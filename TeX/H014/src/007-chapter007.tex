
\chapter{मुक्ति}

हम पहले कह चुके हैं कि ‘कर्म’ शब्द ‘कार्य’ के अतिरिक्त कार्यकारण भाव को भी सूचित करता है~। कोई कार्य, कोई विचार, जो फल उत्पन्न करता है, ‘कर्म’ कहलाता है, इसलिए ‘कर्मविधान’ का अर्थ है कार्यकारण सम्बन्ध का नियम; यदि कारण रहे, तो उसका फल भी अवश्य होगा~। इसका व्यतिक्रम कभी हो नहीं सकता~। भारतीय दर्शन के अनुसार यह ‘कर्मविधान’ समस्त जगत् पर लागू है~। हम जो कुछ देखते हैं, अनुभव करते हैं अथवा जो कुछ कर्म करते हैं, वह एक ओर तो पूर्व कर्म का फल है और दूसरी ओर वही कारण होकर अन्य फल उत्पन्न करता है~। इसके साथ ही साथ हमें यह भी समझ लेना आवश्यक है कि ‘विधान’ अथवा ‘नियम’ शब्द का अर्थ क्या है~। मनोविज्ञान की दृष्टि से इसका अर्थ है - घटना श्रेणियों की पुनरावर्तन की ओर प्रवृत्ति~। जब हम देखते हैं कि एक घटना के बाद कोई दूसरी घटना होती है अथवा दो घटनाएँ साथ ही साथ होती हैं, तब हम सोचते हैं कि इस प्रकार सर्वदा ही होता रहेगा~। हमारे देश के प्राचीन नैयायिक इसे ‘व्याप्ति’ कहते हैं~। उनके मतानुसार नियम सम्बन्धी हमारी समस्त धारणाएँ इसी व्याप्ति के आधार पर होती हैं~। अनेक प्रकार की घटना श्रेणियाँ अपरिवर्तनीय क्रम से हमारे मन में गुँथी हुई रहती हैं~। यही कारण है कि कभी-कभी किसी विषय का अनुभव करते ही वह तुरन्त मन के अन्तर्गत अन्य कुछ बातों से सम्बद्ध हो जाता है~। कोई एक भाव अथवा, हमारे मनोविज्ञान के अनुसार, चित्त में उत्पन्न कोई एक तरंग सदैव उसी प्रकार की अन्य तरंगों को उत्पन्न कर देती है~। इसी को भाव-योग विधान (\enginline{Law of the Association of Ideas}) कहते हैं, और ‘कार्यकारण सम्बन्ध’ इसी ‘व्याप्ति’ नामक योगविधान का एक पहलू मात्र है~। अन्तर्जगत् तथा बाह्य जगत् दोनों में ‘नियमतत्त्व’ अथवा नियम की कल्पना एक ही है, और वह है - यह आशा रखना कि एक घटना के बाद दूसरी एक विशिष्ट घटना होगी और इस क्रमपरम्परा की पुनरावृत्ति होती रहेगी~। यदि ऐसा हो, तो फिर वास्तव में प्रकृति में कोई नियम नहीं है~। कार्यतः यह कहना भूल होगी कि पृथ्वी में गुरुत्वाकर्षण शक्ति है अथवा पृथ्वी के किसी स्थान में कोई नियम विद्यमान है~। हमारा मन जिस प्रणाली से कुछ घटना श्रेणियों की धारणा करता है, उस प्रणाली को ही हम नियम कहते हैं, और यह हमारे मन में ही स्थित है~। मान लो, कुछ घटनाएँ एक के बाद दूसरी अथवा एक साथ घटीं~। इससे हमारे मन में यह दृढ धारणा हो गयी कि भविष्य में नियमित रूप से पुनः-पुनः ऐसा होगा~। और इस प्रकार हमारा मन यह ग्रहण करने में समर्थ हो गया कि सारी घटना श्रेणी किस प्रकार घटित हो रही है~। बस इसी को हम ‘नियम’ कहते हैं~।

‘अब प्रश्न यह है कि नियम के सर्वव्यापी होने का क्या अर्थ है~। हमारा जगत् अनन्त सत्ता का वह अंश है, जो हमारे देश के मनौवैज्ञानिकों के शब्दों में, ‘देश-काल-निमित्त’ द्वारा सीमाबद्ध है~। इससे यह निश्चित है कि नियम केवल इस सीमाबद्ध जगत् में ही सम्भव है, इसके परे कोई नियम सम्भव नहीं~। जब कभी हम जगत् की चर्चा करते हैं, तो उससे हमारा अभिप्राय होता है सत्ता का केवल वही अंश, जो हमारे मन द्वारा सीमाबद्ध है। केवल यह इन्द्रियगोचर जगत् ही - जिसे हम देख, सुन और अनुभव कर सकते हैं, स्पर्श कर सकते हैं, जिसे विचार और कल्पना में ला सकते हैं - नियमों के आधीन है~। पर इसके बाहर और कहीं नियम का प्रभाव नहीं, क्योंकि हमारे मन और इन्द्रियगोचर संसार से परे कार्य-कारण भाव की पहुँच हो नहीं सकती। जो कुछ भी हमारे मन और इन्द्रियों के अतीत है, वह कार्यकारण के नियम द्वारा बद्ध नहीं है; क्योंकि इन्द्रियातीत पदार्थ में मन का सम्बन्ध या योग नहीं हो सकता, और इस प्रकार के भावसम्बन्ध या भाव-योग बिना कार्य-कारण सम्बन्ध ही नहीं हो सकता~। जब यह ‘अस्तित्व’ या सत्ता नामरूप के बन्धनों में जकड़ जाती है, तभी यह कार्य-कारण नियम के सामने सिर झुकाती है, और तब यह ‘नियम’ के आधीन कही जाती है, क्योंकि सभी नियमों का मूल है यही कार्य-कारण सम्बन्ध~। अतएव इससे यह स्पष्ट है कि ‘स्वाधीन इच्छा’ नामक कोई चीज नहीं हो सकती~। ‘स्वाधीन इच्छा’ यह शब्दप्रयोग ही स्वविरोधी है; क्योंकि इच्छा क्या है, हम जानते हैं; और जो कुछ हम जानते हैं; सब इस जगत् के ही अन्तर्गत है; तथा जो कुछ हमारे इस जगत् के अन्तर्गत है, वह सभी देश-काल-निमित्त के साँचे में ढला हुआ है~। अतएव, जो कुछ हम जानते हैं, या सम्भवतः जान सकते हैं, वह सभी कुछ कार्य-कारण नियम के अधीन है; और जो कुछ कार्य-कारण नियमाधीन होता है, वह क्या कभी स्वाधीन हो सकता है? उसके ऊपर अन्यान्य वस्तुएँ अपना कार्य करती हैं, और वह स्वयं भी एक समय कारण बन जाता है~। बस इसी प्रकार सब चल रहा है~। परन्तु जो इच्छा के रूप में परिणत हो जाता है, जो पहले इच्छा के रूप में नहीं था, परन्तु बाद में देश-काल-निमित्त के साँचे में पड़ने से मानवी इच्छा हो गया, वह अवश्य स्वाधीन है; और इस देश-काल-निमित्त के साँचे से जब यह इच्छा मुक्त हो जाएगी, तो वह पुनः स्वतन्त्र हो जाएगा~। स्वाधीनता या मुक्तावस्था से वह आता है, आकर इस बन्धनरूपी साँचे में पड़ जाता है और फिर उससे निकलकर पुनः स्वाधीनता को प्राप्त हो जाता है~।

प्रश्न पूछा गया था कि यह जगत् कहाँ से आया है, किसमें अवस्थित है और फिर किसमें इसका लय हो जाता है? इसका उत्तर दिया गया कि मुक्तावस्था से इसकी उत्पत्ति होती है, बन्धन में इसकी अवस्थिति है और मुक्ति में ही इसका लय होता है~। अतएव जब हम यह कहते हैं कि मनुष्य उसी अनन्त सत्ता का प्रकाश मात्र है, तो उससे हमारा तात्पर्य यही होता है कि वह उस अनन्त सत्ता का एक अत्यन्त क्षुद्र अंश मात्र है~। यह शरीर तथा यह मन, जो हमें दिखायी देता है, समग्र प्रकृत मनुष्य का एक अंश मात्र है - उसी अनन्त पुरुष का केवल एक क्षुद्र अंश है~। यह सारा ब्रह्माण्ड उसी अनन्त पुरुष का एक अंश है~। और हमारे समस्त विधान, हमारे सारे बन्धन, हमारा आनन्द, विषाद, सुख, हमारी आशा आकांक्षा सभी केवल इस क्षुद्र जगत् के अन्तर्गत हैं~। हमारी उन्नति अवनति सभी इस क्षुद्र जगत् के अन्तर्गत है~। अतएव आपने देखा, इस जगत् के - इस मनःकल्पित जगत् के चिरकाल तक रहने की आशा करना और स्वर्ग जाने की अभिलाषा करना कैसी नासमझी है! स्वर्ग और है क्या? हमारे इस जगत् की पुनरावृत्ति ही तो! आप यह स्पष्ट देख सकते है कि इस अखिल अनन्त सत्ता को अपने इस सान्त जगत् के समान कर लेने की इच्छा करना कैसी नासमझी की बात है, कैसा असम्भव व्यापार है! अतएव यदि कोई मनुष्य यह कहे कि जिस भाव में वह अभी है, उसी में चिरकाल तक रहेगा, जो कुछ अभी उसके पास है, उसे ही लेकर सदा के लिए विद्यमान रहेगा, अथवा, जैसा कि मैं कभी-कभी कहा करता हूँ, यदि वह ‘आरामपूर्ण धर्म’ की इच्छा करे, तो तुम यह निश्चय जान लो कि वह इतना गिर चुका है कि वह अपनी वर्तमान अवस्था से अधिक उच्च और कुछ सोच ही नहीं सकता - अपनी क्षुद्र वर्तमान परिस्थिति के अतिरिक्त अन्य किसी परिस्थिति की धारणा तक नहीं कर सकता~। वह अपने अनन्त स्वरूप को भूल चुका है, और उसकी सारी भावनाएँ क्षुद्र सुख, दुःख और ईर्ष्या आदि ही में आबद्ध हैं~। इस सान्त जगत् को ही वह अनन्त मान लेता है; केवल इतना ही नहीं, वह इस अज्ञान को किसी भी हालत में छोड़ना नहीं चाहता~। एक जोंक के समान वह इस जीवन से चिपका रहता है~। प्राण भले ही जायँ, पर वह यह तृष्णा कभी न छोड़ेगा! हमारे इस छोटेसे ज्ञात संसार के बाहर कौन जाने और भी कितने असंख्य प्रकार के सुख-दुःख, जीव जन्तु, विधि विधान, उन्नति के नियम और कार्य-कारण सम्बन्ध विद्यमान हैं~। पर उससे क्या? आखिर वे सब भी तो हमारी अनन्त प्रकृति के केवल एक भाग मात्र ही हैं~।

मुक्तिलाभ करने के लिए हमें इस ससीम विश्व के परे जाना होगा; मुक्ति यहाँ प्राप्त नहीं हो सकती~। पूर्ण साम्यावस्था का लाभ, अथवा ईसाई लोग जिसे ‘बुद्धि से अतीत शान्ति’ कहते हैं, उसकी प्राप्ति इस जगत् में नहीं हो सकती, और न स्वर्ग में ही अथवा न किसी ऐसे स्थान में ही जहाँ हमारे मन और विचार जा सकते हैं, जहाँ हम इन्द्रियों द्वारा किसी प्रकार का अनुभव प्राप्त कर सकते हैं अथवा जहाँ हमारी कल्पनाशक्ति काम कर सकती है~। इस प्रकार के किसी भी स्थान में हमें मुक्ति नहीं प्राप्त हो सकती, क्योंकि ऐसे सब स्थान निश्चय ही हमारे जगत् के अन्तर्गत होंगे, और यह जगत् तो देश, काल और निमित्त के बन्धनों से जकड़ा हुआ है~। सम्भव है, कुछ ऐसे भी स्थान हों, जो हमारी इस पृथ्वी की अपेक्षा अधिक सूक्ष्म हों, जहाँ के सुखभोग यहाँ से अधिक उत्कट हों, परन्तु वे स्थान भी तो हमारे विश्व के ही अन्तर्गत होंगे, और इसी कारण नियमों की सीमा के भीतर भी होंगे~। अतएव हमें इस विश्व के परे जाना होगा~। और वास्तव में सच्चा धर्म तो वहाँ आरम्भ होता है, जहाँ इस क्षुद्र जगत् का अन्त हो जाता है~। वहाँ इन छोटे-छोटे सुख, दुःख और ज्ञान का अन्त हो जाता है और प्रकृत धर्म आरम्भ होता है~। जब तक हम जीवन के प्रति इस तृष्णा को नहीं छोड़ते, इन क्षणभंगुर सान्त विषयों के प्रति अपनी प्रबल आसक्ति का त्याग नहीं करते, तब तक इस जगत् से अतीत उस असीम मुक्ति की एक झलक भी पाने की आशा करना व्यर्थ है~। अतएव यह नितान्त युक्तियुक्त है कि मानवहृदय की समस्त उदात्त स्पृहाओं की चरम गति - मुक्ति - को प्राप्त करने का केवल एक ही उपाय है, और वह है इस क्षुद्र जीवन का त्याग, इस क्षुद्र जगत् का त्याग, इस पृथ्वी का त्याग, स्वर्ग का त्याग, शरीर का, मन का एवं सीमाबद्ध सभी वस्तुओं का त्याग~। यदि हम मन एवं इन्द्रियगोचर इस छोटेसे जगत् से अपनी आसक्ति हटा लें, तो उसी क्षण हम मुक्त हो जाएँगे~। बन्धन से मुक्त होने का एकमात्र उपाय है सारे नियमों के बाहर चले जाना - कार्य-कारण शृंखला के बाहर हो जाना~।

किन्तु इस संसार के प्रति आसक्ति का त्याग करना बड़ा कठिन है~। बहुत ही थोड़े लोग ऐसा कर पाते हैं~। हमारे शास्त्रों में इसके लिए दो मार्ग बताये गये हैं~। एक ‘नेति, नेति’ (यह नहीं, यह नहीं) कहलाता है और दूसरा ‘इति, इति’~। पहला मार्ग निवृत्ति का है, जिसमें ‘नेति, नेति’ करते हुए सर्वस्व का त्याग करना पड़ता है, और दूसरा है प्रवृत्ति का, जिसमें ‘इति, इति’ करते हुए सब वस्तुओं का भोग करके फिर उनका त्याग किया जाता है~। निवृत्ति मार्ग अत्यन्त कठिन है, यह केवल प्रबल इच्छाशक्ति सम्पन्न तथा विशेष उन्नत महापुरुषों के लिए ही साध्य है~। उनके कहने भर की देर है, “नहीं, मुझे यह नहीं चाहिए”, कि बस उनका शरीर और मन तुरन्त उनकी आज्ञा का पालन करता है, और वे संसार के बाहर चले जाते हैं~। परन्तु ऐसे लोग बहुत ही दुर्लभ हैं~। यही कारण है कि अधिकांश लोग प्रवृत्ति मार्ग ग्रहण करते हैं~। इसमें उन्हें संसार में से ही होकर जाना पड़ता है, और इन बन्धनों को तोड़ने के लिए इन बन्धनों की ही सहायता लेनी पड़ती है~। यह भी एक प्रकार का त्याग है - अन्तर इतना ही है कि यह धीरे-धीरे, क्रमशः सब पदार्थों को जानकर, उनका भोग करके और इस प्रकार उनके सम्बन्ध में अनुभव लाभ करके प्राप्त होता है~। इस प्रकार विषयों का स्वरूप भलीभाँति जान लेने से मन अन्त में उन सब को छोड़ देने में समर्थ हो जाता है और आसक्तिशून्य बन जाता है~। अनासक्ति के प्रथमोक्त मार्ग का साधन है विचार, और दूसरे का कर्म~। प्रथम मार्ग ज्ञानयोगी का है, - वह सभी कर्मों का त्याग करता है; दूसरा कर्मयोगी का है, - उसे निरन्तर कर्म करते रहना पड़ता है~। इस जगत् में प्रत्येक मनुष्य को कर्म करना ही पड़ेगा, केवल वही व्यक्ति कर्म से परे है, जो सम्पूर्ण रूप से आत्मतृप्त है, जिसे आत्मा के अतिरिक्त अन्य कोई भी कामना नहीं, जिसका मन आत्मा को छोड़ अन्यत्र कहीं भी गमन नहीं करता, जिसके लिए आत्मा ही सर्वस्व है~। शेष सभी व्यक्तियों को तो कर्म अवश्य ही करना पड़ेगा~। जिस प्रकार एक जलस्रोत स्वाधीन भाव से बहते-बहते किसी गढ़े में गिरकर एक भँवर का रूप धारण कर लेता है और उस भँवर में कुछ देर चक्कर काटने के बाद पुनः एक उन्मुक्त स्रोत के रूप में बाहर आकर अनिर्बन्ध रूपसे बह निकलता है, उसी प्रकार यह मनुष्य जीवन भी है~। यह भी भँवर में पड़ जाता है - नाम रूपात्मक जगत् में पड़कर कुछ समय एक गोते खाता हुआ चिल्लाता है, ‘यह मेरा बाप,’ ‘यह मेरी माँ,’ ‘यह मेरा भाई,’ ‘यह मेरा नाम,’ ‘यह मेरा यश,’ आदि आदि~। फिर अन्त में बाहर निकलकर पुनः अपना मुक्तभाव प्राप्त कर लेता है~। समस्त संसार का यही हाल है~। हम चाहे जानते हों या न जानते हो, ज्ञानवश या अज्ञानवस हम सभी इस संसारस्वप्न से होश में आने का यत्न कर रहे हैं~। मनुष्य का सांसारिक अनुभव इसीलिए है कि वह उसे इस जगत् के भँवर से बाहर निकाल दे~।

तो फिर कर्मयोग क्या है? - कर्म के रहस्य का ज्ञान~। हम देखते हैं कि सारा संसार कर्म में रत है~। यह सब किसलिए है? - मुक्तिलाभ के लिए, स्वाधीनता के लिए~। एक छोटे परमाणु से लेकर सर्वोच्च प्राणी तक सभी, ज्ञानवश अथवा अज्ञानवश, एक ही उद्देश्य के लिए कार्य किये जा रहे हैं और वह है - शारीरिक स्वाधीनता, मानसिक स्वाधीनता, आध्यात्मिक स्वाधीनता~। सभी पदार्थ निरन्तर स्वाधीनता पाने की चेष्टा कर रहे हैं, बन्धन से मुक्त होने का प्रयत्न कर रहे हैं~। सूर्य, चन्द्र, पृथ्वी, ग्रह आदि सभी बन्धन से दूर होने की चेष्टा कर रहे हैं~। कहा जा सकता है कि सारा जगत् केन्द्राभिमुखी और केन्द्रापसारी शक्तियों की एक क्रीड़ाभूमि है~। संसार में इधर उधर धक्के खाकर तथा बहुत समय तक चोटें सहकर फिर प्रकृत तत्त्व को जानने की अपेक्षा हमें कर्मयोग द्वारा सहज ही कर्म का रहस्य, कर्म की पद्धति तथा अल्प परिश्रम द्वारा अधिक कार्य करने की रीति ज्ञात हो जाती है~। यदि हमें कर्मयोग के उपयोग का ज्ञान न रहे, तो व्यर्थ ही हमारी बहुतसी शक्ति क्षय हो जाएगी~। कर्मयोग कर्म की एक विद्या ही बना लेता है~। इस विद्या द्वारा तुम यह जान सकते हो कि संसार के समस्त कार्यों का सद्व्यवहार किस प्रकार करना चाहिए~। कर्म तो अवश्यम्भावी है - करना ही पड़ेगा, किन्तु सर्वोच्च ध्येय को सम्मुख रखकर कार्य करो~। कर्मयोग हमें इस बात पर विवश कर देता है कि यह दुनिया केवल दो दिन की है, इसमें से होकर हमें गुजरना ही होगा; किन्तु मुक्ति इसके भीतर नहीं है, उसके लिए तो हमें इस संसार से परे जाना होगा~। संसार से परे जाने के इस मार्ग को प्राप्त करने के लिए हमें धीरे-धीरे परन्तु दृढ़ पगों से इसी संसार में से होकर जाना होगा~। हाँ, कुछ ऐसे विशेष महापुरुष हो सकते हैं, जिनके सम्बन्ध में मैंने अभी कहा है, जो एकदम संसार से अलग खड़े होकर उसे उसी प्रकार त्याग सकते हैं, जिस प्रकार साँप अपनी केंचुली को छोड़कर, एक ओर खड़े होकर उसे देखता है~। ऐसे विशेष महापुरुष कुछ अवश्य हैं, पर अधिकांश व्यक्तियों को तो इस कर्मबहुल संसार में से ही धीरे-धीरे होकर जाना पड़ता है~। और कर्मयोग उसमें अधिक से अधिक कृतकार्य होने की रीति, उसका रहस्य एवं उपाय दिखा देता है~।

कर्मयोग क्या कहता है~। वह कहता है कि तुम निरन्तर कर्म करो, परन्तु कर्म में आसक्ति का त्याग कर दो~। अपने को किसी भी विषय के साथ एक मत कर डालो - अपने मन को सदैव स्वाधीन रखो~। संसार में तुम्हें जो सुख-दुःख दिखायी देते हैं, वे तो विश्व के अवश्यम्भावी व्यापार हैं~। दारिद्र्य, सम्पत्ति, सुख ये सब क्षणभंगुर ही हैं, वास्तव में हमारे प्रकृत स्वभाव से इनका कोई सम्बन्ध नहीं~। हमारा प्रकृत स्वरूप तो सुख और दुःख से एकदम परे है, प्रत्यक्ष और कल्पनागोचर विषयों के बिलकुल अतीत है; परन्तु फिर भी हमें निरन्तर कर्म करते रहना चाहिए~। ‘क्लेश आसक्ति से ही उत्पन्न होता है, कर्म से नहीं~।’ ज्यों ही हम अपने कर्म से अपने-आपको एक कर डालते हैं, त्यों ही ‘क्लेश’ उत्पन्न होता है; परन्तु यदि हम अपने को उससे पृथक् रखें तो हमें वह क्लेश छू तक नहीं सकता~। यदि किसी दूसरे मनुष्य का कोई सुन्दर चित्र जल जाता है, तो देखनेवाले व्यक्ति को कोई दुःख नहीं होता, परन्तु यदि उसका अपना चित्र जल जाय, तो उसे कितना दुःख होता है! ऐसा क्यों? दोनों ही चित्र सुन्दर थे और सम्भव है, दोनों एक ही मूल चित्र की नकल रहे हों; परन्तु एक दशा में उस व्यक्ति को बिलकुल क्लेश नहीं हुआ, पर दूसरी में बहुत हुआ~। इसका कारण यही है कि पहली दशा में वह अपने को चित्र से पृथक् रखता है, परन्तु दूसरी दशा में अपने को उससे एकरूप कर देता है~। यह ‘मैं और मेरा’ ही समस्त क्लेश की जड़ है~। अधिकार की भावना के साथ ही स्वार्थ आ जाता है और स्वार्थपरता से ही क्लेश उत्पन्न होता है~। प्रत्येक स्वार्थपर कार्य और विचार हमें किसी न किसी वस्तु से आसक्त कर देता है और हम तुरन्त ही उस वस्तु के दास बन जाते हैं~। चित्त का प्रत्येक स्पन्दन, जिसमें ‘मैं और मेरे’ की भावना रहती है, हमें उसी क्षण जंजीरों से जकड़कर गुलाम बना देता है~। हम जितना ही ‘मैं’ और ‘मेरा’ कहते हैं, दासत्व का भाव हममें उतना ही बढ़ता जाता है और हमारे क्लेश भी उतने ही अधिक बढ़ जाते हैं~। अतएव कर्मयोग हमें शिक्षा देता है कि हम संसार के समस्त चित्रों के सौन्दर्य का आनन्द उठायें, परन्तु उसमें से किसी भी एक के साथ एकरूप न हो जायँ~। कभी यह न कहो कि यह ‘मेरा’ है~। जब कभी हम यह कहेंगे कि अमुक वस्तु ‘मेरी’ है, तो उसी क्षण क्लेश हमें आ घेरेगा~। अपने मन में भी कभी न कहो कि यह ‘मेरा बच्चा’ है~। बच्चे को लेकर प्यार करो, परन्तु यह न कहो कि वह ‘मेरा’ है~। ‘मेरा’ कहने से ही क्लेश उत्पन्न होगा~। ‘मेरा घर’, ‘मेरा शरीर’ आदि न कहो~। कठिनाई तो यहीं पर है~। शरीर न तो तुम्हारा है, न मेरा और न अन्य किसी का~। ये शरीर तो प्रकृति के नियमों के अनुसार आते जाते रहते हैं, परन्तु हम बिलकुल मुक्त हैं - केवल साक्षी मात्र हैं~। जिस प्रकार एक चित्र या एक दीवाल स्वाधीन नहीं है, उसी प्रकार यह शरीर भी स्वाधीन नहीं है~। फिर हम इस शरीर में ऐसे आसक्त क्यों हो? एक चित्रकार एक चित्र बना देता है - और बस चल देता है~। आसक्ति की यह स्वार्थी भावना न उठने दो कि ‘मैं इस पर अपना अधिकार जमा लूँ~।’ ज्यों ही यह भावना उत्पन्न होगी, त्यों ही क्लेश आरम्भ हो जाएगा~।

अतएव, कर्मयोग शिक्षा देता है कि सब से पहले तुम स्वार्थपरता के अंकुर के बढ़ने की इस प्रवृत्ति को नष्ट कर दो~। और जब तुममें इसके दमन की क्षमता आ जाय, तो मन को बस वही रोक लो, स्वार्थपरता की इन लहरों में उसे मत बह जाने दो~। फिर तुम संसार में चले जाओ और यथाशक्ति कर्म करो~। फिर तुम सब से मिल सकते हो, जहाँ चाहो जा सकते हो, तुम्हें कुछ भी स्पर्श न कर सकेगा~। पानी में रहते हुए भी जिस प्रकार पद्मपत्र को पानी स्पर्श नहीं कर सकता और न उसे भिगा सकता है, उसी प्रकार तुम भी संसार में निर्लिप्त भाव से रह सकोगे~। इसी को ‘वैराग्य’ कहते हैं, इसी को कर्मयोग की नींव - + अनासक्ति - कहते हैं~। मैंने तुम्हें बताया ही है कि अनासक्ति के बिना किसी भी प्रकार की योगसाधना नहीं हो सकती~। + अनासक्ति ही समस्त योग-साधना की नींव है~। हो सकता है कि जिस मनुष्य ने अपना घर छोड़ दिया है, अच्छे वस्त्र पहनना छोड़ दिया है, अच्छा भोजन करना छोड़ दिया है और जो मरुस्थल में जाकर रहने लगा है, वह भी एक घोर विषयासक्त व्यक्ति हो~। उसकी एकमात्र सम्पत्ति - उसका शरीर - ही उसका सर्वस्व हो जाय और वह उसी के सुख के लिए सतत प्रयत्न करे~। अनासक्ति बाह्य शरीर पर निर्भर नहीं है, वह तो मन पर निर्भर है~। ‘मैं और मेरे’ की जंजीर तो मन में ही रहती है~। यदि शरीर और इन्द्रियगोचर विषयों के साथ इस जंजीर का सम्बन्ध न रहे, तो फिर हम कहीं भी क्यों न रहें, हम बिलकुल अनासक्त रहेंगे~। हो सकता है कि एक व्यक्ति राजसिंहासन पर बैठा हो, परन्तु फिर भी बिलकुल अनासक्त हो; और दूसरी ओर यह भी सम्भव है कि एक व्यक्ति चिथड़ों में हो, पर फिर भी वह बुरी तरह आसक्त हो~। पहले हमें इस प्रकार की अनासक्ति प्राप्त कर लेनी होगी, और फिर सतत कार्य करते रहना होगा~। यद्यपि यह है बड़ा कठिन, परन्तु फिर भी कर्मयोग हमें अनासक्त होने की रीति सिखा देता है~।

आसक्ति का सम्पूर्ण त्याग करने के दो उपाय है, प्रथम उपाय उन लोगों के लिए है, जो न तो ईश्वर में विश्वास करते हैं और न किसी बाहरी सहायता में~। वे अपने-अपने कौशल एवं उपायों का अवलम्बन करें~। उन्हें अपनी ही इच्छाशक्ति, मनःशक्ति एवं विचार का अवलम्बन करके कार्य करना होगा - उन्हें दृढ़तापूर्वक कहना होगा, ‘मैं अनासक्त होऊँगा ही~।’ जो ईश्वर पर विश्वास करते हैं, उनके लिए एक दूसरा मार्ग है, जो इसकी अपेक्षा बहुत सरल है~। वे समस्त कर्मफलों को ईश्वर को अर्पित करके कर्म करते जाते हैं, इसलिए कर्मफल में कभी आसक्त नहीं होते~। वे जो कुछ देखते हैं, अनुभव करते है, सुनते अथवा करते हैं, वह सब भगवान के लिए ही होता है~। हम जो कुछ भी सत् कार्य करें, उससे हमें किसी प्रकार की प्रशंसा अथवा लाभ की आशा नहीं करनी चाहिए~। वह तो सब प्रभु का ही है~। सारे फल उन्हीं के श्रीचरणों में अर्पित कर दो~। हमें तो एक किनारे खड़े हो यह सोचना चाहिए कि हम तो केवल प्रभु के - अपने स्वामी के आज्ञाकारी भृत्य हैं और हमारी प्रत्येक कर्म प्रवृत्ति प्रतिक्षण उन्हीं के पास से आ रही है। -

\begin{verse}
यत्करोषि यदश्नासि यज्जुहोसि ददासि यत्~।\\ यत्तपस्यसि कौन्तेय तत्कुरुष्व मदर्पणम्~॥ - गीता, ९~।२७
\end{verse}

- तुम जो कुछ पूजा करो, ध्यान करो, अथवा कर्म करो, सब उन्हीं को अर्पण कर दो और स्वयं निश्चिन्त हो जाओ~। हम शान्ति से रहें - पूर्ण शान्ति से रहें, और अपना सम्पूर्ण शरीर, मन, यहाँ तक कि अपना सर्वस्व श्रीभगवान के समक्ष चिर बलिस्वरूप दे दें~। अग्नि में घी की आहुतियाँ देने की अपेक्षा दिन रात केवल यही एक महान् आहुति - अपने इस क्षुद्र ‘अहं’ की आहुति - देते रहो~। “संसार में धन की खोज में लगे हुए, हे प्रभु, मैंने केवल तुम्ही को एकमात्र धन पाया; मैं तुम्हारे श्रीचरणों में आत्मसमर्पण करता हूँ~। संसार में किसी प्रेमास्पद की खोज करते-करते, हे नाथ, केवल तुम्ही को मैंने एकमात्र प्रेमास्पद पाया; मैं तुम्हारे श्रीचरणों में आत्मसमर्पण करता हूँ~।” हमें चाहिए कि हम दिन रात यही दुहराते रहें और कहें, “हे प्रभु! मुझे कुछ नहीं चाहिए~। कोई वस्तु चाहे अच्छी हो, चाहे बुरी, मुझे उससे तनिक भी प्रयोजन नहीं~। मैं सब कुछ तुम्हीं को समर्पण करता हूँ~।”

रात दिन हमें इस तथाकथित भासमान ‘अहं’ का त्याग करते रहना चाहिए, जब तक कि यह अभ्यास के रूप में परिणत न हो जाय, तब तक कि यह हमारे शरीर की शिरा-शिरा में, नस-नस में और मस्तिष्क में व्याप्त न हो जाय और हमारा सम्पूर्ण शरीर ही प्रतिक्षण आत्मत्याग के इस भाव के आधीन न हो जाय~। फिर जहाँ हमारी इच्छा हो, हम जा सकते हैं; हमें फिर कुछ भी स्पर्श न कर सकेगा~। चाहे हम गोले-बारूद की तुमुल आवाज से पूर्ण रणक्षेत्र में भी क्यों न चले जायँ, फिर भी हम सदैव मुक्त, स्वाधीन और शान्त ही रहेंगे~।

कर्मयोग हमें इस बात की शिक्षा देता है कि ‘कर्तव्य’ की जो भावना है, वह एक निम्न श्रेणी की चीज है - ‘कर्तव्यबुद्धि’ से कोई कार्य केवल निम्न भूमि में ही किया जाता है~। फिर भी हममें से प्रत्येक को अपने कर्तव्य कर्म करने ही होंगे~। परन्तु हम देखते हैं कि कर्तव्य की यह भावना ही अनेक बार हमारे दुःखों का एकमात्र कारण होती है~। कर्तव्य हमारे लिए एक प्रकार का रोग सा हो जाता है और हमें सदा उसी दिशा में खींचता रहता है~। यह हमें पक्का जकड़ लेता है और हमारे पूरे जीवन को दुःखपूर्ण कर देता है~। यह तो मनुष्यजीवन के लिए महा विभीषिका स्वरूप है~। यह कर्तव्यबुद्धि ग्रीष्मकाल के मध्याह्न सूर्य की नाई है, जो मनुष्य की अन्तरात्मा को दग्ध कर देती है~। जरा कर्तव्य के उन बेचारे गुलामों की ओर तो देखो! उनका कर्तव्य तो उन्हें इतनी भी छुट्टी नहीं देता कि वे पूजा पाठ अथवा स्नान ध्यान कर सकें~। कर्तव्य उन्हें प्रतिक्षण घेरे रहता है~। वे बाहर जाते हैं और काम करते हैं, कर्तव्य सदा उनके सिर पर सवार रहता है~। वे घर आते हैं और फिर अगले दिन का काम सोचने लगते हैं; कर्तव्य उन पर सवार हो रहता है~। यह तो एक गुलाम की जिन्दगी हुई! फिर एक दिन ऐसा आ जाता है कि वे कसेकसाये घोड़े की तरह सड़क पर ही गिरकर मर जाते हैं! कर्तव्य की यह सर्वसाधारण कल्पना है~। परन्तु अनासक्त होकर एक स्वतन्त्र व्यक्ति की तरह कार्य करना तथा समस्त कर्म भगवान् को समर्पण कर देना ही असल में हमारा एकमात्र कर्तव्य है~। हमारे समस्त कर्तव्य तो उन्हीं के हैं~। कितने सौभाग्य की बात है कि हम इस संसार में भेजे गये हैं~। हम बस अपने निर्दिष्ट कार्य करते जा रहे हैं! कौन जाने, हम उन्हें अच्छे कर रहे हैं या बुरे? उन्हें उत्तम रूप से करने पर भी हम फल की आकांक्षा न करेंगे और बुरी तरह से करने पर भी हम चिन्तित न होंगे~। निश्चिन्त होकर स्वाधीन भाव से शान्ति के साथ कर्म करते जाओ~। पर हाँ, इस प्रकार की अवस्था प्राप्त कर लेना जरा टेढ़ी ही खीर है~। दासत्व को कर्तव्य कह देना, अथवा चर्म के प्रति चर्म की घृणित आसक्ति को कर्तव्य कह देना कितना सरल है! मनुष्य संसार में धन अथवा अन्य किसी प्रिय वस्तु की प्राप्ति के लिए एड़ी-चोटी का पसीना एक करता रहता है~। यदि उससे पूछो, “ऐसा क्यों कर रहे हो?” तो झट उत्तर देता है, “यह तो मेरा कर्तव्य है~।” पर असल में यह तो धन के लिए अस्वाभाविक तृष्णा मात्र है~। और इस तृष्णा के ऊपर कुछ फूल चढ़ाकर वे उसे ढके रखने की चेष्टा करते हैं~।

तब जिसे सब लोग कर्तव्य कहा करते हैं, वह फिर क्या है? वह है केवल आसक्ति - शरीर की गुलामी~। जब कोई आसक्ति दृढ़मूल हो जाती है, तो उसे ही हम कर्तव्य कहने लगते हैं~। उदाहरणार्थ, जहाँ विवाह की प्रथा नहीं है, उन सब देशों में पति पत्नी में आपस में कोई कर्तव्य नहीं होता~। क्रमशः समाज में जब विवाह प्रथा आ जाती है, तब पति पत्नी एक साथ रहने लगते हैं~। उनका यह एक साथ रहना शारीरिक आसक्ति के कारण ही होता है~। और कई पीढ़ियों के बाद जब उनका यह एकत्र वास एक प्रथा सी हो जाता है, तब फिर यही एक कर्तव्य के रूप में परिणत हो जाता है~। यह तो एक प्रकार की चिरस्थायी व्याधि सी है~। यदि एक-आध बार यह प्रबल रूप में प्रतीत होती है, तो उसे हम व्याधि कह देते हैं और यदि यह सामान्य भाव से चिरस्थायी हो जाती है, तो इसे हम प्रकृति या स्वभाव कहने लगते हैं~। जो भी हो, पर है वह एक रोग ही~। आसक्ति चिरस्थायी होकर जब हमारा स्वभाव बन जाती है, तो उसे हम ‘कर्तव्य’ के बड़े नाम से ढक देते हैं~। फिर हम उसके ऊपर फूल चढ़ाते हैं, उसके सामने बाजे बजाते हैं, मन्त्रोच्चारण करते हैं~। तब यह समस्त संसार उसी कर्तव्य के लिए आपस में लड़ने भिड़ने लगता है और एक दूसरे का धन अपहरण करने लगता है~।

कर्तव्य वहीं तक अच्छा है, जहाँ तक कि यह पशुत्व भाव को रोकने में सहायता प्रदान करता है~। उन निम्नतम श्रेणी के मनुष्यों के लिए जो और किसी उच्चतर आदर्श की धारणा ही नहीं कर सकते, शायद कर्तव्य की यह भावना किसी हद तक अच्छी हो, परन्तु जो कर्मयोगी बनना चाहते हैं, उन्हें तो कर्तव्य के इस भाव को एकदम त्याग देना चाहिए~। असल में हमारे या तुम्हारे लिए कोई कर्तव्य है ही नहीं~। जो कुछ तुम संसार को देना चाहते हो अवश्य दो, परन्तु कर्तव्य के नाम पर नहीं~। उसके लिए कुछ चिन्ता तक मत करो~। बाध्य होकर कुछ भी मत करो~। बाध्य होकर भला क्यों करोगे? जो कुछ भी तुम बाध्य होकर करते हो, उससे आसक्ति उत्पन्न होती है~। तुम्हारा अपना कोई कर्तव्य क्यों होना चाहिए?

“सब कुछ ईश्वर को ही अर्पण कर दो~।” इस संसार की भयानक भट्टी में, जिसमें कर्तव्यरूपी अग्नि सभी को झुलसाती रहती है, तुम उस ईश्वरार्पण भावरूपी अमृत-चषक का पान करो और प्रसन्न रहो~। हम सब तो केवल उन प्रभु की ही इच्छा का पालन कर रहे हैं और किसी प्रकार के पुरस्कार अथवा दण्ड से हमारा कोई सम्बन्ध नहीं~। यदि तुम पुरस्कार के इच्छुक हो तो तुम्हे साथ ही दण्ड भी स्वीकार करना पड़ेगा~। दण्ड से छुटकारा पाने का केवल यही उपाय है कि तुम पुरस्कार का भी त्याग कर दो~। क्लेश से मुक्त होने का एकमात्र उपाय यही है कि तुम सुख की भावना का भी त्याग कर दो, क्योंकि ये दोनों चीजें एक ही में गुँथी हुई हैं~। यदि एक ओर सुख है, तो दूसरी ओर क्लेश; एक ओर जीवन है, तो दूसरी ओर मृत्यु~। मृत्यु से छुटकारा पाने का एकमात्र उपाय यही है कि जीवन के प्रति आसक्ति का त्याग कर दो~। जीवन और मृत्यु दोनों एक ही वस्तु हैं - एक ही वस्तु दो विभिन्न पहलू मात्र~। अतएव ‘दुःखशून्य सुख’ एवं ‘मृत्युशून्य जीवन’ की भावना, सम्भव है, स्कूल के छोटे-छोटे बच्चों के लिए बड़ी मधुर हो, परन्तु एक चिन्तनशील व्यक्ति को तो यही प्रतीत होता है कि ये दोनों बातें परस्पर विरोधी हैं और यह समझकर वह इन दोनों का परित्याग कर देता है~। जो कुछ तुम करो, उसके लिए किसी प्रकार की प्रशंसा अथवा पुरस्कार की आशा मत रखो~। ज्यों ही हम कोई सत् कार्य करते हैं, त्यों ही हम उसके लिए प्रशंसा की आशा करने लगते हैं~। ज्यों ही हम किसी सत् कार्य में चन्दा देते हैं, त्यों ही हम चाहने लगते हैं कि हमारा नाम अखबारों में खूब चमक उठे~। ऐसी वासनाओं का फल दुःख के अतिरिक्त और क्या होगा? संसार से कई सर्वश्रेष्ठ महापुरुष उठ गये, पर संसार ने उन्हे जाना तक नहीं~। देखा जाय तो भगवान् बुद्ध तथा ईसा मसीह भी उन महापुरुषों की तुलना में द्वितीय श्रेणी के हैं; परन्तु संसार उन महापुरुषों के बारे में कुछ जानता तक नहीं~। प्रत्येक देश में ऐसे सैंकड़ों महापुरुष हुए हैं, परन्तु सदैव वे अपना कार्य चुपचाप ही करते रहे~। चुपचाप वे अपना जीवन व्यतीत करते हैं और चुपचाप इस संसार से चले जाते हैं; समय पर उनकी चिन्तनराशि बुद्धों और ईसा मसीहों में व्यक्त भाव धारण करती है, और संसार केवल जान पाता है इन्हीं बुद्धों और ईसा मसीहों को~। सर्वश्रेष्ठ महापुरुषगण अपने ज्ञान से किसी प्रकार की यशप्राप्ति की कामना नहीं रखते~। ऐसे महापुरुष तो केवल संसार के हित के लिए अपने भाव छोड़ जाते हैं; वे अपने लिए किसी बात का दावा नहीं करते और न अपने नाम पर कोई सम्प्रदाय अथवा धर्मप्रणाली ही स्थापित कर जाते हैं~। उनका स्वभाव ही इन बातों का विरोधी होता है~। ये महापुरुष शुद्धसात्त्विक होते हैं; वे केवल प्रेम से द्रवीभूत होकर रहते हैं~। मैंने एक ऐसा योगी\footnote{पवहारी बाबा~।}देखा है~। वे भारतवर्ष में एक गुफा में रहते हैं~। मैंने जितने भी अद्भुत महापुरुष देखे, उसमें से वे एक हैं~। वे अपना ‘मैं-पन’ यहाँ तक खो चुके हैं कि उनमें से मनुष्य भाव बिलकुल निकल गया है और उनके हृदय पर केवल ईश्वरीय भाव ने ही सम्पूर्ण रूप से अधिकार जमा लिया है~। यदि कोई प्राणी उनके एक हाथ में काट लेता हैं, तो उसे वे दूसरा हाथ भी दे देते हैं और कहते हैं, ‘यह तो प्रभु की इच्छा है~।’ उनके लिए जो कुछ भी उनके पास आता है, सब प्रभु से ही आता है~। वे अपने को लोगों के सामने प्रकट नहीं करते, परन्तु फिर भी वे प्रेम तथा मधुर एवं सत्य भावों के प्रस्रवणस्वरूप हैं~।

इसके बाद फिर वे लोग हैं, जिनमें अपेक्षाकृत अधिक रजःशक्ति होती है~। वे सिद्ध पुरुषों के भावों को ग्रहण करके फिर उनका संसार में प्रचार करते हैं~। सर्वश्रेष्ठ महापुरुषगण चुपचाप सत्य एवं उदात्त भावों का संग्रह करते हैं, और ये दूसरे - बुद्ध अथवा ईसा मसीह जैसे - एक स्थान से दूसरे स्थान जाकर भ्रमण करके उनका प्रचार करते हैं~। गौतम बुद्ध के जीवनचरित्र में हम पाते हैं कि वे अपने को निरन्तर यही कहते आये कि वे पचीसवें बुद्ध थे~। उनके पहले के चौबीस बुद्धों के इतिहास का हमें कोई ज्ञान नहीं, परन्तु फिर भी यह निश्चित है कि ये हमारे ऐतिहासिक बुद्ध अवश्य ही उन बुद्धों द्वारा डाली हुई भित्ति पर ही अपना धर्मप्रासाद निर्माण कर गये हैं~। सर्वश्रेष्ठ महापुरुषगण शान्त, नीरव एवं अपरिचित होते हैं~। विचार की प्रचण्ड शक्ति से वे भली भाँति परिचित रहते हैं~। उनमें यह दृढ़ विश्वास होता है कि यदि वे किसी पर्वत की गुफा में जाकर उसके द्वार बन्द करके केवल चार पाँच सद्विचारों का ही मनन कर इस संसार से चल बसें, तो वे चार पाँच विचार ही अनन्त काल तक विश्व में व्याप्त रहेंगे~। वास्तव में ऐसे विचार पर्वतों को भी चीरकर बाहर निकल आएँगे, समुद्रों को पार कर जाएँगे और सारे संसार भर में व्याप्त हो जाएँगे~। वे मनुष्यों के हृदय एवं मस्तिष्क में इतने गहरे घुस जाएँगे कि उनमें से कुछ ऐसे लोग उत्पन्न होंगे, जो उन्हें कार्यरूप में परिणत करेंगे~। ये पूर्वोक्त सात्त्विक व्यक्ति भगवान् के इतने समीप है कि इनके लिए कर्मशील होना, परोपकार करना तथा इस संसार में धर्म प्रचार आदि कार्य करना सचमुच असम्भव सा है~। कर्मी व्यक्ति चाहे जितने भी भले क्यों हों, उनमें कुछ न कुछ अज्ञान रह ही जाता है~। जब हमारे स्वभाव में कुछ न कुछ अपवित्रता अवशिष्ट रहती है, तभी हम कार्य कर सकते हैं~। कर्म के पीछे साधारणतया कोई हेतु या आसक्ति रहना यह तो कर्म के स्वभाव में ही है~। जो एक क्षुद्र पक्षी के पतन तक पर भी दृष्टि रखते हैं, उन सतत क्रियाशील विधाता के समक्ष मनुष्य भला अपने कार्य की इतनी बड़ाई कैसे कर सकता है? जब वे संसार के छोटे से छोटे प्राणी की भी चिन्ता रखते हैं, तब मनुष्य के लिए ऐसा सोचना क्या घोर ईश्वर-निन्दा नहीं है? हमें तो उनके सामने ससम्भ्रम, नतमस्तक खड़े होकर केवल यही कहना चाहिए, ‘आपकी इच्छा पूर्ण हो~।’ सर्वश्रेष्ठ पुरुष तो कार्य कर ही नहीं सकते, क्योंकि उनमें किसी प्रकार की आसक्ति नहीं होती~। जो आत्मा में ही आनन्द करते हैं, जो आत्मा में ही तृप्त रहते हैं और जो आत्मा के साथ सदा के लिए एक हो गये हैं, उनके लिए कोई कर्म शेष नहीं रह जाता~। यही सर्वश्रेष्ठ मानव हैं~। इनके अतिरिक्त अन्य सभी को कर्म करना पड़ेगा~। पर इस प्रकार कर्म करते समय हमें यह कभी न सोचना चाहिए की हम इस संसार में भी किसी छोटे से छोटे प्राणी तक की तनिक भी सहायता कर सकते हैं~। असल में वह हम बिलकुल नहीं कर सकते~। संसाररूपी इस शिक्षालय में परोपकार के इन कार्यों द्वारा तो हम केवल अपनी ही सहायता करते हैं~। कर्म करने का यही सच्चा दृष्टिकोण है~। अतएव यदि हम इसी भाव से कर्म करें, यदि सदा यही सोचें कि इस समय जो हम कार्य कर रहे हैं, वह तो हमारे लिए एक बड़े सौभाग्य की बात है, तो फिर हम कभी भी किसी वस्तु में आसक्त न होंगे~। इस विश्व में हम तुम जैसे लाखों लोग मन ही मन सोचा करते हैं कि हम एक महान व्यक्ति है; परन्तु एक दिन हमारी मौत हो जाती है और बस पाँच मिनट बाद ही संसार हमें भूल जाता है~। किन्तु ईश्वर का जीवन अनन्त है~। “यदि इस सर्वशक्तिमान प्रभु की इच्छा न हो, तो एक क्षण के लिए भी कौन जीवित रह सकता है, एक क्षण के लिए भी कौन साँस ले सकता है?” वे ही सतत कर्मशील विधाता हैं~। समस्त शक्ति उन्हीं की है और उन्हीं की आज्ञावर्तिनी है~।

\begin{verse}
“भयादस्याग्निस्तपति भयात्तपति सूर्यः।\\ भयादिन्द्रश्च वायुश्च मृत्युर्धावति पंचमः॥”\footnote{कठोपनिषद् - २~।३~।३}
\end{verse}

\noindent- उन्हीं की आज्ञा से वायु चलती है, सूर्य प्रकाशित होता है, पृथ्वी अवस्थित है और मृत्यु इस संसार में विचरण करती है~। वे ही सब कुछ हैं~। हम तो उनकी केवल उपासना मात्र कर सकते हैं~। कर्मों के समस्त फलों को त्याग दो, भले के लिए ही भला करो - तभी पूर्ण अनासक्ति प्राप्त होगी~। तब हृदयग्रन्थि छिन्न हो जाएगी और हम पूर्ण मुक्ति प्राप्त कर लेंगे~। यह मुक्ति ही वास्तव में कर्मयोग का लक्ष्य है~।

