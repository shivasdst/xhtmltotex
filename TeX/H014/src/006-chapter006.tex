
\chapter{अनासक्ति ही पूर्ण आत्मत्याग है}

जिस प्रकार हमारे शरीर, मन और वचन द्वारा किया हुआ प्रत्येक कार्य हमारे पास फल के रूप में फिर से वापस आ जाता है, उसी प्रकार हमारे कार्य दूसरे व्यक्तियों पर तथा उनके कार्य हमारे ऊपर अपना प्रभाव डाल सकते हैं~। शायद तुम सभी ने देखा होगा कि जब मनुष्य कोई बुरे कार्य करता है, तो क्रमशः वह अधिकाधिक बुरा बनता जाता है, और इसी प्रकार जब वह अच्छे कार्य करने लगता है, तो दिनोंदिन सबल होता जाता है और उसकी प्रवृत्ति सदैव सत्कार्य करने की ओर झुकती जाती है~। कर्म का प्रभाव इस प्रकार जो इतना जोर पकड़ता जाता है, उसका स्पष्टीकरण केवल एक ही प्रकार से हो सकता है, और वह यह कि एक मन दूसरे मन के ऊपर क्रिया-प्रतिक्रिया द्वारा असर डाल सकता है~। इसे स्पष्ट करने के लिए हम पदार्थविज्ञान से एक दृष्टान्त ले सकते हैं~। जब मैं कोई कार्य करता हूँ, तो कहा जा सकता है कि मेरा मन एक विशिष्ट प्रकार की कम्पनावस्था में होता है; उस समय अन्य जितने मन उस प्रकार की अवस्था में होगे, उनकी प्रवृत्ति यह होगी कि वे मेरे मन से प्रभावित हो जायँ~। यदि एक कमरे में भिन्न-भिन्न वाद्ययन्त्र एक सुर में बाँध दिये जायँ, तो आप सब ने देखा होगा कि एक को छेड़ने से अन्य सबों की भी प्रवृत्ति उसी प्रकार का सुर निकालने की होने लगती है~। इसी प्रकार जो जो मन एक सुर में बँधे हैं, उन सब के ऊपर एक विशेष विचार का समान प्रभाव पड़ेगा~। हाँ, यह सत्य है कि विचार का मन पर यह प्रभाव दूरी अथवा अन्य कारणों से न्यूनाधिक अवश्य हो जाएगा, परन्तु मन पर प्रभाव होने की सम्भावना सदैव बनी रहेगी~। मान लो, मैं एक बुरा कार्य कर रहा हूँ~। उस समय मेरे मन में एक विशेष प्रकार का कम्पन होगा और संसार के अन्य सब मन, जो उसी प्रकार की स्थिति में हैं, सम्भवतः मेरे मन के कम्पन से प्रभावित हो जाएँगे~। इसी प्रकार जब मैं कोई अच्छा कार्य करता हूँ तो मेरे मन में एक दूसरे प्रकार का कम्पन होता है, और उस प्रकार के कम्पनशील सारे मनों पर मेरे मन के प्रभाव पड़ने की सम्भावना रहती है~। एक मन का दूसरे मन पर यह प्रभाव कम्पन की न्यूनाधिक शक्ति के अनुसार कम या अधिक हुआ करता है~।

उपर्युक्त दृष्टान्त को यदि हम कुछ और आगे ले जायँ, तो कह सकते हैं कि जिस प्रकार कभी-कभी आलोक तरंगों को अपनी गन्तव्य वस्तु तक पहुँचने के लिए लाखों वर्ष लग जाते हैं, उसी प्रकार विचार तरंगे भी कभीकभी सैकड़ों वर्ष तक आकाश में भ्रमण करती रहती हैं, जब तक कि अन्त में उन्हें कोई ऐसा पदार्थ नहीं मिल जाता, जिसके साथ वे एकरूप हो कार्य कर सकें~। अतएव यह नितान्त सम्भव है कि हमारा यह वायुमण्डल अच्छी और बुरी दोनों प्रकार की विचार तरंगो से व्याप्त है~। प्रत्येक मस्तिष्क से निकला हुआ प्रत्येक विचार मानो इसी प्रकार भ्रमण करता रहता है, जब तक कि उसे एक योग्य आधार प्राप्त नहीं हो जाता~। और जो मन इस प्रकार के विचार ग्रहण करने के लिए अपने को उन्मुक्त किये हुए है, वह तुरन्त ही उन्हें अपना लेगा~। अतएव जब कोई मनुष्य कोई दुष्कर्म करता है, तो वह अपने मन को किसी एक विशिष्ट सुर में ले आता है; और उसी सुर की जितनी भी तरंगे पहले से ही आकाश में अवस्थित हैं, वे सब उसके मन में घुस जाने की चेष्टा करती हैं~। यही कारण है कि एक दुष्कर्मी साधारणतः अधिकाधिक दुष्कर्म करता जाता है~। उसके कर्म क्रमशः प्रबलतर होते जाते हैं~। यही बात सत्कर्म करनेवाले के लिए भी घटती है; वह अपने को वातावरण की समस्त शुभ तरंगो को ग्रहण करने के लिए मानो खोल देता है और इस प्रकार उसके सत्कर्म अधिकाधिक शक्तिसम्पन्न होते जाते हैं~। अतएव हम देखते हैं कि दुष्कर्म करने में हमें दो प्रकार का भय है~। पहला तो यह कि हम अपने को चारों ओर की अशुभ तरंगों के लिए खोल देते हैं; और दूसरा यह कि हम स्वयं ऐसी अशुभ तरंग का निर्माण कर देते हैं; जिसका प्रभाव दूसरों पर पड़ता है, फिर चाहे वह सैकड़ों वर्ष बाद ही क्यों न हो~। दुष्कर्म द्वारा हम केवल अपना ही नहीं वरन् दूसरों का भी अहित करते हैं, और सत्कर्म द्वारा हम अपना तथा दूसरों का भी भला करते हैं~। मनुष्य की अन्य आभ्यान्तरिक शक्तियों के समान ये शुभ और अशुभ शक्तियाँ भी बाहर से बल संचित करती हैं~।

कर्मयोग के अनुसार, बिना फल उत्पन्न किये कोई भी कर्म नष्ट नहीं हो सकता~। प्रकृति की कोई भी शक्ति उसे फल उत्पन्न करने से रोक नहीं सकती~। यदि मैं कोई बुरा कर्म करूँ, तो उसका फल मुझे भोगना ही पड़ेगा; विश्व में ऐसी कोई ताकत नहीं, जो इसे रोक सके~। इसी प्रकार, यदि मैं कोई सत्कार्य करूँ, तो विश्व में ऐसी कोई शक्ति नहीं, जो उसके शुभ फल को रोक सके~। कारण से कार्य होता ही है; इसे कोई भी रोक नहीं सकता~। अब हमारे सामने कर्मयोग के सम्बन्ध में सूक्ष्म एवं गम्भीर विषय उपस्थित होता है~। हमारे सत् और असत् कर्म आपस में घनिष्ठ रूप से सम्बद्ध हैं; इन दोनों के बीच हम निश्चित रूप से एक रेखा खींचकर यह नहीं बता सकते कि अमुक कार्य नितान्त शुभ है और अमुक अशुभ~। ऐसा कोई भी कर्म नही हैं, जो एक ही समय शुभ और अशुभ दोनों फल न उत्पन्न करे~। यही देखिये, मैं आप लोगों से बात कर रहा हूँ; सम्भवतः आपमें से कुछ लोग सोचते होंगे कि मैं एक भला कार्य कर रहा हूँ~। परन्तु साथ ही साथ शायद मैं हवा में रहनेवाले असंख्य छोटे-छोटे कीटाणुओं को भी नष्ट करता जा रहा हूँ~। और इस प्रकार एक दृष्टि से मैं बुरा भी कर रहा हूँ~। हमारे निकट के लोगों पर, जिन्हें हम जानते हैं, यदि किसी कार्य का प्रभाव शुभ पड़ता है, तो हम उसे शुभ कार्य कहते हैं~। उदाहरणार्थ, आप लोग मेरे इस व्याख्यान को अच्छा कहेंगे, परन्तु वे कीटाणु ऐसा कभी न कहेंगे~। कीटाणुओं को आप नहीं देख रहे हैं, पर अपने आप को देख रहे है~। मेरी वक्तृता का जैसा प्रभाव आप पर पड़ता है, वह आप स्पष्ट देख सकते हैं, किन्तु उसका प्रभाव उन कीटाणुओं पर कैसा पड़ता है, यह आप नहीं जानते~। इसी प्रकार, यदि हम अपने असत् कर्मों का भी विश्लेषण करें, तो हमें ज्ञात होगा कि सम्भवतः उनसे भी कहीं न कहीं किसी न किसी प्रकार का शुभ फल हुआ है~। - “जो शुभ कर्मों में भी कुछ न कुछ अशुभ, तथा अशुभ कर्मों में भी कुछ न कुछ शुभ देखते हैं, वास्तव में उन्हींने कर्म का रहस्य समझा है~।”

\newpage

हाँ, तो इससे हमने क्या सीखा? - यही कि हम चाहे जितना भी प्रयत्न क्यों न करें, ऐसा कोई कर्म नहीं हो सकता, जो सम्पूर्णतः पवित्र हो अथवा सम्पूर्णतः अपवित्र~। यहाँ ‘पवित्रता’ या ‘अपवित्रता’ से हमारा तात्पर्य है अहिंसा या हिंसा~। बिना दूसरों को नुकसान पहुँचाये हम साँस तक नहीं ले सकते~। अपने भोजन का प्रत्येक ग्रास हम किसी दूसरे के मुँह से छीनकर खाते हैं~। यहाँ तक कि हमारा अस्तित्व भी दूसरे प्राणियों के जीवन को हटाकर होता है~। चाहे मनुष्य हो, पशु हो अथवा कीटाणु, किसी न किसी को हटाकर ही हम अपना अस्तित्व स्थिर रखते हैं~। ऐसी दशा में यह स्वाभाविक ही है कि कर्म द्वारा पूर्णता कभी नहीं प्राप्त हो सकती~। हम भले ही अनन्त काल तक कर्म करते रहें, परन्तु इस जटिल संसारव्यूह से कभी छुटकारा न होगा~। हम चाहे निरन्तर कार्य करते रहें, परन्तु इस शुभ और अशुभ कर्मफलरूपी बन्धन का कहीं अन्त न होगा~।

दूसरी विचारणीय बात है - कर्म का उद्देश्य क्या है? हम देखते हैं कि प्रत्येक देश के अधिकांश व्यक्तियों की यह धारणा है कि एक समय ऐसा आएगा, जब यह संसार पूर्णता को प्राप्त हो जाएगा; तब यहाँ न तो किसी प्रकार का रोग रहेगा, न शोक, न दुष्टता, न मृत्यु~। वैसे तो यह एक बड़ा सुन्दर विचार है और एक अज्ञानी को कार्य में प्रेरणा देने के लिए बड़ी अच्छी खुराक है; परन्तु यदि हम क्षण भर भी ध्यानपूर्वक सोचें, तो हमें सहज ही ज्ञात हो जाएगा कि ऐसा कभी नहीं हो सकता~। और यह हो भी कैसे सकता है, जब हम जानते हैं कि भलाई और बुराई एक ही सिक्के के चित और पट हैं? ऐसा भी कहीं हो सकता है कि भलाई हो और उसके साथ बुराई न हो? तब फिर पूर्णता का अर्थ क्या है? सच पूछा जाय, तो ‘पूर्ण जीवन’ शब्द ही स्वविरोधात्मक है~। जीवन तो हमारे एवं प्रत्येक बाह्य वस्तु के बीच एक प्रकार का निरन्तर द्वन्द्वसा है~। प्रतिक्षण हम बाह्य प्रकृति से संघर्ष करते रहते हैं, और यदि उसमें हमारी हार हो जाय, तो हमारा जीवनदीप ही बुझ जाता है~। आहार और हवा के लिए निरन्तर चेष्टा का नाम ही है जीवन। यदि हमें भोजन या हवा न मिले, तो हमारी मृत्यु हो जाती है~। जीवन कोई आसानी से चलनेवाली सरल चीज नहीं है - यह तो एक प्रकार का सम्मिश्रित व्यापार है~। बहिर्जगत् और अन्तर्जगत् का घोर द्वन्द्व ही जीवन कहलाता है~। इस प्रकार यह स्पष्ट है कि जब यह द्वन्द्व समाप्त हो जाएगा, तो जीवन का भी अन्त हो जाएगा~।

\vskip 2pt

उपर्युक्त आदर्श सुख की बात का अर्थ है - इस सांसारिक द्वन्द्व का अन्त हो जाना~। परन्तु तब तो जीवन का भी अन्त हो जाएगा; क्योंकि द्वन्द्व का अन्त उसी समय होता है, जब स्वयं जीवन ही चला जाता है~।

\vskip 2pt

हम यह देख ही चुके हैं कि संसार का उपकार करना अपना ही उपकार करना है~। दूसरों के लिए किये गये कार्य का मुख्य फल है - अपनी स्वयं की आत्मशुद्धि~। दूसरों के प्रति निरन्तर भलाई करते रहने से हम स्वयं को भूलने का प्रयत्न करते रहते हैं~। और यह आत्मविस्मृति ही एक बहुत बड़ी शिक्षा है, जो हमें जीवन में सीखनी है~। मनुष्य मूर्खतावश सोचता है कि वह अपने को सुखी बना सकता है, परन्तु वर्षों के घोर संघर्ष के बाद उसकी आँखें खुलती हैं और वह यह अनुभव करता है कि वास्तविक सुख तो स्वार्थपरता को नष्ट कर देने में है, और अपने को सुखी बनानेवाला अन्य कोई नहीं, केवल वह स्वयं ही है~।

\vskip 2pt

परोपकार का प्रत्येक कार्य, सहानुभूति का प्रत्येक विचार, दूसरों की सहायतार्थ किया गया प्रत्येक कर्म, प्रत्येक भला कार्य हमारे क्षुद्र अहंभाव को प्रतिक्षण घटाता रहता है और हममें यह भावना उत्पन्न करता है कि हम किसी से बड़े नहीं; और इसीलिए ये सब कार्य श्रेष्ठ हैं~। ज्ञानी, भक्त और कर्मी तीनों इस बात पर एकमत हैं~। सर्वोच्च आदर्श है - चिरकाल के लिए सम्पूर्ण रूप से आत्मत्याग, जिसमें किसी प्रकार का ‘मैं’ नहीं, केवल ‘तू’ ही ‘तू’ है~। हमारे जाने या बिना जाने, कर्मयोग हमें इसी लक्ष्य की ओर ले जाता है~।

\vskip 2pt

सम्भव है, एक धर्मप्रचारक निर्गुण की बात सुनकर चौंक उठे~। उसका शायद यही दृढ़ मत हो कि ईश्वर सगुण है और वह अपने निजत्व, अपने स्वतन्त्र व्यक्तितत्त्व को - इस व्यक्तित्त्व के बारे में उसकी धारणा चाहे जैसी भी हो - कायम रखने का इच्छुक हो; परन्तु यदि उसके नीतिविषयक विचार वास्तव में शुद्ध हैं, तो उनका आधार सर्वोच्च आत्मत्याग के अतिरिक्त और कुछ हो ही नहीं सकता~।

यह सम्पूर्ण आत्मत्याग ही सारी नीति की नींव है~। मनुष्य, पशु देवता सब के लिए यही एक मूल भाव है, जो समस्त नैतिक आदर्शों में व्याप्त हैं~।

इस संसार में हमें कई प्रकार के मनुष्य मिलेंगे~। प्रथम तो वे, जो देव-प्रकृति पुरुष कहे जा सकते हैं~। वे पूर्ण आत्मत्यागी होते हैं, अपने जीवन की भी बाजी लगाकर दूसरों का भला करते हैं~। ये सर्वश्रेष्ठ पुरुष हैं~। यदि किसी देश में ऐसे सौ मनुष्य भी रहें, तो उस देश को फिर किसी बात की चिन्ता नहीं~। परन्तु खेद हैं, ऐसे लोग बहुत-बहुत कम है! दूसरें वे साधुप्रकृति मनुष्य हैं, जो दूसरों की भलाई तब तक करते हैं, जब तक उनका स्वयं का कोई नुकसान न हो; और तीसरे वे आसुरी प्रकृति के लोग हैं, जो अपनी भलाई के लिए दूसरों का नुकसान तक करने में नहीं हिचकिचाते~। एक संस्कृत कवि ने चौथी श्रेणी भी बतायी है, जिसको हम कोई नाम नहीं दे सकते~। वे लोग ऐसे होते हैं कि अकारण ही दूसरों का नुकसान करते रहते हैं~। जिस प्रकार सर्वोच्च स्तर पर साधु-महात्मागण भला करने के लिए ही दूसरों का भला करते रहते हैं, उसी प्रकार सब से निम्न स्तर पर ऐसे लोग भी है, जो केवल बुरा करने के लिए ही दूसरों का बुरा करते रहते हैं~। ऐसा करने से उन्हें कोई लाभ नहीं होता - यह तो उनकी प्रकृति ही है~।

संस्कृत में दो शब्द हैं - प्रवृत्ति और निवृत्ति~। प्रवृत्ति का अर्थ है - किसी वस्तु की ओर प्रवर्तन या गमन, और निवृत्ति का अर्थ है - किसी वस्तु से निवर्तन या दूर गमन~। ‘किसी वस्तु की ओर प्रवर्तन’ का ही अर्थ है हमारा यह संसार - यह ‘मैं’ और ‘मेरा’~। इस ‘मैं’ को धनसम्पत्ति, प्रभुत्व, नामयश द्वारा सर्वदा बढ़ाने का यत्न करना, जो कुछ मिले उसी को पकड़ रखना, सारे समय सभी वस्तुओं को इस ‘मैं’-रूपी केन्द्र में ही संग्रहित करना - इसी का नाम है ‘प्रवृत्ति’~। यह प्रवृत्ति ही मनुष्यमात्र का स्वाभाविक भाव है, - चहुँ ओर से जो कुछ मिले, लेना और सब को केन्द्र में एकत्रित करते जाना~। और वह केन्द्र है उसका अपना मधुर ‘अहं’~। जब यह वृत्ति घटने लगती है, जब निवृत्ति का उदय होता है, तभी नीति और धर्म का आरम्भ होता है~। ‘प्रवृत्ति’ और ‘निवृत्ति’ दोनों ही कर्मस्वरूप हैं~। एक असत् कर्म है और दूसरा सत्~। निवृत्ति ही सारी नीति एवं सारे धर्म की नींव है; और इसकी पूर्णता ही सम्पूर्ण ‘आत्मत्याग’ है, जिसके प्राप्त हो जाने पर मनुष्य दूसरों के लिए अपना शरीर, मन, यहाँ तक कि अपना सर्वस्व निछावर कर देता है~। तभी मनुष्य को कर्मयोग में सिद्धि प्राप्त होती है~। सत्कार्यों का यही सर्वोच्च फल है~। किसी मनुष्य ने चाहे एक भी दर्शनशास्त्र न पढ़ा हो, किसी प्रकार ईश्वर में विश्वास न किया हो और अभी भी न करता हो, चाहे उसने अपने जीवन भर में एक बार भी प्रार्थना न की हो, परन्तु केवल सत्कार्यों की शक्ति उसे यदि उस अवस्था में ले जाय, जहाँ वह दूसरों के लिए अपना जीवन और सब कुछ उत्सर्ग करने को तैयार रहे, तो हमें समझना चाहिए कि वह उसी लक्ष्य को पहुँच गया है, जहाँ एक भक्त अपनी उपासना द्वारा तथा एक ज्ञानी अपने ज्ञान द्वारा पहुँचता है~। अतएव आपने देखा, ज्ञानी, कर्मी और भक्त तीनों एक ही स्थान पर पहुँचते हैं - एक ही स्थान पर आकर मिल जाते हैं; और वह स्थान है - आत्मत्याग~। विभिन्न दर्शनों और धर्मों में आपस में कितना ही मतभेद क्यों न हो; जो व्यक्ति अपना जीवन दूसरों के लिए अर्पण करने को उद्यत रहता है, उसके समक्ष सभी मनुष्य ससम्भ्रम उठ खड़े होते हैं - उसके सामने भक्तिभाव से माथा नवाते हैं~। यहाँ किसी प्रकार के मतामत का प्रश्न नहीं है - यहाँ तक कि वे लोग भी, जो धर्मसम्बन्धी समस्त विचारों पर नाक-भौं सिकोड़ते हैं, जब इस प्रकार का सम्पूर्ण आत्मत्यागपूर्ण कोई कार्य देखते हैं, तो उसके प्रति श्रद्धासम्पन्न हुए बिना नहीं रह सकते~। क्या आपने यह नहीं देखा, एक कट्टर मतान्ध ईसाई भी जब एडविन अर्नोल्ड के \enginline{Light of Asia} (एशिया का आलोक) नामक ग्रन्थ को पढ़ता है, तो वह भी बुद्ध के प्रति किस प्रकार श्रद्धालु हो जाता है? और ये वे बुद्ध थे, जिन्होंने किसी ईश्वर का प्रचार नहीं किया, आत्मत्याग के अतिरिक्त जिन्होंने अन्य किसी भी बात का प्रचार नहीं किया~। इसका कारण केवल यह है कि मतान्ध व्यक्ति यह नहीं जानता कि उसका स्वयं का जीवनलक्ष्य और उन लोगों का जीवनलक्ष्य, जिन्हें वह अपना विरोधी समझता है, बिलकुल एक ही है~। एक उपासक अपने हृदय में निरन्तर ईश्वरीभाव एवं साधुभाव रखते हुए अन्त में उस एक ही स्थान पर पहुँचता है और कहता है, “प्रभो, जैसी तेरी मरजी~।” वह अपने नाम से कुछ बचा नहीं रखता~। यही आत्मत्याग है~। एक ज्ञानी भी अपने ज्ञान द्वारा देखता है कि उसका यह तथाकथित भासमान ‘अहं’ केवल एक भ्रम है; और इस तरह वह उसे बिना किसी हिचकिचाहट के त्याग देता है~। यह भी आत्मत्याग के अतिरिक्त और कुछ नहीं है~। अतएव हम देखते हैं कि कर्म, भक्ति और ज्ञान तीनों यहाँ पर आकर मिल जाते हैं~। प्राचीन काल के बड़े-बड़े धर्मप्रचारकों ने जब हमें यह सिखाया था कि “ईश्वर जगत् से भिन्न है, जगत् से परे है,” तो असल में उसका मर्म यही था~। जगत् एक चीज है और ईश्वर दूसरी; और यह भेद बिलकुल सत्य है~। जगत् से उनका तात्पर्य है स्वार्थपरता~। निःस्वार्थता ही ईश्वर है~। एक मनुष्य चाहे रत्नखचित सिंहासन में आसीन हो, सोने के महल में रहता हो, परन्तु यदि वह पूर्ण रूप से निःस्वार्थ है तो वह ब्रह्म में ही स्थित है~। परन्तु एक दूसरा मनुष्य चाहे झोपड़ी में ही क्यों न रहता हो, चिथड़े क्यों न पहनता हो, सर्वथा दीनहीन ही क्यों न हो, पर यदि वह स्वार्थी है, तो हम कहेंगे कि वह संसार में घोर रूप से लिप्त है~।

हाँ तो हम यह कह रहे थे कि बिना कुछ बुरा किये हम न तो भला कर सकते हैं और न बिना कुछ भला किये बुरा ही~। तो अब प्रश्न यह है कि यह सब जानते हुए हम कर्म करें किस प्रकार? इस समस्या की मीमांसा करने के लिए इस संसार में अनेकानेक सम्प्रदाय उठ खड़े हुए, जो बड़ी लापरवाही से यह प्रचार कर गये कि धीरे-धीरे आत्महत्या कर लेना ही इस संसार से निस्तार पाने का एकमात्र उपाय है~। क्योंकि, मनुष्य यदि जीवित रहे, तो अनेक छोटे-छोटे जन्तुओं और पौधों का नाश करके अथवा अन्य किसी न किसी का कुछ न कुछ अनिष्ट करके ही तो रह सकता है~। इसीलिए उनके मतानुसार इस संसारचक्र से छूटने का एकमात्र उपाय है मृत्यु! जैनियों ने अपने सर्वोच्च आदर्श के रूप में इसी का प्रचार किया है~। यह शिक्षा ऊपर से तर्कसंगत तो अवश्य प्रतीत होती है~। परन्तु इसकी ठीक-ठीक मीमांसा गीता में पायी जाती हैं; और वह है + अनासक्ति - अपने जीवन के समस्त कार्य करते हुए भी किसी में आसक्त न होना~। यह जान लो कि संसार में होते हुए भी तुम संसार से नितान्त पृथक् हो और जहाँ जो कुछ भी तुम कर रहे हो, वह अपने लिए नहीं है~। यदि कोई कार्य तुम अपने लिए करोगे, तो उसका फल तुम्हें ही भोगना पड़ेगा~। यदि वह सत्कार्य है, तो तुम्हें उसका अच्छा फल मिलेगा और यदि बुरा है, तो बुरा~। परन्तु जो कोई भी कार्य हो, यदि तुम वह अपने लिए नहीं करते, तो उसका प्रभाव तुम पर नहीं पड़ेगा~। इस भाव को स्पष्ट करने के लिए हमारे शास्त्रों में बड़े सुन्दर ढंग से कहा है, “यदि किसी में यह बोध रहे कि मैं इसे अपने लिए बिलकुल नहीं कर रहा हूँ, तो फिर वह चाहे समस्त संसार की ही क्यों न हत्या कर डाले अथवा स्वयं ही क्यों न हत हो जाय, पर वास्तव में वह न तो हत्या करता है और न हत ही होता है~।” इसीलिए कर्मयोग हमें शिक्षा देता है, “संसार को मत छोड़ो, संसार में ही रहो, जितना चाहो सांसारिक भाव ग्रहण करो~। परन्तु यदि यह सब तुम्हारे ही भोग के लिए हो, तो फिर तुम्हारा कर्म करना व्यर्थ है~।” तुम्हारा लक्ष्य भोग नहीं होना चाहिए~। पहले अहंभाव को नष्ट कर डालो, और फिर समस्त संसार को आत्मस्वरूप देखो~। यही तो प्राचीन ईसाई लोग भी कहा करते थे - ‘वृद्ध मनुष्य को नष्ट कर डालना चाहिए~।’ इस ‘वृद्ध मनुष्य’ का अर्थ है यह स्वार्थपर भाव कि यह संसार हमारे ही भोग के लिए बना है~। अज्ञ माता-पिता अपने बच्चे को यह प्रार्थना करने की शिक्षा देते हैं, “हे प्रभो, तूने यह सूर्य और चन्द्रमा मेरे लिए ही बनाये हैं”, मानो उस ईश्वर को सिवाय इसके कि वह इन बच्चों के लिए यह सब पैदा करता रहे और कोई काम नहीं था! अपने बच्चों को ऐसी मूर्खतापूर्ण शिक्षा मत दो~। फिर एक दूसरे प्रकार के भी मूर्ख लोग हैं, जो हमें सिखाते हैं कि ये सब जानवर हमारे मारने-खाने के लिए ही बनाये गये हैं और यह सारा संसार मनुष्य के भोग के लिए है~। यह सब निरी मूर्खता है~। एक शेर भी कह सकता है कि मनुष्य की उत्पत्ति मेरे ही लिए हुई है और ईश्वर से प्रार्थना कर सकता है, “हे प्रभो, मनुष्य कितना दुष्ट है कि वह अपने को मेरे सामने उपस्थित नहीं कर देता, जिससे मैं उसे खा जाऊँ~। देखिये, मनुष्य आपका नियम भंग कर रहा है~।” यदि संसार की उत्पत्ति हमारे लिए हुई है तो हम भी संसार के लिए ही पैदा किये गये हैं~। यह बड़ी कुत्सित धारणा है कि यह संसार हमारे भोग के लिए ही बनाया गया है, और इसी भयानक धारणा से हम बद्ध रहते हैं~। वास्तव में यह संसार हमारे लिए नहीं हैं~। प्रति वर्ष लाखों लोग इसमें से बाहर चले जाते हैं, परन्तु उधर संसार की कोई नजर तक नहीं~। लाखों फिर आ जाते हैं~। संसार जैसे हमारे लिए है, वैसे ही हम भी संसार के लिए हैं~।

अतएव ठीक ढंग से कर्म करने के लिए यह आवश्यक है कि पहले हम आसक्ति का भाव त्याग दें~। दूसरी बात यह कि हमें अपने-आपको कर्म से एक नहीं कर देना चाहिए~। हम एक साक्षी के समान रहें और अपना काम करते चलें~। मेरे गुरुदेव कहा करते थे, “अपने बच्चों के प्रति वही भावना रखो, जो एक दाई की होती है” वह तुम्हारे बच्चे को गोद में लेती है, उसे खिलाती है और उसको इस प्रकार प्यार करती है, मानो वह उसी का बच्चा हो~। पर ज्यों ही तुम उसे काम से अलग कर देते हो, त्यों ही वह अपना बोराबिस्तर समेट तुरन्त घर छोड़ने को तैयार हो जाती है~। उन बच्चों के प्रति उसका जो इतना प्रेम था, उसे वह बिलकुल भूल जाती है~। एक साधारण दाई को तुम्हारे बच्चों को छोड़कर दूसरे के बच्चों को लेने में तनिक भी दुःख न होगा~। तुम भी अपने बच्चों के प्रति यही भाव धारण करो~। तुम्हीं उनकी दाई हो - और यदि तुम्हारा ईश्वर में विश्वास है, तो विश्वास करो कि ये सब चीजें, जिन्हें तुम अपनी समझते हो, वास्तव में ईश्वर की हैं~। अत्यन्त कमजोरी कभी-कभी बड़ी साधुता और सबलता का रूप धारण कर लेती है~। यह सोचना कि मेरे ऊपर कोई निर्भर है तथा मैं किसी का भला कर सकता हूँ, अत्यन्त दुर्बलता का चिह्न है~। यह अहंकार ही समस्त आसक्ति की जड़ है, और इस आसक्ति से ही समस्त दुःखों की उत्पत्ति होती है~। हमें अपने मन को यह भलीभाँति समझा देना चाहिए कि इस संसार में हमारे ऊपर कोई भी निर्भर नहीं है~। एक भिखारी भी हमारे दान पर निर्भर नहीं~। किसी भी जीव को हमारी दया की आवश्यकता नहीं, संसार का कोई भी प्राणी हमारी सहायता का भूखा नहीं~। सब की सहायता प्रकृति से होती है~। यदि हममें से लाखों लोग न भी रहे, तो भी उन्हें सहायता मिलती रहेगी~। तुम्हारे हमारे न रहने से प्रकृति के द्वार बन्द न हो जाएँगे~। दूसरों की सहायता करके हम जो स्वयं शिक्षा लाभ कर रहे हैं, यही तो हमारे तुम्हारे लिए परम सौभाग्य की बात है~। जीवन में सीखने योग्य यही सब से बड़ी बात है~। जब हम पूर्ण रूप से इसे सीख लेंगे, तो हम फिर कभी दुःखी न होंगे; तब हम समाज में कहीं भी जाकर उठ-बैठ सकते हैं, इससे हमारी कोई हानि न होगी~। तुम चाहे विवाहित हो, तुम्हारे दल के दल नौकर हों, बड़ा भारी राज्य हो, पर यदि तुम इस तत्त्व को हृदय में रखकर कार्य करते हो कि यह संसार मेरे भोग के लिए नहीं है और इसे मेरी सहायता की कतई आवश्यकता नहीं, तो यह सब रहने पर भी तुम्हारा कुछ न बिगड़ेगा~। हो सकता है, इसी साल तुम्हारे कई मित्रों का निधन हो गया हो~। तो क्या भला संसार उनके फिर वापस आने के लिए रुका हुआ है? क्या इसकी गति शिथिल हो गयी है? नहीं, ऐसा नहीं हुआ~। यह तो जारी ही है~। अतएव अपने मन से यह विचार निकाल दो कि तुम्हें इस संसार के लिए कुछ करना हैं~। संसार को तुम्हारी सहायता की तनिक भी आवश्यकता नहीं~। मनुष्यों का यह सोचना निरी मूर्खता है कि वह संसार की सहायता के लिए पैदा हुआ है~। यह केवल अहंकार है~। निरी स्वार्थपरता है, जो धर्म की आड़ में हमारे सामने आती है~। जब तुम्हारे मन में इतना ही नहीं, बल्कि तुम्हारे स्नायुओं और मांसपेशियों तक में यह शिक्षा भलीभाँति भिद जाएगी कि संसार तुम्हारे अथवा अन्य किसी के ऊपर निर्भर नहीं है, तो कर्म से तुम्हें फिर किसी प्रकार की दुःखरूपी प्रतिक्रिया न होगी~। यदि तुम किसी मनुष्य को कुछ दे दो और उससे किसी प्रकार की आशा न करो, यहाँ तक कि उससे कृतज्ञता प्रकाशन की भी इच्छा न करो, तो यदि वह मनुष्य कृतघ्न भी हो, तो भी उसकी कृतघ्नता का कोई प्रभाव तुम्हारे ऊपर न पड़ेगा, क्योंकि तुमने तो कभी किसी बात की आशा ही नहीं की थी और न यही सोचा था कि तुम्हें उससे बदले में कुछ पाने का अधिकार है~। तुमने तो उसे वही दिया, जो उसे प्राप्य था~। उसे वह चीज अपने कर्म से ही मिली, और अपने कर्म से ही तुम उसके दाता बने~। यदि तुम किसी को कोई चीज दो, तो उसके लिए तुम्हें घमण्ड क्यों होना चाहिए? तुम तो केवल उस धन अथवा दान के वाहक मात्र हो, और संसार अपने कर्मों द्वारा उसे पाने का अधिकारी है~। फिर तुम्हें अभिमान क्यों होना चाहिए? जो कुछ तुम संसार को देते हो वह आखिर है ही कितनी? जब तुममें अनासक्ति का भाव आ जाएगा, तब फिर तुम्हारे लिए न तो कुछ अच्छा रह जाएगा, न बुरा~। वह तो केवल स्वार्थपरता ही है, जिसके कारण तुम्हें अच्छाई या बुराई दिख रही है~। यह समझना बहुत कठिन है, परन्तु धीरे-धीरे समझ सकोगे कि संसार की कोई भी वस्तु तुम्हारे ऊपर तब तक अपना प्रभाव नहीं डाल सकती, जब तक कि तुम स्वयं ही उसे अपना प्रभाव डालने दो~। मनुष्य की आत्मा के ऊपर किसी शक्ति का प्रभाव नहीं पड़ सकता, जब तक कि वह मनुष्य स्वयं अपने को गिराकर मूर्ख न बना ले तथा उस शक्ति के वश में न हो जाय~। अतएव अनासक्ति के द्वारा तुम किसी भी प्रकार की शक्ति पर विजय प्राप्त कर सकते हो और उसे अपने ऊपर प्रभाव डालने से रोक सकते हो~। यह कह देना बड़ा सरल है कि जब तक तुम किसी चीज को अपने ऊपर प्रभाव न डालने दो, तब तक वह तुम्हारा कुछ नहीं कर सकती~। परन्तु जो सचमुच अपने ऊपर किसी का प्रभाव नहीं पड़ने देता, तथा बहिर्जगत् के प्रभावों से जो न सुखी होता है, न दुःखी - उसका लक्षण क्या है? वह लक्षण यह है कि सुख अथवा दुःख में उस मनुष्य का मन सदा एकसा रहता है, सभी अवस्थाओं में उसकी मनोदशा समान रहती है~।

भारतवर्ष में एक महापुरुष हो गये हैं~। इनका नाम व्यास था~। ये बहुत बड़े ऋषि थे और वेदान्तसूत्र के प्रणेता के नाम से प्रसिद्ध हैं~। इनके पिता ने पूर्णत्व प्राप्त करने का बहुत यत्न किया था, परन्तु वे असफल रहे~। उनके पितामह तथा प्रपितामह ने भी पूर्णत्व प्राप्ति के लिए बहुत चेष्टा की थी, किन्तु वे भी सफलकाम न हो सके थे~। स्वयं व्यासदेव भी पूर्ण रूप से सफल न हो सके; परन्तु उनके पुत्र शुकदेव जन्म से ही सिद्ध थे~। व्यासदेव अपने पुत्र को तत्त्वज्ञान की शिक्षा देने लगे~। और स्वयं यथाशक्ति शिक्षा देने के बाद उन्होंने शुकदेव को राजा जनक की राजसभा में भेज दिया~। जनक एक बहुत बड़े राजा थे और ‘विदेह’ नाम से प्रसिद्ध थे~। ‘विदेह’ का अर्थ है ‘शरीर से पृथक्’~। यद्यपि वे राजा थे, फिर भी उन्हें इस बात का तनिक भी भान न था कि वे एक शरीरधारी हैं~। उन्हें तो सदा यही ध्यान रहता था कि वे आत्मा हैं~। बालक शुक उनके पास शिक्षा ग्रहण करने के लिए भेजे गये~। इधर राजा को यह मालूम था कि व्यास मुनि का पुत्र उनके पास तत्त्वज्ञान की शिक्षा प्राप्त करने आ रहा है, और इसलिए उन्होंने पहले से ही कुछ प्रबन्ध कर रखा था~। जब बालक राजमहल के द्वार पर आया, तो सन्तरियों ने उसकी ओर कोई विशेष ध्यान नहीं दिया~। उन्होंने बस उसे बैठने के लिए एक आसन भर दे दिया~। इस आसन पर वह बालक लगातार तीन दिन बैठा रहा; न तो कोई उससे कुछ बोला और न किसी ने यही पूछा कि वह कौन है और क्या चाहता है~। बालक शुक इतने बड़े ऋषि के पुत्र थे, उनके पिता का देश भर में सम्मान था और वे स्वयं भी प्रतिष्ठित थे, परन्तु फिर भी उन क्षुद्र सन्तरियों ने उन पर कोई ध्यान न दिया~। इसके बाद अचानक राजा के मन्त्री तथा बड़े-बड़े राज्याधिकारी वहाँ पर आये और उन्होंने उनका अत्यन्त सम्मान के साथ स्वागत किया~। वे उन्हें अन्दर एक सुशोभित गृह में लिवा ले गये, इत्रों से स्नान कराया, सुन्दर वस्त्र पहनाये और आठ दिन तक उन्हें सब प्रकार के विलास में रखा~। परन्तु शुकदेव के प्रशान्त चेहरे पर तनिक भी अन्तर न हुआ~। बालक शुक आज भी विलासों के बीच वैसे ही थे, जैसे कि उस दिन, जब वे महल के द्वार पर बैठे हुए थे~। इसके बाद उन्हें राजा के सम्मुख लाया गया~। राजा सिंहासन पर बैठे थे, और वहाँ नाचगान तथा अन्य आमोद-प्रमोद हो रहे थे~। राजा ने बालक शुक के हाथ में लबालब दूध से भरा हुआ एक प्याला दिया और उनसे कहा, “इसे लेकर इस दरबार की सात बार प्रदक्षिणा कर आओ, पर देखो, एक बूँद भी दूध न गिरे~।” बालक शुक ने दूध का प्याला ले लिया और संगीत की ध्वनि एवं अनेक सुन्दरियों के बीच प्रदक्षिणा करने को उठे~। राजा की आज्ञानुसार वे सात बार चक्कर लगा आये, परन्तु दूध का एक बूँद भी न गिरा~। बालक शुक का अपने मन पर ऐसा संयम था कि बिना उनकी इच्छा के संसार की कोई भी वस्तु उन्हें आकर्षित नहीं कर सकती थी~। प्रदक्षिणा कर चुकने के बाद जब वे दूध का प्याला लेकर राजा के सम्मुख उपस्थित हुए, तो उन्होंने कहा, “वत्स, जो कुछ तुम्हारे पिता ने तुम्हें सिखाया है तथा जो कुछ तुमने स्वयं सीखा है, उससे अधिक मैं तुम्हें और कुछ नहीं सिखा सकता~। तुमने सत्य को जान लिया है~। जाओ, अपने घर वापस जाओ~।”

अतएव हमने देखा कि जिस मनुष्य ने अपने स्वयं के ऊपर अधिकार प्राप्त कर लिया है, उसके ऊपर संसार की कोई भी चीज अपना प्रभाव नहीं डाल सकती, उसके लिए किसी प्रकार का बन्धन शेष नहीं रह जाता~। उसका मन स्वतन्त्र हो जाता है~। और केवल ऐसा ही पुरुष संसार में रहने योग्य है~। बहुधा हम देखते हैं कि लोगों की संसार के सम्बन्ध में दो प्रकार की धारणाएँ होती हैं~। कुछ लोग निराशावादी होते हैं~। वे कहते हैं, “संसार कैसा भयानक है, कैसा दुष्ट है!” दूसरे लोग आशावादी होते हैं और कहते हैं, “अहा! संसार कितना सुन्दर है, कितना अद्भुत है!” जिन लोगों ने अपने मन पर विजय नहीं प्राप्त की है, उनके लिए यह संसार या तो बुराइयों से भरा है, या अधिक से अधिक, अच्छाइयों और बुराइयों का एक मिश्रण है~। परन्तु यदि हम अपने मन पर विजय प्राप्त कर लें, तो यही संसार सुखमय हो जाता है~। फिर हमारे ऊपर किसी भी बात के अच्छे या बुरे भाव का असर न होगा - हमें कहीं भी विशृंखलता दिखायी न देगी, हमारे लिए सभी कुछ सामंजस्यपूर्ण हो जाएगा~। देखा जाता है, जो लोग आरम्भ में संसार को नरक कुण्ड समझते हैं, वे ही यदि आत्मसंयम की साधना में सफल हो जाते हैं, तो इस संसार को ही स्वर्ग समझने लगते हैं~। यदि हम सच्चे कर्मयोगी हैं और इस अवस्था को प्राप्त करने के लिए अपने को शिक्षित करना चाहते हैं, तो हम चाहे जिस अवस्था से आरम्भ करें यह निश्चित है कि हमें अन्त में पूर्ण आत्म-त्याग का लाभ होगा ही~। और ज्यों ही इस कल्पित ‘अहं’ का नाश हो जाएगा, त्यों ही वही संसार, जो हमें पहले अमंगल से भरा प्रतीत होता था, अब स्वर्गस्वरूप और परमानन्द से पूर्ण प्रतीत होने लगेगा~। यहाँ की हवा तक बदलकर मधुमय हो जाएगी और प्रत्येक व्यक्ति भला प्रतीत होने लगेगा~। यही है कर्मयोग की चरम गति, और यही है उसकी पूर्णता या सिद्धि~। हमारे भिन्न भिन्न योग आपस में विरोधी नहीं हैं~। प्रत्येक अन्त में हमें एक ही स्थान में ले जाता है और पूर्णत्व की प्राप्ति करा देता है~। पर हाँ, प्रत्येक का दृढ़ अभ्यास आवश्यक है~। सारा रहस्य अभ्यास में ही है~। पहले श्रवण करो, फिर मनन करो और फिर उसे अमल में लाओ~। यह बात प्रत्येक योग के सम्बन्ध में सत्य है~। पहले तुम इसके बारे में सुनो और समझो कि इसका मर्म क्या है~। यदि कुछ बाते आरम्भ में स्पष्ट न हों, तो निरन्तर श्रवण एवं मनन से वे स्पष्ट हो जाती हैं~। सब बातों को एकदम समझ लेना बड़ा कठिन है~। फिर भी, उनका स्पष्टीकरण आखिर तुम्ही में तो है~। वास्तव में कभी किसी व्यक्ति ने किसी दूसरे को नहीं सिखाया~। हममें से प्रत्येक को अपने-आपको सिखाना होगा~। बाहर के गुरु तो केवल उद्दीपक कारण मात्र हैं, जो हमारे अन्तःस्थ गुरु को सब विषयों का मर्म समझने के लिए उद्बोधित कर देते हैं~। तब बहुतसी बातें हमारी स्वयं की विचारशक्ति से स्पष्ट हो जाती हैं और उनका अनुभव हम अपनी ही आत्मा में करने लगते हैं; और यह अनुभूति ही हमारी प्रबल इच्छाशक्ति में परिणत हो जाती है~। पहले भाव, और फिर इच्छाशक्ति~। इस इच्छाशक्ति से कर्म करने की वह जबरदस्त शक्ति पैदा होती है, जो हमारी प्रत्येक नस, प्रत्येक शिरा और प्रत्येक पेशी में कार्य करती रहती है, जब तक कि हमारा समस्त शरीर इस निष्काम कर्मयोग का एक यन्त्र ही नहीं बन जाता~। और इसके फलस्वरूप हमें अपना वांछित पूर्ण आत्मत्याग एवं परम निःस्वार्थता प्राप्त हो जाती है~। यह प्राप्ति किसी प्रकार के मतामत या विश्वास के ऊपर निर्भर नहीं है~। चाहे कोई ईसाई हो, यहूदी अथवा जेन्टाइल - इससे कोई अन्तर नहीं पड़ता~। प्रश्न तो यह है कि क्या तुम निःस्वार्थ हो? यदि तुम हो, तो चाहे तुमने एक भी धार्मिक ग्रन्थ का अध्ययन न किया हो, चाहे तुम किसी भी गिर्जा या मन्दिर में न गये हो, फिर भी तुम पूर्णता को प्राप्त हो जाओगे~। प्रत्येक योग इसमें समर्थ है कि वह बिना किसी दूसरे योग की सहायता के भी मनुष्य पूर्ण बना दे; क्योंकि उन सब योगों का लक्ष्य एक ही है~। कर्मयोग, ज्ञानयोग तथा भक्तियोग - सभी मुक्तिलाभ के लिए साक्षात् और स्वतन्त्र उपाय हो सकते हैं~। “सांख्य योगो पृथक् बालाः प्रवदन्ति, न पण्डिताः~।” “केवल अज्ञ लोग ही कहते हैं कि कर्म और ज्ञान भिन्न-भिन्न हैं, ज्ञानी लोग नहीं~।” ज्ञानी यह जानता है कि यद्यपि ऊपर से ये लोग एक दूसरे से विभिन्न प्रतीत होते हैं, परन्तु अन्त में वे सब एक ही लक्ष्य में ले जाते हैं, और वह लक्ष्य है पूर्णता~।

