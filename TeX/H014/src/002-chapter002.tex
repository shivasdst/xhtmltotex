
\chapter{अपने-अपने कार्यक्षेत्र में सब बड़े हैं}

सांख्यमत के अनुसार प्रकृति त्रिगुणमयी है~। ये तीन गुण हैं - सत्त्व, रज तथा तम~। बाह्य जगत् में इन तीन गुणों का प्रकाश साम्यावस्था, क्रियाशीलता तथा जड़ता के रूप में दिखायी देता है~। तम की व्याख्या अन्धकार अथवा कर्मशून्यता के रूप में होती है, रज की कर्मशीलता अर्थात् आकर्षण एवं विकर्षण के रूप में, और सत्त्व इन दोनों की साम्यावस्था तथा संयमरूप होता है~।

प्रत्येक व्यक्ति इन तीन गुणों से निर्मित है~। कभी-कभी जब तमोगुण प्रबल होता है, तो हम सुस्त हो जाते हैं, तब मानो हिलडुल तक नहीं सकते और निष्कर्म होकर कुछ विशिष्ट भावनाओं के दास हो जाते हैं~। फिर कभीकभी कर्मशीलता का प्राबल्य होता है; और कभी-कभी इन दोनों के संयमरूप सत्त्व की प्रबलता होती है, जिससे मन शान्त भाव धारण करता है~। फिर, भिन्न-भिन्न मनुष्यों में इन गुणों में से कोई एक सबसे प्रबल होता है~। एक मनुष्य में कर्मशून्यता, सुस्ती और आलस्य के गुण प्रबल रहते हैं; दूसरे में कर्मशीलता, उत्साह एवं शक्ति के, और तीसरे में हम शान्ति, मृदुता एवं माधुर्य का भाव देखते है, जो पूर्वोक्त दोनों गुणों अर्थात् कर्मशीलता एवं कर्मशून्यता का सामंजस्यस्वरूप होता है~। इस प्रकार सम्पूर्ण सृष्टि में - पशुओं, वृक्षों और मनुष्यों में - हमें इन विभिन्न गुणों का, न्युनाधिक मात्रा में, वैशिष्ट्यपूर्ण प्रकाश दिखायी देता है~।

कर्मयोग का सम्बन्ध मुख्यतः इन तीन गुणों से है~। कर्मयोग हमें यह बतलाकर कि उनका स्वरूप क्या है तथा उनका किस प्रकार उपयोग होना चाहिए, हमें अपना कार्य अच्छी तरह से करने की शिक्षा देता है~। मानवसमाज एक श्रेणीबद्ध संस्था है, इसके अन्तर्गत सभी मनुष्य एक-एक श्रेणी में विभक्त और भिन्न-भिन्न सोपान में अवस्थित हैं~। हम सभी जानते हैं कि सदाचार तथा कर्तव्य किसे कहते हैं; परन्तु फिर भी हम देखते हैं कि भिन्नभिन्न देशों में सदाचार के सम्बन्ध में अलग-अलग धारणाएँ हैं~। एक देश में जो बात सदाचारयुक्त मानी जाती है, सम्भव है, दूसरे देश में वही नितान्त दुराचार समझी जाय~। उदाहरणार्थ, एक देश में चचेरे भाई-बहिन आपस में विवाह कर सकते हैं, परन्तु दूसरे देश में यही बात बहुत बुरी मानी जाती है~। किसी देश में लोग अपनी साली से विवाह कर सकते हैं, परन्तु यही बात दूसरे देश में बड़ी खराब समझी जाती है~। फिर कहीं-कहीं लोग एकही बार विवाह कर सकते हैं, और कहीं-कहीं कई बार, इत्यादि-इत्यादि~। इसी प्रकार, सदाचार की अन्यान्य बातों के सम्बन्ध में भी विभिन्न देशों में बहुत भिन्न-भिन्न धारणाएँ रहती हैं~। फिर भी हमारी यह धारणा है कि सदाचार का एक सार्वभौमिक आदर्श अवश्य है~।

यही बात कर्तव्य के बारे में भी है~। भिन्न-भिन्न जातियों में कर्तव्य के बारे में अलग-अलग धारणा होती है~। किसी देश में यदि एक मनुष्य कुछ विशिष्ट कार्य नहीं करता, तो लोग उस पर दोषारोपण करते हैं; परन्तु अन्य किसी देश में यदि वह मनुष्य वही कार्य करता है, तो वहाँ के लोग कहते हैं कि उसने ठीक नहीं किया~। फिर भी हम जानते हैं कि कर्तव्य का एक सार्वभौमिक आदर्श अवश्य है~। इसी प्रकार, एक समाज सोचता है कि कुछ विशिष्ट बातें ही कर्तव्य-स्वरूप हैं; परन्तु दूसरे समाज का विचार बिलकुल विपरीत होता है और वह उन कार्यों को करना एक पातक समझेगा~। अब हमारे सम्मुख दो मार्ग खुले हैं~। एक अज्ञानी का, जो सोचता है कि सत्य का मार्ग केवल एक ही है तथा अन्य सब भ्रमात्मक हैं; और दूसरा ज्ञानी का, जो यह मानता है कि हमारी मानसिक दशा तथा परिस्थिति के अनुसार कर्तव्य तथा सदाचार भिन्न-भिन्न हो सकते हैं~। अतएव जानने योग्य प्रधान बात यह है कि कर्तव्य तथा सदाचार के विभिन्न स्तर होते हैं, और एक अवस्था के, एक परिस्थिति के कर्तव्य दूसरी परिस्थिति के कर्तव्य नहीं हो सकते~।

उदाहरणार्थ सब महापुरुषों का उपदेश है कि “अशुभ के प्रतिकार की चेष्टा नहीं करनी चाहिए” अशुभ का अप्रतिकार ही सर्वोच्च नैतिक आदर्श है~। हम जानते हैं कि यदि हम लोग इस कहावत को पूर्णतः चरितार्थ करने लगे, तो समाज के सारे बन्धन ही छिन्न-भिन्न हो जायँ~। चोर और लुटेरे हमारी जानमाल पर हाथ मारने और मनमानी करने लगे~। यदि इस प्रकार का ‘अप्रतिकार-धर्म’ एक दिन भी आचरण में लाया गया, तो बड़ी गड़बड़ी मच जाएगी~। परन्तु फिर भी अपने हृदय के अन्तस्तल से हम ‘अप्रतिकार’रूप उपदेश की सत्यता भीतर-ही-भीतर अनुभव करते रहते हैं, हमें यह सर्वोच्च आदर्श प्रतीत होता है; परन्तु यदि केवल इस मत का ही प्रचार किया जाय, तब तो अधिकांश मनुष्यों को ही अन्यायकर्मी कहकर तिरस्कृत कर देना होगा~। इतना ही नही, बल्कि इसके द्वारा मनुष्यों को सदा यही अनुभव होने लगेगा कि वे अन्याय ही कर रहे हैं~। उनके हृदय में प्रत्येक कार्य के बारे में संकल्प-विकल्पसा होने लगेगा, उनका मन दुर्बल हो जाएगा तथा अन्य किसी व्यसन की अपेक्षा आत्म-धिक्कार उनमें अधिक दुर्गुणों का संचार कर देगा~। जो व्यक्ति अपने प्रति घृणा करने लगा है, उसके पतन का द्वार खुल चुका है~। और यही बात जाति के सम्बन्ध में भी घटती है~।

हमारा पहला कर्तव्य यह है कि हम अपने प्रति घृणा न करें; क्योंकि आगे बढ़ने के लिए यह आवश्यक है कि पहले हम स्वयं में विश्वास रखें और फिर ईश्वर में~। जिसे स्वयं में विश्वास नहीं, उसे ईश्वर में कभी भी विश्वास नहीं हो सकता~। अतएव हमारे लिए जो एकमात्र रास्ता रह जाता है, वह यह कि हम समझ ले कि कर्तव्य तथा सदाचार की धारणा विभिन्न परिस्थितियों के अनुसार बदलती रहती है~। यह बात नहीं कि जो मनुष्य अशुभ का प्रतिकार कर रहा है, वह कोई ऐसा काम है, जो सदा और स्वभावतः अन्यायपूर्ण है, वरन् भिन्न-भिन्न परिस्थितियों के अनुसार अशुभ का प्रतिकार करना उसका कर्तव्य ही हो सकता है~।

सम्भव है, श्रीमद्भगवद्गीता का द्वितीय अध्याय पढ़कर तुम पाश्चात्य देशवालों में से बहुतों को आश्चर्य हुआ हो, क्योंकि उस अध्याय में भगवान श्रीकृष्ण ने अर्जुन को कपटी तथा डरपोक कहा है और वह इसलिए कि अर्जुन ने अपने सम्बन्धियों तथा मित्रों से यह कहकर लड़ने अथवा संघर्ष करने से इनकार कर दिया था कि ‘अहिंसा परम धर्म है~।’ हमारे लिए समझने की यह एक बड़ी बात है कि प्रत्येक कार्य की दोनों चरम विपरीत अवस्थाएँ एकसदृश दिखायी देती हैं~। चरम ‘अस्ति’ और चरम ‘नास्ति’ दोनों सदैव एक-समान दिखायी देती हैं~। उदाहरणार्थ, आलोक का स्पन्दन यदि अत्यन्त मृदु होता है, तो हम उसे नहीं देख सकते; और इसी प्रकार जब वह अत्यन्त द्रुत होता है, तब भी हम उसे देखने में असमर्थ होते हैं~। ‘शब्द’ के सम्बन्ध में भी ठीक ऐसा ही है~। न तो उसके बहुत मृदु होने पर हम उसे सुन सकते हैं और न उसके बहुत उच्च होने पर~। इसी प्रकार का भेद ‘प्रतिकार’ तथा ‘अप्रतिकार’ में है~। एक मनुष्य इसलिए प्रतिकार नहीं करता कि वह कमजोर है, सुस्त है, असमर्थ है; दूसरी ओर एक दूसरा मनुष्य है, जो यह जानता है कि यदि वह चाहे तो जबरदस्त प्रतिकार कर सकता है, परन्तु फिर भी वह केवल अप्रतिकार ही नहीं करता, वरन् अपने शत्रुओं के प्रति शुभ कामनाएँ भी प्रकट करता है~। अतः वह मनुष्य, जो दुर्बलता के कारण प्रतिकार नहीं करता, पापग्रस्त होता है और इसलिए अप्रतिकार से कोई लाभ नहीं उठा सकता; परन्तु दूसरा मनुष्य यदि प्रतिकार करे, तो वह भी पाप का भागी होता है~। बुद्ध ने जो अपने राजवैभव तथा सिंहासन छोड़ दिया, उसे हम सच्चा त्याग कह सकते हैं; परन्तु एक भिखारी के सम्बन्ध में त्याग का कोई प्रश्न ही नहीं उठता, क्योंकि उसके पास तो त्याग करने को कुछ है ही नहीं~। अतएव जब हम ‘अप्रतिकार’ तथा ‘आदर्श प्रेम’ की बात करते है, तब यह विशेष रूप से नजर रखना आवश्यक है कि हम किस विषय की ओर लक्ष्य कर रहे हैं~। हमें पहले यह ध्यानपूर्वक सोच लेना चाहिए कि हममें प्रतिकार की शक्ति है भी या नहीं~। तब फिर शक्तिशाली होते हुए भी यदि हम प्रतिकार न करें, तो वास्तव में हम एक महान् कार्य करते हैं; परन्तु यदि हम प्रतिकार कर ही न सकते हों, और फिर भी भ्रमवश यही सोचते रहें कि हम उच्च प्रेम की प्रेरणा से ही कार्य कर रहे हैं, तो यह पहले के ठीक विपरीत ही होगा~। अपने विपक्ष में शक्तिशाली सेना को खड़ी देखकर अर्जुन बुजदिल हो गया; उसके ‘प्रेम’ने उसे अपने देश तथा राजा के प्रति अपने कर्तव्य को भुला दिया~। इसीलिए तो भगवान श्रीकृष्ण ने उससे कहा कि तू ढोंगी है, “एक ज्ञानी के सदृश तू बातें तो करता है, परन्तु तेरे कर्म बुजदिल जैसे हैं~। इसलिए तू उठ, खड़ा हो और युद्ध कर~।”

यह है कर्मयोग का असली भाग~। कर्मयोगी वही है, जो समझता है कि सर्वोच्च आदर्श ‘अप्रतिकार’ है, जो जानता है कि यह अप्रतिकार ही मनुष्य की आन्तरिक शक्ति का उच्चतम विकास है और जो यह भी जानता है कि जिसे हम ‘अन्याय का प्रतिकार’ कहते हैं, वह इस अप्रतिकार-रूप उच्चतम शक्ति की प्राप्ति के मार्ग में केवल एक सीढ़ी मात्र है~। इस सर्वोच्च आदर्श को प्राप्त करने के पहले अन्याय का प्रतिकार करना मनुष्य का कर्तव्य है~। पहले वह कार्य करे, युद्ध करे, यथाशक्ति प्रतिद्वन्द्विता करे~। जब उसमें प्रतिकार की यह शक्ति आ जाएगी, केवल तभी ‘अप्रतिकार’ उसके लिए एक गुणस्वरूप होगा~।

अपने देश में एक बार एक व्यक्ति के साथ मेरी मुलाकात हुई~। मैं पहले से ही जानता था कि वह आलसी, बुद्धिहीन और अज्ञ है~। उसे कुछ जानने की स्पृहा भी न थी~। सारांश यह कि वह पशुवत् अपना जीवन व्यतीत करता था~। उसने मुझसे प्रश्न किया, “भगवान की प्राप्ति के लिए मुझे क्या करना चाहिए? मैं किस प्रकार मुक्त हो सकूँगा?” मैंने उससे पूछा, “क्या तुम झूठ बोल सकते हो?” उसने उत्तर दिया, “नहीं~।” मैंने कहा, “तब तुम पहले झूठ बोलना सीखो~। पशुवत् अथवा एक लट्ठे के सदृश जड़वत् जीवन यापन करने की अपेक्षा झूठ बोलना कहीं अच्छा है~। तुम अकर्मण्य हो~। अवश्य तुम उस सर्वोच्च निष्क्रिय अवस्था तक पहुँचे नहीं, जो सब कर्मों से परे और परम शान्तिपूर्ण है~। और तो और, तुम इतने जड़भावापन्न हो कि एक बुरा कार्य करने की भी तुममें ताकत नहीं!” अवश्य इतने तामसिक पुरुष बहुधा नहीं होते, और सच पूछो तो मैं उससे मजाक ही कर रहा था~। पर मेरा मतलब यह था कि सम्पूर्ण निष्क्रिय अवस्था या शान्तभाव प्राप्त करने के लिए मनुष्य को कर्मशीलता में से होकर जाना होगा~।

आलस्य का प्रत्येक दशा में त्याग करना चाहिए~। क्रियाशीलता का अर्थ है ‘प्रतिकार’~। मानसिक तथा शारीरिक समस्त दुर्बलताओं का प्रतिकार करो; और जब तुम इस प्रतिकार में सफल हो जाओगे, तभी शान्ति प्राप्त होगी~। यह कहना बड़ा सरल है कि ‘किसी से घृणा मत करो; किसी अशुभ का प्रतिकार मत करो’, परन्तु हम जानते हैं कि इसे कार्यरूप में परिणत करना क्या चीज है~। जब सारे समाज की आँखें हमारी ओर फिरती हैं, तो हम अप्रतिकार का स्वांग भले ही करें, परन्तु हमारे हृदय में प्रतिकार-वासना की टोंच सदैव बनी रहती है~। यथार्थ अप्रतिकार के भाव से हृदय में जो शान्ति अनुभूत होती है, उसका अभाव हमें निरन्तर खलता रहता है; हमें ऐसा लगता है कि प्रतिकार करना ही अच्छा है~। यदि तुम्हें धन की इच्छा है और साथ ही तुम्हें यह भी मालूम है कि जो मनुष्य धन का इच्छुक है, उसे संसार दुष्ट कहता है, तो सम्भव है, तुम धन प्राप्त करने के लिए प्राणपण से चेष्टा करने का साहस न करो, परन्तु फिर भी तुम्हारा मन दिन-रात धन के पीछे-ही-पीछे दौड़ता रहेगा~। पर यह तो सरासर कपटता है और इससे कोई लाभ नहीं होता~। संसार में कूद पड़ो, और जब तुम इसके समस्त सुख और दुःख भोग लोगे, तभी त्याग आएगा - तभी शान्ति प्राप्त होगी~। अतएव प्रभुत्व-लाभ की अथवा अन्य जो कुछ तुम्हारी वासना हो, वह सब पहले पूरी कर लो; और जब तुम्हारी सारी वासनाएँ पूर्ण हो जाऐंगी तब एक समय ऐसा आएगा, जब तुम्हें यह मालूम हो जाएगा कि वे सब चीजें बहुत छोटी हैं~। परन्तु जब तक तुम्हारी वह वासना तृप्त नहीं होती, जब तक तुम उस कर्मशीलता में से होकर नहीं जा चुकते, तब तक तुम्हारे लिए उस शान्तभाव एवं आत्मसमर्पण तक पहुँचना नित्तान्त असम्भव है~। यह अहिंसा-तत्व, यह ‘अप्रतिकार-धर्म’ गत हजारों वर्षों से प्रचारित होता आया है - प्रत्येक व्यक्ति ही इसके बारे में बचपन से सुनता आया है, परन्तु फिर भी आज संसार में हमें बहुत कम लोग दिखायी देते हैं, जो वास्तव में उस स्थिति तक पहुँच सके हैं~। मैंने लगभग आधे संसार का भ्रमण कर डाला है, परन्तु मुझे शायद ऐसे बीस मनुष्य भी नहीं मिले, जो वास्तव में शान्त तथा अहिंसा प्रकृतिवाले हों~।

प्रत्येक मनुष्य का कर्तव्य है कि वह अपना आदर्श लेकर उसे अपने जीवन में ढालने का प्रयत्न करे~। बजाय इसके कि वह दूसरों के आदर्शों को लेकर चरित्र गढ़ने की चेष्टा करे, अपने ही आदर्श का अनुसरण करना सफलता का अधिक निश्चित मार्ग है~। सम्भव है, दूसरे का आदर्श वह अपने जीवन में ढालने में कभी समर्थ न हो~। उदाहरणार्थ, यदि हम एक छोटे बच्चे से एकदम बीस मील चलने को कह दें, तो या तो वह बेचारा मर जाएगा, या यदि हजार में से एक-आध रेंगता-राँगता कहीं पहुँचा भी, तो वह अधमरा हो जाएगा~। बस हम भी संसार के साथ ऐसा ही करने का प्रयत्न करते हैं~। किसी समाज के सब स्त्री-पुरुष न एक मन के होते हैं, न एक ही योग्यता के और न एक ही शक्ति के~। अतएव, उनमें से प्रत्येक का आदर्श भी भिन्न-भिन्न होना चाहिए; और इन आदर्शों में से एक का भी उपहास करने का हमें कोई अधिकार नहीं~। अपने आदर्श को प्राप्त करने के लिए प्रत्येक को जितना हो सके यत्न करने दो~। फिर यह भी ठीक नहीं कि मैं तुम्हारे अथवा तुम मेरे आदर्श द्वारा जाँचे जाओ~। आम की तुलना इमली से नहीं होनी चाहिये और न इमली की आम से। आम की तुलना के लिए आम ही लेना होगा, और इमली के लिए इमली~। इसी प्रकार हमें अन्य सब के सम्बन्ध में भी समझना चाहिए~।

बहुत्व में एकत्व ही सृष्टि का नियम है~। प्रत्येक स्त्री-पुरुष में व्यक्तिगत रूप से कितना भी भेद क्यों न हो, उन सब के पीछे वह एकत्व ही विद्यमान है~। स्त्री-पुरुषों के भिन्न-भिन्न चरित्र एवं उनकी अलग-अलग श्रेणियाँ सृष्टि की स्वाभाविक विभिन्नता मात्र हैं~। अतएव एक ही आदर्श द्वारा सब की जाँच करना अथवा सब के सामने एक ही आदर्श रखना किसी भी प्रकार उचित नहीं है~। ऐसा करने से केवल एक अस्वाभाविक संघर्ष उत्पन्न हो जाता है और फल यह होता है कि मनुष्य स्वयं से ही घृणा करने लगता है तथा धार्मिक एवं उच्च बनने से रुक जाता है~। हमारा कर्तव्य तो यह है कि हम प्रत्येक को उसके अपने उच्चतम आदर्श को प्राप्त करने के लिए प्रोत्साहित करें, तथा उस आदर्श को सत्य के जितना निकटवर्ती हो सके लाने की चेष्टा करें~।

हम देखते हैं कि हिन्दू नीतिशास्त्र में यह तत्त्व बहुत प्राचीन काल से ही\break अपनाया जा चुका है; और हिन्दुओं के धर्मशास्त्र तथा नीति-सम्बन्धी पुस्तकों में\break ब्रह्मचर्य, गृहस्थ, वानप्रस्थ तथा संन्यास इन सब विभिन्न आश्रमों के लिए भिन्न-भिन्न विधियों का वर्णन है~।

हिन्दू शास्त्रों के अनुसार मानवजाति के साधारण कर्तव्यों के\break अतिरिक्त प्रत्येक मनुष्य के जीवन में कुछ विशेष-विशेष कर्तव्य होते हैं~। एक हिन्दू\break को पहले ब्रह्मचर्याश्रम अर्थात् छात्रजीवन का अवलम्बन करना पड़ता है; उसके बाद वह विवाह करके गृहस्थ हो जाता है; वृद्धावस्था में गृहस्थाश्रम से अवकाश लेकर वह वानप्रस्थ धर्म का अवलम्बन करता है; और अन्त में वह संसार को त्यागकर संन्यासी हो जाता है~।

जीवन के इन भिन्न-भिन्न आश्रमों में भिन्न-भिन्न कर्तव्य होते हैं~। वास्तव में इन आश्रमों में कोई किसी से श्रेष्ठ नहीं है; एक गृहस्थ का जीवन भी उतना ही श्रेष्ट है, जितना कि एक ब्रह्मचारी का, जिसने अपना जीवन धर्मकार्य के लिए उत्सर्ग कर दिया है~। सड़क का भंगी भी उतना ही उच्च तथा श्रेष्ठ है, जितना कि एक सिंहासनारूढ़ राजा~। थोडी देर के लिए उसे गद्दी पर से उतार दो और उसे मेहतर का काम दो, फिर देखें, वह कैसा काम करता है~। इसी प्रकार उस मेहतर को राजा बना दो; देखें, वह कैसे राज्य चलाता है~। यह कहना व्यर्थ है कि ‘गृहस्थ से संन्यासी श्रेष्ठ है~।’ संसार को छोड़कर, स्वच्छन्द और शान्त जीवन में रहकर ईश्वरोपासना करने की अपेक्षा संसार में रहते हुए ईश्वर की उपासना करना बहुत कठिन है~। आज तो भारत में जीवन के ये चार आश्रम घटकर केवल दो ही रह गये हैं - गृहस्थ एवं संन्यास~। गृहस्थ विवाह करता है और नागरिक बनकर अपने कर्तव्यों का पालन करता है; तथा संन्यासी अपनी समस्त शक्तियों को केवल ईश्वरोपासना एवं धर्मोपदेश में लगा देता है~।

मैं अब महानिर्वाण-तन्त्र से गृहस्थ के कर्तव्य-सम्बन्धी कुछ श्लोक उद्धृत करता हूँ~। यह सुनकर तुम देखोगे कि किसी व्यक्ति के लिए गृहस्थ होकर अपने सब कर्तव्यों का उचित रूप से पालन करना कितना कठिन है~।

\begin{verse}
ब्रह्मनिष्ठो गृहस्थः स्यात् ब्रह्मज्ञानपरायणः।\\ यद्यत्कर्म प्रकुर्वीत तद्ब्रह्मणि समर्पयेत्॥ - ८~।२३
\end{verse}

\newpage

गृहस्थ को ब्रह्मनिष्ठ होना चाहिए तथा ब्रह्मज्ञान का लाभ ही उसके जीवन का चरम लक्ष्य होना चाहिए~। परन्तु फिर भी उसे निरन्तर अपने सब कर्म करते रहना चाहिए - अपने कर्तव्यों का पालन करते रहना चाहिए; और अपने समस्त कर्मों के फलों को ईश्वर को अर्पण कर देना चाहिए~।

कर्म करके कर्मफल की आकांक्षा न करना, किसी मनुष्य की सहायता करके उससे किसी प्रकार की कृतज्ञता की आशा न रखना, कोई सत्कर्म करके भी इस बात की ओर नजर तक न देना कि वह हमें यश और कीर्ति देगा अथवा नहीं, इस संसार में सबसे कठिन बात है~। संसार जब तारीफ करने लगता है, तब एक निहायत बुजदिल भी बहादुर बन जाता है~। समाज के समर्थन तथा प्रशंसा से एक मूर्ख भी वीरोचित कार्य कर सकता है; परन्तु अपने आसपास के लोगों की निन्दा-स्तुति की बिलकुल परवाह न करते हुए सर्वदा सत्कार्य में लगे रहना वास्तव में सबसे बड़ा त्याग है~।

\begin{verse}
न मिथ्या भाषणं कुर्यात् न च शाठ्यं समाचरेत्।\\ देवतातिथिपूजासु गृहस्थो निरतो भवेत्॥ -८~।२४
\end{verse}

गृहस्थ का प्रधान कर्तव्य जीविकोपार्जन करना है, परन्तु उसे ध्यान रखना चाहिए कि वह झूठ बोलकर, दूसरों को धोखा देकर तथा चोरी करके ऐसा न करे, और उसे यह भी याद रखना चाहिए कि उसका जीवन ईश्वरसेवा तथा गरीबों के लिए ही है~।

\begin{verse}
मातरं पितरञ्चैव साक्षात् प्रत्यक्षदेवताम्।\\ मत्वा गृही निषेवेत सदा सर्वप्रयत्नतः॥ -८।२५
\end{verse}

यह समझकर कि माता और पिता ईश्वर के साक्षात् रूप हैं, गृहस्थ को चाहिए कि वह उन्हें सदैव सब प्रकार से प्रसन्न रखे~।

\begin{verse}
तुष्टायां मातरि शिवे तुष्टे पितरि पार्वति।\\ तव प्रीतिर्भवेद्देवि परब्रह्म प्रसीदति॥ -८।२६
\end{verse}

यदि उसके माता-पिता प्रसन्न रहते हैं, तो ईश्वर उसके प्रति प्रसन्न होते हैं~।

\begin{verse}
औद्धत्यं परिहासं च तर्जनं परिभाषणम्~।\\ पित्रोरग्रे न कुर्वीत यदीच्छेदात्मनो हितम्~॥\\ मातरं पितरं वीक्ष्य नत्वोत्तिष्ठेत् ससंभ्रमः~।\\ विनाज्ञया नोपविशेत् संस्थितः पितृशासने~॥ - ८।३०-३१
\end{verse}

अपने माता-पिता के सम्मुख औद्धत्य, परिहास, चंचलता अथवा क्रोध प्रकट न करे~। वह पुत्र वास्तव में श्रेष्ठ है, जो अपने माता-पिता के प्रति एक भी कटु शब्द नहीं कहता~। माता-पिता के दर्शन कर उसे चाहिए कि वह उन्हें आदरपूर्वक प्रणाम करे~। उनके आने पर वह खड़ा हो जाय और जब तक वे उससे बैठने को न कहें, तब तक न बैठे~।

\begin{verse}
मातरं पितरं पुत्रं दारानतिथिसोदरान्~।\\ हित्वा गृही न भुञ्जीयात् प्राणैः कण्ठगतैरपि~॥ - ८।३३
\end{verse}

\begin{verse}
वञ्चयित्वा गुरून् बन्धून् यो भुङ्क्ते स्वोदरम्भरः~।\\ इहैव लोके गर्ह्योऽसौ परत्र नारकी भवेत्~॥ - ८।३४
\end{verse}

जो गृहस्थ अपने माता, पिता, बच्चों, स्त्री तथा अतिथि को बिना भोजन कराये स्वयं कर लेता है, वह पाप का भागी होता है~।

\begin{verse}
जनन्या वर्धितो देहो जनकेन प्रयोजितः~।\\ स्वजनैः शिक्षितः प्रीत्या सोऽधमस्तान् परित्यजेत्~॥ \\ एषामर्थे महेशानि कृत्वा कष्टशतान्यपि~।\\ प्रीणयेत् सततं शक्त्या धर्मो ह्येष सनातनः~॥ - ८।३६-३७
\end{verse}

पिता-माता द्वारा ही यह शरीर उत्पन्न हुआ है, अतएव उन्हें प्रसन्न करने के लिए मनुष्य को हजार-हजार कष्ट भी सहने चाहिए~।

\begin{verse}
न भार्यां ताडयेत् क्वापि मातृवत् पालयेत् सदा~।\\ न त्यजेत् घोरकष्टेऽपि यदि साध्वी पतिव्रता~॥\\ स्थितेषु स्वीयदारेषु स्वियमन्यां न संस्पृशेत् \\ दुष्टेन चेतसा विद्वान् अन्यथा नारकी भवेत्~॥\\ विरले शयने वासं त्यजेत् प्राज्ञः परस्त्रिया~।\\ अयुक्तभाषणञ्चैव स्त्रियं शौर्यं न दर्शयेत्~॥\\ धनेन वाससा प्रेम्णा श्रद्धयामृतभाषणैः~।\\ सततं तोषयेत् दारान् नाप्रियं क्वचिदाचरेत्~॥ - ८।३९-४२
\end{verse}

\delimiter

\begin{verse}
यस्मिन्नरे महेशानि तुष्टा भार्या पतिव्रता~।\\ सर्वो धर्मः कृतस्तेन भवतीप्रिय एव सः~॥ - ८।४४
\end{verse}

इसी प्रकार मनुष्य का अपनी स्त्री के प्रति भी कर्तव्य है~। गृहस्थ को अपनी स्त्री को कभी धुड़कना न चाहिए~। और उसका मातृवत् पालन करना चाहिए~। यदि उसकी स्त्री साध्वी और पतिव्रता है, तो वह कष्ट में भी उसका त्याग न करे~। जो मनुष्य अपनी स्त्री के अतिरिक्त किसी दूसरी स्त्री का कलुषित मन से चिन्तन करता है, वह घोर नरक में जाता है~। ज्ञानी मनुष्य को चाहिए कि वह पर स्त्री के साथ निर्जन में शयन या वास न करे~। स्त्रियों के सम्मुख अनुचित वाक्य न कहे, और न ‘मैंने यह किया, वह किया’ आदि कहकर अपने मुख से अपनी बड़ाई ही करे~। अपनी स्त्री को धन, वस्त्र, प्रेम, श्रद्धा एवं अमृततुल्य वाक्य द्वारा प्रसन्न रखे और उसे किसी प्रकार क्षुब्ध न करे~। हे पार्वती, जो पुरुष अपनी पतिव्रता स्त्री का प्रेमभाजन बनने में सफल होता है, उसे समझो कि अपने स्वधर्म के आचरण में सफलता मिल गयी~। ऐसा व्यक्ति तुम्हारा प्रिय होता है~।

\begin{verse}
चतुर्वर्षावधि सुतान् लालयेत् पालयेत् सदा~।\\ ततः षोडषपर्यन्तं गुणान् विद्याञ्च शिक्षयेत्~॥\\ विशत्यब्दाधिकान् पुत्रान् प्रेरयेत् गृहकर्मसु~।\\ ततस्तांस्तुल्यभावेन मत्वा स्नेहं प्रदर्शयेत्~॥\\ कन्याप्येव पालनीया शिक्षणीयातियत्नतः~।\\ देया वराय विदुषे धनरत्नसमन्विता~॥ ८।४५-४७
\end{verse}

पुत्र-कन्या के प्रति गृहस्थ के निम्नलिखित कर्तव्य हैं:-

चार वर्ष की अवस्था तक पुत्रों का खूब लाड़-प्यार करना चाहिए, फिर सोलह वर्ष की अवस्था तक उन्हें नानाविध सद्गुणों और विद्याओं की शिक्षा देनी चाहिए~। जब वे बीस वर्ष के हो जायँ, तो उन्हें किसी गृहकर्म में लगा देना चाहिए~। तब पिता को चाहिए कि वह उन्हें अपनी बराबरी का समझकर उनके प्रति स्नेह-प्रदर्शन करे~। ठीक इसी तरह कन्याओं का भी लालन-पालन करना चाहिए; उनकी शिक्षा बहुत ध्यानपूर्वक होनी चाहिए, और जब उनका विवाह हो, तो पिता को उन्हें धन-आभूषणादि देने चाहिए~।

\begin{verse}
एवं क्रमेण भ्रातृंश्च स्वसृभ्रातृसुतानपि~।\\ ज्ञातीन् मित्राणि भृत्यांश्च पालयत्तोषयेद् गृही~॥\\ ततः स्वधर्मनिरतानेकग्रामनिवासिनः~।\\ अभ्यागतानुदासीनान् गृहस्थः परिपालयेत्~॥\\ यद्येवं नाचरेद्देवि गृहस्थो विभवे सति~।\\ पशुरेव स विज्ञेयः स पापी लोकगर्हितः~॥ ८।४८-५०
\end{verse}

इसी प्रकार गृहस्थ को अपने भाई-बहन, भतीजे, भांजे तथा अन्य सगेसम्बन्धी, मित्र एवं नौकरों का भी पालन करना चाहिए और उन्हें सन्तुष्ट रखना चाहिए~। फिर गृहस्थ को यह भी चाहिए कि वह स्वधर्मरत अपने ग्रामवासियों, अभ्यागतों और उदासीनों का पालन करे~। हे देवि, धनसम्पन्न होते हुए भी जो गृहस्थ अपने कुटुम्बियों तथा निर्धनों की सहायता नहीं करता, वह निन्दनीय और पापी है, उसे तो पशुतुल्य ही समझना चाहिए~।

\begin{verse}
निद्रालस्यं देहयत्नं केशविन्यासमेव च~।\\ आसक्तिमशने वस्त्रे नातिरिक्तं समाचरेत्~॥\\ युक्ताहारो युक्तनिद्रो मितवाङ् मितमैथुनः~।\\ स्वच्छो नम्रः शुचिर्दक्षो युक्तः स्यात् सर्वकर्मसु~॥ ८।५१-५२
\end{verse}

गृहस्थ को अत्यन्त निद्रा, आलस्य, देह की सेवा, केशविन्यास तथा भोजन-वस्त्र में आसक्ति का त्याग करना चाहिए~। उसे आहार, निद्रा, भाषण, मैथुन इत्यादि सब बातें परिमित रूप से करनी चाहिए~। उसे अकपट, नम्र, बाह्याभ्यन्तरशौच\-सम्पन्न, निरालस्य और उद्योगशील होना चाहिए~।

\begin{verse}
शूरः शत्रौ विनीतः स्यात् बान्धवे गुरुसन्निधौ~॥ ८।५३
\end{verse}

गृहस्थ को अपने शत्रु के सामने शूर होना चाहिए और गुरु एवं बन्धुजनों के समक्ष नम्र~।

शत्रु के सम्मुख शूरता प्रकट करके उसे उस पर शासन करना चाहिए~। यह गृहस्थ का आवश्यक कर्तव्य है~। गृहस्थ को घर में कोने में बैठकर रोना और ‘अहिंसा परमो धर्मः’ कहकर खाली गाल न बजाना चाहिए~। यदि वह शत्रु के सम्मुख वीरता नहीं दिखाता है, तो वह अपने कर्तव्य की अवहेलना करता है~। किन्तु अपने बन्धु-बान्धव, आत्मीय-स्वजन एवं गुरु के निकट उसे गौ के समान शान्त एवं निरीह भाव अवलम्बन करना चाहिए~।

\begin{verse}
जुगुप्सितान् न मन्येत नावमन्येत मानिनः। - ८।५३
\end{verse}

निन्दित असत् व्यक्ति को वह सम्मान न दे और सम्माननीय व्यक्ति का आदर करे~।

असत् व्यक्ति के प्रति सम्मान प्रदर्शित करना गृहस्थ का कर्तव्य नहीं है, क्योंकि ऐसा करने से वह असद् विषय को आश्रय देता है~। और यदि सम्मानयोग्य व्यक्ति को वह सम्मान नहीं देता है, तो भी बड़ा अन्याय करता है~।

\begin{verse}
सोहार्दं व्यहारारांश्च प्रवृत्तिं प्रकृतिं नृणाम्~।\\ सहवासेन तर्केश्च विदित्वा विश्वसेत्ततः॥ - ८।५४
\end{verse}

एक साथ रहकर, विशेष निरीक्षण के द्वारा वह पहले मनुष्य का स्नेह, व्यवहार, प्रवृत्ति और प्रकृति जान ले, फिर उस पर विश्वास करे~।

ऐरे-गैरे जिस किसी भी व्यक्ति के साथ वह मित्रता न कर बैठे~। जिसके साथ उसे मित्रता करने की इच्छा हो, उसके कार्यकलाप तथा अन्य लोगों के साथ उसके व्यवहार की वह पहले भली-भाँति जाँच कर ले और फिर उससे मित्रता करे~।

\begin{verse}
स्वीयं यशः पौरुषं च गुप्तये कथितं च यत्।\\ कृतं यदुपकाराय धर्मज्ञो न प्रकाशयेत्~॥ ८।५६
\end{verse}

धर्मज्ञ गृही व्यक्ति को चाहिए कि वह अपना यश, पौरुष, दूसरों की बतायी हुई गुप्त बात तथा दूसरों के प्रति उसने जो कुछ उपकार किया है, इन सबका वर्णन सर्वसाधारण के सम्मुख न करे~।

उसे अपने वैभव अथवा अभाव आदि की भी बात नहीं करनी चाहिए~। उसे अपने धन पर गर्व करना उचित नहीं~। ऐसे विषय वह गुप्त ही रखे~। यही उसका धर्म है~। यह केवल सांसारिक अभिज्ञता नहीं है; यदि कोई मनुष्य ऐसा नहीं करता, तो वह दुर्नीतिपरायण कहा जा सकता है।

गृहस्थ सारे समाज की नींव-सदृश है; वही मुख्य धन उपार्जन करनेवाला होता है~। निर्धन, दुर्बल, स्त्री-बच्चे आदि जो सब कार्य करने योग्य नहीं है, वे गृहस्थ के ऊपर ही निर्भर रहते हैं~। अतएव गृहस्थ को कुछ कर्तव्य करने पड़ते हैं~। और ये कर्तव्य ऐसे होने चाहिए कि उनका साधन करते-करते वह अपने हृदय में शक्ति का विकास अनुभव करे और ऐसा न सोचे कि वह अपने आदर्शानुसार कार्य नहीं कर रहा है~। इसी कारण -

\begin{verse}
जुगुप्सितप्रवृत्तौ च निश्चतेऽपि पराजये~।\\ गुरुणा लघुना चापि यशस्वी न विवादयेत्~॥ - ८।५७
\end{verse}

यदि उसने कोई अन्याय अथवा निन्दित कार्य कर डाला है, तो उसे दूसरों के सम्मुख प्रकट नहीं करना चाहिए~। इसी प्रकार यदि वह ऐसी किसी बात में लगा है, जिसमें वह अपनी सफलता निश्चित मानता है, तो उसे उसकी भी चर्चा नहीं करनी चाहिए~। इस प्रकार आत्मदोष प्रकट करने से कोई लाभ तो होता नहीं, बल्कि उलटा इसके द्वारा मनुष्य हतोत्साहित हो जाता है, और इस प्रकार उसके कर्तव्य-कर्मों में बाधा पड़ती है~। उसने जो अन्याय कर्म किया है, उसका फल तो उसे भोगना ही पड़ेगा~। किन्तु उसे फिर ऐसी चेष्टा करनी चाहिए, जिससे वह सत्कर्म कर सके~। संसार सर्वदा शक्तिमान् तथा दृढ़चित्त व्यक्ति के प्रति ही सहानुभूति प्रकट करता है।

\begin{verse}
विद्याधनयशोधर्मान् यतमान उपार्जयेत्~।\\ व्यसनं चासतां संगं मिथ्याद्रोहं परित्यजेत्~॥ - ८।५८
\end{verse}

उसे चाहिए कि वह यत्नपूर्वक विद्या, धन, यश और धर्म का उपार्जन करे तथा व्यसन (द्यूत-क्रीड़ा आदि), कुसंग, मिथ्याभाषण एवं परद्रोह का परित्याग करे~।

उसे सबसे पहले ज्ञानलाभ के लिए चेष्टा करनी चाहिए~। फिर उसे धनोपार्जन के लिए भी यत्न करना चाहिए~। यही उसका कर्तव्य है, और यदि वह अपने इस कर्तव्य को नहीं करता, तो उसकी गणना मनुष्यों में नहीं~। जो गृहस्थ धनोपार्जन की चेष्टा नहीं करता, वह दुर्नीतिपरायण एवं निकम्मा है~। यदि वह आलस्यभाव से जीवन यापन करता है और उसी में सन्तुष्ट रहता है, तो वह असत्-प्रकृतिवाला है; क्योंकि उसके ऊपर अनेकों व्यक्ति निर्भर रहते हैं~। यदि वह यथेष्ट धन उपार्जन करता है, तो उससे सैकड़ों का पालन-पोषण होता है~।

यदि तुम्हारे इस शहर में सैकड़ों लोगों ने धनी बनने की चेष्टा न की होती, तो यह सभ्यता, ये अनाथाश्रम और ये हवेलियाँ कहाँ से आतीं?

ऐसी दशा में धनोपार्जन करना कोई अन्याय नहीं है, क्योंकि यह धन वितरण के लिए ही होता है~। गृहस्थ ही समाज-जीवन का केन्द्र है~। उसके लिए धन कमाना तथा उसका सत्कर्मों में व्यय करना ही उपासना है~। जिस प्रकार एक संन्यासी को अपनी कुटी में बैठकर की हुई उपासना उसके मुक्तिलाभ में सहायक होती है, उसी प्रकार एक गृहस्थ की भी सदुपाय तथा सदुद्देश्य से धनी होने की चेष्टा उसके मुक्तिलाभ में सहायक होती है; क्योंकि इन दोनों में ही हम, ईश्वर तथा जो कुछ ईश्वर का है, उस सबके प्रति भक्ति से उत्पन्न हुए आत्म-समर्पण एवं आत्मत्याग का ही प्रकाश पाते है; भेद है केवल प्रकाश के रूप भर में~।

बहुधा देखा जाता है कि लोग ऐसे कार्यों में प्रवृत्त हो जाते है, जो उनकी शक्ति के बाहर होते हैं~। इसका फल यही होता है कि उन्हें फिर अपनी उद्देश्य सिद्धि के लिए दूसरों को धोखा देना पड़ता है~।

फिर -

\begin{verse}
अवस्थानुगताश्चेष्टाः समयानुगताः क्रियाः।\\ तस्मादवस्थां समयं वीक्ष्य कर्म समाचरेत्~॥ - ८।५९
\end{verse}

प्रयत्न अवस्था पर और क्रिया समय पर अवलम्बित रहती है~। अतएव अवस्था और समय के अनुसार ही कार्य करना चाहिए~।

सभी बातों में इस ‘समय’ की ओर विशेष दृष्टि रखनी चाहिए~। एक समय जिसमें असफलता हुई है, सम्भव है उसी में दूसरे समय पूरी सफलता प्राप्त हो जाय~।

\begin{verse}
सत्यं मृदु प्रियं धीरो वाक्यं हितकरं वदेत्~।\\ आत्मोत्कर्षं तथा निन्दां परेषां परिवर्जयेत्~॥ - ८।६२
\end{verse}

धीर गृहस्थ को सत्य, मृदु, प्रिय तथा हितकर वचन बोलने चाहिए~। वह अपने उत्कर्ष की चर्चा न करे और दूसरों की निन्दा करना छोड़ दे~।

\begin{verse}
जलाशयाश्च वृक्षाश्च विश्रामगृहध्वनि~।\\ सेतुः प्रतिष्ठितो येन तेन लोकत्रयं जितम्~॥ - ८।६३
\end{verse}

जो व्यक्ति सब लोगों की सुविधा के लिए जलाशय खुदवाता है, वृक्ष लगाता है, धर्मशालाएँ तथा सेतु निर्माण करता है, वह तीनों लोगों को जीत लेता है।

बड़े-बड़े योगियों को जो पद प्राप्त होता है, उसी की ओर इन सब कर्मों को करनेवाला भी अग्रसर होता रहता है।

ऊपर कहे हुए वाक्यों द्वारा यह स्पष्ट है कि कर्मयोगसम्बन्धी इन नीतिवाक्यों को कार्य-रूप में परिणत करना ही गृहस्थ का मुख्य कर्तव्य है~। सर्वदा क्रियाशील रहना कर्मयोग का एक अंग है - यही गृहस्थ का कर्तव्य है।

उक्त तन्त्र-ग्रन्थ में एक और श्लोक इस प्रकार है:-

\begin{verse}
न बिभेति रणाद् यो वै संग्रामेऽप्यपराङ्मुखः।\\ धर्मयुद्धे मृतो वापि तेन लोकत्रयं जितम्~॥ - ८।६७
\end{verse}

जो मनुष्य युद्ध में नहीं डरता, पीठ नहीं दिखाता और जो धर्मयुद्ध में मृत्यु को प्राप्त होता है, वह तीनों लोकों को जीत लेता है~।

यदि स्वदेश अथवा स्वधर्म के लिए युद्ध करते-करते मनुष्य की मृत्यु हो जाय, तो योगीजन जिस पद को ध्यान द्वारा पाते हैं, वही पद उस मनुष्य को भी मिलता है~। इससे यह स्पष्ट है कि जो एक मनुष्य का कर्तव्य है, वह दूसरे मनुष्य का कर्तव्य नहीं भी हो सकता; परन्तु साथ ही, शास्त्र किसी के भी कर्तव्य को हीन अथवा उन्नत नहीं कहते~। विभिन्न देश, काल तथा पात्र के अनुसार कर्तव्य भी विभिन्न होते हैं; और हम जिस अवस्था में रहें; उसी के उपयोगी कर्तव्य हमें करने चाहिए।

इन सब से हमें एक भाव यह मिलता है कि दुर्बलता मात्र ही सर्वथा घृण्य और परित्याज्य है~। हमारे दर्शन, धर्म अथवा कर्म के भीतर - हमारी समस्त शास्त्रीय शिक्षाओं के भीतर - यही एक मुख्य भाव है, जो मुझे पसन्द आता है~। यदि तुम वेदों को पढ़ो, तो देखोगे कि उसमें ‘नाभयेत्’ ‘अभीः’ अर्थात किसी से भी डरना नहीं चाहिए - यह बात बार-बार कथित हुई है~। भय दुर्बलता का चिह्न है, और यह दुर्बलता ही मनुष्य को ईश्वरप्राप्ति के मार्ग से हटाकर उसे नाना प्रकार के पापकर्मों की ओर खींच लेती है~। इसलिए संसार के उपहास अथवा व्यंग की ओर तनिक भी ध्यान न देकर मनुष्य को निर्भय होकर अपना कर्तव्य करते रहना चाहिए~।

यदि कोई मनुष्य संसार से विरक्त होकर ईश्वरोपासना में लग जाय, तो उसे यह नहीं समझना चाहिए कि जो लोग संसार में रहकर संसार के हित के लिए कार्य करते है, वे ईश्वर की उपासना नहीं करते, और न अपने स्त्री-बच्चों के लिए संसार में रहनेवाले गृहस्थों को ही यह सोचना चाहिए कि जिन लोगों ने संसार का त्याग कर दिया है, वे आलसी और घृणित जीव है~। अपने-अपने स्थान में सभी बड़े हैं~।

इस सम्बन्ध में मुझे एक कहानी का स्मरण आता है~। एक राजा था~। उसके राज्य में जब कभी कोई संन्यासी आते, तो उनसे वह सदैव एक प्रश्न पूछा करता था - “संसार का त्याग कर जो संन्यास ग्रहण करता है, वह श्रेष्ठ है, या संसार में रहकर जो गृहस्थ के समस्त कर्तव्यों को करता जाता है, वह श्रेष्ठ है?” अनेक विद्वान् लोगों ने उसके इस प्रश्न का उत्तर देने का प्रयत्न किया~। कुछ लोगों ने कहा कि संन्यासी श्रेष्ठ है~। यह सुनकर राजा ने इसे सिद्ध करने को कहा। जब वे सिद्ध न कर सके तो उन्हें विवाह करके गृहस्थ हो जाने की आज्ञा दी~। कुछ और लोग आये और उन्होंने कहा ‘स्वधर्मपरायण गृहस्थ ही श्रेष्ठ है।” राजा ने उनसे भी उनकी बात के लिए प्रमाण माँगा~। पर जब वे प्रमाण न दे सके, तो राजा ने उन्हें भी गृहस्थ हो जाने की आज्ञा दी।

अन्त में एक तरुण संन्यासी आये~। राजा ने उनसे भी उसी प्रकार प्रश्न किया~। संन्यासी ने कहा, “हे राजन् अपने-अपने स्थान में दोनों ही श्रेष्ठ हैं, कोई भी कम नहीं है।” राजा ने उसका प्रमाण माँगा~। संन्यासी ने उत्तर दिया, “हाँ, मैं इसे सिद्ध कर दूँगा, परन्तु तुम्हें मेरे साथ आना होगा और कुछ दिन मेरे ही समान जीवन व्यतीत करना होगा~। तभी मैं तुम्हें अपनी बात का प्रमाण दे सकूँगा।” राजा ने संन्यासी की बात स्वीकार कर ली और उनके पीछे-पीछे हो लिया~। वह उन संन्यासी के साथ अपने राज्य की सीमा को पार कर अनेक देशों में से होता हुआ एक बड़े राज्य में आ पहुँचा~। उस राज्य की राजधानी में एक बड़ा उत्सव मनाया जा रहा था~। राजा और संन्यासी ने संगीत और नगाड़ों के शब्द सुने तथा डौंडी पीटनेवालों की आवाज भी~। लोग सड़कों पर सुसज्जित होकर कतारों में खड़े थे~। उसी समय कोई एक विशेष घोषणा की जा रही थी~। उपरोक्त राजा तथा संन्यासी भी यह सब दखेने के लिए वहाँ खड़े हो गये~। घोषणा करनेवाले ने चिल्लाकर कहा, “इस देश की राजकुमारी का स्वयंवर होनेवाला है~।”

राजकुमारियों का अपने लिए इस प्रकार पति चुनना भारतवर्ष में एक पुराना रिवाज था~। अपने भावी पति के सम्बन्ध में प्रत्येक राजकुमारी के अलग-अलग विचार होते थे~। कोई अत्यन्त रूपवान् पति चाहती थी, कोई अत्यन्त विद्वान्, कोई अत्यन्त धनवान, आदि-आदि~। अड़ोस-पड़ोस के राज्यों के राजकुमार सुन्दर-से-सुन्दर ढंग से अपने को सजाकर राजकुमारी के सम्मुख उपस्थित होते थे~। कभी-कभी उन राजकुमारों के भी भाट होते थे, जो उनके गुणों का गान करते तथा यह दर्शाते थे कि उन्हीं का वरण किया जाय~। राजकुमारी को एक सजे हुए सिंहासन पर बिठाकर आलीशान ढंग से सभा के चारों ओर ले जाया जाता था~। वह उन सब के सामने जाती तथा उनका गुणगान सुनती~। यदि उसे कोई राजकुमार नापसन्द होता, तो वह अपने वाहकों से कहती, “आगे बढ़ो”, और उसके पश्चात् उस नापसन्द राजकुमार का कोई ख्याल तक न किया जाता था~। यदि राजकुमारी किसी राजकुमार से प्रसन्न हो जाती, तो वह उसके गले में वरमाला डाल देती और वह राजकुमार उसका पति हो जाता था।

जिस देश में यह राजा और संन्यासी आये हुए थे, उस देश में इसी प्रकार का एक स्वयंवर हो रहा था~। यह राजकुमारी संसार में अद्वितीय सुन्दरी थी और उसका भावी पति ही उसके पिता के बाद उसके राज्य का उत्तराधिकारी होनेवाला था~। इस राजकुमारी का विचार एक अत्यन्त सुन्दर पुरुष से विवाह करने का था, परन्तु उसे योग्य व्यक्ति मिलता ही न था~। कई बार उसके लिए स्वयंवर रच गये, पर राजकुमारी को अपने मन का पति न मिला~। इस बार का स्वयंवर बड़ा सुन्दर था; अन्य सभी अवसरों की अपेक्षा इस बार अधिक लोग आये थे~।

राजकुमारी रत्नजटित सिंहासन पर बैठकर आयी और उसके वाहक उसे एक राजकुमार के सामने से दूसरे के सामने ले गये~। परन्तु उसने किसी की ओर देखा तक नहीं~। सभी लोग निराश हो गये और सोचने लगे कि क्या अन्य अवसरों की भाँति इस बार का स्वयंवर भी असफल ही रहेगा~। इतने ही में वहाँ एक दूसरा तरुण संन्यासी आ पहुँचा~। वह इतना सुन्दर था कि मानो सूर्यदेव ही आकाश छोड़कर स्वयं पृथ्वी पर उतर आये हों~। वह आकर सभा के एक ओर खड़ा हो गया और जो कुछ हो रहा था, उसे देखने लगा~। राजकुमारी का सिंहासन उसके समीप आया और ज्योंही उसने उस सुन्दर संन्यासी को देखा, त्योंही वह रुक गयी और उसके गले में वरमाला डाल दी~। तरुण संन्यासी ने एकदम माला को रोक लिया और यह कहते हुए, ‘छिः, छिः, यह क्या है?’ उसे फेंक दिया~। उसने कहा, “मैं संन्यासी हूँ, मुझे विवाह से क्या प्रयोजन?” उस देश के राजा ने सोचा कि शायद निर्धन होने के कारण यह राजकुमारी से विवाह करने का साहस नहीं कर रहा है~। अतएव उसने उससे कहा, “देखो, मेरी कन्या के साथ तुम्हें मेरा आधा राज्य अभी मिल जाएगा, और सम्पूर्ण राज्य मेरी मृत्यु के बाद!” और यह कहकर उसने संन्यासी के गले में फिर माला डाल दी~। उस युवा संन्यासी ने माला फिर निकालकर फेंक दी और कहा, “छिः, यह सब क्या झंझट है, मुझे विवाह से क्या मतलब?” और यह कहकर वह तुरन्त सभा छोड़कर चला गया~।

इधर राजकुमारी इस युवा पर इतनी मोहित हो गयी कि उसने कह दिया, “मैं इसी मनुष्य से विवाह करूँगी, नहीं तो प्राण त्याग दूँगी।” और राजकुमारी संन्यासी के पीछे-पीछे उसे लौटा लाने के लिए चल पड़ी~। इसी अवसर पर अपने पहले संन्यासी ने, जो राजा को यहाँ लाये थे, राजा से कहा, “राजन् चलिये, इन दोनों के पीछे-पीछे हम लोग भी चले।” निदान, वे उनके पीछे-पीछे काफी फासला रखते हुए चलने लगे~। वह युवा संन्यासी, जिसने राजकुमारी से विवाह करने से इनकार कर दिया था, कई मील निकल गया और अन्त में एक जंगल में घुस गया~। उसके पीछे राजकुमारी थी, और उन दोनों के पीछे ये दोनों।

तरुण संन्यासी उस वन से भलीभाँति परिचित था तथा वहाँ के सारे जटिल रास्तों का उसे ज्ञान था~। वह एकदम एक रास्ते में घुस गया और अदृश्य हो गया~। राजकुमारी उसे फिर देख न सकी~। उसे काफी देर ढूँढ़ने के बाद अन्त में वह एक वृक्ष के नीचे बैठ गयी और रोने लगी, क्योंकि उसे बाहर निकलने का मार्ग नहीं मालूम था~। इतने में यह राजा और संन्यासी उसके पास आये और उससे कहा, “बेटी, रोओ मत, हम तुम्हें इस जंगल के बाहर निकाल ले चलेंगे, परन्तु अभी बहुत अँधेरा हो गया है, जिससे रास्ता ढूँढ़ना सहज नहीं~। यहीं एक बड़ा पेड़ है, आओ, इसी के नीचे हम सब विश्राम करें और सबेरा होते ही हम तुम्हें मार्ग बता देंगे।”

अब, उस पेड की एक डाली पर एक छोटी चिड़िया, उसकी स्त्री तथा उसके तीन बच्चे रहते थे~। उस चिड़िया ने पेड़ के नीचे इन तीन लोगों को देखा और अपने स्त्री से कहा, “देखो, हमारे यहाँ ये लोग अतिथि हैं, जाड़े का मौसम है, हम लोग क्या करें? हमारे पास आग तो है नहीं।” यह कहकर वह उड़ गया और एक जलती हुई लकड़ी का टुकड़ा अपनी चोंच में दबा लाया और उसे अतिथियों के सामने गिरा दिया~। उन्होंने उसमें लकड़ी लगा-लगाकर खूब आग तैयार कर ली; परंतु चिड़िया को फिर भी सन्तोष न हुआ~। उसने अपनी स्त्री से फिर कहा, “बताओ, अब हमें क्या करना चाहिए?” ये लोग भूखे हैं, और इन्हें खिलाने के लिए हमारे पास कुछ भी नहीं है~। हम लोग गृहस्थ हैं और हमारा धर्म है कि जो कोई हमारे घर आये, उसे हम भोजन करायें~। जो कुछ मेरी शक्ति में है, मुझे अवश्य करना चाहिए; मैं उन्हें अपना यह शरीर ही दे दूँगा।” ऐसा कहकर वह आग में कूद पड़ा और भुन गया~। अतिथियों ने उसे आग में गिरते देखा, उसे बचाने का यत्न भी किया, परन्तु बचा न सके~। उस चिड़िया की स्त्री ने अपने पति का सुकृत देखा और अपने मन में कहा, “ये तो तीन लोग हैं, उनके भोजन के लिए केवल एक ही चिड़िया पर्याप्त नहीं~। पत्नी के रूप में मेरा यह कर्तव्य है कि अपने पति के परिश्रमों को मैं व्यर्थ न जाने दूँ~। वे मेरा भी शरीर ले लें।” और ऐसा कहकर वह भी आग में गिर गयी और भुन गयी।

इसके बाद जब उन तीन छोटे बच्चों ने देखा कि उन अतिथियों के लिए इतना तो पर्याप्त न होगा, तो उन्होंने आपस में कहा, “हमारे मातापिता से जो कुछ बन पड़ा उन्होंने किया, परन्तु फिर भी उतना पूरा न पड़ेगा~। अब हमारा धर्म है कि हम उनके कार्य को पूरा करें - हमें अपने शरीर भी दे देने चाहिए।” और यह कहकर वे सब भी आग में कूद पड़े।

यह सब देखकर ये तीनों लोग बहुत चकित हुए~। इन चिड़ियों को वे खा ही कैसे सकते थे! रात को वे बिना भोजन किये ही रहे~। प्रातःकाल राजा तथा संन्यासी ने राजकुमारी को जंगल का मार्ग दिखला दिया और वह अपने पिता के घर वापस चली गयी~।

तब संन्यासी ने राजा से कहा, “देखो राजन्, तुम्हें अब ज्ञात हो गया है कि अपने-अपने स्थान में सब बड़े हैं~। यदि तुम संसार में रहना चाहते हो, तो इन चिड़ियों के समान रहो, दूसरो के लिए अपना जीवन दे देने को सदैव तत्पर रहो~। और यदि तुम संसार छोड़ना चाहते हो, तो उस युवा संन्यासी के समान होओ, जिसके लिए वह परम सुन्दरी स्त्री और एक राज्य भी तृणवत् था~। यदि गृहस्थ होना चाहते हो तो दूसरों के हित के लिए अपना जीवन अर्पित कर देने के लिए तैयार रहो~। और यदि तुम्हें संन्यासजीवन की इच्छा है तो सौन्दर्य, धन तथा अधिकार की ओर आँख तक न उठाओ~। अपने-अपने स्थान में सब श्रेष्ठ हैं, परन्तु एक का कर्तव्य दूसरे का कर्तव्य नहीं हो सकता।”

