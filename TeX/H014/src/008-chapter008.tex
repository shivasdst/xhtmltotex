
\chapter{कर्मयोग का आदर्श}

वेदान्त का सब से उदात्त तत्त्व यह है कि हम एक ही लक्ष्य पर भिन्न भिन्न मार्गों से पहुँच सकते हैं~। मैंने इन्हें साधारण रूप से चार मार्गों में विभाजित किया है और वे हैं - कर्ममार्ग, भक्तिमार्ग, योगमार्ग और ज्ञानमार्ग~। परन्तु साथ ही तुम्हें यह भी स्मरण रखना चाहिए कि ये बिलकुल पृथक्-पृथक् विभाग नहीं हैं~। प्रत्येक एक दूसरे के अन्तर्गत है~। किन्तु प्राधान्य के अनुसार ही ये विभाग किये गये हैं~। ऐसी बात नहीं कि तुम्हें कोई ऐसा व्यक्ति मिले, जिसमें कर्म करने के अतिरिक्त दूसरी कोई शक्ति न हो, अथवा जिसमें केवल भक्ति या केवल ज्ञान के अतिरिक्त और कुछ न हो~। ये विभाग केवल मनुष्य की प्रधान प्रवृत्ति अथवा गुणप्राधान्य के अनुसार किये गये हैं~। हमने देखा है कि अन्त में ये सब मार्ग एक ही लक्ष्य में जाकर एक हो जाते हैं~। सारे धर्म और सारी साधन प्रणाली हमें उसी एक चरम लक्ष्य की ओर ले जा रही हैं~।

वह चरम लक्ष्य क्या है, यह बताने का यत्न मैं पहले ही कर चुका हूँ~। मेरे मतानुसार, वह है मुक्ति~। एक छोटे से परमाणु से लेकर मनुष्य तक, अचेतन प्राणहीन जड़ वस्तु से लेकर सर्वोच्च मानवात्मा तक जो कुछ भी हम इस विश्व में देखते हैं, अनुभव करते या श्रवण करते हैं, वे सब के सब मुक्ति की ही चेष्टा कर रहे हैं~। असल में इस मुक्ति लाभ के लिए संग्राम का ही फल है - यह जगत्~। इस जगत्रूप मिश्रण में प्रत्येक परमाणू दूसरे परमाणुओं से पृथक् हो जाने की चेष्टा कर रहा है, पर दूसरे उसे आबद्ध करके रखे हुए हैं~। हमारी पृथ्वी सूर्य से दूर भागने की चेष्टा कर रही है तथा चन्द्रमा, पृथ्वी से~। प्रत्येक वस्तु अनन्त विस्तारोन्मुख है~। इस संसार में हम जो कुछ भी देखते हैं, उस सबका मूल आधार मुक्तिलाभ के लिए यह संग्राम ही है~। इसी की प्रेरणा से साधु उपासना करता है और चोर, चोरी~। जब कार्यप्रणाली अनुचित होती है, तो उसे हम बुरी कहते हैं, और जब कार्यप्रणाली का प्रकाश उचित तथा उच्च होता हैं, तो उसे हम अच्छा या श्रेष्ठ कहते हैं~। परन्तु दोनों दशाओं में प्रेरणा एक ही होती है, और वह है मुक्तिलाभ के लिए चेष्टा~। साधु अपनी बद्ध दशा को सोचकर कातर हो उठता है, वह उससे छुटकारा पाने की इच्छा करता है और इसलिए ईश्वरोपासना करता है~। इधर चोर भी यह सोचकर परेशान हो जाता है कि उसके पास अमुक वस्तुएँ नहीं हैं~। वह उस अभाव से छुटकारा पाने की - मुक्त होने की - कामना करता है और इसलिए चोरी करता है~। चेतन अथवा अचेतन समस्त प्रकृति का लक्ष्य यह मुक्ति ही है~। जाने या अनजाने सारा जगत् इसी लक्ष्य की ओर पहुँचने का यत्न कर रहा है~। पर हाँ, यह अवश्य है कि मुक्ति के सम्बन्ध में एक साधु की धारणा एक चोर की धारणा से नितान्त भिन्न होती है, यद्यपि वे दोनों ही छुटकारा पाने की प्रेरणा से कार्य कर रहे हैं~। साधु मुक्ति के लिए प्रयत्न करके अनन्त अनिर्वचनीय आनन्द का अधिकारी हो जाता है, परन्तु चोर के तो बन्धनों पर बन्धन बढ़ते ही जाते हैं~।

प्रत्येक धर्म में मुक्तिलाभ की इस प्रकार चेष्टा का विकास पाया जाता है~। यही सारी नीति की, सारी निःस्वार्थपरता की नींव है~। निःस्वार्थपरता का अर्थ है - ‘मैं यह क्षुद्रशरीर हूँ’ इस भाव से परे होना~। जब हम किसी मनुष्य को कोई सत् कार्य करते, दूसरों की सहायता करते देखते हैं, तो उसका तात्पर्य यह है कि यह व्यक्ति ‘मैं और मेरे’ के क्षुद्र वृत्त में आबद्ध होकर नहीं रहना चाहता~। इस स्वार्थपरता के वृत्त के बाहर बस ‘यहीं तक’ जाया जा सकता है, इस प्रकार की कोई निर्दिष्ट सीमा नहीं है~। सारी श्रेष्ठ नीतिप्रणालियाँ यही शिक्षा देती हैं कि सम्पूर्ण स्वार्थत्याग ही चरम लक्ष्य है~। मान लो, किसी मनुष्य ने इस सम्पूर्ण स्वार्थत्याग को प्राप्त कर लिया, - तो फिर उसकी क्या अवस्था हो जाती है? फिर वह अमुक-अमुक नामवाला पहले का सामान्य व्यक्ति नहीं रह जाता~। उसे तो फिर अनन्त विस्तार लाभ हो जाता है~। फिर उसका पहले का वह क्षुद्र व्यक्तित्व सदा के लिए नष्ट हो जाता है - अब तो वह अनन्तस्वरूप हो जाता है~। इस अनन्त विकास की प्राप्ति ही असल में समस्त धार्मिक एवं नैतिक शिक्षाओं का लक्ष्य है~। व्यक्तित्ववादी जब इस तत्त्व को दार्शनिक रूप में रखा हुआ देखता है, तो वह सिहर उठता है~। परन्तु साथ ही जब वह स्वयं नीति का प्रचार करता है, तो आखिर वह क्या करता है? - वह भी इस तत्त्व का ही प्रचार करता है~। वह भी मनुष्य की निःस्वार्थपरता की कोई सीमा निर्दिष्ट नहीं करता~। मान लो, इस व्यक्तित्ववाद के अनुसार एक मनुष्य सम्पूर्ण रूप से अनासक्त हो गया~। तो हम उसमें तथा अन्य सम्प्रदायों के पूर्ण सिद्ध व्यक्तियों में क्या भेद पाते हैं? वह तो विश्व के साथ एकरूप हो गया है; और इस प्रकार एकरूप हो जाना ही तो सभी मनुष्यों का लक्ष्य है~। केवल बेचारे व्यक्तित्ववादी में इतना साहस नहीं कि वह अपनी युक्तियों का, यथार्थ सिद्धान्त पर पहुँचने तक, अनुसरण कर सके~। निःस्वार्थ कर्म द्वारा मानवजीवन की चरमावस्था इस मुक्ति का लाभ कर लेना ही कर्मयोग है~। अतएव हमारा प्रत्येक स्वार्थपूर्ण कार्य हमारे अपने इस लक्ष्य की ओर पहुँचने में बाधक होता है तथा प्रत्येक निःस्वार्थ कर्म हमें उस चरम अवस्था की ओर आगे बढ़ाता है~। इसीलिए ‘नीतिसंगत’ और ‘नीतिविरूद्ध’ की यही एकमात्र व्याख्या हो सकती है कि जो स्वार्थपर है, वह ‘नीतिविरूद्ध’ है और जो निःस्वार्थपर है, वह ‘नीतिसंगत’ है~।

परन्तु यदि हम कुछ विशिष्ट कर्तव्यों की मीमांसा करें, तो इतनी सरल और सीधी व्याख्या दे देने से काम न चलेगा~। जैसा मैं पहले ही कह चुका हूँ, विभिन्न परिस्थितियों में कर्तव्य भिन्न-भिन्न हो जाते हैं~। जो एक कार्य अवस्था में\break निःस्वार्थ होता है, हो सकता है, वही किसी दूसरी अवस्था में बिलकुल स्वार्थपर हो जाय~। अतः कर्तव्य की हम केवल एक साधारण व्याख्या ही कर सकते हैं~।\break परन्तु कार्यविशेषों की कर्तव्याकर्तव्यता पूर्णतया देशकाल-पात्र पर ही निर्भर रहेगी~। एक देश में एक प्रकार का आचरण नीतिसंगत माना जाता है, परन्तु सम्भव है, वही किसी दूसरे देश में अत्यन्त नीतिविरुद्ध माना जाय, क्योंकि भिन्न-भिन्न देशों में भिन्न-भिन्न परिस्थितियाँ होती है~। समस्त प्रकृति का अन्तिम ध्येय मुक्ति है और यह मुक्ति केवल पूर्ण निःस्वार्थता द्वारा ही प्राप्त की जा सकती है~। प्रत्येक स्वार्थशून्य कार्य, प्रत्येक निःस्वार्थ विचार, प्रत्येक निःस्वार्थ वाक्य हमें इसी ध्येय की ओर ले जाता है, और इसीलिए हम उसे नीतिसंगत कहते हैं~। तुम देखोगे कि यह व्याख्या प्रत्येक धर्म एवं प्रत्येक नीतिप्रणाली में लागू होती है~। नीतितत्त्व के मूल के सम्बन्ध में भिन्न-भिन्न देशों में भिन्न-भिन्न धारणाएँ हो सकती है~। कुछ दर्शनों में नीतितत्त्व का मूल सम्बन्ध परमपुरुष परमात्मा में लगाते हैं~। यदि तुम उन सम्प्रदायों के किसी व्यक्ति से पूछो कि हमें अमुक कार्य क्यों करना चाहिए अथवा अमुक क्यों नहीं, तो वह उत्तर देगा कि “ईश्वर की ऐसी ही आज्ञा है~।” उनके नीतितत्त्व का मूल चाहे जो हो, पर उसका सार असल में यही है कि ‘वयं’ की चिन्ता न करो, ‘अहं’ का त्याग करो~। परन्तु फिर भी, नीतितत्त्व के सम्बन्ध में इस प्रकार की उच्च धारणा रहने पर भी अनेक व्यक्ति अपने इस क्षुद्र व्यक्तित्व के त्याग करने की कल्पना से सिहर उठते हैं, जो मनुष्य अपने इस क्षुद्र व्यक्तित्व से जकड़ा रहना चाहता है, उससे हम पूछें, “अच्छा, जरा ऐसे पुरुष की ओर तो देखो, जो नितान्त निःस्वार्थ हो गया है, जिसकी अपने स्वयं के लिए कोई चिन्ता नहीं है, जो अपने लिए कोई भी कार्य नहीं करता, जो अपने लिए एक शब्द भी नहीं कहता; और फिर बताओ कि उसका ‘निजत्व’ कहाँ है?” जब तक वह अपने स्वयं के लिए विचार करता है, कोई कार्य करता है या कुछ कहता है, तभी तक उसे अपने ‘निजत्व’ का बोध रहता है~। परन्तु यदि उसे केवल दूसरों के सम्बन्ध में ध्यान है, जगत् के सम्बन्ध में ही ध्यान है, तो फिर उसका ‘निजत्व’ भला कहाँ रहा? उसका तो सदा के लिए लोप हो चुका है~।

अतएव कर्मयोग, निःस्वार्थपरता और सत्कर्म द्वारा मुक्ति लाभ करने की एक विशिष्ट प्रणाली है~। कर्मयोगी को किसी भी प्रकार के धर्ममत का अवलम्बन करने की आवश्यकता नहीं~। वह ईश्वर में भी चाहे विश्वास करे अथवा न करे, आत्मा के सम्बन्ध में भी अनुसंधान करें या न करे, किसी प्रकार का दार्शनिक विचार भी करे अथवा न करे, इससे कुछ बनता बिगड़ता नहीं~। उसके सम्मुख उसका बस अपना निःस्वार्थपरता लाभरूप एक विशिष्ट ध्येय रहता है और अपने प्रयत्न द्वारा ही उसे उसकी प्राप्ति कर लेनी पड़ती है~। उसके जीवन का प्रत्येक क्षण ही मानो प्रत्यक्ष अनुभव होना चाहिए, क्योंकि उसे तो अपनी समस्या का समाधान किसी भी प्रकार के मतामत की सहायता न लेकर केवल कर्म द्वारा ही करना होता है, जब कि ज्ञानी उसी समस्या का समाधान अपने ज्ञान और आन्तरिक प्रेरणा द्वारा तथा भक्त अपनी भक्ति द्वारा करता है~।

अब दूसरा प्रश्न आता है - यह कर्म क्या है? संसार के प्रति उपकार करने का क्या अर्थ है? क्या हम सचमुच संसार का कोई उपकार कर सकते है? उपकार का अर्थ यदि ‘पूर्ण उपकार’ लिया जाय, तो उत्तर है - नहीं; परन्तु सापेक्ष दृष्टि से - हाँ। संसार के प्रति ऐसा कोई भी उपकार नहीं किया जा सकता, जो चिरस्थायी हो। यदि ऐसा कभी सम्भव होता, तो यह संसार इस रूप में कभी न रहता, जैसा उसे हम आज देख रहे हैं~। हम किसी मनुष्य की भूख अल्प समय के लिए भले ही शान्त कर दें, परन्तु बाद में वह फिर भूखा हो जाएगा~। किसी व्यक्ति को हम जो भी कुछ सुख दे सकते हैं, वह क्षणिक ही होता है~। सुख और दुःखरूपी इस सतत होनेवाले रोग का कोई भी सदा के लिए उपचार नहीं कर सकता - इस सतत गतिमान सुख-दुःखरूपी चक्र को कोई भी चिरकाल के लिए रोक नहीं सकता~। क्या संसार को हम कोई चिरन्तन सुख दे सकते हैं? नहीं, यह कभी सम्भव नहीं हो सकता~। समुद्र के जल में बिना किसी एक जगह गर्त पैदा किये हम एक भी लहर नहीं उठा सकते~। इस संसार में सारी शक्तियों की समष्टि सदैव समान रहती है~। यह न तो कम हो सकती है, न अधिक~। उदाहरणार्थ हम मानवजाति का इतिहास ही ले ले, जैसा कि हमें आज ज्ञात है~। क्या हमें सदैव वही सुख-दुःख, वही हर्ष-विषाद तथा अधिकार का वही तारतम्य पगपग पर नहीं दिखायी देता? क्या कुछ लोग अमीर तो कुछ गरीब, कुछ बड़े तो कुछ छोटे, कुछ स्वस्थ तो कुछ रोगी नहीं है? प्राचीन काल में मिस्रवासियों, ग्रीसवालों और रोमनों की जो अवस्था थी, वही आज अमेरिकावालों की भी है~। जहाँ तक हमें इस संसार के इतिहास से पता चलता है, यही दशा सदैव रही है; परन्तु फिर भी हम देखते है कि सुख दुःख की इस अनिवार्य भिन्नता के होते हुए भी साथ ही साथ उसे घटाने के प्रयत्न भी सदैव होते रहे हैं~। इतिहास के प्रत्येक युग में ऐसे हजारों स्त्री पुरुष हुए हैं, जिन्होंने दूसरों के लिए जीवनपथ को सुगम बनाने के लिए अविरत परिश्रम किया~। पर वे कभी इसमें सफल न हो सके~। हम तो केवल एक गेंद को एक जगह से दूसरी जगह फेंकने का खेल खेल सकते हैं~। हमने यदि शरीर से दुःख को निकाल बाहर किया, तो देखते हैं कि वह मन में जा बैठा~। यह ठीक दान्ते के उस नर्क-चित्र जैसा है; - कंजूसों को सोने का एक बड़ा गोला दिया गया है और उनसे उस गोले को पहाड़ के ऊपर ढकेलकर चढ़ाने के लिए कहा गया है~। परन्तु प्रत्येक बार ज्योंही वे उसे थोड़ासा ऊपर ढकेल पाते हैं कि वह लुढ़ककर नीचे आ जाता है~। इसी प्रकार यह संसारचक्र घूम रहा है~। सतयुग के सम्बन्ध में हमारी बातचीत बहुत सुन्दर है, परन्तु उसी प्रकार, जैसे स्कूल के बच्चों के लिए किस्से कहानी! उससे अधिक और कुछ नहीं~। जो सब जातियाँ सतयुग का लुभावना स्वप्न देखा करती हैं, वे अपने मन में यह भावना रखती हैं कि उस सतयुग के आने पर संसार की अन्य जातियों की अपेक्षा शायद उन्हें ही उसका सब से अधिक लाभ मिल जाएगा! सतयुग के सम्बन्ध में यह कैसा निःस्वार्थ भाव है?

अतएव यह सिद्ध हुआ कि हम इस संसार के सुख को नहीं बढ़ा सकते, और इसी प्रकार न दुःख को ही~। इस संसार में शुभ और अशुभ शक्तियों की समष्टि सदैव समान रहेगी~। हम उसे सिर्फ यहाँ से वहाँ और वहाँ से यहाँ ढकेलते रहते हैं; परन्तु यह निश्चित है कि वह सदैव समान रहेगी, क्योंकि वैसा रहना ही उसका स्वभाव है~। यह ज्वार भाटा, यह चढ़ाव उतार तो संसार की प्रकृति ही है~। इसके विपरीत सोचना तो वैसा ही युक्तिसंगत होगा, जैसा यह कहना कि मृत्यु बिना जीवन सम्भव है~। ऐसा कहना निरी मूर्खता है, क्योंकि जीवन कहने से ही मृत्यु का बोध होता है, और सुख कहने से दुःख का~। एक चिराग सतत जलता जा रहा है और यही तो उसका जीवन है~। यदि तुम्हें जीवन की अभिलाषा हो, तो उसके लिए तुम्हें प्रतिक्षण मरना होगा~। जीवन और मृत्यु एक ही चीज के विभिन्न पहलू हैं - केवल अलग-अलग दृष्टिकोणों से भिन्न-भिन्न दिखायी मात्र देते हैं~। वे एक ही तरंग के उत्थान और पतन हैं, और दोनों को मिलाने से ही एक सम्पूर्ण वस्तु बनती है~। एक व्यक्ति पतन को देखता है और निराशावादी बन जाता है; दूसरा उत्थान देखता है और आशावादी बन जाता है~। बालक पाठशाला जाता है, माता-पिता उसकी पूरी देखभाल करते हैं; तब उसे हर एक वस्तु सुखप्रद मालूम होती है~। उसकी आवश्यकताएँ बिलकुल साधारण हुआ करती हैं, वह बड़ा आशावादी बन जाता है~। पर एक वृद्ध को देखो, जिसे संसार के अनेक अनुभव हो चुके हैं; - वह अपेक्षाकृत शान्त हो जाता है और उसकी गर्मी काफी ठण्डी पड़ जाती है~। इसी प्रकार, वे प्राचीन जातियाँ, जिन्हें चहुँ ओर अपने पूर्वगौरव के केवल ध्वंसावशेष ही दृष्टिगोचर होते हैं, स्वभावतः नूतन जातियों की अपेक्षा कम आशावादी होती हैं~। भारतवर्ष में एक कहावत है, ‘हजार वर्ष तक शहर और फिर हजार वर्ष तक जंगल~।’ शहर का जंगल में तथा जंगल का शहर में इस प्रकार परिवर्तन सर्वत्र ही होता रहता है, और लोग इस तस्वीर को जिस पहलू से देखते हैं, उसी के अनुसार वे आशावादी या निराशावादी बन जाते हैं~।

इसके बाद अब हम साम्यभाव के सम्बन्ध में विचार करेंगे~। उपर्युक्त सतयुग सम्बन्धी धारणा से अनेक व्यक्तियों को कार्य करने की प्रेरणा मिली है~। बहुत से धर्म इसका अपने धर्म के एक अंग के रूप में प्रचार किया करते हैं~। उनकी धारणा है कि परमेश्वर इस जगत् का शासन करने के लिए स्वयं आ रहे हैं, और उनके आने पर फिर लोगों में किसी प्रकार का अवस्था-भेद न रह जाएगा~। जो लोग इस बात का प्रचार करते हैं, वे अवश्य मतान्ध है; किन्तु उनमें सचमुच बड़ी आन्तरिकता होती है~। ईसाई धर्म का प्रचार भी तो इसी मोह मतान्धता द्वारा हुआ था और यही कारण है कि ग्रीक एवं रोमन गुलाम इसकी ओर इतने आकृष्ट हुए थे~। उनका यह दृढ़ विश्वास हो गया था कि इस सतयुगी धर्म में गुलामी बिलकुल न रह जाएगी, अन्न वस्त्र की भी बिलकुल कमी न रहेगी, और इसलिए वे हजारों की तादाद में ईसाई होने लगे~। जिन ईसाइयों ने इस भाव का प्रथम प्रचार किया, वे वास्तव में अज्ञानी मतान्ध व्यक्ति थे, परन्तु उनका विश्वास निष्कपट था~। आजकल के जमाने में इसी सतयुगी भावना ने साम्य, स्वाधीनता और भ्रातृभाव का रूप धारण कर लिया है~। पर यह भी एक मतान्धता है~। यथार्थ साम्यभाव न तो कभी संसार में हुआ है, और न कभी होने की आशा है~। यहाँ हम सब समान हो ही कैसे सकते हैं? इस प्रकार के असम्भव साम्यभाव का फल तो मृत्यु ही होगा~। जगत् की उत्पत्ति तथा उसकी स्थिति का कारण क्या है? - साम्य का अभाव, केवल वैषम्यभाव~। जगत् की प्रारम्भिक अवस्था में - प्रलयावस्था में - ही सम्पूर्ण साम्यभाव हो सकता है~। तब फिर इन सब निर्माणशील विभिन्न शक्तियों का उद्भव किस प्रकार होता है? - विरोध, प्रतियोगिता एवं प्रतिद्वन्द्विता द्वारा ही~। थोड़ी देर के लिए मान लो कि संसार के सब भौतिक परमाणु सम्पूर्ण साम्यवस्था में स्थित हो गये, - तो फिर सृष्टि रहेगी कहाँ? विज्ञान हमें सिखाता है कि यह असम्भव है~। स्थिर जल को हिला दो; तुम देखोगे कि प्रत्येक जलबिन्दु फिर से स्थिर होने की चेष्टा करता है, एक दूसरे की ओर इसी हेतु दौड़ता है~। इसी प्रकार यह जगत्-प्रपंच उत्पन्न हुआ है और उसके अन्तर्गत समस्त शक्तियाँ एवं समस्त पदार्थ अपने नष्ट साम्यभाव को पुनः प्राप्त करने के लिए चेष्टा कर रहे हैं~। पुनः वैषम्यावस्था आती है और उससे पुनः इस सृष्टिरूप मिश्रण की उत्पत्ति हो जाती है~। वैषम्य ही सृष्टि की नींव है~। परन्तु साथ ही वे शक्तियाँ भी, जो साम्यभाव स्थापित करने की चेष्टा करती हैं, सृष्टि के लिए उतनी ही आवश्यक हैं, जितनी कि वे, जो उस साम्यभाव को नष्ट करने का प्रयत्न करती हैं~।

सम्पूर्ण साम्यभाव अर्थात् समस्त प्रतिद्वन्द्वी शक्तियों का सम्पूर्ण सामंजस्य इस संसार में कभी नहीं हो सकता~। तुम्हारी उस अवस्था को प्राप्त करने के पूर्व ही सारा संसार किसी भी प्रकार के जीवन के लिए सर्वथा अयोग्य बन जाएगा, और वहाँ कोई भी प्राणी न रहेगा~। अतएव हम देखते हैं कि सतयुग अथवा सम्पूर्ण साम्यभाव की ये धारणाएँ इस संसार में केवल असम्भव ही नहीं, वरन् यदि हम इन्हें सम्पूर्ण रूप से कार्यरूप में परिणत करें, तो वे हमें निश्चय प्रलय की ओर ले जाएँगी~। वह क्या चीज है, जो मनुष्य मनुष्य में भेद स्थापित करती है? - वह है मस्तिष्क की भिन्नता~। आजकल के दिनों में एक पागल के अतिरिक्त और कोई भी यह न कहेगा कि हम सब मस्तिष्क की समान शक्ति लेकर उत्पन्न हुए हैं~। हम सब संसार में विभिन्न शक्तियाँ लेकर आते हैं~। कोई बड़ा आदमी होकर आता है, कोई छोटा~। इस जन्मगत विभिन्नता को अतिक्रमण करने का कोई मार्ग नहीं है~। अमेरिकन-इन्डियन लोग इस देश में हजारो वर्ष रहे और तुम्हारे जो पूर्वज यहाँ आये, उनकी संख्या बहुत कम थी~। परन्तु उन्हीं थोड़ेसे व्यक्तियों ने इस देश में क्या-क्या परिवर्तन कर दिये हैं! यदि सभी लोग समान हों, तो उन इन्डियनों ने इस देश को उन्नत करके बड़े-बड़े नगर आदि क्यों नहीं बना दिये? क्यों वे चिरकाल तक जंगलों में शिकार करते हुए घूमते रहे? तुम्हारे पूर्वजों के साथ इस देश में एक दूसरे ही प्रकार की दिमागी शक्ति, एक दूसरे ही प्रकार की संस्कार-समष्टि आ गयी और उसके फलस्वरूप यह परिस्थिति उत्पन्न हो गयी है~। सम्पूर्ण साम्यभाव का अर्थ है मृत्यु~। जब तक यह संसार बना रहेगा, तब तक वैषम्यभाव रहेगा ही~। और यह सतयुग अथवा साम्यभाव तभी आएगा, जब कल्प का अन्त हो जाएगा~। उसके पहले पूर्ण साम्यभाव नहीं आ सकता~। परन्तु फिर भी साम्यभाव की यह धारणा हमारे लिए कार्य में प्रवृत्ति देनेवाली एक प्रबल शक्ति है~। जिस प्रकार सृष्टि के लिए वैषम्य उपयोगी है, उसी प्रकार उस वैषम्य को घटाने की चेष्टा भी नितान्त आवश्यक है~। जिस प्रकार वैषम्य न होने से सृष्टि नहीं रह सकती, उसी प्रकार मुक्ति एवं ईश्वर के पास लौट जाने की चेष्टा बिना भी सृष्टि नहीं रह सकती~। कर्म करने के पीछे मनुष्य का जो हेतु रहता है, वह इन दो शक्तियों के तारतम्य से ही निश्चित होता है~। और कर्म करने के ये भिन्न भिन्न उद्देश्य चिरकाल तक विद्यमान रहेंगे - कुछ बन्धन की ओर ले जाएँगे और कुछ मुक्ति की ओर~।

संसार का यह ‘चक्र के भीतर चक्र’ एक बड़ा भयानक यन्त्र है~। इसके भीतर हाथ पड़ा नहीं कि हम गये~। हम सभी सोचते हैं कि अमुक कर्तव्य पूरा होते ही हमें छुट्टी मिल जाएगी, हम चैन की साँस लेंगे; पर उस कर्तव्य का मुश्किल से एक अंश भी समाप्त नहीं हो पाता कि एक दूसरा कर्तव्य सिर पर आ खड़ा होता है~। संसार का यह प्रचण्ड शक्तिशाली, जटिल यन्त्र हम सबों को खींचे ले जा रहा है~। इससे बाहर निकलने के केवल दो ही उपाय हैं~। एक तो यह कि उस यन्त्र से सारा नाता ही तोड़ दिया जाय - वह यन्त्र चलता रहे, हम एक ओर खड़े रहें और अपनी समस्त वासनाओं का त्याग कर दें~। अवश्य यह कह देना तो बड़ा सरल है, परन्तु इसे अमल में लाना असम्भव-सा है~। मैं नहीं कह सकता कि दो करोड़ आदमियों में से एक भी ऐसा कर सकेगा~। दूसरा उपाय है - हम इस संसारक्षेत्र में उतर आयें और कर्म का रहस्य जान लें~। इसी को कर्मयोग कहते हैं~। इस संसारयन्त्र से दूर न भागो, वरन् इसके अन्दर ही खड़े होकर कर्म का रहस्य सीख लो~। भीतर रहकर कौशल से कर्म करके बाहर निकल आना सम्भव है~। इस यन्त्र के भीतर से ही बाहर निकल आने का मार्ग है~।

अब हमने देखा कि कर्म क्या है~। यह प्रकृति की नींव का एक अंश है और सदैव ही चलता रहता है~। जो ईश्वर में विश्वास करते हैं, वे इसे अपेक्षाकृत अधिक अच्छी तरह समझ सकेंगे, क्योंकि वे जानते हैं कि ईश्वर कोई ऐसे एक ‘असमर्थ’ पुरुष नहीं हैं, जिन्हें हमारी सहायता की आवश्यकता है~। यद्यपि यह जगत् अनन्त काल तक चलता रहेगा, फिर भी हमारा ध्येय मुक्ति ही है, निःस्वार्थता ही हमारा लक्ष्य है; और कर्मयोग के मतानुसार उस ध्येय की प्राप्ति कर्म द्वारा ही करनी होगी~। संसार को पूर्ण रूप से सुखी बनाने की जो सब भावनाएँ हैं, वे मतान्ध व्यक्तियों के लिए प्रेरणाशक्ति के रूप में भले ही अच्छी हों, पर हमें यह भी जान लेना चाहिए कि मतान्धता से जितना लाभ होता है, उतनी ही हानि भी होती है~। कर्मयोगी प्रश्न करते हैं कि तुम्हें कर्म करने के लिए मुक्ति को छोड़ अन्य कोई उद्देश्य क्यों होना चाहिए? सब प्रकार के सांसारिक उद्देश्य के अतीत हो जाओ~। तुम्हें केवल कर्म करने का अधिकार है, कर्मफल में तुम्हारा कोई अधिकार नहीं - ‘कर्मण्येवाधिकारस्ते मा फलेषु कदाचन~।’ कर्मयोगी कहते हैं कि मनुष्य अध्यवसाय द्वारा इस सत्य को जान सकता है और इसे कार्यरूप में परिणत कर सकता है~। जब परोपकार करने की इच्छा उसके रोम-रोम में भिद जाती है, तो फिर उसे किसी बाहरी उद्देश्य की कोई आवश्यकता नहीं रह जाती~। हम भलाई क्यों करें? - इसलिए कि भलाई करना अच्छा है~। कर्मयोगी का कथन है कि जो स्वर्ग प्राप्त करने की इच्छा से भी सत्कर्म करता है, वह भी अपने को बन्धन में डाल लेता है~। किसी कार्य में यदि थोड़ीसी भी स्वार्थपरता रहे, तो वह हमें मुक्त करने के बदले हमारे पैरों में और एक बेड़ी डाल देता है~।

अतएव, एकमात्र उपाय है - समस्त कर्मफलों का त्याग कर देना, अनासक्त हो जाना~। यह जान रखो कि न तो यह संसार हम है और न हम यह संसार, न हम यह शरीर हैं और न वास्तव में हम कोई कर्म ही करते हैं~। हम हैं आत्मा - हम अनन्त काल से विश्राम और शान्ति का आनन्द भोग रहे हैं~। हम क्यों किसी के बन्धन में पड़े? यह कह देना बड़ा सरल है कि हम पूर्णरूप से अनासक्त रहें, परन्तु ऐसा हो किस तरह? बिना किसी स्वार्थ के किया हुआ प्रत्येक सत्-कार्य हमारे पैरों में और एक बेड़ी डालने के बदले पहले की ही एक बेड़ी तोड़ देता है~। बिना किसी बदले की आशा से संसार में भेजा गया प्रत्येक शुभ विचार संचित होता जाएगा, - वह हमारे पैरों में से एक बेड़ी को काट देगा और हमें अधिकाधिक पवित्र बनाता जाएगा, जब तक कि हम पवित्रतम मनुष्य के रूप में परिणत नहीं हो जाते~। पर हो सकता है, यह सब आप लोगों को केवल एक अस्वाभाविक और कोरी दार्शनिक बात ही जान पड़े जो कार्य में परिणत नहीं की जा सकती~। मैंने भगवद्गीता के विरोध में अनेक युक्तियाँ पढ़ी हैं, और कई लोगों का यह सिद्धान्त है कि बिना किसी हेतु के हम कुछ कर्म कर ही नहीं सकते~। उन्होंने शायद मतान्धता से रहित कोई निःस्वार्थ कर्म कभी देखा ही नहीं है, इसीलिए वे ऐसा कहा करते हैं~।

अब अन्त में संक्षेप में मैं तुम्हें एक ऐसे व्यक्ति के बारे में बताऊँगा, जो सचमुच कर्मयोग की शिक्षाओं को प्रत्यक्ष अमल में लाये थे~। वे कर्मयोगी थे बुद्ध देव - एकमात्र वे ही इन सब बातों को सम्पूर्ण रूप से अमल में ला सके थे~। भगवान् बुद्ध को छोड़कर संसार के अन्य सभी महापुरुषों की निःस्वार्थ कर्मप्रवृत्ति के पीछे कोई न कोई बाह्य उद्देश्य अवश्य था~। एक इन्हें छोड़कर संसार के अन्य सब महापुरुष दो श्रेणियों में विभक्त किये जा सकते हैं - एक तो वे, जो अपने को संसार में अवतीर्ण भगवान का अवतार कहते थे, और दूसरे वे, जो अपने को केवल ईश्वर का दूत मानते थे; ये दोनों अपने कार्यों की प्रेरणाशक्ति बाहर से लेते थे, बहिर्जगत् से ही पुरस्कार की आशा करते थे - चाहे उनकी भाषा कितनी भी आध्यात्मिकतापूर्ण क्यों न रही हो~। परन्तु एकमात्र बुद्ध ही ऐसे महापुरुष थे, जो कहते थे, “मैं ईश्वर के बारे में तुम्हारे मत-मतान्तरों को जानने की परवाह नहीं करता~। आत्मा के बारे में विभिन्न सूक्ष्म मतों पर बहस करने से क्या लाभ? भला करो और भले बनो~। बस यही तुम्हें निर्वाण की ओर अथवा जो कुछ सत्य हो उसकी ओर ले जाएगा~।”

उनके कार्यों के पीछे व्यक्तिगत उद्देश्य का लवलेश भी न था~। और उनकी अपेक्षा अधिक कार्य भला किस व्यक्ति ने किया है? इतिहास में मुझे जरा एक ऐसा चरित्र तो दिखाओ, जो सब से ऊपर इतना ऊँचा उठ गया हो~। सारी मानवाजाति ने ऐसा केवल एक ही चरित्र उत्पन्न किया है - इतना उन्नत दर्शन! इतनी अद्भुत सहानुभूति! सर्वश्रेष्ठ दर्शन का प्रचार करते हुए भी इन महान् दार्शनिक के हृदय में क्षुद्रतम प्राणी के प्रति भी गहरी सहानुभूति थी~। और फिर भी वे स्वयं के लिए किसी प्रकार का दावा नहीं कर गये~। वास्तव में वे ही आदर्श कर्मयोगी हैं, पूर्णरूपेण हेतुशून्य होकर उन्हीं ने कर्म किया है; और मानवजाति का इतिहास यह दिखाता है कि सारे संसार में उनके सदृश श्रेष्ठ महात्मा और कोई पैदा नहीं हुआ~। उनके साथ किसी की तुलना नहीं हो सकती~। हृदय तथा मस्तिष्क के पूर्ण सामंजस्य भाव के वे सर्वश्रेष्ठ उदाहरण हैं; आत्मशक्ति का जितना विकास उनमें हुआ, उतना और किसी में नहीं हुआ~। संसार में वे ही सर्वप्रथम श्रेष्ठ सुधारक हैं~। उन्हीं ने सर्वप्रथम साहसपूर्वक कहा था, “केवल कुछ प्राचीन हस्तलिखित पोथियों में कोई बात लिखी है इसीलिए उस पर विश्वास मत कर लो, उस बात को इसलिये भी न मान लो कि उस पर तुम्हारा जातीय विश्वास है अथवा बचपन से ही तुम्हें उस पर विश्वास कराया गया है; वरन् तुम स्वयं उस पर विचार करो, और विशेष रूप से विश्लेषण करने के बाद यदि देखो कि उससे तुम्हारा तथा दूसरों का भी कल्याण होगा, तभी उस पर विश्वास करो, उसी के अनुसार अपना जीवन बिताओ तथा दूसरों को भी उसी के अनुसार चलने में सहायता पहुँचाओ~।”

केवल वही व्यक्ति सब की अपेक्षा उत्तम रूप से कार्य करता है, जो पूर्णतया निःस्वार्थ है, जिसे न तो धन की लालसा है, न कीर्ति की और न किसी अन्य वस्तु की ही~। और मनुष्य जब ऐसा करने में समर्थ हो जाएगा, तो वह भी एक बुद्ध बन जाएगा और उसके भीतर से ऐसी कार्यशक्ति प्रकट होगी, जो संसार की अवस्था को सम्पूर्ण रूप से परिवर्तित कर सकती है~। वस्तुतः ऐसा ही व्यक्ति कर्मयोग के चरम आदर्श का एक ज्वलन्त उदाहरण है~।

