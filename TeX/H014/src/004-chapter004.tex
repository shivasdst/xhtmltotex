
\chapter{कर्तव्य क्या है}

कर्मयोग का तत्त्व समझने के लिए यह जान लेना आवश्यक है कि कर्तव्य क्या है~। यदि मुझे कोई काम करना है तो पहले मुझे यह जान लेना चाहिए कि वह मेरा कर्तव्य है, और तभी मैं उसे कर सकता हूँ~। विभिन्न जातियों में, विभिन्न देशों में इस कर्तव्य के सम्बन्ध में भिन्न-भिन्न धारणाएँ हैं~। एक मुसलमान कहता है कि जो कुछ कुरान-शरीफ में लिखा है, वही मेरा कर्तव्य हैं; इसी प्रकार एक हिन्दू कहता है कि जो कुछ मेरे वेदों में लिखा है, वही मेरा कर्तव्य है; फिर एक ईसाई की दृष्टि में जो कुछ उसकी बाइबिल में लिखा है, वही उसका कर्तव्य है~। इससे हमें स्पष्ट दीख पड़ता है कि जीवन में अवस्था, काल एवं जाति के भेद से कर्तव्य के सम्बन्ध में धारणाएँ भी बहुविध होती हैं~। अन्यान्य सार्वभौमिक भावसूचक शब्दों की तरह ‘कर्तव्य’ शब्द की भी ठीक-ठीक व्याख्या करना दुरूह है~। व्यावहारिक जीवन में उसकी परिणति तथा उसके फलाफलों द्वारा ही हमें उसके सम्बन्ध में कुछ धारणा हो सकती है~।

जब हमारे सामने कुछ बातें घटती हैं, तो हम सब लोगों में उस सम्बन्ध में एक विशेष रूप से कार्य करने की स्वाभाविक अथवा पूर्वसंस्कारानुयायी प्रवृत्ति उदित हो जाती है और प्रवृत्ति के उदय होने पर मन उस घटना के सम्बन्ध में सोचने लगता है~। कभी तो वह यह सोचता है कि इस प्रकार की अवस्था में इसी तरह कार्य करना उचित है, फिर किसी दूसरे समय उसी प्रकार की अवस्था होने पर भी पूर्वोक्त रूप से कार्य करना अनुचित प्रतीत होता है~। कर्तव्य के सम्बन्ध में सर्वत्र साधारण धारणा यही देखी जाती है कि सत्पुरुषगण अपने विवेक के आदेशानुसार कर्म किया करते हैं~। परन्तु वह क्या है, जिससे एक कर्म ‘कर्तव्य’ बन जाता है? जीने मरने की समस्या के समय एक ईसाई के सामने गो-मांस का एक टुकड़ा रहने पर भी यदि वह अपनी प्राणरक्षा के लिए उसे नहीं खाता अथवा किसी दूसरे मनुष्य के प्राण बचाने के लिए वह मांस नहीं दे देता तो उसे निश्चय ही ऐसा लगेगा कि उसने अपना कर्तव्य नहीं किया~। परन्तु इसी अवस्था में यदि एक हिन्दू स्वयं वह गो-मांस का टुकड़ा खा ले अथवा किसी दूसरे हिन्दू को दे दे, तो अवश्य उसे भी ठीक उसी प्रकार यह लगेगा कि उसने अपना कर्तव्य नहीं किया~। हिन्दू जाति की शिक्षा तथा संस्कार ही ऐसे हैं, जिनके कारण उसके हृदय में ऐसे भाव जागृत हो जाते हैं~।

पिछली शताब्दी में भारतवर्ष में डाकुओं का एक मशहूर दल था, जिन्हें ठग कहते थे~। वे किसी मनुष्य को मार डालना तथा उसका धन छीन लेना अपना कर्तव्य समझते थे~। वे जितने अधिक मनुष्यों को मारने में समर्थ होते थे, उतना ही अपने को श्रेष्ठ समझते थे~। साधारणतया यदि एक मनुष्य सड़क पर जाकर किसी दूसरे मनुष्य को बन्दूक से मार डाले, तो निश्चय ही उसे यह सोचकर दुःख होगा कि कर्तव्य-भ्रष्ट हो उसने अनुचित कार्य कर डाला है~। परन्तु यदि वही मनुष्य एक फौज में सिपाही की हैसियत से एक नहीं बल्कि बीसों आदमियों को भी मार डाले, तो उसे यह सोचकर अवश्य प्रसन्नता होगी कि उसने अपना कर्तव्य बहुत सुन्दर ढंग से निबाहा~। इस प्रकार यह स्पष्ट है कि केवल किसी कार्यविशेष का विचार करने से ही हमारा कर्तव्य निर्धारित नहीं होता~।

अतएव केवल बाह्य कार्यों के आधार पर कर्तव्य की व्याख्या करना नितान्त असम्भव है~। अमुक कार्य कर्तव्य है तथा अमुक अकर्तव्य - कर्तव्याकर्तव्य का इस प्रकार विभाग-निर्देश नहीं किया जा सकता~। परन्तु फिर भी आन्तरिक दृष्टिकोण (\enginline{Subjective side}) से कर्तव्य की व्याख्या हो सकती है~। यदि किसी कर्म द्वारा हम भगवान् की ओर बढ़ते हैं, तो वह सत् कर्म है और वह हमारा कर्तव्य है; परन्तु जिस कर्म द्वारा हम नीचे गिरते हैं, वह बुरा है; वह हमारा कर्तव्य नहीं~। आन्तरिक दृष्टिकोण से देखने पर हमें यह प्रतीत होता है कि कुछ कार्य ऐसे होते हैं, जो हमें उन्नत बनाते हैं, और दूसरे ऐसे, जो हमें नीचे ले जाते हैं और पशुवत् बना देते हैं~। किन्तु विभिन्न व्यक्तियों में कौनसा कार्य किस तरह का भाव उत्पन्न करेगा, यह निश्चित रूप से बताना असम्भव है~। सभी युगों में समस्त सम्प्रदायों और देशों के मनुष्यों द्वारा मान्य यदि कर्तव्य का कोई एक सार्वभौमिक भाव रहा है, तो वह है - “परोपकारः पुण्याय, पापाय परपीड़नम्~।” - अर्थात् परोपरकार ही पुण्य है, और दूसरों को दुःख पहुँचाना ही पाप है~।

श्रीमद्भगवद्गीता में जन्मगत तथा अवस्थागत कर्तव्यों का बारम्बार वर्णन है~। जीवन के विभिन्न कर्तव्यों के प्रति मनुष्य का जो मानसिक और नैतिक दृष्टिकोण रहता है, वह अनेक अंशों में उसके जन्म और उसकी अवस्था द्वारा नियमित होता है~। इसीलिए अपनी सामाजिक अवस्था के अनुरूप एवं हृदय तथा मन को उन्नत बनानेवाले कार्य करना ही हमारा कर्तव्य है~। परन्तु वह विशेष रूप से ध्यान रखना चाहिए कि सभी देश और समाज में एक ही प्रकार के आदर्श एवं कर्तव्य प्रचलित नहीं है~। इस विषय में हमारी अज्ञता ही एक जाति की दूसरी के प्रति घृणा का मुख्य कारण है~। एक अमेरिकानिवासी समझता है कि उसके देश की प्रथाएँ ही सर्वोत्कृष्ट हैं, अतएव जो कोई उसकी प्रथाओ के अनुसार बर्ताव नहीं करता, वह दुष्ट है~। इस प्रकार एक हिन्दू सोचता है कि उसी के रस्म-रिवाज संसार भर में ठीक और सर्वोत्तम हैं, और जो उनका पालन नहीं करता, वह महा दुष्ट है~। हम सहज ही इस भ्रम में पड़ जाते हैं, और ऐसा होना बहुत स्वाभाविक भी है~। परन्तु यह बहुत अहितकर है; संसार में परस्पर के प्रति सहानुभूति के अभाव एवं पारस्पारिक घृणा का यह प्रधान कारण है~। मुझे स्मरण है, जब मैं इस देश में आया और जब मैं शिकागो प्रदर्शनी में से जा रहा था, तो किसी आदमी ने पीछे से मेरा साफा खींच लिया~। मैंने पीछे घूमकर देखा, तो अच्छे कपड़े पहने हुए एक सज्जन दिखायी पड़े~। मैंने उनसे बातचीत की और जब उन्हें यह मालूम हुआ कि मैं अंग्रेजी भी जानता हूँ तो वे बहुत शरमिन्दा हुए~। इसी प्रकार, उसी सम्मेलन में एक दूसरे अवसर पर एक मनुष्य ने मुझे धक्का दे दिया; पीछे घूमकर जब मैंने उससे कारण पूछा, तो वह भी बहुत लज्जित हुआ और हकला-हकलाकर मुझसे माफी माँगते हुए कहने लगा, “आप ऐसी पोशाक क्यों पहनते हैं?” इन लोगों की सहानुभूति बस् अपनी ही भाषा और वेशभूषा तक सीमित थी~। शक्तिशाली जातियाँ कमजोर जातियों पर जो अत्याचार करती हैं, उसका अधिकांश इसी दुर्भावना के कारण होता है~। मानवमात्र के प्रति मानव का जो बन्धुभाव रहता है, उसको यह सोख लेता है~। सम्भव है, वह मनुष्य जिसने मेरी पोशाक के बारे में पूछा था तथा जो मेरे साथ मेरी पोशाक के कारण ही दुर्व्यवहार करना चाहता था, एक भला आदमी रहा हो, एक सन्तानवत्सल पिता और एक सभ्य नागरिक रहा हो; परन्तु उसकी स्वाभाविक सहृदयता का अन्त बस् उसी समय हो गया, जब उसने मुझ-जैसे एक व्यक्ति को दूसरे वेश में देखा~। सभी देशों में विदेशियों को अनेक अत्याचार सहने पड़ते हैं, क्योंकि वे यह नहीं जानते कि परदेश में अपने को कैसे बचाये~। और इस प्रकार वे उन देशवाशियों के प्रति अपने देश में भूल धारणाएँ साथ ले जाते हैं~। मल्लाह, सिपाही और व्यापारी दूसरे देशों में ऐसे अद्भुत व्यवहार किया करते हैं, जैसा अपने देश में करना वे स्वप्न में भी नहीं सोच सकेंगे~। शायद यही कारण है कि चीनी लोग यूरोप और अमेरिका निवासियों को ‘विदेशी भूत’ कहा करते हैं~। पर यदि उन्हें पश्चिमी देश की सज्जनता तथा उसकी नम्रता का भी अनुभव हुआ होता, तो वे शायद ऐसा न कहते~।

अतएव हमें जो एक बात विशेष रूप से ध्यान में रखनी चाहिए, वह यह है कि हम दूसरे के कर्तव्यों को उसी की दृष्टि से देखें, दूसरों के रीति-रिवाजों को अपने रीति-रिवाज के मापदण्ड से न जाँचें~। यह हमें विशेष रूप से जान लेना चाहिए कि हमारी धारणा के अनुसार सारा संसार नहीं चल सकता, हमें ही सारे संसार के साथ मिल-जुलकर चलना होगा, सारा संसार कभी भी हमारे भाव के अनुकूल नहीं चल सकता~। इस प्रकार हम देखते हैंं कि देश-काल-पात्र के अनुसार हमारे कर्तव्य कितने बदल जाते हैं~। और सबसे श्रेष्ठ कर्म तो यह है कि जिस विशिष्ट समय पर हमारा जो कर्तव्य हो, उसी को हम भलीभाँति निबाहें~। पहले तो हमें जन्म से प्राप्त कर्तव्य को करना चाहिए, और उसे कर चुकने के बाद, समाज-जीवन में हमारे ‘पद’ के अनुसार जो कर्तव्य हो, उसे सम्पन्न करना चाहिए~। प्रत्येक व्यक्ति जीवन में किसी न किसी अवस्था में अवस्थित है; उसके लिए पहले उसी अवस्थानुयायी कर्म करना आवश्यक है~। मानव-स्वभाव की एक विशेष कमजोरी यह है कि वह स्वयं अपनी ओर कभी नजर नहीं फेरता~। वह तो सोचता है कि मैं भी राजा के सिंहासन पर बैठने योग्य हूँ~। और यदि मान लिया जाय कि वह है भी, तो सबसे पहले उसे यह दिखा देना चाहिए कि वह अपने वर्तमान पद का कर्तव्य भली भाँति कर चुका है~। ऐसा होने पर तब उसके सामने उच्चतर कर्तव्य आएँगे~। जब संसार में हम लगन से काम शुरू करते हैं, तो प्रकृति हमें चारों ओर से धक्के देने लगती है और शीघ्र ही हमें इस योग्य बना देती है कि हम अपना वास्तविक पद निर्धारित कर सकें~। जो जिस कार्य के उपयुक्त नहीं है, वह दीर्घ काल तक उस पद में रहकर सबको सन्तुष्ट नहीं कर सकता~। अतएव प्रकृति हमारे लिए जिस कर्तव्य का विधान करती है, उसका विरोध करना व्यर्थ है~। यदि कोई मनुष्य छोटा कार्य करे, तो उसी कारण वह छोटा नहीं कहा जा सकता~। कर्तव्य के केवल ऊपरी रूप से ही मनुष्य की उच्चता या नीचता का निर्णय करना उचित नहीं, देखना तो यह चाहिए कि वह अपना कर्तव्य किस भाव से करता है~।

बाद में हम देखेंगे कि कर्तव्य की यह धारणा भी परिवर्तित हो जाती है और यह भी देखेंगे कि सबसे श्रेष्ठ कार्य तो तभी होता है, जब उसके पीछे किसी प्रकार स्वार्थ की प्रेरणा न हो~। फिर भी यह स्मरण रखना चाहिए कि कर्तव्य-ज्ञान से किया हुआ कर्म ही हमें कर्तव्य-ज्ञान से अतीत कर्म की ओर ले जाता है~। और तब कर्म उपासना में परिणत हो जाता है, - इतना ही नहीं, वरन् उस समय कर्म का अनुष्ठान केवल कर्म के लिए ही होता है~। फिर हमें प्रतीत होगा कि कर्तव्य, चाहे वह नीति पर अधिष्ठित हो अथवा प्रेम पर, उसका उद्देश्य वही है, जो अन्य किसी योग का - अर्थात् ‘कच्चे मैं’ को क्रमशः घटाते-घटाते बिलकुल नष्ट कर देना, जिससे अन्त में ‘पक्का मैं’ अपनी असली महिमा में प्रकाशित हो जाय, तथा निम्न स्तर में अपनी शक्तियों का क्षय होने से रोकना, जिससे आत्मा अधिकाधिक उच्च भूमि में प्रकाशमान हो सके~। नीच वासनाओं के उदय होने पर भी यदि हम उन्हें वश में ले आयें, तो उससे हमारी आत्मा की महिमा का विकास होता रहता है~। कर्तव्य पालन में भी इस स्वार्थत्याग की आवश्यकता अनिवार्य है~। इसी प्रकार ज्ञान अथवा अज्ञानवश सारी समाजसंस्था संगठित हुई है; वह मानो एक कार्यक्षेत्र है - सत्-असत् की एक परीक्षाभूमि है~। इस कार्यक्षेत्र में स्वार्थपूर्ण वासनाओं को धीरे-धीरे कम करते हुए हम मनुष्य के प्रकृत स्वरूप के अनन्त विकास का पथ खोल देते हैं~।

पर कर्तव्य का पालन शायद ही कभी मधुर होता हो~। कर्तव्यचक्र तभी हलका और आसानी से चलता है, जब उसके पहियो में प्रेमरूपी चिकनाई लगी होती है, नहीं तो यह निरन्तर एक घर्षणसा ही है~। यदि ऐसा न हो, तो माता-पिता अपने बच्चों के प्रति, बच्चे अपने माता-पिता के प्रति, पति अपनी स्त्री के प्रति तथा स्त्री अपने पति के प्रति अपना अपना कर्तव्य कैसे निभा सकें? क्या इस घर्षण के उदाहरण हमें अपने दैनिक जीवन में सदैव दिखायी नहीं देते? कर्तव्य पालन की मधुरता प्रेम में ही है, और प्रेम का विकास केवल स्वतन्त्रता में होता है~। परन्तु सोचो तो सही, इन्द्रियों का, क्रोध का, ईर्ष्या का तथा मनुष्य के जीवन में प्रतिदिन होनेवाली अन्य सैकड़ों छोटी-छोटी बातों का गुलाम होकर रहना क्या स्वतन्त्रता है? अपने जीवन के इन सब क्षुद्र संघर्षों में सहिष्णुता धारण करना ही स्वतन्त्रता की सर्वोच्च अभिव्यक्ति है~। स्त्रियाँ स्वयं अपने चिड़चिड़े एवं ईर्ष्यापूर्ण स्वभाव की गुलाम होकर अपने पतियों को दोष दिया करती है~। वे दावा करती हैं कि हम स्वाधीन हैं; परन्तु वे नहीं जानती कि ऐसा करने से वे स्वयं को निरी गुलाम सिद्ध कर रही है~। और यही हाल उन पतियों का भी है, जो सदैव अपनी स्त्रियों में दोष देखा करते हैं~।

पावित्र्य ही स्त्री और पुरुष का सर्वप्रथम धर्म है~। ऐसा उदाहरण शायद ही कहीं हो कि एक पुरुष - वह चाहे जितना भी पथभ्रष्ट क्यों न हो गया हो - अपनी नम्र, प्रेमपूर्ण तथा पतिव्रता स्त्री द्वारा ठीक रास्ते पर न लाया जा सके~। संसार अभी भी उतना गिरा नहीं है~। हम बहुधा संसार में बहुत से निर्दय पतियों तथा पुरुषों के भ्रष्टाचरण के बारे में सुनते रहते हैं; परन्तु क्या यह बात सच नहीं है कि संसार में उतनी ही निर्दय तथा भ्रष्ट स्त्रियाँ भी है?

यदि अमरीका की सभी स्त्रियाँ इतनी शुद्ध और पवित्र होतीं जितना कि वे दावा करती हैं, तो मुझे पूरा विश्वास है कि समस्त संसार में एक भी अपवित्र मनुष्य न रह जाता~। ऐसा कौनसा पाशविक भाव है, जिसे पावित्र्य और सतीत्व पराजित नहीं कर सकता? एक शुद्ध पतिव्रता स्त्री, जो अपने पति को छोड़कर अन्य सब पुरुषों को पुत्रवत् समझती है तथा उनके प्रति माता का भाव रखती है, धीरे-धीरे अपनी पवित्रता की शक्ति में इतनी उन्नत हो जाएगी कि एक अत्यन्त पाशविक प्रवृत्तिवाला मनुष्य भी उसके सान्निध्य में पवित्र वातावरण का अनुभव करेगा~। इसी प्रकार प्रत्येक पति को, अपनी स्त्री को छोड़कर अन्य सब स्त्रियों को अपनी माता, बहिन अथवा पुत्री के समान देखना चाहिए~। विशेषकर उस मनुष्य को, जो धर्म का प्रचारक होना चाहता है, यह आवश्यक है कि वह प्रत्येक स्त्री को मातृवत् देखे और उसके साथ सदैव तद्रूप व्यवहार करे~।

मातृपद ही संसार में सबसे श्रेष्ठ पद है, क्योंकि यही एक ऐसा पद है, जिससे अधिक से अधिक निःस्वार्थता की शिक्षा प्राप्त हो सकती है - निःस्वार्थ कार्य किया जा सकता है~। केवल भगवत्प्रेम ही माता के प्रेम से उच्च है, अन्य सब तो निम्न श्रेणी के हैं~। माता का कर्तव्य है कि पहले वह अपने बच्चों का सोचे और फिर अपना; परन्तु उसके बजाय यदि मातापिता सर्वदा पहले अपने ही बारे में सोचें तो फल यह होगा कि उनमें तथा उनके बच्चों में वही सम्बन्ध स्थापित हो जाएगा, जो चिड़ियों तथा उनके बच्चों में होता है~। चिड़ियों के बच्चे जब उड़ने योग्य हो जाते हैं, तो अपने माँ-बाप को पहिचानते तक नहीं~। वास्तव में वह पुरुष धन्य है, जो स्त्री को ईश्वर के मातृभाव की प्रतिमूर्ति समझता है; और वह स्त्री भी धन्य है, जो पुरुष का ईश्वर के पितृभाव की प्रतिमूर्ति मानती है; तथा वे बच्चे भी धन्य हैं, जो अपने माता-पिता को भगवान् का ही रूप मानते हैं~।

हमारी उन्नति का एकमात्र उपाय यह है कि हम पहले वह कर्तव्य करें, जो हमारे हाथ में है~। और इस प्रकार धीरे-धीरे शक्तिसंचय करते हुए क्रमशः हम सर्वोच्च अवस्था को प्राप्त कर सकते हैं~। किसी भी कर्तव्य को घृणा की दृष्टि से नहीं देखना चाहिए~। मैं पुनः कहता हूँ, जो व्यक्ति अपेक्षाकृत निम्न कार्य करता है, वह किसी उच्चतर कार्य करनेवाले की अपेक्षा निम्नतर श्रेणी का नहीं हो जाता~। केवल मनुष्य के कर्तव्य का रूप देखकर उसकी उच्चता-नीचता का विचार करने से नहीं बनेगा, देखना तो यह होगा कि वह उस कर्तव्य का पालन किस ढंग से करता है~। कार्य करने की उसकी शक्ति और ढंग से ही उसकी जाँच की जानी चाहिए~।

एक तरुण संन्यासी किसी बन में गया~। वहाँ उसने दीर्घ काल तक ध्यान-भजन तथा योगाभ्यास किया~। अनेक वर्षों की कठिन तपस्या के बाद एक दिन जब वह एक वृक्ष के नीचे बैठा था, तो उसके ऊपर वृक्ष से कुछ सूखी पत्तियाँ आ गिरी~। उसने ऊपर निगाह उठायी, तो देखा कि एक कौआ और एक बगुला पेड़ पर लड़ रहे हैं~। यह देखकर संन्यासी को बहुत क्रोध आया~। उसने कहा, “यह क्या! तुम्हारा इतना साहस कि तुम ये सूखी पत्तियाँ मेरे सिर पर फेंको?” इन शब्दों के साथ संन्यासी की क्रुद्ध आँखों से आग की एक ज्वाला-सी निकली, और वे बेचारी दोनों चिड़ियाँ उससे जलकर भस्म हो गयीं~। अपने में यह शक्ति देखकर वह संन्यासी बड़ा खुश हुआ; उसने सोचा, ‘वाह, अब तो मैं दृष्टि मात्र से कौए-बगुले को भस्म कर सकता हूँ~।’ कुछ समय बाद भिक्षा के लिए वह एक गाँव को गया~। गाँव में जाकर वह एक दरवाजे पर खड़ा हुआ और पुकारा, “माँ, कुछ भिक्षा मिले~।” भीतर से आवाज आयी, “थोड़ा रुको, मेरे बेटे~।” संन्यासी ने मन में सोचा, “अरे दुष्टा, तेरा इतना साहस कि तू मुझसे प्रतीक्षा कराये! अब भी तू मेरी शक्ति नहीं जानती?” संन्यासी ऐसा सोच ही रहा था कि भीतर से फिर एक आवाज आयी, “बेटा, अपने को इतना बड़ा मत समझ~। यहाँ न तो कोई कौआ है और न बगुला~।” यह सुनकर संन्यासी को बड़ा आश्चर्य हुआ~। बहुत देर तक खड़े रहने के बाद अन्त में घर में से एक स्त्री निकली और उसे देखकर संन्यासी उसके चरणों पर गिर पड़ा और बोला, “माँ, तुम्हें यह सब कैसे मालूम हुआ?” स्त्री ने उत्तर दिया, “बेटा, न तो मैं तुम्हारा योग जानती हूँ और न तुम्हारी तपस्या~। मैं तो एक साधारण स्त्री हूँ~। मैंने तुम्हें इसलिए थोड़ी देर रोका था कि मेरे पतिदेव बीमार है और मैं उनकी सेवा-शुश्रूषा में संलग्न थी~। यही मेरा कर्तव्य है~। सारे जीवन भर मैं इसी बात का यत्न करती रही हूँ कि मैं अपना कर्तव्य पूर्ण रूप से निबाहूँ~। जब मैं अविवाहित थी, तब मैंने अपने माता-पिता के प्रति कन्या का कर्तव्य किया और अब जब मेरा विवाह हो गया है, तो मैं अपने पतिदेव के प्रति पत्नी का कर्तव्य करती हूँ~। बस यही मेरा योगाभ्यास है~। अपना कर्तव्य करने से ही मेरे दिव्य चक्षु खुल गये हैं, जिससे मैंने तुम्हारे विचारों को जान लिया और मुझे इस बात का भी पता चल गया कि तुमने वन में क्या किया है~। यदि तुम्हें इससे भी कुछ उच्चतर तत्त्व जानने की इच्छा है, तो अमुक नगर के बाजार में जाओ, वहाँ तुम्हें एक व्याध मिलेगा~। वह तुम्हें कुछ ऐसी बातें बतलाएगा, जिन्हें सुनकर तुम बड़े प्रसन्न होगे~।” संन्यासी ने विचार किया, “भला मैं उस शहर में उस व्याध के पास क्यों जाऊँ?” परन्तु उसने अभी जो घटना देखी, उसे सोचकर उसकी आँखे कुछ खुल गयी~। अतएव वह उस शहर को गया~। जब वह शहर के नजदीक आया, तो उसने दूर से एक बड़े मोटे व्याध को बाजार में बैठे हुए और बड़े-बड़े छुरों से मांस काटते हुए देखा~। वह लोगों से अपना सौदा कर रहा था~। संन्यासी ने मन ही मन सोचा, “हरे! हरे! क्या यही वह व्यक्ति है, जिससे मुझे शिक्षा मिलेगी? दिखता तो यह शैतान का अवतार है!” इतने में व्याध ने संन्यासी की ओर देखा और कहा, “महाराज, क्या उस स्त्री ने आपको मेरे पास भेजा है? कृपया बैठ जाइये~। मैं जरा अपना काम समाप्त कर लूँ~।” संन्यासी ने सोचा, “यहाँ मुझे क्या मिलेगा?” खैर, वह बैठ गया~। इधर व्याध अपना काम लगातार करता रहा और जब वह अपना काम पूरा कर चुका, तो उसने अपने रुपये पैसे समेटे और संन्यासी से कहा, “चलिये महाराज, घर चलिये~।” घर पहुँचकर व्याध ने उन्हें आसन दिया और कहा, “आप यहाँ थोड़ा ठहरिये~।” व्याध अपने घर में चला गया~। उसने अपने वृद्ध माता-पिता को स्नान कराया, उन्हें भोजन कराया और उन्हें प्रसन्न करने के लिए जो कुछ कर सकता था, किया~। उसके बाद वह उस संन्यासी के पास आया और कहा, “महाराज, आप मेरे पास आये हैं~। अब बताइये, मैं आपकी क्या सेवा कर सकता हूँ?” संन्यासी ने उससे आत्मा तथा परमात्मा सम्बन्धी कुछ प्रश्न किये और उनके उत्तर में व्याध ने उसे जो उपदेश दिया, वही महाभारत में ‘व्याध-गीता’ के नाम से प्रसिद्ध है~। ‘व्याध-गीता’ में हमें वेदान्तदर्शन की बहुत उच्च बातें मिलती हैं~।

जब व्याध अपना उपदेश समाप्त कर चुका, तो संन्यासी को बड़ा आश्चर्य हुआ और उसने कहा, “फिर आप ऐसे क्यों रहते हैं? इतने ज्ञानी होते हुए भी आप व्याध क्यों हैं, इतना निन्दित और कुत्सित कार्य क्यों करते हैं?” व्याध ने उत्तर दिया, “वत्स, कोई भी कर्तव्य निन्दित नहीं है~। कोई भी कर्तव्य अपवित्र नहीं है~। मैं जन्म से ही इस परिस्थिति में हूँ, यही मेरा प्रारब्धलब्ध कर्म है~। बचपन से ही मैंने यह व्यापार सीखा है, परन्तु इसमें मेरी आसक्ति नहीं है~। कर्तव्य के नाते मैं इसे उत्तम रूप से किये जाता हूँ~। मैं गृहस्थ के नाते अपना कर्तव्य करता हूँ और अपने माता-पिता को प्रसन्न रखने के लिए जो कुछ मुझसे बन पड़ता है, करता हूँ~। न तो मैं तुम्हारा योग जानता हूँ और न मैं कभी संन्यासी ही हुआ~। संसार छोड़कर मैं कभी वन में नहीं गया~। परन्तु फिर भी जो कुछ तुमने मुझसे सुना तथा देखा, वह सब मुझे अनासक्त भाव से अपनी अवस्था के अनुरूप कर्तव्य का पालन करने से ही प्राप्त हुआ है~।”

भारतवर्ष में एक बहुत बड़े महात्मा\footnote{पवहारी बाबा एक प्रसिद्ध महात्मा थे~। इनका आश्रम गाजीपुर में था~। स्वामी विवेकानन्दजी कृत इनका एक संक्षिप्त जीवनचरित भी है, जो हिन्दी में रामकृष्ण मठ, नागपुर द्वारा प्रकाशित हुआ है~।} हैं~। अपने जीवन में मैंने जितने बड़े बड़े महात्मा देखे, उनमें से वे एक हैं~। वे एक बड़े अद्भुत व्यक्ति हैं; कभी किसी को उपदेश नहीं देते; यदि तुम उनसे कोई प्रश्न पूछो भी, तो भी वे उसका उत्तर नहीं देते~। गुरु का पद ग्रहण करने में वे बड़े संकुचित होते हैं~। यदि तुम उनसे आज एक प्रश्न पूछो और उसके बाद कुछ दिन प्रतीक्षा करो, तो किसी दिन अपनी बातचीत में वे उस प्रश्न को उठाकर उस पर बड़ा सुन्दर प्रकाश डालते हैं~। उन्होंने मुझे एक बार कर्म का रहस्य बताया था~। उन्होंने कहा, “साधन और सिद्धि को एकरूप समझो~।” अर्थात् साधनाकाल में साधन में ही मन प्राण अर्पण कर कार्य करो, क्योंकि उसकी चरम अवस्था का नाम ही सिद्धि है~। जब तुम कोई कर्म करो, तब अन्य किसी बात का विचार ही मत करो~। उसे एक उपासना के - बड़ी से बड़ी उपासना के बतौर करो, और उस समय तक के लिए उसमें अपना सारा तन मन लगा दो~। यही बात हमने उपरोक्त कथा में भी देखी है~। व्याध एवं वह स्त्री - दोनों ने अपना-अपना कर्तव्य बड़ी प्रसन्नता से तथा तन्मय होकर किया और उसका फल यह हुआ कि उन्हें दिव्य ज्ञान प्राप्त हुआ~। इससे हमें यह स्पष्ट प्रतीत होता है कि जीवन की किसी भी अवस्था में, कर्मफल में बिना आसक्ति रखे यदि कर्तव्य उचित रूप से किया जाय, तो उससे हमें परमपद की प्राप्ति होती है~।

कर्मफल में आसक्ति रखनेवाला व्यक्ति अपने भाग्य में आये हुए कर्तव्य पर भिनभिनाता है~। अनासक्त पुरुष को सब कर्तव्य एकसमान हैं~। उसके लिए तो वे कर्तव्य स्वार्थपरता तथा इन्द्रियपरायणता को नष्ट करके आत्मा को मुक्त कर देने के लिए शक्तिशाली साधन हैं~। हम अपने कर्तव्य पर जो भिनभिनाते हैं, उसका कारण यह है कि हम सब अपने को बहुत समझते हैं और अपने को बहुत योग्य समझा कहते हैं, यद्यपि हम वैसे हैं नहीं~। प्रकृति ही सदैव कड़े नियम से हमारे कर्मों के अनुसार उचित कर्मफल का विधान करती है; इसमें तनिक भी हेरफेर नहीं हो सकता और इसलिए अपनी ओर से चाहे हम किसी कर्तव्य को स्वीकार करने के लिए भले ही अनिच्छुक हों, फिर भी वास्तव में हमारे कर्मफल के अनुसार ही हमारे कर्तव्य निर्दिष्ट होंगे~। स्पर्धा से ईर्ष्या उत्पन्न होती है और उससे हृदय की कोमलता नष्ट हो जाती है~। असन्तुष्ट तथा तकरारी पुरुष के लिए सभी कर्तव्य नीरस होते हैं~। उसे तो कभी भी किसी चीज से सन्तोष नहीं होता और फलस्वरूप उसका जीवन दूभर हो उठना और असफल हो जाना स्वाभाविक है~। हमें चाहिए कि हम काम करते रहें; जो कुछ भी हमारा कर्तव्य हो, उसे करते रहे; अपना कन्धा सदैव काम से भिड़ाये रखें~। और तभी हमारा पथ ज्ञानालोक से आलोकित हो जाएगा~।

