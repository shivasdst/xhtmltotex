
\chapter{कर्मरहस्य}

दूसरों की शारीरिक आवश्यकताओं की पूर्ति करके उनकी सहायता करना महान कर्म अवश्य है, परन्तु अभाव की मात्रा जितनी अधिक रहती है तथा सहायता या उपकार जितनी अधिक दूर तक अपना असर कर सकता है, उसी मात्रा में वह उपकार भी उच्चतर होता है~। यदि एक मनुष्य के अभाव एक घण्टे के लिए हटाये जा सकें, तो यह उसकी सहायता आवश्यक है और यदि एक साल के लिए हटाये जा सकें, तो यह उससे भी अधिक सहायता है; पर यदि उसके अभाव सदा के लिए दूर कर दिये जायँ, तो सचमुच वह उसके लिए सब से अधिक सहायता होगी~। केवल आध्यात्मिक ज्ञान ही ऐसा है, जो हमारे दुःखों को सदा के लिए नष्ट कर दे सकता है; अन्य किसी प्रकार के ज्ञान से तो हमारी आवश्यकताओं की पूर्ति केवल अल्प समय के लिए ही होती है~। केवल आत्मविषयक ज्ञान द्वारा ही हमारी अभाववृत्ति तक का सदा के लिए अन्त हो सकता है~। अतएव किसी मनुष्य की आध्यात्मिक सहायता करना ही उसकी सबसे बड़ी सहायता करना है~। और जो मनुष्य को पारमार्थिक ज्ञान दे सकता है, वही मानवसमाज का सबसे बड़ा हितैषी है~। हम देखते भी हैं कि जिन व्यक्तियों ने मनुष्य की आध्यात्मिक सहायता की है, वे ही वास्तव में महान् शक्तिशाली थे~। कारण यह है कि आध्यात्मिकता ही हमारे जीवन के समस्त कृत्यों का सच्चा आधार है~। आध्यात्मिक शक्तिशाली पुरुष चाहे तो किसी भी विषय में क्षमतासम्पन्न हो सकता है~। और जब तक मनुष्य में आध्यात्मिक बल नहीं आता, तब तक उसकी भौतिक आवश्यकताएँ भी तो भलीभाँति तृप्त नहीं हो सकती~।

आध्यात्मिक सहायता से नीचे है - बौद्धिक विकास में सहायता करना~। पर साथ ही यह ज्ञान-दान भोजन तथा वस्त्र के दान से कही श्रेष्ठ है, इतना ही नहीं, वरन् प्राणदान से भी उच्च है, क्योंकि ज्ञान ही मनुष्य का प्रकृत जीवन है~। अज्ञान ही मृत्यु है, और ज्ञान जीवन~। यदि जीवन अन्धकारमय है और अज्ञान तथा क्लेश में बीतता है, तो वास्तव में ऐसे जीवन का मूल्य कुछ भी नहीं है~। ज्ञान-दान से नीचे है शारीरिक सहायता का दान~। अतएव हमारे सम्मुख जब दूसरों की सहायता का प्रश्न उपस्थित हो तो हमें इस भूल धारणा से सदा बचे रहने का प्रयत्न करना चाहिए कि शारीरिक सहायता ही एकमात्र सहायता है~। वास्तव में शारीरिक सहायता तो सब सहायताओं में केवल अन्तिम ही नहीं, वरन् निम्नतम श्रेणी की भी है, क्योंकि इसके द्वारा चिरतृप्ति नहीं हो सकती~। भूखे रहने से जो कष्ट होता है, उसका परिहार भोजन कर लेने से ही हो जाता है, परन्तु वह भूख पुनः लौट आती है~। हमारे क्लेशों का अन्त तो केवल तभी हो सकता है, जब हम तृप्त होकर सब प्रकार के अभावों से परे हो जायँ~। तब क्षुधा हमें पीड़ित नहीं कर सकती और न कोई क्लेश अथवा दुःख ही हमें विचलित कर सकता है~। अतएव, जो सहायता हमें आध्यात्मिक बल देती है, वह सर्वश्रेष्ठ है; उससे नीचे है बौद्धिक अथवा मानसिक सहायता, और सबसे नीचे है शारीरिक सहायता का स्थान~।

केवल शारीरिक सहायता द्वारा ही संसार के दुःखों से छुटकारा नहीं हो सकता~। जब तक मनुष्य का स्वभाव ही परिवर्तित नहीं हो जाता, तब तक ये शारीरिक आवश्यकताएँ सदा बनी ही रहेंगी और फलस्वरूप क्लेशों का सदैव अनुभव भी होता रहेगा~। कितनी भी शारीरिक सहायता उनका पूर्ण उपचार नहीं कर सकती~। इस दुःख-समस्या की केवल एक ही मीमांसा है और वह है - समस्त मानवजाति को पवित्र कर देना~। अपने चारों ओर हम जो दुःख-क्लेश देखते हैं, उन सबका केवल एक ही मूल कारण हैं - अज्ञान~। मनुष्य को ज्ञानालोक दो, उसे आध्यात्मिक-बल-सम्पन्न करो~। यदि हम यह करने में समर्थ हों - यदि सभी मनुष्य पवित्र, आध्यात्मिकबलसम्पन्न और सुशिक्षित हो जायँ, केवल तभी संसार में से दुःख का अन्त हो सकेगा, अन्यथा नहीं। देश के प्रत्येक घर को हम सदावर्त में भले ही परिणत कर दें, देश को अस्पतालों से भले ही भर दें, परन्तु जब तक मनुष्य का चरित्र परिवर्तित नहीं होता, तब तक दुःख-क्लेश बना ही रहेगा~।

\newpage

भगवद्गीता में हमें इस बात का बारम्बार उपदेश मिलता है कि हमें निरन्तर कर्म करते रहना चाहिए~। कर्म स्वभावतः ही सत्-असत् से मिश्रित होता है~। हम ऐसा कोई भी कर्म नहीं कर सकते, जिससे कहीं कुछ भला न हो; और ऐसा भी कोई कर्म नहीं है, जिससे कहीं-न-कहीं कुछ हानि न हो~। प्रत्येक कर्म अनिवार्य रूप से गुण-दोष से मिश्रित रहता है~। परन्तु फिर भी शास्त्र हमें सतत कर्म करते रहने का ही आदेश देते हैं~। सत् और असत् दोनों का अपना अलग-अलग फल होगा~। सत् कर्मों का फल सत् होगा और असत् कर्मों का फल असत्~। परन्तु सत् और असत् दोनों ही आत्मा के लिए बन्धनस्वरूप हैं~। इस सम्बन्ध में गीता का कथन है कि यदि हम अपने कर्मों में आसक्त न हों तो हमारी आत्मा पर किसी प्रकार का बन्धन नहीं पड़ सकता~।

अब हम देखेंगे कि ‘कर्मों में अनासक्ति’ का तात्पर्य क्या है~। गीता का मूल सूत्र यह है कि निरन्तर कर्म करते रहो, परन्तु उसमें आसक्त मत होओ~। जिस ओर मन का विशेष झुकाव होता है, स्थूल रूप से उसे ही ‘संस्कार’ कह सकते हैं~। यदि मन को एक तालाब मान लिया जाय, तो उसमें उठनेवाली प्रत्येक लहर जब शान्त हो जाती है तो वास्तव में वह बिलकुल नष्ट नहीं हो जाती, वरन् चित्त में एक प्रकार का चिह्न छोड़ जाती है तथा ऐसी सम्भावना का निर्माण कर जाती है, जिससे वह लहर दुबारा फिर से उठ सके~। इस चिह्न तथा इस लहर के फिर से उठने की सम्भावना को मिलाकर हम ‘संस्कार’ कह सकते हैं~। हमारा प्रत्येक कार्य, हमारा प्रत्येक अंग-संचालन, हमारा प्रत्येक विचार हमारे चित्त पर इसी प्रकार का एक संस्कार छोड़ जाता है; और यद्यपि ये संस्कार ऊपरी दृष्टि से स्पष्ट न हों, तथापि ये अज्ञात रूप से अन्दर-ही-अन्दर कार्य करने में विशेष प्रबल होते हैं~। हम प्रतिमुहूर्त जो कुछ हैं, वह इन संस्कारों के समुदाय द्वारा ही नियमित होता है~। मैं इस मुहूर्त जो कुछ हूँ, वह मेरे अतीत जीवन के समस्त संस्कारों का प्रभाव है~। इसे ही प्रकृत दृष्टि से ‘चरित्र’ कहते हैं और प्रत्येक मनुष्य का चरित्र इन संस्कारों की समष्टि द्वारा ही नियमित होता है~। यदि शुभ संस्कारों का प्राबल्य रहे, तो मनुष्य का चरित्र अच्छा होता है और यदि अशुभ संस्कारों का, तो बुरा~। यदि एक मनुष्य निरन्तर बुरे शब्द सुनता रहे, बुरे विचार सोचता रहे, बुरे कर्म करता रहे, तो उसका मन भी बुरे संस्कारों से पूर्ण हो जाएगा और बिना उसके जाने ही वे संस्कार उसके समस्त विचारों तथा कार्यों पर अपना प्रभाव डाल देंगे~। असल में ये बुरे संस्कार निरन्तर अपना कार्य करते रहते हैं~। अतएव बुरे संस्कार-सम्पन्न होने के कारण उस व्यक्ति के कार्य भी बुरे होंगे - वह एक बुरा आदमी बन जाएगा - इसके सिवाय अन्यथा होना असम्भव है~। ये बुरे संस्कार उसमें दुष्कर्म करने की प्रबल प्रवृत्ति उत्पन्न कर देंगे; वह तो इन संस्कारों के हाथ एक यन्त्र सा होकर रह जाएगा, वे उसे बलपूर्वक दुष्कर्म करने के लिए बाध्य करेंगे~। इसी प्रकार यदि एक मनुष्य अच्छे विचार रखे और सत्कार्य करे, तो उसके इन संस्कारों का प्रभाव भी अच्छा ही होगा तथा उसकी इच्छा न होते हुए भी वे उसे सत्कार्य करने के लिए प्रवृत्त करेंगे~। जब मनुष्य इतने सत्कार्य एवं सत्चिन्तन कर चुकता है कि उसकी इच्छा न होते हुए भी उसमें सत्कार्य करने की एक अनिवार्य प्रवृत्ति उत्पन्न हो जाती है, तब फिर यदि वह दुष्कर्म करना भी चाहे तो इन सब संस्कारों की समष्टि स्वरूप उसका मन उसे ऐसा करने से फौरन रोक देगा, इतना ही नहीं, वरन् उसके ये संस्कार उसे उस मार्ग पर से हटा देंगे~। तब वह अपने संस्कारों के हाथ एक कठपुतलीजैसा हो जाएगा~। जब ऐसी स्थिति हो जाती है, तभी उस मनुष्य का चरित्र गठित कहलाता है~।

जिस प्रकार कछुआ अपने सब अंगों को खपड़े के अन्दर समेट लेता है और तब उसे चाहे हम मार ही क्यों न डालें, उसके टुकड़े-टुकड़े ही क्यों न कर डालें, पर वह बाहर नहीं निकलता, इसी प्रकार जिस मनुष्य ने अपने मन एवं इन्द्रियों को वश में कर लिया है, उसका चरित्र भी सदैव स्थिर रहता है~। वह अपनी आभ्यन्तरिक शक्तियों को वश में रखता है और उसकी इच्छा के विरुद्ध संसार की कोई भी वस्तु उसके मन पर कार्य नहीं कर सकती~। मन के ऊपर इस प्रकार सद्विचारों एवं सुसंस्कारों का निरन्तर प्रभाव पड़ते रहने से सत्कार्य करने की प्रवृत्ति प्रबल हो जाती है और इसके फलस्वरूप हम इन्द्रियों (कर्मेंद्रिय तथा ज्ञानेंद्रिय दोनों) को वश में करने में समर्थ होते हैं~। तभी हमारा चरित्र प्रतिष्ठित होता है, तभी हम सत्यलाभ कर सकते हैं~। ऐसा ही मनुष्य सदैव निरापद रहता है, इससे किसी भी प्रकार की बुराई नहीं हो सकती~। वह फिर कहीं भी रहे, उसके लिए कोई धोखा नहीं है~। इन शुभ संस्कारों से सम्पन्न होने की अपेक्षा एक और भी अधिक उच्चतर अवस्था है और वह है - मुक्तिलाभ की इच्छा~। तुम्हें यह स्मरण रखना चाहिए कि सभी योगों का ध्येय आत्मा की मुक्ति है और प्रत्येक योग समान रूप से उसी ध्येय की ओर ले जाता है~। भगवान् बुद्ध ने ध्यान से तथा ईसा मसीह ने प्रार्थना द्वारा जिस अवस्था की प्राप्ति की थी, मनुष्य केवल कर्म द्वारा भी उस अवस्था को प्राप्त कर सकता है~। बुद्ध ज्ञानी थे, और ईसा मसीह भक्त; पर वे दोनों एक ही ध्येय को पहुँचे थे~। मुक्ति का अर्थ है सम्पूर्ण स्वाधीनता - शुभ और अशुभ दोनों प्रकार के बन्धनों से छुटकारा पा जाना~। इसे समझना जरा कठिन है~। लोहे की जंजीर भी एक जंजीर है और सोने की जंजीर भी एक जंजीर है~। यदि हमारी उँगली में एक काँटा चुभ जाय तो उसे निकालने के लिए हम एक दूसरा काँटा काम में लाते हैं, परन्तु जब वह निकल जाता है तो हम दोनों को ही फेंक देते है~। हमें फिर दूसरे काँटे को रखने की कोई आवश्यकता नहीं रह जाती, क्योंकि दोनों आखिर काँटे ही तो हैं~। इसी प्रकार कुसंस्कारों का नाश शुभ संस्कारों द्वारा करना चाहिए और मन के खराब विचारों को अच्छे विचारों द्वारा दूर करते रहना चाहिए, जब तक कि समस्त कुविचार लगभग नष्ट न हो जायँ अथवा पराजित न हो जायें या वशीभूत होकर मन में कहीं एक कोने में न पड़े रह जायँ~। परन्तु उसके उपरान्त शुभ संस्कारों पर भी विजय प्राप्त करना आवश्यक है~। तभी जो ‘आसक्त’ था, वह ‘अनासक्त’ हो जाता है~। कर्म करो, अवश्य करो, पर उस कर्म अथवा विचार को अपने मन के ऊपर कोई गहरा प्रभाव न डालने दो~। लहरें आयें और जायें, माँसपेशियों और मस्तिष्क से बड़े-बड़े कार्य होते रहें, पर देखना, वे आत्मा पर किसी प्रकार का गहरा प्रभाव न डालने पायें~।

अब प्रश्न यह है कि यह हो कैसे सकता है~। हम देखते हैं कि हम जिस किसी कर्म में लिप्त हो जाते हैं, उसका संस्कार हमारे मन में रह जाता है~। मान लो, सारे दिन में मैं सैकड़ों आदमियों से मिला और उन्हीं में एक ऐसे व्यक्ति से भी मिला, जिससे मुझे प्रेम है~। तब यदि रात को सोते समय मैं उन सब लोगों को स्मरण करने का प्रयत्न करूँ, तो देखूँगा कि मेरे सम्मुख केवल उसी व्यक्ति का चेहरा आता है जिसे मैं प्रेम करता हूँ, भले ही उसे मैंने केवल एक ही मिनट के लिए देखा हो~। उसके अतिरिक्त अन्य सब व्यक्ति अन्तर्हित हो जाएगे~। ऐसा क्यों? - इसलिए कि इस व्यक्ति के प्रति मेरी विशेष आसक्ति ने मेरे मन पर अन्य सभी लोगों की अपेक्षा एक अधिक गहरा प्रभाव डाल दिया था~। शरीरविज्ञान की दृष्टि से तो सभी व्यक्तियों का प्रभाव एक सा ही हुआ था~। प्रत्येक व्यक्ति का चेहरा नेत्रपट पर उतर आया था और मस्तिष्क में उसके चित्र भी बन गये थे~। परन्तु फिर भी मन पर इन सबका प्रभाव एक-समान न था~। सम्भवतः अधिकतर व्यक्तियों के चेहरे एकदम नये थे, जिनके बारे में मैने पहले कभी विचार भी न किया होगा; परन्तु वह एक चेहरा, जिसकी मुझे केवल एक झलक ही मिली थी, भीतर तक समा गया! शायद इस चेहरे का चित्र मेरे मन में वर्षों से रहा हो और मैं उसके बारे में सैकड़ों बातें जानता होऊँ; अतः उसकी इस एक झलक ने ही मेरे मन में उन सैकड़ों सोती हुई यादगारों को जगा दिया~। और इसलिए शेष अन्य सब चेहरों को देखने के समवेत फलस्वरूप मन में जितने सब संस्कार पड़े, इस एक चेहरे को देखने से मेरे मानसपटल पर उन सबकी अपेक्षा सौगुना अधिक संस्कार पड़ गया~। इसी कारण उसने मन के ऊपर सहज ही इतना प्रबल प्रभाव जमा दिया~।

{ अतएव अनासक्त होओ; कार्य होते रहने दो - मस्तिष्क के केन्द्र अपना-अपना कार्य करते रहें - निरन्तर कार्य करते रहे, परन्तु एक लहर को भी अपने मन पर प्रभाव मत डालने दो~। संसार में इस प्रकार कर्म करो, मानो तुम एक विदेशी पथिक हो, दो दिन के लिए यहाँ आये हो~। कर्म तो निरन्तर करते रहो, परन्तु अपने को बन्धन में मत डालो; बन्धन बड़ा भयानक है~। संसार हमारी निवासभूमि नहीं है; यह तो उन सोपानों में से एक है; जिनमें से होकर हम जा रहे हैं~। सांख्यदर्शन के\parfillskip=0pt}\newpage उस महावाक्य को मत भूलो, “समस्त प्रकृति आत्मा के लिए है, आत्मा प्रकृति के लिए नहीं~।”

प्रकृति के अस्तित्व का प्रयोजन आत्मा की शिक्षा के निमित्त ही है, इसका और कोई अर्थ नहीं~। उसका अस्तित्व इसीलिए है कि आत्मा को ज्ञानलाभ हो जाय तथा ज्ञान द्वारा आत्मा अपने को मुक्त कर ले~। यदि हम यह बात निरन्तर ध्यान में रखे, तो हम प्रकृति में कभी आसक्त न होंगे; हमें यह ज्ञान हो जाएगा कि प्रकृति हमारे लिए एक पुस्तकसदृश है जिसका हमें अध्ययन करना है; और जब हमें उससे आवश्यक ज्ञान प्राप्त हो जाएगा तो फिर वह पुस्तक हमारे लिए किसी काम की नहीं रहेगी~। परन्तु इसके विपरीत हो यह रहा है कि हम अपने को प्रकृति में ही मिला दे रहे हैं; यह सोच रहे हैं कि आत्मा प्रकृति के लिए है, आत्मा शरीर के लिए है; और जैसी कि एक कहावत है, हम सोचते हैं, ‘मनुष्य खाने के लिए ही जीवित रहता है, न कि जीवित रहने के लिए खाता है’; और यह भूल हम निरन्तर करते रहते हैं~। प्रकृति को ही ‘अहं’ मानकर हम प्रकृति में आसक्त बने रहते हैं~। और ज्यों ही इस आसक्ति का प्रादुर्भाव होता है, त्योंही आत्मा पर प्रबल संस्कार का निर्माण हो जाता है, जो हमें बन्धन में डाल देता है और जिसके कारण हम मुक्तभाव से कार्य न करके दास की तरह कार्य करते रहते हैं~।

इस सारी शिक्षा का सार यही है कि तुम्हें एक स्वामी के समान कार्य करना चाहिए, न कि एक दास की तरह~। कर्म तो निरन्तर करते रहो, परन्तु एक दास के समान मत करो~। सब लोग किस प्रकार कर्म कर रहे है, क्या यह तुम नहीं देखते? इच्छा होने पर भी कोई आराम नहीं ले सकता~। ९९ प्रतिशत लोग तो दासों की तरह कार्य करते रहते हैं और उसका फल होता है दुःख; ये सब कार्य स्वार्थपर होते हैं~। मुक्तभाव से कर्म करो, प्रेमसहित कर्म करो~। प्रेम शब्द का यथार्थ अर्थ समझना बहुत कठिन है~। बिना स्वाधीनता के प्रेम आ ही नहीं सकता~। दास में सच्चा प्रेम होना सम्भव नहीं~। यदि तुम एक गुलाम मोल ले लो और उसे जंजीरों से बाँधकर उससे अपने लिए कार्य कराओ, तो वह कष्ट उठाकर किसी प्रकार कार्य करेगा अवश्य, पर उसमें किसी प्रकार का प्रेम नहीं रहेगा~। इसी तरह जब हम संसार के लिए दासवत् कर्म करते हैं, तो इसके प्रति हमारा प्रेम नहीं रहता और इसलिए वह सच्चा कर्म नहीं हो सकता~। हम अपने बन्धु-बान्धवों के लिए जो कर्म करते हैं, जहाँ तक कि हम अपने स्वयं के लिए भी जो कर्म करते हैं, उसके बारे में भी ठीक यही बात है~।

स्वार्थ के लिए किया गया कार्य दास का कार्य है~। और कोई कार्य स्वार्थ के लिए है अथवा नहीं, इसकी पहचान यह है कि प्रेम के साथ किया हुआ प्रत्येक कार्य आनन्ददायक होता है~। सच्चे प्रेम के साथ किया हुआ कोई भी कार्य ऐसा नहीं है, जिसके फलस्वरूप शान्ति और आनन्द न आये~। प्रकृत सत्ता, प्रकृत ज्ञान तथा प्रकृत प्रेम - ये तीनों चिरकाल के लिए परस्पर सम्बद्ध हैं~। असल में ये एक ही में तीन हैं~। जहाँ एक रहता है, वहाँ शेष दो भी अवश्य रहते हैं~। ये उस अद्वितीय सच्चिदानन्द के ही त्रिविध रूप हैं~। जब वह सत्ता सान्त तथा सापेक्ष रूप में प्रतीत होती है, तो हम उसे विश्व के रूप में देखते हैं~। वह ज्ञान भी सांसारिक वस्तुविषयक ज्ञान के रूप में परिणत हो जाता है, तथा वह आनन्द मानव-हृदय में विद्यमान समस्त प्रकृत प्रेम की नींव हो जाता है~। अतएव सच्चे प्रेम से प्रेमी अथवा उसके प्रेमपात्र को भी कष्ट नहीं हो सकता~।

उदाहरणार्थ, मान लो एक मनुष्य एक स्त्री से प्रेम करता है~। वह चाहता है कि वह स्त्री केवल उसी के पास रहे; अन्य पुरुषों के प्रति उस स्त्री के प्रत्येक व्यवहार से उसमें ईर्ष्या का उद्रेक होता है~। वह चाहता है कि वह स्त्री उसी के पास बैठे, उसी के पास खड़ी रहे तथा उसी की इच्छानुसार खाये-पिये और चले-फिरे~। वह स्वयं उस स्त्री का गुलाम हो गया है और चाहता है कि वह स्त्री भी उसकी गुलाम होकर रहे~। यह तो प्रेम नहीं है~। यह तो गुलामी का एक प्रकार का विकृत भाव है, जो ऊपर से प्रेम-जैसा दिखायी देता है~। यह प्रेम नहीं हो सकता, क्योंकि वह क्लेशदायक है; यदि वह उस मनुष्य की इच्छानुसार न चले, तो उससे उस मनुष्य को कष्ट होता है~। वास्तव में सच्चे प्रेम की प्रतिक्रिया दुःखप्रद तो होती ही नहीं~। उससे तो केवल आनन्द ही होता है~। और यदि उससे ऐसा न होता हो, तो समझ लेना चाहिए कि वह प्रेम नहीं है, बल्कि वह और ही कोई चीज है, जिसे हम भ्रमवश प्रेम कहते हैं~। जब तुम अपने पति, अपनी स्त्री, अपने बच्चों, यहाँ तक कि समस्त विश्व को इस प्रकार प्रेम करने में सफल हो सकोगे कि उससे किसी भी प्रकार दुःख, ईर्ष्या अथवा स्वार्थपरता रूप कोई प्रतिक्रिया नहीं होगी, केवल तभी तुम ठीक-ठीक अनासक्त हो सकोगे~।

भगवान् श्रीकृष्ण अर्जुन से कहते हैं, “हे अर्जुन, यदि मैं कर्म करने से एक क्षण के लिए भी रुक जाऊँ, तो सारा विश्व ही नष्ट हो जाय~। मुझे कर्म से किसी भी प्रकार का लाभ नहीं; मैं ही जगत् का एकमात्र प्रभु हूँ - फिर भी मैं कर्म क्यों करता हूँ? - इसलिए कि मुझे संसार से प्रेम है~।” ईश्वर अनासक्त है~। क्यों? - इसलिए कि वे सच्चे प्रेमी हैं~। उस सच्चे प्रेम से ही हम अनासक्त हो सकते हैं~। जहाँ कहीं आसक्ति है, वहाँ जान लेना चाहिए कि वह केवल भौतिक आकर्षण है - केवल कुछ जड़ कणों के प्रति आकर्षण हो रहा है - मानो कोई एक चीज दो वस्तुओं को लगातार निकटतर खींचे ला रही है; और यदि वे दोनों वस्तुएँ काफी निकट नहीं आ सकतीं, तो फिर कष्ट उत्पन्न होता है~। परन्तु जहाँ सच्चा प्रेम है, वहाँ भौतिक आकर्षण बिलकुल नहीं रहता~। ऐसे प्रेम चाहे सहस्रों योजन दूर पर क्यों न रहे, उनका प्रेम सदैव वैसा ही रहता है, वह प्रेम कभी नष्ट नहीं होता, उससे कभी कोई क्लेशदायक प्रतिक्रिया नहीं होती~।

इस प्रकार की अनासक्ति प्राप्त करना लगभग सारे जीवन भर का कार्य है~। परन्तु इसका लाभ होते ही हमें अपनी प्रेमसाधना का लक्ष्य प्राप्त हो जाता है और हम मुक्त हो जाते हैं~। तब हम प्रकृति के बन्धन से छूट जाते हैं और उसके असली स्वरूप को जान लेते हैं~। फिर वह हमें बन्धन में नहीं डाल सकती~। तब हम बिलकुल स्वाधीन हो जाते हैं और कर्म के फलाफल की ओर ध्यान नहीं देते~। फिर कौन परवाह करता है कि कर्मफल क्या होगा?

\newpage

जब तुम अपने बच्चों को कोई चीज देते हो, तो क्या उसके बदले में उनसे कुछ माँगते हो? यह तो तुम्हारा कर्तव्य है कि तुम उनके लिए काम करो, और बस वहीं पर बात खत्म हो जाती है~। इसी प्रकार, किसी दूसरे पुरुष, किसी नगर अथवा देश के लिए तुम जो कुछ करो, उसके प्रति भी वैसा ही भाव रखो; उनसे किसी प्रकार के बदले की आशा न रखो~। यदि तुम सदैव ऐसा ही भाव रख सको कि तुम केवल दाता ही हो, जो कुछ तुम देते हो, उससे तुम किसी प्रकार के प्रत्युपकार की आशा नहीं रखते, तो उस कर्म से तुम्हें किसी प्रकार की आसक्ति नहीं होगी~। आसक्ति तभी आती है, जब हम बदले की आशा रखते हैं~।

यदि दासवत् कार्य करने से स्वार्थपरता और आसक्ति उत्पन्न होती है, तो अपने मन का स्वामी बनकर कार्य करने से अनासक्ति द्वारा उत्पन्न आनन्द का लाभ होता है~। हम बहुधा अधिकार और न्याय की बातें किया करते हैं, परन्तु वे सब केवल एक बच्चे की बोली के समान हैं~। मनुष्य के चरित्र का नियमन करनेवाली दो चीजें होती हैं~। एक जोर-जुल्म और दूसरी दया~। जोर-जुल्म का उपयोग करना सदैव स्वार्थपरतावश ही होता है~। बहुधा सभी स्त्री-पुरुष अपनी शक्ति एवं सुविधा का यथासम्भव उपयोग करने का प्रयत्न करते हैं~। दया दैवी सम्पत्ति है~। भले बनने के लिए हमें दयायुक्त होना चाहिए; यहाँ तक कि न्याय और अधिकार भी दया पर ही प्रतिष्ठित होने चाहिए~। कर्मफल की लालसा तो हमारी आध्यात्मिक उन्नति के मार्ग में बाधक है; इतना ही नहीं अन्त में उससे क्लेश भी उत्पन्न होता है~। दया और निःस्वार्थपरता को कार्य रूप में परिणत करने का एक और उपाय है - और वह है कर्मों को उपासनारूप मानना, यदि हम साकार ईश्वर में विश्वास रखते हैं~। यहाँ हम अपने समस्त कर्मों के फल ईश्वर को ही समर्पित कर देते हैं~। और इस प्रकार उनकी उपासना करते हुए हमें इस बात का कोई अधिकार नहीं रह जाता कि हम अपने किये हुए कर्मों के बदले में मानवजाति से कुछ माँगें~। प्रभु स्वयं निरन्तर कार्य करते रहते हैं और वे सारी आसक्ति से परे हैं~। जिस प्रकार पानी कमल के पत्ते को नहीं भिगो सकता, उसी प्रकार कोई कर्म भी फलासक्ति उत्पन्न करके निःस्वार्थी पुरुष को बन्धन में नहीं डाल सकता~। अहं-शून्य और अनासक्त पुरुष किसी जनपूर्ण और पापमय नगर के बीच ही क्यों न रहे, पर पाप उन्हें स्पर्श तक न कर सकेगा~।

निम्नलिखित कहानी सम्पूर्ण स्वार्थत्याग का एक दृष्टान्त है~। कुरुक्षेत्र के युद्ध के बाद पाँचों पाण्डवों ने एक बड़ा भारी यज्ञ किया~। उसमें निर्धनों को बहुतसा दान दिया गया~। सभी लोगों ने उस यज्ञ की महत्ता एवं ऐश्वर्य पर आश्चर्य प्रकट किया और कहा कि ऐसा यज्ञ संसार में इसके पहले कभी नहीं हुआ था~। यज्ञ के बाद उस स्थान पर एक छोटासा नेवला आया~। नेवले का आधा शरीर सुनहला था और शेष आधा भूरा~। वह नेवला उस यज्ञभूमि की मिट्टी पर लोटने लगा~। थोड़ी देर बाद उसने दर्शकों से कहा, “तुम सब झूठे हो~। यह कोई यज्ञ नहीं है~।” लोगों ने कहा, “क्या! तुम कहते क्या हो! यह कोई यज्ञ ही नहीं है? तुम जानते हो, इस यज्ञ में कितना खर्च हुआ है, गरीबों को कितने हीरे-जवाहरात बाँटे गये हैं, जिससे वे सब के सब धनी एवं खुशहाल हो गये है? यह तो इतना बड़ा यज्ञ था कि ऐसा शायद ही किसी मनुष्य ने किया हो~।” परन्तु नेवले ने कहा, “सुनो, एक छोटेसे गाँव में एक निर्धन ब्राह्मण रहता था, साथ ही उसकी स्त्री, पुत्र और पुत्र-वधू~। वे लोग बड़े गरीब थे~। पूजा-पाठ से उन्हें जो कुछ मिलता, उसी पर उनका निर्वाह होता था~। एक बार उस गाँव में तीन साल तक अकाल पड़ा, जिससे उस बेचारे ब्राह्मण के दुःख कष्ट की पराकाष्ठा हो गयी~। एक बार तो सारे कुटुम्ब को पाँच दिन तक उपवास करना पड़ा~। छठे दिन वह ब्राह्मण भाग्यवश कहीं से थोड़ासा जौ का आटा ले आया~। उस आटे के चार भाग किये गये~। उन्होंने उसकी रोटी बनायी और ज्यों ही वे उसे खाने बैठे कि किसी ने दरवाजे पर खटखटाया~। पिता ने उठकर दरवाजा खोला, तो देखते हैं कि बाहर एक अतिथि खड़ा है~। भारतवर्ष में अतिथि बड़ा पवित्र माना जाता है~। वह तो उस समय के लिए ‘नारायण’ ही समझा जाता है और उसके साथ तद्रूप व्यवहार भी किया जाता है~। अतएव उस गरीब ब्राह्मण ने कहा, ‘महाराज, पधारिये, आपका स्वागत है~।’ और उसने अतिथि के सामने अपना भाग रख दिया~। अतिथि उसे जल्दी खा गया और बोला, ‘अरे, आपने तो मुझे और भी मार डाला~। मैं दस दिन का भूखा हूँ और भोजन के इस छोटे टुकड़े ने तो मेरी भूख और भी बढ़ा दी~।’ तब स्त्री ने अपने पति से कहा, ‘आप मेरा भी भाग दे दीजिये~।’ पति ने कहा, ‘नहीं, ऐसा नहीं होगा~।’ परन्तु स्त्री अपनी बात पर अड़ी रही और कहा, ‘वह बेचारा गरीब भूखा है, हमारे यहाँ आया है~। गृहस्थ की हैसियत से हमारा यह धर्म है कि हम उसे भोजन करायें~। यह देखकर कि आप उसे अधिक नहीं दे सकते, पत्नी के नाते मेरा यह कर्तव्य है कि मैं उसे अपना भी भाग दे दूँ~।’ ऐसा कह उसने भी अपना भाग अतिथि को दे दिया~। अतिथि ने वह भी खा लिया और कहा, ‘मैं तो भूख से अभी भी जल रहा हूँ~।’ तब लड़के ने कहा, ‘आप मेरा भाग भी लीजिये, क्योंकि पुत्र का यह धर्म है कि वह पिता के कर्तव्यों को पूरा करने में उन्हें सहायता दे~।’ अतिथि ने वह भी खा लिया, परन्तु फिर भी उसकी तृप्ति नहीं हुई~। अतएव बहू ने भी उसे अपना भाग दे दिया~। अब यह पर्याप्त हो गया और अतिथि ने उनको आशीर्वाद दे बिदा ली~।

“उसी रात वे चारों बेचारे भूख से पीड़ित हो मर गये~। उस आटे के कुछ कण इधर उधर जमीन पर बिखर गये थे और जब मैंने उन पर लोट लगायी, तो मेरा आधा शरीर सुनहला हो गया, जैसा कि तुम अभी देख ही रहे हो~। उस समय से मैं संसार भर में भ्रमण कर रहा हूँ और चाहता हूँ कि किसी दूसरी जगह भी मुझे ऐसा ही यज्ञ देखने को मिले; परन्तु वैसा यज्ञ मुझे कहीं देखने को नहीं मिला~। मेरा शेष आधा शरीर किसी दूसरी जगह सुनहला न हो सका~। इसीलिए तो कहता हूँ कि यह कोई यज्ञ ही नहीं है~।”

त्याग का यह भाव भारतवर्ष से धीरे-धीरे लुप्त होता जा रहा है; महानुभाव व्यक्तियों की संख्या धीरे-धीरे कम होती जा रही है~। जब बचपन में मैंने अंग्रेजी पढ़ना आरम्भ किया था, उस समय मैंने एक अंग्रेजी की पुस्तक पढ़ी, जिसमें एक ऐसे कर्तव्यपरायण बालक का वर्णन था, जिसने काम करके जो कुछ उपार्जन किया था, उसका कुछ भाग अपनी वृद्ध माता को दे दिया था~। उस बालक के इस कृत्य की प्रशंसा पुस्तक के तीन-चार पृष्ठों में गायी गयी थी~। परन्तु इसमें कौन सा असाधारणत्व है? कोई भी हिन्दू बालक उस कहानी की नीतिशिक्षा को नहीं समझ सकता~। और मुझे भी उसका महत्त्व आज ही समझ में आ रहा है, जब मैं इस पश्चिमी रिवाज को सुनता तथा देखता हूँ कि यहाँ प्रत्येक मनुष्य अपने-अपने लिए ही है~। इस देश में ऐसे भी लोग अनेक हैं, जो सब कुछ अपने ही लिए रख लेते हैं, - उनके पिता, माता, स्त्री और बच्चों की फिर चाहे जैसी दशा क्यों न हो~। एक गृहस्थ का ऐसा आदर्श तो कदापि न होना चाहिए~।

अब तुमने देखा, कर्मयोग का अर्थ क्या है~। उसका अर्थ है - मौत के मुँह में भी बिना तर्क-वितर्क के सबकी सहायता करना~। भले ही तुम लाखों बार ठगे जाओ, पर मुँह से एक बात तक न निकालो; और तुम जो कुछ भले कार्य कर रहे हो, उनके सम्बन्ध में सोचो तक नहीं~। निर्धन के प्रति किये गये उपकार पर गर्व मत करो और न उससे कृतज्ञता की ही आशा रखो; बल्कि उलटे तुम्हीं उसके कृतज्ञ होओ - यह सोचकर कि उसने तुम्हें दान देने का एक अवसर दिया है~। अतएव यह स्पष्ट है कि एक आदर्श संन्यासी होने की अपेक्षा एक आदर्श गृहस्थ होना अधिक कठिन है~। यथार्थ कर्ममय जीवन यथार्थ त्यागमय जीवन की अपेक्षा यदि अधिक कठिन नहीं, तो कम से कम उसके बराबर कठिन तो अवश्य है~।

