
\chapter{कर्म का चरित्र पर प्रभाव}

\vskip 2pt

\noindent कर्म शब्द ‘कृ’ धातु से निकला है; ‘कृ’ धातु का अर्थ है करना~। जो कुछ किया जाता है, वही कर्म है~। इस शब्द का पारिभाषिक अर्थ ‘कर्मफल’ भी होता है~। दार्शनिक दृष्टि से यदि देखा जाय, तो इसका अर्थ कभी-कभी वे फल होते हैं, जिनका कारण हमारे पूर्व कर्म रहते हैं~। परन्तु कर्मयोग में ‘कर्म’ शब्द से हमारा मतलब केवल ‘कार्य’ ही है~। मानवजाति का चरम लक्ष्य ज्ञानलाभ है~। प्राच्य दर्शनशास्त्र हमारे सम्मुख एकमात्र यही लक्ष्य रखता है~। मनुष्य का अन्तिम ध्येय सुख नहीं वरन् ज्ञान है; क्योंकि सुख और आनन्द का तो एक न एक दिन अन्त हो ही जाता है~। अतः यह मान लेना कि सुख ही चरम लक्ष्य है, मनुष्य की भारी भूल है~। संसार में सब दुःखों का मूल यही है कि मनुष्य अज्ञानवश यह समझ बैठता है कि सुख ही उसका चरम लक्ष्य है~। पर कुछ समय के बाद मनुष्य को यह बोध होता है कि जिसकी ओर वह जा रहा है, वह सुख नहीं वरन् ज्ञान है, तथा सुख और दुःख दोनों ही महान् शिक्षक हैं, और जितनी शिक्षा उसे सुख से मिलती है, उतनी ही दुःख से भी~। सुख और दुःख ज्यों-ज्यों आत्मा पर से होकर जाते रहते हैं, त्यों-त्यों वे उसके ऊपर अनेक प्रकार के चित्र अंकित करते जाते हैं~। और इन चित्रों अथवा संस्कारों की समष्टि के फल को ही हम मानव का ‘चरित्र’ कहते हैं~। यदि तुम किसी मनुष्य का चरित्र देखो, तो प्रतीत होगा कि वास्तव में वह उसकी मानसिक प्रवृत्तियों एवं मानसिक झुकाव की समष्टि ही है~। तुम यह भी देखोगे कि उसके चरित्रगठन में सुख और दुःख दोनों ही समान रूप से उपादानस्वरूप है~। चरित्र को एक विशिष्ट ढाँचे में ढालने में अच्छाई और बुराई दोनों का समान अंश रहता है, और कभीकभी तो दुःख-सुख से भी बड़ा शिक्षक हो जाता है~। यदि हम संसार के महापुरुषों के चरित्र का अध्ययन करें, तो मैं कह सकता हूँ कि अधिकांश दशाओं में हम यही देखेंगे कि सुख की अपेक्षा दुःख ने तथा सम्पत्ति की अपेक्षा दारिद्र्य ने ही उन्हें अधिक शिक्षा दी है एवं प्रशंसा की अपेक्षा निन्दारूपी आघात ने ही उसकी अन्तःस्थ ज्ञानाग्नि को अधिक प्रस्फुरित किया है~।

अब, यह ज्ञान मनुष्य में अन्तर्निहित है~। कोई भी ज्ञान बाहर से नहीं आता, सब अन्दर ही है~। हम जो कहते हैं कि मनुष्य ‘जानता’ है उसे ठीक ठीक मनोवैज्ञानिक भाषा में व्यक्त करने पर हमें कहना चाहिए कि वह ‘आविष्कार करता है~। मनुष्य जो कुछ सीखता है, वह वास्तव में ‘आविष्कार करना’ ही है। ‘आविष्कार’ का अर्थ है - मनुष्य का अपनी अनन्त ज्ञानस्वरूप आत्मा के ऊपर से आवरण को हटा लेना~। हम कहते हैं कि न्यूटन ने गुरुत्वाकर्षण का आविष्कार किया~। तो क्या वह आविष्कार कहीं एक कोने में बैठा हुआ न्यूटन की प्रतीक्षा कर रहा था? नहीं, वह उसके मन में ही था~। जब समय आया तो उसने उसे ढूँढ निकाला~। संसार ने जो कुछ ज्ञान लाभ किया है, वह मन से ही निकला है~। विश्व का असीम पुस्तकालय तुम्हारे मन में ही विद्यमान है~। बाह्य जगत् तो तुम्हें अपने मन को अध्ययन में लगाने के लिए उद्दीपक तथा सहायक मात्र है; परन्तु प्रत्येक समय तुम्हारे अध्ययन का विषय तुम्हारा मन ही है~। सेव का गिरना न्यूटन के लिए उद्दीपक कारणस्वरूप हुआ और उसने अपने मन का अध्ययन किया~। उसने अपने मन में पूर्व से स्थित भाव-शृंखला की कड़ियों को एक बार फिर से व्यवस्थित किया तथा उनमें एक नयी कड़ी का आविष्कार किया~। उसी को हम गुरुत्वाकर्षण का नियम कहते हैं~। यह न तो सेव में था और न पृथ्वी के केन्द्र में स्थित किसी अन्य वस्तु में ही~। अतएव समस्त ज्ञान, चाहे वह व्यावहारिक हो अथवा पारमार्थिक, मनुष्य के मन में ही निहित है~। बहुधा यह प्रकाशित न होकर ढका रहता है~। और जब आवरण धीरे-धीरे हटता जाता है, तो हम कहते है कि ‘हमें ज्ञान हो रहा है’~। ज्यों-ज्यों इस आविष्करण की क्रिया बढ़ती जाती है, त्यों-त्यों हमारे ज्ञान की वृद्धि होती जाती है~। जिस मनुष्य पर से यह आवरण उठता जा रहा है, वह अन्य व्यक्तियों की अपेक्षा अधिक ज्ञानी है, और जिस मनुष्य पर यह आवरण तह-पर-तह पड़ा है, वह अज्ञानी है~। जिस मनुष्य पर से यह आवरण बिलकुल चला जाता है, वह सर्वज्ञ पुरुष कहलाता है~। अतीत में कितने ही सर्वज्ञ पुरुष हो चुके हैं और मेरा विश्वास है कि अब भी बहुत से होंगे तथा आगामी युगों में भी ऐसे असंख्य पुरुष जन्म लेंगे~। जिस प्रकार एक चकमक पत्थर के टुकड़े में अग्नि निहित रहती है, उसी प्रकार मनुष्य के मन में ज्ञान रहता है~। उद्दीपककारण घर्षणस्वरूप हो इस ज्ञानाग्नि को प्रकाशित कर देता है~। ठीक ऐसा ही हमारे समस्त भावों और कार्यों के सम्बन्ध में भी है~। यदि हम शान्त होकर स्वयं का अध्ययन करें, तो प्रतीत होगा कि हमारा हँसना-रोना, सुख-दुःख, हर्ष-विषाद, हमारी शुभ कामनाएँ एवं शाप, स्तुति और निन्दा ये सब हमारे मन के ऊपर बहिर्जगत् के अनेक घात-प्रतिघात के फलस्वरूप उत्पन्न हुए हैं, और हमारा वर्तमान चरित्र इसी का फल है~। ये सब घात-प्रतिघात मिलकर ‘कर्म’ कहलाते है~। आत्मा की आभ्यन्तरिक अग्नि तथा उसकी अपनी शक्ति एवं ज्ञान को बाहर प्रकट करने के लिए जो मानसिक अथवा भौतिक घात उस पर पहुँचाये जाते हैं, वे ही कर्म हैं~। यहाँ कर्म शब्द का उपयोग व्यापक रूप में किया गया है~। इस प्रकार, हम सब प्रतिक्षण ही कर्म करते रहते हैं~। मैं तुमसे बातचीत कर रहा हूँ - यह कर्म है, तुम सुन रहे हो - यह भी कर्म है; हमारा साँस लेना, चलना आदि भी कर्म है; जो कुछ हम करते हैं, वह शारीरिक हो अथवा मानसिक, सब कर्म ही है; और हमारे ऊपर वह अपना चिह्न अंकित कर जाता है~।

कई कार्य ऐसे भी होते हैं, जो मानो अनेक छोटे-छोटे कर्मों की समष्टि हैं~। उदाहरणार्थ, यदि हम समुद्र के किनारे खड़े हो और लहरों को किनारे से टकराते हुए सुनें, तो ऐसा मालूम होता है कि एक बड़ी भारी आवाज हो रही है~। परन्तु हम जानते हैं कि एक बड़ी लहर असंख्य छोटी-छोटी लहरों से बनी है~। और यद्यपि प्रत्येक छोटी लहर अपना शब्द करती है, परन्तु फिर भी वह हमें सुन नहीं पड़ता~। पर ज्योंही ये सब शब्द आपस में मिलकर एक हो जाते हैं, त्योंही बड़ी आवाज सुनायी देती है~। इसी प्रकार हृदय की प्रत्येक धड़कन से कार्य हो रहा है~। कई कार्य ऐसे होते हैं, जिनका हम अनुभव करते हैं; वे हमारे इन्द्रियग्राह्य हो जाते हैं, पर साथ ही वे अनेक छोटे-छोटे कार्यों की समष्टिस्वरूप हैं~। यदि तुम सचमुच किसी मनुष्य के चरित्र को जाँचना चाहते हो, तो उसके बड़े कार्यों पर से उसकी जाँच मत करो~। एक मूर्ख भी किसी विशेष अवसर पर बहादुर बन जाता है~। मनुष्य के अत्यन्त साधारण कार्यों की जाँच करो, और असल में वे ही ऐसी बातें हैं, जिनसे तुम्हें एक महान् पुरुष के वास्तविक चरित्र का पता लग सकता है~। आकस्मिक अवसर तो छोटे-से-छोटे मनुष्य को भी किसी-न-किसी प्रकार का बड़प्पन दे देते हैं~। परन्तु वास्तव में बड़ा तो वही है, जिसका चरित्र सदैव और सब अवस्थाओं में महान् रहता है~।

मनुष्य का जिन सब शक्तियों के साथ सम्बन्ध आता है, उनमें से कर्मों की वह शक्ति सब से प्रबल है, जो मनुष्य के चरित्रगठन पर प्रभाव डालती है~। मनुष्य तो मानो एक प्रकार का केन्द्र है, और वह संसार की समस्त शक्तियों को अपनी ओर खींच रहा है, तथा इस केन्द्र में उन सारी शक्तियों को आपस में मिलाकर उन्हें फिर एक बड़ी तरंग के रूप में बाहर भेज रहा है~। यह केन्द्र ही ‘प्रकृत मानव’ (आत्मा) है; यह सर्वशक्तिमान् तथा सर्वज्ञ है और समस्त विश्व को अपनी ओर खींच रहा है~। भला-बुरा, सुख-दुःख सब उसकी ओर दौड़े जा रहे हैं, और जाकर उसके चारों ओर मानो लिपटे जा रहे हैं~। और वह उन सब में से चरित्र-रूपी महाशक्ति का गठन करके उसे बाहर भेज रहा है~। जिस प्रकार किसी चीज को अपनी ओर खींच लेने की उसमें शक्ति है, उसी प्रकार उसे बाहर भेजने की भी है~।

संसार में हम जो सब कार्य-कलाप देखते हैं, मानव-समाज में जो सब गति हो रही है, हमारे चारों ओर जो कुछ हो रहा है, वह सारा-का सारा केवल मन का ही खेल है - मनुष्य की इच्छाशक्ति का प्रकाश मात्र है~। अनेक प्रकार के यन्त्र, नगर, जहाज, युद्धपोत आदि सभी मनुष्य की इच्छाशक्ति के विकास मात्र हैं~। मनुष्य की यह इच्छाशक्ति चरित्र से उत्पन्न होती है और वह चरित्र कर्मों से गठित होता है~। अतएव, कर्म जैसा होगा, इच्छाशक्ति का विकास भी वैसा ही होगा~। संसार में प्रबल इच्छाशक्ति सम्पन्न जितने महापुरुष हुए है, वे सभी धुरन्धर कर्मी थे~। उनकी इच्छाशक्ति ऐसी जबरदस्त थी कि वे संसार को भी उलट-पुलट कर सकते थे~। और यह शक्ति उन्हें युग-युगान्तर तक निरन्तर कर्म करते रहने से प्राप्त हुई थी~। बुद्ध एवं ईसा मसीह में जैसी प्रबल इच्छाशक्ति थी, वह एक जन्म में प्राप्त नहीं की जा सकती~। और उसे हम आनुवंशिक शक्तिसंचार भी नहीं कह सकते, क्योंकि हमें ज्ञात है कि उनके पिता कौन थे~। हम नहीं कह सकते कि उनके पिता के मुँह से मनुष्य-जाति की भलाई के लिए शायद कभी एक शब्द भी निकला हो~। जोसेफ (ईसा मसीह के पिता) के समान तो असंख्य बढ़ई हो गये और आज भी है; बुद्ध के पिता के सदृश लाखों छोटे-छोटे राजा हो चुके हैं~। अतः यदि वह बात केवल आनुवंशिक शक्तिसंचार के ही कारण हुई हो, तो इसका स्पष्टीकरण कैसे कर सकते हो कि इस छोटे से राजा को, जिसकी आज्ञा का पालन शायद उसके स्वयं के नौकर भी नहीं करते थे, ऐसा एक सुन्दर पुत्र-रत्न लाभ हुआ, जिसकी उपासना लगभग आधा संसार करता है? इसी प्रकार, जोसेफ नामक बढ़ई तथा संसार में लाखों लोगों द्वारा ईश्वर के समान पूजे जानेवाले उसके पुत्र ईसा मसीह के बीच जो अन्तर है, उसका स्पष्टीकरण कहाँ? आनुवंशिक शक्तिसंचार के सिद्धान्त द्वारा तो इसका स्पष्टीकरण नहीं हो सकता~। बुद्ध और ईसा इस विश्व में जिस महाशक्ति का संचार कर गये, वह आयी कहाँ से? उस महान् शक्ति का उद्भव कहाँ से हुआ? अवश्य, युग-युगान्तरों से वह उस स्थान में ही रही होगी, और क्रमशः प्रबलतर होते-होते अन्त में बुद्ध तथा ईसा मसीह के रूप में समाज में प्रकट हो गयी, और आज तक चली आ रही है~।

यह सब कर्म द्वारा ही नियमित होता है~। यह सनातन नियम है कि जब तक कोई मनुष्य किसी वस्तु का उपार्जन न करे, तब तक वह उसे प्राप्त नहीं हो सकती~। सम्भव है, कभी-कभी हम इस बात को न मानें, परन्तु आगे चलकर हमें इसका दृढ़ विश्वास हो जाता है~। एक मनुष्य चाहे समस्त जीवन भर धनी होने के लिए एड़ी-चोटी का पसीना एक करता रहे, हजारों मनुष्यों को धोखा दे, परन्तु अन्त में वह देखता है कि वह सम्पत्तिशाली होने का अधिकारी नहीं था~। तब जीवन उसके लिए दुःखमय और दूभर हो उठता है~। हम अपने भौतिक सुखों के लिए भिन्न-भिन्न चीजों को भले ही इकट्ठा करते जायँ, परन्तु वास्तव में जिसका उपार्जन हम अपने कर्मों द्वारा करते हैं, वही हमारा होता है~। एक मूर्ख संसार भर की सारी पुस्तकें मोल लेकर भले ही अपने पुस्तकालय में रख ले, परन्तु वह केवल उन्हीं को पढ़ सकेगा, जिनको पढ़ने का वह अधिकारी होगा, और यह अधिकार कर्म द्वारा ही प्राप्त होता है~। हम किसके अधिकारी हैं, हम अपने भीतर क्या-क्या ग्रहण कर सकते हैं, इस सबका निर्णय कर्म द्वारा ही होता है~। अपनी वर्तमान अवस्था के जिम्मेदार हम ही है; और जो कुछ हम होना चाहें, उसकी शक्ति भी हमीं में है~। यदि हमारी-वर्तमान अवस्था हमारे ही पूर्व कर्मों का फल है, तो यह निश्चित है कि जो कुछ हम भविष्य में होना चाहते हैं, वह हमारे वर्तमान कार्यों द्वारा ही निर्धारित किया जा सकता है~। अतएव हमें यह जान लेना आवश्यक है कि कर्म किस प्रकार किये जायँ~। सम्भव है, तुम कहो, “कर्म करने की शैली जानने से क्या लाभ? संसार में प्रत्येक मनुष्य किसी-न-किसी प्रकार से तो काम करता ही रहता है~।” परन्तु यह भी ध्यान रखना चाहिए कि शक्तियों का निरर्थक क्षय भी कोई चीज होती है~। गीता का कथन है, “कर्मयोग का अर्थ है - कुशलता से अर्थात वैज्ञानिक प्रणाली से कर्म करना~।” कर्मानुष्ठान की विधि ठीक-ठीक जानने से मनुष्य को श्रेष्ठ फल प्राप्त हो सकता है~। यह स्मरण रखना चाहिए कि समस्त कर्मों का उद्देश्य है मन के भीतर पहले से ही स्थित शक्ति को प्रकट कर देना - आत्मा को जागृत कर देना~। प्रत्येक मनुष्य के भीतर पूर्ण शक्ति और पूर्ण ज्ञान विद्यमान है~। भिन्न-भिन्न कर्म इन महान् शक्तियों को जागृत करने तथा बाहर प्रकट कर देने में साधन मात्र हैं~।

मनुष्य नाना प्रकार के हेतु लेकर कार्य किया करता है, क्योंकि बिना हेतु के कार्य हो ही नहीं सकता~। कुछ लोग यश चाहते हैं, और वे यश के लिए काम करते हैं~। दूसरे पैसा चाहते हैं, और वे पैसे के लिए काम करते हैं~। फिर कोई अधिकार प्राप्त करना चाहते हैं, और वे अधिकार के लिए काम करते हैं~। कुछ और स्वर्ग पाना चाहते हैं, और वे उसी के लिए प्रयत्न करते हैं~। फिर कुछ लोग अपने बाद अपना नाम छोड़ जाने के इच्छुक होते हैं~। चीन देश में प्रथा है कि मनुष्य की मृत्यु के बाद ही उसे उपाधि दी जाती है; किसी ने यदि बहुत अच्छा कार्य किया, तो उसके मृत-पिता अथवा पितामह को कोई सम्माननीय उपाधि दे दी जाती है~। कुछ लोग उसी के लिए यत्न करते हैं~। विचार करके देखने पर यह प्रथा हमारे यहाँ की अपेक्षा अच्छी ही कही जा सकती है~। इस्लाम धर्म के कुछ सम्प्रदायों के अनुयायी इस बात के लिए आजन्म काम करते रहते हैं कि मृत्यु के बाद उनकी एक बड़ी कब्र बने~। मैं कुछ ऐसे सम्प्रदायों को जानता हूँ, जिनमें बच्चे के पैदा होते ही उसके लिए एक कब्र बना दी जाती हैं, और यही उन लोगों के अनुसार मनुष्य का सब से जरूरी काम होता है~। जिसकी कब्र जितनी बड़ी और सुन्दर होती है, वह उतना ही अधिक सुखी समझा जाता है~। कुछ लोग प्रायश्चित के रूप में कर्म किया करते हैं, अर्थात् अपने जीवन भर अनेक प्रकार के दुष्ट कर्म कर चुकने के बाद एक मन्दिर बनवा देते हैं, अथवा पुरोहितों को कुछ धन दे देते हैं, जिससे कि वे उनके लिए मानो स्वर्ग का टिकट खरीद देंगे! वे सोचते हैं कि बस इससे रास्ता साफ हो गया, अब हम निर्विघ्न चले जाएँगे~। इस प्रकार, मनुष्य को कार्य में लगानेवाले बहुत से उद्देश्य रहते हैं, ये उनमें से कुछ हुए~।

अब कार्य के लिए ही कार्य - इस सम्बन्ध में हम कुछ आलोचना करें~। प्रत्येक देश में कुछ ऐसे नर-रत्न होते हैं, जो केवल कर्म के लिए ही कर्म करते हैं~। वे नाम-यश अथवा स्वर्ग की भी परवाह नहीं करते~। वे केवल इसलिए कर्म करते हैं कि उससे दूसरों की भलाई होती है~। कुछ लोग ऐसे भी होते हैं, जो और भी उच्चतर उद्देश्य लेकर गरीबों के प्रति भलाई तथा मनुष्य-जाति की सहायता करने के लिए अग्रसर होते हैं, क्योंकि भलाई में उनका विश्वास है और उसके प्रति प्रेम है~। देखा जाता है कि नाम तथा यश के लिए किया गया कार्य बहुधा शीघ्र फलित नहीं होता~। ये चीजें तो हमें उस समय प्राप्त होती हैं, जब हम वृद्ध हो जाते हैं और जिन्दगी की आखिरी घड़ियाँ गिनते रहते हैं~। यदि कोई मनुष्य निःस्वार्थता से कार्य करे, तो क्या उसे कोई फलप्राप्ति नहीं होती? असल में तभी तो उसे सर्वोच्च फल की प्राप्ति होती है और सच पूछा जाय, तो निःस्वार्थता अधिक फलदायी होती है, पर लोगों में इसका अभ्यास करने का धीरज नहीं रहता~। व्यावहारिक दृष्टि से भी यह अधिक लाभदायक है~। प्रेम, सत्य तथा निःस्वार्थता नीतिसम्बन्धी आलंकारिक वर्णन मात्र नहीं है, वे तो हमारे सर्वोच्च आदर्श है, क्योंकि वे शक्ति की महान् अभिव्यक्ति हैं~। पहली बात यह है कि यदि कोई मनुष्य पाँच दिन, उतना क्यों, पाँच मिनट भी बिना भविष्य का चिन्तन किये, बिना स्वर्ग, नरक या अन्य किसी के सम्बन्ध में सोचे, निःस्वार्थता से काम कर सके तो वह एक महापुरुष बन सकता है~। यद्यपि इसे कार्यरूप में परिणत करना कठिन है, फिर भी अपने हृदय के अन्तस्तल से हम इसका महत्त्व समझते हैं और जानते हैं कि इससे क्या भलाई होती है~। यह शक्ति की महत्तम अभिव्यक्ति है - इसके लिए प्रबल संयम की आवश्यकता है~। अन्य सब बहिर्मुखी कर्मों की अपेक्षा इस आत्मसंयम में शक्ति का अधिक प्रकाश होता है~। एक चार घोड़ोंवाली गाड़ी उतार पर बड़ी आसानी से धड़धड़ाती हुई आ सकती है, अथवा सईस घोड़ों को रोक सकता है~। इन दोनों में अधिक शक्ति का विकास किसमें होता है? घोड़ों को छोड़ देने में, अथवा उन्हें रोकने में? एक तोप का गोला हवा में काफी दूर तक चला जाता है और फिर गिर पड़ता है~। परन्तु दूसरा दीवार से टकराकर रुक जाने से उतनी दूर नहीं जा सकता, पर उस टकराने से बड़ी आग-सी निकलती है~। इसी प्रकार, मन की सारी बहिर्मुखी गति किसी स्वार्थपूर्ण उद्देश्य की ओर दौड़ती रहने से छिन्न-भिन्न होकर बिखर जाती है, वह फिर तुम्हारे पास लौटकर तुम्हारे शक्तिविकास में सहायक नहीं होती~। परन्तु यदि उसका संयम किया जाय, तो उससे शक्ति की वृद्धि होती है~। इस आत्मसंयम से एक महान् इच्छाशक्ति का प्रादुर्भाव होता है; वह एक ऐसे चरित्र का निर्माण करता है, जो जगत् को अपने इशारे पर चला सकता है~। अज्ञानियों को इस रहस्य का पता नहीं रहता, परन्तु फिर भी वे मनुष्य-जाति पर शासन करने के इच्छुक रहते हैं~। एक अज्ञानी पुरुष भी यदि धीरज रखे, तो समस्त संसार पर शासन कर सकता है~। यदि वह कुछ वर्ष तक धीरज रखे तथा अपने इस अज्ञानसुलभ जगत्-शासन के भाव को संयत कर ले, तो इस भाव के समूल नष्ट होते ही वह संसार पर शासन कर सकेगा~। परन्तु जिस प्रकार कुछ पशु अपने से दो-चार कदम आगे कुछ नहीं देख सकते, इसी प्रकार हममें से अधिकांश लोग भविष्य के बारे में नितान्त अदूरदर्शी होते हैं~। हमारा संसार मानो एक क्षुद्र वर्तुल-सा होता है, हम बस उसी में आबद्ध रहते हैं~। हममें दूरदर्शिता के लिए धैर्य नहीं रहता और इसीलिए हम दुष्ट और नीच हो जाते हैं~। यह हमारी कमजोरी है - शक्तिहीनता है~।

अत्यन्त सामान्य कर्मों को भी घृणा की दृष्टि से नहीं देखना चाहिए~। जो मनुष्य कोई श्रेष्ठ आदर्श नहीं जानता, उसे स्वार्थ दृष्टि से ही - नाम-यश के लिए ही - काम करने दो~। परन्तु यह आवश्यक है कि प्रत्येक मनुष्य को उच्चतर ध्येयों की ओर बढ़ने तथा उन्हें समझने का प्रबल यत्न करते रहना चाहिए~। “हमें कर्म करने का ही अधिकार हैं, कर्म फल में हमारा कोई अधिकार नहीं~।” कर्मफलों को एक ओर रहने दो, उनकी चिन्ता हमें क्यों हो? यदि तुम किसी मनुष्य की सहायता करना चाहते हो, तो इस बात की कभी चिन्ता न करो कि उस आदमी का व्यवहार तुम्हारे प्रति कैसा रहता है~। यदि तुम एक श्रेष्ठ एवं भला कार्य करना चाहते हो, तो यह सोचने का कष्ट मत करो कि उसका फल क्या होगा~।

अब कर्म के इस आदर्श के सम्बन्ध में एक कठिन प्रश्न उठता है~। कर्मयोगी के लिए सतत कर्मशीलता आवश्यक है; हमें सदैव कर्म करते रहना चाहिए~। बिना कार्य के हम एक क्षण भी नहीं रह सकते~। तो फिर प्रश्न यह है कि आराम के बारे में क्या होगा? यहाँ इस जीवन-संग्राम के एक ओर है कर्म, जिसके तीव्र भँवर में फँसे हम लोग चक्कर काट रहे हैं और दूसरी ओर है शान्ति - सभी मानो निवृत्तिमुखी हैं, चारों ओर सब शान्त, स्थिर - किसी प्रकार का कोलाहल नहीं, केवल प्रकृति अपने जीवों, पुष्पों और गिरि-गुफाओं के साथ विराज रही है~। पर इन दोनों में से कोई भी पूर्ण आदर्श का चित्र नहीं है~। यदि एक ऐसा मनुष्य जिसे एकान्तवास का अभ्यास है, संसार के चक्कर में घसीट लाया जाय, तो उसका उसी प्रकार ध्वंस हो जाएगा, जिस प्रकार समुद्र की गहराई में रहनेवाली एक विशेष प्रकार की मछली पानी की सतह पर लाये जाते ही टुकड़े-टुकड़े होकर मर जाती है; क्योंकि सतह पर पानी का वह दबाव नहीं है, जिसके कारण वह जीवित रहती थी~। इसी प्रकार, एक ऐसा मनुष्य, जो सांसारिक तथा सामाजिक जीवन के कोलाहल का अभ्यस्त रहा है, यदि किसी शान्त स्थान को ले जाया जाय, तो क्या वह शान्तिपूर्वक रह सकता है? कदापि नहीं~। उसे क्लेश होता है, और सम्भव है उसका मस्तिष्क ही फिर जाय~। आदर्श पुरुष तो वे हैं, जो परम शान्त एवं निस्तब्धता के बीच भी तीव्र कर्म का, तथा प्रबल कर्मशीलता के बीच भी मरुस्थल की शान्ति एवं निस्तब्धता का अनुभव करते हैं~। उन्होंने संयम का रहस्य जान लिया है - अपने ऊपर विजय प्राप्त कर चुके हैं~। किसी बड़े शहर की भरी हुई सड़कों के बीच से जाने पर भी उनका मन उसी प्रकार शान्त रहता है, मानो वे किसी निःशब्द गुफा में हों, और फिर भी उनका मन सारे समय कर्म में तीव्र रूप से लगा रहता है~। यही कर्मयोग का आदर्श है, और यदि तुमने यह प्राप्त कर लिया है, तो तुम्हे वास्तव में कर्म का रहस्य ज्ञात हो गया~।

परन्तु हमें शुरू से आरम्भ करना पड़ेगा~। जो कार्य हमारे सामने आते जायँ, उन्हें हम हाथ में लेते जायँ और शनैःशनैः हम अपने को दिन-प्रतिदिन\break निःस्वार्थ बनाने का प्रयत्न करें~। हमें कर्म करते रहना चाहिए तथा यह पता लगाना\break चाहिए कि उस कार्य के पीछे हमारा हेतु क्या है~। ऐसा होने पर हम देख पाएँगे कि\break आरम्भावस्था में प्रायः हमारे सभी कार्यों का हेतु स्वार्थपूर्ण रहता है~। किन्तु धीरे-धीरे यह स्वार्थपरायणता अध्यवसाय से नष्ट हो जाएगी, और अन्त में वह समय आ जाएगा, जब हम वास्तव में स्वार्थ से रहित होकर कार्य करने के योग्य हो सकेंगे~। हम सभी यह आशा कर सकते हैं कि जीवन-पथ में अग्रसर होते-होते किसी-न-किसी दिन वह समय अवश्य ही आएँगा, जब हम पूर्ण रूप से निःस्वार्थ बन जाएँगे; और ज्योंही हम उस अवस्था को प्राप्त कर लेंगे, हमारी समस्त शक्तियाँ केन्द्रीभूत हो जाएँगी तथा हमारा आभ्यन्तरिक ज्ञान प्रकट हो जाएगा~।

