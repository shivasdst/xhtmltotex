
\chapter*{वक्तव्य}

\begin{center}
\textbf{(प्रथम संस्करण)}
\end{center}

हिन्दी जनता के सम्मुख ‘कर्मयोग’ पुस्तक का यह संस्करण रखते हमें बड़ी प्रसन्नता होती है। ‘कर्मयोग’ स्वामी विवेकानन्दजी की एक अत्यन्त महत्त्वपूर्ण पुस्तक है। इस पुस्तक में उनके जो व्याख्यान संकलित किये गये हैं, उनका मुख्य उद्देश्य मनुष्य जीवन को गढ़ना ही है। इन व्याख्यानों को पढ़ने से हमें ज्ञात होगा कि स्वामीजी के विचार किस उच्च कोटि के तथा हमारे दैनिक जीवन के लिए कितने उपयोगी रहे हैं। आज की परिस्थिति में संसार के लिए कर्मयोग का असली रूप समझ लेना बहुत आवश्यक है और विशेषकर भारतवर्ष के लिए। हम आशा करते हैं कि यह पुस्तक भिन्न भिन्न क्षेत्रों में कार्य करनेवाले सज्जनों के लिए बहुत ही उपयोगी सिद्ध होगी। आवश्यकता इतनी ही है कि इसके भावों एवं विचारों का मनन कर उन्हें कार्यरूप में परिणत किया जाय।

पं. डा. विद्याभास्करजी शुक्ल, एम. एस-सी., पी-एच. डी., प्रोफेसर, कॉलेज ऑफ साइन्स, नागपुर के प्रति हम परम कृतज्ञ हैं जिन्होंने इस पुस्तक का मूल अंग्रेजी से हिन्दी में अनुवाद किया है। इस पुस्तक में मूल ग्रंथ का ओज, उसकी शैली तथा भाव ज्यों के त्यों रखे गए हैं।

हमें विश्वास है कि जनता को इस पुस्तक से विशेष लाभ होगा।

\bigskip

\noindent नागपुर\hfill\textbf{- प्रकाशक}

\noindent दि. २५-५-१९५०

