
\chapter{ಪ್ರಾಚ್ಯ ಮತ್ತು ಪಾಶಾತ್ಯಾ}

\begin{center}
ಪೀಠಿಕೆ
\end{center}

ವಿಶಾಲವೂ ಆಳವೂ ಆಗಿದ್ದು ಉಕ್ಕಿ ರಭಸವಾಗಿ ಹರಿಯುತ್ತಿರುವ ನದಿಗಳು; ಅವುಗಳ ತೀರದಲ್ಲಿ ನಂದನವನವನ್ನು ನಾಚಿಸುವ ಉದ್ಯಾನವನಗಳು; ಅವುಗಳ ನಡುವೆ ಅತ್ಯುತ್ತಮವಾದ ಕಲಾಕೃತಿಗಳಿಂದ ಅಲಂಕೃತವಾದ, ಗಗನ ಚುಂಬಿ ಗಳಾದ ಅಮೃತ ಶಿಲೆಯ ಸುಂದರ ಅರಮನೆಗಳು; ಅವುಗಳ ಅಕ್ಕಪಕ್ಕಗಳಲ್ಲೂ ಹಿಂದೆ ಮುಂದೆಯೂ ಇನ್ನೇನು ಉರುಳಿ ಬೀಳಲಿರುವ ಮಣ್ಣಿನ ಗೋಡೆಗಳಿಂದಲೂ ಕುಸಿಯುತ್ತಿರುವ ಛಾವಣಿಗಳಿಂದಲೂ ಕೂಡಿದ ಗುಡಿಸಲುಗಳ ಗುಂಪುಗಳು; ಅವುಗಳ ಗಳುಗಳು ಅಸ್ಥಿಪಂಜರಗಳಂತೆ ಕಾಣುತ್ತಿವೆ; ಅಲ್ಲಲ್ಲಿ ಚಿಂದಿಯುಟ್ಟು ಚಿಕ್ಕವರು ದೊಡ್ಡವರು ಓಡಾಡುತ್ತಿದ್ದಾರೆ, ಅವರ ಮುಖಗಳಲ್ಲಿ ನೂರಾರು ವರ್ಷಗಳ ದುಃಖ ದಾರಿದ್ರ್ಯಗಳ ಆಳವಾದ ಗೆರೆಗಳು ಕಾಣುತ್ತಿವೆ; ಎತ್ತೆತ್ತಲೂ ಬಡಕಲಾದ ಹಸುಗಳು ಎತ್ತುಗಳು ಎಮ್ಮೆಗಳು; ಅವುಗಳ ಕಣ್ಣುಗಳಲ್ಲೂ ಅದೇ ಯಾತನೆಯ ನೋಟವೇ- ಇದು ನಮ್ಮ ಇಂದಿನ ಭಾರತ.

ಅರಮನೆಗಳ ಪಕ್ಕದಲ್ಲೇ ಶಿಥಿಲವಾದ ಗುಡಿಸಲುಗಳು; ದೇವಸ್ಥಾನಗಳ ಸಮೀಪ ದಲ್ಲೇ ಕಸದ ರಾಶಿಗಳು; ಬೆಲೆ ಬಾಳುವ ಉಡುಗೆ ತೊಡುಗೆಗಳನ್ನು ತೊಟ್ಟು ಓಡಾಡುತ್ತಿರುವವರ ಪಕ್ಕದಲ್ಲಿ ತಿಂದು ಕೊಬ್ಬಿದ ಶ‍್ರೀಮಂತರ ಮೇಲೆ ತಮ್ಮ ದೈನ್ಯದಿಂದ ಕೂಡಿದ ಕಾಂತಹೀನವಾದ ಕಣ್ಣುಗಳ ನೋಟವನ್ನು ಹರಿಸುತ್ತಿರುವ ಹಸಿದು ಬಳಲಿದ ಜನರು, ಅವರ ಪಕ್ಕದಲ್ಲಿ ಕೌಪೀನ ಮಾತ್ರವನ್ನು ಧರಿಸಿದ ಸಂನ್ಯಾಸಿ-ಇದು ನಮ್ಮ ಮಾತೃಭೂಮಿ.

ಭೀಕರವಾದ ಪ್ಲೇಗು ಕಾಲರಗಳಿಂದಾದ ಸಂಪೂರ್ಣ ನಾಶ; ಈ ರಾಷ್ಟ್ರದ ಜೀವಸತ್ತ್ವವನ್ನೇ ಹೀರುತ್ತಿರುವ ಮಲೇರಿಯಾ; ಜನರ ಎರಡನೆಯ ಸ್ವಭಾವವೊ ಎಂಬ್ಂತಹ ಉಪವಾಸ, ಅರೆಯೂಟ, ಮೃತ್ಯು ಸ್ವರೂಪಿಯಾದ ಕ್ಷಾಮದ ದುರಂತ ನೃತ್ಯ, ರೋಗರುಜಿನಗಳ ದುಃಖದಾರಿದ್ರ್ಯಗಳ ಕುರುಕ್ಷೇತ್ರ; ಆಸೆ ಚಟುವಟಿಕೆ ಸುಖ ಸಾಹಸಗಳನ್ನು ಕಳೆದುಕೊಂಡು ಸತ್ತ ಜನರ ಎಲುಬುಗಳು ಚಲ್ಲಾಪಿಲ್ಲಿಯಾಗಿ ಹರಡಿರುವ ಮಹಾಸ್ಮಶಾಣ; ಅದರ ನಡುವೆ ಮೋಕ್ಷದ ಹೊರತು ಬದುಕಿನಲ್ಲಿ ಬೇರಾವ ಗುರಿಯೂ ಇಲ್ಲದೆ ಆತ್ಮಾನುಸಂಧಾನದಲ್ಲಿ ಆಳವಾಗಿ ಮುಳುಗಿ ಗಂಭೀರ ಮೌನದಲ್ಲಿ ಕುಳಿತಿರುವ ಯೋಗಿ- ಭಾರತದಲ್ಲಿ ಪ್ರವಾಸವನ್ನು ಕೈಗೊಂಡಿರುವ ಐರೋಪ್ಯನು ಕಾಣುವುದು ಈ ದೃಶ್ಯವನ್ನು.

ಹೊರನೋಟಕ್ಕೆ ಮಾತ್ರ ಮನುಷ್ಯರಂತೆ ಕಾಣುತ್ತಿರುವ ಮೂವತ್ತು ಕೋಟಿ ಮಾನವ ಜೀವರಾಶಿಗಳು; ಅವರು ತಮ್ಮವರಿಂದಲೂ ವಿದೇಶೀಯರಿಂದಲೂ, ತಮ್ಮ ಧರ್ಮದವರಿಂದಲೂ ಪರಧರ್ಮದವರಿಂದಲೂ ತುಳಿಯಲ್ಪಟ್ಟು ಸತ್ತ್ವಹೀನರಾದವರು; ಅವರು ಗುಲಾಮರಂತೆ ಕಾರ್ಯೋತ್ಸಾಹವನ್ನು ಕಳೆದುಕೊಂಡು ಶ್ರಮಸಹಿಷ್ಣುಗಳಾಗಿ ಯಾತನೆಯನ್ನು ಅನುಭವಿಸುತ್ತಿರುವವರು; ಹತಾಶರಾದ ಅವರು ಯಾವುದೇ ಭೂತಭವಿಷ್ಯಗಳೂ ಇಲ್ಲದವರು; ಎಷ್ಟೇ ಕಷ್ಟವಾದರೂ ಇಂದಿನ ಬದುಕನ್ನು ಹೇಗೊ ನಡೆಸುತ್ತಿರುವವರು; ಗುಲಾಮ ಯೋಗ್ಯವಾದ ಮಾತ್ಸರ್ಯ ಸ್ವಭಾವದಿಂದ ಕೂಡಿದ ಅವರು ನೆರೆಹೊರೆಯವರ ಸುಖ ಸಂಪತ್ತುಗಳನ್ನು ಸಹಿಸಲಾರರು; ತಮ್ಮಲ್ಲಿನ ಎಲ್ಲ ಅಸೆಗಳೂ ಸತ್ತು ಹೋಗಿರುವ ಶ್ರದ್ಧಾಹೀನರು ಅವರು; ಅವರ ಶಸ್ತ್ರಾಸ್ತ್ರಗಳೆಂದರೆ ಕೀಳುರೀತಿಯ ಮೋಸ, ದ್ರೋಹ ಹಾಗೂ ನರಿಯ ಕುತಂತ್ರ, ಅವರು ಬಲಶಾಲಿಗಳ ಕಾಲಿನ ಧೂಳನ್ನು ನೆಕ್ಕುತ್ತಾರೆ, ಮತ್ತು ತಮಗಿಂತ ದುರ್ಬಲರಾದವರ ಮೇಲೆ ದೌರ್ಜನ್ಯ ನಡೆಸುತ್ತಾರೆ. ಸಹಜವಾಗಿಯೇ ದುರ್ಬಲರಿಗೂ ಭವಿಷ್ಯದ ವಿಷಯದಲ್ಲಿ ಹತಾಶ ರಾದವರಿಗೂ ಇರುವ ಅಸಹ್ಯ ರಾಕ್ಷಸೀಯ ಮೌಢ್ಯಗಳಿಂದ ಕೂಡಿದವರು ಅವರು. ಯಾವ ನೈತಿಕ ಹಿನ್ನೆಲೆಯೂ ಇಲ್ಲದ ಇಂಥ ಮೂವತ್ತು ಕೋಟಿ ಚೇತನಗಳು, ಕೊಳೆತು ನಾರುತ್ತಿರುವ ಹೆಣದ ಮೇಲೆ ಕ್ರಿಮಿ ಕೀಟಗಳು ಹೇಗೋ ಹಾಗೆ, ಭಾರತದ ಶರೀರವನ್ನು ಮುತ್ತಿಕೊಂಡಿವೆ. ಇಂಗ್ಲಿಷ್​ ಅಧಿಕಾರಿಯೊಬ್ಬನಿಗೆ ನಮ್ಮನ್ನು ಕುರಿತಂತೆ ಸಹಜವಾಗಿ ಮೂಡುವ ಚಿತ್ರ ಇದು.

ಹೊಸದಾಗಿ ಸಂಪಾದಿಸಿದ ಅಂಧಕಾರವೆಂಬ ಮದವು ತಲೆಗೇರಿ ಹುಚ್ಚ ರಾದವರು, ಒಳಿತು ಕೆಡುಕುಗಳ ನಡವೆ ವಿವೇಚನೆಯಿಲ್ಲದವರು, ಕಾಡಿನ ಕ್ರೂರ ಮೃಗಗಳಂತೆ ಭಯವನ್ನು ಹುಟ್ಟಿಸುವವರು, ಹೆಂಡತಿಯ ಗುಲಾಮರು, ಕಾಮಾ ಸಕ್ತರು, “ಮದ್ಯದಲ್ಲಿ ಮುಳುಗಿಹೋಗಿ ಬ್ರಹ್ಮಚರ್ಯ, ಪಾವಿತ್ರ್ಯ ಇವುಗಳ ಕಲ್ಪನೆಯೇ ಇಲ್ಲದೆ ಶುದ್ಧವಾದ ದಾರಿ ಅಭ್ಯಾಸಗಳ ಪರಿಚಯವೇ ಇಲ್ಲದವರು,” ಭೌತದ್ರವ್ಯದಲ್ಲಿ ಮಾತ್ರ ನಂಬಿಕೆಯಿಟ್ಟು ಅದನ್ನು ವಿವಿಧ ಪ್ರಯೋಜನಗಳಿಗೆ ಬಳಸುವ ನಾಗರೀಕತೆಯನ್ನು ಆಶ್ರಯಿಸಿದವರು, ಇತರ ದೇಶಗಳನ್ನೂ ಇತರರ ಸಂಪತ್ತನ್ನೂ ಬಲಾತ್ಕಾರ ಮೋಸ ದ್ರೋಹಗಳಿಂದ ಅಪಹರಿಸುವವರು, ಪರದಲ್ಲಿ ನಂಬಿಕೆಯಿಲ್ಲದವರು, ದೇಹಾತ್ಮ ಬುದ್ಧಿಯುಳ್ಳವರು, ಇಂದ್ರಿಯಗಳು ಮತ್ತು ಅವುಗಳನ್ನು ತಣಿಸುವ ಸಂಪತ್ತುಗಳು ಇವೆ ಸರ್ವಸ್ವವೆಂದು ಭಾವಿಸುವವರು-ಹೀಗೆ ಭಾರತೀಯರ ಕಣ್ಣಿನಲ್ಲಿ ಪಾಶ್ಚಾತ್ಯರು ಸಾಕ್ಷಾತ್​ ಅಸುರರು.

ಎರಡೂ ಪಕ್ಷದ ವೀಕ್ಷಕರ ದೃಷ್ಟಿಕೋನಗಳು ಇವು. ಇವು ಪರಸ್ಪರರ ವಿಷಯದಲ್ಲಿ ವಿಚಾರ ಹೀನವಾದ ಆಳವಲ್ಲದ ಜ್ಞಾನ ಅಥವಾ ಅಜ್ಞಾನದಿಂದ ಮೂಡಿದ ದೃಷ್ಟಿಗಳು. ಪಾಶ್ಚಾತ್ಯರು ಅದರಲ್ಲೂ ಯೂರೋಪಿನವರು ಭಾರತಕ್ಕೆ ಬರುತ್ತಾರೆ; ನಮ್ಮ ನಗರಗಳ ಆರೋಗ್ಯಕರ ಭಾಗಗಳಲ್ಲಿ ವಾಸಿಸುತ್ತಾರೆ; ನಮ್ಮ ಊರಿನ ಇತರ ಭಾಗಗಳನ್ನು ತಮ್ಮ ದೇಶದ ಸುಂದರವಾದ, ಚೆನ್ನಾಗಿ ಯೋಜಿತವಾದ ಭಾಗಗಳಿಗೆ ಹೋಲಿಸುತ್ತಾರೆ; ಅವರ ಸಂಪರ್ಕವಿರುವುದು ಯಾರು ತಮ್ಮ ಕೈಕೆಳಗೆ ಕೆಲಸ ಮಾಡುತ್ತಾರೊ ಅಂಥವರೊಂದಿಗೆ ಮಾತ್ರ. ನಿಜ, ದುಃಖ ದಾರಿದ್ರ್ಯಗಳು ಭಾರತ ದಲ್ಲಿರುವಷ್ಟು ಬೇರೆಲ್ಲಿಯೂ ಇಲ್ಲ, ಅಲ್ಲದೆ ಧೂಳು ಕೊಳೆಗಳು ಎಲ್ಲ ಕಡೆಯೂ ತುಂಬಿಕೊಂಡಿವೆ ಎಂಬುದೂ ನಿಜ. ಹೀಗಾಗಿ ಇಂಥ ಹೊಲಸು, ದಾಸ್ಯ ಅವನತಿಗಳ ನಡುವೆ ಏನಾದರೂ ಒಳ್ಳೆಯದು ಇರಬಹುದು ಎಂಬುದನ್ನು ಕಲ್ಪಿಸಿಕೊಳ್ಳುವುದೂ ಕೂಡ ಐರೋಪ್ಯ ಮನಸ್ಸಿಗೆ ಅಸಾಧ್ಯ.

ಇತ್ತ ನಾವಾದರೋ ಐರೋಪ್ಯರು ವಿವೇಚನೆಯೇ ಇಲ್ಲದೆ ಏನು ಸಿಕ್ಕಿದರೂ ತಿನ್ನುತ್ತಾರೆಂದು ಭಾವಿಸುತ್ತೇವೆ. ನಮಗಿರುವಂತೆ ಅವರಿಗೆ ಶುಚಿಯ ಕಲ್ಪನೆಯಿಲ್ಲ, ಜಾತೀಭೇದವನ್ನು ಲೆಕ್ಕಿಸುವುದಿಲ್ಲ, ಸ್ತ್ರೀಯರೊಡನೆ ಮುಕ್ತವಾಗಿ ವ್ಯವಹರಿಸುತ್ತಾರೆ, ಮದ್ಯಪಾನ ಮಾಡುತ್ತಾರೆ, ಸ್ತ್ರೀಪುರುಷರು ಪರಸ್ಪರ ತೆಕ್ಕೆಯಲ್ಲಿದ್ದು ಸಾರ್ವಜನಿಕ ವಾಗಿ ನರ್ತಿಸುತ್ತಾರೆ, ಇಂಥ ದೇಶದವರಲ್ಲಿ ಏನು ತಾನೆ ಒಳ್ಳೆಯದಿದ್ದೀತು ಎಂದು ಆಶ್ಚರ್ಯಚಕಿತರಾಗಿ ನಮಗೆ ನಾವೇ ಪ್ರಶ್ನೆ ಹಾಕಿಕೊಳ್ಳುತ್ತೇವೆ.

ಆದರೆ ಈ ಅಭಿಪ್ರಾಯಗಳಿಗೆಲ್ಲ ಮೂಲವು ಒಳನೋಟವಿಲ್ಲದೆ ಕೇವಲ ಹೊರಗಿ ನಿಂದ ಮೇಲೆ ಮೇಲೆ ಮಾತ್ರ ಸಂಗತಿಗಳನ್ನು ಗಮನಿಸುವುದಾಗಿದೆ. ವಿದೇಶಿಯರು ನಮ್ಮೊಂದಿಗೆ ಬೆರೆಯಲು ನೀವು ಬಿಡುತ್ತಿಲ್ಲ. ಅವರನ್ನು ಮ್ಲೇಚ್ಛರೆಂದು ಕರೆಯುತ್ತೇವೆ. ಅವರೂ ಅಷ್ಟೇ, ನಮ್ಮನ್ನು ಗುಲಾಮರೆಂದು ಭಾವಿಸಿ ನಿಗ್ಗರ್ಸ್​ ಎಂದು ಕರೆದು ದ್ವೇಷಿಸುತ್ತಾರೆ. ಎರಡು ಪಕ್ಷಗಳ ಅಭಿಪ್ರಾಯವೂ ಸ್ವಲ್ಪಮಟ್ಟಿಗೆ ಸತ್ಯವಿರ ಬಹುದು. ಆದರೂ ಎರಡು ಪಕ್ಷದವರೂ ಹಿಂದೆ ಇರುವ ಸತ್ಯವನ್ನು ಅರಿತಿಲ್ಲ.

ಪ್ರತಿಯೊಬ್ಬ ಮಾನವನಲ್ಲಿ ಒಂದು ಭಾವವಿದೆ. ಬಾಹ್ಯ ಮಾನವನು ಈ ಭಾವದ ಬಹಿಃಪ್ರಕಾಶ ಮಾತ್ರ, ಭಾಷೆ ಮಾತ್ರ. ಇದರಂತೆಯೇ ಪ್ರತಿಯೊಂದು ರಾಷ್ಟ್ರಕ್ಕೂ ಒಂದು ಪ್ರತ್ಯೇಕ ರಾಷ್ಟ್ರಭಾವವಿದೆ. ಅದು ಜಗತ್ತಿನ ಕಾರ್ಯವನ್ನು ಸಾಗಿಸುತ್ತದೆ. ಅದು ಜಗ್ಧತ್ತಿನ ಉಳಿವಿಗೆ ಅವಶ್ಯಕ. ಎಂದು ಈ ಅವಶ್ಯಕತೆ ಇಲ್ಲವೋ ಅಂದು ಅ ಭಾವಕ್ಕೆ ಕೇಂದ್ರವಾದ ವ್ಯಕ್ತಿ ಅಥವಾ ದೇಶ ನಾಶವಾಗುವುದು. ಒಳಗಿನಿಂದ ಮತ್ತು ಹೊರಗಿನಿಂದ ಇಷ್ಟೊಂದು ದುಃಖ ದಾರಿದ್ರ್ಯಗಳಿಗೆ ತುತ್ತಾದರೂ ಭಾರತೀಯರು ಇನ್ನೂ ಬದುಕಿರುವರು. ಇದಕ್ಕೆ ಕಾರಣ ಜಗತ್ತಿನ ರಕ್ಷಣೆಗೆ ಆವಶ್ಯಕವಾದ ರಾಷ್ಟ್ರೀಯ ಭಾವ ನಮ್ಮಲ್ಲಿರುವುದು. ಯೂರೋಪ್​ ದೇಶೀಯ ರಲ್ಲಿಯೂ ಒಂದು ರಾಷ್ಟ್ರೀಯ ಭಾವವಿದೆ. ಅದಿಲ್ಲದೆ ಇದ್ದರೆ ಜಗತ್ತು ಸಾಗುವುದಿಲ್ಲ. ಆದ ಕಾರಣವೇ ಅವರು ಅಷ್ಟು ಪ್ರಬಲರಾಗಿರುವುದು. ತನ್ನ ಶಕ್ತಿಯನ್ನೆಲ್ಲಾ ಕಳೆದುಕೊಂಡರೆ ವ್ಯಕ್ತಿಯು ಒಂದು ಕ್ಷಣವಾದರೂ ಬದುಕಿರುವನೇನು? ಒಂದು ರಾಷ್ಟ್ರವು ಅಲ್ಲಿಯ ವ್ಯಕ್ಕಿಗಳ ಸಮಷ್ಟಿ ಮಾತ್ರ. ಅದು ನಿಶ್ಯಕ್ತಗೊಂಡು ಪೂರ್ಣವಾಗಿ ನಿಷ್ಕರ್ಮವಾದರೆ ಹೇಗೆ ಬದುಕಬಲ್ಲದು? ಸಾವಿರಾರು ವರ್ಷಗಳಿಂದ ಹಲವು ಬಗೆಯ ಕೋಟಲೆಗೆ ತುತ್ತಾದರೂ ಹಿಂದೂ ಜನಾಂಗ ಏಕೆ ನಾಶವಾಗಲಿಲ್ಲ? ನಮ್ಮ ರೀತಿನೀತಿಗಳು ಅಷ್ಟು ಹೀನವಾಗಿದ್ದರೆ ಪ್ರಪಂಚದಿಂದ ಏತಕ್ಕೆ ಇನ್ನೂ ನಾವು ಕಣ್ಮರೆಯಾಗದೆ ಇರುವೆವು? ವಿದೇಶೀ ಆಕ್ರಮಣಕಾರರು ನಮ್ಮನ್ನು ನಾಶ ಮಾಡಲು ಯಾವ ಪ್ರಯತ್ನವನ್ನು ತಾನೆ ಮಾಡದೆ ಬಿಟ್ಟಿದಾರೆ? ಇತರ ನಾಗರೀಕ ದೇಶಗಳಲ್ಲಿ ಆದಂತೆ ಹಿಂದೂ ಜನಾಂಗ ಏಕೆ ನಿರ್ನಾಮವಾಗಲಿಲ್ಲ? ಭರತ ಖಂಡವು ಜನ ಶೂನ್ಯವಾದ ಮರುಭೂಮಿಯಾಗಿ ಏಕೆ ಆಗಲಿಲ್ಲ? ವಿದೇಶೀ ಯರು ಅಮೇರಿಕಾ, ಆಸ್ಟ್ರೇಲಿಯಾ, ಆಫ್ರಿಕ ದೇಶಗಳಲ್ಲಿ ಮಾಡಿರುವಂತೆ ಇಲ್ಲಿಗೂ ಬಂದು ಫಲವತ್ತಾದ ಭೂಮಿಯಲ್ಲೆಲ್ಲ ಆಕ್ರಮಿಸಿಕೊಂಡು ಇಲ್ಲಿ ಏ್ಧಕೆ ಕೃಷಿ ಮಾಡಲಿಲ್ಲ?

ವಿದೇಶೀಯರೆ! ನೀವು ತಿಳಿದಿಕೊಂಡಿರುವಷ್ಟು ಬಲಾಢ್ಯರು ನೀವಲ್ಲ. ಅದು ಬರಿಯ ಬಯಲು ಭ್ರಾಂತಿ. ಭಾರತದಲ್ಲಿಯೂ ಒಂದು ಶಕ್ತಿ ಇದೆ, ಒಂದು ಸತ್ತ್ವವಿದೆ. ಮೊದಲು ಇದನ್ನು ಗಮನಿಸಿ. ಜೊತೆಗೆ ಇದನ್ನು ಜ್ಞಾಪಕದಲ್ಲಿಡಿ! ಜಗತ್ತಿನ ನಾಗರೀಕತೆಯ ಭಂಡಾರಕ್ಕೆ ನಮ್ಮ ಕಾಣಿಕೆಯ ಅವಶ್ಯಕತೆಯಿದೆ. ಆದಕಾರಣವೇ ಭಾರತ ಇನ್ನೂ ಜೀವಂತವಾಗಿರುವುದು. ಆಚಾರದಲ್ಲಿ, ಭಾವದಲ್ಲಿ, ಆಲೋಚನೆ ಯಲ್ಲಿ ಪಾಶ್ಚಾತ್ಯರನ್ನು ಅನುಕರಿಸುತ್ತ “ಯುರೋಪಿಯನ್ನರೆ, ನಾವು ನರಪಶು ಗಳಾಗಿದ್ದೇವೆ; ನೀವು ನಮ್ಮನ್ನು ಉದ್ಧಾರಮಾಡಬೇಕು” ಎಂದು ಗೋಗರೆಯುವ ನಮ್ಮ ದೇಶದ ಜನರೆ, ನೀವು ಕೂಡ ಇದನ್ನು ಗಮನಿಸಿ: ಏಸುವು ಭಾರತಕ್ಕೆ ಬಂದಿದ್ದಾನೆ, ಕಾಲವು ಪಕ್ವವಾದಾಗ ದೇವರ ಆಜ್ಞೆಯು ಸಾರ್ಥಕ್ಯವನ್ನು ಪಡೆಯುತ್ತದೆ ಎಂದು ಕೂಗಾಡುತ್ತೀರಲ್ಲ, ನೀವು ಕೂಡ ಗಮನಿಸಿ: ಏಸುವೂ ಬರಲಿಲ್ಲ, ಯಹೋವನೂ ಬರಲಿಲ್ಲ, ಮುಂದೆ ಬರುವುದೂ ಇಲ್ಲ. ಅವರು ತಮ್ಮ ಮನೆಮಠಗಳನ್ನು ದುರಸ್ತುಗೊಳಿಸುವುದರಲ್ಲಿ ನಿರತರಾಗುವರು. ಇಲ್ಲಿಗೆ ಬರುವುದಕ್ಕೆ ಸಮಯವಿಲ್ಲ ಅವರಿಗೆ. ಇಲ್ಲಿ ಹಿಂದಿನಂತೆಯೇ ವೃದ್ಧ ಶಿವ ಕುಳಿತಿರು ವನು, ರಕ್ತಸಿಕ್ತಳಾದ ಕಾಳಿಯು ತನ್ನ ಬಳಗದೊಂದಿಗೆ ಹಾಗೆಯೇ ಇದ್ದಾಳೆ. ವ್ರಜ ಗೋಪಾಲನು ಎಂದಿನಂತೆಯೇ ಕೊಳಲನ್ನು ಊದುತ್ತಿರುವನು. ಈ ವೃದ್ಧ ಶಿವ ಒಮ್ಮೆ ವೃಷಭಾರೂಢನಾಗಿ, ಡಮರು ಬಾರಿಸುತ್ತ ಒಂದು ಕಡೆ ಸುಮಾತ್ರ, ಬೋರ್ನಿಯೊ, ಸಿಲಿಬೆಸ್​, ಆಸ್ಟ್ರೇಲಿಯ, ಅಮೇರಿಕಾ ದೇಶದವರೆಗೂ ಹೋಗಿರು ವನು, ಮತ್ತೊಂದು ಕಡೆ ಟಿಬೆಟ್​, ಚೈನಾ, ಜಪಾನ್​ ಮತ್ತು ಸೈಬೀರಿಯಾವರೆಗೂ ಹೋಗಿರುವನು, ಈಗಲೂ ಹೋಗುತ್ತಿರುವನು. ಕಾಳಿಕಾ ಮಾತೆ ಚೈನಾ ಜಪಾನ್​ ದೇಶಗಳಲ್ಲಿ ಕೂಡ ಇನ್ನೂ ಪೂಜಿಸಿಕೊಳ್ಳುತ್ತಿರುವಳು. ಇವಳನ್ನೇ ಕ್ರೈಸ್ತರು ಏಸುವಿನ ತಾಯಿ ಮೇರಿ ಎಂದು ಪರಿವರ್ತಿಸಿ ಪೂಜಿಸುತ್ತಿರುವರು. ಹಿಮಾಲಯವನ್ನು ನೋಡಿ, ಅದಕ್ಕೆ ಉತ್ತರದಲ್ಲಿ ಕೈಲಾಸವಿದೆ. ಅದೇ ವೃದ್ಧಶಿವನ ನೆಲೆ. ಅದನ್ನು ದಶಶಿರ ರಾವಣ ತನ್ನ ಇಪ್ಪತ್ತು ಭುಜಗಳಿಂದ ಕೂಡ ಅಲುಗಿಸಲಾಗಲಿಲ್ಲ. ಇನ್ನು ಈ ಬಡ ಪಾದ್ರಿಗಳು ಏನು ಮಾಡಬಲ್ಲರು! ಭಾರತದಲ್ಲಿ ಚಿರಕಾಲ ವೃದ್ಧ ಶಿವ ತನ್ನ ಡಮರು ಬಾರಿಸುತ್ತಲೇ ಇರುವನು, ಮಹಾಕಾಳಿಯು ಬಲಿ ಭಕ್ಷಿಸುತ್ತಲೇ ಇರುವಳು. ಪ್ರೇಮಮಯನಾದ ಶ‍್ರೀಕೃಷ್ಣ ಕೊಳಲನ್ನು ಬಾರಿಸುತ್ತಲೇ ಇರುವನು. ಹಿಮಾಲಯದಂತೆ ಅವರು ಭದ್ರವಾಗಿರುವರು. ಪಾದ್ರಿ ಗೀದ್ರಿಗಳಾರೂ ಅವರನ್ನು ಅಲುಗಿಸಲಾರರು. ನಿಮಗೆ ಅವರನ್ನು ಸಹಿಸಲು ಆಗದೆ ಇದ್ದರೆ ದೂರ ಸರಿಯಿರಿ. ನಿಮ್ಮ ನಾಲ್ಕೈದು ಜನರಿಗಾಗಿ ಇಡೀ ದೇಶ ತನ್ನ ತಾಳ್ಮೆಯನ್ನು ಕಳೆದುಕೊಂಡು ನಾಶ ವಾಗಬೇಕೆ? ನೀವೇ ನಿಮಗೆ ಯೋಗ್ಯವಾದ ಸ್ಥಳಕ್ಕೆ ಏತಕ್ಕೆ ಹೋಗಬಾರದು? ವಿಶಾಲ ಜಗತ್ತು ನಿಮ್ಮೆದುರಿಗಿದೆ? ಇಲ್ಲ, ಅವ್ಧರು ಹಾಗೆ ಮಾಡ್ಧುವ್ಧುದ್ಧಿಲ್ಲ. ಹಾಗೆ ಮಾಡ್ಧ್ಧುವ್ಧುದ್ಧಕ್ಕೆ ಧೈರ್ಯ್ಧವ್ಧೆಲ್ಲ್ಧಿದೆ? ನಮ್ಮ ವೃದ್ಧ ಶಿವನ ಉಪ್ಪು ತಿಂದು ಅವನನ್ನು ದೂರುವರು. ವಿದೇಶೀ ದೇವತೆಗಳನ್ನು ಹೊಗಳುವರು. ಅಬ್ಬ! ವಿದೇಶೀ ಯರೆದುರಿಗೆ ನಿರ್ಲಜ್ಜರಾಗಿ ನಾವು ನೀಚರು, ಕ್ಷುದ್ರರು, ಅವನತಿಹೊಂದಿ ದ್ದೇವೆ, ನಮ್ಮಲ್ಲಿರುವುದೆಲ್ಲ ಕೆಟ್ಟುಹೋಗಿದೆ ಎಂದು ಅಳುವವರಿಗೆ ನಾನು ಹೀಗೆ ಹೇಳುತ್ತೇನೆ: “ಅದೆಲ್ಲ ನಿಜವಿರಬಹುದು. ನೀವು ಸತ್ಯವಾದಿಗಳಾದುದರಿಂದ ನಿಮ್ಮನ್ನು ನಂಬದೆ ಇರುವುದಕ್ಕೆ ಕಾರಣವಿಲ್ಲ. ಆದರೆ ‘ನಾವು’ ಎಂಬ ಪದದಲ್ಲಿ ಇಡೀ ದೇಶವನ್ನೆಲ್ಲ ಏತಕ್ಕೆ ಸೇರಿಸುವಿರಿ? ಇದು ಎಂತಹ ಸಭ್ಯತೆ?”

ಯಾವುದೋ ಒಂದು ಜನಾಂಗಕ್ಕೆ ಎಲ್ಲಾ ಒಳ್ಳೆಯ ಗುಣಗಳು ಮೀಸಲಾಗಿವೆ ಎಂಬುದು ಭ್ರಾಂತಿ. ಇದನ್ನು ನಾವು ಮೊದಲು ಗಮನಿಸಬೇಕು. ವ್ಯಕ್ತಿಗಳಲ್ಲಿರು ವಂತೆ ಜನಾಂಗಗಳಲ್ಲಿಯೂ ಕೆಲವು ಒಳ್ಳೆಯ ಗುಣಗಳು ಇತರರಿಗಿಂತ ಪ್ರಧಾನ ವಾಗಿರಬಹುದು.

ನಮ್ಮ ಜನಾಂಗದಲ್ಲಿ ಮೋಕ್ಷಪ್ರಾಪ್ತಿಯ ಇಚ್ಚೆ ಪ್ರಧಾನವಾಗಿದೆ. ಪಾಶ್ಚಾತ್ಯರಲ್ಲಿ ಧರ್ಮವು ಪ್ರಧಾನವಾಗಿದೆ. ನಾವು ಇಚ್ಚಿಸುವುದು ಮುಕ್ತಿ; ಅವರು ಇಚ್ಚಿಸುವುದು ಧರ್ಮ. ಇಲ್ಲಿ ಧರ್ಮ ಶಬ್ದದ ವ್ಯವಹಾರ ಮೀಮಾಂಸಕರ ಅರ್ಥದಲ್ಲಿದೆ. ಧರ್ಮವೆಂದರೇನು? ಇಹಲೋಕದಲ್ಲಿ ಮತ್ತು ಪರಲೋಕದಲ್ಲಿ ಸುಖ ಭೋಗ ಗಳನ್ನು ಪಡೆಯುವಂತೆ ಪ್ರೇರೇಪಿಸುವುದೇ ಧರ್ಮ. ಧರ್ಮವು ಕರ್ಮವನ್ನು ಅವಲಂಬಿಸಿದೆ. ವ್ಯಕ್ತಿಯು ಹಗಲು ರಾತ್ರಿ ಸುಖವನ್ನು ಅರಸುವನು. ಅದಕ್ಕಾಗಿ ಕರ್ಮ ಮಾಡುವನು.

ಮುಕ್ತಿ ಎಂದರೇನು? ಈ ಜೀವನದ ಭೋಗವು ಗುಲಾಮಗಿರಿ. ಬರುವ ಜನ್ಮದ ಭೋಗವು ಅದರಂತೆಯೇ. ಏಕೆಂದರೆ ಈ ಜಗತ್ತಾಗಲಿ, ಮುಂದಿನ ಜಗತ್ತಾಗಲಿ ಪ್ರಕೃತಿಯ ನಿಯಮಗಳನ್ನು ಮೀರಿಲ್ಲ. ಇಹಲೋಕ ದಾಸ್ಯಕ್ಕೂ ಪರಲೋಕದ ದಾಸ್ಯಕ್ಕೂ ಇರುವ ವ್ಯತ್ಯಾಸ ಕಬ್ಬಿಣದ ಮತ್ತು ಚಿನ್ನದ ಸರಪಳಿಗೆ ಇರುವ ವ್ಯತ್ಯಾಸದಂತೆ. ಹೀಗೆ ಯಾವುದು ಬೋಧಿಸುತ್ತದೆಯೋ ಅದೇ ಮುಕ್ತಿ. ಭೋಗ ಎಲ್ಲಿಯಾದರೂ ಆಗಲಿ, ಪ್ರಕೃತಿಯ ನಿಯಮಾವಳಿಯೊಳಗೆ ಇರುವ ಪರಿಯಂತರವೂ ಅದು ಮೃತ್ಯುವಿಗೆ ಒಳಗಾಗುವಂಥದು, ಶಾಶ್ವತವಾಗಿರ ಲಾರದು. ಅದಕ್ಕೇ ಮನುಷ್ಯ ಮುಕ್ತನಾಗಬೇಕು; ದೇಹದ ಬಂಧನದಿಂದ ಪಾರಾಗ ಬೇಕು. ದಾಸ್ಯದಿಂದ ಪ್ರಯೋಜನವಿಲ್ಲ. ಈ ಮೋಕ್ಷಪಥ ಭರತವರ್ಷದಲ್ಲಿ ಮಾತ್ರ ಇದೆ. ಬೇರೆ ಕಡೆ ಇಲ್ಲ. ಇದರಂತೆಯೇ ನೀವು ಹಲವು ವೇಳೆ ಕೇಳಿರುವಂತೆ ಮುಕ್ತ ಪುರುಷರು ಭರತ ವರ್ಷದಲ್ಲಿ ಮಾತ್ರ ಇರುವರು, ಬೇರೆಡೆ ಇಲ್ಲ ಎಂಬ ನುಡಿ ಸತ್ಯ. ಮುಂದೆ ಅವ್ಧರು ಬೇರೆ ದೇಶ್ಧ್ಧಗ್ಧಳ್ಧಲ್ಲೂ ಇರ್ಧುತ್ತ್ಧಾರೆ ಎಂಬ್ಧ್ಧುದೂ ಅಷ್ಟೇ ಸತ್ಯ; ಇದು ಒಳ್ಳೆಯದು, ಆನಂದದ ವಿಷಯ.

ಭಾರತದಲ್ಲಿಯೂ ಒಂದು ಕಾಲದಲ್ಲಿ ಯಾವುದು ಧರ್ಮವೋ ಅದು ಮೋಕ್ಷಕ್ಕೆ ಸಂವಾದಿಯಾಗಿತ್ತು. ಆ ಕಾಲದಲ್ಲಿ ಮೋಕ್ಷಾಕಾಂಕ್ಷಿಗಳಾದ ಶುಕ, ವ್ಯಾಸ, ಸನಕಾದಿ ಮುನಿಗಳೂ ಇದ್ದರು. ಅದೇ ಕಾಲದಲ್ಲಿ ಧರ್ಮೋಪಾಸಕರಾದ ಯುಧಿಷ್ಠಿರ, ಅರ್ಜುನ, ದುರ್ಯೋಧನ, ಭೀಷ್ಮ, ಕರ್ಣ ಮುಂತಾದವರೂ ಇದ್ದರು. ಬೌದ್ಧ ಧರ್ಮವು ಪ್ರಚಾರವಾದ ಮೇಲೆ ಧರ್ಮವು ಸಂಪೂರ್ಣವಾಗಿ ನಿರ್ಲಕ್ಷಿಸಲ್ಪಟ್ಟಿತು, ಮೋಕ್ಷ ಮಾರ್ಗವೊಂದೇ ಪ್ರಮುಖವಾಯಿತು. ಅದಕ್ಕೇ ಅಗ್ನಿಪುರಾಣವು ರೂಪಕ ರೀತಿಯಲ್ಲಿ ವಿವರಿಸುವಾಗ, ‘ಗಯಾಸುರ ಅಂದರೆ ಬುದ್ಧನು ಮೋಕ್ಷ ಮಾರ್ಗವನ್ನು ಎಲ್ಲರಿಗೂ ತೋರಿ ಜಗತ್ತನ್ನು ನಾಶಮಾಡಲು ಪ್ರಯತ್ನಿಸಿದನು; ಆಗ ದೇವತೆಗಳು ಸಭೆ ಸೇರಿ ಉಪಾಯ ಹೂಡಿ ಅವನನ್ನು ಪರಲೋಕಕ್ಕೆ ಕಳುಹಿಸಿ ದರು’ ಎಂದಿದೆ. ಇದರ ನಿಜವಾದ ಅಭಿಪ್ರಾಯ ಇದು, ನಾವು ಈಚೆಗೆ ಕೇಳುತ್ತಿರುವ ದೇಶದ ದುರ್ಗತಿಗೆ ಕಾರಣ ಈ ಧರ್ಮದ ಅಭಾವ. ಇಡೀ ದೇಶವು ಮೋಕ್ಷ ಧರ್ಮ ಪಾರಾಯಣತೆಯಲ್ಲಿ ಮಗ್ನವಾದರೆ ಅದೇನೊ ಒಳ್ಳೆಯದೆ. ಆದರೆ ಅದು ಸಾಧ್ಯವೇ? ಭೋಗವಿಲ್ಲದೆ ತ್ಯಾಗ ಬರಲಾರದು. ಮೊದಲು ಭೋಗಿಸಿ, ನಂತರ ನೀವು ತ್ಯಜಿಸಬಲ್ಲಿರಿ. ಅದಲ್ಲದೆ ಇಡೀ ದೇಶ ನಿವೃತ್ತಿ ಮಾರ್ಗವನ್ನು ಅನುಸರಿಸಿದರೆ ತನಗೆ ಬೇಕಾದುದನ್ನು ಪಡೆಯಲಾರದು, ತನ್ನಲ್ಲಿರುವುದನ್ನೂ ಅದು ಕಳೆದುಕೊಳ್ಳುವುದು. ಕೈಗೆ ಸಿಕ್ಕಿದ ಹಕ್ಕಿಯೂ ಹಾರಿತು, ಪೊದೆಯಲ್ಲಿರುವ ಹಕ್ಕಿಯೂ ದಕ್ಕಲಿಲ್ಲ. ಯಾವ ಸಮಯದಲ್ಲಿ ಬೌದ್ಧರ ಒಂದೊಂದು ಮಠದಲ್ಲಿ ಸಹಸ್ರಾರು ಜನ ಭಿಕ್ಷುಗಳಿದ್ದರೋ ಆಗ ನಮ್ಮ ದೇಶದ ಅವನತಿ ಪ್ರಾರಂಭವಾಯಿತು. ಬೌದ್ಧರು, ಜೈನರು, ಕ್ರೈಸ್ತರು, ಮುಸಲ್ಮಾನರು ಇವರುಗಳು ಎಲ್ಲರಿಗೂ ಒಂದೇ ಕಾನೂನನ್ನು ಭ್ರಾಂತಿಯಿಂದ ಅನ್ವಯಿಸುವರು. ಇದೊಂದು ದೊಡ್ಡ ತಪ್ಪು. ವ್ಯಕ್ತಿಯ ಸ್ವಭಾವಕ್ಕೆ ತಕ್ಕಂತೆ ಶಿಕ್ಷಣ, ವ್ಯವಸಾಯ, ನಿಯಮ, ಬದಲಾಗಬೇಕು. ಬಲಾತ್ಕಾರದಿಂದ ಅವರೆಲ್ಲರನ್ನೂ ಒಂದೇ ಸಮನಾಗಿ ಮಾಡುವುದಕ್ಕೆ ಪ್ರಯತ್ನಿಸುವುದರಿಂದ ಪ್ರಯೋಜನವೇನು? ಬೌದ್ಧರು “ಮೋಕ್ಷಕ್ಕೆ ಸದೃಶವಾದುದು ಮತ್ತಾವುದಿರುವುದು? ಎಲ್ಲರೂ ಮೋಕ್ಷಸಾಧನೆಗಾಗಿ ಪ್ರಯತ್ನಿಸಿ” ಎನ್ನುವರು. ಇದು ಸಾಧ್ಯವೆ ಎಂದು ನಾನು ಪ್ರಶ್ನಿಸುತ್ತೇನೆ. ಹಿಂದೂ ಗ್ರಂಥಗಳು ಸಾರುವುದು ಇದು: “ನೀವು ಗೃಹಸ್ಥರು. ನಿಮಗೆ ಇದರ ಅವಶ್ಯಕತೆ ಅಷ್ಟಾಗಿಲ್ಲ. ನೀವು ನಿಮ್ಮ ಸ್ವಧರ್ಮವನ್ನು ಅನುಸರಿಸಿರಿ.” ಒಂದು ಮೊಳ ಹಾರಲಾರದವನು ಇಡೀ ಸಾಗರದ ಮೇಲೆ ಲಂಕಾನಗರಕ್ಕೆ ಒಂದೇ ನೆಗೆತದಲ್ಲಿ ಹಾರುವನಂತೆ! ಇದು ಸಾಧ್ಯವೇ? ನಿಮ್ಮ ಕುಟುಂಬವನ್ನೇ ಪೋಷಿಸಲಾರಿರಿ, ಇಬ್ಬರಿಗೆ ನೀವು ಹೊಟ್ಟೆ ತುಂಬ ಊಟ ಹಾಕಲಾರಿರಿ, ಇಬ್ಬರು ಸೇರಿ ಜನರಿಗೆ ಹಿತಕರವಾದ ಒಂದು ಕೆಲಸವನ್ನು ಮಾಡಲಾರಿರಿ; ಆದರೂ ಮೋಕ್ಷಕಾತರರಾಗಿ ಓಡುತ್ತಿರುವಿರಿ! ಹಿಂದೂಶಾಸ್ತ್ರ ಹೀಗೆ ಹೇಳುತ್ತಿರುವುದು: “ನಿಸ್ಸಂಶಯವಾಗಿ ಮೋಕ್ಷವು ಧರ್ಮ ಕ್ಕಿಂತ ಹೆಚ್ಚು. ಆದರೆ ಮೊದಲು ಧರ್ಮವನ್ನು ಆಚರಿಸಿ ಮುಗಿಸಬೇಕು.” ಬೌದ್ಧರಿಂದ ಇಲ್ಲೇ ಭ್ರಮೆ ಬಂದದ್ದು. ಇದರಿಂದಲೇ ಅನೇಕ ಉತ್ಪಾತಗಳು ಪ್ರಾರಂಭವಾದವು. ಅಹಿಂಸೆಯು ಪರಮ ಧರ್ಮ, ದೊಡ್ಡ ನೀತಿಯೇನೋ ಸರಿ. ಆದರೆ ಶಾಸ್ತ್ರ ಹೇಳುವುದು ಇದು: “ನೀವು ಗೃಹಸ್ಥರು; ನಿಮ್ಮ ಕೆನ್ನೆಗೆ ಯಾವನಾದರೂ ಒಂದು ಏಟು ಕೊಟ್ಟರೆ ಅದಕ್ಕೆ ಬದಲು ನೀವು ಅವನಿಗೆ ಹತ್ತು ಪೆಟ್ಟು ಕೊಡದೆ ಇದ್ದರೆ ಪಾಪ ಮಾಡುತ್ತೀರಿ. ನಿಮ್ಮನ್ನು ಕೊಲ್ಲುವುದಕ್ಕೆ ಒಬ್ಬ ಬಂದರೆ ಅವನನ್ನು ಕೊಲ್ಲುವುದರಲ್ಲಿ ತಪ್ಪಿಲ್ಲ, ಆತ ಬ್ರಾಹ್ಮಣನಾದರೂ ಚಿಂತೆಯಿಲ್ಲ” -ಮನು (VIII ೩೨೦).ಇದು ಸತ್ಯ, ಇದನ್ನು ಎಂದಿಗೂ ಮರೆಯ ಕೂಡದು. ವೀರರೇ ಪ್ರಪಂಚವನ್ನು ಅನುಭವಿಸುವರು. ಸಾಹಸವನ್ನು ಮೆರೆಯಿರಿ; ಸಾಮ, ದಾನ, ಭೇದ, ದಂಡ ನೀತಿಗಳನ್ನು ಅನುಸರಿಸಿ. ವೈರಿಯನ್ನು ಎದುರಿಸಿ. ಜಗತ್ತನ್ನು ಭೋಗಿಸಿ. ಆಗ ನೀವು ಧಾರ್ಮಿಕರು, ಇನ್ನೊಬ್ಬರಿಂದ ಒದೆಸಿಕೊಂಡು, ತುಳಿಸಿಕೊಂಡು, ಅಪಮಾನವನ್ನು ಸಹಿಸುತ್ತಿದ್ದರೆ, ಇಹಲೋಕದಲ್ಲೇ ನರಕ ಭೋಗ, ಪರಲೋಕವೂ ಅಷ್ಟೆ. ಇದು ಶಾಸ್ತ್ರಮತ. ನಿಮ್ಮ ಸ್ವಧರ್ಮವನ್ನು ಆಚರಿಸಿ. ಇದೇ ಸತ್ಯಸ್ಯ ಸತ್ಯ. ಅನ್ಯಾಯ ಮಾಡಬೇಡಿ, ಅತ್ಯಾಚಾರ ಮಾಡಬೇಡಿ. ಸಾಧ್ಯವಾದಷ್ಟು ಪರೋಪಕಾರ ಮಾಡಿ. ಗೃಹಸ್ಥರು ಮತ್ತೊಬ್ಬರ ಅನ್ಯಾಯವನ್ನು ಸಹಿಸುವುದು ಪಾಪ. ಈ ಸಮಯದಲ್ಲಿ ಅವರಿಗೆ ತಕ್ಕ ಶಿಕ್ಷೆಯನ್ನು ಕೊಡಬೇಕು. ಗೃಹಸ್ಥನು ಉತ್ಸಾಹದಿಂದ, ಶೌರ್ಯದಿಂದ ದ್ರವ್ಯವನ್ನು ಅರ್ಜಿಸಬೇಕು.ಅದರ ಸಹಾಯದಿಂದ ಬಂಧು ಬಳಗ ಪರಿಚಾರಕರನ್ನು ರಕ್ಷಿಸಿ ಅನ್ಯರಿಗೆ ಹಿತವನ್ನು ಮಾಡಬೇಕು. ನೀವು ಇದನ್ನು ಮಾಡದೆ ಹೋದರೆ ಹೇಗೆ ಮಾನವರು ಎಂದು ಹೇಳಿಕೊಳ್ಳಬಲ್ಲಿರಿ? ನೀವು ಇನ್ನೂ ಗೃಹಸ್ಥರೂ ಕೂಡ ಆಗಿಲ್ಲ, ಆಗಲೇ ಮೋಕ್ಷದ ಮಾತು!

ಧರ್ಮವು ಕರ್ಮವನ್ನು ಅವಲಂಬಿಸಿದೆ ಎಂಬುದನ್ನು ಹಿಂದೆಯೇ ಹೇಳಿದೆವು. ಧಾರ್ಮಿಕನ ಕರ್ತವ್ಯ ಸದಾ ಕರ್ಮಶೀಲನಾಗಿರುವುದು. ಕೆಲವು ಮೀಮಾಂಸಕರ ಅಭಿಪ್ರಾಯ ಕೂಡ, ಯಾವುದು ಕರ್ಮವನ್ನು ಹೇಳುವುದಿಲ್ಲವೊ ಅದು ವೇದಾಂಗವಲ್ಲ ಎಂದು. ಜೈಮಿನಿಯ ಒಂದು ಸೂತ್ರ \textbf{ಆಮ್ನಾಯಸ್ಯ ಕ್ರಿಯಾರ್ಥತ್ವಾ ದಾನರ್ಥಕ್ಯಮತದರ್ಥಾನಾಮ್​-} “ವೇದಗಳ ಗುರಿ ಕರ್ಮ, ಯಾವುದು ಕರ್ಮವನ್ನು ಹೇಳುವುದಿಲ್ಲವೊ ಅದು ತನ್ನ ಗುರಿಯನ್ನು ಕಳೆದುಕೊಳ್ಳುವುದು”, ಎನ್ನುತ್ತದೆ.

“ಓಂಕಾರಧ್ಯಾನದಿಂದ ಸಕಲವೂ ಸಿದ್ಧಿಸುವುದು. ಹರಿನಾಮ ಭಜನೆಯಿಂದ ಸಕಲಪಾಪಗಳು ಪರಿಹಾರವಾಗುವುವು. ಭಗವಂತನಲ್ಲಿ ಶರಣಾಗತನಾದರೆ ಸರ್ವ ಪ್ರಾಪ್ತಿ” ಇಂತಹ ಶಾಸ್ತ್ರವಾಕ್ಯಗಳು, ಸಾಧುವಾಕ್ಯಗಳು ಸತ್ಯ. ಲಕ್ಷಾಂತರ ಜನ ಓಂಕಾರ ಧ್ಯಾನ ಮಾಡುತ್ತಿರುವರು, ಹರಿನಾಮೋಚ್ಚಾರಣೆಯಿಂದ ಉನ್ಮತ್ತರಾಗು ತ್ತಿರುವರು, ಪ್ರಭು, ನಿನ್ನ ಇಚ್ಚೆಯಂತಾಗಲಿ, ನಾನು ನಿನಗೆ ಸಂಪೂರ್ಣವಾಗಿ ಶರಣಾಗತನಾಗಿರುವೆನು, ಎನ್ನುತ್ತಿರುವರು. ಆದರೆ ಇದರಿಂದ ಏನು ದೊರಕುತ್ತಿದೆ? ಇದರ ಅರ್ಥವನ್ನು ನಾವು ಚೆನ್ನಾಗಿ ತಿಳಿದುಕೊಳ್ಳಬೇಕು. ಯಾರಲ್ಲಿ ಜಪ ಯಥಾರ್ಥ ವಾಗಿದೆ? ಯಾರಲ್ಲಿ ಹರಿನಾಮ ಸ್ಮರಣೆ ವಜ್ರದಂತೆ ಅಮೋಘವಾಗಿದೆ? ಯಾರು ನಿಜವಾಗಿ ಹರಿ ಶರಣರು? ಕರ್ಮದಿಂದ ಯಾರ ಚಿತ್ತ ಶುದ್ಧಿಯಾಗಿದೆಯೊ ಅವರು; ಅಂದರೆ, ಯಾರು ಧಾರ್ಮಿಕರೊ ಅವರು.

ಪ್ರತಿಯೊಂದು ವ್ಯಕ್ತಿಯೂ ಒಂದು ಬಗೆಯ ಶಕ್ತಿಪ್ರಕಾಶಕ್ಕೆ ಕೇಂದ್ರ ವಾಗಿದ್ದಾನೆ. ಪೂರ್ವಕರ್ಮ ಫಲದಿಂದ ಈ ಶಕ್ತಿ ಸಂಚಿತವಾಗಿದೆ, ಇದರಿಂದ ನಾವು ಜನ್ಮ ತಾಳುತ್ತೇವೆ. ಎಲ್ಲಿಯವರೆಗೂ ಈ ಶಕ್ತಿ ಕಾರ್ಯರೂಪದಲ್ಲಿ ಪ್ರಕಾಶಿತವಾಗಿಲ್ಲವೊ, ಅಲ್ಲಿಯವರೆಗೂ ಯಾರು ಸುಮ್ಮನೆ ಇರಬಲ್ಲರು? ಭೋಗವನ್ನು ನಾಶಮಾಡಬಲ್ಲರು? ಅಲ್ಲಿಯವರೆಗೂ ವ್ಯಕ್ತಿಯು ತನ್ನ ದುಷ್ಕರ್ಮ ಸುಕರ್ಮಗಳಿಗೆ ತಕ್ಕಂತೆ ಸುಖಿಸುತ್ತಲೋ ದುಃಖಿಸುತ್ತಲೋ ಇರಬೇಕು. ಭೋಗ ಮತ್ತು ಕರ್ಮವನ್ನು ತೊರೆಯಲು ಆಗದೆ ಇರುವಾಗ ಸುಕರ್ಮವನ್ನು ಮಾಡುವುದು ದುಷ್ಕರ್ಮ ಮಾಡುವುದಕ್ಕಿಂತ ಮೇಲಲ್ಲವೇ? ಶ‍್ರೀ ರಾಮಪ್ರಸಾದ ಹೀಗೆ ಹೇಳುತ್ತಿದ್ದ: “ಒಳ್ಳೆಯದು ಕೆಟ್ಟದ್ದು ಎಂಬ ಎರಡು ಬಗೆಯ ಕರ್ಮಗಳ ಬಗ್ಗೆ ಜನರು ಮಾತನಾಡುತ್ತಾರೆ. ಅದರಲ್ಲಿ ಒಳ್ಳೆಯದು ಕೆಟ್ಟದಕ್ಕಿಂತ ಮೇಲು.”

ಈಗ ನಾವು ಯಾವ ಒಳ್ಳೆಯದನ್ನು ಅನುಸರಿಸಬೇಕು? ಮೋಕ್ಷಾಕಾಂಕ್ಷಿಯ ಒಳ್ಳೆಯದು ಒಂದು, ಧರ್ಮಾಕಾಂಕ್ಷಿಯ ಒಳ್ಳೆಯದು ಇನ್ನೊಂದು. ಈ ಸಂದೇಶವನ್ನು ಗೀತಾಚಾರ್ಯ ಶ‍್ರೀಕೃಷ್ಣನು ಅತಿ ಸುಂದರವಾಗಿ ಅಲ್ಲಿ ವಿವರಿಸಿರುವನು. ಈ ಮಹಾ ಸತ್ಯದ ಆಧಾರದ ಮೇಲೆ ಹಿಂದೂಗಳ ವರ್ಣಾಶ್ರಮ ಮತ್ತು ಸ್ವಧರ್ಮ ಸಿದ್ಧಾಂತ ನಿಂತಿರುವುದು.

\begin{verse}
ಅದ್ವೇಷ್ಟಾ ಸರ್ವಭೂತಾನಾಂ ಮೈತ್ರಃ ಕರುಣ ಏವ ಚ~॥\\ನಿರ್ಮಮೋ ನಿರಹಂಕಾರ ಸಮದುಃಖ ಸುಖಃ ಕ್ಷಮೀ~॥ (ಗೀತೆ-೧೨:೧೩)
\end{verse}

“ಯಾವನು ಯಾರನ್ನೂ ದ್ವೇಷಿಸದೆ ಮೈತ್ರಿ ಮತ್ತು ದಯೆಗಳಿಂದ ಕೂಡಿರುವನೊ, ನಾನು, ನನ್ನದು ಎಂಬ ಅಭಿಮಾನವಿಲ್ಲದೆ ಸುಖದುಃಖಗಳಲ್ಲಿ ಸಮನಾಗಿರುವನೊ, ಇವನು ಮಾತ್ರ ಮುಕ್ತಿಗೆ ಅರ್ಹ”.

\begin{verse}
ಕ್ಲೈಬ್ಯಂ ಮಾ ಸ್ಮ ಗಮಃ ಪಾರ್ಥ ನೈತತ್ತ್ವಯ್ಯುಪಪದ್ಯತೇ~॥\\ಕ್ಷುದ್ರಂ ಹೃದಯದೌರ್ಬಲ್ಯಂ ತ್ಯಕ್ತ್ವೋತ್ತಿಷ್ಠ ಪರಂತಪ~॥ (ಗೀತೆ-೨:೩)
\end{verse}

“ಪಾರ್ಥ, ಹೇಡಿಯಾಗಬೇಡ, ನಿನಗೆ ಇದು ತರವಲ್ಲ. ಈ ಕ್ಷುದ್ರವಾದ ಹೃದಯ ದೌರ್ಬಲ್ಯವನ್ನು ತೊರೆದು ಯುದ್ಧಕ್ಕೆ ಅಣಿಯಾಗು”.

\begin{verse}
ತಸ್ಮಾ ತ್ತ್ವಮುತ್ತಿಷ್ಠ ಯಶೋ ಲಭಸ್ವ\\ಜಿತ್ವಾ ಶತ್ರೂನ್​ ಭುಂಕ್ಷ್ವ ರಾಜಂ ಸಮೃದ್ಧಮ್​~॥
\end{verse}

\begin{verse}
ಮಯೈವೈತೇ ನಿಹತಾಃ ಪೂರ್ವಮೇವ\\ನಿಮಿತ್ತ ಮಾತ್ರಂ ಭವ ಸವ್ಯಸಾಚಿನ್​~॥ (ಗೀತೆ-೧೧:೩೩)
\end{verse}

“ಆದಕಾರಣ ಏಳು ಅರ್ಜುನ, ಯಶಸ್ಸನ್ನು ಪಡೆ. ಶತ್ರುಗಳನ್ನು ಗೆದ್ದು ಸಮೃದ್ಧ ವಾಗಿರುವ ರಾಜ್ಯವನ್ನು ಅನುಭವಿಸು. ಇವರು ಮೊದಲೇ ನನ್ನಿಂದ ಹತರಾಗಿರುವರು, ನೀನು ನಿಮಿತ್ತ ಮಾತ್ರನಾಗು”.

ಇತ್ಯಾದಿ ಧರ್ಮಪ್ರಾಪ್ತಿಯ ಮಾರ್ಗವನ್ನು ಭಗವಂತನ ಗೀತೆಯಲ್ಲಿ ತೋರಿರುವನು. ಕೆಲಸ ಮಾಡುವಾಗ ಆವಶ್ಯಕವಾಗಿ ಸ್ವಲ್ಪ ಒಳ್ಳೆಯದೂ, ಸ್ವಲ್ಪ ಕೆಟ್ಟದ್ದೂ ಮಿಶ್ರವಾಗಿಯೇ ಇರುವುವು. ಸ್ವಲ್ಪ ಮಟ್ಟಿಗಾದರೂ ಪಾಪವನ್ನು ಮಾಡಬೇಕಾಗುವುದು. ಆದರೆ ಏನು? ಅದು ಹಾಗೇ ಇರಲಿ. ಏನೂ ಇಲ್ಲದೆ ಇರುವುದಕ್ಕಿಂತ ಏನೋ ಸ್ವಲ್ಪ ಇರುವುದು ಮೇಲಲ್ಲವೆ? ಅರ್ಧೆಯ್ಧೂಟ ಉಪ್ಧವ್ಧಾಸ್ಧಕ್ಕ್ಧಿಂತ ಮೇಲ್ಧಲ್ಲ್ಧವೆ? ಒಳ್ಳ್ಧೆಯ್ಧದು ಕೆಟ್ಟ್ಧದ್ದ್ಧರ್ಧಿಂದ ಕೂಡ್ಧಿದ್ದ್ಧರೂ ಕೆಲ್ಧಸ್ಧಮ್ಧಾಡ್ಧದೆ ಶುದ್ಧ ಸ್ಧೋಮ್ಧಾರ್ಧಿಯ್ಧಾಗಿ ಇರ್ಧುವ್ಧುದ್ಧಕ್ಕ್ಧಿಂತ ಮೇಲ್ಧಲ್ಲ್ಧವೇ? ಹಸು ಸುಳ್ಳು ಹೇಳುವುದಿಲ್ಲ, ಕಲ್ಲು ಕಳ್ಳತನ ಮಾಡುವುದಿಲ್ಲ. ಆದರೆ ಹಸು ಹಸುವಾಗಿಯೇ ಇರುವುದು, ಕಲ್ಲು ಕಲ್ಲಿನಂತೆಯೇ ಇರುವುದು. ಮಾನವನು ಕಳ್ಳತನ ಮಾಡುವನು, ಸುಳ್ಳು ಹೇಳುವನು; ಆದರೂ ಅವನು ದೇವನೂ ಆಗಬಲ್ಲ. ಎಂದು ಮನುಷ್ಯ ಸತ್ವಗುಣ ಪ್ರಧಾನಾ ವಸ್ಥೆಯಲ್ಲಿ ರುವನೋ ಆಗ ನಿಷ್ಕ್ರಿಯನಾಗುವನು. ಅವನಿಗೆ ಪರಮಧ್ಯಾನಾವಸ್ಥೆ ಲಭಿಸುವುದು. ರಜೋಗುಣ ಪ್ರಧಾನಾವಸ್ಥೆಯಲ್ಲಿರುವಾಗ ಒಳ್ಳೆಯದು, ಕೆಟ್ಟದ್ದು ಎರಡೂ ಕರ್ಮಗಳನ್ನು ಮಾಡುವನು. ಯಾವ ಅವಸ್ಥೆಯಲ್ಲಿ ತಮೋಗುಣ ಪ್ರಧಾನ ವಾಗಿರುವುದೋ ಆಗ ಪುನಃ ನಿಷ್ಕ್ರಿಯನಾಗಿ ಜಡನಾಗುವನು. ಹೊರಗಡೆ ನೋಡಿದರೆ ಅವನಲ್ಲಿ ಸತ್ವಗುಣ ಪ್ರಧಾನವಾಗಿದೆಯೆ ತಮ್ಧೋಗ್ಧುಣ ಪ್ರಧ್ಧಾನ್ಧವ್ಧಾಗ್ಧಿದ್ಧೆಯೆ ಎಂದು ಹೇಗೆ ನಿರ್ಧರಿಸುವಿರಿ? ನೀವೇ ಹೇಳಿ, ನಾವು ಸುಖದುಃಖಗಳನ್ನು ಮೀರಿ, ಕ್ರಿಯಾ ಹೀನರಾಗಿ ಶುದ್ಧ ಸಾತ್ವಿಕಾವಸ್ಥೆಯಲ್ಲಿರುವೆವೆ, ಅಥವಾ ಶಕ್ತಿಯ ಅಭಾವದಿಂದ ಪ್ರಾಣಹೀನರಾಗಿ, ಜಡರಾಗಿ, ಕ್ರಿಯಾಹೀನರಾಗಿ, ತಾಮಸಿಕ ಅವಸ್ಥೆಯಲ್ಲಿರುವೆವೆ? ಈ ಪ್ರಶ್ನೆಗೆ ಉತ್ತರ ಹೇಳಿ. ನಿಮ್ಮ ಮನಸ್ಸನ್ನೇ ಕೇಳಿ. ಆಗ ಸತ್ಯ ಏನು ಎಂಬುದು ತಿಳಿಯುತ್ತದೆ. ಉತ್ತರಕ್ಕಾಗಿ ಕಾಯುವುದರಿಂದ ಏನು ಪ್ರಯೋಜನ? ಹಣ್ಣಿನಿಂದ ಮರದ ಸ್ವರೂಪ ತಿಳಿಯುತ್ತದೆ. ಸತ್ವ ಪ್ರಧಾನ ವ್ಯಕ್ತಿ ನಿಷ್ಕ್ರಿಯನಾಗುತ್ತಾನೆ, ಶಾಂತನಾಗುತ್ತಾನೆ. ಆದರೆ ಅವನ ನಿಷ್ಕ್ರಿಯೆ ಮಹಾ ಶಕ್ತಿಗೆ ಕೇಂದ್ರೀಭೂತವಾಗಿದೆ. ಅವನ ಶಾಂತಿ ಮಹಾವೀರ್ಯಕ್ಕೆ ಜನನಿಯಾಗಿದೆ. ಅಂತಹ ಮಹಾ ಪುರುಷನು ನಮ್ಮಂತೆ ಕೈಕಾಲುಗಳಿಂದ ಕೆಲಸಮಾಡ ಬೇಕಾಗಿಲ್ಲ.ಕೇವಲ ಅವನ ಇಚ್ಛಾ ಮಾತ್ರದಿಂದ ಸರ್ವಕರ್ಮವೂ ಸಂಪೂರ್ಣವಾಗುವುದು. ಅಂತಹ ವ್ಯಕ್ತಿ ಸತ್ವಗುಣ ಪ್ರಧಾನ ಬ್ರಾಹ್ಮಣ. ಎಲ್ಲರಿಗೂ ಪೂಜ್ಯನು ಅವನು. ಅವನು “ನನ್ನನ್ನು ಪೂಜಿಸಿ” ಎಂದು ಎಲ್ಲರ ಮನೆ ಬಾಗಿಲಿಗೂ ಹೋಗಿ ಹೇಳಬೇಕೇನು? ಜಗದಂಬೆ ತನ್ನ ಸ್ವಂತ ಕೈಗಳಿಂದ ಅವನ ಲಲಾಟದ ಮೇಲೆ ಚಿನ್ನದ ಅಕ್ಷರಗಳಲ್ಲಿ “ಈ ಮಹಾ ಪುರುಷನನ್ನು ಎಲ್ಲರೂ ಪೂಜಿಸಿ” ಎಂದು ಬರೆಯುವಳು. ಜಗತ್ತು ತಲೆಬಾಗಿ ಅವನನ್ನು ಗೌರವಿಸುವುದು. ಅವನೇ-

\begin{verse}
ಅದ್ವೇಷ್ಟಾ ಸರ್ವಭೂತಾನಾಂ ಮೈತ್ರಃ ಕರುಣ ಏವ ಚ~।\\ನಿರ್ಮಮೋ ನಿರಹಂಕಾರಃ ಸಮದುಃಖಸುಖಃ ಕ್ಷಮೀ~॥
\end{verse}

ಯಾರು ಮಣಮಣ ಎಂದು ಮೂಗಿನಿಂದ ಹೆಂಗಸರಂತೆ ಗುಜುಗುಟ್ಟುವರೋ, ಯಾರ ಧ್ವನಿ ಒಂದು ವಾರದಿಂದ ಉಪವಾಸ ಮಾಡುತ್ತಿರುವರು ಎಂಬ ಭ್ರಾಂತಿಯನ್ನು ಹುಟ್ಟಿಸುವುದೋ, ಹರಿದುಹೋದ ಚಿಂದಿಯ ಬೊಂತೆಯಂತೆ ಯಾರು ಅವರನ್ನು ಒದೆಯಲಿ, ತೆಪ್ಪಗಿರುವರೋ, ಅವರು ನಿಮ್ನತಮ ಶ್ರೇಣಿಯ ತಮೋ ಗುಣದ ಪ್ರಕಾಶ; ಇದು ಮೃತ್ಯು ಚಿಹ್ನೆ, ಇದು ಸತ್ವಗುಣವಲ್ಲ, ಬರೀ ದುರ್ಗಂಧ. ಅರ್ಜುನನಿಗೂ ಅಂತಹ ಅವಸ್ಥೆ ಪ್ರಾಪ್ತವಾಗಿತ್ತು. ಇದನ್ನು ಹೋಗಲಾಡಿಸುವುದಕ್ಕಾಗಿಯೇ ಶ‍್ರೀಕೃಷ್ಣ ಗೀತೆಯನ್ನು ಹೇಳಬೇಕಾಗಿ ಬಂದದ್ದು. ಭಗವಂತನ ವದನದಿಂದ ಮೊದಲು ಬಿದ್ದ ಪದಗಳೇ ಪರಿಗಣಿಸಿ: “ಕ್ಲೈಬ್ಯಂ ಮಾ ಸ್ಮ ಗಮಃ ಪಾರ್ಥ ನೈತತ್ತ್ವಯ್ಯುಪಪದ್ಯತೇ” ಎನ್ನುವನು. ನಂತರ “ತಸ್ಮಾತ್ತ್ವಂ ಉತ್ತಿಷ್ಠ ಯಶೋ ಲಭಸ್ವ” ಎನ್ನುವನು.

ಜೈನ ಮತ್ತು ಬೌದ್ಧರ ಪ್ರಭಾವಕ್ಕೆ ಒಳಗಾಗಿ ನಾವೆಲ್ಲರೂ ತಮೋಗುಣೋ ಪಾಸಕರಾಗಿರುವೆವು. ಸಾವಿರ ವರ್ಷಗಳಿಂದ ದೇಶದಲ್ಲಿ ಹರಿನಾಮ ಧ್ವನಿಯು ನಭೋಮಂಡಲವನ್ನು ವ್ಯಾಪಿಸಿದೆ, ಆದರೆ ದೇವರು ಅದಕ್ಕೆ ಕಿವಿಯನ್ನೇ ಕೊಡುವುದಿಲ್ಲ. ಅವನು ಏತಕ್ಕೆ ಕೇಳಬೇಕು? ಮನುಷ್ಯನೇ ಮತ್ತೊಬ್ಬ ಬೇಕೂಫನ ಮಾತನ್ನು ಕೇಳುವುದಿಲ್ಲ. ಹೀಗಿರುವಾಗ ದೇವರು ಕೇಳುವನೆ? ಈಗ ಗೀತೆಯಲ್ಲಿ ಬರುವ ಪರಮಾತ್ಮನ ವಾಕ್ಯವನ್ನು ಕೇಳುವುದು ನಮ್ಮ ಕರ್ತವ್ಯ: “ಪಾರ್ಥ, ಹೇಡಿತನ ನಿನಗೆ ತರವಲ್ಲ. ಯುದ್ಧ ಮಾಡಿ ಯಶಸ್ಸನ್ನು ಅನುಭವಿಸು.”

ಈಗ ಪ್ರಾಚ್ಯ ಪಾಶ್ಚಾತ್ಯ ವಿಷಯಕ್ಕೆ ಬರೋಣ. ಮೊದಲು ಈ ತಮಾಷೆ ಯನ್ನು ನೋಡಿ. ಯುರೋಪಿಯನ್ನರ ದೇವನಾದ ಕ್ರಿಸ್ತ ಹೀಗೆ ಬೋಧಿಸುವನು: “ಯಾರನ್ನೂ ದ್ವೇಷಿಸಬೇಡಿ; ಒಂದು ಕೆನ್ನೆಗೆ ಹೊಡೆದರೆ ಮತ್ತೊಂದು ಕೆನ್ನೆಯನ್ನು ತೋರಿಸಿ. ಕರ್ಮಗಿರ್ಮವನ್ನೆಲ್ಲ ಬಿಟ್ಟು ಪರಲೋಕಕ್ಕೆ ತೆರಳಲು ಅನುವಾಗಿರಿ; ಈ ಪ್ರಪಂಚವು ಇನ್ನು ಕೆಲವೇ ದಿನಗಳಲ್ಲಿ ನಾಶವಾಗುವುದು.” ಆದರೆ ಭಗವದ್ಗೀತೆಯ ನಮ್ಮ ದೇವನು ಹೀಗೆ ಉಪದೇಶ ಮಾಡುವನು: “ಉತ್ಸಾಹದಿಂದ ಕರ್ಮಮಾಡಿ, ಶತ್ರುಗಳನ್ನು ನಾಶಮಾಡಿ, ಪ್ರಪಂಚವನ್ನು ಅನುಭವಿಸಿ.” ಈಗ ಎಲ್ಲಾ ತಲೆ ಕೆಳಗಾಗಿದೆ. ಯುರೋಪಿಯನ್ನರು ಏಸುವಿನ ಮಾತನ್ನು ಕೇಳುತ್ತಿಲ್ಲ. ಮಹಾ ರಜೋಗುಣಿಗಳಾಗಿ, ಕಾರ್ಯಶೀಲರಾಗಿ, ಬಹಳ ಉತ್ಸಾಹದಿಂದ ದೇಶದೇಶಾಂತರಗಳಲ್ಲಿ ಸುಖಭೋಗಗಳನ್ನು ಪಡೆದು ಆನಂದಿಸುತ್ತಿರುವರು. ಆದರೆ ನಾವು ಗಂಟುಮೂಟೆಕಟ್ಟಿ ಕೋಣೆಯ ಮೂಲೆಯಲ್ಲಿ ಕುಳಿತು ಹೀಗೆ ಮೃತ್ಯುಚಿಂತನೆ ಮಾಡುತ್ತಿರುವೆವು. \textbf{“ನಲಿನೀ ದಲಗತಜಲಮತಿತರಲಂ ತದ್ವಜ್ಜೀವನಮತಿಶಯಚಪಲಂ”} - ಮನುಷ್ಯನ ಬಾಳು ಕಮಲಪತ್ರದ ಮೇಲೆ ಸುಳಿದಾಡುತ್ತಿರುವ ಜಲಬಿಂದುವಿನಂತೆ ಅತಿ ಚಂಚಲ, ಕ್ಷಣಿಕ. ಯಮನ ಭಯದಿಂದ ನಾಡಿನಲ್ಲಿ ರಕ್ತವು ಘನೀಭೂತವಾಗುತ್ತಿದೆ. ಶರೀರವೆಲ್ಲ ಕಂಪಿಸುತ್ತದೆ. ಯಮನಿಗೂ ನಮ್ಮ ಮೇಲೆ ಕ್ರೋಧ ಬಂದಿರುವಂತೆ ತೋರುವುದು. ಅದಕ್ಕೇ ಹಲವಾರು ರೋಗರುಜಿನಗಳು ದೇಶದಲ್ಲಿ ಧಾಳಿ ಯಿಟ್ಟಿರುವುವು. ಗೀತೆಯ ಉಪದೇಶವನ್ನು ಯಾರು ಅನುಸರಿಸುತ್ತಿರುವರು? ಯೂರೋಪಿನ್ನರು! ಏಸುವಿನ ಇಚ್ಛೆಯಂತೆ ಯಾರು ಕೆಲಸಮಾಡುತ್ತಿರುವರು? ಶ‍್ರೀಕೃಷ್ಣನ ವಂಶಜರು! ಇದನ್ನು ಚೆನ್ನಾಗಿ ನಾವು ತಿಳಿದುಕೊಳ್ಳಬೇಕಾಗಿದೆ. ಮೋಕ್ಷ ಮಾರ್ಗಕ್ಕೆ ಪ್ರಥಮ ಉಪದೇಶ ನಮಗೆ ದೊರಕುವುದು ವೇದದಲ್ಲಿ. ಇದರ ನಂತರ ಬುದ್ಧ ಕ್ರಿಸ್ತ ಇವರೆಲ್ಲ ಅದನ್ನೆ ಅಲ್ಲಿಂದ ತೆಗೆದುಕೊಂಡುರುವರು. ಅವರು ಸಂನ್ಯಾಸಿಗಳಾಗಿದ್ದರು; ಅವರಿಗೆ ಶತ್ರುಗಳು ಯಾರೂ ಇರಲಿಲ್ಲ; ಅವರು ಎಲ್ಲರನ್ನೂ ಪ್ರೀತಿಸುತ್ತಿದ್ದರು, ಇದು ಅವರಿಗೆ ಒಳ್ಳೆಯದಾಗಿತ್ತು. ಏತಕ್ಕೆ ಬಲವಂತ ದಿಂದ ಎಲ್ಲರನ್ನೂ ಮೋಕ್ಷಮಾರ್ಗದ ಕಡೆಗೆ ಸೆಳೆಯಬೇಕಾಗಿತ್ತು? ಸುಮ್ಮನೆ ತಿಕ್ಕುವುದರಿಂದ ಸೌಂದರ್ಯ ಬರುವುದೆ? ಬಲವಂತದಿಂದ ಪ್ರೇಮ ಹುಟ್ಟು ವುದೇ? ಎಂದಿಗೂ ಇಲ್ಲ. ಮೋಕ್ಷವನ್ನು ಬಯಸದವನಿಗೆ, ಅಥವಾ ಮೋಕ್ಷಕ್ಕೆ ಅರ್ಹನಲ್ಲದವನಿಗೆ ಬುದ್ಧ ಅಥವಾ ಕ್ರಿಸ್ತನು ಉಪದೇಶಿಸಿದ್ದೇನು? ಏನೂ ಇಲ್ಲ. ಮೋಕ್ಷವನ್ನು ಪಡೆಯಿರಿ, ಇಲ್ಲವೆ ನೀವು ನಾಶವಾಗುವಿರಿ, ಎಂಬ ಎರಡೇ ಮಾರ್ಗ ಗಳನ್ನು ಜನರ ಮುಂದೆ ಇಟ್ಟರು. ಮಧ್ಯಮಾರ್ಗವೇ ಇಲ್ಲ, ಮೋಕ್ಷವಲ್ಲದೆ ಬೇರೆ ಏನನ್ನೂ ಸಾಧಿಸಲು ಮಾರ್ಗವೇ ಇಲ್ಲ. ಪ್ರಪಂಚವನ್ನು ಸ್ವಲ್ಪ ಅನುಭವಿ ಸುವುದಕ್ಕೆ ದಾರ್ಧಿಯೇ ಇಲ್ಲ. ದಾರಿ ಇಲ್ಲದಿರುವುದು ಮಾತ್ರವಲ್ಲ, ಅದಕ್ಕೆ ಆತಂಕ ಗಳನ್ನೂ ತಂದೊಡ್ಡಿರುವರು. ಕೇವಲ ವೈದಿಕ ಧರ್ಮದಲ್ಲಿ ಮಾತ್ರ ಧರ್ಮ, ಅರ್ಥ, ಕಾಮ, ಮೋಕ್ಷ ಎಂಬ ನಾಲ್ಕು ಪುರುಷಾರ್ಥಗಳಿಗೆ ಸ್ಥಳವಿದೆ. ಬುದ್ಧ ನಮ್ಮ ಸರ್ವನಾಶ ಮಾಡಿದನು. ಕ್ರಿಸ್ತ ಗ್ರೀಸ್​ ಮತ್ತು ರೋಮಿಗೆ ಸರ್ವನಾಶ ತಂದನು. ದೈವವಶದಿಂದ ಯೂರೋಪಿಯನ್ನರು ಕಾಲಕ್ರಮೇಣ ಪ್ರಾಟೆಸ್ಟಂಟ ರಾದರು, ಚರ್ಚು ಬೋಧಿಸಿದ ರೀತಿಯ ಕ್ರಿಸ್ತನ ಬೋಧನೆಗೆ ತಿಲತರ್ಪಣ ಕೊಟ್ಟು ಸುಖಿಗಳಾದರು. ಭಾರತದಲ್ಲಿ ಪುನಃ ಕುಮಾರಿಲನು ಕರ್ಮಮಾರ್ಗವನ್ನು ತಂದನು, ಶಂಕರ ರಾಮಾನುಜರು ಚತುರ್ವಿಧ ಪುರುಷಾರ್ಥವನ್ನು ಪುನಃ ಜಾರಿಗೆ ತಂದರು. ಆಗ ದೇಶವನ್ನು ಕಾಪಾಡಲು ಸಾಧ್ಯವಾಯಿತು. ಆದರೆ ಭರತಖಂಡದಲ್ಲಿ ಮೂವತ್ತು ಕೋಟಿ ಜನರನ್ನು ಜಾಗ್ರತಗೊಳಿಸಬೇಕಾಗಿದೆ, ಅದಕ್ಕೆ ತಡವಾಗುವುದು. ಮೂವತ್ತು ಕೋಟಿ ಜನರನ್ನು ಒಂದು ದಿನದಲ್ಲಿ ಜಾಗ್ರತರನ್ನಾಗಿ ಮಾಡಲು ಸಾಧ್ಯವೇ?

ಬೌದ್ಧ ಮತ್ತು ವೈದಿಕ ಧರ್ಮಗಳ ಉದ್ದೇಶ ಒಂದೆ. ಆದರೆ ಬುದ್ಧನು ಅನುಸರಿಸಿದ ಮಾರ್ಗ ಸರಿಯಲ್ಲ. ಅವನ ಮಾರ್ಗ ಸರಿಯಾಗಿದ್ದಿದ್ದರೆ ನಮಗೆ ಸರ್ವನಾಶ ಏತಕ್ಕೆ ಆಗುತ್ತಿತ್ತು? ಎಲ್ಲಾ ಕಾಲಾನಂತರ ಆಯಿತು ಎಂದರೆ ಸರಿಯಲ್ಲ, ಕಾಲ ಮಾತ್ರ ಕೆಲಸಮಾಡುವುದೆ, ಕಾರ್ಯಕಾರಣಗಳನ್ನು ಮೀರಿ?

ಉದ್ದೇಶ ಒಂದೇ ಆದರೂ ಉಚಿತ ಉಪಾಯದ ಅಭಾವದಿಂದ ಬೌದ್ಧರು ಭರತವರ್ಷವನ್ನು ರಸಾತಳಕ್ಕೆ ಒಯ್ದರು. ನಾನು ಹೀಗೆ ಹೇಳಿದರೆ ನನ್ನ ಬೌದ್ಧ ಮಿತ್ರರಿಗೆ ಕೋಪ ಬರಬಹುದು. ಆದರೆ ವಿಧಿಯಿಲ್ಲ. ಸತ್ಯವನ್ನುಹೇಳಲೇಬೇಕಾಗಿದೆ. ಪರಿಣಾಮ ಏನು ಬೇಕಾದರೂ ಆಗಲಿ. ವೈದಿಕ ದಾರಿಯೇ ಉಚಿತವಾದುದು. ಜಾತಿ ಧರ್ಮ, ಸ್ವಧರ್ಮವೇ ವೈದಿಕಧರ್ಮ ಮತ್ತು ಸಮಾಜದ ಭಿತ್ತಿ. ಮತ್ತೆ ಕೆಲವು ನನ್ನ ಮಿತ್ರರಿಗೆ ನಾನು ಭಾರತೀಯರನ್ನು ಹೊಗಳುತ್ತಿರುವೆನೆಂದು ಕೋಪ ಬಂದಿರಬಹುದು. ಇಂತಹವರನ್ನು ಹೊಗಳಿ ಬರುವ ಪ್ರಯೋಜನವೇನು ಎಂದು ನಾನು ಅವರನ್ನು ಕೇಳುತ್ತೇನೆ. ನಾನು ಸಾಯುತ್ತಿದ್ದರೂ ಅವರು ನೋಡಿ ಒಂದು ಮುಷ್ಟಿ ಅನ್ನವನ್ನು ಕೊಡಲಾರರು. ಇಲ್ಲಿರುವ ದುರ್ಭಿಕ್ಷಗ್ರಸ್ಥ ಅನಾಥರಿಗೆ ಪರದೇಶದಿಂದ ಯಾಚಿಸಿ ಏನಾದರೂ ತಂದರೆ, ಅದನ್ನು ತೆಗೆದುಕೊಳ್ಳಲು ಯತ್ನಿಸುವರು. ಅದು ಸಿಕ್ಕದೇ ಇದ್ದರೆ ಬೈಯುತ್ತಾರೆ, ನನ್ನ ಚಾರಿತ್ರ್ಯಕ್ಕೆ ಮಸಿ ಬಳಿಯುತ್ತಾರೆ. ಇವರೇ ನಮ್ಮ ಶಿಕ್ಷಿತ ದೇಶಬಂಧುಗಳು. ನಮ್ಮ ದೇಶದವರೆಲ್ಲ ಹೀಗೆ ಇರುವರು. ಇವರನ್ನು ಹೊಗಳಿ ಪ್ರಯೋಜನವಿಲ್ಲ ಎಂಬುದು ನನಗೆ ಚೆನ್ನಾಗಿ ಗೊತ್ತು. ಹುಚ್ಚರಂತೆ ಅವರನ್ನು ನೋಡಬೇಕು. ಅವರಿಗೆ ಔಷಧಿ ಕೊಡಲು ಹೋದರೆ ಅವರಿಂದ ಪೆಟ್ಟು ತಿನ್ನಲು ಸಿದ್ಧರಾಗಿರಬೇಕು. ಆದರೆ ಯಾರು ಅದನ್ನು ಸಹಿಸಿ ಔಷಧಿಯನ್ನು ಅವರು ಗಂಟಲಿಗೆ ತುರುಕುವರೋ ಅವರೇ ಅವರ ನಿಜವಾದ ಸ್ನೇಹಿತರು.

ಜಾತಿಧರ್ಮ, ಸ್ವಧರ್ಮ ಎಲ್ಲಾ ದೇಶಗಳಲ್ಲಿಯೂ ಸಾಮಾಜಿಕ ಉನ್ನತಿಯ ಉಪಾಯ, ಅದೇ ಮುಕ್ತಿ ಸೋಪಾನ. ಈ ಜಾತಿಧರ್ಮ ಸ್ವಧರ್ಮ ನಾಶ್ಧವ್ಧಾದ್ಧರೆ ನ್ಧಮ್ಮ ದೇಶ್ಧವು ಅಧ್ಧಃಪ್ಧತ್ಧನ್ಧವ್ಧಾಗ್ಧುವ್ಧುದು. ಯಾವ್ಧುದ್ಧನ್ನು ಜಾತ್ಧಿಧ್ಧರ್ಮ ಸ್ವಧ್ಧರ್ಮ ಎಂದು ಈಗ ಉನ್ನತ ಜಾತಿಯವರು ತಿಳಿದುಕೊಂಡಿರುವರೊ ಅದು ಈಗ ಬರೀ ತಲೆಕೆಳಗು. ಇದೊಂದು ಮಹೋತ್ಪಾತ. ದೇಶವನ್ನು ಇದರಿಂದ ರಕ್ಷಿಸಬೇಕು. ಮೇಲಿನ ಜಾತಿಯವರು ನಾವು ಜಾತಿ ಧರ್ಮವನ್ನೆಲ್ಲ ತಿಳಿದುಕೊಂಡಿರುವೆವು ಎಂದು ಭಾವಿಸುವರು. ಆದರೆ ಇದು ಭ್ರಾಂತಿ; ಅವರು ತಮ್ಮ ಗ್ರಾಮದ ಆಚಾರವನ್ನೇ ವೇದೋಕ್ತ ಸನಾತನಧರ್ಮ ಎಂದು ಭಾವಿಸುವರು. ಅವರು ಅಧಿಕಾರವನ್ನೆಲ್ಲ ತಾವೆ ತೆಗೆದುಕೊಂಡು ನಾಶವಾಗುತ್ತಿರುವರು. ನಾನು ಗುಣಗತ ಜಾತ್ಧಿಯ ವಿಚ್ಧ್ಧಾರ್ಧವ್ಧನ್ನು ಹೇಳ್ಧುತ್ತ್ಧಿಲ್ಲ. ವಂಶ್ಧಗ್ಧತ, ಜನ್ಮ್ಧಗ್ಧತ ಜಾತ್ಧಿಯ್ಧನ್ನು ಕುರ್ಧಿತು ಹೇಳ್ಧುತ್ತ್ಧಿರ್ಧುವ್ಧುದು. ಗುಣ್ಧಗ್ಧತ ಜಾತಿಯು ಪುರಾತನವೆಂಬುದು ನನಗೆ ಗೊತ್ತಿದೆ. ನಾಲ್ಕೈದು ತಲೆಗಳ ನಂತರ ಇದು ವಂಶಗತವಾಗುವುದು. ಇದು ಶೋಚನೀಯ. ಹೀಗೆ ನಮ್ಮ ದೇಶದ ಜನ ಜೀವನದ ಪ್ರಾಣನಾಡಿಗೆ ಧಕ್ಕೆಯುಂಟಾಯಿತು. ಇಲ್ಲದಿದ್ದರೆ ನಾವೇಕೆ ಈ ಅವನತಿಗೆ ಇಳಿಯುತ್ತಿದ್ದೆವು? ಗೀತೆಯಲ್ಲಿ \textbf{“ಸಂಕರಸ್ಯ ಚ ಕರ್ತಾಸ್ಯಾ ಮುಪಹನ್ಯಾಮಿಮಾಃ ಪ್ರಜಾಃ”} - ಆಗ ನಾನು ವರ್ಣಸಂಕರಕ್ಕೆ ಕಾರಣನಾಗಿ ಇವರು ನಾಶವಾಗುತ್ತಾರೆ ಎಂದು ಏಕೆ ಹೇಳಿದೆ? ಎಂದರೆ ಇಂತಹ ಘೋರ ವರ್ಣಸಂಕರ ಹೇಗೆ ಆಯಿತು? ಎಲ್ಲ ಬಣ್ಣ ಏತಕ್ಕೆ ಕಪ್ಪಾಯಿತು? ಸತ್ವಗುಣವು ರಜೋಗುಣ ಪ್ರಧಾನವಾದ ತಮೋಗುಣ ಹೇಗೆ ಆಯಿತು? ಇದೊಂದು ದೊಡ್ಡ ಕಥೆ. ಅದನ್ನು ಮತ್ತೊಮ್ಮೆ ನಿಧಾನವಾಗಿ ಹೇಳುತ್ತೇನೆ. ಸದ್ಯಕ್ಕೆ ಇದನ್ನು ನಾವು ತಿಳಿದುಕೊಳ್ಳಬೇಕು. ಜಾತಿಧರ್ಮವು ಸರಿಯಾಗಿದ್ದರೆ ದೇಶವು ಅಧೋಗತಿಗೆ ಬರುವುದಿಲ್ಲ. ಇದು ಸತ್ಯವಾದರೆ ನಮ್ಮ ನಾಶಕ್ಕೆ ಕಾರಣ ಯಾವುದು? ಈಗ ಜಾತಿಧರ್ಮ ಹಾಳಾಗಿದೆ. ನೀವು ಯಾವುದನ್ನು ಈಗ ಜಾತಿಧರ್ಮವೆನ್ನುತ್ತೀರೋ ಅದು ಅದಕ್ಕೆ ವಿರೋಧ. ನೀವು ಮತ್ತೊಮ್ಮೆ ನಿಮ್ಮ ಪುರಾಣ ಮತ್ತು ಶಾಸ್ತ್ರಗಳನ್ನು ಓದಿ. ಅಲ್ಲಿ ಯಾವುದನ್ನು ಜಾತಿ ಧರ್ಮ ವೆನ್ನುತ್ತಾರೊ ಅದು ಈಗ ಸಂಪೂರ್ಣವಾಗಿ ಮಾಯವಾಗಿದೆ. ಅದನ್ನೆ ಪುನಃ ತರುವುದಕ್ಕೆ ಪ್ರಯತ್ನಿಸಿ. ಆಗ ಅದರಿಂದ ದೇಶಕ್ಕೆ ಕಲ್ಯಾಣವಾಗುವುದು. ನಾನು ಯಾವುದನ್ನು ತಿಳಿದು ಕೊಂಡಿರುವೆನೊ ಅದನ್ನು ನಾನು ಸ್ಪಷ್ಟವಾಗಿ ಹೇಳುತ್ತಿರುವೆನು. ನಾನು ನಿಮ್ಮ ಕಲ್ಯಾಣಕ್ಕಾಗಿ ಹೊರದೇಶದಿಂದ ಆಮದಾಗಿ ಬಂದಿಲ್ಲ. ನಿಮ್ಮ ಹಳೆಯ ಆಚಾರಕ್ಕೆಲ್ಲ ವೈಜ್ಞಾನಿಕ ರೀತಿಯ ವ್ಯಾಖ್ಯಾನ ಕೊಡುವುದಿಲ್ಲ. ವಿದೇಶ ಬಂಧುಗಳು ಅದನ್ನು ಪುಕ್ಕಟೆಯಾಗಿ ಮಾಡಲು ಸಿದ್ಧರಿರುವರು. ಅವರಿಗೆ ಸ್ವಲ್ಪ ಶಹಭಾಸ್​ ಎಂದರೆ ಸಾಕು. ನಿಮ್ಮ ಮುಖಕ್ಕೆ ಕೆಸರು ಬಿದ್ದರೆ ಅದು ನನಗೂ ಬೀಳುವುದು. ಇದರಿಂದ ಏನು ಪ್ರಯೋಜನ?

ಪ್ರತಿಯೊಂದು ಜನಾಂಗಕ್ಕೂ ಒಂದು ಉದ್ದೇಶವಿದೆ ಎಂದು ಪೂರ್ವದಲ್ಲಿಯೇ ಹೇಳಿರುವೆವು. ಪ್ರಕೃತಿಯ ನಿಯಮಾನುಸಾರ ಮತ್ತು ಮಹಾಪುರುಷರ ಪ್ರತಿಭೆಯ ಬಲದಿಂದ ಪ್ರತಿಯೊಂದು ಜನಾಂಗದ ರೀತಿ ನೀತಿಗಳೂ ಅದರ ಉದ್ದೇಶ ಸಾಧನೆಗಾಗಿ ರಚಿಸಲ್ಪಟ್ಟಿವೆ. ಪ್ರತಿಜನಾಂಗದ ಜೀವನದಲ್ಲಿಯೂ ಅದರ ಉದ್ದೇಶ ಸಾಧನೆಗಾಗಿ ಇರುವ ರೀತಿ ನೀತಿಗಳನ್ನು ಬಿಟ್ಟರೆ ಉಳಿದುವೆಲ್ಲವೂ ಅಷ್ಟೇನೂ ಪ್ರಾಮುಖ್ಯವಲ್ಲ. ಇಂತಹ ಪ್ರಾಮುಖ್ಯವಲ್ಲದ ರೀತಿ ನೀತಿಗಳು ಅಭಿವೃದ್ಧಿಯಾಗಲಿ ನಾಶವಾಗಲಿ, ಇದರಿಂದ ಅತಂಕವೇನೂ ಇಲ್ಲ. ಎಂದು ಪ್ರಧಾನ ಉದ್ದೇಶಕ್ಕೆ ಅಘಾತವಾಗುವುದೋ ಆಗ ಆ ಜನಾಂಗ ನಾಶವಾಗುವುದು.

ನಾವು ಹುಡುಗರಾಗಿದ್ದಾಗ ಒಂದು ಕಥೆಯನ್ನು ಕೇಳಿದ್ದೆವು. ಒಬ್ಬ ರಾಕ್ಷಸಿಯ ಪ್ರಾಣ, ಪಕ್ಷಿಯಲ್ಲಿತ್ತು. ಆ ಪಕ್ಷಿ ನಾಶವಾಗುವವರೆಗೂ ಯಾವ ಪ್ರಕಾರ ದಿಂದಲೂ ಆ ರಾಕ್ಷಸಿ ನಾಶವಾಗುತ್ತಿರಲಿಲ್ಲ. ಜನಾಂಗದ ಜೀವನವೂ ಹೀಗೆ. ಯಾವುದು ರಾಷ್ಟ್ರ ಜೀವನಕ್ಕೆ ಮುಖ್ಯವಲ್ಲವೊ ಅದನ್ನು ಅಷ್ಟು ಕಳೆದುಕೊಂಡರೂ ಚಿಂತೆಯಿಲ್ಲ. ಯಥಾರ್ಥ ರಾಷ್ಟ್ರ ಜೀವನಕ್ಕೆ ಎಂದು ಆಘಾತ ಬರುವುದೋ ಆಗ ಬಹಳ ವೇಗದಿಂದ ಪ್ರತಿಘಾತ ಬರುವುದು.

ಮೂರು ವರ್ತಮಾನಕಾಲದ ರಾಷ್ಟ್ರಗಳನ್ನು ತುಲನೆ ಮಾಡೋಣ. ಅವುಗಳ ಇತಿಹಾಸ ನಮಗೆ ಹೆಚ್ಚು ಪರಿಚಯವಿದೆ. ಅವೇ ಫ್ರೆಂಚ್​, ಆಂಗ್ಲೇಯ ಮತ್ತು ಹಿಂದೂ ಜನಾಂಗಗಳು. ರಾಜಕೀಯ ಸ್ವಾತಂತ್ರ್ಯ ಫ್ರೆಂಚ್​ ಜನಾಂಗದ ಮೇರು ದಂಡ. ಫ್ರೆಂಚ್​ ಜನರು ಎಲ್ಲಾ ರೀತಿಯ ಅತ್ಯಾಚಾರಗಳನ್ನೂ ಶಾಂತಿಯಿಂದ ಸಹಿಸಿಕೊಳ್ಳುವರು. ತೆರಿಗೆಯ ಹೊರೆಯನ್ನು ಅವರ ಮೇಲೆ ಹೊರಿಸಿ; ಗೊಣಗಾಡದೆ ಸಹಿಸುವರು. ಬಲಾತ್ಕಾರದಿಂದ ಎಲ್ಲರನ್ನೂ ಸೇನೆಗೆ ಸೇರುವಂತೆ ಮಾಡಿ; ಅವರು ಗೊಣಗುವುದಿಲ್ಲ. ಆದರೆ ಎಂದು ಅವರ ಸ್ವಾತಂತ್ರ್ಧ್ಯದ ಮೇಲೆ ಇತರರು ಕೈ ಹಾಕುವರೋ ಆಗ ಇಡೀ ಜನಾಂಗವು ಉನ್ಮತ್ತರಂತೆ ಪ್ರತಿಘಾತಕ್ಕೆ ತಕ್ಷಣ ಸಿದ್ಧವಾಗುವುದು. ಯಾರಿಗೂ ಅವರ ಸ್ವಾತಂತ್ರ್ಯವನ್ನು ಅಪಹರಿಸಲು ಶಕ್ತಿಯಿಲ್ಲ. ಇದೇ ಫ್ರೆಂಚ್​ ಜನಾಂಗದ ಚರಿತ್ರೆಯ ಮೂಲಮಂತ್ರ. ಜ್ಞಾನಿ, ಮೂರ್ಖ, ದರಿದ್ರ, ಧನಿಕ, ಉಚ್ಚವಂಶೀಯ, ನೀಚವಂಶೀಯ ಎಲ್ಲರಿಗೂ ರಾಜ್ಯಾಂಗ ಶಾಸನದಲ್ಲಿ, ಸಾಮಾಜಿಕ ಸ್ವಾತಂತ್ರ್ಯದಲ್ಲಿ, ಸಮಾನ ಅಧಿಕಾರವಿದೆ. ಯಾರು ಅದರ ಗೋಜಿಗೆ ಹೋಗುವರೊ ಅವರು ಅದರ ಫಲವನ್ನು ಅನುಭವಿಸ ಬೇಕಾಗುವುದು.

ಆಂಗ್ಲೇಯರ ಸ್ವಭಾವದಲ್ಲಿ ಕೊಟ್ಟು ತೆಗೆದುಕೊಳ್ಳುವ ವೈಶ್ಯನ ಬುದ್ಧಿ ಪ್ರಧಾನ. ಇಂಗ್ಲಿಷರಿಗೆ ಸಂಪತ್ತನ್ನು ನ್ಯಾಯಬದ್ಧವಾಗಿ ಹಂಚಿಕೊಳ್ಳುವುದು ಅತ್ಯಂತ ಮುಖ್ಯ. ಇಂಗ್ಲಿಷರು ತಮ್ಮ ರಾಜನ ಮತ್ತು ಉನ್ನತ ವರ್ಗದವರ ಅಧಿಕಾರವನ್ನು ಮಾನ್ಯ ಮಾಡುತ್ತಾರೆ. ಆದರೆ ಅವರು ರಾಜನಿಗೆ ಒಂದು ಕಾಸು ಕೊಡ ಬೇಕಾದರೂ ಅದರ ಲೆಕ್ಕಾಚಾರವನ್ನು ಪರೀಕ್ಷಿಸುವರು. ರಾಜನಿರುವನು; ಅದು ಒಳ್ಳೆಯದು. ಅವನನ್ನು ಆದರಿಸುವರು. ಆದರೆ ಅವನು ಏನಾದರೂ ದುಡ್ಡು ಕೇಳಿದರೆ, ಅದರ ಆವಶ್ಯಕತೆ ಮತ್ತು ಅದರಿಂದ ಬರುವ ಪ್ರಯೋಜನ ಇದರ ಲೆಕ್ಕಾಚಾರವನ್ನು ಅವನು ನೋಡಬೇಕು. ಅನಂತರ ಮಾತ್ರ ಕಾಸು ಬಿಚ್ಚುವನು. ಹಿಂದೆ ಒಬ್ಬ ರಾಜನು ಬಲಾತ್ಕಾರದಿಂದ ಹಣವನ್ನು ಜನರಿಂದ ವಸೂಲು ಮಾಡಲು ಪ್ರಯತ್ನಿಸಿದ. ಅದರಿಂದ ದೊಡ್ಡ ವಿಪ್ಲವವಾಗಿ ಜನರು ರಾಜನನ್ನೇ ಕೊಂದರು.

ರಾಜಕೀಯ ಮತ್ತು ಸಾಮಾಜಿಕ ಸ್ವಾತಂತ್ರ್ಯ ಒಳ್ಳೆಯ ವಿಷಯವೆಂದು ಹಿಂದೂ ಹೇಳುವನು. ಆದರೆ ಮುಖ್ಯವಾದದ್ದು ಆಧ್ಯಾತ್ಮಿಕ ಮುಕ್ತಿ. ಇದು ನಮ್ಮ ಜನಾಂಗದ ಗುರಿ. ವೈದಿಕ, ಜೈನ, ಬೌದ್ಧ, ದ್ವೈತ, ವಿಶಿಷ್ಟಾದ್ವೈತ, ಎಲ್ಲರೂ ಈ ವಿಷಯದಲ್ಲಿ ಏಕಮತೀಯರು. ಮುಕ್ತಿಭಾವನೆಯ ತಂಟೆಗೆ ಹೋಗದೇ ಇದ್ದರೆ ಏನೂ ಆಗುವುದಿಲ್ಲ. ಅದನ್ನು ಬಿಟ್ಟು ನೀವು ಏನು ಮಾಡಿದರೂ ಹಿಂದೂ ಸುಮ್ಮನೇ ಇರುವನು. ನೀವು ಅದರ ತಂಟೆಗೆ ಹೋದರೆ ನಿಮ್ಮ ಸರ್ವನಾಶ ಸಿದ್ಧ. ಅವನ ಸರ್ವಸ್ವವನ್ನೂ ದೋಚಿ, ಅವನನ್ನು ಒದೆಯಿರಿ, ಅವನನ್ನು ಗುಲಾಮ ಅಥವಾ ಇನ್ನೇನಾದರೂ ಕರೆಯಿರಿ, ಅವನ್ನೆಲ್ಲ ಅವನು ಲೆಕ್ಕಕ್ಕೇ ತೆಗೆದುಕೊಳ್ಳುವುದಿಲ್ಲ. ಆದರೆ ಮುಕ್ತಿಯ ಬಾಗಿಲನ್ನು ಮಾತ್ರ ಯಾವಾಗಲೂ ತೆರೆದಿರಿ. ಎಷ್ಟು ಜನ ಪಠಾಣರು ಹೊರಗಿನಿಂದ ಬಂದು ಹೋದರು. ಯಾರೂ ಸ್ಥಿರವಾಗಿ ರಾಜ್ಯವನ್ನು ಕಟ್ಟಲು ಆಗಲಿಲ್ಲ. ಏಕೆಂದರೆ ಹಿಂದೂಧರ್ಮವನ್ನು ಅವರು ಕೆಣಕಿದರು. ಆದರೆ ಮೊಗಲರ ರಾಜ್ಯ ಬಲವಾಗಿ ದೃಢವಾಗಿ ಪ್ರತಿಷ್ಠಿತವಾಯಿತು. ಅದಕ್ಕೆ ಕಾರಣ ಮೊಗಲರು ಹಿಂದೂಧರ್ಮವನ್ನು ಕೆಣಕಲಿಲ್ಲ. ಹಿಂದೂಗಳೇ ಮೊಗಲರ ಸಿಂಹಾಸನಕ್ಕೆ ಆಧಾರವಾಗಿದ್ದರು. ಜಹಾಂಗೀರ್​, ಷಾಜಹಾನ್​, ದಾರಶುಕೋ, ಇವರ ತಾಯಂದಿರು ಹಿಂದೂಗಳು. ಆದರೆ ನೋಡಿ! ಭಾಗ್ಯಹೀನ ಔರಂಗಜೇಬನು ಎಂದು ಹಿಂದೂಧರ್ಮದ ಮೇಲೆ ಆಘಾತವನ್ನು ಪ್ರಾರಂಭಿಸಿದನೊ ಆಗ ಅಷ್ಟು ದೊಡ್ಡ ಮೊಗಲರಾಜ್ಯ ಸ್ವಪ್ನದಂತೆ ಮಂಗಮಾಯವಾಯಿತು. ಆಂಗ್ಲೇಯರು ಕೂಡ ಇಷ್ಟು ದಿನ ಸಿಂಹಾಸನದ ಮೇಲೆ ಏತಕ್ಕೆ ಇಷ್ಟು ಭದ್ರವಾಗಿ ಇರುವರು? ಅದಕ್ಕೆ ಕಾರಣ ಅವರು ಯಾವ ಕಾಲದಲ್ಲೂ ಧರ್ಮವನ್ನು ಕೆಣಕಲಿಲ್ಲ. ಆದರೆ ಎಂದು ಪಾದ್ರಿಗಳ ಪಡೆ ಇದನ್ನು ಸ್ವಲ್ಪ ಕೆಣಕಿತೊ ಆಗ ೧೮೫೭ನೇ ಸಿಪಾಯಿ ದಂಗೆ ಯಾಯಿತು. ಎಲ್ಲಿಯವರೆವಿಗೂ ಆಂಗ್ಲೇಯರು ಅದನ್ನು ತಿಳಿದುಕೊಂಡಿರುವರೋ, ಇದನ್ನು ಪರಿಪಾಲಿಸುತ್ತಿರುವರೋ, ಅಲ್ಲಿಯವರೆವಿಗೂ ಅವರ ಸಿಂಹಾಸನಕ್ಕೆ ಭರತಖಂಡದಲ್ಲಿ ಚ್ಯುತಿಯಿಲ್ಲ. ವಿದ್ವಾಂಸರಾದ ದೂರದರ್ಶಿ ಆಂಗ್ಲೇಯರು ಇದನ್ನು ತಿಳಿದು ಕೊಂಡಿರುವರು. ಲಾರ್ಡ್​ ರಾಬರ್ಟ್​ನ “ಭಾರತದಲ್ಲಿ ೪೧ ವರುಷಗಳು” ಎಂಬ ಪುಸ್ತಕವನ್ನು ಓದಿ.

ಆ ರಾಕ್ಷಸಿಯ ಪ್ರಾಣಪಕ್ಷಿ ಎಲ್ಲಿದೆ ಎನ್ನುವುದು ಈಗ ಗೊತ್ತಾಗುವುದು. ಅದು ಧರ್ಮದಲ್ಲಿದೆ. ಏಕೆಂದರೆ ಅದನ್ನು ಯಾರೂ ನಾಶಮಾಡದೆ ಇದ್ದುದರಿಂದ, ದೇಶವು ಎಷ್ಟು ಆಪತ್ತು ವಿಪತ್ತುಗಳಲ್ಲಿ ಸಾಗಿದ್ದರೂ ಇನ್ನೂ ಉಳಿದಿರುವುದು. ಒಬ್ಬ ಭಾರತೀಯ ವಿದ್ವಾಂಸನು ರಾಷ್ಟ್ರ ಪ್ರಾಣವನ್ನು ಧರ್ಮದಲ್ಲಿ ಇಡುವ ಆವಶ್ಯ ಕತೆಯೇನು? ಉಳಿದ ದೇಶಗಳಂತೆ ಅವನ್ನು ಸಾಮಾಜಿಕ ಮತ್ತು ರಾಜಕೀಯದ ಸ್ವಾತಂತ್ರ್ಯದಲ್ಲಿ ಏತಕ್ಕೆ ಇಡಬಾರದು?ಎಂದು ಪ್ರಶ್ನಿಸುವನು. ಹೀಗೆ ಮಾತನಾಡು ವುದು ಸುಲಭ. ಕೇವಲ ತರ್ಕಕ್ಕೋಸ್ಕರ ಧರ್ಮ, ಕರ್ಮ ಮುಂತಾದುವೆಲ್ಲ ಮಿಥ್ಯ ವೆಂದು ವಾದಿಸಿದರೂ ಏನು ಆಗುವುದು ನೋಡಿ. ಉದಾಹರಣೆಗೆ ಒಂದೇ ಬೆಂಕಿಯು ಬೇರೆ ಬೇರೆ ಆಕಾರಗಳನ್ನು ತಾಳುವಂತೆ, ಒಂದೇ ಮಹಾಶಕ್ತಿಯು ಫ್ರಾನ್ಸಿ ನಲ್ಲಿ ರಾಜಕೀಯ ಸ್ವಾತಂತ್ರ್ಯದಂತೆ, ಆಂಗ್ಲೇಯರಲ್ಲಿ ವಾಣಿಜ್ಯ ವಿಸ್ತಾರದಂತೆ, ಹಿಂದೂಗಳ ಹೃದಯದಲ್ಲಿ ಮುಕ್ತಿಲಾಭೇಚ್ಚೆಯಂತೆ ರೂಪವನ್ನು ತಾಳಿರುವುದು. ಈ ಮಹಾಶಕ್ತಿಯ ಪ್ರೇರಣೆಯಿಂದ ಹಲವು ಶತಮಾನಗಳಿಂದಲೂ ನಾನಾ ಬಗೆಯ ಸುಖದುಃಖಗಳನ್ನು ಅನುಭವಿಸಿ ಫ್ರೆಂಚರ ಮತ್ತು ಅಂಗ್ಲೇಯರ ಶೀಲವು ರೂಪುಗೊಂಡಿದೆ. ಇದೇ ಪ್ರೇರಣೆಯಿಂದಲೇ ಸಾವಿರಾರು ವರ್ಷ ಗಳ ಅವಧಿಯಲ್ಲಿ ತಮ್ಮ ಅದೃಷ್ಟದ ಏಳುಬೀಳುಗಳ ನಡುವೆಯೂ ಹಿಂದೂ ಗಳು ರಾಷ್ಟ್ರೀಯ ಶೀಲವನ್ನು ರೂಪಿಸಿಕೊಂಡಿದ್ದಾರೆ. ನಾನು ಈಗ ನಿಮ್ಮನ್ನು ಗಂಭೀರವಾದ ಒಂದು ಪ್ರಶ್ನೆಯನ್ನು ಕೇಳುತ್ತೇನೆ, ಲಕ್ಷಾಂತರ ವರುಷ ಗಳಿಂದಲೂ ಬಂದ ನಮ್ಮ ಸ್ವಭಾವವನ್ನು ಬಿಡುವುದು ಸುಲಭವೊ ಅಥವಾ ಕೆಲವು ವರುಷಗಳಿಂದೀಚೆಗೆ ನಮ್ಮ ಸ್ವಭಾವಕ್ಕೆ ಕಸಿಗೊಳಿಸಿದ ವಿದೇಶೀ ಸ್ವಭಾವವನ್ನು ಬಿಡುವುದು ಸುಲಭವೊ? ಇದನ್ನು ವಿಚಾರಮಾಡಿ. ಆಂಗ್ಲೇಯ ತನ್ನ ಯೋಧ ಸ್ವಭಾವವನ್ನು ತೊರೆದು ಧರ್ಮವನ್ನೇ ತಮ್ಮ ಬದುಕಿನ ಏಕೈಕ ಗುರಿಯನ್ನಾಗಿ ಮಾಡಿಕೊಂಡು ಶಾಂತರಾಗಿ, ಧ್ಯಾನಾಸಕ್ತರಾಗಿ ಏತಕ್ಕೆ ಕುಳಿತು ಕೊಳ್ಳಬಾರದು?

ವಾಸ್ತ್ಧವ್ಧಿಕ ಅಂಶ ಇದು. ಈ ನದಿಯು ಉಗಮಸ್ಥಾನವಾದ ಬೆಟ್ಟದಿಂದ ಸಾವಿರಾರು ಮೈಲಿ ಹರಿದುಬಂದಿದೆ. ಅದನ್ನು ಪುನಃ ಬೆಟ್ಟದ ಮೇಲೆ ಕಳುಹಿಸು ವುದಕ್ಕೆ ಆಗುವುದೆ? ಹಾಗೆ ಪ್ರಯತ್ನಿಸಿದರೆ ಪರಿಣಾಮವೆ, ನೀರು ಅಲ್ಲಿ ಹರಿದು, ವ್ಯರ್ಥವಾಗಿ, ಕೊನೆಗೆ ನದಿ ಬತ್ತಿಹೋಗುವುದು. ನದಿ ಹೇಗಾದರೂ ಸಮುದ್ರವನ್ನು ಸೇರುವುದು. ಒಂದೆರಡು ದಿನ ಮುಂಚೆಯೋ ತಡವಾಗಿಯೋ ಆಗಬಹುದು. ಶುಭ್ರ ಸುಂದರ ಸ್ಥಳದಲ್ಲಿ ಹರಿದೋ ಇಲ್ಲವೇ, ಮಲಿನವಾದ ಸ್ಥಳದಲ್ಲಿ ಹರಿದೋ ಕೊನೆಗೆ ಕಡಲನ್ನು ಸೇರುವುದು. ಹತ್ತುಸಾವಿರ ವರುಷಗಳ ಕಾಲದ ನಮ್ಮ ರಾಷ್ಟ್ರೀಯ ಜೀವನವು ತಪ್ಪು ಹಾದಿಯಲ್ಲಿ ನಡೆದಿದ್ದರೂ ಈಗ ಬೇರಾವ ಉಪಾಯವೂ ಇಲ್ಲ. ನಾವು ಈ ಸಮಯದಲ್ಲಿ ಹೊಸ ಶೀಲವನ್ನು ಸೃಷ್ಟಿಸಲು ಪ್ರಯತ್ನಿಸಿದರೆ ಅದು ಮೃತ್ಯುವಿನಲ್ಲಿ ಪರ್ಯವಸಾನವಾಗುವುದು.

ದಯವಿಟ್ಟು ಕ್ಷಮಿಸಿ. ನಮ್ಮ ರಾಷ್ಟ್ರೀಯ ಆದರ್ಶವು ಸರಿಯಾಗಿರಲಿಲ್ಲ ಎಂದು ಭಾವಿಸುವುದು ತಪ್ಪು. ಮೊದಲು ಅನ್ಯದೇಶಗಳಿಗೆ ಹೋಗಿ, ನಿಮ್ಮ ಕಣ್ಣುಗಳಿಂದ ನಿಮ್ಮದೇ ದೃಷ್ಟಿಯಿಂದ ಅಲ್ಲಿ ಅವರ ರೀತಿ, ನೀತಿ, ಅಧ್ಯಯನ ಮುಂತಾದುವನ್ನು ಗಮನಿಸಿ, ಅವನ್ನೇ ಕುರಿತು ವಿಚಾರ ಮಾಡಿ. ನಂತರ ನಿಮ್ಮ ಶಾಸ್ತ್ರ, ಪುರಾಣ ಸಾಹಿತ್ಯಗಳನ್ನು ಓದಿ. ಅನಂತರ ಸಮಸ್ತ ಭಾರತ ಯಾತ್ರೆ ಮಾಡಿ. ಅಲ್ಲಿಯ ಭಿನ್ನ ಭಿನ್ನ ಜನರ ಹಲವು ಆಚಾರಗಳನ್ನು ಪರೀಕ್ಷಿಸಿ. ಬುದ್ಧಿವಂತರಂತೆ ನೋಡಿ, ದಡ್ಡರಂತೆ ಅಲ್ಲ. ಆಗ ಸ್ಪಷ್ಟವಾಗಿ ಗೊತ್ತಾಗುವುದು. ಈ ಜನಾಂಗ ಬದುಕಿದೆ, ಉಸಿರಾಡುತ್ತಿದೆ. ಕೇವಲ ನಿದ್ರಿಸುತ್ತಿದೆ. ಅಂತರಾಳದಲ್ಲಿ ಚೇತನವು ಮಿಡಿಯುತ್ತಿದೆ. ಮೇಲುನೋಟಕ್ಕೆ ಮೃತ್ಯುವಿನ ಬೂದಿಯು ಮುಚ್ಚಿಕೊಂಡಂತೆ ಕಂಡರೂ, ಅದರ ಅಡಿಯಲ್ಲಿ ರಾಷ್ಟ್ರೀಯ ಜೀವನದ ಚಿಂತೆಯು ಉರಿಯುವುದನ್ನು ನೋಡಬಹುದು. ಈ ದೇಶದ ಪ್ರಾಣ ಧರ್ಮ, ಭಾಷೆ ಧರ್ಮ, ಭಾವ ಧರ್ಮ; ನಿಮ್ಮ ರಾಜನೀತಿ, ಸಮಾಜ ನೀತಿ, ನಗರ ನಿರ್ಮನೀಕರಣ, ಪ್ಲೇಗ್​ ನಿವಾರಣೆ, ದುರ್ಭಿಕ್ಷಪೀಡಿತರಿಗೆ ಅನ್ನದಾನ ಹಿಂದಿನಿಂದಲೂ ಹೇಗೆ ಜರುಗುತ್ತಿತ್ತೋ ಹಾಗೆಯೇ ಜರುಗುವುದು, ಅದೇ ಧರ್ಮದ ಮೂಲಕ, ಅನ್ಯಥಾ ಇಲ್ಲ. ನೀವು ಸುಮ್ಮನೇ ಕೂಗಾಡಿದರೆ ಪ್ರಯೋಜನವಿಲ್ಲ.

ಎಲ್ಲಾ ದೇಶಗಳಲ್ಲಿಯೂ ನಿಯಮ ಒಂದೇ. ಯಾವುದನ್ನು ಕೆಲವು ಶಕ್ತಿವಂತರು ಮಾಡುತ್ತಾರೋ ಅದು ಆಗುತ್ತದೆ. ಉಳಿದವರು ಕುರಿಮಂದೆಯಂತೆ ಅವರನ್ನು ಅನುಸರಿಸುವರು. ನನ್ನ ಮಿತ್ರರೆ! ನಾನು ನಿಮ್ಮ ಪಾರ್ಲಿಮೆಂಟ್​, ಸೆನೇಟ್​, ಮತ ಚಲಾವಣೆ, ಬಹುಮತ, ಮತಪೆಟ್ಟಿಗೆ, ಎಲ್ಲಾ ನೋಡಿರುವೆನು. ಯಾವುದನ್ನು ಶಕ್ತಿವಂತನಾದವನು ಮಾಡುತ್ತಾನೆಯೋ ಅದನ್ನೇ ಉಳಿದವರು ಅನುಸರಿಸುವರು. ಭರತವರ್ಷದಲ್ಲಿ ಶಕ್ತಿವಂತರಾರು? ಯಾರು ಧರ್ಮವೀರರೋ ಅವರೇ ನಮ್ಮ ಸಮಾಜದ ನಾಯಕರು. ಹೊಸ ರೀತಿ ನೀತಿಗಳು ಸಮಾಜಕ್ಕೆ ಅವಶ್ಯಕವಾಗಿ ಬೇಕಾದರೆ ಅವರೇ ಅದನ್ನು ಕೊಡುವರು. ನಾವು ಅದನ್ನು ಕೇಳಿ ಅವರು ಹೇಳಿದಂತೆ ಮಾಡುವೆವು. ನಮ್ಮಲ್ಲಿರುವ ವ್ಯತ್ಯಾಸವೇನೆಂದರೆ, ಪಾಶ್ಚಾತ್ಯ ದೇಶಗಳಲ್ಲಿರುವ ಬಹುಮತ, ಓಟು, ಮತಪೆಟ್ಟಿಗೆ, ಮುಂತಾದ ಗಲಭೆ ತಿಕ್ಕಾಟಗಳು ನಮ್ಮಲ್ಲಿಲ್ಲ.

ಓಟು, ಮತಪೆಟ್ಟಿಗೆ ಮುಂತಾದವುಗಳ ಮೂಲಕ ಪಶ್ಚಿಮದ ಜನರಿಗೆ ದೊರಕುವ ಶಿಕ್ಷಣ ನಮ್ಮವರಿಗೆ ಇಲ್ಲ. ಆದರೆ ಎಲ್ಲ ಯೂರೋಪ್​ ದೇಶಗಳಲ್ಲಿ ನಡೆಯುವ, ರಾಜಕೀಯ ನೆಪದಲ್ಲಿ ಚೋರರು ಬಡವರ ರಕ್ತವನ್ನು ಹೀರಿ ತಾವು ಮಾತ್ರ ಅಭಿವೃದ್ಧಿಯಾಗುವ, ಆ ದೌರ್ಜನ್ಯ ನಮ್ಮಲ್ಲಿಲ್ಲ. ಪಾಶ್ಚಾತ್ಯ ದೇಶಗಳಲ್ಲಿ ರಾಜಕೀಯದ ಹೆಸರಿನಲ್ಲಿ ನಡೆಯುತ್ತಿರುವ ಕಪಟ, ಲಂಚಕೋರತನ, ಹಗಲು ದರೋಡೆ, ಇವನ್ನು ನೋಡಿದಾಗ ಹತಾಶರಾಗುವಿರಿ. ಅಲ್ಲಿ ಮನೆಯ ಬಾಗಿಲಿಗೇ ಹಾಲನ್ನು ತಂದರೂ ಕೊಳ್ಳುವವರಿಲ್ಲ. ಹೆಂಡದಂಗಡಿಯಲ್ಲಿ ಜನಸಂದಣಿ ಕಿಕ್ಕಿರಿದಿದೆ; ಬಡ ಪತಿವ್ರತಾ ನಾರಿಯರ ಮಾನರಕ್ಷಣೆಗೆ ಒಂದು ಚೂರು ವಸ್ತ್ರವಿಲ್ಲ, ಗಣಿಕಾ ಸ್ತ್ರೀಯರು ವೇಷಭೂಷಣಗಳಿಂದ ಮೆರೆಯುತ್ತಿರುವರು; ಹಣವಂತರು ರಾಜಶಾಸನವನ್ನು ತಮ್ಮ ಸ್ವಾಧೀನದಲ್ಲಿ ಇಟ್ಟುಕೊಂಡಿರುವರು; ಪ್ರಜೆಗಳನ್ನು ಲೂಟಿ ಮಾಡಿ ಅವರ ರಕ್ತವನ್ನು ಹೀರುತ್ತಿರುವರು; ಜನಗಳನ್ನು ಸೈನಿಕರನ್ನಾಗಿ ಮಾಡಿ ದೇಶ ದೇಶಾಂತರಗಳಿಗೆ ಸಾಯುವುದಕ್ಕೆ ಕಳುಹಿಸುವರು. ಅವರು ಗೆದ್ದರೆ ಸೋತ ರಾಷ್ಟ್ರಗಳಿಂದ ಬರುವ ಧನ ಧಾನ್ಯ ಇವರಿಗೆ ದೊರಕುವುದು. ಆದರೆ ಆಶ್ರಿತ ರಾಷ್ಟ್ರಗಳು! ಸಾಯುವುದೇ ಅವರ ಹಣೆಯ ಬರಹ. ಇದು ರಾಜಕೀಯ! ಮಿತ್ರರೆ! ನೀವು ಗಾಬರಿಯಾಗಬೇಡಿ; ಆಶ್ಚರ್ಯಚಕಿತರಾಗಬೇಡಿ.

ಈ ಒಂದು ವಿಚಾರವನ್ನು ಆಲೋಚಿಸಿ ನೋಡಿ. ವ್ಯಕ್ತಿಯು ನಿಯಮವನ್ನು ಮಾಡುವನೊ ಅಥವಾ ನಿಯಮವು ವ್ಯಕ್ತಿಯನ್ನು ಮಾಡುವುದೋ? ವ್ಯಕ್ತಿಯು ಹಣವನ್ನು ಮಾಡುವನೊ ಅಥವಾ ಹಣವು ವ್ಯಕ್ತಿಯನ್ನು ಮಾಡುವುದೋ? ವ್ಯಕ್ತಿಯು ಕೀರ್ತಿ ಗೌರವಗಳನ್ನು ಗಳಿಸುತ್ತಾನೆಯೊ ಆಥವಾ ಕೀರ್ತಿ ಗೌರವಗಳು ವ್ಯಕ್ತಿಯನ್ನು ಮಾಡುವುವೊ?

ಮಿತ್ರರೇ! ಮೊದಲು ಮಾನವರಾಗಿ, ಆಗ ಉಳಿದುದೆಲ್ಲ ತಮಗೆ ತಾವೇ ನಿಮ್ಮನ್ನು ಅನುಸರಿಸುವುದನ್ನು ನೋಡುವಿರಿ. ಪರಸ್ಪರ ದ್ವೇಷಾಸೂಯೆಗಳನ್ನು ತೊರೆಯಿರಿ, ಸದುದ್ದೇಶ, ಸತ್ಸಹಾಯ ಮತ್ತು ಒಳ್ಳೆಯ ಧೈರ್ಯಗಳನ್ನು ಅವಲಂಬಿಸಿ. ನೀವು ಪರಸ್ಪರರ ವಿರುದ್ಧ ಬೊಗಳುವುದನ್ನು ನಿಲ್ಲಿಸಿ. ಮನುಷ್ಯರಾಗಿ ಹುಟ್ಟಿದ ಮೇಲೆ ಸ್ವಲ್ಪವಾದರೂ ಕೀರ್ತಿಯನ್ನು ಇಲ್ಲಿ ಬಿಟ್ಟು ಹೋಗಿ. “ತುಲಸಿ, ಜಗತ್ತಿಗೆ ಬಂದಾಗ ಜಗತ್ತು ನಕ್ಕಿತು, ನೀನು ಅತ್ತೆ; ಹಾಗೆಯೇ ನೀನು ತೆರಳುವಾಗ ನೀನು ನಗು, ಜನ ಅಳಲಿ.” ನೀವು ಹೀಗೆ ಮಾಡಿದರೆ ಮಾತ್ರ ಮನುಷ್ಯರು. ಇಲ್ಲದೆ ಇದ್ದರೆ ನಿಮ್ಮಿಂದ ಏನು ಪ್ರಯೋಜನ?

ಮಿತ್ರರೆ, ನಿಮಗೆ ಇನ್ನೊಂದು ವಿಷಯ ಹೇಳಬೇಕೆಂದಿರುವೆನು. ನಾವು ಅನ್ಯ ಜನಾಂಗದಿಂದ ಎಷ್ಟೋ ಕಲಿಯುವುದು ನಿಜವಾಗಿ ಇರುವುದು. ಯಾರು ತಾವು ಮತ್ತೊಬ್ಬರಿಂದ ತಿಳಿದುಕೊಳ್ಳುವುದು ಏನೂ ಇಲ್ಲವೆಂದು ಭಾವಿಸುವರೆಗೂ ಅವರು ಮರಣೋನ್ಮುಖರಾಗುವರು. ಯಾವ ರಾಷ್ಟ್ರವು ತಾನು ಸರ್ವಜ್ಞ ಎಂದು ಭಾವಿಸುವುದೋ, ಅದು ಅವನತಿಗೆ ಅತಿ ನಿಕಟದಲ್ಲಿರುವುದು. ಬದುಕಿರುವವರೇ ಕಲಿತುಕೊಳ್ಳುತ್ತಲೇ ಇರುತ್ತೇವೆ. ಆದರೆ ಇದನ್ನು ನಾವು ಜ್ಞಾಪಕದಲ್ಲಿಟ್ಟುಕೊಳ್ಳ ಬೇಕು. ನೀವು ಯಾವುದನ್ನು ಕಲಿತುಕೊಳ್ಳುತ್ತೀರೋ ಅದು ನಿಮ್ಮ ರೀತಿಯಲ್ಲೇ ಆಗಿರಬೇಕು. ನಿಮ್ಮ ತಿರುಳನ್ನು ತ್ಯಜಿಸದೆ ಉಳಿದವರಿಂದ ಕಲಿತುಕೊಳ್ಳಬೇಕು. ಊಟ ಎಲ್ಲಾ ದೇಶದಲ್ಲೂ ಒಂದೇ. ಆದರೆ ನಾವು ಭೋಜನಕಾಲದಲ್ಲಿ ನೆಲದ ಮೇಲೆ ಕುಳಿತುಕೊಳ್ಳುವೆವು. ಅವರು ಕುರ್ಚಿ ಮೇಲೆ ಕುಳ್ಧಿತು ಮೇಜ್ಧಿನ ಮೇಲೆ ಭೋಜನ ಮಾಡುವರು. ನಾನು ಅವರಂತೆ ಊಟ ಮಾಡಬೇಕಾದರೆ ಬಹಳ ಹೊತ್ತು ಕುಳಿತು ಕೊಂಡು ಯಮಯಾತನೆ ಅನುಭವಿಸಬೇಕು. ಆದಕಾರಣ ಅವರ ಭೋಜನವನ್ನು ನಮ್ಮ ರೀತಿಯಲ್ಲಿ ಮಾಡುತ್ತೇವೆ. ಈ ಪ್ರಕಾರವಾಗಿಯೇ ನಾವು ವಿದೇಶೀಯರಿಂದ ಏನನ್ನಾದರೂ ಕಲಿಯಬೇಕಾದರೆ ನಮ್ಮ ವಾಸ್ತವಿಕ ಜನಾಂಗದ ತತ್ವವನ್ನು ರಕ್ಷಿಸಿಕೊಂಡು ಅವರೊಂದಿಗೆ ಅವನ್ನು ಅಭ್ಯಾಸಮಾಡಬೇಕು. ನಾನು ಇದನ್ನು ಕೇಳುತ್ತೇನೆ; ಬಟ್ಟೆಯಿಂದ ಮನುಷ್ಯನಾಗುವನೆ ಅಥವಾ ಮನುಷ್ಯ ಬಟ್ಟೆಯನ್ನು ಮಾಡುವನೆ? ಪ್ರತಿಭಾವಂತನು ಎಂತಹ ಪೋಷಾಕಿನಲ್ಲಿದ್ದರೂ ಗೌರವಕ್ಕೆ ಪಾತ್ರನಾಗುತ್ತಾನೆ, ನನ್ನಂತಹ ಮೂಢನು ಅಗಸರ ಕತ್ತೆಯಂತೆ ಎಷ್ಟು ಬಟ್ಟೆಯ ಗಂಟನ್ನು ಹೊತ್ತಿದ್ದರೂ ಯಾರೂ ಗೌರವಿಸುವುದಿಲ್ಲ.

\section{ರೀತಿನೀತಿಗಳು}

ಈ ಮುನ್ನುಡಿ ಬಹಳ ದೀರ್ಘವಾಯಿತು. ಆದರೆ ಇದಾದ ಮೇಲೆ ಎರಡು ಜನಾಂಗಗಳನ್ನು ತುಲನೆಮಾಡುವುದು ಸುಲಭ. ಅವರು ಯೋಗ್ಯರೆ, ನಾವು ಯೋಗ್ಯರೆ, ಯಾರನ್ನು ಹೊಗಳುವುದು, ಯಾರನ್ನು ತೆಗಳುವುದು. ತ್ರಾಸಿನ ಎರಡು ಕಡೆಯೂ ಸಮನಾಗಿದೆ. ಆದರೆ ಒಳ್ಳೆಯ ಗುಣದಲ್ಲಿ ತಾರತಮ್ಯ ಮತ್ತು ಭಿನ್ನತೆ ಇವೆ, ಅಷ್ಟೆ.

ನಮ್ಮ ವಿಚಾರಗಳ ಪ್ರಕಾರ ಮನುಷ್ಯನು ಮೂರು ವಸ್ತುಗಳ ಸಂಘಟನೆ ಯಿಂದ ಆಗಿರುವನು: ಶರೀರ, ಮನಸ್ಸು, ಆತ್ಮ. ಮೊದಲು ಶರೀರದ ವಿಷಯ ವನ್ನು ತೆಗೆದುಕೊಳ್ಳೋಣ. ಅದೇ ಮನುಷ್ಯನಲ್ಲಿರುವ ಅತ್ಯಂತ ಹೊರಗಿನ ವಸ್ತು.

ನೋಡಿ; ಶರೀರದಲ್ಲಿ ಎಷ್ಟೊಂದು ಭೇದವಿದೆ; ಮುಖ, ಕೂದಲು, ಎತ್ತರ, ಬಣ್ಣ ಮುಂತಾದವುಗಳಲ್ಲಿ ಎಷ್ಟು ವ್ಯತ್ಯಾಸವಿದೆ.

ಆಧುನಿಕ ತಜ್ಞರ ಅಭಿಪ್ರಾಯದಲ್ಲಿ ಬಣ್ಣದ ವ್ಯತ್ಯಾಸಕ್ಕೆ ಕಾರಣ ವರ್ಣಸಂಕರ. ಶೀತ ಮತ್ತು ಉಷ್ಣದೇಶದ ಹವಾಗುಣದಿಂದ ಸ್ವಲ್ಪ ವ್ಯತ್ಯಾಸವೇನೋ ಆಗುವುದು. ಆದರೆ ಬಣ್ಣಕ್ಕೆ ಮುಖ್ಯ ಕಾರಣ ಪಿತೃಗಳು. ಅತಿ ಶೀತ ದೇಶದಲ್ಲಿಯೂ ಕಪ್ಪು ಬಣ್ಣದವರಿರುವರು. ಅತಿ ಉಷ್ಣ ದೇಶದಲ್ಲಿಯೂ ಗೌರವರ್ಣದವರಿರುವರು. ಕೆನಡಾ ದೇಶದ ಆದಿವಾಸಿಗಳು ಮತ್ತು ಉತ್ತರ ಧ್ರುವದ ಸಮೀಪದಲ್ಲಿ ವಾಸಿಸುವ ಎಸ್ಕಿಮೋ ಜನರು ಅಷ್ಟೇನೂ ಬಿಳುಪಲ್ಲ. ಆದರೆ ವಿಷುವದ್ರೇಕೆಗೆ ಸಮೀಪ ದಲ್ಲಿರುವ ಬೋರ್ನಿಯೋ, ಸಿಲೆಬಿಸ್​ ದ್ವೀಪವಾಸಿಗಳ ಬಣ್ಣ ಬಿಳುಪು.

ಹಿಂದೂ ಶಾಸ್ತ್ರರೀತ್ಯಾ, ಹಿಂದೂಗಳಲ್ಲಿರುವ ಬ್ರಾಹ್ಮಣ, ಕ್ಷತ್ರಿಯ, ವೈಶ್ಯರು, ಮತ್ತು ಹೊರಗಿನ ಚೀನ, ಹೂಣ, ದರದ, ಪಹಲಾವ, ಯವನ, ಶಕ ಇವರೆಲ್ಲ ಆರ್ಯರು. ಶಾಸ್ತ್ರದಲ್ಲಿ ಬರುವ ಚೀನಾ ಜನಾಂಗ ವರ್ತಮಾನ ಚೀಣೀಯರಲ್ಲ. ಅದೂ ಅಲ್ಲದೆ ಆಗಿನ ಕಾಲದಲ್ಲಿ ಅವರು ಚೀಣೀಯರು ಎಂದು ಹೇಳಿಕೊಳ್ಳು ತ್ತಿರಲಿಲ್ಲ. ಈಗಿನ ಕಾಶ್ಮೀರಕ್ಕೆ ಈಶಾನ್ಯದಲ್ಲಿ ಆಗ ಚೀಣೀ ಹೆಸರಿನ ಒಂದು ದೊಡ್ಡ ಜನಾಂಗವಿತ್ತು. ಯಾವುದನ್ನು ದರದಾ ಎಂದು ಹೇಳುತ್ತೇವೆಯೋ ಅದು ಭಾರತ ಮತ್ತು ಆಫ್ಘಾನಿಸ್ತಾನದ ಮಧ್ಯೆ ಇದ್ದ ಒಂದು ಗುಡ್ಡದ ಜನಾಂಗ. ಅದು ಈಗಲೂ ಇದೆ. ಪ್ರಾಚೀನ ಚೀಣಾ ಜನಾಂಗದ ಕೆಲವರು ಇನ್ನೂ ಇರುವರು. ದರದ್​ಸ್ಥಾನ ಈಗಲೂ ಇದೆ. ರಾಜತರಂಗಿಣಿ ಎಂಬ ಕಾಶ್ಮೀರದ ಇತಿಹಾಸದಲ್ಲಿ ಬಾರಿ ಬಾರಿಗೆ ದರದರಾಜನ ಪ್ರಭುತ್ವ ಪರಿಚಯವಾಗುವುದು. ಹೂಣರೆಂಬ ಪ್ರಾಚೀನ ಜನಾಂಗ ಬಹುಕಾಲದ ಹಿಂದೆ ಭಾರತವರ್ಷದ ವಾಯವ್ಯದಲ್ಲಿ ರಾಜ್ಯವಾಳುತ್ತಿತ್ತು. ಈಗ ತಿಬೆಟ್ಟಿನವರು ಹೂಣರೆಂದು ಕರೆದುಕೊಳ್ಳುವರು. ಆದರೆ ನಿಜವಾಗಿ ನೋಡಿದರೆ ಅವರು ಹ್ಯೂನರು.

ಮನು ಉಲ್ಲೇಖಿಸಿರುವ ಹೂಣರು ಆಧುನಿಕ ತಿಬೆಟ್ಟಿನವರಲ್ಲ. ಆಧುನಿಕ ತಿಬೆಟ್ಟಿನವರು ಪೂರ್ವದ ಆರ್ಯಹೂಣರು ಮತ್ತು ಮಧ್ಯ ಏಷ್ಯಾದಿಂದ ಬಂದ ಒಂದು ಮೊಗಲ್​ ಜಾತಿಯ ಸಂಮಿಶ್ರಣದಿಂದ ಆಗಿರಬಹುದು.

ಪ್ರೆಜ್ವಲ್ಸ್ಕಿ ಮತ್ತು ಡ್ಯೂಕ್​ಡಿ ಆರ್ಲೆಯನ್ಸ್​ ಎಂಬ ರಷ್ಯನ್​ ಮತ್ತು ಫ್ರೆಂಚ್​ ಪ್ರವಾಸಿಗಳ ಮತದಲ್ಲಿ, ತಿಬೆಟ್ಟಿನಲ್ಲಿ ಈಗಲೂ ಕೂಡ ಮುಖಚರ್ಯೆ ಮತ್ತು ಕಣ್ಣುಗಳಲ್ಲಿ ಆರ್ಯರನ್ನು ಹೋಲುವ ಜನರಿರುವರು. ಗ್ರೀಕರನ್ನು ಯವನರೆಂದು ಕರೆಯುತ್ತಿದ್ದರು. ಈ ಯವನರೆಂಬ ಹೆಸರಿನ ಮೇಲೆ ಹಲವು ವಾದವಿವಾದ ಎದ್ದಿವೆ. ಕೆಲವರು ಅಯೋನಿಯ ಎಂಬ ಪ್ರದೇಶದಲ್ಲಿ ವಾಸಿಸುತ್ತಿದ್ದ ಗ್ರೀಕರಿಗೆ ಯವನರೆಂಬ ಹೆಸರು ಎನ್ನುವರು. ಅಶೋಕನ ಪಾಳಿಭಾಷೆಯ ಶಿಲಾಶಾಸನಗಳು ಅವರನ್ನು ಯೋನರೆಂದು ಕರೆಯುವುವು. ಈ ಯೋನ ಎಂಬ ಪದದಿಂದ ಸಂಸ್ಕೃತದ ‘ಯವನ’ ಬಂದಿತೆಂದು ಹೇಳುವರು. ಕೆಲವು ಭಾರತೀಯ ವಿದ್ವಾಂಸರ ಅಭಿಪ್ರಾಯದಲ್ಲಿ ಯವನವೆಂಬುದು ಗ್ರೀಕರಿಗೆ ಅನ್ವಯಿಸುವುದಿಲ್ಲ. ಆದರೆ ಈ ಅಭಿಪ್ರಾಯಗಳೆಲ್ಲ ತಪ್ಪು. ಯವನವೇ ಆದಿಶಬ್ದ. ಕೇವಲ ಹಿಂದೂಗಳು ಮಾತ್ರ ಯವನರೆಂದು ಕರೆಯುತ್ತಿರಲಿಲ್ಲ. ಪುರಾತನ ಈಜಿಪ್ಟ್​ ಮತ್ತು ಬ್ಯಾಬಿ ಲೋನಿಯಾದವರೂ ಕೂಡ ಗ್ರೀಕರನ್ನು ಯವನರೆಂದೇ ಕರೆದರು. ಪಹಲ್ವಿ ಎಂದರೆ ಪ್ರಾಚೀನ ಪರ್ಷಿಯನರು. ಅವರು ಪಹಲ್ವಿ ಭಾಷೆಯನ್ನು ಆಡುತ್ತಿದ್ದರು. ಖಶ್​ ಎಂದರೆ ಹಿಮಾಲಯ ಪರ್ವತದ ಸಮೀಪದಲ್ಲಿ ವಾಸಿಸುತ್ತಿದ್ದ ಅರೆನಾಗರಿಕ ಆರ್ಯ ಜನಾಂಗ. ಈಗಲೂ ಅವರಿಗೆ ಅದೇ ಹೆಸರು. ಈ ಪ್ರಕಾರ ವರ್ತಮಾನ ಯೂರೋಪಿಯನ್ನರು ಖಶ್​ ವಂಶಜರು. ಪೂರ್ವದಲ್ಲಿ ಅನಾಗರಿಕರಾದ ಆರ್ಯರೆಲ್ಲ ಖಶ್​ ಜನರು.

ಆಧುನಿಕ ಪಂಡಿತರ ಮತದಲ್ಲಿ ಆರ್ಯರು ಗುಲಾಬಿ ಬಣ್ಣದವರು, ಕಪ್ಪು ಅಥವಾ ಕೆಂಪು ಬಣ್ಣದ ಕೂದಲು, ನೇರವಾದ ಮೂಗು ಮತ್ತು ವಿಶಾಲವಾದ ಕಣ್ಣುಗಳುಳ್ಳವರು. ಕಪಾಲದ ರಚನೆಯು ಕೂದಲ ಬಣ್ಣಕ್ಕೆ ಅನುಗುಣವಾಗಿ ಸ್ವಲ್ಪ ವ್ಯತ್ಯಾಸವಾಗುತ್ತಿತ್ತು. ಉಳಿದ ಕಪ್ಪು ಬಣ್ಣದವರೊಂದಿಗೆ ಸಂಮಿಶ್ರಣವಾದ ಮೇಲೆ ಅವರ ಬಣ್ಣವೂ ಕಪ್ಪಾಗುತ್ತಾ ಬಂದಿತು. ಇವರ ಅಭಿಪ್ರಾಯದಲ್ಲಿ ಹಿಮಾಲಯಕ್ಕೆ ಪಶ್ಚಿಮದಲ್ಲಿರುವ ಕೆಲವರು ಮಾತ್ರ ಶುದ್ಧ ಆರ್ಯರು, ಉಳಿದವರೆಲ್ಲರೂ ಮಿಶ್ರ ವಾಗಿ ಹೋಗಿರುವರು. ಇಲ್ಲದೆ ಇದ್ದರೆ ಅವರು ಹೇಗೆ ಕಪ್ಪಾಗಬೇಕು? ಯುರೋಪಿನ ವಿದ್ವಾಂಸರು ಇಷ್ಟು ಹೊತ್ತಿಗೆ ಇದನ್ನು ತಿಳಿದುಕೊಂಡಿರಬೇಕು: ದಕ್ಷಿಣದಲ್ಲಿ ಅನೇಕ ಮಕ್ಕಳಿಗೆ ಹುಟ್ಟುವಾಗ ಕೆಂಪು ಕೂದಲಿದ್ದು ಬರುಬರುತ್ತಾ ಎರಡು ಮೂರು ವರ್ಷದಲ್ಲಿ ಕಪ್ಪಾಗುವುದು ಮತ್ತು ಹಿಮಾಲಯದಲ್ಲಿ ಅನೇಕರಿಗೆ ಕೆಂಪುಕೂದಲು, ನೀಲಿ ಅಥವಾ ಕಂದುಬಣ್ಣದ ಕಣ್ಣುಗಳಿವೆ.

ಪಂಡಿತರು ಈ ವಿಷಯದಲ್ಲಿ ಬೇಕಾದಷ್ಟು ಚರ್ಚಿಸಲಿ. ಹಿಂದೂಗಳು ಹಿಂದಿನಿಂದಲೂ ಆರ್ಯರೆಂದು ಕರೆಸಿಕೊಳ್ಳುತ್ತಿರುವರು. ಶುದ್ಧ ಅಥವಾ ಮಿಶ್ರಿತ ರಕ್ತದವರಾಗಲಿ ಹಿಂದೂಗಳು ಆರ್ಯರು. ಯೂರೋಪಿಯನ್ನರಿಗೆ ನಾವು ಕಪ್ಪಾಗಿರುವುದರಿಂದ ಹಾಗೆ ಕರೆಯಲು ಇಚ್ಚೆ ಇಲ್ಲದೆ ಇದ್ದ್ಧರೆ ಅವರು ಬೇರೆ ಹೆಸರನ್ನು ಇಟ್ಟುಕೊಳ್ಳಲಿ. ಅದರಿಂದ ನಮಗೇನು?

ಬೆಳ್ಳಗಿರಲಿ, ಕಪ್ಪಗಿರಲಿ ಪ್ರಪಂಚದಲ್ಲಿ ಎಲ್ಲರಿಗಿಂತ ಹಿಂದೂಗಳು ಸುಂದರಾ ಕೃತಿಯುಳ್ಳವರು. ನಾನು ನನ್ನ ದೇಶದವರ ಬಗ್ಗೆ ಜಂಬಕೊಚ್ಚಿಕೊಳ್ಳುತ್ತಿಲ್ಲ. ಇದು ಜಗತ್​ಪ್ರಸಿದ್ಧವಾಗಿದೆ. ಈ ದೇಶದಲ್ಲಿ ಶೇಕಡಾವಾರು ಸ್ತ್ರೀಪುರುಷರಲ್ಲಿ ಸುಂದರವಾಗಿರುವಷ್ಟು ಮತ್ತಾವ ದೇಶದಲ್ಲಿ ಇರುವರು? ಇದನ್ನು ವಿಚಾರಮಾಡಿ ನೋಡಿ, ಉಳಿದ ದೇಶಗಳಲ್ಲಿ ಸುಂದರವಾಗಿ ಕಾಣಬೇಕಾದರೆ ಎಷ್ಟೊಂದು ಸಲಕರಣೆ ಬೇಕು. ಆದರೆ ಭರತಖಂಡದಲ್ಲಿ ಅತ್ಯಲ್ಪ ಸಾಕು. ಏಕೆಂದರೆ ನಮ್ಮ ಶರೀರದ ಅಧಿಕಾಂಶ ಹೊರಗೆ ಕಾಣುತ್ತಿರುವುದು. ಉಳಿದ ದೇಶಗಳಲ್ಲಿ ಹೇಗಾದರೂ ಬಟ್ಟೆಯ ಸಹಾಯದಿಂದ ಕುರೂಪಿಗಳೂ ಸುಂದರವಾಗಿ ಕಾಣಲು ಪ್ರಯತ್ನಿಸುವರು.

ಆರೋಗ್ಯ ದೃಷ್ಟಿಯಿಂದ ಪಾಶ್ಚಾತ್ಯರು ನಮಗಿಂತ ಎಷ್ಟೋ ಮೇಲು. ಅಲ್ಲಿ ನಲವತ್ತು ವರುಷದ ಪುರುಷ, ಐವತ್ತು ವರುಷದ ಸ್ತ್ರೀ ಇನ್ನೂ ಯುವಕರು. ನಿಜ, ಏಕೆಂದರೆ ಅವರು ಚೆನ್ನಾಗಿ ತಿನ್ನುತ್ತಾರೆ, ಬಟ್ಟೆಬರೆ ಧರಿಸುತ್ತಾರೆ. ಶುಭ್ರ ವಾದ ಸ್ಥಳದಲ್ಲಿ ವಾಸಿಸುತ್ತಾರೆ. ಇದೆಲ್ಲಕಿಂತ ಹೆಚ್ಚಾದುದೆಂದರೆ ಅವರಲ್ಲಿ ಬಾಲ್ಯ ವಿವಾಹವಿಲ್ಲ. ನಮ್ಮ ದೇಶದಲ್ಲೂ ಕೆಲವು ಬಲಾಢ್ಯ ಜಾತಿಗಳಿವೆ. ಗೂರ್ಖಾ, ಪಂಜಾಬಿ, ಜಾಟ್​, ಆಫ್ರಿದಿ ಮುಂತಾದವರಲ್ಲಿ ಮದುವೆಯ ವಯಸ್ಸು ಎಷ್ಟು ಎಂಬುದನ್ನು ವಿಚಾರಿಸಿ. ನಂತರ ಶಾಸ್ತ್ರವನ್ನು ಓದಿ. ಬ್ರಾಹ್ಮಣನಿಗೆ ಮೂವತ್ತು ವರ್ಷಕ್ಕೆ ಮದುವೆ, ಕ್ಷತ್ರಿಯನಿಗೆ ಇಪ್ಪತ್ತೈದು, ವೈಶ್ಯನಿಗೆ ಇಪ್ಪತ್ತಕ್ಕೆ ಎಂದು ಹೇಳಿದೆ. ಪಾಶ್ಚಾತ್ಯರಿಗೂ ನಮಗೂ ಆಯುಸ್ಸು, ಬಲ, ವೀರ್ಯದಲ್ಲಿ ಎಷ್ಟೋ ವ್ಯತ್ಯಾಸಗಳಿವೆ. ನಾವು ನಲ್ವತ್ತು ವರ್ಷ ಮುಟ್ಟುವುದರಲ್ಲೆ ನಮ್ಮ ಬಲ ಬುದ್ಧಿಯಲ್ಲಿ ಅವನತಾವಸ್ಥೆಗೆ ಇಳಿದಿರುವೆವು. ಆದರೆ ಪಾಶ್ಚಾತ್ಯರಲ್ಲಿ ಅದೇ ವಯಸ್ಸಿನಲ್ಲಿ ಅವರು ಜೀವನವನ್ನು ಆಗತಾನೆ ಪ್ರಾರಂಭಿಸಿರುವರು.

ನಾವು ಶಾಕಾಹಾರಿಗಳು, ನಮ್ಮ ರೋಗಗಳಲ್ಲಿ ಅಧಿಕಾಂಶ ಜೀರ್ಣಕೋಶಕ್ಕೆ ಅನ್ವಯಿಸಿರುವುದು. ನಮ್ಮ ಮುದುಕ ಮುದುಕಿಯರು ಮುಕ್ಕಾಲು ಪಾಲು ಹೊಟ್ಟೆಯ ಜಾಡ್ಯದಿಂದ ಸಾಯುವರು. ಪಾಶ್ಚಾತ್ಯರು ಮಾಂಸಾಹಾರಿಗಳು. ಅವರು ರೋಗದಲ್ಲಿ ಬಹುಪಾಲು ಹೃದಯಕ್ಕೆ ಸಂಬಂಧಪಟ್ಟಿರುವುದು. ಪಾಶ್ಚಾತ್ಯರಲ್ಲಿ ಬಹುಪಾಲು ವೃದ್ಧರು ಹೃದಯ ಮತ್ತು ಶ್ವಾಸಕೋಶಕ್ಕೆ ಸಂಬಂಧಿಸಿದ ಜಾಡ್ಯದಿಂದ ಮೃತ ರಾಗುವರು. ಒಬ್ಬ ಪಾಶ್ಚಾತ್ಯ ವಿದ್ಯಾವಂತ ಡಾಕ್ಟರ್​ ಹೀಗೆ ವಿಚಾರಮಾಡಿರುವರು: ಜೀರ್ಣಕೋಶಕ್ಕೆ ಸಂಬಂಧಪಟ್ಟ ರೋಗಿಗಳು ನಿರುತ್ಸಾಹಿಗಳಾಗಿ ವೈರಾಗ್ಯದ ಕಡೆ ತಿರುಗುತ್ತಾರೆ; ಹೃದಯ ಮತ್ತು ಅದಕ್ಕೆ ಮೇಲಿರುವ ದೇಹದ ಭಾಗದ ಖಾಯಿಲೆಗೆ ತುತ್ತಾದವರಲ್ಲಿ ಕೊನೆಯ ತನಕ ಆಸೆ ಮತ್ತು ವಿಶ್ವಾಸವಿರುವುದು. ಕಾಲರಾ ರೋಗಿ ಪ್ರಾರಂಭದಲ್ಲೆ ಮೃತ್ಯುವಿಗೆ ನಡುಗುತ್ತಾನೆ. ಕ್ಷಯರೋಗಿ ಕೊನೆಯವರೆಗೆ ತಾನು ಗುಣಮುಖಿಯಾಗುವನೆಂಬ ವಿಶ್ವಾಸವನ್ನು ಇಟ್ಟುಕೊಂಡಿರುವನು. ಈ ಕಾರಣ ದಿಂದಲೆ ಭಾರತೀಯರು ಯಾವಾಗಲೂ ಮೃತ್ಯುಚಿಂತನೆ ಮತ್ತು ವೈರಾಗ್ಯದ ಭಾವವನ್ನು ಮೆಲುಕು ಹಾಕುವುದು! ಇದಕ್ಕೆ ಸರಿಯಾದ ಉತ್ತರ ನನಗೆ ಇದುವರೆಗೂ ದೊರೆತಿಲ್ಲ, ಇದು ಆವಶ್ಯಕ ವಿಚಾರಣೀಯ ವಿಷಯ.

ನಮ್ಮ ದೇಶದಲ್ಲಿ ಕೂದಲು ಹಲ್ಲಿಗೆ ಸಂಬಂಧಿಸಿದ ಖಾಯಿಲೆ ಕಡಿಮೆ. ಪಾಶ್ಚಾತ್ಯರಲ್ಲಿ ಸ್ವಾಭಾವಿಕವಾಗಿ ಒಳ್ಳೆಯ ಹಲ್ಲು ಇರುವುದು ಬಹಳ ಕಡಿಮೆ. ಎಲ್ಲಿ ನೋಡಿದರೂ ಬೋಳು ತಲೆ ಹೆಚ್ಚು. ನಮ್ಮ ಸ್ತ್ರೀಯರು ಒಡವೆ ಧರಿಸುವುದಕ್ಕಾಗಿ ಮೂಗು ಕಿವಿಗಳನ್ನು ಚುಚ್ಚಿಸಿಕೊಳ್ಳುತ್ತಿದ್ದರು. ಪಾಶ್ಚಾತ್ಯರಲ್ಲಿ ಈಗ ಹೆಚ್ಚು ಮಂದಿ ಇದನ್ನು ಮಾಡುವುದಿಲ್ಲ. ಆದರೆ ಅವರು ಸೊಂಟವನ್ನು ಬಿಗಿದು ವಿಕಾರಮಾಡಿ, ಪಿತ್ತಕೋಶ ಯಕೃತ್​ಗಳಗೆ ಅಡತಣೆ ತಂದು, ಚೆನ್ನಾಗಿ ಕಾಣಬೇಕೆಂದು ಯಮ ಯಾತನೆ ಅನುಭವಿಸುವರು. ಜೊತೆಗೆ ಚನ್ನಾಗಿ ಕಾಣಬೇಕೆಂದು ಹೊತ್ತಿರುವ ಬಟ್ಟೆಯ ಭಾರ ಬೇರೆ. ಪಾಶ್ಚಾತ್ಯ ಪೋಷಾಕು ಕೆಲಸ ಮಾಡುವುದಕ್ಕೆ ಅನುಕೂಲ. ಕೆಲವು ಧನಿಕ ಸ್ತ್ರೀಯರ ಉಡುಪನ್ನು ಬಿಟ್ಟರೆ ಉಳಿದವರ ಪೋಷಾಕು ಅಷ್ಟು ಚೆನ್ನಾಗಿಲ್ಲ. ನಮ್ಮ ಸ್ತ್ರೀಯರ ಸೀರೆ, ಪುರುಷರ ಪಂಚೆ, ಉತ್ತರೀಯ, ಪೇಟಕ್ಕೆ ಸರಿ ಸಮಾನವಾದುದು ಪೃಥ್ವಿಯಲ್ಲೇ ಇಲ್ಲ. ಸ್ವಾಭಾವಿಕವಾಗಿ ನೆರಿನೆರಿಯಂತೆ ಬೀಳುವ ಪೋಷಾಕಿನ ಸೌಂದರ್ಯದ ಮುಂದೆ ಬಿಗಿಯಾದ ಬಟ್ಟೆ ನಿಲ್ಲಲಾರದು. ನಮ್ಮ ಬಟ್ಟೆಬರೆ ನೆರಿಗೆಗಳಿಂದ ಕೂಡಿರುವುದರಿಂದ ಕೆಲಸಕ್ಕೆ ಸರಿಯಲ್ಲ. ಕೆಲಸ ಮಾಡುವಾಗ ಅವೆಲ್ಲ ಹಾಳಾಗುತ್ತವೆ. ಅವರ ಷೋಕಿ ಬಟ್ಟೆಯಲ್ಲಿ, ನಮ್ಮ ಷೋಕಿ ಒಡವೆಯಲ್ಲಿ ಹೆಚ್ಚು. ಈಗ ನಮ್ಮ ಷೋಕಿ ಸ್ವಲ್ಪ ಬಟ್ಟೆಯ ಕಡೆಗೂ ಹೋಗಿದೆ. ಸ್ತ್ರೀಯರ ಪೋಷಾಕಿನ ಸೌಂದರ್ಯದ ಕೇಂದ್ರ ಪ್ಯಾರಿಸ್​, ಪುರುಷರಿಗೆ ಲಂಡನ್​. ಪ್ಯಾರಿಸ್ಸಿನ ನರ್ತಕಿಯರು ಹೊಸ ಹೊಸ ಷೋಕಿಯನ್ನು ಧರಿಸಿದರೆ ಅದನ್ನು ಅನುಸರಿಸುವುದಕ್ಕೆ ಎಲ್ಲರೂ ಧಾವಿಸುವರು. ಆಗ ಪ್ರಖ್ಯಾತ ಬಟ್ಟೆ ವ್ಯಾಪಾರಿಗಳು ಹೊಸ ಪೋಷಾಕನ್ನು ಪ್ರಚಾರಕ್ಕೆ ತರುವರು. ಇದಕ್ಕಾಗಿ ಪ್ರತಿವರ್ಷವೂ ಎಷ್ಟು ಕೋಟಿ ರೂಪಾಯಿಗಳನ್ನು ಅವರು ಖರ್ಚುಮಾಡುವರೊ ಗೊತ್ತಿಲ್ಲ. ನವ ಪೋಷಾಕನ್ನು ಸೃಷ್ಟಿಸುವುದು ಅವರಿಗೊಂದು ಕಲೆಯಾಗಿದೆ. ಸ್ತ್ರೀಯ ಬಣ್ಣ ಮತ್ತು ಕೂದಲಿಗೆ ಯಾವ ಪೋಷಾಕು ಹೋಲುತ್ತದೆ, ಆಕೆಯ ಶರೀರದಲ್ಲಿ ಯಾವ ಭಾಗವನ್ನು ಮುಚ್ಚಬೇಕು, ಯಾವುದನ್ನು ಪ್ರಕಾಶಿಸಬೇಕು, ಇದರಲ್ಲಿ ವಿಚಾರವಾಗಿ ಬೇಕಾದಷ್ಟು ಸಂಶೋಧನೆ ಮಾಡಿ ಪೋಷಾಕನ್ನು ಹೊಲಿಯುವರು. ಉಚ್ಚ ಶ್ರೇಣಿಯ ಕೆಲವು ಸ್ತ್ರೀಯರು ಯಾವ ಉಡುಪನ್ನು ಧರಿಸುತ್ತಾರೆಯೋ, ಉಳಿದವರೂ ಅದನ್ನೇ ಧರಿಸಬೇಕು. ಇಲ್ಲದೇ ಹೋದರೆ ಅವರು ಜಾತಿ ಭ್ರಷ್ಟ ರಾಗುವರು! ಇದೇ ಫ್ಯಾಷನ್​! ಈ ಫ್ಯಾಷನ್​ ಕೂಡ ಕಾಲ ಕಾಲಕ್ಕೆ ಬದಲಾಯಿಸು ತ್ತಿರುವುದು. ವರ್ಷದ ನಾಲ್ಕು ಋತುಗಳಲ್ಲಿ ನಾಲ್ಕು ಸಾರಿಯಾದರೂ ಬದಲಾಯಿಸುವುದು. ಯಾರು ಶ‍್ರೀಮಂತರೋ ಅವರು ದೊಡ್ಡ ದೊಡ್ಡ ಅಂಗಡಿಗಳಲ್ಲಿ ಪೋಷಾಕನ್ನು ಹೊಲಿಸುವರು. ಯಾರು ಮಧ್ಯಮ ದರ್ಜೆಯಲ್ಲಿ ರುವರೋ ಅವರು ತಮ್ಮ ಮನೆಯಲ್ಲಿಯೇ ಮಹಿಳೆಯರಿಂದ ಹೊಲಿಸುವರು; ಅಥವಾ ತಾವೇ ಹೊಲಿದುಕೊಳ್ಳುವರು. ಹೊಸ ಫ್ಯಾಷನ್ನು ತಮ್ಮ ಹಿಂದಿನದರ ಹತ್ತಿರ ಹತ್ತಿರ ಬಂದರೆ ತಾವೇ ಈ ಹಿಂದಿನ ಉಡುಪನ್ನು ಮಾರ್ಪಡಿಸಿಕೊಳ್ಳುವರು. ಇದು ಸಾಧ್ಯವಿಲ್ಲದೇ ಇದ್ದರೆ ಹೊಸದನ್ನು ಕೊಳ್ಳುವರು. ಶ‍್ರೀಮಂತರು ತಮ್ಮ ಹಳೆಯ ಪೋಷಾಕನ್ನು ಆಶ್ರಿತರಿಗೆ ಮತ್ತು ನೌಕರರಿಗೆ ಕೊಡುವರು, ಮಧ್ಯಮ ತರಗತಿಯವರಾದರೋ ಅದನ್ನು ಮಾರಿಬಿಡುವರು. ಈ ಮಾದರಿ ಬಟ್ಟೆಯು ಯೂರೋಪಿನ ಆಶ್ರಿತ ರಾಷ್ಟ್ರಗಳಾದ ಆಫ್ರಿಕ, ಏಷ್ಯ, ಆಸ್ಟ್ರೇಲಿಯಾ ಮುಂತಾದ ದೇಶಗಳಿಗೆ ಬಿಕರಿಗೆ ಹೋಗುವುವು. ಅತಿ ಶ‍್ರೀಮಂತರು ಪ್ಯಾರಿಸ್ಸಿ ನಿಂದ ತಮ್ಮ ಪೋಷಾಕನ್ನು ತರಿಸುವರು. ಉಳಿದವರು ಅದರಂತೆ ತಮ್ಮ ದೇಶದಲ್ಲೇ ಹೊಲಿಸುವರು. ಆದರೆ ಸ್ತ್ರೀಯರ ಟೋಪಿ ಮಾತ್ರ ಫ್ರಾನ್ಸಿನಿಂದಲೇ ಬರಬೇಕು. ಆಂಗ್ಲೇಯ ಮತ್ತು ಜರ್ಮನ್​ ಸ್ತ್ರೀಯರ ಪೋಷಾಕು ಅಷ್ಟು ಚೆನ್ನಾಗಿಲ್ಲ. ಅವರಲ್ಲಿ ಕೆಲವು ಶ‍್ರೀಮಂತರನ್ನು ಬಿಟ್ಟರೆ ಉಳಿದವರು ಪ್ಯಾರಿಸ್​ ಫ್ಯಾಷನ್ನನ್ನು ಅನುಕರಿಸುವುದಿಲ್ಲ. ಅದಕ್ಕೇ ಉಳಿದ ದೇಶದ ಸ್ತ್ರೀಯರು ಇವರನ್ನು ನೋಡಿ ನಗುವರು. ಬಹುಪಾಲು ಆಂಗ್ಲೇಯ ಪುರುಷರು ಉತ್ತಮ ಪೋಷಾಕನ್ನು ಧರಿಸುವರು. ಅಮೇರಿಕಾ ಸ್ತ್ರೀ ಪುರುಷರು ಯಾವ ಭೇದವೂ ಇಲ್ಲದೆ ಸುಂದರ ಪೋಷ್ಧಾಕ್ಧನ್ನು ಧರ್ಧಿಸ್ಧುವ್ಧರು. ಅಮ್ಧೇರ್ಧಿಕಾ ಸರ್ಧಕ್ಧಾರ ಆಮ್ಧದ್ಧಾಗಿ ಬರ್ಧುವ ಬಟ್ಟ್ಧೆಯ್ಧಮೇಲೆ ಬೇಕ್ಧಾದ್ಧಷ್ಟು ಸುಂಕ್ಧವ್ಧನ್ನು ಹೊರ್ಧಿಸ್ಧಿದ್ಧರೂ ಅಮ್ಧೇರ್ಧಿಕಾ ಸ್ತ್ರೀಯ್ಧ್ಧರು ಪ್ಯಾರ್ಧಿಸ್ಸ್ಧಿನ ಪೋಷಾಕನ್ನೂ, ಪುರುಷರು ಲಂಡನ್ನಿನ ಪೋಷಾಕನ್ನೂ ಕೊಳ್ಳುವರು. ಹಲವು ಬಗೆಯ ಬಣ್ಣದ ಉಣ್ಣೆಯ ಮತ್ತು ರೇಷ್ಮೆಯ ಬಟ್ಟೆಯನ್ನು ಜಾರಿಗೆ ತರುತ್ತಿರುವರು. ಸಾವಿರಾರು ಜನ ಹೊಸ ಹೊಸ ಪೋಷಾಕನ್ನು ತಯಾರಿಸುತ್ತಿರುವರು, ಕಾರ್ಮಿಕರನ್ನು ಕೆಲಸಕ್ಕೆ ತೆಗೆದುಕೊಳ್ಳುತ್ತಿರುವರು. ಅವರ ಪೋಷಾಕು ಅತಿ ಆಧುನಿಕವಲ್ಲದಿದ್ದರೆ ಜನರು ಆಡಿಕೊಳ್ಳುವರು! ನಮ್ಮಲ್ಲಿ ಪೋಷಾಕಿನ ಫ್ಯಾಷನ್​ ಇಲ್ಲದೇ ಇದ್ದರೂ ಒಡವೆಯ ಫ್ಯಾಷನ್​ ಇದೆ. ಪಶ್ಚಿಮದಲ್ಲಿ ರೇಶ್ಮೆ, ಉಣ್ಣೆ, ಅರಳೆಬಟ್ಟೆಯ ವ್ಯಾಪಾರಸ್ಥರು ಹಗಲು ರಾತ್ರಿ ಜನರ ಪೋಷಾಕು ಯಾವ ರೀತಿ ಮಾರ್ಪಾಡಾಗುತ್ತಿದೆ, ಯಾವುದನ್ನು ಮೆಚ್ಚುವರು ಎಂಬುದನ್ನು ಜಾಗರೂಕರಾಗಿ ಗಮನಿಸುವರು. ಕೆಲವು ವೇಳೆ ತಾವೆ ಒಂದು ಹೊಸ ಫ್ಯಾಷನ್​ ಕಂಡು ಹಿಡಿದು ಜನರಿಗೆ ಆಕರ್ಷಣೀಯ ವಾಗುವಂತೆ ಮಾಡುವರು. ವ್ಯಾಪಾರಿಯು ಜನರ ಮೆಚ್ಚಿಗೆಗೆ ಪಾತ್ರನಾದರೆ ಅವನು ಶ‍್ರೀಮಂತನಾದಂತೆಯೆ. ಫ್ರಾನ್ಸಿನಲ್ಲಿ ಮೂರನೇ ನೆಪೋಲಿಯನ್ನಿನ ಕಾಲದಲ್ಲಿ ಅವನ ರಾಣಿ ಯೂಜಿನಿ ಪಶ್ಚಿಮದ ವೇಷಭೂಷಣದ ಅಧಿಷ್ಠಾತ್ರಿ ದೇವಿ ಯಾಗಿದ್ದಳು, ಅವಳಿಗೆ ಕಾಶ್ಮೀರಿ ಶಾಲು ಬಹಳ ಇಷ್ಟ. ಆಗಿನ ಕಾಲದಲ್ಲಿ ಕೋಟ್ಯಂತರ ರೂಪಾಯಿ ಬೆಲೆಬಾಳುವ ಶಾಲುಗಳನ್ನು ಕಾಶ್ಮೀರದಿಂದ ಯೂರೋಪಿಗೆ ರವಾನಿಸುತ್ತಿದ್ದರು. ನೆಪೋಲಿಯನ್​ನ ಪತನವಾದ ಮೇಲೆ ಫ್ಯಾಷನ್​ ಬದಲಾಗಿ ಕಾಶ್ಮೀರಿ ಶಾಲುಗಳಿಗೆ ಗಿರಾಕಿಗಳಿರಲಿಲ್ಲ. ನಮ್ಮ ವರ್ತಕರು ಯಾವಾಗಲೂ ಹಳೆಯ ಹಾದಿಯಲ್ಲೇ ಹೋಗುತ್ತಾರೆ. ನಮ್ಮವರು ಸಮಯಾ ನುಸಾರ ಯಾವುದು ಹೊಸ ಫ್ಯಾಷನ್​ ಎನ್ನುವುದನ್ನು ಗಮನಿಸಿ ಜನರು ಗಮನ ಸೆಳೆಯುವುದಕ್ಕೆ ಪ್ರಯತ್ನಿಸುವುದಿಲ್ಲ. ಅದಕ್ಕೆ ಕಾಶ್ಮೀರದ ವ್ಯಾಪಾರಕ್ಕೇ ಧಕ್ಕೆ ಬಂದಿತು. ದೊಡ್ಡ ದೊಡ್ಡ ವರ್ತಕರು ಗರೀಬರಾದರು.

ಇದೇ ಪ್ರಪಂಚ. ಯಾರು ಜಾಗ್ರತರಾಗಿರುವರೊ ಅವರು ಮುಂದೆ ಬರು ವರು; ಯಾರು ನಿದ್ರಿಸುತ್ತಿರುವರೊ ಅವರು ಹಿಂದೆ ಬೀಳುವರು. ಒಬ್ಬ ಮತ್ತೊಬ್ಬನಿಗೆ ಕಾಯುವನು ಎಂದಿರುವಿರೇನು? ಪಾಶ್ಚಾತ್ಯರು ಜನರನ್ನು ಆಕರ್ಷಿಸಲು ಯಾವಾಗಲೂ ಹೊಸಹೊಸ ಪೋಷಾಕನ್ನು ತಯಾರಿಸುವರು. ಪರಿಸ್ಥಿತಿಯನ್ನು ಹತ್ತು ಕಣ್ಣುಗಳಿಂದ ನೋಡಿ, ಇನ್ನೂರು ಕೈಗಳಿಂದ ಕೆಲಸ ಮಾಡುವಂತೆ ಇರುವರು. ನಮ್ಮವರು ಶಾಸ್ತ್ರದಲ್ಲಿ ಏನು ಹೇಳಿಲ್ಲವೋ ಅದನ್ನು ಮಾಡುವಂತಯೇ ಇಲ್ಲ. ಅದ್ದರಿಂದ ನಾವು ಹಳೆಯ ಜಾಡಿನಲ್ಲಿಯೇ ಸುತ್ತುತ್ತಿದ್ದೇವೆ. ಹೊಸದಕ್ಕಾಗಿ, ಸ್ವಂತಿಕೆಗಾಗಿ ಪ್ರಯತ್ನಿಸುವ ಪ್ರಕೃತಿಯೇ ನಮ್ಮಲ್ಲಿಲ್ಲ. ಹಾಗೆ ಮಾಡಲು ಬೇಕಾದ ಶಕ್ತಿಯೇ ನಮ್ಮಲ್ಲಿಲ್ಲ. ಹೊಟ್ಟೆಗೆ ಹಿಟ್ಟಿಲ್ಲದ ಕೂಗು ದೇಶ ದಾದ್ಯಂತವೂ ವ್ಯಾಪಿಸಿದೆ. ಇದು ಯಾರ ತಪ್ಪು? ನಮ್ಮದು. ಇದರ ಪರಿಹಾರ ಕ್ಕಾಗಿ ನಾವು ಯಾವುದನ್ನೂ ಮಾಡುವುದಿಲ್ಲ. ಸುಮ್ಮನೆ ಅರಚಿಕೊಳ್ಳುತ್ತಿರುವೆವು. ನಿಮ್ಮ ಬಿಲಗಳಿಂದ ಹೊರಗೆ ಬಂದು ಕಣ್ಣು ಬಿಟ್ಟು ಪ್ರಪಂಚ ಹೇಗೆ ಮುಂದುವರಿ ಯುತ್ತಿದೆ ಎಂಬುದನ್ನು ನೋಡಿ. ಅಗ ಜ್ಞಾನನೇತ್ರ ತೆರೆದು ಅವಶ್ಯಕ ಕರ್ತವ್ಯ ಯಾವುದು ಮತ್ತು ಅದನ್ನು ಹೇಗೆ ಮಾಡಬೇಕೆಂದು ಹೊಳೆಯುವುದು. ದೇವಾಸುರರ ಪ್ರಸಂಗ ನಿಮಗೆ ಗೊತ್ತಿದೆ. ದೇವತೆಗಳು ಆಸ್ತಿಕರು, ಅವರಿಗೆ ಆತ್ಮನಲ್ಲಿ ವಿಶ್ವಾಸ, ಈಶ್ವರ, ಪರಲೋಕ ಇವುಗಳಲ್ಲಿ ನಂಬಿಕೆ ಇದ್ದುವು. ಅಸುರರು ಈಗಿನ ಜೀವನಕ್ಕೆ ಪ್ರಾಮುಖ್ಯವನ್ನು ಕೊಡುತ್ತಿದ್ದರು. ಪೃಥ್ವಿಯನ್ನು ಭೋಗಿಸಿ ಶರೀರವನ್ನು ಸುಖವಾಗಿಡುತ್ತಿದ್ದರು. ಈ ಸಮಯದಲ್ಲಿ, ನಾವು ದೇವತೆಗಳು ಮೇಲೋ, ಅಸುರರು ಮೇಲೋ, ಎಂಬುದನ್ನು ಪರಿಗಣಿಸುತ್ತಿಲ್ಲ. ಆದ್ಧರೆ ಪುರ್ಧ್ಧಾಣ್ಧವ್ಧನ್ನು ಓ್ಧದ್ಧಿದ್ಧರೆ ಅಸ್ಧುರ್ಧರು ಮನ್ಧುಷ್ಯ್ಧರ್ಧನ್ನೇ ಹೋಲ್ಧುತ್ತ್ಧಿದ್ದ್ಧರು. ಮತ್ತು ದೇವತೆಗಳಿಗಿಂತ ವೀರರಾಗಿದ್ದರು. ದೇವತೆಗಳು ಅನೇಕ ವಿಷಯಗಳಲ್ಲಿ ನಿಜವಾಗಿಯೂ ಅಸುರರಿಗಿಂತ ಕೆಳಮಟ್ಟದಲ್ಲಿದ್ದರು. ಪ್ರಾಚ್ಯ ಪಾಶ್ಚಾತ್ಯಗಳನ್ನು ತಿಳಿದುಕೊಳ್ಳಬೇಕಾದರೆ, ಹಿಂದೂ ಗಳನ್ನು ದೇವತೆಗಳೊಂದಿಗೆ ಪಾಶ್ಚಾತ್ಯರನ್ನು ಅಸುರರೊಂದಿಗೆ ಹೋಲಿಸಬೇಕು.

ಮೊದಲು ಇಬ್ಬರ ದೇಹಶೌಚದ ವಿಷಯವಾಗಿ ನೋಡುವ. ಪವಿತ್ರವಾದು ದೆಂದರೆ ಬಾಹ್ಯ ಮತ್ತು ಆಂತರಿಕ ಶುದ್ಧಿ. ದೇಹ ಶುದ್ಧಿ ನೀರಿನಿಂದ ಆಗುವುದು. ಹಿಂದೂಗಳಷ್ಟು ದೇಹವನ್ನು ಶುಚಿಯಾಗಿ ಇಡುವವರು ಪ್ರಪಂಚದಲ್ಲಿ ಮತ್ತಾರೂ ಇಲ್ಲ. ಹಿಂದೂಗಳಷ್ಟು ನೀರನ್ನು ಧಾರಾಳವಾಗಿ ಉಪಯೋಗಿಸುವವರು ಅಪರೂಪ, ಇವರಂತೆ ನೀರಿನಲ್ಲಿ ಮುಳುಗುವವರೇ ವಿರಳ. ಆಂಗ್ಲೇಯರು ಭರತಖಂಡಕ್ಕೆ ಬಂದ ಮೇಲೆ ಸ್ನಾನಮಾಡುವುದನ್ನು ಇಂಗ್ಲೆಂಡಿನಲ್ಲಿ ಕೊಂಚ ಪ್ರಚಾರಕ್ಕೆ ತಂದಿರುವರು. ಈಗಲೂ ಇಂಗ್ಲೆಂಡಿನಲ್ಲಿ ಓದುತ್ತಿದ್ದ ವಿದ್ಯಾರ್ಥಿಗಳನ್ನು ವಿಚಾರಿಸಿ ನೋಡಿ, ಸ್ನಾನಕ್ಕೆ ಎಷ್ಟು ತೊಂದರೆ ಅಲ್ಲಿ ಎನ್ನುವುದನ್ನು. ಪಾಶ್ಚಾತ್ಯರು ವಾರಕ್ಕೊಮ್ಮೆ ಸ್ನಾನ ಮಾಡುತ್ತಾರೆ. ಮತ್ತು ಆಗ ತಮ್ಮ ಒಳಬಟ್ಟೆಯನ್ನು ಬದಲಾಯಿಸುತ್ತಾರೆ. ಕೆಲವು ಶ‍್ರೀಮಂತರು ಪ್ರತಿದಿನವೂ ಸ್ನಾನಮಾಡುತ್ತಾರೆ. ಅಮೇರಿಕಾದಲ್ಲಿ ಪ್ರತಿದಿನ ಸ್ನಾನ ಮಾಡುವವರ ಸಂಖ್ಯೆ ಅಧಿಕವಾಗಿದೆ! ಜರ್ಮನರು ವಾರಕ್ಕೆ ಒಮ್ಮೆ; ಫ್ರಾನ್ಸ್​ ಮತ್ತು ಇತರ ದೇಶಗಳಲ್ಲಿ ಅಪರೂಪಕ್ಕೆ ಯಾವುದಾದರೊಂದು ದಿನ! ಸ್ಪೆಯಿನ್​ ಮತ್ತು ಇಟಲಿ ಶಾಖವಾದ ದೇಶ. ಆದರೂ ಸ್ನಾನಮಾಡುವವರ ಸಂಖ್ಯೆ ಕಡಿಮೆ. ಬೇಕಾದಷ್ಟು ಬೆಳ್ಳುಳ್ಳಿ ತಿನ್ನುವರು. ಬೆವರು ಧಾರಾಳವಾಗಿ ಸುರಿಯುವುದು. ಆದರೂ ಸ್ನಾನವಿಲ್ಲ. ಅವರ ದುರ್ಗಂಧಕ್ಕೆ ಭೂತಗಳೂ ಓಡಿಹೊಗಬೇಕು, ಮನುಷ್ಯರ ಪಾಡೇನು! ಸ್ನಾನವೆಂದರೆ ಅರ್ಥವೇನು? ಕೈಕಾಲು, ಮುಖ, ಹೊರಗೆ ತೋರುವ ಭಾಗ ವನ್ನು ತೊಳೆಯುವುದು. ಸಭ್ಯತೆಯ ರಾಜಧಾನಿ, ಭೋಗ ವಿಲಾಸದಲ್ಲಿ ಸ್ವರ್ಗ, ಶಿಲ್ಪ ಕೇಂದ್ರವಾದ ಪ್ಯಾರಿಸ್ಸಿಗೆ ನನ್ನ ಶ‍್ರೀಮಂತ ಸ್ನೇಹಿತನು ನನ್ನನ್ನು ಕರೆದು ಕೊಂಡು ಹೋದನು. ಒಂದು ದೊಡ್ಡ ಅರಮನೆಯಂತೆ ಇರುವ ಹೋಟೆಲಿನಲ್ಲಿ ನಾವು ಇಳಿದುಕೊಂಡೆವು. ರಾಜ ಭೋಗದಂತೆ ಊಟ ಕೊಡುತ್ತಿದ್ದರು. ಆದರೆ ಸ್ನಾನದ ಹೆಸರೇ ಇಲ್ಲ. ಎರಡು ದಿನ ಹೇಗೋ ತಾಳಿಕೊಂಡು ಕೊನೆಗೆ ಅಸಾಧ್ಯವಾಗಿ ನನ್ನ ಸ್ನೇಹಿತನಿಗೆ “ಈ ರಾಜಭೋಗ ನಿನಗೇ ಇರಲಿ. ಇಲ್ಲಿಂದ ಪಾರಾಗುವುದಕ್ಕೆ ನಾನು ಯತ್ನಿಸುತ್ತಿರುವೆನು. ಇಷ್ಟು ಶಾಖ, ಆದರೂ ಸ್ನಾನ ಮಾಡುವುದಕ್ಕೆ ಅವಕಾಶವೇ ಇಲ್ಲ. ಹೆಚ್ಚು ಇಲ್ಲಿದ್ದರೆ ಹುಚ್ಚು ನಾಯಿಯಂತೆ ಆಗುವ ಸಂಭವವಿದೆ” ಎಂದೆ. ನನ್ನ ಸ್ನೇಹಿತ ಇದನ್ನು ಕೇಳಿ ದುಃಖಿತನಾಗಿ ಹೋಟೇಲಿನ ಯಜಮಾನನ ಮೇಲೆ ಕುಪಿತನಾದನು. ಇಲ್ಲಿ ನೀವು ಇರ ಬೇಕಾಗಿಲ್ಲ, ಬೇರೆ ಉತ್ತಮ ಸ್ಥಳಕ್ಕೆ ಹೋಗೋಣವೆಂದು ನನಗೆ ಹೇಳಿದನು.

ಸುಮಾರು ಹನ್ನೆರಡು ಹೋಟೆಲುಗಳನ್ನು ಹೋಗಿ ನೋಡಿದೆನು. ಎಲ್ಲೂ ಸ್ನಾನಮಾಡುವುದಕ್ಕೆ ಅವಕಾಶವಿಲ್ಲ. ಸ್ನಾನಮಾಡುವುದಕ್ಕೆ ಬೇರೆ ಸ್ಥಳವಿತ್ತು. ಅಲ್ಲಿ ನಾಲ್ಕೈದು ರೂಪಾಯಿ ಕೊಟ್ಟು ಒಮ್ಮೆ ಸ್ನಾನಮಾಡಬೇಕು. ಆ ದಿನ ಮಧ್ಯಾಹ್ನದ ಪೇಪರಿನಲ್ಲಿ, ಒಬ್ಬ ಮುದುಕಿ ಸ್ನಾನಕ್ಕೆ ಹೋಟೆಲಿನಲ್ಲಿ ನೀರನ್ನು ಮುಟ್ಟಿದಾಗ ಮೃತಳಾದಳೆಂಬ ಸುದ್ದಿ ಇತ್ತು. ಡಾಕ್ಟರ್​ ಇದಕ್ಕೆ ಯಾವ ಕಾರಣವನ್ನಾದರೂ ಕೊಡಲಿ. ಆದರೆ ನನ್ನ ಕಾರಣ ಬೇರೆ ಇದೆ. ಜೀವನದಲ್ಲಿ ಪ್ರಥಮಬಾರಿ ಅಷ್ಟೊಂದು ನೀರನ್ನು ಮುಟ್ಟಿದುದರಿಂದ ಆದ ಶಾಕ್​ನಿಂದ ಆಕೆ ಮೃತಳಾಗಿರ ಬೇಕು! ಇದರಲ್ಲಿ ಏನೂ ಅತಿಶಯೋಕ್ತಿ ಇಲ್ಲ. ರಷ್ಯ ಮತ್ತು ಇತರ ಜನರು ತುಂಬ ಕೊಳಕು. ತಿಬೆಟ್ಟಿನಿಂದ ಈ ಕೊಳಕು ಜನಾಂಗ ಪ್ರಾರಂಭವಾಗುವುದು. ಅಮೆರಿಕಾದಲ್ಲಿ ಪ್ರತಿಯೊಂದು ವಾಸಗೃಹಕ್ಕೂ ಒಂದು ಸ್ನಾನದ ಮನೆ, ನಲ್ಲಿ ಇರುವುದು.

ನಮಗೂ ಅವರಿಗೂ ಇರುವ ಅಂತರವನ್ನು ಇಲ್ಲಿ ನೋಡಿ. ಹಿಂದೂ ಸ್ನಾನ ಮಾಡುವುದು ಏತಕ್ಕೆ? ಪಾಪ ನಿವಾರಣೆಗೆ. ಪಾಶ್ಚಾತ್ಯರು ತಮ್ಮ ಕೈಕಾಲು ಮುಖಗಳನ್ನು ತೊಳೆಯುವುದು ಶುಭ್ರತೆಗೆ. ನಮ್ಮಲ್ಲಿ ಸ್ನಾನಮಾಡುವುದೆಂದರೆ ಮೈಮೇಲೆ ನೀರು ಸುರಿದುಕೊಳ್ಳುವುದು ಅಷ್ಟೆ. ಮೈಮೇಲೆ ಕೊಳೆ, ಎಣ್ಣೆಜಿಡ್ಡು ಇರಲಿ, ಸುಮ್ಮನೆ ನೀರು ಹಾಕಿಕೊಂಡರೆ ಸಾಕು. ನಮ್ಮ ದಾಕ್ಷಿಣಾತ್ಯ ಸಹೋದರ ಅಷ್ಟು ದೊಡ್ಡ ದೊಡ್ಡ ನಾಮಗಳನ್ನು ಹಾಕಿಕೊಳ್ಳುತ್ತಾನೆ. ಅದನ್ನು ಅಳಿಸಬೇಕಾದರೆ ಒಂದು ಉಜ್ಜುವ ಕಲ್ಲೆ ಬೇಕು! ನಮಗೆ ಸ್ನಾನ ಮಾಡುವುದು ಸುಲಭ. ಎಲ್ಲಾದರೂ ನೀರಿನಲ್ಲಿ ಮುಳುಗಿದರೆ ಸಾಕು. ಪಾಶ್ಚಾತ್ಯರಿಗೆ ಹಾಗೆ ಆಗುವುದಿಲ್ಲ. ಅವರಿಗೆ ಪೋಷಾಕನ್ನು ಬಿಚ್ಚಬೇಕಾದರೆ ಒಂದು ಘಂಟೆ ಬೇಕು. ಎಷ್ಟು ಗುಂಡಿ, ಕಾಜಾ ಇವೆ! ನಮಗೆ ಶರೀರವನ್ನು ತೋರುವುದರಲ್ಲಿ ಲಜ್ಜೆಯೇ ಇಲ್ಲ. ಅವರು ಹಾಗಲ್ಲ, ತಂದೆ ಎದುರಿಗೆ ಮಗ ಬೆತ್ತಲೆ ಇರಬಹುದು. ಆದರೆ ಸ್ತ್ರೀಯರೆದುರಿಗೆ ಆಪಾದಮಸ್ತಕವೂ ಪೋಷಾಕಿನಿಂದ ಮರೆಯಾಗಿರಬೇಕು.

ಬಾಹ್ಯ ಶೌಚದ ಆಚಾರ ಉಳಿದ ಆಚಾರದಂತೆ ಮಿತಿಮೀರಿ ಹೋದಾಗ ಅನಾಚಾರವಾಗುವುದು. ಶರೀರಕ್ಕೆ ಸಂಬಂಧಿಸಿದ ಕಾರ್ಯವನ್ನೆಲ್ಲ ಅತಿ ಗುಪ್ತವಾಗಿ ಮಾಡಬೇಕೆಂಬುದೇ ಯುರೋಪಿಯನ್ನರ ಮತ. ಇದು ಒಳ್ಳೆಯ ಮಾತು. ಇತರರೆದುರಿಗೆ ಉಗುಳುವುದು ಕೆಟ್ಟದು, ಬಾಯನ್ನು ಇತರರೆದುರಿಗೆ ಮುಕ್ಕಳಿ ಸುವುದು ನಾಚಿಕೆಗೇಡು. ಜನರಿಗೆ ನಾಚಿ, ಊಟವಾದ ಮೇಲೆ ಬಾಯನ್ನೇ ಮುಕ್ಕಳಿಸುವುದಿಲ್ಲ. ಇದರ ಪರಿಣಾಮವಾಗಿ ಹಲ್ಲು ಕೆಡುವುದು, ಇದು ಸಭ್ಯತೆಗೆ ಅಂಜಿದ ಅನಾಚಾರ. ಆದರೆ ನಾವು ಎಲ್ಲರೊಂದಿಗೆ ಶಬ್ದ ಮಾಡುತ್ತಾ ಕುಳಿತು ಕೊಂಡು ಬಾಯಿ ಮುಕ್ಕಳಿಸುವೆವು. ಇದು ಅತ್ಯಾಚಾರ. ಇದನ್ನು ಮೌನದಲ್ಲಿ ಗುಪ್ತವಾಗಿ ಮಾಡಬೇಕೆನ್ನುವುದು ಸರಿ. ಆದರೆ ಸಮಾಜಕ್ಕೆ ಅಂಜಿ ಮಾಡದೆ ಇರುವುದೂ ದೋಷ.

ದೇಶಭೇದಕ್ಕೆ ತಕ್ಕಂತೆ ಯಾವ ಅಭ್ಯಾಸಗಳು ಅನಿವಾರ್ಯವಾಗಿವೆಯೊ ಅವನ್ನು ಸಮಾಜವು ಶಾಂತರೀತಿಯಿಂದ ಅನುಭವಿಸುತ್ತದೆ. ನಮ್ಮಂತಹ ಬಿಸಿಲು ದೇಶದಲ್ಲಿ ಊಟಮಾಡುವಾಗ ಅರ್ಧ ಕೊಡ ನೀರು ಕುಡಿದು ಬಿಡುವೆವು. ನಂತರ ತೇಗದೆ ಏನು ಮಾಡಬೇಕು? ಆದರೆ ಪಾಶ್ಚಾತ್ಯರಲ್ಲಿ ತೇಗುವುದು ಅಸಭ್ಯವರ್ತನೆ. ಆದರೆ ಅಲ್ಲಿ ಊಟಮಾಡುವಾಗ ಮೂಗನ್ನು ಒರೆಸಿಕೊಳ್ಳುವುದ ಕ್ಕಾಗಿ ಕರವಸ್ತ್ರವನ್ನು ಉಪಯೋಗಿಸುವರು. ಅದು ಅಸಭ್ಯ ವರ್ತನೆಯಲ್ಲ. ಆದರೆ ನಮ್ಮಲ್ಲಿ ಅದು ಅಸಹ್ಯ. ಆದರೆ ಶೀತ ದೇಶದಲ್ಲಿ ಅವರು ಪದೇ ಪದೇ ಹಾಗೆ ಒರಸಿಕೊಳ್ಳದೆ ವಿಧಿಯಿಲ್ಲ.

ಹಿಂದೂಗಳು ಕೊಳೆಯನ್ನು ಅತಿ ನಿಕೃಷ್ಟವಾಗಿ ಕಾಣುವರು. ಆದರೂ ನಾವು ಅನೇಕ ವೇಳೆ ಬಹಳ ಕೊಳಕರು. ನಾವು ಕೊಳೆಯನ್ನು ಅಷ್ಟು ತಾತ್ಸಾರದಿಂದ ಕಾಣುವೆವು. ಅದನ್ನು ಮುಟ್ಟಿದೊಡನೆಯೇ ಸ್ನಾನಮಾಡುವೆವು. ಅದನ್ನು ಮುಟ್ಟದೆ ಇರುವುದರಿಂದ ಅದು ಮನೆ ಮುಂದೆಯೇ ಕೊಳೆತು ನಾರುವುದು. ಅದನ್ನು ಮುಟ್ಟಕೂಡದು. ಇಷ್ಟು ಜೋಪಾನವಾಗಿದ್ದರೆ ಸಾಕು. ನಾವು ಒಂದು ನರಕ ಕುಂಡದಲ್ಲಿ ವಾಸಿಸುತ್ತಿರುವೆವು ಎಂಬುದನ್ನು ಭಾವಿಸುವುದೇ ಇಲ್ಲ. ಒಂದು ಅನಾಚಾರದ ಭಯದಿಂದ ಅದಕ್ಕಿಂತ ಗುರುತರ ಪಾಪವನ್ನು ಮಾಡುತ್ತೇವೆ. ಯಾರು ಮನೆಯಲ್ಲೇ ಕಸವನ್ನು ಇಟ್ಟುಕೊಂಡಿರುವರೋ ಅವರು ಅವಶ್ಯಕವಾಗಿ ಪಾಪಿಗಳು, ಇದರಲ್ಲಿ ಸಂದೇಹವೇನಿದೆ? ಇದರಿಂದ ಬರುವ ಪಾಪಭೋಗಕ್ಕೆ ಇನ್ನೊಂದು ಜನ್ಮಕ್ಕೆ ಕಾಯಬೇಕಾಗಿಲ್ಲ. ಈ ಜನ್ಮದಲ್ಲೇ ಅದು ದೊರಕುವುದು.

ಪಾಶ್ಚಾತ್ಯ ಜನರ ಮೇಲೆ ಲಕ್ಷೀ ಸರಸ್ವತಿಯರಿಬ್ಬರೂ ಪ್ರಸನ್ನರಾಗಿರುವರು. ಭೋಗ, ಸಾಮಗ್ರಿಗಳನ್ನು ಸಂಪಾದಿಸುವುದರಲ್ಲಿ ಮಾತ್ರ ಅವರು ನಿಲ್ಲುವುದಿಲ್ಲ. ಅವರು ಎಲ್ಲಾ ಕೆಲಸದಲ್ಲೂ ಒಂದು ವಿಧದ ಸೌಂದರ್ಯ, ನಾಜೂಕನ್ನು ತರುವರು. ತಿನ್ನುವುದರಲ್ಲೂ ಕುಡಿಯುವುದರಲ್ಲೂ ಮನೆಯ ಒಳಗೆ ಹೊರಗೆ ಎಲ್ಲಾ ಕಡೆ ಯಲ್ಲೂ ಶುಭ್ರವಾಗಿರುವಂತೆ ಮಾಡುವರು. ನಮ್ಮ ದೇಶದಲ್ಲಿ ಹಿಂದೆ ಸಂಪತ್ತು ಇದ್ದಾಗ ಹಾಗೆ ಇತ್ತು. ನಾವು ಈಗ ಹೆಚ್ಚು ಗರೀಬರಾಗಿದ್ದೇವೆ. ಅದಕ್ಕಿಂತ ಭೀಕರ ವಾದುದು ಯಾವುದೆಂದರೆ, ಎರಡು ರೀತಿ ನಾವು ಅಧೋಗತಿಗೆ ಇಳಿಯುತ್ತಿರು ವೆವು. ಮತ್ತೊಬ್ಬರನ್ನು ಅನುಸರಿಸುವ ಭರದಲ್ಲಿ ನಮ್ಮ ಸ್ವಂತ ಆಚಾರವನ್ನು ಎಸೆಯುತ್ತಿರುವೆವು. ನಮ್ಮ ಜನಾಂಗದ ರೀತಿ ನೀತಿ ಮಾಯವಾಗುತ್ತಿದೆ. ಪಾಶ್ಚಾತ್ಯರ ಆಚಾರವನ್ನು ಕಲಿತುಕೊಳ್ಳುತ್ತಲೂ ಇಲ್ಲ. ನಡೆನುಡಿ ಎಲ್ಲಾ ನಮ್ಮದೇ ಒಂದು ರೀತಿ ಇತ್ತು ಹಿಂದೆ. ಅದು ಈಗ ಹೋಯಿತು. ಪಾಶ್ಚಾತ್ಯರ ಆಚಾರ ವ್ಯವಹಾರಗಳನ್ನೂ ಕಲಿಯಲಿಲ್ಲ. ಹಳೆಯ ಕಾಲದ ಜಪ, ಪೂಜೆ, ಅಧ್ಯಯನ ಮುಂತಾದುವನ್ನೂ ಆಚೆಗೆ ಎಸೆಯುತ್ತಿರುವೆವು. ಆದರೆ ಅದರ ಬದಲು ಉತ್ತಮವಾಗಿರುವುದಾವುದೂ ಬೇರು ಬಿಡುತ್ತಿಲ್ಲ. ಇಂದಿನ ನಮ್ಮ ದುಃಸ್ಥಿತಿ ಈ ಎರಡರ ಮಧ್ಯೆ ತೊಳಲಾಡುವುದಾಗಿದೆ. ಭವಿಷ್ಯವಂಗ ದೇಶಕ್ಕೆ ಇನ್ನೂ ಸರಿಯಾದ ತಳಹದಿ ಹಾಕಿಲ್ಲ. ಎಲ್ಲಕಿಂತ ಹೆಚ್ಚಾಗಿ ನಮ್ಮ ಕಲೆಗಳಗೆ ಹೆಚ್ಚು ಅಪಾಯ ತಾಕಿದೆ. ಹಿಂದಿನ ಕಾಲದಲ್ಲಿ ನಮ್ಮ ಅಜ್ಜಿಯರು ಅಕ್ಕಿಯ ಹಿಟ್ಟಿನಲ್ಲಿ ಮನೆಯಲ್ಲಿ ಅತಿ ಸುಂದರವಾದ ಹಸೆಗಳನ್ನು ಬರೆಯುತ್ತಿದ್ದರು. ಬಾಗಿಲ ಮೇಲೆ ಗೋಡೆಯ ಮೇಲೆಲ್ಲಾ ಚಿತ್ರಿಸುತ್ತಿದ್ದರು. ಊಟಕ್ಕೆ ಬಾಳೆ ಎಲೆ ಹಾಕುವಾಗ ಅದನ್ನು ಸುಂದರವಾಗಿ ಕತ್ತರಿಸುತ್ತಿದ್ದರು. ಹಲವು ಬಗೆಯ ಭೋಜನವನ್ನು ಎಲೆಯ ಮೇಲೆ ಸುಂದರವಾಗಿ ಅಣಿಮಾಡುತ್ತಿದ್ದರು. ಆ ಕಲೆಯೆಲ್ಲ ಈಗ ಮಾಯವಾಗುತ್ತಿದೆ.

ಹೊಸದನ್ನು ಕಲಿತುಕೊಳ್ಳಬೇಕು, ನಮ್ಮದರೊಂದಿಗೆ ಸೇರಿಸಿಕೊಳ್ಳಬೇಕೆಂಬು ದೇನೋ ನಿಜ. ಆದರೆ ಅದನ್ನು ಮಾಡಬೇಕಾದರೆ ಸುಮ್ಮನೆ ಹಳೆಯದನ್ನೆಲ್ಲ ಆಚೆಗೆ ಬಿಸುಡಬೇಕೇ? ಹೊಸದಾಗಿ ನೀವು ಏನು ಕಲಿತಿರುವಿರಿ? ಏನೂ ಇಲ್ಲ. ಕೆಲವು ಹೊಸ ಪದಗಳು ಮಾತ್ರ. ಯಾವ ಹೊಸ ಕಲೆ ಅಥವಾ ವಿಜ್ಞಾನವನ್ನು ಕಲಿತಿ ರುವಿರಿ? ಈಗಲೂ ಕೂಡ ಹೋಗಿ ನೋಡಿ, ದೂರದ ಹಳ್ಳಿಯಲ್ಲಿರುವ ಇಟ್ಟಿಗೆ ಕೆಲಸ ಮತ್ತು ಮರಗೆಲಸವನ್ನು! ನಿಮ್ಮ ಊರಿನ ಬಡಗಿ ಚೆನ್ನಾದ ಒಂದು ಜೊತೆ ಬಾಗಿಲನ್ನು ತಯಾರು ಮಾಡಲಾರ. ಅದು ಒಂದು ಗುಡಿಸಲಿಗೊ ಅಥವಾ ಬಂಗಲೆಗೊ, ಅದೂ ಕೂಡ ಗೊತ್ತು ಮಾಡಲು ಆಗುವುದಿಲ್ಲ. ಅವರು ಹೊರಗಿ ನಿಂದ ಬರುವ ಉಪಕರಣಗಳನ್ನು ಕೊಳ್ಳುವುದರಲ್ಲಿ ಮಾತ್ರ ನಿಪುಣರು. ಅದೇ ಬಡಗಿ ಕೆಲಸವೆಂದು ಭಾವಿಸುವರು. ಅಯ್ಯೋ, ಎಂತಹ ದುಃಸ್ಥಿತಿ ನಮಗೆ ಪ್ರಾಪ್ತವಾಗಿದೆ! ನಮ್ಮದಾಗಿ ಇದ್ದದೆಲ್ಲಾ ಮಾಯವಾಗುತ್ತಿದೆ. ಆದರೂ ಹೊರಗಿನವರಿಂದ ನಾವು ಕಲಿತಿರುವುದು ಮಾತನಾಡುವುದನ್ನು ಮಾತ್ರ. ಬರೇ ಓದು, ಮಾತು! ಬಂಗಾಳ ಮತ್ತು ಐರ್ಲೆಂಡಿನಲ್ಲಿರುವವರು ಒಂದೆ ಅಚ್ಚಿನವರು. ಸುಮ್ಮನೆ ಮಾತು ಮಾತು! ಇವರಿಬ್ಬರೂ ತುಂಬಾ ಚತುರರು ಮಾತಿನಲ್ಲಿ. ಎಲ್ಲಿ ಒಂದು ಚೂರು ಕೆಲಸ ಮಾಡಬೇಕಾಗಿ ಬರುವುದೊ ಆಗ ಪತ್ತೆಯಿಲ್ಲ. ಇದಲ್ಲದೆ ಇಡೀ ಜೀನಮಾನದಲ್ಲಿ ಜಗಳ ಬೇರೆ.

ಪಾಶ್ಚಾತ್ಯರು ತಮಗೆ ಸೇರಿದ ಎಲ್ಲವನ್ನೂ ಶುಭ್ರವಾಗಿಟ್ಟಿರುವರು. ಅತಿ ದರಿದ್ರನೂ ಕೂಡ ಇದಕ್ಕೆ ಗಮನಕೊಡುವನು. ಈ ಶುಭ್ರತೆಯನ್ನು ಅನುಸರಿಸಲೇಬೇಕು. ಶುಭ್ರವಾದ ಪೋಷಾಕು ಇಲ್ಲದೇ ಇದ್ದರೆ ಯಾರೂ ಅವನಿಗೆ ಕೆಲಸ ಕೊಡುವುದಿಲ್ಲ. ಅವರ ಅಡಿಗೆ ಪರಿಚಾರಕರೆಲ್ಲ ಶುಭ್ರವಾದ ಬಟ್ಟೆಯನ್ನು ಧರಿಸುವರು. ತಮ್ಮ ಮನೆಯನ್ನು ಪ್ರತಿದಿನವೂ ಗುಡಿಸಿ, ಸಾರಿಸಿ ಶುಭ್ರವಾಗಿಟ್ಟಿರುವರು. ಸಭ್ಯತೆಯ ಒಂದು ಕುರುಹೇ ಸಿಕ್ಕಾಪಟ್ಟೆ ಸಾಮಾನನ್ನು ಹರಡದಿರುವುದು. ಅವುಗಳನ್ನು ಅದರದರ ಸ್ಥಾನದಲ್ಲಿ ಇಡಬೇಕು. ಅವರ ಅಡಿಗೆ ಮನೆ ನಿರ್ಮಲನಾಗಿರುತ್ತದೆ. ಬೇಕಾದಷ್ಟು ಬೆಳಕು ಬರುವುದು. ತರಕಾರಿ ಸಿಪ್ಪೆ ಮುಂತಾದ ಕಸವನ್ನು ತಾತ್ಕಾಲಿಕವಾಗಿ ಒಂದು ಕಡೆ ಸೇರಿಸಿಡುವರು. ನಂತರ ಕಸಗುಡಿಸುವವರು ಅದಕ್ಕಾಗಿ ನಿಯಮಿಸಿದ ಒಂದು ಸ್ಥಳಕ್ಕೆ ಊರ ಹೊರಗೆ ಒಯ್ಯುವರು. ಅವನ್ನು ಮನೆಮುಂದೆ ರಸ್ತೆಯ ಬಳಿ ಬಿಸಾಡುವುದಿಲ್ಲ.

ಶ‍್ರೀಮಂತರ ಸೌಧ ಮುಂತಾದುವನ್ನು ನೋಡುವುದು ಕಣ್ಣಿಗೆ ಒಂದು ಹಬ್ಬ. ಹಗಲು ರಾತ್ರಿ ಅವು ಅತಿ ನಿರ್ಮಲವಾಗಿರುವುವು. ಇದಲ್ಲದೆ ಪ್ರಪಂಚದ ಇತರ ಕಡೆಗಳಿಂದ ಕಲಾವಸ್ತುಗಳನ್ನು ಸಂಗ್ರಹಿಸಿ ತಮ್ಮ ಕೋಣೆಯಲ್ಲಿ ಅಲಂಕರಿಸುವ ಅಭ್ಯಾಸ ಬೇರೆ ಇದೆ. ಸಧ್ಯಕ್ಕೆ ನಾವು ಅವರಂತೆ ಕಲಾಕೃತಿಗಳನ್ನು ಸಂಗ್ರಹ ಮಾಡಬೇಕಾಗಿಲ್ಲ. ಆದರೆ ಯಾವುದು ನಮ್ಮಲ್ಲಿ ಆಗಲೇ ಇದೆಯೊ, ಅದು ಹಾಳಾಗದಂತೆ ನೋಡಿಕೊಳ್ಳಬೇಡವೆ! ಚಿತ್ರ ಮತ್ತು ಶಿಲ್ಪಕಲೆಯಲ್ಲಿ ಅವರಂತೆ ಆಗಬೇಕಾಗರೆ ನಮಗೆ ಇನ್ನೂ ಬಹಳ ದಿನಬೇಕು. ನಾವೆಂದಿಗೂ ಅದರಲ್ಲಿ ನಿಪುಣರಾಗಿರಲಿಲ್ಲ. ಪಾಶ್ಚಾತ್ಯರನ್ನು ಅನುಕರಿಸಿ ಒಬ್ಬರಿಬ್ಬರು ರವಿವರ್ಮರನ್ನು ನಾವು ಸೃಷ್ಟಿಸಿರಬಹುದು. ಆದರೆ ಬಣ್ಣದ ಗೊಂಬೆಗಳನ್ನು ಮಾಡುವುದರಲ್ಲಿ ನಮ್ಮನರು ತುಂಬ್ ಉತ್ತಮರು. ಅವರು ತಮ್ಮ ಕೃತಿಯಲ್ಲಿ ಬಣ್ಣಗಳ ವೈವಿಧ್ಯತೆಯ ಪ್ರಕಾಶದಲ್ಲಿ ಧೈರ್ಯವನ್ನಾದರೂ ತೋರುವರು; ರವಿವರ್ಮನ ಚಿತ್ರವನ್ನು ನೋಡುವಾಗ ಒಬ್ಬ ನಾಚಿಕೆಯಿಂದ ತಲೆ ತಗ್ಗಿಸಬೇಕಾಗಿದೆ. ಅದಕ್ಕಿಂತ ಪೂರ್ವದಿಂದಲೂ ಬಂದಿರುವ ದುರ್ಗಾದೇವಿ ಬಣ್ಣದ ವಿಗ್ರಹ, ಜಯಪುರದ ಕೆಲವು ಚಿತ್ರಗಳು ಎಷ್ಟೋ ಉತ್ತಮ. ಪಾಶ್ಚಾತ್ಯಶಿಲ್ಪ ಮತ್ತು ಚಿತ್ರಕಲೆಗೆ ಸಂಬಂಧಿಸಿದ ನನ್ನ ಕೆಲವು ಭಾವನೆಗಳನ್ನು ಮುಂದೆ ಯಾವಾಗಲಾದರೊಮ್ಮೆ ವಿವರಿಸುತ್ತೇನೆ. ಅದು ದೊಡ್ಡ ವಿಷಯ, ಅದನ್ನು ಕುರಿತು ಇಲ್ಲಿ ವಿವರಿಸ ಲಾಗುವುದಿಲ್ಲ.


\section{ಆಹಾರ ಮತ್ತು ಅಡಿಗೆ}

ಅಡಿಗೆಯಲ್ಲಿ ವಿಲಾಯಿತಿ ಪದ್ಧಿತಿಯನ್ನು ಈಗ ಸ್ವಲ್ಪ ಕೇಳಿ. ನಮ್ಮಲ್ಲಿ ಅಡಿಗೆ ಸಮಯದಲ್ಲಿ ಗಮನಿಸುವ ಮಡಿ ಎಂಬ ಆಚಾರ ಇನ್ನೆಲ್ಲಿಯೂ ಇಲ್ಲ. ಆದರೆ ಬಡಿಸುವ ವಿಧಾನದಲ್ಲಿ ಇಂಗ್ಲಿಷಿನವರಲ್ಲಿ ಇರುವಂತೆ ಕ್ರಮ, ಶುಭ್ರತೆ ನಮ್ಮಲ್ಲಿ ಇಲ್ಲ. ಪ್ರತಿದಿನ ಬೆಳಿಗ್ಗೆ ನಮ್ಮ ಅಡಿಗೆಯವನು ಸ್ನಾನ ಮಾಡುವನು. ಅಡಿಗೆ ಮನೆಗೆ ಹೋಗುವಾಗ ಬಟ್ಟೆ ಬದಲಾಯಿಸುವನು. ಪಾತ್ರೆಯನ್ನೆಲ್ಲ ಚೆನ್ನಾಗಿ ತೊಳೆದು ಒಲೆಯನ್ನೂ ಶುದ್ಧಗೊಳಿಸುತ್ತಾನೆ. ಅಡಿಗೆ ಮಾಡುವಾಗ ತನ್ನ ದೇಹದ ಯಾವುದಾದರೂ ಭಾಗವನ್ನು ಕೈಯಿಂದ ಮುಟ್ಟಿದರೆ ಅದನ್ನು ಪುನಃ ತೊಳೆದಲ್ಲದೆ ಅದರಿಂದ ಸಾಮಾನನ್ನು ಮುಟ್ಟುವುದಿಲ್ಲ. ಪಾಶ್ಚಾತ್ಯ ಅಡಿಗೆಯವನು ಸ್ನಾನ ಮಾಡುವುದೇ ಅಪರೂಪ. ಅಡಿಗೆ ಮಾಡುವಾಗ ಸೌಟಿನಿಂದ ರುಚಿನೋಡಿ ಅದೇ ಸೌಟನ್ನು ಪಾತ್ರೆಯೊಳಗೆ ಅದ್ದಲು ಅನುಮಾನಿಸುವುದೇ ಇಲ್ಲ. ಮೂಗಿನಿಂದ ಬರುವ ಗೊಣ್ಣೆಯನ್ನು ಕರವಸ್ತ್ರದಿಂದ ಒರಸಿ ಅದೇ ಕೈಯಿಂದ ರೊಟ್ಟಿಯ ಹಿಟ್ಟನ್ನು ನಾದುವನು. ಹೊರಗಿನಿಂದ ಅಡಿಗೆ ಮನೆಗೆ ಬಂದರೆ ಕೈ ತೊಳೆಯಬೇಕೆಂಬ ಆಲೋಚನೆಯೇ ಇಲ್ಲ. ತಕ್ಷಣ ಅಡಿಗೆ ಶುರುಮಾಡುವನು. ಆದರೂ ಬೆಳ್ಳಗಿರುವ ಕೋಟು, ಟೋಪಿ, ಮೈಮೇಲೆ. ಅವನು ರೊಟ್ಟಿಗೆ ಕಲಸಿರುವ ಹಿಟ್ಟಿನ ಮೇಲೆ ಕುಣಿಯುತ್ತಿದ್ದರೂ ಇರಬಹುದು, ತನ್ನ ದೇಹದ ಭಾರದಿಂದ ಚೆನ್ನಾಗಿ ಮೃದು ವಾಗಲಿ ಎಂದು. ಆ ತಾಂಡವ ನೃತ್ಯದ ಸಮಯದಲ್ಲಿ ಮೈಬೆವರು ಅದಕ್ಕೆ ಬಿದ್ದರೂ ಬೀಳಬಹುದು.(ಅದೃಷ್ಟವಶಾತ್​ ಈಗ ಅದಕ್ಕೆ ಯಂತ್ರಗಳನ್ನು ಕಂಡು ಹಿಡಿದಿರುವರು.) ಇಷ್ಟೊಂದು ಅನಾಚಾರವಾಗಿ ರೊಟ್ಟಿ ಆದಮೇಲೆ ಅದನ್ನು ಶುಭ್ರವಾದ ತಟ್ಟೆಯಲ್ಲಿಟ್ಟು, ಮೇಲೆ ಬಿಳಿಯ ವಸ್ತ್ರದಿಂದ ಮುಚ್ಚಿರುವರು. ಅದನ್ನು ಶುಭ್ರವಾದ ಪೋಷಾಕು ಹಾಕಿಕೊಂಡ ಪರಿಚಾರಕ ತನ್ನ ಕೈಗೆ ಕೈಚೀಲಗಳನ್ನು ಧರಿಸಿ, ಆ ತಟ್ಟೆಯನ್ನು ಚೆನ್ನಾಗಿ ವಸ್ತ್ರದಿಂದ ಅಲಂಕರಿಸಿದ ಮೇಲೆ ಮೇಜಿನ ಮೇಲೆ ತಂದಿಡುವನು. ಇಲ್ಲಿ ಪರಿಚಾರಕ ಹಾಕಿಕೊಂಡಿರುವ ಕೈಚೀಲಗಳನ್ನು ಗಮನಿಸಿ. ಬರೇ ಬೆರಳು ಪದಾರ್ಥದ ಮೇಲೆ ಎಲ್ಲಿ ಬೀಳುವುದೋ ಎಂಬ ಅಂಜಿಕೆ!

ನಮ್ಮ ಆಚಾರವನ್ನು ಗಮನಿಸಿ. ನಮ್ಮ ಬ್ರಾಹ್ಮಣರ ಅಡಿಗೆಯವನು ಮೊದಲು ಸ್ನಾನಮಾಡಿಕೊಂಡು ತೊಳೆದ ಪಾತ್ರೆಯಲ್ಲಿ ಅಡಿಗೆ ಮಾಡುವನು. ಆದರೆ ಅವನು ಸಗಣಿಯಿಂದ ಸಾರಿಸಿದ ಬರೀ ನೆಲದ ಮೇಲೆ ಇರುವ ತಟ್ಟೆಯಲ್ಲಿ ಬಡಿಸುವನು. ಪ್ರತಿದಿನ ಅವನು ಬಟ್ಟೆಯನ್ನು ಒಗೆದಿದ್ದರೂ ಅದು ಒಗೆದಂತೆಯೇ ಕಾಣುವುದಿಲ್ಲ. ಕೆಲವು ವೇಳೆ ತಟ್ಟೆಗೆ ಬದಲು ಬಾಳೆಯ ಎಲೆಯನ್ನು ಹಾಕುವರು, ಅದು ಹರಿದು ಹೋಗಿದ್ದರೆ ಎಲೆ ಮೇಲಿರುವ ಕಲಸಿದ ಅನ್ನ ಕೆಳಗೆ ಇರುವ ಸಗಣಿಯೊಂದಿಗೆ ಬೆರೆತು ಒಂದು ಹೊಸ ರುಚಿಯನ್ನು ಕೊಡುವುದು.

ನಾವು ಚೆನ್ನಾಗಿ ಸ್ನಾನಮಾಡಿ ಆದ ಮೇಲೆ ಎಣ್ಣೆಯಿಂದ ಅಂಟುತ್ತಿರುವ ಕೊಳಕು ಬಟ್ಟೆಯನ್ನು ಹಾಕಿಕೊಳ್ಳುವೆವು. ಪಾಶ್ಚಾತ್ಯರು ಸ್ನಾನಮಾಡದೆ ಕೊಳೆದೇಹದ ಮೇಲೆ ಶುಭ್ರವಾದ ಬಟ್ಟೆಯನ್ನು ಧರಿಸುವರು. ಇದನ್ನು ನಾವು ಚೆನ್ನಾಗಿ ತಿಳಿದುಕೊಳ್ಳಬೇಕು, ಇಲ್ಲೇ ಇರುವುದು ಪಾಶ್ಚಾತ್ಯರಿಗೂ ಪೌರಸ್ತ್ಯರಿಗೂ ಇರುವ ಮುಖ್ಯ ವ್ಯತ್ಯಾಸ. ಹಿಂದೂಗಳ ಅಂತರ್​ ದೃಷ್ಟಿ ಪಾಶ್ಚಾತ್ಯರ ಬಹಿರ್​ ದೃಷ್ಟಿ ಅವರ ಆಚಾರ ವ್ಯವಹಾರಗಳಲ್ಲೆಲ್ಲಾ ಕಾಣುವುದು. ಹಿಂದೂ ಯಾವಾಗಲೂ ಒಳಗೆ ನೋಡುವನು, ಪಾಶ್ಚಾತ್ಯ ಹೊರಗೆ ನೋಡುವನು. ಹಿಂದೂ ವಜ್ರವನ್ನು ಹರಕು ಬಟ್ಟೆಯಲ್ಲಿ ಬಚ್ಚಿಡುವಂತೆ, ಪಾಶ್ಚಾತ್ಯ ಒಂದು ಹಿಡಿ ಮಣ್ಣನ್ನು ಚಿನ್ನದ ಪೆಟ್ಟಿಗೆಯಲ್ಲಿಡುವಂತೆ! ಹಿಂದೂ ತನ್ನ ದೇಹವನ್ನು ಶುದ್ಧವಾಗಿಡುವುದಕ್ಕೆ ಸ್ನಾನಮಾಡುವನು, ಅವನ ಬಟ್ಟೆ ಎಷ್ಟು ಕೊಳೆಯಾಗಿದ್ದರೂ ಚಿಂತೆಯಿಲ್ಲ. ಪಾಶ್ಚಾತ್ಯ ಶುಭ್ರವಾದ ಬಟ್ಟೆಯನ್ನು ಧರಿಸುವನು. ಒಳಗೆ ಎಷ್ಟು ಕೊಳೆಯಿದ್ದರೇ ನಂತೆ! ಹಿಂದೂ ತನ್ನ ಮನೆಯ ಒಳಗೆ ಕೋಣೆ, ನೆಲ, ಬಾಗಿಲು, ಎಲ್ಲವನ್ನೂ ಶುಭ್ರವಾಗಿಡುವನು. ಬಾಗಿಲಿನ ಹೊರಗೆ ಬೇಕಾದಷ್ಟು ಕಸದ ಗುಡ್ಡೆಯಿದ್ದರೂ ಚಿಂತೆಯಿಲ್ಲ! ಪಾಶ್ಚಾತ್ಯ ಮನೆಯ ಒಳಗಿನ ನೆಲಕ್ಕೆ ಸುಂದರವಾದ ಜಮ ಖಾನವನ್ನು ಹಾಸುವನು. ಆದರೆ ಕೆಳಗೆ ಇರುವ ಧೂಳು ಕೊಳೆ ಕಣ್ಣಿಗೆ ಕಾಣದೆ ಇದ್ದರೆ ಸರಿ! ಹಿಂದೂ ಮನೆಯ ಚರಂಡಿ ನೀರನ್ನು ರಸ್ತೆಗೆ ಬಿಡುವನು. ದುರ್ವಾಸನೆ ಪರವಾ ಇಲ್ಲ; ಪಾಶ್ಚಾತ್ಯ ದೇಶದಲ್ಲಿ ನೆಲದೊಳಗೆ ಚರಂಡಿ. ಅದೇ ಟೈಫಾಯಿಡ್​ ಜ್ವರಕ್ಕೆ ಮೂಲ. ಹಿಂದೂ ಒಳಗಿನದನ್ನು ತೊಳೆಯುವನು. ಪಾಶ್ಚಾತ್ಯ ಹೊರಗಿನದನ್ನು ತೊಳೆಯುವನು.

ನಮಗೆ ಬೇಕಾಗಿರುವುದು ಶುಭ್ರ ದೇಹ ಮತ್ತು ಶುಭ್ರ ವಸನ. ಬಾಯಿ ಮುಕ್ಕು ಳಿಸುವುದು, ಹಲ್ಲುಜ್ಜುವುದು ಇವನ್ನೆಲ್ಲ ಏಕಾಂತದಲ್ಲಿ ಮಾಡಬೇಕು. ವಾಸಿಸುವ ಮನೆ ನಿರ್ಮಲವಾಗಿರಬೇಕು. ಹೊರಗೆ ರಸ್ತೆ ಮುಂತಾದುವೂ ನಿರ್ಮಲವಾಗಿರ ಬೇಕು. ಅಡಿಗೆಯವನು ತನ್ನ ದೇಹ ಮತ್ತು ಬಟ್ಟೆಯನ್ನು ನಿರ್ಮಲವಾಗಿಟ್ಟಿರ ಬೇಕು. ಶುಭ್ರವಾದ ಸ್ಥಳದಲ್ಲಿ ಕುಳಿತುಕೊಂಡು ಆಹಾರವನ್ನು ಶುಚಿಯಾದ ತಟ್ಟೆ ಬಟ್ಟಲಿಂದ ತೆಗೆದುಕೊಳ್ಳಬೇಕು.ಆಚಾರವೇ ಧರ್ಮಕ್ಕೆ ಮೊದಲ ಮೆಟ್ಟಿಲು. ದೇಹ ಮತ್ತು ಮನಸ್ಸು ಶುದ್ಧವಾಗಿರಬೇಕು, ಎಲ್ಲದರಲ್ಲಿಯೂ ಶುಚಿಯಾಗಿರ ಬೇಕು. ಇದೇ ಮುಖ್ಯ. ಆಚಾರಭ್ರಷ್ಟನಿಗೆ ಧರ್ಮ ದೊರಕುವುದೇ? ಆಚಾರ ಭ್ರಷ್ಟ ನಿಗೆ ಒದಗುವ ದುರ್ಗತಿಯನ್ನು ನಿಮ್ಮ ಕಣ್ಣುಮುಂದಯೇ ನೋಡುವುದಿಲ್ಲವೇ? ನಾವು ಇಷ್ಟು ಅನುಭವಿಸಿದರೂ ಬುದ್ದಿ ಕಲಿಯಬೇಡವೇ? ಕಾಲರಾ, ಮಲೇರಿಯಾ, ಪ್ಲೇಗಿಗೆ ಇಂಡಿಯ ತವರೂರಾಗಿದೆ. ಲಕ್ಷಾಂತರ ಜನರನ್ನು ಆಹುತಿ ತೆಗೆದುಕೊಳ್ಳುತ್ತಿವೆ. ಇದು ಯಾರ ದೋಷ? ನಮ್ಮ ದೋಷ. ನಾವು ಮಹಾ ಅನಾಚಾರಿಗಳು.

ಆಹಾರ ಶುದ್ಧಿಯಿಂದ ಮನಸ್ಸು ಶುದ್ಧಿಯಾಗುವುದು, ಮನಶ್ಶುದ್ಧಿಯಾದರೆ ಆತ್ಮದ ಸ್ಮೃತಿ ಸ್ಥಿರವಾಗುವುದು. ಈ ಶಾಸ್ತ್ರವಾಕ್ಯವನ್ನು ಎಲ್ಲಾ ಸಂಪ್ರದಾಯದವರೂ ಒಪ್ಪುವರು. ಶಂಕರಾಚಾರ್ಯರು ಆಹಾರವೆಂದರೆ ಇಂದ್ರಿಯದ ಮೂಲಕ ತೆಗೆದುಕೊಳ್ಳುವುದು ಎನ್ನುವರು. ರಾಮನುಜಾಚಾರ್ಯರು ಆಹಾರವೆಂದರೆ ಭೋಜ್ಯ ವಸ್ತು ಎನ್ನುವರು. ಇವೆರಡೂ ಸರಿ ಎಂದು, ಎರಡನ್ನೂ ಸ್ವೀಕರಿಸಬೇಕೆಂಬುದು ಪ್ರಾಜ್ಞರ ಅಭಿಪ್ರಾಯ. ಸರಿಯಾದ ಆಹಾರವಿಲ್ಲದೆ ಇದ್ದರೆ ಇಂದ್ರಿಯ ತನ್ನ ಕೆಲಸವನ್ನು ಹೇಗೆ ಮಾಡಬಲ್ಲುದು? ಅಶುದ್ಧ ಆಹಾರ ಇಂದ್ರಿಯ ಗ್ರಹಣಶಕ್ತಿ, ಪಟುತ್ವಗಳನ್ನು ತಗ್ಗಿಸುವುದು, ಇಲ್ಲವೆ ಅದು ತನಗೆ ವಿರೋಧವಾಗಿ ಹೋಗು ವಂತೆ ಮಾಡುವುದು. ಅಜೀರ್ಣ ದೋಷದಿಂದ ಒಂದು ಇನ್ನೊಂದರ ಭ್ರಾಂತಿಯನ್ನು ಹುಟ್ಟಿಸುವುದು. ಉಪವಾಸ ಮಾಡಿದರೆ ದೃಷ್ಟಿ ಮುಂತಾದ ಇಂದ್ರಿಯ ಗ್ರಹಣ ಕಡಿಮೆಯಾಗುವುದು. ಇದು ಎಲ್ಲರಿಗೂ ಗೊತ್ತಾದ ವಿಷಯ. ಕೆಲವು ಆಹಾರ ಮನಸ್ಸು ಮತ್ತು ದೇಹದ ಮೇಲೆ ಬೇರೆ ಬೇರೆ ಪರಿಣಾಮವನ್ನು ಉಂಟುಮಾಡುವುದು. ಈ ನಿಯಮವೇ ಹಿಂದೂಗಳ ಆಹಾರದ ವಿಧಿ ನಿಷೇಧದ ಹಿಂದೆ ಇರುವುದು. ಅನೇಕವೇಳೆ ಇದರ ಹಿಂದೆ ಇರುವ ಅರ್ಥ ಮರೆತುಹೋಗಿ ಹೊರಗಿನ ಕರಟಕ್ಕೆ ಜಗಳ ಕಾಯುತ್ತಾ ಬಹಳ ಜೋಪಾನದಿಂದ ಅದನ್ನು ಕಾಯುತ್ತಿರುವೆವು.

ರಾಮಾನುಜಾಚಾರ್ಯರು ಖಾದ್ಯ ಪದಾರ್ಥಗಳಲ್ಲಿ ಮೂರು ದೋಷಗಳನ್ನು ಹೇಳುವರು. ಜಾತಿದೋಷ: ಇದು ತಿನ್ನುವ ಪದಾರ್ಥದಲ್ಲಿದೆ. ಈರುಳ್ಳಿ, ಬೆಳ್ಳುಳ್ಳಿ ಇತ್ಯಾದಿ. ಇದನ್ನು ಹೆಚ್ಚು ತಿಂದರೆ ಮನಸ್ಸು ಅಸ್ಥಿರವಾಗಿ ಬುದ್ಧಿ ಭ್ರಮಣವಾಗು ವುದು! ಆಶ್ರಯ ದೋಷವೆಂದರೆ ಯಾವ ವ್ಯಕ್ತಿಯ ಮೂಲಕ ಆಹಾರ ಬರುವುದೋ ಅದು. ದುಷ್ಟನಿಂದ ಬರುವ ಅನ್ನವನ್ನು ಉಂಡರೆ ದುಷ್ಟಬುದ್ಧಿ ಹೆಚ್ಚುವುದು. ಉತ್ತಮನು ಕೊಡುವ ಅನ್ನವನ್ನು ಉಂಡರೆ ಸದ್ಬುದ್ಧಿ ಬರುವುದು. ನಿಮಿತ್ತ ದೋಷ ಎಂದರೆ ಕೊಳೆ, ಕ್ರಿಮಿ, ಕೂದಲು ಇದರಿಂದ ಕೂಡಿದ ಆಹಾರ. ಇದನ್ನು ಸೇವಿಸಿದರೆ ಮನಸ್ಸು ಅಪವಿತ್ರವಾಗುವುದು. ಇದರಲ್ಲಿ ಜಾತಿ ಮತ್ತು ನಿಮಿತ್ತ ದೋಷದಿಂದ ಪಾರಾಗುವುದು ಸುಲಭ. ಆದರೆ ಆಶ್ರಯದೋಷದಿಂದ ಪಾರಾಗುವುದು ಬಹಳ ಕಷ್ಟ. ಈ ಆಶ್ರಯದೋಷದಿಂದ ಪಾರಾಗುವುದಕ್ಕೆ ನಮ್ಮಲ್ಲಿ ಅಸ್ಪೃಶ್ಯತೆ ಇರುವುದು. ಅನೇಕ ವೇಳೆ ಇದನ್ನು ತಪ್ಪು ತಿಳಿದಿಕೊಂಡು ಇದೊಂದು ವಿಪರೀತ ಲೋಕಾಚಾರವಾಗುವುದು. ಈ ಲೋಕಾಚಾರವನ್ನು ಬಿಟ್ಟು ಮಹಾ ಪುರುಷರು ಯಾವ ರೀತಿ ಆಚರಿಸುವರೋ ಅದನ್ನು ಅನುಸರಿಸು ವುದು ಮೇಲು. ಚೈತನ್ಯಾದಿ ಜಗದ್ಗುರುಗಳ ಚರಿತ್ರೆಯನ್ನು ಓದಿ. ಈ ಸಂಬಂಧ ದಲ್ಲಿ ಅವರು ಹೇಗೆ ವ್ಯವಹರಿಸುವರು ಎಂಬುದನ್ನು ಗಮನಿಸಿ. ಆಹಾರದಲ್ಲಿ ಜಾತಿದೋಷಕ್ಕೆ ಹಿಂದೂ ಗಮನ ಕೊಡುವಷ್ಟು ಪೃಥ್ವಿಯಲ್ಲಿ ಮತ್ತಾರೂ ಕೊಡುವುದಿಲ್ಲ. ಪ್ರಪಂಚದಲ್ಲಿ ಯಾರೂ ನಮ್ಮಷ್ಟು ಜಾತಿ ದೋಷದಿಂದ ಪಾರಾದ ಅನ್ನವನ್ನು ತಯಾರು ಮಾಡುವುದಿಲ್ಲ. ಆದರೆ ನಿಮಿತ್ತ ದೋಷ ಭಯಾನಕ ರೂಪವನ್ನು ತಾಳುತ್ತಿರುವುದರಿಂದ ಅದಕ್ಕೆ ಗಮನ ಕೊಡುವುದು ಒಳ್ಳೆಯದು. ಮಿಠಾಯಿಯನ್ನು ಅಂಗಡಿಯಲ್ಲಿ ಕೊಳ್ಳುವುದು ರೂಢಿಯಾಗಿದೆ. ನಿಮಿತ್ತ ದೋಷದಿಂದ ಇದು ಎಷ್ಟು ಅಪವಿತ್ರ ಎಂಬುದನ್ನು ನೀವೇ ಊಹಿಸಿಕೊಳ್ಳ ಬಹುದು. ಅದನ್ನು ಎಲ್ಲರಿಗೂ ಕಾಣುವಂತೆ ಪ್ರದರ್ಶಿಸುವರು. ಅದರ ಮೇಲೆ ಕ್ರಿಮಿಕೀಟ, ಧೂಳು, ಕೊಳೆ ಕುಳಿತಿರುವುದು, ಅದೂ ಅಲ್ಲದೆ ಅದು ಎಷ್ಟು ದಿನದ ಸಾಮಗ್ರಿಯೋ ಹೇಳತೀರದು. ಪ್ರತಿಯೊಂದು ಮನೆಯಲ್ಲಿಯೂ ಕಾಣುವ ಅಗ್ನಿಮಾಂದ್ಯ, ನಗರಗಳ ಜನರಲ್ಲಿ ಸಾಮಾನ್ಯವಾಗಿರುವ ಮಧುಮೇಹ ಇವೇ ಇದರ ಪ್ರತಿಫಲ. ಗ್ರಾಮವಾಸಿಗಳಲ್ಲಿ ಈ ರೋಗ ಇಲ್ಲದೇ ಇರುವುದಕ್ಕೆ ಕಾರಣ, ಇಂತಹ ವಿಷ ಪದಾರ್ಥಗಳನ್ನು ತಮ್ಮ ಇಚ್ಚೆ ಬಂದಾಗ ಕೊಂಡುಕೊಳ್ಳುವತಂಹ ಅಂಗಡಿ ಇಲ್ಲದೇ ಇರುವುದು. ಇದನ್ನು ಮುಂದೆ ವಿಶದವಾಗಿ ಹೇಳುತ್ತೇನೆ.

ಆಹಾರದ ವಿಷಯದಲ್ಲಿ ಹಿಂದೆ ಇದ್ದ ಪ್ರಾಚೀನ ನಿಯಮಗಳು ಮೇಲಿನವು.ಆದರೆ ಈಗಲೂ ಎಷ್ಟೋ ಭಿನ್ನಾಭಿಪ್ರಾಯಗಳಿವೆ. ಮಾಂಸಾಹಾರ ವನ್ನು ಸ್ವೀಕರಿಸುವುದು ಒಳ್ಳೆಯದೆ, ಕೆಟ್ಟದ್ದೆ ಅಥವಾ ಬರೀ ಶಾಖಾಹಾರದ ಮೇಲೆ ಜೀವಿಸಬೇಕೆ, ಮಾಂಸಾಹಾರದಿಂದ ಪ್ರಯೋಜನವುಂಟೆ ಎಂಬುದರ ವಿಷಯದಲ್ಲಿ ಹಲವು ಭಿನ್ನಾಭಿಪ್ರಾಯಗಳಿವೆ. ಒಂದು ಪಕ್ಷದವರು ಯಾವ ಪ್ರಕಾರದಿಂದಲೂ ಜೀವಹತ್ಯೆ ಮಾಡಕೂಡದು ಎನ್ನುವರು. ಮತ್ತೊಬ್ಬರು ನಿಮ್ಮ ಮಾತನ್ನು ನಿಲ್ಲಿಸಿ, ಹತ್ಯೆ ಇಲ್ಲದೆ ಇದ್ದರೆ ಬದುಕುವುದಕ್ಕೆ ಸಾಧ್ಯವಿಲ್ಲ ಎನ್ನುವರು. ಶಾಸ್ತ್ರಾಭಿಪ್ರಾಯ ದಲ್ಲಿ ಏಕಮತವಿಲ್ಲ. ಒಂದು ಕಡೆ ಯಜ್ಞದಲ್ಲಿ ಹತ್ಯಮಾಡಿ, ಉಳಿದ ಕಡೆ ಕೂಡದು ಎನ್ನುವುದು. ಹಿಂದೂಶಾಸ್ತ್ರದ ಪ್ರಕಾರ ಯಜ್ಞಸ್ಥಳ ಒಂದರಲ್ಲಿ ಬಿಟ್ಟು ಉಳಿದ ಕಡೆ ಪ್ರಾಣಿಹತ್ಯೆ ಮಾಡಿದರೆ ಪಾಪ. ಯಜ್ಞದಲ್ಲಿ ಬಲಿಕೊಟ್ಟಾದ ಮೇಲೆ ಸಂತೋಷದಿಂದ ಆ ಮಾಂಸವನ್ನು ಭಕ್ಷಿಸಬಹುದು. ಇಷ್ಟೇ ಅಲ್ಲ. ಶ್ರಾದ್ಧ ಮುಂತಾದ ಸಮಯದಲ್ಲಿ ಗೃಹಸ್ಥರು ಬಲಿ ಕೊಡಬೇಕೆಂದೂ, ಹಾಗೆ ತಪ್ಪಿದರೆ ಪಾಪ ಬರುವುದೆಂದೂ ಸಾರುವುದು. ಶ್ರಾದ್ಧ, ಮುಂತಾದ ಸಮಯದಲ್ಲಿ ಬಡಿಸಿದ ಮಾಂಸಾಹಾರವನ್ನು ಭಕ್ಷಿಸದೇ ಇದ್ದರೆ ಮುಂದಿನ ಜನ್ಮಗಳಲ್ಲಿ ಮೃಗಗಳಾಗಿ ಹುಟ್ಟುವರೆಂದು ಮನು ಹೇಳುವನು. ಆದರೆ ಜೈನ, ಬೌದ್ಧ ಮತ್ತು ವೈಷ್ಣವರು ಅಂತಹ ಶಾಸ್ತ್ರವನ್ನು ನಾವು ನಂಬುವುದಿಲ್ಲವೆಂದು ತಿರಸ್ಕರಿಸಿದರು. ಯಾವ ಸಮಯದಲ್ಲೂ ಪ್ರಾಣಿವಧೆಯನ್ನು ಮಾಡಬಾರದೆಂಬುದೇ ಅವರ ಮತ. ಯಾರು ಯಜ್ಞಗಳನ್ನು ಮಾಡುತ್ತಿದ್ದರೊ ಮತ್ತು ಯಾರು ಅತಿಥಿಗಳಿಗೆ ಮಾಂಸಾಹಾರವನ್ನು ಕೊಡುತ್ತಿದ್ದರೊ ಅವರನ್ನು ಅಶೋಕನು ಶಿಕ್ಷಿಸುತ್ತಿದ್ದ ಎಂಬುದನ್ನು ಕೇಳುವೆವು. ಆಧುನಿಕ ವೈಷ್ಣವರ ರಾಮ, ಕೃಷ್ಣ ಮುಂತಾದವರೆಲ್ಲ ಮದ್ಯ, ಮಾಂಸವನ್ನು ಸೇವಿಸುತ್ತಿದ್ದರೆಂಬುದನ್ನು ರಾಮಾಯಣ ಮಹಾಭಾರತಗಳಲ್ಲಿ ನೋಡುತ್ತೇವೆ.

\begin{verse}
ಸೀತಾಮಾದಾಯ ಬಾಹುಭ್ಯಾಂ ಮಧು ಮೈರೇಯಕಂ ಶುಚಿ~॥\\ಪಾಯಯಾಮಾಸ ಕಾಕುಸ್ಸ್ಥಃ ಶಚೀಮಿಂದ್ರೋ ಯಥಾಮೃತಮ್​~॥\\ಮಾಂಸಾನಿ ಚ ಸಮೃಷ್ಟಾನಿ ವಿವಿಧಾನಿ ಫಲಾನಿ ಚ~॥\\ರಾಮಸ್ಯಾಭ್ಯವಹಾರಾರ್ಥಂ ಕಿಂಕರಾಸ್ತೂರ್ಣಮಾಹರನ್​~॥
\end{verse}

\begin{flushright}
(ರಾಮಾಯಣ: ಉತ್ತರಕಾಂಡ ೪೨)
\end{flushright}

“ಶ‍್ರೀರಾಮನು ಸೀತೆಯನ್ನು ಎರಡು ಬಾಹುಗಳಿಂದ ಆಲಿಂಗಿಸಿ ಶುದ್ಧ ಮೈರೇಯ ಮದ್ಯವನ್ನು, ಇಂದ್ರನು ಶಚಿಗೆ ಅಮೃತವನ್ನು ನೀಡುವಂತೆ, ನೀಡಿದನು. ಪರಿಚಾರಕರು ಶ‍್ರೀರಾಮನಿಗೆ ಹಲವು ಬಗೆಯ ಮಾಂಸ ಮತ್ತು ಫಲಗಳನ್ನು ಬಡಿಸಿದರು.”

\begin{verse}
ಸುರಾಘಟಸಹಸ್ರೇಣ ಮಾಂಸಭೂತೌದನೇನ ಚ\\ಯಕ್ಷ್ಯೇ ತ್ವಾಂ ಪ್ರೀಯತಾಂ ದೇವಿ ಪುರೀಂ ಪುನರುಪಾಗತಾ~॥
\end{verse}

\begin{flushright}
(ರಾಮಾಯಣ: ಅಯೋಧ್ಯಾಕಾಂಡ ೫೨)
\end{flushright}

“ಮಹಾತಾಯಿ ಗಂಗೆ, ದಯೆತಾಳು, ನಾನು ಹಿಂತಿರುಗಿ ಬರುವಾಗ ಮದ್ಯದ ಸಾವಿರ ಹಂಡೆಗಳನ್ನು, ಬೇಕಾದಷ್ಟು ಮಾಂಸದೊಂದಿಗೆ ಬೇಯಿಸಿದ ಅನ್ನವನ್ನು ಅರ್ಪಿಸುತ್ತೇನೆ” ಎಂದು ಸೀತಾದೇವಿ ಗಂಗಾನದಿಯನ್ನು ದಾಟುವ ಕಾಲದಲ್ಲಿ ಪ್ರಾರ್ಥಿಸಿದಳು.

\begin{verse}
ಉಭೌ ಮಧ್ವಾಸವಕ್ಷೀಪ್ತಾವುಭೌ ಚಂದನರೂಷಿತೌ~॥\\ಸ್ರಗ್ವಿನೌ ವರವಸ್ತ್ರೌ ತೌ ದಿವ್ಯಾಭರಣ ಭೂಷಿತೌ~॥
\end{verse}

“ಕೃಷ್ಣಾರ್ಜುನರಿಬ್ಬರೂ ಗಂಧ ಲೇಪಿಸಿಕೊಂಡು ಪುಷ್ಪದಿಂದ ಅಲಂಕೃತ ರಾಗಿ ಸುಂದರ ವಸ್ತ್ರಗಳನ್ನು ಧರಿಸಿ ಮಧುರವಾದ ಮದ್ಯವನ್ನು ಸೇವಿಸಿದ್ದುದನ್ನು ನಾನು ಕಂಡೆನು” (ಮಹಾಭಾರತ, ಉದ್ಯೋಗಪರ್ವ, ಅದ್ಯಾಯ ೫೮; ೫)

ಪಾಶ್ಚಾತ್ಯರಲ್ಲಿ ಮಾಂಸಾಹಾರವು ದೇಹಕ್ಕೆ ಒಳ್ಳೆಯದೆ, ಕೆಟ್ಟದ್ದೆ, ಇದು ಶಾಕಾ ಹಾರಕ್ಕಿಂತ ಹೆಚ್ಚು ಶಕ್ತಿವರ್ಧಕವೆ ಇಲ್ಲವೆ ಎಂಬುದು ಪ್ರಶ್ನೆ. ಒಂದು ಪಕ್ಷದವರು ಯಾರು ಮಾಂಸಾಹಾರವನ್ನು ಸೇವಿಸುವರೋ ಅವರು ರೋಗದಿಂದ ನರಳುವರೆಂದು ಹೇಳುವರು. ಮತ್ತೊಂದು ಪಕ್ಷದವರು ಇದನ್ನು ವಿರೋಧಿಸಿ ಇದೆಲ್ಲ ಬರೀ ಭ್ರಾಂತಿ, ಇದು ಸತ್ಯವಾಗಿದ್ದರೆ ಹಿಂದೂಗಳೇ ಪ್ರಪಂಚದಲ್ಲಿ ಎಲ್ಲರಿಗಿಂತ ದೃಢಕಾಯರಾಗಿ ರ ಬೇಕಿತ್ತು; ಮಾಂಸಾಹಾರವೇ ಮುಖ್ಯವಾದ ಇಂಗ್ಲೀಷರು, ಅಮೇರಿಕನ್ನರು ಮುಂತಾದವರು ಹಲವು ರೋಗ ರುಜಿನಗಳಿಗೆ ತುತ್ತಾಗಿ ಎಂದೊ ನಾಶವಾಗಿ ಹೋಗಬೇಕಾಗಿತ್ತು- ಎಂದು ವಾದಿಸುವರು. ಕೆಲವರು ಮೇಕೆಮಾಂಸ ತಿಂದರೆ ಬುದ್ಧಿ ಹಾಗೆಯೇ ಆಗುವುದೆಂದೂ, ಹಂದಿಯದು ತಿಂದರೆ ಅದರಂತೆ ಆಗುವು ದೆಂದೂ, ಮೀನಿನ ಮಾಂಸ ತಿಂದರೆ ಅದರಂತೆ ಆಗುವುದೆಂದೂ ಹೇಳುವರು. ಮತ್ತೊಂದು ಪಕ್ಷದವ್ಧರು ಈ ವಾದ್ಧಸ್ಧರ್ಧಣ್ಧಿಯ್ಧನ್ನೇ ಉಪ್ಧಯ್ಧೋಗ್ಧಿಸ್ಧುತ್ತ ಆಲ್ಧೂಗ್ಧಡ್ಡೆ ತಿಂದ್ಧರೆ ಮೆದ್ಧ್ಧುಳು ಹಾಗೇ ಆಗುವುದೆಂದೂ, ಶಾಕಾಹಾರವನ್ನು ತಿಂದರೆ ಮೆದುಳು ಅದರಂತೆ ಜಡವಾಗುವುದೆಂದೂ, ಹೇಳುವರು. ಪ್ರಾಣಿಯ ಮೆದುಳನ್ನು ಹೊಂದು ವುದು ಜಡವನ್ನು ಹೋಲುವ ತರಕಾರಿಗಿಂತ ಮೇಲಲ್ಲವೆ ಎನ್ನುವರು. ಕೆಲವರು ಮಾಂಸಾಹಾರದಲ್ಲಿರುವ ಗುಣವೇ ಶಾಕಾಹಾರದಲ್ಲಿಯೂ ಇರುವುದೆಂದು ಹೇಳು ವರು. ಮತ್ತೊಬ್ಬರು ಅದನ್ನು ಹಾಸ್ಯ ಮಾಡುತ್ತ ಹೌದು, ಅದು ಗಾಳಿಯಲ್ಲೂ ಇದೆ, ಅದನ್ನೇ ಏತಕ್ಕೆ ಸೇವಿಸಿ ಜೀವಿಸಬಾರದು ಎನ್ನುವರು. ಕೆಲವರು ಶಾಕಾಹಾರಿ ಗಳು ಶ್ರಮಸಹಿಷ್ಣುಗಳು, ಬೇಕಾದಷ್ಟು ಕೆಲಸ ಮಾಡಬಲ್ಲರು ಎನ್ನುವರು. ಪ್ರತಿ ಪಕ್ಷದವರು ಅದು ಸತ್ಯವಾಗಿದ್ದರೆ ಇಂದು ಪ್ರಪಂದಲ್ಲಿ ಅವರೇ ಅಗ್ರಗಣ್ಯರಾಗಿರ ಬೇಕಾಗಿತ್ತು. ಆದರೆ ಹಾಗಿಲ್ಲ; ಯಾರು ಮಾಂಸಾಹಾರಿಗಳೊ ಅವರು ಪ್ರಮುಖ ವಾದ ಸ್ಥಾನಗಲ್ಲಿರುವರು. ಯಾರು ಮಾಂಸಾಹಾರವನ್ನು ಅನುಮೋದಿಸುವರೊ ಅವರು ಹಿಂದೂ ಮತ್ತು ಚೈನೀಯರನ್ನು ನೋಡಿ ಎಷ್ಟು ಬಡವರಾಗಿ ರುವರು ಎನ್ನುವರು. ಅವರು ಮಾಂಸ ತಿನ್ನುವುದಿಲ್ಲ.ಅಕ್ಕಿ ಮತ್ತು ಸಿಕ್ಕಿದ ಕೆಲವು ತರಕಾರಿಗಳ ಮೇಲೆ ಜೀವಿಸುವುದರಿಂದ ಇಂತಹ ದುಃಸ್ಥಿತಿಯಲ್ಲಿರುವರು. ಹಿಂದೆ ಜಪಾನಿಯರೂ ಅದೇ ಸ್ಥಿತಿಯಲ್ಲಿದ್ದರು. ಆದರೆ ಮಾಂಸಾಹಾರವನ್ನು ಪ್ರಾರ್ಧಂಭ್ಧಿಸ್ಧಿದ್ಧೊಡ್ಧನ್ಧೆಯೇ ಅವ್ಧರ ಚರ್ಧಿತ್ರ್ಧೆಯ್ಧಲ್ಲಿ ಹೊಸ ಅಧ್ಯ್ಧಾಯ ಪ್ರಾರಂಭವಾಯಿತು. ಇಂಡಿಯಾ ಸೈನ್ಯದಳದಲ್ಲಿ ಸುಮಾರು ಒಂದೂವರೆ ಲಕ್ಷ ಭಾರತೀಯ ಸಿಪಾಯಿ ಗಳಿರುವರು. ಅವರಲ್ಲಿ ಎಷ್ಟು ಜನ ಶಾಕಾಹಾರಿಗಳು? ವಿಚಾರಿಸಿ. ಅವರಲ್ಲಿ ಬಲಾಢ್ಯರಾದ ಸಿಕ್ಕರು ಮತ್ತು ಗೂರ್ಖರು ಎಂದೂ ಶಾಕಾಹಾರಿಗಳಲ್ಲ. ಒಬ್ಬರು ಅಜೀರ್ಣಕ್ಕೆ ಮಾಂಸವೇ ಕಾರಣ ಎನ್ನುವರು. ಮತ್ತೊಬ್ಬರು ಅದೆಲ್ಲ ಭ್ರಾಂತಿ, ಹೊಟ್ಟೆಯ ಜಾಡ್ಯದಿಂದ ನರಳುವವರೆಲ್ಲ ಶಾಕಾಹಾರಿಗಳು ಎನ್ನುವರು. ಶಾಕಾ ಹಾರದಿಂದ ಸುಲಭವಾಗಿ ಮಲವಿಸರ್ಜನೆಯಾಗಬಹುದು. ಆದರೆ, ಅದಕ್ಕಾಗಿ ಪ್ರಪಂಚದಲ್ಲಿರುವವರೆಲ್ಲ ಶಾಕಾಹಾರವನ್ನೇ ತೆಗೆದುಕೊಳ್ಳಬೇಕೆ?

ಬಹುದಿನದಿಂದಲೂ ಮಾಂಸಾಹಾರಿಗಳು ಹೆಚ್ಚು ಯೋಧರು ಮತ್ತು ಚಿಂತನ ಶೀಲರು ಆಗಿರುವರು. ಭರತಖಂಡದಲ್ಲಿ ಎಂದು ಹೋಮದಿಂದ ಧೂಮವೇಳು ತ್ತಿತ್ತೋ, ಅವರು ಯಜ್ಞಶಿಷ್ಟ ಮಾಂಸವನ್ನು ತಿನ್ನುತ್ತಿದ್ದರೊ, ಆಗ ಹಿಂದುಗಳು ಪರಾಕ್ರಮಶಾಲಿಗಳೂ ವಿದ್ವಾಂಸರೂ ಆಗಿದ್ದರು ಎಂಬುದು ಮಾಂಸಾಹಾರಿಗಳ ವಾದ. ಎಂದು ಶಾಕಾಹಾರಿಗಳಾದ ಬಾಬಾಜಿಗಳಾದರೋ ಅಂದಿನಿಂದ ಅವರಲ್ಲಿ ಹೆಚ್ಚು ವೀರರಾಗಲಿ, ವಿದ್ವಾಂಸರಾಗಲಿ ಇಲ್ಲವೆನ್ನುವರು. ಈ ಅಭಿಪ್ರಾಯವನ್ನು ಅನುಸರಿಸಿ ನಮ್ಮ ದೇಶದ ಕೆಲವು ಮಾಂಸಾಹಾರಿಗಳು ಅದನ್ನು ಬಿಟ್ಟಿಲ್ಲ. ಆರ್ಯಸಮಾಜದಲ್ಲಿ ಈ ವಿವಾದ ನಡೆಯುತ್ತಿದೆ. ಒಂದು ಪಕ್ಷದವರು ಅವಶ್ಯಕವಾಗಿ ಮಾಂಸ ಸೇವಿಸಬೇಕೆಂದೂ, ಮತ್ತೊಬ್ಬರು ಅದನ್ನು ಎಂದೂ ಸೇವಿಸಬಾರದು, ಅಧರ್ಮವೆಂದೂ ಎನ್ನುವರು. ಇದರಿಂದ ಪರಸ್ಪರ ದ್ವೇಷ, ಅಸೂಯೆ ಕಲಹ ಉತ್ಪನ್ನವಾಗುತ್ತದೆ. ಎಲ್ಲ ಪಕ್ಷದವರನ್ನು ಕೇಳಿ ಆದಮೇಲೆ ನನ್ನ ಮನಸ್ಸಿನ ಅಭಿಪ್ರಾಯವಿದು: ಹಿಂದೂ ಸರಿಯಾದ ಮಾರ್ಗದಲ್ಲಿರುವನು, ಅಂದರೆ, ಹಿಂದೂಶಾಸ್ತ್ರದ ಸಿದ್ಧಾಂತ ಹೇಳುವುದು ಆಹಾರ ಒಬ್ಬನ ಜನ್ಮಕರ್ಮದ ಮೇಲೆ ನಿಂತಿರಬೇಕು ಎನ್ನುವುದು ಸರಿ. ಆದರೆ ಈಗಿನ ಹಿಂದೂಗಳು ಶಾಸ್ತ್ರವನ್ನು ಅನುಸರಿಸುತ್ತಿಲ್ಲ; ಅಥವಾ ಹಿಂದಿನ ಆಚಾರ್ಯರನ್ನೂ ಅನುಸರಿಸುತ್ತಿಲ್ಲ.

ಮಾಂಸಾಹಾರ ಬರ್ಬರ, ಶಾಕಾಹಾರ ಪವಿತ್ರವೇನೋ ನಿಜ. ಇದನ್ನು ಯಾರೂ ಚರ್ಚಿಸುವುದಿಲ್ಲ. ಯಾರು ಆಧ್ಯಾತ್ಮಿಕ ಜೀವನವನ್ನು ನಡೆಸಬೇಕು ಎಂದು ಇಚ್ಚಿಸುವರೋ ಅವರಿಗೆ ಶಾಕಾಹಾರ ಒಳ್ಳೆಯದು. ಆದರೆ ಯಾರು ಹಗಲು ರಾತ್ರಿ ಜೀವನದಲ್ಲಿ ದುಡಿಯಬೇಕೊ ಅವರು ಮಾಂಸಾಹಾರವನ್ನು ಸೇವಿಸಬೇಕು. ಎಲ್ಲಿಯವರೆಗೂ ಸಮಾಜದಲ್ಲಿ ಬಲವಂತನಿಗೆ ಜಯ ವೆಂಬುದು ಇದೆಯೊ ಅಲ್ಲಿಯವರೆವಿಗೂ ಮಾಂಸಾಹಾರಿಯಾಗಿರಬೇಕು. ಇಲ್ಲದೆ ಇದ್ದರೆ ದುರ್ಬಲರು ಬಲಾಢ್ಯರ ಪಾದದಡಿ ಸಿಕ್ಕಿ ನಾಶವಾಗುವರು. ಎಲ್ಲೋ ಕೆಲವು ವ್ಯಕ್ತಿಗಳಿಗೆ ಶಾಕಾಹಾರದಿಂದ ಪ್ರಯೋಜನವಾಗಿದೆಯೆಂದು ಹೇಳಿದರೆ ಸಾಲದು. ಒಂದು ದೇಶವನ್ನು ಮತ್ತೊಂದು ದೇಶದೊಂದಿಗೆ ಹೋಲಿಸಿ ನಿರ್ಣಯಕ್ಕೆ ಬನ್ನಿ.

ಶಾಕಾಹಾರಿಗಳಲ್ಲೂ ಭಿನ್ನಾಭಿಪ್ರಾಯವಿದೆ. ಅಕ್ಕಿ, ಆಲೂಗಡ್ಡೆ, ಗೋಧಿ, ಬಾರ್ಲಿ ಮುಂತಾದ ಪಿಷ್ಟಪ್ರಧಾನ ಆಹಾರದಿಂದ ಪ್ರಯೋಜವಿಲ್ಲ, ಇವೆಲ್ಲ ರೋಗಕಾರಿ. ಯಾವ ಆಹಾರ ದೇಹದಲ್ಲಿ ಸಕ್ಕರೆಯನ್ನು ಉತ್ಪತ್ತಿಮಾಡುವುದೊ ಅದು ಆರೋಗ್ಯಕ್ಕೆ ಅಪಾಯಕಾರಿ. ಕುದುರೆ ಮತ್ತು ದನಗಳಿಗೆ ಇವನ್ನು ಕೊಟ್ಟರೂ ಅವುಗಳಿಗೆ ರೋಗ ಬರುವುದು. ಆದರೆ ಅವನ್ನೇ ಹೊರಗೆ ಹುಲ್ಲು ಮೇಯಲು ಬಿಟ್ಟರೆ ಪುನಃ ಮೇಲಾಗುವುವು. ಹುಲ್ಲು ಮತ್ತು ಇತರ ಹಸಿ ಪದಾರ್ಥದಲ್ಲಿ ಪಿಷ್ಟಸಾಮಗ್ರಿ ಕಡಿಮೆ. ವನಚರ ಪ್ರಾಣಿಗಳು ಬೀಜ ಮತ್ತು ಹಸಿರು ಪದಾರ್ಥವನ್ನು ಸೇವಿಸುವುವು. ಗೋಧಿ, ಆಲೂಗೆಡ್ಡೆ ಮುಂತಾದವುಗಳನ್ನು ಸೇವಿಸುವುದಿಲ್ಲ. ಒಂದು ವೇಳೆ ಹಾಗೆ ಸೇವಿಸಿದರೆ ಅವು ಇನ್ನೂ ಮಾಗದೆ ಹಸಿ ಇರುವಾಗ. ಆಗ ಅದರಲ್ಲಿ ಪಿಷ್ಟ ಹೆಚ್ಚು ಇರುವುದಿಲ್ಲ. ಕೆಲವರು ದೀರ್ಘ ಆಯಸ್ಸಿಗೆ ಸುಟ್ಟ ಮಾಂಸ, ಬೇಕಾದಷ್ಟು ಹಾಲು ಹಣ್ಣೇ ಸರಿಯಾದ ಆಹಾರ ವೆನ್ನುವರು. ಯಾರು ನಿತ್ಯ ಹಣ್ಣನ್ನು ಹೆಚ್ಚು ಸೇವಿಸುವರೋ ಅವರು ಹೆಚ್ಚು ಕಾಲ ಬದುಕುವರು. ಏಕೆಂದರೆ ಹಣ್ಣಿನಲ್ಲಿರುವ ದ್ರವಭಾಗವು ಮೂಳೆಯ ಮೇಲೆ ಕವಿಯುವ ವೃದ್ಧಾಪ್ಯಕ್ಕೆ ಕಾರಣವಾದ ಭಾಗವನ್ನು ಕರಗಿಸುವುದು.

ಸರ್ವಸಮ್ಮತ ಸಿದ್ಧಾಂತ ಇನ್ನೂ ಆಗಿಲ್ಲ. ಈ ಬಗೆಹರಿಸಲಾರದ ಪ್ರಶ್ನೆಯಲ್ಲಿ ಎಲ್ಲರೂ ಸರ್ವಸಾಮಾನ್ಯವಾಗಿ ಒಪ್ಪುವುದೇ, ಸುಲಭವಾಗಿ ಜೀರ್ಣವಾಗುವ ಪುಷ್ಟಿಯಾದ ಆಹಾರವನ್ನು ಸೇವಿಸಬೇಕು ಎನ್ನುವುದು. ಆಹಾರ ಅಲ್ಪವಾದರೂ ಹೆಚ್ಚು ಪುಷ್ಟಿವರ್ಧಕವಾಗಿರಬೇಕು. ಅತ್ಯಲ್ಪ ಕಾಲದಲ್ಲಿ ಜೀರ್ಣಿಸಬೇಕು. ಇಲ್ಲದೇ ಇದ್ದರೆ ಹೆಚ್ಚು ಆಹಾರವನ್ನು ಸೇವಿಬೇಕಾಗುವುದು. ಇಡೀ ದಿನವೆಲ್ಲ ಅದನ್ನು ಅರಗಿಸಿವುದರಲ್ಲೇ ಕಳೆಯುವುದು. ಆಹಾರವನ್ನು ಜೀರ್ಣಿಸಿಕೊಳ್ಳುವುದಕ್ಕೆ ಎಲ್ಲಾ ಶಕ್ತಿ ವ್ಯಯವಾದರೆ ಉಳಿದುದಕ್ಕೆ ಏನು ಮಿಗುವುದು?

ಕರಿದ ಪದಾರ್ಥವೆಲ್ಲಾ ವಿಷ. ಮಿಠಾಯಿ ಅಂಗಡಿ ಯಮನ ಮನೆ. ಉಷ್ಣ ದೇಶದಲ್ಲಿ ಎಷ್ಟು ತುಪ್ಪ ಅಥವಾ ಎಣ್ಣೆಯನ್ನು ಕಡಿಮೆ ತೆಗೆದುಕೊಂಡರೆ ಅಷ್ಟು ಒಳ್ಳೆಯದು. ಬೆಣ್ಣೆಯು ತುಪ್ಪಕ್ಕಿಂತ ಹೆಚ್ಚು ಸುಲಭವಾಗಿ ಜೀರ್ಣವಾಗುತ್ತದೆ. ಮೈದಾಹಿಟ್ಟು ನೋಡುವುದಕ್ಕೆ ಚೆನ್ನಾಗಿರುವುದು. ಆದರೆ ಅದರಲ್ಲಿ ಸತ್ತ್ವವಿಲ್ಲ. ಇದಕ್ಕಿಂತ ಗೋಧಿ ಒಳ್ಳೆಯದು. ವಂಗದೇಶ್ಧಕ್ಕೆ ಈಗ್ಧಲೂ ಹಳ್ಳ್ಧಿಯ ಕಡೆ ಮಾಡ್ಧುವ ಅಡ್ಧಿಗೆ ರೀತಿ ಒಳ್ಳ್ಧೆಯ್ಧದು. ಯಾವ ಹಿಂದಿನ ವಂಗ ಕವಿ ಪೂರಿ ಖಚೋರಿಯನ್ನು ಹೊಗಳಿರುವನು? ವಂಗದೇಶಕ್ಕೂ ಪೂರಿ ಖಚೋರಿ ವಾಯುವ್ಯ ದೇಶದಿಂದ ಬಂದಿದೆ. ಅಲ್ಲಿಯೂ ಜನರು ಅತಿ ವಿರಳವಾಗಿ ತಿನ್ನುವರು. ಅಲ್ಲಿಯೂ ಕೇವಲ ಪ್ರತಿದಿನ ತುಪ್ಪದಲ್ಲೇ ಕರಿದ ಪದಾರ್ಥಗಳ ಮೇಲೆ ಜೀವಿಸುವವರನ್ನು ನೋಡಿಲ್ಲ. ಮಥುರೆಯ ಚೌಬಿ ಪೈಲ್ವಾನರಿಗೆ ಪೂರಿ ಮಿಠಾಯಿ ಕಂಡರೆ ಇಷ್ಟ. ಕೆಲವು ವರ್ಷಗಳಲ್ಲೇ ಅವರ ಜೀರ್ಣಶಕ್ತಿ ಮಂದವಾಗಿ ಹಲವು ಬಗೆಯ ಚೂರ್ಣಗಳನ್ನು ಸೇವಿಸಬೇಕಾಗುವುದು.

ಬಡವರು ತಿನ್ನುಲು ಸಿಕ್ಕದೆ ಸಾಯುವರು. ಶ‍್ರೀಮಂತರಿಗೆ ಯಾವುದನ್ನು ತಿನ್ನ ಬೇಕೊ ತಿಳಿಯದೆ ಸಾಯುವರು. ತಿಂದದ್ದೆಲ್ಲ ಆಹಾರವಲ್ಲ. ಯಾವುದನ್ನು ಚೆನ್ನಾಗಿ ಜೀರ್ಣಿಸಿಕೊಳ್ಳಬಲ್ಲೆವೋ ಅದೇ ಒಳ್ಳೆಯ ಆಹಾರ. ಸಿಕ್ಕಿದ ಆಹಾರವನ್ನೆಲ್ಲ ತಿನ್ನುವುದ ಕ್ಕಿಂತ ಉಪವಾಸ ಮಾಡುವುದು ಒಳ್ಳೆಯದು. ಮಿಠಾಯಿ ಅಂಗಡಿಯ ಸಾಮಾನು ಗಳಲ್ಲಿ ಪುಷ್ಟಿಕರವಾದುದು ಯಾವುದೂ ಇಲ್ಲ. ಅದು ವಿಷಪ್ರಾಯ. ಹಿಂದಿನ ಕಾಲದಲ್ಲಿ ಎಲ್ಲೋ ಅಪರೂಪಕ್ಕೆ ಒಮ್ಮೆ ಅದನ್ನು ತೆಗೆದುಕೊಳ್ಳುತ್ತಿದ್ದರು. ಆದರೆ ಈಗ ಪಟ್ಟಣದಲ್ಲಿ ವಾಸಮಾಡುವವರು, ವಿಶೇಷವಾಗಿ ಯಾರು ಹಳ್ಳಿಯಿಂದ ಬಂದಿರುವರೊ ಅವರು ಪ್ರತಿದಿನವೂ ಸಿಹಿಪದಾರ್ಥಗಳನ್ನು ತಿನ್ನುತ್ತಾರೆ. ಅದು ದೊಡ್ಡ ತಪ್ಪು. ಅವರು ಅಗ್ನಿಮಾಂದ್ಯದಿಂದ ಬೇಗ ಸಾಯುವುದರಲ್ಲಿ ಏನು ಆಶ್ಚರ್ಯವಿದೆ? ಹಸಿವಾಗಿದ್ದರೆ ಮಿಠಾಯಿ ಮುಂತಾದ ತುಪ್ಪದಲ್ಲಿ ಕರಿದ ಎಲ್ಲವನ್ನೂ ಬಚ್ಚಲಿಗೆ ಬಿಸಾಡಿ ಸ್ವಲ್ಪ ಅಕ್ಕಿಯ ಪುರಿಯನ್ನು ತಿನ್ನಿ. ಅದು ಬೆಲೆ ಕಡಿಮೆ, ಹೆಚ್ಚು ಆರೋಗ್ಯಕರ. ಅನ್ನ, ಬೇಳೆ, ಚಪಾತಿ, ಮೀನು, ತರಕಾರಿ, ಹಾಲು ಇದ್ದರೆ ಸಾಕು ಆಹಾರದಲ್ಲಿ. ಬೇಳೆಯನ್ನು ತೆಗೆದುಕೊಳ್ಳುವಾಗ ದಕ್ಷಿಣ ದೇಶ ದವರು ಮಾಡುವಂತೆ ಬೇಯುಸಿ, ಅದರ ನೀರನ್ನು ಮಾತ್ರ ತೆಗೆದುಕೊಂಡು ಉಳಿದುದನ್ನು ದನಕ್ಕೆ ಕೊಡಿ. ಸಾಧ್ಯವಾದರೆ ಮಾಂಸವನ್ನು ತೆಗೆದುಕೊಳ್ಳಬಹುದು. ಮಸಾಲೆ ಆಹಾರವಲ್ಲ, ಅದಕ್ಕೆ ವಾಯವ್ಯದ ಜನರು ಬೆರಸುವಷ್ಟು ಮಸಾಲೆ ಹಾಕಬೇಕಾಗಿಲ್ಲ. ಅದನ್ನು ಹೆಚ್ಚಾಗಿ ಸೇವಿಸುವುದು ದುರಭ್ಯಾಸದ ಬಲದಿಂದ. ಬೇಳೆ ಒಳ್ಳೆಯ ಆಹಾರ. ಆದರೆ ಆದನ್ನು ಜೀರ್ಣಿಸಿಕೊಳ್ಳುವುದು ಕಷ್ಟ. ಹಸಿರು ಬಟಾಣಿಯಿಂದ ಮಾಡಿದ ಸಾರು ಸುಲಭವಾಗಿ ಜೀರ್ಣವಾಗುವುದು, ಮತ್ತು ರುಚಿಯಾಗಿಯೂ ಇರುವುದು. ಪ್ಯಾರಿಸ್​ನಲ್ಲಿ ಬಟಾಣಿ ಸಾರು ಅತಿ ಪ್ರಖ್ಯಾತ ವಾಗಿದೆ. ಬಟಾಣಿಯನ್ನು ಮೊದಲು ಬೇಯಿಸಿ ಅದನ್ನು ಅರೆದು ನೀರಿನೊಡನೆ ಬೆರೆಸಿ, ಒಂದು ಬಟ್ಟೆಯಿಂದ ಅದನ್ನು ಶೋಧಿಸಿ, ಚರಟವೆಲ್ಲ ಹೊರಗೆ ಉಳಿಯು ವುದು. ನಂತರ ಅದನ್ನು ಅರಶಿಣ ಮೆಣಸು ಇವನ್ನು ಸೇರಿಸಿ ಸ್ವಲ್ಪ ತುಪ್ಪದ ಒಗ್ಗರಣೆಯನ್ನು ಹಾಕಿದರೆ ರುಚಿಕರವಾದ, ಆರೋಗ್ಯಕರವಾದ ಆಹಾರ. ಮಾಂಸಾಹಾರಿಗಳು ಇದನ್ನು ಕುರಿಯ ಅಥವಾ ಮೀನಿನ ತಲೆಯೊಂದಿಗೆ ಬೇಯಿಸಿದರೆ ತುಂಬಾ ರುಚಿಕರವಾಗುವುದು.

ಇಂಡಿಯಾ ದೇಶದಲ್ಲಿ ಇಷ್ಟೊಂದು ಬಹುಮೂತ್ರ ಕಾಯಿಲೆಗೆ ಬಹು ಮುಖ್ಯ ಕಾರಣ ಅಜೀರ್ಣ. ಎಲ್ಲೋ ಕೆಲವರಿಗೆ ಇದು ಮಾನಸಿಕ ಪರಿಶ್ರಮದಿಂದ ಆಗುವುದು. ಬೊಜ್ಜು ಹೊಟ್ಟೆಯೆ ಇದರ ಚಿಹ್ನೆ. ಊಟ ಮಾಡುವುದೆಂದರೆ ಸುಮ್ಮನೆ ತಿನ್ನುವುದೆ?ಒಬ್ಬನು ಎಷ್ಟನ್ನು ಜೀರ್ಣಿಸಿಕೊಳ್ಳಬಲ್ಲನೋ ಅದೇ ಮಿತಾಹಾರ. ತೆಳ್ಳಗಾಗುವುದು ಅಥವಾ ದಪ್ಪನಾಗುವುದಕ್ಕೆ ಕಾರಣ ಅಜೀರ್ಣ. ಮಧುಮೇಹದ ಚಿಹ್ನೆ ಸ್ವಲ್ಪ ಕಾಣುತ್ತಿದ್ದರೆ ನಿರಾಶರಾಗಬೇಡಿ. ನಮ್ಮ ದೇಶದಲ್ಲಿ ಅದನ್ನು ನಾವು ಹೆಚ್ಚಾಗಿ ಗಮನಿಸಬೇಕಾಗಿಲ್ಲ. ನಿಮ್ಮ ಆಹಾರಕ್ಕೆ ಹೆಚ್ಚು ಗಮನ ವನ್ನು ಕೊಡಿ. ಅಜೀರ್ಣ ಪ್ರಾಪ್ತವಾಗದಂತೆ ಸಾಧ್ಯವಾದಷ್ಟು ಬಯಲಿನಲ್ಲಿರಿ. ದೂರ ಗಾಳಿ ಸಂಚಾರಮಾಡಿ. ಕಷ್ಟಪಟ್ಟು ಕೆಲಸ ಮಾಡಿ. ಕಾಲಿನ ಮಾಂಸಖಂಡ ಕಬ್ಬಿಣದಂತೆ ಗಟ್ಟಿಯಾಗಿರಬೇಕು. ಕೆಲಸದಲ್ಲಿ ಇದ್ದರೆ ರಜಾ ತೆಗೆದುಕೊಂಡು ಹಿಮಾಲಯದ ಬದರಿಕಾಶ್ರಮಕ್ಕೆ ಯಾತ್ರೆ ಹೊರಡಿ. ನೀವು ಸುಮಾರು ಇನ್ನೂರು ಮೈಲಿ ಏರಿಳಿತವನ್ನು ಕಾಲುನಡಿಗೆಯಲ್ಲೆ ಮಾಡಿದರೆ ಮಧುಮೇಹದ ಭೂತ ಪರಾರಿ ಯಾಗುವುದನ್ನು ನೋಡುವಿರಿ. ವೈದ್ಯ ನಿಮ್ಮ ಸಮೀಪಕ್ಕೆ ಬರದಿರಲಿ. ಅನೇಕರು ಒಳ್ಳೆಯದಕ್ಕಿಂತ ಕೆಟ್ಟದ್ದನ್ನು ಮಾಡುವರು. ಸಾಧ್ಯವಾದ ಮಟ್ಟಿಗೆ ಔಷಧಿಯನ್ನು ಸೇವಿಸಬೇಡಿ. ಅನೇಕ ವೇಳೆ ಖಾಯಿಲೆಗಿಂತ ಈ ಔಷಧವೇ ಬೇಗ ರೋಗಿಯನ್ನು ಕೊಲ್ಲುವುದು. ಸಾಧ್ಯವಾದರೆ ಪೂಜಾ ರಜಾದಿನಗಳಲ್ಲಿ ಪ್ರತಿ ವರ್ಷವೂ ಕಲ್ಕತ್ತೆಯಿಂದ ನಿಮ್ಮ ಹಳ್ಳಿಗೆ ಕಾಲು ನಡಿಗೆಯಲ್ಲಿ ಹೋಗಿ. ನಮ್ಮ ದೇಶದಲ್ಲಿ ಶ‍್ರೀಮಂತನೆಂದರೆ ಸೋಮಾರಿತನಕ್ಕೆ ಚಿಹ್ನೆಯಾಗಿರುವನು. ಇನ್ನೊಬ್ಬರನ್ನು ಹಿಡಿದುಕೊಂಡು ನಡೆಯಬೇಕು. ಇನ್ನೊಬ್ಬರ ಸಹಾಯದಿಂದ ಊಟಮಾಡಬೇಕು. ಇವನು ಜೀವಾವಧಿ ರೋಗಿ. ಯಾರು ಪೂರಿಯ ಅಂಚಿನ ಭಾಗ ಜೀರ್ಣವಾಗುವುದಿಲ್ಲವೆಂದು ಅದನ್ನು ಬಿಟ್ಟು, ಒಳಗಿನದನ್ನು ಮಾತ್ರ ಸೇವಿಕುವರೋ ಅವರು ಆಗಲೇ ಜೀವನದಲ್ಲಿ ಮೃತರಂತೆ ಇರುವರು. ಒಂದು ಸಲ ಇಪ್ಪತ್ತು ಮೈಲಿ ನಡೆಯದೆ ಇದ್ದರೆ ಅವನು ಮನುಷ್ಯನೋ ಕೀಟವೋ! ಯಾರು ರೋಗ, ಮತ್ತು ಅಕಾಲ ಮೃತ್ಯುವಿಗೆ ಆಮಂತ್ರಣ ಕೊಡುವರೋ ಅವರನ್ನು ಯಾರು ರಕ್ಷಿಸಬಲ್ಲರು?

ಹಳೆಯ ಬ್ರೆಡ್​ ಕೂಡ ಕೆಟ್ಟದ್ದು. ಅದನ್ನು ಸೇವಿಸಬೇಡಿ. ಯೀಸ್ಟ್​ ನೊಂದಿಗೆ ಮೈದಾಹಿಟ್ಟನ್ನು ಬೆರಸಿ ಬಹಳ ಕಾಲ ಇಟ್ಟರೆ ಕೆಡುವುದು. ಹಳಸಿದ ಯಾವುದನ್ನೂ ತೆಗೆದುಕೊಳ್ಳಬೇಡಿ. ನಮ್ಮ ಶಾಸ್ತ್ರಪ್ರಕಾರ ಹಳಸಿದ ಪದಾರ್ಥವನ್ನು ಸ್ವೀಕರಿಸಬಾರದೆಂಬುದು ಒಳ್ಳೆಯದು. ಸಿಹಿಪದಾರ್ಥ ಹುಳಿಯಾದರೆ ಅದನ್ನು ಶುಕ್ತವೆನ್ನುವರು. ಅದನ್ನು ಸೇವಿಸುವುದು ಶಾಸ್ತ್ರಕ್ಕೆ ವಿರುದ್ಧ. ಆದರೆ ಮೊಸರನ್ನು ಸ್ವೀಕರಿಸಬಹುದು. ಬ್ರೆಡ್​ ತೆಗೆದುಕೊಳ್ಳಬೇಕಾದರೆ ಬೆಂಕಿಯ ಮೇಲೆ ಚೆನ್ನಾಗಿ ಕಾಯಿಸಿ ತೆಗೆದುಕೊಳ್ಳಬೇಕು.

ಅಶುದ್ಧ ನೀರು, ಅಶುದ್ಧ ಆಹಾರ ಎಲ್ಲಾ ರೋಗಕ್ಕೆ ತವರು. ಅಮೇರಿಕಾ ದೇಶದಲ್ಲಿ ನೀರಿನ ಶುದ್ಧಿ ಒಂದು ಚಟವಾಗಿದೆ. ಫಿಲ್ಟರ್​ ಕಾಲ ಆಗಿ ಹೋಯಿತು, ಅದು ಕೇವಲ ನೀರನ್ನು ಮಾತ್ರ ಶೋಧಿಸುವುದು. ಅದರಲ್ಲಿರುವ ಸೂಕ್ಷ್ಮ ಕಾಲರಾ, ಪ್ಲೇಗ್​ ಕ್ರಿಮಿಗಳು ಹಾಗೆಯೇ ಇರುವುವು. ಈ ಫಿಲ್ಟರೇ ಕಾಲಕ್ರಮೇಣ ಜಾಡ್ಯದ ತವರೂರಾಗುವುದು. ಕಲ್ಕತ್ತೆಗೆ ಮೊದಲು ಈ ಫಿಲ್ಟರ್​ ಬಂದಾಗ ಐದು ವರ್ಷ ಕಾಲರಾ ಇರಲಿಲ್ಲ. ಈಚೆಗೆ ಹಿಂದಿನಂತೆಯೇ ಆಗಿರುವುದು. ಏಕೆಂದರೆ ದೊಡ್ಡ ಫಿಲ್ಟರ್​ ಸಾಂಕ್ರಾಮಿಕ ಕ್ರಿಮಿಗಳ ನೆಲೆಮನೆಯಾಗಿದೆ. ಮೂರು ಕಾಲಿನ ಬೊಂಬಿನ ಏಣಿಯ ಮೇಲೆ ಮೂರು ಮಡಕೆಗಳನ್ನು ಇಟ್ಟು ನೀರನ್ನು ಶೋಧಿಸು ವುದೇ ಸರಳವಾದ ಮಾರ್ಗ. ಆದರೆ ಎರಡು ಅಥವಾ ಮೂರು ದಿನಕ್ಕೊಮ್ಮೆ ಹೊಸ ಇದ್ದಿಲು, ಮರಳನ್ನು ಉಪಯೋಗಿಸಬೇಕು. ಇಲ್ಲವೆ ಅದನ್ನು ಚೆನ್ನಾಗಿ ಕಾಯಿಸಿ ಪುನಃ ಉಪಯೋಗಿಸಬೇಕು. ಒಂದು ಬಟ್ಟೆಯಲ್ಲಿ ಪಟಿಕವನ್ನು ಹಾಕಿ ನಂತರ ನೀರನ್ನು ಶೋಧಿಸುವುದು ಕಲ್ಕತ್ತೆಯ ಸಮೀಪದಲ್ಲಿರುವ ಗಂಗಾನದಿಯ ತೀರದ ಹಳ್ಳಿಗಳಲ್ಲಿ ಒಂದು ರೂಢಿಯಾಗಿದೆ.ಇದೂ ಒಂದು ಅತಿ ಶ್ರೇಷ್ಠ ಮಾರ್ಗ. ಪಟಿಕ ನೀರಿನಲ್ಲಿ ರುವ ಧೂಳು ಮತ್ತು ಕ್ರಿಮಿಗಳನ್ನೆಲ್ಲ ತಳಕ್ಕೆ ತರುವುದು. ಈ ಸಣ್ಣ ಪ್ರಯೋಗ ನಲ್ಲಿಯ ನೀರಿಗಿಂತಲೂ ಉತ್ತಮವಾಗಿರುವುದು. ಹೊರಗಿನಿಂದ ಬರುವ ಫಿಲ್ಟರು ಗಳೆಲ್ಲಕ್ಕಿಂತ ಇದು ಮೇಲು. ನೀರನ್ನು ಚೆನ್ನಾಗಿ ಕಾಯಿಸಿದರೆ ಯಾವ ಅಪಾಯವೂ ಇಲ್ಲ. ಪಟಿಕದಿಂದ ನೀರಿನ ಕಶ್ಮಲವೆಲ್ಲ ತಳಭಾಗಕ್ಕೆ ಹೋದ ಮೇಲೆ ಆ ನೀರನ್ನು ಬೇರೆ ಪಾತ್ರಗೆ ಹಾಕಿ ಚೆನ್ನಾಗಿ ಕುದಿಸಿ ಕುಡಿಯಿರಿ. ವಿಲಾಯಿತಿ ಯಿಂದ ಬಂದ ಫಿಲ್ಟರ್​ ಗಿಲ್ಟರ್​ ಮುಂತಾದುವನ್ನೆಲ್ಲ ಆಚೆಗೆ ಬಿಸಾಡಿ. ಅಮೆರೀಕಾದೇಶದಲ್ಲಿ ಕುಡಿಯುವ ನೀರನ್ನು ತಯಾರು ಮಾಡಬೇಕಾದರೆ ಮೊದಲು ನೀರನ್ನು ದೊಡ್ಡ ಯಂತ್ರದಲ್ಲಿ ಆವಿಯಾಗಿ ಮಾಡುವರು. ನಂತರ ಬೇರೆ ಪಾತ್ರೆಯಲ್ಲಿ ಅದನ್ನೆಲ್ಲ ತಣ್ಣಗೆ ಮಾಡಿ ನೀರು ಮಾಡಿ ಶುದ್ಧ ಗಾಳ್ಧಿಯ ಸಂಪ್ಧರ್ಕ್ಧವ್ಧನ್ನು ಕೊಡ್ಧುವ್ಧರು. ಕಾಯ್ಧಿಸ್ಧುವ್ಧಾಗ ನೀರ್ಧಿಂದ ಹೋದ ಗಾಳಿ ಪುನಃ ನೀರಿನೊಳಗೆ ಬರಲಿ ಎಂದು ಈ ಉಪಾಯ. ಈ ನೀರು ಅತಿ ಶುದ್ದ. ಪ್ರತಿಯೊಂದು ಮನೆಯಲ್ಲಿಯೂ ಇದನ್ನು ಉಪಯೋಗಿಸುವರು.

ನಮ್ಮಲ್ಲಿ ಯಾರು ಸ್ವಲ್ಪ ಶ‍್ರೀಮಂತರೋ ಅವರು ಮಕ್ಕಳಿಗೆ ಬೇಕಾದಷ್ಟು ತುಪ್ಪದಲ್ಲಿ ಕರಿದ ತಿಂಡಿ ಮಿಠಾಯಿ ಮುಂತಾದುವನ್ನು ಕೊಡುವರು. ಬರೀ ಅನ್ನ ಚಪಾತಿ ಕೊಟ್ಟರೆ ಜನ ಏನು ತಿಳಿದುಕೊಳ್ಳುತ್ತಾರೊ ಎಂದು ಭಾವಿಸುವರು. ಹಾಗೆ ಬೆಳೆಸಿದ ಮಕ್ಕಳು ಇನ್ನು ಹೇಗೆ ಇರುವರು? ವಕ್ರ ಮೂರ್ತಿಗಳಾಗಿ, ಸೋಮಾರಿ ಗಳಾಗಿ, ಕೆಲಸಕ್ಕೆ ಬಾರದ, ಬೆನ್ನುಲುಬೇ ಇಲ್ಲದೆ, ಶುದ್ಧ ಮೂರ್ಖರಾಗುವರು. ಬಲಾಢ್ಯರಾದ ಆಂಗ್ಲೇಯರು ಹಗಲು ರಾತ್ರಿ ಕಷ್ಟಪಟ್ಟು ಕೆಲಸಮಾಡುತ್ತಾರೆ. ಅವರದು ಶೀತದೇಶ. ಅವರು ಕೂಡ ಮಿಠಾಯಿ ಮತ್ತು ತುಪ್ಪದಿಂದ ಕರಿದ ತಿಂಡಿ ಯನ್ನು ತಿನ್ನಲು ಅಂಜುವರು. ನಾವು ಅಗ್ನಿ ಕುಂಡದಲ್ಲೇ ಇರುವೆವು. ಒಂದು ಮನೆಯಿಂದ ಮತ್ತೊಂದು ಮನೆಗೆ ಹೋಗುವುದಿಲ್ಲ. ನಾವು ತಿನ್ನುವುದೇನು? ಪೂರಿ, ಖಚೋರಿ, ಮಿಠಾಯಿ, ಎಲ್ಲಾ ತುಪ್ಪ, ಎಣ್ಣೆಯಲ್ಲಿ ಕರಿದ ಸಾಮಾನು. ಹಿಂದಿನ ಕಾಲದಲ್ಲಿ ಬಂಗಾಳದ ಗ್ರಾಮೀಣ ಜಮೀನ್ದಾರರು ಇಪ್ಪತ್ತು ಮೂವತ್ತು ಮೈಲಿ ನಡೆಯುವುದು ಸಾಧಾರಣವಾಗಿತ್ತು. ಎರಡು ವೇಳೆ ಇಪ್ಪತ್ತು ಕೊಯ್​ ಮೀನು ಮತ್ತು ಅವುಗಳ ಮೂಳೆಯನ್ನೆಲ್ಲ ತಿಂದು ಪೂರೈಸುತ್ತಿದ್ದರು. ಅವರು ನೂರು ವರುಷದವರೆಗೂ ಬದುಕಿರುತ್ತಿದ್ದರು. ಈಗ ಅವರ ಮಕ್ಕಳು ಮೊಮ್ಮಕ್ಕ ಳು ಕಲ್ಕತ್ತೆಗೆ ಬಂದು ಡೊಡ್ಡ ಬಾಬುಗಳಂತೆ ನಟಿಸುವರು, ಕನ್ನಡಕ ಹಾಕಿಕೊಳ್ಳು ವರು, ಅಂಗಡಿಯಿಂದ ಮಿಠಾಯಿಯನ್ನೇ ಕೊಳ್ಳುವರು. ಒಂದು ಬೀದಿಯಿಂದ ಮತ್ತೊಂದು ಬೀದಿಗೆ ಹೋಗುವುದಕ್ಕೆ ಗಾಡಿ ಮಾಡುವರು. ನಂತರ ಮೂತ್ರಾತಿ ಸಾರವೆಂದು ಪೇಚಾಡುವರು. ಇದು ಅವರು ನಾಗರಿಕರಾದುದಕ್ಕೆ ಕಲ್ಕತ್ತೀಕರಣ ಗೊಂಡುದಕ್ಕೆ ಪ್ರಾಯಶ್ಚಿತ್ತ. ವೈದ್ಯರು, ಹಕೀಮರು ಅವರ ಅಂತ್ಯವನ್ನು ಮತ್ತೂ ಹತ್ತಿರಕ್ಕೆ ತರುವರು. ವೈದ್ಯರು ಸರ್ವಜ್ಞರೆಂದು ಭಾವಿಸುವರು, ಔಷಧಿಯಿಂದ ಎಲ್ಲಾ ರೋಗವನ್ನು ಪರಿಹರಿಸುವೆವೆಂದು ತಿಳಿಯುವರು. ಸ್ವಲ್ಪ ತೇಗು ಬಂದರೂ ಯಾವುದಾದರೂ ಚೂರ್ಣವನ್ನೂ, ಕಷಾಯವನ್ನೊ ಕೊಡುವರು. ಈ ವೈದ್ಯರ ತಲೆಗೆ, ರೋಗಿಗಳಿಗೆ ಔಷಧಿಯಿಂದ ದೂರವಿದ್ದು, ಪ್ರತಿದಿನವೂ ನಾಲ್ಕೈದು ಮೈಲಿ ನಡೆಯಿರಿ ಎಂದು ಹೇಳುವುದು ಹೊಳೆಯುವುದೇ ಇಲ್ಲ.

ನಾನು ಹಲವು ದೇಶಗಳನ್ನು ನೋಡಿರುವೆನು; ಹಲವು ಬಗೆಯ ಆಹಾರವನ್ನು ಸೇವಿಸಿರುವೆನು. ಆದರೆ ನಮ್ಮ ದೇಶದ ಅನ್ನ, ತೊವ್ವೆ, ಸಾರು, ಪಲ್ಯ ಅವಕ್ಕೆ ಸರಿಸಮನಾದುದು ಯಾವುದೂ ಇಲ್ಲ. ಬಾಳೆಹೂವಿನ ಪಲ್ಯವನ್ನು ರುಚಿ ನೋಡಲು ಮತ್ತೊಂದು ಜನ್ಮವನ್ನಾದರೂ ಬಯಸುವೆನು. ಹಲ್ಲಿರುವಾಗ ಅದರ ಉಪಯೋಗ ವನ್ನು ತಿಳಿಯದೆ ಇರುವುದು ಶೋಚನೀಯ. ಪಾಶ್ಚಾತ್ಯರನ್ನು ನಾವು ಆಹಾರ. ದಲ್ಲೂ ಏತಕ್ಕೆ ಅನುಸರಿಸಬೇಕು? ಇದನ್ನು ಎಷ್ಟು ಜನ ಮಾಡಲು ಸಾಧ್ಯ? ನಮ್ಮ ದೇಶಕ್ಕೆ ಸರಿಯಾಗಿರುವುದು ಪೂರ್ವ ಬಂಗಾಳದಲ್ಲಿ ರೂಢಿಯಾಗಿರುವ ಆಹಾರ. ಶುದ್ಧವಾಗಿದೆ, ಸುಲಭ ಬೆಲೆ, ಪುಷ್ಟಿವರ್ಧಕ. ನೀವು ಆಹಾರವನ್ನು ಅನುಸರಿಸಲು ಪಶ್ಚಿಮಕ್ಕೆ ಹೆಚ್ಚು ತಿರುಗಿದಷ್ಟೂ ಕೆಟ್ಟದ್ದು, ನೀವು ಹೆಚ್ಚು ಅನಾಗರಿಕರಾಗುವ ಸಂಭವವಿದೆ. ನೀವೆಲ್ಲ ಕಲ್ಕತ್ತೆಯ ಪುರಜನರು, ಶುದ್ಧ ನಾಗರಿಕರು. ಅಂಗಡಿಯ ಮಿಠಾಯಿ ಎನ್ನುವ ಮೋಹಿನಿಯ ಬಲೆಗೆ ಬಿದ್ದಿರುವಿರಿ. ಬಂಕೂರವು ಅಕ್ಕಿಯ ಪುರಿಯನ್ನು ದಾಮೋದರ ನದಿಗೆ ಚೆಲ್ಲಿತು. ಅಲ್ಲಿನ ಜನರು ಉದ್ದಿನಬೇಳೆ ತೊವ್ವೆ ಬಿಸಾಡಿದರು. ಢಾಕಾ, ವಿಕ್ರಮಪುರ ಜನರು ತಮ್ಮ ಹಳೆಯ ಅಡಿಗೆ ವಿಧಾನವನ್ನೆಲ್ಲ ಮರೆತರು. ಬೇರೆ ಮಾತಿನಲ್ಲಿ ಹೇಳುವುದಾದರೆ, ಎಲ್ಲಾ ನಾಗರಿಕರಾಗಿ ಬಿಟ್ಟರು! ನಾಗರಿಕರಾದ ಪುರಜನರೇ! ನೀವು ನಾಶವಾದಿರಿ. ಇತರರನ್ನೂ ನಾಶದ ಕಡೆಗೆ ಒಯ್ಯುತ್ತಿರುವಿರಿ. ಆದರೂ ನಾವು ನಾಗರಿಕರೆಂದು ಹೆಮ್ಮೆಕೊಚ್ಚಿಕೊಳ್ಳುವಿರಿ. ನಮ್ಮ ದೇಶದ ಜನರು ಮೂಢರು, ಊರಿನ ಗಲೀಜನ್ನೆಲ್ಲಾ ತಿಂದು ಆಮಶಂಕೆ ಅಜೀರ್ಣದಿಂದ ನರಳುವರು.ಅದು ತಮಗೆ ಸರಿಯಾದ ಆಹಾರವಲ್ಲವೆಂದು ಒಪ್ಪಿ ಕೊಳ್ಳುವುದಿಲ್ಲ. ಕಲ್ಕತ್ತೆಯ ಗಾಳಿ ತಮಗೆ ಒಗ್ಗುವುದಿಲ್ಲ, ಅದರಲ್ಲಿ ನೀರಿನ ಆವಿ ಇದೆ ಎಂದು ಹೇಳಿಕೊಳ್ಳುವರು. ಎಲ್ಲಾ ವಿಧದಲ್ಲಿಯೂ ಅವರು ನಗರದ ಜನರು ಆಗಬೇಕು ತಾನೆ!

ಆಹಾರ, ಉಡುಪು, ಇವಕ್ಕೆ ಸಂಬಂಧಪಟ್ಟ ವಿಷಯ ಈಗ ಸಂಕ್ಷೇಪವಾಗಿ ಆಯಿತು. ಪಾಶ್ಚಾತ್ಯರು ಏನು ತಿನ್ನುತ್ತಾರೆ; ಕ್ರಮೇಣ ಅದು ಹೇಗೆ ಬದಲಾ ಯಿಸಿದೆ ಅದನ್ನು ವಿವರಿಸುತ್ತೇನೆ.

ಎಲ್ಲ ದೇಶಗಳಲ್ಲಿಯೂ ಬಡವರು ಯಾವುದಾದರೊಂದು ಬಗೆಯ ಧಾನ್ಯವನ್ನು ಸೇವಿಸುವರು. ಸೊಪ್ಪು, ತರಕಾರಿ, ಮೀನು, ಮಾಂಸ ಇವು ದುಬಾರಿವಸ್ತುಗಳು. ಅವನ್ನು ಚಟ್ನಿಯಂತೆ ಸ್ವಲ್ಪ ಸೇವಿಸುವರು. ಯಾವ ಧಾನ್ಯವನ್ನು ದೇಶದಲ್ಲಿ ಯಥೇಚ್ಚವಾಗಿ ಬೆಳೆಯುವರೋ, ಯಾವುದಕ್ಕೆ ಬೆಲೆ ಕಡಿಮೆಯೋ ಅದನ್ನೇ ಅಲ್ಲಿಯ ಬಡ ಜನರು ಸೇವಿಸುವರು. ಬಂಗಾಳ, ಒರಿಸ್ಸ, ಮದ್ರಾಸು, ಮಲಬಾರ್​, ಮುಂತಾದ ಕಡೆ ಮುಖ್ಯ ಆಹಾರ ಅಕ್ಕಿ, ಬೇಳೆ, ತರಕಾರಿ. ಕೆಲವು ವೇಳೆ ಮೀನು, ಮಾಂಸವನ್ನು ಚಟ್ನಿಯಂತೆ ಮಾತ್ರ ಉಪಯೋಗಿಸುವರು. ಉಳಿದ ಕಡೆ ಶ‍್ರೀಮಂತರ ಆಹಾರ ಗೋಧಿಯಿಂದ ಮಾಡಿದ ಚಪಾತಿ. ಸಾಧಾರಣ ಜನರು ನವಣೆ, ಹಾರಕ, ಜೋಳ ಮುಂತಾದ ಧಾನ್ಯದಿಂದ ಮಾಡಿದ ರೊಟ್ಟಿ ತೆಗೆದುಕೊಳ್ಳುವರು.

ಭಾರತದಲ್ಲಿ ಎಲ್ಲಾ ಕಡೆಯಲ್ಲಿಯೂ ರೊಟ್ಟಿ ಅಥವಾ ಅನ್ನಕ್ಕೆ ರುಚಿ ಹುಟ್ಟಿಸುವ ಸಲುವಾಗಿ ಮಾತ್ರ ತರಕಾರಿ, ಬೇಳೆ, ಮೀನು, ಮಾಂಸವನ್ನು ಉಪಯೋಗಿಸುವರು. ಅದನ್ನು ಸಂಸ್ಕೃತದಲ್ಲಿ ವ್ಯಂಜನವೆಂದು ಹೇಳುವರು. ಪಂಜಾಬ್​, ರಜಪುಟಾಣ, ದಖನ್​ ಮುಂತಾದ ಕಡೆ ಶ‍್ರೀಮಂತರು ಪ್ರತಿದಿವವೂ ಹಲವು ಬಗೆಯ ಮಾಂಸವನ್ನು ತೆಗೆದುಕೊಂಡರೂ ಅವರ ಮುಖ್ಯ ಅಹಾರ ಚಪಾತಿ ಅಥವಾ ಅನ್ನ. ಯಾರು ದಿನಕ್ಕೆ ಒಂದು ಪೌಂಡಿನಷ್ಟು ಮಾಂಸ ತೆಗೆದುಕೊಳ್ಳುವನೊ ಅವನು ದಿನಕ್ಕೆ ಎರಡು ಪೌಂಡಿನಷ್ಟು ಅನ್ನ ಅಥವಾ ಚಪಾತಿಯನ್ನು ಅದರ ಜೊತೆಗೆ ತೆಗೆದುಕೊಳ್ಳುವನು.

ಇದರಂತೆ ಪಾಶ್ಚಾತ್ಯ ದೇಶಗಳಲ್ಲಿ ಬಡವರು ಬ್ರೆಡ್​ ಮತ್ತು ಆಲೂಗೆಡ್ಡೆ ಯನ್ನು ಸೇವಿಸುವರು. ಮಾಂಸವನ್ನು ತೆಗೆದುಕೊಳ್ಳುವುದು ಅಪರೂಪ. ತೆಗೆದು ಕೊಂಡರೆ ಅದನ್ನು ಚಟ್ನಿಯಂತೆ ಭಾವಿಸುವರು. ಹೆಚ್ಚು ಉಷ್ಣವಿರುವ ಸ್ಪೆಯಿನ್​, ಪೋರ್ಚುಗಲ್​, ಇಟಲಿ ಮುಂತಾದ ಕಡೆ ಬೇಕಾದಷ್ಟು ದ್ರಾಕ್ಷಿಯನ್ನು ಬೆಳೆಯು ವರು. ದ್ರಾಕ್ಷಿಯಿಂದ ಮಾಡಿದ ಮದ್ಯ ಸುಲಭ ಬೆಲೆಗೆ ದೊರೆಯುವುದು. ಹೆಚ್ಚಾಗಿ ಕುಡಿಯದಿದ್ದರೆ ಇದರಿಂದ ಅಮಲು ಬರುವುದಿಲ್ಲ, ದೇಹಕ್ಕೂ ಒಳ್ಳೆಯದು. ಆ ದೇಶದ ಬಡವರು ಮೀನು, ಮಾಂಸಕ್ಕೆ ಬದಲಾಗಿ ದ್ರಾಕ್ಷಾರಸವನ್ನು ಉಪಯೋಗಿಸುವರು. ಆದರೆ ರಷ್ಯಾ, ಸ್ವೀಡನ್​, ನಾರ್ವೆ ಮುಂತಾದ ಯೂರೋಪಿನ ಶೀತದೇಶಗಳಲ್ಲಿ ಗೋಧಿಯಿಂದ ಮಾಡಿದ ಬ್ರೆಡ್​, ಆಲೂಗಡ್ಡೆ, ಒಣ ಮೀನು ಇವನ್ನು ಬಡವರು ಸೇವಿಸುವರು.

ಯೂರೋಪಿನ ಶ‍್ರೀಮಂತರು ಮತ್ತು ಅಮೇರಿಕಾದೇಶದ ಎಲ್ಲಾ ಜನರು ಊಟ ಬೇರೆ. ಅವರ ಮುಖ್ಯ ಆಹಾರ ಮೀನು ಮತ್ತು ಮಾಂಸ. ಬ್ರೆಡ್​, ಅನ್ನ ಮುಂತಾದುವನ್ನು ಚಟ್ನಿಯಂತೆ ಸೇವಿಸುವರು. ಅಮೇರಿಕಾ ದೇಶದಲ್ಲಿ ಬ್ರೆಡ್​ ತೆಗೆದುಕೊಳ್ಳುವುದು ಎಲ್ಲೊ ಸ್ವಲ್ಪ. ಮೀನು ಅಥವಾ ಮಾಂಸ ಬಡಿಸಿದರೆ ಅದನ್ನು ಬ್ರಡ್​ ಅಥವಾ ಅನ್ನದ ಸಹಾಯವಿಲ್ಲದೆ ಸೇವಿಸುವರು. ಅದಕ್ಕೆ ತಟ್ಟೆಯನ್ನು ಪ್ರತಿ ಸಲವೂ ಬದಲಾಯಿಸಬೇಕು. ಹತ್ತು ಬಗೆಯ ಆಹಾರವಿದ್ದರೆ ಅದನ್ನು ಹತ್ತು ಸಲ ಬದಲಾಯಿಸಬೇಕು. ಈ ರೀತಿ ನಾವು ಆಹಾರವನ್ನು ಸೇವಿಸಬೇಕಾದರೆ ಅದನ್ನು ಹೀಗೆ ಬಡಿಸಬೇಕು: ಮೊದಲು ಸೂಕ್ತ, ಕಹಿಯಾದ ಪಲ್ಯ, ಆ ತಟ್ಟೆಯನ್ನು ಬದಲಾಯಿಸಿದ ಮೇಲೆ ತೊವ್ವೆ; ಇದರಂತೆಯೇ ಸೂಪ್​; ನಂತರ ಅನ್ನ, ಪೂರಿ, ಹೀಗೆ ಪ್ರತಿಸಲವೂ ತಟ್ಟೆ ಬದಲಾಯಿಸಬೇಕು. ಈ ರೀತಿ ಬಡಿಸುವುದರಿಂದ ಆಗುವ ಪ್ರಯೋಜನವೇನೆಂದರೆ ಹಲವು ಬಗೆಯನ್ನು ಸ್ವಲ್ಪ ಸ್ವ್ಧಲ್ಪ ಸೇವಿಸುವರು. ಯಾವುದನ್ನೂ ಹೆಚ್ಚು ಸೇವಿಸುವುದಕ್ಕೆ ಅವಕಾಶವಿಲ್ಲ. ಫ್ರೆಂಚ್​ ಜನ ಬೆಳಗ್ಗೆ ಕಾಫಿ, ಒಂದೆರಡು ಚೂರು ಬ್ರೆಡ್​ ಮತ್ತು ಬೆಣ್ಣೆ; ಮಧ್ಯಾಹ್ನ ಸಾಧಾರಣವಾಗಿ ಮೀನು ಮಾಂಸ; ಅವರ ಮುಖ್ಯವಾದ ಊಟ ನಡೆಯುವುದು ರಾತ್ರಿ. ಇಟಲಿ ಮತ್ತು ಸ್ಪೆಯಿನ್​ ಜನರು ಫ್ರೆಂಚರಂತೆ. ಜರ್ಮನರು ಹೆಚ್ಚು ಸಲ ತಿನ್ನುವರು - ದಿನಕ್ಕೆ ಐದಾರು ಸಲ. ಪ್ರತಿ ಸಲವೂ ಹೆಚ್ಚು ಕಡಿಮೆ ಸ್ವಲ್ಪ ಮಾಂಸ ಇದ್ದೇ ಇರುವುದು. ಇಂಗ್ಲೀಷರು ದಿನಕ್ಕೆ ಮೂರು ಸಲ; ಬೆಳಿಗ್ಗೆ ಉಪಹಾರ ಬಹಳ ಕಡಿಮೆ; ಮಧ್ಯೆ ಮಧ್ಯೆ ಕಾಫಿ ಅಥವಾ ಟೀ ತೆಗೆದುಕೊಳ್ಳುವರು. ಅಮೇರಿಕಾದವರೂ ದಿನಕ್ಕೆ ಮೂರು ಸಲ ತೆಗೆದುಕೊಳ್ಳುವರು. ಪ್ರತಿಸಲವೂ ಮಾಂಸ ತೆಗೆದುಕೊಳ್ಳುವರು. ಈ ದೇಶಗಳಲ್ಲೆಲ್ಲಾ ರಾತ್ರಿ ಊಟವೇ ಮುಖ್ಯ. ಶ‍್ರೀಮಂತರ ಹತ್ತಿರ ಫ್ರೆಂಚ್​ ಅಡ್ಧುಗ್ಧೆಯ್ಧವ್ಧರು ಇರ್ಧುವ್ಧರು. ಅವ್ಧರು ಫ್ರ್ಧೆಂಚ್​ ರೀತ್ಧಿಯ್ಧಲ್ಲಿ ಅಡಿಗೆಯನ್ನು ಮಾಡುವರು. ಊಟಕ್ಕೆ ಮುಂಚೆ ಉಪ್ಪಿನ ಮೀನು ಅಥವಾ ಯಾವುದಾದರೂ ಬಗೆಯ ಚಟ್ನಿಯನ್ನು ಕೊಡುವರು. ಇದು ಹಸಿವನ್ನು ಹೆಚ್ಚಿಸುವುದಕ್ಕೆ. ನಂತರ ಸೂಪ್​. ಇದಾದ ಮೇಲೆ ಇಂದಿನ ರೂಢಿಯ ಪ್ರಕಾರ ಹಣ್ಣು, ಮೀನು, ಮಾಂಸದ ಪಲ್ಯ; ಇದಾದ ಮೇಲೆ ಸುಟ್ಟ ಮಾಂಸದ ತುಂಡು, ಜೊತೆಗೆ ಸ್ವಲ್ಪ ತರಕಾರಿ, ನಂತರ ಹಕ್ಕಿಯ ಮಾಂಸ; ನಂತರ ಸಿಹಿ ಪದಾರ್ಥ, ಕೊನೆಗೆ ರುಚಿಕರವಾದ ಐಸ್​ಕ್ರೀಮ್​. ಶ‍್ರೀಮಂತರ ಊಟದಲ್ಲಿ ಪ್ರತಿ ಸಲ ತಟ್ಟೆಯನ್ನು ಬದಲಾಯಿಸುವಾಗಲೂ, ಮದ್ಯವನ್ನು ಬದಲಾಯಿಸುವರು. ಹಾಕ್​, ಕ್ಲಾರೆಟ್​, ತಣ್ಣಗಿರುವ ಷಾಂಪೇನ್​ ಇವನ್ನು ಒಂದಾದ ಮೇಲೊಂದರಂತೆ ಬಡಿಸುವರು. ಪ್ರತಿಸಲ ತಟ್ಟೆ ಬದಲಾಯಿಸುವಾಗಲೂ ಚಾಕು, ಸ್ಪೂನ್​, ಫೋರ್ಕ್​ ಇವನ್ನು ಬದಲಾಯಿಸುವರು. ಊಟವಾದ ಮೇಲೆ ಹಾಲಿಲ್ಲದ ಕಾಫಿ ಮತ್ತು ಅತಿ ಸಣ್ಣ ಗ್ಲಾಸಿನಲ್ಲಿ ಮದ್ಯವನ್ನು ತರುವರು. ಕೊನೆಗೆ ಧೂಮಪಾನ. ಪ್ರತಿ ತಟ್ಟೆಗೂ ಯಾರು ಬಗೆ ಬಗೆಯ ಮದ್ಯವನ್ನು ಕೊಡುವರೊ ಅವರನ್ನು ಅತಿ ಶ‍್ರೀಮಂತರೆಂದು ಪರಿಗಣಿಸುವರು. ಒಂದು ಊಟ ಕೊಡುವುದಕ್ಕೆ ಅಲ್ಲಿ ಅಷ್ಟೊಂದು ಹಣವನ್ನು ಖರ್ಚು ಮಾಡುವರು. ಇದು ನಮ್ಮ ದೇಶದ ತಕ್ಕಮಟ್ಟಿನ ಶ‍್ರೀಮಂತನನ್ನು ಮುಳುಗಿಸಲು ಸಾಕು.

ಆರ್ಯರು ಊಟದ ಸಮಯದಲ್ಲಿ ಚಕ್ಕಲುಮಕ್ಕಲು ಹಾಕಿಕೊಂಡು ಒಂದು ಮಣೆಯ ಮೇಲೆ ಕುಳಿತುಕೊಳ್ಳುತ್ತಿದ್ದರು. ಒರಗಿಕೊಳ್ಳುವುದಕ್ಕೆ ಹಿಂದೆ ಒಂದು ಮಣೆ. ಊಟ ಮಾಡುವುದಕ್ಕೆ ಒಂದು ಲೋಹದ ತಟ್ಟೆಯನ್ನು ಒಂದು ಮಣೆಯ ಮೇಲೆ ಇಡುತ್ತಿದ್ದರು. ಈ ಅಭ್ಯಾಸ ಪಂಜಾಬ್​, ರಾಜಪುಟಾನ, ಮಹಾರಾಷ್ಟ್ರ ಮತ್ತು ಗುಜರಾತುಗಳಲ್ಲಿ ಈಗಲೂ ಇದೆ. ಬಂಗಾಳ, ಒರಿಸ್ಸ, ತೆಲಂಗಾಣಾ ಮತ್ತು ಮಲ ಬಾರಿನ ಜನರು ಊಟದ ತಟ್ಟೆಯನ್ನು ಮಣೆಯ ಮೇಲೆ ಇಡುವುದಿಲ್ಲ, ಅದರ ಬದಲು ಬಾಳೆಯ ಎಲೆ ಅಥವಾ ತಟ್ಟೆಯನ್ನು ನೆಲದ ಮೇಲೆಹಾಕಿಕೊಳ್ಳುವರು. ಮೈಸೂರಿನ ಮಹಾರಾಜರು ಕೂಡ ಹೀಗೆಯೇ ಮಾಡುವರು. ಮಹಮ್ಮದೀಯರು ಊಟ ಮಾಡುವಾಗ ದೊಡ್ಡ ಬಿಳಿಯ ಬಟ್ಟೆಯ ಮೇಲೆ ಕುಳಿತು ಕೊಳ್ಳುವರು. ಬರ್ಮ ಮತ್ತು ಜಪಾನಿನ ಜನರು ಊಟದ ತಟ್ಟೆಯನ್ನು ನೆಲದ ಮೇಲೆ ಇಟ್ಟು ಮಂಡಿಯ ಮೇಲೆ ಕುಳಿತುಕೊಂಡು ಊಟಮಾಡುವರು. ಚೈನಾ ದೇಶೀಯರು ಕುರ್ಚಿಯ ಮೇಲೆ ಕುಳಿತುಕೊಳ್ಳುವರು. ಮುಂದೆ ಊಟಕ್ಕೆ ಒಂದು ಮೇಜು ಇರುವುದು; ಊಟದ ಸಮಯದಲ್ಲಿ ಚಮಚ ಮತ್ತು ಎರಡು ಕಡ್ಡಿಗಳನ್ನು ಉಪಯೋಗಿಸುವರು. ಹಿಂದಿನ ಕಾಲದಲ್ಲಿ ಗ್ರೀಕ್​ ಮತ್ತು ರೋಮನ್​ ಜನರು ಊಟ ಮಾಡುವುದಕ್ಕೆ ಮೇಜನ್ನೇ ಉಪಯೋಗಿಸುತ್ತಿದ್ದರು. ಮಂಚದ ಮೇಲೆ ಒರಗಿಕೊಂಡು ಕುಳಿತು ಕೈಬೆರಳಿಂದಲೆ ಊಟ ಮಾಡುತ್ತಿದ್ದರು. ಯೂರೋಪ್​ ದೇಶೀಯರು ಕುರ್ಚಿಯ ಮೇಲೆ ಕುಳಿತುಕೊಂಡು ಮೇಜಿನಿಂದ ಕೈಬೆರಳುಗಳ ಸಹಾಯದಿಂದಲೇ ಊಟಮಾಡುತ್ತಿದ್ದರು. ಆದರೆ, ಈಗ ಚಮಚ ಫೋರ್ಕ್​ ಮುಂತಾದುವನ್ನು ಉಪಯೋಗಿಸುವರು. ಚೈನಿಯರ ರೀತಿ ಊಟ ಮಾಡುವುದು ಒಂದು ಕಸರತ್ತು. ಅದಕ್ಕೆ ತುಂಬಾ ಕುಶಲತೆ ಬೇಕು. ಹೇಗೆ ನಮ್ಮ ಪಾನ್ ವಾಲಾಗಳು ವೀಳ್ಯದೆಲೆ ಮಡಿಸುವುದಕ್ಕೆ ಎರಡು ಕಡ್ಡಿಗಳನ್ನು ಉಪಯೋಗಿಸು ವರೋ, ಹಾಗೆಯೇ ಚೈನಾದೇಶಿಯರು ಎರಡು ಕಡ್ಡಿಗಳ ಸಹಾಯದಿಂದ ತರಕಾರಿ ಯನ್ನು ಹಿಡಿದುಕೊಂಡು ಬಾಯಿಗೆ ರವಾನಿಸುವರು. ಪುನಃ ಅದೇ ಕಡ್ಡಿಗಳ ಸಹಾಯದಿಂದ ಅನ್ನದ ತಟ್ಟೆಯನ್ನು ಬಾಯಿಯ ಹತ್ತಿರ ಇಟ್ಟುಕೊಂಡು ಅನ್ನವನ್ನು ಬಾಯೊಳಗೆ ತಳ್ಳುವರು.

ಎಲ್ಲಾ ದೇಶದ ಪೂರ್ವಿಕರು ತಮಗೆ ಏನು ಸಿಕ್ಕುತ್ತಿತ್ತೊ ಅದನ್ನು ತಿನ್ನುತ್ತಿದ್ದರು. ಒಂದು ದೊಡ್ಡ ಪ್ರಾಣಿಯನ್ನು ಕೊಂದರೆ ಒಂದು ತಿಂಗಳ ತನಕ ಇಟ್ಟುಕೊಂಡು ತಿನ್ನುತ್ತಿದ್ದರು. ಅದು ಕೊಳೆತರೂ ಆಚೆಗೆ ಎಸೆಯುತ್ತಿರಲಿಲ್ಲ. ಕ್ರಮೇಣ ಅವರು ಸಭ್ಯರಾಗಿ ವ್ಯವಸಾಯವನ್ನು ಕಲಿತರು. ಹಿಂದಿನ ಕಾಲದಲ್ಲಿ ಪ್ರತಿದಿನ ಬೇಟೆ ಆಡು ವಾಗಲೂ ಪ್ರಾಣಿಗಳು ಸಿಕ್ಕುತ್ತಿರಲಿಲ್ಲ. ಅದಕ್ಕೇ ಮೃಗ ಸಿಕ್ಕಿದಾಗ ಹೊಟ್ಟೆ ತುಂಬ್ ತಿಂದು ನಾಲ್ಕೈದು ದಿನ ಉಪವಾಸ ಮಾಡುತ್ತಿದ್ದರು. ನಂತರ ಅದರಿಂದ ಪಾರಾದರು. ಕೃಷಿಯಿಂದ ಪ್ರತಿದಿನದ ಆಹಾರ ದೊರಕುತ್ತದೆ. ಆದರೂ ಊಟದ ಸಮಯದಲ್ಲಿ ಸ್ವಲ್ಪ ಕೊಳೆತ ಮಾಂಸವನ್ನು ತಿನ್ನುವುದು ಹಿಂದಿನ ರೂಢಿಯ ಅವಶೇಷವಾಗಿದೆ. ಆಗ ಕೊಳೆತ ಮಾಂಸ ಭೋಜನದ ಒಂದು ಅವಶ್ಯಕ ಪದಾರ್ಥವಾಗಿತ್ತು. ಈಗ ಅದು ಚಟ್ನಿ ಉಪ್ಪಿನಕಾಯಿ ರೂಪವನ್ನು ತಾಳಿದೆ.

ಎಸ್ಕಿಮೋ ಜನ ಹಿಮದ ಮೇಲೆ ವಾಸಿಸುವರು. ಅಲ್ಲಿ ಯಾವ ಧಾನ್ಯವನ್ನೂ ಬೆಳೆಯಲು ಆಗುವುದಿಲ್ಲ. ಅವರ ಪ್ರತಿದಿನ ಆಹಾರ ಮೀನು, ಮಾಂಸ, ಯಾವಾಗಲಾದರೂ ಅಗ್ನಿಮಾಂದ್ಯವಾದರೆ ಚೂರು ಕೊಳೆತ ಮಾಂಸವನ್ನು ಪುನಃ ಹಸಿಯನ್ನು ಪಡೆಯುವುದಕ್ಕೆ ತಿನ್ನುವರು.

ಈಗಲೂ ಯೂರೋಪಿಯನ್ನರು ಪಕ್ಷಿ ಮತ್ತು ಇನ್ನೂ ವನಚರ ಪ್ರಾಣಿಗಳನ್ನು ಕೊಂದ ತಕ್ಷಣ ಉಪಯೋಗಿಸುವುದಿಲ್ಲ. ಸ್ವಲ್ಪ ವಾಸನೆ ಬರುವ ತನಕ ಅದನ್ನು ನೇತು ಹಾಕುವರು. ಕಲ್ಕತ್ತೆಯಲ್ಲಿ ಸ್ವಲ್ಪ ಕೊಳೆತ ಜಿಂಕೆ ಮಾಂಸ ಮಾರ್ಕೆಟ್ಟಿಗೆ ಬಂದೊ ಡನೆ ಖರ್ಚಾಗಿ ಹೋಗುವುದು. ಕೆಲವು ಬಗೆಯ ಮೀನು ಸ್ವಲ್ಪ ಹಳೆಯದಾದರೆ ಜನರಿಗೆ ಇಷ್ಟ. ಯೂರೋಪಿನ ಕೆಲವು ಭಾಗಗಳಲ್ಲಿ ಸ್ವಲ್ಪ ವಾಸನೆ ಬರುವ ಚೀಸ್​ ಜನರಿಗೆ ಇಷ್ಟ. ಶಾಕಾಹಾರಿಗಳಿಗೂ ಈರುಳ್ಳಿ ಬೆಳ್ಳುಳ್ಳಿ ಬೇಕು. ದಕ್ಷಿಣ ದೇಶದ ಬ್ರಾಹ್ಮಣರ ಅಡಿಗೆಯಲ್ಲಿ ಅದು ಇದ್ದೇ ಇರಬೇಕು. ಆದರೆ ಹಿಂದೂ ಶಾಸ್ತ್ರ ಅದನ್ನು ನಿಷೇಧಿಸಿದೆ. ಈರುಳ್ಳಿ, ಬೆಳ್ಳುಳ್ಳಿ, ಸಾಕಿದ ಕೋಳಿ, ಹಂದಿ, ಇವು ಬ್ರಾಹ್ಮಣ ನಿಗೆ ನಿಷೇಧವಸ್ತು. ಯಾರು ಇದನ್ನು ಭಕ್ಷಿಸುವರೊ ಅವರು ತಮ್ಮ ಜಾತಿಯನ್ನು ಕಳೆದುಕೊಳ್ಳುವರು. ಅದಕ್ಕೆ ಆಚಾರಶೀಲ ಬ್ರಾಹ್ಮಣರು ಅದನ್ನು ಬಿಟ್ಟು ಅದರ ಬದಲು ವಾಸನೆಯಲ್ಲಿ ಅದಕ್ಕಿಂತ ಭಯಂಕರವಾದ ಇಂಗನ್ನು ಉಪಯೋಗಿಸಲು ಪ್ರಾರಂಭಿಸಿದರು. ಇದರಂತೆ ಹಿಮಾಲಯ ಪ್ರಾಂತ್ಯದಲ್ಲಿ ವಾಸಿಸುವ ಬ್ರಾಹ್ಮಣರು ಬೆಳ್ಳುಳ್ಳಿಯಂತೆ ವಾಸನೆ ಬರುವ ಒಂದು ಬಗೆಯ ಹುಲ್ಲನ್ನು ಉಪ ಯೋಗಿಸುವರು. ಇದರಲ್ಲಿ ದೋಷವೇನು? ಶಾಸ್ತ್ರ ಇದನ್ನು ನಿಷೇಧಿಸಿಲ್ಲ.

ಎಲ್ಲಾ ಧರ್ಮಗಳಲ್ಲಿಯೂ ತಿನ್ನುವುದಕ್ಕೆ ಕುಡಿಯುವುದಕ್ಕೆ ಹಲವು ವಿಧಿ ನಿಷೇಧಗಳಿವೆ. ಕ್ರೈಸ್ತರಲ್ಲಿ ಮಾತ್ರ ಇದು ಇಲ್ಲ. ಜೈನರು ಮತ್ತು ಬೌದ್ಧರು ಎಂದಿಗೂ ಮಾಂಸ ಮೀನು ತಿನ್ನುವುದಿಲ್ಲ. ಜೈನರು ನೆಲದೊಳಗೆ ಬೆಳೆಯುವ ಆಲೂಗೆಡ್ಡೆ, ಮೂಲಂಗಿ ಮುಂತಾದುವನ್ನು ತಿನ್ನುವುದಿಲ್ಲ. ಏಕೆಂದರೆ ಆ ತರಕಾರಿಗಳನ್ನು ಅಗೆಯುವಾಗ ನೆಲದಲ್ಲಿರುವ ಕ್ರಿಮಿಕೀಟಗಳ ವಧೆಯ ಪಾಪ ಬರುವುದೆಂದು ಭಯ. ರಾತ್ರಿ ಹೊತ್ತು ಅವರು ಊಟಮಾಡುವುದಿಲ್ಲ. ಹಾಗೆ ಮಾಡಿದರೆ ಯಾವುದಾದರೂ ಕ್ರಿಮಿಕೀಟ ಬಾಯೊಳಗೆ ಹೋಗಬಹುದು. ಯಹೂದ್ಯರು ಯಾವ ಮೀನಿನ ಮೇಲೆ ಪೊರೆ ಇರುವುದಿಲ್ಲವೊ ಅದನ್ನು ಭಕ್ಷಿಸುವುದಿಲ್ಲ; ಹಂದಿ, ಮೆಲಕು ಹಾಕದ, ಸೀಳಿದ ಕಾಲ್ಗೊರಸಿಲ್ಲದ ಪ್ರಾಣಿಗಳನ್ನಾಗಲಿ ತಿನ್ನುವುದಿಲ್ಲ. ಮೀನು ಮಾಂಸ ಬೇಯಿಸುವ ಕಡೆ, ಹಾಲು ಮತ್ತು ಅದಕ್ಕೆ ಸಂಬಂಧಪಟ್ಟ ವಸ್ತುವನ್ನು ತಂದರೆ ಮಾಡಿದುದನ್ನೆಲ್ಲ ಆಚೆಗೆ ಎಸೆಯುವರು. ಅದಕ್ಕೆ ಆಚಾರ ಶೀಲರಾದ ಯಹೂದ್ಯರು ಅನ್ಯ ದೇಶೀಯರು ಮಾಡಿದ ಅಡಿಗೆಯನ್ನು ಸೇವಿಸುವುದಿಲ್ಲ. ಹಿಂದೂಗಳಂತೆಯೇ ನೈವೇದ್ಯ ಮಾಡದ ಮಾಂಸವನ್ನು ಭಕ್ಷಿಸುವುದಿಲ್ಲ. ಬಂಗಾಳ ಮತ್ತು ಪಂಜಾಬುಗಳಲ್ಲಿ ದೇವಿಗೆ ಅರ್ಪಿಸಿದ ಮಾಂಸಕ್ಕೆ ಮಹಾಪ್ರಸಾದವೆಂದು ಹೆಸರು. ಮಹಾಪ್ರಸಾದವಲ್ಲದ ಮಾಂಸವನ್ನು ಯಹೂದ್ಯರು ಭಕ್ಷಿಸುವುದಿಲ್ಲ. ಹಿಂದೂಗಳಂತೆಯೇ ಅವ್ಧರು ಕೂಡ ಮಾಂಸವನ್ನು ಸಿಕ್ಕಿದ ಕಡೆ ಕೊಂಡುಕೊಳ್ಳಲಾರರು. ಮಹಮ್ಮದೀಯರು ಯಹೂದ್ಯರ ಹಲವು ನಿಯಮಗನ್ನು ಅನುಸರಿಸುವರು. ಆದರೆ ಅವರಂತೆ ಒಂದು ಅತಿರೇಕಕ್ಕೆ ಹೋಗುವುದಿಲ್ಲ. ಒಂದೇ ಊಟದಲ್ಲಿ ಹಾಲು, ಮೀನು, ಮಾಂಸವನ್ನು ಭಕ್ಷಿಸುವುದಿಲ್ಲ. ಆದರೆ ಅವು ಒಂದೇ ಅಡಿಗೆಮನೆಯಲ್ಲಿದ್ದರೆ ಅಥವಾ ಒಂದಕ್ಕೊಂದು ತಾಕಿದರೆ ಆಚಾರವೇನೂ ಕೆಡುವುದಿಲ್ಲ. ಆಹಾರದಲ್ಲಿ ಹಿಂದೂಗಳಿಗೂ ಯಹೂದ್ಯರಿಗೂ ಎಷ್ಟೋ ಹೋಲಿಕೆ ಇದೆ. ಆದರೆ ಹಿಂದೂಗಳು ತಿನ್ನುವ ಕಾಡುಹಂದಿಯನ್ನು ಯಹೂದ್ಯರು ತಿನ್ನುವುದಿಲ್ಲ. ಹಿಂದೂಗಳಿಗೂ ಮಹಮ್ಮದೀಯರಿಗೂ ಇರುವ ಪರಸ್ಪರ ದ್ವೇಷದಿಂದ ಪಂಜಾಬಿನಲ್ಲಿ ಯಾವುದನ್ನು ಮಹಮ್ಮದೀಯರು ಮಾಡುವುದಿಲ್ಲವೋ ಅದನ್ನು ಹಿಂದೂಗಳು ಮಾಡುತ್ತಾರೆ. ಅಲ್ಲಿಯ ಹಿಂದೂಗಳಿಗೆ ಕಾಡುಹಂದಿಯ ಮಾಂಸ ಅತ್ಯಾವಶ್ಯಕವಾದ ವಸ್ತುವಾಗಿದೆ. ರಾಜಪುತ್ರರಿಗೆ ಕಾಡುಹಂದಿಯನ್ನು ಬೇಟೆಯಾಡುವುದು, ಅದರ ಮಾಂಸವನ್ನು ಭಕ್ಷಿಸುವುದು ಧರ್ಮದ ಒಂದು ಭಾಗ. ದಖನ್ನಿನಲ್ಲಿ ಬ್ರಾಹ್ಮಣರನ್ನು ಬಿಟ್ಟರೆ ಉಳಿದವರೆಲ್ಲರೂ ಹಂದಿಮಾಂಸವನ್ನು ಸೇವಿಸುವುದು ರೂಢಿಯಾಗಿದೆ. ಹಿಂದೂಗಳು ಕಾಡುಕೋಳಿಯನ್ನು ತಿನ್ನುತ್ತಾರೆ, ಆದರೆ ಸಾಕಿದ ಕೋಳಿಯನ್ನು ತಿನ್ನುವುದಿಲ್ಲ.

ಬಂಗಾಳದಿಂದ ನೇಪಾಳದವರೆಗೆ ಮತ್ತು ಕಾಶ್ಮೀರದ ಎಲ್ಲೆಯವರೆಗೂ ಹಿಮಾಲಯದಲ್ಲಿ ಜನರು ಆಹಾರ ವಿಷಯದಲ್ಲಿ ಒಂದೇ ಪದ್ಧತಿಯನ್ನು ಅನುಸರಿಸುವರು. ಈ ಭಾಗಗಳಲ್ಲಿ ಇಂದಿನವರೆಗೂ, ಬಹುಮಟ್ಟಿಗೆ ಮನುವಿನ ನಿಯಮಗಳು ಆಚರಣೆಯಲ್ಲಿವೆ. ಆದರೆ ಅವು ಬಂಗಾಳ, ಬಿಹಾರ, ಅಲಹಾಬಾದ್​, ನೇಪಾಳ ಈ ಭಾಗಗಳಿಗಿಂತ ಹೆಚ್ಚಾಗಿ ಕೊಮಾಓನ್​ನಿಂದ ಕಾಶ್ಮೀರದವರೆಗಿನ ಭಾಗಗಳಲ್ಲಿ ಹೆಚ್ಚು ರೂಢಿಯಲ್ಲಿವೆ. ಉದಾಹರಣೆದೆ ಬಂಗಾಳಿಗಳು ಕೋಳಿ ಅಥವಾ ಕೋಳಿಮೊಟ್ಟೆ ತಿನ್ನುವುದಿಲ್ಲ. ಆದರೆ ಬಾತಿನ ಮೊಟ್ಟೆ ತಿನ್ನುವರು. ನೇಪಾಳಿಯರೂ ಇವರಂತೆಯೇ. ಕುಮಾಓನಿನಾಚೆ ಇದೂ ಇಲ್ಲ. ಕಾಶ್ಮೀರದವರು ಸಂತೋಷದಿಂದ ಕಾಡುಬಾತಿನ ಮೊಟ್ಟೆ ತಿನ್ನುವರು. ಆದರೆ ಸಾಕಿದ ಹಕ್ಕಿಯ ಮೊಟ್ಟೆಯನ್ನು ತಿನ್ನುವುದಿಲ್ಲ. ಅಲಹಾಬಾದಿನಿಂದ ಹಿಡಿದು, ಹಿಮಾಲಯಾ ಪ್ರಾಂತ್ಯವನ್ನುಳಿದು ಭಾರತೀಯರಲ್ಲಿ ಯಾರು ಆಡಿನ ಮಾಂಸವನ್ನು ತಿನ್ನುತ್ತಾರೋ ಅವರು ಕೋಳಿಯನ್ನು ತಿನ್ನುತ್ತಾರೆ.

ಆಹಾರದ ವಿಷಯದ ವಿಧಿ ನಿಷೇಧಗಳೆಲ್ಲ ಬಹುಮಟ್ಟಿಗೆ ಕೇವಲ ಆರೋಗ್ಯ ದೃಷ್ಟಿಯಿಂದ ಆದುವು. ಆದರೆ ಪ್ರತಿಯೊಂದು ಆಹಾರವನ್ನೂ ಈ ದೃಷ್ಚಿಯಿಂದ ನೋಡಲು ಆಗುವುದಿಲ್ಲ. ಕೋಳಿಗಳು ಮತ್ತು ಹಂದಿಗಳು ಸಿಕ್ಕಿದ್ದನ್ನೆಲ್ಲ ತಿನ್ನುತ್ತವೆ ಮತ್ತು ಅವುಗಳು ತೀರ ಕೊಳಕು. ಅದಕ್ಕೇ ಅವನ್ನು ತಿನ್ನುವುದು ನಿಷೇಧ. ಕಾಡಿನಲ್ಲಿರುವುವು ಏನು ತಿನ್ನುತ್ತವೊ ಗೊತ್ತಿಲ್ಲ. ಅದಕ್ಕೇ ಅವನ್ನು ತೆಗೆದುಕೊಳ್ಳಲು ಯಾವ ಅಭ್ಯಂತರವೂ ಇಲ್ಲ. ಸಾಕಿದ ಪ್ರಾಣಿಗಿಂತ ಕಾಡಿನಲ್ಲಿರುವ ಪ್ರಾಣಿ ಆರೋಗ್ಯವಾಗಿರುತ್ತದೆ.

ದೇಹದಲ್ಲಿ ಹೆಚ್ಚು ಆಮ್ಲ (acidity) ಇದ್ದರೆ ಹಾಲನ್ನು ಜೀರ್ಣಿಸಿ ಕೊಳ್ಳುವುದು ಕಷ್ಟ. ಕೆಲವು ವೇಳೆ ಇಂತಹ ಸ್ಥಿತಿಯಲ್ಲಿ ಒಂದು ಲೋಟ ಹಾಲನ್ನು ಗಟಗಟನೆ ಅವಸರದಲ್ಲಿ ಹೀರಿದ್ದರಿಂದಲೇ ಪ್ರಾಣಾಪಾಯವಾಗಿರುವ ಸಂಭವವು ಉಂಟು. ಮಗು ತಾಯಿಯ ಎದೆಯಿಂದ ಹಾಲನ್ನು ತೆಗೆದು ಕೊಳ್ಳುವಂತೆ, ಅಂದರೆ, ಸ್ವಲ್ಪ ಸ್ವಲ್ಪವಾಗಿ ತೆಗೆದುಕೊಳ್ಳಬೇಕು. ಆಗ ಸುಲಭವಾಗಿ ಜೀರ್ಣಿಸಿ ಕೊಳ್ಳಬಹುದು. ಇಲ್ಲದೇ ಇದ್ದರೆ ಕಷ್ಟ. ಇದನ್ನು ಅರಗಿಸಿಕೊಳ್ಳುವುದೇ ಕಷ್ಟ ವಾಗಿರುವಾಗ ಮಾಂಸದೊಂದಿಗೆ ತೆಗೆದುಕೊಂಡರೆ ಇನ್ನೂ ತೊಂದರೆಯಾಗು ವುದು. ಅದಕ್ಕೇ ಯಹೂದ್ಯರಲ್ಲಿ ಹಾಲು, ಮಾಂಸವನ್ನು ಒಟ್ಟಿಗೆ ತೆಗೆದು ಕೊಳ್ಳುವುದು ನಿಷಿದ್ಧ.

ಅವಿವೇಕಿಗಳಾದ ತಾಯಂದಿರು ಬಲಾತ್ಕಾರದಿಂದ ಮಕ್ಕಳಿಗೆ ಅತಿಯಾಗಿ ಹಾಲು ಕುಡಿಸುವರು. ಕೆಲವು ತಿಂಗಳಾದ ಮೇಲೆ ಮಗುವಿನ ಜೀವಕ್ಕೆ ಸಂಚಕಾರ ಬಂದಾಗ ತಲೆ ಹೊಡೆದುಕೊಳ್ಳುವಳು. ಈಗಿನ ವೈದ್ಯರು ವಯಸ್ಕರೂ ಕೂಡ ಒಂದು ಪೈಂಟ್​ ಹಾಲು ಮಾತ್ರ ತೆಗೆದುಕೊಳ್ಳಬೇಕು ಎನ್ನುವರು. ಅದನ್ನು ನಿಧಾನವಾಗಿ ಕುಡಿಯಬೇಕು. ಮಕ್ಕಳಿಗೆ ಬುಡ್ಡಿಯ ಮೂಲಕ ಹಾಲು ಕುಡಿಸು ವುದು ಒಳ್ಳೆಯ ಅಭ್ಯಾಸ. ನಮ್ಮ ತಾಯಂದಿರಿಗೆ ಮನೆಕೆಲಸ ಬಹಳ. ದಾದಿಯು ಅಳುವ ಮಗುವನ್ನು ತೊಡೆಯ ಮೇಲೆ ತೆಗೆದುಕೊಂಡು ಬಲಾತ್ಕಾರವಾಗಿ ಬಾಯಿಗೆ ಹಾಲನ್ನು ಸುರಿಯುವಳು. ಇದರ ಪರಿಣಾಮವಾಗಿ ಮಕ್ಕಳಿಗೆ ಯಕೃತ್​ ಕೆಟ್ಟು ಚೆನ್ನಾಗಿ ಬೆಳೆಯುವುದಿಲ್ಲ. ಈ ಹಾಲೇ ಅದಕ್ಕೆ ಮೃತ್ಯುವಾಗುವುದು. ಯಾವ ಮಕ್ಕಳು ಇದರಿಂದ ಪಾರಾಗುವುವೊ ಅವೇ ಮುಂದೆ ದೃಢಕಾಯ ವಾಗವುವು. ಹಿಂದಿನ ಕಾಲದ ಬಾಣಂತಿ ಮನೆ, ಮಕ್ಕಳಿಗೆ ಕೊಡುತ್ತಿದ್ದ ಬಿಸಿ ನೀರಿನ ಸ್ನಾನ, ಇವನ್ನು ಆಲೋಚಿಸಿ ನೋಡಿ! ಈ ಯಾತನೆಯಿಂದ ಮಗು ಬಾಣಂತಿ ಜೀವ ಸಹಿತ ಉಳಿದುದು ಭಗವಂತನ ದಯೆ ಎನ್ನಬೇಕು.


\section{ಉಡುಪಿನ ಸಭ್ಯತೆ}

ಪ್ರತಿಯೊಂದು ದೇಶದಲ್ಲಿಯೂ ವ್ಯಕ್ತಿಯ ಸಭ್ಯತೆಯನ್ನು ಸ್ವಲ್ಪಮಟ್ಟಿಗೆ ಅವನ ಉಡುಪಿನ ಮೇಲೆ ನಿಷ್ಕರ್ಷಿಸುವರು. ಬಂಗಾಳದ ಗ್ರಾಮಸ್ಥರು ಹೇಳುವಂತೆ, ಒಬ್ಬ ವ್ಯಕ್ತಿಗೆ ಬರುವ ವರಮಾನ ತಿಳಿಯದೆ ಅವನು ಸಾಮಾನ್ಯನೊ ಸಭ್ಯನೊ ಎಂದು ತಿಳಿಯುವುದು ಹೇಗೆ? ವರಮಾನ ಮಾತ್ರವಲ್ಲ; ಅವನ ಉಡುಪನ್ನು ನೋಡಿದರೆ ಮಾತ್ರ ಗೊತ್ತಾಗುವುದು. ಹೆಚ್ಚು ಕಡಿಮೆ ಪ್ರಪಂಚದಲ್ಲೆಲ್ಲ ಇದು ಇರುವುದೇ ಹೀಗೆ. ಬಂಗಾಳದಲ್ಲಿ ಬರೀ ಮುಂಡು ಸುತ್ತಿಕೊಂಡು ಯಾವ ಸಭ್ಯನೂ ಬೀದಿಯಲ್ಲಿ ನಡೆಯುವುದಿಲ್ಲ. ಉಳಿದ ಕಡೆಗಳಲ್ಲಿ ಪೇಟವಿಲ್ಲದೆ ಹೊರಗೆ ಹೋಗು ವುದಿಲ್ಲ. ಪಾಶ್ಚಾತ್ಯ ದೇಶಗಳಲ್ಲಿ ಎಲ್ಲರಿಗೂ ಮುಂದಾಳುಗಳು ಫ್ರೆಂಚರು. ಅವರ ಉಡುಪು ಮತ್ತು ಆಹಾರವನ್ನು ಎಲ್ಲರೂ ಅನುಕರಿಸುವರು. ಯೂರೋಪಿನ ಹಲವು ಭಾಗಗಳಲ್ಲಿ ಉಡುಪು ಆಹಾರ ಭಿನ್ನಭಿನ್ನವಾಗಿದ್ದರೂ, ಒಬ್ಬ ಹೆಚ್ಚು ಸಂಪಾದಿಸಿ ಶ‍್ರೀಮಂತನಾದರೆ, ತನ್ನ ಹಿಂದಿನ ಉಡುಪನ್ನು ತೊರೆದು ಫ್ರೆಂಚರನ್ನು ಅನುಕರಿಸುವನು. ಡಚ್​ ರೈತರ ಉಡುಪು ಕಾಬೂಲಿಯರ ಪಾಯಜಾಮೆಯಂತೆ ಇದೆ. ಗ್ರೀಕರು ಉದ್ದನೆಯ ನಿಲುವಂಗಿ ಹಾಕಿಕೊಳ್ಳುವರು. ರಷ್ಯನರು ತಿಬೆಟ್ಟಿ ನವರಂತೆ ಬಟ್ಟೆ ಹಾಕಿಕೊಳ್ಳುವರು. ಆದರೆ ಇವರೆಲ್ಲ “ಜೆಂಟಲ್​ ಮ್ಯಾನ್​” ಆದೊಡನೆಯೆ ಫ್ರೆಂಚರಂತೆ ಅಂಗಿ ಷರಾಯಿಯನ್ನು ಧರಿಸುತ್ತಾರೆ. ಹೆಂಗಸರ ವಿಷಯವನ್ನು ಹೇಳಬೇಕಾಗಿಯೇ ಇಲ್ಲ. ಸ್ವಲ್ಪ ಶ‍್ರೀಮಂತರಾದರೆ ಸಾಕು, ಪ್ಯಾರಿಸ್ಸಿ ನಿಂದಲೇ ಅವರಿಗೆ ಉಡುಪು ಬೇಕು. ಪಾಶ್ಚಾತ್ಯರಲ್ಲಿ ಅಮೆರಿಕ, ಇಂಗ್ಲೆಂಡ್​, ಜರ್ಮನಿಯ ದೇಶದ ಐಶ್ವರ್ಯವಂತರ ಉಡುಪೆಲ್ಲಾ ಫ್ರೆಂಚ್​ ಫ್ಯಾಶನ್ನಿನಲ್ಲಿ ತಯಾರಾಗುವುದು. ಈ ಅಭ್ಯಾಸ ಯೂರೋಪಿನ ಎಲ್ಲಾ ದೇಶಕ್ಕೂ ಹಬ್ಬುತ್ತಿದೆ. ಯೂರೋಪೆಲ್ಲ ಫ್ರಾನ್ಸಿನ ಅನುಕರಣೆಯಂತೆ ಕಾಣುವುದು. ಆದರೆ ಈಗ ಪುರುಷರು ಉಡುಪನ್ನು ಲಂಡನ್ನಿನಲ್ಲಿ ಫ್ರಾನ್ಸಿಗಿಂತ ಚೆನ್ನಾಗಿ ತಯಾರಿಸುವರು. ಅದಕ್ಕೇ ಈಗ ಪುರುಷರ ಉಡುಪನ್ನು ಲಂಡ್ಧನ್ನ್ಧಿನ್ಧಲ್ಲ್ಧಿಯೂ, ಸ್ತ್ರೀಯ್ಧರ ಉಡ್ಧುಪ್ಧನ್ನು ಪ್ಯಾರಿಸ್ಸಿನಲ್ಲಿಯೂ ತಯಾರಿಸುವರು. ಯಾರು ತುಂಬಾ ಶ‍್ರೀಮಂತರೋ ಅವರು ಈ ಎರಡು ಸ್ಥಳಗಳಿಂದ ಉಡುಪನ್ನು ತರಿಸುವರು. ಅಮೇರಿಕಾ ದೇಶದಲ್ಲಿ ಹೊರಗಿನಿಂದ ಬರುವ ಬಟ್ಟೆಗೆ ಬೇಕಾದಷ್ಟು ಸುಂಕ ಹಾಕುವರು. ಆದರೂ ಅಲ್ಲಿ ಮಹಿಳೆಯರಿಗೆ ಲಂಡನ್​ ಮತ್ತು ಪ್ಯಾರಿಸ್ಸಿನಿಂದಲೇ ಅವು ಬರಬೇಕು. ಅಮೇರಿಕಾ ದೇಶೀಯರಿಗೆ ಮಾತ್ರ ಹೀಗೆ ಮಾಡುವುದಕ್ಕೆ ಸಾಧ್ಯ. ಅವರು ಈಗ ಕುಬೇರನ ಮಕ್ಕಳು.

ಪ್ರಾಚೀನ ಆರ್ಯರು ಧೋತ್ರ ಮತ್ತು ಉತ್ತರೀಯವನ್ನು ಧರಿಸುತ್ತಿದ್ದರು. ಕ್ಷತ್ರಿಯರು ಯುದ್ಧದಲ್ಲಿ ಷರಾಯಿ ಮತ್ತು ನಿಲುವಂಗಿಯನ್ನು ತೊಡುತ್ತಿದ್ದರು. ಉಳಿದ ಕಾಲದಲ್ಲಿ ಧೋತ್ರ ಮತ್ತು ಉತ್ತರೀಯವನ್ನು ಧರಿಸುತ್ತಿದ್ದರು. ತಲೆಗೆ ಪೇಟವನ್ನು ಕಟ್ಟುತ್ತಿದ್ದರು. ಅದೇ ಅಭ್ಯಾಸ ಈಗ ಬಂಗಾಳ ದೇಶವೊಂದನ್ನು ಬಿಟ್ಟು ಉಳಿದ ಕಡೆಯಲ್ಲೆಲ್ಲಾ ರೂಢಿಯಲ್ಲಿದೆ. ದೇಹದ ಇತರ ಭಾಗದ ಮೇಲೆ ಏನಾದರೂ ಇರಲಿ, ಇಲ್ಲದೆ ಇರಲಿ, ತಲೆಯ ಮೇಲೆ ರುಮಾಲು ಇರಬೇಕು. ಹಿಂದಿನ ಕಾಲದಲ್ಲಿ ಸ್ತ್ರೀಪುರುಷರಲ್ಲಿ ಇದೇ ರೂಢಿಯಲ್ಲಿತ್ತು. ಬೌದ್ಧರ ಕಾಲದಲ್ಲಿ ಕೆತ್ತಿದ ವಿಗ್ರಹಗಳಲ್ಲಿ ಸ್ತ್ರೀಪುರುಷರಿಬ್ಬರಿಗೂ ಕೇವಲ ಒಂದು ಕೌಪೀನ ಮಾತ್ರ ವಿದೆ. ದೊರೆಯಾದ ಬುದ್ಧರ ತಂದೆ ಕೂಡ ಸಿಂಹಾಸನದ ಮೇಲೆ ಕುಳಿತಾಗ ಬರೀ ಕೌಪೀನವನ್ನು ಮಾತ್ರ ಧರಿಸಿರುವುದು ಕೆಲವು ಕಡೆ ಕಾಣುವುದು. ಅದರಂತೆ ಅವನ ತಾಯಿ ಕೂಡ. ಆದರೆ ಜೊತೆಗೆ ಅವಳ ತೋಳು ಕಾಲುಗಳಿಗೆ ಆಭರಣ ಗಳಿವೆ. ಆದರೆ ಎಲ್ಲರಿಗೂ ಪೇಟವಿದೆ. ಬೌದ್ಧ ಚಕ್ರವರ್ತಿ ಧರ್ಮಾಶೋಕನು ಡಮರುವಿನಂತೆ ಇರುವ ಸಿಂಹಾಸನದ ಮೇಲೆ ಪಂಚೆ ಉಟ್ಟುಕೊಂಡು ಉತ್ತರೀಯವನ್ನು ಹೊದ್ದು ಕುಳಿತು ಎದುರಿಗೆ ನೃತ್ಯ ಮಾಡುತ್ತಿರುವ ಬಾಲಕಿಯ ರನ್ನು ನೋಡುತ್ತಿರುವನು. ನರ್ತಕಿಯರ ಉಡಿಗೆ ತುಂಬಾ ಕಡಿಮೆ. ಸೊಂಟದಿಂದ ಮಂಡಿಯವರೆಗೆ ಒಂದು ತುಂಡು ಬಟ್ಟೆ ಮಾತ್ರ. ಆದರೆ ತಲೆಯ ಮೇಲೆ ರುಮಾಲು ಇದೆ. ಇದೇ ಮುಖ್ಯ. ಆದರೆ ಸುತ್ತಲೂ ಇರುವ ಸಾಮಂತರು ಷರಾಯಿ, ಉದ್ದವಾದ ಅಂಗಿ ಇವನ್ನು ಧರಿಸಿರುವರು. ಋತುಪರ್ಣ ರಾಯನಿಗೆ ನಳನು ಸಾರಥಿಯಾಗಿದ್ದಾಗ ದಮಯಂತಿ ಸ್ವಯಂವರಕ್ಕೆ ಋತಪರ್ಣ ಹೋಗುತ್ತಿರುವಾಗ, ನಳ ರಥವನ್ನು ನಡೆಸುತ್ತಿದ್ದ. ಆ ವೇಗಕ್ಕೆ ಋತುಪರ್ಣನ ಉತ್ತರೀಯ ಹಾರಿಹೋಯಿತು. ಉತ್ತರೀಯವಲ್ಲದೆ ಋತುಪರ್ಣ ಸ್ವಯಂವರವನ್ನು ಸೇರಿದನು. ಧೋತ್ರ ಮತ್ತು ಉತ್ತರೀಯ ಹಿಂದಿನಿಂದಲೂ ಬಂದ ಆರ್ಯರ ಉಡುಪು. ಆದಕಾರಣ ಈಗಲೂ ಯಾವುದಾದರೂ ಕ್ರಿಯಾಕರ್ಮಗಳನ್ನು ಮಾಡಬೇಕಾದರೆ ಧೋತ್ರ ಮತ್ತು ಉತ್ತರೀಯವನ್ನು ಮಾತ್ರ ಧರಿಸುವರು.

ಪ್ರಾಚೀನ ಗ್ರೀಕ್​ ಮತ್ತು ರೋಮನ್ನರ ಪೋಷಾಕು ಕೂಡ ಧೋತ್ರ ಮತ್ತು ಉತ್ತರೀಯ. ಅವರು ಉದ್ದವಾದ ಒಂದು ಬಟ್ಟೆಯಲ್ಲಿ ಧೋತ್ರ ಮತ್ತು ಉತ್ತರೀಯ ವನ್ನು ಮಾಡುತ್ತಿದ್ದರು. ಇದಕ್ಕೆ “ಟೋಗಾ” ಎನ್ನುವರು. ಈಗ ಅದರ ಅಪಭ್ರಂಶ “ಚೋಗ”. ಕೆಲವು ವೇಳೆ ಅಂಗಿಯನ್ನು ಹಾಕಿಕೊಳ್ಳುತ್ತಿದ್ದರು. ಯುದ್ಧ ಕಾಲದಲ್ಲಿ ಷರಾಯಿ ಮತ್ತು ಕೋಟನ್ನು ಧರಿಸುತ್ತಿದ್ದರು. ಸ್ತ್ರೀಯರ ಪೋಷಾಕು ಕೂಡ ಉದ್ದವಾದ ಚೌಕಾಕಾರದಲ್ಲಿರುವ ಒಂದು ಬಟ್ಟೆ. ಅದನ್ನು ಎದೆಯ ಮೇಲೆ ಒಂದು ಸಲ, ಸೊಂಟದ ಮೇಲೆ ಒಂದು ಸಲ ಕಟ್ಟುತ್ತಿದ್ದರು. ಮೇಲೆ ಕಟ್ಟುವಾಗ ಕೈ ಸಂಚಾರಕ್ಕೆ ಅವಕಾಶವಾಗುವಂತೆ ಎರಡು ಸೂಜಿಯಿಂದ ಸಿಕ್ಕಿಸುತ್ತಿದ್ದರು. ಉತ್ತರ ಹಿಮಾಲಯದಲ್ಲಿ ಪಹಾಡಿ ಜನರು ಈಗಲೂ ತಮ್ಮ ಕಂಬಳಿಯನ್ನು ಆ ರೀತಿ ಧರಿಸುತ್ತಾರೆ. ಈ ದೊಡ್ಡ ಬಟ್ಟೆಯ ಮೇಲೆ ಒಂದು ಉತ್ತರೀಯ ಇರುತ್ತಿದ್ದರು. ಈ ಪೋಷಾಕು ಸುಂದರವಾಗಿಯೂ ಸೊಗಸಾಗಿಯೂ ಇರುತ್ತಿತ್ತು.

ಪ್ರಾಚೀನ ಕಾಲದಿಂದಲೂ ಇರಾನೀಯರು ಮಾತ್ರ ಕತ್ತರಿಸಿ ಹೊಲಿಗೆ ಬಟ್ಟೆಯನ್ನು ಉಪಯೋಗಿಸುತ್ತಿದ್ದರು. ಇದನ್ನು ಅವರು ಬಹುಶಃ ಚೈನೀಯರಿಂದ ಕಲಿತಿರಬಹುದು. ಚೈನೀದೇಶೀಯರು ಸಭ್ಯತೆಗೆ, ಭೋಗವಿಲಾಸಕ್ಕೆ, ಸುಖ ಸ್ವಚ್ಛಂದಕ್ಕೆ ಆದಿ ಗುರುಗಳು. ಅನಾದಿಕಾಲದಿಂದಲೂ ಚೈನೀಯರು ಮೇಜಿನ ಮೇಲೆ ಊಟ ಮಾಡುತ್ತಿದ್ದರು. ಕುರ್ಚಿಯ ಮೇಲೆ ಕುಳಿತುಕೊಳ್ಳುತ್ತಿದ್ದರು. ಊಟ ಮಾಡುವ ಸಮಯದಲ್ಲೂ ಹಲವಾರು ಯಂತ್ರ ತಂತ್ರಗಳನ್ನು ಉಪಯೋಗಿಸುತ್ತಿದ್ದರು. ಹಲವು ಬಗೆಯ ಅಂಗಿ, ಟೋಪಿ, ಷರಾಯಿಯನ್ನು ಧರಿಸುತ್ತಿದ್ದರು.

ಅಲೆಕ್ಸಾಂಡರ್​ ಇರಾನನ್ನು ಗೆದ್ದ ಮೇಲೆ ಧೋತ್ರ, ಉತ್ತರೀಯ ಬಿಟ್ಟು ಪಾಯಿಜಾಮ ಧರಿಸಲು ಪ್ರಾರಂಭಿಸಿದರು. ಇದನ್ನು ನೋಡಿ ಆತನ ದೇಶೀಯ ಸೈನಿಕರಿಗೆ ಆಗಲಿಲ್ಲ. ಅವನ ಮೇಲೆ ದಂಗೆ ಏಳುವುದರಲ್ಲಿದ್ದರು. ಆದರೆ ಅಲೆಕ್ಸಾಂಡರ್​ ಎದೆಗೆಡದೆ ಹೊಸ ಪೋಷಾಕನ್ನು ಬಲವಂತಾಗಿ ಪ್ರಚಾರಕ್ಕೆ ತಂದನು.

ಉಷ್ಣದೇಶದಲ್ಲಿ ಬಟ್ಟೆಯ ಆವಶ್ಯಕತೆ ಇರುವುದಿಲ್ಲ. ನಾಚಿಕೆಯನ್ನು ಹೋಗ ಲಾಡಿಸಲು ಕೌಪೀನ ಒಂದು ಸಾಕು. ಉಳಿದುವೆಲ್ಲ ಕೇವಲ ಅಲಂಕಾರ. ಶೀತ ದೇಶದಲ್ಲಿ ಜನರು ಅನಾಗರಿಕರಾಗಿದ್ದಾಗ ಸದಾ ಚಳಿಯಿಂದ ಪೀಡಿತರಾಗಿ ಪ್ರಾಣಿಯ ಚರ್ಮವನ್ನು ಹೊದೆಯುತ್ತಿದ್ದರು. ಕ್ರಮೇಣ, ನಾಗರಿಕರಾದಂತೆ ಕಂಬಳಿಯನ್ನು ಹೊದೆಯುವುದನ್ನು ಕಲಿತರು. ಅನಂತರ ಕೋಟು, ಷರಾಯಿ ಎಲ್ಲಾ ಜಾರಿಗೆ ಬಂದವು. ಶೀತ ದೇಶದಲ್ಲಿ ದೇಹದ ಮೇಲೆ ಧರಿಸುವ ಒಡವೆ ಯನ್ನು ಪ್ರದರ್ಶಿಸಲು ಆಗಿವುದಿಲ್ಲ. ಆದಕಾರಣವೆ ಒಡವೆಯ ಮೇಲೆ ಇರುವ ಆಸೆ ಉಡುಪಿನ ಮೇಲೆ ಹೋಗುವುದು. ಇಂಡಿಯಾ ದೇಶದಲ್ಲಿ ಒಡವೆ ಫ್ಯಾಶನ್​ ಹೇಗೆ ಬದಲಾಯಿಸುವುದೋ ಹಾಗೆಯೇ ಪಾಶ್ಚಾತ್ಯರಲ್ಲಿ ಉಡುಪು ಬದಲಾಯಿಸು ವುದು. ಶೀತ ದೇಶದಲ್ಲಿ ದೇಹದ ಎಲ್ಲಾ ಭಾಗವನ್ನೂ ಬಟ್ಟೆಯಿಂದ ಮುಚ್ಚದೆ ಯಾರ ಎದುರಿಗಾದರೂ ಕಾಣುವುದು ಅಸಭ್ಯತೆ. ಲಂಡನ್ನಿನಲ್ಲಿ ಚೆನ್ನಾಗಿ ಬಟ್ಟೆ ಹಾಕಿಕೊಳ್ಳದೆ ಮನೆಯಿಂದ ಹೊರಗೆ ಹೋಗುವುದಕ್ಕೇ ಆಗುವುದಿಲ್ಲ. ಪಾಶ್ಚಾತ್ಯ ದೇಶಾದಲ್ಲಿ ಹೆಂಗಸರು ತಮ್ಮ ಕಾಲನ್ನು ತೋರುವುದೂ ಅಸಭ್ಯತೆ. ಆದರೆ ನೃತ್ಯ ಮಾಡುವಾಗ ಮುಖ, ತೋಳು, ಎದೆಯ ಭಾಗವನ್ನು ತೋರಿದರೆ ಪರವಾಗಿಲ್ಲ. ನಮ್ಮ ಕಡೆ ಹೆಂಗಸರಿಗೆ ಮುಖವನ್ನು ತೋರಿಸುವುದಕ್ಕೆ ತುಂಬಾ ನಾಚಿಕೆ. ಅದಕ್ಕೇ ಮುಸುಕು ಹಾಕಿಕೊಳ್ಳುವುದು. ಕಾಲನ್ನು ಬೇಕಾದರೆ ತೋರಿಸಬಹುದು. ರಾಜಪುತ್ರಸ್ಥಾನ ಮತ್ತು ಹಿಮಾಲಯ ಪ್ರಾಂತ್ಯದಲ್ಲಿ ಸೊಂಟವೊಂದು ಬಿಟ್ಟು ಮಿಕ್ಕ ದೇಹವನ್ನೆಲ್ಲ ಮುಚ್ಚಿಕೊಳ್ಳುವರು.

ಪಾಶ್ಚಾತ್ಯ ದೇಶಗಳಲ್ಲಿ ನರ್ತಕಿ ಮತ್ತು ವೇಶ್ಯೆಯರು ಪುರುಷರನ್ನು ಆಕರ್ಷಿಸು ವುದಕ್ಕಾಗಿ ಬಹಳ ಕಡಿಮೆ ಬಟ್ಟೆ ಉಪಯೋಗಿಸುವರು. ಅವರ ನೃತ್ಯವೆಂದರೆ ಸಂಗೀತಕ್ಕೆ ಸರಿಯಾಗಿ ದೇಹದ ಅವಯವಗಳನ್ನು ಚಲನವಲನಗಳ ಮೂಲಕ ತೋರಿಸುವುದು. ನಮ್ಮ ದೇಶದಲ್ಲಿ ಗೌರವಸ್ಥ ಹೆಂಗಸರು ಕೂಡ ಹೆಚ್ಚು ಬಟ್ಟೆಯನ್ನು ಉಪಯೋಗಿಸುವುದಿಲ್ಲ. ಆದರೆ ನರ್ತಕಿಯರು ಬೇಕಾದಷ್ಟು ಬಟ್ಟೆಯನ್ನು ಉಪಯೋಗಿಸುವರು. ಪಾಶ್ಚಾತ್ಯರಲ್ಲಿ ಹಗಲು ಹೊತ್ತು ಹೆಂಗಸರು ಪೋಷಾಕಿನಿಂದ ತಮ್ಮ ದೇಹವನ್ನು ಯಾವಾಗಲೂ ಮುಚ್ಚಿಕೊಳ್ಳುವರು. ಅದಕ್ಕೇ ಅವರು ಕಡಿಮೆ ಬಟ್ಟೆ ಹಾಕಿ ಕಂಡಾಗ ಆಕರ್ಷಣೆ ಹೆಚ್ಚು. ನಮ್ಮ ಹೆಂಗಸರು ಯಾವಾಗಲೂ ಮನೆಯಲ್ಲಿಯೇ ಇರುವರು. ಹೆಚ್ಚು ಪೋಷಾಕನ್ನು ಉಪಯೋಗಿ ಸುವುದಿಲ್ಲ. ಆದಕಾರಣವೇ ಹೆಚ್ಚು ಬಟ್ಟೆ ಧರಿಸಿ ಆಕರ್ಷಿಸುವರು. ಮಲಬಾರಿನಲ್ಲಿ ಸ್ತ್ರೀಪುರುಷರಿಬ್ಬರೂ ಒಂದು ಪಂಚೆಯನ್ನು ಮಾತ್ರ ಧರಿಸುವರು. ಬಂಗಾಳ ದೇಶದಲ್ಲಿ ಹೆಂಗಸರು ಕಡಿಮೆ ಪೋಷಾಕು ಧರಿಸಿದ್ದರೂ ಪುರುಷರ ಮುಂದೆ ಮುಸುಕನ್ನು ಹಾಕಿಕೊಳ್ಳುವರು.

ಚೈನಾದೇಶವನ್ನು ಉಳಿದ ಎಲ್ಲ ಕಡೆಗಳಲ್ಲಿಯೂ ಸಭ್ಯತೆಯ ವಿಷಯದಲ್ಲಿ ವಿಚಿತ್ರವಾದ ಭಾವನೆಗಳಿರುವುದನ್ನು ನಾನು ಗಮನಿಸಿದ್ದೇನೆ. ಕೆಲವು ವಿಷಯಗಳಲ್ಲಿ ಅವರು ಅತಿರೇಕಕ್ಕೆ ಹೋಗುತ್ತಾರೆ. ಇನ್ನೂ ಕೆಲವು ವಿಷಯಗಳಲ್ಲಿ ಯಾವುದು ದೋಷಪೂರ್ಣ ಎನಿಸುತ್ತದೆಯೋ ಅದು ಅವರಿಗೆ ಹಾಗೆಂದು ತೋರುತ್ತಲೇ ಇರಲಿಲ್ಲ. ಚೈನಾದಲ್ಲಿ ಸ್ತ್ರೀಪುರುಷರಿಬ್ಬರೂ ಆಪಾದಮಸ್ತಕ ಬಟ್ಟೆಯಿಂದ ಮುಚ್ಚಿ ಕೊಳ್ಳುತ್ತಾರೆ. ಚೀನೀಯರೆಲ್ಲ ಕನ್ಫ್ಯೂಷಿಯಸ್​ ಮತ್ತು ಬುದ್ಧನ ಅನುಯಾಯಿಗಳು. ಅದ್ಧರಿಂದ ನಯನೀತಿಯಲ್ಲಿ ಹೆಚ್ಚು ಕಟ್ಟುನಿಟ್ಟು; ಅಶ್ಲೀಲ ಭಾಷೆ, ಪುಸ್ತಕ, ಚಿತ್ರ, ವರ್ತನೆ ಮುಂತಾದುವನ್ನು ಅವರು ಸಹಿಸರು. ಆ ವಿಷಯದಲ್ಲಿ ದೋಷಿಗಳನ್ನು ತಕ್ಷಣವೇ ಶಿಕ್ಷಿಸುವರು. ಕ್ರೈಸ್ತ ಮಿಷನರಿಗಳು ಬೈಬಲನ್ನು ಚೀನೀಭಾಷೆಗೆ ಅನುವಾದಿಸಿದರು. ಬೈಬಲಲ್ಲಿ ಎಂಥ ಅಶ್ಲೀಲ ಭಾಗಗಳಿವೆಯೆಂದರೆ ಹಿಂದೂಗಳ ಪುರಾಣಗಳೂ ಅವುಗಳ ಮುಂದೆ ತಲೆತಗ್ಗಿಸಬೇಕು! ಅಂಥ ಅಸಭ್ಯ ಭಾಗಗಳನ್ನು ಓದಿ ಚೀನೀಯರಿಗೆ ದಿಗ್ಭ್ರಮೆಯಾಯಿತು. ಅವರು ದೇಶದಲ್ಲಿ ಬೈಬಲ್​ ಪ್ರಚಾರವಾಗುವುದಕ್ಕೆ ಬಿಡಲಿಲ್ಲ. ಇಷ್ಟೇ ಅಲ್ಲದೆ, ಮಿಷನರಿ ಸ್ತ್ರೀಯರು ರಾತ್ರಿಯ ಉಡುಪು ಧರಿಸಿ ಗಂಡಸರೊಂದಿಗೆ ಮುಕ್ತವಾಗಿ ಬೆರೆಯುತ್ತಿದ್ದರು. ಸರಳ ಸ್ವಭಾವದ ಚೀನೀಯರು ಇದನ್ನೆಲ್ಲ ನೋಡಿ ಹೇಸಿ ದೊಡ್ಡ ರಾದ್ಧಾಂತ ಎಬ್ಬಿಸಿದರು. “ಹೋ! ಇದು ಘೋರ! ಈ ಧರ್ಮ ನಮ್ಮ ಹುಡುಗರನ್ನೆಲ್ಲ ಹಾಳುಮಾಡುವು ದಕ್ಕೆ ಬಂದಿದೆ. ಈ ಬೈಬಲನ್ನು ಓದಿ, ಈ ಅರೆಬತ್ತಲೆ ಸ್ತ್ರೀಯರ ಬಲೆಗೆ ಬಿದ್ದು ನಮ್ಮವರು ಹಾಳಾಗಿ ಹೋಗುತ್ತಾರೆ” - ಎಂದು ಗುಲ್ಲೆಬ್ಬಿಸಿದರು. ಆದ್ದರಿಂದಲೇ ಚೀನೀಯರಿಗೆ ಕ್ರೈಸ್ತಧರ್ಮವನ್ನು ಕಂಡರಾಗದು. ಇಲ್ಲದಿದ್ದರೆ, ಚೀನೀಯರು ಪರಧರ್ಮ ಸಹಿಷ್ಣುಗಳು. ಮಿಷನರಿಗಳು ಈಗ ಬೈಬಲ್ಲಿನಲ್ಲಿ ಆಕ್ಷೇಪಣೀಯ ಭಾಗ ಗಳನ್ನು ತೆಗೆದು ಹಾಕಿ ಅದರ ಹೊಸ ಮುದ್ರಣವನ್ನು ಪ್ರಕಟಿಸಿರುವರೆಂದು ಕೇಳಿ ದ್ದೇನೆ. ಆದರೆ ಇದು ಚೀನೀಯರಿಗೆ ಇನ್ನೂ ಸಂದೇಹಕ್ಕೆ ಆಸ್ಪದವನ್ನುಂಟುಮಾಡಿದೆ.


\section{ಶಿಷ್ಟಾಚಾರ}

ಪಶ್ಚಿಮದಲ್ಲೂ ಶಿಷ್ಟಾಚಾರ ಮತ್ತು ಸಭ್ಯತೆಗಳ ವಿಷಯದ ಭಾವನೆಗಳು ದೇಶದಿಂದ ದೇಶಕ್ಕೆ ಬದಲಾಗುತ್ತದೆ. ಇಂಗ್ಲಿಷರು ಮತ್ತು ಅಮೆರಿಕನರದು ಒಂದು ಬಗೆ, ಫ್ರೆಂಚರದು ಬೇರೆ, ಜರ್ಮನರೂ ಅಷ್ಟೆ. ಟಿಬೆಟ್ಟಿನವರು ಮತ್ತು ರಷ್ಯನರಲ್ಲಿ ಸಮಾನವಾದ ಕೆಲವು ಆಚಾರಗಳವೆ ತುರ್ಕಿಯವರಿಗೆ ತಮ್ಮದೇ ಶಿಷ್ಟಾಚಾರಗಳಿವೆ. ಯೂರೋಪ್​ ಮತ್ತು ಅಮೇರಿಕಾ ದೇಶದವರು ತಮ್ಮ ಸ್ನಾನ, ಮಲಮೂತ್ರ ವಿಸರ್ಜನೆ ಮುಂತಾದುವನ್ನು ಹೆಚ್ಚು ಗೋಪ್ಯದಲ್ಲಿ ನಿರ್ವಹಿಸಲು ಬಯಸುತ್ತಾರೆ. ನಾವು ಶಾಖಾಹಾರಿಗಳು, ಉಷ್ಣ ದೇಶಲ್ಲಿರುವುದರಿಂದ ಮಧ್ಯೆ ಮಧ್ಯೆ ನೀರು ಕುಡಿಯುತ್ತೇವೆ. ಉತ್ತರ ಪ್ರಾಂತ್ಯದಲ್ಲಿ ರೈತರು ಹೊತ್ತಿಗೆ ಎರಡು ಪೌಂಡು ಹುರಿಟ್ಟು ತಿನ್ನುತ್ತಾನೆ. ಬಾಯಾರಿಕೆ ಆದಂತೆ ಬಾವಿಯಿಂದ ನೀರು ಸೇದಿ ಕುಡಿಯಲು ಪ್ರಾರಂಭಿಸುವನು. ಬಿಸಿಲು ಕಾಲದಲ್ಲಿ ಬಾಯಾರಿದವನಿಗೆ ಒಂದು ಸಣ್ಣ ಗಳುವಿನ ಕೊಳವಿ ಮೂಲಕ ನೀರು ಕೊಡಲು ಮನೆಯಲ್ಲಿ ಅರವಟ್ಟಿಗೆಯನ್ನು ಇಡುತ್ತೇವೆ. ಇದರಿಂದ ರಹಸ್ಯವಾಗಿ ನೀರು ಕುಡಿಯಲು ಸಾಧ್ಯವಿಲ್ಲ. ಹಸುವಿನ ಗೊಂತು, ಕುದುರೆ ಲಾಯಗಳನ್ನು ಸಿಂಹ ಅಥವಾ ಹುಲಿಯ ಪಂಜರದೊಂದಿಗೆ ಹೋಲಿಸಿ. ನಾಯಿಯನ್ನು ಮೇಕೆಯೊಂದಿಗೆ ಹೋಲಿಸಿ. ಪಾಶ್ಚಾತ್ಯರ ಆಹಾರ ಮುಖ್ಯವಾಗಿ ಮಾಂಸ. ಶೀತ ದೇಶದಲ್ಲಿ ಅವರು ನೀರು ಕುಡಿಯವುದೇ ಅಪರೂಪ. ಶ‍್ರೀಮಂತರು ಸಣ್ಣ ಲೋಟದಲ್ಲಿ ಮದ್ಯ ತೆಗೆದುಕೊಳ್ಳುವರು. ಫ್ರೆಂಚರಿಗೆ ನೀರು ಕಂಡರೆ ಆಗದು. ಅಮೆರಿಕಾದವರು ಮಾತ್ರ ಯಥೇಚ್ಛವಾಗಿ ಜಲಪಾನ ಮಾಡುವರು. ಏಕೆಂದರೆ ಬೇಸಗೆಯಲ್ಲಿ ಬಿಸಿಲುತಾಪ ಹೆಚ್ಚು ಅಲ್ಲಿ. ನ್ಯೂಯಾಕ್​ ಕಲ್ಕತ್ತೆಗಿಂತ ಹೆಚ್ಚು ಉಷ್ಣ. ಜರ್ಮನರು ಬೇಕಾದಷ್ಟು ಬೀರ್​ ಕುಡಿಯುವರು, ಆದರೆ ಊಟದೊಂದಿಗೆ ಅಲ್ಲ.

ಶೀತದೇಶದಲ್ಲಿ ನೆಗಡಿಯಾಗುವ ಪ್ರಮೇಯ ಯಾವಾಗಲೂ ಹೆಚ್ಚು. ಅವರಿಗೆ ಸೀನದೆ ವಿಧಿಯಿಲ್ಲ. ಉಷ್ಣದೇಶಗಳಲ್ಲಿ ಭೋಜನ ಸಮಯದಲ್ಲಿ ಬೇಕಾದಷ್ಟು ನೀರು ಕುಡಿಯಲೇ ಬೇಕು. ಆದಕಾರಣ ತೇಗದೇ ಇರುವುದಕ್ಕೆ ಆಗುವುದಿಲ್ಲ. ಈ ಅಭ್ಯಾಸವನ್ನು ಇಲ್ಲಿ ಗಮನಿಸಿ. ಪಾಶ್ಚಾತ್ಯ ದೇಶದಲ್ಲಿ ಕ್ಷಮಿಸಲಾಗದ ತಪ್ಪು ಇದು. ಆದರೆ ಕರವಸ್ತ್ರವನ್ನು ತೆಗೆದು ಮೂಗಿನ ಗೊಣ್ಣೆಯನ್ನು ಎಲ್ಲರೆದುರಿಗೆ ಒರಸಿದರೆ ಪರವಾಗಿಲ್ಲ. ನಮ್ಮಲ್ಲಿ ಆತಿಥೇಯನಿಗೆ ಅತಿಥಿಯು ತೇಗದ ಹೊರತು ಸಮಾಧಾನ ವಿಲ್ಲ. ಇದು ಹೊಟ್ಟೆತುಂಬಿದ ಚಿಹ್ನೆ. ಪಾಶ್ಚಾತ್ಯರೊಂದಿಗೆ ಊಟ ಮಾಡುವಾಗ ಹೀಗೆ ಮಾಡಿದರೆ ಅವರು ಏನು ತಿಳಿದುಕೊಳ್ಳುತ್ತಾರೆ?

ಇಂಗ್ಲೆಂಡ್​ ಮತ್ತು ಅಮೆರಿಕಾ ದೇಶಗಳಲ್ಲಿ ಸ್ತ್ರೀಯರೆದುರಿಗೆ ಅಜೀರ್ಣ ಅಥವಾ ಯಾವುದೇ ಉದರ ರೋಗಗಳ ಪ್ರಸ್ತಾಪವನ್ನು ಮಾಡಕೂಡದು. ಸ್ತ್ರೀಯರು ಮುದುಕಿಯರಾಗಿದ್ದರೆ ಅಥವಾ ಅವರು ಹೆಚ್ಚು ಪರಿಚಿತರಾಗಿದ್ದರೆ ಚಿಂತೆಯಿಲ್ಲ. ಫ್ರಾನ್ಸ್​ ದೇಶದಲ್ಲಿ ಜನ ಇದನ್ನು ಅಷ್ಟು ಗಮನಿಸುವುದಿಲ್ಲ. ಜರ್ಮನರು ಗಮನಿಸುವುದು ಇದಕ್ಕಿಂತಲೂ ಕಡಿಮೆ.

ಇಂಗ್ಲಿಷ್​ ಮತ್ತು ಅಮೆರಿಕಾದವರು ಹೆಂಗಸರೊಂದಿಗೆ ಮಾತನಾಡುವಾಗ ಅತಿ ಜೋಪಾನವಾಗಿರಬೇಕು. ಕಾಲಿನ ಹೆಸರನ್ನು ಕೂಡ ಹೇಳಕೂಡದು. ಫ್ರೆಂಚರು ನಮ್ಮಂತೆ ಎಲ್ಲರೆದುರಿಗೆ ಮಾತನಾಡುವರು. ಜರ್ಮನ್​ ಮತ್ತು ರಷ್ಯನರು ಎಲ್ಲರೆದುರಿಗೆ ಅಶ್ಲೀಲವಾಗಿ ಮಾತನಾಡಲು ಸಂಕೋಚಪಡುವುದಿಲ್ಲ.

ಪ್ರೇಮದ ವಿಷಯವನ್ನು ಎಲ್ಲರೆದುರಿಗೆ ಮಾತನಾಡುವರು-ತಾಯಿ, ಮಗ, ಸಹೋದರಿಯರು, ಸಹೋದರರು, ತಂದೆ ಇವರೆದುರಿಗೆಲ್ಲಾ ಮಾತನಾಡುವರು. ತಂದೆ ಮಗಳನ್ನು ಆಕೆ ಮುಂದೆ ಮದುವೆಯಾಗಲಿರುವ ವರನ ವಿಷಯವಾಗಿ ಹಲವು ಪ್ರಶ್ನೆಗಳನ್ನು ಕೇಳಿ ಹಾಸ್ಯಮಾಡುವನು. ಅಂತಹ ಸಮಯದಲ್ಲಿ ಫ್ರೆಂಚ್​ ಕನ್ಯೆ ಲಜ್ಜೆಯಿಂದ ತಲೆಬಾಗುವಳು. ಆಂಗ್ಲೇಯಳು ನಾಚುವಳು. ಅಮೆರಿಕಾದವಳು ಮಾತಿಗೆ ಸರಿಯಾದ ಉತ್ತರವನ್ನು ಕೊಡುವಳು. ಅವರಲ್ಲಿ ಚುಂಬನ ಆಲಿಂಗನ ಗಳಲ್ಲಿ ದೋಷವೇನೂ ಇಲ್ಲ. ಈ ವಿಷಯವನ್ನು ಕುರಿತು ಎಲ್ಲರೆದುರಿಗೆ ಮಾತನಾಡ ಬಹುದು. ಆದರೆ ನಮ್ಮಲ್ಲಿ ಪ್ರೇಮಕ್ಕೆ ಸಂಬಂಧಪಟ್ಟ ಹೆಸರನ್ನು ಕೂಡ ಹಿರಿಯ ರೆದುರಿಗೆ ಹೇಳಕೂಡದು.

ಪಾಶ್ಚಾತ್ಯರು ಈಗ ಶ‍್ರೀಮಂತರಾಗಿರುವರು. ಒಬ್ಬನ ಪೋಷಾಕು ಅತಿ ಶುಭ್ರವಾಗಿ ಅತಿ ನೂತನ ಫ್ಯಾಶನ್​ ಇಲ್ಲದೇ ಇದ್ದರೆ ಅವನನ್ನು ಸಭ್ಯ ಮನುಷ್ಯ ನೆಂದು ಪರಿಗಣಿಸುವುದಿಲ್ಲ. ಸಮಾಜದ ಎಲ್ಲರೊಂದಿಗೆ ಬೆರೆಯಲಾರ. ಸಭ್ಯ ಮನುಷ್ಯ ಒಂದು ದಿನಕ್ಕೆ ತನ್ನ ಕಾಲರ್​ ಮತ್ತು ಷರ್ಟನ್ನು ಎರಡು ಮೂರು ವೇಳೆ ಯಾದರೂ ಬದಲಾಯಿಸಬೇಕು. ಬಡವರು ಇದನ್ನು ಮಾಡಲಾರರು. ಹೊರಗೆ ಕಾಣುವ ಬಟ್ಟೆಯಲ್ಲಿ ಒಂದು ಚುಕ್ಕೆಯಾಗಲೀ ಗೆರ್ಧೆಯ್ಧಾಗ್ಧಲೀ ಇರಕೂಡದು. ಎಷ್ಟೇ ಬಿಸಿಲಾಗಿರಲಿ, ಹೊರಗೆ ಹೋಗುವಾಗ ಕೈಚೀಲವಿರಬೇಕು. ಇಲ್ಲದೆ ಇದ್ದರೆ ಕೈಕೊಳೆ ಯಾದೀತೆಂಬ ಭಯ. ಹೆಂಗಸರಿಗೆ ಕೊಳೆ ಕೈಯಿಂದ ಕರಲಾಘವ ಕೊಡುವುದು ಅಸಭ್ಯತೆ. ಸಭ್ಯ ಸಮಾಜದಲ್ಲಿರುವಾಗ ಉಗುಳುವುದು, ಬಾಯಿ ಮುಕ್ಕಳಿಸುವುದು, ಹಲ್ಲಿನ ಮಧ್ಯೆ ಸಿಕ್ಕಿಕೊಂಡಿರುವ ಕಸವನ್ನು ತೆಗೆಯುವುದು, ಇಂತಹ ಕೆಲಸವನ್ನೇ ನಾದರೂ ಮಾಡಿದರೆ ಅವನನ್ನು ಚಂಡಾಲನೆಂದು ಇತರರು ನೋಡಿ ತಿರಸ್ಕರಿಸುವರು.

ಶಕ್ತಿ ಪೂಜೆಯೆ ಪಾಶ್ಚಾತ್ಯ ಧರ್ಮ. ವಾಮಾಚಾರದವರು ಸ್ತ್ರೀಯರನ್ನು ಪೂಜಿಸುವಂತೆ ಇರುವುದು ಅದು. ತಾಂತ್ರಿಕರು ಹೇಳುವಂತೆ “ಎಡಗಡೆ ಹೆಂಗಸು, ಬಲಗಡೆ ಮಧ್ಯದಿಂದ ತುಂಬಿದ ಬಟ್ಟಲು, ಮುಂದೆ ಮಸಾಲೆ ಹಾಕಿ ಬೇಯಿಸಿದ ಬಿಸಿಬಿಸಿ ಮಾಂಸ-ತಾಂತ್ರಿಕಧರ್ಮ ಅತಿ ಗಹನವಾದುದು. ಯೋಗಿ ಗಳಿಗೆ ಅದನ್ನು ಅರ್ಥಮಾಡಿ ಕೊಳ್ಳುವುದು ಕಷ್ಟ!” ಈ ಶಕ್ತಿ ಪೂಜೆ ಸಾಧಾರಣ ವಾಗಿ ಬಹಿರಂಗವಾಗಿ ಇದೆ. ಇಲ್ಲಿ ಮಾತೃ ಭಾವವೂ ಇರುವುದು. ಯೂರೋಪಿನಲ್ಲಿ ಪ್ರಾಟೆಸ್ಟಂಟರ ಪ್ರಾಬಲ್ಯ ಕಡಿಮೆ. ಕ್ಯಾಥೋಲಿಕರ ಪ್ರಾಬಲ್ಯವೇ ಹೆಚ್ಚು. ಈ ಧರ್ಮದಲ್ಲಿ ಜೆಹೋವ, ಜೀಸಸ್​, ಅಥವಾ ತ್ರಿಮೂರ್ತಿಗಳು ಗೌಣ. ಅಲ್ಲಿ ಪೂಜೆ ತಾಯಿಗೆ. ಬಾಲ ಏಸುವನ್ನು ಅವಳು ಎತ್ತಿಕೊಂಡಿರುವಳು. ಲಕ್ಷ ಸ್ಥಳದಿಂದ - ಅರಮನೆ, ಗುಡಿಸಲು, ಚರ್ಚು ಎಲ್ಲ ಕಡೆಯಿಂದಲೂ ಹಗಲು ರಾತ್ರಿ ‘ತಾಯಿ’ ಎಂಬ ಧ್ವನಿಯೊಂದು ಕೇಳಿ ಬರುತ್ತದೆ. ಚಕ್ರವರ್ತಿ ‘ಹೇ ತಾಯಿ’ ಎನ್ನುವನು, ಸೇನಾಪತಿ ‘ಹೇ ತಾಯಿ’ ಎನ್ನುವನು, ಧ್ವಜವನ್ನು ಹಿಡಿದ ಯೋಧ ‘ಹೇ ತಾಯಿ’ ಎನ್ನುವನು. ಸಾಗರದ ಮಧ್ಯದಲ್ಲಿರುವ ನಾವಿಕ ‘ಹೇ ತಾಯಿ’ ಎನ್ನುವನು. ಚಿಂದ್ಧಿಯ್ಧನ್ನ್ಧುಟ್ಟ ಬೆಸ್ತ ‘ಹೇ ತಾಯಿ’ ಎ್ಧನ್ನ್ಧುವ್ಧರು. ದಾರ್ಧಿಯ್ಧಲ್ಲಿ ನಡ್ಧ್ಧೆವ ಭಿಕ್ಷ್ಧುಕ ‘ಹೇ ತಾಯಿ’ ಎನ್ನ್ಧುವ್ಧನು. ಕೋಟಿ ಕೊರಲುಗಳ ಧನ್ಯ ಮೇರಿ, ಧನ್ಯ ಮೇರಿ ಎಂಬ ನಿನಾದ ದೇಶದಲ್ಲೆಲ್ಲಾ ತುಂಬಿ ತುಳುಕಾಡುತ್ತಿದೆ.

ನಂತರವೇ ಸ್ತ್ರೀ ಪೂಜೆ. ಈ ಶಕ್ತಿ ಪೂಜೆ ಕಾಮವಾಸನೆಯಿಂದ ಕೂಡಿಲ್ಲ. ಇದು ಕುಮಾರಿ ಮತ್ತು ಮುತ್ತೈದೆಯರ ಪೂಜೆ. ಕಾಶಿ, ಕಾಳಿಘಾಟ್​ ಮತ್ತು ಇತರ ಪವಿತ್ರ ಸ್ಥಳಗಳಲ್ಲಿ ಇದನ್ನು ಮಾಡುತ್ತಾರೆ. ಇದು ಕಾಲ್ಪನಿಕವಲ್ಲ, ವಾಸ್ತವಿಕ ಪೂಜೆ. ನಮ್ಮ ಶಕ್ತಿಪೂಜೆಯು ನಡೆಯುವುದು ತೀರ್ಥಸ್ಥಳಗಳಲ್ಲಿ ಮತ್ತು ಕೆಲವು ಸಂದರ್ಭಗಳಲ್ಲಿ ಮಾತ್ರ. ಪಶ್ಚಿಮದಲ್ಲಿ ಶಕ್ತಿ ಪೂಜೆ ಎಲ್ಲಾ ಕಾಲದಲ್ಲಿ, ಎಲ್ಲಾ ದೇಶದಲ್ಲೂ ವರ್ಷದ ಹನ್ನೆರಡು ತಿಂಗಳೂ ನಡೆಯುವುದು. ಮೊದಲು ಸ್ತ್ರೀಗೆ ಉನ್ನತ ಸ್ಥಾನವನ್ನು ಕೊಡುವರು. ಅವಳ ಬಟ್ಟೆ, ಆಹಾರ, ಬಯಕೆ, ಸುಖ ಇತ್ಯಾದಿ ಗಳ ಕಡೆ ಹೆಚ್ಚು ಗಮನ ಕೊಡುವರು. ಎಲ್ಲಾ ಬಗೆಯ ಪ್ರಥಮ ಗೌರವವೂ ಅವಳಿಗೆ ಸಲ್ಲುವುದು. ರೂಪವತಿ ಯುವತಿಯರಿಗೆ ಮಾತ್ರ ಇದು ಅನ್ವಯಿಸುವುದಿಲ್ಲ. ಈ ಶಕ್ತಿಪೂಜೆ ಪ್ರತಿಯೊಬ್ಬ ಸ್ತ್ರೀಗೂ ಅವಳು ಪರಿಚಿತಳಾಗಿರಲಿ, ಅಪರಿಚಿತಳಾಗಿರಲಿ ಸಲ್ಲುವುದು. ಮಹಮ್ಮದೀಯ ಮೂರ್​ ಜನಾಂಗದವರು ಸ್ಪೆಯಿನ್​ ದೇಶವನ್ನು ಗೆದ್ದು ಎಂಟು ಶತಮಾನಗಳು ಅದನ್ನು ಆಳಿದಾಗ ಯೂರೋಪಿನಲ್ಲಿ ಈ ಶಕ್ತಿಪೂಜೆಯನ್ನು ಪ್ರಾರಂಭಿಸಿದರು. ಇವರೇ ಮೊದಲು ಯೂರೋಪಿನಲ್ಲಿ ಪಾಶ್ಚಾತ್ಯ ನಾಗರಿಕತೆಯ ಮತ್ತು ಶಕ್ತಿಪೂಜೆಯ ಬೀಜವನ್ನು ಬಿತ್ತಿದವರು. ಕೆಲವು ಕಾಲದ ನಂತರ ಮೂರರು ಶಕ್ತಿಪೂಜೆಯನ್ನು ಮರೆತರು. ಇದರಿಂದಲೇ ಅವರು ಶಕ್ತಿ ಸಂಪತ್ತನ್ನು ಕಳೆದುಕೊಂಡು ಅನಾಗರೀಕರಾಗಿ ಆಫ್ರಿಕಾಮೂಲೆಯಲ್ಲಿ ಅನಾಮಧೇಯರಾಗಿರುವರು. ಇವರ ನಂತರ ಯೂರೋಪು ಇವರು ಶಕ್ತಿ ಮತ್ತು ನಾಗರೀಕತೆಗಳನ್ನು ತೆಗೆದುಕೊಂಡಿತು. “ತಾಯಿ” ಮೂರರನ್ನು ತೊರೆದು ಕ್ರಸ್ತರ ಮನೆಯಲ್ಲಿ ರಾರಾಜಿಸುತ್ತಿರುವಳು


\section{ಫ್ರಾನ್ಸ್ - ಪ್ಯಾರಿಸ್}

ಯೂರೋಪ್​ ಅಂದರೇನು? ಏಷ್ಯಾ, ಆಫ್ರಿಕಾ, ಅಮೆರಿಕಾ ದೇಶದ ಕಪ್ಪು, ಕಂದು, ಹಳದಿ ಬಣ್ಣದ ಜನ ಯೂರೋಪಿನ ಪಾದದಡಿಯಲ್ಲಿ ಏಕೆ ಮಣಿಯು ವರು? ಕಲಿಯುಗದಲ್ಲಿ ಯೂರೋಪ್​ ನಿವಾಸಿಗಳು ಏಕಮಾತ್ರ ಶಾಸನ ಕರ್ತರಾಗಿರುವುದು ಏಕೆ? ಈ ಯೂರೋಪನ್ನು ತಿಳಿದುಕೊಳ್ಳಬೇಕಾದರೆ ಫ್ರಾನ್ಸಿನ ಮೂಲಕ ತಿಳಿದುಕೊಳ್ಳಬೇಕು. ಪಶ್ಚಿಮದಲ್ಲಿರುವ ಎಲ್ಲ ಉತ್ಕೃಷ್ಟತೆಗೂ ಮೂಲ ಇದು. ಪೃಥ್ವಿಯ ಆಧಿಪತ್ಯವಿರುವುದು ಯೂರೋಪಿನ ಕೈಯಲ್ಲಿ. ಯೂರೋಪಿನ ಮಹಾಕೇಂದ್ರ ಪ್ಯಾರಿಸ್​. ಪ್ಯಾರಿಸ್​ನಲ್ಲಿ ಪಾಶ್ಚಾತ್ಯ ಸಭ್ಯತೆ, ರೀತಿ ನೀತಿ, ಬೆಳಕು, ಅಂಧಕಾರ, ಒಳ್ಳೆಯದು, ಕೆಟ್ಟದ್ದು ಎಲ್ಲವೂ ಪಕ್ವವಾಗಿ ಪ್ರಬುದ್ಧ ಸ್ಥಿತಿಗೆ ಬರುತ್ತವೆ.

ಈ ಪ್ಯಾರಿಸ್​ ನಗರ ಒಂದು ಮಹಾ ಸಮುದ್ರದಂತೆ ಇದೆ. ಇದರಲ್ಲಿ ಎಷ್ಟೋ ಅನರ್ಘ್ಯ ವಜ್ರ ವೈಢೂರ್ಯ ರತ್ನಗಳೂ ಇವೆ. ನಕ್ರ ತಿಮಿಂಗಿಲವೆಂಬ ಮಹಾ ಭಯಾನಕ ಕಡಲಜಂತುಗಳೂ ಇವೆ. ಯೂರೋಪಿನ ಕರ್ಮಕ್ಷೇತ್ರ ಫ್ರಾನ್ಸ್​. ಚೈನಾ ದೇಶದ ಸ್ವಲ್ಪ ಭಾಗವನ್ನು ಬಿಟ್ಟರೆ ಇಂತಹ ಸುಂದರ ದೇಶ ಪೃಥ್ವಿಯಲ್ಲಿ ಮತ್ತೆ ಎಲ್ಲಿಯೂ ಇಲ್ಲ. ಬಹಳ ಉಷ್ಣವೂ ಇಲ್ಲ, ಶೀತವೂ ಇಲ್ಲ. ಅಧಿಕ ನೀರೂ ಇಲ್ಲ, ಕಡಿಮೆ ನೀರೂ ಇಲ್ಲ. ನಿರ್ಮಲ ಆಕಾಶ, ಆಹ್ಲಾದಕರವಾದ ಸೂರ್ಯ, ಹುಲ್ಲುಗಾವಲು, ಸಣ್ಣ ಸಣ್ಣ ಬೆಟ್ಟಗಳು, ಎಲ್ಮ್​ ಮತ್ತು ಓಕ್​ ಮರಗಳು, ಸಣ್ಣ ಸಣ್ಣ ನದಿ ಮತ್ತು ಝರಿ, ಇಂತಹ ಸುಂದರ ದೃಶ್ಯ ಪೃಥ್ವಿಯಲ್ಲಿ ಬೇರೆ ಎಲ್ಲಿರುವುದು? ಮೋಹಕವಾದ ನೀರು ಸುಂದರಸ್ಥಳ, ಉನ್ಮತ್ತರಂತೆ ಮಾಡುವ ವಾಯು, ಸುಂದರ ಆಕಾಶ ಎಲ್ಲಿ ದೊರಕುವುದು? ಪ್ರಕೃತಿ ಸುಂದರವಾಗಿದೆ; ಜನರು ಸೌಂದರ್ಯ ಪ್ರಿಯರು. ಬಾಲಕರು, ವೃದ್ಧರು, ಸ್ತ್ರೀಪುರುಷರು, ಧನಿಕದರಿದ್ರರು ತಮ್ಮ ಮನೆ, ಬಾಗಿಲು, ಕೋಣೆ, ರಸ್ತೆ, ಮೈದಾನ, ತೋಟ ಎಲ್ಲವನ್ನೂ ಸುಂದರವಾಗಿಡುವರು. ದೇಶ ಒಂದು ಮನೋರಂಜಕ ಚಿತ್ರದಂತೆ ಕಾಣುವುದು. ಜಪಾನ್​ ದೇಶವನ್ನು ಬಿಟ್ಟರೆ ಇಂಥಾ ಸೌಂದರ್ಯೋಪಾಸನೆ ಪ್ರಪಂಚದಲ್ಲಿ ಬೇರೆ ಯಾವ ಕಡೆಯೂ ಕಾಣುವುದಿಲ್ಲ. ಅಮರಾವತಿಯಂತೆ ಕಾಣುವುವು, ಇಲ್ಲಿನ ಅಂತಸ್ತಿನ ಮನೆಗಳು. ನಂದನವನದಂತೆ ಉದ್ಯಾನ, ಉಪವನ. ರೈತರ ಹೊಲಗದ್ದೆ, ಮನೆಗಳನ್ನೂ ಕೂಡ ನೋಡುವುದಕ್ಕೆ ಮನೋಹರವಾಗಿ ಕಾಣುವಂತೆ ಮಾಡಿರುವರು. ಈ ಪ್ರಯತ್ನ ದಲ್ಲಿ ಅವರು ಜಯಶೀಲರಾಗಿರುವರು.

ಪ್ರಾಚೀನಕಾಲದಿಂದಲೂ ಫ್ರಾನ್ಸ್​ ಗೌಲ್​, ರೋಮನ್​, ಫ್ರಾಂಕ್​ ಜನರ ಸಂಘರ್ಷಣೆಯ ಸ್ಥಾನವಾಗಿತ್ತು. ರೋಮನ್​ ಸಾಮ್ರಾಜ್ಯವು ನಾಶವಾದ ಮೇಲೆ ಯೂರೋಪಿನಲ್ಲಿ ಫ್ರಾಂಕರು ಪ್ರಖ್ಯಾತರಾದರು. ಇವರ ಚಕ್ರವರ್ತಿ ಚಾರ್ಲ್​ ಮೆಯ್ನ್​ ಬಲಪ್ರಯೋಗದಿಂದ ಕ್ತೈಸ್ತ ಧರ್ಮವನ್ನು ಯೂರೋಪಿನಲ್ಲಿ ಪ್ರಚಾರಕ್ಕೆ ತಂದನು. ಈ ಫ್ರಾಂಕ್​ ಜನರ ಮೂಲಕ ಏಷ್ಯಾದಲ್ಲಿ ಯೂರೋಪ್​ ಪ್ರಖ್ಯಾತಿಗೆ ಬಂದಿತು. ಆದಕಾರಣ ಈಗಲೂ ನಾವು ಯೂರೋಪಿಯನ್ನರನ್ನು ಫ್ರಾಂಕಿ, ಫಿರಂಗಿ, ಪ್ಲಾಂಕಿ ಅಥವಾ ಫೆಲಿಂಗ್​ ಎಂದು ಕರೆಯುವುದು.

ಸಭ್ಯತೆಗೆ ನಿವಾಸಸ್ಥಾನವಾದ ಗ್ರೀಕರು ಸೋತು ಅನಾಮಧೇಯರಾದರು. ರೋಮ್​ ಚಕ್ರಾಧಿಪತ್ಯವು ಬರ್ಬರ ಆಕ್ರಮಣ ತರಂಗದೆದುರು ನುಚ್ಚು ನೂರಾಯಿತು. ಯೂರೋಪಿನ ಕಾಂತಿ ಕುಂದಿತು. ಈ ಸಮಯದಲ್ಲಿ ಏಷ್ಯಾದಲ್ಲಿ ಮತ್ತೊಂದು ಬರ್ಬರ ಜನಾಂಗ ಮೇಲೆದ್ದಿತು. ಅದೇ ಅರಬ್ಬೀ ಜನಾಂಗ. ಈ ಅರಬ್ಬೀ ಜನಾಂಗದ ತರಂಗ ಬಹಳ ವೇಗದಿಂದ ಪೃಥ್ವಿಯನ್ನು ಆಕ್ರಮಿಸಲು ಪ್ರಾರಂಭಿಸಿತು. ಮಹಾ ಬಲ ಶಾಲಿಗಳಾದ ಪಾರಸೀಯರು ಇವರೆದುರಿಗೆ ಸೋತು ಮುಸಲ್ಮಾನ ಧರ್ಮವನ್ನು ಸ್ವೀಕರಿಸಬೇಕಾಯಿತು. ಪರಿಣಾಮವಾಗಿ ಇಸ್ಲಾಂ ಧರ್ಮವು ಹೊಸರೂಪವನ್ನು ಪಡೆಯಿತು. ಅರಬ್ಬೀ ಧರ್ಮವು ಪಾರಸೀ ನಾಗರಕತೆಯೊಂದಿಗೆ ಬೆರೆಯಿತು.

ಅರಬರ ಕತ್ತಿಯ ಸಹಾಯದಿಂದ ಪಾರಸೀ ನಾಗರಿಕತೆಯು ಕ್ರಮೇಣ ಸುತ್ತಲೂ ಹಬ್ಬತೊಡಗಿತು. ಪಾರಸೀ ನಾಗರೀಕತೆ ಗ್ರೀಕ್​ ಮತ್ತು ಹಿಂದೂ ದೇಶ ದಿಂದ ಬಂದುದು. ಪೂರ್ವ ಪಶ್ಚಿಮಗಳೆರಡು ಕಡೆಯಿಂದಲೂ ಬಹಳ ವೇಗದಿಂದ ಮುಸಲ್ಮಾನ ತರಂಗ ಯೂರೋಪನ್ನು ಆಕ್ರಮಿಸತೊಡಗಿತು. ಅಂಧಕಾರದಿಂದ ತುಂಬಿದ ಯೂರೋಪಿನಲ್ಲಿ ಜ್ಞಾನ ಕಿರಣಗಳು ಹಬ್ಬಲು ಮೊದಲಾಯಿತು. ಗ್ರೀಕರ ಜ್ಞಾನ, ಶಿಲ್ಪಾದಿ ಕಲೆಗಳು, ಬರ್ಬರಾಕ್ರಾಂತ ಇಟಲಿಗೆ ಪ್ರವೇಶಿಸಿದುವು. ಅದರ ರಾಜಧಾನಿಯಾದ ರೋಮ್​ನಗರದ ಮೃತ ಶರೀರದಲ್ಲಿ ಪ್ರಾಣ ಸ್ಪಂದನ ವಾಯಿತು. ಈ ಸ್ಪಂದನ ಫ್ಲಾರೆನ್ಸ್​ ನಗರದಲ್ಲಿ ಪ್ರಬಲ ರೂಪವನ್ನು ತಾಳಿತು. ಪ್ರಾಚೀನ ಇಟಲಿಯಲ್ಲಿ ನವಜೀವನ ಆರಂಭವಾಯಿತು. ಇದನ್ನೇ “ರಿನೈಸಾನ್ಸ್​” ನವ ಜೀವನ ಎನ್ನುವರು. ಈ ನವಜನ್ಮ ಇಟಲಿಗೆ ಪುನರ್ಜನ್ಮವಷ್ಟೆ. ಯೂರೋಪಿನ ಇತರ ಪ್ರಾಂತ್ಯಕ್ಕೆ ಇದು ಪ್ರಥಮ ಜನ್ಮ ಮಾತ್ರವಾಗಿತ್ತು. ಯೂರೋಪ್​ ಕ್ರಿ.ಶ. ಹದಿನಾರನೆಯ ಶತಮಾನದಲ್ಲಿ ಉದಿಸಿತು. ಆಗ ಮೊಗಲ್​ ಚಕ್ರವರ್ತಿಗಳಾದ ಅಕ್ಬರ್​, ಜಹಾಂಗೀರ್​, ಶಾಜಹಾನ್​ ತಮ್ಮ ಚಕ್ರಾಧಿಪತ್ಯವನ್ನು ಭರತಖಂಡದಲ್ಲಿ ಸುದೃಢವಾಗಿ ಸ್ಥಾಪಿಸಿದ್ದರು.

ಇಟಲಿ ಬಹಳ ಪ್ರಾಚೀನ ಜನಾಂಗ. ನವಜೀವನದ ಕರೆಗೆ ಓಗೊಟ್ಟು ಒಮ್ಮೆ ಎದ್ದು ಪುನಃ ದೀರ್ಘನಿದ್ರೆಯಲ್ಲಿ ತಲ್ಲೀನವಾಯಿತು. ಆ ಸಮಯದಲ್ಲಿ ಭರತ ಖಂಡವೂ ಕಾರಾಣಾಂತರದಿಂದ ಸ್ವಲ್ಪ ಜಾಗ್ರತವಾಯಿತು. ಅಕ್ಬರನಿಂದ ಹಿಡಿದು ಮೂರು ತಲೆಯವರೆಗೆ, ವಿದ್ಯೆ, ಬುದ್ಧಿ, ಶಿಲ್ಪಾದಿಕಲೆಗೆ ಹೆಚ್ಚು ಪ್ರೋತ್ಸಾಹ ದೊರಕು ತ್ತಿತ್ತು. ಭರತಖಂಡವೂ ಒಂದು ಪ್ರಾಚೀನ ಜನಾಂಗ. ಹಲವಾರು ಕಾರಣಗಳಿಂದ ಇಟಲಿಯಂತೆ ಇದೂ ತಮಸ್ಸಿನಕಡೆ ತೆರಳಿತು.

ಯೂರೋಪಿನಲ್ಲಿ ಇಟಲಿಯ ನವಚೇತನದ ಅಲೆಯು ಬಲವಾದ ನವ ಫ್ರಾಂಕ್​ ಜನಾಂಗವನ್ನು ವ್ಯಾಪಿಸಿತು. ನಾಲ್ಕು ಕಡೆಯಿಂದಲೂ ಬರುತ್ತಿದ್ದ ನಾಗರಿಕತೆಯ ಪ್ರವಾಹ ಫ್ಲಾರೆನ್ಸಿನಲ್ಲಿ ಸಂಧಿಸಿ ಹೊಸ ರೂಪವನ್ನು ಧಾರಣೆ ಮಾಡಿತು. ಈ ನವ ಜಾಗ್ರತ ಮಹಾಶಕ್ತಿಯನ್ನು ಅರಗಿಸಿಕೊಳ್ಳಲು ಇಟಲಿಗೆ ಸಾಧ್ಯವಾಗಲಿಲ್ಲ. ಭಾರತ ವರ್ಷದಲ್ಲಿ ಆದಂತೆ ಈ ಜಾಗ್ರತಿ ಇಟಲಿಯಲ್ಲೇ ಪರಿಸಮಾಪ್ತವಾಗಬೇಕಾಗಿತ್ತು. ಯೂರೋಪಿನ ಭಾಗ್ಯವಶದಿಂದ ನವೀನ ಫ್ರಾಂಕ್​ ಜನಾಂಗ ಇದನ್ನು ಆದರ ಪೂರ್ವಕವಾಗಿ ಸ್ವೀಕರಿಸಿತು. ನವೀನ ರಕ್ತಸಂಪನ್ನರಾದ ಫ್ರಾಂಕರು ನವೀನ ಜಾತಿ ತರಂಗದ ಮೇಲೆ ಬಹಳ ಸಾಹಸದಿಂದ ತಮ್ಮ ರಾಷ್ಟ್ರೀಯ ನೌಕೆಯನ್ನು ತೇಲಿ ಬಿಟ್ಟರು. ಕ್ರಮೇಣ ಆ ವೇಗವು ಪ್ರಬಲವಾಗುತ್ತಾ ಬಂದಿತು. ಈ ಪ್ರವಾಹದ ಒಂದು ಧಾರೆ ಸಹಸ್ರಮುಖವಾಗಿ ಹರಿಯತೊಡಗಿತು. ಯೂರೋಪಿನ ಇತರ ಜನಾಂಗಗಳೂ ಅದನ್ನು ನಾಲೆಯ ಮೂಲಕ ತಮ್ಮ ತಮ್ಮ ದೇಶಕ್ಕೆ ಹರಿಸಿಕೊಂಡು, ಅದಕ್ಕೆ ತಮ್ಮ ಜೀವನ “ಶಕ್ತಿ” ಸಾಹಸವನ್ನು ಮಿಶ್ರ ಮಾಡಿ, ಅದರ ವೇಗ ಮತ್ತು ವಿಸ್ತಾರವನ್ನು ಇನ್ನೂ ಅಧಿಕಗೊಳಸಿದರು. ಈ ತರಂಗವೇ ಕಾಲಕ್ರಮೇಣ ಭಾರತ ವರ್ಷಕ್ಕೂ ತಲುಪಿತು. ಈ ತರಂಗ ಲಹರಿ ಜಪಾನ್​ ಸಮುದ್ರತೀರವನ್ನು ಮುಟ್ಟಿತು. ಜಪಾನರು ಈ ಜಲಪಾನ ಮಾಡಿ ಹೊಸ ಜೀವ ತಾಳುತ್ತಿರುವರು. ಏಷ್ಯಾದಲ್ಲಿ ಜಪಾನ್​ ನನೀನ ಜನಾಂಗ.

ಪ್ಯಾರಿಸ್​ ನಗರವು ಯೂರೋಪ್​ ನಾಗರಿಕತೆಗೆ ಉಗಮಸ್ಥಾನ, ಗಂಗೆಗೆ ಗೋಮುಖಿ ಇದ್ದಂತೆ. ಈ ವಿರಾಟ್​ ನಗರಿ ಭೂಸ್ವರ್ಗ, ಅಮರಾವತಿ, ಸದಾನಂದ ನಗರಿ. ಪ್ಯಾರಿಸ್ಸಿನ ಭೋಗ, ವಿಲಾಸ, ಆನಂದ ಲಂಡನ್ನಿನಲ್ಲಿ ಇಲ್ಲ, ಬರ್ಲಿನಿನಲ್ಲಿ ಇಲ್ಲ, ಯೂರೋಪಿನ ಮತ್ತಾವ ನಗರಿಯಲ್ಲಿಯೂ ಇಲ್ಲ. ಲಂಡನ್​ ಮತ್ತು ನ್ಯೂ ಯಾರ್ಕ್​ಗಳಲ್ಲಿ ಧನವಿದೆ. ಬರ್ಲಿನಿನಲ್ಲಿ ವಿದ್ಯೆ ಬುದ್ಧಿ ಯಥೇಷ್ಟವಿದೆ. ಆದರೆ ಬೇರೆಲ್ಲಿಯೂ ಫ್ರೆಂಚ್​ ನೆಲದ ಮಣ್ಣಿನ ಗುಣವಿಲ್ಲ. ಎಲ್ಲಕ್ಕಿಂತ ಹೆಚ್ಚಾಗಿ ಫ್ರೆಂಚರ ಅಸಾಧಾರಣ ಪ್ರತಿಭೆ, ಮನೋಧರ್ಮಗಳಿಲ್ಲ. ಉಳಿದ ಕಡೆಗಳಲ್ಲಿ ಐಶ್ವರ್ಯ, ವಿದ್ಯೆ, ಬುದ್ಧಿ ಇರಬಹುದು, ಪ್ರಾಕೃತಿಕ ಸೌಂದರ್ಯ ಇರಬಹುದು - ಆದರೆ ಮುಖ್ಯವಾಗಿ ಜನಶಕ್ತಿ ಎಲ್ಲಿರುವುದು? ಪ್ರಾಚೀನ ಗ್ರೀಕರು ಪುನರ್ಜನ್ಮ ತಾಳಿ ಫ್ರೆಂಚರಾಗಿರುವರು ಎನ್ನಬಹುದು. ಯಾವಾಗಲೂ ಆನಂದ ಉತ್ಸಾಹದಿಂದ ಕೂಡಿರುವರು ಇವರು. ಅತಿ ಲಘುವಾಗಿರುವರು. ಎಲ್ಲಾ ಕಾರ್ಯಕ್ಕೂ ಉತ್ಸಾಹ ತೋರುವರು. ಏನಾದರೂ ಸ್ವಲ್ಪ ತೊಂದರೆ ಬಂದರೆ ನಿರುತ್ಸಾಹಿ ಳಾಗುವರು. ಆದರೆ ಈ ನಿರುತ್ಸಾಹ ಅವರ ಮುಖದ ಮೇಲೆ ಬಹಳ ಕಾಲವಿರುವುದಿಲ್ಲ. ಪುನಃ ನವೀನ ಉತ್ಸಾಹ, ವಿಶ್ವಾಸ ಅವರ ಮೊಗದಲ್ಲಿ ಅರಳುವುದು.

ಪ್ಯಾರಿಸ್​ ವಿಶ್ವವಿದ್ಯಾನಿಲಯವು ಯೂರೋಪಿನ ಆದರ್ಶ ವಿಶ್ವವಿದ್ಯಾ ನಿಲಯ. ಪ್ರಪಂಚದಲ್ಲಿರುವ ವೈಜ್ಞಾನಿಕ ಸಂಸ್ಥೆಗಳು ಪ್ಯಾರಿಸ್ಸಿನ ಸಂಸ್ಥೆಯ ಒಂದು ನಕಲು. ಫ್ರಾನ್ಸ್​ ಇತರರಿಗೆ ವಸಾಹತು ಸಾಮ್ರಾಜ್ಯ ಸ್ಥಾಪನೆಯ ಶಿಕ್ಷಾಗುರು. ಎಲ್ಲಾ ಭಾಷೆಗಳಲ್ಲೂ ಇರುವ ಯುದ್ಧಕ್ಕೆ ಸಂಬಂದಪಟ್ಟ ಶಬ್ದಗಳು ಹೆಚ್ಚಾಗಿ ಫ್ರೆಂಚ್​ ಭಾಷೆಯವು. ಯೂರೋಪಿನ ಭಾಷೆಗಳೆಲ್ಲ ಫ್ರೆಂಚ್​ ಭಾಷೆಯಿಂದ ಶೈಲಿಯನ್ನು ಅನುಸರಿಸಿವೆ. ಪ್ಯಾರಿಸ್​ ನಗರ ದರ್ಶನ, ವಿಜ್ಞಾನ, ಶಿಲ್ಪಾದಿ ಕಲೆಗಳ ತವರೂರು. ಎಲ್ಲರೂ ಇವರನ್ನೇ ಅನುಕರಿಸುವರು.

ಪ್ಯಾರಿಸ್​ನಲ್ಲಿರುವವರು ನಗರದಲ್ಲಿರುವವರಂತೆ, ಉಳಿದವರು ಹಳ್ಳಿಗಾಡಿ ನವರಂತೆ. ಯಾವುದನ್ನು ಪ್ಯಾರಿಸ್ಸಿನವರು ಈಗ ಮಾಡುವರೋ ಅದನ್ನೇ ಇನ್ನು ಐವತ್ತು ಅಥವಾ ಇಪ್ಪತ್ತೈದು ವರ್ಷಗಳಲ್ಲಿ ಜರ್ಮನಿ, ಲಂಡನ್ನಿನವರು ಅನು ಸರಿಸುವರು. ಅದು ವಿದ್ಯೆಗೆ ಸಂಬಂಧಿಸಿರಬಹುದು, ಶಿಲ್ಪಶಾಸ್ತ್ರಕ್ಕೆ ಸಂಬಂಧಿಸಿರ ಬಹುದು, ಅಥವಾ ಸಾಮಾಜಿಕ ರೀತಿ ನೀತಿಗೆ ಸಂಬ್ಂಧಿಸಿರಬಹುದು. ಫ್ರಾನ್ಸ್​ ನಾಗರಿಕತೆ ಸ್ಕಾಟ್​ಲೆಂಡನ್ನು ಮುಟ್ಟಿತು. ಅಲ್ಲಿಯ ರಾಜ ಇಂಗ್ಲೆಂಡಿಗೂ ರಾಜ ನಾದಾಗ ಅದು ಜಾಗ್ರತವಾಯಿತು. ಸ್ಕಾಟ್ಲೆಂಡಿನ ಸ್ಟೂಯರ್ಟ್ ವಂಶಜರು ಇಂಗ್ಲೆಂಡನ್ನು ಆಳುತ್ತಿದ್ದಾಗ, ರಾಯಲ್​ ಸೊಸೈಟಿ ಮುಂತಾದವು ಸ್ಥಾಪಿತವಾದುವು.

ಫ್ರಾನ್ಸ್​ ದೇಶವು ಸ್ವಾತಂತ್ರ್ಯದ ತವರು. ಈ ಪ್ಯಾರಿಸ್​ ಮಹಾನಗರಿಯಿಂದಲೇ ಪ್ರಜಾಶಕ್ತಿಯು ಯೂರೋಪಿನ ಬುನಾದಿಯನ್ನು ಅಲುಗಾಡಿಸಿತು.ಅದರಿಂದ ಯೂರೋಪಿನ ಸ್ವರೂಪವೇ ಬದಲಾಗಿ ಹೊಸ ಯೂರೋಪು ಅಸ್ತಿತ್ವಕ್ಕೆ ಬಂದಿತು. “ಸ್ವಾತಂತ್ರ್ಯ, ಸಮಾನತೆ, ಭ್ರಾತೃತ್ವ” ಈ ಶಬ್ದಗಳನ್ನು ಫ್ರಾನ್ಸಿನಲ್ಲಿ ಈಗಲೂ ಯಾರೂ ಕೇಳುತ್ತಿಲ್ಲ. ಫ್ರಾನ್ಸ್​ ಈಗ ಬೇರೆ ಭಾವವನ್ನು, ಬೇರೆ ಉದ್ದೇಶವನ್ನು ಅನುಸರಿಸುತ್ತಿದೆ. ಆದರೂ ಇತರ ದೇಶಗಳಲ್ಲಿ ಫ್ರಾನ್ಸಿನ ವಿಪ್ಲವ ಸಂದೇಶ ಕಾರ್ಯ ರೂಪದಲ್ಲಿರುವುದು.

ಇಂಗ್ಲೆಂಡಿನ ಒಬ್ಬ ಪ್ರಸಿದ್ಧ ವಿಜ್ಞಾನಿ ಒಂದು ದಿನ ನನಗೆ ಹೀಗೆ ಹೇಳಿದನು: “ಪ್ಯಾರಿಸ್​ ಪೃಥ್ವಿಯ ಕೇಂದ್ರ. ಯಾವ ದೇಶವು ಪ್ಯಾರಿಸ್ಸಿನೊಂದಿಗೆ ಅತಿ ನಿಕಟ ಸಂಬಂಧವನ್ನು ಕಲ್ಪಿಸಿಕೊಳ್ಳುವುದೊ ಅದು ಅದೇ ಪ್ರಮಾಣದಲ್ಲಿ ಪ್ರಗತಿ ಹೊಂದುವುದು.” ಈ ಮಾತು ಅತಿಶಯೋಕ್ತಿಯಾದರೂ ಇದರಲ್ಲಿ ಸ್ವಲ್ಪ ಸತ್ಯ ವಿದೆ. ಯಾವುದಾದರೊಂದು ನವೀನ ಭಾವವನ್ನು ಪ್ರಪಂಚದಲ್ಲಿ ಹರಡಬೇಕಾ ದರೂ ಅದರ ಪ್ರಸಾರಕ್ಕೆ ಕೇಂದ್ರ ಪ್ಯಾರಿಸ್​. ಪ್ಯಾರಿಸ್ಸಿನ ನಾಗರಿಕರ ಮೆಚ್ಚುಗೆಯನ್ನು ಪಡೆದ ಧ್ವನಿಯು ಯೂರೋಪಿನಲ್ಲೆಲ್ಲಾ ಪ್ರತಿಧ್ವನಿತವಾಗುವುದು. ಯಾವ ಸಂಗೀತ, ಶಿಲ್ಪ, ಚಿತ್ರ, ನೃತ್ಯ, ಕಲೆಗೆ ಪ್ಯಾರಿಸ್ಸಿನಲ್ಲಿ ಗೌರವ ಸಿಕ್ಕುವುದೊ ಅದಕ್ಕೆ ಇತರ ದೇಶ ಗಳಲ್ಲೂ ಅದು ಸಿಕ್ಕುವುದರಲ್ಲಿ ಸಂದೇಹವಿಲ್ಲ.

ನಮ್ಮ ದೇಶದಲ್ಲಿ ನಾವು ಪ್ಯಾರಿಸ್ಸಿನ ದೋಷಗಳ ಬಗ್ಗೆ ಮಾತ್ರ ಕೇಳುತ್ತೇವೆ. ಪ್ಯಾರಿಸ್​ ನಗರ ಭಯಂಕರ, ನರಕಕುಂಡ ಎನ್ನುವರು. ಕೆಲವು ಆಂಗ್ಲೇಯರು ಇದನ್ನೇ ಹೇಳುವರು. ಇತರ ದೇಶದ ಕೆಲವು ಶ‍್ರೀಮಂತರಿಗೆ ಪ್ಯಾರಿಸ್ಸಿನಲ್ಲಿ ವಿಷಯ ವಾಸನಾ ತೃಪ್ತಿಯಲ್ಲದೆ ಬೇರೇನೂ ಕಾಣುವುದಿಲ್ಲ. ಅವರಿಗೆ ಅದು ಅನೈತಿಕತೆಯ ಮತ್ತು ವಿಷಯ ತೃಪ್ತಿಯ ಕೇಂದ್ರ. ಲಂಡನ್​, ನ್ಯೂಯಾರ್ಕ್​, ಬರ್ಲಿನ್​, ವಿಯನ್ನಾ ಮುಂತಾದ ನಗರಗಳಲ್ಲೂ ಪರಿಸ್ಥತಿ ಹಾಗೇ ಇದೆ. ಇರುವ ವ್ಯತ್ಯಾಸ ಇದು: ಇತರ ದೇಶಗಳಲ್ಲಿ ಇಂದ್ರಿಯ ಸುಖವನ್ನು ಅನಾಗರಿಕರಂತೆ ಅನುಭವಿಸುವರು. ಆದರೆ ನಾಗರಿಕವಾದ ಪ್ಯಾರಿಸ್ಸಿನ ಕಸ ಕೂಡ ಸ್ವರ್ಣಧೂಳಿಯಿಂದ ಕೂಡಿದೆ. ಇತರ ದೇಶದ ಪೈಶಾಚಿಕ ಭೋಗದೊಂದಿಗೆ, ಪ್ಯಾರಿಸ್ಸಿನ ವಿಲಾಸಪ್ರಿಯತೆಯನ್ನು ತುಲನೆಮಾಡಿ ನೋಡಿದರೆ, ಕೆಸರಿನಲ್ಲಿ ಕ್ರೀಡಿಸುವ ಹಂದಿಯನ್ನು ಗರಿಗೆದರಿ ಕುಣಿ ಯುವ ನವಿಲಿನೊಂದಿಗೆ ಹೋಲಿಸಿದಂತೆ ಇರುವುದು.

ಭೋಗವಿಲಾಸದ ಇಚ್ಚೆ ಯಾವ ಜನಾಂಗದಲ್ಲಿ ಇಲ್ಲ? ಇದಲ್ಲದೇ ಇದ್ದರೆ ಶ‍್ರೀಮಂತರಾದವರೆಲ್ಲ ಪ್ಯಾರಿಸ್ಸಿಗೆ ಏತಕ್ಕೆ ಧಾವಿಸುವರು? ರಾಜರು ಚಕ್ರವರ್ತಿಗಳು, ತಮ್ಮ ಹೆಸರನ್ನು ಮರೆಮಾಚಿಕೊಂಡು ಪ್ಯಾರಿಸ್ಸಿನ ವಿಲಾಸ ಕುಂಡದಲ್ಲಿ ಸ್ನಾನಮಾಡಿ ಸಂತೋಷಿಸಲು ಏತಕ್ಕೆ ಹೋಗುವರು? ಇಚ್ಚೆ ಎಲ್ಲಾ ದೇಶದಲ್ಲೂ ಇದೆ.ಅದನ್ನು ತೃಪ್ತಿ ಪಡಿಸಿಕೊಳ್ಳುವುದಕ್ಕೆ ಎಲ್ಲರೂ ಕಾತರರಾಗಿರುವರು. ಇರುವ ಭೇದವಿಷ್ಟೆ. ಪ್ಯಾರಿಸ್ಸಿನವರು ಇದರಲ್ಲಿ ಸಿದ್ಧಹಸ್ತರು. ಹೇಗೆ ಬೋಗಿಸಬೇಕೆಂಬುದನ್ನು ಅರಿತಿರು ವವರು; ವಿಲಾಸಪ್ರಿಯತೆಯ ತುತ್ತ ತುದಿಯನ್ನು ಮುಟ್ಟಿರುವರು.

ಅತಿ ಹೀನ ನೃತ್ಯ, ತಮಾಷೆಯೆಲ್ಲ ಪರದೇಶೀಯರ ವಿಲಾಸಕ್ಕೆ ಮಾತ್ರ. ಫ್ರೆಂಚರು ತುಂಬಾ ಚತುರರು. ಕೆಲಸಕ್ಕೆ ಬಾರದ ಖರ್ಚು ಮಾಡುವುದಿಲ್ಲ. ಆ ಮಹಾವಿಲಾಸ, ಹೋಟೆಲು ಊಟ - ಇಂತಹ ಒಂದು ಭೋಜನ ಸಾಕು, ಒಬ್ಬನ ಸರ್ವನಾಶಕ್ಕೆ. -ಇದೆಲ್ಲಾ ಇರುವುದು ವಿದೇಶಿ ಧನಿಕರಿಗೆ. ಫ್ರೆಂಚರು ಸಭ್ಯರು. ಆದರೆ ಸಂಪನ್ನರು, ಗೌರವಸ್ಥರು, ಅತಿಥಿ ಸತ್ಕಾರಪರರು. ಇತರರಿಂದ ದುಡ್ಡು ಸುಲಿಯುವುದರಲ್ಲಿ ನಿಪುಣರು. ನಂತರ ತಮ್ಮೊಳಗೇ ಅವರನ್ನು ನೋಡಿ ನಗುವರು.

ಫ್ರೆಂಚರ ಸಮಾಜದಲ್ಲಿ ಮತ್ತೊಂದನ್ನು ಗಮನಿಸಬೇಕಾಗಿದೆ. ಅಮೆರಿಕಾ, ಇಂಗ್ಲಿಷ್​, ಜರ್ಮನ್​ ಸಮಾಜಗಳು ಎಲ್ಲರಿಗಾಗಿ ತೆರೆದಿವೆ; ವಿದೇಶಿಯರು ಅದನ್ನು ಒಳಹೊಕ್ಕು ತಿಳಿದುಕೊಳ್ಳಹುದು. ನಾಲ್ಕೈದಿ ದಿನ ಪರಿಚಯವಾದ ಮೇಲೆ ಅಮೆರಿಕಾದವರು ತಮ್ಮ ಮನೆಯಲ್ಲೇ ಬಂದು ಇರಬಹುದೆಂಬ ನಿಮಂತ್ರಣವನ್ನು ಕೊಡುವರು. ಜರ್ಮನಿಯರೂ ಅವರಂತೆಯೇ. ಆಂಗ್ಲೇಯರು ಪರಿಚಯ ಹೆಚ್ಚಾ ದ ಮೇಲೆ ಮಾಡುವರು. ಆದರೆ ಫ್ರೆಂಚ್​ ಸಮಾಜವೇ ಬೇರೆ. ಒಬ್ಬ ಅತಿ ನಿಕಟ ಪರಿಚಿತನಾದ ಹೊರತು ಮನೆಯಲ್ಲಿ ಬಂದು ಇರಿ ಎಂದು ಅವನಿಗೆ ಎಂದಿಗೂ ನಿಮಂತ್ರಣ ಕೊಡುವುದಿಲ್ಲ. ಅಪರಿಚಿತನಿಗೆ ಇಂತಹ ಒಂದು ಅವಕಾಶ ಸಿಕ್ಕಿದಾಗ, ಅವರ ಸಮಾಜವನ್ನು ನೋಡಿ ತಿಳಿದುಕೊಂಡಾಗ, ತಾನು ಕೇಳಿದುದಕ್ಕಿಂತ ಅದು ಬೇರೆ ಯಾಗಿರುವುದು ತಿಳಿಯುವುದು. ವಿದೇಶಿಯರು ಕಲ್ಕತ್ತೆಗೆ ಬಂದು ಮಚುವಾಬ ಜಾರ್​ ನೋಡಿ ಇಡೀ ದೇಶದ ಜನರೇ ಹೀಗೆ ಎಂದರೆ ಎಷ್ಟು ಅನ್ಯಾಯವಾಗು ವುದೊ ಅದರಂತೆಯೆ ಪ್ಯಾರಿಸ್ಸಿನ ವಿಷಯದಲ್ಲೂ ಕೂಡ. ಫ್ರಾನ್ಸಿನಲ್ಲಿ ಅವಿವಾಹಿತ ಹುಡುಗಿಯರು ನಮ್ಮ ದೇಶದಂತೆಯೇ ಸುರಕ್ಷಿತರು. ಅವರು ಸ್ವೇಚ್ಛೆಯಿಂದ ಸಮಾಜದಲ್ಲಿ ಬೆರೆಯಲು ಆಗುವುದಿಲ್ಲ. ವಿವಾಹಾನಂತರ ತನ್ನ ಪತಿಯೊಡನೆ ಎಲ್ಲಿಗಾದರೂ ಹೋಗಿ ಸಮಾರಂಭಗಳಲ್ಲಿ ಭಾಗಿಯಾಗಬಹುದು. ನಮ್ಮಂತೆಯೇ ವಿವಾಹವನ್ನು ತಂದೆ ತಾಯಿಗಳು ಏರ್ಪಡಿಸುವರು. ಅವರು ಸಂತೋಷಪ್ರಿಯ ರಾದುದರಿಂದ ಯಾವುದಾದರೊಂದು ವಿಶೇಷ ಸಮಯದಲ್ಲಿ ನರ್ತಕಿಯರ ನೃತ್ಯ ವಿಲ್ಲದೆ ಅದು ಪೂರ್ಣವಾಗುವುದಿಲ್ಲ. ನಮ್ಮ ದೇಶದಲ್ಲೂ ವಿವಾಹ, ಪೂಜೆ ಮುಂತಾದ ಸಮಯದಲ್ಲಿ ನರ್ತಕಿಯರಿಂದ ನೃತ್ಯ ಮಾಡಿಸುವರು. ಆಂಗ್ಲೇಯರು ಧೂಳು, ಮಂಜಿನಿಂದ ಕವಿದ ದೇಶದಲ್ಲಿ ವಾಸಿಸುವುದರಿಂದ ಹಾಸ್ಯಪ್ರಿಯರಲ್ಲ; ಯಾವಾಗಲೂ ಜೋಲು ಮುಖ ಮಾಡಿಕೊಂಡಿರುವರು. ಅವರ ದೃಷ್ಟಿಯಲ್ಲಿ ಇಂತಹ ನೃತ್ಯ ಮನೆಯಲ್ಲಿ ನಡೆಯುವುದು ಅಶ್ಲೀಲ. ಆದರೆ ರಂಗಭೂಮಿಯ ಮೇಲೆ ಆದರೆ ಪರವಾಗಿಲ್ಲ. ಅವರ ನೃತ್ಯ ನಮ್ಮ ಕಣ್ಣಿಗೆ ಅಶ್ಲೀಲವಾಗಿ ಕಾಣಬಹುದು. ಆದರೆ ಅವರಿಗೆ ಇದು ಅಭ್ಯಾಸವಾಗಿರುವುದರಿಂದ ಹಾಗೆ ಕಾಣುವುದಿಲ್ಲ. ಇಂತಹ ನೃತ್ಯಕಾಲದಲ್ಲಿ ಹುಡುಗಿಯರ ಭುಜ ಕತ್ತು ಕಾಣಬಹುದು. ಇದು ಅಶ್ಲೀಲವಲ್ಲ. ಇಂಗ್ಲೀಷರು ಮತ್ತು ಅಮೆರಿಕಾದವರು ಇಂತಹ ನೃತ್ಯವನ್ನು ಅಕ್ಷೇಪಿಸುವುದಿಲ್ಲ. ಆದರೆ ಮನೆಗೆ ಹೋಗಿ ಫ್ರೆಂಚರ ಸಮಾಜವನ್ನು ಟೀಕಿಸಬಹುದು.

ಸ್ತ್ರೀ ಸಂಬಂಧದಲ್ಲಿ ಪೃಥ್ವಿಯ ಎಲ್ಲಾ ದೇಶದಲ್ಲೂ ಒಂದೇ ಬಗೆಯ ಆಚಾರ ರೂಢಿಯಲ್ಲಿದೆ. ಗಂಡಸರು ಅನ್ಯ ಸ್ತ್ರೀಯೊಂದಿಗೆ ಸಂಪರ್ಕವನ್ನು ಇಟ್ಟುಕೊಂಡಿದ್ದರೆ ಅದು ಮಹಾಪರಾಧವಲ್ಲ. ಆದರೆ ಸ್ತ್ರೀ ಹಾಗೆ ಇದ್ದರೆ ಅದನ್ನು ಭಯಂಕರವಾಗಿ ಭಾವಿಸುವರು. ಫ್ರೆಂಚರು ಈ ವಿಷಯದಲ್ಲಿ ಸ್ವಲ್ಪ ಸ್ವತಂತ್ರರು. ಬೇರೆ ದೇಶದಲ್ಲಿ ಶ‍್ರೀಮಂತರು ಹೇಗೆ ಆ ವಿಷಯದಲ್ಲಿ ನಡೆಯುವರೋ ಹಾಗೆಯೇ ಇಲ್ಲಿಯೂ ಕೂಡ ಟೀಕೆಗೆ ಅಷ್ಟು ಗಮನ ಕೊಡುವುದಿಲ್ಲ. ಪಾಶ್ಚಾತ್ಯ ದೇಶದಲ್ಲಿ ಅವಿವಾಹಿತ ಪುರುಷರೂ ಹೀಗೆ. ಅಲ್ಲಿ ಯುವಕ ವಿದ್ಯಾರ್ಥಿ ಅಲ್ಲಿ ಸ್ತ್ರೀಯರ ವಿಷಯದಲ್ಲಿ ಸಂಕೋಚ ಪ್ರವೃತ್ತಿಯವನಾಗಿದ್ದರೆ ಅದೊಂದು ನ್ಯೂನತೆ ಎಂದು ತಂದೆ ತಾಯಿಗಳು ಭಾವಿಸುವರು. ಏಕೆಂದರೆ ಮುಂದೆ ಅವನು ಪೌರುಷ ಹೀನನಾಗುತ್ತಾನೆ ಎಂಬ ಅಂಜಿಕೆ. ಪಶ್ಚಿಮದಲ್ಲಿ ಪುರುಷನಿಗಿರಬೇಕಾದ ದೊಡ್ಡಗುಣ ಎಂದರೆ ಧೈರ್ಯ. ಅವರ \enginline{Virtue} ವಿಗೂ ನಮ್ಮ ವೀರತ್ವ ಎರಡಕ್ಕೂ ಒಂದೇ ಅರ್ಥ. \enginline{Virtue} ಎಂಬುದರ ವ್ಯುತ್ಪತ್ತಿಯನ್ನು ನೋಡಿ. ಪುರುಷನಲ್ಲಿ ಯಾವುದು ಒಳ್ಳೆಯದು ಎಂದು ಕರೆಯುತ್ತಾರೆ ಎಂಬುದನ್ನು ನೋಡಿ. ಸ್ತ್ರೀಯರಿಗೆ ಪಾತಿವ್ರತ್ಯ ಅತ್ಯಮೂಲ್ಯ ಗುಣ ವೇನೋ ನಿಜ. ಒಬ್ಬ ಹೆಂಗಸು ಹಲವು ಪತಿಯರನ್ನು ಏಕಕಾಲದಲ್ಲಿ ವರಿಸುವುದ ಕ್ಕಿಂತ ಒಬ್ಬ ಪುರುಷ ಹಲವು ಸ್ತ್ರೀಯರನ್ನು ಮದುವೆಯಾದರೆ ಸಮಾಜಕ್ಕೆ ಅಷ್ಟು ಅಪಾಯವಿಲ್ಲ. ಹೆಂಗಸು ಬಹು ಪತಿಯರನ್ನು ವರಿಸಿದರೆ ಜನಾಂಗ ಕ್ರಮೇಣ ನಾಶ ವಾಗುವುದು. ಆದಕಾರಣವೇ ಎಲ್ಲಾ ದೇಶಗಳಲ್ಲಿಯೂ ಸ್ತ್ರೀಯರ ಪಾತಿವ್ರತ್ಯವನ್ನು ರಕ್ಷಿಸಲು ಅಷ್ಟು ಪ್ರಯತ್ನ ಪಡುವರು. ಈ ಮಾನವ ಪ್ರಯತ್ನದ ಹಿಂದೆ ಪ್ರಕೃತಿ ಯ ಕೈವಾಡವಿದೆ. ಪ್ರಕೃತಿಯ ಸ್ವಭಾವ ಸಂತಾನೋತ್ಪತ್ತಿ. ಪಾತಿವ್ರತ್ಯ ಇದಕ್ಕೆ ಸಹಕಾರಿ. ಪುರುಷನ ಬ್ರಹ್ಮಚರ್ಯ ವ್ರತಕ್ಕಿಂತ ಸ್ತ್ರೀಯರ ಪಾತಿವ್ರತ್ಯಕ್ಕೆ ಹೆಚ್ಚು ಪ್ರಾಧಾನ್ಯ ನೀಡುವುದರಿಂದ ಪ್ರಕೃತಿಯ ಕೆಲಸಕ್ಕೆ ಹೆಚ್ಚು ಸಹಾಯ ದೊರಕುತ್ತದೆ.

ಈ ವಿಷಯವನ್ನು ಕುರಿತು ನಾನು ಹೇಳುತ್ತಿರುವುದಕ್ಕೆ ಕಾರಣ ಪ್ರತಿಯೊಂದು ಜನಾಂಗಕ್ಕೂ ಒಂದು ನೈತಿಕ ಜೀವನೋದ್ದೇಶವಿದೆ ಎಂಬುದನ್ನು ಎತ್ತಿ ತೋರಿಸು ವುದು. ಪ್ರತಿಯೊಂದು ಜನಾಂಗದ ರೀತಿ ನೀತಿಗಳನ್ನು ಆ ಉದ್ದೇಶದ ದೃಷ್ಟಿಯಿಂದ ವಿಚಾರಮಾಡಬೇಕು. ಪಾಶ್ಚಾತ್ಯರನ್ನು ಅವರ ದೃಷ್ಟಿಯಿಂದಲೇ ಅಳೆಯಬೇಕೇ ಹೊರತು ನಮ್ಮ ದೃಷ್ಟಿಯಿಂದ ಅಳೆಯುವುದು ಶುದ್ಧ ತಪ್ಪು. ನಮ್ಮ ಜೀವ ನೋದ್ದೇಶ ಅವರ ಉದ್ದೇಶಕ್ಕೆ ವಿರುದ್ಧವಾದುದಾಗಿದೆ. ಸಂಸ್ಕೃತದಲ್ಲಿ ಬ್ರಹ್ಮಚಾರಿ ಎಂದರೆ ಯಾರು ಕಾಮವನ್ನು ಗೆದ್ದಿರುವನೊ ಅವನು. ವಿದ್ಯಾರ್ಥಿ ಮತ್ತು ಬ್ರಹ್ಮಚಾರಿ ಎರಡೂ ಪರ್ಯಾಯ ಪದಗಳು. ನಮ್ಮ ಉದ್ದೇಶ ಮೋಕ್ಷ. ಬ್ರಹ್ಮಚರ್ಯವಿಲ್ಲದೆ ಅದು ಹೇಗೆ ಸಿದ್ಧಿಸುವುದು? ಆದ್ದರಿಂದ ನಮ್ಮ ತರುಣರಿಗೆ ಅದು ವಿದ್ಯಾರ್ಥಿ ದೆಸೆಯಲ್ಲಿ ಅನಿವಾರ್ಯವಾದುದು ಎಂದು ವಿಧಿಸಲಾಗಿದೆ. ಪಶ್ಚಿಮದವರ ಉದ್ದೇಶ ಭೋಗ; ಇದಕ್ಕೆ ಬ್ರಹ್ಮಚರ್ಯದ ಆವಶ್ಯಕತೆ ಅಷ್ಟಾಗಿ ಇಲ್ಲ.

ಪ್ಯಾರಿಸ್ಸಿನಷ್ಟು ಸುಂದರನಗರ ಪ್ರಪಂಚದಲ್ಲಿ ಮತ್ತೆಲ್ಲೂ ಇಲ್ಲ. ಹಿಂದೆ ಪ್ಯಾರಿಸ್​ ಬೇರೆ ಒಂದು ರೀತಿಯಾಲ್ಲಿತ್ತು. ಕಾಶಿಯಲ್ಲಿ ಬಂಗಾಳಿಗಳು ಇರುವ ಭಾಗದಂತೆ ಇತ್ತು. ಬೀದಿ ಗಲ್ಲಿಗಳು ಏಳುಬೀಳುಗಳಿಂದ ಕೂಡಿತ್ತು. ರಸ್ತೆಯ ಎದುರುಬದುರಿ ಗಿರುವ ಮನೆಗಳನ್ನು ಮೇಲೆ ಒಂದರಿಂದ ಮತ್ತೊಂದಕ್ಕೆ ಹೋಗುವಂತೆ ಸೇರಿಸಿದ್ದರು. ಗೋಡೆಯ ಪಕ್ಕದಲ್ಲೆ ಬಾವಿ. ಕಳೆದ ಪ್ಯಾರಿಸ್​ ಪ್ರದರ್ಶನದಲ್ಲಿ ಪುರಾತನ ಪ್ಯಾರಿಸ್​ ನಗರಿಯನ್ನು ತೋರಿಸಿದ್ದರು. ಈಗ ಪುರಾತನ ಪ್ಯಾರಿಸ್ಸೆಲ್ಲಿ? ಕ್ರಮೇಣ ಕ್ರಾಂತಿ, ಯುದ್ಧ, ದಂಗೆಯ ಹಾವಳಿಗಳಿಂದ ಅದು ನಾಶವಾಗಿ ಈಗ ನವೀನ, ವಿಶಾಲ, ಸುಂದರ ಪ್ಯಾರಿಸ್​ ಅದರ ಸ್ಥಳದಲ್ಲಿರುವುದು.

ವರ್ತಮಾನ ಪ್ಯಾರಿಸ್​ ಮುಕ್ಕಾಲು ಪಾಲೆಲ್ಲಾ ಮೂರನೇ ನೆಪೋಲಿಯನ್ನನ ನಿರ್ಮಾಣ. ಹಿಂದಿನ ಚಕ್ರವರ್ತಿಯ ಪತನಕಾಲದಲ್ಲಿ ಪ್ಯಾರಿಸ್​ನ ಪುನರ್ರಚನೆ ಪ್ರಾರಂಭವಾಯಿತು. ಅದನ್ನು ಇವನು ಪೂರ್ಣಗೊಳಿಸಿದನ್ನು ಫ್ರೆಂಚ್​ ಕ್ರಾಂತಿಗೆ ಮುಂಚೆ ಅಲ್ಲಿಯ ರಾಜರು ಹೇಗೆ ಪ್ರಜೆಗಳನ್ನು ಪೀಡಿಸಿತ್ತಿದ್ದರೆಂಬುದನ್ನು ಚರಿತ್ರೆ ಯನ್ನು ಓದಿದವರಿಗೆ ಹೇಳಬೇಕಾಗಿಲ್ಲ. ಕೇವಲ ಕತ್ತಿಯ ಬಲದಿಂದ ರಕ್ತಪಾತ ಮಾಡಿ ಮೂರನೇ ನೆಪೋಲಿಯನನು ಚಕ್ರವರ್ತಿಯಾದನು. ಫ್ರಾನ್ಸಿನ ಕ್ರಾಂತಿಯಾದ ಮೇಲೆ ಜನರು ಅಸ್ಥಿರವಾಗಿದ್ದರು. ಕೆಲವರನ್ನು ಸುಖಿಗಳನ್ನಾಗಿ ಮಾಡುವುದಕ್ಕೆ ಮತ್ತು ಗರೀಬರಿಗೆ ಕೆಲಸ ಕೊಡುವುದಕ್ಕೆ ದೊಡ್ಡ ದೊಡ್ಡ ರಸ್ತೆ, ನಾಟ್ಯಶಾಲೆ, ಘಾಟ್​ ಮುಂತಾದುವನ್ನು ಕಟ್ಟಿಸತೊಡಗಿದನು. ಪ್ಯಾರಿಸ್ಸಿನ ಹತ್ತಿರವಿದ್ದ ಮಂದಿರ, ಸ್ತಂಭ, ಸ್ಮಾರಕಗಳನ್ನು ಹಾಗೆಯ ಬಿಟ್ಟು ನಗರವನ್ನು ವಿಸ್ತರಿಸಿದನು. ರಸ್ತೆ, ಘಾಟ್​ ಎಲ್ಲವನ್ನೂ ಹೊಸದಾಗಿ ಮಾಡಿಸಿದನು. ವಿಶಾಲವಾದ ನೇರವಾದ ರಸ್ತೆಗಳು ನಗರದಲ್ಲೆಲ್ಲಾ ಆದುದಲ್ಲದೆ ಹಲವು ಹಳೆಯ ಮನೆ ಜೀರ್ಣೋದ್ಧಾರವಾದುವು. ಹೊಸ ಮನೆಗಳನ್ನು ಕಟ್ಟಿಸಿದನು. ನಗರದ ಸುತ್ತಲೂ ಅದರ ಎರಡರಷ್ಟು ವಿಶಾಲವಾದ ಕೋಟೆಯನ್ನು ಕಟ್ಟಿಸಿದನು. ನಂದನವನಗಳು, ಡಿ ಆನ್​ಜಸ್​ ಮೊದಲಾದ ಬಡಾವಣೆ ಮತ್ತು ಪ್ರಪಂಚದಲ್ಲೆಲ್ಲಾ ಅತಿ ಸುಂದರವಾದ ಸಾಲುಮರದ ರಸ್ತೆ \enginline{(Champs Elysess)} ಇವನ್ನೆಲ್ಲಾ ಮಾಡಿದನು. ಈ ರಸ್ತೆ ತುಂಬಾ ವಿಶಾಲವಾಗಿರುವುದು. ರಸ್ತೆಯ ಮಧ್ಯದಲ್ಲಿ ಮತ್ತು ಎರಡು ಕಡೆಗಳಲ್ಲಿ ಉದ್ಯಾನಗಳಿವೆ. ಈ ರಸ್ತೆ ಪಶ್ಚಿಮದಲ್ಲಿ ವೃತ್ತಾಕಾರ ವಾಗಿ ನಗರದ ಮುಂಭಾಗವಾದ \enginline{Place de la Concorde} ಎಂದು ಆಗಿದೆ.ಇದರ ಸುತ್ತಲೂ ಫ್ರಾನ್ಸಿನ ಎಂಟು ನಗರಗಳನ್ನು ಪ್ರತಿನಿಧಿಸುವ ಎಂಟು ಸ್ತ್ರೀ ಪ್ರತಿಮೆಗಳು ನಿರ್ಮಾಣಗೊಂಡಿವೆ. ಇದರಲ್ಲಿ ಒಂದು ಸ್ಟ್ರಾಸ್​ ಬರ್ಸ್​ನ ಪ್ರತೀಕ. ಈ ನಗರವನ್ನು ಜರ್ಮನರು (1870) ಯುದ್ಧದಲ್ಲಿ ಫ್ರೆಂಚರಿಂದ ತೆಗೆದುಕೊಂಡರು. ಆ ದುಃಖವನ್ನು ಫ್ರೆಂಚರು ಇನ್ನೂ ಮರೆತಿಲ್ಲ. ಯಾವಾಗಲೂ ಆ ವಿಗ್ರಹ ಹೂವು ಹಾರಗಳಿಂದ ಕಂಗೊಳಿಸುತ್ತಿರುವುದು. ಗತಿಸಿದ ಬಂಧುಗಳ ಗೋರಿಯ ಸಮೀಪದಲ್ಲಿ ಜನರು ಹೇಗೆ ಹಾರ ತುರಾಯಿಯನ್ನು ಅರ್ಪಿಸುವರೋ ಅದರಂತೆ ಈ ವಿಗ್ರಹಕ್ಕೆ ಇಲ್ಲಿಯ ಜನರು ಕಳೆದು ಹೋದ ಆ ನಗರದ ಆತ್ಮ ತೃಪ್ತಿಗಾಗಿ ಶೃಂಗರಿಸುತ್ತಿರುವರು.

ಡೆಲ್ಲಿಯಲ್ಲಿರುವ ಚಾಂದ್ನಿ ಚೌಕ ಒಂದಾನೊಂದು ಕಾಲದಲ್ಲಿ ಹೀಗಿತ್ತು ಎಂದು ಭಾಸವಾಗುವುದು. ಹತ್ತಿರ ಹತ್ತಿರದಲ್ಲೆ ಜಯಸ್ತಂಭ, ವಿಜಯ ತೋರಣ, ಸ್ತ್ರೀ ಪುರುಷರು, ಸಿಂಹ, ಮುಂತಾದವುಗಳ ಶಿಲಾಪ್ರತಿಮೆಗಳನ್ನು ಮಾಡಿರುವರು.

ಮಹಾವೀರ ನೆಪೋಲಿಯನ್ನನ ಜ್ಞಾಪಕಾರ್ಥವಾಗಿ ವೆಂಡೋಮಿ ಎಂಬ ಕಡೆ ಮಿಶ್ರಲೋಹದಿಂದ ಮಾಡಿದ ಬೃಹತ್​ ವಿಜಯಸ್ತಂಭವಿದೆ. ಅದರ ಕೆಳಗೆ ಸುತ್ತಲೂ ನಾಲ್ಕುಕಡೆ ನೆಪೋಲಿಯನ್​ ಗೆದ್ದ ಯುದ್ಧವನ್ನು ಉಲ್ಲೇಖಿಸಿರುವರು. ಅದರ ಮೇಲೆ ನೆಪೋಲಿಯನ್​ ಬೋನಪಾರ್ಟಿಯ ವಿಗ್ರಹವಿದೆ. \enginline{Place de la Bastille} ಎಂಬ ಕಡೆ \enginline{Column of July} ಎಂಬುದು ಇದೆ. ಅದು 1789 ರ ಕ್ರಾಂತಿಯ ಸ್ಮಾರಕ. ಅದು ಮುಂದೆ ಸೆರೆಮನೆಯಾಯಿತು. ಆಗಿನ ಕಾಲದಲ್ಲಿ ರಾಜ ನಿರಂಕುಶ ಪ್ರಭು. ಯಾರು ಅವನ ಕೋಪಕ್ಕೆ ಪಾತ್ರರಾಗುತ್ತಿದ್ದರೋ ಅವರನ್ನು ಆ ಸೆರೆಮನೆಗೆ ತಳ್ಳುತ್ತಿದ್ದರು. ಯಾವ ವಿಚಾರಣೆಯನ್ನೂ ನಡೆಸುತ್ತಿರಲಿಲ್ಲ.ರಾಜ ಒಂದು ಆಜ್ಞೆಯನ್ನು ಬರೆದುಕೊಡುತ್ತಿದ್ದನು. ಅದರ ಹೆಸರು \enginline{Letter de Cachet}. ಇದು ಇದ್ದರೆ ಸಾಕು, ಆ ವ್ಯಕ್ತಿ ದೋಷಿಯೆ ನಿರ್ದೋಷಿಯೆ ಯಾವುದನ್ನೂ ವಿಚಾರಿಸದೆ ಸೆರೆಮನೆಗೆ ತಳ್ಳುತ್ತಿದ್ದರು. ಅಲ್ಲಿ ಒಮ್ಮೆ ಹೋದರೆ ಪುನಃ ಬರುವಂತೆ ಇಲ್ಲ. ರಾಜನ ಪ್ರೀತಿಗೆ ಪಾತ್ರರಾದವರು ಇತರರ ಮೇಲೆ ತಮಗೆ ಇರುವ ಸೇಡನ್ನು ತೀರಿಸಿಕೊಳ್ಳಬೇಕಾದರೆ ರಾಜನಿಂದ ಈ ಆಜ್ಞಾಪತ್ರವನ್ನು ತೆಗೆದುಕೊಂಡು ತಮ್ಮ ವೈರಿಯನ್ನು ಈ ಸೆರೆಮನೆಗೆ ತಳ್ಳುತ್ತಿದ್ದರು. ಕೊನೆಗೆ ಪ್ರಜೆಗಳು ಈ ಅತ್ಯಾಚಾರವನ್ನು ಸಹಿಸಲಾರದೆ ಕ್ರೋಧೋನ್ಮತ್ತರಾಗಿ ದಂಗೆ ಎದ್ದರು. ವ್ಯಕ್ತಿಗತ ಸ್ವಾತಂತ್ರ್ಯ, ಎಲ್ಲರೂ ಸಮಾನರು, ಯಾರೂ ಮೇಲಲ್ಲ, ಕೀಳಲ್ಲ ಎಂಬ ಧ್ವನಿಯಿಂದ ದೇಶವೆಲ್ಲ ಅನುರಣಿತ ವಾಯಿತು. ಪ್ಯಾರಿಸ್ಸಿನ ಜನ ಕ್ರೋಧಾವೇಶದಿಂದ ರಾಜ ರಾಣಿಯರ ಮೇಲೆ ಧಾಳಿ ನಡೆಸಿದರು. ಆ ಸಮಯದಲ್ಲಿ ರಾಜನ ಅತ್ಯಾಚಾರದ ಚಿಹ್ನೆಯಂತಿದ್ದ ಬ್ಯಾಸ್ಟಿಲಿ ಯನ್ನು ಧ್ವಂಸಮಾಡಿ ಅಂದಿನ ರಾತ್ರಿಯನ್ನು ಗೀತ, ನೃತ್ಯ ಆಮೋದಗಳಲ್ಲಿ ಕಳೆದರು. ರಾಜ ತಪ್ಪಿಸಿಕೊಳ್ಳುವುದಕ್ಕೆ ಪ್ರಯತ್ನಿಸಿದ. ಜನರು ಅವನನ್ನು ಪುನಃ ಸೆರೆಹಿಡಿದರು. ಅವನ ಮಾವನು, ಅಳಿಯನ ಸಹಾಯಕ್ಕೆ ಆಸ್ಟ್ರಿಯಾದಿಂದ ಸೇನೆಯನ್ನು ಕಳುಹಿಸು ತ್ತಿರುವನು ಎಂಬ ಸುದ್ದಿಯನ್ನು ಕೇಳಿ ಜನರು ಫ್ರಾನ್ಸಿನ ರಾಜ ರಾಣಿಯರನ್ನು ಕೊಲೆ ಮಾಡಿದರು. ಫ್ರಾನ್ಸ್​ ಅಂದಿನಿಂದ ಸ್ವಾತಂತ್ರ್ಯ ಮತ್ತು ಸಮತ್ವದ ಹೆಸರಿನಲ್ಲಿ ಹುಚ್ಚಾಯಿತು. ಫ್ರಾನ್ಸಿನಲ್ಲಿ ಪ್ರಜಾತಂತ್ರ ಸ್ಥಾಪಿತವಾಯಿತು. ರಾಜಮನೆತನ ದವರನ್ನೆಲ್ಲ ಕೊಲೆಮಾಡಲನುವಾದರು. ಅನೇಕರು ತಮ್ಮ ಬಿರುದು ಬಾವಲಿಗಳನ್ನು ಬಿಸುಟು ಸಾಮಾನ್ಯ ಪ್ರಜೆಗಳೊಂದಿಗೆ ಬೆರೆಯತೊಡಗಿದ್ದರು. ಪ್ರಪಂಚದಲ್ಲಿ ಎಲ್ಲರಿಗೂ ರಾಜರಿಂದ ಸ್ವತಂತ್ರರಾಗುವಂತೆ ಕರೆಕಳುಹಿಸಿದರು. “ಜಾಗ್ರತರಾಗಿ, ದ್ರೋಹಿಗಳಾದ ರಾಜರನ್ನು ನಾಶಮಾಡಿ, ಸ್ವತಂತ್ರರಾಗಿ ಸಮಾನಸ್ಕಂಧರಾಗಿ.” ಆ ಸಮಯದಲ್ಲಿ ಯೂರೋಪಿನ ಅರಸರು ಭಯದಿಂದ ಕಂಪಿಸತೊಡಗಿದರು. ಈ ಜ್ವಾಲೆ ತಮ್ಮ ದೇಶಕ್ಕೂ ವ್ಯಾಪಿಸಿ ತಮ್ಮ ಸಿಂಹಾಸನವನ್ನು ಎಲ್ಲಿ ದಹಿಸುವುದೋ ಎಂದು ಭಯಗ್ರಸ್ಥರಾಗಿ ಅದನ್ನು ಅಡಗಿಸಲು ನಾಲ್ಕು ಕಡೆಯಿಂದಲೂ ಫ್ರಾನ್ಸಿನ ಮೇಲೆ ಧಾಳಿಯಿಟ್ಟರು. ಫ್ರಾನ್ಸ್​ ತನ್ನ ಮಕ್ಕಳಿಗೆಲ್ಲ “ಜನ್ಮ ಭೂಮಿ ಅಪಾಯ ದಲ್ಲಿದೆ” ಎಂಬ ಕರೆ ಕಳುಹಿಸಿತು. ಈ ಕರೆಯನ್ನು ಕೇಳಿ ದೇಶವೆಲ್ಲ ಕ್ಷಣದಲ್ಲಿ ಜಾಗ್ರತವಾಯಿತು. ಆಬಾಲವೃದ್ಧರೂ, ಸ್ತ್ರೀಪುರುಷರೂ ಫ್ರಾನ್ಸಿನ ರಾಷ್ಟ್ರಗೀತೆ ಯಾದ “ \enginline{La-Marseillaise}” ಯನ್ನು ಉತ್ಸಾಹಪೂರ್ಣವಾಗಿ ಹಾಡುತ್ತಾ ಸೇನೆಗೆ ಸೇರುವುದಕ್ಕೆ ಧಾವಿಸಿ ಬಂದರು. ದೀನ ಫ್ರೆಂಚ್​ ಜನರ ತಂಡ ಚಿಂದಿಬಟ್ಟೆ ಹೊದ್ದು, ಬರೀ ಕಾಲಿನಲ್ಲಿ ನಡುಗಿಸುವ ಚಳಿಯಲ್ಲಿ ಅರೆಹೊಟ್ಟೆ ಯಿದ್ದರೂ ಹೆಗಲ ಮೇಲೆ ಬಂದೂಕನ್ನು ಹೊತ್ತುಕೊಂಡು, ದುರ್ಜನರ ನಾಶಕ್ಕೆ, ತಮ್ಮ ಪವಿತ್ರ ದೇಶೋದ್ಧಾರಕ್ಕೆ ಸಿದ್ಧರಾಗಿ, ಯೂರೋಪಿನ ಪ್ರಚಂಡ ಸೇನಾ ಸಮೂಹವನ್ನೇ ಧೈರ್ಯದಿಂದ ಎದುರಿಸಿತು. ಇಡೀ ಯೂರೋಪಿಗೆ ಇದರ ವೇಗವನ್ನು ಸಹಿಸುವುದಕ್ಕೆ ಅಸಾಧ್ಯವಾಯಿತು. ಆ ಮಹಾಸೇನೆಗೆ ಅಧಿಪತಿಯಾದ ಒಬ್ಬ ವೀರ ಮುಂದೆ ನಿಂತು ಸಿಂಹನಾದ ಮಾಡಿದನು. ಅವನ ಒಂದು ಕೈ ಬೆರಳ ಸಲ್ಲೆಗೆ ಇಡಿ ಪೃಥ್ವಿ ಕಂಪಿಸುತ್ತಿತ್ತು. ಅವನೇ ನೆಪೋಲಿಯನ್​ ಬೋನಪಾರ್ಟೆ. ಭ್ರಾತೃತ್ವ, ಸಾಮ್ಯ, ಸ್ವಾತಂತ್ರ್ಯಗಳ ಭಾವವನ್ನು ಕೋವಿ ಮತ್ತು ಬಲದಿಂದ ಯೂರೋಪಿನ ಮಜ್ಜೆ ಮಾಂಸಗಳಿಗೂ ಪ್ರವೇಶಿಸುವಂತೆ ಮಾಡಿದನು. ಇದಾದ ನಂತರ ಫ್ರಾನ್ಸ್​ ದೇಶವನ್ನು ಸುವ್ಯವಸ್ಥಿತ ಸ್ಥಿತಿಗೆ ತರುವುದಕ್ಕಾಗಿ ಅವನು ಚಕ್ರವರ್ತಿಯಾದನು.

ಮಕ್ಕಳಾಗದೇ ಇದ್ದುದರಿಂದ ತನ್ನ ಸುಖದುಃಖಸಂಗಿನಿ ಭಾಗ್ಯಲಕ್ಷ್ಮಿ ರಾಣಿ ಜೋಸೆಫೈನ್​ಳನ್ನು ತ್ಯಾಗಮಾಡಿ ಆಸ್ಟ್ರಿಯಾ ರಾಜಕುಮಾರಿಯನ್ನು ಮದುವೆ ಯಾದನು. ಜೊಸೆಫೈನಳನ್ನು ತ್ಯಜಿಸಿದ ಮೇಲೆ ನೆಪೋಲಿಯನ್​ ಭಾಗ್ಯಹೀನ ನಾದನು. ರಷ್ಯಾ ಮೇಲೆ ಧಾಳಿ ಇಟ್ಟಾಗ ಅಲ್ಲಿಯ ಹಿಮದಲ್ಲಿ ಅವನ ಸೇನೆಯೆಲ್ಲಾ ನಾಶವಾಯಿತು. ಯೂರೋಪಿನವರಿಗೆ ಈ ಅವಕಾಶ ದೊರಕಿ ಸಿಂಹಾಸನ ತ್ಯಜಿಸು ವಂತೆ ಬಲಾತ್ಕರಿಸಿ ದ್ವೀಪಾಂತರ ಒಂದರಲ್ಲಿ ಅವನನ್ನು ಸೆರೆಯಲ್ಲಿಟ್ಟರು. ಹಿಂದಿನ ರಾಜವಂಶಕ್ಕೆ ಸೇರಿದವನೊಬ್ಬನನ್ನು ಸಿಂಹಾಸನದ ಮೇಲೆ ಕೂರಿಸಿದರು. ನೆಪೋಲಿಯನ್​ ಸೆರೆಯಿಂದ ತಪ್ಪಿಸಿಕೊಂಡು ಪುನಃ ಫ್ರಾನ್ಸಿಗೆ ಬಂದನು. ಫ್ರಾನ್ಸ್​ ಆತನನ್ನು ಆಹ್ವಾನಿಸಿ ಪುನಃ ರಾಜನನ್ನಾಗಿ ಮಾಡಿತು. ಹಳೆಯ ರಾಜ ಓಡಿ ಹೋದನು. ಆದರೆ ನೆಪೋಲಿಯನ್ನನ ಭಾಗ್ಯವೆಂದೋ ನಾಶವಾಗಿತ್ತು. ಪುನಃ ಅದು ಹಿಂತಿರುಗಿ ಬರಲಿಲ್ಲ. ಇಡೀ ಯೂರೋಪ್​ ಅವನನ್ನು ಎದುರಿಸಿ ವಾಟರ್​ಲೂ ಕದನದಲ್ಲಿ ಸೋಲಿಸಿತು. ನೆಪೋಲಿಯನ್​ ಒಂದು ಇಂಗ್ಲೀಷರ ಯುದ್ದದ ಹಡಗಿನಲ್ಲಿ ಶರಣಾ ಗತನಾದನು. ಆಂಗ್ಲರು ಅವನು ಸಾಯುವವರೆಗೂ ಸೆಂಟ್​ ಹೇಲಿನಾ ದ್ವೀಪದಲ್ಲಿ ಸೆರೆಯಿಟ್ಟರು. ಪುನಃ ಹಿಂದಿನ ರಾಜಮನೆತನದವನೊಬ್ಬನನ್ನು ಸಿಂಹಾಸನದ ಮೇಲೆ ಕುಳ್ಳಿರಿಸಿದರು. ಫ್ರೆಂಚರು ಅವನ ಮೇಲೆ ದಂಗೆ ಎದ್ದರು. ಅವನನ್ನು ಓಡಿಸಿ ಪುನಃ ಪ್ರಜಾಪ್ರಭುತ್ವವನ್ನು ಸ್ಥಾಪಿಸಿದರು. ಕ್ರಮೇಣ ಮಹಾವೀರ ನೆಪೋಲಿಯನ್ನನ ಸಂಬಂಧಿ, ಮೂರನೆ ನೆಪೋಲಿಯನ್​ ಫ್ರೆಂಚರ ಪ್ರೀತಿಗೆ ಪಾತ್ರನಾದನು. ಆತ ರಾಜಪಿತೂರಿ ನಡೆಸಿ ರಾಜನಾದ. ಕೆಲವು ದಿನ ಅವನು ದರ್ಪದಿಂದ ರಾಜ್ಯವಾಳಿ ದನು. ಜರ್ಮನಿಯರೊಂದಿಗೆ ಯುದ್ಧ ಮಾಡುವಾಗ ಫ್ರೆಂಚರು ಸೋತರು. ಆಗ ರಾಜನನ್ನು ತೊರೆದು ಪುನಃ ಪ್ರಜಾಪ್ರಭುತ್ವವನ್ನು ಸ್ಥಾಪಿಸಿದರು. ಅಂದಿನಿಂದ ಇಂದಿನವರೆಗೂ ಅದು ಪ್ರಜಾಪ್ರಭುತ್ವವಾಗಿಯೇ ಉಳಿದಿರುವುದು.


\section{ನಾಗರೀಕತೆಯ ಮುನ್ನಡೆ}

ವಿಕಾಸವಾದವು ಹಿಂದೂ ದರ್ಶನಗಳೆಲ್ಲಕ್ಕೂ ಮೂಲಾಧಾರವಾಗಿದೆ. ಅದು ಯೂರೋಪಿನ ಭೌತ ವಿಜ್ಞಾನವನ್ನು ಈಗ ಪ್ರವೇಶಿಸಿದೆ. ಭರತಖಂಡವನ್ನು ಬಿಟ್ಟರೆ ಉಳಿದೆಲ್ಲರೂ ಜಗತ್ತು ಸಂಪೂರ್ಣ ಪ್ರತ್ಯೇಕವಾದ ಬೇರೆ ಬೇರೆ ಅಂಶಗಳಿಂದ ಆಗಿದೆ ಎಂದು ನಂಬುವರು. ದೇವರು, ಪ್ರಕೃತಿ, ಮನುಷ್ಯ ಎಲ್ಲವೂ ಬೇರೆ ಬೇರೆ, ಅದರಂತೆಯೇ ಮೃಗ, ಪ್ರಾಣಿ, ಕೀಟ, ಮರ, ಮಣ್ಣು, ಕಲ್ಲು, ಲೋಹ ಇವೆಲ್ಲವನ್ನೂ ದೇವರು ಪ್ರತ್ಯೇಕ ಪ್ರತ್ಯೇಕವಾಗಿ ಆದಿಯಲ್ಲಿ ಸೃಷ್ಟಿಸಿದನೆಂದು ನಂಬುತ್ತಾರೆ.

ಜ್ಞಾನವೆಂದರೆ ವೈವಿದ್ಯತೆಯ ಹಿಂದೆ ಐಕ್ಯತೆಯನ್ನು ನೋಡುವುದು. ಯಾವುದು ಹೊರಗೆ ಪ್ರತ್ಯೇಕವಾಗಿ ಕಾಣುವುದೋ ಅದರ ಅಂತರಾಳದಲ್ಲಿ ಐಕ್ಯತೆ ಇದೆ. ಆ ಐಕ್ಯತೆಯನ್ನು ಕಂಡುಕೊಳ್ಳಲು ಸಹಾಯ ಮಾಡುವ ಸಂಬಂಧವನ್ನೇ ನಿಯಮ ಎಂದು ಕರೆಯುತ್ತಾರೆ. ಅದೇ ಪ್ರಕೃತಿ ನಿಯಮ.

ನಮ್ಮ ವಿದ್ಯೆ, ಬುದ್ಧಿ, ಚಿಂತನೆಗಳೆಲ್ಲ ಆಧ್ಯಾತ್ಮಿಕವೆಂಬುದನ್ನು ಹಿಂದೆ ಹೇಳಿರು ವೆವು. ಅವೆಲ್ಲವೂ ಧರ್ಮದಲ್ಲಿ ಅಭಿವ್ಯಕ್ತವಾಗುತ್ತವೆ. ಪಾಶ್ಚಾತ್ಯರಲ್ಲಿ ಆ ಎಲ್ಲಾ ಅಭಿವ್ಯಕ್ತಿಗಳಿಗೂ ಬಹಿರ್ಮುಖವಾದುವು-ಭೌತಿಕ ಮತ್ತು ಸಾಮಾಜಿಕ ಸ್ತರಗಳಲ್ಲಿ. ಭಾರತವರ್ಷದ ಪ್ರಾಚೀನ ತತ್ತ್ವಜ್ಞರು ಕ್ರಮೇಣ ಪ್ರತ್ಯೇಕತೆಯ ಭಾವನೆಯು ದೋಷಯುಕ್ತವಾದುದು ಎಂದು ತಿಳಿದರು. ಪ್ರತ್ಯೇಕವಾದರೂ ಅದರ ಅಂತರಾಳ ದಲ್ಲಿ ಒಂದು ಸಂಬಂಧವಿದೆ ಎಂಬುದನ್ನು ಕಂಡುಕೊಂಡರು. ಮಣ್ಣು, ಕಲ್ಲು, ಗಿಡ, ಪ್ರಾಣಿ, ಮನುಷ್ಯ, ದೇವ, ಈಶ್ವರ ಇವುಗಳ ಹಿಂದೆ ಏಕತ್ವ ಇದೆ. ಅದ್ವೈತ ವಾದಿಯು ಇದರ ಚರಮ ಸೀಮೆಯನ್ನು ಮುಟ್ಟಿರುವನು. ಇವುಗಳೆಲ್ಲ ಒಂದ ರಿಂದ ಆಗಿರುವುದೆಂದು ಸಾರುತ್ತಾನೆ. ಸತ್ಯವಾಗಿ ನೋಡಿದರೆ ಅಧ್ಯಾತ್ಮ ಮತ್ತು ಆದಿ ಭೌತಿಕ ಜಗತ್ತು ಒಂದು. ಅದೇ ಬ್ರಹ್ಮ. ಪ್ರತ್ಯೇಕವನ್ನು ನೋಡುವುದು ದೋಷ. ಅದೇ ಮಾಯೆ, ಅವಿದ್ಯೆ, ಅಜ್ಞಾನ. ಇದೇ ಜ್ಞಾನದ ಚರಮಸೀಮೆ!

ಭಾರತವರ್ಷದ ಮಾತನ್ನು ಬಿಟ್ಟುಬಿಡಿ. ಉಳಿದ ವಿದೇಶೀಯರಿಗೆ ಇದು ಗೊತ್ತಾಗದೇ ಇದ್ದರೆ ಅವರು ವಿದ್ವಾಂಸರು ಹೇಗೆ ಆಗುತ್ತಾರೆ? ಅವರಲ್ಲಿ ಬಹು ಜನ ಇದನ್ನು ಜಡ ವಿಜ್ಞಾನದ ದ್ವಾರ ತಿಳಿದುಕೊಳ್ಳುವರು. ಆ ಒಂದು ಹೇಗೆ ಅನೇಕವಾಯಿತು ಎಂಬುದನ್ನು ನಾವು ಹೇಳಲಾರೆವು, ಅವರೂ ಹೇಳಲಾರರು. ನಾವು ಇದಕ್ಕೆ ಸಂಬಂಧಪಟ್ಟ ಒಂದು ಸಿದ್ಧಾಂತವನ್ನು ಮಾಡಿರುವೆವು. ನಮ್ಮ ವಿಷಯಬುದ್ಧಿಗೆ ಮೀರಿರುವುದರಿಂದ ಅದನ್ನು ನಾವು ತಿಳಿದುಕೊಳ್ಳಲಾರೆವು ಎಂಬುದೇ ಆ ಸಿದ್ಧಾಂತ. ಆದರೆ ಆ ಒಂದು ಯಾವ ಯಾವ ರೂಪದಲ್ಲಿ ಧರಿಸುತ್ತದೆ, ಯಾವ ರೀತಿ ಅದು ಜೀವತ್ವದಲ್ಲಿ ಮತ್ತು ವ್ಯಕ್ತಿತ್ವದಲ್ಲಿ ಪರಿಣಾಮ ವಾತ್ತದೆ ಎಂಬುದನ್ನು ತಿಳಿದುಕೊಳ್ಳಬಹುದು. ಅದೇ ವಿಜ್ಞಾನ.

ಪಾಶ್ಚಾತ್ಯರಲ್ಲಿ ಈಗ ಹೆಚ್ಚು ಕಡಿಮೆ ಎಲ್ಲರೂ ವಿಕಾಸವಾದಿಗಳು. ಸಣ್ಣ ದೊಂದು ಪ್ರಾಣಿ ಬದಲಾವಣೆ ಹೊಂದಿ ದೊಡ್ಡದಾಗುತ್ತದೆ. ದೊಡ್ಡ ಪ್ರಾಣಿ ಸಣ್ಣದಾಗುತ್ತದೆ, ಕೆಲವು ವೇಳೆ ನಾಶವಾಗಿಯೂ ಹೋಗುವುದು. ಇದರಂತೆಯೇ ಮನುಷ್ಯರು ಹುಟ್ಟುವಾಗಲೇ ನಾಗರೀಕರಾಗುವುದಿಲ್ಲ. ಮನುಷ್ಯರೆಲ್ಲರೂ ಮೊದಲು ಅನಾಗರಿಕರಾಗಿದ್ದು ಕ್ರಮೇಣ ನಾಗರಿಕರಾದರು. ಇದಕ್ಕೆ ವಿರುದ್ದ ವಾದುದನ್ನು ಅವರು ನಂಬುವುದಿಲ್ಲ. ಏಕೆಂದರೆ ಕೆಲವು ಶತಮಾನದ ಹಿಂದೆ ಅವರ ದೇಶದಲ್ಲೆ ಕಾಡುಜನರಿದ್ದರು.ಆ ಅವಸ್ಥೆಯಿಂದ ಇಂತಹ ಅತ್ಯಲ್ಪ ಕಾಲದಲ್ಲಿ ಒಂದು ನಾಗರೀಕತೆ ಹುಟ್ಟಿದೆ. ಅದರಂತೆಯೇ ಮನುಷ್ಯರೆಲ್ಲ ಹೀಗೆ ಅನಾಗರಿಕ ಅವಸ್ಥೆಯಿಂದ ನಾಗರಿಕ ಅವಸ್ಥೆಗೆ ಬಂದರು ಮತ್ತು ಇನ್ನೂ ಬರುತ್ತಿರುವರು, ಎಂದು ಪಾಶ್ಚಾತ್ಯರು ಹೇಳುವರು.

ಕಾಡು ಜನರು ಪ್ರಾಚೀನ ಕಾಲದಲ್ಲಿ ಕಲ್ಲು ಮರಗಳ ಉಪಕರಣಗಳಿಂದ ಕೆಲಸ ಮಾಡುತ್ತಿದ್ದರು. ಚರ್ಮವನ್ನೂ ಮರದ ಎಲೆಗಳನ್ನೂ ಧರಿಸುತ್ತಿದ್ದರು. ಬೆಟ್ಟದ ಗುಹೆಗಳಲ್ಲೊ, ಹಕ್ಕಿಯ ಗೂಡಿನಂತಹ ಜೋಪಡಿಯಲ್ಲೋ ಕಾಲ ಕಳೆಯುತ್ತಿದ್ದರು. ಇದಕ್ಕೆ ಪ್ರಮಾಣ ಎಲ್ಲಾ ದೇಶಗಳಲ್ಲಿಯೂ ಭೂಶೋಧನೆ ಮಾಡುವಾಗ ದೊರಕಿರುವುದು. ಈಗಲೂ ಕೆಲವು ಕಡೆ ಜನರು ಇನ್ನೂ ಹಾಗೆಯೇ ಜೀವಿಸುತ್ತಿರುವುದು ಕಾಣುವುದು. ಕ್ರಮೇಣ ಲೋಹದ ಉಪಯೋಗವನ್ನು ಮನುಷ್ಯನು ತಿಳಿದುಕೊಂಡನು. ಮೃದು ಲೋಹಗಳಾದ ಟಿನ್​, ತಾಮ್ರ ಮುಂತಾದವನ್ನು ಮಿಶ್ರಮಾಡಿ ಉಪಕರಣ ಮಾಡುವುದನ್ನು ಕಲಿತನು. ಪ್ರಾಚೀನ ಗ್ರೀಕರಿಗೆ, ಬ್ಯಾಬಿಲೋನಿಯಾ ನಿವಾಸಿಗಳಿಗೆ ಮತ್ತು ಈಜಿಪ್ಟಿನವರಿಗೆ ಬಹಳ ಕಾಲದವರೆಗೆ ಲೋಹದ ಉಪಯೋಗ ಗೊತ್ತಿರಲಿಲ್ಲ. ಅವರು ನಾಗರಿಕರಾಗಿ ಪುಸ್ತಕ ಬರೆದು ಚಿನ್ನ ಬೆಳ್ಳಿಯನ್ನೂ ಉಪಯೋಗಿಸುತ್ತಿದ್ದರೂ ಇತರ ಲೋಹದ ಬಳಕೆ ಗೊತ್ತಿರಲಿಲ್ಲ. ಆ ಸಮಯದಲ್ಲಿ ಅಮೆರಿಕಾದ ಆದಿನಿವಾಸಿಗಳಾದ ಮೆಕ್ಸಿಕೊ, ಪೆರು, ಮಾಯಾ ಮುಂತಾದ ನಿವಾಸಿಗಳು ಸುಮಾರು ನಾಗರಿಕರಾಗಿದ್ದರು. ದೊಡ್ಡ ದೊಡ್ಡ ಗುಡಿಗಳನ್ನು ಕಟ್ಟಿದರು. ಚಿನ್ನ ಬೆಳ್ಳಿಯನ್ನು ಸರ್ವೇ ಸಾಮಾನ್ಯವಾಗಿ ಉಪಯೋಗಿಸುತ್ತಿದ್ದರು. ಚಿನ್ನ, ಬೆಳ್ಳಿಯ ಮೇಲೆ ಸ್ಪೆಯಿನ್​ ದೇಶಕ್ಕೆ ಇದ್ದ ಆಸೆಯೆ ಅವರ ನಾಶಕ್ಕೆ ಕಾರಣವಾಯಿತು. ಅವರು ಇವನ್ನೆಲ್ಲ ಕಲ್ಲಿನ ಉಪಕರಣದಿಂದ ತಯಾರು ಮಾಡುತ್ತಿದ್ದರು. ಕಬ್ಬಿಣದ ಹೆಸರು ಕೂಡ ಅವರಿಗೆ ಗೊತ್ತಿರಲಿಲ್ಲ.

ಆದಿಯಲ್ಲಿ ಮನುಷ್ಯನು ಬಿಲ್ಲು ಬಾಣ ಅಥವಾ ಬಲೆಯಿಂದ ಮೃಗ ಪಕ್ಷಿ ಮೀನುಗಳನ್ನು ಹಿಡಿದು ಜೀವಿಸುತ್ತಿದ್ದನು. ಕ್ರಮೇಣ ಕೃಷಿ, ಗೋರಕ್ಷಣೆಗಳನ್ನು ಕಲಿತನು. ದುಷ್ಟಮೃಗಗಳನ್ನು ಪಳಗಿಸಿ ಅವುಗಳಿಂದ ಕೆಲಸ ಮಾಡಿಸಿಕೊಳ್ಳುತ್ತಿದ್ದನು. ಅಥವಾ ಅವನ್ನು ಆಹಾರಕ್ಕಾಗಿ ಉಪಯೋಗಿಸುತ್ತಿದ್ದರು. ದನ, ಕುದುರೆ, ಆನೆ, ಒಂಟೆ, ಕುರಿ, ಆಡು ಹಲವು ಬಗೆಯ ಹಕ್ಕಿ ಇವನ್ನೆಲ್ಲ ಪಳಗಿಸಿದನು. ಇವುಗಳಲ್ಲಿ ನಾಯಿ ಮನುಷ್ಯನ ಆದಿ ಸ್ನೇಹಿತ.

ಕ್ರಮೇಣ ಕೃಷಿ ಪ್ರಚಾರಕ್ಕೆ ಬಂದಿತು. ಈಗ ಮನುಷ್ಯರು ಉಪಯೋಗಿಸುವ ಹಣ್ಣು, ಗೆಡ್ಡೆ, ತರಕಾರಿ, ದ್ವಿದಳ ಧಾನ್ಯ, ಇವು ಹಿಂದೆ ಕಾಡಿನಲ್ಲಿ ಮನುಷ್ಯನ ಪೋಷಣೆಗೆ ಒಳಗಾಗದೆ ಬೆಳೆಯುತ್ತಿದ್ದಾಗ ಈಗಿನದಕ್ಕಿಂತ ಬಹಳ ಭಿನ್ನವಾಗಿದ್ದುವು. ನಂತರ ಮನುಷ್ಯನ ವ್ಯವಸಾಯದಿಂದ ಅವು ರುಚಿಕರವಾಗಿ ನೋಡಲು ಸುಂದರವಾದವು. ಪ್ರಕೃತಿಯೂ ಹಗಲು ರಾತ್ರಿ ಪರಿವರ್ತನೆ ಮಾಡುತ್ತಿರುವುದು. ಪಶುಪಕ್ಷಿಗಳ ಶರೀರ ಸಂಸರ್ಗದಿಂದ ಈಗ ಪರಿವರ್ತನೆಗೆ ಒಳಗಾಗಿ ಹಲವು ಬಗೆಯ ನವೀನ ಮೃಗ ಪಕ್ಷಿ ಗಿಡತರುಗಳು ಸೃಷ್ಟಿಯಾಗುತ್ತಿವೆ. ಮನುಷ್ಯನು ಅಸ್ತಿತ್ವಕ್ಕೆ ಬರುವುದಕ್ಕೆ ಮುನ್ನ ಪ್ರಕೃತಿಯು ಮರ, ಗಿಡ ಮತ್ತು ಪ್ರಾಣಿಗಳನ್ನು ನಿಧಾನವಾಗಿ, ಮೃದುವಾಗಿ ಬದಲಾಯಿಸುತ್ತಿತ್ತು. ಮನುಷ್ಯನು ಅಸ್ತಿತ್ವಕ್ಕೆ ಬಂದ ಮೇಲೆ ಆ ಪರಿವರ್ತನೆಯನ್ನು ವೇಗವಾಗಿ ಮಾಡಲು ಪ್ರಾರಂಭಿಸಿದನು. ಒಂದು ದೇಶದ ಗಿಡ ಮರ ಬಳ್ಳಿಗಳನ್ನು ಮತ್ತೊಂದು ದೇಶಕ್ಕೆ ಒಯ್ದು ಪರಸ್ಪರ ಮಿಶ್ರಣದಿಂದ ಹೊಸ ಜೀವ ಜಂತು ಗಿಡ ಮರ ಬಳ್ಳಿಯನ್ನು ಸೃಷ್ಟಿಸತೊಡಗಿದನು.

ಆದಿಯಲ್ಲಿ ವಿವಾಹ ಪದ್ಧತಿ ಇರಲಿಲ್ಲ. ಕ್ರಮೇಣ ವಿವಾಹಸಂಬಂಧ ಸ್ಥಾಪಿತ ವಾಯಿತು. ಹಿಂದೆ ಎಲ್ಲಾ ಸಮಾಜಗಳಲ್ಲಿಯೂ ವಿವಾಹಸಂಬಂಧ ತಾಯಿಯ ಮೇಲೆ ನಿರ್ಧಾರವಾಗುತ್ತಿತ್ತು. ತಂದೆ ಯಾರು ಎಂದು ನಿಶ್ಚಯವಾಗಿ ಹೇಳುವುದಕ್ಕೆ ಆಗುತ್ತಿರಲಿಲ್ಲ. ತಾಯಿಯ ಹೆಸರಿನ ಮೇಲೆ ಮಕ್ಕಳ ಹೆಸರನ್ನು ಇಡುತ್ತಿದ್ದರು. ಅವರ ಸಂಪತ್ತೆಲ್ಲಾ ಸ್ತ್ರೀಯರ ವಶದಲ್ಲಿತ್ತು. ಅವರು ಮಕ್ಕಳ ಲಾಲನೆಪಾಲನೆ ಮಾಡುತ್ತಿದ್ದರು. ಕ್ರಮೇಣ ಆಸ್ತಿಯು ಪುರುಷನ ವಶವಾದ ಮೇಲೆ ಸ್ತ್ರೀಯು ಅವನ ವಶವಾದಳು. ಪುರುಷನು ಹೀಗೆ ಹೇಳತೊಡಗಿದನು: “ಧನ ಧಾನ್ಯವೆಲ್ಲ ನನ್ನದು. ನಾನು ಇದನ್ನು ಬೆಳೆದಿರುವೆನು. ಇಲ್ಲವೆ ಕೊಳ್ಳೆ ಹೊಡೆದು ತಂದಿರುವೆನು. ಯಾರಾ ದರೂ ಇದರಲ್ಲಿ ಪಾಲಿಗೆ ಬಂದರೆ ಅವರೊಂದಿಗೆ ಹೋರಾಡುವೆನು. ಇದರಂತೆಯೆ ಸ್ತ್ರೀಯರೆಲ್ಲ ನನ್ನವರು. ಯಾರಾದರೂ ಇವರ ಮೇಲೆ ಕೈಮಾಡಿದರೆ ಅವರು ನನ್ನ ಶತ್ರು.” ಹೀಗೆ ವರ್ತಮಾನಕಾಲದ ವಿವಾಹ ಪದ್ಧತಿ ಜಾರಿಗೆ ಬಂದಿತು. ಧನ ಧಾನ್ಯಗಳಂತೆ ಸ್ತ್ರೀಯೂ ಪುರುಷನ ಸ್ವತ್ತಾದಳು. ಹಿಂದಿನ ಕಾಲದಲ್ಲಿ ಒಂದು ಗುಂಪಿನ ಗಂಡಸು ಮತ್ತೊಂದು ಗುಂಪಿನ ಹುಡುಗಿಯನ್ನು ಮದುವೆಮಾಡಿಕೊಳ್ಳುತ್ತಿದ್ದನು. ಆಗಲೂ ಸ್ತ್ರೀಯರನ್ನು ಬಲಾತ್ಕಾರದಿಂದ ಸಳೆದುಕೊಂಡು ಹೋಗುತ್ತಿದ್ದರು. ಕ್ರಮೇಣ ಬಲಾತ್ಕಾರದಿಂದ ತೆಗೆದುಕೊಂಡು ಹೋಗುವ ಪದ್ಧತಿ ನಿಂತುಹೋಯಿತು. ಎರಡು ಪಕ್ಷದವರ ಒಪ್ಪಿಗೆಯ ಮೇಲೆ ಮದುವೆಯಾಗುತ್ತಿತ್ತು. ಆದರೆ ಈಗಲೂ ಹಿಂದಿನ ಅವಶೇಷ ಉಳಿದುಕೊಂಡಿರುವುದು. ಮದುವೆಯ ಕಾಲದಲ್ಲಿ ವರನ ಮೇಲೆ ನಟನೆಯ ಆಕ್ರಮಣವನ್ನು ಎಲ್ಲಾ ಕಡೆಯೂ ಮಾಡುವರು. ಬಂಗಾಳ ಮತ್ತು ಯೂರೋಪ್​ ದೇಶದಲ್ಲಿ ಹಿಡಿ ಅಕ್ಕಿಯನ್ನು ವರನ ಮೇಲೆ ಚೆಲ್ಲುವರು. ಉತ್ತರ ಇಂಡಿಯಾದಲ್ಲಿ ವಧುವಿನ ಕಡೆಯವರು ವರನ ಕಡೆಯವರನ್ನು ನಿಂದಿಸುವರು.

ನಂತರ ಸಮಾಜ ಸೃಷ್ಟಿ ಮೊದಲಾಯಿತು. ದೇಶ ಭೇದಕ್ಕೆ ತಕ್ಕಂತೆ ಸಮಾಜವೂ ಭಿನ್ನರೂಪವನ್ನು ತಾಳಿತು. ಸಮುದ್ರ ತೀರದಲ್ಲಿರುವವರು ಮೀನು ಹಿಡಿದು ಜೀವಿಸ ತೊಡಗಿದರು. ಪ್ರಸ್ಥಭೂಮಿಯಲ್ಲಿರುವವರು ವ್ಯವಸಾಯದಿಂದ ಜೀವಿಸತೊಡಗಿದರು. ಪರ್ವತಗಳಲ್ಲಿ ವಾಸಿಸುತ್ತಿದ್ದವರು ಕುರಿ ಮೇಯಿಸತೊಡಗಿದರು. ಮರಳುಕಾಡಿನ ಕಡೆ ಇರುವವರು ಒಂಟೆ ಆಡು ಮುಂತಾದುವನ್ನು ಸಾಕಿ ಜೀವನೋಪಾಯವನ್ನು ಸಂಪಾದಿಸುತ್ತಿದ್ದರು. ಉಳಿದವರು ಕಾಡಿನಲ್ಲಿ ಬೇಟೆಯ ಕೆಲಸದಲ್ಲಿ ನಿರತರಾದರು. ಮೈದಾನ ಪ್ರದೇಶದಲ್ಲಿದ್ದವರು ವ್ಯವಸಾಯವನ್ನು ಕಲಿತರು. ಜೀವನೋಪಾಯಕ್ಕೆ ಅವರಿಗೆ ಕಷ್ಟವಿರಲಿಲ್ಲ. ವಿಚಾರಮಾಡುವುದಕ್ಕೆ ವಿಶೇಷ ಕಾಲ ದೊರೆತು ಇತರರಿಗಿಂತ ಹೆಚ್ಚು ನಾಗರಿಕರಾದರು. ನಾಗರಿಕರಾದಂತೆ ದುರ್ಬಲರಾಗುತ್ತಾ ಬಂದರು. ಹಗಲು ರಾತ್ರೆ ಬಯಲಿನಲ್ಲಿ ವಾಸಿಸುತ್ತ ಮಾಂಸಾಹಾರವನ್ನು ಭಕ್ಷಿಸುತ್ತಿದ್ದವರು ಕೇವಲ ಮನೆಯಲ್ಲಿ ವಾಸಮಾಡಿ ಶಾಕಾಹಾರವನ್ನು ಸೇವಿಸುತ್ತಿದ್ದವರಿಗಿಂತ ಹೆಚ್ಚು ಪರಾಕ್ರಮ ಶಾಲಿಗಳಾದರು. ಬೇಟೆಗಾರರು, ಕುರುಬರು, ಬೆಸ್ತರು ತಮ್ಮ ಆಹಾರಕ್ಕೆ ಅಭಾವ ವಾದಾಗ ಹತ್ತಿರದ ಊರಿಗೆ ಹೋಗಿ ಕೊಳ್ಳೆ ಹೊಡೆಯುತ್ತಿದ್ದರು. ಪುರನಿವಾಸಿಗಳು ಆತ್ಮ ರಕ್ಷಣೆಗಾಗಿ ಒಟ್ಟು ಕಲೆತರು. ಹೀಗೆ ಸಣ್ಣ ಸಣ್ಣ ರಾಜ್ಯಗಳು ಸೃಷ್ಟಿಯಾದವು.

ದೇವತೆಗಳು(ದೈವೀಸಂಪತ್ತಿನ ಮನುಷ್ಯರು) ಧಾನ್ಯ ತರಕಾರಿ ಮುಂತಾದುವನ್ನು ಸೇವಿಸುತ್ತಿದ್ದರು. ನಾಗರಿಕರಾಗಿ ಹಳ್ಳಿ, ಊರುಗಳಲ್ಲಿ ವಾಸಿಸುತ್ತಿದ್ದರು, ನೇಯ್ದ ಬಟ್ಟೆಯನ್ನು ಧರಿಸುತ್ತಿದ್ದರು. ಅಸುರರು (ಆಸುರೀಸಂಪತ್ತಿನ ಜನ) ಬೆಟ್ಟ, ಗುಡ್ಡ, ಕಾಡು, ಕಡಲ ಕರೆಗಳಲ್ಲಿ ವಾಸವಾಗಿದ್ದರು. ಕಾಡಿನ ಪ್ರಾಣಿಗಳು, ಗೆಡ್ಡೆ, ಗೆಣಸು, ಹಣ್ಣು ಜೊತೆಗೆ ಊರ ಜನರಿಗೆ ಕಾಡಿನ ದನ ಕರು ಆಡು ಕುರಿಗಳನ್ನು ಕೊಟ್ಟು ಪಡೆಯುವ ದವಸಧಾನ್ಯಗಳ ಮೇಲೆ ಜೀವಿಸುತ್ತಿದ್ದರು. ಚರ್ಮವನ್ನು ಉಡುತ್ತಿದ್ದರು. ದೇವತೆಗಳು ಶರೀರದಲ್ಲಿ ದುರ್ಬಲರಾಗಿದ್ದರು. ಕಷ್ಟ ಸಹಿಸುವುದಕ್ಕೂ ಆಗುತ್ತಿರಲಿಲ್ಲ. ಅಸುರರು ಹಲವು ವೇಳೆ ಉಪವಾಸ ಮಾಡಬೇಕಾದ ಪ್ರಮೇಯ ಬೀಳುತ್ತಿದ್ದುದರಿಂದ ಎಂತಹ ಕಷ್ಟವನ್ನಾದರೂ ಸಹಿಸಬಲ್ಲವರಾಗಿದ್ದರು.

ಅಸುರರಿಗೆ ಆಹಾರದ ಅಭಾವವಾದಾಗ ಕಾನನಗಳಿಂದ ಸಮುದ್ರದ ಕರೆಯಿಂದ ನಗರಗಳ ಮತ್ತು ಹಳ್ಳಿಗಳ ಮೇಲೆ ಧಾಳಿಯಿಡುತ್ತಿದ್ದರು. ಕೆಲವು ವೇಳೆ ಅಸುರರು ದೇವತೆಗಳಿಂದ ಐಶ್ವರ್ಯ, ಧಾನ್ಯ ಮುಂತಾದುವನ್ನು ಸುಲಿಗೆ ಮಾಡುತ್ತಿದ್ದರು. ಯಾವಾಗ ಸಾಕಾದಷ್ಟು ದೇವತೆಗಳು ಒಗ್ಗಟ್ಟಾಗಲಿಲ್ಲವೊ ಅವರು ಅಸುರರ ಕೈಯಲ್ಲಿ ಸಾಯುವುದರಲ್ಲಿ ಸಂದೇಹವಿರಲಿಲ್ಲ. ದೇವತೆಗಳು ಬುದ್ಧಿಶಾಲಿಗಳಾದುದರಿಂದ ಅಸ್ತ್ರ ಶಸ್ತ್ರಗಳನ್ನು ತಯಾರುಮಾಡುತ್ತಿದ್ದರು. ಬ್ರಹ್ಮಾಸ್ತ್ರ, ಗರುಡಾಸ್ತ್ರ, ವೈಷ್ಣವಾಸ್ತ್ರ, ಶೈವಾಸ್ತ್ರ, ಇವೆಲ್ಲ ದೇವತೆಗಳ ಅಸ್ತ್ರಗಳು. ಅಸುರರಲ್ಲಿ ಇದ್ದ ಶಸ್ತ್ರ ಸಾಧಾರಣ ವಾಗಿತ್ತು. ಆದರೆ ದೇಹದಲ್ಲಿ ಅಸಾಧ್ಯ ಬಲವಿತ್ತು. ಅನೇಕ ವೇಳೆ ಅಸುರರು ದೇವತೆ ಗಳನ್ನು ಸೋಲಿಸಿದರೂ ಅವರಂತೆ ಸಭ್ಯರಾಗಲು ಇಚ್ಚಿಸಲಿಲ್ಲ. ಕೃಷಿಯನ್ನೂ ಕಲಿಯ ಲಿಲ್ಲ. ಬುದ್ಧಿಯನ್ನೂ ರೂಢಿಸಲಿಲ್ಲ. ವಿಜಯಿಗಳಾದ ಅಸುರರು ದೇವತೆಗಳ ಸ್ವರ್ಗ ದಲ್ಲಿ ರಾಜ್ಯವಾಳುತ್ತಿದಾಗ, ದೇವತೆಗಳು ತಮ್ಮ ಬುದ್ಧಿ ಕೌಶಲ್ಯದಿಂದ ಬಹಳ ಬೇಗ ಅವರನ್ನು ತಮ್ಮ ಗುಲಾಮರನ್ನಾಗಿ ಮಾಡಿಕೊಳ್ಳುತ್ತಿದ್ದರು. ಕೆಲವು ವೇಳೆ ಅಸುರರು ಸುಲಿಗೆ ಆದಮೇಲೆ ಹಳೆಯ ನಿವಾಸಕ್ಕೆ ಹಿಂತಿರುಗುತ್ತಿದ್ದರು. ದೇವತೆ ಗಳೆಲ್ಲ ಐಕ್ಯಮತ್ಯದಿಂದ ಕೂಡಿದಾಗ ಅಸುರರನ್ನು ಸೋಲಿಸಿ ಬೆಟ್ಟಕ್ಕೊ, ಕಾಡಿಗೊ, ಸಮುದ್ರದ ಕಡೆಗೊ ಓಡಿಸುತ್ತಿದ್ದರು. ಕ್ರಮೇಣ ಪ್ರತಿಯೊಂದು ಪಕ್ಷಕ್ಕೂ ಜನ ಸೇರಿ ಹೆಚ್ಚಿತು. ಲಕ್ಷಾಂತರ ದೇವಾಸುರರು ತಮ್ಮ ತಮ್ಮಲ್ಲಿ ಐಕ್ಯಮತ್ಯರಾದರು. ಭಯಂಕರ ಮನಸ್ತಾಪ ಮತ್ತು ಯುದ್ಧ ಮೊದಲಾಯಿತು. ಇದರ ಜತೆಗೆ ಎರಡೂ ಜನಾಂಗಗಳು ಪರಸ್ಪರ ಮಿಶ್ರಣಗೊಂಡವು.

ಈ ಎರಡೂ ಜನಾಂಗಗಳ ಸಂಪರ್ಕದಿಂದ ಇಂದಿನ ನಮ್ಮ ಸಮಾಜ, ರೀತಿ, ನೀತಿ ಬೆಳೆದುವು. ಹೊಸ ಭಾವನೆಗಳುದಿಸಿದವು, ಹೊಸ ವಿಜ್ಞಾನಗಳು ಬೆಳೆದುವು. ಒಂದು ಪಂಗಡವು ಜನತೆಗೆ ಆವಶ್ಯಕವಾದ ಮತ್ತು ಸುಖಕರವಾದ ಪದಾರ್ಥಗಳನ್ನು ಬುದ್ಧಿವಂತಿಕೆಯಿಂದಲೂ ಶ್ರಮದಿಂದಲೂ ತಯಾರು ಮಾಡಲು ಪ್ರಾರಂಭಿಸಿತು. ಎರಡನೆ ಪಂಗಡವು ಅವರ ರಕ್ಷಣೆಗೆ ನಿಂತಿತು. ಬುದ್ಧಿವಂತರಾದ ಕೆಲವರು ಒಂದು ಪದಾರ್ಥವನ್ನು ಒಂದು ಕಡೆಯಿಂದ ಮತ್ತೊಂದು ಕಡೆಗೆ ಸಾಗಿಸುವ ಕೆಲಸವನ್ನು ತೆಗೆದುಕೊಂಡರು; ಅವರ ತೊಂದರೆಗೆ ತಕ್ಕ ಪ್ರತಿಫಲವಾಗಿ ಬೆಲೆಯ ಬಹುಭಾಗವನ್ನು ಲಾಭದಂತೆ ತೆಗೆದುಕೊಂಡರು. ಒಬ್ಬ ನೆಲ ಉಳುವನು, ಎರಡನೆಯವರು ಬೆಳೆಯನ್ನು ರಕ್ಷಿಸುವನು, ಮೂರನೆಯವನು ಮಾರುವನು. ನಾಲ್ಕನೆಯವನು ಕೊಳ್ಳುವನು. ಉಳುವವನಿಗೆ ಸಿಕ್ಕುತ್ತಿದ್ದುದು ಬಹಳ ಕಡಿಮೆ. ಧಾನ್ಯವನ್ನು ಕಾಯುವವನು ಬಲಾತ್ಕಾರದಿಂದ ತಾನು ಎಷ್ಟು ಸುಲಿಗೆ ಮಾಡುವುದಕ್ಕೆ ಸಾಧ್ಯವೊ ಅಷ್ಟನ್ನೂ ಕಸಿದುಕೊಳ್ಳುತ್ತಿದ್ದನು. ಧಾನ್ಯವನ್ನು ಮಾರುವ ಉದ್ದೇಶದಿಂದ ಅದನ್ನು ಕೊಳ್ಳುತ್ತಿದ್ದ ವರ್ತಕನಿಗೆ ಲಾಭದ ಸಿಂಹಪಾಲು ಸಿಕ್ಕುತ್ತಿತ್ತು. ಕೊನೆಗೆ ಕೊಳ್ಳುವವನು ಈ ದಂಡವನ್ನೆಲ್ಲಾ ತೆತ್ತು ಈ ಅನ್ಯಾಯದ ಭಾರದಲ್ಲಿ ನರಳಬೇಕಾಗಿತ್ತು. ರಕ್ಷಿಸುವವನು ರಾಜನಾದ. ಒಂದು ಸ್ಥಳದಿಂದ ಮತ್ತೊಂದು ಸ್ಥಳಕ್ಕೆ ಸಾಮಾನನ್ನು ಸಾಗಿಸುವವನು ವರ್ತಕನಾದ. ಇವರಿಬ್ಬರೂ ಏನನ್ನೂ ಬೆಳೆಯಲಿಲ್ಲ. ಆದರೂ ಮುಖ್ಯ ಭಾಗ್ಯ ವನ್ನೆಲ್ಲಾ ಸೆಳೆಯುತ್ತಿದ್ದರು. ರೈತನ ದುಡಿತದಿಂದ ಇವರ ಸಂಪತ್ತನ್ನು ಗಳಿಸಿದರು. ಇದನ್ನೆಲ್ಲಾ ಬೆಳೆಸುತ್ತಿದ್ದ ರೈತ ಕೊನೆಗೆ ಕೆಲವು ವೇಳೆ ತಿನ್ನುವುದಕ್ಕೆ ಏನೂ ಇಲ್ಲದೆ ಉಪವಾಸದಿಂದ ದೇವರನ್ನು ಪ್ರಾರ್ಥಿಸಬೇಕಾಗಿತ್ತು.

ಕ್ರಮೇಣ ಸಮಸ್ಯೆ ಜಟಿಲವಾಗುತ್ತ ಬಂದಿತು. ನಮ್ಮ ವರ್ತಮಾನ ಸಮಾಜ ಆದರಿಂದ ಉತ್ಪನ್ನವಾಗಿದೆ. ಪೂರ್ವಚಿಹ್ನೆಗಳು ಒಂದೇ ಸಲ ನಾಶವಾಗುವುದಿಲ್ಲ. ಯಾರು ಹಿಂದೆ ಕುರುಬರು, ಬೆಸ್ತರು ಆಗಿದ್ದರೊ ಅವರು ದರೋಡೆಕಾರರು, ಕಡಲು ಗಳ್ಳರು ಆದರು. ಬೇಟೆಯಾಡುವುದಕ್ಕೆ ಕಾಡಿಲ್ಲ. ಕುರಿ ಮೇಯಿಸುವುದಕ್ಕೆ ತಪ್ಪಲಿಲ್ಲ. ಆಕಸ್ಮಾತ್​ ಸಭ್ಯ ಸಮಾಜದಲ್ಲಿ ಜನ್ಮ ಧಾರಣೆ ಮಾಡಿದುದರಿಂದ ಪೂರ್ವದಂತೆ ಜೀವನ ನಡೆಸಲು ಅವಕಾಶವಿರಲಿಲ್ಲ. ಹಿಂದಿನ ಸಂಸ್ಕಾರಗಳಿಂದ ಪ್ರೇರಿತರಾಗಿ ಲೂಟಿ, ದರೋಡೆಯ ವೃತ್ತಿಯನ್ನು ಅನುಸರಿಸಿದರು. ಪೂಜ್ಯ ಪ್ರಾತಃಸ್ಮರಣೀಯ ಸ್ತ್ರೀಯರಾದ ಅಹಲ್ಯಾ, ತಾರಾ, ಮಂಡೋದರಿ, ಕುಂತಿ, ದ್ರೌಪದಿ ಇವರ ವಂಶಜರು ಒಂದೇ ಸಾರಿ ಒಬ್ಬ ಗಂಡನಿಗಿಂತ ಹೆಚ್ಚು ಮದುವೆಯಾಗಲು ಅವಕಾಶವಿಲ್ಲದ ಕಾರಣ ಪಾತಿವ್ರತ್ಯದಿಂದ ಜಾರಿದರು. ಭಿನ್ನ ಭಿನ್ನ ಭಾವ ಸಂಸ್ಕಾರ, ಸಭ್ಯ ಅಸಭ್ಯ, ದೇವ ಅಸುರರು, ಎಲ್ಲಾ ಮಿಶ್ರವಾಗಿ ಆಧುನಿಕ ಸಮಾಜ ಉತ್ಪನ್ನವಾಯಿತು. ಆದ ಕಾರಣವೆ ಪ್ರತಿಯೊಂದು ಸಮಾಜಲ್ಲಿಯೂ ಸಾಧು ನಾರಾಯಣ, ಚೋರ ನಾರಾಯಣರನ್ನು ನೋಡುತ್ತೇವೆ. ಯಾವ ಸಮಾಜದಲ್ಲಿ ದೈವೀ, ಯಾವ ಸಮಾಜ ಅಸುರೀ ಎಂಬುದನ್ನು ಅಲ್ಲಿಯ ಬಹು ಜನರು ಯಾವ ಸ್ವಭಾವದವರು ಎನ್ನುವುದರ ಮೇಲೆ ನಿರ್ಣಯಿಸುತ್ತಿದ್ದರು.

ಏಷ್ಯದ ಸಂಸ್ಕೃತಿ ಗಂಗೆ, ಯಂಗ್​ಟ್ಸಕಿಯಾಂಗ್​ ಮತ್ತು ಯೂಫ್ರಟಿಸ್​ ನದೀ ತೀರದ ಫಲವತ್ತಾದ ಭೂಭಾಗದಲ್ಲಿ ಮೊದಲಾಯಿತು. ಈ ಸಂಸ್ಕೃತಿಯ ತಳಹದಿ ವ್ಯವಸಾಯ. ಇವರಲ್ಲಿ ದೈವೀ ಪ್ರಕೃತಿ ಪ್ರಧಾನ. ಯೂರೋಪಿನ ಸಂಸ್ಕೃತಿ ಜನ್ಮ ಧಾರಣೆಯಾದುದು ಕಡಲತೀರ, ಇಲ್ಲವೆ ಗುಡ್ಡಗಾಡುಗಳಲ್ಲಿ. ಅಲ್ಲಿ ಅಸುರೀ ಪ್ರಕೃತಿ ಪ್ರಧಾನ. ಲೂಟಿ ದರೋಡೆಯೇ ಅವರ ಸಂಸ್ಕೃತಿಯ ತಳಹದಿ.

ಈಗಿನಕಾಲದ ಅಭಿಪ್ರಾಯದ ಪ್ರಕಾರ ಮಧ್ಯ ಏಷ್ಯದ ಮಧ್ಯಭಾಗ ಮತ್ತು ಅರಬ್ಬೀ ಮರುಭೂಮಿ ಅಸುರರ ಮೂಲಸ್ಥಾನ. ಅಲ್ಲಿಂದ ಅಸುರರ ದಳ ಕ್ರಮೇಣ ಹರಡುತ್ತ ಬಂದು ದೇವತೆಗಳನ್ನು ಪ್ರಪಂಚದ ನಾನಾ ಭಾಗಗಳಿಗೆ ತಳ್ಳಿದರು.

ಯೂರೋಪ್​ ಖಂಡದಲ್ಲಿ ಒಂದು ಆದಿವಾಸಿಗಳ ತಂಡವಿತ್ತು. ಪರ್ವತ ಗುಹೆ ಗಳಲ್ಲಿ ಅವರು ವಾಸಮಾಡುತ್ತಿದ್ದರು. ಇವರಲ್ಲಿ ಕೆಲವರು ಬುದ್ಧಿವಂತರು. ಆಳ ವಿಲ್ಲದ ನೀರಿನ ಮೇಲೆ ತೊಲೆ ನೆಟ್ಟು ಅದರ ಮೇಲೆ ಮನೆ ಕಟ್ಟಿಕೊಂಡು ವಾಸಿಸು ತ್ತಿದ್ದರು. ಅವರು ಕಲ್ಲಿನಿಂದ ಮಾಡಿದ ಆಯುಧಗಳನ್ನು ಉಪಯೋಗಿಸುತ್ತಿದ್ದರು.

ಕ್ರಮೇಣ ಏಷ್ಯಾಖಂಡದ ಜನರು ಪ್ರವಾಹವು ಯೂರೋಪಿನ ಮೇಲೆ ಧಾಳಿ ಇಟ್ಟಿತು. ಇದರ ಪರಿಣಾಮವಾಗಿ ಕೆಲವು ಸ್ಥಳಗಳು ತಕ್ಕಮಟ್ಟಿನ ನಾಗರಿಕತೆಯನ್ನು ಪಡೆದುಕೊಂಡುವು. ರಷ್ಯಾ ದೇಶದ ಕೆಲವು ಕಡೆಗಳ ಭಾಷೆಯು ದ್ರಾವಿಡ ಭಾಷೆಗಳನ್ನು ಹೋಲುತ್ತದೆ. ಉಳಿದವರೆಲ್ಲ ಅನೇಕ ಕಾಲದವರೆಗೆ ಬರ್ಬರರಂತೆಯೇ ಇದ್ದರು. ಏಷ್ಯಾ ಮೈನರಿನಿಂದ ಒಂದು ಸುಸಂಸ್ಕೃತ ಜನಾಂಗವು ಬಂದು ಯೂರೋಪಿನ ತೀರದ ದ್ವೀಪದಲ್ಲಿ ನೆಲಸಿದ್ದು ಮುಂದಿನ ಉತ್ತಮ ಸಂಸ್ಕೃತಿಗೆ ನಾಂದಿಯಾಯಿತು. ನಾವು ಅವರನ್ನು ಯವನರೆಂದೂ ಯೂರೋಪಿನವರು ಅವರನ್ನು ಗ್ರೀಕರೆಂದೂ ಕರೆಯುವರು.

ನಂತರ ಇಟಲಿಯಲ್ಲಿದ್ದ ರೋಮನ್​ ಬರ್ಬರ ಜನಾಂಗವು ಸಭ್ಯ ಯುಟ್ರಸ್ಕನ್ನ್​ ರನ್ನು ಗೆದ್ದು ಅವರ ವಿದ್ಯಾ ಬುದ್ಧಿಯನ್ನೆಲ್ಲ ಹೀರಿಕೊಂಡು ತಾವೂ ಸುಸಂಸ್ಕೃತ ರಾದರು. ಕ್ರಮೇಣ ನಾಲ್ಕು ದಿಕ್ಕಿನಲ್ಲಿಯೂ ರೋಮನ್ನರು ಪ್ರಬಲರಾದರು. ಯೂರೋಪಿನ ದಕ್ಷಿಣ ಮತ್ತು ಪೂರ್ವ ಭಾಗವನ್ನೆಲ್ಲ ತಮ್ಮ ಆಳ್ವಿಕೆಗೆ ಒಳಪಡಿಸಿ ದರು. ಉತ್ತರ ಭಾಗದ ಕಾಡುಗಳಲ್ಲಿ ವಾಸಿಸುತ್ತಿದ್ದ ಬರ್ಬರರು ಮಾತ್ರ ಸ್ವತಂತ್ರ ರಾಗಿದ್ದರು. ಕಾಲಕ್ರಮೇಣ ರೋಮನ್ನರು ಐಶ್ವರ್ಯ ಮದೋನ್ಮತ್ತರಾಗಿ ದುರ್ಬಲ ರಾದರು. ಆ ಸಮಯದಲ್ಲಿ ಏಷ್ಯಾ ಖಂಡದ ಅಸುರ ಸೇನಾದಳ ಯೂರೋಪಿಗೆ ಧಾಳಿ ಇಟ್ಟಿತು. ಈ ಅಸುರರ ಧಾಳಿಗೆ ಹಿಮ್ಮಟ್ಟಿ ಉತ್ತರ ಯೂರೋಪಿನ ಬರ್ಬರರು ಸ್ಥಾನಚ್ಯುತರಾಗಿ ದಕ್ಷಿಣದಲ್ಲಿದ್ದ ರೋಮನ್​ ಚಕ್ರಾಧಿಪತ್ಯದ ಮೇಲೆ ಬಿದ್ದು ಅದನ್ನು ನಾಶಮಾಡಿದರು. ಏಷ್ಯದ ದಳವನ್ನು ಎದುರಿಸಲು ಯೂರೋಪಿನ ಬರ್ಬರರು ರೋಮ್​, ಗ್ರೀಕ್​ ಜನರೊಂದಿಗೆ ಮಿಶ್ರವಾಗಿ ಹೊಸದೊಂದು ಜನಾಂಗವಾಯಿತು. ಆ ಸಮಯದಲ್ಲಿ ರೋಮನ್ನರಿಂದ ಸೋಲಿಸಲ್ಪಟ್ಟ ಯಹೂದ್ಯರು ತಮ್ಮ ಧರ್ಮ ಸಮೇತ ಯೂರೋಪಿನ ಹಲವು ಭಾಗಗಳಲ್ಲಿ ಚದುರಿಹೋದರು. ಇದರಿಂದಾಗಿ ಹೊಸ ಕ್ರೈಸ್ತಧರ್ಮವೂ ಹರಡಿತು. ಈ ವಿಭಿನ್ನ ಜನಾಂಗಗಳು, ಅವರ ಧರ್ಮಗಳು, ವಿಚಾರಗಳು, ಈ ಎಲ್ಲ ಅಸುರರ ದಳಗಳು ನಿರಂತರ ಹೋರಾಟ ಮತ್ತು ಯುದ್ಧಗಳ ಬೆಂಕಿಯಲ್ಲಿ ಬೆಂದುವು. ಮಹಾಮಾಯೆಯ ಮೂಸೆಯಲ್ಲಿ ಅವು ಕರಗಿ, ವಿಶ್ರಣಗೊಂಡು ಆಧುನಿಕ ಯೂರೋಪ್​ ಜನಾಂಗವು ಸೃಷ್ಟಿಯಾಯಿತು.

ಅತ್ಯಂತ ಅನಾಗರಿಕ ಬರ್ಬರ ಜನಾಂಗವೊಂದು ಯೂರೋಪಿನಲ್ಲಿ ಅಸ್ತಿತ್ವಕ್ಕೆ ಬಂತು. ಹಿಂದುಗಳ ಕಪ್ಪು ಬಣ್ಣದಿಂದ ಹಿಡಿದು, ಉತ್ತರದ ಹಾಲಿನಂತಹ ಬಿಳಿಯ ಬಣ್ಣದ ಜನರು, ಕಪ್ಪು, ಕಂದು, ಕೆಂಪು, ಬಿಳಿ ಕೂದಲಿನವರು, ಕಪ್ಪು, ಬೂದು, ನೀಲಿ ಬಣ್ಣಗಳ ಕಣ್ಣಿನವರು, ಹಿಂದುಗಳ ಸೊಗಸಾದ ಮುಖ, ಮೂಗು ಕಣ್ಣುಗಳನ್ನು ಳ್ಳವರು, ಅಥವಾ ಚೀಣೀಯರಂತೆ ಚಪ್ಪಟೆ ಮುಖವುಳ್ಳವರು- ಈ ಎಲ್ಲರೂ ಆ ಬರ್ಬರರಲ್ಲಿ ಇದ್ದರು. ಅವರು ಕೆಲವು ಕಾಲ ತಮ್ಮೊಳಗೇ ಹೋರಾಡುತ್ತಿದ್ದರು. ಕಡಲ್ಗಳ್ಳಂತೆ ಜೀವಿಸುತ್ತಿದ್ದ ಉತ್ತರದವರು ತಕ್ಕಮಟ್ಟಿಗೆ ಸುಸಂಸ್ಕೃತರಾದ ಜನಾಂಗದವರನ್ನು ಹಿಂಸಿಸಿ ಕೊಲ್ಲುತ್ತಿದ್ದರು. ಈ ನಡುವೆ ಈಸಾಯಿ ಧರ್ಮದ ಇಬ್ಬರು ಗುರುಗಳು, ಪೋಪ್​ ಮತ್ತು ಕಾನ್​ಸ್ಟಾಂಟಿನೋಪಲ್ಲಿನ ಪಾಟ್ರಿಯಾರ್ಕ್​ ಇವರು ಯೂರೋಪಿನ ಪಶುಪ್ರಾಯ ಬರ್ಬರ ಜನಾಂಗ, ಅವರ ರಾಜರಾಣಿಯರು ಇವರ ಮೇಲೆ ತಮ್ಮ ಅಧಿಕಾರವನ್ನು ಸ್ಥಾಪಿಸಲು ಪ್ರಾರಂಭಿಸಿದರು.

ಈ ಸಮಯದಲ್ಲಿ ಅರಬ್ಬೀ ದೇಶದ ಮುಸಲ್ಮಾನ ಧರ್ಮ ಹುಟ್ಟಿತು. ಅಲ್ಲಿನ ಜನ ಒಬ್ಬ ಮಹಾಪುರುಷನ ಪ್ರೇರಣೆಯಿಂದ, ಮರಳುಗಾಡಿನಲ್ಲಿ ಅದಮ್ಯ ಸಾಹಸದಿಂದ ಪೃಥ್ವಿಯ ಮೇಲೆ ಧಾಳಿ ಮಾಡಿದರು. ಪಶ್ಚಿಮ ಮತ್ತು ಪೂರ್ವ ದಿಕ್ಕುಗಳಿಂದ ಈ ತರಂಗ ಯೂರೋಪನ್ನು ಪ್ರವೇಶಿಸಿತು. ಈ ಪ್ರವಾಹದಲ್ಲಿ ಭಾರತ ಮತ್ತು ಪ್ರಾಚೀನ ಗ್ರೀಕರ ವಿದ್ಯೆ, ಸಂಸ್ಕೃತಿಗಳು ಯುರೋಪನ್ನು ಪ್ರವೇಶಿಸಿ ದುವು.

ಏಷ್ಯಾದ ಮಧ್ಯಭಾಗದಲ್ಲಿರುವ ಸೆಲ್​ಜಕ್​ ಟಾರ್ಟರ್​ ಎಂಬ ಅಸುರದಳವು ಇಸ್ಲಾಮ್​ ಧರ್ಮವನ್ನು ಸ್ವೀಕರಿಸಿತು. ಏಷ್ಯಾ ಮೈನರ್​ ಮುಂತಾದ ಸ್ಥಳವನ್ನು ಅವರು ಕ್ರಮೇಣ ಗೆದ್ದರು. ಭರತಖಂಡವನ್ನು ಜಯಿಸಬೇಕೆಂಬ ಅರಬ್ಬರ ಹಲವು ಪ್ರಯತ್ನಗಳು ವಿಫಲವಾದುವು. ಇಡೀ ಪ್ರಪಂಚವನ್ನು ನುಂಗಿದ ಇಸ್ಲಾಂ, ಭರತಖಂಡದ ಎದುರು ಕುಂಠಿತವಾಯಿತು. ಒಮ್ಮೆ ಅವರು ಸಿಂಧೂ ದೇಶವನ್ನು ಆಕ್ರಮಿಸಿದರು. ಆದರೆ ಅದನ್ನು ಇಟ್ಟುಕೊಳ್ಳಲು ಆಗಲಿಲ್ಲ. ಇದರ ನಂತರ ಅವರು ಪ್ರಯತ್ನ ಪಡಲಿಲ್ಲ.

ಕೆಲವು ಶತಮಾನಗಳ ಮೇಲೆ ಟರ್ಕರು ಬೌದ್ಧಧರ್ಮವನ್ನು ತೊರೆದು ಮಹಮ್ಮ ದೀಯರಾದ ಮೇಲೆ ಮಾತ್ರ ಅವರು ಪಾರಸಿ, ಹಿಂದೂ ಎಲ್ಲರನ್ನೂ ದಾಸರನ್ನಾಗಿ ಮಾಡಿದರು. ಭಾರತವರ್ಷವನ್ನು ಗೆದ್ದ ಮುಸಲ್ಮಾನ್​ ವಿಜಯಗಳಲ್ಲಿ ಯಾರೂ ಅರಬ್ಬರಲ್ಲ, ಪರ್ಷಿಯನರೂ ಅಲ್ಲ. ಅವರೆಲ್ಲ ತುರ್ಕಿ ಅಥವಾ ಟಾರ್ಟರ್​ ಜನಾಂಗ ದವರು.ರಜಪುತಾನದಲ್ಲಿ ಮಹಮ್ಮದೀಯ ಧಾಳಿಕಾರರನ್ನೆಲ್ಲ ತುರುಕರೆಂದು ಕರೆಯುತ್ತಿದ್ದರು. ಇದು ಐತಿಹಾಸಿಕ ಸತ್ಯ. ಈಗಲೂ ರಜಪುತಾನದಲ್ಲಿ “ತುರುಕರು ಬಹಳ ಜೋರಿನವರು” ಎಂದು ಚಾರಣರೆಂಬುವರು ಹಾಡುತ್ತಾರೆ. ಕುತುಬುದ್ದೀನನ ನಂತರ ಮೊಗಲರವರೆಗಿನ ಚಕ್ರವರ್ತಿಗಳೆಲ್ಲ ಟಾರ್ಟರರು. ಟಿಬೆಟ್ಟಿನವರು ಯಾವ ಜನಾಂಗಕ್ಕೆ ಸೇರಿದವರೋ, ಅದೇ ಜನಾಂಗದವರು ಅವರು. ಆದರೆ ಅವರು ಬೌದ್ಧ ಧರ್ಮ ಬಿಟ್ಟು ಮಹಮ್ಮದೀಯರಾಗಿರುವರು, ಮತ್ತು ಹಿಂದೂ ಮತ್ತು ಪಾರ್ಸಿ ಜನಾಂಗಗಳೊಂದಿಗೆ ಮದುವೆಯಾಗಿ ಅವರ ಚಪ್ಪಟೆ ಮೂಗು ಬದಲಾಯಿಸಿದೆ. ಇವರೇ ಹಿಂದಿನ ಕಾಲದ ಅಸುರ ಜನಾಂಗ. ಇಂದಿಗೂ ಅವರು ಕಾಬೂಲ್​, ಪರ್ಷಿಯಾ, ಅರೇಬಿಯಾ, ಕಾನ್ಸ್​ಟ್ಯಾಂಟಿಲೋಪಲ್​, ಕಾಂದಹಾರ್​ ದೇಶಗಳ ಸಿಂಹಾಸನಗಳ ಮೇಲೆ ಇರುವರು. ಪರ್ಷಿಯನ್ನರು ಈಗಲೂ ತುರ್ಕರ ಗುಲಾಮರು. ವಿಶಾಲ ಚೈನಾ ಚಕ್ರಾಧಿಪತ್ಯ ಕೂಡ ಮಂಚೂರಿಯದ ಟಾರ್ಟರರ ವಶದಲ್ಲಿರುವುದು. ಈ ಮಂಚೂಗಳು ತಮ್ಮ ಬೌದ್ಧಧರ್ಮವನ್ನು ತ್ಯಜಿಸಿಲ್ಲ, ಲಾಮಾನ ಶಿಷ್ಯರಾಗಿರು ವರು. ಈ ಅಸುರರಿಗೆ ವಿದ್ಯಾರ್ಜನೆ, ಬುದ್ಧಿಯ ಬೆಳವಣಿಗೆ ಇವುಗಳಲ್ಲಿ ಆಸಕ್ತಿಯೇ ಇಲ್ಲ.ಅವರಿಗೆ ಹೋರಾಟವೊಂದೇ ಗೊತ್ತಾಗುವುದು. ಈ ರಕ್ತದ ಸಂಮಿಶ್ರಣವಿಲ್ಲದೆ ಕಾದಾಟಕ್ಕೆ ಸ್ಪೂರ್ತಿ ಬರಲಾರದು. ಉತ್ತರ ಯೂರೋಪಿನಲ್ಲಿ, ಅದರಲ್ಲೂ ರಷ್ಯಾ ದಲ್ಲಿ ನಾವು ಈ ಹೋರಾಟದ ಸ್ವಭಾವವನ್ನು ನೋಡುತ್ತೇವೆ. ಅವರ ನಾಡಿಯಲ್ಲಿ ಮುಕ್ಕಾಲು ಪಾಲು ಟಾರ್ಟರ ರಕ್ತವಿದೆ. ದೇವಾಸುರರ ಯುದ್ಧ ಇನ್ನೂ ಬಹಳ ಕಾಲ ಮುಂದುವರಿಯುವುದು.ದೇವತೆಗಳು ಅಸುರರ ಕನ್ಯೆಯನ್ನು ವಿವಾಹವಾಗು ವರು. ಅಸುರರು ದೇವಕನ್ಯೆಯರನ್ನು ಹೊತ್ತುಕೊಂಡು ಹೋಗುವರು. ಈ ಪ್ರಕಾರ ಪ್ರಬಲವಾದ ವರ್ಣಸಂಕರ ನಡೆಯುವುದು.

ಟಾರ್ಟರರು ಅರಬ್ಬೀ ಕಲೀಫನ ಸಿಂಹಾಸನವನ್ನು ಅಕ್ರಮಣ ಮಾಡಿದರು. ಈಸಾಯಿಧರ್ಮದ ತೀರ್ಥಸ್ಥಳವಾದ ಜೆರೂಸಲೆಮ್ಮನ್ನು ಸ್ವಾಧೀನಪಡಿಸಿಕೊಂಡು ಕ್ರೈಸ್ತರು ಅಲ್ಲಿಗೆ ಬಾರದಂತೆ ಆಜ್ಞೆಮಾಡಿ, ಹಲವು ಕ್ರೈಸ್ತರನ್ನು ಕೊಂದರು. ಕ್ರೈಸ್ತ ಚರ್ಚುಗಳ ಮುಖ್ಯಸ್ಥರು ಕೋಪದಿಂದ ಹುಚ್ಚೆದ್ದು ತಮ್ಮ ಅನಾಗರಿಕ ಅನುಯಾಯಿ ಗಳನ್ನು ಉದ್ರೇಕಗೊಳಿಸಿದರು. ಅವರು, ತಮ್ಮ ಸರದಿಯಲ್ಲಿ ರಾಜರುಗಳನ್ನೂ, ಅವರ ಪ್ರಜೆಗಳನ್ನೂ ಮಹಮ್ಮದೀಯರ ಮೇಲೆ ಸೇಡು ತೀರಿಸಿಕೊಳ್ಳುವಂತೆ ಉದ್ರೇಕಿಸಿದರು. ದಳವಾದ ಮೇಲೆ ದಳ ಯೂರೋಪಿನ ಬರ್ಬರ ಸೇನೆ, ಧರ್ಮ ದ್ರೋಹಿಗಳ ಹತೋಟಿಯಿಂದ ಜೆರೂಸಲೆಮ್ಮನ್ನು ಬಿಡುಗಡೆಗೊಳಿಸಲು ಏಷ್ಯಾ ಮೈನರಿಗೆ ಧಾಳಿ ಇಟ್ಟಿತು. ಹಲವರು ಯುದ್ಧದಲ್ಲಿ ಸತ್ತರು. ಕೆಲವರು ಕಾಯಿಲೆಯಲ್ಲಿ ಸತ್ತರು. ಉಳಿದವರು ಮಹಮ್ಮದೀಯರ ಖಡ್ಗಕ್ಕೆ ಆಹುತಿಯಾದರು. ಒಂದು ಬರ್ಬರ ಸೇನೆ ನಾಶವಾಯಿತೆಂದರೆ ಮತ್ತೊಂದು ಸೇನೆ ಕಾಡುಮೃಗದ ಛಲದಿಂದ ಬರುತ್ತಿತ್ತು. ಬರ್ಬರ ಸೇನೆ ತಮ್ಮವರನ್ನೇ ಲೂಟಿಮಾಡುತ್ತಿದ್ದರು. ಏನೂ ಆಹಾರ ಸಿಕ್ಕದೇ ಇರುವಾಗ ಮಹಮ್ಮದೀಯರನ್ನೇ ಕೊಂದು ತಿನ್ನುತ್ತಿದ್ದರು. ಈ ವಿಷಯ ಈಗಲೂ ಪ್ರಸಿದ್ಧವಾಗಿದೆ: ಆಂಗ್ಲೇಯರ ದೊರೆಯಾದ ರಿಚರ್ಡನಿಗೆ ಮುಸಲ್ಮಾನ ಜನರ ಮಾಂಸ ಬಹಳ ಇಷ್ಟ.

ಕಾಡುಜನರಿಗೂ ನಾಗರಿಕರಿಗೂ ಯುದ್ಧವಾದರೆ ಏನು ಆಗಬೇಕೊ ಅದು ಆಯಿತು. ಯೂರೋಪಿಯನ್ನರಿಗೆ ಜೆರೂಸಲೆಮ್ಮಿನ ಮೇಲಿನ ಅಧಿಕಾರ ಸಿಕ್ಕಲಿಲ್ಲ. ಆದರೆ ಯೂರೋಪು ನಾಗರೀಕವಾಗುವುದಕ್ಕೆ ಅವಕಾಶವಾಯಿತು. ಚರ್ಮ ಹೊತ್ತು, ಹಸಿ ಮಾಂಸ ತಿನ್ನುತ್ತಿದ್ದ ಅನಾಗರಿಕ ಆಂಗ್ಲೇಯ, ಜರ್ಮನಿ, ಪ್ರಾನ್ಸ್​ ದೇಶಗಳು ಏಷ್ಯಾದಿಂದ ನಾಗರೀಕತೆಯನ್ನು ಕಲಿತವು. ಇಟಲಿ ಮತ್ತು ಇತರ ದೇಶಗಳ ಒಂದು ಕ್ರೈಸ್ತ ಸೈನಿಕ ದಳವು, ನಮ್ಮ ನಾಗಾಗಳಂತೆ ಇರುವವರು, ಅವರು ದರ್ಶನಶಾಸ್ತ್ರವನ್ನು ಕಲಿಯಲು ಮೊದಲು ಮಾಡಿದರು. ಅವರ ಒಂದು ಪಂಗಡದದವರಾದ \enginline{Knights Templer} ಗಳು ಅದ್ವೈತವಾದಿಗಳಾದರು. ಕೊನೆಗೆ ಅವರು ಈಸಾಯಿ ಧರ್ಮವನ್ನು ನಗೆಗೀಡು ಮಾಡಿದರು. ಅವರ ಹತ್ತಿರ ಬೇಕಾದಷ್ಟು ಧನವಿತ್ತು. ಪೋಪನ ಆಜ್ಞೇಯ ಪ್ರಕಾರ ಧರ್ಮೋದ್ಧಾರದ ಹೆಸರಿನಲ್ಲಿ ಯೂರೋಪ್​ ದೊರೆಗಳು \enginline{Knights Templer} ರನ್ನು ನಾಶಮಾಡಿ, ಅವರ ಧನವನ್ನು ಸೂರೆಗೊಂಡರು.

ಬೇರೆಕಡೆ ಮೂರ್​ ಎಂಬ ಒಂದು ಮುಸಲ್ಮಾನ ದಳವು ಸ್ಪೆಯಿನ್​ ದೇಶದಲ್ಲಿ ನಾಗರಿಕ ರಾಜ್ಯವನ್ನು ಸ್ಥಾಪನೆ ಮಾಡಿತು. ಇದು ಹಲವು ಬಗೆಯ ವಿದ್ಯೆಗಳಿಗೂ ಮತ್ತು ಕಲೆಗಳಿಗೂ ಕೇಂದ್ರವಾಯಿತು. ಇಲ್ಲಿ ಯೂರೋಪಿನ ಮೊದಲನೇ ವಿಶ್ವ ವಿದ್ಯಾನಿಲಯ ಅಸ್ತಿತ್ವಕ್ಕೆ ಬಂತು. ಯೂರೋಪಿನ ಹಲವು ಕಡೆಗಳಿಂದ, ಇಟಲಿ, ಫ್ರಾನ್ಸ್​ ಮತ್ತು ಬಹುದೂರದ ಇಂಗ್ಲೆಂಡಿನಿಂದಲೂ ವಿದ್ಯಾರ್ಥಿಗಳು ಇಲ್ಲಿಗೆ ವಿದ್ಯಾಭ್ಯಾಸಕ್ಕೆ ಬರುತ್ತಿದ್ದರು. ರಾಜವಂಶಕ್ಕೆ ಸೇರಿದವರ ಮಕ್ಕಳು, ಯುದ್ಧ ವಿದ್ಯೆ, ಆಚಾರ, ಸಂಸ್ಕೃತಿ ಮುಂತಾದುವನ್ನು ಕಲಿಯಲು ಪ್ರಾರಂಭಿಸಿದರು. ಮನೆ, ಮಠ, ಸ್ಮಾರಕ ಮಂದಿರಗಳನ್ನು ಹೊಸ ರೀತಿಯಲ್ಲಿ ಕಟ್ಟಿದರು.

ಇಡೀ ಯೂರೋಪ್​ ಒಂದು ಮಹಾಸೇನೆಯ ನಿವಾಸ ಸ್ಥಾನವಾಯಿತು. ಈ ಸ್ಥಿತಿ ಈಗಲೂ ಇದೆ. ಮುಸಲ್ಮಾನರು ಒಂದು ದೇಶವನ್ನು ಗೆದ್ದರೆ ರಾಜ ದೊಡ್ಡ ಭಾಗವನ್ನು ಇಟ್ಟುಕೊಡು ಉಳಿದುದನ್ನು ತನ್ನ ಸೇನಾಪತಿಗಳಿಗೆ ಹಂಚುತ್ತಿದ್ದನು. ಇವರು ರಾಜನಿಗೆ ಯಾವ ಹಣವನ್ನೂ ಕೊಡುತ್ತಿರಲಿಲ್ಲ. ಆದರೆ ಅವನಿಗೆ ಅವಶ್ಯಕ ವಾದಾಗ ಸೇನಾಸಹಾಯ ಮಾಡಬೇಕಾಗಿತ್ತು. ಇದರಿಂದ ಅಣಿಯಾದ ಒಂದು ದೊಡ್ಡ ದಂಡನ್ನು ಯಾವಾಗಲೂ ಇಟ್ಟುಕೊಂಡಿರುವ ತಾಪತ್ರಯ ತಪ್ಪಿತು. ಬೇಕಾದಾಗ ದೊಡ್ಡದೊಂದು ದಂಡು ಬೇಗ ಸಿದ್ಧವಾಗುತ್ತಿತ್ತು. ಈ ರೀತಿ ಈಗಲೂ ರಾಜಪುತ್ರ ಸ್ಥಾನದಲ್ಲಿದೆ. ಪಾಶ್ಚಾತ್ಯ ದೇಶಕ್ಕೆ ಈ ಭಾವನೆ ಮುಸಲ್ಮಾನರಿಂದ ಬಂದಿತು. ಯೂರೋಪ್​ ದೇಶೀಯರು ಇದನ್ನು ಸ್ವೀಕರಿಸಿದರು. ಆದರೆ ಮಹಮ್ಮದೀಯರಲ್ಲಿ ರಾಜ, ಸಾಮಂತರು, ಸೇನಾನಿಗಳು ಮತ್ತು ಸೈನ್ಯ ಇರುತ್ತಿತ್ತು. ಸಾಮಾನ್ಯ ಪ್ರಜೆಗಳು ಬೇರೆ ಇದ್ದರು. ಯುದ್ಧ ಸಮಯದಲ್ಲಿ ಅವರಿಗೆ ಅಪಾಯವಿರುತ್ತಿರಲಿಲ್ಲ.ಆದರೆ ಯೂರೋಪಿನಲ್ಲಿ ರಾಜ, ಸಾಮಂತ, ಸೇನಾನಿ, ಸಿಪಾಯಿ ಎಲ್ಲರೂ ಪ್ರಜೆಗಳನ್ನು ಪೀಡಿಸಲು ಪ್ರಾರಂಭಿಸಿದರು. ಅವರನ್ನು ಗುಲಾಮರನ್ನಾಗಿ ಮಾಡಿದರು. ಪ್ರತಿ ಯೊಬ್ಬರೂ ಯಾವುದಾದರೂ ಸಾಮಂತ ರಾಜನ ಅಧೀನದಲ್ಲಿರಬೇಕಾಗಿತ್ತು. ಯಾವ ಸಮಯದಲ್ಲಿ ಆಜ್ಞೆ ಬಂದರೆ ಆಗ ಯುದ್ಧ ಮಾಡಲು ಸಿದ್ಧನಾಗಿರ ಬೇಕಾಗಿತ್ತು.

ಪಾಶ್ಚಾತ್ಯರು ಹೆಮ್ಮೆ ಪಡುವ “ಸಂಸ್ಕೃತಿಯ ಅಭ್ಯುದಯ” ಎನ್ನುವುದರ ಅರ್ಥ ವೇನು? ಜಯಪ್ರದವಾಗಿ ತಮ್ಮ ಆಸೆಯನ್ನು ಅಧರ್ಮದಿಂದಲಾದರೂ ಈಡೇರಿಸಿ ಕೊಳ್ಳುವುದು. ಕಳ್ಳತನ, ಸುಳ್ಳು, ಕೊಲೆ, ಇವನ್ನೇ ಕೆಲವು ವೇಳೆ ಧರ್ಮವೆಂದು ಸಾರುವುದು. ತನ್ನ ಜೊತೆಯಲ್ಲಿ ಹೋದ ಮಹಮ್ಮದೀಯ ಪಹರೆಯವರು ಹಸಿವಿ ನಿಂದ ಒಂದು ಚೂರು ಮಾಂಸವನ್ನು ಕದ್ದುದು ತಪ್ಪು ಎಂದು ಸ್ಟಾನ್ಲಿ ಅವರನ್ನು ಚಾವಟಿಯಿಂದ ಹೊಡೆದನು! ಇದನ್ನು ಅವರು ಸಮರ್ಥಿಸುತ್ತಾರೆ. ಇದು ಯೂರೋಪ್​ ದೇಶದ ದೊಡ್ಡದೊಂದು ಆದರ್ಶ. “ನಾನು ಇಲ್ಲಿಗೆ ಬರಬೇಕೆಂದು ಇರುವೆನು, ಬಂದು ನೆಲೆ ನಿಲ್ಲುತ್ತೇನೆ, ನೀವು ಇಲ್ಲಿಂದ ತೊಲಗಿ” ಎಂಬ ನೀತಿ ತತ್ವವನ್ನು ಅದು ಸಮರ್ಥಿಸುತ್ತದೆ. ಇದಕ್ಕೆ ಉದಾಹರಣೆ ಚರಿತ್ರೆಯಲ್ಲಿ ಬೇಕಾದಷ್ಟು ದೊರಕುವುದು. ಎಲ್ಲಿಗೆ ಯೂರೋಪಿಯನ್ನರು ಹೋಗಿರುವರೋ ಅಲ್ಲಿಯ ಆದಿ ನಿವಾಸಿಗಳನ್ನು ನಿರ್ಮೂಲ ಮಾಡಿರುವರು. ಈ ನಾಗರಿಕತೆಗೆ ಸೇರಿದವರು ಲಂಡನ್ನಿ ನಲ್ಲಿ ದಾಂಪತ್ಯ ಜೀವನದ ಭ್ರಷ್ಟತೆಯನ್ನೂ, ಫ್ರಾನ್ಸಿನಲ್ಲಿ ಹೆಂಡತಿ ಮಕ್ಕಳನ್ನು ಅನಾಥರಾಗಿ ಬಿಟ್ಟು ಗಂಡ ಆತ್ಮಹತ್ಯೆ ಮಾಡಿಕೊಳ್ಳುವುದನ್ನೂ ಒಂದು ಪಾತಕವೆಂದು ಎಣಿಸುವುದಿಲ್ಲ. ಇದನ್ನು ಒಂದು ದೋಷವೆನ್ನುತ್ತಾರೆ.

ಇಸ್ಲಾಂ ನಾಗರಿಕತೆಯು ಮೂರು ಶತಮಾನಗಳಲ್ಲಿ ವೇಗವಾಗಿ ಹೇಗೆ ಹರಡಿತು ಎಂಬುದನ್ನು ಅದೇ ಅವಧಿಯಲ್ಲಿ ಕ್ರೈಸ್ತಧರ್ಮವು ಹರಡಿದುದರೊಂದಿಗೆ ಹೋಲಿಸಿ ನೋಡಿ. ಕ್ರೈಸ್ತಧರ್ಮವು ತಾನು ಮೊದಲ ಮೂರು ಶತಮಾನಗಳಲ್ಲಿ ತನ್ನ ಅಸ್ತಿತ್ವ ವನ್ನು ಜಗತ್ತಿಗೆ ತೋರಿಸುವುದರಲ್ಲಿ ಯಶಸ್ವಿಯಾಗಲಿಲ್ಲ. ಕಾನ್​ಸ್ಟಂಟೈನನ ಖಡ್ಗವು ಅದಕ್ಕೆ ಆಸರೆಕೊಟ್ಟ ಮೇಲೆ ಮಾತ್ರ ಅದರ ಪ್ರಭಾವ ಪ್ರಾರಂಭವಾಯಿತು. ಕ್ರೈಸ್ತ ಸಂಸ್ಕೃತಿ ಆಧ್ಯಾತ್ಮಿಕ ಮತ್ತು ಲೌಕಿಕ ಜ್ಞಾನ ವಿಕಾಸಕ್ಕೆ ಯಾವ ಪ್ರೋತ್ಸಾಹವನ್ನು ಕೊಟ್ಟಿತು? ಭೂಮಿ ಸುತ್ತುತ್ತಿದೆ ಎಂದು ಸಾರಿದ ವಿದ್ವಾಂಸನಿಗೆ ಯಾವ ಬಹುಮಾನವನ್ನು ಕ್ರೈಸ್ತರು ಕೊಟ್ಟರು? ಯಾವ ವಿಜ್ಞಾನಿಯನ್ನು ಕ್ರೈಸ್ತ ಚರ್ಚಿನವರು ಪ್ರೋತ್ಸಾಹಿಸುವರು? ಕ್ರೈಸ್ತರ ಸಾಹಿತ್ಯವು ಸಿವಿಲ್​ ಅಥವಾ ಕ್ರಿಮಿನಲ್​ ನ್ಯಾಯ, ಕಲೆ, ವಾಣಿಜ್ಯ ಮುಂತಾದ ಯಾವ ಶಾಖೆಗಳಿಗೆ ನಿರಂತರವಾಗಿ ಯಾವುದೇ ನೆರವನ್ನು ನೀಡಿದೆಯೇನು? ಈಗಲೂ ಕ್ರೈಸ್ತಧರ್ಮವು ಲೌಕಿಕ ಸಾಹಿತ್ಯಕ್ಕೆ ಬೆಂಬಲ ನೀಡುವುದಿಲ್ಲ. ಆಧುನಿಕ ವಿದ್ಯೆ ಮತ್ತು ವಿಜ್ಞಾನಗಳಲ್ಲಿ ಆಳವಾದ ಪರಿಣತಿಯನ್ನು ಹೊಂದಿರುವ ಆದರೂ ಪ್ರಾಮಾಣಿಕವಾಗಿ ಕ್ರಿಶ್ಚಿಯನ್​ ಆಗಿರುವುದು ಸಾಧ್ಯವೇ? ನ್ಯೂ ಟೆಸ್ಟ್​ಮೆಂಟಿನಲ್ಲಿ ಪ್ರತ್ಯಕ್ಷ ಅಥವಾ ಅಪ್ರತ್ಯಕ್ಷ ರೂಪದಲ್ಲಿ ಯಾವ ವಿಜ್ಞಾನ ಅಥವಾ ಶಿಲ್ಪ ಕಲೆಯನ್ನೂ ಪ್ರಶಂಸಿಸಿಲ್ಲ. ಆದರೆ ಯಾವ ವಿಜ್ಞಾನವಾಗಲಿ, ಶಿಲ್ಪವಾಗಲೀ ಪ್ರತ್ಯಕ್ಷವಾಗಿ ಆಗಲೀ, ಅಪ್ರತ್ಯಕ್ಷವಾಗಿ ಆಗಲೀ, ಖುರಾನ್​ ಅಥವಾ ಹಫೀಸಿನ ಹಲವು ಷರೀಫಗಳಲ್ಲಿ ಅನುಮೋದನೆಯನ್ನು ಪಡೆಯದೆ ಇಲ್ಲ. ಯೂರೋಪಿನ ಮಹಾನ್​ ಚಿಂತಕರಾದ ವಾಲ್ಟೈರ್​ ಡಾರ್ವಿನ್​, ಬುಕನರ್​, ಫ್ಲಮಾರಿಯನ್​, ವಿಕ್ಟರ್​ ಹ್ಯೂಗೊ ಮುಂತಾದವರನ್ನು ಕ್ರೈಸ್ತ ಪ್ರವಾದಿ ಧರ್ಮದವರು ನಿಂದಿಸಿದರು. ಆದರೆ ಇಸ್ಲಾಂ ಧರ್ಮ ಅಂಥವರನ್ನು ದೇವರಲ್ಲಿ ನಂಬಿಕೆಯುಳ್ಳವರು, ಆದರೆ ಪ್ರವಾದಿ ಮಹಮ್ಮದನಲ್ಲಿ ಅವರಿಗೆ ನಂಬಿಕೆ ಇಲ್ಲವೆಂದು ಹೇಳುವರು. ಇವೆರಡು ಧರ್ಮಗಳಲ್ಲಿ ಯಾವುದು ಅಭ್ಯುದಯಕ್ಕೆ ಸಹಾಯ ಮಾಡಿತು, ಯಾವುದು ಆತಂಕ ತಂದೊಡ್ಡಿತು ಎಂಬುದನ್ನು ವಿಮರ್ಶಿಸಿ ನೋಡಬೇಕು. ಎಲ್ಲಿ ಇಸ್ಲಾಂ ಧರ್ಮ ಹೋಗಿದೆಯೋ ಅಲ್ಲಿನ ಆದಿವಾಸಿಗಳು ಇನ್ನೂ ಇರುವರು. ಅವರ ಭಾಷೆ, ಮಾತು, ಆಚಾರ ಇನ್ನೂ ಇವೆ. ಈಸಾಯಿ ಧರ್ಮ ಇಂತಹ ಔದಾರ್ಯವನ್ನು ಎಲ್ಲಿ ತೋರಿದೆ? ಸ್ಪೆಯಿನ್​ ದೇಶದ ಅರಬ್ಬರು, ಆಸ್ಟ್ರೇಲಿಯಾ ಮತ್ತು ಅಮೆರಿಕಾದ ಆದಿವಾಸಿಗಳು ಈಗ ಎಲ್ಲಿರುವರು? ಯೂರೋಪಿನ ಯಹೂದ್ಯರಿಗೆ ಯಾವ ಸತ್ಕಾರ ಈಗ ಸಲ್ಲುತ್ತಿದೆ? ದಾನ ಒಂದನ್ನು ಬಿಟ್ಟರೆ ಇನ್ನಾವ ಕಾರ್ಯ ಪದ್ಧತಿಯೂ ಕ್ರೈಸ್ತ ಧರ್ಮಗ್ರಂಥದಲ್ಲಿ ಅನುಮೋದಿತವಾಗಿಲ್ಲ. ಎಲ್ಲಾ ಅದಕ್ಕೆ ವಿರೋಧವಾಗಿಯೇ ಆಗಿದೆ. ಯೂರೋಪಿನಲ್ಲಿ ಏನಾದರೂ ಸ್ವಲ್ಪ ಉನ್ನತಿ ಆಗಿದ್ದರೆ ಅದೆಲ್ಲ ಧರ್ಮಕ್ಕೆ ವಿರುದ್ಧವಾಗಿದೆ. ಈಗ ಏನಾದರೂ ಕ್ರೈಸ್ತಧರ್ಮವು ಪ್ರಬಲವಾಗಿದ್ದಿದ್ದರೆ ಡಾರ್ವಿನ್​, ಕೋಕ್​, ಪ್ಯಾಸ್ಟರ್​ ಮುಂತಾದವರನ್ನು ಜೀವಸಹಿತ ಸುಡುತ್ತಿತ್ತು. ವರ್ತಮಾನ ಯೂರೋಪಿನಲ್ಲಿ ಕ್ರೈಸ್ತ ಧರ್ಮ ಮತ್ತು ಅಭಿವೃದ್ಧಿ ಪರಸ್ಪರ ವಿರುದ್ಧವಾದವು. ನಾಗರಿಕತೆಯು ಈಗ ತನ್ನ ಆಜನ್ಮ ಶತ್ರುವಾದ ಕ್ರೈಸ್ತ ಧರ್ಮದ ಮೇಲೆ ಸೊಂಟಕಟ್ಟಿ, ಪಾದ್ರಿಗಳನ್ನು ಪದಚ್ಯುತರನ್ನಾಗಿ ಮಾಡಿ, ವಿದ್ಯೆ ಮತ್ತು ಪರೋಪಕಾರಗಳನ್ನು ಅದರ ಕೈಗಳಿಂದ ಕಸಿದುಕೊಳ್ಳುವುದರಲ್ಲಿದೆ. ಮೂರ್ಖ ಜನಸಾಮಾನ್ಯರ ಸಹಾಯವಿಲ್ಲದೆ ಇದ್ದರೆ ಕ್ರೈಸ್ತಧರ್ಮ ಎಂದೋ ಕಣ್ಮರೆಯಾಗು ತ್ತಿತ್ತು. ಏಕೆಂದರೆ ಪಟ್ಟಣಗಳಲ್ಲಿರುವ ಕೂಲಿಕಾರರು ಈಗಲೂ ಕ್ರೈಸ್ತಧರ್ಮದ ಶತ್ರುಗಳು. ಇದನ್ನೇ ಇಸ್ಲಾಂ ಧರ್ಮದೊಂದಿಗೆ ಹೋಲಿಸಿನೋಡಿ. ಮಹಮ್ಮದೀಯ ದೇಶಗಳಲ್ಲಿ ಎಲ್ಲಾ ಶಾಸನಗಳೂ ಇಸ್ಲಾಂಧರ್ಮವನ್ನೇ ಸಂಪೂರ್ಣವಾಗಿ ಆಧರಿಸಿವೆ. ಇಸ್ಲಾಂ ಧರ್ಮಪ್ರಚಾರಕರನ್ನು ಸರ್ಕಾರದ ಉದ್ಯೋಗಿಗಳು ಗೌರವಿಸುತ್ತಾರೆ. ಇತರ ಧರ್ಮಪ್ರಚಾರಕರಿಗೂ ಅವರು ಗೌರವ ಸಲ್ಲಿಸುವರು.

ಯೂರೋಪಿನ ಸಂಸ್ಕೃತಿಯನ್ನು ಒಂದು ಬಟ್ಟೆಗೆ ಹೋಲಿಸಿ ಹೀಗೆ ವಿವರಿಸಬಹುದು: ಸಮಶೀತೋಷ್ಣವಲಯದಲ್ಲಿರುವ ಬೆಟ್ಟ ಮತ್ತು ಕಡಲ ತಡಿಯೆ ಅದರ ಮಗ್ಗ. ಯುದ್ಧಪ್ರಿಯ ಬಲಿಷ್ಠ ಜನಾಂಗಗಳ ಸಮಷ್ಟಿಯಿಂದ ಉದ್ಭವಿಸಿದ ಜನರೇ ಇದಕ್ಕೆ ಹತ್ತಿ. ಅದರ ಹಾಸುಹೊಕ್ಕುಗಳು, ಆತ್ಮರಕ್ಷಣೆ ಮತ್ತು ಧರ್ಮರಕ್ಷಣಾರ್ಥವಾಗಿ ಸರ್ವದಾ ಯುದ್ಧ ಮಾಡುವುದು. ಬಲಾಢ್ಯನೆ ಸರಿ, ಬಲವಿಲ್ಲದವನು ಇತರರ ಸಾಮರ್ಥ್ಯದ ಛಾಯೆಯಲ್ಲಿ ನಿಲ್ಲಬೇಕು. ಇದರಿಂದ ತಯಾರಾದ ವಸ್ತುವೇ ವ್ಯಾಪಾರ. ಈ ಸಂಸ್ಕೃತಿ ಹರಡುವುದಕ್ಕೆ ರಾಜಮಾರ್ಗವೇ ಖಡ್ಗ. ಶಕ್ತಿ ಸಾಮರ್ಥ್ಯವೇ ಇದಕ್ಕೆ ಸಹಾಯ. ಇದರ ಗುರಿಯೇ ಭೋಗ.

ನಮ್ಮ ಸಂಸ್ಕೃತಿ ಹೇಗಿರುವುದು? ಆರ್ಯರು ಶಾಂತಿಪ್ರಿಯರು. ಕೃಷಿಯಿಂದ ಜೀವಿಸು ವರು. ತಮ್ಮ ಸಂಸಾರವನ್ನು ಆತಂಕವಿಲ್ಲದೆ ಸಾಕಿ ಸಲಹುವುದಕ್ಕೆ ಯಥೇಷ್ಟವಾದ ಅವಕಾಶ ದೊರಕಿದರೆ ಅದರಿಂದಲೇ ಅವರಿಗೆ ಸಂತೋಷ ಮತ್ತು ತೃಪ್ತಿ. ಅಂತಹ ಬದುಕಿನಲ್ಲಿ ಜನರಿಗೆ ಹೆಚ್ಚು ಬಿಡುವು ದೊರಕುತ್ತದೆ. ಚಿಂತನಶೀಲರಾಗಿ ನಾಗರಿಕತೆಯನ್ನು ಬೆಳೆಸುವುದಕ್ಕೆ ಆಗ ಹೆಚ್ಚಿನ ಅವಕಾಶ ದೊರಕುತ್ತದೆ. ಜನಕ ಮಹಾರಾಜನು ತಾನೇ ಕೈಗಳಿಂದ ನೆಲವನ್ನು ಉಳುತ್ತಿದ್ದ. ಆ ಕಾಲದಲ್ಲಿ ಅವನೇ ಸರ್ವ ಶ್ರೇಷ್ಠ ಆತ್ಮಜ್ಞಾನಿಯಾಗಿದ್ದ. ಮೊದಲಿನಿಂದಲೂ ಇಲ್ಲಿ ಋಷಿ, ಮುನಿ, ಯೋಗಿಗಳು ಜನಿಸಿದರು. ಮೊದಲಿನಿಂದಲೇ ಅವರು ಸಂಸಾರ ಮಿಥ್ಯೆ ಎಂಬುದನ್ನು ಅರಿತಿದ್ದರು. ಜಗಳ ಕದನಗಳಿಂದ ಪ್ರಯೋಜನವಿಲ್ಲ, ಯಾವ ಸುಖವನ್ನು ಹುಡುಕು ತ್ತಿರುವಿರೋ ಅದು ದೊರಕುವುದು ಶಾಂತಿಯಿಂದ. ಆ ಶಾಂತಿ ಸಿಕ್ಕಬೇಕಾದರೆ ಶರೀರ ಭೋಗ ವಿಸರ್ಜಿಸಬೇಕು. ಸುಖವು ದೇಹ ಪೋಷಣೆಯಿಂದ ಸಿಕ್ಕುವುದಿಲ್ಲ, ಮನಸ್ಸು ಬುದ್ಧಿಗಳನ್ನು ರೂಢಿಸುವುದರಿಂದ ಮಾತ್ರ ದೊರಕುವುದು.

ಜ್ಞಾನಿಗಳು ಕಾಡುಗಳನ್ನು ಕಡಿದು ಅವನ್ನು ಕೃಷಿಗೆ ಯೋಗ್ಯವಾದ ಭೂಮಿ ಯನ್ನಾಗಿ ಮಾಡಿದರು. ಅಂಥ ಭೂಮಿಯಲ್ಲಿ ಯಜ್ಞವೇದಿಕೆಯನ್ನು ಕಟ್ಟಿದರು. ಇಲ್ಲಿಂದ ಯಜ್ಞಧೂಮವು ನಿರ್ಮಲಾಕಾಶಕ್ಕೆ ಏರಿತು. ಆ ಶಾಂತ ವಾತಾವರಣದಲ್ಲಿ ವೇದಮಂತ್ರಗಳು ಪ್ರತಿಧ್ವನಿತಗೊಂಡುವು. ದನಕರುಗಳು ಯಾವ ಅಂಜಿಕೆಯೂ ಇಲ್ಲದೆ ಅಲ್ಲಿ ಮೇಯತೊಡಗಿದವು. ವಿದ್ಯೆ ಮತ್ತು ಧರ್ಮದ ಪದತಳದಲ್ಲಿ ಖಡ್ಗ ನಿಂತಿತು. ಖಡ್ಗದ ಏಕೈಕ ಕಾರ್ಯ ಧರ್ಮರಕ್ಷಣೆ, ಜನ ಮತ್ತು ದನದ ಪಾಲನೆ. ಆಪತ್ತಿನಲ್ಲಿರುವವರನ್ನು ರಕ್ಷಿಸುವವನು ಕ್ಷತ್ರಿಯನಾದ. ನೇಗಿಲು ಮತ್ತು ಕತ್ತಿಯನ್ನು ಆಳುತ್ತಿದ್ದುದು ಸರ್ವರಕ್ಷಕವಾದ ಧರ್ಮ. ಅದೇ ರಾಜರ ರಾಜ. ನಿದ್ರಿಸುವಾಗಲೂ ಅದು ಚಿರಜಾಗ್ರತವಾಗಿರುವುದು. ಧರ್ಮದ ಆಶ್ರಯದಲ್ಲಿ ಎಲ್ಲರೂ ಸ್ವತಂತ್ರ ರಾಗಿದ್ದರು.

ಆರ್ಯರೆನ್ನುವರು ಹೊರಗಡೆಯಿಂದ ಭರತಖಂಡದ ಮೇಲೆ ದಾಳಿಯಿಟ್ಟು ಇಲ್ಲಿನ ಆದಿವಾಸಿ ಜನರ ಭೂಮಿಯನ್ನು ಕಬಳಿಸಿ, ಅವರನ್ನು ನಿರ್ನಾಮಗೊಳಿಸಿ ಇಲ್ಲಿ ನೆಲಸಿದರು ಎಂದು ಯೂರೋಪಿನ ಪಂಡಿತರು ಹೇಳುವುದು ಶುದ್ಧ ಅರ್ಥಹೀನ. ಆಶ್ಚರ್ಯವೆಂದರೆ ನಮ್ಮ ಪಂಡಿತರೂ ಅದಕ್ಕೆ ತಥಾಸ್ತು ಹೇಳುತ್ತಾರೆ. ಈ ಭಯಂಕರ ಸುಳ್ಳುಗಳನ್ನೆಲ್ಲ ನಮ್ಮ ಮಕ್ಕಳಿಗೆ ಬೋಧಿಸಲಾಗುತ್ತಿದೆ. ಇದು ಮಹಾ ಅನ್ಯಾಯ.

ನಾನು ಅಲ್ಪಜ್ಞ. ನನಗೇನೂ ವಿದ್ವತ್ತು ಅಷ್ಟು ಇಲ್ಲ. ಆದರೆ ನನಗೆ ಗೊತ್ತಿರುವ ಆಧಾರದ ಮೇಲೆ ಪ್ಯಾರಿಸ್​ ಕಾಂಗ್ರೆಸಿನಲ್ಲಿ ಈ ಸಿದ್ಧಾಂತವನ್ನು ತೀವ್ರವಾಗಿ ವಿರೋಧಿಸಿದ್ದೇನೆ. ಭಾರತೀಯ ಮತ್ತು ಐರೋಪ್ಯ ವಿದ್ವಾಂಸರೊಡನೆ ಈ ವಿಷಯವನ್ನು ನಾನು ಚರ್ಚಿಸುತ್ತಿರುವೆನು. ಸರಿಯಾದ ಸಮಯದಲ್ಲಿ ಇನ್ನೂ ಕೆಲವು ಆಕ್ಷೇಪಣೆಗಳನ್ನು ಎತ್ತುತ್ತೇನೆ. ನಾನು ಇಷ್ಟನ್ನು ನಮ್ಮ ಪಂಡಿತರಿಗೆ ಹೇಳುತ್ತೇನೆ: “ನೀವು ವಿದ್ವಾಂಸರು, ನಿಮ್ಮ ಹಳೆಯ ಶಾಸ್ತ್ರ ಪುರಾಣಗಳನ್ನು ಹುಡುಕಿ ನೀವೇ ನಿರ್ಣಯಕ್ಕೆ ಬನ್ನಿ.”

ಐರೋಪ್ಯರು ಎಲ್ಲಿ ಹೋಗಲಿ, ಒಂದು ಅವಕಾಶ ಸಿಕ್ಕಿದೊಡನೆ ಅಲ್ಲಿಯ ಆದಿವಾಸಿಗಳನ್ನು ನಾಶಮಾಡಿ ಆ ಸ್ಥಳದಲ್ಲಿ ಸುಖವಾಗಿ ವಾಸಿಸುವರು. ಆದಕಾರಣವೇ ಪಾಶ್ಚಾತ್ಯರು ಆರ್ಯರೂ ಕೂಡ ಹೀಗೆಯೇ ಮಾಡಿರಬೇಕೆಂದು ಊಹಿಸುವರು! ಪಾಶ್ಚಾತ್ಯರು ತಮ್ಮ ಮನೆಯಲ್ಲಿ ತಮ್ಮದೇ ಆದಾಯವನ್ನು ನೆಚ್ಚಿಕೊಂಡು ತಾವು ತೆಪ್ಪಗೆ ಇದ್ದರೆ ಇತರರು ಅವರನ್ನು ನಿಕೃಷ್ಟ ದೃಷ್ಟಿಯಿಂದ ನೋಡುವರು. ಅದಕ್ಕೆ ಯಾವ ದೇಶವನ್ನು ಲೂಟಿ ಮಾಡೋಣ, ದರೋಡೆ ಮಾಡೋಣ ಎಂದು ಅವರು ಮೆಲಕು ಹಾಕುತ್ತಿರಬೇಕು. ಆದಕಾರಣವೇ ಆರ್ಯರು ಹಾಗೆಯೇ ಮಾಡಿರಬೇಕೆಂದು ನಿರ್ಧರಿಸುವರು. ಅದಕ್ಕೆ ಆಧಾರವೆಲ್ಲ ಬರೀ ಊಹೆ, ನಿಮ್ಮ ಅನುಮಾನ, ಊಹೆ ಯನ್ನು ನೀವೇ ಇಟ್ಟುಕೊಳ್ಳಿ.

ಯಾವ ವೇದದಲ್ಲಿ, ಯಾವ ಸೂಕ್ತದಲ್ಲಿ ಆರ್ಯರು ಪರದೇಶದಿಂದ ಬಂದರೆಂಬು ದನ್ನು ನೋಡುವಿರಿ? ಆದಿ ಜನರನ್ನು ಕೊಂದರು ಎಂಬುದಕ್ಕೆ ಆಧಾರವೆಲ್ಲಿ? ಇಂತಹ ಪ್ರಲಾಪದಿಂದ ಪ್ರಯೋಜವೇನು? ರಾಮಾಯಣವನ್ನು ಓದಿ ಸಾರ್ಥಕವಾಯಿತು. ಅದರ ಆಧಾರದ ಮೇಲೆ ಇಂತಹ ಸುಳ್ಳನ್ನು ಏತಕ್ಕೆ ತಯಾರುಮಾಡುವಿರಿ?

ರಾಮಾಯಣವೆಂದರೇನು? ಆರ್ಯರು ದಂಡಕಾರಣ್ಯವನ್ನು ಜಯಿಸಿದ ಕಥೆಯೆ? ರಾಮಚಂದ್ರ ಸುಸಭ್ಯ ಜನಾಂಗದ ದೊರೆ. ಅವನು ಯಾರೊಂದಿಗೆ ಯುದ್ಧಮಾಡು ವನು? ಲಂಕಾನಗರಿಯ ರಾಜನಾದ ರಾವಣನೊಡನೆ. ದಯವಿಟ್ಟು ರಾಮಾಯಣ ವನ್ನು ಇನ್ನೊಮ್ಮೆ ಓದಿ. ರಾವಣ ರಾಮನಿಗಿಂತ ಹೆಚ್ಚು ನಾಗರಿಕನಾಗಿದ್ದನು, ಕಡಿಮೆಯಿಲ್ಲ. ಲಂಕಾ ಸಂಸ್ಕೃತಿ ಅಯೋಧ್ಯೆಯ ಸಂಸ್ಕೃತಿಗಿಂತ ಕೀಳಲ್ಲ, ಮೇಲಾಗಿತ್ತು. ವಾನರರು ಮತ್ತು ಇತರ ದಕ್ಷಿಣ ಇಂಡಿಯಾದವರನ್ನು ಎಂದು ಗೆದ್ದದ್ದು? ಅವರೆಲ್ಲ ರಾಮಚಂದ್ರನ ಸ್ನೇಹಿತರು. ರಾಮಚಂದ್ರ ಯಾವ ವಾಲಿಯ, ಯಾವ ಗುಹಕನ ರಾಜ್ಯವನ್ನು ಆಕ್ರಮಿಸಿದನು?

ಕೆಲವು ಕಡೆ ಆರ್ಯರಿಗೂ ದಸ್ಯುಗಳಿಗೂ ಯುದ್ಧ ನಡೆದಿರಬಹುದು. ಕೆಲವು ಕಪಟ ಋಷಿಗಳು ಬಕಧ್ಯಾನ ಮಾಡಿ ಯಾವಾಗ ದಸ್ಯುಗಳು ಬಂದು ತಮಗೆ ವಿಘ್ನ ತಂದಾರು ಎಂದು ಕಾಯುತ್ತಿದ್ದರು, ಕೆಲವು ವೇಳೆ ಹಾಗೆ ತೊಂದರೆ ಕೊಟ್ಟಾಗ ತಕ್ಷಣ ಊರಿಗೆ ಹೋಗಿ ರಾಜನಿಗೆ ದೂರು ಹೇಳುವರು. ಅವರು ಶಸ್ತ್ರಾಸ್ತ್ರ ಸನ್ನದ್ಧರಾಗಿ ಬಂದು ದಸ್ಯುಗಳೊಡನೆ ಯುದ್ದ ಮಾಡುತ್ತಿದ್ದರು. ಪಾಪ ದಸ್ಯುಗಳು ಕೋಲುಕಲ್ಲುಗಳೊಡನೆ ಎಷ್ಟುಕಾಲ ತಾನೆ ಯುದ್ಧ ಮಾಡಿಯಾರು? ಅವರು ಹತರಾದರು, ಇಲ್ಲವೆ ಓಡಿ ಹೋದರು. ಇದರಲ್ಲಿ ಕೆಲವು ಕಥೆಗಳು ನಿಜವಾಗಿರಬಹುದು. ಆದರೂ ಆರ್ಯರು ಅವರ ಊರನ್ನು ಸ್ವಾಧೀನ ಪಡಿಸಿಕೊಂಡರು ಎನ್ನುವುದಕ್ಕೆ ಆಧಾರವೆಲ್ಲಿರುವುದು? ರಾಮಾಯಣದಲ್ಲಿ ಎಲ್ಲಿ ನಿಮಗೆ ಆಧಾರ ಸಿಕ್ಕುವುದು?

ಆರ್ಯ ಸಭ್ಯತಾರೂಪಿ ವಸ್ತ್ರಕ್ಕೆ ವಿಶಾಲ ನದಿ, ನದ, ಉಷ್ಣ ಪ್ರಧಾನ ಸಮತಲ ಕ್ಷೇತ್ರ ಮಗ್ಗ. ನಾನಾ ಪ್ರಕಾರ ಸುಸಭ್ಯ, ಅರ್ಧ ಸಭ್ಯ ಅಸಭ್ಯ ಆರ್ಯ ಪ್ರಧಾನ ಮನುಷ್ಯರೇ ಈ ವಸ್ತ್ರಕ್ಕೆ ಹತ್ತಿ. ಇದರ ತಂತು ವರ್ಣಾಶ್ರಮಾಚಾರ; ಅದರ ಗುರಿ ಪ್ರಕೃತಿ ದ್ವಂದ್ವ ಮತ್ತು ಸಂಘರ್ಷಣೆಯ ನಿವಾರಣೆ.

ಐರೋಪ್ಯರೇ! ನೀವು ಯಾವ ದೇಶವನ್ನು ಮುಂದಕ್ಕೆ ತಂದಿರುವಿರಿ? ಎಲ್ಲಿ ಅಬಲ ಜನಾಂಗವಿತ್ತೊ ಅದನ್ನು ಆಮೂಲಾಗ್ರವಾಗಿ ನಾಶಮಾಡಿರುವಿರಿ. ನೀವು ಅವರ ಸ್ಥಳವನ್ನು ಆಕ್ರಮಿಸಿರುವಿರಿ, ಅವರು ನಿರ್ನಾಮರಾದರು. ನಿಮ್ಮ ಅಮೇರಿಕಾ, ನ್ಯೂಜಿಲೇಂಡ್​, ಆಸ್ಟ್ರೇಲಿಯಾ ಶಾಂತಿಸಾಗರದ ದ್ವೀಪಗಳು, ದಕ್ಷಿಣ ಆಫ್ರಿಕಾ, ಇವುಗಳ ಚರಿತ್ರೆ ಏನು? ಅಲ್ಲಿಯ ಆದಿವಾಸಿಗಳು ಇಂದೆಲ್ಲಿ? ಅವರನ್ನೆಲ್ಲಾ ನಾಶ ಮಾಡಿದಿರಿ. ಕಾಡುಮೃಗಗಳಂತೆ ಅವರನ್ನು ಬೇಟೆಯಾಡಿದಿರಿ. ಎಲ್ಲಿ ನಿಮಗೆ ನಾಶ ಮಾಡುವುದಕ್ಕೆ ಸಾಧ್ಯವಾಗುಲಿಲ್ಲವೊ ಅಲ್ಲಿ ಮಾತ್ರ ಜನ ಜೀವಿಸಿರುವರು.

ಭಾರತ ವರ್ಷ ಹೀಗೆ ಎಂದೂ ಆಚರಿಸಲಿಲ್ಲ. ಆರ್ಯರು ಬಹಳ ದಯಾಳುಗಳು, ಅವರಿಗೆ ಅಖಂಡ ಸಮುದ್ರದಂತೆ ವಿಶಾಲ ಹೃದಯವಿತ್ತು. ದೈವೀ ಪ್ರತಿಭಾ ಸಂಪನ್ನ ಬುದ್ಧಿಶಕ್ತಿ ಇತ್ತು. ಅಲ್ಲಿ ಕ್ಷಣಿಕವೂ, ತೋರಿಕೆಗೆ ರಮಣೀಯವೂ ಆದರೆ ನಿಜವಾಗಿ ಪಾಶವಿಕವಾದ ಭಾವನೆ ಎಂದಿಗೂ ಇರಲಿಲ್ಲ. ನಮ್ಮ ದೇಶದ ಮೂರ್ಖರೇ! ಆರ್ಯರು ಕಾಡು ಜನಾಂಗವನ್ನು ನಾಶಮಾಡಿದರೆ ವರ್ಣಾಶ್ರಮವನ್ನು ಏಕೆ ಸೃಷ್ಟಿಸುತ್ತಿದ್ದರು?

ಐರೋಪ್ಯರ ಉದ್ದೇಶ ಎಲ್ಲರನ್ನೂ ನಾಶಮಾಡಿ ತಾವು ಮಾತ್ರ ಬದುಕ ಬೇಕೆಂಬುದು. ಆರ್ಯರ ಉದ್ದೇಶ ಎಲ್ಲರನ್ನು ತಮ್ಮ ಸಮನಾಗಿ ಮಾಡಬೇಕು, ತಮ ಗಿಂತ ಉತ್ತಮರನ್ನಾಗಿ ಮಾಡಬೇಕೆಂಬುದು. ಯೂರೋಪಿನಲ್ಲಿ ಸಭ್ಯತೆಗೆ ದಾರಿ ಖಡ್ಗದ ಮೂಲಕ. ಭರತಖಂಡದಲ್ಲಿ ವರ್ಣದ ವಿಭಾಗದ ಮೂಲಕ. ಒಬ್ಬನ ಶಿಕ್ಷಣ ಮತ್ತು ಸಭ್ಯತೆಗೆ ತಕ್ಕಂತೆ ಮೇಲೇರಿಹೋಗಲು ಇರುವ ಮೆಟ್ಟಲು ಸಾಲು ವರ್ಣವಿಭಾಗ. ಯೂರೋಪಿನಲ್ಲಿ ಬಲಶಾಲಿಗೇ ಜಯ, ಬಲಹೀನನಿಗೆ ಮೃತ್ಯು. ಭರತವರ್ಷದಲ್ಲಿ ಪ್ರತ್ಯೇಕ ಸಾಮಾಜಿಕ ನಿಯಮಗಳಿರುವುದು ದುರ್ಬಲರ ರಕ್ಷಣೆಗಾಗಿ.

\begin{center}
(ಮೂಲ ಬಂಗಾಳಿಯಿಂದ ಇಂಗ್ಲಿಷಿಗೆ ಅನುವಾದಗೊಂಡದ್ದು.)
\end{center}

