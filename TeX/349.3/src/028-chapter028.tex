
\chapter{ನಾವು ಜನ್ಮವೆತ್ತಿದ ಧರ್ಮ}

(ಢಾಕಾದಲ್ಲಿ ೧೯೦೧, ಮಾರ್ಚ್ ೩೧ ರಂದು ಬೃಹತ್​ ಬಹಿರಂಗ ಸಭೆಯನ್ನು ಉದ್ದೇಶಿಸಿ ಸ್ವಾಮೀಜಿಯವರು ಈ ಮೇಲಿನ ವಿಷಯದ ಮೇಲೆ ಇಂಗ್ಲಿಷ್​ನಲ್ಲಿ ಎರಡು ಗಂಟೆ ಮಾತನಾಡಿದರು. ಈ ಕೆಳಗಿನದು ಅದರ ಬಂಗಾಳಿ ವರದಿಯ ಭಾಷಾಂತರ)

ಪ್ರಾಚೀನ ಕಾಲದಲ್ಲಿ ನಮ್ಮ ದೇಶವು ಆಧ್ಯಾತ್ಮಿಕ ಕ್ಷೇತ್ರದಲ್ಲಿ ಅದ್ಭುತ ಪ್ರಗತಿಯನ್ನು ಸಾಧಿಸಿತ್ತು. ಆ ಪ್ರಾಚೀನ ಇತಿಹಾಸವನ್ನು ಪುನಃ ಇಂದು ಸ್ಮರಿಸಿಕೊಳ್ಳೋಣ. ಆದರೆ ಗತಕಾಲದ ವೈಭವವನ್ನು ಮೆಲುಕು ಹಾಕುವುದರಿಂದ ಸಂಭವಿಸಬಹುದಾದ ಒಂದು ಅಪಾಯವೇನೆಂದರೆ ಹೊಸದನ್ನು ಸಾಧಿಸಲು ಪ್ರಯತ್ನಪಡದಿರುವುದು, ಹಾಗೂ ಗತಕಾಲದ ವೈಭವದಲ್ಲಿಯೇ ಸಂತೃಪ್ತರಾಗಿ ಅದರ ವಿಷಯವಾಗಿ ಜಂಭಕೊಚ್ಚಿಕೊಳ್ಳುತ್ತ ನಿಶ್ಚೇಷ್ಟಿತರಾಗಿಬಿಡುವುದು. ಈ ವಿಷಯದಲ್ಲಿ ನಾವು ಎಚ್ಚರಿಕೆಯಿಂದಿರಬೇಕು. ಹಿಂದಿನ ಕಾಲದಲ್ಲಿ ಸತ್ಯ ಸಾಕ್ಷಾತ್ಕಾರವನ್ನು ಮಾಡಿಕೊಂಡ ಅನೇಕ ಋಷಿ ಮಹರ್ಷಿಗಳೇನೋ ನಿಸ್ಸಂದೇಹವಾಗಿ ಇದ್ದರು. ಪ್ರಾಚೀನ ಕಾಲದ ಮಹಿಮಾ ಸ್ಮರಣೆ ಪ್ರಯೋಜನವಾಗಬೇಕಾದರೆ ನಾವೂ ಅವರಂತೆಯೇ ಋಷಿಗಳಾಗಬೇಕು. ಅದು ಮಾತ್ರವಲ್ಲ, ಹಿಂದೆ ನಮ್ಮ ದೇಶದಲ್ಲಿದ್ದ ಮಹರ್ಷಿಗಳನ್ನು ಕೂಡ ನಾವು ಮೀರಿಸಬಲ್ಲೆವು ಎಂಬುದು ನನ್ನ ದೃಢನಿಶ್ಚಯ. ನಾವು ಹಿಂದೆ ಬಹಳ ಸಾಧಿಸಿದ್ದೆವು; ನನಗೆ ಇದರಲ್ಲಿ ಹೆಮ್ಮೆಯುಂಟು. ಅದನ್ನು ಕುರಿತು ಆಲೋಚಿಸಿದರೆ ಆನಂದವಾಗುವುದು. ಈಗಿನ ಅವನತಿಯನ್ನು ನೋಡಿ ನಾನು ನಿರಾಶನೂ ಆಗುವುದಿಲ್ಲ. ನಮಗೆ ಭವ್ಯ ಭವಿಷ್ಯವಿದೆ ಎಂಬುದರಲ್ಲಿ ನನಗೆ ತುಂಬು ಭರವಸೆಯುಂಟು. ಏಕೆಂದರೆ, ಬೀಜವು ಒಂದು ಮರವಾಗುವುದಕ್ಕೆ ಮುಂಚೆ ಸಂಪೂರ್ಣ ಬದಲಾಗುವುದು. ಅದು ನಾಶವಾದಂತೆಯೇ ಕಾಣುತ್ತದೆ. ಅದರಂತೆಯೆ ಇಂದಿನ ಅವನತಿಯ ಅಂತರಾಳದಲ್ಲಿ ನಮ್ಮ ಧರ್ಮದ ಮುಂಬರುವ ಶ್ರೇಷ್ಠತೆಯು ಸುಪ್ತಾವಸ್ಥೆಯಲ್ಲಿರಬಹುದು, ಹಿಂದಿಗಿಂತಲೂ ಹೆಚ್ಚು ಮಹಿಮಾನ್ವಿತವಾಗಿ ಜಾಗ್ರತವಾಗಲು ಸಿದ್ಧವಾಗಿರಬಹುದು.

\vskip 4pt

ನಾವು ಜನ್ಮವೆತ್ತಿದ ಧರ್ಮದಲ್ಲಿರುವ ಸಾಮಾನ್ಯ ಭಾವನೆಗಳಾವುವು ಎಂಬುದನ್ನು ಪರ್ಯಾಲೋಚಿಸೋಣ. ಮೇಲುನೋಟಕ್ಕೆ ನಮ್ಮ ವಿಭಿನ್ನ ಪಂಗಡಗಳಲ್ಲಿ ಎಷ್ಟೋ ವ್ಯತ್ಯಾಸಗಳು ಕಾಣುತ್ತವೆ ಎಂಬುದು ನಿಜ. ಕೆಲವರು ಅದ್ವೈತಿಗಳು, ಕೆಲವರು ವಿಶಿಷ್ಟಾದ್ವೈತಿಗಳು, ಮತ್ತೆ ಕೆಲವರು ದ್ವೈತಿಗಳು. ಕೆಲವರು ಅವತಾರಗಳನ್ನು ನಂಬುವರು, ಕೆಲವರು ಮೂರ್ತಿಪೂಜೆಯನ್ನು ನಂಬುತ್ತಾರೆ, ಮತ್ತೆ ಕೆಲವರು ನಿರಾಕಾರತತ್ತ್ವವನ್ನು ಪ್ರತಿಪಾದಿಸುತ್ತಾರೆ. ಹಾಗೆಯೇ ಆಚಾರ ವ್ಯವಹಾರಗಳಲ್ಲಿಯೂ ಎಷ್ಟೋ ಭಿನ್ನತೆಗಳಿವೆ. ಜಾಟರು ಮಹಮ್ಮದೀಯರೊಡನೆ ಮತ್ತು ಕ್ರೈಸ್ತರೊಡನೆ ವಿವಾಹ ಸಂಬಂಧವನ್ನು ಬೆಳಸಿದರೂ ಅವರಿಗೆ ಬಹಿಷ್ಕಾರವಿಲ್ಲ. ಯಾವ ಹಿಂದೂ ದೇವಸ್ಥಾನಕ್ಕಾದರೂ ಆತಂಕವಿಲ್ಲದೆ ಅವರು ಪ್ರವೇಶಿಸಬಹುದು. ಪಂಜಾಬಿನ ಹಲವು ಗ್ರಾಮಗಳಲ್ಲಿ ಹಂದಿಯನ್ನು ತಿನ್ನದವರನ್ನು ಹಿಂದೂಗಳೆಂದು ಪರಿಗಣಿಸುವುದೇ ಇಲ್ಲ. ನೇಪಾಳದಲ್ಲಿ ಬ್ರಾಹ್ಮಣರು ನಾಲ್ಕೂ ವರ್ಣದವರನ್ನು ಮದುವೆಯಾಗಬಹುದು. ಆದರೆ ಬಂಗಾಳದಲ್ಲಿ ಬ್ರಾಹ್ಮಣನು ತನ್ನ ಜಾತಿಯಲ್ಲೇ ಉಪಜಾತಿಯವರೊಡನೆ ಮದುವೆಯಾಗುವಂತಿಲ್ಲ. ಹೀಗೆ ಹಲವು ಆಚಾರಗಳಿವೆ. ಇಷ್ಟೊಂದು ಭಿನ್ನತೆಗಳಿದ್ದರೂ ಎಲ್ಲಾ ಹಿಂದೂಗಳಲ್ಲಿಯೂ ಒಂದು ಏಕತೆಯನ್ನು ನೋಡುವೆವು– ಅದೇ ಯಾವ ಹಿಂದೂವೂ ಗೋಮಾಂಸ ಭಕ್ಷಣ ಮಾಡದಿರುವುದು. ಹೀಗೆಯೇ ಹಿಂದೂಧರ್ಮದ ವಿಭಿನ್ನ ಪಂಗಡಗಳಲ್ಲಿ ಎಷ್ಟೋ ಸಾಮಾನ್ಯ ಭಾವನೆಗಳಿವೆ.

\vskip 4pt

ಶಾಸ್ತ್ರಗಳನ್ನು ಕುರಿತು ಚರ್ಚಿಸುವಾಗ ಒಂದು ಸಂಗತಿ ಬಹಳ ಮುಖ್ಯವಾಗಿ ತೋರುವುದು ಯಾವ ಧರ್ಮದಲ್ಲಿ ಒಂದು ಅಥವಾ ಹಲವು ಶಾಸ್ತ್ರಗಳು ಇವೆಯೋ ಅದು ಮಾತ್ರವೇ ಕ್ಷಿಪ್ರ ಪ್ರಗತಿಯನ್ನು ಸಾಧಿಸಿದೆ ಮತ್ತು ಎಷ್ಟೇ ವಿರೋಧಿಗಳಿದ್ದರೂ ಅವನ್ನೆಲ್ಲ ಎದುರಿಸಿ ಇಂದಿಗೂ ಉಳಿದುಕೊಂಡು ಬಂದಿದೆ. ಆದರೆ ಗ್ರೀಕ್​ ಧರ್ಮವು ಎಷ್ಟು ಸುಂದರವಾಗಿದ್ದರೂ ಶಾಸ್ತ್ರದ ಆಧಾರವಿಲ್ಲದೆ ಇದ್ದುದರಿಂದ ಮಾಯವಾಯಿತು. ಯಹೂದ್ಯರ ಧರ್ಮವು ಹಳೆಯ ಒಡಂಬಡಿಕೆಯ ಮೇಲೆ ಭದ್ರವಾಗಿ ನಿಂತಿರುವುದು. ಇದರಂತೆಯೇ ಜಗತ್ತಿನಲ್ಲೆಲ್ಲಾ ಬಹಳ ಪುರಾತನವೆಂದು ಪರಿಗಣಿಸಲ್ಪಟ್ಟ ವೇದಗಳು ಶಾಸ್ತ್ರವಾಗಿ ಉಳ್ಳ ಹಿಂದೂಧರ್ಮವು ಕೂಡ. ವೇದವನ್ನು ಕರ್ಮಕಾಂಡ ಮತ್ತು ಜ್ಞಾನಕಾಂಡವೆಂದು ವಿಭಾಗಿಸಿರುವರು. ಒಳ್ಳೆಯದಕ್ಕೆ, ಕೆಟ್ಟದಕ್ಕೆ, ಅಂತೂ ಕರ್ಮಕಾಂಡ ಈಗ ಜಾರಿಯಲ್ಲಿ ಇಲ್ಲ. ಎಲ್ಲೋ ಮಹಾರಾಷ್ಟ್ರ ದೇಶದಲ್ಲಿ ಕೆಲವು ವೇಳೆ ಯಜ್ಞಮಾಡಿ ಮೇಕೆಯನ್ನು ಬಲಿ ಕೊಡುವ ಕೆಲವು ಬ್ರಾಹ್ಮಣರು ಇರುವರು. ಮದುವೆ ಶ್ರಾದ್ಧ ಮುಂತಾದ ಸಮಯಗಳಲ್ಲಿ, ನಾವು ಉಪಯೋಗಿಸುವ ಮಂತ್ರದಲ್ಲಿ ವೇದಗಳ ಕರ್ಮಕಾಂಡದ ಕೆಲವು ಲಕ್ಷಣಗಳು ಕಾಣುವುವು. ಆದರೆ ಅದು ಪುನಃ ಹಿಂದಿನಂತೆ ಪ್ರಬಲವಾಗುವ ಸಂಭವವಿಲ್ಲ. ಒಮ್ಮೆ ಕುಮಾರಿಲ ಭಟ್ಟನು ಅದರ ಪುನರುತ್ಥಾನ ಮಾಡಲು ಪ್ರಯತ್ನಿಸಿ ವಿಫಲನಾದನು.

\vskip 4pt

ವೇದಗಳ ಜ್ಞಾನಕಾಂಡಕ್ಕೆ ಉಪನಿಷತ್​ ಅಥವಾ ವೇದಾಂತವೆಂದು ಹೆಸರು. ಎಲ್ಲಿಯಾದರೂ ಆಚಾರ್ಯರು ಶ್ರುತಿಯನ್ನು ಉದಾಹರಿಸಿದರೆ ಅದು ಯಾವಾಗಲೂ ಉಪ\break ನಿಷತ್ತೇ ಆಗಿರುತ್ತದೆ. ಹಿಂದೂಗಳ ಈಗಿನ ಧರ್ಮವೇ ವೇದಾಂತ. ಭರತ ಖಂಡದಲ್ಲಿ ಯಾವುದಾದರೂ ಒಂದು ಪಂಗಡ ಸುಭದ್ರವಾಗಿ ಜನಾದರಣೀಯವಾಗಬೇಕಾದರೆ ಅದು ವೇದಾಂತದ ಪ್ರಮಾಣದ ಮೇಲೆ ನಿಂತಿರಬೇಕು. ದ್ವೈತಿಗಳಾಗಲೀ, ಅದ್ವೈತಿಗಳಾಗಲೀ, ಅವರು ವೇದಾಂತವನ್ನೇ ಪ್ರಮಾಣವಾಗಿ ಸ್ವೀಕರಿಸಬೇಕು. ವೈಷ್ಣವರು ಕೂಡಾ, ತಮ್ಮ ಸಿದ್ಧಾಂತಗಳ ಸತ್ಯತೆಯನ್ನು ಪ್ರಮಾಣೀಕರಿಸಲು, ಗೋಪಾಲತಾಪಿನೀ ಉಪನಿಷತ್ತನ್ನೇ ಆಶ್ರಯಿಸಬೇಕು. ಹೊಸ ಪಂಗಡದವರು ಉಪನಿಷತ್ತಿನಲ್ಲಿ ತಮ್ಮ ಧರ್ಮ ಪ್ರತಿಪಾದನೆಗೆ ಆಧಾರ ಸಿಕ್ಕದೆ ಇದ್ದರೆ ಹೊಸ ಒಂದು ಉಪನಿಷತ್ತನ್ನೇ ಸೃಷ್ಟಿಸಿ, ಅದು ಬಹಳ ಹಿಂದಿನ ಗ್ರಂಥ ಎಂದು ತೋರುವುದಕ್ಕೆ ಪ್ರಯತ್ನಿಸುವರು. ಹಿಂದೆ ಇಂತಹವು ಎಷ್ಟೋ ಇದ್ದವು.

\vskip 4pt

ವೇದಗಳು ಬಹಳ ಹಿಂದಿನ ಕಾಲದಲ್ಲಿ ಮನುಷ್ಯರು ಬರೆದ ಗ್ರಂಥ ಎಂದು ಹಿಂದೂಗಳು ಒಪ್ಪಿಕೊಳ್ಳುವುದಿಲ್ಲ. ಅವು ಅಗಾಧ ಆಧ್ಯಾತ್ಮಿಕ ಸತ್ಯಗಳ ಸಂಗ್ರಹ ಎಂದು ಅವರು ಹೇಳುತ್ತಾರೆ. ಕೆಲವು ವೇಳೆ ಆ ಸತ್ಯಗಳು ಸುಪ್ತವಾಗಿರುವುವು, ಕೆಲವು ವೇಳೆ ವ್ಯಕ್ತವಾಗಿರುವುವು. ಭಾಷ್ಯಕಾರ ಸಾಯಣಾಚಾರ್ಯರು ಎಲ್ಲೋ ಒಂದು ಕಡೆ \textbf{ಯೋ ವೇದೇಭ್ಯೋಽ\-ಖಿಲಂ ಜಗತ್​ ನಿರ್ಮಮೇ} – “ವೇದಗಳಿಂದ ಅಖಿಲ ಜಗತ್ತನ್ನು ಸೃಷ್ಟಿಸಿದನು” ಎನ್ನುತ್ತಾರೆ. ವೇದಸಂಗ್ರಹಕಾರರನ್ನು ಯಾರೂ ನೋಡಿಲ್ಲ, ಹಾಗೆ ಯಾರನ್ನಾದರೂ ಊಹಿಸುವುದಕ್ಕೂ ಅಸಾಧ್ಯ. ಋಷಿಗಳು ಮಂತ್ರ ಅಥವಾ ಸನಾತನ ನಿಯಮವನ್ನು ಕಂಡುಹಿಡಿದವರು. ಅನಾದಿ ಕಾಲದಿಂದಲೂ ಇದ್ದ ವೇದಜ್ಞಾನರಾಶಿಯನ್ನು ಇವರು ಸಂದರ್ಶಿಸಿದರು ಅಷ್ಟೆ.

\vskip 4pt

ಈ ಋಷಿಗಳು ಯಾರು? ಯಾರು ಸರಿಯಾದ ಮಾರ್ಗದಲ್ಲಿ ಧರ್ಮವನ್ನು ಸಾಕ್ಷಾತ್ಕಾರ ಮಾಡಿಕೊಂಡಿರುವರೋ, ಅವರೇ ಋಷಿಗಳು–ಅವರು ಮ್ಲೇಚ್ಛರಾದರೂ ಚಿಂತೆಯಿಲ್ಲ, ಎಂದು ವಾತ್ಸ್ಯಾಯನ ಹೇಳುತ್ತಾನೆ. ಆದಕಾರಣವೇ ಹಿಂದಿನ ಕಾಲದಲ್ಲಿ ವಿವಾಹೇತರ ಸಂಬಂಧದಿಂದ ಜನಿಸಿದ ವಸಿಷ್ಠ, ಬೆಸ್ತರ ಮಗ ವ್ಯಾಸ, ಗಂಡನಾರೆಂದು ತಿಳಿಯದ ದಾಸಿಯ ಸುತ ನಾರದ, ಇವರೆಲ್ಲಾ ಮಹಾಋಷಿಗಳಾಗಿದ್ದರು. ಯಾರು ಸತ್ಯ ಸಾಕ್ಷಾತ್ಕಾರ ಮಾಡಿಕೊಂಡಿರುವರೋ ಅವರಲ್ಲಿ ವ್ಯತ್ಯಾಸವನ್ನು ನೋಡಬಾರದು ಎಂಬುದು ಇದರಿಂದ ಸ್ಪಷ್ಟವಾಗುತ್ತದೆ. ಈಗ ತಾನೇ ಹೆಸರಿಸಲ್ಪಟ್ಟವರೆಲ್ಲ ಶ್ರೇಷ್ಠ ಋಷಿಗಳಾಗಿರುವಾಗ ಕುಲೀನ ಬ್ರಾಹ್ಮಣರಾದ ನೀವು, ಎಂತಹ ಶ್ರೇಷ್ಠ ಋಷಿಗಳಾಗಬಹುದು! ಆ ಋಷಿತ್ವವನ್ನು ಪಡೆಯಲು ಪ್ರಯತ್ನಿಸಿ, ಗುರಿ ಸೇರುವ ತನಕ ಬಿಡಬೇಡಿ. ಇಡಿಯ ಪ್ರಪಂಚ ಸ್ವಭಾವತಃ ನಿಮಗೆ ಮಣಿಯುವುದು. ಋಷಿಗಳಾಗಿ, ಅದೇ ಶಕ್ತಿಯ ರಹಸ್ಯ.

\vskip 4pt

ಈ ವೇದವೇ ನಮ್ಮ ಪ್ರಮಾಣ. ಅದನ್ನು ಓದುವುದಕ್ಕೆ ಎಲ್ಲರಿಗೂ ಅಧಿಕಾರವಿದೆ.

\smallskip

\begin{longtable}{@{\hspace{-10pt}}l@{}}
\textbf{“ಯಥೇಮಾಂ ವಾಚಂ ಕಲ್ಯಾಣೀಮಾವದಾನಿ ಜನೇಭ್ಯಃ ।} \\
\textbf{ಬ್ರಹ್ಮರಾಜನ್ಯಾಭ್ಯಾಂ ಶೂದ್ರಾಯ} \\
\textbf{ಚಾರ್ಯಾಯ ಚ ಸ್ವಾಯ ಚಾರಣಾಯ ॥”} \\
\end{longtable}

\hfill (ಶುಕ್ಲ ಯಜುರ್ವೇದ ಮಾಧ್ಯಂದಿನಶಾಖಾ ೨೬ ಅಧ್ಯಾಯ ಮಂತ್ರ ೨).

\smallskip

ವೇದಗಳಿಂದ, ಅವನ್ನು ಓದಲು ಎಲ್ಲರಿಗೂ ಅಧಿಕಾರವಿಲ್ಲ ಎಂಬ ಯಾವುದಾದರೂ ಪ್ರಮಾಣವನ್ನು ತೋರಬಲ್ಲಿರಾ? ಪುರಾಣಗಳೇನೋ, ವೇದಗಳ ಕೆಲವು ಭಾಗಗಳನ್ನು ಓದಲು ಕೆಲವು ಜಾತಿಯವರಿಗೆ ಅಧಿಕಾರವಿದೆ, ಮತ್ತೆ ಕೆಲವು ಜಾತಿಯವರಿಗೆ ಅಧಿಕಾರವಿಲ್ಲ ಎಂದು ಹೇಳಬಹುದು. ವೇದಗಳ ಈ ಭಾಗ ಸತ್ಯಯುಗಕ್ಕೆ ಸೇರಿದ್ದು, ಆ ಭಾಗ ಕಲಿಯುಗಕ್ಕೆ ಸೇರಿದ್ದು, ಎಂದೆಲ್ಲ ಹೇಳಬಹುದು. ಆದರೆ ಇದನ್ನು ನೆನಪಿನಲ್ಲಿಡಿ. ವೇದಗಳು ಈ ರೀತಿ ಹೇಳುವುದಿಲ್ಲ. ಪುರಾಣಗಳು ಮಾತ್ರ ಹೀಗೆ ಹೇಳುವುವು. ಸೇವಕರು ಸ್ವಾಮಿಗೆ ಆಜ್ಞೆ ಮಾಡಬಲ್ಲನೆ? ಸ್ಮೃತಿ, ಪುರಾಣ, ತಂತ್ರ ಇವೆಲ್ಲ ಎಲ್ಲಿಯವರೆಗೆ ವೇದಕ್ಕೆ ಅಧೀನವಾಗಿರುವುವೋ, ಅಲ್ಲಿಯವರೆಗೆ ಇವನ್ನು ಒಪ್ಪಿಕೊಳ್ಳಬಹುದು. ಎಲ್ಲಿ ಅವು ವೇದಗಳನ್ನು ವಿರೋಧಿಸುವುವೋ, ಅಲ್ಲಿ ಅವನ್ನು ನಿರಾಕರಿಸಬಹುದು. ಆದರೆ ಈಗಿನ ಕಾಲದಲ್ಲಿ ನಾವು ಪುರಾಣಗಳಿಗೆ ವೇದಕ್ಕಿಂತ ಮೇಲಿನ ಸ್ಥಾನವನ್ನು ಕೊಟ್ಟಿರುವೆವು. ವೇದಾಧ್ಯಯನವು ವಂಗದೇಶದಿಂದ ಸಂಪೂರ್ಣ ಮಾಯವಾಗಿದೆ. ಪ್ರತಿಯೊಂದು ಮನೆಯಲ್ಲಿಯೂ ಸಾಲಿಗ್ರಾಮದೊಂದಿಗೆ ವೇದವನ್ನು ಇಟ್ಟು ಯುವಕರು ವೃದ್ಧರು ಸ್ತ್ರೀಯರು ಎಲ್ಲರೂ ಅದನ್ನು ಆರಾಧಿಸುವ ಸುದಿನ ಬೇಗ ಬರಲೆಂದು ನಾನು ಹಾರೈಸುತ್ತೇನೆ.

\vskip 4pt

ವೇದಗಳನ್ನು ಕುರಿತಂತೆ ಪಾಶ್ಚಾತ್ಯ ವಿದ್ವಾಂಸರ ಸಿದ್ಧಾಂತಗಳಲ್ಲಿ ನನಗೆ ನಂಬಿಕೆಯಿಲ್ಲ. ಇಂದು ವೇದಗಳಿಗೆ ಒಂದು ನಿರ್ದಿಷ್ಟ ಕಾಲವನ್ನು ನಿಗದಿಪಡಿಸುವರು, ನಾಳೆ ಅದನ್ನು ಬದಲಾಯಿಸಿ ಒಂದು ಸಾವಿರ ವರ್ಷ ಮುಂದೆ ತಳ್ಳುವರು. ಹೀಗೆಯೇ ಮಾಡುತ್ತಿರುವರು. ಪುರಾಣಗಳು ಎಲ್ಲಿಯವರೆಗೆ ವೇದಗಳನ್ನು ಒಪ್ಪುವುವೋ, ಅಲ್ಲಿಯವರೆಗೆ ಅವಕ್ಕೆ ಪ್ರಮಾಣ, ಇಲ್ಲದಿದ್ದರೆ ಇಲ್ಲ. ವೇದಭಾವನೆಗೆ ವಿರುದ್ಧವಾದ ಹಲವು ವಿಷಯಗಳು ಪುರಾಣಗಳಲ್ಲಿವೆ. ಪುರಾಣಗಳಲ್ಲಿ ಒಬ್ಬ ಹತ್ತುಸಾವಿರ ವರುಷ ಬದುಕಿದನೆಂದೂ, ಮತ್ತೊಬ್ಬ ಇಪ್ಪತ್ತು ಸಾವಿರ ವರುಷ ಬದುಕಿದನೆಂದೂ, ಇದೆ. ಆದರೆ ವೇದದಲ್ಲಿ ಶತಾಯುರ್ವೈ ಪುರುಷಃ– “ಮನುಷ್ಯನ ಆಯಸ್ಸು ನೂರು ವರುಷ” ಎಂದಿದೆ. ಇಂತಹ ಸಂದರ್ಭದಲ್ಲಿ ನಾವೂ ಯಾವುದನ್ನು ಒಪ್ಪಿಕೊಳ್ಳುವುದು? ನಿಜವಾಗಿಯೂ ವೇದಗಳನ್ನು. ಇಂತಹ ಅಸಂಬದ್ಧ ವಿಷಯಗಳಿದ್ದರೂ ನಾವು ಪುರಾಣಗಳನ್ನು ನಿಕೃಷ್ಟ ದೃಷ್ಟಿಯಿಂದ ನೋಡುವುದಿಲ್ಲ. ಅವುಗಳಲ್ಲಿ ಯೋಗ, ಭಕ್ತಿ, ಜ್ಞಾನ, ಕರ್ಮ ಮುಂತಾದವುಗಳಿಗೆ ಸಂಬಂಧಿಸಿದ ಹಲವು ಸುಂದರವಾದ ಧಾರ್ಮಿಕ ಉಪದೇಶಗಳಿವೆ. ನಾವು ಇವನ್ನೇನೋ ಸ್ವೀಕರಿಸಲೇಬೇಕು. ಅನಂತರ ತಂತ್ರಗಳಿವೆ. ‘ತಂತ್ರ’ ಎಂಬ ಶಬ್ದದ ನಿಜವಾದ ಅರ್ಥ ‘ಶಾಸ್ತ್ರ’ ಎಂದು, ಉದಾಹರಣೆಗೆ ಕಾಪಿಲ ತಂತ್ರ. ಆದರೆ ತಂತ್ರ ಶಬ್ದವನ್ನು ಸಾಮಾನ್ಯವಾಗಿ ಸೀಮಿತ ಅರ್ಥದಲ್ಲಿ ಉಪಯೋಗಿಸುತ್ತಾರೆ. ಬೌದ್ಧಧರ್ಮವನ್ನು ಸ್ವೀಕರಿಸಿ ಅಹಿಂಸಾತತ್ತ್ವವನ್ನು ಪ್ರಚಾರ ಮಾಡುತ್ತಿದ್ದ ರಾಜರುಗಳ ಆಧಿಪತ್ಯದಲ್ಲಿ ವೈದಿಕ ಯಾಗಯಜ್ಞಗಳ ಆಚರಣೆ ಗತಕಾಲದ ವಿಷಯವಾಯಿತು. ಅರಸರಿಗೆ ಅಂಜಿ ಯಾಗ ಸಮಯದಲ್ಲಿ ಪ್ರಾಣಿಗಳನ್ನು ಕೊಲ್ಲುತ್ತಿರಲಿಲ್ಲ. ಆದರೆ ಕ್ರಮೇಣ ಹಿಂದೂ ಧರ್ಮದಿಂದ ಬೌದ್ಧಧರ್ಮಕ್ಕೆ ಮತಾಂತರಗೊಂಡವರೇ ಯಾಗಯಜ್ಞಗಳ ಉತ್ತಮಾಂಶಗಳನ್ನು ತೆಗೆದುಕೊಂಡು ರಹಸ್ಯವಾಗಿ ಅವುಗಳನ್ನು ಅಭ್ಯಸಿಸುತ್ತಿದ್ದರು. ಇದರಿಂದಲೇ ತಂತ್ರಶಾಸ್ತ್ರ ಹುಟ್ಟಿತು. ವಾಮಾಚಾರ ಮುಂತಾದ ಕೆಲವು ಹೀನ ಅನುಷ್ಠಾನಗಳನ್ನು ಬಿಟ್ಟರೆ ತಂತ್ರವು ಜನರು ತಿಳಿದುಕೊಂಡಿರುವಷ್ಟು ಕೆಟ್ಟದೇನೂ ಅಲ್ಲ. ಅವುಗಳಲ್ಲಿ ಎಷ್ಟೋ ಉತ್ತಮವಾದ ಗಹನ ವೇದಾಂತ ಭಾವನೆಗಳಿವೆ. ನಿಜವಾಗಿಯೂ ವೇದಗಳಲ್ಲಿರುವ ಬ್ರಾಹ್ಮಣ ಭಾಗವನ್ನು ಸ್ವಲ್ಪ ವ್ಯತ್ಯಾಸಮಾಡಿ ತಂತ್ರಶಾಸ್ತ್ರದಲ್ಲಿ ಸೇರಿಸಿರುವರು. ಈಗಿನ ಕಾಲದಲ್ಲಿ ಆಚರಣೆಯಲ್ಲಿರುವ, ಕರ್ಮಕಾಂಡಕ್ಕೆ ಸೇರಿದ ಹಲವು ಪೂಜೆ ಮತ್ತು ವಿಧಿವಿಧಾನಗಳನ್ನು ತಂತ್ರಶಾಸ್ತ್ರದಲ್ಲಿರುವಂತೆಯೇ ಆಚರಿಸುವರು.

\newpage

ಈಗ ನಮ್ಮ ಧಾರ್ಮಿಕ ತತ್ತ್ವಗಳನ್ನು ಕುರಿತು ಸ್ವಲ್ಪ ವಿಮರ್ಶಿಸೋಣ. ನಮ್ಮ ವಿಭಿನ್ನ ಮತಗಳ ನಡುವೆ ಎಷ್ಟೋ ವ್ಯತ್ಯಾಸಗಳು ಮತ್ತು ವಿರೋಧಗಳು ಇದ್ದರೂ ಎಲ್ಲಕ್ಕೂ ಸಮಾನವಾದ ಅನೇಕ ವಿಷಯಗಳಿವೆ. ಹೆಚ್ಚಿನ ಎಲ್ಲ ಮತದವರೂ ಈಶ್ವರ, ಆತ್ಮ ಮತ್ತು ಜಗತ್​–ಈ ಮೂರರ ಅಸ್ತಿತ್ವಗಳನ್ನು ಒಪ್ಪುತ್ತಾರೆ. ಈಶ್ವರನೇ ಸೃಷ್ಟಿ–ಸ್ಥಿತಿ–ಪ್ರಳಯಗಳಿಗೆ ಕಾರಣ. ಸಾಂಖ್ಯರೊಬ್ಬರನ್ನು ಬಿಟ್ಟರೆ ಉಳಿದವರೆಲ್ಲರೂ ಇದನ್ನು ಒಪ್ಪಿಕೊಳ್ಳುವರು. ಅನಂತರ ಆತ್ಮ ಮತ್ತು ಪುನರ್ಜನ್ಮ ತತ್ತ್ವವಿದೆ. ಅಸಂಖ್ಯ ಜೀವಿಗಳು ತಮ್ಮ ಕರ್ಮಕ್ಕೆ ತಕ್ಕಂತೆ ಒಂದಾದಮೇಲೊಂದು ದೇಹವನ್ನು ಧರಿಸುತ್ತ ಜನನ ಮರಣಗಳ ಚಕ್ರದಲ್ಲಿ ಸಾಗಿ ಹೋಗುತ್ತಿವೆ. ಇದೇ ಸಂಸಾರ ವಾದ ಅಥವಾ ಪುನರ್ಜನ್ಮ ವಾದ. ಅನಂತರ ಆದಿ ಅಂತ್ಯಗಳಿಲ್ಲದ ಜಗತ್ತೊಂದು ಇದೆ. ಕೆಲವರು, ಈ ಮೂರು ಒಂದೇ ಸತ್ಯದ ವಿವಿಧ ಮುಖಗಳು ಎಂತಲೂ, ಕೆಲವರು ಈ ಮೂರು ಸಂಪೂರ್ಣ ಬೇರೆ ಬೇರೆ ಎಂತಲೂ, ಉಳಿದವರು ಇನ್ನೂ ಬೇರೆ ಬೇರೆ ರೀತಿ ಹೇಳಿದರೂ ಎಲ್ಲರೂ ಈ ಮೂರನ್ನೂ ಒಪ್ಪಿಕೊಳ್ಳುವುದರಲ್ಲಿ ಏಕಮತೀಯರು.

\vskip 4pt

ಬಹಳ ಹಿಂದಿನ ಕಾಲದಿಂದಲೂ ಹಿಂದೂಗಳಿಗೆ, ಆತ್ಮ ಎಂದರೆ ಮನಸ್ಸಿಗಿಂತ ಬೇರೆ ಎಂದು ತಿಳಿದಿತ್ತು. ಆದರೆ ಪಾಶ್ಯಾತ್ಯರಿಗೆ ಮನಸ್ಸನ್ನು ಮೀರಿ ಹೋಗಲಾಗಲಿಲ್ಲ. ಪಾಶ್ಚಾತ್ಯರ ದೃಷ್ಟಿಯಲ್ಲಿ ಈ ಜಗತ್ತು ಸುಖಮಯವಾದುದು, ಇದು ಅವರಿಗೆ ಭೋಗಭೂಮಿ. ಆದರೆ ಪ್ರಾಚ್ಯರು, ಈ ಸಂಸಾರವು ಅನಿತ್ಯ, ದುಃಖಮಯ, ಮಿಥ್ಯೆ; ಆತ್ಮನನ್ನು ಇಲ್ಲಿಯ ತಾತ್ಕಾಲಿಕ ಸುಖಭೋಗಗಳಿಗೆ ಮಾರಿಕೊಳ್ಳಕೂಡದು ಎಂದು ಭಾವಿಸುವರು. ಆದಕಾರಣವೇ ಪಾಶ್ಚಾತ್ಯರು ಸಂಘಬದ್ಧ ಕರ್ಮಗಳಲ್ಲಿ ಪಟುಗಳು. ಪ್ರಾಚ್ಯರು ಅಂತರ್ಜಗತ್ತಿನ ಅನ್ವೇಷಣೆಯಲ್ಲಿ ಹೆಚ್ಚು ಸಾಹಸವನ್ನು ತೋರುವರು.

\vskip 4pt

ಈಗ ಹಿಂದೂಧರ್ಮದ ಉಳಿದ ಒಂದೆರಡು ವಿಷಯಗಳನ್ನು ತೆಗೆದು ಕೊಳ್ಳೋಣ. ಭಗವಂತನ ಅವತಾರ ಸಿದ್ಧಾಂತವಿದೆ. ವೇದಗಳಲ್ಲಿ ಕೇವಲ ಮತ್ಸ್ಯಾವತಾರದ ಪ್ರಸ್ತಾಪವಿದೆ. ಎಲ್ಲರೂ ಭಗವಂತನ ಅವತಾರ ಸಿದ್ಧಾಂತವನ್ನು ನಂಬುತ್ತಾರೊ ಅಥವಾ ಇಲ್ಲವೊ ಎನ್ನವುದಲ್ಲ ಮುಖ್ಯ ವಿಷಯ. ಆದರೆ ಈ ಅವತಾರವಾದದ ನಿಜವಾದ ಅರ್ಥವೇ ಮಾನವ ಪೂಜೆ. ಮಾನವನಲ್ಲಿ ದೇವರನ್ನು ನೋಡುವುದೇ ನಿಜವಾದ ದೇವರ ದರ್ಶನ. ಹಿಂದೂಗಳು ಪ್ರಕೃತಿಯ ಮೂಲಕ ಪ್ರಕೃತಿಯ ದೇವನೆಡೆಗೆ ಹೋಗುವುದಿಲ್ಲ, ಮಾನವನ ಮೂಲಕ ಮಾನವ ದೇವನೆಡೆಗೆ ಹೋಗುವರು.

\vskip 4pt

ಅನಂತರ ವಿಗ್ರಹಾರಾಧನೆ ಇದೆ. ಎಲ್ಲಾ ಪವಿತ್ರ ಕರ್ಮಗಳಲ್ಲಿಯೂ ಪಂಚ ದೇವತೆಗಳನ್ನು ಪೂಜೆಮಾಡಬೇಕೆಂದು ನಮ್ಮ ಶಾಸ್ತ್ರ ವಿಧಿಸಿರುವುದು. ಉಳಿದ ದೇವತೆಗಳೆಲ್ಲ ಆಯಾಯಾ ಸ್ಥಾನಗಳ ಹೆಸರುಗಳು. ಈ ಪಂಚದೇವತೆಗಳೂ ಕೂಡ ಒಂದೇದೇವರ ವಿವಿಧ ಹೆಸರುಗಳು. ಬಾಹ್ಯ ಪ್ರತಿಮಾ ಪೂಜೆಯು ಅತಿ ಕೆಳಗಿನ ದರ್ಜೆಯ ಪೂಜೆ ಎಂದು ನಮ್ಮ ಶಾಸ್ತ್ರಗಳೆಲ್ಲಾ ಸಾರುವುವು. ಆದರೆ ಹೀಗೆ ಪೂಜಿಸುವುದು ತಪ್ಪು ಎಂದು ಅರ್ಥವಲ್ಲ. ಈಗ ರೂಢಿಯಲ್ಲಿರುವ ಪ್ರತಿಮಾಪೂಜೆಯಲ್ಲಿ ಎಷ್ಟೋ ಅನಾವಶ್ಯಕವಾಗಿರುವುದು ಸೇರಿದ್ದರೂ ನಾನು ಇದನ್ನು ಖಂಡಿಸುವುದಿಲ್ಲ. ಆ ಸಂಪ್ರದಾಯಶೀಲ ವಿಗ್ರಹಾರಾಧಕ ಬ್ರಾಹ್ಮಣನ ಪಾದರಜ ನನ್ನ ಭಾಗ್ಯಕ್ಕೆ ದೊರಕದೇ ಇದ್ದಿದ್ದರೆ ನಾನು ಎಲ್ಲಿರುತ್ತಿದ್ದೆ!

ವಿಗ್ರಹಾರಾಧನೆಯನ್ನು ದೂರುವ ಸುಧಾರಕರಿಗೆ ನಾನು ಹೀಗೆ ಹೇಳುತ್ತೇನೆ:\break “ಸಹೋದರರೇ! ಬಾಹ್ಯ ಆಚಾರಗಳು ಮತ್ತು ಸಹಾಯಗಳನ್ನೆಲ್ಲಾ ತ್ಯಜಿಸಿ ನೀವು ಭಗವಂತನನ್ನು ನಿರಾಕಾರವಾಗಿ ಆರಾಧಿಸಲು ನಿಮಗೆ ಶಕ್ತಿ ಇದ್ದರೆ ಮಾಡಿ. ಹೀಗೆ ಯಾರಿಗೆ ಮಾಡುವುದಕ್ಕೆ ಸಾಧ್ಯವಿಲ್ಲವೋ, ಅವರನ್ನು ಏಕೆ ದೂರುವಿರಿ? ಒಂದು ಸುಂದರ ವಿಶಾಲ ಸೌಧ, ಪ್ರಾಚೀನ ಕಾಲದ ವೈಭವಾನ್ವಿತ ಅವಶೇಷ, ಅದನ್ನು ಉಪಯೋಗಿಸದೆ, ಸರಿಯಾಗಿ ನೋಡಿಕೊಳ್ಳದೆ, ಈಗ ಪಾಳುಬಿದ್ದಿದೆ. ಅದರಲ್ಲೆಲ್ಲಾ ಕೊಳೆ ಧೂಳು ಕವಿದಿರಬಹುದು. ಅದರ ಕೆಲವು ಭಾಗಗಳು ಕುಸಿಯುತ್ತಲೂ ಇರಬಹುದು. ಅದಕ್ಕೆ ನೀವೇನು ಮಾಡುವಿರಿ? ಅದನ್ನು ಶುಚಿಮಾಡಿ ಅವಶ್ಯಕವಾದ ದುರಸ್ತಿ ಮಾಡುವಿರೋ ಅಥವಾ ಅದನ್ನೆಲ್ಲಾ ಧ್ವಂಸಮಾಡಿ, ಇನ್ನೂ ಸ್ಥಿರತೆಯನ್ನು ಸಾಧಿಸಬೇಕಾಗಿರುವ ಕ್ಷುದ್ರ ಆಧುನಿಕ ಯೋಜನೆಯಂತೆ, ಒಂದು ಕೆಲಸಕ್ಕೆ ಬಾರದ ಕಟ್ಟಡವನ್ನು ಅದರ ಸ್ಥಳದಲ್ಲಿ ಕಟ್ಟಲು ಯತ್ನಿಸುವಿರೊ? ನಾವು ಅದನ್ನು ಸುಧಾರಿಸಬೇಕು, ಎಂದರೆ ಅದನ್ನು ಕ್ರಮೇಣ ಶುದ್ಧಿಮಾಡಿ, ಸರಿಪಡಿಸಿ, ವಾಸಯೋಗ್ಯವನ್ನಾಗಿ ಮಾಡಬೇಕು. ಇಡಿ ಕಟ್ಟಡವನ್ನೇ ಧ್ವಂಸಮಾಡುವುದಲ್ಲ. ಅಲ್ಲಿಗೆ ಸುಧಾರಣೆ ಕೆಲಸ ಕೊನೆಗಾಣುವುದು. ಹಳೆಯ ಕಟ್ಟಡವನ್ನು ಸರಿಮಾಡಿದ ಮೇಲೆ ಆ ಸುಧಾರಣೆಯಿಂದ ಮತ್ತೇನು ಪ್ರಯೋಜನ? ಈ ಸುಧಾರಣೆಯನ್ನು ಸಾಧ್ಯವಾದರೆ ಮಾಡಿ, ಇಲ್ಲದಿದ್ದರೆ ಸುಮ್ಮನಿರಿ.” ನಮ್ಮ ದೇಶದ ಸುಧಾರಕರ ತಂಡದವರು ತಮ್ಮದೇ ಒಂದು ಪಂಗಡವನ್ನು ಸ್ಥಾಪಿಸಲು ಯತ್ನಿಸುವರು. ಅವರು ಒಳ್ಳೆಯ ಕೆಲಸವನ್ನೇನೋ ಮಾಡಿರುವರು. ಅದಕ್ಕೆ ಭಗವಂತನು ಅವರನ್ನು ಆಶೀರ್ವದಿಸಲಿ! ಹಿಂದೂಗಳಾದ ನೀವು ಹಿಂದೂ ಸಮುದಾಯದಿಂದ ಏತಕ್ಕೆ ಬೇರೆಯಾಗಬೇಕು? ಹಿಂದೂಗಳೆಂದು ಹೇಳಿಕೊಳ್ಳುವುದಕ್ಕೆ ಏತಕ್ಕೆ ಲಜ್ಜೆ? ಹಿಂದುತ್ವವೇ ನಮ್ಮ ಶ್ರೇಷ್ಠ ಅನರ್ಘ್ಯ ಆಸ್ತಿ.

ನಮ್ಮ ದೇಶಬಾಂಧವರೇ, ಅಮೃತ ಪುತ್ರರೆ! ಈ ಜನಾಂಗದ ಹಡಗು ಹಲವು ಶತಮಾನಗಳಿಂದ ಸಂಚರಿಸುತ್ತಿದೆ. ಅನರ್ಘ್ಯ ರತ್ನದಂತಿರುವ ಸಂಸ್ಕೃತಿಯನ್ನು ಇತರ ದೇಶಗಳಿಗೆ ಹೊತ್ತು ಜಗತ್ತನ್ನೇ ಉತ್ತಮಸ್ಥಾನಕ್ಕೆ ತರುತ್ತಿದೆ. ಈ ಜನಾಂಗವೆಂಬ ನಮ್ಮ ನೌಕೆ ಹಲವು ಶತಮಾನಗಳಿಂದ ಕೋಟ್ಯಂತರ ಜೀವಿಗಳನ್ನು ಭವಸಾಗರದ ಆಚೆ ದುಃಖಾತೀತ ಸ್ಥಿತಿಗೆ ಕೊಂಡೊಯ್ದಿದೆ. ಆದರೆ ಇಂದು ನಮ್ಮ ತಪ್ಪಿನಿಂದಲೋ, ಅಥವಾ ಮತ್ತಾವ ಕಾರಣದಿಂದಲೋ, ಈ ಹಡಗು ಸೋರಲು ಮೊದಲಾಗಿ ಅದಕ್ಕೆ ಅಪಾಯವೊದಗಿರಬಹುದು. ಅದರಲ್ಲಿರುವ ನೀವು ಈಗ ಏನು ಮಾಡಬಲ್ಲಿರಿ? ಅದನ್ನು ದೂರುತ್ತ ನಿಮ್ಮ ನಿಮ್ಮಲ್ಲೇ ಜಗಳ ಕಾಯುವಿರೇನು? ನೀವೆಲ್ಲ ಕಲೆತು ಆ ರಂಧ್ರವನ್ನು ಮುಚ್ಚಲು ಸಾಧ್ಯವಾದಷ್ಟು ಪ್ರಯತ್ನ ಪಡುವುದಿಲ್ಲವೇ? ನಮ್ಮ ಹೃದಯದ ರಕ್ತವನ್ನು ಸಂತೋಷದಿಂದ ಇದಕ್ಕೆ ಅರ್ಪಿಸೋಣ. ನಾವು ಪ್ರಯತ್ನದಲ್ಲಿ ಸೋತರೆ ನಾವೆಲ್ಲಾ ಒಟ್ಟಿಗೆ ಕಲೆತು, ಶಾಪ ಕೊಡದೆ, ಆಶೀರ್ವಾದವನ್ನು ಉಚ್ಚರಿಸುತ್ತ ಮುಳುಗೋಣ.

ಬ್ರಾಹ್ಮಣರೇ! ನಿಮ್ಮ ಜಾತಿಕುಲಗಳ ಹೆಮ್ಮೆ ವ್ಯರ್ಥ. ಅದರಿಂದ ಪಾರಾಗಿ. ನಿಮ್ಮ ಶಾಸ್ತ್ರ ಪ್ರಕಾರದ ಬ್ರಾಹ್ಮಣ್ಯ ನಿಮ್ಮಲ್ಲಿ ಇಂದು ಇಲ್ಲ. ಏಕೆಂದರೆ ಮ್ಲೇಚ್ಛರ ಕೈಕೆಳಗೆ ನೀವು ಬಹಳ ಕಾಲದಿಂದ ಇರುವಿರಿ. ನಿಮ್ಮ ಪೂರ್ವಿಕರ ಮಾತಿನಲ್ಲಿ ನಿಮಗೆ ಇನ್ನೂ ನಂಬಿಕೆ ಇದ್ದರೆ, ಈ ಕ್ಷಣವೇ– ಹಿಂದೆ ಕುಮಾರಿಲನು, ಒಬ್ಬ ಬೌದ್ಧ ಗುರುವಿನಲ್ಲಿ ಕಲಿತು, ಅನಂತರ ಅವರನ್ನು ಸೋಲಿಸಿ, ಹಲವು ಬೌದ್ಧರ ಸಾವಿಗೆ ಕಾರಣನಾದೆನೆಂದು, ಅದರ ಪ್ರಾಯಶ್ಚಿತ್ತಕ್ಕಾಗಿ ದಹಿಸುತ್ತಿದ್ದ ಹೊಟ್ಟಿನ ರಾಶಿಯಲ್ಲಿ ಬಿದ್ದಂತೆ–ನೀವೂ ಬಿದ್ದು ಪ್ರಾಯಶ್ಚಿತ್ತ ಮಾಡಿಕೊಳ್ಳಿ. ಅದನ್ನು ಮಾಡುವುದಕ್ಕೆ ನಿಮಗೆ ಧೈರ್ಯವಿಲ್ಲದೇ ಇದ್ದರೆ, ನಿಮ್ಮ ದೌರ್ಬಲ್ಯವನ್ನು ಒಪ್ಪಿಕೊಂಡು ಎಲ್ಲರಿಗೂ ಸಹಾಯಮಾಡಿ, ಜ್ಞಾನಾಗಾರದ ಬಾಗಿಲನ್ನು ಎಲ್ಲರಿಗೂ ತೆರೆಯಿರಿ, ತುಳಿತಕ್ಕೊಳಗಾದ ಜನ ಸಾಮಾನ್ಯರಿಗೆ ಮತ್ತೊಮ್ಮೆ ಅವರಿಗೆ ನ್ಯಾಯವಾಗಿ ಸಲ್ಲುವ ಹಕ್ಕು ಬಾಧ್ಯತೆಗಳನ್ನು ಕೊಡಿ.

