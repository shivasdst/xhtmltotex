
\chapter{ಭಾರತದ ಭವಿಷ್ಯ}

\begin{center}
(ಮದ್ರಾಸಿನ ಉಪನ್ಯಾಸ)
\end{center}

ಜಗತ್ತಿನಲ್ಲಿ ಜ್ಞಾನವು ಮೊಟ್ಟಮೊದಲು ಉದಯಿಸಿದ್ದು ಈ ಪುರಾತನ ಭೂಮಿಯಲ್ಲಿಯೇ. ಆಧ್ಯಾತ್ಮಿಕ ಪ್ರವಾಹದ ಸ್ಥೂಲ ಪ್ರತೀಕದಂತೆ ಸಮುದ್ರೋಪಮವಾದ ನದಿಗಳು ಹರಿಯುತ್ತಿರುವುದು ಈ ನಾಡಿನಲ್ಲಿಯೇ. ಇಲ್ಲಿ ಶ್ರೇಣಿಶ್ರೇಣಿಯಾಗಿ ಪಸರಿಸಿರುವ ತುಷಾರಭೂಷಣ ಹಿಮಾಲಯ ಪರ್ವತಸ್ತೋಮವು ಗಗನ ರಹಸ್ಯವನ್ನು ಭೇದಿಸುವಂತೆ ಇದೆ. ಋಷಿಗಳ, ಮಹಾಮುನಿಗಳ ಪಾದಧೂಳಿ ಸೋಂಕಿದ ಪವಿತ್ರ ಭೂಮಿ ಇದು. ಮಾನವನ ಸ್ವರೂಪ ಮತ್ತು ಅಂತರ್ಜಗತ್ತನ್ನು ಕುರಿತು ಜಿಜ್ಞಾಸೆ ಮೊದಲಾದುದು ಇಲ್ಲಿ. ಆತ್ಮದ ಅಮರತ್ವ, ಅಂತರ್ಯಾಮಿ ಈಶ್ವರ, ಜಗತ್ತು ಮತ್ತು ಜೀವಿಗಳಲ್ಲಿ ಓತಪ್ರೋತನಾಗಿರುವ ಪರಮಾತ್ಮನ ಕಲ್ಪನೆ, ಧರ್ಮದ ಮತ್ತು ತತ್ತ್ವದ ಅತಿ ಶ್ರೇಷ್ಠ ಭಾವನೆಗಳು ಇಲ್ಲಿ ತಮ್ಮ ಪರಾಕಾಷ್ಠೆಯನ್ನು ಮುಟ್ಟಿದವು. ಇಲ್ಲಿಂದ ಮಹಾಸಾಗರದ ಅಲೆಗಳಂತೆ ತತ್ತ್ವ ಮತ್ತು ಅಧ್ಯಾತ್ಮ ಭಾವನೆಗಳು ಪುನಃ ಪುನಃ ಹೊರಟು ಇಡೀ ಪ್ರಪಂಚವನ್ನು ತೋಯಿಸಿರುವುವು. ಅವನತಿಗೆ ಇಳಿಯುತ್ತಿರುವ ಮಾನವ ಜನಾಂಗಕ್ಕೆ ನವಚೇತನವನ್ನು ತುಂಬಲು ಮತ್ತೊಮ್ಮೆ ಆ ಪ್ರವಾಹವು ಉಕ್ಕಿ ಹರಿಯಬೇಕಾಗಿದೆ. ಶತಶತಮಾನಗಳ ಆಘಾತವನ್ನು, ವಿದೇಶೀಯರ ನೂರಾರು ಆಕ್ರಮಣಗಳನ್ನು, ನೂರಾರು ಆಚಾರ ವ್ಯವಹಾರಗಳ ಘರ್ಷಣೆಯನ್ನು ಸಹಿಸಿದ ದೇಶ ಇದು. ಅವಿನಾಶಿಯಾದ ವೀರ್ಯದಿಂದ, ಮತ್ತು ನಾಶಗೊಳಿಸಲು ಸಾಧ್ಯವೇ ಇಲ್ಲದ ಪ್ರಾಣಶಕ್ತಿಯಿಂದ ಹೇಗೆ ಅನಾದಿ ಅನಂತ ಅಮೃತ ಸ್ವರೂಪವಾಗಿದೆಯೋ ಹಾಗೆಯೇ ನಮ್ಮ ಭಾರತ ಭೂಮಿ ಕೂಡ. ನಾವು ಅಂಥ ದೇಶದ ಸಂತಾನರು.

ಭಾರತ ಪುತ್ರರೇ, ನಾನಿಂದು ಅನುಷ್ಠಾನಕ್ಕೆ ಸಂಬಂಧಿಸಿದ ಕೆಲವು ವಿಷಯಗಳನ್ನು ಹೇಳುವೆನು. ನಮ್ಮ ಭರತಖಂಡದ ಗತವೈಭವದ ನೆನಪನ್ನು ನಿಮಗೆ ಉಂಟುಮಾಡುವ ಉದ್ದೇಶ ಇದು. ಅನೇಕ ವೇಳೆ ಈ ಗತಕಾಲದ ವೈಭವವನ್ನು ಕುರಿತು ಚಿಂತಿಸುವುದು ವಿನಾಶಕಾರಿ, ಅದರಿಂದ ಪ್ರಯೋಜನವಿಲ್ಲ, ನಾವು ಭವಿಷ್ಯದತ್ತ ಮಾತ್ರ ನೋಡಬೇಕು ಎಂದು ಹೇಳುವರು. ಅದೇನೋ ಸತ್ಯ. ಆದರೆ ಹಿಂದಿನದರ ಮೇಲೆ ಮಾತ್ರ ಮುಂದೆ ಬರಲಿರುವುದನ್ನು ನಿರ್ಮಿಸಬಹುದು. ಆದ್ದರಿಂದ ನಿಮಗೆ, ಸಾಧ್ಯವಾಗುವಷ್ಟೂ ಹಿಂದೆ ನೋಡಿ, ಹಿಂದಿನ ಅಮೃತ ಚಿಲುಮೆಯಿಂದ ತೃಪ್ತರಾಗುವವರೆಗೆ ಪಾನ ಮಾಡಿ, ಅನಂತರ ಮುಂದೆ ನೋಡಿ, ಮುಂದೆ ಅಡಿ ಇಡಿ, ಭರತಖಂಡವನ್ನು ಹಿಂದಿಗಿಂತ ಉಜ್ವಲತರಮಾಡಿ, ಮಹತ್ತರವಾದುದನ್ನಾಗಿ ಮಾಡಿ ಉನ್ನತ ಶಿಖರಕ್ಕೆ ಕೊಂಡೊಯ್ಯಿರಿ ಎಂದು ಹೇಳುತ್ತೇನೆ. ನಮ್ಮ ಪೂರ್ವಜರು ಮಹಾಮಹಿಮರಾಗಿದ್ದರು. ಅದನ್ನು ಮೊದಲು ಸ್ಮರಿಸಬೇಕು. ನಮ್ಮ ವ್ಯಕ್ತಿತ್ವವು ಯಾವುದರಿಂದ ಆಗಿದೆ ಎಂಬುದನ್ನೂ ನಮ್ಮ ನಾಡಿನಲ್ಲಿ ಎಂತಹ ರಕ್ತ ಹರಿಯುತ್ತಿದೆ ಎಂಬುದನ್ನೂ ಮೊದಲು ಅರಿಯಬೇಕು. ಅದು ಹಿಂದೆ ಏನನ್ನು ಸಾಧಿಸಿತು ಎನ್ನುವುದರ ಮೇಲೆ ಶ್ರದ್ಧೆ ಇರಬೇಕು. ಆ ಗತವೈಭವದ ಅರಿವಿನಿಂದ ಮತ್ತು ಅದರ ಮೇಲಿನಶ್ರದ್ಧೆಯಿಂದ ಹಿಂದಿಗಿಂತ ಹೆಚ್ಚು ವೈಭವಯುತವಾದ ಭರತಖಂಡವನ್ನು ನಿರ್ಮಿಸಬೇಕು. ಹಿಂದೆ ದುರ್ದೆಶೆಯ ಮತ್ತು ಅವನತಿಯ ಸಮಯ ಇತ್ತು. ನಮಗೆ ಅದೆಲ್ಲ ಗೊತ್ತಿದೆ. ಅದಕ್ಕೆ ಹೆಚ್ಚು ಪ್ರಾಮುಖ್ಯವನ್ನು ನಾನು ಕೊಡುವುದಿಲ್ಲ. ಅಂತಹ ಸಮಯಗಳು ಅವಶ್ಯವಾಗಿದ್ದವು. ವಿಶಾಲವೃಕ್ಷವು ಸುಂದರವಾದ ಮಾಗಿದ ಫಲವನ್ನು ಕೊಡುವುದು. ಅದು ನೆಲಕ್ಕೆ ಬಿದ್ದು, ಕೊಳೆತು, ನಾಶವಾಗುವುದು. ಆ ನಾಶವೇ ಭವಿಷ್ಯದ ತರುವು ಬೇರು ಬಿಡುವುದಕ್ಕೆ ಕಾರಣವಾಗುವುದು. ಮುಂದಿನದು ಹಿಂದಿನದಕ್ಕಿಂತ ಬಹುಶಃ ಉತ್ತಮತರ ತರುವಾಗಬಹುದು. ನಾವೆಲ್ಲ ಸಾಗಿ ಬಂದ ಅವನತಿಯ ಕಾಲವು ಅವಶ್ಯಕವಾಗಿತ್ತು. ಈ ಅವನತಿಯಿಂದ ಭವಿಷ್ಯ ಭಾರತ ಉದ್ಭವಿಸುವುದು. ಅದಾಗಲೇ ಚಿಗುರುತ್ತಿದೆ. ಪ್ರಥಮ ತಳಿರುಗಳಾಗಲೇ ಕಾಣುತ್ತಿವೆ. ಊರ್ಧ್ವಮೂಲದ ಭೀಮ ವೃಕ್ಷವದು ಆಗಲೇ ಗೋಚರಿಸುತ್ತಿದೆ. ನಾನಿಂದು ಮಾತನಾಡುವುದು ಅದರ ವಿಷಯವಾಗಿಯೇ.

ಭರತಖಂಡದ ಸಮಸ್ಯೆಗಳು ಇತರ ದೇಶಗಳ ಸಮಸ್ಯೆಗಳಿಗಿಂತ ಹೆಚ್ಚು ಜಟಿಲ ಮತ್ತು ಗುರುತರವಾಗಿವೆ. ಜನಾಂಗ, ಧರ್ಮ, ಭಾಷೆ, ಸರ್ಕಾರ ಇವುಗಳೆಲ್ಲ ಸೇರಿ ಒಂದು ರಾಷ್ಟ್ರವಾಗುವುದು. ಜಗತ್ತಿನ ಇತರ ಜನಾಂಗಗಳನ್ನು ನಮ್ಮ ಜನಾಂಗದೊಂದಿಗೆ ತುಲನೆಮಾಡಿ ನೋಡಿದರೆ, ಇತರ ಜನಾಂಗಗಳ ಸಂಘಟನೆಗೆ ಸೇರಿದ ಸಾಮಗ್ರಿಗಳು ನಮ್ಮ ದೇಶಕ್ಕಿಂತ ಕಡಮೆಯಾಗಿವೆ. ಇಲ್ಲಿ ಆರ್ಯ, ದ್ರಾವಿಡ, ಟಾರ್ಟರ್​, ತುರ್ಕಿ, ಮೊಗಲ್​, ಯೂರೋಪಿನ ರಾಷ್ಟ್ರಗಳು-ಈ ಎಲ್ಲ ಜನಾಂಗಗಳ ರಕ್ತವು ಹರಿಯುತ್ತಿದೆ. ಒಂದು ಅತಿ ವಿಚಿತ್ರವಾದ ಭಾಷಾಮೇಳವೇ ಇಲ್ಲಿದೆ. ಆಚಾರ ವ್ಯವಹಾರಗಳ ದೃಷ್ಟಿಯಿಂದ, ಯೂರೋಪಿಯನರಿಗೂ ಪ್ರಾಚ್ಯರಾಷ್ಟ್ರಗಳಿಗೂ ಇರುವ ಭಿನ್ನತೆಗಿಂತ ಹೆಚ್ಚು ಭಿನ್ನತೆಯು ಎರಡು ಭಾರತೀಯ ಜನಾಂಗಗಳ ನಡುವೆ ಇದೆ.

ನಮ್ಮ ಏಕಮಾತ್ರ ಸಮಾನ ಭೂಮಿಕೆಯೆಂದರೆ ಹಿಂದಿನಿಂದ ಬಂದ ಪವಿತ್ರ ಪರಂಪರೆ ಮತ್ತು ಧರ್ಮ. ಅದೊಂದೇ ಏಕಮಾತ್ರ ಸಾಧಾರಣ ಭೂಮಿ. ಅದರ ಮೇಲೆ ಮಾತ್ರ ನಮ್ಮ ರಾಷ್ಟ್ರವನ್ನು ನಿರ್ಮಿಸುವುದು ಸಾಧ್ಯ. ಯೂರೋಪಿನಲ್ಲಿ ರಾಜನೀತಿಯ ಭಾವನೆ ಜನಾಂಗದ ಏಕತೆಗೆ ಕಾರಣ. ಏಷ್ಯಾಖಂಡದಲ್ಲಿ ಧಾರ್ಮಿಕ ಆದರ್ಶಗಳು ಜನಾಂಗದ ಏಕತೆಗೆ ಮೂಲ. ಭವಿಷ್ಯದ ಭಾರತದ ಸಂಘಟನೆಗೆ ಧಾರ್ಮಿಕ ಏಕತೆ ಮೊದಲನೆಯ ಆವಶ್ಯಕತೆ. ಭರತಖಂಡದ ಆದ್ಯಂತವು ಒಂದು ಧರ್ಮವನ್ನೊಪ್ಪಿಕೊಳ್ಳಬೇಕು. ಒಂದು ಧರ್ಮವೆಂದರೆ ಏನು? ಕ್ರೈಸ್ತರು, ಮಹಮ್ಮದೀಯರು, ಬೌದ್ಧರು ತಿಳಿದುಕೊಂಡಿರುವಂತಹ ಒಂದು ಧರ್ಮವಲ್ಲ. ನಮ್ಮ ವಿಭಿನ್ನ ಸಂಪ್ರದಾಯಗಳಲ್ಲಿ, ಅವರವರ ಸಿದ್ಧಾಂತಗಳು ಪರಸ್ಪರ ಎಷ್ಟೇ ವಿರುದ್ಧವಾಗಿರಲಿ, ಅವುಗಳಲ್ಲೆಲ್ಲಾ ಒಂದು ಸರ್ವಸಾಮಾನ್ಯ ತಳಹದಿ ಇದೆ. ಈ ಧರ್ಮಗಳಲ್ಲಿ ಕೆಲವು ಸಾಮಾನ್ಯ ಭೂಮಿಕೆಗಳು ಇವೆ. ಅವುಗಳ ಪರಿಧಿಯಲ್ಲಿ ಈ ನಮ್ಮ ಧರ್ಮವು ಅನಂತ ವೈವಿಧ್ಯವನ್ನು ಒಪ್ಪಿಕೊಳ್ಳುತ್ತದೆ. ಅನಂತವಾದ ಆಲೋಚನಾ ಸ್ವಾತಂತ್ರ್ಯವನ್ನೂ ತಮ್ಮದೇ ರೀತಿಯಲ್ಲಿ ಜನರು ಬಾಳುವ ಅವಕಾಶವನ್ನೂ ಅದು ಒಪ್ಪಿಕೊಳ್ಳುತ್ತದೆ. ನಮಗೆಲ್ಲಾ ಇದು ಗೊತ್ತಿದೆ. ನಮ್ಮಲ್ಲಿ ಯಾರು ಸ್ವಲ್ಪ ಆಲೋಚನೆ ಮಾಡಬಲ್ಲರೋ ಅವರಿಗಾದರೂ ಇದು ಗೊತ್ತಿದೆ. ಪ್ರಾಣಶಕ್ತಿಯನ್ನು ನೀಡುವ ನಮ್ಮ ಧರ್ಮದ ಈ ಸರ್ವಸಾಮಾನ್ಯ ತತ್ತ್ವಗಳನ್ನು ಎಲ್ಲರಿಗೂ ಪ್ರಚಾರಮಾಡಬೇಕಾಗಿದೆ. ಈ ದೇಶದಲ್ಲಿರುವ ಸ್ತ್ರೀ, ಪುರುಷ, ಮಕ್ಕಳೆಲ್ಲರೂ ಇದನ್ನು ತಿಳಿದುಕೊಳ್ಳಲಿ, ತಮ್ಮ ನಿತ್ಯ ಜೀವನದಲ್ಲಿ ಅನುಷ್ಠಾನಕ್ಕೆ ತರಲಿ. ಇದೇ ನಮ್ಮ ಪ್ರಥಮ ಕರ್ತವ್ಯ. ಆದುದರಿಂದ ಇದನ್ನು ನಾವು ಸಾಧಿಸಬೇಕು.

ಏಷ್ಯಾಖಂಡದಲ್ಲಿ, ಅದರಲ್ಲೂ ಭರತಖಂಡದಲ್ಲಿ ಜನಾಂಗ, ಭಾಷೆ, ಸಮಾಜ, ದೇಶ ಇವುಗಳಿಗೆ ಸಂಬಂಧಪಟ್ಟ ಜಟಿಲ ಸಮಸ್ಯೆಗಳೆಲ್ಲಾ ಧಾರ್ಮಿಕ ಸಂಘಟನಾ ಶಕ್ತಿಯ ಎದುರಿಗೆ ಮಾಯವಾಗುವುದನ್ನು ನೋಡುವೆವು. ಭಾರತೀಯ ಮನಸ್ಸಿಗೆ ಧಾರ್ಮಿಕ ಆದರ್ಶಕ್ಕಿಂತ ಶ್ರೇಷ್ಠವಾದುದು ಯಾವುದೂ ಇಲ್ಲ ಎಂಬುದನ್ನು ನಾವು ಬಲ್ಲೆವು. ಧರ್ಮವೇ ಭಾರತ ಜೀವನದ ಮೂಲಮಂತ್ರ. ಎಲ್ಲಿ ವಿರೋಧವು ಅತಿ ಅಲ್ಪವಾಗಿರುವುದೋ ಅಲ್ಲಿ ಮಾತ್ರ ನಾವು ಕೆಲಸಮಾಡುವುದು ಸಾಧ್ಯ. ಭರತ ಖಂಡದಲ್ಲಿ ಧಾರ್ಮಿಕ ಆದರ್ಶವೇ ಸರ್ವೋಚ್ಚ ಆದರ್ಶವೆಂಬುದು ಸತ್ಯ ಮಾತ್ರವಲ್ಲ; ಇಲ್ಲಿ ಕೆಲಸಮಾಡುವುದಕ್ಕೆ ಇರುವುದು ಅದೊಂದೇ ಮಾರ್ಗ. ಅದನ್ನು ಮೊದಲು ಬಲಗೊಳಿಸದೆ ಬೇರಾವ ಮಾರ್ಗದಿಂದ ನಾವು ಪ್ರಯತ್ನಿಸಿದರೂ ಅದು ವಿನಾಶಕಾರಿಯಾಗುತ್ತದೆ. ಆದುದರಿಂದ ಭವಿಷ್ಯ ಭಾರತದ ನಿರ್ಮಾಣಕ್ಕೆ ಮೊದಲ ಹಂತವೆಂದರೆ ಧಾರ್ಮಿಕ ಏಕತೆಯನ್ನು ಸಾಧಿಸುವುದು. ಹಿಂದೂಗಳಾದ ನಾವು ದ್ವೈತಿಗಳಾಗಲೀ, ವಿಶಿಷ್ಟಾದ್ವೈತಿಗಳಾಗಲೀ, ಅದ್ವೈತಿಗಳಾಗಲೀ, ಶೈವರಾಗಲೀ, ವೈಷ್ಣವರಾಗಲೀ, ಪಾಶುಪತರಾಗಲೀ, ಯಾವ ವಿಭಿನ್ನ ಸಂಪ್ರದಾಯಕ್ಕೆ ಸೇರಿದ್ದರೂ ಎಲ್ಲರಿಗೂ ಸಮಾನವಾದ ಕೆಲವು ಭಾವನೆಗಳಿವೆ ಎಂಬುದನ್ನು ತಿಳಿದುಕೊಳ್ಳಬೇಕು. ನಮ್ಮ ಹಿತಕ್ಕೆ, ನಮ್ಮ ದೇಶದ ಹಿತಕ್ಕೆ, ನಮ್ಮಲ್ಲಿರುವ ವೈಮನಸ್ಯಗಳನ್ನು ಮತ್ತು ಭಿನ್ನಾಭಿಪ್ರಾಯಗಳನ್ನು ಮರೆಯುವ ಸಮಯ ಸನ್ನಿಹಿತವಾಗಿದೆ. ಈ ಮನಸ್ತಾಪ ಸಂಪೂರ್ಣವಾಗಿ ತಪ್ಪು ಎಂಬುದನ್ನು ಖಂಡಿತವಾಗಿ ತಿಳಿದುಕೊಳ್ಳಿ. ನಮ್ಮ ಶಾಸ್ತ್ರಗಳು ಇದನ್ನು ನಿಷೇಧಿಸುವುವು. ನಮ್ಮ ಪೂರ್ವಿಕರು ಇದನ್ನು ನಿಷೇಧಿಸುವರು. ಯಾವ ಮಹಾಪುರುಷರ ಕುಲಕ್ಕೆ ನಾವು ಸೇರಿದವರೋ, ಯಾರ ರಕ್ತ ನಮ್ಮ ನಾಡಿಯಲ್ಲಿ ಹರಿಯುತ್ತಿದೆಯೋ ಅವರು, ಅಲ್ಪ ವಿಷಯಗಳಿಗೆ ಹೋರಾಡುತ್ತಿರುವ ನಮ್ಮನ್ನು ನಿಕೃಷ್ಟದೃಷ್ಟಿಯಿಂದ ನೋಡುವರು.

ಹೋರಾಟವನ್ನು ನಿಲ್ಲಿಸಿದರೆ ಇತರ ಉನ್ನತಿಗಳು ಆವಶ್ಯಕವಾಗಿ ಆಗುವುವು. ರಕ್ತವು ಪುಷ್ಟಿಯಾಗಿದ್ದರೆ, ಶುದ್ಧವಾಗಿದ್ದರೆ ಯಾವ ವಿಷಕ್ರಿಮಿಯೂ ಆ ದೇಹದಲ್ಲಿರಲಾರದು. ಅಧ್ಯಾತ್ಮವೇ ನಮ್ಮ ಜೀವನ ರಕ್ತ. ಅದು ಶುದ್ಧವಾಗಿದ್ದರೆ, ಪುಷ್ಟಿಯಾಗಿದ್ದರೆ, ಎಲ್ಲವೂ ಸರಿಯಾಗಿರುತ್ತದೆ. ಆ ರಕ್ತ ಪರಿಶುದ್ಧವಾಗಿದ್ದರೆ ರಾಜಕೀಯ, ಸಾಮಾಜಿಕ ಮತ್ತು ಇತರ ಪ್ರಾಪಂಚಿಕ ಕುಂದು ಕೊರತೆಗಳೆಲ್ಲಾ, ನಮ್ಮ ದೇಶದ ದಾರಿದ್ರ್ಯ ಕೂಡ, ನಿವಾರಣೆಯಾಗುವುವು. ಏಕೆಂದರೆ ವಿಷಕ್ರಿಮಿಯನ್ನು ಹೊರಗೆ ಅಟ್ಟಿದರೆ ಮತ್ತಾವುದೂ ರಕ್ತವನ್ನು ಪ್ರವೇಶಿಸಲಾರದು. ಆಧುನಿಕ ಔಷಧ ಶಾಸ್ತ್ರದ ಪ್ರಕಾರ ರೋಗಕ್ಕೆ ಎರಡು ಕಾರಣಗಳಿರಬೇಕು: ಒಂದು ಹೊರಗೆ ಇರುವ ವಿಷಕ್ರಿಮಿ, ಎರಡನೆಯದು ನಮ್ಮ ದೇಹಸ್ಥಿತಿ. ದೇಹವು ವಿಷಕ್ರಿಮಿಗೆ ಅವಕಾಶ ಕೊಡದೇ ಇರುವ ಸ್ಥಿತಿಯಲ್ಲಿರುವವರೆಗೆ, ದೇಹವು ದುಃಸ್ಥಿತಿಗಿಳಿದು ವಿಷಕ್ರಿಮಿಗಳು ಅದರೊಳಗೆ ನುಗ್ಗಿ ಅಲ್ಲಿ ವೃದ್ಧಿಯಾಗದವರೆಗೆ ಯಾವ ವಿಷಕ್ರಿಮಿಯೂ ನಮಗೆ ರೋಗವನ್ನು ತರಲಾರದು. ಪ್ರತಿಕ್ಷಣವೂ ಕೋಟ್ಯಂತರ ವಿಷಕ್ರಿಮಿಗಳು ನಿರಂತರವಾಗಿ ನಮ್ಮ ದೇಹದ ಮೂಲಕ ಸಂಚರಿಸುತ್ತಿವೆ. ಎಲ್ಲಿಯವರೆಗೆ ಶರೀರ ಪುಷ್ಟಿಯಾಗಿರುವುದೋ ಅಲ್ಲಿಯವರೆಗೆ ಅದರ ಅರಿವೇ ನಮಗಿರುವುದಿಲ್ಲ. ದೇಹವು ದುರ್ಬಲವಾದಾಗ ಮಾತ್ರ ಈ ಕ್ರಿಮಿಗಳು ದೇಹವನ್ನು ಆಕ್ರಮಿಸಿ ರೋಗಕ್ಕೆ ಕಾರಣವಾಗುವುವು. ಇದರಂತೆಯೇ ರಾಷ್ಟ್ರದ ಜೀವನ ಕೂಡ. ದೇಹವು ದುರ್ಬಲವಾಗಿರುವಾಗ, ದೇಶದ ರಾಜಕೀಯ, ಸಾಮಾಜಿಕ, ಶಿಕ್ಷಣ ಅಥವಾ ಬೌದ್ಧಿಕ ಕ್ಷೇತ್ರದಲ್ಲಿರುವ ರೋಗಾಣುಗಳು ದೇಹವನ್ನು ಸೇರಿ ರೋಗಕ್ಕೆ ಕಾರಣವಾಗುವುವು. ಇದನ್ನು ನಿವಾರಿಸಬೇಕಾದರೆ ರೋಗದ ಮೂಲಕ ಹೋಗಬೇಕು. ರಕ್ತದಲ್ಲಿರುವ ಕಶ್ಮಲವನ್ನೆಲ್ಲಾ ತೆಗೆದು ಹಾಕಿಬಿಡಬೇಕು. ಮನುಷ್ಯನನ್ನು ಬಲಾಢ್ಯನನ್ನಾಗಿ ಮಾಡುವುದೇ ಈಗ ಮಾಡಬೇಕಾದ ಚಿಕಿತ್ಸೆ, ರಕ್ತವನ್ನು ಶುದ್ಧಿಮಾಡಬೇಕು, ದೇಹವನ್ನು ಪುಷ್ಟಿಗೊಳಿಸಬೇಕು. ಇದರಿಂದ ದೇಹಕ್ಕೆ ವಿಷಕ್ರಿಮಿಗಳನ್ನು ತಡೆಗಟ್ಟಿ ಹೊರಕ್ಕೆ ಓಡಿಸಲು ಸಾಧ್ಯವಾಗುತ್ತದೆ.

ನಮ್ಮ ತೇಜಸ್ಸು, ನಮ್ಮ ಬಲ, ಜನಾಂಗ ಜೀವನದ ಮೂಲಭಿತ್ತಿಯೇ ಧರ್ಮ ಎಂಬುದನ್ನು ನಾವು ನೋಡಿರುವೆವು. ನಮ್ಮ ಜನಾಂಗದ ಕೇಂದ್ರವು ಧರ್ಮವಾಗಿರುವುದು ಭವಿಷ್ಯದೃಷ್ಟಿಯಿಂದ ಸರಿಯೆ ತಪ್ಪೆ, ಹಿತವೆ ಅಹಿತವೆ, ಎಂದು ನಾನು ವಿಮರ್ಶಿಸುತ್ತಿಲ್ಲ. ಶುಭಕ್ಕೊ, ಅಶುಭಕ್ಕೊ, ಅದು ನಮ್ಮ ಜನಾಂಗದ ಕೇಂದ್ರವಾಗಿದೆ. ಇದರಿಂದ ಪಾರಾಗಲು ಶಕ್ಯವಿಲ್ಲ. ಅದು ಹೀಗಿದೆ, ಎಂದೆಂದಿಗೂ ಹೀಗೆಯೇ ಇರುವುದು. ಧರ್ಮದಲ್ಲಿ ನನಗೆ ಇರುವಷ್ಟು ಶ್ರದ್ಧೆ ನಿಮಗೆ ಇಲ್ಲದೆ ಇದ್ದರೂ ನೀವು ಅದಕ್ಕೆ ಬೆಂಬಲ ನೀಡಬೇಕಾಗಿದೆ. ನೀವು ಅದರಿಂದ ಬದ್ಧರಾಗಿರುವಿರಿ. ಅದನ್ನು ತ್ಯಜಿಸಿದರೆ ನಿಮ್ಮ ಸರ್ವನಾಶವಾಗುವುದು. ಅದೇ ನಮ್ಮ ಜನಾಂಗದ ಜೀವನ, ಅದನ್ನು ನಾವು ಬಲಪಡಿಸಬೇಕು. ನೀವು ಧರ್ಮವನ್ನು ರಕ್ಷಿಸಿದುದರಿಂದಲೇ, ಅದಕ್ಕಾಗಿ ಸರ್ವಸ್ವವನ್ನು ನೀವು ತ್ಯಾಗಮಾಡಿದುದರಿಂದಲೇ ಹಲವು ಶತಮಾನಗಳ ಆಘಾತವನ್ನು ಸಹಿಸುವುದು ನಿಮಗೆ ಸಾಧ್ಯವಾಯಿತು. ನಿಮ್ಮ ಪೂರ್ವಿಕರು ಎಲ್ಲ ಕೋಟಲೆಗಳನ್ನೂ ಮೃತ್ಯುವನ್ನೂ ಧೈರ್ಯದಿಂದ ಎದುರಿಸಿ ಧರ್ಮವನ್ನು ರಕ್ಷಿಸಿದರು.

ವಿದೇಶೀ ಆಕ್ರಮಣಕಾರರು ಒಂದಾದ ಮೇಲೊಂದರಂತೆ ದೇವಸ್ಥಾನಗಳನ್ನು ಧ್ವಂಸಮಾಡಿದರು. ಆ ಸೇನಾಸಮೂಹ ಹಿಂದೆ ಸರಿದ ತಕ್ಷಣವೇ ಹೊಸ ದೇವಸ್ಥಾನಗಳು ಮೇಲೆದ್ದವು. ದಕ್ಷಿಣ ಭಾರತದ ಕೆಲವು ಪುರಾತನ ದೇವ ಮಂದಿರಗಳು ಮತ್ತು ಗುಜರಾತಿನ ಸೋಮನಾಥದಂತಹ ದೇವಾಲಯಗಳು ಹಲವು ಗ್ರಂಥ ರಾಶಿಗಳಿಗಿಂತ ಮಿಗಿಲಾಗಿ ಜ್ಞಾನವನ್ನು ನೀಡುತ್ತವೆ, ಜನಾಂಗ ಜೀವನದ ಚರಿತ್ರೆಯ ಬಗ್ಗೆ ಅಂತರ್​ದೃಷ್ಟಿಯನ್ನು ಕೊಡಬಲ್ಲವು. ಈ ಗುಡಿಗಳ ಮೇಲೆ ಇರುವ ನೂರಾರು ಆಕ್ರಮಣದ ಚಿಹ್ನೆಗಳನ್ನು, ನೂರಾರು ಪುನರುತ್ಥಾನದ ಚಿಹ್ನೆಗಳನ್ನು ಗಮನಿಸಿ. ಬಾರಿ ಬಾರಿ ನಷ್ಟವಾದರೂ ಧ್ವಂಸದ ಅವಶೇಷದಿಂದ ಉದಯಿಸಿ, ನವಜೀವನವನ್ನು ತಾಳಿ ಮತ್ತೆ ಅವು ಹಿಂದಿನಂತೆ ಪ್ರಬಲವಾಗಿರುವುವು. ಇದೇ ನಮ್ಮ ಜನಾಂಗದ ಮನಸ್ಸು, ಇದೇ ನಮ್ಮ ಜೀವನ ಪ್ರವಾಹ. ಇದನ್ನು ಅನುಸರಿಸದರೆ ಮಹತ್ವಕ್ಕೆ ಏರುವಿರಿ. ಇದನ್ನು ತ್ಯಜಿಸಿದರೆ ಸಾಯುವಿರಿ. ಈ ಜೀವನ ಪ್ರವಾಹದಿಂದ ದೂರ ಸರಿದೊಡನೆಯೇ ಸರ್ವನಾಶವಾಗುವುದು. ಇತರ ವಿಷಯಗಳು ಅನವಶ್ಯಕ ಎಂದು ನಾನು ಹೇಳುವುದಿಲ್ಲ. ರಾಜಕೀಯ ಮತ್ತು ಸಾಮಾಜಿಕ ಪ್ರಗತಿಗಳು ಅನವಶ್ಯಕವೆಂದು ಹೇಳುವುದಿಲ್ಲ. ನಾನು ಹೇಳುವುದು ಇದು: ಇಲ್ಲಿ ಅವು ಗೌಣ, ಇಲ್ಲಿ ಧರ್ಮವೇ ಪ್ರಧಾನ. ಇದನ್ನು ಗಮನದಲ್ಲಿಡಿ. ಭಾರತೀಯನಿಗೆ ಮೊದಲು ಧರ್ಮ, ಅನಂತರ ಇತರ ಹವ್ಯಾಸಗಳು, ಮೊದಲು ಇದನ್ನು ನಾವು ಬಲಪಡಿಸಬೇಕು. ಇದನ್ನು ಮಾಡುವುದು ಹೇಗೆ? ನಾನು ನನ್ನ ಭಾವನೆಗಳನ್ನು ನಿಮ್ಮ ಮುಂದೆ ಇಡುತ್ತೇನೆ. ಬಹಳ ವರ್ಷಗಳಿಂದ ಇವು ನನ್ನ ಮನದಲ್ಲಿದ್ದವು. ಮದ್ರಾಸಿನಿಂದ ಅಮೆರಿಕಾ ದೇಶಕ್ಕೆ ಹೋಗುವುದಕ್ಕೂ ಮುಂಚೆಯೇ ಇದ್ದವು. ಆ ಭಾವನೆಗಳನ್ನು ಪ್ರಚಾರ ಮಾಡುವುದಕ್ಕೆ ಮಾತ್ರ ನಾನು ಅಮೆರಿಕಾ ಮತ್ತು ಇಂಗ್ಲೆಂಡ್​ ದೇಶಗಳಿಗೆ ಹೋದುದು. ವಿಶ್ವಧರ್ಮ ಸಮ್ಮೇಳನ ಮುಂತಾದುವನ್ನು ನಾನು ಲೆಕ್ಕಿಸಲಿಲ್ಲ. ಅದೊಂದು ಅವಕಾಶ ಮಾತ್ರ. ಈ ನನ್ನ ಭಾವನೆಗಳೇ ಸಂಚರಿಸುವಂತೆಮಾಡಿದವು.

ನಮ್ಮ ಶಾಸ್ತ್ರಗಳಲ್ಲಿ ಹುದುಗಿರುವ, ಮಠ ಮತ್ತು ಅರಣ್ಯಗಳಲ್ಲಿ ಅಡಗಿರುವ ಕೆಲವೇ ವ್ಯಕ್ತಿಗಳ ಸ್ವತ್ತಾಗಿರುವ ಆಧ್ಯಾತ್ಮಿಕ ರತ್ನಗಳನ್ನು ಬಯಲಿಗೆ ತರಬೇಕು. ಯಾರಲ್ಲಿ ಅವು ರಹಸ್ಯವಾಗಿರುವುವೋ ಅಂತಹ ಕೆಲವು ವ್ಯಕ್ತಿಗಳಿಂದ ಹೊರಗೆ ತರುವುದು ಮಾತ್ರವಲ್ಲ, ದುರ್ಭೇದ್ಯವಾದ ಯಾವ ಭಾಷಾಪೆಟ್ಟಿಗೆಯಲ್ಲಿ ಅವು ಸುರಕ್ಷಿತವಾಗಿವೆಯೋ, ಅಂತಹ ಶತಶತಾಬ್ದಿಗಳ ಸಂಸ್ಕೃತ ಶಬ್ದಜಾಲದಿಂದ ಅವನ್ನು ಬಿಡಿಸಬೇಕಾಗಿದೆ. ಸಂಕ್ಷೇಪವಾಗಿ ಹೇಳಬೇಕಾದರೆ ಅವನ್ನು ಜನರಿಗೆ ಸುಲಭವಾಗಿ ನಿಲುಕುವಂತೆ ಮಾಡಬೇಕು. ಈ ತತ್ತ್ವಗಳನ್ನು ಹೊರಗೆ ತರಬೇಕು. ಇವು ಭಾರತದ ಪ್ರತಿಯೊಬ್ಬರ ಸಂಪತ್ತು ಆಗಬೇಕು - ಅವರಿಗೆ ಸಂಸ್ಕೃತ ಗೊತ್ತಿರಲಿ, ಇಲ್ಲದಿರಲಿ, ಚಿಂತೆ ಇಲ್ಲ. ಈ ಮಾರ್ಗದಲ್ಲಿರುವ ಮೊದಲನೆಯ ಅಡಚಣೆಯೇ ನಮ್ಮ ಶ್ರೇಷ್ಠ ಸಂಸ್ಕೃತ ಭಾಷೆ. ಸಾಧ್ಯವಾದರೆ ನಮ್ಮ ದೇಶದವರೆಲ್ಲರೂ ಸಂಸ್ಕೃತದಲ್ಲಿ ವಿದ್ಯಾವಂತರಾಗುವವರೆಗೆ ಈ ಅಡಚಣೆಯನ್ನು ನಿವಾರಿಸುವುದು ಕಷ್ಟ. ನಾನು ಇಡೀ ಜೀವನವೆಲ್ಲ ಈ ಭಾಷೆಯ ಅಧ್ಯಯನವನ್ನು ಮಾಡುತ್ತಿರುವೆನು, ಆದರೂ ಪ್ರತಿಯೊಂದು ಹೊಸ ಗ್ರಂಥವೂ ನನಗೆ ಹೊಸದಾಗಿ ತೋರುತ್ತದೆ ಎಂದರೆ ಆ ಭಾಷೆ ಎಷ್ಟು ಕ್ಲಿಷ್ಟವಾದುದು ಎಂಬುದು ನಿಮಗೆ ಅರಿವಾಗುತ್ತದೆ. ಈ ಭಾಷೆಯನ್ನು ವಿವರವಾಗಿ ಅಧ್ಯಯನ ಮಾಡುವುದಕ್ಕೆ ಸಮಯವಿಲ್ಲದವರಿಗೆ ಕಷ್ಟ ಎಷ್ಟಿರಬಹುದು! ಆದಕಾರಣವೇ ಭಾವನೆಗಳನ್ನು ದೇಶಭಾಷೆಗಳ ಮೂಲಕ ಜನರಿಗೆ ಪ್ರಚಾರಮಾಡಬೇಕು. ಜೊತೆಗೆ ಸಂಸ್ಕೃತ ವಿದ್ಯಾಭ್ಯಾಸವೂ ನಡೆಯಬೇಕು. ಏಕೆಂದರೆ ಸಂಸ್ಕೃತಭಾಷೋಚ್ಚಾರಣೆಯೆ ಜನರಿಗೆ ಗೌರವ, ಶಕ್ತಿ, ತೇಜಸ್ಸು ಇವನ್ನು ನೀಡುವುದು. ಕೆಳಗಿರುವ ಜನರನ್ನು ಮೇಲೆತ್ತಲು ರಾಮಾನುಜ, ಚೈತನ್ಯ, ಕಬೀರ್​ ಮುಂತಾದವರು ಮಾಡಿದ ಪ್ರಯತ್ನ ಬಹಳ ಶೀಘ್ರದಲ್ಲಿ, ಅವರ ಕಾಲದಲ್ಲೇ ಪ್ರತಿಫಲವನ್ನು ನೀಡಿತು. ಆದರೂ ಅನಂತರ ಅಭಿವೃದ್ಧಿ ಮುನ್ನಡೆಯಲಿಲ್ಲ. ಈ ಶ್ರೇಷ್ಠ ಆಚಾರ್ಯರ ಕಾಲಾನಂತರ ಒಂದು ಶತಮಾನದ ಒಳಗೇ ಅವರ ಉಪದೇಶಗಳು ತಮ್ಮ ಪ್ರಭಾವವನ್ನುಕಳೆದುಕೊಂಡವು. ಇದಕ್ಕೆ ನಾವು ಕಾರಣ ಕೊಡಬೇಕಾಗಿದೆ. ಅದರ ರಹಸ್ಯ ಇಲ್ಲಿದೆ: ಅವರು ಕೆಳಗಿನ ವರ್ಗದವರನ್ನು ಮೇಲೆತ್ತಿದರು. ಇವರು ಮೇಲೆ ಬರಬೇಕೆಂದು ಅವರೆಲ್ಲರೂ ಆಶಿಸಿದ್ದರು. ಆದರೆ ಜನಸಾಮಾನ್ಯರಲ್ಲಿ ಸಂಸ್ಕೃತ ಭಾಷೆಯನ್ನು ಪ್ರಚಾರ ಮಾಡುವುದಕ್ಕೆ ಅವರು ಪ್ರಯತ್ನಿಸಲಿಲ್ಲ. ಜನಸಾಮಾನ್ಯರು ಸಂಸ್ಕೃತ ಭಾಷೆಯನ್ನು ಓದದಂತೆ ಮಾಡುವುದರ ಮೂಲಕ ಭಗವಾನ್​ ಬುದ್ಧದೇವನೂ ಒಂದು ತಪ್ಪನ್ನು ಮಾಡಿದನು. ಅವನಿಗೆ ಶೀಘ್ರವಾಗಿ ಫಲ ಬೇಕಾಗಿತ್ತು. ಆದ್ದರಿಂದ ಜನರ ಅಂದಿನ ಭಾಷೆಯಾದ ಪಾಲಿಯಲ್ಲಿ ಪ್ರಚಾರಮಾಡಿದನು. ಅದೊಂದು ಮಹಾಕಾರ್ಯ. ಜನರಾಡುವ ಭಾಷೆಯಲ್ಲಿ ಮಾತನಾಡಿದನು, ಜನರೂ ಅದನ್ನು ತಿಳಿದುಕೊಂಡರು. ಇದೊಂದು ಮಹಾ ಸಾಹಸ. ಇದರಿಂದ ಭಾವನೆ ವೇಗವಾಗಿ ಬಹಳ ದೂರ ಪ್ರಚಾರವಾಗುವುದಕ್ಕೆ ಸಾಧ್ಯವಾಯಿತು. ಅದರ ಜೊತೆಗೆ ಸಂಸ್ಕೃತ ಭಾಷೆಯೂ ಹರಡಬೇಕಾಗಿತ್ತು. ಜ್ಞಾನ ಬಂತು, ಗೌರವ ಬರಲಿಲ್ಲ, ಸಂಸ್ಕೃತಿ ಬರಲಿಲ್ಲ. ಸಂಸ್ಕೃತಿ ಮಾತ್ರ ಆಘಾತವನ್ನು ಸಹಿಸಬಲ್ಲದು. ಕೇವಲ ಜ್ಞಾನರಾಶಿಗೆ ಅದನ್ನು ಎದುರಿಸುವ ಶಕ್ತಿ ಇಲ್ಲ. ಜಗತ್ತಿಗೆ ಜ್ಞಾನರಾಶಿಯನ್ನೇ ಕೊಡಬಹುದು. ಆದರೆ ಅದರಿಂದ ಹೆಚ್ಚು ಪ್ರಯೋಜನವಿಲ್ಲ. ಸಂಸ್ಕೃತಿಯು ರಕ್ತದಲ್ಲಿ ಹರಿಯಬೇಕು. ಆಧುನಿಕ ಕಾಲದಲ್ಲಿ ಬೇಕಾದಷ್ಟು ಜ್ಞಾನಸಂಪತ್ತನ್ನು ಪಡೆದ ದೇಶಗಳನ್ನು ನೋಡಿರುವೆವು. ಆದರೆ ಅದರಿಂದ ಏನು ಪ್ರಯೋಜನ? ಅವರು ವ್ಯಾಘ್ರಗಳಂತೆ, ಕಾಡುಜನರಂತೆ ಇರುವರು. ಅವರಲ್ಲಿ ಸಂಸ್ಕೃತಿ ಇಲ್ಲ. ನಾಗರಿಕತೆಯಂತೆ ಅವರ ಜ್ಞಾನ ಕೂಡ ತೋರಿಕೆಗೆ ಮಾತ್ರ. ಅವರನ್ನು ಸ್ಪಲ್ಪ ಕೆರೆದರೆ ಸಾಕು ಕಾಡುಮನುಷ್ಯ ಕಾಣುವನು. ಆಗುವುದೇ ಹೀಗೆ, ಇದೇ ಅಪಾಯ. ದೇಶಭಾಷೆಯಲ್ಲಿ ಜನರಿಗೆ ಪ್ರಚಾರಮಾಡಿ, ಭಾವನೆಗಳನ್ನು ನೀಡಿ. ಅವರು ವಿಷಯ ಗ್ರಹಣಮಾಡಿಕೊಳ್ಳುವರು. ಆದರೆ ಅವರಿಗೆ ಮತ್ತೊಂದು ಆವಶ್ಯಕತೆ ಇದೆ. ಸಂಸ್ಕೃತಿಯನ್ನು ಕೊಡಿ. ನೀವು ಅವರಿಗೆ ಇದನ್ನು ಕೊಡುವ ತನಕ ಅವರ ಪ್ರಗತಿ ಸ್ಥಿರವಾಗಿರಲಾರದು. ಸಂಸ್ಕೃತಭಾಷೆ ಬಲ್ಲ ಮತ್ತೊಂದು ಜಾತಿ ಹುಟ್ಟುವುದು. ಅದು ಇತರರ ಮೇಲೆ ಹೋಗಿ ಅವರನ್ನು ಆಳುವುದು. ಕೆಳಗಿನ ವರ್ಗದಲ್ಲಿರುವವರು ಮೇಲೆ ಬರಬೇಕಾದರೆ ಒಂದೇ ಸುರಕ್ಷಿತ ಮಾರ್ಗ ಇರುವುದು, ಅದೇ ಸಂಸ್ಕೃತ ಭಾಷೆಯನ್ನು ಕಲಿಯುವುದು. ಮೇಲಿನ ವರ್ಣದವರೊಡನೆ ಹೋರಾಡುವುದು, ಜಗಳ ಕಾಯುವುದು, ಅವರಿಗೆ ವಿರೋಧವಾಗಿ ಬರೆಯುವುದು ಇದರಿಂದ ವ್ಯರ್ಥ ಮನಸ್ತಾಪ, ಹೋರಾಟ. ಈ ಜನಾಂಗ ಆಗಲೇ ದುರ್ದೈವವಶಾತ್​ ಭಿನ್ನಭಿನ್ನವಾಗಿದೆ, ಅದು ಮತ್ತೂ ಹೆಚ್ಚು ಭಿನ್ನಭಿನ್ನವಾಗುವುದು. ಜಾತಿಗಳನ್ನು ಒಂದೇ ಸಮನಾಗಿ ಮಾಡಬೇಕಾದರೆ, ಕೆಳಗಿನ ವರ್ಣದವರು ಮೇಲಿನವರ್ಣದವರ ಪ್ರಬಲತೆಗೆ ಕಾರಣವಾದ ವಿದ್ಯೆಯನ್ನು ಮತ್ತು ಸಂಸ್ಕೃತಿಯನ್ನು ಸ್ವೀಕರಿಸಬೇಕು. ಅದಾದರೆ ನಿಮಗೆ ಬೇಕಾದುದು ಸಿದ್ಧಿಸುವುದು.

ಈ ಸಂದರ್ಭದಲ್ಲಿ ಕೇವಲ ಮದ್ರಾಸಿಗೆ ಅನ್ವಯಿಸುವ ಒಂದು ಪ್ರಶ್ನೆಯನ್ನು ಪರ್ಯಾಲೋಚಿಸುವೆ. ದ್ರಾವಿಡರು ಎಂಬ ಬೇರೆ ಜನಾಂಗ ದಕ್ಷಿಣ ಭರತಖಂಡದಲ್ಲಿ ಇದ್ದಿತೆಂದು ಒಂದು ಸಿದ್ಧಾಂತವಿದೆ. ಇದು ಉತ್ತರದಲ್ಲಿರುವ ಆರ್ಯರಿಗಿಂತ ಸಂಪೂರ್ಣ ಬೇರೆ ಎಂದೂ, ದಕ್ಷಿಣದೇಶದಲ್ಲಿರುವ ಬ್ರಾಹ್ಮಣರು ಮಾತ್ರ ಉತ್ತರದ ಆರ್ಯರೆಂದೂ, ಉಳಿದವರೆಲ್ಲರೂ ಬೇರೆ ಜಾತಿಗೆ ಸೇರಿದವರೆಂದೂ ಹೇಳುವರು. ಭಾಷಾಶಾಸ್ತ್ರಜ್ಞರೇ, ದಯವಿಟ್ಟು ಕ್ಷಮಿಸಿ. ನಿಮ್ಮ ಊಹೆ ನಿರಾಧಾರವಾದುದು. ನಿಮ್ಮ ವಾದಕ್ಕೆ ಪ್ರಮಾಣ ಒಂದೇ - ಅದು ಉತ್ತರ ಮತ್ತು ದಕ್ಷಿಣ ಭಾಷೆಗಳಲ್ಲಿ ಇರುವ ವ್ಯತ್ಯಾಸ. ಮತ್ತಾವ ವ್ಯತ್ಯಾಸವನ್ನೂ ನಾನು ಕಾಣೆ. ಇಲ್ಲಿ ಎಷ್ಟೋ ಜನರು ಉತ್ತರ ದೇಶದವರು ಇರುವರು. ನನ್ನ ಯೂರೋಪಿನ ಸ್ನೇಹಿತರಿಗೆ, ನಮ್ಮಲ್ಲಿ ಔತ್ತರೇಯರು ಯಾರು ದಾಕ್ಷಿಣಾತ್ಯರು ಯಾರು ಎಂದು ಈ ಸಭೆಯಲ್ಲಿ ಪ್ರತ್ಯೇಕಿಸಿ ಎಂದೂ ಕೇಳುವೆನು. ಎಲ್ಲಿದೆ ವ್ಯತ್ಯಾಸ? ಎಲ್ಲೋ ಸ್ವಲ್ಪ ಭಾಷಾ ವ್ಯತ್ಯಾಸವಷ್ಟೆ. ಇಲ್ಲಿಗೆ ಬಂದ ಬ್ರಾಹ್ಮಣರು ಸಂಸ್ಕೃತ ಭಾಷೆಯನ್ನು ಮಾತಾಡುವ ಬ್ರಾಹ್ಮಣರು, ಅವರು ದ್ರಾವಿಡ ಭಾಷೆಯನ್ನು ಕಲಿತು ಸಂಸ್ಕೃತವನ್ನು ಮರೆತರು ಎನ್ನುವರು. ಇತರ ಜಾತಿಯವರೂ ಏತಕ್ಕೆ ಹಾಗೆಯೇ ಮಾಡಿರಬಾರದು? ಇತರರೂ ಕೂಡ ಒಬ್ಬರಾದ ಮೇಲೊಬ್ಬರು ಉತ್ತರದಿಂದ ಬಂದು ತಮ್ಮ ಭಾಷೆಯನ್ನು ಮರೆತು ದ್ರಾವಿಡ ಭಾಷೆಯನ್ನು ಏತಕ್ಕೆ ಕಲಿತಿರಬಾರದು? ಈ ವಾದಸರಣಿ ಉಭಯ ಪಕ್ಷಗಳಿಗೂ ಅನ್ವಯಿಸುವುದು. ಇಂತಹ ಕೆಲಸಕ್ಕೆ ಬಾರದ ಸಿದ್ಧಾಂತವನ್ನು ನಂಬಬೇಡಿ. ಇಲ್ಲಿಂದ ಮಾಯವಾದ ದ್ರಾವಿಡ ಜನಾಂಗ ಹಿಂದೆ ಇದ್ದಿರಬಹುದು. ಎಲ್ಲೊ ಅಳಿದುಳಿದ ಕೆಲವರು ಕಾಡುಗಳಲ್ಲಿ ಮತ್ತಿತರ ಕಡೆಗಳಲ್ಲಿ ವಾಸಿಸುತ್ತಿದ್ದರೇನೋ. ಬಹುಶಃ ಅವರ ಭಾಷೆಯನ್ನು ಸ್ವೀಕರಿಸಿರಬಹುದು. ಈಗಿರುವವರೆಲ್ಲ ಉತ್ತರದಿಂದ ಬಂದ ಆರ್ಯರು. ಹಿಂದೂ ದೇಶವೆಲ್ಲ ಆರ್ಯ ಭೂಮಿ, ಬೇರೆಯಲ್ಲ.

ಶೂದ್ರರು ಈ ದೇಶದ ಮೂಲ ನಿವಾಸಿಗಳು ಎಂಬ ಭಾವನೆಯೊಂದು ಇದೆ. ಅವರಾರು? ಅವರು ದಸ್ಯುಗಳು. ಇತಿಹಾಸವು ಪುನರಾವೃತ್ತಿಗೊಳ್ಳುತ್ತದೆ ಎನ್ನುತ್ತಾರೆ. ಅಮೆರಿಕಾದವರು, ಇಂಗ್ಲಿಷರು, ಪೋರ್ಚುಗೀಸರು ಮತ್ತು ಡಚ್ಚರು ಆಫ್ರಿಕಾದೇಶದ ನೀಗ್ರೋ ಜನರನ್ನು ಹಿಡಿದು ಅವರಿಂದ ಕೆಲಸ ಮಾಡಿಸುತ್ತಿದ್ದರು. ಗುಲಾಮಗಿರಿಯಲ್ಲೇ ಜಾತಿಸಂಕರದಿಂದ ಹುಟ್ಟಿದ ಸಂತಾನರು ಗುಲಾಮರಾಗಿಯೇ ಬಹಳ ಕಾಲ ಉಳಿದರು. ಈ ಉದಾಹರಣೆಯ ಆಧಾರದ ಮೇಲೆ ಇಲ್ಲಿಯೂ ಹಾಗೆಯೇ ಆಗಿರಬೇಕೆಂದು ಊಹಿಸುವರು. ನಮ್ಮ ಪ್ರಾಕ್ತನ ಶಾಸ್ತ್ರಜ್ಞರು ಭರತಖಂಡದಲ್ಲೆಲ್ಲ ಕಪ್ಪು ಕಣ್ಣಿನ ಅನಾರ್ಯರೇ ಮೊದಲು ಇದ್ದರೆಂದೂ, ಆರ್ಯರು ಎಲ್ಲೋ ಹೊರಗಿನಿಂದ ಬಂದರೆಂದೂ ಹೇಳುವರು. ದೇವರೇ ಬಲ್ಲ ಅವರು ಎಲ್ಲಿಂದ ಬಂದರೋ! ಕೆಲವರು ಆರ್ಯರು ಮಧ್ಯ ಟಿಬೆಟ್ಟಿನಿಂದ ಬಂದರೆಂದೂ, ಮತ್ತೆ ಕೆಲವರು ಮಧ್ಯ ಏಷ್ಯಾದಿಂದ ಬಂದರೆಂದೂ, ದೇಶ ಭಕ್ತ ಆಂಗ್ಲೇಯರು ಆರ್ಯರೆಲ್ಲ ಕೆಂಪು ಕೂದಲಿನವರೆಂದೂ ಮತ್ತೆ ಕೆಲವರು ತಮಗೆ ತೋರಿದಂತೆ ಅವರೆಲ್ಲ ಕಪ್ಪು ಕೂದಲಿನವರೆಂದೂ ಹೇಳುವರು. ಗ್ರಂಥಕರ್ತ ಕಪ್ಪು ಕೂದಲಿನವನಾದರೆ ಆರ್ಯರು ಕಪ್ಪು ಕೂದಲಿನವರು! ಈಚೆಗೆ ಆರ್ಯರು ಸ್ವಿಟ್ಜರ್​ಲೆಂಡ್​ ಸರೋವರದ ಬಳಿ ಇದ್ದರೆಂದು ಸಪ್ರಮಾಣವಾಗಿ ತೋರುವ ಪ್ರಯತ್ನ ಬೇರೆ ಆಗಿದೆ. ಅವರೆಲ್ಲರೂ, ಈ ಸಿದ್ಧಾಂತ ಸಮೇತ ಆ ಸರೋವರದ ಪಾಲಾದರೂ ನನಗೆ ವ್ಯಥೆಯಿಲ್ಲ. ಕೆಲವರು ಆರ್ಯರು ಉತ್ತರಧ್ರುವದ ಬಳಿ ಇದ್ದರೆಂದು ಈಗ ಹೇಳುವರು. ದೇವರು ಆರ್ಯರನ್ನು ಮತ್ತು ಅವರ ನಿವಾಸಸ್ಥಳವನ್ನು ಆಶೀರ್ವದಿಸಲಿ! ಈ ಸಿದ್ಧಾಂತ ಸತ್ಯವೆಂದು ತೋರುವುದಕ್ಕೆ ನಮ್ಮ ಶಾಸ್ತ್ರಗಳಲ್ಲಿ ಒಂದು ಪದವೂ ಇಲ್ಲ. ಆರ್ಯರು ಹೊರಗಿನಿಂದ ಇಲ್ಲಿಗೆ ಬಂದರು ಎಂದು ತೋರುವುದಕ್ಕೆ ಒಂದು ಆಧಾರವೂ ಇಲ್ಲ. ಹಿಂದಿನ ಭರತಖಂಡದಲ್ಲಿ ಈಗಿನ ಆಫ್ಘಾನಿಸ್ಥಾನ ಕೂಡ ಸೇರಿತ್ತು. ಇಲ್ಲಿಗೆ ಮುಕ್ತಾಯವಾಗುವುದು. ಶೂದ್ರರೆಲ್ಲ ಅನಾರ್ಯರು, ಅನಾಗರಿಕ ಗುಂಪು ಎಂಬುದು ಕುತರ್ಕ, ನ್ಯಾಯಬದ್ಧವಲ್ಲ. ಆಗಿನ ಕಾಲದಲ್ಲಿ ಎಲ್ಲೋ ಕೆಲವು ಆರ್ಯರು ಇಲ್ಲಿಗೆ ಬಂದು ನೆಲಸಿ ಲಕ್ಷಾಂತರ ದಸ್ಯುಗಳನ್ನು ಆಳುತ್ತಿದ್ದರೆಂಬುದು ಸಾಧ್ಯವಿಲ್ಲ; ಈ ದಸ್ಯುಗಳು ಅವರನ್ನು ಐದು ನಿಮಿಷದಲ್ಲಿ ಚಟ್ನಿಮಾಡಿ ಪೂರೈಸುತ್ತಿದ್ದರು. ಇದಕ್ಕೆ ಸರಿಯಾದ ವಿವರಣೆ ನಮಗೆ ಮಹಾಭಾರತದಲ್ಲಿ ದೊರಕುವುದು. ಸತ್ಯಯುಗದ ಆದಿಯಲ್ಲಿ ಕೇವಲ ಬ್ರಾಹ್ಮಣರು ಮಾತ್ರ ಇದ್ದರು. ಬೇರೆ ಬೇರೆ ವೃತ್ತಿಗಳಿಗೆ ಅನುಸಾರವಾಗಿ ಅವರು ಭಿನ್ನ ಭಿನ್ನ ಪಂಗಡಗಳಾಗಿ ವಿಭಾಗವಾಗುತ್ತಾ ಹೋದರು ಎಂದು ಅದು ಹೇಳುವುದು. ಇದೊಂದೇ ನಿಜವಾದ ಯುಕ್ತಿಪೂರಿತ ವಿವರಣೆ. ಮುಂದೆ ಬರುವ ಸತ್ಯಯುಗದಲ್ಲಿ ಎಲ್ಲಾ ಜಾತಿಗಳೂ ಒಂದಾಗಬೇಕು.

ವರ್ಣಸಮಸ್ಯೆ ಭರತಖಂಡದಲ್ಲಿ ಈ ರೂಪವನ್ನು ತಾಳುವುದು, ಮೇಲಿನ ವರ್ಗದವರನ್ನು ಕೆಳಗೆ ಎಳೆಯುವುದಲ್ಲ, ಬ್ರಾಹ್ಮಣರನ್ನು ನಾಶಗೊಳಿಸುವುದಲ್ಲ. ಭರತಖಂಡದಲ್ಲಿ ಮಾನವಕೋಟಿಯ ಆದರ್ಶ ಬ್ರಾಹ್ಮಣ. ಶ‍್ರೀಶಂಕರಾಚಾರ್ಯರು ಗೀತಾಭಾಷ್ಯದ ಪ್ರಾರಂಭದಲ್ಲಿ ಇದನ್ನು ಸುಂದರವಾಗಿ ವಿವರಿಸುವರು. ಬ್ರಾಹ್ಮಣರ ಬ್ರಹ್ಮತ್ವವನ್ನು ಕಾಪಾಡುವುದಕ್ಕೆ, ಅದನ್ನು ಪ್ರಚಾರಮಾಡುವುದಕ್ಕೆ ಶ‍್ರೀಕೃಷ್ಣ ಅವತಾರವೆತ್ತಿದನು ಎಂದು ಅವರು ಹೇಳುವರು. ಇದೇ ಅವನ ಅವತಾರದ ಪರಮ ಉದ್ದೇಶ. ಈ ಬ್ರಾಹ್ಮಣ, ಬ್ರಹ್ಮಜ್ಞಪುರುಷ, ಈ ಆದರ್ಶ ಸಿದ್ಧಮಾನವ ಇರುವುದು ಅತ್ಯಾವಶ್ಯಕ. ಬ್ರಹ್ಮಜ್ಞಪುರುಷ ಎಂದಿಗೂ ಮಾಯವಾಗಕೂಡದು. ಈಗ ಇರುವ ವರ್ಣಗಳಲ್ಲಿ ಎಷ್ಟೋ ಲೋಪದೋಷಗಳಿವೆ. ಆದರೆ ಎಲ್ಲರೂ ಬ್ರಾಹ್ಮಣರಿಗೆ ಈ ಗೌರವವನ್ನು ಕೊಡುವುದಕ್ಕೆ ಸಿದ್ಧರಾಗಿರಬೇಕೆಂದು ನಮಗೆ ಗೊತ್ತಿದೆ. ಉಳಿದ ವರ್ಗಗಳಿಗೆಲ್ಲಕ್ಕಿಂತ ಹೆಚ್ಚಾಗಿ ಬ್ರಾಹ್ಮಣ ವರ್ಗದಲ್ಲಿ ಹೆಚ್ಚು ಬ್ರಹ್ಮಜ್ಞರಿರುವರು. ಇದು ಸತ್ಯ. ಉಳಿದ ವರ್ಣದವರಿಂದ ಅವರಿಗೆ ಇದಕ್ಕಾಗಿ ಗೌರವ ಸಲ್ಲಬೇಕಾಗಿದೆ. ಅವರ ಲೋಪದೋಷಗಳನ್ನು ನಿರ್ಭೀತಿಯಿಂದ ಧೈರ್ಯವಾಗಿ ತೋರಬೇಕು. ಆದರೆ ಅವರಿಗೆ ಸಲ್ಲುವ ಗೌರವವನ್ನೂ ತೋರಬೇಕು. \enginline{Give every man his due–} ಪ್ರತಿಯೊಬ್ಬನಿಗೂ ನ್ಯಾಯವಾಗಿ ಸಲ್ಲಬೇಕಾದ ಗೌರವವನ್ನು ಸಲ್ಲಿಸಿ - ಎಂಬ ಇಂಗ್ಲಿಷ್​ ಹಳೆಯ ನಾಣ್ನುಡಿಯನ್ನು ನೆನಪಿನಲ್ಲಿಡಿ. ಆದ್ದರಿಂದ ಮಿತ್ರರೆ, ಸುಮ್ಮನೆ ಜಾತಿಗಳೊಂದಿಗೆ ಹೋರಾಡಿ ಪ್ರಯೋಜನವಿಲ್ಲ. ಇದರಿಂದ ಏನು ಪ್ರಯೋಜನ?

ಇದು ನಮ್ಮಲ್ಲಿ ಹೆಚ್ಚಾಗಿ ಭೇದಗಳನ್ನು ಕಲ್ಪಿಸುತ್ತದೆ, ಹೆಚ್ಚು ದುರ್ಬಲರನ್ನಾಗಿ ಮಾಡುವುದು, ಹೆಚ್ಚು ಅಧೋಗತಿಗೆ ಒಯ್ಯುವುದು. ಪ್ರತ್ಯೇಕ ಹಕ್ಕುಬಾಧ್ಯತೆಗಳ ಕಾಲ ಆಗಿಹೋಯಿತು. ಭರತಖಂಡದಿಂದ ಅಂತಹ ಕಾಲ ಎಂದೋ ಹೋಯಿತು. ಆಂಗ್ಲೇಯರ ಆಳ್ವಿಕೆಯಿಂದಾದ ಒಂದು ಮಹದುಪಕಾರ ಇದು. ಪ್ರತ್ಯೇಕ ಹಕ್ಕನ್ನು ಧ್ವಂಸಮಾಡಿದ ಮಹಮ್ಮದೀಯ ಆಳ್ವಿಕೆಗೂ ನಾವು ಋಣಿಗಳು. ಅವರ ಆಳ್ವಿಕೆಯೆಲ್ಲಾ ಕೆಟ್ಟದ್ದಲ್ಲ. ಯಾವುದೂ ಬರಿಯ ಒಳ್ಳೆಯದೂ ಅಲ್ಲ, ಕೆಟ್ಟದ್ದೂ ಅಲ್ಲ. ಭರತಖಂಡದಲ್ಲಿ ಮಹಮ್ಮದೀಯ ಆಕ್ರಮಣ ಮುಕ್ತಿಯಂತೆ ಬಂತು. ಅದಕ್ಕಾಗಿಯೆ ನಮ್ಮ ದೇಶದಲ್ಲಿ ಐದನೆಯ ಒಂದು ಪಾಲು ಮಹಮ್ಮದೀಯರಾದರು. ಕತ್ತಿಯೇ ಇದನ್ನೆಲ್ಲಾ ಸಾಧಿಸಲಿಲ್ಲ. ಕೇವಲ ಬಂದೂಕು ಮತ್ತು ಕತ್ತಿ ಎರಡೇ ಇದನ್ನೆಲ್ಲಾ ಸಾಧಿಸಿದವು ಎಂದು ಭಾವಿಸುವುದು ಮೌಢ್ಯದ ಪರಾಕಾಷ್ಠೆ. ನೀವು ಜೋಪಾನವಾಗಿಲ್ಲದೇ ಇದ್ದರೆ ಐದನೆಯ ಒಂದು ಪಾಲು ಅಲ್ಲ, ಮದ್ರಾಸಿನಲ್ಲಿ ಅರ್ಧ ಪಾಲು ಜನರು ಕ್ರೈಸ್ತರಾಗುವರು. ಮಲಬಾರಿನಲ್ಲಿ ನಾನು ನೋಡಿರುವುದಕ್ಕಿಂತ ಹಾಸ್ಯಾಸ್ಪದವಾದುದು ಮತ್ತೆಲ್ಲಿಯಾದರೂ ಇರುವುದೇ? ಸತ್ಕುಲಪ್ರಸೂತನು ಇರುವ ಬೀದಿಯಲ್ಲಿ ಪರೆಯನಿಗೆ ನಡೆಯಲು ಅವಕಾಶವಿಲ್ಲ. ಅವನು ತನ್ನ ಹೆಸರನ್ನು ಬದಲಾಯಸಿ ಒಂದು ಇಂಗ್ಲೀಷ್​ ಹೆಸರನ್ನೋ ಅಥವಾ ಮಹಮ್ಮದೀಯ ಹೆಸರನ್ನೋ ಇಟ್ಟುಕೊಂಡರೆ ಪರವಾಯಿಲ್ಲ. ಇದರಿಂದ ನೀವು ಅವರನ್ನು ಯಾರು ಎಂದು ಊಹಿಸಬೇಕಾಗಿದೆ? ಮಲೆಯಾಳಿಗಳೆಲ್ಲಾ ಹುಚ್ಚರು. ಅವರ ಮನೆ ಹುಚ್ಚರ ಆಸ್ಪತ್ರೆ. ಭರತಖಂಡದ ಇತರರೆಲ್ಲಾ, ಇವರು ತಮ್ಮ ನಡತೆಯನ್ನು ಸರಿಮಾಡಿಕೊಂಡು ಬುದ್ಧಿವಂತರಾಗುವವರೆಗೆ ಇವರನ್ನು ನಿಕೃಷ್ಟ ದೃಷ್ಟಿಯಿಂದ ಕಾಣಬೇಕು. ನಾಚಿಕೆಗೇಡು! ಇಂತಹ ಕ್ರೂರ ಅನಾಗರಿಕ ಆಚಾರ ಅವರಲ್ಲಿದೆ. ಅವರ ಮಕ್ಕಳು ಉಪವಾಸದಿಂದ ಸಾಯುವರು. ಅವರು ಧರ್ಮವನ್ನು ಬದಲಾಯಿಸಿದರೆ ಹೊಟ್ಟೆ ತುಂಬಾ ಊಟ ಸಿಕ್ಕುವುದು. ಜಾತಿವಿಷಯದಲ್ಲಿ ಇನ್ನು ಮುಂದೆ ಹೋರಾಟವಿರಕೂಡದು.

ಉಚ್ಚವರ್ಗದವರನ್ನು ಕೆಳಗೆ ಎಳೆಯುವುದಲ್ಲ, ಕೆಳಗಿನವರನ್ನು ಮೇಲಕ್ಕೆ ತರುವುದೇ ಪರಿಹಾರ. ನಮ್ಮ ಶಾಸ್ತ್ರಗಳಲ್ಲಿ ಸೂಚಿಸಿರುವುದೆಲ್ಲಾ ಈ ಮಾರ್ಗವನ್ನೇ. ಭರತಖಂಡದ ಶಾಸ್ತ್ರಗಳ ಪೂರ್ಣ ಪರಿಚಯವಿಲ್ಲದವರು, ನಮ್ಮ ಪೂರ್ವಿಕರ ಬೃಹತ್​ ಯೋಜನಾಕ್ರಮವನ್ನು ತಿಳಿದುಕೊಳ್ಳಲು ಸ್ವಲ್ಪವೂ ಸಾಮರ್ಥ್ಯವಿಲ್ಲದವರು, ಇದಕ್ಕೆ ವಿರುದ್ಧವಾಗಿ ಏನನ್ನೇ ಹೇಳಿದರೂ, ನಮ್ಮ ಶಾಸ್ತ್ರಗಳು ಸೂಚಿಸಿದ್ದು ಈ ಮಾರ್ಗವಲ್ಲದೆ ಬೇರೆಯಲ್ಲ ಎಂಬುದು ಸ್ಪಷ್ಟವಾಗಿದೆ. ನಮ್ಮ ಶಾಸ್ತ್ರಗಳ ಸರಿಯಾದ ಪರಿಚಯ ಇಲ್ಲದವರಿಗೆ ಇದು ಗೊತ್ತಾಗುವುದಿಲ್ಲ. ಯಾರಿಗೆ ಬುದ್ಧಿ ಇದೆಯೋ, ಯಾರು ಅವರ ರೀತಿಯನ್ನು ಸವಿಸ್ತಾರವಾಗಿ ತಿಳಿದುಕೊಳ್ಳಬಲ್ಲರೋ, ಅವರು ತಿಳಿದುಕೊಳ್ಳವರು. ಅವರು ಬದಿಗೆ ಸರಿದು ಅನಾದಿಕಾಲದಿಂದಲೂ ಭರತಖಂಡವು ಅನುಸರಿಸಿದ ಹಾದಿಯಲ್ಲಿ ಹೋಗುತ್ತಾರೆ. ಅವರು ಆಧುನಿಕ ಕಾಲದ ಮತ್ತು ಪೂರ್ವಕಾಲದ ಶಾಸ್ತ್ರಗಳಲ್ಲಿ ಇದನ್ನು ಕ್ರಮಕ್ರಮವಾಗಿ ನಿರೂಪಿಸಿರುವುದನ್ನು ಮನಗಾಣುವರು. ಅವರ ಯೋಜನೆ ಏನು? ಆದರ್ಶದ ತುತ್ತತುದಿ ಬ್ರಾಹ್ಮಣ. ಅದರ ಪ್ರಾರಂಭ ಚಂಡಾಲ. ಚಂಡಾಲನನ್ನು ಬ್ರಾಹ್ಮಣನನ್ನಾಗಿ ಮಾಡುವುದೇ ಅವರ ಮುಖ್ಯ ಗುರಿ. ಕ್ರಮೇಣ ಅವರಿಗೆ ಹೆಚ್ಚು ಹೆಚ್ಚು ಹಕ್ಕುಬಾಧ್ಯತೆಗಳು ಬರುವುವು. ಕೆಲವು ಶಾಸ್ತ್ರಗಳಿವೆ, ಅವು ಹೀಗೆ ಹೇಳುವುವು: “ಶೂದ್ರನು ವೇದವನ್ನು ಕೇಳಿದರೆ ಅವನ ಕಿವಿಗೆ ಕಾದ ಸೀಸವನ್ನು ಸುರಿಯಿರಿ. ಅವನು ಒಂದು ಶ್ಲೋಕವನ್ನು ಉಚ್ಚರಿಸಿದರೆ ನಾಲಗೆಯನ್ನು ಕತ್ತರಿಸಿ. ಬ್ರಾಹ್ಮಣನನ್ನು ‘ಎಲೈ ಬ್ರಾಹ್ಮಣನೇ’ ಎಂದರೆ ಅವನ ನಾಲಗೆಯನ್ನು ಕತ್ತರಿಸಿ.” ಇದೊಂದು ನಿಸ್ಸಂದೇಹವಾಗಿ ಅನಾಗರಿಕ ರಾಕ್ಷಸೀ ಕೃತ್ಯ, ಅದನ್ನು ಹೇಳಲೇಬೇಕಾಗಿಲ್ಲ. ಸ್ಮೃತಿಕಾರರನ್ನು ದೂರಿ ಪ್ರಯೋಜನವಿಲ್ಲ. ಅವರು ಆಗಿನ ಕಾಲದಲ್ಲಿ ಕೆಲವು ಜಾತಿಯಲ್ಲಿ ಬಳಕೆಯಲ್ಲಿದ್ದ ಆಚಾರಗಳನ್ನು ಬರೆದಿರುವರು. ಇಂತಹ ರಾಕ್ಷಸೀಯ ವ್ಯಕ್ತಿಗಳು ಕೆಲವರು ಪೂರ್ವಕಾಲದಲ್ಲಿ ಹುಟ್ಟಿದರು. ಹೆಚ್ಚು ಕಡಮೆ ಇಂತಹ ಆಸುರೀ ವ್ಯಕ್ತಿಗಳು ಎಲ್ಲಾ ಕಾಲದಲ್ಲಿಯೂ ಇದ್ದೇ ಇರುವರು. ಕ್ರಮೇಣ ಅವರು ಉಗ್ರತೆಯನ್ನು ಸ್ವಲ್ಪ ತಗ್ಗಿಸಿರುವರು. “ಶೂದ್ರರಿಗೆ ತೊಂದರೆ ಕೊಡಬೇಡಿ, ಆದರೆ ಉತ್ತಮ ವಿಷಯಗಳನ್ನು ಆವರಿಗೆ ಬೋಧಿಸಬೇಡಿ” ಎನ್ನುವರು. ಕ್ರಮೇಣ ಇತರ ಸ್ಮೃತಿಗಳಲ್ಲಿ, ಯಾವುದು ಇಂದು ಹೆಚ್ಚು ಬಳಕೆಯಲ್ಲಿರುವುದೊ ಅವುಗಳಲ್ಲಿ, ಶೂದ್ರರು ಬ್ರಾಹ್ಮಣರ ಆಚಾರ ವ್ಯವಹಾರಗಳನ್ನು ಅನುಸರಿಸಿದರೆ ಒಳ್ಳೆಯದು ಎಂದು ಹೇಳಿದೆ. ಹಾಗೆ ಮಾಡಲು ಅಗತ್ಯವಾದ ಪ್ರೋತ್ಸಾಹವನ್ನು ಶೂದ್ರರಿಗೆ ನೀಡಬೇಕು. ಅದನ್ನು ಹೇಗೆ ತಿಳಿದುಕೊಳ್ಳುವುದು ಎಂಬುದನ್ನು ಸವಿಸ್ತಾರವಾಗಿ ಹೇಳಲು ನನಗೆ ಸಮಯವಿಲ್ಲ. ವಾಸ್ತವಿಕ ಅಂಶಗಳಿಗೆ ಬಂದರೆ, ಈ ಜಾತಿಗಳೆಲ್ಲಾ ಕ್ರಮೇಣ ಮೇಲೆ ಬರಬೇಕು. ಸಾವಿರಾರು ಉಪಪಂಗಡಗಳಿವೆ. ಬ್ರಾಹ್ಮಣರು ಆಗಲೇ ಕೆಲವರನ್ನು ಸೇರಿಸುತ್ತಿರುವರು. ಇತರರು ತಾವು ಬ್ರಾಹ್ಮಣರೆಂದು ಹೇಳಿಕೊಳ್ಳುವುದಕ್ಕೆ ಏನು ಅಭ್ಯಂತರವಿದೆ? ಅನೇಕ ಕಟ್ಟುನಿಟ್ಟಾದ ನಿಯಮಾವಳಿಗಳಿಂದ ಕೂಡಿದ ಜಾತಿಗಳು ಹೀಗೆ ಸೃಷ್ಟಿಯಾದವು. ಹಲವು ಜಾತಿಗಳಿವೆ, ಪ್ರತಿಯೊಂದರಲ್ಲಿಯೂ ಹತ್ತು ಸಾವಿರ ಜನರಿರುವರು ಎಂದು ಊಹಿಸೋಣ. ಇವರೆಲ್ಲಾ ಒಟ್ಟು ಕಲೆತು ನಾವೆಲ್ಲಾ ಬ್ರಾಹ್ಮಣರು ಎಂದರೆ ಯಾರೂ ಅವರನ್ನು ತಡೆಯುವಂತೆ ಇಲ್ಲ. ನಾನು ಜೀವನದಲ್ಲೇ ಇದನ್ನು ನೋಡಿರುವೆನು. ಕೆಲವು ಜಾತಿಗಳು ಪ್ರಬಲವಾಗುವುವು. ಅವರೆಲ್ಲಾ ಒಪ್ಪಿದರೆ ಯಾರು ಅವರನ್ನು ವಿರೋಧಿಸಬಲ್ಲರು? ಪ್ರತಿಯೊಂದು ಜಾತಿಯೂ ಪ್ರತ್ಯೇಕವಾಗಿತ್ತು, ಇತರ ಜಾತಿಯ ಪಾಡಿಗೆ ಹೋಗುತ್ತಿರಲಿಲ್ಲ. ಒಂದು ಜಾತಿಯ ಹಲವು ಉಪಜಾತಿಗಳು ಕೂಡ ಒಂದು ಮತ್ತೊಂದರ ಗೋಜಿಗೆ ಹೋಗುತ್ತಿರಲಿಲ್ಲ. ಶಂಕರಾಚಾರ್ಯ ಮುಂತಾದ ಯುಗಾವತಾರ ವ್ಯಕ್ತಿಗಳು ಹಲವು ಜಾತಿಗಳನ್ನು ಸೃಷ್ಟಿಸಿದರು. ಅವರು ಮಾಡಿರುವ ಅತಿ ವಿಚಿತ್ರವಾದ ಘಟನೆಗಳನ್ನೆಲ್ಲಾ ಹೇಳಲಾರೆ. ನಿಮ್ಮಲ್ಲಿ ಹಲವರು ಅದನ್ನು ಒಪ್ಪದೆ ಇರಬಹುದು. ಆದರೆ ನನ್ನ ಪ್ರಯಾಣದಲ್ಲಿ ಮತ್ತು ಅನುಭವದಲ್ಲಿ ಇದನ್ನೆಲ್ಲಾ ನಾನು ಕಂಡು ಹಿಡಿದು ಒಂದು ಅದ್ಭುತ ನಿರ್ಣಯಕ್ಕೆ ಬಂದಿರುವೆನು. ಬೆಲೂಚಿ ದೇಶದ ಕೆಲವು ಜನರನ್ನು ಕ್ಷತ್ರಿಯರನ್ನಾಗಿ ಮಾಡುತ್ತಿದ್ದರು. ಬೆಸ್ತರಪಡೆಯನ್ನೆಲ್ಲಾ ಬ್ರಾಹ್ಮಣರನ್ನಾಗಿ ಮಾಡುತ್ತಿದ್ದರು. ಅವರೆಲ್ಲಾ ಋಷಿಗಳು, ಮಹಾಮುನಿಗಳು. ಅವರ ಸ್ಮೃತಿಗೆ ನಾವು ಗೌರವ ಕೊಡಬೇಕು. ನೀವೆಲ್ಲಾ ಋಷಿಮುನಿಗಳಾಗಿ. ಇಲ್ಲೇ ಗೂಢ ರಹಸ್ಯವಿರುವುದು. ಹೆಚ್ಚು ಕಡಿಮೆ ನಾವೆಲ್ಲಾ ಋಷಿಗಳಾಗುವೆವು. ಋಷಿ ಎಂದರೆ ಏನು? ಪರಿಶುದ್ಧನು. ಮೊದಲು ಪರಿಶುದ್ಧರಾಗಿ, ಅನಂತರ ಶಕ್ತಿ ಬರುವುದು. “ನಾನು ಋಷಿ” ಎಂದು ಸುಮ್ಮನೆ ಹೇಳಿದರೆ ಪ್ರಯೋಜನವಿಲ್ಲ. ನೀವು ಋಷಿಗಳಾದರೆ ಇತರರು ಅರಿವಿಲ್ಲದೆ ನಿಮ್ಮನ್ನು ಅನುಸರಿಸುವರು. ನಿಮ್ಮಿಂದ ಯಾವುದೋ ಹೇಳಲಾರದ್ದೊಂದು ಹೊರಹೊಮ್ಮುವುದು. ಇದೇ ಅವರು ನಿಮ್ಮನ್ನು ಅನುಸರಿಸುವಂತೆ ಮಾಡುವುದು, ನೀವು ಹೇಳಿದ್ದನ್ನು ಕೇಳುವಂತೆ ಮಾಡುವುದು. ಅರಿವಿಲ್ಲದೆ ತಮ್ಮ ಇಚ್ಛೆಗೆ ವಿರೋಧವಾಗಿಯಾದರೂ ನೀವು ಹೇಳಿದ್ದನ್ನು ಕೇಳುವಂತೆ ಮಾಡುವುದು. ಇದು ಋಷಿತ್ವ.

ಇನ್ನು ವಿವರಗಳ ಮಾತು. ಅವನ್ನು ಕಾರ್ಯರೂಪಕ್ಕೆ ತರಲು ಹಲವು ತಲೆ ಮಾರುಗಳೇ ಬೇಕಾಗುತ್ತವೆ. ಈ ಕಲಹ ಕೊನೆಗಾಣುವುದಕ್ಕೆ ಇದೊಂದು ಸಲಹೆ ಮಾತ್ರ. ಆಧುನಿಕ ಕಾಲದಲ್ಲಿ ಜಾತಿಯ ಸಂಬಂಧದಲ್ಲಿ ಇಷ್ಟು ತರ್ಕ ವಿತರ್ಕ ನಡೆಯುವುದನ್ನು ನೋಡಿ ನನಗೆ ವ್ಯಥೆಯಾಗುವುದು. ಇದು ಕೊನೆಗಾಣಬೇಕು. ಇದು ಉಭಯಪಕ್ಷದವರಿಗೂ ವ್ಯರ್ಥ. ಉಚ್ಚವರ್ಗದ ಬ್ರಾಹ್ಮಣರ ದೃಷ್ಟಿಯಿಂದ ಇದು ಪ್ರಯೋಜನವಿಲ್ಲ. ಪ್ರತ್ಯೇಕ ಹಕ್ಕುಬಾಧ್ಯತೆಗಳ ಕಾಲ ಆಗಿಹೋಯಿತು. ಉಚ್ಚ ಜಾತಿವರ್ಗದವರು ತಮ್ಮ ಸಮಾಧಿಯನ್ನು ತಾವು ಎಷ್ಟು ಬೇಗ ಸಿದ್ಧ ಮಾಡಿಕೊಳ್ಳುತ್ತಾರೆ, ಅವರಿಗೆ ಅದು ಅಷ್ಟು ಒಳ್ಳೆಯದೇ. ಕಾಲ ವಿಳಂಬ ಮಾಡಿದಷ್ಟೂ ಕೊಳೆಯುವುದು, ಅಷ್ಟೂ ಭಯಂಕರವಾಗುವುದು ಅದರ ಕೊನೆಗಾಲ. ಭರತಖಂಡದಲ್ಲಿ ಇತರರ ಮುಕ್ತಿಗೆ ಕೆಲಸ ಮಾಡುವುದು ಬ್ರಾಹ್ಮಣರ ಕರ್ತವ್ಯ. ಅವರು ಅದನ್ನು ಮಾಡಿದರೆ ಬ್ರಾಹ್ಮಣರು. ಮಾಡದೆ ದ್ರವ್ಯಾರ್ಜನೆಗೆ ಹೊರಟರೆ ಅವರು ಬ್ರಾಹ್ಮಣರಲ್ಲ. ಯಾರು ಯೋಗ್ಯ ಬ್ರಾಹ್ಮಣರೋ ಅವರಿಗೆ ಮಾತ್ರ ನೀವು ಸಹಾಯ ಮಾಡಬೇಕು. ಅದರಿಂದ ಸ್ವರ್ಗ ಲಭಿಸುವುದು. ಆದರೆ ಅಯೋಗ್ಯನಿಗೆ ದಾನ ಮಾಡಿದರೆ ನರಕವೇ ಗತಿ ಎಂದು ನಮ್ಮ ಶಾಸ್ತ್ರಗಳು ಹೇಳುವುವು. ನೀವು ಈ ವಿಚಾರದಲ್ಲಿ ಜೋಪಾನವಾಗಿರಬೇಕು. ದ್ರವ್ಯಾರ್ಜನೆಯ ವೃತ್ತಿ ಯಾರಿಗೆ ಇಲ್ಲವೋ ಅವನೇ ಬ್ರಾಹ್ಮಣ. ದ್ರವ್ಯಾರ್ಜನೆಯ ವೃತ್ತಿ ಬ್ರಾಹ್ಮಣನಿಗಲ್ಲ, ಇತರೆ ವರ್ಗದವರಿಗೆ. ಬ್ರಾಹ್ಮಣರನ್ನು ನಾನು ಕೇಳಿಕೊಳ್ಳುತ್ತೇನೆ: ಅವರು ಭಾರತೀಯ ಜನರಿಗೆ ತಮಗೆ ತಿಳಿದಿರುವುದನ್ನು ಕಲಿಸಿಕೊಟ್ಟು, ಶತಮಾನಗಳಿಂದ ಸಂಗ್ರಹಿಸಿದ ತಮ್ಮ ಸಂಸ್ಕೃತಿಯನ್ನು ಅವರಿಗೆ ಕೊಟ್ಟು ಅವರನ್ನು ಮೇಲೆತ್ತಲು ಶ್ರಮಪಡಬೇಕು. ನಿಜವಾದ ಬ್ರಾಹ್ಮಣರ ಆದರ್ಶವೇನು ಎಂದು ತಿಳಿದುಕೊಳ್ಳುವುದು ಭರತಖಂಡದ ಬ್ರಾಹ್ಮಣರ ಕರ್ತವ್ಯ. ಬ್ರಾಹ್ಮಣರಿಗೆ ಇಷ್ಟೊಂದು ಗೌರವವನ್ನು ಮತ್ತು ಹಕ್ಕನ್ನು ಕೊಡುವುದಕ್ಕೆ ಕಾರಣ ಅವರು “ಸದ್ಗುಣಗಳ ಭಂಡಾರ” ಆಗಿರುವುದು ಎಂದು ಮನು ಹೇಳುತ್ತಾನೆ:

\begin{verse}
\textbf{ಬ್ರಾಹ್ಮಣೋ ಜಾಯಮಾನೋ ಹಿ ಪೃಥಿವ್ಯಾಮಧಿಜಾಯತೇ~।}\\\textbf{ಈಶ್ವರಃ ಸರ್ವಭೂತಾನಾಂ ಧರ್ಮಕೋಶಸ್ಯ ಗುಪ್ತಯೇ~॥}
\end{verse}

\vauthor{ಮನುಸ್ಮೃತಿ ೧.೯೯}

ಅವನು ಈ ಭಂಡಾರವನ್ನು ತೆರೆದು ಅನರ್ಘ್ಯರತ್ನಗಳನ್ನು ಜಗತ್ತಿಗೆ ನೀಡಬೇಕು. ಅವನೇ ಭರತಖಂಡದ ಪ್ರಥಮ ಪ್ರಚಾರಕನೆಂಬುದು ನಿಜ. ಜೀವನದ ಪರಮಗುರಿಯನ್ನು ಸೇರುವುದಕ್ಕೆ ಇತರರಿಗಿಂತ ಮುಂಚೆ ಸರ್ವತ್ಯಾಗಮಾಡಿದವನು ಅವನು. ಇತರ ಜಾತಿಗಳಿಗಿಂತ ಮುಂದೆ ಹೋದುದು ಅವನ ತಪ್ಪಲ್ಲ. ಇತರ ಜಾತಿಯವರು ಅವನಂತೆ ತಿಳಿದುಕೊಂಡು ಏಕೆ ಮಾಡಲಿಲ್ಲ? ಅವರೇಕೆ ಸುಮ್ಮನೆ ಕುಳಿತು ಸೋಮಾರಿಗಳಾಗಿರಬೇಕು? ಬ್ರಾಹ್ಮಣರು ಮುಂದೆ ಹೋಗುವುದಕ್ಕೆ ಏಕೆ ಅವಕಾಶಕೊಡಬೇಕು?

ಇಂತಹ ಅವಕಾಶವನ್ನು ಪಡೆಯುವುದೊಂದು, ಅದನ್ನು ದುರುದ್ದೇಶಕ್ಕೆ ಬಳಸುವುದು ಮತ್ತೊಂದು. ಅಧಿಕಾರವನ್ನು ಸ್ವಾರ್ಥಕ್ಕೆ ಉಪಯೋಗಿಸಿಕೊಳ್ಳುವುದು ಮಹಾ ಪಾಪ. ಅದನ್ನು ಒಳ್ಳೆಯದಕ್ಕೆ ಮಾತ್ರ ಉಪಯೋಗಿಸಬೇಕು. ಹಲವು ಶತಮಾನಗಳಿಂದ ಸಂಗ್ರಹಿಸಿಟ್ಟ ಸಾಂಸ್ಕೃತಿಕನಿಧಿಗೆ ಬ್ರಾಹ್ಮಣ ಈಗ ರಕ್ಷಕನಾಗಿರುವನು. ಅದನ್ನು ಅವನು ಈಗ ಎಲ್ಲರಿಗೂ ದಾನಮಾಡಬೇಕು. ಅವನು ಜನರಿಗೆ ಕೊಡದೆ ಇದ್ದ ಕಾರಣ ಮಹಮ್ಮದೀಯರ ದಾಳಿ ಸಾಧ್ಯವಾಯಿತು. ಈ ಅನರ್ಘ್ಯ ರತ್ನರಾಶಿಯನ್ನು ಮೊದಲಿನಿಂದ ಜನರಿಗೆ ಕೊಡದ ಕಾರಣ, ಕಳೆದ ಒಂದು ಸಾವಿರ ವರ್ಷದಿಂದಲೂ ಭರತಖಂಡಕ್ಕೆ ಬರಲು ಬಯಸಿದ ಎಲ್ಲರ ಗುಲಾಮರಾಗಿರುವೆವು ನಾವು. ಇದರಿಂದಲೇ ನಾವು ಅಧೋಗತಿಗೆ ಬಂದುದು. ನಮ್ಮ ಪೂರ್ವಿಕರು ಸಂಗ್ರಹಿಸಿಟ್ಟ ಅನರ್ಘ್ಯ ರತ್ನರಾಶಿ ಇರುವ ಗುಹೆಯನ್ನು ಭೇದಿಸಿ ಹೊರಗೆ ತೆಗೆದು ಎಲ್ಲರಿಗೂ ದಾನಮಾಡುವುದು ನಮ್ಮ ಪ್ರಥಮ ಕರ್ತವ್ಯ. ಬ್ರಾಹ್ಮಣ ಇದನ್ನು ಮೊದಲು ಮಾಡಬೇಕು. ಬಂಗಾಳ ದೇಶದಲ್ಲಿ ಒಂದು ಹಳೆಯ ಮೂಢನಂಬಿಕೆ ಇದೆ. ಯಾವ ನಾಗರಹಾವು ಕಚ್ಚಿದೆಯೋ ಅದೇ ಹಾವು ಪುನಃ ವಿಷವನ್ನು ಕಚ್ಚಿದವನಿಂದ ಹೀರಿದರೆ ಕಡಿಸಿಕೊಂಡವನು ಬದುಕುವನು ಎನ್ನುವರು. ಹಾಗಾದರೆ ಬ್ರಾಹ್ಮಣ ಮೊದಲು ತನ್ನ ವಿಷವನ್ನು ಪುನಃ ಹೀರಿಕೊಳ್ಳಲಿ. ಬ್ರಾಹ್ಮಣೇತರರಿಗೆ, ತಾಳಿ, ಅವಸರಪಡಬೇಡಿ ಎನ್ನುವೆನು. ಬ್ರಾಹ್ಮಣರೊಂದಿಗೆ ಹೋರಾಡುವುದಕ್ಕೆ ಪ್ರತಿಯೊಂದು ಅವಕಾಶವನ್ನೂ ತೆಗೆದುಕೊಳ್ಳಬೇಡಿ. ನಿಮ್ಮ ತಪ್ಪಿನಿಂದ ನೀವು ಅನುಭವಿಸುತ್ತಿರುವಿರಿ. ಇದನ್ನು ನಾನು ಆಗಲೇ ತೋರಿರುವೆನು. ಅಧ್ಯಾತ್ಮವಿದ್ಯೆ, ಸಂಸ್ಕೃತಜ್ಞಾನ ಇವನ್ನು ಮರೆಯುವಂತೆ ಯಾರು ನಿಮಗೆ ಹೇಳಿದರು? ಇಂದಿನವರೆಗೂ ನೀವು ಏನು ಮಾಡುತ್ತಿದ್ದಿರಿ? ನೀವು ಏತಕ್ಕೆ ತಾತ್ಸಾರದಿಂದ ಇದ್ದಿರಿ? ಇತರರಿಗೆ ನಿಮಗಿಂತ ಹೆಚ್ಚು ಬುದ್ಧಿ ಇತ್ತು, ಶಕ್ತಿ ಇತ್ತು, ಛಲವಿತ್ತು ಎಂದು ಈಗ ಏತಕ್ಕೆ ಅಸಮಾಧಾನಪಡುವಿರಿ? ವೃತ್ತಪತ್ರಿಕೆಗಳಲ್ಲಿ ಟೀಕಿಸುವುದು, ವಾದಿಸುವುದು, ನಿಮ್ಮ ಮನೆಗಳಲ್ಲಿ ಜಗಳ ಕಾಯುವುದು - ಇವು ಪಾಪಕರ. ಇವನ್ನು ತ್ಯಜಿಸಿ, ಬ್ರಾಹ್ಮಣರ ಸಂಸ್ಕೃತಿಯನ್ನು ಗಳಿಸುವುದಕ್ಕೆ ಯತ್ನಿಸಿ. ಆಗ ಕಾರ್ಯಸಾಧನೆಯಾಗುವುದು. ನೀವೇಕೆ ಸಂಸ್ಕೃತ ವಿದ್ವಾಂಸರಾಗಬಾರದು? ಎಲ್ಲಾ ಜಾತಿಗಳಿಗೂ ಸಂಸ್ಕೃತ ಭಾಷೆಯನ್ನು ಹರಡುವುದಕ್ಕೆ ಕೋಟ್ಯಂತರ ರೂಪಾಯಿಗಳನ್ನು ಏತಕ್ಕೆ ಖರ್ಚು ಮಾಡಬಾರದು? ಅದೇ ಪ್ರಶ್ನೆ. ಇದನ್ನು ಮಾಡಿದೊಡನೆಯೇ ಬ್ರಾಹ್ಮಣನಿಗೆ ನೀವು ಸರಿಸಮರಾಗುತ್ತೀರಿ; ಭಾರತದಲ್ಲಿ ಶಕ್ತಿ ಲಾಭದ ಗೂಢರಹಸ್ಯವೇ ಇದು.

ಭಾರತದಲ್ಲಿ ಸಂಸ್ಕೃತ ಭಾಷೆ ಮತ್ತು ಗೌರವ ಒಟ್ಟಿಗೆ ಹೋಗುವುದು. ನಿಮ್ಮಲ್ಲಿ ಅದು ಇದ್ದರೆ ಯಾರಿಗೂ ನಿಮ್ಮನ್ನು ಟೀಕಿಸಲು ಆಗುವುದಿಲ್ಲ. ಇದೇ ಮುಖ್ಯ ರಹಸ್ಯ. ಇದನ್ನು ಅನುಸರಿಸಿ. ಜಗತ್ತೆಲ್ಲ ಅದ್ವೈತಿ ಹೇಳುವಂತೆ ಭ್ರಮೆಯಾಗಿರುವುದು. ಸಂಕಲ್ಪವೇ ಶಕ್ತಿ. ಯಾರು ಪ್ರಚಂಡ ಇಚ್ಛಾಶಕ್ತಿಯ ಅಧಿಕಾರಿಯೋ ಅವರು ಒಂದು ಕಾಂತಿಯಿಂದ ಪರಿವೃತರಾದವರಂತೆ, ಉಳಿದೆಲ್ಲರನ್ನೂ ತಮ್ಮ ಮಾನಸಿಕ ಸ್ಥಿತಿಗೆ ತರುವರು. ಅಂತಹ ಮಹಾಪುರುಷರು ಬರುವರು. ಅಂತಹ ಪ್ರಬಲ ವ್ಯಕ್ತಿ ಬಂದರೆ ಅವನು ತನ್ನ ಭಾವನೆಯನ್ನು ತನ್ನ ವ್ಯಕ್ತಿತ್ವದ ಮೂಲಕ ನಮ್ಮ ಮೇಲೆ ಬೀರುವನು. ನಮ್ಮಲ್ಲಿಯೂ ಹಲವರಿಗೆ ಅದೇ ಭಾವನೆ ಬರುವುದು. ನಾವು ಶಕ್ತಿಶಾಲಿಗಳಾಗುವೆವು. ಸಂಸ್ಥೆಯಲ್ಲಿ ಹೇಗೆ ಇಷ್ಟೊಂದು ಶಕ್ತಿ ಇರುವುದು? ಸಂಸ್ಥೆ ಜಡವೆಂದು ಭಾವಿಸಬೇಡಿ. ಒಂದು ಉದಾಹರಣೆಯನ್ನು ತೆಗೆದುಕೊಳ್ಳೋಣ. ನಾಲ್ಕು ಕೋಟಿ ಆಂಗ್ಲೇಯರು ಮೂವತ್ತು ಕೋಟಿ ಭಾರತೀಯರನ್ನು ಹೇಗೆ ಆಳುವರು? ಈ ಪ್ರಶ್ನೆಯ ಉತ್ತರಕ್ಕೆ ಮನೋವಿಜ್ಞಾನ ಏನು ಹೇಳುವುದು? ನಾಲ್ಕು ಕೋಟಿ ಜನರು ತಮ್ಮ ಸಂಕಲ್ಪವನ್ನೆಲ್ಲಾ ಒಟ್ಟುಗೂಡಿಸುವರು. ಇದರಿಂದ ಅದ್ಭುತ ಶಕ್ತಿ ಉತ್ಪತ್ತಿಯಾಗುವುದು. ನಿಮ್ಮ ಮೂವತ್ತು ಕೋಟಿ ಜನರ ಇಚ್ಛೆಯೂ ಬೇರೆ ಬೇರೆಯಾಗಿರುವುದು. ಭವ್ಯ ಭರತಖಂಡವನ್ನು ನಾವು ಸೃಷ್ಟಿಸಬೇಕಾದರೆ, ಅದರ ರಹಸ್ಯವೆಲ್ಲ, ಸಂಘಟನೆಯಲ್ಲಿ, ಶಕ್ತಿ ಸಂಗ್ರಹದಲ್ಲಿ, ಮತ್ತು ಇಚ್ಛಾಶಕ್ತಿಯನ್ನು ಏಕತ್ರ ಮಾಡುವುದರಲ್ಲಿದೆ.

ಋಗ್ವೇದ ಸಂಹಿತೆಯ ಶ್ಲೋಕವೊಂದು ಮನಸ್ಸಿಗೆ ಬರುವುದು:

\begin{verse}
\textbf{“ಸಂಗಚ್ಛಧ್ವಂ ಸಂವದಧ್ವಂ ಸಂ ವೋ ಮನಾಂಸಿ}\\\textbf{ಜಾನತಾಮ್​! ದೇವಾ ಭಾಗಂ ಯಥಾ ಪೂರ್ವೇ”}
\end{verse}

“ನೀವೆಲ್ಲ ಒಂದು ಮನಸ್ಸಿನವರಾಗಿ, ಒಂದು ಭಾವನೆಯವರಾಗಿ, ಪ್ರಾಚೀನ ಕಾಲದಲ್ಲಿ ಒಮ್ಮತದವರಾದುದರಿಂದ ದೇವತೆಗಳಿಗೆ ಬಲಿಯನ್ನು ಸ್ವೀಕರಿಸಲು ಸಾಧ್ಯವಾಯಿತು.” ಮನುಷ್ಯರೂ ಏಕಮತದವರಾಗಿರುವುದರಿಂದಲೇ ದೇವತೆಗಳನ್ನು ಪೂಜಿಸಲು ಸಾಧ್ಯವಾಯಿತು. ಒಮ್ಮತವೇ ಸಮಾಜಶಕ್ತಿಯ ರಹಸ್ಯ. ದ್ರಾವಿಡ, ಆರ್ಯ, ಬ್ರಾಹ್ಮಣ, ಬ್ರಾಹ್ಮಣೇತರ ಎಂಬ ಪ್ರಶ್ನೆಯ ಮೇಲೆ ನೀವು ಹೋರಾಡಿದಷ್ಟೂ ಭವಿಷ್ಯ ಭಾರತ ನಿರ್ಮಾಣಕ್ಕೆ, ಮಾಡಬೇಕಾದ ಶಕ್ತಿ ಸಂಗ್ರಹಕ್ಕೆ ದೂರವಾಗುವೆವು. ಇದನ್ನು ಗಮನದಲ್ಲಿಡಿ. ಭವಿಷ್ಯ ಭಾರತ ಸಂಪೂರ್ಣವಾಗಿ ಇದರ ಮೇಲೆ ನಿಂತಿದೆ. ಈ ಇಚ್ಛಾಶಕ್ತಿಯನ್ನು ಸಂಗ್ರಹಿಸುವುದು, ಅದನ್ನು ಏಕೀಭೂತ ಮಾಡುವುದು, ಶತಮುಖವಾಗಿ ಹರಿಯುತ್ತಿರುವುದನ್ನು ಏಕಮುಖವಾಗಿ ಮಾಡುವುದು, ಇದೇ ರಹಸ್ಯ. ಪ್ರತಿಯೊಬ್ಬ ಚೈನೀಯನು ಬೇರೆಯಾಗಿ ಆಲೋಚಿಸುವನು. ಕೆಲವು ಜನ ಜಪಾನೀಯರು ಒಂದೇ ರೀತಿ ಆಲೋಚಿಸುವರು. ಅದರ ಫಲ ನಿಮಗೇ ಗೊತ್ತಿದೆ. ಜಗತ್ತಿನ ಇತಿಹಾಸದಲ್ಲೆಲ್ಲಾ ಹೀಗೇ ಆಗುತ್ತಿರುವುದು. ಎಲ್ಲೆಡೆಯಲ್ಲಿಯೂ ಅಲ್ಪಸಂಖ್ಯಾತ ಚಿಕ್ಕ ಚಿಕ್ಕ ರಾಷ್ಟ್ರಗಳು ದೊಡ್ಡ ದೊಡ್ಡ ರಾಷ್ಟ್ರಗಳನ್ನು ಆಳುತ್ತಿವೆ. ಇದು ಸ್ವಾಭಾವಿಕವಾಗಿಯೇ ಇದೆ. ಸಣ್ಣ ಸಣ್ಣ ರಾಷ್ಟ್ರಗಳಿಗೆ ತಮ್ಮ ಭಾವನೆಗಳನ್ನು ಕೇಂದ್ರೀಕರಿಸುವುದು ಸುಲಭ. ಅವು ಅದರಿಂದ ಅಭಿವೃದ್ಧಿಯಾಗುವುದು. ಹೆಚ್ಚು ಜನರಿರುವ ದೇಶದಲ್ಲಿ ಇದು ಬಹಳ ಕಷ್ಟ. ಚೆಲ್ಲಾಪಿಲ್ಲಿಯಾಗಿರುವುದರಿಂದ ಮತ್ತು ದೊಂಬಿಯಾಗಿರುವುದರಿಂದ ಅವರೆಲ್ಲರೂ ಒಟ್ಟಿಗೆ ಕಲೆಯುವುದಿಲ್ಲ. ಈ ಮತಭೇದವೆಲ್ಲಾ ಒಂದೇ ಸಲ ನಿಲ್ಲಬೇಕು.

ನಮ್ಮಲ್ಲಿ ಮತ್ತೊಂದು ದೊಡ್ಡ ದೋಷವಿದೆ. ಮಹಿಳೆಯರು ದಯವಿಟ್ಟು ಕ್ಷಮಿಸಬೇಕು. ಹಲವು ಶತಮಾನಗಳ ಗುಲಾಮಗಿರಿಯಿಂದ ನಾವೆಲ್ಲ ಸ್ತ್ರೀ ರಾಜ್ಯಕ್ಕೆ ಸೇರಿದವರಂತೆ ಇರುವೆವು. ಈ ದೇಶದಲ್ಲಾಗಲಿ ಅಥವಾ ಅನ್ಯದೇಶದಲ್ಲಾಗಲಿ ಮೂರು ಸ್ತ್ರೀಯರನ್ನು ಐದು ನಿಮಿಷ ಒಟ್ಟಿಗೆ ಸೇರುವಂತೆ ಮಾಡುವುದು ಕಷ್ಟ. ಆಗಲೇ ಅವರು ಜಗಳ ಕಾಯುವರು. ಐರೋಪ್ಯ ದೇಶಗಳಲ್ಲಿ ದೊಡ್ಡ ದೊಡ್ಡ ಸ್ತ್ರೀ ಸಂಸ್ಥೆಗಳಿವೆ. ಸ್ತ್ರೀ ಶಕ್ತಿಯ ವಿಷಯವಾಗಿ ದೊಡ್ಡ ದೊಡ್ಡ ಘೋಷಣೆ ಮಾಡುವರು. ಅನಂತರ ತಮ್ಮೊಳಗೇ ಜಗಳ ಕಾಯುವರು. ಒಬ್ಬ ಗಂಡಸು ಬಂದು ಅವರನ್ನೆಲ್ಲ ಆಳುವನು. ನಾವೆಲ್ಲ ಹಾಗೆ ಇರುವೆವು. ನಾವೆಲ್ಲ ಸ್ತ್ರೀಯರು. ಒಬ್ಬ ಸ್ತ್ರೀ, ಸ್ತ್ರೀಯರಿಗೆ ನಾಯಕಳಾದರೆ ಅವಳನ್ನು ದೂರುವುದಕ್ಕೆ ಪ್ರಾರಂಭಿಸುವರು, ಖಂಡಿಸುವರು, ಅವಳನ್ನು ಸುಮ್ಮನೆ ಇರಿಸುವರು. ಒಬ್ಬ ಗಂಡಸು ಬಂದು ಅವರನ್ನು ಮಧ್ಯೆ ಮಧ್ಯೆ ಹೊಡೆದು ಬೈದರೂ ಪರವಾಗಿಲ್ಲ. ಇಂತಹ ವಶೀಕರಣ ಅವರಿಗೆ ಅಭ್ಯಾಸವಾಗಿದೆ. ಪ್ರಪಂಚದಲ್ಲೆಲ್ಲಾ ಇಂತಹ ವಶೀಕಾರಕ ಸಮ್ಮೋಹಿನೀ ವಿದ್ಯೆಯವರು ಇರುವರು. ಇದರಂತೆಯೇ ಯಾರಾದರೂ ನಮ್ಮ ದೇಶೀಯರು ಪ್ರಖ್ಯಾತರಾಗಲು ಪ್ರಯತ್ನಿಸಿದರೆ ಅವರನ್ನು ಕೆಳಗೆ ಎಳೆಯಲು ನಾವೆಲ್ಲಾ ಪ್ರಯತ್ನಿಸುವೆವು. ಹೊರಗಿನವರು ಬಂದು ನಮ್ಮನ್ನು ಒದ್ದರೆ ಚಿಂತೆಯಿಲ್ಲ. ಇದು ನಮಗೆ ಅಭ್ಯಾಸವಾಗಿದೆ ಯಲ್ಲವೆ? ಗುಲಾಮರು ಮಹಾನ್​ ಪ್ರಭುಗಳಾಗಬೇಕು! ಮೊದಲು ಗುಲಾಮ ತನದಿಂದ ಪಾರಾಗಿ.

ಬರುವ ಐವತ್ತು ವರ್ಷಗಳವರೆಗೆ ಈ ಮಾತೃಭೂಮಿಯೇ ನಮ್ಮ ಆರಾಧನೆಯ ಇಷ್ಟದೈವವಾಗಬೇಕು. ಇನ್ನುಳಿದ ವ್ಯರ್ಥ ದೇವತೆಗಳೆಲ್ಲ ಕೆಲವು ಕಾಲ ನಮ್ಮಿಂದ ಮರೆಯಾಗಲಿ. ಜಾಗ್ರತನಾಗಿರುವ ದೇವರು ಇದೊಂದೇ, ಇದೇ ನಮ್ಮ ಜನಾಂಗ. ಎಲ್ಲೆಲ್ಲೂ ಅವನ ಕಾಲುಗಳೇ, ಎಲ್ಲೆಲ್ಲೂ ಅವನ ಕೈಗಳೇ, ಎಲ್ಲೆಲ್ಲೂ ಅವನ ಕಿವಿಗಳೇ. ಅವನೇ ಸರ್ವವ್ಯಾಪಿಯಾಗಿರುವನು. ಇತರ ದೇವತೆಗಳೆಲ್ಲ ನಿದ್ರಿಸುತ್ತಿರುವರು. ನಮ್ಮ ಸುತ್ತಲೂ ಇರುವ ವಿರಾಟ್​ ಮಹೇಶ್ವರನ ಆರಾಧನೆಯನ್ನು ತೊರೆದು ಕೆಲಸಕ್ಕೆ ಬಾರದ ಇತರ ದೇವರನ್ನು ಏಕೆ ಅರಸಿಕೊಂಡು ಹೋಗುವುದು? ನಾವು ಇದನ್ನು ಆರಾಧಿಸಿದರೆ ಇತರ ದೇವತೆಗಳನ್ನು ಆರಾಧಿಸಿದಂತೆ. ಅರ್ಧ ಮೈಲನ್ನು ಕೂಡ ತೆವಳಿಕೊಂಡು ಹೋಗಲಾರೆವು, ಆದರೂ ಹನುಮನಂತೆ ಸಾಗರವನ್ನೇ ನೆಗೆಯಲು ಇಚ್ಛಿಸುವೆವು! ಇದು ಸಾಧ್ಯವಿಲ್ಲ. ಪ್ರತಿಯೊಬ್ಬನೂ ಯೋಗಿಯಾಗುವವನೆ! ಪ್ರತಿಯೊಬ್ಬನೂ ಧ್ಯಾನಮಗ್ನನಾಗುವನೆ! ಇದು ಸಾಧ್ಯವಿಲ್ಲ. ಹಗಲೆಲ್ಲ ಪ್ರಾಪಂಚಿಕತೆಯಲ್ಲಿ, ಕರ್ಮದಲ್ಲಿ, ನಿರತರಾಗಿ ರಾತ್ರಿ ಕಣ್ಣುಮುಚ್ಚಿಕೊಂಡು ಪ್ರಾಣಾಯಾಮ ಮಾಡುವುದು! ಇದು ಅಷ್ಟು ಸುಲಭವೆ? ಮೂಗಿನ ಮೂಲಕ ನೀವು ಮೂರು ವೇಳೆ ಪ್ರಾಣಾಯಾಮ ಮಾಡಿಬಿಟ್ಟರೆ ಋಷಿಗಳು ಆಕಾಶದಿಂದ ನಿಮ್ಮ ಹತ್ತಿರ ಕೆಳಗೆ ಇಳಿಯುವರೆ? ಇದು ತಮಾಷೆಯೇ? ಇದರಲ್ಲಿ ಅರ್ಥವಿಲ್ಲ. ಮೊದಲು ಬೇಕಾಗಿರುವುದು ಚಿತ್ತಶುದ್ಧಿ. ಅದು ಹೇಗೆ ಸಿದ್ಧಿಸುವುದು? ಪ್ರಥಮದಲ್ಲಿ ಮಾಡಬೇಕಾದ ಪೂಜೆಯೇ ನಮ್ಮ ಸುತ್ತಲೂ ಇರುವ ವಿರಾಟನ ಪೂಜೆ. \enginline{‘Worship’} (ಪೂಜೆ) ಎನ್ನುವುದೇ ಸಂಸ್ಕೃತ ಶಬ್ದಕ್ಕೆ ಸರಿಯಾದ ಪರ್ಯಾಯ ಪದ. ಇಂಗ್ಲೀಷಿನಲ್ಲಿ ಮತ್ತಾವ ಪದವೂ ಇದನ್ನು ಹೋಲುವುದಿಲ್ಲ. ಮನುಷ್ಯರು ಪ್ರಾಣಿಗಳು ಇವರೆಲ್ಲ ನಮ್ಮ ದೇವರು. ನಾವು ಮೊದಲು ಪೂಜಿಸಬೇಕಾದ ದೇವರೆ ನಮ್ಮ ದೇಶಬಾಂಧವರು. ಅವರನ್ನು ಪೂಜಿಸಬೇಕು. ನಮ್ಮನಮ್ಮಲ್ಲಿ ಜಗಳ ಕಾಯುವುದು, ಪರಸ್ಪರ ಅಸೂಯೆಪಡುವುದು ನಿಲ್ಲಬೇಕು. ಇದೊಂದು ಭಯಂಕರ ಕರ್ಮ. ಇದಕ್ಕಾಗಿ ನಾವೆಲ್ಲ ಇಂದು ವ್ಯಥೆಪಡುತ್ತಿರುವೆವು. ಆದರೂ ನಾವು ಜಾಗೃತರಾಗಿಲ್ಲ.

ವಿಷಯವೂ ವಿಸ್ತಾರವಾಗಿದೆ, ಅದನ್ನು ಎಲ್ಲಿ ನಿಲ್ಲಿಸಬೇಕೊ ಗೊತ್ತಿಲ್ಲ. ಮದ್ರಾಸಿನಲ್ಲಿ ಏನು ಮಾಡಬೇಕೆಂದಿರುವೆನು ಎಂಬುದನ್ನು ಸಂಕ್ಷೇಪವಾಗಿ ಹೇಳಿ ಭಾಷಣವನ್ನು ಮುಕ್ತಾಯಗೊಳಿಸುವೆನು. ರಾಷ್ಟ್ರದ ಆಧ್ಯಾತ್ಮಿಕ ಮತ್ತು ಲೌಕಿಕ ಶಿಕ್ಷಣದ ಮೇಲೆ ನಮಗೆ ಒಂದು ಹತೋಟಿ ಇರಬೇಕು. ನಿಮಗೆ ಇದು ಗೊತ್ತಾಗುವುದೆ? ಇದನ್ನು ನೀವು ಕನಸು ಕಾಣಬೇಕು, ಮಾತನಾಡಬೇಕು, ಆಲೋಚಿಸಬೇಕು, ಇದನ್ನು ಅನುಷ್ಠಾನಕ್ಕೆ ತರಬೇಕು. ಅಲ್ಲಿಯವರೆಗೂ ನಮ್ಮ ದೇಶಕ್ಕೆ ಮೋಕ್ಷ ಇಲ್ಲ. ನಿಮಗೆ ಈಗ ದೊರೆಯುತ್ತಿರುವ ಶಿಕ್ಷಣದಲ್ಲಿ ಕೆಲವು ಒಳ್ಳೆಯ ಅಂಶಗಳಿವೆ. ಆದರೆ ಅದರಲ್ಲಿ ಒಂದು ಮಹಾಲೋಪವಿದೆ, ಅದು ಒಳ್ಳೆಯ ಅಂಶಗಳನ್ನೂ ನಿಷ್ಕ್ರಿಯಗೊಳಿಸಿದೆ. ಇದು ಪುರುಷಸಿಂಹರನ್ನು ತಯಾರು ಮಾಡುವ ವಿದ್ಯಾಭ್ಯಾಸವಲ್ಲ. ಇದು ಕೇವಲ ನಿಷೇಧಮಯವಾದ ವಿದ್ಯಾಭ್ಯಾಸ. ನಿಷೇಧಮಯವಾದ ವಿದ್ಯಾಭ್ಯಾಸ ಅಥವಾ ನಿಷೇಧ ಭಾವನೆಯ ಮೇಲೆ ನಿಂತ ಯಾವ ತರಬೇತಿಯಾಗಲೀ ಮೃತ್ಯುವಿಗಿಂತ ಹಾನಿಕರ. ಮಗುವನ್ನು ಶಾಲೆಗೆ ಕಳುಹಿಸುವರು. ಮೊದಲು ಅದು ಕಲಿಯುವುದೇ ತಂದೆ ಮೂರ್ಖ ಎಂಬುದನ್ನು, ಎರಡನೆಯದು ಅದರ ತಾತ ಹುಚ್ಚನೆನ್ನುವುದು, ಮೂರನೆಯದೇ ಗುರುಗಳೆಲ್ಲ ಆಷಾಢಭೂತಿಗಳು, ನಾಲ್ಕನೆಯದೇ ಪವಿತ್ರ ಶಾಸ್ತ್ರಗಳೆಲ್ಲ ಸುಳ್ಳಿನ ಕಂತೆ ಎನ್ನುವುದು. ಅವನಿಗೆ ಹದಿನಾರು ವರುಷ ತುಂಬುವ ಸಮಯಕ್ಕೆ ಅವನೊಂದು ನಿಷೇಧಮಯದ ಕಂತೆಯಾಗುವನು. ಅವನು ನಿರ್ಜೀವನಾಗಿ, ನಿತ್ರಾಣನಾಗುವನು. ಇದರ ಪರಿಣಾಮವಾಗಿ ಐವತ್ತು ವರುಷಗಳ ಇಂತಹ ವಿದ್ಯಾಭ್ಯಾಸದಿಂದ ಮೂರು ಪ್ರಾಂತಗಳಲ್ಲಿ ಒಬ್ಬ ಸ್ವತಂತ್ರವಾಗಿ ವಿಚಾರ ಮಾಡುವ ವ್ಯಕ್ತಿಯೂ ಇದರ ಮೂಲಕ ಬಂದಿಲ್ಲ. ಸ್ವತಂತ್ರವಾಗಿ ಆಲೋಚನೆ ಮಾಡುವ ವ್ಯಕ್ತಿಗಳೆಲ್ಲ ವಿದ್ಯಾಭ್ಯಾಸವನ್ನು ಇಲ್ಲಿ ಪಡೆಯಲಿಲ್ಲ. ಹೊರಗೆ ಪಡೆದರು ಅಥವಾ ಪುನಃ ಹಿಂದಿನ ಕಾಲದ ವಿದ್ಯಾಕೇಂದ್ರಗಳಿಗೆ ತಮ್ಮ ಮೂಢನಂಬಿಕೆಯಿಂದ ಪಾರಾಗಲು ಹೋದರು. ವಿದ್ಯಾಭ್ಯಾಸವೆಂದರೆ ನಿಮ್ಮ ತಲೆಗೆ ತುಂಬಿದ ವಿಷಯಗಳ ಮೊತ್ತವಲ್ಲ. ಅದು ಅಲ್ಲಿ ಜೀರ್ಣವಾಗದೆ ಚೆಲ್ಲಾಪಿಲ್ಲಿಯಾಗಿ ಬಿದ್ದಿರುವುದು. ಜೀವನವನ್ನು ನಿರ್ಮಾಣ ಮಾಡುವಂತಹ, ಪುರುಷಸಿಂಹರನ್ನು ಮಾಡುವಂತಹ, ಶೀಲ ಸಂಪತ್ತಿಗೆ ಸಹಾಯ ಮಾಡುವಂತಹ, ವಿಚಾರಗಳನ್ನು ಗ್ರಹಿಸುವಂತಹ ವಿದ್ಯಾಭ್ಯಾಸ ನಮಗೆ ಬೇಕು. ನೀವು ಐದು ವಿಷಯಗಳನ್ನು ತಿಳಿದು, ಅವನ್ನು ನಿಮ್ಮ ಜೀವನದಲ್ಲಿ, ಶೀಲದಲ್ಲಿ ಅನುಷ್ಠಾನಕ್ಕೆ ತಂದರೆ ಇಡಿಯ ಪುಸ್ತಕಭಂಡಾರವನ್ನೇ ಕಂಠಪಾಠ ಮಾಡಿದವನಿಗಿಂತ ಹೆಚ್ಚು ವಿದ್ಯೆ ನಿಮಗೆ ಇದೆ. \textbf{“ಯಥಾ ಖರಶ್ಚಂದನಭಾರವಾಹೀ ಭಾರಸ್ಯ ವೇತ್ತಾ ನ ತು ಚಂದನಸ್ಯ”} -ಗಂಧದ ಮರವನ್ನು ಹೊರುವ ಕತ್ತೆಗೆ ಅದರ ಭಾರ ಮಾತ್ರ ಗೊತ್ತು, ಅದರ ಗುಣ ಗೊತ್ತಿಲ್ಲ. ವಿದ್ಯಾಭ್ಯಾಸ ಕೇವಲ ವಿಷಯ ಸಂಗ್ರಹವಾದರೆ ಪುಸ್ತಕಾಲಯಗಳೇ ದೊಡ್ಡ ಮುನಿಗಳು, ವಿಶ್ವಕೋಶಗಳೇ ದೊಡ್ಡ ಋಷಿಗಳಾಗುತ್ತಿದ್ದವು. ನಮ್ಮ ಆದರ್ಶವೇನೆಂದರೆ ಪಾರಮಾರ್ಥಿಕ ಮತ್ತು ಲೌಕಿಕ ಶಿಕ್ಷಣಗಳೆರಡೂ ನಮ್ಮ ಅಧೀನದಲ್ಲಿರಬೇಕು; ಸಾಧ್ಯವಾದ ಮಟ್ಟಿಗೆ ಅವು ರಾಷ್ಟ್ರೀಯವಾಗಿರಬೇಕು, ರಾಷ್ಟ್ರೀಯ ಪದ್ಧತಿಯಲ್ಲಿರಬೇಕು.

ಇದೊಂದು ವಿರಾಟ್​ ಯೋಜನೆಯೇನೋ ಸರಿ. ಅದು ಎಂದಿಗಾದರೂ ಸಾಧ್ಯವಾಗುವುದೇನೋ ನನಗೆ ಗೊತ್ತಿಲ್ಲ. ಆದರೆ ನಾವು ಕೆಲಸವನ್ನು ಪ್ರಾರಂಭಿಸಬೇಕು. ಅದನ್ನು ಪ್ರಾರಂಭಿಸುವುದು ಹೇಗೆ? ಉದಾಹರಣೆಗೆ ಮದ್ರಾಸನ್ನೇ ತೆಗೆದುಕೊಳ್ಳಿ. ನಮಗೆ ಧರ್ಮವೇ ಪ್ರಧಾನವಾಗಿರುವುದರಿಂದ ಒಂದು ದೇವಸ್ಥಾನವಿರಬೇಕು. ಆಗ ವಿವಿಧ ಪಂಗಡಗಳೆಲ್ಲ ಕಾದಾಡುವುವು ಎಂದು ಹೇಳಬಹುದು. ಆದರೆ ಅದನ್ನು ಯಾವ ಪಂಗಡಕ್ಕೂ ಸೇರದ ಗುಡಿಯನ್ನಾಗಿ ಮಾಡುತ್ತೇವೆ. ಅಲ್ಲಿ “ಓಂ” ಎಂಬುದೇ ಮುಖ್ಯ ಚಿಹ್ನೆ. ಅದು ಎಲ್ಲ ಪಂಗಡಗಳ ಅತ್ಯಂತ ಮುಖ್ಯವಾದ ಚಿಹ್ನೆಯಾಗಿದೆ. ಯಾವುದಾದರೂ ಪಂಗಡದವರು “ಓಂ” ಎನ್ನುವುದು ತಮ್ಮ ಚಿಹ್ನೆಯಾಗುವುದಿಲ್ಲ ಎಂದು ಭಾವಿಸಿದರೆ ತಾವು ಹಿಂದೂಗಳೆಂದು ಹೇಳಿಕೊಳ್ಳುವುದಕ್ಕೆ ಅವರಿಗೆ ಅಧಿಕಾರವಿಲ್ಲ. ಪ್ರತಿಯೊಂದು ಪಂಗಡದವರೂ ತಮ್ಮ ತಮ್ಮ ಸಂಪ್ರದಾಯಕ್ಕೆ ತಕ್ಕಂತೆ ಹಿಂದೂಧರ್ಮವನ್ನು ವಿವರಿಸಬಹುದು. ಆದರೆ ನಮಗೆಲ್ಲ ಒಂದು ಸಾಮಾನ್ಯ ದೇವಸ್ಥಾನವಿರಬೇಕು. ಬೇರೆ ಕಡೆ ನಿಮಗೆ ಇಷ್ಟವಾದ ವಿಗ್ರಹಗಳನ್ನು, ಚಿಹ್ನೆಗಳನ್ನು ನೀವು ಇಟ್ಟುಕೊಂಡಿರಬಹುದು. ಆದರೆ ಇಲ್ಲಿ ಯಾರು ನಿಮ್ಮೊಂದಿಗೆ ಒಪ್ಪಿಕೊಳ್ಳುವುದಿಲ್ಲವೋ ಅವರೊಡನೆ ವ್ಯಾಜ್ಯ ಮಾಡಬೇಡಿ. ಇಲ್ಲಿ ಬೇರೆ ಬೇರೆ ಪಂಗಡಕ್ಕೆ ಸೇರಿದ ಸಾಮಾನ್ಯ ಸಿದ್ಧಾಂತವನ್ನು ಇಲ್ಲಿ ಪ್ರಚಾರ ಮಾಡುವುದಕ್ಕೆ ಅವಕಾಶವಿರಬೇಕು. ನಿಮಗೆ ಏನು ತೋರುವುದೋ ಅದನ್ನು ಹೇಳಿ. ಅದು ಜಗತ್ತಿಗೆ ಬೇಕಾಗಿದೆ. ನೀವು ಇತರರನ್ನು ಹೇಗೆ ನೋಡುವಿರಿ ಎಂಬುದನ್ನು ಕೇಳುವುದಕ್ಕೆ ಜಗತ್ತಿಗೆ ಸಮಯವಿಲ್ಲ. ಅದನ್ನು ನಿಮ್ಮೊಳಗೆಯೇ ಇಟ್ಟುಕೊಳ್ಳಿ.

ಎರಡನೆಯದಾಗಿ, ಈ ದೇವಸ್ಥಾನದ ಅಂಗವಾಗಿ ಧರ್ಮ ಪ್ರಚಾರವನ್ನು ತರಬೇತು ಮಾಡುವ ಒಂದು ಸಂಸ್ಥೆಯಿರಬೇಕು. ಅವರು ಲೌಕಿಕ ಮತ್ತು ಪಾರಮಾರ್ಥಿಕ ವಿಷಯಗಳನ್ನು ಪ್ರಚಾರಮಾಡಲು ಹೋಗಬೇಕು. ನಾವು ಆಗಲೇ ಧರ್ಮವನ್ನು ಮನೆಯಿಂದ ಮನೆಗೆ ಕೊಡುತ್ತಿರುವೆವು. ಜೊತೆಗೆ ಲೌಕಿಕ ಜ್ಞಾನವನ್ನೂ ಕೊಡೋಣ. ಅದನ್ನು ಸುಲಭವಾಗಿ ಮಾಡಬಹುದು. ಕ್ರಮೇಣ ಬೋಧಕರ ಮತ್ತು ಪ್ರಚಾರಕರ ಮೂಲಕ ಕೆಲಸ ಹರಡುವುದು. ಕ್ರಮೇಣ ಇಂಡಿಯಾ ದೇಶದಲ್ಲಿ ಮಾತ್ರವಲ್ಲ, ಇತರ ಕಡೆಗಳಲ್ಲಿಯೂ ಇಂತಹ ದೇವಸ್ಥಾನಗಳನ್ನು ಸ್ಥಾಪಿಸೋಣ. ಇದು ನನ್ನ ಯೋಜನೆ. ಇದು ಬೃಹತ್ತಾಗಿ ಕಾಣಬಹುದು. ಆದರೆ ಇದು ಬಹಳ ಆವಶ್ಯಕ. ಹಣವೆಲ್ಲಿದೆ ಎಂದು ನೀವು ಕೇಳಬಹುದು. ಹಣ ಬೇಕಾಗಿಲ್ಲ. ಹಣವಲ್ಲ ಮುಖ್ಯ. ಕಳೆದ ಹನ್ನೆರಡು ವರುಷಗಳಿಂದ ನನಗೆ ಮುಂದಿನ ಊಟ ಎಲ್ಲಿ ದೊರಕುವುದು ಎಂಬುದು ಗೊತ್ತಿರಲಿಲ್ಲ. ನನಗೆ ಬೇಕಾದ ಹಣ ಮತ್ತು ಎಲ್ಲವೂ ಸಿಕ್ಕಲೇಬೇಕು. ಅವು ನನ್ನ ಗುಲಾಮರು. ನಾನು ಅವುಗಳ ಗುಲಾಮನಲ್ಲ. ಹಣ ಬರಲೇಬೇಕು. ಆದರೆ ವ್ಯಕ್ತಿಗಳೆಲ್ಲಿ? ಅದೇ ಪ್ರಶ್ನೆ. ಮದ್ರಾಸಿನ ಯುವಕರೇ, ನನ್ನ ಭರವಸೆಯೆಲ್ಲಾ ನಿಮ್ಮ ಮೇಲಿದೆ. ನಿಮ್ಮ ದೇಶದ ಕರೆಗೆ ಕಿವಿಗೊಡುವಿರೇನು? ನನ್ನ ಮಾತಿನಲ್ಲಿ ನಿಮಗೆ ನಂಬಿಕೆ ಇರುವುದಾದರೆ ನಿಮ್ಮಲ್ಲಿ ಪ್ರತಿಯೊಬ್ಬರಿಗೂ ಭವ್ಯವಾದ ಭವಿಷ್ಯವಿದೆ. ನಿಮ್ಮಲ್ಲಿ ನಿಮಗೆ ಅಪಾರವಾದ ನಂಬಿಕೆ ಬೆಳೆಯಲಿ. ಮಗುವಾಗಿದ್ದಾಗ ನನ್ನಲ್ಲಿ ಶ್ರದ್ಧೆ ಇತ್ತು. ಅದನ್ನೇ ನಾನು ಈಗ ಅನುಸರಿಸುವೆನು. ಅನಂತಶಕ್ತಿ ಪ್ರತಿ ಜೀವಿಯ ಅಂತರಾಳದಲ್ಲಿ ಇರುವುದು ಎಂಬ ಆತ್ಮಶ್ರದ್ಧೆ ನಿಮ್ಮಲ್ಲಿರಲಿ. ನೀವು ಇಡೀ ಭರತಖಂಡವನ್ನು ಜಾಗ್ರತಗೊಳಿಸುವಿರಿ. ಆಗ ನಾವು ಜಗತ್ತಿನ ಇತರ ದೇಶಗಳಿಗೆಲ್ಲ ಹೋಗುವೆವು. ಜಗತ್ತಿನಲ್ಲಿ ರಾಷ್ಟ್ರಗಳನ್ನು ನಿರ್ಮಾಣ ಮಾಡುತ್ತಿರುವ ಶಕ್ತಿಯೊಂದಿಗೆ ನಮ್ಮ ಭಾವನೆಯೂ ಮಿಲನವಾಗುವುದು. ಭರತಖಂಡದಲ್ಲಿ ಮತ್ತು ವಿದೇಶಗಳಲ್ಲಿ ಎಲ್ಲಾ ಜನಾಂಗಗಳ ಜೀವನದ ಅಂತರಾಳಕ್ಕೆ ನಾವು ಹೋಗಬೇಕು. ಇದನ್ನು ಸಾಧಿಸುವುದಕ್ಕೆ ಕೆಲಸಮಾಡಬೇಕು. ಆ ಕಾರ್ಯಕ್ಕೆ ನನಗೆ ಯುವಕರು ಬೇಕು. “ಆಶಿಷ್ಠರೂ, ಬಲಿಷ್ಠರೂ, ದ್ರಢಿಷ್ಠರೂ, ಮೇಧಾವಿಗಳು ಆದಂತಹ ಯುವಕರು ಮಾತ್ರ ಭಗವಂತನನ್ನು ಪಡೆಯಬಹುದು” ಎಂದು ವೇದ ಸಾರುವುದು. ತಾರುಣ್ಯದ ಶಕ್ತಿ ಇರುವಾಗ ನಿಮ್ಮ ಭವಿಷ್ಯವನ್ನು ನಿರ್ಧರಿಸಬೇಕಾಗಿದೆ. ಮುದುಕರಾಗಿ ಶಕ್ತಿ ಕುಗ್ಗಿದ ಮೇಲೆ ಅಲ್ಲ. ನವತಾರುಣ್ಯದ ಉತ್ಸಾಹದಲ್ಲಿ ಅದನ್ನು ನಿರ್ಧರಿಸಬೇಕಾಗಿದೆ. ಕೆಲಸಮಾಡಿ, ಈಗ ತಾನೇ ವಿಕಸಿತವಾದ, ಯಾರೂ ಮುಟ್ಟದ, ಮೂಸಿ ನೋಡದ, ಹೂವುಗಳನ್ನು ಭಗವಂತನ ಅಡಿದಾವರೆಯಲ್ಲಿಡುವುದಕ್ಕೆ ಇದೇ ಸಮಯ. ದೇವರು ಇದನ್ನು ಮಾತ್ರ ಪರಿಗ್ರಹಿಸುವನು. ಬಾಳು ಕ್ಷಣಿಕ, ಜಾಗ್ರತರಾಗಿ. ವಕೀಲರಾಗುವುದಕ್ಕಿಂತ, ಜಗಳ ಕಾಯುವುದಕ್ಕಿಂತ, ಹೆಚ್ಚಾಗಿ ಮಹತ್ಕಾರ್ಯಗಳನ್ನು ಸಾಧಿಸಬೇಕಾಗಿದೆ. ನಿಮ್ಮ ಜನಾಂಗದ ಹಿತಕ್ಕೆ, ಮಾನವಕೋಟಿಯ ಹಿತಕ್ಕೆ, ತ್ಯಾಗಮಾಡುವುದು ಮಹಾಕಾರ್ಯ. ಈ ಪ್ರಪಂಚದಲ್ಲಿ ಏನಿದೆ? ನೀವು ಹಿಂದೂಗಳು, ಜೀವನ ಅನಂತವೆಂಬ ಭಾವನೆ ನಿಮಗೆ ಸ್ವಾಭಾವಿಕವಾಗಿರುವುದು. ಕೆಲವು ಯುವಕರು ನಾಸ್ತಿಕ ವಿಚಾರವನ್ನು ನನ್ನ ಹತ್ತಿರ ಮಾತನಾಡುವರು. ಹಿಂದೂ ನಾಸ್ತಿಕನಾಗಬಲ್ಲ ಎಂಬುದನ್ನು ನಾನು ನಂಬಲಾರೆ. ಯೂರೋಪಿನ ಗ್ರಂಥಗಳನ್ನು ಅವನು ಓದಿ ನಾನು ನಾಸ್ತಿಕ ಎಂದು ಕೆಲವು ಕಾಲ ಭ್ರಮಿಸಿರಬಹುದು. ಆದರೆ ಇದು ಕೇವಲ ತಾತ್ಕಾಲಿಕ. ಇದನ್ನು ನೀವು ನಂಬಲಾರಿರಿ. ಅದು ಸಾಧ್ಯವಿಲ್ಲದ ಮಾತು. ಇಂತಹ ಅಸಾಧ್ಯವನ್ನು ಪ್ರಯತ್ನಿಸಬೇಡಿ. ನಾನು ಹುಡುಗನಾಗಿದ್ದಾಗ ಒಮ್ಮೆ ಪ್ರಯತ್ನಪಟ್ಟೆ. ಆದರೆ ಸಾಧ್ಯವಾಗಲಿಲ್ಲ. ಬಾಳು ಕ್ಷಣಿಕ, ಆತ್ಮ ಅಮರ, ಅನಂತ. ಮೃತ್ಯು ಎನ್ನುವುದೊಂದು ಸತ್ಯವಾಗಿರುವಾಗ ಒಂದು ಮಹಾ ಆದರ್ಶವನ್ನು ಸ್ವೀಕರಿಸಿ, ಅದಕ್ಕೆ ನಮ್ಮ ಇಡೀ ಬಾಳನ್ನು ಧಾರೆ ಎರೆಯೋಣ. ಇದು ನಮ್ಮ ಶಪಥವಾಗಲಿ. ತನ್ನ ಮಕ್ಕಳ ಮುಕ್ತಿಗೆ ಪುನಃ ಪುನಃ ಬರುತ್ತೇನೆ ಎಂದು ಸಾರಿದ ಶ‍್ರೀಕೃಷ್ಣ ಭಗವಾನ್​ ನಮ್ಮನ್ನು ಆಶೀರ್ವದಿಸಿ ನಮ್ಮ ಉದ್ದೇಶ ಸಿದ್ಧಿಗೆ ಸಹಾಯಮಾಡಲಿ.

