
\chapter{ರಾಮನಾಡಿನ ಬಿನ್ನವತ್ತಳೆಗೆ ಉತ್ತರ}

ರಾಮನಾಡಿನಲ್ಲಿ ಅಲ್ಲಿನ ರಾಜರು ಸ್ವಾಮಿ ವಿವೇಕಾನಂದರಿಗೆ ಈ ಮುಂದಿನ ಬಿನ್ನವತ್ತಳೆ\break ಯನ್ನು ಸಮರ್ಪಿಸಿದರು:

\vskip 2pt

\textbf{ಪರಮಪೂಜ್ಯ ಸ್ವಾಮೀಜಿಯವರೆ,}

ಶ‍್ರೀಪರಮಹಂಸ, ಯತಿರಾಜ, ದಿಗ್ವಿಜಯಕೋಲಾಹಲ, ಸರ್ವಮತ ಸಂಪ್ರತಿಪನ್ನ, ಪರಮಯೋಗೀಶ್ವರ, ಶ‍್ರೀಮತ್​ ಭಗವಾನ್​ ಶ‍್ರೀರಾಮಕೃಷ್ಣ ಪರಮಹಂಸ ಕರಕಮಲ–ಸಂಜಾತ, ರಾಜಾಧಿರಾಜ ಸೇವಿತ, ಶ‍್ರೀ ವಿವೇಕಾನಂದ ಸ್ವಾಮಿಯವರೇ –

\vskip 2pt

ಅತ್ಯಂತ ಪ್ರಾಚೀನವೂ ಐತಿಹಾಸಿಕವೂ ಆಗಿದ್ದು ರಾಮನಾಥಪುರ ಅಥವಾ ರಾಮನಾಡೆಂದು ಪ್ರಸಿದ್ಧವಾಗಿರುವ ಸೇತುಬಂಧ ರಾಮೇಶ್ವರಂ ಸಂಸ್ಥಾನದ ಪ್ರಜೆಗಳಾದ ನಾವು ನಮ್ಮ ಈ ತಾಯ್ನಾಡಿಗೆ ತಮ್ಮನ್ನು ಅತ್ಯಂತ ಹಾರ್ದಿಕವಾಗಿ ಸ್ವಾಗತಿಸಲು ತಮ್ಮ ಅಪ್ಪಣೆಯನ್ನು ಬೇಡುತ್ತೇವೆ. ಮಹಾಸ್ವಾಮಿಯವರು ಭಾರತದ ಭೂಮಿಯ ಮೇಲೆ, ಅದರಲ್ಲೂ ಮಹಾವೀರನಾದ ಶ‍್ರೀಭಗವಾನ್​ ರಾಮಚಂದ್ರನು ತನ್ನ ಪಾದಸ್ಪರ್ಶದಿಂದ ಪವಿತ್ರಗೊಳಿಸಿದ ಸಮುದ್ರತೀರದ ಮೇಲೆ ತಾವು ಪಾದಾರ್ಪಣ ಮಾಡಿದಾಗ ತಮಗೆ ನಮ್ಮ ಹೃತ್ಪೂರ್ವಕ ಗೌರವವನ್ನು ಮೊದಲು ಸಲ್ಲಿಸುವ ಅಪೂರ್ವ ಅವಕಾಶವು ನಮಗೆ ದೊರೆತುದಕ್ಕಾಗಿ ನಾವು ಧನ್ಯರಾಗಿದ್ದೇವೆ.

\vskip 2pt

ಪುರಾತನವೂ ಭವ್ಯವೂ ಆದ ನಮ್ಮ ಧರ್ಮದ ಅಂತರ್ನಿಹಿತ ಮೌಲ್ಯಗಳನ್ನೂ ಶ್ರೇಷ್ಠತೆಯನ್ನೂ ಪಶ್ಚಿಮದ ಮಹಾಮತಿಗಳಿಗೆ ತಿಳಿಸಲು ತಾವು ಮಾಡಿದ ಪ್ರಶಂಸನೀಯವಾದ ಪ್ರಯತ್ನಗಳು ಅಭೂತಪೂರ್ವವಾದ ಯಶಸ್ಸನ್ನು ಪಡೆದ ಪ್ರಕ್ರಿಯೆಯನ್ನು ನಾವು ನಿಜವಾದ ಹೆಮ್ಮೆಯಿಂದಲೂ ಸಂತೋಷದಿಂದಲೂ ಗಮನಿಸುತ್ತಾ ಬಂದಿದ್ದೇವೆ.

\vskip 2pt


ಯೂರೋಪ್​ ಮತ್ತು ಅಮೆರಿಕಾಗಳಲ್ಲಿ ಸುಸಂಸ್ಕೃತ ಶ್ರೋತೃಗಳಿಗೆ ಹಿಂದೂಧರ್ಮವು ವಿಶ್ವಧರ್ಮದ ಎಲ್ಲ ಲಕ್ಷಣಗಳನ್ನು ಹೊಂದಿದೆ ಎಂಬುದನ್ನೂ ಎಲ್ಲ ಜನಾಂಗಗಳ, ಪಂಥಗಳ, ಸ್ತ್ರೀಪುರುಷರ ಪ್ರವೃತ್ತಿಗಳಿಗೂ ಹೊಂದಿಕೊಂಡು ಅವರ ಅಗತ್ಯಗಳನ್ನು ಪೂರೈಸಬಲ್ಲದು ಎಂಬುದನ್ನೂ ತಾವು ತಮ್ಮ ಅಪೂರ್ವ ವಾಗ್ಮಿತೆಯಿಂದ, ಸರಳವೂ ಸ್ಪಷ್ಟವೂ ಆದ ಭಾಷೆಯಲ್ಲಿ ತಿಳಿಸಿದ್ದೀರಿ. ನಿಃಸ್ವಾರ್ಥ ಮನೋಭಾವದಿಂದಲೂ ಅತ್ಯುತ್ಕೃಷ್ಟವಾದ ಉದ್ದೇಶಗಳಿಂದಲೂ ಪ್ರೇರಿತರಾಗಿ, ಅತ್ಯಂತ ಪರಿಶ್ರಮದಿಂದ ತಾವು ಸತ್ಯಶಾಂತಿಗಳ ಸಂದೇಶವನ್ನು ಸಾರಲು ಮತ್ತು ಯೂರೋಪು ಅಮೆರಿಕಗಳ ಫಲವತ್ತಾದ ಭೂಮಿಯಲ್ಲಿ ಭಾರತದ ಆಧ್ಯಾತ್ಮಿಕ ವಿಜಯ ವೈಭವಗಳ ಧ್ವಜವನ್ನು ನೆಡುವುದಕ್ಕೆಂದು ಅಸೀಮವಾದ ಸಾಗರಗಳನ್ನು ದಾಟಿಹೋಗಿದ್ದೀರಿ. ಮಹಾ ಯತಿವರ್ಯರಾದ ತಾವು ತಮ್ಮ ಬೋಧನೆ ಅನುಷ್ಠಾನಗಳಿಂದ ವಿಶ್ವಭ್ರಾತೃತ್ವದ ಸಾಧ್ಯತೆಯನ್ನೂ ಪ್ರಾಮುಖ್ಯವನ್ನೂ ತೋರಿಸಿಕೊಟ್ಟಿದ್ದೀರಿ. ಎಲ್ಲಕ್ಕಿಂತ ಮಿಗಿಲಾಗಿ, ತಾವು ಪಶ್ಚಿಮದಲ್ಲಿ ಪಟ್ಟ ಪರಿಶ್ರಮವು ಭಾರತದ ನಿರುತ್ಸಾಹಿ ಪ್ರಜೆಗಳಲ್ಲಿ ಪರೋಕ್ಷವಾಗಿ ತಮ್ಮ ಪ್ರಾಚೀನಧರ್ಮದ ಮಹತ್ತು ವೈಭವಗಳ ವಿಷಯದಲ್ಲಿ ಜಾಗೃತಿಯನ್ನು ಉಂಟುಮಾಡಿ, ತಮ್ಮ ಅನರ್ಘ್ಯವಾದ ಧರ್ಮವನ್ನು ಅಧ್ಯಯನಮಾಡಿ ಅದನ್ನು ಆಚರಿಸುವ ವಿಷಯದಲ್ಲಿ ನಿಜವಾದ ಆಸಕ್ತಿಯನ್ನು ಸೃಷ್ಟಿಸಿದೆ.

\vskip 2pt

ಪೂರ್ವ ಪಶ್ಚಿಮ ದೇಶಗಳಲ್ಲಿ ಆಧ್ಯಾತ್ಮಿಕ ಪುನರುತ್ಥಾನದ ವಿಷಯದಲ್ಲಿ ತಾವು ಪಟ್ಟಿರುವ ಶ್ರಮಕ್ಕಾಗಿ ತಮಗೆ ಕೃತಜ್ಞತೆಗಳನ್ನು ಮಾತುಗಳಲ್ಲಿ ಸಲ್ಲಿಸಲು ಅಸಮರ್ಥರಾಗಿದ್ದೇವೆ. ತಮ್ಮ ನಿಷ್ಠಾವಂತ ಶಿಷ್ಯರಲ್ಲಿ ಒಬ್ಬರಾದ ನಮ್ಮ ರಾಜರ ವಿಷಯದಲ್ಲಿ ತಾವು ಸದಾ ತೋರುತ್ತಾ ಬಂದಿರುವ ದಯೆಯನ್ನೂ, ತಾವು ಮೊಟ್ಟಮೊದಲು ಅವರ ಪ್ರಾಂತ್ಯದಲ್ಲಿ ಪಾದಾರ್ಪಣ ಮಾಡಿದ್ದಕ್ಕಾಗಿ ಅವರು ತಾಳಿರುವ ಹೆಮ್ಮೆ ಮತ್ತು ಗೌರವಗಳನ್ನೂ ಉಲ್ಲೇಖಿಸದೆ ಈ ಬಿನ್ನವತ್ತಳೆಯನ್ನು ಮುಗಿಸುವುದು ಅಸಾಧ್ಯ.

\vskip 2pt

ಕೊನೆಯದಾಗಿ ಜಗದೀಶ್ವರನು ಪೂಜ್ಯಪಾದರಿಗೆ ದೀರ್ಘಾಯುಷ್ಯವನ್ನೂ ಆರೋಗ್ಯವನ್ನೂ, ತಾವು ಕೈಗೊಂಡಿರುವ ಒಳ್ಳೆಯ ಕೆಲಸವನ್ನು ನಡೆಸಲು ಅಗತ್ಯವಾದ ಶಕ್ತಿಯನ್ನೂ ನೀಡಲೆಂದು ಪ್ರಾರ್ಥಿಸುತ್ತೇವೆ.

\medskip
\medskip

\begin{longtable}[r]{@{}l@{}}
ಗೌರವ ಮತ್ತು ಪ್ರೇಮಪೂರ್ವಕವಾಗಿ \\
ತಮ್ಮ ಅತ್ಯಂತ ನಿಷ್ಠಾವಂತರೂ ವಿಧೇಯರೂ ಆದ \\
ಶಿಷ್ಯರು ಮತ್ತು ಸೇವಕರು 
\end{longtable}

\medskip


ಸ್ವಾಮೀಜಿಯವರು ಈ ಕೆಳಗಿನಂತೆ ಉತ್ತರಿಸಿದರು:


\vskip 2pt

ಸುದೀರ್ಘ ರಾತ್ರಿಯು ಕಡೆಗೆ ಇಂದು ಕೊನೆಗಾಣುವಂತೆ ತೋರುತ್ತಿದೆ. ಬಹು ಕಾಲದ ಕಷ್ಟಸಂಕಟಗಳು ಕೊನೆಗೆ ಇಂದು ಸಮಾಪ್ತಿಗೊಳ್ಳುವಂತಿದೆ. ಇದುವರೆಗೆ ಶವದಂತೆ ಇದ್ದ ಶರೀರವು ಇಂದು ಸಚೇತವಾಗುತ್ತಿದೆ. ವಾಣಿಯೊಂದು ನಮಗೆ ಕೇಳಿ ಬರುತ್ತಿದೆ. ಇತಿಹಾಸ, ಸಂಪ್ರದಾಯಗಳು ಇಣುಕಿ ಕೂಡ ನೋಡಲು ಸಾಧ್ಯವಿಲ್ಲದಷ್ಟು ಬಹುಪುರಾತನವಾದ ಕಾಲಗರ್ಭದಲ್ಲಿ ಉಗಮಿಸಿ, ಜ್ಞಾನ, ಕರ್ಮ, ಭಕ್ತಿಗಳ ಅನಂತ ಹಿಮಾಚಲ ಗಿರಿಶಿಖರಗಳಲ್ಲಿ ಪ್ರತಿಧ್ವನಿಗೊಂಡು ಆ ವಾಣಿಯು ಹೊಮ್ಮಿ ಬರುತ್ತಿದೆ. ಮೃದುವಾದರೂ ಸದೃಢವಾದ, ತನ್ನ ಸಂದೇಶವನ್ನು ಅಸಂದಿಗ್ಧವಾಗಿ ಸಾರುವ ಆ ವಾಣಿಯು ದಿನಗಳು ಉರುಳಿದಂತೆ ಹೆಚ್ಚು ಹೆಚ್ಚು ತೀವ್ರವಾಗುತ್ತಾ ಬರುತ್ತಿದೆ. ಅದೋ ನೋಡಿ! ನಿದ್ರಿಸುತ್ತಿರುವುದು ಜಾಗೃತವಾಗುತ್ತಿದೆ. ಹಿಮಾಲಯದ ತಂಗಾಳಿಯಂತೆ ಅದು ಮೃತಪ್ರಾಯವಾದ ಮೂಳೆ ಮಾಂಸಗಳಿಗೆ ಪ್ರಾಣದಾನ ಮಾಡುತ್ತಿದೆ. ಜಡತೆಯು ಬೀಳ್ಕೊಳ್ಳುತ್ತಿದೆ. ನಮ್ಮೀ ಮಾತೃಭೂಮಿಯು ತನ್ನದೀರ್ಘ ನಿದ್ರೆಯನ್ನು ತ್ಯಜಿಸಿ ಜಾಗೃತವಾಗುತ್ತಿರುವುದನ್ನು ಕುರುಡರು ಮಾತ್ರ ಕಾಣ\-ಲಾರರು, ಮೂರ್ಖರು ಮಾತ್ರ ಕಾಣಲೊಲ್ಲರು. ಅವಳನ್ನು ಇನ್ನು ಯಾರೂ ತಡೆಯ\-ಲಾರರು. ಆಕೆ ಇನ್ನು ಎಂದೆಂದೂ ನಿದ್ದೆ ಹೋಗುವುದಿಲ್ಲ. ಯಾವ ಬಾಹ್ಯಶಕ್ತಿಯೂ ಇನ್ನು ಅವಳನ್ನು ಹಿಮ್ಮೆಟ್ಟಿಸಲಾರದು. ಏಕೆಂದರೆ ಆ ಮಹಾಕಾಳಿಯು ಪ್ರತಿಷ್ಠಾಪನೆ ಗೊಳ್ಳುತ್ತಿದ್ದಾಳೆ.

\eject

ಮಹಾರಾಜರೇ, ರಾಮನಾಡಿನ ಸಜ್ಜನರೇ! ನೀವು ನೀಡಿದ ಹಾರ್ದಿಕ ಸ್ವಾಗತಕ್ಕೂ ತೋರಿಸಿರುವ ಪ್ರೀತಿಗೂ ನನ್ನ ಹೃತ್ಪೂರ್ವಕ ಕೃತಜ್ಞತೆಗಳನ್ನು ಸಲ್ಲಿಸಬಯಸುತ್ತೇನೆ. ಏಕೆಂದರೆ ಎಲ್ಲಾ ಭಾಷೆಗಳಿಗಿಂತ ಹೃದಯವು ಹೃದಯದೊಂದಿಗೆ ಹೆಚ್ಚು ಆತ್ಮೀಯವಾಗಿ ಮಾತನಾಡುತ್ತದೆ. ಆತ್ಮವು ಮತ್ತೊಂದು ಆತ್ಮದೊಂದಿಗೆ ಸ್ಪಷ್ಟವಾದ ಭಾಷೆಯಲ್ಲಿ ಮೌನವಾಗಿ ಮಾತನಾಡುವುದು. ಅದು ನನ್ನ ಹೃದಯಕ್ಕೆ ಅರಿವಾಗುತ್ತಿದೆ. ರಾಮನಾಡಿನ ಮಹಾರಾಜರೇ! ನಮ್ಮ ಧರ್ಮದ ಮತ್ತು ಮಾತೃಭೂಮಿಯ ಪರವಾಗಿ ಏನಾದರೂ ಈ ದೀನ ವ್ಯಕ್ತಿಯು ಸ್ವಲ್ಪ ಕೆಲಸವನ್ನು ವಿದೇಶಗಳಲ್ಲಿ ಮಾಡಿದ್ದರೆ, ನಮ್ಮ ದೇಶದಲ್ಲಿ ಅಜ್ಞಾತವಾಗಿ, ಗುಪ್ತವಾಗಿ ಬಿದ್ದಿರುವ ರತ್ನರಾಶಿಯ ಕಡೆಗೆ ಜನರ ಗಮನವನ್ನು ಆಕರ್ಷಿಸಿ ಸ್ವದೇಶದ ಮೇಲೆ ವಿಶ್ವಾಸ ಹುಟ್ಟುವಂತೆ ನಾನು ಪ್ರಯತ್ನಿಸಿದ್ದರೆ, ಅಜ್ಞಾನಾಂಧಕಾರದಲ್ಲಿ ತೊಳಲುತ್ತಾ, ಬಾಯಾರಿಕೆಯಿಂದ ನರಳಿ ಸಾಯುತ್ತಾ, ಚರಂಡಿಯ ನೀರನ್ನು ಕುಡಿಯುವ ಬದಲು ತಮ್ಮ ಮನೆಯ ಬಳಿಯಲ್ಲೇ ಹರಿಯುತ್ತಿರುವ ಅಮೃತ ಪ್ರವಾಹವನ್ನು ಪಾನಮಾಡಿ ಎಂದು ಹೇಳಿದ್ದರೆ, ನಮ್ಮ ದೇಶಬಾಂಧವರನ್ನು ಕಾರ್ಯೋನ್ಮುಖರನ್ನಾಗಿ ಮಾಡಿದ್ದರೆ, ಭಾರತದ ಪ್ರಾಣವೇ ಧರ್ಮ; ಅದು ಹೋದರೆ ರಾಜನೀತಿ, ಸಾಮಾಜಿಕ ಪ್ರಗತಿ ಕುಬೇರನ ಐಶ್ವರ್ಯ ಇವುಗಳೆಲ್ಲ ಇದ್ದರೂ ಭಾರತ ನಾಶವಾಗುವುದು ಎಂಬ ಭಾವನೆಯನ್ನು ನನ್ನ ಬಾಂಧವರಲ್ಲಿ ಮೂಡುವಂತೆ ಮಾಡಿದ್ದರೆ, ಅಥವಾ ಹೊರಗೆ ಈ ಭಾವನೆ ಹರಡಿದ್ದರೆ, ಅದಕ್ಕೆಲ್ಲಾ, ರಾಮನಾಡಿನ ರಾಜರೇ, ಇಡಿಯ ದೇಶವು ನಿಮಗೆ ಋಣಿ. ಏಕೆಂದರೆ ವಿದೇಶಗಳಿಗೆ ಹೋಗಬೇಕೆಂಬ ಭಾವನೆಯನ್ನು ಮೊದಲು ಕೊಟ್ಟವರು ನೀವು. ಅದನ್ನು ಕಾರ್ಯಗತಗೊಳಿಸಬೇಕೆಂದು ಪದೇ ಪದೇ ಒತ್ತಾಯ ಮಾಡಿದವರು ನೀವು. ಅಂತರ್​ದೃಷ್ಟಿಯ ಬಲದಿಂದ ಭವಿಷ್ಯದಲ್ಲಿ ಏನಾಗುವುದೆಂಬುದನ್ನು ಅರಿತಿರುವವರಂತೆ ನನ್ನನ್ನು ಕೈಹಿಡಿದು ನಡೆಸಿ ನನ್ನಲ್ಲಿ ಉತ್ಸಾಹ ತುಂಬಿದಿರಿ. ನಾನು ಪಡೆದ ಯಶಸ್ಸಿಗೆ ಸಂತೋಷ ಪಡುವವರಲ್ಲಿ ನೀವೇ ಮೊದಲಿಗ\-ರಾದುದು ನ್ಯಾಯವಾಗಿಯೇ ಇದೆ, ಮತ್ತು ನಾನು ಹಿಂತಿರುಗಿ ಬರುವಾಗ ಮೊದಲು ನಿಮ್ಮ ಪ್ರಾಂತ್ಯದಲ್ಲಿ ಕಾಲಿಡುತ್ತಿರುವುದು ಸರಿಯಾಗಿಯೇ ಇದೆ.

ನೀವು ಈಗ ತಾನೆ ಹೇಳಿದಂತೆ ನಾವು ಮಹಾಕಾರ್ಯಗಳನ್ನು ಸಾಧಿಸಬೇಕಾಗಿದೆ, ಮಹಾಶಕ್ತಿಯನ್ನು ವ್ಯಕ್ತಗೊಳಿಸಬೇಕಾಗಿದೆ. ನಾವು ಇತರ ರಾಷ್ಟ್ರಗಳಿಗೆ ಹಲವು ವಿಷಯಗಳನ್ನು ಬೋಧಿಸಬೇಕಾಗಿದೆ. ನಮ್ಮ ಮಾತೃಭೂಮಿಯು ದರ್ಶನ ಅಧ್ಯಾತ್ಮ, ನೀತಿ, ಮಾಧುರ್ಯ, ಕೋಮಲತೆ, ಪ್ರೀತಿ ಇವುಗಳ ತವರೂರು. ಇವು ಈಗಲೂ ಇವೆ. ನಾನು ಪಡೆದಿರುವ ಲೋಕಾನುಭವವು ಇದನ್ನು ಮತ್ತೂ ದೃಢಪಡಿಸಿರುವುದು. ಈಗಲೂ ಭರತಖಂಡವು ಈ ಗುಣಗಳಲ್ಲಿ ಪ್ರಪ್ರಥಮ ಸ್ಥಾನದಲ್ಲಿದೆ ಎಂದು ಮುಕ್ತಕಂಠದಿಂದ ಸಾರುತ್ತೇನೆ. ಈ ಒಂದು ಸಣ್ಣ ಪ್ರಸಂಗವನ್ನು ತೆಗೆದುಕೊಳ್ಳಿ. ಕಳೆದ ನಾಲ್ಕೈದು ವರ್ಷಗಳಿಂದ ಎಷ್ಟೋ ರಾಜಕೀಯ ಬದಲಾವಣೆಗಳು ಆಗಿವೆ. ಪಾಶ್ಚಾತ್ಯ ದೇಶಗಳಲ್ಲೆಲ್ಲಾ ಪ್ರಚಂಡ ಸಂಸ್ಥೆಗಳು ಸಮಾಜದ ಸ್ಥಿತಿಯನ್ನೇ ಬದಲಾಯಿಸುವುದಕ್ಕೆ ಎಷ್ಟೋ ಆಂದೋಲನಗಳನ್ನು ನಡೆಸಿ ಸ್ವಲ್ಪಮಟ್ಟಿನ ಜಯವನ್ನು ಗಳಿಸಿವೆ. ಈ ವಿಷಯವನ್ನೇನಾದರೂ ಕೇಳಿದ್ದೀರಾ ಎಂದು ನಮ್ಮ\break ದೇಶದವರನ್ನು ವಿಚಾರಿಸಿ. ಅವರಿಗೆ ಇದಾವುದೂ ತಿಳಿಯದು. ಆದರೆ ಚಿಕಾಗೋ ನಗರದಲ್ಲಿ ಒಂದು ವಿಶ್ವಧರ್ಮ ಸಮ್ಮೇಳನ ನಡೆಯಿತು, ಭರತಖಂಡದಿಂದ ಒಬ್ಬ ಸಂನ್ಯಾಸಿಯನ್ನು ಅಲ್ಲಿಗೆ ಕಳುಹಿಸಿಕೊಟ್ಟರು, ಆ ದೇಶದ ಜನರು ಹೃತ್ಪೂರ್ವಕವಾಗಿ ಅವನನ್ನು ಸ್ವಾಗತಿಸಿದರು, ಆತ ಇನ್ನೂ ಅಲ್ಲಿ ಕೆಲಸ ಮಾಡುತ್ತಿರುವನು ಎಂಬ ಸಂಗತಿಗಳು ಇಲ್ಲಿಯ ಒಬ್ಬ ದರಿದ್ರ ಭಿಕ್ಷುಕನಿಗೂ ಗೊತ್ತಿದೆ. ನಮ್ಮ ಜನರು ಮೂಢರು, ಅವರಿಗೆ ವಿದ್ಯಾಭ್ಯಾಸ ಬೇಕಾಗಿಲ್ಲ, ಅವರಿಗೆ ವಿಷಯ ಸಂಗ್ರಹದ ಕುತೂಹಲವಿಲ್ಲ ಎಂದು ಹೇಳುವುದನ್ನು ನಾನು ಕೇಳಿದ್ದೇನೆ. ನಾನೂ ಒಂದಾನೊಂದು ಕಾಲದಲ್ಲಿ ಹೀಗೆಯೇ ಮೂಢನಂತೆ ಭಾವಿಸಿದ್ದೆ. ಆದರೆ ಬರಿಯ ಊಹೆಗಿಂತ, ದೇಶಗಳನ್ನು ಸುತ್ತುವವರು ಅವಸರದಲ್ಲಿ ಬರೆದ ಪುಸ್ತಕಗಳನ್ನು ಓದುವುದಕ್ಕಿಂತ, ಅನುಭವ ದೊಡ್ಡ ಗುರು. ಈ ಅನುಭವದಿಂದ ನಮ್ಮ ಜನರು ಮೂಢರಲ್ಲ, ಮಂದಬುದ್ಧಿಯವರಲ್ಲ, ಇತರ ಜನಾಂಗದವರಂತೆ ಅವರೂ ವಿಷಯ ಸಂಗ್ರಹವನ್ನು ಮಾಡಲು ಕುತೂಹಲಿಗಳಾಗಿರುವರು ಎಂಬುದೂ ಗೊತ್ತಾಯಿತು. ಆದರೆ ಪ್ರತಿಯೊಂದು ದೇಶವೂ ವಹಿಸಬೇಕಾದ ಪಾತ್ರವೇ ಬೇರೆ. ಅದರಂತೆಯೇ ಪ್ರತಿಯೊಂದು ದೇಶಕ್ಕೂ ಸ್ವಾಭಾವಿಕವಾಗಿ ಅದರದೇ ಆದ ವೈಶಿಷ್ಟ್ಯ ಮತ್ತು ವೈಯಕ್ತಿಕತೆಗಳು ಇವೆ. ಸರ್ವರಾಷ್ಟ್ರಗಳ ಸ್ವರಮೇಳದಲ್ಲಿ ಪ್ರತಿಯೊಂದು ದೇಶವೂ ಒಂದೊಂದು ರಾಗದಂತೆ ಇದೆ. ಇದೇ ಅದರ ಜೀವಾಳ, ಸಾರ, ಇದೇ ಅದರ ಬೆನ್ನೆಲುಬು; ಅದರ ರಾಷ್ಟ್ರ ಜೀವನದ ತಳಹದಿ. ಧರ್ಮವು ಮಾತ್ರವೇ ಈ ದೇಶದ ಬೆನ್ನೆಲುಬು, ತಳಹದಿ ಮತ್ತು ಜೀವನದ ಕೇಂದ್ರ. ಇತರರು ರಾಜಕೀಯವನ್ನು ಕುರಿತು ಮಾತನಾಡಲಿ, ವ್ಯಾಪಾರದಿಂದ ಐಶ್ವರ್ಯವನ್ನು ಗಳಿಸುವುದರ ವೈಭವವನ್ನು ಕುರಿತು ಹೇಳಲಿ, ವ್ಯಾಪಾರದ ಶಕ್ತಿಯನ್ನೂ ಅದು ಹರಡಬೇಕಾದುದರ ಅಗತ್ಯವನ್ನೂ ಕುರಿತು ಹೇಳಲಿ, ಭೌತಿಕ ಸ್ವಾತಂತ್ರ್ಯದ ಉಗಮದ ಮಹಿಮೆಯನ್ನು ಕುರಿತು ಮಾತ\-ನಾಡಲಿ – ಇದಾವುದನ್ನೂ ಹಿಂದೂ ಅರ್ಥ ಮಾಡಿಕೊಳ್ಳಲಾರ, ಅರ್ಥಮಾಡಿಕೊಳ್ಳುವುದಕ್ಕೆ ಇಚ್ಛಿಸುವುದೂ ಇಲ್ಲ. ಆದರೆ ಅವನ ಹತ್ತಿರ ಅಧ್ಯಾತ್ಮ, ದೇವರು, ಧರ್ಮ, ಆತ್ಮ, ಅನಂತ, ಧಾರ್ಮಿಕ ಸ್ವಾತಂತ್ರ್ಯ ಇವುಗಳ ವಿಷಯ ಮಾತನಾಡಿ; ಒಬ್ಬ ಕನಿಷ್ಠ ಕೂಲಿಯೂ ಕೂಡ ಈ ವಿಷಯಗಳನ್ನು ಕುರಿತು ಇತರ ದೇಶದ ಘನ ವಿದ್ವಾಂಸರಿಗಿಂತ ಹೆಚ್ಚು ತಿಳಿದುಕೊಂಡಿರುವನು. ಸಭಿಕರೇ, ಪ್ರಪಂಚಕ್ಕೆ ನಾವು ಬೋಧಿಸಬೇಕಾಗಿರುವುದೊಂದು ಇದೆ ಎಂದು ಹೇಳಿದೆನು. ಆದಕಾರಣವೇ ಈ ದೇಶವು ನೂರಾರು ವರ್ಷಗಳಿಂದ ದಬ್ಬಾಳಿಕೆಗೆ ತುತ್ತಾಗಿದ್ದರೂ, ಸಹಸ್ರಾರು ವರ್ಷಗಳಿಂದ ಪರಾಧೀನತೆಯಲ್ಲಿದ್ದರೂ ಪರಕೀಯರ ಕ್ರೌರ್ಯಕ್ಕೆ ತುತ್ತಾಗಿದ್ದರೂ, ಬದುಕಿಕೊಂಡಿದೆ, ಈ ದೇಶ ಇಂದೂ ಜೀವಂತವಾಗಿದೆ. ಇದಕ್ಕೆ ಕಾರಣ ಏನೆಂದರೆ ಇದು ಇನ್ನೂ ಈಶ್ವರನನ್ನು ನೆಚ್ಚಿಕೊಂಡಿರುವುದು ಮತ್ತು ಅಧ್ಯಾತ್ಮ ಮತ್ತು ಧರ್ಮವೆಂಬ ಅನರ್ಘ್ಯ ರತ್ನಗಳನ್ನು ಆಶ್ರಯಿಸಿರುವುದು.

ಈ ದೇಶದಲ್ಲಿ ಧರ್ಮ ಮತ್ತು ಅಧ್ಯಾತ್ಮ ಪ್ರವಾಹಗಳು ಇನ್ನೂ ಹರಿಯುತ್ತಿವೆ. ಪ್ರಪಂಚವನ್ನೆಲ್ಲಾ ಇವು ತುಂಬಿ ರಾಜಕೀಯ ಮಹತ್ವಾಕಾಂಕ್ಷೆ ಮತ್ತು ಸಾಮಾಜಿಕ ಗೊಂದಲಗಳಲ್ಲಿ ಸಿಕ್ಕಿ ಜರ್ಝರಿತವಾಗಿ ನಿರಾಶೆಗೊಂಡು ಮೃತ್ಯುಪ್ರಾಯವಾಗಿರುವ ಪಾಶ್ಚಾತ್ಯ ದೇಶಗಳಿಗೆ ಹೊಸಬೆಳಕನ್ನು, ಹೊಸ ಬಾಳನ್ನು ನೀಡಬೇಕಾಗಿದೆ. ಸಮರಸವೂ ವಿರಸವೂ ಆದ ಅನೇಕ ಸ್ವರಗಳ ನಡುವೆ ಭಾರತದ ವಾತಾವರಣವನ್ನು ತುಂಬಿರುವ ನಾನಾಧ್ವನಿಗಳ ಗೊಂದಲದ ನಡುವೆ ಸರ್ವೋಚ್ಚವೂ, ಸ್ಪಷ್ಟವೂ, ಪೂರ್ಣವೂ ಆದ ಒಂದು ಧ್ವನಿ ಕೇಳುತ್ತಿದೆ. ಅದು “ತ್ಯಾಗ”. ‘ತ್ಯಾಗಮಾಡು’ ಎಂಬುದೇ ಭಾರತದ ಧರ್ಮಗಳ ಧ್ಯೇಯವಾಕ್ಯ. ಈ ಪ್ರಪಂಚ ಒಂದೆರಡು ದಿನಗಳ ಭ್ರಾಂತಿ; ಜೀವನ ಕ್ಷಣಿಕ. ಇದರಿಂದ ಬಹಳ ದೂರದಲ್ಲಿ ಅನಂತ ಅಪಾರ ರಾಜ್ಯವಿದೆ. ಅದನ್ನು ಮಹಾವೀರ, ಮನೀಷಿಗಳ ಹೃದಯದ ಜ್ಯೋತಿಯು ಬೆಳಗುತ್ತಿದೆ. ಅವರು ಈ ಅನಂತ ಜಗತ್ತನ್ನು ಕೂಡ ಒಂದು ಕಸದ ಗುಂಡಿಯಂತೆ ನೋಡಬಲ್ಲರು. ಇದನ್ನು ಮೀರಿ ಹೋಗಬೇಕು, ಅನಂತ ಕಾಲವನ್ನು ಕೂಡ ಅವರು ಗಣನೆಗೆ ತರುವುದಿಲ್ಲ. ಅವರಿಗೆ ಕಾಲವೇ ಇಲ್ಲ, ಕಾಲಾತೀತರಾಗಿ ಹೋಗುವರು. ದೇಶವನ್ನು ಕೂಡ ಅವರು ಲೆಕ್ಕಿಸು\-ವುದಿಲ್ಲ. ಅದನ್ನು ಮೀರಿ ಹೋಗಲು ಬಯಸುವರು. ಕಾಲದೇಶಗಳನ್ನು ಮೀರಿ ಹೋಗುವುದೇ ಧರ್ಮದ ಸಾರ. ಈ ದೇಶದ ಒಂದು ಲಕ್ಷಣವೇ ಈ ಅತೀಂದ್ರಿಯ ದರ್ಶನ, ಅತೀತರಾಗಬೇಕೆಂಬ ಹೋರಾಟ. ಪ್ರಕೃತಿಯ ತೆರೆಯನ್ನು ಸೀಳಿ, ಎಂತಹ ಅಪಾಯವನ್ನಾದರೂ ಎದುರಿಸಿ, ಎಷ್ಟು ಕಷ್ಟವಾದರೂ ಆಗಲಿ ಅದರ ಹಿಂದಿರುವ ಪರಂಜ್ಯೋತಿಯ ಕ್ಷಣಿಕ ದರ್ಶನವನ್ನಾದರೂ ಪಡೆಯಬೇಕೆಂಬುದೇ ನಮ್ಮ ಆದರ್ಶ. ಆದರೆ ದೇಶದ ಜನರೆಲ್ಲಾ ಸರ್ವಸಂಗ ಪರಿತ್ಯಾಗ ಮಾಡಲಾರರು. ಅವರಿಗೆ ನೀವು ಪ್ರೋತ್ಸಾಹ ಕೊಡಬೇಕಾದರೆ ಇಲ್ಲಿದೆ ಮಾರ್ಗ. ನಿಮ್ಮ ರಾಜನೀತಿ, ಸಮಾಜ ಸುಧಾರಣೆ, ಐಶ್ವರ್ಯ ಸಂಪಾದನೆ, ವ್ಯಾಪಾರ, ಇವುಗಳ ಕತೆಯೆಲ್ಲ ನೀರುಹಕ್ಕಿಯ ಮೇಲಿನ ನೀರಿನಂತೆ ಜಾರಿ ಹೋಗುವುದು. ನೀವು ಜಗತ್ತಿಗೆ ಬೋಧಿಸಬೇಕಾಗಿರುವುದು ಈ ಆಧ್ಯಾತ್ಮಿಕತೆಯನ್ನು. ನಾವು ಆ ದೇಶಗಳಿಂದ ಏನನ್ನಾದರೂ ಕಲಿಯುವುದಿದೆಯೆ? ಹೌದು; ವಿಜ್ಞಾನಶಾಸ್ತ್ರ, ಸಂಘಟನಾಶಕ್ತಿ, ಅಧಿಕಾರವನ್ನು ಬಳಸುವ ರೀತಿ, ವ್ಯಕ್ತಿಗಳ ಸಾಮರ್ಥ್ಯವನ್ನು ವ್ಯವಸ್ಥೆಗೊಳಿಸುವ ಕೌಶಲ, ಕಿರಿದಾದುದರಿಂದಲೂ ಹಿರಿದಾದುದನ್ನು ಪಡೆಯುವ ಕಲೆ; ಇಂತಹ ಕೆಲವನ್ನು ಕಲಿಯಬೇಕಾಗಿದೆ. ಆದರೆ ಯಾರಾದರೂ ಭರತಖಂಡದಲ್ಲಿ, ಕೇವಲ ತಿನ್ನುವುದು ಕುಡಿಯುವುದು ವಿನೋದ ಇವೇ ಜೀವನದ ಗುರಿ ಎಂದು ಭಾವಿಸಿದರೆ, ದೇವರ ಬದಲು ಪ್ರಪಂಚವನ್ನು ತಂದರೆ, ಅವನು ಮಿಥ್ಯಾವಾದಿ. ಈ ಪುಣ್ಯಭೂಮಿಯಲ್ಲಿ ಅವನಿಗೆ ಸ್ಥಳವಿಲ್ಲ. ಭಾರತೀಯರು ಅವನಿಗೆ ಕಿವಿ ಗೊಡುವುದಿಲ್ಲ. ಪಾಶ್ಚಾತ್ಯ ನಾಗರಿಕತೆ ಕೋರೈಸುತ್ತಿರುವಾಗಲೂ, ಅದರ ನಯ ನಾಜೂಕುಗಳು, ಅದರ ಅದ್ಭುತ ಶಕ್ತಿಯ ಅಭಿವ್ಯಕ್ತಿ ಇವುಗಳು ಇದ್ದಾಗ್ಯೂ, ಅವರೆದುರಿಗೆ ನಿಂತು ನಾನು ಇದೆಲ್ಲಾ ಭ್ರಾಂತಿ, ಇದೆಲ್ಲ ಮಿಥ್ಯೆಯೆಂದು ಮುಕ್ತ ಕಂಠದಿಂದ ಸಾರಿ ಹೇಳುತ್ತೇನೆ. ಅವೆಲ್ಲವೂ ನಿಸ್ಸಾರ. ಈಶ್ವರನೊಬ್ಬನೇ ಸತ್ಯ, ಆತ್ಮ ಒಂದೇ ಸತ್ಯ, ಅಧ್ಯಾತ್ಮ ಒಂದೇ ಸತ್ಯ, ಅವನ್ನು ಮಾತ್ರ ನೆಚ್ಚಿ.

ಆದರೂ ನಮ್ಮಲ್ಲಿ ಉಚ್ಚತಮ ಆದರ್ಶಕ್ಕೆ ಸಿದ್ಧರಲ್ಲದ ಅನೇಕರಿಗೆ ಅವರ ಅಗತ್ಯಕ್ಕೆ ತಕ್ಕ ಸ್ವಲ್ಪಮಟ್ಟಿನ ಪ್ರಾಪಂಚಿಕತೆ ಕಲ್ಯಾಣಕಾರಿ. ಈ ದೋಷವನ್ನು ಪ್ರತಿಯೊಂದು ದೇಶದಲ್ಲೂ ಪ್ರತಿಯೊಂದು ಸಮಾಜದಲ್ಲೂ ನೋಡುವೆವು. ಭರತಖಂಡದಲ್ಲಿ ನಾವು ಕೂಡ\break ಇದೇ ತಪ್ಪನ್ನು ಈಚೆಗೆ ಮಾಡಿರುವುದು ವಿಷಾದಕರ. ಅದು ಯಾವುದೆಂದರೆ ಇನ್ನೂ ಯೋಗ್ಯರಲ್ಲದವರಿಗೆ ಶ್ರೇಷ್ಠ ಸತ್ಯವನ್ನು ಬಲಾತ್ಕಾರವಾಗಿ ಸ್ವೀಕರಿಸಿ ಎಂದು ಹೇಳುವುದು. ನನ್ನ ಮಾರ್ಗ ನಿನ್ನದಾಗದೇ ಇರಬಹುದು. ನಮಗೆಲ್ಲರಿಗೂ ಗೊತ್ತಿರುವಂತೆ ಹಿಂದೂಗಳ ಆದರ್ಶ ಸಂನ್ಯಾಸ. ಪ್ರತಿಯೊಬ್ಬರೂ ಪ್ರಪಂಚವನ್ನು ತ್ಯಜಿಸಬೇಕೆಂದು ನಮ್ಮ ಶಾಸ್ತ್ರ ವಿಧಿಸುವುದು. ಈ ಪ್ರಪಂಚವನ್ನು ಅನುಭವಿಸಿದ ಹಿಂದೂಗಳೆಲ್ಲರೂ ಜೀವನದ ಕೊನೆಗಾಲದಲ್ಲಿ ಇದನ್ನು ತ್ಯಜಿಸಬೇಕು. ಯಾರು ಹೀಗೆ ಮಾಡುವುದಿಲ್ಲವೋ ಅವರು ಹಿಂದೂಗಳಲ್ಲ, ಹಿಂದೂಗಳೆಂದು ಹೇಳಿಕೊಳ್ಳುವುದಕ್ಕೆ ಅವರಿಗೆ ಅಧಿಕಾರವಿಲ್ಲ. ಈ ಕ್ಷಣಭಂಗುರ ಪ್ರಪಂಚವನ್ನು ಅನುಭವಿಸಿ ತ್ಯಜಿಸುವುದು ಆದರ್ಶವೆನ್ನುವುದು ನಮಗೆ ಗೊತ್ತಿದೆ. ಈ ಸಂಸಾರದೊಳಗೆ ಏನೂ ಇಲ್ಲ, ಇದು ಬರೀ ಪೊಳ್ಳು, ನಿಸ್ಸಾರ, ಇದರಲ್ಲಿ ಬೂದಿ ಮಾತ್ರ ಇರುವುದು ಎಂದು ತಿಳಿದು ತ್ಯಜಿಸಿ ನಿವೃತ್ತಿ ಮಾರ್ಗವನ್ನು ಅವಲಂಬಿಸಬೇಕು. ಮನಸ್ಸು ಹಿಂತಿರುಗಿ ಹೋಗಬೇಕು, ಪ್ರವೃತ್ತಿ ನಿಂತು ನಿವೃತ್ತಿ ಪ್ರಾರಂಭವಾಗಬೇಕು, ಇದೇ ಆದರ್ಶ. ಆದರೆ ಪ್ರಪಂಚವನ್ನು ಸ್ವಲ್ಪ ಅನುಭವಿಸಿದ ಮೇಲೆ ಮಾತ್ರ ಈ ಆದರ್ಶ ಸಾಧ್ಯ. ಮಗುವಿಗೆ ತ್ಯಾಗಜೀವನದ ಸತ್ಯವನ್ನು ಬೋಧಿಸುವುದು ಸಾಧ್ಯವಿಲ್ಲ. ಅದು ಹುಟ್ಟು ಆಶಾವಾದಿ. ಅದರ ಜೀವನವೆಲ್ಲಾ ಪಂಚೇಂದ್ರಿಯಗಳಲ್ಲಿರುವುದು. ಅದರಲ್ಲಿ ಆನಂದಿಸುವುದೇ ಅದರ ಏಕಮಾತ್ರ ಗುರಿ. ಪ್ರತಿಯೊಂದು ಸಮಾಜದಲ್ಲಿಯೂ ಶಿಶುಸಮಾನರಾದ ಜನರಿರುವರು. ಅವರಿಗೆ ಸ್ವಲ್ಪ ಸುಖಭೋಗದ ಅನುಭವಬೇಕಾಗಿದೆ. ಆಗ ಮಾತ್ರ ಅದರ ಅನಿತ್ಯತೆಯನ್ನು ತಿಳಿಯುವರು. ಅನಂತರ ತ್ಯಾಗ ಬುದ್ಧಿ ಬರುವುದು. ನಮ್ಮ ಶಾಸ್ತ್ರಗಳು ಅಂಥವರಿಗೂ ಅವಕಾಶವನ್ನು ಕಲ್ಪಿಸಿವೆ. ಆದರೆ ದುರದೃಷ್ಟವಶಾತ್​ ಅನಂತರದ ಕಾಲದಲ್ಲಿ, ಯಾವ ನಿಯಮಗಳು ಸಂನ್ಯಾಸಿಗೆ ಅನ್ವಯಿಸುವುವೋ ಅವನ್ನೇ ಎಲ್ಲರ ಮೇಲೂ ಹೇರುವ ಸ್ವಭಾವ ಕಂಡು ಬರುತ್ತದೆ. ಇದೊಂದು ಮಹಾಪರಾಧ! ಇದಿಲ್ಲದೆ ಇದ್ದರೆ ಭರತಖಂಡದಲ್ಲಿ ಇಷ್ಟೊಂದು ದಾರಿದ್ರ್ಯ ದುಃಖ ಇರುತ್ತಿರಲಿಲ್ಲ. ತನಗೆ ಪ್ರಯೋಜನವಿಲ್ಲದ ಧರ್ಮದ ಮತ್ತು ಆಧ್ಯಾತ್ಮಿಕ ನಿಯಮಗಳ ಜಾಲದಲ್ಲಿ ಪಾಪ, ಬಡವ ಸಿಕ್ಕಿ ನರಳುವನು. ದೂರ ನಿಲ್ಲಿ! ಸಂಸಾರವನ್ನು ಅವನು ಸ್ವಲ್ಪ ಅನುಭವಿಸಲಿ. ಅನಂತರ ಅವನು ತಾನೇ ವಿಮುಖನಾಗುವನು. ತ್ಯಾಗ ಬಂದೇ ಬರುವುದು. ಪಾಶ್ಚಾತ್ಯರಿಂದ ನಾವು ಸ್ವಲ್ಪ ಕಲಿತುಕೊಳ್ಳುವಾಗ ಈ ಮಾರ್ಗದಲ್ಲಿ ಬಹಳ ಜೋಪಾನವಾಗಿರಬೇಕು; ಪಾಶ್ಚಾತ್ಯ ಭಾವನೆಗಳನ್ನು ಮೈಗೂಡಿಸಿಕೊಂಡವರಲ್ಲಿ ಇಂದಿನ ಬಹುಪಾಲು ಜನರು ನಿಷ್ಪ್ರಯೋಜಕರು.

ಭರತಖಂಡದ ಶ್ರೇಯಸ್ಸಿನ ದಾರಿಯಲ್ಲಿ ಎರಡು ಆತಂಕಗಳಿವೆ. ಒಂದು ಪೂರ್ವಾಚಾರ ಪರಾಯಣತೆ; ಮತ್ತೊಂದು ಆಧುನಿಕ ಪಾಶ್ಚಾತ್ಯ ಸಂಸ್ಕೃತಿಯ ಅನುಕರಣೆ. ಇವೆರಡರಲ್ಲಿ ನಾನು ಪೂರ್ವಾಚಾರಿಗಳನ್ನು ಮೆಚ್ಚುತ್ತೇನೆ, ಆಧುನಿಕ ಕೃತಕ ಪಾಶ್ಚಾತ್ಯ ಸಂಸ್ಕೃತಿಯನ್ನು ಅನುಕರಿಸುವವರನ್ನಲ್ಲ. ಪೂರ್ವಾಚಾರ ಪರಾಯಣರು ಅಜ್ಞರಾಗಿರಬಹುದು, ಅವರಲ್ಲಿ ಅಷ್ಟು ನಯವಿಲ್ಲದೆ ಇರಬಹುದು, ಆದರೆ ಅವರು ಮನುಷ್ಯರು. ಅವರಿಗೆ ಶ್ರದ್ಧೆ ಇದೆ, ಕಾಲಿನಮೇಲೆ ತಾವು ನಿಂತುಕೊಳ್ಳಬಲ್ಲರು. ಪಾಶ್ಚಾತ್ಯರನ್ನು ಅನುಸರಿಸುವವರಲ್ಲಿ ವ್ಯಕ್ತಿತ್ವವೇ ಇಲ್ಲ. ಅವರು ಮೂಲೆ ಮೂಲೆಗಳಿಂದ ಹಲವು ವಿಷಯಗಳನ್ನು ಸಂಗ್ರಹ ಮಾಡಿ ಮನಸ್ಸನ್ನು ಒಂದು ಉಗ್ರಾಣವನ್ನಾಗಿ ಮಾಡಿಕೊಂಡಿರುವರು. ಆ ವಿಷಯಗಳನ್ನು ಜೀರ್ಣಿಸಿಕೊಂಡಿಲ್ಲ; ಅವುಗಳಲ್ಲಿ ಒಂದು ರೀತಿಯಿಲ್ಲ, ರಚನೆಯಿಲ್ಲ. ಅವರು ಸ್ಥೈರ್ಯದಿಂದ ನಿಲ್ಲಲಾರರು, ತತ್ತರಿಸುತ್ತಿರುವರು. ಅವರ ಕೆಲಸಕ್ಕೆ ಪ್ರೋತ್ಸಾಹ ದೊರಕುತ್ತಿರುವುದು ಎಲ್ಲಿಂದ? ಆಂಗ್ಲೇಯರು ಭುಜತಟ್ಟಿ ಕೊಡುವ ಸ್ವಲ್ಪ ಶಹಭಾಷ್​ಗಿರಿಯಿಂದ. ಅವರು ಮಾಡುವ ಸಮಾಜ ಸುಧಾರಣೆಯ ಯೋಜನೆಗಳ ಹಿಂದೆ, ಅವರು ಸಮಾಜದ ಕೆಲವು ಅನಿಷ್ಟ ಪದ್ಧತಿಗಳನ್ನು ಅತಿ ಕಟುವಾಗಿ ಹಳಿಯುವುದರ ಹಿಂದೆ, ಯಾರೋ ಕೆಲವು ಆಂಗ್ಲೇಯರು ಉತ್ತೇಜನವನ್ನು ಕೊಡುತ್ತಿರುವರು. ನಿಮ್ಮ ಕೆಲವು ಆಚಾರಗಳನ್ನು ಹೀನವಾದುವೆಂದು ಏತಕ್ಕೆ ಟೀಕಿಸುತ್ತಾರೆ? ಪಾಶ್ಚಾತ್ಯರು ಹಾಗೆ ಹೇಳುವರು ಎಂದು. ಅವರು ಕೊಡುವ ಕಾರಣವೇ ಇದು. ನಾನು ಇದನ್ನು ಒಪ್ಪಿಕೊಳ್ಳುವುದಿಲ್ಲ. ನಿಮ್ಮ ಶಕ್ತಿಯ ಮೇಲೆ ನಿಂತು ಹೋರಾಡಿ. ಈ ಪ್ರಪಂಚದಲ್ಲಿ ಏನಾದರೂ ಪಾತಕವಿದ್ದರೆ ಅದೇ ದೌರ್ಬಲ್ಯ. ಎಲ್ಲ ಬಗೆಯ ದೌರ್ಬಲ್ಯದಿಂದಲೂ ಪಾರಾಗಿ. ದೌರ್ಬಲ್ಯವೇ ಮಹಾಪಾತಕ, ದೌರ್ಬಲ್ಯವೇ ಮರಣ. ಈ ಸ್ತಿಮಿತವಿಲ್ಲದ\break ಮನುಷ್ಯರು ಇನ್ನೂ ಸ್ಪಷ್ಟ ವ್ಯಕ್ತಿಗಳಾಗಿಲ್ಲ. ಇವರನ್ನು ನಾವು ಏನೆಂದು ಕರೆಯೋಣ? ಗಂಡಸರೇ, ಹೆಂಗಸರೇ, ಪ್ರಾಣಿಗಳೇ? ಆದರೆ ಪೂರ್ವಾಚಾರಪರಾಯಣರಲ್ಲಿ ಶ್ರದ್ಧೆ ಇತ್ತು, ಅವರು ಧೀರರಾಗಿದ್ದರು. ಇನ್ನೂ ನಮ್ಮಲ್ಲಿ ಕೆಲವು ಅಂತಹ ಅತ್ಯುತ್ತಮ ಉದಾಹರಣೆಗಳಿವೆ. ನಿಮ್ಮ ರಾಮನಾಡಿನ ರಾಜರೇ ನಿಮ್ಮ ಎದುರಿಗೆ ಇರುವ ಒಂದು ಉದಾಹರಣೆ. ಇಡಿಯ ಭರತಖಂಡದಲ್ಲೇ ಇವರಿಗಿಂತ ಹೆಚ್ಚು ಶ್ರದ್ಧಾವಂತ ಹಿಂದೂ ಇಲ್ಲ. ಪ್ರಾಚ್ಯ–ಪಾಶ್ಚಾತ್ಯ ವಿಷಯಗಳಲ್ಲಿ ಇವರಿಗಿಂತ ಹೆಚ್ಚು ಮತ್ತೊಬ್ಬ ರಾಜನಿಗೆ ತಿಳಿಯದು. ಇವರು ಪ್ರತಿಯೊಂದು ದೇಶದಿಂದಲೂ ಒಳ್ಳೆಯದನ್ನು ಸ್ವೀಕರಿಸಬಲ್ಲರು. “ಅಂತ್ಯಜರಿಂದಲೂ ಶ್ರದ್ಧೆಯಿಂದ ಒಳ್ಳೆಯ ವಿಷಯವನ್ನು ಕಲಿಯಿರಿ. ಮುಕ್ತಿ ಮಾರ್ಗವು ಒಬ್ಬ ಪರೆಯನ ಬಾಯಿಂದ ಬಂದರೂ ಶ್ರದ್ಧಾ ಭಕ್ತಿಗಳಿಂದ ಕಲಿಯಿರಿ. ಸ್ತ್ರೀರತ್ನವು ಅತಿ ನೀಚ ಕುಲದಿಂದ ಬಂದಿದ್ದರೂ ಅವಳನ್ನು ಸತಿಯಾಗಿ ಸ್ವೀಕರಿಸಿ” –ಧರ್ಮಶಾಸ್ತ್ರದ ಕರ್ತೃವಾದ ಆ ಪ್ರಖ್ಯಾತ ಮನು ವಿಧಿಸಿದ ಅಸದೃಶ ನಿಯಮಗಳು ಇವು. ಇದು ಸತ್ಯ. ಸ್ವತಂತ್ರವಾಗಿದ್ದು ನಿಮಗೆ ಸಾಧ್ಯವಾದಷ್ಟನ್ನು ಜೀರ್ಣಿಸಿಕೊಳ್ಳಿ. ಎಲ್ಲಾ ದೇಶಗಳಿಂದಲೂ ಕಲಿತುಕೊಳ್ಳಿ. ನಿಮಗೆ ಯಾವುದು ಉಪಯೋಗಕರವಾಗುವುದೋ ಅದನ್ನು ತೆಗೆದುಕೊಳ್ಳಿ. ಆದರೆ ನಾವು ಹಿಂದೂಗಳಾದುದರಿಂದ ಸರ್ವವೂ ನಮ್ಮ ರಾಷ್ಟ್ರಧ್ಯೇಯಕ್ಕೆ ಅಧೀನವಾಗಿರಬೇಕು. ಪ್ರತಿಯೊಬ್ಬನಿಗೂ ತನ್ನ ಸಂಸ್ಕಾರಕ್ಕೆ ಅನುಗುಣವಾಗಿ ಒಂದೊಂದು ಧ್ಯೇಯವಿದೆ. ನಿಮ್ಮಲ್ಲಿ ಎಲ್ಲರಲ್ಲಿಯೂ ಅಮೋಘವಾದ, ಪರಂಪರಾಗತವಾಗಿ ಬಂದ ಸುಸಂಸ್ಕೃತಿ ಇದೆ. ಅದು ನಿಮ್ಮ ಭವ್ಯ ರಾಷ್ಟ್ರದ ಅನಂತ ಭೂತಕಾಲದ ಕೊಡುಗೆ. ಕೋಟ್ಯಂತರ ಪೂರ್ವಿಕರು ನಿಮ್ಮನ್ನು ಮತ್ತು ಇಂದು ನೀವು ಮಾಡುವ ಪ್ರತಿಯೊಂದು ಕಾರ್ಯವನ್ನು ಗಮನಿಸುತ್ತಿರುವಂತೆ ತೋರುತ್ತಿದೆ. ಆದ್ದರಿಂದ ಎಚ್ಚರಿಕೆಯಿಂದಿರಿ. ಪ್ರತಿಯೊಬ್ಬ ಹಿಂದೂ ಮಗುವಿನ ಆದರ್ಶ ಯಾವುದು? ಬ್ರಾಹ್ಮಣನ ಕರ್ತವ್ಯವನ್ನು ಮನು ವಿವರಿಸುವಾಗ:

\begin{verse}
\textbf{“ಬ್ರಾಹ್ಮಣೋ ಜಾಯಮಾನೋ ಹಿ ಪೃಥಿವ್ಯಾಮಧಿಜಾಯತೇ~।}\\\textbf{ಈಶ್ವರಃ ಸರ್ವಭೂತಾನಾಂ ಧರ್ಮಕೋಶಸ್ಯ ಗುಪ್ತಯೇ”}
\end{verse}

\vskip   -0.5cm

ಬ್ರಾಹ್ಮಣನ ಜನನ ಧರ್ಮರಕ್ಷಣೆಗಾಗಿಯೇ ಎಂದು ಮುಕ್ತಕಂಠದಿಂದ ಸಾರುವುದನ್ನು ನೀವು ಓದಿಲ್ಲವೇ? ಧರ್ಮರಕ್ಷಣೆ ಭರತಖಂಡದಲ್ಲಿ ಹುಟ್ಟುವ ಪ್ರತಿಯೊಬ್ಬ ಸ್ತ್ರೀ–ಪುರುಷ–\- ಮಕ್ಕಳ ಕರ್ತವ್ಯ. ಕೇವಲ ಬ್ರಾಹ್ಮಣರ ಕರ್ತವ್ಯ ಮಾತ್ರವಲ್ಲ. ಜೀವನದ ಪ್ರತಿಯೊಂದು\break ಕರ್ತವ್ಯವೂ ಆ ಪ್ರಧಾನ ವಿಷಯಕ್ಕೆ ಅಧೀನವಾಗಿರಬೇಕು. ಸಂಗೀತದಲ್ಲೂ ಸಮರಸದ ನಿಯಮವೆಂದರೆ ಇದೇ. ರಾಜಕೀಯವೇ ಪ್ರಧಾನವಾದ ಮೂಲ ಮಂತ್ರವಾಗಿ ಉಳ್ಳ ಕೆಲವು ರಾಷ್ಟ್ರಗಳಿಗೆ ಧರ್ಮ ಮುಂತಾದುವೆಲ್ಲ ಆ ಒಂದು ಆದರ್ಶಕ್ಕೆ ಅಧೀನವಾಗಿರಬೇಕು. ಹಿಂದೂಗಳ ಪ್ರಧಾನ ಜೀವನೋದ್ದೇಶಗಳು ಧರ್ಮ ಮತ್ತು ವೈರಾಗ್ಯ. ಹಿಂದೂಗಳ ಮೂಲಮಂತ್ರವೇ, ಜಗತ್ತು ಕ್ಷಣಿಕ, ಭ್ರಮೆ ಮಾತ್ರ, ಮಿಥ್ಯೆ ಎಂಬುದು. ಉಳಿದ ಜ್ಞಾನವೆಲ್ಲ, ಅದು ವಿಜ್ಞಾನವಾಗಲಿ, ಸಾಹಿತ್ಯವಾಗಲಿ, ಭೋಗವಾಗಲಿ, ಅಧಿಕಾರವಾಗಲಿ, ಐಶ್ವರ್ಯ, ಕೀರ್ತಿಗಳಾಗಲಿ ಎಲ್ಲವೂ ಈ ಒಂದು ಮೂಲಮಂತ್ರಕ್ಕೆ ಅಧೀನವಾಗಿರಬೇಕು. ಜನಾಂಗದ ಆಧ್ಯಾತ್ಮಿಕತೆಗೆ ಮತ್ತು ಪರಿಶುದ್ಧತೆಗೆ ಅವನು ಕಲಿತ ಪಾಶ್ಚಾತ್ಯ ವಿಜ್ಞಾನಶಾಸ್ತ್ರ, ಅವನ ಐಶ್ವರ್ಯ, ಅಧಿಕಾರ, ಯಶಸ್ಸು, ಎಲ್ಲಾ ಅಧೀನವಾಗಿರಬೇಕು. ಇದೇ ನಿಜವಾದ ಹಿಂದೂ ಶೀಲದ ರಹಸ್ಯ. ಪೂರ್ವಾಚಾರ–ಪರಾಯಣರಲ್ಲಿ ಜನಾಂಗದ ಮೂಲ ಜೀವಾಳವಾದ ಆಧ್ಯಾತ್ಮಿಕತೆ ಇದೆ. ಪಾಶ್ಚಾತ್ಯ ಆದರ್ಶಕ್ಕೆ ಪರವಶರಾದವರ ಕೈಯಲ್ಲಿ ಕೃತಕ ವಜ್ರಗಳಿವೆ; ನಮ್ಮ ಬಾಳಿಗೆ ಬೆಳಕನ್ನು ಕೊಡುವ ಅಧ್ಯಾತ್ಮವಿಲ್ಲ. ಇವರಿಬ್ಬರಲ್ಲಿ ನೀವೆಲ್ಲರೂ ಪೂರ್ವಾಚಾರ ಪರಾಯಣರನ್ನು ಮೆಚ್ಚುವಿರಿ ಎಂದು ನಿಸ್ಸಂಶಯವಾಗಿ ಹೇಳಬಲ್ಲೆ. ಏಕೆಂದರೆ ಅವರಿಗೆ ಸ್ವಲ್ಪ ಭರವಸೆ ಇದೆ. ಅವರ ಜೀವನದಲ್ಲಿ ಜನಾಂಗದ ಆದರ್ಶ ಹರಿಯುತ್ತಿದೆ, ಅವರ ಬಾಳಿಗೆ ಒಂದು ಆಸರೆ ಇದೆ. ಅವರು ಬದುಕುವವರು. ಉಳಿದವರು ನಶಿಸುವರು. ವ್ಯಕ್ತಿ ಜೀವನದಲ್ಲಿ ಪ್ರಾಣಾಪಾಯವಾಗುವಂತಹ ಕೇಡು ದೇಹಕ್ಕೆ ಆಗಿಲ್ಲದೆ ಇರುವಾಗ, ಉಳಿದ ಗೌಣ ವ್ಯಾಧಿಗಳು ಎಷ್ಟಿದ್ದರೂ ಚಿಂತೆಯಿಲ್ಲ. ಅವು ಅವನನ್ನು ಕೊಲ್ಲಲಾರವು. ಇದರಂತೆಯೇ ನಮ್ಮ ಜೀವನದ ಮೂಲಮಂತ್ರಕ್ಕೆ ಎಲ್ಲಿಯವರೆವಿಗೂ ಧಕ್ಕೆ ಬರುವುದಿಲ್ಲವೋ ಅಲ್ಲಿಯವರೆವಿಗೂ ಯಾವುದೂ ನಮ್ಮ ರಾಷ್ಟ್ರವನ್ನು ನಾಶಮಾಡಲಾರದು. ಆದರೆ ಇದನ್ನು ಲಕ್ಷ್ಯದಲ್ಲಿಡಿ. ನೀವು ಅಧ್ಯಾತ್ಮವನ್ನು ತ್ಯಜಿಸಿ, ಪಾಶ್ಚಾತ್ಯ ನಾಗರಿಕತೆಯ ಆಕರ್ಷಣೆಗೆ ಒಳಗಾದರೆ ಮೂರು ತಲೆ ಮಾರಿನ ಹೊತ್ತಿಗೆ ನಿಮ್ಮ ಜನಾಂಗ ನಾಶವಾಗುವುದು. ಏಕೆಂದರೆ ಇದರಿಂದ ರಾಷ್ಟ್ರದ\break ಬೆನ್ನೆಲುಬು ಮುರಿಯುತ್ತದೆ. ಯಾವ ತಳಹದಿಯ ಮೇಲೆ ರಾಷ್ಟ್ರ ಸೌಧವು ನಿರ್ಮಿತವಾಗಿರುವುದೊ ಅದು ಕುಸಿದು ಬೀಳುತ್ತದೆ. ಇದರ ಪರಿಣಾಮವೇ ಸರ್ವನಾಶ.

ಆದ್ದರಿಂದ ಮಿತ್ರರೇ, ನಮ್ಮ ಜನಾಂಗದ ಶ್ರೇಯಸ್ಸಿಗೆ ಇರುವುದೊಂದೇ ಮಾರ್ಗ. ಅದು ಪೂರ್ವಿಕರು ಇತ್ತ ಅಮೂಲ್ಯ ಧರ್ಮಧನವನ್ನು ರಕ್ಷಿಸುವುದು. ಒಂದು ದೇಶದ ಪ್ರಖ್ಯಾತ ರಾಜರು, ತಮ್ಮ ವಂಶದ ಮೂಲಪುರುಷರು ರಾಜರಲ್ಲ, ಕೋಟೆ ಕೊತ್ತಲಗಳಲ್ಲಿ ವಾಸಿಸುತ್ತಿದ್ದ ಮತ್ತು ಸುಲಿಗೆಮಾಡುತ್ತಿದ್ದ ಪಾಳೆಯಗಾರರಲ್ಲ, ಕಾಡಿನಲ್ಲಿ ವಾಸಿಸುತ್ತಿದ್ದ\break ಅರೆಬತ್ತಲೆ ಋಷಿಗಳು ಎಂದು ನಿಷ್ಕರ್ಷಿಸುವುದಕ್ಕೆ ಪ್ರಯತ್ನಿಸುವುದನ್ನು ಕೇಳಿರುವಿರಾ? ಈ ದೇಶದಲ್ಲಿ ಮಾತ್ರ ಅದು ಸಾಧ್ಯ. ಉಳಿದ ದೇಶಗಳಲ್ಲಿ ಧಾರ್ಮಿಕ ವ್ಯಕ್ತಿಗಳು ತಮ್ಮ ಪೂರ್ವಿಕರು ರಾಜರಾಗಿದ್ದರೆಂದು ತೋರಲೆತ್ನಿಸುವರು. ಆದರೆ ಇಲ್ಲಿ ರಾಜಮಹಾರಾಜರು\break ಗಳೂ ತಮ್ಮ ಪೂರ್ವಿಕರು ಋಷಿಗಳಾಗಿದ್ದರೆಂದು ಹೇಳುವರು. ನಿಮಗೆ ಅಧ್ಯಾತ್ಮದಲ್ಲಿ ನಂಬಿಕೆ ಇರಲಿ ಇಲ್ಲದಿರಲಿ, ರಾಷ್ಟ್ರದ ಹಿತಕ್ಕಾಗಿ ಅಧ್ಯಾತ್ಮವನ್ನು ನಂಬಿ ಅದನ್ನು ರಕ್ಷಿಸಬೇಕು. ಅನಂತರ ಮತ್ತೊಂದು ಕೈಯಿಂದ ಇತರ ದೇಶಗಳಿಂದ ಏನನ್ನು ಬೇಕಾದರೂ ಪಡೆಯಿರಿ. ಆದರೆ ಅವು ಯಾವಾಗಲೂ ಪ್ರಧಾನ ಆದರ್ಶಕ್ಕೆ ಅಧೀನವಾಗಿರಬೇಕು. ಇದರಿಂದ ಅದ್ಭುತವಾದ, ವೈಭವಪೂರ್ಣವಾದ ಭವಿಷ್ಯ ಭಾರತವು ಮೂಡಿಬರುವುದು. ಅದು ಹಿಂದಿಗಿಂತ ಹೆಚ್ಚು ಭವ್ಯವಾಗುವುದು. ಅದು ಬರುತ್ತಿದೆ ಎಂದು ನಿಸ್ಸಂದೇಹವಾಗಿ ಹೇಳಬಲ್ಲೆ. ಹಿಂದಿನ ಋಷಿಗಳನ್ನು ಮೀರಿಸಿದ ಋಷಿಗಳು ಉದಯಿಸುವರು. ನಿಮ್ಮ ಪೂರ್ವಿಕರು ತೃಪ್ತರಾಗುವುದು ಮಾತ್ರವಲ್ಲ, ಇಷ್ಟು ಮಹಿಮಾವಂತರಾದ, ಪ್ರಖ್ಯಾತರಾದ ತಮ್ಮ ವಂಶಜರನ್ನು ನೋಡಿದಾಗ ಹೆಮ್ಮೆಪಡುವರು.

ಸಹೋದರರೇ, ಎಲ್ಲರೂ ಕಷ್ಟಪಟ್ಟು ದುಡಿಯೋಣ. ನಿದ್ರೆಗೆ ಸಮಯವಲ್ಲ ಇದು. ನಮ್ಮ ಇಂದಿನ ಕಾರ್ಯದ ಮೇಲೆ ಭವಿಷ್ಯ ಭಾರತ ನಿಂತಿದೆ. ನಮಗಾಗಿ ಆಕೆ ಕಾಯುತ್ತಿರು\-ವಳು. ಆಕೆ ಕೇವಲ ನಿದ್ರಿಸುತ್ತಿರುವಳು, ಅಷ್ಟೆ. ಜಾಗೃತರಾಗಿ, ಏಳಿ, ಎದ್ದು ನೋಡಿ. ಹಿಂದಿಗಿಂತ ಹೆಚ್ಚು ವೈಭವಪೂರ್ಣಳಾಗಿ, ಬಲಿಷ್ಠಳಾಗಿ ನಮ್ಮ ಭಾರತಮಾತೆಯು ಶೋಭಾಯಮಾನವಾಗಿರುವ ಸನಾತನ ಸಿಂಹಾಸನದ ಮೇಲೆ ಮಂಡಿಸಿರುವುದನ್ನು ನೋಡಿ. ದೇವರ ಭಾವನೆಯು ನಮ್ಮ ಮಾತೃಭೂಮಿಯಲ್ಲಿ ಪರಿಪುಷ್ಟವಾಗಿ ಬೆಳೆದಷ್ಟು ಬೇರೆ ದೇಶಗಳಲ್ಲಿ ಬೆಳೆದಿಲ್ಲ. ಇಂತಹ ಭಾವನೆ ಅನ್ಯ ದೇಶಗಳಲ್ಲಿ ಇರಲೇ ಇಲ್ಲ. ನಿಮಗೆ ಇದನ್ನು ಕೇಳಿ ಆಶ್ಚರ್ಯವಾಗಬಹುದು. ಆದರೆ ನಮ್ಮ ಶಾಸ್ತ್ರಗಳಲ್ಲಿ ಇರುವಂತಹ ದೇವರ ಭಾವನೆಯನ್ನು ಇತರರ ಧರ್ಮಶಾಸ್ತ್ರಗಳಲ್ಲಿ ಎಲ್ಲಿಯಾದರೂ ತೋರಿ ನೋಡೋಣ! ಅವರಲ್ಲಿ ಕೇವಲ ಒಂದು ಕೋಮಿನ ದೇವರುಗಳಿವೆ. ಯೆಹೂದ್ಯರ ದೇವರಂತಹ, ಅರಬ್ಬರ ದೇವರಂತಹ ಭಾವನೆಗಳಿವೆ. ಆ ದೇವರುಗಳು ಇತರ ಕೋಮಿನ ದೇವರುಗಳೊಂದಿಗೆ ಹೋರಾಡುತ್ತಿರುವರು. ಆದರೆ ಪರಮದಯಾಳು, ತಂದೆ, ತಾಯಿ, ಸಖ, ನಮ್ಮ ಸಖರ ಸಖನಾದ, ನಮ್ಮ ಪರಮಾತ್ಮನಾದ ಈಶ್ವರನ ಭಾವನೆ ಇಲ್ಲಿ ಮಾತ್ರ ಸಾಧ್ಯ. ಯಾರು ಶೈವರ ಶಿವನೋ, ವೈಷ್ಣವರ ವಿಷ್ಣುವೋ, ಮೀಮಾಂಸಕರ ಕರ್ಮವೋ, ಬೌದ್ಧರ ಬುದ್ಧನೋ, ಜೈನರ ಜಿನನೋ, ಕ್ರೈಸ್ತರ ಕ್ರಿಸ್ತನೋ, ಯೆಹೂದ್ಯರ ಯಹೋವನೋ, ಮಹಮ್ಮದೀಯರ ಅಲ್ಲನೋ, ಯಾರು ಪ್ರತಿಯೊಂದು ಧರ್ಮದ ದೇವರಾಗಿರುವನೋ, ವೇದಾಂತಿಗಳ ಬ್ರಹ್ಮನಾಗಿರುವನೋ ಅವನು ಸರ್ವವ್ಯಾಪಿಯಾಗಿರುವನು; ಅವನ ಮಾಹಾತ್ಮ್ಯೆಯನ್ನು ಈ ದೇಶ ಮಾತ್ರ ತಿಳಿದಿರುವುದು. ಈ ಆದರ್ಶವನ್ನು ಅನುಷ್ಠಾನದಲ್ಲಿ ತರುವುದಕ್ಕೆ ಅವನು ನಮ್ಮನ್ನು ಆಶೀರ್ವದಿಸಲಿ, ಸಹಾಯವನ್ನು ನೀಡಲಿ, ಶಕ್ತಿ ನೀಡಲಿ.

\begin{verse}
\textbf{ಓಂ ಸಹ ನಾವವತು~। ಸಹ ನೌ ಭುನಕ್ತು~।\\ಸಹ ವೀರ್ಯಂ ಕರವಾವಹೈ~।}\\\textbf{ತೇಜಸ್ವಿನಾವಧೀತಮಸ್ತು ಮಾ ವಿದ್ವಿಷಾವಹೈ~॥}
\end{verse}

(ನಾವು ಯಾವುದನ್ನು ಕೇಳಿರುವೆವೋ, ಓದಿರುವೆವೋ, ಅದು ನಮ್ಮ ಆಹಾರವಾಗಲಿ, ಅದು ನಮ್ಮಲ್ಲಿ ಶಕ್ತಿಯಾಗಲಿ, ಪರಸ್ಪರ ಸಹಾಯಕ್ಕೆ ಅದು ನಮ್ಮಲ್ಲಿ ಪೌರುಷವಾಗಲಿ; ಗುರುಶಿಷ್ಯರಲ್ಲಿ ದ್ವೇಷವಿಲ್ಲದಿರಲಿ.)

\begin{center}
ಶಾಂತಿಃ ಶಾಂತಿಃ ಶಾಂತಿಃ ಹರಿಃ ಓಂ
\end{center}

