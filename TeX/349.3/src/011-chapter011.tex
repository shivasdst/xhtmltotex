
\chapter{ನನ್ನ ಸಮರನೀತಿ}

\begin{center}
(ಮದ್ರಾಸಿನ ವಿಕ್ಟೋರಿಯ ಹಾಲಿನಲ್ಲಿ ಮಾಡಿದ ಭಾಷಣ)
\end{center}

ಮೊನ್ನೆ ನನಗೆ ಹೆಚ್ಚು ಜನಸಂದಣಿಯ ದೆಸೆಯಿಂದ ಉಪನ್ಯಾಸವನ್ನು ಪೂರೈಸಲು ಸಾಧ್ಯವಾಗಲಿಲ್ಲ. ಮದ್ರಾಸಿನ ಪುರಜನರು ಏಕಪ್ರಕಾರವಾಗಿ ತೋರಿದ ದಯೆಗೆ ನಾನು ಧನ್ಯವಾದಗಳನ್ನು ಅರ್ಪಿಸುತ್ತೇನೆ. ನೀವು ಅಭಿನಂದನಾ ಪತ್ರದಲ್ಲಿ ನನ್ನ ವಿಷಯವಾಗಿ ವ್ಯಕ್ತಪಡಿಸಿರುವ ಸುಂದರ ಭಾವನೆಗಳಿಗೆ ನಾನು ಹೇಗೆ ನಿಮಗೆ ಕೃತಜ್ಞತೆಯನ್ನು ತೋರಬೇಕೆಂಬುದು ನನಗೆ ತಿಳಿಯದು. ಭಗವಂತನ ಅಡಿದಾವರೆಯಲ್ಲಿ, ನಿಮ್ಮ ಪ್ರಶಂಸೆಗೆ ತಕ್ಕಂತೆ ನನ್ನನ್ನು ಯೋಗ್ಯನನ್ನಾಗಿ ಮಾಡು, ನಮ್ಮ ಧರ್ಮಕ್ಕಾಗಿ ಮತ್ತು ಮಾತೃಭೂಮಿಗಾಗಿ ಬದುಕಿರುವ ತನಕ ದುಡಿಯಲು ಶಕ್ತಿಯನ್ನು ನೀಡು, ಎಂದು ಪ್ರಾರ್ಥನೆಯನ್ನು ಮಾತ್ರ ಮಾಡುವೆನು.

ನನ್ನಲ್ಲಿ ಎಷ್ಟೋ ದೋಷಗಳಿದ್ದರೂ ಸ್ವಲ್ಪ ಸಾಹಸವಿದೆ. ಭರತವರ್ಷವು ಪಾಶ್ಚಾತ್ಯ ದೇಶಗಳಿಗೆ ಕೊಡಬೇಕಾದ ಒಂದು ಸಂದೇಶವಿತ್ತು. ಅದನ್ನು ನಾನು ನಿರ್ಭೀತಿಯಿಂದ ಕೊಟ್ಟೆ. ನಾನು ಇಂದು ಉಪನ್ಯಾಸ ಮಾಡಬೇಕೆಂಬ ವಿಷಯವನ್ನು ಪ್ರಸ್ತಾಪಿಸುವುದಕ್ಕೆ ಮುಂಚೆ ಕೆಲವು ಸಾಹಸಪೂರ್ಣ ಮಾತುಗಳನ್ನು ನಿಮ್ಮಲ್ಲಿ ನಿವೇದಿಸಬೇಕೆಂದಿರುವೆನು. ನನ್ನ ಜಯಕ್ಕೆ ಪ್ರತಿಬಂಧಕವನ್ನು ತಂದೊಡ್ಡಿ, ಸಾಧ್ಯವಾದರೆ ನನ್ನನ್ನು ನಿರ್ನಾಮ ಮಾಡುವಂತಹ ಸನ್ನಿವೇಶ ನನ್ನ ಸುತ್ತಲೂ ಬೆಳೆಯುತ್ತಿತ್ತು. ದೇವರಿಗೆ ಧನ್ಯವಾದ. ಇಂತಹ ಪ್ರಯತ್ನಗಳು ಯಾವಾಗಲೂ ನಿಷ್ಫಲವಾಗುವಂತೆ ಈಗಲೂ ಆಯಿತು. ಆದರೂ ಕಳೆದ ಮೂರು ವರುಷಗಳಿಂದಲೂ ಕೆಲವು ಭಿನ್ನಾಭಿಪ್ರಾಯಗಳು ಇವೆ. ನಾನು ಎಲ್ಲಿಯವರೆವಿಗೂ ಪಾಶ್ಚಾತ್ಯ ದೇಶದಲ್ಲಿದ್ದೆನೊ ಅಲ್ಲಿಯವರೆಗೆ ನಾನು ಒಂದೇ ಒಂದು ಶಬ್ದವನ್ನು ಉಚ್ಚರಿಸಲಿಲ್ಲ. ಈ ವಿಷಯದಲ್ಲಿ ಮೌನವಾಗಿದ್ದೆ. ನಾನೀಗ ಮಾತೃಭೂಮಿಯ ಮೇಲೆ ನಿಂತಿರುವಾಗ ಅದನ್ನು ವಿವರಿಸುವೆನು. ಈ ಮಾತಿನ ಪರಿಣಾಮವನ್ನು ನಾನು ಲೆಕ್ಕಿಸುವುದಿಲ್ಲ. ಇದರಿಂದ ಯಾವ ಭಾವನೆ ನಿಮ್ಮಲ್ಲಿ ಮೂಡುವುದೋ ಅದನ್ನೂ ಲೆಕ್ಕಿಸುವುದಿಲ್ಲ. ನಾನು ನಾಲ್ಕು ವರುಷಗಳ ಹಿಂದೆ ದಂಡ- ಕಮಂಡಲು ಧಾರಿಯಾಗಿ ನಿಮ್ಮ ನಗರವನ್ನು ಪ್ರವೇಶಿಸಿದ ಸಂನ್ಯಾಸಿಯಂತೆಯೇ ಈಗಲೂ ಇರುವೆನು. ಇಡೀ ಪ್ರಪಂಚ ನನಗೆ ತೆರೆದಿರುವುದು. ಹೆಚ್ಚು ಭೂಮಿಕೆ ಇಲ್ಲದೆ ನಾನಿನ್ನು ಪ್ರಾರಂಭಿಸುವೆ.

ಪ್ರಥಮತಃ ಥಿಯಾಸಫಿಕಲ್​ ಸೊಸೈಟಿಯ ವಿಷಯದಲ್ಲಿ ನಾನು ಸ್ವಲ್ಪ ಹೇಳಬೇಕಾಗಿದೆ. ಆ ಸಮಾಜ ಭರತಖಂಡಕ್ಕೆ ಸ್ವಲ್ಪ ಸಹಾಯಮಾಡಿದೆ. ಅದಕ್ಕಾಗಿ ಎಲ್ಲಾ ಹಿಂದೂಗಳು ಅವರಿಗೆ ಋಣಿಗಳು. ಅದರಲ್ಲೂ ಶ‍್ರೀಮತಿ ಆನಿಬೆಸೆಂಟರಿಗೆ ಹೆಚ್ಚು ಋಣಿಗಳು. ನನಗೆ ಅವರ ವಿಷಯ ಹೆಚ್ಚು ತಿಳಿಯದು. ಆದರೆ ತಿಳಿದ ಸ್ವಲ್ಪದರಿಂದಲೇ ಅವರು ಹೃತ್ಪೂರ್ವಕವಾಗಿ ನಮ್ಮ ಭರತಖಂಡದ ಹಿತೈಷಿಯಾಗಿದ್ದಾರೆ ನಮ್ಮ ಮಾತೃಭೂಮಿಯ ಉದ್ಧಾರಕ್ಕೆ ತಮ್ಮ ಕೈಲಾದುದನ್ನೆಲ್ಲಾ ಅವರು ಮಾಡುತ್ತಿರುವರು ಎಂದು ಹೇಳುವೆನು. ಅದಕ್ಕಾಗಿ ಪ್ರತಿಯೊಬ್ಬ ಭಾರತೀಯನೂ ಅವರಿಗೆ ಕೃತಜ್ಞನು. ಈಶ್ವರನ ಆಶೀರ್ವಾದ ಸದಾ ಅವರ ಮೇಲೆ ಮಳೆಗರೆಯಲಿ. ಇದು ಬೇರೆ, ಥಿಯಾಸಫಿ ಸಂಘವನ್ನು ಸೇರುವುದು ಬೇರೆ. ಅಭಿಮಾನ, ಗೌರವ, ಪ್ರೀತಿ ಬೇರೆ, ಹೇಳಿದುದನ್ನೆಲ್ಲ ವಿವೇಚನೆ ಇಲ್ಲದೆ, ವಿಮರ್ಶೆ ಇಲ್ಲದೆ, ವಿಶ್ಲೇಷಣೆ ಇಲ್ಲದೆ ಒಪ್ಪಿಕೊಳ್ಳುವುದು ಬೇರೆ. ಥಿಯಾಸಫಿಸ್ಟರು ನನಗೆ ಅಮೆರಿಕಾ ಯೂರೋಪ್​ ದೇಶಗಳಲ್ಲಿ ದೊರೆತ ಜಯಕ್ಕೆ ಕಾರಣರಾದರೆಂದು ವದಂತಿ ಹಬ್ಬಿರುತ್ತದೆ. ಆದರೆ ಆ ಮಾತೆಲ್ಲ ತಪ್ಪು, ಅಸತ್ಯ ಎಂದು ಹೇಳಬೇಕಾಗಿದೆ. ಉದಾರಭಾವನೆ, ಭಿನ್ನಾಭಿಪ್ರಾಯಗಳಿಗೆ ಮನ್ನಣೆ ಕೊಡುವುದು, ಎಂಬ ಹಲವು ಮಾತುಗಳನ್ನು ನಾವು ಕೇಳುವೆವು. ಅದೇನೊ ಒಳ್ಳೆಯದೆ. ಆದರೆ ನಿಜಾಂಶ ಬೇರೆ. ನಾನು ಹೇಳುವುದನ್ನೆಲ್ಲ ಎಲ್ಲಿಯವರೆವಿಗೂ ಮತ್ತೊಬ್ಬನು ನಂಬುತ್ತಾನೆಯೊ ಅಲ್ಲಿಯವರೆಗೆ ಮಾತ್ರ ನಮ್ಮ ಸಹಾನುಭೂತಿ. ಆದರೆ ಯಾವಾಗ ಅವನು ನಂಬುವುದಿಲ್ಲವೋ ಆಗ ಸಹೃದಯತೆ, ಪ್ರೇಮಗಳು ಮಾಯವಾಗುವುವು. ಸ್ವಾರ್ಥಾಕಾಂಕ್ಷಿಗಳಾದ ಮತ್ತೆ ಕೆಲವರು ಇರುವರು. ದೇಶದಲ್ಲಿ ಯಾವುದಾದರೂ ಘಟನೆ ತಮಗೆ ಅಡ್ಡಿಯಾದರೆ ಅವರಲ್ಲಿ ಅಸೂಯಾಜ್ವಾಲೆ ಏಳುವುದು. ಅದಕ್ಕೆ ಮಿತಿ ಇಲ್ಲ. ಏನು ಮಾಡಬೇಕೊ ಅವರಿಗೆ ತಿಳಿಯುವುದಿಲ್ಲ. ಹಿಂದೂಗಳು ತಮ್ಮ ಧರ್ಮದ ಕಳೆಯನ್ನು ಕೀಳಲು ಪ್ರಯತ್ನಪಟ್ಟರೆ ಇದರಿಂದ ಕ್ರೈಸ್ತರಿಗೆ ಏನು ತೊಂದರೆ? ಹಿಂದೂಗಳು ತಮ್ಮ ಸಮಾಜವನ್ನು ಸುಧಾರಣೆಗೊಳಿಸಲು ಪ್ರಯತ್ನಪಟ್ಟರೆ ಬ್ರಹ್ಮ ಸಮಾಜ ಮತ್ತು ಇತರೆ ಸುಧಾರಕರಿಗೆ ಏನು ತೊಂದರೆ? ಅವರೇಕೆ ವಿರೋಧಿಸಬೇಕು? ಅವರೇಕೆ ಈ ಚಟುವಟಿಕೆಗಳಿಗೆ ಪರಮ ಶತ್ರುಗಳಾಗಬೇಕು? ಏಕೆ ಎಂದು ಕೇಳುತ್ತೇನೆ. ಅವರ ಅಸೂಯೆ ದ್ವೇಷ ಅಷ್ಟು ಕಟುವಾಗಿದೆ, ಏತಕ್ಕೆ ಹೀಗೆ ಎಂದು ಕೇಳಲೂ ಆಗುವುದಿಲ್ಲ.

ನಾಲ್ಕು ವರ್ಷಗಳ ಹಿಂದೆ ನಾನು ಭಿಕಾರಿಯಾಗಿ, ಅಜ್ಞಾತನಾಗಿ, ಸ್ನೇಹಿತನಿಲ್ಲದ ಸಂನ್ಯಾಸಿಯಾಗಿದ್ದಾಗ, ಕಡಲನ್ನು ದಾಟಿ ಅಮೇರಿಕಾ ದೇಶಕ್ಕೆ, ಯಾವ ಪರಿಚಯ ಪತ್ರವೂ ಇಲ್ಲದೆ, ಸ್ನೇಹಿತರೂ ಇಲ್ಲದೆ ಹೋಗುತ್ತಿದ್ದಾಗ, ಥಿಯಾಸಫಿಕಲ್​ ಸೊಸೈಟಿಯ ಅಧ್ಯಕ್ಷರನ್ನು ಕಂಡೆ. ಅವರು ಅಮೆರಿಕಾ ದೇಶದವರಾಗಿದ್ದುದರಿಂದ, ಭರತ ಖಂಡದ ಹಿತೈಷಿಗಳಾದುದರಿಂದ ಅಮೆರಿಕಾ ದೇಶದಲ್ಲಿ ಯಾರಿಗಾದರೂ ಒಂದು ಪರಿಚಯಪತ್ರವನ್ನು ಕೊಡುವರು ಎಂದು ಊಹಿಸಿದ್ದೆ. ಅವರು, “ನೀನು ನಮ್ಮ ಸಂಘಕ್ಕೆ ಸೇರುವೆಯಾ?” ಎಂದು ಕೇಳಿದರು. ನಾನು, “ಇಲ್ಲ, ಹೇಗೆ ಸೇರಲಿ? ನಿಮ್ಮ ಸಿದ್ಧಾಂತಗಳನ್ನೆಲ್ಲಾ ನಾನು ಒಪ್ಪುವುದಿಲ್ಲ” ಎಂದೆ. ಅವರು, “ಹಾಗಾದರೆ ನಿನಗೇನೂ ನಾನು ಮಾಡಲಾರೆ, ಕ್ಷಮಿಸು” ಎಂದರು. ಇದು ನನಗೆ ಸಹಾಯಮಾಡಿದಂತೆ ಆಗಲಿಲ್ಲ. ನಾನು ನಿಮಗೆ ತಿಳಿದಿರುವಂತೆ ಮದ್ರಾಸಿನ ಕೆಲವು ಸ್ನೇಹಿತರ ಸಹಾಯದಿಂದ ಅಮೆರಿಕಾವನ್ನು ಸೇರಿದೆ. ಅವರಲ್ಲಿ ಬಹುಪಾಲು ಜನರು ಇಲ್ಲಿರುವರು. ಅವರಲ್ಲಿ ಒಬ್ಬರು ಮಾತ್ರ ಇಲ್ಲ. ಅವರೇ ನ್ಯಾಯಮೂರ್ತಿಗಳಾದ ಸುಬ್ರಹ್ಮಣ್ಯ ಅಯ್ಯರ್​. ಅವರು ನನ್ನ ಚಿರಕೃತಜ್ಞತೆಗೆ ಪಾತ್ರರು. ಅವರಲ್ಲಿ ಮಹಾವ್ಯಕ್ತಿಗೆ ಸಲ್ಲುವ ಮುಂದಾಲೋಚನೆ ಇದೆ. ನನ್ನ ಜೀವನದ ಅತಿ ನಿಕಟ ಸ್ನೇಹಿತರಲ್ಲಿ ಒಬ್ಬರು ಅವರು. ನಿಜವಾದ ಸ್ನೇಹಿತರು, ಭರತಮಾತೆಯ ಪುತ್ರರು. ವಿಶ್ವಧರ್ಮಸಮ್ಮೇಳನ ಪ್ರಾರಂಭವಾಗುವುದಕ್ಕೆ ಕೆಲವು ತಿಂಗಳುಗಳ ಮುಂಚೆ ಅಮೆರಿಕಾ ದೇಶವನ್ನು ಸೇರಿದೆ. ನನ್ನಲ್ಲಿದ್ದ ಹಣ ಅಲ್ಪ; ಅದು ಬೇಗ ಮುಗಿಯಿತು. ಚಳಿಗಾಲ ಬಂತು. ಬೇಸಿಗೆ ಕಾಲದ ತೆಳು ಬಟ್ಟೆ ಮಾತ್ರ ನನ್ನಲ್ಲಿ ಇತ್ತು. ಆ ಚಳಿದೇಶದಲ್ಲಿ ಏನು ಮಾಡಬೇಕೊ ತಿಳಿಯಲಿಲ್ಲ. ನಾನೇನಾದರೂ ಭಿಕ್ಷೆ ಎತ್ತಲು ಹೋಗಿದ್ದರೆ ನನ್ನನ್ನು ಸೆರೆಮನೆಗೆ ಒಯ್ಯುತ್ತಿದ್ದರು. ಆಗ ಜೇಬಿನಲ್ಲಿ ಕೆಲವು ಡಾಲರು ಮಾತ್ರ ಉಳಿದಿದ್ದುವು. ಮದ್ರಾಸಿನ ನನ್ನ ಸ್ನೇಹಿತರಿಗೆ ತಂತಿಯನ್ನು ಕಳುಹಿಸಿದೆ. ಈ ವಾರ್ತೆ ಥಿಯಾಸಫಿಸ್ಟರಿಗೆ ಗೊತ್ತಾಗಿ, “ಈಗ ಈ ಪಿಶಾಚಿ ಸಾಯುತ್ತಿದೆ. ದೇವರು ನಮ್ಮನ್ನು ಈ ಪೀಡೆಯಿಂದ ತಪ್ಪಿಸಿದ” ಎಂದರು. ಇದೇ ಅವರು ನನಗೆ ಮಾಡಿದ ಸಹಾಯ! ಇದನ್ನು ನಾನು ವ್ಯಕ್ತಪಡಿಸುತ್ತಿರಲಿಲ್ಲ. ಆದರೆ ನನ್ನ ದೇಶ ಬಾಂಧವರು ಇಚ್ಛೆಪಟ್ಟ ಕಾರಣ ಹೇಳಬೇಕಾಯಿತು. ಮೂರು ವರ್ಷದಿಂದ ಈ ಸುದ್ದಿಯನ್ನು ಎತ್ತಲಿಲ್ಲ, ಮೌನವಾಗಿದ್ದೆ. ಇಂದು ಅದು ಹೊರಗೆ ಬಂದಿದೆ. ಇಷ್ಟು ಮಾತ್ರವಲ್ಲ. ವಿಶ್ವಧರ್ಮಸಮ್ಮೇಳನದಲ್ಲಿ ಕೆಲವು ಥಿಯಾಸಫಿಸ್ಟರನ್ನು ನೋಡಿದೆ. ಅವರ ಹತ್ತಿರ ಕಲೆತು ಮಾತನಾಡಲು ಪ್ರಯತ್ನಪಟ್ಟೆ. ಆದರೆ ಅವರ ಮೊಗದ ಮೇಲೆ, “ದೇವತೆಗಳೊಂದಿಗೆ ಇರಬಯಸುವ ಈ ಕ್ಷುದ್ರ ಕೀಟವಾವುದು!” ಎಂಬ ತಿರಸ್ಕಾರಭಾವ ಮುದ್ರಿತವಾದಂತೆ ಇತ್ತು. ವಿಶ್ವಧರ್ಮಸಮ್ಮೇಳನದಲ್ಲಿ ಕೀರ್ತಿ ಲಭಿಸಿದ ಮೇಲೆ ನಾನು ಬಹುವಾಗಿ ಶ್ರಮಪಡಬೇಕಾಯಿತು. ಆದರೆ ಥಿಯಾಸಫಿಸ್ಟರು ಎಲ್ಲಾ ಕಡೆಗಳಲ್ಲಿಯೂ ಆತಂಕವನ್ನು ತಂದೊಡ್ಡಲು ಪ್ರಯತ್ನಿಸಿದರು. ಯಾರಾದರೂ ಥಿಯಾಸಫಿಸ್ಟರು ನನ್ನ ಉಪನ್ಯಾಸಕ್ಕೆ ಹೋದರೆ ಅವರಿಗೆ ಸಂಘದಿಂದ ಬಹಿಷ್ಕಾರವಾಗುವುದೆಂದು ಹೆದರಿಸಿ\-ದರು. ಯಾರು ಥಿಯಾಸಫಿಯ ಅಂತರಂಗಕ್ಕೆ ಸೇರಿದವರೊ ಅವರು ಕುತುಮಿ ಮತ್ತು ಮೋರಿಯಾ ಇವರಿಂದಲೇ ಉಪದೇಶ ಪಡೆಯಬೇಕು. ಅವರ ಜೀವಂತ ಪ್ರತಿನಿಧಿಗಳೆ, ಮಿಸ್ಟರ್​ ಜಡ್ಜ್ ಮತ್ತು ಮಿಸಸ್​ ಅನಿಬೆಸಂಟ್​. ಅಂತರಂಗಕ್ಕೆ ಸೇರುವುದು ಎಂದರೆ ಸ್ವಾತಂತ್ರ್ಯಕ್ಕೆ ತಿಲತರ್ಪಣ ಕೊಡುವುದೆಂದರ್ಥ. ನಿಜವಾಗಿ ನಾನು ಹೀಗೆ ಯಾವುದನ್ನೂ ಮಾಡಲಾರೆ. ಇದನ್ನು ಯಾರಾದರೂ ಮಾಡಿದರೆ ಅವರನ್ನು ಹಿಂದೂಗಳೆಂದು ಕರೆಯಲಾರೆ. ನನಗೆ ಮಿಸ್ಟರ್​ ಜಡ್ಜ್‌ರ ಮೇಲೆ ಬಹಳ ಗೌರವವಿತ್ತು. ಆತ ಗುಣವಂತ, ಉದಾರಿ, ಸರಳ ಸ್ವಭಾವದವರು. ಥಿಯಾಸಫಿಸ್ಟರ ಯೋಗ್ಯತಮ ಪ್ರತಿನಿಧಿ. ಅವರಿಗೂ ಅನಿಬೆಸೆಂಟರಿಗೂ ತಮ್ಮ ತಮ್ಮ ಮಹಾತ್ಮನೇ ಸರಿ ಎಂಬ ವಿಷಯದಲ್ಲಿರುವ ಚರ್ಚೆಯನ್ನು ಕುರಿತು ದೂರಲು ನನಗೆ ಅಧಿಕಾರವಿಲ್ಲ. ಇದರಲ್ಲಿ ವಿಚಿತ್ರವೇನೆಂದರೆ ಇಬ್ಬರೂ ಒಬ್ಬ ಮಹಾತ್ಮನನ್ನೇ ಹೇಳುವರು. ನಿಜಾಂಶ ದೇವರಿಗೇ ಗೊತ್ತು. ಆತನೇ ನ್ಯಾಯಾಧಿಪತಿಯಾಗಿರುವಾಗ, ಎದುರಿಗೆ ಇರುವ ತಕ್ಕಡಿ ಒಂದೇ ಸಮನಾಗಿರುವಾಗ, ಯಾರಿಗೂ ಯಾರ ಪರವಾಗಿಯೂ ತೀರ್ಪನ್ನು ಕೊಡಲು ಅಧಿಕಾರವಿಲ್ಲ.

ಅಮೆರಿಕಾ ದೇಶದಲ್ಲೆಲ್ಲಾ ಅವರು ನನಗೆ ಸಹಾಯ ಮಾಡಿದ್ದು ಹೀಗೆ. ಅವರು ಕ್ರೈಸ್ತ ಪಾದ್ರಿಗಳ ಗುಂಪಿಗೆ ಸೇರಿ ನನ್ನನ್ನು ವಿರೋಧಿಸಿದರು. ಕ್ರೈಸ್ತ ಪಾದ್ರಿಗಳು ನನಗೆ ವಿರೋಧವಾಗಿ ಹೇಳದೆ ಇರುವ ಅಸತ್ಯವೆ ಇಲ್ಲ. ನಾನು ಅಪರಿಚಿತ ದೇಶದಲ್ಲಿ ಯಾವ ಸ್ನೇಹಿತರೂ ಇಲ್ಲದೆ ಇರುವಾಗ, ಪ್ರತಿ ಊರಿನಲ್ಲಿಯೂ ಅವರು ನನ್ನ ಶೀಲದ ಮೇಲೆ ದೋಷಾರೋಪಣೆ ಮಾಡಿದರು. ಪ್ರತಿಯೊಂದು ಮನೆಯಿಂದಲೂ ನನ್ನನ್ನು ಓಡಿಸುವುದಕ್ಕೆ ಪ್ರಯತ್ನಿಸಿದರು. ನನ್ನ ಪ್ರತಿಯೊಬ್ಬ ಸ್ನೇಹಿತನನ್ನೂ ವೈರಿಯನ್ನಾಗಿ ಮಾಡಲು ಪ್ರಯತ್ನಿಸಿದರು. ನನ್ನ ಅನ್ನವನ್ನು ಕಸಿಯಲು ಪ್ರಯತ್ನಿಸಿದರು. ಈ ಪ್ರಸಂಗದಲ್ಲಿ ನಮ್ಮ ದೇಶದವರೊಬ್ಬರು ಅವರೊಡನೆ ಭಾಗಿಯಾಗಿದ್ದರು ಎನ್ನುವುದಕ್ಕೆ ದುಃಖವಾಗುವುದು. ಆತ ಹಿಂದೂ ಸುಧಾರಕ ಸಂಘದ ಮುಂದಾಳು. ಪ್ರತಿದಿನ ‘ಕ್ರಿಸ್ತ ಭರತಖಂಡಕ್ಕೆ ಬಂದನು’ ಎಂದು ಬೋಧಿಸುತ್ತಿರುವನು. ಕ್ರಿಸ್ತ ಭರತಖಂಡಕ್ಕೆ ಹೀಗೆಯೇ ಬರುವುದು? ಭರತಖಂಡವನ್ನು ಸುಧಾರಿಸುವುದು ಹೀಗೆಯೇ? ಈ ಭದ್ರ ಮನುಷ್ಯ ಬಾಲ್ಯದಿಂದಲೂ ನನಗೆ ಪರಿಚಿತ. ನನ್ನ ಆಪ್ತ ಮಿತ್ರರಲ್ಲಿ ಒಬ್ಬ. ನಾನು ಆತನನ್ನು ನೋಡಿದಾಗ ತುಂಬಾ ಸಂತೋಷವಾಯಿತು. ಹಲವು ಕಾಲದಿಂದ ನಮ್ಮ ದೇಶದವರನ್ನು ನಾನು ಕಂಡಿರಲಿಲ್ಲ. ಆದರೆ ಆತ ನನಗೆ ಮಾಡಿದ ಉಪಕಾರ ಇದು. ವಿಶ್ವಧರ್ಮ ಸಮ್ಮೇಳನ ನನ್ನನ್ನು ಕೊಂಡಾಡಲು ಪ್ರಾರಂಭಿಸಿದ ದಿನದಿಂದಲೇ, ಚಿಕಾಗೊ ನಗರದಲ್ಲಿ ನಾನು ಪ್ರಖ್ಯಾತನಾದ ದಿನದಿಂದಲೇ ಆತನ ಧ್ವನಿ ಬದಲಾಯಿತು. ಗೋಪ್ಯವಾಗಿ ನನ್ನ ಪತನಕ್ಕೆ ಸಾಧ್ಯವಾದುದನ್ನೆಲ್ಲ ಆತ ಮಾಡಿದ. ಕ್ರಿಸ್ತ ಭರತಖಂಡಕ್ಕೆ ಬರುವ ರೀತಿಯೆ ಇದು? ಇಪ್ಪತ್ತು ವರುಷ ಕ್ರಿಸ್ತನ ಅಡಿದಾವರೆಯಲ್ಲಿ ಕುಳಿತು ಇದನ್ನೇ ಆತ ಕಲಿತಿರುವುದು? ನಮ್ಮ ಪ್ರಖ್ಯಾತ ಸುಧಾರಕರು ಕ್ರೈಸ್ತಧರ್ಮ ಮತ್ತು ಕ್ರೈಸ್ತಶಕ್ತಿ ಭರತಖಂಡವನ್ನು ಉದ್ಧರಿಸುವುದೆಂದು ಸಾರುತ್ತಿರುವರು. ಇದೇ ಏನು ಅದನ್ನು ಸಾಧಿಸುವ ರೀತಿ? ನಿಜವಾಗಿ ಈತ ಅದಕ್ಕೆ ಉದಾಹರಣೆಯಾದರೆ, ಪರಿಸ್ಥಿತಿ ಆಶಾಜನಕವಾಗಿಲ್ಲ.

ಮತ್ತೊಂದು ವಿಷಯ. ಸಮಾಜ ಸುಧಾರಕರ ಒಂದು ಮಾಸ ಪತ್ರಿಕೆಯಲ್ಲಿ ನನ್ನನ್ನು ಶೂದ್ರನೆಂದು ಕರೆದು ಸಂನ್ಯಾಸಿಯಾಗಲು ನನಗೆ ಅಧಿಕಾರವಿಲ್ಲವೆಂದು ಹೇಳಿರುವರು. ಇದಕ್ಕೆ ಉತ್ತರವಾಗಿ ಬ್ರಾಹ್ಮಣರು “ಯಮಾಯ ಧರ್ಮರಾಜಾಯ ಚಿತ್ರಗುಪ್ತಾಯ ವೈ ನಮಃ” ಎಂದು ಉಚ್ಚರಿಸಿ ಯಾರ ಪಾದಗಳನ್ನು ಪೂಜಿಸುವರೊ ಅವರು ನನ್ನ ವಂಶದ ಮೂಲಪುರುಷರು; ಅವರು ಪರಿಶುದ್ಧ ಕ್ಷತ್ರಿಯರು ಎನ್ನುವೆನು. ನೀವು ನಿಮ್ಮ ಪುರಾಣ ಪುಣ್ಯಕಥೆಯನ್ನು ನಂಬಿದರೆ, ನನ್ನ ವರ್ಣವು ಮಾಡಿದ ಇತರ ಸೇವೆಯನ್ನು ಬಿಟ್ಟರೂ, ಅದು ಹಲವು ಶತಮಾನಗಳವರೆಗೆ ಅರ್ಧ ಭರತಖಂಡವನ್ನೇ ಆಳಿತೆಂಬುದು ಗೊತ್ತಾಗುವುದು. ನನ್ನ ವರ್ಣವನ್ನು ನೀವು ಬಿಟ್ಟರೆ ಭರತಖಂಡದ ಇಂದಿನ ಸಂಸ್ಕೃತಿಯಲ್ಲಿ ಉಳಿಯುವುದೇನು? ವಂಗಭೂಮಿಯಲ್ಲೇ ನಮ್ಮ ಕುಲ, ಶ್ರೇಷ್ಠ ತತ್ತ್ವಜ್ಞಾನಿಗಳನ್ನು, ಕವಿಗಳನ್ನು, ಇತಿಹಾಸಕಾರರನ್ನು, ಪ್ರಾಕ್ತನ ಶಾಸ್ತ್ರಜ್ಞರನ್ನು, ಪ್ರಖ್ಯಾತ ಧರ್ಮ ಬೋಧಕರನ್ನು ಕೊಟ್ಟಿರುವುದು. ಆಧುನಿಕ ಭರತಖಂಡದ ಪ್ರಖ್ಯಾತ ವಿಜ್ಞಾನಿ ನಮ್ಮ ಕುಲೋದ್ಭವನು. ಇದನ್ನು ಚರ್ಚಿಸುವವರಿಗೆ ನಮ್ಮ ದೇಶದ ಚರಿತ್ರೆ ಸ್ವಲ್ಪ ಗೊತ್ತಿರಬೇಕಾಗಿತ್ತು. ಮೂರು ವರ್ಣಗಳ ವಿಷಯ ಗೊತ್ತಿರಬೇಕಾಗಿತ್ತು. ಬ್ರಾಹ್ಮಣ ಕ್ಷತ್ರಿಯ ವೈಶ್ಯರೆಂಬ ತ್ರಿವರ್ಣದವರಿಗೂ ಸಂನ್ಯಾಸದ ಹಕ್ಕಿದೆ ಎಂಬುದು ಅವರಿಗೆ ಗೊತ್ತಿರಬೇಕಾಗಿತ್ತು. ಇದನ್ನು ಸುಮ್ಮನೆ ಹೇಳುತ್ತಿರುವೆನು. ಆದರೆ ನನ್ನನ್ನು ಶೂದ್ರನೆಂದು ಕರೆದುದರಿಂದ ನನಗೆ ವ್ಯಥೆಯಾಗುವುದಿಲ್ಲ. ನಮ್ಮ ಪೂರ್ವಿಕರು ದೀನರನ್ನು ನಿಕೃಷ್ಟ ದೃಷ್ಟಿಯಿಂದ ನೋಡಿದುದಕ್ಕೆ ಇದೊಂದು ಪ್ರಾಯಶ್ಚಿತ್ತವೆಂದು ತಿಳಿಯುತ್ತೇನೆ. ನಾನೊಬ್ಬ ಹೊಲೆಯನಾದರೂ ಪರಮ ಸಂತೋಷ. ಚಂಡಾಲನ ಮನೆಯನ್ನು ಗುಡಿಸಬಯಸಿದ ಬ್ರಾಹ್ಮಣ ಶ್ರೇಷ್ಠನ ಶಿಷ್ಯ ನಾನು. ಚಂಡಾಲ ಇದಕ್ಕೆ ಒಪ್ಪಿಗೆ ಕೊಡಲಿಲ್ಲ. ಬ್ರಾಹ್ಮಣ ಸಂನ್ಯಾಸಿಗೆ ಆತ ತನ್ನ ಮನೆಯನ್ನು ಗುಡಿಸಲು ಹೇಗೆ ಒಪ್ಪಿಗೆ ಕೊಡುವನು? ನನ್ನ ಗುರುಗಳು ಅರ್ಧರಾತ್ರಿ ಎದ್ದು ಚಂಡಾಲನ ಮನೆಯನ್ನು ಯಾರಿಗೂ ತಿಳಿಯದೆ ಪ್ರವೇಶಿಸಿ, ಅವನ ಕಕ್ಕಸನ್ನು ಗುಡಿಸಿ, ತಮ್ಮ ಕೇಶರಾಶಿಯಿಂದ ಆ ಸ್ಥಳವನ್ನು ಒರೆಸಿದರು. ಇದನ್ನು ಅವರು ಹಲವು ದಿನಗಳವರೆಗೆ ಮಾಡಿದರು. ಅವರ ದಾಸಾನುದಾಸನಾಗ ಬೇಕೆಂಬುದೇ ಅವರ ಇಚ್ಛೆ. ಅವರ ಪಾದಗಳನ್ನು ನಾನು ಶಿರದಲ್ಲಿ ಧರಿಸಿರುವೆ. ಅವರೇ ನನ್ನ ಆದರ್ಶ. ಆ ಗುರುಗಳ ಜೀವನವನ್ನು ಅನುಕರಿಸಲು ಪ್ರಯತ್ನಿಸುತ್ತೇನೆ. ಎಲ್ಲರ ಸೇವಕನಾಗಿ ಹಿಂದೂವು ತಾನು ಉತ್ತಮನಾಗಲು ಪ್ರಯತ್ನಿಸುವನು. ಹಿಂದೂಗಳು ಜನಸಾಧಾರಣರನ್ನು ಉದ್ಧರಿಸುವುದು ಹೀಗೆ, ಬಾಹ್ಯ ಸಹಾಯವನ್ನು ನಿರೀಕ್ಷಿಸುವುದರಿಂದ ಅಲ್ಲ. ಭಾರತೀಯನು ತನ್ನ ಸ್ನೇಹಿತನನ್ನು ಪರದೇಶದಲ್ಲಿ ಉಪವಾಸವಿರುವಂತೆ ಮಾಡಿರುವುದು ಇಪ್ಪತ್ತು ವರುಷಗಳ ಪಾಶ್ಚಾತ್ಯ ಸಂಸ್ಕೃತಿಯ ಪರಿಣಾಮ. ಇದಕ್ಕೆ ಕಾರಣ ಆತನಿಗಿಂತ ನಾನು ಹೆಚ್ಚು ಜನಾದರಣೀಯನಾಗಿದ್ದು; ದುಡ್ಡು ಮಾಡುವುದಕ್ಕೆ ನಾನು ಅವನಿಗೆ ಆತಂಕವೆಂದು ಅವನು ಭಾವಿಸಿದುದು. ಮತ್ತೊಂದೇ ಶುದ್ಧ ಸನಾತನ ಹಿಂದೂಗಳು ತಮ್ಮ ದೇಶದಲ್ಲಿಯೇ ಏನು ಮಾಡುತ್ತಾರೆ ಎಂಬುದಕ್ಕೆ ಉದಾಹರಣೆ. ನಮ್ಮ ಸುಧಾರಕರು ಚಂಡಾಲನಿಗೆ ಕೂಡ ಸೇವೆಯನ್ನು ಸಲ್ಲಿಸುವಂತಹ ಬದುಕನ್ನು ಬಾಳಲಿ. ಆಗ ನಾನು ಅವರ ಪದತಳದಲ್ಲಿ ಕುಳಿತು ಕಲಿತುಕೊಳ್ಳುತ್ತೇನೆ. ಅದಕ್ಕೆ ಮುಂಚೆ ಅಲ್ಲ. ಎಳ್ಳಿನಷ್ಟು ಅನುಷ್ಠಾನ, ರಾಶಿರಾಶಿ ಮಾತಿಗಿಂತ ಮೇಲು.

ಈಗ ಮದ್ರಾಸಿನ ಸಮಾಜಸುಧಾರಕರ ವಿಷಯಕ್ಕೆ ಬರುತ್ತೇನೆ. ಅವರು ನನ್ನನ್ನು ದಯೆಯಿಂದ ಕಂಡಿರುವರು. ಅವರು ನನ್ನನ್ನು ಶ್ಲಾಘಿಸಿರುವರು. ಮದ್ರಾಸಿನ ಸುಧಾರಕರಿಗೂ ಬಂಗಾಳದ ಸುಧಾರಕರಿಗೂ ವ್ಯತ್ಯಾಸವಿದೆ ಎಂದು ಅವರು ಹೇಳುವುದನ್ನು ನಾನು ಒಪ್ಪುತ್ತೇನೆ. ನಾನು ಹಲವು ಬಾರಿ ಹೇಳಿರುವಂತೆ ಮದ್ರಾಸು ಈಗ ಒಳ್ಳೆಯ ಸ್ಥಿತಿಯಲ್ಲಿದೆ ಎಂಬುದು ಎಲ್ಲರಿಗೂ ಗೊತ್ತಿರಬಹುದು. ಬಂಗಾಳದಂತೆ ಇದು ಕ್ರಿಯೆ ಮತ್ತು ಪ್ರತಿಕ್ರಿಯೆಯ ಘರ್ಷಣಕ್ಕೆ ಸಿಕ್ಕಿಲ್ಲ. ಇಲ್ಲಿ ಮೊದಲಿನಿಂದಲೂ ನಿಧಾನವಾದ ಬೆಳವಣಿಗೆ ಇದೆ. ಇಲ್ಲಿ ಪ್ರತಿಕ್ರಿಯೆಯಲ್ಲ ಇರುವುದು, ಬೆಳವಣಿಗೆ. ವಂಗದೇಶದಲ್ಲಿ ಹಲವು ಕಡೆ ಒಂದು ನವಜಾಗೃತಿ ಆಗಿದೆ. ಆದರೆ ಮದ್ರಾಸಿನಲ್ಲಿರುವುದು ಸ್ವಾಭಾವಿಕ ಬೆಳವಣಿಗೆ. ಸುಧಾರಕರು ಹೇಳುವಂತೆ ಎರಡು ಜನಾಂಗಗಳಿಗೂ ಇರುವ ವ್ಯತ್ಯಾಸ ಇದು ಎಂಬುದನ್ನು ನಾನು ಒಪ್ಪಿಕೊಳ್ಳುವೆನು. ಆದರೆ ಅವರಿಗೆ ತಿಳಿಯದ ಒಂದು ವ್ಯತ್ಯಾಸವಿದೆ. ಇಂತಹ ಕೆಲವು ಸಮಾಜಗಳು ನನ್ನನ್ನು ತಮ್ಮ ಸಂಘಕ್ಕೆ ಸೇರುವಂತೆ ಹೆದರಿಸುವುವು ಎಂದು ತೋರುವುದು. ಆದರೆ ಹೀಗೆ ಮಾಡುವುದೊಂದು ವಿಚಿತ್ರ. ಹದಿನಾಲ್ಕು ವರುಷಗಳಿಂದ ಉಪವಾಸವನ್ನು ಎದುರಿಸಿದವನನ್ನು, ಮಾರನೆಯ ದಿನ ಎಲ್ಲಿ ಊಟ ಸಿಕ್ಕುವುದು, ಎಲ್ಲಿ ಮಲಗುವುದು ಎಂದು ಗೊತ್ತಿಲ್ಲದವನನ್ನು, ಇಷ್ಟು ಸುಲಭವಾಗಿ ಅಂಜಿಸುವುದಕ್ಕೆ ಆಗುವುದಿಲ್ಲ. ಶಾಖವಾದ ಬಟ್ಟೆ ಇಲ್ಲದೆ, ಥರ್ಮಾಮೀಟರ್​ ಸೊನ್ನೆಯ ಕೆಳಗೆ ಮೂವತ್ತು ಡಿಗ್ರಿ ತೋರಿಸುತ್ತಿದ್ದ ಕಡೆ ಬಾಳಿದವನನ್ನು, ಮತ್ತೊಂದು ಹೊತ್ತಿನ ಊಟ ಎಲ್ಲಿ ಸಿಗುವುದೆಂದು ಗೊತ್ತಿಲ್ಲದವನನ್ನು, ಇಂಡಿಯಾ ದೇಶದಲ್ಲಿ ಅಷ್ಟು ಸುಲಭವಾಗಿ ಅಂಜಿಸುವುದಕ್ಕೆ ಆಗುವುದಿಲ್ಲ. ನಾನು ಅವರಿಗೆ ಹೇಳುವ ಮೊದಲನೆಯ ಮಾತು ಇದು: ನನ್ನ ಸ್ವಂತ ಇಚ್ಛೆಯೊಂದು ಇದೆ. ಸ್ವಲ್ಪ ಅನುಭವವೂ ಇದೆ. ಜಗತ್ತಿಗೆ ಕೊಡಬೇಕಾದ ಒಂದು ಸಂದೇಶವೂ ಇದೆ. ಅದನ್ನು ದಯಾದಾಕ್ಷಿಣ್ಯವಿಲ್ಲದೆ, ಮುಂದೆ ಏನಾಗಬಹುದು ಎಂಬುದನ್ನು ಗಮನಿಸದೇ, ಸಾರುತ್ತೇನೆ. ಸುಧಾರಕರಿಗೆ, ನಾನು ಅವರಿಗಿಂತ ದೊಡ್ಡ ಸುಧಾರಕ ಎಂದು ಹೇಳುತ್ತೇನೆ. ಅವರು ಚೂರುಪಾರುಗಳನ್ನು ಸುಧಾರಿಸಲು ಯತ್ನಿಸುವರು. ನನಗೆ ಆಮೂಲಾಗ್ರ ಸುಧಾರಣೆ ಬೇಕು. ಮಾರ್ಗವನ್ನು ಕುರಿತು ಮಾತ್ರ ನಮ್ಮಲ್ಲಿ ವ್ಯತ್ಯಾಸ. ಅವರದು ಧ್ವಂಸಮಾರ್ಗ. ನನ್ನದು ನಿರ್ಮಾಣ ಮಾರ್ಗ. ನನಗೆ ಸುಧಾರಣೆಯಲ್ಲಿ ನಂಬಿಕೆಯಿಲ್ಲ. ನನಗೆ ಬೆಳವಣಿಗೆಯಲ್ಲಿ ನಂಬಿಕೆಯುಂಟು. ನಾನು ದೇವರ ಸ್ಥಾನದಲ್ಲಿ ನಿಂತು ಸಮಾಜಕ್ಕೆ “ಹೀಗೆ ಚಲಿಸಬೇಕು” ಎಂದು ಅಪ್ಪಣೆ ಮಾಡಲಾರೆ. ರಾಮನು ಸೇತುವೆಯನ್ನು ಕಟ್ಟುವಾಗ ಇದ್ದ ಅಳಿಲಿನಂತೆ ನಾನು ಇರಬೇಕೆಂದು ಬಯಸುವೆನು. ಅದು ತನ್ನ ಪಾಲಿಗೆ ಬಂದ ಮರಳನ್ನು ಸೇತುವೆಗೆ ಕೊಡಹಿ ಧನ್ಯವಾಗುತ್ತಿತ್ತು. ಅದು ನನ್ನ ಸ್ಥಿತಿ. ಈ ರಾಷ್ಟ್ರೀಯ ಮಹಾ ಯಂತ್ರವು ಶತಮಾನಗಳಿಂದ ಕೆಲಸ ಮಾಡಿದೆ. ರಾಷ್ಟ್ರಜೀವನದ ಮಹಾ ನದಿಯು ನಮ್ಮ ಕಣ್ಣೆದುರಿಗೆ ಹರಿಯುತ್ತಿದೆ. ಇದು ಒಳ್ಳೆಯದೆ. ಇದರ ಗತಿ ಯಾವುದು ಎಂದು ಯಾರಿಗೆ ಗೊತ್ತಿದೆ? ಇದನ್ನು ಹೇಳುವುದಕ್ಕೆ ಯಾರಿಗೆ ಎದೆಗಾರಿಕೆ ಇದೆ? ಸಹಸ್ರಾರು ಸನ್ನಿವೇಶಗಳು ಇದರಲ್ಲಿ ಮಿಲನವಾಗಿ ಕೆಲವು ವೇಳೆ ಮಂದವಾಗಿ, ಕೆಲವು ವೇಳೆ ವೇಗವಾಗಿ ಚಲಿಸುವಂತೆ ಮಾಡುತ್ತಿವೆ. ಅದರಗತಿಯನ್ನು ನಿರ್ಧರಿಸುವ ಧೈರ್ಯ ಯಾರಿಗಿದೆ? ಗೀತೆಯು ಹೇಳುವಂತೆ ನಮ್ಮ ಕರ್ತವ್ಯ, ಫಲಾಪೇಕ್ಷೆ ಇಲ್ಲದೆ ಕರ್ಮ ಮಾಡುವುದು. ಜನಜೀವನಕ್ಕೆ ಬೇಕಾದ ಸಾಮಗ್ರಿ ಒದಗಿದರೆ, ಬೆಳವಣಿಗೆ ಅದರ ನಿಯಮಾನುಸಾರ ಆಗುವುದು. ಹೇಗೆ ಬೆಳೆಯಬೇಕೆಂದು ಯಾರೂ ಅದನ್ನು ಬಲಾತ್ಕರಿಸ\-ಲಾರರು. ನಮ್ಮ ಸಮಾಜದಲ್ಲಿ ಬೇಕಾದಷ್ಟು ಲೋಪದೋಷಗಳಿವೆ. ಹಾಗೆಯೇ ಎಲ್ಲಾ ಸಮಾಜಗಳಲ್ಲಿಯೂ ಇವೆ. ಇಲ್ಲಿ ವಿಧವೆಯರ ಕಂಬನಿಗಳಿಂದ ಕೆಲವು ವೇಳೆ ಭೂಮಿಯು ತೊಯ್ದಿರುವುದು. ಅಲ್ಲಿ ಪಾಶ್ಚಾತ್ಯರಲ್ಲಿ, ಮದುವೆಯಾಗದ ಹೆಂಗಸರ ಬೇಗುದಿಯಿಂದ ಆಕಾಶವು ತುಂಬಿದೆ. ದಾರಿದ್ರ್ಯದ ಮಹಾದುಃಖವು ಇಲ್ಲಿ, ವಿಲಾಸೋನ್ಮಾದದ ಜೀವನದಿಂದ ಬರುವ ಜುಗುಪ್ಸೆ ಆ ಜನಾಂಗದಲ್ಲಿ. ಇಲ್ಲಿ ತಿನ್ನಲೂ ಏನೂ ಸಿಕ್ಕದೆ ಜನರು ಆತ್ಮಹತ್ಯೆ ಮಾಡಿಕೊಳ್ಳಬೇಕೆನ್ನುವರು. ಅಲ್ಲಿ ತಿನ್ನಲು ಜನರಿಗೆ ಯಥೇಚ್ಛವಾಗಿರುವುದರಿಂದ ಆತ್ಮಹತ್ಯೆ ಮಾಡಿಕೊಳ್ಳುವರು. ದೋಷವು ಎಲ್ಲಾ ಕಡೆಗಳಲ್ಲಿಯೂ ಇದೆ. ಅದೊಂದು ಬಹಳ ಕಾಲದಿಂದ ಇರುವ ವಾತರೋಗದಂತೆ; ಕಾಲಿನಿಂದ ಓಡಿಸಿದರೆ ತಲೆಗೆ ಹೋಗುವುದು, ಅಲ್ಲಿಂದ ಓಡಿಸಿದರೆ ಮತ್ತೆಲ್ಲೋ ಓಡುವುದು. ಅದರ ಸ್ಥಳದಿಂದ ಸ್ಥಳಕ್ಕೆ ಓಡುವುದು ಅಷ್ಟೆ. ಮಕ್ಕಳೇ, ಕೇಳಿ, ದೋಷವನ್ನೆಲ್ಲಾ ನಿವಾರಣೆ ಮಾಡುವುದಕ್ಕೆ ಪ್ರಯತ್ನ ಮಾಡುವುದು ಸರಿಯಾದ ಮಾರ್ಗವಲ್ಲ. ಒಳ್ಳೆಯದು ಕೆಟ್ಟದ್ದು ಎರಡೂ ಒಂದೇ ನಾಣ್ಯದ ಎರಡು ಮುಖಗಳಂತೆ ಎಂದು ನಮ್ಮ ತತ್ತ್ವಶಾಸ್ತ್ರಗಳು ಹೇಳುತ್ತವೆ. ನಿಮಗೆ ಒಂದು ಬೇಕಾದರೆ ಮತ್ತೊಂದನ್ನು ಸ್ವೀಕರಿಸಬೇಕು. ಸಮುದ್ರದ ಒಂದು ಕಡೆ ಉಬ್ಬರವಿದ್ದರೆ ಮತ್ತೊಂದು ಕಡೆ ಇಳಿತವಿರಲೇಬೇಕು. ಜೀವನವೆಲ್ಲ ದೋಷಮಯ. ಬೇರೊಂದನ್ನು ಕೊಲ್ಲದೆ ಉಸಿರನ್ನು ಎಳೆಯಲಾರೆವು. ಮತ್ತೊಂದು ಬಾಯಿಯಿಂದ ಕಿತ್ತುಕೊಳ್ಳದೆ ನಾವು ಒಂದು ತುತ್ತನ್ನೂ ಊಟ ಮಾಡಲಾರೆವು. ಇದೇ ನಿಯಮ, ಇದೇ ತತ್ತ್ವ. ನಾವು ತಿಳಿದುಕೊಳ್ಳಬೇಕಾದ ಏಕಮಾತ್ರ ವಿಷಯವೆಂದರೆ, ದೋಷಕ್ಕೆ ವಿರುದ್ಧವಾಗಿ ಮಾಡುವ ಕೆಲಸವೆಲ್ಲ ವ್ಯಕ್ತಿ ದೃಷ್ಟಿಯಿಂದ ಮಾತ್ರ, ವಸ್ತು ದೃಷ್ಟಿಯಿಂದಲ್ಲ. ದೋಷಕ್ಕೆ ವಿರುದ್ಧವಾಗಿ ನಾವು ಎಷ್ಟು ಮಾತನಾಡಿದರೂ ಅದು ಕೇವಲ ವ್ಯಕ್ತಿ ಶಿಕ್ಷಣದ ದೃಷ್ಟಿಯಿಂದ ಮಾತ್ರ. ದೋಷವನ್ನು ವಿರೋಧಿಸಿ ಕೆಲಸ ಮಾಡುವುದು ಎಂದರೆ ಇದು. ಇದು ನಮ್ಮ ಉದ್ವಿಗ್ನತೆಯನ್ನು ಪರಿಹರಿಸಿ ಮತಭ್ರಾಂತಿಯನ್ನು ಹೋಗಲಾಡಿಸಬೇಕು. ಎಲ್ಲಿ ಮತಭ್ರಾಂತರಾದ ಸುಧಾರಕರಿದ್ದಾರೋ ಅವರೆಲ್ಲಾ ತಮ್ಮ ಗುರಿಯನ್ನೇ ಕಳೆದು\-ಕೊಂಡರು ಎಂಬುದನ್ನು ಚರಿತ್ರೆಯು ತೋರಿಸಿದೆ. ಅಮೆರಿಕಾ ದೇಶದಲ್ಲಿ ಗುಲಾಮಗಿರಿಯನ್ನು ಕೊನೆಗಾಣಿಸಿ ನೀಗ್ರೋ ಜನರಿಗೆ ಸ್ವಾತಂತ್ರ್ಯವನ್ನು ಮತ್ತು ಅವರ ಹಕ್ಕನ್ನು ಅವರಿಗೆ ಕೊಡುವುದಕ್ಕೆ ಮಾಡಿದ ಸಾಹಸದ ಮಹಾಸಮಾಜ ಸುಧಾರಣೆಗೆ ಮಿಗಿಲಾದುದನ್ನು ನಾವು ಆಲೋಚಿಸಲಾರೆವು. ನಿಮಗೆ ಆ ವಿಷಯವೆಲ್ಲ ತಿಳಿದಿದೆ. ಇದರ ಫಲವೇನಾಯಿತು? ಗುಲಾಮಗಿರಿ ರದ್ದಾದ ಮೇಲೆ ಅವರ ಸ್ಥಿತಿ ಹಿಂದಿಗಿಂತ ನೂರುಪಾಲು ಕಠಿಣವಾಗಿದೆ.\break ರದ್ದಾಗುವುದಕ್ಕೆ ಮುಂಚೆ ನೀಗ್ರೋಗಳು ಮತ್ತೊಬ್ಬರ ಅಧೀನದಲ್ಲಿದ್ದರು. ಆದಕಾರಣ ಅವರು ಸಾಯದಂತೆ ಅವರನ್ನು ನೋಡಿಕೊಳ್ಳಬೇಕಾಗಿತ್ತು. ಇಂದು ಅವರು ಯಾರಿಗೂ ಸೇರಿಲ್ಲ. ಅವರ ಜೀವಕ್ಕೆ ಬೆಲೆಯೇ ಇಲ್ಲ. ಯಾವುದಾದರೊಂದು ಆರೋಪಣೆಯ ಮೇಲೆ ಅವರನ್ನು ಜೀವಸಹಿತ ಸುಡುವರು. ಅವರನ್ನು ಗುಂಡಿಕ್ಕಿ ಕೊಲ್ಲುವರು. ಕೊಂದವರನ್ನು ಹಿಡಿಯು\-ವುದಕ್ಕೆ ಯಾವ ಕಾನೂನೂ ಇಲ್ಲ. ಏಕೆಂದರೆ ಅವರು ನೀಗ್ರೋಗಳು; ಮನುಷ್ಯರೂ\break ಅಲ್ಲ, ಮೃಗಗಳೂ ಅಲ್ಲ. ಮತಭ್ರಾಂತಿ ಅಥವಾ ಕಾನೂನಿನಿಂದ ದೋಷ ನಿವಾರಣೆ ಮಾಡಲು ಪ್ರಯತ್ನಪಟ್ಟ ಪರಿಣಾಮ ಇದು. ಮತಭ್ರಾಂತಿಯ ಪ್ರೇರಣೆಯಿಂದ ಆದ ಎಲ್ಲಾ ಒಳ್ಳೆಯ ಕಾರ್ಯಗಳ ಪರಿಣಾಮ ಅಷ್ಟೇ ಎಂದು ಚರಿತ್ರೆ ತೋರಿಸಿದೆ. ನಾನು ಇದನ್ನು ಕಣ್ಣಾರೆ ನೋಡಿರುವೆನು. ನನ್ನ ಸ್ವಂತ ಅನುಭವ ಅದನ್ನು ಕಲಿಸಿರುವುದು. ಆದಕಾರಣವೆ ದೂರುವ ಯಾವ ಸಂಸ್ಥೆಗೂ ನಾನು ಸೇರುವುದಿಲ್ಲ. ಏತಕ್ಕೆ ದೂರಬೇಕು? ಎಲ್ಲಾ ಸಮಾಜಗಳಲ್ಲೂ ಲೋಪದೋಷಗಳಿವೆ. ಅದು ಎಲ್ಲರಿಗೂ ಗೊತ್ತು. ಇಂದಿನ ಪ್ರತಿಯೊಂದು ಮಗುವಿಗೂ ಅದು ಗೊತ್ತು. ಮಗುವು ವೇದಿಕೆಯ ಮೇಲೆ ನಿಂತು ಹಿಂದೂ ಸಮಾಜದ ನ್ಯೂನಾತಿರೇಕದ ಮೇಲೆ ಒಂದು ದೊಡ್ಡ ಉಪನ್ಯಾಸವನ್ನೇ ಕೊಡಬಲ್ಲದು. ಜಗತ್ತನ್ನು ಸುತ್ತುವ ಪ್ರತಿಯೊಬ್ಬ ಅವಿದ್ಯಾವಂತ ವಿದೇಶೀಯನೂ ಭರತಖಂಡಕ್ಕೆ ಬಂದು, ರೈಲಿನಲ್ಲಿ ಹೋಗುತ್ತಿರುವಾಗ ಈ ದೇಶವನ್ನು ನೋಡಿ ಇಲ್ಲಿನ ಭಯಾನಕ ದೋಷಗಳ ಮೇಲೆ ಬಹಳ ಪಾಂಡಿತ್ಯದಿಂದ ಕೂಡಿದ ಉಪನ್ಯಾಸವನ್ನೇ ಕೊಡುವನು. ನಮ್ಮಲ್ಲಿ ದೋಷಗಳಿವೆ ಎಂದು ನಾವು ಒಪ್ಪಿಕೊಳ್ಳುತ್ತೇವೆ. ಎಲ್ಲರೂ ದೋಷವನ್ನೇ ತೋರಬಹುದು. ಆದರೆ ಯಾರು ಅದರಿಂದ ಪಾರಾಗಲು ಮಾರ್ಗವನ್ನು ತೋರಬಲ್ಲರೋ ಅವರೇ ಮಾನವ ಬಂಧುಗಳು. ಮುಳುಗುತ್ತಿರುವ ಹುಡುಗ ಹಾಗೂ ತತ್ತ್ವಜ್ಞಾನಿಯಂತೆ-ಹುಡುಗ ನೀರಿನಲ್ಲಿ ಮುಳುಗುತ್ತಿದ್ದಾಗ ತತ್ತ್ವಜ್ಞಾನಿ ಅವನಿಗೆ ಬೋಧನೆ ಮಾಡುತ್ತಿದ್ದ. ಹುಡುಗ “ನನ್ನನ್ನು ಮೊದಲು ನೀರಿನಿಂದ ಮೇಲೆತ್ತಿ” ಎಂದು ಅರಚಿಕೊಂಡನು. ಹೀಗೆ ನಮ್ಮ ಜನರೂ ಕೂಗುತ್ತಿರುವರು. “ಉಪನ್ಯಾಸಗಳು ಬೇಕಾದಷ್ಟು ಆಗಿವೆ. ಸಮಾಜಗಳು, ಪತ್ರಿಕೆಗಳು ಬೇಕಾದಷ್ಟು ಇವೆ. ನಮ್ಮ ಉದ್ಧಾರಕ್ಕೆ ಕೈ ನೀಡುವವನೆಲ್ಲಿ? ಯಾರು ನಮ್ಮನ್ನು ನಿಜವಾಗಿ ಪ್ರೀತಿಸುವವರು? ನಿಜವಾಗಿ ನಮ್ಮ ಮೇಲೆ ಸಹಾನುಭೂತಿ ಯಾರಿಗೆ ಇದೆ?” ಅಯ್ಯೊ, ಮೊದಲು ಬೇಕಾಗಿರುವುದು ಸಹಾಯವನ್ನು ನೀಡುವ ವ್ಯಕ್ತಿ. ಇಲ್ಲೇ ಸಮಾಜ ಸುಧಾರಕರಿಗೂ ನನಗೂ ಇರುವ\break ಭಿನ್ನಾಭಿಪ್ರಾಯ. ನೂರು ವರುಷಗಳಿಂದ ಅವರು ಇಲ್ಲಿರುವರು. ಒಳ್ಳೆಯದಾದುದನ್ನು ಅವರು ಏನು ತಾನೆ ಮಾಡಿದ್ದಾರೆ? ದ್ವೇಷಕಾರಕ, ನಿಂದಾಸ್ಪದವಾದ ಸಾಹಿತ್ಯ ರಾಶಿಯನ್ನಲ್ಲದೆ ಮತ್ತೇನನ್ನು ಸೃಷ್ಟಿಸಿದ್ದಾರೆ? ದೇವರ ದಯೆಯಿಂದ ಅದು ಇಲ್ಲದೆ ಇದ್ದರೇ ಒಳ್ಳೆಯದಾಗುತ್ತಿತ್ತು! ಅವರು ಸನಾತನಿಗಳನ್ನು ದೂರಿರುವರು, ದ್ವೇಷಿಸಿರುವರು. ಕೊನೆಗೆ ವಿರೋಧಿ\-ಗಳು ಕೂಡ ಮುಯ್ಯಿಗೆ ಮುಯ್ಯಿ ಕೊಡುವುದಕ್ಕೆ ಇವರಂತೆಯೇ ಆಗಿರುವರು. ಇದರ ಪರಿಣಾಮವಾಗಿಯೇ ಪ್ರತಿಯೊಂದು ದೇಶ ಭಾಷೆಯಲ್ಲಿಯೂ ಜನಾಂಗದ ಗೌರವಕ್ಕೆ, ದೇಶದ ಗೌರವಕ್ಕೆ ಚ್ಯುತಿ ತರುವ ಸಾಹಿತ್ಯರಾಶಿ ಬೆಳೆದಿದೆ. ಇದು ಸುಧಾರಣೆಯೇ? ಇದು ದೇಶವನ್ನು ಪ್ರಖ್ಯಾತಿಯ ಶಿಖರಕ್ಕೆ ಏರಿಸುವುದೇ? ಇದು ಯಾರ ತಪ್ಪು?

\vskip   4pt

ನಾವು ಮತ್ತೊಂದು ಮುಖ್ಯವಾದ ವಿಷಯವನ್ನು ಕುರಿತು ಆಲೋಚಿಸಬೇಕಾಗಿದೆ. ಭರತಖಂಡದಲ್ಲಿ ಯಾವಾಗಲೂ ನಮ್ಮನ್ನು ರಾಜರು ಆಳುತ್ತಿದ್ದರು. ರಾಜರೇ ಎಲ್ಲಾ ಶಾಸನಗಳನ್ನೂ ಮಾಡುತ್ತಿದ್ದರು. ಈಗ ರಾಜರಿಲ್ಲ. ಆ ಕಾರ್ಯವನ್ನು ಮಾಡುವವರೇ ಇಲ್ಲ. ಸರ್ಕಾರಕ್ಕೆ ಅದನ್ನು ಮಾಡುವ ಧೈರ್ಯವಿಲ್ಲ. ಜನರ ಅಭಿಪ್ರಾಯಕ್ಕೆ ತಕ್ಕಂತೆ ಸರ್ಕಾರ ಹೊಂದಿಕೊಂಡು ಹೋಗಬೇಕಾಗಿದೆ. ಆರೋಗ್ಯಕರವಾದ, ಬಲವಾದ ಪ್ರಜಾಭಿಪ್ರಾಯವನ್ನು ಸೃಷ್ಟಿಸುವುದಕ್ಕೆ ಬಹಳ ಕಾಲ ಹಿಡಿಯುವುದು. ಅಲ್ಲಿಯವರೆಗೂ ನಾವು ಕಾಯಬೇಕು. ಸಮಾಜ ಸುಧಾರಣೆಯ ಸಮಸ್ಯೆಯೆಲ್ಲಾ ಹೀಗೆ ಆಗುವುದು. ಯಾರಿಗೆ ಸುಧಾರಣೆ ಬೇಕಾಗಿದೆಯೋ ಮೊದಲು ಅವರನ್ನು ಅಣಿಮಾಡಿ. ಜನರೆಲ್ಲಿದ್ದಾರೆ? ಅಲ್ಪಸಂಖ್ಯಾತರ ದಬ್ಬಾಳಿಕೆಯು ಜಗತ್ತಿನಲ್ಲೆಲ್ಲ ಅತಿ ಘೋರ ಅತ್ಯಾಚಾರ. ಕೆಲವು ಆಚಾರಗಳು ದೋಷಪೂರಿತ ಎಂದು ಕೆಲವೇ ವ್ಯಕ್ತಿಗಳು ಆಲೋಚಿಸಿದರೆ ದೇಶ ಜಾಗ್ರತವಾಗುವುದಿಲ್ಲ. ಜನಾಂಗ ಏಕೆ ಜಾಗ್ರತವಾಗುತ್ತಿಲ್ಲ? ಮೊದಲು ಜನರಿಗೆ ಶಿಕ್ಷಣವನ್ನು ಕೊಡಿ; ನಿಮ್ಮ ಶಾಸನ ಸಭೆಯನ್ನು ರಚಿಸಿ. ಅನಂತರ ಶಾಸನ ಬರುವುದು. ಮೊದಲು ಅಧಿಕಾರವನ್ನು ಸೃಷ್ಟಿಸಿ. ಅದರ ಆಧಾರದ ಮೇಲೆ ಶಾಸನ ಜನಿಸುವುದು. ರಾಜರು ಮಾಯವಾದರು. ಹೊಸ ಅಧಿಕಾರವೆಲ್ಲಿ? ಜನಶಕ್ತಿ ಎಲ್ಲಿ? ಅದನ್ನು ತನ್ನಿ. ನೀವು ಸಮಾಜ ಸುಧಾರಣೆ ಮಾಡಬೇಕಾದರೂ ಮೊದಲು ಜನರಲ್ಲಿ ಶಿಕ್ಷಣವನ್ನು ಹರಡಬೇಕು. ಆ ಸಮಯ ಬರುವವರೆಗೆ ನೀವು ಕಾಯಬೇಕು. ಕಳೆದ\break ಶತಮಾನದಲ್ಲಿ ಜನರು ಬಯಸುತ್ತಿದ್ದ ಸುಧಾರಣೆಗಳೆಲ್ಲ ಕೇವಲ ಅಲಂಕಾರಗಳಂತೆ ಇದ್ದವು. ಮೊದಲನೆಯ ಎರಡು ವರ್ಣಗಳಿಗೆ ಮಾತ್ರ ಈ ಸುಧಾರಣೆಗಳೆಲ್ಲ ಅನ್ವಯಿಸುವುವು. ವಿಧವಾ ವಿವಾಹ ಶೇಕಡ ಎಪ್ಪತ್ತು ಮಹಿಳೆಯರಿಗೆ ಅನ್ವಯಿಸುವುದಿಲ್ಲ. ಈ ಸುಧಾರಣೆಗಳೆಲ್ಲ ಮೇಲಿನ ವರ್ಣದಲ್ಲಿರುವ, ಕೆಳಗಿನ ವರ್ಣದವರ ದುಡಿತದಿಂದ ಕೃತವಿದ್ಯರಾದ, ಕೆಲವರಿಗೆ ಮಾತ್ರ ಅನ್ವಯಿಸುವುವು. ತಮ್ಮ ಮನೆ ಗುಡಿಸುವುದಕ್ಕೆ ಎಲ್ಲಾ ಪ್ರಯತ್ನಗಳನ್ನೂ ಮಾಡಿರುವರು. ಆದರೆ ಇದು ಸಮಾಜ ಸುಧಾರಣೆ ಅಲ್ಲ. ನೀವು ಸಮಸ್ಯೆಯ ಮೂಲಕ ಹೋಗಬೇಕು. ಇದನ್ನೇ ನಾನು ಆಮೂಲಾಗ್ರ ಸುಧಾರಣೆ ಎನ್ನುವುದು. ಮೊದಲು ಅಲ್ಲಿ ಬೆಂಕಿ ಇಡಿ, ಹತ್ತಿಕೊಂಡು ಉರಿಯಲಿ. ಆಗ ಭಾರತ ರಾಷ್ಟ್ರ ನಿರ್ಮಾಣವಾಗುವುದು. ಸಮಸ್ಯೆ ಬಹಳ ಜಟಿಲವಾಗಿರುವುದರಿಂದ, ವಿಶಾಲವಾದುದರಿಂದ, ಸುಧಾರಣೆ ಅಷ್ಟು ಸುಲಭವಲ್ಲ. ಅವಸರಪಡಬೇಕಾಗಿಲ್ಲ. ನೂರಾರು ವರುಷಗಳಿಂದ ಈ ಸಮಸ್ಯೆ ಎಲ್ಲರಿಗೂ ಗೊತ್ತಿದೆ.

\vskip   4pt

ಇಂದು ವಿಶೇಷವಾಗಿ ದಕ್ಷಿಣ ದೇಶದಲ್ಲಿ ಬೌದ್ಧಧರ್ಮ ಮತ್ತು ಬೌದ್ಧಧರ್ಮದ ಅಜ್ಞೇಯತಾವಾದ ಇವುಗಳ ವಿಷಯ ಮಾತನಾಡುವುದು ಒಂದು ಫ್ಯಾಷನ್​ ಆಗಿದೆ. ಇಂದು ನಮ್ಮಲ್ಲಿರುವ ಅವನತಿಗೆ ಬೌದ್ಧರೇ ಮುಕ್ಕಾಲುಪಾಲು ಕಾರಣರೆಂಬುದು ಅವರಿಗೆ ಗೊತ್ತೇ ಇಲ್ಲ. ಬೌದ್ಧಧರ್ಮ ನಮ್ಮ ಪಾಲಿಗೆ ಬಿಟ್ಟ ಆಸ್ತಿ ಇದು. ಬೌದ್ಧರ ಉನ್ನತಿ-ಅವನತಿಗಳ ವಿಷಯವನ್ನು ತಿಳಿಯದವರು ಅವರ ವಿಷಯವಾಗಿ ಬರೆದ ಪುಸ್ತಕಗಳಲ್ಲಿ ಬೌದ್ಧಧರ್ಮ ಪ್ರಸಾರಕ್ಕೆ ಮುಖ್ಯಕಾರಣ ಅದರ ಅದ್ಭುತ ನೀತಿ ಮತ್ತು ಗೌತಮ ಬುದ್ಧನ ಅದ್ಭುತ ಜೀವನ ಎನ್ನುವರು. ಭಗವಾನ್​ ಬುದ್ಧದೇವನ ಮೇಲೆ ಭಕ್ತಿ ಗೌರವಗಳು ನನಗೆ ಉಂಟು. ಆದರೆ ಇದನ್ನು ಗಮನಿಸಿ: ಬೌದ್ಧ ಧರ್ಮ ಹರಡುವುದಕ್ಕೆ ಕಾರಣ ಅದರ ಸಿದ್ಧಾಂತವೂ ಅಲ್ಲ, ಆ ಪ್ರಖ್ಯಾತ ಮೂಲಪುರುಷನ ವ್ಯಕ್ತಿತ್ವವೂ ಅಲ್ಲ. ಅದಕ್ಕೆ ಮುಖ್ಯ ಕಾರಣ ಬೌದ್ಧರು ಕಟ್ಟಿದ ವಿಹಾರಗಳು, ಕೆತ್ತಿದ ವಿಗ್ರಹಗಳು, ಜನರ ಮನವನ್ನು ಸೆಳೆಯುವ ಹಲವು ಬಾಹ್ಯಾಚಾರಗಳು. ಬೌದ್ಧಧರ್ಮ ಹರಡಿದ್ದು ಹೀಗೆ. ಹಿಂದೂಗಳ ಮನೆಯಲ್ಲಿ ಹವಿಸ್ಸನ್ನು\break ಕೊಡುವುದಕ್ಕೆ ನಿರ್ಮಿಸುತ್ತಿದ್ದ ಸಣ್ಣ ಸಣ್ಣ ಯಜ್ಞವೇದಿಕೆಗಳಿಗೆ, ಬೌದ್ಧರ ಅದ್ಭುತ ದೇಗುಲಗಳನ್ನೂ, ಆಕರ್ಷಣೀಯ ಬಾಹ್ಯಾಚಾರಗಳನ್ನೂ ಎದುರಿಸುವ ಶಕ್ತಿ ಇರಲಿಲ್ಲ. ಕಾಲಕ್ರಮೇಣ ಬೌದ್ಧಧರ್ಮದ ಈ ಎಲ್ಲವೂ ಅಧೋಗತಿಗೆ ಇಳಿದುವು. ಅದು ಎಷ್ಟರ ಮಟ್ಟಿಗೆ ಅನಾಚಾರಗಳ ರಾಶಿಯಾಯಿತೆಂದರೆ ಅದನ್ನು ಈ ಸಭೆಯಲ್ಲಿ ಹೇಳುವುದಕ್ಕೆ ಆಗುವುದಿಲ್ಲ. ಯಾರು ಈ ವಿಷಯವನ್ನು ತಿಳಿದುಕೊಳ್ಳಬೇಕೆಂದು ಬಯಸುವರೋ ಅವರು ದಕ್ಷಿಣ ಭಾರತದ ದೇವಸ್ಥಾನಗಳಲ್ಲಿರುವ ಕೆತ್ತನೆಗಳನ್ನು ನೋಡಬಹುದು. ಇವೆಲ್ಲಾ ನಮಗೆ ಬಂದದ್ದು ಬೌದ್ಧಧರ್ಮದ ಕೊಡುಗೆಯಾಗಿ. ಅನಂತರ ಮಹಾನ್​ ಸುಧಾರಕರಾದ ಶಂಕರಾಚಾರ್ಯರು ಮತ್ತು ಅವರ ಅನುಯಾಯಿಗಳು ಪ್ರಾಬಲ್ಯಕ್ಕೆ ಬಂದರು. ಅಂದಿನಿಂದ ಇಂದಿನವರೆಗೆ ನೂರಾರು ವರುಷಗಳ ಕಾಲ, ಕ್ರಮೇಣ ಜನಸಾಮಾನ್ಯರಿಗೆ ಪರಿಶುದ್ಧವಾದ ವೇದಾಂತ ಸಿದ್ಧಾಂತದ ಪರಿಚಯ ಮಾಡಿಕೊಡಲು ಯತ್ನಿಸಿರುವರು. ಈ ಸುಧಾರಕರಿಗೆ ಸಮಾಜದ ಲೋಪ ದೋಷಗಳು ಗೊತ್ತಿದ್ದುವು. ಆದರೂ ಅವರು ಅದನ್ನು ಅಲ್ಲಗಳೆಯಲಿಲ್ಲ.\break ``ನಿಮ್ಮಲ್ಲಿರುವುದೆಲ್ಲ ತಪ್ಪು, ಅದನ್ನೆಲ್ಲಾ ಕಿತ್ತೊಗೆಯಿರಿ” ಎಂದು ಅವರು ಹೇಳಲಿಲ್ಲ.\break ಹಾಗೆಂದಿಗೂ ಅದು ಆಗಲಾರದು. ಇಂದು ನನ್ನ ಸ್ನೇಹಿತ ಡಾಕ್ಟರ್​ ಬರೋಸ್​ ಮುನ್ನೂರು ವರ್ಷಗಳಲ್ಲಿ ಕ್ರೈಸ್ತಧರ್ಮ, ಗ್ರೀಕ್​ ಮತ್ತು ರೋಮನ್​ ಪ್ರಾಬಲ್ಯವನ್ನು ತಗ್ಗಿಸಿತು ಎಂದು ಬರೆದುದನ್ನು ಓದಿದೆನು. ಯಾರು ಯೂರೋಪ್​, ಗ್ರೀಸ್​, ರೋಮ್​ ನೋಡಿರುವರೋ ಅವರು ಹೇಳುವ ಮಾತಲ್ಲ ಇದು. ಪ್ರಾಟೆಸ್ಟೆಂಟ್​ ಜನಾಂಗದಲ್ಲಿಯೂ ರೋಮ್​ ಮತ್ತು ಗ್ರೀಕ್​ ಧರ್ಮಗಳ ಪ್ರಾಬಲ್ಯವಿದೆ. ಆದರೆ ಹೆಸರು ಬದಲಾಗಿದೆ. ಹಳೆಯ ದೇವರಿಗೆ ಹೊಸ ನಾಮಕರಣ ಮಾಡಿರುವರು. ದೇವಿಯರು ಮೇರಿಗಳಾಗಿರುವರು. ದೇವರು ಸಂತರಾಗಿರುವರು. ಆಚಾರಗಳು ಹೊಸ ರೂಪವನ್ನು ತಾಳಿರುವುವು. ಪಾಂಟಿಪೆಕ್ಸ್ ಮಾಕ್ಸಿಮಸ್​ ಎಂಬ ಹಳೆಯ ಬಿರುದು ಕೂಡ ಇರುವುದು. ಬದಲಾವಣೆ ತಕ್ಷಣ ಆಗಲಾರದು. ಶಂಕರಾಚಾರ್ಯರಿಗೆ ಇದು ತಿಳಿದಿತ್ತು. ರಾಮಾನುಜಾಚಾರ್ಯರಿಗೆ ಇದು ತಿಳಿದಿತ್ತು. ಅವರಿಗೆ ಇದ್ದ ಏಕಮಾತ್ರ ಮಾರ್ಗವೇ ಕ್ರಮೇಣ ಜನರನ್ನು ಉತ್ತಮ ಆದರ್ಶದ ಕಡೆಗೆ ಕರೆದುಕೊಂಡು ಬರುವುದು. ಅವರು ಬೇರೆ ಮಾರ್ಗವನ್ನು ಹಿಡಿದಿದ್ದರೆ ಮಿಥ್ಯಾಚಾರಿಗಳಾಗುತ್ತಿದ್ದರು. ಅವರ ಧರ್ಮದ ಮೂಲ ಸಿದ್ಧಾಂತವೇ ವಿಕಾಸ. ಜೀವವು ಹಲವು ಅನುಭವಗಳ ಮೂಲಕ ಕ್ರಮೇಣ ಪರಮಾದರ್ಶದ ಕಡೆಗೆ ಹೋಗುವುದು. ಆದಕಾರಣ ಇವೆಲ್ಲ ಆವಶ್ಯಕ. ಇವೆಲ್ಲ ಸಹಾಯ ಮಾಡುವುವು. ಅವನ್ನು ಯಾರು ಧಿಕ್ಕರಿಸಬಲ್ಲರು?

\vskip   4pt

ವಿಗ್ರಹಾರಾಧನೆ ಕೆಟ್ಟದೆಂದು ಹೇಳುವುದು ಈಗ ರೂಢಿಯಾಗಿಹೋಗಿದೆ. ಎಲ್ಲರೂ ಸ್ವಲ್ಪವೂ ವಿಮರ್ಶೆ ಇಲ್ಲದೆ ಇದನ್ನು ಒಪ್ಪಿಕೊಳ್ಳುವರು. ನಾನೂ ಒಮ್ಮೆ ಹೀಗೆಯೇ ಆಲೋಚಿ\-ಸಿದ್ದೆ. ಅದಕ್ಕೆ ಪ್ರಾಯಶ್ಚಿತ್ತರೂಪವಾಗಿ ಎಲ್ಲವನ್ನೂ ವಿಗ್ರಹಾರಾಧನೆಯ ಮೂಲಕ ಕಲಿತವರ ಪದತಳದಲ್ಲಿ ಕುಳಿತು ಕಲಿತುಕೊಳ್ಳಬೇಕಾಯಿತು. ಅವರೇ ಶ‍್ರೀರಾಮಕೃಷ್ಣ ಪರಮಹಂಸರು. ವಿಗ್ರಹಾರಾಧನೆಯಿಂದ ರಾಮಕೃಷ್ಣ ಪರಮಹಂಸರಂಥವರನ್ನು ನಾವು ತಯಾರು\-ಮಾಡುವ ಹಾಗೆ ಇದ್ದರೆ, ನಿಮಗೆ ಸುಧಾರಕರ ದಳ ಬೇಕೆ ಅಥವಾ ವಿಗ್ರಹಗಳು ಬೇಕೆ? ನನಗೆ ಉತ್ತರ ಬೇಕಾಗಿದೆ. ಶ‍್ರೀರಾಮಕೃಷ್ಣ ಪರಮಹಂಸರಂತಹ ಒಬ್ಬರನ್ನು ನೀವು ಪಡೆಯುವ ಹಾಗೆ ಇದ್ದರೆ ಸಹಸ್ರಾರು ವಿಗ್ರಹಗಳನ್ನು ಬೇಕಾದರೆ ತೆಗೆದುಕೊಳ್ಳಿ. ದೇವರು ನಿಮಗೆ ಒಳ್ಳೆಯದು ಮಾಡಲಿ. ನಿಮಗೆ ಸಾಧ್ಯವಾದ ಮಾರ್ಗದಲ್ಲಿ ಅಂತಹ ದೈವೀ ವ್ಯಕ್ತಿಯನ್ನು ಸೃಷ್ಟಿಸಿ. ಆದರೂ ವಿಗ್ರಹಾರಾಧನೆಯನ್ನು ಹಳಿಯುವರು. ಅದಕ್ಕೆ ಕಾರಣವೇನು? ಯಾರಿಗೂ ಗೊತ್ತಿಲ್ಲ. ಏಕೆಂದರೆ ನೂರಾರು ವರ್ಷಗಳ ಹಿಂದೆ ಯೆಹೂದ್ಯನೊಬ್ಬನು ವಿಗ್ರಹಾ\-ರಾಧನೆಯನ್ನು ದೂರಿದನಂತೆ. ಅಂದರೆ ತನ್ನ ವಿಗ್ರಹವನ್ನಲ್ಲದೆ ಉಳಿದವರ ವಿಗ್ರಹವನ್ನೆಲ್ಲ ಅವನು ದೂರಿದನು! ದೇವರನ್ನು ಯಾವುದಾದರೂ ಸುಂದರ ವಿಗ್ರಹದಲ್ಲಿ ಅಥವಾ ಚಿಹ್ನೆಯಲ್ಲಿ ತೋರಿದ್ದರೆ ಅದು ಭಯಾನಕ ಪಾಪ, ಕೆಟ್ಟದ್ದು. ಆದರೆ ಅದನ್ನು ಒಂದು ಸಣ್ಣ ಪೆಟ್ಟಿಗೆಯಂತೆ ಚಿತ್ರಿಸಿ, ಎರಡು ಕಡೆಗಳಲ್ಲಿಯೂ ಅದರ ಮೇಲೆ ಎರಡು ದೇವದೂತರನ್ನು ಕೂರಿಸಿ, ಮೇಲೆ ಮೋಡವೊಂದು ತೇಲುತ್ತಿದ್ದರೆ ಅದು ಅತಿ ಪವಿತ್ರ! ದೇವರು ಪಾರಿವಾಳ\-ದಂತೆ ಬಂದರೆ ಪವಿತ್ರೋತ್ತಮ. ಆದರೆ ಹಸುವಿನಂತೆ ಬಂದರೆ ಹೀದನ್​, ಘೋರಪಾಪ, ಮೌಢ್ಯ. ಅದನ್ನು ವಿರೋಧಿಸಬೇಕು! ಪ್ರಪಂಚವೇ ಹೀಗೆ. ಅದಕ್ಕೇ ಕವಿ, “ಮನುಷ್ಯರು ಎಂತಹ ಹುಚ್ಚರು” ಎಂದು ಹೇಳುವುದು. ಮತ್ತೊಬ್ಬರ ದೃಷ್ಟಿಯಿಂದ ನೋಡುವುದು ಎಷ್ಟು ಕಷ್ಟ. ಇದೇ ಮಾನವ ಜಾತಿಯ ಮಹಾದೋಷ. ಇದೇ ದ್ವೇಷಾಸೂಯೆಗಳಿಗೆ ಹೋರಾಟಕ್ಕೆ ಕಾರಣ. ಮಕ್ಕಳು, ಮೀಸೆಯಿರುವ ಮಕ್ಕಳು, ಮದ್ರಾಸಿನಿಂದ ಹೊರಗೆ ಹೋಗದವರು, ಸಹಸ್ರಾರು ಪ್ರಾಚೀನ ಸಂಸ್ಕಾರಗಳಿಂದ ಕೂಡಿದ ಮೂವತ್ತು ಕೋಟಿ ಜನರಿಗೆ ಅಪ್ಪಣೆ ಮಾಡಲೆತ್ನಿಸುವರು. ನಿಮಗೆ ನಾಚಿಕೆಯಾಗುವುದಿಲ್ಲವೇ? ಇಂತಹ ದೇವನಿಂದೆಯಿಂದ ಪಾರಾಗಿ, ಮೊದಲು ನಿಮ್ಮ ಕರ್ತವ್ಯವನ್ನು ತಿಳಿಯಿರಿ. ಅಂಕೆ ಇಲ್ಲದ ಹುಡುಗರೆ, ನಿಮಗೆ ಬಿಳಿಕಾಗದದ ಮೇಲೆ ಸ್ವಲ್ಪ ಗೀಚುವುದಕ್ಕೆ ಬರುವುದೆಂದು, ಅದನ್ನು ಯಾರೋ ಮೂಢರು ಪ್ರಕಾಶಿಸುವರೆಂದು ನೀವು ಏನು ಪ್ರಪಂಚಕ್ಕೆಯೇ ವಿದ್ಯಾದಾನ ಮಾಡುವಿರೆಂದು ಭಾವಿಸಿರುವಿರೇನು? ನೀವೇ ಭಾರತದ ಸಾರ್ವಜನಿಕ ಅಭಿಪ್ರಾಯ ಎಂದು ಭಾವಿಸಿರುವಿರೇನು? ಮದ್ರಾಸಿನ ಸುಧಾರಕರಿಗೆ ನಾನು ಇದನ್ನು ಹೇಳಬೇಕಾಗಿದೆ: ನನಗೆ ಅವರ ಮೇಲೆ ಅತಿ ಗೌರವ, ವಿಶ್ವಾಸಗಳು ಇವೆ. ಅವರ ಹೃದಯ ವೈಶಾಲ್ಯ, ದೇಶ ಪ್ರೇಮ, ದೀನದಲಿತರ ಮೇಲೆ ಅವರಿಗೆ ಇರುವ ವಿಶ್ವಾಸ ಇವುಗಳಿಗಾಗಿ ನಾನು ಅವರನ್ನು ಪ್ರೀತಿಸುತ್ತೇನೆ. ನಾನು ಅವರಿಗೆ ಸಹೋದರ ಸಹಜ ಪ್ರೇಮದಿಂದ ಹೇಳುವುದೇನೆಂದರೆ, ಅವರ ಮಾರ್ಗ ಸರಿಯಲ್ಲ ಎಂದು. ನೂರಾರು ವರುಷಗಳಿಂದ ಇಂಥ ಪ್ರಯತ್ನಗಳು ವಿಫಲವಾಗಿವೆ. ಹೊಸದೊಂದು ಪ್ರಯೋಗವನ್ನು ಮಾಡಿ ನೋಡೋಣ.

\vskip   4pt

ಭರತಖಂಡಕ್ಕೆ ಸುಧಾರಕರ ಬರಗಾಲ ಎಂದಾದರೂ ಇತ್ತೇ? ಭರತಖಂಡದ ಇತಿಹಾಸವನ್ನು ನೀವು ಓದಿರುವಿರಾ? ರಾಮಾನುಜ ಯಾರು? ಶಂಕರ ಯಾರು? ನಾನಕ್​ ಯಾರು? ಚೈತನ್ಯ ಯಾರು? ಕಬೀರ ಯಾರು? ದಾದು ಯಾರು? ಒಂದಾದಮೇಲೊಂದು ಮೂಡಿದ ಪ್ರಥಮವರ್ಗದ ತಾರಾ ಶ್ರೇಣಿಗಳಂತೆ ಇರುವ ಧರ್ಮ ಪ್ರಚಾರಕರು ಯಾರು? ರಾಮಾನು\-ಜರು ಅಂತ್ಯಜರಿಗಾಗಿ ಮರುಗಲಿಲ್ಲವೇ? ಚಂಡಾಲರನ್ನು ಕೂಡ ತಮ್ಮ ಪಂಗಡಕ್ಕೆ ಸೇರಿಸಿಕೊಳ್ಳಲು ಅವರು ತಮ್ಮ ಜೀವಮಾನವೆಲ್ಲ ಪ್ರಯತ್ನ ಪಡಲಿಲ್ಲವೆ? ಮಹಮದೀಯರನ್ನು ತಮ್ಮ ಪಂಗಡಕ್ಕೆ ಸೇರಿಸಲು ಅವರು ಪ್ರಯತ್ನ ಪಡಲಿಲ್ಲವೇ? ನಾನಕ್​ ಹಿಂದೂಗಳೊಂದಿಗೆ ಮತ್ತು ಮಹಮದೀಯರೊಂದಿಗೆ ಕಲೆತು ಹೊಸ ಸೌಹಾರ್ದ ಭಾವವನ್ನು ತರಲು ಯತ್ನಿಸಲಿಲ್ಲವೇ? ಅವರೆಲ್ಲ ಪ್ರಯತ್ನಪಟ್ಟರು. ಅವರ ಪ್ರಯತ್ನ ಈಗಲೂ ಸಾಗುತ್ತಿದೆ. ವ್ಯತ್ಯಾಸವಿಷ್ಟೆ. ಅವರಲ್ಲಿ ಇಂದಿನ ಸಮಾಜ ಸುಧಾರಕರ ಅಟ್ಟಹಾಸವಿರಲಿಲ್ಲ. ಆಧುನಿಕ ಸಮಾಜ ಸುಧಾರಕರ ಬಾಯಿಯಿಂದ ಹೊರಬರುವ ಶಾಪದ ಸುರಿಮಳೆ ಅವರಲ್ಲಿ ಇರಲಿಲ್ಲ. ಅವರು ಯಾರನ್ನೂ ನಿಂದಿಸುತ್ತಿರಲಿಲ್ಲ. ಎಲ್ಲರನ್ನೂ ಆಶೀರ್ವದಿಸಿದರು. ಜನಾಂಗ ಯಾವಾಗಲೂ ಮುಂದುವರಿಯುತ್ತಿರಬೇಕೆಂದು ಅವರು ಹೇಳಿದರು. ಅವರು “ಹಿಂದೂಗಳೆ ನೋಡಿ, ಹಿಂದೆ ನೀವು ಚೆನ್ನಾಗಿ ಮಾಡಿರುವಿರಿ. ಆದರೆ ಸಹೋದರರೆ, ಈಗ ಅದಕ್ಕಿಂತಲೂ ಚೆನ್ನಾಗಿ ಮಾಡೋಣ” ಎನ್ನುತ್ತಿದ್ದರು. “ನೀವು ದುಷ್ಟರಾಗಿದ್ದೀರಿ, ಈಗ ಒಳ್ಳೆಯವರಾಗಿ” ಎನ್ನಲಿಲ್ಲ. ಇಲ್ಲೇ ಎಷ್ಟೋ ವ್ಯತ್ಯಾಸವಾಗುವುದು. ನಮ್ಮ ಸ್ವಭಾವಕ್ಕೆ ಅನುಗುಣವಾಗಿ ನಾವು\break ಬೆಳೆಯಬೇಕು. ಪಾಶ್ಚಾತ್ಯರು ನಮ್ಮ ಮುಂದೆ ಇಟ್ಟ ಆದರ್ಶವನ್ನು ಅನುಕರಿಸುವು\break ದರಿಂದೇನೂ ಪ್ರಯೋಜನವಿಲ್ಲ. ಹಾಗೆ ಆಗದೇ ಇರುವುದು ದೇವರ ದಯದಿಂದ! ನಮ್ಮನ್ನು ಕಾಯಿಸಿ, ಬಡಿದು ಇತರ ರಾಷ್ಟ್ರಗಳ ರೂಪಕ್ಕೆ ತರುವುದು ಸಾಧ್ಯವಿಲ್ಲ. ಇತರ ರಾಷ್ಟ್ರಗಳ ಸಾಮಾಜಿಕ ಸಂಸ್ಥೆಗಳನ್ನು ನಾನು ಅಲ್ಲಗಳೆಯುವುದಿಲ್ಲ. ಅವು ಅವರಿಗೆ ಒಳ್ಳೆ\-ಯದು, ನಮಗಲ್ಲ. ಅವರ ಅಮೃತ ನಮಗೆ ವಿಷವಾಗಬಹುದು. ನಾವು ಕಲಿಯಬೇಕಾದ ಮೊದಲನೆಯ ನೀತಿಯೇ ಇದು. ಅವರ ವೈಜ್ಞಾನಿಕ ಸಮಾಜದ ಸಂಸ್ಕಾರಕ್ಕೆ ತಕ್ಕಂತೆ ಈಗ ಅವರ ಆಚಾರ ವ್ಯವಹಾರಗಳಿವೆ. ಸಹಸ್ರಾರು ವರುಷಗಳ ನಮ್ಮ ಕರ್ಮಕ್ಕೆ ಅನುಗುಣವಾಗಿ ನಾವು ಬೆಳೆಯಬೇಕಾಗಿದೆ.

\vskip   4pt

ಹಾಗಾದರೆ ನನ್ನ ಯೋಜನೆ ಏನು? ಪುರಾತನ ಮಹಾತ್ಮರ ಆಲೋಚನೆಯನ್ನು ಅನುಸರಿಸುವುದೇ ನನ್ನ ಯೋಜನೆ. ನಾನು ಅವರೇನು ಮಾಡಿದರೆಂಬುದನ್ನು ಓದಿರುವೆನು. ಈಶ್ವರೇಚ್ಛೆಯಿಂದ ಅವರು ಯಾವ ರೀತಿ ಕಾರ್ಯ ಮಾಡಿದರು ಎಂಬುದನ್ನು ಕಂಡುಹಿಡಿದಿರುವೆನು. ಅವರು ಪ್ರಬಲ ಸಮಾಜ ಸಂಘಟನಕಾರರು. ನಮ್ಮ ಜೀವನಕ್ಕೆ ಶಕ್ತಿ ಪವಿತ್ರತೆ ಇವನ್ನು ಮತ್ತು ಹೊಸ ಬಾಳನ್ನು ಕೊಟ್ಟರು. ಅವರು ಅದ್ಭುತ ಕಾರ್ಯಗಳನ್ನು ಸಾಧಿಸಿದರು. ನಾವೂ ಅದ್ಭುತ ಕಾರ್ಯಗಳನ್ನು ಸಾಧಿಸಬೇಕಾಗಿದೆ. ಈಗ ಕಾಲ ಸ್ವಲ್ಪ ಬದಲಾಗಿದೆ. ಅದಕ್ಕೆ ಸರಿಯಾಗಿ ನಮ್ಮ ಕಾರ್ಯವನ್ನು ಹೊಂದಿಸಿಕೊಳ್ಳಬೇಕಾಗಿದೆ ಅಷ್ಟೆ. ಪ್ರತಿಯೊಂದು ವ್ಯಕ್ತಿಗೆ ಇರುವಂತೆ, ಪ್ರತಿಯೊಂದು ದೇಶಕ್ಕೂ ಒಂದು ಜೀವನೋದ್ದೇಶವಿದೆ. ಅದೇ ಜೀವನದ ಪ್ರಧಾನ ಸ್ವರ. ಉಳಿದ ಸ್ವರಗಳು ಅದರಲ್ಲಿ ಮಿಲನವಾಗಿ ಐಕ್ಯಗೊಳ್ಳುವುವು. ಇಂಗ್ಲೆಂಡಿನಂಥ ಒಂದು ದೇಶದಲ್ಲಿ ರಾಜಕಾರಣ ಅದರ ಜೀವಾಳ, ಮತ್ತೊಂದು ದೇಶದಲ್ಲಿ ಕಲಾಕೌಶಲ್ಯ ಅದರ ಜೀವಾಳ. ಭರತಖಂಡದಲ್ಲಿ ಧರ್ಮವೇ ಅದರ ಕೇಂದ್ರ. ರಾಷ್ಟ್ರ ಜೀವನದ ಉಳಿದ ಸ್ವರಗಳೆಲ್ಲ ಇದಕ್ಕೆ ಅಧೀನ. ಹಲವು ಶತಮಾನಗಳಿಂದ ಒಂದೇ ದಿಕ್ಕಿನಲ್ಲಿ ಹರಿದು ಅದರ ಸ್ವಭಾವ ಆ ಜನಾಂಗದ ಮುಖ್ಯ ಗುಣವಾಗಿದೆ. ಯಾರಾದರೂ ಅದನ್ನು ಬದಲಾಯಿಸಲು ಪ್ರಯತ್ನಪಟ್ಟರೆ, ಆ ಉದ್ಯಮದಲ್ಲಿ ಜಯಶೀಲರಾದರೆ, ಆ ರಾಷ್ಟ್ರ ನಾಶವಾದಂತೆ. ನೀವು ಧರ್ಮವನ್ನು ತ್ಯಜಿಸಿ, ರಾಜಕೀಯವನ್ನಾಗಲೀ ಸಾಮಾಜಿಕವನ್ನಾಗಲೀ ಅಥವಾ ಇತರ ಯಾವುದನ್ನಾಗಲೀ ನಿಮ್ಮ ಜೀವನದ ಮುಖ್ಯ ಗುರಿಯನ್ನಾಗಿ ಮಾಡಲು ಪ್ರಯತ್ನಿಸಿದರೆ ಅದರ ಪರಿಣಾಮವಾಗಿ ನಿಮ್ಮ ರಾಷ್ಟ್ರ ನಾಶವಾಗುವುದು. ಇದನ್ನು ತಡೆಯುವುದಕ್ಕೆ ಎಲ್ಲರೂ ಎಲ್ಲಾ ಕಾರ್ಯಗಳನ್ನೂ ಧರ್ಮದ ಮೂಲಕ ಮಾಡಲು ಪ್ರಯತ್ನಿಸಬೇಕು. ನಿಮ್ಮ ಪ್ರತಿಯೊಂದು ಕಾರ್ಯದಲ್ಲಿಯೂ ಧಾರ್ಮಿಕ ಹಿನ್ನೆಲೆ ಅನುರಣಿತವಾಗುತ್ತಿರಲಿ. ನನಗೆ ಅಮೆರಿಕಾ ದೇಶೀಯರಿಗೆ ಧರ್ಮವನ್ನು ಕೂಡ, ಸಮಾಜದ ಮೇಲೆ ಅದರಿಂದ ಯಾವ ಪರಿಣಾಮ ಉಂಟಾಗುವುದೆಂಬುದನ್ನು ತೊರೆದು, ಬೋಧಿಸಲಾಗಲಿಲ್ಲ. ಧರ್ಮ ಯಾವರೀತಿ ರಾಜಕೀಯದಲ್ಲಿ ಅದ್ಭುತ ಪರಿಣಾಮವನ್ನುಂಟು ಮಾಡುವುದೆಂಬುದನ್ನು ಬಿಟ್ಟು ಇಂಗ್ಲೆಂಡಿನಲ್ಲಿ ನನಗೆ ವೇದಾಂತವನ್ನು ಬೋಧಿಸಲಾಗಲಿಲ್ಲ. ಇದರಂತೆಯೇ ಭರತಖಂಡದಲ್ಲಿ ಸಮಾಜ ಸುಧಾರಣೆಯನ್ನು ಕೂಡ ಅದರ ಮೂಲಕ ಹೇಗೆ ನಮ್ಮ ಆಧ್ಯಾತ್ಮಿಕ ಸಂಪತ್ತನ್ನು ಹೆಚ್ಚಿಸಬಹುದು ಎಂಬುದನ್ನು ಲಕ್ಷ್ಯದಲ್ಲಿಟ್ಟುಕೊಂಡು ಬೋಧಿಸಬೇಕು. ರಾಜಕೀಯವನ್ನು ಕೂಡ, ಭರತಖಂಡಕ್ಕೆ ಅತ್ಯಾವಶ್ಯಕವಾಗಿ ಬೇಕಾಗಿರುವ ಅಧ್ಯಾತ್ಮವನ್ನು ಎಷ್ಟರ ಮಟ್ಟಿಗೆ ಉತ್ತಮಪಡಿಸುವುದು ಎಂಬುದನ್ನು ಲಕ್ಷ್ಯದಲ್ಲಿಟ್ಟುಕೊಂಡು ಬೋಧಿಸಬೇಕು. ಪ್ರತಿಯೊಬ್ಬನೂ ತನಗೆ ಬೇಕಾದುದನ್ನು ಆರಿಸಿಕೊಳ್ಳಬೇಕು. ಅದರಂತೆಯೇ ಪ್ರತಿಯೊಂದು ರಾಷ್ಟ್ರವೂ ಕೂಡ. ಹಲವು ಶತಮಾನಗಳ ಹಿಂದೆಯೇ ನಮಗೆ ಬೇಕಾದುದನ್ನು ನಾವು ಆರಿಸಿಕೊಂಡೆವು. ನಾವು ಈಗ ಅದರಂತೆ ನಡೆದುಕೊಳ್ಳಬೇಕಾಗಿದೆ. ಆದರೆ ಇದೇನೂ ಅಂತಹ ಕೆಟ್ಟ ಆಯ್ಕೆಯಲ್ಲ. ಸಂಸಾರವನ್ನು ಬಿಟ್ಟು ಅಧ್ಯಾತ್ಮವನ್ನು ನೆನೆಯುವುದು, ಮನುಷ್ಯನನ್ನು ಬಿಟ್ಟು ದೇವರನ್ನು ನೆನೆಯುವುದು ಅಂತಹ ಕೆಟ್ಟ ಆಯ್ಕೆಯೆ? ಪರಲೋಕದಲ್ಲಿ ಅಚಲ ನಂಬಿಕೆ, ಇಹಲೋಕವನ್ನು ನಿಕೃಷ್ಟ ದೃಷ್ಟಿಯಿಂದ ನೋಡುವುದು; ತೀವ್ರ ತ್ಯಾಗಶಕ್ತಿ, ಭಗವಂತನಲ್ಲಿ ಶ್ರದ್ಧೆ, ಅಮರವಾದ ಜೀವಾತ್ಮನಲ್ಲಿ ಶ್ರದ್ಧೆ, ಇವೆಲ್ಲ ನಮ್ಮಲ್ಲಿವೆ. ಇವನ್ನು ತ್ಯಜಿಸುವುದಕ್ಕೆ ನಿಮ್ಮಲ್ಲಿ ಯಾರಿಗಾದರೂ ಸಾಧ್ಯವೇ ಎಂದು ಕೇಳುತ್ತೇನೆ. ಇದು ಸಾಧ್ಯವಿಲ್ಲ. ನಿಮ್ಮನ್ನು ಜಡವಾದಿಗಳೆಂದು ನಂಬಿಸಬಹುದು. ಕೆಲವು ಕಾಲ ಹಾಗೆ ಮಾತನಾಡಬಹುದು. ಆದರೆ ನೀವು ಅಂತರಾಳದಲ್ಲಿ ಏನಾಗಿರುವಿರಿ ಎಂಬುದು ನನಗೆ ಗೊತ್ತು. ನಾನು ನಿಮ್ಮನ್ನು ಕೈ ಹಿಡಿದು ಕರೆದರೆ, ದೇವರ ಪರಮ ಭಕ್ತರಂತೆ ಹಿಂಬಾಲಿಸುವಿರಿ. ನಿಮ್ಮ ಸ್ವಭಾವವನ್ನು ಹೇಗೆ ಬದಲಾಯಿಸ\-ಬಲ್ಲಿರಿ?

\vskip   4pt

ಭರತಖಂಡದಲ್ಲಿ ಎಲ್ಲಾ ವಿಧದ ಜಾಗೃತಿ ಫಲಕಾರಿಯಾಗಬೇಕಾದರೆ ಮೊದಲು ಧಾರ್ಮಿಕ ಜಾಗೃತಿ ಆವಶ್ಯಕ. ರಾಜಕೀಯ ಮತ್ತು ಸಾಮಾಜಿಕ ವಿಷಯಗಳಿಂದ ಭರತಖಂಡವನ್ನು ತುಂಬುವುದಕ್ಕಿಂತ ಮುಂಚೆ ಆಧ್ಯಾತ್ಮಿಕ ಭಾವನೆಯ ಮಳೆಗರೆಯಿರಿ. ಮೊದಲು ನಾವು ಲಕ್ಷ್ಯದಲ್ಲಿಡಬೇಕಾದುದು ಇದು. ನಮ್ಮ ಉಪನಿಷತ್​-ಶಾಸ್ತ್ರ- ಪುರಾಣಗಳಲ್ಲಿರುವ ಅದ್ಭುತ ಸತ್ಯಗಳು ಪ್ರಚಾರವಾಗಬೇಕು. ಮಠಗಳಿಂದ, ಕಾನನಗಳಿಂದ, ತಮಗೇ ಇವು ಮೀಸಲಾಗಿವೆ ಎಂದು ಭಾವಿಸಿದ ಕೆಲವು ಜನರಿಂದ ಹೊರಗೆ ತಂದು ಎಲ್ಲರಿಗೂ ಪ್ರಚಾರ ಮಾಡಬೇಕು. ಈ ಭಾವನೆ ಬೆಂಕಿಯಂತೆ ಉತ್ತರದಿಂದ ದಕ್ಷಿಣಕ್ಕೆ, ಪೂರ್ವದಿಂದ ಪಶ್ಚಿಮಕ್ಕೆ, ಹಿಮಾಲಯದಿಂದ ಕನ್ಯಾಕುಮಾರಿಯವರೆಗೆ, ಸಿಂಧುವಿನಿಂದ ಬ್ರಹ್ಮಪುತ್ರದವರೆಗೆ ಹರಿಯಬೇಕು. ಪ್ರತಿಯೊಬ್ಬರೂ ಇದನ್ನು ತಿಳಿದುಕೊಳ್ಳಬೇಕು. “ಮೊದಲು ಶ್ರವಣ, ಅನಂತರ ಮನನ, ಅದರ ಅನಂತರ ನಿದಿ ಧ್ಯಾಸನ” ಎಂದು ಹೇಳುವರು. ಮೊದಲು ಜನರು ಇದನ್ನು ಕೇಳಲಿ. ಅವರ ಶಾಸ್ತ್ರದಲ್ಲಿರುವ ಮಹಾಸತ್ಯವನ್ನು ಅವರು ಕೇಳುವುದಕ್ಕೆ ಯಾರು ಸಹಾಯ ಮಾಡುವರೋ ಅವರು ಅದಕ್ಕಿಂತ ಹೆಚ್ಚು ಪವಿತ್ರ ಕರ್ಮವನ್ನು ಇಂದು ಮಾಡಲಾರರು. ಭಗವಾನ್​ ವ್ಯಾಸರು ಹೇಳುವಂತೆ ಕಲಿಯುಗದಲ್ಲಿ ಒಂದು ಕರ್ಮ ಉಳಿದಿದೆ. ಯಾಗಯಜ್ಞಗಳಿಂದ, ಪ್ರಚಂಡ ತಪಸ್ಸಿನಿಂದ ಈಗ ಪ್ರಯೋಜನವಿಲ್ಲ. ಕರ್ಮದಲ್ಲಿ ಉಳಿದಿರುವುದೊಂದೇ, ಅದೇ ದಾನ, ದಾನದಲ್ಲಿ ಅಧ್ಯಾತ್ಮ-ಜ್ಞಾನದಾನ ಪರಮೋತ್ಕೃಷ್ಟವಾದುದು. ಎರಡನೆಯದೇ ಸಾಧಾರಣ ವಿದ್ಯಾದಾನ, ಕೊನೆಯದೇ ಅನ್ನದಾನ. ಈ ಅದ್ಭುತ ಉದಾರ ಜನಾಂಗವನ್ನು ನೋಡಿ. ಈ ಬಡದೇಶದಲ್ಲಿ ಜನರು ಮಾಡಿರುವ ದಾನವನ್ನು ನೋಡಿ. ಜನರ ಅತಿಥಿ ಸತ್ಕಾರವನ್ನು ನೋಡಿ. ಉತ್ತರದಿಂದ ದಕ್ಷಿಣದವರೆಗೆ ಕೈಯಲ್ಲಿ ಕಾಸು ಇಲ್ಲದೇ ಪ್ರಯಾಣ ಮಾಡುವವನೂ ಅತ್ಯುತ್ತಮ ವಸ್ತುಗಳನ್ನು ಪಡೆಯಬಲ್ಲ, ಎಲ್ಲರೂ ಎಲ್ಲಾ ಸಮಯದಲ್ಲಿಯೂ ಸ್ನೇಹಿತನಂತೆ ಅವನನ್ನು ಉಪಚರಿಸುವರು. ಯಾವ ಭಿಕ್ಷುಕನೂ ಉಪವಾಸವಿರಬೇಕಾಗಿಲ್ಲ.

ಈ ದಾನಶೀಲ ದೇಶದಲ್ಲಿ ಅಧ್ಯಾತ್ಮ ಜ್ಞಾನದಾನವೆಂಬ ಮೊದಲನೆಯ ಶಕ್ತಿಯನ್ನು ತೆಗೆದು\-ಕೊಳ್ಳೋಣ. ಇದು ಭರತಖಂಡದಲ್ಲಿ ಮಾತ್ರ ಕೊನೆಗಾಣಕೂಡದು. ಇದು ಪ್ರಪಂಚದಲ್ಲೆಲ್ಲಾ ಹರಡಬೇಕು. ಹಿಂದೆ ಇದು ರೂಢಿಯಲ್ಲಿತ್ತು. ಭರತಖಂಡದ ಆಲೋಚನೆ ಹೊರಗೆ ಹೋಗಿಲ್ಲ, ನಾನೇ ಮೊದಲು ಹೊರಗೆ ಹೋಗಿ ಉಪನ್ಯಾಸ ಮಾಡಿದವನು ಎಂದರೆ, ಅವರಿಗೆ ನಮ್ಮ ದೇಶದ ಚರಿತ್ರೆಯ ಅರಿವೇ ಇಲ್ಲ. ಪದೇ ಪದೇ ಹೀಗೆ ಆಗುತ್ತಿದೆ. ಪ್ರಪಂಚಕ್ಕೆ ಆವಶ್ಯಕವಾದಾಗೆಲ್ಲಾ ಎಂದಿಗೂ ಬತ್ತದ ಈ ಆಧ್ಯಾತ್ಮಿಕ ಪ್ರವಾಹ ಭರತಖಂಡದ ಹೊರಗೆ ಇರುವವರಿಗೆಲ್ಲ ಹೋಗಿದೆ. ಪ್ರಚಂಡ ಸೇನಾಚಲನೆ, ರಣಕಹಳೆಯ ಧ್ವನಿ ಇವುಗಳಿಂದ ರಾಜಕೀಯ ಭಾವನೆ ಹರಡಬಹುದು. ಲೌಕಿಕ ಮತ್ತು ಸಾಮಾಜಿಕ ಭಾವನೆಗಳನ್ನು ಕತ್ತಿಯ ಮತ್ತು ಕೋವಿಯ ಬಲದ ಮೂಲಕ ಕೊಡಬಹುದು. ಆದರೆ ಆಧ್ಯಾತ್ಮಿಕ ಭಾವನೆಯನ್ನು ಯಾವಾಗಲೂ ಮೌನವಾಗಿ ಮಾತ್ರ ಕೊಡಲು ಸಾಧ್ಯ. ಯಾರಿಗೂ ಕೇಳದೆ, ಕಾಣದೆ ಬೀಳುವ ಹಿಮಮಣಿ ಗುಲಾಬಿಯ ಹೂವುಗಳನ್ನು ಅರಳಿಸುವಂತೆ, ಪುನಃ ಪುನಃ ಪ್ರಪಂಚಕ್ಕೆ ಭರತಖಂಡವು ಈ ದಾನವನ್ನು ಮಾಡಿದೆ. ಯಾವಾಗಲಾದರೂ ಪ್ರಬಲ ರಾಷ್ಟ್ರಗಳು ಮೇಲೆದ್ದು ಅನ್ಯರಾಷ್ಟ್ರಗಳನ್ನು ಒಟ್ಟಿಗೆ ಸೇರಿಸಿ, ರಸ್ತೆ ವಾಹನಸಂಚಾರ ಮುಂತಾದುವಕ್ಕೆ ಅವಕಾಶಮಾಡಿದೊಡನೆಯೆ, ಭರತಖಂಡ ತಕ್ಷಣ ಜಾಗೃತವಾಗಿ ವಿಶ್ವಕಲ್ಯಾಣವೆಂಬ ಸಮಷ್ಟಿಗೆ ಅಧ್ಯಾತ್ಮವನ್ನು ಧಾರೆ ಎರೆದಿರುವುದು. ಇದು ಬುದ್ಧನ ಕಾಲಕ್ಕೆ ಮುಂಚೆಯೇ ಆಯಿತು. ಇದರ ಅವಶೇಷಗಳನ್ನು ಚೈನ, ಏಷ್ಯಾಮೈನರ್​, ಮಲಯ ಪರ್ಯಾಯ ದ್ವೀಪದ ಮಧ್ಯಭಾಗ-ಇವುಗಳಲ್ಲಿ ನೋಡಬಹುದು. ಪ್ರಖ್ಯಾತ ಯವನ ವಿಜಯ ಅಂದಿನ ಕಾಲದಲ್ಲಿ ಗೊತ್ತಾದ ಪ್ರಪಂಚದ ನಾಲ್ಕು ಭಾಗಗಳನ್ನೂ ಒಂದುಗೂಡಿಸಿದಾಗ ಹೀಗೆ ಆಯಿತು. ಆಗ ಭರತಖಂಡದ ಆಧ್ಯಾತ್ಮಿಕ ಪ್ರವಾಹ ಹರಿದುಹೋಯಿತು. ಇಂದು ಹೆಮ್ಮೆಯನ್ನು ಕೊಚ್ಚಿಕೊಳ್ಳುವ ಪಾಶ್ಚಾತ್ಯ ಸಂಸ್ಕೃತಿ ಅಂದಿನ ಕಾಲದ ಅವಶೇಷವಷ್ಟೆ. ಈಗ ಅದೇ ಅವಕಾಶ ಒದಗಿದೆ. ಆಂಗ್ಲಶಕ್ತಿ ಹಿಂದೆ ಎಂದೂ ಇಲ್ಲದಂತೆ ಪ್ರಪಂಚವನ್ನೆಲ್ಲಾ ಒಟ್ಟುಗೂಡಿಸಿರುವುದು. ಪ್ರಪಂಚದ ಒಂದು ಮೂಲೆಯಿಂದ ಮತ್ತೊಂದು ಮೂಲೆಗೆ ಇಂಗ್ಲೆಂಡಿನ ರಸ್ತೆಗಳು ಇವೆ. ಆಂಗ್ಲ ಮೇಧಾಶಕ್ತಿಯಿಂದ ಇಂದು ಪ್ರಪಂಚ ಹಿಂದೆ ಎಂದೂ ಇಲ್ಲದಂತೆ ನಿಕಟವಾಗಿ ಸೇರಿಸಲ್ಪಟ್ಟಿದೆ. ಮಾನವ ಇತಿಹಾಸದಲ್ಲೇ ಇದುವರೆಗೆ ಇಲ್ಲದ ರೀತಿಯಲ್ಲಿ ವ್ಯಾಪಾರ ಕೇಂದ್ರಗಳು ನಿರ್ಮಿತವಾಗಿವೆ. ತಕ್ಷಣವೆ ತಿಳಿದೋ ತಿಳಿಯದೆಯೋ ಭರತಖಂಡ ಜಾಗ್ರತವಾಗಿ ಅಧ್ಯಾತ್ಮವನ್ನು ಧಾರೆ ಎರೆಯುವುದು. ಅದು ಈ ಮಾರ್ಗವಾಗಿ ಪ್ರಪಂಚದ ಮೂಲೆ ಮೂಲೆಗಳನ್ನೂ ಸೇರುವುದು. ನಾನು ಅಮೆರಿಕಾ ದೇಶಕ್ಕೆ ಹೋದುದು ನನ್ನಿಂದ ಅಲ್ಲ ಅಥವಾ ನಿಮ್ಮಿಂದಲೂ ಅಲ್ಲ, ಭಾರತದ ಅದೃಷ್ಟವನ್ನು ಕಾಯುತ್ತಿರುವ ವಿಧಾತ ನನ್ನನ್ನು ಕಳುಹಿಸಿದನು. ಹೀಗೆಯೆ ನನ್ನಂತಹ ನೂರಾರು ಜನರನ್ನು ಪ್ರಪಂಚದ ಎಲ್ಲ ಕಡೆಗೂ ಕಳುಹಿಸುವನು. ಪ್ರಪಂಚದ ಯಾವ ಶಕ್ತಿಯೂ ಇದನ್ನು ತಡೆಯಲಾರದು. ಜೊತೆಗೆ ಇದನ್ನೂ ಮಾಡಬೇಕಾಗಿದೆ. ನೀವು ನಿಮ್ಮ ಧರ್ಮವನ್ನು ಬೋಧಿಸುವುದಕ್ಕೆ ಹೊರಗೆ ಹೋಗಬೇಕು; ಪ್ರಪಂಚದ ಎಲ್ಲಾ ದೇಶಗಳಿಗೆ ಎಲ್ಲಾ ಜನಾಂಗಗಳಿಗೆ ಅದನ್ನು ಬೋಧಿಸಬೇಕು. ನಾವು ಮಾಡಬೇಕಾದ ಪ್ರಥಮ ಕರ್ತವ್ಯವಿದು. ಅಧ್ಯಾತ್ಮ ಬೋಧಿಸಿದ ಮೇಲೆ ನಿಮಗೆ ಬೇಕಾದ ಲೌಕಿಕ ಜ್ಞಾನವೆಲ್ಲಾ ಬಂದೇ ಬರುವುದು. ಧರ್ಮವಿಲ್ಲದೆ ಕೇವಲ ಲೌಕಿಕ ಜ್ಞಾನವನ್ನು ಮಾತ್ರ ಪಡೆಯಲು ನೀವು ಯತ್ನಿಸಿದರೆ ಪ್ರಯೋಜನವಿಲ್ಲ. ಜನರ ಮೇಲೆ ಇದು ಪರಿಣಾಮಕಾರಿಯಾಗುವುದಿಲ್ಲ. ಬೌದ್ಧರ ಭಾರೀ ಚಳುವಳಿ ಕೂಡ ನಿಷ್ಪ್ರಯೋಜಕವಾದುದಕ್ಕೆ ಇದೇ ಕಾರಣ.

ಆದಕಾರಣ, ನನ್ನ ಸ್ನೇಹಿತರೇ, ನಮ್ಮ ಧರ್ಮವನ್ನು ಭರತಖಂಡದಲ್ಲಿ ಮತ್ತು ಹೊರಗೆ ಪ್ರಚಾರಮಾಡುವಂತಹ ಯುವಕರನ್ನು ತರಬೇತು ಮಾಡುವಂತಹ ಕೇಂದ್ರವನ್ನು ತೆರೆಯಬೇಕೆಂಬುದೇ ನನ್ನ ಯೋಜನೆ. ನಮಗೆ ಇಂದು ಆವಶ್ಯಕವಾಗಿ ಬೇಕಾಗಿರುವುದು ಜನ, ಜನ. ಉಳಿದುವೆಲ್ಲ ಅನಂತರ ಬರುವುವು. ವೀರ್ಯವಂತ​, ತೇಜಸ್ವಿ, ಶ್ರದ್ಧಾಸಂಪನ್ನ, ಕೊನೆಯವರೆಗೂ ನಿಷ್ಕಪಟಿಗಳಾದ ಯುವಕರು ಬೇಕಾಗಿದ್ದಾರೆ. ಇಂತಹ ನೂರು ಜನರು ಸಿಕ್ಕಿದರೆ ಪ್ರಪಂಚವನ್ನೇ ಬದಲಾಯಿಸಬಹುದು. ಇಚ್ಛಾಶಕ್ತಿ ಎಲ್ಲಕ್ಕಿಂತ ಪ್ರಬಲವಾದುದು. ಅದರ ಎದುರಿಗೆ ಎಲ್ಲವೂ ಮಣಿಯಬೇಕು. ಅದು ಸ್ವಯಂ ಭಗವಂತನಿಂದ ಬರುವುದು. ಪರಿಶುದ್ಧವಾದ ಪ್ರಬಲ ಇಚ್ಛೆ ಸರ್ವಶಕ್ತವಾದುದು. ಅದರಲ್ಲಿ ನಿಮಗೆ ಶ್ರದ್ಧೆ ಇಲ್ಲವೆ? ನಿಮ್ಮ ಧರ್ಮದ ಮಹಾಸತ್ಯವನ್ನು ಜಗತ್ತಿಗೆ ಬೋಧಿಸಿ. ಪ್ರಪಂಚ ಅದಕ್ಕಾಗಿ ಕಾಯುತ್ತಿದೆ. ಹಲವು ಶತಮಾನಗಳಿಂದ ಜನರಿಗೆ ಹೀನ ಅವಸ್ಥೆಯ ವಿಷಯವನ್ನು ಹೇಳಿದ್ದಾಯಿತು. ಅವರು ಕೆಲಸಕ್ಕೆ ಬಾರದವರೆಂದು ಅವರಿಗೆ ಹೇಳಿದ್ದಾಯಿತು. ಪ್ರಪಂಚದಲ್ಲೆಲ್ಲಾ ಸಾಮಾನ್ಯರಿಗೆ ಅವರು ಮನುಷ್ಯರೇ ಅಲ್ಲವೆಂದು ಸಾರಿದ್ದಾಯಿತು. ಹಲವು ಶತಮಾನಗಳಿಂದ ಅವರು ಅಂಜುಕುಳಿಗಳಾಗಿ ಈಗ ಅವರು ಮೃಗ ಸಮಾನರಾಗಿರುವರು. ಆತ್ಮದ ವಿಷಯವನ್ನು ಕೇಳುವುದಕ್ಕೇ ಅವರಿಗೆ ಅವಕಾಶವಿರಲಿಲ್ಲ. ಅವರಲ್ಲಿ ಅತಿ ನೀಚರಲ್ಲಿಯೂ ಆತ್ಮವಿದೆ. ಅದಕ್ಕೆ ಹುಟ್ಟು ಸಾವು ಇಲ್ಲ. ಕತ್ತಿ ಅದನ್ನು ಕಡಿಯಲಾರದು, ಬೆಂಕಿ ದಹಿಸಲಾರದು, ಗಾಳಿ ಒಣಗಿಸಲಾರದು. ಆದಿ ಅಂತ್ಯಗಳಿಲ್ಲದ ಪರಿಶುದ್ಧವಾದ ಸರ್ವವ್ಯಾಪಿಯಾದ ಸರ್ವಶಕ್ತನಾದ ಆತ್ಮ ಅದು. ಮೊದಲು ತಮ್ಮಲ್ಲಿ ತಮಗೆ ಶ್ರದ್ಧೆ ಇರಲಿ. ಇಂಗ್ಲಿಷಿನವರಿಗೂ ನಿಮಗೂ ಇರುವ ವ್ಯತ್ಯಾಸಕ್ಕೆ ಕಾರಣವೇನು? ಅವರು ತಮ್ಮ ಧರ್ಮ ಕರ್ತವ್ಯ ಮುಂತಾದುವನ್ನು ಮಾತನಾಡಿಕೊಳ್ಳಲಿ. ನನಗೆ ಅದರ ವ್ಯತ್ಯಾಸ ಗೊತ್ತಾಗಿದೆ, ವ್ಯತ್ಯಾಸ ಇಲ್ಲಿದೆ: ಆಂಗ್ಲೇಯನಿಗೆ ಆತ್ಮವಿಶ್ವಾಸವಿದೆ. ನಿಮಗೆ ಇಲ್ಲ. ಆತ ಆಂಗ್ಲೇಯನಾದುದರಿಂದ ಏನನ್ನು ಬೇಕಾದರೂ ತಾನು ಮಾಡಬಲ್ಲೆನೆಂದು ತಿಳಿಯುವನು. ಆತನಲ್ಲಿರುವ ದೇವರನ್ನು ಅದು ಹೊರಗೆ ತರುವುದು. ತನಗೆ ತೋರಿದುದನ್ನು ಅವನು ಸಾಧಿಸಬಲ್ಲ. ಆದರೆ ಪ್ರತಿದಿನ ನಿಮಗೆ, ನಿಮ್ಮಿಂದ ಏನೂ ಆಗುವುದಿಲ್ಲ, ನಿಷ್ಪ್ರಯೋಜಕರು ನೀವು ಎಂದು ಬೋಧಿಸಲಾಗಿದೆ. ಅದರಂತೆಯೇ ನೀವು ಆಗುತ್ತಿರುವಿರಿ. ನಮಗೆ ಇಂದು ಬೇಕಾಗಿರುವುದು ಶಕ್ತಿ. ಅದಕ್ಕಾಗಿಯೇ ಆತ್ಮವಿಶ್ವಾಸವಿರಲಿ. ನಾವು ದುರ್ಬಲರಾಗಿರುವೆವು. ಅದಕ್ಕಾಗಿಯೇ ಈ ರಹಸ್ಯ, ಈ ಮಾಯಾಮಂತ್ರಗಳೆಲ್ಲ ನಮ್ಮನ್ನು ಆವರಿಸಿರುವುವು. ಇವುಗಳಲ್ಲಿ ಸ್ವಲ್ಪ ಸತ್ಯವಿರಬಹುದು. ಆದರೆ ಇವು ನಮ್ಮನ್ನು ಸರ್ವನಾಶಮಾಡಿವೆ. ಮೊದಲು ನಿಮ್ಮ ನರಗಳನ್ನು ದೃಢಪಡಿಸಿ. ನಮಗೆ ಇಂದು ಬೇಕಾಗಿರುವುದು ಕಬ್ಬಿಣದಂತಹ ಮಾಂಸಖಂಡಗಳು, ಉಕ್ಕಿನಂತಹ ನರಗಳು. ನಾವು ಬೇಕಾದಷ್ಟು ಅತ್ತಿರುವೆವು. ಇನ್ನು ಸಾಕು. ನಿಮ್ಮ ಕಾಲಿನ ಮೇಲೆ ನಿಂತು ಪುರುಷಸಿಂಹರಾಗಿ. ಪುರುಷಸಿಂಹರನ್ನು ಮಾಡುವ ಧರ್ಮ ನಮಗಿಂದು ಬೇಕಾಗಿದೆ. ಪುರುಷಸಿಂಹರನ್ನು ಮಾಡುವ ಸಿದ್ಧಾಂತ ನಮಗೆ ಬೇಕಾಗಿದೆ. ಸರ್ವತೋಮುಖವಾದ ಪುರುಷಸಿಂಹರನ್ನು ಮಾಡುವ ವಿದ್ಯಾಭ್ಯಾಸ ನಮಗೆ ಇಂದು ಬೇಕಾಗಿದೆ. ಸತ್ಯದ ಪರೀಕ್ಷೆ ಇಲ್ಲಿದೆ. ಯಾವುದು ನಿಮ್ಮನ್ನು ದೈಹಿಕವಾಗಿ, ಬೌದ್ಧಿಕವಾಗಿ, ಆಧ್ಯಾತ್ಮಿಕವಾಗಿ ದುರ್ಬಲರನ್ನಾಗಿ ಮಾಡುವುದೋ ಅದನ್ನು ವಿಷದಂತೆ ತ್ಯಜಿಸಿ. ಅದರಲ್ಲಿ ಜೀವವಿಲ್ಲ. ಅದು ಸತ್ಯವಾಗಿರಲಾರದು. ಸತ್ಯವು ಶಕ್ತಿವರ್ಧಕ, ಪರಿಶುದ್ಧ. ಸತ್ಯವೇ ಅನಂತಜ್ಞಾನ. ಸತ್ಯವು ಶಕ್ತಿ ಕೊಡಬೇಕು, ಬೆಳಕು ಕೊಡಬೇಕು, ಸ್ಫೂರ್ತಿ ಕೊಡಬೇಕು. ರಹಸ್ಯಗಳಲ್ಲಿ ಸ್ವಲ್ಪ ಸತ್ಯವಿದ್ದರೂ ಅವು ನಮ್ಮನ್ನು ಯಾವಾಗಲೂ ದುರ್ಬಲರನ್ನಾಗಿ ಮಾಡುವುವು. ನನಗೆ ಇಡೀ ಜೀವನದ ಅನುಭವವಿದೆ ಎಂದು ನಂಬಿ. ರಹಸ್ಯವು ದುರ್ಬಲರನ್ನಾಗಿ ಮಾಡುವುದೆಂದು ನಾನು ಸಿದ್ಧಾಂತ ಮಾಡಿರುವೆನು. ನಾನು ಭರತಖಂಡವನ್ನೆಲ್ಲಾ ಸಂಚರಿಸಿರುವೆನು. ಗುಹೆಗಳಲ್ಲಿ ಹುಡುಕಿರುವೆನು. ನಾನು ಹಿಮಾಲಯದಲ್ಲಿ ಸಂಚರಿಸಿರುವೆನು. ಜೀವನಾದ್ಯಂತವೂ ಅಲ್ಲೇ ಜನಿಸಿದ್ದ ಜನರನ್ನು ನಾನು ನೋಡಿರುವೆನು. ನಾನು ನನ್ನ ದೇಶವನ್ನು ಪ್ರೀತಿಸುತ್ತೇನೆ. ಈಗಿರುವ ಸ್ಥಿತಿಗಿಂತ ನೀವು ದುರ್ಬಲರಾಗುವುದನ್ನೂ ಅಧೋಗತಿಗೆ ಬರುವುದನ್ನೂ ನಾನು ಸಹಿಸಲಾರೆ. ಆದಕಾರಣವೆ ನಾನು ನಿಮಗಾಗಿ, ಸತ್ಯಕ್ಕಾಗಿ, ಅಧೋಗತಿಯ ಕಡೆಗೆ ಜಾರುತ್ತಿರುವ ಭರತಖಂಡವನ್ನು ತಡೆಯಲು, “ನಿಲ್ಲಿ, ಇನ್ನು ಜಾರಬೇಡಿ” ಎಂದು ಕೂಗಬೇಕಾಗಿದೆ. ದುರ್ಬಲರನ್ನಾಗಿ ಮಾಡುವ ರಹಸ್ಯವಿದ್ಯೆಯನ್ನು ತ್ಯಜಿಸಿ ಧೀರರಾಗಿ. ನಿಮ್ಮ ಉಪನಿಷತ್ತುಗಳಿಗೆ ಹೋಗಿ. ಪ್ರಕಾಶಮಾನವಾದ ಶಕ್ತಿವರ್ಧಕ ದಿವ್ಯದರ್ಶನಶಾಸ್ತ್ರಗಳು ಅವು. ಉಪನಿಷತ್ತುಗಳ ಈ ದರ್ಶನವನ್ನು ಸ್ವೀಕರಿಸಿ. ಮಹಾನ್​ ಸತ್ಯಗಳು ಜಗತ್ತಿನಲ್ಲಿ ಅತ್ಯಂತ ಸಹಜವಾದುವುಗಳು; ನಿಮ್ಮ ಅಸ್ತಿತ್ವದಷ್ಟೇ ಸರಳವಾದುವು. ಉಪನಿಷತ್ತುಗಳ ಸತ್ಯ ನಿಮ್ಮ ಮುಂದೆ ಇದೆ. ಅದನ್ನು ಸ್ವೀಕರಿಸಿ, ಅದನ್ನು ಅನುಷ್ಠಾನಕ್ಕೆ ತನ್ನಿ. ಅದರಿಂದ ಭರತಖಂಡದ ಉದ್ಧಾರವಾಗುವುದು.

\vskip   4pt

ಇನ್ನೊಂದು ಮಾತು. ಅದಾದ ಮೇಲೆ ಉಪನ್ಯಾಸವನ್ನು ಮುಕ್ತಾಯ ಗೊಳಿಸುವೆನು. ದೇಶಭಕ್ತಿಯ ವಿಷಯವಾಗಿ ಮಾತನಾಡುವರು. ನನಗೂ ದೇಶ ಭಕ್ತಿಯಲ್ಲಿ ವಿಶ್ವಾಸವಿದೆ. ದೇಶಭಕ್ತಿಯನ್ನು ಕುರಿತಂತೆ ನನ್ನದೇ ಆದ ಒಂದು ಆದರ್ಶವಿದೆ. ಮಹತ್ಕಾರ್ಯಸಾಧನೆಗೆ ಮೂರು ವಿಷಯಗಳು ಅತ್ಯಾವಶ್ಯಕ. ಮೊದಲನೆಯದೇ ಹೃದಯವಂತಿಕೆ. ಬುದ್ಧಿಯಲ್ಲಿ ಮತ್ತು ವಿಚಾರಶಕ್ತಿಯಲ್ಲಿ ಏನಿದೆ? ಕೆಲವು ಹೆಜ್ಜೆಗಳು ಹೋಗಿ ಅಲ್ಲಿ ನಿಲ್ಲುವುವು. ಸ್ಫೂರ್ತಿ ಬರುವುದು ಹೃದಯದಿಂದ. ಪ್ರೇಮವು ಅಸಾಧ್ಯವಾದ ಬಾಗಿಲನ್ನೂ ತೆರೆಯುವುದು. ಪ್ರಪಂಚದ ರಹಸ್ಯಕ್ಕೆಲ್ಲಾ ಪ್ರೇಮವೇ ಬಾಗಿಲು. ಭಾವೀ ದೇಶಭಕ್ತರೇ, ಸಮಾಜ ಸುಧಾರಕರೇ, ಮೊದಲು ದೇಶಭಕ್ತಿಯನ್ನು ಹೃದಯದಲ್ಲಿ ಅನುಭವಿಸಿ. ಕೋಟ್ಯಂತರ ದೇವಸಂತಾನರು ಮತ್ತು ಋಷಿಸಂತಾನರು ಪಶುಸಂತಾನರಾಗಿರುವರೆಂದು ನಿಮಗೆ ಹೃದಯದಲ್ಲಿ ವ್ಯಥೆ ಇದೆಯೇ? ಕೋಟ್ಯಂತರ ಜನರು ಈಗ ಉಪವಾಸದಿಂದ ನರಳುತ್ತಿರುವರು, ಹಿಂದಿನಿಂದಲೂ ಕೋಟ್ಯಂತರ ಜನರು ಅದರಿಂದ ನರಳುತ್ತಿದ್ದರು ಎಂಬುದು ಗೊತ್ತೇ? ಅಜ್ಞಾನವು ಕಾರ್ಮೋಡದಂತೆ ಭಾರತವನ್ನೆಲ್ಲಾ ವ್ಯಾಪಿಸಿದೆ ಎಂಬುದು ನಿಮಗೆ ಗೊತ್ತಾಗುತ್ತಿದೆಯೆ? ಇದರಿಂದ ನೀವು ಅಶಾಂತರಾಗಿದ್ದೀರಾ? ಇದರಿಂದ ನಿಮ್ಮ ನಿದ್ರೆಗೆ ಭಂಗ ಬಂದಿದೆಯೆ? ಇದು ನಿಮ್ಮ ರಕ್ತದಲ್ಲಿ ವ್ಯಾಪಿಸಿದೆಯೆ? ನಾಡಿಯಲ್ಲಿ ಸಂಚರಿಸಿ ನಿಮ್ಮ ಹೃದಯಚಲನೆಯೊಂದಿಗೆ ಸ್ಪಂದಿಸುತ್ತಿದೆಯೆ? ಇದು ನಿಮ್ಮನ್ನು ಉನ್ಮತ್ತರನ್ನಾಗಿ ಮಾಡುತ್ತಿದೆಯೆ? ಈ ಸರ್ವನಾಶದ ದುಃಖ ನಿಮ್ಮನ್ನು ವ್ಯಾಪಿಸಿ, ನಿಮ್ಮ ಕೀರ್ತಿ ಯಶಸ್ಸು ಹೆಂಡಿರು ಮಕ್ಕಳು ಆಸ್ತಿ ಮತ್ತು ದೇಹವನ್ನು ನೀವು ಮರೆತಿರುವಿರಾ? ನೀವು ಇದನ್ನು ಮಾಡಿರುವಿರಾ? ದೇಶಭಕ್ತರಾಗುವುದಕ್ಕೆ ಇದೇ ಮೊದಲನೆಯ ಹೆಜ್ಜೆ. ನಿಮ್ಮಲ್ಲಿ ಅನೇಕರು ತಿಳಿದುಕೊಂಡಿರುವಂತೆ ನಾನು ವಿಶ್ವಧರ್ಮ ಸಮ್ಮೇಳನಕ್ಕಾಗಿ ಅಮೆರಿಕಾ ದೇಶಕ್ಕೆ ಹೋಗಲಿಲ್ಲ. ಈ ಉದ್ವೇಗದೈತ್ಯ ನನ್ನ ಹೃದಯವನ್ನು ಮೆಟ್ಟಿಕೊಂಡು ಪ್ರಚೋದಿಸುತ್ತಿತ್ತು. ನನ್ನ ದೇಶೀಯರಿಗೆ ಸೇವೆ ಮಾಡುವ ದಾರಿ ತಿಳಿಯದೆ ಹನ್ನೆರಡು ವರುಷಗಳವರೆಗೆ ದೇಶದಲ್ಲಿ ಸಂಚರಿಸಿದೆ. ಆದಕಾರಣವೆ ನಾನು ಅಮೆರಿಕಾದೇಶಕ್ಕೆ ಹೋದೆ. ಇದು ನನ್ನ ಪರಿಚಯವಿದ್ದ ನಿಮಗೆಲ್ಲಾ ಗೊತ್ತಿದೆ. ವಿಶ್ವಧರ್ಮ ಸಮ್ಮೇಳನವನ್ನು ಯಾರು ಲಕ್ಷಿಸಿದರು? ನನ್ನ ಸ್ವಂತ ದೇಶೀಯರು ಪ್ರತಿದಿನ ಅಧಃಪಾತಾಳಕ್ಕೆ ಇಳಿಯುತ್ತಿದ್ದರು. ಅವರನ್ನು ಯಾರು ಲೆಕ್ಕಿಸಿದರು? ಇದೇ ನನ್ನ ಪ್ರಥಮ ಹೆಜ್ಜೆ.

\vskip   4pt

ನೀವು ಹೃದಯದಲ್ಲಿ ಅನುಭವಿಸಬಹುದು. ಆ ಉದ್ವೇಗವನ್ನು ಬರಿಯ ಬಾಯಿ ಮಾತಿನಲ್ಲಿ ವ್ಯರ್ಥಗೊಳಿಸದೆ, ಈ ಸಮಸ್ಯೆಯಿಂದ ಪಾರಾಗುವುದಕ್ಕೆ ಯಾವುದಾದರೊಂದು ಮಾರ್ಗವನ್ನು ಕಂಡುಹಿಡಿದಿರುವಿರಾ? ಅವರನ್ನು ದೂರದೆ ಸಹಾಯವನ್ನು ನೀಡುವ, ವರ್ತಮಾನ ಮೃತ್ಯುಸ್ಥಿತಿಯಿಂದ ಅವರನ್ನು ಮೇಲೆತ್ತಿ ಅವರ ದುಃಖವನ್ನು ಶಮನಗೊಳಿಸುವ ಸಹಾನುಭೂತಿಯ ನುಡಿಯನ್ನು ಉಚ್ಚರಿಸಬಲ್ಲಿರಾ?

\vskip   4pt

ಇಷ್ಟೇ ಅಲ್ಲ. ಪರ್ವತೋಪಮ ಆತಂಕಗಳನ್ನು ದಾಟಬಲ್ಲ ಇಚ್ಛಾಶಕ್ತಿ ನಿಮ್ಮಲ್ಲಿದೆಯೆ? ಇಡೀ ಪ್ರಪಂಚವೇ ಕತ್ತಿಯನ್ನು ಹಿರಿದು ನಿಮ್ಮನ್ನು ಎದುರಿಸಿದರೂ, ಸರಿ ಎಂದು\break ತೋರಿದುದನ್ನು ಮಾಡುವ ಧೈರ್ಯ ನಿಮ್ಮಲ್ಲಿದೆಯೆ? ನಿಮ್ಮ ಹೆಂಡತಿ ಮಕ್ಕಳು ಅದನ್ನು ವಿರೋಧಿಸಿದರೂ, ನಿಮ್ಮ ದ್ರವ್ಯ ಹೋದರೂ, ಕೀರ್ತಿ ನಶಿಸಿದರೂ, ಐಶ್ವರ್ಯ ಮಾಯವಾದರೂ, ನೀವು ಅದನ್ನು ಬಿಡದೆ ಇರಬಲ್ಲಿರಾ? ಅದನ್ನೇ ಹಿಂಬಾಲಿಸುತ್ತಾ, ಎಡೆಬಿಡದೆ ಗುರಿಯೆಡೆಗೆ ಧಾವಿಸಬಲ್ಲಿರಾ? ಭರ್ತೃಹರಿ ಹೇಳುವಂತೆ,

\begin{verse}
\textbf{ನಿಂದಂತು ನೀತಿನಿಪುಣಾ ಯದಿ ವಾ ಸ್ತುವಂತು}\\\textbf{ಲಕ್ಷ್ಮೀಃ ಸಮಾವಿಶತು ಗಚ್ಛತು ವಾ ಯಥೇಷ್ಟಮ್​~।}\\\textbf{ಅದ್ವೈವ ವಾ ಮರಣಮಸ್ತು ಯುಗಾಂತರೇ ವಾ}\\\textbf{ನ್ಯಾಯ್ಯಾತ್​ ಪಥಃ ಪ್ರವಿಚಲಂತಿ ಪದಂ ನ ಧೀರಾಃ~॥}
\end{verse}

\vskip   4pt

“ನೀತಿ ನಿಪುಣರು ಹೊಗಳಲಿ ಅಥವಾ ತೆಗಳಲಿ, ಲಕ್ಷ್ಮಿ ಒಲಿಯಲಿ ಅಥವಾ ತ್ಯಜಿಸಲಿ, ಮೃತ್ಯು ಇಂದೇ ಬರಲಿ ಅಥವಾ ಯುಗಾಂತರಗಳ ಮೇಲೆ ಬರಲಿ; ಧೀರರು ಮಾತ್ರ ಸತ್ಯ ಪಥದಿಂದ ವಿಮುಖರಾಗುವುದಿಲ್ಲ.” ಈ ದೃಢತೆ ನಿಮ್ಮಲ್ಲಿ ಇದೆಯೆ? ನಿಮ್ಮಲ್ಲಿ ಈ ಮೂರು ಗುಣಗಳಿದ್ದರೆ ಪ್ರತಿಯೊಬ್ಬರೂ ಮಹಾ ಅದ್ಭುತ ಕಾರ್ಯವನ್ನು ಸಾಧಿಸಬಲ್ಲಿರಿ. ನೀವು ವೃತ್ತಪತ್ರಿಕೆಯಲ್ಲಿ ಬರೆಯಬೇಕಾಗಿಲ್ಲ, ಉಪನ್ಯಾಸ ಮಾಡಬೇಕಿಲ್ಲ. ನಿಮ್ಮ ಮುಖವೇ ಕಾಂತಿಯಿಂದ ಕೋರೈಸುವುದು. ನೀವು ಒಂದು ಗುಹೆಯಲ್ಲಿ ವಾಸಿಸುತ್ತಿದ್ದರೂ ನಿಮ್ಮ ಭಾವನೆಗಳು ಕಲ್ಲುಬಂಡೆಯನ್ನು ತೂರಿ ಪ್ರಪಂಚದಲ್ಲೆಲ್ಲಾ ನೂರಾರು ವರ್ಷಗಳವರೆಗೆ ಸ್ಪಂದಿಸುವುವು. ಯಾವುದಾದರೂ ಅದಕ್ಕೆ ಸೂಕ್ತವಾದ ವ್ಯಕ್ತಿಯ ಆಶ್ರಯ ಸಿಕ್ಕಿ ಅವನ ಮೂಲಕ\break ಅನುಷ್ಠಾನಕ್ಕೆ ಬರುವವರೆಗೂ ಅದು ಸಂಚರಿಸುತ್ತಿರುವುದು. ಆಲೋಚನೆ ಮತ್ತು ನಿಷ್ಕಪಟ ಪರಿಶುದ್ಧ ಉದ್ದೇಶದ ಶಕ್ತಿ ಇದು.

\vskip   4pt

ನಿಮಗೆ ಆಗಲೇ ತಡವಾಗುತ್ತಿದೆ ಎಂದು ಶಂಕಿಸುತ್ತಿರುವೆನು. ಆದರೆ ಇದೇ ಕೊನೆಯ ಮಾತು.

ನನ್ನ ಸ್ವದೇಶವಾಸಿಗಳೇ, ಸ್ನೇಹಿತರೇ, ಮಕ್ಕಳೇ ನಮ್ಮ ರಾಷ್ಟ್ರನೌಕೆಯು ಭವಸಾಗರದ ಮೇಲೆ ಕೋಟ್ಯಂತರ ಜೀವಿಗಳನ್ನು ಪಾರುಗಾಣಿಸಿದೆ. ಹಲವು ಶತಾಬ್ದಗಳಿಂದ ಕಡಲಿನ ಮೇಲೆ ಸಂಚರಿಸಿ ಕೋಟ್ಯಂತರ ಜೀವಿಗಳನ್ನು ಅಮೃತತ್ವದ ಕಡೆಗೆ ಒಯ್ದಿದೆ. ಇಂದು ಬಹುಶಃ ನಿಮ್ಮ ಅಜಾಗರೂಕತೆಯಿಂದಲೇ ಹಡಗು ಶಿಥಿಲವಾಗಿದೆ. ರಂಧ್ರ ಬಿದ್ದಿದೆ. ಇದನ್ನು ಈಗ ನಿಂದಿಸುವಿರಾ? ಪ್ರಪಂಚದಲ್ಲಿ ಎಲ್ಲಕ್ಕಿಂತ ಮಹತ್ಕಾರ್ಯವನ್ನು ಸಾಧಿಸಿದ ಪವಿತ್ರತಮ ವಸ್ತುವಿನ ಮೇಲೆ ನಿಂದೆಯ ಬಾಣವನ್ನು ಕರೆಯುವುದು ತರವೆ? ಈ ಜನಾಂಗದ ಹಡಗಿನಲ್ಲಿ, ಅಂದರೆ ನಮ್ಮ ಜನಾಂಗದಲ್ಲಿ, ಕುಂದು ಕೊರತೆಗಳಿದ್ದರೆ ನಾವು ಅದರ ಮಕ್ಕಳು, ಹೋಗಿ ಆ ರಂಧ್ರವನ್ನು ಮುಚ್ಚೋಣ. ನಮ್ಮ ಹೃದಯದ ರಕ್ತವನ್ನಾದರೂ ಕೊಟ್ಟು ಸಂತೋಷದಿಂದ ಈ ಕಾರ್ಯವನ್ನು ಸಾಧಿಸೋಣ. ಸಾಧ್ಯವಿಲ್ಲದೆ ಇದ್ದರೆ ಪ್ರಾಣವನ್ನು ಅರ್ಪಿಸೋಣ. ನಮ್ಮ ತಲೆಯನ್ನಾದರೂ ಛೇದಿಸಿ ರಂಧ್ರವನ್ನು ಮುಚ್ಚೋಣ. ಆದರೆ ಎಂದಿಗೂ ನಿಂದಿಸಬೇಡಿ. ಈ ಸಮಾಜದ ಮೇಲೆ ಒಂದಾದರೂ ಕಟುಟೀಕೆ ಇಲ್ಲದಿರಲಿ. ಇದರ ಗತ ಮಾಹಾತ್ಮ್ಯೆಗೆ ನಾನು ಇದನ್ನು ಪ್ರೀತಿಸುತ್ತೇನೆ. ನೀವೆಲ್ಲ ಭಗವಂತನ ಮಕ್ಕಳು, ಮಹಿಮಾವಂತ ಪೂರ್ವಿಕರ ಸಂತಾನರು ಎಂದು ಪ್ರೀತಿಸುತ್ತೇನೆ. ನಾನು ಹೇಗೆ ನಿಮ್ಮನ್ನು ಶಪಿಸಬಲ್ಲೆ? ಎಂದಿಗೂ ಇಲ್ಲ. ನಿಮಗೆ ಸರ್ವಪ್ರಕಾರದಿಂದಲೂ ಕಲ್ಯಾಣವಾಗಲಿ. ಮಕ್ಕಳೇ, ನನ್ನ ಯೋಜನೆಗಳನ್ನೆಲ್ಲ ನಿಮಗೆ ಹೇಳಲು ಬಂದಿರುವೆನು. ನೀವು ಅದನ್ನು ಆಲಿಸಿದರೆ ನಿಮ್ಮೊಂದಿಗೆ ಕೆಲಸ ಮಾಡಲು ಸಿದ್ಧನಾಗಿರುವೆನು. ನೀವು ಕೇಳದೆ ನನ್ನನ್ನು ಭರತಖಂಡದಿಂದ ನೂಕಿದರೂ ನಾನು ಪುನಃ ಬಂದು ನಾವೆಲ್ಲಾ ಮುಳುಗುತ್ತಿರುವೆವು ಎಂದು ಹೇಳುವೆನು. ನಾನು ನಿಮ್ಮ ಸಮೀಪದಲ್ಲಿರಲು ಬಂದಿರುವೆನು. ನಾವೆಲ್ಲಾ ಮುಳುಗಲೇಬೇಕಾದರೆ ಒಟ್ಟಿಗೆ ಮುಳುಗೋಣ. ಎಂದಿಗೂ ಶಾಪವನ್ನು ಉಚ್ಚರಿಸದೆ ಇರೋಣ.

