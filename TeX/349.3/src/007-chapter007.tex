
\chapter{ಮನಮಧುರೆಯ ಬಿನ್ನವತ್ತಳೆಗೆ ಉತ್ತರ}

ಮನಮಧುರೆಯಲ್ಲಿ ಶಿವಗಂಗೆ ಮತ್ತು ಮನಮಧುರೆಯ ಜಮೀನುದಾರರು ಮತ್ತು ಇತರ ಪೌರರು ಸ್ವಾಮೀಜಿಯವರಿಗೆ ಈ ಕೆಳಗಿನ ಬಿನ್ನವತ್ತಳೆಯನ್ನು ಅರ್ಪಿಸಿದರು.

\textbf{ಪರಮಪೂಜ್ಯ ಸ್ವಾಮೀಜಿಯವರೆ,}

ಮನಮಧುರೆ ಮತ್ತು ಶಿವಗಂಗೆಯ ಜಮೀನುದಾರರು ಮತ್ತು ಇತರ ಪೌರರಾದ ನಾವು ತಮಗೆ ಅತ್ಯಂತ ಹೃತ್ಪೂರ್ವಕವಾದ ಸ್ವಾಗತವನ್ನು ಬಯಸುತ್ತೇವೆ. ನಮ್ಮ ಜೀವನದ ಅತ್ಯಂತ ಉತ್ಸಾಹಪೂರ್ಣ ಕ್ಷಣಗಳಲ್ಲಿ, ನಾವು ಆಶಾಪೂರ್ಣ ಕನಸುಗಳನ್ನು ಕಾಣುತ್ತಿರುವಾಗ, ನಮ್ಮ ಹೃದಯಕ್ಕೆ ಅತ್ಯಂತ ಸಮೀಪದವರಾದ ತಮ್ಮ ಭೌತಿಕ ಸಾಮೀಪ್ಯವನ್ನು ನಾವು ನಿರೀಕ್ಷಿಸಿಯೇ ಇರಲಿಲ್ಲ. ತಾವು ಶಿವಗಂಗೆಗೆ ಬರುವುದು ಅಸಾಧ್ಯವೆಂಬ ತಂತಿಸಮಾಚಾರವನ್ನು ಕೇಳಿ ನಾವು ಖಿನ್ನಮನಸ್ಕರಾಗಿದ್ದೆವು. ಮುಂದೆ ತಾವು ಬರುವ ಆಶಾಕಿರಣವು ಕಾಣಿಸಿಕೊಳ್ಳದೆ ಇದ್ದಿದ್ದರೆ ನಮ್ಮ ನಿರಾಸೆ ಅತ್ಯಂತ ತೀವ್ರವಾಗಿರುತ್ತಿತ್ತು. ತಮ್ಮ ಸಾನ್ನಿಧ್ಯದಿಂದ ನಮ್ಮ ನಗರವನ್ನು ಪಾವನಗೊಳಿಸಲು ತಾವು ಒಪ್ಪಿದುದನ್ನು ನಾವು ಮೊದಲು ಕೇಳಿದಾಗ ನಮ್ಮ ಜೀವನದಲ್ಲಿ ಪರಮೋಚ್ಚವಾದುದನ್ನು ಸಾಧಿಸಿದಷ್ಟು ಆನಂದವುಂಟಾಯಿತು. ತಾವು ಬರುವುದಕ್ಕೆ ನಿರಾಕರಿಸಿದಾಗ ತಮ್ಮನ್ನು ನೋಡಲಾಗುವುದಿಲ್ಲವೇನೋ ಎಂಬ ಭಯ ನಮ್ಮನ್ನು ಬಹಳವಾಗಿ ಪೀಡಿಸುತ್ತಿತ್ತು. ಆದರೆ ನಮ್ಮ ಕಳಕಳಿಯ ಪ್ರಾರ್ಥನೆಗೆ ಒಪ್ಪಿಗೆಯಿತ್ತು ತಾವು ನಮ್ಮನ್ನು ಅನುಗ್ರಹಿಸಿದ್ದೀರಿ.

ದೂರಪ್ರಯಾಣದ ಅಸಾಧ್ಯ ಕಷ್ಟನಷ್ಟಗಳನ್ನು ಎದುರಿಸಿ ಉದಾತ್ತ ತ್ಯಾಗ ಮನೋಭಾವದಿಂದ ಪೂರ್ವದ ಶ್ರೇಷ್ಠತಮ ಸಂದೇಶವನ್ನು ಪಶ್ಚಿಮಕ್ಕೆ ನೀಡಿದುದು, ಈ ಕಾರ್ಯಸಾಧನೆಯಲ್ಲಿ ತಾವು ತೋರಿದ ಅಪಾರ ದಕ್ಷತೆ, ಮತ್ತು ಈ ಲೋಕ ಕಲ್ಯಾಣ ಪ್ರಯತ್ನಗಳಿಗೆ ಲಭಿಸಿದ ಅದ್ಭುತ ಅಸದೃಶ ಯಶಸ್ಸು –ಇವು ತಮಗೆ ಎಂದೆಂದಿಗೂ ಕ್ಷೀಣಿಸದ ಕೀರ್ತಿಯನ್ನು ನೀಡಿದೆ. ಕೇವಲ ಲೌಕಿಕ ಆವಶ್ಯಕತೆಗಳನ್ನು ಪೂರೈಸುವ ಪಾಶ್ಚಿಮಾತ್ಯ ಜಡವಾದವು ಭಾರತೀಯ ಧಾರ್ಮಿಕ ಭಿತ್ತಿಯನ್ನು ಸೀಳುತ್ತಿರುವಾಗ, ನಮ್ಮ ಋಷಿಮುನಿಗಳ ಹೇಳಿಕೆಗಳು ಮತ್ತು ಬರಹ\-ಗಳಿಗೆ ಕೊನೆಗಾಲವು ಪ್ರಾಪ್ತವಾಗುತ್ತಿರುವಾಗ ತಮ್ಮಂತಹ ಆಚಾರ್ಯರ ಆಗಮನದಿಂದ ಧಾರ್ಮಿಕ ಪ್ರಗತಿಯ ಇತಿಹಾಸದಲ್ಲಿ ಒಂದು ಹೊಸ ಯುಗವು ಪ್ರಾರಂಭವಾದಂತಾಗಿದೆ. ಮುಂದೆ ಕಾಲವು ತುಂಬಿಬಂದಾಗ ಭಾರತೀಯ ದರ್ಶನವೆಂಬ ಚಿನ್ನವನ್ನು ಮುಸುಕಿರುವ ಧೂಳನ್ನು ಹೋಗಲಾಡಿಸಿ, ಅದಕ್ಕೆ ತಮ್ಮ ತೀಕ್ಷ್ಣಬುದ್ಧಿಶಕ್ತಿಯ ಟಂಕಸಾಲೆಯಲ್ಲಿ ಹೊಸ ರೂಪವನ್ನು ಕೊಟ್ಟು ಇಡೀ ವಿಶ್ವದಲ್ಲಿ ಅದನ್ನು ಚಲಾವಣೆಗೊಳಿಸುತ್ತೀರಿ ಎಂಬ ಭರವಸೆಯನ್ನು ಇಟ್ಟುಕೊಂಡಿದ್ದೇವೆ. ಚಿಕಾಗೊ ಧರ್ಮಸಮ್ಮೇಳನದಲ್ಲಿ ಭಿನ್ನ ಭಿನ್ನ ಧರ್ಮಾನುಯಾಯಿಗಳ ಮುಂದೆ ಭಾರತೀಯ ಧರ್ಮಧ್ವಜವನ್ನು ಉದಾತ್ತ ರೀತಿಯಲ್ಲಿ ಯಶಸ್ವಿಯಾಗಿ ಹಾರಿಸಿದುದು, ತಾವು ಮುಂದೆ ಅತ್ಯಲ್ಪ ಕಾಲದಲ್ಲಿಯೇ–ರಾಜಕೀಯ ಕ್ಷೇತ್ರದ ತಮ್ಮ ಸಮಕಾಲೀನರಂತೆಯೆ – ಸೂರ್ಯ ಮುಳುಗದ ಸಾಮ್ರಾಜ್ಯದ (ಇಂಗ್ಲೆಂಡ್​) ಮೇಲೆ ಪ್ರಭುತ್ವವನ್ನು ಸಾಧಿಸುತ್ತೀರಿ ಎಂಬ ಭರವಸೆಯನ್ನು ನಮ್ಮಲ್ಲಿ ಉಂಟುಮಾಡುತ್ತಿದೆ. ಆದರೆ ಒಂದು ವ್ಯತ್ಯಾಸವೇನೆಂದರೆ ಅವರದು ಭೌತಿಕ ಸಾಮ್ರಾಜ್ಯ, ತಮ್ಮದು ಮಾನಸಿಕ ಸಾಮ್ರಾಜ್ಯ. ಬ್ರಿಟಿಷ್​ ಸಾಮ್ರಾಜ್ಯವು ತನ್ನ ವೈಶಾಲ್ಯದಲ್ಲಿ ಮತ್ತು ಜನ ಹಿತಕಾರಿ ಆಳ್ವಿಕೆಯಲ್ಲಿ ರಾಜಕೀಯ\- ಇತಿಹಾಸದಲ್ಲಿಯೇ ಒಂದು ಹೊಸ ದಾಖಲೆಯನ್ನು ಸೃಷ್ಟಿಸಿರುವಂತೆ, ತಾವು ಅತ್ಯಂತ ಅನಾಸಕ್ತಿಯಿಂದ ಸ್ವೀಕರಿಸಿರುವ ಪ್ರೀತಿಯ ಕಾರ್ಯವನ್ನು ಸಮಾಪ್ತಿಗೊಳಿಸಲು ಭಗವಂತನು ತಮಗೆ ದೀರ್ಘಾಯುಷ್ಯವನ್ನು ದಯಪಾಲಿಸಲಿ ಎಂದೂ, ಮತ್ತು ತಾವು ಹಿಂದಿನ ಎಲ್ಲ ಆಧ್ಯಾತ್ಮಿಕ ಆಚಾರ್ಯರನ್ನು ಮೀರುವಂತಾಗಲಿ ಎಂದೂ ಆಶಿಸುತ್ತೇವೆ.

\begin{longtable}[r]{@{}r@{}}
ಪರಮಪೂಜ್ಯರಾದ ತಮ್ಮ \\
ಅತ್ಯಂತ ಕರ್ತವ್ಯನಿಷ್ಠ ಸೇವಕರು. \\
\end{longtable}

ಸ್ವಾಮೀಜಿಯವರು ಈ ಕೆಳಗಿನಂತೆ ಉತ್ತರಿಸಿದರು:

ಹಾರ್ದಿಕವೂ, ದಯಾಪೂರ್ಣವೂ ಆದ ಅಭಿನಂದನೆಯ ಮೂಲಕ ನೀವು ನನಗೆ ನೀಡಿದ ಸುಖಾಗಮನಕ್ಕೆ ತಕ್ಕ ಕೃತಜ್ಞತೆಯನ್ನು ನಾನು ವ್ಯಕ್ತಪಡಿಸಲಾರೆ. ನಾನು ಎಷ್ಟೇ ಇಷ್ಟಪಟ್ಟರೂ ಈಗ ದೊಡ್ಡ ಉಪನ್ಯಾಸವನ್ನು ನೀಡುವ ಸ್ಥಿತಿಯಲ್ಲಿ ಇಲ್ಲ. ಸಂಸ್ಕೃತ ಬಲ್ಲ ನಮ್ಮ ಸ್ನೇಹಿತರು ಬೇಕಾದಷ್ಟು ಸುಂದರವಾದ ವಿಶೇಷಣಗಳ ಮಾಲೆಯನ್ನೇ ನನ್ನ ಮೇಲೆ ಹೇರಿದ್ದಾರೆ. ಆದರೂ ನನಗೆ ಒಂದು ದೇಹ ಇದೆ, ಮತ್ತು ಅದು ಜಡವಸ್ತುವಿನ ನಿಯಮವನ್ನು ಅನುಸರಿಸುತ್ತದೆ. ಆದ್ದರಿಂದ ಸ್ಥೂಲ ಕಾಯಕ್ಕೆ ದಣಿವು ಎಂಬುದು ಸ್ವಾಭಾವಿಕ.

ಪಾಶ್ಚಾತ್ಯ ದೇಶಗಳಲ್ಲಿ ನಾನು ಮಾಡಿದ ಅತ್ಯಲ್ಪ ಕೆಲಸಕ್ಕಾಗಿ ದೇಶಾದ್ಯಂತವೂ ತೋರುತ್ತಿರುವ ಅಪಾರವಾದ ಸಂತೋಷ ಮತ್ತು ಮೆಚ್ಚುಗೆಯನ್ನು ಕಂಡು ನನಗೆ ಆನಂದವಾಗಿದೆ. ನಾನು ಇದನ್ನು ಈ ದೃಷ್ಟಿಯಿಂದ ಮಾತ್ರ ನೋಡುತ್ತೇನೆ: ಭವಿಷ್ಯದಲ್ಲಿ ಬರುವ ಮಹಾವ್ಯಕ್ತಿಗಳಿಗೆ ಇದನ್ನೆಲ್ಲಾ ಸಮರ್ಪಿಸುತ್ತೇನೆ. ನಾನು ಮಾಡಿದ ಅತ್ಯಲ್ಪ ಕಾರ್ಯಕ್ಕೇ ಇಷ್ಟೊಂದು ಪ್ರಶಂಸೆ ದೊರೆತಿರುವಾಗ, ನಮ್ಮ ಅನಂತರ ಬರುವ ಧರ್ಮವೀರರನ್ನು, ಜಗತ್ತನ್ನೇ ಜಾಗೃತಗೊಳಿಸಬಲ್ಲ ಮಹಾವ್ಯಕ್ತಿಗಳನ್ನು, ನಮ್ಮ ದೇಶ ಎಷ್ಟೊಂದು ಪ್ರಶಂಸಿಸಬಹುದು! ಭಾರತವು ಧರ್ಮಭೂಮಿ. ಹಿಂದೂವಿಗೆ ಧರ್ಮ ಒಂದೇ ತಿಳಿಯುವುದು. ಶತಮಾನಗಳ ನಮ್ಮ ಶಿಕ್ಷಣ ಆ ದಾರಿಯಲ್ಲೇ ಸಾಗಿದೆ. ಇದರ ಪರಿಣಾಮವಾಗಿ ಧರ್ಮವೇ ನಮ್ಮ ಜೀವನದ ಚರಮಗುರಿಯಾಗಿದೆ. ಇದು ಸತ್ಯವೆಂಬುದು ನಿಮಗೆಲ್ಲಾ ಗೊತ್ತು. ಪ್ರತಿಯೊಬ್ಬನೂ ವರ್ತಕನಾಗಬೇಕಾಗಿಲ್ಲ; ಪ್ರತಿಯೊಬ್ಬನೂ ಶಾಲಾ ಉಪಾಧ್ಯಾಯನಾಗಬೇಕಾಗಿಲ್ಲ; ಪ್ರತಿಯೊಬ್ಬನೂ ಯೋಧನಾಗಬೇಕಾಗಿಲ್ಲ. ಆದರೆ ಜಗತ್ತಿನಲ್ಲಿ ಪ್ರತಿಯೊಂದು ದೇಶವೂ ವಿಶ್ವಸಾಮರಸ್ಯಕ್ಕೆ ತನ್ನ ಪಾಲಿನದನ್ನು ನಿವೇದಿಸಬೇಕಾಗಿದೆ.

ಭಗವಂತನ ಇಚ್ಛೆಯಂತೆ ಬಹುಶಃ ನಾವು ಈ ರಾಷ್ಟ್ರಗಳ ಸ್ವರ ಸಮ್ಮೇಳನದಲ್ಲಿ ಅಧ್ಯಾತ್ಮಗೀತೆಯನ್ನು ಹಾಡಬೇಕಾಗಿದೆ. ಯಾವ ದೇಶವಾದರೂ ಹೆಮ್ಮೆ ಪಡಬಹುದಾದ ನಮ್ಮ ಪ್ರಖ್ಯಾತ ಪೂರ್ವಿಕರು ನಮಗಿತ್ತ ಆ ಭವ್ಯಪರಂಪರೆ ಇನ್ನೂ ಮಾಯವಾಗಿಲ್ಲವೆಂಬುದನ್ನು ನೋಡಿ ನನಗೆ ಅತ್ಯಾನಂದವಾಗುತ್ತದೆ. ಜನಾಂಗದ ಭವಿಷ್ಯದಲ್ಲಿ ಭರವಸೆಯನ್ನೂ, ದೃಢಶ್ರದ್ಧೆಯನ್ನೂ ಇದು ನನಗೆ ನೀಡುತ್ತದೆ. ನನಗೆ ಆನಂದವನ್ನು ನೀಡುತ್ತಿರುವುದು ನನಗೆ ದೊರೆಯುತ್ತಿರುವ ಸನ್ಮಾನವಲ್ಲ, ಆದರೆ ನಮ್ಮ ರಾಷ್ಟ್ರದ ಕೇಂದ್ರ ಇನ್ನೂ ಸುಭದ್ರವಾಗಿದೆ ಎಂಬುದು. ಭರತಖಂಡ ಇನ್ನೂ ಜೀವಿಸುತ್ತಿದೆ. ಅದು ಸತ್ತಿರುವುದೆಂದು ಹೇಳುವವರು ಯಾರು? ಪಾಶ್ಚಾತ್ಯರು ನಾವು ಕರ್ಮಪಟುಗಳಾಗಿರುವುದನ್ನು ನೋಡಬಯಸುವರು. ಸಮರಾಂಗಣದಲ್ಲಿ ನಮ್ಮ ಚಟುವಟಿಕೆಯನ್ನು ಅವರು ನೋಡಬಯಸಿದರೆ ನಿರಾಶರಾಗುವರು. ಅದು ನಮ್ಮ ಕ್ಷೇತ್ರವಲ್ಲ. ಸೈನಿಕ ರಾಷ್ಟ್ರವು ಧಾರ್ಮಿಕ ಕ್ಷೇತ್ರದಲ್ಲಿ ಕ್ರಿಯಾಶೀಲವಾಗಿರುವುದನ್ನು ನಾವು ನೋಡಬಯಸಿದರೆ ಹೇಗೆ ನಿರಾಶರಾಗುತ್ತೇವೆಯೋ ಹಾಗೆಯೇ ಇದು. ಅವರು ಇಲ್ಲಿ ಬಂದು ನಮ್ಮನ್ನು ನೋಡಲಿ, ನಾವೂ ಕ್ರಿಯಾಶೀಲರಾಗಿರುವೆವು, ನಮ್ಮ ರಾಷ್ಟ್ರ ಹಿಂದಿನಂತೆಯೇ ಜೀವಂತವಾಗಿದೆ ಎಂಬುದು ಅವರಿಗೆ ತಿಳಿಯುತ್ತದೆ. ನಾವು ಅಧೋಗತಿಗೆ ಬಂದಿರುವೆವು ಎಂಬ ಭಾವನೆಯನ್ನೇ ನಾವು ಹೋಗಲಾಡಿಸಬೇಕು. ಅದೇನೋ ಒಳ್ಳೆಯದೆ.

ಆದರೆ ನಾನು ಕೆಲವು ಕಟು ಮಾತುಗಳನ್ನು ಹೇಳಬೇಕಾಗಿದೆ. ನೀವದನ್ನು ತಪ್ಪಾಗಿ ಭಾವಿಸುವುದಿಲ್ಲ ಎಂದು ಆಶಿಸುತ್ತೇನೆ. ಪಾಶ್ಚಾತ್ಯರ ಜಡವಾದವು ನಮ್ಮನ್ನು ಕೊಚ್ಚಿಕೊಂಡುಹೋಗಿದೆ ಎಂದು ಈಗ ತಾನೆ ಹೇಳಲಾಯಿತು. ಆದರೆ ಇದು ಪಾಶ್ಚಾತ್ಯರ ತಪ್ಪಲ್ಲ. ಬಹುಪಾಲು ತಪ್ಪು ನಮ್ಮದೇ. ವೇದಾಂತಿಗಳಾದ ನಾವು ಎಲ್ಲವನ್ನೂ ಆತ್ಮಪರೀಕ್ಷಾ ದೃಷ್ಟಿಯಿಂದ, ವೈಯಕ್ತಿಕ ದೃಷ್ಟಿಯಿಂದ ನೋಡಬೇಕು. ನಮಗೆ ನಾವೇ ಹಾನಿ ಮಾಡಿಕೊಳ್ಳುವವರೆಗೆ ಜಗತ್ತಿನ ಯಾವುದೂ ನಮ್ಮನ್ನು ಹಾನಿಗೆ ಗುರಿ ಮಾಡಲಾರದೆಂಬುದು ವೇದಾಂತಿಗಳಾದ ನಮಗೆ ವೇದ್ಯ. ಭರತಖಂಡದ ಜನಸಂಖ್ಯೆಯಲ್ಲಿ ಐದನೆಯ ಒಂದು ಭಾಗ\break ಮಹಮ್ಮದೀಯರಾಗಿರುವರು. ಸ್ವಲ್ಪ ಹಿಂದಿನ ಕಾಲಕ್ಕೆ ಹೋದರೆ ಮೂರನೆಯ ಎರಡು ಭಾಗ ಬೌದ್ಧರಾಗಿದ್ದರು. ಈಗ ಐದನೆಯ ಒಂದು ಭಾಗ ಮಹಮ್ಮದೀಯರು. ಕ್ರೈಸ್ತ ರಾಗಲೇ ಹತ್ತು ಲಕ್ಷದ ಮೇಲೆ ಇರುವರು.

ಇದು ಯಾರ ತಪ್ಪು? ಎಂದಿಗೂ ಮರೆಯದ ಭಾಷೆಯಲ್ಲಿ ಇತಿಹಾಸಜ್ಞರೊಬ್ಬರು ಹೀಗೆ ಹೇಳಿರುತ್ತಾರೆ: ಎಂದಿಗೂ ಬತ್ತದ ಅಮೃತಪ್ರವಾಹ ಪಕ್ಕದಲ್ಲೇ ಹರಿಯುತ್ತಿರುವಾಗ ಈ ದುರ್ಭಾಗಿಗಳು ಏಕೆ ಉಪವಾಸದಿಂದ ನರಳಿ ಸಾಯಬೇಕು? ನಮ್ಮ ಧರ್ಮವನ್ನು ತ್ಯಜಿಸಿದ ಈ ಜನರಿಗೆ ನಾವು ಏನು ಮಾಡಿದೆವು, ಅವರು ಏತಕ್ಕೆ ಮಹಮ್ಮದೀಯರಾದರು ಎಂಬುದೇ ಪ್ರಶ್ನೆ. ಇಂಗ್ಲೆಂಡಿನಲ್ಲಿ ವೇಶ್ಯಾವೃತ್ತಿಗೆ ಇಳಿಯುವುದರಲ್ಲಿ ಪ್ರಾಮಾಣಿಕಳಾದ ತರುಣಿಯೊಬ್ಬಳನ್ನು ಕುರಿತು ಯಾರೋ ಹಾಗೆ ಮಾಡಬೇಡ ಎಂದಾಗ ಅವಳು ಹೀಗೆ ಹೇಳಿದಳು: “ಜನರ ಸಹಾನುಭೂತಿಯನ್ನು ಪಡೆಯುವುದಕ್ಕೆ ಅದೊಂದೇ ಮಾರ್ಗ. ಈಗ ಯಾರೂ ನನ್ನ ಸಹಾಯಕ್ಕೆ ಬರುವುದಿಲ್ಲ. ನಾನು ಪತಿತಳಾದ ಮೇಲೆ ನನಗಾಗಿ ಮರುಗುವ ಸ್ತ್ರೀಯರು ನನ್ನ ನೆರವಿಗೆ ಬಂದು ಎಲ್ಲ ಸಹಾಯವನ್ನೂ ಒದಗಿಸುವರು”. ಈ ಮತಾಂತರಗೊಂಡವರ ಬಗ್ಗೆ ನಾವಿಂದು ಅಳುತ್ತಿರುವೆವು. ಆದರೆ ಅವರಿಗೆ ಹಿಂದೆ ನಾವು ಏನು ಮಾಡಿದೆವು? ನಮ್ಮಲ್ಲಿ ಪ್ರತಿಯೊಬ್ಬರೂ ಈ ಪ್ರಶ್ನೆಯನ್ನು ಕೇಳಿಕೊಳ್ಳಬೇಕು: ನಾವು ಕಲಿತಿರುವುದೇನು? ನಾವು ಸತ್ಯಜ್ಯೋತಿಯನ್ನು ಹಿಡಿದುಕೊಂಡಿರುವೆವೆ, ಹಿಡಿದುಕೊಂಡಿದ್ದರೆ ಎಷ್ಟು ದೂರ ಅದನ್ನು ತೆಗೆದು\-ಕೊಂಡು ಹೋಗಿರುವೆವು? ನಾವು ಆಗ ಅವರಿಗೆ ಸಹಾಯ ಮಾಡಲಿಲ್ಲ. ಇದು ನಮ್ಮದೇ ದೋಷ, ನಮ್ಮದೇ ಕರ್ಮ. ನಾವು ಯಾರನ್ನೂ ದೂಷಿಸಬೇಕಾಗಿಲ್ಲ, ನಮ್ಮ ಕರ್ಮವನ್ನೇ ನಾವು ದೂಷಿಸಬೇಕು.

ಜಡವಾದ, ಇಸ್ಲಾಂಧರ್ಮ, ಕ್ರೈಸ್ತಧರ್ಮ ಇವು ಯಾವುವೂ, ನಾವು ಅವಕ್ಕೆ ಅವಕಾಶಕೊಡದೆ ಇದ್ದಿದ್ದರೆ, ಭರತಖಂಡದಲ್ಲಿ ಅಭಿವೃದ್ಧಿಯಾಗುತ್ತಿರಲಿಲ್ಲ. ಕೆಟ್ಟ ಆಹಾರ, ಆಹಾರದ ಕೊರತೆ ಮುಂತಾದವುಗಳಿಂದ ನಮ್ಮ ದೇಹ ದುರ್ಬಲವಾಗದೆ ಇದ್ದರೆ ಯಾವ ವಿಷಕ್ರಿಮಿಗಳಿಗೂ ನಮ್ಮ ಮೇಲೆ ದಾಳಿಯಿಡಲಾಗುವುದಿಲ್ಲ. ಆರೋಗ್ಯವಂತನು ವಿಷಕ್ರಿಮಿಗಳ ಮಧ್ಯದಲ್ಲಿದ್ದರೂ ಸ್ವಲ್ಪವೂ ಅವುಗಳ ಪ್ರಭಾವಕ್ಕೆ ಬಲಿಯಾಗದೆ ಪಾರಾಗಬಲ್ಲ. ಆದರೆ ನಾವು ಉತ್ತಮರಾಗುವುದಕ್ಕೆ ಇನ್ನೂ ಸಮಯವಿದೆ. ಕೆಲಸಕ್ಕೆ ಬಾರದ ಹಳೆಯ ವಿಷಯಗಳನ್ನು ಕುರಿತು ಚರ್ಚಿಸುವುದನ್ನು ತ್ಯಜಿಸಿ. ಕಳೆದ ಆರು ಏಳು ಶತಮಾನಗಳ ನಮ್ಮ ಅವನತಿಯ ಕಾಲದಲ್ಲಿ, ನೂರಾರು ಮಂದಿ ವಿದ್ಯಾವಂತರು, ನಾವು ನೀರನ್ನು ಎಡಗೈಯಿಂದ ಕುಡಿಯಬೇಕೇ ಅಥವಾ ಬಲಗೈಯಿಂದ ಕುಡಿಯಬೇಕೇ, ಕೈಗಳನ್ನು ಮೂರು ಸಲ ತೊಳೆಯಬೇಕೆ ಅಥವಾ ನಾಲ್ಕು ಸಲ ತೊಳೆಯಬೇಕೆ, ಬಾಯನ್ನು ಐದು ಸಲ ಮುಕ್ಕಳಿಸಬೇಕೆ ಅಥವಾ ಆರು ಸಲ ಮುಕ್ಕಳಿಸಬೇಕೆ ಇವೇ ಮುಂತಾದುವನ್ನು ಕುರಿತು ಚರ್ಚಿಸುತ್ತಿದ್ದುದನ್ನು ಯೋಚಿಸಿ ನೋಡಿ! ಇಂತಹ ಘನ ವಿಷಯಗಳನ್ನು ಕುರಿತು ಚರ್ಚಿಸುತ್ತ, ಅವುಗಳ ಮೇಲೆ ವಿದ್ವತ್ಪೂರ್ಣ ಗ್ರಂಥಗಳನ್ನು ಬರೆಯುವುದರಲ್ಲಿಯೇ ಜೀವನ ಕಳೆಯುವ ವ್ಯಕ್ತಿಗಳಿಂದ ನೀವು ಏನನ್ನು ನಿರೀಕ್ಷಿಸಬಲ್ಲಿರಿ! ನಮ್ಮ ಧರ್ಮ ಅಡುಗೆ ಮನೆಯನ್ನು ಸೇರುವ ಅಪಾಯವಿದೆ. ಮುಕ್ಕಾಲು ಪಾಲು ಜನ ಈಗ ವೇದಾಂತಿಗಳೂ ಅಲ್ಲ, ಪೌರಾಣಿಕರೂ ಅಲ್ಲ, ಶಾಕ್ತರೂ ಅಲ್ಲ. ನಾವು “ಮುಟ್ಟಬೇಡಿ ಎನ್ನುವವರು,” ಆಗಿರುವೆವು. ನಮ್ಮ ಧರ್ಮ ಅಡಿಗೆ ಮನೆಯಲ್ಲಿದೆ. ಅಡಿಗೆ ಮಾಡುವ ಪಾತ್ರೆಯೇ ನಮ್ಮ ದೇವರು. “ನಾನು ಮಡಿ, ನನ್ನನ್ನು ಮುಟ್ಟಬೇಡಿ” ಇದೇ ನಮ್ಮ ಧರ್ಮವಾಗಿದೆ. ಹೀಗೆ ಮತ್ತೊಂದು ಶತಮಾನ ಕಳೆದರೆ ನಾವೆಲ್ಲಾ ಹುಚ್ಚರ ಆಸ್ಪತ್ರೆಯನ್ನು ಸೇರಬೇಕಾಗುವುದು. ಜೀವನದ ಉನ್ನತ ಸಮಸ್ಯೆಗಳನ್ನು ಗ್ರಹಿಸಲಾಗದ, ಬುದ್ಧಿ ಮಂದವಾಗಿರುವುದರ ಚಿಹ್ನೆ ಇದು. ಸ್ವಂತಿಕೆ ಎನ್ನುವುದೆಲ್ಲ ಹೊರಟುಹೋಗಿದೆ. ಮನಸ್ಸು ದುರ್ಬಲವಾಗಿ ನಿಷ್ಕ್ರಿಯವಾಗಿ, ಚಿಂತನಾಶಕ್ತಿಯನ್ನೇ ಕಳೆದುಕೊಂಡಿದೆ. ಅದು ಅತ್ಯಂತ ಕಿರಿದಾದ ವೃತ್ತದಲ್ಲಿಯೇ ಸುತ್ತುತ್ತಿರುವುದು. ಮೊದಲು ಈ ಪರಿಸ್ಥಿತಿ ಬದಲಾಗಬೇಕು. ಅನಂತರ ಜಾಗೃತರಾಗಿ, ಬಲಶಾಲಿಗಳಾಗಿ, ಕಾರ್ಯಶೀಲರಾಗಬೇಕು. ಆಗ ನಮ್ಮ ಪ್ರಾಚೀನರು ನಮಗಾಗಿ ಬಿಟ್ಟ ಅನರ್ಘ್ಯ ಅನಂತ ರತ್ನರಾಶಿಗೆ ಹಕ್ಕುದಾರರಾಗುತ್ತೇವೆ. ಇಡಿಯ ಜಗತ್ತಿಗೆ ಇಂದು ಈ ನಿಧಿ ಬೇಕಾಗಿದೆ. ಇದನ್ನು ಹಂಚದಿದ್ದರೆ ಜಗತ್ತು ನಾಶವಾಗುತ್ತದೆ. ಇದನ್ನು ಹೊರಗೆ ತನ್ನಿ, ಪ್ರಚಾರ ಮಾಡಿ, ಎಲ್ಲರಿಗೂ ಹಂಚಿ. ವ್ಯಾಸ ಮಹಾನುಭಾವರು ಕಲಿಯುಗದಲ್ಲಿ ದಾನ ಒಂದೇ ಧರ್ಮ ಎನ್ನುತ್ತಾರೆ. ಎಲ್ಲಾ ದಾನಕ್ಕಿಂತ ಅಧ್ಯಾತ್ಮದಾನ ಪರಮೋತ್ಕೃಷ್ಟ ದಾನ; ಅನಂತರ ಸಾಧಾರಣ ವಿದ್ಯಾದಾನ; ಅದಕ್ಕಿಂತ ಕೆಳಗಿನದು ಪ್ರಾಣದಾನ, ಕೊನೆಯದು ಅನ್ನದಾನ. ಅನ್ನದಾನವನ್ನು ನಾವು ಬೇಕಾದಷ್ಟು ಮಾಡಿರುವೆವು. ಮತ್ತಾವ ರಾಷ್ಟ್ರದವರೂ ನಮ್ಮಷ್ಟು ದಾನಿಗಳಲ್ಲ. ಭಿಕ್ಷುಕನ ಮನೆಯಲ್ಲಿ ಒಂದು ಚೂರು ರೊಟ್ಟಿ ಇರುವವರೆಗೆ ಅವನು ಅದರಲ್ಲಿ ಅರ್ಧವನ್ನು ಮತ್ತೊಬ್ಬನಿಗೆ ಕೊಡುವನು. ಈ ಸ್ಥಿತಿಯನ್ನು ನಾವು ಭರತಖಂಡದಲ್ಲಿ ಮಾತ್ರ ನೋಡಬಹುದು. ಇದು ನಮ್ಮಲ್ಲಿ ಸಾಕಷ್ಟು ಇದೆ. ಈಗ ಲೌಕಿಕ ಜ್ಞಾನದಾನ, ಅಧ್ಯಾತ್ಮದಾನಗಳನ್ನು ತೆಗೆದುಕೊಳ್ಳೋಣ. ನಾವೆಲ್ಲಾ ಧೀರರಾಗಿದ್ದರೆ, ಅಂಜದೇ ಇದ್ದರೆ ನಿರ್ವಂಚನೆಯಿಂದ ಕೆಲಸಕ್ಕೆ ಕೈಹಾಕಿದರೆ, ಇನ್ನು ಇಪ್ಪತ್ತೈದು ವರ್ಷಗಳಲ್ಲಿ ಈ ಸಮಸ್ಯೆಗಳನ್ನೆಲ್ಲಾ ಪರಿಹರಿಸಬಹುದು. ನಮಗೆ ಇನ್ನಾವ ಆತಂಕವೂ ಇರುವುದಿಲ್ಲ. ಭರತಖಂಡವೆಲ್ಲಾ ಪುನಃ ಆರ್ಯಾವರ್ತವಾಗುವುದು.

ನಾನು ನಿಮಗೆ ಹೇಳಬೇಕೆಂಬುದು ಇಷ್ಟೆ. ನಾನು ಯೋಜನೆಗಳನ್ನು ಕುರಿತು ಹೆಚ್ಚು ಮಾತನಾಡುವವನಲ್ಲ. ಮೊದಲು ಮಾಡಿ ತೋರಿಸಬೇಕು. ಅನಂತರ ಯೋಜನೆಗಳನ್ನು ಕುರಿತು ಮಾತನಾಡುತ್ತೇನೆ. ನನ್ನ ಯೋಜನೆಗಳು ಎಷ್ಟೋ ಇವೆ. ದೇವರ ಇಚ್ಛೆ ಇದ್ದರೆ, ನನಗೆ ಆಯುಸ್ಸು ಸಾಕಷ್ಟು ಇದ್ದರೆ ನಾನು ಅವುಗಳನ್ನು ಕಾರ್ಯರೂಪಕ್ಕೆ ತರುತ್ತೇನೆ. ನಾನು ಇದರಲ್ಲಿ ಜಯಿಸುವೆನೋ ಇಲ್ಲವೋ ತಿಳಿಯದು. ಆದರೆ ಜೀವನದಲ್ಲಿ ಒಂದು ಮಹಾ ಆದರ್ಶವನ್ನು ಸ್ವೀಕರಿಸಿ ಅದಕ್ಕಾಗಿ ಬಾಳನ್ನು ಅರ್ಪಿಸುವುದೇ ದೊಡ್ಡ ಸೌಭಾಗ್ಯ. ಇಲ್ಲದೇ ಇದ್ದರೆ ಗಿಡಮರಗಳಂತೆ ಬಾಳುವ ಈ ಕ್ಷುದ್ರ ಮಾನವ ಜೀವನದಿಂದ ಪ್ರಯೋಜನವೇನು? ಅದನ್ನು ಯಾವುದಾದರೊಂದು ಮಹಾಧ್ಯೇಯಕ್ಕೆ ಮೀಸಲಾಗಿಡುವುದೇ ಒಂದು ಭಾಗ್ಯ. ಭರತಖಂಡದಲ್ಲಿ ಸಾಧಿಸಬೇಕಾದ ಮಹತ್ಕಾರ್ಯವೇ ಇದು. ಈಗ ಉಂಟಾಗಿರುವ ಧಾರ್ಮಿಕ ಜಾಗೃತಿಯನ್ನು ನಾನು ಸ್ವಾಗತಿಸುತ್ತೇನೆ. ಕಬ್ಬಿಣ ಕಾದಿರುವಾಗ ಅದನ್ನು ಬಡಿಯುವ\break ಅವಕಾಶವನ್ನು ನಾನು ಕಳೆದುಕೊಂಡರೆ ಅನಂತರ ಪಶ್ಚಾತ್ತಾಪ ಪಡಬೇಕಾಗುವುದು.

