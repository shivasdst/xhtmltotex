
\chapter{ನಾನು ಕಲಿತುದೇನು?}

(ಢಾಕಾದಲ್ಲಿ ಸ್ವಾಮಿ ವಿವೇಕಾನಂದರು ಇಂಗ್ಲೀಷ್​ನಲ್ಲಿ ಎರಡು ಉಪನ್ಯಾಸಗಳನ್ನು ನೀಡಿದರು. ಮೊದಲನೆಯದು ‘ನಾನು ಕಲಿತುದೇನು’ ಎಂಬುದು, ಎರಡನೆಯದು, ‘ನಮ್ಮ ಧರ್ಮ’ ಎಂಬುದು. ಈ ಕೆಳಗಿನದು, ೧೯೦೧, ಮಾರ್ಚ್​ ೩೦ರಂದು ಮಾಡಿದ ಮೊದಲನೆಯ ಉಪನ್ಯಾಸದ ಬಂಗಾಳಿ ವರದಿಯ ಭಾಷಾಂತರವಾಗಿದೆ.)

\vskip 5pt

ಮೊದಲನೆಯದಾಗಿ ಪೂರ್ವ ಬಂಗಾಳಕ್ಕೆ ಬಂದು ಈ ಭಾಗದ ನಿಕಟ ಜ್ಞಾನವನ್ನು ಪಡೆಯುವುದಕ್ಕೆ ನನಗೆ ಸಿಕ್ಕಿದ ಈ ಅವಕಾಶಕ್ಕಾಗಿ ನನ್ನ ಸಂತೋಷವನ್ನು ವ್ಯಕ್ತಪಡಿಸುತ್ತೇನೆ. ಪಾಶ್ಚಾತ್ಯ ನಾಗರಿಕ ದೇಶಗಳಲ್ಲಿ ನಾನು ಬೇಕಾದಷ್ಟು ಅಲೆದಾಡಿದ್ದರೂ ನಮ್ಮ ದೇಶದ ಈ ಭಾಗದ ಪರಿಚಯವೇ ನನಗೆ ಇರಲಿಲ್ಲ. ದೊಡ್ಡ ದೊಡ್ಡ ನದಿಗಳು, ಫಲವತ್ತಾದ ಭೂಮಿ, ಸುಂದರವಾದ ಗ್ರಾಮಗಳಿಂದ ಕೂಡಿದ ನಮ್ಮ ಸ್ವಂತ ಬಂಗಾಳ ದೇಶವನ್ನು ನೋಡುವುದಕ್ಕೆ ನನಗೆ ಅವಕಾಶವೇ ಇದುವರೆಗೆ ದೊರಕಿರಲಿಲ್ಲ. ನಮ್ಮ ಬಂಗಾಳದೇಶದ ನೆಲದಲ್ಲಿ ಮತ್ತು ಜಲದಲ್ಲಿ ಇಷ್ಟೊಂದು ಸೌಂದರ್ಯವಿದೆ ಎಂದು ನನಗೆ ತಿಳಿದಿರಲಿಲ್ಲ. ಆದರೆ ನನಗಾದ ಪ್ರಯೋಜನ ಯಾವುದೆಂದರೆ, ಜಗತ್ತಿನ ಇತರ ಭಾಗಗಳನ್ನು ನೋಡಿದ ಮೇಲೆ ನಮ್ಮ ದೇಶದ ಸೌಂದರ್ಯವನ್ನು ನಾನು ಹೆಚ್ಚು ನೋಡಿ ಆನಂದಿಸಬಲ್ಲವನಾಗಿದ್ದೇನೆ.

\vskip 5pt

ಇದೇ ರೀತಿಯೇ ನಾನು ಬೇರೆ ಬೇರೆ ಮತಗಳಲ್ಲಿ ಧರ್ಮವನ್ನು ಅರಸಿದೆ, ಹೊರ ದೇಶಗಳ ಆದರ್ಶವನ್ನು ಅನುಸರಿಸುತ್ತಿರುವ ಪಂಗಡಗಳ ಬಾಗಿಲಲ್ಲಿ ನಿಂತು ಬೇಡಿದೆ. ನಮ್ಮ ದೇಶದ, ಜನಾಂಗದ ಧರ್ಮದಲ್ಲೇ ಇಷ್ಟೊಂದು ಭವ್ಯತೆ ಮತ್ತು ಸೌಂದರ್ಯವಿರುವುದೆಂದು ನಾನು ಭಾವಿಸಿಯೇ ಇರಲಿಲ್ಲ. ಜಗತ್ತಿನಲ್ಲಿ ಹಿಂದೂ ಧರ್ಮವು ಅತ್ಯಂತ ತೃಪ್ತಿದಾಯಕ ಧರ್ಮವೆಂಬುದನ್ನು ಈಗ ಕೆಲವು ವರುಷಗಳ ಹಿಂದೆ ತಿಳಿದುಕೊಂಡೆನು. ನಮ್ಮ ದೇಶದಲ್ಲಿ ಇಂತಹ ಅಪೂರ್ವ ಧರ್ಮವಿದ್ದರೂ, ನಮ್ಮ ಜನರಿಗೆ ಧರ್ಮದಲ್ಲಿ ಆಸಕ್ತಿಯೇ ಇಲ್ಲದಿರುವುದನ್ನು ನೋಡಿದರೆ ವ್ಯಸನವಾಗುತ್ತದೆ. ನಮ್ಮ ಶ್ರೇಷ್ಠ ಮಾತೃಭೂಮಿಯಲ್ಲಿ ಈಗ ಪ್ರಚಾರದಲ್ಲಿರುವ, ಆಧ್ಯಾತ್ಮಿಕತೆಯ ವಿಕಾಸಕ್ಕೆ ಹಿತಕಾರಿಯಲ್ಲದ, ಪಾಶ್ಚಾತ್ಯ ಭಾವನೆಯಿಂದ ಪ್ರೇರಿತವಾದ ಜಡವಾದದ ವಾತಾವರಣದಲ್ಲಿ ಜನರು ಬೆಳೆಯಬೇಕಾಗಿರುವುದೇ ಇದಕ್ಕೆ ಕಾರಣ ಎಂಬುದು ನನಗೆ ಗೊತ್ತು.

\vskip 5pt

ಇಂದು ನಮ್ಮಲ್ಲಿ ಕೆಲವು ಸುಧಾರಕರು ಇರುವರು. ಅವರು ನಮ್ಮ ಧರ್ಮವನ್ನು ಸುಧಾರಿಸಬೇಕು ಎನ್ನುತ್ತಾರೆ, ಎಂದರೆ ಹಿಂದೂ ಜನಾಂಗದ ಪುನರುತ್ಥಾನಕ್ಕಾಗಿ ಧರ್ಮವನ್ನು ತಲೆಕೆಳಗೆ ಮಾಡಲು ಅವರು ಬಯಸುತ್ತಾರೆ. ಅವರಲ್ಲಿ ಕೆಲವರು ಆಲೋಚನಾಪರರಿರುವರು ನಿಜ. ಆದರೆ ಅವರಲ್ಲಿ ಅನೇಕರು ತಾವು ಏನು ಮಾಡುತ್ತಿರುವೆವು\break ಎಂಬುದನ್ನು ಆಲೋಚಿಸಿದೆ ಅನ್ಯರನ್ನು ಅಂಧರಾಗಿ ಅನುಕರಿಸುತ್ತ ಮೂರ್ಖರಂತೆ ವರ್ತಿಸುತ್ತಿರುವರು. ಈ ಗುಂಪಿಗೆ ಸೇರಿದ ಸುಧಾರಕರು ನಮ್ಮ ಧರ್ಮಕ್ಕೆ ಹೊರಗಿನ ಭಾವನೆಗಳನ್ನು ತರಲು ಬಹಳ ತವಕಪಡುತ್ತಿರುವರು. ಹಿಂದೂಧರ್ಮದಲ್ಲಿ ವಿಗ್ರಹಾರಾಧನೆ ಎಂದರೇನು? ಅದು ಒಳ್ಳೆಯದೆ ಕೆಟ್ಟದ್ದೆ ಎಂಬುದನ್ನು ವಿಚಾರಿಸುವುದಕ್ಕೆ ಪ್ರಯತ್ನ ಪಡುವುದಿಲ್ಲ. ಬರಿಯ ಮತ್ತೊಬ್ಬರ ಭಾವನೆಯನ್ನು ತೆಗೆದುಕೊಂಡು ಹಿಂದೂ ಧರ್ಮವನ್ನು ದೂರುವರು. ಮತ್ತೊಂದು ಪಕ್ಷದ ಜನರಿರುವರು–ಅವರು ಹಿಂದೂಧರ್ಮದಲ್ಲಿರುವ ಪ್ರತಿಯೊಂದು ಆಚಾರಕ್ಕೆ ಯಾವುದಾದರೂ ವೈಜ್ಞಾನಿಕ ವಿವರಣೆ ಕೊಡಲು ಯತ್ನಿಸುವರು, ಮತ್ತು ಅವರು ಯಾವಾಗಲೂ ವಿದ್ಯುಚ್ಛಕ್ತಿ, ಅಯಸ್ಕಾಂತಶಕ್ತಿ, ವಾಯುಸ್ಪಂದನ ಮುಂತಾದ ವಿಷಯಗಳ ಬಗ್ಗೆ ಮಾತನಾಡುತ್ತಾರೆ. ಯಾರಿಗೆ ಗೊತ್ತು, ಒಂದು ದಿನ ಅವರು ದೇವರನ್ನೇ ವಿದ್ಯುತ್​ ಸ್ಪಂದನದ ರಾಶಿ ಎನ್ನಬಹುದು! ಹೇಗಾದರೂ ಆಗಲಿ ಜಗನ್ಮಾತೆ ಅವರನ್ನು ಹರಸಲಿ. ಅವಳೇ ತನ್ನ ಕೆಲಸವನ್ನು ಹಲವು ರೀತಿಗಳಲ್ಲಿ ಹಲವು ಸ್ವಭಾವದ ಜನರ ಮೂಲಕ ಮಾಡಿಸುತ್ತಿರುವಳು.

ಇವರಿಗೆ ವಿರೋಧವಾಗಿ ಸನಾತನಿಗಳ ಗುಂಪೊಂದು ಇದೆ. ಅವರು “ನನಗೆ ಗೊತ್ತಿಲ್ಲ, ನಿಮ್ಮ ಈ ಸೂಕ್ಷ್ಮ ತರ್ಕವನ್ನೆಲ್ಲಾ ತಿಳಿದುಕೊಳ್ಳುವುದಕ್ಕೆ ನನಗೆ ಇಚ್ಛೆ ಇಲ್ಲ. ನನಗೆ ದೇವರು ಬೇಕು. ಆತ್ಮ ಬೇಕು. ನಾನು ಎಲ್ಲಿ ಜಗತ್ತಿಲ್ಲವೋ, ಸುಖ ದುಃಖ ಗಳಿಲ್ಲವೋ, ಪರಮಾನಂದ ನೆಲಸಿರುವುದೋ, ಆ ಅತೀತ ಸ್ಥಿತಿಯನ್ನು ಮುಟ್ಟಲು ಬಯಸುತ್ತೇನೆ” ಎನ್ನುವರು. ಶ್ರದ್ಧೆಯಿಂದ ಗಂಗಾನದಿಯಲ್ಲಿ ಮಿಂದರೆ ಮುಕ್ತಿ ಸಿಕ್ಕುವುದೆಂದು ಅವರು ಭಾವಿಸುವರು. ಶ್ರದ್ಧಾಭಕ್ತಿಗಳಿಂದ ಶಿವ, ರಾಮ, ವಿಷ್ಣು, ಯಾವ ದೇವರನ್ನು ಆರಾಧಿಸಿದರೂ, ನಿಮಗೆ ಮೋಕ್ಷ ಸಿಕ್ಕುವುದೆಂದು ಅವರು ಹೇಳುವರು. ಇಂತಹ ಗಟ್ಟಿಮುಟ್ಟಾದ ಹಳೆಯ ಪರಂಪರೆಗೆ ನಾನು ಸೇರಿದವನೆಂಬುದು ನನಗೆ ಹೆಮ್ಮೆಯ ವಿಷಯ.

ಮತ್ತೊಂದು ಗುಂಪಿನವರು ದೇವರು ಮತ್ತು ಜಗತ್ತು ಇವೆರಡನ್ನೂ ಅನುಸರಿಸಬೇಕು ಎಂದು ಉಪದೇಶಿಸುತ್ತಾರೆ. ಅವರು ಪ್ರಾಮಾಣಿಕರಲ್ಲ. ಅವರು ತಮ್ಮ ಹೃದಯದಲ್ಲಿರುವುದನ್ನೇ ಹೇಳುವುದಿಲ್ಲ. ಮಹಾಪುರುಷರು ಏನು ಬೋಧಿಸುವರು? – “ರಾಮನಿರುವ ಕಡೆ ಕಾಮನಿಲ್ಲ. ಕಾಮನಿರುವ ಕಡೆ ರಾಮನಿಲ್ಲ. ಹಗಲು ಮತ್ತು ರಾತ್ರಿ ಇವು ಏಕಕಾಲದಲ್ಲಿ ಇರಲಾರವು.” ಪುರಾತನ ಋಷಿವಾಣಿ ಹೇಳುತ್ತದೆ: “ದೇವರು ನಿನಗೆ ಬೇಕಾದರೆ ಕಾಮಕಾಂಚನಗಳನ್ನು ತ್ಯಜಿಸಬೇಕು. ಈ ಸಂಸಾರ ಅನಿತ್ಯ, ಜೊಳ್ಳು, ಮಿಥ್ಯೆ. ಇದನ್ನು ತ್ಯಜಿಸಿದ ಹೊರತು ನೀವೆಷ್ಟು ಪ್ರಯತ್ನ ಪಟ್ಟರೂ ದೇವರು ನಿಮಗೆ ದೊರಕುವುದಿಲ್ಲ. ಇದು ನಿಮಗೆ ಸಾಧ್ಯವಿಲ್ಲದೇ ಇದ್ದರೆ ನೀವು ದುರ್ಬಲರೆಂದು ಒಪ್ಪಿಕೊಳ್ಳಿ, ಆದರ್ಶವನ್ನು ಕೆಳಗೆ ಎಳೆಯಬೇಡಿ. ಕೊಳೆತು ನಾರುವ ಶವದ ಮೇಲೆ ಚಿನ್ನದ ತಗಡನ್ನು ಹೊದಿಸಬೇಡಿ.” ಅವರ ದೃಷ್ಟಿಯಲ್ಲಿ, ನಿಮಗೆ ಆಧ್ಯಾತ್ಮಿಕತೆ ಬೇಕಾದರೆ, ದೇವರು ಬೇಕಾದರೆ, ಮೊದಲು ಈ ಮರೆಮಾಚುವಿಕೆಯಿಂದ ಪಾರಾಗಬೇಕು; ಆತ್ಮವಂಚನೆಯಿಂದ ಪಾರಾಗಬೇಕು.

\newpage

ನಾನೇನು ಕಲಿತಿರುವೆ? ಈ ಪುರಾತನ ಸಂಪ್ರದಾಯದಿಂದ ನಾನೇನು ಕಲಿತಿರುವೆ?

\begin{longtable}{@{}l@{}}
\textbf{ದುರ್ಲಭಂ ತ್ರಯಮೇವೈತತ್​ ದೇವಾನುಗ್ರಹಹೇತುಕಮ್​ ।} \\
\textbf{ಮನುಷ್ಯತ್ವಂ ಮುಮುಕ್ಷುತ್ವಂ ಮಹಾಪುರುಷಸಂಶ್ರಯಃ ॥} \\
\end{longtable}

“ಈ ಮೂರು ಬಹಳ ದುರ್ಲಭ, ದೇವರ ದಯೆಯಿಂದ ಮಾತ್ರ ಲಭಿಸುವುವು: ಮನುಷ್ಯ ಜನ್ಮ, ಮುಕ್ತನಾಗಬೇಕೆಂಬ ಆಸೆ ಮತ್ತು ಮಹಾಪುರುಷರ ಸಂಗ.” ಮೊದಲು ಬೇಕಾಗಿರುವುದೇ ಮನುಷ್ಯತ್ವ. ಏಕೆಂದರೆ ಈ ಜನ್ಮದಲ್ಲಿ ಮಾತ್ರ ಮುಕ್ತಿ ಸಾಧ್ಯ. ಅನಂತರವೇ ಮುಮುಕ್ಷುತ್ವ. ವ್ಯಕ್ತಿಗೆ ಮತ್ತು ಮತಕ್ಕೆ ತಕ್ಕಂತೆ ಸಾಕ್ಷಾತ್ಕಾರದ ಮಾರ್ಗ ಬೇರೆ ಬೇರೆಯಾದರೂ, ಕೆಲವು ಪಂಗಡದವರು ಈ ಜ್ಞಾನವನ್ನು ಪಡೆಯಲು ತಮಗೆಯೇ ವಿಶೇಷ ಹಕ್ಕುಗಳಿವೆ ಮತ್ತು ತಮ್ಮ ಮಾರ್ಗವೆ ಸರಿಯಾದುದು ಎಂದು ಭಾವಿಸಿದರೂ, ಅದು ಜೀವನದ ವಿವಿಧ ಹಂತಗಳಿಗೆ ಅನುಸಾರವಾಗಿ ವ್ಯತ್ಯಾಸವಾದರೂ, ವಿರೋಧದ ಅಂಜಿಕೆ ಇಲ್ಲದೆ, ಈ ಮುಮುಕ್ಷುತ್ವವಿಲ್ಲದೆ ಭಗವಂತನ ಸಾಕ್ಷಾತ್ಕಾರವಿಲ್ಲವೆಂದು ಹೇಳಬಹುದು. ಮುಮುಕ್ಷುತ್ವ ಎಂದರೇನು? ಮೋಕ್ಷಕ್ಕಾಗಿ ತೀವ್ರ ಆಕಾಂಕ್ಷೆ, ಸುಖ ದುಃಖಗಳ ತುಮುಲದಿಂದ ಪಾರಾಗಿ ಹೋಗಬೇಕೆಂಬ ಉತ್ಕಟ ಇಚ್ಛೆ, ಪ್ರಪಂಚದ ಮೇಲೆ ತೀವ್ರ ಜಿಗುಪ್ಸೆ. ಭಗವಂತನನ್ನು ನೋಡಬೇಕೆಂಬ ಆಸೆ ಉತ್ಕಟವಾಗಿ ದಾರುಣವಾದಾಗ ಮಾತ್ರ ನೀವು ಆ ಪರಮ ಪದವನ್ನು ಪಡೆಯಲು ಯೋಗ್ಯರಾಗುತ್ತೀರಿ.

\vskip 5pt

ಮತ್ತೊಂದು ಬೇಕಾಗಿದೆ–ಅದೇ ಮಹಾಪುರುಷರ ಸಂಗ. ಗುರಿಯನ್ನು ಸೇರಿದ ಮಹಾಪುರುಷರ ಬಾಳಿನಂತೆ ನಮ್ಮ ಜೀವನವನ್ನು ರೂಢಿಸಿಕೊಳ್ಳಬೇಕು. ಪ್ರಪಂಚದ ಮೇಲೆ ಜಿಗುಪ್ಸೆ, ಭಗವಂತನನ್ನು ಪಡೆಯಬೇಕೆಂಬ ತೀವ್ರ ಆಕಾಂಕ್ಷೆ ಇವೆರಡೇ ಸಾಲದು. ಗುರುವಿನಿಂದ ಉಪದೇಶ ಅವಶ್ಯಕ. ಏತಕ್ಕೆ? ಇದರಿಂದ ಆ ಶಕ್ತಿ ಮೂಲಕ್ಕೆ ಒಂದು ಸಂಬಂಧವನ್ನು ಕಲ್ಪಿಸಿಕೊಳ್ಳುವಿರಿ. ಹಲವು ತಲೆಮಾರುಗಳಿಂದ ಗುರುವಿನಿಂದ ಶಿಷ್ಯನಿಗೆ ಈ ಶಕ್ತಿ ಅಖಂಡವಾಗಿ ಹರಿದುಬಂದಿದೆ. ಶಿಷ್ಯನು ಗುರುವನ್ನು ಮಾರ್ಗದರ್ಶಕ, ಸ್ನೇಹಿತ, ತತ್ತ್ವದರ್ಶಿ ಎಂದು ಸ್ವೀಕರಿಸಬೇಕು. ಆಧ್ಯಾತ್ಮಿಕ ಜೀವನದಲ್ಲಿ ಮುಂದುವರಿಯಬೇಕಾದರೆ ಗುರುವು ಅತಿ ಮುಖ್ಯ. ನಾನು ಯಾರನ್ನು ನನ್ನ ಗುರುವನ್ನಾಗಿ ಸ್ವೀಕರಿಸಬೇಕು? \textbf{ಶ್ರೋತ್ರಿಯೋ ಅವೃಜಿನೋ, ಅಕಾಮಹತೋ ಯೋ ಬ್ರಹ್ಮವಿತ್ತಮಃ} – “ವೇದಗಳನ್ನು ಅಧ್ಯಯನ ಮಾಡಿದವನೂ, ಪಾಪದೂರನೂ, ಕಾಮಹೀನನೂ, ಬ್ರಹ್ಮಜ್ಞರಲ್ಲಿ ಉತ್ತಮನೂ ಆಗಿರಬೇಕು.” ಶ್ರೋತ್ರಿಯ ಎಂದರೆ ಕೇವಲ ಶಾಸ್ತ್ರಜ್ಞಾನ ಉಳ್ಳವನು ಎಂದು ಅಲ್ಲ. ಶಾಸ್ತ್ರಗಳ ಸೂಕ್ಷ್ಮರಹಸ್ಯವನ್ನು ತಿಳಿದವನು, ಅವನ್ನು ತನ್ನ ಜೀವನದಲ್ಲಿ ಮನಗಂಡವನು. “ಹಲವು ಶಾಸ್ತ್ರಗಳನ್ನು ಓದಿ ಅವರು ಅರಗಿಳಿಗಳಾಗಿರುವರು, ಪಂಡಿತರಲ್ಲ. ಶಾಸ್ತ್ರದ ಒಂದು ಶಬ್ದವನ್ನಾದರೂ ಓದಿ ದೇವರ ಮೇಲೆ ಯಾರು ಪ್ರೇಮವನ್ನು ಪಡೆದಿರುವರೋ ಅವರೇ ಪಂಡಿತರು.” ಕೇವಲ ಪುಸ್ತಕ ಪಂಡಿತರಿಂದ ಪ್ರಯೋಜನವಿಲ್ಲ. ಈಗಿನ ಕಾಲದಲ್ಲಿ ಎಲ್ಲರೂ ಗುರುಗಳಾಗ ಬಯಸುವರು. ಒಬ್ಬ ಭಿಕಾರಿಯೂ ಲಕ್ಷ ರೂಪಾಯಿಗಳನ್ನು ದಾನ ಮಾಡಲು ಇಚ್ಛಿಸುವನು! ಗುರುವಿನಲ್ಲಿ ಪಾಪವಿರಕೂಡದು. ಆತ ಯಾವ ಆಸೆಗೂ ತುತ್ತಾಗಿರಬಾರದು. ಇತರರಿಗೆ ಹಿತವನ್ನು ಮಾಡುವುದಲ್ಲದೆ ಮತ್ತಾವ ಉದ್ದೇಶವೂ ಅವನಲ್ಲಿರಕೂಡದು. ಅವನು ಅಹೇತುಕ ಕೃಪಾಸಾಗರನಾಗಿರಬೇಕು. ಹೆಸರು ಕೀರ್ತಿ ಅಥವಾ ಇನ್ನು ಯಾವುದೇ ಸ್ವಾರ್ಥ ಲಾಭಕ್ಕಾಗಿ ಅವನು ಜ್ಞಾನವನ್ನು ನೀಡಬಾರದು. \textbf{ಅವನು ಪೂರ್ತಿ ಬ್ರಹ್ಮಜ್ಞಾನಿಯಾಗಿರಬೇಕು. ಅಂಗೈ ಮೇಲಿನ ನೆಲ್ಲಿಕಾಯಿಯಂತೆ ಸಾಕ್ಷಾತ್ಕಾರ ಅವನ ವಶದಲ್ಲಿರಬೇಕು.} ಇಂತಹವನೆ ಗುರು ಎಂದು ಶ್ರುತಿ ಹೇಳುವುದು. ಇಂತಹ ಗುರುವಿನೊಂದಿಗೆ ಆಧ್ಯಾತ್ಮಿಕ ಸಂಬಂಧವನ್ನು ಕಲ್ಪಿಸಿಕೊಂಡರೆ ಆಗ ಭಗವತ್ಸಾಕ್ಷಾತ್ಕಾರವಾಗುವುದು, ಭಗವದ್ದರ್ಶನ ಸುಲಭ ಸಾಧ್ಯವಾಗುತ್ತದೆ.

\vskip 5pt

ಉಪದೇಶ ಪಡೆದಾದ ಮೇಲೆ ಶಿಷ್ಯನು ಸತತ ಅದನ್ನು ಅಭ್ಯಾಸಮಾಡಬೇಕು. ಇಷ್ಟದೇವತೆಯ ಮೇಲೆ ನಿರಂತರ ಧ್ಯಾನದ ಮೂಲಕ ಸತ್ಯ ಸಾಕ್ಷಾತ್ಕಾರಕ್ಕಾಗಿ ಪ್ರಯತ್ನಿಸಬೇಕು. ನಿಮಗೆ ದೇವರನ್ನು ಪಡೆಯಬೇಕೆಂಬ ತೀವ್ರ ಹಂಬಲ ಇದ್ದರೂ, ಗುರು ದೊರೆತಿದ್ದರೂ ಯಾವುದನ್ನು ನೀವು ಕಲಿತಿರುವಿರೊ ಅದನ್ನು ಅಭ್ಯಾಸ ಮಾಡದ ಹೊರತು ಸಾಕ್ಷಾತ್ಕಾರ ದೊರಕುವುದಿಲ್ಲ. ಇವೆಲ್ಲ ನಿಮ್ಮಲ್ಲಿ ದೃಢವಾಗಿ ನೆಲಸಿದ ಮೇಲೆ ನೀವು ಗುರಿಯನ್ನು ಸೇರುವಿರಿ. ಆದಕಾರಣವೇ ಹಿಂದೂಗಳೇ, ಶ್ರೇಷ್ಠ ಆರ್ಯ ಸಂತಾನರೆ, ನಮ್ಮ ಧರ್ಮದ ಆದರ್ಶವನ್ನು ಎಂದೂ ಮರೆಯಬೇಡಿ. ಹಿಂದೂಗಳ ಪರಮ ಆದರ್ಶ ಈ ಸಂಸಾರದಿಂದ ಪಾರಾಗುವುದು. ಈ ಸಂಸಾರವನ್ನು ತ್ಯಜಿಸುವುದು ಮಾತ್ರವಲ್ಲ, ಸ್ವರ್ಗವನ್ನೂ ತ್ಯಜಿಸಬೇಕು. ಅಶುಭವನ್ನು ತೊರೆಯುವುದು ಮಾತ್ರವಲ್ಲ, ಶುಭವನ್ನೂ ತೊರೆಯುವುದು. ಈ ವ್ಯಕ್ತ ಪ್ರಪಂಚವನ್ನು ಮೀರಿ ಸಚ್ಚಿದಾನಂದನಲ್ಲಿ ಐಕ್ಯವಾಗಬೇಕು.

