
\chapter{ವೇದಾಂತ ದರ್ಶನದ ವಿವಿಧ ಮುಖಗಳು}

\begin{center}
(ಕಲ್ಕತ್ತೆಯಲ್ಲಿ ಮಾಡಿದ ಭಾಷಣ)
\end{center}

ಬಹಳ ಹಿಂದೆ, ಎಲ್ಲಿ ಲಿಪಿಬದ್ಧ ಇತಿಹಾಸ, ಇಲ್ಲ, ಕಿಂವದಂತಿ ಎಂಬ ಕಥೆಗಳ ಮಂದಪ್ರಕಾಶ ಕೂಡ ತೂರಿಹೋಗಲಾರದೋ, ಅಲ್ಲಿ ಅನಂತ ಕಾಲಗರ್ಭದಿಂದಲೂ ಒಂದು ಕಾಂತಿ ಕೋರೈಸುತ್ತಿದೆ. ಬಾಹ್ಯಪ್ರಕೃತಿಗೆ ಅನುಗುಣವಾಗಿ ಕೆಲವು ವೇಳೆ ಉಜ್ವಲವಾಗಿಯೂ, ಕೆಲವು ವೇಳೆ ಮಂದವಾಗಿಯೂ, ಆದರೆ ಎಂದಿಗೂ ನಾಶವಾಗದೆ, ಸ್ಥಿರವಾಗಿ, ಮೌನವಾಗಿ, ಅಗೋಚರವಾಗಿ, ಮೃದುವಾಗಿಯಾದರೂ ಅತಿಪ್ರಬಲವಾಗಿ ತನ್ನ ಕಾಂತಿಯನ್ನು ಬೀರುತ್ತಾ - ಮುಂಜಾನೆ ಯಾರಿಗೂ ಕಾಣದ, ಯಾರ ಅರಿವಿಗೂ ಬಾರದ ಹಿಮಮಣಿಯು ಅತಿ ಸುಂದರವಾದ ಗುಲಾಬಿಯನ್ನು ಅರಳಿಸುವಂತೆ-ಭರತಖಂಡದ ಮೇಲೆ ಮಾತ್ರವಲ್ಲ, ಇಡಿಯ ಭಾವನಾ ಜಗತ್ತಿನ ಮೇಲೆ ಅದು ತನ್ನ ಪ್ರಭಾವವನ್ನು ಬೀರುತ್ತಿದೆ. ಅದು ಯಾವುದೆಂದರೆ ಉಪನಿಷತ್ತುಗಳ ಭಾವನೆ, ವೇದಾಂತ ತತ್ತ್ವದ ಪ್ರಭಾವ. ಭರತಖಂಡದಲ್ಲಿ ಎಂದು ಈ ಭಾವನೆ ವಿಕಸಿತವಾಯಿತೋ ಯಾರಿಗೂ ಗೊತ್ತಿಲ್ಲ. ಅದು ಎಂದು ನಡೆಯಿತು ಎಂದು ಊಹಿಸುವುದೂ ಸಾಧ್ಯವಿಲ್ಲ. ಪಾಶ್ಚಾತ್ಯ ವಿದ್ವಾಂಸರ ಊಹೆಗಳು ಎಷ್ಟರ ಮಟ್ಟಿಗೆ ಪರಸ್ಪರ ವಿರುದ್ಧವಾಗಿವೆ ಎಂದರೆ ಯಾವ ನಿರ್ದಿಷ್ಟ ಕಾಲವನ್ನೂ ಅದಕ್ಕೆ ಆರೋಪಿಸುವುದಕ್ಕೆ ಆಗುವುದಿಲ್ಲ. ಆದರೆ ಹಿಂದೂಗಳಾದ ನಾವು ಆಧ್ಯಾತ್ಮಿಕ ದೃಷ್ಟಿಯಿಂದ ಅದಕ್ಕೆ ಒಂದು ಆದಿಯಿದೆ ಎಂದು ಒಪ್ಪಿಕೊಳ್ಳುವುದಿಲ್ಲ. ವೇದಾಂತ ದರ್ಶನವು ಆಧ್ಯಾತ್ಮಿಕ ರಾಜ್ಯದಲ್ಲಿ ಮಾನವನು ಮಾಡಿದ ಪ್ರಥಮ ಆಲೋಚನೆ ಎಂತಲೂ ಮತ್ತು ಅದೇ ಪರಾಕಾಷ್ಠೆ ಎಂತಲೂ ನಾನು ಸಂಕೋಚವಿಲ್ಲದೆ ಹೇಳುತ್ತೇನೆ.

ಈ ವೇದಾಂತದ ಮಹಾ ಸಮುದ್ರದಿಂದ ಕಾಲಕಾಲಕ್ಕೆ ಜ್ಞಾನದ ಅಲೆಗಳೆದ್ದು ಪೂರ್ವ ಮತ್ತು ಪಶ್ಚಿಮ ದೇಶಗಳಿಗೆ ಹೋಗಿವೆ. ಬಹಳ ಹಿಂದಿನ ಕಾಲದಲ್ಲಿ ಪಶ್ಚಿಮದಲ್ಲಿ ಅಥೆನ್ಸ್, ಅಲೆಗ್ಸಾಂಡ್ರಿಯಾ, ಅಂಟಿಯೋಕ್​ ಮುಂತಾದ ಕಡೆಗಳಲ್ಲಿ ಗ್ರೀಕರ ಮೇಲೆ ತನ್ನ ಪ್ರಭಾವವನ್ನು ಅದು ಬೀರಿತು. ನಿಸ್ಸಂದೇಹವಾಗಿ ಸಾಂಖ್ಯ ಸಿದ್ಧಾಂತವು ಪುರಾತನ ಗ್ರೀಕರ ಮೇಲೆ ತನ್ನ ಪ್ರಭಾವವನ್ನು ಬೀರಿದೆ. ಸಾಂಖ್ಯ ಮತ್ತು ಇತರ ದರ್ಶನಗಳೆಲ್ಲಾ ಉಪನಿಷತ್ತುಗಳ ಅಥವಾ ವೇದಾಂತದ ಮೇಲೆ ಪ್ರತಿಷ್ಠಿತವಾಗಿವೆ. ಭರತಖಂಡದಲ್ಲೂ ಹಿಂದೆ ಇದ್ದ ಮತ್ತು ಈಗಿರುವ ವಿಭಿನ್ನ ಪಂಗಡಗಳ ಗೊಂದಲದಲ್ಲಿಯೂ, ಅವರಿಗೆಲ್ಲಾ ಒಂದು ಪ್ರಮಾಣ, ಒಂದು ಸಾಮಾನ್ಯ ನೆಲೆ ಉಪನಿಷತ್ತುಗಳು. ನೀವು ದ್ವೈತಿಗಳೋ, ವಿಶಿಷ್ಟಾದ್ವೈತಿಗಳೋ, ಅದ್ವೈತಿಗಳೋ, ಅಥವಾ ಶುದ್ಧಾದ್ವೈತಿಗಳೋ, ಅಥವಾ ಇನ್ನು ಯಾವುದಾದರೂ ಸಿದ್ಧಾಂತವನ್ನು ಅನುಸರಿಸುವವರಾಗಿರಿ, ಅವೆಲ್ಲಕ್ಕೂ ಪ್ರಮಾಣದಂತೆ ನಿಮ್ಮ ಹಿಂದೆ ನಿಲ್ಲುವ ಶಾಸ್ತ್ರ ಉಪನಿಷತ್ತುಗಳು. ಯಾವ ಸಿದ್ಧಾಂತವು ಉಪನಿಷತ್ತನ್ನು ಅನುಸರಿಸುವುದಿಲ್ಲವೋ ಅದು ಆಸ್ತಿಕ ಮತವಲ್ಲ. ಜೈನರು ಮತ್ತು ಬೌದ್ಧರು ಉಪನಿಷತ್ತುಗಳನ್ನು ಅಲ್ಲಗಳೆದುದರಿಂದಲೇ ಭರತಖಂಡದಲ್ಲಿ ಅವರಿಗೆ ಸ್ಥಳವಿಲ್ಲವಾಯಿತು. ನಮಗೆ ಗೊತ್ತಿರಲಿ ಗೊತ್ತಿಲ್ಲದೇ ಇರಲಿ; ವೇದಾಂತವು ಈ ವಿಭಿನ್ನ ಪಂಗಡಗಳನ್ನೆಲ್ಲಾ ಪ್ರವೇಶಿಸಿದೆ. ಅನಂತರ ಶಾಖೋಪಶಾಖೆಗಳಿಂದ ಕೂಡಿದ ಹಿಂದೂ ಧರ್ಮದಂತಹ ಬೃಹತ್​ ವಟವೃಕ್ಷದಲ್ಲಿ ವೇದಾಂತದ ಪ್ರಭಾವ ಹಾಸು ಹೊಕ್ಕಾಗಿದೆ. ನಮಗೆ ತಿಳಿದೊ, ತಿಳಿಯದೆಯೊ ನಾವು ವೇದಾಂತವನ್ನು ಆಲೋಚಿಸುವೆವು, ವೇದಾಂತದಲ್ಲಿ ಜೀವಿಸುವೆವು, ವೇದಾಂತದಲ್ಲಿ ಉಸಿರಾಡುವೆವು, ವೇದಾಂತದಲ್ಲೇ ಸಾಯುವೆವು. ಪ್ರತಿಯೊಬ್ಬ ಹಿಂದೂವು ಹೀಗೆ ಮಾಡುತ್ತಿರುವನು. ಭರತ ಖಂಡದಲ್ಲಿ ವೇದಾಂತವನ್ನು ಪ್ರಚಾರ ಮಾಡುವುದು ಅನವಶ್ಯಕವಾಗಿ ಕಾಣುವುದು. ಇದೊಂದೆ ಪ್ರಚಾರಕ್ಕೆ ಯೋಗ್ಯವಾದುದು. ಈ ಯುಗಕ್ಕೆ ಇದನ್ನು ಬೋಧಿಸಬೇಕಾಗಿದೆ. ಎಲ್ಲಾ ಹಿಂದೂ ಪಂಗಡಗಳೂ ನಾನು ಹೇಳಿದಂತೆ ಉಪನಿಷತ್ತಿಗೆ ಅಧೀನವಾಗಿರಬೇಕು. ಆದರೆ ಈ ಪಂಗಡಗಳಲ್ಲಿ ಎಷ್ಟೋ ತೋರಿಕೆಯ ವಿರುದ್ಧಾಭಿಪ್ರಾಯಗಳಿವೆ. ಅನೇಕ ವೇಳೆ ಪ್ರಾಚೀನ ಋಷಿಗಳು ಕೂಡ ಉಪನಿಷತ್ತುಗಳ ಅಂತರಾಳದಲ್ಲಿರುವ ಸಮನ್ವಯ ಭಾವವನ್ನು ಅರ್ಥಮಾಡಿಕೊಳ್ಳಲಿಲ್ಲ. ಅನೇಕ ಸಲ ಋಷಿಗಳು ತಮ್ಮ ತಮ್ಮಲ್ಲೇ ಎಷ್ಟರಮಟ್ಟಿಗೆ ಕಾದಾಡಿದರೆಂದರೆ “ಭಿನ್ನ ಮತವಿಲ್ಲದ ಮುನಿಗಳೇ ಇಲ್ಲ” ಎಂಬುದು ನಾಣ್ನುಡಿಯಾಯಿತು. ಉಪನಿಷತ್ತು ದ್ವೈತ ವಿಶಿಷ್ಟಾದ್ವೈತ, ಅದ್ವೈತ ಭಾವನೆಗಳನ್ನು ಕೊಟ್ಟರೂ ಅವುಗಳಲ್ಲಿ ಸಾಮರಸ್ಯವನ್ನು ಉಂಟುಮಾಡುವ ವಿವರಣೆಯನ್ನು ತೋರುವುದು ಆವಶ್ಯಕ. ಇದನ್ನೇ ಜಗತ್ತಿಗೆ ತಿಳಿಸಬೇಕಾಗಿದೆ. ಇದು ಭರತಖಂಡಕ್ಕೆ ಬೇಕಾದಷ್ಟೇ ಹೊರಗಿನ ಜಗತ್ತಿಗೂ ಬೇಕಾಗಿದೆ.

ಈಶ್ವರ ಕೃಪೆಯಿಂದ ಒಬ್ಬ ಮಹಾಪುರುಷನ ಪದತಳದಲ್ಲಿ ಕುಳಿತು ಕಲಿತುಕೊಳ್ಳುವ ಮಹಾ ಸೌಭಾಗ್ಯ ನನಗೆ ಒದಗಿತ್ತು. ಅವರ ಇಡೀ ಜೀವನ ಉಪನಿಷತ್ತುಗಳ ಮಹಾಸಮನ್ವಯದಂತೆ ಇತ್ತು. ಉಪದೇಶಕ್ಕಿಂತ ಸಾವಿರಪಾಲು ಹೆಚ್ಚಾಗಿ ಅವರ ಜೀವವು ಉಪನಿಷತ್ತುಗಳ ಜೀವಂತ ಭಾಷ್ಯದಂತೆ ಇತ್ತು. ಉಪನಿಷತ್ತುಗಳೇ ಮಾನವ ರೂಪವನ್ನು ಧಾರಣ ಮಾಡಿದಂತೆ ಇತ್ತು. ಬಹುಶಃ ಆ ಸಮನ್ವಯ ನನ್ನಲ್ಲಿ ಸ್ವಲ್ಪ ಇರಬಹುದು. ಅದನ್ನು ನಿಮಗೆ ವಿವರಿಸುವುದಕ್ಕೆ ಸಾಧ್ಯವಾಗುವುದೋ ಇಲ್ಲವೋ ಗೊತ್ತಿಲ್ಲ. ಆದರೆ ಇದು ನನ್ನ ಪ್ರಯತ್ನ. ವೇದಾಂತ ಸಂಪ್ರದಾಯಗಳು ಪರಸ್ಪರ ವಿರುದ್ಧವಲ್ಲ. ಒಂದು ಮತ್ತೊಂದರ ಪೂರಕ. ಒಂದು ಮತ್ತೊಂದರ ಪೂರೈಕೆ. ‘ತತ್ತ್ವಮಸಿ’ ಎಂಬ ಅದ್ವೈತ ಶಿಖರವನ್ನು ಮುಟ್ಟುವವರೆಗೆ ಒಂದು ಮತ್ತೊಂದಕ್ಕೆ ಒಯ್ಯುವ ಮೆಟ್ಟಲು ಎಂಬುದನ್ನು ತೋರುವುದೇ ನನ್ನ ಜೀವನದ ಗುರಿ. ಭರತ ಖಂಡದಲ್ಲಿ ಕರ್ಮಕಾಂಡ ಪ್ರಬಲವಾಗಿದ್ದ ಕಾಲವೊಂದಿತ್ತು. ಅಲ್ಲಿಯೂ ಎಷ್ಟೋ ಒಳ್ಳೆಯ ಭಾವನೆಗಳು ನಿಸ್ಸಂಶಯವಾಗಿ ಇವೆ. ಈಗಲೂ ನಮ್ಮ ನಿತ್ಯದ ಪೂಜಾ ವಿಧಾನದ ಬಹುಭಾಗ ಅದನ್ನೇ ಅನುಸರಿಸುವುದು. ಆದರೂ ವೇದಗಳ ಕರ್ಮಕಾಂಡ ಈಗ ಭರತಖಂಡದಿಂದ ಸಂಪೂರ್ಣ ಮಾಯವಾಗಿದೆ. ವೇದಗಳ ಕರ್ಮಕಾಂಡವನ್ನು ನಮ್ಮ ನಿತ್ಯ ಜೀವನದಲ್ಲಿ ಅನುಸರಿಸುವುದು ಬಹಳ ಕಡಮೆ. ನಮ್ಮ ನಿತ್ಯಜೀವನದಲ್ಲಿ ನಾವು ಹೆಚ್ಚಾಗಿ ಪುರಾಣಗಳನ್ನು ಅಥವಾ ತಂತ್ರವನ್ನು ಅನುಸರಿಸುತ್ತೇವೆ. ಕೆಲವು ವೇದಗಳ ಭಾಗಗಳನ್ನು ಬ್ರಾಹ್ಮಣರು ಹೇಳುವಾಗಲೂ ಅವನ್ನು ಪುರಾಣಗಳಿಗೆ ಮತ್ತು ತಂತ್ರಕ್ಕೆ ಹೊಂದಿಸಿಕೊಂಡು ಹೇಳುವರು. ನಾವು ಕರ್ಮಕಾಂಡವನ್ನು ಅನುಸರಿಸುವ ವೈದಿಕರೆಂದು ಕರೆದುಕೊಳ್ಳುವುದು ಸೂಕ್ತವಾಗಿ ತೋರುವುದಿಲ್ಲ. ಆದರೆ ನಾವುಗಳೆಲ್ಲಾ ವೇದಾಂತಿಗಳು ಎನ್ನುವುದು ಸತ್ಯ. ಹಿಂದೂಗಳೆಂದು ಹೇಳಿಕೊಳ್ಳುವವರೆಲ್ಲಾ ವೇದಾಂತಿಗಳೆಂದು ಹೇಳಿಕೊಳ್ಳಲಿ. ವೇದಾಂತಿಗಳು ಎಂಬ ಹೆಸರಿನಲ್ಲಿ ದ್ವೈತಿಗಳು, ಅದ್ವೈತಿಗಳು, ವಿಶಿಷ್ಟಾದ್ವೈತಿಗಳು ಎಲ್ಲರೂ ಸೇರುತ್ತಾರೆ.

ಭರತಖಂಡದಲ್ಲಿ ಈಗ ಇರುವ ಮುಖ್ಯ ಸಂಪ್ರದಾಯಗಳು, ದ್ವೈತ ಮತ್ತು ಅದ್ವೈತ. ಈ ಸಂಪ್ರದಾಯಗಳಲ್ಲಿ ಇರುವ ಸಣ್ಣ ಸಣ್ಣ ಭಿನ್ನಾಭಿಪ್ರಾಯಗಳು, ಇವುಗಳ ಆಧಾರದ ಮೇಲೆ ತಾವು ಶುದ್ಧಾದ್ವೈತಿಗಳು ಅಥವಾ ವಿಶಿಷ್ಟಾದ್ವೈತಿಗಳು ಎಂದು ಅವರು ಬೇರೆ ಬೇರೆ ಕರೆದುಕೊಳ್ಳುವುದು ಅಷ್ಟು ಮುಖ್ಯವಲ್ಲ. ಅವರೆಲ್ಲಾ ದ್ವೈತಿಗಳು ಇಲ್ಲವೆ ಅದ್ವೈತಿಗಳು. ಈಗಿರುವ ಸಂಪ್ರದಾಯಗಳಲ್ಲಿ ಕೆಲವು ಬಹಳ ಆಧುನಿಕವಾದವು, ಮತ್ತೆ ಕೆಲವು ಬಹಳ ಹಿಂದಿನದರ ರೂಪಾಂತರ ಎಂದುತೋರುತ್ತದೆ. ಒಬ್ಬರನ್ನು ರಾಮಾನುಜರ ಜೀವನಕ್ಕೆ ಮತ್ತು ತತ್ತ್ವಕ್ಕೆ ಸೇರಿದವರೆಂದೂ, ಮತ್ತೊಬ್ಬರನ್ನು ಶಂಕರಾಚಾರ್ಯರ ಜೀವನಕ್ಕೆ ಮತ್ತು ಉಪದೇಶಕ್ಕೆ ಸೇರಿದವರೆಂದೂ ಕರೆಯುತ್ತೇವೆ.

ರಾಮಾನುಜರು ಇತ್ತೀಚಿನ ಅತಿ ಪ್ರಮುಖ ದ್ವೈತ ಸಿದ್ಧಾಂತಿಗಳು. ಉಳಿದವರೆಲ್ಲರೂ ಪ್ರತ್ಯಕ್ಷವಾಗಿ ಇಲ್ಲವೇ ಪರೋಕ್ಷವಾಗಿ ಅವರನ್ನು ಅನುಸರಿಸಿರುವರು. ಮೂಲ ಸಿದ್ಧಾಂತದ ವಿಷಯದಲ್ಲಿ ಮಾತ್ರವಲ್ಲದೆ, ಮತ ಸಂಪ್ರದಾಯದ ಸಣ್ಣಪುಟ್ಟ ವಿವರಗಳಲ್ಲಿಯೂ ಅವರನ್ನು ಅನುಸರಿಸಿರುವರು. ರಾಮಾನುಜರನ್ನು ಭರತಖಂಡದ ಇತರ ದ್ವೈತ ವೈಷ್ಣವ ಸಂಪ್ರದಾಯದವರೊಂದಿಗೆ ಹೋಲಿಸಿದರೆ, ಅವರ ಸಂಸ್ಥೆ ಬೋಧನೆಯ ರೀತಿ ಮುಂತಾದುವುಗಳಲ್ಲಿ ಎಷ್ಟೋ ಹೋಲಿಕೆ ಇರುವುದನ್ನು ನೋಡಿ ಆಶ್ಚರ್ಯವಾಗುವುದು. ದಕ್ಷಿಣದಲ್ಲಿ ಮಧ್ವಮುನಿ ಎಂಬ ಮತ್ತೊಬ್ಬ ಪ್ರಚಾರಕರಿದ್ದರು. ಚೈತನ್ಯ ಮಹಾಪ್ರಭು ಇವರ ತತ್ತ್ವವನ್ನೇ ಅನುಸರಿಸಿ ವಂಗದೇಶದಲ್ಲಿ ಪ್ರಚಾರ ಮಾಡಿದರು. ದಕ್ಷಿಣದಲ್ಲಿ ಶೈವ ವಿಶಿಷ್ಟಾದ್ವೈತಿಗಳೆಂಬ ಮತ್ತೊಂದು ಸಂಪ್ರದಾಯದವರು ಇರುವರು. ಶೈವರು, ದಕ್ಷಿಣದಲ್ಲಿ ಕೆಲವು ಕಡೆ ಮತ್ತು ಸಿಲೋನಿನಲ್ಲಿ ಬಿಟ್ಟು, ಇತರ ಕಡೆಗಳಲ್ಲೆಲ್ಲಾ ಅದ್ವೈತಿಗಳು. ಅವರು ವಿಷ್ಣುವಿಗೆ ಬದಲು ಶಿವನನ್ನು ಪೂಜಿಸುವರು. ಜೀವಸಿದ್ಧಾಂತದಲ್ಲಿ ಹೊರತು ಉಳಿದ ಎಲ್ಲಾ ಭಾಗಗಳಲ್ಲಿಯೂ ಅವರು ರಾಮಾನುಜರಂತೆಯೇ. ರಾಮಾನುಜ ಸಂಪ್ರದಾಯದವರು ಜೀವವನ್ನು ಅಣುವೆಂದು ಹೇಳುವರು. ಶಂಕರಾಚಾರ್ಯರ ಅನುಯಾಯಿಗಳು ಜೀವವನ್ನು ವಿಭು ಎನ್ನುವರು. ಎಷ್ಟೋ ಅದ್ವೈತ ಪಂಥಗಳಿವೆ. ಹಿಂದಿನ ಕಾಲದಲ್ಲಿಯೂ ಎಷ್ಟೋ ಪಂಥಗಳಿದ್ದುವು. ಆದರೆ ಶಂಕರಾಚಾರ್ಯರು ಅವನ್ನು ಸಂಪೂರ್ಣ ಸ್ವೀಕರಿಸಿ ತಮ್ಮಲ್ಲಿ ಅಳವಡಿಸಿಕೊಂಡಿರುವರು. ಕೆಲವು ಟೀಕಾಚಾರ್ಯರು ಶಂಕರರನ್ನೇ ಟೀಕಿಸುವರು. ಉದಾಹರಣೆಗೆ ವಿಜ್ಞಾನ ಭಿಕ್ಷು ಅದ್ವೈತಿಯಾದರೂ ಶಂಕರಾಚಾರ್ಯರ ಮಾಯಾವಾದವನ್ನು ಖಂಡಿಸುವನು. ಮಾಯಾವಾದವನ್ನು ನಂಬದ ಎಷ್ಟೋ ಪಂಥಗಳಿದ್ದುವು.\break ಅವರು ಶಂಕರಾಚಾರ್ಯರನ್ನು ಪ್ರಚ್ಛನ್ನಬೌದ್ಧರೆಂದು ಟೀಕಿಸುತ್ತಿದ್ದರು.\break ಶಂಕರರು ಮಾಯಾವಾದವನ್ನು ಬೌದ್ಧರಿಂದ ತೆಗೆದುಕೊಂಡು ಅದನ್ನು\break ವೇದಾಂತದಲ್ಲಿ ಸೇರಿಸಿದರು ಎಂದು ಹೇಳುವರು. ಅದು ಹೇಗಾದರೂ ಇರಲಿ, ಅಂತೂ ಈಗ ಅದ್ವೈತಿಗಳೆಲ್ಲರೂ ಶಂಕರಾಚಾರ್ಯರ ಅನುಯಾಯಿಗಳಾಗಿರುವರು. ದಕ್ಷಿಣ ಮತ್ತು ಉತ್ತರ ದೇಶಗಳಲ್ಲಿ ಶಂಕರಾಚಾರ್ಯರ ಅನುಯಾಯಿಗಳು ಪ್ರಖ್ಯಾತ ಅದ್ವೈತ ಪ್ರಚಾರಕರು. ಶಂಕರಾಚಾರ್ಯರ ಪ್ರಭಾವ, ವಂಗ ಪಂಜಾಬ್​ ಕಾಶ್ಮೀರ ದೇಶಗಳಿಗೆ ಅಷ್ಟು ಹರಡಲಿಲ್ಲ. ದಕ್ಷಿಣದಲ್ಲಿ ಸ್ಮಾರ್ತರೆಲ್ಲಾ ಶಂಕರಾಚಾರ್ಯರ ಅನುಯಾಯಿಗಳು. ಉತ್ತರ ಭಾರತ ಕಾಶಿಯನ್ನು ಕೇಂದ್ರವಾಗಿ ಮಾಡಿಕೊಂಡು ಅವರು ಅದ್ಭುತ ಪ್ರಭಾವವನ್ನು ಬೀರುತ್ತಿರುವರು.

ಶಂಕರ ರಾಮಾನುಜರಿಬ್ಬರೂ ತಾವು ಹೊಸ ತತ್ತ್ವವನ್ನು ಜಾರಿಗೆ ತಂದವರೆಂದು ಹೇಳುವುದಿಲ್ಲ. ರಾಮಾನುಜಾಚಾರ್ಯರು ಬೋಧಾಯನ ವೃತ್ತಿಯನ್ನು ಮಾತ್ರ ಅನುಸರಿಸುತ್ತಿರುವೆ ಎಂದು ಹೇಳುವರು. \textbf{“ಭಗವದ್ಬೋಧಾಯನಕೃತಾಂ ವಿಸ್ತೀರ್ಣಾಂ ಬ್ರಹ್ಮಸೂತ್ರವೃತ್ತಿಂ ಪೂರ್ವಾಚಾರ್ಯಾಃ ಸಂಚಿಕ್ಷಿಪುಃ ತನ್ಮತಾನುಸಾರೇಣ ಸೂತ್ರಾಕ್ಷರಾಣಿ ವ್ಯಾಖ್ಯಾಸ್ಯಂತೇ”} - ಬೋಧಾಯನರು ಬರೆದ ವಿಸ್ತೀರ್ಣವಾದ ಬ್ರಹ್ಮ ಸೂತ್ರವೃತ್ತಿಯನ್ನು ಪೂರ್ವಾಚಾರ್ಯರು ಸಂಕ್ಷಿಪ್ತವಾಗಿ ಹೇಳಿದರು; ಅವರ ಅಭಿಪ್ರಾಯದಂತೆ ಸೂತ್ರವನ್ನು ವ್ಯಾಖ್ಯಾನ ಮಾಡಿರುವೆನು ಎಂದು ರಾಮಾನುಜಾಚಾರ್ಯರು ಶ‍್ರೀಭಾಷ್ಯದ ಪ್ರಾರಂಭದಲ್ಲಿ ವಿವರಿ\-ಸಿರುವರು. ಅವರು ಆ ಬೋಧಾಯನ ವೃತ್ತಿಯನ್ನು ಸಂಕ್ಷಿಪ್ತವಾಗಿ ಹೇಳಿರುವರು. ನಮ್ಮಲ್ಲಿ ಈಗ ಇರುವುದೇ ಅದು. ಬೋಧಾಯನ ವೃತ್ತಿಯನ್ನು ನೋಡುವುದಕ್ಕೆ ನನಗೂ ಅವಕಾಶ ಸಿಕ್ಕಲಿಲ್ಲ. ಸ್ವಾಮಿ ದಯಾನಂದ ಸರಸ್ವತಿಗಳು ಬೋಧಾಯನ ವೃತ್ತಿಯೊಂದನ್ನಲ್ಲದೆ ಉಳಿದ ಸೂತ್ರ ಭಾಷ್ಯವನ್ನೆಲ್ಲಾ ತಿರಸ್ಕರಿಸಿರುವರು. ರಾಮಾನುಜರನ್ನು ಸಮಯ ಬಂದಾಗಲೆಲ್ಲಾ ಟೀಕಿಸಿದರೂ ದಯಾನಂದ ಸರಸ್ವತಿಯವರೂ ಬೋಧಾಯನ ವೃತ್ತಿಯನ್ನು ನೋಡಿದ್ದಂತೆ ಕಾಣುವುದಿಲ್ಲ. ಭರತಖಂಡದಲ್ಲೆಲ್ಲಾ ಅದನ್ನು ಹುಡುಕಿರುವೆನು. ಆದರೂ ಅದು ನನಗೆ ಸಿಕ್ಕಲಿಲ್ಲ. ರಾಮಾನುಜರು ತಾವು ಬೋಧಾಯನರ ಭಾವನೆಯನ್ನೇ, ಕೆಲವು ವೇಳೆ ಅವರ ವಾಕ್ಯವನ್ನೇ ತೆಗೆದುಕೊಂಡು ಶ‍್ರೀಭಾಷ್ಯದಲ್ಲಿ ಸಂಕ್ಷಿಪ್ತವಾಗಿ ಹೇಳುತ್ತಿರುವುದಾಗಿ ಒಪ್ಪಿಕೊಳ್ಳುವರು. ಶಂಕರಾಚಾರ್ಯರು ಕೂಡ ಅದನ್ನೇ ಮಾಡಿದ\-ರಂತೆ. ಶಂಕರಾಚಾರ್ಯರು ತಮ್ಮ ಭಾಷ್ಯದಲ್ಲಿ ತಮಗಿಂತ ಹಿಂದಿನ ಭಾಷ್ಯಕಾರನ ಹೆಸರನ್ನು ಕೆಲವು ವೇಳೆ ಹೇಳುವರು. ಅವರ ಗುರುಗಳು ಮತ್ತು\break ಗುರುಗಳ ಗುರುಗಳು ಕೂಡ ಅದೇ ವೇದಾಂತ ಸಂಪ್ರದಾಯಕ್ಕೆ ಸೇರಿದವರು. ಕೆಲವು ವಿಷಯಗಳಲ್ಲಿ ಶಂಕರರಿಗಿಂತ ಧೈರ್ಯವಾಗಿ ಮುನ್ನುಗ್ಗಿರುವರು. ಆದಕಾರಣ ಶಂಕರಾಚಾರ್ಯರು ಕೂಡ ಹೊಸದಾಗಿ ಏನನ್ನೂ ಬೋಧಿಸಲಿಲ್ಲ ಎಂಬುದು ನಿಜವಾಯಿತು. ಶಂಕರಾಚಾರ್ಯರು ಕೂಡ ರಾಮಾನುಜರು ಬೋಧಾಯನದಿಂದ ತೆಗೆದುಕೊಂಡಂತೆ ಇತರರ ಯಾವುದೋ ಭಾಷ್ಯದಿಂದ ಭಾವನೆಗಳನ್ನು ತೆಗೆದುಕೊಂಡಿರಬೇಕು. ಆದರೆ ಅದು ಯಾವ ಭಾಷ್ಯವೆಂದು ಈಗ ಹೇಳಲು ಸಾಧ್ಯವಿಲ್ಲ.

ನೀವು ಕೇಳಿದ, ನೋಡಿದ ದರ್ಶನಗಳೆಲ್ಲಾ ಉಪನಿಷತ್ತುಗಳನ್ನು ಆಧರಿಸಿದೆ. ದಾರ್ಶನಿಕರು ಯಾವಾಗಲಾದರೂ ಶ್ರುತಿಯನ್ನು ಉದ್ಧರಿಸುತ್ತಿದ್ದೇವೆ ಎಂದರೆ ಉಪನಿಷತ್ತುಗಳನ್ನು ಎಂದು ಅರ್ಥ. ಅವರು ಯಾವಾಗಲೂ ಉಪನಿಷತ್ತು\-ಗಳನ್ನೇ ಉದ್ಧರಿಸುತ್ತಾರೆ. ಉಪನಿಷತ್ತುಗಳನ್ನು ಅನುಸರಿಸಿ ಹಲವು ತತ್ತ್ವಗಳು ಅಸ್ತಿತ್ವಕ್ಕೆ ಬಂದವು. ಆದರೆ ಭಾರತದಲ್ಲಿ ವ್ಯಾಸರ ತತ್ತ್ವವು ಬೀರಿದಷ್ಟು ಪ್ರಭಾವವನ್ನು ಮತ್ತಾವುದೂ ಬೀರಲು ಸಾಧ್ಯವಾಗಲಿಲ್ಲ. ವ್ಯಾಸರ ತತ್ತ್ವಕೂಡ ಅದಕ್ಕೆ ಹಿಂದೆ ಇದ್ದ ಸಾಂಖ್ಯ ಸಿದ್ಧಾಂತದ ವಿಕಾಸ ಮಾತ್ರವಾಗಿದೆ, ಮತ್ತು ಭಾರತದ, ಅಷ್ಟೇಕೆ ಇಡೀ ಜಗತ್ತಿನ ತತ್ತ್ವ ಸಂಪ್ರದಾಯಗಳೆಲ್ಲಾ ಕಪಿಲನಿಗೆ ಚಿರಋಣಿಗಳು. ಮನಃಶಾಸ್ತ್ರ ಮತ್ತು ದರ್ಶನ ಶಾಸ್ತ್ರಗಳಲ್ಲಿ ಅವನು ಭರತಖಂಡದ ಅತ್ಯಂತ ಪ್ರಖ್ಯಾತ ವ್ಯಕ್ತಿ. ಜಗತ್ತಿನಲ್ಲೆಲ್ಲಾ ನಾವು ಕಪಿಲನ ಪ್ರಭಾವವನ್ನು ನೋಡಬಹುದು. ಎಲ್ಲಿ ಒಂದು ಶಾಸ್ತ್ರೀಯ ಸಿದ್ಧಾಂತವಿದೆಯೋ ಅಲ್ಲೆಲ್ಲಾ ಕಪಿಲನ ಪ್ರಭಾವವನ್ನು ನೋಡಬಹುದು. ಸಾವಿರಾರು ವರ್ಷಗಳು ಹಿಂದೆ ಹೋದರೂ ತೇಜಸ್ವಿ, ಗೌರವಯುಕ್ತ, ಅಪೂರ್ವ ಪ್ರತಿಭಾಶಾಲಿ ಕಪಿಲ ಕಂಗೊಳಿಸುತ್ತಿರುವನು. ಇಂಡಿಯಾ ದೇಶದಲ್ಲಿ ಎಲ್ಲಾ ವಿಭಿನ್ನ ಪಂಥಗಳೂ ಕಪಿಲನ ಮನಃಶಾಸ್ತ್ರದ ಮತ್ತು ತತ್ತ್ವಗಳ ಸ್ವಲ್ಪ ಭಾಗವನ್ನು ಬಿಟ್ಟು ಉಳಿದೆಲ್ಲವನ್ನೂ ಸ್ವೀಕರಿ\-ಸಿರುವರು. ನಮ್ಮ ದೇಶದಲ್ಲೇ ನ್ಯಾಯಸಿದ್ಧಾಂತದವರಿಗೆ ಭರತಖಂಡದ ತತ್ತ್ವಶಾಸ್ತ್ರದ ಮೇಲೆ ಅಷ್ಟು ಪ್ರಭಾವವನ್ನು ಬೀರಲಾಗಲಿಲ್ಲ. ಅವರು ಸಾಮಾನ್ಯ, ವಿಶೇಷ, ಜಾತಿ, ದ್ರವ್ಯ, ಗುಣ ಮುಂತಾದ ಜಟಿಲವಾದ ಪಾರಿಭಾಷಿಕ ಶಬ್ದಗಳಲ್ಲೇ ನಿರತರಾಗಿದ್ದರು. ಅವನ್ನು ಓದುವುದಕ್ಕೆ ಒಂದು ಜೀವನವೇ ಸಾಲದು. ಅವರು ಕೇವಲ ತರ್ಕದಲ್ಲೇ ನಿರತರಾಗಿದ್ದು ತತ್ತ್ವವನ್ನು ವೇದಾಂತಿಗಳಿಗೆ ಬಿಟ್ಟರು. ಭರತಖಂಡದ ಇತ್ತೀಚಿನ ತತ್ತ್ವ ಸಂಪ್ರದಾಯದವರೆಲ್ಲಾ ವಂಗ ನೈಯಾಯಿಕರ ಪಾರಿಭಾಷಿಕ ಶಬ್ದಗಳನ್ನು ಬಳಸಿಕೊಂಡಿರುವರು. ಜಗದೀಶ, ಗದಾಧರ, ಶಿರೋಮಣಿ ಮುಂತಾದವರು ನದಿಯಾದಲ್ಲಿ ಎಷ್ಟು ಪ್ರಖ್ಯಾತರಾಗಿದ್ದರೋ,\break ಮಲಬಾರಿನ ಕೆಲವು ನಗರಗಳಲ್ಲೂ ಅಷ್ಟೇ ಪ್ರಖ್ಯಾತರಾಗಿರುವರು. ಆದರೆ ವ್ಯಾಸ ಪ್ರಣೀತ ವೇದಾಂತ ದರ್ಶನವು ಭಾರತದಲ್ಲಿ ದೃಢವಾಗಿ ಪ್ರತಿಷ್ಠಿತವಾಗಿದೆ. ಆ ದರ್ಶನವು ಬೋಧಿಸುವ ಬ್ರಹ್ಮದಷ್ಟೇ ಅದೂ ಶಾಶ್ವತವಾಗಿ ಉಳಿಯಿತು. ವ್ಯಾಸದರ್ಶನದಲ್ಲಿ ಯುಕ್ತಿಯು ಸಂಪೂರ್ಣವಾಗಿ ಶ್ರುತಿಗೆ ಅಧೀನವಾಗಿದೆ. ವ್ಯಾಸರು ಯುಕ್ತಿಯಿಂದ ಯಾವುದನ್ನೂ ಸಿದ್ಧಾಂತಗೊಳಿಸಲು ಪ್ರಯತ್ನಿಸಲಿಲ್ಲ ಎಂದು ಶಂಕರಾಚಾರ್ಯರು ಒಂದು ಕಡೆ ಹೇಳುವರು. ಅವರು ಸೂತ್ರಗಳನ್ನು ಬರೆದದ್ದು ವೇದಾಂತಶಾಸ್ತ್ರ ಪುಷ್ಪಗಳನ್ನೆಲ್ಲಾ ಒಂದು ಸೂತ್ರದಲ್ಲಿ ಬಂಧಿಸುವುದಕ್ಕಾಗಿ. ಎಲ್ಲಿಯವರೆಗೂ ಅವರ ಸೂತ್ರಗಳು ಉಪನಿಷತ್ತುಗಳಿಗೆ ಅಧೀನವಾಗಿವೆಯೋ ಅಲ್ಲಿಯವರೆಗೂ ಅವಕ್ಕೆ ಮಾನ್ಯತೆ, ಇಲ್ಲದೆ ಇದ್ದರೆ ಇಲ್ಲ.

\vskip 2pt

ನಾನು ಈಗಾಗಲೇ ಹೇಳಿದಂತೆ ಭರತಖಂಡದ ಸಂಪ್ರದಾಯಗಳೆಲ್ಲಾ ವ್ಯಾಸ ಸೂತ್ರಗಳನ್ನು ಅತಿ ಶ್ರೇಷ್ಠ ಪ್ರಮಾಣವೆಂದು ಒಪ್ಪಿಕೊಳ್ಳುವುವು. ಪ್ರತಿಯೊಂದು ಹೊಸ ಸಂಪ್ರಾಯದವರೂ ಕೂಡ ತಮ್ಮ ಭಾವನೆಗೆ ತಕ್ಕಂತೆ ಅವುಗಳಿಗೆ ಹೊಸ ಭಾಷ್ಯವನ್ನು ಬರೆದಿದ್ದಾರೆ. ಕೆಲವು ವೇಳೆ ಈ ಭಾಷ್ಯಕಾರರ ಮತಗಳಲ್ಲಿ ತುಂಬಾ ವ್ಯತ್ಯಾಸವಿದೆ. ಮತ್ತೆ ಕೆಲವು ವೇಳೆ ಅವರು ಮೂಲಸೂತ್ರಗಳನ್ನು ವಿಕೃತಗೊಳಿಸಿರುವುದನ್ನು ನೋಡಿದರೆ ಜುಗುಪ್ಸೆಯಾಗುವುದು. ವ್ಯಾಸಸೂತ್ರಗಳಿಗೆ ಅತಿ ಶ್ರೇಷ್ಠ ಪ್ರಮಾಣ ಗ್ರಂಥವೆಂಬ ಸ್ಥಾನ ದೊರೆತಿದೆ. ಅವುಗಳ ಮೇಲೆ ಬೇರೊಂದು ಭಾಷ್ಯವನ್ನು ಬರೆದಲ್ಲದೆ ಯಾರೂ ಹೊಸ ಸಂಪ್ರದಾಯವನ್ನು ಭಾರತದಲ್ಲಿ ಜಾರಿಗೆ ತರಲಾರರು.

\vskip 2pt

ವ್ಯಾಸಸೂತ್ರದ ಅನಂತರದ ಪ್ರಮಾಣವೇ ವಿಶ್ವವಿಖ್ಯಾತ ಭಗವದ್ಗೀತೆ. ಶಂಕರಾಚಾರ್ಯರು ಗೀತೆಯನ್ನು ಪ್ರಚಾರಮಾಡಿದರು. ಅವರ ಕೀರ್ತಿ ಇದರ ಮೇಲೆ ನಿಂತಿದೆ. ಅದರ ಮೇಲೆ ಅತಿ ಸುಂದರವಾದ ವ್ಯಾಖ್ಯಾನವನ್ನು ಬರೆದರು.\break ಈ ಮಹಾಪುರುಷರು ತಮ್ಮ ಮಹಾಜೀವನದಲ್ಲಿ ಮಾಡಿದ ಅದ್ಭುತ\break ಕಾರ್ಯಗಳಲ್ಲಿ ಇದು ಒಂದು. ಸನಾತನ ಸಿದ್ಧಾಂತದ ಸಂಸ್ಥಾಪಕರೆಲ್ಲಾ ಇವರನ್ನು ಅನುಸರಿಸುವರು. ಅವರಲ್ಲಿ ಪ್ರತಿಯೊಬ್ಬರೂ ಗೀತೆಯ ಮೇಲೆ ಒಂದೊಂದು ಭಾಷ್ಯವನ್ನು ಬರೆದಿರುವರು.

ಹಲವು ಉಪನಿಷತ್ತುಗಳು ಇವೆ, ನೂರೆಂಟು ಎಂದು ಹೇಳುವರು. ಅವುಗಳಲ್ಲಿ ಕೆಲವು ಬಹಳ ಇತ್ತೀಚಿನವು. ಅಲ್ಲಾನನ್ನು ಸ್ತುತಿಸಿರುವ ಅಲ್ಲೋಪನಿಷತ್ತು ಇತ್ತೀಚಿನದು. ಅದರಲ್ಲಿ ಮಹಮ್ಮದನನ್ನು ರಜಸುಲ್ಲಾನನ್ನಾಗಿ ಮಾಡಿದರು.\break ಅಕ್ಬರನ ಕಾಲದಲ್ಲಿ ಹಿಂದೂ-ಮಹಮ್ಮದೀಯರ ಸಖ್ಯ ಬೆಳಸುವುದಕ್ಕಾಗಿ ಇದನ್ನು ರಚಿಸಿದರು ಎನ್ನುವರು. ಸಂಹಿತೆಯಲ್ಲಿ ಬರುವ ಅಲ್ಲಾ ಅಥವಾ ಇಲ್ಲಾ ಎಂಬ ಪದವನ್ನು ತೆಗೆದುಕೊಂಡು ಅದರ ಮೇಲೆ ಒಂದು ಉಪನಿಷತ್ತನ್ನು ರಚಿಸಿದರು. ಅದರ ಅರ್ಥ ಏನೇ ಆಗಿರಲಿ, ಅದರಲ್ಲಿ ಮಹಮ್ಮದನು ರಜಸುಲ್ಲಾ ಎಂದು ಹೇಳಿದೆ. ಇಂತಹ ಹಲವು ಪಂಥೀಯ ಉಪನಿಷತ್ತುಗಳಿವೆ. ಅವುಗಳೆಲ್ಲಾ ಕೇವಲ ಇತ್ತೀಚಿನದೆಂದು ತೋರುವುದು. ಇದನ್ನು ಬರೆಯುವುದೂ ಸುಲಭ. ಸಂಹಿತೆಯ ಭಾಷೆ ಅತಿ ಪ್ರಾಚೀನವಾದುದು. ಅಲ್ಲಿ ವ್ಯಾಕರಣವೇ ಇಲ್ಲ.

ಕೆಲವು ವರ್ಷಗಳ ಹಿಂದೆ ನನಗೆ ವೈದಿಕ ವ್ಯಾಕರಣವನ್ನು ಓದಬೇಕೆಂದು ಆಸೆಯಾಯಿತು. ಬಹಳ ಉತ್ಸಾಹದಿಂದ ಪಾಣಿನಿ ಸೂತ್ರಗಳನ್ನು ಮತ್ತು ಮಹಾಭಾಷ್ಯವನ್ನು ಓದಲು ಪ್ರಯತ್ನಿಸಿದೆ. ಬಹಳ ಆಶ್ಚರ್ಯವಾದುದೇನೆಂದರೆ ವೈದಿಕ ವ್ಯಾಕರಣದಲ್ಲಿ ಸಾಧಾರಣ ನಿಯಮಗಳಿಗಿಂತ ಅಪವಾದಗಳೇ ಹೆಚ್ಚಾಗಿರು\-ವುದು. ಒಂದು ನಿಯಮವನ್ನು ಮಾಡುವರು. ಅನಂತರ ಒಂದು ವಿನಾಯಿತಿ ತರುವರು. ಈ ನಿಯಮ ಇದಕ್ಕೆ ಅನ್ವಯಿಸುವುದಿಲ್ಲ ಎನ್ನುವರು. ಯಾರಿಗಾದರೂ ಏನನ್ನಾದರೂ ಬರೆಯುವುದಕ್ಕೆ ಎಷ್ಟೊಂದು ಸ್ವಾತಂತ್ರ್ಯವಿದೆ ಎಂಬುದನ್ನು ನೋಡಿ! ಯಾಸ್ಕರ ನಿರುಕ್ತ ಒಂದೇ ರಕ್ಷಣೆ. ಆದರೂ ಇದರಲ್ಲಿ ಹಲವು ಸಮಾನಾರ್ಥ ಪದಗಳಿವೆ. ಹೀಗಿದ್ದರೆ ನಿಮಗೆ ಇಚ್ಛೆ ಬಂದಷ್ಟು ಉಪನಿಷತ್ತುಗಳನ್ನು ಬರೆಯುವುದು ಎಷ್ಟು ಸುಲಭ! ಸ್ವಲ್ಪ ಸಂಸ್ಕೃತ ಭಾಷೆಯ ಪರಿಚಯವಿದ್ದು ಹಿಂದಿನ ಸಂಸ್ಕೃತ ಪದಗಳಂತೆ ಕಾಣುವ ಪದಗಳನ್ನು ಸೃಷ್ಟಿಸಿದರೆ ಸಾಕು. ವ್ಯಾಕರಣದ ಅಂಜಿಕೆಯೇ ಇಲ್ಲ ಅನಂತರ ರಜಸು ಲ್ಲಾನೋ ಯಾವನಾದರೂ ಒಬ್ಬ ಸುಲ್ಲಾನನ್ನು ತಂದರೂ ಸಾಕು. ಹೀಗೆ ಹಲವು ಉಪನಿಷತ್ತುಗಳನ್ನು ತಯಾರು ಮಾಡಿರುವರು. ಈಗಲೂ ಕೂಡ ತಯಾರು ಮಾಡುತ್ತಾ ಇರುವರು ಎಂದು ಕೇಳಿದೆ. ಪ್ರತಿಯೊಂದು ಸಂಪ್ರದಾಯದವರೂ ಇಂಡಿಯಾ ದೇಶದ ಕೆಲವು ಕಡೆಗಳಲ್ಲಿ ಇಂತಹ ಉಪನಿಷತ್ತುಗಳನ್ನು ಸೃಷ್ಟಿಸಲು ನಿಸ್ಸಂದೇಹವಾಗಿ ಪ್ರಯತ್ನಪಡುತ್ತಿರಬಹುದು. ಆದರೆ ಉಪನಿಷತ್ತುಗಳಲ್ಲಿ ಕೆಲವು ಅತಿ ಪ್ರಾಚೀನವಾದುವು, ಮತ್ತು ಪರಿಶುದ್ಧವಾದುವೆಂದು ನಿರ್ವಿವಾದವಾಗಿ ಹೇಳಬಹುದು. ಅವುಗಳ ಮೇಲೆ ಪ್ರಸಿದ್ಧ ಭಾಷ್ಯಕಾರರಾದ ಶಂಕರಾಚಾರ್ಯರು ಭಾಷ್ಯವನ್ನು ಬರೆದಿರುವರು. ಅನಂತರ ರಾಮಾನುಜರೇ ಮುಂತಾದವರು ಅವುಗಳ ಮೇಲೆ ಭಾಷ್ಯಗಳನ್ನು ಬರೆದಿರುವರು.

ಉಪನಿಷತ್ತುಗಳ ಒಂದೆರಡು ಭಾವನೆಗಳನ್ನು ನಿಮ್ಮ ಗಮನಕ್ಕೆ ತರಬೇಕೆಂದಿರು ವೆನು. ಇವು ಒಂದು ಜ್ಞಾನಸಾಗರ. ನನ್ನಂತಹ ಅಲ್ಪಮತಿಗಳಿಗೂ ಕೂಡ ಇವುಗಳ ಮೇಲೆ ಮಾತನಾಡಬೇಕಾದರೆ ಹಲವು ವರ್ಷಗಳು ಹಿಡಿಯುವುವು. ಇದು ಒಂದು ಉಪನ್ಯಾಸದಿಂದ ಸಾಧ್ಯವಿಲ್ಲ. ಒಂದೆರಡು ಭಾವನೆಗಳನ್ನು ಮಾತ್ರ ನಿಮ್ಮ ಗಮನಕ್ಕೆ ತರುವೆನು. ಮೊದಲನೆಯದಾಗಿ ಅವು ಜಗತ್ತಿನ ಶ್ರೇಷ್ಠ ಕಾವ್ಯಗಳು. ವೇದಗಳಲ್ಲಿ ಸಂಹಿತೆಯ ಭಾಗವನ್ನು ಓದುತ್ತಿದ್ದರೆ ಮಧ್ಯೆ ಮಧ್ಯೆ ಅತಿ ಸುಂದರವಾಗಿರುವ ಭಾಗಗಳು ದೊರಕುವುವು. ಅದರಲ್ಲಿ ಪ್ರಳಯದ ಅಂಧಕಾರವನ್ನು ವಿವರಿಸುವಾಗ “ತಮ ಆಸೀತ್​ ತಮಸಾ ಗೂಢಮಗ್ರೇ” ತಮಸ್ಸು ತಮಸ್ಸಿನಲ್ಲಿ\break ಐಕ್ಯವಾಗಿತ್ತು ಎಂದು ಹೇಳಿದೆ. ಇದನ್ನು ಓದುವಾಗ ಕಾವ್ಯದ ಅಪೂರ್ವ\break ಭವ್ಯತೆಯ ಅನುಭವವಾಗುವುದು. ಇದನ್ನು ಗಮನಿಸಿ. ಭರತಖಂಡದಲ್ಲಿ ಮತ್ತು ಹೊರಗೆ ಭವ್ಯತೆಯ ಭಾವವನ್ನು ಚಿತ್ರಿಸಲು ಪ್ರಯತ್ನಪಟ್ಟಿರುವರು. ಆದರೆ ಭಾರತದ ಹೊರಗಿನವರ ದೃಷ್ಟಿಯಲ್ಲಿ ಭವ್ಯತೆ ಎಂದರೆ ಬಾಹುಬಲದ ಭವ್ಯತೆಯೇ, ಅಥವಾ ಹೊರಜಗತ್ತು, ಭೌತವಸ್ತು ಅಥವಾ ಆಕಾಶ ಇವುಗಳ ಅನಂತತೆಯೇ. ಮಿಲ್ಟನ್​, ಡಾಂಟೆ ಅಥವಾ ಈಚಿನ ಅಥವಾ ಪುರಾತನ ಪ್ರಸಿದ್ಧ ಕವಿಗಳು ಅನಂತವನ್ನು ವಿವರಿಸಬೇಕಾದರೆ ಹೊರಗೆ ಹೋಗಿ ಬಾಹ್ಯ ಪ್ರಕೃತಿಯ ಮೂಲಕ ಅನಂತದ ಭಾವನೆಯನ್ನು ಕೊಡಲು ಯತ್ನಿಸುವರು. ಇಲ್ಲಿಯೂ ಆ ಪ್ರಯತ್ನವನ್ನು ಮಾಡಿರುವರು. ಸಂಹಿತೆಯಲ್ಲಿ ಅನಂತಾಕಾಶವನ್ನು ಅತಿ ಸುಂದರವಾಗಿ ಇನ್ನೆಲ್ಲಿಯೂ ಇದಕ್ಕೆ ಸಮಾನವಾದದ್ದು ಇಲ್ಲವೆಂಬಂತೆ ಚಿತ್ರಿಸಿರುವರು. “ತಮ ಆಸೀತ್​ ತಮಸಾ ಗೂಢಂ” ಎಂಬ ಒಂದು ವರ್ಣನೆಯನ್ನು\break ನೋಡಿ. ಕತ್ತಲೆಯನ್ನು ಮೂವರು ಕವಿಗಳು ವರ್ಣಿಸಿರುವ ರೀತಿಯನ್ನು\break ಗಮನಿಸಿ. ನಮ್ಮ ಕಾಳಿದಾಸನ ವರ್ಣನೆಯನ್ನೇ ತೆಗೆದುಕೊಳ್ಳಿ: ಸೂಚೀಭೇದ್ಯ ಅಂಧಕಾರ (ಸೂಜಿಯಿಂದ ಭೇದಿಸುವಷ್ಟು ಘನೀಭೂತವಾದ ಕತ್ತಲೆ). ಮಿಲ್ಟನ್​: “ಬೆಳಕಿಲ್ಲ, ನೋಡಬಹುದಾದ ಅಂಧಕಾರ.” ಆದರೆ ಉಪನಿಷತ್ತಿಗೆ ಬಂದರೆ,\break “ಅಂಧಕಾರ ಅಂಧಕಾರವನ್ನು ವ್ಯಾಪಿಸಿತ್ತು. ಅಂಧಕಾರ ಅಂಧಕಾರದಲ್ಲಿ\break ಹೊಕ್ಕಿತ್ತು” ಎಂದಿದೆ. ಉಷ್ಣವಲಯದಲ್ಲಿ ವಾಸಿಸುವವರಿಗೆ ಇದು ಚೆನ್ನಾಗಿ ವೇದ್ಯವಾಗುವುದು. ವರ್ಷಾಕಾಲದಲ್ಲಿ ಮಳೆ ಬೀಳುವಾಗ ಕ್ಷಣದಲ್ಲಿ ಆಕಾಶವೆಲ್ಲಾ ಕಪ್ಪು ಮೋಡಗಳಿಂದ ಆವೃತವಾಗುವುದು, ಮತ್ತೂ ಹೆಚ್ಚು ಕಪ್ಪು ಮುಗಿಲುಗಳು ಅವುಗಳ ಮೇಲೆ ಕವಿಯುವುವು. ಕಾವ್ಯ ಹೀಗೆ ಮುಂದುವರಿಯುವುದು. ಆದರೆ ಸಂಹಿತೆಯಲ್ಲಿ ಇದೆಲ್ಲಾ ಬಾಹ್ಯವರ್ಣನೆ. ಎಲ್ಲಾ ಕಡೆಯಂತೆ ಇಲ್ಲಿಯೂ ಜೀವನದ ಮಹಾ ಸಮಸ್ಯೆಗಳನ್ನು ಅರಿಯುವುದಕ್ಕೆ ಬಾಹ್ಯ ಜಗತ್ತಿನ ಮೂಲಕ ಪ್ರಯತ್ನಿಸುವರು.

ಹೀಗೆ ಪ್ರಾಚೀನ ಗ್ರೀಕರು ಮತ್ತು ಆಧುನಿಕ ಯೂರೋಪಿಯನರು ಜೀವನಕ್ಕೆ ಮತ್ತು ಜಗತ್ತಿಗೆ ಕಾರಣವಾದ ಪರಮಾತ್ಮ ತತ್ತ್ವಕ್ಕೆ ಸಂಬಂಧಿಸಿದ ಸಮಸ್ಯೆಗಳ ಪರಿಹಾರವನ್ನು ಬಾಹ್ಯ ಪ್ರಕೃತಿಯಲ್ಲಿ ಹುಡುಕುವರು. ಅದರಂತೆಯೇ ನಮ್ಮ ಪೂರ್ವಿಕರೂ ಪ್ರಯತ್ನಪಟ್ಟರು. ಯೂರೋಪಿಯನ್ನರು ಈ ಪ್ರಯತ್ನದಲ್ಲಿ ಸೋತಂತೆ ನಮ್ಮವರೂ ಸೋತರು. ಪಾಶ್ಚಾತ್ಯರು ಅಲ್ಲಿಂದ ಮುಂದೆ ಹೋಗಲಿಲ್ಲ. ಅವರು ಅಲ್ಲೇ ದಾರಿತಪ್ಪಿ ನಿಂತರು. ನಮ್ಮ ಪೂರ್ವಿಕರಿಗೂ ಇದು ಅಸಾಧ್ಯವೆಂದು ಗೊತ್ತಾಯಿತು. ಇಂದ್ರಿಯಗಳಿಗೆ ಈ ವಿಷಯವನ್ನು ತಿಳಿದುಕೊಳ್ಳುವುದು ಅಸಾಧ್ಯವೆಂದು ಧೈರ್ಯವಾಗಿ ಅವರು ಒಪ್ಪಿಕೊಂಡರು. ಉಪನಿಷತ್ತುಗಳು ನೀಡುವ ವಿವರಣೆಗಿಂತ ಉತ್ತಮವಾದುದು ಬೇರೆಲ್ಲಿಯೂ ದೊರಕುವುದಿಲ್ಲ. \textbf{“ಯತೋ ವಾಚೋ ನಿವರ್ತಂತೇ ಅಪ್ರಾಪ್ಯ ಮನಸಾ ಸಹ”}, ಎಲ್ಲಿಂದ ಮಾತು ಮನಸ್ಸಿನೊಂದಿಗೆ ಹಿಂತಿರುಗುವುದೋ, \textbf{“ನ ತತ್ರ ಚಕ್ಷುರ್ಗಚ್ಛತಿನ ವಾಗ್​ ಗಚ್ಛತಿ”} ಅಲ್ಲಿಗೆ ಕಣ್ಣು ಹೋಗಲಾರದು, ಮಾತೂ ಹೋಗಲಾರದು” ಇಂದ್ರಿಯಗಳಿಗೆ ಅದನ್ನು ಗ್ರಹಿಸುವುದು ಅಸಾಧ್ಯ ಎಂದು ವಿವರಿಸುವ ಹಲವು ವಾಕ್ಯಗಳಿವೆ. ಆದರೆ ಅವರು ಅಲ್ಲೇ ನಿಲ್ಲಲಿಲ್ಲ. ಮನುಷ್ಯನ ಅಂತರಂಗದ ಕಡೆ ತಿರುಗಿದರು. ತಮ್ಮ ಅಂತರಾಳದಿಂದ ಅದಕ್ಕೆ ಉತ್ತರವನ್ನು ಕಂಡುಹಿಡಿಯಲು ಯತ್ನಿಸಿದರು.\break ಅವರು ಆಂತರ್ಮುಖಿಗಳಾದರು. ಬಾಹ್ಯ ಪ್ರಕೃತಿಯಲ್ಲಿ ಅವರಿಗೆ ಏನೂ\break ದೊರಕಲಿಲ್ಲ. ದೊರಕುವ ಸಂಭವವೂ ಇಲ್ಲ. ಅಲ್ಲಿ ಯಾವ ಉತ್ತರವೂ ದೊರಕದು. ಅದನ್ನು ತ್ಯಜಿಸಿದರು. ಜಡ ಅಚೇತನ ಪ್ರಕೃತಿ ನಮಗೆ ಸತ್ಯವನ್ನು ಕೊಡಲಾರದು ಎಂದು ಅದನ್ನು ತ್ಯಜಿಸಿದರು. ಜ್ಯೋತಿರ್ಮಯ ಜೀವಾತ್ಮನ ಕಡೆ ತಿರುಗಿದರು. ಅಲ್ಲಿ ಉತ್ತರ ಸಿಕ್ಕಿತು.

\textbf{“ತಮೇವೈಕಂ ಜಾನಥ ಆತ್ಮಾನಂ ಅನ್ಯಾ ವಾಚೋ ವಿಮುಂಚಥ.”}\break (ಮುಂಡಕೋಪನಿಷತ್​)- ಈ ಆತ್ಮವನ್ನು ಮಾತ್ರ ತಿಳಿಯಿರಿ, ಉಳಿದ ವ್ಯರ್ಥಾ\-ಲಾಪವನ್ನು ತ್ಯಜಿಸಿ. ಆತ್ಮನಲ್ಲಿ ಎಲ್ಲಾ ಸಮಸ್ಯೆಗಳಿಗೂ ಸಮಾಧಾನ ದೊರಕಿತು. ಪರಮಾತ್ಮ, ಈಶ್ವರ, ಜಗದೊಡೆಯ ಈ ಎಲ್ಲರಿಗೆ ಸಂಬಂಧಪಟ್ಟ ವಿಷಯಗಳು ಅಲ್ಲಿ ದೊರೆತವು. ಜೀವಾತ್ಮನಿಗೂ ಪರಮಾತ್ಮನಿಗೂ ಇರುವ ಸಂಬಂಧ, ಮಾನವ ಕರ್ತವ್ಯ, ಅವನಿಗೂ ಇತರರಿಗೂ ಇರುವ ಸಂಬಂಧ, ಈ ಎಲ್ಲ ಜ್ಞಾನ ಅಲ್ಲಿ ದೊರಕಿತು. ಆತ್ಮತತ್ತ್ವದ ವರ್ಣನೆಯಲ್ಲಿ ಜಗತ್ತಿನ ಅತಿ ಶ್ರೇಷ್ಠ ಕಾವ್ಯವಿದು. ಜಡ ಭಾಷೆಯ ಮೂಲಕ ಆತ್ಮನನ್ನು ವಿವರಿಸುವ ಯತ್ನ ಇನ್ನು ಇಲ್ಲ.\break ಅದಕ್ಕಾಗಿ ಇತ್ಯಾತ್ಮಕ \enginline{(positive)} ಭಾಷೆಯನ್ನೆಲ್ಲಾ ತೊರೆದಿರುವರು. ಅನಂತತೆಯ ಭಾವವನ್ನು ಮನಸ್ಸಿಗೆ ತರುವುದಕ್ಕೆ ಇಂದ್ರಿಯಗಳನ್ನು ಆಶ್ರಯಿಸುವ ಪ್ರಯತ್ನವೇ ಇಲ್ಲ. ಬಾಹ್ಯ, ಜಡ, ಅಚೇತನ ಆಕಾಶಸಂಬಂಧದ ಇಂದ್ರಿಯಗ್ರಾಹ್ಯ ಅನಂತ ಇನ್ನು ಮೇಲೆ ಇಲ್ಲ. ಅದರ ಬದಲು ಇಂತಹ ಸುಂದರ ಭಾವನೆಗಳು ಬರುವುವು:

\vskip 2pt

\begin{longtable}[r]{@{}l@{}}
\textbf{ನ ತತ್ರ ಸೂರ್ಯೋ ಭಾತಿ ನ ಚಂದ್ರತಾರಕಂ} \\
\textbf{ನೇಮಾ ವಿದ್ಯುತೋ ಭಾಂತಿ ಕುತೋಽಯಮಗ್ನಿಃ ।} \\
\textbf{ತಮೇವ ಭಾಂತಮನುಭಾತಿ ಸರ್ವಂ} \\
\textbf{ತಸ್ಯ ಭಾಸಾ ಸರ್ವಮಿದಂ ವಿಭಾತಿ ॥} \\
\end{longtable}

\vskip 2pt

“ಅಲ್ಲಿ ಸೂರ್ಯ ಬೆಳಗಲಾರ, ಚಂದ್ರ ಬೆಳಗಲಾರ, ತಾರೆಗಳು ಬೆಳಗ\-ಲಾರವು. ಅಲ್ಲಿ ಮಿಂಚು ಬೆಳಗಲಾರದು. ಇನ್ನು ಈ ಅಗ್ನಿಯ ಪಾಡೇನು!” ಪೃಥ್ವಿಯಲ್ಲಿ ಯಾವ ಕಾವ್ಯವು ಇದಕ್ಕಿಂತ ಸುಂದರವಾಗಿರಬಲ್ಲದು! ಇಂತಹ ಕಾವ್ಯವು ಮತ್ತೆಲ್ಲಿಯೂ ದೊರಕದು. ಅಪೂರ್ವ ಕಠೋಪನಿಷತ್ತನ್ನು ತೆಗೆದುಕೊಳ್ಳಿ. ರಚನೆ ಎಷ್ಟು ಸುಂದರವಾಗಿದೆ ನೋಡಿ! ಎಷ್ಟು ಮನೋಹರ ರೀತಿಯಲ್ಲಿ ಕಾವ್ಯವನ್ನು ಆರಂಭಮಾಡಿರುವರು ನೋಡಿ! ಆ ಸಣ್ಣ ಬಾಲಕ ನಚಿಕೇತನ ಹೃದಯದಲ್ಲಿ ಶ್ರದ್ಧೆ ಬಂದು ಯಮನ ದರ್ಶನಕ್ಕೆ ಹೋಗುವನು. ಅತಿ ಕುಶಲಿಯಾದ ಗುರು, ಸ್ವಯಂ ಯಮನೇ ಜನನ ಮರಣಗಳ ರಹಸ್ಯವನ್ನು ವಿವರಿಸುವನು. ಆ ಹುಡುಗನ ಅನ್ವೇಷಣೆ ಯಾವುದು? ಮೃತ್ಯುವಿನ ರಹಸ್ಯವನ್ನು ಅರಿಯುವುದು!

ನೀವು ಉಪನಿಷತ್ತುಗಳ ಬೋಧನೆಯಲ್ಲಿ ಗಮನದಲ್ಲಿಡಬೇಕಾದ\break ಎರಡನೆಯ ಅಂಶವೇ ಅವುಗಳ ಅಪೌರುಷೇಯತ್ವ. ಹಲವು ಹೆಸರುಗಳು, ಹಲವು ವ್ಯಕ್ತಿಗಳು, ಆಚಾರ್ಯರು ಉಪನಿಷತ್ತುಗಳಲ್ಲಿ ಬಂದರೂ ಯಾರೂ ಉಪನಿಷತ್ತುಗಳಿಗೆ ಪ್ರಮಾಣ ಪುರುಷರಲ್ಲ, ಯಾವ ಒಂದು ಶ್ಲೋಕಕ್ಕೂ ಅವರ ಜೀವನದ ಆಧಾರ ಬೇಕಿಲ್ಲ. ಅವರು ಹಿನ್ನೆಲೆಯಲ್ಲಿ ಗೋಚರವಾಗದೆ, ಭಾವನೆಗೆ ನಿಲುಕದೆ ಸಂಚರಿಸುತ್ತಿರುವ ಛಾಯಾಮೂರ್ತಿಗಳಂತೆ ಇರುವರು. ಉಪನಿಷತ್ತುಗಳ ಅಪೂರ್ವ ಮಹಿಮೆ, ಜ್ಯೋತಿರ್ ಮಯವೂ ತೇಜೋಮಯವೂ ಆದ ಮಂತ್ರಗಳಲ್ಲಿರುವ ಶಕ್ತಿಯಲ್ಲಿದೆ, ವ್ಯಕ್ತಿ ವಿಶೇಷದ ಮೇಲೆ ಇಲ್ಲ. ಇಪ್ಪತ್ತು ಯಾಜ್ಞವಲ್ಕ್ಯರು ಹುಟ್ಟಿ ಬಾಳಿ ಅಳಿದರೂ ಚಿಂತೆಯಿಲ್ಲ. ಗ್ರಂಥ ಅಲ್ಲಿದೆ. ಆದರೂ ಅವು ಯಾವ ವ್ಯಕ್ತಿಗೂ ವಿರುದ್ಧವಾಗಿಲ್ಲ. ಇದುವರೆಗೆ ಹುಟ್ಟಿದ ಮತ್ತು ಮುಂದೆ ಹುಟ್ಟಲಿರುವ ವ್ಯಕ್ತಿಗಳನ್ನೆಲ್ಲಾ ಸ್ವೀಕರಿಸುವಷ್ಟು ಅವು ವಿಶಾಲವಾಗಿದೆ. ವ್ಯಕ್ತಿಗಳು ಅವತಾರಗಳು ಋಷಿಗಳ ಆರಾಧನೆಗೆ ವಿರುದ್ಧವಾಗಿ ಅವು ಏನನ್ನೂ ಹೇಳುವು\-ದಿಲ್ಲ. ಬದಲಿಗೆ, ಅವರ ಆರಾಧನೆಯನ್ನು ಸಮರ್ಥಿಸುತ್ತವೆ. ಆದರೂ ಅವು ಯಾವ ವ್ಯಕ್ತಿಯ ಮೇಲೂ ನಿಂತಿಲ್ಲ. ಉಪನಿಷತ್ತಿನ ಅಪೌರುಷೇಯ ಭಾವನೆಯು ಅವು ಸಾರುವ ದೇವರಷ್ಟೇ ಅದ್ಭುತವಾದುದು. ಉಪನಿಷತ್ತುಗಳು ಋಷಿಗೆ, ಯೋಚನಾಪರನಿಗೆ, ತತ್ವಜ್ಞಾನಿಗೆ, ಯುಕ್ತಿಮತಿಗೆ, ಆಧುನಿಕ ವಿಜ್ಞಾನಿಗಳು ಅಪೇಕ್ಷಿಸುವಷ್ಟು ಅಪೌರುಷೇಯವಾಗಿದೆ. ಇವೇ ನಮ್ಮ ಶಾಸ್ತ್ರ, ಕ್ರೈಸ್ತರಿಗೆ ಬೈಬಲ್​ ಹೇಗೋ ಮಹಮ್ಮದೀಯರಿಗೆ ಖುರಾನು ಹೇಗೋ, ಬೌದ್ಧರಿಗೆ ತ್ರಿಪಿಟಿಕಗಳು ಹೇಗೋ, ಹಾಗೆಯೇ ಹಿಂದೂಗಳಿಗೆ ಉಪನಿಷತ್ತುಗಳು. ಇವೇ\break ನಮ್ಮ ಶಾಸ್ತ್ರ, ಮತ್ತಾವುದೂ ಅಲ್ಲ. ಪುರಾಣ, ತಂತ್ರ, ಇತರ ಶಾಸ್ತ್ರಗಳು,\break ವ್ಯಾಸಸೂತ್ರ ಕೂಡ, ಎರಡನೆಯ, ಮೂರನೆಯ ದರ್ಜೆಗೆ ಸೇರಿದವು. ವೇದಗಳು ಮುಖ್ಯ. ಮನುಸ್ಮೃತಿ, ಪುರಾಣ ಮತ್ತು ಇತರ ಶಾಸ್ತ್ರಗಳೆಲ್ಲಾ ಉಪನಿಷತ್ತುಗಳ ಪ್ರಮಾಣವನ್ನು ಎಲ್ಲಿಯವರೆಗೆ ಒಪ್ಪಿಕೊಳ್ಳುವುವೋ, ಅಲ್ಲಿಯವರೆಗೆ\break ಮಾತ್ರ ಸ್ವೀಕಾರಯೋಗ್ಯ. ಯಾವಾಗ ಒಪ್ಪಿಕೊಳ್ಳುವುದಿಲ್ಲವೋ ಆಗ\break ನಿರ್ದಾಕ್ಷಿಣ್ಯವಾಗಿ ಅವನ್ನು ತ್ಯಜಿಸಬೇಕು. ಇದನ್ನು ನಾವು ಯಾವಾಗಲೂ ಜ್ಞಾಪಕದಲ್ಲಿಡಬೇಕು. ಆದರೆ ದುರದೃಷ್ಟವಶಾತ್​ ಭರತಖಂಡದಲ್ಲಿ ಇದನ್ನು ಮರೆತರು. ಒಂದು ಸಣ್ಣ ಗ್ರಾಮದ ಆಚಾರ ಈಗ ಉಪನಿಷತ್ತಿಗಿಂತ ಹೆಚ್ಚು ಪ್ರಧಾನವಾಗಿದೆ. ಬಂಗಾಳ ದೇಶದ ಗ್ರಾಮದಲ್ಲಿರುವ ಭಾವನೆಯೊಂದಕ್ಕೆ ವೇದದಷ್ಟೇ ಪ್ರಾಧಾನ್ಯವಿದೆ. ವೇದಕ್ಕಿಂತ ಹೆಚ್ಚೇ ಇದೆ. ಆಚಾರ ಎಂಬ ಒಂದು ಪದದ ಪ್ರಭಾವ ಎಷ್ಟು ವಿಚಿತ್ರವಾಗಿದೆ. ನೋಡಿ! ಹಳ್ಳಿಯವನಿಗೆ ಕರ್ಮಕಾಂಡದ ಪ್ರತಿಯೊಂದು ಅಂಶವನ್ನೂ ಅನುಸರಿಸುವುದೆ ಆಚಾರಶೀಲತೆಯ ಪರಮಾವಧಿ. ಯಾರು ಅದನ್ನು ಅನುಸರಿಸುವುದಿಲ್ಲವೋ ಅವರನ್ನು “ನೀವು ಹಿಂದೂಗಳಲ್ಲ,\break ಹೊರಡಿ” ಎನ್ನುವೆವು. ದೌರ್ಭಾಗ್ಯವಶವಾಗಿ ನನ್ನ ತಾಯ್ನಾಡಿನಲ್ಲಿ ಈಗ\break ಕೆಲವರು ತಂತ್ರಶಾಸ್ತ್ರವನ್ನು ತೆಗೆದುಕೊಂಡು, ಈ ತಂತ್ರವನ್ನು ಎಲ್ಲರೂ ಅನುಸರಿಸಬೇಕು ಎನ್ನುವರು. ಯಾವನು ಹೀಗೆ ಮಾಡುವುದಿಲ್ಲವೋ ಆತ ಅವರ ದೃಷ್ಟಿಯಲ್ಲಿ ಆಚಾರವಂತನಲ್ಲ. ಆದ್ದರಿಂದ ಉಪನಿಷತ್ತುಗಳು ಪ್ರಮುಖ ಪ್ರಮಾಣ ಎಂಬುದನ್ನು ನೆನಪಿನಲ್ಲಿಡುವುದು ನಮಗೆ ಒಳ್ಳೆಯದು. ಗೃಹ್ಯ, ಶ್ರೌತ ಸೂತ್ರಗಳು ಕೂಡ ವೇದಗಳ ಪ್ರಮಾಣಕ್ಕೆ ಅಧೀನ. ಅವು ನಮ್ಮ ಪೂರ್ವಜರ, ಮಹಾಋಷಿಗಳ ವಾಣಿ. ನೀವೂ ಹಿಂದೂಗಳಾಗಬೇಕಾದರೆ ಅವನ್ನು ನಂಬಬೇಕು. ಈಶ್ವರ ಎಂದರೆ ನೀವು ಯಾವ ವಿಚಿತ್ರ ಭಾವನೆಯನ್ನು ಬೇಕಾದರೂ ಇಟ್ಟುಕೊಂಡಿರಬಹುದು. ಆದರೆ ವೇದ ಪ್ರಮಾಣವನ್ನು ಒಪ್ಪದೇ ಇದ್ದರೆ ನೀವು ನಾಸ್ತಿಕರು. ಕ್ರೈಸ್ತರ, ಬೌದ್ಧರ ಶಾಸ್ತ್ರಗಳಿಗೂ ನಮ್ಮ ಶಾಸ್ತ್ರಕ್ಕೂ ಇರುವ ವ್ಯತ್ಯಾಸ ಇದು. ಅವೆಲ್ಲ ಕೇವಲ ಪುರಾಣಗಳು, ಶಾಸ್ತ್ರಗಳಲ್ಲ. ಅವುಗಳಲ್ಲಿ ಪ್ರಳಯಕ್ಕೆ ಸಂಬಂಧಪಟ್ಟ ವಿಷಯ, ಆಳುವ ಅರಸರ ಮತ್ತು ವಂಶಗಳ ಹೆಸರು, ಮಹಾಪುರುಷರ ಜೀವನ ಮುಂತಾದವು ಇವೆ. ಅವು ಪೌರಾಣಿಕ ವಸ್ತುಗಳು. ವೇದಗಳೊಂದಿಗೆ ಎಲ್ಲಿಯವರೆಗೆ ಅವು ಹೊಂದಿಕೊಳ್ಳುವುವೋ ಅಲ್ಲಿಯವರೆಗೆ ಅವಕ್ಕೆ ಗೌರವ. ಎಲ್ಲಿಯವರೆಗೆ ಬೈಬಲ್ಲು ಮತ್ತು ಜಗತ್ತಿನ ಇತರ ಧರ್ಮಗ್ರಂಥಗಳು ವೇದಗಳೊಂದಿಗೆ ಹೊಂದಿಕೊಳ್ಳುವುವೋ ಅಲ್ಲಿಯವರೆಗೆ ಸರಿ. ಅವು ಹೊಂದಿಕೊಳ್ಳದಾಗ ಅವನ್ನು ಸ್ವೀಕರಿಸಬೇಕಾಗಿಲ್ಲ. ಅದರಂತೆಯೇ ಖುರಾನು ಕೂಡ. ಇಲ್ಲಿ ಎಷ್ಟೋ ನೈತಿಕ ವಿಷಯಗಳಿವೆ. ಎಲ್ಲಿಯವರೆಗೆ ಅವು ವೇದಗಳೊಂದಿಗೆ\break ಹೊಂದಿಕೊಳ್ಳುವುವೋ ಅಲ್ಲಿಯವರೆಗೆ ಅವಕ್ಕೆ ಪುರಾಣದಷ್ಟೇ ಬೆಲೆ ಇದೆ,\break ಅದಕ್ಕಿಂತ ಹೆಚ್ಚು ಇಲ್ಲ. ಮುಖ್ಯವಾದದ್ದೇನೆಂದರೆ ವೇದಗಳು ಎಂದೂ ಬರೆಯಲ್ಪಡಲಿಲ್ಲ. ಅವು ಯಾವುದೋ ಒಂದು ಸಮಯದಲ್ಲಿ ಅಸ್ತಿತ್ವಕ್ಕೆ ಬಂದುವಲ್ಲ. ಒಬ್ಬ ಕ್ರೈಸ್ತಪಾದ್ರಿ ಒಂದು ಸಲ, ತಮ್ಮ ಶಾಸ್ತ್ರವು ಚಾರಿತ್ರಿಕವಾಗಿದೆ, ಅದಕ್ಕಾಗಿಯೇ ಸತ್ಯ ಎಂದನು. ಅದಕ್ಕೆ ನಾನು ಹೀಗೆಂದೆ: “ನಮ್ಮದು ಚರಿತ್ರೆಯಲ್ಲ, ಅದಕ್ಕೆ ಅದು ಸತ್ಯ. ನಿಮ್ಮದು ಚರಿತ್ರೆಯಂತೆ ಇರುವುದರಿಂದ ಯಾರೋ ಕೆಲವು ದಿನದ ಹಿಂದೆ ಅದನ್ನು ಬರೆದಿರಬೇಕು. ನಿಮ್ಮದು ಮಾನವಕೃತ, ನಮ್ಮದು ಹಾಗಲ್ಲ. ನಮ್ಮ ಶಾಸ್ತ್ರದ ಅನೈತಿಹಾಸ ಕರ್ತವ್ಯವೇ ಅದರ ಸತ್ಯತೆಗೆ ಪ್ರಮಾಣ.” ಈಗಿನ ಕಾಲದಲ್ಲಿ ವೇದಕ್ಕೂ ಇತರ ಶಾಸ್ತ್ರಗಳಿಗೂ ಇರುವ ಸಂಬಂಧ ಇದು.

ಈಗ ಉಪನಿಷತ್ತುಗಳ ಬೋಧನೆಗೆ ಬರೋಣ. ಅಲ್ಲಿ ಹಲವು ಬಗೆಯ ಭಾವನೆಯ ಮಂತ್ರಗಳಿವೆ. ಕೆಲವು ದ್ವೈತದೃಷ್ಟಿಗೆ ಸೇರಿದವು ಮತ್ತೆ ಕೆಲವು ಅದ್ವೈತಕ್ಕೆ ಸೇರಿದವು. ಆದರೆ ಎಲ್ಲಾ ವಿಭಿನ್ನ ಸಂಪ್ರದಾಯದವರೂ ಒಪ್ಪಿಕೊಳ್ಳುವಂತಹ ಕೆಲವು ಸಿದ್ಧಾಂತಗಳಿವೆ. ಅವುಗಳಲ್ಲಿ ಮೊದಲನೆಯದು ಸಂಸಾರ ಅಥವಾ ಜೀವಾತ್ಮನ ಪುನರ್ಜನ್ಮ. ಎರಡನೆಯದಾಗಿ, ಮನಃಶಾಸ್ತ್ರಕ್ಕೆ ಸಂಬಂಧಿಸಿದ ವಿಷಯದಲ್ಲಿ ಅವರಲ್ಲಿ ಏಕಮತವಿದೆ. ಮೊದಲು ದೇಹವಿದೆ, ಅದರ ಹಿಂದೆ ಸೂಕ್ಷ್ಮಶರೀರವೆಂದು ಕರೆಯಲ್ಪಡುವ ಮನಸ್ಸಿದೆ. ಅದರ ಹಿಂದೆ ಜೀವ ಇದೆ. ಪಾಶ್ಚಾತ್ಯ ಮತ್ತು ಪ್ರಾಚ್ಯ ಮನಶ್ಶಾಸ್ತ್ರಕ್ಕೆ ಇದೇ ವ್ಯತ್ಯಾಸ. ಪಾಶ್ಚಾತ್ಯ ಮನಶ್ಶಾಸ್ತ್ರದಲ್ಲಿ ಮನಸ್ಸೇ ಜೀವ. ಇಲ್ಲಿ ಹಾಗಲ್ಲ. ಮನಸ್ಸನ್ನು ಅಂತಃಕರಣ ಎನ್ನುವರು. ಇದು ಜೀವನ ಕೈಯಲ್ಲಿ ಒಂದು ಯಂತ್ರದಂತೆ. ಇದರ ಮೂಲಕ ಜೀವನು ದೇಹದಲ್ಲಿ ಅಥವಾ ಬಾಹ್ಯ ಪ್ರಪಂಚದಲ್ಲಿ ಕೆಲಸ ಮಾಡುವನು. ಇದನ್ನು ಎಲ್ಲರೂ ಒಪ್ಪುವರು. ಜೀವಾತ್ಮನಿಗೆ ಆದಿ ಅಂತ್ಯಗಳಿಲ್ಲ. ಮೋಕ್ಷ ಸಿಕ್ಕುವವರೆಗೆ ಅವನು ಹಲವು ಜನ್ಮಗಳನ್ನು ಪಡೆಯುತ್ತಾ ಹೋಗುವನು. ಎಲ್ಲರೂ ಇದನ್ನು ನಂಬುವರು. ಎಲ್ಲ ಪಂಗಡದವರಲ್ಲೂ ಒಮ್ಮತವಿರುವ ಇನ್ನೊಂದು ಅತಿ ಮುಖ್ಯ ವಿಷಯವಾವುದೆಂದರೆ, ಎಲ್ಲವೂ ಆಗಲೇ ಆತ್ಮನಲ್ಲಿದೆ ಎನ್ನುವುದು. ಇದೇ ಪ್ರಾಚ್ಯರಿಗೂ ಪಾಶ್ಚಾತ್ಯರಿಗೂ ಇರುವ ಬಹಳ ಮುಖ್ಯವಾದ ವ್ಯತ್ಯಾಸ. ಇಲ್ಲಿ \enginline{inspiration} (ಯಾವುದೋ ಹೊರಗಿನಿಂದ ಬಂದು ಪ್ರಚೋದಿಸುವುದು) ಎಂಬುದಿಲ್ಲ. ಎಲ್ಲಾ \enginline{expiration} (ಅಂತರಾಳದಿಂದ ಹೊರಹೊಮ್ಮುವುದು) ಶಕ್ತಿ, ಪಾವಿತ್ರ್ಯ, ಪರಿಪೂರ್ಣತೆ ಆಗಲೇ ಆತ್ಮನಲ್ಲಿವೆ. ಯೋಗಿಗಳು ಪಡೆಯಬೇಕೆನ್ನುವ ಅಣಿಮಾ, ಲಘಿಮಾ ಮುಂತಾದ ಶಕ್ತಿಗಳನ್ನು ಹೊಸದಾಗಿ ಪಡೆದುಕೊಳ್ಳಬೇಕಾಗಿಲ್ಲ. ಅವೆಲ್ಲಾ ಆಗಲೇ ಆತ್ಮನಲ್ಲಿರುವುವು. ಅವನ್ನು ವ್ಯಕ್ತಗೊಳಿಸಬೇಕಾಗಿದೆ ಅಷ್ಟೇ. ನಿಮ್ಮ ಕಾಲಿನ ಕೆಳಗೆ ಹರಿದಾಡುತ್ತಿರುವ ಅತಿ ಕ್ಷುದ್ರ ಕೀಟದಲ್ಲಿಯೂ ಕೂಡ ಅಣಿಮಾದಿ ಅಷ್ಟಸಿದ್ಧಿಗಳಿವೆ. ವ್ಯತ್ಯಾಸವಿರುವುದು ದೇಹದಲ್ಲಿ ಮಾತ್ರ. ಉತ್ತಮ ಶರೀರ ದೊರೆತೊಡನೆಯೇ ಅವು ವ್ಯಕ್ತವಾಗುವುವು. ಅವು ಆಗಲೇ ಅಲ್ಲಿವೆ. \textbf{‘ನಿಮಿತ್ತಮಪ್ರಯೋಜಕಂ ಪ್ರಕೃತೀನಾಂ ವರಣಭೇದಸ್ತು ತತಃ ಕ್ಷೇತ್ರಿಕವತ್​’} - ಒಳ್ಳೆಯ ಮತ್ತು ಕೆಟ್ಟ ಕ್ರಿಯೆಗಳೇ ನಮ್ಮ ಸ್ವಭಾವದ ಬದಲಾವಣೆಗೆ ಮುಖ್ಯ ಕಾರಣಗಳಲ್ಲ. ವಿಕಾಸಕ್ಕೆ ಇರುವ ಆತಂಕವನ್ನು ಅವು ನಿವಾರಿಸುವುವು -\break ರೈತನು ನೀರು ಹರಿಯುವುದಕ್ಕೆ ಇರುವ ಆತಂಕವನ್ನು ತೆಗೆಯುವುದರಿಂದ\break ಕಾಲುವೆಯ ನೀರು ಹರಿಯುವಂತೆ. ಇಲ್ಲಿ ಪತಂಜಲಿಯು ದೂರದ ದೊಡ್ಡ ಕೆರೆಯಿಂದ ನೀರನ್ನು ತರುವ ಒಬ್ಬ ರೈತನ ಉಪಮಾನವನ್ನು ಕೊಡುವನು. ಕೆರೆಯಲ್ಲಿ ಆಗಲೇ ನೀರು ತುಂಬಾ ಇದೆ. ತನ್ನ ಗದ್ದೆಗೂ ಕೆರೆಗೂ ಮಧ್ಯೆ ಮಣ್ಣಿನ ಆತಂಕವನ್ನು ತೆಗೆದೊಡನೆಯೇ ನೀರು ಸ್ವಭಾವತಃ ಹರಿದು ಬರುವುದು. ಹಾಗೆಯೇ ಪಾವಿತ್ರ್ಯ, ಪರಿಪೂರ್ಣತೆ, ಆಗಲೇ ಆತ್ಮನಲ್ಲಿವೆ. ಅದರ ಮೇಲೆ ಇರುವ ಆವರಣವೇ ವ್ಯತ್ಯಾಸಕ್ಕೆ ಕಾರಣ. ಆವರಣ ಹೋದ ಮೇಲೆ ಆತ್ಮ ಪರಿಶುದ್ಧವಾಗುವುದು, ಆತ್ಮನ ಶಕ್ತಿ ವ್ಯಕ್ತವಾಗುವುದು. ಪ್ರಾಚ್ಯ ಮತ್ತು ಪಾಶ್ಚಾತ್ಯರಿಗೆ ಇರುವ ಭಾವನೆಯ ವ್ಯತ್ಯಾಸವೇ ಇದು. ನಾವು ಹುಟ್ಟು ಪಾಪಿಗಳು ಎಂಬ ಪಾಶ್ಚಾತ್ಯ ಭಾವನೆಯನ್ನು ನಂಬದೇ ಇರುವುದರಿಂದ ನಾವೆಲ್ಲಾ ಜನ್ಮತಃ ಪಾಪಿಗಳು ಎಂದು ಅವರು ನಿಂದಿಸುವರು. ಸ್ವಭಾವತಃ ನಾವು ಪಾಪಿಗಳಾದರೆ ನಾವು ಎಂದಿಗೂ ಸಜ್ಜನರಾಗಲಾರೆವು. ಏಕೆಂದರೆ ಸ್ವಭಾವ ಹೇಗೆ ಬದಲಾಗಬಲ್ಲದು? ಇದನ್ನು ಅವರು ಆಲೋಚಿಸುವುದೇ ಇಲ್ಲ. ಅದು ಬದಲಾಗುತ್ತದೆ ಎಂಬ ವಾದವು ಸ್ವವಿರೋಧಿಯಾದುದು. ಅದು ಸ್ವಭಾವ ಆಗುವುದಿಲ್ಲ. ನಾವು ಇದನ್ನು ನೆನಪಿನಲ್ಲಿಡಬೇಕು. ಅದ್ವೈತಿಗಳು, ದ್ವೈತಿಗಳೆಲ್ಲರೂ ಇದನ್ನು ಒಪ್ಪಿಕೊಳ್ಳುವರು.

ಎಲ್ಲ ವಿಭಿನ್ನ ಸಂಪ್ರದಾಯದವರೂ ಒಪ್ಪಿಕೊಳ್ಳುವ ಇನ್ನೊಂದು ಭಾವನೆಯೇ ದೇವರು. ದೇವರನ್ನು ಕುರಿತ ಅವರ ಭಾವನೆಗಳಲ್ಲಿ ವ್ಯತ್ಯಾಸವಿರಬಹುದು. ದ್ವೈತಿಗಳು ಸಗುಣ ದೇವರನ್ನು ಮಾತ್ರ ಒಪ್ಪಿಕೊಳ್ಳುವರು. ಈ ಸಗುಣ ಎಂದರೆ ಏನೆಂಬುದನ್ನು ನೀವು ಸ್ವಲ್ಪ ಚೆನ್ನಾಗಿ ತಿಳಿದುಕೊಳ್ಳಬೇಕು. ಸಗುಣದೇವರು ಅಥವಾ ಈಶ್ವರ ಎಂದರೆ ಅವನಿಗೊಂದು ದೇಹವಿದೆ, ಅವನೆಲ್ಲೋ ಸಿಂಹಾಸನದ ಮೇಲೆ ಕುಳಿತು ಪ್ರಪಂಚವನ್ನು ಆಳುತ್ತಿರುವನು ಎಂದು ಅಲ್ಲ. ಅವನಲ್ಲಿ ಗುಣಗಳಿವೆ ಎಂದು ಅರ್ಥ. ಈಶ್ವರನ ಎಷ್ಟೋ ವರ್ಣನೆಗಳಿವೆ. ಈ ಪ್ರಪಂಚದ ಸೃಷ್ಟಿ - ಸ್ಥಿತಿ-ಲಯಗಳಿಗೆಲ್ಲಾ ಈಶ್ವರನೇ ಕಾರಣನೆಂದು ಎಲ್ಲಾ ಸಂಪ್ರದಾಯದವರೂ ಒಪ್ಪಿಕೊಳ್ಳುವರು. ಅದ್ವೈತಿಗಳು ಇನ್ನೂ ಮುಂದೆ ಹೋಗಿ ಈಶ್ವರನಿಗಿಂತಲೂ ಮೇಲಿನ ಸಗುಣ ನಿರ್ಗುಣ ದೇವರನ್ನು ನಂಬುತ್ತಾರೆ. ವಿಶೇಷಣವೇ ಇಲ್ಲದಾಗ ಯಾವ ಗುಣವಾಚಕವೂ ಆ ಸತ್ಯವನ್ನು ನಿರ್ದೇಶಿಸಲಾರದು. ಅದ್ವೈತಿಗಳು ಅದಕ್ಕೆ ಬೇರೆ ಯಾವ ಗುಣಗಳನ್ನೂ ಆರೋಪಿಸದೆ ಕೇವಲ ಸಚ್ಚಿದಾನಂದ ಎಂದು ಮಾತ್ರ ಹೇಳುವರು. ಶಂಕರಾಚಾರ್ಯರು ಹೇಳಿದುದೇ ಇದನ್ನು. ಉಪನಿಷತ್ತಿನ ಋಷಿಗಳು ಇದಕ್ಕೂ ಮುಂದೆ ಹೋಗಿ ಅದಕ್ಕೆ ಯಾವುದೇ ಗುಣಗಳನ್ನೂ ಆರೋಪಿಸುವುದು ಸಾಧ್ಯವಿಲ್ಲ, ಅದನ್ನು “ನೇತಿ, ನೇತಿ” ಎಂದು ಮಾತ್ರ ಹೇಳಬಹುದು ಎನ್ನುವರು.

ಭರತಖಂಡದ ವಿಭಿನ್ನ ಸಂಪ್ರದಾಯದವರೆಲ್ಲಾ ಇದನ್ನು ಒಪ್ಪಿಕೊಳ್ಳುವರು. ದ್ವೈತಿಗಳಲ್ಲಿ ರಾಮಾನುಜರನ್ನು ಆಧುನಿಕ ದ್ವೈತದ ಪ್ರತಿನಿಧಿ ಎಂದು ನಾನು ಭಾವಿಸುವೆನು. ಭರತಖಂಡದಲ್ಲಿ ಬೇರೆ ಕಡೆಗಳಲ್ಲಿ ಜನ್ಮಧಾರಣ ಮಾಡಿದ ಆಚಾರ್ಯರ ಪರಿಚಯ ವಂಗದೇಶದವರಿಗೆ ಅಷ್ಟು ಇಲ್ಲದೇ ಇರುವುದು ವಿಷಾದಕರ. ಮಹಮ್ಮದೀಯರ ಕಾಲದಲ್ಲಿ, ನಮ್ಮ ಚೈತನ್ಯ ಮಹಾಪ್ರಭು ಹೊರತು, ಮಿಕ್ಕ ಪ್ರಖ್ಯಾತ ಆಚಾರ್ಯರೆಲ್ಲ ದಕ್ಷಿಣದಲ್ಲಿ ಹುಟ್ಟಿದವರು. ವಾಸ್ತವವಾಗಿ ದಕ್ಷಿಣ ಭಾರತದ ಪ್ರತಿಭೆಯೇ ಇಂದು ಇಡೀ ಭಾರತವನ್ನು ನಿಜವಾಗಿಯೂ\break ಆಳುತ್ತಿರುವುದು. ಏಕೆಂದರೆ ಚೈತನ್ಯರು ಕೂಡ ಮಧ್ವಾಚಾರ್ಯರ ಸಂಪ್ರ\-ದಾಯಕ್ಕೆ ಸೇರಿದವರು. ರಾಮಾನುಜರ ದೃಷ್ಟಿಯಲ್ಲಿ, ಈಶ್ವರ, ಜೀವ, ಜಗತ್​ ಇವು ಶಾಶ್ವತವಾದವು. ಜೀವಿಗಳು ಅನಾದಿ, ಎಂದೆಂದಿಗೂ ಇರುವುವು, ತಮ್ಮ ವ್ಯಕ್ತಿತ್ವವನ್ನು ಚಿರಕಾಲ ಇಟ್ಟುಕೊಂಡಿರುವುವು. ಎಂದೆಂದಿಗೂ ನಿಮ್ಮ ಮತ್ತು ನಮ್ಮ ಆತ್ಮ ಚಿರಕಾಲವೂ ಬೇರೆಯಾಗಿರುವುದು. ಅದರಂತೆಯೇ ಜೀವರಷ್ಟೇ ಮತ್ತು ದೇವರಷ್ಟೇ ಸತ್ಯವಾದ ಜಗತ್ತು ಕೂಡ ಬೇರೆಯಾಗಿರುವುದು. ದೇವರು ಜೀವಿಯಲ್ಲಿ ಹಾಸುಹೊಕ್ಕಾಗಿರುವನು, ಅದರ ಸಾರವಾಗಿರುವನು. ಅವನು ಅಂತರ್ಯಾಮಿ. ಈ ದೃಷ್ಟಿಯಿಂದ ರಾಮಾನುಜರು ಕೆಲವು ವೇಳೆ ಜೀವಾತ್ಮ ಪರಮಾತ್ಮರ ಸ್ವಭಾವ ಒಂದೇ ಎಂದು ಹೇಳುವರು. ಪ್ರಳಯದಲ್ಲಿ ಪ್ರಕೃತಿಯೆಲ್ಲಾ ಸಂಕುಚಿತವಾದಾಗ ಜೀವಿಗಳು ಸೂಕ್ಷ್ಮವಾಗಿ ಅಣುವಿನಂತೆ ಸ್ವಲ್ಪ ಕಾಲ ಇರುವುವು. ಕಲ್ಪದ ಆದಿಯಲ್ಲಿ ಇವೆಲ್ಲ ಪುನಃ ತಮ್ಮ ತಮ್ಮ ಕರ್ಮಕ್ಕೆ ತಕ್ಕಂತೆ ವ್ಯಕ್ತವಾಗಿ, ಪರಿಣಾಮ ಹೊಂದುವುವು. ಆತ್ಮನ ಪಾವಿತ್ರ್ಯವನ್ನು ಮತ್ತು ಪರಿಪೂರ್ಣತೆಯನ್ನು ಸಂಕುಚಿತ ಮಾಡುವ ಪ್ರತಿಯೊಂದು ಕ್ರಿಯೆಯೂ ದುಷ್ಕರ್ಮಗಳು; ಈ ಗುಣಗಳು ವ್ಯಕ್ತವಾಗಿ ವಿಕಾಸವಾಗುವಂತೆ ಯಾವುವು ಮಾಡುವುವೋ ಅವೆಲ್ಲ ಪುಣ್ಯ ಕರ್ಮಗಳು ಎಂದು ರಾಮಾನುಜರು ಹೇಳುವರು. ಯಾವುದು ಆತ್ಮವಿಕಾಸಕ್ಕೆ ಸಹಾಯ ಮಾಡುವುದೋ ಅದು ಒಳ್ಳೆಯದು. ಯಾವುದು ಸಂಕೋಚವಾಗುವಂತೆ ಮಾಡುವುದೋ, ಅದು ಕೆಟ್ಟದ್ದು. ಜೀವಿ ಭಗವಂತನ ದಯೆಯಿಂದ ಮೋಕ್ಷ ಪಡೆಯುವವರೆಗೆ ಸಂಕುಚಿತವಾಗುತ್ತಾ ವಿಕಾಸವಾಗುತ್ತಾ ಹೋಗುವುದು. ಯಾರು ಪರಿಶುದ್ಧರೋ ಭಗವಂತನ ದಯೆಗೆ ಹೋರಾಡುವರೋ ಅವರಿಗೆಲ್ಲಾ ಅದು ದೊರಕುವುದು.

\vskip 3pt

\textbf{“ಆಹಾರಶುದ್ಧೌ ಸತ್ತ್ವಶುದ್ಧಿಃ ಸತ್ತ್ವಶುದ್ಧೌ ಧ್ರುವಾ ಸ್ಮೃತಿಃ”} ಎಂಬ ಪ್ರಖ್ಯಾತ ಶ್ರುತಿ ವಾಕ್ಯವಿದೆ. ಆಹಾರ ಶುದ್ಧವಾದರೆ ಸತ್ತ್ವಶುದ್ಧಿಯಾಗುವುದು. ಸತ್ತ್ವಶುದ್ಧಿಯಾದರೆ ಸೃತಿ-ಭಗವಂತನ ಸ್ಮೃತಿ, ಅಥವಾ ಅದ್ವೈತಿಯಾಗಿದ್ದರೆ ನಮ್ಮ ಪೂರ್ಣತೆಯ ಸ್ಮೃತಿ, ದೃಢವಾಗುವುದು. ಅದು ಸತ್ಯವಾಗಿ ಸ್ಥಿರವಾಗಿ ನಿತ್ಯವಾಗುವುದು. ಈ ವಿಷಯದಲ್ಲಿ ದೊಡ್ಡದೊಂದು ಚರ್ಚೆ ಇದೆ. ಮೊದಲನೆಯದಾಗಿ ಸತ್ತ್ವ ಎಂದರೆ ಏನು? ಸಾಂಖ್ಯರು ಮತ್ತು ಎಲ್ಲರೂ ಒಪ್ಪಿಕೊಳ್ಳುವಂತೆ ನಮ್ಮ ದೇಹ ಮೂರು ವಸ್ತುಗಳಿಂದ ಆಗಿದೆ. ತಮಸ್ಸು, ರಜಸ್ಸು, ಸತ್ತ್ವ ಇವು ಗುಣಗಳು ಎಂದು ಸಾಧಾರಣವಾಗಿ ತಿಳಿಯುವರು. ಅವು ಗುಣಗಳಲ್ಲ, ಪ್ರಪಂಚವು ಯಾವುದರಿಂದ ಆಗಿದೆಯೋ ಆ ಮೂಲವಸ್ತುಗಳು. ಆಹಾರ ಶುದ್ಧಿಯಾದರೆ ಸತ್ವಾಂಶ ಶುದ್ಧಿಯಾಗುವುದು. ವೇದಾಂತದ ಒಂದು ಗುರಿಯೇ ಈ ಸತ್ತ್ವವನ್ನು ಪಡೆಯುವುದು. ಆತ್ಮವು ಆಗಲೇ ಪರಿಶುದ್ಧವಾಗಿದೆ, ಪರಿಪೂರ್ಣವಾಗಿದೆ. ವೇದಾಂತದ ದೃಷ್ಟಿಯಿಂದ ಅದು ರಜೋ ಗುಣಗಳಿಂದ ಮುಚ್ಚಲ್ಪಟ್ಟಿವೆ. ಸತ್ತ್ವ ಎಲ್ಲಕ್ಕಿಂತಲೂ ಪ್ರಕಾಶಮಾನವಾದುದು. ಆತ್ಮಜ್ಯೋತಿ, ಬೆಳಕು ಹೇಗೆ ಗಾಜಿನ ಮೂಲಕ ಪ್ರಕಾಶಿಸುವುದೋ ಹಾಗೆ ಸತ್ತ್ವದ ಮೂಲಕ ಪ್ರಕಾಶಿಸುವುದು. ತಮೋ ರಜೋ ಗುಣಗಳು ಹೋಗಿ ಸತ್ತ್ವಗುಣ ಉಳಿದರೆ ಆತ್ಮನ ಪೂರ್ಣತೆ ಪರಿಶುದ್ಧತೆ ವ್ಯಕ್ತವಾಗಿ, ಆತ್ಮನು ಹೆಚ್ಚು ಪ್ರಕಾಶಮಾನವಾಗುವನು.

ಆದಕಾರಣವೇ ಈ ಸತ್ತ್ವವನ್ನು ಪಡೆಯುವುದು ಮುಖ್ಯ. “ಆಹಾರ ಶುದ್ಧವಾದ ಮೇಲೆ” ಎಂದು ಶ್ರುತಿ ಹೇಳುತ್ತದೆ. ‘ಆಹಾರ’ವೆಂದರೆ ರಾಮಾನುಜರು\break ನಾವು ಭಕ್ಷಿಸುವ ವಸ್ತು ಎನ್ನುವರು. ತಮ್ಮ ಸಿದ್ಧಾಂತದಲ್ಲಿ ಇದಕ್ಕೆ ವಿಶೇಷ ಪ್ರಾಧಾನ್ಯವನ್ನು ಕೊಟ್ಟಿರುವರು. ಇದು ನಮ್ಮ ಭರತಖಂಡದಲ್ಲಿ ವಿಭಿನ್ನ ಪಂಥಗಳ ಮೇಲೂ ತನ್ನ ಪ್ರಭಾವವನ್ನು ಬೀರಿದೆ. ಆದಕಾರಣ ಆಹಾರ ಎಂದರೆ ಏನು ಎಂಬುದನ್ನು ತಿಳಿದುಕೊಳ್ಳಬೇಕು. ರಾಮಾನುಜರ ದೃಷ್ಟಿಯಲ್ಲಿ ಇದೊಂದು ಅತಿ ಮುಖ್ಯ ವಿಷಯ. ನಮ್ಮ ಜೀವನದಲ್ಲಿ ಆಹಾರವು ಹೇಗೆ ಅಶುದ್ಧವಾಗುವುದು ಎಂಬುದನ್ನು ರಾಮಾನುಜರು ಹೇಳುತ್ತಾರೆ. ಮೂರು ದೋಷಗಳಿಂದ ಆಹಾರ ಅಶುದ್ಧವಾಗುವುದು ಎನ್ನುವರು. ಮೊದಲನೆಯದು ಜಾತಿ ದೋಷ. ಇದು ಆಹಾರದ ಸ್ವಭಾವದಲ್ಲೇ ಇರುವ ದೋಷ, ಉದಾಹರಣೆಗೆ ಈರುಳ್ಳಿ, ಬೆಳ್ಳುಳ್ಳಿ ಮುಂತಾದ ವಾಸನೆಯ ಪದಾರ್ಥಗಳು. ಎರಡನೆಯದು ಆಶ್ರಯ ದೋಷ. ಯಾರಿಂದ ಆಹಾರ ಬರುವುದೋ ಅವರ ದೋಷವೂ ಅದನ್ನು ಸ್ವೀಕರಿಸು\-ವವನ ಮೇಲೆ ಪರಿಣಾಮವನ್ನುಂಟುಮಾಡುವುದು. ಅನೇಕ ಮಹಾತ್ಮರು ಇದನ್ನು ತಮ್ಮ ಇಡಿಯ ಜೀವನದಲ್ಲಿ ಅನುಸರಿಸಿರುವುದನ್ನು ನೋಡಿರುವೆನು. ಯಾರು ಆಹಾರ ತಂದರು, ಅದನ್ನು ಮುಟ್ಟಿದರು, ಎಂಬುದು ಕೂಡ ಅವರಿಗೆ ತಿಳಿಯುತ್ತಿತ್ತು. ನಾನೇ ನನ್ನ ಜೀವನದಲ್ಲಿ ಇದನ್ನು ಒಂದು ಸಲವಲ್ಲ, ನೂರಾರು ಸಲ ನೋಡಿದ್ದೇನೆ. ಅನಂತರ ನಿಮಿತ್ತದೋಷ. ಆಹಾರಕ್ಕೆ ಕೊಳೆ ಮುಂತಾದುವು ಸೇರುವುವು. ಇದಕ್ಕೆ ನಾವು ಈಗ ಸ್ವಲ್ಪ ಹೆಚ್ಚು ಗಮನ ಕೊಡಬೇಕು. ಕೊಳೆ ಕಷ್ಮಲ ಕೂದಲು ಮುಂತಾದವುಗಳಿಂದ ಕೂಡಿದ ಆಹಾರವನ್ನು ತೆಗೆದುಕೊಳ್ಳುವುದು ಭಾರತ ದೇಶದಲ್ಲಿ ವಾಡಿಕೆಯಾಗಿದೆ. ಈ ಮೂರು ದೋಷಗಳಿಂದ ಆಹಾರವನ್ನು ಪಾರುಮಾಡಿದರೆ ಸತ್ತ್ವ ಶುದ್ಧಿಯಾಗುವುದು. ಅನಂತರ ಅಧ್ಯಾತ್ಮ\break ಸಾಧನೆ ಬಹಳ ಸುಲಭವಾಗುತ್ತದೆ. ಶುದ್ಧ ಆಹಾರದಿಂದಲೇ ಧಾರ್ಮಿಕ\-ರಾಗುವ\break ಹಾಗಿದ್ದರೆ ಎಲ್ಲರೂ ಧಾರ್ಮಿಕರಾಗುವರು. ಈ ಮೂರು ದೋಷಗಳಿಂದ\break ಪಾರಾಗುವುದು ಪ್ರಪಂಚದಲ್ಲಿ ಕಷ್ಟವಲ್ಲ. ಶಂಕರಾಚಾರ್ಯರು ಆಹಾರ\break ವೆಂದರೆ ಮನಸ್ಸು ಸ್ವೀಕರಿಸುವ ಆಲೋಚನೆ, ಅದು ಶುದ್ಧವಾದರೆ ಸತ್ತ್ವ\break ಶುದ್ಧವಾಗುವುದು, ಅದಕ್ಕೆ ಮುಂಚೆ ಅಲ್ಲ ಎನ್ನುತ್ತಾರೆ. ನೀವು ಏನನ್ನು ಬೇಕಾದರೂ ತೆಗೆದುಕೊಳ್ಳಬಹುದು. ಆಹಾರ ಒಂದೇ ಸತ್ತ್ವವನ್ನು ಶುದ್ಧಿ ಮಾಡುವ ಹಾಗಿದ್ದರೆ ಒಂದು ಕಪಿಗೆ ಬರಿಯ ಅನ್ನ ಹಾಲು ಕೊಡಿ-ಅದು ದೊಡ್ಡ ಯೋಗಿಯಾಗುವುದೇನೋ ನೋಡೋಣ. ಹಾಗಿದ್ದಿದ್ದರೆ ದನ ಜಿಂಕೆಗಳು ದೊಡ್ಡ ಯೋಗಿಗಳಾಗಬೇಕಾಗಿತ್ತು- ಹೆಚ್ಚು ಸ್ನಾನದಿಂದ ಸ್ವರ್ಗ ಸಿಕ್ಕುವ ಹಾಗಿದ್ದರೆ ಮೀನು ಮೊದಲು ಸ್ವರ್ಗಕ್ಕೆ ಹೋಗುವುದು ಎನ್ನುವಂತೆ, ಶಾಕಾಹಾರದಿಂದ ಸ್ವರ್ಗ ಸಿಕ್ಕುವ ಹಾಗೆ ಇದ್ದರೆ ಜಿಂಕೆ ದನಗಳು ಮೊದಲು ಅಲ್ಲಿಗೆ ಹೋಗುತ್ತವೆ ಎಂದು ಹೇಳಿದಂತೆ.

ಇದಕ್ಕೆ ಪರಿಹಾರವೇನು? ಎರಡೂ ಅವಶ್ಯಕ. ಶಂಕರಾಚಾರ್ಯರ ಆಹಾರ ವಿವರಣೆ ಮುಖ್ಯ. ಶುದ್ಧ ಆಹಾರ ಮನಶ್ಶುದ್ಧಿಯನ್ನು ಪಡೆಯಲು ಸಹಾಯ ಮಾಡುತ್ತದೆ. ಎರಡಕ್ಕೂ ನಿಕಟ ಸಂಬಂಧವಿದೆ. ಎರಡೂ ಇರಬೇಕು. ಭರತಖಂಡದ ಈಗಿನ ನ್ಯೂನತೆ ಏನೆಂದರೆ ಶಂಕರಾಚಾರ್ಯರ ವಿವರಣೆಯನ್ನು ಮರೆತು ಕೇವಲ ಆಹಾರ ಶುದ್ಧಿಗೆ ಪ್ರಾಧಾನ್ಯ ನೀಡಿರುವುದು. ಅದಕ್ಕಾಗಿಯೇ ಧರ್ಮವು ಅಡಿಗೆ ಮನೆಗೆ ನುಗ್ಗಿದೆ ಎಂದು ನಾನು ಹೇಳಿದರೆ ಜನರಿಗೆ ನನ್ನ ಮೇಲೆ ಕೋಪ ಬರುತ್ತದೆ. ನೀವು ನನ್ನೊಂದಿಗೆ ಮದ್ರಾಸಿಗೆ ಬಂದಿದ್ದರೆ ನಾನು ಹೇಳುವುದನ್ನು ಒಪ್ಪಿಕೊಳ್ಳುತ್ತಿದ್ದಿರಿ. ಬಂಗಾಳಿಗಳು ಅವರಿಗಿಂತ ಮೇಲು. ಮದ್ರಾಸಿನಲ್ಲಿ ಆಹಾರವನ್ನು ಯಾರಾದರೂ ನೋಡಿದರೆ ಅದನ್ನು ಬಿಸಾಡುವರು. ಇಷ್ಟು ಮಾಡಿದರೂ ಜನರೇನೂ ಉತ್ತಮರಾಗಿ ಕಾಣುವುದಿಲ್ಲ. ಯಾವುದಾದರೂ ವಿಶೇಷ ಆಹಾರವನ್ನು ಸೇವಿಸಿದರೆ, ಜನರ ದೃಷ್ಟಿಯಿಂದ ಆಹಾರವನ್ನು ಸಂರಕ್ಷಿಸಿದರೆ ಮುಕ್ತಿ ಸಿಕ್ಕುವ ಹಾಗೆ ಇದ್ದರೆ, ಅವರೆಲ್ಲ ಇಷ್ಟು ಹೊತ್ತಿಗೆ ಸಿದ್ಧಪುರಷರಾಗಿರಬೇಕಾಗಿತ್ತು. ಆದರೆ ಅವರು ಆಗಿಲ್ಲ.

ಇವೆರಡನ್ನೂ ಒಟ್ಟುಗೂಡಿಸಿ ಒಂದು ಮಾಡಬೇಕು. ಆದರೆ ಕುದುರೆಯ ಮುಂದೆ ಗಾಡಿ ಕಟ್ಟಬೇಡಿ. ಆಹಾರ, ವರ್ಣಾಶ್ರಮ ಮುಂತಾದ ವಿಷಯಗಳ ಮೇಲೆ ಜನರು ಬಹಳ ಮಾತನಾಡುತ್ತಿರುವರು. ವಂಗೀಯರ ಗದ್ದಲ ಇದರಲ್ಲಿ ಬಹಳ ಜೋರಾಗಿದೆ.

ವರ್ಣಾಶ್ರಮದ ವಿಷಯ ನಿಮಗೆ ಏನು ಗೊತ್ತಿದೆ ಎಂದು ನಿಮ್ಮನ್ನು ಕೇಳುತ್ತೇನೆ. ದೇಶದಲ್ಲಿ ಈಗ ನಾಲ್ಕು ಜಾತಿಗಳಿವೆಯೆ? ಉತ್ತರ ಹೇಳಿ. ನನಗೆ ವರ್ಣಗಳು ಕಾಣುತ್ತಿಲ್ಲ. ‘ತಲೆಯೇ ಇಲ್ಲ ತಲೆನೋವು’ ಎಂಬ ಬಂಗಾಳಿ ಗಾದೆಯಂತಿದೆ ನೀವು ವರ್ಣಾಶ್ರಮವನ್ನು ವಿಚಾರಿಸುತ್ತಿರುವುದು. ನಾಲ್ಕು ವರ್ಣಗಳಲ್ಲಿ ಕೇವಲ ಬ್ರಾಹ್ಮಣರು ಮತ್ತು ಶೂದ್ರರನ್ನು ಮಾತ್ರ ನೋಡುತ್ತೇವೆ. ಕ್ಷತ್ರಿಯ ವೈಶ್ಯರು ಎಲ್ಲಿದ್ದಾರೆ? ಬ್ರಾಹ್ಮಣರು ಅವರಿಗೆ ಯಜ್ಞೋಪವೀತವನ್ನು ಕೊಟ್ಟು ಹಿಂದುಗಳಂತೆ ವೇದಗಳನ್ನು ಅಧ್ಯಯನ ಮಾಡುವಂತೆ ಏತಕ್ಕೆ ಹೇಳಕೂಡದು? ವೈಶ್ಯರು ಮತ್ತು ಕ್ಷತ್ರಿಯರಿಲ್ಲದೆ ಕೇವಲ ಬ್ರಾಹ್ಮಣ ಮತ್ತು ಶೂದ್ರರು ಮಾತ್ರ ಇದ್ದರೆ, ಅಂತಹ ಸ್ಥಳದಲ್ಲಿ ಬ್ರಾಹ್ಮಣರು ಇರಕೂಡದು. ಗಂಟು ಮೂಟೆ ಕಟ್ಟಿ ಹೊರಡಿ. ಯಾರು ಮ್ಲೇಚ್ಛರ ಕೆಳಗೆ ಇರುವರೋ, ಅವರ ಆಹಾರವನ್ನು ಸ್ವೀಕರಿಸುವರೋ, ಅಂತಹವರಿಗೆ ಶಾಸ್ತ್ರಗಳು ಯಾವ ಪ್ರಾಯಶ್ಚಿತ್ತವನ್ನು ಹೇಳುವುದು ಗೊತ್ತೆ? ತಾವೇ ಚಿತೆಗೆ ಬೆಂಕಿ ಹಚ್ಚಿಕೊಂಡು ಬೇಯುವುದು. ನೀವು ಸಾವಿರ ವರ್ಷಗಳಿಂದ ಮ್ಲೇಚ್ಛರ ಕೆಳಗೆ ಇದ್ದೀರಿ. ಗುರುಗಳಂತೆ ನಟಿಸಿ ಕಪಟಿಗಳಂತೆ ಆಚರಿಸುವಿರೇನು? ನಿಮಗೆ ಶಾಸ್ತ್ರಗಳಲ್ಲಿ ನಂಬಿಕೆ ಇದ್ದರೆ ನೀವು ಅಲೆಗ್ಸಾಂಡರನೊಂದಿಗೆ ಹೋದ ಬ್ರಾಹ್ಮಣನಂತೆ ಮಾಡಿಕೊಳ್ಳಿ. ಆತ ಮ್ಲೇಚ್ಛರ ಆಹಾರ ತಿಂದಿರಬಹುದೆಂದು ಆತ್ಮಹತ್ಯೆಯನ್ನು ಮಾಡಿಕೊಂಡನು. ನೀವೂ ಹಾಗೆ ಮಾಡಿಕೊಳ್ಳಿ. ಆಗ ಇಡೀ ದೇಶ ನಿಮ್ಮ ಪದತಳಕ್ಕೆ ಬೀಳುವುದು. ನಿಮಗೆ ಶಾಸ್ತ್ರದಲ್ಲಿ ನಂಬಿಕೆ ಇಲ್ಲ. ಆದರೂ ಇತರರನ್ನು ನಂಬಿಸುವುದಕ್ಕೆ ಪ್ರಯತ್ನಪಡುತ್ತಿರುವಿರಿ. ಈಗ ಅದನ್ನು ಮಾಡುವುದಕ್ಕೆ ಆಗುವು\-ದಿಲ್ಲವೆಂದು ನಿಮ್ಮ ದುರ್ಬಲತೆಯನ್ನು ಒಪ್ಪಿಕೊಂಡರೆ, ಇತರರ ದುರ್ಬಲತೆಯನ್ನೂ ಕ್ಷಮಿಸಿ. ಇತರ ವರ್ಣದವರನ್ನೂ ಸ್ವೀಕರಿಸಿ ಮೇಲೆತ್ತಿ. ಅವರೂ ವೇದವನ್ನು ಓದಲಿ. ಅವರೂ ಜಗತ್ತಿನ ಇತರ ಆರ್ಯರಂತೆ ಉತ್ತಮರಾಗಲಿ. ವಂಗ ಬ್ರಾಹ್ಮಣರೇ, ನೀವೂ ಕೂಡ ಪ್ರಪಂಚದ ಇತರ ಆರ್ಯರಂತೆ ಆಗಿ.

ನಿಮ್ಮ ದೇಶವನ್ನೇ ನಾಶಮಾಡುತ್ತಿರುವ ಅನಿಷ್ಟ ವಾಮಾಚಾರವನ್ನು ತ್ಯಜಿಸಿ. ನೀವು ಭಾರತದೇಶದ ಬೇರೆ ಭಾಗವನ್ನು ನೋಡಿಲ್ಲ. ನಮ್ಮ ಸಮಾಜದಲ್ಲಿ ಎಷ್ಟೊಂದು ವಾಮಾಚಾರವಿದೆ ಎಂಬುದನ್ನು ನೋಡಿದಾಗ, ನಮ್ಮ ಸಂಸ್ಕೃತಿಯನ್ನು ಕುರಿತು ಎಷ್ಟೇ ಜಂಭಕೊಚ್ಚಿಕೊಂಡರೂ ನಾಚಿಕೊಳ್ಳಬೇಕಾಗುವುದು. ವಾಮಾಚಾರ ಪಂಗಡ ವಂಗ ಸಮಾಜವನ್ನೆಲ್ಲಾ ಮುತ್ತುತ್ತಿದೆ. ಹಗಲು ಆಚಾರವನ್ನು ಘಂಟಾ ಘೋಷವಾಗಿ ಸಾರುವರು. ರಾತ್ರಿ ಹೊತ್ತು ಅನಿಷ್ಟ ವ್ಯಭಿಚಾರವನ್ನು ನಡೆಸುವರು. ಇವರಿಗೆ ಹಲವು ಶಾಸ್ತ್ರಗಳು ಬೇರೆ ಸಹಾಯಕ್ಕೆ ಬರುತ್ತವೆ!\break ಹೀಗೆ ಮಾಡಬೇಕೆಂದು ಶಾಸ್ತ್ರಗಳೇ ಅವರಿಗೆ ವಿಧಿಸುತ್ತವೆ! ವಂಗೀಯರಿಗೆ ಇದು ಗೊತ್ತು. ವಂಗ ಶಾಸ್ತ್ರವೇ ವಾಮಾಚಾರ ತಂತ್ರಗಳು. ಗಾಡಿಗಟ್ಟಲೆ ಇಂತಹ ಪುಸ್ತಕಗಳು ಪ್ರತಿ ವರ್ಷವೂ ತಯಾರಾಗುತ್ತಿವೆ. ನಿಮ್ಮ ಮಕ್ಕಳಿಗೆ ಶ್ರುತಿಯನ್ನು ಕೊಡುವುದನ್ನು ಬಿಟ್ಟು ಅವರ ಮನಸ್ಸನ್ನು ಇಂತಹ ವಿಷದಿಂದ ತುಂಬುವಿರಿ. ಕಲ್ಕತ್ತಾ ನಗರದ ಪಿತರೇ, ವಾಮಾಚಾರ-ತಂತ್ರಗಳನ್ನು ಮತ್ತು ಅವುಗಳ ಭಾಷಾಂತರವನ್ನು ನಿಮ್ಮ ಗಂಡು, ಹೆಣ್ಣು ಮಕ್ಕಳಿಗೆ ಕೊಡುವುದಕ್ಕೆ ನಿಮಗೆ ನಾಚಿಕೆಯಾಗುವುದಿಲ್ಲವೇ? ಅವರ ಮನಸ್ಸು ಹಾಳಾಗಿ ಇವೇ ಹಿಂದೂಶಾಸ್ತ್ರ ಎಂದು ಅವರು ಭಾವಿಸುವರು. ಅವರು ವೇದ ಗೀತಾ ಉಪನಿಷತ್ತು ಇಂತಹ ನಿಜವಾದ ಶಾಸ್ತ್ರಗಳನ್ನು ಓದಲಿ.

ಭರತಖಂಡದ ದ್ವೈತಿಗಳ ದೃಷ್ಟಿಯಲ್ಲಿ, ಜೀವಾತ್ಮಗಳು ಕೊನೆಯವರೆಗೂ ಬೇರೆಯಾಗಿಯೇ ಇರುವುವು. ದೇವರು ಜಗತ್ತನ್ನು ಆಗಲೇ ಇರುವ ವಸ್ತು\-ವಿನಿಂದ ಸೃಷ್ಟಿಸಿರುವನು. ಅವನು ಕೇವಲ ನಿಮಿತ್ತ ಕಾರಣ. ಅದ್ವೈತಿಗಳ ದೃಷ್ಟಿಯಲ್ಲಿ ದೇವರು ಉಪಾದಾನ ಕಾರಣ ಮತ್ತು ನಿಮಿತ್ತಕಾರಣ ಎರಡೂ ಹೌದು. ಅವನು ಕೇವಲ ಸೃಷ್ಟಿಕರ್ತ ಮಾತ್ರ ಅಲ್ಲ, ಅವನು ಸೃಷ್ಟಿಯನ್ನು ತನ್ನಿಂದಲೇ\break ರಚಿಸುವನು. ಇದೇ ಅದ್ವೈತಿಯ ದೃಷ್ಟಿ. ಅಪರಿಪಕ್ವವಾದ ದ್ವೈತಿಗಳ ಗುಂಪಿರು\-ವುದು. ಅವರು ದೇವರು ತನ್ನಿಂದ ಸೃಷ್ಟಿಯನ್ನು ರಚಿಸಿ ಆದಮೇಲೆ ಸೃಷ್ಟಿಗಿಂತ ಸಂಪೂರ್ಣ ಬೇರೆಯಾಗಿರುವನು ಎಂದೂ; ಪ್ರಪಂಚದಲ್ಲಿ ಎಲ್ಲವೂ ಅವನ ಅಧೀನ ಎಂದೂ ಹೇಳುವರು. ಮತ್ತೆ ಕೆಲವರು, ದೇವರು ತನ್ನಿಂದಲೇ ಈ ಸೃಷ್ಟಿಯನ್ನು ರಚಿಸಿದನು, ಜೀವಿಗಳು ಕೊನೆಗೆ ತಮ್ಮ ವ್ಯಕ್ತಿತ್ವವನ್ನು ತೊರೆದು ನಿರ್ವಾಣವನ್ನು ಪಡೆಯುವರು, ಆಗ ಅನಂತವಾಗು ವರು ಎನ್ನುವರು. ಈ ಪಂಥದವರೂ ಈಗ ಮಾಯವಾಗಿರುವರು. ಆಧುನಿಕ ಭರತ ಖಂಡದಲ್ಲಿರುವ ಅದ್ವೈತಿಗಳೆಲ್ಲಾ ಶಂಕರಾಚಾರ್ಯರ ಅನುಯಾಯಿಗಳು. ಶಂಕರಾಚಾರ್ಯರ ಮತದ ಪ್ರಕಾರ ದೇವರು ಸೃಷ್ಟಿಯ ಉಪಾದಾನಕಾರಣ ಮತ್ತು ನಿಮಿತ್ತ ಕಾರಣ. ಆದರೆ ಇದು ಮಾಯೆಯ ಮೂಲಕ; ವಾಸ್ತವಿಕವಾಗಿ ದೇವರು ಪ್ರಪಂಚವಾಗಿಲ್ಲ. ಪ್ರಪಂಚವೆಂಬುದೇ ಇಲ್ಲ. ದೇವರೊಬ್ಬನೇ ಇರುವುದು. ಈ ಮಾಯಾ\-ಭಾವನೆ ನಾವು ಅದ್ವೈತದಲ್ಲಿ ತಿಳಿದುಕೊಳ್ಳಬೇಕಾದ ಅತಿ ಮುಖ್ಯ ವಿಷಯ. ನಮ್ಮ ತತ್ತ್ವದಲ್ಲೆಲ್ಲ ಅತಿ ಕಷ್ಟವಾದ ವಿಷಯ ಇದು. ಇದನ್ನು ನಿಮಗೆ ವಿವರಿಸುವುದಕ್ಕೆ\break ನನಗೆ ಕಾಲಾವಕಾಶವಿಲ್ಲ. ಪಾಶ್ಚಾತ್ಯ ತತ್ತ್ವಶಾಸ್ತ್ರದ ಪರಿಚಯವಿದ್ದವರಿಗೆ\break ಕ್ಯಾಂಟನಲ್ಲಿ ಇಂತಹ ಕೆಲವು ಭಾವನೆಗಳಿರುವುದು ಗೊತ್ತಾಗುವುದು. ಆದರೆ ಯಾರು ಕ್ಯಾಂಟನ ವಿಚಾರವಾಗಿ ಪ್ರೊಫೆಸರ್​ ಮ್ಯಾಕ್ಸ್‌ಮುಲ್ಲರ್​ ಬರೆದಿರುವು\-ದನ್ನು ಓದಿರುವರೋ ಅವರು ಬಹಳ ಜೋಪಾನವಾಗಿರಬೇಕು. ಅದರಲ್ಲಿ\break ಒಂದು ತಪ್ಪು ಭಾವನೆ ಇದೆ. ಕಾಲ-ದೇಶ-ನಿಮಿತ್ತಗಳು ಮಾಯೆಯಿಂದ\break ಭಿನ್ನವಲ್ಲ ಎಂದು ಶಂಕರಾಚಾರ್ಯರು ಮೊದಲು ಕಂಡುಹಿಡಿದರು.\break ಶಂಕರಾಚಾರ್ಯರ ಭಾಷ್ಯಗಳಲ್ಲಿ ಇದಕ್ಕೆ ಸಂಬಂಧಿಸಿದ ಕೆಲವು ವಾಕ್ಯಗಳನ್ನು ಹುಡುಕಿ ನನ್ನ ಸ್ನೇಹಿತರಾದ ಪ್ರೊಫೆಸರ್​ ಮ್ಯಾಕ್ಸ್ ಮುಲ್ಲರಿಗೆ ಕಳುಹಿಸಿದೆ. ಆ ಭಾವನೆ ಕೂಡ ಭರತಖಂಡದಲ್ಲಿತ್ತು. ಅದ್ವೈತ ದರ್ಶನದ ಈ ಮಾಯಾ ಸಿದ್ಧಾಂತ ಅತಿ ವಿಚಿತ್ರವಾದುದು. ಬ್ರಹ್ಮ ಒಂದೇ ಇರುವುದು. ಈ ವೈವಿಧ್ಯ ಮಾಯೆಯಿಂದಾಗಿದೆ. ಬ್ರಹ್ಮವು ಏಕ, ಅದೇ ಪರಮ ಗುರಿ. ಇದೇ ಭಾರತೀಯರಿಗೂ ಪಾಶ್ಚಾತ್ಯರಿಗೂ ಇರುವ ಅಂತ್ಯವಿಲ್ಲದ ವ್ಯತ್ಯಾಸ. ಸಾವಿರಾರು ವರುಷಗಳ ಕಾಲ ಭರತಖಂಡವು ಜಗತ್ತಿಗೆ ಒಂದು ಸವಾಲನ್ನು ಹಾಕಿದೆ. ಬೇರೆ ಬೇರೆ ದೇಶಗಳು ಈ ಸವಾಲನ್ನು ಸ್ವೀಕರಿಸಿದವು. ಆದರೆ ಆ ದೇಶಗಳವರು ಮಾಯವಾದರು. ನೀವು ಮಾತ್ರ ಇರುವಿರಿ. ಆ ಸವಾಲು ಯಾವುದೆಂದರೆ, ಈ ಸೃಷ್ಟಿಯೊಂದು ಭ್ರಾಂತಿ, ಮಾಯೆ ಎಂಬುದು. ನೀವು ನೆಲದ ಮೇಲೆ ಕುಳಿತು ಬೆರಳಿನಿಂದ ಊಟ ಮಾಡುವಿರೋ, ನೀವು ಅರಮನೆಯಲ್ಲಿ ವಾಸಿಸುವಿರೋ ಅಥವಾ ಅತಿ ದರಿದ್ರ ಭಿಕಾರಿಗಳಾಗಿರುವಿರೋ, ಎಲ್ಲರೂ ಮೃತ್ಯುವಶರಾಗಲೇಬೇಕು. ಎಲ್ಲಾ\break ಮಾಯೆ. ಇದೇ ಭರತಖಂಡದ ಸನಾತನ ಪಲ್ಲವಿ. ಪುನಃ ಪುನಃ ಹೊಸ\break ರಾಷ್ಟ್ರಗಳೆದ್ದು ಇದನ್ನು ವಿರೋಧಿಸುವುದಕ್ಕೆ ಪ್ರಯತ್ನಿಸುತ್ತಿವೆ. ಅಲ್ಲಿನ ಜನರು ಭೋಗವನ್ನೇ ಪರಮ ಧ್ಯೇಯವನ್ನಾಗಿ ಮಾಡಿಕೊಂಡು ಅಧಿಕಾರದಿಂದ ಪ್ರಖ್ಯಾತರಾಗುವರು, ಅಧಿಕಾರ ಮತ್ತು ಭೋಗದ ಪರಮಾವಧಿಯನ್ನು ಮುಟ್ಟುವರು, ಸಾಕಾದಷ್ಟು ಅಧಿಕಾರವನ್ನು ಭೋಗಕ್ಕೆ ಉಪಯೋಗಿಸಿ ಮರುಕ್ಷಣ ನಾಶವಾಗುವರು. ಎಲ್ಲಾ ಮಾಯೆಯೆಂದು ನಾವು ನೋಡುವುದರಿಂದಲೇ ನಾವು ಇನ್ನೂ ಇರುವುದು. ಮಾಯಾಸುತರು ಎಂದೆಂದಿಗೂ ಬದುಕಿರುವರು, ಭೋಗದ ಸುತರು ಸಾಯುವರು.

\newpage

ಇಲ್ಲಿ ಮತ್ತೊಂದು ವ್ಯತ್ಯಾಸವಿದೆ. ಜರ್ಮನ್​ ತತ್ತ್ವ ಶಾಸ್ತ್ರದಲ್ಲಿ ಹೆಗಲ್​ ಮತ್ತು ಶೋಫನ್ಹೆರ್​ ಇವರು ಪ್ರಯತ್ನಿಸಿದಂತೆಯೇ ಅದೇ ಭಾವನೆಯನ್ನು ಪ್ರಾಚೀನ ಭರತಖಂಡದಲ್ಲಿ ಹಿಂದೆ ತಂದಿದ್ದರು. ಅದೃಷ್ಟವಶಾತ್​ ಹೆಗಲ್​ನ ಭಾವನೆಯನ್ನು ಅಂಕುರದಲ್ಲೇ ನಾಶಮಾಡಿದರು. ಅದು ಬೆಳೆದು ಮರವಾಗಿ ನಾಶಕಾರಕ ರೆಂಬೆಗಳನ್ನು ಭರತಖಂಡದ ಮೇಲೆ ಹರಡಲು ಅವಕಾಶ ಕೊಡಲಿಲ್ಲ. ಹೆಗಲ್​ನ ಭಾವನೆ, ಏಕವಾದ ಪರವಸ್ತುವು ಅವ್ಯವಸ್ಥಿತವಾಗಿದೆ, ವ್ಯಕ್ತಿ ಅದಕ್ಕಿಂತ ಉತ್ತಮ, ಅವ್ಯಕ್ತಕ್ಕಿಂತ ವ್ಯಕ್ತಪ್ರಪಂಚ ಮೇಲು, ಮುಕ್ತಿಗಿಂತ ಸಂಸಾರ ಮೇಲು ಎಂಬುದು. ನೀವು ಸಂಸಾರದಲ್ಲಿ ಅದರ ಚಟುವಟಿಕೆಗಳಲ್ಲಿ ನಿರತರಾದಷ್ಟೂ ನೀವು ಮೇಲೆ ಇರುವಿರಿ. ನಾವು ಮನೆಗಳನ್ನು ಹೇಗೆ ಕಟ್ಟುತ್ತೇವೆ, ಬೀದಿಗಳನ್ನು ಹೇಗೆ ಗುಡಿಸುತ್ತೇವೆ, ಇಂದ್ರಿಯಗಳ ಮೂಲಕ ಹೇಗೆ ಸುಖಪಡುತ್ತೇವೆ ಎಂಬುದು ನಿಮಗೆ ಕಾಣುವುದಿಲ್ಲವೆ, ಎಂದು ಆ ಪಂಥದವರು ಕೇಳುತ್ತಾರೆ. ದುಃಖ, ಭಯ, ಮನಸ್ತಾಪಗಳನ್ನು ಅವರು ಭೋಗದಿಂದ ಮರೆಮಾಚುತ್ತಿರುವರು.

ಆದರೆ ನಮ್ಮ ದಾರ್ಶನಿಕರಾದರೂ ಮೊದಲಿನಿಂದಲೂ ವಿಕಾಸವೆಂದು ಕರೆಯಲ್ಪಡುವ ಪ್ರತಿಯೊಂದು ಅಭಿವ್ಯಕ್ತಿಯೂ, ಅವ್ಯಕ್ತವೂ ವ್ಯಕ್ತವಾಗಲೆಣಿಸುವ ವ್ಯರ್ಥ ಪ್ರಯತ್ನ ಎಂದು ಸಾರಿದರು. ಇಡೀ ಬ್ರಹ್ಮಾಂಡದ ಸೃಷ್ಟಿಕರ್ತರಾದ\break ಮಹಾಮಹಿಮರೇ, ಕೆಸರಿನ ಗುಂಡಿಯಲ್ಲಿ ಪ್ರತಿಬಿಂಬಿಸಲು ಏತಕ್ಕೆ ಪ್ರಯತ್ನಿಸು\-ತ್ತಿರುವಿರಿ! ಕೆಲವು ಕಾಲ ಪ್ರಯತ್ನಪಟ್ಟು ಇವೆಲ್ಲ ನಿರಾಶೆ ಎಂದು ತೋರಿದ ಮೇಲೆ ಬಂದೆಡೆಗೆ ಹಿಂತಿರುಗುವಿರಿ. ಇದೇ ವೈರಾಗ್ಯ, ಧರ್ಮದ ಪ್ರಾರಂಭ. ಧರ್ಮ, ನೀತಿ ಮತ್ತು ತ್ಯಾಗಗಳಿಲ್ಲದೆ ಹೇಗೆ ಸಾಧ್ಯ? ತ್ಯಾಗವೇ ಮುಖ್ಯ. “ತ್ಯಜಿಸಿ ತ್ಯಜಿಸಿ” ಎನ್ನುವುದು ವೇದ. ತ್ಯಾಗವೊಂದೇ ಇರುವ ಮಾರ್ಗ. \textbf{“ನ ಪ್ರಜಯಾ ಧನೇನ ತ್ಯಾಗೇನೈಕೇ ಅಮೃತತ್ವಮಾನಶುಃ”} - ಸಂತಾನದಿಂದಲ್ಲ, ದ್ರವ್ಯದಿಂದಲ್ಲ,\break ಕೇವಲ ತ್ಯಾಗದಿಂದ ಕೆಲವರು ಮುಕ್ತಿಯನ್ನು ಗಳಿಸಿದರು. ಹಿಂದೂ ಶಾಸ್ತ್ರಗಳು ಸಾರುವುದೇ ಇದನ್ನು. ಸಿಂಹಾಸನದ ಮೇಲೆ ಕುಳಿತಿದ್ದರೂ ಅನೇಕ ಮಹಾಪುರುಷರು ತ್ಯಾಗಿಗಳಾಗಿದ್ದರು. ಜನಕ ಕೂಡ ತ್ಯಾಗ ಮಾಡಬೇಕಾಯಿತು,\break ಅವನಿಗಿಂತ ಹೆಚ್ಚು ತ್ಯಾಗಿಗಳಾರು? ಆಧುನಿಕ ಕಾಲದಲ್ಲಿ ನಮಗೆಲ್ಲ ಜನಕರೆಂದು ಕರೆಸಿಕೊಳ್ಳಲು ಆಸೆ. ನಾವೆಲ್ಲ ಹೊಟ್ಟೆಗಿಲ್ಲದೆ, ಬಟ್ಟೆಗಿಲ್ಲದೆ ನರಳುತ್ತಿರುವ ಮಕ್ಕಳ ಜನಕರು (ತಂದೆ)! ಜನಕ ಎಂಬ ಪದವನ್ನು ಈ ಅರ್ಥದಲ್ಲಿ ಮಾತ್ರ ನಾವು ಉಪಯೋಗಿಸಬಹುದು. ಹಿಂದಿನ ಜನಕನಲ್ಲಿದ್ದ ಜ್ಯೋತಿರ್ಮಯ ದೈವೀ ಗುಣಗಳಿಲ್ಲ. ಇವರೇ ನಮ್ಮ ಆಧುನಿಕ ಕಾಲದ ಜನಕರು. ಈ ಜನಕತ್ವವನ್ನು ಬಿಟ್ಟು ಸೀದಾ ರಸ್ತೆಗೆ ಬನ್ನಿ. ತ್ಯಾಗಮಾಡಿದರೆ ನಿಮಗೆ ಧರ್ಮ ದೊರಕುವುದು. ಅದು ಸಾಧ್ಯವಿಲ್ಲದೇ ಇದ್ದರೆ ಜಗತ್ತಿನಲ್ಲಿರುವ ಪ್ರಾಚ್ಯ ಪಾಶ್ಚಾತ್ಯ ಗ್ರಂಥಗಳನ್ನೆಲ್ಲ ಓದಬಹುದು. ಪುಸ್ತಕ ಭಂಡಾರವನ್ನೆಲ್ಲ ಪೂರೈಸಬಹುದು. ಪ್ರಪಂಚದಲ್ಲೇ ದೊಡ್ಡ ಪಂಡಿತರಾಗಬಹುದು. ಆದರೆ ನಿಮ್ಮಲ್ಲಿ ಕರ್ಮಕಾಂಡ ಒಂದೇ ಇದ್ದರೆ ಏನೂ ಪ್ರಯೋಜನವಿಲ್ಲ. ಅಲ್ಲಿ ಆಧ್ಯಾತ್ಮಿಕತೆ ಏನೂ ಇಲ್ಲ. ತ್ಯಾಗದಿಂದ ಮಾತ್ರ ಆಮೃತತ್ವವನ್ನು ಪಡೆಯಬಹುದು. ತ್ಯಾಗವೇ ಶಕ್ತಿ. ಈ ಶಕ್ತಿ ಪ್ರಪಂಚವನ್ನು ಕೂಡ ನಿರ್ಲಕ್ಷಿಸುವುದು. \textbf{“ಬ್ರಹ್ಮಾಂಡಂ ಗೋಷ್ಪದಾಯತೇ.”} ಬ್ರಹ್ಮಾಂಡವೆಲ್ಲ ಗೋವಿನ ಪಾದದಿಂದಾದ ಗುರುತಿನಂತಾಗುವುದು.

ತ್ಯಾಗವೇ ಭಾರತದ ಸನಾತನ ಪತಾಕೆ. ಇದನ್ನೇ ಭರತಖಂಡಪ್ರಪಂಚದಲ್ಲೆಲ್ಲಾ ಹಾರಿಸುವುದು. ನಾಶವಾಗುತ್ತಿರುವ ಜನಾಂಗಗಳಿಗೆ, ಎಲ್ಲಾ ದುರಾಚಾರಿಗಳಿಗೆ, ದೌರ್ಜನ್ಯಪರರಿಗೆ, ಪದೇ ಪದೇ “ತ್ಯಾಗ” ಎಂಬ ಸಂದೇಶದಿಂದ ಎಚ್ಚರಿಕೆ ಕೊಡುತ್ತದೆ. ಹಿಂದೂಗಳೇ, ಈ ಪತಾಕೆಯನ್ನು ಎಂದಿಗೂ ತ್ಯಜಿಸಬೇಡಿ. ಅದನ್ನು ಎತ್ತಿ ಹಿಡಿಯಿರಿ. ನೀವು ದುರ್ಬಲರಾಗಿ ತ್ಯಾಗ ಮಾಡುವುದಕ್ಕೆ ಶಕ್ತಿ ಇಲ್ಲದೇ ಇದ್ದರೂ ಆದರ್ಶವನ್ನು ಕೆಳಗೆ ಎಳೆಯಬೇಡಿ. “ನನ್ನ ಕೈಲಾಗುವುದಿಲ್ಲ. ನಾನು\break ತ್ಯಜಿಸಲಾರೆ” ಎನ್ನಿ. ಆದರೆ ಮಿಥ್ಯಾಚಾರಿಗಳಾಗಬೇಡಿ. ಗ್ರಂಥಪೀಡನೆ ಮಾಡಬೇಡಿ, ವಾದಮಾಡಬೇಡಿ. ಇತರರ ಕಣ್ಣಿಗೆ ಮಣ್ಣು ಹಾಕಲು ಪ್ರಯತ್ನಿಸಬೇಡಿ. ಇದನ್ನು ಮಾಡಬೇಡಿ. ನಮ್ಮ ಕೈಲಾಗುವುದಿಲ್ಲ ಎಂದು ಒಪ್ಪಿಕೊಳ್ಳಿ. ತ್ಯಾಗಭಾವನೆ ಅತಿ ಶ್ರೇಷ್ಠವಾದುದು. ಕೋಟ್ಯಂತರ ಜನರು ಸೋತು ಕೇವಲ ಹತ್ತು ಅಥವಾ ಒಬ್ಬರೇ ಜಯಶಾಲಿಗಳಾಗಿ ಹಿಂತಿರುಗಿದರೆ ಏನಂತೆ! ಸತ್ತ ಲಕ್ಷಾಂತರ ಜನರೇ ಧನ್ಯರು. ಅವರು ಬಲಿದಾನವೇ ವಿಜಯವನ್ನು ಸಂಪಾದಿಸಿರುವುದು. ವೈದಿಕ ಸಂಪ್ರದಾಯದ ಒಂದು ಆದರ್ಶ ಈ ತ್ಯಾಗ. ಬೊಂಬಾಯಿ ಪ್ರಾಂತ್ಯದಲ್ಲಿರುವ ವಲ್ಲಭಾಚಾರ್ಯರ ಪಂಗಡ ಒಂದು ಮಾತ್ರ ಇದಕ್ಕೆ ಹೊರತಾಗಿದೆ. ತ್ಯಾಗ ಎಲ್ಲಿಲ್ಲವೊ ಅಲ್ಲಿ ಏನಾಗುವುದೆಂಬುದು ನಿಮ್ಮಲ್ಲಿ ಅನೇಕರಿಗೆ ಗೊತ್ತಿದೆ. ನಮಗೆ ಸಂಪ್ರದಾಯ ಬೇಕು. ಮೈಗೆ ಬೂದಿ ಬಳಿದುಕೊಳ್ಳುವ, ಕೈಯನ್ನು\break ಯಾವಾಗಲೂ ಎತ್ತಿರುವ ಭಯಾನಕ ಸಂಪ್ರದಾಯಗಳೂ ಬೇಕು. ಅದು\break ಅಸ್ವಾಭಾವಿಕವಾದರೂ ನಮಗೆ ಅವು ಬೇಕು. ಅವುಗಳ ಮುಖ್ಯ ಪಲ್ಲವಿಯೇ ತ್ಯಾಗ. ಭರತಖಂಡದಲ್ಲಿ ಧಾಳಿ ಇಡುತ್ತಿರುವ, ನಮ್ಮನ್ನು ಬಲಹೀನರನ್ನಾಗಿ ಮಾಡುವ ಆಮೋದ ಪ್ರಮೋದಗಳಿಗೆ ಬಲಿಯಾಗುವುದಕ್ಕೆ ಮುಂಚೆ, ನಮಗೆ ಎಚ್ಚರಿಕೆ ಕೊಡುವುದಕ್ಕೆ ಈ ತ್ಯಾಗಭಾವನೆ ಬೇಕು. ನಮಗೆ ಸ್ವಲ್ಪ ತಪಸ್ಸು ಬೇಕು. ಪುರಾತನ ಕಾಲದಲ್ಲಿ ತ್ಯಾಗವೇ ಭಾರತವನ್ನು ಗೆದ್ದಿತು, ಈಗ ಅದು ಭಾರತವನ್ನು ಗೆಲ್ಲಬೇಕಾಗಿದೆ. ಈಗಲೂ ಭಾರತದಲ್ಲಿ ತ್ಯಾಗವು ಅತ್ಯುತ್ತಮ ಮತ್ತು ಪರಮಶ್ರೇಷ್ಠ ಆದರ್ಶವಾಗಿದೆ. ಬುದ್ಧ, ರಾಮಾನುಜ, ರಾಮಕೃಷ್ಣ\break ಪರಮಹಂಸರ ನಾಡಾದ, ಕರ್ಮಕಾಂಡಕ್ಕೆ ವಿರುದ್ಧವಾಗಿ ಬೋಧಿಸಿದ ತ್ಯಾಗಭೂಮಿಯಾದ ಭಾರತದಲ್ಲಿ ಇಂದಿಗೂ ಸರ್ವತ್ಯಾಗಿಗಳಾದ ಜೀವನ್ಮುಕ್ತರಾದ ನೂರಾರು ಮಂದಿ ಇರುವರು.

ಇಂಥ ದೇಶ ತನ್ನ ಆದರ್ಶವನ್ನು ಬಿಟ್ಟುಬಿಡುವುದೆ? ಖಂಡಿತ ಇಲ್ಲ. ಪಾಶ್ಚಾತ್ಯ ವಿಲಾಸ ಜೀವನದ ಆದರ್ಶಕ್ಕೆ ಮಾರುಹೋದ ಜನರಿರಬಹುದು, ಪಾಶ್ಚಾತ್ಯರಿಗೆ ಶಾಪವಾಗಿರುವ ಜಗತ್ತಿಗೇ ಶಾಪವಾಗಿರುವ ಇಂದ್ರಿಯಲೋಲುಪತೆಯಲ್ಲಿ ಸಾವಿರಾರು ಜನರು ಉನ್ಮತ್ತರಾಗಿರಬಹುದು, ಆದರೂ ಧರ್ಮವನ್ನೇ ಸತ್ಯವೆಂದು ನಂಬಿರುವ ಸಾವಿರಾರು ಜನರು ನಮ್ಮ ಮಾತೃಭೂಮಿಯಲ್ಲಿರುವರು, ಅವರು ಯಾವುದೇ ಬೆಲೆ ತೆತ್ತಾದರೂ ಸದಾ ತ್ಯಾಗಕ್ಕೆ ಸಿದ್ಧರಾಗಿರುವರು.

ಎಲ್ಲಾ ಪಂಥಗಳಲ್ಲಿಯೂ ರೂಢಿಯಲ್ಲಿರುವ ಮತ್ತೊಂದು ಭಾವನೆಯನ್ನು ಮುಂದಿಡುವೆನು. ಇದು ಕೂಡ ವಿಸ್ತಾರವಾದ ವಿಷಯ. ಧರ್ಮವನ್ನು ಸಾಕ್ಷಾತ್ಕಾರ ಮಾಡಿಕೊಳ್ಳಬೇಕು ಎಂಬ ಅಪೂರ್ವ ಭಾವನೆ ಈ ನಮ್ಮ ದೇಶದಲ್ಲಿ ಮಾತ್ರ ಇರುವುದು. \textbf{“ನಾಯಮಾತ್ಮಾ ಪ್ರವಚನೇನ ಲಭ್ಯೋ ನ ಮೇಧಯಾ ನ ಬಹುನಾ ಶ್ರುತೇನ”} - ಹೆಚ್ಚು ಮಾತನಾಡುವುದರಿಂದ ಆತ್ಮವು ದೊರಕುವುದಿಲ್ಲ. ಹೆಚ್ಚು ಬುದ್ಧಿಯಿಂದಲೂ ದೊರಕುವುದಿಲ್ಲ. ನಮ್ಮ ಶಾಸ್ತ್ರವೊಂದೇ ಪ್ರಪಂಚದಲ್ಲಿ ಕೇವಲ ಶಾಸ್ತ್ರ ಅಧ್ಯಯನದಿಂದ ಆತ್ಮ ಸಾಕ್ಷಾತ್ಕಾರ ಆಗುವುದಿಲ್ಲವೆಂದು ಹೇಳುವುದು. ಮಾತು ಉಪನ್ಯಾಸ ಯಾವುದೂ ಪ್ರಯೋಜನವಿಲ್ಲ. ಸತ್ಯವನ್ನು ಸ್ವತಃ ಸಾಕ್ಷಾತ್ಕಾರ ಮಾಡಿಕೊಳ್ಳಬೇಕು. ಅದು ಗುರುವಿನಿಂದ ಶಿಷ್ಯನಿಗೆ ಬರುವುದು. ಈ ಅಂತರ್ದೃಷ್ಟಿ ಶಿಷ್ಯನಿಗೆ ಬಂದಾಗ ಆತಂಕಗಳೆಲ್ಲ ದೂರವಾಗಿ ಆತ್ಮಸಾಕ್ಷಾತ್ಕಾರ ಸಿದ್ಧಿಸುವುದು.

ಮತ್ತೊಂದು ಭಾವನೆ. ಕುಲಗುರು ಅಥವಾ ವಂಶಪಾರಂಪರ್ಯವಾಗಿ ಬಂದ ಗುರುಗಳು ಎಂಬ ವಿಚಿತ್ರ ಪದ್ಧತಿ ವಂಗದೇಶದಲ್ಲಿ ರೂಢಿಯಲ್ಲಿದೆ. “ನಮ್ಮ ತಂದೆ ನಿಮ್ಮ ಗುರುವಾಗಿದ್ದರು. ಈಗ ನಾನು ನಿನ್ನ ಗುರುವಾಗುವೆನು. ನನ್ನ ತಂದೆ ನಿನ್ನ ತಂದೆಯ ಗುರು, ಅದಕ್ಕೆ ನಾನು ನಿನ್ನ ಗುರು.” ಗುರು ಎಂದರೆ ಏನು? ಶ್ರುತಿಗಳು ಈ ವಿಷಯದಲ್ಲಿ ಏನು ಹೇಳುತ್ತವೆಯೋ ನೋಡೋಣ. ಯಾರಿಗೆ ಶಾಸ್ತ್ರಸಾರ ಗೊತ್ತಿದೆಯೊ ಅವನು ಗುರು. ಸುಮ್ಮನೆ ಪುಸ್ತಕ ಪಾಂಡಿತ್ಯವಲ್ಲ, ಶಾಸ್ತ್ರಗಳ ಅರ್ಥ ಗೊತ್ತಿರಬೇಕು. “ಗಂಧದ ಮರದ ಭಾರವನ್ನು ಹೊರುವ ಕತ್ತೆಗೆ ಅದರ ಭಾರ ಮಾತ್ರ ವೇದ್ಯವಾಗುವುದು, ಅದರ ಗುಣವಲ್ಲ.” ಹಾಗೆಯೇ ಪಂಡಿತರೂ ಕೂಡ. ನಮಗೆ ಅಂತಹವರು ಬೇಕಾಗಿಲ್ಲ. ತಮಗೇ ಸಾಕ್ಷಾತ್ಕಾರವಾಗಿಲ್ಲದೇ ಇದ್ದವರು ಏನು ತಿಳಿಸಬಲ್ಲರು? ನಾನು ಕಲ್ಕತ್ತೆಯಲ್ಲಿ ಹುಡುಗನಾಗಿದ್ದಾಗ ಧರ್ಮವನ್ನು ಹುಡುಕಿಕೊಂಡು ಮನೆ ಮನೆ ಅಲೆಯುತ್ತಿದ್ದೆ. ದೊಡ್ಡ ಉಪನ್ಯಾಸವನ್ನು ಕೇಳಿ ಆದ ಮೇಲೆ ಉಪನ್ಯಾಸ ನೀಡಿದವರನ್ನು “ನೀವು ದೇವರನ್ನು ಕಂಡಿರು\-ವಿರಾ?” ಎಂದು ಕೇಳುತ್ತಿದ್ದೆ. ದೇವರನ್ನು ನೋಡಿದ್ದೀರಾ ಎಂಬ ಪ್ರಶ್ನೆ ಕೇಳಿ ಅವರಿಗೆ ಆಶ್ಚರ್ಯವಾಗುತ್ತಿತ್ತು. ದೇವರನ್ನು ನೋಡಿರುವೆನು ಎಂದು ಹೇಳಿ\break ದವರು ರಾಮಕೃಷ್ಣ ಪರಮಹಂಸರೊಬ್ಬರೇ. ಇಷ್ಟು ಮಾತ್ರವಲ್ಲ, “ಅವನನ್ನು ನೋಡುವುದಕ್ಕೆ ನಿನಗೂ ಸಹಾಯ ಮಾಡುತ್ತೇನೆ” ಎಂದರು. ಶಾಸ್ತ್ರಗಳನ್ನು ಹಿಂಸಿಸಿ ಹಲವು ಬಗೆಯ ಅರ್ಥಗಳನ್ನು ಕೊಡುವವನು ಗುರುವಲ್ಲ.

\begin{verse}
\textbf{ವಾಗ್ವೈಖರೀ ಶಬ್ದಝರೀ ಶಾಸ್ತ್ರವ್ಯಾಖ್ಯಾನಕೌಶಲಂ~।}\\\textbf{ವೈದುಷ್ಯಂ ವಿದುಷಾಂ ತದ್ವದ್ಭುಕ್ತಯೇ ನ ತು ಮುಕ್ತಯೇ~॥}
\end{verse}

“ಹಲವು ಬಗೆಯ ಮಾತುಗಳ ಚಮತ್ಕಾರ, ನಿರರ್ಗಳ ಶಬ್ದ ಪ್ರಯೋಗ, ಶಾಸ್ತ್ರಗಳಿಗೆ ಹಲವು ಬಗೆಯ ವಿವರಣೆ ಕೊಡುವ ಕೌಶಲ ಕೇವಲ ಪಂಡಿತರ\break ರಂಜನೆಗೆ ಮಾತ್ರ, ಮುಕ್ತಿಗಲ್ಲ.” ಶ್ರೋತ್ರಿಯ-ಶ್ರುತಿಸಾರವನ್ನು ತಿಳಿದಿರುವವನು, ಅವೃಜಿನ-ಪಾಪದೂರನು, ಅಕಾಮಹತ-ಕಾಮದಿಂದ ಪಾರಾದವನು,\break ಇವನೇ ನಿಜವಾದ ಗುರು. ನಮಗೆ ಕಲಿಸಿ ಹಣ ಮಾಡುವ ಇಚ್ಛೆ ಅವನಿಗೆ\break ಇರುವುದಿಲ್ಲ. ಅವನೇ ಶಾಂತನೂ ಸಾಧುವೂ ಆಗಿರುತ್ತಾನೆ. ಅವನು ಚಿಗುರು\break ಹೂವುಗಳನ್ನು ಮರಗಳಿಗೆ ತರುವ ವಸಂತದಂತೆ ಇರುವನು. ಅದು ಮರದಿಂದ\break ಏನನ್ನೂ ಆಶಿಸುವುದಿಲ್ಲ. ಅದರ ಸ್ವಭಾವವೇ ಒಳ್ಳೆಯದನ್ನು ಮಾಡುವುದು. ಹಾಗೆ ಮಾಡಿ ಅದು ಸುಮ್ಮನೆ ಇರುವುದು. ಅಂಥವನೇ ಗುರು. \textbf{“ತೀರ್ಣಾಃ ಸ್ವಯಂ ಭೀಮಭವಾರ್ಣವಂ ಜನಾನ್​ ಅಹೇತುನಾನ್ಯಾನಪಿ ತಾರಯನ್ತಃ”} - ಯಾವನು ಈ ಅಗಾಧವಾದ ಸಂಸಾರಸಾಗರವನ್ನು ದಾಟಿರುವನೋ, ಯಾವ ಪ್ರತಿಫಲಾಪೇಕ್ಷೆಯೂ ಇಲ್ಲದೆ ಯಾವನು ಇತರರೂ ಸಂಸಾರ ಸಾಗರವನ್ನು ದಾಟಲು ಸಹಾಯಮಾಡುವನೊ ಆವನು ಮಾತ್ರ ಗುರು, ಇತರರಲ್ಲ.

\begin{verse}
\textbf{ಅವಿದ್ಯಾಯಾಮಂತರೇ ವರ್ತಮಾನಾಃ}\\\textbf{ಸ್ವಯಂ ಧೀರಾಃ ಪಂಡಿತಂ ಮನ್ಯಮಾನಾಃ~।}\\\textbf{ದಂದ್ರಮ್ಯಮಾಣಾಃ ಪರಿಯಂತಿ ಮೂಢಾಃ}\\\textbf{ಅಂಧೇನೈವ ನೀಯಾಮಾನಾ ಯಥಾಂಧಾಃ~॥}
\end{verse}

“ಅವರೇ ಅವಿದ್ಯೆಯ ಕೂಪದಲ್ಲಿರುವರು. ಆದರೂ ತಮಗೆ ಎಲ್ಲಾ ತಿಳಿದಿರುವುದೆಂದು ಭಾವಿಸುವರು. ಇತರರಿಗೆ ಸಹಾಯಮಾಡಲು ಕಾತರರಾಗಿರುವರು. ಅವರೇ ತಪ್ಪು ದಾರಿಯಲ್ಲಿ ಆಲೆದಾಡುತ್ತಿರುವರು. ಕುರುಡನು ಕುರುಡನಿಗೆ ದಾರಿ ತೋರುವಂತೆ ಇಬ್ಬರೂ ಬಾವಿಗೆ ಬೀಳುವರು” (ಕಠೋಪನಿಷತ್​). ನಿಮ್ಮ ಈಗಿನ ಸ್ಥಿತಿಯನ್ನು ಅವರೊಂದಿಗೆ ಹೋಲಿಸಿ ನೋಡಿ. ನೀವು ವೇದಾಂತಿಗಳು, ಬಹಳ ಆಚಾರಶೀಲರು, ಅಲ್ಲವೇ? ನಾನು ನಿಮ್ಮನ್ನು ಹೆಚ್ಚು ಆಚಾರಶೀಲರನ್ನಾಗಿ ಮಾಡಲಿಚ್ಛಿಸುವೆನು. ನೀವು ಹೆಚ್ಚು ಆಚಾರಶೀಲರಾದಷ್ಟೂ ಹೆಚ್ಚು ವಿವೇಕಿಗಳಾಗುವಿರಿ. ಆಧುನಿಕ ಆಚಾರ ಶೀಲತೆಯನ್ನು ಹೆಚ್ಚು ಪರಿಗಣಿಸಿದಷ್ಟೂ ಹುಚ್ಚರಾಗುವಿರಿ. ನಿಮ್ಮ ಹಿಂದಿನ ಆಚಾರ ಶೀಲತೆಗೆ ಹಿಂದಿರುಗಿ. ಏಕೆಂದರೆ ಆಗಿನ ಕಾಲದಲ್ಲಿ ಶಾಸ್ತ್ರಗಳಿಂದ ಬಂದ ಪ್ರತಿಯೊಂದು ವಾಣಿಯ ಪ್ರತಿಯೊಂದು ಆಚರಣೆಯ ಹಿಂದೆಯೂ ಧೀರ ಸ್ಥಿರ ಶುದ್ಧ ಹೃದಯವಿತ್ತು, ಪ್ರತಿಯೊಂದು ಸತ್ಯವಾಗಿತ್ತು. ಅನಂತರ ಕಲೆ ವಿಜ್ಞಾನ ಧರ್ಮ ಮತ್ತು ಪ್ರತಿಯೊಂದು ಕಾರ್ಯಕ್ಷೇತ್ರದಲ್ಲಿಯೂ ಅಧೋಗತಿ ಆರಂಭವಾಯಿತು. ಅದರ ಕಾರಣವನ್ನು ವಿಚಾರಿಸಲು ಸಮಯವಿಲ್ಲ. ಆಗಿನ ಕಾಲದಲ್ಲಿ ಬರೆದ ಗ್ರಂಥಗಳೆಲ್ಲ ಈ ಅಧೋಗತಿಯನ್ನು ವ್ಯಕ್ತಪಡಿಸುವುವು - ಧೈರ್ಯದ ಬದಲು ಗೋಳು, ಅಳು. ಸ್ಫೂರ್ತಿ ಇದ್ದ ಹಿಂದಿನ ಕಾಲಕ್ಕೆ ಹೋಗಿ, ಮತ್ತೊಮ್ಮೆ ಬಲಾಢ್ಯರಾಗಿ, ಹಿಂದಿನ ಅಮೃತ ಚಿಲುಮೆಯಲ್ಲಿ ಪಾನಮಾಡಿ. ಭಾರತವನ್ನು ಪುನರುಜ್ಜೀವನಗೊಳಿಸುವುದಕ್ಕೆ ಇದೊಂದೇ ದಾರಿ.

ಅದ್ವೈತದ ಪ್ರಕಾರ ನಮ್ಮಲ್ಲಿ ಇರುವ ವ್ಯಕ್ತಿತ್ವವು ಒಂದು ಭ್ರಾಂತಿ. ಜಗತ್ತಿನಲ್ಲೆಲ್ಲಾ ಇದೊಂದು ಬಿಡಿಸಲಾರದ ಪ್ರಶ್ನೆ. ನೀವು ಒಬ್ಬನಿಗೆ ಅವನು ಒಬ್ಬ ವ್ಯಕ್ತಿಯಲ್ಲ ಎಂದು ಹೇಳಿ. ತಕ್ಷಣ ತನ್ನ ವ್ಯಕ್ತಿತ್ವ ಏನಾಗಿ ಹೋಗುವುದೋ ಎಂದು ಅವನು ಭಯಭ್ರಾಂತನಾಗುವನು. ಅದ್ವೈತಿಯು ನಿಮಗೆ ಎಂದಿಗೂ ವ್ಯಕ್ತಿತ್ವ ಇರಲೇ ಇಲ್ಲ ಎನ್ನುವನು. ನೀವು ಪ್ರತಿಕ್ಷಣವೂ ಬದಲಾಗುತ್ತಿದ್ದೀರಿ. ಮಗುವಾಗಿದ್ದಾಗ ಒಂದು ರೀತಿ ಆಲೋಚನೆ ಮಾಡುತ್ತಿದ್ದಿರಿ. ಈಗ ಮನುಷ್ಯರಾಗಿರುವಿರಿ. ಈಗೊಂದು ಬಗೆಯ ಆಲೋಚನೆ ಮಾಡುವಿರಿ. ಮುದುಕರಾದ ಮೇಲೆ ಬೇರೊಂದು ಬಗೆ ಆಲೋಚಿಸುವಿರಿ. ಪ್ರತಿಯೊಬ್ಬರೂ ಬದಲಾಗುತ್ತಿರುವರು. ಹೀಗಿರುವಾಗ ನಿಮ್ಮ ವ್ಯಕ್ತಿತ್ವ ಎಲ್ಲಿದೆ? ಅದು ನಿಜವಾಗಿ ದೇಹ, ಮನಸ್ಸು, ಆಲೋಚನೆಗಳಲ್ಲಿಲ್ಲ. ಅದರಾಚೆ ಆತ್ಮನಿರುವುದು. ಆತ್ಮನೇ ಬ್ರಹ್ಮ. ಎರಡು ಅನಂತ ಇರಲಾರದು ಎಂದು ಅದ್ವೈತ ಹೇಳುವುದು. ಇರುವುದು ಒಂದೇ ಒಂದು ವ್ಯಕ್ತಿ. ಅದೇ ಅನಂತ. ನೇರವಾದ ಮಾತಿನಲ್ಲಿ ಹೇಳುವುದಾದರೆ, ನಾವು ವಿಚಾರಮತಿಗಳು, ಆಲೋಚಿಸಬೇಕಾಗಿದೆ. ವಿಚಾರ ಎಂದರೆ ಏನು? ಹೆಚ್ಚು ಕಡಿಮೆ, ವಿಶ್ಲೇಷಣೆಯೆ ವಿಚಾರ, ಕೊನೆ ಮುಟ್ಟುವವರೆಗೂ ವಿಶ್ಲೇಷಿಸುವುದು. ಸಾಂತವಾದುದು ಅನಂತದೊಡನೆ ಒಂದಾದಾಗಲೇ ಆತ್ಯಂತಿಕ ವಿಶ್ರಾಂತಿಯನ್ನು ಪಡೆಯುತ್ತದೆ. ಸಾಂತವಾದ ವಸ್ತುವನ್ನು ವಿಶ್ಲೇಷಣೆ ಮಾಡುತ್ತಾ ಹೋಗಿ. ಅನಂತವನ್ನು ಸೇರುವವರೆಗೆ ನೀವು ಎಲ್ಲಿಯೂ ನಿಲ್ಲಲಾರಿರಿ. ಆ ಅನಂತ ಒಂದೇ ಇರುವುದು. ಉಳಿದುದೆಲ್ಲ ಮಾಯೆ. ಮತ್ತಾವುದೂ ನಿಜವಾಗಿ ಇಲ್ಲ. ಯಾವುದಾದರೂ ವಸ್ತುವಿನಲ್ಲಿ ಸತ್ಯಾಂಶವಿದ್ದರೆ ಅದೇ ಬ್ರಹ್ಮ. ನಾವೇ ಆ ಬ್ರಹ್ಮ, ನಾಮರೂಪಗಳೆಲ್ಲ ಮಾಯೆ. ನಾಮ ರೂಪಗಳನ್ನು ತೆಗೆದುಬಿಟ್ಟರೆ ನಾವೆಲ್ಲ ಒಂದೆ. ಆದರೆ ‘ನಾನು’ ಎಂಬ ಪದದ ವಿಷಯದಲ್ಲಿ ನಾವು ಎಚ್ಚರಿಕೆಯಿಂದಿರಬೇಕು. ನಾನು ಬ್ರಹ್ಮನಾದರೆ ಯಾವ ಕೆಲಸವನ್ನಾದರೂ ನಾನು ಏಕೆ ಮಾಡಲಾಗುವುದಿಲ್ಲ - ಎಂದು ಸಾಮಾನ್ಯವಾಗಿ ಜನರು ಕೇಳುತ್ತಾರೆ. ಆದರೆ\break ಇಲ್ಲಿ ಆ ಪದವನ್ನು ಬೇರೆ ಅರ್ಥದಲ್ಲಿ ಉಪಯೋಗಿಸುತ್ತಿರುವೆವು. ನೀವು\break ಎಲ್ಲಿಯವರೆಗೂ ಬದ್ಧರೆಂದು ಭಾವಿಸುವಿರೋ ಅಲ್ಲಿಯವರೆಗೆ ನೀವು ಬ್ರಹ್ಮನಲ್ಲ. ಅವನಿಗೆ ಯಾವುದೂ ಬೇಕಾಗಿಲ್ಲ. ಅವನು ಅಂತರ್ಜ್ಯೋತಿ. ಅವನ ಆನಂದ ಸುಖವೆಲ್ಲ ಆಂತರ್ಯದಲ್ಲಿರುವುದು. ಅವನು ಆತ್ಮತೃಪ್ತನಾಗಿರುವನು. ಅವನಿಗೆ ಏನೂ ಬೇಕಾಗಿಲ್ಲ, ಅವನು ಸಂಪೂರ್ಣ ನಿರ್ಭೀತನು, ಸಂಪೂರ್ಣ ಸ್ವತಂತ್ರನು. ಅದೇ ಬ್ರಹ್ಮ, ಆ ಬ್ರಹ್ಮಸ್ವರೂಪದಲ್ಲಿ ನಾವೆಲ್ಲ ಒಂದೇ.

ಇಲ್ಲೇ ದ್ವೈತವಾದಿಗಳಿಗೂ, ಅದ್ವೈತವಾದಿಗಳಿಗೂ ದೊಡ್ಡ ಅಂತರವಿರುವುದು. ಶ್ರೇಷ್ಠ ಭಾಷ್ಯಕಾರರಾದ ಶಂಕರಾಚಾರ್ಯರೇ, ಕೆಲವು ವೇಳೆ ಮಂತ್ರಗಳಿಗೆ ಕೊಡುವ ಅರ್ಥ ನನಗೆ ಸರಿ ಕಾಣುವುದಿಲ್ಲ. ಕೆಲವು ವೇಳೆ ರಾಮಾನುಜರು ಕೊಡುವ ಮಂತ್ರಗಳ ವಿವರಣೆಯೂ ಸ್ಪಷ್ಟವಾಗಿಲ್ಲ. ನಮ್ಮ ಪಂಡಿತರಲ್ಲಿ ಕೂಡ, ಈ ಮತಗಳಲ್ಲಿ ಒಂದು ಸತ್ಯವಾಗಿರಬೇಕು, ಉಳಿದದ್ದು ಮಿಥ್ಯೆಯಾಗಿರಬೇಕು ಎಂಬ ಭಾವನೆ ಬಂದುಹೋಗಿದೆ. ಭಾರತವಿನ್ನೂ ಜಗತ್ತಿಗೆ ನೀಡಬೇಕಾಗಿರುವ ಆ ಮಹೋನ್ನತವಾದ \textbf{‘ಏಕಂ ಸತ್​ ವಿಪ್ರಾಃ ಬಹುಧಾ ವದಂತಿ’} ಎಂಬ ಭಾವನೆ ಶುತ್ರಿಯಲ್ಲಿ ಬಂದಿದ್ದರೂ ಅವರು ಗಮನಿಸುವುದಿಲ್ಲ. ನಮ್ಮ ರಾಷ್ಟ್ರ ಜೀವನದ ಮೂಲ ಮಂತ್ರ ಇದು, ನಮ್ಮ ರಾಷ್ಟ್ರ ಜೀವನದ ಸಮಸ್ಯೆಯನ್ನು ಪರಿಹರಿಸುವುದೆಂದರೆ \textbf{‘ಏಕಂ ಸದ್ವಿಪ್ರಾ ಬಹುಧಾ ವದಂತಿ’} ಎಂಬ ಈ ಭಾವನೆಯನ್ನು ಅನುಷ್ಠಾನಕ್ಕೆ ತರುವುದು. ಕೆಲವು ಆಧ್ಯಾತ್ಮಿಕ ಜೀವಿಗಳನ್ನು, ಜ್ಞಾನಿಗಳನ್ನು ಬಿಟ್ಟರೆ ಈ ಸಂದೇಶವನ್ನು ನಾವು ಯಾವಾಗಲೂ ಮರೆತಿರುವೆವು. ಪಂಡಿತರಲ್ಲಿ ಶೇಕಡ ತೊಂಭತ್ತೆಂಟು ಮಂದಿ ಅದ್ವೈತ, ವಿಶಿಷ್ಟಾದ್ವೈತ ಮತ್ತು ದ್ವೈತ-ಇವುಗಳಲ್ಲಿ ಯಾವುದೋ ಒಂದು ಮಾತ್ರ ಸತ್ಯವಾಗಿರಬೇಕೆಂದು ದೃಢವಾಗಿ ನಂಬುತ್ತಾರೆ. ನೀವು ಕಾಶಿಗೆ ಹೋಗಿ ಯಾವುದಾದರೂ ಘಾಟಿನಲ್ಲಿ ಐದು ನಿಮಿಷ ಕುಳಿತರೆ ನಾನು ಹೇಳುವುದು ಸತ್ಯವೆಂದು ನಿಮಗೆ ಗೊತ್ತಾಗುವುದು. ಅವರಲ್ಲಿ ಗೂಳಿಯುದ್ಧ ನಡೆಯುವುದನ್ನು ನೀವು ಅಲ್ಲಿ ನೋಡಬಹುದು.

ಇಂತಹ ಪರಿಸ್ಥಿತಿಯಲ್ಲಿ ಒಬ್ಬ ಮಹಾಪುರುಷರು ಉದಿಸಿದರು. ಅವರ ಜೀವನವೇ ಭರತಖಂಡದ ವಿಭಿನ್ನ ಸಂಪ್ರದಾಯಗಳ ಸಮನ್ವಯ ಕ್ಷೇತ್ರವಾಗಿತ್ತು. ಅವರೇ ಶ‍್ರೀರಾಮಕೃಷ್ಣ ಪರಮಹಂಸರು. ದ್ವೈತ ಅದ್ವೈತಗಳೆರಡೂ ಆವಶ್ಯಕ ಎಂಬುದನ್ನು ಅವರ ಜೀವನವು ತೋರಿಸುತ್ತದೆ. ಇವು ಖಗೋಳ ಶಾಸ್ತ್ರದಲ್ಲಿ ಭೂಕೇಂದ್ರ \enginline{(Geocentric)} ಸೂರ್ಯಕೇಂದ್ರ \enginline{(Heliocentric)} ವೆಂಬ ಸಿದ್ಧಾಂತಗಳಿರುವಂತೆ. ಮಗುವಿಗೆ ಖಗೋಳಶಾಸ್ತ್ರವನ್ನು ಕಲಿಸುವಾಗ ಮೊದಲು ಭೂಕೇಂದ್ರವನ್ನು ಕಲಿಸುವರು; ಅದಕ್ಕೆ ಸಂಬಂಧಪಟ್ಟ ಸಮಸ್ಯೆಯನ್ನು ಅವನು ಕಲಿತುಕೊಳ್ಳುವನು. ಖಗೋಳ ಶಾಸ್ತ್ರದ ಸೂಕ್ಷ್ಮ ವಿಷಯಗಳಿಗೆ ಬಂದಾಗ ಸೂರ್ಯ ಕೇಂದ್ರ ಆವಶ್ಯಕ. ಆಗ ಅವನು ವಿಷಯವನ್ನು ಚೆನ್ನಾಗಿ ತಿಳಿದುಕೊಳ್ಳುವನು. ದ್ವೈತವೇ ಪಂಚೇಂದ್ರಿಯಗಳ ಸ್ವಾಭಾವಿಕ ಭಾವನೆ. ಎಲ್ಲಿಯವರೆಗೆ ನಾವು ಇಂದ್ರಿಯಗಳಿಗೆ ಬದ್ಧರಾಗಿರುವೆವೋ ಅಲ್ಲಿಯವರೆಗೆ, ಸಗುಣ ದೇವರನ್ನು ಮಾತ್ರ ನೋಡಲು ಸಾಧ್ಯ, ಬೇರೆ ಇಲ್ಲ. ಹಾಗೆಯೇ ಜಗತ್ತನ್ನೂ ಈಗಿರುವಂತೆ ನೋಡಲೇ ಬೇಕಾಗಿದೆ. ರಾಮಾನುಜರು, “ಎಲ್ಲಿಯವರೆಗೆ ನೀನು ದೇಹವೆಂದು ಭಾವಿಸುವೆಯೋ, ಮನಸ್ಸು ಎಂದು ಭಾವಿಸುವೆಯೋ, ಜೀವ ಎಂದು ಭಾವಿಸುವೆಯೋ, ಅಲ್ಲಿಯವರೆಗೆ ಪ್ರತಿಯೊಂದು ಇಂದ್ರಿಯ ಗ್ರಹಣದ ಹಿಂದೆ ಜೀವ, ಜಗತ್​ ಮತ್ತು ಇವೆರಡಕ್ಕೂ ಕಾರಣವಾದ ಈಶ್ವರನ ಭಾವನೆ ಇರುತ್ತದೆ” ಎನ್ನುತ್ತಾರೆ. ಮಾನವನ ಜೀವನದಲ್ಲಿ ಒಂದು ಸಮಯ ಬರಬಹುದು. ಆಗ ಶರೀರ ಭಾವನೆ ಮಾಯವಾಗುವುದು. ಮನಸ್ಸು ಕ್ರಮೇಣ ಸೂಕ್ಷ್ಮವಾಗುತ್ತಾ ಬಂದು ಸಂಪೂರ್ಣ ಮಾಯವಾದಂತಾಗುತ್ತದೆ. ಆಗ ನಮ್ಮನ್ನು ಅಂಜಿಸುವ, ದುರ್ಬಲರನ್ನಾಗಿ ಮಾಡುವ ದೇಹಕ್ಕೆ ಬಂಧಿಸುವ ಭಾವನೆಗಳೆಲ್ಲ ಮಾಯವಾಗುವುವು. ಆಗ ಮಾತ್ರ ಈ ಪ್ರಾಚೀನ ಮಹಾಬೋಧನೆಯ ಸತ್ಯ ನಮಗೆ ಮನವರಿಕೆಯಾಗುವುದು:

\begin{verse}
\textbf{ಇಹೈವ ತೈರ್ಜಿತಃ ಸರ್ಗೋ ಯೇಷಾಂ ಸಾಮ್ಯೇ ಸ್ಥಿತಂ ಮನಃ~।}\\\textbf{ನಿರ್ದೋಷಂ ಹಿ ಸಮಂ ಬ್ರಹ್ಮ ತಸ್ಮಾತ್​ ಬ್ರಹ್ಮಣಿ ತೇ ಸ್ಥಿತಾಃ~॥}
\end{verse}

“ಯಾರ ಮನಸ್ಸು ಸಮತ್ವದಲ್ಲಿ ಸ್ಥಿರವಾಗಿರುವುದೋ ಅವರು ಈ ಜನ್ಮದಲ್ಲಿಯೇ ಜನನಮರಣಗಳಿಂದ ಪಾರಾಗಿರುವರು. ಬ್ರಹ್ಮವು ನಿರ್ದೋಷವಾದುದು; ಎಲ್ಲರಿಗೂ ಸಮ. ಅಂತಹವರು ಬ್ರಹ್ಮನಲ್ಲಿ ಐಕ್ಯವಾಗಿರುವರು.”

\begin{verse}
\textbf{ಸಮಂ ಪಶ್ಯನ್​ ಹಿ ಸರ್ವತ್ರ ಸಮವಸ್ಥಿತಮೀಶ್ವರಮ್​~।}\\\textbf{ನ ಹಿನಸ್ತ್ಯಾತ್ಮನಾತ್ಮಾನಂ ತತೋ ಯಾತಿ ಪರಾಂ ಗತಿಮ್​~॥}
\end{verse}

“ಸರ್ವತ್ರ ಸಮನಾಗಿರುವ ದೇವರನ್ನು ನೋಡಿ, ಜ್ಞಾನಿಯು ಆತ್ಮನಿಂದ\break ಆತ್ಮನಿಗೆ ಹಾನಿಮಾಡದೇ ಪರಮಗತಿಯನ್ನು ಪಡೆಯುವನು.”

