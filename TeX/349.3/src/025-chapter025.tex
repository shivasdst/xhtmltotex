
\chapter{ಇಂಗ್ಲೆಂಡಿನಲ್ಲಿ ಭಾರತೀಯ ಆಧ್ಯಾತ್ಮಿಕ ಪ್ರಭಾವ}

(ಸೋದರಿ ನಿವೇದಿತಾ ೧೮೯೮, ಮಾರ್ಚ್​ ೧೧ರಂದು ಕಲಕತ್ತೆಯ ಸ್ಟಾರ್​ ಥಿಯೇಟರಿನಲ್ಲಿ ‘ಇಂಗ್ಲೆಂಡಿನಲ್ಲಿ ಭಾರತೀಯ ಆಧ್ಯಾತ್ಮಿಕ ಭಾವನೆಗಳ ಪ್ರಭಾವ’ ಎಂಬ ವಿಷಯದ ಮೇಲೆ ಉಪನ್ಯಾಸ ನೀಡಿದರು. ಆ ಸಭೆಯ ಅಧ್ಯಕ್ಷತೆ\break ವಹಿಸಿದ್ದ ಸ್ವಾಮಿ ವಿವೇಕಾನಂದರು ನಿವೇದಿತಾರನ್ನು ಪರಿಚಯಿಸುತ್ತ ಈ ಕೆಳಗಿನಂತೆ ಮಾತನಾಡಿದರು.)

ಮಹಿಳೆಯರೆ ಮತ್ತು ಮಹನೀಯರೆ,

ಏಷ್ಯಾಖಂಡದ ಪೂರ್ವ ರಾಷ್ಟ್ರಗಳಲ್ಲಿ ನಾನು ಪ್ರಯಾಣ ಮಾಡುತ್ತಿದ್ದಾಗ ಆ ದೇಶಗಳಲ್ಲಿ ಭಾರತೀಯ ಆಧ್ಯಾತ್ಮಿಕ ಭಾವನೆಗಳು ಪ್ರಚಲಿತವಿರುವುದನ್ನು ನಾನು ಗಮನಿಸಿದೆನು. ಚೈನಾ ಮತ್ತು ಜಪಾನಿನ ದೇವಸ್ಥಾನಗಳ ಗೋಡೆಗಳ ಮೇಲೆ ಬರೆದಿದ್ದ ಕೆಲವು ಪ್ರಖ್ಯಾತ ಸಂಸ್ಕೃತ ಮಂತ್ರಗಳನ್ನು ಓದಿದಾಗ ನನಗೆಷ್ಟು ಆಶ್ಚರ್ಯ ವಾಯಿತೆಂಬುದನ್ನು ನೀವು ಊಹಿಸಬಹುದು. ಅವನ್ನು ಹಳೆಯ ಬಂಗಾಳಿ ಲಿಪಿಯಲ್ಲಿ ಬರೆದಿದ್ದರು ಎಂಬುದನ್ನು ಕೇಳಿದರೆ ನಿಮಗೆ ತುಂಬಾ ಸಂತೋಷವಾಗಬಹುದು. ವಂಗದೇಶದ ನಮ್ಮ ಪೂರ್ವಿಕರು ಭಾರತೀಯ ಚಿಂತನೆಗಳ ಪ್ರಚಾರದಲ್ಲಿ ತೋರಿದ ಉತ್ಸಾಹಕ್ಕೆ ಇದು ಒಂದು ಸ್ಮಾರಕವಾಗಿದೆ.

ಏಷ್ಯಾಖಂಡದ ದೇಶಗಳಲ್ಲಿ ಭಾರತೀಯ ಆಧ್ಯಾತ್ಮಿಕ ಭಾವನೆಗಳ ಪ್ರಭಾವವು ಅತ್ಯಂತ ವಿಸ್ತೃತವೂ ಗಮನಾರ್ಹವೂ ಆಗಿರುವಂತೆಯೇ ಪಾಶ್ಚಾತ್ಯ ದೇಶಗಳಲ್ಲಿಯೂ, ಅವುಗಳ ಸಂಸ್ಕೃತಿಯ ಆಳಕ್ಕೆ ಹೋದರೆ, ಇದೇ ಪ್ರಭಾವವು ಈಗಲೂ ಇರುವುದನ್ನು ನಾನು ಗುರುತಿಸಿದೆನು. ಭಾರತೀಯ ಆಧ್ಯಾತ್ಮಿಕ ಚಿಂತನೆಗಳು ಪೂರ್ವಕಾಲದಲ್ಲಿ ಇಲ್ಲಿಂದ ಪೂರ್ವ ದೇಶಗಳಿಗೂ ಪಶ್ಚಿಮ ದೇಶಗಳಿಗೂ ಹರಡಿರು\-ವುದು ಈಗ ಐತಿಹಾಸಿಕ ಸತ್ಯವಾಗಿದೆ. ಜಗತ್ತು ಎಷ್ಟರ ಮಟ್ಟಿಗೆ ಭಾರತೀಯ ಆಧ್ಯಾತ್ಮಿಕತೆಗೆ ಋಣಿಯಾಗಿದೆ, ಭರತಖಂಡದ ಆಧ್ಯಾತ್ಮಿಕ ಶಕ್ತಿಗಳು ಹಿಂದೆ ಮತ್ತು ಈಗ ಮಾನವ ಜನಾಂಗದ ಪ್ರಗತಿಯಲ್ಲಿ ಎಂತಹ ಮಹತ್ವದ ಪಾತ್ರವನ್ನು ವಹಿಸಿವೆ ಎಂಬುದು ಎಲ್ಲರಿಗೂ ತಿಳಿದ ವಿಷಯ. ಇದು ಹಳೆಯ\break ಮಾತಾಯಿತು.

ನನ್ನ ಗಮನಕ್ಕೆ ಬಂದಿರುವ ಮತ್ತೊಂದು ವಿಶೇಷ ಸಂಗತಿ ಏನೆಂದರೆ, ಮಾನವ ಜನಾಂಗ ಮತ್ತು ಸಮಾಜದ ಪ್ರಗತಿಯ ಸಾಧನೆಯಲ್ಲಿ ಅದ್ಭುತ ಆಂಗ್ಲೋ - ಸ್ಯಾಕ್ಸನ್​ ಜನಾಂಗದ ಕೊಡುಗೆ ಮಹತ್ತರವಾದುದು. ನಾನು ಇನ್ನೂ ಸ್ವಲ್ಪ ಮುಂದೆ ಹೋಗಿ, ಆಂಗ್ಲೋ-ಸ್ಯಾಕ್ಸನ್​ರ ಶಕ್ತಿ ಇಲ್ಲದೇ ಇದ್ದಿದ್ದರೆ, ನಾವೀಗ ಇಲ್ಲಿ ಮಾಡುತ್ತಿರುವಂತೆ, ಭಾರತೀಯ ಆಧ್ಯಾತ್ಮಿಕ ಭಾವನೆಯ ಪರಿಣಾಮವನ್ನು ಕುರಿತು ಚರ್ಚಿಸುವುದಕ್ಕೂ ಆಗುತ್ತಿರಲಿಲ್ಲ ಎನ್ನುತ್ತೇನೆ. ಪಶ್ಚಿಮ ದೇಶಗಳಿಂದ ನಮ್ಮ ದೇಶಕ್ಕೆ ಹಿಂತಿರುಗಿದ ಮೇಲೆ, ಇಲ್ಲಿಯೂ ಅದೇ ಆಂಗ್ಲೋ-ಸ್ಯಾಕ್ಸನ್​ ಶಕ್ತಿಯು, ತನ್ನಲ್ಲಿ ಎಷ್ಟೋ ಕುಂದು ಕೊರತೆಗಳಿದ್ದರೂ ಅಪೂರ್ವವಾದ ತನ್ನ ಒಳ್ಳೆಯ ಸ್ವಭಾವವನ್ನು ಉಳಿಸಿಕೊಂಡು, ಕೆಲಸ ಮಾಡುತ್ತಿರುವುದನ್ನು ನೋಡುತ್ತಿರುವೆನು. ಅಂತಿಮವಾಗಿ ಒಂದು ಮಹಾ ಪರಿಣಾಮ ಉಂಟಾಗುತ್ತದೆ ಎಂದು ನಂಬುತ್ತೇನೆ. ಬ್ರಿಟಿಷರ ವಿಕಾಸ ಮತ್ತು ಪ್ರಗತಿಯ ಭಾವನೆಗಳು ನಮ್ಮನ್ನು ಮುಂದುವರಿಯುವಂತೆ ಬಲಾತ್ಕರಿಸುತ್ತಿವೆ. ಪಾಶ್ಚಾತ್ಯ ನಾಗರಿಕತೆಯ ಮೂಲ ಗ್ರೀಕ್​ ಸಂಸ್ಕೃತಿ. ಮತ್ತು ತಮ್ಮ ಜನರ ಭಾವನೆಗಳನ್ನು ಸ್ಪಷ್ಟವಾಗಿ ವ್ಯಕ್ತಪಡಿಸುವುದೇ ಗ್ರೀಕ್​ ನಾಗರಿಕತೆಯ ಮುಖ್ಯ ಲಕ್ಷಣ ಎಂಬುದನ್ನು ನಾವು ನೆನಪಿನಲ್ಲಿಡಬೇಕು. ಭರತಖಂಡದಲ್ಲಿ ನಾವು ವಿಚಾರಶೀಲರು. ದುರದೃಷ್ಟವಶಾತ್​ ಕೆಲವು ವೇಳೆ ನಾವು ಎಷ್ಟೊಂದು ಆಳವಾಗಿ ಆಲೋಚಿಸುತ್ತೇವೆ ಎಂದರೆ ನಮ್ಮ ಶಕ್ತಿಯೆಲ್ಲ ಅದರಲ್ಲೇ ವ್ಯಯವಾಗಿ ಬಾಹ್ಯ ಅಭಿವ್ಯಕ್ತಿಗೆ ಏನೂ ಉಳಿಯಲೇ ಇಲ್ಲ. ಆದಕಾರಣ ನಾವು ಜಗತ್ತಿನ ಎದುರಿಗೆ ನಮ್ಮನ್ನು ನಾವು ಪ್ರಕಾಶಗೊಳಿಸಲಿಲ್ಲ. ಇದರ ಪರಿಣಾಮ ಏನಾಯಿತು? ನಮ್ಮಲ್ಲಿರುವ ಎಲ್ಲವನ್ನೂ ಮುಚ್ಚಿಡುವುದಕ್ಕೆ ಪ್ರಯತ್ನ ಪಟ್ಟೆವು. ಈ ಮುಚ್ಚಿಡುವಿಕೆ ಮೊದಲು ಕೆಲವು ವ್ಯಕ್ತಿಗಳ ಚಾಳಿಯಾಗಿ ಅನಂತರ ಇಡೀ ದೇಶದ ಒಂದು ಸ್ವಭಾವವಾಯಿತು. ನಮ್ಮಲ್ಲಿ ಪ್ರಕಾಶಪಡಿಸುವ ಶಕ್ತಿಯ ಕೊರತೆ ಎಷ್ಟರ ಮಟ್ಟಿಗೆ ಇದೆಯೆಂದರೆ ಇತರರು ನಮ್ಮನ್ನು ಒಂದು ಸತ್ತ ಜನಾಂಗ ಎಂದು ಭಾವಿಸುವಂತಾಗಿದೆ. ನಾವು ಪ್ರಕಾಶಪಡಿಸದೆ ಹೇಗೆ ಬಾಳಬಲ್ಲೆವು? ಪಾಶ್ಚಾತ್ಯ ನಾಗರಿಕತೆಯ ಮೂಲಭಿತ್ತಿಯೇ ವಿಕಾಸ ಮತ್ತು ಅಭಿವ್ಯಕ್ತಿ. ಭರತ ಖಂಡದಲ್ಲಿ ಆಂಗ್ಲೋ-ಸ್ಯಾಕ್ಸನ್​ ಜನಾಂಗವು ಮಾಡುತ್ತಿರುವ ಕೆಲಸದಲ್ಲಿ ಈ ಅಂಶದ ಕಡೆಗೆ ನಿಮ್ಮ ಲಕ್ಷ್ಯವನ್ನು ಸೆಳೆಯುತ್ತೇನೆ; ಅದು ತನ್ನನ್ನು ತಾನು ವ್ಯಕ್ತಪಡಿಸಲು ನಮ್ಮ ರಾಷ್ಟ್ರವನ್ನು ಪ್ರಚೋದಿಸುತ್ತದೆ. ಆ ಪ್ರಬಲ ಜನಾಂಗವೇ ಒದಗಿಸಿರುವ ಸಮೂಹ ಮಾಧ್ಯಮಗಳ ಮೂಲಕ ಭಾರತವು ತನ್ನ ಗುಪ್ತನಿಧಿ ಯನ್ನು ಹೊರ ಜಗತ್ತಿಗೆ ನೀಡಲು ಅದು ಹುರಿದುಂಬಿಸುತ್ತಿದೆ. ಭರತ ಖಂಡಕ್ಕೆ ಆಂಗ್ಲೋ-ಸ್ಯಾಕ್ಸನ್​ರು ಒಂದು ಭವಿಷ್ಯವನ್ನು ಸೃಷ್ಟಿಸಿರುವರು. ನಮ್ಮ ಪೂರ್ವಿಕರ ಭಾವನೆಗಳು ಅದ್ಬುತ ವೇಗದಿಂದ ಹರಡುತ್ತಿವೆ. ನಮ್ಮ ಪೂರ್ವಿಕರು ಸತ್ಯ ಮತ್ತು ಮುಕ್ತಿಯ ಸಂದೇಶವನ್ನು ಜಗತ್ತಿಗೆ ಸಾರಿದಾಗ ಎಷ್ಟು ಸೌಕರ್ಯಗಳು ಅವರಿಗೆ ಇದ್ದುವು? ಭಗವಾನ್​ ಬುದ್ಧನು ವಿಶ್ವಸಹೋದರತ್ವದ ಭಾವನೆಯನ್ನು ಹೇಗೆ ಪ್ರಚಾರ ಮಾಡಿದ? ಆಗಲೂ ನಮ್ಮ ಪ್ರಿಯ ಭರತಖಂಡದಲ್ಲಿ ನಿಜವಾದ ಸುಖವನ್ನು ಪಡೆಯುವುದಕ್ಕೆ ಬೇಕಾದಷ್ಟು ಸೌಕರ್ಯಗಳಿದ್ದುವು ಮತ್ತು ಜಗತ್ತಿನ ಒಂದು ಮೂಲೆಯಿಂದ ಮತ್ತೊಂದು ಮೂಲೆಗೆ ನಮ್ಮ ಭಾವನೆಗಳನ್ನು ಬಹಳ ಸುಲಭವಾಗಿ ಕಳುಹಿಸಬಹುದಾಗಿತ್ತು. ಈಗ ಆಂಗ್ಲೋ-ಸ್ಯಾಕ್ಸನ್​ ಜನಾಂಗವನ್ನು ಕೂಡ ನಮ್ಮ ಭಾವನೆಗಳು ಮುಟ್ಟಿರುವುವು. ಈಗ ಪರಸ್ಪರ ಭಾವ ವಿನಿಮಯ ನಡೆಯುತ್ತಿದೆ. ನಮ್ಮ ಸಂದೇಶವನ್ನು ಅವರು ಕೇಳುತ್ತಿರುವರು; ಕೇಳುವುದು ಮಾತ್ರವಲ್ಲ ಅದಕ್ಕೆ ಒಲಿಯುತ್ತಿರುವರು. ನಮ್ಮ ಧ್ಯೇಯ ಸಾಧನೆಗೆ ಇಂಗ್ಲೆಂಡ್​ ಆಗಲೇ ತನ್ನ ಅತಿ ಮೇಧಾವಿಗಳಲ್ಲಿ ಕೆಲವರನ್ನು ಕೊಟ್ಟಿರುವುದು. ಈ ವೇದಿಕೆಯ ಮೇಲೆ\break ಇರುವ ನನ್ನ ಸ್ನೇಹಿತೆ ಮಿಸ್​ ಮುಲ್ಲರರ ವಿಷಯವನ್ನು ನಿಮ್ಮಲ್ಲಿ ಎಲ್ಲರೂ ಕೇಳಿರ\-ಬಹುದು. ಅವರ ಪರಿಚಯ ನಿಮ್ಮಲ್ಲಿ ಅನೇಕರಿಗೆ ಇರಬಹುದು. ಸದ್ವಂಶದಲ್ಲಿ ಹುಟ್ಟಿರುವ, ಒಳ್ಳೆಯ ವಿದ್ಯಾವಂತರಾಗಿರುವ ಈ ಮಹಿಳೆ ಭರತಖಂಡದ ಮೇಲಿನ ಪ್ರೀತಿಯಿಂದ ತಮ್ಮ ಇಡಿಯ ಜೀವನವನ್ನು ನಮಗೆ ಕೊಟ್ಟಿರುವರು.\break ಭರತ ಖಂಡವನ್ನು ತಮ್ಮ ಮನೆಯನ್ನಾಗಿ, ತಮ್ಮ ಸಂಸಾರವನ್ನಾಗಿ ಮಾಡಿ\-ಕೊಂಡಿರುವರು. ಭರತಖಂಡದ ಹಿತಕ್ಕೆ, ಭರತಖಂಡದ ಪುನರುತ್ಥಾನಕ್ಕೆ ತಮ್ಮ ಜೀವನವನ್ನು ತೆರುತ್ತಿರುವ ಮತ್ತೊಬ್ಬ ಪ್ರಖ್ಯಾತ ಆಂಗ್ಲ ಮಹಿಳೆಯ ಹೆಸರು ನಿಮಗೆ ಚಿರಪರಿಚಿತವಾಗಿದೆ. ಅವರೇ ಶ‍್ರೀಮತಿ ಬೆಸೆಂಟ್​. ಇದೇ ಧ್ಯೇಯೋದ್ದೇಶ ಇರುವ ಇಬ್ಬರು ಅಮೆರಿಕಾ ಮಹಿಳೆಯರನ್ನು ಇಂದು ಇದೇ ವೇದಿಕೆಯ\break ಮೇಲೆ ಕಾಣುತ್ತಿದ್ದೀರಿ. ನಮ್ಮ ಬಡದೇಶಕ್ಕೆ ಸಾಧ್ಯವಾದಷ್ಟು ಒಳ್ಳೆಯದನ್ನು ಮಾಡಲು ಅವರೂ ಸಿದ್ಧರಾಗಿರುವರು. ಇದೇ ಸಂದರ್ಭದಲ್ಲಿ ಈ ನಮ್ಮ ದೇಶದ ಮತ್ತೊಬ್ಬರನ್ನು ನಿಮ್ಮ ನೆನಪಿಗೆ ತರಲು ಇಚ್ಛಿಸುತ್ತೇನೆ. ಅವರು ಇಂಗ್ಲೆಂಡ್​ ಮತ್ತು ಅಮೆರಿಕವನ್ನು ನೋಡಿರುವರು. ಅವರಲ್ಲಿ ನನಗೆ ತುಂಬ ಭರವಸೆ ಇದೆ. ಅವರನ್ನು ನಾನು ಗೌರವಿಸುತ್ತೇನೆ ಮತ್ತು ಪ್ರೀತಿಸುತ್ತೇನೆ. ಅವರು ನಮ್ಮ ದೇಶದ ಹಿತಕ್ಕೆ ಸ್ಥಿರವಾಗಿ, ಮೌನವಾಗಿ ದುಡಿಯುತ್ತಿರುವರು. ಅವರು ಶ್ರೇಷ್ಠ ಆಧ್ಯಾತ್ಮಿಕ ವ್ಯಕ್ತಿಗಳು - ಅವರೇ ಶ‍್ರೀ ಮೋಹಿನೀ ಮೋಹನ ಚಟರ್ಜಿ. ಅವರಿಗೆ ಬೇರೆ ಕಾರ್ಯಕ್ರಮವಿಲ್ಲದೇ ಇದ್ದಿದ್ದರೆ ಅವರು ಇಂದು ಇಲ್ಲಿರುತ್ತಿದ್ದರು. ಈಗ ಇಂಗ್ಲೆಂಡ್​ ಭಾರತಕ್ಕೆ ಮಿಸ್​ ಮಾರ್ಗರೇಟ್​ ನೋಬೆಲ್​ ಎಂಬ ಮತ್ತೊಂದು ನಾರೀರತ್ನವನ್ನು ಕಾಣಿಕೆಯಾಗಿ ಕೊಟ್ಟಿದೆ. ಅವರಿಂದ ನಾವು ಬಹಳವಾದ ಸೇವೆಯನ್ನು ನಿರೀಕ್ಷಿಸುತ್ತಿರುವೆವು. ನಾನು ಇನ್ನು ಹೆಚ್ಚು ಮಾತನಾಡದೆ ಮಿಸ್​ ನೊಬೆಲ್​ ಅವರ ಪರಿಚಯವನ್ನು ನಿಮಗೆ ಮಾಡಿಸುತ್ತೇನೆ. ಈಗ ಅವರು ಮಾತನಾಡುವರು.

ಸೋದರಿ ನಿವೇದಿತಾ ಅವರ (ಮಾರ್ಗರೆಟ್​ ನೋಬೆಲ್​) ಸ್ವಾರಸ್ಯಕರವಾದ ಉಪನ್ಯಾಸ ಮುಗಿದ ಮೇಲೆ ಸ್ವಾಮಿ ವಿವೇಕಾನಂದರು ಪುನಃ ಮಾತನಾಡಿದರು:

ನಾನೆಲ್ಲೋ ಸ್ವಲ್ಪ ಮಾತ್ರ ಹೇಳಬೇಕಾಗಿದೆ. ಭಾರತೀಯರಾದ ನಾವು\break ಏನಾದರೂ ಸಾಧಿಸಬಲ್ಲೆವು ಎಂಬುದನ್ನು ನಂಬುತ್ತೇವೆ. ಭಾರತೀಯರಲ್ಲಿ ಬಂಗಾಳಿಗಳು ಇದನ್ನು ಕೇಳಿ ನಗಬಹುದು. ಆದರೆ ನಾನು ಹಾಗೆ ನಗುವು\-ದಿಲ್ಲ. ನಿಮ್ಮಲ್ಲಿ ಒಂದು ಜಾಗೃತಿಯನ್ನು ಉಂಟುಮಾಡುವುದೇ ನನ್ನ ಜೀವನದ ಉದ್ದೇಶ. ನೀವು ಅದ್ವೈತಿಗಳೋ, ವಿಶಿಷ್ಟಾದ್ವೈತಿಗಳೋ, ದ್ವೈತಿಗಳೋ, ಚಿಂತೆಯಿಲ್ಲ. ಆದರೆ ನಿಮ್ಮ ಲಕ್ಷ್ಯವನ್ನು ಒಂದು ಮುಖ್ಯ ವಿಷಯದ ಕಡೆ ಸೆಳೆಯುತ್ತೇನೆ, ದುರದೃಷ್ಟ ವಶಾತ್​ ಅದನ್ನು ನಾವು ಯಾವಾಗಲು ಮರೆಯುತ್ತೇವೆ - ಅದೇನೆಂದರೆ: “ಹೇ ಮಾನವ, ನಿನ್ನಲ್ಲಿ ನಿನಗೆ ಶ್ರದ್ಧೆ ಇರಲಿ.” ಇದರಿಂದ ನಮಗೆ ದೇವರಲ್ಲಿ ಶ್ರದ್ಧೆ ಹುಟ್ಟುವುದು. ನೀವು ದ್ವೈತಿಗಳಾಗಿರಲೀ, ಅದ್ವೈತಿಗಳಾಗಿರಲೀ, ನೀವು ಯೋಗವನ್ನಾದರೂ ನಂಬಿ, ಶಂಕರಾಚಾರ್ಯರನ್ನಾದರೂ\break ನಂಬಿ, ನೀವು ವ್ಯಾಸರನ್ನಾದರೂ ಅನುಸರಿಸಿ, ವಿಶ್ವಾಮಿತ್ರನನ್ನಾದರೂ ಅನುಸರಿಸಿ. ಇದಾವುದೂ ಮುಖ್ಯವಲ್ಲ. ಆದರೆ ಮುಖ್ಯವಾಗಿ ಈ ವಿಷಯದಲ್ಲಿ\break ಭರತಖಂಡದ ಭಾವನೆಯು ಜಗತ್ತಿನ ಇತರ ದೇಶಗಳ ಭಾವನೆಗಳಿಗಿಂತ ಬೇರೆ ಅಗಿದೆ. ಬೇರೆ ಎಲ್ಲ ದೇಶಗಳಲ್ಲಿ ಮತ್ತು ಧರ್ಮಗಳಲ್ಲಿ ಆತ್ಮಶಕ್ತಿಯನ್ನು ಕಡೆಗಣಿಸಲಾಗಿದೆ-ಇದನ್ನು ಸ್ವಲ್ಪ ನೆನಪಿನಲ್ಲಿ ಇಡೋಣ. ಅವರು ಆತ್ಮನನ್ನು\break ಶಕ್ತಿಹೀನ, ದುರ್ಬಲ, ಜಡ ಎಂದು ಭಾವಿಸುತ್ತಾರೆ. ಆದರೆ ಭರತಖಂಡದಲ್ಲಿ\break ನಾವು ಆತ್ಮನನ್ನು ಆದಿ - ಅಂತ್ಯರಹಿತನೆಂದೂ, ಎಲ್ಲಾ ಕಾಲದಲ್ಲಿಯೂ\break ಪರಿಪೂರ್ಣನಾಗಿರುವನು ಎಂದೂ ನಂಬುತ್ತೇವೆ. ನೀವು ಯಾವಾಗಲೂ ಉಪನಿಷತ್ತಿನ ಬೋಧನೆಗಳನ್ನು ಜ್ಞಾಪಕದಲ್ಲಿಟ್ಟಿರಬೇಕು.

ನಿಮ್ಮ ಜೀವನದ ಮಹಾಧ್ಯೇಯವನ್ನು ಸ್ಮರಿಸಿಕೊಳ್ಳಿ. ಭಾರತೀಯರ\break ಅದರಲ್ಲೂ ಬಂಗಾಳಿಗಳ ಮೇಲೆ ವಿದೇಶೀ ಭಾವನೆಗಳು ಧಾಳಿ ಮಾಡಿವೆ. ಅವು ನಮ್ಮ ರಾಷ್ಟ್ರೀಯ ಧರ್ಮದ ಅಂತಃಸತ್ವವನ್ನೇ ಹೀರುತ್ತಿವೆ. ನಾವೇಕೆ ಇಷ್ಟು ಹಿಂದುಳಿದಿರುವೆವು? ನಮ್ಮಲ್ಲಿ ಶೇಕಡ ತೊಂಬತ್ತೊಂಬತ್ತು ಮಂದಿ ವಿದೇಶೀ ಆದರ್ಶಗಳಿಂದಲೇ ತುಂಬಿ ಹೋಗಿರುವರಲ್ಲ ಏಕೆ? ನಮ್ಮ ರಾಷ್ಟ್ರೀಯ ಗೌರವವನ್ನು ಉಳಿಸಿಕೊಳ್ಳಬೇಕಾದರೆ ಈ ಪ್ರವೃತ್ತಿ ಹೊರಟುಹೋಗಬೇಕು. ನಾವು ಮೇಲೇಳಬೇಕಾದರೆ ಪಾಶ್ಚಾತ್ಯರಿಂದ ಹಲವು ವಿಷಯಗಳನ್ನು ಕಲಿಯಬೇಕಾಗಿದೆ ಎನ್ನುವುದನ್ನು ಮರೆಯಬಾರದು. ನಾವು ಪಾಶ್ಚಾತ್ಯರಿಂದ ಅವರ ಕಲೆ ಮತ್ತು ವಿಜ್ಞಾನಗಳನ್ನು ಕಲಿಯಬೇಕು. ಅವರು ಧರ್ಮ ಮತ್ತು ಅಧ್ಯಾತ್ಮವನ್ನು ಅರಿತು ಅರಗಿಸಿಕೊಳ್ಳುವುದಕ್ಕೆ ನಮ್ಮಲ್ಲಿಗೆ ಬರಬೇಕು. ಹಿಂದೂಗಳಾದ ನಾವು ಜಗತ್ತಿಗೇ ಗುರುಗಳು ಎಂಬುದನ್ನು ನಂಬಬೇಕು. ರಾಜಕೀಯ ಹಕ್ಕು ಮುಂತಾದ ವಿಷಯಗಳ ಬಗ್ಗೆ ನಾವು ಇಲ್ಲಿ ಗದ್ದಲ ಎಬ್ಬಿಸುತ್ತಿರುವೆವು. ಅದೇನೋ ಸರಿ. ಹಕ್ಕು ಸವಲತ್ತು ಮುಂತಾದವುಗಳು ಸ್ನೇಹದಿಂದ ಮಾತ್ರ ಬರ ಬಲ್ಲವು. ಹಾಗೂ ಸ್ನೇಹವು ಇಬ್ಬರು ಸರಿಸಮಾನರಲ್ಲಿ ಮಾತ್ರ ಸಾಧ್ಯ. ಇಬ್ಬರಲ್ಲಿ ಒಬ್ಬ ಭಿಕ್ಷುಕನಾಗಿದ್ದರೆ ಸ್ನೇಹ ಹೇಗೆ ಸಾಧ್ಯ? ಹಾಗೆ ಹೇಳುವುದೇನೋ ಸುಲಭ. ಆದರೆ ಪರಸ್ಪರ ಸಹಕಾರವಿಲ್ಲದೆ ನಾವೆಂದಿಗೂ ಬಲಾಢ್ಯರಾಗಲಾರೆವು. ಆದ್ದರಿಂದ ಇಂಗ್ಲೆಂಡ್​ ಮತ್ತು ಅಮೆರಿಕಾ ದೇಶಗಳಿಗೆ ಹೋಗಿ-ಭಿಕ್ಷುಕರಂತಲ್ಲ ಧಾರ್ಮಿಕ ಗುರುಗಳಂತೆ-ಎನ್ನುತ್ತೇವೆ. ಕೊಟ್ಟು ತೆಗೆದುಕೊಳ್ಳುವ ನಿಯಮವನ್ನು ನಮ್ಮ\break ಶಕ್ತಿಗನುಗುಣವಾಗಿ ಉಪಯೋಗಿಸಬೇಕು. ಈ ಜೀವನದಲ್ಲಿ ನಮ್ಮನ್ನು ಸುಖಿಗಳನ್ನಾಗಿ ಮಾಡುವ ಮಾರ್ಗವನ್ನು ನಾವು ಏಕೆ ಅವರಿಗೆ ತೋರಬಾರದು? ಎಲ್ಲಕ್ಕಿಂತ ಮಿಗಿಲಾಗಿ ಮಾನವಕೋಟಿಯ ಹಿತಕ್ಕೆ ಯತ್ನಿಸಿ; ನಿಮ್ಮ ಕ್ಷುದ್ರ ಸಾಂಪ್ರದಾಯಿಕತೆಯ ಹೆಮ್ಮೆಯನ್ನು ತ್ಯಜಿಸಿ. ಸಾವು ಎಲ್ಲರಿಗೂ ನಿಶ್ಚಯ. ಅತ್ಯದ್ಭುತವಾದ ಐತಿಹಾಸಿಕ ಘಟನೆಯನ್ನು ಗಮನದಲ್ಲಿಡಿ: ಜಗತ್ತಿನ ರಾಷ್ಟ್ರಗಳೆಲ್ಲ ಭಾರತದ ಪದತಳದಲ್ಲಿ ಮೌನವಾಗಿ ಕುಳಿತು, ಅದರ ಶಾಸ್ತ್ರಗಳಲ್ಲಿರುವ ಸನಾತನ ತತ್ತ್ವಗಳನ್ನು ಕಲಿಯಬೇಕಾಗಿದೆ. ಭಾರತವು ನಾಶವಾಗುವುದಿಲ್ಲ, ಚೈನಾ ನಾಶವಾಗುವುದಿಲ್ಲ. ಜಪಾನ್​ ನಾಶವಾಗುವುದಿಲ್ಲ. ಆದ್ದರಿಂದ ನಮ್ಮ ಮೂಲಭಿತ್ತಿ ಅಧ್ಯಾತ್ಮ ಎಂಬುದನ್ನು ನಾವು ಯಾವಾಗಲೂ ನೆನಪಿನಲ್ಲಿಡಬೇಕು. ತನ್ನ ಧರ್ಮವು ಶುದ್ಧ ಅಧ್ಯಾತ್ಮ ಎಂದು ಯಾವ ಹುಡುಗ ನಂಬುವುದಿಲ್ಲವೋ ಅವನು ಹಿಂದುವಲ್ಲ. ಹಿಂದೆ ಕಾಶ್ಮೀರದಲ್ಲಿ ಒಬ್ಬ ಮಹಮ್ಮದೀಯ ಮುದುಕಿಯೊಂದಿಗೆ ಮಾತನಾಡುತ್ತಿದ್ದಾಗ “ನಿನ್ನದು ಯಾವ ಧರ್ಮ?” ಎಂದು ಕೇಳಿದೆ “ದೇವರಿಗೆ ಜಯವಾಗಲಿ, ಅವನ ದಯೆಯಿಂದ ನಾನು ಮಹಮ್ಮದೀಯ ಧರ್ಮಕ್ಕೆ ಸೇರಿದವಳು” ಎಂದಳು. ಅನಂತರ ಒಬ್ಬ ಹಿಂದೂವನ್ನು ನಿಮ್ಮ ಧರ್ಮ ಯಾವುದು ಎಂದು ಕೇಳಿದೆ. ಅವನು ಸುಮ್ಮನೆ “ನಾನು ಹಿಂದೂ” ಎಂದ. ಕಠೋಪನಿಷತ್ತಿನಲ್ಲಿ ಬರುವ ‘ಶ್ರದ್ಧೆ’ ಎಂಬ ಅಪೂರ್ವ ಪದ ನನಗೆ ಜ್ಞಾಪಕಕ್ಕೆ ಬರುವುದು. ನಚಿಕೇತನ ಜೀವನದಲ್ಲಿ ಜ್ವಲಂತ ಶ್ರದ್ಧೆಗೆ ಒಂದು ಉದಾಹರಣೆ ದೊರಕುವುದು. ಆತ್ಮಶ್ರದ್ಧೆಯನ್ನು ಸಾರುವುದೇ ನನ್ನ ಜೀವನದ ಧ್ಯೇಯ. ಶ್ರದ್ಧೆಯೇ ಮಾನವ ಕೋಟಿಯ ಮತ್ತು ಎಲ್ಲಾ ಧರ್ಮಗಳ ಅತ್ಯಂತ ಶಕ್ತಿಪೂರ್ಣವಾದ ಅಂಶ ಎಂಬುದನ್ನು ಮತ್ತೊಮ್ಮೆ ಹೇಳುತ್ತೇನೆ.

ಮೊದಲು ಆತ್ಮಶ್ರದ್ಧೆ ಇರಲಿ. ಒಬ್ಬನು ಸಣ್ಣಗುಳ್ಳೆಯಾಗಿ ಮತ್ತೊಬ್ಬನು ಪರ್ವತಾಕಾರದ ಅಲೆಯಾಗಿರಬಹುದು, ಆದರೆ ಗುಳ್ಳೆ ಮತ್ತು ಆ ಬೃಹದಾಕಾರದ ಅಲೆ ಎರಡರ ಹಿಂದೆಯೂ ಒಂದೇ ಅನಂತ ಸಾಗರವಿದೆ ಎಂಬುದನ್ನು ತಿಳಿದುಕೊಳ್ಳಿ. ಎಲ್ಲರಿಗೂ ಭರವಸೆ ಇದೆ, ಎಲ್ಲರಿಗೂ ಮೋಕ್ಷವಿದೆ. ಪ್ರತಿಯೊಬ್ಬನೂ ನಿಧಾನವಾಗಿಯೋ, ವೇಗವಾಗಿಯೋ, ಮಾಯಾಬಂಧನದಿಂದ ಪಾರಾಗಬೇಕಾಗಿದೆ. ಇದೇ ನಾವು ಮಾಡಬೇಕಾದ ಪ್ರಥಮ ಕರ್ತವ್ಯ. ಅನಂತ ಭರವಸೆಯಿಂದ ಅನಂತ ಸ್ಫೂರ್ತಿ ಜನಿಸುವುದು. ಆ ಶ್ರದ್ಧೆ ನಮ್ಮಲ್ಲಿ ಉದಯಿ\-ಸಿದರೆ, ವ್ಯಾಸ-ಅರ್ಜುನರ ಕಾಲದ ರಾಷ್ಟ್ರೀಯ ಶ್ರೇಷ್ಠತೆಯನ್ನು ನಾವು ಪುನಃ ಪಡೆಯಬಹುದು. ಮಾನವ ಕೋಟಿಯ ಭವ್ಯತಮ ಸಿದ್ಧಾಂತವನ್ನು ಸಾರಿದ ಸಮಯ ಅದು. ಆಧ್ಯಾತ್ಮಿಕ ಭಾವನೆ ಮತ್ತು ದರ್ಶನಗಳಲ್ಲಿ ನಾವು ಇಂದು ತುಂಬ ಹಿಂದೆ ಬಿದ್ದಿರುವೆವು. ಭರತಖಂಡದಲ್ಲಿ ಹಿಂದೆ ಬೇಕಾದಷ್ಟು ಅಧ್ಯಾತ್ಮವಿತ್ತು. ಈ ಆಧ್ಯಾತ್ಮಿಕ ಮಹಿಮೆಯಿಂದಲೇ ಭಾರತವು ಆಗ ಅಸ್ತಿತ್ವದಲ್ಲಿದ್ದ ಜಗತ್ತಿನ ಎಲ್ಲ ಜನಾಂಗಗಳಲ್ಲಿಯೂ ಅತಿ ದೊಡ್ಡ ರಾಷ್ಟ್ರವಾಗಿತ್ತು. ಸಂಪ್ರದಾಯವನ್ನು ನಂಬಬಹುದಾದರೆ, ಅಂತಹ ಸುದಿನ ನಮಗೆ ಮತ್ತೊಮ್ಮೆ ಬರುವುದು ಮತ್ತು ಅದು ನಿಮ್ಮನ್ನು ಅವಲಂಬಿಸಿದೆ. ವಂಗದೇಶದ ತರುಣರೇ, ಶ‍್ರೀಮಂತರನ್ನು ನೆಚ್ಚಬೇಡಿ. ಜಗತ್ತಿನಲ್ಲಿ ಅದ್ಭುತ ಕೆಲಸಗಳನ್ನೆಲ್ಲಾ ಸಾಧಿಸಿದವರು ದೀನರು. ವಂಗದೇಶದ ದೀನರೇ, ಮುಂದೆ ಬನ್ನಿ, ನೀವೂ ಏನನ್ನು ಬೇಕಾದರೂ ಸಾಧಿಸಬಲ್ಲಿರಿ, ಎಲ್ಲವನ್ನೂ ಸಾಧಿಸಲೇಬೇಕು. ನೀವು ದೀನರಾದರೂ ಹಲವರು ನಿಮ್ಮನ್ನು ಅನುಸರಿಸುವರು. ಸ್ಥಿರತೆಯನ್ನು ಪಡೆಯಿರಿ. ಎಲ್ಲಕ್ಕಿಂತ ಹೆಚ್ಚಾಗಿ ಪರಿಶುದ್ಧರಾಗಿ, ನಿಷ್ಠಾವಂತರಾಗಿ. ನಿಮ್ಮ ಭವಿಷ್ಯದಲ್ಲಿ ನಂಬಿಕೆಯಿಡಿ. ವಂಗದೇಶದ ತರುಣರೇ, ಇದನ್ನು ಗಮನಿಸಿ, ನೀವು ಭರತಖಂಡದ ವಿಮೋಚನೆ ಮಾಡಬೇಕಾಗಿದೆ, ನಿಮಗೆ ಇದರಲ್ಲಿ ನಂಬಿಕೆ ಇರಬಹುದು, ಇಲ್ಲದೆ ಇರಬಹುದು. ಇಂದೋ, ನಾಳೆಯೋ ಅದು ನೆರವೇರುತ್ತದೆ ಎಂದು ಭಾವಿಸಬೇಡಿ. ನನ್ನ ದೇಹವನ್ನು ಮತ್ತು ಆತ್ಮವನ್ನು ನಂಬುವಂತೆ ಇದನ್ನೂ ನಾನು ನಂಬುತ್ತೇನೆ. ವಂಗ ತರುಣರೇ, ಅದಕ್ಕಾಗಿಯೇ ನನ್ನ ಹೃದಯ ನಿಮ್ಮನ್ನು ಪ್ರೀತಿಸುವುದು. ದುಡ್ಡಿಲ್ಲದ ನಿಮ್ಮ ಮೇಲೆ ಅದು ನಿಂತಿದೆ; ನೀವು ದೀನರು, ಅದಕ್ಕೆ ನೀವು ಅದನ್ನು ಸಾಧಿಸುವಿರಿ. ನಿಮ್ಮಲ್ಲಿ ಏನೂ ಇಲ್ಲ, ಅದಕ್ಕೇ ನೀವು ನಿಷ್ಠಾವಂತರಾಗುವಿರಿ. ನೀವು ನಿಷ್ಠಾವಂತರಾಗುವುದರಿಂದಲೇ ಸರ್ವವನ್ನೂ ತ್ಯಾಗಮಾಡಬಲ್ಲಿರಿ. ಈಗ ನಾನು ಇದನ್ನೇ ನಿಮಗೆ ಹೇಳುತ್ತಿರುವುದು. ಮತ್ತೊಮ್ಮೆ ನಾನು ನಿಮಗೆ ಇದನ್ನು ಹೇಳುತ್ತೇನೆ. ಇದೇ ನಿಮ್ಮ ಜೀವನದ ಉದ್ದೇಶ. ನೀವು ಯಾವ ತತ್ತ್ವವನ್ನಾದರೂ ತೆಗೆದುಕೊಳ್ಳಿ ಚಿಂತೆ ಇಲ್ಲ. ಮಾನವ ಕೋಟಿಯ ಪರಿಪೂರ್ಣತೆಯಲ್ಲಿ ಸೌಹಾರ್ದಪೂರ್ಣ ಮತ್ತು ಸ್ನೇಹ ಪೂರಿತ ಅನಂತ ವಿಶ್ವಾಸವು ಭಾರತದಲ್ಲೆಲ್ಲ ಪಸರಿಸಿದೆ ಎಂಬುದನ್ನು ನಿಮಗೆ ಸಪ್ರಮಾಣವಾಗಿ ತೋರಬಲ್ಲೆ. ಈ ವಿಶ್ವಾಸ ಭರತಖಂಡದಲ್ಲೆಲ್ಲಾ ಪ್ರಚಾರವಾಗಲಿ.

