
\chapter{ಅಮೆರಿಕ ವೃತ್ತಪತ್ರಿಕೆಯ ವರದಿಗಳು}

\begin{center}
\textbf{ಹಿಂದೂದೇಶ, ಅದರ ಧರ್ಮ ಮತ್ತು ಆಚಾರಗಳು\supskpt{\footnote{* Complete Work of Swamy Vivekananda, Vol. III P.465}}}
\end{center}

\begin{center}
\textbf{(ಸೇಲಂ ಈವನಿಂಗ್​ ನ್ಯೂಸ್​, ಆಗಸ್ಟ ೨೯, ೧೮೯೩)}
\end{center}

\vskip -0.4cm

ನಿನ್ನೆ ಮಧ್ಯಾಹ್ನದ ಮೇಲೆ ತುಂಬಾ ಶೆಖೆಯಿದ್ದರೂ \enginline{Thoughts and work club} ಗೆ ಸೇರಿದ ಅನೇಕ ಜನರು ತಮ್ಮ ಅತಿಥಿಗಳೊಡನೆ ಹಿಂದೂ ಸಂನ್ಯಾಸಿ ಸ್ವಾಮಿ ವಿವೇಕಾನಂದರನ್ನು ನೋಡಲು \enginline{Wesley Chapel} ನಲ್ಲಿ ನೆರೆದಿದ್ದರು. ಈಗ ಸ್ವಾಮೀಜಿ ಈ ದೇಶದಲ್ಲಿ ಸಂಚರಿಸುತ್ತಿರುವರು. ವೇದಗಳ ಬೋಧನೆಯ ಮೇಲೆ ನಿಂತಿರುವ ಹಿಂದೂಧರ್ಮದ ವಿಷಯವನ್ನು ಅವರ ಬಾಯಿಂದ ಕೇಳಲು ಜನರು ಅಲ್ಲಿ ಸೇರಿದ್ದರು. ಹಿಂದೂದೇಶದ ವರ್ಣಗಳು ಕೇವಲ ಸಾಮಾಜಿಕ ವಿಭಾಗಗಳೆಂದೂ ಅವಕ್ಕೂ ಧರ್ಮಕ್ಕೂ ಯಾವ ಸಂಬಂಧವೂ ಇಲ್ಲವೆಂದೂ ಅವರು ತಿಳಿಸಿದರು.

ಜನಸಾಧಾರಣರಲ್ಲಿರುವ ಬಡತನವನ್ನು ಅವರು ಹೆಚ್ಚಾಗಿ ವಿವರಿಸಿದರು. ಉತ್ತರ ಅಮೆರಿಕ ದೇಶಕ್ಕಿಂತ ಸಣ್ಣದಾಗಿರುವ ಇಂಡಿಯಾದೇಶದಲ್ಲಿ ಇಪ್ಪತ್ತಮೂರು ಕೋಟಿ ಜನರು ಇರುವರು ಎಂದು ಹೇಳಿದರು. ಇವರಲ್ಲಿ ಮೂರು ಕೋಟಿ ಜನರು ತಿಂಗಳಿಗೆ ಸುಮಾರು ಐವತ್ತು ಸೆಂಟುಗಳಿಗಿಂತ ಕಡಿಮೆ ಸಂಪಾದಿಸುವವರು ಎಂದರು. ಇನ್ನು ಕೆಲವು ವೇಳೆ ಒಂದು ಜಿಲ್ಲೆಯ ಜನರು ಒಂದು ಬಗೆಯ ತಿನ್ನಬಹುದಾದಂತಹ ಹೂವನ್ನು ಬೇಯಿಸಿ ಅದನ್ನು ತಿಂದು ತಮ್ಮ ಬಾಳನ್ನು ಸಾಗಿಸುವರು ಎಂದು ತಿಳಿಸಿದರು.

ಇನ್ನು ಬೇರೆ ಜಿಲ್ಲೆಗಳಲ್ಲಿ ಪುರುಷರು ಮಾತ್ರ ಅನ್ನವನ್ನು ಊಟಮಾಡುವರು. ಹೆಂಗಸರು ಮತ್ತು ಮಕ್ಕಳು ಅನ್ನ ಬೇಯಿಸಿದ ಗಂಜಿಯ ನೀರನ್ನು ಕುಡಿದು ಹಸಿವನ್ನು ನೀಗಿಕೊಳ್ಳುವರು. ಏನಾದರೂ ಭತ್ತದ ಬೆಳೆ ಬರದೇ ಹೋದರೆ ಕ್ಷಾಮ ಬರುವುದು. ಅರ್ಧ ಜನರಿಗೆ ದಿನಕ್ಕೆ ಒಂದೇ ಹೊತ್ತಿನ ಊಟ ಸಿಕ್ಕುತ್ತದೆ. ಉಳಿದ ಅರ್ಧ ಜನಕ್ಕೆ ಮುಂದಿನ ಊಟ ಎಲ್ಲಿ ಹೇಗೆ ಸಿಕ್ಕುತ್ತದೆಯೋ ಗೊತ್ತಿರುವುದಿಲ್ಲ. ವಿವೇಕಾನಂದರ ದೃಷ್ಟಿಯಲ್ಲಿ ಭರತಖಂಡಕ್ಕೆ ಅತಿ ಜರೂರಾಗಿ ಬೇಕಾಗಿರುವುದು, ಹೆಚ್ಚು ಧರ್ಮವೂ ಅಲ್ಲ, ಉತ್ತಮವಾದ ಧರ್ಮವೂ ಅಲ್ಲ. ಅವರು ವಿವರಿಸಿದಂತೆ ಅವರಿಗೆ ಹೆಚ್ಚು ಹೊಟ್ಟೆಗೆ ಮತ್ತು ಬಟ್ಟೆಗೆ ಬೇಕಾಗಿದೆ. ಭರತಖಂಡದಲ್ಲಿ ದಾರಿದ್ರ್ಯದಿಂದ ದುಃಖ ಪಡುತ್ತಿರುವ ಕೋಟ್ಯಂತರ ಜನರ ವಿಷಯದಲ್ಲಿ ಅಮೆರಿಕಾ ದೇಶೀಯರಿಗೆ ಕರುಣೆ ಹುಟ್ಟುವಂತೆ ಮಾಡುವುದಕ್ಕಾಗಿ ಅವರು ಇಲ್ಲಿಗೆ ಬಂದಿರುವರು.

ತಮ್ಮ ದೇಶದ ಜನರು ಮತ್ತು ಅವರ ಧರ್ಮದ ವಿಷಯವಾಗಿ ಅವರು ಬಹಳಕಾಲ ಮಾತನಾಡಿದರು. ಅವರು ಮಾತನಾಡುತ್ತಿದ್ದಾಗ \enginline{Central Baptist Church} ನ ಡಾ~॥ ಎಫ್​.ಎ. ಗಾರ್ಡನರ್​ ಮತ್ತು ರೆವೆರೆಂಡ್​ ಎಸ್​.ಎಫ್​.ನಾಬ್ಸ್​ ಅವರು ಸ್ವಾಮಿಗಳಿಗೆ ಹಲವು ಪ್ರಶ್ನೆಗಳನ್ನು ಹಾಕಿದರು. ಸ್ವಾಮೀಜಿ ಉತ್ತರ ಕೊಡುತ್ತಾ ಅಲ್ಲಿರುವ ಮಿಷನರಿಗಳ ಬಳಿ ಬಹಳ ಒಳ್ಳೆಯ ಸಿಂದ್ಧಾಂತಗಳಿವೆ ಎಂದೂ, ಒಳ್ಳೆಯ ಅಭಿಪ್ರಾಯದಿಂದಲೇ ಅವರ ಕೆಲಸಗಳು ಪ್ರಾರಂಭವಾಗಿವೆ ಎಂದೂ; ಆದರೆ ದೇಶದ ಕೈಗಾರಿಕೆ ಅಭಿವೃದ್ಧಿಯ ವಿಷಯದಲ್ಲಿ ಅವರು ಯಾವ ಸಹಾಯವನ್ನೂ ಮಾಡಲಿಲ್ಲವೆಂದೂ ಹೇಳಿದರು. ಮುಂದುವರಿದು ಸ್ವಾಮೀಜಿ, ಅಮೆರಿಕನರು ಧರ್ಮವನ್ನು ಬೋಧಿಸಲು ಭಾರತಕ್ಕೆ ಮಿಷನರಿಗಳನ್ನು ಕಳುಹಿಸುವ ಬದಲು ಭಾರತೀಯರಿಗೆ ಕೈಗಾರಿಕೆಗಳನ್ನು ಕಲಿಸಲು ತಜ್ಞರನ್ನು ಕಳುಹಿಸುವುದು ಒಳ್ಳೆಯದು ಎಂದು ಹೇಳಿದರು.

ಜನರ ಕಷ್ಟಕಾಲದಲ್ಲಿ ಪಾದ್ರಿಗಳು ಸಹಾಯ ಮಾಡಲಿಲ್ಲವೆ, ಮತ್ತು ಹಲವು ಶಾಲೆಗಳನ್ನು ಸ್ಥಾಪಿಸಿ ವ್ಯಾವಹಾರಿಕ ರೀತಿಯಲ್ಲಿ ಅಲ್ಲಿಯ ಜನರಿಗೆ ಸಹಾಯ ಮಾಡಲಿಲ್ಲವೆ ಎಂದು ಕೆಲವರು ಕೇಳಿದರು. ಉಪನ್ಯಾಸಕರು ಅದಕ್ಕೆ ಉತ್ತರವಾಗಿ ಹಾಗೆ ಕೆಲವು ವೇಳೆ ಮಾಡಿರುವ\break ರೆಂದೂ, ಆದರೆ ಅದೇನು ಅವರಿಗೆ ಅಷ್ಟೊಂದು ಕೀರ್ತಿಯನ್ನು ತರುವಂತಹ ಕೆಲಸವಲ್ಲ\break ವೆಂದೂ ಹೇಳಿದರು. ಏಕೆಂದರೆ ಅಲ್ಲಿರುವ ಕಾನೂನಿನ ಪ್ರಕಾರ ಅಂತಹ ಕಾಲದಲ್ಲಿ ಜನರ ಮೇಲೆ ಯಾವ ಪ್ರಭಾವವನ್ನೂ ಬೀರಕೂಡದೆಂದು ನಿರ್ಬಂಧವಿದೆ ಎಂದು ಹೇಳಿದರು.

ಭರತಖಂಡದಲ್ಲಿರುವ ಸ್ತ್ರೀಯರ ದುಃಸ್ಥಿತಿಯನ್ನು ವಿವರಿಸಿದರು. ಹಿಂದೂ ಗಂಡಸರಿಗೆ ಅಲ್ಲಿಯ ಹೆಂಗಸರ ಮೇಲೆ ಅಷ್ಟೊಂದು ಗೌರವವಿದೆಯೆಂದೂ, ಅವರು ಮನೆಯಿಂದ ಹೊರಗೆ ಹೋಗುವುದಕ್ಕೆ ಹೆಂಗಸರಿಗೆ ಅವಕಾಶವನ್ನೇ ಕೊಡುವುದಿಲ್ಲವೆಂದೂ ಹೇಳಿದರು. ಗಂಡನು ಕಾಲವಾದರೆ ಅವನೊಡನೆ ಹೆಂಡತಿಯು ಚಿತೆಗೇರುವ ಹಳೆಯ ಪದ್ಧತಿಯನ್ನೂ ಅವರು ವಿವರಿಸಿದರು. ಅವರು ಹಾಗೆ ಮಾಡುತ್ತಿದ್ದುದಕ್ಕೆ ಗಂಡನ ಮೇಲಿರುವ ಪ್ರೇಮವೇ ಕಾರಣವೆಂದರು. ಅವರು ತಮ್ಮ ಗಂಡನನ್ನು ಬಿಟ್ಟಿರಲು ಇಚ್ಛೆಪಡುತ್ತಿರಲಿಲ್ಲ ಎಂದರು. ಮದುವೆಯಲ್ಲಿ ಒಂದಾದ ಅವರು ಸಾವಿನಲ್ಲಿಯೂ ಒಂದಾಗಿ ಇರಬಯಸಿದ್ದರು ಎಂದರು.

ವಿಗ್ರಹಾರಾಧನೆ ಮತ್ತು ಜಗನ್ನಾಥನ ರಥದ ಗಾಲಿಗಳ ಅಡಿಯಲ್ಲಿ ಸಿಕ್ಕಿಕೊಂಡು ಪ್ರಾಣಬಿಡುವುದು ಮುಂತಾದ ವಿಷಯಗಳನ್ನು ಕುರಿತು ಅವರನ್ನು ಕೇಳಲಾಯಿತು. ಸ್ವಾಮೀಜಿ ಎರಡನೆಯದಕ್ಕೆ ಹಿಂದೂಗಳು ಕಾರಣರಲ್ಲವೆಂದೂ, ಕೆಲವು ಧರ್ಮಾಂಧರು, ಅದರಲ್ಲಿಯೂ ಕುಷ್ಠರೋಗಿಗಳು ಜೀವನದ ಮೇಲಿನ ಜಿಗುಪ್ಸೆಯಿಂದ ಹೀಗೆ ಮಾಡು\break ತ್ತಿದ್ದದು ಎಂತಲೂ ಉತ್ತರ ಹೇಳಿದರು.

ಉಪನ್ಯಾಸಕರು ತಾವು ಅಮೆರಿಕಾ ದೇಶದಲ್ಲಿ ಏನು ಮಾಡಬೇಕೆಂದಿರುವೆ ಎಂಬುದನ್ನು ವಿವರಿಸಿದರು: ಕೈಗಾರಿಕೆಗಳಿಗೆ ಸಹಾಯಕವಾಗುವಂತಹ ಸಂನ್ಯಾಸಿಗಳನ್ನು ತರಬೇತುಗೊಳಿಸು\break ವುದೇ ತಮ್ಮ ಉದ್ದೇಶವೆಂದರು. ಸಂನ್ಯಾಸಿಗಳು ಜನರಿಗೆ ಕೈಗಾರಿಕಾ ಕುಶಲತೆಯನ್ನು ಬೋಧಿಸಿ ಅವರ ಆರ್ಥಿಕ ಮಟ್ಟವನ್ನು ಉತ್ತಮಗೊಳಿಸಲು ಇದು ಸಹಾಯವಾಗುವುದು ಎಂದರು.

ಇಂದಿನ ಮಧ್ಯಾಹ್ನ ವಿವೇಕಾನಂದರು, ಅಮೆರಿಕಾ ದೇಶದ ಮಕ್ಕಳಾಗಲೀ, ಯುವಕ\break ರಾಗಲೀ ಕೇಳಲು ಕುತೂಹಲಿಗಳಾಗಿದ್ದರೆ, “ಭರತಖಂಡದ ಮಕ್ಕಳು” ಎಂಬ ವಿಷಯವನ್ನು, ಕುರಿತು ಮಾತನಾಡುವರು. ಅವರ ಉಪನ್ಯಾಸವನ್ನು ನಾರ್ತ್​ ಸ್ಟ್ರೀಟ್​ನ ೧೬೬ನೇ ನಂಬರ್​ ಮನೆಯಲ್ಲಿ ಕೇಳಬಹುದು. ಶ‍್ರೀಮತಿ ವುಡ್ಸ್​ ಅವರು ತಮ್ಮ ಉದ್ಯಾನವನ್ನು ಇದಕ್ಕಾಗಿ ಸಂತೋಷದಿಂದ ಬಿಟ್ಟುಕೊಟ್ಟಿರುವರು. ನೋಡುವುದಕ್ಕೆ ಸ್ವಾಮೀಜಿ ಮನೋಹರವಾಗಿರು\break ವರು. ಸ್ವಲ್ಪ ಕಪ್ಪು, ಆದರೂ ಸುಂದರವಾಗಿರುವರು. ಒಂದು ದೊಡ್ಡ ಹಳದಿಯ ನಿಲುವಂಗಿ\break ಯನ್ನು ತೊಟ್ಟುಕೊಂಡಿರುವರು. ಸೊಂಟದ ಸಮೀಪದಲ್ಲಿ ಅದನ್ನು ಒಂದು ದಾರದಿಂದ ಕಟ್ಟಿಕೊಂಡಿರುವರು. ಒಂದು ಹಳದಿಯ ರುಮಾಲು ತಲೆಯ ಮೇಲೆ ಇದೆ. ಅವರು ಸಂನ್ಯಾಸಿಯಾಗಿರುವುದರಿಂದ ಅವರಿಗೆ ಯಾವ ಜಾತಿಯೂ ಇಲ್ಲ. ಅವರು ಯಾರೊಡನೆ ಬೇಕಾದರೂ ಊಟ ಮಾಡುವರು.

\begin{center}
\textbf{(ಡೈಲಿ ಗೆಜೆಟ್​ ಆಗಸ್ಟ ೨೯, ೧೮೯೩)}
\end{center}

\vskip -0.35cm

ಇಂಡಿಯಾ ದೇಶದ ರಾಜ ಸ್ವಾಮಿ ವಿವೇಕಾನಂದರು\footnote{* ಅಮೆರಿಕಾ ಸುದ್ದಿಗಾರರು ಸ್ವಾಮಿ ವಿವೇಕಾನಂದರ ಹೆಸರಿಗೆ ಹಿಂದೆ ರಾಜ, ಬ್ರಾಹ್ಮಣ, ಪುರೋಹಿತ ಮುಂತಾದ ಹೆಸರುಗಳನ್ನು ಹಾಕುತ್ತಿದ್ದರು} ಸೇಲಂನಲ್ಲಿರುವ \enginline{Thought and works} ಕ್ಲಬ್ಬಿನ ಅತಿಥಿಗಳಾಗಿದ್ದರು. ನಿನ್ನೆ ಮಧ್ಯಾಹ್ನ ಸಭೆ ವೆಸ್ಲಿ ಚರ್ಚಿನಲ್ಲಿ ಸೇರಿತ್ತು. ಹಲವು ಜನ ಮಾನ್ಯ ಮಹಿಳೆಯರು ಮತ್ತು ಮಹನೀಯರು ಪ್ರಖ್ಯಾತರಾದ ವಿವೇಕಾನಂದರಿಗೆ ಅಮೆರಿಕಾ ರೀತಿಯಲ್ಲಿ ಹಸ್ತ ಲಾಘವವನ್ನು ಕೊಟ್ಟರು. ಅವರು ಒಂದು ಕಿತ್ತಲೆ ಬಣ್ಣದ ನಿಲುವಂಗಿಯನ್ನು ತೊಟ್ಟಿದ್ದರು. ಅದರ ಮೇಲೆ ಕೆಂಪು ನಡುಪಟ್ಟಿಯನ್ನು ಬಿಗಿದಿದ್ದರು. ಹಳದಿಯ ರುಮಾಲನ್ನು ಕಟ್ಟಿಕೊಂಡಿದ್ದರು. ಆ ಪೇಟದ ಕುಚ್ಚು ಹೆಗಲಿನ ಮೇಲೆ ಆಡುತ್ತಿತ್ತು. ಸ್ವಾಮೀಜಿಯವರು ಅದನ್ನು ಕರವಸ್ತ್ರವಾಗಿ ಉಪಯೋಗಿಸುತ್ತಿದ್ದರು. ಕಾಲಿಗೆ ಕಾಂಗ್ರೆಸ್​ ಬೂಟ್ಸುಗಳನ್ನು ಹಾಕಿಕೊಂಡಿದ್ದರು.

ಭರತಖಂಡದ ಜನರ ಧರ್ಮ ಮತ್ತು ಅವರ ಆಚಾರ ವ್ಯವಹಾರಗಳ ವಿಷಯವಾಗಿ ಅವರು ದೀರ್ಘ ಭಾಷಣ ಮಾಡಿದರು. ಅವರು ಮಾತನಾಡುತ್ತಿದ್ದಾಗ ಸೆಂಟ್ರಲ್​ ಬ್ಯಾಪ್ಟಿಸ್ಟ್​ ಚರ್ಚಿನ ಡಾಕ್ಟರ್​ ಎಫ್​.ಎ. ಗಾರ್ಡನರ್​ ಮತ್ತು ರೆವೆರೆಂಡ್​ ಎಸ್​.ಎಫ್​. ನಾಬ್ಸ್​ ಅವರು ಅನೇಕ ಕ್ಲಿಷ್ಟ ಪ್ರಶ್ನೆಗಳನ್ನು ಹಾಕಿದರು. ಭರತಖಂಡದಲ್ಲಿರುವ ಪಾದ್ರಿಗಳ ಬಳಿ ಒಳ್ಳೆಯ ಸಿದ್ಧಾಂತಗಳಿವೆ, ಮತ್ತು ಅವರು ಒಳ್ಳೆಯ ದೃಷ್ಟಿಯಿಂದಲೇ ಕೆಲಸಮಾಡುತ್ತಿರುವರು. ಆದರೆ ಜನರನ್ನು ಕೈಗಾರಿಕೆಯಲ್ಲಿ ತರಬೇತು ಮಾಡುವುದಕ್ಕೆ ಯಾವ ಪ್ರಯತ್ನಗಳನ್ನೂ ಅವರು ಮಾಡಿಲ್ಲವೆಂದು ಹೇಳಿದರು. ಅಮೆರಿಕಾ ದೇಶದವರು ಅಲ್ಲಿಯ ಜನರಿಗೆ ಧಾರ್ಮಿಕ ವಿಷಯಗಳನ್ನು ಹೇಳಲು ಯಾರನ್ನಾದರೂ ಕಳುಹಿಸುವ ಬದಲು ಕೈಗಾರಿಕೆಯ ವಿದ್ಯೆಯನ್ನು ಕಲಿಸಬಲ್ಲಂತಹ ಕೆಲವರನ್ನು ಕಳುಹಿಸುವುದು ಮೇಲು ಎಂದರು.

ಅಲ್ಲಿರುವ ಸ್ತ್ರೀಪುರುಷರ ವಿಷಯವನ್ನು ಕುರಿತು ಮಾತನಾಡುವಾಗ ಇಂಡಿಯಾ ದೇಶದ ಗಂಡಂದಿರು ಎಂದಿಗೂ ಸುಳ್ಳು ಹೇಳುವವರಲ್ಲ; ತಮ್ಮ ಹೆಂಡತಿಯನ್ನು\break ಹಿಂಸಿಸುವವರಲ್ಲ ಎಂದು ಹೇಳಿದರು. ಅವರು ಮಾಡದೇ ಇರುವ ಇನ್ನೂ ಹಲವಾರು ಪಾಪಕೃತ್ಯಗಳನ್ನು ಕುರಿತು ಹೇಳಿದರು.

ಕಷ್ಟಕಾಲದಲ್ಲಿ ಪಾದ್ರಿಗಳು ಇಂಡಿಯಾ ದೇಶದ ಜನರಿಗೆ ಸಹಾಯ ಮಾಡಲಿಲ್ಲವೆ, ಉಪಾಧ್ಯಾಯರಿಗೆ ಶಿಕ್ಷಣವನ್ನು ಕಲಿಸುವ ಟ್ರೈನಿಂಗ್​ ಸ್ಕೂಲು ಮುಂತಾದವುಗಳನ್ನು ನಡೆಸ\break ಲಿಲ್ಲವೆ ಎಂದು ಕೇಳಿದಾಗ, ಹೌದು ಕೆಲವು ವೇಳೆ ಅವರು ಹಾಗೆ ಮಾಡಿದರು, ಆದರೆ ತಮಗೆ ಒಳ್ಳೆಯ ಹೆಸರು ಬರುವ ರೀತಿಯಲ್ಲಿ ಅದು ಅಲ್ಲ ಎಂದರು. ಅಂತಹ ಸಮಯದಲ್ಲಿ ಜನರ ಮೇಲೆ ತಮ್ಮ ಪ್ರಭಾವವನ್ನು ಬೀರಲು ಕಾನೂನು ನಿಷೇಧಿಸುವುದು ಎಂದರು.

ಭರತಖಂಡದ ನಾರಿಯರ ದುಃಸ್ಥಿತಿಯನ್ನು ಕುರಿತು ಅವರು ವಿವರಿಸಿದರು. ಹಿಂದೂ ದೇಶದಲ್ಲಿರುವ ಪುರುಷರಿಗೆ ತಮ್ಮ ಸ್ತ್ರೀಯರ ವಿಷಯದಲ್ಲಿ ಅಷ್ಟೊಂದು ಗೌರವವಿರುವುದ\break ರಿಂದಲೇ, ಅವರು ಸ್ತ್ರೀಯರಿಗೆ ಹೊರಗೆ ಹೋಗಲು ಕೂಡ ಅವಕಾಶವನ್ನು ಕೊಡುವುದಿಲ್ಲ ಎಂದರು. ಹಿಂದೂ ನಾರಿಯರನ್ನು ಅಷ್ಟೊಂದು ಗೌರವದಿಂದ ನೋಡುವುದರಿಂದ ಅವರಿಗೆ ಇತರರೊಡನೆ ಬೆರೆಯಲು ಅವಕಾಶವನ್ನೇ ಕೊಟ್ಟಿಲ್ಲವೆಂದರು. ಗಂಡನು ಕಾಲವಾದ ಮೇಲೆ ಅವನೊಡನೆ ಅವನ ಹೆಂಡತಿಯನ್ನು ಜೀವ ಸಹಿತ ಸುಡುವ ಹಳೆಯ ಆಚಾರವನ್ನು ಕುರಿತು, ಅವರು ತಮ್ಮ ಗಂಡಂದಿರನ್ನು ಅಷ್ಟು ಪ್ರೀತಿಸುತ್ತಿದ್ದರೆಂದೂ, ಅವರಿಲ್ಲದೆ ಪ್ರಪಂಚದಲ್ಲಿ ತಮಗೆ ಬಾಳಲು ಇಷ್ಟ ವಿರಲಿಲ್ಲವೆಂದೂ ತಿಳಿಸಿದರು. ಅವರು ಮದುವೆಯಾದಾಗ ಒಟ್ಟಿಗೆ ಬಂದುದರಿಂದ ಮರಣದಲ್ಲಿಯೂ ಒಟ್ಟಿಗೆ ಹೋಗಬೇಕೆಂದು ಆಶಿಸಿದರು ಎಂದರು.

ವಿಗ್ರಹಾರಾಧನೆ ಮತ್ತು ಜಗನ್ನಾಥನ ಚಕ್ರದ ಮುಂದೆ ಬಿದ್ದು ಸಾಯುವುದು ಈ ವಿಚಾರಗಳನ್ನು ಪ್ರಶ್ನಿಸಿದಾಗ, ಜನ ಚಕ್ರದ ಕೆಳಗೆ ಸಿಕ್ಕಿ ಸತ್ತರೆ ಅದಕ್ಕಾಗಿ ಹಿಂದೂಗಳನ್ನು ದೂರಕೂಡದೆಂದೂ, ಹಾಗೆ ಮಾಡುವವರು ಕೆಲವು ಧರ್ಮಾಂಧರು ಮತ್ತು ಕುಷ್ಠರೋಗಿಗಳು ಎಂದರು.

ವಿಗ್ರಹಾರಾಧನೆಯ ವಿಷಯವನ್ನು ಕುರಿತು, ಕ್ರೈಸ್ತರು ತಾವು ಪ್ರಾರ್ಥಿಸುವಾಗ ಏನನ್ನು ಆಲೋಚಿಸುತ್ತಾರೆ ಎಂದು ಕೇಳಿದರು. ಕೆಲವರು ತಾವು ಚರ್ಚನ್ನು ಕುರಿತು ಆಲೋಚಿಸುತ್ತೇವೆ ಎಂತಲೂ, ಮತ್ತೆ ಕೆಲವರು ದೇ-ವ-ರು ಎಂಬ ಅಕ್ಷರಗಳನ್ನು ಕುರಿತು ಯೋಚಿಸುತ್ತೇವೆ ಎಂತಲೂ ಹೇಳಿದರು. ಹಿಂದೂಗಳಾದರೋ ವಿಗ್ರಹಗಳು ಕುರಿತು ಯೋಚಿಸುತ್ತಾರೆ ಎಂದರು. ಪಾಪ, ಸಾಧಾರಣ ಜನರಿಗೆ ವಿಗ್ರಹಗಳು ಆವಶ್ಯಕ. ಹಿಂದಿನ ಕಾಲದಲ್ಲಿ ಧರ್ಮವು ಆಗತಾನೆ ಪ್ರಾರಂಭವಾದಾಗ ಸ್ತ್ರೀಯರು ಪ್ರತಿಭೆಗೆ ಮತ್ತು ಮಾನಸಿಕ ಶಕ್ತಿಗೆ ಬಹಳ ಪ್ರಖ್ಯಾತರಾಗಿದ್ದರೆಂದು ಹೇಳಿದರು. ಹೀಗೆ ಇದ್ದರೂ ಆಧುನಿಕ ಕಾಲದ ಸ್ತ್ರೀಯರು ಅಧೋಗತಿಗೆ ಇಳಿದಿರುವರು ಎಂಬುದನ್ನು ಸ್ವಾಮೀಜಿ ಒಪ್ಪಿಕೊಂಡಂತೆ ತೋರಿತು. ತಿನ್ನುವುದು, ಕುಡಿಯುವುದು, ಹರಟೆ ಮತ್ತು ಪರರ ವಿಷಯ ಚರ್ಚಿಸುವುದು, ಇವುಗಳಲ್ಲದೆ ಬೇರೆ ಏನನ್ನೂ ಅವರು ಮಾಡುತ್ತಿಲ್ಲ ಎಂದರು.

\delimiter

\begin{center}
\textbf{(ಸೇಲಂ ಈವಿನಿಂಗ್​ ನ್ಯೂಸ್​, ಸೆಪ್ಟೆಂಬರ್​ ೧, ೧೮೯೩)}
\end{center}

\vskip -0.35cm

ಈ ಊರಿನಲ್ಲಿ ಕೆಲವು ದಿನಗಳು ತಂಗಿರುವ ಭರತಖಂಡದ ವಿದ್ವಾಂಸರಾದ ಸಂನ್ಯಾಸಿಗಳು ಈಸ್ಟ್​ ಚರ್ಚ್​ನಲ್ಲಿ ಭಾನುವಾರ ಸಾಯಂಕಾಲ ೨-೨೦ಕ್ಕೆ ಮಾತನಾಡುವರು. ಕಳೆದ ಭಾನುವಾರ ಸಾಯಂಕಾಲ ಸ್ವಾಮಿ ವಿವೇಕಾನಂದರು ಆನಿಸ್ಕ್ವಾಮ್​ನಲ್ಲಿ ಇರುವ ಎಪಿಸ್​\break ಕೋಪಲ್​ ಚರ್ಚ್​ನಲ್ಲಿ ಮಾತನಾಡಿದರು. ಸ್ವಾಮೀಜಿ ಅವರಿಗೆ ಗೌರವವನ್ನು ತೋರಿದ ಹಾರ್​ವರ್ಡ್​ ಯೂನಿವರ್ಸಿಟಿಯ ಪ್ರೊಫೆಸರ್​ ರೈಟ್​ ಮತ್ತು ಅಲ್ಲಿಯ ಪಾದ್ರಿಗಳು ಅವರನ್ನು ಅಲ್ಲಿಗೆ ಸ್ವಾಗತಿಸಿದ್ದರು.

ಸೋಮವಾರ ರಾತ್ರಿ ಅವರು ಸಾರಟೋಗಾಕ್ಕೆ ಹೊರಡುವರು. ಅವರು ಅಲ್ಲಿ\break \enginline{Social Science Association} ನಲ್ಲಿ ಮಾತನಾಡುವರು. ಇಂಡಿಯಾ ದೇಶದ ಉತ್ತಮ ವಿಶ್ವವಿದ್ಯಾನಿಲಯಗಳಲ್ಲಿ ಕಲಿತ ಎಲ್ಲರಂತೆ ವಿವೇಕಾನಂದರು ಸುಲಭವಾಗಿ, ಶುದ್ಧವಾಗಿ ಇಂಗ್ಲಿಷ್​ನಲ್ಲಿ ಮಾತನಾಡುವರು. ಕಳೆದ ಮಂಗಳವಾರ ಸ್ವಾಮಿ ವಿವೇಕಾನಂದರು ಅಲ್ಲಿಯ ಹುಡುಗರಿಗೆ ಭರತಖಂಡದ ಮಕ್ಕಳ ವಿಷಯವಾಗಿ ಅವರ ಆಟಪಾಟಗಳು, ಶಾಲೆ, ಆಚಾರ ವ್ಯವಹಾರಗಳು ಮುಂತಾದ ವಿಷಯದಲ್ಲಿ ಮಾಡಿದ ಸರಳ ಉಪನ್ಯಾಸ ಬಹಳ ಪ್ರಯೋಜನಕಾರಿಯಾಗಿತ್ತು ಮತ್ತು ಸ್ವಾರಸ್ಯಕರವಾಗಿತ್ತು. ಒಂದು ಸಣ್ಣ ಹುಡುಗಿ, ತನ್ನ ಉಪಾಧ್ಯಾಯಿನಿ ತನ್ನನ್ನು ಅಷ್ಟು ಬಲವಾಗಿ ಹೊಡೆದಳೆಂದೂ, ತನ್ನ ಕೈಯೇ ಮುರಿದುಹೋಗುವಂತೆ ಇತ್ತು ಎಂಬುದನ್ನೂ ಕೇಳಿದಾಗ ಅವರ ಹೃದಯ ಕರಗಿ ಹೋಯಿತು. ವಿವೇಕಾನಂದರು ಆ ದೇಶದಲ್ಲಿ ಎಲ್ಲಾ ಸಂನ್ಯಾಸಿಗಳು ಮಾಡುವಂತೆ, ತಮ್ಮ ದೇಶದಲ್ಲಿ ಸಂಚಾರ ಮಾಡಿಕೊಂಡು ಧರ್ಮ, ಬ್ರಹ್ಮಚರ್ಯ, ಸಹೋದರಭಾವ ಮುಂತಾದುವನ್ನು ಬೋಧಿಸಬೇಕಾಗಿರುವುದರಿಂದ ಅವರ ಕಣ್ಣಿಗೆ ಯಾವ ಒಳ್ಳೆಯದೂ ಬೀಳದೆ ಇರುವುದಿಲ್ಲ, ಅಥವಾ ಘೋರ ಪಾಪಕೃತ್ಯಗಳು ಕೂಡ ಇವರ ಲಕ್ಷ್ಯಕ್ಕೆ ಬಾರದೆ ಇರುವುದಿಲ್ಲ. ಇತರ ಧರ್ಮಗಳಿಗೆ ಸೇರಿದವರ ವಿಷಯದಲ್ಲಿ ಅವರು ತುಂಬು ಔದಾರ್ಯದಿಂದ ವರ್ತಿಸುತ್ತಾರೆ. ತಮ್ಮ ಅಭಿಪ್ರಾಯವನ್ನು ಒಪ್ಪಿಕೊಳ್ಳದೆ ಇರುವವರಿಗೆ ಕೂಡ ಅವರು ಒಳ್ಳೆಯ ಮಾತುಗಳನ್ನು ಮಾತ್ರ ಹೇಳುತ್ತಾರೆ.

\delimiter

\begin{center}
\textbf{(ಡೈರಿ ಗೆಜೆಟ್​, ಸೆಪ್ಟೆಂಬರ್​ ೫, ೧೮೯೩)}
\end{center}

\vskip -0.35cm

ಇಂಡಿಯಾದ ರಾಜಾ ಸ್ವಾಮಿ ವಿವೇಕಾನಂದರು, ಭಾನುವಾರ ಸಾಯಂಕಾಲ ಈಸ್ಟ್​ ಚರ್ಚ್​ನಲ್ಲಿ ಭರತಖಂಡದ ಧರ್ಮ ಮತ್ತು ಅಲ್ಲಿಯ ಬಡಜನರ ವಿಷಯವಾಗಿ ಮಾತನಾಡಿದರು. ಅನೇಕ ಜನರು ನೆರೆದಿದ್ದರು. ಆದರೆ ಮಾತನಾಡಿದ ರಸಭರಿತವಾದ ವಿಷಯ ಮತ್ತು ಅವರ ವ್ಯಕ್ತಿತ್ವಕ್ಕೆ ತಕ್ಕಷ್ಟು ಸಭಿಕರು ನೆರೆದಿದ್ದರು ಎಂದು ಹೇಳಲಾಗುವುದಿಲ್ಲ. ಸಂನ್ಯಾಸಿಗಳು ತಮ್ಮ ದೇಶೀಯ ಉಡುಪನ್ನು ಧರಿಸಿದ್ದರು. ಅವರು ಸುಮಾರು ನಲವತ್ತು ನಿಮಿಷಗಳ ಕಾಲ ಮಾತನಾಡಿದರು. ಇಂದಿನ ಭರತಖಂಡವು ಐವತ್ತು ವರ್ಷಗಳ ಹಿಂದಿನ ಭರತಖಂಡವಲ್ಲವೆಂದೂ, ಇಂದು ಅಲ್ಲಿ ಅತ್ಯಾವಶ್ಯಕವಾಗಿ ಬೇಕಾಗಿರುವುದು-ಜನರಿಗೆ ಕೈಗಾರಿಕೆ ಮತ್ತು ಸಾಮಾಜಿಕ ವಿಷಯಗಳನ್ನು ಕಲಿಸಬಲ್ಲ ಮಿಷನರಿಗಳು, ಧರ್ಮವನ್ನು ಕಲಿಸುವವರಲ್ಲ-ಎಂದರು. ಹಿಂದೂಗಳಿಗೆ ಬೇಕಾದ ಧರ್ಮವೆಲ್ಲ ಇದೆ. ಹಿಂದೂಧರ್ಮ ಪ್ರಪಂಚದಲ್ಲೆಲ್ಲ ಬಹಳ ಪುರಾತನವಾಗಿರುವ ಧರ್ಮ ಎಂದರು. ಈ ಸಂನ್ಯಾಸಿ ಬಹಳ ಚೆನ್ನಾಗಿ ಮಾತನಾಡಬಲ್ಲರು. ಸಭಿಕರ ಮನಸ್ಸನ್ನು ತಮ್ಮ ಕಡೆಗೆ ಆಕರ್ಷಿಸಿಕೊಂಡರು.

\delimiter

\vskip -1.2cm

\begin{center}
\textbf{(ಡೈಲಿ ಸಾರಟೋಗಿಯನ್​, ಸೆಪ್ಟೆಂಬರ್​ ೬, ೧೮೯೩)}
\end{center}

\vskip -0.35cm

ಅನಂತರ ಮದ್ರಾಸಿನಿಂದ ಬಂದ, ಹಿಂದೂಸ್ತಾನದಲ್ಲೆಲ್ಲಾ ಈಗಾಗಲೇ ಉಪದೇಶಗಳನ್ನು ನೀಡಿದ್ದ ಸಂನ್ಯಾಸಿ ವೇದಿಕೆಯನ್ನು ಅಲಂಕರಿಸಿದರು. ಅವರಿಗೆ ಸಮಾಜ ವಿಜ್ಞಾನದಲ್ಲಿ ತುಂಬಾ ಆಸಕ್ತಿ. ಅವರು ಬುದ್ಧಿವಂತರು, ಮತ್ತು ಸ್ವಾರಸ್ಯವಾಗಿ ಮಾತನಾಡುವರು. ಅವರು ಭರತಖಂಡದಲ್ಲಿ ಮಹಮದೀಯ ಆಳ್ವಿಕೆ ಎಂಬ ವಿಷಯದ ಮೇಲೆ ಮಾತನಾಡಿದರು.

ಇಂದಿನ ಕಾರ್ಯಕ್ರಮದಲ್ಲಿ ಕೆಲವು ಸ್ವಾರಸ್ಯವಾದ ವಿಷಯಗಳಿವೆ. ಹಾರ್​ ವರ್ಡ್​ನಿಂದ ಬರುವ ಕರ್ನಲ್​ ಜಾಕೋಬ್​ ಗ್ರೀನ್​ ಅವರು “ಬೈಮೆಟಾಲಿಸಮ್​” ಎಂಬ ವಿಷಯದ ಮೇಲೆ ಲೇಖನವನ್ನು ಓದುವರು. ಈ ದಿನವೂ ವಿವೇಕಾನಂದರು “ಭರತಖಂಡದಲ್ಲಿ ಬೆಳ್ಳಿಯ ಉಪಯೋಗ” ಎಂಬ ವಿಷಯದ ಮೇಲೆ ಮಾತನಾಡುವರು.

\delimiter

\vskip -1.2cm

\begin{center}
\textbf{ವಿಶ್ವಸಮ್ಮೇಳನ್ನದಲ್ಲಿ ಹಿಂದೂಗಳು}\\ (ಬಾಸ್ಟನ್​ ಈವನಿಂಗ್​ ಟ್ರಾನ್ಸ್​ಕ್ರಿಪ್ಟ್​, ಸೆಪ್ಟೆಂಬರ್​ ೩೦, ೧೮೯೩)
\end{center}

\vskip -0.35cm

ಚಿಕಾಗೊ, ಸೆಪ್ಟೆಂಬರ್​ ೨೩.

\enginline{Art Palace}ನ ಪ್ರವೇಶದ ಎಡಭಾಗದಲ್ಲಿ ಒಂದು ಕೋಣೆ ಇದೆ. ಅದರ ಮೇಲೆ “ನಂ-೧-ಪ್ರವೇಶವಿಲ್ಲ” ಎಂದು ಬರೆದಿರುವುದು. ಇಲ್ಲಿಗೆ ಕಾಂಗ್ರೆಸ್ಸಿಗೆ ಬಂದ ಉಪನ್ಯಾಸಕ\break ರೆಲ್ಲ, ಈಗಲೋ ಅನಂತರವೋ, ತಮ್ಮ ತಮ್ಮೊಳಗೆ ಮಾತನಾಡಿ ಕೊಳ್ಳುವುದಕ್ಕೊ, ಅಥವಾ ಪ್ರೆಸಿಡೆಂಟ್​ ಬೋನಿಯವರೊಂದಿಗೆ ಮಾತನಾಡುವುದಕ್ಕೊ ನೆರೆಯುವರು. ಅವರ ಖಾಸಗಿ ಕೋಣೆಯು ಕಟ್ಟಡದ ಒಂದು ಮೂಲೆಯಲ್ಲಿ ಇದೆ. ಅದರ ಮಡಿಸುವ ಬಾಗಿಲನ್ನು ಯಾರೂ ಪ್ರವೇಶಿಸದಂತೆ ನೋಡಿಕೊಂಡಿರುವರು. ಅದು ಕೂಡ ಅದರ ಮೂಲಕ ಜನರು ಇಣಿಕಿ ನೋಡುವುದಕ್ಕೆ ಕೂಡ ಆಗದಷ್ಟು ದೂರದಲ್ಲಿರುವುದು. ಆ ಪವಿತ್ರವಾದ ಕೊಠಡಿಗೆ ಕಾಂಗ್ರೆಸ್ಸಿಗೆ ಬಂದ ಪ್ರತಿನಿಧಿಗಳಿಗೆ ಮಾತ್ರ ಪ್ರವೇಶ. ಆದರೂ ಒಳಗೆ ಪ್ರವೇಶಿಸುವುದು ಅಷ್ಟೇನೂ ಅಸಾಧ್ಯವಾದುದಲ್ಲ, ಕೊಲಂಬಸ್ಸಿನ ಹಾಲಿನ ವೇದಿಕೆಯ ಮೇಲೆ ಪ್ರತಿನಿಧಿಗಳನ್ನು ನೋಡುವುದಕ್ಕಿಂತ ಇಲ್ಲಿ ಪ್ರವೇಶಿಸಿ ಪ್ರಖ್ಯಾತ ಪ್ರತಿನಿಧಿಗಳ ನಿಕಟ ಪರಿಚಯವನ್ನು ಪಡೆಯಬಹುದು.

ಈ ಒಳಕೋಣೆಯಲ್ಲಿ ನಮ್ಮ ಕಣ್ಣಿಗೆ ಬೀಳುವ ಅತಿ ಆಕರ್ಷಕನಾದ ವ್ಯಕ್ತಿಯೇ ಬ್ರಾಹ್ಮಣ ಸಂನ್ಯಾಸಿಯಾದ ಸ್ವಾಮಿ ವಿವೇಕಾನಂದರು. ಅವರು ಎತ್ತರವಾದ, ದೃಢವಾದ ಮೈಕಟ್ಟಿನವರು. ಅತಿಭವ್ಯವಾದ ಹಿಂದೂಸ್ತಾನೀ ಜನರ ಆಕೃತಿ, ಚೆನ್ನಾಗಿ ಕ್ಷೌರಮಾಡಿಕೊಂಡ ಮುಖ, ಅತಿ ಸ್ಪಷ್ಟವಾದ ಚಚ್ಚೌಕವಾದ ಮುಖದಾಕೃತಿ, ಬಿಳಿಯ ಹಲ್ಲುಗಳು, ಅತಿ ಸುಂದರವಾಗಿ ರಚಿತವಾಗಿರುವ ತುಟಿಗಳು. ಅವರು ಮಾತನಾಡುವಾಗ ಸ್ವಾಭಾವಿಕವಾಗಿ ಅವು ದಯೆಯಿಂದ ಕೂಡಿದ ನಗೆಯೊಂದಿಗೆ ಅರಳುತ್ತವೆ. ಅವರ ಸುಂದರವಾದ ತಲೆಯ ಮೇಲೆ ನಿಂಬೆಹಣ್ಣಿನ ಬಣ್ಣದ್ದೊ ಅಥವಾ ಕೆಂಪು ಬಣ್ಣದ್ದೊ ರುಮಾಲು ಇರುತ್ತದೆ. ದೇಹದ ಮೇಲೆ ಧರಿಸಿರುವ ದೊಡ್ಡ ನಿಲುವಂಗಿ ಮೊಣಕಾಲುಗಳನ್ನು ದಾಟಿ ಬರುತ್ತದೆ. ಸೊಂಟದ ಮೇಲೆ ಅದನ್ನು ಕಟ್ಟಿಕೊಂಡಿರುವರು. ಕಡುಗೆಂಪು ಮತ್ತು ಕಿತ್ತಲೇ ಬಣ್ಣದ್ದೊ ಆ ನಿಲುವಂಗಿ. ಅವರು ಅತ್ಯಂತ ಸೊಗಸಾದ ಇಂಗ್ಲೀಷ್​ ಭಾಷೆಯನ್ನು ಮಾತನಾಡುವರು. ಯಾವ ಪ್ರಶ್ನೆಯನ್ನು ಕೇಳಿದರೂ ಅದಕ್ಕೆ ತಕ್ಷಣವೇ ಉತ್ತರ ಕೊಡುವರು.

ಅವರ ನಡೆನುಡಿಯಲ್ಲಿ ಸರಳತನವಿದ್ದರೂ, ಅವರು ಸ್ತ್ರೀಯರೊಡನೆ ಮಾತನಾಡು ವಾಗ ಒಂದು ಅಂತರವನ್ನು ಕಾಪಾಡಿಕೊಳ್ಳುತ್ತಾರೆ. ಇದು ಅವರು ಆರಿಸಿಕೊಂಡಿರುವ ವ್ರತಕ್ಕೆ -(ಎಂದರೆ ಸಂನ್ಯಾಸಾಶ್ರಮಕ್ಕೆ) ಅನುಗುಣವಾಗಿದೆ. ಅವರ ಸಂನ್ಯಾಸಿಯ ಜೀವನಕ್ಕೆ ಸಂಬಂಧಪಟ್ಟ ನಿಯಮವನ್ನು ಪ್ರಶ್ನಿಸಿದಾಗ “ನಾನು ನನಗೆ ತೋರಿದಂತೆ ಮಾಡಬಲ್ಲೆ. ನಾನು ಸ್ವತಂತ್ರ. ನಾನು ಕೆಲವು ವೇಳೆ ಹಿಮಾಲಯ ಪರ್ವತಗಳಲ್ಲಿ ಇರುವೆ, ಕೆಲವು ವೇಳೆ ಕಲ್ಕತ್ತೆಯ ಬೀದಿಗಳಲ್ಲಿ ಇರುವೆ, ಇನ್ನೊಂದು ಊಟ ನನಗೆ ಎಲ್ಲಿ ಸಿಕ್ಕುವುದೊ ಅದು ನನಗೆ ಗೊತ್ತಿಲ್ಲ. ನನ್ನ ಹತ್ತಿರ ನಾನು ದುಡ್ಡನ್ನು ಇಟ್ಟು ಕೊಳ್ಳುವುದಿಲ್ಲ. ಜನರು ಚಂದಾ ಎತ್ತಿ ನನ್ನನ್ನು ಇಲ್ಲಿಗೆ ಕಳುಹಿಸಿದರು” ಎಂದರು. ತಮ್ಮ ಹತ್ತಿರ ನಿಂತಿದ್ದ ಇಂಡಿಯಾ ದೇಶದ ಒಬ್ಬರಿಬ್ಬರನ್ನು ನೋಡಿ ನಗುತ್ತ “ಇವರು ನನ್ನನ್ನು ನೋಡಿಕೊಳ್ಳುವರು” ಎಂದರು. ಎಂದರೆ ಚಿಕಾಗೊ ನಗರದಲ್ಲಿ ಆಗುವ ಖರ್ಚನ್ನು ಮತ್ತಾರೊ ವಹಿಸುತ್ತಾರೆ ಎಂದು ಅರ್ಥ ಬರುವಂತೆ ನುಡಿದರು. ತಾವು ಸಾಧಾರಣವಾಗಿ ಧರಿಸುವ ಸಂನ್ಯಾಸಿಯ ಉಡುಪನ್ನೇ ಇಲ್ಲಿ ಧರಿಸಿದ್ದೀರಾ ಎಂದು ಪ್ರಶ್ನಿಸಿದಾಗ, “ಈಗ ಹಾಕಿಕೊಂಡಿರುವುದು ಬಹಳ ಒಳ್ಳೆಯ ಉಡುಪು. ನಾನು ನನ್ನ ದೇಶದಲ್ಲಿ ಚಿಂದಿಪಂದಿಯನ್ನು ಉಡುವೆ. ಬರಿ ಕಾಲಿನಲ್ಲಿ ಹೊಗುವೆ. ನಾನು ಜಾತಿಯನ್ನು ನಂಬುತ್ತೇನೆಯೇ ಎಂದು ಪ್ರಶ್ನಿಸುವಿರಾ? ಜಾತಿ ಸಮಾಜಕ್ಕೆ ಸಂಬಂಧಪಟ್ಟ ವಿಚಾರ. ಧರ್ಮಕ್ಕೂ ಅದಕ್ಕೂ ಏನೂ ಸಂಬಂಧವಿಲ್ಲ. ಎಲ್ಲಾ ಜಾತಿಯವರೂ ನನ್ನ ಬಳಿಗೆ ಬರುತ್ತಾರೆ” ಎಂದರು.

ವಿವೇಕಾನಂದರ ಚಹರೆಯನ್ನು ನೋಡಿದರೆ ಅವರು ಕುಲೀನ ವಂಶಕ್ಕೆ ಸೇರಿದವರೆಂಬುದನ್ನು ನಿಸ್ಸಂದೇಹವಾಗಿ ಹೇಳಬಹುದು. ತಾವಾಗಿಯೆ ಸ್ವೀಕರಿಸಿದ ಬಡತನ ಮತ್ತು ಪರಿವ್ರಾಜಕರಾಗಿ ಹಲವು ವರುಷಗಳು ಮಾಡಿದ ಸಂಚಾರ ಇವುಗಳಿಂದ ಅವರು ತಮ್ಮ ಜನ್ಮಸಿದ್ಧವಾದ ಗೌರವಸ್ಥಾನವನ್ನು ಕಳೆದುಕೊಂಡಿಲ್ಲ. ಅವರ ವಂಶದ ಹೆಸರು ಕೂಡ ಗೊತ್ತಿಲ್ಲ. ಸಂನ್ಯಾಸಿಗಳಾದಾಗ ಅವರು ವಿವೇಕಾನಂದ ಎಂಬ ಹೆಸರನ್ನು ತೆಗೆದುಕೊಂಡರು. ಸ್ವಾಮಿ ಎಂಬುದು ಅವರಿಗೆ ಕೊಟ್ಟಿರುವ ಗೌರವದ ಹೆಸರು. ಅವರ ವಯಸ್ಸು ಮೂವತ್ತು ವರುಷಗಳಿಗೆ ಮೀರಿಹೋದಂತೆ ಕಾಣುವುದಿಲ್ಲ. ಅವರು ಈ ಆಶ್ರಮಕ್ಕಾಗಿಯೇ ಹುಟ್ಟಿದವ\break ರಂತೆ, ಜನನಮರಣಗಳಿಗೆ ಅತೀತವಾದುದನ್ನು ಧ್ಯಾನ ಮಾಡುವುದಕ್ಕಾಗಿಯೇ ಬಂದವ\break ರಂತೆ ತೋರುವರು. ಅವರ ಜೀವನದಲ್ಲಿ ಯಾವ ಘಟನೆಯು ಅವರನ್ನು ಈ ಮಾರ್ಗಕ್ಕೆ ಬರುವಂತೆ ಮಾಡಿರಬಹುದೊ ಎಂಬುದು ನಮಗೆ ತಿಳಿಯದು.

“ನಾನು ಪ್ರತಿಯೊಬ್ಬ ಸ್ತ್ರೀಯರಲ್ಲಿಯೂ ಜಗನ್ಮಾತೆಯನ್ನೇ ನೋಡುವಾಗ ಏತಕ್ಕೆ ಮದುವೆಯಾಗಬೇಕು? ನಾನು ಇವುಗಳನ್ನೆಲ್ಲಾ ತ್ಯಜಿಸಿರುವುದು ಏತಕ್ಕೆ? ಜನನಮರಣಗಳಿಂದ ಕೂಡಿದ ಸಂಸಾರದಿಂದ ಪಾರಾಗುವುದಕ್ಕಾಗಿ. ನಾನು ಸತ್ತರೆ ತಕ್ಷಣವೇ ದೇವರಲ್ಲಿ ಒಂದಾಗಬೇಕೆಂದು ಆಶಿಸುವೆ. ನಾನೊಬ್ಬ ಬುದ್ಧನಾಗುತ್ತೇನೆ.”

ಇದರಿಂದ ವಿವೇಕಾನಂದರು ತಾವು ಬೌದ್ಧರು ಎಂದು ಹೇಳಿಕೊಳ್ಳುವುದಿಲ್ಲ. ಈಗ ಯಾವ ಜಾತಿಗೂ ಯಾವ ಹೆಸರಿಗೂ ಸೇರದವರು. ಅವರು ಹಿಂದೆ ಉತ್ತಮ ಬ್ರಾಹ್ಮಣ ವಂಶಕ್ಕೆ ಸೇರಿದವರು. ಈಗ ಅವರು ಶ್ರೇಷ್ಠರಾದ ಭಾವಜೀವಿಗಳಾದ ಆತ್ಮತ್ಯಾಗ ಮಾಡಿದ ಸಂನ್ಯಾಸಿ ಅಥವಾ ಸಾಧುಗಳ ವರ್ಗಕ್ಕೆ ಸೇರಿದವರು.

ಅವರು ತಮ್ಮ ಹತ್ತಿರವಿರುವ ಕೆಲವು ಪುಸ್ತಿಕೆಗಳನ್ನು ಹಂಚುತ್ತಿರುವರು. ಅವು ಇವರ ಗುರುಗಳಾದ ಶ‍್ರೀರಾಮಕೃಷ್ಣ ಪರಮಹಂಸರಿಗೆ ಸಂಬಂಧಪಟ್ಟವುಗಳು. ಅವರೊಬ್ಬ ಹಿಂದೂ ಭಕ್ತರು. ಅವರು ತಮ್ಮ ಶಿಷ್ಯರ ಮೇಲೆ ಅದ್ಭುತವಾದ ಪ್ರಭಾವವನ್ನು ಬೀರಿರುವರು. ಇದರ ಪರಿಣಾಮವಾಗಿ ಅವರ ಕಾಲಾನಂತರ ಅನೇಕರು ಸಂ ಸಂನ್ಯಾಸಿಗಳಾದರು. ಮಜುಮ್​\break ದಾರರು ಕೂಡ ಅವರನ್ನು ತಮ್ಮ ಗುರುಗಳೆಂದು ಭಾವಿಸುವರು. ಮಜುಮ್​ದಾರರಾದರೊ ಕ್ರಿಸ್ತನ ಬೋಧನೆಯ ಮೇರೆಗೆ ಈ ಪ್ರಪಂಚದಲ್ಲಿ ಇದ್ದರೂ ಪ್ರಾಪಂಚಿಕತೆ ಇಲ್ಲದೆ ಇರುವ ಸಂದೇಶವನ್ನು ಬೋಧಿಸುವರು.

ವಿಶ್ವಧರ್ಮ ಸಮ್ಮೇಳನದಲ್ಲಿ ವಿವೇಕಾನಂದರು ಮಾಡಿದ ಉಪನ್ಯಾಸದ ಭಾವನೆಯು ನಮ್ಮ ಮೇಲಿರುವ ಆಗಸದಷ್ಟು ವಿಶಾಲವಾಗಿತ್ತು. ಕೊನೆಗೆ ಬರಬಹುದಾದ ವಿಶ್ವವ್ಯಾಪಿ\break ಯಾದ ಧರ್ಮದಲ್ಲಿರುವಂತೆ, ಎಲ್ಲಾ ಧರ್ಮಗಳ ಶ್ರೇಷ್ಠ ವಿಷಯಗಳು ಅವರ ಉಪನ್ಯಾಸ\break ದಲ್ಲಿದ್ದವು. ಪ್ರಪಂಚದಲ್ಲಿ ಎಲ್ಲಾ ಮಾನವರಿಗೂ ಔದಾರ್ಯವನ್ನು ತೋರುವುದು, ಭಗವದ್​ ಪ್ರೇಮದಿಂದ ಪ್ರೇರಿತವಾಗಿ ಒಳ್ಳೆಯ ಕರ್ಮಗಳನ್ನು ಮಾಡುವುದು, ಅವನು ಕೊಡುವ ಶಿಕ್ಷೆಗೆ ಅಂಜಿ ಅಥವಾ ಅವನು ಕೊಡುವ ಬಹುಮಾನದ ಆಸೆಗೆ ಮಾಡುವುದಲ್ಲ-ಇತ್ಯಾದಿ ವಿಷಯಗಳೆಲ್ಲ ಅವರ ಉಪನ್ಯಾಸದಲ್ಲಿದ್ದವು. ತಮ್ಮ ವ್ಯಕ್ತಿತ್ವ ಮತ್ತು ಬೋಧನೆಯ ಭವ್ಯತೆ ಇವುಗಳಿಂದ ಸ್ವಾಮೀಜಿ ಅವರು ವಿಶ್ವಧರ್ಮ ಸಮ್ಮೇಳನದಲ್ಲಿ ಜನರ ಅನುರಾಗವನ್ನು ಪಡಕೊಂಡಿರುವರು. ಸುಮ್ಮನೆ ವೇದಿಕೆಯ ಮೇಲೆ ಸಂಚರಿಸಿದರೆ ಸಾಕು, ಜನರು ಕರತಾಡನ ಮಾಡುವರು. ಸಹಸ್ರಾರು ಜನರು ನೀಡುವ ಗೌರವವನ್ನು ಸ್ವಲ್ಪವೂ ಅಹಂಕಾರವಿಲ್ಲದೆ ಎಳೆ ಹಸುಳೆಯಂತೆ ಸ್ವೀಕರಿಸುವರು. ತ್ಯಾಗ ಮತ್ತು ಬಡತನಗಳ ಸ್ಥಿತಿಯಲ್ಲಿದ್ದ ದೀನನಾದ ಯುವಕ ಬ್ರಾಹ್ಮಣ ಸಂನ್ಯಾಸಿಗೆ ಕೀರ್ತಿ ಮತ್ತು ಐಶ್ವರ್ಯಗಳು ಒಂದು ನವೀನವಾದ ಅನುಭವಗಳೇ ಆಗಿರಬೇಕು. ಥಿಯಾಸಫಿಸ್ಟರು ಬಹಳ ದೃಢವಾಗಿ ನಂಬುವ ಹಿಮಾಲಯದಲ್ಲಿರುವ ಮಹಾತ್ಮರ ವಿಷಯವಾಗಿ ಅವರನ್ನು ಪ್ರಶ್ನಿಸಿದಾಗ, ಅವರು ಸರಳವಾಗಿ “ನಾನು ಇಂತಹ ಯಾರನ್ನೂ ಇದು ವರೆಗೂ ಕಂಡಿಲ್ಲ” ಎಂದು ಹೇಳಿದರು. ಆದರೆ ಅವರ ಮಾತಿನಲ್ಲಿ “ಅಂತಹ ಮಹಾತ್ಮರು ಇರಬಹುದೇನೊ, ಆದರೆ ಹಿಮಾಲಯ ನನಗೆ ಚೆನ್ನಾಗಿ ಪರಿಚಯ ವಿದ್ದರೂ ನನಗೆ ಅಂಥವರಲ್ಲಿ ಒಬ್ಬರ ದರ್ಶನವೂ ಆಗಿಲ್ಲ” ಎಂಬ ಅರ್ಥ ಬರುವಂತೆ ಇತ್ತು.

\begin{center}
\textbf{ವಿಶ್ವಧರ್ಮ ಸಮ್ಮೇಳನದಲ್ಲಿ}\\ (ದಿ ಡುಬುಕ್ಯೂ, ಅಯೋವ, ಟೈಮ್ಸ್​, ಸೆಪ್ಟೆಂಬರ್​ ೨೯, ೧೮೯೩)
\end{center}

\vskip -0.3cm

ವಿಶ್ವದ ಜಾತ್ರೆ-ಸೆಪ್ಟೆಂಬರ್​ ೨೮-(ವಿಶೇಷ)-ವಿಶ್ವಧರ್ಮ ಸಮ್ಮೇಳನವು ಬಹಳ ಕಟುವಾದ ಪರಿಸ್ಥಿತಿಯನ್ನು ಮುಟ್ಟಿತು. ಕೃತಕ ಸಭ್ಯತೆಯ ವಾತಾವರಣವನ್ನೇನೊ ಸ್ವಲ್ಪ ಕಾಪಾಡಿ\break ಕೊಂಡಿದ್ದರು. ಆದರೆ ಅದರ ಹಿಂದೆ ದ್ವೇಷದ ಭಾವನೆ ಇತ್ತು. ರೆವರೆಂಡ್​ ಜೋಸೆಫ್​ಕುಕ್​ ಅವರು ಹಿಂದೂಗಳನ್ನು ಬಹಳ ಕಟುವಾಗಿ ಟೀಕಿಸಿದರು. ಅದಕ್ಕೆ ಪ್ರತಿಯಾಗಿ ಹಿಂದೂಗಳು ಅವರನ್ನು ಖಂಡಿಸಿದರು. ಕುಕ್​ ಅವರು, ದೇವರು ವಿಶ್ವವನ್ನು ಸೃಷ್ಟಿಸಿಲ್ಲ ಎಂದು ಹೇಳುವುದು ಅಕ್ಷಮ್ಯವಾದ ಅವಿವೇಕ ಎಂದರು. ಏಷ್ಯಾಖಂಡದವರು, ಆದಿ ಇರುವ ವಿಶ್ವ ಒಂದು ನಿಜವಾದ ಅಸಂಬದ್ಧ ಪ್ರಲಾಪ ಎಂದರು. ಬಿಷಪ್​ ಜೆ.ಪಿ. ನ್ಯೂಮೀನ್​ ಅವರು, ಓಹಿಯೋ ನದೀ ತೀರದಲ್ಲಿ ದೂರವಾಗಿ ನಿಂತುಕೊಂಡು, ಪೌರಸ್ತ್ಯರು ಮಿಷನರಿಗಳ ವಿಷಯದಲ್ಲಿ ಅಪಪ್ರಚಾರ ಮಾಡಿ ಅಮೆರಿಕಾ ದೇಶದ ಕ್ರೈಸ್ತರಿಗೆಲ್ಲ ಅಪಮಾನ ಮಾಡಿರುವರು ಎಂದರು. ಪೌರಸ್ತ್ಯರಾದರೋ ಎದುರಾಳಿಯನ್ನು ಕೆರಳಿಸುವಂತಹ ಶಾಂತಿ ಮತ್ತು ತಿರಸ್ಕಾರದ ನಗೆಯೊಂದಿಗೆ, ಇದೆಲ್ಲ ಬರೀ ಬಿಷಪ್ಪರ ಅಜ್ಞಾನಜನ್ಯ ಅಭಿಪ್ರಾಯ ಎಂದರು.

\begin{center}
\textbf{ಬೌದ್ಧ ತತ್ತ್ವ}
\end{center}

\vskip -0.3cm

ಪ್ರತ್ಯಕ್ಷವಾಗಿ ಕೇಳಿದ ಪ್ರಶ್ನೆಗೆ ಉತ್ತರವಾಗಿ ಮೂರು ಜನ ಬೌದ್ಧ ವಿದ್ವಾಂಸರು, ಸರಳವಾದ, ಸುಂದರವಾದ ಭಾಷೆಯಲ್ಲಿ, ತಮ್ಮ ಧರ್ಮದಲ್ಲಿ ದೇವರು, ಮನುಷ್ಯ ಮತ್ತು ಜಗತ್ತು ಇವುಗಳಿಗೆ ಸಂಬಂಧಿಸಿದ ಮೂಲಭೂತ ಅಭಿಪ್ರಾಯಗಳನ್ನು ವ್ಯಕ್ತಪಡಿಸಿದರು.

(ಮುಂದೆ ಬರುವುದು “ಬುದ್ಧನಿಗೆ ಪ್ರಪಂಚದ ಋಣ” ಎಂಬ ಧರ್ಮಪಾಲರ ಲೇಖನದ ಸಾರಾಂಶ. ಅವರು ಈ ಲೇಖನವನ್ನು ಓದುವುದಕ್ಕೆ ಮುಂಚೆ ಸಿಂಹಳ ಭಾಷೆಯ ಪ್ರಾರ್ಥನೆಯಿಂದ ಪ್ರಾರಂಭ ಮಾಡಿದರು ಎಂದು ನಾವು ಬೇರೆಯವರಿಂದ ಕೇಳಿದೆವು. ಲೇಖನ ನಂತರ ಮುಂದುವರಿಯುವುದು.)

ಅವರ (ಸ್ವಾಮೀಜಿ) ವಾಗ್​ವೈಖರಿಯು ಚಿಕಾಗೊ ಸಭಿಕರು ಕೇಳಿದ ಉಪನ್ಯಾಸಗಳಲ್ಲೆಲ್ಲ ತುಂಬಾ ಸುಂದರವಾಗಿತ್ತು. ಡೆಮಾಸ್ಥನೀಸ್​ ಇವರನ್ನು ಮೀರಿಸಿರಲಾರ.

\begin{center}
\textbf{ಜಗಳಗಂಟಿಯ ಟೀಕೆಗಳು}
\end{center}

\vskip -0.3cm

ಹಿಂದೂ ಸಂನ್ಯಾಸಿಯಾದ ಸ್ವಾಮಿ ವಿವೇಕಾನಂದರು ಬೌದ್ಧಧರ್ಮವನ್ನು ಕುರಿತು ಮಾತನಾಡಿದ ಧರ್ಮಪಾಲರಷ್ಟು ಅದೃಷ್ಟಶಾಲಿಗಳಾಗಿರಲಿಲ್ಲ. ಅವರು ಮುಂಚೆಯೇ ಕೋಪಿಷ್ಠರಾಗಿದ್ದರು. ಅಥವಾ ಬಹಳಬೇಗ ಹಾಗೆ ಆದರು. ಅವರು ಒಂದು ಗೈರಿಕ ವಸನದ ನಿಲುವಂಗಿಯನ್ನು ತೊಟ್ಟಿದ್ದರು, ಒಂದು ಕಂದು ಬಣ್ಣದ ರುಮಾಲನ್ನು ಧರಿಸಿದ್ದರು. ವೇದಿಕೆಯ ಮೇಲೆ ನಿಂತೊಡನೆಯೇ ಕ್ರೈಸ್ತಜನಾಂಗಗಳನ್ನು ಬಹಳ ಕಟುವಾಗಿ ಹೀಗೆ ಟೀಕಿಸಿದರು: “ಪೌರಸ್ತ್ಯ ದೇಶಗಳಿಂದ ಬಂದ ನಾವು ಪ್ರತಿದಿನವೂ ಇಲ್ಲಿ ಕುಳಿತುಕೊಂಡು, ನಾವೆಲ್ಲ ಕ್ರೈಸ್ತಧರ್ಮವನ್ನು ಸ್ವೀಕರಿಸಬೇಕೆಂದೂ, ಏಕೆಂದರೆ ಕ್ರೈಸ್ತ ಜನಾಂಗಗಳೇ ಪ್ರಪಂಚದಲ್ಲಿ ಅತಿ ಪ್ರಗತಿಗಾಮಿಯಾಗಿ ಇರುವುದೆಂದೂ ಕೇಳುತ್ತಿರುವೆವು. ನಮ್ಮ ಸ್ಥಿತಿಯನ್ನು ನೋಡಿಕೊಂಡರೆ ಪ್ರಪಂಚದಲ್ಲೆಲ್ಲ ಅತ್ಯಂತ ಶ‍್ರೀಮಂತ ರಾಷ್ಟ್ರವಾದ ಇಂಗ್ಲೆಂಡ್​ ತನ್ನ ಕಾಲುಗಳನ್ನು \enginline{250,000,000} ಏಷ್ಯನರ ಕತ್ತಿನ ಮೇಲೆ ಇಟ್ಟಿರುವುದನ್ನು ನೋಡುತ್ತೇವೆ. ನಾವು ಚರಿತ್ರೆಯನ್ನು ಓದಿ ನೋಡಿದರೆ, ಯೂರೋಪಿನಲ್ಲಿರುವ ಕ್ರೈಸ್ತರ ಪ್ರಗತಿ ಸ್ಪೇಯಿನ್​ ದೇಶದಿಂದ ಆರಂಭವಾಯಿತು ಎಂದು ತೋರುವುದು. ಸ್ಪೇನಿನ ಏಳಿಗೆ ಪ್ರಾರಂಭವಾದುದು ಮೆಕ್ಸಿಕೋ ದೇಶವನ್ನು ಆಕ್ರಮಣ ಮಾಡಿದ ದಿನದಿಂದ. ಕ್ರೈಸ್ತಧರ್ಮದ ಏಳಿಗೆ ಇರುವುದು ಇತರರ ಕೊಲೆಯ ಮೇಲೆ. ಹಿಂದೂಗಳು ಈ ರೀತಿ ಏಳಿಗೆಯನ್ನು ಹೊಂದಲಾರರು.”

ಹೀಗೆಯೇ ಉಪನ್ಯಾಸ ಪ್ರಾರಂಭವಾಯಿತು. ಪ್ರತಿಯೊಬ್ಬ ಉಪನ್ಯಾಸಕನೂ ಹಿಂದಿನ\break ವರಿಗಿಂತ ಕಟುವಾಗಿ ಟೀಕಿಸಲು ಪ್ರಾರಂಭಿಸಿದನು.

\delimiter

\vskip -0.35cm

\begin{center}
\textbf{(ಔಟ್​ ಲುಕ್​, ಅಕ್ಟೋಬರ್​ ೭, ೧೮೯೩)}
\end{center}

\vskip -0.35cm

....ಇಂಡಿಯಾ ದೇಶದಲ್ಲಿ ಕ್ರೈಸ್ತರು ಮಾಡಿರುವ ಕೆಲಸ ಎಂಬುದರ ವಿಷಯವನ್ನು ಮಾತನಾಡಲು ಆಕರ್ಷಣೀಯವಾದ ಗೈರಿಕವಸನವನ್ನು ಧರಿಸಿದ ವಿವೇಕಾನಂದರು ಎದ್ದರು. ಅವರು ಕ್ರೈಸ್ತ ಮಿಶನರಿಗಳ ಕೆಲಸವನ್ನು ಜರಿದರು. ಅವರಿಗೆ ಕ್ರೈಸ್ತಧರ್ಮ ಚೆನ್ನಾಗಿ ಗೊತ್ತಿಲ್ಲ ಎನ್ನುವುದು ವ್ಯಕ್ತವಾಗುವುದೆಂದರು. ಅದರಂತೆಯೇ ಸಹಸ್ರಾರು ವರುಷಗಳ ಜಾತಿ ಮತಗಳ ಭಾವನೆಗಳೊಡನೆ ಹುಟ್ಟುವ ಹಿಂದೂಗಳನ್ನೂ ಕೂಡ ಪಾದ್ರಿಗಳು ಅರ್ಥಮಾಡಿ ಕೊಂಡಿಲ್ಲವೆಂದರು. ತಮ್ಮ ಪ್ರಕಾರ ಪಾದ್ರಿಗಳು ಭಾರತಕ್ಕೆ ಬಂದಿರುವುದು, ಭಾರತೀಯರ ಅತ್ಯಂತ ಪವಿತ್ರವಾದ ಧಾರ್ಮಿಕ ನಂಬಿಕೆಗಳ ಮೇಲೆ ನಿಂದೆಯ ಮಳೆಗರೆಯುವುದಕ್ಕೆ ಮತ್ತು ತಾವು ಯಾರಿಗೆ ಉಪದೇಶ ನೀಡಬೇಕಾಗಿದೆಯೋ ಆ ತಮ್ಮ ಜನರ ನೀತಿ ಮತ್ತು ಅಧ್ಯಾತ್ಮಗಳ ಅಡಿಗಲ್ಲನ್ನೇ ನಾಶಪಡಿಸುವುದಕ್ಕೆ.

\delimiter

\begin{center}
\textbf{(ಕ್ರಿಟಿಕ್​, ಅಕ್ಟೋಬರ್​ ೭, ೧೮೯೩)}
\end{center}

\vskip -0.3cm

ವಿಶ್ವಧರ್ಮ ಸಮ್ಮೇಳನದ ಅತ್ಯಂತ ಆಕರ್ಷಕ ವ್ಯಕ್ತಿಗಳೆಂದರೆ ಸಿಲೋನಿನ ಬೌದ್ಧ ಭಿಕ್ಷು\break ಗಳಾದ ಎಚ್​. ಧರ್ಮಪಾಲ ಮತ್ತು ಹಿಂದೂ ಸಂನ್ಯಾಸಿಯಾದ ಸ್ವಾಮಿ ವಿವೇಕಾನಂದರು. ಮೊದಲನೆ ಉಪನ್ಯಾಸಕರು ಹೀಗೆ ಹೇಳಿದರು: “ಸತ್ಯಾನ್ವೇಷಣೆಯಲ್ಲಿ ಶಾಸ್ತ್ರ, ನಂಬಿಕೆಗಳು ಅಡ್ಡವಾಗಿ ಬಂದರೆ ಅವುಗಳನ್ನು ತ್ಯಜಿಸಿ. ಯಾವ ಪೂರ್ವನಿಶ್ಚಿತ ಅಭಿಪ್ರಾಯವೂ ಇಲ್ಲದೇ ವಿಚಾರಮಾಡುವುದನ್ನು ಅಭ್ಯಾಸಮಾಡಿ. ಪ್ರೀತಿಗಾಗಿ ಎಲ್ಲವನ್ನೂ ಪ್ರೀತಿಸುವುದನ್ನು ಕಲಿ\break ಯಿರಿ. ನಿರ್ಭಯವಾಗಿ ನಿಮ್ಮ ಅಭಿಪ್ರಾಯಗಳನ್ನು ವ್ಯಕ್ತಪಡಿಸಿ, ಪರಿಶುದ್ಧವಾದ ಜೀವನವನ್ನು ನಡೆಸಿ. ಸತ್ಯದ ಜ್ಯೋತಿ ನಿಮಗೆ ದಾರಿಯನ್ನು ತೋರುವುದು.” ಈ ಸಭೆಯಲ್ಲಿ ಆದ ಸಣ್ಣ ಪುಟ್ಟ ಭಾಷಣಗಳೆಲ್ಲ ಬಹಳ ಚೆನ್ನಾಗಿಯೇ ಇದ್ದುವು. ಅಪೋಲೋ ಕ್ಲಬ್ಬಿನ, ಹಲೆಲೂಜಾ ಗಾನ ಮೇಳದ ಹಾಡುಗಾರರು ಅಂತ್ಯದಲ್ಲಿ ಅತಿಸುಂದರವಾಗಿ ಮುಕ್ತಾಯದ ಹಾಡನ್ನು ಹಾಡಿದರು. ಆದರೂ ವಿಶ್ವಧರ್ಮ ಸಮ್ಮೇಳನದ ಉದ್ದೇಶವನ್ನೂ, ಅದರ ವ್ಯಾಪ್ತಿಯನ್ನೂ ಮತ್ತು ಜನರ ಮೇಲೆ ಅದು ಬೀರಿರುವ ಸತ್​ಪ್ರಭಾವವನ್ನೂ ಹಿಂದೂ ಸಂನ್ಯಾಸಿಗಳಾದ ವಿವೇಕಾನಂದರು ವ್ಯಕ್ತಪಡಿಸಿದಂತೆ ಮತ್ತೆ ಯಾರೂ ವ್ಯಕ್ತಪಡಿಸಲಿಲ್ಲ. ನಾನು ಅವರ ಉಪನ್ಯಾಸವನ್ನು ಪೂರ್ಣವಾಗಿ ಇಲ್ಲಿ ವರದಿ ಮಾಡುತ್ತೇನೆ. ಅದು ಸಭಿಕರ ಮೇಲೆ ಯಾವ ರೀತಿ ಪರಿಣಾಮವನ್ನು ಉಂಟುಮಾಡಿತು ಎಂಬ ಸೂಚನೆಯನ್ನು ಮಾತ್ರ ಕೊಡುತ್ತೇನೆ. ಅವರು ಭಗವಂತನಿಂದ ಪ್ರೇರಿತರಾದ ವಾಗ್ಮಿಗಳು. ಹಳದಿ ಮತ್ತು ಕಿತ್ತಳೆ ವರ್ಣಗಳ ಅವರ ಉಡುಪಿನ ಸುಂದರ ಹಿನ್ನೆಲೆಯಲ್ಲಿ, ಅವರ ದೃಢವೂ, ಅವರ ತೀಕ್ಷ್ಣ ಮತಿಯನ್ನು ಪ್ರತಿಬಿಂಬಿಸುವಂಥದೂ ಆದ ಅವರ ಮುಖವು ಅವರ ಅಪೂರ್ವಶ್ರದ್ಧೆಯಿಂದ ಸುಮಧುರ ಕಂಠದಿಂದ ಬಂದ ಮಾತಿನಷ್ಟೇ ಮನೋಹರವಾಗಿತ್ತು. (ಸ್ವಾಮೀಜಿ ಅವರ ಮುಕ್ತಾಯಭಾಷಣದ ಬಹುಭಾಗವನ್ನು ಬರೆದಾದ ಮೇಲೆ ಲೇಖನ ಹೀಗೆ ಮುಂದುವರಿಯಿತು):

ಈ ಸಮ್ಮೇಳನದ ಅತ್ಯಂತ ಮುಖ್ಯವಾದ ಪರಿಣಾಮವೆಂದರೆ, ಬೇರೆ ಬೇರೆ ದೇಶಗಳಲ್ಲಿ ಕೆಲಸ ಮಾಡುವ ಕ್ರೈಸ್ತಪಾದ್ರಿಗಳ ವಿಷಯದಲ್ಲಿ ಮೂಡಿದ ಭಾವನೆಗಳು; ಬುದ್ಧಿವಂತ\break ರಿಂದಲೂ, ವಿದ್ವಾಂಸರಿಂದಲೂ ತುಂಬಿದ ಪೌರಸ್ತ್ಯ ದೇಶಗಳಿಗೆ ಬೋಧಿಸಲು, ಅರೆಕಲಿತ ಪಾದ್ರಿಗಳನ್ನು ಕಳುಹಿಸುವುದು ಎಂತಹ ಹೆಡ್ಡತನ ಎಂಬುದು ಇಂಗ್ಲೆಂಡ್​ ಮತ್ತು ಅಮೆರಿಕಾ ದೇಶದ ಜನರಿಗೆ ಈಗ ಗೊತ್ತಾದಂತೆ ಹಿಂದೆ ಎಂದೂ ಗೊತ್ತಾಗಿರಲಿಲ್ಲ. ನಾವು ಪರಕೀಯರ ಧರ್ಮವನ್ನು ಪ್ರಸ್ತಾಪಿಸಬೇಕಾದರೆ ಔದಾರ್ಯ ಮತ್ತು ಸಹಾನುಭೂತಿ ಇವುಗಳು ಅಗತ್ಯ. ಇವನ್ನು ಪಡೆದಿರುವ ಬೋಧಕರು ಅತಿ ವಿರಳ. ಬೌದ್ಧರು ಹೇಗೆ ನಮ್ಮಿಂದ ಕಲಿಯಬೇಕೋ ಹಾಗೇಯೇ ಬೌದ್ಧರಿಂದ ನಾವೂ ಕಲಿಯಬೇಕಾದುದು ಇದೆ ಎಂಬುದನ್ನೂ ಮತ್ತು ಅತ್ಯಂತ ಶ್ರೇಷ್ಠವಾದ ಪ್ರಭಾವವನ್ನು ಸೌಹಾರ್ದದ ಮೂಲಕ ಮಾತ್ರ ಬೀರಬಹುದು ಎಂಬುದನ್ನೂ ನಾವು ಅರಿಯಬೇಕು.

\delimiter

\textbf{ಚಿಕಾಗೊ ೩ನೇ ಅಕ್ಟೋಬರ್​, ೧೮೯೩} \versenum{\textbf{ಲೂಸಿ ಮನ್​ರೋ}}

ಅಕ್ಟೋಬರ್​ ೧ನೇ ತಾರೀಕಿನ ನ್ಯೂಯಾರ್ಕ್​ ವರ್ಲ್ಡ್​ ಪತ್ರಿಕೆಯ ಮಹತ್ವ ಪೂರಿತವಾದ ಸಮ್ಮೇಳನದ ಪರವಾಗಿ ಒಂದು ಹೇಳಿಕೆಯನ್ನು ಕೊಡಬೇಕೆಂದು ಪ್ರತಿಯೊಬ್ಬ ಪ್ರತಿನಿಧಿ\break ಯನ್ನೂ ಕೇಳಿಕೊಂಡಾಗ ಸ್ವಾಮೀಜಿ ಅವರು, ಭಗವದ್ಗೀತೆಯಿಂದ ಒಂದನ್ನೂ, ವ್ಯಾಸರಿಂದ ಮತ್ತೊಂದನ್ನೂ ಹೇಳಿದರು:

“ಪ್ರತಿಯೊಂದು ಧರ್ಮದಲ್ಲಿಯೂ, ಮಣಿಗಳ ಮಧ್ಯದಲ್ಲಿರುವ ಸೂತ್ರದಂತೆ ಇರು\break ವವನು ನಾನೆ.” “ಎಲ್ಲಾ ಧರ್ಮಗಳಲ್ಲಿಯೂ ಪವಿತ್ರರೂ ಪರಿಪೂರ್ಣರೂ, ಸಾಧುಸ್ವಭಾವ\break ದವರೂ ಇರುವರು. ಆದಕಾರಣ ಅವರೆಲ್ಲ ಒಂದೇ ಗುರಿಯ ಕಡೆಗೆ ನಮ್ಮನ್ನು ಒಯ್ಯುವರು. ವಿಷದಿಂದ ಹೇಗೆ ಅಮೃತ ತರಲು ಸಾಧ್ಯ?”

\delimiter

\begin{center}
\textbf{ವ್ಯಕ್ತಿಯ ವಿಶೇಷ ಲಕ್ಷಣಗಳು}\\ (ಕ್ರಿಟಿಕ್​, ಅಕ್ಟೋಬರ್​ ೭, ೧೮೯೩)
\end{center}

\vskip -0.3cm

....ವಿಶ್ವಧರ್ಮ ಸಮ್ಮೇಳನದ ಉಪನ್ಯಾಸಗಳು ನಡೆದ ಮೇಲೆ ನಮ್ಮ ಕಣ್ಣುಗಳು ತೆರೆದವು. ಪುರಾತನ ಧರ್ಮಗಳಲ್ಲಿ, ಆಧುನಿಕರಿಗೆ ಬೇಕಾದ ಅತಿ ಸುಂದರವಾದ ಭಾವನೆಗಳು ಅಡಗಿವೆ ಎಂಬುದು ನಮಗೆ ಗೊತ್ತಾಯಿತು. ನಾವು ಇದನ್ನು ಒಂದು ಸಲ ಸ್ಪಷ್ಟವಾಗಿ ಅರಿತಮೇಲೆ, ಅವುಗಳನ್ನು ವಿವರಿಸುವವರ ವಿಷಯದಲ್ಲಿ ನಮ್ಮ ಆಸಕ್ತಿ ಕೆರಳಿತು. ಸಮ್ಮೇಳನ ಮುಕ್ತಾಯಗೊಂಡ ಮೇಲೆ, ಅದನ್ನು ತಿಳಿದುಕೊಳ್ಳುವುದಕ್ಕೆ ನಮಗೆ ಸದವಕಾಶ ದೊರೆತುದು ಸ್ವಾಮಿ ವಿವೇಕಾನಂದರ ಉಪನ್ಯಾಸಗಳ ಮೂಲಕ. ಅವರು ಇನ್ನೂ ಚಿಕಾಗೊ ನಗರದಲ್ಲಿ ಇರುವರು. ಅವರು ಅಮೆರಿಕಾ ದೇಶಕ್ಕೆ ಬಂದಾಗ, ಭರತಖಂಡದಲ್ಲಿ ಹೊಸ ಕೈಗಾರಿಕೆಗಳನ್ನು ಸ್ಥಾಪಿಸುವ ಸಲುವಾಗಿ ಅಮೆರಿಕಾದೇಶೀಯರ ನೆರವು ಪಡೆಯುವುದು ಅವರ ಒಂದು ಉದ್ದೇಶವಾಗಿತ್ತು. ಆದರೆ ಸದ್ಯಕ್ಕೆ ಅವರು ಇದನ್ನು ತ್ಯಜಿಸಿರುವರು. ಅಮೆರಿಕಾ ದೇಶದ ಜನರು ಪ್ರಪಂಚದಲ್ಲೆಲ್ಲ ಬಹಳ ಉದಾರಿಗಳು ಎಂಬುದು ಗೊತ್ತಾಗಿ, ಯಾವುದಾದರೂ ಉದ್ದೇಶವಿರುವ ಪ್ರತಿಯೊಬ್ಬನೂ ಅದನ್ನು ಸಾಧಿಸಲು ಹಣಸಂಪಾದನೆಗಾಗಿ ಇಲ್ಲಿಗೆ ಬರುವನು. ಭರತಖಂಡದ ಬಡವರಿಗೂ ಅಮೆರಿಕಾ ದೇಶದ ಬಡವರಿಗೂ ಇರುವ ವ್ಯತ್ಯಾಸವನ್ನು ಕೇಳಿದಾಗ, ಅಮೆರಿಕಾ ದೇಶದ ಬಡವರು ಇಂಡಿಯಾ ದೇಶದ ರಾಜರಿಗೆ ಸಮ ಎಂದರು. ನ್ಯೂಯಾರ್ಕಿನ ಬಹಳ ಕೀಳಾದ ಸ್ಥಳವನ್ನು ಅವರಿಗೆ ತೋರಿಸಿದ ಮೇಲೆಯೂ, ಅವರ ದೇಶದ ದೃಷ್ಟಿಯಿಂದ ಸೌಖ್ಯವಾಗಿಯೇ ಮತ್ತು ಚೆನ್ನಾಗಿಯೇ ಇದೆ ಎಂದರು.

ವಿವೇಕಾನಂದರು ಬ್ರಾಹ್ಮಣರಲ್ಲಿ ಬ್ರಾಹ್ಮಣರು. ಸಂನ್ಯಾಸಿಗಳ ಸಂಘವನ್ನು ಸೇರು\break ವುದಕ್ಕಾಗಿ ಅವರು ತಮ್ಮ ಜಾತಿಯನ್ನು ಬಿಟ್ಟರು. ಸಂನ್ಯಾಸ ಧರ್ಮದಲ್ಲಿ ಜಾತಿಗೌರವಗಳನ್ನು ತಾವಾಗಿಯೇ ತ್ಯಜಿಸುವರು. ಆದರೂ ಅವರ ವ್ಯಕ್ತಿತ್ವದ ಮೇಲೆ ಆ ಜನಾಂಗದ ಚಿಹ್ನೆ ಇದೆ. ಅವರ ಸಂಸ್ಕೃತಿ, ವಾಗ್ಮಿತೆ ಆಕರ್ಷಣೀಯವಾದ ವ್ಯಕ್ತಿತ್ವ ಇವು ಹಿಂದೂ ಸಂಸ್ಕೃತಿಯ ವಿಷಯದಲ್ಲಿ ನಮಗೆ ಹೊಸ ಭಾವನೆಯನ್ನು ಕೊಟ್ಟಿವೆ. ಅವರು ಸ್ವಾರಸ್ಯವಾದ ವ್ಯಕ್ತಿಗಳು. ಗೈರಿಕವಸನದ ಹಿನ್ನೆಲೆಯಲ್ಲಿ, ಶೋಭಿಸುವ ಅವರ ಸುಂದರವಾದ ತೇಜಸ್ಸಿನಿಂದ ಕೂಡಿದ ಚಲಿಸುತ್ತಿರುವ ಮುಖಮಂಡಲ, ಆಳವಾದ ಸುಶ್ರಾವ್ಯವಾದ ಧ್ವನಿ ಇವುಗಳೆಲ್ಲ ಅವರನ್ನು ನಾವು ಒಲಿಯುವಂತೆ ಮಾಡುವುವು. ಸಾಹಿತ್ಯ ಮಂಡಳಿಗಳು ಅವರನ್ನು ಮಾತನಾಡಲು ಆಮಂತ್ರಿಸಿದುದರಲ್ಲಿ ಆಶ್ಚರ್ಯವೇನೂ ಇಲ್ಲ. ಬುದ್ಧನ ಜೀವನ ಮತ್ತು ಅವನ ಧರ್ಮ ನಮಗೆ ಚಿರಪರಿಚಿತವಾಗುವಂತೆ ಅವರು ಚರ್ಚುಗಳಲ್ಲಿ ಬೋಧಿಸಿರುವರು. ಅವರು ಯಾವ ಟಿಪ್ಪಣಿಗಳೂ ಇಲ್ಲದೇ ಮಾತನಾಡುವರು. ಅವರು ತಮ್ಮ ವಿಷಯಗಳನ್ನು ಮತ್ತು ನಿರ್ಣಯಗಳನ್ನು ಅತ್ಯಂತ ಕಲಾಮಯವಾಗಿ ನಮಗೆಲ್ಲಾ ತೃಪ್ತಿಯಾಗುವಂತೆ ವಿವರಿಸಬಲ್ಲರು. ಕೆಲವು ವೇಳೆ ಅತಿ ಸ್ಫೂರ್ತಿಯಿಂದ ಕೂಡಿದ ವಾಗ್ವೈಖರಿಯ ಶಿಖರಕ್ಕೆ ಏರುವುದನ್ನು ನಾವು ನೋಡುತ್ತೇವೆ. ತುಂಬಾ ವಿದ್ಯಾವಂತನಾದ ಜೆಸ್ಯೂಟ್​ ಮಿಷನರಿಯಷ್ಟೆ ಇವರು ವಿದ್ಯಾವಂತರು, ಮತ್ತು ವಿನಯ ಸಂಪನ್ನರು. ಅವರ ಮನೋಭಾವವು ಜೆಸ್ಯೂಟರ ಮನೋಭಾವದಂತೆಯೇ ಇದೆ. ಅವರು ಮಾತಿನ ಮಧ್ಯದಲ್ಲಿ ಉಪಯೋಗಿಸುವ\break ವ್ಯಂಗ್ಯೋಕ್ತಿ, ಕತ್ತಿಯಂತೆ ಹರಿತ ವಾಗಿದ್ದರೂ, ಅವು ಅಷ್ಟು ಸೂಕ್ಷ್ಮವಾಗಿರುವುದರಿಂದ ಹಲವು ಸಭಿಕರಿಗೆ ಅವು ಗೋಚರಕ್ಕೇ ಬರುವುದಿಲ್ಲ. ಆದರೆ ಅವರು ಎಂದಿಗೂ ಅವಿನಯ\break ವನ್ನು ವ್ಯಕ್ತಪಡಿಸುವುದಿಲ್ಲ, ಅವರು ನಮ್ಮ ಆಚಾರ ವ್ಯವಹಾರಗಳನ್ನು ಎಂದೂ ನೇರವಾಗಿ, ಕಟುವಾಗಿ ಟೀಕಿಸುವುದಿಲ್ಲ. ಸಧ್ಯಕ್ಕೆ ಅವರು ಧರ್ಮ ಮತ್ತು ತಮ್ಮ ದಾರ್ಶನಿಕರ ವಿಷಯಗಳ ಕುರಿತು ಮಾತ್ರ ನಮಗೆ ತಿಳಿಸುವುದರಲ್ಲಿ ತೃಪ್ತರಾಗಿರುವರು. ನಾವೆಲ್ಲ ಮೂರ್ತಿಪೂಜೆಯನ್ನು ಮೀರಿ ಹೋಗುವ ಕಾಲವನ್ನು (ಈಗ ಅದು ಅನೇಕ ಅಜ್ಞಾನಿಗಳಾದ ಜನಸಾಧಾರಣರಿಗೆ ಆವಶ್ಯಕವಾಗಿದೆ), ಯಾವುದೇ ಪೂಜೆಯನ್ನು ಮೀರಿ ಹೋಗುವ ಕಾಲವನ್ನು, ಪ್ರಕೃತಿಯಲ್ಲಿ ದೇವರು ಇರುವನು ಎಂಬುದನ್ನು ನೋಡುವ ಮತ್ತು ಮಾನವ\break ನಲ್ಲಿಯೇ ಪವಿತ್ರತೆಯನ್ನು ನೋಡುವ, ಎಲ್ಲಕ್ಕೂ ಅವನನ್ನೇ ಜವಾಬ್ದಾರನನ್ನಾಗಿ ಮಾಡುವ ಕಾಲವು ಪ್ರಾಪ್ತವಾಗುವುದನ್ನು ಅವರು ಆಶಿಸುವರು. ಪ್ರಪಂಚವನ್ನು ಅಗಲುತ್ತಿರುವ ಬುದ್ಧ ಹೇಳಿದಂತೆ “ನಿಮ್ಮ ಉದ್ಧಾರವನ್ನು ನೀವೇ ಮಾಡಿಕೊಳ್ಳಬೇಕಾಗಿದೆ. ನಾನು ನಿಮಗೆ ಸಹಾಯ ಮಾಡಲಾರೆ. ಮತ್ತಾರೂ ನಿಮಗೆ ಸಹಾಯ ಮಾಡಲಾರರು. ನಿಮಗೆ ನೀವೆ ಸಹಾಯವನ್ನು ಮಾಡಿಕೊಳ್ಳಬೇಕಾಗಿದೆ” ಎಂದು ವಿವೇಕಾನಂದರು ಹೇಳಿದರು.

\begin{flushright}
\textbf{ಲೂಸಿ ಮನ್‍ರೋ}
\end{flushright}

\begin{center}
\textbf{ಪುನರ್ಜನ್ಮ}\\ (ಇವಾನ್​ಸ್ಟನ್​ ಇಂಡೆಕ್ಸ್​, ಅಕ್ಟೋಬರ್​ ೭, ೧೮೯೩)
\end{center}

\vskip -0.3cm

ಕಳೆದವಾರ ಕಾಂಗ್ರಗೇಶನಲ್​ ಚರ್ಚಿನಲ್ಲಿ ಕೆಲವು ಉಪನ್ಯಾಸಗಳನ್ನು ಕೊಟ್ಟರು. ಅಲ್ಲಿ ಕೊಟ್ಟ ಉಪನ್ಯಾಸಗಳು ಈಗತಾನೆ ಮುಕ್ತಾಯಗೊಂಡ ವಿಶ್ವಧರ್ಮ ಸಮ್ಮೇಳನದ ಉಪನ್ಯಾಸಗಳಂತೆ ಇದ್ದುವು. ಉಪನ್ಯಾಸಕರು ಸ್ವೀಡನ್ನಿನ ಡಾ~॥ ಕಾರನ್​ ವಾನ್​ ಬರ್ಗನ್​ ಮತ್ತು ಹಿಂದೂ ಸನ್ಯಾಸಿಯಾದ ಸ್ವಾಮಿ ವಿವೇಕಾನಂದರು. ಸ್ವಾಮಿ ವಿವೇಕಾನಂದರು ವಿಶ್ವಧರ್ಮ ಸಮ್ಮೇಳನಕ್ಕೆ ಬಂದ ಭರತಖಂಡದ ಪ್ರತಿನಿಧಿಗಳು. ಅವರು ತಮ್ಮ ವಿಚಿತ್ರವಾದ ಪೋಷಾಕು, ಆಕರ್ಷಕವಾದ ವ್ಯಕ್ತಿತ್ವ, ಪ್ರಖರವಾದ ವಾಗ್ಮಿತೆ ಮತ್ತು ಹಿಂದೂ ತತ್ತ್ವಗಳ ಅತಿ ಸುಂದರವಾದ ವಿವರಣೆ, ಇವುಗಳ ಮೂಲಕ ಜನರ ಮನಸ್ಸನ್ನು ಸೂರೆಗೊಂಡಿರುವರು. ಚಿಕಾಗೊ ನಗರದ ಜನರು ಅವರನ್ನು ಒಂದೇ ಸಮನಾಗಿ ಸ್ತುತಿಸುತ್ತಿರುವರು. ಮೂರು ದಿನ ಸಂಜೆ ಅವರ ಉಪನ್ಯಾಸಗಳನ್ನು ಏರ್ಪಡಿಸಿದ್ದರು.

(ಶನಿವಾರ ಮತ್ತು ಮಂಗಳವಾರ ಕೊಟ್ಟ ಉಪನ್ಯಾಸಗಳನ್ನು ಯಾವ ಟೀಕೆಯೂ ಇಲ್ಲದೆ ವರದಿಮಾಡಿದ್ದರು. ಅನಂತರ ಲೇಖನ ಹೀಗೆ ಮುಂದುವರಿದಿದೆ):

ಗುರುವಾರ ಸಾಯಂಕಾಲ ಅಕ್ಟೋಬರ್​ ೫ನೇ ತಾರೀಖು ವಾನ್​ಬರ್ಗನ್​ ಅವರು “ಹಲ್​ ಡೈನ್​ ಬೀಮಿಶ್​: ಸ್ವೀಡನ್ನಿನ ರಾಜಪುತ್ರಿಯರು ಎಂಬ ಸಂಸ್ಥೆಯ ಸ್ಥಾಪಕರು” ಎಂಬ ವಿಷಯದ ಮೇಲೆ ಮಾತನಾಡಿದರು. ಹಿಂದೂ ಸಂನ್ಯಾಸಿಯು (ಸ್ವಾಮಿ ವಿವೇಕಾನಂದರು) “ಪುನರ್ಜನ್ಮ” ಎಂಬ ವಿಷಯದ ಮೇಲೆ ಮಾತನಾಡಿದರು. ಉಪನ್ಯಾಸವು ಬಹಳ ಸೊಗ\break ಸಾಗಿತ್ತು. ಪ್ರಪಂಚದ ಈ ಭಾಗದ ಜನರಿಗೆ ಪರಿಚಿತವಲ್ಲದ ವಿಷಯವಾಗಿತ್ತು ಅದು. ಜೀವಿಯ ಪುನರ್ಜನ್ಮದ ಸಿದ್ಧಾಂತ ಈ ದೇಶದಲ್ಲಿ ಇತ್ತೀಚೆಗೆ ಬಂದ ಭಾವನೆ, ಮತ್ತು ಅದನ್ನು ತಿಳಿದುಕೊಂಡಿರುವವರು ಅತ್ಯಂತ ವಿರಳ ಮಂದಿಗಳು. ಆದರೆ ಪೌರಸ್ತ್ಯ ದೇಶದಲ್ಲಾದರೋ ಇದು ಚೆನ್ನಾಗಿ ಗೊತ್ತಿರುವ ವಿಷಯ. ಅಲ್ಲಿಯ ಬಹುಕಾಲ ಜನರ ಧರ್ಮದ ತಳಹದಿಯೇ ಅದು ಎಂದು ಬೇಕಾದರೂ ಹೇಳಬಹುದು. ಯಾರು ಅದನ್ನು ತಮ್ಮ ಮತದ ಸಿದ್ಧಾಂತ ಎಂದು ಸ್ವೀಕರಿಸುವುದಿಲ್ಲವೋ ಅವರು ಇದಕ್ಕೆ ವಿರೋಧವಾಗಿ ಏನನ್ನೂ ಹೇಳುವುದಿಲ್ಲ. ಈ ಸಿದ್ಧಾಂತದ ವಿಷಯದಲ್ಲಿ ನಾವು ಮುಖ್ಯವಾಗಿ ನಿಶ್ಚಯಿಸಬೇಕಾದುದೇನೆಂದರೆ ಹಿಂದಿನ ಜನ್ಮವಿತ್ತೇ ಎಂಬುದು. ಈಗ ನಮಗೊಂದು ಜನ್ಮವಿದೆ ಎಂದು ಗೊತ್ತಿದೆ. ಮತ್ತು ಮುಂದಿನ ನಮ್ಮ ಸ್ಥಿತಿಯನ್ನು ಕುರಿತು ಯಾವ ಸಂದೇಹವೂ ಇಲ್ಲ. ಆದರೂ ಭೂತಕಾಲ ಎನ್ನುವುದು ಇಲ್ಲದೆ ವರ್ತಮಾನ ಇರುವುದು ಹೇಗೆ ಸಾಧ್ಯ? ಆಧುನಿಕ ವಿಜ್ಞಾನ, ಭೌತವಸ್ತು ಈಗ ಇದೆ ಮತ್ತು ಮುಂದೆಯೂ ಇರುವುದು ಎಂದು ಸಿದ್ಧಾಂತಗೊಳಿಸಿದೆ. ಸೃಷ್ಟಿ ಎಂದರೆ ಆಕಾರದಲ್ಲಿ ಒಂದು ಬದಲಾವಣೆ, ಅಷ್ಟೆ. ನಾವು ಶೂನ್ಯದಿಂದ ಬರಲಿಲ್ಲ. ಕೆಲವರು ದೇವರೇ ಎಲ್ಲಕ್ಕೂ ಕಾರಣ ಎಂದು ನಂಬಿ, ಈಗ ಇರುವ ವಸ್ತುಗಳಿಗೆಲ್ಲ ಇದೇ ಸಾಕಷ್ಟು ಕಾರಣ ಎಂದು ನಿಸ್ಸಂದೇಹವಾಗಿ ಹೇಳುವರು. ಆದರೆ ನಾವು ಎಲ್ಲಾ ವಿಷಯಗಳಲ್ಲಿಯೂ ಕಾರ್ಯಕಾರಣ ಸಂಬಂಧವನ್ನು ಪರ್ಯಾಲೋಚಿಸಬೇಕಾಗಿದೆ. ಎಲ್ಲಿಂದ ಯಾವಾಗ ಭೌತವಸ್ತು ಬಂದಿತು ಎಂಬುದನ್ನು ತಿಳಿಯಬೇಕಾಗಿದೆ. ಭವಿಷ್ಯವಿದೆ ಎಂದು ಪ್ರತಿ ಪಾದಿಸುವ ವಾದಗಳೇ ಭೂತ ಒಂದು ಇತ್ತು ಎಂಬುದನ್ನು ಪ್ರತಿಪಾದಿಸುತ್ತವೆ. ದೈವೇಚ್ಛೆ ಎಂಬುದಲ್ಲದೆ ಬೇರೆ ಕಾರಣಗಳು ಇರಬೇಕಾದುದು ಆವಶ್ಯಕ. ಈ ವಿಷಯದಲ್ಲಿ ಆನುವಂಶಿಕತೆ ಸಾಕಷ್ಟು ಪ್ರಬಲವಾದ ಕಾರಣವನ್ನು ಒದಗಿಸಲು ಶಕ್ತವಾಗಿಲ್ಲ. ಕೆಲವರು ತಮಗೆ ಹಿಂದಿನ ಜನ್ಮದ ಪ್ರಜ್ಞೆ ಇಲ್ಲ ಎಂದು ಹೇಳುವರು. ಅನೇಕರಿಗೆ ತಮ್ಮ ಹಿಂದಿನ ಜೀವನದ ಸ್ಮರಣೆ ಇರುವ ಪ್ರಸಂಗಗಳಿವೆ. ಇಲ್ಲಿ ಒಂದು ಸಿದ್ಧಾಂತದ ಮೊಳಕೆಯಿದೆ. ಜನರು ಮೂಕಪ್ರಾಣಿಗಳಿಗೆ ದಯೆ ತೋರುವುದರಿಂದ, ಮನುಷ್ಯರು ಪ್ರಾಣಿಯಾಗಿ ಜನ್ಮ ಧರಿಸಿದ್ದಿರಬಹುದು ಎಂಬುದನ್ನು ಹಿಂದೂಗಳು ನಂಬುತ್ತಾರೆ. ಕೇವಲ ಮೂಢನಂಬಿಕೆಗಳ ವಿನಃ ಪ್ರಾಣಿಗಳಿಗೆ ದಯೆ ತೋರುವುದಕ್ಕೆ ಬೇರಾವುದಾದರೂ ಕಾರಣವಿರಬಹುದು ಎಂಬುದನ್ನು ಊಹಿಸುವುದು ಅವರಿಗೆ ಸಾಧ್ಯವಿಲ್ಲ. ಪುರಾತನ ಕಾಲದ ಹಿಂದೂ ಋಷಿಯೊಬ್ಬ ಯಾವುದು ನಮ್ಮನ್ನು ಉನ್ನತಿಗೆ ಏರಿಸುತ್ತದೆಯೋ ಅದೇ ಧರ್ಮ ಎನ್ನುವರು. ಮೃಗೀಯತೆಯನ್ನು ನಿವಾರಿಸಿದ ಮೇಲೆ ಬರುವ ಮಾನವತ್ವವು ದೇವತ್ವಕ್ಕೆ ಆಸ್ಪದವಾಗುವುದು. ಪುನರ್ಜನ್ಮ ಸಿದ್ಧಾಂತವು ಮನುಷ್ಯನು ಭೂಮಿಯಲ್ಲೇ ಜನ್ಮವೆತ್ತಬೇಕೆನ್ನುವುದಿಲ್ಲ. ಅವನ ಆತ್ಮವು ಇನ್ನೂ ಉತ್ತಮ ಜೀವಿಗಳಿರುವ ಲೋಕಗಳಿಗೆ ಹೋಗಬಹುದು. ಅಲ್ಲಿ ಪಂಚೇಂದ್ರಿಯಗಳ ಬದಲು ಎಂಟು ಇಂದ್ರಿಯಗಳನ್ನು ಪಡೆಯಬಹುದು. ಹೀಗೆಯೇ ಮುಂದುವರಿಯುತ್ತಾ ಹೋಗಿ ಪರಿಪೂರ್ಣತೆಯ ಶಿಖರವನ್ನು ಮುಟ್ಟಬಹುದು, ಮತ್ತು ಮುಕ್ತಾತ್ಮರಿರುವ ಲೋಕದ ಆನಂದವನ್ನು ಅನುಭವಿಸಲು ಹಕ್ಕುದಾರನಾಗಿ ಈ ಪ್ರಪಂಚವನ್ನು ಮರೆಯುವನು.

\delimiter

\begin{center}
\textbf{ಹಿಂದೂ ನಾಗರಿಕತೆ}
\end{center}

\vskip -0.3cm

(ಅಕ್ಟೋಬರ್​ ೯ನೇ ತಾರೀಖು ಸ್ಟ್ರೀಟರಿನಲ್ಲಿ ನಡೆದ ಉಪನ್ಯಾಸಕ್ಕೆ ಅನೇಕ ಜನರು ಬಂದಿದ್ದರೂ ಸ್ಟ್ರೀಟರ್​ ಡೈಲಿ ಫ್ರೀಪ್ರೆಸ್​ ಕೆಳಗಿನ ನೀರಸವಾದ ವರದಿಯನ್ನು ಪ್ರಕಟಿಸಿತು.)

ಅಪೆರಾ ಹೌಸಿನಲ್ಲಿ ಶನಿವಾರ ರಾತ್ರಿ ಪ್ರಖ್ಯಾತ ಹಿಂದೂವು ನೀಡಿದ ಉಪನ್ಯಾಸವು ಬಹಳ ಸ್ವಾರಸ್ಯಕರವಾಗಿತ್ತು. ಭಾಷಾಶಾಸ್ತ್ರಗಳ ತುಲನಾತ್ಮಕ ಅಧ್ಯಯನದಿಂದ, ಆರ್ಯರ ಮತ್ತು ಹೊಸ ಜಗತ್ತಿನಲ್ಲಿರುವ ಅವರ ಪೀಳಿಗೆಯವರ, ಬಹುಕಾಲದಿಂದಲೂ ಒಪ್ಪಿಕೊಂಡು ಬಂದಿರುವ ಪರಸ್ಪರ ಸಂಬಂಧವನ್ನು ಅವರು ಪುನಃ ಸಾಧಿಸಿದರು. ಭರತಖಂಡದಲ್ಲಿರುವ ವರ್ಣಾಶ್ರಮ ಪದ್ಧತಿ ಸರಿ ಎಂದು ಅವರು ಮೃದುವಾಗಿ ಪ್ರತಿಪಾದಿಸಿದರು. ಇದು ಅಲ್ಲಿಯ ಮುಕ್ಕಾಲುಪಾಲು ಜನರನ್ನು ಅತಿ ಹೀನಾಯವಾದ ಗುಲಾಮಗಿರಿಯಲ್ಲಿಟ್ಟಿದೆ. ಆಕಾಶದಲ್ಲಿ ಉಲ್ಕೆಗಳು ಕಾಣಿಸಿಕೊಂಡು ಮರೆಯಾಗುವಂತೆ, ಹಲವು ಶತಮಾನಗಳಿಂದ ಜಗತ್ತಿನ ರಾಷ್ಟ್ರಗಳ ಉನ್ನತಿ ಅವನತಿಗಳನ್ನು ಸಾಕ್ಷಿಯಂತೆ ನೋಡುತ್ತಿರುವ ಅಂದಿನ ಭಾರತವೇ ಇಂದೂ ಇದೆ ಎಂದು ಅವರು ಹೆಮ್ಮೆಯಿಂದ ಹೇಳಿದರು. ತಮ್ಮ ಜನರಂತೆ ಸ್ವಾಮೀಜಿ ಹಿಂದಿನದನ್ನು ಪ್ರೀತಿಸುತ್ತಾರೆ. ಅವರು ತಮ್ಮ ಸ್ವಾರ್ಥಕ್ಕಾಗಿ ಜೀವಿಸಿಲ್ಲ, ದೇವರಿಗಾಗಿ ಜೀವಿಸಿರುವರು. ಅವರು ದೇಶದಲ್ಲಿ ಭಿಕ್ಷಾಟನೆ ಮತ್ತು ಪರ್ಯಟನೆಗೆ ಹೆಚ್ಚು ಪ್ರಾಧಾನ್ಯ ನೀಡುತ್ತಾರೆ. ಅಲ್ಲಿ ಅವರೇನೂ ಉಪನ್ಯಾಸ ಕೊಡುವುದರಲ್ಲಿ ಅಷ್ಟೇನು ಪ್ರಖ್ಯಾತರಾಗಿರಲಿಲ್ಲ. ಮನೆಯಲ್ಲಿ ಅಡುಗೆಯು ಮುಗಿದ ನಂತರ, ಬರುವ ಯಾರಿಗಾದರೂ ಕಾದುಕೊಂಡಿದ್ದು, ಅವನು ಬಂದ ಮೇಲೆ, ಅವನಿಗೆ ಅವರು (ಭಾರತೀಯರು) ಮೊದಲು ಬಡಿಸುವರು. ಅನಂತರ ಪ್ರಾಣಿಗಳು, ಮನೆಯ ಆಳುಗಳು ನಂತರ ಮನೆಯ ಗಂಡಸರು, ಕೊನೆಗೆ ಹೆಂಗಸರು ಊಟಮಾಡುವರು. ಹುಡುಗರಿಗೆ ಹತ್ತು ವರುಷಗಳಾದಾಗ ಅವರನ್ನು ಕರೆದುಕೊಂಡು ಹೋಗಿ ಒಬ್ಬ ಗುರುವಿನ ಮನೆಯಲ್ಲಿ ಬಿಟ್ಟು, ಹತ್ತರಿಂದ ಇಪ್ಪತ್ತು ವರುಷಗಳವರೆಗೆ ಅವರನ್ನು ಅಲ್ಲಿ ಇರುವಂತೆ ಮಾಡುವರು. ಅನಂತರ ಅವರು ತಮ್ಮ ಹಿಂದಿನ ಉದ್ಯೋಗವನ್ನು ಅನುಸರಿಸಲು ಹಿಂದಿರುಗುವರು. ಇಲ್ಲವೆ ಕೊನೆಮೊದಲಿಲ್ಲದೆ ಸಂಚರಿ\break ಸುತ್ತ, ಬೋಧಿಸುತ್ತ, ಪ್ರಾರ್ಥಿಸುತ್ತ ಕಾಲಕಳೆಯುವರು. ತಿನ್ನುವುದಕ್ಕೆ ತಮಗೆ ಜನರು ಏನನ್ನು ಕೊಡುವರೊ ಅದನ್ನು ಮಾತ್ರ ತೆಗೆದುಕೊಳ್ಳುವರು. ಹಣವನ್ನು ಮುಟ್ಟುವುದೇ ಇಲ್ಲ. ವಿವೇಕಾನಂದರು ಈ ಗುಂಪಿಗೆ ಸೇರಿದವರು. ವಯಸ್ಸಾದ ಗಂಡಸರು ಪ್ರಪಂಚವನ್ನು ತ್ಯಜಿಸಿ, ಕೆಲವು ಕಾಲ ಧ್ಯಾನ ಮತ್ತು ಅಧ್ಯಯನದಲ್ಲಿ ಕಳೆದಾದ ಮೇಲೆ ತಾವು ಪರಿಶುದ್ಧರಾದೆ\break ವೆಂದು ಭಾವಿಸುವರು. ಅನಂತರ ಅವರು ಕೂಡ ಬೋಧಿಸುತ್ತ ಹೋಗುವರು. ಬೌದ್ಧಿಕ ಬೆಳವಣಿಗೆಗೆ ಬೇಕಾದಷ್ಟು ವಿರಾಮವಿರಬೇಕು ಎಂದು ವಿವೇಕಾನಂದರು ಹೇಳಿದರು. ಯಾರನ್ನು ಕೊಲಂಬಸನು ಅನಾಗರಿಕ ಸ್ಥಿತಿಯಲ್ಲಿರುವುದನ್ನು ಕಂಡನೊ ಆ ರೆಡ್​ ಇಂಡಿಯನ್ನ\break ರನ್ನು ವಿದ್ಯಾವಂತರನ್ನಾಗಿ ಮಾಡದಿದ್ದುದಕ್ಕಾಗಿ ಅಮೆರಿಕಾ ದೇಶದವರನ್ನು ದೂರಿದರು. ಇಲ್ಲಿ ಅವರಿಗೆ ಈ ವಿಷಯದಲ್ಲಿರುವ ಅಜ್ಞಾನವನ್ನು ವ್ಯಕ್ತ ಗೊಳಿಸಿಕೊಂಡರು. ಅವರ ಉಪನ್ಯಾಸವು ಅತ್ಯಂತ ಸಂಕ್ಷಿಪ್ತವಾಗಿದ್ದುದು ವಿಶಾದನೀಯ. ಎಷ್ಟೋ ವಿಷಯಗಳನ್ನು ಅವರು ಹೇಳಲೇ ಇಲ್ಲ. ಹೇಳಿದುದಕ್ಕಿಂತ ಇನ್ನೂ ಮುಖ್ಯವಾದ ವಿಷಯಗಳೂ ಎಷ್ಟೋ ಇದ್ದವು. ಅವುಗಳನ್ನು ಕುರಿತು ಇವರು ಮಾತನಾಡಲಿಲ್ಲ. (ಈ ವರದಿಯಿಂದ, ಅಮೆರಿಕದ ವೃತ್ತಪತ್ರಿಕೆಗಳು, ಯಾವುದಾದರೂ ಕಾರಣದಿಂದಾಗಿ, ಯಾವಾಗಲೂ ಸ್ವಾಮೀಜಿಯವರ ವಿಷಯದಲ್ಲಿ ತೋರಿಸಬೇಕಾದಷ್ಟು ಉತ್ಸಾಹವನ್ನು ತೋರಿಸಲಿಲ್ಲ ಎಂಬುದು ಸ್ಪಷ್ಟವಾಗುತ್ತದೆ.)

\begin{center}
\textbf{ಒಂದು ಸ್ವಾರಸ್ಯವಾದ ಉಪನ್ಯಾಸ}\\ (ವಿಸ್​ಕಾನ್​ಸಿನ್​ ಸ್ಟೇಟ್​ ಜರ್ನಲ್​, ನವೆಂಬರ್​ ೨೧, ೧೮೯೩)
\end{center}

ಕಳೆದ ರಾತ್ರಿ ಮಾಡಿಸನ್​ನಲ್ಲಿರುವ ಕಾಂಗ್ರಗೇಶನಲ್​ಚರ್ಚಿನಲ್ಲಿ ಪ್ರಖ್ಯಾತ ವ್ಯಕ್ತಿಗಳಾದ ವಿವೇಕಾನಂದರು ಕೊಟ್ಟ ಉಪನ್ಯಾಸ ಬಹಳ ಸ್ವಾರಸ್ಯಕರವಾಗಿತ್ತು. ಆ ಉಪನ್ಯಾಸದಲ್ಲಿ ಉತ್ತಮ ತಾತ್ತ್ವಿಕ ಮತ್ತು ಧಾರ್ಮಿಕ ಭಾವನೆಗಳು ಇದ್ದುವು. ಕ್ರೈಸ್ತರು ಕ್ರೈಸ್ತೇತರರಾದ ಅವರ ಹಲವು ಬೋಧನೆಗಳನ್ನು ಅನುಸರಿಸಬಹುದು. ವಿಶ್ವದಷ್ಟು ವಿಶಾಲವಾಗಿದೆ ಅವರ ಮತ. ಅದು ಎಲ್ಲಾ ಧರ್ಮಗಳನ್ನೂ ಒಳಗೊಳ್ಳುತ್ತದೆ, ಸತ್ಯ ಎಲ್ಲಿದ್ದರೂ ಚಿಂತೆಯಿಲ್ಲ ಅದನ್ನು ಸ್ವೀಕರಿಸುವುದು. ಭರತಖಂಡದ ಧರ್ಮಗಳಲ್ಲಿ, ಮತಾಂಧತೆ ಮೂಢನಂಬಿಕೆ ಮತ್ತು ಕೆಲಸಕ್ಕೆ ಬಾರದ ಆಚಾರಗಳು ಇಲ್ಲ ಎಂದು ಅವರು ಸಾರಿದರು.

\delimiter

\begin{center}
\textbf{ಹಿಂದೂಧರ್ಮ}\\ (ಮಿನಿಯಪೊಲಿಸ್​ ಸ್ಟಾರ್​, ನವೆಂಬರ್​ ೨೫, ೧೮೯೩)
\end{center}

\vskip -0.3cm

ಸ್ವಾಮಿ ವಿವೇಕಾನಂದರು ಕಳೆದ ರಾತ್ರಿ ಮಿನಿಯಪೊಲಿಸ್​ನಲ್ಲಿ ಇರುವ ಪ್ರಥಮ ಯೂನಿಟೇರಿಯನ್​ ಚರ್ಚಿನಲ್ಲಿ ಹಿಂದೂಧರ್ಮದ ಮೇಲೆ ಮಾತನಾಡಿದರು. ಸನಾತನವೂ ಸತ್ಯಪೂರ್ಣವೂ ಆದ ತತ್ತ್ವಗಳನ್ನೊಳಗೊಂಡಿರುವ ಬ್ರಾಹ್ಮಣಧರ್ಮವನ್ನು ಆಕರ್ಷಕವಾದ ರೀತಿಯಲ್ಲಿ ವಿವರಿಸಿದರು. ಉಪನ್ಯಾಸಕ್ಕೆ ನೆರೆದವರಲ್ಲಿ ಮೇಧಾವಿಗಳಾದ ಸ್ತ್ರೀಪುರುಷರು ಇದ್ದರು. ಏಕೆಂದರೆ ಪೆರಿಪಾಟಿಕ್​ ಸಂಘದವರು ಸ್ವಾಮೀಜಿ ಅವರನ್ನು ಕರೆಸಿದ್ದರು. ಈ ಅವಕಾಶದಲ್ಲಿ ಭಾಗಿಗಳಾದ ಸ್ನೇಹಿತರಲ್ಲಿ ಕೆಲವರು ವಿವಿಧ ಮತದ ಕ್ರೈಸ್ತಪಾದ್ರಿಗಳೂ, ವಿದ್ಯಾರ್ಥಿಗಳೂ ಮತ್ತು ಪಂಡಿತರೂ ಇದ್ದರು. ವಿವೇಕಾನಂದರು ಬ್ರಾಹ್ಮಣ ಸಂನ್ಯಾಸಿಗಳು. ಅವರು ತಮ್ಮ ದೇಶೀಯ ವೇಷದಲ್ಲಿ ಉಪನ್ಯಾಸ ಮಾಡಿದರು. ಅವರು ತಲೆಗೆ ಒಂದು ರುಮಾಲನ್ನು ಕಟ್ಟಿಕೊಂಡಿದ್ದರು, ದೇಹದ ಮೇಲೆ ಗೈರಿಕ ನಿಲುವಂಗಿ ಇತ್ತು. ಅದನ್ನು ಸೊಂಟದ ಮೇಲೆ ಕೆಂಪು ವಸ್ತ್ರದಿಂದ ಕಟ್ಟಿಕೊಂಡಿದ್ದರು. ಅದರ ಕೆಳಗೆ ಬಣ್ಣದ ಷರಾಯಿಯನ್ನು ಹಾಕಿಕೊಂಡಿದ್ದರು.

ಅವರು ತಮ್ಮ ಧರ್ಮದ ವಿಷಯವನ್ನು ಬಹಳ ಶ್ರದ್ಧೆಯಿಂದ ವಿವರಿಸಿದರು. ಅವರು ನಿಧಾನವಾಗಿ ಸ್ಪಷ್ಟವಾಗಿ ಮಾತನಾಡುತ್ತಿದ್ದರು. ವೇಗವಾದ ಚಲನ ವಲನಗಳಿಗೆ ಅವಕಾಶ ಕೊಡದೆ ಮಾತಿನ ಮಾಧುರ್ಯದಿಂದ ತಮ್ಮ ಪ್ರಭಾವವನ್ನು ಬೀರಿದರು. ಅವರು ಪ್ರತಿಯೊಂದು ಶಬ್ದವನ್ನೂ ತೂಗಿ ಉಚ್ಚರಿಸುತ್ತಿದ್ದರು. ಆಡಿದ ಪ್ರತಿಯೊಂದು ಮಾತು ಎಲ್ಲರ ಹೃದಯಗಳಿಗೂ ತಾಕುವಂತೆ ಇತ್ತು. ಹಿಂದೂ ಧರ್ಮದ ಸರಳವಾದ ತತ್ತ್ವಗಳನ್ನು ಅವರು ವಿವರಿಸಿದರು. ಕ್ರೈಸ್ತಧರ್ಮಕ್ಕೆ ವಿರೋಧವಾಗಿ ಏನನ್ನೂ ಹೇಳದಿದ್ದರೂ ಎಲ್ಲರ ಮುಂದೆ ಬ್ರಾಹ್ಮಣಧರ್ಮವನ್ನು ಪ್ರತಿಪಾದಿಸುವುದಕ್ಕೆ ಎಷ್ಟನ್ನು ಉದಾಹರಿಸಬೇಕೋ ಅಷ್ಟನ್ನು ಮಾತ್ರ ಕ್ರೈಸ್ತಧರ್ಮದಿಂದ ತೆಗೆದುಕೊಂಡರು. ಆತ್ಮವು ಸ್ವಭಾವತಃ ಪವಿತ್ರವಾದುದು ಎಂಬುದು ಹಿಂದೂಧರ್ಮದಲ್ಲಿ ಸರ್ವಾದರಣೀಯವೂ ಅತ್ಯಂತ ಪ್ರಮುಖವೂ ಆದ ಭಾವನೆ. ಆತ್ಮ ಸ್ವಭಾವತಃ ಪರಿಪೂರ್ಣವಾದದು; ಧರ್ಮವೆಂದರೆ ಆಗಲೆ ಮನುಷ್ಯನಲ್ಲಿ ಇರುವ ಪವಿತ್ರತೆಯನ್ನು ವ್ಯಕ್ತಗೊಳಿಸುವುದು ಮಾತ್ರವಾಗಿದೆ. ಈಗಿನ ಸ್ಥಿತಿ ಮನುಷ್ಯನ ಹಿಂದಿನ ಮತ್ತು ಮುಂದಿನ ಸ್ಥಿತಿಯನ್ನು ಬೇರ್ಪಡಿಸುವ ಒಂದು ಎಲ್ಲೆಯಂತೆ ಇದೆ. ಅವನಲ್ಲಿರುವ ಎರಡು ಸ್ವಭಾವಗಳಲ್ಲಿ ಒಳ್ಳೆಯದು ಹೆಚ್ಚಾಗಿದ್ದರೆ ಅವನು ಉತ್ತಮ ಲೋಕಕ್ಕೆ ಹೋಗುವನು, ಕೆಟ್ಟದ್ದು ಹೆಚ್ಚಾಗಿದ್ದರೆ, ಅಧೋಗತಿಗೆ ಇಳಿಯುವನು. ಇವೆರಡು ಸ್ವಭಾವ\break ಗಳೂ ಯಾವಾಗಲೂ ಮನುಷ್ಯನಲ್ಲಿ ಕೆಲಸ ಮಾಡುತ್ತಿವೆ. ಯಾವುದು ಅವನನ್ನು ಮೇಲೆತ್ತು\break ವುದೋ ಅದೇ ಪುಣ್ಯ. ಯಾವುದು ಅವನನ್ನು ಅಧೋಗತಿಗೆ ಒಯ್ಯುವುದೋ ಅದೇ ಪಾಪ.

ವಿವೇಕಾನಂದರು ಪ್ರಥಮ ಯೂನಿಟೇರಿಯನ್​ ಚರ್ಚಿನಲ್ಲಿ ನಾಳೆ ಬೆಳಿಗ್ಗೆ ಭಾಷಣ ಮಾಡುವರು.

\delimiter

\section[ಧರ್ಮಕ್ಷೇತ್ರದಲ್ಲಿ ವ್ಯಾಪಾರದ ಸ್ವಭಾವದವರು]{ಧರ್ಮಕ್ಷೇತ್ರದಲ್ಲಿ ವ್ಯಾಪಾರದ ಸ್ವಭಾವದವರು\protect\footnote{* C.W. Vol. VII P.- 416}}

\begin{center}
(೧೮೯೩, ನವೆಂಬರ್​ ೨೬ ರಂದು ಮಿನಿಯಪೊಲಿಸ್​ನಲ್ಲಿ ಮಾಡಿದ ಭಾಷಣದ ಸಾರಾಂಶ, ಮಿನಿಯಪೊಲಿಸ್​ ಜರ್ನಲ್​ನಲ್ಲಿ ವರದಿಯಾದದ್ದು)
\end{center}

\vskip -0.45cm

ನಿನ್ನೆ ಬೆಳಿಗ್ಗೆ ಯೂನಿಟೇರಿಯನ್​ ಚರ್ಚ್​ ಸಭಿಕರಿಂದ ಕಿಕ್ಕಿರಿದಿತ್ತು. ಸ್ವಾಮಿ ವಿವೇಕಾನಂದ\break ರೆನ್ನುವ ಬ್ರಾಹ್ಮಣ ಸಂನ್ಯಾಸಿಯು ಪೌರಸ್ತ್ಯ ಧರ್ಮದ ಬಗ್ಗೆ ಮಾತನಾಡುವುದನ್ನು ಕೇಳುವುದಕ್ಕೆ ಅವರೆಲ್ಲ ನೆರೆದಿದ್ದರು. ಅವರು ಕಳೆದ ಬೇಸಗೆಯಲ್ಲಿ ಚಿಕಾಗೊ ನಗರದ ವಿಶ್ವಧರ್ಮ ಸಮ್ಮೇಳನದಲ್ಲಿ ಭಾಗವಹಿಸಿದ ಪ್ರಮುಖ ವ್ಯಕ್ತಿಗಳಾಗಿದ್ದರು. ಪೆರಿಪೇಟಿಟಿಕ್​ ಕ್ಲಬ್ಬಿನವರು ಬ್ರಾಹ್ಮಣಧರ್ಮದ ಅ ಸುಪ್ರಸಿದ್ಧ ಪ್ರತಿನಿಧಿಯನ್ನು ಮಿನಿಯಪೊಲಿಸ್​ಗೆ ಕರೆದುಕೊಂಡು ಬಂದಿದ್ದರು. ಅವರು ಕಳೆದ ಶುಕ್ರವಾರ ಆ ಕ್ಲಬ್ಬಿನ ಆಶ್ರಯದಲ್ಲಿ ಮಾತನಾಡಿದ್ದರು. ನಿನ್ನೆಯ ಉಪನ್ಯಾಸವನ್ನು ನೀಡುವುದಕ್ಕಾಗಿ ಅವರನ್ನು ವಾರದ ಕೊನೆಯವರೆಗೂ ಇಲ್ಲಿಯೇ ಇರಬೇಕೆಂದು ಕೋರಿಕೊಂಡಿದ್ದರು.

ಡಾಕ್ಟರ್​ ಎಚ್​.ಎಂ. ಸೈಮನ್​ ಅವರು ಶ್ರದ್ಧೆ, ಭರವಸೆ, ಪ್ರೀತಿ, ಮುಂತಾದುವುಗಳ ಮೇಲೆ ಸಂತಪಾಲನ ಬೋಧನೆಗಳನ್ನು ಓದಿ, ಇವುಗಳಲ್ಲೆಲ್ಲಾ ತುಂಬಾ ಶ್ರೇಷ್ಠವಾದುದೇ ಪ್ರೀತಿ ಎಂದರು. ಅವರು ಮೇಲಿನದನ್ನು ಓದಿ ಆದಮೇಲೆ ಇತರ ಧರ್ಮ ಗ್ರಂಥಗಳಿಂದಲೂ ಕೆಲವು ಭಾಗಗಳನ್ನು ಓದಿದರು. ಅದೇ ವಿಷಯವನ್ನು ಬೋಧಿಸುವ ಬ್ರಾಹ್ಮಣ ಶಾಸ್ತ್ರಗಳಿಂದ, ಮುಸ್ಲಿಂ ಧರ್ಮದಿಂದ ಕೆಲವು ಭಾಗಗಳನ್ನೂ, ಹಿಂದೂ ಸಾಹಿತ್ಯದಿಂದ ಕೆಲವು ಪದ್ಯಗಳನ್ನೂ ಓದಿದರು. ಪಾಲನ ಬೋಧನೆಗೂ ಇವುಗಳಿಗೂ ಯಾವ ಭಿನ್ನಾಭಿಪ್ರಾಯವೂ ಇರಲಿಲ್ಲ.

ಎರಡನೆಯ ಪ್ರಾರ್ಥನೆಯನ್ನು ಓದಿ ಆದಮೇಲೆ ಸ್ವಾಮಿ ವಿವೇಕಾನಂದರನ್ನು\break ಪರಿಚಯಿಸಲಾಯಿತು. ಅವರು ವೇದಿಕೆಯನ್ನೇರಿದೊಡನೆಯೆ ಹಿಂದೂ ಕಥೆಯೊಂದನ್ನು ಹೇಳಿ ಸಭಿಕರನ್ನು ಮುಗ್ಧರನ್ನಾಗಿ ಮಾಡಿದರು. ಅವರು ಸುಂದರವಾದ ಇಂಗ್ಲಿಷ್​ ಭಾಷೆಯಲ್ಲಿ ಮಾತನಾಡಿದರು:

“ನಾನು ನಿಮಗೆ ಐದು ಜನ ಕುರುಡರನ್ನು ಕುರಿತು ಒಂದು ಕಥೆಯನ್ನು ಹೇಳುತ್ತೇನೆ. ಭಾರತ ದೇಶದ ಒಂದು ಗ್ರಾಮದಲ್ಲಿ ಒಂದು ಮೆರವಣಿಗೆ ನಡೆಯಿತು. ಊರಿನ ಜನರೆಲ್ಲ ಮೆರವಣಿಗೆಯನ್ನು ನೋಡಲು ನೆರೆದಿದ್ದರು. ಆ ಮೆರವಣಿಗೆಯಲ್ಲಿ ಒಂದು ಅಲಂಕೃತವಾದ ಆನೆಯಿತ್ತು. ಜನರಿಗೆ ಇದನ್ನು ನೋಡಿ ಸಂತೋಷವಾಯಿತು. ಆದರೆ ಐದು ಜನ ಕುರುಡರಿಗೆ ನೋಡುವುದಕ್ಕೆ ಆಗಲಿಲ್ಲ. ಅದರ ಆಕಾರ ಹೇಗಿರುವುದೋ ಎಂಬುದನ್ನು ತಿಳಿದುಕೊಳ್ಳುವುದಕ್ಕಾಗಿ ಅದನ್ನು ಮುಟ್ಟಲು ಅವರು ಯತ್ನಿಸಿದರು. ಅದಕ್ಕೆ ಸಮ್ಮತಿಯನ್ನು ಆನೆಯ ಮಾಲೀಕರು ಕೊಟ್ಟರು. ಮೆರವಣಿಗೆ ಹೊರಟು ಹೋದ ಮೇಲೆ ಈ ಐದು ಜನ ಕುರುಡರು ಜನರೊಡನೆ ತಮ್ಮ ಹಳ್ಳಿಗೆ ಹಿಂತಿರುಗುವಾಗ ಆನೆಯ ವಿಷಯವನ್ನು ಮಾತನಾಡಿಕೊಂಡರು. ಒಬ್ಬ ಕುರುಡ “ಆನೆ ಒಂದು ಗೋಡೆಯಂತೆ ಇತ್ತು” ಎಂದ. “ಇಲ್ಲ ಹಾಗಲ್ಲ ಇದ್ದುದ್ದು, ಅದು ಒಂದು ಹಗ್ಗದ ತುಂಡಿನಂತೆ ಇತ್ತು” ಎಂದ ಮತ್ತೊಬ್ಬ. ಮೂರನೆಯವನು, “ಇಲ್ಲ, ನೀನು ಹೇಳುವುದು ತಪ್ಪು, ನಾನು ಅದನ್ನು ಮುಟ್ಟಿ ನೋಡಿದೆ; ಅದೊಂದು ದೊಡ್ಡ ಹಾವಿನಂತೆ ಇತ್ತು” ಎಂದನು. ಮಾತಿಗೆ ಸ್ವಲ್ಪ ಕಾವೇರಿತು. ನಾಲ್ಕನೆಯವನು “ಆನೆ ಒಂದು ದಿಂಬಿನಂತೆ ಇದ್ದಿತು” ಎಂದ. ಮಾತು ಕೊನೆಗೆ ಜಗಳದಲ್ಲಿ ಪರ್ಯವಸಾನವಾಯಿತು. ಐದು ಕುರುಡರು ಕಾದಾಡುವುದಕ್ಕೆ ಪ್ರಾರಂಭಿಸಿದರು. ಅಲ್ಲಿಗೆ ದೃಷ್ಟಿ ಸರಿಯಾಗಿದ್ದ ಒಬ್ಬ ಮನುಷ್ಯ ಬಂದು “ಸ್ನೇಹಿತರೆ, ಏತಕ್ಕೆ ಜಗಳಕಾಯುತ್ತಿರುವಿರಿ?” ಎಂದು ಕೇಳಿದನು. ಅವನಿಗೆ ಅವರು ತಮ್ಮ ವಿವಾದವನ್ನೆಲ್ಲಾ ವಿವರಿಸಿದರು. ಹೊಸಬ ಹೇಳಿದ: “ನೀವೆಲ್ಲ ಸರಿ. ನಿಮ್ಮ ತೊಂದರೆ ಏನು ಎಂದರೆ ನೀವೆಲ್ಲ ಆನೆಯ ಬೇರೆ ಬೇರೆ ಭಾಗಗಳನ್ನು ಮುಟ್ಟಿದ್ದೀರಿ. ಗೋಡೆಯಂತೆ ಇದ್ದುದೇ ಅದರ ಪಕ್ಕಗಳು, ಹಗ್ಗದಂತೆ ಇದ್ದುದೇ ಅದರ ಬಾಲ, ಹಾವಿನಂತೆ ಇದ್ದುದು ಅದರ ಸೊಂಡಿಲು, ಅದರ ಕಾಲಿನ ಅಡಿಗಳೇ ದಿಂಬಿನಂತೆ ಇದ್ದುದು. ನಿಮ್ಮ ಜಗಳವನ್ನು ನಿಲ್ಲಿಸಿ. ನೀವೆಲ್ಲ ಸರಿ. ಆದರೆ ನೀವು ಆನೆಯನ್ನು ಬೇರೆ ಬೇರೆ ದೃಷ್ಟಿಯಿಂದ ನೋಡುತ್ತಿರುವಿರಿ.”

ಧರ್ಮ ಇಂದು ಇಂತಹ ಜಗಳದಲ್ಲಿ ಸಿಕ್ಕಿಹಾಕಿಕೊಂಡಿದೆ ಎಂದರು ಸ್ವಾಮೀಜಿ. ಪಾಶ್ಚಾತ್ಯ ದೇಶದವರು ತಮ್ಮದೊಂದೇ ನಿಜವಾದ ಧರ್ಮ ಎಂದು ಭಾವಿಸುವರು. ಪೌರಸ್ತ್ಯರು ಕೂಡ ಹಾಗೆಯೇ ಭಾವಿಸುವರು. ಇಬ್ಬರದೂ ತಪ್ಪೇ. ದೇವರು ಎಲ್ಲಾ ಧರ್ಮಗಳಲ್ಲಿಯೂ ಇರುವನು.

ಪಾಶ್ಚಾತ್ಯ ಚಿಂತನೆಯನ್ನು ಕುರಿತಂತೆ ಹಲವು ವಿಚಾರಪೂರ್ಣವಾದ ಟೀಕೆಗಳು ಅವರ ಉಪನ್ಯಾಸದಲ್ಲಿ ಇದ್ದವು. ಕ್ರೈಸ್ತರ ಧರ್ಮವನ್ನು ಕುರಿತು ಅದೊಂದು ವ್ಯಾಪಾರಿಯ ಧರ್ಮ ಎಂದು ಹೇಳಿದರು. ಅವರು ಯಾವಾಗಲೂ ದೇವರನ್ನು “ದೇವರೆ ನಮಗೆ ಅದನ್ನು ಕೊಡು, ಇದನ್ನು ಕೊಡು, ಅದನ್ನು ಮಾಡು, ಇದನ್ನು ಮಾಡು” ಎಂದು ಗೋಗರೆಯುವರು. ಹಿಂದೂಗಳಿಗೆ ಇದರ ಅರ್ಥವಾಗಲಿಲ್ಲ. ದೇವರನ್ನು ಹಾಗೆ ಬೇಡುವುದು ತಪ್ಪು ಎಂದು ಅವರು ಭಾವಿಸುವರು. ಒಬ್ಬ ನಿಜವಾಗಿ ಧಾರ್ಮಿಕ ವ್ಯಕ್ತಿಯಾದರೆ ಬೇಡುವ ಬದಲು ಕೊಡಬೇಕು. ಹಿಂದುವು ದೇವರಿಗೆ ಮತ್ತು ಅವನ ಭಕ್ತರಿಗೆ ಕೊಡುವುದನ್ನು ನಂಬುವನು. ದೇವರಿಂದ ಏನನ್ನೂ ಕೇಳ ಬಯಸುವುದಿಲ್ಲ. ಪಾಶ್ಚಾತ್ಯರಲ್ಲಿ ಅನೇಕ ಜನ, ಎಲ್ಲಿಯವರೆಗೆ ತಮಗೆ ಬೇಕಾದುದೆಲ್ಲವೂ ದೊರಕುವುದೋ ಅಲ್ಲಿಯವರೆಗೆ ದೇವರನ್ನು ಕೊಂಡಾಡುವರು; ಎಂದು ಅವರು ಕೇಳುವುದು ಸಿಕ್ಕುವುದಿಲ್ಲವೋ ಆಗ ದೇವರನ್ನು ಮರೆಯುವರು. ಆದರೆ ಹಿಂದುವು ಹಾಗಲ್ಲ. ಅವನು ದೇವರನ್ನು ಪ್ರೇಮಮೂರ್ತಿಯಾಗಿ ನೋಡುತ್ತಾನೆ. ಹಿಂದೂಧರ್ಮವು ದೇವರನ್ನು ತಾಯಿಯಂತೆ ಮತ್ತು ತಂದೆಯಂತೆಯೂ ನೋಡುವುದು. ಏಕೆಂದರೆ ಮಾತೃಭಾವ ಪ್ರೀತಿಗೆ ಹೆಚ್ಚು ಹತ್ತಿರವಾಗಿದೆ. ಪಾಶ್ಚಾತ್ಯದೇಶದಲ್ಲಿ ಕ್ರೈಸ್ತರು ವಾರವೆಲ್ಲ ಕಷ್ಟಪಟ್ಟು ಕೆಲಸಮಾಡಿ ದುಡ್ಡು ಬಂದ ಮೇಲೆ ದೇವರಿಗೆ ಧನ್ಯವಾದವನ್ನು ಅರ್ಪಿಸಿ ಹಣವನ್ನು ತನ್ನ ಜೇಬಿಗೆ ಇಳಿಬಿಡುವರು. ಹಿಂದುವು ಹಣವನ್ನುಗಳಿಸಿ ಅದನ್ನು ದೀನದರಿದ್ರರ ಸೇವೆಗೆ ಉಪಯೋಗಿಸಿ, ಅದರ ಮೂಲಕ ದೇವರಿಗೆ ಅದನ್ನು ಅರ್ಪಿಸುವನು. ಹೀಗೆ ಪೌರಸ್ತ್ಯ ಮತ್ತು ಪಾಶ್ಚಾತ್ಯ ಭಾವನೆಗಳ ಹೋಲಿಕೆಯನ್ನು ಮಾಡಿದರು. ದೇವರ ವಿಷಯವಾಗಿ ಮಾತನಾಡುತ್ತಾ ವಿವೇಕಾನಂದರು ಒಟ್ಟಿನಲ್ಲಿ ಹೀಗೆ ಹೇಳಿದರು: “ಪಾಶ್ಚಾತ್ಯರಾದ ನೀವು ನಿಮಗೆ ಒಬ್ಬ ದೇವರು ಇರುವನು ಎಂದು ಭಾವಿಸುವಿರಿ. ದೇವರು ಇರುವುದೆಂದರೆ ಏನು? ನಿಜವಾಗಿ ದೇವರು ನಿಮ್ಮಲ್ಲಿ ಇರುವುದಾದರೆ ನಿಮ್ಮಲ್ಲಿ ಏಕೆ ಇಷ್ಟೊಂದು ದುಷ್ಕಾರ್ಯಗಳು ನಡೆಯುವುದು? ಹತ್ತರಲ್ಲಿ ಒಂಭತ್ತು ಜನ ಆಷಾಢಭೂತಿಗಳು ಆಗಿರುವುದು ಏಕೆ? ದೇವರು ಇರುವ ಕಡೆ ಆಷಾಢಭೂತಿತನ ಇರಲಾರದು. ನಿಮಗೆ ದೇವರನ್ನು ಪೂಜಿಸುವುದಕ್ಕೆ ದೊಡ್ಡ ದೊಡ್ಡ ಅರಮನೆಗಳು ಇವೆ. ನೀವು ವಾರಕ್ಕೆ ಒಂದು ಸಲ ಸ್ವಲ್ಪ ಹೊತ್ತು ಅಲ್ಲಿಗೆ ಹೋಗುವಿರಿ. ಆದರೆ ಎಷ್ಟು ಕಡಿಮೆ ಮಂದಿ ದೇವರನ್ನು ಪೂಜಿಸುವುದಕ್ಕೆ ಹೋಗುವರು? ಪಾಶ್ಚಾತ್ಯದೇಶಗಳಲ್ಲಿ ಚರ್ಚಿಗೆ ಹೋಗುವುದು ಒಂದು ಪ್ಯಾಷನ್. ಹಲವರು ಈ ಕಾರಣದಿಂದಲೇ ಅಲ್ಲಿಗೆ ಹೋಗುವರು. ಹಾಗಿರುವಾಗ ಪಾಶ್ಚಾತ್ಯರಾದ ನಿಮಗೆ, ದೇವರು ನಿಮಗೆ ಮಾತ್ರ ಮೀಸಲಾಗಿರುವನು ಎಂದು ಹೇಳುವುದಕ್ಕೆ ಏನು ಅಧಿಕಾರವಿದೆ?''

ಈ ಸಮಯದಲ್ಲಿ ಸಭಿಕರು ಸ್ವಪ್ರೇರಣೆಯಿಂದ ಕರತಾಡನ ಮಾಡಿದರು. ವಿವೇಕಾ\break ನಂದರು ಭಾಷಣವನ್ನು ಮುಂದುವರಿಸಿದರು. “ಹಿಂದೂಗಳಾದ ನಾವು ಪ್ರೀತಿಗಾಗಿ ದೇವರನ್ನು ಪ್ರೀತಿಸುವುದನ್ನು ನಂಬುತ್ತೇವೆ. ಅವನು ನಮಗೆ ಏನನ್ನು ಕೊಡುತ್ತಾನೋ ಅದಕ್ಕಲ್ಲ. ಆದರೆ ದೇವರು ಪ್ರೇಮಸ್ವರೂಪನಾಗಿರುವುದರಿಂದ ಅವನನ್ನು ಪ್ರೀತಿಸುತೇವೆ. ಎಲ್ಲಿಯವರೆಗೆ ಪ್ರೀತಿಗಾಗಿ ಅವನನ್ನು ಪ್ರೀತಿಸುವುದಿಲ್ಲವೋ ಅಲ್ಲಿಯವರೆಗೆ ದೇಶಕ್ಕೇ ಆಗಲಿ, ಜನಾಂಗಕ್ಕೇ ಆಗಲಿ, ಧರ್ಮಕ್ಕೇ ಆಗಲಿ ದೇವರು ದೊರಕಲಾರನು. ಪಾಶ್ಚಾತ್ಯರಾದ ನೀವು ವ್ಯಾಪಾರದಲ್ಲಿ ಗಟ್ಟಿಗರು. ಹೊಸ ಹೊಸ ವಿಷಯಗಳನ್ನು ಕಂಡುಹಿಡಿಯುವುದರಲ್ಲಿ ಗಟ್ಟಿಗರು, ಪೌರಸ್ತ್ಯರಾದ ನಾವು ಧರ್ಮದಲ್ಲಿ ಗಟ್ಟಿಗರು. ನೀವು ವಾಣಿಜ್ಯವನ್ನು ನಿಮ್ಮ ವ್ಯವಹಾರವನ್ನಾಗಿ ಮಾಡಿಕೊಂಡಿರುವಿರಿ. ನಾವು ಧರ್ಮವನ್ನೇ ವ್ಯವಹಾರವನ್ನಾಗಿ ಮಾಡಿಕೊಂಡಿರುವೆವು. ನೀವು ಇಂಡಿಯಾ ದೇಶಕ್ಕೆ ಬಂದು ಅಲ್ಲಿಯ ಹೊಲದಲ್ಲಿ ಕೆಲಸ ಮಾಡುವ ಕೆಲಸಗಾರರೊಡನೆ ಮಾತನಾಡಿದರೆ, ಅವರಿಗೆ ರಾಜಕೀಯಕ್ಕೆ ಸಂಬಂಧಿಸಿದಂತೆ ಯಾವ ಅಭಿಪ್ರಾಯವೂ ಇಲ್ಲವೆನ್ನುವುದು ಗೊತ್ತಾಗುವುದು, ಅವರಿಗೆ ರಾಜಕೀಯದ ಬಗ್ಗೆ ಏನೂ ಗೊತ್ತಿಲ್ಲ. ಆದರೆ ಅವರ ಹತ್ತಿರ ನೀವು ಧಾರ್ಮಿಕ ವಿಷಯವನ್ನು ಕುರಿತು ಮಾತನಾಡಿದರೆ, ಅವರಲ್ಲಿ ಅತ್ಯಂತ ಕನಿಷ್ಠನಿಗೂ, ದ್ವೈತ ಅದ್ವೈತ ಮುಂತಾದ ತತ್ತ್ವಗಳ ವಿಷಯ ಗೊತ್ತಿರುತ್ತದೆ.

ನೀವು ಅವರನ್ನು ಯಾವ ಸರ್ಕಾರವು ನಿಮ್ಮನ್ನು ಆಳುತ್ತಿದೆ ಎಂದು ಕೇಳಿದರೆ ಅವನು 'ನಮಗೆ ಗೊತ್ತಿಲ್ಲ, ನಾವು ತೆರಿಗೆ ಕೊಡುತ್ತೇವೆ, ಅಷ್ಟೇ ನಮಗೆ ಗೊತ್ತಿರುವುದು' ಎಂದು ಹೇಳುತ್ತಾರೆ. ನಾನು ನಿಮ್ಮ ಕೂಲಿಕಾರರೊಡನೆ ಮಾತನಾಡಿರುವೆನು. ಅವರಿಗೆ ರಾಜಕೀಯ ವಿಷಯಗಳೆಲ್ಲ ಚೆನ್ನಾಗಿ ಗೊತ್ತಿದೆ. ಅವರು ಡೆಮಾಕ್ರಾಟಿಕ್ ಅಥವಾ ರಿಪಬ್ಲಿಕ್ ಪಾರ್ಟಿಗೆ ಸೇರಿದವರು. ತಮಗೆ ಬೆಳ್ಳಿಯ ಸ್ಟಾಂಡರ್ಡ್ ಬೇಕೆ, ಚಿನ್ನದ ಸ್ಟಾಂಡರ್ಡ್ ಬೇಕೆ ಎನ್ನುವುದು ಚೆನ್ನಾಗಿ ಗೊತ್ತಿದೆ ಆದರೆ ನೀವು ಅವರೊಡನೆ ಧರ್ಮದ ವಿಷಯವನ್ನು ಕುರಿತು ಮಾತನಾಡಿದರೆ, ಅವರಿಗೆ ಇಂಡಿಯಾ ದೇಶದ ಬೇಸಾಯಗಾರರಂತೆ, ಏನೂ ಗೊತ್ತಿಲ್ಲ, ಅವರು ಒಂದು ಚರ್ಚಿಗೆ ಹೋಗುವರು. ಆ ಚರ್ಚು ಏನನ್ನು ನಂಬುವುದೋ ಅದು ಅವರಿಗೆ ಗೊತ್ತಿಲ್ಲ ಚರ್ಚಿಗೆ ಚಂದಾ ಹಣವನ್ನು ಕೊಡುವರು. ಅದಲ್ಲದೆ ಅವರಿಗೆ ದೇವರು ಅಥವಾ ಇನ್ನಾವ ಆಧ್ಯಾತ್ಮಿಕ ವಿಷಯವೂ ಗೊತ್ತಿಲ್ಲ.” 

ಇಂಡಿಯಾ ದೇಶದಲ್ಲಿರುವ ಮೂಢನಂಬಿಕೆಗಳನ್ನು ಅವರು ಒಪ್ಪಿಕೊಂಡರು. “ಆದರೆ ಯಾವ ದೇಶದಲ್ಲಿ ಅದು ಇಲ್ಲ?” ಎಂದು ಕೇಳಿದರು. ಕೊನೆಗೆ ಉಪನ್ಯಾಸವನ್ನು ಮುಕ್ತಾಯ\break ಗೊಳಿಸುತ್ತ, ರಾಷ್ಟ್ರಗಳು ದೇವರು ತಮ್ಮ ಸ್ವತ್ತು ಎಂದು ಭಾವಿಸುತ್ತವೆ ಎಂದರು. ಎಲ್ಲಾ ದೇಶದವರಿಗೂ ದೇವರು ಇರುವನು. ಸತ್ಪ್ರೇರಣೆಗಳೆಲ್ಲವೂ ದೇವರೇ, ಪೌರಸ್ತ್ಯ ಮತ್ತು ಪಾಶ್ಚಾತ್ಯರು ದೇವರಿಗಾಗಿ ಹಂಬಲಿಸುವುದನ್ನು ಕಲಿಯಬೇಕಾಗಿದೆ. ನೀರಿನಲ್ಲಿರುವವನು ಗಾಳಿಗಾಗಿ ತವಕಿಸುವುದಕ್ಕೆ ಆ ಹಂಬಲವನ್ನು ಹೋಲಿಸಿದರು. ಅವನಿಗೆ ಗಾಳಿ ಬೇಕಾಗಿದೆ. ಅದಿಲ್ಲದೆ ಅವನು ಬಾಳಲಾರ. ಪಾಶ್ಚಾತ್ಯರಿಗೆ ದೇವರು ಹಾಗೆ ಬೇಕಾದಾಗ ಇಂಡಿಯಾ ದೇಶದವರು ಅವರನ್ನು ಸ್ವಾಗತಿಸುತ್ತಾರೆ. ಏಕೆಂದರೆ ಆಗ ದೇವರೊಡನೆ ಮಿಷನರಿಗಳು ಅವರ ಬಳಿಗೆ ಬರುತ್ತಾರೆ. ಇಂಡಿಯಾ ದೇಶದವರಿಗೆ ದೇವರ ಬಗ್ಗೆ ಗೊತ್ತಿಲ್ಲ ಎಂದು ಭಾವಿಸದೆ, ಯಾವ ಮತಾಂಧತೆಯೂ ಇಲ್ಲದೆ ಹೃದಯದಲ್ಲಿ ಪ್ರೀತಿಯಿಂದ ಬರಬೇಕು.

\delimiter

\begin{center}
(ಡೆಸ್ ಮಾನೀಸ್ ನ್ಯೂಸ್, ನವೆಂಬರ್ 28, 1893) \footnote{* C.W. Vol. VIII P. 482}
\end{center}

\vskip -0.35cm

ಬಹಳ ದೂರದ ಇಂಡಿಯಾದಿಂದ ಬಂದಿರುವ ಪ್ರತಿಭಾನ್ವಿತ ವಿದ್ವಾಂಸರಾದ ಸ್ವಾಮಿ ವಿವೇಕಾನಂದರು ಕಳೆದ ರಾತ್ರಿ (ನವೆಂಬರ್ 27) ಸೆಂಟ್ರಲ್ ಚರ್ಚ್‌ನಲ್ಲಿ ಮಾತನಾಡಿದರು. ವಿಶ್ವಮೇಳದ ಪ್ರಯುಕ್ತ ಚಿಕಾಗೊ ನಗರದಲ್ಲಿ ನಡೆದ ವಿಶ್ವಧರ್ಮ ಸಮ್ಮೇಳನಕ್ಕೆ ಭರತಖಂಡದ ಮತ್ತು ಅಲ್ಲಿಯ ಧರ್ಮದ ಪ್ರತಿನಿಧಿಗಳಾಗಿ ಅವರು ಬಂದಿದ್ದರು. ರೆವರೆಂಡ್ ಎಚ್.ಓ. ಬ್ರಿಡನ್ ಅವರು ಸಭಿಕರಿಗೆ ಸ್ವಾಮೀಜಿ ಅವರನ್ನು ಪರಿಚಯ ಮಾಡಿಕೊಟ್ಟರು. ಸ್ವಾಮೀಜಿ ಅವರು ಎದ್ದು ನಿಂತು ಸಭಿಕರಿಗೆ ವಂದಿಸಿ ಹಿಂದೂಧರ್ಮದ ಮೇಲೆ ಉಪನ್ಯಾಸ ಮಾಡಿದರು. ಅವರ ಉಪನ್ಯಾಸ ಯಾವ ಒಂದು ವಿಚಾರ ಕ್ರಮವನ್ನೂ ಅನುಸರಿಸಲಿಲ್ಲ. ತಮ್ಮ ಧರ್ಮ ಇತರ ಧರ್ಮಗಳಿಗೆ ಸಂಬಂಧಪಟ್ಟಂತೆ ತಮ್ಮದೇ ಆದ ಕೆಲವು ಸಿದ್ಧಾಂತಗಳನ್ನು ಅವರು ಮಂಡಿಸಿದರು. ಒಬ್ಬನು ಪರಿಪೂರ್ಣ ಕೈಸ್ತನಾಗಿರಬೇಕಾದರೆ, ಅವನು ಎಲ್ಲಾ ಧರ್ಮಗಳನ್ನು ಸ್ವೀಕರಿಸಬೇಕು ಎಂದು ಅವರು ಹೇಳಿದರು. ಒಂದು ಧರ್ಮದಲ್ಲಿ ಇಲ್ಲದೇ ಇರುವುದು ಮತ್ತೊಂದು ಧರ್ಮದಿಂದ ಸಿಕ್ಕುವುದು, ನಿಜವಾದ ಕ್ರಿಸ್ತಾನುಯಾಯಿಗೆ ಇವುಗಳೆಲ್ಲ ಸರಿ ಮತ್ತು ಆವಶ್ಯಕ. ನೀವು ಮಿಷನರಿಯನ್ನು ನಮ್ಮ ದೇಶಕ್ಕೆ ಕಳುಹಿಸಿದರೆ ಅವನು ಹಿಂದೂ ಕ್ರೈಸ್ತನಾಗುವನು, ನಾನು ಇಲ್ಲಿಗೆ ಬಂದರೆ ಕ್ರೈಸ್ತ ಹಿಂದೂವಾಗುವೆನು. ಇಲ್ಲಿಯ ಜನರನ್ನು ನೀವು ಮತಾಂತರಗೊಳಿಸುವಿರಾ ಎಂದು ನನ್ನನ್ನು ಅನೇಕರು ಕೇಳಿರುವರು. ನಾನು ಇದನ್ನು ಅವಮಾನ ಎಂದು, ಎಣಿಸುತ್ತೇನೆ. \footnote{* ವರದಿಗಾರನು ಅಲ್ಲಿ ಉಪಸ್ಥಿತನಿದ್ದರೂ, ಮತಾಂತರದ ಬಗ್ಗೆ ಸ್ವಾಮೀಜಿಯವರ ವಾದವನ್ನು ಸರಿಯಾಗಿ ಗ್ರಹಿಸಲಿಲ್ಲ ಎಂದು ತೋರುತ್ತದೆ. ಸ್ವಾಮೀಜಿ ಅವರ ಭಾವನೆ ಆಗಲೆ ಗೊತ್ತಿದ್ದವರಿಗೆ ಅರ್ಥವಾಗುವಂತಹ ಕೆಲವು ವಿಷಯಗಳನ್ನು ಮಾತ್ರ ಅವನು ಗ್ರಹಿಸಲು ಸಾಧ್ಯವಾಯಿತು.}ನಾನು ಮತಾಂತರಗೊಳಿಸುವುದನ್ನು ನಂಬುವುದಿಲ್ಲ. ಇಂದು ಪಾಪಿಯಾದ ಮನುಷ್ಯನು ನಮ್ಮಲ್ಲಿ ಇರುವನು, ಅವನನ್ನು ಮತಾಂತರಗೊಳಿಸಿದರೆ ಅವನು ಬರುಬರುತ್ತಾ ಪವಿತ್ರನಾಗುವನು ಎಂದು ನೀವು ಭಾವಿಸು\break ತ್ತೀರಿ. ಈ ಬದಲಾವಣೆ ಎಲ್ಲಿಂದ ಬರುವುದು? ಇದನ್ನು ನೀವು ಹೇಗೆ ವಿವರಿಸುತ್ತೀರಿ? ಮನುಷ್ಯನಿಗೆ ಹೊಸ ಜೀವವೇನೂ ಬಂದಿಲ್ಲ, ಏಕೆಂದರೆ ಜೀವವು ಸಾಯಲೇಬೇಕು. ದೇವರು ಅವನನ್ನು ಬದಲಾಯಿಸಿದನು ಎನ್ನುತ್ತೀರಿ. ದೇವರು ಪರಿಪೂರ್ಣ, ಎಲ್ಲಾ ಶಕ್ತಿಯೂ ಅವನಲ್ಲಿ ಇದೆ. ಅವನು ಪಾವಿತ್ರ್ಯವೇ ಆಗಿರುವನು. ಮನುಷ್ಯನು ಮತಾಂತರ\break ಗೊಂಡ ಮೇಲೆ ಅವನ ದೇವರು ಹಿಂದಿನ ದೇವರೇ ಆಗಿರುತ್ತಾನೆ. ಆದರೆ ತಾನು ಮನುಷ್ಯನಿಗೆ ಕೊಟ್ಟ ಪಾವಿತ್ರ್ಯವನ್ನು ಮಾತ್ರ ಕಳೆದುಕೊಂಡಿರುತ್ತಾನೆ. ನಮ್ಮ ದೇಶದಲ್ಲಿ ಎರಡು ಪದಗಳಿಗೆ, ಈ ದೇಶದಲ್ಲಿ ಅವಕ್ಕೆ ಇರುವ ಅರ್ಥಕ್ಕಿಂತ ಸಂಪೂರ್ಣ ಬೇರೆಯಾದ ಅರ್ಥವಿದೆ. ಆ ಎರಡು ಪದಗಳು ಯಾವುವೆಂದರೆ ಧರ್ಮ ಮತ್ತು ಮತ, ಧರ್ಮ ಎಲ್ಲಾ ಧರ್ಮಗಳನ್ನು ಒಳಗೊಳ್ಳುತ್ತದೆ, ಎಂಬುದು ನಮ್ಮ ನಿಲುವು, ನಾವು ಎಲ್ಲವನ್ನೂ ಸಹಿಸುತ್ತೇವೆ, ಅಸಹಿಷ್ಣುತೆಯನ್ನು ಮಾತ್ರ ಸಹಿಸುವುದಿಲ್ಲ. ಅನಂತರ ಮತ ಎಂಬ ಪದವಿದೆ. ನಾವು ಸರಿ ನೀವು ತಪ್ಪು ಎಂದು ಹೇಳುತ್ತ ಹೊರಗೆ ಮಾತ್ರ ಪ್ರೀತಿಯನ್ನು ವ್ಯಕ್ತಪಡಿಸುವ 'ಮಧುರ' ವ್ಯಕ್ತಿಗಳೆಲ್ಲ ಈ ಪದದಡಿಯಲ್ಲಿ ಬರುತ್ತಾರೆ. ಇದು ನನಗೆ ಎರಡು ಕಪ್ಪೆಗಳ ಒಂದು ಕಥೆಯನ್ನು ನೆನಪಿಗೆ ತರುವುದು. ಒಂದು ಕಪ್ಪೆ ಬಾವಿಯಲ್ಲಿ ಹುಟ್ಟಿತು. ಅದು ತನ್ನ ಜೀವಮಾನವನ್ನೆಲ್ಲ ಅಲ್ಲೇ ಕಳೆಯಿತು. ಒಂದು ದಿನ ಸಮುದ್ರದ ಕಪ್ಪೆಯೊಂದು ಬಾವಿಗೆ ಬಿತ್ತು ಅವು ಸಮುದ್ರದ ವಿಷಯವನ್ನು ಮಾತನಾಡಲು ಉಪಕ್ರಮಿಸಿದವು, ಬಾವಿಯ ಕಪ್ಪೆ ಸಮುದ್ರದ ಕಪ್ಪೆಯನ್ನು ಕುರಿತು ಸಮುದ್ರ ಎಷ್ಟುದೊಡ್ಡದು ಎಂದು ಕೇಳಿತು. ಅದಕ್ಕೆ ಸಮರ್ಪಕವಾದ ಉತ್ತರ ಸಿಕ್ಕಲಿಲ್ಲ. ಆಗ ಬಾವಿಯ ಕಪ್ಪೆ ಒಂದು ಕಡೆಯಿಂದ ಮತ್ತೊಂದು ಕಡೆಗೆ ನೆಗೆದು, ಸಮುದ್ರ ಇಷ್ಟು ದೊಡ್ಡದೊ ಎಂದು ಕೇಳಿತು. ಅದಕ್ಕೆ ಹೊರಗಿನಿಂದ ಬಂದ ಕಪ್ಪೆ ಹೌದು ಎಂದಿತು. ಕಪ್ಪೆ ಇನ್ನೊಂದು ಸಲ ಒಂದು ಕಡೆಯಿಂದ ಮತ್ತೊಂದು ಕಡೆಗೆ ನೆಗೆದು ಸಮುದ್ರ ಇಷ್ಟು ದೊಡ್ಡದೋ, ಎಂದು ಕೇಳಿತು. ಹೌದು ಎಂಬ ಉತ್ತರವನ್ನು ಕೇಳಿ, ಬಾವಿಯ ಕಪ್ಪೆಯು 'ಈ ಕಪ್ಪೆ ಸುಳ್ಳು ಹೇಳುತ್ತಿರಬೇಕು, ಇದನ್ನು ನನ್ನ ಬಾವಿಯಿಂದ ಆಚೆಗೆ ಕಳುಹಿಸುತ್ತೇನೆ' ಎಂದು ಮನಸ್ಸು ಮಾಡಿತು, ಮತಗಳು ಇಂತಹ ಗುಂಪಿಗೆ ಸೇರಿದವು. ಅವು ತಮಗಿರುವ ನಂಬಿಕೆಗಳನ್ನೇ ಯಾರು ಹೊಂದಿಲ್ಲವೋ ಅವರನ್ನು ತುಳಿಯಲು ಮತ್ತು ಆಚೆಗೆ ಕಳುಹಿಸಲು ಪ್ರಯತ್ನಿಸುವುವು.

\vskip -0.5cm

\delimiter


\section[ಹಿಂದೂ ಸಂನ್ಯಾಸಿ]{ಹಿಂದೂ ಸಂನ್ಯಾಸಿ\protect\footnote{* C.W. Vol. III P. 484}}

\begin{center}
(ಅಪೀಲ್​ ಅವಲಾಂಚ್​, ಜನವರಿ ೧೬, ೧೮೯೪)
\end{center}

\vskip -0.35cm

ಮೆಂಫಿಸ್​ನ ಆಡಿಟೋರಿಯಂನಲ್ಲಿ ಇಂದಿನ ರಾತ್ರಿ ಉಪನ್ಯಾಸ ಮಾಡಲಿರುವ ಹಿಂದೂ ಸಂನ್ಯಾಸಿಗಳಾದ ವಿವೇಕಾನಂದರು, ಈ ದೇಶದಲ್ಲಿ ಧಾರ್ಮಿಕ ವೇದಿಕೆಗಳ ಮೇಲೂ, ಇತರ ವಿಷಯಗಳನ್ನು ಕುರಿತು ಉಪನ್ಯಾಸಗಳನ್ನೇರ್ಪಡಿಸುವ ವೇದಿಕೆಗಳ ಮೇಲೂ ಕಾಣಿಸಿಕೊಳ್ಳುತ್ತಿರುವ ಒಂದು ಅಪೂರ್ವ ವ್ಯಕ್ತಿ. ಅವರ ಅನುಪಮವಾದ ವಾಗ್ವೈಖರಿ, ರಹಸ್ಯವಾದ ವಿಷಯಗಳ ಮೇಲೆ ಅವರಿಗೆ ಇರುವ ಅಗಾಧ ಅನುಭವ, ವಾದ ಮಾಡುವುದರಲ್ಲಿ ಅವರಿಗೆ ಇರುವ ಕೌಶಲ ಮತ್ತು ಅವರಲ್ಲಿರುವ ಶ್ರದ್ಧೆ, ಉತ್ಸಾಹಗಳು, ವಿಶ್ವಧರ್ಮ ಸಮ್ಮೇಳನಕ್ಕೆ ಬಂದಿದ್ದ ಜಗತ್ತಿನ ಚಿಂತನಶೀಲರನ್ನೆಲ್ಲ ಆಕರ್ಷಿಸಿದುವು. ಅಂದಿನಿಂದ ಅಮೆರಿಕಾ ದೇಶದಲ್ಲಿ ಅವರು ಕೊಟ್ಟ ಉಪನ್ಯಾಸಗಳನ್ನು ಕೇಳಿದ ಸಹಸ್ರಾರು ಜನರು ಅವರನ್ನು ಮೆಚ್ಚಿರುವರು.

ಸಂಭಾಷಣೆಯಲ್ಲಿ ಅವರು ಬಹಳ ಹರ್ಷಪ್ರದರಾದ ವ್ಯಕ್ತಿಗಳು. ಅವರು ಬಳಸುವ ಪದಗಳಾದರೊ ಆಂಗ್ಲ ಭಾಷೆಯ ಅನರ್ಘ್ಯರತ್ನಗಳಂತೆ ಇವೆ. ಅವರು ತಮ್ಮ ನಡೆ ನುಡಿಗಳಲ್ಲಿ ಪಾಶ್ಚಾತ್ಯ ಸಭ್ಯತೆಯ ಅತ್ಯಂತ ಸುಸಂಸ್ಕೃತ ವ್ಯಕ್ತಿಗಳಂತೆ ಕಾಣುತ್ತಾರೆ. ಒಬ್ಬ ಸಂಗಾತಿಯಾಗಿ ಅವರು ಅಪೂರ್ವ ಮೋಹಕ ವ್ಯಕ್ತಿ. ಸಂಭಾಷಣೆಯಲ್ಲಂತೂ ಪಾಶ್ಚಾತ್ಯ ಜಗತ್ತಿನ ಯಾವ ನಗರದ ದಿವಾನ್​ ಖಾನೆಗಳಲ್ಲಿಯೂ ಅವರನ್ನು ಮೀರಿಸುವವರು ಸಿಕ್ಕಲಾರರು. ಅವರು ಇಂಗ್ಲಿಷ್​ ಭಾಷೆಯನ್ನು ಸ್ಪಷ್ಟವಾಗಿ ಮಾತ್ರ ಮಾತನಾಡುವುದಲ್ಲ, ಸರಾಗವಾಗಿಯೂ ಮಾತನಾಡುವರು. ಅವರ ಭಾವನೆಗಳಾದರೋ ಹೊಳೆಯುತ್ತಿರುವ ರತ್ನಗಳಂತೆ, ಅವ್ಯಾಹತವಾಗಿ ಅವರ ನಾಲಿಗೆಯಿಂದ ಹೊರಬೀಳುತ್ತವೆ. ಉಕ್ಕಿ ಹರಿಯುವ ಅವರ ಆಲಂಕಾರಿಕ ಭಾಷೆ ಎಲ್ಲರನ್ನೂ ಚಕಿತಗೊಳಿಸುತ್ತದೆ.

ಸ್ವಾಮಿ ವಿವೇಕಾನಂದರು ತಾವು ಜನ್ಮವೆತ್ತಿದ ಮತ ಅಥವಾ ಬಾಲ್ಯದಲ್ಲಿ ಪಡೆದ ಶಿಕ್ಷಣಕ್ಕೆ ಅನುಸಾರವಾಗಿ ಬ್ರಾಹ್ಮಣರಾಗಿದ್ದರು. ಆದರೆ ಹಿಂದೂಧರ್ಮಕ್ಕೆ ಸೇರಿದ ಮೇಲೆ ತಮ್ಮ ಜಾತಿ ಮುಂತಾದುವನ್ನೆಲ್ಲಾ ತ್ಯಜಿಸಿ ಹಿಂದೂ ಬೋಧಕರಾದರು, ಅಥವಾ ಪೌರಸ್ತ್ಯ ದೇಶಗಳಲ್ಲಿ ರೂಢಿಯಲ್ಲಿರುವಂತೆ, ಅವರೊಬ್ಬ ಸಂನ್ಯಾಸಿಯಾದರು. ಅವರು ಪ್ರಕೃತಿಯ ಅದ್ಭುತವಾದ ರಹಸ್ಯಮಯವಾದ ಕಾರ್ಯಗಳ ಅತಿ ನಿಕಟ ವಿದ್ಯಾರ್ಥಿಯಾಗಿರುವರು. ಪ್ರಕೃತಿ ಭಗವಂತನ ಒಂದು ಅದ್ಭುತವಾದ ಭಾವನೆ. ಪೌರಸ್ತ್ಯ ದೇಶಗಳಲ್ಲಿ ಅನೇಕ ಉನ್ನತ ಶಿಕ್ಷಣ ಸಂಸ್ಥೆಗಳಲ್ಲಿ ಹಲವು ಕಾಲ ವಿದ್ಯಾರ್ಥಿಯಾಗಿಯೂ ಮತ್ತು ಉಪಾಧ್ಯಾಯರಾಗಿಯೂ ಕೆಲಸ ಮಾಡಿದ ಮೇಲೆ ಜಗತ್ತಿನಲ್ಲೆಲ್ಲ ಈ ಯುಗದ ಮಹಾ ಚಿಂತನಶೀಲ ವಿದ್ವಾಂಸರು ಎನ್ನಿಸಿಕೊಳ್ಳುವಷ್ಟು ಜ್ಞಾನವನ್ನು ಪಡೆದಿರುವರು.

ವಿಶ್ವಧರ್ಮ ಸಮ್ಮೇಳನದಲ್ಲಿ ಅವರು ಮಾಡಿದ ಮೊದಲನೇ ಉಪನ್ಯಾಸದಿಂದಲೇ, ಆ ಅಭೂತಪೂರ್ವ ಧಾರ್ಮಿಕ ಸಂಘದಲ್ಲಿ ನಾಯಕತ್ವಕ್ಕೆ ಏರಿರುವರು.

ಸಮ್ಮೇಳನದ ಕಾಲದಲ್ಲಿ ಅವರು ತಮ್ಮ ಧರ್ಮದ ಪರವಾಗಿ ಮಾಡಿದ ಭಾಷಣಗಳನ್ನು ಅನೇಕ ವೇಳೆ ಸಭಿಕರು ಕೇಳಿರುವರು, ಮತ್ತು ಇಂಗ್ಲಿಷ್​ ಭಾಷೆಯಲ್ಲಿ ಶೋಭಾಯಮಾನವಾಗಿರುವ ಅತ್ಯಂತ ಸುಂದರವಾದ ಚಿಂತನೆಯ ರತ್ನಗಳ ನೆರವಿನಿಂದ, ಮಾನವನು ಇತರ ಮಾನವರಿಗೆ ಮತ್ತು ದೇವರಿಗೆ ಸಲ್ಲಿಸಬೇಕಾದ ಉನ್ನತ ಕರ್ತವ್ಯಗಳನ್ನು ಅವರು ವಿವರಿಸಿದರು. ಅವರು ಭಾವನೆಯನ್ನು ವ್ಯಕ್ತಗೊಳಿಸುವುದರಲ್ಲಿ ಕಲೆಗಾರರು, ತಮ್ಮ ನಂಬಿಕೆಗಳಲ್ಲಿ ಆದರ್ಶವಾದಿ, ವೇದಿಕೆಯ ಮೇಲೆ ನಟನಚತುರರು.

ಮೆಂಫಿಸ್​ಗೆ ಬಂದಾಗಿನಿಂದಲೂ ಅವರು ಶ‍್ರೀ ಹು.ಎಲ್​. ಬ್ರಿನ್​ಕ್ಲಿ ಅವರ ಅತಿಥಿಗಳಾಗಿ\break ರುವರು. ಅಲ್ಲಿಗೆ ತಮ್ಮ ಗೌರವವನ್ನು ತೋರುವುದಕ್ಕಾಗಿ ಹಗಲು ರಾತ್ರಿ ಬರುವ ಹಲವರನ್ನು ಸ್ವಾಮೀಜಿ ಸ್ವಾಗತಿಸಿರುವರು. ಟೆನಿಸೀ ಕ್ಲಬ್ಬಿನಲ್ಲಿಯೂ ಅವರು ಅನೌಪಚಾರಿಕ ಅತಿಥಿಗಳಾಗಿರುವರು. ಶನಿವಾರ ಸಾಯಂಕಾಲ ಶ‍್ರೀಮತಿ ಎಸ್​.ಆರ್​. ಶಫರ್ಡ್​ ಅವರು ಏರ್ಪಡಿಸಿದ್ದ ಸತ್ಕಾರ ಕೂಟದಲ್ಲಿ ಅವರು ಅತಿಥಿಗಳಾಗಿದ್ದರು. ಕರ್ನಲ್​ ಆರ್​. ಬಿ. ಸ್ನೋಡನ್​ ಅವರು ಸುಪ್ರಸಿದ್ಧ ಸ್ವಾಮೀಜಿ ಅವರ ಗೌರವಾರ್ಥವಾಗಿ ಒಂದು ಭೋಜನ ಕೂಟವನ್ನು ಭಾನುವಾರ ಆನ್ಸ್​ಡೇಲಿನಲ್ಲಿರುವ ತಮ್ಮ ಮನೆಯಲ್ಲಿ ಏರ್ಪಡಿಸಿದ್ದರು. ಅಲ್ಲಿ ಅವರು ಅಸಿಸ್ಟೆಂಟ್​ ಬಿಷಪ್​ ಆಗಿರುವ ಥಾಮಸ್​ ಎಫ್​. ಗೈಲರ್​, ರೆವರೆಂಡ್​ ಡಾಕ್ಟರ್​ ಜಾಜ್​ ಪ್ಯಾಟರ್ಸನ್​ ಮತ್ತು ಇತರ ಪಾದ್ರಿಗಳನ್ನು ಸಂದರ್ಶಿಸಿದರು.

ನಿನ್ನೆ ಮಧ್ಯಾಹ್ನ ಹತ್ತೊಂಬತ್ತನೇ ಶತಮಾನದ ಕ್ಲಬ್ಬಿಗೆ ಸೇರಿದ ರಂಡಾಲ್ಫ್​ ಬಿಲ್ಡಿಂಗಿನ ಸಭಾಂಗಣದಲ್ಲಿ ಬಹುಸಂಖ್ಯೆಯಲ್ಲಿ ನೆರೆದಿದ್ದ ನವನಾಗರಿಕ ಸಭಿಕರನ್ನು ಉದ್ದೇಶಿಸಿ ಅವರು ಮಾತನಾಡಿದರು. ಇವತ್ತು ಆಡಿಟೋರಿಯಂನಲ್ಲಿ ಅವರು ಹಿಂದೂ ಧರ್ಮದ ಮೇಲೆ ಮಾತನಾಡುವರು.

\delimiter


\section[ಮಾನವನ ಗುರಿ]{ಮಾನವನ ಗುರಿ\protect\footnote{* C.W. Vol. VII P. 419}}

\begin{center}
(ಮೆಂಫಿಸ್​ನಲ್ಲಿ ೧೭ನೇ ಜನವರಿ ೧೮೯೪ರಲ್ಲಿ ಮಾಡಿದ ಉಪನ್ಯಾಸದ ಸಾರಾಂಶ, ಅಪೀಲ್​ ಆವ್​ಲಾಂಚ್​ ಪತ್ರಿಕೆಯ ವರದಿ)
\end{center}

\vskip -0.35cm

ಸಭಿಕರು ಸುಮಾರಾಗಿಯೇ ಸೇರಿದ್ದರು. ಊರಿನ ಶ್ರೇಷ್ಠ ಮೇಧಾವಿಗಳು ಮತ್ತು ಸಂಗೀತಗಾರರು ಅವರಲ್ಲಿದ್ದರು. ಪ್ರಖ್ಯಾತರಾದ ವಕೀಲರು ಮತ್ತು ಆರ್ಥಿಕ ಸಂಸ್ಥೆಯ ಪ್ರತಿನಿಧಿಗಳು ಇದ್ದರು. ಅಮೆರಿಕಾ ದೇಶದ ಕೆಲವು ವಾಗ್ಮಿಗಳೊಂದಿಗೆ ಹೋಲಿಸಿ ನೋಡಿದರೆ ಉಪನ್ಯಾಸಕರಲ್ಲಿ ಒಂದು ವ್ಯತ್ಯಾಸ ಕಂಡುಬರುತ್ತದೆ. ಗಣಿತ ಪ್ರಾಧ್ಯಾಪಕನು ಒಂದು ಬೀಜಗಣಿತದ ಲೆಕ್ಕವನ್ನು ತನ್ನ ವಿದ್ಯಾರ್ಥಿಗಳಿಗೆ ಹೇಗೆ ಬೋಧಿಸುವನೊ ಹಾಗೆ ಉಪನ್ಯಾಸಕರು ತಮ್ಮ ಚಿಂತನೆಗಳನ್ನು ಸಮರ್ಥಿಸುವರು. ವಿವೇಕಾನಂದರಿಗೆ ತಮ್ಮ ಶಕ್ತಿ ಮತ್ತು ಯೋಗ್ಯತೆಗಳಲ್ಲಿ ಚೆನ್ನಾಗಿ ನಂಬಿಕೆ ಇದೆ. ಎಲ್ಲಾ ವಾದಗಳಲ್ಲೂ ಸಮರ್ಥವಾಗಿ ಎದುರಿಸ\break ಬಲ್ಲಂತಹ ದಿಟ್ಟತನದಿಂದ ಅವರು ಮಾತನಾಡುವರು. ತಾರ್ಕಿಕವಾಗಿ ನಿರ್ಣಯಗೊಳಿಸ\break ಲಾಗದ ಯಾವ ಭಾವನೆಗಳನ್ನಾಗಲೀ, ಹೇಳಿಕೆಗಳನ್ನಾಗಲೀ ಅವರು ಮಂಡಿಸುವುದಿಲ್ಲ. ಅವರ ಉಪನ್ಯಾಸದ ಬಹುಭಾಗ ಇಂಗರ್​ಸಾಲ್​ನ ತತ್ತ್ವವನ್ನು ಹೋಲುವುದು. ಕ್ರೈಸ್ತರು ನಂಬುವಂತಹ ಮರಣಾನಂತರ ಶಿಕ್ಷೆ ಮತ್ತು ಅವರ ದೇವರ ಭಾವನೆ ಇವುಗಳನ್ನು ಸ್ವಾಮೀಜಿ ಅವರು ನಂಬುವುದಿಲ್ಲ. ಅವರು ಮನಸ್ಸನ್ನು ಅಮರ ಎಂದು ಭಾವಿಸುವುದಿಲ್ಲ. ಏಕೆಂದರೆ ಅದು ಆಶ್ರಿತ ವಸ್ತು. ಯಾವುದು ಅಸ್ವತಂತ್ರವಾಗಿರುವುದೊ ಅದು ಅಮರವಾಗಿರಲಾರದು. ಅವರು ಹೀಗೆ ಹೇಳುವರು: “ದೇವರು ವಿಶ್ವದ ಯಾವುದೋ ಒಂದು ಮೂಲೆಯಲ್ಲಿ ಕುಳಿತುಕೊಂಡು ನಾವಿಲ್ಲಿ ಮಾಡುವ ಒಳ್ಳೆಯ ಮತ್ತು ಕೆಟ್ಟ ಕರ್ಮಗಳಿಗೆ ಬಹುಮಾನ ಮತ್ತು ಶಿಕ್ಷೆಯನ್ನು ಕೊಡುವ ರಾಜನೇನಲ್ಲ.” ಮಾನವನು ಸತ್ಯವನ್ನು ಅರಿಯುವ ಒಂದು ಸಮಯ ಬರುವುದು, ಆಗ ಅವನು ಎದ್ದು ನಿಂತು ‘ನಾನೇ ದೇವರು, ಅವನ ಜೀವದ ಜೀವ’ ಎಂದು ಹೇಳುವನು. ನಮ್ಮ ನೈಜಸ್ವಭಾವ, ನಮ್ಮಲ್ಲಿರುವ ಅಮರ ತತ್ತ್ವವೇ ದೇವರಾಗಿರುವಾಗ ದೇವರು ಎಲ್ಲೊ ದೂರದಲ್ಲಿರುವನು ಎಂದು ಏತಕ್ಕೆ ಬೋಧಿಸುತ್ತೀರಿ?

“ನಿಮ್ಮ ಧರ್ಮವು ಮಾನವ ಆದಿಯಲ್ಲಿ ಪಾಪಿಯಾಗಿದ್ದನು ಎಂಬುದನ್ನು ಕೇಳಿ ಭ್ರಾಂತರಾಗಬೇಡಿ. ಅದೇ ಧರ್ಮ ಮಾನವ ಆದಿಯಲ್ಲಿ ಪವಿತ್ರಾತ್ಮನಾಗಿದ್ದನು ಎಂಬುದನ್ನೂ ಬೋಧಿಸುವುದು. ಆದಮ್​ ಪತಿತನಾದಾಗ ಅವನು ಪವಿತ್ರ ಸ್ಥಿತಿಯಿಂದ ಪತಿತನಾ\break ದನು (ಕರತಾಡನ). ಪವಿತ್ರತೆಯೇ ನಮ್ಮ ನೈಜ ಸ್ವಭಾವ. ಅದನ್ನು ಪುನಃ ಪಡೆಯುವುದೇ ಎಲ್ಲಾ ಧರ್ಮಗಳ ಗುರಿ. ಮನುಷ್ಯರೆಲ್ಲ ಪವಿತ್ರಾತ್ಮರು, ಒಳ್ಳೆಯವರು. ಇದಕ್ಕೆ ವಿರೋಧವಾಗಿ ಕೆಲವು ಪ್ರಶ್ನೆಗಳನ್ನು ಕೇಳಬಹುದು. ಹಾಗಾದರೆ ಕೆಲವರು ಏತಕ್ಕೆ ಮೃಗದಂತೆ ಇರುವರು ಎಂದು ಕೇಳಬಹುದು. ನೀವು ಯಾರನ್ನು ಪಶುಸಮಾನ ಎನ್ನುವಿರೊ ಅವನು ಕಸದಲ್ಲಿ ಬಿದ್ದಿರುವ ಒಂದು ವಜ್ರದಂತೆ. ಆ ಕೊಳೆಯನ್ನು ಕೊಡವಿದರೆ ಅದೊಂದು ಚೆನ್ನಾಗಿರುವ ವಜ್ರವಾಗುವುದು. ಅದರ ಮೇಲೆ ಎಂದೂ ಧೂಳು ಕವಿದೇ ಇರಲಿಲ್ಲವೇನೋ ಎನ್ನುವಷ್ಟು ಶುದ್ಧವಾಗಿ ಕಾಣುವುದು. ಪ್ರತಿಯೊಂದು ಜೀವವೂ ಒಂದು ದೊಡ್ಡ ವಜ್ರ ಎನ್ನುವುದನ್ನು ನಾವು ಒಪ್ಪಬೇಕಾಗಿದೆ.”

“ನಮ್ಮ ಸಹೋದರನನ್ನು ಪಾಪಿ ಎಂದು ಕರೆಯುವುದಕ್ಕಿಂತ ಹೀನವಾಗಿರುವುದು ಮತ್ತೊಂದು ಇಲ್ಲ. ಒಂದು ಹೆಣ್ಣು ಸಿಂಹ ಒಂದು ಸಲ ಕುರಿಯ ಮಂದೆಯ ಮೇಲೆ ಬಿತ್ತು. (ಆ ಸಮಯದಲ್ಲಿ ಅದು ಗರ್ಭಿಣಿಯಾಗಿದ್ದುದರಿಂದ ಒಂದು ಮರಿಯನ್ನು ಹಾಕಿ ಸತ್ತುಹೋಯಿತು.) ತಬ್ಬಲಿಯಾದ ಸಿಂಹದ ಮರಿಗೆ ಕುರಿಯೇ ಹಾಲು ಕುಡಿಸಿತು. ಅದು ಕುರಿಯ ಮಂದೆಯೊಂದಿಗೆ ಬೆಳೆದು ಹುಲ್ಲನ್ನು ತಿನ್ನುವುದನ್ನೂ ಕಲಿತುಕೊಂಡಿತು. ಒಂದು ದಿನ ಒಂದು ಮುದಿ ಸಿಂಹ ಕುರಿಯ ಮಂದೆಯಲ್ಲಿದ್ದ ಈ ಸಿಂಹವನ್ನು ಕಂಡಿತು. ಅದನ್ನು ಕುರಿಯ ಮಂದೆಯಿಂದ ಬಿಡಿಸಲು ಯತ್ನಿಸಿತು. ಆದರೆ ಅದು ಮುದಿ ಸಿಂಹವು ಬಂದೊಡನೆಯೇ ಕುರಿಯ ಮಂದೆಯೊಡನೆ ತಪ್ಪಿಸಿಕೊಂಡು ಓಡಿಹೋಯಿತು. ಮುದಿ ಸಿಂಹ ಹೊಂಚುಹಾಕಿಕೊಂಡಿದ್ದು ಒಂದು ಸಲ ಒಂಟಿಯಾಗಿ ಮೇಯುತ್ತಿದ್ದಾಗ ಅದನ್ನು ಹಿಡಿದುಕೊಂಡು ಹತ್ತಿರ ಇದ್ದ ಒಂದು ಕುಂಟೆಯ ಸಮೀಪಕ್ಕೆ ಹೋಯಿತು. ಆ ಸಿಂಹ ಮಂದೆಯಲ್ಲಿದ್ದ ಸಿಂಹಕ್ಕೆ “ನೀನು ಕುರಿಯಲ್ಲ, ಸಿಂಹ. ನಿನ್ನ ಮತ್ತು ನನ್ನ ನೆರಳನ್ನು ನೀರಿನಲ್ಲಿ ನೋಡಿಕೊ” ಎಂದು ಹೇಳಿತು. ಅದು ನೀರಿನಲ್ಲಿ ಪ್ರತಿಬಿಂಬಿತವಾದ ತನ್ನ ನೆರಳನ್ನು ನೋಡಿಕೊಂಡು ತಾನು ಸಿಂಹ ಕುರಿಯಲ್ಲ ಎಂದು ತಿಳಿದುಕೊಂಡಿತು. ನಾವುಗಳು ಕುರಿ ಎಂದು ಭಾವಿಸಕೂಡದು, ಕುರಿಯಂತೆ ಅರಚಿಕೊಳ್ಳುತ್ತ ಹುಲ್ಲನ್ನು ಮೇಯುತ್ತಿರಬಾರದು, ಸಿಂಹಗಳಾಗಬೇಕು.”

“ನಾನು ನಾಲ್ಕು ತಿಂಗಳಿಂದ ಅಮೆರಿಕಾದೇಶದಲ್ಲಿರುವೆನು. ಮಸಾಚುಸೆಟ್ಸ್​ನಲ್ಲಿ\break ಅಪರಾಧಿಗಳನ್ನು ತಿದ್ದುವುದಕ್ಕಾಗಿ ಇರುವ ಸೆರೆಮನೆಯನ್ನು ನೋಡಿದೆನು. ಆ ಜೈಲಿನಲ್ಲಿರುವ ಅಧಿಕಾರಿಗೆ, ಅಪರಾಧಿಗಳನ್ನು ಅಲ್ಲಿಗೆ ಯಾವ ತಪ್ಪಿಗಾಗಿ ಕಳುಹಿಸುವರೊ ಗೊತ್ತಿಲ್ಲ. ಅಲ್ಲಿ ಅಪರಾಧಿಗಳನ್ನು ದಯೆಯಿಂದ ನೋಡಿಕೊಳ್ಳತ್ತಾರೆ. ಮತ್ತೊಂದು ಊರಿನಲ್ಲಿ ಬಹಳ ವಿದ್ಯಾವಂತರ ಸಂಪಾದಕತ್ವದಲ್ಲಿ ನಡೆಯುತ್ತಿರುವ ಮೂರು ವೃತ್ತಪತ್ರಿಕೆಗಳು ಇದ್ದವು. ಅಪರಾಧಿಗಳಿಗೆ ಉಗ್ರವಾದ ಶಿಕ್ಷೆ ಅವಶ್ಯಕ ಎಂದು ಈ ಪತ್ರಿಕೆಗಳು ವಾದಿಸುತ್ತವೆ. ಮತ್ತೊಂದು ಪತ್ರಿಕೆಯಾದರೊ ದಯೆ ಶಿಕ್ಷೆಗಿಂತ ಹೆಚ್ಚು ಪರಿಣಾಮಕಾರಿ ಎಂದು ಸಾರುವುದು. ಒಂದು ವೃತ್ತಪತ್ರಿಕೆ ಅಂಕಿ ಅಂಶಗಳ ಆಧಾರದ ಮೇಲೆ ಉಗ್ರ ಶಿಕ್ಷೆಯನ್ನು ಪಡೆದ ನೂರು ಅಪರಾಧಿಗಳಲ್ಲಿ ಐವತ್ತು ಜನ ಮಾತ್ರ ತಮ್ಮ ಜೀವನವನ್ನು ಸರಿಪಡಿಸಿಕೊಂಡಿರುವರು ಎಂದೂ ಬಹಳ ಕಡಿಮೆ ಶಿಕ್ಷೆ ಪಡೆದ ಪ್ರಯೋಜನದಿಂದ ಸುಮಾರು ನೂರರಲ್ಲಿ ತೊಂಬತ್ತು ಜನ ಜೀವನದಲ್ಲಿ ಉದ್ಯೋಗಗಳನ್ನು ಅವಲಂಬಿಸಿ ಉತ್ತಮರಾದರು ಎಂದೂ ಸಾರುತ್ತದೆ.”

“ಧರ್ಮವು ಮಾನವನ ಸ್ವಭಾವದಲ್ಲಿರುವ ದೌರ್ಬಲ್ಯದಿಂದ ಬಂದುದಲ್ಲ. ಕ್ರೂರ ನಿರಂಕುಶ ಅಧಿಪತಿಯ ಭಯದಿಂದ ಧರ್ಮವು ಹುಟ್ಟಿರುವುದಲ್ಲ. ಧರ್ಮ ಎಂದರೆ ಪ್ರೀತಿ, ವಿಕಾಸವಾಗುವುದು, ವಿಸ್ತಾರವಾಗುವುದು, ಬೆಳೆಯುವುದು ಎಂದು. ಒಂದು ಗಡಿಯಾರವನ್ನು ತೆಗೆದುಕೊಳ್ಳಿ. ಅದರೊಳಗೆ ಒಂದು ಸಣ್ಣ ಸ್ಪ್ರಿಂಗ್​ ಮತ್ತು ಯಂತ್ರವಿದೆ. ಸ್ಪ್ರಿಂಗ್​ ಅನ್ನು ತಿರುಗಿಸಿದರೆ ಅದು ತನ್ನ ಸಹಜ ಸ್ಥಿತಿಗೆ ಬರುವುದಕ್ಕೆ ಪ್ರಯತ್ನಿಸುವುದು. ನೀವು ಗಡಿಯಾರದಲ್ಲಿನ ಸ್ಪ್ರಿಂಗಿನ ಹಾಗೆ. ಪ್ರತಿಯೊಂದು ಗಡಿಯಾರಕ್ಕೂ ಒಂದೇ ಬಗೆಯ ಸ್ಪ್ರಿಂಗ್​ ಇರಬೇಕಾಗಿಲ್ಲ. ಹಾಗೆಯೇ ನಮಗೆಲ್ಲರಿಗೂ ಒಂದೇ ಬಗೆಯ ಧರ್ಮ ಇರಬೇಕಾಗಿಲ್ಲ. ನಾವೇತಕ್ಕೆ ಜಗಳ ಕಾಯಬೇಕು? ನಮ್ಮಲ್ಲಿ ಎಲ್ಲರಿಗೂ ಒಂದೇ ಬಗೆಯ ಆಲೋಚನೆ ಇದ್ದರೆ ಈ ಪ್ರಪಂಚದ ಸರ್ವನಾಶ ವಾಗುತ್ತಿತ್ತು. ಬಾಹ್ಯಚಲನೆಯನ್ನು ನಾವು ಕ್ರಿಯೆ ಎನ್ನುವೆವು. ಆಂತರಿಕ ಚಲನೆಯನ್ನು ಆಲೋಚನೆ ಎನ್ನುವೆವು. ಕಲ್ಲು ನೆಲಕ್ಕೆ ಬೀಳುವುದು. ಅದು ಆಕರ್ಷಣ ಶಕ್ತಿಯಿಂದ ಆಯಿತು ಎಂದು ಹೇಳುತ್ತೀರಿ. ಕುದುರೆ ಗಾಡಿಯನ್ನು ಎಳೆಯುವುದು, ದೇವರು ಕುದುರೆಯನ್ನು ಎಳೆಯುವನು. ಇದೇ ಚಲನೆಯ ನಿಯಮ. ಸುಳಿಗಳು ನೀರಿನ ಪ್ರವಾಹದ ಶಕ್ತಿಯನ್ನು ತೋರುವುದು. ಪ್ರವಾಹವನ್ನು ತಡೆದರೆ ನೀರು ಕಲುಷಿತ ವಾಗುತ್ತದೆ. ಚಲನೆಯೇ ಜೀವನ. ನಮ್ಮಲ್ಲಿ ಐಕ್ಯತೆ ಮತ್ತು ವೈವಿಧ್ಯ ಇರಬೇಕು. ಗುಲಾಬಿಯ ಹೂವನ್ನು ಯಾವ ಹೆಸರಿನಿಂದ ಕರೆದರೂ ಪರಿಮಳದಲ್ಲಿ ವ್ಯತ್ಯಾಸ ವಾಗುವುದಿಲ್ಲ. ನಿಮ್ಮ ಧರ್ಮವನ್ನು ಯಾವ ಹೆಸರಿನಿಂದ ಬೇಕಾದರೂ ಕರೆಯಿರಿ ಚಿಂತೆಯಿಲ್ಲ.

(ಅನಂತರ ಸ್ವಾಮೀಜಿ ಕುರುಡರು ಆನೆಯನ್ನು ವರ್ಣಿಸಿದ ಕತೆಯನ್ನು ನಿರೂಪಿಸಿದರು.)

“ಇಂಡಿಯಾ ದೇಶದಲ್ಲಿ ಒಬ್ಬ ಸಾಧು ಹೀಗೆ ಹೇಳುವನು: “ನೀವು ಮರಳನ್ನು ಗಾಣಕ್ಕೆ ಕೊಟ್ಟು ಎಣ್ಣೆಯನ್ನು ತೆಗೆದೆವು ಎಂದರೆ ನಂಬಬಲ್ಲೆ, ಮೊಸಳೆಯ ಬಾಯಿ ಯಲ್ಲಿರುವ ಹಲ್ಲನ್ನು ಅದರಿಂದ ಕಚ್ಚಿಸಿಕೊಳ್ಳದೆ ಕಿತ್ತು ಹಾಕಿದೆ ಎಂದರೆ ನಂಬಬಲ್ಲೆ. ಆದರೆ ಮತಾಂಧನು ಬದಲಾಯಿಸಿರುವನು ಎಂದರೆ ನಂಬಲಾರೆ! ಧರ್ಮದಲ್ಲಿ ಏತಕ್ಕೆ ಇಷ್ಟೊಂದು ಭಿನ್ನಾಭಿಪ್ರಾಯಗಳಿವೆ ಎಂದು ನೀವು ಕೇಳಬಹುದು. ಅದಕ್ಕೆ ಉತ್ತರ ಇದು: ಬೆಟ್ಟಗುಡ್ಡಗಳಿಂದ ಬರುವ ಸಾವಿರಾರು ನದಿಗಳು ಕೊಟ್ಟಕೊನೆಗೆ ಮಹಾಸಾಗರವನ್ನು ಸೇರಿಯೇ ಸೇರುವುವು. ಇದರಂತೆಯೇ ಹಲವು ಧರ್ಮಗಳು ಅವೆಲ್ಲ ಕೊನೆಗೆ ನಮ್ಮನ್ನು ದೇವರ ಸಮೀಪಕ್ಕೆ ಒಯ್ಯುವುದಕ್ಕಾಗಿಯೇ ಇರುವುವು. ನೀವು ಕಳೆದ ೧೯೦೦ ವರುಷಗಳಿಂದ ಯಹೂದಿಗಳನ್ನು ನಾಶಮಾಡಲು ಯತ್ನಿಸುತ್ತಿರುವಿರಿ. ಅವರನ್ನು ಏತಕ್ಕೆ ನಿರ್ನಾಮ ಮಾಡಲು ಆಗಲಿಲ್ಲ? ಮೌಢ್ಯ ಮತ್ತು ಧರ್ಮಾಂಧತೆ ಎಂದಿಗೂ ಸತ್ಯವನ್ನು ನಾಶಮಾಡ\break ಲಾರದು ಎಂದು ಪ್ರತಿಧ್ವನಿ ಉತ್ತರವೀಯುವುದು.”

ಉಪನ್ಯಾಸಕರು ಇದೇ ಸರಣಿಯಲ್ಲಿ ಸುಮಾರು ಎರಡು ಘಂಟೆಗಳವರೆಗೆ ಮಾತನಾಡಿದರು. ಕೊನೆಗೆ “ಒಬ್ಬರು ಮತ್ತೊಬ್ಬರಿಗೆ ಸಹಾಯ ಮಾಡೋಣ, ಇತರರನ್ನು ಧ್ವಂಸ ಮಾಡದೆ ಇರೋಣ” ಎಂಬ ನುಡಿಯಿಂದ ತಮ್ಮ ಉಪನ್ಯಾಸವನ್ನು ಮುಕ್ತಾಯ ಮಾಡಿದರು.

\delimiter


\section[ಔದಾರ್ಯಕ್ಕಾಗಿ ಒಂದು ಮನವಿ]{ಔದಾರ್ಯಕ್ಕಾಗಿ ಒಂದು ಮನವಿ\protect\footnote{* C.W. Vol. III P. 486}}

\begin{center}
ಮೆಂಫಿಸ್​ ಕಮರ್ಷಿಯಲ್​, ಜನವರಿ ೧೭, ೧೮೯೪)
\end{center}

\vskip -0.35cm

ನಿನ್ನೆ ರಾತ್ರಿ ಆಡಿಟೋರಿಯಂನಲ್ಲಿ ಹಿಂದೂಧರ್ಮದ ಮೇಲೆ ಖ್ಯಾತ ಹಿಂದೂ ಸಂನ್ಯಾಸಿ ಸ್ವಾಮಿ ವಿವೇಕಾನಂದರು ಮಾಡಿದ ಉಪನ್ಯಾಸವನ್ನು ಕೇಳಲು ಅನೇಕ ಜನರು ನೆರೆದಿದ್ದರು. ಜಡ್ಜ್​ ಆರ್​.ಜೆ. ಮೋರ್ಗನ್​ ಅವರು ಸಂಕ್ಷೇಪವಾದ ಮತ್ತು ಮಾಹಿತಿಪೂರ್ಣವಾದ ತಮ್ಮ ಭಾಷಣದಲ್ಲಿ ಉಪನ್ಯಾಸಕರ ಪರಿಚಯವನ್ನು ಮಾಡಿ ಕೊಟ್ಟರು. ಆ ಸಮಯದಲ್ಲಿ ಅವರು ಪ್ರಖ್ಯಾತ ಆರ್ಯ ಜನಾಂಗದ ಬೆಳವಣಿಗೆಯ ಚಿತ್ರವನ್ನು ನೀಡಿ, ಈ ಬೆಳವಣಿಗೆಯಿಂದಲೇ ಯುರೋಪಿಯನ್ನರು ಮತ್ತು ಹಿಂದೂಗಳು ಬಂದಿರುವರೆಂದು ಹೇಳಿ, ಉಪನ್ಯಾಸಕರಿಗೂ ಅಮೆರಿಕಾದೇಶೀಯರಿಗೂ ಒಂದು ಸಂಬಂಧವನ್ನು ಕಲ್ಪಿಸಿದರು.

ಪ್ರಖ್ಯಾತರಾದ ಪೌರಸ್ತ್ಯರನ್ನು (ವಿವೇಕಾನಂದರನ್ನು) ಧಾರಾಳವಾದ ಕರ ತಾಡನಗಳಿಂದ ಸ್ವಾಗತಿಸಲಾಯಿತು. ಬಹಳ ಆಸಕ್ತಿಯಿಂದ ಅವರ ಉಪನ್ಯಾಸವನ್ನೆಲ್ಲ ಕೇಳಿದರು. ಉತ್ತಮ ಅಂಗಸೌಷ್ಟವದಿಂದ ಕೂಡಿದ ಅವರು ಸುಂದರವಾಗಿ ಕಾಣುತ್ತಿದ್ದರು. ಕಂದುಬಣ್ಣದ ರೇಷ್ಮೆಯ ನಿಲುವಂಗಿಯನ್ನು ತೊಟ್ಟಿದ್ದರು. ಕಪ್ಪು ವಸ್ತ್ರದ ಪಟ್ಟಿಯನ್ನು ನಡುವಿನ ಮೇಲೆ ಕಟ್ಟಿಕೊಂಡಿದ್ದರು. ಕಪ್ಪು ಶರಾಯಿಯನ್ನು ಹಾಕಿ ಕೊಂಡಿದ್ದರು. ತಲೆಯ ಮೇಲೆ ಹಳದಿಯ ಬಣ್ಣದ ರೇಶ್ಮೆಯ ರುಮಾಲನ್ನು ಕಟ್ಟಿಕೊಂಡು ಕುಚ್ಚನ್ನು ಕೆಳಗೆ ಇಳಿಬಿಟ್ಟಿದ್ದರು. ಅವರ ಭಾಷಣದ ವೈಖರಿ ಬಹಳ ಚೆನ್ನಾಗಿದೆ. ಸುಂದರವಾದ ಪದಗಳನ್ನು ಬಳಸುವರು. ಭಾಷೆ ವ್ಯಾಕರಣಬದ್ಧವಾಗಿದೆ. ಕೆಲವು ವೇಳೆ ಉಚ್ಚಾರಣೆ ಮಾಡುವಾಗ ಯಾವುದಾದರೂ ಸ್ವರದ ಮೇಲೆ ಹೆಚ್ಚು ಒತ್ತು ನೀಡುತ್ತಾರೆ. ಇದೊಂದೇ ಸ್ವಲ್ಪ ತಪ್ಪಾಗಿ ಕಾಣುವುದು. ಲಕ್ಷ್ಯವಿಟ್ಟು ಕೇಳುತ್ತಿದ್ದ ಶ್ರೋತೃಗಳು ಅವರ ಯಾವ ಶಬ್ದವನ್ನೂ ಬಿಡಲಿಲ್ಲ. ಅವರು ಆಸಕ್ತಿ ಇಟ್ಟು ಕೇಳಿದ್ದಕ್ಕೆ ತಕ್ಕ ಪ್ರತಿಫಲವೇ ದೊರಕಿತು. ಉಪನ್ಯಾಸದಲ್ಲಿ ಸ್ವತಂತ್ರವಾದ ಭಾವನೆಗಳು, ಬೇಕಾದಷ್ಟು ವಿಷಯ ಸಂಗ್ರಹ ಮತ್ತು ಉದಾತ್ತ ವಿಚಾರಗಳು ಇದ್ದವು. ವಿಶ್ವದಲ್ಲಿ ಒಂದು ಔದಾರ್ಯ ಮನೋಭಾವ ಉದಿಸಬೇಕು ಎಂಬುದಕ್ಕೆ ಮಾಡಿದ ಕರೆ ಎಂದು ಆ ಉಪನ್ಯಾಸದ ಬಗ್ಗೆ ಹೇಳಬಹುದು. ಇದನ್ನು ಭರತಖಂಡದ ಧರ್ಮದ ಮೂಲಕ ವಿವರಿಸಿದರು. ಔದಾರ್ಯ, ಪ್ರೀತಿಯ ಮನೋಭಾವ ಇವೇ ಎಲ್ಲಾ ಉತ್ತಮ ಧರ್ಮಗಳ ಮೂಲ ಸ್ಫೂರ್ತಿ. ಯಾವ ಬಗೆಯ ಧರ್ಮವಾಗಲೀ, ಅದು ಸಾಧಿಸಬೇಕಾದ ಗುರಿ, ಇವೇ ಆಗಿರಬೇಕು ಎಂದರು.

ಹಿಂದೂಧರ್ಮದ ಮೇಲೆ ಅವರು ಕೊಟ್ಟ ಉಪನ್ಯಾಸವು ಕೇವಲ ಆ ವಿಷಯಕ್ಕೆ ಮಾತ್ರ ಸೀಮಿತವಾಗಿರಲಿಲ್ಲ. ಅವರ ಉದ್ದೇಶ ಹಿಂದೂಧರ್ಮದ ಮೂಲ ಸ್ಫೂರ್ತಿಯ ಕೆಲವು ವಿಷಯಗಳನ್ನು ಹೇಳಬೇಕೆಂಬುದು, ಅದರಲ್ಲಿ ಬರುವ ಕಥೆಗಳು ಮತ್ತು ಆಚಾರಗಳನ್ನು ವಿವರಿಸುವುದಾಗಿರಲಿಲ್ಲ. ಅವರು ತಮ್ಮ ಧರ್ಮಕ್ಕೆ ಸೇರಿದ ಎಲ್ಲೋ ಕೆಲವು ಆಚಾರಗಳನ್ನು ಮಾತ್ರ ಹೇಳಿದರು. ಆದರೆ ಅವುಗಳನ್ನು ಬಹಳ ಸ್ಪಷ್ಟವಾಗಿ ಎಲ್ಲರಿಗೂ ಅರ್ಥವಾಗುವ ರೀತಿಯಲ್ಲಿ ಹೇಳಿದರು. ಹಿಂದೂಧರ್ಮಕ್ಕೆ ಸೇರಿದ ಅನುಭಾವ ವಿಷಯಗಳನ್ನು ಅವರು ವಿವರವಾಗಿ ಹೇಳಿದರು.ಅವುಗಳಿಂದಲೇ, ಅನೇಕ ವೇಳೆ ತಪ್ಪಾಗಿ ವಿವರಿಸಲ್ಪಟ್ಟಿರುವ ಪುನರ್ಜನ್ಮ ಸಿದ್ಧಾಂತ ಬಂದಿರುವುದು. ಅವರ ಧರ್ಮವು ಕಾಲದ ವ್ಯತ್ಯಾಸವನ್ನು ನಂಬುವುದಿಲ್ಲ ಎಂಬುದನ್ನು ವಿವರಿಸಿದರು. ಎಲ್ಲರೂ ವರ್ತಮಾನ ಮತ್ತು ಭವಿಷ್ಯಕಾಲವನ್ನು ನಂಬುವಂತೆ ಬ್ರಹ್ಮಧರ್ಮವು ಭೂತಕಾಲವನ್ನೂ ನಂಬುವುದು. ಅವರ ಧರ್ಮವು ಮಾನವನ ‘ಆದಿಪಾಪ’ವನ್ನು ನಂಬುವುದಿಲ್ಲ ಎಂಬುದನ್ನು ಸ್ಪಷ್ಟಪಡಿಸಿದರು. ಮಾನವ ಪೂರ್ಣನಾಗ\break ಬಲ್ಲ. ಅದಕ್ಕಾಗಿ ಅವನು ಮಾಡಿದ ಪ್ರಯತ್ನ ಯಾವುದೂ ವಿಫಲವಾಗುವುದಿಲ್ಲ ಎಂದು ಆ ಧರ್ಮ ನಂಬುತ್ತದೆ. ಪರಿಶುದ್ಧರಾಗುವುದು, ಉತ್ತಮರಾಗುವುದು ಇವು ಭರವಸೆಯ ಮೇಲೆ ನಿಂತಿರಬೇಕು. ಮಾನವನ ಪ್ರಗತಿ ಎಂದರೆ ಅವನಲ್ಲಿ ಹಿಂದೆ ಇದ್ದ ಪೂರ್ಣತೆಗೆ ಹಿಂತಿರುಗುವುದು ಎಂದು ಅರ್ಥ. ಈ ಪರಿಪೂರ್ಣತೆಯು ಪಾವಿತ್ರ್ಯ ಮತ್ತು ಪ್ರೀತಿಯ ಅಭ್ಯಾಸದಿಂದ ಸಿದ್ಧಿಸುವುದು. ಇಲ್ಲಿ ಸ್ವಾಮೀಜಿ ಅವರು ತಮ್ಮ ದೇಶದ ಜನ ಈ ಗುಣಗಳನ್ನು ಹೇಗೆ ಅಭ್ಯಾಸ ಮಾಡಿರುವರು ಎಂಬುದನ್ನು ವಿವರಿಸಿದರು. ಭರತ ಖಂಡವು ದಬ್ಬಾಳಿಕೆಗೆ ತುತ್ತಾದ ಹಲವು ಜನಾಂಗಗಳಿಗೆ ಆಶ್ರಯವನ್ನು ಕೊಟ್ಟಿದೆ. ಟೈಟಸ್​ ಜೆರುಸಲೇಮ್​ ನಗರವನ್ನು ಧ್ವಂಸಮಾಡಿ, ಅಲ್ಲಿಯ ದೇವಾಲಯವನ್ನು ಹಾಳು ಮಾಡಿದಾಗ ಅಲ್ಲಿಂದ ಬಂದ ಯಹೂದ್ಯರಿಗೆ ಭರತಖಂಡ ಆಶ್ರಯವನ್ನು ಕೊಟ್ಟ ಸಂಗತಿಯನ್ನು ಅವರು ಉದಾಹರಿಸಿ\break ದರು.

ಹಿಂದೂಗಳು ಬಾಹ್ಯ ಆಕಾರಕ್ಕೆ ಅಷ್ಟು ಪ್ರಾಮುಖ್ಯತೆಯನ್ನು ಕೊಡುವುದಿಲ್ಲ ಎಂಬುದನ್ನು ಕಣ್ಣಿಗೆ ಕಟ್ಟುವಂತೆ ವಿವರಿಸಿದರು. ಕೆಲವು ವೇಳೆ ಒಂದೇ ಮನೆಯಲ್ಲಿರುವವರ ಇಷ್ಟದೇವತೆಗಳು ಬೇರೆ ಬೇರೆ ಆಗಿರಬಹುದು. ಆದರೆ ದೇವರ ಮುಖ್ಯ ಗುಣವಾದ ಪ್ರೀತಿಯನ್ನು ಆರಾಧಿಸಿ ಎಲ್ಲರೂ ಒಂದೇ ದೇವರನ್ನು ಪೂಜಿಸುತ್ತಿರುವರು ಎಂದರು. ಹಿಂದೂಗಳು ಎಲ್ಲಾ ಧರ್ಮಗಳಲ್ಲಿಯೂ ಒಳ್ಳೆಯ ಅಂಶಗಳಿವೆ ಎಂದು ನಂಬುವರು, ಧರ್ಮಗಳೆಲ್ಲ ಮನುಷ್ಯನು ಪವಿತ್ರನಾಗುವುದಕ್ಕೆ ಮಾಡಿದ ಸನ್ನಾಹಗಳು, ಆದಕಾರಣ ಎಲ್ಲಾ ಧರ್ಮಗಳನ್ನು ಗೌರವಿಸಬೇಕು ಎಂದರು. ಇದಕ್ಕೆ ಅವರು ವೇದಗಳಿಂದ (?)ಉದಾಹರಣೆಗಳನ್ನು ನೀಡಿದರು. ಧರ್ಮಗಳೆಲ್ಲ ಒಂದು ಚಿಲುಮೆಯಿಂದ ನೀರನ್ನು ತರುವುದಕ್ಕೆ ತಂದ ಹಲವು ಆಕಾರಗಳುಳ್ಳ ಪಾತ್ರೆಗಳು ಎಂದರು. ಪಾತ್ರೆಗಳ ಆಕಾರ ವಿಧವಿಧವಾಗಿರುವುವು. ಆದರೆ ಅವುಗಳೊಳಗೆ ಎಲ್ಲರೂ ತುಂಬಿಸುವುದಕ್ಕೆ ಬಂದಿರುವುದು ಸತ್ಯವೆಂಬ ಜಲವನ್ನು. ದೇವರಿಗೆ ಎಲ್ಲಾ ಧರ್ಮಗಳೂ ಗೊತ್ತಿದೆ. ಯಾವ ಹೆಸರಿನಿಂದ ಅವನನ್ನು ಕರೆದರೂ ಅದು ತನ್ನ ಹೆಸರೇ ಎಂದು ಗೊತ್ತಿದೆ.

ಕ್ರೈಸ್ತರು ಯಾವ ದೇವರನ್ನು ಪೂಜಿಸವರೋ ಅದೇ ದೇವರನ್ನು ಹಿಂದೂಗಳು ಪೂಜಿಸುತ್ತಾರೆ. ಹಿಂದೂಗಳಲ್ಲಿರುವ ಬ್ರಹ್ಮ, ವಿಷ್ಣು, ಮಹೇಶ್ವರರೆಂಬ ತ್ರಿಮೂರ್ತಿಗಳು ಸೃಷ್ಟಿ ಸ್ಥಿತಿ ಪ್ರಲಯ ಕಾರ್ಯದಲ್ಲಿ ತೊಡಗಿರುವ ಭಗವಂತನ ವಿವಿಧ ಅಭಿವ್ಯಕ್ತಿಗಳು. ಮೂರು ದೇವರನ್ನೂ ಬೇರೆ ಬೇರೆ ಎಂದು ಭಾವಿಸುವುದು ತಪ್ಪು. ಸಾಧಾರಣ ಮಾನವರಿಗೆ ನೈತಿಕ ವಿಷಯವನ್ನು ಚೆನ್ನಾಗಿ ಒತ್ತಿ ಹೇಳಲು ಮಾಡಿದ ಪ್ರಯತ್ನ ಇದು. ಅದರಂತೆಯೇ ಹಿಂದೂದೇವತೆಗಳ ವಿಗ್ರಹಗಳು ಕೂಡ ದಿವ್ಯ ಗುಣಗಳ ಸಂಕೇತಗಳು ಅಷ್ಟೆ.

ಹಿಂದೂಗಳ ಅವತಾರ ಸಿದ್ಧಾಂತವನ್ನು ವಿವರಿಸುವಾಗ ಅವರು ಶ‍್ರೀಕೃಷ್ಣನ ಜೀವನವನ್ನು ಉಲ್ಲೇಖಿಸಿದರು. ಅವನು ನಿಷ್ಕಳಂಕ ಗರ್ಭಧಾರಣೆಯ ಮೂಲಕ ಜನಿಸಿದವನು, ಮತ್ತು ಅವನ ಜೀವನ ಕಥೆ ಜೀಸಸ್​ ಕ್ರಿಸ್ತನ ಜೀವನವನ್ನು ಬಹು ಮಟ್ಟಿಗೆ ಹೋಲುವುದು. ಶ‍್ರೀಕೃಷ್ಣನ ಉಪದೇಶ, ಪ್ರೀತಿಗಾಗಿ ಪ್ರೀತಿ ಎಂಬುದು. “ದೇವರ ಅಂಜಿಕೆ ಧರ್ಮದ ಪ್ರಾರಂಭವಾದರೆ, ದೇವರ ಪ್ರೇಮ ಧರ್ಮದ ಮುಕ್ತಾಯ” ಎಂದು ಹೇಳಿದರು.

ಅವರ ಇಡೀ ಉಪನ್ಯಾಸವನ್ನು ಇಲ್ಲಿ ಉಲ್ಲೇಖಿಸಲು ಸಾಧ್ಯವಿಲ್ಲ. ಅದು ಮಾನವ ಸಹೋದರತ್ವಕ್ಕಾಗಿ ಮಾಡಿದ ಒಂದು ಅದ್ಭುತವಾದ ಬಿನ್ನಹ ಮತ್ತು ಅತಿ ಸುಂದರವಾದ ಧರ್ಮವನ್ನು ಪ್ರತಿಪಾದಿಸಲು ಮಡಿದ ಚಮತ್ಕಾರವಾದ ವಾಗ್ವೈಖರಿ. ಅವರ ಭಾಷಣದ ಮುಕ್ತಾಯ ತುಂಬಾ ಸೊಗಸಾಗಿತ್ತು. ತಾವು ಕ್ರಿಸ್ತನನ್ನು ಸ್ವೀಕರಿಸುವುದಕ್ಕೆ ಸಿದ್ಧನಾಗಿರುವೆ, ಆದರೆ ಕೃಷ್ಣ ಮತ್ತು ಬುದ್ಧರಿಗೂ ನಾವು ತಲೆಬಾಗಬೇಕು ಎಂದರು. ನಮ್ಮ ನಾಗರಿಕತೆಯಲ್ಲಿರುವ ಕ್ರೌರ್ಯವನ್ನು ನೋಡಿದಾಗ, ಪ್ರಗತಿಯ ಹೆಸರಿನಲ್ಲಿ ನಡೆಯುತ್ತಿರುವ ಈ ಕ್ರೌರ್ಯಗಳಿಗೆ ಕ್ರಿಸ್ತನು ಜವಾಬ್ದಾರನೆನ್ನ ಲಾಗುವುದಿಲ್ಲ, ಎಂದರು.


\section[ಪುನರ್ಜನ್ಮ]{ಪುನರ್ಜನ್ಮ\protect\footnote{* C.W. Vol. VII P. 423}}

\begin{center}
(೧೮೯೪ರ ಜನವರಿ ೧೯ರಂದು ಮೆಂಫಿಸ್​ನಲ್ಲಿ ಮಾಡಿದ ಭಾಷಣ ‘ಅಪೀಲ್​ ಆವಲಾಂಚ್​’ ಪತ್ರಿಕೆಯ ವರದಿ)
\end{center}

\vskip -0.35cm

ನೀಳವಾದ ಉಡುಪನ್ನು ಧರಿಸಿ, ತಲೆಗೆ ರುಮಾಲನ್ನು ಸುತ್ತಿದ್ದ ಸ್ವಾಮಿ ವಿವೇಕಾನಂದರು ಪುನಃ ನಿನ್ನೆ ರಾತ್ರಿ ಮೂರನೆಯ ಬೀದಿಯಲ್ಲಿರುವ ಲಾ ಸೆಲಿಟಿ ಅಕಾಡೆಮಿಯಲ್ಲಿ ಸಾಕಷ್ಟು ಸಂಖ್ಯೆಯಲ್ಲಿ ನೆರೆದಿದ್ದ, ಗುಣಗ್ರಾಹಿಗಳಾದ ಸಭಿಕರನ್ನು ಉದ್ದೇಶಿಸಿ ಉಪನ್ಯಾಸ ಮಾಡಿದರು.

ಉಪನ್ಯಾಸದ ವಿಷಯ ಜೀವಿಯ ಪುನರ್ಜನ್ಮ. ಬಹುಶಃ ವಿವೇಕಾನಂದರು ಈ ವಿಷಯದ ಮೇಲೆ ಮಾತನಾಡಿದಾಗ ತೋರಿದ ವಾಗ್ಮಿತೆಯನ್ನು ಬೇರೆ ವಿಷಯದ ಮೇಲೆ ತೋರಲಿಲ್ಲ ಎಂದು ಭಾವಿಸಬಹುದು. ಪೌರಸ್ತ್ಯ ದೇಶಗಳಲ್ಲಿ ಪುನರ್ಜನ್ಮ ಸಿದ್ಧಾಂತವನ್ನು ಬಹಳವಾಗಿ ನಂಬುವರು. ಆ ಸಿದ್ಧಾಂತವನ್ನು ದೇಶದ ಒಳಗೆ ಮತ್ತು ಹೊರಗೆ ಎಲ್ಲಿ ಬೇಕಾದರೂ, ಸಮರ್ಥಿಸಲು ಸಿದ್ಧರಾಗಿರುವರು. ವಿವೇಕಾನಂದರು ಹೀಗೆ ಹೇಳಿದರು:

“ಬಹುಶಃ ನಿಮ್ಮಲ್ಲಿ ಬಹುಜನರಿಗೆ, ಪುರಾತನ ಧರ್ಮಗಳಲ್ಲೆಲ್ಲಾ ಅತ್ಯಂತ ಪುರಾತನ ಧಾರ್ಮಿಕನಂಬಿಕೆ ಇದು ಎಂಬುದು ಗೊತ್ತಿಲ್ಲದೆ ಇರಬಹುದು. ಇದು ಪಾರ್ಸಿಗಳಿಗೆ, ಯಹೂದ್ಯರಿಗೆ ತಿಳಿದಿತ್ತು. ಕ್ರೈಸ್ತಧರ್ಮದ ಪ್ರಪ್ರಥಮದಲ್ಲಿ ಬಂದ ಪಾದ್ರಿಗಳು ಇದನ್ನು ನಂಬಿದ್ದರು. ಅರಬ್​ ಜನರಲ್ಲಿ ಕೂಡ ಇದೊಂದು ಸಾಮಾನ್ಯ ನಂಬಿಕೆಯಾಗಿತ್ತು. ಇದು ಈಗಲೂ ಕೂಡ ಹಿಂದೂಗಳಲ್ಲಿ ಮತ್ತು ಬೌದ್ಧರಲ್ಲಿ ಇರುವುದು.

“ವಿಜ್ಞಾನದ ಯುಗ ಬರುವವರೆಗೂ ಈ ಸ್ಥಿತಿ ಮುಂದುವರಿಯಿತು. ವಿಜ್ಞಾನ ಎಂದರೆ ಶಕ್ತಿಯ ವಿಷಯವನ್ನು ಕುರಿತು ಆಲೋಚಿಸುವುದು. ಈಗ ಪಾಶ್ಚಾತ್ಯರಾದ ನೀವು ಪುನರ್ಜನ್ಮ ಸಿದ್ಧಾಂತವು ನೀತಿಜೀವನಕ್ಕೆ ಸಹಾಯಕವಲ್ಲ ಎಂದು ಭಾವಿಸುವಿರಿ. ಪುನರ್ಜನ್ಮದ ಸಿದ್ಧಾಂತವನ್ನು ಚೆನ್ನಾಗಿ ತಿಳಿದುಕೊಳ್ಳಬೇಕಾದರೆ, ಅದಕ್ಕೆ ಸಂಬಂಧ ಪಟ್ಟ ತಾರ್ಕಿಕ ಮತ್ತು ತಾತ್ವಿಕ ವಿಷಯಗಳನ್ನು ತಿಳಿದುಕೊಳ್ಳಬೇಕಾದರೆ, ನಾವು ಅದನ್ನು ಆಮೂಲಾಗ್ರವಾಗಿ ಪರಿಶೀಲಿಸ\break ಬೇಕು. ನಾವೆಲ್ಲ ಈ ಪ್ರಪಂಚವನ್ನು ಆಳುವ ನೀತಿಬದ್ಧವಾದ ಒಬ್ಬ ವ್ಯಕ್ತಿಯನ್ನು ಒಪ್ಪಿಕೊಳ್ಳುತ್ತೇವೆ. ಆದರೂ ಪ್ರಕೃತಿಯಲ್ಲಿ ನ್ಯಾಯಕ್ಕಿಂತ ಅನ್ಯಾಯವೇ ಕಾಣುತ್ತದೆ. ಒಬ್ಬನು ಒಳ್ಳೆಯ ಅವಕಾಶಗಳುಳ್ಳ ವಾತಾವರಣದೊಳಗೆ ಜನಿಸುವನು. ಅವನಿಗೆ ಜೀವನದಲ್ಲಿ ಎಲ್ಲಾ ಅನುಕೂಲವಾಗಿ ಒದಗುವುದು. ಎಲ್ಲಾ ಸುಖಪ್ರದವಾಗಿರುತ್ತವೆ, ಮೇಲುಮೇಲಕ್ಕೆ ಹೋಗಲು ಸಹಕಾರಿಯಾಗಿರುತ್ತವೆ. ಮತ್ತೊಬ್ಬ ಹುಟ್ಟುವನು. ಅವನಿಗೆ ಹೆಜ್ಜೆಹೆಜ್ಜೆಗೆ ಸುತ್ತಲಿರುವ ಜನಗಳಿಂದ ಕಂಟಕಗಳು ಬರುತ್ತಿರುತ್ತವೆ. ಅವನು ಸಮಾಜದಿಂದ ತಿರಸ್ಕರಿಸಲ್ಪಟ್ಟವ\break ನಾಗಿ ದುಃಖದಾರಿದ್ರ್ಯದಲ್ಲಿ ನರಳಿ ಸಾಯುವನು. ಸುಖವನ್ನು ಹಂಚುವುದರಲ್ಲಿ ಏತಕ್ಕೆ ಇಷ್ಟೊಂದು ಪಕ್ಷಪಾತ?

“ಪುನರ್ಜನ್ಮ ಸಿದ್ಧಾಂತವು ಇಂತಹ ವ್ಯತ್ಯಾಸಕ್ಕೆ ಕಾರಣವನ್ನು ಕೊಡುವುದು. ಇದು ನಮ್ಮನ್ನು ನೀತಿಬಾಹಿರರನ್ನಾಗಿ ಮಾಡುವ ಬದಲು ನ್ಯಾಯದ ವಿಷಯವನ್ನು ತಿಳಿಸುತ್ತದೆ. ನಿಮ್ಮಲ್ಲಿ ಕೆಲವರು ‘ಇದು ದೇವರ ಇಚ್ಛೆ,’ ಎಂದು ಹೇಳಬಹುದು. ಅದು ಉತ್ತರವೇ ಅಲ್ಲ. ಅದು ಅವೈಜ್ಞಾನಿಕ. ಪ್ರತಿಯೊಂದಕ್ಕೂ ಒಂದು ಕಾರಣವಿದೆ. ದೇವರೇ ಎಲ್ಲಕ್ಕೂ ಕಾರಣ ಎಂದು ಹೇಳಿದರೆ ದೇವರನ್ನು ದೊಡ್ಡ ಅನೀತಿವಂತನನ್ನಾಗಿ ಮಾಡಿದಂತಾಗುವುದು. ಆದರೆ ಜಡವಾದವು ಮತ್ತೊಂದರಷ್ಟೇ ಅತಾರ್ಕಿಕ ವಾದುದು. ನಮಗೆ ತಿಳಿದಿರುವ ಮಟ್ಟಿಗೆ ಎಲ್ಲಕ್ಕೂ ಒಂದು ಕಾರಣವಿರಬೇಕು. ಆದ್ದರಿಂದ ಈ ಕಾರಣಗಳಿಂದಾಗಿ ನಮಗೆ ಪುನರ್ಜನ್ಮ ಸಿದ್ಧಾಂತ ಆವಶ್ಯಕ. ಇಲ್ಲಿ ನಾವೆಲ್ಲ ಹುಟ್ಟಿರುವೆವು. ಇದೇ ಮೊದಲನೆ ಸೃಷ್ಟಿಯೇ? ಸೃಷ್ಟಿ ಎಂದರೆ ಶೂನ್ಯದಿಂದ ಏನಾದರೂ ಬರುವುದೇ? ಸರಿಯಾಗಿ ವಿಶ್ಲೇಷಣೆ ಮಾಡಿದರೆ ಶೂನ್ಯದಿಂದ ಏನಾದರೂ ಬರುವುದು ಎನ್ನುವುದು ಅವಿವೇಕ. ಇದು ಸೃಷ್ಟಿಯಲ್ಲ, ಅಭಿವ್ಯಕ್ತಿ.

“ಕಾರಣವಿಲ್ಲದೆ ಏನೂ ಬರಲಾರದು. ನಾನು ಬೆಂಕಿಗೆ ಕೈಯನ್ನಿಟ್ಟ ತಕ್ಷಣವೇ ಕೈ ಸುಡುವುದು. ಬೆಂಕಿಗೆ ಕೈ ಇಟ್ಟುದು ಸುಡುವುದಕ್ಕೆ ಕಾರಣ ಎಂದು ನನಗೆ ಗೊತ್ತಿದೆ. ಪ್ರಕೃತಿಯ ವಿಷಯವನ್ನು ತೆಗೆದುಕೊಂಡರೆ, ಪ್ರಕೃತಿಯು ಇಲ್ಲದ ಕಾಲವೆ ಎಂದಿಗೂ ಇರಲಿಲ್ಲ. ಏಕೆಂದರೆ ಕಾರಣ ಯಾವಾಗಲೂ ಇದ್ದೇ ಇರುವುದು. ಕೇವಲ ವಾದಕ್ಕಾಗಿ, ಏನೂ ಇಲ್ಲದ ಒಂದು ಕಾಲವಿತ್ತು ಎಂದು ಭಾವಿಸೋಣ. ಆಗ ಈಗ ನಮ್ಮೆದುರಿಗೆ ಇರುವ ಭೌತವಸ್ತುಗಳ ರಾಶಿ ಎಲ್ಲಿತ್ತು? ಹೊಸದಾಗಿ ಏನನ್ನಾದರೂ ಸೃಷ್ಟಿಸುವುದೆಂದರೆ ವಿಶ್ವಕ್ಕೆ ಅಷ್ಟರ ಮಟ್ಟಿನ ಹೊಸ ಶಕ್ತಿಯನ್ನು ಸೇರಿಸಿದಂತೆ. ಇದು ಸಾಧ್ಯವಿಲ್ಲ. ಹಳೆಯ ವಸ್ತುಗಳನ್ನು ಬೇಕಾದರೆ ಹೊಸದಾಗಿ ಮಾಡಬಹುದು. ಆದರೆ ಪ್ರಪಂಚಕ್ಕೆ ಹೊಸದಾಗಿ ಏನನ್ನೂ ಸೇರಿಸುವುದಕ್ಕೆ ಆಗುವುದಿಲ್ಲ.

“ಪುನರ್ಜನ್ಮ ಸಿದ್ಧಾಂತವನ್ನು ನಾವು ಗಣಿತಶಾಸ್ತ್ರದ ಮೂಲಕ ಪ್ರತಿಪಾದಿಸುವುದಕ್ಕೆ ಆಗುವುದಿಲ್ಲ. ತರ್ಕಶಾಸ್ತ್ರದ ಪ್ರಕಾರ ಊಹೆ ಮತ್ತು ಸಿದ್ಧಾಂತಗಳನ್ನು ನಂಬಕೂಡದು. ಆದರೆ ಜೀವನವನ್ನು ವಿವರಿಸುವುದಕ್ಕೆ ಪುನರ್ಜನ್ಮದಷ್ಟು ಉತ್ತಮವಾದ ಬೇರಾವ ಸಿದ್ಧಾಂತವೂ ಇಲ್ಲ ಎಂಬುದು ನನ್ನ ಅಭಿಪ್ರಾಯ.

“ನಾನು ಮಿನಿಯಾಪೊಲಿಸ್​ ಊರನ್ನು ಬಿಟ್ಟಾಗ ರೈಲಿನಲ್ಲಿ ಒಂದು ವಿಚಿತ್ರ ಘಟನೆ ನಡೆಯಿತು. ರೈಲಿನಲ್ಲಿ ಒಬ್ಬ ದನಕಾಯುವವನು ಇದ್ದನು. ಅವನು ಒರಟು ಸ್ವಭಾವದವನು. ಅವನು ಅತಿ ಆಚಾರಶೀಲರಾದ ಪ್ರಿಸ್ಬಿಟೇರಿಯನ್​ ಪಂಗಡಕ್ಕೆ ಸೇರಿದವನು. ಅವನು ನನ್ನ ಬಳಿಗೆ ಬಂದು ನಾನು ಯಾವ ಊರಿನವನೆಂದು ವಿಚಾರಿಸಿದನು. ನಾನು ಇಂಡಿಯಾ ದೇಶದವನೆಂದು ಹೇಳಿದೆ. ‘ನೀವು ಯಾವ ಧರ್ಮದವರು? ಎಂದು ಕೇಳಿದ. ‘ಹಿಂದೂ’ ಎಂದುತ್ತರಿಸಿದೆ. ‘ಹಾಗಾದರೆ ನೀವು ನರಕಕ್ಕೆ ಹೋಗಲೇಬೇಕು ’ ಎಂದನು. ಅವನಿಗೆ ನಾನು ಪುನರ್ಜನ್ಮ ಸಿದ್ಧಾಂತವನ್ನು ಹೇಳಿದೆ. ಇದನ್ನು ಅವನಿಗೆ ವಿವರಿಸಿದ ಮೇಲೆ ಅವನು ತಾನು ಯಾವಾಗಲೂ ಅದರಲ್ಲಿ ನಂಬಿದ್ದೆ ಎಂದು ಹೇಳಿದನು. ಏಕೆಂದರೆ ಅವನು ಒಂದು ದಿನ ಸೌದೆಯನ್ನು ಒಡೆಯುತ್ತಿದ್ದಾಗ, ಅವನ ತಂಗಿ ಅವನ ಬಟ್ಟೆಯನ್ನು ಹಾಕಿಕೊಂಡು ಬಂದು ನಾನು ಹಿಂದೆ ಗಂಡಸಾಗಿದ್ದೆ ಎಂದು ಹೇಳಿದಳು. ಅದಕ್ಕೇ ಅವನು ಪುನರ್ಜನ್ಮ ಸಿದ್ಧಾಂತವನ್ನು ನಂಬುವುದು. ಈ ಸಿದ್ಧಾಂತದ ಆಧಾರವೇ ಇದು: ಯಾರ ಕೆಲಸ ಒಳ್ಳೆಯದಾಗಿರುವುದೊ ಅವನು ಮೇಲು ಮಟ್ಟಕ್ಕೆ ಸೇರಿದವನು, ಅದರಂತೆಯೇ ಕೀಳು ಕೆಲಸ ಮಾಡುವವನು ಕೀಳು ಮಟ್ಟಕ್ಕೆ ಸೇರಿರುವನು.

“ಈ ಸಿದ್ಧಾಂತದಲ್ಲಿ ಮತ್ತೊಂದು ಒಳ್ಳೆಯ ಅಂಶವಿದೆ. ಇದು ನೀತಿಗೆ ಕ್ರಿಯೋತ್ತೇಜಕ ಶಕ್ತಿಯನ್ನು ಕೊಡುವುದು. ಆಗಿದ್ದು ಆಗಿಹೋಯಿತು. ಆದರೆ ಇನ್ನೂ ಚೆನ್ನಾಗಿ ಆಗಿದ್ದರೆ ಉತ್ತಮ ಎನ್ನುವುದು. ಪುನಃ ನಿಮ್ಮ ಬೆರಳನ್ನು ಬೆಂಕಿಗೆ ಇಡಬೇಡಿ. ಪ್ರತಿಯೊಂದು ಕ್ಷಣವೂ ಅದು ನಮಗೆ ಹೊಸ ಅವಕಾಶವನ್ನು ಕಲ್ಪಿಸಿಕೊಡುವುದು.”

ವಿವೇಕಾನಂದರು ಈ ರೀತಿ ಸ್ವಲ್ಪ ಹೊತ್ತು ಮಾತನಾಡಿದರು. ಸಭಿಕರು ಮಧ್ಯೆ ಮಧ್ಯೆ ಕರತಾಡನ ಮಾಡುತ್ತಿದ್ದರು.

ಸ್ವಾಮಿ ವಿವೇಕಾನಂದರು ಪುನಃ ಈ ದಿನ ಮಧ್ಯಾಹ್ನ ಲಾ ಸೆಲಿಟಿ ಅಕಾಡೆಮಿಯಲ್ಲಿ ನಾಲ್ಕು ಘಂಟೆಗೆ “ಭರತಖಂಡದ ಆಚಾರ ವಿಚಾರಗಳು” ಎಂಬ ವಿಷಯದ ಮೇಲೆ ಉಪನ್ಯಾಸ ಮಾಡುವರು.


\section[ತುಲನಾತ್ಮಕ ಮತಧರ್ಮಶಾಸ್ತ್ರ]{ತುಲನಾತ್ಮಕ ಮತಧರ್ಮಶಾಸ್ತ್ರ\protect\footnote{* C.W. Vol. VII P. 425}}

\begin{center}
(ಮೆಂಫಿಸ್​ನಲ್ಲಿ ಜನವರಿ ೨೧, ೧೮೯೪ರಲ್ಲಿ ಮಾಡಿದ ಉಪನ್ಯಾಸ, ಅಪೀಲ್​ ಅವಲಾಂಚ್​ ಪತ್ರಿಕೆಯ ವರದಿ)
\end{center}

\vskip -0.3cm

“ತುಲನಾತ್ಮಕ ಮತಧರ್ಮಶಾಸ್ತ್ರಗಳು” ಎಂಬುದೇ ಸ್ವಾಮಿ ವಿವೇಕಾನಂದರು ಕಳೆದ ರಾತ್ರಿ ಹಿಬ್ರು ಯುವಕ ಸಂಘದ ಹಾಲಿನಲ್ಲಿ ಮಾಡಿದ ಭಾಷಣದ ವಿಷಯ. ಅದು ಉಪನ್ಯಾಸಗಳ ಸರಣಿಯಲ್ಲಿ ಅತ್ಯಂತ ಮುಖ್ಯವಾದ ಉಪನ್ಯಾಸವಾಗಿತ್ತು. ವಿದ್ವಾಂಸರಾದ ಉಪನ್ಯಾಸಕರ (ಸ್ವಾಮೀಜಿಯ) ಮೇಲೆ ಜನರಿಗಿರುವ ಗೌರವ ಈ ಉಪನ್ಯಾಸದಿಂದ ಹೆಚ್ಚಿತು ಎನ್ನುವುದರಲ್ಲಿ ಸಂದೇಹವಿಲ್ಲ.

ಇದುವರೆಗೂ ಸ್ವಾಮಿ ವಿವೇಕಾನಂದರು ಯಾವುದಾದರೂ ಧರ್ಮಸಂಸ್ಥೆಯ ಸಹಾ\break ಯಾರ್ಥವಾಗಿ ಉಪನ್ಯಾಸ ಮಾಡುತ್ತಿದ್ದರು. ಇವುಗಳಿಗೆ ಅವರಿಂದ ಸಾಕಷ್ಟು ಆರ್ಥಿಕವಾಗಿ ಸಹಾಯವಾಗಿರುವುದರಲ್ಲಿ ಸಂಶಯವಿಲ್ಲ. ಆದರೆ ಕಳೆದ ರಾತ್ರಿ ಅವರು ತಮ್ಮ ಸಹಾರ್ಯಾಥವಾಗಿ\break ವಾಗಿ ಉಪನ್ಯಾಸವನ್ನು ಮಾಡಿದರು. ಉಪನ್ಯಾಸವನ್ನು ಮಿಸ್ಟರ್​ ಹು. ಎಲ್​. ಬ್ರಿಂಕ್ಲೇ ಅವರು ಅಣಿಗೊಳಿಸಿದ್ದರು. ಅವರು ವಿವೇಕಾನಂದರ ದೊಡ್ಡ ಸ್ನೇಹಿತರೂ ಅವರನ್ನು ಮೆಚ್ಚುವವರೂ ಆಗಿರುವರು. ಈ ಊರಿನಲ್ಲಿ ಪೌರಸ್ತ್ಯ ದೇಶದ ಪ್ರಖ್ಯಾತ ವ್ಯಕ್ತಿಯ ಕೊನೆಯ ಉಪನ್ಯಾಸವನ್ನು ಕೇಳಲು ಸುತ್ತಮುತ್ತಲಿನಿಂದ ಸುಮಾರು ಇನ್ನೂರು ಜನ ಬಂದಿದ್ದರು.

ತಮ್ಮ ಉಪನ್ಯಾಸದ ವಿಷಯಕ್ಕೆ ಸಂಬಂಧಿಸಿದಂತೆ ಅವರು ಎತ್ತಿದ ಮೊದಲನೆ ಪ್ರಶ್ನೆ ಯಾವುದೆಂದರೆ - “ಧರ್ಮಗಳಲ್ಲಿ ನಿಜವಾಗಿಯೂ, ಅವುಗಳ ಮತತತ್ತ್ವಗಳು ಸೂಚಿಸು\break ವಂತೆ, ವ್ಯತ್ಯಾಸಗಳು ಇರಬಲ್ಲವೆ” ಎಂಬುದು.

ಅವರು, ಈಗ ಧರ್ಮಗಳಲ್ಲಿ ಯಾವ ವ್ಯತ್ಯಾಸವೂ ಇಲ್ಲ ಎಂದು ಸಾಧಿಸಿದರು. ಎಲ್ಲಾ ಧರ್ಮಗಳು ಹಿಂದಿನಿಂದ ಇದುವರೆಗೆ ಹೇಗೆ ಮುಂದುವರಿದು ಬಂದಿವೆ ಎಂಬುದನ್ನು ತೋರಿದರು. ದೇಶದ ಭಾವನೆಯ ವಿಷಯದಲ್ಲಿ ಸಾಧಾರಣವಾಗಿ ಅನಾಗರಿಕ ಮಾನವರಲ್ಲಿ ಕೆಲವು ವ್ಯತ್ಯಾಸಗಳು ಇದ್ದಿರಬಹುದು. ಆದರೆ ಜಗತ್ತು ನೈತಿಕ ಮತ್ತು ಬೌದ್ಧಿಕ ವಿಷಯದಲ್ಲಿ ಮುಂದುವರಿದಂತೆ ವ್ಯತ್ಯಾಸಗಳು ಕಡಿಮೆಯಾಗುತ್ತ ಬಂದವು. ಕೊನೆಗೆ ಅವುಗಳೆಲ್ಲ ಈಗ ಮಾಯವಾಗಿವೆ. ಈಗ ಸರ್ವವ್ಯಾಪಿಯಾದ ಸಿದ್ಧಾಂತವೊಂದು ಜಾರಿಯಲ್ಲಿದೆ-ಅದೇ ನಿರಪೇಕ್ಷವಾದ ಅಸ್ತಿತ್ವಕ್ಕೆ ಸಂಬಂಧಿಸಿದುದು.

ಯಾವುದಾದರೂ ಒಂದು ಬಗೆಯ ದೇವರನ್ನು ನಂಬದ ಯಾವ ಕಾಡುಜನರೂ ಇಲ್ಲ ಎಂದು ಅವರು ಹೇಳಿದರು.

“ಆಧುನಿಕ ವಿಜ್ಞಾನ ಇದು ಅಪೌರುಷೇಯವಾಗಿ ಬಂದುದೆ ಇಲ್ಲವೆ ಎಂಬ ವಿಷಯದಲ್ಲಿ ಏನನ್ನೂ ಹೇಳುವುದಿಲ್ಲ. ಕಾಡುಜನಾಂಗದವರಲ್ಲಿರುವ ಪ್ರೀತಿ ಅಷ್ಟು ತೀವ್ರವಾದುದಲ್ಲ. ಅವರು ಭಯದಲ್ಲಿ ಬಾಳುವರು. ಅವರ ಮೂಢ ಕಲ್ಪನೆಗಳಲ್ಲಿ ನಮಗೆ ತೊಂದರೆಯನ್ನು ಕೊಡುವ ಕೆಲವು ಭೂತಪ್ರೇತಗಳಿವೆ. ಅವರು ಅವುಗಳ ಅಂಜಿಕೆಯಿಂದ ತತ್ತರಿಸುವರು. ತನಗೆ ಯಾವುದು ಇಷ್ಟವೋ ಅದನ್ನು ಮಾಡಿದರೆ ಅವಕ್ಕೆ ತೃಪ್ತಿಯಾಗುವುದು ಎಂದು ಭಾವಿಸುವನು. ತನಗೆ ಯಾವುದು ಸಮಾಧಾನವನ್ನು ತರುವುದೋ ಅದೇ ಆ ಪ್ರೇತಕ್ಕೂ ಸಮಾಧಾನವನ್ನು ತರುವುದು ಎಂದು ಭಾವಿಸುವನು. ಇದಕ್ಕಾಗಿ ಅವನು ತನ್ನ ಕಾಡು ಜನರಿಗೆ ವಿರೋಧವಾಗಿಯಾದರೂ ಕೆಲಸ ಮಾಡುವನು.”

ಉಪನ್ಯಾಸಕರು ಚಾರಿತ್ರಿಕವಾಗಿ ಕಾಡುಮನುಷ್ಯನು ಪಿತೃಪೂಜೆಯಿಂದ ಆನೆಯ ಪೂಜೆಗೆ ಬಂದನೆಂದೂ, ಅಲ್ಲಿಂದ ಮಳೆ ಮತ್ತು ಸಿಡಿಲುಗಳ ದೇವತೆಯ ಪೂಜೆಗೆ ಬಂದ\break ನೆಂದೂ ತೋರಿದರು. ಆಗಿನ ಧರ್ಮದಲ್ಲಿ ಹಲವು ದೇವತೆಗಳ ಆರಾಧನೆಗಳು ಇದ್ದವು. “ಸೂರ್ಯೋದಯದ ಸೌಂದರ್ಯ, ಸೂರ್ಯಾಸ್ತಮದ ಭವ್ಯತೆ, ರಹಸ್ಯಾತ್ಮಕವಾಗಿ ಕಾಣುವ ಅನಂತ ತಾರಾಖಚಿತ ನಭೋಮಂಡಲ, ಸಿಡಿಲು ಮತ್ತು ಮಿಂಚುಗಳ ಹಿಂದೆ ಇರುವ ಅಲೌಕಿಕತೆ ಇವು ಕಾಡುಮನುಷ್ಯನನ್ನು ಅದ್ಭುತವಾಗಿ ಆಕರ್ಷಿಸಿದವು. ಅವನು ಇವುಗಳನ್ನು ವಿವರಿಸಲಾರದೆ ಹೋದನು. ತನ್ನ ಕಣ್ಣೆದುರಿಗೆ ಇರುವ ಅನಂತಾಕಾಶವನ್ನು ಯಾರೋ ಎಲ್ಲಕ್ಕಿಂತಲೂ ಪರಾಕ್ರಮಶಾಲಿಯಾದ ವ್ಯಕ್ತಿಯು ಆಳುತ್ತಿರುವನು ಎಂದು ಭಾವಿಸಿದರು” ಎಂದರು ವಿವೇಕಾನಂದರು.

ಅನಂತರ ಏಕೇಶ್ವರ ಭಾವನೆಯ ಕಾಲವು ಬಂದಿತು. ಹಲವು ದೇವರುಗಳು ಮಾಯವಾಗಿ ಒಂದೇ ದೇವರಲ್ಲಿ ಪರ್ಯವಸಾನವಾಯಿತು. ಅವನೇ ದೇವರ ದೇವ, ಮತ್ತು ವಿಶ್ವೇಶ್ವರ. ಅನಂತರ ಉಪನ್ಯಾಸಕರು “ನಾವು ಅವನಲ್ಲಿ ಜೀವಿಸಿರುವೆವು. ಅವನಲ್ಲೆ ಚಲಿಸುತ್ತಿರುವೆವು, ಅವನೇ ಚಲನೆ” ಎಂದು ಹೇಳುವ ಸ್ಥಿತಿಯನ್ನು ಮುಟ್ಟುವವರೆಗೆ ಆರ್ಯರು ಮುಂದುವರಿದ ಬಗೆಯನ್ನು ವಿವರಿಸಿದರು. ಅನಂತರ ತತ್ತ್ವಶಾಸ್ತ್ರದಲ್ಲಿ ಯಾವು\break ದನ್ನು ವಿಶ್ವದೇವೈಕ್ಯವಾದ \enginline{(Pantheism)} ವೆನ್ನುತ್ತಾರೆಯೋ ಆ ಕಾಲವು ಬಂದಿತು. ಈ ಜನಾಂಗವು ಹಲವು ದೇವರುಗಳು, ಒಂದು ದೇವರು ಮತ್ತು ಈ ಪ್ರಪಂಚವೇ ದೇವರು ಎಂಬ ಭಾವನೆಗಳನ್ನೆಲ್ಲ ತಿರಸ್ಕರಿಸಿ, “ನನ್ನ ಆತ್ಮನ ಆತ್ಮನೇ ನಿಜವಾದ ಅಸ್ತಿತ್ವ, ನನ್ನ ಪ್ರಕೃತಿಯೇ ನನ್ನ ಅಸ್ತಿತ್ವ, ಅದು ನನ್ನಲ್ಲಿಗೆ ವಿಸ್ತರಿಸುವುದು” ಎಂದು ಹೇಳಿತು.

ವಿವೇಕಾನಂದರು ಅನಂತರ ಬೌದ್ಧಧರ್ಮವನ್ನು ತೆಗೆದುಕೊಂಡರು. ಬೌದ್ಧರು ದೇವರು ಇದ್ದಾನೆ ಎನ್ನಲೂ ಇಲ್ಲ, ಇಲ್ಲ ಎನ್ನಲೂ ಇಲ್ಲ, ಎಂದರು. ಈ ವಿಷಯದಲ್ಲಿ ಬುದ್ಧನನ್ನು ಪ್ರಶ್ನಿಸಿದಾಗ ಅವನು ಹೀಗೆ ಮಾತ್ರ ಹೇಳುತ್ತಿದ್ದ: “ನೀವು ದುಃಖವನ್ನು ನೋಡುತ್ತಿರುವಿರಿ. ಅದನ್ನು ತಗ್ಗಿಸುವುದಕ್ಕೆ ಯತ್ನಿಸಿ.” ಬೌದ್ಧರ ಪ್ರಕಾರ ದುಃಖವು ಯಾವಾಗಲೂ ಇರುವುದು. ಸಮಾಜದಲ್ಲಿ ಅದು ಎಷ್ಟು ಮಟ್ಟಿಗೆ ಇದೆ ಎಂಬುದು ವ್ಯಕ್ತವಾಗುವುದು. ಮಹಮ್ಮದೀಯರು ಹಿಬ್ರುಗಳ, ಹಳೆಯ ಒಡಂಬಡಿಕೆಯನ್ನೂ, ಕ್ತೈಸ್ತರ ಹೊಸ ಒಡಂಬಡಿಕೆಯನ್ನೂ ನಂಬುತ್ತಿದ್ದರು. ಅವರು ಕ್ರೈಸ್ತರನ್ನು ಇಷ್ಟಪಡುವುದಿಲ್ಲ. ಏಕೆಂದರೆ ಕ್ರೈಸ್ತರು ದೈವದ್ವೇಷಿಗಳು, ಮನುಷ್ಯ ಪೂಜೆಯನ್ನು ಬೋಧಿಸುತ್ತಾರೆ. ಮಹಮ್ಮದ್​ ತನ್ನ ಚಿತ್ರವನ್ನು ಕೂಡ ಇಟ್ಟುಕೊಳ್ಳ\break ಕೂಡದೆಂದು ಹೇಳಿದ್ದನು.

ಅವರು ಹೇಳಿದರು: “ಅನಂತರ ಏಳುವ ಪ್ರಶ್ನೆಯೇ ಇದು; ಈ ಧರ್ಮಗಳೆಲ್ಲ ಸತ್ಯವೇ? ಅಥವಾ ಇವುಗಳಲ್ಲಿ ಕೆಲವು ಸತ್ಯ ಮತ್ತೆ ಕೆಲವು ಅಸತ್ಯವೇ? ಎಲ್ಲ ಧರ್ಮಗಳೂ, ನಿರಪೇಕ್ಷ ಅನಂತ ಅಸ್ತಿತ್ವವನ್ನು ತಲುಪಿದ್ದವು. ಧರ್ಮದ ಗುರಿಯೇ ಏಕತೆ. ನಮ್ಮ ಕಣ್ಣೆದುರಿಗೆ ಕಾಣುವ ವೈವಿಧ್ಯತೆಗಳೆಲ್ಲ ಏಕತೆಯ ಅಭಿವ್ಯಕ್ತಿ. ಧರ್ಮಗಳನ್ನು ವಿಶ್ಲೇಷಣೆ ಮಾಡಿ ನೋಡಿದರೆ, ಮಾನವನು ಅಸತ್ಯದಿಂದ ಸತ್ಯದ ಕಡೆಗೆ ಬರುತ್ತಿಲ್ಲ, ಕೆಳಗಿನ ಸತ್ಯದಿಂದ ಮೇಲಿನ ಸತ್ಯದ ಕಡೆಗೆ ಬರುತ್ತಿರುವನು ಎಂಬುದು ಕಂಡು ಬರುತ್ತದೆ.

“ಒಬ್ಬನು ಒಂದು ಸೈಜಿನ ಅಂಗಿಯನ್ನು ಹಲವರಿಗೆ ತರುವನು. ಕೆಲವರು ಆ ಅಂಗಿ ತಮಗೆ ಸರಿಯಾಗುವುದಿಲ್ಲ ಎಂದು ಹೇಳುವರು. “ನೀವು ಹೋಗಿ, ನಿಮಗೆ ಅಂಗಿ ಸಿಕ್ಕುವುದಿಲ್ಲ”, ಎನ್ನುವನು. ನೀವು ಕ್ರೈಸ್ತ್ರಪಾದ್ರಿಯನ್ನು, ತನ್ನ ಸಿದ್ಧಾಂತ ಮತ್ತು ನಂಬಿಕೆಗಳಿಗೆ ವಿರೋಧವಾಗಿರುವ ಮತಗಳ ವಿಷಯದಲ್ಲಿ ಪ್ರಶ್ನೆ ಮಾಡಿದರೆ ಅವನು ‘ಓ, ಅವರು ಕ್ರೈಸ್ತರಲ್ಲ’ ಎನ್ನುವನು. ಆದರೆ ನಮ್ಮಲ್ಲಿ ಇವುಗಳಿಗಿಂತ ಉತ್ತಮವಾದ ಬೋಧನೆಯಿದೆ. ನಮ್ಮ ಸ್ವಭಾವವೆ, ನಮ್ಮ ಪ್ರೀತಿ ಮತ್ತು ವಿಜ್ಞಾನ ಇದಕ್ಕಿಂತ ಹೆಚ್ಚಿನದನ್ನು ನಮಗೆ ಬೋಧಿಸುವುವು. ಹರಿಯುವ ನದಿಯಲ್ಲಿರುವ ಸುಳಿಯಂತೆ ಅವುಗಳಿಲ್ಲದೇ ಇದ್ದರೆ ನದಿ ಕಲುಷಿತವಾಗುವುದು. ಅಭಿಪ್ರಾಯ ಭೇದವನ್ನು ನಾವು ನಾಶಮಾಡಿದರೆ, ನಾವು ಚಿಂತನಾ ಶಕ್ತಿಯನ್ನೇ ಕಳೆದುಕೊಳ್ಳುತ್ತೇವೆ. ಚಲನೆ ಆವಶ್ಯಕ. ಭಾವನೆಯೆ ಮನಸ್ಸಿನ ಚಲನೆ. ಅದು ನಿಂತರೆ ಮೃತ್ಯು ಸನ್ನಿಹಿತವಾದಂತೆ.

“ನೀವು ಒಂದು ನೀರಿರುವ ಗ್ಲಾಸಿನ ತಳಭಾಗದಲ್ಲಿ ಗಾಳಿಯ ಕಣವನ್ನು ಇಟ್ಟರೆ, ಅದು ತಕ್ಷಣವೇ ಮೇಲಿರುವ ಅನಂತವಾದ ಗಾಳಿಯೊಡನೆ ಒಂದಾಗಲು ಪ್ರಯತ್ನಿಸುವುದು. ಇದರಂತೆಯೇ ಆತ್ಮನೂ ಕೂಡ. ಇದು ಈ ದೇಹದ ಬಂಧನದಿಂದ ಪಾರಾಗಿ ಶುದ್ಧವಾದ ತನ್ನ ಸ್ವಭಾವವನ್ನು ಪಡೆಯಲು ಯತ್ನಿಸುತ್ತಿದೆ. ಅದು ತನ್ನ ಅನಂತತೆಯ ಸ್ಥಿತಿಯನ್ನು ಮತ್ತೆ ಪಡೆಯಲು ಬಯಸುತ್ತದೆ. ಇದು ಎಲ್ಲಾ ಕಡೆಯಲ್ಲಿಯೂ ಹೀಗೆಯೇ. ಕ್ರೈಸ್ತರಲ್ಲಿ ಆಗಲೀ, ಬೌದ್ಧರಲ್ಲಿ ಆಗಲೀ ಮಹಮ್ಮದೀಯರಲ್ಲಾಗಲೀ, ಆಜ್ಞೇಯತಾವಾದಿಗಳಲ್ಲಾಗಲೀ ಪಾದ್ರಿಯಲ್ಲಿ ಆಗಲೀ-ಎಲ್ಲರಲ್ಲೂ ಜೀವವು ಪರಿಪೂರ್ಣವಾಗಲು ಯತ್ನಿಸುತ್ತಿದೆ. ನದಿ ಬೆಟ್ಟದ ಮೇಲಿನಿಂದ ವಕ್ರವಕ್ರವಾಗಿ ಹರಿದು ಸಮುದ್ರವನ್ನು ಸೇರಲು ಸಾವಿರಾರು ಮೈಲಿಗಳು ಪ್ರಯಾಣ ಮಾಡುವುದು. ಮನುಷ್ಯನಾದರೋ ಆ ನದಿಗೆ, ತನ್ನ ಮೂಲಕ್ಕೆ ಹಿಂತಿರುಗಿ ಹೋಗಿ ಬೇರೆ ಕಡೆಯಲ್ಲಿ ಹುಟ್ಟಿ ಹರಿದು ಬಾ ಎಂದು ಆಜ್ಞಾಪಿಸುವನು! ಹಾಗೆ ಆಜ್ಞಾಪಿಸುವವನು ಒಬ್ಬ ಮೂರ್ಖ. ನೀವು ಪುರಾತನವಾದ ಜೆರೂಸಲಮಿನ ಪವಿತ್ರ ಪರ್ವತದಿಂದ ಹುಟ್ಟಿರುವಿರಿ. ನಾನಾದರೊ ಆಗಸವನ್ನು ಚುಂಬಿಸುವ ಶಿಖರಗಳುಳ್ಳ\break ಹಿಮಾಲಯದಿಂದ ಹುಟ್ಟಿ ಬಂದಿರುವೆನು. ನಾನು ನಿಮಗೆ, ನೀವು ತಪ್ಪು, ನನ್ನಂತೆ ಹುಟ್ಟಿ ಬನ್ನಿ ಎಂದು ಹೇಳುವುದಿಲ್ಲ. ಅದು ಮೂರ್ಖತೆಗಿಂತ ಹೆಚ್ಚು ತಪ್ಪಾಗುವುದು. ನಿಮ್ಮ ನಂಬಿಕೆ\break ಗಳನ್ನು ನೀವು ತ್ಯಜಿಸಬೇಡಿ. ಸತ್ಯವನ್ನು ಎಂದಿಗೂ ನೀವು ಕಳೆದುಕೊಳ್ಳುವುದಿಲ್ಲ. ಶಾಸ್ತ್ರಗಳು ಹಾಳಾಗಿ ಹೋಗಬಹುದು, ದೇಶ ನಿರ್ನಾಮವಾಗಿ ಹೋಗಬಹುದು. ಆದರೆ ಸತ್ಯವನ್ನು ಯಾರೋ ರಕ್ಷಿಸಿ ಅನಂತರ ಅದನ್ನು ಪುನಃ ಸಮಾಜಕ್ಕೆ ಕೊಡುವರು. ಇದು ದೇವರು ಯಾವಾಗಲೂ ಮಾನವನಿಗೆ ವ್ಯಕ್ತವಾಗುತ್ತಿರುವನು ಎಂಬುದನ್ನು ತೋರುವುದು.”


\section[ಇಂಡಿಯಾ ದೇಶದ ಆಚಾರ ವ್ಯವಹಾರಗಳು]{ಇಂಡಿಯಾ ದೇಶದ ಆಚಾರ ವ್ಯವಹಾರಗಳು\protect\footnote{* C.W. Vol. III P. 488}}

\begin{center}
(ಅಪಿಲ್​ ಅವಲಾಂಚ್​, ಜನವರಿ ೨೧, ೧೮೯೪)
\end{center}

\vskip -0.35cm

ಹಿಂದೂ ಸಂನ್ಯಾಸಿಗಳಾದ ಸ್ವಾಮಿ ವಿವೇಕಾನಂದರು ನಿನ್ನೆ ಮಧ್ಯಾಹ್ನ (ಮೆಂಫಿಸ್​ನ) ಲಾ ಸೆಲಿಟಿ ಅಕಾಡೆಮಿಯಲ್ಲಿ ಒಂದು ಉಪನ್ಯಾಸವನ್ನು ಕೊಟ್ಟರು. ತುಂಬಾ ಮಳೆ ಬೀಳುತ್ತಿದ್ದುದರಿಂದ ಬಹಳ ಕಡಿಮೆ ಜನ ಉಪನ್ಯಾಸಕ್ಕೆ ಬಂದಿದ್ದರು. ಅವರು ಮಾಡಿದ ಉಪನ್ಯಾಸದ ವಿಷಯ “ಇಂಡಿಯಾ ದೇಶದ ಆಚಾರ ವ್ಯವಹಾರಗಳು” ಎಂಬುದಾಗಿತ್ತು.

ಸ್ವಾಮಿ ವಿವೇಕಾನಂದರು ಕೆಲವು ಧಾರ್ಮಿಕ ಭಾವನೆಯ ಸಿದ್ಧಾಂತಗಳನ್ನು ವಿವರಿಸು\break ತ್ತಿರುವರು. ಈ ನಗರದ ಮತ್ತು ಅಮೆರಿಕಾ ದೇಶದ ಬೇರೆ ಬೇರೆ ನಗರಗಳ ಪ್ರಮುಖ ವಿದ್ವಾಂಸರು ಅವನ್ನು ಸ್ವೀಕರಿಸುತ್ತಿರುವರು.

ಕ್ರೈಸ್ತ ಪಾದ್ರಿಗಳು ಬೋಧಿಸುವ ಸಂಪ್ರದಾಯಬದ್ಧ ನಂಬಿಕೆಗಳಿಗೆ ಸ್ವಾಮೀಜಿ ಅವರ ಬೋಧನೆಯು ಮೃತ್ಯುಪ್ರಾಯವಾಗಿದೆ. ಇಂಡಿಯಾ ದೇಶದಲ್ಲಿ ಅಜ್ಞಾನದಲ್ಲಿ ನರಳುವ\break ಜನರಿಗೆ ಜ್ಞಾನೋದಯವನ್ನು ಮಾಡಿಸಬೇಕೆಂದು ಅಮೆರಿಕಾ ದೇಶದ ಕ್ರೈಸ್ತರು ವಿಶ್ವಪ್ರಯತ್ನ ಮಾಡುತ್ತಿರುವರು. ಆದರೆ ನಮ್ಮ ಪೂರ್ವಿಕರು ಬೋಧಿಸಿದ ಕ್ರೈಸ್ತಧರ್ಮವಾದರೊ\break ವಿವೇಕಾನಂದರ ಪೌರಸ್ತ್ಯ ಧರ್ಮಜ್ಯೋತಿಯ ಮುಂದೆ ಕಳೆಗುಂದುವುದು. ಅಮೆರಿಕಾ ದೇಶದಲ್ಲಿ ಸುಶಿಕ್ಷಿತರ ಮನಸ್ಸಿನಲ್ಲಿ ಹಿಂದೂಧರ್ಮದ ಭಾವನೆಗೆ ಹೆಚ್ಚು ಆಶ್ರಯ ದೊರಕುವುದು.

ಯಾವುದಾದರೊಂದು ಖಯಾಲಿಯ ದಿನ ಇಂದು. ವಿವೇಕಾನಂದರು ಬಹಳ ದಿನಗಳ ಬಯಕೆಯೊಂದನ್ನು ಪೂರ್ಣ ಮಾಡುತ್ತಿರುವಂತೆ ತೋರುವುದು. ಬಹುಶಃ ಅವರ ದೇಶದ ಅಪೂರ್ವ ಮೇಧಾವಿಗಳಲ್ಲಿ ಅವರೊಬ್ಬರು. ಅವರಲ್ಲಿ ಎಲ್ಲರನ್ನೂ ಆಕರ್ಷಿಸುವ ಒಂದು ಅಪೂರ್ವ ವರ್ಚಸ್ಸು ಇದೆ. ಸಭಿಕರಾದರೊ ಅವರ ವಾಗ್ಮಿತೆಗೆ ಮುಗ್ಧರಾಗಿ ಹೋಗುವರು. ಅವರು ಬಹಳ ಉದಾರಿಗಳಾದರೂ ಸಂಪ್ರದಾಯಬದ್ಧ ಕ್ರೈಸ್ತಧರ್ಮದಲ್ಲಿ ಮೆಚ್ಚುವಂತಹ ಯಾವುದೂ ಅವರಿಗೆ ಕಾಣುವುದಿಲ್ಲ. ಈ ಊರಿಗೆ ಬಂದ ಎಲ್ಲಾ ಉಪನ್ಯಾಸಕರು ಮತ್ತು ಪಾದ್ರಿಗಳಿಗಿಂತ ಹೆಚ್ಚಾಗಿ ವಿವೇಕಾನಂದರಿಗೆ ಜನ ಮೆಂಫಿಸ್​ನಲ್ಲಿ ಆಕರ್ಷಿತರಾಗಿರುವರು.

ಈ ದೇಶದಲ್ಲಿ ಹಿಂದೂ ಸಂನ್ಯಾಸಿಯನ್ನು ಸ್ವಾಗತಿಸುವಂತೆ ಭರತಖಂಡದಲ್ಲಿ ಕ್ರೈಸ್ತ ಪಾದ್ರಿಯನ್ನು ಸ್ವಾಗತಿಸಿದರೆ, ಕ್ರೈಸ್ತೇತರ ದೇಶಗಳಲ್ಲಿ ಕ್ರಿಸ್ತನ ಬೋಧನೆ ಚೆನ್ನಾಗಿ ಪ್ರಚಾರ\break ವಾಗಲು ಕಾರಣವಾಗುವುದು. ಅವರು ನಿನ್ನೆ ಮಾಡಿದ ಉಪನ್ಯಾಸ ಚಾರಿತ್ರಿಕ ದೃಷ್ಟಿಯಿಂದ ಸ್ವಾರಸ್ಯವಾಗಿತ್ತು. ಪುರಾತನ ಕಾಲದಿಂದ ಇಂದಿನವರೆಗಿನ ಅವರ ದೇಶದ ಇತಿಹಾಸ ಮತ್ತು ಸಂಪ್ರದಾಯಗಳ ಪರಿಚಯ ಅವರಿಗೆ ಚೆನ್ನಾಗಿ ಇದೆ. ಆ ದೇಶದ ಮುಖ್ಯವಾದ ಹಲವು ಸ್ಥಳ ಮತ್ತು ವಸ್ತುಗಳನ್ನು ಬಹಳ ಸರಳವಾಗಿಯೂ ಮನೋಹರವಾಗಿಯೂ ಚಿತ್ರಿಸಬಲ್ಲರು.

ಅವರು ಉಪನ್ಯಾಸಮಾಡುತ್ತಿದ್ದಾಗ ಸಭೆಯಲ್ಲಿದ್ದ ಹಲವು ಸ್ತ್ರೀಯರು ಅನೇಕ ಪ್ರಶ್ನೆಗಳನ್ನು ಕೇಳುತ್ತಿದ್ದರು. ಅವರ ಪ್ರಶ್ನೆಗಳಿಗೆಲ್ಲ ಸ್ವಲ್ಪವೂ ಸಂದೇಹವಿಲ್ಲದೆ ಉತ್ತರ ಕೊಟ್ಟರು. ಎಲ್ಲೋ ಒಬ್ಬ ಹೆಂಗಸು ಅವರನ್ನು ಧಾರ್ಮಿಕ ಚರ್ಚೆಗೆ ಎಳೆಯುವ ಉದ್ದೇಶದಿಂದ ಪ್ರಶ್ನೆಯನ್ನು ಕೇಳಿದಾಗ ಮಾತ್ರ ಅದಕ್ಕೆ ಉತ್ತರ ಕೊಡಲಿಲ್ಲ. ತಾವು ಮಾತನಾಡುತ್ತಿರುವ ಮುಖ್ಯವಾದ ವಿಷಯವನ್ನು ಬಿಟ್ಟು ಬೇರೆ ವಿಷಯವನ್ನು ತೆಗೆದುಕೊಳ್ಳುವುದಕ್ಕೆ ತಮಗೆ ಇಷ್ಟವಿಲ್ಲವೆಂದೂ, ಬೇರೆ ಸಮಯದಲ್ಲಿ ಜೀವಿಯ ಪುನರ್ಜನ್ಮ ಮುಂತಾದ ವಿಷಯದಲ್ಲಿ ಬೇಕಾದರೆ ತಮ್ಮ ಅಭಿಪ್ರಾಯ ಕೊಡುವೆನೆಂದೂ ಹೇಳಿದರು.

ಅವರು ತಮ್ಮ ಭಾಷಣದಲ್ಲಿ, ತಮ್ಮ ಅಜ್ಜ ಮೂರನೇ ವಯಸ್ಸಿನಲ್ಲಿ ಮದುವೆಯಾದ\break ರೆಂದೂ, ತಮ್ಮ ತಂದೆ ಹದಿನೆಂಟನೇ ವಯಸ್ಸಿನಲ್ಲಿ ಮದುವೆಯಾದರೆಂದೂ, ತಾವು ಮದುವೆಯಾಗಲೇ ಇಲ್ಲವೆಂದೂ ತಿಳಿಸಿದರು. ಸಂನ್ಯಾಸಿ ಮದುವೆಯಾಗ ಕೂಡದೆಂಬ ನಿಷಿದ್ಧವೇನೂ ಇಲ್ಲವೆಂದೂ\footnote{* ಸ್ವಾಮೀಜಿ ಅವರು ಸಂನ್ಯಾಸಿಗಳು ಮದುವೆಯ ವಿಷಯದಲ್ಲಿ ಬಹುಶಃ ಮೇಲಿನದನ್ನು ಹೇಳಿರಲಾರರು. ಇದು ಬರೀ ವರದಿಗಾರನ ಆರೋಪವಿರಬೇಕು. ಏಕೆಂದರೆ ಸಂನ್ಯಾಸಿ ಏನಾದರೂ ಮದುವೆಯಾದರೆ ಅವನನ್ನು ಭ್ರಷ್ಟನೆಂದು ಹೇಳುತ್ತಾರೆ. ಅವನು ಇನ್ನು ಮೇಲೆ ಆ ಆಶ್ರಮದಲ್ಲಿರಲಾರ.} ಹಾಗೆ ಏನಾದರೂ ಅವರು ಮದುವೆಯಾದರೆ, ಆಕೆಯೂ ಕೂಡ ಅದೇ ಅಧಿಕಾರ ಮತ್ತು ಹಕ್ಕುಗಳುಳ್ಳ ಸಂನ್ಯಾಸಿನಿಯಾಗುವಳೆಂದೂ, ತನ್ನ ಪತಿ\break ಯಂತೆಯೇ ಅವಳು ಸಮಾಜದಲ್ಲಿ ಒಂದು ಸ್ಥಾನವನ್ನು ಪಡೆಯುವಳೆಂದೂ ಹೇಳಿದರು.

ಒಂದು ಪ್ರಶ್ನೆಗೆ ಉತ್ತರವಾಗಿ, ಭಾರತದಲ್ಲಿ ಯಾವ ಕಾರಣದಿಂದಲೂ ವಿವಾಹ ವಿಚ್ಛೇದನಕ್ಕೆ ಅವಕಾಶವಿಲ್ಲವೆಂದರು. ಆದರೆ ಮದುವೆಯಾಗಿ ಹದಿನಾಲ್ಕು ವರುಷಗಳು ಆದ ಮೇಲೆ ಮಕ್ಕಳಾಗದೆ ಇದ್ದರೆ ಹೆಂಡತಿಯ ಒಪ್ಪಿಗೆಯನ್ನು ಪಡೆದು ಗಂಡ ಬೇಕಾದರೆ ಮತ್ತೊಂದು ಮದುವೆಯನ್ನು ಮಾಡಿಕೊಳ್ಳಬಹುದು. ಆದರೆ ಅವಳು ಆಕ್ಷೇಪಿಸಿದರೆ ಅವನು ಹಾಗೆ ಮದುವೆಯಾಗುವಂತಿಲ್ಲ. ವಿವೇಕಾನಂದರು ಹಳೆಯ ದೇವಸ್ಥಾನಗಳು, ಮಸೀದಿಗಳು, ಇವುಗಳನ್ನು ಕುರಿತು ಕೊಟ್ಟ ವಿವರಣೆಗಳು ಅತಿ ಅಮೋಘವಾಗಿದ್ದವು. ಈಗಿನ ಕಾಲದ ಅತ್ಯಂತ ನುರಿತ ಕೈಕೆಲಸಗಾರರಿಗಿಂತ ಆಗಿನ ಕಾಲದವರಿಗೆ ಹೆಚ್ಚು ವೈಜ್ಞಾನಿಕ ಜ್ಞಾನವಿತ್ತು ಎಂಬುದನ್ನು ಇದು ತೋರುವುದು.

ಸ್ವಾಮಿ ವಿವೇಕಾನಂದರು \enginline{Y.M.H.A} ಹಾಲಿನಲ್ಲಿ ಕೊನೆಯ ಬಾರಿ ಇಂದಿನ ರಾತ್ರಿ ಮಾತನಾಡುವರು. ಅವರು ಚಿಕಾಗೊವಿನಲ್ಲಿರುವ ಸ್ಲೇಟನ್​ ಲೈಸಿಯಂ ಬ್ಯೂರೊವಿನ ಕಂಟ್ರಾಕ್ಟಿನಲ್ಲಿ ಇರುವರು. ಅವರು ನಾಳೆ ಚಿಕಾಗೊ ನಗರಕ್ಕೆ ಹೋಗುವರು. ಅವರು ೨೫ನೇ ತಾರೀಖು ಅಲ್ಲಿ ಮಾತನಾಡಬೇಕಾಗಿದೆ.

\delimiter

\begin{center}
(ಡೆಟ್ರಾಯಿಟ್​ ಟ್ರಿಬ್ಯೂನ್​, ೧೪ನೇ ಫೆಬ್ರುವರಿ ೧೮೯೪)
\end{center}

\vskip -0.27cm

ಕಳೆದ ಸಂಜೆ ಪ್ರಖ್ಯಾತ ಹಿಂದೂ ಸಂನ್ಯಾಸಿಗಳಾದ ಸ್ವಾಮಿ ವಿವೇಕಾನಂದರ ಉಪನ್ಯಾಸವನ್ನು ಕೇಳುವ ಮತ್ತು ಅವರನ್ನು ನೋಡುವ ಸುಯೋಗ ಹಲವರಿಗೆ ಒದಗಿತ್ತು. ಅವರು ಬ್ರಹ್ಮಸಮಾಜಕ್ಕೆ ಸೇರಿದವರು. ಅವರು ಯೂನಿಟಿ ಕ್ಲಬ್ಬಿನ ಆಶ್ರಯದಲ್ಲಿ ಯೂನಿಟೇರಿಯನ್​ ಚರ್ಚ್​ನಲ್ಲಿ ಮಾತನಾಡಿದರು. ಅವರು ತಮ್ಮ ದೇಶೀ ಪೋಷಾಕಿನಲ್ಲಿದ್ದರು. ಸುಂದರವಾದ ಮುಖವುಳ್ಳ ಮತ್ತು ದೃಢಕಾಯರಾದ ಅವರು ನೋಡಲು ಚೆನ್ನಾಗಿದ್ದರು. ಅವರ ವಾಗ್ಧಾರೆ ಸಭಿಕರನ್ನು ಮುಗ್ಧಗೊಳಿಸಿತು. ಮಧ್ಯೆ ಅನೇಕ ವೇಳೆ ಕರತಾಡನಗಳು ಆದವು. ಅವರು ‘ಭರತಖಂಡದ ಆಚಾರ ವ್ಯವಹಾರಗಳು’ ಎನ್ನುವದರ ಮೇಲೆ ಮಾತನಾಡಿದರು. ಅವರು ವಿಷಯವನ್ನು ಅತಿ ಸುಂದರವಾದ ಇಂಗ್ಲಿಷ್​ ಭಾಷೆಯಲ್ಲಿ ವರ್ಣಿಸಿದರು. ಅವರು ತಮ್ಮ ದೇಶವನ್ನು ಇಂಡಿಯಾ ಎಂದು ಕರೆಯುವುದಿಲ್ಲವೆಂದೂ ತಾವು ಹಿಂದೂಗಳು ಎಂದು ಹೇಳಿಕೊಳ್ಳುವುದಿಲ್ಲವೆಂದೂ ಹೇಳಿದರು. ದೇಶದ ಹೆಸರು ಹಿಂದೂಸ್ತಾನ, ತಾವು ಬ್ರಾಹ್ಮಣರು. ಹಿಂದಿನ ಕಾಲದಲ್ಲಿ ಅವರು ಸಂಸ್ಕೃತವನ್ನು ಮಾತನಾಡುತ್ತಿದ್ದರು. ಆ ಭಾಷೆಯಲ್ಲಿ ತಮ್ಮ ಭಾವನೆಗಳನ್ನೆಲ್ಲ ಚೆನ್ನಾಗಿ ವಿವರಿಸುತ್ತಿದ್ದರು. ಆದರೆ ಈಗ ಅದೆಲ್ಲ ಮಾಯವಾಗಿದೆ. ಸಂಸ್ಕೃತದಲ್ಲಿ ಜೂಪಿಟರ್​ ಎಂದರೆ ಸ್ವರ್ಗದಲ್ಲಿರುವ ತಂದೆ ಎಂದು ಅರ್ಥ. ಭರತ ಖಂಡದ ಉತ್ತರದ ಭಾಷೆಗಳೆಲ್ಲ ಹೆಚ್ಚು ಕಡಿಮೆ ಒಂದೇ ರೀತಿ ಇದೆ. ಆದರೆ ಅವರು ದಕ್ಷಿಣ ದೇಶಕ್ಕೆ ಹೋದರೆ ಅಲ್ಲಿಯ ಜನರೊಡನೆ ಮಾತನಾಡಲಾರರು. ತಂದೆ, ತಾಯಿ, ಸಹೋದರ, ಸಹೋದರಿ ಮುಂತಾದ ಅರ್ಥವನ್ನು ನೀಡುವ ಇಂಗ್ಲಿಷ್​ ಶಬ್ದಗಳಿಗೆ ಸಂಸ್ಕೃತದಲ್ಲಿ ಹೆಚ್ಚು ಕಡಿಮೆ ಸಮಾನವಾದ ಶಬ್ದಗಳಿವೆ. ಇವೇ ಮುಂತಾದ ಕೆಲವು ಕಾರಣಗಳು ನಾವೆಲ್ಲ ಒಂದೇ ಮೂಲ ಜನಾಂಗವಾದ ಆರ್ಯಕುಲಕ್ಕೆ ಸೇರಿದವರೆಂಬುದನ್ನು ತೋರಿಸುತ್ತದೆ ಎಂದರು. ಈ ಜನಾಂಗಕ್ಕೆ ಸೇರಿದ ವಿಭಿನ್ನ ಶಾಖೆಗಳೆಲ್ಲ ತಮ್ಮ ಪ್ರತ್ಯೇಕತೆಯನ್ನು ಕಳೆದುಕೊಂಡಿವೆ.

ನಾಲ್ಕು ವರ್ಣಗಳು ಇದ್ದವು. ಅವರೇ ಬ್ರಾಹ್ಮಣರು, ಕ್ಷತ್ರಿಯರು, ವೈಶ್ಯರು ಮತ್ತು ಶೂದ್ರರು. ಮೊದಲನೆ ಮೂರು ವರ್ಣಗಳಲ್ಲಿ ಮಕ್ಕಳಿಗೆ ಹತ್ತು, ಹನ್ನೊಂದು, ಹದಿಮೂರು ವರ್ಷಗಳಾದಾಗ ಅವರನ್ನು ಗುರುಕುಲದಲ್ಲಿ ಬಿಡುತ್ತಿದ್ದರು. ಅವರು ಕ್ರಮವಾಗಿ ಮೂವತ್ತು, ಇಪ್ಪತ್ತೈದು, ಇಪ್ಪತ್ತು ವರುಷಗಳವರೆಗೆ ಅಲ್ಲಿರುತ್ತಿದ್ದರು. ಹಿಂದಿನ ಕಾಲದಲ್ಲಿ ಹುಡುಗ ಹುಡುಗಿಯರು ಇಬ್ಬರಿಗೂ ವಿದ್ಯೆಯನ್ನು ಹೇಳಿಕೊಡುತ್ತಿದ್ದರು. ಆದರೆ ಈಗ ಹುಡುಗರಿಗೆ ಮಾತ್ರ ಕಲಿಸುವರು. ಬಹು ಕಾಲದಿಂದ ಬಂದ ದೋಷವನ್ನು ಹೋಗಲಾಡಿಸಲು ಈಗ ಪ್ರಯತ್ನ ನಡೆಯುತ್ತಿದೆ. ದೇಶದಲ್ಲಿ ಪ್ರಚಲಿತವಾದ ತಾತ್ತ್ವಿಕ ಮತ್ತು ಕಾನೂನಿಗೆ ಸಂಬಂಧಿಸಿದ ವಿಷಯಗಳಿಗೆ ಪುರಾತನ ಕಾಲದ ಸ್ತ್ರೀಯರ ಕೊಡುಗೆಯೂ ಇದೆ. ಇದು ಅನಾಗರಿಕರು ಭರತಖಂಡವನ್ನು ಆಳಲು ಪ್ರಾರಂಭಿಸುವುದಕ್ಕೆ ಮುಂಚೆ. ಹಿಂದೂವಿನ ದೃಷ್ಟಿಯಲ್ಲಿ ಈಗ ಸ್ತ್ರೀಗೂ ತಕ್ಕ ಹಕ್ಕುಬಾಧ್ಯತೆಗಳು ಇವೆ. ಅವಳು ತನಗೆ ಸಂಬಂಧಪಟ್ಟ ಆಸ್ತಿಗೆ ತಾನೆ ಹಕ್ಕುದಾರಳು, ಕಾನೂನು ಕೂಡ ಅವಳ ಪರವಾಗಿದೆ.

ವಿದ್ಯಾರ್ಥಿ ಗುರುಕುಲದಿಂದ ಹಿಂತಿರುಗಿದ ಮೇಲೆ ಮದುವೆಯಾಗಿ ಗೃಹಸ್ಥನಾಗಿ ಬಾಳಬಹುದು. ಗಂಡ ಹೆಂಡತಿಯರಿಬ್ಬರೂ ಕೆಲಸ ಮಾಡಬೇಕು. ಇಬ್ಬರಿಗೂ ಅವರವರ ಹಕ್ಕುಗಳಿವೆ. ಕ್ಷತ್ರಿಯರಲ್ಲಿ ಅನೇಕ ವೇಳೆ ರಾಜಪುತ್ರಿಯರು ತಮ್ಮ ಪತಿಯನ್ನು ತಾವೇ ಆರಿಸಿಕೊಳ್ಳುತ್ತಿದ್ದರು. ಆದರೆ ಬೇರೆ ವರ್ಣಗಳಲ್ಲಿ ತಂದೆತಾಯಿಗಳೇ ಅವರ ಮದುವೆಗೆ ಅಣಿ ಮಾಡುವರು. ಬಾಲ್ಯ ವಿವಾಹವನ್ನು ತಪ್ಪಿಸಲು ಸಾಧ್ಯವಾದಷ್ಟು ಈಗ ಪ್ರಯತ್ನಿಸುತ್ತಿರುವರು. ಮದುವೆಯ ಕೆಲವು ಆಚಾರಗಳು ಬಹಳ ಚೆನ್ನಾಗಿವೆ. ಪ್ರತಿಯೊಬ್ಬರೂ ಮತ್ತೊಬ್ಬರ ಹೃದಯವನ್ನು ಮುಟ್ಟಿ ದೇವರು ಮತ್ತು ಅಲ್ಲಿ ನೆರೆದ ಸಭೆಯ ಎದುರಿಗೆ ತಾವು ಎಂದೆಂದಿಗೂ ಇನ್ನೊಬ್ಬರನ್ನು ತ್ಯಜಿಸುವುದಿಲ್ಲ ಎಂದು ಮಾತು ಕೊಡುವರು. ಒಬ್ಬ ತಾನು ಮದುವೆ\break ಯಾಗದೆ ಇದ್ದರೆ ಪುರೋಹಿತನಾಗಲಾರ. ಸಾರ್ವಜನಿಕ ಪೂಜೆ ಮಾಡುವಾಗ ಅವನು ಯಾವಾಗಲೂ ತನ್ನ ಹೆಂಡತಿಯೊಂದಿಗೆ ಇರುವನು. ಹಿಂದೂ ಐದು ಯಜ್ಞಗಳನ್ನು ಮಾಡುವನು; ದೇವಯಜ್ಞ, ಪಿತೃಯಜ್ಞ, ಋಷಿಯಜ್ಞ, ನೃಯಜ್ಞ, ಮತ್ತು ಭೂತಯಜ್ಞ. ಹಿಂದೂವಿನ ಮನೆಯಲ್ಲಿ ಎಲ್ಲಿಯವರೆಗೆ ಏನಾದರೂ ಇರುವುದೋ ಅಲ್ಲಿಯವರೆಗೆ ಅವನು ಅತಿಥಿಗೆ ಅದನ್ನು ಕೊಡಬೇಕು. ಅವನು ತೃಪ್ತನಾದ ಮೇಲೆ ಮಕ್ಕಳಿಗೆ, ಅನಂತರ ತಂದೆತಾಯಿ\break ಗಳು ಊಟ ಮಾಡುವರು. ಈ ಪ್ರಪಂಚದಲ್ಲಿರುವ ಕಡು ಬಡವರು ಭಾರತೀಯರು. ಆದರೂ ಕ್ಷಾಮಕಾಲದಲ್ಲಿ ಹೊರತು ಯಾರೂ ಉಪವಾಸದಿಂದ ಸಾಯುವುದಿಲ್ಲ. ಜನರನ್ನು ನಾಗರಿಕರನ್ನಾಗಿ ಮಾಡುವುದು ಒಂದು ದೊಡ್ಡ ಕೆಲಸ. ಪರಸ್ಪರ ಹೋಲಿಕೆಯನ್ನು ಮಾಡುತ್ತಾ, ಇಂಗ್ಲೆಂಡಿನಲ್ಲಿ ನಾನ್ನೂರು ಜನಕ್ಕೆ ಒಬ್ಬ ಕುಡುಕ ಇರುವನು. ಆದರೆ ಇಂಡಿಯಾ ದೇಶದಲ್ಲಿಯಾದರೋ ಹತ್ತು ಲಕ್ಷಕ್ಕೆ ಒಬ್ಬ ಸಿಕ್ಕಬಹುದು ಎಂದು ವಿವೇಕಾನಂದರು ಹೇಳಿದರು. ಸತ್ತವರನ್ನು ದಹನ ಮಾಡುವ ಒಂದು ವಿವರಣೆಯನ್ನು ಕೊಟ್ಟರು. ಯಾರಾದರೂ ಒಬ್ಬ ದೊಡ್ಡ ಜಮೀನ್ದಾರನಲ್ಲದೆ ಇದ್ದರೆ ಶವ ಸಂಸ್ಕಾರವನ್ನು ಬಹಿರಂಗವಾಗಿ ಮಾಡುವುದಿಲ್ಲ. ಹದಿನೈದು ದಿನಗಳಾದ ಮೇಲೆ ಉಪವಾಸ ಮಾಡಿ ಬಡವರಿಗೆ ದಾನಧರ್ಮಗಳನ್ನು ಮಾಡುವರು. ಇಲ್ಲವೆ ಗತಿಸಿದವರ ಹೆಸರಿನಲ್ಲಿ ಯಾವುದಾದರೊಂದು ಸಂಸ್ಥೆಯನ್ನು ತೆರೆಯುವರು. ನೈತಿಕ ವಿಷಯದಲ್ಲಿ ಭಾರತೀಯರು ಬೇರೆಲ್ಲ ರಾಷ್ಟ್ರಗಳನ್ನೂ ಮೀರಿಸುವರು.

\delimiter


\section[ಹಿಂದೂದರ್ಶನ]{ಹಿಂದೂದರ್ಶನ\protect\footnote{* C.W. Vol. III P. 492}}

\begin{center}
(ಡೆಟ್ರಾಯಿಟ್​ ಫ್ರೀಪ್ರೆಸ್​, ಫೆಬ್ರವರಿ ೧೬, ೧೮೯೪)
\end{center}

\vskip -0.35cm

ಹಿಂದೂ ಸಂನ್ಯಾಸಿಗಳಾದ ಸ್ವಾಮಿ ವಿವೇಕಾನಂದರು ತಮ್ಮ ಎರಡನೆ ಉಪನ್ಯಾಸವನ್ನು ನಿನ್ನೆ ಸಾಯಂಕಾಲ ಯೂನಿಟೇರಿಯನ್​ ಚರ್ಚ್​ನಲ್ಲಿ ನೀಡಿದರು. ಗುಣ ಗ್ರಾಹಿಗಳಾದ ಶ್ರೋತೃಗಳು ಬಹುಸಂಖ್ಯೆಯಲ್ಲಿ ನೆರೆದಿದ್ದರು. ಹಿಂದೂದರ್ಶನದ ವಿಷಯವಾಗಿ ಸಾಕಷ್ಟು ಮಾಹಿತಿಯನ್ನು ನಿರೀಕ್ಷಿಸುತ್ತಿದ್ದ ಸಭಿಕರಿಗೆ ಎಲ್ಲೋ ಸ್ವಲ್ಪ ಪ್ರಮಾಣದಲ್ಲಿ ಮಾತ್ರ ತೃಪ್ತಿ\break ಯಾಯಿತು. ಬುದ್ಧನ ತತ್ತ್ವವನ್ನು ಪ್ರಾಸಂಗಿಕವಾಗಿ ಪ್ರಸ್ತಾಪಿಸಿದರು. ಬೌದ್ಧಧರ್ಮವು ಪ್ರಪಂಚದ ಮೊದಲನೆ ಮಿಷನರಿ ಧರ್ಮ ಎಂದು ಉಪನ್ಯಾಸಕರು ಹೇಳಿದಾಗ ಕರತಾಡನ\break ಗಳು ಆದವು. ಅದು ಒಂದು ತೊಟ್ಟು ರಕ್ತವನ್ನು ಕೂಡ ಹರಿಸದೆ ಈ ಪ್ರಪಂಚದಲ್ಲೆಲ್ಲ ಹೆಚ್ಚು ಮಂದಿ ಅನುಯಾಯಿಗಳನ್ನು ಪಡೆದಿರುವುದು. ಆದರೆ ಅವರು ಸಭಿಕರಿಗೆ ಬುದ್ಧನ ಧರ್ಮವನ್ನಾಗಲೀ, ತತ್ತ್ವವನ್ನಾಗಲಿ ಏನನ್ನೂ ಹೇಳಲಿಲ್ಲ. ಅವರು ಕ್ರೈಸ್ತಧರ್ಮವನ್ನು ಕುರಿತು ಅನೇಕ ಕಡೆ ಟೀಕಿಸಿದರು. ಈ ಧರ್ಮವನ್ನು ಬೇರೆ ಬೇರೆ ದೇಶಗಳಲ್ಲಿ ಹರಡಿದುದ\break ರಿಂದ ಉಂಟಾಗಿರುವ ತೊಂದರೆಗಳನ್ನು ಕುರಿತು ಅವರು ಪ್ರಸ್ತಾಪಿಸಿದರು. ಆದರೆ ತಮ್ಮ ದೇಶದ ಜನರ ಸಾಮಾಜಿಕ ಸ್ಥಿತಿ ಮತ್ತು ಯಾವ ದೇಶದವರನ್ನು ಉದ್ದೇಶಿಸಿ ಅವರು ಮಾತನಾಡುತ್ತಿರುವರೋ ಆ ದೇಶದ ಸಾಮಾಜಿಕ ಸ್ಥಿತಿ-ಇವುಗಳನ್ನು ಹೋಲಿಸುವುದನ್ನು ಜಾಣತನದಿಂದ ಬಿಟ್ಟರು. ಸಾಧಾರಣವಾಗಿ ಹಿಂದೂ ದಾರ್ಶನಿಕರು ಒಬ್ಬನಿಗೆ ಕೆಳಗಿನ ಸತ್ಯದಿಂದ ಮೇಲಿನ ಸತ್ಯಕ್ಕೆ ಹೋಗುವಂತೆ ಬೋಧಿಸುವರು ಎಂದರು. ಆದರೆ ಹೊಸ ಕ್ರೈಸ್ತ ಸಿದ್ಧಾಂತವನ್ನು ಸ್ವೀಕರಿಸುವವನು ತನ್ನ ಹಿಂದಿನ ನಂಬಿಕೆಗಳನ್ನೆಲ್ಲ ಸಂಪೂರ್ಣ ತ್ಯಜಿಸಿ, ಹೊಸ ನಂಬಿಕೆಗಳನ್ನೆಲ್ಲ ಸ್ವೀಕರಿಸಬೇಕಾಗುತ್ತದೆ. ನಮ್ಮಲ್ಲಿ ಎಲ್ಲರಿಗೂ ಒಂದೇ ಧರ್ಮವಿರುವಂತೆ ಸಾಧಿಸುವುದು ಒಂದು ವ್ಯರ್ಥ ಪ್ರಯತ್ನ ಎಂದರು ಅವರು. “ಮನಸ್ಸಿನಲ್ಲಿ ಘರ್ಷಣೆಯಾಗು\break ವಂತಹ ವಿಷಯಗಳು ಬರದೆ ಇದ್ದರೆ, ಯಾವ ಭಾವನೆಯೂ ಬರಲಾರದು. ಬದಲಾವಣೆಯ ಪ್ರತ್ಯುದ್ರೇಕ, ಹೊಸ ಬೆಳಕು, ಹಳೆಯದರ ಜಾಗದಲ್ಲಿ ಹೊಸದನ್ನು ಸ್ಥಾಪಿಸುವುದು-ಇವೇ ಭಾವೋದ್ವೇಗವನ್ನು ಕೆರಳಿಸುವುದು.”

\begin{center}
(ಮೊದಲನೆ ಉಪನ್ಯಾಸವು ಕೆಲವರ ವಿರೋಧಕ್ಕೆ ಕಾರಣವಾದದ್ದರಿಂದ ಫ್ರೀಪ್ರೆಸ್​ ವಾರ್ತಾಕಾರನು ಬಹಳ ಜೋಪಾನವಾಗಿ ವರದಿಮಾಡಿದ್ದನು. ಆದರೆ ಅದೃಷ್ಟವಶಾತ್​ ಡೆಟ್ರಾಯಿಟ್​ ಟ್ರಿಬ್ಯೂನ್​ ಒಂದೇ ಸಮನೆ ಸ್ವಾಮೀಜಿ ಅವರ ಭಾವನೆಯನ್ನು ಸರಿ ಎಂದು ಸಾಧಿಸಿತು. ಆದಕಾರಣ ಫೆಬ್ರವರಿ ೧೬ನೇ ತಾರೀಖಿನ ಪತ್ರಿಕೆಯಲ್ಲಿ ‘ಹಿಂದೂ ದರ್ಶನದ ಮೇಲಿನ ಉಪನ್ಯಾಸದ ಕೆಲವು ಭಾವನೆಗಳು ಬರುವುವು. ಆದರೆ ಟ್ರಿಬ್ಯೂನ್​ ವಾರ್ತಾಕಾರನು ಬಹಳ ಸಂಕ್ಷೇಪವಾಗಿ ಟಿಪ್ಪಣಿಗಳನ್ನು ತೆಗೆದು ಕೊಂಡಿದ್ದನು.)
\end{center}

\begin{center}
(ಡೆಟ್ರಾಯಿಟ್​ ಟ್ರಿಬ್ಯೂನ್​, ಫೆಬ್ರವರಿ ೧೬, ೧೮೯೪)
\end{center}

\vskip -0.27cm

ಬ್ರಾಹ್ಮಣ ಸಂನ್ಯಾಸಿಯಾದ ಸ್ವಾಮಿ ವಿವೇಕಾನಂದರು ಪುನಃ ನೆನ್ನೆ ಸಾಯಂಕಾಲ ಯೂನಿಟೇರಿಯನ್​ ಚರ್ಚ್​ನಲ್ಲಿ ಉಪನ್ಯಾಸ ಮಾಡಿದರು. ಅವರ ಉಪನ್ಯಾಸದ ವಿಷಯ ‘ಹಿಂದೂದರ್ಶನ’ ಎಂಬುದು. ಉಪನ್ಯಾಸಕರು ಕೆಲವು ಕಾಲ ಸಾಮಾನ್ಯ ತತ್ತ್ವ ಮತ್ತು ಭೌತಾತೀತ ವಿಷಯವನ್ನು ಕುರಿತು ಹೇಳಿದರು. ಆದರೆ ಕೇವಲ ಧರ್ಮಕ್ಕೆ ಸಂಬಂಧಪಟ್ಟ ವಿಷಯಗಳನ್ನು ಮಾತ್ರ ತಾವು ಹೇಳುತ್ತೇವೆ ಎಂದರು. ಜೀವನ ಅಸ್ತಿತ್ವವನ್ನು ನಂಬುವ ಒಂದು ಪಂಗಡದವರು ಇರುವರು. ಆದರೆ ದೇವರ ವಿಷಯದಲ್ಲಿ ಅವರು ಆಜ್ಞೇಯತಾವಾದಿಗಳು. ಬೌದ್ಧಧರ್ಮ ಶ್ರೇಷ್ಠ ನೈತಿಕ ಧರ್ಮ. ಆದರೆ ಬೌದ್ಧ ಸಗುಣ ದೇವರಿಲ್ಲದೆ ಬಹಳ ಕಾಲ ಇರಲು ಸಾಧ್ಯವಾಗಲಿಲ್ಲ. ಮತ್ತೊಂದು ಪಂಗಡದವರಾದರೆ ಜೈನರು ಜೀವನ ಅಸ್ತಿತ್ವವನ್ನು ನಂಬುತ್ತಾರೆ, ಆದರೆ ಪ್ರಪಂಚವನ್ನು ಯಾರೊ ದೇವರು ಆಳುತ್ತಿರುವನು ಎಂಬುದನ್ನು ನಂಬುವುದಿಲ್ಲ. ಇಂಡಿಯಾದೇಶದಲ್ಲಿ ಈ ಪಂಗಡಕ್ಕೆ ಸೇರಿದ ಲಕ್ಷಾಂತರ ಜನರು ಇದ್ದರು. ಈ ಪಂಗಡಕ್ಕೆ ಸೇರಿದ ಭಿಕ್ಷುಗಳು ತಮ್ಮ ಬಾಯಿಯ ಮೇಲೆ ಒಂದು ಬಟ್ಟೆಯನ್ನು ಕಟ್ಟಿಕೊಳ್ಳುವರು. ಏಕೆಂದರೆ ತಮ್ಮ ಬಿಸಿಗಾಳಿ ಯಾರಾದರೂ ಮನುಷ್ಯರಿಗೋ ಅಥವಾ ಮೃಗಕ್ಕೋ ತಾಕಿದರೆ ಸಾವು ಬರುವುದು ಎಂದು ಅವರು ಭಾವಿಸುವರು.

ಸಂಪ್ರದಾಯಸ್ಥರೆಲ್ಲ ಶ್ರುತಿ ಪ್ರಮಾಣವನ್ನು ನಂಬುವರು. ಕೆಲವರು ಶಾಸ್ತ್ರದಲ್ಲಿರುವ ಪ್ರತಿಯೊಂದು ಅಕ್ಷರವೂ ದೇವರಿಂದ ಬರುವುದು ಎಂದು ನಂಬುವರು. ಅನೇಕ ಧರ್ಮಗಳಲ್ಲಿ ಪದಗಳ ಅರ್ಥವನ್ನು ಹಿಗ್ಗಿಸುತ್ತಾ ಹೋಗುತ್ತಾರೆ. ಆದರೆ ಹಿಂದೂಗಳಲ್ಲಿ ಅವರಿಗೆ ಸಂಸ್ಕೃತ ಭಾಷೆಯಿದೆ. ಅಲ್ಲಿ ಆಯಾ ಪದಗಳಿಗೆ ನಿರ್ದಿಷ್ಟವಾದ ಅರ್ಥ ಮತ್ತು ಯುಕ್ತಿಗಳು ಇವೆ.

ಪ್ರಖ್ಯಾತ ಪೌರಸ್ತ್ಯ ಉಪನ್ಯಾಸಕರು ಈಗ ನಮಗೆ ಇರುವ ಪಂಚೇಂದ್ರಿಯಗಳಿಗಿಂತ ಬಲವಾದ ಆರನೆಯ ಇಂದ್ರಿಯ ಒಂದಿದೆ ಎಂದು ಭಾವಿಸುವರು. ಇದೇ ಶ್ರುತಿ ಪ್ರಮಾಣದ ಸತ್ಯತೆಗೆ ಕಾರಣ. ಒಬ್ಬನು ಜಗತ್ತಿನಲ್ಲಿರುವ ಧರ್ಮಕ್ಕೆ ಸಂಬಂಧಪಟ್ಟ ಪುಸ್ತಕಗಳನ್ನೆಲ್ಲಾ ಓದಿರಬಹುದು, ಆದರೂ ಅತಿ ದುರಾತ್ಮನಾಗಿರಬಹುದು. ಆಪ್ತವಾಕ್ಯ ಎಂದರೆ ಮುಂದೆ ಕಂಡುಹಿಡಿಯಲ್ಪಟ್ಟ ಅಧ್ಯಾತ್ಮಕ್ಕೆ ಸಂಬಂಧಪಟ್ಟ ವಿಷಯಗಳು.

ಕೆಲವರು ತೆಗೆದುಕೊಳ್ಳುವ ಎರಡನೆಯ ದೃಷ್ಟಿಯೇ ಆದಿ ಅಂತ್ಯವಿಲ್ಲದ ಸೃಷ್ಟಿ. ಸೃಷ್ಟಿ ಇಲ್ಲದ ಒಂದು ಕಾಲವಿತ್ತು ಎಂದು ಭಾವಿಸಿದರೆ ಅಗ ದೇವರು ಏನು ಮಾಡುತ್ತಿದ್ದ? ಎಂಬ ಪ್ರಶ್ನೆ ಏಳುತ್ತದೆ. ಹಿಂದೂವಿನ ದೃಷ್ಟಿಯಲ್ಲಿ ಸೃಷ್ಟಿ ಎನ್ನುವುದು ಕೇವಲ ಆಕಾರಕ್ಕೆ ಸಂಬಂಧಿಸಿರುವುದು. ಒಬ್ಬನಿಗೆ ಒಳ್ಳೆಯ ಆರೋಗ್ಯವಿದೆ, ಒಳ್ಳೆಯ ಮನೆತನದಲ್ಲಿ ಹುಟ್ಟುವನು, ಒಳ್ಳೆಯ ಧರ್ಮಾತ್ಮನಾಗಿ ಬೆಳೆಯುವನು. ಮತ್ತೊಬ್ಬ ಅಂಗಹೀನನಾಗಿ ಹುಟ್ಟುವನು, ದುರಾತ್ಮನಾಗುವನು, ಬೇಕಾದಷ್ಟು ಕಷ್ಟವನ್ನು ಅನುಭವಿಸುವನು. ಪವಿತ್ರನಾದ ಧರ್ಮಾತ್ಮನಾದ ದೇವರು ಒಬ್ಬನನ್ನು ಒಳ್ಳೆಯವನನ್ನಾಗಿಯೂ ಮತ್ತೊಬ್ಬನನ್ನು ಕೆಟ್ಟವನನ್ನಾ\break ಗಿಯೂ ಏತಕ್ಕೆ ಸೃಷ್ಟಿಸಿದ? ಇಲ್ಲಿ ಮನುಷ್ಯನಿಗೆ ಆಯ್ದು ಕೊಳ್ಳುವುದಕ್ಕೆ ಅವಕಾಶವೇ ಇಲ್ಲ. ದುಷ್ಕರ್ಮಿಯಲ್ಲಿ ಪಾಪದ ಬುದ್ಧಿಯಿರುತ್ತದೆ. ಧರ್ಮ ಅಧರ್ಮಗಳೆಂದರೆ ಏನು ಎಂಬುದನ್ನು ಉಪನ್ಯಾಸಕರು ವಿವರಿಸಿದರು. ದೇವರೇ ಎಲ್ಲವನ್ನೂ ಮಾಡಿದರೆ ವಿಜ್ಞಾನ ಕೊನೆಗೊಂಡಂತೆಯೇ. ಮನುಷ್ಯ ಪಾಪದಲ್ಲಿ ಎಷ್ಟು ಆಳಕ್ಕೆ ಹೋಗಬಲ್ಲ? ಮನುಷ್ಯ ಪುನಃ ಮೃಗದ ಸ್ಥಿತಿಗೆ ಹೋಗಲು ಸಾಧ್ಯವೆ?

ವಿವೇಕಾನಂದರು ತಾವು ಹಿಂದೂಗಳಾಗಿ ಹುಟ್ಟಿರುವುದು ಸಂತೋಷ ಎಂದರು. ರೋಮನ್ನರು ಜೆರುಸಲೆಂ ನಗರವನ್ನು ನಾಶಮಾಡಿದಾಗ ಹಲವು ಸಾವಿರ ಯಹೂದ್ಯರು ಭರತಖಂಡದಲ್ಲಿ ಆಶ್ರಯವನ್ನು ಪಡೆದರು. ಅರಬ್ಬರು ಪಾರ್ಸಿಯವರನ್ನು ಅವರ ದೇಶದಿಂದ ಓಡಿಸಿದಾಗ ಸಹಸ್ರಾರು ಜನ ಪಾರ್ಸಿಗಳು ಭರತಖಂಡದಲ್ಲಿ ಆಶ್ರಯವನ್ನು ಪಡೆದರು. ಅವರಲ್ಲಿ ಯಾರಿಗೂ ತೊಂದರೆಯನ್ನು ಕೊಡಲಿಲ್ಲ. ಹಿಂದೂಗಳು ಎಲ್ಲಾ ಧರ್ಮಗಳೂ ಸತ್ಯ ಎನ್ನುವುದನ್ನು ನಂಬುವರು. ಆದರೆ ಅವರ ಧರ್ಮ ಎಲ್ಲಕ್ಕಿಂತಲೂ ಪುರಾತನವಾದುದು. ಹಿಂದೂಗಳು ಎಂದಿಗೂ ಮಿಷನರಿಗಳಿಗೆ ತೊಂದರೆಯನ್ನು ಕೊಡುವುದಿಲ್ಲ. ಮೊದಲನೆ ಮಿಷನರಿಗಳು ಭರತಖಂಡಕ್ಕೆ ಬರದಂತೆ ಮಾಡಿದವರು ಇಂಗ್ಲೀಷರು. ಆ ಸಮಯದಲ್ಲಿ ಹಿಂದೂ ಒಬ್ಬನು ಮುಂದೆ ಬಂದು ಅವರ ಪರವಾಗಿ ಮಾತನಾಡಿ ಆಶ್ರಯಸ್ಥಾನವನ್ನು ಕೊಟ್ಟನು. ಧರ್ಮ ಎಂದರೆ ಎಲ್ಲವನ್ನು ಒಪ್ಪಿಕೊಳ್ಳುವುದು. ಧರ್ಮವನ್ನು, ಕೆಲವರು ಕುರುಡರು ಆನೆಯನ್ನು ತಿಳಿಯಲು ಮಾಡಿದ ಪ್ರಯತ್ನಕ್ಕೆ ಹೋಲಿಸಿದರು. ಪ್ರತಿಯೊಬ್ಬರೂ ಆನೆಯ ಬೇರೆ ಬೇರೆ ಭಾಗಗಳನ್ನು ಮುಟ್ಟುವರು. ಅದರಿಂದ ಆನೆ ಹೀಗಿರಬಹುದು ಎಂಬುದನ್ನು ನಿರ್ಧರಿಸಿದರು. ಪ್ರತಿಯೊಬ್ಬನೂ ತನ್ನ ತನ್ನ ದೃಷ್ಟಿಯಿಂದ ಸರಿಯೇ. ಆದರೆ ಅವರು ನೋಡಿರುವುದನ್ನೆಲ್ಲ ಒಟ್ಟಿಗೆ ಸೇರಿಸಬೇಕು, ಒಟ್ಟು ಆನೆಯಾಗಬೇಕಾದರೆ. ಹಿಂದೂ ದಾರ್ಶನಿಕರು, ಮನುಷ್ಯ ಸತ್ಯದಿಂದ ಸತ್ಯಕ್ಕೆ ಪ್ರಯಾಣ\break ಮಾಡವನು, ಕೆಳಗಿನ ಸತ್ಯದಿಂದ ಮೇಲಿನ ಸತ್ಯಕ್ಕೆ ಪ್ರಯಾಣ ಮಾಡುವನು ಎಂದು ಹೇಳುತ್ತಾರೆ. ಒಂದು ಕಾಲ ಬರುವುದು, ಆಗ ಎಲ್ಲರೂ ಒಂದೇ ಸಮನಾಗಿ ಆಲೋಚಿಸುವರು ಎಂದು ಭಾವಿಸುವುದು ಒಂದು ನನಸಾಗದ ಕನಸು ಅಷ್ಟೆ. ಹಾಗೇನಾದರೂ ಆದರೆ ಧರ್ಮವೇ ನಾಶವಾಗುವುದು. ಪ್ರತಿಯೊಂದು ಧರ್ಮವೂ ಸಣ್ಣ ಸಣ್ಣ ಮತಗಳಾಗಿ ಕವಲೊಡೆಯುವುದು. ಅದರಲ್ಲಿ ಪ್ರತಿಯೊಬ್ಬರೂ ತಾವೇ ಸತ್ಯವೆಂದೂ ಇತರರು ಸುಳ್ಳು ಎಂದು ಭಾವಿಸುವರು. ಧರ್ಮಕ್ಕಾಗಿ ಅನ್ಯರನ್ನು ಹಿಂಸೆ ಮಾಡುವುದು ಬೌದ್ಧರಿಗೆ ಗೊತ್ತೇ ಇಲ್ಲ. ಅವರೇ ಮೊದಲನೇ ಮಿಷನರಿಗಳನ್ನು ಪ್ರಪಂಚಕ್ಕೆ ಕಳುಹಿಸಿದುದು. ಒಂದು ತೊಟ್ಟು ರಕ್ತವನ್ನೂ ಚೆಲ್ಲದೆ ಕೋಟ್ಯಂತರ ಜನರನ್ನು ಬೌದ್ಧರನ್ನಾಗಿ ಮಾಡಿದವರು ತಾವೆಂದು ಅವರೊಬ್ಬರು ಮಾತ್ರ ಹೇಳಿಕೊಳ್ಳಬಲ್ಲರು. ಹಿಂದೂಗಳಲ್ಲಿ ಎಷ್ಟೇ ಲೋಪದೋಷಗಳು ಇರಲಿ ಅವರು ಧರ್ಮದ ಹೆಸರಿನಲ್ಲಿ ಅನ್ಯರನ್ನು ಪೀಡಿಸಲಿಲ್ಲ. ಕ್ರೈಸ್ತ ದೇಶಗಳಲ್ಲೆಲ್ಲ ಧರ್ಮದ ಹೆಸರಿನಲ್ಲಿ ಅಷ್ಟೊಂದು ಹಿಂಸೆ ಇರುವುದಕ್ಕೆ ಕ್ರೈಸ್ತರು ಹೇಗೆ ಅವಕಾಶ ಕೊಟ್ಟರು ಎಂಬುದನ್ನು ತಿಳಿದುಕೊಳ್ಳಬೇಕು ಎಂದು ಉಪನ್ಯಾಸಕರು ಹೇಳಿದರು.

\delimiter


\section[ಪವಾಡಗಳು]{ಪವಾಡಗಳು\protect\footnote{* C.W. Vol. III P. 495}}

\begin{center}
(ಈವನಿಂಗ್​ ನ್ಯೂಸ್​, ಫೆಬ್ರವರಿ ೧೭, ೧೮೯೪)
\end{center}

\vskip -0.3cm

“ನನ್ನ ಧರ್ಮ ಸತ್ಯ ಎಂದು ಸಾಧಿಸುವುದಕ್ಕೆ, ‘ಈವನಿಂಗ್​ ನ್ಯೂಸ್​’ ಕೋರುವಂತೆ, ನಾನು ಯಾವ ಪವಾಡಗಳನ್ನೂ ಮಾಡುವವನಲ್ಲ.” ಈ ಪೇಪರಿನ ಒಬ್ಬ ಬಾತ್ಮೀದಾರರಿಗೆ ಸ್ವಾಮಿ ವಿವೇಕಾನಂದರು ಈ ವಿಷಯದ ಬಗ್ಗೆ ನ್ಯೂಸ್​ ಸಂಪಾದಕೀಯವನ್ನು ನೋಡಿದ ಮೇಲೆ, ಮೇಲಿನಂತೆ ಉತ್ತರ ಕೊಟ್ಟರು. “ಮೊದಲನೆಯದಾಗಿ ನಾನು ಪವಾಡವನ್ನು ಮಾಡುವವನಲ್ಲ. ಎರಡನೆಯದಾಗಿ, ನಾನು ಸೇರಿರುವ ಹಿಂದೂಧರ್ಮ ಯಾವ ಪವಾಡಗಳ ಮೇಲೂ ನಿಂತಿಲ್ಲ. ನಾವು ಪವಾಡಗಳನ್ನು ಒಪ್ಪಿಕೊಳ್ಳುವುದಿಲ್ಲ. ನಮ್ಮ ಪಂಚೇಂದ್ರಿಯಗಳಿಗೆ ನಿಲುಕದಂತಹ ವಿಚಿತ್ರಗಳಿವೆ. ಆದರೆ ಅವೆಲ್ಲ ಯಾವುದೋ ನಿಯಮಕ್ಕೆ ಬದ್ಧವಾಗಿವೆ. ನಮ್ಮ ಧರ್ಮ ಇವುಗಳಾವುದನ್ನೂ ಗಮನಿಸುವುದಿಲ್ಲ. ಇಂಡಿಯಾ ದೇಶದಲ್ಲಿ ನಡೆಯುವ ಹಲವು ಅದ್ಭುತಗಳು, ಪರದೇಶದ ವೃತ್ತ ಪತ್ರಿಕೆಗಳಲ್ಲಿ ಪ್ರಕಟವಾದವುಗಳು. ಇವೆಲ್ಲ ಕೇವಲ ಕೈ ಚಳಕ ಅಥವಾ ಸುಪ್ತಿ ಆವಾಹಕನಿಂದ ಆದ ಕ್ರಿಯೆಗಳು (hypnotism). ಯಾವ ಸಾಧು ಪುರುಷನೂ ಇವುಗಳನ್ನು ಮಾಡುವುದಕ್ಕೆ ಹೋಗುವುದಿಲ್ಲ. ಸಂತೆಯಲ್ಲಿ ದುಡ್ಡನ್ನು ವಸೂಲಿ ಮಾಡಲು ಇಂತಹ ಡೊಂಬರಾಟವನ್ನು ಅವರು ತೋರಿಸುತ್ತ ಹೋಗುವುದಿಲ್ಲ. ಯಾರು ಸತ್ಯವನ್ನು ತಿಳಿದುಕೊಳ್ಳಬೇಕೆಂದು ಆಶಿಸುವರೊ ಅವರು ಮಾತ್ರ ಅಂತಹ ಸಾಧುಪುರುಷರನ್ನು ನೋಡಬಹುದು, ಮತ್ತು ತಿಳಿದುಕೊಳ್ಳಬಹುದು. ಕೇವಲ ಬಾಲಸಹಜ ಕುತೂಹಲಕ್ಕೆ ಅವರು ಒಳಗಾಗುವುದಿಲ್ಲ.”

\delimiter


\section[ಮಾನವನ ದಿವ್ಯತೆ]{ಮಾನವನ ದಿವ್ಯತೆ\protect\footnote{* C.W. Vol. III P. 496}}

\begin{center}
(ಡೆಟ್ರಾಯಿಟ್​ ಫ್ರೀ ಪ್ರೆಸ್​, ಫೆಬ್ರವರಿ ೧೮, ೧೮೯೪)
\end{center}

ಹಿಂದೂ ದಾರ್ಶನಿಕರು ಮತ್ತು ಸಂನ್ಯಾಸಿಗಳಾದ ಸ್ವಾಮಿ ವಿವೇಕಾನಂದರು ಕಳೆದ ರಾತ್ರಿ ಯೂನಿಟೇರಿಯನ್​ ಚರ್ಚ್​ನಲ್ಲಿ ತಮ್ಮ ಉಪನ್ಯಾಸಗಳನ್ನು ಅಥವಾ ಬೋಧನೆಯನ್ನು ಮುಕ್ತಾಯಗೊಳಿಸಿದರು. ಅವರ ಕೊನೆಯ ಉಪನ್ಯಾಸದ ವಿಷಯ “ಮಾನವನ ದಿವ್ಯತೆ” ಎಂಬುದು. ಹವಾಗುಣ ಅಷ್ಟು ಚೆನ್ನಾಗಿ ಇಲ್ಲದೇ ಇದ್ದರೂ, ಪೌರಸ್ತ್ಯ ಸಹೋದರನು (ಹೀಗೆ ಕರೆಯುವುದನ್ನು ಅವರು ಇಷ್ಟಪಡುತ್ತಾರೆ) ಬರುವುದಕ್ಕೆ ಅರ್ಧ ಗಂಟೆ ಮುಂಚೆಯೇ ಜನರು ಬಾಗಿಲಿನವರೆಗೂ ಕಿಕ್ಕಿರಿದು ನೆರೆದಿದ್ದರು. ಉತ್ಸಾಹದಿಂದ ಕೇಳುತ್ತಿದ್ದ ಸಭಿಕರಲ್ಲಿ ಜೀವನದ ಎಲ್ಲಾ ಕಾರ್ಯಕ್ಷೇತ್ರಗಳಿಗೆ ಸೇರಿದವರೂ ಇದ್ದರು. ಲಾಯರುಗಳು, ನ್ಯಾಯಾಧಿ\break ಪತಿಗಳು, ಬೈಬಲನ್ನು ಬೋಧಿಸುವ ಪಾದ್ರಿಗಳು, ವರ್ತಕರು ಮತ್ತು ರಬ್ಬಿಯೂ ಕೂಡ ಇದ್ದರು. ಪ್ರತಿದಿನವೂ ಅವರು ಉಪನ್ಯಾಸವನ್ನು ಉತ್ಸಾಹದಿಂದ ಕೇಳುವುದಕ್ಕೆ ಬರುತ್ತಿದ್ದ ಮಹಿಳೆಯರ ವಿಷಯವಾಗಿ ಹೇಳಲೇ ಬೇಕಾಗಿಲ್ಲ. ಅವರು ಕಂದು ಬಣ್ಣದ ಆಗಂತುಕ ವ್ಯಕ್ತಿಯ ಮೇಲೆ ತಮ್ಮ ಮೆಚ್ಚುಗೆಯ ಪುಷ್ಪವೃಷ್ಟಿಯನ್ನು ಕರೆದಿರುವರು. ವಿವೇಕಾನಂದರು ವೇದಿಕೆಯ ಮೇಲೆ ಎಷ್ಟು ಚೆನ್ನಾಗಿ ಉಪನ್ಯಾಸ ಮಾಡಬಲ್ಲರೋ ಹಾಗೆಯೇ ದಿವಾನ್​ ಖಾನೆಯ ಸಂಭಾಷಣೆಯಲ್ಲಿಯೂ ಅಗ್ರಗಣ್ಯರು.

ನೆನ್ನೆ ಮಾಡಿದ ಉಪನ್ಯಾಸವು ಹಿಂದಣ ಉಪನ್ಯಾಸಗಳಂತೆ ಹೆಚ್ಚು ವಿವರಣಾತ್ಮಕ ವಾಗಿರಲಿಲ್ಲ. ಸುಮಾರು ಎರಡು ಗಂಟೆಗಳ ಕಾಲ ವಿವೇಕಾನಂದರು ಮಾನವ ಮತ್ತು ದೇವರಿಗೆ ಸಂಬ್ಂಧಪಟ್ಟ ತಾತ್ತ್ವಿಕ ವಿಷಯವಾಗಿ ರಮ್ಯವಾದ ವಾಗ್​ಜಾಲವನ್ನು ನೆಯ್ದರು. ಅವರ ಉಪನ್ಯಾಸ ಎಷ್ಟು ತರ್ಕಬದ್ಧವಾಗಿತ್ತು ಎಂದರೆ ಅದರ ಮುಂದೆ ವಿಜ್ಞಾನವೂ ಸಾಮಾನ್ಯ ವಿಷಯವಾಗಿ ಕಂಡುಬಂದಿತು. ಅವರು ತರ್ಕಬದ್ಧವಾದ ಒಂದು ಸುಂದರ ವಸ್ತ್ರವನ್ನೇ ಹೆಣೆದಿದ್ದರು. ಅವರ ದೇಶದಲ್ಲಿ ಕೈಮಗ್ಗದಿಂದ ಮಾಡಿದ, ವಿವಿಧ, ಬಣ್ಣಗಳಿಂದ ಕೂಡಿದ, ಅತ್ಯಂತ ಮೋಹಕವಾದ ಪೌರಸ್ತ್ಯ ಸುವಾಸನೆಗಳಿಂದ ಕೂಡಿದ ವಸ್ತ್ರದಂತೆ ಅವರ ಉಪನ್ಯಾಸವು ವರ್ಣರಂಜಿತವಾಗಿದ್ದು ಆಕರ್ಷಣೀಯವಾಗಿತ್ತು, ಮತ್ತು ಮನನ ಮಾಡಲು ತೃಪ್ತಿಕರವಾಗತ್ತು. ಈ ಕಂದು ಬಣ್ಣದ ವ್ಯಕ್ತಿ ಚಿತ್ರಕಾರನು ಬಣ್ಣವನ್ನು ಉಪಯೋಗಿಸುವಂತೆ ಕಾವ್ಯಮಯವಾದ ರೂಪಕಗಳನ್ನು ಉಪಯೋಗಿಸುವರು. ಎಲ್ಲಿ ಬಣ್ಣವನ್ನು ಬಳಿಯಬೇಕೆ ಅಲ್ಲಿ ಅದನ್ನು ಇಡುವರು. ಪರಿಣಾಮವಾಗಿ ಅದು ಹೆಚ್ಚು ವಿಲಕ್ಷಣವಾಗಿ ಕಂಡರೂ ಅಲ್ಲಿ ಒಂದು ವಿಚಿತ್ರವಾದ ಆಕರ್ಷಣೆಯಿತ್ತು. ವೇಗವಾಗಿ ಒಂದಾದ ಮೇಲೊಂದರಂತೆ ಬರುತ್ತಿದ್ದ ತರ್ಕಬದ್ಧವಾದ ನಿರ್ಣಯಗಳು ವಿವಿಧಾಕಾರದಲ್ಲಿ, ವಿವಿಧ ಬಣ್ಣಗಳಲ್ಲಿ ಹೊಳೆಯುವ ಗಾಜಿನ ಮಣಿಗಳಂತೆ ಕಂಗೊಳಿಸುತ್ತಿದ್ದವು. ಉಪನ್ಯಾಸಕರ ಅದ್ಭುತ ಕೌಶಲ್ಯಕ್ಕೆ ತಕ್ಕ ಪ್ರತಿಫಲವೋ ಎಂಬಂತೆ ಉತ್ಸಾಹಪೂರ್ಣವಾದ ಕರತಾಡನಗಳು ಸಭಿಕರಿಂದ ಬರುತ್ತಿದ್ದುವು.

ಜನರು ಅನೇಕ ಪ್ರಶ್ನೆಗಳನ್ನು ಕೇಳಿರುವರು ಎಂಬ ಹೇಳಿಕೆಯೊಂದಿಗೆ ಉಪನ್ಯಾಸವನ್ನು ಪ್ರಾರಂಭಿಸಿದರು. ಇವುಗಳಲ್ಲಿ ಹಲವನ್ನು ಏಕಾಂತವಾಗಿ ಉತ್ತರಿಸುವೆನು ಎಂದು ಹೇಳಿದರು. ಆ ಪ್ರಶ್ನೆಗಳಲ್ಲಿ ಬಹಿರಂಗವಾಗಿ ವೇದಿಕೆಯ ಮೇಲಿನಿಂದ ಉತ್ತರ ಕೊಡುವುದಕ್ಕೆ ಮೂರನ್ನು ಆರಿಸಿಕೊಂಡರು. ಆ ಪ್ರಶ್ನೆಗಳು ಯಾವುವೆಂದರೆ:

\vskip -0.15cm

\begin{myquote}
“ಭರತಖಂಡದ ಜನರು ತಮ್ಮ ಮಕ್ಕಳನ್ನು ಮೊಸಳೆಗಳ ಬಾಯಿಗೆ ಕೊಡುವರೆ?”\\ “ಜಗನ್ನಾಥನ ರಥಕ್ಕೆ ಸಿಕ್ಕಿಕೊಂಡು ಆತ್ಮಹತ್ಯೆ ಮಾಡಿಕೊಳ್ಳುವರೆ?” \\ ವಿಧವೆಯರನ್ನು ಅವರ ಗಂಡನ ಶವದೊಡನೆ ಸುಡುವರೆ?”
\end{myquote}

\vskip -0.27cm

ಮೊದಲನೆಯ ಪ್ರಶ್ನೆಯನ್ನು, ಅಮೆರಿಕಾ ದೇಶೀಯನು ಪರದೇಶಕ್ಕೆ ಹೋದರೆ ಅಲ್ಲಿ ಇವನನ್ನು, ರೆಡ್​ ಇಂಡಿಯನ್ನರು ನ್ಯೂಯಾರ್ಕ್​ ನಗರಿ ಬೀದಿಗಳಲ್ಲಿ ಓಡಿ ಹೋಗುತ್ತಿರುವರೆ, ಎಂದು ಮುಂತಾದ ಕಥೆಗಳಿಗೆ ಅವನು ಹೇಗೆ ಉತ್ತರ ಕೊಡುವನೋ ಹಾಗೆಯೇ ಇದನ್ನು ಪರಿಗಣಿಸಿದರು. ಪ್ರಶ್ನೆ ತುಂಬಾ ಹಾಸ್ಯಾಸ್ಪದವಾಗಿತ್ತು. ಅದಕ್ಕೆ ಒಂದು ಸರಿಯಾದ ಉತ್ತರ ಕೊಡುವಷ್ಟು ಯೋಗ್ಯವಾಗಿರಲಿಲ್ಲ. ಕೆಲವು ಒಳ್ಳೆಯ ಜನರು, ಆದರೆ ವಿಷಯ ಗೊತ್ತಿಲ್ಲದೆ ಇರುವವರು, ಏತಕ್ಕೆ ಹೆಣ್ಣು ಮಕ್ಕಳನ್ನು ಮೊಸಳೆಗಳ ಬಾಯಿಗೆ ಕೊಡುತ್ತಾರೆ ಎಂದು ಕೇಳಿದಾಗ, ಸ್ವಾಮೀಜಿ ತಮಾಷೆಯಿಂದ ನಗುತ್ತ, ಬಹುಶಃ ಹೆಣ್ಣು ಮಕ್ಕಳು ಮೃದುವಾಗಿರುವುದರಿಂದ ಆ ಅನಾಗರಿಕದೇಶದ ನದಿಗಳಲ್ಲಿ ವಾಸಿಸುವ ಮೊಸಳೆಗಳು ಚೆನ್ನಾಗಿ ಅಗಿದು ಜೀರ್ಣಿಸಿಕೊಳ್ಳುವುದಕ್ಕೆ ಸುಲಭವಾಗಿದ್ದುದರಿಂದಲೋ ಏನೋ ಎಂದು ಹಾಸ್ಯವಾಗಿ ಉತ್ತರ ಕೊಟ್ಟರು. ಜಗನ್ನಾಥ ರಥಕ್ಕೆ ಸಿಕ್ಕಿ ಆತ್ಮಹತ್ಯೆ ಮಾಡಿಕೊಳ್ಳುವ ಪ್ರಸಂಗವನ್ನು ತೆಗೆದುಕೊಂದು ಬಹುಶಃ ಕೆಲವರು ಉತ್ಸಾಹದಿಂದ ರಥದ ಹಗ್ಗವನ್ನು ಹಿಡಿದುಕೊಂಡು ಎಳೆಯುವುದರಲ್ಲಿ ಭಾಗಿಗಳಾಗಬೇಕು ಎಂದು ಪ್ರಯತ್ನಿಸಿದಾಗ ಜಾರಿಬಿದ್ದು ಗಾಲಿಗೆ ಸಿಕ್ಕಿ ಸತ್ತಿರಬಹುದು, ಎಂದರು. ಇಂತಹ ಕೆಲವು ಪ್ರಸಂಗಗಳನ್ನು ಉತ್ಪ್ರೇಕ್ಷೆ ಮಾಡಿ ಸತ್ಯದೂರವಾಗಿ ಮಾಡಿರಬೇಕು ಮತ್ತು ಇತರ ಸಜ್ಜನರು ಇದನ್ನು ಕೇಳಿ ದಿಗ್ಭ್ರಾಂತರಾಗಿರ\break ಬೇಕು. ವಿವೇಕಾನಂದರು ವಿಧವೆಯರನ್ನು ಸುಡುವುದು ಸುಳ್ಳು ಎಂದು ಹೇಳಿದರು. ವಿಧವೆಯರು ಕೆಲವು ವೇಳೆ ತಾವೇ ಬೆಂಕಿಗೆ ಬಿದ್ದುದು ನಿಜ. ಇಂತಹ ಸಂದರ್ಭಗಳಲ್ಲಿ ಆತ್ಮಹತ್ಯೆಯನ್ನು ವಿರೋಧಿಸುವ ಪುರೋಹಿತರು ಮತ್ತು ಗುರು ಹಿರಿಯರು ಹೀಗೆ ಮಾಡುವುದು ಒಳ್ಳೆಯದಲ್ಲ ಎಂದು ಅವರಿಗೆ ಎಷ್ಟೋ ಬೋಧಿಸಿದ್ದರು. ಆದರೆ ಪತಿವ್ರತೆಯರಾದ ವಿಧವೆಯರು ತಾವು ಗಂಡನನ್ನು ಅನುಸರಿಸಿ ಹೋಗುತ್ತೇವೆ ಎಂದು ಹಟ ಹಿಡಿದಾಗ ಅವರನ್ನು ಮುಂಚೆ ಒಂದು ಅಗ್ನಿ ಪರೀಕ್ಷೆಗೆ ಗುರಿಮಾಡುತ್ತಿದ್ದರು. ಅದಾವುದೆಂದರೆ, ಬೆಂಕಿಯ ಉರಿಗೆ ತಮ್ಮ ಕೈಯನ್ನು ಒಡ್ಡುವುದು. ಅದು ಸುಡುತ್ತಿರುವಾಗ ಅವರು ಸ್ವಲ್ಪವೂ ಹಿಂದೆಗೆಯದೇ ಇದ್ದರೆ ಅನಂತರ ಚಿತೆಯಲ್ಲಿ ಬೀಳುವುದಕ್ಕೆ ಅವಕಾಶ ಕೊಡುತ್ತಿದ್ದರು. ಆದರೆ ಪ್ರೀತಿಸುತ್ತಿದ್ದ ಪತಿಯನ್ನು ಮೃತ್ಯುವಿನ ನಂತರವೂ ಅನುಸರಿಸಿದ ಸತಿಯರು ಇಂಡಿಯ ದೇಶ ಒಂದರಲ್ಲೇ ಅಲ್ಲ, ಇನ್ನೂ ಬೇರೆ ದೇಶಗಳಲ್ಲಿಯೂ ಇರುವರು. ಇಂತಹ ಆತ್ಮಹತ್ಯೆಗಳು ಪ್ರತಿಯೊಂದು ದೇಶಗಳಲ್ಲಿಯೂ ಆಗಿವೆ. ಇದು ಎಲ್ಲಾ ದೇಶದಲ್ಲಿಯೂ ಇರುವ ಅಸಾಧಾರಣವಾದ ಒಂದು ಆಚಾರ. ಇಲ್ಲ, ಇಂಡಿಯಾದೇಶದಲ್ಲಿ ಜನರು ಸ್ತ್ರೀಯರನ್ನು ಸುಡುವುದಿಲ್ಲ ಅಥವಾ ಮಂತ್ರವಾದಿನಿಯರನ್ನು ಅವರು ಎಂದೂ ಸುಟ್ಟಿಲ್ಲ- ಎಂದು ಉಪನ್ಯಾಸಕರು ಒತ್ತಿಹೇಳಿದರು.

ಅನಂತರ ಅವರು ಮುಖ್ಯ ಉಪನ್ಯಾಸವನ್ನು ಪ್ರಾರಂಭ ಮಾಡಿದರು ವಿವೇಕಾನಂದರು ಜೀವಿಯ ದೇಹ ಮನಸ್ಸು ಆತ್ಮ ಇವುಗಳನ್ನು ವಿವರಿಸತೊಡಗಿದರು. ದೇಹ ಕೇವಲ ಒಂದು ಕೋಶದಂತೆ. ಮನಸ್ಸು ತಾತ್ಕಾಲಿಕವಾಗಿ ವಿಚಿತ್ರವಾಗಿ ಕೆಲಸ ಮಾಡುವುದು. ಆದರೆ ಆತ್ಮನಿಗಾದರೊ ಪ್ರತ್ಯೇಕ ವೈಶಿಷ್ಟ್ಯವಿದೆ. ಆತ್ಮನ ಅನಂತತೆಯನ್ನು ಅರಿಯುವುದೇ ಮುಕ್ತಿ. ನಮ್ಮ ಮನಸ್ಸಿಗೆ ಒಪ್ಪುವ ರೀತಿಯಲ್ಲಿ ವಾದಿಸುತ್ತ ಉಪನ್ಯಾಸಕರು ಪ್ರತಿಯೊಂದು ಆತ್ಮವೂ ಸ್ವತಂತ್ರವಾದುದು ಎಂದು ಹೇಳಿದರು. ಅದೇನಾದರೂ ಬದ್ಧವಾಗಿದ್ದರೆ ಅನಂತರ ಅದು ಅಮರತ್ವವನ್ನು ಪಡೆಯುತ್ತಿರಲಿಲ್ಲ. ಇದರ ಸಾಕ್ಷಾತ್ಕಾರ ಒಬ್ಬನಿಗೆ ಹೇಗೆ ಬರುತ್ತದೆ ಎಂಬುದನ್ನು ವಿವರಿಸುತ್ತ ಅವರ ದೇಶದಲ್ಲಿ ರೂಢಿಯಲ್ಲಿರುವ ಒಂದು ಕಥೆಯನ್ನು ಹೇಳಿದರು. ಒಂದು ಗರ್ಭಿಣಿ ಸಿಂಹವು ಕುರಿಯ ಮಂದೆಯ ಮೇಲೆ ಬಿದ್ದಾಗ ಒಂದು ಮರಿಯನ್ನು ಹಾಕಿ ತೀರಿಕೊಂಡಿತು. ಕುರಿಗಳು ಸಿಂಹದ ಮರಿಗೆ ಹಾಲನ್ನು ಕೊಟ್ಟು ಸಂರಕ್ಷಿಸಿದವು. ಆ ಸಿಂಹದ ಮರಿ ಕುರಿಯ ಮಂದೆಯಲ್ಲಿಯೇ ಬೆಳೆಯುತ್ತಾ ಹೋಯಿತು. ಆ ಕುರಿಗಳಂತೆಯೇ ವ್ಯವಹರಿಸತೊಡಗಿತು. ಆದರೆ ಒಂದು ದಿನ ಮತ್ತೊಂದು ಸಿಂಹ ಬಂತು. ಅದು ಕುರಿಯ ಮಂದೆಯಲ್ಲಿದ್ದ ಸಿಂಹವನ್ನು ಒಂದು ಬಾವಿಯ ಸಮೀಪಕ್ಕೆ ಕರೆದು ಕೊಂಡು ಹೋಗಿ ತನ್ನ ನೆರಳು ಮತ್ತು ಅದರ ನೆರಳು ಎರಡನ್ನೂ ತೋರಿಸಿತು. ಆಗ ಕುರಿಯ ಮಂದೆಯಲ್ಲಿದ್ದ ಸಿಂಹ ತಾನು ಕೂಡ ಸಿಂಹವೇ ಎಂಬುದನ್ನು ಅರಿತು ಕೊಂಡಿತು. ಅದೂ ಸಿಂಹದಂತೆ ಘರ್ಜಿಸಿತು. ತಾನೂ ಸಿಂಹ ಎಂಬುದನ್ನು ಅದು ಅರಿತುಕೊಂಡಿತು. ಅನೇಕ ಜನರು ಕುರಿಯಂತೆ ಇರುವ ಸಿಂಹಗಳು. ಒಂದು ಮೂಲೆಗೆ ಹೋಗಿ ತಾವು ಪಾಪಿಗಳು, ಕೆಲಸಕ್ಕೆ ಬಾರದವರು ಎಂದು ತಮ್ಮನ್ನು ಕೀಳಾಗಿ ಭಾವಿಸುವರು. ತಮ್ಮಲ್ಲೆ ಹುದುಗಿರುವ ಪೂರ್ಣತೆಯನ್ನಾಗಲೀ ಪವಿತ್ರತೆಯನ್ನಾಗಲೀ ಇನ್ನೂ ತಿಳಿದುಕೊಂಡಿಲ್ಲ. ಸ್ತ್ರೀಪುರುಷರ ಅಹಂಕಾರವೇ ಜೀವ. ಜೀವ ಸ್ವತಂತ್ರವಾಗಿದ್ದರೆ ಅದನ್ನು ಪೂರ್ಣವಾದ ಅನಂತದಿಂದ ಹೇಗೆ ಪ್ರತ್ಯೇಕಿಸಲು ಸಾಧ್ಯ? ದೊಡ್ಡ ಸೂರ್ಯನು ಸರೋವರದ ಮೇಲೆ ಬೆಳಗುತ್ತಿರುವನು. ಆಗ ಅನೇಕ ಪ್ರತಿಬಿಂಬಗಳು ಕಾಣುವುವು. ಪ್ರತಿಬಿಂಬಗಳಿಗೆಲ್ಲ ಮೂಲ ಸೂರ್ಯನೇ ಎಂದು ಗೊತ್ತಾಗಿದ್ದರೂ ಪ್ರತಿಯೊಂದು ಪ್ರತಿಬಿಂಬವನ್ನೂ ಬೇರೆ ಬೇರೆ ಎಂದು ಭಾವಿಸುವಂತೆ ಇದೆ ಇದು. ಜೀವಿಗೆ ಯಾವ ಲಿಂಗವೂ ಇಲ್ಲ. ನಿರಪೇಕ್ಷವಾದ ಮುಕ್ತಿಯನ್ನು ಅದು ಪಡೆದ ಮೇಲೆ, ದೈಹಿಕವಾದ ಲೈಂಗಿಕತೆಯಿಂದ ಅದಕ್ಕೇನು? ಈ ವಿಷಯದಲ್ಲಿ ಉಪನ್ಯಾಸಕರು ಸ್ವೀಡನ್​ಬರ್ಗ​ನ ತತ್ತ್ವ ಅಥವಾ ಧರ್ಮದ ವಿಷಯವನ್ನೂ ಕೂಲಂಕುಷವಾಗಿ ವಿವರಿಸಿದರು. ಹಿಂದೂಗಳ ಭಾವನೆಗೂ ಇತ್ತೀಚೆಗೆ ಬಂದ ಈ ಮಹಾತ್ಮನ ಭಾವನೆಗೂ ಇರುವ ಸಾದೃಶ್ಯ ಸ್ಪಷ್ಟವಾಗಿ ಗೊತ್ತಾಯಿತು. ಸ್ವೀಡನ್​ ಬರ್ಗನು ಹಿಂದಿನ ಬ್ರಾಹ್ಮಣ ವರ್ಗದ ಐರೋಪ್ಯ ಉತ್ತರಾಧಿಕಾರಿಯಂತೆ ತೋರಿದನು. ಸ್ವೀಡನ್​ಬರ್ಗನು ಆಧುನಿಕ ವೇಷವನ್ನು ತೊಟ್ಟ ಪ್ರಾಚೀನ ಭಾವನೆಗಳ ಪ್ರತಿನಿಧಿಯಂತೆ ಇದ್ದನು. ಅತಿ ಶ್ರೇಷ್ಠ ಫ್ರೆಂಚ್​ ಕಾದಂಬರಿಕಾರ ಮತ್ತು ತತ್ತ್ವ ಜ್ಞಾನಿಯೊಬ್ಬನು (ಬಾಲ್ಜಾಕ್​) ಒಬ್ಬ ಪರಿಪೂರ್ಣ ವ್ಯಕ್ತಿಯ ಜೀವನವನ್ನು ಚಿತ್ರಿಸುವಾಗ ಸ್ವೀಡನ್​ಬರ್ಗನ ಕೆಲವು ಭಾವನೆಯನ್ನು ಸೇರಿಸುವುದು ಯೋಗ್ಯವೆಂದು ಭಾವಿಸುವನು. ಪ್ರತಿಯಂದು ವ್ಯಕ್ತಿಯ ಅಂತರಾಳದಲ್ಲಿಯೂ ಪರಿಪೂರ್ಣತೆ ಇದೆ. ಅದು ಅವನ ದೇಹದ ಹಿಂದೆ ರಹಸ್ಯವಾದ ಗುಹೆಯಲ್ಲಿ ಅವಿತಿರುವುದು. ದೇವರು ತನ್ನಲ್ಲಿರುವ ಪರಿಪೂರ್ಣತೆಯನ್ನು ಮನುಷ್ಯನಿಗೆ ಸ್ವಲ್ಪ ದಾನ ಮಾಡಿರುವುದರಿಂದ ಮನುಷ್ಯ ಪವಿತ್ರನಾದ ಎಂದು ಭಾವಿಸಿದರೆ, ಇದನ್ನು ದಾನ ಮಾಡಿದ ಮೇಲೆ ದೇವರ ಪೂರ್ಣತೆ ಅಷ್ಟು ಕಡಿಮೆಯಾದಂತಾಗುವುದು. ಮಾರ್ಪಡಿಸಲಾಗದ ವೈಜ್ಞಾನಿಕ ನಿಯಮವೇ, ಆತ್ಮವು ವೈಯಕ್ತಿಕವಾದುದು, ಪೂರ್ಣತೆ ಆಗಲೇ ಅದರಲ್ಲಿ ಸುಪ್ತವಾಗಿದೆ ಎಂಬುದು. ಮತ್ತೆ ಅದನ್ನು ಪಡೆಯುವುದೇ ಮುಕ್ತಿ, ಅಥವಾ ವೈಯಕ್ತಿಕ ಅನಂತತೆಯನ್ನು ಪಡೆಯುವುದು. ಪ್ರಕೃತಿ, ದೇವರು, ಧರ್ಮ ಇವೆಲ್ಲಾ ಒಂದೇ.

ಧರ್ಮಗಳೆಲ್ಲ ಒಳ್ಳೆಯವೆ. ಒಂದು ಗಾಜಿನ ಬುಡ್ಡಿಯ ಕೆಳಗೆ ಇರುವ ಗಾಳಿಯ ಗುಳ್ಳೆ, ಮೇಲಿರುವ ಅನಂತ ಗಾಳಿಯಲ್ಲಿ ಒಂದಾಗಬಯಸುವುದು. ಎಣ್ಣೆ, ವಿನಿಗರ್​ ಮತ್ತು ಇತರ ದ್ರವ್ಯಗಳಲ್ಲಿ, ಅದರ ಸಾಂದ್ರತೆಯ ತರತಮಕ್ಕೆ ತಕ್ಕಂತೆ, ಗುಳ್ಳೆ ಮೇಲೇಳುವುದಕ್ಕೆ ಹಿಡಿವ ಕಾಲ ವ್ಯತ್ಯಾಸವಾಗುವುದು. ಇದರಂತೆಯೇ ಜೀವವು ಹಲವು ವಾತಾವರಣಗಳ ಮೂಲಕ ತನ್ನ ಅನಂತತೆಯನ್ನು ಪಡೆಯಲು ಯತ್ನಿಸುವುದು. ಒಂದು ಧರ್ಮ ಕೆಲವರಿಗೆ ಸರಿಯಾಗುವುದು. ಏಕೆಂದರೆ ಅವರ ಜೀವನದ ಅಭ್ಯಾಸ, ಆನುವಂಶಿಕವಾಗಿ ಬಂದ ಸ್ವಭಾವಗಳು, ಮತ್ತು ಅಲ್ಲಿಯ ಹವಾಗುಣದ ಪ್ರಭಾವ- ಇವುಗಳಿಗೆ ಅನುಗುಣವಾದ ಧರ್ಮವನ್ನು ಅವರು ಅನುಸರಿಸಬೇಕಾಗುತ್ತದೆ. ಇದೇ ಕಾರಣದಿಂದಾಗಿಯೇ ಬೇರೊಂದು ಧರ್ಮ ಅವರಿಗೆ ಅನ್ವಯಿಸದು. ಇರುವುದೆಲ್ಲ ಶ್ರೇಷ್ಠ, ಎಂಬುದೇ ಉಪನ್ಯಾಸದ ಉಪಸಂಹಾರದ ಸಾರಾಂಶ ಎನ್ನಬಹುದು. ಒಂದು ದೇಶದ ಧರ್ಮವನ್ನು ಇದ್ದಕ್ಕೆ ಇದ್ದಂತೆ ಬದಲಾವಣೆ ಮಾಡುವುದು, ಆಲ್ಪ್ಸ್ ಪರ್ವತದಿಂದ ಹರಿದು ಬರುವ ನದಿಯನ್ನು ನೋಡಿ ಅದು ಹರಿದು ಬರುವ ಮಾರ್ಗದಲ್ಲಿ ತಪ್ಪು ಕಂಡುಹಿಡಿದಂತೆ. ಮತ್ತೊಬ್ಬನು ಒಂದು ಮಹಾ ಪ್ರವಾಹ ಹಿಮಾಲಯದಿಂದ ಹರಿದುಕೊಂಡು ಬರುತ್ತಿರುವುದನ್ನು ನೋಡುವನು. ಅದು ಹಲವಾರು ತಲೆಮಾರುಗಳಿಂದ, ಸಹಸ್ರಾರು ವರುಷಗಳಿಂದಲೂ ಹರಿದು ಬರುತ್ತಿದೆ. ಅದನ್ನು ನೋಡಿ ಹತ್ತಿರದ ಒಳ್ಳೆಯ ಮಾರ್ಗವನ್ನು ಆ ನದಿ ಆರಿಸಿಕೊಂಡಿಲ್ಲ ಎಂದು ಟೀಕಿಸುವನು. ಕ್ರೈಸ್ತರು, ದೇವರು ಎಂದರೆ ಮೇಲೆ ಎಲ್ಲೋ ಕುಳಿತುಕೊಂಡಿರುವನು ಎಂದು ಭಾವಿಸುವರು. ಕ್ರೈಸ್ತನು ಸ್ವರ್ಗದಲ್ಲಿದ್ದರೂ, ಅಲ್ಲಿಯ ಚಿನ್ನದ ಬೀದಿಯ ಕೊನೆಯಲ್ಲಿ ನಿಂತು, ಬೇರೆ ಲೋಕಕ್ಕೂ ಅದಕ್ಕೂ ಇರುವ ವ್ಯತ್ಯಾಸವನ್ನು ನೋಡದೆ ಇದ್ದರೆ ಅವನು ಸ್ವರ್ಗಸುಖವನ್ನು ಅನುಭವಿಸಲಾರ. ಸುವರ್ಣ ನಿಯಮಕ್ಕೆ ಬದಲಾಗಿ ಹಿಂದೂಗಳು ಎಲ್ಲಾ ನಿಃಸ್ವಾರ್ಥತೆಯೂ ಪುಣ್ಯ, ಸ್ವಾರ್ಥವೇ ಪಾಪ ಎಂದು ನಂಬುವರು. ಈ ನಂಬಿಕೆಯ ಮೂಲಕ ಜೀವಿ ಸರಿಯಾದ ಕಾಲದಲ್ಲಿ ಪೂರ್ಣತೆ ಮತ್ತು ಅನಂತತೆಯನ್ನು ಪಡೆಯಲು ಸಾಧ್ಯವಾಗುವುದು. ಸುವರ್ಣ ನಿಯಮವು ಎಷ್ಟು ಅಯೋಗ್ಯವಾದುದು ಎಂದು ಟೀಕಿಸಿದರು ವಿವೇಕಾನಂದರು. ಯಾವಾಗಲೂ ನಾನು, ನಾನು ಎನ್ನುವುದೇ ಕ್ರೈಸ್ತಧರ್ಮ. ನಾನು ಇತರರಿಂದ ಏನನ್ನು ನಿರೀಕ್ಷಿಸುತ್ತೇನೆಯೋ ಅದರಂತೆಯೇ ಇತರರನ್ನು ನೋಡುವುದು-ಎಂಬುದು ಜಿಗುಪ್ಸಾಕಾರಕ, ಅನಾಗರಿಕ, ಮೃಗೀಯ ಭಾವನೆ. ಆದರೆ ಅವರು ಕ್ರೈಸ್ತಧರ್ಮವನ್ನು ತೆಗಳಲು ಬಯಸಲಿಲ್ಲ. ಏಕೆಂದರೆ ಯಾರು ಅದರಿಂದ ತೃಪ್ತರಾಗಿರುವರೊ ಅವರಿಗೆ ಅದು ಸರಿಯಾಗಿರುವುದು. ಮಹಾಪ್ರವಾಹ ಇದುವರೆಗೆ ಹರಿದ ದಿಕ್ಕಿನಲ್ಲಿ ಹರಿದುಕೊಂಡು ಹೋಗಲಿ. ಪ್ರಕೃತಿಯೇ ಸಮಸ್ಯೆಯನ್ನು ಬಗೆಹರಿಸುವಾಗ ವ್ಯಕ್ತಿಯು ಅದರ ದಿಕ್ಕನ್ನು ಬದಲಾಯಿಸುವುದಕ್ಕೆ ಯತ್ನಿಸುವುದು ಮೂರ್ಖತನ. ಆಧ್ಯಾತ್ಮಿಕವಾದಿ ಮತ್ತು ಅದೃಷ್ಟವಾದಿಗಳಾದ ವಿವೇಕಾನಂದರು ಎಲ್ಲಾ ಸರಿಯಾಗಿರುವುದೆಂದೂ, ತಾವು ಕ್ರೈಸ್ತರನ್ನು ಮತಾಂತರಗೊಳಿಸುವುದಕ್ಕೆ ಇಚ್ಛಿಸುವುದಿಲ್ಲವೆಂದೂ ಹೇಳಿದರು. ಅವರು ಕ್ರೈಸ್ತರು; ಅದು ಒಳ್ಳೆಯದೇ; ತಾವು ಹಿಂದೂಗಳು ಅದೂ ಒಳ್ಳೆಯದೆ. ಅವರ ದೇಶದಲ್ಲಿ ಜನರು ಬುದ್ಧಿಯ ಬೆಳವಣಿಗೆಗೆ ತಕ್ಕಂತೆ ವಿವಿಧ ಮತಗಳನ್ನು ರೂಪಿಸಿರುವರು. ಇವುಗಳೆಲ್ಲಾ ಆಧ್ಯಾತ್ಮಿಕ ವಿಕಾಸದ ವಿವಿಧ ಹಂತಗಳನ್ನು ಸೂಚಿಸುತ್ತವೆ. ಹಿಂದೂ ಧರ್ಮ ಅಹಂಕಾರದ ಮೇಲೆ ನಿಂತಿಲ್ಲ. ಎಂದಿಗೂ ತನ್ನ ಅಹಂಕಾರವನ್ನು ಅದು ಮೆರೆಸಲಿಲ್ಲ. ಎಂದಿಗೂ ಬಹುಮಾನ ಮತ್ತು ಶಿಕ್ಷೆಯನ್ನು ಜನರಿಗೆ ಅದು ತೋರಿಸಲಿಲ್ಲ. ನಿಃಸ್ವಾರ್ಥದಿಂದ ಒಬ್ಬನಿಗೆ ಮುಕ್ತಿ ದೊರಕುವುದು ಎಂಬುದನ್ನು ಅದು ತೋರಿಸಿತು. ಕ್ರೈಸ್ತಧರ್ಮ ಸಾಕ್ಷಾತ್​ ದೇವರಿಂದ ಬಂತೆಂದೂ ಪ್ರಪಂಚದಲ್ಲಿ ದೇವರು ಯಾರೋ ಕೆಲವರಿಗೆ ಕಾಣಿಸಿಕೊಂಡನೆಂದೂ ಹೇಳಿ ಲಂಚ ಕೊಟ್ಟು ಜನರನ್ನು ಕ್ರೈಸ್ತರನ್ನಾಗಿ ಪರಿವರ್ತಿಸುವುದು ಅತ್ಯಂತ ಹೀನಕೃತ್ಯ. ಇದು ಮನುಷ್ಯನನ್ನು ಬಹಳ ಅಧೋಗತಿಗೆ ಒಯ್ಯುವುದು. ಅವರು ಹೇಳುವಂತೆಯೇ ಕ್ರೈಸ್ತ ಮತತತ್ತ್ವಗಳನ್ನು ಶಬ್ದಾರ್ಥದಂತೆಯೇ ಒಪ್ಪಿಕೊಂಡರೆ, ಅದನ್ನು ಒಪ್ಪಿಕೊಳ್ಳುವ\break ಧರ್ಮಾಂಧರ ನೀತಿಯ ಮಟ್ಟ ಬಹಳ ಹೀನಸ್ಥಿತಿಗೆ ಬರುವುದು, ಪೂರ್ಣತೆಯನ್ನು ಪಡೆಯುವ ಕಾಲವನ್ನು ವಿಳಂಬ ಮಾಡುವುದು.

\delimiter

\begin{center}
(ಡೆಟ್ರಾಯಿಟ್​ ಟ್ರಿಬ್ಯೂನ್​, ಫೆಬ್ರವರಿ ೧೮, ೧೮೯೪)
\end{center}

ಸ್ವಾಮಿ ವಿವೇಕಾನಂದರು ಯೂನಿಟೇರಿಯನ್​ ಚರ್ಚ್​ನಲ್ಲಿ ನೆನ್ನೆ ರಾತ್ರಿ ವಿಧವೆಯರನ್ನು ಎಂದಿಗೂ ಇಂಡಿಯಾ ದೇಶದಲ್ಲಿ ಜೀವ ಸಹಿತ ಸುಡುವುದಿಲ್ಲವೆಂದೂ ಹಾಗೆ ಆದ ಕೆಲವು ಘಟನೆಗಳು ವಿಧವೆಯರೇ ತಾವು ಹಾಗೆ ಇಷ್ಟಪಟ್ಟು ಆದುದು ಎಂದೂ ಸಾರಿದರು. ಒಬ್ಬ ಚಕ್ರವರ್ತಿ ಈ ಆಚಾರವನ್ನು ನಿಷೇಧಿಸಿದ್ದನು. ಆದರೆ ಆ ಅಭ್ಯಾಸ ಹೇಗೋ ಪುನಃ ಆರಂಭವಾಯಿತು. ಈಗ ಇಂಗ್ಲೀಷ್​ ಸರ್ಕಾರ ಅದನ್ನು ನಿಲ್ಲಿಸಿದೆ. ಧರ್ಮಾಂಧರು. ಕ್ರೈಸ್ತರಲ್ಲಿಯೂ ಮತ್ತು ಎಲ್ಲಾ ಕಡೆಗಳಲ್ಲಿಯೂ ಇರುವರು. ಭಾರತದಲ್ಲಿ ತಪಸ್ಸಿಗಾಗಿ ತಮ್ಮ ಕೈಯನ್ನು ದಿನವೆಲ್ಲ ತಮ್ಮ ತಲೆಯ ಮೇಲೆ ಎತ್ತಿ ಹಿಡಿಯುತ್ತಿದ್ದ ಧರ್ಮಾಂಧರು ಕೆಲವರು ಇದ್ದರು. ಕೈ ಗಟ್ಟಿಯಾಗಿ ಆ ಸ್ಥಿತಿಯಲ್ಲಿಯೇ ಅನಂತರ ಇರುತ್ತಿತ್ತು. ಇದರಂತೆಯೇ ಒಂದೇ ಸ್ಥಿತಿಯಲ್ಲಿ ನಿಂತುಕೊಳ್ಳುವುದು ಒಂದು ಬಗೆಯ ತಪಸ್ಸು. ಇಂತಹ ಜನರು ಕಾಲಕ್ರಮೇಣ ತಮ್ಮ ಕೆಳಭಾಗದ ಅಂಗಗಳ ಮೇಲೆ ಹತೋಟಿಯನ್ನೆಲ್ಲ ಕಳೆದುಕೊಂಡವರಾಗಿ ನಂತರ ಅವರಿಗೆ ನಡೆಯುವುದಕ್ಕೆ ಆಗುತ್ತಿರಲಿಲ್ಲ. ಎಲ್ಲಾ ಧರ್ಮಗಳೂ ಸತ್ಯ. ದೇವರು ಅಪ್ಪಣೆ ಮಾಡಿರುವನು ಎಂದು ಜನರು ನೀತಿಯನ್ನು ಅನುಸರಿಸುವುದಿಲ್ಲ, ಹಾಗೆ ಇರುವುದು ಒಳ್ಳೆಯದು ಎಂದು ನೀತಿಯನ್ನು ಅನುಸರಿಸುವರು. ಹಿಂದೂಗಳು ಮತಾಂತರವನ್ನು ಒಪ್ಪುವುದಿಲ್ಲ, ಇದನ್ನು ಒಂದು ಅವಿವೇಕ ಎಂದು ಭಾವಿಸುವರು. ಎಂದು ವಿವೇಕಾನಂದರು ಹೇಳಿದರು. ಪರಸ್ಪರ ಸಂಬಂಧ, ವಾತಾವರಣ, ವಿದ್ಯೆ ಇವುಗಳೆಲ್ಲ ಹಲವು ಧರ್ಮಗಳು ಇರುವುದಕ್ಕೆ ಕಾರಣ. ಒಂದು ಧರ್ಮಕ್ಕೆ ಸೇರಿದವನು, ಮತ್ತೊಂದು ಧರ್ಮವನ್ನು ಸುಳ್ಳು ಎನ್ನುವುದು ದೊಡ್ಡ ಮೂರ್ಖತನ. ಏಷ್ಯಾ ದೇಶದವರು ಅಮೆರಿಕಾ ದೇಶಕ್ಕೆ ಬಂದು ಹರಿಯುತ್ತಿರುವ ಮಿಸಿಸಿಪ್ಪಿ ನದಿಯನ್ನು ನೋಡಿ “ನೀನು ತಪ್ಪು ದಾರಿಯಲ್ಲಿ ಹರಿಯುತ್ತಿರುವೆ. ನೀನು ಮೂಲಕ್ಕೆ ಹೋಗಿ ಸರಿಯಾದ ದಾರಿಯಲ್ಲಿ ಪುನಃ ಹರಿದುಕೊಂಡು ಬಾ” ಎಂದರೆ ಹೇಗೆ ಆಗುವುದೋ ಹಾಗೆಯೇ ಮತ್ತೊಬ್ಬರ ಧರ್ಮದಲ್ಲಿ ತಪ್ಪನ್ನು ಕಂಡುಹಿಡಿಯುವುದಾಗಿರುತ್ತದೆ. ಅಮೆರಿಕಾ ದೇಶದವನು ಆಲ್ಪ್ಸ್​ ಪರ್ವತಕ್ಕೆ ಹೋಗಿ, ಜರ್ಮನ್​ ಸಮುದ್ರಕ್ಕೆ ಹರಿಯುವ ನದಿಯನ್ನು ನೋಡಿ ಅದರ ಗತಿ ತುಂಬಾ ವಕ್ರವಾಗಿದೆ, ಅದು ನೇರವಾಗಿ ಹರಿಯಬೇಕು ಎನ್ನುವುದು ಎಷ್ಟು ಮೂರ್ಖತನವೊ ಹಾಗೆಯೇ ಇದೂ ಕೂಡ. ಸುವರ್ಣ ನಿಯಮ ಈ ಪ್ರಪಂಚದಷ್ಟೇ ಹಳೆಯದು, ನೀತಿಗೆ ಸಂಬಂಧಪಟ್ಟುದಕ್ಕೆಲ್ಲ ಮೂಲ ಇದೇ ಎಂದು ಅವರು ಸಾರಿದರು. ಮನುಷ್ಯ ಒಂದು ಸ್ವಾರ್ಥದ ಕಂತೆ. ನರಕದ ಬೆಂಕಿಯ ಸಿದ್ಧಾಂತವು ಅರ್ಥಹೀನ ಎಂದರು. ದುಃಖವಿದೆ ಎಂಬುದನ್ನು ತಿಳಿದುಕೊಂಡಾಗ ಎಂದಿಗೂ ಪೂರ್ಣವಾದ ಸುಖ ಇರಲಾರದು. ಕೆಲವು ಧರ್ಮಕ್ಕೆ ಸೇರಿದವರು ದೇವರನ್ನು ಪ್ರಾರ್ಥಿಸುವ ವಿಧಾನವನ್ನು ಅವರು ಟೀಕಿಸಿದರು. ಹಿಂದೂವು ಕಣ್ಣನ್ನು ಮುಚ್ಚಿಕೊಂಡು ಮನಸ್ಸನ್ನು ಅಂತರ್ಮುಖ ಮಾಡಿಕೊಳ್ಳುತ್ತಿದ್ದ. ಆದರೆ ತಾವು ನೋಡಿದ ಕೆಲವು ಕ್ರೈಸ್ತರಾದರೋ ಪ್ರಾರ್ಥಿಸುವಾಗ ಆಕಾಶದ ಕಡೆ, ದೇವರು ಸ್ವರ್ಗದಲ್ಲಿ ತನ್ನ ಸಿಂಹಾಸನದ ಮೇಲೆ ಕುಳಿತಿರುವನೇನೊ ಎಂಬಂತೆ, ಮೇಲೆ ನೋಡುತ್ತಿದ್ದರು ಎಂದರು. ಧರ್ಮದ ವಿಷಯದಲ್ಲಿ ಎರಡು ಅತಿರೇಕಗಳಿವೆ. ಒಬ್ಬನು ನಾಸ್ತಿಕ, ಮತ್ತೊಬ್ಬನು ಧರ್ಮಾಂಧ. ನಾಸ್ತಿಕನಲ್ಲಿ ಸ್ವಲ್ಪ ಒಳ್ಳೆಯದು ಇದೆ. ಆದರೆ ಧರ್ಮಾಂಧ\break ನಾದರೋ ಕೇವಲ ಸ್ವಾರ್ಥಿ. ಯಾರೊ ಅನಾಮಧೇಯ ವ್ಯಕ್ತಿಯೊಬ್ಬರು ತಮಗೆ ಕ್ರಿಸ್ತನ ಹೃದಯದ ಚಿತ್ರವನ್ನು ಕಳುಹಿಸಿದ್ದಕ್ಕೆ ಅವರು ಧನ್ಯವಾದವನ್ನು ಅರ್ಪಿಸಿದರು. ಆ ರೀತಿ ಕಳಿಸುವುದು ಮತಾಂಧತೆಯ ಅಭಿವ್ಯಕ್ತಿ ಎಂದು ಅವರು ಭಾವಿಸಿದರು. ಮತಭ್ರಾಂತರು ಯಾವ ಧರ್ಮಕ್ಕೂ ಸೇರಿದವರಲ್ಲ. ಅವರು ತಾವೇ ಒಂದು ಪ್ರತ್ಯೇಕ ಗುಂಪಿಗೆ ಸೇರಿದವರು.

\delimiter


\section[ಭಗವತ್ ಪ್ರೇಮ]{ಭಗವತ್ ಪ್ರೇಮ\protect\footnote{* C.W. Vol. III P. 503}}

\begin{center}
(ಡೆಟ್ರಾಯಿಟ್​ ಟ್ರಿಬ್ಯೂನ್​, ಫೆಬ್ರವರಿ ೨೧, ೧೮೯೪)
\end{center}

\vskip -0.3cm

ನೆನ್ನೆ ರಾತ್ರಿ ಸ್ವಾಮಿ ವಿವೇಕಾನಂದರ ಉಪನ್ಯಾಸವನ್ನು ಕೇಳುವುದಕ್ಕೆ ಮೊದಲನೆ ಯೂನಿಟೇರಿಯನ್​ ಚರ್ಚ್​ನಲ್ಲಿ ಜನರು ಕಿಕ್ಕಿರಿದು ನೆರೆದಿದ್ದರು. ಸಭಿಕರು ಜಫರ್​ ಸನ್​ ಎವಿನ್ಯೂ ಇಂದಲೂ ಮತ್ತು ವುಡ್​ವರ್ಡ್​ ಎವಿನ್ಯೂವಿನ ಮೇಲು ಭಾಗದಿಂದಲೂ ಬಂದಿ\break ದ್ದರು. ಸಭಿಕರಲ್ಲಿ ಬಹುಪಾಲು ಜನ ಮಹಿಳೆಯರು. ಉಪನ್ಯಾಸವನ್ನು ಅವರು ತುಂಬಾ ಉತ್ಸಾಹಭರಿತರಾಗಿ ಕೇಳುತ್ತಿದ್ದರು. ಬ್ರಾಹ್ಮಣ ವಿವೇಕಾನಂದರ ಮಾತುಗಳಿಗೆ ಕರತಾಡನವನ್ನು ಮಾಡುತ್ತಿದ್ದರು.

ಉಪನ್ಯಾಸಕರು ಹೇಳುತ್ತಿದ್ದ ಪ್ರೇಮವು ಕಾಮಕ್ಕೆ ಸಂಬಂಧಪಟ್ಟುದಲ್ಲ, ಅದು ಭರತಖಂಡದಲ್ಲಿ ಜನರು ದೇವರಿಗೆ ತೋರುವ ಪರಿಶುದ್ಧವಾದ ಪ್ರೀತಿ. ಉಪನ್ಯಾಸದ ಪ್ರಾರಂಭದಲ್ಲಿ “ದೇವರ ಮೇಲೆ ಭಾರತೀಯನಿಗೆ ಇರುವ ಪ್ರೇಮ” ಎಂಬುದು ಉಪನ್ಯಾಸದ ವಿಷಯ ಎಂದರು. ಆದರೆ ಅವರು ಆ ವಿಷಯದ ಮೇಲೆ ಮಾತನಾಡಲಿಲ್ಲ. ಉಪನ್ಯಾಸದ ಬಹುಪಾಲು ಕ್ರೈಸ್ತಧರ್ಮದ ಟೀಕೆಯಾಗಿತ್ತು. ಭಾರತೀಯನ ಧರ್ಮವಾಗಲೀ, ದೇವರ ಮೇಲೆ ಅವನಿಗೆ ಇರುವ ಪ್ರೀತಿಯಾಗಲಿ ಉಪನ್ಯಾಸದ ಪ್ರಧಾನ ವಿಷಯವಾಗಿರಲಿಲ್ಲ. ಉಪನ್ಯಾಸದಲ್ಲಿ ಹಲವು ವಿಷಯಗಳನ್ನು ಚಾರಿತ್ರಿಕ ಮಹಾವ್ಯಕ್ತಿಗಳ ನಿದರ್ಶನದ ಮೂಲಕ ವಿವರಿಸಿದರು. ಅವರು ಹೇಳಿದ ನಿದರ್ಶನಗಳು ಪ್ರಖ್ಯಾತ ಮೊಗಲ್​ ಚಕ್ರವರ್ತಿಗಳಿಗೆ ಸಂಬಂಧಪಟ್ಟಿದ್ದವು, ಹಿಂದೂ ರಾಜರಿಗೆ ಸೇರಿದುವಲ್ಲ.

ಧರ್ಮದಲ್ಲಿ ಎರಡು ಬಗೆಯ ಜನರು ಇರುವರು, ಮೊದಲನೆಯವರು ಜ್ಞಾನ ಮಾರ್ಗಾವಲಂಬಿಗಳು, ಎರಡನೆಯವರು ಭಕ್ತಿ ಮಾರ್ಗಾವಲಂಬಿಗಳು. ಜ್ಞಾನಮಾರ್ಗಿಗಳ ಉದ್ದೇಶ ಅನುಭಾವ, ಭಕ್ತಿ ಮಾರ್ಗಿಗಳ ಉದ್ದೇಶ ಪ್ರೇಮ. ಪ್ರೇಮ ಎಂದರೆ ತ್ಯಾಗ. ಅದು ಎಂದಿಗೂ ತೆಗೆದುಕೊಳ್ಳುವುದಿಲ್ಲ, ಯಾವಾಗಲೂ ಕೊಡುವುದು. ಹಿಂದೂಗಳು ದೇವರಿಂದ ಏನನ್ನೂ ಕೇಳುವುದಿಲ್ಲ, ಮುಕ್ತಿಗಾಗಿಯಾಗಲಿ, ಮುಂದಿನ ಸುಖ ಜೀವನಕ್ಕಾಗಿ ಆಗಲೀ ಅವರು ದೇವರನ್ನು ಪ್ರಾರ್ಥಿಸುವುದಿಲ್ಲ. ಅದರ ಬದಲು ಪರಮ ಪ್ರೇಮದಿಂದ ದೇವರಿಗೆ ತಮ್ಮ ಸರ್ವಸ್ವವನ್ನೂ ಅರ್ಪಿಸುವರು. ಭಗವಂತನಿಗಾಗಿ ತೀವ್ರವಾಗಿ ವ್ಯಾಕುಲಪಟ್ಟಾಗ ಮಾತ್ರ ಆ ಸುಂದರ ಪ್ರೇಮಸ್ಥಿತಿಯನ್ನು ಪಡೆಯಬಹುದು. ಆಗ ದೇವರು ಪೂರ್ಣವಾಗಿ ಭಕ್ತನೆಡೆಗೆ ಬರುವನು.

ದೇವರನ್ನು ಮೂರು ದೃಷ್ಟಿಯಿಂದ ನೋಡಬಹುದು. ಒಂದನೆಯದು ಅವನನ್ನು ಸರ್ವಶಕ್ತನಾದ ವ್ಯಕ್ತಿಯೆಂದು ತಿಳಿದುಕೊಂಡು ಅವನಿಗೆ ಅಡ್ಡ ಬಿದ್ದು ಪೂಜಿಸುವುದು. ಎರಡನೆಯದು ಅವನನ್ನು ತಂದೆಯೆಂದು ಪೂಜಿಸುವುದು. ಭರತಖಂಡದಲ್ಲಿ ತಂದೆ ಯಾವಾಗಲೂ ಮಕ್ಕಳನ್ನು ಶಿಕ್ಷಿಸುವನು, ತಂದೆಯನ್ನು ಪ್ರೀತಿಸುವುದರಲ್ಲಿ ಸ್ವಲ್ಪ ಅಂಜಿಕೆ ಇರುತ್ತದೆ. ಮತ್ತೊಂದು ವಿಧವೇ ದೇವರನ್ನು ತಾಯಿ ಎಂದು ಭಾವಿಸುವುದು. ಭರತಖಂಡದಲ್ಲಿ ತಾಯಿಯನ್ನು ನಿಜವಾಗಿ ಪ್ರೀತಿಸುತ್ತಾರೆ ಮತ್ತು ಗೌರವಿಸುತ್ತಾರೆ. ಭಾರತೀಯರು ದೇವರನ್ನು ನೋಡುವ ದೃಷ್ಟಿಯಿದು.

ನಿಜವಾದ ಭಕ್ತನು ಭಗವತ್​ ಪ್ರೇಮದಲ್ಲಿ ಉನ್ಮತ್ತನಾಗಿ ಹೋಗಿರುತ್ತಾನೆ; ಅವನಿಗೆ, ಬೇರೆ ಧರ್ಮದವರಿಗೆ, ‘ನೀವು ಹೋಗುತ್ತಿರುವ ಮಾರ್ಗ ಸರಿಯಲ್ಲ’ ಎಂದು ಹೇಳಿ ಅವರನ್ನು ತನ್ನ ಮಾರ್ಗಕ್ಕೆ ತರಲು ಸಮಯವೇ ಇರುವುದಿಲ್ಲ, ಎಂದು ವಿವೇಕಾನಂದರು ಹೇಳಿದರು.

\delimiter

\begin{center}
(ಡೆಟ್ರಾಯಿಟ್​ ಜರ್ನಲ್​)
\end{center}

\vskip -0.3cm

ಈ ಊರಿನಲ್ಲಿ ಉಪನ್ಯಾಸಗಳನ್ನು ಮಾಡುತ್ತಿರುವ ಬ್ರಾಹ್ಮಣ ಸಂನ್ಯಾಸಿ ವಿವೇಕಾನಂದರನ್ನು ಮತ್ತೊಂದು ವಾರ ಇರುವಂತೆ ಮಾಡಿದರೆ ಡೆಟ್ರಾಯಿಟ್​ನಲ್ಲಿರುವ ಅತಿ ದೊಡ್ಡ ಸಭಾಂಗಣದಲ್ಲಿಯೂ ಸ್ವಾಮಿಗಳ ಉಪನ್ಯಾಸವನ್ನು ಕೇಳಲು ಬರುವವರನ್ನು ಹಿಡಿಸುವುದು ಕಷ್ಟವಾಗುವುದು. ಜನರಿಗೆ ಅವರೊಂದು ಖಯಾಲಿ ಆಗಿರುವರು. ಕಳೆದ ಸಾಯಂಕಾಲ ಉಪನ್ಯಾಸಕ್ಕೆ ಹಾಲಿನ ಪ್ರತಿಯೊಂದು ಸ್ಥಳವೂ ಭರ್ತಿಯಾಗಿತ್ತು. ಅನೇಕರು ಉಪನ್ಯಾಸ ಮುಗಿಯುವವರೆಗೆ ನಿಂತುಕೊಂಡೇ ಕೇಳಿದರು.

ಉಪನ್ಯಾಸಕರು ‘ಭಗವತ್​ ಪ್ರೇಮ’ ಎಂಬ ವಿಷಯದ ಮೇಲೆ ಮಾತನಾಡಿದರು. ಅವರು ಭಕ್ತಿ ಎಂದರೆ ಇತ್ತ ವಿವರಣೆ ಇದು: “ಅಲ್ಲಿ ಸ್ವಾರ್ಥ ಎಳ್ಳಷ್ಟೂ ಇರುವುದಿಲ್ಲ. ನಮ್ಮ ಪ್ರೀತಿ ವಿಶ್ವಾಸಗಳಿಗೆ ಪಾತ್ರನಾದ ಭಗವಂತನನ್ನು ಸ್ತುತಿಸುವುದು ವಿನಃ ಮತ್ತೆ ಯಾವುದೂ ಅಲ್ಲಿ ಇರುವುದಿಲ್ಲ.” ಪ್ರೀತಿ ಎಂದರೆ ಭಗವಂತನಿಗೆ ಬಾಗಿ ನಮಸ್ಕರಿಸುವುದು, ಮತ್ತೆ ಅವನಿಂದ ಏನನ್ನೂ ಪ್ರತಿಫಲ ಕೇಳುವುದಲ್ಲ, ಎಂದರು. ಭಗವತ್​ ಪ್ರೀತಿ ಎಂದರೆ ಬೇರೆ. ನಮಗೆ ದೇವರು ಅತ್ಯಂತ ಆವಶ್ಯಕ ಎಂದು ನಾವು ಅವನನ್ನು ಸ್ವೀಕರಿಸುವುದಿಲ್ಲ, ಕೇವಲ ಸ್ವಾರ್ಥ ದೃಷ್ಟಿಯಿಂದ ನಾವು ಅವನ ಬಳಿಗೆ ಹೋಗುವುದಿಲ್ಲ. ಅವರ ಉಪನ್ಯಾಸದಲ್ಲಿ ಬೇಕಾದಷ್ಟು ಕಥೆಗಳು ಮತ್ತು ಉದಾಹರಣೆಗಳು ಇದ್ದುವು. ಅವುಗಳೆಲ್ಲ ಭಗವತ್​ ಪ್ರೇಮದಲ್ಲಿರುವ ಸ್ವಾರ್ಥವನ್ನು ತೋರುವುದಕ್ಕೆ ಆಗಿತ್ತು. ಕ್ರೈಸ್ತರ ಬೈಬಲ್ಲಿನಲ್ಲಿ ಸಾಲಮನ್ನನ ಪ್ರಾರ್ಥನೆಗಳು ಅತ್ಯಂತ ಸುಂದರವಾದ ಭಾಗ, ಅದನ್ನು ಬೈಬಲ್ಲಿನಿಂದ ತೆಗೆದು ಹಾಕುವರು ಎಂಬುದನ್ನು ಕೇಳಿ ತಮಗೆ ವಿಷಾದವಾಗಿದೆ ಎಂದು ಹೇಳಿದರು. ಕೊನೆಗೆ ನಿರ್ಣಾಯಕವಾಗಿ ಹೀಗೆ ಹೇಳಿದರು: “ಭಗವತ್ಪ್ರೀತಿಯು ‘ನನಗೆ ಅದರಿಂದ ಏನು ಲಾಭ’ ಎಂಬ ಸಿದ್ಧಾಂತದ ಮೇಲೆ ನಿಂತಿರುವಂತೆ ಕಾಣುತ್ತದೆ. ಕ್ರೈಸ್ತರ ಪ್ರೇಮ ಸ್ವಾರ್ಥದಿಂದ ಕೂಡಿದೆ. ಅವರು ಯಾವಾಗಲೂ ದೇವರಿಂದ ಏನಾದರೂ ಕೋರುತ್ತಲೇ ಇರುತ್ತಾರೆ. ಆಧುನಿಕ ಧರ್ಮ ಒಂದು ಹವ್ಯಾಸ ಮತ್ತು ಷೋಕಿ ಆಗಿದೆ. ಕುರಿಯ ಮಂದೆಯಂತೆ ಜನರು ಚರ್ಚಿಗೆ ಹೋಗುವರು.”


\section[ಏಷ್ಯಾಜ್ಯೋತಿಯ ಧರ್ಮವಾದ ಬೌದ್ಧ ಧರ್ಮ]{ಏಷ್ಯಾಜ್ಯೋತಿಯ ಧರ್ಮವಾದ ಬೌದ್ಧ ಧರ್ಮ\protect\footnote{* C.W. Vol. VII P. 429}}

\begin{center}
(ಡೆಟ್ರಾಯಿಟ್​ನಲ್ಲಿ ೧೮೯೪, ಮಾರ್ಚ್​ ೧೯ರಂದು ಮಾಡಿದ ಭಾಷಣದ ಡೆಟ್ರಾಯ್ಟ್​ ಟ್ರಿಬ್ಯೂನ್​, ಪತ್ರಿಕೆಯ ವರದಿ)
\end{center}

\vskip -0.3cm

ವಿವೇಕಾನಂದರು ಕಳೆದ ರಾತ್ರಿ ಅಡಿಟೋರಿಯ್​ನಲ್ಲಿ ಸುಮಾರು ನೂರ ಐವತ್ತು ಜನ ಶ್ರೋತೃಗಳನ್ನು ಉದ್ದೇಶಿಸಿ ಉಪನ್ಯಾಸವನ್ನು ಮಾಡಿದರು. (‘ಜರ್ನಲ್​’ ಪ್ರಕಾರ ಐನೂರು ಜನರು) ಉಪನ್ಯಾಸ “ಏಷ್ಯಾ ಜ್ಯೋತಿಯ ಧರ್ಮವಾದ ಬೌದ್ಧ ಧರ್ಮ” ಎಂಬುದು. ಮಿಸ್ಟರ್​ ಡಿಕನ್​ಸನ್​ ಅವರು ಉಪನ್ಯಾಸಕರ ಪರಿಚಯ ಮಾಡಿಕೊಡುತ್ತ, “ಈ ಧರ್ಮ ಒಳ್ಳೆಯದು ಆ ಧರ್ಮ ಕೆಟ್ಟದ್ದು ಎಂದು ಯಾರು ಹೇಳಬಲ್ಲರು, ಯಾರು ಇವುಗಳನ್ನು ಪ್ರತ್ಯೇಕಿಸಬಲ್ಲ ಗೆರೆಯನ್ನು ಹಾಕಬಲ್ಲರು?” ಎಂದರು.

ಭರತಖಂಡದ ಪೂರ್ವ ಧರ್ಮಗಳ ವಿಷಯವಾಗಿ ವಿವೇಕಾನಂದರು ದೀರ್ಘವಾಗಿ ಉಪನ್ಯಾಸ ಮಾಡಿದರು. ಯಜ್ಞಪೀಠದ ಮೇಲೆ ಪ್ರಾಣಿಗಳನ್ನು ಬಲಿಕೊಡುತ್ತಿದ್ದುದನ್ನು ಕುರಿತು ಹೇಳಿದರು. ಬುದ್ಧನ ಜನನ ಮತ್ತು ಜೀವನ; ಈ ಪ್ರಪಂಚ ಏತಕ್ಕೆ ಬಂತು, ನಾವೇತಕ್ಕೆ ಇಲ್ಲಿರುವೆವು ಎಂಬ ದಿಗ್ಭ್ರಮೆಗೊಳಿಸುವ ಪ್ರಶ್ನೆಗಳನ್ನು ಅವನು ತಾನೇ ಹಾಕಿಕೊಂಡುದ್ದು; ಈ ಸೃಷ್ಟಿ ಮತ್ತು ಜೀವನ ಇವುಗಳ ರಹಸ್ಯವನ್ನು ಕಂಡುಹಿಡಿಯುವುದಕ್ಕೆ ಅವನು ಮಾಡಿದ ಪ್ರಯತ್ನಗಳು ಮತ್ತು ಕೊನೆಯಲ್ಲಿ ಅವನು ಕಂಡುಹಿಡಿದ ಪರಿಹಾರೋಪಾಯಗಳು-ಇವುಗಳ ವಿಷಯವನ್ನು ಅವರು ಹೇಳಿದರು.

ಬುದ್ಧನು ಎಲ್ಲರನ್ನೂ ಮೀರಿ ನಿಂತಿದ್ದನು. ಇವನೊಬ್ಬನ ವಿಷಯದಲ್ಲಿ ಮಾತ್ರ ಅವನ ಸ್ನೇಹಿತರಾಗಲೀ, ವೈರಿಗಳಾಗಲೀ, ಎಲ್ಲರೂ, ಅವನು ಜಗದ ಕಲ್ಯಾಣಕ್ಕಾಗಿ ಅಲ್ಲದೆ ಬೇರಾವ ಉದ್ದೇಶಕ್ಕಾಗಲೀ ಬದುಕಿರಲಿಲ್ಲ, ಎನ್ನುವರು.

ವಿವೇಕಾನಂದರು ಹೇಳಿದರು: “ಅವನು ಜೀವದ ಪುನರ್ಜನ್ಮದ ವಿಷಯವನ್ನು\break ಬೋಧಿಸಲಿಲ್ಲ. ಒಂದು ಅಲೆ ಮತ್ತೊಂದು ಅಲೆಗೆ ಹೇಗೆ ಕಾರಣವೋ ಹಾಗೆ ಒಂದು ಜನ್ಮ ಮತ್ತೊಂದು ಜನ್ಮಕ್ಕೆ ಕಾರಣ. ಮುಂದಿನ ಅಲೆಗೆ ತನ್ನ ವೇಗವನ್ನು ಮಾತ್ರ ರವಾನಿಸಿ ಹಿಂದಿನ ಅಲೆ ನಾಶವಾಗುವುದು. ಅವನು ದೇವರು ಇರುವನು ಎಂದೂ ಬೋಧಿಸಲಿಲ್ಲ ಅಥವಾ ಅವನು ದೇವರು ಇಲ್ಲ ಎಂತಲೂ ಹೇಳಲಿಲ್ಲ.”

“ ‘ನಾವೇತಕ್ಕೆ ಒಳ್ಳೆಯವರಾಗಿರಬೇಕು?’” ಎಂದು ಅವನ ಶಿಷ್ಯರು ಕೇಳಿದರು.

“ ‘ಏಕೆಂದರೆ ನೀವು ಒಳ್ಳೆಯದನ್ನು ಹಿಂದಿನವರಿಂದ ಪಡೆದಿರುವಿರಿ. ನೀವು ಕೂಡ ನಿಮ್ಮ ಮುಂದೆ ಬರುವವರಿಗೆ ಒಳ್ಳೆಯತನದ ಆಸ್ತಿಯನ್ನು ಬಿಟ್ಟು ಹೋಗಿ. ನಾವೆಲ್ಲರೂ ಒಳ್ಳೆಯದಕ್ಕಾಗಿಯೇ ಒಳ್ಳೆಯದನ್ನು ಮಾಡುವ ಆದರ್ಶದ ವ್ಯಾಪ್ತಿಗೆ ಸಹಾಯ ಮಾಡೋಣ.’

“ಬುದ್ಧನು ಮೊದಲನೇ ಪ್ರವಾದಿಯಾಗಿದ್ದನು. ಅವನು ಯಾರನ್ನೂ ನಿಂದಿಸಲಿಲ್ಲ, ಅಥವಾ ತನ್ನ ವಿಷಯದಲ್ಲಿ ಪ್ರಶಂಸೆಯನ್ನು ಮಾಡಿಕೊಳ್ಳಲಿಲ್ಲ. ಆಧ್ಯಾತ್ಮಿಕ ಪ್ರಪಂಚದಲ್ಲಿ ನಮ್ಮ ಗುರಿಯನ್ನು ನಾವೇ ಸೇರಬೇಕಾಗಿದೆ ಎಂಬುದನ್ನು ಅವನು ನಂಬಿದ್ದನು.

“ಬುದ್ಧನು ಮರಣೋನ್ಮುಖನಾಗಿರುವಾಗ ಹೀಗೆ ಹೇಳಿದನು: ‘ನಾನು ನಿಮಗೆ (ಪರಮ ಗುರಿಯ ವಿಷಯವನ್ನು) ಹೇಳಲಾರೆ, ಅಥವಾ ಇನ್ನು ಯಾರಾದರೂ ಈ ವಿಷಯವನ್ನು ಹೇಳಲಾರರು. ನೀವು ಯಾರ ಆಸರೆಯ ಮೇಲೆಯೂ ನಿಲ್ಲಬೇಡಿ. ನಿಮ್ಮ ನಿರ್ವಾಣಕ್ಕೆ ನೀವೇ ಪ್ರಯತ್ನ ಮಾಡಬೇಕಾಗಿದೆ.’

“ಮನುಷ್ಯ ಮನುಷ್ಯರಲ್ಲಿ, ಮನುಷ್ಯ ಮತ್ತು ಪ್ರಾಣಿಗಳಲ್ಲಿ, ಒಬ್ಬರು ಮೇಲು, ಮತ್ತೊಬ್ಬರು ಕೀಳು ಎಂಬ ಭಾವನೆಗಳನ್ನು ಅವನು ಖಂಡಿಸಿದನು. ಸುರಾಪಾನಕ್ಕೆ ವಿರೋಧವಾಗಿ ನಿಂತವರಲ್ಲಿ ಅವನೇ ಮೊದಲಿಗನು. ‘ಒಳ್ಳೆಯವರಾಗಿ ಒಳ್ಳೆಯದನ್ನು ಮಾಡಿ’, ಎಂದು ಅವನು ಬೋಧಿಸಿದನು. ‘ಯಾರಾದರೂ ಒಬ್ಬ ದೇವರಿದ್ದರೆ, ನೀವು ಒಳ್ಳೆಯವರಾಗಿದ್ದರೆ ಅವನು ನಿಮಗೆ ದೊರಕುವನು. ಒಂದು ವೇಳೆ ದೇವರಿಲ್ಲದೆ ಇದ್ದರೆ ಒಳ್ಳೆಯವರಾಗಿರುವುದು ಒಳ್ಳೆಯದೇ. ಮನುಷ್ಯನೇ ತನ್ನ ದುಃಖಕ್ಕೆಲ್ಲ ಕಾರಣ. ಅವನ ಒಳ್ಳೆಯದಕ್ಕೆಲ್ಲ ಅವನೇ ಕಾರಣಕರ್ತನು’ ಎಂದು ಹೇಳಿದನು.

“ಮಿಷನರಿ ಪದ್ಧತಿಯನ್ನು ತಂದವರಲ್ಲಿ ಬುದ್ಧನು ಮೊತ್ತ ಮೊದಲಿಗನು, ಭರತ ಖಂಡದಲ್ಲಿ ದಬ್ಬಾಳಿಕೆಗೆ ತುತ್ತಾದ ಲಕ್ಷಾಂತರ ಜನರಿಗೆ ಉದ್ಧಾರಕನಂತೆ ಅವನು ಬಂದನು. ಅವರಿಗೇ ಬುದ್ಧನ ತತ್ತ್ವ ಅರ್ಥವಾಗಲಿಲ್ಲ. ಆದರೆ ಅವನ ಜೀವನ ಮತ್ತು ಬೋಧನೆಯನ್ನು ಕಂಡರು, ಅವನನ್ನು ಅನುಸರಿಸಿದರು.”

ಮುಕ್ತಾಯದಲ್ಲಿ ಬೌದ್ಧಧರ್ಮವೇ ಕ್ರೈಸ್ತಧರ್ಮಕ್ಕೆ ತಳಹದಿ ಎಂದರು. ಕ್ಯಾಥೋಲಿಕ್​ ಚರ್ಚ್​ ಬೌದ್ಧಧರ್ಮದಿಂದ ಬಂದಿತು ಎಂದು ವಿವೇಕಾನಂದರು ಹೇಳಿದರು.

\delimiter


\section[ಭಾರತ ಮಹಿಳೆಯರು]{ಭಾರತ ಮಹಿಳೆಯರು\protect\footnote{* C.W. Vol. III P. 305}}

\begin{center}
(ಡೆಟ್ರಾಯಿಟ್​ ಫ್ರೀಪ್ರೆಸ್​, ಮಾರ್ಚ್​ ೨೫, ೧೮೯೪)
\end{center}

\vskip -0.4cm

ವಿವೇಕಾನಂದರು ನಿನ್ನೆಯ ರಾತ್ರಿ ಯೂನಿಟೇರಿಯನ್​ ಚರ್ಚ್​ನಲ್ಲಿ ‘ಭಾರತ ಮಹಿಳೆಯರು’ ಎಂಬ ವಿಷಯದ ಮೇಲೆ ಉಪನ್ಯಾಸ ಮಾಡಿದರು. ಪ್ರಾಚೀನ ಭರತಖಂಡದ ಸ್ತ್ರೀಯರ ವಿಷಯವಾಗಿ ಮಾತನಾಡುತ್ತಾ, ಶಾಸ್ತ್ರಗಳಲ್ಲಿ ಸ್ತ್ರೀಯರಿಗೆ ನೀಡಿರುವ ಗೌರವವನ್ನು ವಿವರಿಸಿದರು. ಸ್ತ್ರೀಯರಲ್ಲಿ ಕೂಡ ಮಹಾಸಾಧ್ವಿಯರು ಇದ್ದರು. ಅವರಲ್ಲಿದ್ದ ಆಧ್ಯಾತ್ಮಿಕತೆ ಶ್ಲಾಘನೀಯವಾಗಿತ್ತು. ಪೌರಸ್ತ್ಯ ನಾರಿಯರನ್ನು ಪಾಶ್ಚಾತ್ಯ ಆದರ್ಶದ ದೃಷ್ಟಿಯಿಂದ ಅಳೆಯುವುದು ಸರಿಯಲ್ಲ ಎಂದರು. ಪಾಶ್ಚಾತ್ಯ ದೇಶದಲ್ಲಿ ಸ್ತ್ರೀ ಹೆಂಡತಿ, ಪೌರಾತ್ಯ ದೇಶದಲ್ಲಿ ಸ್ತ್ರೀ ತಾಯಿ. ಹಿಂದೂಗಳು ಮಾತೃ ಭಾವವನ್ನು ಪೂಜಿಸುತ್ತಾರೆ. ಸಂನ್ಯಾಸಿಗಳು ಕೂಡ ತಾಯಿಗೆ ದಂಡವತ್​ ನಮಸ್ಕಾರ ಮಾಡಬೇಕು. ಸ್ತ್ರೀಯರು ಪಾತಿವ್ರತ್ಯವನ್ನು ಬಹಳವಾಗಿ ಗೌರವಿಸುವರು. ವಿವೇಕಾನಂದರು ಇಂದು ಮಾಡಿದ ಉಪನ್ಯಾಸ ಅತ್ಯಂತ ಆಸಕ್ತಿಪೂರ್ಣವಾಗಿತ್ತು. ಅವರನ್ನು ಬಹಳ ಆದರದಿಂದ ಬರಮಾಡಿಕೊಳ್ಳಲಾಯಿತು.

\delimiter

\begin{center}
(ಡೆಟ್ರಾಯಿಟ್​ ಈವ್​ನಿಂಗ್​ ನ್ಯೂಸ್​, ಮಾರ್ಚ್​ ೨೫, ೧೮೯೪)
\end{center}

\vskip -0.3cm

ಸ್ವಾಮಿ ವಿವೇಕಾನಂದರು ಕಳೆದ ರಾತ್ರಿ ಯೂನಿಟೇರಿಯನ್​ ಚರ್ಚ್​ನಲ್ಲಿ “ಮಹಿಳೆ\break ಯರು-ಹಿಂದಿನ ಕಾಲದಲ್ಲಿ, ಮಧ್ಯಯುಗದಲ್ಲಿ ಮತ್ತು ಈಗ” ಎನ್ನುವ ವಿಷಯದ ಮೇಲೆ ಮಾತನಾಡಿದರು. ಭರತಖಂಡದಲ್ಲಿ ಸ್ತ್ರೀಯರನ್ನು ಪ್ರತ್ಯಕ್ಷ ದೇವತೆ ಎಂದು ಜನರು ಭಾವಿಸುವರು. ಅವಳು ತನ್ನ ಇಡೀ ಜೀವನವನ್ನು ತಾಯಿತನಕ್ಕೆ ಅರ್ಪಿಸುವಳು. ಆದರ್ಶ ತಾಯಿಯಾಗಬೇಕಾದರೆ ಪತಿವ್ರತೆಯಾಗಿರಬೇಕು. ಭರತಖಂಡದಲ್ಲಿ ಯಾವ ತಾಯಿಯೂ ಮಕ್ಕಳನ್ನು ತ್ಯಜಿಸುವುದಿಲ್ಲವೆಂದರು. ಇದಕ್ಕೆ ವಿರೋಧವಾಗಿದ್ದರೆ ತೋರಿಸಿ ಎಂದರು. ಭಾರತದ ಹುಡುಗಿಯರು, ಅಮೆರಿಕಾದ ಹುಡುಗಿಯರಂತೆ, ತಮ್ಮ ದೇಹದ ಅರ್ಧ ಭಾಗವನ್ನು ಪರಪುರುಷರು ನೋಡುವಂತೆ ಮಾಡಿಕೊಳ್ಳಬೇಕಾಗಿ ಬಂದರೆ, ಅವರು ಬೇಕಾದರೆ ತಮ್ಮ ಪ್ರಾಣವನ್ನಾದರೂ ಬಿಟ್ಟಾರು; ಎಂದಿಗೂ ಹಾಗೆ ಮಾಡಲಾರರು-ಎಂದರು. ಇಂಡಿಯಾ ದೇಶವನ್ನು ಆ ದೇಶದ ಆದರ್ಶ ದೃಷ್ಟಿಯಿಂದ ನಾವು ನೋಡಬೇಕೇ ಹೊರತು, ನಿಮ್ಮ ದೇಶದ ಆದರ್ಶದ ದೃಷ್ಟಿಯಿಂದ ಅಲ್ಲ-ಎಂದರು.

\delimiter

\begin{center}
(ಟ್ರಿಬ್ಯೂನ್​, ಏಪ್ರಿಲ್​ ೧, ೧೮೯೪)
\end{center}

\vskip -0.3cm

ಸ್ವಾಮಿ ವಿವೇಕಾನಂದರು ಡೆಟ್ರಾಯಿಟ್​ನಲ್ಲಿ ಇದ್ದಾಗ ಅವರು ಹಲವರೊಡನೆ\break ಸಂಭಾಷಣೆ ಮಾಡಿದರು. ಆಗ ಭಾರತೀಯ ನಾರಿಯರ ವಿಷಯದಲ್ಲಿ ಹಾಕಿದ ಪ್ರಶ್ನೆಗಳಿಗೆ ಉತ್ತರ ಕೊಟ್ಟರು. ಅವರು ಈ ಸಮಯದಲ್ಲಿ ನೀಡಿದ ಮಾಹಿತಿಗಳಿಂದಾಗಿ ಈ ವಿಷಯದ ಮೇಲೆ ಒಂದು ಬಹಿರಂಗ ಉಪನ್ಯಾಸವನ್ನು ಮಾಡ ಬೇಕೆಂದು ಜನರು ಕೋರಿಕೊಂಡರು. ಆದರೆ ಅವರು ಯಾವ ಟಿಪ್ಪಣಿಯೂ ಇಲ್ಲದೆ ಮಾತನಾಡುವುದರಿಂದ ಸಂಭಾಷಣೆಯ ಸಮಯದಲ್ಲಿ ಅವರು ಹೇಳಿದ ಕೆಲವಂಶಗಳು ಬಹಿರಂಗ ಉಪನ್ಯಾಸದಲ್ಲಿ ಇರಲಿಲ್ಲ. ಅವರ ಸ್ನೇಹಿತರಿಗೆ ಆಗ ಸ್ವಲ್ಪ ನಿರಾಸೆ ಆಯಿತು. ಸ್ವಾಮೀಜಿ ಅವರು ಒಂದು ದಿನ ಮಧ್ಯಾಹ್ನದ ಸಂಭಾಷಣೆಯಲ್ಲಿ ಹೇಳಿದ ಕೆಲವು ವಿಷಯಗಳನ್ನು ಮಹಿಳೆಯೊಬ್ಬಳು ಬರೆದಿಟ್ಟರುವರು. ಅದು ಈಗ ಪ್ರಥಮ ಬಾರಿ ಪ್ರಕಟವಾಗುತ್ತಿದೆ:

ಉನ್ನತ ಹಿಮಾಲಯ ಪರ್ವತಗಳಿರುವ ಪ್ರಸ್ಥಭೂಮಿಗೆ ಆರ್ಯರು ಮೊದಲು\break ಬಂದರು. ಅಲ್ಲಿ ಇಂದಿಗೂ ಶುದ್ಧವಾದ ಬ್ರಾಹ್ಮಣರು ಇರುವರು. ಪಾಶ್ಚಾತ್ಯರಿಗೆ ಇದನ್ನು ಕೇವಲ ಕಲ್ಪಿಸಿಕೊಳ್ಳುವುದಕ್ಕೆ ಮಾತ್ರ ಸಾಧ್ಯ. ಅವರು ಮನೋವಾಕ್ಕಾಯವಾಗಿ ಪರಿಶುದ್ಧ\break ವಾಗಿರುವರು. ಅವರು ಎಷ್ಟು ಸತ್ಯಸಂಧರಾಗಿರುವರು ಎಂದರೆ ಒಂದು ಚಿನ್ನದ ನಾಣ್ಯ ಬೀದಿಯಲ್ಲಿ ಬಿದ್ದಿದ್ದರೆ ಇಪ್ಪತ್ತು ವರ್ಷಗಳಾದರೂ ಅದು ಅಲ್ಲಿಯೇ ಬಿದ್ದಿರುತ್ತದೆ. ಅವರು ನೋಡಲು ಎಷ್ಟು ಸುಂದರವಾಗಿರುವರು ಎಂಬುದನ್ನು ವಿವೇಕಾನಂದರ ಮಾತಿನಲ್ಲಿಯೇ ಹೇಳೋಣ: “ಹೊಲದಲ್ಲಿ ಒಬ್ಬ ಹುಡುಗಿಯನ್ನು ನೋಡಿದರೆ, ಇಷ್ಟು ಸುಂದರವಾದ ವ್ಯಕ್ತಿಯನ್ನು ಭಗವಂತ ಸೃಷ್ಟಿಸಿರುವನೇ ಎಂದು ಆಶ್ಚರ್ಯ ಪಡಬೇಕು!” ಅವರ ಆಕಾರ ಚೆನ್ನಾಗಿದೆ, ಕಣ್ಣು ಮತ್ತು ತಲೆ ಕಪ್ಪಾಗಿವೆ. ಅವರ ದೇಹದ ಬಣ್ಣವು ಬೆರಳನ್ನು ಚುಚ್ಚಿದಾಗ ರಕ್ತದ ಹನಿ ಹಾಲಿನ ಗ್ಲಾಸಿನಲ್ಲಿ ಬಿದ್ದರೆ ಹಾಲು ಹೇಗೆ ಕಾಣುವುದೊ ಹಾಗೆ ಇತ್ತು. ಇವರೇ ಮಿಶ್ರವಾಗದ ಕೊಳೆಯಾಗದ ಪವಿತ್ರ ಹಿಂದೂಗಳು.

ಅವರ ಆಸ್ತಿಯ ಕಾನೂನಿನ ಪ್ರಕಾರ ಪತ್ನಿಯ ವರದಕ್ಷಿಣೆ ಅವಳಿಗೆ ಮಾತ್ರ ಸೇರಿದ್ದು. ಅದು ಎಂದಿಗೂ ಗಂಡನದು ಆಗಲಾರದು. ಅವಳು ಅದನ್ನು ಗಂಡನ ಅಪ್ಪಣೆಯಿಲ್ಲದೆ ಇತರರಿಗೆ ಕೊಡಬಹುದು, ಇಲ್ಲವೇ ಮಾರಬಹುದು. ಅವಳಿಗೆ ಯಾವ ಬಹುಮಾನ ಬರಲಿ, ಅದನ್ನು ಅವಳ ಗಂಡನೇ ಕೊಟ್ಟಿದ್ದರೂ, ಅದು ಅವಳಿಗೆ ಮಾತ್ರ ಸೇರಿದ್ದು; ತನಗೆ ತೋಚಿದಂತೆ ಅದನ್ನು ವಿನಿಯೋಗಿಸಬಹುದು.

ಸ್ತ್ರೀಯರು ಅಂಜಿಕೆ ಇಲ್ಲದೆ ಹೊರಗೆ ಓಡಾಡುವರು. ಅವಳು ಮುಕ್ತವಾಗಿ ಓಡಾಡು\break ವಷ್ಟರ ಮಟ್ಟಿಗೆ ಅವಳಿಗೆ ಸುತ್ತಮುತ್ತಲಿನವರ ಮೇಲೆ ಭರವಸೆಯಿರುತ್ತದೆ. ಹಿಮಾಲಯದಲ್ಲಿ ಹೆಂಗಸರಿಗೆ ಯಾವ ಜನಾನವು ಇಲ್ಲ. ಭಾರತದಲ್ಲಿ ಮಿಷನರಿಗಳು ಹೋಗಲಾಗದ ಒಂದು ಭಾಗವಿದೆ. ಅಲ್ಲಿಗೆ ಹೋಗಬೇಕಾದರೆ ಬಹಳ ಕಷ್ಟಪಟ್ಟು ಬೆಟ್ಟ ಗುಡ್ಡಗಳನ್ನು ದಾಟಿಕೊಂಡು ಹೋಗಬೇಕು. ಆಗ ಮಾತ್ರ ಅವರನ್ನು ನೋಡಲು ಸಾಧ್ಯ. ಮಹಮ್ಮದೀಯರಿಗೂ ಕ್ರೈಸ್ತರಿಗೂ ಇವರ ಪರಿಚಯವಿಲ್ಲ.

\delimiter


\section{ಇಂಡಿಯಾ ದೇಶದ ಆದಿವಾಸಿಗಳು}

ಇಂಡಿಯಾ ದೇಶದ ಕಾಡಿನಲ್ಲಿ ಹಲವು ಅನಾಗರೀಕ ಕಾಡು ಜನರು ಇರುವರು. ಅವರು ಬಹಳ ಅನಾಗರಿಕರು. ನರಭಕ್ಷಕರಿಗಿಂತಲೂ ಅನಾಗರಿಕರು. ಅವರೇ ಭಾರತದ ಆದಿವಾಸಿ\break ಗಳು. ಅವರು ಆರ್ಯರೂ ಅಲ್ಲ, ಹಿಂದೂಗಳೂ ಅಲ್ಲ.

ಹಿಂದೂಗಳು ದೇಶದಲ್ಲಿ ನೆಲೆಸಿ ವಿಸ್ತಾರವಾಗಿ ಹಬ್ಬಿದ ಮೇಲೆ, ಅವರಲ್ಲಿ ಹಲವು ಬಗೆಯ ಮಿಶ್ರಣವನ್ನು ನೋಡುತ್ತೇವೆ. ಸೂರ್ಯ ಧಗಧಗಿಸುತ್ತಿದ್ದು ಅದರ ಕೆಳಗೆ ಇದ್ದವರ ಮೈಬಣ್ಣ ಕಪ್ಪಾಯಿತು.

ಹಿಮಾಲಯ ಪ್ರಾಂತ್ಯಗಳಲ್ಲಿರುವ ಶುದ್ಧ ಶ್ವೇತ ವರ್ಣದ ಜನರು ಕಂದು ಬಣ್ಣದ ಭಾರತೀಯ ಹಿಂದೂಗಳಾಗಬೇಕಾದರೆ ಐದು ತಲೆಮಾರುಗಳು ಬೇಕಾಗುತ್ತದೆ.

ವಿವೇಕಾನಂದರಿಗೆ ಅವರಿಗಿಂತ ಬಿಳುಪಾಗಿರುವ ಸಹೋದರರು, ಅವರಿಗಿಂತ ಕಪ್ಪಾಗಿ\break ರುವ ಮತ್ತೊಬ್ಬ ಸಹೋದರರೂ ಇರುವರು. ಅವರ ತಂದೆ ತಾಯಿಗಳು ಬಿಳುಪಾಗಿರುವರು. ಸ್ತ್ರೀಯರನ್ನು ಮಹಮ್ಮದೀಯರ ಅಂಜಿಕೆಯಿಂದ ಜನಾನದಲ್ಲಿ ಇಟ್ಟಿರುವುದರಿಂದ ಅವರು ಪುರುಷರಿಗಿಂತ ಬಿಳುಪಾಗಿರುವರು. ವಿವೇಕಾನಂದರಿಗೆ ಮೂವತ್ತೊಂದು ವಯಸ್ಸು.

\delimiter


\section{ಅಮೆರಿಕಾ ದೇಶದ ಪುರುಷರ ಅಣಕ}

\vskip -0.15cm

ವಿವೇಕಾನಂದರು ಕಿರುನಗೆಯನ್ನು ಬೀರುತ್ತ ಕಣ್ಣನ್ನು ಹೊರಳಿಸಿ, ಅಮೆರಿಕಾ ದೇಶದ ಪುರುಷರನ್ನು ಕಂಡರೆ ತಮಗೆ ತಮಾಷೆ ಆಗುವುದು ಎಂದರು. ಅವರು ಸ್ತ್ರೀಯರನ್ನು ಪೂಜಿಸುತ್ತೇವೆ ಎನ್ನುವರು. ಅದರೆ ವಿವೇಕಾನಂದರ ದೃಷ್ಟಿಯಲ್ಲಿ ಅಮೆರಿಕಾ ಪುರುಷರು ಪೂಜಿಸುವುದು ಸ್ತ್ರೀಯರಲ್ಲಿರುವ ಯೌವನ ಮತ್ತು ಸೌಂದರ್ಯವನ್ನು. ಅವರೆಂದಿಗೂ ಸುಕ್ಕಿಹೋದ ಮುಖ ಅಥವಾ ನೆರೆತ ಕೂದಲನ್ನು ಪ್ರೀತಿಸುವುದಿಲ್ಲ. ಅಮೆರಿಕಾದ ಪುರುಷರಿಗೆ ಹಿಂದೆ ಒಂದು ತಂತ್ರ ಗೊತ್ತಿತ್ತು-ಅದು ಆನುವಂಶಿಕವಾಗಿ ಬಂದಿರಬಹುದು-ಅದೇನೆಂದರೆ ವೃದ್ಧ ಮಹಿಳೆಯರನ್ನು ಸುಟ್ಟು ಹಾಕುವುದು, ಎಂದು ವಿವೇಕಾನಂದರು ನಂಬುತ್ತಾರೆ. ಆಧುನಿಕ ಚರಿತ್ರೆ ಇದನ್ನು ಮಂತ್ರವಾದಿನಿಯರನ್ನು ಸುಡುವುದು ಎಂದು ಹೇಳುವುದು. ಪುರುಷರು ಮಂತ್ರವಾದಿಗಳಲ್ಲಿ ತಪ್ಪು ಕಂಡುಹಿಡಿದು ಅವರಿಗೆ ಶಿಕ್ಷೆಯನ್ನು ಕೊಟ್ಟರು. ಆ ಸ್ತ್ರೀಯರ ವೃದ್ಧಾಪ್ಯವೇ ಆ ಶಿಕ್ಷೆಗೆ ಕಾರಣವಾಗಿರಬೇಕು. ಸಜೀವವಾಗಿ ಸ್ತ್ರೀಯರನ್ನು ದಹಿಸುವುದು ಬರೀ ಹಿಂದೂಗಳಲ್ಲಿ ಮಾತ್ರ ಒಂದು ವಾಡಿಕೆಯಲ್ಲ ಎಂಬುದು ಅದರಿಂದ ಗೊತ್ತಾಗುವುದು. ಕ್ರೈಸ್ತ ಚರ್ಚ್​ನವರು ವೃದ್ಧನಾರಿಯರನ್ನು ಜೀವ ಸಹಿತ ಬೆಂಕಿಯಲ್ಲಿ ಸುಟ್ಟಿರುವು\break ದನ್ನು ನೆನಸಿಕೊಂಡರೆ, ಹಿಂದೂ ವಿಧವೆಯರನ್ನು ಸುಡುವುದು ಅಷ್ಟು ಭಯಾನಕವಾಗೇನೂ ತೋರುವುದಿಲ್ಲ.


\section{ಸುಡುವುದನ್ನು ಹೋಲಿಸಿರುವುದು}

\vskip -0.15cm

ಹಿಂದೂ ವಿಧವೆಯು ತನ್ನ ಮರಣ ಸಂಕಟದ ಸಮಯದಲ್ಲಿ ಔತಣ ಗಾನಗಳಲ್ಲಿ ತೊಡಗಿರು\break ತ್ತಾಳೆ. ಸಾಯುವಾಗ ತನ್ನ ಬೆಲೆಬಾಳುವ ಆಭರಣ ವಸ್ತುಗಳನ್ನೆಲ್ಲಾ ಧರಿಸಿಕೊಂಡು ಇರುತ್ತಿ\break ದ್ದಳು. ಹೀಗೆ ಮಾಡುವುದು ತನಗೆ ಮತ್ತು ತನ್ನ ವಂಶಕ್ಕೆ ಶ್ರೇಯಸ್ಕರವೆಂದು ಅವಳು ಭಾವಿಸಿದ್ದಳು. ಅವಳನ್ನು ಧರ್ಮವೀರಳೆಂದು ಕರೆಯುತ್ತಿದ್ದರು. ವಂಶಾವಳಿಯಲ್ಲಿ ಅವಳ\break ಹೆಸರನ್ನು ಬರೆದು ಇಡುತ್ತಿದ್ದರು.

ಈ ಆಚಾರ ಎಷ್ಟೇ ಭಯಾನಕವಾಗಿದ್ದರೂ ಕ್ರೈಸ್ತರು ಮಂತ್ರವಾದಿನಿಯರನ್ನು ಸುಡು\break ವುದರೊಂದಿಗೆ ಹೋಲಿಸಿದರೆ ಇದು ಮೇಲಾಗಿ ತೋರುವುದು. ಮೊದಲಿನಿಂದಲೇ ಆ ಮಂತ್ರವಾದಿನಿಯನ್ನು ಪಾಪಿ ಎಂದು ಪರಿಗಣಿಸಿ, ಕೊಳೆತು ನಾರುತ್ತಿರುವ ಕೋಣೆಯಲ್ಲಿ ಅವಳನ್ನು ಕೂಡಿಹಾಕಿ, ಅವಳಿಂದ ತನ್ನ ತಪ್ಪಿನ ಒಪ್ಪಿಗೆಯನ್ನು ಪಡೆಯುವುದಕ್ಕೆ ಅಸಾಧ್ಯವಾದ ಯಾತನೆಯನ್ನು ಅವಳಿಗೆ ಕೊಡುತ್ತಿದ್ದರು. ಅಯೋಗ್ಯವಾದ ವಿಚಾರಣೆಗೆ ಅವಳನ್ನು ಅನಂತರ ಗುರಿಪಡಿಸುತ್ತಿದ್ದರು. ಸುತ್ತಲೂ ಜನ ಅಣಕಿಸುತ್ತಿರುವಾಗ ಅವಳನ್ನು ಸುಡಲು ಕರೆದುಕೊಂಡು ಹೋಗುವರು. ಅವಳು ಬೆಂಕಿಗೆ ಸಿಕ್ಕಿ ಯಮ ಯಾತನೆ ಪಡುತ್ತಿರುವಾಗ ಅದನ್ನು ನೋಡುತ್ತಿರುವವರು, ಇದೊಂದು ಸಣ್ಣ ಯಾತನೆ ದೇಹಕ್ಕೆ, ಪರಲೋಕಕ್ಕೆ ಹೋದ ಮೇಲೆ ಆತ್ಮವು ಎಂದೆಂದಿಗೂ ನರಕದ ಬೆಂಕಿಯಲ್ಲಿ ಬೇಯಬೇಕಾಗುತ್ತಿತ್ತು ಎಂದು ಸಮಾಧಾನ ಪಟ್ಟುಕೊಳ್ಳುತ್ತಿದ್ದರು.

\delimiter


\section{ತಾಯಂದಿರು ಪವಿತ್ರರು}

\vskip -0.15cm

ಹಿಂದೂಗಳು ತಾಯಿತನವನ್ನು ಪೂಜ್ಯ ದೃಷ್ಟಿಯಿಂದ ನೋಡಬೇಕೆಂಬುದನ್ನು ಕಲಿಯು\break ತ್ತಾರೆ, ಎಂದರು ವಿವೇಕಾನಂದರು. ತಾಯಿ ಹೆಂಡತಿಗಿಂತ ಮುಖ್ಯ. ತಾಯಿ ಪವಿತ್ರಾತ್ಮಳು. ಹಿಂದೂವಿಗೆ ದೇವರನ್ನು ತಾಯಿ ಎನ್ನುವುದು ತಂದೆ ಎನ್ನುವುದಕ್ಕಿಂತ ಪ್ರಿಯವಾಗಿರುವುದು. ಸ್ತ್ರೀಯರು ಯಾವ ವರ್ಗಕ್ಕೆ ಸೇರಿರಲಿ ಅವರಿಗೆ ದೇಹ ದಂಡನೆಯ ಶಿಕ್ಷೆಯನ್ನು ವಿಧಿಸು\break ವುದಿಲ್ಲ. ಸ್ತ್ರೀಯೇನಾದರೂ ಯಾರನ್ನಾದರೂ ಕೊಂದರೆ ಅವಳನ್ನು ಸಾಯಿಸುವುದಿಲ್ಲ. ಅವಳನ್ನು ಕತ್ತೆಯ ಮೇಲೆ ಹಿಂದು ಮುಂದಾಗಿ ಕುಳ್ಳಿರಿಸಿ ಮೆರವಣಿಗೆ ಮಾಡುವರು. ಹೀಗೆ ಅವಳು ಹೋಗುವಾಗ ತಮಟೆ ಹೊಡೆಯುವವನು ಅವಳು ಮಾಡಿದ ಅಪರಾಧವನ್ನು ಸಾರುವನು. ಇದಾದ ಮೇಲೆ ಹಾಗೆಯೇ ಅವಳನ್ನು ಬಿಟ್ಟು ಬಿಡುವರು. ಮುಂದೆ ಅವಳು ಅಪರಾಧವೆಸಗದೆ ಇರುವಂತೆ ಮಾಡುವುದಕ್ಕೆ ಈ ಮಾನಭಂಗವೇ ಸಾಕಾಗುತ್ತದೆ ಎಂದು ಅವರು ಭಾವಿಸುವರು. ಅವಳೇನಾದರೂ ಪಶ್ಚಾತ್ತಾಪ ಮಾಡಿಕೊಳ್ಳಬೇಕೆಂದು ಆಶಿಸಿದರೆ ಕೆಲವು ಧರ್ಮ ಸಂಸ್ಥೆಗಳು ಅದಕ್ಕಾಗಿ ಇವೆ. ಅವಳು ಅಲ್ಲಿ ಪರಿಶುದ್ಧಳಾಗಬಹುದು. ಅನಂತರ ತಾನು ಇಚ್ಛೆಪಟ್ಟರೆ ಒಬ್ಬ ಸಂನ್ಯಾಸಿನಿಯಾಗಬಹುದು.

ವಿವೇಕಾನಂದರಿಗೆ ಈ ಸಮಯದಲ್ಲಿ ಒಂದು ಪ್ರಶ್ನೆಯನ್ನು ಕೇಳಿದರು. ಹಾಗೆ ಅವಳೇ\break ನಾದರೂ ಒಂದು ಧರ್ಮ ಸಂಸ್ಥೆಗೆ ಸೇರಿದರೆ, ಅವಳನ್ನು ನೋಡುವುದಕ್ಕೆ ಮೇಲಿನವರು ಯಾರೂ ಇಲ್ಲದೇ ಇದ್ದರೆ, ಇದು, ಯಾವ ಆಶ್ರಮವನ್ನು ಹಿಂದೂಗಳು ಅತ್ಯಂತ ಪವಿತ್ರ ದೃಷ್ಟಿಯಿಂದ ನೋಡುವರೋ, ಅದಕ್ಕೆ ಕಳಂಕ ತರುವುದಿಲ್ಲವೇ? ವಿವೇಕಾನಂದರು ಇದನ್ನು ಒಪ್ಪಿಕೊಂಡರು. ಆದರೆ ಹೇಳಿದರು: ಜನರಿಗೂ ಸಂನ್ಯಾಸಿಗೂ ಮಧ್ಯೆ ಯಾರೂ ಇಲ್ಲ ಎಂದು. ಸಂನ್ಯಾಸಿ ಎಲ್ಲಾ ವರ್ಣಗಳಿಗೂ ಅತೀತನಾಗಿ ಹೋಗಿರುವನು. ಬ್ರಾಹ್ಮಣ ಕೀಳು ಕುಲದಲ್ಲಿ ಹುಟ್ಟಿದ ಹಿಂದೂವನ್ನು ಮುಟ್ಟುವುದಿಲ್ಲ. ಆದರೆ ಅವನು ಅಥವಾ ಅವಳು ಏನಾದರೂ ಸಂನ್ಯಾಸಿಯಾದರೆ ಅತಿ ದೊಡ್ಡ ಬ್ರಾಹ್ಮಣನು ಕೂಡ ಅವರಿಗೆ ನಮಸ್ಕಾರ ಮಾಡುವನು. ಜನರು ಸಂನ್ಯಾಸಿಗಳನ್ನು ನೋಡಿಕೊಳ್ಳಬೇಕಾಗಿದೆ-ಆದರೆ ಅವನು ಆ ಸಂನ್ಯಾಸವ್ರತವನ್ನು ಸತ್ಯವಾಗಿ ಪರಿಪಾಲಿಸುತ್ತಿರುವನು ಎಂದು ನಂಬಿಕೊಂಡಿರುವ ತನಕ. ಯಾವಾಗಲಾದರೊಮ್ಮೆ ಅವನು ಕಪಟಿ ಎಂದು ಕಂಡುಬಂದರೆ ಅವನನ್ನು ತಿರಸ್ಕರಿಸುವರು, ಸುಳ್ಳುಗಾರ ಎಂದು ಕರೆಯುವರು. ಅವರ ದೃಷ್ಟಿಯಲ್ಲಿ ಅವನು ತುಂಬಾ ಅಧೋಗತಿಗೆ ಬಂದವನು. ಅವನು ಬರೀ ಭಿಕ್ಷಾಟನೆ ಮಾಡುವವನಾಗುವನು. ನಂತರ ಅವನು ಯಾವ ಗೌರವಕ್ಕೂ ಪಾತ್ರನಾಗುವುದಿಲ್ಲ.


\section{ಇವರ ಭಾವನೆಗಳು}

\vskip -0.15cm

ಸ್ತ್ರೀ ಎಲ್ಲಿಗೆ ಬೇಕಾದರೂ ಹೋಗಬಹುದು. ರಾಜಕುಮಾರರನ್ನು ಬೇಕಾದರೂ ನೋಡಲು ಹೋಗಬಹುದು. ವಿದ್ಯಾಕಾಂಕ್ಷಿಗಳಾದ ಗ್ರೀಕರು ಕಲಿಯುವುದಕ್ಕೆ ಭರತ ಖಂಡಕ್ಕೆ ಬಂದಾಗ ಎಲ್ಲಾ ಮನೆಯವರು ಅವರಿಗೆ ಸ್ವಾಗತವನ್ನು ಕೊಟ್ಟರು. ಆದರೆ ಮಹಮ್ಮದೀಯರು ಖಡ್ಗಧಾರಿಗಳಾಗಿ ಬಂದಾಗ, ಕ್ರೈಸ್ತರು ಕೋವಿಯಿಂದ ಸನ್ನದ್ಧರಾಗಿ ಬಂದಾಗ ಹಿಂದೂಗಳ ಮನೆಬಾಗಿಲುಗಳು ಮುಚ್ಚಲ್ಪಟ್ಟವು. ಇಂತಹ ಅತಿಥಿಗಳನ್ನು ಅವರು ಗೌರವಿಸುತ್ತಿರಲಿಲ್ಲ. ವಿವೇಕಾನಂದರು ಇದನ್ನು ರಮ್ಯವಾಗಿ ಬಣ್ಣಿಸುವರು: “ಹುಲಿಯು ಬಂದರೆ ಅದು ಹೋಗುವ ತನಕ ಬಾಗಿಲನ್ನು ಹಾಕಿಕೊಳ್ಳುತ್ತೇವೆ.”

ಮಾನವತೆಗೆ ಉತ್ತಮ ಭವಿಷ್ಯವಿದೆ ಎಂಬ ಸ್ಫೂರ್ತಿ ಅಮೆರಿಕಾ ದೇಶದಿಂದ ತಮಗೆ ಬಂದಿದೆ ಎಂದೂ, ನಮ್ಮ ಭವಿಷ್ಯ ಮತ್ತು ಜಗತ್ತಿನ ಭವಿಷ್ಯ ಇಂದಿನ ಶಾಸನ ಕರ್ತನ ಮೇಲೆ ನಿಂತಿಲ್ಲ, ಸ್ತ್ರೀಯರ ಮೇಲೆ ಇದೆ ಎಂದರು. “ನಿಮ್ಮ ದೇಶದ ಉದ್ಧಾರ ನಿಂತಿರುವುದು ನಿಮ್ಮ ಸ್ತ್ರೀಯರ ಮೇಲೆ” ಎಂಬುದಾಗಿ ಅವರು ಅಭಿಪ್ರಾಯಪಟ್ಟರು.


\section[ಒಂದು ಚರ್ಚೆ]{ಒಂದು ಚರ್ಚೆ\protect\footnote{* C.W. Vol. V P. 297}}

\begin{center}
(ಅಮೆರಿಕಾ ದೇಶದ ಹಾರ್ವರ್ಡ್​ ವಿಶ್ವವಿದ್ಯಾನಿಲಯದ ಗ್ರಾಜುಯೇಟ್​ ಫಿಲಾಸಫಿಕಲ್​ ಸೊಸೈಟಿಯಲ್ಲಿ ಸ್ವಾಮೀಜಿ ೧೮೯೬ನೇ ಮಾರ್ಚ್​ ೨೫ರಂದು ವೇದಾಂತ ತತ್ತ್ವದ ಮೇಲೆ ಉಪನ್ಯಾಸ ಮಾಡಿದ ನಂತರ ನಡೆದ ಪ್ರಶ್ನೋತ್ತರಗಳು)
\end{center}

\vskip -0.3cm

ಪ್ರಶ್ನೆ: ಇಂಡಿಯಾದಲ್ಲಿ ಈಗ ನಡೆಯುತ್ತಿರುವ ತಾತ್ತ್ವಿಕ ಚಟುವಟಿಕೆಗಳ ಬಗ್ಗೆ ತಿಳಿಯ\break ಬೇಕೆಂಬ ಅಪೇಕ್ಷೆ ಇದೆ. ಈಗ ಎಷ್ಟರ ಮಟ್ಟಿಗೆ ಈ ಪ್ರಶ್ನೆಗಳನ್ನು ಚರ್ಚಿಸುತ್ತಿರುವರು?

ಸ್ವಾಮೀಜಿ: ನಾನು ಈಗ ತಾನೆ ಹೇಳಿದಂತೆ ಇಂಡಿಯಾದ ಬಹುಪಾಲು ಜನರು ದ್ವೈತಿಗಳು, ಅದ್ವೈತಿಗಳ ಸಂಖ್ಯೆ ಕಡಿಮೆ. ಚರ್ಚೆಗೊಳಗಾಗುವ ಮುಖ್ಯ ವಿಷಯವೇ ಮಾಯೆ ಮತ್ತು ಜೀವ. ನಾನು ಈ ದೇಶಕ್ಕೆ ಮೊದಲು ಬಂದಾಗ ಇಲ್ಲಿಯ ಕೂಲಿಗಳಿಗೆ ಈಗಿನ ರಾಜಕೀಯ ಸ್ಥಿತಿ ಚೆನ್ನಾಗಿ ಗೊತ್ತಿರುವುದು ತಿಳಿಯಿತು. ಆದರೆ ನಾನು ಅವರನ್ನು ಧರ್ಮ ಎಂದರೇನು, ಬೇರೆ ಬೇರೆ ಪಂಗಡಗಳ ಮುಖ್ಯ ಸಿದ್ಧಾಂತವೇನು ಎಂದು ಕೇಳಿದಾಗ, ‘ನಮಗೆ ಗೊತ್ತಿಲ್ಲ, ನಾವು ಚರ್ಚಿಗೆ ಹೋಗುತ್ತೇವೆ ಅಷ್ಟೇ’ ಎಂದರು. ನಾನು ಇಂಡಿಯಾ ದೇಶದಲ್ಲಿ ರೈತನ ಹತ್ತಿರ ಹೋಗಿ ನಿಮ್ಮನ್ನು ಆಳುವವರು ಯಾರು ಗೊತ್ತೆ ಎಂದು ಕೇಳಿದರೆ, ನನಗೆ ಗೊತ್ತಿಲ್ಲ, ನಾನು ಕಂದಾಯ ಮಾತ್ರ ಕಟ್ಟುತ್ತೇನೆ’ ಎನ್ನುವನು. ಆದರೆ ನಾನು ಅವನನ್ನು ನಿಮ್ಮ ಧರ್ಮ ಯಾವುದು ಎಂದು ಕೇಳಿದರೆ ಅವನು ನಾನು ದ್ವೈತಿ ಎಂದು ಹೇಳಿ ಮಾಯಾ ಎಂದರೇನು ಜೀವ ಎಂದರೇನು ಎಂಬುದನ್ನು ಬೇಕಾದರೆ ವಿವರಿಸಬಲ್ಲ. ಅವನಿಗೆ ಓದುವುದಕ್ಕೆ ಬರೆಯುವುದಕ್ಕೆ ಬರುವುದಿಲ್ಲ, ಇದನ್ನೆಲ್ಲಾ ಅವನು ಸಾಧುಗಳಿಂದ ಕೇಳಿ ತಿಳಿದು ಕೊಂಡಿರುವನು. ಇದನ್ನು ಕುರಿತು ಚರ್ಚಿಸುವುದಕ್ಕೆ ಅವನಿಗೆ ತುಂಬಾ ಆಸೆ. ದಿನದ ಕೆಲಸವಾದ ಮೇಲೆ ಮರದ ಕೆಳಗೆ ಕುಳಿತುಕೊಂಡು ಇವನ್ನು ಚರ್ಚಿಸುವರು.

ಪ್ರಶ್ನೆ: ಹಿಂದೂಗಳಲ್ಲಿ ಸಾಂಪ್ರದಾಯಿಕತೆ ಎಂದರೆ ಅರ್ಥವೇನು?

ಸ್ವಾಮೀಜಿ: ಸದ್ಯಕ್ಕೆ ಸಾಂಪ್ರದಾಯಿಕತೆ ಎಂದರೆ ಊಟಮಾಡುವುದು ಮತ್ತು ಮದುವೆಗೆ ಸಂಬಂಧಪಟ್ಟ ಕೆಲವು ನಿಯಮಗಳನ್ನು ಅನುಸರಿಸುವುದಾಗಿದೆ. ಇದಲ್ಲದೆ ಅವನು ಯಾವ ಸಿದ್ಧಾಂತವನ್ನು ಬೇಕಾದರೂ ನಂಬಬಹುದು. ಇಂಡಿಯಾ ದೇಶಕ್ಕೆಲ್ಲ ಅನ್ವಯಿಸುವ ಒಂದು ವ್ಯವಸ್ಥಿತ ಮಠ ಇಲ್ಲ. ಆದಕಾರಣ ಇಂತಹ ಆಚಾರಗಳನ್ನೇ ಪಾಲಿಸಬೇಕೆಂಬುದನ್ನು ವಿಧಿಸುವ ಸಂಸ್ಥೆ ಇಲ್ಲ: ಸಾಮಾನ್ಯವಾಗಿ ಯಾರು ವೇದಗಳನ್ನು ನಂಬುವರೋ ಅವರನ್ನು ಸಂಪ್ರದಾಯಸ್ಥರು ಎನ್ನುತ್ತೇವೆ. ಆದರೆ ಹಲವು ದ್ವೈತಿಗಳು ವೇದಗಳಿಗಿಂತ ಹೆಚ್ಚಾಗಿ ಪುರಾಣಗಳನ್ನು ನಂಬುತ್ತಾರೆ.

ಪ್ರಶ್ನೆ: ಹಿಂದುದರ್ಶನ ಗ್ರೀಕರ ಸ್ಟೋಯಿಕ್​ ದರ್ಶನದ ಮೇಲೆ ಯಾವ ಪ್ರಭಾವವನ್ನು ಬೀರಿದೆ?

ಸ್ವಾಮೀಜಿ: ಅಲೆಗ್ಜಾಂಡ್ರಿಯಾದವರ ಮೂಲಕ ಹಿಂದುದರ್ಶನ ಗ್ರೀಕಿನ ದರ್ಶನದ ಮೇಲೆ ತನ್ನ ಪ್ರಭಾವವನ್ನು ಬೀರಿರಬಹುದು. ಫೈಥಾಗೋರಸ್​ ಸಾಂಖ್ಯ ಸಿದ್ಧಾಂತದ ಪ್ರಭಾವಕ್ಕೆ ಒಳಗಾದನೆಂಬ ಊಹೆಯೊಂದಿದೆ. ಹೇಗಾದರೂ ಆಗಲಿ ವೇದಗಳ ತತ್ತ್ವವನ್ನು ಯುಕ್ತಿಯ ಮೂಲಕ ಸಮನ್ವಯಗೊಳಿಸಲು ಪ್ರಯತ್ನಿಸಿದವರಲ್ಲಿ ಸಾಂಖ್ಯರೇ ಮೊತ್ತಮೊದಲಿ\break ಗರು ಎಂದು ನಾವು ಭಾವಿಸುತ್ತೇವೆ. ವೇದಗಳಲ್ಲಿ ಕೂಡ ಕಪಿಲರ ಹೆಸರು ಬರುವುದು ‘ಋಷಿಂ ಪ್ರಸೂತಂ ಕಪಿಲಂ ಯಸ್ತಮಗ್ರೇ’ (ಮೊದಲಲ್ಲಿ ಹುಟ್ಟಿದ ಋಷಿಗಳೇ ಕಪಿಲರು).

ಪ್ರಶ್ನೆ: ಸಾಂಖ್ಯ ದರ್ಶನಕ್ಕೂ ಪಾಶ್ಚಾತ್ಯ ವಿಜ್ಞಾನಗಳಿಗೂ ಏನು ವಿರೋಧ?

ಸ್ವಾಮೀಜಿ: ವಿರೋಧವೇನೂ ಇಲ್ಲ. ನಾವು ಅವರೊಂದಿಗೆ ಸೌಹಾರ್ದದಿಂದ ಇರುವೆವು. ನಮ್ಮ ಪರಿಣಾಮವಾದ, ಆಕಾಶ ಮತ್ತು ಪ್ರಾಣ ಇವು ನಿಮ್ಮ ಆಧುನಿಕ ದರ್ಶನಗಳಲ್ಲಿ ಇರುವಂತೆಯೇ ಇವೆ. ನಿಮ್ಮಲ್ಲಿ ಪರಿಣಾಮವಾದವನ್ನು ಕುರಿತು ಇರುವ ನಂಬಿಕೆಯೇ ನಮ್ಮ ಯೋಗಿಗಳಲ್ಲಿಯೂ ಇತ್ತು, ಮತ್ತು ಸಾಂಖ್ಯರಲ್ಲಿಯೂ ಇತ್ತು. ಪತಂಜಲಿ ತನ್ನ ಸೂತ್ರದಲ್ಲಿ “ಜಾತ್ಯನ್ತರಪರಿಣಾಮಃ ಪ್ರಕೃತ್ಯಾಪೂರಾತ್​” ಪ್ರಕೃತಿಯ ಆಪೂರಣದಿಂದ ಜಾತ್ಯನ್ತರ ಪರಿಣಾಮವಾಗುವುದು ಎಂದು ಹೇಳುತ್ತಾನೆ. ಅವನು ಕೊಡುವ ವಿವರಣೆಯಲ್ಲಿ ವ್ಯತ್ಯಾಸವಿದೆ. ಅವನು ವಿಕಾಸಕ್ಕೆ ಕೊಡುವ ವಿವರಣೆ ಆಧ್ಯಾತ್ಮಿಕವಾದುದು. ಅವನು ‘ನಿಮಿತ್ತಮಪ್ರಯೋಜಕಂ ಪ್ರಕೃತೀನಾಂ ವರಣಭೇದಸ್ತು ತತಃ ಕ್ಷೇತ್ರಿಕವತ್​’ ರೈತ ತನ್ನ ಗದ್ದೆಯ ಹತ್ತಿರ ಹರಿಯುತ್ತಿರುವ ನಾಲೆಯಿಂದ ನೀರನ್ನು ಹಾಯಿಸಬೇಕಾದರೆ ತನ್ನ ಗದ್ದೆಗೆ ನೀರು ಹರಿದು ಬರುವುದಕ್ಕೆ ಇರುವ ಆತಂಕವನ್ನು ಮಾತ್ರ ತೆಗೆಯಬೇಕಾಗಿದೆ ಎನ್ನುವನು. ಆದಕಾರಣ ಪ್ರತಿಯೊಬ್ಬನೂ ಅನಂತಾತ್ಮನಾಗಿರುವನು. ಹಲವು ಸನ್ನಿವೇಶಗಳು ಮತ್ತು ಉಪಾಧಿಗಳು ಅದನ್ನು ಕಾಣದಂತೆ ಮಾಡಿವೆ. ಅದನ್ನು ತೆರೆದೊಡನೆಯೆ ಅದು ವ್ಯಕ್ತವಾಗುವುದು. ಪ್ರಾಣಿಯಲ್ಲಿ ಆಗಲೆ ಮನುಷ್ಯನಾಗುವುದು ತಡೆಯಲ್ಪಟ್ಟಿತ್ತು. ಸನ್ನಿವೇಶದ ಅವಕಾಶ ದೊರಕಿದೊಡನೆ ಅವನು ಮನುಷ್ಯನಾದನು. ಪುನಃ ಸರಿಯಾದ ಅವಕಾಶ ದೊರಕಿದ ಮೇಲೆ ಮಾನವನಲ್ಲಿ ಹುದುಗಿರುವ ದೇವನು ಪ್ರಕಾಶಿತನಾಗುವನು. ನಮಗೆ ಈಗಿನ ಸಿದ್ಧಾಂತಗಳೊಂದಿಗೆ ಮನಸ್ತಾಪಕ್ಕೆ ಕಾರಣವೇ ಇಲ್ಲ. ಸಾಂಖ್ಯದೃಷ್ಟಿಯ ಪ್ರಕಾರ ಗ್ರಹಣಕ್ರಿಯೆ ಹೇಗೆ ಆಗುವುದೆಂಬುದು ಈಗಿನ ಶಾರೀರಿಕ ಶಾಸ್ತ್ರದ ನಿಯಮದಂತೆಯೇ ಇದೆ.

ಪ್ರಶ್ನೆ: ಆದರೆ ನಿಮ್ಮ ಮಾರ್ಗ ಬೇರೆಯಾಗಿದೆಯಲ್ಲ?

ಸ್ವಾಮೀಜಿ: ಹೌದು, ಮನಸ್ಸಿನ ಶಕ್ತಿಗಳನ್ನು ಏಕಾಗ್ರಗೊಳಿಸುವುದೊಂದೇ ಜ್ಞಾನಕ್ಕೆ ಮಾರ್ಗವೆಂದು ನಾವು ಭಾವಿಸುತ್ತೇವೆ. ಬಾಹ್ಯ ವಿಜ್ಞಾನದಲ್ಲಿ ಮನಸ್ಸನ್ನು ಏಕಾಗ್ರಗೊಳಿಸು\break ವುದು ಎಂದರೆ, ಯಾವುದಾದರೂ ಬಾಹ್ಯ ವಸ್ತುವಿನ ಮೇಲೆ ನಿಲ್ಲಿಸುವುದು ಎಂದು ಅರ್ಥ. ಆಂತರಿಕ ವಿಜ್ಞಾನದಲ್ಲಿ ಅದನ್ನು ಅಂತರ್ಮುಖ ಮಾಡುವುದು. ಈ ಮನಸ್ಸಿನ ಏಕಾಗ್ರತೆಯನ್ನೇ ಯೋಗ ಎನ್ನುವುದು.

ಪ್ರಶ್ನೆ: ಅಂತಹ ಏಕಾಗ್ರತೆಯ ಸಮಯದಲ್ಲಿ ಈ ಸಿದ್ಧಾಂತದ ಸತ್ಯಗಳು ಪ್ರತ್ಯಕ್ಷವಾಗು\break ವುದೆ?

ಸ್ವಾಮೀಜೀ: ಯೋಗಿಗಳು ಇನ್ನೂ ಹೆಚ್ಚು ಗೊತ್ತಾಗುವುದು ಎನ್ನುವರು. ಏಕಾಗ್ರತೆ\break ಯಿಂದ ಬಾಹ್ಯ ಮತ್ತು ಆಂತರಿಕ ಸತ್ಯಗಳೆಲ್ಲ ಪ್ರತ್ಯಕ್ಷವಾಗುವುವು ಎನ್ನುವರು.

ಪ್ರಶ್ನೆ: ಈ ವಿಶ್ವದ ಸೃಷ್ಟಿತತ್ತ್ವವನ್ನು ಕುರಿತು ಅದ್ವೈತಿಯ ಭಾವನೆ ಏನು?

ಸ್ವಾಮೀಜಿ: ಅದ್ವೈತಿಗಳು ಸೃಷ್ಟಿ ಮುಂತಾದುವೆಲ್ಲಾ ಮಾಯೆಯಲ್ಲಿದೆ ಎನ್ನುವರು. ನಿಜವಾಗಿ ಇವು ಯಾವುವೂ ಇಲ್ಲ. ನಾವು ಬದ್ಧರಾಗಿರುವವರೆಗೆ ಈ ದೃಶ್ಯಗಳನ್ನು ನೋಡಲೇಬೇಕಾಗಿದೆ. ದೃಶ್ಯಗಳು ಬರುವಾಗ ಇವು ಒಂದು ಕ್ರಮದಲ್ಲಿ ಇರುವುವು. ಇವುಗಳಾಚೆ ಯಾವ ಕ್ರಮವೂ ಇಲ್ಲ, ನಿಯಮವೂ ಇಲ್ಲ. ಬರೀ ಸ್ವಾತಂತ್ರವು ಅಲ್ಲಿರುವುದು.

ಪ್ರಶ್ನೆ: ಅದ್ವೈತವು ದ್ವೈತಕ್ಕೆ ವಿರೋಧಿಯೆ?

ಸ್ವಾಮೀಜಿ: ಉಪನಿಷತ್ತಿನಲ್ಲಿ ಒಂದು ನಿರ್ದಿಷ್ಟ ಸಿದ್ಧಾಂತ ಇಲ್ಲದೇ ಇರುವುದರಿಂದ, ಅನಂತರ ಬಂದ ದಾರ್ಶನಿಕರು ತಮ್ಮ ಸಿದ್ಧಾಂತಕ್ಕೆ ಸರಿಹೊಂದುವ ಶ್ಲೋಕಗಳನ್ನು ತೆಗೆದುಕೊಳ್ಳುವುದು ಸುಲಭವಾಗಿತ್ತು. ಉಪನಿಷತ್ತುಗಳನ್ನೇ ಯಾವಾಗಲೂ ತೆಗೆದುಕೊಳ್ಳಬೇಕಾ\break ಗಿತ್ತು. ಇಲ್ಲದೇ ಇದ್ದರೆ ಅವರ ದರ್ಶನಕ್ಕೆ ಒಂದು ನೆಲೆ ಸಿಕ್ಕುತ್ತಿರಲಿಲ್ಲ. ಆದರೂ ಉಪನಿಷತ್ತಿನಲ್ಲಿ ಭಿನ್ನ ಭಿನ್ನ ತತ್ತ್ವಗಳ ಭಾವನೆಗಳನ್ನೆಲ್ಲಾ ನಾವು ನೋಡುತ್ತೇವೆ. ಅದ್ವೈತವು ದ್ವೈತಭಾವನೆಗೆ ವಿರೋಧವಲ್ಲ ಎಂಬುದೇ ನಮ್ಮ ಮತ. ನಾವು ದ್ವೈತವನ್ನು ಮೂರು ಮೆಟ್ಟಲಲ್ಲಿ ಒಂದು ಎನ್ನುತ್ತೇವೆ. ಧರ್ಮವು ಯಾವಾಗಲೂ ಮೂರು ಮೆಟ್ಟಲನ್ನು ಒಪ್ಪಿಕೊಳ್ಳುವುದು. ಮೊದಲನೆ\break ಯದೆ ದ್ವೈತ. ಇನ್ನು ಸ್ವಲ್ಪ ಮೇಲೆ ಹೋದಾಗ ಅದು ವಿಶಿಷ್ಟಾದ್ವೈತ. ಕೊನೆಗೆ ಅವನು ವಿಶ್ವದಲ್ಲಿ ಐಕ್ಯತಾ ಭಾವನೆಯನ್ನು ಪಡೆಯುತ್ತಾನೆ. ಆದ ಕಾರಣ ಈ ಮೂರು ಒಂದಕ್ಕೊಂದು ವಿರೋಧವಲ್ಲ, ಒಂದು ಮತ್ತೊಂದಕ್ಕೆ ಪೂರಕವಾಗಿದೆ.

ಪ್ರಶ್ನೆ: ಮಾಯೆ ಅಥವಾ ಅಜ್ಞಾನ ಏತಕ್ಕೆ ಇದೆ?

ಸ್ವಾಮೀಜಿ: ಕಾರ್ಯಕಾರಣನಿಮಿತ್ತದ ಆಚೆ ಇರುವುದರ ಕಾರಣವನ್ನು ಹೇಳಲಾಗುವುದಿಲ್ಲ. ಈ ಪ್ರಶ್ನೆಯನ್ನು ಮಾಯೆಯ ಆವರಣದೊಳಗೆ ಇರುವಾಗ ಮಾತ್ರ ಕೇಳಬಹುದು. ನೀವು ಈ ಪ್ರಶ್ನೆಯನ್ನು ತಾರ್ಕಿಕವಾಗಿ ಕೇಳಿದರೆ ನಾವು ಉತ್ತರ ಕೊಡುತ್ತೇವೆ, ಅದಕ್ಕೆ ಮುಂಚೆ ನಮಗೆ ಉತ್ತರ ಹೇಳಲು ಹಕ್ಕಿಲ್ಲ.

ಪ್ರಶ್ನೆ: ಸಗುಣದೇವರು ಮಾಯೆಗೆ ಸೇರಿರುವನೆ?

ಸ್ವಾಮೀಜಿ: ಹೌದು, ಬ್ರಹ್ಮನೇ ಮಾಯಾತೆರೆಯ ಮೂಲಕ ಸಾಕಾರದೇವರಂತೆ ಕಾಣುವನು. ಪ್ರಕೃತಿಯ ಅಧೀನದಲ್ಲಿರುವ ಬ್ರಹ್ಮನೇ ಜೀವ, ಪ್ರಕೃತಿಯನ್ನು ಆಳುತ್ತಿರುವವನೇ ಈಶ್ವರ, ಅಥವಾ ಸಗುಣ ದೇವರು. ಇಲ್ಲಿಂದ ಒಬ್ಬನು ಸೂರ್ಯನನ್ನು ನೋಡುವುದಕ್ಕೆ ಹೊರಟರೆ ಮೊದಲು ಅವನು ಒಂದು ಸಣ್ಣ ಸೂರ್ಯನನ್ನು ನೋಡುತ್ತಾನೆ. ಹಾಗೆಯೇ ಮುಂದೆಮುಂದೆ ಹೋಗುತ್ತಿದ್ದರೆ ಅದನ್ನು ದೊಡ್ಡದೊಡ್ಡದಾಗಿ ನೋಡುತ್ತಿರುತ್ತಾನೆ. ಅವನು ಮುಂದುವರಿಯುತ್ತಿದ್ದಾಗ ಪ್ರತಿಯೊಂದು ಸ್ಥಳದಿಂದಲೂ ತೋರಿಕೆಗೆ ಬೇರೆ ಸೂರ್ಯನನ್ನೇ ನೋಡುತ್ತಿದ್ದನು. ಆದರೂ ಅವನು ನೋಡುತ್ತಿದ್ದುದು ಒಂದೇ ಸೂರ್ಯ ಎಂದು ನಮಗೆ ಗೊತ್ತಿದೆ. ಇವುಗಳೆಲ್ಲಾ ಆ ಪರಬ್ರಹ್ಮನ ದರ್ಶನವೇ ಆಗಿದೆ. ಎಲ್ಲವೂ ಸತ್ಯ, ಒಂದೂ ಸುಳ್ಳಲ್ಲ. ಆದರೆ ಅವುಗಳೆಲ್ಲ ಕೆಳಗಿನ ಹಂತಗಳು ಎಂದು ಮಾತ್ರ ಹೇಳಬಹುದು.

ಪ್ರಶ್ನೆ: ಬ್ರಹ್ಮ ಸಾಕ್ಷಾತ್ಕಾರಕ್ಕೆ ಯಾವುದು ಮಾರ್ಗ?

ಸ್ವಾಮೀಜಿ: ನಾವು ಎರಡು ಮಾರ್ಗಗಳಿವೆ ಎನ್ನುವೆವು. ಒಂದು ನೇತಿ ಮಾರ್ಗ ಮತ್ತೊಂದು ಇತಿ ಮಾರ್ಗ. ಜಗತ್ತೆಲ್ಲ ಸಾಗುತ್ತಿರುವುದೇ ಇತಿ ಮಾರ್ಗದಲ್ಲಿ. ಅದೇ ಪ್ರೀತಿ. ಈ ಪ್ರೀತಿಯ ವೃತ್ತವನ್ನು ನಾವು ಮೇರೆಮೀರಿ ವಿಸ್ತರಿಸಿದರೆ ವಿಶ್ವಪ್ರೇಮವನ್ನು ಪಡೆಯುತ್ತೇವೆ. ಎರಡನೆಯದೆ ನೇತಿ ಮಾರ್ಗ. ಮನಸ್ಸಿನಲ್ಲಿ ಬಾಹ್ಯ ಮುಖವಾಗುವ ಪ್ರತಿಯೊಂದು ಅಲೆಯನ್ನೂ ನಿಲ್ಲಿಸಬೇಕು. ಕೊನೆಗೆ ಮನಸ್ಸು ನಿರ್ನಾಮವಾದಂತೆ ಆಗುವುದು. ಆಗ ಸತ್ಯ ಕಂಗೊಳಿಸುವುದು. ಇದನ್ನೇ ನಾವು ಸಮಾಧಿ ಎನ್ನುವುದು”

ಪ್ರಶ್ನೆ: ಹಾಗಾದರೆ ಅದು ವಿಷಯಿಯನ್ನು ವಿಷಯದಲ್ಲಿ ಲೀನಗೊಳಿಸಿದಂತೆ ಆಗುತ್ತದೆ, ಅಲ್ಲವೆ?

ಉತ್ತರ: ಅದು ವಿಷಯಿಯನ್ನು ವಿಷಯದಲ್ಲಿ ಲೀನಗೊಳಿಸುವುದಲ್ಲ ಬದಲಾಗಿ ವಿಷಯವನ್ನು ವಿಷಯಿಯಲ್ಲಿ ಲೀನಗೊಳಿಸುವುದು. ಈ ಜಗತ್ತು ವಾಸ್ತವವಾಗಿ ಮಾಯವಾಗುತ್ತದೆ. “ನಾನು” ಇರುತ್ತೇನೆ. ಇರುವ ಒಂದು ನಾನು ಮಾತ್ರವೇ.

ಪ್ರಶ್ನೆ: ಭರತಖಂಡದ ಸಿದ್ಧಾಂತವೆಲ್ಲ ಪಾಶ್ಚಾತ್ಯ ಪ್ರಭಾವದ ಪರಿಣಾಮವಿರಬಹುದು ಎಂದು ಜರ್ಮನಿಯ ಕೆಲವು ದಾರ್ಶನಿಕರು ಭಾವಿಸುವರು.

ಸ್ವಾಮೀಜಿ:ನಾನು ಇದಕ್ಕೆ ಬೆಲೆ ಕೊಡುವುದಿಲ್ಲ. ಅವರ ಊಹೆ ನಿರಾಧಾರವಾದುದು. ಇಂಡಿಯಾ ದೇಶದ ಭಕ್ತಿ ಪಾಶ್ಚಾತ್ಯರ ಭಕ್ತಿಯಂತೆ ಅಲ್ಲ. ನಮ್ಮ ಭಕ್ತಿಯ ಮುಖ್ಯ ಭಾವನೆಯಲ್ಲಿ ಭಯ ಎಂಬುದೇ ಇಲ್ಲ. ಅಲ್ಲಿ ಯಾವಾಗಲೂ ದೇವರ ಪ್ರೀತಿಯೊಂದೇ ಇರುವುದು. ಅಲ್ಲಿ ಮೊದಲಿನಿಂದ ಕೊನೆಯವರೆವಿಗೂ ಭಯದಿಂದ ಪೂಜೆ ಮಾಡುವುದು ಎಂಬುದು ಇಲ್ಲವೇ ಇಲ್ಲ, ಬರೀ ಪ್ರೇಮಾರಾಧನೆ ಮಾತ್ರ ಇದೆ. ಎರಡನೆಯದಾಗಿ ಅವರ ಊಹೆ ಅನಗತ್ಯವಾದುದು. ಭಕ್ತಿಯ ಪ್ರಸ್ತಾಪ ಅತ್ಯಂತ ಪುರಾತನ ಉಪನಿಷತ್ತಿನಲ್ಲಿ ಇದೆ. ಇದು ಬೈಬಲ್ಲಿಗಿಂತ ಬಹಳ ಪುರಾತನವಾದದ್ದು. ಭಕ್ತಿಯ ಅಂಕುರವನ್ನು ನಾವು ಸಂಹಿತೆಯಲ್ಲಿಯೂ ನೋಡುತ್ತೇವೆ. ಭಕ್ತಿ ಎಂಬುದು ಪಾಶ್ಚಾತ್ಯ ಪದವಲ್ಲ, ಇದು ಶ್ರದ್ಧೆ ಎಂಬ ಪದದ ಮೂಲಕ ಬಂದಿದೆ.

ಪ್ರಶ್ನೆ: ಹಿಂದೂಗಳು ಕ್ರೈಸ್ತ ಧರ್ಮವನ್ನು ಯಾವ ದೃಷ್ಟಿಯಿಂದ ನೋಡುವರು?

ಸ್ವಾಮೀಜಿ: ಅದು ಒಳ್ಳೆಯದು ಎನ್ನುತ್ತಾರೆ. ವೇದಾಂತವು ಎಲ್ಲರನ್ನೂ ಸ್ವೀಕರಿಸುವುದು. ನಮ್ಮಲ್ಲಿ ಒಂದು ವಿಚಿತ್ರ ಭಾವನೆ ಇದೆ. ನನಗೆ ಒಂದು ಮಗು ಇದೆ ಎಂದು ಊಹಿಸಿ. ನಾನು ಅದಕ್ಕೆ ಯಾವ ಧರ್ಮವನ್ನೂ ಉಪದೇಶಿಸುವುದಿಲ್ಲ. ಅದಕ್ಕೆ ಪ್ರಾಣಾಯಾಮ, ಮನಸ್ಸಿನ ಏಕಾಗ್ರತೆ ಮತ್ತು ಒಂದು ಪ್ರಾರ್ಥನೆಯನ್ನು ಮಾತ್ರ ಹೇಳಿಕೊಡುತ್ತೇನೆ. ಆದರೆ ಅದು ನಿಮ್ಮ ಪ್ರಾರ್ಥನೆಯಂತೆ ಅಲ್ಲ, ಅದು ಹೀಗಿದೆ: ‘ಈ ಪ್ರಪಂಚವನ್ನು ಸೃಷ್ಟಿಸಿದವನನ್ನು ಕುರಿತು ನಾನು ಧ್ಯಾನಿಸುತ್ತೇನೆ. ಅವನು ನನ್ನ ಬುದ್ಧಿಯನ್ನು ಪ್ರಚೋದಿಸಲಿ.’ ಅವನು ಮೊದಲು ಇದನ್ನು ಕಲಿತು ಅನಂತರ ಹಲವು ಸಿದ್ಧಾಂತಗಳನ್ನು, ಹಲವು ಗುರುಗಳು ಏನು ಹೇಳುತ್ತಾರೊ ಅದನ್ನು ಕೇಳುತ್ತಾ ಹೋಗುವನು. ಯಾರು ತನಗೆ ಶ್ರೇಷ್ಠ ಎಂದು ಭಾವಿಸುವನೋ ಅವರನ್ನು ಆರಿಸಿಕೊಳ್ಳುವನು. ಅವನು ಗುರುವಾಗುವನು, ಈತ ಶಿಷ್ಯನಾಗುವನು. ಶಿಷ್ಯನು ಗುರುವಿಗೆ ‘ನೀವು ಬೋದಿಸುವ ತತ್ತ್ವ ನನಗೆ ಶ್ರೇಷ್ಠವಾಗಿ ಕಾಣುವುದು. ನನಗೆ ಅದನ್ನು ಬೋಧಿಸಿ’ ಎನ್ನುವನು. ನಮ್ಮ ಮೂಲ ಭಾವನೆಯೆ, ‘ನಿಮ್ಮ ಸಿದ್ಧಾಂತವೆಂದೂ ನನ್ನ ಸಿದ್ಧಾಂತವಾಗಲಾರದು, ಹಾಗೆಯೇ ನನ್ನ ಸಿದ್ಧಾಂತವೆಂದೂ ನಿಮ್ಮ ಸಿದ್ಧಾಂತವಾಗಲಾರದು’ ಎಂಬುದು. ಪ್ರತಿಯೊಬ್ಬನೂ ತನ್ನದೇ ಮಾರ್ಗವನ್ನು ಅನುಸರಿಸಬೇಕಾಗಿದೆ. ನನ್ನ ಮಗಳದು ಒಂದು ಮಾರ್ಗ, ಮಗನದು ಒಂದು ಮಾರ್ಗ, ನನ್ನದು ಮತ್ತೊಂದು ಮಾರ್ಗವಾಗಿರ\break ಬಹುದು. ಪ್ರತಿಯೊಬ್ಬರಿಗೂ ಒಂದೊಂದು ಇಷ್ಟದೇವತೆಯಿರುವುದು. ಅವರು ಅದನ್ನು ಮನಸ್ಸಿನಲ್ಲಿಟ್ಟು ಕೊಂಡಿರುತ್ತಾರೆ. ಇದು ನನಗೆ ಮತ್ತು ನನ್ನ ಗುರುವಿಗೆ ಮಾತ್ರ ಗೊತ್ತಿರುತ್ತದೆ. ಏಕೆಂದರೆ ಇಷ್ಟದೇವತೆಯ ವಿಷಯದಲ್ಲಿ ನಮಗೆ ಕಚ್ಚಾಡುವುದಕ್ಕೆ ಇಷ್ಟವಿಲ್ಲ. ಆದಕಾರಣ ಸಾಮಾನ್ಯ ತತ್ತ್ವ ಮತ್ತು ಸಾಮಾನ್ಯ ಸಾಧನೆಯನ್ನು ಮಾತ್ರ ಎಲ್ಲರಿಗೂ ಹೇಳುತ್ತೇವೆ. ಒಂದು ಹಾಸ್ಯಾಸ್ಪದವಾದ ಉದಾಹರಣೆಯನ್ನು ತೆಗೆದುಕೊಳ್ಳಿ. ನನಗೆ ಒಂದು ಕಾಲಮೇಲೆ ನಿಂತುಕೊಳ್ಳುವುದರಿಂದ ಉತ್ತಮವಾಗಬಹುದು. ಆದರೆ ಎಲ್ಲ ರಿಗೂ ಅದನ್ನು ವಿಧಿಸುವುದು ಹಾಸ್ಯಾಸ್ಪದವಾಗುವುದು. ನಾನು ದ್ವೈತಿಯಾಗಿರಬಹುದು. ನನ್ನ ಹೆಂಡತಿ ಅದ್ವೈತಿಯಾಗಿರಬಹುದು. ನನ್ನ ಮಗನೊಬ್ಬ ಕ್ರಿಸ್ತ, ಬುದ್ಧ, ಮಹಮ್ಮದನನ್ನೋ ಪೂಜಿಸುತ್ತಿರಬಹುದು. ಅವನು ಎಲ್ಲಿಯವರೆಗೂ ಜಾತಿಯ ಕಾಯಿದೆಯನ್ನು ಮೀರಿ ಹೋಗುವುದಿಲ್ಲವೋ ಅಲ್ಲಿಯವರೆಗೆ ಪರವಾಗಿಲ್ಲ, ಅದು ಅವನ ಇಷ್ಟವಾಗಿರುವುದು.

ಪ್ರಶ್ನೆ: ಹಿಂದೂಗಳೆಲ್ಲಾ ವರ್ಣವನ್ನು ನಂಬುವರೆ?

ಸ್ವಾಮೀಜಿ: ನಂಬಲೇಬೇಕಾಗಿದೆ. ಅವರಿಗೆ ಅದರಲ್ಲಿ ನಂಬಿಕೆ ಇಲ್ಲದಿರಬಹುದು, ಆದರೆ ಅದನ್ನು ಒಪ್ಪದೆ ವಿಧಿಯಿಲ್ಲ.

ಪ್ರಶ್ನೆ: ಪ್ರಾಣಾಯಾಮ ಮತ್ತು ಏಕಾಗ್ರತೆಯನ್ನು ಎಲ್ಲರೂ ಅಭ್ಯಾಸ ಮಾಡುವರೆ?

ಸ್ವಾಮೀಜಿ: ಹೌದು, ಕೆಲವರು ತಮ್ಮ ಧರ್ಮ ಪ್ರಕಾರ ಅದನ್ನು ಪಾಲಿಸುವರು. ಇಂಡಿಯಾ ದೇಶದ ದೇವಸ್ಥಾನಗಳು ಇಲ್ಲಿಯ ಚರ್ಚುಗಳಂತೆ ಅಲ್ಲ. ಅವೆಲ್ಲಾ ನಾಳೆ ಮಾಯವಾಗಿ ಹೋದರೂ, ಯಾವ ಹಾನಿಯು ಆಗುವುದಿಲ್ಲ. ಸ್ವರ್ಗಕ್ಕೆ ಹೋಗುವುದಕ್ಕೆ, ಮಕ್ಕಳನ್ನು ಪಡೆಯುವುದಕ್ಕೆ, ಅಥವಾ ಇನ್ನೂ ಯಾವುದಾದರೂ ಬಯಕೆಯನ್ನು ಈಡೇರಿಸಿಕೊಳ್ಳುವುದ\break ಕ್ಕಾಗಿ ದೇವಾಲಯಗಳನ್ನು ಕಟ್ಟಿಸುವರು. ಅಲ್ಲಿ ಪೂಜೆ ಮಾಡುವುದಕ್ಕೆ ಕೆಲವು ಪೂಜಾರಿಗಳನ್ನು ನೇಮಿಸುವರು. ನಾವು ಅಲ್ಲಿಗೆ ಹೋಗಲೇಬೇಕಾಗಿಲ್ಲ. ನಮ್ಮ ಪೂಜೆಯೆಲ್ಲಾ ಮನೆಯೊಳಗೇ ನಡೆಯುವುದು. ಪ್ರತಿಯೊಂದು ಮನೆಯಲ್ಲಿಯೂ ದೇವರ ಪೂಜೆಗಾಗಿ ಪ್ರತ್ಯೇಕವಾದ ಕೋಣೆಯಿರುತ್ತದೆ. ಮಗುವಿಗೆ ಉಪನಯನವಾದ ಮೇಲೆ ಮೊದಲು ಸ್ನಾನ ಮಾಡಿ ಪೂಜೆಮಾಡಬೇಕು. ಅವನ ಪೂಜೆ ಎಂದರೆ ಪ್ರಾಣಾಯಾಮ, ಧ್ಯಾನ ಮತ್ತು ಕೆಲವು ಮಂತ್ರಗಳನ್ನು ಉಚ್ಚರಿಸುವುದು. ಅವನು ನೇರವಾಗಿ ಕುಳಿತುಕೊಳ್ಳಬೇಕು. ದೇಹವನ್ನು ಆರೋಗ್ಯವಾಗಿಡುವುದಕ್ಕೆ ಮನಸ್ಸಿಗೆ ಸಾಧ್ಯ ಎಂದು ನಾವು ನಂಬುತ್ತೇವೆ. ಒಬ್ಬ ಪೂಜೆ ಮಾಡಿಕೊಂಡು ಹೋದಮೇಲೆ ಮತ್ತೊಬ್ಬನು ಬಂದು ಮೌನವಾಗಿ ತನ್ನ ಪೂಜೆಯನ್ನು ಮಾಡಿಕೊಂಡು ಹೋಗುವನು. ಕೆಲವು ವೇಳೆ ಏಕಕಾಲದಲ್ಲಿ ಒಂದೇ ಪೂಜಾಮಂದಿರದಲ್ಲಿ ಮೂರು ನಾಲ್ಕು ಜನರಿರಬಹುದು. ಆದರೆ ಪ್ರತಿಯೊಬ್ಬನ ಇಷ್ಟದೇವತೆಯೇ ಬೇರೆ. ಹೀಗೆ ದಿನಕ್ಕೆ ಎರಡು ವೇಳೆಯಾದರೂ ಪೂಜೆಯನ್ನು ಮಾಡುತ್ತಾರೆ.

ಪ್ರಶ್ನೆ: ನೀವು ಹೇಳುವ ಐಕ್ಯತಾಭಾವನೆ ಕೇವಲ ಒಂದು ಆದರ್ಶವೋ ಅಥವಾ ಯಾರಾದರೂ ಅದನ್ನು ಪ್ರತ್ಯಕ್ಷ ಮಾಡಿಕೊಂಡಿರುವರೊ?

ಸ್ವಾಮೀಜಿ: ನಾವು ಇದನ್ನು ಪಡೆಯುವುದು ಸಾಧ್ಯ ಎನ್ನುತ್ತೇವೆ. ಇದು ಬರೀ ಬಾಯಿ ಮಾತಾದರೆ ಇದರಿಂದ ಏನೂ ಪ್ರಯೋಜನವಿಲ್ಲ. ಮೂರು ವಿಷಯಗಳನ್ನು ವೇದಗಳು ಬೋಧಿಸುತ್ತವೆ. ಮೊದಲು ಆತ್ಮನ ವಿಷಯವಾಗಿ ಶ್ರವಣಮಾಡಬೇಕು, ಅನಂತರ ಅದನ್ನು ವಿಮರ್ಶೆ ಮಾಡಬೇಕು, ಅದರ ಮೇಲೆ ಧ್ಯಾನ ಮಾಡಬೇಕು. ಮೊದಲು ಕೇಳಿದಾಗ ಅದನ್ನು ಕುರಿತು ಯೋಚಿಸಬೇಕು. ಅದನ್ನು ಕೇವಲ ಮೂಢ ನಂಬಿಕೆಯಿಂದ ನಂಬುವುದಲ್ಲ; ಚೆನ್ನಾಗಿ ಅರ್ಥಮಾಡಿಕೊಂಡು ನಂಬಬೇಕು. ಅದನ್ನು ಆಲೋಚನೆ ಮಾಡಿದ ಮೇಲೆ ಅದರ ಮೇಲೆ ಧ್ಯಾನಮಾಡಬೇಕು, ಅನಂತರ ಸಾಕ್ಷಾತ್ಕಾರ ಮಾಡಿಕೊಳ್ಳಬೇಕು. ಇದೇ ಧರ್ಮ. ಬರೀ ನಂಬಿಕೆಯೇ ಧರ್ಮವಲ್ಲ. ಧರ್ಮ ಎಂಬುದು ಒಂದು ಇಂದ್ರಿಯಾತೀತ ಅನುಭವ ಎನ್ನುತ್ತೇವೆ.

ಪ್ರಶ್ನೆ: ಈ ಸ್ಥಿತಿಯನ್ನು ನೀವು ಪಡೆದರೆ ಇದನ್ನು ವಿವರಿಸಲು ಆಗುವುದೇ?

ಸ್ವಾಮೀಜಿ: ಇಲ್ಲ, ಆದರೆ ಅದರ ಪರಿಣಾಮದ ಮೂಲಕ ಅದನ್ನು ಅರಿಯಬಹುದು. ದಡ್ಡ ನಿದ್ದೆಗೆ ಹೋದರೆ ಅಲ್ಲಿಂದ ಬರುವಾಗಲೂ ದಡ್ಡನೇ ಆಗಿರುವನು, ಇಲ್ಲವೆ ಅದಕ್ಕಿಂತಲೂ ಹೀನನಾಗಿರುವನು. ಆದರೆ ಸಮಾಧಿಗೆ ಹೋಗಿ ಬರುವಾಗ ಅವನು ಜ್ಞಾನಿಯಾಗಿ\break ರುವನು, ಋಷಿಯಾಗಿರುವನು, ಮಹಾಪುರುಷನಾಗಿರುವನು. ಅದೇ ಎರಡು ಅವಸ್ಥೆಗೂ ಇರುವ ವ್ಯತ್ಯಾಸವನ್ನು ತೋರುವುದು.

ಪ್ರಶ್ನೆ: ನಾನು ಮತ್ತೊಂದು ಪ್ರಶ್ನೆಯನ್ನು ಕೇಳಬೇಕೆಂದಿರುವೆನು. ಆತ್ಮ ಸಮ್ಮೋಹನ \enginline{(Self-Hypenotism)} ಮಾಡಿಕೊಳ್ಳುವವರನ್ನು ನೀವು ನೋಡಿರುವಿರಾ? ಹಿಂದಿನ ಕಾಲದಲ್ಲಿ ಇಂಡಿಯಾ ದೇಶದಲ್ಲಿ ಇದನ್ನು ಮಾಡಿಕೊಳ್ಳುತ್ತಿದ್ದರು. ಎಂದು ಹೇಳುವರು. ಆಧುನಿಕ ಭರತಖಂಡದಲ್ಲಿ ಅದು ಹೆಚ್ಚಾಗಿ ಇಲ್ಲದೆ ಇರಬಹುದು.

ಸ್ವಾಮೀಜಿ: ನೀವು ಈ ದೇಶದಲ್ಲಿ ಯಾವುದನ್ನು ಸಮ್ಮೋಹೀನೀ ವಿದ್ಯೆ \enginline{(Hypenotism)} ಎನ್ನುವಿರೋ ಅದೆಲ್ಲೋ ನಿಜವಾದುದರ ಒಂದು ಅಂಶ ಮಾತ್ರ. ಹಿಂದುಗಳು ಇದನ್ನು ಆತ್ಮಸಮ್ಮೋಹನ \enginline{(Self-Hypenotism)} ಎನ್ನುವರು. ಅವರು ನೀವು ಈಗಾಗಲೇ ಸಮ್ಮೋಹಿತರಾಗಿರುವಿರಿ ಎನ್ನುವರು. ಈಗ ಮಾಡಿಕೊಳ್ಳಬೇಕಾಗಿರುವುದು ಆ ಅವಸ್ಥೆ\break ಯಿಂದ ಹೊರಬರುವುದು. ಹಾಗೆ ಹೊರಬಂದಾಗ, ‘ಅಲ್ಲಿ ಸೂರ್ಯ ಬೆಳಗಲಾರ, ಚಂದ್ರ ತಾರಾವಳಿಗಳೂ ಬೆಳಗಲಾರವು. ಮಿಂಚೂ ಅಲ್ಲಿ ಬೆಳಗಲಾರದು. ಈ ಬೆಂಕಿಯ ಮಾತೇಕೆ ಇನ್ನು? ಅದು ಬೆಳಗಿದರೆ ಎಲ್ಲವೂ ಅದರಿಂದ ಬೆಳಗುವುದು.’ ಇದು ಸಮ್ಮೋಹನಕ್ಕೆ ಒಳಗಾಗುವುದಲ್ಲ, ಅದರಿಂದ ಹೊರಬರುವುದು. ಈ ವಿಷಯಗಳು ಸತ್ಯವೆಂದು ಬೋಧಿ\break ಸುವ ಇತರ ಧರ್ಮಗಳೆಲ್ಲ ಒಂದು ವಿಧವಾದ ಸಮ್ಮೋಹನ ವಿದ್ಯೆಯನ್ನು ಅಭ್ಯಾಸಮಾಡು\break ತ್ತಿವೆ. ಅದ್ವೈತಿಯೊಬ್ಬನೇ ಸಮ್ಮೋಹನಕ್ಕೆ ಇಚ್ಛೆಪಡುವುದಿಲ್ಲ. ಅವನೊಬ್ಬನೇ ‘ಎಲ್ಲಾ ವಿಧವಾದ ದ್ವೈತ ಭಾವನೆಯಲ್ಲೂ ಒಂದಲ್ಲ ಮತ್ತೊಂದು ಸಮ್ಮೋಹನವಿದೆ’ ಎಂದು ಅರಿಯಬಲ್ಲ. ಆದರೆ ಅದ್ವೈತಿಗಳು, ವೇದಗಳನ್ನು ಕೂಡ ಆಚೆಗೆ ಎಸೆಯಿರಿ, ಸಾಕಾರ ದೇವರನ್ನು ಆಚೆಗೆ ಎಸೆಯಿರಿ, ದೇಹ ಮತ್ತು ಮನಸ್ಸನ್ನು ಕೂಡ ಆಚೆಗೆ ಎಸೆಯಿರಿ. ನಿಮ್ಮನ್ನು ಸಮ್ಮೋ ಹನಗೊಳಿಸುವ ಯಾವುದೂ ಇಲ್ಲದಿರಲಿ ಎನ್ನುವರು. ‘ಯಾವುದನ್ನು ವಿವರಿಸಲಾರದೆ ಮಾತು ಮತ್ತು ಮನಸ್ಸು ಹಿಂತಿರುಗಿ ಬರುವುದೋ’ ಆ ಬ್ರಹ್ಮಾನಂದವನ್ನು ಪಡೆದರೆ ಮತ್ತಾವ ಅಂಜಿಕೆಯೂ ಇಲ್ಲ. ಇದೇ ಸಮ್ಮೋಹನವನ್ನು ನಿರ್ಮೂಲ ಮಾಡುವುದು. ‘ನನ್ನಲ್ಲಿ ಪಾಪ ಪುಣ್ಯಗಳಿಲ್ಲ, ಸುಖ ದುಃಖಗಳಿಲ್ಲ, ನನಗೆ ವೇದ ಯಜ್ಞಗಳಾವುವೂ ಬೇಕಿಲ್ಲ. ನಾನು ಆಹಾರವೂ ಅಲ್ಲ, ಅದನ್ನು ತಿನ್ನುವ ಕ್ರಿಯೆಯೂ ಅಲ್ಲ, ತಿನ್ನುವವನೂ ಅಲ್ಲ, ಏಕೆಂದರೆ ನಾನೇ ಸಚ್ಚಿದಾನಂದ ಸ್ವರೂಪ’. ನಮಗೆ ಸಮ್ಮೋಹನದ ವಿಷಯವೆಲ್ಲ ಗೊತ್ತಿದೆ. ನಮ್ಮಲ್ಲಿ ಒಂದು ಮನಶ್ಶಾಸ್ತ್ರವಿದೆ, ಅದನ್ನು ಪಾಶ್ಚಾತ್ಯರು ಈಗತಾನೇ ತಿಳಿಯಲೆತ್ನಿಸುವರು. ಆದರೆ ಅದನ್ನು ಇನ್ನೂ ಚೆನ್ನಾಗಿ ತಿಳಿದುಕೊಂಡಿಲ್ಲ ಎಂದು ಹೇಳಲು ವಿಷಾದಪಡುವೆ.

ಪ್ರಶ್ನೆ: ನೀವು ಸೂಕ್ಷ್ಮದೇಹವನ್ನು ಏನೆಂದು ಕರೆಯುತ್ತೀರಿ?

ಸ್ವಾಮೀಜಿ: ಅದನ್ನು ನಾವು ಲಿಂಗಶರೀರ ಎನ್ನುತ್ತೇವೆ. ಈ ದೇಹ ನಾಶವಾದ ಮೇಲೆ ಅದು ಮತ್ತೊಂದು ದೇಹವನ್ನು ಧರಿಸುವುದು. ಶಕ್ತಿಯು ದ್ರವ್ಯವಿಲ್ಲದೆ ಇರಲಾರದು. ಆದಕಾರಣ ಸ್ವಲ್ಪ ಸ್ಥೂಲದೇಹ ಉಳಿಯುವುದು. ಇದರ ಮೂಲಕ ಅಂತಃಕರಣಗಳು ಬೇರೊಂದು ದೇಹವನ್ನು ತಯಾರು ಮಾಡುವುವು. ಪ್ರತಿಯೊಬ್ಬರೂ ತಮ್ಮ ದೇಹವನ್ನು ಸೃಷ್ಟಿಸಿಕೊಳ್ಳುತ್ತಿರುವರು. ಮನಸ್ಸೆ ದೇಹವನ್ನು ತಯಾರು ಮಾಡುವುದು. ನಾನೊಬ್ಬ ಋಷಿ\break ಯಾದರೆ ನನ್ನ ಮಿದುಳು ಒಬ್ಬ ಋಷಿಯ ಮಿದುಳಾಗಿ ಪರಿವರ್ತನೆ ಹೊಂದುವುದು. ಈ ಜೀವನದಲ್ಲೆ ಮಾನವ ತನ್ನ ದೇಹವನ್ನು ದೇವರ ದೇಹವನ್ನಾಗಿ ಪರಿವರ್ತಿಸಬಲ್ಲ ಎಂದು ಯೋಗಿಗಳು ಹೇಳುವರು.

ಯೋಗಿಗಳು ಹಲವು ಅದ್ಭುತ ವಿಷಯಗಳನ್ನು ತೋರುವರು. ಸಾಸಿವೆ ಕಾಳಿನಷ್ಟು ಅನುಷ್ಠಾನ ಪರ್ವತದಷ್ಟು ಸಿದ್ಧಾಂತಕ್ಕಿಂತ ಮೇಲು. ನನಗೆ ಕೆಲವು ವಸ್ತುಗಳು ಕಂಡಿಲ್ಲ ಎಂದರೆ ಅದು ಸುಳ್ಳು ಎಂದು ಹೇಳಲಾಗುವುದಿಲ್ಲ. ಅಭ್ಯಾಸದ ಮೂಲಕ ಅದ್ಭುತವಾದ ಶಕ್ತಿಯನ್ನು ಪಡೆಯಬಹುದೆಂದು ಹಿಂದೂಗಳ ಶಾಸ್ತ್ರಗಳು ಸಾರುವುವು. ಪ್ರತಿದಿನ ಕೆಲವು ಕಾಲ ಅಭ್ಯಾಸ ಮಾಡಿದರೆ ನಾವು ಸ್ವಲ್ಪ ಸ್ವಲ್ಪ ಅದರ ಪರಿಣಾಮವನ್ನು ನೋಡಬಹುದು. ಇದರಿಂದ ಇದೆಲ್ಲ ಒಂದು ಬೂಟಾಟಿಕೆ ಅಲ್ಲ ಎಂಬುದು ಗೊತ್ತಾಗುವುದು. ಯೋಗಿಗಳು ಶಾಸ್ತ್ರಗಳಲ್ಲಿ ಹೇಳಿರುವ ಅದ್ಭುತವಾದ ವಿಷಯಗಳನ್ನೆಲ್ಲಾ ವೈಜ್ಞಾನಿಕವಾಗಿ ವಿವರಿಸುವರು. ಈ ಪವಾಡಗಳು ಎಲ್ಲಾ ದೇಶಗಳಲ್ಲೂ ಹೇಗೆ ಬಳಕೆಗೆ ಬಂದವು ಎಂಬುದೇ ಒಂದು ಆಶ್ಚರ್ಯ. ಯಾರು ಇದೆಲ್ಲಾ ಸುಳ್ಳು, ಇದಕ್ಕೆ ವಿವರಣೆಗಳೇನೂ ಅಗತ್ಯವಿಲ್ಲ ಎನ್ನುವನೋ ಅವನು ವಿಚಾರವಾದಿಯಲ್ಲ. ನೀವು ಅದನ್ನು ಶುದ್ಧ ಸುಳ್ಳು ಎಂದು ತೋರಿಸುವವರೆಗೆ ಅದನ್ನು ನಿರಾಕರಿಸುವುದಕ್ಕೆ ನಿಮಗೆ ಅಧಿಕಾರವಿಲ್ಲ. ಇದಕ್ಕೆ ಯಾವ ತಳಹದಿಯೂ ಇಲ್ಲ ಎಂದು ತೋರಿಸಿದರೆ ಮಾತ್ರ ಇದೆಲ್ಲ ಶುದ್ಧ ಸುಳ್ಳು ಎಂದು ಹೇಳಲು ನಿಮಗೆ ಅಧಿಕಾರವಿದೆ. ಆದರೆ ನೀವು ಅದನ್ನು ಮಾಡಿಲ್ಲ. ಆದರೆ ಯೋಗಿಗಳಾದರೋ, ಇವು ಪವಾಡಗಳಲ್ಲ, ಇದನ್ನು ಈಗಲೂ ಕೂಡ ಮಾಡಬಹುದು ಎನ್ನುವರು. ಭರತಖಂಡದಲ್ಲಿ ಎಷ್ಟೋ ಅದ್ಭುತಗಳನ್ನು ಈಗಲೂ ಮಾಡುತ್ತಿರುವರು. ಆದರೆ ಇದಾವುದನ್ನೂ ಪವಾಡದಂತೆ ಮಾಡುವುದಿಲ್ಲ. ಈ ವಿಷಯವನ್ನು ಕುರಿತಂತೆ ಅನೇಕ ಪುಸ್ತಕಗಳಿವೆ. ಇನ್ನಾವುದನ್ನೂ ಮಾಡದೆ ಕೇವಲ ವೈಜ್ಞಾನಿಕವಾಗಿ ಮನಶ್ಶಾಸ್ತ್ರವನ್ನು ಅಧ್ಯಯನ ಮಾಡಿದ್ದರೆ ಆ ಕೀರ್ತಿ ಯೋಗಿಗಳಿಗೆ ಸೇರಬೇಕಾಗಿದೆ.

ಪ್ರಶ್ನೆ: ಯೋಗಿಗಳು ಮಾಡಬಲ್ಲ ಅದ್ಭುತಗಳಾವುವು ಎಂಬುದನ್ನು ನೀವು ಸ್ಪಷ್ಟವಾಗಿ ವಿವರಿಸುವಿರಾ?

ಸ್ವಾಮೀಜಿ: ಯೋಗಿಗಳು ನೀವು ಉಳಿದ ವಿಜ್ಞಾನಕ್ಕೆ ಎಷ್ಟು ನಂಬಿಕೆ ಮತ್ತು ಶ್ರದ್ಧೆಯನ್ನು ಕೊಡುತ್ತೀರೊ ಅದಕ್ಕಿಂತ ಹೆಚ್ಚಾಗಿ ಇದಕ್ಕೇನೂ ಬೇಡ ಎನ್ನುವರು. ಪ್ರಯೋಗ ಮಾಡುವುದಕ್ಕೆ ಅಗತ್ಯವಾದ ಸಾಮಾನ್ಯ ಶ್ರದ್ಧೆ ಇದ್ದರೆ ಸಾಕು. ಯೋಗಿಯ ಆದರ್ಶ ಅದ್ಭುತವಾದುದು. ಮನೋಬಲದಿಂದ ಮಾಡಿದ ಕೆಲವು ಕೆಳಮಟ್ಟದ ಘಟನೆಗಳನ್ನು ನಾನು ನೋಡಿರುವೆನು. ಆದಕಾರಣ ಅವರು ಮೇಲುತರದ ಘಟನೆಗಳನ್ನೂ ಮಾಡಬಲ್ಲರೆಂಬುದನ್ನು ನಂಬುವೆನು. ಯೋಗಿಯ ಆದರ್ಶವು ಸರ್ವಜ್ಞತ್ವ ಮತ್ತು ಸರ್ವಶಕ್ತಿತ್ವದ ಮೂಲಕ ಶಾಶ್ವತ ಶಾಂತಿಯನ್ನೂ ಪ್ರೀತಿಯನ್ನೂ ಸಾಧಿಸುವುದು. ಒಬ್ಬ ಯೋಗಿಯನ್ನು ಸರ್ಪ ಕಚ್ಚಿತು. ಅವನು ನೆಲದ ಮೇಲೆ ಬಿದ್ದನು. ಸಾಯಂಕಾಲ ಪುನಃ ಪ್ರಜ್ಞೆ ಬಂದಿತು. ಏನಾಯಿತು, ಎಂದು ಕೇಳಿದಾಗ, “ನನ್ನ ಪ್ರಿಯತಮನಿಂದ ಒಬ್ಬ ದೂತ ಬಂದಿದ್ದ” ಎಂದನು. ದ್ವೇಷ, ಅಸೂಯೆ, ಕೋಪವೆಲ್ಲ ಇವನಲ್ಲಿ ಭಸ್ಮೀಭೂತವಾಗಿ ಹೋಗಿತ್ತು. ಅವನನ್ನು ಯಾವುದೂ ಕುಪಿತನನ್ನಾಗಿ ಮಾಡಲಾರದಾಗಿತ್ತು. ಅವನು ಯಾವಾಗಲೂ ಅನಂತಪ್ರೇಮವೇ ಆಗಿದ್ದನು. ಆ ಪ್ರೇಮಶಕ್ತಿಗೆ ಅಸಾಧ್ಯವಾದುದು ಯಾವುದೂ ಇರಲಿಲ್ಲ. ಅವನೇ ನಿಜವಾದ ಯೋಗಿ. ಇತರ ಸಿದ್ಧಿಗಳು ಗೌಣ, ಕೇವಲ ನಿಮಿತ್ತಮಾತ್ರ. ಇವುಗಳನ್ನು ಪಡೆಯಬೇಕೆಂಬುದು ಅವನ ಗುರಿಯಲ್ಲ. ಯೋಗಿಯಲ್ಲದವರೆಲ್ಲ ಗುಲಾಮರು ಎನ್ನುವನು ಯೋಗಿ. ಅವರೆಲ್ಲ ಹೊಟ್ಟೆಗೆ, ಬಟ್ಟೆಗೆ, ಹೆಂಡರಿಗೆ, ಮಕ್ಕಳಿಗೆ, ಹಣಕ್ಕೆ ದಾಸರು, ದೇಹಕ್ಕೆ ದಾಸರು, ಹೆಸರು ಕೀರ್ತಿಗೆ ದಾಸರು, ಇನ್ನೂ ಹಲವು ವಸ್ತುಗಳಿಗೆ ದಾಸರು. ಇವುಗಳಾವುದಕ್ಕೂ ದಾಸನಲ್ಲದವನು ಮಾತ್ರ ನಿಜವಾದ ಯೋಗಿ. “ಯಾರು ಸಮತ್ವದಲ್ಲಿ ನೆಲೆಸಿರುವರೋ ಅವರು ಬದುಕಿರುವಾಗಲೇ ಪ್ರಪಂಚದ ಬಂಧನದಿಂದ ಪಾರಾಗಿರುವರು. ದೇವರು ಪರಿಶುದ್ಧನು, ಎಲ್ಲರಿಗೂ ಒಂದೇ. ಇಂತಹವರೇ ನಿಜವಾಗಿ ದೇವರಲ್ಲಿ ಬಾಳುತ್ತಿರುವವರು”.

ಪ್ರಶ್ನೆ: ಯೋಗಿಗಳು ಜಾತಿ ವ್ಯವಸ್ಥೆಯನ್ನು ಒಪ್ಪಿಕೊಳ್ಳುವರೆ?

ಸ್ವಾಮೀಜಿ: ಇಲ್ಲ, ಮಾನಸಿಕವಾಗಿ ಇನ್ನೂ ವಿಕಾಸವಾಗದವರು ಮಾತ್ರ ಇದನ್ನು ಒಪ್ಪಿಕೊಳ್ಳುವರು.

ಪ್ರಶ್ನೆ: ಇಂಡಿಯಾ ದೇಶದಲ್ಲಿರುವ ಉಷ್ಣದ ಹವಾಗುಣಕ್ಕೂ ಸಮಾಧಿಗೂ ಏನಾದರೂ ಸಂಬಂಧವಿದೆಯೇ?

ಸ್ವಾಮೀಜಿ: ಇರುವಂತೆ ಕಾಣುವುದಿಲ್ಲ. ಈ ತಾತ್ತ್ವಿಕ ಆಲೋಚನೆಗಳನ್ನೆಲ್ಲಾ ಮಾಡಿದುದು ಸಮುದ್ರಮಟ್ಟಕ್ಕೆ ಸುಮಾರು ಹದಿನೈದುಸಾವಿರ ಅಡಿ ಎತ್ತರವಿರುವ ಹಿಮಾಲಯದ ಮೇಲೆ. ಅಲ್ಲಿ ಧ್ರುವದ ಸಮೀಪದಲ್ಲಿರುವಂತೆ ಇದೆ ವಾಯುಗುಣ.

ಪ್ರಶ್ನೆ: ಶೀತದೇಶಗಳಲ್ಲಿ ಆಧ್ಯಾತ್ಮಿಕ ಜೀವನದಲ್ಲಿ ಜಯಪ್ರದವಾಗುವುದು ಸಾಧ್ಯವೆ?

ಸ್ವಾಮೀಜಿ: ಇದು ಅನುಷ್ಠಾನ ಸಾಧ್ಯ. ಪ್ರಪಂಚದಲ್ಲಿ ಏಕಮಾತ್ರ ಅನುಷ್ಠಾನ ಸಾಧ್ಯ\break ವಾದುದು ಅಂದರೆ ಇದೇ. ನಿಮ್ಮಲ್ಲಿ ಪ್ರತಿಯೊಬ್ಬರೂ ಆಜನ್ಮ ವೇದಾಂತಿಗಳೆಂದು ನಾವು ಸಾರುತ್ತೇವೆ. ನೀವು ಪ್ರತಿಕ್ಷಣ ಪ್ರತಿಯೊಂದು ವಸ್ತುವಿನೊಂದಿಗೂ ತಾದಾತ್ಮ್ಯಭಾವವನ್ನು ವ್ಯಕ್ತಗೊಳಿಸು\break ತ್ತಿರುವಿರಿ. ಪ್ರತಿಯೊಂದು ಸಲ ನೀವು ಪ್ರಪಂಚಕ್ಕೆ ಅನುಕಂಪವನ್ನು ತೋರುವಾಗಲೂ ನೀವು ನಿಜವಾದ ವೇದಾಂತಿಗಳಾಗಿರುವಿರಿ. ಇದು ನಿಮಗೆ ಗೊತ್ತಿಲ್ಲ ಅಷ್ಟೆ. ನೀವು ಏತಕ್ಕೆ ನೀತಿವಂತರಾಗಿರಬೇಕೆಂಬುದನ್ನು ಅರಿಯದೇ ನೀತಿವಂತರಾಗಿರುವಿರಿ. ಮಾನವಸ್ವಭಾವ\break ವನ್ನೆಲ್ಲಾ ವಿಶ್ಲೇಷಣೆ ಮಾಡಿ, ಅದನ್ನು ಒಂದು ಸಿದ್ಧಾಂತಮಾಡಿ ಹೇಗೆ ಪ್ರಜ್ಞಾಪೂರ್ವಕವಾಗಿ ನೈತಿಕರಾಗಿರಬೇಕೆಂಬುದನ್ನು ಇದು ಬೋಧಿಸಿತು. ಇದೇ ಎಲ್ಲಾ ಧರ್ಮದ ಸಾರ.

ಪ್ರಶ್ನೆ: ಪಾಶ್ಚಾತ್ಯರಲ್ಲಿ ಪ್ರತಿಯೊಬ್ಬರೂ ತಮ್ಮನ್ನು ಬಹುತತ್ತ್ವವಾದಿಗಳನ್ನಾಗಿ ಮಾಡುವ ಸಮಾಜವಿರೋಧಿಯಾದ ತತ್ತ್ವವುಳ್ಳವರು ಮತ್ತು ಪೌರಸ್ತ್ಯರು ನಮಗಿಂತ ಹೆಚ್ಚು ಸಹಾನು\break ಭೂತಿಯುಳ್ಳವರು ಎಂದು ನೀವು ಹೇಳುವಿರೇನು?

ಸ್ವಾಮೀಜಿ: “ಪಾಶ್ಚಾತ್ಯರು ಹೆಚ್ಚು ಕ್ರೂರಿಗಳು, ಪೌರಸ್ತ್ಯರು ಎಲ್ಲ ಪ್ರಾಣಿಗಳ ವಿಷಯದಲ್ಲೂ ಹೆಚ್ಚು ದಯಾಪರರು ಎಂಬುದು ನಮ್ಮ ಭಾವನೆ. ನಿಮ್ಮ ಸಂಸ್ಕೃತಿ ಇತ್ತೀಚಿನದಾಗಿರು\break ವುದೇ ಇದಕ್ಕೆ ಕಾರಣ. ದಯಾಪರರಾಗಬೇಕಾದರೆ ಅದಕ್ಕೆ ಹೆಚ್ಚು ಕಾಲಬೇಕು. ನಿಮ್ಮಲ್ಲಿ ಬೇಕಾದಷ್ಟು ಶಕ್ತಿ ಇದೆ. ನೀವು ಮನಸ್ಸನ್ನು ಹೆಚ್ಚು ಏಕಾಗ್ರಮಾಡಿಲ್ಲ. ನೀವು ಮೃದುಸ್ವಭಾವ\break ದವರಾಗಬೇಕಾದರೆ, ಸಜ್ಜನರಾಗಬೇಕಾದರೆ, ಅದಕ್ಕೆ ಕಾಲಬೇಕು. ಆದರೆ ಈ ಭಾವನೆ ಪ್ರತಿಯೊಬ್ಬ ಭಾರತೀಯನ ಧಮನಿಯಲ್ಲಿಯೂ ಅನುರಣಿತವಾಗುತ್ತಿದೆ. ನಾನು ಹಳ್ಳಿಯಲ್ಲಿ ಜನರಿಗೆ ರಾಜಕೀಯವನ್ನು ಬೋಧಿಸಲು ಹೋದರೆ ಅವರಿಗೆ ಅದು ಅರ್ಥವಾಗುವುದಿಲ್ಲ. ಆದರೆ ನಾನು ಅವರಿಗೆ ವೇದಾಂತವನ್ನು ಬೋಧಿಸಲು ಹೋದರೆ, ‘ಸ್ವಾಮೀಜಿ, ಈಗ ಸರಿ’ ಎನ್ನುವರು. ಇಂದಿಗೂ ವೈರಾಗ್ಯ ಇಂಡಿಯಾ ದೇಶದಲ್ಲೆಲ್ಲಾ ಇರುವುದು. ಈಗ ನಾವು ಬಹಳ ಹೀನಸ್ಥಿತಿಗೆ ಇಳಿದಿರುವೆವು. ಆದರೂ ರಾಜರು ಸಿಂಹಾಸನವನ್ನು ತೊರೆದು ಬೇಕಾದರೆ ಭಿಕಾರಿಗಳಾಗಿ ಅಲೆದಾಡುವರು.”

ಕೆಲವು ಕಡೆ ಚರಕದಲ್ಲಿ ನೂಲುತ್ತಿರುವ ಸಾಧಾರಣ ಹಳ್ಳಿಯ ಹುಡುಗಿ ‘ನನಗೆ ದ್ವೈತವನ್ನು ಬೋಧಿಸಬೇಡಿ. ಚರಕದ ಗಾಲಿ ‘ಸೋಽಹಂ ಸೋಽಹಂ’ (ನಾನೇ ಅದು) ಎನ್ನುತ್ತಿದೆ ಎನ್ನುವಳು.’ ನೀವು ಅವರ ಬಳಿಗೆ ಹೋಗಿ, ‘ಅದ್ವೈತವನ್ನು ಮಾತನಾಡಿದರೂ ಆ ಕಲ್ಲಿನೆದುರಿಗೆ ಏತಕ್ಕೆ ನಮಸ್ಕರಿಸುತ್ತೀರಿ’ ಎಂದು ಕೇಳಿ. ಅವರು `ನಿಮಗೆ ಧರ್ಮವೆಂದರೆ ಯಾವುದನ್ನೋ ನಂಬುವುದಾಗಿದೆ, ಆದರೆ ನಮಗೆ ಧರ್ಮ ಎಂದರೆ ಸಾಕ್ಷಾತ್ಕಾರ’ ಎನ್ನುವರು. ಮಿಥ್ಯಾಪ್ರಪಂಚವೆಲ್ಲ ಮಾಯವಾದಾಗ, ನಾನು ಸತ್ಯವನ್ನು ಕಂಡಾಗ ಮಾತ್ರ, ನಾನು ವೇದಾಂತಿ. ಅಲ್ಲಿಯವರೆಗೂ ಪಾಮರನಿಗೂ ನನಗೂ ಏನೂ ವ್ಯತ್ಯಾಸವಿಲ್ಲ. ಅದಕ್ಕೆ ನಾನು ಈ ಕಲ್ಲಿಗೆ ನಮಸ್ಕಾರ ಮಾಡುವುದು, ದೇವಸ್ಥಾನಕ್ಕೆ ಹೋಗುವುದು, ಆ ಪರಮ ಸತ್ಯವನ್ನು ತಿಳಿದುಕೊಳ್ಳುವುದಕ್ಕಾಗಿ. ನಾನು ಅದನ್ನು ಕೇಳಿರುವೆನು, ಆದರೆ ಅದನ್ನು ನೋಡಬೇಕಾಗಿದೆ, ಅನುಭವಿಸಬೇಕಾಗಿದೆ, ಎನ್ನುವನು. ‘ಚಮತ್ಕಾರವಾಗಿ ಮಾತನಾಡುವುದು, ಹಲವು ಬಗೆಯ ಭಾಷ್ಯಗಳನ್ನು ಶಾಸ್ತ್ರಕ್ಕೆ ಕೊಡುವುದು, ಇವೆಲ್ಲಾ ಕೇವಲ ಪಂಡಿತರಂಜನೆಗೇ ಹೊರತು ಮುಕ್ತಿಗಲ್ಲ’ (ಶಂಕರ). ಸಾಕ್ಷಾತ್ಕಾರವೇ ನಮಗೆ ಆ ಮುಕ್ತಿಯನ್ನು ಕೊಡಬೇಕಾಗಿದೆ.

ಪ್ರಶ್ನೆ: ಜಾತಿ ಭಾವನೆಯೊಂದಿಗೆ ಈ ಆಧ್ಯಾತ್ಮಿಕ ಸ್ವಾತಂತ್ರ್ಯವು ಹೊಂದಿಕೊಳ್ಳ ಬಲ್ಲುದೆ?

ಸ್ವಾಮೀಜಿ: ಎಂದಿಗೂ ಇಲ್ಲ. ಜಾತಿ ಭೇದ ಎಂದಿಗೂ ಇರಕೂಡದು ಎನ್ನುವರು. ಯಾರು ಜಾತಿಯಲ್ಲಿರುವರೋ ಅವರು ಕೂಡ ಜಾತಿಯ ಬಗ್ಗೆ ಎಲ್ಲಾ ಒಳ್ಳೆಯದು ಎನ್ನುವುದಿಲ್ಲ. ಇದಕ್ಕಿಂತ ಮೇಲಾಗಿರುವುದನ್ನು ನೀವು ತೋರಿದರೆ ಆಗ ಈಗಿರುವುದನ್ನು ಬಿಡುತ್ತೇವೆ ಎನ್ನುತ್ತಾರೆ. ಜಾತಿ ಎಲ್ಲಿ ಇಲ್ಲ? ನಿಮ್ಮ ದೇಶದಲ್ಲೇ ಒಂದು ಜಾತಿಯನ್ನು ಮಾಡಲು ನೀವು ಯಾವಾಗಲೂ ಪ್ರಯತ್ನಿಸುತ್ತಿರುವಿರಿ. ಒಬ್ಬನಿಗೆ ಒಂದು ಚೀಲ ಡಾಲರ್​ ಸಿಕ್ಕಿತೆಂದರೆ ನಾನು ನಾಲ್ಕುನೂರು ಜನ ಶ‍್ರೀಮಂತರಲ್ಲಿ ಒಬ್ಬ ಎನ್ನುವನು. ನಾವು ಮಾತ್ರ ಒಂದು ಸ್ಥಿರವಾದ ಜಾತಿ ವ್ಯವಸ್ಥೆಯನ್ನು ಮಾಡಿರುವೆವು. ಇತರ ದೇಶಗಳು ಅದಕ್ಕಾಗಿ ಹೋರಾಡುತ್ತಿವೆ, ಆದರೆ ಇನ್ನೂ ಆ ಪ್ರಯತ್ನದಲ್ಲಿ ಜಯಿಸಿಲ್ಲ. ನಮ್ಮಲ್ಲಿ ಎಷ್ಟೋ ಮೂಢನಂಬಿಕೆಗಳು ಇವೆ, ಲೋಪ ದೋಷಗಳಿವೆ. ನಿಮ್ಮ ದೇಶದಿಂದ ಇನ್ನಷ್ಟು ಮೂಢನಂಬಿಕೆ ಮತ್ತು ದೋಷಗಳನ್ನು ತೆಗೆದುಕೊಂಡರೆ ಅವು ನಮ್ಮನ್ನು ಉತ್ತಮಸ್ಥಿತಿಗೆ ತರಬಲ್ಲವೆ? ಜಾತಿವ್ಯವಸ್ಥೆ ಇರುವುದರಿಂದ ಈಗ ಮೂವತ್ತು ಕೋಟಿ ಜನಕ್ಕೆ ಊಟ ಸಿಗುತ್ತಿದೆ. ನಿಸ್ಸಂದೇಹವಾಗಿ, ಈ ಸಂಸ್ಥೆಯಲ್ಲಿ ಏನೂ ದೋಷವಿಲ್ಲದೇ ಇಲ್ಲ. ಜಾತಿಪದ್ಧತಿ ಇಲ್ಲದೇ ಇದ್ದರೆ ಓದುವುದಕ್ಕೆ ಸಂಸ್ಕೃತ ಪುಸ್ತಕಗಳೇ ಇರುತ್ತಿರಲಿಲ್ಲ. ಈ ಜಾತಿಪದ್ಧತಿ ಒಂದು ದೊಡ್ಡ ಕೋಟೆಯಂತೆಯೇ ಆಯಿತು. ಹೊರಗಿನಿಂದ ಸೇನಾಸಮೂಹ ಎಷ್ಟು ಇದರ ಮೇಲೆ ಬಿದ್ದರೂ ಅದನ್ನು ಒಡೆಯುವುದಕ್ಕೆ ಆಗಲಿಲ್ಲ. ಜಾತಿವ್ಯವಸ್ಥೆಯ ಆವಶ್ಯಕತೆ ಇನ್ನೂ ಹೋಗಿಲ್ಲ. ಆದ್ದರಿಂದಲೇ ಅದು ಉಳಿದಿರುವುದು. ಈಗಿರುವ ಜಾತಿಪದ್ಧತಿ ಏಳನೂರು ವರುಷಗಳ ಹಿಂದೆ ಇದ್ದ ಜಾತಿಯಂತಲ್ಲ. ಹೊರಗಿನಿಂದ ಬಿದ್ದ ಪ್ರತಿಯೊಂದು ಪೆಟ್ಟೂ ಜಾತಿಪದ್ಧತಿಯನ್ನು ಮತ್ತೂ ಭದ್ರಮಾಡಿದೆ.

ಇಂಡಿಯಾ ದೇಶವೊಂದೇ ಅನ್ಯರನ್ನು ಜಯಿಸುವುದಕ್ಕೆ ಹೊರಗೆ ಹೋಗಿಲ್ಲ ಎಂಬುದು ನಿಮಗೆ ಗೊತ್ತೇ? ಪ್ರಖ್ಯಾತನಾದ ಅಶೋಕ ಚಕ್ರವರ್ತಿ ಮುಂದೆ ಬರುವ ತನ್ನ ವಂಶಜರು ಯಾರೂ ಇತರರನ್ನು ಜಯಿಸುವುದಕ್ಕೆ ಹೋಗಕೂಡದೆಂದು ಸಾರಿದನು. ಇತರರು ನಮಗೆ ಪ್ರಚಾರಕರನ್ನು ಕಳುಹಿಸಬೇಕೆಂದು ಇದ್ದರೆ ಕಳುಹಿಸಿಕೊಡಲಿ. ಆದರೆ ನಮಗೆ ತೊಂದರೆ ಕೊಡದಿರಲಿ. ಎಲ್ಲರೂ ಭರತಖಂಡವನ್ನು ಗೆಲ್ಲುವುದಕ್ಕೆ ಏತಕ್ಕೆ ಬರಬೇಕು? ಇತರ ರಾಷ್ಟ್ರಗಳಿಗೆ ಭಾರತೀಯರು ಏನಾದರೂ ತೊಂದರೆ ಕೊಟ್ಟಿರುವರೆ? ತಮಗೆ ಸಾಧ್ಯವಿದ್ದಷ್ಟು ಇತರರಿಗೆ ಸಹಾಯಮಾಡಿದರು. ಅವರು ಇತರರಿಗೆ ವಿಜ್ಞಾನ, ತತ್ತ್ವ ಮುಂತಾದುವನ್ನು ಬೋಧಿಸಿ, ಭರತಖಂಡದ ಮೇಲೆ ದಂಡಯಾತ್ರೆ ಮಾಡಿದ ಅನಾಗರಿಕರನ್ನು ಸುಸಂಸ್ಕೃತ\break ರನ್ನಾಗಿ ಮಾಡಿದರು. ಆದರೆ ಅದಕ್ಕೆ ಪ್ರತಿಫಲ ಕೊಲೆ, ದರೋಡೆ, ಹೀದನ್​, ಮುಟ್ಠಾಳರು ಎಂಬ ಬೈಗುಳದ ಸುರಿಮಳೆ! ಪಾಶ್ಚಾತ್ಯರು ಭರತಖಂಡದ ವಿಷಯವಾಗಿ ಬರೆದ ಪುಸ್ತಕಗಳನ್ನು ನೋಡಿ; ಪ್ರವಾಸಿಗಳು ಬರೆದ ಕಥೆಗಳನ್ನು ಓದಿ. ಅವರಿಗೆ ಯಾವ ತೊಂದರೆ ಕೊಟ್ಟಿದ್ದಕ್ಕೆ ಈ ಪ್ರತೀಕಾರವನ್ನು ಅವರು ತೀರಿಸಿಕೊಳ್ಳುತ್ತಿರುವರು?

ಪ್ರಶ್ನೆ: ನಾಗರಿಕತೆಯನ್ನು ಕುರಿತಂತೆ ವೇದಾಂತದ ಭಾವನೆ ಏನು?

ಉತ್ತರ: ನೀವೆಲ್ಲಾ ತಾತ್ತ್ವಿಕರು. ಒಬ್ಬ ಶ‍್ರೀಮಂತ ಮತ್ತೊಬ್ಬ ಬಡವನಾದರೆ ಅವರಲ್ಲಿ ವ್ಯತ್ಯಾಸವಿದೆ ಎಂದು ನೀವು ಭಾವಿಸುವುದಿಲ್ಲ. ಈ ವಿಜ್ಞಾನ ಮತ್ತು ಯಂತ್ರ ಸಲಕರಣೆಗಳಿಂದ ಪ್ರಯೋಜನವೇನು? ಇವುಗಳಿಂದೆಲ್ಲ ಒಂದು ಪ್ರಯೋಜನವಿದೆ. ಅದೇ ಸ್ವಲ್ಪ ಜ್ಞಾನವನ್ನು ಹರಡುವುದು. ಆದರೆ ನೀವು ಯಂತ್ರಗಳಿಂದ ಬಯಕೆಗಳನ್ನು ತೃಪ್ತಿಪಡಿಸಲಿಲ್ಲ, ಅದನ್ನು ಇನ್ನೂ ವೃದ್ಧಿ ಮಾಡಿರುವಿರಿ. ಹೀಗಾಗಿ ಯಂತ್ರಗಳು ದಾರಿದ್ರ್ಯದ ಸಮಸ್ಯೆಯನ್ನು ಪರಿಹರಿಸುವುದಿಲ್ಲ. ಮನುಷ್ಯನ ಹೋರಾಟವನ್ನು ಮತ್ತೂ ಹೆಚ್ಚು ಮಾಡುವುದು, ಸ್ಪರ್ಧೆ ಹೆಚ್ಚಾಗುವುದು. ಪ್ರಕೃತಿಗೆ ತನ್ನದೇ ಆದ ಯಾವ ಮೌಲ್ಯವಿದೆ? ಒಂದು ತಂತಿ ಮೂಲಕ ವಿದ್ಯುತ್​ ಶಕ್ತಿಯನ್ನು ಕಳುಹಿಸುವ ಒಬ್ಬ ವಿಜ್ಞಾನಿಗೆ ನೀವು ಸ್ಮಾರಕವನ್ನು ಏತಕ್ಕೆ ಕಟ್ಟುವುದಕ್ಕೆ ಹೋಗುತ್ತೀರಿ? ಪ್ರಕೃತಿ ಇದಕ್ಕೆ ಲಕ್ಷ ಪಾಲಷ್ಟು ಮಾಡುವುದಿಲ್ಲವೆ? ಪ್ರಕೃತಿಯಲ್ಲಿ ಆಗಲೆ ಎಲ್ಲವೂ ಇದೆಯಲ್ಲವೆ? ಇದನ್ನು ಪಡೆದು ಪ್ರಯೋಜನವೇನು? ಅದಾಗಲೇ ಅಲ್ಲಿರುವುದು. ಅದು ಅಭಿವೃದ್ಧಿಗೆ ಸಹಕಾರಿಯಷ್ಟೆ. ಜೀವವು ಅಂಗಸಾಧನೆಯನ್ನು ಮಾಡುತ್ತಿರುವ ಒಂದು ಗರಡಿಯ ಮನೆ ಈ ಪ್ರಪಂಚ. ಈ ಅಂಗಸಾಧನೆ ಆದ ಮೇಲೆ ನಾವು ದೇವರಾಗುವೆವು. ಆದಕಾರಣ ಎಲ್ಲದರ ಪ್ರಯೋಜನವನ್ನೂ ಅವು ಎಷ್ಟು ಮಟ್ಟಿಗೆ ನಮ್ಮನ್ನು ದೇವತೆಗಳನ್ನಾಗಿ ಮಾಡಬಹುದು ಎಂಬುದರ ಮೇಲೆ ನಿಷ್ಕರ್ಷಿಸಬೇಕಾಗಿದೆ. ಮನುಷ್ಯನಲ್ಲಿ ಆಗಲೇ ಸುಪ್ತವಾಗಿರುವ ಪಾವಿತ್ರ್ಯವನ್ನು ಯಾವುದು ವ್ಯಕ್ತಗೊಳಿಸುವುದೋ ಅದೇ ಸಂಸ್ಕೃತಿ.”

ಪ್ರಶ್ನೆ: ಬೌದ್ಧರಲ್ಲಿ ಏನಾದರೂ ಜಾತಿಯ ಕಟ್ಟುನಿಟ್ಟುಗಳಿವೆಯೆ?

ಸ್ವಾಮೀಜಿ: ಬೌದ್ಧರಲ್ಲಿ ಹೆಚ್ಚು ಜಾತಿಗಳು ಇರಲಿಲ್ಲ. ಇಂಡಿಯಾ ದೇಶದಲ್ಲಿ ಬೌದ್ಧರು ಇರುವುದು ಬಹಳ ಕಡಿಮೆ. ಬುದ್ಧ ಒಬ್ಬ ಸಮಾಜ ಸುಧಾರಕ. ಆದರೂ ಬೌದ್ಧದೇಶಗಳಲ್ಲಿ ಜಾತಿಯನ್ನು ನಿರ್ಮಿಸಲು ಬಲವಾದ ಪ್ರಯತ್ನಗಳಾಗಿವೆ. ಅದರಲ್ಲಿ ಯಶಸ್ವಿಗಳಾಗಲಿಲ್ಲ, ಅಷ್ಟೆ. ಬೌದ್ಧರಲ್ಲಿ ಜಾತಿ ಎಂಬುದೇ ಇಲ್ಲ ಎನ್ನಬಹುದು. ಆದರೆ ಇದನ್ನು ಕುರಿತೇ ಅವರು ಹೆಮ್ಮೆಪಡುವರು.

ಬುದ್ಧ ವೇದಾಂತ ಸಂನ್ಯಾಸಿಗಳಲ್ಲಿ ಒಬ್ಬ. ಈಚೆಗೆ ಹೇಗೆ ಜನ ಬೇರೆ ಪಂಥಗಳನ್ನು ಪ್ರಾರಂಭಮಾಡುತ್ತಿರುವರೋ ಹಾಗೆಯೇ ಅವನೂ ಕೂಡ ಬೇರೊಂದು ಪಂಥವನ್ನು ಸ್ಥಾಪಿಸಿದನು. ಯಾವುದನ್ನು ಈಗ ಬೌದ್ಧಧರ್ಮ ಎನ್ನುವರೋ ಇದು ಅವನದಲ್ಲ. ಇದು ಅವನಿಗಿಂತಲೂ ಪುರಾತನವಾದುದು. ಬುದ್ಧ ಮಹಾನ್​ವ್ಯಕ್ತಿ. ಅವನು ಈ ಭಾವನೆಗೆ ಒಂದು ಶಕ್ತಿಯನ್ನು ತುಂಬಿದ. ಬೌದ್ಧರ ಒಂದು ವಿಶೇಷವೆ ಅವರ ಸಾಮಾಜಿಕ ಅಂಶ. ಬ್ರಾಹ್ಮಣ ಮತ್ತು ಕ್ಷತ್ರಿಯರು ಯಾವಾಗಲೂ ನಮ್ಮ ಗುರುಗಳಾಗಿದ್ದರು. ಅನೇಕ ಉಪನಿಷತ್ತುಗಳನ್ನು ಕ್ಷತ್ರಿಯರೇ ಬರೆದಿರುವರು. ವೇದಗಳ ಕರ್ಮಕಾಂಡ ಮಾತ್ರ ಬ್ರಾಹ್ಮಣರಿಂದ ಬಂದಿತು. ಇಂಡಿಯಾ ದೇಶದ ಅನೇಕ ಗುರುಗಳು ಕ್ಷತ್ರಿಯರಾಗಿದ್ದರು. ಅವರ ಬೋಧನೆ ಯಾವಾಗಲೂ ಉದಾರವಾಗಿತ್ತು. ಇಬ್ಬರು ಬ್ರಾಹ್ಮಣರು ವಿನಃ ಉಳಿದ ಬ್ರಾಹ್ಮಣ ಪ್ರವಾದಿ\break ಗಳ ಬೋಧನೆಯೆಲ್ಲ ಕೆಲವು ಪಂಥಗಳಿಗೆ ಮಾತ್ರ ಅನ್ವಯಿಸುವುದಾಗಿತ್ತು. ಇಂದು ದೇವರ ಅವತಾರವೆಂದು ಪೂಜಿಸುತ್ತಿರುವ ರಾಮ, ಕೃಷ್ಣ, ಬುದ್ಧ ಮುಂತಾದವರೆಲ್ಲ ಕ್ಷತ್ರಿಯರು.

ಪ್ರಶ್ನೆ: ಜಾತಿ, ಆಚಾರ, ಶಾಸ್ತ್ರ ಇವು ಆತ್ಮಸಾಕ್ಷಾತ್ಕಾರಕ್ಕೆ ಸಹಾಯ ಮಾಡುವುವೆ?

ಸ್ವಾಮೀಜಿ: ಒಬ್ಬನಿಗೆ ಸಾಕ್ಷಾತ್ಕಾರವಾದ ಮೇಲೆ ಅವನ್ನೆಲ್ಲಾ ತ್ಯಜಿಸುವನು. ಎಲ್ಲಿಯವರೆಗೂ ಜಾತಿ ಆಚಾರ ಗ್ರಂಥಗಳೆಲ್ಲ ಒಬ್ಬನಿಗೆ ಸಹಾಯ ಮಾಡುವುವೋ ಅಲ್ಲಿಯವರೆಗೆ ಅವು ಒಳ್ಳೆಯವು. ಆದರೆ ಯಾವಾಗ ಇವು ಆತಂಕವಾಗುವುವೊ ಆಗ ನಾವು ಅವನ್ನು ಬದಲಾಯಿಸಬೇಕಾಗಿದೆ. “ಜ್ಞಾನಿಗಳು ಅಜ್ಞಾನಿಗಳು ಮಾಡುತ್ತಿರುವುದನ್ನು ಎಂದಿಗೂ ಹಳಿಯಕೂಡದು, ಅಥವಾ ಅವರಲ್ಲಿರುವ ಶ್ರದ್ಧೆಗೆ ಭಂಗ ತರಕೂಡದು. ಜ್ಞಾನಿಗಳು ತಮ್ಮ ಕರ್ಮದ ಮೂಲಕ ಅವರನ್ನು ಸರಿಯಾದ ಮಾರ್ಗಕ್ಕೆ ತರಬೇಕು.”

ಪ್ರಶ್ನೆ: ವೇದಾಂತವು ವ್ಯಕ್ತಿತ್ವ ಮತ್ತು ನೀತಿ ಇವೆರಡನ್ನೂ ಹೇಗೆ ವಿವರಿಸುವುದು?

ಸ್ವಾಮೀಜಿ: ನಿಜವಾದ ವ್ಯಕ್ತಿತ್ವವೇ ಬ್ರಹ್ಮ. ಈಗಿರುವ ವ್ಯಕ್ತೀಕರಣವು ಮಾಯೆಯ ಮೂಲಕ ಆಗಿದೆ. ಇದು ಕೇವಲ ತೋರಿಕೆ. ನಿಜವಾಗಿ ಇದು ಬ್ರಹ್ಮವೇ ಆಗಿದೆ. ನಿಜವಾಗಿ ಇರುವುದೊಂದು, ಆದರೆ ಮಾಯೆಯಲ್ಲಿ ಅದು ಅನೇಕವಾದಂತೆ ಕಾಣುತ್ತಿದೆ. ವೈವಿಧ್ಯ ಇರುವುದು ಮಾಯೆಯಲ್ಲಿ, ಈ ಮಾಯೆಯಲ್ಲಿದ್ದರೂ ಅದು ಆ ಏಕಮೇವಾದ್ವಿತೀಯಕ್ಕೆ ಹಿಂತಿರುಗುವುದಕ್ಕೆ ಯತ್ನಿಸುತ್ತಿದೆ. ಪ್ರತಿಯೊಂದು ದೇಶದ ನೀತಿ ಮತ್ತು ಧರ್ಮ ಇದನ್ನು ವಿವರಿಸುತ್ತದೆ. ಏಕೆಂದರೆ ಅದು ಆತ್ಮನ ನಿಜವಾದ ಆವಶ್ಯಕತೆಯಾಗಿದೆ. ಅದು ತನ್ನ ಐಕ್ಯತೆಯನ್ನು ಅರಸುತ್ತಿದೆ. ಆ ಐಕ್ಯತೆಯನ್ನು ಪಡೆಯುವ ಹೋರಾಟವನ್ನೇ ನಾವು ನೀತಿ ಎನ್ನುವುದು. ಆದಕಾರಣವೆ ನಾವು ಯಾವಾಗಲೂ ಇದನ್ನು ಅಭ್ಯಾಸ ಮಾಡಬೇಕು.

ಪ್ರಶ್ನೆ: ನೀತಿಯ ಬಹುಭಾಗ ವ್ಯಕ್ತಿಗಳ ಪರಸ್ಪರ ಸಂಬಂಧಕ್ಕೆ ಸೇರಿಲ್ಲವೆ?

ಸ್ವಾಮೀಜಿ: ನೀತಿ ಎಂದರೆ ಅದೇ. ಬ್ರಹ್ಮ ಮಾಯೆಯ ಒಳಗೆ ಬರುವುದಿಲ್ಲ.

ಪ್ರಶ್ನೆ: ನೀವು ಜೀವನೇ ಬ್ರಹ್ಮ ಎಂದಿರಿ. ಆದರೆ ಜೀವನಿಗೆ ಜ್ಞಾನವಿದೆಯೆ ಎಂಬುದನ್ನು ನಾನು ಕೇಳಬೇಕೆಂದು ಇದ್ದೆ.

ಸ್ವಾಮೀಜಿ: ಆವಿರ್ಭಾವದ ಸ್ಥಿತಿಯಲ್ಲಿ ವ್ಯಕ್ತಿತ್ವವಿರುವುದು. ಈ ಸ್ಥಿತಿಯಲ್ಲಿರುವ ಜ್ಯೋತಿ\break ಯನ್ನೇ ಜ್ಞಾನವೆನ್ನುವುದು. ಈ ಜ್ಞಾನವನ್ನು ಬ್ರಹ್ಮಜ್ಞಾನ ಎನ್ನುವುದು ಸರಿಯಲ್ಲ. ಬ್ರಹ್ಮ ಸಾಪೇಕ್ಷ ಜ್ಞಾನಕ್ಕೆ ಅತೀತವಾಗಿರುವನು.

ಪ್ರಶ್ನೆ: ಆ ಸ್ಥಿತಿ ಸಾಪೇಕ್ಷ ಜ್ಞಾನವನ್ನು ಒಳಗೊಂಡಿರುವುದೆ?

ಸ್ವಾಮೀಜಿ: ಹೌದು, ಈ ದೃಷ್ಟಿಯಲ್ಲಿ-ಒಂದು ಚೂರು ಚಿನ್ನದಿಂದ ಬೇರೆಬೇರೆ ಒಡವೆಗಳನ್ನು ಮಾಡುವಂತೆ, ಈ ಸ್ಥಿತಿಯನ್ನು ಹಲವು ಜ್ಞಾನಕ್ಷೇತ್ರಗಳಾಗಿ ಮಾಡಬಹುದು. ಇದು ಸಮಾಧಿಯ ಸ್ಥಿತಿ. ಇಲ್ಲಿ ಪ್ರಜ್ಞೆ ಮತ್ತು ಅಪ್ರಜ್ಞೆಗಳೆರಡೂ ಇರುವುವು. ಈ ಸ್ಥಿತಿ ಪಡೆದವನಿಗೆ ಜ್ಞಾನವೆಂಬುದೆಲ್ಲ ಇದೆ. ಅವನು ಆ ಜ್ಞಾನದ ಪ್ರಜ್ಞೆಯನ್ನು ಪಡೆಯಬೇಕಾದರೆ ಒಂದು ಮೆಟ್ಟಲು ಕೆಳಗೆ ಬರಬೇಕು. ಜ್ಞಾನವು ಕೆಳಗಿನ ಸ್ಥಿತಿ. ಮಾಯೆಯಲ್ಲಿ ಮಾತ್ರ ಜ್ಞಾನವು ದೊರಕುವುದು.


\section[ಅಮೇರಿಕದ ಬೋಸ್ಟನಿನ ಟ್ವೆಂಟಿಯತ್ ಸೆಂಚುರಿ ಕ್ಲಬ್ಬಿನಲ್ಲಿ]{ಅಮೇರಿಕದ ಬೋಸ್ಟನಿನ ಟ್ವೆಂಟಿಯತ್ ಸೆಂಚುರಿ ಕ್ಲಬ್ಬಿನಲ್ಲಿ\protect\footnote{* C.W. Vol. V P. 310}}

ಪ್ರಶ್ನೆ: ವೇದಾಂತವು ಮಹಮ್ಮದೀಯರ ಮೇಲೆ ಯಾವ ಪ್ರಭಾವವನ್ನಾದರೂ ಬೀರಿದೆಯೆ?

ಸ್ವಾಮೀಜಿ: ವೇದಾಂತ ಧಾರ್ಮಿಕ ಔದಾರ್ಯವು ಮಹಮ್ಮದೀಯರ ಮೇಲೆ ದೊಡ್ಡ ಪರಿಣಾಮವನ್ನು ಉಂಟುಮಾಡಿದೆ. ಇಂಡಿಯಾ ದೇಶದ ಮಹಮ್ಮದೀಯ ಧರ್ಮ ಇತರ ದೇಶದ ಮಹಮ್ಮದೀಯ ಧರ್ಮದಂತೆ ಅಲ್ಲ. ಹೊರಗಿನಿಂದ ಬಂದ ಮಹಮ್ಮದೀಯರು ಇಲ್ಲಿರುವವರಿಗೆ, ಇತರ ಧರ್ಮದವರೊಡನೆ ಹಾಗೆ ಸೌಹಾರ್ದದಿಂದ ಇರಬೇಡಿ ಎಂದಾಗಲೇ, ಮಹಮ್ಮದೀಯರ ಗುಂಪು ಧರ್ಮದ ಅಮಲಿನಲ್ಲಿ ಹುಚ್ಚರಾಗಿ ಹೋರಾಡುವುದು.

ಪ್ರಶ್ನೆ: ವೇದಾಂತ ಜಾತಿಯನ್ನು ಒಪ್ಪಿಕೊಳ್ಳುವುದೆ?

ಸ್ವಾಮೀಜಿ: ಜಾತಿಪದ್ಧತಿ ವೇದಾಂತಕ್ಕೆ ವಿರುದ್ಧವಾದುದು. ಜಾತಿ ಕೇವಲ ಸಾಮಾಜಿಕ ಆಚಾರ. ನಮ್ಮ ಮಹಾಗುರುಗಳೆಲ್ಲ ಅದನ್ನು ನಾಶಮಾಡಲು ಯತ್ನಿಸಿರುವರು. ಬೌದ್ಧರಿಂದ\break ಪ್ರಾರಂಭಿಸಿ ಇದುವರೆಗೆ ಎಲ್ಲರೂ ಜಾತಿಯನ್ನು ಹಳಿದಿರುವರು. ಆದರೆ ಪ್ರತಿಸಲವೂ ಜಾತಿಯ ಬಂಧನ ಬಿಗಿಯಾಗುತ್ತಾ ಬರುತ್ತಿದೆ. ಜಾತಿಯು ರಾಜಕೀಯ ಸಂಸ್ಥೆಯ ಒಂದು ಬೆಳವಣಿಗೆ. ಇದು ಆನುವಂಶಿಕವಾಗಿ ಬಂದ ವ್ಯಾಪಾರ ಸಂಸ್ಥೆ. ಯೂರೋಪಿನೊಂದಿಗೆ ಆಗುತ್ತಿರುವ ವ್ಯಾಪಾರ ಸ್ಪರ್ಧೆ ಎಲ್ಲಾ ಬೋಧನೆಗಳಿಗಿಂತ ಹೆಚ್ಚಾಗಿ ಜಾತಿಪದ್ಧತಿಯನ್ನು ನಿರ್ಮೂಲ ಮಾಡಿದೆ.

ಪ್ರಶ್ನೆ: ವೇದಗಳ ವೈಶಿಷ್ಟ್ಯ ಏನು?

ಸ್ವಾಮೀಜಿ: ಅವುಗಳ ಒಂದು ವೈಶಿಷ್ಟ್ಯವೇ, ಅವು ಮಾತ್ರವೇ ತಮ್ಮನ್ನೂ ಮೀರಿ ಹೋಗಿ ಎಂದು ಬೋಧಿಸುವುದು. ತಾವು ಕೇವಲ ಬಾಲಬುದ್ಧಿಯವರಿಗಾಗಿ ಮಾತ್ರ ಬರೆಯಲ್ಪಟ್ಟವುಗಳೆಂದು ಎನ್ನುವುವು. ನೀವು ಬೆಳೆದಂತೆ ಅವನ್ನೂ ಮೀರಿ ಹೋಗಬೇಕು.

ಪ್ರಶ್ನೆ: ಜೀವ ಯಾವಾಗಲೂ ಸತ್ಯ ಎಂದು ಭಾವಿಸುವಿರಾ?

ಸ್ವಾಮೀಜಿ: ಮಾನವನ ಜೀವ ಅದರ ಆಲೋಚನೆಗಳಿಂದ ಕೂಡಿದೆ. ಅವು ಪ್ರತಿಕ್ಷಣವೂ ಬದಲಾಗುತ್ತಿವೆ. ಆದಕಾರಣ ಅವು ನಿತ್ಯಸತ್ಯವಾಗಿರಲಾರವು. ಇವು ಕೇವಲ ತೋರಿಕೆಗೆ ಮಾತ್ರ ಸತ್ಯ. ಜೀವದಲ್ಲಿ ಸ್ಮೃತಿ ಮತ್ತು ಆಲೋಚನೆಗಳು ಇವೆ. ಇದು ಹೇಗೆ ಸತ್ಯವಾಗಬಲ್ಲದು?

ಪ್ರಶ್ನೆ: ಬೌದ್ಧಧರ್ಮ ಏತಕ್ಕೆ ಇಂಡಿಯಾದೇಶದಲ್ಲಿ ಅವನತಿ ಹೊಂದಿತು?

ಸ್ವಾಮೀಜಿ: ಬೌದ್ಧಧರ್ಮ ನಿಜವಾಗಿ ಅವನತಿಗೆ ಬರಲಿಲ್ಲ. ಅದೊಂದು ಅದ್ಭುತವಾದ ಸಾಮಾಜಿಕ ಚಳುವಳಿ. ಬುದ್ಧನಿಗೆ ಮುಂಚೆ ಯಾಗ ಮುಂತಾದುವುಗಳಲ್ಲಿ ಹಲವು ಪ್ರಾಣಿಗಳನ್ನು ಬಲಿಕೊಡುತ್ತಿದ್ದರು. ಜನರು ಮದ್ಯಪಾನ ಮಾಡುತ್ತಿದ್ದರು, ಮಾಂಸ ತಿನ್ನುತ್ತಿದ್ದರು. ಬುದ್ಧನ ಬೋಧನಾನಂತರ ಮದ್ಯಪಾನ ಮತ್ತು ಪ್ರಾಣಿವಧೆ ಸಂಪೂರ್ಣವಾಗಿ ನಿಂತು\break ಹೋದವು ಎನ್ನಬಹುದು.


\section[ಬ್ರುಕ್‍ಲಿನ್ ಎಥಿಕಲ್ ಸೊಸೈಟಿಯಲ್ಲಿ]{ಬ್ರುಕ್‍ಲಿನ್ ಎಥಿಕಲ್ ಸೊಸೈಟಿಯಲ್ಲಿ\protect\footnote{* C.W. Vol. V P. 312}}

\begin{center}
(ಬ್ರುಕ್​ಲಿನ್​, ಯು.ಎನ್​.ಎ.)
\end{center}

ಪ್ರಶ್ನೆ: ಪ್ರಪಂಚದಲ್ಲಿ ಪಾಪವಿದ್ದರೂ, ದುಃಖಸಂಕಟಗಳಿದ್ದರೂ ಹೇಗೆ ನೀವು ಆಶಾ\break ಭಾವನೆಯನ್ನು ಇಟ್ಟುಕೊಂಡಿರಬಲ್ಲಿರಿ?

ಸ್ವಾಮೀಜಿ: ಪಾಪ ಇದೆ ಎಂಬುದನ್ನು ನೀವು ಮೊದಲು ದೃಢಪಡಿಸಿದ ಮೇಲೆ ನಾನು ಉತ್ತರ ಹೇಳುವೆನು. ಆದರೆ ವೇದಾಂತ ಪಾಪವಿದೆ ಎಂಬುದನ್ನು ಒಪ್ಪಿಕೊಳ್ಳುವುದಿಲ್ಲ. ಸುಖವಿಲ್ಲದ ನಿತ್ಯವ್ಯಥೆ ನಿಜವಾಗಿಯೂ ಪಾಪವೇನೋ ಹೌದು. ಆದರೆ ತಾತ್ಕಾಲಿಕ ನೋವು ಮತ್ತು ವ್ಯಥೆಗಳು ಒಬ್ಬನನ್ನು ಮಾರ್ದವಗೊಳಿಸಿ ಉದಾತ್ತ ಜೀವಿಯನ್ನಾಗಿ ಮಾಡಿ ನಿತ್ಯಾನಂದದ ಕಡೆಗೆ ಸಹಾಯ ಮಾಡಿದರೆ ಅವು ಪಾಪವಲ್ಲ. ಅದರ ಬದಲು ಅವು ಅತ್ಯಂತ ಶ್ರೇಷ್ಠ ಎನ್ನಬಹುದು. ಅದು ನಮ್ಮನ್ನು ಅನಂತತೆಯ ಕಡೆಗೆ ಒಯ್ಯುವುದಿಲ್ಲ ಎಂದು ಗೊತ್ತಾಗದೆ ಯಾವುದನ್ನೂ ಪಾಪ ಎನ್ನಲಾಗುವುದಿಲ್ಲ.

ಭೂತಪ್ರೇತಗಳ ಆರಾಧನೆ ಹಿಂದೂಧರ್ಮದ ಅಂಶವಲ್ಲ. ಮಾನವಕೋಟಿ ಇನ್ನೂ ಮುಂದುವರಿಯುತ್ತಿರುವ ಸ್ಥಿತಿಯಲ್ಲಿದೆ. ಎಲ್ಲರೂ ಒಂದೇ ಮಟ್ಟಕ್ಕೆ ಸೇರಿಲ್ಲ. ಆದಕಾರಣವೆ ಪ್ರಪಂಚದಲ್ಲಿ ಕೆಲವರು ಇತರರಿಗಿಂತ ಹೆಚ್ಚು ಪರಿಶುದ್ಧರಾಗಿರುವರು, ಉದಾರಿಗಳಾಗಿರು\break ವರು. ಪ್ರತಿಯೊಬ್ಬರಿಗೂ ತಾವು ಇರುವ ವಾತಾವರಣದಲ್ಲಿ ಇನ್ನೂ ಉತ್ತಮರಾಗಲು ಅವಕಾಶವಿರುವುದು. ನಮ್ಮನ್ನು ನಾವು ಅಲ್ಲಗಳೆಯುವುದಕ್ಕೆ ಆಗುವುದಿಲ್ಲ. ನಮ್ಮಲ್ಲಿರುವ ಶಕ್ತಿಯನ್ನು ನಾಶಮಾಡುವುದಕ್ಕಾಗಲೀ ಕುಂಠಿತ ಮಾಡುವುದಕ್ಕಾಗಲೀ ಆಗುವುದಿಲ್ಲ. ಆದರೆ ಅದನ್ನು ಬೇರೊಂದು ಮಾರ್ಗಕ್ಕೆ ತಿರುಗಿಸುವ ಅಧಿಕಾರ ನಮಗಿದೆ.

ಪ್ರಶ್ನೆ: ಸರ್ವವ್ಯಾಪಕವಾದ ಸತ್ಯ ಒಂದು ಇದೆ ಎಂಬುದು ಕೇವಲ ನಮ್ಮ ಮನಸ್ಸಿನ ಭ್ರಾಂತಿಯಲ್ಲವೆ?

ಸ್ವಾಮೀಜಿ: ನನ್ನ ದೃಷ್ಟಿಯಲ್ಲಿ ಬಾಹ್ಯಜಗತ್ತು ನಿಜವಾಗಿ ಇರುವುದು. ನಮ್ಮ ಮನಸ್ಸಿಗಿಂತ ಪ್ರತ್ಯೇಕವಾಗಿ ಅದು ಇರುವುದು. ಸೃಷ್ಟಿಯೆಲ್ಲಾ ವಿಕಾಸವಾದದ ನಿಯಮದಂತೆ ಮುಂದೆ ಮುಂದೆ ಮೇಲು ಮೇಲಕ್ಕೆ ಹೋಗುತ್ತಿದೆ. ಇದು ಭೌತಿಕಪರಿಣಾಮದಂತೆ ಅಲ್ಲ. ಭೌತಿಕಪರಿಣಾಮ ಚೇತನದ ಪರಿಣಾಮಕ್ಕೆ ಒಂದು ಉದಾಹರಣೆಯಾಗಿದೆಯೇ ಹೊರತು, ಅದು ಚೇತನದ ಪರಿಣಾಮವನ್ನು ವಿವರಿಸಲಾರದು. ನಾವೀಗ ಇರುವ ಪ್ರಪಂಚದ ವಾತಾವರಣದಲ್ಲಿ ನಾವಿನ್ನೂ ನಿಜವಾದ ವ್ಯಕ್ತಿಗಳಾಗಿಲ್ಲ. ನಾವು ಮೇಲಿನ ಮಟ್ಟಕ್ಕೆ ಏರುವ ತನಕ, ನಮ್ಮಲ್ಲಿರುವ ಅಧ್ಯಾತ್ಮ ವ್ಯಕ್ತವಾಗುವುದಕ್ಕೆ ಒಂದು ಪರಿಶುದ್ಧವಾದ ಮಾಧ್ಯಮವು ಸಿಕ್ಕುವ ತನಕ, ನಮಗೆ ವ್ಯಕ್ತಿತ್ವ ಬಂದಿದೆ ಎನ್ನಲಾಗುವುದಿಲ್ಲ.

ಪ್ರಶ್ನೆ:ಕುರುಡಾಗಿ ಹುಟ್ಟಿರುವುದು ಮಗುವಿನ ಪಾಪವೆ ಅಥವಾ ಅದರ ತಂದೆತಾಯಿಗಳ ಪಾಪವೆ ಎಂದು ಕ್ರಿಸ್ತನನ್ನು ಕೇಳಿದ ಪ್ರಶ್ನೆಗೆ ನೀವು ಏನು ಉತ್ತರವನ್ನು ಕೊಡುತ್ತೀರಿ?

ಸ್ವಾಮೀಜಿ: ಇಲ್ಲಿ ಪಾಪದ ಪ್ರಶ್ನೆಯೇನೂ ಏಳುವುದಿಲ್ಲ. ಆ ಮಗು ತನ್ನ ಹಿಂದಿನ ಜನ್ಮದಲ್ಲಿ ಏನನ್ನೊ ಮಾಡಿದುದರ ಕರ್ಮವಿರಬೇಕು. ನನ್ನ ದೃಷ್ಟಿಯಲ್ಲಿ ಇದನ್ನು ವಿವರಿಸಬೇಕಾದರೆ ಪೂರ್ವಜನ್ಮವನ್ನು ಒಪ್ಪಿಕೊಂಡರೆ ಮಾತ್ರ ಸಾಧ್ಯ.

ಪ್ರಶ್ನೆ: ನಮ್ಮ ಜೀವವು ಮರಣದ ಸಮಯದಲ್ಲಿ ಆನಂದವನ್ನು ಅನುಭವಿಸುವುದೆ?

ಸ್ವಾಮೀಜಿ: ಸಾವು ಕೇವಲ ನಮ್ಮ ಸ್ಥಿತಿಯ ಬದಲಾವಣೆ ಅಷ್ಟೆ. ಕಾಲ ದೇಶಗಳು ನಮ್ಮಲ್ಲಿವೆ; ನಾವು ಕಾಲದೇಶದಲ್ಲಿ ಇಲ್ಲ. ನಮ್ಮ ಜೀವನವನ್ನು ಈಗ ನಮಗೆ ಕಾಣಿಸುವ ಅಥವಾ ಕಾಣಿಸದಿರುವ ಜಗತ್ತಿನಲ್ಲಿ ಪರಿಶುದ್ಧ ಮಾಡಿದಷ್ಟೂ ಎಲ್ಲಾ ಸೌಂದರ್ಯದ ಮತ್ತು ಆನಂದದ ಕೇಂದ್ರವಾದ ಭಗವಂತನನ್ನು ಸಮೀಪಿಸುವೆವು ಎಂಬುದನ್ನು ತಿಳಿದರೆ ಸಾಕು.

ಪ್ರಶ್ನೆ: ಹಿಂದೂಗಳ ಜನ್ಮಾಂತರ ಸಿದ್ಧಾಂತ ಹೇಗಿದೆ?

ಸ್ವಾಮೀಜಿ: ಅದು ವಿಜ್ಞಾನಿಯ ಶಕ್ತಿಸಾತತ್ಯ \enginline{(Conservation Of Energy)} ಎಂಬ ನಿಯಮದಂತೆಯೇ ಇದೆ. ಈ ಸಿದ್ಧಾಂತವನ್ನು ನಮ್ಮ ದೇಶದ ದಾರ್ಶನಿಕರೊಬ್ಬರು ಪ್ರಚಾರಕ್ಕೆ ತಂದರು. ಪುರಾತನ ಋಷಿಗಳು ಹೊಸದಾಗಿ ಆದ ಸೃಷ್ಟಿಯನ್ನು ನಂಬಲಿಲ್ಲ. ಸೃಷ್ಟಿ ಎಂದರೆ ಶೂನ್ಯದಿಂದ ಏನೋ ಬರಬಹುದು ಎಂಬ ಅರ್ಥಕ್ಕೆ ಅವಕಾಶವಿದೆ. ಇದೆಂದಿಗೂ ಸಾಧ್ಯವಿಲ್ಲ. ಕಾಲಕ್ಕೆ ಹೇಗೆ ಒಂದು ಆದಿ ಇಲ್ಲವೊ ಹಾಗೆಯೇ ಸೃಷ್ಟಿಗೂ ಒಂದು ಆದಿ ಎಂಬುದಿಲ್ಲ. ದೇವರು ಮತ್ತು ಸೃಷ್ಟಿ ಆದಿ ಅಂತ್ಯ ರಹಿತ; ಅವು ಸಮಾನಾಂತರ ರೇಖೆಗಳಂತೆ ಇವೆ. ನಮ್ಮ ಸೃಷ್ಟಿ ಸಿದ್ಧಾಂತವೇ, ಅದು ಹಿಂದೆ ಇತ್ತು, ಈಗ ಇದೆ, ಮುಂದೆಯೂ ಇರುವುದು ಎಂಬುದು. ಶಿಕ್ಷೆಯಲ್ಲ ಒಂದು ಪ್ರತಿಕ್ರಿಯೆ. ಪಾಶ್ಚಾತ್ಯರು ಇಂಡಿಯಾ ದೇಶದಿಂದ ಒಂದನ್ನು ಕಲಿಯಬೇಕಾಗಿದೆ. ಅದೇ, ಪರಧರ್ಮ ಸಹಿಷ್ಣುತೆ. ಎಲ್ಲಾ ಧರ್ಮಗಳೂ ಒಳ್ಳೆಯವೇ, ಏಕೆಂದರೆ ಅವುಗಳ ಸಾರವೆಲ್ಲ ಒಂದೇ.

ಪ್ರಶ್ನೆ: ಭಾರತೀಯ ಸ್ತ್ರೀಯರನ್ನು ಮೇಲಕ್ಕೆ ಏತಕ್ಕೆ ತಂದಿಲ್ಲ?

ಸ್ವಾಮೀಜಿ: ಇದಕೆಲ್ಲ ಕಾರಣ ಹಲವು ಕಾಲದಿಂದ ಆಗಿರುವ ಅನಾಗರಿಕ ಅನ್ಯದೇಶೀಯರ ಆಕ್ರಮಣ. ಇದು ಸ್ವಲ್ಪಮಟ್ಟಿಗೆ ಭಾರತೀಯರ ತಪ್ಪೂ ಆಗಿದೆ.

ಸ್ವಾಮಿ ವಿವೇಕಾನಂದರೊಡನೆ ಹಿಂದೂಧರ್ಮವು ಅನ್ಯಮತೀಯರನ್ನು ತಮ್ಮ ಧರ್ಮಕ್ಕೆ ಸೇರಿಸಿಕೊಳ್ಳುವ ಧರ್ಮವಲ್ಲ ಎಂದಾಗ ಸ್ವಾಮೀಜಿ ಅವರು ಹೀಗೆಂದರು- “ಬುದ್ಧ ಹೇಗೆ ಪ್ರಾಚ್ಯ ದೇಶಕ್ಕೆ ಒಂದು ಸಂದೇಶವಿತ್ತನೋ ಹಾಗೆ ಪಾಶ್ಚಾತ್ಯ ದೇಶಕ್ಕೆ ನಾನು ಒಂದು ಸಂದೇಶವನ್ನು ಕೊಡುವವನಿದ್ದೇನೆ.

ಪ್ರಶ್ನೆ: ಹಿಂದೂಧರ್ಮದ ಪೂಜಾದಿ ಆಚಾರಗಳನ್ನು ಅಮೇರಿಕಾ ದೇಶದಲ್ಲಿ ಜಾರಿಗೆ ತರುವಿರಾ?

ಸ್ವಾಮೀಜಿ: ನಾನು ಬರಿಯ ತತ್ತ್ವವನ್ನು ಮಾತ್ರ ಬೋಧಿಸುತ್ತೇನೆ.

ಪ್ರಶ್ನೆ: ಮುಂದಿನ ನರಕಭಯವನ್ನು ತೆಗೆದುಬಿಟ್ಟರೆ ಮಾನವನನ್ನು ನಿಗ್ರಹಿಸುವುದಕ್ಕೆ ಆಗುವುದಿಲ್ಲ ಎಂದು ನೀವು ಭಾವಿಸುವುದಿಲ್ಲವೆ?

\newpage

ಸ್ವಾಮೀಜಿ: ಇಲ್ಲ, ನರಕಭಯಕ್ಕಿಂತ ಪ್ರೀತಿಯ ಮತ್ತು ಭರವಸೆಯ ಮೂಲಕ ಅವನು ಹೆಚ್ಚು ಉತ್ತಮನಾಗುತ್ತಾನೆ.


\section[ಯೋಗ, ವೈರಾಗ್ಯ, ತಪಸ್ಸು ಮತ್ತು ಪ್ರೇಮ]{ಯೋಗ, ವೈರಾಗ್ಯ, ತಪಸ್ಸು ಮತ್ತು ಪ್ರೇಮ\protect\footnote{* C.W. Vol. V P. 319}}

ಪ್ರಶ್ನೆ: ಯೋಗ ದೇಹವನ್ನು ದೃಢವಾಗಿ ಆರೋಗ್ಯವಾಗಿ ಇಡಬಲ್ಲದೆ?

ಉತ್ತರ: ಹೌದು ಅದು ಖಾಯಿಲೆಗಳು ಬರದಂತೆ ನೋಡಿಕೊಳ್ಳುವುದು. ತಮ್ಮ ದೇಹವನ್ನೇ ಬಾಹ್ಯದಂತೆ ನೋಡುವುದು ಕಷ್ಟವಾಗಿರುವುದರಿಂದ ಇತರರ ವಿಷಯದಲ್ಲಿ ಇದು ಚೆನ್ನಾಗಿ ಪರಿಣಾಮಕಾರಿಯಾಗುವುದು. ಹಾಲು ಹಣ್ಣು ಯೋಗಿಗಳಿಗೆ ಶ್ರೇಷ್ಠವಾದ ಆಹಾರ.

ಪ್ರಶ್ನೆ: ವೈರಾಗ್ಯವಿದ್ದರೇನೆ ಆನಂದವನ್ನು ಪಡೆಯಲು ಸಾಧ್ಯ ಹೌದೆ?

ಉತ್ತರ: ವೈರಾಗ್ಯವು ಪ್ರಾರಂಭದಲ್ಲಿ ಬಹಳ ಯಾತನಾಮಯವಾಗಿರುವುದು.\break ಯಾವಾಗ ಅದು ಪರಿಪಕ್ವವಾಗುವುದೊ ಆಗ ಅದು ಅನಂತಾನಂದವನ್ನು ಕೊಡಬಲ್ಲದು.

ಪ್ರಶ್ನೆ: ತಪಸ್ಸು ಎಂದರೆ ಏನು?

ಉತ್ತರ: ತಪಸ್ಸಿನಲ್ಲಿ ಮೂರು ವಿಧಗಳಿವೆ. ಶಾರೀರಿಕ, ವಾಚಿಕ, ಮಾನಸಿಕ. ಮೊದಲನೆ\break ಯದೆ ಇತರರಿಗೆ ಸೇವೆ ಮಾಡುವುದು, ಎರಡನೆಯದೆ ಸತ್ಯವನ್ನು ಹೇಳುವುದು, ಮೂರನೆ\break ಯದೆ ಮನೋನಿಗ್ರಹ ಮತ್ತು ಏಕಾಗ್ರತೆ.

ಪ್ರಶ್ನೆ: ಮುಕ್ತಾತ್ಮನಲ್ಲಿಯೂ ಮತ್ತು ಒಂದು ಇರುವೆಯಲ್ಲಿಯೂ ಒಂದೇ ಚೈತನ್ಯವಿರುವು\break ದನ್ನು ನಾವು ಏತಕ್ಕೆ ನೋಡಲಾರೆವು?

ಉತ್ತರ: ಕಾಲಕ್ರಮೇಣ ಈ ಸ್ಥಿತಿಯನ್ನು ಸಾಧಿಸಹುದು.

ಪ್ರಶ್ನೆ: ಪೂರ್ಣತೆಯನ್ನು ಪಡೆಯದೆ ಇದ್ದರೆ ಬೋಧಿಸುವುದಕ್ಕೆ ಆಗುವುದಿಲ್ಲವೆ?

ಉತ್ತರ: ಇಲ್ಲ. ನನ್ನ ಗುರುವಿನ ಮತ್ತು ನನ್ನ ಸಂನ್ಯಾಸಿಶಿಷ್ಯರೆಲ್ಲರನ್ನೂ ಭಗವಂತ ತನ್ನ ಅನುಗ್ರಹದಿಂದ ಪರಿಪೂರ್ಣರನ್ನಾಗಿ ಮಾಡಲಿ. ಈ ಕಾರಣದಿಂದ ಅವರು ಬೋಧನೆಯ ಕೆಲಸಕ್ಕೆ ಅರ್ಹರಾಗಲಿ.

ಪ್ರಶ್ನೆ: ಗೀತೆಯಲ್ಲಿ ಬರುವ ಶ‍್ರೀಕೃಷ್ಣನ ವಿಶ್ವರೂಪ ದರ್ಶನದಲ್ಲಿ ವ್ಯಕ್ತವಾಗಿರುವ ಭಗವದ್​ವೈಭವವು ಬೇರೆ ಕಡೆ ಬರುವ ಅವನಿಗೂ ಗೋಪಿಯರಿಗೂ ಇರುವ ಪ್ರೀತಿಗಿಂತ ಶ್ರೇಷ್ಠವಾದುದೆ?

ಉತ್ತರ: ನಾವು ಪ್ರೀತಿಸುವ ವ್ಯಕ್ತಿಯಲ್ಲಿ ದೈವತ್ವದ ಭಾವನೆ ಇಲ್ಲದೇ ಇದ್ದರೆ, ಅದು ಭಗವದ್​ವೈಭವ ಜ್ಞಾನಕ್ಕಿಂತ ತುಂಬ ಕೀಳಾದುದು. ಹಾಗಿಲ್ಲದೇ ಇದ್ದರೆ ದೇಹವನ್ನು ಪ್ರೀತಿಸುವವರೆಲ್ಲಾ ಮುಕ್ತಿಯನ್ನು ಪಡೆಯಬಹುದಾಗಿತ್ತು.

