
\chapter{ವೇದಾಂತ}

ಜಾಫ್ನಾದ ಹಿಂದೂಗಳು ಸ್ವಾಮಿ ವಿವೇಕಾನಂದರನ್ನು ಸ್ವಾಗತಿಸುತ್ತಾ ಈ ಕೆಳಗಿನ ಬಿನ್ನವತ್ತಳೆ\-ಯನ್ನು ಅರ್ಪಿಸಿದರು:

\textbf{ಪೂಜ್ಯರೇ,}

ಜಾಫ್ನಾದ ಹಿಂದೂಗಳಾದ ನಾವು ಸಿಲೋನಿನ ಮುಖ್ಯ ಹಿಂದೂ ಕೇಂದ್ರವಾದ ಈ ಪ್ರದೇಶಕ್ಕೆ ತಮ್ಮನ್ನು ಹೃತ್ಪೂರ್ವಕವಾಗಿ ಸ್ವಾಗತಿಸುತ್ತೇವೆ. ಲಂಕಾದ್ವೀಪದ ಈ ಭಾಗಕ್ಕೆ ಬರಲು ಕೃಪೆಯಿಟ್ಟು ಒಪ್ಪಿದುದಕ್ಕಾಗಿ ತಮಗೆ ನಾವು ಅತ್ಯಂತ ಕೃತಜ್ಞರಾಗಿದ್ದೇವೆ.

ಎರಡು ಸಾವಿರ ವರ್ಷಗಳಿಗೂ ಹಿಂದೆ ನಮ್ಮ ಪೂರ್ವಿಕರು ದಕ್ಷಿಣ ಭಾರತದಿಂದ ಬಂದು ಇಲ್ಲಿ ನೆಲಸಿದರು. ಜಾಫ್ನಾದ ದೊರೆಗಳು ಹಿಂದುಧರ್ಮಕ್ಕೆ ರಕ್ಷಣೆಯನ್ನು ನೀಡಿದ್ದರು. ಆದರೆ ಮುಂದೆ ಪೋರ್ಚುಗೀಸರು ಮತ್ತು ಡಚ್ಚರು ಈ ಭಾಗವನ್ನು ಆಕ್ರಮಿಸಿಕೊಂಡ ಮೇಲೆ ನಮ್ಮ ಧರ್ಮದ ಆಚರಣೆಗೆ ಅಡ್ಡಿ ಆತಂಕಗಳು ಒದಗಿದವು. ಸಾರ್ವಜನಿಕವಾಗಿ ಧರ್ಮವನ್ನು ಆಚರಿಸುವುದನ್ನು ನಿಷೇಧಿಸಲಾಯಿತು. ನಮ್ಮ ಪವಿತ್ರ ದೇವಾಲಯಗಳನ್ನು ನೆಲಸಮ ಮಾಡಲಾಯಿತು. ಹಿಂಸೆಯ ಕ್ರೂರ ಹಸ್ತಗಳು ಹಾಗೆ ನೆಲಸಮಗೊಳಿಸಿದ ದೇವಾಲಯಗಳಲ್ಲಿ ಎರಡು ಜಗತ್ಪ್ರಸಿದ್ಧವಾದ ದೇವಾಲಯಗಳೂ ಸೇರಿದ್ದವು. ಹೊರದೇಶದಿಂದ ಬಂದ ಜನರು ನಮ್ಮ ಪೂರ್ವಿಕರ ಮೇಲೆ ಕ್ರೈಸ್ತಧರ್ಮವನ್ನು ಹೇರಲು ಎಷ್ಟೇ ಪ್ರಯತ್ನಿಸಿದರೂ ಅವರು ತಮ್ಮ ಧರ್ಮಕ್ಕೆ ದೃಢವಾಗಿ ಅಂಟಿಕೊಂಡಿದ್ದರು, ಮತ್ತು ನಮ್ಮ ಭವ್ಯ ಸಂಸ್ಕೃತಿಯ ರೂಪದಲ್ಲಿ ನಮಗೆ ಅದನ್ನು ಬಳುವಳಿಯಾಗಿ ನೀಡಿದ್ದಾರೆ. ಬ್ರಿಟಿಷರ ಆಡಳಿತದಲ್ಲಿ ಧರ್ಮದ ಪುನರುತ್ಥಾನವು ಮಹತ್ತಾದ ರೀತಿಯಲ್ಲಿ ನಡೆದಿದೆ. ಅಷ್ಟೇ ಅಲ್ಲದೆ ಪವಿತ್ರ ಸ್ಥಾನಗಳನ್ನು ಜೀರ್ಣೋದ್ಧಾರಗೊಳಿಸುವ ಪ್ರಯತ್ನಗಳೂ ನಡೆಯುತ್ತಿವೆ.

ತಾವು ನಮ್ಮ ಧರ್ಮಕ್ಕೆ ಉದಾತ್ತವೂ ನಿಃಸ್ವಾರ್ಥವೂ ಆದ ರೀತಿಯಲ್ಲಿ ಸೇವೆಯನ್ನು ಸಲ್ಲಿಸಿದ್ದೀರಿ. ವೇದಪ್ರಣೀತವಾದ ಸತ್ಯಜ್ಯೋತಿಯನ್ನು ಸರ್ವ ಧರ್ಮಸಮ್ಮೇಳನಕ್ಕೆ\break ಒಯ್ದಿದ್ದೀರಿ. ಅಮೆರಿಕ ಮತ್ತು ಇಂಗ್ಲೆಂಡುಗಳಲ್ಲಿ ಭಾರತದ ದಿವ್ಯದರ್ಶನಗಳಲ್ಲಿ ಪ್ರತಿಪಾದಿತವಾಗಿರುವ ಸತ್ಯಗಳನ್ನು ಪ್ರಚಾರ ಮಾಡಿದ್ದೀರಿ. ಹಿಂದೂ ಧರ್ಮದ ಸತ್ಯಗಳನ್ನು ಪಶ್ಚಿಮ ಜಗತ್ತಿಗೆ ಪರಿಚಯಿಸಿ ಪೂರ್ವ ಪಶ್ಚಿಮಗಳನ್ನು ಹತ್ತಿರಕ್ಕೆ ತಂದಿದ್ದೀರಿ. ಈ ಎಲ್ಲ ಸಾಧನೆಗಳಿಗಾಗಿ ನಾವು ತಮಗೆ ನಮ್ಮ ಹೃತ್ಪೂರ್ವಕ ಕೃತಜ್ಞತೆಗಳನ್ನು ಸಮರ್ಪಿಸಬಯಸುತ್ತೇವೆ. ಜಡವಾದದ ಇಂದಿನ ದಿನಗಳಲ್ಲಿ ಧರ್ಮಶ್ರದ್ಧೆ ಕುಂದುತ್ತಿದೆ. ಆಧ್ಯಾತ್ಮಿಕ ಸತ್ಯದ ಅನ್ವೇಷಣೆಯನ್ನು ಕಡೆಗಣಿಸಲಾಗಿದೆ. ಅಂತಹ ಸಮಯದಲ್ಲಿ ನಮ್ಮ ಪ್ರಾಚೀನ ಧರ್ಮದ ಪುನರುದ್ಧಾರಕ್ಕಾಗಿ ಒಂದು ಚಳುವಳಿಯನ್ನು ಪ್ರಾರಂಭಿಸಿರುವುದಕ್ಕಾಗಿಯೂ ತಮಗೆ ನಾವು ಕೃತಜ್ಞತೆಗಳನ್ನು ಸಲ್ಲಿಸುತ್ತೇವೆ.

ನಮ್ಮ ಧರ್ಮದ ಹೃದಯವು ಎಷ್ಟು ವಿಶಾಲವಾದುದು ಎಂಬುದನ್ನು ಪಶ್ಚಿಮದ ಜನರಿಗೆ ತಿಳಿಸಿದ್ದೀರಿ. ಪಶ್ಚಿಮದ ತತ್ತ್ವಶಾಸ್ತ್ರವು ಊಹಿಸಲೂ ಸಾಧ್ಯವಿಲ್ಲದಷ್ಟು ವಿಷಯಗಳು ಭಾರತದ ತತ್ತ್ವಶಾಸ್ತ್ರಗಳಲ್ಲಿ ಅಡಗಿವೆ ಎಂಬುದನ್ನು ಪಾಶ್ಚಾತ್ಯರಿಗೆ ಚೆನ್ನಾಗಿ ಮನದಟ್ಟು ಮಾಡಿಸಿದ್ದೀರಿ. ಇದಕ್ಕಾಗಿ ನಾವು ತಮಗೆ ಎಷ್ಟು ಋಣಿ ಎಂಬುದನ್ನು ಮಾತುಗಳಲ್ಲಿ ಹೇಳಿ ಮುಗಿಸಲಾರೆವು.

ಪಶ್ಚಿಮದಲ್ಲಿ ತಾವು ನಡೆಸಿದ ಕೆಲಸಕಾರ್ಯಗಳನ್ನು ನಾವು ಅತ್ಯಂತ ಎಚ್ಚರಿಕೆಯಿಂದ ಗಮನಿಸುತ್ತಿದ್ದೇವೆ ಎಂಬುದನ್ನಾಗಲಿ, ಧರ್ಮದ ಕ್ಷೇತ್ರಗಳಲ್ಲಿ ಶ್ರದ್ಧೆಯಿಂದ ದುಡಿದು ತಾವು ಸಾಧಿಸಿದ ಯಶಸ್ಸನ್ನು ಗಮನಿಸಿ ನಮ್ಮ ಹೃದಯಗಳು ಸಂತೋಷ ಭರಿತವಾಗಿವೆ ಎಂಬು\-ದನ್ನಾಗಲಿ ತಮಗೆ ಒತ್ತಿ ಹೇಳಬೇಕಾದ ಅಗತ್ಯವಿಲ್ಲ. ಪಶ್ಚಿಮ ದೇಶದ ಬೌದ್ಧಿಕ ಚಟುವಟಿಕೆಗಳ, ನೈತಿಕ ಪ್ರಗತಿಯ, ಧಾರ್ಮಿಕ ಅನ್ವೇಷಣೆಯ ಮಹಾನ್​ ಕೇಂದ್ರಗಳ ಪತ್ರಿಕೆಗಳು ತಮ್ಮನ್ನು ಮೆಚ್ಚಿ ಲೇಖನಗಳನ್ನು ಬರೆದಿವೆ. ನಮ್ಮ ಧಾರ್ಮಿಕ ಸಾಹಿತ್ಯಕ್ಕೆ ನೀವು ನೀಡಿದ ಅಮೂಲ್ಯ ಕಾಣಿಕೆಗಳನ್ನು ಕೊಂಡಾಡಿವೆ. ಇವೆಲ್ಲ ಮಹತ್ತಾದ ಪ್ರಯತ್ನಗಳಿಗೆ ಸಾಕ್ಷ್ಯಗಳಾಗಿವೆ.

ತಾವು ನಮ್ಮ ನಾಡಿಗೆ ಭೇಟಿಕೊಟ್ಟಿದ್ದೀರಿ. ತಮ್ಮಂತೆಯೇ ನಿಜವಾದ ಎಲ್ಲ ಆಧ್ಯಾತ್ಮಿಕ ಜ್ಞಾನಕ್ಕೂ ವೇದಗಳೇ ಆಧಾರ ಎಂಬುದನ್ನು ನಂಬಿರುವ ನಾವು ಮತ್ತೆ ಮತ್ತೆ ತಾವು ನಮ್ಮಲ್ಲಿಗೆ ಬರುತ್ತಾ ಇರಿ ಎಂಬ ಆಸೆಯನ್ನು ವ್ಯಕ್ತಪಡಿಸಲು ಇಚ್ಛಿಸುತ್ತೇವೆ. ಇದುವರೆಗೂ ತಮ್ಮ ಭವ್ಯವಾದ ಕಾರ್ಯಚಟುವಟಿಕೆಗಳಿಗೆ ಎಲ್ಲರಿಗೂ ಎದ್ದು ಕಾಣುವಂತಹ ಯಶಸ್ಸಿನ ಕಿರೀಟವನ್ನು ತೊಡಿಸಿದ ಭಗವಂತನು ತಮಗೆ ತಮ್ಮ ಉದಾತ್ತವಾದ ಕಾರ್ಯವನ್ನು ಮುಂದುವರಿಸಲು ಅಗತ್ಯವಾದ ಶಕ್ತಿಯನ್ನೂ ದೀರ್ಘಾಯುಷ್ಯವನ್ನೂ ನೀಡಲಿ ಎಂದು ಪ್ರಾರ್ಥಿಸುತ್ತೇವೆ.

\begin{flushright}
ಇತಿ ತಮ್ಮ ವಿಧೇಯರು\\ಜಾಫ್ನದ ಹಿಂದೂಗಳ ಪರವಾಗಿ.
\end{flushright}

ಮೇಲಿನ ಬಿನ್ನವತ್ತಳೆಗೆ ಸ್ವಾಮೀಜಿ ಸೂಕ್ತವಾದ ಉತ್ತರವನ್ನು ನೀಡಿದರು. ಮರುದಿನ ಸಂಜೆ ವೇದಾಂತವನ್ನು ಕುರಿತು ಅವರು ನೀಡಿದ ಉಪನ್ಯಾಸದ ವರದಿಯನ್ನು ಇಲ್ಲಿ ನೀಡಲಾಗಿದೆ.

ಮಾತನಾಡಬೇಕೆಂದಿರುವ ವಿಷಯ ಬೃಹತ್ತಾಗಿದೆ. ಕಾಲವಿರುವುದು ಅಲ್ಪ. ಒಂದು ಉಪನ್ಯಾಸದಲ್ಲಿ ಹಿಂದೂಧರ್ಮದ ವಿಷಯಗಳನ್ನೆಲ್ಲಾ ವಿಸ್ತಾರವಾಗಿ ಹೇಳುವುದು ಅಸಾಧ್ಯ. ಸಾಧ್ಯವಾದಷ್ಟು ಸುಲಭ ಭಾಷೆಯಲ್ಲಿ ಹಿಂದೂಧರ್ಮದ ಮುಖ್ಯ ವಿಷಯಗಳನ್ನು ನಿಮ್ಮ ಮುಂದಿಡಲು ಯತ್ನಿಸುತ್ತೇನೆ. ಇಂದು ನಮ್ಮನ್ನು ನಾವು ಕರೆದುಕೊಳ್ಳುತ್ತಿರುವ ಹಿಂದೂ ಎಂಬ ಪದ ತನ್ನ ಅರ್ಥವನ್ನೆಲ್ಲಾ ಕಳೆದುಕೊಂಡಿದೆ. ಇದಕ್ಕೆ ಹಿಂದೆ ಇದ್ದ ಅರ್ಥ ಸಿಂಧೂನದಿಯ ಆಚೆಕಡೆ ಎಂದರೆ ಪೂರ್ವಕ್ಕೆ ಇದ್ದವರು ಎಂದು. ಪುರಾತನ ಪಾರ್ಸಿಯವರು ಇದನ್ನು ಹಿಂದೂ ಎಂದು ತಿರುಗಿಸಿದರು. ಸಿಂಧೂ ನದಿಯ ಪೂರ್ವಕ್ಕೆ ಇರುವವರನ್ನೆಲ್ಲಾ ಹಿಂದೂಗಳೆಂದು ಕರೆದರು. ಈ ಪದ ಬಂದಿರುವುದು ಹೀಗೆ. ಮಹಮ್ಮದೀಯರ\break ಆಳ್ವಿಕೆಯ ಕಾಲದಲ್ಲಿ ನಾವೇ ಆ ಪದವನ್ನು ತೆಗೆದುಕೊಂಡೆವು. ಆ ಪದವನ್ನು ಉಪಯೋಗಿಸಿದರೆ ಏನೂ ತೊಂದರೆ ಇಲ್ಲ. ಆದರೆ ಹಿಂದೆ ಇದ್ದ ಅರ್ಥವೆಲ್ಲಾ ಮಾಯವಾಗಿದೆ. ಸಿಂಧೂ ನದಿಯ ಈ ಪಾರ್ಶ್ವದಲ್ಲಿರುವವರೆಲ್ಲಾ ಹಿಂದಿನಂತೆ ಒಂದು ಧರ್ಮವನ್ನು ಅನುಸರಿಸುತ್ತಿಲ್ಲ. ಆ ಪದದ ಅರ್ಥದಲ್ಲಿ ಹಿಂದೂಗಳು ಮಾತ್ರವಲ್ಲ, ಇಂಡಿಯಾ ದೇಶದಲ್ಲಿ ಇರುವ ಮಹಮ್ಮದೀಯರು, ಕ್ರೈಸ್ತರು, ಜೈನರು ಎಲ್ಲರೂ ಸೇರಿರುವರು. ಆದಕಾರಣ ನಾನು ಹಿಂದೂ ಎಂಬ ಪದವನ್ನು ಉಪಯೋಗಿಸುವುದಿಲ್ಲ. ಮತ್ತಾವ ಪದವನ್ನು ಉಪಯೋಗಿಸುವುದು? ಮತ್ತೊಂದು ಪದವೇ ವೇದವನ್ನು ಅನುಸರಿಸುವ ವೈದಿಕರು ಅಥವಾ ವೇದಾಂತಿಗಳು. ಎಲ್ಲಾ ಧರ್ಮಗಳು ಯಾವುದಾದರೊಂದು ಶಾಸ್ತ್ರಕ್ಕೆ ಗೌರವವನ್ನು ತೋರುತ್ತವೆ. ಇದು ದೇವವಾಣಿ ಅಥವಾ ಯಾವುದಾದರೊಬ್ಬ ಅತಿಮಾನವನು ಇದನ್ನು ಹೇಳಿದನು ಎಂದು ನಂಬುವೆವು. ಈ ಶಾಸ್ತ್ರವೇ ಆಯಾಯ ಧರ್ಮಗಳ ತಳಹದಿ. ಆಧುನಿಕ ಪಾಶ್ಚಾತ್ಯ ವಿದ್ಯಾವಂತರ ದೃಷ್ಟಿಯಲ್ಲಿ ಇಂತಹ ಶಾಸ್ತ್ರಗಳಲ್ಲಿ ಅತಿ ಪುರಾತನವಾದುದೇ ಹಿಂದೂಗಳ ವೇದ. ವೇದಗಳ ಸಂಕ್ಷಿಪ್ತ ಪರಿಚಯ ಆವಶ್ಯಕ.

ವೇದಗಳು ಮಾನವಕೃತವಲ್ಲ. ಅವುಗಳ ಕಾಲವನ್ನು ಇದುವರೆವಿಗೂ ಯಾರೂ ನಿಶ್ಚಯಿ\-ಸಿಲ್ಲ, ನಿಶ್ಚಯಿಸುವಂತೆಯೂ ಇಲ್ಲ. ನಮ್ಮ ದೃಷ್ಟಿಯಲ್ಲಿ ವೇದಗಳು ಅನಾದಿ ಮತ್ತು ಅನಂತ. ಇಲ್ಲಿ ಒಂದು ಮುಖ್ಯ ವಿಷಯವನ್ನು ಗಮನಿಸಬೇಕು. ಉಳಿದ ಧರ್ಮಗಳು ತಮ್ಮ ಶಾಸ್ತ್ರವನ್ನು ದೇವರೋ, ದೇವದೂತರೋ ದೇವಮಾನವರೋ ಮನುಷ್ಯನಿಗೆ ಉಸುರಿದರೆಂದು ನಂಬುವರು. ಆದರೆ ಹಿಂದೂಗಳ ದೃಷ್ಟಿಯಲ್ಲಿ ವೇದಗಳಿಗೆ ಬೇರೆ ಪ್ರಮಾಣವೇ ಇಲ್ಲ. ಅವು ತಮಗೆ ತಾವೇ ಪ್ರಮಾಣವಾಗಿವೆ. ಅವನ್ನು ಯಾರೂ ಬರೆಯಲಿಲ್ಲ, ಅನಾದಿಕಾಲದಿಂದಲೂ ಅವು ಇದ್ದವು. ಹೇಗೆ ಸೃಷ್ಟಿಗೆ ಆದಿ ಅಂತ್ಯಗಳಿಲ್ಲವೋ ಹಾಗೆಯೇ ಈಶ್ವರನ ಜ್ಞಾನರಾಶಿಯಾದ ವೇದಗಳಿಗೆ ಆದಿ ಅಂತ್ಯಗಳಿಲ್ಲ. ಈ ಜ್ಞಾನರಾಶಿಯನ್ನು ವೇದವೆನ್ನುತ್ತೇವೆ. ವೇದಾಂತವೆಂಬ ಜ್ಞಾನರಾಶಿಯನ್ನು ಋಷಿಗಳು ಕಂಡುಹಿಡಿದರು. ಋಷಿ ಎಂಬ ಪದದ ಅರ್ಥ ಮಂತ್ರದ್ರಷ್ಟಾ ಎಂದು, ಅಂದರೆ ಮಂತ್ರವನ್ನು ಕಂಡವನು ಎಂದು; ಅದನ್ನು ಸೃಷ್ಟಿಸಿದವನೆಂದು ಅಲ್ಲ. ವೇದದ ಯಾವುದೋ ಒಂದು ಮಂತ್ರವು ಒಬ್ಬ ಋಷಿಯಿಂದ ಬಂದಿದೆ ಎಂದು ಹೇಳಿದರೆ, ಅದನ್ನು ಆತ ತನ್ನ ಮನಸ್ಸಿನಲ್ಲಿ ಸೃಷ್ಟಿಸಿದನೆಂದಾಗಲಿ, ಬರೆದನೆಂದಾಗಲಿ ತಿಳಿಯಬೇಡಿ. ಆಗಲೇ ಇದ್ದ ಆ ಭಾವವನ್ನು ಆತ ಸಂದರ್ಶಿಸಿದ ಎಂದು ಅರ್ಥ. ಅನಾದಿಯಿಂದಲೂ ಅದು ವಿಶ್ವದಲ್ಲಿ ಇತ್ತು. ಋಷಿಯು ಅದನ್ನು ಆವಿಷ್ಕಾರ ಮಾಡಿದನು. ಋಷಿಗಳು ಆಧ್ಯಾತ್ಮಿಕ ಜ್ಞಾನರಾಶಿಯನ್ನು ಆವಿಷ್ಕಾರ ಮಾಡಿದವರು.

ವೇದಗಳು ಕರ್ಮಕಾಂಡ ಮತ್ತು ಜ್ಞಾನಕಾಂಡ ಎಂದು ಎರಡು ಭಾಗವಾಗಿವೆ. ಕರ್ಮಕಾಂಡದಲ್ಲಿ ಹಲವು ಯಾಗಯಜ್ಞಗಳಿವೆ. ಅವುಗಳಲ್ಲಿ ಹಲವನ್ನು ಆಧುನಿಕ ಕಾಲದಲ್ಲಿ ಅನುಷ್ಠಾನಕ್ಕೆ ಅಸಾಧ್ಯವೆಂದು ತೊರೆದಿರುವರು. ಉಳಿದವು ಯಾವುದಾದರೊಂದು\break ರೂಪದಲ್ಲಿ ಈಗಲೂ ಇವೆ. ಕರ್ಮಕಾಂಡದಲ್ಲಿ ಬರುವ ಮುಖ್ಯ ವಿಷಯವಾದ ಸಾಧಾರಣ ಮಾನವನ ಕರ್ತವ್ಯಗಳು, ಬ್ರಹ್ಮಚರ್ಯ, ಗಾರ್ಹಸ್ಥ್ಯ, ವಾನಪ್ರಸ್ಥ, ಸಂನ್ಯಾಸಾಶ್ರಮಗಳಿಗೆ ಸಂಬಂಧಿಸಿದ ಕರ್ತವ್ಯಗಳು ಈಗಲೂ ಸ್ವಲ್ಪ ಹೆಚ್ಚು ಕಡಮೆ ಹಾಗೆಯೇ ಇವೆ. ಎರಡನೆಯದೇ ಜ್ಞಾನಕಾಂಡ; ನಮ್ಮ ಧರ್ಮದ ಆಧ್ಯಾತ್ಮಿಕ ಭಾಗವಿರುವುದೇ ಅಲ್ಲಿ. ಇದೇ ವೇದಾಂತ, ವೇದಗಳ ಚರಮ ಲಕ್ಷ್ಯ. ವೇದಗಳ ಜ್ಞಾನಸಾರವನ್ನೇ ವೇದಾಂತ ಎಂದು ಕರೆಯುವುದು. ಇದೇ ಉಪನಿಷತ್ತುಗಳು. ಅದ್ವೈತಿಗಳಾಗಲಿ, ವಿಶಿಷ್ಟಾದ್ವೈತಿಗಳಾಗಲಿ, ದ್ವೈತಿಗಳಾಗಲಿ, ಶೈವರಾಗಲಿ, ವೈಷ್ಣವರಾಗಲಿ, ಶಾಕ್ತರಾಗಲಿ, ಸೌರರಾಗಲಿ, ಗಾಣಪತ್ಯರಾಗಲಿ ಅಥವಾ ಹಿಂದೂಧರ್ಮದ ಪರಿಧಿಯೊಳಗೆ ಬರಲು ಯತ್ನಿಸುವ ಮತ್ತಾವ ಧರ್ಮವೇ ಆಗಲಿ, ಎಲ್ಲವೂ ವೇದಗಳ ಭಾಗವಾದ ಉಪನಿಷತ್ತುಗಳನ್ನು ಪ್ರಮಾಣವಾಗಿ ಸ್ವೀಕರಿಸಲೇಬೇಕು. ಅವಕ್ಕೆ ತಮ್ಮದೇ ಭಾಷ್ಯಗಳಿರಬಹುದು, ತಮಗೆ ತೋರಿದ ರೀತಿಯಲ್ಲಿ ಅವನ್ನು ವಿವರಿಸಬಹುದು ಆದರೆ ಅವುಗಳ ಪ್ರಮಾಣವನ್ನು ಮಾತ್ರ ಒಪ್ಪಿಕೊಳ್ಳಬೇಕು. ಆದಕಾರಣ ಅವರನ್ನು ಹಿಂದೂಗಳೆಂದು ಕರೆಯದೆ ವೇದಾಂತಿಗಳೆಂದು ಕರೆಯಲಿಚ್ಛಿಸುತ್ತೇನೆ. ಪುರಾತನ ಸಾಂಪ್ರದಾಯಿಕ ದಾರ್ಶನಿಕರೆಲ್ಲಾ ವೇದಾಂತದ ಪ್ರಮಾಣವನ್ನು ಒಪ್ಪಲೇಬೇಕು. ಆಧುನಿಕ ಕಾಲದ ಧರ್ಮಗಳಲ್ಲಿರುವ ಹಲವು ಶಾಖೋಪಶಾಖೆಗಳೆಲ್ಲ, ಅವು ಎಷ್ಟೇ ಅಸ್ಪಷ್ಟವಾಗಿರಲಿ, ಗ್ರಾಮ್ಯವಾಗಿರಲಿ, ಅವನ್ನು ನಾವು ಪರೀಕ್ಷಿಸಿದರೆ ಅವುಗಳ ಮೂಲ ಉಪನಿಷತ್ತಿನಲ್ಲಿರುವುದು ಕಾಣುತ್ತದೆ. ಉಪನಿಷತ್ತಿನ ಭಾವ ನಮ್ಮ ಜನಾಂಗದ ಅಂತರಾಳಕ್ಕೆ ಹೋಗಿದೆ. ಹಿಂದೂಗಳ ಅತಿ ಗ್ರಾಮ್ಯ ಧರ್ಮದ ಚಿಹ್ನೆಗಳನ್ನು ನೋಡಿದರೂ ಅಲ್ಲಿ ಉಪನಿಷತ್ತಿನ ಕೆಲವು ಭಾವನೆಗಳನ್ನು ಗುರುತಿಸಿ ಆಶ್ಚರ್ಯಪಡುತ್ತೇವೆ. ಉಪನಿಷತ್ತು ಕಾಲಕ್ರಮೇಣ ಚಿಹ್ನೆಯಾಗುವುದು, ರೂಪಕವಾಗುವುದು. ಉಪನಿಷತ್ತಿನ ಆಧ್ಯಾತ್ಮಿಕ ಮತ್ತು ತಾತ್ತ್ವಿಕ ಭಾವನೆಗಳು ಈಗ ನಮ್ಮಲ್ಲಿವೆ. ಅವು ಪ್ರತಿ ಮನೆಯಲ್ಲಿ ಪೂಜಿಸಲ್ಪಡುವ ಪ್ರತೀಕಗಳಾಗಿ ಮಾರ್ಪಟ್ಟಿವೆ. ನಮ್ಮಲ್ಲಿ ಈಗ ಬಳಕೆಯಲ್ಲಿರುವ ಸಂಕೇತಗಳೆಲ್ಲಾ ವೇದಾಂತದಿಂದ ಬಂದವು. ವೇದಾಂತದಲ್ಲಿ ಇವನ್ನು ಸಂಕೇತಗಳಾಗಿ ಬಳಸಿದ್ದಾರೆ. ಈ ಭಾವನೆ ನಮ್ಮ ಜನಜೀವನದಲ್ಲಿ ಸಂಕೇತಗಳಾಗಿ ಪ್ರವೇಶಿಸಿ ನಮ್ಮ ಜೀವನದ ಅಂಗವಾಗಿ ಹಾಸುಹೊಕ್ಕಾಗಿದೆ.

ವೇದಾಂತದ ಅನಂತರ ಸ್ಮೃತಿ ಬರುವುದು. ಋಷಿಗಳೇ ಇದನ್ನು ಬರೆದವರು. ಇದರ ಪ್ರಮಾಣ ವೇದಾಂತಕ್ಕೆ ಅಧೀನವಾಗಿದೆ. ಆಯಾಯ ಧರ್ಮಾನುಯಾಯಿಗಳಿಗೆ ಇತರ ಧರ್ಮಗ್ರಂಥಗಳು ಹೇಗೆ ಇವೆಯೋ ಇದೂ ಹಾಗೆಯೇ. ಇದನ್ನು ಬರೆದವರು ಋಷಿಗಳೆಂದು ನಾವು ಒಪ್ಪಿಕೊಳ್ಳುತ್ತೇವೆ. ಈ ದೃಷ್ಟಿಯಲ್ಲಿ ಇತರ ಧರ್ಮಗಳ ಶಾಸ್ತ್ರಗಳಂತೆ ಇದೂ ಕೂಡ. ಆದರೆ ಸ್ಮೃತಿಯೇ ಪರಮ ಪ್ರಮಾಣವಲ್ಲ. ವೇದಾಂತಕ್ಕೆ ವಿರುದ್ಧವಾದುದು ಏನಾದರೂ ಇದ್ದರೆ, ಆಗ ಸ್ಮೃತಿಯನ್ನು ತ್ಯಜಿಸಬಹುದು. ಅದಕ್ಕೆ ಪ್ರಮಾಣವಿಲ್ಲ. ಈ ಸ್ಮೃತಿಗಳು ಕಾಲಕಾಲಕ್ಕೆ ಬದಲಾಗಿವೆ. ಒಂದು ಸ್ಮೃತಿಯು ಸತ್ಯಯುಗಕ್ಕೆ, ಒಂದು ಸ್ಮೃತಿಯು ತ್ರೇತಾಯುಗಕ್ಕೆ, ಮತ್ತೊಂದು ದ್ವಾಪರಯುಗಕ್ಕೆ, ಒಂದು ಕಲಿಯುಗಕ್ಕೆ ಪ್ರಮಾಣವೆಂಬುದನ್ನು ನೋಡುತ್ತೇವೆ. ಜನಾಂಗದ ಸ್ಥಿತಿಗತಿಗಳು ಬದಲಾದ ಹಾಗೆ, ಹಲವು ಸಮಸ್ಯೆಗಳನ್ನು ಜನಾಂಗಗಳು ಇದಿರಿಸಿದಾಗ ಅವರ ಆಚಾರ ವ್ಯವಹಾರಗಳು ಬದಲಾದವು. ಜನಾಂಗದ ಆಚಾರ ವ್ಯವಹಾರಗಳನ್ನು ಕ್ರಮಪಡಿಸುವುದೇ ಸ್ಮೃತಿಯ ಮುಖ್ಯ ಗುರಿಯಾದುದರಿಂದ ಅವೂ ಕಾಲಕಾಲಕ್ಕೆ ಬದಲಾಗಬೇಕಾಯಿತು. ಇದನ್ನು ಜ್ಞಾಪಕದಲ್ಲಿಡಬೇಕೆಂದು ನಿಮಗೆ\break ಹೇಳುತ್ತೇನೆ. ವೇದಾಂತದಲ್ಲಿರುವ ಅಧ್ಯಾತ್ಮ ತತ್ತ್ವಗಳು ಬದಲಾಗುವುದಿಲ್ಲ. ಏತಕ್ಕೆ?\break ಅವೆಲ್ಲವೂ ಮಾನವ ಮತ್ತು ಜಗತ್ತಿಗೆ ಸಂಬಂಧಿಸಿದ ಸನಾತನ ತತ್ತ್ವಗಳ ಮೇಲೆ ನಿಂತಿವೆ. ಅವೆಂದಿಗೂ ಬದಲಾಗುವುದಿಲ್ಲ. ಆತ್ಮ, ಪರಲೋಕ, ಮುಂತಾದವು ಎಂದಿಗೂ ಬದಲಾಗುವುದಿಲ್ಲ. ಸಾವಿರಾರು ವರ್ಷಗಳ ಹಿಂದೆಯೂ ಹೀಗೆಯೇ ಇದ್ದವು. ಈಗಲೂ ಹೀಗೆಯೇ ಇರುವುವು, ಲಕ್ಷಾಂತರ ವರ್ಷಗಳ ಅನಂತರವೂ ಹೀಗೆಯೇ ಇರುವುವು. ಆದರೆ ನಮ್ಮ ಸಾಮಾಜಿಕ ಸ್ಥಿತಿಯ ಮೇಲೆ ನಿಂತ ಧಾರ್ಮಿಕ ಅನುಷ್ಠಾನಗಳು ಸಮಾಜವು ಬದಲಾದಂತೆ ಬದಲಾಗಬೇಕು. ಕೆಲವು ನಿಯಮಗಳು, ಕೆಲವು ಕಾಲದೇಶಗಳಿಗೆ ಸರಿಹೋಗುವುದೇ ಹೊರತು ಅನ್ಯ ಕಾಲದೇಶಗಳಿಗೆ ಅಲ್ಲ. ಒಂದು ಬಗೆಯ ಆಹಾರ ಒಂದು ಕಾಲಕ್ಕೆ ಸರಿ, ಬೇರೆ ಕಾಲಕ್ಕೆ ಅಲ್ಲ, ಅದು ಆ ಕಾಲಕ್ಕೆ ಮಾತ್ರ ಯೋಗ್ಯ. ವಾಯುಗುಣ ಬದಲಾದಾಗ, ಇನ್ನೂ ಹಲವು ಸಮಸ್ಯೆಗಳನ್ನು ಎದುರಿಸಬೇಕಾದಾಗ ಸ್ಮೃತಿಯು ಆಹಾರ ಮುಂತಾದುವನ್ನು ಬದಲಾಯಿಸಿತು. ಆಧುನಿಕ ಕಾಲದಲ್ಲಿ ಕೆಲವು ಬದಲಾವಣೆಗಳು ಆವಶ್ಯಕವಾದರೆ ಅವನ್ನು ಮಾಡಬೇಕು. ಅದನ್ನು ಹೇಗೆ ಮಾಡಬೇಕೆಂಬುದನ್ನು ಋಷಿಗಳು ಬಂದು ಹೇಳುವರು. ಆದರೆ ನಮ್ಮ ಆಧ್ಯಾತ್ಮಿಕ ತತ್ತ್ವಗಳಾವುವೂ ಬದಲಾಗುವ ಹಾಗಿಲ್ಲ. ಅವು ಹಾಗೆಯೇ ಇರುವುವು.

ಇದಾದ ಅನಂತರ ಪುರಾಣ ಬರುವುದು. ಪುರಾಣಗಳಿಗೆ ಪಂಚಲಕ್ಷಣಗಳಿವೆ. ಇತಿಹಾಸ, ಸೃಷ್ಟಿತತ್ತ್ವ, ಇತ್ಯಾದಿಗಳನ್ನು ಹಲವು ಕಥೆಗಳ ಮೂಲಕ ವಿವರಿಸುವ ಧಾರ್ಮಿಕ ತತ್ತ್ವಗಳು ಇವುಗಳಲ್ಲಿ ಅಡಕವಾಗಿವೆ. ವೇದಧರ್ಮವನ್ನು ಜನಸಾಮಾನ್ಯರಲ್ಲಿ ಹರಡುವುದಕ್ಕಾಗಿ ಇವನ್ನು ಬರೆದಿರುವರು. ವೇದಗಳ ಭಾಷೆ ಅತಿ ಪುರಾತನ. ಪಂಡಿತರಿಗೂ ಅದರ ಕಾಲವನ್ನು ನಿರ್ಣಯಿಸುವುದು ಕಷ್ಟ. ಪುರಾಣವನ್ನು ಜನರು ಆಗಿನ ಭಾಷೆಯಲ್ಲಿ, ಅಂದರೆ ಆಧುನಿಕ ಸಂಸ್ಕೃತದಲ್ಲಿ ಬರೆದರು. ಅದು ಪಂಡಿತರಿಗಾಗಿ ಅಲ್ಲ, ಸಾಧಾರಣ ಜನರಿಗಾಗಿ. ಸಾಧಾರಣ ಜನರು ತತ್ತ್ವವನ್ನು ತಿಳಿದುಕೊಳ್ಳಲಾರರು. ಆ ಗಹನ ವಿಷಯಗಳನ್ನು ಅಂದಿನ ಕಾಲದ ಐತಿಹಾಸಿಕ ಘಟನೆಗಳ ಮೂಲಕ, ಋಷಿಗಳ ಮತ್ತು ರಾಜರ ಜೀವನದ ಮೂಲಕ ವಿವರಿಸಿರುವರು. ಧರ್ಮದ ಸನಾತನ ತತ್ತ್ವಗಳನ್ನು ವಿವರಿಸುವುದಕ್ಕೆ ಋಷಿಗಳು ಪುರಾಣವನ್ನು ಉಪಯೋಗಿಸಿಕೊಂಡರು.

ತಂತ್ರಶಾಸ್ತ್ರವೆಂಬುದೂ ಇದೆ. ಅದು ಕೆಲವು ವಿಷಯಗಳಲ್ಲಿ ಪುರಾಣವನ್ನು ಹೋಲುವುದು. ಕೆಲವು ತಂತ್ರಗಳಲ್ಲಿ ಪುರಾತನ ಕರ್ಮಕಾಂಡದ ಭಾವನೆಗಳನ್ನು ಪುನಶ್ಚೇತನಗೊಳಿಸುವ ಪ್ರಯತ್ನ ನಡೆದಿದೆ.

ಈ ಗ್ರಂಥಗಳೇ ಹಿಂದೂ ಶಾಸ್ತ್ರ. ಈ ದೇಶದಲ್ಲಿ, ಮುಕ್ಕಾಲುಪಾಲು ಕಾಲ\break ವನ್ನೆಲ್ಲಾ ಅಧ್ಯಾತ್ಮ ವಿದ್ಯೆಗೆ ಮೀಸಲಾಗಿಟ್ಟ ಈ ಜನಾಂಗದಲ್ಲಿ, ಇಷ್ಟು ಪವಿತ್ರ ಗ್ರಂಥರಾಶಿ\break ಇರುವುದರಿಂದ ಭಿನ್ನ ಭಿನ್ನ ಪಂಗಡಗಳಿರುವುದು ಸಹಜವೇ ಸರಿ. ಇನ್ನೂ ಸಾವಿರಾರು ಪಂಗಡಗಳು ಇಲ್ಲದೇ ಇರುವುದೇ ಒಂದು ಆಶ್ಚರ್ಯ. ಕೆಲವು ವಿಷಯಗಳಲ್ಲಿ ಈ ಪಂಗಡಗಳಿಗೆ ಎಷ್ಟೋ ವ್ಯತ್ಯಾಸವಿದೆ. ಇವುಗಳಿಗೆ ಇರುವ ವ್ಯತ್ಯಾಸವನ್ನಾಗಲಿ ಅಥವಾ ಇವುಗಳಲ್ಲಿ ಇರುವ ಆಧ್ಯಾತ್ಮಿಕ ವಿಷಯಗಳನ್ನಾಗಲಿ ವಿಶದವಾಗಿ ತಿಳಿದುಕೊಳ್ಳುವುದಕ್ಕೆ ಸಮಯವಿಲ್ಲ. ಪ್ರತಿಯೊಬ್ಬ ಹಿಂದುವೂ ನಂಬಬೇಕಾದ ಭಿನ್ನ ಭಿನ್ನ ಪಂಗಡಗಳ ಸಾಮಾನ್ಯ ಮೂಲತತ್ತ್ವವನ್ನು ಮಾತ್ರ ತೆಗೆದುಕೊಳ್ಳುತ್ತೇವೆ.

ಮೊದಲು ಸೃಷ್ಟಿಯ ವಿಷಯವನ್ನು ತೆಗೆದುಕೊಳ್ಳೋಣ. ಪ್ರತಿಯೊಬ್ಬ ಹಿಂದೂವು ಈ ಸಂಸಾರ, ಈ ಪ್ರಕೃತಿ ಅನಾದಿ ಮತ್ತು ಅನಂತವೆಂದು ತಿಳಿಯುವನು. ಎಲ್ಲೋ ಕೆಲವು ದಿನಗಳ ಹಿಂದೆ ಈ ಪ್ರಪಂಚವನ್ನು ದೇವರು ಸೃಷ್ಟಿಸಿ ಅನಂತರ ಇದುವರೆಗೆ ಅವನು ನಿದ್ರಿಸುತ್ತಿರುವನೆಂದು ಅಲ್ಲ. ಸೃಷ್ಟಿ ಇನ್ನೂ ಆಗುತ್ತಿದೆ. ದೇವರು ಅನುಗಾಲವೂ ಸೃಷ್ಟಿಸು\-ತ್ತಿರುವನು. ಅವನಿಗೆ ವಿರಾಮವೆಂದಿಗೂ ಇಲ್ಲ. “ನಾನು ಒಂದು ಕ್ಷಣ ಸುಮ್ಮನಿದ್ದರೂ ಸೃಷ್ಟಿ ನಾಶವಾಗುವುದು” ಎಂದು ಶ‍್ರೀಕೃಷ್ಣ ಗೀತೆಯಲ್ಲಿ ಹೇಳುವುದನ್ನು ಗಮನಿಸಿ. ಸೃಷ್ಟಿಯಲ್ಲಿ ಅವನು ಕೆಲಸ ಮಾಡದ ಸಮಯವೇ ಇರಲಿಲ್ಲ. ಆದರೆ ನಮ್ಮಲ್ಲಿ ಪ್ರಳಯ ತತ್ತ್ವವಿದೆ. ಸಂಸ್ಕೃತದಲ್ಲಿ ಸೃಷ್ಟಿ ಎಂಬ \enginline{creation} ಎಂದು ಅರ್ಥವಲ್ಲ. ಅದಕ್ಕೆ \enginline{projection} ಎಂಬ ಅರ್ಥವೇ ಸರಿ ಹೊಂದುತ್ತದೆ. ಇಂಗ್ಲಿಷಿನಲ್ಲಿ \enginline{creation} ಎಂಬ ಪದಕ್ಕೆ ದುರದೃಷ್ಟ ವಶಾತ್​ ಶೂನ್ಯದಿಂದ ಸೃಷ್ಟಿ, ಮುಂಚೆ ಏನೂ ಇರಲಿಲ್ಲ, ಅನಂತರ ಏನೋ ಅಲ್ಲೊಂದು ಉದಯಿಸಿತು ಎಂಬ ಭಯಾನಕ ಅಸಂಸ್ಕೃತ ಅರ್ಥದ ಸಂಬಂಧವಿದೆ. ಅದನ್ನು ನಂಬಿ ಎಂದು ಹೇಳಿ ನಾನು ನಿಮಗೆ ಅವಮಾನ ಮಾಡುವುದಿಲ್ಲ. ನಾವು ಬಳಸಬೇಕಾದ ಪದ \enginline{projection}. ಈ ಪ್ರಕೃತಿ ಯಾವಾಗಲೂ ಇರುವುದು. ಅದು ಸೂಕ್ಷ್ಮವೂ ಅವ್ಯಕ್ತವೂ ಆಗುತ್ತದೆ. ಕೆಲವು ಕಾಲದ ಮೇಲೆ ಪುನಃ ಹೊರಗೆ ಆವಿರ್ಭವಿಸುತ್ತದೆ. ಹಿಂದಿನ ಕಲ್ಪದಂತೆಯೇ ಸಂಮಿಶ್ರಣಗೊಂಡು ಹಲವು ರೂಪಗಳನ್ನು ತಾಳಿ, ಕೆಲವು ಕಾಲ ಹಾಗಿದ್ದು, ಪುನಃ ವಿಭಾಗವಾಗಿ ಅವ್ಯಕ್ತವಾಗುವುದು. ಅದಾದ ಮೇಲೆ ಪುನಃ ವ್ಯಕ್ತವಾಗುವುದು. ಅನಂತಕಾಲದವರೆವಿಗೂ ಅಲೆಯಂತೆ ಮೇಲೆದ್ದು ಬೀಳುತ್ತಿರುವುದು. ಕಾಲದೇಶಕಾರಣಗಳೆಲ್ಲಾ ಪ್ರಕೃತಿಯಲ್ಲಿರುವುವು. ಅದಕ್ಕೆ ಒಂದು ಆದಿ ಇತ್ತು ಎನ್ನುವುದು ಅಸಮಂಜಸ. ಆದಿ ಅಂತ್ಯಗಳ ಪ್ರಶ್ನೆ ಏಳುವುದಿಲ್ಲ. ನಮ್ಮ ಶಾಸ್ತ್ರದಲ್ಲಿ ಆದಿ ಅಂತ್ಯಗಳ ಪ್ರಶ್ನೆ ಇದ್ದರೆ ಅವು ಕೇವಲ ಒಂದು ಕಲ್ಪದ ಆದಿ ಅಂತ್ಯಗಳೆಂದು ಭಾವಿಸಬೇಕು. ಅದಕ್ಕಿಂತ ಹೆಚ್ಚಿಗೆ ಏನೂ ಅಲ್ಲ.

ಸೃಷ್ಟಿಗೆ ಕಾರಣ ಯಾವುದು? ದೇವರು. ಇಂಗ್ಲಿಷಿನ \enginline{God} ಎಂಬ ಶಬ್ದವನ್ನು ನಾನು ಯಾವ ಅರ್ಥದಲ್ಲಿ ಉಪಯೋಗಿಸುತ್ತೇನೆ? ಸಾಧಾರಣವಾಗಿ ಆ ಪದವನ್ನು ಉಪಯೋಗಿಸುವ ರೀತಿಯಲ್ಲಿ ಅಲ್ಲ – ಎಷ್ಟೋ ವ್ಯತ್ಯಾಸವಿದೆ. ಇಂಗ್ಲಿಷಿನಲ್ಲಿ ಸರಿಯಾದ ಬೇರೆ ಪದವಿಲ್ಲ. ಆದಕಾರಣ ಸಂಸ್ಕೃತದ ಬ್ರಹ್ಮ ಎಂಬ ಪದವನ್ನು ಉಪಯೋಗಿಸುತ್ತೇನೆ. ಅವನೇ ಸೃಷ್ಟಿಗೆಲ್ಲಾ ಮುಖ್ಯ ಕಾರಣ. ಬ್ರಹ್ಮನೆಂದರೆ ಯಾರು? ಅವನು ನಿತ್ಯ ಶುದ್ಧ ಸದಾ ಜಾಗೃತ ಸರ್ವಶಕ್ತಿಮಾನ್​ ಸರ್ವಜ್ಞ ಪರಮದಯಾಮಯ ಸರ್ವವ್ಯಾಪಿ ನಿರಾಕಾರ ಅಖಂಡ ಸ್ವರೂಪನು. ಅವನು ಈ ವಿಶ್ವವನ್ನು ಸೃಷ್ಟಿಸುವನು. ಅವನು ಯಾವಾಗಲೂ ಪ್ರಪಂಚವನ್ನು ಸೃಷ್ಟಿಸುತ್ತ ರಕ್ಷಿಸುತ್ತ ಇದ್ದರೆ ಎರಡು ಸಮಸ್ಯೆಗಳು ಬರುವುವು. ಈ ಜಗತ್ತಿನಲ್ಲಿ ಪಕ್ಷಪಾತವಿದೆ. ಒಬ್ಬನು ಸುಖದಲ್ಲಿ ಹುಟ್ಟುವನು, ಮತ್ತೊಬ್ಬನು ದುಃಖದಲ್ಲಿ ಹುಟ್ಟುವನು; ಒಬ್ಬ ಶ‍್ರೀಮಂತ, ಮತ್ತೊಬ್ಬ ಗರೀಬ. ಇದು ಪಕ್ಷಪಾತವನ್ನು ತೋರುವುದು. ಈ ಪ್ರಪಂಚದಲ್ಲಿ ಕ್ರೌರ್ಯವೂ ಇದೆ. ಇಲ್ಲಿ ಒಂದು ಪ್ರಾಣಿ ಜೀವಿಸಿರಬೇಕಾದರೆ ಮತ್ತೊಂದು ಪ್ರಾಣಿ ನಾಶವಾಗಬೇಕು. ಒಂದು ಮೃಗ ಮತ್ತೊಂದನ್ನು ಕೊಲ್ಲುವುದು. ಪ್ರತಿಯೊಬ್ಬನೂ ಸಾಧ್ಯವಾದ ಮಟ್ಟಿಗೆ ಇತರರಿಗಿಂತ ಮೇಲಿರಲು ಯತ್ನಿಸುವನು. ಪೈಪೋಟಿ, ಕ್ರೌರ್ಯ, ದುಃಖ ಇವುಗಳಿಂದ ಮತ್ತು ಹಗಲೂ ರಾತ್ರಿ ಜೀವವನ್ನು ಕರಗಿಸುವಂತಹ ಹಾಹಾಕಾರದಿಂದ ಜಗತ್ತು ತುಂಬಿದೆ. ಇದು ದೇವರ ಸೃಷ್ಟಿಯಾದರೆ, ಆ ದೇವರು ಕ್ರೌರ್ಯಕ್ಕಿಂತ ಭಯಾನಕನಾಗಿರಬೇಕು. ಅವನು ಮನುಷ್ಯನು ಕಲ್ಪಿಸಿಕೊಳ್ಳುವ ಸೈತಾನನಂತೆ ಕಠೋರನಾಗಿರಬೇಕು. ಈ ಪೋಟಾಪೋಟಿ ಇರುವುದಕ್ಕೆ, ತಾರತಮ್ಯವಿರುವುದಕ್ಕೆ, ದೇವರು ಕಾರಣನಲ್ಲ ಎಂದು ವೇದಾಂತವು ಸಾರುವುದು. ಇದನ್ನು ಮಾಡುವವರು ಯಾರು? ನಾವು. ಮೋಡವು ಎಲ್ಲಾ ಕಡೆಗಳಲ್ಲಿಯೂ ಮಳೆಯನ್ನು ಕರೆಯುತ್ತದೆ. ಆದರೆ ಯಾವ ಹೊಲವು ಸರಿಯಾಗಿ ಉಳಲ್ಪಟ್ಟಿದೆಯೋ ಅದು ಮಾತ್ರ ಇದರಿಂದ ಪ್ರಯೋಜನವನ್ನು ಪಡೆಯುವುದು. ಯಾವ ಹೊಲವನ್ನು ಉತ್ತಿಲ್ಲವೋ, ನಿರ್ಲಕ್ಷ್ಯದಿಂದ ನೋಡಿರುವೆವೋ ಅದಕ್ಕೆ ಇದರ ಪ್ರಯೋಜನವಿಲ್ಲ. ಅದು ಮೋಡದ ತಪ್ಪಲ್ಲ. ದೇವರ ದಯೆ ನಿತ್ಯವಾದುದು ಮತ್ತು ಅವಿಕಾರಿಯಾದುದು. ನಾವೇ ಈ ತಾರತಮ್ಯವನ್ನು ತರುವೆವು. ಆದರೆ ಕೆಲವರು ಸುಖದಲ್ಲಿ ಹುಟ್ಟುವರು ಮತ್ತೆ ಕೆಲವರು ದುಃಖದಲ್ಲಿ ಹುಟ್ಟುವರು; ಇದನ್ನು ಹೇಗೆ ವಿವರಿಸುವುದು? ಅವರು ಈ ವ್ಯತ್ಯಾಸಕ್ಕೆ ಕಾರಣರಲ್ಲ! ಈ ಜನ್ಮದಲ್ಲಿ ಅಲ್ಲ, ಕಳೆದ ಜನ್ಮದಲ್ಲಿ ಮಾಡಿದ ಕರ್ಮವು ಕಾರಣವಾಗಿರಬಹುದು.

ನಾವೆಲ್ಲರೂ, ಹಿಂದೂಗಳಾಗಲೀ ಬೌದ್ಧರಾಗಲೀ ಜೈನರಾಗಲೀ ಒಪ್ಪಿಕೊಳ್ಳುವ ಎರಡನೇ ನಿಯಮಕ್ಕೆ ಬರುತ್ತೇನೆ. ಜೀವವು ಅನಂತ, ಅದು ಶೂನ್ಯದಿಂದ ಆಗಿಲ್ಲ. ಅದು ಹಾಗೆ ಆಗಲಾರದು. ಅಂತಹ ಜೀವವನ್ನು ಹೊಂದಿ ಪ್ರಯೋಜನವಿಲ್ಲ. ಕಾಲದಲ್ಲಿ ಮೊದಲಾದುದೆಲ್ಲಾ ಒಂದಲ್ಲ ಒಂದು ದಿನ ಕಾಲದಲ್ಲಿ ನಾಶವಾಗಬೇಕು. ಜೀವವು ನಿನ್ನೆ ತಾನೆ ಆರಂಭವಾಗಿದ್ದರೆ ಅದು ನಾಳೆ ಅಂತ್ಯವನ್ನು ಹೊಂದಲೇ ಬೇಕು. ಸರ್ವನಾಶವೇ ಪರಿಣಾಮ. ಜೀವ ಯಾವಾಗಲೂ ಇರಬೇಕು. ನಾವು ಈ ವಿಷಯವನ್ನು ಇಂದು ತಿಳಿದುಕೊಳ್ಳಬೇಕಾದರೆ ಅಷ್ಟೊಂದು ವಿಚಾರಶಕ್ತಿಯ ಆವಶ್ಯಕತೆ ಇಲ್ಲ. ಆಧುನಿಕ ವಿಜ್ಞಾನ ಶಾಸ್ತ್ರಗಳೆಲ್ಲಾ ಈ ವಿಷಯದಲ್ಲಿ ನಮ್ಮ ನೆರವಿಗೆ ಬರುತ್ತಿವೆ. ಅವು ನಮ್ಮ ಶಾಸ್ತ್ರಗಳ ಸಿದ್ಧಾಂತವನ್ನು ಬಾಹ್ಯ ಘಟನೆ\-ಗಳ ಮೂಲಕ ಪ್ರಮಾಣಮಾಡಿ ತೋರುತ್ತಿವೆ. ಇದು ಆಗಲೇ ನಿಮಗೆ ಗೊತ್ತಿದೆ. ಪ್ರತಿಯೊಬ್ಬರೂ ಅನಂತ ಭೂತಕಾಲದ ಪರಿಣಾಮ. ಕವಿಗಳು ಹೇಳಲಿಚ್ಛಿಸುವಂತೆ ಪ್ರಕೃತಿಯ ಕರಗಳಿಂದ ಮಿಂಚಿನಂತೆ ಹೊಸದಾಗಿ ಶಿಶು ಈಗ ತಾನೇ ಪ್ರಪಂಚಕ್ಕೆ ಇಳಿಯಲಿಲ್ಲ. ಆ ಮಗುವಿನಲ್ಲಿ ಅನಂತ ಭೂತಕಾಲದ ಸಂಸ್ಕಾರವಿದೆ. ಒಳ್ಳೆಯದಕ್ಕೊ ಕೆಟ್ಟದಕ್ಕೊ ಪೂರ್ವಕರ್ಮವನ್ನು ಸವೆಯಿಸಲು ಅದು ಪ್ರಪಂಚಕ್ಕೆ ಬರುವುದು. ಅದೇ ವ್ಯತ್ಯಾಸಕ್ಕೆ ಕಾರಣ. ಇದೇ ಕರ್ಮನಿಯಮ. ಪ್ರತಿಯೊಬ್ಬನೂ ತನ್ನ ಅದೃಷ್ಟಕ್ಕೆ ತಾನೇ ಹೊಣೆಗಾರ. ವಿಧಿವಾದ, ಅದೃಷ್ಟವಾದ ಮುಂತಾದುವನ್ನೆಲ್ಲಾ ಇದು ತಳ್ಳಿಹಾಕುವುದು. ಈಶ್ವರ ಮತ್ತು ಜೀವ ಇವರನ್ನು ರಾಜಿ ಮಾಡುವ ಏಕಮಾತ್ರ ಸಿದ್ಧಾಂತವನ್ನು ಇದು ತೋರುವುದು. ನಾವು ಏನನ್ನು ಅನುಭವಿಸುತ್ತೇವೆಯೋ ಅದಕ್ಕೆ ನಾವೇ ಜವಾಬ್ದಾರರು. ನಮ್ಮ ಅದೃಷ್ಟಕ್ಕೆ ನಾವೇ ಹೊಣೆ, ನಾವೇ ಅದರ ಪರಿಣಾಮ. ಆದಕಾರಣ ನಾವು ಸ್ವತಂತ್ರರು. ನಾನು ದುಃಖಿಯಾಗಿದ್ದರೆ ನಾನೇ ಅದಕ್ಕೆ ಕಾರಣ. ಇಚ್ಛೆಪಟ್ಟರೆ ನಾನು ಸುಖಿಯಾಗಿರಬಹುದೆಂಬುದನ್ನು ಇದೇ ತೋರುವುದು. ನಾನು ಅಪವಿತ್ರನಾಗಿದ್ದರೆ ನಾನೇ ಅದಕ್ಕೆ ಕಾರಣ. ಇಚ್ಛೆಪಟ್ಟರೆ ಪವಿತ್ರನಾಗಬಲ್ಲೆ ಎಂಬುದನ್ನು ಇದು ತೋರುವುದು. ಮಾನವನ ಇಚ್ಛಾಶಕ್ತಿಯು ಎಲ್ಲಾ ಪರಿಸ್ಥಿತಿಗಳನ್ನು ಮೀರಿ ನಿಲ್ಲುತ್ತದೆ. ಮಾನವನ ಪ್ರಬಲ ವಿರಾಟ್​ ಅನಂತ ಇಚ್ಛಾಶಕ್ತಿಯ ಮತ್ತು ಸ್ವಾತಂತ್ರ್ಯದ ಎದುರಿಗೆ ಪ್ರಕೃತಿಯು ಬಾಗಿ ಶರಣಾಗಿ ಗುಲಾಮಳಾಗಬೇಕು. ಇದೇ ಕರ್ಮಸಿದ್ಧಾಂತದ ಪರಿಣಾಮ.

ಸ್ವಭಾವತಃ ಅನಂತರ ಬರುವ ಪ್ರಶ್ನೆಯೇ ಆತ್ಮವೆಂದರೇನು ಎಂಬುದು. ನಮ್ಮ ಶಾಸ್ತ್ರದ ಪ್ರಕಾರ ಆತ್ಮನನ್ನು ಅರಿಯದೆ ಈಶ್ವರನನ್ನು ತಿಳಿದುಕೊಳ್ಳಲಾರೆವು. ಬಾಹ್ಯ ಪ್ರಕೃತಿಯ ಸಂಶೋಧನೆಯಿಂದ ಅತೀಂದ್ರಿಯ ವಸ್ತುವಿನ ದರ್ಶನ ಪಡೆಯಲು ಭರತಖಂಡದಲ್ಲಿ ಮತ್ತು ಹೊರಗೆ ಪ್ರಯತ್ನಗಳು ನಡೆದಿವೆ. ಇವೆಲ್ಲ ಹೇಗೆ ಸಂಪೂರ್ಣವಾಗಿ ನಿಷ್ಪ್ರಯೋಜಕವಾಗಿದೆ ಎಂಬುದು ನಮಗೆಲ್ಲಾ ಗೊತ್ತಿದೆ. ಅತೀಂದ್ರಿಯ ವಸ್ತುವಿನ ಅನುಭವವನ್ನು ಪಡೆಯುವ ಬದಲು ಹೆಚ್ಚು ಹೆಚ್ಚು ಪ್ರಪಂಚವನ್ನು ಪರೀಕ್ಷಿಸಿದಷ್ಟು ನಾವು ಹೆಚ್ಚು ಹೆಚ್ಚು ಪ್ರಾಪಂಚಿಕರಾಗುತ್ತೇವೆ. ಹೆಚ್ಚು ಹೆಚ್ಚು ನಾವು ಪ್ರಾಪಂಚಿಕ ವಸ್ತುಗಳಲ್ಲಿ ಪರಿಣತರಾದಷ್ಟೂ ಮೊದಲು ಇದ್ದ ಸ್ವಲ್ಪ ಆಧ್ಯಾತ್ಮಿಕತೆ ಮಾಯವಾಗುವುದು ಕಾಣುತ್ತದೆ. ಆದ್ದರಿಂದ ಅಧ್ಯಾತ್ಮಕ್ಕೆ ಇದಲ್ಲ ದಾರಿ, ಪರಮ ಜ್ಞಾನಕ್ಕೆ ಇದಲ್ಲ ದಾರಿ. ಇದು ಮಾನವನ ಹೃದಯದ ಮೂಲಕ ಬರಬೇಕು, ಆತ್ಮದ ಮೂಲಕ ಬರಬೇಕು. ಬಾಹ್ಯಕ್ರಿಯೆಗಳು ಅತೀಂದ್ರಿಯ ವಸ್ತುವನ್ನು ಅನಂತವನ್ನು ವಿವರಿಸಲಾರವು. ಆಂತರಿಕವಾದುದು ಮಾತ್ರ ಅದನ್ನು ವಿವರಿಸಬಲ್ಲದು. ಮಾನವನ ಆತ್ಮದ ಪರೀಕ್ಷೆಯ ಮೂಲಕ ಮಾತ್ರ ನಾವು ದೇವರನ್ನು ತಿಳಿದುಕೊಳ್ಳಬಹುದು. ಭರತಖಂಡದಲ್ಲಿ ಭಿನ್ನ ಭಿನ್ನ ಪಂಗಡದವರಲ್ಲಿ ಜೀವಾತ್ಮನ ವಿಷಯದಲ್ಲಿ ಭಿನ್ನಾಭಿಪ್ರಾಯಗಳಿವೆ. ಅವರೆಲ್ಲರೂ ಒಪ್ಪಿಕೊಳ್ಳುವ ಕೆಲವು ವಿಷಯಗಳೂ ಇವೆ. ಆತ್ಮಕ್ಕೆ ಆದಿ ಅಂತ್ಯಗಳಿಲ್ಲವೆಂದು ನಾವು ಒಪ್ಪುತ್ತೇವೆ. ಅದು ಸ್ವಭಾವತಃ ಅಮೃತವಾದುದು. ಸರ್ವವಿಧ ಶಕ್ತಿ, ಆನಂದ, ಪಾವಿತ್ರ್ಯ, ಸರ್ವವ್ಯಾಪಿತ್ವ ಮತ್ತು ಸರ್ವಜ್ಞತ್ವ ಇವು ಪ್ರತಿಯೊಂದು ಆತ್ಮನಲ್ಲೂ ಅಂತರ್ನಿಹಿತವಾಗಿವೆ. ನಾವು ಜ್ಞಾಪಕದಲ್ಲಿಡಬೇಕಾದ ಶ್ರೇಷ್ಠಭಾವನೆ ಇದು. ಪ್ರತಿಯೊಬ್ಬ ಮಾನವನಲ್ಲಿಯೂ ಪ್ರತಿಯೊಂದು ಪ್ರಾಣಿಯಲ್ಲಿಯೂ ಸರ್ವಶಕ್ತನೂ ಸರ್ವಜ್ಞನೂ ಆದ ಆತ್ಮನು ನೆಲೆಸಿರುವನು. ಅತ್ಯಂತ ದುರ್ಬಲನಲ್ಲಿಯೂ ದುಷ್ಟನಲ್ಲಿಯೂ ಅತ್ಯಂತ ಕ್ಷುದ್ರಪ್ರಾಣಿ\- ಯಲ್ಲಿಯೂ ಒಂದೇ ಆತ್ಮನು ನೆಲೆಸಿರುವನು. ಆತ್ಮನಲ್ಲಿ ಅಲ್ಲ ವ್ಯತ್ಯಾಸ ಇರುವುದು ಅಭಿವ್ಯಕ್ತಿಯಲ್ಲಿ. ನನಗೂ ಅತಿ ಕ್ಷುದ್ರ ಕೀಟಕ್ಕೂ ಇರುವ ವ್ಯತ್ಯಾಸ ಅದರ ಅಭಿವ್ಯಕ್ತಿಯಲ್ಲಿ ಮಾತ್ರ. ಒಂದು ಸಿದ್ಧಾಂತದ ದೃಷ್ಟಿಯಿಂದ ನಾನು ಅದು ಎರಡೂ ಒಂದೇ, ನನ್ನ ಸಹೋದರ ಅದು. ನನ್ನಲ್ಲಿರುವಂತೆಯೇ ಅದರಲ್ಲಿಯೂ ಆತ್ಮವಿದೆ. ಭರತಖಂಡದ ಬೋಧನೆಯ ಮಹಾತತ್ತ್ವ ಇದು. ಮಾನವ ಸಹೋದರತ್ವ ಎಂಬ ಭಾವನೆ ಭರತಖಂಡದಲ್ಲಿ ಮೃಗ ಕ್ರಿಮಿಕೀಟ ಇವುಗಳಿಂದ ಕೂಡಿದ ವಿರಾಟ್​ ಜೀವನದ ಸಹೋದರತ್ವ ಭಾವವಾಗುವುದು. ಇವುಗಳೆಲ್ಲಾ ನಮ್ಮ ದೇಹವೆ. ಅದಕ್ಕೇ ನಮ್ಮ ಶಾಸ್ತ್ರ “ಜ್ಞಾನಿಯು, ಒಬ್ಬನೇ ಈಶ್ವರ ಎಲ್ಲಾ ಭೂತಗಳಲ್ಲೂ ಇರುವನೆಂದು ತಿಳಿದು ಎಲ್ಲವನ್ನೂ ಆರಾಧಿಸುವನು” ಎನ್ನುವುದು. ಅದಕ್ಕೆಯೇ ಭರತಖಂಡದಲ್ಲಿ ಜನರು ದೀನರು ಪ್ರಾಣಿಗಳು ಮುಂತಾದವರನ್ನು ದಯಾ ದೃಷ್ಟಿಯಿಂದ ನೋಡುವುದು. ಆತ್ಮನ ವಿಷಯದಲ್ಲಿ ನಾವೆಲ್ಲರೂ ಒಪ್ಪಿಕೊಳ್ಳುವ ಒಂದು ಸಾಮಾನ್ಯ ಭಾವನೆ ಇದು.

ಈಗ ನಾವು ಸ್ವಾಭಾವಿಕವಾಗಿ ಈಶ್ವರ ತತ್ತ್ವದ ವಿಚಾರಕ್ಕೆ ಬರುತ್ತೇವೆ. ಆತ್ಮನ ವಿಚಾರವಾಗಿ ಮತ್ತೊಂದು ವಿಷಯ. ಆಂಗ್ಲಭಾಷೆಯನ್ನು ಓದುವವರಿಗೆ \enginline{soul, mind} ಎಂಬ ಪದಗಳು ಭ್ರಮೆಯನ್ನುಂಟುಮಾಡುವುವು. ನಾವು ಹೇಳುವ ಆತ್ಮ ಮತ್ತು ಆಂಗ್ಲಭಾಷೆಯ ಇವು ಬೇರೆ ಬೇರೆ ಅರ್ಥವನ್ನು ಕೊಡುತ್ತವೆ. ನಾವು ಯಾವುದನ್ನು ಮನಸ್ಸು \enginline{(mind)} ಎನ್ನುವೆವೋ ಅದನ್ನು ಪಾಶ್ಚಾತ್ಯರು \enginline{soul} ಎನ್ನುವರು. ಇಪ್ಪತ್ತು ವರ್ಷಗಳ ಹಿಂದೆ ಸಂಸ್ಕೃತ ತತ್ತ್ವಜ್ಞಾನದಿಂದ ಈ ಪದ ಸಿಕ್ಕುವವರೆಗೆ ಅವರಿಗೆ ಜೀವಾತ್ಮದ ಕಲ್ಪನೆ ಇರಲಿಲ್ಲ. ದೇಹ ಇಲ್ಲಿದೆ, ಇದರಾಚೆ ಮನಸ್ಸು. ಆದರೆ ಮನಸ್ಸೇ ಆತ್ಮವಲ್ಲ. ಅದು ಸೂಕ್ಷ್ಮ ಶರೀರ. ಅದು ತನ್ಮಾತ್ರದಿಂದ ಆದುದು. ಜನ್ಮ ಜನ್ಮಾಂತರಗಳವರೆಗೂ ಅದು ಇರುವುದು. ಮನಸ್ಸಿನ ಹಿಂದೆ ಆತ್ಮವಿರುವುದು. ನಾವು ಅದನ್ನು \enginline{soul} ಅಥವಾ \enginline{mind} ಎಂದು ಭಾಷಾಂತರ ಮಾಡಲಾಗು\-ವುದಿಲ್ಲ. ನಾವು ಅದನ್ನು ಆತ್ಮ ಅಥವಾ ಪಾಶ್ಚಾತ್ಯ ದಾರ್ಶನಿಕರು ಹೇಳುವಂತೆ \enginline{self} ಎಂದು ಕರೆಯಬೇಕು. ನೀವು ಯಾವ ಪದವನ್ನಾದರೂ ಉಪಯೋಗಿಸಿ, ಆದರೆ ಇದನ್ನು ಮಾತ್ರ ನೆನಪಿಟ್ಟಿರಿ–ಆತ್ಮವು ಮನಸ್ಸಿಗಿಂತ ಬೇರೆ ದೇಹಕ್ಕಿಂತ ಬೇರೆ. ಈ ಆತ್ಮವು ಜನನ ಮರಣಗಳಲ್ಲಿ ತನ್ನ ಸೂಕ್ಷ್ಮ ಶರೀರದೊಂದಿಗೆ ಹೋಗುವುದು. ಜ್ಞಾನ ಪಡೆದಾದ ಮೇಲೆ, ಪೂರ್ಣತೆಯ ಗುರಿಯನ್ನು ಮುಟ್ಟಿದ ಮೇಲೆ, ಜನನ ಮರಣಗಳು ಅದಕ್ಕೆ ಇಲ್ಲ. ಆಗ ಅದು ಸೂಕ್ಷ್ಮ ಶರೀರವನ್ನು ಬೇಕಾದರೆ ಇಟ್ಟುಕೊಂಡಿರಬಹುದು ಇಲ್ಲದೆ ಇದ್ದರೆ ತ್ಯಜಿಸಿ ಅನಂತಕಾಲದವರೆಗೆ ಸ್ವತಂತ್ರವಾಗಿ ಮುಕ್ತವಾಗಿರಬಹುದು. ಆತ್ಮನ ಗುರಿ ಸ್ವಾತಂತ್ರ್ಯ. ಇದು ನಮ್ಮ ಧರ್ಮದ ಒಂದು ವೈಶಿಷ್ಟ್ಯ. ನಮ್ಮಲ್ಲಿಯೂ ಸ್ವರ್ಗ ನರಕಗಳಿವೆ. ಆದರೆ ಅವು ಅನಂತವಲ್ಲ. ಸ್ವಭಾವತಃ ಅವು ಹಾಗೆ ಇರಲಾರವು. ಸ್ವರ್ಗವೇನಾದರೂ ಇದ್ದರೆ ಅದು ನಮ್ಮ ಪ್ರಪಂಚದ ದೊಡ್ಡ ಪ್ರಮಾಣದ ಪ್ರತಿರೂಪವಾಗಿರುತ್ತದೆ. ಅಲ್ಲಿ ಇಲ್ಲಿಗಿಂತ ಹೆಚ್ಚಿನ ಸುಖಭೋಗಗಳು ಇರಬಹುದು. ಆದರೆ ಇದರಿಂದ ಜೀವನಿಗೇ ಹಾನಿ. ಇಂತಹ ಸ್ವರ್ಗಗಳು ಎಷ್ಟೋ ಇವೆ. ಫಲಾಪೇಕ್ಷೆಯಿಂದ ಯಾರು ಇಲ್ಲಿ ಸತ್ಕರ್ಮವನ್ನು ಮಾಡುವರೋ ಅವರು ಕಾಲವಾದರೆ ಇಂತಹ ಸ್ವರ್ಗದಲ್ಲಿ ಇಂದ್ರ ಮುಂತಾದವರಂತೆ ಹುಟ್ಟುವರು. ದೇವತೆಗಳು ಎಂದರೆ ಕೆಲವು ಪದವಿಗಳ ಹೆಸರುಗಳು. ಅವರೂ ಮನುಷ್ಯರಾಗಿದ್ದರು; ಅವರು ಸತ್ಕರ್ಮದಿಂದ ದೇವತೆಗಳಾದರು. ನೀವು ಓದುವ ಇಂದ್ರಾದಿ ದೇವತೆಗಳು ಹೆಸರುಗಳ ವ್ಯಕ್ತಿಗಳ ಹೆಸರುಗಳಲ್ಲ, ಪದವಿಗಳ ಹೆಸರುಗಳು. ಸಾವಿರಾರು ಇಂದ್ರರಿರುವರು. ನಹುಷ ಪ್ರಖ್ಯಾತ ರಾಜನಾಗಿದ್ದ: ಕಾಲವಾದ ಮೇಲೆ ಇಂದ್ರನಾದ, ಅದೊಂದು ಪದವಿ. ಒಂದು ಜೀವಿ ಸತ್ಕರ್ಮದಿಂದ ಮೇಲೆ ಹೋಗಿ ಇಂದ್ರನಾಗುವನು. ಅಲ್ಲಿ ಕೆಲವು ಕಾಲ ಇರುವನು. ಅವನು ಕಾಲವಾದ ಮೇಲೆ ಪುನಃ ಮಾನವನಾಗಿ ಹುಟ್ಟವನು. ಆದರೆ ಮನುಷ್ಯ ದೇಹವೇ ಸರ್ವಶ್ರೇಷ್ಠ. ಕೆಲವು ದೇವತೆಗಳು ಉನ್ನತ ಸ್ತರಗಳನ್ನು ಸೇರುವ ಇಚ್ಛೆಯಿಂದ ಸ್ವರ್ಗದ ಸುಖಭೋಗಗಳನ್ನು ತ್ಯಜಿಸಬಹುದು. ಆದರೆ ಈ ಪ್ರಪಂಚದಲ್ಲಿ ಐಶ್ವರ್ಯ, ಅಧಿಕಾರ, ಭೋಗ ಬಹುಪಾಲು ಜನರನ್ನು ಹೇಗೆ ಭ್ರಾಂತರನ್ನಾಗಿ ಮಾಡುವುವೋ ಹಾಗೆಯೇ ಅವು ದೇವತೆಗಳನ್ನೂ ಭ್ರಾಂತರನ್ನಾಗಿ ಮಾಡುವುವು. ದೇವತೆಗಳು ತಮ್ಮ ಸತ್ಕರ್ಮಗಳ ಫಲವನ್ನು ಅನುಭವಿಸಿದ ಮೇಲೆ ಪತಿತರಾಗಿ ಮಾನವ ಜನ್ಮವನ್ನು ಪಡೆಯುವರು. ಆದಕಾರಣವೇ ಈ ಭೂಮಿಯು ಕರ್ಮಭೂಮಿ. ನಾವು ಈ ಭೂಮಿಯಲ್ಲಿಯೇ ಮೋಕ್ಷವನ್ನು ಪಡೆಯಲು ಸಾಧ್ಯ. ಆದ್ದರಿಂದ ಸ್ವರ್ಗಲೋಕಗಳನ್ನು ಕೂಡ ನಾವು ಪಡೆಯಲು ಯೋಗ್ಯವಾದುವಲ್ಲ. ಆದರೆ ನಾವು ಯಾವುದನ್ನು ಇಚ್ಛಿಸಬೇಕು? ಮುಕ್ತಿಯನ್ನು.

ಅತಿ ಉತ್ಕೃಷ್ಟ ಸ್ವರ್ಗದಲ್ಲಿದ್ದರೂ ನೀವು ಪ್ರಕೃತಿಯ ಗುಲಾಮರು ಎಂದು ನಮ್ಮ ಶಾಸ್ತ್ರಗಳು ಹೇಳುತ್ತವೆ. ಇಪ್ಪತ್ತು ಸಾವಿರ ವರುಷಗಳವೆರೆಗೆ ನೀವು ರಾಜರಾದರೆ ತಾನೆ ಪ್ರಯೋಜನವೇನು? ಎಲ್ಲಿಯವರೆಗೆ ನಿಮಗೆ ದೇಹವಿದೆಯೋ, ಎಲ್ಲಿಯವರೆಗೆ ನೀವು ಸುಖಕ್ಕೆ ದಾಸರೋ, ಎಲ್ಲಿಯವರೆಗೆ ನೀವು ಕಾಲದೇಶ ಎಲ್ಲೆಯೊಳಗೆ ಇರುವಿರೋ, ಅಲ್ಲಿಯವರೆಗೆ ನೀವು ಗುಲಾಮರು. ಬಾಹ್ಯ ಮತ್ತು ಆಂತರಿಕ ಪ್ರಕೃತಿಯಿಂದ ಮುಕ್ತನಾಗುವುದೇ ಗುರಿ. ಪ್ರಕೃತಿಯು ನಿಮ್ಮ ಪಾದಕ್ಕೆ ಬೀಳಬೇಕು. ನೀವು ಅದನ್ನು ತುಳಿಯಬೇಕು. ಮುಕ್ತರಾಗಿ ನಿಮ್ಮ ಮಹಿಮೆಯಲ್ಲಿ ನೀವು ಪ್ರತಿಷ್ಠಿತರಾಗಬೇಕು. ನಿಮಗೆ ಇನ್ನು ಜನನವಿಲ್ಲ,\break ಮರಣವೂ ಇಲ್ಲ. ಇನ್ನು ಮೇಲೆ ಸುಖವಿಲ್ಲ. ಅಂದರೆ ದುಃಖವೂ ಇಲ್ಲ. ಆಗ ನೀವು ಸರ್ವಾತೀತ ಅವ್ಯಕ್ತ ಅವಿನಾಶಿ ಆನಂದಕ್ಕೆ ಅಧಿಕಾರಿಯಾಗುವಿರಿ. ನಾವು ಇಲ್ಲಿ ಯಾವುದನ್ನು ಸುಖ ಮತ್ತು ಆನಂದ ಎನ್ನುವೆವೋ ಅದು ಆ ಆನಂದದ ಕಣ ಮಾತ್ರ. ಅನಂತ ಆನಂದವೇ ನಮ್ಮ ಗುರಿ.

ಆತ್ಮವು ಲಿಂಗಾತೀತವಾದುದು. ಅದನ್ನು ಪುರುಷ ಅಥವಾ ಸ್ತ್ರೀ ಎಂದು ಹೇಳಲಾರೆವು. ಲಿಂಗವೂ ದೇಹಕ್ಕೆ ಮಾತ್ರ ಅನ್ವಯಿಸುವುದು. ಆತ್ಮನ ಸಂಬಂಧವಾಗಿ ಹೇಳುವ ಸ್ತ್ರೀ ಪುರುಷ ಭಾವನೆಗಳೆಲ್ಲ ಕೇವಲ ಭ್ರಾಂತಿ. ಅದು ದೇಹಕ್ಕೆ ಮಾತ್ರ ಅನ್ವಯಿಸುವುದು. ಹಾಗೆಯೇ ವಯಸ್ಸೂ ಕೂಡ. ಆತ್ಮಕ್ಕೆ ವಯಸ್ಸಾಗುವುದಿಲ್ಲ, ಆ ಪುರಾಣ ಪುರುಷ ಯಾವಾಗಲೂ ಹಾಗೆಯೇ ಇರುವನು. ಅದು ಪ್ರಪಂಚಕ್ಕೆ ಬಂದದ್ದು ಹೇಗೆ? ಅದಕ್ಕೆ ಒಂದು ಉತ್ತರ ನಮ್ಮ ಶಾಸ್ತ್ರದಲ್ಲಿ ಇದೆ. ಈ ಬಂಧನಕ್ಕೆಲ್ಲಾ ಕಾರಣ ಅವಿದ್ಯೆ. ನಾವು ಜ್ಞಾನದಿಂದ ಬದ್ಧರಾಗಿರುವೆವು. ಜ್ಞಾನವು ನಮ್ಮನ್ನು ಮುಕ್ತರನ್ನಾಗಿ ಮಾಡುವುದು. ಈ ಜ್ಞಾನ ಬರುವುದು ಹೇಗೆ? ಭಕ್ತಿಯಿಂದ ಭಗವಂತನ ಪೂಜೆಯಿಂದ, ಎಲ್ಲರೂ ಭಗವಂತನ ದೇವಾಲಯಗಳೆಂದು ಪ್ರೀತಿಸುವುದರಿಂದ, ಅದು ಸಾಧ್ಯ. ಅವನು ಎಲ್ಲರಲ್ಲಿಯೂ ಇರುವನು. ಪರಮ ಭಕ್ತಿಯಿಂದ ಜ್ಞಾನ ಲಭಿಸುವುದು. ಅಜ್ಞಾನ ಮಾಯವಾಗುವುದು, ಜೀವಿ ಮುಕ್ತನಾಗುವನು.

ನಮ್ಮ ಶಾಸ್ತ್ರದಲ್ಲಿ ಎರಡು ಬಗೆಯ ದೇವರ ಭಾವನೆಗಳು ಇವೆ; ಒಂದು ಸಗುಣ, ಮತ್ತೊಂದು ನಿರ್ಗುಣ. ಸಗುಣವೆಂದರೆ ಸರ್ವವ್ಯಾಪಿಯಾದ ಸೃಷ್ಟಿಕರ್ತ, ರಕ್ಷಕ ಮತ್ತು ಸಂಹಾರಕ. ಅವನೇ ವಿಶ್ವಕ್ಕೆ ನಿತ್ಯ ತಾಯಿ ತಂದೆ. ಆದರೆ ಅವನು ಪ್ರಕೃತಿಗಿಂತ ಮತ್ತು ಜೀವಿಗಳಿಗಿಂತ ಪ್ರತ್ಯೇಕನಾಗಿರುವನು. ಮೋಕ್ಷವೆಂದರೆ ಅವನ ಹತ್ತಿರ ಬರುವುದು, ಅವನಲ್ಲಿ ಬಾಳುವುದು. ನಿರ್ಗುಣ ದೇವರ ಭಾವನೆ ಮತ್ತೊಂದು ಇರುವುದು. ಇಲ್ಲಿ ಗುಣವಾಚಕಗಳನ್ನು ಅನಾವಶ್ಯಕವೆಂದೂ ಅಯುಕ್ತವೆಂದೂ ತೆಗೆದು ಹಾಕುತ್ತಾರೆ. ಅವ್ಯಕ್ತವಾದ ಸರ್ವವ್ಯಾಪಿಯಾದ ನಿರ್ಗುಣವೊಂದೇ ಉಳಿಯುವುದು. ಅದನ್ನು ಜ್ಞಾತೃ ಎನ್ನಲಾಗುವುದಿಲ್ಲ. ಜ್ಞಾನ ಮಾನವ ಮನಸ್ಸಿಗೆ ಮಾತ್ರ ಸೇರಿದ್ದು. ಅದನ್ನು ವಿಚಾರಶಕ್ತಿಯುಳ್ಳ ಜೀವಿ ಎಂದೂ ಕರೆಯಲಾಗುವುದಿಲ್ಲ. ಏಕೆಂದರೆ ವಿಚಾರವು ದೌರ್ಬಲ್ಯದ ಚಿಹ್ನೆ. ಅದನ್ನು ಸೃಷ್ಟಿಕರ್ತ ಎನ್ನಲಾರೆವು. ಬಂಧನದಲ್ಲಿ ಇಲ್ಲದೇ ಇದ್ದರೆ ಯಾರೂ ಸೃಷ್ಟಿಸಲಾರರು. ಅದಕ್ಕೆ ಇರುವ ಬಂಧನ ಯಾವುದು? ಆಸೆಯ ತೃಪ್ತಿಗಾಗಿ ಮಾತ್ರ ಎಲ್ಲರೂ ಕೆಲಸ ಮಾಡುವುದು. ಅದಕ್ಕೆ ಇರುವ ಆಸೆ ಯಾವುದು? ಕೊರತೆಯ ಪೂರ್ಣತೆಗೆ ಯಾರಾದರೂ ಕೆಲಸ ಮಾಡಬಹುದು. ಅದಕ್ಕೆ ಇರುವ ಕೊರತೆ ಯಾವುದು? ವೇದಗಳು ಈ ನಿರ್ಗುಣ ದೇವರನ್ನು ‘ಅದು’ ಎಂದು ನಿರ್ದೇಶಿಸುತ್ತವೆ. ‘ಅವನು’ ಎಂದು ಕರೆಯುವುದಿಲ್ಲ. ‘ಅವನು’ ಎಂದರೆ ಒಂದು ಲಿಂಗವಾಚಕವಾಗುತ್ತದೆ, ಒಂದು ವ್ಯಕ್ತಿಯನ್ನು ನಿರ್ದೇಶಿಸುತ್ತದೆ. ಆದ್ದರಿಂದ ವೇದಗಳು ‘ಅದು’ ಎಂದು ಬೋಧಿಸುತ್ತವೆ. ಈ ಸಿದ್ಧಾಂತವನ್ನೇ ಅದ್ವೈತವೆನ್ನುವುದು.

ಈ ನಿರ್ಗುಣದೊಂದಿಗೆ ನಮ್ಮ ಸಂಬಂಧವೆಂತಹುದು? ನಾವೇ ಅದು. ನಾವು ಮತ್ತು ಅದು ಎರಡೂ ಒಂದೇ. ಎಲ್ಲರೂ ಆ ನಿರ್ಗುಣದ ಅಭಿವ್ಯಕ್ತಿಗಳು. ಎಲ್ಲರಿಗೂ ಅದೇ ಮೂಲವಸ್ತು. ನಾವು ಅವ್ಯಕ್ತ ಅನಂತದಿಂದ ಬೇರೆ ಎಂದು ಭಾವಿಸುವುದೇ ದುಃಖಕ್ಕೆ ಕಾರಣ. ಮುಕ್ತಿ ಎಂದರೆ ಈ ಅದ್ವೈತ ನಿರ್ಗುಣವೇ ನಾವು ಎಂದು ಭಾವಿಸುವುದು. ಸಂಕ್ಷೇಪವಾಗಿ ಹೇಳುವುದಾದರೆ ನಮ್ಮ ಶಾಸ್ತ್ರಗಳಲ್ಲಿ ದೇವರನ್ನು ಕುರಿತಂತೆ ದೊರಕುವ ಎರಡು ಭಾವನೆಗಳು ಇವು.

ಇಲ್ಲಿ ಕೆಲವು ವಿವರಣೆಗಳು ಆವಶ್ಯಕ. ನಿರ್ಗುಣ ಬ್ರಹ್ಮವಾದದಿಂದ ಮಾತ್ರವೇ ಯಾವುದೇ ಬಗೆಯ ನೀತಿತತ್ತ್ವವನ್ನು ಪ್ರತಿಪಾದಿಸುವುದು ಸಾಧ್ಯ. ಎಲ್ಲಾ ದೇಶಗಳಲ್ಲಿಯೂ ಅತಿ ಪುರಾತನ ಕಾಲದಿಂದಲೂ ಇತರರನ್ನು ನಿಮ್ಮಂತೆಯೇ ನೋಡಿ. ನಿಮ್ಮಂತೆಯೇ ಪ್ರೀತಿಸಿ, ಎಂಬ ಸತ್ಯವನ್ನು ಬೋಧಿಸಿರುವರು. ಭರತಖಂಡದಲ್ಲಿ ಎಲ್ಲರನ್ನೂ ನೀವೇ ಆಗಿದ್ದೀರಿ ಎಂಬಂತೆ ಪ್ರೀತಿಸಿ ಎಂದು ಬೋಧಿಸಿರುವರು. ಪ್ರಾಣಿಗಳು ಮತ್ತು ಮನುಷ್ಯರ ನಡುವೆ ಯಾವ ವ್ಯತ್ಯಾಸವನ್ನೂ ನಾವುಮಾಡುವುದಿಲ್ಲ. ಆದರೆ ಇದಕ್ಕೆ ಸರಿಯಾದ ಕಾರಣ ಗೊತ್ತಿರಲಿಲ್ಲ. ಇತರರನ್ನು ನಮ್ಮಂತೆ ಪ್ರೀತಿಸಿದರೆ ಪ್ರಯೋಜನವೇನು ಎಂದು ಗೊತ್ತಿರಲಿಲ್ಲ. ನಿರ್ಗುಣ ಬ್ರಹ್ಮವಾದದಲ್ಲಿ ಮಾತ್ರ ಇದಕ್ಕೆ ವಿವರಣೆ ಇದೆ. ಈ ಸೃಷ್ಟಿಯೆಲ್ಲಾ ಒಂದು, ಏಕ, ಎಲ್ಲಾ ಜೀವನವೂ ಒಂದು. ನಾನು ಮತ್ತೊಂದನ್ನು ಪೀಡಿಸಿದರೆ ನನ್ನನ್ನು ನಾನೇ ಪೀಡಿಸಿಕೊಳ್ಳುತ್ತಿರುವೆನು. ಅನ್ಯರನ್ನು ಪ್ರೀತಿಸಿದರೆ ನನ್ನನ್ನು ನಾನೇ ಪ್ರೀತಿಸುತ್ತಿರುವೆನು. ಆದ್ದರಿಂದ ನಾವು ಮತ್ತೊಬ್ಬರಿಗೆ ಏಕೆ ನೋವನ್ನುಂಟುಮಾಡಕೂಡದು ಎಂಬುದು ತಿಳಿಯುತ್ತದೆ. ಈ ನೀತಿಗೆ ಕಾರಣ ನಮಗೆ ನಿರ್ಗುಣ ಬ್ರಹ್ಮವಾದದಲ್ಲಿ ದೊರಕುವುದು. ಈಗ ಸಗುಣ ಈಶ್ವರನ ಸ್ಥಾನವೇನು ಎಂಬ ಪ್ರಶ್ನೆ ಏಳುವುದು. ಸಗುಣೋಪಾಸನೆಯಲ್ಲಿ ಬರುವ ಅದ್ಭುತ ಭಕ್ತಿಯನ್ನು ನಾನು ಬಲ್ಲೆ. ಜನರಿಗೆ ಭಕ್ತಿಯ ಆವಶ್ಯಕತೆ ಮತ್ತು ಅದರಲ್ಲಿರುವ ಶಕ್ತಿ ನನಗೆ ಗೊತ್ತು. ಆದರೆ ದೇಶದಲ್ಲಿ ನಮಗೆ ಇಂದು ಬೇಕಾಗಿರುವುದು ಅಳುವಲ್ಲ, ಸ್ವಲ್ಪ ಶಕ್ತಿ. ಎಲ್ಲಾ ಮೂಢನಂಬಿಕೆಗಳನ್ನೂ ಕಿತ್ತೊಗೆದು, ನಾನೇ ಅವ್ಯಕ್ತಬ್ರಹ್ಮನೆಂದು ನಿರ್ಭೀತಿಯಿಂದ ನಿಂತರೆ, ಎಂತಹ ಶಕ್ತಿಗಣಿ ಅಲ್ಲಿರುವುದು! ಯಾವುದಕ್ಕೆ ನಾನು ಅಂಜುವುದು? ಪ್ರಕೃತಿ ನಿಯಮಗಳನ್ನೂ ನಾನು ಲೆಕ್ಕಿಸುವುದಿಲ್ಲ, ಮೃತ್ಯುವನ್ನು ನೋಡಿ ನಗುವೆನು. ನಾನು ಅನಂತವಾದ ಸನಾತನವಾದ ಚ್ಯುತಿಯಿಲ್ಲದ ನಾಶವಿಲ್ಲದ ನನ್ನ ಆತ್ಮಮಹಿಮೆಯ ಮೇಲೆ ನಿಲ್ಲುವೆನು. ಯಾವ ಶಸ್ತ್ರವೂ ಅದನ್ನು ಭೇದಿಸಲಾರದು, ಶಾಖ ಒಣಗಿಸಲಾರದು, ಬೆಂಕಿ ಸುಡಲಾರದು, ನೀರು ಕರಗಿಸಲಾರದು. ಅದು ಅನಂತ, ಜನನಮರಣಾತೀತ ಮತ್ತು ಆದಿ–ಅಂತ್ಯ ರಹಿತ. ಅದರ ಬೃಹತ್ತಿನ ಮುಂದೆ ಸೂರ್ಯಚಂದ್ರಾದಿಗಳು ಸಾಗರದ ಬಿಂದುಗಳಂತೆ ಕಾಣುವುವು. ಅದರ ಮಹಿಮೆಯ ಎದುರಿಗೆ ದೇಶ ಮಾಯವಾಗುವುದು. ಕಾಲ ತನ್ನ ಅಸ್ತಿತ್ವವನ್ನೇ ಕಳೆದುಕೊಳ್ಳುತ್ತದೆ. ಇಂತಹ ಮಹಾ ಆತ್ಮವನ್ನು ನಾವು ನಂಬಬೇಕು; ಅದರಿಂದ ಶಕ್ತಿ ಬರುವುದು; ನೀವು ಇಚ್ಛಿಸುವುದೆಲ್ಲಾ ಪ್ರಾಪ್ತವಾಗುವುದು. ದುರ್ಬಲರೆಂದು ಯೋಚಿಸಿದರೆ ದುರ್ಬಲವಾಗುವಿರಿ. ಶಕ್ತಿವಂತರೆಂದು ಯೋಚಿಸಿದರೆ ಶಕ್ತಿವಂತರಾಗುವಿರಿ. ಅಶುದ್ಧರೆಂದು ಭಾವಿಸಿದರೆ ಅಶುದ್ಧರಾಗುವಿರಿ. ಶುದ್ಧರೆಂದು ಯೋಚಿಸಿದರೆ ಶುದ್ಧರಾಗುವಿರಿ. ದುರ್ಬಲರೆಂದು ಭಾವಿಸದೆ, ತೇಜಸ್ವಿಗಳು ನಾವು, ಸರ್ವಶಕ್ತಿವಂತರು ನಾವು, ಜ್ಞಾನಿಗಳು ನಾವು ಎಂದು ಭಾವಿಸಿ ಎನ್ನುವುದು ಅದ್ವೈತವಾದ. ಅದನ್ನು ನಾನು ಇದುವರೆಗೆ ವ್ಯಕ್ತಪಡಿಸದೆ ಇದ್ದರೆ ಚಿಂತೆಯಿಲ್ಲ. ಅದು ನನ್ನಲ್ಲಿರುವುದು, ಅನಂತ ಜ್ಞಾನ ನನ್ನಲ್ಲಿದೆ, ಅನಂತಶಕ್ತಿ, ಪವಿತ್ರತೆ, ಸ್ವಾತಂತ್ರ್ಯವೆಲ್ಲ ನನ್ನಲ್ಲಿದೆ. ನಾನೇಕೆ ಈ ಜ್ಞಾನವನ್ನು ವ್ಯಕ್ತಪಡಿಸಬಾರದು? ನನಗೆ ಅದರಲ್ಲಿ ಶ್ರದ್ಧೆಯಿಲ್ಲ. ಶ್ರದ್ಧೆ ಹುಟ್ಟಲಿ, ಇವೆಲ್ಲ ವ್ಯಕ್ತವಾಗುವುವು. ನಿರ್ಗುಣಬ್ರಹ್ಮ ಭಾವನೆಯು ಬೋಧಿಸುವುದು ಇದನ್ನು. ಬಾಲ್ಯದಿಂದಲೇ ನಿಮ್ಮ ಮಕ್ಕಳನ್ನು ಪೌರುಷವಂತರನ್ನಾಗಿ ಮಾಡಿ. ಅವರಿಗೆ ದುರ್ಬಲತೆಯನ್ನು ಮತ್ತು ಬರಿಯ ಬಾಹ್ಯಾಚಾರಗಳನ್ನು ಬೋಧಿಸಬೇಡಿ. ಬಲಾಢ್ಯರನ್ನಾಗಿ ಮಾಡಿ. ಅವರು ತಮ್ಮ ಕಾಲಿನ ಮೇಲೆ ತಾವು ನಿಂತುಕೊಳ್ಳಲಿ, ಧೀರರಾಗಲಿ, ವಿಜಯಿಗಳಾಗಲಿ, ಸಹಿಷ್ಣುಗಳಾಗಲಿ, ಆತ್ಮನ ಮಹಾತ್ಮೆಯನ್ನು ಮೊದಲು ಅರಿಯಲಿ. ನಿಮಗೆ ಈ ಭಾವನೆ ವೇದಾಂತದಲ್ಲಿ ಮಾತ್ರ ದೊರಕುವುದು. ಉಳಿದ ಧರ್ಮಗಳಲ್ಲಿರುವಂತೆಯೇ ಅದರಲ್ಲಿಯೂ ಪ್ರೀತಿ ಪೂಜೆ ಮುಂತಾದವು ಬಹಳ ಇದೆ. ಆದರೆ ಜೀವನೋತ್ಸಾಹವನ್ನು ತುಂಬುವ ಅದ್ಭುತವಾದ ಈ ಆತ್ಮಭಾವನೆ ಇಲ್ಲಿ ಮಾತ್ರ ಇದೆ. ಜಗತ್ತಿನಲ್ಲೆಲ್ಲಾ ಒಂದು ಕ್ರಾಂತಿಯನ್ನು ತರುವ ಭಾವನೆ ಇಲ್ಲಿ ಮಾತ್ರ ಇದೆ. ಜಡವಿಜ್ಞಾನವನ್ನು ಧರ್ಮದೊಂದಿಗೆ ಸಮನ್ವಯಗೊಳಿಸುವುದು ಇದು ಮಾತ್ರ.

ನಮ್ಮ ಧರ್ಮದ ಮುಖ್ಯ ತತ್ತ್ವಗಳನ್ನು ನಿಮ್ಮೆದುರಿಗೆ ಇಡಲು ಯತ್ನಿಸಿರುವೆ. ಇದನ್ನು ಹೇಗೆ ಪ್ರಯೋಗಿಸಬೇಕು, ಹೇಗೆ ಅನುಷ್ಠಾನಕ್ಕೆ ತರಬೇಕು ಎಂಬ ವಿಷಯದ ಮೇಲೆ ಸ್ವಲ್ಪ ಹೇಳಬೇಕಾಗಿದೆ. ಸ್ವಾಭಾವಿಕವಾಗಿ ಭರತಖಂಡದಲ್ಲಿ ಇರುವ ಪರಿಸ್ಥಿತಿಗಳಿಗನುಸಾರವಾಗಿ ಹಲವು ಪಂಥಗಳು ಇರಲೇಬೇಕು. ವಾಸ್ತವವಾಗಿ ಹಲವು ಪಂಥಗಳಿವೆ. ಆದರೆ ಒಬ್ಬರು ಮತ್ತೊಬ್ಬರೊಂದಿಗೆ ವ್ಯಾಜ್ಯವಾಡುವುದಿಲ್ಲ. ಇದೊಂದು ವೈಚಿತ್ರ್ಯ. ವೈಷ್ಣವರು ನರಕಕ್ಕೆ ಹೋಗುತ್ತಾರೆಂದು ಶೈವರು ಹೇಳುವುದಿಲ್ಲ. ಶೈವರು ನರಕಕ್ಕೆ ಹೋಗುತ್ತಾರೆಂದು ವೈಷ್ಣವರು ಹೇಳುವುದಿಲ್ಲ. ಶೈವರು “ಇದು ನಮ್ಮ ದಾರಿ, ನಿಮ್ಮದು ಬೇರೆ ದಾರಿ ಇದೆ. ಕೊನೆಗೆ ನಾವಿಬ್ಬರೂ ಸಂಧಿಸುವೆವು” ಎನ್ನುವರು. ಭರತಖಂಡದಲ್ಲಿ ಎಲ್ಲರಿಗೂ ಇದು ಗೊತ್ತಿದೆ. ಇದೇ ಇಷ್ಟದೇವತಾ ಸಿದ್ಧಾಂತ. ಅತಿ ಪುರಾತನ ಕಾಲದಿಂದಲೇ ಭಗವಂತನ ಉಪಾಸನೆಗೆ ಹಲವು ಆಕಾರಗಳಿವೆ ಎಂಬುದನ್ನು ಒಪ್ಪಿಕೊಂಡಿರುವರು. ಒಬ್ಬೊಬ್ಬರ ಪ್ರಕೃತಿಗೆ ಒಂದೊಂದು ಮಾರ್ಗವಿರಬೇಕೆಂಬುದನ್ನೂ ಒಪ್ಪಿಕೊಂಡಿರುವರು. ನಿಮ್ಮ ದಾರಿ ನಮ್ಮದಲ್ಲ; ಬಹುಶಃ ನನಗೆ ಅದು ಅಪಾಯಕರವಿರಬಹುದು. ಎಲ್ಲರಿಗೂ ಒಂದೇ ಮಾರ್ಗವಿದೆ ಎಂದು ನಂಬುವುದು ಅಪಾಯಕಾರಿ, ಅರ್ಥವಿಲ್ಲದ್ದು. ಆ ಭಾವನೆಯಿಂದ ನಾವು ಪಾರಾಗಬೇಕು. ಪ್ರತಿಯೊಬ್ಬರೂ ಒಂದೇ ಧರ್ಮಕ್ಕೆ ಸೇರಿ, ಒಂದೇ ಮಾರ್ಗವನ್ನು ಅನುಸರಿಸುವ ದುರ್ದಿನ ಪ್ರಪಂಚಕ್ಕೆ ಬರದಿರಲಿ. ಆಗ ಧರ್ಮ ಮತ್ತು ಆಧ್ಯಾತ್ಮಿಕ ಭಾವನೆಗಳು ನಾಶವಾಗುವುವು. ವೈವಿಧ್ಯವೆ ಜೀವನದ ರಹಸ್ಯ. ಇದು ಪೂರ್ಣ ನಾಶವಾದರೆ ಸೃಷ್ಟಿಯೇ ನಾಶವಾಗುವುದು. ನಮ್ಮ ಭಾವನೆಯಲ್ಲಿ ವೈವಿಧ್ಯ ಇದ್ದರೆ ನಾವು ಜೀವಿಸಿರುವೆವು. ವೈವಿಧ್ಯ ಇದೆ ಎಂದು ನಾವು ಹೋರಾಡಬೇಕಾಗಿಲ್ಲ. ನಿನ್ನ ದಾರಿ ನಿನಗೆ ಸರಿ, ನನಗಲ್ಲ. ನನ್ನ ದಾರಿ ನನಗೆ ಸರಿ, ನಿನಗಲ್ಲ. ನನ್ನ ದಾರಿಗೆ ನನ್ನ “ಇಷ್ಟ” ವೆಂದು ಸಂಸ್ಕೃತದಲ್ಲಿ ಹೆಸರು. ಪ್ರಪಂಚದಲ್ಲಿನ ಯಾವ ಧರ್ಮದೊಂದಿಗೂ ನಾವು ಕಾದಾಡುವುದಿಲ್ಲ. ಇದನ್ನು ಮರೆಯದಿರಿ. ನಮ್ಮಲ್ಲಿ ಪ್ರತಿಯೊಬ್ಬರಿಗೂ ಬೇರೆ ಬೇರೆ ಇಷ್ಟದೇವತೆ ಇದೆ. ಜನರು ಹೊರಗಿನಿಂದ ಬಂದು “ಇದೊಂದೇ ಮಾರ್ಗ” ವೆಂದು ಬಲಾತ್ಕರಿಸಿದರೆ ನಾವು ಅವರನ್ನು ನೋಡಿ ನಗುವೆವು. ದೇವರೆಡೆಗೆ ಬೇರೆ ದಾರಿಯಲ್ಲಿ ಹೋಗಲಿಚ್ಛಿಸುವವರನ್ನು ನಾಶಮಾಡಬೇಕೆಂದು ಬಯಸುವವರು ಪ್ರೀತಿ ಎಂಬ ಮಾತನ್ನೆತ್ತುವುದು ಅಪಹಾಸ್ಯ. ಅಂಥ ಪ್ರೀತಿಗೆ ಅರ್ಥವಿಲ್ಲ. ತಮ್ಮ ಮಾರ್ಗವನ್ನು ಬಿಟ್ಟು ಬೇರೆ ಮಾರ್ಗದಲ್ಲಿ ಹೋಗುವವನನ್ನೇ ಸಹಿಸಲಾರದವನು ಪ್ರೀತಿಯನ್ನು ಹೇಗೆ ಬೋಧಿಸಬಲ್ಲ? ಅದು ಪ್ರೀತಿಯಾದರೆ ದ್ವೇಷವೆಂದರೇನು? ಯಾವ ಧರ್ಮದೊಂದಿಗೂ ನಮಗೆ ವೈಮನಸ್ಯವಿಲ್ಲ. ಕ್ರಿಸ್ತ, ಮಹಮ್ಮದ್​, ಬುದ್ಧ ಅಥವಾ ಇನ್ನೂ ಯಾವುದಾದರೂ ದೇವದೂತರನ್ನು ಅವರು ಆರಾಧಿಸಲಿ. ಚಿಂತೆಯಿಲ್ಲ. “ಒಳ್ಳೆಯದು, ಸಹೋದರನೇ ನಾನು ನಿನಗೆ ಸಹಾಯ ಮಾಡುವೆನು. ಆದರೆ ನನ್ನ ದಾರಿಯಲ್ಲಿ ನಾನು ನಡೆಯುವುದಕ್ಕೆ ಸ್ವಾತಂತ್ರ್ಯ ಕೊಡು. ಅದು ನನ್ನ ಇಷ್ಟ. ನಿನ್ನ ಮಾರ್ಗ ನಿನಗೆ ಶ್ರೇಷ್ಠವಿರಬಹುದು. ಆದರೆ ನನಗೆ ಅದು ಅಪಾಯಕಾರಿಯಾಗಿರಬಹುದು. ಯಾವ ಆಹಾರ ಒಳ್ಳೆಯದೆಂದು ನನ್ನ ಅನುಭವವೇ ನನಗೆ ಹೇಳುವುದು. ವೈದ್ಯರ ಒಂದು ಸೇನೆ ನನಗೆ ಅದನ್ನು ಹೇಳಬೇಕಾಗಿಲ್ಲ. ನನ್ನ ಅನುಭವದಿಂದ ಯಾವ ಮಾರ್ಗ ನನಗೆ ಹಿತಕಾರಿ ಎಂಬುದು ನನಗೆ ಗೊತ್ತಿದೆ” ಎನ್ನುವನು ಹಿಂದೂ. ಅದೇ ನಮ್ಮ ಗುರಿ, ಇಷ್ಟ. ದೇವಸ್ಥಾನ, ಪ್ರತೀಕ, ವಿಗ್ರಹ, ಇವು ನಿಮ್ಮ ಆತ್ಮಸಾಕ್ಷಾತ್ಕಾರಕ್ಕೆ ಸಹಾಯ ಮಾಡಿದರೆ ಅವನ್ನು ಸ್ವೀಕರಿಸಿ. ಇನ್ನೂರು ವಿಗ್ರಹಗಳನ್ನು ಬೇಕಾದರೆ ಇಟ್ಟುಕೊಳ್ಳಿ. ಕೆಲವು ಆಕಾರಗಳು ಮತ್ತು ಅನುಷ್ಠಾನಗಳು ಭಗವತ್​ ಸಾಕ್ಷಾತ್ಕಾರಕ್ಕೆ ಸಹಾಯ ಮಾಡಿದರೆ ಅಗತ್ಯವಾಗಿ ಅವನ್ನು ಸ್ವೀಕರಿಸಿ. ಯಾವ ಯಾವ ವಿಗ್ರಹ ಮಂತ್ರ–ಆಚಾರಗಳು ನಿಮ್ಮನ್ನು ದೇವರ ಸಮೀಪಕ್ಕೆ ಒಯ್ಯುವುವೋ ಅವನ್ನೆಲ್ಲ ಸ್ವೀಕರಿಸಿ, ಆದರೆ ಅವಕ್ಕಾಗಿ ಹೋರಾಡಬೇಡಿ. ನೀವು ಕಾದಾಡುವುದಕ್ಕೆ ಪ್ರಾರಂಭಿಸಿದೊಡನೆ, ದೇವರಿಗೆ ವಿಮುಖರಾಗುವಿರಿ. ಮುಂದೆ ಹೋಗುವುದಿಲ್ಲ, ಹಿಮ್ಮುಖವಾಗಿ ಪ್ರಾಣಿಗಳ ಕಡೆಗೆ ಹಿಂದೆ ಹೋಗುವಿರಿ.

ಇವು ನಮ್ಮ ಧರ್ಮದ ಕೆಲವು ಭಾವನೆಗಳು. ಎಲ್ಲವನ್ನೂ ನಾವು ಸ್ವೀಕರಿಸುತ್ತೇವೆ. ಯಾವುದನ್ನೂ ನಿರಾಕರಿಸುವುದಿಲ್ಲ. ನಮ್ಮ ಜಾತಿಪದ್ಧತಿ ಮತ್ತು ಇನ್ನೂ ಹಲವು ಆಚಾರಗಳು ಧರ್ಮದೊಂದಿಗೆ ಬೆರೆತು ಹೋಗಿರುವಂತೆ ಕಂಡರೂ ಅವು ನಿಜವಾಗಿಯೂ ಹಾಗಿಲ್ಲ. ನಮ್ಮ ಜನಾಂಗದ ರಕ್ಷಣೆಗೆ ಇವು ಆವಶ್ಯಕ, ಅಷ್ಟೆ. ರಾಷ್ಟ್ರ ರಕ್ಷಣೆಗೆ ಯಾವಾಗ ಅವು ಆವಶ್ಯಕವಾಗುವುದಿಲ್ಲವೋ ಆಗ ಅವೂ ಸ್ವಾಭಾವಿಕವಾಗಿ ಮಾಯವಾಗುವುವು. ಆದರೆ ನನಗೆ ವಯಸ್ಸಾದಂತೆಲ್ಲಾ ಪುರಾತನಕಾಲದಿಂದಲೂ ಬಂದ ಈ ಸಂಸ್ಥೆಗಳನ್ನು ಗೌರವದಿಂದ ಕಾಣುತ್ತಿದ್ದೇನೆ. ಮುಕ್ಕಾಲುಪಾಲು ಇವೆಲ್ಲಾ ಕೆಲಸಕ್ಕೆ ಬಾರದವು ಎಂದು ಆಲೋಚಿಸುತ್ತಿದ್ದ ಕಾಲ ಒಂದಿತ್ತು. ಆದರೆ ವಯಸ್ಸಾದಂತೆಲ್ಲಾ ಅವುಗಳಲ್ಲಿ ಯಾವುದನ್ನೂ ತುಚ್ಛವಾಗಿ ಕಾಣಲಾರೆ. ಪ್ರತಿಯೊಂದರಲ್ಲಿಯೂ ಹಲವು ಶತಮಾನಗಳ ಪ್ರಯೋಗದ ಫಲವಿದೆ. ನಿನ್ನೆ ಹುಟ್ಟಿದ ಹಾಗೂ ನಾಳೆ ಕಾಲವಾಗಬಹುದಾದ ಮಗುವೊಂದು ನನ್ನ ಬಳಿಗೆ ಬಂದು ನನ್ನ ಯೋಜನೆಗಳನ್ನೆಲ್ಲಾ ಬದಲಾಯಿಸು ಎನ್ನುವುದು. ಅದರ ಮಾತಿನಂತೆ ನಾನು ನನ್ನ ಯೋಜನೆಗಳನ್ನೆಲ್ಲ ಬದಲಾಯಿಸಿದರೆ ನಾನೇ ಮೂರ್ಖ, ಬೇರಾರೂ ಅಲ್ಲ. ಹಲವು ದೇಶಗಳಿಂದ ಬರುತ್ತಿರುವ ಬುದ್ಧಿವಾದ ಈ ಬಗೆಯದು. ‘ನೀವೇ ಒಂದು ಸದೃಢ ಸಮಾಜವನ್ನು ತಯಾರುಮಾಡಿದಾಗ ನಿಮ್ಮ ಮಾತನ್ನು ಕೇಳುತ್ತೇನೆ’ ಎಂದು ಈ ಹರಟೆಮಲ್ಲರಿಗೆ ಹೇಳಿ. ‘ಒಂದು ಭಾವನೆಯನ್ನು ಎರಡು ದಿನ ಇಟ್ಟುಕೊಂಡಿರಲಾರಿರಿ, ಆಗಲೇ ಜಗಳ ಕಾಯುವಿರಿ, ನಿರಾಶರಾಗುವಿರಿ. ನೀವು ವಸಂತ ಋತುವಿನಲ್ಲಿ ಹುಟ್ಟುವ ಚಿಟ್ಟೆಗಳಂತೆ; ಅವುಗಳಂತೆಯೇ ಐದು ನಿಮಿಷದಲ್ಲಿ ಕಾಲವಾಗುವಿರಿ. ಗುಳ್ಳೆಯಂತೆ ಮೇಲೇಳುವಿರಿ; ಅವುಗಳಂತೆಯೇ ಒಡೆದು ಹೋಗುವಿರಿ. ನಮ್ಮ ಸಮಾಜದಂತಹ ಸುಭದ್ರ ಸಮಾಜವನ್ನು ಮೊದಲು ತಯಾರುಮಾಡಿ; ಹಲವು ಶತಮಾನಗಳಾದರೂ ಅಚ್ಚಳಿಯದೆ ನಿಲ್ಲುವಂತಹ ಸಮಾಜವನ್ನೂ, ನಿಯಮಾವಳಿಯನ್ನೂ ರಚಿಸಿ. ಆಗಬೇಕಾದರೆ ನಿಮ್ಮೊಂದಿಗೆ ಈ ವಿಷಯದ ಬಗ್ಗೆ ಮಾತಾಡಬಹುದು. ಅಲ್ಲಿಯವರೆಗೂ ನೀವೆಲ್ಲ ಅಮಲೇರಿದ ಮಕ್ಕಳು’ ಎಂದು ಅವರಿಗೆ ಹೇಳಿ.

ಧಾರ್ಮಿಕ ವಿಷಯವಾಗಿ ನಾನು ಮಾತಾಡಬೇಕೆಂದುದು ಆಯಿತು. ಈಗಿನ ಕಾಲದ ಒಂದು ಆವಶ್ಯಕ ವಿಷಯವನ್ನು ನಿಮಗೆ ಜ್ಞಾಪಿಸಿ ಪೂರೈಸುವೆನು. ಮಹಾ ಭಾರತದ ಕರ್ತೃ\break ವಾದ ವ್ಯಾಸನಿಗೆ ಧನ್ಯವಾದ. ಅವನು “ಕಲಿಯುಗದಲ್ಲಿ ದಾನವೊಂದೇ ಧರ್ಮ” ಎಂದು ಸಾರಿರುವನು. ತಪಸ್ಸು ಕಠಿಣ; ಯೋಗ ಈ ಯುಗದಲ್ಲಿ ಸಾಧ್ಯವಿಲ್ಲ. ಇತರರಿಗೆ ಸಹಾಯಮಾಡುವ ಧರ್ಮ ಇಂದು ಬೇಕಾಗಿದೆ. ದಾನವೆಂದರೇನು? ಎಲ್ಲಾ ದಾನಗಳಿಗಿಂತ ಶ್ರೇಷ್ಠವಾದುದು ಜ್ಞಾನದಾನ. ಅದರ ಅನಂತರ ವಿದ್ಯಾದಾನ; ಅನಂತರ ಪ್ರಾಣದಾನ; ವಸ್ತ್ರ–ಅನ್ನ–ಪಾನಾದಿಗಳು ಎಲ್ಲವುಗಳಿಗಿಂತ ಕನಿಷ್ಠ ದಾನಗಳು. ಯಾರು ಅಧ್ಯಾತ್ಮ ವಿದ್ಯೆಯನ್ನು ದಾನ ಮಾಡುವರೋ ಅವರು ಜೀವಿಗಳನ್ನು ಹಲವು ಜನ್ಮಗಳಿಂದ ಪಾರುಮಾಡುವರು. ಯಾರು ಲೌಕಿಕ ವಿದ್ಯೆಯನ್ನು ದಾನ ಮಾಡುವರೋ ಅವರು ಅಧ್ಯಾತ್ಮಜ್ಞಾನದ ಕಡೆ ಜನರ ಗಮನವನ್ನು ಸೆಳೆಯುತ್ತಾರೆ. ಉಳಿದ ದಾನಗಳೆಲ್ಲಾ ಇದಕ್ಕಿಂತ ಕನಿಷ್ಠ. ಪ್ರಾಣದಾನ ಕೂಡ ಇದಕ್ಕಿಂತ ಕೆಳಗಿನದು. ಅದಕ್ಕಾಗಿಯೇ ಇದನ್ನು ತಿಳಿದುಕೊಳ್ಳುವುದು ಆವಶ್ಯಕ. ಅಧ್ಯಾತ್ಮದಾನಕ್ಕಿಂತ ಉಳಿದವೆಲ್ಲಾ ಕನಿಷ್ಠ. ಶ್ರೇಷ್ಠವಾದ ಅಮೋಘವಾದ ದಾನವೇ ಆಧ್ಯಾತ್ಮಿಕ ವಿದ್ಯೆಯನ್ನು ಜನರಿಗೆ ಹಂಚುವುದು. ನಮ್ಮ ಶಾಸ್ತ್ರದಲ್ಲಿ ಎಂದಿಗೂ ಬತ್ತದ ಆಧ್ಯಾತ್ಮಿಕ ಚಿಲುಮೆ\break ಇದೆ. ಈ ಪುಣ್ಯಭೂಮಿಯನ್ನು ಬಿಟ್ಟರೆ ಮತ್ತೆಲ್ಲಿಯೂ ಅಧ್ಯಾತ್ಮವನ್ನು ಅಷ್ಟರಮಟ್ಟಿಗೆ\break ಅನುಷ್ಠಾನದಲ್ಲಿ ಇಟ್ಟಿರುವಂತಹ ಉದಾಹರಣೆ ದೊರಕುವುದಿಲ್ಲ. ನನಗೆ ಸ್ವಲ್ಪಮಟ್ಟಿಗೆ ಜಗತ್ತಿನ ಅನುಭವ ಇದೆ. ಇತರ ದೇಶಗಳಲ್ಲಿ ಬೇಕಾದಷ್ಟು ಮಾತನಾಡುವರು. ಅವನ್ನೆಲ್ಲ ಜೀವನದಲ್ಲಿ ಅನುಷ್ಠಾನಕ್ಕೆ ತಂದವರು ಇಲ್ಲಿ ಮಾತ್ರ ಇರುವರು. ಮಾತನಾಡುವುದಲ್ಲ ಧರ್ಮ. ಅರಗಿಳಿಗಳೂ ಮಾತನಾಡುವುವು. ಈಗಿನ ಕಾಲದಲ್ಲಿ ಯಂತ್ರವೂ ಮಾತನಾಡುವುದು. ಆದರೆ ತ್ಯಾಗ ಜೀವನವನ್ನು ತೋರಿ, ಆಧ್ಯಾತ್ಮಿಕ ಜೀವನವನ್ನು ತೋರಿ, ಸಹಿಷ್ಣುತಾ ಜೀವನವನ್ನು ತೋರಿ, ಅಪಾರ ಪ್ರೇಮ ಜೀವನವನ್ನು ತೋರಿ–ಇದೇ ಆಧ್ಯಾತ್ಮಿಕ ಪುರುಷನ ಲಕ್ಷಣ. ಇಂತಹ ಭಾವನೆಗಳು, ಇಂತಹ ಉದಾತ್ತವಾದ ನಿದರ್ಶನಗಳು ನಮ್ಮ ದೇಶದಲ್ಲಿರುವಾಗ, ಯೋಗಿಗಳ ಬುದ್ಧಿ ಹೃದಯಗಳಲ್ಲಿ ಇರುವ ಇಂತಹ ಅನರ್ಘ್ಯ ರತ್ನಗಳು ಶ‍್ರೀಮಂತರ, ಬಡವರ, ಪಂಡಿತಪಾಮರರ, ಉಚ್ಚನೀಚರ ಸರ್ವಸಾಮಾನ್ಯ ಆಸ್ತಿಯಾಗದೆ ಇರುವುದು ಅತ್ಯಂತ ಶೋಚನೀಯ. ಭರತಖಂಡಕ್ಕೆ ಮಾತ್ರವಲ್ಲ, ಪ್ರಪಂಚದ ಮೂಲೆಮೂಲೆಗಳಲ್ಲೂ ಇದನ್ನು ಸಾರಬೇಕು. ಇದು ನಮ್ಮ ಪರಮ ಕರ್ತವ್ಯಗಳಲ್ಲಿ ಒಂದು. ನೀವು ಮತ್ತೊಬ್ಬರಿಗೆ ಹೆಚ್ಚು ಸಹಾಯಮಾಡಿದಷ್ಟೂ ಅದರಿಂದ ನಿಮಗೇ ಉಪಯೋಗ. ನೀವು ನಿಜವಾಗಿಯೂ ನಿಮ್ಮ ಧರ್ಮವನ್ನು ಪ್ರೀತಿಸುವುದಾದರೆ, ದೇಶವನ್ನು ಪ್ರೀತಿಸುವುದಾದರೆ, ನಿಮ್ಮ ಪವಿತ್ರ ಗ್ರಂಥಗಳಲ್ಲಿ ಹುದುಗಿರುವ ರತ್ನಗಳನ್ನು ತಂದು ಅವನ್ನು ಯೋಗ್ಯ ಜನರಿಗೆ ಹರಡುವುದಕ್ಕೆ ದುಡಿಯುವುದು ನಿಮ್ಮ ಆದ್ಯ ಕರ್ತವ್ಯ.

ಎಲ್ಲಕ್ಕಿಂತ ಹೆಚ್ಚಾಗಿ ಒಂದು ವಿಷಯ ಅತ್ಯಂತ ಆವಶ್ಯಕ. ಹಲವು ಶತಮಾನಗಳಿಂದ ನಾವು ಭಯಾನಕ ಅಸೂಯಾ ಭಾವನೆಯಲ್ಲಿ ಸಿಕ್ಕಿ ನರಳುತ್ತಿರುವೆವು. ಇತರರನ್ನು ಕಂಡರೆ ನಮಗೆ ಯಾವಾಗಲೂ ಅಸೂಯೆ; ಇವನಿಗೇಕೆ ಅಗ್ರಸ್ಥಾನ ನನಗೇಕೆ ಇಲ್ಲ? ದೇವರ ಪೂಜೆಯಲ್ಲೂ ನಮಗೆ ಅಗ್ರಸ್ಥಾನ ಬೇಕು. ಇಂತಹ ಗುಲಾಮಗಿರಿಗೆ ನಾವು ಬಂದಿರುವೆವು. ನಾವು ಇದರಿಂದ ಪಾರಾಗಬೇಕಾಗಿದೆ. ಭರತಖಂಡದಲ್ಲಿ ಇಂದು ಯಾವುದಾದರೂ ಮಹಾಪಾತಕವಿದ್ದರೆ ಅದೇ ಈ ಗುಲಾಮಗಿರಿ. ಪ್ರತಿಯೊಬ್ಬರೂ ಆಜ್ಞಾಪಿಸುವವರೆ. ಆಜ್ಞಾಧಾರಕರಾಗಲು ಯಾರೂ ಇಚ್ಛಿಸುವುದಿಲ್ಲ. ಹಿಂದಿನಕಾಲದ ಬ್ರಹ್ಮಚರ್ಯಾಶ್ರಮ ಈಗ ಇಲ್ಲದಿರುವುದೇ ಇದಕ್ಕೆ ಕಾರಣ. ಮೊದಲು ಅಪ್ಪಣೆ ಪಾಲಿಸುವುದನ್ನು ಕಲಿಯಿರಿ. ಆಜ್ಞೆ ಕೊಡುವುದು ಅನಂತರ ಬರುವುದು. ಮೊದಲು ಹೇಗೆ ಸೇವಕನಾಗಬೇಕೆಂಬುದನ್ನು ಕಲಿಯಿರಿ. ಅನಂತರ ಸ್ವಾಮಿಯಾಗಲೂ ಯೋಗ್ಯರಾಗುವಿರಿ. ಅಸೂಯೆಯನ್ನು ತೊರೆಯಿರಿ. ಆಗ ಇನ್ನೂ ಆಗಬೇಕಾದ ಮಹತ್ಕಾರ್ಯಗಳನ್ನು ಸಾಧಿಸುವಿರಿ. ನಮ್ಮ ಪೂರ್ವಿಕರು ಅದ್ಭುತ ಕಾರ್ಯಗಳನ್ನು ಸಾಧಿಸಿದರು. ಅವುಗಳನ್ನು ನಾವು ಗೌರವಪೂರ್ವಕವಾದ ಹೆಮ್ಮೆಯಿಂದ ನೋಡುತ್ತೇವೆ. ನಾವೂ ಮಹತ್ಕಾರ್ಯಗಳನ್ನು ಸಾಧಿಸುತ್ತೇವೆ. ನಮ್ಮ ಅನಂತರ ಬರುವವರು ಹೆಮ್ಮೆಯಿಂದ ಮತ್ತು ಆಶೀರ್ವಾದದ ದೃಷ್ಟಿಯಿಂದ ನಮ್ಮನ್ನು ನೋಡಲಿ. ನಮ್ಮ ಪೂರ್ವಿಕರು ಸಾಧಿಸಿದ ಮಹತ್ಕಾರ್ಯಗಳೂ ಅಲ್ಪ ಎಂದು ಎನ್ನಿಸುವಂತಹ ಮಹತ್ಕಾರ್ಯಗಳನ್ನು ಇಲ್ಲಿರುವ ಪ್ರತಿಯೊಬ್ಬರೂ ಭಗವಂತನ ದಯೆಯಿಂದ ಸಾಧಿಸುವರು.

