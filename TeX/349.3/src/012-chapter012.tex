
\chapter{ಭಾರತೀಯರ ಜೀವನದಲ್ಲಿ ವೇದಾಂತದ ಅನುಷ್ಠಾನ}

\begin{center}
(ಮದ್ರಾಸಿನಲ್ಲಿ ಮಾಡಿದ ಉಪನ್ಯಾಸ)
\end{center}

ನಮ್ಮ ಜನಾಂಗವನ್ನು ಮತ್ತು ಧರ್ಮವನ್ನು ಪ್ರತಿನಿಧಿಸುವ ಒಂದು ಪದ ಬಹಳ ಪ್ರಚಲಿತವಾಗಿದೆ. ವೇದಾಂತವೆಂದರೆ ಏನೆಂದು ತಿಳಿಯಬೇಕಾದರೆ ‘ಹಿಂದೂ’ ಎಂಬ ಪದವನ್ನು ಸ್ವಲ್ಪ ವಿವರಿಸಬೇಕಾಗಿದೆ. ಪುರಾತನ ಪರ್ಶಿಯನರು ಸಿಂಧೂ ನದಿಯನ್ನು ಹಿಂದೂ ಎಂದು ಕರೆಯುತ್ತಿದ್ದರು. ಸಂಸ್ಕೃತದ ‘ಸ’ –ಕಾರವನ್ನು ಪರ್ಶಿಯನರು ‘ಹ್​’ ಎಂದು ಉಚ್ಚರಿಸುತ್ತಿದ್ದರು. ಹೀಗಾಗಿ ‘ಸಿಂಧೂ’ ಪದವು ‘ಹಿಂದೂ’ ಎಂದಾಯಿತು. ಗ್ರೀಕರಿಗೆ ಹಕಾರವನ್ನು ಉಚ್ಚರಿಸಲು ಕಷ್ಟವಾಗಿ ಅವರು ಅದನ್ನು ತ್ಯಜಿಸಿಯೇಬಿಟ್ಟರು ಎಂಬುದು ನಿಮಗೆಲ್ಲ ಗೊತ್ತಿದೆ. ಹೀಗೆ ನಾವು ಇಂಡಿಯನರಾದೆವು. ಹಿಂದೂಗಳು ಎಂದರೆ ಸಿಂಧೂ ನದಿಯ ಆಚೆಯ ಭಾಗದಲ್ಲಿ ವಾಸಿಸುವವರು ಎಂಬ ಅರ್ಥ ಹಿಂದೆ ಇತ್ತು. ಆದರೆ ಈಗ ಅದು ಆ ಅರ್ಥವನ್ನುಕಳೆದುಕೊಂಡಿದೆ. ಏಕೆಂದರೆ ಈಗ ಸಿಂಧೂ ನದಿಯ ಈಚೆಗೆ ಇರುವ ಜನರೆಲ್ಲ ಒಂದೇ ಧರ್ಮಕ್ಕೆ ಸೇರಿದವರಲ್ಲ. ಇಲ್ಲಿ ಹಿಂದೂಗಳಿರುವರು, ಜೊತೆಗೆ ಮಹಮ್ಮದೀಯರು, ಕ್ರೈಸ್ತರು, ಪಾರ್ಸಿಗಳು, ಬೌದ್ಧರು, ಜೈನರು ಎಲ್ಲರೂ ಇರುವರು. ಮೂಲ ಶಬ್ದಾರ್ಥದ ಪ್ರಕಾರ ಹಿಂದೂ ಎಂಬುದು ಇವರೆಲ್ಲರಿಗೂ ಅನ್ವಯಿಸಬೇಕು. ಆದರೆ ಧಾರ್ಮಿಕ ದೃಷ್ಟಿಯಿಂದ ಇವರೆಲ್ಲರನ್ನು ಹಿಂದೂ ಎಂದು ಕರೆಯಲಾಗುವುದಿಲ್ಲ. ನಮ್ಮ ಧರ್ಮಕ್ಕೆ ಒಂದು ಸರ್ವಸಾಮಾನ್ಯವಾದ ಹೆಸರನ್ನು ಕಂಡುಹಿಡಿಯುವುದು ಕಷ್ಟವಾಗಿದೆ. ಏಕೆಂದರೆ ಹಲವು ಬಗೆಯ ಧರ್ಮಗಳು, ಭಾವನೆಗಳು, ಆಚಾರ ವ್ಯವಹಾರಗಳು– ಇವುಗಳನ್ನೆಲ್ಲ ಯಾವ ಪ್ರತ್ಯೇಕ ಹೆಸರು, ಪಂಥ, ಸಂಸ್ಥೆ ಇಲ್ಲದೆ ಒಟ್ಟುಗೂಡಿಸಲಾಗಿದೆ. ನಮ್ಮ ಎಲ್ಲ ಮತಗಳಿಗೂ ಸಾಮಾನ್ಯವಾದ ಒಂದು ಅಂಶವೆಂದರೆ ಬಹುಶಃ ಎಲ್ಲ ಮತಗಳ ಅನುಯಾಯಿಗಳಿಗೂ ವೇದಗಳಲ್ಲಿರುವ ನಂಬಿಕೆ. ವೇದಗಳ ಪರಮ ಪ್ರಾಮಾಣ್ಯವನ್ನು ಒಪ್ಪದವರಾರಿಗೂ ಹಿಂದೂಗಳೆಂದು ಕರೆಸಿಕೊಳ್ಳುವ ಅಧಿಕಾರವಿಲ್ಲವೆಂದು ತೋರುತ್ತದೆ.

ಈ ವೇದರಾಶಿ ನಿಮಗೆ ತಿಳಿದಿರುವಂತೆ ಜ್ಞಾನಕಾಂಡ, ಕರ್ಮಕಾಂಡವೆಂದು ಎರಡು ಭಾಗವಾಗಿದೆ. ಕರ್ಮಕಾಂಡದಲ್ಲಿ ಹಲವು ಯಾಗಯಜ್ಞಗಳ ವಿಷಯವಿದೆ. ಅವುಗಳಲ್ಲಿ ಬಹುಭಾಗ ಈಗ ಬಳಕೆಯಲ್ಲಿ ಇಲ್ಲ. ವೇದಗಳ ಆಧ್ಯಾತ್ಮಿಕ ಉಪದೇಶಗಳನ್ನು ಒಳಗೊಂಡಿರುವುದೇ ಜ್ಞಾನಕಾಂಡ. ಇದನ್ನೇ ಉಪನಿಷತ್ತು ಅಥವಾ ವೇದಾಂತ ಎಂದು ಕರೆಯುವರು. ನಮ್ಮ ಆಚಾರ್ಯರು, ತತ್ತ್ವಜ್ಞರು, ಗ್ರಂಥಕರ್ತರು – ಅವರು ದ್ವೈತಿಗಳಾಗಲಿ, ವಿಶಿಷ್ಟಾದ್ವೈತಿಗಳಾಗಲಿ ಅಥವಾ ಅದ್ವೈತಿಗಳಾಗಲಿ – ಎಲ್ಲರೂ ಇದನ್ನೇ ಪರಮಪ್ರಮಾಣವೆಂದು ಉದಾಹರಿಸುತ್ತಾರೆ. ಅವನ ತತ್ತ್ವ ಯಾವುದೇ ಆಗಿರಲಿ, ಅವನು ಯಾವುದೇ ಪಂಥಕ್ಕೆ ಸೇರಿರಲಿ, ಭಾರತದಲ್ಲಿ ಪ್ರತಿಯೊಬ್ಬನಿಗೂ ಉಪನಿಷತ್ತುಗಳೇ ಪ್ರಮಾಣ. ಅವನು ಅವನ್ನು ಒಪ್ಪದಿದ್ದರೆ ಅವನ ಮತ ನಾಸ್ತಿಕವೆನಿಸುವುದು. ಆದ್ದರಿಂದ ಆಧುನಿಕ ಕಾಲದಲ್ಲಿ ಪ್ರತಿಯೊಬ್ಬ ಹಿಂದೂವಿಗೂ ಅನ್ವಯಿಸಬಹುದಾದ ಏಕಮಾತ್ರ ಪದವೇ ‘ವೇದಾಂತಿ’ ಅಥವಾ ‘ವೈದಿಕ’ ಎನ್ನುವುದು. ನಾನು ಯಾವಾಗಲೂ ಈ ಅರ್ಥದಲ್ಲಿಯೇ ‘ವೇದಾಂತ ತತ್ತ್ವ’ ಮತ್ತು ‘ವೇದಾಂತ’ ಎಂಬ ಪದಗಳನ್ನು ಉಪಯೋಗಿಸುವುದು. ನಾನು ಇದನ್ನು ಹೆಚ್ಚು ವಿಶದಪಡಿಸಲು ಇಚ್ಛಿಸುತ್ತೇನೆ. ಏಕೆಂದರೆ ಈಚೆಗೆ ವೇದಾಂತವೆಂದರೆ ಕೇವಲ ಅದ್ವೈತವೆಂದು ತಿಳಿಯುವುದು ರೂಢಿಯಾಗಿಹೋಗಿದೆ. ಉಪನಿಷತ್ತುಗಳ ಆಧಾರದ ಮೇಲೆ ನಿಂತ ಹಲವು ತತ್ತ್ವಗಳಲ್ಲಿ ಅದ್ವೈತ ತತ್ತ್ವವೂ ಒಂದು ಮಾತ್ರ ಎಂದು ನಮಗೆ ಗೊತ್ತಿದೆ. ಉಪನಿಷತ್ತುಗಳ ಮೇಲೆ ಅದ್ವೈತಿಗಳಿಗೆ ಇರುವಷ್ಟೇ ಪೂಜ್ಯಭಾವ ವಿಶಿಷ್ಟಾದ್ವೈತಿಗಳಿಗೂ ಇದೆ. ಅದ್ವೈತಿಗಳೂ ವೇದಾಂತದ ಪ್ರಮಾಣವನ್ನು ಒಪ್ಪಿಕೊಳ್ಳುವಷ್ಟೇ ವಿಶಿಷ್ಟಾದ್ವೈತಿಗಳೂ ಒಪ್ಪಿಕೊಳ್ಳುತ್ತಾರೆ. ದ್ವೈತಿಗಳು ಕೂಡ ಹೀಗೆಯೇ. ಭಾರತದಲ್ಲಿ ಪ್ರತಿಯೊಂದು ಪಂಗಡದವರೂ ಕೂಡ ಹೀಗೆಯೆ. ಆದರೂ ಜನಸಾಧಾರಣರಲ್ಲಿ ವೇದಾಂತವೆಂದರೆ ಅದ್ವೈತವೆಂಬ ಭಾವನೆ ಬಂದುಹೋಗಿದೆ. ಇದಕ್ಕೆ ಕೊಂಚ ಕಾರಣವೂ ಇದೆ. ವೇದಗಳು ನಮ್ಮ ಪ್ರಮಾಣ ಗ್ರಂಥಗಳಾಗಿದ್ದರೂ ಅವುಗಳ ತತ್ತ್ವಗಳನ್ನು ವಿವರಿಸುವುದಕ್ಕೆ ಮುಂದೆ ಬರೆಯಲ್ಪಟ್ಟ ಸ್ಮೃತಿಗಳೂ ಪುರಾಣಗಳೂ ನಮ್ಮಲ್ಲಿವೆ. ಆದರೆ ವೇದಗಳಿಗಿರುವ ಪ್ರಾಮಾಣ್ಯ ಇವುಗಳಿಗಿಲ್ಲ. ಈ ಪುರಾಣ ಸ್ಮೃತಿಗಳಿಗೂ ಶ್ರುತಿಯ ಭಾಗಕ್ಕೂ ಎಲ್ಲಿಯಾದರೂ ಅಭಿಪ್ರಾಯ ಭೇದವಿದ್ದರೆ ಶ್ರುತಿಯನ್ನೇ ಅನುಸರಿಸಿ ಸ್ಮೃತಿಯನ್ನು ಕೈಬಿಡಬೇಕೆಂದು ನಿಯಮ. ಶ್ರೇಷ್ಠ ಅದ್ವೈತಾಚಾರ್ಯರಾದ ಶಂಕರರು ತಮ್ಮ ಸಿದ್ಧಾಂತವನ್ನು ಪ್ರತಿಪಾದಿಸುವಾಗ ಪ್ರಮಾಣಕ್ಕಾಗಿ ಹೆಚ್ಚಾಗಿ ಉಪನಿಷತ್ತು\-ಗಳನ್ನೇ ಉದಾಹರಿಸುತ್ತಾರೆ. ಬಹಳ ಅಪರೂಪಕ್ಕೆ ಯಾವುದಾದರೂ ಒಂದು ವಿಷಯವನ್ನು ವಿವರಿಸುವಾಗ ಶ್ರುತಿಯಲ್ಲಿ ಪ್ರಮಾಣವು ದೊರೆಯದೆ ಇದ್ದರೆ ಆಗ ಮಾತ್ರ ಸ್ಮೃತಿಯನ್ನು ಉದಾಹರಿಸುತ್ತಾರೆ. ಆದರೆ ಇತರ ಸಂಪ್ರದಾಯವದವರು ಶ್ರುತಿಗಿಂತ ಹೆಚ್ಚಾಗಿ ಸ್ಮೃತಿಯನ್ನೇ ಉದಾಹರಿಸುತ್ತಾರೆ. ದ್ವೈತಮತಕ್ಕೆ ಹೆಚ್ಚು ಹೆಚ್ಚು ಸಮೀಪವಾದ ಸಿದ್ಧಾಂತಗಳು ಸ್ಮೃತಿಯನ್ನು ಉದಾಹರಿಸುವುದು ಹೆಚ್ಚುತ್ತದೆ. ಇದು ವೇದಾಂತಿಗಳಿಂದ ನಾವು ನಿರೀಕ್ಷಿಸಬಹುದಾದದ್ದಲ್ಲ. ಹೀಗೆ ಇವರು ಪುರಾಣ ಪ್ರಮಾಣಕ್ಕೆ ಹೆಚ್ಚು ಪ್ರಾಧಾನ್ಯ ನೀಡಿದುದಕ್ಕಾಗಿಯೇ ಬಹುಶಃ ಅದ್ವೈತವನ್ನು ಪರಮಶ್ರೇಷ್ಠ ವೇದಾಂತವೆಂದು ಪರಿಗಣಿಸಿರಬಹುದು.

ಏನೇ ಆದರೂ ವೇದಾಂತವು ಭಾರತೀಯ ಧಾರ್ಮಿಕ ಜೀವನದ ಪೂರ್ಣ ಕ್ಷೇತ್ರವನ್ನು ಆವರಿಸಬೇಕು. ವೇದಾಂತವು ವೇದದ ಒಂದು ಭಾಗವಾದುದರಿಂದ ಅದು ಎಲ್ಲರೂ ಒಪ್ಪುವಂತಹ ನಮ್ಮಲ್ಲಿರುವ ಅತಿ ಪುರಾತನ ಗ್ರಂಥ. ಆಧುನಿಕ ವಿದ್ವಾಂಸರ ಅಭಿಪ್ರಾಯ ಏನೇ ಆಗಿದ್ದರೂ ವೇದಗಳ ಬೇರೆ ಬೇರೆ ಭಾಗಗಳು ಬೇರೆ ಬೇರೆ ಕಾಲದಲ್ಲಿ ಬರೆಯಲ್ಪಟ್ಟವೆಂಬುದನ್ನು ಹಿಂದೂಗಳು ಒಪ್ಪುವುದಿಲ್ಲ. ವೇದಗಳೆಲ್ಲ ಒಟ್ಟಿಗೆ ಒಂದೇ ಕಾಲದಲ್ಲಿ ಆದುವೆಂದು ಅವರು ನಂಬುತ್ತಾರೆ. ಆದರೆ ನಿಜವಾಗಿ ಹೇಳಬೇಕೆಂದರೆ ಅವರು ಈ ವೇದಗಳು ಹೊಸದಾಗಿ ಹುಟ್ಟಿದವು ಎನ್ನುವುದನ್ನೇ ನಂಬುವುದಿಲ್ಲ. ಅವು ಯಾವಾಗಲೂ ಭಗವಂತನ\break ಮನಸ್ಸಿನಲ್ಲಿದ್ದವು. ವೇದಾಂತವು ದ್ವೈತ, ವಿಶಿಷ್ಟಾದ್ವೈತ ಮತ್ತು ಅದ್ವೈತಗಳನ್ನು ಒಳಗೊಳ್ಳುತ್ತದೆ. ನನ್ನ ದೃಷ್ಟಿಯಲ್ಲಿ ವೇದಾಂತವೆಂದರೆ ಇದೇ. ಬೌದ್ಧರು ಮತ್ತು ಜೈನರು ಒಪ್ಪುವುದಾದರೆ ಅವರ ಧರ್ಮಗಳ ಅಂಶಗಳನ್ನು ನಾವು ಸೇರಿಸಿಕೊಳ್ಳುತ್ತೇವೆ. ನಮ್ಮ ಹೃದಯ ಅಷ್ಟು ವಿಶಾಲವಾಗಿದೆ. ಆದರೆ ಒಪ್ಪದೇ ಇರುವವರು ಅವರೇ. ನಾವು ಸಿದ್ಧವಾಗಿರುವೆವು. ತೀಕ್ಷ್ಣ ವಿಶ್ಲೇಷಣೆಗೆ ಒಳಪಡಿಸಿ ನೋಡಿದರೆ ಬೌದ್ಧಧರ್ಮವು ತನ್ನ ಸಾರವನ್ನೆಲ್ಲ ಉಪನಿಷತ್ತಿನಿಂದ ಪಡೆದಿದೆ ಎಂಬುದು ತಿಳಿದುಬರುತ್ತದೆ. ಬೌದ್ಧ ಧರ್ಮದ ಶ್ರೇಷ್ಠ ಮತ್ತು ಅದ್ಭುತ ನೈತಿಕ ಭಾವನೆಗಳು ಒಂದಲ್ಲಾ ಒಂದು ರೂಪದಲ್ಲಿ ಅಕ್ಷರಶಃ ಉಪನಿಷತ್ತುಗಳಲ್ಲಿವೆ. ಇದರಂತೆಯೇ ಜೈನಧರ್ಮದ ಉತ್ತಮ ಸಿದ್ಧಾಂತಗಳು ಕೂಡಾ ಉಪನಿಷತ್ತುಗಳಲ್ಲಿವೆ. ಆದರೆ ಜೈನರ ವಿಚಿತ್ರ ನಡವಳಿಕೆಗಳು ಮಾತ್ರ ಅಲ್ಲಿಲ್ಲ. ಮುಂದೆ ಬೆಳವಣಿಗೆ ಹೊಂದಿದ ಭಾರತೀಯ ಧಾರ್ಮಿಕ ಭಾವನೆಗಳೆಲ್ಲವೂ ಉಪನಿಷತ್ತುಗಳಲ್ಲಿ ಬೀಜರೂಪದಲ್ಲಿವೆ. ಕೆಲವು ವೇಳೆ ಸಾಕಷ್ಟು ಪ್ರಮಾಣವಿಲ್ಲದೆ ಭಕ್ತಿಯ ಆದರ್ಶವು ಉಪನಿಷತ್ತಿನಲ್ಲಿ ಇಲ್ಲವೆನ್ನುತ್ತಾರೆ. ಯಾರು ಉಪನಿಷತ್ತುಗಳ ಅಧ್ಯಯನ ಮಾಡಿರುವರೊ ಅವರಿಗೆ ಇದು ಸತ್ಯವಲ್ಲವೆಂದು ಗೊತ್ತಾಗುತ್ತದೆ. ನೀವು ಅರಸು\-ವುದಾದರೆ ಸಾಕಷ್ಟು ಭಕ್ತಿಯ ಭಾವನೆಗಳು ಪ್ರತಿಯೊಂದು ಉಪನಿಷತ್ತಿನಲ್ಲಿಯೂ ದೊರೆಯುತ್ತವೆ. ಅನಂತರ ಪುರಾಣಗಳಲ್ಲಿ ಮತ್ತು ಇತರ ಸ್ಮೃತಿಗಳಲ್ಲಿ ಪೂರ್ಣ ಬೆಳವಣಿಗೆಯನ್ನು ಹೊಂದಿದ ಈ ಭಕ್ತಿಯ ಭಾವನೆಗಳೆಲ್ಲ ಉಪನಿಷತ್ತಿನಲ್ಲಿ ಅಂಕುರಾವಸ್ಥೆಯಲ್ಲಿವೆ. ಭಕ್ತಿಯ ರೂಪರೇಖೆ ಆಗಲೇ ಉಪನಿಷತ್ತಿನಲ್ಲಿದೆ. ಅನಂತರದ ಪುರಾಣಗಳು ಅದಕ್ಕೆ ಪೂರ್ಣರೂಪವನ್ನು ನೀಡಿದವು, ಅಷ್ಟೇ. ಭರತಖಂಡದಲ್ಲಿ ಸ್ಪಷ್ಟವಾಗಿ ಅನಂತರ ಪ್ರಕಾಶಕ್ಕೆ ಬಂದ ಯಾವ ಒಂದು ಆದರ್ಶವೂ ಉಪನಿಷತ್ತಿನಲ್ಲಿ ಇಲ್ಲದೇ ಇಲ್ಲ. ಸಾಕಷ್ಟು ಉಪನಿಷತ್ತಿನ ಜ್ಞಾನವಿಲ್ಲದ ಕೆಲವರು ಭಕ್ತಿಯು ಹೊರದೇಶದಿಂದ ಬಂತೆಂದು ಪ್ರಮಾಣೀಕರಿಸಲು ವಿಚಿತ್ರವಾಗಿ ಪ್ರಯತ್ನಿಸುತ್ತಾರೆ. ಆದರೆ ನಿಮಗೆ ತಿಳಿದಿರುವಂತೆ ಇದೆಲ್ಲ ವ್ಯರ್ಥ. ಭಕ್ತಿಯ ಎಲ್ಲ ಅಂಶಗಳೂ ಉಪನಿಷತ್ತುಗಳಲ್ಲಿರುವುದು ಮಾತ್ರವಲ್ಲದೆ ಅವಕ್ಕೂ ಹಿಂದಿನ ಸಂಹಿತಾ ಭಾಗದಲ್ಲಿಯೂ ಕಂಡುಬರುತ್ತದೆ. ಭಕ್ತಿಗೆ ಸಂಬಂಧಿಸಿದ ಪೂಜೆ, ಪ್ರೀತಿ ಮುಂತಾದ ಭಾವನೆಗಳೆಲ್ಲ ಅಲ್ಲಿವೆ. ಭಕ್ತಿಯ ಆದರ್ಶಗಳು ಕ್ರಮೇಣ ಉನ್ನತ ಮಟ್ಟಕ್ಕೆ ಏರಿದವು, ಅಷ್ಟೆ. ಸಂಹಿತಾ ಭಾಗದಲ್ಲಿ ದುಃಖಕ್ಕೆ ಮತ್ತು ಅಂಜಿಕೆಗೆ ಸಂಬಂಧಪಟ್ಟ ಧಾರ್ಮಿಕ ಭಾವನೆಗಳು ಕೆಲವು ಕಡೆ ಕಂಡುಬರುತ್ತವೆ. ಪೂಜಕನು ವರುಣ ಅಥವಾ ಇತರ ದೇವತೆಗಳ ಮುಂದೆ ಭಯದಿಂದ ನಡುಗುವುದನ್ನು ನಾವು ಸಂಹಿತೆಯಲ್ಲಿ ಕಾಣುತ್ತೇವೆ. ಅವರು ಪಾಪಭಾವನೆಯಿಂದ ಪೀಡಿತರಾಗಿರುವುದನ್ನೂ ನೋಡುತ್ತೇವೆ. ಆದರೆ ಉಪನಿಷತ್ತಿನಲ್ಲಿ ಇಂತಹ ಭಾವನೆಗಳಿಗೆ ಸ್ಥಳವಿಲ್ಲ. ಅಲ್ಲಿ ಅಂಜಿಕೆಯ ಧರ್ಮವಿಲ್ಲ. ಅಲ್ಲಿರುವುದು ಪ್ರೇಮ ಮತ್ತು ಜ್ಞಾನಮಯ ಧರ್ಮ.

ಉಪನಿಷತ್ತುಗಳು ನಮ್ಮ ಶಾಸ್ತ್ರಗ್ರಂಥಗಳು. ಅವು ವಿವಿಧ ರೀತಿಯಲ್ಲಿ ವಿವರಿಸಲ್ಪಟ್ಟಿವೆ ಮತ್ತು ನಾವು ಮೊದಲೇ ಹೇಳಿರುವಂತೆ ಎಲ್ಲಿ ಪುರಾಣಕ್ಕೂ ಮತ್ತು ವೇದಗಳಿಗೂ ಭಿನ್ನಾಭಿಪ್ರಾಯ ಬರುವುದೋ ಅಲ್ಲಿ, ಪುರಾಣಕ್ಕೆ ಪ್ರಾಮಾಣ್ಯವಿಲ್ಲ. ಆದರೆ ಅನುಷ್ಠಾನ ದೃಷ್ಟಿಯಿಂದ ನೋಡಿದರೆ ನಾವು ಶೇಕಡ ತೊಂಬತ್ತರಷ್ಟು ಪೌರಾಣಿಕರು, ಹತ್ತರಷ್ಟು\break ವೈದಿಕರು ಆಗಿರುವೆವು. ನಮ್ಮಲ್ಲಿ ಪ್ರಚಲಿತವಿರುವ ಎಷ್ಟೋ ಪರಸ್ಪರ ವಿರುದ್ಧ ಆಚಾರಗಳಿಗೆ, ಹಾಗೆಯೇ ನಮ್ಮ ಸಮಾಜದಲ್ಲಿರುವ ಎಷ್ಟೋ ಧಾರ್ಮಿಕ ಅಭಿಪ್ರಾಯಗಳಿಗೆ ಹಿಂದೂ ಶಾಸ್ತ್ರಗಳಲ್ಲಿ ಯಾವ ಪ್ರಮಾಣವೂ ಇಲ್ಲ. ಎಷ್ಟೋ ವೇಳೆ ನಾವು ಪುಸ್ತಕಗಳಲ್ಲಿ ಓದುವ ಮತ್ತು ದೇಶದಲ್ಲಿ ಕಾಣುವ ವಿಚಿತ್ರ ಆಚರಣೆಗಳಿಗೆ ಪ್ರಮಾಣವು ವೇದಗಳಲ್ಲಿ ದೊರೆಯದೆ ಇರುವುದು ಮಾತ್ರವಲ್ಲದೆ ಸ್ಮೃತಿ ಅಥವಾ ಪುರಾಣಗಳಲ್ಲಿಯೂ ದೊರೆಯುವುದಿಲ್ಲ. ಅವು ಕೇವಲ ಲೋಕಾಚಾರವಾಗಿವೆ. ಆದರೂ ತಿಳಿವಳಿಕೆ ಸಾಲದ ಪ್ರತಿಯೊಬ್ಬ ಹಳ್ಳಿಗನೂ ಯಾವುದೋ ಒಂದು ಲೋಕಾಚಾರ ನಷ್ಟವಾದರೆ ತಾನು ಹಿಂದುತ್ವವನ್ನೇ ಕಳೆದುಕೊಳ್ಳುವೆನು ಎಂದು ಭಾವಿಸುತ್ತಾನೆ. ಅವನ ಮನಸ್ಸಿನಲ್ಲಿ ವೇದಾಂತ ಮತ್ತು ಈ ಕ್ಷುದ್ರ ಲೋಕಾಚಾರ ಎರಡೂ ಬೇರ್ಪಡಿಸಲಾರದ ರೀತಿಯಲ್ಲಿ ಸೇರಿ ಕಲಸು ಮೇಲೋಗರವಾಗಿದೆ. ಶಾಸ್ತ್ರದಲ್ಲಿ ತಾನು ಮಾಡುತ್ತಿರುವುದಕ್ಕೆ ಆಧಾರವೇ ಇಲ್ಲ. ಅದನ್ನು ಬಿಟ್ಟರೆ ಯಾವ ಹಾನಿಯೂ ಇಲ್ಲ. ಅಷ್ಟೇ ಅಲ್ಲ ಅದರಿಂದ ತಾನು ಮತ್ತೂ ಉತ್ತಮನಾಗಬಹುದೆಂಬುದನ್ನು ಅವನು ಯೋಚಿಸಲಾರ. ಎರಡನೆಯದಾಗಿ ನಮ್ಮ ಶಾಸ್ತ್ರಗಳಲ್ಲಿ ಮತ್ತೊಂದು ತೊಡಕಿದೆ. ನಮ್ಮ ಶಾಸ್ತ್ರಗ್ರಂಥ ತುಂಬ ವಿಶಾಲವಾದುದು. ಪತಂಜಲಿಪ್ರಣೀತ ‘ಮಹಾಭಾಷ್ಯ’ ವೆಂಬ ಭಾಷಾಶಾಸ್ತ್ರ ಗ್ರಂಥದಲ್ಲಿ ಸಾಮವೇದದಲ್ಲಿ ಸಹಸ್ರ ಶಾಖೆಗಳಿವೆ ಎಂದು ಬರೆದಿರುವುದನ್ನು ಓದುತ್ತೇವೆ. ಅವೆಲ್ಲ ಈಗ ಎಲ್ಲಿ? ಯಾರಿಗೂ ಗೊತ್ತಿಲ್ಲ. ಅದರಂತೆಯೇ ಪ್ರತಿಯೊಂದು ವೇದದಲ್ಲಿಯೂ ಅದರ ಬಹುಭಾಗ ಲುಪ್ತವಾಗಿ ಎಲ್ಲೋ ಸ್ವಲ್ಪ ಮಾತ್ರ ಉಳಿದಿರುವುದು. ಒಂದೊಂದು ವಂಶದವರು ಒಂದೊಂದು ವೇದಶಾಖೆಯನ್ನು ತೆಗೆದುಕೊಂಡರು. ಈ ವಂಶ ತಾನಾಗಿ ನಾಶವಾಯಿತು, ಇಲ್ಲವೇ ಅನ್ಯದೇಶೀಯರ ಕೈಗೆ ಸಿಕ್ಕಿ ನಾಶವಾಯಿತು. ಇದರ ಜತೆಗೆ ಅವರ ಪಾಲಿನ ವೇದಶಾಖೆಯೂ ಮಾಯವಾಯಿತು. ಈ ವಿಷಯವನ್ನು ನಾವು ಯಾವಾಗಲೂ ಮನಸ್ಸಿನಲ್ಲಿಟ್ಟಿರಬೇಕು. ಏಕೆಂದರೆ ಹೊಸ ಭಾವನೆಗಳನ್ನು ಹರಡುವವರು ಅಥವಾ ವೇದಕ್ಕೆ ವಿರೋಧವಾಗಿ ಏನನ್ನಾದರೂ ಬೋಧಿಸುವವರು, ಈ ವಿಷಯಕ್ಕೆ ಹೆಚ್ಚು ಪ್ರಾಧಾನ್ಯ ಕೊಡುವರು. ಲೋಕಾಚಾರವು ಶ್ರುತಿಗೆ ವಿರೋಧವಾಗಿರುವಾಗ ಅವರು, ಅದು ಹಾಗಿಲ್ಲವೆಂದೂ ಈಗ ನಷ್ಟವಾಗಿ ಹೋಗಿರುವ ವೇದಗಳ ಶಾಖೆಗಳಲ್ಲಿ ಆ ಲೋಕಾಚಾರವು ಇದ್ದಿತೆಂದೂ ವಾದಿಸುತ್ತಾರೆ. ನಮ್ಮ ಶ್ರುತಿಗಳನ್ನು ಅಧ್ಯಯನ ಮಾಡುವ ಮತ್ತು ಅವನ್ನು ವಿವರಿಸುವ ಈ ಹಲವು ತರಹದ ವಿಧಾನಗಳ ಮಧ್ಯದಲ್ಲಿ ಅವುಗಳ ಅಂತರಾಳದಲ್ಲಿರುವ ಏಕಮತವನ್ನು ಕಂಡುಹಿಡಿಯುವುದು ಬಹಳ ಕಷ್ಟ. ಈ ವಿವಿಧ ಪಂಗಡಗಳು ಮತ್ತು ಉಪಪಂಗಡಗಳಿಗೆ ಒಂದು ಸಾಮಾನ್ಯವಾದ ಹಿನ್ನೆಲೆ ಇರಲೇಬೇಕೆಂದು ನಮಗೆ ಅನಿಸುತ್ತದೆ. ಅಲ್ಲಿ ಒಂದು ಸಾಮರಸ್ಯವಿರಬೇಕು ಯಾವುದೋ ಒಂದು ಸಾಮಾನ್ಯ ಯೋಜನೆಯ ಆಧಾರದ ಮೇಲೆ ಈ ಸಣ್ಣ ಪುಟ್ಟ ಕಟ್ಟಡಗಳನ್ನೆಲ್ಲ ಕಟ್ಟಿರಬೇಕು. ಮೇಲುನೋಟಕ್ಕೆ ಅತ್ಯಂತ ಗೊಂದಲಮಯವಾಗಿ ಕಾಣುವ ನಮ್ಮ ಈ ಧರ್ಮಕ್ಕೆ ಒಂದು ಸಾಮಾನ್ಯ ತಳಹದಿ ಇದ್ದಿರಲೇಬೇಕು. ಹಾಗಿಲ್ಲದೆ ಇದ್ದಿದ್ದರೆ ಅದು ಇಷ್ಟುಕಾಲ ಉಳಿಯುತ್ತಿರಲಿಲ್ಲ.

ನಮ್ಮ ಭಾಷ್ಯಕಾರರ ವಿಷಯಕ್ಕೆ ಬಂದರೆ ಇನ್ನೊಂದು ತೊಂದರೆ ಕಂಡುಬರುತ್ತದೆ. ಅದ್ವೈತ ಭಾಷ್ಯಕಾರನು ಅದ್ವೈತಕ್ಕೆ ಸಂಬಂಧಪಟ್ಟ ಶ್ರುತಿಯ ಭಾಗವನ್ನು ಹಾಗೆಯೇ ಇಡುತ್ತಾನೆ. ಆದರೆ ದ್ವೈತಕ್ಕೆ ಸಂಬಂಧಪಟ್ಟ ಭಾಗವನ್ನು ಹಿಸುಕಿ ಅದಕ್ಕೆ ವಿಚಿತ್ರ ಅರ್ಥ ಬರುವಂತೆ ಮಾಡುತ್ತಾನೆ. ಕೆಲವು ವೇಳೆ ‘ಅಜ’ (ಹುಟ್ಟಿಲ್ಲದುದು) ಮೇಕೆಯಾಗುವುದು. ಬದಲಾವಣೆ ಇಷ್ಟು ವಿಚಿತ್ರವಾಗಿದೆ! ಭಾಷ್ಯಕಾರನ ಅನುಕೂಲಕ್ಕಾಗಿ ಅಜ ಮೇಕೆಯಾಗುವುದು. ಇದರಂತೆಯೇ ಅಥವಾ ಇದಕ್ಕಿಂತಲೂ ವಿಚಿತ್ರವಾಗಿ ದ್ವೈತಿಗಳು ಶ್ರುತಿವಾಕ್ಯಗಳನ್ನು ವಿವರಿಸುವರು. ದ್ವೈತಕ್ಕೆ ಸಂಬಂಧ ಪಟ್ಟುದನ್ನೆಲ್ಲಾ ಹಾಗೆಯೇ ಉಳಿಸಿಕೊಂಡು ಅದ್ವೈತಕ್ಕೆ ಸಂಬಂಧಪಟ್ಟುದಕ್ಕೆ ತಮ್ಮ ಮನಸ್ಸಿಗೆ ತೋರಿದಂತೆ ಅರ್ಥ ಕೊಡುವರು. ಸಂಸ್ಕೃತ ಭಾಷೆಯು ಅತ್ಯಂತ ಜಟಿಲವಾದುದು. ವೇದದ ಭಾಷೆಯು ಅತಿ ಪ್ರಾಚೀನ ಸಂಸ್ಕೃತ. ಸಂಸ್ಕೃತ ಭಾಷಾಶಾಸ್ತ್ರ ಎಷ್ಟು ಪ್ರೌಢವಾಗಿದೆಯೆಂದರೆ ಯಾವುದಾದರೂ ಒಂದು ಪದದ ಅರ್ಥದ ಮೇಲೆ ಎಷ್ಟು ವರ್ಷಗಳಾದರೂ ಚರ್ಚೆಮಾಡಬಹುದು. ಪಂಡಿತರಿಗೆ ಮನಸ್ಸು ಬಂದರೆ ಯಾವ ಹರಟೆಯನ್ನಾದರೂ ತೆಗೆದುಕೊಂಡು, ತಮ್ಮ ವಾದ ಶಕ್ತಿಯ ಮೂಲಕ ವ್ಯಾಕರಣ ನಿಯಮಗಳನ್ನು ಉದಾಹರಿಸಿ ಅದು ಶುದ್ಧ ಸಂಸ್ಕೃತವೆಂದು ತೋರಬಲ್ಲರು. ಉಪನಿಷತ್ತುಗಳನ್ನು ತಿಳಿದುಕೊಳ್ಳುವುದಕ್ಕೆ ಇರುವ ತೊಡಕುಗಳು ಇವು. ಕಟ್ಟಾ ದ್ವೈತಿಯಾಗಿರುವಷ್ಟೇ ಅದ್ವೈತಿಯೂ ಆಗಿದ್ದ, ಪರಮಭಕ್ತನಾಗಿರುವಷ್ಟೇ ಜ್ಞಾನಿಯೂ ಆಗಿದ್ದ ಒಬ್ಬರೊಂದಿಗೆ ಇರುವ ಭಾಗ್ಯ ನನ್ನದಾಗಿತ್ತು. ಅವರ ಬಳಿಯಲ್ಲಿ ವಾಸಿಸುತ್ತಿರುವಾಗ ಉಪನಿಷತ್ತು ಮತ್ತು ಇತರ ಶಾಸ್ತ್ರಗಳನ್ನು, ಕೇವಲ ಭಾಷ್ಯಕಾರರನ್ನು ಸುಮ್ಮನೆ ಅನುಸರಿಸದೆ, ಸ್ವತಂತ್ರವಾಗಿ ಬೇರೊಂದು ದೃಷ್ಟಿಯಿಂದ ಅಧ್ಯಯನಮಾಡಬೇಕು ಎನ್ನಿಸಿತು. ನನ್ನ ಸಂಶೋಧನೆ ಮತ್ತು ಅಭಿಪ್ರಾಯದ ಪ್ರಕಾರ ಈ ಶಾಸ್ತ್ರವಾಕ್ಯಗಳು ಪರಸ್ಪರ ವಿರೋಧವಾಗಿಲ್ಲ. ಆದ್ದರಿಂದ ನಮಗೆ ಗ್ರಂಥಪೀಡನೆಯ ಗೊಡವೆಯೇ ಬೇಕಿಲ್ಲ. ಶಾಸ್ತ್ರಗಳು ಅತಿ ಸುಂದರವಾಗಿವೆ. ಅವು ಅತಿ ಅದ್ಭುತವಾಗಿವೆ. ಅವುಗಳಲ್ಲಿ ಪರಸ್ಪರ ವಿರೋಧವಿಲ್ಲ. ಅವುಗಳಲ್ಲಿ ಸಾಮರಸ್ಯವಿದೆ. ಒಂದು ಭಾವನೆ ಮತ್ತೊಂದು ಭಾವನೆಗೆ ಒಯ್ಯುವುದು. ನನಗೆ ತೋರಿದ ಒಂದು ಭಾವನೆಯೇ ಎಲ್ಲ ಉಪನಿಷತ್ತುಗಳೂ ಉಪಾಸನೆಯ ದ್ವೈತಭಾವದಿಂದ ಪ್ರಾರಂಭವಾಗಿ ಪರಮ ಅದ್ವೈತ ಭಾವನೆಯಲ್ಲಿ ಕೊನೆಗಾಣುವುವು ಎಂಬುದು.

ನಾನು ಈಗತಾನೆ ಪ್ರಸ್ತಾಪಿಸಿದ ವ್ಯಕ್ತಿಯ ಜೀವನದ ಬೆಳಕಿನಲ್ಲಿ ನೋಡಿದರೆ ದ್ವೈತಿಗಳು ಮತ್ತು ಅದ್ವೈತಿಗಳು ಒಬ್ಬರು ಮತ್ತೊಬ್ಬರೊಂದಿಗೆ ಹೋರಾಡಬೇಕಾಗಿಲ್ಲ, ಇಬ್ಬರಿಗೂ ಸ್ಥಳವಿದೆ, ರಾಷ್ಟ್ರೀಯ ಜೀವನದಲ್ಲಿ ಇಬ್ಬರಿಗೂ ಉತ್ತಮ ಸ್ಥಾನವಿದೆ. ದ್ವೈತಿಯೂ ಇರಬೇಕು. ಏಕೆಂದರೆ ಅವನು ಅದ್ವೈತಿಯಷ್ಟೇ ಜನಾಂಗದ ಧಾರ್ಮಿಕ ಜೀವನದ ಅವಿಭಾಜ್ಯ ಅಂಗವಾಗಿದ್ದಾನೆ. ಒಬ್ಬರು ಮತ್ತೊಬ್ಬರಿಲ್ಲದೆ ಜೀವಿಸಲಾರರು. ಇಬ್ಬರೂ ಪರಸ್ಪರ ಪೂರಕ. ಒಂದು ಕಟ್ಟಡ, ಮತ್ತೊಂದು ಶಿಖರ, ಒಂದು ಮೂಲಬೇರು ಮತ್ತೊಂದು ಫಲ. ಆದ್ದರಿಂದ ಉಪನಿಷತ್ತಿನ ಮೂಲವನ್ನು ತಿರುಚುವುದು ತಿಳಿಗೇಡಿತನವೆಂದು ನಾನು ಭಾವಿಸುತ್ತೇನೆ. ಉಪನಿಷತ್ತಿನ ಭಾಷೆ ಅಪೂರ್ವವಾದುದು. ಅದು ಶ್ರೇಷ್ಠತಮ ದರ್ಶನಶಾಸ್ತ್ರವೂ ಮಾನವಜಾತಿಗೆ ಮುಕ್ತಿ ಪಥ ದರ್ಶಕವೂ ಆಗಿರುವುದು ಮಾತ್ರವಲ್ಲದೆ ಅದು ಜಗತ್ತಿನಲ್ಲೇ\break ಭವ್ಯತೆಯನ್ನು ಚಿತ್ರಿಸುವ ಅತ್ಯುತ್ತಮ ಸಾಹಿತ್ಯ. ನಮಗೆ ಅಲ್ಲಿ ಮಾನವ ಮನಸ್ಸಿನ ವೈಯಕ್ತಿಕತೆ ಮತ್ತು ಹಿಂದೂ ಮನಸ್ಸಿನ ಅಂತರ್​ದೃಷ್ಟಿಪರಾಯಣತೆ ಇವುಗಳ ಪರಿಚಯ ಅತ್ಯಂತ ಸ್ಪಷ್ಟವಾಗಿ ಆಗುತ್ತದೆ.

ಇತರ ದೇಶಗಳಲ್ಲೂ ಭವ್ಯತೆಯ ವರ್ಣನೆಗಳನ್ನು ನೋಡುತ್ತೇವೆ. ಆದರೆ ಎಲ್ಲಾ ಕಡೆಗಳಲ್ಲಿಯೂ ಅವರ ಆದರ್ಶವು ಭೌತಿಕತೆಯಲ್ಲಿ ಭವ್ಯತೆಯನ್ನು ಕಾಣುವ ಪ್ರಯತ್ನವಾಗಿದೆ. ಉದಾಹರಣೆಗೆ ಮಿಲ್ಟನ್​, ಡಾಂಟೆ, ಹೋಮರ್​ ಅಥವಾ ಉಳಿದ ಪಾಶ್ಚಾತ್ಯ ಕವಿಗಳನ್ನು ತೆಗೆದುಕೊಳ್ಳಿ. ಅವರ ಕಾವ್ಯಗಳಲ್ಲೂ ಭವ್ಯತೆಯ ವರ್ಣನೆಗಳಿವೆ. ಆದರೆ ಅವುಗಳೆಲ್ಲ ಇಂದ್ರಿಯಗಳ ಮೂಲಕ ಅನಂತತೆಯನ್ನು ಹಿಡಿಯುವ ಪ್ರಯತ್ನ, ಅನಂತ ಆಕಾಶದ ಆದರ್ಶವನ್ನು ಮುಟ್ಟುವ ಪ್ರಯತ್ನ. ಸಂಹಿತೆಯ ಭಾಗದಲ್ಲಿಯೂ ಅದೇ ಪ್ರಯತ್ನ ನಡೆದಿದೆ. ಸೃಷ್ಟಿಯನ್ನು ವಿವರಿಸುವ ಅಪೂರ್ವ ಋಕ್​ ಮಂತ್ರಗಳು ನಿಮಗೆ ಪರಿಚಿತವಾಗಿರಬಹುದು. ಅಲ್ಲಿ ಅನಂತ ಆಕಾಶದ ವರ್ಣನೆಯ ಮೂಲಕ ಪರಮೋಚ್ಚ ಭವ್ಯತೆಯ ಭಾವನೆಗಳು ವ್ಯಕ್ತವಾಗಿವೆ. ಆದರೆ ಅವರು ಇದರ ಸಹಾಯದಿಂದ ಅನಂತಸ್ವರೂಪ ಪ್ರಾಪ್ತವಾಗಲಾರದೆಂಬುದನ್ನು ಬಹುಬೇಗ ಕಂಡುಹಿಡಿದರು. ದೇಶದ ಅನಂತತೆ, ಬಾಹ್ಯಪ್ರಕೃತಿಯ ಅನಂತತೆ ಇವುಗಳಿಗೆ ಅವರ ಮನಸ್ಸಿನಲ್ಲಿ ಹೊರಬರಲು ಪ್ರಯತ್ನಿಸುತ್ತಿದ್ದ ಭಾವನೆಗಳನ್ನು ವ್ಯಕ್ತಗೊಳಿಸಲು ಅಸಾಧ್ಯವಾಯಿತು. ಆದ್ದರಿಂದ ಅವರು ಬೇರೆ ರೀತಿಯ ವಿವರಣೆಗೆ ಕೈಹಾಕಬೇಕಾಯಿತು. ಉಪನಿಷತ್ತಿನಲ್ಲಿ ಭಾಷೆಯು ಹೊಸ ರೂಪವನ್ನು ತಾಳಿತು. ಅದು ಬಹುಪಾಲು ನಿಷೇಧಾತ್ಮಕವಾಗಿದೆ. ಕೆಲವು ಕಡೆ ಅವ್ಯವಸ್ಥಿತವಾಗಿದೆ. ಕೆಲವೊಮ್ಮೆ ನಿಮ್ಮನ್ನು ಇಂದ್ರಿಯಗಳಾಚೆ ಕರೆದುಕೊಂಡು ಹೋಗಿ ನಿಮಗೆ ಗ್ರಹಿಸಲಾಗದುದನ್ನು ನಿರ್ದೇಶಿಸುತ್ತದೆ. ನೀವದನ್ನು ಭಾವಿಸಲಾಗದಿದ್ದರೂ ಅದರ ಇರುವಿಕೆಯನ್ನು ಸಂದೇಹಿಸಲಾರಿರಿ. ಪ್ರಪಂಚದಲ್ಲಿ ಮತ್ತಾವ ಭಾಷೆ ಇದಕ್ಕೆ ಸರಿಸಮವಾಗಿರುವುದು?

\begin{verse}
\textbf{ನ ತತ್ರ ಸೂರ್ಯೋ ಭಾತಿ ನ ಚಂದ್ರತಾರಕಮ್​~।}\\\textbf{ನೇಮಾ ವಿದ್ಯುತೋ ಭಾನ್ತಿ ಕುತೋಽಯಮಗ್ನಿಃ~॥}
\end{verse}

\hfill —ಕಠೋಪನಿಷತ್​

“ಅಲ್ಲಿ ಸೂರ್ಯ ಪ್ರಕಾಶಿಸಲಾರ, ಚಂದ್ರನೂ ಇಲ್ಲ, ತಾರೆಯೂ ಇಲ್ಲ. ಮಿಂಚು ಕೂಡ ಅಲ್ಲಿ ಪ್ರಕಾಶಿಸಲಾರದು. ಇನ್ನು ಸಾಮಾನ್ಯ ಅಗ್ನಿಯ ಪಾಡೇನು!” ಇಡೀ ಜಗತ್ತಿನ ಸಮಗ್ರ ದಾರ್ಶನಿಕ ಭಾವನೆಯ ಪರಿಪೂರ್ಣ ಚಿತ್ರ, ಹಿಂದೂಗಳ ಒಟ್ಟು ಚಿಂತನೆಯ ಸಾರ, ಮಾನವನ ಮೋಕ್ಷಾಕಾಂಕ್ಷೆಯ ಸಮಗ್ರ ಕಲ್ಪನೆ–ಇವು ಈ ಕೆಳಗಿನ ಮಂತ್ರದಲ್ಲಿ ವ್ಯಕ್ತವಾಗಿರುವಂತೆ, ಇಂತಹ ಭವ್ಯವಾದ ಭಾಷೆಯಲ್ಲಿ, ಇಂತಹ ಅದ್ಭುತವಾದ ರೂಪಕದ ಮೂಲಕ, ಇನ್ನೆಲ್ಲಿ ವರ್ಣಿತವಾಗಿದೆ?

\begin{longtable}{@{\hspace{-18pt}}l@{}}
\textbf{ದ್ವಾ ಸುಪರ್ಣಾ ಸಯುಜಾ ಸಖಾಯಾ ಸಮಾನಂ ವೃಕ್ಷಂ ಪರಿಷಸ್ವಜಾತೇ ।} \\
\textbf{ತಯೋರನ್ಯಃ ಪಿಪ್ಪಲಂ ಸ್ವಾದ್ವತ್ತ್ಯನಶ್ನನ್ನನ್ಯೋ ಅಭಿಚಾಕಶೀತಿ ॥} \\
\textbf{ಸಮಾನೇ ವೃಕ್ಷೇ ಪುರುಷೋ ನಿಮಗ್ನೋಽನೀಶಯಾ} \\
\textbf{ಶೋಚತಿ ಮುಹ್ಯಮಾನಃ ।} \\
\textbf{ಜುಷ್ಟಂ ಯದಾ ಪಶ್ಯತ್ಯನ್ಯಮೀಶಮಸ್ಯ ಮಹಿಮಾನಮಿತಿ ವೀತಶೋಕಃ ॥} \\
\end{longtable}

\hfill —ಮುಂಡಕ ಉಪನಿಷತ್​

“ಒಂದೇ ಮರದಲ್ಲಿ ಎರಡು ಸುಂದರ ಗರಿಗಳ ಹಕ್ಕಿಗಳಿವೆ. ಎರಡೂ ಅನ್ಯೋನ್ಯ ಪ್ರೀತಿಯಿಂದ ಇವೆ. ಒಂದು ಹಣ್ಣನ್ನು ತಿನ್ನುತ್ತಿದೆ. ಮತ್ತೊಂದು ಏನನ್ನೂ ತಿನ್ನದೆ ಶಾಂತವಾಗಿ ಕುಳಿತಿದೆ. ಕೆಳಗಿನ ರೆಂಬೆಯಲ್ಲಿ ಕುಳಿತಿರುವ ಹಕ್ಕಿ ಒಂದು ಸಲ ಕಹಿ ಮತ್ತೊಂದು ಸಲ ಸಿಹಿ ಹಣ್ಣನ್ನು ತಿಂದು, ಸುಖದುಃಖ ಅನುಭವಿಸುತ್ತದೆ. ಮೇಲಿರುವ ಹಕ್ಕಿ ಶಾಂತವಾಗಿ ಗಂಭೀರವಾಗಿದೆ. ಸಿಹಿ ಹಣ್ಣನ್ನಾಗಲಿ ಕಹಿ ಹಣ್ಣನ್ನಾಗಲಿ ತಿನ್ನುತ್ತಿಲ್ಲ, ಸುಖದುಃಖಗಳನ್ನೂ ಲೆಕ್ಕಿಸುವುದಿಲ್ಲ. ತನ್ನ ಸ್ವಯಂಪ್ರಭೆಯಲ್ಲಿ ತಲ್ಲೀನವಾಗಿರುವುದು.” ಇದೇ ಮಾನವ ಜೀವಿಯ ಚಿತ್ರ. ಮಾನವನು ಜೀವನದ ಕಹಿಸಿಹಿಗಳನ್ನು ತಿನ್ನುತ್ತಿರುವನು. ಕಾಂಚನ, ಇಂದ್ರಿಯ ಸುಖ, ಜೀವನದ ಪೊಳ್ಳು ಆಶೋತ್ತರಗಳು ಇವನ್ನರಸುತ್ತಾ ಹುಚ್ಚನಂತೆ ಧಾವಿಸುತ್ತಿರುವನು. ಮತ್ತೊಂದು ಕಡೆ ಉಪನಿಷತ್ತು ಜೀವಿಯನ್ನು ರಥಿಗೂ, ಇಂದ್ರಿಯಗಳನ್ನು ನಿಗ್ರಹಿಸದ ಕುದುರೆಗಳಿಗೂ ಹೋಲಿಸು ವುದು. ಪ್ರಪಂಚದ ಕೆಲಸಕ್ಕೆ ಬಾರದ ವಸ್ತುಗಳನ್ನು ಅರಸುವವರ ಬಾಳು ಹೀಗೆ. ಮಕ್ಕಳು ಸುಂದರ ಸವಿ ಕನಸುಗಳನ್ನು ಕಲ್ಪಿಸಿಕೊಳ್ಳುತ್ತಿರುವರು. ಕೊನೆಗೆ ಇವೆಲ್ಲಾ ಪುಡಿಪುಡಿಯಾಗುವುದನ್ನು ನೋಡುವರು. ಮುದುಕರು ತಮ್ಮ ಹಿಂದಿನ ನೆನಪನ್ನು ಮೆಲಕು ಹಾಕುತ್ತಿರುವರು. ಆದರೂ ಸಂಸಾರ ಜಾಲದಿಂದ ತಪ್ಪಿಸಿಕೊಳ್ಳುವುದಕ್ಕೆ ದಾರಿಯನ್ನು ಅವರು ಕಾಣರು. ಇದೇ ಜಗತ್ತು. ಆದರೂ ಪ್ರತಿಯೊಬ್ಬರ ಜೀವನದಲ್ಲಿ ಒಂದು ಸುವರ್ಣ ಕ್ಷಣ ಲಭಿಸುವುದು. ಅತ್ಯಂತ ದಾರುಣ ದುಃಖದಲ್ಲಿರುವಾಗ ಆಗಲಿ, ಅಥವಾ ಅತ್ಯಾನಂದದಲ್ಲಿ ತನ್ಮಯರಾಗಿರುವಾಗ ಆಗಲಿ, ಸೂರ್ಯನನ್ನು ಕಾಣದಂತೆ ಮಾಡಿದ್ದ ಮೋಡ ಕ್ಷಣಕಾಲ ಮಾಯವಾದಂತೆ ತೋರುವುದು. ನಮಗೆ ಇಚ್ಛೆ ಇಲ್ಲದೇ ಇದ್ದರೂ ಅತೀಂದ್ರಿಯ ಸತ್ಯದ ಕ್ಷಣಿಕ ದರ್ಶನ ಲಭಿಸುವುದು. ಅದು ಇಂದ್ರಿಯ ಸುಖವನ್ನು ಮೀರಿರುವುದು, ಪ್ರಪಂಚದ ವ್ಯರ್ಥ ಆಸೆ ಸುಖ ದುಃಖಗಳನ್ನು ಮೀರಿರುವುದು, ಪ್ರಕೃತಿಯಾಚೆ, ನಮ್ಮ ಇಹಪರಗಳ\break ಕಲ್ಪನೆಯನ್ನು ಮೀರಿರುವುದು, ಧನ ಯಶಸ್ಸು ಸಂತಾನ ತೃಷ್ಣೆಗಳನ್ನು ಮೀರಿರುವುದು. ಮನುಷ್ಯನಿಗೆ ಕ್ಷಣಕಾಲದ ಈ ದಿವ್ಯದೃಷ್ಟಿ ಪ್ರಾಪ್ತವಾದಾಗ ಅವನು ಮರದ ಮೇಲೆ ಸುಖ ದುಃಖಗಳನ್ನು ಭೋಗಿಸದೆ ಶಾಂತವಾಗಿ ಆತ್ಮತೃಪ್ತವಾಗಿ ತನ್ನ ಮಹಿಮೆಯಲ್ಲಿ ತಲ್ಲೀನವಾಗಿರುವ ಹಕ್ಕಿಯನ್ನು ನೋಡುವನು. ಭಗವದ್​ಗೀತೆ ಹೇಳುವಂತೆ:

\begin{verse}
\textbf{ಯಸ್ತ್ವಾತ್ಮರತಿರೇವ ಸ್ಯಾದಾತ್ಮತೃಪ್ತಶ್ಚ ಮಾನವಃ~।}\\\textbf{ಆತ್ಮನ್ಯೇವ ಚ ಸಂತುಷ್ಟಸ್ತಸ್ಯ ಕಾರ್ಯಂ ನ ವಿದ್ಯತೇ~॥}
\end{verse}

“ಯಾರು ಆತ್ಮನಲ್ಲೇ ರಮಿಸುವರೋ, ಆತ್ಮತೃಪ್ತರಾಗಿರುವರೋ, ಆತ್ಮ ಸಂತುಷ್ಟರಾಗಿರುವರೋ ಅವರು ಮಾಡಬೇಕಾಗಿರುವ ಕರ್ತವ್ಯ ಯಾವುದಿರುವುದು?” ಅವರೇಕೆ ಗೊಣಗಾಡಬೇಕು? ಮಾನವನಿಗೆ ಅತೀಂದ್ರಿಯ ಸತ್ಯದ ಕ್ಷಣಿಕ ನೋಟ ದೊರಕುವುದು. ಪುನಃ ಮರೆಯುವನು. ಜೀವನ ವೃಕ್ಷದ ಕಹಿ ಸಿಹಿ ಹಣ್ಣುಗಳನ್ನು ತಿನ್ನುತ್ತಾ ಹೋಗುವನು. ಕೆಲವು ಕಾಲವಾದ ಮೇಲೆ ಪುನಃ ಕ್ಷಣಿಕ ನೋಟವೊಂದು ಲಭಿಸುವುದು. ಕೆಳಗಿರುವ ಹಕ್ಕಿ ಪೆಟ್ಟು ತಿಂದಂತೆಲ್ಲಾ ಮೇಲೆ ಮೇಲಕ್ಕೆ ಹೋಗುವುದು. ಇನ್ನೂ ಹೆಚ್ಚು ಪೆಟ್ಟು ಬೀಳುವ ಅದೃಷ್ಟವಿದ್ದರೆ ಮೇಲಿರುವ, ತನ್ನ ಪ್ರಾಣದಂತಿರುವ ಗೆಳೆಯನ ಹತ್ತಿರ ಹೋಗುವುದು. ಅದನ್ನು ಸಮೀಪಿಸಿದಂತೆಲ್ಲ ಅದರ ಕಾಂತಿ ತನ್ನ ಮೇಲೆ ಬೀಳುತ್ತಿರುವುದು ಕಾಣುತ್ತದೆ.\break ಕೆಳಗಿನ ಹಕ್ಕಿ ಮೇಲಿನದನ್ನು ಸಮೀಪಿಸಿದಂತೆಲ್ಲ ಬದಲಾಗುತ್ತಾ ಬರುವುದು.\break ಅದು ಹತ್ತಿರ ಹತ್ತಿರ ಬಂದಂತೆಲ್ಲಾ ಮಾಯವಾದಂತೆ ತೋರಿ ಕೊನೆಗೆ ಒಂದೇ ಹಕ್ಕಿ ಉಳಿಯುವುದು. ಅದು ನಿಜವಾಗಿಯೂ ಇರಲಿಲ್ಲ. ಅಲ್ಲಾಡುವ ಎಲೆಗಳ ಮಧ್ಯದಲ್ಲಿ ಶಾಂತವಾಗಿ ಗಂಭೀರವಾಗಿದ್ದ ಹಕ್ಕಿಯ ಪ್ರತಿಬಿಂಬದಂತೆ ಮಾತ್ರ ಇತ್ತು ಎರಡನೇ ಹಕ್ಕಿ. ಇದೆಲ್ಲ ಮೇಲಿರುವ ಹಕ್ಕಿಯ ಮಹಿಮೆ. ಆಗ ಅದು ಆತ್ಮತೃಪ್ತವಾಗಿ ಶಾಂತವಾಗಿ ಅಂಜಿಕೆ ಇಲ್ಲದೆ ಇರುತ್ತದೆ. ಈ ಉಪಮಾನದಲ್ಲಿ ಉಪನಿಷತ್ತು ನಮ್ಮನ್ನು ದ್ವೈತಸಿದ್ಧಾಂತವಾದದಿಂದ ಪರಮಾದ್ವೈತಕ್ಕೆ ಒಯ್ಯುತ್ತದೆ.

\vskip   2pt

ಹೀಗೆ ಬೇಕಾದಷ್ಟು ಉದಾಹರಣೆಗಳನ್ನು ಕೊಡಬಹುದು. ಆದರೆ ಉಪನಿಷತ್ತಿನ ಅಪೂರ್ವ ಕವಿತ್ವ, ಭವ್ಯತೆಯ ಚಿತ್ರ, ಮಹೋಚ್ಚ ಭಾವಸಮೂಹ ಇವನ್ನು ವಿವರಿಸಲು ಈಗ ಸಮಯವಿಲ್ಲ. ಆದರೆ ನಾನು ನಿಮಗೆ ಮತ್ತೊಂದು ವಿಷಯವನ್ನು ಹೇಳಬೇಕಾಗಿದೆ. ಅಲ್ಲಿಯ ಭಾಷೆ ಮತ್ತು ಭಾವನೆ ನೇರವಾಗಿ ಹರಿದು ಬರುವುವು, ಸುತ್ತಿಗೆಯ ಪೆಟ್ಟಿನಂತೆ ಬೀಳುವುವು, ಹಿರಿದ ಅಲಗಿನಂತೆ ನಿಮ್ಮ ಮೇಲೆ ಎರಗುವುವು. ಅವುಗಳ ಅರ್ಥವನ್ನು ತಿಳಿದುಕೊಳ್ಳಲು ಕಷ್ಟವಿಲ್ಲ. ಆ ಸಂಗೀತದ ಪ್ರತಿಯೊಂದು ಸ್ವರದಲ್ಲಿಯೂ ಶಕ್ತಿ ಇರುವುದು, ಅದು ಪೂರ್ಣಪ್ರಭಾವವನ್ನು ನಮ್ಮ ಮೇಲೆ ಬೀರುವುದು. ಅಲ್ಲಿ ಯಾವ ವಿಧದ ಅಸ್ಪಷ್ಟತೆಯೂ ಇಲ್ಲ. ಒಂದು ಅಸಂಬದ್ಧ ಭಾವನೆಯೂ ಇಲ್ಲ. ಬುದ್ಧಿ ಭ್ರಮಣೆಗೊಳಿಸುವ ಯಾವ ಜಟಿಲತೆಯೂ ಇಲ್ಲ. ಅಲ್ಲಿ ಅವನತಿಯ ಚಿಹ್ನೆ ಇಲ್ಲ. ಅತಿಯಾದ ಉಪಮಾನ ರೂಪಕಗಳನ್ನು, ರಾಶಿರಾಶಿ ಗುಣವಾಚಕಗಳನ್ನು ಸೇರಿಸಿ ಮೂಲಾರ್ಥವನ್ನೇ ಕೆಡಿಸಿ ಓದುಗರ ತಲೆತಿರುಗುವಂತೆ ಮಾಡಿ, ಅವರನ್ನು ವ್ಯೂಹದಲ್ಲಿ ಸಿಕ್ಕಿಸಿ, ದಾರಿಗಾಣದಂತೆ ಮಾಡುವ ಪ್ರಯತ್ನವಿಲ್ಲ ಅಲ್ಲಿ. ಇಂತಹ ಚಿಹ್ನೆ ಅಲ್ಲಿ ಯಾವುದೂ ಇಲ್ಲ. ಅದು ಮಾನವಸಾಹಿತ್ಯವಾಗಿದ್ದರೆ, ಜನಾಂಗವು ತನ್ನ ರಾಷ್ಟ್ರೀಯ ಓಜಸ್ಸನ್ನು ಕಳೆದು ಕೊಳ್ಳುವುದಕ್ಕಿಂತ ಮುಂಚೆ ಇದನ್ನು ಸೃಷ್ಟಿಸಿರಬೇಕು.

\vskip   2pt

ಪ್ರತಿಯೊಂದು ಮಂತ್ರದಲ್ಲಿಯೂ ಉಪನಿಷತ್​ “ಶಕ್ತಿ, ಶಕ್ತಿ” ಎಂದು ಸಾರುತ್ತದೆ. ನಾವು ಜ್ಞಾಪಕದಲ್ಲಿಡಬೇಕಾದ ಒಂದು ಮಹಾ ವಿಷಯ ಇದು. ನನ್ನ ಜೀವನದಲ್ಲಿ ಕಲಿತ ಒಂದು ಮಹಾನೀತಿ ಇದು. ಹೇ ಮಾನವ, ಶಕ್ತನಾಗು, ದುರ್ಬಲನಾಗಬೇಡ ಎನ್ನುತ್ತದೆ ಉಪನಿಷತ್ತು. ಮಾನವಸಹಜ ದೌರ್ಬಲ್ಯಗಳು ಎಂಬುದು ಇಲ್ಲವೆ, ಎಂದು ಕೇಳುತ್ತಾನೆ ಮಾನವ. ಇವೆ ಎನ್ನುತ್ತದೆ ಉಪನಿಷತ್ತು. ಆದರೆ ಹೆಚ್ಚು ದೌರ್ಬಲ್ಯವು ಅದನ್ನು ಗುಣಪಡಿಸಬಲ್ಲದೆ? ಕೊಳೆಯಿಂದ ಕೊಳೆಯನ್ನು ತೊಳೆಯಬಲ್ಲೆಯಾ? ಪಾಪವು ಪಾಪವನ್ನು ಗುಣಪಡಿಸುವುದೆ? ದೌರ್ಬಲ್ಯವು ದೌರ್ಬಲ್ಯವನ್ನು ಹೋಗಲಾಡಿಸುವುದೆ? ಹೇ ಮಾನವ, ಶಕ್ತಿ ಶಕ್ತಿ, ಎದ್ದು ನಿಲ್ಲು, ಧೀರನಾಗು ಎನ್ನುತ್ತವೆ, ಉಪನಿಷತ್ತುಗಳು. ವಿಶ್ವಸಾಹಿತ್ಯದಲ್ಲಿ ಇದೊಂದು ಮಾತ್ರ “ಅಭೀಃ”, “ಅಭೀಃ” ಎಂದು ಪುನಃ ಪುನಃ ಸಾರುವುದು. ಜಗತ್ತಿನ ಮತ್ತಾವ ಸಾಹಿತ್ಯದಲ್ಲಿಯೂ ಇದನ್ನು ಮಾನವನಿಗಾಗಲಿ ದೇವರಿಗಾಗಲಿ ವಿಶೇಷಣವಾಗಿ ಉಪಯೋಗಿಸಿಲ್ಲ. ಅಭೀಃ, ನಿರ್ಭಯತೆ. ನನ್ನ ಮನಸ್ಸಿನಲ್ಲಿ ಗತಕಾಲದ ಪಾಶ್ಚಾತ್ಯ ಚಕ್ರವರ್ತಿ ಅಲೆಗ್ಸಾಂಡರನ ಚಿತ್ರ ಬರುವುದು. ಅವನು ಸಿಂಧೂನದಿಯ ತೀರದ ಕಾಡಿನಲ್ಲಿ ನಮ್ಮ ದೇಶದ ಸಂನ್ಯಾಸಿಯೊಬ್ಬನೊಂದಿಗೆ ಮಾತನಾಡುವುದು ಜ್ಞಾಪಕಕ್ಕೆ ಬರುವುದು. ಅವನು ಮಾತನಾಡುತ್ತಿದ್ದ ವೃದ್ಧ ಬಹುಶಃ ನಗ್ನನಾಗಿ ಕಲ್ಲುಬಂಡೆಯೊಂದರ ಮೇಲೆ ಕುಳಿತಿದ್ದನು. ಅಲೆಗ್ಸಾಂಡರ್​ ಅವನ ಜ್ಞಾನಕ್ಕೆ ವಿಸ್ಮಿತನಾಗಿ “ಗ್ರೀಸ್​ ದೇಶಕ್ಕೆ ಬಾ” ಎಂದು ಕೀರ್ತಿಯ ಮತ್ತು ಹಣದ ಆಸೆಯನ್ನು ತೋರಿಸಿದನು. ಸಂನ್ಯಾಸಿ ಆತನು ಒಡ್ಡಿದ ಹೊನ್ನಿನ ವಿಷಯಕ್ಕೆ ನಕ್ಕನು, ಒಡ್ಡಿದ ಪ್ರಲೋಭನೆಯನ್ನು ನೋಡಿ ನಕ್ಕನು. ಎಲ್ಲವನ್ನೂ ನಿರಾಕರಿಸಿದನು. ಚಕ್ರವರ್ತಿ ತನ್ನ ಅಧಿಕಾರದ ದರ್ಪದಿಂದ, “ನೀನು ಬರದೇ ಇದ್ದರೆ ನಿನ್ನನ್ನು ಕೊಲ್ಲುತ್ತೇನೆ” ಎನ್ನುವನು. ಆಗ ಸಂನ್ಯಾಸಿ ನಕ್ಕು, “ಈಗ ಹೇಳಿದಂತಹ ಸುಳ್ಳನ್ನು ನೀನು ಜೀವನದಲ್ಲಿ ಎಂದೂ ಹೇಳಿಲ್ಲ. ಬಾಹ್ಯ ಪ್ರಪಂಚದ ಚಕ್ರವರ್ತಿಯಾದ ನೀನು ನನ್ನನ್ನು ಕೊಲ್ಲುವೆಯಾ? ಎಂದಿಗೂ ಇಲ್ಲ! ಎಂದಿಗೂ ಹುಟ್ಟದ, ನಾಶವಾಗದ ಆತ್ಮ ನಾನು. ನನಗೆ ಜನನ ಮರಣಗಳಿಲ್ಲ, ನಾನು ಅನಂತ, ಸರ್ವವ್ಯಾಪಿ, ಸರ್ವಜ್ಞ. ನೀನು ಹಸುಳೆ. ನನ್ನನ್ನು ಕೊಲ್ಲುವೆಯಾ!” ಎಂದನು. ಅದು ಶಕ್ತಿ! ಅದು ಶಕ್ತಿ! ನನ್ನ ಸ್ನೇಹಿತರೇ, ದೇಶ ಬಾಂಧವರೆ, ನಾನು ಉಪನಿಷತ್ತನ್ನು ಹೆಚ್ಚು ಹೆಚ್ಚಾಗಿ ಓದಿದಂತೆಲ್ಲಾ ನಿಮಗಾಗಿ ಮರುಗುವೆನು. ಅಲ್ಲಿ ಅನುಷ್ಠಾನಯೋಗ್ಯ ಸತ್ಯವಿದೆ. ಶಕ್ತಿ, ಶಕ್ತಿ ಬೇಕು ನಮಗೆ. ನಮಗೆ ಬೇಕಾಗಿರುವುದು ಶಕ್ತಿ. ಯಾರು ನಮಗೆ ಅದನ್ನು ನೀಡಬಲ್ಲರು? ನಮ್ಮನ್ನು ದುರ್ಬಲರನ್ನಾಗಿ ಮಾಡುವುದಕ್ಕೆ ಎಷ್ಟೋ ಮಂದಿ ಇರುವರು, ಎಷ್ಟೋ ಕಂತೆ ಪುರಾಣಗಳಿವೆ. ಪ್ರತಿಯೊಂದು ಪುರಾಣವನ್ನೂ ನಾವು ಹುಡುಕಿದರೆ ಪ್ರಪಂಚದ ಮುಕ್ಕಾಲು ಪಾಲು ಪುಸ್ತಕ ಭಂಡಾರವನ್ನು ಹಿಡಿಸುವಷ್ಟು ಕಥೆಗಳು ದೊರಕುವುವು. ಕಳೆದ ಒಂದು ಸಾವಿರ ವರ್ಷಗಳಿಂದ ನಮ್ಮ ಜನಾಂಗವನ್ನು ದುರ್ಬಲರನ್ನಾಗಿ ಮಾಡುವುದನ್ನೆಲ್ಲ ನಾವು ಪಡೆದಿರುವೆವು. ಆ ಸಮಯದಲ್ಲಿ ನಮ್ಮ ಜನಾಂಗದ ಗುರಿ ಇದೊಂದು ಮಾತ್ರ ಆಗಿತ್ತು – ಅದೇ ನಮ್ಮನ್ನು ಹೆಚ್ಚು ಹೆಚ್ಚು ದುರ್ಬಲರನ್ನಾಗಿ ಮಾಡಿ, ನಮ್ಮನ್ನು ತುಳಿಯಲು ಬರುವವರು ಪದತಳದಲ್ಲಿ ಹರಿದಾಡುವ ಎರೆಹುಳುಗಳಂತೆ ಮಾಡುವುದು. ಅದಕ್ಕಾಗಿಯೇ ನನ್ನ ಸ್ನೇಹಿತರೆ, ಸಹೋದರರೇ, ನಿಮ್ಮ ರಕ್ತ ಸಂಬಂಧಿಯಾಗಿ ನಿಮ್ಮೊಂದಿಗೆ ಬಾಳಿ ಅಳಿಯುವ ಒಬ್ಬನಂತೆ ನಿಮಗೆ ಹೇಳುತ್ತೇನೆ: ನಮಗೆ ಬೇಕಾಗಿರುವುದು ಶಕ್ತಿ, ಶಕ್ತಿ, ಪ್ರತಿ ಸಲವೂ ಶಕ್ತಿ. ಉಪನಿಷತ್ತುಗಳು ಶಕ್ತಿಯ ಮಹಾಗಣಿ. ಜಗತ್ತನ್ನೆಲ್ಲ ಜಾಗ್ರತಗೊಳಿಸುವಷ್ಟು ಶಕ್ತಿ ಅಲ್ಲಿದೆ. ಜಗತ್ತಿಗೆ ಅದರ ಮೂಲಕ ಜೀವದಾನ ಮಾಡಬಹುದು, ಶಕ್ತಿಯನ್ನು ನೀಡಬಹುದು, ಸ್ಫೂರ್ತಿಯನ್ನು ನೀಡಬಹುದು. ಜಗತ್ತಿಗೆ ಸೇರಿದ ಎಲ್ಲಾ ದೇಶಗಳ ಕೋಮುಗಳ ದುರ್ಬಲರಿಗೆ, ದುಃಖಿಗಳಿಗೆ, ದಬ್ಬಾಳಿಕೆಗೆ ತುತ್ತಾದವರಿಗೆ ತಮ್ಮ ಕಾಲಿನ ಮೇಲೆ ನಿಂತು ಮುಕ್ತರಾಗಿ ಎಂದು ಉಚ್ಚ\break ಕಂಠದಿಂದ ಸಾರುತ್ತದೆ ಉಪನಿಷತ್ತು. ದೈಹಿಕ ಸ್ವಾತಂತ್ರ್ಯ, ಮಾನಸಿಕ ಸ್ವಾತಂತ್ರ್ಯ, ಆಧ್ಯಾತ್ಮಿಕ ಸ್ವಾತಂತ್ರ್ಯವೇ ಉಪನಿಷತ್ತಿನ ಮೂಲಮಂತ್ರ.

ಈ ಒಂದು ಶಾಸ್ತ್ರ ಮಾತ್ರ ಜಗತ್ತಿನಲ್ಲಿ ಮೋಕ್ಷವನ್ನು ಕುರಿತು ಹೇಳದೆ ಸ್ವಾತಂತ್ರ್ಯವನ್ನು ಕುರಿತು ಹೇಳುತ್ತದೆ. ಪ್ರಕೃತಿಬಂಧನದಿಂದ ಪಾರಾಗಿ, ದೌರ್ಬಲ್ಯದಿಂದ ಪಾರಾಗಿ. ಈ ಸ್ವಾತಂತ್ರ್ಯ ಆಗಲೇ ನಿಮ್ಮಲ್ಲಿದೆ ಎಂದು ತೋರಿಸಿಕೊಡುತ್ತದೆ. ಉಪನಿಷತ್ತಿನ ಸಂದೇಶದ ಇನ್ನೊಂದು ವೈಶಿಷ್ಟ್ಯ ಇದು. ಒಬ್ಬನು ದ್ವೈತಿಯಾಗಿರಬಹುದು, ಚಿಂತೆಯಿಲ್ಲ. ಆದರೆ ಆತ್ಮವು ಸ್ವಭಾವತಃ ಪರಿಪೂರ್ಣವೆಂದು ಅವನು ಒಪ್ಪಲೇಬೇಕು. ಅದು ಕೆಲವು ಕ್ರಿಯೆಗಳಿಂದ ಸಂಕುಚಿತವಾಗುವುದು ಅಷ್ಟೆ. ರಾಮಾನುಜಾಚಾರ್ಯರ ಸಂಕೋಚ–ವಿಕಾಸವಾದವು\break ಆಧುನಿಕ ವಿಕಾಸವಾದದಂತೆ ಮತ್ತು ಪುನರಾವರ್ತನವಾದದಂತೆ \enginline{(atavism)} ಇದೆ. ಜೀವವು ಸಂಕುಚಿತವಾಗುವುದು, ಅದರ ಶಕ್ತಿ ಸುಪ್ತವಾಗುವುದು. ಪುನಃ ಒಳ್ಳೆಯ ಕರ್ಮದಿಂದ ಮತ್ತು ಒಳ್ಳೆಯ ಆಲೋಚನೆಯಿಂದ ವಿಕಾಸವಾಗಿ ತನ್ನ ಸ್ವಾಭಾವಿಕ ಪರಿಪೂರ್ಣತೆಯನ್ನು ಪ್ರಕಾಶಗೊಳಿಸುವುದು. ಆದರೆ ಅದ್ವೈತಿಗಳು ಪ್ರಕೃತಿಯಲ್ಲಿ ಮಾತ್ರ ವಿಕಾಸವನ್ನು ಒಪ್ಪುವರು, ಆತ್ಮನಲ್ಲಿ ಅಲ್ಲ. ಎದುರಿಗೆ ಒಂದು ತೆರೆ ಇದೆ, ಅದರಲ್ಲಿ ಒಂದು ಸಣ್ಣ ರಂಧ್ರವಿದೆ ಎಂದು ತಿಳಿದುಕೊಳ್ಳೋಣ. ನಾನು ತೆರೆಯ ಹಿಂದೆ ನಿಂತುಕೊಂಡು ರಂಧ್ರದ ಮೂಲಕ ಜನಸಂದಣಿಯನ್ನು ನೋಡುತ್ತಿರುವೆನು. ನನಗೆ ಕೆಲವರ ಮುಖ ಮಾತ್ರ ಕಾಣುತ್ತದೆ. ರಂಧ್ರವು ದೊಡ್ಡದಾಗುತ್ತಾ ಹೋದಂತೆ ನಾನು ಹೆಚ್ಚು ಹೆಚ್ಚು ಜನರನ್ನು ನೋಡಬಹುದು. ಯಾವಾಗ ರಂಧ್ರವು ತೆರೆಯಷ್ಟೇ ದೊಡ್ಡದಾಗುವುದೊ ಆಗ ಜನಸಂದಣಿಯೆಲ್ಲಾ ಕಾಣುವುದು. ಆಗ ನನಗೂ ನಿಮಗೂ ಮಧ್ಯದಲ್ಲಿ ಏನೂ ಇರುವುದಿಲ್ಲ. ನೀವು ಬದಲಾಗಲಿಲ್ಲ, ನಾನೂ ಬದ\-ಲಾಗಲಿಲ್ಲ. ಬದಲಾವಣೆಯೆಲ್ಲ ತೆರೆಯಲ್ಲಿ. ನೀವು ಮೊದಲಿನಿಂದ ಕೊನೆಯವರೆಗೂ ಹಾಗೆಯೇ ಇದ್ದಿರಿ. ತೆರೆ ಮಾತ್ರ ಬದಲಾಯಿತು. ವಿಕಾಸದ ಬಗ್ಗೆ ಅದ್ವೈತಿಯ ಅಭಿಪ್ರಾಯವಿದು: ಪ್ರಕೃತಿಯ ವಿಕಾಸ, ಆತ್ಮದ ಅಭಿವ್ಯಕ್ತಿ. ಆತ್ಮವು ಯಾವ ಕಾರಣದಿಂದಲೂ ಸಂಕೋಚವಾಗುವುದಿಲ್ಲ, ಅದು ಬದಲಾಗುವುದಿಲ್ಲ. ಅದು ಅನಂತ. ಮಾಯಾ ತೆರೆಯಿಂದ ಆವೃತವಾದಂತೆ ಇತ್ತು. ಆ ತೆರೆಯು ಬರುಬರುತ್ತಾ ತೆಳುವಾಗುತ್ತಾ ಬಂದಂತೆ, ಆತ್ಮನ ಸ್ವಭಾವಸಿದ್ಧ ಮಹಿಮೆ ಹೆಚ್ಚು ಹೆಚ್ಚು ವ್ಯಕ್ತವಾಗುತ್ತಾ ಬರುತ್ತದೆ. ಜಗತ್ತು ಭರತಖಂಡದಿಂದ ಕಲಿಯಬೇಕಾದ ಒಂದು ಮಹಾಸಿದ್ಧಾಂತ ಇದು. ಇತರರು ಏನನ್ನಾದರೂ ಮಾತನಾಡಲಿ, ಎಷ್ಟೇ ಜಂಭ ಕೊಚ್ಚಿಕೊಳ್ಳಲಿ. ಯಾವ ಸಮಾಜವೂ ಇದನ್ನು ಒಪ್ಪಿಕೊಳ್ಳದೆ ಇರಲು ಸಾಧ್ಯವಿಲ್ಲವೆಂಬುದನ್ನು ಕ್ರಮೇಣ ಅವರು ತಿಳಿಯುತ್ತಾರೆ. ಈಗ ಎಲ್ಲಾ ಕ್ಷೇತ್ರಗಳಲ್ಲಿಯೂ ಎಂತಹ ಮಹತ್ತರ ಬದಲಾವಣೆ ಆಗುತ್ತಿದೆ ಎಂಬುದು ನಿಮಗೆ ಗೊತ್ತಿಲ್ಲವೆ? ಪ್ರತಿಯೊಂದನ್ನೂ ಅದು ಒಳ್ಳೆಯದೆಂದು ಗೊತ್ತಾಗುವವರೆಗೆ ಅದು ಕೆಟ್ಟದ್ದು ಎಂದು ತಿಳಿಯುವುದು ರೂಢಿಯಾಗಿದ್ದುದು ನಿಮಗೆ ಗೊತ್ತಿಲ್ಲವೆ? ವಿದ್ಯಾಭ್ಯಾಸದಲ್ಲಿ, ಅಪರಾಧಿಗಳನ್ನು ಶಿಕ್ಷಿಸುವುದರಲ್ಲಿ, ಹುಚ್ಚರಿಗೆ ಮತ್ತು ರೋಗಿಗಳಿಗೆ ಚಿಕಿತ್ಸೆ ಮಾಡುವುದರಲ್ಲಿ ಇದೇ ನಿಯಮವನ್ನು ಅನುಸರಿಸುತ್ತಿದ್ದರು. ಆಧುನಿಕ ನಿಯಮವು ದೇಹವು ಸ್ವಭಾವತಃ ಆರೋಗ್ಯವಾಗಿರುವುದು ಅದು ತಾನೇ ರೋಗವನ್ನು ಪರಿಹರಿಸಿಕೊಳ್ಳುವುದು ಎಂದು ಹೇಳುತ್ತದೆ. ಔಷಧಿ ದೇಹದ ಶಕ್ತಿಯನ್ನು ಸಂರಕ್ಷಿಸುವುದು, ಅಷ್ಟೆ. ಅಪರಾಧಿಗಳ ವಿಷಯದಲ್ಲಿ ಅದು ಏನು ಹೇಳುವುದು? ಮಾನವ ಎಂತಹ ದುಷ್ಟನಾಗಿರಲಿ, ಅವನ ಅಂತರಾಳದಲ್ಲಿ ಎಂದಿಗೂ ಬದಲಾಗದ ದೈವತ್ವವಿದೆ. ನಾವು ಅಪರಾಧಿಗಳನ್ನು ಈ ರೀತಿ ನೋಡಿಕೊಳ್ಳಬೇಕು. ಆದ್ದರಿಂದಲೇ ಸುಧಾರಣಾಲಯ \enginline{(reformatory)} ಪಶ್ಚಾತ್ತಾಪಾಲಯ \enginline{(pentitentiary)} ಇತ್ಯಾದಿಗಳನ್ನು ತೆರೆಯುತ್ತಿರುವರು. ಇದರಂತೆಯೇ ಪ್ರತಿಯೊಂದು ಕ್ಷೇತ್ರದಲ್ಲಿಯೂ ಕೂಡ ಪರಿವರ್ತನೆ ಆಗಿದೆ. ಪ್ರತಿಯೊಂದು ಜೀವಿಯಲ್ಲಿಯೂ ದೈವತ್ವವಿದೆ ಎಂಬ ಭಾರತೀಯ ಭಾವನೆ ತಿಳಿದೊ ತಿಳಿಯದೆಯೊ ಜಗತ್ತಿನ ಇತರ ಕಡೆಗಳಲ್ಲಿ ವ್ಯಕ್ತವಾಗುತ್ತಿದೆ. ನಿಮ್ಮ ಶಾಸ್ತ್ರಗಳಲ್ಲಿ ಬೇರೆ ರಾಷ್ಟ್ರಗಳು ಒಪ್ಪಿಕೊಳ್ಳಲೇಬೇಕಾದ ವಿವರಣೆ ದೊರೆಯುತ್ತದೆ. ಒಬ್ಬರು ಮತ್ತೊಬ್ಬರನ್ನು ನೋಡುವ ರೀತಿಯೇ ಬದಲಾಗುತ್ತದೆ. ಯಾವಾಗಲೂ ಮಾನವನ ದೌರ್ಬಲ್ಯವನ್ನೇ ಎತ್ತಿತೋರಿಸುವ ಹಳೆಯ ಭಾವನೆ ತೊಲಗಬೇಕು. ಈ ಶತಮಾನದಲ್ಲಿ ಅವುಗಳೆಲ್ಲ ನಾಶವಾಗಬೇಕು. ಈಗ ಜನರು ನಮ್ಮನ್ನು ಟೀಕಿಸಬಹುದು. ಪಾಪವೇ ಇಲ್ಲ ಎಂಬ ರಾಕ್ಷಸೀ ಭಾವನೆಯನ್ನು ಬೋಧಿಸುತ್ತಿರುವನು ಎಂದು ಜಗತ್ತಿನಾದ್ಯಂತವೂ ಜನರು ನನ್ನನ್ನು ಟೀಕಿಸುತ್ತಿದ್ದಾರೆ. ಒಳ್ಳೆಯದು, ಮುಂದೆ ಅವರ ವಂಶಜರೇ, ನಾನು ಅಧರ್ಮವನ್ನು ಬೋಧಿಸಿದವನಲ್ಲ, ಧರ್ಮವನ್ನು ಬೋಧಿಸಿದವನು ಎಂದು ನನ್ನನ್ನು ಆಶೀರ್ವದಿಸುತ್ತಾರೆ. ನಾನು ಧರ್ಮ ಬೋಧಕ, ಅಧರ್ಮ ಬೋಧಕನಲ್ಲ. ನಾನು ಅಜ್ಞಾನ ಬೋಧಕನಲ್ಲ, ಜ್ಞಾನ ಬೋಧಕನೆಂದು ಹೆಮ್ಮೆ ಪಡುತ್ತೇನೆ.

ನಮ್ಮ ಉಪನಿಷತ್ತುಗಳಿಂದ ಜಗತ್ತು ನಿರೀಕ್ಷಿಸುತ್ತಿರುವ ಎರಡನೆಯ ಮಹಾ ಭಾವನೆಯೇ ವಿಶ್ವದ ಅಖಂಡತ್ವ. ಪ್ರಾಚೀನಕಾಲದ ಎಲ್ಲೆಯ ಗೆರೆ ಮತ್ತು ಭೇದ ಭಾವನೆಗಳು ಕ್ಷಿಪ್ರವಾಗಿ ಮಾಯವಾಗುತ್ತಿವೆ. ವಿದ್ಯುತ್ತು ಮತ್ತು ಉಗಿಯಶಕ್ತಿ ಜಗತ್ತಿನ ಬೇರೆ ಬೇರೆ ಭಾಗಗಳಿಗೆ ಸಂಬಂಧವನ್ನು ಏರ್ಪಡಿಸುತ್ತಿವೆ. ಇದರ ಪರಿಣಾಮವಾಗಿ ಹಿಂದೂಸ್ಥಾನದ ಹೊರಗೆಲ್ಲಾ ರಾಕ್ಷಸರು, ಭೂತಪ್ರೇತಗಳು ಇವೆ ಎಂದು ಹಿಂದೂಗಳು ಹೇಳುವುದಿಲ್ಲ. ಅಥವಾ ಕ್ರೈಸ್ತರು, ಇಂಡಿಯಾ ದೇಶದವರೆಲ್ಲಾ ನರಮಾಂಸ ಭಕ್ಷಕರು, ಕಾಡು ಜನರು ಎಂದು ಹೇಳುವುದಿಲ್ಲ. ನಮ್ಮ ದೇಶದಿಂದ ಹೊರಗೆ ಹೋದರೆ ಅಲ್ಲೂ ನಮ್ಮ ಸಹೋದರ ಮಾನವನನ್ನೇ ನೋಡುವೆವು. ಅಲ್ಲಿಯೂ ನಮ್ಮ ಸಹಾಯಕ್ಕೆ ಅನುಕಂಪದಿಂದ ತುಂಬಿದ ಹೃದಯ ಕಾದಿದೆ, ನಮಗೆ ಮಂಗಳವನ್ನು ಕೋರುವರು. ಕೆಲವು ವೇಳೆ ನಮ್ಮ ದೇಶದವರಿಗಿಂತ ಅವರು ಒಳ್ಳೆಯವರು. ಅವರು ಇಲ್ಲಿಗೆ ಬಂದರೆ, ಅದೇ ಸಹೋದರತ್ವ ಪ್ರೋತ್ಸಾಹ ಆಶೀರ್ವಾದ ಅವರಿಗೆ ಕಾದಿವೆ. ನಮ್ಮ ದುಃಖಕ್ಕೆ ಕಾರಣ ಅಜ್ಞಾನವೆಂದು ಉಪನಿಷತ್ತುಗಳು ಸಾರುತ್ತವೆ. ಸಾಮಾಜಿಕವಾಗಲೀ ಆಧ್ಯಾತ್ಮಿಕವಾಗಲೀ, ಎಲ್ಲ ಕ್ಷೇತ್ರದಲ್ಲಿಯೂ ಇದು ಸತ್ಯ. ಒಬ್ಬರು ಮತ್ತೊಬ್ಬರನ್ನು ದ್ವೇಷಿಸುವಂತೆ ಮಾಡುವುದು ಅಜ್ಞಾನ. ಅಜ್ಞಾನದಿಂದಾಗಿ ನಾವು ಮತ್ತೊಬ್ಬರನ್ನು ತಿಳಿಯಲಾರೆವು, ನಾವು ಮತ್ತೊಬ್ಬರನ್ನು ಪ್ರೀತಿಸಲಾರೆವು. ಪರಸ್ಪರ ಪರಿಚಯವಾದಮೇಲೆ ಪ್ರೀತಿ ಉದಯಿಸುವುದು. ಅದು ಬರಲೇಬೇಕು. ಏಕೆಂದರೆ ನಾವೆಲ್ಲ ಒಂದೇ ಅಲ್ಲವೆ? ಅಖಂಡತ್ವ ತಾನಾಗಿಯೇ ಬರುವುದನ್ನು ನಾವು ಕಾಣುವೆವು. ಸಾಮಾಜಿಕ ಮತ್ತು ರಾಜಕೀಯ ಕ್ಷೇತ್ರಗಳಲ್ಲಿಯೂ ಇಪ್ಪತ್ತು ವರ್ಷಗಳ ಹಿಂದೆ ಕೇವಲ ಒಂದೇ ದೇಶದ ಸಮಸ್ಯೆಯಾಗಿದ್ದುದನ್ನು ಈಗ ಕೇವಲ ಆ ದೇಶದ ದೃಷ್ಟಿಯಿಂದಲೇ ಬಗೆಹರಿಸಲಾಗುವುದಿಲ್ಲ. ಅವು ಮಹತ್ತಾಗುತ್ತಿರುವುವು, ಭೀಮಾಕಾರವನ್ನು ತಾಳುತ್ತಿವೆ. ಅವನ್ನು ಅಂತರಾಷ್ಟ್ರೀಯ ದೃಷ್ಟಿಯಿಂದ ಮಾತ್ರ ಪರಿಹರಿಸಬಹುದು. ವಿಶ್ವಸಂಸ್ಥೆ, ವಿಶ್ವಯೋಜನೆ, ವಿಶ್ವಕಾನೂನು ಇವು ಈ ಕಾಲಕ್ಕೆ ಆವಶ್ಯಕವಾಗಿವೆ. ಇವು ನಮ್ಮ ಅಖಂಡತ್ವವನ್ನು ತೋರುವುವು. ವಿಜ್ಞಾನದಲ್ಲಿಯೂ ಪ್ರತಿದಿನವೂ ಭೌತದ್ರವ್ಯದ ಸಂಬಂಧದಲ್ಲಿ ವಿಶಾಲವಾದ ದೃಷ್ಟಿ ಬೆಳೆಯುತ್ತಿದೆ. ಜಗತ್ತೆಲ್ಲ ಒಂದು ರಾಶಿ, ಜಡವಸ್ತುವಿನ ಒಂದು ಮಹಾಸಾಗರ \enginline{(ocean of matter)} ಎಂದು ಹೇಳಬಹುದು. ಅಲ್ಲಿ ನಾನು, ನೀವು ಸೂರ್ಯ ಚಂದ್ರರೆಲ್ಲ ಹಲವು ಸುಳಿಗಳ ಹೆಸರುಗಳಲ್ಲದೆ ಬೇರೆಯಲ್ಲ. ಮಾನಸಿಕ ದೃಷ್ಟಿಯಿಂದ ನೋಡಿದರೆ ಇದೊಂದು ಆಲೋಚನೆಯ ಮಹಾಸಾಗರ. ಇದರಲ್ಲಿ ನಾನು ನೀವುಗಳೆಲ್ಲ ಅಲೆಗಳು. ಆಧ್ಯಾತ್ಮಿಕ ದೃಷ್ಟಿಯಿಂದ ನೋಡಿದರೆ ಇದು ಅಚಲವಾದುದು, ಬದಲಾಗುವುದಿಲ್ಲ. ಇದು ಏಕವಾದ, ಬದಲಾಗದ, ಅಖಂಡವಾದ ಆತ್ಮ. ನೀತಿಯ ಆವಶ್ಯಕತೆಯನ್ನು ಸಾರುವವರು ಅದಕ್ಕೆ ಪ್ರಮಾಣವನ್ನು ನಮ್ಮ ಶಾಸ್ತ್ರದಲ್ಲಿ ನೋಡಬಹುದು. ನೀತಿಯ ವಿವರಣೆ, ಅದರ ಮೂಲ ಈಗ ಜಗತ್ತಿಗೆ ಬೇಕಾಗಿದೆ. ಇದು ಕೂಡ ನಮ್ಮ ಶಾಸ್ತ್ರದಲ್ಲಿ ದೊರಕುವುದು.

\vskip   4pt

ಭರತಖಂಡದಲ್ಲಿ ಇರುವ ನಮಗೆ ಏನು ಬೇಕಾಗಿದೆ? ಹೊರಗಿನವರಿಗೆ ಈ ಭಾವನೆಗಳು ಆವಶ್ಯಕವಾಗಿದ್ದರೆ ನಮಗೆ ಇವು ಇಪ್ಪತ್ತರಷ್ಟು ಆವಶ್ಯಕವಿದೆ. ಏಕೆಂದರೆ ಉಪನಿಷತ್ತು ಎಷ್ಟೇ ಶ್ರೇಷ್ಠವಾಗಿದ್ದರೂ, ನಮ್ಮ ಪೂರ್ವಿಕರು ಕೀರ್ತಿಶೇಷರಾದ ಮಹಾಋಷಿಗಳಾಗಿದ್ದರೂ, ಇತರ ಅನೇಕ ಜನಾಂಗಗಳೊಂದಿಗೆ ಹೋಲಿಸಿದರೆ ನಾವು ದುರ್ಬಲರು, ಅತಿ ದುರ್ಬಲರಾಗಿರುವೆವು. ಮೊದಲು ನಮ್ಮ ಶಾರೀರಿಕ ದೌರ್ಬಲ್ಯ ಈ ಶಾರೀರಿಕ ದೌರ್ಬಲ್ಯವೇ ನಮ್ಮ ಕಷ್ಟದ ಮೂರನೆಯ ಒಂದು ಪಾಲಿಗಾದರೂ ಕಾರಣ. ನಾವು ಶುದ್ಧ ಸೋಮಾರಿಗಳು, ಕೆಲಸ ಮಾಡುವುದಿಲ್ಲ. ನಮ್ಮಲ್ಲಿ ಒಗ್ಗಟ್ಟಿಲ್ಲ. ಅನ್ಯರನ್ನು ಪ್ರೀತಿಸುವುದಿಲ್ಲ; ನಾವು ಬರಿಯ ಸ್ವಾರ್ಥಿಗಳು. ಪರಸ್ಪರರನ್ನು ದ್ವೇಷಿಸದೆ, ಅಸೂಯೆ ಪಡದೆ ಮೂರು ಜನರು ಕೂಡ ಒಟ್ಟಿಗೆ ಕಲೆಯಲಾರೆವು. ಇದೇ ನಮ್ಮ ಸ್ಥಿತಿ; ನಾವು ಚೆಲ್ಲಾಪಿಲ್ಲಿಯಾಗಿರುವ ದೊಂಬಿಯಂತೆ ಇರುವೆವು, ಅತ್ಯಂತ ಸ್ವಾರ್ಥಿಗಳಾಗಿರುವೆವು. ಹಣೆಯ ಮೇಲೆ ಮತ ಚಿಹ್ನೆಯನ್ನು ಹೀಗೆ ಇಡಬೇಕೆ, ಹಾಗೆ ಇಡಬೇಕೆ ಎಂದು ಬೇಕಾದಷ್ಟು ಹೋರಾಟವಾಗಿದೆ. ಒಬ್ಬನ ದೃಷ್ಟಿ ನನ್ನ ಆಹಾರವನ್ನು ಅಶುದ್ಧ ಮಾಡುವುದೆ ಇಲ್ಲವೆ ಎಂಬ ಗಹನ ವಿಷಯದ ಮೇಲೆ ಬೇಕಾದಷ್ಟು ಗ್ರಂಥಗಳನ್ನು ಬರೆದಿರುವರು! ಹಲವು ಶತಮಾನಗಳಿಂದ ಇದರಲ್ಲೇ ನಾವು ನಿರತರಾಗಿರುವೆವು. ಇಂತಹ ಅದ್ಭುತ ಅನ್ವೇಷಣೆಯಲ್ಲಿ ತಮ್ಮ ಪೂರ್ಣ ಮಾನಸಿಕ ಶಕ್ತಿಯನ್ನೆಲ್ಲಾ ವ್ಯರ್ಥಗೊಳಿಸಿದ ಜನಾಂಗದಿಂದ ಯಾವುದೇ ಉತ್ತಮವಾದುದನ್ನೂ ನಿರೀಕ್ಷಿಸಲಾರೆವು. ನಮಗೆ ನಾಚಿಕೆಯಾಗುವುದಿಲ್ಲವೆ?ಕೆಲವು ವೇಳೆ ಆಗುವುದು. ಆದರೆ ಇವನ್ನು ಲಘುವಾಗಿ ಎಣಿಸಿದರೂ ನಾವು ಇವನ್ನು ತೊರೆಯಲಾರೆವು. ಹಲವು ವಿಷಯಗಳನ್ನು ಆಲೋಚಿಸುತ್ತೇವೆ. ಕಾರ್ಯರೂಪಕ್ಕೆ ತರುವುದಿಲ್ಲ. ಗಿಳಿಯಂತೆ ಮಾತನಾಡುವುದೊಂದು ನಮ್ಮ ಬಾಳಿನ ಚಾಳಿಯಾಗಿದೆ. ಏನನ್ನೂ ಅನುಷ್ಠಾನಕ್ಕೆ ತರುವುದಿಲ್ಲ. ಇದಕ್ಕೆ ಕಾರಣವೇನು? ಶಾರೀರಿಕ ದುರ್ಬಲತೆ. ಇಂತಹ ದುರ್ಬಲ ಮಿದುಳು ಏನನ್ನೂ ಸಾಧಿಸಲಾರದು. ನಾವು ಅದನ್ನು ಪುಷ್ಟಿಗೊಳಿಸಬೇಕು. ಮೊದಲು ನಮ್ಮ ಯುವಕರು ಬಲಿಷ್ಠರಾಗಬೇಕು. ಅನಂತರ ಧರ್ಮ. ನನ್ನ ತರುಣ ಸ್ನೇಹಿತರೇ, ಮೊದಲು ಬಲಿಷ್ಠರಾಗಿ. ಇದು ನನ್ನ ಬುದ್ಧಿವಾದ. ಗೀತಾಧ್ಯಯನಕ್ಕಿಂತ ಪುಟಬಾಲ್​ ಆಟದಿಂದ ಮುಕ್ತಿಗೆ ಹತ್ತಿರ ಹೋಗುವಿರಿ. ಇವು ಕೆಚ್ಚಿನ ಮಾತುಗಳು. ಆದರೂ ನಾನು ಇದನ್ನು ನಿಮಗೆ ಹೇಳಬೇಕಾಗಿದೆ. ನಾನು ನಿಮ್ಮನ್ನು ಪ್ರೀತಿಸುತ್ತೇನೆ. ನಿಮ್ಮ ನ್ಯೂನತೆ ನನಗೆ ಗೊತ್ತಿದೆ. ನನಗೆ ಸ್ವಲ್ಪ ಅನುಭವವಾಗಿದೆ. ನಿಮ್ಮ ಬಾಹುಗಳ ಮಾಂಸಖಂಡದಲ್ಲಿ ಸ್ವಲ್ಪ ಹೆಚ್ಚು ಶಕ್ತಿ ಇದ್ದರೆ, ನೀವು ಗೀತೆಯನ್ನು ಇನ್ನೂ ಚೆನ್ನಾಗಿ ತಿಳಿದುಕೊಳ್ಳಬಲ್ಲಿರಿ. ನಿಮ್ಮಲ್ಲಿ ಹೆಚ್ಚು ಬಿಸಿ ರಕ್ತವಿದ್ದರೆ ಗೀತೆಯಲ್ಲಿ ಕೃಷ್ಣನ ಪ್ರತಿಭೆ ಮತ್ತು ಅದ್ಭುತ ಶಕ್ತಿ ಇವನ್ನು ತಿಳಿದುಕೊಳ್ಳಬಲ್ಲಿರಿ. ನಿಮ್ಮ ಕಾಲಿನ ಮೇಲೆ ನಿಮ್ಮ ದೇಹ ದೃಢವಾಗಿ ನಿಂತು, ನಿಮ್ಮನ್ನು ನೀವು ಪುರುಷರೆಂದು ಪರಿಗಣಿಸಿದಾಗ ಉಪನಿಷತ್ತಿನ ಮಹಿಮೆಯನ್ನು ಚೆನ್ನಾಗಿ ತಿಳಿದುಕೊಳ್ಳಬಲ್ಲಿರಿ, ಆತ್ಮನ ಮಹಿಮೆಯನ್ನು ಚೆನ್ನಾಗಿ ತಿಳಿದುಕೊಳ್ಳಬಲ್ಲಿರಿ. ಹೀಗೆ ನಾವು ಉಪನಿಷತ್ತಿನ ಭಾವನೆಗಳನ್ನು ನಮ್ಮ ಆವಶ್ಯಕತೆಗೆ ತಕ್ಕಂತೆ ಅನುಷ್ಠಾನಕ್ಕೆ ತರಬಹುದು.

\vskip   4pt

ಅದ್ವೈತವನ್ನೇ ನಾನು ಬೋಧಿಸುತ್ತೇನೆಂದು ಜನರಿಗೆ ನನ್ನ ಮೇಲೆ ಬೇಜಾರು. ದ್ವೈತವನ್ನಾಗಲೀ, ಅದ್ವೈತವನ್ನಾಗಲೀ, ಅಥವಾ ಮತ್ತಾವ ತತ್ತ್ವವನ್ನಾಗಲೀ ಬೋಧಿಸುವುದು ನನ್ನ ಗುರಿಯಲ್ಲ. ನಮಗೆ ಇಂದು ಬೇಕಾಗಿರುವುದೇ ಆತ್ಮನಿಗೆ ಸಂಬಂಧ ಪಟ್ಟ ಮಹಾಭಾವನೆ, ಅದರ ಅನಂತ ಶಕ್ತಿ, ಪರಿಶುದ್ಧತೆ ಮತ್ತು ಪರಿಪೂರ್ಣತೆ. ನನಗೆ ಒಂದು ಮಗುವಿದ್ದರೆ ಅದು ಹುಟ್ಟಿದಾಗಿನಿಂದಲೂ ಅದಕ್ಕೆ “ನೀನು ಪರಿಶುದ್ಧಾತ್ಮ” ಎಂದು ಹೇಳುತ್ತಿದ್ದೆ. ಪುರಾಣದಲ್ಲಿ ಬರುವ ರಾಣಿ ಮದಾಲಸೆಯ ಸುಂದರವಾದ ಕಥೆಯನ್ನು ನೀವು ಕೇಳಿರಬಹುದು. ಅವಳಿಗೆ ಮಗುವಾದ ತಕ್ಷಣವೇ ತೊಟ್ಟಿಲಲ್ಲಿ ಇಟ್ಟು ತೂಗುವಾಗ, “ನೀನು ಪರಿಶುದ್ದಾತ್ಮ, ಕಳಂಕರಹಿತ, ಪಾಪದೂರ, ಶಕ್ತಿಶಾಲಿ, ಪರಮಶ್ರೇಷ್ಠ” ಎಂದು ಹಾಡುತ್ತಿದ್ದಳು. ಅದರಲ್ಲಿ ತುಂಬಾ ಸತ್ಯವಿದೆ. ನೀವು ಶ್ರೇಷ್ಠರೆಂದು ಭಾವಿಸಿ, ಶ್ರೇಷ್ಠರಾಗುವಿರಿ. ಪ್ರಪಂಚವನ್ನೆಲ್ಲಾ ಪರ್ಯಟನೆ ಮಾಡಿ ನಾನು ಏನನ್ನು ಕಲಿತಿರುವೆನು? ತಾವು ಪಾಪಿಗಳೆಂದು ಅವರು ಹೇಳಿಕೊಳ್ಳಬಹುದು – ಆಂಗ್ಲೇಯರು ತಾವು ನಿಜವಾಗಿಯೂ ಪಾಪಿಗಳೆಂದು ನಂಬಿದ್ದರೆ ಮಧ್ಯ ಆಫ್ರಿಕಾದಲ್ಲಿರುವ ನೀಗ್ರೋ ಮೇಲಾಗುತ್ತಿರಲಿಲ್ಲ. ಸದ್ಯಕ್ಕೆ ಅವರು ನಂಬದೇ ಇರುವುದು ಒಳ್ಳೆಯದು. ಅದರ ಬದಲು ಆಂಗ್ಲೇಯನು ತಾನು ವಿಶ್ವಕ್ಕೆ ಒಡೆಯ ಎಂದು ಭಾವಿಸುವನು. ತಾನು ಮಹಾನ್​ ವ್ಯಕ್ತಿ ಜಗತ್ತಿನಲ್ಲಿ ಏನನ್ನು ಬೇಕಾದರೂ ಸಾಧಿಸಲು ಸಾಧ್ಯ ಎಂದು ಭಾವಿಸುವನು. ಬೇಕಾದರೆ ಸೂರ್ಯಲೋಕಕ್ಕೆ ಅಥವಾ ಚಂದ್ರಲೋಕಕ್ಕೆ ತಾನು ಹೋಗಬಲ್ಲೆನೆಂದು ಭಾವಿಸುವನು. ಇದೇ ಅವನನ್ನು ಪ್ರಖ್ಯಾತನನ್ನಾಗಿ ಮಾಡುವುದು. ತನ್ನ ಪಾದ್ರಿ ಹೇಳುವಂತೆ ಆವನು ತಾನೊಬ್ಬ ಪಾಪಿ, ಮುಂದೆ ಶಾಶ್ವತವಾಗಿ ನರಕದಲ್ಲಿ ಕೊಳೆಯಬೇಕಾಗುವುದೆಂದು ನಂಬಿದ್ದರೆ, ಆತ ಈಗಿನ ಆಂಗ್ಲೇಯನಾಗುತ್ತಿರಲಿಲ್ಲ. ಪ್ರತಿಯೊಂದು\break ದೇಶದಲ್ಲಿಯೂ ಹೀಗೆಯೇ. ಪಾದ್ರಿಗಳು ಏನೇ ಹೇಳಲಿ, ಮೂಢನಂಬಿಕೆಗಳು ಎಷ್ಟೇ ಇರಲಿ ಮಾನವನಲ್ಲಿರುವ ದೈವತ್ವ ಪ್ರಕಾಶಕ್ಕೆ ಬರುತ್ತಿರುವುದು. ನಾವು ಶ್ರದ್ಧೆಯನ್ನು ಕಳೆದುಕೊಂಡಿರುವೆವು. ಆಂಗ್ಲೇಯ ಸ್ತ್ರೀ ಪುರುಷರಲ್ಲಿ ಸಾವಿರದಲ್ಲಿ ಒಂದು ಪಾಲು ಶ್ರದ್ಧೆ ಕೂಡ ನಮ್ಮಲ್ಲಿಲ್ಲ ಎಂದರೆ ನಂಬುತ್ತೀರಾ! ಇದರಲ್ಲಿ ಉತ್ಪ್ರೇಕ್ಷೆ ಇಲ್ಲ. ಹೀಗೆ ಹೇಳಬೇಕಾಗಿದೆ, ವಿಧಿಯಿಲ್ಲ. ಆಂಗ್ಲೇಯ ಸ್ತ್ರೀ–ಪುರಷರಿಗೆ ನಮ್ಮ ಆದರ್ಶದ ಮೇಲೆ ಮನಸ್ಸಾದರೆ ಅದರಲ್ಲೇ ಹುಚ್ಚರಾಗಿ, ತಮ್ಮ ಜನರು ಅವರನ್ನು ಅಣಕಿಸುತ್ತಿರುವರು ಎಂಬುದನ್ನು ಕೂಡ ಗಮನಿಸದೆ, ಅವರೇ ನಮ್ಮನ್ನು ಆಳುತ್ತಿದ್ದಾಗ್ಯೂ, ನಮ್ಮ ದೇಶಕ್ಕೆ ಅದನ್ನು ಬೋಧಿಸಲು ಬರುವುದನ್ನು ನೋಡುವುದಿಲ್ಲವೆ? ನಿಮ್ಮಲ್ಲಿ ಎಷ್ಟು ಜನರು ಹಾಗೆ ಮಾಡಬಲ್ಲಿರಿ? ನೀವು ಏತಕ್ಕೆ ಹಾಗೆ ಮಾಡಬಾರದು? ನಿಮಗೆ ಆದು ತಿಳಿದಿಲ್ಲವೆ? ಅವರಿಗಿಂತ ನಿಮಗೆ ಹೆಚ್ಚು ತಿಳಿದಿದೆ. ಎಷ್ಟು ಬೇಕೋ ಅದಕ್ಕಿಂತ ಹೆಚ್ಚು ಬುದ್ಧಿ ನಿಮ್ಮಲ್ಲಿದೆ. ಅದೇ ಬಂದಿರುವ ಕಷ್ಟ. ಇದಕ್ಕೆ ಕಾರಣ ನಿಮ್ಮ ರಕ್ತ ಬರಿಯ ನೀರಿನಂತಿರುವುದು, ನಿಮ್ಮ ಮಿದುಳು ಜಡವಾಗಿದೆ, ದೇಹ ದುರ್ಬಲವಾಗಿದೆ. ದೇಹವನ್ನು ಬಲಗೊಳಿಸಬೇಕು. ಶಾರೀರಕ ದುರ್ಬಲತೆಯೇ ಇದಕ್ಕೆ ಕಾರಣ, ಬೇರೆಯಲ್ಲ. ಕಳೆದ ನೂರು ವರ್ಷಗಳಿಂದ ಸುಧಾರಣೆ, ಆದರ್ಶ ಮುಂತಾದುವನ್ನು ಕುರಿತು ಮಾತನಾಡುತ್ತಿರುವಿರಿ. ಅನುಷ್ಠಾನದ ಸಮಯ ಬಂದರೆ ನಿಮ್ಮ ಸುಳಿವೇ ಇಲ್ಲ. ಇದರಿಂದ ನಿಮ್ಮ ಮೇಲೆ ಜಗತ್ತಿಗೆ ತಾತ್ಸಾರವುಂಟಾಗಿದೆ, ಸುಧಾರಣೆ ಎಂಬ ಪದವೇ ಹಾಸ್ಯಾಸ್ಪದವಾಗಿದೆ. ಇದಕ್ಕೆ ಕಾರಣವೇನು? ನಿಮಗೆ ಗೊತ್ತಿಲ್ಲವೆ? ಅದು ನಿಮಗೆ ಚೆನ್ನಾಗಿ ಗೊತ್ತಿದೆ. ನೀವು ದುರ್ಬಲರು, ದುರ್ಬಲರು, ದುರ್ಬಲರು–ಇದೇ ಇದಕ್ಕೆ ಮೂಲಕಾರಣ. ನಿಮ್ಮ ದೇಹ ದುರ್ಬಲ, ಮನಸ್ಸು ದುರ್ಬಲ, ನಿಮ್ಮಲ್ಲಿ ಆತ್ಮಶ್ರದ್ಧೆ ಇಲ್ಲ. ನನ್ನ ಸಹೋದರರೇ, ಶತಶತಮಾನಗಳಿಂದಲೂ, ಕಳೆದ ಸಾವಿರ ವರುಷಗಳಿಂದಲೂ, ಜಾತಿ ಭಾವನೆ, ರಾಜರು, ಪರದೇಶೀಯರ ಮತ್ತು ಸ್ವಜನರ ದಬ್ಬಾಳಿಕೆ, ಇವುಗಳಿಂದ ನಿಮ್ಮ ಸತ್ವವೆಲ್ಲಾ ನಾಶವಾಗಿರುವುದು! ನಿಮಗೆ ಬೆನ್ನೆಲುಬೇ ಇಲ್ಲ. ನೀವು ಹರಿದಾಡುತ್ತಿರುವ ಕ್ರಿಮಿಕೀಟಗಳಂತೆ. ಯಾರು ನಿಮಗೆ ಶಕ್ತಿಯನ್ನು ನೀಡಬಲ್ಲರು? ಶಕ್ತಿ, ಶಕ್ತಿಯೇ ನಿಮಗೆ ಬೇಕಾಗಿರುವುದು. ಇದನ್ನೇ ನಾನು ಒತ್ತಿ ಹೇಳುತ್ತೇನೆ. ಶಕ್ತಿಯನ್ನು ಪಡೆಯಬೇಕಾದರೆ ಮೊದಲನೆಯ ಹೆಜ್ಜೆಯೇ ಉಪನಿಷತ್ತಿನ ತಳಪಾಯದ ಮೇಲೆ ನಿಂತು “ನಾನು ಆತ್ಮ” ಎಂದು ನಂಬುವುದು. “ನನ್ನನ್ನು ಯಾವ ಕತ್ತಿಯೂ ಕತ್ತರಿಸಲಾರದು, ಯಾವ ಶಸ್ತ್ರವೂ ತಿವಿಯಲಾರದು, ಬೆಂಕಿ ದಹಿಸಲಾರದು, ಗಾಳಿ ಒಣಗಿಸಲಾರದು. ನಾನು ಸರ್ವಶಕ್ತ, ಸರ್ವಜ್ಞ” ಎಂದು ನಮ್ಮನ್ನು ಉದ್ಧರಿಸುವ ಇಂತಹ ಮಂತ್ರಗಳನ್ನು ಪಠಿಸಿ. ನಾವು ದುರ್ಬಲರೆಂದು ಹೇಳಬೇಡಿ. ನಾವು ಏನನ್ನು ಬೇಕಾದರೂ ಮಾಡಬಲ್ಲೆವು. ನಮಗೆ ಮಾಡಲು ಅಸಾಧ್ಯವಾಗಿ ಇರುವುದು ಯಾವುದಿದೆ? ಏನನ್ನುಬೇಕಾದರೂ ಸಾಧಿಸಬಲ್ಲೆವು. ನಮ್ಮಲ್ಲಿ ಎಲ್ಲರಲ್ಲಿಯೂ ಪರಂಜ್ಯೋತಿಯಾದ ಆತ್ಮನಿರುವನು. ಇದನ್ನು ನಂಬೋಣ. ನಚಿಕೇತನಿಗಿದ್ದಂತಹ ಶ್ರದ್ಧೆ ಇರಲಿ. ತಂದೆ ಯಜ್ಞ ಮಾಡುತ್ತಿದ್ದಾಗ ನಚಿಕೇತನಿಗೆ ಶ್ರದ್ಧೆ ಬಂತು. ನಿಮ್ಮಲ್ಲಿ ಪ್ರತಿಯೊಬ್ಬರಿಗೂ ಅಂತಹ ಶ್ರದ್ಧೆ ಬರಲಿ. ವೀರರಂತೆ, ಧೀರರಂತೆ ನಿಂತು, ಪ್ರಚಂಡ ಬುದ್ಧಿಶಕ್ತಿಯಿಂದ ದೇವರಂತೆ ನೀವು ಜಗಚ್ಚಾಲಕರಾಗಿ ಎಂದು ನಾನು ಆಶಿಸುತ್ತೇನೆ. ಉಪನಿಷತ್ತಿನಿಂದ ದೊರಕುವ ಶಕ್ತಿ ಇದು, ಶ್ರದ್ಧೆ ಇದು.

ಉಪನಿಷತ್ತುಗಳು ಕೇವಲ ಸಂನ್ಯಾಸಿಗೆ ಮಾತ್ರವೇನು! ಅವುಗಳು ರಹಸ್ಯ ವಿದ್ಯೆಯೇ? ಪ್ರಾಚೀನ ಕಾಲದಲ್ಲಿ ಇದು ಕೇವಲ ಸಂನ್ಯಾಸಿಗೆ ಮೀಸಲಾದ ಶಾಸ್ತ್ರವಾಗಿತ್ತು. ಅವನು ಅರಣ್ಯವಾಸಿಯಾದನು. ಶಂಕರಾಚಾರ್ಯರು ಕೊಂಚ ದಯೆ ತಾಳಿ ಗೃಹಸ್ಥರೂ ಉಪನಿಷತ್ತನ್ನು ಓದಬಹುದು, ಅದರಿಂದ ಅವರಿಗೆ ಅಪಾಯವಿಲ್ಲ ಎಂದು ಸಾರಿದರು. ಆದರೂ ಉಪ\-ನಿಷತ್ತು ಸಂನ್ಯಾಸಿಯ ಅರಣ್ಯ ಜೀವನವನ್ನು ಕುರಿತು ಹೇಳುತ್ತದೆ ಎಂಬ ಅಭಿಪ್ರಾಯ ಉಂಟಾಗಿದೆ. ನಾನು ನಿಮಗೆ ಹೇಳಿದಂತೆ, ವೇದಗಳ ಮೇಲೆ ಅಧಿಕಾರಯುತವಾದ ಅಸದೃಶವಾದ ಭಾಷ್ಯವನ್ನು, ಆ ವೇದಗಳನ್ನು ಪ್ರಚೋದಿಸಿದ ಶ‍್ರೀಕೃಷ್ಣನೇ ಗೀತೆಯಲ್ಲಿ ಮಾಡಿರುವನು. ಜಗತ್ತಿನ ಎಲ್ಲಾ ಕಾರ್ಯಕ್ಷೇತ್ರಗಳಲ್ಲಿರುವ ಸರ್ವರಿಗೂ ಉಪನಿಷತ್ತಿನ ಭಾವನೆಗಳು ಅನ್ವಯಿಸುವುವು. ಈ ವೇದಾಂತದ ಭಾವನೆಗಳು ಹೊರಗೆ ಬರಬೇಕು. ಅವು ಕೇವಲ ಅರಣ್ಯದಲ್ಲಿ ಮತ್ತು ಗಿರಿ ಗುಹೆಗಳಲ್ಲಿ ಮಾತ್ರ ಇರುವುದಲ್ಲ. ನ್ಯಾಯಾಸ್ಥಾನದಲ್ಲಿ, ಪ್ರಾರ್ಥನಾಲಯದಲ್ಲಿ, ಬಡವನ ಕುಟೀರದಲ್ಲಿ ವೇದಾಂತ ರೂಢಿಗೆ ಬರಬೇಕು. ಮೀನನ್ನು ಹಿಡಿಯುವ ಬೆಸ್ತರಿಗೂ, ಅಧ್ಯಯನ ಮಾಡುತ್ತಿರುವ ವಿದ್ಯಾರ್ಥಿಗಳಿಗೂ, ವೇದಾಂತ ಸಹಾಯಕ್ಕೆ ಬರಬೇಕು. ಸ್ತ್ರೀ ಪುರುಷ ಬಾಲಕರೆಲ್ಲರೂ, ಅವರು ಎಲ್ಲಿಯೇ ಇರಲಿ, ಯಾವ ಕೆಲಸವನ್ನಾದರೂ ಮಾಡುತ್ತಿರಲಿ, ಎಲ್ಲರಿಗೂ ಇದು ಅನ್ವಯಿಸುವುದು. ಇದರಲ್ಲಿ ಅಂಜಿಕೊಳ್ಳುವುದಕ್ಕೆ ಏನಿದೆ? ಬೆಸ್ತರು ಮುಂತಾದವರು ಹೇಗೆ ಉಪನಿಷತ್ತಿನ ಆದರ್ಶಗಳನ್ನು ಅನುಷ್ಠಾನಕ್ಕೆ ತರಬಲ್ಲರು? ಅದಕ್ಕೆ ಮಾರ್ಗವನ್ನು ತೋರಿರುವರು. ಅದು ಅನಂತವಾಗಿದೆ, ಧರ್ಮ ಅನಂತವಾಗಿದೆ. ಯಾರೂ ಅದನ್ನು ಮೀರಿಹೋಗಲಾರರು. ಯಾವುದನ್ನೇ ಆದರೂ ಪ್ರಾಮಾಣಿಕತೆಯಿಂದ ಮಾಡಿದರೆ ಅದೆಲ್ಲ ನಿಮಗೆ ಹಿತಕಾರಿ. ಅತ್ಯಂತ ಕನಿಷ್ಠವಾದುದನ್ನೂ ಚೆನ್ನಾಗಿ ಮಾಡಿದರೆ ಅದರಿಂದ ಅದ್ಭುತ ಫಲಪ್ರಾಪ್ತಿ. ಪ್ರತಿಯೊಬ್ಬರೂ ತಮಗೆ ಸಾಧ್ಯವಾದ ಮಟ್ಟಿಗೆ ಮಾಡಲಿ. ಬೆಸ್ತ ತಾನು ಆತ್ಮನೆಂದು ತಿಳಿದರೆ ಅವನು ಒಳ್ಳೆಯ ಬೆಸ್ತನಾಗುವನು. ವಿದ್ಯಾರ್ಥಿ ತಾನು ಆತ್ಮನೆಂದು ತಿಳಿದರೆ ಒಳ್ಳೆಯ ವಿದ್ಯಾರ್ಥಿಯಾಗುವನು. ವಕೀಲನು ತಾನು ಆತ್ಮನೆಂದು ತಿಳಿದರೆ ಆತ ಒಳ್ಳೆಯ ವಕೀಲನಾಗುವನು. ಇದರ ಪರಿಣಾಮವಾಗಿ ಜಾತಿ ಯಾವಾಗಲೂ ಉಳಿಯುವುದು. ವಿವಿಧ ಜಾತಿಗಳಾಗಿ ವಿಭಾಗವಾಗುವುದೇ ಸಮಾಜದ ಸ್ವಭಾವ. ಆದರೆ ವಿಶೇಷಹಕ್ಕುಗಳು ಮಾತ್ರ ಹೋಗುವುವು. ಜಾತಿ ಪ್ರಕೃತಿ ನಿರ್ಮಿತ. ನಾನು ಸಮಾಜದಲ್ಲಿ ಒಂದು ಕರ್ತವ್ಯವನ್ನು ಮಾಡಬಹುದು, ನೀವು ಮತ್ತೊಂದು ಕರ್ತವ್ಯವನ್ನು ಮಾಡಬಹುದು. ನೀವೊಂದು ರಾಜ್ಯವನ್ನಾಳಬಹುದು. ನಾನೊಂದು ಪಾದರಕ್ಷೆಯನ್ನು ರಿಪೇರಿ ಮಾಡಬಹುದು. ಆದರೆ ಆದರಿಂದ ನೀವು ನನಗಿಂತ ಶ್ರೇಷ್ಠರೆಂದಾಗಲಿಲ್ಲ, ಏಕೆಂದರೆ ನೀವು ಪಾದರಕ್ಷೆಯನ್ನು ಹೊಲಿಯಬಲ್ಲಿರಾ? ನಾನು ರಾಜ್ಯವನ್ನು ಆಳಬಲ್ಲನೇ? ನಾನು ಪಾದರಕ್ಷೆ ಹೊಲಿಯುವುದರಲ್ಲಿ ನಿಪುಣ. ನೀವು ವೇದಾಧ್ಯಯನದಲ್ಲಿ ನಿಪುಣರಿರಬಹುದು. ಆದಕಾರಣ ನಿಮಗೆ ನನ್ನನ್ನು ತುಳಿಯಲು ಅಧಿಕಾರವಿಲ್ಲ. ಒಬ್ಬ ಕೊಲೆ ಮಾಡಿದರೆ ಅವನನ್ನು ಹೊಗಳುವುದೇಕೆ? ಮತ್ತೊಬ್ಬ ಸೇಬಿನ ಹಣ್ಣನ್ನು ಕದ್ದರೆ ಅವನನ್ನು ಗಲ್ಲಿಗೇಕೆ ಏರಿಸುವುದು? ಈ ತರದ ಸ್ಥಿತಿ ಮಾಯವಾಗಬೇಕು. ಜಾತಿ ಒಳ್ಳೆಯದು. ನಮ್ಮ ಜೀವನದ ಸಮಸ್ಯೆಯನ್ನು ಬಗೆಹರಿಸುವ ಮಾರ್ಗ ಇದೊಂದೇ. ಮಾನವರು ಜಾತಿಗಳಾಗಿ ಭಾಗವಾಗಬೇಕಾಗಿದೆ. ನೀವು ಇದರಿಂದ ಪಾರಾಗಲಾರಿರಿ. ನೀವು ಎಲ್ಲಿ ಹೋದರೂ ಅಲ್ಲಿ ಜಾತಿ ಇರುವುದು. ಆದರೆ ಅಲ್ಲೆಲ್ಲಾ ಈ ವಿಶೇಷ ಹಕ್ಕುಗಳನ್ನು ಬೋಧಿಸಿದರೆ, ಅವನು “ನಿನ್ನಂತೆಯೇ ನಾನೂ ಉತ್ತಮ; ನಾನು ಬೆಸ್ತ, ನೀನು ತತ್ವಜ್ಞಾನಿ. ಆದರೆ ನಿನ್ನಲ್ಲಿರುವಂತೆಯೇ ನನ್ನಲ್ಲಿಯೂ ದೇವರು ಇರುವನು” ಎನ್ನುತ್ತಾನೆ. ನಮಗೆ ಬೇಕಾಗಿರುವುದು ಅದು. ಎಲ್ಲರಿಗೂ ಸಮಾನ ಅವಕಾಶ, ಯಾರಿಗೂ ಪ್ರತ್ಯೇಕ ಹಕ್ಕು ಇಲ್ಲ. ಭಗವಂತ ಎಲ್ಲರಲ್ಲಿಯೂ ಇರುವನೆಂಬುದನ್ನು ಸಕಲರಿಗೂ ಬೋಧಿಸೋಣ. ಪ್ರತಿಯೊಬ್ಬರೂ ತಮ್ಮ ಮುಕ್ತಿಯನ್ನು ತಾವು ಸಾಧಿಸಲಿ.

ಬೆಳವಣಿಗೆಗೆ ಸ್ವಾತಂತ್ರ್ಯವೇ ಪ್ರಥಮ ಆವಶ್ಯಕತೆ ‘ಈ ಸ್ತ್ರೀ ಅಥವಾ ಮಗುವಿನ ಮುಕ್ತಿಯನ್ನು ನಾನು ಸಾಧಿಸುತ್ತೇನೆ’ ಎಂದು ಹೇಳಿದರೆ ಅದು ತಪ್ಪು, ನೂರುಬಾರಿ ತಪ್ಪು. ಅನೇಕ ವೇಳೆ ನನ್ನನ್ನು ಜನರು ವಿಧವೆಯರ ಪ್ರಶ್ನೆಯನ್ನು ಹೇಗೆ ಬಗೆಹರಿಸುವಿರಿ? ಎಂದು ಮುಂತಾಗಿ ಕೇಳುವರು. ನಾನು ವಿಧವೆಯೆ, ನನ್ನ ಹತ್ತಿರ ಈ ಪ್ರಶ್ನೆಯನ್ನು ಕೇಳುತ್ತಿರಲ್ಲ– ಎಂದು ನಾನು ಕೇಳುತ್ತೇನೆ. ನಾನು ಸ್ತ್ರೀಯೆ. ಪುನಃ ಪುನಃ ಈ ಪ್ರಶ್ನೆಯನ್ನು ಕೇಳುತ್ತೀರಲ್ಲ? ಸ್ತ್ರೀ–ಸಮಸ್ಯೆಯನ್ನು ಬಗೆಹರಿಸಲು ನೀವಾರು? ನೀವೇನು ದೇವರೆ–ಪ್ರತಿಯೊಬ್ಬ ವಿಧವೆಯನ್ನು ಸ್ತ್ರೀಯನ್ನೂ ಆಳುವುದಕ್ಕೆ? ಬದಿಗೆ ಸರಿಯಿರಿ! ಅವರು ತಮ್ಮ ಸಮಸ್ಯೆಯನ್ನು ತಾವೇ ಬಗೆಹರಿಸಿಕೊಳ್ಳುವರು. ಕ್ರೂರಿಗಳೇ, ಇತರರನ್ನು ನಾವು ಉದ್ಧಾರ ಮಾಡುತ್ತೇವೆಯೆಂದು ತಿಳಿಯುವಿರಾ? ಬದಿಗೆ ಸರಿಯಿರಿ! ದೇವರು ಎಲ್ಲರನ್ನು ನೋಡಿಕೊಳ್ಳುವನು. ಸರ್ವಜ್ಞರೆಂದು ತಿಳಿಯಲು ನೀವಾರು? ಪಾಷಂಡರೆ! ಈಶ್ವರನ ಮೇಲೆ ನಿಮಗೆ ಅಧಿಕಾರವಿದೆ ಎಂದು ತಿಳಿಯುವುದಕ್ಕೆ ನಿಮ್ಮ ಎದೆಗಾರಿಕೆ ಎಷ್ಟು! ಪ್ರತಿಯೊಂದು ಜೀವಿಯೂ ಭಗವಂತನ ಆತ್ಮನೆಂದು ತಿಳಿಯದೆ! ಮೊದಲು ನಿಮ್ಮ ಕರ್ಮವನ್ನು ಗಮನಿಸಿ. ಆಗಲೇ ನಿಮ್ಮ ಮೇಲೆ ಹೊರೆಯಷ್ಟು ಕರ್ಮವಿದೆ ಕ್ಷಯಿಸುವುದಕ್ಕೆ. ನಿಮ್ಮ ದೇಶ ನಿಮ್ಮನ್ನು ವೇದಿಕೆಯ ಮೇಲೆ ಕುಳ್ಳಿರಿಸಬಹುದು. ನಿಮ್ಮ ಸಮಾಜ ಬೇಕಾದಷ್ಟು ನಿಮಗೆ ಪ್ರೋತ್ಸಾಹವನ್ನು ಕೊಡಬಹುದು. ನಿಮ್ಮನ್ನು ಮೂಢರು ಹೊಗಳಬಹುದು. ಆದರೆ ದೇವರು ಎಂದಿಗೂ ನಿದ್ರಿಸುವುದಿಲ್ಲ, ಇಂದೊ ನಾಳೆಯೋ ಕರ್ಮಫಲ ಲಭಿಸುವುದರಲ್ಲಿ ಸಂದೇಹವಿಲ್ಲ.

ಪ್ರತಿಯೊಬ್ಬ ಸ್ತ್ರೀ ಪುರುಷರನ್ನು ದೇವರಂತೆ ನೋಡಿ. ನೀವು ಯಾರಿಗೂ ಸಹಾಯಮಾಡಲಾರಿರಿ. ಬೇಕಾದರೆ ನೀವು ಸೇವೆ ಮಾಡಬಹುದು. ನಿಮಗೆ ಆ ಭಾಗ್ಯ ಬಂದರೆ ನೀವು ಭಗವಂತನ ಮಕ್ಕಳನ್ನು ಪೂಜಿಸಬಹುದು; ಆತನನ್ನು ಪೂಜಿಸಬಹುದು. ಭಗವಂತನು ತನ್ನ ಮಕ್ಕಳ ಸೇವೆಗೆ ನಿಮಗೆ ಅವಕಾಶ ಒದಗಿಸಿಕೊಟ್ಟರೆ ನೀವೇ ಧನ್ಯರು ಎಂದು ಭಾವಿಸಿ. ನಿಮ್ಮ ಸಮಾನವಿಲ್ಲವೆಂದು ಭಾವಿಸಬೇಡಿ. ಆ ಸದವಕಾಶ ಇತರರಿಗೆ ದೊರೆಯದೆ ಇರುವಾಗ ನಿಮಗೆ ದೊರೆತುದಕ್ಕೆ ನೀವೇ ಧನ್ಯರು. ಅದನ್ನು ಒಂದು ಪೂಜೆಯಂತೆ ಮಾಡಿ. ನಾವು\break ಬಡವನಲ್ಲಿ ದೇವರನ್ನು ನೋಡಬೇಕು. ನಮ್ಮ ಮುಕ್ತಿಗಾಗಿ ನಾವು ಅವರ ಪೂಜೆ ಮಾಡಬೇಕು. ದೀನರು, ದುಃಖಿಗಳು ನಮ್ಮ ಮೋಕ್ಷಕ್ಕಾಗಿಯೇ ಇರುವರು. ರೋಗಿಯಂತೆ, ಹುಚ್ಚನಂತೆ, ಕುಷ್ಠನಂತೆ, ಪಾಪಿಯಂತೆ ದೇವರು ಬರುವನು. ಇವು ನನ್ನ ಕೆಚ್ಚಿನ ನುಡಿಗಳು. ಭಗವಂತನನ್ನು ಹಲವು ರೂಪಗಳಲ್ಲಿ ಪೂಜಿಸುವುದಕ್ಕೆ ನನಗೆ ಅವಕಾಶ ದೊರಕಿಸಿರುವುದು ಮಹಾ ಪುಣ್ಯವೆಂದು ಸಾರುತ್ತೇನೆ. ಮತ್ತೊಬ್ಬರನ್ನು ಆಳುವುದರಿಂದ ಅವರಿಗೆ ಒಳ್ಳೆಯದನ್ನು ಮಾಡುತ್ತೇವೆ. ಎಂಬ ಭಾವನೆಯನ್ನು ತೊರೆಯಿರಿ. ಗಿಡಕ್ಕೆ ಬೆಳೆಯಲು ಸಹಾಯ ಮಾಡಿದಂತೆ ಮಾತ್ರ ನಾವು ಮಾಡಬಹುದು. ಬೀಜದ ಬೆಳವಣಿಗೆಗೆ ಸಾಮಗ್ರಿ ಒದಗಿಸಬಹುದು. ಮಣ್ಣು ನೀರು ಗಾಳಿ ಗೊಬ್ಬರಗಳನ್ನು ಕೊಡಬಹುದು. ಬೀಜ ಮಾತ್ರ, ತನ್ನ ಸ್ವಭಾವಕ್ಕೆ ಅನುಗುಣವಾಗಿ ತನಗೆ ಬೇಕಾದುದನ್ನು ಹೀರಿ ಜೀರ್ಣಿಸಿಕೊಂಡು ತನ್ನ ನಿಯಮದಂತೆ ಬೆಳೆಯುವುದು.

ಪ್ರಪಂಚದಲ್ಲಿ ಜ್ಞಾನವನ್ನು ಪ್ರಚಾರಮಾಡಿ. ಜ್ಞಾನವನ್ನು, ಹೆಚ್ಚು ಜ್ಞಾನವನ್ನು ಪ್ರಚಾರಮಾಡಿ. ಪ್ರತಿಯೊಬ್ಬರಿಗೂ ಜ್ಞಾನ ಬರಲಿ. ಪ್ರತಿಯೊಬ್ಬರೂ ಭಗವಂತನ ಸಾಕ್ಷಾತ್ಕಾರವನ್ನು ಪಡೆಯುವವರೆಗೆ ನಮ್ಮ ಕರ್ತವ್ಯ ಮುಗಿಯುವಂತೆ ಇಲ್ಲ. ಬಡವರಿಗೆ ಜ್ಞಾನವನ್ನು ನೀಡಿ. ಶ‍್ರೀಮಂತರಿಗೆ ಹೆಚ್ಚು ಜ್ಞಾನವನ್ನು ನೀಡಿ. ಏಕೆಂದರೆ ಶ‍್ರೀಮಂತರಿಗೆ ಅದು ಬಡವರಿ\-ಗಿಂತ ಹೆಚ್ಚು ಬೇಕಾಗಿದೆ. ಪಾಮರರಿಗೆ ಜ್ಞಾನವನ್ನು ನೀಡಿ, ಪಂಡಿತರಿಗೂ ಜ್ಞಾನವನ್ನು ನೀಡಿ. ಏಕೆಂದರೆ ಆಧುನಿಕ ವಿದ್ಯಾವಂತರಲ್ಲಿ ವಿದ್ಯಾಭಿಮಾನ ಹೆಚ್ಚಾಗಿದೆ! ಎಲ್ಲರಿಗೂ ಜ್ಞಾನಜ್ಯೋತಿಯನ್ನು ನೀಡಿ, ಉಳಿದುದನ್ನು ದೇವರಿಗೆ ಬಿಡಿ. “ಕೆಲಸ ಮಾಡುವುದಕ್ಕೆ ಮಾತ್ರ ನಮಗೆ ಅಧಿಕಾರ, ಅದರ ಫಲಕ್ಕೆ ಅಲ್ಲ. ಕರ್ಮವು ಫಲಕ್ಕೆ ಕಾರಣವಾಗದಿರಲಿ. ಆದರೆ ಕರ್ಮವಿಲ್ಲದೆ ಇರಬೇಡ” ಎಂದು ಶ‍್ರೀಕೃಷ್ಣ ಸಾರುತ್ತಾನೆ.

ಹಲವು ಶತಮಾನಗಳ ಹಿಂದೆ ನಮ್ಮ ಪೂರ್ವಿಕರಿಗೆ ಈ ಮಹೋಚ್ಚ ಸಿದ್ಧಾಂತವನ್ನು ಸಾರಿದ ಭಗವಂತನು ಈ ಸಂದೇಶವನ್ನು ಅನುಷ್ಠಾನಕ್ಕೆ ತರಲು ನಮಗೆ ಶಕ್ತಿ ಸಹಾಯಗಳನ್ನು ಅನುಗ್ರಹಿಸಲಿ.

