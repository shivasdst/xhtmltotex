
\chapter{ಭಕ್ತಿ}

\begin{center}
(೧೮೯೭ರ ನವೆಂಬರ್​ ೯ರಂದು ಲಾಹೋರಿನಲ್ಲಿ ನಡೆದ ಉಪನ್ಯಾಸ)
\end{center}

ಉಪನಿಷತ್ತಿನ ಘನಗರ್ಜನೆಯ ಪ್ರವಾಹದೊಂದಿಗೆ ಬಹಳ ದೂರದಿಂದ ಬರುತ್ತಿರುವ ಧ್ವನಿಯೊಂದು ಕೇಳುವುದು. ಅದು ಕೆಲವು ವೇಳೆ ತಾರಸ್ವರಕ್ಕೇರುತ್ತದೆ. ಆದರೆ ವೇದಾಂತ ಸಾಹಿತ್ಯದಲ್ಲೆಲ್ಲ ಅದರ ಧ್ವನಿ ಸ್ಪಷ್ಟವಾಗಿದ್ದರೂ ಅಷ್ಟೇನೂ ಗಟ್ಟಿಯಾಗಿಲ್ಲ. ಉಪನಿಷತ್ತಿನ ಪ್ರಧಾನ ಉದ್ದೇಶ ಭೂಮದ ಭಾವವನ್ನು ಮತ್ತು ಚಿತ್ರವನ್ನು ನಮಗೆ ಕೊಡುವುದು. ಅನೇಕ ವೇಳೆ ಈ ಭಾವ–ಗಾಂಭೀರ್ಯದಿಂದ ಕಾವ್ಯವು ಅಲ್ಲಿ ಇಲ್ಲಿ ವ್ಯಕ್ತವಾಗುವುದು:

\begin{verse}
\textbf{ನ ತತ್ರ ಸೂರ್ಯೋ ಭಾತಿ ನ ಚಂದ್ರತಾರಕಂ~।}\\\textbf{ನೇಮಾ ವಿದ್ಯುತೋ ಭಾಂತಿ ಕುತೋಽಯಮಗ್ನಿಃ~॥}
\end{verse}

“ಅಲ್ಲಿ ಸೂರ್ಯನು ಪ್ರಕಾಶಿಸುವುದಿಲ್ಲ, ಚಂದ್ರನೂ ಪ್ರಕಾಶಿಸುವುದಿಲ್ಲ, ತಾರಾವಳಿಗಳೂ ಪ್ರಕಾಶಿಸುವುದಿಲ್ಲ. ಇನ್ನು ಬೆಂಕಿಯ ಮಾತಿನ್ನೇನು ಹೇಳುವುದು?”

ಈ ಅಪೂರ್ವ ಹೃದಯಸ್ಪರ್ಶಿ ಕವಿತ್ವವನ್ನು ಕೇಳುತ್ತ ಕೇಳುತ್ತ ಇದ್ದರೆ ಅದು ನಮ್ಮನ್ನು ಇಂದ್ರಿಯ ಜಗತ್ತಿನಾಚೆಗೆ – ಬುದ್ಧಿಯ ಜಗತ್ತಿನಾಚೆಗೆ ಕೂಡ – ಒಯ್ಯುವುದು; ಯಾವುದನ್ನು ನಾವು ತಿಳಿದುಕೊಳ್ಳಲಾರೆವೋ, ಆದರೆ ಯಾವುದು ಸದಾ ನಮ್ಮಲ್ಲಿರುವುದೋ ಅಂತಹ ಜಗತ್ತಿಗೆ ಕರೆದೊಯ್ಯುವುದು. ಈ ಮಹಾ ಭಾವದ ಛಾಯೆಯಂತೆ ಮತ್ತೊಂದು ಭಾವನೆ ಅನುಸರಿಸುತ್ತಿದೆ. ಇದು ನಮ್ಮ ನಿತ್ಯ ಜೀವನಕ್ಕೆ ಸಹಕಾರಿಯಾದುದು ಮಾನವ ಜೀವನದ ಎಲ್ಲಾ ಕಾರ್ಯಕ್ಷೇತ್ರಗಳಿಗೆ ಅದು ಪ್ರವಹಿಸಬೇಕು. ಕ್ರಮೇಣ ಆ ಭಾವನೆ ಸ್ಪಷ್ಟವಾಗುತ್ತಾ ಮುಂದೆ ಪುರಾಣದಲ್ಲಿ ಬೃಹದಾಕಾರವನ್ನು ತಾಳುವುದು. ಅದೇ ಭಕ್ತಿ. ಭಕ್ತಿಯ ಬೀಜ ಆಗಲೇ ಅಲ್ಲಿ ಇದೆ. ಅದು ಸಂಹಿತೆಯಲ್ಲೂ ಇದೆ. ಉಪನಿಷತ್ತಿನಲ್ಲಿ ಈ ಬೀಜ ಸ್ಪಲ್ಪ ವಿಕಾಸವಾಗಿದೆ. ಆದರೆ ಪುರಾಣದಲ್ಲಿ ಇದನ್ನು ವಿಸ್ತಾರವಾಗಿ ವಿವರಿಸುವರು.

ಭಕ್ತಿಯನ್ನು ನಾವು ತಿಳಿದುಕೊಳ್ಳಬೇಕಾದರೆ ನಮ್ಮ ಪುರಾಣವನ್ನು ತಿಳಿದುಕೊಳ್ಳಬೇಕು. ಈಚೆಗೆ ಪುರಾಣದ ಪ್ರಮಾಣದ ವಿಷಯವಾಗಿ ಬೇಕಾದಷ್ಟು ಚರ್ಚೆ ನಡೆದಿದೆ. ಅಸ್ಪಷ್ಟವಾದ ಹಲವು ಭಾಗಗಳನ್ನು ತೆಗೆದುಕೊಂಡು ಅವನ್ನು ಟೀಕಿಸಿರುವರು. ಅವು ಆಧುನಿಕ ವಿಜ್ಞಾನದ ಪರೀಕ್ಷೆಯನ್ನು ಅನೇಕ ಸಂದರ್ಭಗಳಲ್ಲಿ ಎದುರಿಸಲಾರವು ಎಂದು ಮುಂತಾಗಿ ಹೇಳಿರುವರು. ಪೌರಾಣಿಕ ಹೇಳಿಕೆಗಳು ವೈಜ್ಞಾನಿಕವೆ ಅಲ್ಲವೇ ಎನ್ನುವುದಾಗಲೀ, ಅವುಗಳಲ್ಲಿ ಬರುವ ಭೂಗೋಳ ಮತ್ತು ಖಗೋಳ ಶಾಸ್ತ್ರಗಳಿಗೆ ಸಂಬಂಧಿಸಿದ ವಿಷಯಗಳ ಸತ್ಯಾಸತ್ಯತೆಯಾಗಲಿ ಮುಖ್ಯವಲ್ಲ. ಆದರೆ ಅವುಗಳಲ್ಲಿ ಬರುವ ಅತ್ಯಂತ ಮುಖ್ಯವಾದ ವಿಷಯವೆಂದರೆ ಭಕ್ತಿ. ಪ್ರತಿಯೊಂದು ಪುರಾಣದಲ್ಲಿಯೂ ಮೊದಲಿನಿಂದ ಕೊನೆಯವರೆಗೆ ಭಕ್ತಿ ಸಿದ್ಧಾಂತವನ್ನು ಸಾಧು–ಸಂತರ, ರಾಜರ ಕಥೆಯ ಮೂಲಕ ವಿವರಿಸಿರುವುದು ಕಂಡುಬರುತ್ತದೆ. ಅತಿ ಸುಂದರವಾದ ಭಕ್ತಿಗೆ ನಿದರ್ಶನವನ್ನು ಕೊಡುವುದಕ್ಕೆ ಪುರಾಣಗಳಿವೆ ಎಂದು ತೋರುವುದು. ಇದು ನಾನು ಹೇಳಿದಂತೆ ಸಾಧಾರಣ ಮನುಷ್ಯನಿಗೆ ಬಹಳ ಹತ್ತಿರವಾಗಿರುವುದು. ವೇದಾಂತದ ಪೂರ್ಣ ಜ್ಯೋತಿಯನ್ನು ತಾಳಬಲ್ಲವರು ಅದನ್ನು ತಿಳಿದುಕೊಳ್ಳಬಲ್ಲವರು ಬಹಳ ಅಪರೂಪ. ವೇದಾಂತಿಗೆ ಮೊದಲು ಬೇಕಾಗಿರುವುದೇ ‘ಅಭೀಃ’ ನಿರ್ಭಯತೆ. ವೇದಾಂತಿಯಾಗುವುದಕ್ಕೆ ಮುಂಚೆ ಅವನ ದೌರ್ಬಲ್ಯ ಹೋಗಬೇಕು. ಇದು ಎಷ್ಟು ಕಷ್ಟವೆನ್ನುವುದು ನಮಗೆ ಗೊತ್ತಿದೆ. ಯಾರು ಪ್ರಪಂಚದ ಹವ್ಯಾಸವನ್ನೆಲ್ಲಾ ತೊರೆದಿರುವರೋ, ಯಾರಿಗೆ ಹೇಡಿಯಾಗುವುದಕ್ಕೆ ಬಹಳ ಕಡಿಮೆ ಬಂಧನಗಳಿವೆಯೋ, ಅಂತಹವರು ಕೂಡ ಕೆಲವು ಕ್ಷಣ ಎಷ್ಟು ದುರ್ಬಲರಾಗುತ್ತಾರೆ, ಮೃದುವಾಗುತ್ತಾರೆ, ಅಂಜುತ್ತಾರೆ! ಹಾಗಿರುವಾಗ ಹಲವಾರು ಬಂಧನಗಳುಳ್ಳವರಿಗೆ ಇದು, ಇನ್ನೆಷ್ಟು ಕಷ್ಟವಾಗಿರಬೇಕು! ತಮ್ಮ ಒಳಗೆ ಹೊರಗೆ ನೂರಾರು ವಿಷಯಗಳಿಗೆ ಅವರು ದಾಸರಾಗಿರುತ್ತಾರೆ. ಪ್ರತಿಕ್ಷಣವೂ ದಾಸರಂತೆ ಜೀವನ ಹೊರೆಯಬೇಕಾಗುತ್ತದೆ. ಅಂತಹವರಿಗೆ ಪುರಾಣಗಳು ಅತಿ ಸುಂದರವಾದ ಸಂದೇಶವನ್ನು ನೀಡುವುವು.

\vskip 5pt

ಅವರಿಗಾಗಿ ಮೃದು ಮಧುರ ಕಾವ್ಯರಸವನ್ನು ಪುರಾಣಗಳು ಹರಿಸುವವು. ಅಂತಹವರಿಗೆ ಅತಿ ಅದ್ಭುತವಾದ ರೋಮಾಂಚಕಾರಿಯಾದ ಧ್ರುವ, ಪ್ರಹ್ಲಾದ ಮುಂತಾದ ಭಕ್ತರ ಸಾವಿರಾರು ಕಥೆಗಳನ್ನು ಹೇಳುವವು. ಈ ದೃಷ್ಟಾಂತಗಳ ಉದ್ದೇಶ ಭಕ್ತಿಯನ್ನು ಅನುಷ್ಠಾನಗೊಳಿಸುವುದು. ಪುರಾಣವು ವೈಜ್ಞಾನಿಕವಾದುದೆಂದು ನೀವು ನಂಬದೇ ಇದ್ದರೂ, ಪುರಾಣದಲ್ಲಿ ಬರುವ ಪ್ರಹ್ಲಾದ, ಧ್ರುವ ಅಥವಾ ಮತ್ತಾವುದಾದರೂ ಸಾಧು–ಸಂತರ ಜೀವನದಿಂದ ಪ್ರಭಾವಿತರಾಗದವರು ನಿಮ್ಮಲ್ಲಿ ಯಾರೂ ಇಲ್ಲ. ನಮ್ಮ ಕಾಲದಲ್ಲೇ ಪುರಾಣಗಳ ಪ್ರಭಾವವನ್ನು ಒಪ್ಪಿಕೊಳ್ಳುವುದಲ್ಲದೆ, ಹಿಂದಿನ ಕಾಲದಲ್ಲಿ ಅಧೋಗತಿಗೆ ಇಳಿದಿದ್ದ ಬೌದ್ಧ ಧರ್ಮ ನಮಗೆ ಕೊಡುವುದಕ್ಕೆ ಸಾಧ್ಯವಾದುದಕ್ಕಿಂತ ಹೆಚ್ಚು ಶ್ರೇಷ್ಠವಾದ ಎಲ್ಲರಿಗೂ ಗ್ರಹಿಸಲು ಯೋಗ್ಯವಾದ ಸಾಧಾರಣ ಧರ್ಮವನ್ನು ಕೊಟ್ಟಿದ್ದಕ್ಕಾಗಿ ನಾವು ಅವುಗಳಿಗೆ ಕೃತಜ್ಞರಾಗಿರಬೇಕು. ಸುಲಭವಾದ, ಮೃದುವಾದ ಭಕ್ತಿ ಸಾಧನೆಗಳು ಅವುಗಳಲ್ಲಿ ವಿವರಿಸಲ್ಪಟ್ಟಿವೆ. ನಾವು ನಮ್ಮ ನಿತ್ಯ ಜೀವನದಲ್ಲಿ ಅವನ್ನು ಅನುಸರಿಸಬೇಕು. ನಾವು ಮುಂದುವರಿದಂತೆ ಭಕ್ತಿ ಪ್ರೇಮಸಾರವಾಗಿದೆ ಎಂಬುದನ್ನು ನೋಡುವೆವು. ಎಲ್ಲಿಯವರೆಗೂ ವ್ಯಕ್ತಿಗತ ಜಡಪ್ರೀತಿ ಇರುವುದೋ ಅಲ್ಲಿಯವರೆಗೆ ನಾವು ಪುರಾಣಗಳನ್ನು ಮೀರಿ ಹೋಗಲಾರೆವು. ವ್ಯಕ್ತಿಯು ಆಶ್ರಯಕ್ಕೆ ಎಲ್ಲಿಯವರೆಗೂ ಮತ್ತೊಬ್ಬನನ್ನು ಆಶ್ರಯಿಸುತ್ತಾನೆಯೋ, ಅಲ್ಲಿಯವರೆವಿಗೂ ಪುರಾಣ ಒಂದಲ್ಲ ಒಂದು ರೂಪದಲ್ಲಿ ಇದ್ದೇ ತೀರಬೇಕು. ಅವುಗಳ ಹೆಸರನ್ನು ಬದಲಾಯಿಸಬಹುದು, ಅಷ್ಟೇ, ಆಗಲೇ ಇರುವ ಪುರಾಣವನ್ನು ನೀವು ಅಲ್ಲಗಳೆಯಬಹುದು. ಆದರೆ ತಕ್ಷಣವೇ ನೀವೇ ಮತ್ತೊಂದು ಪುರಾಣವನ್ನು ಬರೆಯಬೇಕಾಗುವುದು. ಹಳೆಯ ಪುರಾಣ ನಮಗೆ ಬೇಕಾಗಿಲ್ಲ ಎಂದು ಸಾರುವ ಒಬ್ಬ ಮಹಾತ್ಮ ನಮ್ಮಲ್ಲಿ ಹುಟ್ಟಿದರೆ, ಆ ಮನುಷ್ಯ ಸತ್ತ ಇಪ್ಪತ್ತು ವರ್ಷದೊಳಗೆ ಅವನ ಶಿಷ್ಯರು ಗುರುವಿನ ಹೆಸರಿನಲ್ಲಿ ಮತ್ತೊಂದು ಪುರಾಣವನ್ನು ಬರೆಯುವರು. ಅಷ್ಟೇ ವ್ಯತ್ಯಾಸ.

ಮನುಷ್ಯನ ಸ್ವಭಾವಕ್ಕೆ ಇದು ಅತ್ಯಾವಶ್ಯಕ. ಮಾನವ ದುರ್ಬಲತೆಯನ್ನೆಲ್ಲ ಯಾರು ಮೀರಿ ಹೋಗಿರುವರೋ, ಪರಮಹಂಸರಂತೆ ಧೀರರಾಗಿ, ಮಾಯಾಪಾಶದಿಂದ ಪಾರಾಗಿ ಪ್ರಕೃತಿಯ ಅವಶ್ಯಕತೆಗಳನ್ನೆಲ್ಲಾ ಯಾರು ಮೀರಿ ಹೋಗಿರುವರೋ, ಅಂತಹವರಿಗೆ ಪುರಾಣಗಳು ಅನಾವಶ್ಯಕ. ಅವರೇ ಉತ್ಸಾಹ ಪೂರಿತ ದಿಗ್ವಿಜಯಿಗಳು, ಜಗತ್ತಿನ ದೇವರು. ಸಾಧಾರಣ ಮನುಷ್ಯನಿಗೆ ಪೂಜಿಸುವುದಕ್ಕೆ ಒಂದು ಸಗುಣ ದೇವರಿಲ್ಲದೆ ಸಾಧ್ಯವಿಲ್ಲ. ಹೊರಗೆ ಒಬ್ಬ ದೇವರನ್ನು ಪೂಜಿಸದೆ ಇದ್ದರೆ, ಹೆಂಡತಿ, ಮಗು, ತಂದೆ, ಸ್ನೇಹಿತ, ಗುರು ಅಥವಾ ಇನ್ನಾರನ್ನಾದರೂ ದೇವರಂತೆ ಪೂಜಿಸುವನು. ಗಂಡಸರಿಗಿಂತ ಹೆಂಗಸರಿಗೆ ಈ ಅವಶ್ಯಕತೆ ಜಾಸ್ತಿ. ಬೆಳಕಿನ ಸ್ಪಂದನ ಎಲ್ಲಾ ಕಡೆಯಲ್ಲಿಯೂ ಇರುವುದು. ಕತ್ತಲೆಯಲ್ಲಿಯೂ ಇರಬಹುದು. ಬೆಕ್ಕು ಮುಂತಾದವು ಅಲ್ಲಿ ನೋಡುತ್ತವೆ. ಆದರೆ ನಮಗೆ ಕಾಣಬೇಕಾದರೆ ಅದು ನಮ್ಮ ಕ್ಷೇತ್ರದಲ್ಲಿರಬೇಕು. ನಾವು ನಿರ್ಗುಣ ಮುಂತಾದುವನ್ನು ಕುರಿತು ಮಾತನಾಡಬಹುದು. ಆದರೆ ಎಲ್ಲಿಯವರೆಗೂ ನಾವು ಸಾಧಾರಣ ಮನುಷ್ಯರಾಗಿರುವೆವೋ, ಅಲ್ಲಿಯವರೆಗೆ ದೇವರನ್ನು ಭಾವಿಸುವ ರೀತಿ, ನಾವು ಪೂಜಿಸುವ ರೀತಿ ಮಾನವ ಸಾಧಾರಣವಾಗಿವೆ. “ಈ ದೇಹವೇ ದೇವರ ಶ್ರೇಷ್ಠ ದೇವಾಲಯ”. ಆದಕಾರಣವೇ ಅನೇಕ ಶತಮಾನಗಳಿಂದಲೂ ದೇವರನ್ನು ಮಾನವ ರೂಪದಲ್ಲಿ ಪೂಜಿಸುವ ಪದ್ಧತಿ ರೂಢಿಯಲ್ಲಿದೆ. ಸ್ವಾಭಾವಿಕವಾಗಿ ಇದರಿಂದ ಉಂಟಾಗುವ ಕೆಲವು ಅತಿರೇಕದ ಅಭ್ಯಾಸಗಳನ್ನು ಟೀಕಿಸಬಹುದು. ಆದರೆ ಇದರ ಹೃದಯ ಚೆನ್ನಾಗಿದೆ. ಅತಿರೇಕ ಇದ್ದರೂ, ಉತ್ಪ್ರೇಕ್ಷೆ ಇದ್ದರೂ, ಆ ಸಿದ್ಧಾಂತದ ಅಂತರಾಳದಲ್ಲಿ ಸತ್ಯವಿದೆ, ಮೂಲಭಿತ್ತಿ ದೃಢವಾಗಿದೆ. ಯಾವ ಕಟ್ಟುಕಥೆಯನ್ನಾಗಲೀ, ವಿಜ್ಞಾನಕ್ಕೆ ವಿರೋಧವಾದ ವಿಷಯವನ್ನಾಗಲೀ, ವಿಚಾರ ಮಾಡದೆ ಸ್ವೀಕರಿಸಿ ಎಂದು ನಾನು ಹೇಳುತ್ತಿಲ್ಲ. ದುರಾದೃಷ್ಟವಶಾತ್​ ಪುರಾಣಕ್ಕೆ ಪ್ರವೇಶಿಸಿದ ವಾಮಾಚಾರ ಅಭ್ಯಾಸಗಳನ್ನೆಲ್ಲಾ ನಂಬಿ ಎಂದು ಹೇಳುವುದಿಲ್ಲ. ನಾನು ಹೇಳುವುದು ಇದು: ಅದರಲ್ಲಿ ಒಂದು ಸತ್ಯವಿದೆ. ಅದನ್ನು ಮರೆಯಕೂಡದು. ಧರ್ಮವನ್ನು ಅನುಷ್ಠಾನಯೋಗ್ಯವಾಗಿ ಮಾಡುವುದಕ್ಕಾಗಿ, ಉನ್ನತ ತಾತ್ತ್ವಿಕ ಶಿಖರದಿಂದ ಅದನ್ನು ಕೆಳಗಿಳಿಸಿ ಜನಸಾಮಾನ್ಯರ ದೈನಂದಿನ ಜೀವನದಲ್ಲಿ ಪ್ರವೇಶಿಸುವಂತೆ ಮಾಡುವುದಕ್ಕಾಗಿ ಪುರಾಣಗಳು ಭಕ್ತಿಯನ್ನು ಉಪದೇಶಿಸುತ್ತವೆ. ಅವುಗಳಿರುವುದೇ ಇದಕ್ಕಾಗಿ.

ಅನಂತರ ಉಪನ್ಯಾಸಕರು ಭಕ್ತಿಗೆ ಸಂಬಂಧಿಸಿದ ಬಾಹ್ಯ ಸಲಕರಣೆಯ ಅವಶ್ಯಕತೆಯನ್ನು ಸಮರ್ಥಿಸಿದರು. ಮನುಷ್ಯನು ದೇವರ ದಯೆಯಿಂದ ನಿಂತಕಡೆಯೇ ನಿಲ್ಲಕೂಡದು. ಆದರೆ ಈಗಿರುವ ಪರಿಸ್ಥಿತಿಯನ್ನು ವಿರೋಧಿಸಿ ಪ್ರಯೋಜನವಿಲ್ಲ. ನಾವು ಮಾನವನನ್ನು ಒಬ್ಬ ಆಧ್ಯಾತ್ಮಿಕ ಜೀವಿ ಎಂದು ಎಷ್ಟೇ ಭಾವಿಸಿದರೂ ಈಗ ಅವನೊಬ್ಬ ಭೌತಿಕ ಜೀವಿ. ಈ ಭೌತಿಕ ಜೀವಿಯನ್ನು ಕ್ರಮೇಣ ಆಧ್ಯಾತ್ಮಿಕ ಜೀವಿಯನ್ನಾಗಿ ಮಾಡಬೇಕು. ಈಗಿನ ಕಾಲದಲ್ಲಿ ಶೇಕಡ ೯೯ ಜನಕ್ಕೆ ಆಧ್ಯಾತ್ಮಿಕ ವಿಷಯವನ್ನು ತಿಳಿದುಕೊಳ್ಳುವುದಕ್ಕೆ\break ಆಗುವುದಿಲ್ಲ. ಅದನ್ನು ಕುರಿತು ಮಾತನಾಡುವುದು ಮತ್ತೂ ಕಷ್ಟ. ನಮ್ಮನ್ನು ಮುಂದೆ ನೂಕುತ್ತಿರುವ ಶಕ್ತಿ, ನಾವು ಸಾಧಿಸ ಬೇಕೆಂದಿರುವುದೆಲ್ಲವೂ ಭೌತಿಕವಾದುವುಗಳು, ಹರ್ಬರ್ಟ್​ ಸ್ಪೆನ್ಸರ್​ ಹೇಳಿದಂತೆ ವಿರೋಧ ಎಲ್ಲಿ ಅತಿ ಕಡಮೆಯೋ, ಅಲ್ಲಿ ಮಾತ್ರ ನಾವು ಕೆಲಸಮಾಡಲು ಸಾಧ್ಯ. ಪುರಾಣಗಳು ಜಾಣ್ಮೆಯಿಂದ ಅತ್ಯಂತ ಕಡಮೆ ವಿರೋಧದ ಹಾದಿಯನ್ನೇ ಅನುಸರಿಸುವುವು. ಈ ವಿಷಯದಲ್ಲಿ ಅವು ಅದ್ಭುತ ಯಶಸ್ಸನ್ನೂ ಪಡೆದಿವೆ. ಭಕ್ತಿಯ ಆದರ್ಶವೇನೋ ಆಧ್ಯಾತ್ಮಿಕ. ಆದರೆ ದಾರಿ ಭೌತಿಕ, ಅದಲ್ಲದೆ ಬೇರೆ ದಾರಿಯಿಲ್ಲ. ಭೌತಿಕ ಜಗತ್ತಿನಲ್ಲಿ ಆಧ್ಯಾತ್ಮಿಕಕ್ಕೆ ಸಹಾಯಕವಾದುದೆಲ್ಲವನ್ನೂ ಸ್ವೀಕರಿಸಿ ಪ್ರಯೋಜನ ಪಡೆದುಕೊಳ್ಳಬೇಕು. ಲಿಂಗ ಜಾತಿಭೇದಗಳಿಲ್ಲದೆ ವೇದಾಧ್ಯಯನಕ್ಕೆ ಎಲ್ಲರಿಗೂ ಅಧಿಕಾರವಿದೆಯೆಂದು ಶಾಸ್ತ್ರಗಳು ಸಾರುತ್ತವೆ ಎಂದು ಉಪನ್ಯಾಸಕರು ಸೂಚಿಸಿದರು. ದೇವಸ್ಥಾನಗಳು ಮಾನವನಿಗೆ ಭಗವಂತನನ್ನು ಪ್ರೀತಿಸಲು ಸಹಾಯಕವಾಗುವುದಾದರೆ ಅವುಗಳನ್ನು ಕಟ್ಟಿಸಬೇಕು. ವಿಗ್ರಹಗಳು ಅದೇ ರೀತಿ ಸಹಾಯಕವಾಗುವುದಾದರೆ ಅವನಿಗೆ ಇಪ್ಪತ್ತು ವಿಗ್ರಹಗಳನ್ನು ಬೇಕಾದರೂ ಕೊಡಿ–ಎಂದರು ಅವರು. ಯಾವುದು ಅವನ ಆಧ್ಯಾತ್ಮಿಕ ಜೀವನಕ್ಕೆ ಸಹಾಯಮಾಡುವುದೋ ಅದೆಲ್ಲವನ್ನೂ ತೆಗೆದುಕೊಳ್ಳಲಿ. ಆದರೆ ಅದು ಧಾರ್ಮಿಕವಾಗಿರಬೇಕು. ಅಧರ್ಮ ಸಹಾಯ ಮಾಡುವುದಿಲ್ಲ. ಅದು ಅವನಿಗೆ ಆತಂಕವಾಗುವುದು. ಭರತಖಂಡದಲ್ಲಿ ವಿಗ್ರಹಾರಾಧನೆಯನ್ನು ವಿರೋಧಿಸಿದವರಲ್ಲಿ ಕಬೀರ ಒಬ್ಬನು. ಆದರೆ ಸಗುಣ ದೇವರಲ್ಲಿ ನಂಬದೆ ದೊಡ್ಡ ದೊಡ್ಡ ಧರ್ಮವನ್ನು ಸ್ಥಾಪಿಸಿದವರೂ ತತ್ತ್ವವನ್ನು ಜನರಿಗೆ ನಿರ್ಭೀತಿಯಿಂದ ಬೋಧಿಸಿದವರೂ ಭರತಖಂಡದಲ್ಲಿ ಎಷ್ಟೋ ಜನ ಇದ್ದರು. ಆದರೂ ಅವರು ವಿಗ್ರಹಾರಾಧನೆಯನ್ನು ದೂರಲಿಲ್ಲ. ಹೆಚ್ಚು ಎಂದರೆ ಅದು ಅಷ್ಟು ಉತ್ತಮ ರೀತಿಯ ಪೂಜೆ ಎಂದು ಅವರು ಒಪ್ಪಿಕೊಳ್ಳುವುದಿಲ್ಲ. ಯಾವ ಪುರಾಣವೂ ಇದನ್ನು ಉತ್ತಮ ಪೂಜೆ ಎಂದು ಸಾರುವುದಿಲ್ಲ. ಯಹೂದ್ಯರು ಹಿಂದೆ ವಿಗ್ರಹಾರಾಧಕರಾಗಿದ್ದರೆಂಬುದನ್ನು ಚಾರಿತ್ರಿಕ ದೃಷ್ಟಿಯಿಂದ ಉಪನ್ಯಾಸಕರು ತೋರಿದರು. ಜಿಹೋವನು ಪೆಟ್ಟಿಗೆಯಲ್ಲಿರುತ್ತಾನೆಂದು ಯಹೂದ್ಯರು ನಂಬಿದ್ದರು. ಇತರರು ವಿಗ್ರಹಾರಾಧನೆ ಕೆಟ್ಟದ್ದು ಎಂದುದರಿಂದ ಅದನ್ನು ದೂರುವುದನ್ನು ಉಪನ್ಯಾಸಕರು ಖಂಡಿಸಿದರು. ವಿಗ್ರಹ ಅಥವಾ ಯಾವುದಾದರೂ ಬಾಹ್ಯ ವಸ್ತು ನಮ್ಮ ಆಧ್ಯಾತ್ಮಿಕ ಜೀವನಕ್ಕೆ ಸಹಾಯ ಮಾಡಿದರೆ ಅದನ್ನು ಉಪಯೋಗಿಸಿಕೊಳ್ಳಲು ಅಡ್ಡಿ ಇಲ್ಲದೇ ಇದ್ದರೂ, ಅದನ್ನು ಬಹಳ ಕೆಳಗಿನ ಪೂಜೆ ಎಂದು ಭಾವಿಸಬೇಕೆಂದು ಸಾರಿದರು. ವಿಗ್ರಹಾರಾಧನೆಯು ಅತ್ಯಂತ ಕೆಳಮಟ್ಟದ ಪೂಜೆ ಎಂದು ಹೇಳದ ಒಂದೇ ಒಂದು ಗ್ರಂಥವೂ ನಮ್ಮ ಧರ್ಮದಲ್ಲಿಲ್ಲ. ಏಕೆಂದರೆ ಇದು ಜಡ ವಸ್ತುವಿನ ಮೂಲಕ ಮಾಡುವ ಪೂಜೆ. ಎಲ್ಲರ ಮೇಲೂ ಈ ವಿಗ್ರಹಾರಾಧನೆಯನ್ನು ಹೇರಲು ಮಾಡಿದ ಪ್ರಯತ್ನವನ್ನು ಖಂಡಿಸಲು ತಕ್ಕಭಾಷೆ ಇಲ್ಲ. ಯಾವುದನ್ನು ಪೂಜಿಸಬೇಕು, ಯಾವುದರ ಮೂಲಕ ಪೂಜಿಸಬೇಕು ಎಂದು ಇತರರಿಗೆ ಬಲಾತ್ಕರಿಸುವುದಕ್ಕೆ ನಮಗೆ ಏನು ಅಧಿಕಾರವಿದೆ? ಒಬ್ಬ ಹೇಗೆ ಮುಂದುವರಿಯಬಲ್ಲ ಎಂಬುದು ಇತರರಿಗೆ ಹೇಗೆ ಗೊತ್ತು? ವಿಗ್ರಹವನ್ನೋ ಬೆಂಕಿಯನ್ನೋ ಅಥವಾ ಒಂದು ಕಂಬವನ್ನೋ ಪೂಜಿಸಿಯೂ ಒಬ್ಬನು ಆಧ್ಯಾತ್ಮಿಕ ಜೀವನದಲ್ಲಿ ಮುಂದುವರಿಯುವುದಿಲ್ಲ ಎಂಬುದು ಹೇಗೆ ಗೊತ್ತು? ಅದಕ್ಕೆ ನಮ್ಮ ಗುರುಗಳು ಮಾತ್ರ ಮಾರ್ಗದರ್ಶಕರಾಗಬೇಕು. ಗುರುಶಿಷ್ಯರ ಸಂಬಂಧ ಇದಕ್ಕೆ ಸಹಕಾರಿ. ಇಷ್ಟದೇವತೆಯ ಪ್ರಾಧಾನ್ಯವನ್ನು ಭಕ್ತಿಶಾಸ್ತ್ರಗಳು ಆದಕಾರಣವೇ ಒತ್ತಿ ಹೇಳಿರುವುದು. ಪ್ರತಿಯೊಬ್ಬನೂ ತನ್ನ ಇಷ್ಟದಂತೆ ದೇವರ ಕಡೆಗೆ ಹೋಗಬೇಕು. ಅದೇ ಆತನ ಇಷ್ಟದೇವತೆ. ಇತರರ ಪೂಜೆಯನ್ನು ಅವನು ಸಹಾನುಭೂತಿಯಿಂದ ನೋಡಬೇಕು. ಆದರೆ ತನ್ನ ರೀತಿಯಂತೆಯೇ ತಾನು ಅನುಷ್ಠಾನ ಮಾಡಬೇಕು. ತಾನು ಗುರಿ ಸೇರಿ ಬಾಹ್ಯವಸ್ತುಗಳ ಅವಶ್ಯಕತೆ ಇಲ್ಲದೆ ಇರುವ ಹಂತವನ್ನು ಸೇರುವವರೆಗೂ, ಅವನು ತನ್ನ ಮಾರ್ಗವನ್ನೇ ಅನುಸರಿಸಬೇಕು. ಕುಲಗುರು ಎಂಬ ವಂಶಪಾರಂಪರ್ಯವಾಗಿ ಬರುವ ಗುರುಗಿರಿಯ ಪದ್ಧತಿಯನ್ನು ಕುರಿತು ಸ್ವಲ್ಪ ಎಚ್ಚರಿಕೆ ವಹಿಸುವುದು ಆವಶ್ಯಕ. ಇದು ಭರತಖಂಡದ ಕೆಲವು ಕಡೆ ಜಾರಿಯಲ್ಲಿದೆ. ಗುರು ಯಾರು ಎನ್ನುವುದರ ಬಗ್ಗೆ ಶಾಸ್ತ್ರದಲ್ಲಿ ಹೀಗೆ ಓದುವೆವು: “ಅವನಿಗೆ ಶಾಸ್ತ್ರಸಾರ ಗೊತ್ತಿರಬೇಕು. ಅವನು ಪಾಪದೂರನಾಗಿರಬೇಕು. ದ್ರವ್ಯದಾಸೆಗೆ ಅಥವಾ ಕೀರ್ತಿಯಾಸೆಗೆ ಮತ್ತೊಬ್ಬನಿಗೆ ಬೋಧಿಸಕೂಡದು. ವಸಂತ ಋತು, ಗಿಡಮರಗಳಿಂದ ಏನನ್ನೂ ಕೇಳದೆ ಅವಕ್ಕೆ ಪುನರ್ಜನ್ಮವನ್ನು ಕೊಟ್ಟು ತಳಿರು, ಹೂ, ಫಲ ಬರುವಂತೆ ಮಾಡುತ್ತದೆ. ಅದರಂತೆಯೇ ಗುರು ಕೂಡ ಏನನ್ನೂ ಆಶಿಸದೆ ಇತರರಿಗೆ ಒಳ್ಳೆಯದನ್ನು ಮಾಡುವನು.” ಇಂತಹವನು ಮಾತ್ರ ಗುರುವಾಗಬಲ್ಲನು, ಇತರರಲ್ಲ. ಮತ್ತೊಂದು ಅಪಾಯವಿದೆ. ಗುರು ಕೇವಲ ಉಪಾಧ್ಯಾಯನಲ್ಲ. ಅಧ್ಯಾಪಕತನ ಬಹಳ ಅಲ್ಪಭಾಗ. ಹಿಂದೂಗಳು ತಿಳಿದಿರುವಂತೆ ಗುರು ಶಿಷ್ಯನಿಗೆ ಅಧ್ಯಾತ್ಮವನ್ನು ಧಾರೆಯೆರೆದು ಕೊಡುವನು. ಒಂದು ಭೌತಿಕ ಉದಾಹರಣೆಯನ್ನು ತೆಗೆದುಕೊಳ್ಳುವುದಾದರೆ, ಉತ್ತಮಾಣುಗಳನ್ನು ರಕ್ತಕ್ಕೆ ಸೇರಿಸದೆ ಇದ್ದರೆ ರೋಗಾಣುಗಳು ರಕ್ತವನ್ನು ಸೇರುವ ಸಂಭವವುಂಟು. ಕೆಟ್ಟ ಗುರುವಿನಿಂದ ಕಲಿತರೆ ಕೆಟ್ಟ ವಿಷಯಗಳನ್ನು ಕಲಿತುಕೊಳ್ಳುವ ಅಪಾಯವಿದೆ. ಕುಲಗುರು ಎಂಬ ಭಾವನೆ ಭರತಖಂಡದಿಂದ ಸಂಪೂರ್ಣ ಮಾಯವಾಗಬೇಕಾಗಿರುವುದು ಆವಶ್ಯಕವಾಗಿದೆ. ಗುರುತ್ವ ಒಂದು ವ್ಯಾಪಾರವಾಗಬಾರದು. ಅದು ನಿಲ್ಲಬೇಕು. ಅದು ಶಾಸ್ತ್ರಕ್ಕೆ ವಿರೋಧ. ಯಾರೂ ಕುಲಗುರುಗಳೆಂದು ಹೇಳಿಕೊಳ್ಳಬಾರದು. ಈಗ ಇರುವ ಕುಲಗುರು ಪದ್ಧತಿಗೆ ಸಹಾಯವನ್ನೂ ನೀಡಬಾರದು.

ಆಹಾರದ ವಿಷಯವನ್ನು ಕುರಿತು ಸ್ವಾಮೀಜಿಯವರು, ಈಗಿನ ಕಾಲದ ಆಹಾರಕ್ಕೆ ಸಂಬಂಧಪಟ್ಟ ಕಟ್ಟುನಿಟ್ಟಿನ ನಿಯಮಗಳೆಲ್ಲ ಕೇವಲ ಬಾಹ್ಯಾಡಂಬರಗಳಾಗಿವೆ ಎಂದರು. ಹಿಂದೆ ಯಾವ ಉದ್ದೇಶದಿಂದ ಅದನ್ನು ಮಾಡಿದ್ದರೋ ಆ ಗುರಿಯನ್ನೇ ಮರೆತುಬಿಟ್ಟಿದ್ದೇವೆ. ಯಾರು ಆಹಾರವನ್ನು ಮುಟ್ಟುವರು ಎಂಬುದರ ಬಗ್ಗೆ ಜೋಪಾನದಿಂದಿರಬೇಕು ಎಂಬ ನಿಯಮವನ್ನು ಪ್ರಸ್ತಾಪಿಸಿ, ಇದರ ಹಿಂದೆ ದೊಡ್ಡ ಮನಃಶ್ಶಾಸ್ತ್ರದ ರಹಸ್ಯವಿದೆ ಎಂದರು. ಆದರೆ ಜನರ ನಿತ್ಯ ಜೀವನದಲ್ಲಿ ಈ ನಿಯಮವನ್ನು ಅನುಸರಿಸುವುದು ಕಷ್ಟ ಅಥವಾ ಅಸಾಧ್ಯ. ಇಲ್ಲಿ ಕೇವಲ ಒಂದು ಪಂಥಕ್ಕೆ ಅನ್ವಯಿಸಬಹುದಾದ ವಿಧಿಯನ್ನೇ ಎಲ್ಲರಿಗೂ ಅನ್ವಯಿಸುವಂತೆ ಮಾಡಿದುದು ತಪ್ಪು. ಯಾರು ಇಡಿಯ ಜೀವನವನ್ನು ಆಧ್ಯಾತ್ಮಿಕ ಜೀವನಕ್ಕೆ ಧಾರೆಯೆರೆದಿದ್ದಾರೋ ಅವರಿಗೆ ಮಾತ್ರ ಇದು ಅನ್ವಯಿಸುವುದು. ಆದರೆ ಮುಕ್ಕಾಲುಪಾಲು ಜನಕ್ಕೆ ಭೋಗೇಚ್ಛೆ ಇನ್ನೂ ತೃಪ್ತಿಯಾಗಿಲ್ಲ. ಅವರ ಮೇಲೆ ಅಧ್ಯಾತ್ಮವನ್ನು ಬಲಾತ್ಕಾರವಾಗಿ ಹೊರಿಸಲಾಗುವುದಿಲ್ಲ.

ಭಕ್ತನು ಹೇಳುವ ಅತಿ ಶ್ರೇಷ್ಠವಾದ ಪೂಜೆಯೇ ಮಾನವಪೂಜೆ, ಯಾವುದಾದರೊಂದು ಬಾಹ್ಯಪೂಜೆ ಇರಬೇಕಾದರೆ ಒಬ್ಬನೋ, ಆರು ಜನರೋ, ಹನ್ನೆರಡು ಜನರೋ, ಬಡವರನ್ನು ತಮ್ಮ ಯೋಗ್ಯತೆಗೆ ತಕ್ಕಂತೆ ಮನೆಗೆ ಕರೆದು ಉಪಚರಿಸುವುದು ಮೇಲು. ನಾರಾಯಣ ದೃಷ್ಟಿಯಿಂದ ಅವರಿಗೆ ಸೇವೆ ಸಲ್ಲಿಸಬೇಕು. ಹಲವು ದೇಶಗಳಲ್ಲಿ ಕಂಡುಬರುವ ದಾನ ಮಾಡುವ ಪದ್ಧತಿಯು ಸಫಲವಾಗದೆ ಇರುವುದಕ್ಕೆ ಕಾರಣ ಅದನ್ನು ಒಳ್ಳೆಯ ಭಾವದಿಂದ ಮಾಡದೆ ಇರುವುದು. ‘ತಗೊ, ತೊಲಗು’ ಎನ್ನುವುದು ದಾನವಲ್ಲ, ಅದು ಅಹಂಕಾರದ ಚಿಹ್ನೆ. ತಾವು ದಾನಿ ಎಂಬುದನ್ನು ಪ್ರಪಂಚಕ್ಕೆ ತೋರಿಸಿ ಸನ್ಮಾನವನ್ನು ಪಡೆಯುವುದೇ ಮುಖ್ಯ ಉದ್ದೇಶವಾಗಿರುತ್ತದೆ. ಸ್ಮೃತಿಯ ಪ್ರಕಾರ, ಕೊಡುವವನು ಸ್ವೀಕರಿಸುವವನಿಗಿಂತ ಕೆಳಗೆ ಇರಬೇಕು, ಮತ್ತು ಸ್ವೀಕರಿಸುವವನು ಸದ್ಯಕ್ಕೆ ದೇವರ ಸ್ಥಾನದಲ್ಲಿ ನಿಲ್ಲಬೇಕು – ಎಂಬುದನ್ನು ಹಿಂದೂಗಳು ತಿಳಿದುಕೊಂಡಿರಬೇಕು. ಪ್ರತಿದಿನವೂ ಮನೆಗೆ ದರಿದ್ರ, ಕುರುಡ, ಅನಾಥ ನಾರಾಯಣರನ್ನು ಕರೆದು ಸೇವೆ ಮಾಡುವಂತೆ ಸ್ವಾಮೀಜಿ ಸಲಹೆ ಇತ್ತರು. ವಿಗ್ರಹಕ್ಕೆ ಮಾಡುವ ಪೂಜೆಯನ್ನು ಇವರಿಗೆ ಮಾಡಬೇಕು. ಊಟ ಕೊಡಬೇಕು, ಬಟ್ಟೆ ಕೊಡಬೇಕು. ಮಾರನೆ ದಿನ ಇತರರಿಗೂ ಹಾಗೆಯೇ ಮಾಡಬೇಕು. ಅವರು ಯಾವ ವಿಧದ ಪೂಜೆಯನ್ನೂ ಧಿಕ್ಕರಿಸಲಿಲ್ಲ. ಆದರೆ ತತ್ಕಾಲಕ್ಕೆ ಭರತಖಂಡಕ್ಕೆ ಬೇಕಾದ ಶ್ರೇಷ್ಠಪೂಜೆಯೇ, ಈ ನಾರಾಯಣ ಸೇವೆ ಎಂದು ಹೇಳಿದರು.

ಕೊನೆಯಲ್ಲಿ, ಭಕ್ತಿಯನ್ನು ಒಂದು ತ್ರಿಭುಜಕ್ಕೆ ಹೋಲಿಸಿದರು. ಮೊದಲನೆ ಕೋನವೇ ಪ್ರೇಮಕ್ಕೆ ಯಾವ ಆಸೆಯೂ ಇಲ್ಲ. ಎರಡನೆಯದು ಪ್ರೇಮಕ್ಕೆ ಅಂಜಿಕೆಯಿಲ್ಲ ಎಂಬುದು. ಬಹುಮಾನ ಅಥವಾ ಮತ್ತಾವುದಾದರೂ ಸಹಾಯಕ್ಕಾಗಿ ಪ್ರೀತಿಸುವುದು ಭಿಕ್ಷುಕನ ಧರ್ಮ, ಅಲ್ಲಿ ನಿಜವಾದ ಧರ್ಮವಿರುವುದು ಬಹಳ ಕಡಿಮೆ. ಜನ ಭಿಕ್ಷುಕರಾಗದಿರಲಿ. ಭಿಕ್ಷೆ ಬೇಡುವುದು ನಾಸ್ತಿಕನ ಚಿಹ್ನೆ. “ಗಂಗಾನದಿಯ ತೀರದಲ್ಲಿದ್ದು ಕುಡಿಯುವ ನೀರಿಗಾಗಿ ಬಾವಿಯನ್ನು ತೋಡುವವನು ಮೂರ್ಖ.” ಅದರಂತೆಯೇ ದೇವರಿಂದ ಪ್ರಾಪಂಚಿಕ ವಸ್ತುಗಳನ್ನು ಬಯಸುವವನು ಕೂಡ. ಭಕ್ತನು ಎದ್ದು ನಿಂತು, “ದೇವರೆ, ನನಗೆ ನಿನ್ನಿಂದ ಏನು ಬೇಡ. ನಿನಗೆ ಏನಾದರೂ ನನ್ನಿಂದ ಬೇಕಾದರೆ ಕೊಡಲು ಸಿದ್ಧನಾಗಿರುವೆನು” ಎಂದು ಹೇಳಲು ಧೈರ್ಯ ಇರಬೇಕು. ಪ್ರೀತಿಗೆ ಅಂಜಿಕೆ ತಿಳಿಯದು. ದುರ್ಬಲ ಕೃಶಳಾದ ಹೆಂಗಸೊಬ್ಬಳು ದಾರಿಯಲ್ಲಿ ಹೋಗುತ್ತಿರುವಾಗ ನಾಯಿಯೊಂದು ಬೊಗಳಿದರೆ ತಕ್ಷಣ ಸಮೀಪವಿರುವ ಮನೆಗೆ ನುಗ್ಗುವಳು. ಆದರೆ ಮಾರನೆಯ ದಿನ ತನ್ನ ಮಗುವನ್ನು ಕರೆದುಕೊಂಡು ದಾರಿಯಲ್ಲಿ ಹೋಗುತ್ತಿರುವಳು ಎಂದು ಊಹಿಸಿ, ಸಿಂಹ ಅವಳನ್ನು ಎದುರಿಸಿದರೆ ಆಗ ಅವಳು ಎಲ್ಲಿರುವಳು ಎಂದು ಊಹಿಸುತ್ತೀರಿ? ಮಗುವಿನ ಪ್ರಾಣರಕ್ಷಣೆಗಾಗಿ ಸಿಂಹದ ಎದುರಲ್ಲಿ ಇರುತ್ತಾಳೆ. ಕೊನೆಯದು, ಪ್ರೀತಿಗಾಗಿ ಪ್ರೀತಿಸುವುದು.

ಪ್ರೀತಿಯೊಂದೇ ದೇವರು ಎಂದು ಭಕ್ತ ಕೊನೆಗೆ ತಿಳಿಯುವನು. ದೇವರ ಅಸ್ತಿತ್ವವನ್ನು ಪ್ರಮಾಣೀಕರಿಸುವುದಕ್ಕೆ ಎಲ್ಲಿಗೆ ಹೋಗಬೇಕು? ಪ್ರೇಮವೇ ಎಲ್ಲಕ್ಕಿಂತ ಸ್ಪಷ್ಟವಾಗಿ ಕಾಣುವ ವಸ್ತು. ಈ ಶಕ್ತಿಯೇ ಸೂರ್ಯ ಚಂದ್ರ ತಾರಾವಳಿಗಳನ್ನು ಸಂಚರಿಸುವಂತೆ ಮಾಡುತ್ತಿರುವುದು. ಈ ಪ್ರೀತಿಯೊಂದೇ ಸ್ತ್ರೀ ಪುರುಷ ಪ್ರಾಣಿಗಳಲ್ಲಿ, ಎಲ್ಲಾ ಕಡೆಗಳಲ್ಲಿಯೂ, ಪ್ರತಿಯೊಂದು ಪರಮಾಣುವಿನಲ್ಲಿಯೂ ವ್ಯಕ್ತವಾಗುತ್ತಿವೆ. ಆಕರ್ಷಣ ಮುಂತಾದ ಬಾಹ್ಯ ಶಕ್ತಿಯಲ್ಲಿ ಅದು ವ್ಯಕ್ತವಾಗುತ್ತದೆ. ಈ ಪ್ರಪಂಚದ ಏಕಮಾತ್ರ ಕ್ರಿಯೋತ್ತೇಜಕ ಶಕ್ತಿ ಪ್ರೇಮವೊಂದೇ. ಇದು ಸರ್ವವ್ಯಾಪಿಯಾಗಿರುವುದು. ಇದು ಸ್ವಯಂ ಭಗವಂತನೇ ಆಗಿರುವುದು.\footnote{\engfoot{The Tribune} ಪತ್ರಿಕೆಯಲ್ಲಿ ಪ್ರಕಟವಾದ ವರದಿಯಿಂದ}

