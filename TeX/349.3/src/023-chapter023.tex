
\chapter{ವೇದಾಂತ}

\begin{center}
(೧೮೯೭ ಇಸವಿ ನವೆಂಬರ್​ ೧೨ರಂದು ಲಾಹೋರಿನಲ್ಲಿ ನೀಡಿದ ಉಪನ್ಯಾಸ)
\end{center}

ನಾವು ಸಾಧಾರಣವಾಗಿ ಎರಡು ಜಗತ್ತುಗಳಲ್ಲಿ ಜೀವಿಸುತ್ತೇವೆ. ಒಂದು ಹೊರಗಿನದು, ಮತ್ತೊಂದು ಒಳಗಿನದು. ಪುರಾತನ ಕಾಲದಿಂದಲೂ ಮಾನವ ಪ್ರಗತಿಯು ಈ ಎರಡು ಜಗತ್ತುಗಳಲ್ಲಿಯೂ ಸಮಾನಾಂತರವಾಗಿ ಸಾಗಿ ಬಂದಿದೆ. ಅನ್ವೇಷಣೆಯು ಪ್ರಾರಂಭವಾದುದು ಹೊರಗಿನ ಜಗತ್ತಿನಲ್ಲಿ. ಮಾನವನು ತನ್ನ ಅಂತರಂಗದ ಅಗಾಧ ಪ್ರಶ್ನೆಗಳಿಗೆ ಉತ್ತರಗಳನ್ನು ಮೊಟ್ಟಮೊದಲು ಹುಡುಕಿದುದು ಬಾಹ್ಯ ಪ್ರಕೃತಿಯಲ್ಲಿ. ತನ್ನ ಹೃದಯದಲ್ಲಿ ಉತ್ಪನ್ನವಾದ ಸೌಂದರ್ಯತೃಷೆಯನ್ನೂ, ಭವ್ಯತಾಪಿಪಾಸೆಯನ್ನೂ ತೃಪ್ತಿಪಡಿಸಲು ಸುತ್ತಮುತ್ತಣ ಸನ್ನಿವೇಶದ ಸಹಾಯವನ್ನು ಕೋರಿದನು; ತನ್ನನ್ನೂ ಮತ್ತು ತನ್ನ ಅನುಭವಗಳನ್ನೂ ಸ್ಥೂಲವಾದ ಮತ್ತು ಪಂಚೇಂದ್ರಿಯ ಸುಲಭವಾದ ವಾಸ್ತವಿಕ ರೀತಿಯಲ್ಲಿ ವ್ಯಕ್ತಗೊಳಿಸಲು ಯತ್ನಿಸಿದನು; ಅದರ ಫಲವಾಗಿ ಭವ್ಯವಾದ ಮಹತ್ತಾದ ಉತ್ತರಗಳು ಮೂಡಿದುವು. ಈಶ್ವರನ ಮತ್ತು ಆತನ ಆರಾಧನೆಯ ವಿಚಾರವಾಗಿ ಅದ್ಭುತ ಭಾವನೆಗಳು ಹೊರ ಹೊಮ್ಮಿದವು. ನಿಜವಾಗಿಯೂ ಈ ಬಾಹ್ಯ ಪ್ರಕೃತಿಯ ಅನ್ವೇಷಣೆಯಿಂದ ಅತ್ಯಂತ ಸುಂದರವೂ ಭಾವಪೂರ್ಣವೂ ಆದ ಕಲ್ಪನೆಗಳು ಮೈದೋರಿದುವು. ಆದರೆ ಕೆಲವು ಕಾಲಾನಂತರದಲ್ಲಿ ಆ ಬಹಿರನ್ವೇಷಣೆಯು ಅಂತರ್ಮುಖವಾಗಿ, ಮನುಷ್ಯನ ದೃಷ್ಟಿಗೆ ಹಿಂದೆ ತೋರಿದುದಕ್ಕಿಂತಲೂ ಅತಿಶಯವಾದ ಭವ್ಯವೂ, ಸುಂದರವೂ, ಅನಂತ ವಿಸ್ತೃತವೂ ಆದ ಮತ್ತೊಂದು ಮಹತ್ತರ ಜಗತ್ತು ಗೋಚರವಾಯಿತು. ವೇದಭಾಗವಾದ ಕರ್ಮಕಾಂಡದಲ್ಲಿ ಅದ್ಭುತ ಧಾರ್ಮಿಕ ಭಾವನೆಗಳಿವೆ. ಸೃಷ್ಟಿ ಸ್ಥಿತಿ ಲಯಕರ್ತನಾದ ಏಕಮಹೇಶ್ವರನ ವಿಚಾರವಾಗಿ ಅದ್ಭುತವಾದ ಕಲ್ಪನೆಗಳಿವೆ. ಹೃದಯವನ್ನು ಕಲಕುವ ಭಾಷೆಯಲ್ಲಿ ಆ ಈಶ್ವರನನ್ನು ವರ್ಣಿಸಿದ್ದಾರೆ. ನಿಮ್ಮಲ್ಲನೇಕರಿಗೆ ಋಗ್ವೇದ ಸಂಹಿತೆಯಲ್ಲಿರುವ ಸೃಷ್ಟಿ ವರ್ಣನೆಯ ಮಹಾಶ್ಲೋಕವು (ನಾಸದೀಯ ಸೂಕ್ತ) ನೆನಪಿನಲ್ಲಿರಬಹುದು. ಬಹುಶಃ ಅದಕ್ಕಿಂತಲೂ ಭವ್ಯತರವಾದ ಮತ್ತೊಂದು ಕವನ ಇದುವರೆಗೆ ನಿರ್ಮಿತವಾಗಿಲ್ಲ ಎಂದು ಹೇಳಿದರೆ ಅತಿಶಯೋಕ್ತಿಯಾಗಲಾರದು. ಆ ಕವನದಲ್ಲಿ ಚಿತ್ರಿತವಾಗಿರುವುದು ಭವ್ಯತೆಯ ಬಾಹ್ಯವರ್ಣನೆ ಮಾತ್ರ. ಅದರಲ್ಲಿ ನಮಗೆ ಸ್ಥೂಲತೆಯ ದರ್ಶನವಾಗುತ್ತದೆ. ಅದರಲ್ಲಿ ಏನೋ ಒಂದು ವಿಧವಾದ ಭೌತವಸ್ತುವಿನ ಛಾಯೆಯಿದೆ. ಎಷ್ಟೇ ಭವ್ಯವಾಗಿದ್ದರೂ ಅದರಲ್ಲಿ ಅನಂತವನ್ನು ಸಾಂತದ ಭಾಷೆಯಲ್ಲಿ ಹೇಳಲು ಪ್ರಯತ್ನಿಸಿದ್ದಾರೆ. ಅದರಲ್ಲಿರುವುದು ಬಾಹ್ಯಶಕ್ತಿಯ ಅನಂತತೆಯೇ ಹೊರತು ಭಾವದ್ದಲ್ಲ. ಆದ್ದರಿಂದಲೇ ವೇದಗಳ ಎರಡನೆಯ ಭಾಗವಾದ ಜ್ಞಾನಕಾಂಡದಲ್ಲಿ ಅನ್ವೇಷಣೆಯ ಅಥವಾ ತತ್ತ್ವ ವಿಚಾರದ ವಿಧಾನ ನಮಗೆ ಇನ್ನೊಂದು ಪರಿಯಾಗಿತೋರುವುದು. ಪ್ರಥಮತಃ ಜಡ ಪ್ರಪಂಚದಿಂದಲೂ ಬಾಹ್ಯ ಪ್ರಕೃತಿಯಿಂದಲೂ ಜೀವನದ ಸಮಸ್ಯೆಗಳಿಗೆ ಪರಿಹಾರವನ್ನು ಹುಡುಕಲಾಯಿತು. \textbf{(ಯಸ್ಯೈತೇ ಹಿಮವಂತೋ ಮಹಿತ್ವಾ} ಯಾರ ಮಹಿಮೆಯನ್ನು ಹಿಮಾಲಯಗಳು ಸಾರುತ್ತಿವೆಯೋ). ಇದೇನೋ ಮಹೋನ್ನತ ಭಾವನೆ. ಆದರೆ ಭರತಖಂಡವು ಅಷ್ಟರಿಂದಲೇ ತೃಪ್ತಗೊಳ್ಳುವ ಹಾಗಿರಲಿಲ್ಲ. ಭಾರತೀಯರ ಮನಸ್ಸು ಬಾಹ್ಯದಿಂದ ಆಂತರ್ಯದ ಕಡೆಗೆ, ಜಡದಿಂದ ಚೇತನದ ಕಡೆಗೆ ತಿರುಗಿತು. “ಮನುಷ್ಯನು ಮೃತನಾದ ಮೇಲೆ ಏನಾಗುತ್ತಾನೆ?” ಎಂಬ ಆರ್ತನಾದ ಮೇಲೆದ್ದಿತು \textbf{ಅಸ್ತೀತ್ಯೇಕೇ ನಾಯಮಸ್ತೀತಿ ಚೈಕೇ,} “ಅವನು ಇರುತ್ತಾನೆಂದು ಕೆಲವರು ಹೇಳುತ್ತಾರೆ, ನಾಶವಾದನೆಂದು ಉಳಿದವರು ಹೇಳುತ್ತಾರೆ. ಓ ಯಮರಾಜನೆ ಸತ್ಯವೇನು?” ಇಲ್ಲಿ ಸಂಪೂರ್ಣ ಬೇರೊಂದು ವಿಧಾನವನ್ನೇ ನೋಡುತ್ತೇವೆ. ಭಾರತೀಯರು ಬಾಹ್ಯಪ್ರಕೃತಿಯಿಂದ ಪಡೆಯಬೇಕಾದುದನ್ನೆಲ್ಲಾ ಪಡೆದರು. ಆದರೆ ಅಷ್ಟರಿಂದಲೇ ತೃಪ್ತರಾಗಲಿಲ್ಲ. ಇನ್ನೂ ಮುಂದೆ ಅನ್ವೇಷಣೆ ನಡೆಸಿದರು. ಅಂತರ್ಜಗತ್ತಿನ ಹೃದಯವನ್ನು ಪ್ರವೇಶಮಾಡಿ ತಮ್ಮ ಆತ್ಮದ ಸಹಾಯದಿಂದಲೇ ಸಮಸ್ಯೆಗಳನ್ನು ಬಿಡಿಸಲು ಸಾಹಸ ಮಾಡಿದರು.

ಮೇಲೆ ಹೇಳಿದ ವೇದಗಳ ಭಾಗಕ್ಕೆ ಉಪನಿಷತ್ತುಗಳು, ವೇದಾಂತ, ಆರಣ್ಯಕ ಅಥವಾ ರಹಸ್ಯವೆಂದು ಹೆಸರು. ಇಲ್ಲಿ ಧರ್ಮವು ತನ್ನ ಬಾಹ್ಯಾಚಾರ, ವೇಷಗಳನ್ನೆಲ್ಲಾ ಸಂಪೂರ್ಣವಾಗಿ ಕಿತ್ತೊಗೆದಿರುವುದನ್ನು ಕಾಣುತ್ತೇವೆ. ಇಲ್ಲಿ ಆತ್ಮದ ವಿಷಯಗಳನ್ನು ಜಡದ ಭಾಷೆಯಲ್ಲಿ ಹೇಳದೆ ಆತ್ಮದ ಭಾಷೆಯಲ್ಲಿಯೇ ಹೇಳಿದೆ. ಅತ್ಯಂತ ಸೂಕ್ಷ್ಮವಾದುದನ್ನು ಅತ್ಯಂತ ಸೂಕ್ಮವಾದ ಭಾಷೆಯಲ್ಲಿಯೇ ಹೇಳಿರುವರು. ಇಲ್ಲಿ ಒರಟಾದುದು, ಜಡವಾದುದು ಯಾವುದೂ ಇಲ್ಲ. ಇಲ್ಲಿ ಸತ್ಯ ಮತ್ತು ಲೌಕಿಕ ವಸ್ತುಗಳ ನಡುವೆ ರಾಜಿಯಾಗಿಲ್ಲ. ಧೈರ್ಯವಾಗಿ, ಸಾಹಸಪೂರ್ಣವಾಗಿ, ವರ್ತಮಾನ ಕಾಲದ ಭಾವನೆಗೆ ನಿಲುಕದ, ಸಂಕುಚಿತ ಭಾವಗಳಿಗೆ ಅತೀತವಾಗಿ, ಎಳ್ಳಿನಿತೂ ಭಯವಿಲ್ಲದೆ, ದಾಕ್ಷಿಣ್ಯವಿಲ್ಲದೆ, ಆತ್ಮದ ವಿಚಾರವಾದ ಪಾವನ ಮಹಾಸತ್ಯಗಳನ್ನು ಮಾನವ ಕುಲಕ್ಕೆ ಉಚ್ಚಕಂಠದಿಂದ ಉದ್ಘೋಷಿಸುತ್ತಾ, ಉಪನಿಷತ್ತಿನ ಆ ತ್ರಿವಿಕ್ರಮ ಮಾನಸರಾದ ಮಹರ್ಷಿಗಳು, ಮಹಿಮಾಮಯರಾಗಿಯೇ ನಿಂತಿದ್ದಾರೆ. ಓ ನನ್ನ ದೇಶಬಾಂಧವರೇ, ಇಂದು ನಿಮ್ಮೆದುರು ಆ ಸತ್ಯ ಸಾಹಸಗಳನ್ನು ಕುರಿತು ಕೈಲಾದ ಮಟ್ಟಿಗೆ ಪ್ರಸ್ತಾಪಿಸಬೇಕೆಂದು ನಾನಿಲ್ಲಿ ನಿಂತಿದ್ದೇನೆ. ಏಕೆಂದರೆ ವೇದಗಳ ಜ್ಞಾನಕಾಂಡವು ಕೂಡ ನಿಜವಾಗಿಯೂ ಒಂದು ಮಹಾಸಮುದ್ರ. ಅದರಲ್ಲಿ ಸ್ವಲ್ಪಭಾಗವನ್ನು ತಿಳಿಯಲು ಬಹು ಜನ್ಮಗಳು ಬೇಕು. ಶ‍್ರೀ ರಾಮಾನುಜರು ಹೇಳಿರುವಂತೆ ಉಪನಿಷತ್ತುಗಳು ನಿಜವಾಗಿಯೂ ವೇದಪುರುಷನ ಭುಜ ಶಿಖರಗಳು. ಉಪನಿಷತ್ತುಗಳೇ ಆಧುನಿಕ ಭಾರತ ವರ್ಷಕ್ಕೆ ನಿಜವಾದ ಧರ್ಮಗ್ರಂಥಗಳು. ವೇದಗಳ ಕರ್ಮಕಾಂಡದ ವಿಷಯವಾಗಿ ಹಿಂದೂಗಳಿಗೆ ಬಹಳ ಗೌರವವಿದೆ. ಆದರೂ ಬಹುಕಾಲದಿಂದಲೂ ಶ್ರುತಿ ಎಂಬ ಪದ ಉಪನಿಷತ್ತುಗಳಿಗೆ ಮಾತ್ರವೇ ಅನ್ವಯಿಸುತ್ತದೆ. ಅಷ್ಟೇ ಅಲ್ಲದೆ ಶ್ರೇಷ್ಠರಾದ ನಮ್ಮ ದಾರ್ಶನಿಕರೆಲ್ಲರೂ, ವ್ಯಾಸ, ಪತಂಜಲಿ, ಗೌತಮ ಮತ್ತು ತತ್ತ್ವಶಾಸ್ತ್ರ ಪಿತಾಮಹನಾದ ಕಪಿಲ ಮಹರ್ಷಿಗಳೆಲ್ಲರೂ, ತಮ್ಮ ಪ್ರಮಾಣಗಳನ್ನು ಉಪನಿಷತ್ತುಗಳಿಂದಲೇ ಆರಿಸಿಕೊಂಡಿದ್ದಾರೆ, ಬೇರೆಲ್ಲಿಂದಲೂ ಅಲ್ಲ. ಏಕೆಂದರೆ ಉಪನಿಷತ್ತುಗಳು ಸನಾತನ ಸತ್ಯದ ಗಣಿಗಳು.

ಕೆಲವು ಸತ್ಯಗಳು ದೇಶಕಾಲಬದ್ಧವಾದವು. ಅವು ವಿಶಿಷ್ಟ ಸನ್ನಿವೇಶಗಳಿಗೆ ಮತ್ತು ವಿಶಿಷ್ಟಕಾಲಕ್ಕೆ ಮಾತ್ರ ಅನ್ವಯಿಸುತ್ತವೆ. ಮತ್ತೆ ಕೆಲವು ಸತ್ಯಗಳು ಮಾನವ ಸ್ವಭಾವವನ್ನು ಅವಲಂಬಿಸಿದಂಥವು. ಆದ್ದರಿಂದ ಅವು ಮಾನವನಿರುವ ತನಕ ಇರಲೇಬೇಕು. ಆ ಸತ್ಯಗಳು ವಿಶ್ವಮಾನ್ಯವಾದವು, ಶಾಶ್ವತವಾದವು, ಹಾಗೂ ಸನಾತನವಾದವು. ಭರತಖಂಡದ ಜನಜೀವನದಲ್ಲಿ ಬಹಳ ಬದಲಾವಣೆಗಳಾಗಿದ್ದರೂ, ನಮ್ಮ ಸಮಾಜದ ರೀತಿ ನೀತಿಗಳು, ಉಡುಗೆ ತೊಡುಗೆಗಳು, ಭೋಜನ ಮತ್ತು ಪೂಜಾ ವಿಧಾನಗಳು ವ್ಯತ್ಯಾಸವಾಗಿದ್ದರೂ, ವೇದಾಂತದ ಈ ಮಹೋತ್ತುಂಗ ಭಾವಗಳು ಮಾತ್ರ ಭವ್ಯವಾಗಿ ಅಚಲವಾಗಿ ಅಜೇಯವಾಗಿ ಅಮರವಾಗಿ ಅಚ್ಯುತಾನಂತವಾಗಿ ಇನ್ನೂ ನಿಂತಿವೆ. ಆದರೂ ಉಪನಿಷತ್ತಿನಲ್ಲಿ ಪ್ರಸ್ಫುಟವಾಗಿರುವ ಭಾವನೆಗಳಿಗೆಲ್ಲ ಕರ್ಮಕಾಂಡದಲ್ಲಿಯೇ ಅಂಕುರಾರ್ಪಣವಾಗಿದೆ. ಉದಾಹರಣೆಗೆ ವೇದಾಂತ ದರ್ಶನಗಳೆಲ್ಲವೂ ಬೋಧಿಸುವ ಬ್ರಹ್ಮಾಂಡದ ವಿಚಾರವೂ, ಭರತಖಂಡದ ದಾರ್ಶನಿಕರೆಲ್ಲರೂ ಸ್ವೀಕರಿಸುವ ಮನಶ್ಶಾಸ್ತ್ರೀಯ ವಿಚಾರಗಳು ಕರ್ಮಕಾಂಡದಲ್ಲಿಯೇ ಉಕ್ತವಾಗಿವೆ. ಆದ್ದರಿಂದ ನಾವು ವೇದಾಂತದ ವಿಚಾರವಾಗಿಯಾಗಲೀ, ಉಪನಿಷತ್ತಿನ ವಿಚಾರವಾಗಿಯಾಗಲೀ, ಮಾತನಾಡುವ ಮೊದಲು ಕರ್ಮಕಾಂಡದ ವಿಚಾರವಾಗಿ ಒಂದೆರಡು ಮಾತುಗಳನ್ನು ಹೇಳಬೇಕಾಗಿದೆ – ಎಲ್ಲಕ್ಕಿಂತಲೂ ಮೊದಲು ‘ವೇದಾಂತ’ ಎಂಬ ಪದವನ್ನು ನಾನು ಯಾವ ಅರ್ಥದಲ್ಲಿ ಉಪಯೋಗಿಸುತ್ತಿದ್ದೇನೆ ಎಂಬುದು.

ಈಗಿನ ಅನೇಕರು ತಪ್ಪು ತಿಳುವಳಿಕೆಯಿಂದ ‘ವೇದಾಂತ’ ಎಂಬ ಪದವನ್ನು ಅದ್ವೈತ ದರ್ಶನ ಎಂಬ ಏಕಮಾತ್ರ ಅರ್ಥದಲ್ಲಿ ಪದೇ ಪದೇ ಉಪಯೋಗಿಸುತ್ತಾರೆ. ಆಧುನಿಕ ಭಾರತದ ವೇದಾಂತದ ಎಲ್ಲ ಶಾಖೆಯವರೂ ಮೂರು ಪ್ರಸ್ಥಾನಗಳ ಅಧ್ಯಯನಕ್ಕೆ ಸಮಾನವಾದ ಪ್ರಾಮುಖ್ಯವನ್ನು ನೀಡುತ್ತಾರೆ. ಮೊದಲನೆಯದು ಶ್ರುತಿ, ಎಂದರೆ ಉಪನಿಷತ್ತು. ಎರಡನೆಯದಾಗಿ ವ್ಯಾಸ ಸೂತ್ರಗಳಿವೆ. ಹಿಂದಿನ ಎಲ್ಲ ದರ್ಶನಗಳ ಆತ್ಯಂತಿಕ ರೂಪವಾಗಿರುವುದರಿಂದ ವ್ಯಾಸ ಸೂತ್ರಗಳು ಅತ್ಯಧಿಕ ಪ್ರಾಮುಖ್ಯವನ್ನು ಪಡೆದಿವೆ. ವಿವಿಧ ದರ್ಶನಗಳು ಪರಸ್ಪರ ವಿರೋಧಿಗಳಲ್ಲ. ಅವು ಒಂದಕ್ಕೊಂದು ಪೂರಕವಾಗಿ, ಒಂದರಮೇಲೊಂದು ನಿಂತಿವೆ ಮತ್ತು ಒಂದರಿಂದೊಂದು ವಿಕಾಸವಾಗಿವೆ. ಅವು ವ್ಯಾಸಸೂತ್ರದಲ್ಲಿ ತಮ್ಮ ಪರಿಪೂರ್ಣತೆಯನ್ನು ಮುಟ್ಟುತ್ತವೆ. ಮೂರನೆಯದಾಗಿ, ಉಪನಿಷತ್ತುಗಳಿಗೆ ಮತ್ತು ವೇದಾಂತದ ಅಸಾಧಾರಣ ಸತ್ಯಗಳನ್ನು ಕ್ರಮಬದ್ಧಗೊಳಿಸುವ ಬ್ರಹ್ಮಸೂತ್ರಗಳಿಗೆ ಮಧ್ಯೆ ವೇದಾಂತದ ದಿವ್ಯ ವ್ಯಾಖ್ಯಾನವಾದ ಭಗವದ್ಗೀತೆಯು ನಿಂತಿದೆ.

ಆದ್ದರಿಂದ ಸಂಪ್ರದಾಯವಂತರೆಂದು ಕರೆಯಿಸಿಕೊಳ್ಳಲು ಆಸೆಪಡುವ ಎಲ್ಲ\break ಪಂಥದವರೂ, ಉಪನಿಷತ್ತು ಭಗವದ್ಗೀತೆ ಮತ್ತು ವ್ಯಾಸಸೂತ್ರಗಳನ್ನು ಪ್ರಮಾಣ ಗ್ರಂಥಗಳೆಂದು ಭಾವಿಸಿಯೇ ಭಾವಿಸುತ್ತಾರೆ. ದ್ವೈತಿಯಾಗಲೀ, ವಿಶಿಷ್ಟಾದ್ವೈತಿಯಾಗಲೀ, ಅದ್ವೈತಿಯಾಗಲೀ, ಪ್ರತಿಯೊಬ್ಬರಿಗೂ ಆ ಮೂರು ಗ್ರಂಥಗಳು ಪ್ರಮಾಣ. ಶಂಕರಾಚಾರ್ಯ, ರಾಮಾನುಚಾರ್ಯ, ಮಧ್ವಾಚಾರ್ಯ, ವಲ್ಲಭಾಚಾರ್ಯ, ಚೈತನ್ಯ ಅಥವಾ ಇನ್ನಾವ ಆಚಾರ್ಯರೇ ಆಗಲಿ, ಹೊಸ ಪಂಥವನ್ನು ಸ್ಥಾಪಿಸಬೇಕೆಂದು ಮನಸ್ಸು ಮಾಡಿದಾಗ, ಆ ಮೂರು ಪ್ರಸ್ಥಾನಗಳನ್ನು ತೆಗೆದುಕೊಂಡು ಹೊಸ ವ್ಯಾಖ್ಯಾನಗಳನ್ನು ಬರೆದಿರುತ್ತಾರೆ, ಎಂಬುದು ಎಲ್ಲರಿಗೂ ತಿಳಿದ ವಿಷಯವಾಗಿದೆ. ಆದ್ದರಿಂದ ಉಪನಿಷತ್ತಿನಿಂದ ಮೂಡಿದ ಒಂದು ದರ್ಶನಕ್ಕೆ ಮಾತ್ರ ವೇದಾಂತವೆಂದು ಹೆಸರು ಕೊಡುವುದು ತಪ್ಪು. ‘ವೇದಾಂತ’ ಎಂಬ ಪದವು ಎಲ್ಲ ದರ್ಶನಗಳನ್ನು ಒಳಗೊಳ್ಳುತ್ತದೆ. ದ್ವೈತಿಗಳು, ವಿಶಿಷ್ಟಾದ್ವೈತಿಗಳೂ ಅದ್ವೈತಿಗಳಂತೆಯೇ ವೇದಾಂತಿಗಳೆಂಬ ಹೆಸರಿಗೆ ಮಾನ್ಯರಾಗಿದ್ದಾರೆ. ಹೆಚ್ಚೇನು? ನಾವು ಉಪಯೋಗಿಸುವ ‘ಹಿಂದೂ’ ಎಂಬ ಪದ ‘ವೇದಾಂತಿ’ ಎಂಬ ಪದಕ್ಕೆ ಪರ್ಯಾಯವೆಂದು ಹೇಳಿದರೆ ಹೆಚ್ಚು ತಪ್ಪಾಗಲಾರದು. ಇನ್ನೊಂದು ವಿಷಯವನ್ನು ನೀವು ನೆನಪಿನಲ್ಲಿಡಬೇಕು. ಈ ಮೂರು ದರ್ಶನಗಳು ಬಹು ಪುರಾತನ ಕಾಲದಿಂದಲೂ ಭರತಖಂಡದಲ್ಲಿವೆ. ಶಂಕರಾಚಾರ್ಯರು ಅದ್ವೈತ ವೇದಾಂತವನ್ನು ಹೊಸತಾಗಿ ಕಂಡುಹಿಡಿದರೆಂದು ತಿಳಿಯಬೇಡಿ. ಅವರು ಹುಟ್ಟುವುದಕ್ಕೆ ಬಹಳ ಮೊದಲೇ ಅದು ರೂಢಿಯಲ್ಲಿತ್ತು. ಶಂಕರಾಚಾರ್ಯರು ಅದರ ಕಟ್ಟಕಡೆಯ ಪ್ರತಿನಿಧಿ ಮಾತ್ರ. ಅದರಂತೆಯೇ ರಾಮಾನುಜರ ದರ್ಶನವೂ ಅವರು ಲೋಕಕ್ಕೆ ಬರುವ ಶತಮಾನಗಳ ಮೊದಲೇ ಇತ್ತು. ಈ ವಿಷಯವು ಅವರು ಬರೆದಿರುವ ವ್ಯಾಖ್ಯಾನಗಳಿಂದಲೇ ತಿಳಿದುಬರುತ್ತದೆ. ಹಾಗೆಯೇ ಇತರ ಎಲ್ಲ ದ್ವೈತ ಸಿದ್ಧಾಂತಗಳೂ ಅವುಗಳು ದರ್ಶನಬದ್ಧವಾಗುವ ಮುನ್ನವೇ ರೂಢಿಯಲ್ಲಿದ್ದವು, ಮತ್ತು ಈ ಎಲ್ಲ ಸಿದ್ಧಾಂತಗಳು ಒಂದಕ್ಕೊಂದು ವಿರೋಧಿಗಳಲ್ಲವೆಂದು ನನ್ನ ಅಲ್ಪಮತಿಗೆ ತಿಳಿದುಬಂದಿದೆ.

ಹೇಗೆ ಷಡ್ದರ್ಶನಗಳು ಒಂದೇ ಮಹಾತತ್ತ್ವದ ಕ್ರಮ ವಿಕಾಸಗಳಂತಿವೆಯೋ, ಮೃದುಸ್ವರದಲ್ಲಿ ಪ್ರಾರಂಭವಾದ ಸಂಗೀತವು ಅದ್ವೈತ ಮಹಾಗಾನದಲ್ಲಿ ಹೇಗೆ ಕೊನೆಗೊಂಡಿದೆಯೋ, ಹಾಗೆಯೇ ಮಾನವನ ಮನಸ್ಸು ದ್ವೈತ ವಿಶಿಷ್ಟಾದ್ವೈತಗಳೆಂಬ ಸೋಪಾನಗಳನ್ನೇರಿ, ಸರ್ವೋನ್ನತ ದರ್ಶನವಾದ ಅದ್ವೈತ ಶಿಖರವನ್ನು ಸೇರಿದೆ. ಆದ್ದರಿಂದ ಈ ಮೂರು ಸಿದ್ಧಾಂತಗಳೂ ಪರಸ್ಪರ ವಿರೋಧಿಗಳಲ್ಲ. ಅವುಗಳು ಪರಸ್ಪರ ವಿರೋಧಿಗಳೆಂದು ಅನೇಕರು ತಪ್ಪಾಗಿ ತಿಳಿದುಕೊಂಡಿದ್ದಾರೆ. ಅದ್ವೈತಿಗಳು ಉಪನಿಷತ್ತಿನ ಅದ್ವೈತ ಭಾಗವನ್ನು ಮಾತ್ರವೇ ವಿಕಾರಗೊಳಿಸದೆ ಉಳಿಸಿಕೊಂಡು ದ್ವೈತ ವಿಶಿಷ್ಟಾದ್ವೈತ ಭಾಗಗಳಿಗೆ ಬಹು ಕಷ್ಟದಿಂದ ತಮ್ಮ ಸಿದ್ಧಾಂತದ ಅರ್ಥ ಬರುವಂತೆ ವ್ಯಾಖ್ಯಾನಗಳನ್ನು ಆರೋಪಿಸಿದ್ದಾರೆ. ಹಾಗೆಯೇ ದ್ವೈತಿಗಳು ದ್ವೈತಭಾಗವನ್ನು ಮಾತ್ರ ತೆಗೆದುಕೊಂಡು ಅದ್ವೈತಭಾಗಗಳಿಗೆ ಬಹುಪ್ರಯಾಸದಿಂದ ದ್ವೈತಾರೋಪಣೆ ಮಾಡಿದ್ದಾರೆ. ಅವರೆಲ್ಲರೂ ಮಹಾಪುರುಷರೇ ಹೌದು; ನಮ್ಮ ಗುರುಗಳೇ ಹೌದು. ಆದರೂ ಗುರುವಿನಲ್ಲಾದರೂ ದೋಷಗಳಿದ್ದರೆ ಹೇಳಲೇಬೇಕು ಎಂಬ ಹೇಳಿಕೆ ಇದೆ. ನನ್ನ ಅಭಿಪ್ರಾಯದಲ್ಲಿ ಅವರೆಲ್ಲರೂ ತಪ್ಪು ಹಾದಿಯನ್ನೇ ಹಿಡಿದಿದ್ದರು. ಆ ಗ್ರಂಥಪೀಡನೆ ನಮಗೇಕೆ? ಆ ಧಾರ್ಮಿಕ ಕೃತ್ರಿಮತೆ ನಮಗೇಕೆ? ಆ ವ್ಯಾಕರಣದ ಕಚ್ಚಾಟವೇಕೆ? ಇಲ್ಲದ ಅರ್ಥವನ್ನು ಗ್ರಂಥಗಳಿಗೆ ನಾವೇಕೆ ಆರೋಪಿಸಬೇಕು? ನಾವೆಂದು ಅಧಿಕಾರ ಭೇದದ ಅದ್ಭುತ ರಹಸ್ಯವನ್ನು ಗ್ರಹಿಸುತ್ತೇವೆಯೋ, ಅಂದು ನಮಗೆ ಆ ಗ್ರಂಥಗಳಲ್ಲಿರುವ ಸಮನ್ವಯ ಭಾವ ಸುಲಭವೂ ಸರಳವೂ ಆಗುತ್ತದೆ.

\textbf{“ಕಸ್ಮಿನ್ನು ಭಗವೋ ವಿಜ್ಞಾತೇ ಸರ್ವಮಿದಂ ವಿಜ್ಞಾತಂ ಭವತಿ”} (ಯಾವುದನ್ನು ತಿಳಿಯುವುದರಿಂದ ಎಲ್ಲವೂ ತಿಳಿಯಲ್ಪಡುತ್ತದೆ) ಎಂಬುದೇ ಉಪನಿಷತ್ತುಗಳ ಮಹಾ ಪ್ರಶ್ನೆ ಮತ್ತು ಸಮಸ್ಯೆ. ಆಧುನಿಕ ಭಾಷೆಯಲ್ಲಿ ಹೇಳುವುದಾದರೆ, ಇತರ ಎಲ್ಲ ದರ್ಶನಗಳ ಗುರಿಯಂತೇ ಉಪನಿಷತ್ತುಗಳ ಗುರಿಯೂ ಸೃಷ್ಟಿಯ ವೈವಿಧ್ಯ ವಿರೋಧಗಳ ಹಿಂದೆ ಇರುವ ಏಕತೆಯನ್ನು ತಿಳಿಯುವುದಾಗಿದೆ. ಜ್ಞಾನವೆಂದರೇನು? ಅನೇಕತೆಯಲ್ಲಿ ಏಕತೆಯನ್ನು ಗ್ರಹಿಸುವುದೇ ಜ್ಞಾನ. ಪ್ರತಿಯೊಂದು ವಿಜ್ಞಾನಶಾಸ್ತ್ರಕ್ಕೂ ಇದೇ ಆಧಾರ ಮತ್ತು ಗಮ್ಯ. ‘ಅನೇಕ’ ದಲ್ಲಿ ‘ಏಕ’ ವನ್ನು ಕಂಡುಹಿಡಿಯುವುದೇ ಮಾನವ ವಿದ್ಯೆಯ ಸರ್ವಸ್ವ. ಎಲ್ಲೋ ಕೆಲವು ವೈವಿಧ್ಯ ಪೂರ್ಣ ವಿಷಯಗಳ ಅಂತರಾಳದಲ್ಲಿರುವ ಏಕತೆಯನ್ನು ತಿಳಿಯುವುದೇ ವಿಜ್ಞಾನ ಶಾಸ್ತ್ರಗಳ ಕಾರ್ಯ ಎಂದ ಮೇಲೆ, ಅಸಂಖ್ಯ ಶಾಖೋಪಶಾಖೆಗಳುಳ್ಳ, ನಾಮರೂಪಮಯವಾದ ಸಮಸ್ತ ವಿಶ್ವದ ಅಂತರಾಳದಲ್ಲಿರುವ ಐಕ್ಯಭಾವವನ್ನು ತಿಳಿಯುವುದು ಪರಮಾದ್ಭುತ ಕಾರ್ಯವಾಗುತ್ತದೆ ಮತ್ತು ಕಷ್ಟತರವಾಗುತ್ತದೆ. ಜಡಚೇತನಗಳೆಂಬ ಭೇದಭಾವದಿಂದಲೂ, ಸಂಪೂರ್ಣ ಗುಣ ವ್ಯತ್ಯಾಸವುಳ್ಳ ತರತರದ ಪದಾರ್ಥಗಳಿಂದಲೂ, ಒಂದನ್ನೊಂದು ವಿರೋಧಿಸುವ ಮತ್ತು ಒಂದರಿಂದೊಂದು ಬೇರೆಯಾಗಿ ಇರುವ ಅನೇಕ ಭಿನ್ನಭಾವಗಳಿಂದಲೂ, ಭಿನ್ನರೂಪಗಳಿಂದಲೂ, ಅಸಂಖ್ಯಲೋಕಗಳಿಂದಲೂ ಕೂಡಿರುವ ಈ ಬ್ರಹ್ಮಾಂಡದಲ್ಲಿ ಏಕಮೇವಾದ್ವಿತೀಯವನ್ನು ಕಂಡುಹಿಡಿಯುವುದೇ ಉಪನಿಷತ್ತು ಕೈಗೊಂಡಿರುವ ಮಹಾಕಾರ್ಯ. ಆ ಸಿದ್ಧಾಂತವನ್ನು ಉಪನಿಷತ್ತುಗಳು ಅರುಂಧತಿ ನ್ಯಾಯದ ಪ್ರಕಾರ ಸಾಧಿಸಿವೆ. ಬಹಳ ಕಿರಿದಾದ ಅರುಂಧತೀ ನಕ್ಷತ್ರವನ್ನು ತೋರಿಸಬೇಕಾದರೆ ಅದರ ಬಳಿಯಿರುವ ಹಿರಿದಾದ ಮತ್ತು ಕಾಂತಿಯುಕ್ತವಾದ ತಾರೆಗಳನ್ನು ತೋರಿಸಿ, ತರುವಾಯ ದೃಷ್ಟಿಯನ್ನು ಅರುಂಧತಿಯ ಕಡೆಗೆ ಕ್ರಮವಾಗಿ ತಿರುಗಿಸುತ್ತಾರೆ. ಉಪನಿಷತ್ತುಗಳು ತಮ್ಮ ಕಾರ್ಯವನ್ನು ಹಾಗೆಯೇ ಸಾಧಿಸಿವೆ. ಕೆಲವು ಗ್ರಂಥಭಾಗಗಳನ್ನು ತೆಗೆದುಕೊಂಡರೆ ಸಾಕು, ನಾನು ಹೇಳುವ ಮಾತಿನ ಅರ್ಥ ವಿಶದವಾಗುತ್ತದೆ. ಆ ಗ್ರಂಥಗಳಲ್ಲಿ ಪ್ರತಿಯೊಂದು ಅಧ್ಯಾಯವೂ ದ್ವೈತಬೋಧೆ ಅಥವಾ ಉಪಾಸನೆಯಿಂದ ಪ್ರಾರಂಭವಾಗುತ್ತದೆ. ಮೊದಲು ಸೃಷ್ಟಿ, ಸ್ಥಿತಿ ಲಯಕರ್ತನಾದ ಈಶ್ವರನ ವಿಚಾರವು ಗೋಚರವಾಗುತ್ತದೆ. ಆತನು ಆರಾಧ್ಯನು, ರಕ್ಷಕನು, ಸ್ವಾಮಿಯು, ಒಳಗಿನ ಮತ್ತು ಹೊರಗಿನ ಎರಡು ಜಗತ್ತುಗಳನ್ನು ನಡೆಸುವವನು. ಆದರೂ ಅವನು ಪ್ರಕೃತಿದೂರನು ಮತ್ತು ಅತೀತನು. ಇನ್ನೂ ಸ್ವಲ್ಪ ಮುಂದುವರಿದರೆ ಅದೇ ಉಪನಿಷತ್ತು ಈಶ್ವರನು ಜಗತ್ತಿನ ಹೊರಗಿಲ್ಲ, ಎಲ್ಲೆಲ್ಲಿಯೂ ಅಂತರ್ಯಾಮಿಯಾಗಿದ್ದಾನೆ ಎಂದು ಬೋಧಿಸುತ್ತದೆ. ತುದಿಯಲ್ಲಿ ಉಪನಿಷತ್ತು ಮೇಲೆ ಹೇಳಿದ ಎಲ್ಲಾ ಭಾವಗಳನ್ನು ಪರಿತ್ಯಜಿಸಿ, ಇರುವುದೆಲ್ಲಾ ಬ್ರಹ್ಮ ಎಂಬ ತತ್ತ್ವವನ್ನು ಬೋಧಿಸುತ್ತದೆ. ಕಡೆಯಲ್ಲಿ ಬ್ರಹ್ಮವೂ ಮನುಷ್ಯನ ಆತ್ಮವೂ ಒಂದೇ ಎಂದು ಉಕ್ತವಾಗುತ್ತದೆ. \textbf{“ತತ್ತ್ವಮಸಿ ಶ್ವೇತಕೇತೋ”} (ಶ್ವೇತಕೇತುವೇ ನೀನೇ ಅದು). ಈ ತತ್ತ್ವಬೋಧೆಯಲ್ಲಿ ಒಂದಿನಿತೂ ಭಿನ್ನಾಭಿಪ್ರಾಯದ ಭೀತಿಯಿಲ್ಲ. ಸತ್ಯವನ್ನು ಧೀರಸತ್ಯವನ್ನು ಧೀರಭಾಷೆಯಲ್ಲಿ ಹೇಳಿದ್ದಾರೆ. ಆ ಸತ್ಯವನ್ನು ಅವರಂತೆಯೇ ಧೀರರಾಗಿ ಬೋಧಿಸಲು ನಾವು ಹಿಂಜರಿಯಬಾರದು. ಈಶ್ವರ ಪ್ರಸಾದದಿಂದ ನಾನು ಆ ಸತ್ಯವನ್ನು ಧೈರ್ಯವಾಗಿ ಬೋಧಿಸುವ ಧೀರ ಬೋಧಕನಾಗಿರುವೆನೆಂದು ನಾನು ನಂಬಿದ್ದೇನೆ.

ಮೊದಲು ನಾವು ತಿಳಿಯಬೇಕಾಗಿರುವ, ಎಲ್ಲ ವೇದಾಂತ ಸಿದ್ಧಾಂತಗಳಿಗೂ ಸಮಾನವಾಗಿರುವ, ಎರಡು ವಿಷಯಗಳಿವೆ–ಒಂದು ಮನಶ್ಶಾಸ್ತ್ರಕ್ಕೆ ಸಂಬಂಧಿಸಿದುದು ಮತ್ತೊಂದು ವಿಶ್ವರಚನೆಗೆ ಸಂಬಂಧಿಸಿದುದು. ಮೊದಲು ಎರಡನೆಯದನ್ನು ತೆಗೆದುಕೊಳ್ಳುತ್ತೇನೆ. ಇಂದು ವಿಜ್ಞಾನ ಶಾಸ್ತ್ರವು ಕಂಡುಹಿಡಿಯುತ್ತಿರುವ ಹೊಸ ಸಂಗತಿಗಳು ನಮಗೆ ದಿಗ್ಭ್ರಮೆ ಹಿಡಿಸುತ್ತಿವೆ. ಅದು ನಮ್ಮ ಕಣ್ತೆರೆದು ನಾವೆಂದೂ ಕಾಣದ ಕನಸು ನಮಗೆ ಕಾಣುವಂತೆ ಮಾಡುತ್ತಿದೆ. ಆದರೆ ಇವುಗಳಲ್ಲನೇಕವು ಬಹಳ ಹಿಂದೆಯೇ ತಿಳಿದಿದ್ದ ವಿಷಯಗಳ ಪುನರಾವಿಷ್ಕಾರವಷ್ಟೆ. ವಿವಿಧ ಶಕ್ತಿಗಳ ನಡುವೆ ಏಕತೆಯಿದೆಯೆಂಬುದನ್ನು ವಿಜ್ಞಾನವು ಇತ್ತೀಚೆಗಷ್ಟೇ ಕಂಡುಹಿಡಿದಿದೆ. ಉಷ್ಣ, ಅಯಸ್ಕಾಂತ ಶಕ್ತಿ, ವಿದ್ಯುಚ್ಛಕ್ತಿ ಇವು ಬೇರೆ ಬೇರೆಯಾಗಿ ತೋರಿದರೂ, ಒಂದನ್ನು ಮತ್ತೊಂದನ್ನಾಗಿ ಪರಿವರ್ತಿಸಬಹುದೆಂದೂ, ಯಾವುದೇ ಹೆಸರಿನಿಂದ ಕರೆದರೂ ಎಲ್ಲವೂ ಒಂದೇ ಶಕ್ತಿಯ ಅಂಶಗಳೆಂದೂ ಮನುಷ್ಯನಿಗೆ ಗೊತ್ತಾಗಿದೆ. ಈ ವಿಷಯವು ಸಂಹಿತೆಯಲ್ಲಿಯೂ ಉಕ್ತವಾಗಿದೆ. ಅತಿ ಪುರಾತನವಾದುದಾದರೂ ಸಂಹಿತೆಯಲ್ಲಿ ನಾನು ಮೇಲೆ ಹೇಳಿದ ಶಕ್ತಿಭಾವವು ಉಲ್ಲೇಖಗೊಂಡಿದೆ. ಆಕರ್ಷಣ ಶಕ್ತಿ, ವಿಕರ್ಷಣ ಶಕ್ತಿ, ಉಷ್ಣ, ಅಯಸ್ಕಾಂತ ಶಕ್ತಿ, ವಿದ್ಯುಚ್ಛಕ್ತಿ, ಎಂಬ ಯಾವ ಹೆಸರಿನಿಂದ ಕರೆದರೂ ಎಲ್ಲ ಶಕ್ತಿಗಳೂ ಆ ಒಂದು ಮೂಲ ಶಕ್ತಿಯ ವಿವಿಧರೂಪಗಳು. ಅಂತಃಕರಣ ವೃತ್ತಿಗಳ ರೂಪದಲ್ಲಿ ಅದು ವ್ಯಕ್ತವಾಗಿರಲಿ, ಅಥವಾ ಬಾಹ್ಯಕ್ರಿಯೆಯ ರೂಪದಲ್ಲಿ ವ್ಯಕ್ತವಾಗಿರಲಿ ಅದರ ಮೂಲ ಒಂದೇ. ಆ ಮೂಲ ಶಕ್ತಿಯನ್ನೇ ಪ್ರಾಣ ಎಂದು ಕರೆಯುತ್ತಾರೆ. ಹಾಗಾದರೆ ಪ್ರಾಣ ಎಂದರೇನು? ಪ್ರಾಣ ಎಂದರೆ ಸ್ಪಂದನ. ವಿಶ್ವವೆಲ್ಲವೂ ತನ್ನ ಆದಿಸ್ಥಿತಿಗೆ ಹಿಂದಿರುಗಿದಾಗ ಈ ಅಪಾರ ಶಕ್ತಿಯಾದ ಪ್ರಾಣವು ಏನಾಗುತ್ತದೆ? ಅದು ಸಂಪೂರ್ಣವಾಗಿ ವಿನಾಶವಾಗುತ್ತದೆಯೋ? ಎಂದಿಗೂ ಇಲ್ಲ. ಅದು ನಾಶವಾದರೆ ಮುಂದಣ ಸೃಷ್ಟಿಯ ಗತಿಯೇನು? ನಮ್ಮ ಶಾಸ್ತ್ರಗಳ ಪ್ರಕಾರ ಸೃಷ್ಟಿಯು ಅಲೆಗಳಂತೆ ಚಲಿಸುತ್ತದೆ – ಉಬ್ಬರ ಇಳಿತ, ಉಬ್ಬರ ಇಳಿತ ಇದೇ ಅದರ ಕ್ರಮ. ಸೃಷ್ಟಿ ಎಂಬ ಪದವನ್ನು \enginline{creation} ಎಂದು ಭಾಷಾಂತರಿಸುವುದು ತಪ್ಪು. ಸೃಷ್ಟಿ ಎಂದರೆ ಅನಂತ ಸತ್ತಿನ ಒಂದು ಪ್ರಕಾಶ ಅಥವಾ ವಿಕಾಸವೇ ಹೊರತು ನೂತನ ನಿರ್ಮಾಣವಲ್ಲ. ಪ್ರಳಯಸ್ಥಿತಿಯಲ್ಲಿ ಸೃಷ್ಟಿಯ ಕಾಣದೆ ಹೋದಾಗ ಎಲ್ಲ ವಸ್ತುಗಳು ಸೂಕ್ಷ್ಮಾತಿ ಸೂಕ್ಷ್ಮಗಳಾಗಿ, ತಮ್ಮ ಆದಿಸ್ಥಿತಿಯಲ್ಲಿ ಕೆಲಕಾಲವಿದ್ದು ಹೊರಗೆ ತೋರಲು ಸಿದ್ಧವಾಗಿ ಸಮತೆಯಲ್ಲಿರುತ್ತವೆ. ಆಗ ಈ ಪ್ರಾಣದ ಅಸಂಖ್ಯ ಶಕ್ತಿ ರೂಪಗಳೆಲ್ಲ ಏನಾಗುತ್ತವೆ? ಆದಿಪ್ರಾಣದೊಂದಿಗೆ ಐಕ್ಯವಾಗುತ್ತದೆ. ಆಗ ಪ್ರಾಣವು ನಿಶ್ಚಲವಾಗುತ್ತದೆ. ಸಂಪೂರ್ಣ ನಿಶ್ಚಲವಲ್ಲ! ಸ್ವಲ್ಪ ಹೆಚ್ಚು ಕಡಮೆ ನಿಶ್ಚಲವೆಂದು ಹೇಳಬಹುದು. ಪ್ರಾಣದ ಆ ಸ್ಥಿತಿಯನ್ನೇ ನಾಸದೀಯ ಸೂಕ್ತವು \textbf{“ಆನೀದವಾತಂ”} – ಸ್ಪಂದನಗಳಿಲ್ಲದೆ ಸ್ಪಂದಿಸುತ್ತಿತ್ತು ಎಂದು ವರ್ಣಿಸುತ್ತದೆ. ಉಪನಿಷತ್ತುಗಳಲ್ಲಿ ಬರುವ ಅನೇಕ ಪಾರಿಭಾಷಿಕ ಪದಗಳ ಅರ್ಥವನ್ನು ತಿಳಿಯುವುದು ಕಷ್ಟ. ಉದಾಹರಣೆಗಾಗಿ ಪೂರ್ವೋಕ್ತ ಪಾಠದಲ್ಲಿರುವ “ವಾತಂ” ಎಂಬ ಪದವನ್ನುಕೆಲವು ಸಾರಿ ‘ಗಾಳಿ’ ಎಂಬರ್ಥದಲ್ಲಿಯೂ, ಕೆಲವು ಸಾರಿ ‘ಚಲನೆ’ ಎಂಬರ್ಥದಲ್ಲಿಯೂ ಉಪಯೋಗಿಸಲಾಗಿದೆ. ಅನೇಕರು ಒಂದನ್ನು ಇನ್ನೊಂದರ್ಥದಲ್ಲಿ ಗ್ರಹಿಸಿ ಮೋಸಹೋಗುತ್ತಾರೆ. ಅಂತಹ ಪ್ರಸಂಗದಲ್ಲಿ ನಾವು ಬಹಳ ಎಚ್ಚರಿಕೆಯಿಂದ ಇರಬೇಕಾಗುತ್ತದೆ. ಪ್ರಳಯ ಕಾಲದಲ್ಲಿ ಪ್ರಾಣ ಆ ಸ್ಥಿತಿಯಲ್ಲಿರುತ್ತದೆ. ಆಗ ಆಕಾಶವೆಂದು ಕರೆಯಲ್ಪಡುವ ಜಡಪ್ರಕೃತಿಯು ಏನಾಗುತ್ತದೆ? ಬೇರೆ ಬೇರೆ ಶಕ್ತಿಗಳು ಜಡಪ್ರಕೃತಿಯನ್ನೆಲ್ಲಾ ವ್ಯಾಪಿಸಿರುವುದರಿಂದ ಅದರ ಜೊತೆಯಲ್ಲಿ ಅವೂ ಆಕಾಶದಲ್ಲಿ ಲೀನವಾಗುತ್ತವೆ. ಮತ್ತೊಮ್ಮೆ ಸೃಷ್ಟಿ ತರಂಗವು ಮೇಲೇಳುವಾಗ ಅದರೊಡನೆ ಅವು ಹೊರಗೆ ತೋರುತ್ತದೆ. ಆಕಾಶವನ್ನು \enginline{Ether} ಎಂದು ಕರೆಯಬಹುದು. ಈ ಆಕಾಶವೇ ಜಡ ಪ್ರಕೃತಿಯ ಆದಿರೂಪ. ಸೃಷ್ಟಿಯ ಸಮಯದಲ್ಲಿ ಈ ಆಕಾಶವು ಪ್ರಾಣ ಕ್ರಿಯೆಯಿಂದ ಸ್ಪಂದಿಸುತ್ತದೆ. ಮುಂದಿನ ಸೃಷ್ಟಿ ಸಮಯದಲ್ಲಿ ಸ್ಪಂದನವು ಅತಿಶಯವಾಗಲು ಆಕಾಶವು ಸೂರ್ಯ, ಚಂದ್ರ, ನಕ್ಷತ್ರ ಮತ್ತು ಗ್ರಹವ್ಯೂಹಗಳೆಂದು ಕರೆಯಲ್ಪಡುವ ನಾನಾ ಅಲೆಗಳಾಗಿ ವ್ಯಕ್ತವಾಗುತ್ತದೆ.

\textbf{“ಯದಿದಂ ಕಿಂಚ ಜಗತ್​ ಸರ್ವಂ ಪ್ರಾಣ ಏಜತಿ ನಿಃಸೃತಮ್​”} — ಪ್ರಾಣಸ್ಪಂದನದಿಂದ ಜಗತ್ಸರ್ವವೂ ಸೃಷ್ಟಿಯಾಗುತ್ತದೆ ಎಂಬುದು ಉಪನಿಷತ್ತಿನ ವಾಣಿ. ಮೇಲೆ ಹೇಳಿದ ಮಂತ್ರಭಾಗವನ್ನು ಚೆನ್ನಾಗಿ ಲಕ್ಷಿಸಿ! ‘ಏಜ–ಕಂಪಿಸುವುದು’ ಎಂಬುದರಿಂದ ‘ಏಜತಿ’ ಬಂದಿದೆ. ‘ನಿಃಸೃತಮ್​’ ಎಂದರೆ ‘ಹೊರಗೆ ತೋರಿದುದು’, ‘ಯದಿದಂ ಕಿಂಚ’ – ಈ ಜಗತ್ತಿನಲ್ಲಿ ಏನಿದೆಯೋ ಅದೆಲ್ಲ.

ಇದು ವಿಶ್ವರಚನೆಯ ವಿಚಾರದ ಒಂದು ಭಾಗ ಮಾತ್ರ. ಇದಲ್ಲದೆ ಇನ್ನೂ ಅನೇಕ ವಿವರಣೆಗಳಿವೆ. ಉದಾಹರಣೆಗೆ, ಸೃಷ್ಟಿಯಾಗುವ ಕ್ರಮವೆಂತು? ಮೊದಲು ಆಕಾಶವಿರುತ್ತದೆ. ಅದೇ ಎಲ್ಲದಕ್ಕೂ ಮೂಲ. ಅದು ಸ್ಪಂದಿಸ ತೊಡಗುತ್ತದೆ. ಆಗ ಅದರಿಂದ ವಾಯು ಉತ್ಪತ್ತಿಯಾಗುತ್ತದೆ. ಈ ವಿಷಯಗಳೆಲ್ಲವೂ ವಿವರಿಸಲ್ಪಟ್ಟಿವೆ. ಮೇಲಿನ ವಿಷಯಗಳಲ್ಲಿರುವ ತತ್ತ್ವದ ಸಾರಾಂಶವೇನೆಂದರೆ, ಸೂಕ್ಷ್ಮದಿಂದ ಸ್ಥೂಲವು ಹೊರಹೊಮ್ಮುತ್ತದೆ, ಎಂಬುದು. ಕೊಟ್ಟ ಕೊನೆಗೆ ಅಭಿವ್ಯಕ್ತವಾಗುವುದು ಯಾವುದೆಂದರೆ ಸ್ಥೂಲಭೂತ. ಇದಕ್ಕೆ ಮುಂಚೆ ಅಭಿವ್ಯಕ್ತಗೊಂಡದ್ದೆಂದರೆ ಸೂಕ್ಷ್ಮ ತನ್ಮಾತ್ರಗಳು. ಆದರೂ ವಿಶ್ವರಚನೆಯನ್ನು ಕುರಿತ ಈ ಮೇಲಿನ ಜಿಜ್ಞಾಸೆಯಲ್ಲಿ ದ್ವೈತಭಾವವಿದೆ. ಐಕ್ಯಭಾವವಿನ್ನೂ ಸಂಭವಿಸಿಲ್ಲ. ಆಕಾಶದ, ಎಂದರೆ ಜಡ ಪ್ರಕೃತಿಯ ಐಕ್ಯಭಾವವಿದೆ. ಪ್ರಾಣದ ಎಂದರೆ ಶಕ್ತಿಗಳ ಐಕ್ಯಭಾವವಿದೆ. ಹಾಗೆಯೇ ಮುಂದುವರಿದರೆ ಪ್ರಾಣ–ಆಕಾಶಗಳಲ್ಲಿ ಐಕ್ಯಭಾವವು ಲಭಿಸುವುದೆ? ಇವೆರಡನ್ನೂ ಕರಗಿಸಿ ಒಂದು ಮಾಡಲು ಸಾಧ್ಯವೇ? ಈ ವಿಷಯದಲ್ಲಿ ನಮ್ಮ ಆಧುನಿಕ ವಿಜ್ಞಾನವು ಮೂಕವಾಗಿದೆ. ಇದಕ್ಕೆ ಉತ್ತರವನ್ನೇನೂ ಅದು ಇನ್ನೂ ಕಂಡುಹಿಡಿದಿಲ್ಲ. ಒಂದು ವೇಳೆ ಅದು ಎಂದಾದರೂ ಕಂಡುಹಿಡಿಯುವ ಪಕ್ಷದಲ್ಲಿ (ಉಪನಿಷತ್ತಿನಲ್ಲಿ ಹೇಳಿರುವ ಪ್ರಾಣ ಆಕಾಶಗಳನ್ನು ಅದು ನಿಧಾನವಾಗಿ ಕಂಡುಹಿಡಿಯುತ್ತಿರುವಂತೆ) ಅದು ಉಪನಿಷತ್ತುಗಳ ಸಿದ್ಧಾಂತಕ್ಕೇ ಬರಬೇಕು.

\newpage

ಮುಂದಿನ ಐಕ್ಯತೆ ಸರ್ವವ್ಯಾಪಿಯಾದ ನಿರ್ಗುಣ ಬ್ರಹ್ಮ. ಬ್ರಹ್ಮ ಎಂಬುದು ಪೌರಾಣಿಕ ನಾಮಧೇಯ, ಅದೇ ಚತುರ್ಮುಖ ಬ್ರಹ್ಮ. ಅದನ್ನೇ ಮನಶ್ಶಾಸ್ತ್ರ ದೃಷ್ಟಿಯಿಂದ ‘ಮಹತ್​’ ಎಂದು ಕರೆದಿದ್ದಾರೆ. ‘ಮಹತ್​’ ನಲ್ಲಿ ಪ್ರಾಣ ಆಕಾಶಗಳೆರಡೂ ಲೀನವಾಗುತ್ತವೆ, ಐಕ್ಯವಾಗುತ್ತವೆ, ವ್ಯಕ್ತಿಯ ಮನಸ್ಸು ಮಿದುಳಿನ ಪಂಜರದಲ್ಲಿ ಸೆರೆಸಿಕ್ಕಿರುವ ಮಹತ್ತಿನ ಒಂದು ಅಲ್ಪಭಾಗ ಮಾತ್ರ. ಎಲ್ಲ ವ್ಯಕ್ತಿಗಳಲ್ಲಿಯೂ ಮೈದೋರಿರುವ ಈ ಮನಸ್ಸುಗಳ ಸಮುದಾಯಕ್ಕೆ ಸಮಷ್ಟಿ ಮನಸ್ಸು ಅಥವಾ ವಿಶ್ವಮನಸ್ಸು ಎಂದು ಹೆಸರು. ಇಲ್ಲಿಗೇ ವಿಶ್ಲೇಷಣೆಯು ಮುಗಿಯಲಿಲ್ಲ; ಮುಂದುವರಿಯಿತು. ನಮ್ಮಲ್ಲಿ ಒಬ್ಬೊಬ್ಬರನ್ನೂ ಒಂದೊಂದು ಪಿಂಡಾಂಡವೆಂದು ಕರೆಯಬಹುದು. ಎಲ್ಲರನ್ನು ಒಳಗೊಂಡಿರುವ ಜಗತ್ತನ್ನು ಬ್ರಹ್ಮಾಂಡವೆಂದು ಕರೆಯಬಹುದು. ಆದರೆ ವ್ಯಷ್ಟಿಯಲ್ಲಿ ಏನೇನಿದೆಯೋ ಅದೆಲ್ಲವೂ ಸಮಷ್ಟಿಯಲ್ಲಿಯೂ ಇದೆ, ಎಂದು ನಾವು ನಿಸ್ಸಂದೇಹವಾಗಿ ಊಹಿಸಬಹುದು. ಆದ್ದರಿಂದ ನಮ್ಮ ಮನಸ್ಸಿನ ವಿಶ್ಲೇಷಣೆಯು ನಮಗೆ ಸಾಧ್ಯವಾದರೆ ವಿಶ್ವಮನಸ್ಸಿನಲ್ಲಿಯೂ, ನಮ್ಮ ಮನಸ್ಸಿನಲ್ಲಿ ನಡೆಯುತ್ತಿರುವಂತಹ ಕ್ರಿಯೆಗಳೇ ನಡೆಯುತ್ತಿವೆ ಎಂದು ಊಹಿಸಬಹುದು. ಹಾಗಾದರೆ ಈ ಮನಸ್ಸು ಏನು ಎಂಬುದು ಮುಂದಿನ ಪ್ರಶ್ನೆ. ಪಾಶ್ಚಾತ್ಯ ದೇಶಗಳಲ್ಲಿ ಈಗಿನ ವಿಜ್ಞಾನವು ಮುಂದುವರಿದಂತೆಲ್ಲಾ ಶರೀರಶಾಸ್ತ್ರವು ಪೂರ್ವಕಾಲದ ಧರ್ಮ ದುರ್ಗಗಳನ್ನೆಲ್ಲ ಆಕ್ರಮಿಸಿಬಿಟ್ಟಿದೆ. ಅಲ್ಲಿಯ ಜನರಿಗೆ ನಿಲ್ಲಲು ತಾವಿಲ್ಲದಂತಾಗಿದೆ. ಅವರು ನಿರಾಶರಾಗಿದ್ದಾರೆ. ಏಕೆಂದರೆ ಆಧುನಿಕ ಶರೀರಶಾಸ್ತ್ರವು ಮಿದುಳನ್ನೇ ಮನಸ್ಸೆಂದು ಹೇಳುತ್ತಿದೆ. ಆದರೆ ಭಾರತೀಯರಾದ ನಮಗೆ ಅದೇನೂ ಹೊಸ ವಿಷಯವಲ್ಲ. ಪ್ರತಿಯೊಬ್ಬ ಹಿಂದೂ ಬಾಲಕನಿಗೂ ಗೊತ್ತು. ಮನಸ್ಸು ಜಡ, ಸೂಕ್ಷ್ಮತರವಾದ ಜಡ ಎಂದು. ಶರೀರವು ಸ್ಥೂಲವಾದುದು. ಅದರ ಹಿಂದೆ ಸೂಕ್ಷ್ಮ ಶರೀರವಿದೆ. ಅದೂ ಕೂಡ ಜಡವೇ; ಆದರೆ ಅದು ಆತ್ಮವಲ್ಲ.

ಈ ಪದವನ್ನು ಇಂಗ್ಲೀಷಿಗೆ ನಾನು ಭಾಷಾಂತರಿಸುವುದಿಲ್ಲ; ಏಕೆಂದರೆ ಈ ಭಾವನೆ ಯೂರೋಪಿನಲ್ಲಿಲ್ಲ. ಆದ್ದರಿಂದ ಇದನ್ನು ಭಾಷಾಂತರಿಸುವುದು ಅಸಾಧ್ಯ. ಕೆಲವು ಜರ್ಮನ್​ ತತ್ತ್ವಶಾಸ್ತ್ರಜ್ಞರು ಆತ್ಮನ್​ ಎಂಬ ಪದವನ್ನು \enginline{`self’} ಎಂದು ತರ್ಜುಮೆ ಮಾಡಿದ್ದಾರೆ. ಆದರೆ ಎಲ್ಲರೂ ಅದನ್ನು ಒಪ್ಪಿ ಪ್ರಯೋಗಿಸುವ ತನಕ ಅದನ್ನು ಉಪಯೋಗಿಸುವುದು ಸಾಧ್ಯವಿಲ್ಲ. ಅವರೇನಾದರೂ ಕರೆದುಕೊಳ್ಳಲಿ. ಅದೇ ನಮ್ಮ ಆತ್ಮ, ಎಲ್ಲದರ ಹಿಂದಿರುವ ನಿಜವಾದ ಪುರುಷ. ಈ ಆತ್ಮವೇ ಜಡ ಮನಸ್ಸನ್ನು ಅಥವಾ ನಮ್ಮ ಮನಃಶಾಸ್ತ್ರಗಳಲ್ಲಿ ಕರೆಯುವಂತೆ ಅಂತಃಕರಣವನ್ನು ತನ್ನ ಉಪಕರಣವನ್ನಾಗಿ ಉಪಯೋಗಿಸುತ್ತದೆ. ಮನಸ್ಸು ಎಂಬುದು ಹಲಕೆಲವು ಸೂಕ್ಷ್ಮೇಂದ್ರಿಯಗಳ ಸಹಾಯದಿಂದ ಶರೀರ ಸ್ಥೂಲಾಂಗಗಳನ್ನು ನಡೆಸುತ್ತದೆ. ಈ ಮನಸ್ಸು ಏನು? ನಿಜವಾಗಿ ನೋಡುವುದು ಕಣ್ಣಲ್ಲ, ನಿಜವಾಗಿ ಕೇಳುವುದು ಕಿವಿಯಲ್ಲ; ಆದರೆ ಇವುಗಳ ಹಿಂದೆ ಇಂದ್ರಿಯಗಳಿವೆ; ಈ ಇಂದ್ರಿಯಗಳು ಹಾಳಾದರೆ ಇಂದ್ರನಿಗಿರುವಂತೆ ಸಹಸ್ರಾಕ್ಷಗಳಿದ್ದರೂ ಏನೂ ಕಾಣಿಸುವುದಿಲ್ಲ – ಈ ವಿಚಾರಗಳು ಯೂರೋಪಿನ ಶಾಸ್ತ್ರಜ್ಞರಿಗೆ ಮೊನ್ನೆ ಮೊನ್ನೆ ಗೊತ್ತಾಗಿವೆ. ದೃಷ್ಟಿ ಎಂದರೆ ಬಾಹ್ಯ ದೃಷ್ಟಿಯಲ್ಲ ಎಂಬ ಆಧಾರದಿಂದಲೇ ನಮ್ಮ ತತ್ತ್ವಶಾಸ್ತ್ರಪ್ರಾರಂಭವಾಗುತ್ತದೆ. ನಿಜವಾದ ದೃಷ್ಟಿಯು ಮಿದುಳಿನ ಸೂಕ್ಷ್ಮೇಂದ್ರಿಯಗಳಿಗೆ ಸೇರಿದುದು. ನೀವು ಅವುಗಳನ್ನು ಏನೆಂದು ಬೇಕಾದರೂ ಕರೆಯಿರಿ. ಅಂತೂ ಕಣ್ಣು ಕಿವಿ ಮೂಗುಗಳೇ ಇಂದ್ರಿಯಗಳಲ್ಲ. ಈ ಇಂದ್ರಿಯ ಸಮುದಾಯ, ಮನಸ್ಸು, ಬುದ್ಧಿ, ಚಿತ್ತ, ಅಹಂಕಾರ – ಇವೆಲ್ಲವನ್ನು ಒಟ್ಟಿಗೆ ಮನಸ್ಸು \enginline{(mind)} ಎಂದು ಕರೆಯುತ್ತಾರೆ. ಆಧುನಿಕ ಶರೀರಶಾಸ್ತ್ರಜ್ಞರು ಮಿದುಳೇ ಮನಸ್ಸು ಎಂದು ಹೇಳಿದರೆ ನೀವೇನೂ ಭಯಪಡಬೇಕಾದುದಿಲ್ಲ. ನಮ್ಮ ಶಾಸ್ತ್ರಜ್ಞರಿಗೇ ಅದು ಯಾವಾಗಲೋ ತಿಳಿದಿತ್ತು. ಅದು ನಮ್ಮ ಧರ್ಮದ ಮೂಲತತ್ತ್ವಗಳಲ್ಲಿ ಒಂದು ಎಂದು ಪಾಶ್ಚಾತ್ಯರಿಗೆ ಉತ್ತರ ಹೇಳಿ.

ಅದಂತಿರಲಿ, ನಾವೀಗ ಮನಸ್ಸು, ಬುದ್ಧಿ, ಚಿತ್ತ ಅಹಂಕಾರಗಳೆಂದರೇನು ಎಂಬುದನ್ನು ತಿಳಿಯಬೇಕು. ಮೊದಲು ಚಿತ್ತವನ್ನು ತೆಗೆದುಕೊಳ್ಳೋಣ. ಅದು ಮಹತ್ತಿನ ಒಂದು ಅಂಶ. ವಿವಿಧ ಅವಸ್ಥೆಗಳನ್ನೊಳಗೊಂಡ ಮನಸ್ಸಿಗಿರುವ ಒಂದು ಮೂಲ ನಾಮ. ಇಲ್ಲಿ ಒಂದು ಸರೋವರವಿದೆ ಎಂದು ಭಾವಿಸಿರಿ. ಬೇಸಗೆಯ ಸಂಜೆಯಲ್ಲಿ ಅದು ನಿಷ್ಪಂದವಾಗಿ ನಿಸ್ತರಂಗವಾಗಿ ಶಾಂತವಾಗುತ್ತದೆ. ಅದನ್ನೇ ‘ಚಿತ್ತ’ ಎಂದು ಕರೆಯೋಣ. ಅಂತಹ ಸರಸ್ಸಿಗೆ ಯಾರೋ ಒಂದು ಕಲ್ಲು ಎಸೆದರೆಂದು ಭಾವಿಸಿ. ಏನಾಗುತ್ತದೆ? ಮೊದಲು ಕಲ್ಲು ನೀರನ್ನು ತಳ್ಳುವ ಕ್ರಿಯೆ; ಆಮೇಲೆ ನೀರು ಕಲ್ಲನ್ನು ತಳ್ಳುವ ಪ್ರತಿಕ್ರಿಯೆ. ಅದರಿಂದ ಅಲೆಯುಂಟಾಗುತ್ತದೆ. ಅಂದರೆ ಆ ಪ್ರತಿಕ್ರಿಯೆಯೇ ಅಲೆಯ ರೂಪದಲ್ಲಿ ತೋರುತ್ತದೆ. ಮೊದಲು ನೀರು ಸ್ವಲ್ಪ ಸ್ಪಂದಿಸುತ್ತದೆ. ತತ್​ಕ್ಷಣವೇ ತರಂಗರೂಪದ ಪ್ರತಿಕ್ರಿಯೆಯನ್ನು ಕಳುಹಿಸುತ್ತದೆ. ಚಿತ್ತವು ಸರೋವರಕ್ಕೆ ಸಮಾನ; ಹೊರಗಿನ ವಸ್ತುಗಳು ಎಸೆದ ಕಲ್ಲಿಗೆ ಸಮಾನ. ಇಂದ್ರಿಯಗಳ ಮೂಲಕ ಚಿತ್ತಕ್ಕೆ ವಿಷಯ ಸನ್ನಿಹಿತವಾಗುತ್ತದೆ. ಇಂದ್ರಿಯಗಳು ವಿಷಯಮುದ್ರೆಗಳನ್ನು ಒಳಗೆ ಕಳುಹಿಸಿದಾಗ ಚಿತ್ತದಲ್ಲಿ ತರಂಗಗಳೇಳುತ್ತವೆ. ಈ ತರಂಗಿತ ಚಿತ್ತವೇ ಮನಸ್ಸಾಗುತ್ತದೆ. ಅಂದರೆ ಚಲನೆಯೇ ಮನಸ್ಸಿನ ಮುಖ್ಯ ಲಕ್ಷಣ. ತರುವಾಯ ಪ್ರತಿಕ್ರಿಯೆ ಉಂಟಾಗುತ್ತದೆ. ಚಿತ್ತದ ಪ್ರತಿಕ್ರಿಯಾವಸ್ಥೆಯೇ ಬುದ್ಧಿ, ಅಥವಾ ಚಿತ್ತದ ನಿಶ್ಚಯಾತ್ಮಕ ಶಕ್ತಿ. ಈ ಬುದ್ಧಿಯೊಡನೆ ‘ಅಹಂ’ ‘ನಾನು’ ಎಂಬ ಮತ್ತೊಂದು ಅವಸ್ಥೆಯು ಚಿತ್ತದಲ್ಲಿ ಉತ್ಪನ್ನವಾಗುತ್ತದೆ. ಕೂಡಲೇ ಬಾಹ್ಯವಸ್ತುವಿನ ಅರಿವುಂಟಾಗುತ್ತದೆ. ನನ್ನ ಕೈಮೇಲೆ ಸೊಳ್ಳೆಯೊಂದು ಕುಳಿತುಕೊಳ್ಳುತ್ತದೆ ಎಂದು ತಿಳಿದುಕೊಳ್ಳಿ. ಆ ಅನುಭವ ಚಿತ್ತಕ್ಕೆ ಹೋಗುತ್ತದೆ. ಚಿತ್ತವು ಸ್ಪಂದಿಸುತ್ತದೆ. ಅದೇ ಮನಸ್ಸು. ಕೂಡಲೇ ಪ್ರತಿಕ್ರಿಯೆಯಾಗುತ್ತದೆ; ಒಡನೆಯೇ ನನ್ನ ಮೈಮೇಲೊಂದು ಸೊಳ್ಳೆ ಕುಳಿತಿದೆಯೆಂದೂ ಅದನ್ನು ನಾನು ಓಡಿಸಬೇಕೆಂದೂ ಭಾಸವಾಗುತ್ತದೆ. ಮೇಲಿನ ಉಪಮೆಯಲ್ಲಿರುವ ಸರೋವರಕ್ಕೂ ಚಿತ್ತಕ್ಕೂ ಒಂದು ವ್ಯತ್ಯಾಸವಿದೆ. ಸರೋವರಕ್ಕೆ ಕಲ್ಲು ಹೊರಗಿನಿಂದ ಬಂದು ಬೀಳುತ್ತದೆ. ಚಿತ್ತಕ್ಕೆ ಹೊರಗಿನಿಂದಲೂ, ಒಳಗಿನಿಂದಲೂ ಬೀಳುತ್ತದೆ. ಈ ಎಲ್ಲದರ ಸಮುದಾಯವೇ ಅಂತಃಕರಣ ಎನ್ನಿಸಿಕೊಳ್ಳುತ್ತದೆ.

ಇದರ ಜೊತೆಗೆ ನೀವು ಇನ್ನೊಂದು ವಿಚಾರವನ್ನು ತಿಳಿದುಕೊಂಡರೆ ನಿಮಗೆ ಅದ್ವೈತ ದರ್ಶನವು ಹೆಚ್ಚು ಸುಲಭವಾಗಿ ತಿಳಿಯುತ್ತದೆ. ಅದೇನೆಂದರೆ: ನೀವೆಲ್ಲ ಮುತ್ತುಗಳನ್ನು ನೋಡಿದ್ದೀರಿ. ಅವುಗಳಾಗುವ ರೀತಿಯನ್ನೂ ತಿಳಿದಿರುತ್ತೀರಿ. ಧೂಳಿನ ಕಣವು ಶುಕ್ತಿಕೆ ಎಂಬ ಸಮುದ್ರ ಪ್ರಾಣಿಯ ಚಿಪ್ಪಿನೊಳಗೆ ಪ್ರವೇಶಿಸುತ್ತದೆ. ಆ ಪ್ರಾಣಿಗೆ ತುರಿಕೆಯಾಗಲು ಅದು ತನ್ನ ದೇಹದಿಂದ ಒಂದು ತೆರನಾದ ದ್ರವವನ್ನು ಸ್ರವಿಸಿ, ಅದರಿಂದ ಧೂಳಿನ ಕಣವನ್ನು ಆವರಿಸುತ್ತದೆ. ಹೀಗೆ ಶುಕ್ತಿಕೆಯ ದ್ರವದಿಂದ ಸುತ್ತಲ್ಪಟ್ಟ ಧೂಳಿಯ ಕಣವೇ ಕಾಲಕ್ರಮೇಣ ಮುತ್ತಿನರೂಪವನ್ನು ತಾಳುತ್ತದೆ. ಹಾಗೆಯೇ ಈ ಮಹಾವಿಶ್ವವೆಲ್ಲವೂ ನಮ್ಮಿಂದ ತಯಾರಾಗುತ್ತಿರುವ ಒಂದು ಮುತ್ತು. ಬಾಹ್ಯ ಪ್ರಪಂಚದಿಂದ ನಮಗೆ ಲಭಿಸುವುದೆಂದರೆ ಒಂದು ಆಘಾತ ಮಾತ್ರ. ಆ ಆಘಾತದ ಅರಿವು ಕೂಡ ನಮ್ಮ ಪ್ರತಿಕ್ರಿಯೆಯಿಂದಲೇ ಲಭಿಸುತ್ತದೆ. ಪ್ರತಿಕ್ರಿಯೆಯಿಂದ ನಮ್ಮ ಮನಸ್ಸಿನ ಸ್ವಲ್ಪ ಭಾಗವನ್ನೇ ಆಘಾತದ ಮೇಲೆ ಅಧ್ಯಾರೋಪ ಮಾಡುತ್ತೇವೆ. ಆ ಜ್ಞಾನವುಂಟಾದಾಗ ಅದು ಆಘಾತವನ್ನು ನಿಮಿತ್ತ ಮಾತ್ರವನ್ನಾಗಿ ಇಟ್ಟುಕೊಂಡ ನಮ್ಮ ಚಿತ್ತದ ರೂಪವೇ ಎಂದು ಗೊತ್ತಾಗುತ್ತದೆ. ಹೊರಗಿನ ವಾಸ್ತವ ಜಗತ್ತಿನಲ್ಲಿ ನಂಬಿಕೆ ಇರುವವರಿಗೂ ಕೂಡ ಈ ವಿಷಯವು ಸ್ಪಷ್ಟವಾಗುತ್ತದೆ. ಶರೀರಶಾಸ್ತ್ರದ ಪ್ರಭಾವವು ಹೆಚ್ಚಾಗಿರುವ ಈ ಕಾಲದಲ್ಲಿ ಅವರು ಅದನ್ನು ಒಪ್ಪಲೇಬೇಕಾಗುತ್ತದೆ. ಹೊರ ಜಗತ್ತನ್ನು ‘ಅ’ ಎಂದು ಇಟ್ಟುಕೊಳ್ಳೋಣ. ನಮಗೆ ತಿಳಿದಿರುವ ಹೊರ ಜಗತ್ತಿನ ಜ್ಞಾನವು ‘ಅ’ ಮತ್ತು ಮನಸ್ಸಿನ ಮಿಶ್ರಣ. ಮನಸ್ಸಿನ ಅಂಶವು ಎಷ್ಟರಮಟ್ಟಿಗೆ ಇದೆಯೆಂದರೆ ‘ಅ’ ಅದರಲ್ಲಿ ಸಂಪೂರ್ಣ ಮುಳುಗಿಹೋಗಿ ಅದು ಅಜ್ಞಾತವಾಗಿಯೂ ಅಜ್ಞೇಯವಾಗಿಯೂ ಉಳಿದುಬಿಟ್ಟಿದೆ. ಆದ್ದರಿಂದ ಒಂದು ವೇಳೆ ಹೊರಗಿನ ಜಗತ್ತೊಂದು ಇರುವ ಪಕ್ಷದಲ್ಲಿ ಅದು ಎಂದೆಂದಿಗೂ ಆಜ್ಞಾತ ಮತ್ತು ಅಜ್ಞೇಯವಾದುದು. ನಮಗೆ ಗೊತ್ತಿರುವುದೇನೆಂದರೆ, ನಮ್ಮ ಮನಸ್ಸು ಎರಕಹೊಯ್ದಂತೆ, ರೂಪಕೊಟ್ಟಂತೆ, ಅದು ನಮಗೆ ಗೋಚರವಾಗುತ್ತಿದೆ ಎಂಬುದು. ಹೀಗೆಯೇ ಒಳಗಿನ ಪ್ರಪಂಚಕ್ಕೂ ಇದೇ ವಾದವು ಅನ್ವಯಿಸುತ್ತದೆ. ನಮ್ಮ ಆತ್ಮಕ್ಕೂ ಇದು ಅನ್ವಯವಾಗುತ್ತದೆ. ಆತ್ಮನನ್ನೂ ಕೂಡ ಮನಸ್ಸಿನ ಮುಖಾಂತರ ತಿಳಿಯಬೇಕು. ಆದ್ದರಿಂದ ನಮಗೆ ಗೊತ್ತಾಗುವುದು ಮನಸ್ಸು ಸಮೇತವಾದ ಆತ್ಮವೇ ಹೊರತು, ಶುದ್ಧ ಆತ್ಮವಲ್ಲ. ಎಂದರೆ ನಮಗೆ ತಿಳಿಯುವುದು ಮನಸ್ಸಿನ ಉಪಾಧಿಯಿಂದ ಬದ್ಧವಾಗಿರುವ, ಮನಸ್ಸಿನ ಬಣ್ಣಗಳನ್ನು ತಾಳಿರುವ, ಮನೋ ಮುದ್ರಿತವಾಗಿರುವ ಆತ್ಮವೇ ಹೊರತು ಬೇರೆಯಲ್ಲ. ಈ ವಿಷಯದ ಪ್ರಸ್ತಾಪವು ಮುಂದೆ ಬರುವುದರಿಂದ ಇದನ್ನು ಮರೆಯಬೇಡಿ.

ಇದಾದ ತರುವಾಯ ನಾವು ತಿಳಿಯಬೇಕಾದ ವಿಷಯವೊಂದಿದೆ. ಜಡ ವಸ್ತುವಿನ ಒಂದು ನಿರಂತರ ಪ್ರವಾಹಕ್ಕೆ ನಾವು ಕೊಟ್ಟಿರುವ ಹೆಸರು ‘ಶರೀರ’ ಎಂದು. ಪ್ರತಿಯೊಂದು ಕ್ಷಣದಲ್ಲಿಯೂ ಅದಕ್ಕೆ ದ್ರವ್ಯಕಣಗಳು ಬಂದು ಸೇರುತ್ತವೆ; ಮತ್ತು ಅದರಿಂದ ಹೊರಟುಹೋಗುತ್ತವೆ. ಅದೊಂದು ಹರಿಯುವ ನದಿಯಂತೆ; ಹೊಸ ನೀರು ಬರುತ್ತದೆ; ಹಳೆಯ ನೀರು ಹೊರಟು ಹೋಗುತ್ತಿರುತ್ತದೆ. ನಾವು ಮಾತ್ರ ನಮ್ಮ ಕಲ್ಪನೆಯ ಸಹಾಯದಿಂದ ಅದನ್ನು ನಿರಂತರವೆಂದೂ ಯಾವಾಗಲೂ ಒಂದೇ ನದಿಯೆಂದೂ ಭಾವಿಸುತ್ತೇವೆ. ನಿಜವಾಗಿ ನೋಡಿದರೆ ‘ನದಿ’ ಎಂಬುದು ಯಾವುದು? ನದಿಯ ನೀರು ಬದಲಾಗುತ್ತದೆ. ಕ್ಷಣಕ್ಷಣವೂ ನದಿಯ ದಡವು ಬದಲಾಗುತ್ತದೆ. ಹಾಗಾದರೆ ನಿಜವಾದ ನದಿ ಯಾವುದು? ನಿಜವಾದ ನದಿ ಎಂದರೆ ಈ ಬದಲಾವಣೆಗಳ ಶ್ರೇಣಿ. ಮನಸ್ಸಿನ ವಿಷಯಕ್ಕೂ ಇದೇ ಮಾತು ಅನ್ವಯಿಸುತ್ತದೆ. ಬೌದ್ಧರ ಸಿದ್ಧಾಂತವಾಗಿರುವ ‘ಕ್ಷಣಿಕ ವಿಜ್ಞಾನ’ ವಾದವು ಇದನ್ನೇ ಸಮರ್ಥಿಸುತ್ತದೆ. ಅರಿಯಲು ಅತ್ಯಂತ ಕಷ್ಟವಾದರೂ, ಆ ವಾದವು ಸಂಪೂರ್ಣವಾಗಿ ತರ್ಕಬದ್ಧ ವಾದುದಾಗಿದೆ. ವೇದಾಂತದ ಕೆಲವು ಅಂಶಗಳಿಗೆ ಅದು ವಿರೋಧವಾಗಿರುವುದರಿಂದ ಆ ವಾದವನ್ನು ಖಂಡಿಸಬೇಕಾಯಿತು. ಅದು ಅದ್ವೈತ ವೇದಾಂತದಿಂದ ಮಾತ್ರ ಹೇಗೆ ಸಾಧ್ಯ ಎಂಬುದನ್ನು ಮುಂದೆ ನಾವು ನೋಡುವೆವು. ಆದ್ವೈತದ ಬಗ್ಗೆ ವಿಲಕ್ಷಣ ಭಾವನೆಗಳಿವೆ. ಕೆಲವರಿಗಂತೂ ಆದ್ವೈತವೆಂದರೆ ಭೀತಿ. ಆದರೂ ಜಗತ್​ ಕಲ್ಯಾಣಕ್ಕೆ ಅದ್ವೈತವೇ ಕಾರಣವಾಗುತ್ತದೆ. ಏಕೆಂದರೆ ವಸ್ತುಸ್ಥಿತಿಯನ್ನು ಅದು ಮಾತ್ರ ಸಮರ್ಪಕವಾಗಿ ವಿವರಿಸುತ್ತದೆ. ದ್ವೈತ ಮತ್ತು ಇತರ ಸಿದ್ಧಾಂತಗಳು ಭಗವದುಪಾಸನೆಗೆ ಬಹಳವಾಗಿ ಸಹಾಯ ಮಾಡುತ್ತವೆ; ಮನಸ್ಸಿಗೆ ತೃಪ್ತಿ ಕೊಡುತ್ತವೆ. ಮನಸ್ಸನ್ನು ಮೇಲೆ ಮೇಲೆ ಕೊಂಡೊಯ್ಯಲೂಬಹುದು. ಆದರೆ ಒಬ್ಬನು ವಿಚಾರಪರನೂ ಧಾರ್ಮಿಕನೂ ಆಗಿರಬೇಕೆಂದು ಬಯಸಿದರೆ ಅಂತಹವನಿಗೆ ಜಗತ್ತಿನಲ್ಲಿ ಅದ್ವೈತವನ್ನು ಬಿಟ್ಟರೆ ಬೇರೆ ಯಾವುದೂ ಇಲ್ಲ. ಸದ್ಯಕ್ಕೆ ಈ ವಿಷಯ ಹಾಗಿರಲಿ. ನಾವು ಮನಸ್ಸನ್ನು ನಿರಂತರವಾಗಿ ನೀರು ಬಂದು ಹೋಗುತ್ತಿರುವ ನದಿಗೆ ಹೋಲಿಸಿದೆವು. ಹಾಗಾದರೆ ಆತ್ಮವೆಂದು ಕರೆಯಲ್ಪಡುವ ಮೂಲಭೂತವಾದ ಐಕ್ಯತತ್ತ್ವವೆಲ್ಲಿದೆ? ವಿಚಾರವೇನೆಂದರೆ, ನಮ್ಮ ದೇಹವು ನಿರಂತರವಾಗಿ ವ್ಯತ್ಯಸ್ತವಾದರೂ, ಮನಸ್ಸು ಹಾಗೆಯೇ ಪ್ರತಿಕ್ಷಣವೂ ಬದಲಾಯಿಸಿದರೂ, ನಮ್ಮಲ್ಲಿ ಬದಲಾಯಿಸದೇ ಇರುವ ಯಾವುದೋ ಒಂದು ತತ್ತ್ವವಿದೆ. ಆ ತತ್ತ್ವವು, ಜಗತ್ತಿನ ಉಳಿದೆಲ್ಲವೂ ಬದಲಾಯಿಸುವುದಿಲ್ಲ ಎಂದು ತೋರುವಂತೆ ಮಾಡುತ್ತದೆ. ಬೇರೆ ಬೇರೆ ದಿಕ್ಕುಗಳಿಂದ ಬರುವ ಕಿರಣಗಳು ಪರದೆ, ಗೋಡೆ ಇತ್ಯಾದಿ ಯಾವುದಾದರೂ ಒಂದು ಸ್ಥಿರವಾದ ಆಧಾರದ ಮೇಲೆ ಬಿದ್ದರೆ ತಾನೇ ಬೇಕೆಂಬ ಐಕ್ಯಭಾವವು ತೋರುವುದು? ಆಗ ಮಾತ್ರವೇ ಆ ಕಿರಣಗಳು ಒಂದು ಪೂರ್ಣರೂಪವನ್ನು ಪಡೆಯುವುದು ಸಾಧ್ಯ. ಹಾಗೆಯೇ, ವಿವಿಧ ಭಾವನೆಗಳು ಬಂದು, ಐಕ್ಯತೆಯನ್ನು ಪಡೆದು, ಒಂದು ಪೂರ್ಣರೂಪವನ್ನು ತಾಳುವಂತೆ ಮಾಡುವ ಒಂದು ಸ್ಥಿರವಾದ ನೆಲೆ ನಮ್ಮಲ್ಲಿ ಯಾವುದಿದೆ? ಮನಸ್ಸು ಆ ನೆಲೆಯಾಗಿರುವುದು ಸಾಧ್ಯವಿಲ್ಲ. ಏಕೆಂದರೆ ಮನಸ್ಸೇ ಚಂಚಲವಾದುದು. ಆದ್ದರಿಂದ ದೇಹ ಮನಸ್ಸುಗಳಲ್ಲದ, ಚಂಚಲವಲ್ಲದ, ಅಚಂಚಲವಾದ, ಶಾಶ್ವತವಾದ, ನಮ್ಮ ಭಾವ ಆಲೋಚನೆ ಅನುಭವಗಳಿಗೆ ಮೂಲ ಪೀಠ ಸ್ವರೂಪವಾಗಿ ಐಕ್ಯಭಾವವನ್ನು ಕೊಡುವ, ಒಂದು ತತ್ತ್ವವಿರಬೇಕು. ಅದನ್ನೇ ಆತ್ಮ ಎಂದು ಕರೆಯುತ್ತಾರೆ. ಜಡ ಪ್ರಕೃತಿಯು ಅದನ್ನು ನೀವು ಸೂಕ್ಷ್ಮವಸ್ತು ಎಂದಾಗಲಿ, ಮನಸ್ಸು ಎಂದಾಗಲಿ ಕರೆಯಿರಿ ಪರಿವರ್ತನಶೀಲವಾದದ್ದು. ಸೂಕ್ಷ್ಮವಸ್ತುವಿಗೆ ಹೋಲಿಸಿದರೆ ಸ್ಥೂಲವಸ್ತು ಅಥವಾ ಬಾಹ್ಯ ಜಗತ್ತು ಕೂಡ ಪರಿವರ್ತನಶೀಲವಾದದ್ದು. ಆದುದರಿಂದ ಬದಲಾವಣೆ ಹೊಂದದ ಅಥವಾ ಅಪರಿವರ್ತನ ಶೀಲವಾದ ಆ ವಸ್ತುವು ಭೌತವಸ್ತುವಿನಿಂದ ಆದದ್ದಾಗಿರಲಾರದು. ಆದ್ದರಿಂದ ಅದು ಚೇತನಾತ್ಮಕವಾದದ್ದಾಗಿರಬೇಕು. ಅದು ಅವಿಕಾರಿಯೂ ಅವಿನಾಶಿಯೂ ಆದದ್ದು.

ಈ ಬಾಹ್ಯ ಜಗತ್ತನ್ನು ಸೃಷ್ಟಿಸಿದವರು ಯಾರು, ಜಡವಸ್ತುವನ್ನು ಸೃಷ್ಟಿಸಿದವರು ಯಾರು, ಮುಂತಾದ ಈ ಸೃಷ್ಟಿಯ ರಚನೆಗೆ ಸಂಬಂಧಿಸಿದ, ಹೊರ ಜಗತ್ತಿನ ವಿಷಯದಲ್ಲಿ\break ಮಾತ್ರ ಉದ್ಭವವಾಗುವ ಹಳೆಯ ವಾದಗಳನ್ನು ಹೊರತು ಪಡಿಸಿದರೆ, ಇನ್ನೊಂದು ಪ್ರಶ್ನೆಯು ಉದ್ಭವವಾಗುತ್ತದೆ. ಸತ್ಯವನ್ನು ಮಾನವನ ಅಂತಃ ಸ್ವಭಾವದಿಂದ ಮಾತ್ರ ತಿಳಿಯುವ ಪ್ರಯತ್ನ ಅದು. ಆತ್ಮನ ವಿಷಯವಾಗಿ ಪ್ರಶ್ನೆಯು ಹೇಗೆ ಉದ್ಭವಿಸಿತೋ ಹಾಗೆಯೇ ಈ ವಿಷಯದಲ್ಲೂ ಅದು ಉದ್ಭವಿಸಿತು. ಪ್ರತಿಯೊಬ್ಬನಲ್ಲಿಯೂ ದೇಹ ಮನಸ್ಸುಗಳಿಂದ ಬೇರೆಯಾದ, ಅಚಂಚಲವಾಗಿರುವ ಒಂದೊಂದು ಆತ್ಮವಿದೆ ಎಂದು ಭಾವಿಸಿದರೂ, ಆ ಅನೇಕ ಆತ್ಮಗಳಲ್ಲಿ ಪರಸ್ಪರವಾಗಿರುವ ಏಕದೃಷ್ಟಿ, ಏಕಭಾವ, ಸಹಾನುಭೂತಿಗಳಿಗೆ ಕಾರಣವಾದ ಒಂದು ಏಕತೆಯು ಬೇಕಷ್ಟೆ? ನನ್ನ ಆತ್ಮವು ನಿಮ್ಮ ಆತ್ಮಗಳ ಮೇಲೆ ಪ್ರಭಾವ ಬೀರುವುದಾದರೂ ಹೇಗೆ? ನನ್ನ ಮತ್ತು ನಿಮ್ಮ ಆತ್ಮಗಳು ಪರಸ್ಪರ ಕ್ರಿಯೆ ಪ್ರತಿಕ್ರಿಯೆಯನ್ನು ತೋರಿಸುವಂತೆ ಮಾಡಬಹುದಾದ ಮಾಧ್ಯಮ ಯಾವುದು? ನಿಮ್ಮ ಆತ್ಮದ ವಿಚಾರವಾಗಿ ನಾನು ಏನನ್ನೇ ಆಗಲಿ ಹೇಗೆ ತಿಳಿಯಬಲ್ಲೆ? ನಿಮ್ಮ ನನ್ನ ಆತ್ಮಗಳಿಗೆ ಮಧ್ಯಸ್ಥವಾಗಿ ಸಂಬಂಧ ಕಲಿಸಿರುವ ವಸ್ತುವಾವುದು? ಈ ಪ್ರಶ್ನೆಗಳಿಗೆ ಉತ್ತರವಾಗಿ ಮತ್ತೊಂದು ಆತ್ಮವನ್ನು ತಾತ್ವಿಕವಾಗಿ ನಾವು ಒಪ್ಪಲೇಬೇಕು. ಆ ಆತ್ಮವು ಬೇರೆ ಎಲ್ಲ ಆತ್ಮಗಳನ್ನೂ ಸಂಬಂಧಿಸುವ ಆತ್ಮವಾಗಿರಬೇಕು, ಇತರ ಎಲ್ಲ ಆತ್ಮಗಳಲ್ಲಿಯೂ ಅಂತರ್ಯಾಮಿಯಾಗಿರಬೇಕು. ಅದು ಎಲ್ಲ ಆತ್ಮಗಳ ಜೀವನ ಸ್ವರೂಪವಾಗಿರಬೇಕು. ಅದರ ಸಹಾಯದಿಂದಲೇ ಇತರ ಎಲ್ಲ ಆತ್ಮಗಳೂ ಪ್ರೇಮ ಸಹಾನುಭೂತಿಗಳಿಂದ ಬದುಕಿ ಬಾಳಿ ಕೆಲಸ ಮಾಡಬೇಕು. ಈ ಆತ್ಮಗಳ ಆತ್ಮವೇ, ಈ ವಿಶ್ವಾತ್ಮವೇ ಜಗದೀಶ್ವರನಾದ ಪರಮಾತ್ಮ. ಇತರ ಆತ್ಮಗಳಂತೆ ಇದೂ ಕೂಡ ಜಡವಾಗಿರದೆ ಚಿನ್ಮಯವಾಗಿರಬೇಕು. ಅದು ಜಡಪ್ರಕೃತಿಯ ನಿಯಮಗಳಿಗೆ ವಿಧೇಯವಾಗಿರಲಾರದು. ಪ್ರಕೃತಿ ನಿಯಮಗಳ ತೀರ್ಪಿಗೆ ಅದು ಬದ್ಧವಾಗಿರಲಾರದು. ಆದುದರಿಂದ ಅದು ಅಮೃತವೂ ಅಚಲವೂ ಅಜೇಯವೂ ಆಗಿರುತ್ತದೆ. \textbf{“ನೈನಂ ಛಿಂದಂತಿ ಶಸ್ತ್ರಾಣಿ ನೈನಂ ದಹತಿ ಪಾವಕಃ~। ನ ಚೈನಂ ಕ್ಲೇದಯಂತ್ಯಾಪೋ ನ ಶೋಷಯತಿ ಮಾರುತಃ~॥ ನಿತ್ಯಃ ಸರ್ವಗತಃ ಸ್ಥಾಣುರಚಲೋಽಯಂ ಸನಾತನಃ~॥”} (ಅದನ್ನು ಬೆಂಕಿ ದಹಿಸದು, ಕತ್ತಿ ಕತ್ತರಿಸದು, ಗಾಳಿ ಒಣಗಿಸದು, ನೀರು ತೋಯಿಸದು, ನಿತ್ಯವು ಸ್ಥಾಣುವೂ ಅಮರವೂ ಸನಾತನವೂ ಆಗಿದೆ). ಗೀತೆಯೂ ವೇದಾಂತವೂ ಈ ಜೀವಾತ್ಮನು ಕೂಡ ವಿಭು ಅಥವಾ ಸರ್ವವ್ಯಾಪಿ ಎಂದು ಹೇಳುತ್ತವೆ. ಭರತಖಂಡದಲ್ಲಿ ಆತ್ಮವನ್ನು ಅಣು ಎಂದು ಹೇಳುವ ಪಂಥಗಳೂ ಇವೆ. ಆದರೆ ಅವುಗಳ ಹೇಳಿಕೆಯ ನಿಜವಾದ ಅರ್ಥ, ಆತ್ಮವೂ ತೋರಿಕೆಗೆ ಅಣು, ತತ್ತ್ವತಃ ವಿಭು ಎಂದು.

ಮತ್ತೊಂದು ವಿಷಯ, ಅದು ದಿಗ್ಭ್ರಮೆ ಹುಟ್ಟಿಸುವಂಥದು, ವಿಶೇಷವಾಗಿ ಭಾರತೀಯವಾದ ಲಕ್ಷಣವುಳ್ಳದ್ದು. ಅದು ಎಲ್ಲ ಪಂಗಡಗಳಿಗೂ ಸಮಾನವಾದ ಭಾವ. ಆದ್ದರಿಂದ ದಯವಿಟ್ಟು ನೀವೆಲ್ಲರೂ ಈ ವಿಷಯವನ್ನು ಗಮನವಿಟ್ಟು ಕೇಳಿ ಮನಸ್ಸಿನಲ್ಲಿಡಬೇಕೆಂದು ಪ್ರಾರ್ಥಿಸುತ್ತೇನೆ. ಏಕೆಂದರೆ ಭಾರತದಲ್ಲಿರುವ ಪ್ರತಿಯೊಂದಕ್ಕೂ ಅದೇ ಅಸ್ತಿರಭಾರವಾಗಿದೆ. ಅದೇನೆಂದರೆ – ಜರ್ಮನ್​ ಮತ್ತು ಇಂಗ್ಲೀಷ್​ ಮೇಧಾವಿಗಳು ಪ್ರತಿಪಾದಿಸಿರುವ ‘ವಿಕಾಸವಾದ’ ದ ವಿಚಾರವನ್ನು ನೀವೆಲ್ಲ ಕೇಳಿದ್ದೀರಿ. ಅದರ ಪ್ರಕಾರ ಬೇರೆ ಬೇರೆ ಪ್ರಾಣಿಗಳ ದೇಹಗಳೆಲ್ಲವೂ ಮೂಲತಃ ಒಂದೇ. ನಮಗೆ ಕಂಡು ಬರುವ ವ್ಯತ್ಯಾಸಗಳು ಒಂದೇ ಶ್ರೇಣಿಯ ವಿವಿಧ ಅಭಿವ್ಯಕ್ತಿಗಳು. ಅತಿ ಕ್ಷುದ್ರವಾದ ಕ್ರಿಮಿಯಿಂದ ಪ್ರಾರಂಭವಾಗಿ ಮಹಾ ಮಹಿಮನಾದ ಮಹರ್ಷಿಯವರೆವಿಗೂ, ಒಂದೇ ದೇಹವು ಒಂದು ಸ್ಥಿತಿಯಿಂದ ಮತ್ತೊಂದು ಸ್ಥಿತಿಗೇರಿ, ಕೀಳ್ಮೆಯಿಂದ ಮೇಲ್ಮೆಗೇರಿ, ಕಟ್ಟಕಡೆಗೆ ಅಂತ್ಯದ ಪರಿಪೂರ್ಣತೆಗೆ ಹೋಗುತ್ತದೆ. ನಮ್ಮ ತತ್ತ್ವಶಾಸ್ತ್ರಗಳಲ್ಲಿಯೂ ಅದೇ ಭಾವನೆ ಇದೆ. ಪತಂಜಲಿ ಯೋಗಿಗಳು ಹೀಗೆ ಹೇಳುತ್ತಾರೆ: \textbf{“ಜಾತ್ಯಂತರ ಪರಿಣಾಮಃ ಪ್ರಕೃತ್ಯಾಪೂರಾತ್​–”} ಒಂದು ಜಾತಿಯೆ ಮತ್ತೊಂದು ಜಾತಿಯಾಗುತ್ತದೆ. ಪರಿಣಾಮ ಎಂದರೆ ಒಂದು ವಸ್ತುವು ಮತ್ತೊಂದು ವಸ್ತುವಾಗುವುದು. ಆದರೆ ಅದು ಆಗುವುದು ಹೇಗೆ? ಈ ವಿಷಯದಲ್ಲಿ ನಮಗೂ ಪಾಶ್ಚಾತ್ಯರಿಗೂ ಭಿನ್ನಾಭಿಪ್ರಾಯವಿರುತ್ತದೆ. ‘ಪ್ರಕೃತ್ಯಾಪೂರಾತ್​’ ಪ್ರಕೃತಿಯ ಆಪೂರಣದಿಂದ ಆಗುತ್ತದೆ ಎಂಬುದು ಪತಂಜಲಿಯ ಅಭಿಪ್ರಾಯ. ಹಾಗೆಂದರೆ, ಪ್ರಕೃತಿಯ ‘ಒಳತುಂಬುವಿಕೆ’ ಯಿಂದ ಎಂದು ಅರ್ಥವಾಗುತ್ತದೆ. ಪಾಶ್ಚಾತ್ಯ ವಿದ್ವಾಂಸರ ಪ್ರಕಾರ ಒಂದು ರೂಪ ಮತ್ತೊಂದಾಗುವುದಕ್ಕೆ ಸ್ಫರ್ಧೆ ಹಾಗೂ ನೈಸರ್ಗಿಕ ಮತ್ತು ಲೈಂಗಿಕ ಆಯ್ಕೆ ಇತ್ಯಾದಿಗಳು \enginline{(natural and sexual selection)} ಕಾರಣವಾಗುತ್ತವೆ. ಆದರೆ “ಒಂದು ರೂಪವು ಮತ್ತೊಂದು ರೂಪವನ್ನು ಧಾರಣೆ ಮಾಡುವುದಕ್ಕೆ ಪ್ರಕೃತಿಯ ಆಪೂರಣವೇ ಕಾರಣವಾಗುತ್ತದೆ” ಎಂಬ ಸಿದ್ಧಾಂತವೂ ಇನ್ನೂ ಹೆಚ್ಚು ಯುಕ್ತಿ ಪೂರ್ವಕವಾಗಿದೆ. ‘ಪ್ರಕೃತಿಯ ಆಪೂರಣ’ ಎಂದರೆ ಅರ್ಥವೇನು? ‘ಅಮೀಬ’ ವೆ \enginline{(amoeba)} ಕ್ರಮೇಣ ಉತ್ತಮ ಸ್ಥಿತಿಗೆ ಏರಿ, ಕಡೆಗೆ ಬುದ್ಧನ ಸ್ಥಿತಿಗೂ ಬರುತ್ತದೆ ಎಂಬುದನ್ನು ನಾವು ಒಪ್ಪುತ್ತೇವೆ. ಆದರೆ ಒಂದು ಯಂತ್ರದಿಂದ ನಮಗೆ ಎಷ್ಟು ಕೆಲಸಬೇಕೊ ಅಷ್ಟನ್ನು ಮಾಡಿಸಿಕೊಳ್ಳುವುದಕ್ಕೆ ಅಗತ್ಯವಾದ ಶಕ್ತಿಯನ್ನು ಇನ್ನೊಂದು ರೂಪದಿಂದ ಆ ಯಂತ್ರಕ್ಕೆ ನಾವು ಕೊಡಲೇಬೇಕು, ಎಂಬುದನ್ನು ಒಪ್ಪಲೇಬೇಕು. ಶಕ್ತಿಯು ಯಾವ ರೂಪವನ್ನು ಧಾರಣೆ ಮಾಡಿದರೂ ಅದರ ಮೊತ್ತ ಹೆಚ್ಚು ಕಡಮೆ ಆಗುವುದಿಲ್ಲ. ಒಂದು ಮೊತ್ತದ ಶಕ್ತಿ ಬೇಕಾದರೆ ಮತ್ತೊಂದು ರೂಪದಿಂದ ಅದನ್ನು ತುಂಬಲೇಬೇಕು. ರೂಪದ ಬದಲಾವಣೆ ಆಗುತ್ತದೆಯೇ ಹೊರತು ಶಕ್ತಿಯ ಪ್ರಮಾಣದ ಬದಲಾವಣೆಯಾಗುವುದಿಲ್ಲ. ಆದುದರಿಂದ ‘ಬುದ್ಧ’ನು ‘ವಿಕಸನ’ದ ಸರಪಣಿಯ ಒಂದು ತುದಿಯಾದರೆ ‘ಅಮೀಬ’ವಾದರೆ ‘ಅಮೀಬ’ವು ಕೂಡ ಬುದ್ಧನಾಗಿರಲೇಬೇಕು. ‘ಬುದ್ಧ’ನು ವ್ಯಕ್ತವಾದ ‘ಅಮೀಬ’ವಾದರೆ ‘ಅಮೀಬ’ವು ಅವ್ಯಕ್ತವಾದ ‘ಬುದ್ಧ’ನಾಗಿರಬೇಕು. ಈ ಜಗತ್ತು ಅನಂತಶಕ್ತಿಯ ವ್ಯಕ್ತರೂಪವಾದ ಪಕ್ಷದಲ್ಲಿ, ಪ್ರಲಯ ಸ್ಥಿತಿಯಲ್ಲಿಯೂ ಆ ಶಕ್ತಿಯು ಅವ್ಯಕ್ತವಾಗಿರಲೇಬೇಕು. ಅದು ಹಾಗಾಗದೇ ಇರಲು ಸಾಧ್ಯವೇ ಇಲ್ಲ. ಹಾಗೆಂದ ಮೇಲೆ ಪ್ರತಿಯೊಂದು ಆತ್ಮವೂ ಅನಂತವಾದುದು ಎಂದು ಸಿದ್ಧಾಂತವಾಗುತ್ತದೆ. ನಮ್ಮ ಪದತಲದಲ್ಲಿ ತೆವಳುತ್ತಿರುವ ಕ್ಷುದ್ರತಮ ಕ್ರಿಮಿಯಿಂದ ಹಿಡಿದು, ಪವಿತ್ರತಮನಾದ ಮಹರ್ಷಿಯವರೆಗೂ, ಎಲ್ಲರ ಆತ್ಮಗಳಲ್ಲಿಯೂ ಅನಂತಶಕ್ತಿ ಅನಂತ ಪಾವಿತ್ರ್ಯ, ಅನಂತವಾದ ಸಕಲವೂ ಅಡಗಿವೆ.

ಅವುಗಳಲ್ಲಿರುವ ಭೇದವು ಅಭಿವ್ಯಕ್ತಿಯ ತರತಮದಲ್ಲಿ ಅಲ್ಲದೆ ಸ್ವರೂಪದಲ್ಲಿ ಅಲ್ಲ. ಕ್ರಿಮಿಯು ಅನಂತ ಶಕ್ತಿಯಲ್ಲಿ ಸ್ವಲ್ಪ ಮಾತ್ರವನ್ನು ವ್ಯಕ್ತಗೊಳಿಸಿದೆ, ಮನುಷ್ಯನು ಸ್ವಲ್ಪ ಹೆಚ್ಚಾಗಿ ವ್ಯಕ್ತಗೊಳಿಸಿದ್ದಾನೆ, ಮಹರ್ಷಿಗಳೂ ಅವತಾರ ಪುರುಷರೂ ಇನ್ನಷ್ಟು ಹೆಚ್ಚಾಗಿ ವ್ಯಕ್ತಗೊಳಿಸುತ್ತಾರೆ. ಇಷ್ಟೇ ವ್ಯತ್ಯಾಸ. ಆದರೆ ಆ ಅನಂತ ಶಕ್ತಿಯೆಲ್ಲಾ ಒಂದೇ. \textbf{“ತತಃ ಕ್ಷೇತ್ರಿಕವತ್​”}, “ರೈತನು ಗದ್ದೆಗೆ ನೀರು ಹಾಯಿಸಿದಂತೆ,” ಎಂದು ಪತಂಜಲಿ ಹೇಳುತ್ತಾರೆ. ತನ್ನ ಗದ್ದೆಗೆ ಕೆರೆಯ ಸಂಬಂಧವಿರುವ ರೈತನು ಕೆರೆಯ ತೂಬನ್ನು ತೆಗೆದನೆಂದರೆ ನೀರು ತನ್ನಷ್ಟಕ್ಕೆ ತಾನೆ, ತನ್ನ ಶಕ್ತಿಯಿಂದಲೇ ನುಗ್ಗಿ ಬರುತ್ತದೆ. ಹರಿಯುವ ಶಕ್ತಿಯನ್ನು ರೈತನು ನೀರಿಗೆ ಕೊಡುವುದು ಬೇಡ; ಕೆರೆಯಲ್ಲಿ ನಿಂತ ನೀರಿಗೆ ಆ ಶಕ್ತಿಯಿದೆ. ಅಂತೆಯೇ ನಮ್ಮಲ್ಲಿ ಪ್ರತಿಯೊಬ್ಬರ ಹಿನ್ನೆಲೆಯಲ್ಲಿಯೂ ಅನಂತ ಶಕ್ತಿ ಪಾವಿತ್ರ್ಯ ಆನಂದಗಳ ಮಹಾಸಾಗರವಿದೆ. ಆದರೆ ಶರೀರಗಳೆಂಬ ತೂಬುಗಳು ಅವು ಸಂಪೂರ್ಣವಾಗಿ ಅಭಿವ್ಯಕ್ತವಾಗದಂತೆ ತಡೆಗಟ್ಟುತ್ತವೆ ಅಷ್ಟೆ.

ಈ ದೇಹಗಳು ಸೂಕ್ಷ್ಮತರವಾದಂತೆಲ್ಲಾ, ತಮೋಗುಣವು ರಜೋಗುಣವಾಗಿ, ರಜೋಗುಣವು ಸತ್ತ್ವಗುಣವಾದಾಗ, ಆ ಶಕ್ತಿ ಪಾವಿತ್ರ್ಯಗಳು ಹೆಚ್ಚು ಹೆಚ್ಚಾಗಿ ವ್ಯಕ್ತವಾಗುತ್ತವೆ. ಆದ್ದರಿಂದಲೇ ನಮ್ಮವರು ಆಹಾರದ ವಿಷಯದಲ್ಲಿ ಹೆಚ್ಚು ಕಟ್ಟುಪಾಡುಗಳನ್ನು ಮಾಡಿದರು. ಆ ಮೂಲಾಭಿಪ್ರಾಯಗಳು ಇಂದು ಹೋಗಿರಬಹುದು, ವಿವಾಹದ ವಿಚಾರದಲ್ಲಿ ಆಗಿರುವಂತೆ. ಪ್ರಸ್ತುತ ವಿಷಯಕ್ಕೆ ಸಂಬಂಧಿಸದೇ ಇದ್ದರೂ, ಈ ಉದಾಹರಣೆಯನ್ನು ತೆಗೆದುಕೊಳ್ಳುತ್ತೇನೆ. ನನಗೆ ಮತ್ತೊಂದು ಅವಕಾಶ ದೊರೆತರೆ ನಾನು ಈ ವಿಷಯವನ್ನು ಕುರಿತು ಹೆಚ್ಚು ಹೇಳುತ್ತೇನೆ. ಈಗ ಇಷ್ಟು ಮಾತ್ರ ಹೇಳುತ್ತೇನೆ: ನಮ್ಮ ವಿವಾಹ ಪದ್ಧತಿಯ ಹಿಂದಿರುವ ಭಾವನೆಗಳೇ ನಿಜವಾದ ನಾಗರಿಕತೆಯನ್ನು ನಿರ್ಮಿಸಲು ಸಾಧ್ಯ. ಒಬ್ಬ ಸ್ತ್ರೀಯಾಗಲೀ, ಪುರುಷನಾಗಲೀ ತಮ್ಮ ತಮ್ಮ ಮನಸ್ಸಿಗೆ ಬಂದ ಸ್ತ್ರೀಯನ್ನಾಗಲೀ ಪುರುಷನನ್ನಾಗಲೀ ವರಿಸಲು ಬಿಟ್ಟರೆ, ವೈಯಕ್ತಿಕ ಸುಖವೇ, ಮೃಗೀಯ ತೃಷ್ಣೆಗಳ ತೃಪ್ತಿಯೇ ವಿವಾಹದ ಗುರಿಯಾದರೆ, ಸಮಾಜಕ್ಕೆ ಕೇಡು ತಪ್ಪದು. ಅಂತಹ ವಿವಾಹದಿಂದ ದುಷ್ಟರೂ ನೀಚರೂ ಆದ ಮಕ್ಕಳೇ ಹುಟ್ಟುತ್ತಾರೆ. ಹೌದು, ಪ್ರತಿಯೊಂದು ದೇಶದಲ್ಲಿಯೂ ಮನುಷ್ಯರು ಒಂದು ಕಡೆ ದುಷ್ಟಮಕ್ಕಳನ್ನು ಹುಟ್ಟಿಸುತ್ತಿದ್ದಾರೆ; ಇನ್ನೊಂದು ಕಡೆ ಮೃಗ ಸದೃಶರಾದ ಸಂತಾನದವರನ್ನು ಹತೋಟಿಯಲ್ಲಿಡುವುದಕ್ಕೆ ಪೋಲೀಸರ ಪಡೆಯನ್ನು ಹೆಚ್ಚಿಸುತ್ತಿದ್ದಾರೆ. ಸಂಭವಿಸಿದ ಕೇಡನ್ನು ಪರಿಹರಿಸುವುದು ಹೇಗೆ ಎಂಬುದಲ್ಲ ಪ್ರಶ್ನೆ. ಆದರೆ ಕೇಡು ಸಂಭವಿಸದಂತೆ ನೋಡಿಕೊಳ್ಳುವುದು ಹೇಗೆ ಎಂಬುದು. ಒಬ್ಬನು ಸಮಾಜದಲ್ಲಿರುವ ತನಕ ಆತನ ವಿವಾಹದಿಂದ ಇತರರಿಗೆ ಕೇಡಾಗಲೀ ಲೇಸಾಗಲೀ ಸಂಭವಿಸಿಯೇ ತೀರುತ್ತದೆ. ಆದ್ದರಿಂದ ಆ ವ್ಯಕ್ತಿಯು ಯಾರನ್ನು ವಿವಾಹವಾಗಬೇಕು, ಯಾರನ್ನು ಬಿಡಬೇಕು, ಎಂದು ನಿರ್ಣಯಿಸುವ ಹಕ್ಕು ಸಮಾಜಕ್ಕೆ ಇರುತ್ತದೆ. ಇದಕ್ಕಾಗಿಯೇ ಅವರು ವಧೂವರರ ಜಾತಿ – ಜಾತಕಗಳನ್ನು ನೋಡುತ್ತಾರೆ. ಇನ್ನೊಂದು ವಿಷಯವನ್ನು ನಿಮಗೆ ತಿಳಿಸುತ್ತೇನೆ. ಕಾಮ ಮಾತ್ರದಿಂದ ಸಂಭವಿಸಿದ ಶಿಶು ಎಂದಿಗೂ ಆರ್ಯನಲ್ಲ ಎಂಬುದು ಮನುವಿನ ವಚನ. ಯಾವ ಶಿಶುವಿನ ಜನನ ಮತ್ತು ಮರಣ ವೈದಿಕ ನಿಯಮಗಳ ಪ್ರಕಾರ ನಡೆಯುತ್ತದೆಯೋ ಆ ಶಿಶುವು ಮಾತ್ರ ಆರ್ಯನು. ಈಗಲಾದರೋ ಎಲ್ಲೆಲ್ಲಿಯೂ ಅನಾರ್ಯ ಶಿಶುಗಳೇ ಹುಟ್ಟುತ್ತಿರುವುದರಿಂದ ಕಲಿಯುಗವೆಂದು ಕರೆಸಿಕೊಳ್ಳುವ ಕೇಡುಗಾಲ ಪ್ರಾರಂಭವಾಗಿದೆ. ನಮ್ಮ ಸಾಮಾಜಿಕ ಸಂಸ್ಥೆಗಳಿಗೆ ಆಧಾರವಾಗಿದ್ದ ಆ ಉನ್ನತ ಭಾವಗಳು ಇಂದು ನಶಿಸಿ ಹೋಗಿವೆ. ನಿಜ; ಆ ಮಹಾತತ್ತ್ವಗಳೆಲ್ಲವನ್ನೂ ನಾವಿಂದು ಆಚರಿಸಲು ಅಶಕ್ತರಾಗಿದ್ದೇವೆ, ಹಾಗೂ ಇನ್ನು ಕೆಲವು ಭಾವನೆಗಳನ್ನು ಅತ್ಯಂತ ವಿಕೃತಗೊಳಿಸಿದ್ದೇವೆ. ಶೋಚನೀಯ ವಿಷಯವೇನೆಂದರೆ, ಈಗಿನ ಮಾತಾಪಿತೃಗಳು ಹಿಂದಿನ ಕಾಲದವರಂತಿಲ್ಲ. ಇಂದಿನ ಸಮಾಜವೂ ಹಿಂದಿನಷ್ಟು ವಿದ್ಯಾವಂತವಾಗಿಲ್ಲ ಮತ್ತು ವ್ಯಕ್ತಿಯ ಬಗ್ಗೆ ಹಿಂದೆ ಇದ್ದ ಕಳಕಳಿ ಈಗಿನ ಸಮಾಜದಲ್ಲಿಲ್ಲ. ಹಿಂದಿನ ಪದ್ಧತಿಗಳನ್ನು ಕಾರ್ಯತಃ ಪ್ರಯೋಗಿಸುವುದರಲ್ಲಿ ನಾವೆಷ್ಟು ತಪ್ಪಿದ್ದರೂ, ಅವುಗಳ ಹಿಂದಿರುವ ತತ್ತ್ವಗಳು ಮಹತ್ತಾದುವು. ಅವುಗಳ ಪ್ರಯೋಗದಲ್ಲಿ ನಾವು ತಪ್ಪು ಹಾದಿ ಹಿಡಿದಿದ್ದರೆ ತಿದ್ದಿಕೊಳ್ಳುವುದು ನಮ್ಮ ಕೆಲಸವೇ ಹೊರತು, ತತ್ತ್ವವನ್ನೇ ಮೂಲೆಗೆ ಎಸೆಯುವುದು ಶ್ರೇಯಸ್ಕರವಲ್ಲ. ವಿವಾಹದ ವಿಚಾರವಾಗಿ ನಾನಾಡಿದ ಮಾತುಗಳೆಲ್ಲವೂ ಆಹಾರಕ್ಕೂ ಸಲ್ಲುತ್ತವೆ. ನಿಯಮಗಳು ಉಪಯೋಗಿಸುವಾಗ ಅಲ್ಲಲ್ಲಿ ಸಣ್ಣ ಸಣ್ಣ ವಿಷಯಗಳಲ್ಲಿ ತಪ್ಪಿ ಹೋಗಿದ್ದೇವೆ. ನಿಜವಾಗಿಯೂ ಬಹಳ ತಪ್ಪಿ ಹೋಗಿದ್ದೇವೆ. ಆದರೆ ಅದಕ್ಕೆ ತತ್ತ್ವವು ದೋಷಿಯಲ್ಲ. ಅದು ನಿತ್ಯವಾದುದು. ಅದನ್ನು ಹೊಸರೀತಿಯಲ್ಲಿ ಮತ್ತೆ ಪ್ರಯೋಗಿಸಬೇಕಾದುದು ನಮ್ಮ ಕರ್ತವ್ಯ.

‘ಆತ್ಮ’ ಎಂಬ ಈ ಮಹಾಭಾವನೆಯನ್ನು ನಮ್ಮ ದೇಶದ ಎಲ್ಲಾ ಪಂಥದವರೂ ನಂಬಲೇಬೇಕು. ದ್ವೈತಿಗಳು ಮಾತ್ರವೇ, “ಈ ‘ಆತ್ಮ’ವು ಪಾಪಕಾರ್ಯಗಳಿಂದ ಸಂಕುಚಿತವಾಗುತ್ತದೆ, ಅದರ ಶಕ್ತಿ ಸ್ವಭಾವಗಳು ಸಂಕೋಚಗೊಳ್ಳುತ್ತವೆ, ಆದರೆ ಪುಣ್ಯಕಾರ್ಯದಿಂದ ಮತ್ತೆ ‘ಆತ್ಮ’ ಕ್ಕೆ ಮೊದಲಿನ ಸ್ಥಿತಿಯು ಲಭಿಸುತ್ತದೆ” ಎಂದು ಹೇಳುತ್ತಾರೆ. ಅದ್ವೈತಿಗಳಾದರೋ “ಆತ್ಮಕ್ಕೆ ಸಂಕೋಚ ವಿಕಾಸಗಳಿಲ್ಲ. ಆದರೆ ಹಾಗಾಗುವಂತೆ ತೋರುತ್ತದೆ” ಎಂದು ಹೇಳುತ್ತಾರೆ. ಅಷ್ಟೇ ವ್ಯತ್ಯಾಸ. ಅಂತೂ ಆತ್ಮವು ಸಕಲ ಶಕ್ತಿಗಳನ್ನೂ ತನ್ನಲ್ಲಿಯೇ ಒಳಗೊಂಡಿದೆ; ಆಕಾಶದಿಂದೇನೂ ಅದಕ್ಕೆ ಇಳಿದುಬರುವುದಿಲ್ಲ, ಎಂಬುದನ್ನು ಎಲ್ಲರೂ ಒಪ್ಪುತ್ತಾರೆ. ಹೆಚ್ಚೇನು? ವೇದಗಳು ಕೂಡ ಹೊರಗಿನಿಂದ ಬರುವುದಿಲ್ಲ. ಆತ್ಮದಿಂದಲೇ ಪ್ರಕಾಶಿತವಾದವು. ಅವು ಪ್ರತಿಯೊಂದು ಆತ್ಮದಲ್ಲಿಯೂ ಜೀವಂತವಾಗಿರುವ ನಿತ್ಯ ನಿಯಮಗಳು. ಇರುವೆಯ ಆತ್ಮದಲ್ಲಿ ಹೇಗೋ ಹಾಗೆಯೇ ದೇವತೆಗಳ ಆತ್ಮದಲ್ಲಿಯೂ ವೇದದ ತತ್ತ್ವಗಳ ಪ್ರಕಾರವೇ ಕೆಲಸ ನಡೆಯುತ್ತದೆ. ಇರುವೆಯು ವಿಕಾಸವಾಗಿ ಋಷಿಯ ದೇಹವನ್ನು ಪಡೆಯುವುದೇ ತಡ, ವೇದಗಳು ಪ್ರಕಾಶಗೊಳ್ಳುತ್ತವೆ. ಇದು ನಾವು ತಿಳಿದು ಕೊಳ್ಳಲೇ ಬೇಕಾಗಿರುವ ಒಂದು ಮಹತ್ತಾದ ವಿಷಯ. ಅನಂತ ಶಕ್ತಿ ನಮ್ಮಲ್ಲಿ ಈಗಾಗಲೇ ಇದೆ; ಮುಕ್ತಿ ನಮ್ಮೊಳಗೆ ಪವಡಿಸಿದೆ. ನೀವು ಆತ್ಮವು ಸಂಕುಚಿತವಾಗುತ್ತದೆ ಎಂದು ಹೇಳಿ ಅಥವಾ ಅದು ಮಾಯೆಯಿಂದ ಆವೃತವಾಗಿದೆ ಎಂದು ಹೇಳಿ, ಅದರಿಂದ ಯಾವ ಬಾಧೆಯೂ ಇಲ್ಲ. ಪರಿಪೂರ್ಣತೆ ಆಗಲೇ ಅಲ್ಲಿದೆ. ಆತ್ಮದ ಪೂರ್ಣತೆಯಲ್ಲಿ ನಿಮಗೆ ಶ್ರದ್ಧೆಯಿರಲೇಬೇಕು. ಪ್ರತಿಯೊಬ್ಬನೂ ಪೂರ್ಣನಾಗುತ್ತಾನೆ, ಮುಕ್ತನಾಗುತ್ತಾನೆ, ಎಂದು ನೀವು ನಂಬಲೇಬೇಕು. ಅತ್ಯಂತ ಹೀನ ಮನುಷ್ಯನಿಗೂ ಬುದ್ಧನಾಗಲು ಅವಕಾಶವಿದೆ. ಇದೇ ಆತ್ಮದ ಸಿದ್ಧಾಂತ.

\newpage

ಈಗ ಒಂದು ಮಹಾ ಘರ್ಷಣೆ ಎದುರಾಗುತ್ತದೆ. ದೇಹ ಮತ್ತು ಮನಸ್ಸುಗಳು ನಿರಂತರವೂ ಬದಲಾಗುವ ಎರಡು ಸರಣಿಗಳು ಎಂಬುದನ್ನು ವಿಶ್ಲೇಷಣೆಯಿಂದ ಕಂಡು ಹಿಡಿದು ಬೌದ್ಧರು, ಆತ್ಮವಿದೆ ಎಂಬ ನಂಬಿಕೆಯು ಅನಾವಶ್ಯಕ ಎನ್ನುತ್ತಾರೆ. ಆತ್ಮ ಸಿದ್ಧಾಂತ ನಮಗೆ ಬೇಡವೇ ಬೇಡ. ಗುಣಗಳೇ ನಮ್ಮ ಕೆಲಸಕ್ಕೆ ಸಾಕಾಗಿರುವಾಗ, ಗುಣಗಳಿರುವ ನಿತ್ಯ ವಸ್ತುವೊಂದನ್ನು (ಗುಣಿಯನ್ನು) ನಾವೇಕೆ ನಿರ್ಮಿಸಬೇಕು? ಒಂದೇ ಕಾರಣವು ಸಮಗ್ರ ವಿವರಣೆಗೂ ಸಾಕಾಗುವ ಪಕ್ಷದಲ್ಲಿ ಎರಡು ಕಾರಣಗಳನ್ನು ಊಹಿಸುವುದು ತರ್ಕದೋಷ. ಬೌದ್ಧರ ಈ ವಾದದಿಂದ ಮೂಲ ವಸ್ತುವನ್ನು ಪ್ರತಿಪಾದಿಸುವ ಸಿದ್ಧಾಂತಗಳೆಲ್ಲವೂ ನೆಲಕ್ಕುರುಳಿದವು. ಯಾವ ದರ್ಶನಗಳು ‘ಗುಣ’ ‘ಗುಣಿ’ ಗಳನ್ನು ಒಪ್ಪಿದ್ದವೋ, ಪ್ರತಿಯೊಬ್ಬನಲ್ಲಿಯೂ ಮನಸ್ಸು ದೇಹಗಳಿಂದ ಪ್ರತ್ಯೇಕವಾದ ಆತ್ಮವಿದೆ ಮತ್ತು ಆ ಒಂದೊಂದು ಆತ್ಮವೂ ಸಂಪೂರ್ಣವಾಗಿ ವಿಶಿಷ್ಟವಾದುದು ಎಂದು ನಂಬಿದ್ದವೊ, ಆ ದರ್ಶನಗಳೆಲ್ಲ ಬಿರುಕುಬಿಟ್ಟವು.

ಸ್ಥೂಲಶರೀರ, ಅದರ ಹಿಂದೆ ಸೂಕ್ಷ್ಮ ಶರೀರ, ಅದರ ಹಿಂದೆ ಆತ್ಮ ಮತ್ತು ಎಲ್ಲ ಆತ್ಮಗಳನ್ನೂ ವ್ಯಾಪಿಸಿದ ಪರಮಾತ್ಮನಿದ್ದಾನೆ ಎಂಬ ದ್ವೈತಸಿದ್ಧಾಂತವು ನಮಗೆ ಸಮ್ಮತವೆ. ಆದರೆ ಆತ್ಮ ಪರಮಾತ್ಮರು ಗುಣಿಗಳು, ಮನಸ್ಸು ಶರೀರ ಇತ್ಯಾದಿಗಳೆಲ್ಲ ಅವುಗಳನ್ನು ಅವಲಂಬಿಸಿರುವ ಗುಣಗಳು, ಎಂಬ ವಿಷಯದಲ್ಲಿ ತೊಂದರೆ ಬರುತ್ತದೆ. ಆ ಗುಣಿಗಳನ್ನು ಯಾರೂ ಕಂಡಿಲ್ಲ. ಅವುಗಳು ಎಂತಿವೆ ಎಂಬುದು ಕಲ್ಪನೆಗೂ ಅತೀತವಾಗಿದೆ. ಹೀಗಿರುವಾಗ ಅಂತಹ ಆತ್ಮ ವಸ್ತುವಿದೆ ಎಂದು ಊಹಿಸುವುದರಿಂದ ಪ್ರಯೋಜನವೇನು? ಆದ್ದರಿಂದ ಕ್ಷಣಿಕವಾದಿಗಳಾಗಿ “ಇರುವುದೆಲ್ಲ ಚಿತ್ತವೃತ್ತಿಗಳ ಪ್ರವಾಹ” ಎಂದು ಏಕೆ ಹೇಳಬಾರದು? ಅದಲ್ಲದೆ ಮತ್ತೊಂದಿಲ್ಲ, ಅವುಗಳಿಗೆ ಏಕತೆಯಿಲ್ಲ; ಕಡಲ ತೆರೆಗಳಂತೆ ಅವುಗಳು ಒಂದನ್ನೊಂದು ಹಿಂಬಾಲಿಸುತ್ತವೆ. ಇಲ್ಲಿ ಪೂರ್ಣತೆಯಿಲ್ಲ; ಏಕಭಾವವಿಲ್ಲ; ವ್ಯಕ್ತಿತ್ವ ಎಂಬುದು ವೃತ್ತಿಗಳ ಪ್ರವಾಹವಲ್ಲದೆ ಬೇರೆಯಲ್ಲ. ಒಂದು ಅಲೆಯು ಮತ್ತೊಂದು ಅಲೆಗೆ ಕಾರಣವಾಗುತ್ತದೆ. ಈ ವೃತ್ತಿಗಳ ಸಂಪೂರ್ಣ ವಿನಾಶವೇ ನಿರ್ವಾಣ” – ಎಂಬುದು ಬೌದ್ಧರ ವಾದ. ದ್ವೈತವು ಇಲ್ಲಿ ಮೂಕವಾಗಿದೆ. ಈ ವಾದವನ್ನು ಖಂಡಿಸಲು ಅದಕ್ಕೆ ಅಸಾಧ್ಯ. ಆದ್ದರಿಂದ ದ್ವೈತದ ಪರಮಾತ್ಮನಿಗೆ ಇಲ್ಲಿ ಸ್ಥಳವೇ ಇಲ್ಲ. ಸರ್ವವ್ಯಾಪಿಯಾದರೂ, ಕೈಯಿಲ್ಲದೆ ಸೃಷ್ಟಿಸುವ, ಕಾಲಿಲ್ಲದೇ ಚಲಿಸುವ ಕುಂಬಾರನು ಘಟವನ್ನು ತಯಾರುಮಾಡುವಂತೆ ವಿಶ್ವವನ್ನು ರಚಿಸುವ, ಸಗುಣ ದೇವರನ್ನು ನಾವು ಖಂಡಿಸಿಯೇ ತೀರುತ್ತೇವೆ ಎಂದು ಬೌದ್ಧರು ಹೇಳುತ್ತಾರೆ. ಈ ಕಲ್ಪನೆ ಬಾಲಿಶವಾದುದು. ಅಂತಹ ದೇವರು ಆರಾಧನೆಗೆ ಅರ್ಹನಲ್ಲ ಎಂಬುದು ಅವರ ಭಾವನೆ. ಏಕೆಂದರೆ ಈ ದುಃಖಮಯವಾದ ಜಗತ್ತನ್ನು ಸೃಷ್ಟಿಸಿದ ದೇವರು ನಿಂದನೆಗೆ ಯೋಗ್ಯನೇ ಹೊರತು ವಂದನೆಗಲ್ಲ. ಅದೂ ಅಲ್ಲದೆ ಅಂತಹ ಪರಮಾತ್ಮನು ನಿಮಗೆಲ್ಲ ತಿಳಿದಿರುವಂತೆ ತರ್ಕ ಸಾಧ್ಯನೂ ಅಲ್ಲ. ‘ರಚನವಾದ’ \enginline{(design theory)} ವನ್ನು ಕ್ಷಣಿಕವಾದಿಗಳು ಹೇಗೆ ಖಂಡಿಸಿದರೆಂಬ ವಿಚಾರ ನಮಗೀಗ ಬೇಡ. ಇಷ್ಟು ಹೇಳಿದರೆ ಸಾಕು. ಕ್ಷಣಿಕವಾದಿಗಳ ಮುಂದೆ ಸಗುಣ ಸಾಕಾರ ಈಶ್ವರನು ಚೂರು ಚೂರಾಗಿ ಕೆಳಗುರುಳುವನು.

ಸತ್ಯವೊಂದೇ ನಮ್ಮ ಗುರಿ ಎಂಬುದು ಅದ್ವೈತಿಗಳ ಧ್ಯೇಯವಾಕ್ಯ. \textbf{“ಸತ್ಯಮೇವ ಜಯತೇ ನಾನೃತಂ, ಸತ್ಯೇನ ಪಂಥಾ ವಿತತೋ ದೇವಯಾನಃ”} (ಸತ್ಯವೇ ಜಯಿಸುತ್ತದೆ; ಅನೃತವಲ್ಲ. ಸತ್ಯದ ದಾರಿಯೊಂದೇ ನಮ್ಮನ್ನು ದೇವಯಾನಕ್ಕೆ ಕೊಂಡೊಯ್ಯುತ್ತದೆ.) ಆದರೆ ಎಲ್ಲರೂ ಅದೇ ಘೋಷಣೆಯನ್ನು ಉಚ್ಚರಿಸುತ್ತ ತಮ್ಮ ನಿಲುವಿನ ಮೂಲಕ ದುರ್ಬಲರ ನಿಲುವನ್ನು ಖಂಡಿಸುತ್ತ ಹೋಗುತ್ತಾರೆ. ಒಬ್ಬನು ದೇವರ ವಿಚಾರದಲ್ಲಿ ಯಾವುದೋ ಒಂದು ದ್ವೈತ ಭಾವನೆಯನ್ನು ಸರಿಯೆಂದು ಒಪ್ಪಿಕೊಂಡು, ವಿಗ್ರಹವನ್ನು ಪೂಜೆಮಾಡುವ ಇನ್ನೊಬ್ಬನೊಡನೆ ಕಾದಾಡುತ್ತಾನೆ. ತಾನು ಮಾತ್ರ ತೀಕ್ಷ್ಣಮತಿಯೆಂದೂ, ವಿಚಾರಪರನೆಂದೂ, ವಿಗ್ರಹಪೂಜೆಯನ್ನು ಖಂಡಿಸಿಬಿಡುತ್ತೇನೆಂದೂ ಭ್ರಮಿಸುತ್ತಾನೆ. ಆದರೆ ಮೂರ್ತಿ ಪೂಜಕನು ಹಿಂದಕ್ಕೆ ತಿರುಗಿ ‘ನಿನ್ನ ಸಗುಣ ದೇವರೂ ಕೇವಲ ಕಾಲ್ಪನಿಕವಾದದ್ದು’ ಎಂದು ಜರಿದರೆ ನಿಮ್ಮ ಯುಕ್ತಿವಾದಿಯು ಏನು ಹೇಳುತ್ತಾನೆ? ಬಹುಶಃ “ಮೂರ್ತಿ ಪೂಜಕನು ನಾಸ್ತಿಕ” ನೆಂದೂ ಕೂಗೆಬ್ಬಿಸುತ್ತಾನೆ. ಕೈಲಾಗದವರಿಗೆ ಕೂಗೇ ದೊಡ್ಡ ಆಧಾರ. ನಮ್ಮ ವಾದವನ್ನು ಖಂಡಿಸುವವರೆಲ್ಲರೂ ನಾಸ್ತಿಕರು ಎಂದು ಹೇಳುವುದು ಪೂರ್ವದಿಂದಲೂ ನಡೆದು ಬಂದ ಪದ್ಧತಿ. ಆದರೆ ಹಾಗೆ ಮಾಡಬಾರದು. ಯುಕ್ತಿವಾದಿಯಾಗುವ ಪಕ್ಷದಲ್ಲಿ ಯಾವಾಗಲೂ ವಿಚಾರಪರನಾಗಿಯೇ ಇರಬೇಕು. ಇಲ್ಲದೆ ಇದ್ದರೆ ತಾನು ಕೇಳುವ ಸೌಕರ್ಯಗಳನ್ನು ಇತರರಿಗೂ ಕೊಡಬೇಕು. ಸಗುಣ ದೇವರಿದ್ದಾನೆ ಆತನಿಗೆ ಗುಣಗಳಿವೆ; ಆತ್ಮಗಳ ಸಂಖ್ಯೆ ಅನಂತವಾಗಿದೆ; ಪ್ರತಿಯೊಂದು ಆತ್ಮವೂ ವಿಶಿಷ್ಟವಾಗಿದೆ” ಇದನ್ನೆಲ್ಲ ಸತರ್ಕವಾಗಿ ಸಿದ್ಧಾಂತ ಮಾಡುವುದು ಹೇಗೆ? ನಿಮ್ಮ ವೈಯಕ್ತಿಕತೆ ಯಾವುದರಲ್ಲಿದೆ? ದೇಹ ದೃಷ್ಟಿಯಿಂದ ನಿಮಗೆ ವೈಶಿಷ್ಟ್ಯವಿಲ್ಲ, ಏಕೆಂದರೆ ಈ ವಿಷಯದಲ್ಲಿ ಆಗಿನ ಬೌದ್ಧರಿಗಿಂತಲೂ ನಮಗೆ ಈಗ ಹೆಚ್ಚು ಗೊತ್ತಾಗಿದೆ. ಈ ಹಿಂದಿನ ಕ್ಷಣವೊಂದರಲ್ಲಿ ಸೂರ್ಯನಲ್ಲಿದ್ದ ವಸ್ತು ಈ ಕ್ಷಣದಲ್ಲಿ ನಿಮ್ಮ ದೇಹದಲ್ಲಿ ಇರಬಹುದು. ಮರುಕ್ಷಣದಲ್ಲಿ ಅದು ಒಂದು ವೃಕ್ಷಕ್ಕೇ ಸೇರಬಹುದು. ಹಾಗಾದರೆ ನಿಮ್ಮ ದೇಹದ ಪೃಥಕ್ತ್ವವೆಲ್ಲಿ? ವೈಯಕ್ತಿಕತೆ ಎಲ್ಲಿ? ಇದೇ ವಾದವು ಮನಸ್ಸಿಗೂ ಅನ್ವಯಿಸುತ್ತದೆ. ನಿಮ್ಮ ಮನಸ್ಸಿನ ‘ಪ್ರತ್ಯೇಕತೆ’ ಎಲ್ಲಿದೆ? ಈ ಹೊತ್ತು ಮನಸ್ಸಿನಲ್ಲಿ ಒಂದು ಆಲೋಚನೆ ಇದ್ದರೆ ನಾಳೆ ಮತ್ತೊಂದಿರುತ್ತದೆ. ಶಿಶುಗಳಾಗಿದ್ದಾಗ ನಿಮ್ಮ ಮನಸ್ಸಿದ್ದಂತೆ ಈಗಿಲ್ಲ. ಮುದಿತನದ ಮನಸ್ಸು ತಾರುಣ್ಯದ ಮನಸ್ಸಲ್ಲ. ಹೀಗಿದ್ದ ಮೇಲೆ ನಿಮ್ಮ ಮನಸ್ಸಿನ ವೈಶಿಷ್ಟ್ಯವೆಲ್ಲಿ? ಅದು ನಮ್ಮ ಪ್ರಜ್ಞೆಯಲ್ಲಿದೆ ಅಥವಾ ಅಹಂಕಾರದಲ್ಲಿದೆ ಎನ್ನಬೇಡಿ. ಏಕೆಂದರೆ ಅದು ನಿಮ್ಮ ಅಸ್ತಿತ್ವದ ಒಂದು ಅಂಶ ಮಾತ್ರವಾಗಿದೆ. ಈಗ ನಾನು ನಿಮ್ಮೊಡನೆ ಮಾತನಾಡುತ್ತಿದ್ದೇನೆ. ನನ್ನ ಎಲ್ಲಾ ಇಂದ್ರಿಯಗಳು ಕೆಲಸಮಾಡುತ್ತಿವೆ. ಆದರೆ ನನಗೆ ಅದರ ಪ್ರಜ್ಞೆಯಿಲ್ಲ (ತಿಳಿವಿಲ್ಲ). ಪ್ರಜ್ಞೆಯಲ್ಲಿರುವುದು ಮಾತ್ರ ಇರುವುದಾದ ಪಕ್ಷದಲ್ಲಿ ಅವುಗಳಲ್ಲವೆಂದೇ ಹೇಳಬೇಕು. ಏಕೆಂದರೆ ನನಗೀಗ ಅವುಗಳ ತಿಳಿವಿಲ್ಲ. ಹಾಗಾದರೆ ನಿಮ್ಮ ನೆಚ್ಚಿನ ಸಗುಣ ದೇವರು ಅಥವಾ ಈಶ್ವರ ಸಿದ್ಧಾಂತಗಳೆಲ್ಲಿ? ಅವುಗಳನ್ನು ಹೇಗೆ ಸಿದ್ಧಗೊಳಿಸಿ ತೋರಿಸುತ್ತೀರಿ?

ಬೌದ್ಧರ ಪ್ರಕಾರ ಈಶ್ವರ ಭಾವವು ತರ್ಕ ವಿರುದ್ಧವಾದುದು ಮಾತ್ರವಲ್ಲ ಅನೈತಿಕವಾದದು; ಏಕೆಂದರೆ ಅದು ಮನುಷ್ಯನನ್ನು ಹೇಡಿಯನ್ನಾಗಿ ಮಾಡುತ್ತದೆ. ಮನುಷ್ಯನು ತಾನೇ ಪ್ರಯತ್ನಿಸದೆ ಯಾವಾಗಲೂ ಇಲ್ಲದ ದೇವರ ಮೇಲೆ ಭಾರಹಾಕುತ್ತಾನೆ; ಆತನನ್ನೆ ಸದಾ ಸಹಾಯಕ್ಕೆ ಕರೆಯುತ್ತಾನೆ. ಆ ಸಹಾಯವೂ ಬರಿಯ ಮಿಥ್ಯೆ. ಹಾಗೆ ಸಹಾಯ ಮಾಡಲು ಯಾರಿಂದಲೂ ಸಾಧ್ಯವಿಲ್ಲ. ಈ ಜಗತ್ತು ಮಾನವ ನಿರ್ಮಿತವಾದುದು; ಸುಮ್ಮನೆ ಒಬ್ಬ ಊಹೆಯ ಸೃಷ್ಟಿಕರ್ತನನ್ನು ನಿರ್ಮಿಸುವುದೇಕೆ? ಆತನನ್ನು ಕಂಡವರಿಲ್ಲ; ಕೇಳಿದವರಿಲ್ಲ. ಆತನಿಂದ ಸಹಾಯ ಪಡೆದವರೂ ಇಲ್ಲ. ಹೀಗಿರಲು ನಿಮ್ಮನ್ನು ನೀವೇ ಹೇಡಿಗಳನ್ನಾಗಿ ಮಾಡಿಕೊಳ್ಳುವುದೇಕೆ? ನಾನು ಕೈಲಾಗದವನು, ನಾನು ಪಾಪಿ ಇತ್ಯಾದಿಯಾಗಿ ಹೇಳಿಕೊಳ್ಳುತ್ತ, ಆ ಊಹೆಯ ಸೃಷ್ಟಿಕರ್ತನ ಮುಂದೆ ನಾಯಿಯಂತೆ ಹಲ್ಲುಕಿರಿದು ಅಂಗಲಾಚಿ ಬೇಡಿಕೊಂಡು ಹೊರಳಾಡುವುದೇ ಜೀವನದ ಪರಮ ಪುರುಷಾರ್ಥವೆಂದು ನಿಮ್ಮ ಮಕ್ಕಳಿಗೇಕೆ ಕಲಿಸುತ್ತೀರಿ? ನೀವು ಮಾಡುವುದು ಮಿಥ್ಯಾಬೋಧೆ ಮಾತ್ರವಲ್ಲ; ನಿಮ್ಮ ಬೋಧೆಯಿಂದ ನಿಮ್ಮ ಮಕ್ಕಳಿಗೆ ಬಹಳ ಕೇಡು ಸಂಭವಿಸುತ್ತದೆ. ಏಕೆಂದರೆ ಚೆನ್ನಾಗಿ ಕೇಳಿ, ಈ ಜಗತ್ತು ಸಮ್ಮೋಹಿನೀ ವಿದ್ಯೆಯಿಂದ ರಚಿತವಾಗಿದೆ. ಬುದ್ಧ ದೇವನು ಹೇಳಿದ ಮೊದಲ ಮಾತುಗಳೇ ಇವು: ನೀವೆಂತು ಭಾವಿಸಿದರೆ ಅಂತಾಗುತ್ತೀರಿ. ನೀವು ಏನನ್ನು ಆಲೋಚಿಸುತ್ತೀರೋ ಅದೇ ಆಗುತ್ತೀರಿ. ಇದು ಸತ್ಯವಾದರೆ “ನನ್ನಿಂದ ಏನೂ ಆಗುವುದಿಲ್ಲ. ನಾನೊಂದು ಕ್ರಿಮಿ; ಮುಗಿಲಿನಾಚೆಯ ಮಹೇಶ್ವರನು ನೆರವಾದ ಹೊರತು, ನನ್ನ ಕೈಯಲ್ಲಿ ಏನೂ ಸಾಗುವುದಿಲ್ಲ” ಎಂದು ಹೇಳಿಕೊಳ್ಳಬೇಡಿ. ಹಾಗೆ ಪ್ರಾರ್ಥಿಸಿದರೆ ನೀವು ದಿನದಿನವೂ ಹೆಚ್ಚು ಹೆಚ್ಚು ದುರ್ಬಲರಾಗುತ್ತೀರಿ. “ನಾನು ಪಾಪಿ, ನನ್ನನ್ನು ಶುದ್ಧಮಾಡು” ಎಂದು ಪದೇ ಪದೇ ಹೇಳಿಕೊಳ್ಳುವುದರಿಂದ ನಾನು ಪಾಪಿ ಎಂಬ ಸೂಚನೆಯನ್ನು ಸ್ವೀಕರಿಸಿದ ಸುಪ್ತಚಿತ್ತವು ನಮ್ಮನ್ನು ಮತ್ತೂ ಹೆಚ್ಚಾದ ದುರಾಚಾರಗಳಿಗೇ ಕೊಂಡೊಯ್ಯುತ್ತದೆ. ಅಷ್ಟೇ ಅಲ್ಲ, ಈ ಸಗುಣ ಸಾಕಾರ ಈಶ್ವರನ ಆರಾಧನೆ ಪ್ರಾರ್ಥನೆಗಳೇ ಮಾನವನ ಅನೇಕ ದುರ್ಬಲತೆಗಳಿಗೂ ಪಾಪಗಳಿಗೂ ಕಾರಣವಾಗುತ್ತವೆ ಎಂಬುದು ಬೌದ್ಧರ ಅಭಿಪ್ರಾಯ. ಈ ಅದ್ಭುತ ಮಾನವ ಜನ್ಮವನ್ನು ಪಡೆದು ನಾಯಿಯಂತೆ ಬಾಳುವುದು ಅತ್ಯಂತ ಅಸಹ್ಯಕರ. ಬೌದ್ಧನು ಹೇಳುತ್ತಾನೆ: “ಓ ವೈಷ್ಣವನೇ, ನಿನ್ನ ಗುರಿ, ಉದ್ದೇಶ, ಆದರ್ಶ ಎಲ್ಲವೂ ವಿಷ್ಣುವಿರುವ ವೈಕುಂಠಕ್ಕೆ ಹೋಗಿ, ಅಲ್ಲಿ ಕೈಕಟ್ಟಿಕೊಂಡು ಅನಂತಕಾಲದವರೆಗೆ ನಿಂತಿರುವುದಾದ ಪಕ್ಷದಲ್ಲಿ, ಅದಕ್ಕಿಂತಲೂ ಆತ್ಮಹತ್ಯೆ ಮಾಡಿಕೊಳ್ಳುವುದೇ ಲೇಸು.” ಅಂತಹ ಗುಲಾಮಗಿರಿಯನ್ನು ತಪ್ಪಿಸಿಕೊಳ್ಳಲೋಸುಗವೇ ಬೌದ್ಧನು ನಿರ್ವಾಣವನ್ನು ಬಯಸುತ್ತಾನೆ. ನಾನು ತಾತ್ಕಾಲಿಕವಾಗಿ ಬೌದ್ಧನಾಗಿ ಈ ವಿಷಯಗಳನ್ನು ಪ್ರತಿಪಾದಿಸುತ್ತಿದ್ದೇನೆ. ಏಕೆಂದರೆ, ಅನೇಕರು ನಿರ್ಗುಣ ನಿರಾಕಾರ ಬ್ರಹ್ಮೋಪಾಸನೆ ದುರ್ನೀತಿಗೆ ಕಾರಣವಾಗುತ್ತದೆ ಎಂದು ಹೇಳುತ್ತಿರುವುದರಿಂದ ಅಂಥವರಿಗೆ, ತಮ್ಮ ಪ್ರತಿಪಕ್ಷದವರ ಮತವೂ ತಿಳಿಯಲಿ ಎಂಬುದು ನನ್ನ ಇಚ್ಛೆ. ಎರಡು ಪಕ್ಷಗಳನ್ನು ಧೈರ್ಯವಾಗಿ ಎದುರಿಸೋಣ.

ಸೃಷ್ಟಿಕರ್ತನಾದ ಸಗುಣ ಸಾಕಾರನಾದ ದೇವರನ್ನು ಪ್ರಮಾಣೀಕರಿಸಲಾಗುವುದಿಲ್ಲ ಎಂಬುದನ್ನು ಈ ಮೊದಲೇ ನೋಡಿದ್ದೇವೆ. ಇಂದು ಇದನ್ನು ನಂಬುವಂತಹ ಶಿಶು ಇದೆಯೇನು? ಕುಂಬಾರನು ಮಡಕೆ ಮಾಡುವಂತೆ ದೇವರು ಸೃಷ್ಟಿಮಾಡುತ್ತಾನಂತೆ. ಹಾಗಾದರೆ ಕುಂಬಾರನೂ ದೇವರಾಗಬೇಕು. ಆದರೆ ‘ಕೈಗಳೂ ತಲೆಯೂ ಇಲ್ಲದೆ ಕುಂಬಾರನು ಮಡಕೆಮಾಡುತ್ತಾನೆ’ ಎಂದು ಯಾರಾದರೂ ನಿಮಗೆ ಹೇಳಿದರೆ ‘ಅವನನ್ನು ಹುಚ್ಚರ ಆಸ್ಪತ್ರೆಗೆ ಕಳುಹಿಸಿ’ ಎನ್ನುತ್ತೀರಿ. ಹಾಗಾದರೆ ದೇವರು ಹಸ್ತ ಮಸ್ತಕಗಳಿಲ್ಲದೆ ಸೃಷ್ಟಿ ಮಾಡುತ್ತಾನೆ, ಎಂದಾಗ ನೀವೇಕೆ ಹಾಗೆ ಹೇಳುವುದಿಲ್ಲ? ಆ ಹಳೆಯ ಬೌದ್ಧರ ಜೊತೆಗೆ ಈಗ ಹೊಸ ವೈಜ್ಞಾನಿಕರೂ “ಯಾರನ್ನು ಕುರಿತು ಜೀವಮಾನವೆಲ್ಲಾ ಮೊರೆಯಿಡುತ್ತಿರೋ, ಆ ನಿಮ್ಮ ಉಪಾಸನೆಯ ದೇವರು, ಆ ನಿಮ್ಮ ಸೃಷ್ಟಿಕರ್ತನು, ನಿಮಗೆ ಎಂದಾದರೂ ಏನಾದರೂ ಸಹಾಯ ಮಾಡಿದ್ದಾನೆಯೊ?” ಎಂಬ ಸವಾಲನ್ನು ಎಸೆಯುತ್ತಾರೆ. ನೀವು ಒಂದು ವೇಳೆ ‘ಹೌದು ಸಹಾಯ ದೊರೆತಿದೆ’ ಎಂದು ಹೇಳಿದರೆ ಆ ಸಹಾಯವೆಲ್ಲ ನಿಮ್ಮ ಪ್ರಯತ್ನದಿಂದಲೇ ಬಂದುದು, ಸುಮ್ಮನೆ ಭ್ರಮೆಯಿಂದ ಕ್ಲೇಶಪಟ್ಟುದೇ ಆಯಿತು. ಅತ್ತು ಕರೆಯದೆ ಇದ್ದಿದ್ದರೆ ಈಗಾಗಿರುವುದಕ್ಕಿಂತಲೂ ಉತ್ತಮವಾದ ಪರಿಣಾಮವಾಗುತ್ತಿತ್ತು, ಎಂದು ಆಧುನಿಕ ವಿಜ್ಞಾನಿಗಳು ಹೇಳುತ್ತಾರೆ. ಈ ನಿಮ್ಮ ‘ಸಗುಣ ಭಗವಂತ’ ನ ಉಪಾಸನೆಯ ದೆಸೆಯಿಂದಲೇ ರಾಜರ ನಿರಂಕುಶ ವರ್ತನೆಯೂ ಪುರೋಹಿತಷಾಹಿಯೂ ಹಬ್ಬಿಹೋಗಿದೆ. ಪುರೋಹಿತಷಾಹಿಯೂ ನಿರಂಕುಶ ಪ್ರಭುಗಳೂ ಯಾವಾಗಲೂ ಜೊತೆಯಾಗಿಯೇ ಇರುವುವು. ಈ ಮಹಾ ಮಿಥ್ಯೆಯ ತಲೆಯನ್ನು ಮೊಟಕಿದ ಹೊರತು, ಲೋಕದಲ್ಲಿ ದಬ್ಬಾಳಿಕೆ ತಪ್ಪಿದ್ದಲ್ಲ ಎಂದು ಬೌದ್ಧರು ಹೇಳುತ್ತಾರೆ. ಎಲ್ಲಿಯವರೆಗೆ ಮನುಷ್ಯನು ತನಗಿಂತಲೂ ಹೆಚ್ಚು ಶಕ್ತನಾದವನೊಬ್ಬನಿಗೆ ಮಣಿಯುತ್ತಾನೋ, ಅವನು ದೇವನಾಗಲೀ ಮನುಜನಾಗಲೀ– ಅಲ್ಲಿಯವರೆಗೆ ಅರ್ಚಕರಿಗೆ ಭಕ್ತರನ್ನು ಶರಣುಮಾಡಿಕೊಳ್ಳುವ ಹಕ್ಕು ಬಾಧ್ಯತೆಗಳು ಇದ್ದೇ ಇರುತ್ತವೆ. ಏಕೆ? ಬಡ ಮಾನವನು ತನಗೂ ತನ್ನ ಆರಾಧನೆಯ ದೇವತೆಗೂ ಮಧ್ಯಸ್ಥಗಾರನೊಬ್ಬನನ್ನು ಕಲ್ಪಿಸಿಯೇ ಕಲ್ಪಿಸುತ್ತಾನೆ. ಆದರೆ ಬ್ರಾಹ್ಮಣ ಪೂಜಾರಿಗಳನ್ನು ವರ್ಜಿಸಿದ ಮಾತ್ರದಿಂದಲೇ ಈ ದೋಷ ಪರಿಹಾರವಾಗುವುದಿಲ್ಲ. ಅವರನ್ನು ಪದಚ್ಯುತಗೊಳಿಸಿದವರೇ ಆ ಪಟ್ಟಕ್ಕೆ ಏರುತ್ತಾರೆ. ಹೊಸಬರು ಹಳಬರಿಗಿಂತಲೂ ಹೆಚ್ಚು ನಿರಂಕುಶ ಕ್ರೂರಿಗಳಾಗುತ್ತಾರೆ. ಭಿಕ್ಷುಕನಿಗೆ ಐಶ್ವರ್ಯ ಸಿಕ್ಕಿದರೆ ಅವನು ಇಡೀ ಪ್ರಪಂಚವನ್ನು ಹುಲ್ಲುಕಡ್ಡಿಗೆ ಸಮಾನವಾಗಿ ಭಾವಿಸುತ್ತಾನೆ. ಅಂತೂ ಸಗುಣ ‘ಭಗವಂತ’ ಭಾವವು ಇರುವ ತನಕ, ಪುರೋಹಿತರು ಇದ್ದೇ ಇರುತ್ತಾರೆ. ಹಾಗೂ ಸಮಾಜದಲ್ಲಿ ಉನ್ನತ ನೈತಿಕತೆಯೂ ಇರಲಾರದು. ನಿರಂಕುಶ ಪ್ರಭುತ್ವವೂ, ಪೌರೋಹಿತ್ಯವೂ ಕೈ ಕೈ ಕುಲುಕಿಕೊಂಡು ನಡೆಯುತ್ತವೆ. ಇದು ಹೀಗಾದುದು ಹೇಗೆ? ಬಹುಪ್ರಾಚೀನ ಕಾಲದಲ್ಲಿ ಕೆಲವು ಜನ ಬಲಿಷ್ಠರು ಕೆಲವು ಅಬಲರನ್ನು ಹಿಡಿದುಕೊಂಡು “ನಮ್ಮ ಮಾತು ಕೇಳದಿದ್ದರೆ ನಿಮ್ಮನ್ನು ಬಲಿ ಹಾಕುತ್ತೇವೆ” ಎಂದು ಆರ್ಭಟಿಸಿದರು. ಅಂದಿನಿಂದ ಈ ಪ್ರಸಂಗ ಪ್ರಾರಂಭವಾಯಿತು. ಮಹದ್ಭಯಂ ವಜ್ರಮುದ್ಯತಮ್​ – ತನ್ನ ಮಾತನ್ನು ಕೇಳದವರನ್ನು ಕೊಲ್ಲುವ ರುದ್ರನ ಭಾವನೆಯಿದು.

ಮತ್ತೆ ಬೌದ್ಧರು ಹೇಳುತ್ತಾರೆ: “ಪ್ರತಿಯೊಂದೂ ಕರ್ಮ ನಿಯಮದ ಫಲ” ಎಂಬ ನಿಮ್ಮ ಅಭಿಪ್ರಾಯ ಯುಕ್ತಿಯುಕ್ತವಾಗಿದೆ. ಅಸಂಖ್ಯ ಆತ್ಮಗಳಿವೆ, ಆ ಆತ್ಮಗಳಿಗೆ ಜನನ ಮರಣಗಳಿಲ್ಲ, ಎಂದು ನೀವು ನಂಬುತ್ತೀರಿ. ಈ ಆತ್ಮಗಳ ಅಸಂಖ್ಯತೆ ಮತ್ತು ಕರ್ಮಸಿದ್ಧಾಂತ ಸಂಪೂರ್ಣ ತರ್ಕಬದ್ಧವಾದುದು. ಪ್ರತಿಯೊಂದು ಕಾರ್ಯಕ್ಕೂ ಒಂದೊಂದು ಕಾರಣವಿದ್ದೇ ಇರುತ್ತದೆ. ಭೂತಕಾಲ, ವರ್ತಮಾನಕಾಲಕ್ಕೆ ಕಾರಣ. ವರ್ತಮಾನ, ಭವಿಷ್ಯತ್ತಿಗೆ ಕಾರಣ. ಹಿಂದೂ “ಕರ್ಮವು ಜಡ, ಅದು ಫಲಕಾರಿಯಾಗಬೇಕಾದರೆ ಚೈತನ್ಯದ ಅವಶ್ಯಕತೆಯಿದೆ;” ಎಂದು ಹೇಳುತ್ತಾನೆ. ಒಂದು ಗಿಡವು ಹಣ್ಣು ಬಿಡುವಂತೆ ಮಾಡಲು ಚೈತನ್ಯ ಬೇಕು ಎಂಬುದು ನಿಜವೇ? ಬೀಜವನ್ನು ನೆಟ್ಟು ನೀರು ಹಾಕಿದೆನೆಂದರೆ ಯಾವ ಚೈತನ್ಯದ ಸಹಾಯವೂ ಬೇಡ, ಗಿಡವಾಗಿರುತ್ತದೆ. ನೀವು “ಚೈತನ್ಯ ಮೊದಲೇ ಅದರಲ್ಲಿತ್ತು” ಎಂದು ಹೇಳಬಹುದು. ಆದರೆ ಆತ್ಮಗಳು ತಾವೇ ಚೈತನ್ಯವಾದುದರಿಂದ ಅವುಗಳಿಗೆ ಇತರ ಚೈತನ್ಯದ ಸಹಾಯ ಅವಶ್ಯವಲ್ಲ. ಆತ್ಮಗಳಲ್ಲಿ ನಂಬಿಕೆಯಿಟ್ಟರೂ ಈಶ್ವರನನ್ನು ನಂಬದ ಜೈನರು ಹೇಳುವಂತೆ, ಮನುಷ್ಯರ ಆತ್ಮಗಳಿಗೆಯೇ ಚೈತನ್ಯವಿರಲಾಗಿ ಅವುಗಳಿಗೆ ಈಶ್ವರನ ಅವಶ್ಯಕತೆಯೇನು? ನಿಮ್ಮ ತರ್ಕ ಎಲ್ಲಿ? ನಿಮ್ಮ ನೈತಿಕತೆ ಎಲ್ಲಿ? ಅದ್ವೈತ ದರ್ಶನವು ದುರ್ನೀತಿಗಳಿಗೆ ಕಾರಣಗಳಾಗುತ್ತದೆಂದು ಅದನ್ನು ಖಂಡಿಸುವ ಮಹನೀಯರು ದ್ವೈತ ಮತಗಳು ಏನು ಮಾಡಿವೆ ಎಂಬುದರ ಕಡೆಗೆ ಸ್ವಲ್ಪ ಕಣ್ಣಿಟ್ಟು ನೋಡಲಿ. ಇಪ್ಪತ್ತು ಸಾವಿರ ಅದ್ವೈತಿ ನೀಚರು ಇರುವುದು ಹೌದಾದರೆ ಇಪ್ಪತ್ತು ಸಾವಿರ ದ್ವೈತಿ ನೀಚರು ಇರುವುದು ಹೌದು. ಸಾಧಾರಣವಾಗಿ ನೋಡಿದರೆ ನೀಚರಾದ ದ್ವೈತಿಗಳೇ ಹೆಚ್ಚು. ಏಕೆಂದರೆ, ಅದ್ವೈತವನ್ನು ಅರ್ಥ ಮಾಡಿಕೊಳ್ಳಲು ಉನ್ನತತರವಾದ ಮನಸ್ಸು ಬೇಕಾಗುತ್ತದೆ. ಅದ್ವೈತಿಗಳಿಗೆ ಯಾವುದನ್ನು ಹೇಳಿಯೂ ಅಂಜಿಸುವುದು ಸಾಧ್ಯವಿಲ್ಲ. ಇಂತಿರಲು ನಿಮ್ಮ ವಾದದ ಗತಿಯೇನು? ಬೌದ್ಧರ ಮುಷ್ಟಿಯಿಂದ ನಿಮಗಂತೂ ಉಳಿಗಾಲವಿಲ್ಲ. ಮಹೇಶ್ವರನಿದ್ದಾನೆ ಎಂಬುದಕ್ಕೆ ನೀವು ವೇದ ಪ್ರಮಾಣಗಳನ್ನು ಕೊಡಬಹುದು. ಆದರೆ ಅವರು ವೇದಗಳನ್ನು ನಂಬುವುದಿಲ್ಲ. ಅವರು ಹೆಳುತ್ತಾರೆ: “ನಮ್ಮ ತ್ರಿಪಿಟಕಗಳು ಈಶ್ವರನಿಲ್ಲ ಎಂದು ಹೇಳುತ್ತವೆ. ನಮ್ಮ ತ್ರಿಪಿಟಕಗಳಿಗೂ ಆದ್ಯಂತಗಳಿಲ್ಲ, ಅವುಗಳನ್ನು ಬರೆದವನು ಬುದ್ಧನಲ್ಲ. ಏಕೆಂದರೆ ಬುದ್ಧನೇ ತಾನು ಅವುಗಳನ್ನು ಉಚ್ಚರಿಸುತ್ತೇನೆ ಎಂದು ಮಾತ್ರ ಹೇಳುತ್ತಾನೆ. ಆದ್ದರಿಂದ ಅವುಗಳೂ ಸನಾತನವಾದುವು” ಇಷ್ಟು ಹೇಳುವುದರ ಜೊತೆಗೆ ಅವರು, “ನಿಮ್ಮ ವೇದಗಳು ಮಿಥ್ಯೆ, ನಮ್ಮ ವೇದಗಳೇ ಸತ್ಯ. ಏಕೆಂದರೆ ನಿಮ್ಮ ವೇದಗಳು ಬ್ರಾಹ್ಮಣ ಪುರೋಹಿತರಿಂದ ನಿರ್ಮಿತವಾಗಿವೆ. ಆದ್ದರಿಂದ ಅವನ್ನು ಬಿಸಾಡಿ” ಎಂಬುದನ್ನೂ ಸೇರಿಸಿದರೆ ನೀವೇನು ಮಾಡುವಿರಿ? ತಪ್ಪಿಸಿಕೊಳ್ಳುವುದು ಹೇಗೆ?

ಇಲ್ಲಿದೆ ಪಾರಾಗಲು ಒಂದು ಹಾದಿ. ಗುಣಗಳು ಮತ್ತು ಗುಣಿಗಳು ಬೇರೆ ಎಂಬ ಮೊದಲನೆಯ ತಾತ್ವಿಕ ಪೂರ್ವಪಕ್ಷವನ್ನು ಸಮಾಲೋಚಿಸುವ ಅದ್ವೈತಿಯು ಅವುಗಳು ಬೇರೆ ಬೇರೆ ಅಲ್ಲ ಎಂದು ಹೇಳುತ್ತಾನೆ. ಗುಣಗಳಿಗೆ ಮತ್ತು ಗುಣಿಗಳಿಗೆ ಭಿನ್ನಭಾವವಿಲ್ಲ. ಸರ್ಪ–ರಜ್ಜು ಭ್ರಮೆ ಎಂಬ ಸಂಪ್ರದಾಯದ ದೃಷ್ಟಾಂತವು ತಮಗೆಲ್ಲ ತಿಳಿದಿದೆ. ಹಾವು ಕಾಣುವಾಗ ಹಗ್ಗ ಕಾಣುವುದೇ ಇಲ್ಲ. ಹಗ್ಗ ಮಾಯವಾಗುತ್ತದೆ. ಒಂದು ಪದಾರ್ಥವನ್ನು ಗುಣ ಮತ್ತು ಗುಣಿ ಎಂದು ಮಾಡುವ ವಿಭಜನೆಯು ತಾತ್ತ್ವಿಕರ ತಲೆಯಲ್ಲಿ ಅಲ್ಲದೆ ವಾಸ್ತವಿಕವಾದ ಅನುಭವದಲ್ಲಿ ಇಲ್ಲ. ನೀವು ಸಾಧಾರಣ ಮನುಷ್ಯರಾದರೆ ಗುಣವನ್ನೇ ನೋಡುತ್ತೀರಿ; ಮಹಾ ಯೋಗಿಗಳಾದರೆ ಗುಣಿಯನ್ನು ನೋಡುತ್ತೀರಿ. ಆದರೆ ಎರಡೂ ಒಟ್ಟಿಗೆ ಕಾಣುವುದು ಸರ್ವಥಾ ಅಸಾಧ್ಯ. ಆದ್ದರಿಂದ, ಬೌದ್ಧರೇ, ಗುಣಗುಣಿಗಳ ಭೇದದ ವಿಷಯದಲ್ಲಿರುವ ಈ ಕಲಹವು ಭ್ರಾಂತಿಮೂಲವಾದುದು, ಅನುಭವವು ಅದನ್ನು ಒಪ್ಪಲಾರದು. ಆದ್ದರಿಂದ ವಸ್ತುವು ಗುಣರಹಿತವಾದುದಾದರೆ ಒಂದೇ ವಸ್ತುವಿರಬೇಕು. ಆತ್ಮದಿಂದ ಗುಣಗಳನ್ನೆಲ್ಲ ತೆಗೆದು ಹಾಕಿ, ಮನಸ್ಸಿನಲ್ಲಿ ಮಾತ್ರ ಇರುವ ಗುಣಗಳು ಆತ್ಮದಲ್ಲಿ ಆಧ್ಯಾರೋಪವಾಗಿವೆ ಎಂದು ತೋರಿಸಿದರೆ, ಇರುವುದೊಂದೇ ಆತ್ಮ ಎಂದು ಸಿದ್ಧಾಂತಪಡಿಸಿದಂತಾಗುತ್ತದೆ. ಏಕೆಂದರೆ ಒಂದು ಆತ್ಮಕ್ಕೂ ಮತ್ತೊಂದು ಆತ್ಮಕ್ಕೂ ಭೇದಭಾವವಿರುವುದು ಗುಣಗಳಿಂದಲೇ. ಒಂದು ಆತ್ಮದಿಂದ ಮತ್ತೊಂದು ಆತ್ಮವು ಬೇರೆಯಾಗಿದೆ ಎಂದು ನಿಮಗೆ ಗೊತ್ತಾಗುವುದು ಹೇಗೆ? ಗುಣ ಭೇದದಿಂದ, ಗುಣಗಳಿಲ್ಲದಿದ್ದರೆ ಭೇದವೆಲ್ಲಿರುತ್ತದೆ? ಆದ್ದರಿಂದ ಅನೇಕಾತ್ಮಗಳ ಮಾತಂತಿರಲಿ, ಎರಡು ಆತ್ಮಗಳೂ ಅಲ್ಲ, ಇರುವುದೊಂದೇ ಆತ್ಮ. ಪರಮಾತ್ಮನೊಬ್ಬನಿದ್ದಾನೆಂದು ನಂಬುವುದೂ ಅನಾವಶ್ಯಕ. ಏಕೆಂದರೆ ಈ ಆತ್ಮವೇ ಆ ಪರಮಾತ್ಮ. ಯಾವ ವಸ್ತುವನ್ನು ಪರಮಾತ್ಮನೆಂದು ಕರೆಯುತ್ತಾರೋ ಅದೇ ಜೀವಾತ್ಮನೆಂದೂ ಕರೆಯಲ್ಪಡುತ್ತಿದೆ. ಆತ್ಮವನ್ನು ಸರ್ವವ್ಯಾಪಿ, ವಿಭು ಎಂದು ಕರೆಯುವ ಸಾಂಖ್ಯರೇ ಮೊದಲಾದ ದ್ವೈತಿಗಳು ಎರಡು ಅನಂತಗಳು ಇವೆ ಎಂಬುದನ್ನು ಸ್ಥಾಪಿಸುವುದು ಹೇಗೆ ಸಾಧ್ಯ? ಒಂದೇ ಒಂದು ಇರುವುದು ಸಾಧ್ಯ. ಹಾಗಾದರೆ ಈ ಅದ್ವಿತೀಯ ವಿಭುವೇ ಆತ್ಮ, ಉಳಿದುದೆಲ್ಲ ಅದರ ಅಭಿವ್ಯಕ್ತಿ ಮಾತ್ರ. ಇದೇ ಬೌದ್ಧರಿಗೆ ತಕ್ಕ ಪ್ರತಿವಾದ. ಆದರೆ ಇಷ್ಟಕ್ಕೆ ಮುಗಿಯಲಿಲ್ಲ.

ಅದ್ವೈತವು ಬರಿಯ ಖಂಡನಾತ್ಮಕವಾದುದಲ್ಲ. ಇತರ ವಾದಿಗಳು ಕದನಕ್ಕಾಗಿ ತನಗೆ ಬಹಳ ಸಮೀಪವರ್ತಿಗಳಾದಾಗ ಮಾತ್ರ ಅದ್ವೈತಿಯು ಖಂಡನೆಯ ಖಡ್ಗವನ್ನು ಬೀಸುತ್ತಾನೆ–ಶತ್ರುನಾಶಕ್ಕಾಗಿ. ಆದರೆ ಅವನದೇ ಒಂದು ಸಿದ್ಧಾಂತವಿದೆ; ಒಂದು ಸ್ಥಾನವಿದೆ. ಇತರ ಮತಗಳನ್ನು ಖಂಡಿಸಿ ತನ್ನ ಮತದ ಪರವಾಗಿ ಪ್ರಮಾಣ ಗ್ರಂಥಗಳನ್ನು ಪ್ರದರ್ಶಿಸುವುದರಿಂದ ಮಾತ್ರವೇ ತೃಪ್ತನಾಗದವನು ಆತನೊಬ್ಬನೇ. ವಿಶ್ವವು ನಿರಂತರ ಚಲನಶೀಲವಾದುದೆಂದು ನೀವು ಹೇಳುತ್ತೀರಿ. ವ್ಯಷ್ಟಿಯಲ್ಲಿ ಎಲ್ಲವೂ ಚಲಿಸುತ್ತಿದೆ ನೀವು ಚಲಿಸುತ್ತಿದ್ದೀರಿ; ಮೇಜು ಚಲಿಸುತ್ತಿದೆ; ಎಲ್ಲೆಲ್ಲಿಯೂ ಚಲನೆ ಇದ್ದೇ ಇದೆ. ಸಂಸಾರ ಅಥವಾ ಜಗತ್​ ಎಂದರೆ ಅದೇ. ಆದುದರಿಂದ ಈ ಜಗತ್ತಿನಲ್ಲಿ ಪ್ರತ್ಯೇಕ ಅಸ್ತಿತ್ವ \enginline{(individuality)} ಎಂಬುದಕ್ಕೆ ಅವಕಾಶವಿಲ್ಲ. ಏಕೆಂದರೆ ಪ್ರತ್ಯೇಕ ಅಸ್ತಿತ್ವ ಎಂದರೆ ವ್ಯತ್ಯಾಸವಾಗದೆ ಇರುವಂತಹುದು ಎಂದು ಅರ್ಥವಾಗುತ್ತದೆ. ವ್ಯತ್ಯಾಸ ಹೊಂದುವ ಪ್ರತ್ಯೇಕ ಅಸ್ತಿತ್ವ ಇರಲಾರದು. ಏಕೆಂದರೆ ವ್ಯತ್ಯಾಸ ಹೊಂದುವಿಕೆಯೂ ಪ್ರತ್ಯೇಕ ಅಸ್ತಿತ್ವವೂ ಪರಸ್ಪರ ವಿರುದ್ಧ ಭಾವಗಳಾಗಿವೆ. ನಮ್ಮ ಈ ಕಿರು ಜಗತ್ತಿನಲ್ಲಿ ಪ್ರತ್ಯೇಕ ಅಸ್ತಿತ್ವ ಎಂಬುದು ಇಲ್ಲವೇ ಇಲ್ಲ. ಬುದ್ಧಿ, ಭಾವ, ಮನಸ್ಸು, ದೇಹ, ಮೃಗ, ಪ್ರಾಣಿ ಎಲ್ಲವೂ ನಿರಂತರವಾಗಿ ಬದಲಾಗುತ್ತಿವೆ. ಈ ವಿಶ್ವವನ್ನೆಲ್ಲಾ ಒಟ್ಟಾಗಿ ತೆಗೆದು ಕೊಂಡಿರಿ ಎಂದು ಭಾವಿಸೋಣ. ಎಂದರೆ ಸಮಷ್ಟಿ ದೃಷ್ಟಿಯಿಂದ ನೋಡಿದರೆ ಈ ಜಗತ್ತು ಬದಲಾಗಬಲ್ಲುದೆ? ಚಲಿಸಬಲ್ಲುದೆ? ಎಂದಿಗೂ ಇಲ್ಲ. ಒಂದು ಪದಾರ್ಥದ ಚಲನೆಯು ನಮಗೆ ಗೊತ್ತಾಗಬೇಕಾದರೆ ಇನ್ನೊಂದು ಪದಾರ್ಥದ ಕಡಮೆ ಚಲನೆಯ ಅಥವಾ ನಿಶ್ಚಲನೆಯ ದೃಷ್ಟಿಯಿಂದಲೇ ಹೊರತು ಬೇರೆ ಸಾಧ್ಯವಿಲ್ಲ. ಆದುದರಿಂದ ಒಟ್ಟಿನಲ್ಲಿ ಈ ಜಗತ್ತು ನಿಶ್ಚಲವಾಗಿದೆ; ನಿರ್ವಿಕಾರಿಯಾಗಿದೆ. ಯಾವಾಗ ನೀವು ಸಮಸ್ತ ವಿಶ್ವದೊಡನೆ ಐಕ್ಯತೆಯನ್ನು ಸಾಧಿಸುತ್ತೀರೊ, ಯಾವಾಗ ನೀವು ನಾನೇ ಜಗತ್ತು ಎಂಬುದನ್ನು ಅರಿಯುತ್ತೀರೋ ಆಗ ಮಾತ್ರ ನಿಮಗೆ ವ್ಯಕ್ತಿತ್ವ ದೊರೆಯುತ್ತದೆ. ಆದುದರಿಂದಲೇ ವೇದಾಂತಿಯು ದ್ವೈತಭಾವವಿರುವ ತನಕ ಭಯ ತಪ್ಪದು ಎಂದು ಹೇಳುತ್ತಾನೆ. ಯಾವಾಗ ತನಗಿಂತಲೂ ಬೇರೆಯಾದುದು ಕಾಣುವುದಿಲ್ಲವೋ, ಯಾವಾಗ ಬೇರೆ ಎಂಬ ಭಾವವಿರುವುದಿಲ್ಲವೋ, ಯಾವಾಗ ಎಲ್ಲ ಒಂದಾಗುವುದೋ ಆಗ ಮಾತ್ರ ಭಯವು ತಪ್ಪುತ್ತದೆ, ಮರಣ ಕೊನೆಗಾಣುತ್ತದೆ, ಮತ್ತು ಆಗ ಮಾತ್ರವೇ ಸಂಸಾರ ಮಾಯವಾಗುತ್ತದೆ. ಆದುದರಿಂದ ಅದ್ವೈತವು, ನಮಗೆ ಈ ರೀತಿ ಬೋಧಿಸುತ್ತದೆ: ವಿಶ್ವದಲ್ಲಿ ಲೀನವಾಗುವುದರಿಂದಲೇ ಮನುಷ್ಯನಿಗೆ ನಿಜವಾದ ವ್ಯಕ್ತಿತ್ವ ದೊರೆಯುತ್ತದೆ, ಪ್ರತ್ಯೇಕತೆಯಿಂದ ಅದು ದೊರೆಯಲಾರದು. ನೀವು ಸಮಷ್ಟಿ ಯಾದಾಗಲೇ ಅಮೃತರಾಗುತ್ತೀರಿ. ನೀವೂ ವಿಶ್ವವೂ ಒಂದಾದಾಗಲೇ ನಿಮಗೆ ಭಯವಿಲ್ಲ, ಮರಣವಿಲ್ಲ. ಆಗ ಮಾತ್ರವೇ ನಿಮಗೆ ಗೋಚರವಾಗುತ್ತದೆ, ವಿಶ್ವ ದೇವರು ನಾನು ಎಂಬುದೆಲ್ಲವೂ ಏಕವಾಗಿವೆ, ಅದ್ವಿತೀಯವಾಗಿವೆ, ಎಂದು. ಅಸ್ತಿತ್ವ ಎಂಬುದು ಅದೇ, ಪೂರ್ಣ ಎಂಬುದು ಅದೇ. ಏಕವೂ ಅಖಂಡವೂ ಆಗಿರುವ ಸತ್​ ನಮಗೂ, ನಮ್ಮಂತೆಯೇ ಮನಸ್ಸುಳ್ಳವರಿಗೂ ಅನೇಕ ಖಂಡವಾದ ಹಾಗೆ ಆದಂತಾಗಿ, ಬಹುತ್ವವುಳ್ಳ ಜಗತ್ತಾಗಿ ಕಾಣಿಸಿಕೊಳ್ಳುತ್ತಿದೆ. ಸ್ವಲ್ಪ ಪುಣ್ಯ ಕರ್ಮಗಳನ್ನು ಮಾಡಿದವರಿಗೆ ಸ್ವಲ್ಪ ಉತ್ತಮವಾದ ಮನಸ್ಸು ಲಭಿಸುವುದರಿಂದ ಸತ್ತಮೇಲೆ ಅವರಿಗೆ ಇಂದ್ರಾದಿಗಳಿಂದ ಕೂಡಿದ ಸ್ವರ್ಗವು ಗೋಚರಿಸಿದಂತಾಗುತ್ತದೆ. ಅದಕ್ಕಿಂತಲೂ ಉತ್ತಮ ಮನಸ್ಕರು ಅದೇ ಸತ್ತನ್ನೇ ಬ್ರಹ್ಮಲೋಕವಾಗಿ ಕಾಣುತ್ತಾರೆ. ಮುಕ್ತರಾದವರು ಅದರಲ್ಲಿ ಮರ್ತ್ಯ, ಸ್ವರ್ಗ ಮೊದಲಾದ ಯಾವ ಲೋಕವನ್ನೂ ಕಾಣುವುದಿಲ್ಲ. ಅವರ ದೃಷ್ಟಿಗೆ ವಿಶ್ವವು ಮಾಯವಾಗಿ ಬ್ರಹ್ಮ ಮಾತ್ರ ನಿಲ್ಲುತ್ತದೆ.

ನಾವು ಈ ಬ್ರಹ್ಮವನ್ನು ತಿಳಿಯಲು ಸಾಧ್ಯವೇ? ಅನಂತದ ವಿಚಾರವಾಗಿ ಸಂಹಿತೆಗಳು ಚಿತ್ರಿಸಿರುವ ಬಾಹ್ಯದೃಷ್ಟಿಯ ಮನೋಹರ ಚಿತ್ರವು ಎಂಥದು ಎಂಬುದನ್ನು ನಿಮಗಾಗಲೇ ತಿಳಿಸಿದ್ದೇನೆ. ಇಲ್ಲಿ ಅದನ್ನು ಕುರಿತು ಮತ್ತೊಂದು ವಿಧವಾದ ಅಂತರ್​ದೃಷ್ಟಿಯ ವ್ಯಾಖ್ಯಾನವಿದೆ. ಮೊದಲನೆಯದರಲ್ಲಿ ಇದ್ದುದು ಬಾಹ್ಯದ ಅಥವಾ ಮೃಣ್ಮಯದ ಅನಂತತೆ; ಇಲ್ಲಿ ಆಂತರ್ಯದ ಅಥವಾ ಚಿನ್ಮಯದ ಅನಂತತೆ; ಅಲ್ಲಿ ಅನಂತವನ್ನು ‘ಇತಿ ಇತಿ’ ಎಂದು ವರ್ಣಿಸಲು ಪ್ರಯತ್ನಿಸಿ ಸೋತು ಹೋಗಿದ್ದಾರೆ. ಆದ್ದರಿಂದ ಇಲ್ಲಿ ಅದನ್ನು ‘ನೇತಿ ನೇತಿ’ ಭಾಷೆಯಲ್ಲಿ ವರ್ಣಿಸಲು ಪ್ರಯತ್ನಿಸಿದ್ದಾರೆ. ಇಲ್ಲಿ ಜಗತ್ತಿದೆ; ಈ ಜಗತ್ತೇ ಬ್ರಹ್ಮ ಎಂದು ಇಟ್ಟುಕೊಂಡರೂ ಆ ಬ್ರಹ್ಮವನ್ನು ನಾವರಿಯಲು ಸಾಧ್ಯವೇ?ನಾವು ಎಂದಿಗೂ ಅದನ್ನು ಅರಿಯಲು ಸಾಧ್ಯವಿಲ್ಲ; ಇದನ್ನು ನೆನಪಿನಲ್ಲಿಡಿ. ಪುನಃ ಪುನಃ ನಿಮ್ಮ ಮನದಲ್ಲಿ ಈ ಸಂದೇಹವು ಮೂಡುತ್ತದೆ. ಇದು ಬ್ರಹ್ಮವಾದರೆ ಇದನ್ನು ನಾವು ತಿಳಿಯುವುದು ಹೇಗೆ? \textbf{ವಿಜ್ಞಾತಾರಮರೇ ಕೇನ ವಿಜಾನೀಯಾತ್​} “ತಿಳಿಯುವವನನ್ನು ತಿಳಿಯುವುದೆಂತು?” ತಿಳಿಯುವವನೇ ಹೇಗೆ ತಿಳಿವಿನ ವಸ್ತುವಾಗಬಲ್ಲನು? ಕಣ್ಣು ಎಲ್ಲವನ್ನೂ ನೋಡುತ್ತದೆ. ತನ್ನನ್ನು ತಾನು ನೋಡಿಕೊಳ್ಳಬಲ್ಲದೆ? ಇಲ್ಲ, ಅದು ಅಸಾಧ್ಯ. ಅರಿಯುವುದೇ ಒಂದು ಅವನತಿಯ ಸ್ಥಿತಿ, ಒಂದು ಪರಿಮಿತಿಯನ್ನು ಕಲ್ಪಿಸಿದಂತೆ. ಆರ್ಯ ಸಂತಾನರೇ, ನೀವು ಇದನ್ನು ಮನಸ್ಸಿನಲ್ಲಿಡಿ. ಇದರಲ್ಲೊಂದು ಮಹಾ ಸತ್ಯವಿದೆ. ನಿಮ್ಮಲ್ಲಿಗೆ ಬಂದಿರುವ ಪಾಶ್ಚಾತ್ಯ ಪ್ರಲೋಭನೆಗಳಿಗೆಲ್ಲ ತಾತ್ತ್ವಿಕವಾದ ಆಧಾರವೆಂದರೆ ‘ಇಂದ್ರಿಯ ಜ್ಞಾನಕ್ಕಿಂತಲೂ ಉನ್ನತತರವಾದ ಜ್ಞಾನವಿಲ್ಲ’ ಎಂಬುದು. ವೇದಗಳನ್ನನುಸರಿಸುವ ಪ್ರಾಚ್ಯರಾದ ನಾವು ‘ಜ್ಞಾನವು ಜ್ಞೇಯಕ್ಕಿಂತ ಕೆಳಮಟ್ಟದ್ದು, ಏಕೆಂದರೆ ಜ್ಞಾನವು ಸ್ವಭಾವತಃ ಒಂದು ಪರಿಮಿತಿ’ ಎಂದು ಹೇಳುತ್ತೇವೆ. ನಾವು ಒಂದು ವಸ್ತುವನ್ನು ತಿಳಿಯಲು ಹೋದರೆ ಅದು ನಮ್ಮ ಮನಸ್ಸಿನಿಂದ ಪರಿಮಿತವಾಗುತ್ತದೆ. ಮುತ್ತಿನ ಚಿಪ್ಪಿನಲ್ಲಿ ಮುತ್ತು ತಯಾರಾಗುವ ದೃಷ್ಟಾಂತವನ್ನು ನೆನಪಿಗೆ ತಂದುಕೊಳ್ಳಿ. ಅರಿಯುವುದೆಂದರೆ ಒಂದು ವಸ್ತುವನ್ನು ನಮ್ಮ ಪ್ರಜ್ಞೆಯ ಪರಿಧಿಯ ಒಳಕ್ಕೆ ತರುವುದು, ಅದನ್ನು ಸಮಗ್ರವಾಗಿ ನೋಡದೆ ಇರುವುದು. ಎಲ್ಲ ಜ್ಞಾನದ ವಿಷಯದಲ್ಲೂ ಇದು ನಿಜ. ಅನಂತದ ವಿಷಯದಲ್ಲಿ ಇದು ಕಡಮೆ ಸತ್ಯವೇನು? ಜ್ಞಾನವು ಹೇಗೆ ಮಿತಿಯನ್ನು ಕಲ್ಪಿಸಬಲ್ಲುದು ಎಂಬುದನ್ನು ನೋಡಿ. ಯಾರು ಎಲ್ಲ ಜ್ಞಾನದ ಅಂತಃಸತ್ವ ವಸ್ತುವೋ, ಯಾರು ಸಾಕ್ಷಿಯೋ, ಯಾರಿಲ್ಲದಿದ್ದರೆ ಜ್ಞಾನವೇ ಇಲ್ಲವೋ, ಯಾರು ನಿರ್ಗುಣನೋ ಯಾರು ವಿಶ್ವದ ಏಕಮಾತ್ರ ಸಾಕ್ಷಿಯೊ, ಯಾರು ನಮ್ಮಾತ್ಮಗಳಲ್ಲಿ ಅಂತಃ ಸಾಕ್ಷಿಯಾಗಿದ್ದಾನೆಯೋ ಅಂತಹವನಿಗೆ ಮೇರೆಯನ್ನು ಕಲ್ಪಿಸುವುದು ಹೇಗೆ? ಅವನನ್ನು ಹೇಗೆ ತಾನೆ ತಿಳಿಯುತ್ತೀರಿ? ಯಾವುದರಿಂದ ಆತನನ್ನು ಸಾಂತ ಗೊಳಿಸುತ್ತೀರಿ? ಪ್ರತಿಯೊಂದೂ, ಈ ಇಡೀ ವಿಶ್ವವೇ ಈ ಮಿಥ್ಯಾ ಪ್ರಯತ್ನದ ಪರಿಣಾಮ. ಈ ಅನಂತಾತ್ಮವು ತನ್ನ ಮುಖವನ್ನೇ ತಾನು ನೋಡಲು ಪ್ರಯತ್ನಿಸಿದಂತಿದೆ. ಅತ್ಯಂತ ಚಿಕ್ಕ ಪ್ರಾಣಿಗಳಿಂದ ಹಿಡಿದು ಅತ್ಯುನ್ನತ ದೇವತೆಗಳವರೆಗೆ ಇರುವ ಪ್ರತಿಯೊಂದೂ ಪರಬ್ರಹ್ಮವನ್ನು ಪ್ರತಿಬಿಂಬಿಸುವ ಕನ್ನಡಿಗಳಂತೆ. ಪರಬ್ರಹ್ಮವು ಇನ್ನೂ ಬೇರೆ ದೇಹಗಳನ್ನು ಧರಿಸುತ್ತಾ ಅವೆಲ್ಲವೂ ಅಸಮರ್ಪಕ ಎಂದು ತಿಳಿದು, ಕೊನೆಗೆ ಮಾನವ ದೇಹದಲ್ಲಿ ಎಲ್ಲವೂ ಸಾಂತವೆಂದು ತಿಳಿಯುತ್ತದೆ. ಸಾಂತದಲ್ಲಿ ಅನಂತದ ಅಭಿವ್ಯಕ್ತಿ ಅಸಾಧ್ಯ. ಅದಾದ ಮೇಲೆ ಪ್ರತಿಗಮನ (ಹಿಂದಕ್ಕೆ ಹೋಗುವುದು) ಪ್ರಾರಂಭವಾಗುತ್ತದೆ. ಅದೇ ‘ವೈರಾಗ್ಯ’. ಇಂದ್ರಿಯಗಳಿಂದ ಹಿಂದಕ್ಕೆ ಸರಿ, ಇಂದ್ರಿಯಗಳ ಕಡೆ ಹೋಗಬೇಡ ಎಂಬುದೇ ವೈರಾಗ್ಯದ ತೂರ್ಯವಾಣಿ. ವೈರಾಗ್ಯದಿಂದಲೇ ನೀತಿಧರ್ಮಗಳ ಆಧಾರವಾದ ಸಂಯಮವು ಲಭಿಸುತ್ತದೆ. ಚೆನ್ನಾಗಿ ನೆನಪಿನಲ್ಲಿಡಿ. ನಮ್ಮ ದೃಷ್ಟಿಯಲ್ಲಿ ವಿಶ್ವವು ಪ್ರಾರಂಭವಾಗುವುದೇ ತಪಸ್ಸಿನ ಮೂಲಕ, ವೈರಾಗ್ಯದ ಮೂಲಕ. ನೀವು ಹಿಂದೆ ಹಿಂದೆ ಸರಿದಂತೆಲ್ಲಾ ಎಲ್ಲ ರೂಪಗಳೂ ನಿಮ್ಮ ಮುಂದೆ ಅಭಿವ್ಯಕ್ತವಾಗುತ್ತವೆ. ಮತ್ತು ಅವುಗಳನ್ನು ಒಂದೊಂದನ್ನಾಗಿ ತ್ಯಜಿಸಿದ ಮೇಲೆಯೇ ನಿಮಗೆ ಸ್ವಸ್ವರೂಪ ದರ್ಶನವಾಗುತ್ತದೆ. ಅದನ್ನೇ ಮೋಕ್ಷ ಅಥವಾ ಮುಕ್ತಿ ಎಂದು ಕರೆಯುತ್ತಾರೆ.

ನಾವಿದನ್ನು ಚೆನ್ನಾಗಿ ಅರಿಯಬೇಕು– \textbf{“ವಿಜ್ಞಾತಾರಮರೇ ಕೇನ ವಿಜಾನೀ ಯಾತ್​”} — ತಿಳಿಯುವವನನ್ನೆ ತಿಳಿಯುವುದೆಂತು? ತಿಳಿಯುವವನು ತಿಳಿಯಲ್ಪಡ ಲಾರನು. ಏಕೆಂದರೆ ತಿಳಿಯಲ್ಪಟ್ಟರೆ ಅದು ತಿಳಿಯುವವನಾಗುವುದಿಲ್ಲ. ಕನ್ನಡಿಯ ಪ್ರತಿಬಿಂಬದಲ್ಲಿ ನಿಮ್ಮ ಕಣ್ಣನ್ನು ನೀವು ನೋಡಿಕೊಂಡರೆ ಆ ಕಣ್ಣು ಎಂದೆಂದಿಗೂ ನಿಮ್ಮದಲ್ಲ; ಅದೊಂದು ಅನ್ಯಪದಾರ್ಥ, ಪ್ರತಿಬಿಂಬ ಮಾತ್ರ. ಆದರೆ ಈ ಆತ್ಮವು, ನೀವೇ ಆಗಿರುವ ಈ ಅನಂತಬ್ರಹ್ಮವು, ಬರಿಯ ಸಾಕ್ಷಿ ಮಾತ್ರವಾಗಿದ್ದರೆ ಅದನ್ನು ಹಾಗೆಂದು ತಿಳಿಯುವುದರಿಂದ ಸಾರ್ಥಕತೆಯೇನು? ಅದು ನಮ್ಮಂತೆ ಬದುಕಲಾರದು, ಚಲಿಸಲಾರದು, ಸುಖಪಡಲಾರದು. ಸಾಕ್ಷಿಯಾದುದಕ್ಕೆ ಸುಖ ಲಭಿಸುತ್ತದೆ ಎಂದರೆ ಜನರಿಗೆ ಅರ್ಥವಾಗುವುದಿಲ್ಲ. ಅದೂ ಅಲ್ಲದೆ ಅನೇಕರು ಹೇಳುತ್ತಾರೆ: “ಅಯ್ಯೋ ಹಿಂದುಗಳಿರಾ, ನೀವೆಲ್ಲ ಬರಿಯ ಸಾಕ್ಷಿಗಳೆಂಬ ಸಿದ್ಧಾಂತವನ್ನು ನಂಬಿ ಜಡರೂ ಕೆಲಸಕ್ಕೆ ಬಾರದವರೂ ಆಗಿಹೋಗಿದ್ದೀರಿ.” ಅದಕ್ಕೆ ಉತ್ತರವಾಗಿ ನಾವು ಹೀಗೆ ಹೇಳಬಹುದು: “ಮೊದಲನೆಯದಾಗಿ, ಸಾಕ್ಷಿಯೊಂದೇ ಭೋಕ್ತೃ, ಎಂದರೆ ಸುಖಪಡತಕ್ಕದ್ದು.” ಎಲ್ಲಿಯಾದರೂ ಒಂದು ಕುಸ್ತಿ ನಡೆಯುತ್ತಿದ್ದರೆ ಹೆಚ್ಚು ಸುಖಾನುಭವ ಮಾಡುವವರು ಯಾರು? ಕುಸ್ತಿ ಆಡುವವರೋ, ಅದನ್ನು ನೋಡುವವರೋ? ಜೀವನದಲ್ಲಿ ನೀವು ಯಾವುದನ್ನು ಸಾಕ್ಷಿಮಾತ್ರರಾಗಿ ಅನುಭವಿಸುತ್ತೀರೋ ಅದನ್ನು ಮಾತ್ರವೇ ಹೆಚ್ಚಾಗಿ ಭೋಗಿಸುತ್ತೀರಿ. ಅಂತಹ ಭೋಗವೇ ಆನಂದ. ಆದುದರಿಂದ ನಾವು ಜಗತ್ತಿನ ಸಾಕ್ಷಿಭೂತರಾದಾಗ ಮಾತ್ರವೇ ಅನಂತ ಆನಂದವು ಪ್ರಾಪ್ತವಾಗುತ್ತದೆ ಆಗಲೇ ನೀವು ಮುಕ್ತರಾಗುವುದು. ಕಾಮವಿಲ್ಲದೆ, ಸ್ವರ್ಗಾಭಿಲಾಷೆಯಿಲ್ಲದೆ, ಕೀರ್ತಿ ಅಪಕೀರ್ತಿಗಳ ಭಾವವಿಲ್ಲದೆ ಕರ್ಮಮಾಡುವುದು ಸಾಕ್ಷಿಯಾದವನಿಗೆ ಮಾತ್ರ ಸಾಧ್ಯ. ಸಾಕ್ಷಿಯಾದವನು ಮಾತ್ರವೇ ಭೋಕ್ತೃ; ಬೇರೆ ಯಾರೂ ಅಲ್ಲ.

ನೈತಿಕ ವಿಚಾರಕ್ಕೆ ಬರೋಣ. ಅದ್ವೈತದ ತಾತ್ತ್ವಿಕ ಮತ್ತು ನೈತಿಕ ಅಂಶಗಳ ನಡುವೆ ಒಂದು ತತ್ತ್ವವಿದೆ. ಅದು ಯಾವುದೆಂದರೆ ಮಾಯಾ ಸಿದ್ಧಾಂತ. ಅದ್ವೈತ ದರ್ಶನದ ಪ್ರತಿಯೊಂದು ವಿಷಯವನ್ನು ತಿಳಿಯಲು ವರ್ಷಗಳೇ ಬೇಕಾಗುತ್ತವೆ; ಹೇಳಲು ತಿಂಗಳುಗಳು ಬೇಕಾಗುತ್ತವೆ. ಆದುದರಿಂದ ನಾನಿಂದು ಆ ವಿಷಯಗಳನ್ನು ಮೇಲೆ ಮೇಲೆ ಮಾತ್ರ ಮುಟ್ಟಿ ತೇಲಿ ಸಾಗಿದರೆ ನೀವು ನನ್ನನ್ನು ಕ್ಷಮಿಸುತ್ತೀರೆಂದು ಭಾವಿಸುತ್ತೇನೆ. ಎಲ್ಲ ಕಾಲದಲ್ಲಿಯೂ ಈ ಮಾಯಾಸಿದ್ಧಾಂತವು ಜನರ ತಿಳುವಳಿಕೆಗೆ ಅತಿ ಕಷ್ಟಸಾಧ್ಯವೆಂದು ತೋರಿ ಬಂದಿದೆ. ನಾನು ನಿಮಗೆ ಹೇಳುತ್ತೇನೆ – ಅದು ಸಿದ್ಧಾಂತವೇ ಅಲ್ಲ. ಕಾಲ ದೇಶ ನಿಮಿತ್ತಗಳೆಂಬ ಭಾವಗಳ ಸಂಯೋಗ ಮಾತ್ರವಾಗಿದೆ. ಕಾಲ ದೇಶ ನಿಮಿತ್ತಗಳು ನಾಮರೂಪಗಳಾಗಿ ಪರಿವರ್ತಿತವಾಗಿದೆ. ಸಮುದ್ರದಲ್ಲಿ ಒಂದು ತರಂಗವಿದೆಯೆಂದು ಭಾವಿಸಿ. ಅದು ನಾಮರೂಪಗಳಿಂದ ಮಾತ್ರವೇ ಸಮುದ್ರದಿಂದ ಬೇರೆಯಾಗಿದೆ. ಈ ನಾಮರೂಪಗಳು ಮತ್ತು ತರಂಗಗಳು ಬೇರೆ ಬೇರೆಯಾಗಿ ಇರಲಾರವು. ಅವು ತರಂಗದೊಂದಿಗೇ ಇರುತ್ತವೆ. ತರಂಗ ಕಣ್ಮರೆಯಾಗಬಹುದು. ಆದರೆ ನಾಮರೂಪಗಳು ಸಂಪೂರ್ಣ ನಾಶವಾದರೂ ಆ ತರಂಗ ರೂಪವಾದ ಜಲದ ಪ್ರಮಾಣ ಇದ್ದೇ ಇರುತ್ತದೆ. ಆದ್ದರಿಂದ ನಿಮಗೂ, ನನಗೂ, ಪ್ರಾಣಿಗಳಿಗೂ, ಮನುಷ್ಯರಿಗೂ, ನರನಿಗೂ, ದೇವತೆಗಳಿಗೂ, ಇತರ ಎಲ್ಲ ವಸ್ತುಗಳಿಗೂ ಪರಸ್ವರ ಭೇದವಿರುವಂತೆ ಕಾಣುವುದು ಈ ‘ಮಾಯೆ’ ಯಿಂದಲೇ. ಅಷ್ಟೇ ಅಲ್ಲ. ಈ ಮಾಯೆಯ ದೆಸೆಯಿಂದಲೇ ನಿತ್ಯ ಮುಕ್ತವಾಗಿರುವ ಆತ್ಮವು ನಾಮ ರೂಪಗಳಿಂದ ವಿವಿಧವೂ ಅಸಂಖ್ಯವೂ ಆಗಿ ಜೀವಗಳಲ್ಲಿ ಬಂಧಿತವಾಗಿರುವಂತೆ ತೋರುತ್ತದೆ. ನಾಮರೂಪಗಳಿಂದ ಬಿಡಿಸಿ, ಅದನ್ನು ಕೇವಲವಾಗಿ ಮಾಡಿದರೆ ವಿಶ್ವವು ಎಂದೆಂದಿಗೂ ಕಣ್ಮರೆಯಾಗಿ ನೀವು ಸ್ವರೂಪಸ್ಥಿತರಾಗುತ್ತೀರಿ. ಇದೇ ‘ಮಾಯೆ’.

ಮತ್ತೆ ಹೇಳುತ್ತೇನೆ: ಅದು ಸಿದ್ಧಾಂತವಲ್ಲ, ವಾಸ್ತವ ಘಟನೆಗಳ ನಿರೂಪಣೆ. ವಸ್ತುಸತ್ತಾವಾದಿಯು \enginline{(realist)} ಈ ಮೇಜು ‘ಇದೆ’ ಎಂದು ಹೇಳಿದಾಗ ಅವನ ಅಭಿಪ್ರಾಯ ಇದು: ಈ ಮೇಜಿಗೆ ತನ್ನದೇ ಆದ ಒಂದು ಸ್ವತಂತ್ರ ಅಸ್ತಿತ್ವವಿದೆ. ವಿಶ್ವದಲ್ಲಿರುವ ಮತ್ತಾವ ವಸ್ತುವಿನ ಅಸ್ತಿತ್ವದ ಆಧಾರದ ಮೇಲೂ ಇದು ನಿಂತಿಲ್ಲ. ಈ ಇಡೀ ಜಗತ್ತು ಸಂಪೂರ್ಣವಾಗಿ ನಾಶವಾದರೂ ಈ ಮೇಜು ಈಗ ಹೇಗಿದೆಯೋ ಹಾಗೆಯೇ ಇರುತ್ತದೆ. ನಾವು ಸ್ವಲ್ಪ ವಿಚಾರಮಾಡಿದರೆ ಇದು ಹಾಗಲ್ಲವೆಂದು ಅರಿವಾಗುವುದು. ಈ ಇಂದ್ರಿಯ ಜಗತ್ತಿನಲ್ಲಿ ಪ್ರತಿಯೊಂದು ವಸ್ತುವೂ, ಇನ್ನೊಂದನ್ನು ಅವಲಂಬಿಸಿದೆ, ಎಲ್ಲವೂ ಪರಸ್ಪರ ಸಂಬಂಧಿಸಿವೆ, ಒಂದು ವಸ್ತುವಿನ ಅಸ್ತಿತ್ವ ಮತ್ತೊಂದನ್ನು ಅವಲಂಬಿಸಿದೆ. ಆದ್ದರಿಂದ ನಮಗಾಗುವ ವಸ್ತುಜ್ಞಾನಕ್ಕೆ ಮೂರು ಮೆಟ್ಟಲುಗಳಿವೆ: ಪ್ರತಿಯೊಂದು ವಸ್ತುವೂ ಬೇರೆ ಬೇರೆಯಾಗಿದೆ ಎಂದು ತಿಳಿಯುವ ಪೃಥಗ್ಭಾವ ಒಂದನೆಯ ಮೆಟ್ಟಲು; ಒಂದಕ್ಕೊಂದು ಸಂಬಂಧಪಟ್ಟಿದೆ ಎಂದು ತಿಳಿಯುವ ಸಂಬಂಧಭಾವವು ಎರಡನೆಯದು; ಏಕವಾದುದನ್ನೇ ಅನೇಕವಾಗಿ ಕಾಣುತ್ತೇವೆ ಎಂಬುದು ಕಡೆಯದು. ಅಜ್ಞಾನಿಗಳ ದೇವರೆಂದರೆ, ಒಬ್ಬ ಸಮರ್ಥನಾದ ದೊಡ್ಡ ಮನುಷ್ಯ; ನಮಗಿಂತಲೂ ಸಾವಿರಪಾಲು ಸಮರ್ಥನು; ಸಾವಿರಪಾಲು ದೊಡ್ಡವನು. ಅವನು ಜಗತ್ತಿನ ಹೊರಗಡೆ ಮುಗಿಲಿನಾಚೆ ಎಲ್ಲಿಯೋ ಆಕಾಶದಲ್ಲಿ ವಾಸಿಸುತ್ತಾನೆ. ಆ ದೇವರು ಹೇಗೆ ಅಸಮಂಜಸ ಮತ್ತು ಅಸಂಪೂರ್ಣನು ಎಂಬುದನ್ನು ನಾವಾಗಲೇ ವಿಚಾರಿಸಿದ್ದೇವೆ. ಅದಾದ ತರುವಾಯ ಮನುಷ್ಯರ ಮನಸ್ಸಿಗೆ ಬರುವ ಈಶ್ವರ ಭಾವವೆಂದರೆ, ಸರ್ವವ್ಯಾಪಿಯೂ, ಸರ್ವತ್ರ ಪ್ರಕಾಶಿತವೂ ಆಗಿರುವ ಒಂದು ಶಕ್ತಿ. ಇದನ್ನೇ ಸಗುಣ ಈಶ್ವರನೆಂದು ಕರೆಯುತ್ತಾರೆ. ‘ಚಂಡಿ’\footnote{ಮಾರ್ಕಂಡೇಯ ಪುರಾಣದಲ್ಲಿ ಬರುವ ಸೃಷ್ಟಿ ಸ್ಥಿತಿ, ಲಯಕರ್ತೃವಾದ ಶಕ್ತಿಯನ್ನು ವರ್ಣಿಸುವ ಚಂಡೀಸ್ತೋತ್ರ} ಯಲ್ಲಿ ದೊರೆಯುವ ಭಾವವು ಇದೆ. ‘ಸಗುಣ–ಈಶ್ವರ’ ಎಂದರೆ ಸಗುಣ–ಈಶ್ವರನಲ್ಲವೆಂಬುದನ್ನು ಮರೆಯಬೇಡಿ. ಅವನು ಸಕಲ ಕಲ್ಯಾಣ ಗುಣಸಂಪನ್ನನು ಮಾತ್ರವೇ ಅಲ್ಲ. ದೇವ ಮತ್ತು ಸೈತಾನ ಎಂಬ ಎರಡು ದೇವರನ್ನು ಇಟ್ಟುಕೊಳ್ಳಲಾರಿರಿ. ಒಬ್ಬನನ್ನೇ ಇಟ್ಟುಕೊಳ್ಳಬೇಕು. ಒಳ್ಳೆಯದಕ್ಕೂ, ಕೆಟ್ಟದಕ್ಕೂ ಅವನೊಬ್ಬನೇ ಕಾರಣವೆಂದು ಹೇಳುವ ಸಾಹಸವನ್ನು ಮಾಡಲೇಬೇಕು. ಅದರಿಂದ ಉದ್ಭವಿಸುವ ತಾರ್ಕಿಕ ಪರಿಣಾಮವನ್ನು ಸ್ವೀಕರಿಸಲೇಬೇಕು. ಏನೇ ಆದರೂ ‘ಏಕ’ ವನ್ನೇ ಇಟ್ಟುಕೊಳ್ಳಬೇಕು. “ಚಂಡಿ” ಯಲ್ಲಿ ಹೀಗೆ ಹೇಳಿದೆ: \textbf{ಯಾ ದೇವೀ ಸರ್ವಭೂತೇಷು ಶಾಂತಿ ರೂಪೇಣ ಸಂಸ್ಥಿತಾ~। ನಮಸ್ತಸ್ಯೈ ನಮಸ್ತಸ್ಯೈ ನಮಸ್ತಸ್ಯೈ ನಮೋ ನಮಃ~॥ ಯಾ ದೇವೀ ಸರ್ವಭೂತೇಷು ಭ್ರಾಂತಿರೂಪೇಣ ಸಂಸ್ಥಿತಾ ನಮಸ್ತಸ್ಯೈ ನಮಸ್ತಸ್ಯೈ ನಮಸ್ತಸ್ಯೈ ನಮೋ ನಮಃ~॥} “ಯಾವ ದೇವಿಯು ಸರ್ವಭೂತಗಳಲ್ಲಿ ಶಾಂತಿ ರೂಪದಿಂದ ನೆಲೆಸಿರುವಳೊ, ಯಾವ ದೇವಿಯು ಸರ್ವಭೂತಗಳಲ್ಲಿ ಭ್ರಾಂತಿ ರೂಪದಿಂದ ನೆಲೆಸಿರುವಳೊ ಅವಳಿಗೆ ನಮಸ್ಕಾರ.” ಈ ಈಶ್ವರನು ಸರ್ವ ರೂಪಾತ್ಮಕನೆಂದು ಕರೆಯುವುದರಿಂದಾಗುವ ಎಲ್ಲ ಪರಿಣಾಮಗಳನ್ನೂ ನೀವು ಸ್ವೀಕರಿಸಬೇಕು. “ಓ ಗಾರ್ಗಿ, ಇದೆಲ್ಲವೂ ಆನಂದವೇ. ಎಲ್ಲೆಲ್ಲಿ ಆನಂದವಿದೆಯೊ ಅಲ್ಲೆಲ್ಲ ದಿವ್ಯತೆಯ ಅಂಶವಿದೆ. ಆ ಏಕಾತ್ಮನು ನನ್ನ ಮುಂದಿರುವ ಈ ದೀಪದಂತೆ. ಇದರ ಬೆಳಕಿನಲ್ಲಿ ಒಳ್ಳೆಯ ಕೆಲಸವನ್ನಾದರೂ ಮಾಡಬಹುದು; ಸುಳ್ಳು ಪತ್ರವನ್ನು ಬೇಕಾದರೂ ಸೃಜಿಸಬಹುದು. ಎರಡಕ್ಕೂ ಬೆಳಕು ಒಂದೇ. ಇದು ಈಶ್ವರ ವಿಚಾರವಾದ ಜ್ಞಾನದಲ್ಲಿ ಎರಡನೆಯ ಮೆಟ್ಟಲು. ಮೂರನೆಯ ಮೆಟ್ಟಲೆಂದರೆ; ದೇವರು ಪ್ರಕೃತಿಯ ಒಳಗಾಗಲೀ, ಹೊರಗಾಗಲೀ, ಇಲ್ಲ; ಆದರೆ ಈಶ್ವರ, ಪ್ರಕೃತಿ, ಆತ್ಮ ಈ ಎಲ್ಲ ಪದಗಳು ಒಂದನ್ನೇ ಸೂಚಿಸುತ್ತವೆ. ಒಂದಕ್ಕೆ ಬದಲಾಗಿ ಇನ್ನೊಂದು ಪದವನ್ನು ಉಪಯೋಗಿಸಬಹುದು. ಎರಡು ವಸ್ತುಗಳನ್ನು ನೀವೆಂದಿಗೂ ಕಾಣಲಾರಿರಿ, ಒಂದೇ ಕಾಲದಲ್ಲಿ ದೇಹ ಮತ್ತು ಆತ್ಮಗಳೆರಡನ್ನೂ ಕಾಣುವುದು ಅಸಾಧ್ಯವಾಗಿ ಇರುವಂತೆ. ಅವುಗಳನ್ನು ಬೇರೆ ಬೇರೆಯಾಗಿ ಕಾಣಲು ಸಾಧ್ಯವೆಂಬುದು ನಿಮಗೆ ತತ್ತ್ವಶಾಸ್ತ್ರದ ಪಾರಿಭಾಷಿಕ ಪದಗಳಿಂದ ಲಭಿಸಿರುವ ಭ್ರಾಂತಿ ಮಾತ್ರವಾಗಿದೆ. ಆ ಭ್ರಾಂತಿಯಿಂದ ನೀವೊಂದು ದೇಹವೆಂದೂ ನಿಮಗೊಂದು ಆತ್ಮವಿದೆಯೆಂದೂ, ನೀವು ದೇಹ ಮತ್ತು ಆತ್ಮಗಳ ಸಮಷ್ಟಿ ಎಂದೂ ಊಹಿಸುತ್ತೀರಿ. ಅದು ಹೇಗಾದೀತು? ನಿಮ್ಮ ಮನಸ್ಸಿನಲ್ಲಿಯೇ ಆಲೋಚಿಸಿ ನೋಡಿ. ಅದು ಅಸಾಧ್ಯವೆಂಬುದು ನಿಮಗೇ ಗೊತ್ತಾಗುತ್ತದೆ. ಯೋಗಿಯಾದವನು ತಾನು ಚೈತನ್ಯ ಎಂದು ತಿಳಿಯುವುದರಿಂದ ಆತನಿಗೆ ದೇಹಬುದ್ಧಿಯು ಇರುವುದೇ ಇಲ್ಲ. ಸಾಧಾರಣನಾದವನು ತಾನು ದೇಹವೆಂದು ತಿಳಿಯುವುದರಿಂದ ಅವನ ಭಾಗಕ್ಕೆ ಚೈತನ್ಯವು ನಾಸ್ತಿ ಎಂದೇ ಹೇಳಬೇಕು. ತತ್ತ್ವ ಶಾಸ್ತ್ರದಲ್ಲಿ ದೇಹ ಮತ್ತು ಆತ್ಮ ಎಂಬ ಪದಗಳನ್ನು ಉಪಯೋಗಿಸುವುದರಿಂದ ನೀವು ಅವೆರಡೂ ಸೇರಿ ಮನುಷ್ಯನಾಗಿದ್ದಾನೆ ಎಂದು ಭ್ರಮಿಸುತ್ತೀರಿ. ಒಂದೇ ಸಾರಿಗೆ ದೇಹವೂ ಆತ್ಮವೂ ಆಗಿರುವುದು ಅಸಂಭವ. ದೇಹಭಾವವಿದ್ದರೆ ಆತ್ಮವಿಲ್ಲ; ಆತ್ಮಭಾವವಿದ್ದರೆ ದೇಹವಿಲ್ಲ. ಆದುದರಿಂದ ಒಂದು ಸಲಕ್ಕೆ ಒಂದು ಮಾತ್ರ ಇರುವುದು ಸಾಧ್ಯ. ಪ್ರಕೃತಿಯು ಕಾಣುವಾಗ ಈಶ್ವರನು ಕಾಣುವುದಿಲ್ಲ. ಈಶ್ವರನು ಕಂಡಾಗ ಪ್ರಕೃತಿಯು ಮಾಯವಾಗುತ್ತವೆ. ಕಾರ್ಯವು ಕಾಣಿಸುವಾಗ ಕಾರಣವು ಎಂದಿಗೂ ಕಾಣಿಸುವುದಿಲ್ಲ, ಕಾರಣವು ಕಂಡುಬಂದರೆ ಕಾರ್ಯವು ಮಾಯವಾಗುತ್ತದೆ. ಆಗ “ಜಗತ್ತೆಲ್ಲಿ? ಅದನ್ನಾರು ನುಂಗಿದರು”? ಎಂದು ಕೂಗುವಂತಾಗುತ್ತದೆ.

“ನಿರಾಕಾರವೂ, ಅಪ್ರಮೇಯವೂ, ನಿರುಪಮವೂ, ನಿರ್ಗುಣವೂ ಆಗಿರುವ ಬ್ರಹ್ಮವು, ಓ ಬುದ್ಧಿವಾನ್​, ಸಮಾಧಿಯಲ್ಲಿ ನಿನ್ನ ಹೃದಯ ಮಧ್ಯೆ ರಂಜಿಸುತ್ತದೆ.” “ಯಾವುದರಲ್ಲಿ ಪ್ರಕೃತಿ ವಿಕೃತಿಗಳೆಲ್ಲ ಎಂದೆಂದಿಗೂ ಕೊನೆಗಾಣುತ್ತವೆಯೋ, ಯಾವುದು ಚಿಂತಾತೀತ ಚಿಂತಾಮಣಿಯಾಗಿದೆಯೋ, ಯಾವುದನ್ನು ವೇದಗಳು ನಮ್ಮ ಅಸ್ತಿತ್ವದ ಅಂತಃಸಾರವೆಂದು ಕರೆಯುತ್ತಿವೆಯೋ ಅದೇ ಸಮಾಧಿಯಲ್ಲಿ ಅನುಭವ ಗೋಚರವಾಗುವ ಬ್ರಹ್ಮ.” “ಜನನ ಮರಣಾತೀತವಾಗಿ ಅನಂತವಾಗಿ ಅನುಪಮವಾಗಿ ಪ್ರಳಯ ಮಹಾ ಜಲರಾಶಿಯಂತೆ – ಮೇಲೆ ಕೆಳಗೆ ಸುತ್ತಮುತ್ತಲು ಹಬ್ಬಿ ತಬ್ಬಿರುವ ನಿಸ್ತರಂಗ ಸ್ತಬ್ಧವಾರಿ ವಿಸ್ತಾರದಂತೆ ಮೌನವಾಗಿ ಶಾಂತವಾಗಿ ಸರ್ವಮತ ದರ್ಶನಗಳನ್ನು ಕ್ಲೇಶ ಕಷ್ಟಗಳನ್ನು ಚಿರವಾಗಿ ಕೊನೆಗಾಣಿಸಿ, ಪರಬ್ರಹ್ಮವು ನಿಮ್ಮ ಹೃದಯಗಳ ಮಧ್ಯೆ ಸಮಾಧಿಯಲ್ಲಿ ಪ್ರಕಾಶಿಸುತ್ತದೆ.” (ವಿವೇಕಚೂಡಾಮಣಿ ೪೦೮–೪೧೦). ಅಂತಹ ಸಾಕ್ಷಾತ್ಕಾರವು ಮಾನವ ಸಾಧ್ಯವಾದುದು. ಅದು ಸಿದ್ಧಿಸಿತೆಂದರೆ ಜಗತ್ತು ನಮ್ಮ ದೃಷ್ಟಿಗೆ ಆಮೂಲಾಗ್ರವಾಗಿ ಬದಲಾಗುತ್ತದೆ. ವಿಶ್ವವು ಈಗ ನಮಗೆ ತೋರುವಂತೆ ಆಮೇಲೆ ತೋರುವುದೇ ಇಲ್ಲ.

ಈ ಪರಬ್ರಹ್ಮವು, ಈ ಸತ್ಯಸ್ಯ ಸತ್ಯವು ಅಜ್ಞಾತವಾದುದೆಂದು ಹಿಂದೆ ಹೇಳಿದೆ. ಆದರೆ ಅದು ಅಜ್ಞೇಯತಾವಾದಿಯ ದೃಷ್ಟಿಯಿಂದ ಹೇಳಿದ ಮಾತಲ್ಲ. ನಾವದನ್ನು ತಿಳಿಯಲಾರದೆ ಇರುವುದಕ್ಕೆ ಕಾರಣವೇನೆಂದರೆ ನಾವು ಈಗಾಗಲೇ ಅದಾಗಿದ್ದೇವೆ. ಅದನ್ನು ತಿಳಿದಿದ್ದೇವೆ ಎಂಬುದು ಶುದ್ಧ ದೈವನಿಂದೆ. ಆ ಬ್ರಹ್ಮವು ಹೇಗೆ ಒಂದು ದೃಷ್ಟಿಯಲ್ಲಿ ಈ ಮೇಜೇ ಆಗಿದ್ದರೂ ಇನ್ನೊಂದು ದೃಷ್ಟಿಯಲ್ಲಿ ಈ ಮೇಜು ಆಗಿಲ್ಲ ಎಂಬುದನ್ನೂ ವಿಚಾರಿಸಿದ್ದೇವೆ. ಈ ಮೇಜಿನ ನಾಮರೂಪಗಳನ್ನು ತೆಗೆದು ಹಾಕಿದರೆ ಉಳಿಯುವುದೇ ಬ್ರಹ್ಮ. ಏಕೆಂದರೆ ಎಲ್ಲ ವಸ್ತುಗಳ ಸಾರಸತ್ವವೂ ಬ್ರಹ್ಮವೇ ಆಗಿರುತ್ತದೆ.

“ನೀನೇ ಸ್ತ್ರೀ, ನೀನೇ ಪುರುಷ, ಯುವಕನೂ ಯುವತಿಯೂ ನೀನೇ, ಕೋಲು ಹಿಡಿದು ತತ್ತರಿಸಿ ನಡೆಯುವ ವೃದ್ಧನೂ ನೀನೇ, ಎಲ್ಲವೂ ನೀನೇ \textbf{(ತ್ವಂ ಸ್ತ್ರೀ ತ್ವಂ ಪುಮಾನಸಿ~। ತ್ವಂ ಕುಮಾರ ಉತ ವಾ ಕುಮಾರೀ~॥ ತ್ವಂ ಜೀರ್ಣೋದಂಡೇನ ವಂಚಸಿ~। ತ್ವಂ ಜಾತೋ ಭವಸಿ ವಿಶ್ವತೋಮುಖಃ~॥} ಶ್ವೇತಾಶ್ವತರ ಉಪನಿಷತ್ತು). ಅದೇ ಅದ್ವೈತ ಸಿದ್ಧಾಂತದ ವಿಷಯ. ಇನ್ನೆರಡು ಮಾತು: ವಸ್ತುಗಳ ಅಂತಃಸತ್ವದ ವಿಷಯವು ನಿಮಗಿಲ್ಲಿ ವಿಶದವಾಗುತ್ತದೆ. ವೈಜ್ಞಾನಿಕರ, ತಾರ್ಕಿಕರ ಮತ್ತು ಇತರರ ವಾದಬಾಣಗಳಿಗೆಲ್ಲ ಅದ್ವೈತ ಒಂದೇ ಪ್ರತಿವಾದ ಬಾಣ. ಬುದ್ಧಿಗೆ ಇಲ್ಲಿ ಬಲವಾದ ಆಧಾರವಿದೆ. ಆದರೆ ವೇದಾಂತಿಯು ಅದ್ವೈತಕ್ಕೆ ನಮ್ಮನ್ನು ಕೊಂಡೊಯ್ಯವ ಇತರ ಸೋಪಾನಗಳನ್ನು ಅಲ್ಲಗಳೆಯುವುದಿಲ್ಲ. ಹಿಂದಕ್ಕೆ ನೋಡಿ ಅವುಗಳನ್ನು ಆಶೀರ್ವದಿಸುತ್ತಾನೆ. ಅವು ಸತ್ಯ ಎಂಬುದು ಅವನಿಗೆ ಗೊತ್ತು. ಆದರೆ ಸತ್ಯವನ್ನು ಅವು ತಪ್ಪಾಗಿ ಗ್ರಹಿಸಿ ತಪ್ಪಾಗಿ ನಿರೂಪಿಸಿವೆ ಎಂಬುದೂ ಅವನಿಗೆ ಗೊತ್ತು. ಅವು ಅದೇ ಸತ್ಯವನ್ನೇ ಹೇಳುತ್ತಿವೆ, ಆದರೆ ಆ ಸತ್ಯವು ಮಾಯೆಯ ಗಾಜಿನ ಮೂಲಕ ಕಂಡದ್ದು, ವಿಕೃತವಾದದ್ದಾದರೂ ಅದು ಸತ್ಯವಲ್ಲದೆ ಬೇರಲ್ಲ. ಯಾವ ದೇವರನ್ನು ಅಜ್ಞಾನಿಯು ಪ್ರಕೃತಿಯ ಹೊರಗೆ ಇರುವಂತೆ ಕಾಣುತ್ತಾನೆಯೋ, ಅಲ್ಪಜ್ಞಾನಿಯು ಅಂತರ್ಯಾಮಿಯಾಗಿ ಕಾಣುತ್ತಾನೆಯೋ, ಜ್ಞಾನಿಯು ತನ್ನದೇ ಆತ್ಮವಾಗಿಯೂ, ಸರ್ವವಿಶ್ವವಾಗಿಯೂ ಕಾಣುತ್ತಾನೆಯೋ, ಅದು ಏಕವೂ ಅದ್ವಿತೀಯವೂ ದೃಷ್ಟಿಭೇದದಿಂದ ಮಾತ್ರವೇ ವಿಭಿನ್ನರೂಪಿಯೂ ಮಾಯೆಯ ವಿವಿಧ ಗಾಜುಗಳ ಮೂಲಕ ನೋಡಲ್ಪಟ್ಟದ್ದೂ, ಬೇರೆ ಬೇರೆಯ ಮನಸ್ಸುಗಳಿಂದ ಗ್ರಹಿಸಲ್ಪಟ್ಟದ್ದೂ ಆಗಿರುವ ಪರಬ್ರಹ್ಮವು. ಎಲ್ಲ ಭಿನ್ನತೆಗಳಿಗೂ ದೃಷ್ಟಿಭೇದವೇ ಕಾರಣ. ಈ ಮೆಟ್ಟಲುಗಳು ಸತ್ಯ ಮಾತ್ರವೆ ಅಲ್ಲ, ಸತ್ಯದಿಂದ ಉತ್ತಮತರ ಸತ್ಯಕ್ಕೆ ನಮ್ಮನ್ನು ಒಯ್ಯುವ ಸೋಪಾನಗಳು. ವಿಜ್ಞಾನಕ್ಕೂ, ಸಾಮಾನ್ಯ ಜ್ಞಾನಕ್ಕೂ ಇರುವ\break ಭೇದವೇನು? ಏನಾದರೂ ಅಪೂರ್ವ ಘಟನೆ ಸಂಭವಿಸಿದಾಗ ಹಳ್ಳಿಯವನೊಬ್ಬನನ್ನು ಅದಕ್ಕೆ ಕಾರಣವೇನು ಎಂದು ಕೇಳಿದರೆ ಅವನು ಯಾವುದಾದರೂ ಪಿಶಾಚಿಯ ಹೆಸರು ಹೇಳುತ್ತಾನೆ. ಏಕೆಂದರೆ ಘಟನೆಗಳ ಅಥವಾ ಕಾರ್ಯಗಳ ಹೊರಗೆ ಕಾರಣಗಳನ್ನು ಹುಡುಕುವುದು ಅಜ್ಞಾನದ ಸ್ವಭಾವ. ಕಲ್ಲು ಕೆಳಗೆ ಬಿದ್ದರೆ ಯಾವುದೋ ಭೂತ ಎಸೆಯಿತೆಂದು ಭಾವಿಸುತ್ತಾನೆ. ಆದರೆ ವಿಜ್ಞಾನಿಯು ಭೂಮಿಯಲ್ಲಿರುವ ಆಕರ್ಷಣ ಶಕ್ತಿಯೇ ಅದಕ್ಕೆ ಕಾರಣವಾಯಿತೆಂದು ತಿಳಿಯುತ್ತಾನೆ.

ಮತಗಳ ಮತ್ತು ವಿಜ್ಞಾನಗಳಲ್ಲಿ ನಡೆಯುವ ದ್ವಂದ್ವಯುದ್ಧ ಎಂಬುದೇನು? ಮತಗಳೆಲ್ಲ ಪ್ರಾಕೃತ ಘಟನೆಗಳಿಗೆ ಅಪ್ರಾಕೃತ ಅಥವಾ ಅತಿಪ್ರಾಕೃತ ಕಾರಣಗಳನ್ನು ಒಡ್ಡಿವೆ. ಆದ್ದರಿಂದಲೇ ಅವು ಸೂರ್ಯ ಚಂದ್ರ ಗಾಳಿ ಬೆಂಕಿ ಎಲ್ಲವುಗಳನ್ನು ಆಳುವ ಒಂದೊಂದು ದೇವತೆಯನ್ನು ಕಲ್ಪಿಸಿವೆ. ಪ್ರತಿಯೊಂದು ಬದಲಾವಣೆಗೂ ಯಾವುದೊ ಒಂದು ದೇವತೆ ಕಾರಣ, ಈ ಕಾರಣವು ವಸ್ತುವಿನ ಹೊರಗೆ ಇರುವುದು ಎಂದು ಮತಗಳು ಹೇಳುತ್ತವೆ. ಆದರೆ ವಿಜ್ಞಾನವಾದರೋ ವಸ್ತುವಿನ ಅಥವಾ ಘಟನೆಯ ಒಳಗಡೆಯೇ ಎಂದರೆ ಅದರ ಸ್ವಭಾವದಲ್ಲಿಯೇ ಕಾರಣವನ್ನು ಹುಡುಕುವುದು. ವಿಜ್ಞಾನಶಾಸ್ತ್ರವು ಮುಂದುವರಿದ ಹಾಗೆಲ್ಲಾ ಅದು ಹೆಜ್ಜೆ ಹೆಜ್ಜೆಗೂ ವಸ್ತುಗಳಿಗೆ ಮತ್ತು ಘಟನೆಗಳಿಗೆ ಸ್ವಾಭಾವಿಕವಾದ ಕಾರಣಗಳನ್ನು ಹುಡುಕಿ ಹುಡುಕಿ ಹೇಳುತ್ತಿದೆ. ಪ್ರಾಕೃತಿಕ ಘಟನೆಗಳಿಗೆ ಕಾರಣಗಳು ದೇವತೆಗಳಲ್ಲ ಎಂದು ಅದು ಹೇಳುತ್ತಿದೆ. ಅದ್ವೈತವೊಂದೇ ಆಧ್ಯಾತ್ಮಿಕ ವಿಷಯಗಳಲ್ಲಿ ಇದೇ ಕೆಲಸವನ್ನು ಮಾಡಿರುವುದರಿಂದ ಅದು ಮಾತ್ರವೇ ಅತ್ಯಂತ ವೈಜ್ಞಾನಿಕ ಧರ್ಮ. ಅದು ಹೇಳುತ್ತದೆ: ಈ ಜಗತ್ತು ವಿಶ್ವದ ಹೊರಗಿರುವ ಯಾವ ದೇವರಿಂದಾಗಲೀ, ಅಥವಾ ಯಾವ ಅತಿ ಪ್ರಾಕೃತ ಶಕ್ತಿಯಿಂದಾಗಲಿ ಸೃಷ್ಟಿಯಾಗಿಲ್ಲ. ಇದು ಸ್ವಯಂ ಸೃಷ್ಟಿಯಾಗುತ್ತದೆ. ಸ್ವಯಂ ಲಯವಾಗುತ್ತದೆ. ಸ್ವಯಂ ಪ್ರಕಾಶಿತವಾಗುತ್ತದೆ. ಇದು ಅನಂತವಾದುದು. ಇದು ಪರಬ್ರಹ್ಮ. “ತತ್ತ್ವಮಸಿ” – “ನೀನೇ ಅದು” ಇದೇ ಅದ್ವೈತಿಯ ಸಿದ್ಧಾಂತ.

ವಿಚಾರಿಸಿ ನೋಡಿ, ಇದೊಂದೇ ನಿಜವಾದ ವೈಜ್ಞಾನಿಕ ಮತವಾಗಲು ಅರ್ಹವಾಗಿದೆ. ಈ ಭರತಖಂಡದಲ್ಲಿ ವಿಜ್ಞಾನದ ಬಗ್ಗೆ ಅರ್ಧ ವಿದ್ಯಾವಂತರಾದವರು ನಡೆಸುತ್ತಿರುವ ಅಸಂಬದ್ಧ ಪ್ರಲಾಪವನ್ನು ಕೇಳಿದರೆ, ಪ್ರತಿದಿನವೂ ನನ್ನ ಕಿವಿಗೆ ಬೀಳುತ್ತಿರುವ ವಿಚಾರವಾದ ಯುಕ್ತಿವಾದ ಮೊದಲಾದುವುಗಳ ಗಲಭೆಗಳನ್ನು ಕೇಳಿದರೆ, ನೀವೆಲ್ಲರೂ ಧೈರ್ಯದಿಂದಲೂ ಸಾಹಸದಿಂದಲೂ ಅದ್ವೈತಿಗಳಾಗಿ, ಬುದ್ಧನು ಹೇಳಿದಂತೆ ಬಹುಜನರ ಸುಖಕ್ಕಾಗಿ, ಬಹುಜನರ ಹಿತಕ್ಕಾಗಿ ಆ ತತ್ತ್ವಗಳನ್ನು ಜನರಿಗೆ ಅಂಜದೆ ಬೋಧಿಸುತ್ತೀರೆಂದು ಹಾರೈಸುತ್ತೇನೆ. ಹಾಗೆ ಮಾಡದೇ ಇದ್ದರೆ ನೀವೆಲ್ಲರೂ ಹೇಡಿಗಳೆಂದು ಭಾವಿಸುತ್ತೇನೆ. ನಿಮ್ಮ ಹೇಡಿತನವನ್ನು ಬಿಡಲಾಗದಿದ್ದರೆ, ಕ್ಲೈಬ್ಯವೇ ನಿಮಗೆ ನೆರವಾಗುವ ಪಕ್ಷದಲ್ಲಿ ಇತರರಿಗೂ ಅದೇ ಸ್ವಾತಂತ್ರ್ಯವನ್ನು ಬಿಟ್ಟುಕೊಡಿ. ಬಡ ಮೂರ್ತಿಪೂಜಕನನ್ನು ಏಕೆ ಭಂಗಿಸುತ್ತೀರಿ? ಅವನನ್ನೇಕೆ ಪಿಶಾಚಿ ಎಂದು ಕರೆಯುತ್ತೀರಿ? ನಿಮಗಿಂತಲೂ ಭಿನ್ನಾಭಿಪ್ರಾಯ ಉಳ್ಳವರಿಗೆಲ್ಲ ನಿಮ್ಮ ಅಭಿಪ್ರಾಯವನ್ನೇ ಬಲಾತ್ಕಾರದಿಂದ ಬೋಧಿಸುತ್ತ ಹೋಗಬೇಡಿ. ವಿದ್ಯಾರ್ಜನೆ ಮಾಡಿರುವ ನೀವೇ ಹೇಡಿಗಳಾಗಿರುವಾಗ, ನೀವೇ ಸಮಾಜಕ್ಕೆ ಹೆದರಿ ನಡೆಯುವಾಗ, ನಿಮ್ಮ ಪುರಾತನ ಮೌಢ್ಯಗಳು ನಿಮ್ಮನ್ನೇ ಬೆದರಿಸುತ್ತಿರುವಾಗ, ಏನೂ ತಿಳಿಯದ ಬಡ ಜನರನ್ನು ಅವು ಎಷ್ಟರ ಮಟ್ಟಿಗೆ ಬೆದರಿಸುತ್ತಿರಬೇಕು? ಇನ್ನೆಷ್ಟರ ಮಟ್ಟಿಗೆ ಬಂಧಿಸಿರಬೇಕು. ಇದು ಅದ್ವೈತದ ನಿಲವು: ಇತರರ ಮೇಲೆ ಕರುಣೆ ಇರಲಿ. ದೇವರ ದಯೆಯಿಂದ ನಾಳೆಯೇ ಎಲ್ಲರೂ ಅದ್ವೈತಿಗಳಾದರೆ! ಮಾತಿನಲ್ಲಲ್ಲ, ಕಾರ್ಯದಲ್ಲಿ. ಆದರೆ ಅದು ಸಾಧ್ಯವಿಲ್ಲ. ಆದಕಾರಣ ಉತ್ತಮವಾದ ಎರಡನೆಯದನ್ನು ಆರಿಸಿಕೊಳ್ಳೋಣ. ಅಜ್ಞಾನಿಗಳನ್ನು ಕೈಹಿಡಿದು ಮೆಟ್ಟಲು ಮೆಟ್ಟಲಾಗಿ ಅವರ ಶಕ್ತಿಗೆ ಅನುಗುಣವಾಗಿ ಹೆಜ್ಜೆಯಿಡಿಸಿ, ಅವರು ಮೇಲೆ ಬರುವಂತೆ ಮಾಡಬೇಕು. ಭಾರತದಲ್ಲಿ ಧಾರ್ಮಿಕ ಬೆಳವಣಿಗೆಯ ಪ್ರತಿಯೊಂದು ಹೆಜ್ಜೆಯೂ ಪ್ರಗತಿಪರವಾದುದು ಎಂಬುದನ್ನು ತಿಳಿಯಿರಿ. ಅಂತಹ ಪ್ರಗತಿಯೂ ಅಧಮದಿಂದ ಉತ್ತಮಕ್ಕೆಂದು ಭಾವಿಸಬೇಡಿ, ಉತ್ತಮದಿಂದ ಉತ್ತಮ ತರಕ್ಕೆಂದು ತಿಳಿಯಿರಿ.

ಅದ್ವೈತಕ್ಕೆ ಸಂಬಂಧಿಸಿದಂತೆ ನೀತಿಯ ವಿಷಯವಾಗಿ ಒಂದೆರಡು ಮಾತುಗಳನ್ನು ಹೇಳಬೇಕಾಗಿದೆ. ಈಚೀಚೆಗೆ ಹುಡುಗರೆಲ್ಲ ಲಘುವಾಗಿ ಮಾತಾಡಲು ತೊಡಗಿದ್ದಾರೆ – ಯಾರು ಹೇಳಿಕೊಟ್ಟರೋ ಏನೋ ದೇವರೇ ಬಲ್ಲ! – ಅದ್ವೈತವು ಜನರನ್ನು ಅನೀತಿಯುತರನ್ನಾಗಿ ಮಾಡುತ್ತದಂತೆ. ಏಕೆಂದರೆ ನಾವೆಲ್ಲರೂ ಒಂದೇ ಆಗಿ, ನಾವೆಲ್ಲರೂ ದೇವರೇ ಆಗಿರುವ ಪಕ್ಷದಲ್ಲಿ ನೀತಿಯಿಂದೇನು ಪ್ರಯೋಜನ ಎಂದು ಅವರು ಕೇಳುತ್ತಾರೆ. ಆದರೆ ಅದು ಕಿರಾತರ ತರ್ಕ; ಅಂತಹ ಕಿರಾತರನ್ನು ಚಾವಟಿಯಿಂದ ಮಾತ್ರವೇ ಹತೋಟಿಯಲ್ಲಿಡಲು ಸಾಧ್ಯ. ಯಾವಾಗಲೂ ಚಾವಟಿಯಿಂದ ಮಾತ್ರವೇ ಹತೋಟಿಗೆ ಬರುವ ಮನುಷ್ಯನು ಮೊದಲು ಆತ್ಮಹತ್ಯೆ ಮಾಡಿಕೊಳ್ಳಬೇಕು. ಚಾವಟಿಯ ಹೊಡೆತ ಇಲ್ಲವೆಂದಾದರೆ ನೀವೆಲ್ಲ ರಕ್ಕಸರಾಗುತ್ತೀರಿ. ಇದು ನಿಮ್ಮ ಸ್ಥಿತಿಯಾದರೆ ನಿಮ್ಮನ್ನೆಲ್ಲ ಸಾಯಿಸುವುದೊಳಿತು. ನೀವೆಲ್ಲ ಬೆತ್ತದ ನಿಯಂತ್ರಣದಲ್ಲಿದ್ದೀರಿ, ನಿಮಗೆಲ್ಲಿಯ ಉದ್ಧಾರ?

ಎರಡನೆಯದಾಗಿ ಅದ್ವೈತವು ಮಾತ್ರವೇ ನೀತಿ ಎಂಬುದನ್ನು ಸಮರ್ಪಕವಾಗಿ ವಿವರಿಸಬಲ್ಲುದು. ಇತರರಿಗೆ ಒಳ್ಳೆಯದು ಮಾಡುವುದು, ನಿಃಸ್ವಾರ್ಥರಾಗಿರುವುದು, ಇವೇ ನೀತಿಯ ಸಾರವೆಂದು ಎಲ್ಲ ಮತಗಳು ಸಾರುತ್ತವೆ. “ನಿಃಸ್ವಾರ್ಥಿಯಾಗಿರು”. ನಾನು ಏಕೆ ನಿಃಸ್ವಾರ್ಥಿಯಾಗಿರಬೇಕು? ಯಾವುದೋ ದೇವರು ಹಾಗೆ ಹೇಳಿದನೆಂತಲೇ? ಹಾಗೆ ಹೇಳುವವನ ಗೋಜಿಗೆ ನಾನು ಬರುವುದಿಲ್ಲ! ಯಾವುದೋ ಧರ್ಮಗ್ರಂಥವು ವಿಧಿಸಿದೆಯೆಂತಲೇ? ವಿಧಿಸಲಿ! ನನಗೇನು? ನನಗೂ ಅದಕ್ಕೂ ಸಂಬಂಧವಿಲ್ಲ. ಅವನವನು ಅವನವನಿಗೇ; ಹಿಂದೆ ಬಿದ್ದವನು ಮಣ್ಣು ಮುಕ್ಕಲಿ; ಹೀಗೆಂಬುದೇ ಈ ಪ್ರಪಂಚದಲ್ಲಿ ಅನೇಕರ ನೀತಿಯಾಗಿದೆ. ನಾನೇಕೆ ನೀತಿವಂತನಾಗಿರಬೇಕು? ಕಾರಣವೇನು? ಭಗವದ್ಗೀತೆಯಲ್ಲಿ ಹೇಳಿರುವ ತತ್ತ್ವವನ್ನು ತಿಳಿದ ಹೊರತು ನಮಗೆ ಸರಿಯಾದ ವಿವರಣೆ ದೊರಕುವುದಿಲ್ಲ. ಗೀತೆಯು ಹೇಳುತ್ತದೆ: “ಎಲ್ಲರನ್ನು ತನ್ನಲ್ಲಿಯೂ ತನ್ನನ್ನು ಎಲ್ಲರಲ್ಲಿಯೂ ಕಾಣುವುದರಿಂದ ಯಾವಾತನು ಸರ್ವಭೂತಗಳಲ್ಲಿಯೂ ಈಶ್ವರನನ್ನೇ ದರ್ಶಿಸುತ್ತಾನೆಯೋ ಅಂತಹ ಮುನಿಯು ಆತ್ಮದಿಂದ ಆತ್ಮನನ್ನು ಕೊಲ್ಲುವುದಿಲ್ಲ.” ನೀನು ಇತರರಿಗೆ ಕೊಟ್ಟ ಪೀಡನೆಯು ನಿನ್ನನ್ನೇ ಪೀಡಿಸುತ್ತದೆ ಎಂಬುದು ಅದ್ವೈತದಿಂದ ಗೊತ್ತಾಗುತ್ತದೆ. ಏಕೆಂದರೆ ಎಲ್ಲವೂ ನೀನೆ. ನಿನಗೆ ತಿಳಿದಿರಲಿ, ತಿಳಿಯದಿರಲಿ, ಎಲ್ಲ ಕೈಗಳಿಂದಲೂ ನೀನು ಕೆಲಸ ಮಾಡುತ್ತೀಯೆ; ಎಲ್ಲ ಕಾಲುಗಳಿಂದಲೂ ನೀನು ನಡೆಯುತ್ತೀಯೆ. ಅರಮನೆಯಲ್ಲಿ ಆನಂದಪಡುವ ದೊರೆಯೂ ನೀನೇ; ಬೀದಿಯಲ್ಲಿ ತಿರುಪೆ ಬೇಡಿ ತೊಳಲುವ ತಿರುಕನೂ ನೀನೇ; ಪಂಡಿತನೂ ನೀನೇ; ಪಾಮರನೂ ನೀನೇ; ಬಲಿಷ್ಠನೂ ನೀನೇ; ದುರ್ಬಲನೂ ನೀನೇ. ಇದನ್ನು ತಿಳಿದು ಸಹಾನುಭೂತಿಯಿಂದಿರು. ಆದಕಾರಣವೇ ನಾನು ಇತರರನ್ನು ನೋಯಿಸಬಾರದು. ಅದಕ್ಕಾಗಿಯೇ ನಾನು ಹಸಿದು ಸಾಯುವ ಸಮಯದಲ್ಲಿಯೇ ಇತರ ಲಕ್ಷಾಂತರ ಬಾಯಿಗಳು ಊಟ ಮಾಡುತ್ತಿರುತ್ತವೆ. ಅವೆಲ್ಲವೂ ನನ್ನವೇ ಆದುದರಿಂದ ‘ನಾನು’ ‘ನನ್ನದು’ ಎಂಬುದನ್ನು ನಾನು ಗಮನಿಸಬಾರದು. ಏಕೆಂದರೆ ವಿಶ್ವವೆಲ್ಲಾ ನನ್ನದೆ. ಅದರಲ್ಲಿರುವ ಅಸಂಖ್ಯ ಸುಖಗಳನ್ನು, ಸಮಸ್ತ ಆನಂದವನ್ನು ಒಂದೇ ಸಾರಿಗೆ ನಾನು ಅನುಭವಿಸುತ್ತಿರುತ್ತೇನೆ. ವಿಶ್ವದಲ್ಲಿ ಐಕ್ಯವಾಗಿರುವ ನನ್ನನ್ನಾರು ಕೊಲ್ಲಬಲ್ಲರು? ಇಲ್ಲಿದೆ ಸಕಲ ನೀತಿಯ ಆಧಾರ. ಇತರರು ನೀತಿಯನ್ನೇನೋ ಬೋಧಿಸುತ್ತಾರೆ. ಆದರೆ ಅದಕ್ಕೆ ಕಾರಣಗಳನ್ನು ಕೊಡಲಾರರು. ಅದ್ವೈತವು ಮಾತ್ರವೇ ಅದಕ್ಕೆ ವಿವರಣೆಯನ್ನು ನೀಡಬಲ್ಲದು.

ಪ್ರಯೋಜನವೇನು? ಶಕ್ತಿಲಾಭ! ನೀವೇ ಜಗತ್ತಿನ ಮೇಲೆ ಎಸೆದಿರುವ ಮಾಯೆಯ ಜಾಲವನ್ನು ಕಿತ್ತು ಎಸೆಯಿರಿ. ಮಾನವ ಕೋಟಿಯನ್ನು ದುರ್ಬಲಗೊಳಿಸುವ ಆಲೋಚನೆಗಳನ್ನು ಮಾತುಗಳನ್ನು ಪ್ರಸಾರ ಮಾಡಬೇಡಿ. ಎಲ್ಲ ಪಾಪಗಳನ್ನು, ಕೇಡುಗಳನ್ನು ಒಂದು ಪದದಲ್ಲಿ ಅಡಕ ಮಾಡಿಬಿಡಬಹುದು. ಆ ಪದ “ದೌರ್ಬಲ್ಯ”. ಎಲ್ಲ ಪಾಪಕಾರ್ಯಗಳನ್ನು ಪ್ರೇರೇಪಿಸುವುದು ದೌರ್ಬಲ್ಯ; ಎಲ್ಲ ಸ್ವಾರ್ಥದ ಮೂಲ ದೌರ್ಬಲ್ಯ. ಇತರರನ್ನು ಹಿಂಸಿಸುವಂತೆ ಮಾಡುವುದು ದೌರ್ಬಲ್ಯ. ಜನರು ನಿಜವಾಗಿ ತಾವು ಏನು ಅಲ್ಲವೋ ಅದರಂತೆ ಕಾಣಿಸಿಕೊಳ್ಳಲು ಪ್ರಯತ್ನಿಸುವುದಕ್ಕೆ ಕಾರಣ ದೌರ್ಬಲ್ಯವೇ. ಎಲ್ಲರಿಗೂ ತಮ್ಮ ತಮ್ಮ ನೈಜ ಸ್ವರೂಪವು ತಿಳಿಯಲಿ; ಎಲ್ಲರೂ ಹಗಲಿರುಳೂ ತಮ್ಮ ಸ್ವರೂಪದ ವಿಚಾರವಾಗಿಯೇ ‘ಸೋಽಹಂ’ ಮಂತ್ರವನ್ನು ಜಪಿಸಲಿ. ತಾಯಿಯ ಸ್ತನ್ಯದೊಡನೆ ಆ ಶಕ್ತಿಭಾವವು ಶಿಶುವಿಗೆ ಲಭಿಸಲಿ. ಶಿವೋಽಹಂ, ಶಿವೋಽಹಂ ಎಂಬ ಶಕ್ತಿಮಂತ್ರವು ಎಲ್ಲರ ಮನಸ್ಸನ್ನೂ ಪ್ರವೇಶಿಸಲಿ. ಇದನ್ನು ಮೊದಲು ಕೇಳಬೇಕು – \textbf{ಶ್ರೋತವ್ಯೋ ಮಂತವ್ಯೋ ನಿದಿಧ್ಯಾಸಿ ತವ್ಯಃ} ಇತ್ಯಾದಿ – ಅನಂತರ ಅದನ್ನು ಕುರಿತು ಮನನ ಮಾಡಲಿ. ಆ ಮಂತ್ರದಿಂದ ಲೋಕವು ಇದುವರೆಗೂ ಕಾಣದಿರುವ ಶಕ್ತಿಗಳೂ ಕಾರ್ಯಗಳೂ ಹೊರಹೊಮ್ಮುವುದರಲ್ಲಿ ಸಂದೇಹವಿಲ್ಲ. ನಾವು ಮಾಡಬೇಕಾದುದೇನು? ಅದ್ವೈತವು ಕಾರ್ಯಕಾರಿಯಲ್ಲವೆಂದು, ಎಂದರೆ ಅದು ಇನ್ನೂ ಲೌಕಿಕದಲ್ಲಿ ವ್ಯಕ್ತವಾಗಿಲ್ಲವೆಂದು ಕೆಲವರು ಹೇಳುತ್ತಾರೆ. ಆ ಹೇಳಿಕೆಯಲ್ಲಿ ಸ್ವಲ್ಪ ಮಟ್ಟಿಗೆ ನಿಜಾಂಶವಿದೆ. ಅಂತೆಯೇ ಪರಿಹಾರವೂ ಇದೆ. ಈ ವೇದಮಂತ್ರವನ್ನು ನೆನಪಿಗೆ ತಂದುಕೊಳ್ಳಿ– “ಓಂ ಇದು ಮಹಾರಹಸ್ಯ, ಇದು ಮಹಾಸಂಪತ್ತು; ಈ ಓಂಕಾರದ ರಹಸ್ಯವನ್ನು ಅರಿತವನಿಗೆ ಬಯಕೆಗಳೆಲ್ಲ ಕೈಗೂಡುತ್ತವೆ.” \textbf{(ಓಮಿತ್ಯೇಕಾಕ್ಷರಂ ಬ್ರಹ್ಮ ಓಮಿತ್ಯೇಕಾಕ್ಷರಂ ಪರಂ ಓಮಿತ್ಯೇಕಾಕ್ಷರಂ ಜ್ಞಾತ್ವಾ ಯೋ ಯದಿಚ್ಛತಿ ತಸ್ಯತತ್​~॥)} ಮೊದಲು ನೀವು ಈ ಓಂಕಾರದ ರಹಸ್ಯವನ್ನು, ನೀವೇ ಅದು ಎಂಬುದನ್ನು ಅರಿಯಬೇಕು. “ತತ್ತ್ವಮಸಿ” ಎಂಬ ಮಂತ್ರದ ರಹಸ್ಯವನ್ನು ತಿಳಿಯಬೇಕು. ಹಾಗೆ ಮಾಡಿದರೇನೇ ನಿಮಗೆ ಏನು ಬೇಕೋ ಅದು ದೊರಕುತ್ತದೆ. ಲೌಕಿಕದಲ್ಲಿಯೂ ಮಹತ್ವವನ್ನು ಸಂಪಾದನೆ ಮಾಡಬೇಕಾದರೆ “ನೀವೇ ಅದು” ಎಂಬುದನ್ನು ನಂಬಬೇಕು. ನಾನೊಂದು ಸಣ್ಣ ಗುಳ್ಳೆಯಾಗಿರಬಹುದು. ನೀವು ಪರ್ವತೋಪಮ ತರಂಗವಾಗಿರಬಹುದು. ಆದರೆ ನಮ್ಮಿಬ್ಬರಿಗೂ ಅನಂತ ಸಮುದ್ರವು ಆಧಾರವೆಂಬುದನ್ನು ತಿಳಿಯಬೇಕು. ಅನಂತವಾದ ಬ್ರಹ್ಮವು ನಮ್ಮೆಲ್ಲರಿಗೂ ಶಕ್ತಿಸಾಗರ. ನಾನು ಮತ್ತು ನೀವು ಆ ಶಕ್ತಿ ನಿಧಿಯಿಂದ ಬೇಕಾದಷ್ಟನ್ನು ಪಡೆದುಕೊಳ್ಳಬಹುದು. ಆದ್ದರಿಂದ ನಿಮ್ಮಲ್ಲಿ ನೀವು ನಂಬಿಕೆ ಇಡಿ. ಆತ್ಮಶ್ರದ್ಧೆಯೇ ಅದ್ವೈತದ ಗುಟ್ಟು. ಮೊದಲು ಆತ್ಮಶ್ರದ್ಧೆ, ತರುವಾಯ, ಉಳಿದುದರಲ್ಲಿ ನಂಬಿಕೆ. ಇದೇ ಅದ್ವೈತದ ಮಹಾರಹಸ್ಯ. ಜಗತ್ತಿನ ಇತಿಹಾಸವನ್ನು ನೋಡಿದರೆ ಆತ್ಮಶ್ರದ್ಧೆಯುಳ್ಳ ಜನಾಂಗಗಳೇ ಶ್ರೇಷ್ಠವೂ ಬಲಶಾಲಿಗಳೂ ಆಗಿವೆ ಎಂಬ ವಿಷಯವು ಗೊತ್ತಾಗುತ್ತದೆ. ಒಂದೊಂದು ರಾಷ್ಟ್ರದ ಇತಿಹಾಸವನ್ನು ತೆಗೆದುಕೊಂಡರೂ ಆತ್ಮಶ್ರದ್ಧೆಯುಳ್ಳ ವ್ಯಕ್ತಿಗಳೇ ಶ್ರೇಷ್ಠರೂ, ಶಕ್ತಿವಂತರೂ, ಆಗಿದ್ದಾರೆಂಬುದು ಸ್ಪಷ್ಟವಾಗುತ್ತದೆ. ಇಲ್ಲಿಗೆ, ಈ ನಮ್ಮ ಭರತಖಂಡಕ್ಕೆ ಒಬ್ಬ ಆಂಗ್ಲೇಯನು ಬಂದನು. ಅವನೊಬ್ಬ ಸಾಮಾನ್ಯ ಕರಣಿಕನಾಗಿದ್ದನು. ಹಣ ಸಾಲದುದಕ್ಕಾಗಿಯೋ, ಯಾವುದೋ ಕಾರಣಗಳಿಗಾಗಿಯೋ ಎರಡು ಸಾರಿ ತಲೆಗೆ ಗುಂಡಿನಿಂದ ಹೊಡೆದುಕೊಂಡು ಮಿದುಳನ್ನು ಹಾರಿಸಿ ಆತ್ಮಹತ್ಯೆ ಮಾಡಿಕೊಳ್ಳಲು ಪ್ರಯತ್ನಿಸಿದನು. ಆದರೆ ಆ ಪ್ರಯತ್ನಗಳು ಸಫಲವಾಗಲಿಲ್ಲ. ತರುವಾಯ ಆತನಲ್ಲಿ ಆತ್ಮಶ್ರದ್ಧೆಯು ಮೂಡಿತು. ತಾನು ಮಹತ್ಕಾರ್ಯಗಳನ್ನು ಮಾಡುವುದಕ್ಕಾಗಿಯೇ ಹುಟ್ಟಿದ್ದೇನೆ ಎಂದು ಆತ ನಂಬಿದನು. ಆ ಮನುಷ್ಯನೇ ಮುಂದೆ ಲಾರ್ಡ್​ ಕ್ಲೈವ್​ ಎಂಬ ಹೆಸರಿನಿಂದ ಒಂದು ದೊಡ್ಡ ಚಕ್ರಾಧಿಪತ್ಯದ ಸ್ಥಾಪಕನಾದನು. ಅವನೆಲ್ಲಿಯಾದರೂ ಪಾದ್ರಿಗಳನ್ನು ನಂಬಿ ಹೊಟ್ಟೆಯ ಮೇಲೆ ತೆವಳುತ್ತಾ “ಓ ದೇವರೇ ನಾನು ದುರ್ಬಲ, ಪಾಪಿ; ನಾನು ಹೀನ ಮನುಷ್ಯ” ಎಂದು ಹೇಳಿಕೊಂಡಿದ್ದರೆ ಅವನೆಲ್ಲಿರುತ್ತಿದ್ದನು? ಹುಚ್ಚರ ಆಸ್ಪತ್ರೆಯಲ್ಲಿ! ನಿಮಗೆ ದುರ್ಬಲಕಾರಿಗಳಾದ ವಿಷಯಗಳನ್ನು ಬೋಧಿಸಿ ಹುಚ್ಚರನ್ನಾಗಿ ಮಾಡಿದ್ದಾರೆ. ನಾನು ಜಗತ್ತನ್ನೆಲ್ಲಾ ಸಂಚರಿಸಿ ಈ ದೌರ್ಬಲ್ಯ ಬೋಧನೆಯ ವಿಷಮಯ ಪರಿಣಾಮವನ್ನೂ, ದೈನ್ಯವನ್ನೂ ಕಣ್ಣಾರೆ ಕಂಡಿದ್ದೇನೆ. ಆ ಬೋಧನೆಗಳು ಮಾನವ ವರ್ಗಕ್ಕೆ ವಿನಾಶಕಾರಿಗಳಾಗಿವೆ. ನಮ್ಮ ಮಕ್ಕಳನ್ನು ಹೀಗೆಯೇ ಬೆಳೆಸಿದ್ದೇವೆ. ಅಂತಹ ಬೋಧನೆಯಲ್ಲಿ ಬೆಳೆದ ನಮ್ಮ ಮಕ್ಕಳು ಅರೆ ಮರುಳಾಗಿರುವುದರಲ್ಲಿ ಮಹದಾಶ್ಚರ್ಯವೇನು?

ಇದೇ ಅದ್ವೈತ ವೇದಾಂತದ ಲೌಕಿಕ ಅಂಶ, ಎಂದರೆ ಕಾರ್ಯಕಾರಿಯಾದ ಅಂಶ. ಆದ್ದರಿಂದ ನಿಮ್ಮಲ್ಲಿ ನೀವು ಶ್ರದ್ಧೆ ಇಡಿ. ನಿಮಗೆ ಲೌಕಿಕ ಸಂಪತ್ತು ಬೇಕಾದರೆ ಸಾಹಸದಿಂದ ಕೆಲಸಮಾಡಿರಿ. ಲಭಿಸಿಯೇ ಲಭಿಸುತ್ತದೆ. ಬೇಕಾದರೆ ಮಾನಸಿಕ ಭೂಮಿಕೆಯಲ್ಲಿ ಸಾಧನೆಮಾಡಿ; ಮಹಾಮೇಧಾವಿಗಳಾಗಿಯೇ ಆಗುತ್ತೀರಿ. ನಿಮಗೆ ಮುಕ್ತಿ ಬೇಕಾದರೆ ಧಾರ್ಮಿಕ ಭೂಮಿಕೆಯಲ್ಲಿ ಸಾಧನೆಮಾಡಿ; ನೀವು ಮುಕ್ತರಾಗಿಯೇ ಆಗುತ್ತೀರಿ. ಪೂರ್ಣಾನಂದದ ನಿರ್ವಾಣವನ್ನು ಹೊಂದಿಯೇ ಹೊಂದುತ್ತೀರಿ. ಒಂದು ದೋಷವೇನೆಂದರೆ ಅದ್ವೈತವು ಇದುವರೆಗೂ ಧರ್ಮಭೂಮಿಕೆಯಲ್ಲಿ ಮಾತ್ರ ಕಾರ್ಯಕಾರಿಯಾಗಿತ್ತು; ಬೇರೆ ಕ್ಷೇತ್ರಗಳನ್ನು ಅದು ಪ್ರವೇಶಿಸಿರಲಿಲ್ಲ. ಈಗ ಅದನ್ನು ವ್ಯವಹಾರ ಕ್ಷೇತ್ರಕ್ಕೆ ಎಳೆದು ತರುವುದು ನಮ್ಮ ಕರ್ತವ್ಯ. ಇನ್ನು ಮುಂದೆ ಅದ್ವೈತವು ‘ರಹಸ್ಯ’ವಾಗಿರಬಾರದು; ಹಿಮಾಲಯ ಪರ್ವತಗಳ, ಗುಹವನಾಂತರಗಳಲ್ಲಿರುವ ಸಾಧು ಸಂನ್ಯಾಸಿಗಳ `ಗುಟ್ಟು’ ಮಾತ್ರ ಆಗಿರಬಾರದು; ಅದು ಸಾಮಾನ್ಯ ಜನಗಳ ದಿನದಿನದ ಉಪಯೋಗದ ವಸ್ತುವಾಗಬೇಕು. ಅದು ಅರಸರ ಅರಮನೆಗಳಲ್ಲಿಯೂ ಬಡವರ ಗುಡಿಸಲಿನಲ್ಲಿಯೂ ತಪಸ್ವಿಗಳ ಗುಹೆಗಳಲ್ಲಿಯೂ ಕಾರ್ಯಕಾರಿಯಾಗಬೇಕು. ಎಲ್ಲೆಲ್ಲಿಯೂ ಅದು ಕಾರ್ಯಗತವಾಗಬಲ್ಲುದು. ನೀವು ಮಹಿಳೆಯಾಗಿರಬಹುದು, ಶೂದ್ರರಾಗಿರಬಹುದು, ಆ ಕಾರಣದಿಂದ ಅಂಜಬೇಕಾಗಿಲ್ಲ. ಏಕೆಂದರೆ ಈ ಧರ್ಮವು ಕೃಷ್ಣನು ಹೇಳುವಂತೆ ಮಹತ್ತಾದುದು – \textbf{ಸ್ವಲ್ಪಮಪ್ಯಸ್ಯ ಧರ್ಮಸ್ಯ ತ್ರಾಯತೇ ಮಹತೋ ಭಯಾತ್​} – ಈ ತತ್ತ್ವದ ಸ್ವಲ್ಪ ಮಾತ್ರ ಅನುಷ್ಠಾನವೂ ಕೂಡ ಮಹತ್ತಾದ ಭಯದಿಂದ ನಮ್ಮನ್ನು ಪಾರು ಮಾಡುತ್ತದೆ.

ಆದ್ದರಿಂದ ಆರ್ಯಮಾತೆಯ ಪುತ್ರರಿರಾ ಎಚ್ಚರಗೊಳ್ಳಿ, ಎದ್ದೇಳಿ; ಸೋಮಾರಿಗಳಾಗಬೇಡಿ; ಗುರಿಯು ದೊರಕುವವರೆವಿಗೂ ನಿಲ್ಲಬೇಡಿ. ಜೀವನದ ಪ್ರತಿಯೊಂದು ಭೂಮಿಕೆಯಲ್ಲಿಯೂ ಅದ್ವೈತವನ್ನು ಅನುಷ್ಠಾನ ಮಾಡುವ ಸಮಯವು ನಿಮಗೀಗ ಪ್ರಾಪ್ತವಾಗಿದೆ. ಆಕಾಶದಲ್ಲಿ ಇರುವ ಅದನ್ನು ಭೂಮಿಗೆ ಎಳೆದು ತರೋಣ. ಇದೇ ನಮ್ಮ ಮುಂದಿರುವ ಆಧುನಿಕ ಕರ್ತವ್ಯ. "ಅದೇ ನಿಮ್ಮ ಕರ್ತವ್ಯ” ಎಂದು ನಮ್ಮ ಪೂರ್ವಿಕರ ಆತ್ಮಗಳು ಕೂಗಿ ಹೇಳುತ್ತಿವೆ. ಸಮಾಜದ ಮೂಲೆ ಮುಡುಕುಗಳನ್ನು ಮುಟ್ಟುವವರೆಗೂ, ಅದು ಸರ್ವರ ಸ್ವತ್ತಾಗುವ ವರೆಗೂ, ಅದು ನಮ್ಮ ಬದುಕಿನ ಅವಿಭಾಜ್ಯ ಅಂಗವಾಗುವವರೆಗೂ, ನಮ್ಮ ನಾಡಿಯನ್ನು ಪ್ರವೇಶಿಸಿ ಪ್ರತಿಯೊಂದು ರಕ್ತಕಣದಲ್ಲಿಯೂ ಅನುರಣಿತವಾಗುವವರೆಗೂ ನಿಮ್ಮ ಈ ಬೋಧನೆಯು ಜಗತ್ತನ್ನೆಲ್ಲ ವ್ಯಾಪಿಸಲಿ.

\vskip 2pt

ಈ ನನ್ನ ಮಾತುಗಳನ್ನು ಕೇಳಿದರೆ ಆಶ್ಚರ್ಯವಾಗಬಹುದು; ಅನುಷ್ಠಾನದಲ್ಲಿ ಅಮೇರಿಕನ್ನರು ನಮಗಿಂತಲೂ ಹೆಚ್ಚಿನ ವೇದಾಂತಿಗಳು. ನ್ಯೂಯಾರ್ಕ್​ ನಗರದಲ್ಲಿ ಸಮುದ್ರ ತೀರದಲ್ಲಿ ನಿಂತು ಅಲ್ಲಿಗೆ ಬರುತ್ತಿದ್ದ ವಿದೇಶಿಯರನ್ನು ನೋಡುತ್ತಿದ್ದೆ. ಪ್ರಪಂಚದ ನಾನಾ ಭಾಗಗಳಿಂದ ಜರ್ಝರಿತರಾಗಿ ಪತಿತರಾಗಿ ಸಣ್ಣ ಪುಟ್ಟ ಬಟ್ಟೆಯ ಗಂಟುಗಳನ್ನು ಹೊತ್ತು, ಚಿಂದಿಬಟ್ಟೆಯನ್ನುಟ್ಟು, ಮನುಷ್ಯರನ್ನು ಮುಖವೆತ್ತಿ ನೋಡಲೂ ಸಾಮರ್ಥ್ಯವಿಲ್ಲದವರಾಗಿ ಅಸಂಖ್ಯಾತ ಜನಗಳು ಅಲ್ಲಿಗೆ ಬರುತ್ತಿದ್ದರು. ಒಬ್ಬ ಪೋಲೀಸರನ್ನು ಕಂಡರೆ ಹೆದರಿ, ದಾರಿಯ ಮತ್ತೊಂದು ಪಕ್ಕಕ್ಕೆ ಹೋಗುತ್ತಿದ್ದರು. ಅಂತಹ ದೀನರು ಆರೇ ತಿಂಗಳಲ್ಲಿ ಸುಸಜ್ಜಿತ ವಸನ ಭೂಷಿತರಾಗಿ ಕತ್ತೆತ್ತಿ ಬೀದಿಯಲ್ಲಿ ಠೀವಿಯಿಂದ ನಡೆಯುತ್ತಿದ್ದರು! ಈ ಅದ್ಭುತ ಮಾರ್ಪಾಡಿಗೆ ಕಾರಣವೇನು? ಅದರಲ್ಲೊಬ್ಬನು ಬಹುಶಃ ಆರ್ಮೇನಿಯಾದಿಂದಲೋ ಅಥವಾ ಇನ್ನಾವ ದೇಶದಿಂದಲೋ ಬಂದವನಾಗಿರಬಹುದು. ಅಲ್ಲಿ ಎಲ್ಲರೂ ಅವನನ್ನು ಗುಲಾಮನೆಂದು ಕರೆದು ಕಾಲಿನಿಂದ ತುಳಿದು, ಅವನು ಯಾವಾಗಲೂ ಹಾಗೆಯೇ ಇರಬೇಕೆಂದು ಹೇಳುತ್ತಿದ್ದರು. ಅವನೆಲ್ಲಿಯಾದರೂ ತಲೆಯೆತ್ತಲು ಪ್ರಯತ್ನಪಟ್ಟರೆ ಆತನ ಜೀವವನ್ನೇ ಹಿಂಡಿ ತೆಗೆಯುತ್ತಿದ್ದರು. ಅಲ್ಲಿ ಅವನ ಸುತ್ತಣ ಗಾಳಿಯೂ ಕೂಡ “ನೀನು ಗುಲಾಮ, ಹುಟ್ಟುಗುಲಾಮ; ನೀನು ಹಾಗೆಯೇ ಇರಬೇಕು. ನೀನು ಹುಟ್ಟು ದರಿದ್ರ, ದಾರಿದ್ರ್ಯವೇ ನಿನ್ನ ಬಾಳು” ಎಂದು ಹೇಳುತ್ತಿತ್ತು. ಆದರೆ ಅವನು ನ್ಯೂಯಾರ್ಕ್​ ಬೀದಿಗಳಲ್ಲಿ ಕಾಲಿಟ್ಟ ಕೂಡಲೆ, ಒಳ್ಳೆಯ ಉಡುಪನ್ನು ಧರಿಸಿದ ದೊಡ್ಡ ಮನುಷ್ಯರೂ ಕೂಡ ತನಗೆ ಹಸ್ತಲಾಘವ ಕೊಡುವುದನ್ನು ಕಂಡನು. ಅಲ್ಲಿ ಒಳ್ಳೆಯ ಬಟ್ಟೆ ಹಾಕಿದವನಿಗೂ ಹಾಕದವನಿಗೂ ಭೇದವೇ ಇರಲಿಲ್ಲ. ಸ್ವಲ್ಪ ದೂರ ಹೋದ ಮೇಲೆ ಒಂದು ಒಳ್ಳೆಯ ಹೋಟೆಲಿನಲ್ಲಿ ದೊಡ್ಡ ಮನುಷ್ಯರು ಮೇಜಿನ ಸುತ್ತ ಊಟ ಮಾಡುತ್ತಿರುವುದನ್ನು ಕಂಡನು. ಅವರು ಇವನನ್ನು ಸಹ ಪಂಕ್ತಿಗೆ ಕರೆದರು. ಹೀಗೆ ಅವನು ಸುತ್ತಾಡಿದಂತೆಲ್ಲಾ ಒಂದು ನೂತನ ಜೀವನಕ್ರಮವನ್ನೂ, ಲೋಕದಲ್ಲಿ ತನ್ನನ್ನು ಗೌರವ ಯೋಗ್ಯವಾದ ಮನುಷ್ಯನೆಂದು ಪರಿಗಣಿಸುವ ಸ್ಥಳವನ್ನು ಕಂಡುಹಿಡಿದನು. ಬಹುಶಃ ಅವನು ಸಂಯುಕ್ತ ಸಂಸ್ಥಾನಗಳ ವಾಷಿಂಗ್​ಟನ್ನಿಗೆ ಹೋಗಬಹುದು. ಬಹುಶಃ ಅಲ್ಲಿ ಅವನು ದೂರದ ಹಳ್ಳಿಗಳಿಂದ ಬಂದ, ಅಷ್ಟೇನೂ ಒಳ್ಳೆಯ ಬಟ್ಟೆಗಳನ್ನು ಹಾಕಿಕೊಂಡಿಲ್ಲದ ರೈತರು ಅಧ್ಯಕ್ಷನೊಡನೆ ಕೈಕುಲಕಿಸುವುದನ್ನು ಕಂಡು, ತಾನೂ ಧೈರ್ಯ ತಂದುಕೊಂಡು ಅವರಂತೆಯೇ ಮಾಡಿದನು. ಹೀಗೆ ಅವನಿಗೆ ಮಾಯೆಯ ತೆರೆಯು ಮೆಲ್ಲಮೆಲ್ಲನೆ ಹರಿದು ಬೀಳತೊಡಗಿತು. ಅವನು ಬ್ರಹ್ಮವೇ. ಸ್ವಾತಂತ್ರ್ಯ ಮೂರ್ತಿಯಾಗಿದ್ದ ಅವನು, ದೌರ್ಬಲ್ಯ–ದಾಸ್ಯಗಳ ಬೋಧನೆಯ ಮಾಯೆಯಿಂದ ಗುಲಾಮನಾಗಿದ್ದವನು ಎಚ್ಚೆತ್ತು ನೋಡುತ್ತಾನೆ, ಜಗತ್ತಿನ ಮನುಷ್ಯರಲ್ಲಿ ತಾನೂ ಒಬ್ಬ ಮಾನ್ಯ ವ್ಯಕ್ತಿಯಾಗಿದ್ದಾನೆ! ನಮ್ಮ ದೇಶದಲ್ಲಿ, ವೇದಾಂತದ ಜನ್ಮ ಸ್ಥಾನವಾದ ಈ ಪುಣ್ಯ ಭೂಮಿಯಲ್ಲಿ, ನಮ್ಮ ಸಾಮಾನ್ಯ ಜನಗಳು ದೌರ್ಬಲ್ಯ – ದಾಸ್ಯಗಳ ಬೋಧನೆಯಿಂದ ಅನೇಕ ಕಾಲದಿಂದಲೂ ಸಮ್ಮೋಹಿತರಾಗಿದ್ದಾರೆ. ಅವರನ್ನು ಮುಟ್ಟಿದರೆ ಮೈಲಿಗೆಯಂತೆ, ಅವರ ಜೊತೆಯಲ್ಲಿ ಕುಳಿತರೆ ಮೈಲಿಗೆಯಂತೆ! ದರಿದ್ರರಾಗಿ ಹುಟ್ಟಿದವರು ದಾರಿದ್ರ್ಯದಲ್ಲಿಯೇ ಇರಬೇಕಂತೆ. ಪರಿಪೂರ್ಣವಾಗಿ ಅವರೆಲ್ಲರೂ ಅಧೋಗತಿಗೆ ಇಳಿದೂ ಇಳಿದೂ ಇನ್ನು ಮುಂದೆ ಇಳಿಯಲಸಾಧ್ಯವಾದ ಸ್ಥಿತಿಗೆ ಬಂದು ಬಿಟ್ಟಿದ್ದಾರೆ. ಏಕೆಂದರೆ ಇನ್ನಾವ ದೇಶದಲ್ಲಿ ಮನುಷ್ಯರು ದನಗಳ ಕೊಟ್ಟಿಗೆಯಲ್ಲಿ ಮಲಗಿ ನಿದ್ರಿಸುತ್ತಾರೆ? ಇದಕ್ಕಾಗಿ ಯಾರನ್ನೂ ದೂಷಿಸಬೇಡಿ. ಅಜ್ಞಾನಿಗಳು ಮಾಡುವ ಅಪರಾಧವನ್ನು ನೀವೂ ಮಾಡಬೇಡಿ. ಕಾರಣವೂ ಕಾರ್ಯವೂ ಎರಡೂ ನಮ್ಮಲ್ಲಿಯೇ ಇವೆ. ದೋಷಕ್ಕೆ ನಾವೇ ಬಾಧ್ಯರು. ಎದ್ದುನಿಂತು ಧೈರ್ಯವಾಗಿ ಆ ದೂಷಣೆಯನ್ನು ಭುಜದ ಮೇಲೆ ಹೊತ್ತುಕೊಳ್ಳಿ. ಸುಮ್ಮನೆ ಇತರರ ಮೇಲೆ ಕೆಸರನ್ನು ಎರಚುತ್ತಾ ಹೋಗಬೇಡಿ. ನಿಮ್ಮಲ್ಲಿರುವ ಕೇಡು ಕಷ್ಟಗಳಿಗೆಲ್ಲ ನೀವು ಮಾತ್ರ ಕಾರಣರು.

ಲಾಹೋರ್​ ನಗರದ ತರುಣರಿರಾ, ಇದನ್ನು ತಿಳಿಯಿರಿ, ಈ ವಂಶ\break ಪಾರಂಪರ್ಯವಾಗಿ ಬಂದ ರಾಷ್ಟ್ರೀಯ ಮಹಾಪಾಪ ನಿಮ್ಮ ತಲೆಯ ಮೇಲಿದೆ ಎಂಬುದನ್ನು ತಿಳಿಯಿರಿ. ನಮಗೆ ಭರವಸೆಯೇ ಇಲ್ಲವಾಗಿದೆ. ನೀವು ಸಾವಿರಾರು ಸುಧಾರಕ ಸಂಘಗಳನ್ನು ನಿರ್ಮಿಸಬಹುದು, ಸಾವಿರಾರು ರಾಜಕೀಯ ಸಭೆಗಳನ್ನು ಏರ್ಪಡಿಸಬಹುದು, ಐವತ್ತು ಸಾವಿರ ಸಂಸ್ಥೆಗಳನ್ನು ಕಟ್ಟಬಹುದು. ಆದರೆ ಆ ಸಹಾನುಭೂತಿ, ಆ ಪ್ರೀತಿ, ಎಲ್ಲರಿಗಾಗಿಯೂ ಮಿಡಿಯುವ ಆ ಹೃದಯ ಇಲ್ಲದಿದ್ದರೆ ಇವುಗಳಿಂದ ಯಾವ ಪ್ರಯೋಜನವೂ ಇಲ್ಲ. ಎಲ್ಲಿಯವರೆಗೆ ಬುದ್ಧನ ಹೃದಯವು ಮತ್ತೆ ಭರತಖಂಡದಲ್ಲಿ ಕಾಣಿಸಿಕೊಳ್ಳುವುದಿಲ್ಲವೋ ಶ‍್ರೀಕೃಷ್ಣನ ಬೋಧನೆಯು ಅನುಷ್ಠಾನಕ್ಕೆ ಬರುವುದಿಲ್ಲವೊ ಅಲ್ಲಿಯವರೆಗೆ ನಮಗೆ ಭರವಸೆಯಿಲ್ಲ. ನೀವು ಸುಮ್ಮನೆ ಐರೋಪ್ಯರನ್ನು ಅನುಕರಿಸಿ, ಸಂಸ್ಥೆಗಳನ್ನು ಕಟ್ಟಿದರೆ ಏನಾಗುತ್ತದೆ? ನಾನು ಕಣ್ಣಾರೆ ಕಂಡ ಒಂದು ಕಥೆಯನ್ನು ಹೇಳುತ್ತೇನೆ. ಕೆಲವು ಯುರೇಸಿಯನರು ಬರ್ಮೀಯರ ಒಂದು ತಂಡವನ್ನು ಲಂಡನ್ನಿಗೆ ಕರೆದೊಯ್ದರು. ಆ ಜನರನ್ನು ಪ್ರದರ್ಶಿಸಿ ಬೇಕಾದಷ್ಟು ಹಣ ಸಂಪಾದನೆ ಮಾಡಿಕೊಂಡು, ಅಲ್ಲಿಂದ ಅವರನ್ನು ಫ್ರಾನ್ಸ್, ಜರ್ಮನಿಗಳಿಗೂ ಕರೆದುಕೊಂಡು ಹೋಗಿ ಹಣ ಸಂಪಾದನೆ ಮಾಡಿ, ಅವರನ್ನು ಹಾಗೆಯೇ ಬಿಟ್ಟುಬಿಟ್ಟರು. ಆ ಬಡಜನರಿಗೆ ಯುರೋಪಿನ ಯಾವ ಭಾಷೆಯ ಒಂದು ಅಕ್ಷರವೂ ತಿಳಿಯದು. ಆದರೆ ಆಸ್ಟ್ರಿಯದಲ್ಲಿದ್ದ ಆಂಗ್ಲ ರಾಯಭಾರಿಗಳು ಅವರನ್ನು ಲಂಡನ್ನಿಗೆ ಕಳುಹಿಸಿದರು. ಲಂಡನ್ನಿನಲ್ಲಿ ಯಾರನ್ನೂ ಅರಿಯದೆ ಅವರು ಕಂಗೆಟ್ಟರು. ಆದರೆ ಒಬ್ಬ\break ಆಂಗ್ಲ ಮಹಿಳೆಯು ಅವರ ಸ್ಥಿತಿಯನ್ನು ಅರಿತು ಅವರನ್ನು ತನ್ನ ಮನೆಯಲ್ಲಿ\break ಇರಿಸಿಕೊಂಡು ಊಟ ಬಟ್ಟೆಗಳನ್ನು ಕೊಟ್ಟು ಸುದ್ದಿಯನ್ನು ವರ್ತಮಾನ ಪತ್ರಿಕೆಗಳಿಗೆ ಕಳುಹಿಸಿದಳು. ಏನಾಯಿತು ಬಲ್ಲಿರಾ? ಮರುದಿನವೇ ಇಡೀ ಇಂಗ್ಲಿಷ್​ ದೇಶವೇ ಜಾಗ್ರತಗೊಂಡಿತು. ಬೇಕಾದಷ್ಟು ದುಡ್ಡು ಸೇರಿತು, ಆ ಜನರಿಗೆ ಎಲ್ಲ ಸಹಾಯವನ್ನು ನೀಡಿ, ಅವರನ್ನು ಬರ್ಮಕ್ಕೆ ಹಿಂದಿರುಗಿ ಕಳುಹಿಸಲಾಯಿತು. ಪಾಶ್ಚಾತ್ಯ ದೇಶಗಳ ರಾಜಕೀಯ ಮತ್ತು ಇತರ ಸಂಸ್ಥೆಗಳೆಲ್ಲವೂ ಇಂತಹ ಸಹಾನುಭೂತಿಯ ಆಧಾರದ ಮೇಲೆಯೇ ನಿಂತಿವೆ. ಅವರು ತಮ್ಮನ್ನೇ ಪ್ರೀತಿಸುವು\-ದಾದರೂ ಕೂಡ ಆ ಪ್ರೀತಿಯೇ ಅವುಗಳ ಸುಭದ್ರ ತಳಹದಿ. ಅವರು ಲೋಕವನ್ನೆಲ್ಲಾ ಪ್ರೀತಿಸದಿರಬಹುದು. ಬರ್ಮೀಯರು ಅವರ ಶತ್ರುಗಳೇ ಆಗಿರಬಹುದು. ಆದರೆ ಇಂಗ್ಲೆಂಡಿನಲ್ಲಿ ಜನರು ತಮ್ಮವರನ್ನೂ ಸತ್ಯ ಮತ್ತು ನ್ಯಾಯವನ್ನು ಅತ್ಯಂತವಾಗಿ ಪ್ರೀತಿಸುತ್ತಾರೆ, ಹಾಗೂ ತಮ್ಮಲ್ಲಿಗೆ ಬಂದ ಪರಕೀಯರನ್ನು ಆದರದಿಂದ ಕಾಣುತ್ತಾರೆ. ಪ್ರತಿಯೊಂದು ಪಾಶ್ಚಾತ್ಯ ದೇಶದಲ್ಲಿಯೂ ನನ್ನನ್ನು ಎಷ್ಟು ಆದರದಿಂದ ಸತ್ಕಾರ ಮಾಡಿದ್ದಾರೆ ಎಂಬುದನ್ನು ನಾನಿಲ್ಲಿ ಹೇಳದೆ ಹೋದರೆ ಮಹಾ ಕೃತಘ್ನನಾಗುತ್ತೇನೆ. ಅಂತಹ ಹೃದಯ ಇಲ್ಲಿ ಎಲ್ಲಿದೆ? ನಾಲ್ಕು ಜನ ಸೇರಿ ಒಂದು ಸಂಘವನ್ನು ಮಾಡಿಕೊಳ್ಳುವುದು ತಡ, ಒಬ್ಬರನ್ನೊಬ್ಬರು ವಂಚಿಸಲು ಪ್ರಯತ್ನಿಸುತ್ತೀರಿ, ಸಂಘ ಒಡೆದು ನುಚ್ಚು ನೂರಾಗುತ್ತದೆ. ಪಾಶ್ಚಾತ್ಯರನ್ನು ಅನುಕರಿಸಿ, ಅವರಂತೆಯೇ ದೊಡ್ಡ ರಾಷ್ಟ್ರವನ್ನು ನಿರ್ಮಾಣ ಮಾಡಬೇಕು ಎನ್ನುತ್ತೀರಿ. ಆದರೆ ಅವಶ್ಯವಿರುವ ತಳಹದಿ ಎಲ್ಲಿ? ನಮ್ಮ ತಳಹದಿಗಳೆಲ್ಲಾ ಮರಳಿನಿಂದ ಮಾಡಿದವು. ಅದರ ಮೇಲೆ ಕಟ್ಟಿದ ಮಂದಿರಗಳು ಕಣ್ಣುಮುಚ್ಚಿ ಬಿಡುವುದರೊಳಗಾಗಿ ನೆಲಸಮವಾಗುತ್ತವೆ.

ಆದುದರಿಂದ ಲಾಹೋರಿನ ತರುಣರಿರಾ, ಅದ್ವೈತ ವೇದಾಂತದ ಮಹಾ ಧ್ವಜವನ್ನು ಮತ್ತೊಮ್ಮೆ ಗಗನಕ್ಕೆ ಹಿಡಿಯಿರಿ. ಅದರ ಹೊರತು ಎಲ್ಲರಲ್ಲಿಯೂ ಯಾವಾಗಲೂ ಏಕಮಾತ್ರ ಈಶ್ವರನನ್ನು ಸಂದರ್ಶಿಸುವ ಸಮದರ್ಶಿತ್ವವೂ ಪ್ರೀತಿಯೂ ಲಭಿಸುವುದು ಅಸಾಧ್ಯ. ಪ್ರೀತಿಯ ಬಾವುಟವನ್ನು ಬಿಚ್ಚಿಹಿಡಿಯಿರಿ. ‘ಎದ್ದೇಳಿ! ಎಚ್ಚರಗೊಳ್ಳಿ! ಗುರಿಯು ದೊರಕುವವರೆಗೂ ಮುಂದೆ ನುಗ್ಗಿ! ನಿಲ್ಲದಿರಿ!’ ‘ಉತ್ತಿಷ್ಠತ, ಜಾಗೃತ ಪ್ರಾಪ್ಯ ವರಾನ್ನಿಬೋಧತ!” ಎದ್ದೇಳಿ, ಇನ್ನೊಮ್ಮೆ ಎದ್ದೇಳಿ!” ಏಕೆಂದರೆ ತ್ಯಾಗವಿಲ್ಲದೆ ಯಾವುದೂ ಸಾಧ್ಯವಿಲ್ಲ. ಇತರರಿಗೆ ನೀವು ಸಹಾಯ ಮಾಡಬೇಕೆಂದಿದ್ದರೆ ಮೊದಲು ನಿಮ್ಮ ಸ್ವಾರ್ಥ ಹೋಗಬೇಕು. ಹೌದು, ಕ್ರಿಸ್ತನು ಹೇಳುವಂತೆ ದೇವರನ್ನು ಸಂಪತ್ತನ್ನು ಒಟ್ಟಿಗೆ ಒಲಿಸಿಕೊಳ್ಳಲಾರಿರಿ. ಒಟ್ಟಿಗೆ ಪೂಜಿಸಲಾರಿರಿ. ವೈರಾಗ್ಯವನ್ನು ಸಾಧಿಸಿರಿ. ನಿಮ್ಮ ಪೂರ್ವಿಕರು ಮಹಾಕಾರ್ಯಗಳನ್ನು ಸಾಧಿಸಲೋಸುಗ ಐಹಿಕ ಸುಖತ್ಯಾಗವನ್ನು ಮಾಡಿದರು. ಆದರೆ ಈಗ ಜನರು ತಮ್ಮ ಸ್ವಂತ ಮೋಕ್ಷಕ್ಕಾಗಿ ಮಾತ್ರ ಐಹಿಕ ಬಿಡುತ್ತಾರೆ. ಎಲ್ಲವನ್ನೂ ತ್ಯಜಿಸಿಬಿಡಿ. ನಿಮ್ಮ ಸ್ವಂತ ಮೋಕ್ಷವನ್ನು ಬಿಸಾಡಿ. ಹೋಗಿ, ಇತರರ ಸೇವೆಮಾಡಿ. ನೀವು ಯಾವಾಗಲೂ ಧೀರ ವಚನಗಳನ್ನು ಆಡುತ್ತಿರುವಿರಿ. ಈಗ ಈ ಕಾರ್ಯಕಾರಿಯಾದ ವೇದಾಂತ ನಿಮ್ಮ ಮುಂದಿದೆ. ಇತರರ ಕ್ಷೇಮಕ್ಕಾಗಿ ನಿಮ್ಮ ಕ್ಷುದ್ರ ಜೀವನವನ್ನು ತ್ಯಾಗ ಮಾಡಿ. ಹೊಟ್ಟೆಗಿಲ್ಲದೆ ಸತ್ತರೇನಂತೆ! ನಾನು ನೀವೂ ನಿಮ್ಮಂತೆ ಇನ್ನೂ ಸಹಸ್ರ ಜನರೂ ಮಡಿದರೇನಂತೆ! ಈ ಜನಾಂಗ ಉಳಿದರೆ ಸಾಕಲ್ಲವೇ? ದೇಶ ಮುಳುಗುತ್ತಿದೆ. ಅಸಂಖ್ಯ ಲಕ್ಷೋಪಲಕ್ಷ ಜೀವಿಗಳ ಅಭಿಶಾಪವು ನಮ್ಮ ತಲೆಯ ಮೇಲೆ ಬಿದ್ದಿದೆ. ಬಾಯಾರಿಕೆಯಿಂದ ಸಾಯುತ್ತಿದ್ದ ಯಾರಿಗೆ, ಪಕ್ಕದಲ್ಲಿ ಅಮೃತ ವಾಹಿನಿ ಹರಿಯುತ್ತಿದ್ದರೂ, ಬಚ್ಚಲ ನೀರನ್ನು ಕುಡಿಯಲು ಕೊಟ್ಟೆವೋ, ಸಮೃದ್ಧತೆಯ ನಡುವೆಯೇ ಯಾರನ್ನು ಹಸಿವೆಯಿಂದ ನರಳುವಂತೆ ಮಾಡಿದೆವೋ, ಅದ್ವೈತವನ್ನು ಕುರಿತು ಮಾತನಾಡುತ್ತ ಹೃತ್ಪೂರ್ವಕವಾಗಿ ಯಾರನ್ನು ದ್ವೇಷಿಸಿದೆವೋ, ಯಾರ ದಮನಕ್ಕಾಗಿ ಲೋಕಾಚಾರ ಸಿದ್ಧಾಂತವನ್ನು ಕಂಡುಹಿಡಿದೆವೊ, ಯಾರಿಗೆ ಬಾಯಿ ಮಾತಿನಲ್ಲಿ ಮಾತ್ರ "ಎಲ್ಲರೂ ಒಂದೇ, ಎಲ್ಲರಲ್ಲಿಯೂ ಈಶ್ವರನಿದ್ದಾನೆ” ಎಂದು ಹೇಳುತ್ತ ಕಾರ್ಯತಃ ಅದನ್ನು ಒಂದಿನಿತೂ ಅಭ್ಯಾಸ ಮಾಡಲಿಲ್ಲವೋ, ಅಂತಹ ಲಕ್ಷೋಪಲಕ್ಷ ಜನಗಳ ಅಭಿಶಾಪವು ಸಿಡಿಲಿನಂತೆ ನಮ್ಮ ತಲೆಯ ಮೇಲೆರಗುತ್ತಿದೆ. "ಆದರೂ ನನ್ನ ಸ್ನೇಹಿತರೇ, ಅದು ಮನಸ್ಸಿನಲ್ಲಿ ಮಾತ್ರ ಇರಬೇಕು, ಅನುಷ್ಠಾನದಲ್ಲಿ ಇರಕೂಡದು!” – ಈ ಕಳಂಕವನ್ನು ತೊಡೆದು ಹಾಕಿ. “ಉತ್ತಿಷ್ಠತ! ಜಾಗ್ರತ!” ಈ ಸಣ್ಣ ಜೀವ ಹೋದರೇನಂತೆ? ಎಲ್ಲರೂ ಸಾಯಲೇಬೇಕು. ಮಹರ್ಷಿಯಾಗಲೀ, ಪಾಪಿಯಾಗಲೀ, ಶ‍್ರೀಮಂತನಾಗಲೀ, ಯಾರ ದೇಹವೂ ಶಾಶ್ವತವಲ್ಲ. ಎದ್ದೇಳಿ! ಎಚ್ಚರಗೊಳ್ಳಿ! ಸಂಪೂರ್ಣ ಪ್ರಾಮಾಣಿಕರಾಗಿರಿ. ಈ ನಮ್ಮ\break ದೇಶದಲ್ಲಿ ಅಪ್ರಾಮಾಣಿಕತೆ ಭೀಕರವಾಗಿದೆ. ನಮಗೀಗ ಬೇಕಾಗಿರುವುದು ಶೀಲ. ಪ್ರಾಣಹೋದರೂ ಹಿಂಜರಿಯದೆ ಕಾರ್ಯವನ್ನು ಸಾಧಿಸುವ ದೃಢಚಿತ್ತವು ಬೇಕಾಗಿದೆ.

“ನೀತಿ ನಿಪುಣರು ನಿಂದಿಸಲಿ, ಹೊಗಳಲಿ; ಲಕ್ಷ್ಮಿ ಕೈ ಸೇರಲಿ, ಅಥವಾ ಹೊರಟು ಹೋಗಲಿ; ಸಾವು ಈಗಲೇ ಬರಲಿ, ಅಥವಾ ಇನ್ನೊಂದು ನೂರು ವರ್ಷದ ಮೇಲೇ ಬರಲಿ; ದೃಢಮನಸ್ಕನು ಸನ್ಮಾರ್ಗದಿಂದ ಒಂದು ಹೆಜ್ಜೆಯನ್ನು ತಪ್ಪಿ ಇಡುವುದಿಲ್ಲ.” ಎದ್ದೇಳಿ! ಎಚ್ಚರಗೊಳ್ಳಿ! ಸಮಯ ಕಳೆಯುತ್ತಿದೆ. ನಮ್ಮ ಶಕ್ತಿಯೆಲ್ಲ ಬರಿಯ ಕಾಡುಹರಟೆಯಲ್ಲಿ ವ್ಯರ್ಥವಾಗುತ್ತಿದೆ. ಎದ್ದೇಳಿ; ಎಚ್ಚರಗೊಳ್ಳಿ! ಸಾಮಾನ್ಯ ವಿಷಯಗಳನ್ನು ಕುರಿತ ಜಗಳಗಳೂ\break ವಾದಗಳೂ ಕೊನೆಗಾಣಲಿ. ಏಕೆಂದರೆ ನಮ್ಮ ಮುಂದೆ ಒಂದು ಮಹಾ ಕಾರ್ಯವು ಬಂದು ನಿಂತಿದೆ – ಮುಳುಗುತ್ತಿರುವ ಲಕ್ಷಾಂತರ ಜನರನ್ನು ಮೇಲೆತ್ತಿ ಉದ್ಧಾರಮಾಡುವ ಮಹಾ ಸೇವಾ ಕಾರ್ಯ. ಮುಸಲ್ಮಾನರು ಈ ದೇಶಕ್ಕೆ ಮೊಟ್ಟಮೊದಲು ಬಂದಾಗ ಇಲ್ಲಿ ಎಷ್ಟು ಕೋಟಿ ಹಿಂದೂಗಳಿದ್ದರು? ಹೇಗೆ ಅವರ ಸಂಖ್ಯೆ ಕಡಮೆಯಾಗುತ್ತಿದೆ ಎಂಬುದನ್ನು ನೋಡಿ! ದಿನ ಕಳೆದ ಹಾಗೆಲ್ಲಾ ಅವರು ಕಡಿಮೆಯಾಗಿ ಕಡೆಗೆ ಅವರು ಸಂಪೂರ್ಣ ಇಲ್ಲವಾಗುವ ಕಾಲವೂ ಬರಬಹುದು. ಅವರು ಮಾಯವಾಗಲಿ, ಆದರೆ ಅವರೊಂದಿಗೆ ಅವರಲ್ಲಿರುವ ಮಹಾ ತತ್ತ್ವಗಳು ಖಿಲವಾಗುತ್ತವೆ; ಅವರಲ್ಲಿ ಎಷ್ಟೇ ದೋಷಗಳಿದ್ಧರೂ, ಎಷ್ಟೇ ಆದರ್ಶದಿಂದ ಚ್ಯುತರಾಗಿದ್ದರೂ ಅವರಿನ್ನೂ ಆ ತತ್ತ್ವಗಳ ಪ್ರತಿನಿಧಿಗಳಾಗಿರುವರು. ಅವರೊಂದಿಗೆ ಆಧ್ಯಾತ್ಮಿಕ ಚಿಂತನೆಯ ಚೂಡಾಮಣಿಯಂತಿರುವ ಈ ಅದ್ಭುತ ಅದ್ವೈತವೂ ಮಾಯವಾಗಿಹೋಗುತ್ತದೆ. ಆದುದರಿಂದ ಎದ್ದೇಳಿ! ಎಚ್ಚರಗೊಳ್ಳಿ! ನಿಮ್ಮ ಬಾಹುಗಳನ್ನು ಚಾಚಿ, ಜಗತ್ತಿನ ಆಧ್ಯಾತ್ಮಿಕ ಜೀವನವನ್ನು ರಕ್ಷಿಸಿ. ಮೊತ್ತಮೊದಲು ನಿಮ್ಮ ದೇಶದಲ್ಲಿಯೇ ಈ ಕಾರ್ಯಸಾಧನೆ ಮಾಡಿ. ನಮಗೀಗ ಬೇಕಾಗಿರುವುದು ಆಧ್ಯಾತ್ಮಿಕತೆಗಿಂತಲೂ ಹೆಚ್ಚಾಗಿ ಅದ್ವೈತವನ್ನು ವ್ಯಾವಹಾರಿಕ ಜಗತ್ತಿನಲ್ಲಿ ಕಾರ್ಯ ರೂಪಕ್ಕಿಳಿಸುವುದು. ಮೊದಲು ರೊಟ್ಟಿ ಅನಂತರ ಅಧ್ಯಾತ್ಮ, ಮೊದಲು ಅನ್ನ ಅಮೇಲೆ ಬ್ರಹ್ಮ. ಅನ್ನವಿಲ್ಲದೆ ಸಾಯುವವರಿಗೆ ಅಧ್ಯಾತ್ಮವನ್ನು ತುರುಕುತ್ತಿದ್ದೇವೆ. ನಮ್ಮವರಿಗೆ ಹೊಟ್ಟೆಗಿಲ್ಲದಿದ್ದರೂ ಧರ್ಮವೇನೋ ಅಜೀರ್ಣವಾಗುವಷ್ಟಾಗಿದೆ. ಆದರೆ ಸಿದ್ಧಾಂತಗಳು ಹಸಿವನ್ನು ಪರಿಹರಿಸಲಾರವು. ನಮ್ಮಲ್ಲಿ ಎರಡು ಮಹಾಶಾಪಗಳಿವೆ: ಒಂದು ದೌರ್ಬಲ್ಯ, ಮತ್ತೊಂದು ದ್ವೇಷ, ಹೃದಯ ಶುಷ್ಕತೆ. ನೀವು ಸಾವಿರ ಮತಗಳನ್ನು ಕುರಿತು ಮಾತನಾಡಬಹುದು, ಲಕ್ಷ ಪಂಥಗಳನ್ನು\break ಸೃಜಿಸಬಹುದು. ಆದರೆ ಎಲ್ಲಿಯವರೆಗೆ ನಿಮ್ಮ ಹೃದಯದಲ್ಲಿ ಕನಿಕರ ಇರುವು\-ದಿಲ್ಲವೋ, ಎಲ್ಲಿಯ ವರೆಗೆ ವೇದಗಳು ಬೋಧಿಸುವಂತೆ ಎಲ್ಲರೂ ನಿಮ್ಮ ಆತ್ಮದ ಅಂಶಗಳೆಂದೂ, ಶ‍್ರೀಮಂತರೂ ದರಿದ್ರರೂ ಮಹರ್ಷಿಗಳೂ ಪಾಪಿಗಳೂ ಎಲ್ಲರೂ ಅನಂತ ಬ್ರಹ್ಮದ ಅಂಶಗಳೆಂದು ತಿಳಿಯುವುದಿಲ್ಲವೋ, ಅಲ್ಲಿಯವರೆಗೆ ನಿಮ್ಮ ಸಿದ್ಧಾಂತ ಬೋಧನೆಯೂ ಮತ ಪ್ರಚಾರವೂ ಕೆಲಸಕ್ಕೆ ಬರುವುದಿಲ್ಲ.

ಮಹನೀಯರೇ, ಅದ್ವೈತ ವೇದಾಂತದ ಕೆಲವು ಪರಮಶ್ರೇಷ್ಠ ಭಾವನೆಗಳನ್ನು\break ನಿಮ್ಮೆದುರು ಹೇಳಲು ಪ್ರಯತ್ನಿಸಿದ್ಧೇನೆ. ಅವುಗಳನ್ನು ಕಾರ್ಯತಃ ಪ್ರಯೋಗಿಸುವ ಕಾಲವೀಗ ಬಂದಿದೆ. ಈ ದೇಶದಲ್ಲಿ ಮಾತ್ರವಲ್ಲ; ಎಲ್ಲ ಕಡೆಗಳಲ್ಲಿಯೂ ಅದ್ವೈತವು ಕಾರ್ಯರೂಪಕ್ಕಿಳಿಯಬೇಕು. ಆಧುನಿಕ ವಿಜ್ಞಾನದ ಪ್ರಚಂಡ ಆಘಾತಗಳಿಂದ ದ್ವೈತಮತಗಳ ಮೃತ್ತಿಕಾ ಮಂದಿರಗಳು ಎಲ್ಲೆಲ್ಲಿಯೂ ನುಚ್ಚುನೂರಾಗುತ್ತಿವೆ. ಈ ದೇಶದಲ್ಲಿ ಮಾತ್ರವೇ ಅಲ್ಲ, ದ್ವೈತಿಗಳು ಶಾಸ್ತ್ರ ಪಾಠಗಳಿಗೆ ವಕ್ರವ್ಯಾಖ್ಯಾನಗಳನ್ನು ಹೇಳಲು ಪ್ರಯತ್ನಿಸುತ್ತಿರುವುದು! ಎಳೆದತ್ತ ಹಿಗ್ಗಲು ಶಾಸ್ತ್ರಗಳೇನು ಇಂಡಿಯಾ ರಬ್ಬರೇ! ಇಲ್ಲಿ ಮಾತ್ರವೇ ಅಲ್ಲ ಮತ ಪ್ರಮುಖರು ಸ್ವಸಂರಕ್ಷಣಾರ್ಥವಾಗಿ ವಿವಿಧೋಪಾಯಗಳನ್ನು ಕಂಡು ಹಿಡಿಯುತ್ತಿರುವುದು! ಯೂರೋಪು – ಅಮೆರಿಕಾಗಳಲ್ಲಿಯೂ ಅಂತಹ ಕಾರ್ಯವು ಭರದಿಂದ ಸಾಗುತ್ತಿದೆ. ಈ ಅದ್ವೈತ ವೇದಾಂತವು ಭರತಖಂಡದಿಂದ ಅಲ್ಲಿಗೂ ದಾಳಿಯಿಡಬೇಕು. ಅದಾಗಲೇ ಅಲ್ಲಿ ಕಾಲಿಟ್ಟಿದೆ. ಅದು ಬೆಳೆದು ಪ್ರವರ್ಧಮಾನಕ್ಕೆ ಬಂದು ಅವರ ನಾಗರಿಕತೆಯನ್ನು ರಕ್ಷಿಸಬೇಕಾಗಿದೆ. ಏಕೆಂದರೆ ಈಗ ಪಾಶ್ಚಾತ್ಯ ದೇಶಗಳಲ್ಲಿ ಪ್ರಾಚೀನ ಸಂಪ್ರದಾಯಗಳೆಲ್ಲ ಸಡಿಲವಾಗಿ, ಕಾಂಚನ ಪೂಜೆಯೇ ಆಧಾರವಾಗಿರುವ ನವೀನ ಜೀವನ ರೀತಿಗಳು ಬೇರೂರುತ್ತಿವೆ. ಸ್ಪರ್ಧೆ ಮತ್ತು ಕಾಂಚನವೇ ಪ್ರಧಾನವಾಗಿರುವ ಆಧುನಿಕ ಪದ್ಧತಿಗಿಂತ ಹಳೆಯ ಒರಟಾದ ಮತಪದ್ಧತಿಯೇ ಉತ್ತಮವಾಗಿತ್ತು. ಯಾವ ರಾಷ್ಟ್ರವೇ ಆಗಲೀ, ಅದು ಎಷ್ಟೇ ಬಲಿಷ್ಠವಾಗಿರಲಿ, ಅಂತಹ ತಳಹದಿಯ ಮೇಲೆ ಕಟ್ಟಿದ ನಾಗರೀಕತೆಗಳೆಲ್ಲ ಮಣ್ಣು ಪಾಲಾಗಿವೆ ಎಂಬುದನ್ನು ಜಗತ್ತಿನ ಇತಿಹಾಸ ಸಾರಿ ಹೇಳುತ್ತಿದೆ. ಮೊದಲು ಅಂತಹ ಸಂಸ್ಕೃತಿಯ ಪ್ರವಾಹವು ನಮ್ಮ ದೇಶಕ್ಕೆ ನುಗ್ಗದಂತೆ ತಡೆಯಬೇಕು. ಆದ್ದರಿಂದ ಎಲ್ಲರಿಗೂ ಅದ್ವೈತವನ್ನು ಬೋಧಿಸಿ. ಇದರಿಂದ ಆಧುನಿಕ ವಿಜ್ಞಾನದ ಆಘಾತದಿಂದ ಧರ್ಮವು ರಕ್ಷಿಸಲ್ಪಡುತ್ತದೆ. ಅಷ್ಟೇ ಅಲ್ಲ, ಇತರ ರಾಷ್ಟ್ರಗಳಿಗೂ ನಾವು ಸಹಾಯಮಾಡಬೇಕು, ನಿಮ್ಮ ಚಿಂತನೆಗಳಿಂದ ಯೂರೋಪ್​ ಮತ್ತು ಅಮೇರಿಕಕ್ಕೆ ಸಹಾಯ ಮಾಡಬೇಕು. ಮತ್ತೊಮ್ಮೆ ನಿಮಗೆ\break ನೆನಪು ಮಾಡುತ್ತೇನೆ – ನಮಗೆ ಇಲ್ಲಿ ಬೇಕಾಗಿರುವ ರಚನಾತ್ಮಕ ಕ್ರಿಯೆ.\break ಮೊದಲನೆಯದಾಗಿ ಅಧೋಗತಿ ಗಿಳಿಯುತ್ತಿರುವ ಲಕ್ಷಾಂತರ ಭಾರತೀಯರನ್ನು ಕೈಹಿಡಿದು ಮೇಲೆತ್ತಬೇಕು. ಭಗವಾನ್​ ಶ‍್ರೀ ಕೃಷ್ಣನ ಈ ಅಮರವಾಣಿಯನ್ನು ನೆನಪಿಸಿಕೊಳ್ಳಿ:

\begin{verse}
\textbf{ಇಹೈವ ತೈರ್ಜಿತಃ ಸರ್ಗೋ ಯೇಷಾಂ ಸಾಮ್ಯೇ ಸ್ಥಿತಂ ಮನಃ~।}\\\textbf{ನಿರ್ದೋಷಂ ಹಿ ಸಮಂ ಬ್ರಹ್ಮ ತಸ್ಮಾತ್​ ಬ್ರಹ್ಮಣಿ ತೇ ಸ್ಥಿತಾಃ~॥}
\end{verse}

“ಯಾರ ಮನಸ್ಸು ಸಮತಾಭಾವವನ್ನು ತಾಳುತ್ತದೆಯೋ ಅವರು ಈ ಜೀವನದಲ್ಲಿಯೇ ಸಂಸಾರವನ್ನು ಗೆದ್ದವರಾಗುತ್ತಾರೆ. ಏಕೆಂದರೆ ಈಶ್ವರನು ನಿರ್ಮಲನು ಮತ್ತು ಸಮದರ್ಶಿ. ಆದುದರಿಂದ ಯಾರಲ್ಲಿ ಆ ಗುಣಗಳಿರುತ್ತವೆಯೋ ಅವರು ಈಶ್ವರನಲ್ಲಿ ಆರೂಢರಾಗುತ್ತಾರೆ.” (ಗೀತಾ)

