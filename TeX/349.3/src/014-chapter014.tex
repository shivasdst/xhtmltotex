
\chapter{ನಮ್ಮ ಪ್ರಸ್ತುತ ಕರ್ತವ್ಯ}

\begin{center}
(ಮದ್ರಾಸ್​ ಟ್ರಿಪ್ಲಿಕೇನ್​ ಸಾಹಿತ್ಯ ಸಂಘದಲ್ಲಿ ನೀಡಿದ ಉಪನ್ಯಾಸ)
\end{center}

\vskip -0.5cm

ದಿನ ಸಾಗಿದಂತೆಲ್ಲಾ ಜೀವನ ಸಮಸ್ಯೆ ಗಾಢವಾಗುತ್ತಿದೆ. ವಿಶಾಲವಾಗುತ್ತಿದೆ. ಬಹಳ ಹಿಂದೆಯೇ ವೇದಾಂತ ಸತ್ಯಗಳನ್ನು ಮೊದಲು ಕಂಡುಹಿಡಿದುಕೊಂಡಾಗಲೇ ಸಮಸ್ತ ಜೀವನದ ಅಖಂಡತ್ವವನ್ನು ಸಾರಿದರು. ಇದೇ ವೇದಾಂತದ ಮೂಲ ಮಂತ್ರ ಮತ್ತು ಬೋಧನೆಯ ಸಾರ. ವಿಶ್ವದಲ್ಲಿ ಒಂದು ಪರಮಾಣುವು ಚಲಿಸುವಾಗಲೂ ಜಗತ್ತನ್ನು ತನ್ನೊಂದಿಗೆ ಚಲಿಸುವಂತೆ ಮಾಡದೆ ಇರಲಾರದು. ಇಡಿಯ ವಿಶ್ವವು ಪ್ರಗತಿಮಾರ್ಗವನ್ನು ಹಿಡಿಯದೆ ಪ್ರಗತಿ ಎಂಬುದಿಲ್ಲ. ಯಾವುದೇ ಸಮಸ್ಯೆಯನ್ನೂ ಕೇವಲ ಒಂದು ದೇಶದ ಅಥವಾ ಜನಾಂಗದ ಸಂಕುಚಿತ ದೃಷ್ಟಿಯಿಂದಲೇ ಬಗೆ ಹರಿಸಲಾಗುವುದಿಲ್ಲ ಎಂಬುದು ದಿನ ಕಳೆದಂತೆ ಸ್ಪಷ್ಟವಾಗುತ್ತಿದೆ. ಪ್ರತಿಯೊಂದು ಭಾವನೆಯೂ ವಿಶ್ವವೆಲ್ಲವನ್ನು ವ್ಯಾಪಿಸುವಷ್ಟು ವಿಸ್ತಾರವಾಗಬೇಕು. ಪ್ರತಿಯೊಂದು ಆದರ್ಶವೂ ಇಡೀ ಮಾನವಕೋಟಿಗೆ ವ್ಯಾಪಿಸುವ ಪರಿಯಂತವೂ ಅಷ್ಟೇ ಅಲ್ಲ, ಇಡೀ ಜೀವಕೋಟಿಗೆ ವ್ಯಾಪಿಸುವ ಪರಿಯಂತವೂ ವಿಶಾಲವಾಗುತ್ತಾಹೋಗಬೇಕು. ಈ ವಿಷಯವು, ನಮ್ಮ ದೇಶವು ಕಳೆದ ಕೆಲವು ಶತಮಾನಗಳಿಂದ, ಹಿಂದೆ ಇದ್ದಂತೆ ಇಲ್ಲದೆ ಇರುವುದನ್ನು ವಿವರಿಸುತ್ತದೆ. ನಮ್ಮ ದೇಶದ ಈ ಅವನತಿಗೆ ಮುಖ್ಯ ಕಾರಣಗಳೆಂದರೆ ನಮ್ಮ ಸಂಕುಚಿತ ದೃಷ್ಟಿ ಮತ್ತು ಸಂಕುಚಿತ ಕರ್ತವ್ಯಕ್ಷೇತ್ರ ಎಂಬುದು ಕಂಡುಬರುತ್ತದೆ.

ಜಗತ್ತಿನಲ್ಲಿ ಎರಡು ಆಶ್ಚರ್ಯಕರವಾದ ರಾಷ್ಟ್ರಗಳು ಒಂದೇ ಜನಾಂಗದಿಂದ ಕವಲೊಡೆದರೂ ಬೇರೆ ಬೇರೆ ವಾತಾವರಣದಲ್ಲಿ ಅಭಿವೃದ್ಧಿ ಹೊಂದಿದವು. ಜೀವನದ ಸಮಸ್ಯೆಗಳಿಗೆ ತಮ್ಮದೇ ಆದ ಬೇರೆ ಬೇರೆ ಉತ್ತರಗಳನ್ನು ಅವು ಕಂಡುಹಿಡಿದುಕೊಂಡವು. ಅವು ಪ್ರಾಚೀನ ಹಿಂದೂದೇಶ ಮತ್ತು ಪ್ರಾಚೀನ ಗ್ರೀಸ್​ ದೇಶ. ಭಾರತೀಯ ಆರ್ಯನ ಉತ್ತರ ದಿಕ್ಕಿಗೆ ಹಿಮಾವೃತ ಪರ್ವತ ಶ್ರೇಣಿಗಳಿವೆ. ಕೆಳಗೆ ಪ್ರಸ್ಥಭೂಮಿಯಲ್ಲಿ, ಹರಿಯುವ ಸಮುದ್ರದಂತಿರುವ ತಿಳಿನೀರಿನ ವಿಶಾಲ ನದಿಗಳಿವೆ. ಇದೇ ಪೃಥ್ವಿಯ ತುದಿಯೋ ಎಂಬಂತೆ ಕಾಣುವ ಅನಂತ ಅರಣ್ಯ ರಾಶಿಯಿದೆ. ಇವು ಭಾರತೀಯ ಆರ್ಯನ ದೃಷ್ಟಿಯು ಅಂತರ್ಮುಖವಾಗುವಂತೆ ಮಾಡಿದವು. ಸುತ್ತಲೂ ಮನೋಹರವಾದ ಪ್ರಕೃತಿಯ ದೃಶ್ಯಗಳು ಇದ್ದುದರಿಂದ ಸೂಕ್ಷ್ಮ ಬುದ್ಧಿಶಾಲಿ ಆರ್ಯನು ಸ್ವಭಾವತಃ ಅಂತರ್ಮುಖಿಯಾದನು. ತಮ್ಮದೇ ಮನಸ್ಸಿನ ವಿಶ್ಲೇಷಣೆಯು ಭಾರತದ ಆರ್ಯರಿಗೆ, ಹೀಗೆ ಅತಿಮುಖ್ಯವಾಯಿತು. ಗ್ರೀಕರು ಪ್ರಪಂಚದ ಕೊನೆಯಂತೆ ಕಾಣುವ ಗ್ರೀಕ್​ ದ್ವೀಪಸ್ತೋಮಕ್ಕೆ ಬಂದರು. ಅವು ಸುಂದರವಾಗಿದ್ದವು. ಆದರೆ ಭವ್ಯವಾಗಿರಲಿಲ್ಲ. ಅಲ್ಲಿ ಪ್ರಕೃತಿಯು ಉದಾರವಾಗಿತ್ತು. ಆದರೆ ನಿರಾಭರಣವಾಗಿತ್ತು. ಹೀಗಾಗಿ ಗ್ರೀಕರ ಮನಸ್ಸು ಸ್ವಾಭಾವಿಕವಾಗಿ ಬಹಿರ್ಮುಖವಾಯಿತು. ಅದು ಬಾಹ್ಯಜಗತ್ತನ್ನು ವಿಶ್ಲೇಷಿಸಲು ಬಯಸಿತು. ಇದರ ಪರಿಣಾಮವಾಗಿಯೇ ಅಂತರಿಕ ವಿಶ್ಲೇಷಣಾತ್ಮಕ ವಿಜ್ಞಾನಗಳು \enginline{(analytical sciences)} ಭರತಖಂಡದಲ್ಲಿ ಜನಿಸಿದವು; ಗ್ರೀಸಿನಲ್ಲಿ ಸಾಮಾನ್ಯ ತತ್ತ್ವಗಳನ್ನು ಕಂಡುಹಿಡಿಯುವ ವಿಜ್ಞಾನ ಶಾಖೆಗಳು \enginline{(science of generalisation)} ಜನಿಸಿದವು.

\vskip   4pt

ಹಿಂದೂ ಸೂಕ್ಷ್ಮಮತಿಯು ತನ್ನದೇ ಪಥದಲ್ಲಿ ಮುಂದುವರಿದು ಅದ್ಭುತ ಫಲವನ್ನು ಸಂಪಾದಿಸಿತು. ಇಂದಿಗೂ ಕೂಡ ತರ್ಕಶಕ್ತಿಯಲ್ಲಿ ಅಪಾರ ಬುದ್ಧಿ ಶಕ್ತಿಯಲ್ಲಿ ಹಿಂದೂಗಳನ್ನು ಮೀರಿಸುವವರು ಈ ಜಗತ್ತಿನಲ್ಲಿ ಯಾರೂ ಇಲ್ಲ. ನಮ್ಮ ದೇಶದ ಹುಡುಗರಿಗೂ ಇತರರಿಗೂ ಸ್ಪರ್ಧೆ ಉಂಟಾದಾಗ ನಮ್ಮವರೇ ಯಾವಾಗಲೂ ಗೆಲ್ಲುವರೆಂಬುದು ನಮಗೆಲ್ಲ ಗೊತ್ತಿದೆ. ಮಹಮ್ಮದೀಯರು ಭರತಖಂಡವನ್ನು ಗೆಲ್ಲುವುದಕ್ಕೆ ಬಹುಶಃ ಒಂದೆರಡು ಶತಮಾನಗಳ ಮುಂಚೆ ನಮ್ಮ ರಾಷ್ಟ್ರೀಯ ಶಕ್ತಿಯು ಕುಂಠಿತವಾಯಿತು. ಆಗ ಈ ನಮ್ಮ ವಿಚಾರಶಕ್ತಿಯೂ ಅತಿರೇಕಕ್ಕೆ ಹೋಗಿ ಅಧಃಪತನಕ್ಕೆ ಇಳಿಯಿತು. ಈ ಅಧೋಗತಿಯನ್ನು ಭರತಖಂಡದ ಸಾಹಿತ್ಯ, ಕಲೆ, ಸಂಗೀತ, ವಿಜ್ಞಾನ ಮುಂತಾದ ಎಲ್ಲ ಕ್ಷೇತ್ರಗಳಲ್ಲಿಯೂ ನೋಡುವೆವು. ಕಲೆಯಲ್ಲಿ ಹಿಂದಿನ ವಿಶಾಲ ಭಾವನೆ ಉಳಿಯಲಿಲ್ಲ. ಕಲ್ಪನಾಂಶದಲ್ಲಿ ಭವ್ಯತೆ, ರೂಪಾಂಶದಲ್ಲಿ ಪರಸ್ಪರ ಅನುರೂಪತೆ ಇವು ಮಾಯವಾದವು. ಅದರ ಬದಲು ಆಲಂಕಾರಿಕ ಶೈಲಿಯ ಪ್ರಬಲ ವ್ಯಾಮೋಹವನ್ನು ನೋಡುತ್ತೇವೆ. ಈ ಜನಾಂಗದ ಸ್ವಂತಿಕೆ ನಾಶವಾದಂತೆ ತೋರಿತು. ಚಿತ್ತವನ್ನು ಕಲಕುವಂತಹ ಪುರಾತನ ವೈದಿಕ ಸಂಗೀತದ ಭಾವನೆಗಳು ಉಳಿಯಲಿಲ್ಲ. ಪ್ರತಿಯೊಂದು ಸ್ವರವೂ ಪ್ರತ್ಯೇಕವಾಗಿ ನಿಂತು ಅದ್ಭುತ ಸಾಮರಸ್ಯವನ್ನು ಉಂಟುಮಾಡುವಂತೆ ಇರಲಿಲ್ಲ. ಪ್ರತಿಸ್ವರವೂ ತನ್ನ ವೈಶಿಷ್ಟ್ಯವನ್ನು ಕಳೆದುಕೊಂಡಿತು. ನಮ್ಮ ಆಧುನಿಕ ಸಂಗೀತವು ಹಲವು ರಾಗ ಮತ್ತು ಸ್ವರಗಳ ಕಲಸುಮೇಲೋಗರ. ಸಂಗೀತದಲ್ಲಿ ಅವನತಿಯ ಚಿಹ್ನೆ ಇದು. ನಿಮ್ಮ ಆದರ್ಶಕ್ಕೆ ಸಂಬಂಧಿಸಿದ ಭಾವನೆಗಳಲ್ಲಿಯೂ ಆಲಂಕಾರಿಕ ವರ್ಣನೆಗಳು ಕಂಡುಬರುತ್ತಿವೆಯೋ ಹೊರತು, ಸ್ವಂತಿಕೆಯೆಂಬುದು ಕಂಡುಬರುವುದಿಲ್ಲ. ನಿಮ್ಮ ವಿಶೇಷ ಕ್ಷೇತ್ರವಾದ ಧರ್ಮದಲ್ಲಿ ಕೂಡ ಭಯಾನಕ ಅವನತಿಯನ್ನು ಕಾಣುತ್ತೇವೆ. ನೀರನ್ನು ಎಡಗೈಯಿಂದ ಕುಡಿಯುವುದೇ ಅಥವಾ ಬಲಗೈಯಿಂದಲೇ ಎಂಬಂತಹ ಅತಿ ಮುಖ್ಯ ಪ್ರಶ್ನೆಗಳನ್ನು ಬಗೆಹರಿಸಲು ನೂರಾರು ವರ್ಷ ನಿರತರಾಗಿರುವ ಜನಾಂಗದಿಂದ ನೀವು ಮತ್ತೇನನ್ನು ನಿರೀಕ್ಷಿಸಬಲ್ಲಿರಿ? ದೇಶದ ಮಹಾ ವಿದ್ವಾಂಸರು, ನೂರಾರು ವರ್ಷಗಳಿಂದ ಅಡಿಗೆಮನೆಯ ವಿಷಯವಾಗಿ, ನಾನು ನಿನ್ನನ್ನು ಮುಟ್ಟಬಹುದೇ, ನೀನು ನನ್ನನ್ನು ಮುಟ್ಟಬಹುದೇ, ಮುಟ್ಟಿದರೆ ಪ್ರಾಯಶ್ಚಿತ್ತವೇನು ಎಂಬ ವಿಚಾರವಾಗಿ ಚರ್ಚಿಸುತ್ತಿದ್ದರೆ, ಇಂತಹ ಅವನತಿಗಿಂತ ಮಿಗಿಲಾದುದು ಮತ್ತೇನಿದೆ? ಜೀವನಿಗೆ ಮತ್ತು ಈಶ್ವರನಿಗೆ ಸಂಬಂಧಪಟ್ಟ ವೇದಾಂತದ ಜಗತ್ತಿನಲ್ಲಿಯೇ ಅತ್ಯುದಾತ್ತವಾದ ಸಿದ್ಧಾಂತಗಳ ಅರ್ಧಭಾಗ ಕಾಡುಪಾಲಾಗಿ, ಕೇವಲ ಸಂನ್ಯಾಸಿಗಳ ಸ್ವತ್ತಾಗಿ ಉಳಿದಿದೆ. ದೇಶದಲ್ಲಿ ಉಳಿದವರೆಲ್ಲ ಸ್ಪೃಶ್ಯಾಸ್ಪೃಶ್ಯತೆಯ ವಿಚಾರವಾಗಿ, ಆಹಾರ, ಬಟ್ಟೆಗಳ ವಿಚಾರವಾಗಿ ಚರ್ಚಿಸುತ್ತಾ ಇರುವರು. ಮಹಮ್ಮದೀಯರ ಆಕ್ರಮಣವು ನಿಸ್ಸಂದೇಹವಾಗಿ ಹಲವು ಒಳ್ಳೆಯ ವಿಷಯಗಳನ್ನು, ನಮಗೆ ಕಲಿಸಿತು–ಅತಿ ನೀಚನೂ ಅತ್ಯುಚ್ಚನಾಗಿ ಏನನ್ನಾದರೂ ಕಲಿಸಲು ಸಾಧ್ಯ. ಆದರೂ ಇದರಿಂದ ಜನಾಂಗದಲ್ಲಿ ಶಕ್ತಿಯನ್ನು ಜಾಗೃತ ಗೊಳಿಸಲಾಗಲಿಲ್ಲ.

ಅನಂತರ ಶುಭಕ್ಕೊ, ಅಶುಭಕ್ಕೊ, ಆಂಗ್ಲೇಯರಿಂದ ಭರತಖಂಡದ ಆಕ್ರಮಣ ಆರಂಭವಾಯಿತು. ಪ್ರತಿಯೊಂದು ಆಕ್ರಮಣವೂ ಕೆಟ್ಟದ್ದು, ನಿಜ; ಆಕ್ರಮಣ ಎಂಬುದೇ ಪಾಪ, ವಿದೇಶೀ ಸರ್ಕಾರವಿರುವುದೊಂದು ದೋಷ. ಅದರಲ್ಲಿ ಸಂದೇಹವೇ ಇಲ್ಲ. ಆದರೂ ಕೆಲವು ವೇಳೆ ಅಶುಭದಿಂದಲೂ ಶುಭವಾಗುವುದು. ಆಂಗ್ಲೇಯರ ಆಕ್ರಮಣದ ಶುಭ ಫಲವೇ ಇದು: ಇಂಗ್ಲೆಂಡ್​ ಮಾತ್ರವಲ್ಲ, ಇಡಿಯ ಯೂರೋಪ್​ ತನ್ನ ನಾಗರೀಕತೆಗೆ ಗ್ರೀಸಿಗೆ ಧನ್ಯವಾದವನ್ನು ಅರ್ಪಿಸಬೇಕು. ಯೂರೋಪಿನ ಪ್ರತಿಯೊಂದರಲ್ಲಿಯೂ ವ್ಯಕ್ತವಾಗುತ್ತಿರುವುದು ಗ್ರೀಕ್​ ಸಂಸ್ಕೃತಿ. ಪ್ರತಿಯೊಂದು ಕಟ್ಟಡದ ಮೇಲೆ, ಉಪಯೋಗಿಸುವ ಪ್ರತಿಯೊಂದು ಸಾಮಾನಿನ ಮೇಲೆ ಗ್ರೀಕ್​ ಮುದ್ರೆ ಇದೆ. ಯೂರೋಪಿನ ವಿಜ್ಞಾನ ಮತ್ತು ಕಲೆ ಗ್ರೀಸಿನವಲ್ಲದೆ ಬೇರೆಯಲ್ಲ. ಇಂದು ಭರತಖಂಡದ ನೆಲದ ಮೇಲೆ ಪುರಾತನ ಗ್ರೀಕನ್ನು ಪುರಾತನ ಹಿಂದೂವನ್ನು ಸಂಧಿಸುತ್ತಿರುವನು. ನಿಧಾನವಾಗಿ, ಮೌನವಾಗಿ, ಒಂದು ಪರಿವರ್ತನೆಯಾಗುತ್ತಿದೆ. ನಾವು ಸುತ್ತಲೂ ನೋಡುವ ಉದಾರ ಜೀವನಪ್ರದ ಪುನರುತ್ಥಾನಗಳು ಈ ಎರಡು ಶಕ್ತಿಗಳ ಒಗ್ಗೂಡುವಿಕೆಯ ಪರಿಣಾಮ. ಇಂದು ನಮ್ಮ ಮುಂದೆ ಮಾನವ ಜೀವನದ ಉದಾರ ವಿಶಾಲ ಭಾವನೆಗಳು ಇವೆ. ಮೊದಲು ನಾವು ಭ್ರಾಂತರಾಗಿ ವಿಷಯವನ್ನು ಸಂಕುಚಿತಗೊಳಿಸುವುದಕ್ಕೆ ಪ್ರಯತ್ನಿಸಿದ್ದರೂ, ಇಂದು ಜಾಗ್ರತವಾಗಿರುವ ಉದಾರ ಪ್ರೇರಣೆಗಳು, ಜೀವನದ ವಿಶಾಲ ಭಾವನೆಗಳು ನಮ್ಮ ಹಿಂದಿನ ಶಾಸ್ತ್ರಗಳಲ್ಲಿರುವ ಭಾವನೆಗಳ ವ್ಯಾಖ್ಯಾನ ಮಾತ್ರ; ನಮ್ಮ ಪೂರ್ವಿಕರ ಮೂಲಭೂತ ತತ್ತ್ವಗಳನ್ನು ಪ್ರಖರ ತಾರ್ಕಿಕ ವಿಶ್ಲೇಷಣೆಗೆ ಒಳಪಡಿಸಿದರೆ ನಮಗೆ ದೊರೆಯುವುದು ಈ ಭಾವನೆಗಳೇ. ವಿಶಾಲವಾಗಬೇಕು, ವಿಸ್ತರಿಸಬೇಕು, ಸಂಯೋಜಿಸಬೇಕು, ಸಾರ್ವಭೌಮಿಕವಾಗಬೇಕು – ಇದೇ ನಮ್ಮ ಗುರಿ. ಆದರೆ ನಮ್ಮ ಶಾಸ್ತ್ರೋಪದೇಶಕ್ಕೆ ವಿರುದ್ಧವಾಗಿ ನಾವು ದಿನೇ ದಿನೇ ಸಂಕುಚಿತರಾಗಿ ಇತರರಿಂದ ಬೇರೆಯಾಗುತ್ತಿದ್ದೇವೆ.

ಪ್ರಗತಿಮಾರ್ಗದಲ್ಲಿ ಹಲವು ವಿಘ್ನಗಳಿವೆ ಅದರಲ್ಲಿ ಒಂದು–ಪೃಥ್ವಿಯಲ್ಲಿ ನಮ್ಮ ಸಮಾನ ಯಾರೂ ಇಲ್ಲವೆಂಬ ದುರಾಗ್ರಹಭಾವನೆ. ನಾನು ಹೃತ್ಪೂರ್ವಕವಾಗಿ ಭರತಖಂಡವನ್ನು ಪ್ರೀತಿಸುತ್ತೇನೆ. ನನಗೆ ದೇಶಭಕ್ತಿ ಇದೆ. ನಮ್ಮ ಪೂರ್ವಿಕರ ಮೇಲೆ ಗೌರವವಿದೆ. ಆದರೂ ಇತರ ದೇಶಗಳಿಂದ ಹಲವು ವಿಷಯಗಳನ್ನು ನಾವು ಕಲಿತುಕೊಳ್ಳಬೇಕಾಗಿದೆ ಎಂದು ಹೇಳದೆ ವಿಧಿಯಿಲ್ಲ. ಎಲ್ಲರ ಪದತಳದಲ್ಲಿ ಕುಳಿತು ಕಲಿಯಲು ನಾವು ಸಿದ್ಧರಾಗಿರಬೇಕು. ಇದನ್ನು ಗಮನಿಸಿ. ಪ್ರತಿಯೊಬ್ಬರೂ ನಮಗೆ ಒಂದು ದೊಡ್ಡ ಪಾಠವನ್ನು ಕಲಿಸಬಹುದು. ಶ್ರೇಷ್ಠ ಸ್ಮೃತಿಕಾರನಾದ ಮನು ಹೀಗೆ ಹೇಳಿದ್ದಾನೆ:

\begin{verse}
ಶ್ರದ್ದಧಾನೋ ಶುಭಾಂ ವಿದ್ಯಾಮಾದದೀತಾವರಾದಪಿ~।\\ ಅನ್ತ್ಯಾದಪಿ ಪರಂ ಧರ್ಮಂ ಸ್ತ್ರೀರತ್ನಂ ದುಷ್ಕುಲಾದಪಿ~॥
\end{verse}

“ಶ್ರದ್ಧಾವಂತನಾದವನು ಕನಿಷ್ಠನಿಂದಲೂ ಒಳ್ಳೆಯ ವಿದ್ಯೆಯನ್ನು ಕಲಿಯಬೇಕು,\break ಅಂತ್ಯಜನಿಂದಲೂ ಮುಕ್ತಿಮಾರ್ಗವನ್ನು ಕಲಿಯಬೇಕು. ಸ್ತ್ರೀರತ್ನವು ನೀಚಕುಲದಲ್ಲಿ ಜನಿಸಿದ್ದರೂ ಅವಳನ್ನು ಸ್ವೀಕರಿಸಬೇಕು.” ಮನುಸಂತಾನರಾದ ನಾವು ಅವನ ಆದೇಶವನ್ನು ಅನುಸರಿಸಬೇಕು. ಇಹ ಮತ್ತು ಪರಜನ್ಮಗಳಿಗೆ ಸಂಬಂಧಿಸಿದ ವಿಷಯವನ್ನು ಯಾರಿಂದಲಾದರೂ ಕಲಿಯಲು ಸಿದ್ಧರಾಗಿರಬೇಕು. ಜೊತೆಗೆ ನಾವು ಕೂಡ ಜಗತ್ತಿಗೆ ಮಹತ್ತಾದುದನ್ನು ಬೋಧಿಸಬೇಕಾಗಿದೆ ಎಂಬುದನ್ನು ಮರೆಯಬಾರದು. ಭಾರತದ ಹೊರಗೆ ಇರುವ ದೇಶಗಳ ಸಹಾಯವಿಲ್ಲದೆ ನಾವು ಇರಲಾರೆವು. ನಾವು ಹಾಗೆ ಇರಲು ಸಾಧ್ಯವೆಂದು ತಿಳಿದುದೇ ನಮ್ಮ ಮೂರ್ಖತನ. ಅದಕ್ಕಾಗಿ ಸಾವಿರಾರು ವರುಷದ ಗುಲಾಮಗಿರಿಯ ಶಿಕ್ಷೆಯನ್ನು ಅನುಭವಿಸಿರುವೆವು. ಭಾರತೀಯರ ಅವನತಿಗೆ ಮುಖ್ಯ ಕಾರಣ, ನಾವು ಹೊರಗೆ ಹೋಗಿ ಜಗತ್ತಿನ ಇತರರೊಂದಿಗೆ ನಮ್ಮನ್ನು ತುಲನೆಮಾಡಿ ನೋಡಿಕೊಳ್ಳದೆ ಇದ್ದದ್ದು. ಸುತ್ತಲೂ ಏನಾಗುತ್ತಿದೆ ಎಂಬುದನ್ನು ನಾವು ಗಮನಿಸಲಿಲ್ಲ. ಅದಕ್ಕೆ ತಕ್ಕ ಶಿಕ್ಷೆಯನ್ನು ಅನುಭವಿಸಿರುವೆವು. ಇನ್ನು ಮುಂದಾದರೂ ಹಾಗೆ ಮಾಡದೆ ಇರೋಣ. ಭಾರತೀಯರು ಭಾರತದ ಹೊರಗೆ ಹೋಗಕೂಡದು ಎಂಬ ಮೂರ್ಖ ಭಾವನೆಯಲ್ಲಿ ಹುರುಳಿಲ್ಲ. ಈ ಭಾವನೆಯನ್ನು ತ್ಯಜಿಸಬೇಕು. ನೀವು ಎಷ್ಟು ಹೊರಗೆ ಹೋಗಿ ಇತರ ರಾಷ್ಟ್ರಗಳನ್ನು ನೋಡಿಕೊಂಡು ಬರುವಿರೋ, ಅಷ್ಟೂ ನಿಮಗೆ ಮತ್ತು ನಿಮ್ಮ ದೇಶಕ್ಕೆ ಒಳ್ಳೆಯದು. ನೀವು ಕಳೆದ ನೂರಾರು ವರುಷಗಳಿಂದ ಹೀಗೆ ಮಾಡಿದ್ದರೆ, ನಮ್ಮನ್ನು ಆಳಲು ಬಂದ ಪ್ರತಿಯೊಂದು ರಾಷ್ಟ್ರದ ಪದತಳದಲ್ಲಿಯೂ ನಾವು ಇರು\-ತ್ತಿರಲಿಲ್ಲ. ವಿಕಾಸವೇ ಜೀವನದ ಪ್ರಥಮ ಚಿಹ್ನೆ. ನೀವು ಬಾಳಬೇಕಾದರೆ ವಿಸ್ತಾರಗೊಳ್ಳಬೇಕು. ವಿಕಾಸ ನಿಂತೊಡನೆಯೇ ಮೃತ್ಯು ಸನ್ನಿಹಿತವಾಗುವುದು ಅಪಾಯ ಸನ್ನಿಹಿತವಾಗುವುದು. ನೀವು ಪ್ರಸ್ತಾಪಿಸುತ್ತಿರುವಿರಿ. ನಾನು ಹೋಗಲೇಬೇಕಾಯಿತು. ಏಕೆಂದರೆ ವಿಕಾಸವೇ ರಾಷ್ಟ್ರ ಜೀವನದ ಪ್ರಥಮ ಚಿಹ್ನೆ. ನಮ್ಮ ರಾಷ್ಟ್ರೀಯ ಜಾಗೃತಿ, ಆಂತರಿಕ ವಿಕಾಸ ನನ್ನನ್ನು ಹೊರಗೆ ಎಸೆಯಿತು. ಸಾವಿರಾರು ಜನರು ಹಾಗೆಯೇ ಎಸೆಯಲ್ಪಡುವರು. ಜನಾಂಗ ಜೀವನದ ಪುನರಭ್ಯುದಯದ ಪ್ರಥಮ ಚಿಹ್ನೆ. ಇದರಿಂದ ಮನುಷ್ಯನ ಜ್ಞಾನಸಮಷ್ಟಿಗೆ, ಸಮಗ್ರ ಜಗತ್ತಿನ ಉನ್ನತಿಗೆ, ನಾವು ಏನನ್ನು ಕಾಣಿಕೆಯಾಗಿ ಕೊಡಬೇಕಾಗಿದೆಯೋ ಅದನ್ನು ಕೊಡುತ್ತಿರುವೆವು.

ಇದು ಹೊಸದಲ್ಲ. ಹಿಂದಿನಿಂದಲೂ ಹಿಂದೂಗಳು ಭರತಖಂಡದಲ್ಲೇ ಇದ್ದರು, ಎಂದೂ ಹೊರಗೆ ಹೋಗಲಿಲ್ಲವೆಂದು ಭಾವಿಸಿದರೆ ಅದು ತಪ್ಪು. ನೀವು ಹಾಗೆ ಭಾವಿಸಿದ್ದರೆ, ನೀವು ಪುಸ್ತಕ ಓದಿಲ್ಲ, ಜನಾಂಗದ ಚರಿತ್ರೆಯನ್ನು ಸರಿಯಾಗಿ ತಿಳಿದು ಕೊಂಡಿಲ್ಲವೆಂದು ಅರ್ಥ. ಯಾವುದೇ ಜನಾಂಗವು ಬಾಳಬೇಕಾದರೆ ಏನನ್ನಾದರೂ ನೀಡಬೇಕು. ನೀವು ಜೀವನವನ್ನು ಕೊಟ್ಟರೆ ಜೀವನ ದೊರಕುವುದು. ನೀವು ಸ್ವೀಕರಿಸಿದರೆ, ಅದಕ್ಕೆ ಪ್ರತಿಯಾಗಿ ಇತರರಿಗೆ ಕೊಡಲೇಬೇಕು. ನಾವು ಸಹಸ್ರಾರು ವರ್ಷಗಳಿಂದ ಬಾಳುತ್ತಿರುವೆವು ಎಂಬುದು ನಿಸ್ಸಂದೇಹವಾಗಿದೆ. ಇದರಿಂದ ತಿಳಿದು ಬರುವುದೇನೆಂದರೆ, ಮೂಢರು ಅನ್ಯಥಾ ಭಾವಿಸಿದರೂ, ನಾವು ಜಗತ್ತಿಗೆ ಯಾವಾಗಲೂ ಕೊಡುತ್ತಿದ್ದೆವು ಎಂಬುದು. ಧರ್ಮ, ದರ್ಶನಗಳು, ಅಧ್ಯಾತ್ಮ ಮತ್ತು ವಿವೇಕ ಇವೇ ಜಗತ್ತಿಗೆ ಭಾರತದ ಕೊಡುಗೆ. ಧಾರ್ಮಿಕ ಪ್ರಚಾರಕ್ಕೆ ಮುಂಚೆ ಸೇನೆ ಹೋಗಿ ಮಾರ್ಗ ರಚಿಸಬೇಕಾಗಿಲ್ಲ. ಜ್ಞಾನ ಮತ್ತು ದಾರ್ಶನಿಕ ತತ್ತ್ವ ರಕ್ತ ಪ್ರವಾಹದ ಮೂಲಕ ಚಲಿಸಬೇಕಾಗಿಲ್ಲ. ಜ್ಞಾನ ಮತ್ತು ತತ್ತ್ವಗಳು ನೆತ್ತರು ಸ್ರವಿಸುವ ಮಾನವ ದೇಹದ ಮೇಲೆ ಸಾಗಿಹೋಗಬೇಕಾಗಿಲ್ಲ. ಅವು ಹಿಂಸೆಯ ಮೂಲಕ ಸಂಚರಿಸುವುದಿಲ್ಲ, ಶಾಂತಿ ಮತ್ತು ಪ್ರೇಮದ ರೆಕ್ಕೆಯ ಮೇಲೆ ಬರುವುವು. ಇದರ ರೀತಿಯೇ ಹೀಗೆ. ಆದಕಾರಣವೇ ನಾವು ಕೊಡಬೇಕಾಗಿತ್ತು. ಒಬ್ಬ ತರುಣಿ ಲಂಡನ್ನಿನಲ್ಲಿ ನನ್ನನ್ನು, “ಹಿಂದೂಗಳಾದ ನೀವು ಏನು ಮಾಡಿರುವಿರಿ? ನೀವು ಯಾವ ಒಂದು ರಾಷ್ಟ್ರವನ್ನೂ ಗೆದ್ದಿಲ್ಲ” ಎಂದು ಮೂದಲಿಸಿದಳು. ಧೀರನಾದ, ವೀರನಾದ, ಕ್ಷಾತ್ರ ಗುಣದ ಆಂಗ್ಲೇಯನ ದೃಷ್ಟಿಯಿಂದ ಇದು ಸರಿ. ಪರರಾಷ್ಟ್ರವನ್ನು ಗೆಲ್ಲುವುದೇ ಪರಮ ಶ್ರೇಷ್ಠತೆಯ ಗುರುತು, ಅವನ ಪಾಲಿಗೆ. ಆದರೆ ನಮ್ಮ ದೃಷ್ಟಿ ಅದಕ್ಕೆ ಸಂಪೂರ್ಣ ವಿರುದ್ಧ. ಭರತಖಂಡದ ಹಿರಿಮೆಗೆ ಕಾರಣ, ನಾನು ಪರ್ಯಾಲೋಚಿಸಿದಂತೆ, ನಾವು ಅನ್ಯರನ್ನು ಗೆಲ್ಲದಿದ್ದದ್ದು. ಹಿಂದೂಧರ್ಮವು ಅನ್ಯರನ್ನು ಗೆಲ್ಲುವ ಧರ್ಮವಲ್ಲವೆಂದು ತಿಳಿವಳಿಕೆ ಸಾಲದ ಅನೇಕರು ಅದನ್ನು ಪ್ರತಿದಿನ ದೂರುತ್ತಿರುವರು. ಅವರಿಗೆ ನಮ್ಮ ಧರ್ಮದ ಪರಿಚಯ ಹೆಚ್ಚು ಆಗಬೇಕಾಗಿದೆ ಎಂದು ವ್ಯಸನದಿಂದ ಹೇಳಬೇಕಾಗಿದೆ. ನನ್ನ ದೃಷ್ಟಿಯಲ್ಲಿ ನಮ್ಮ ಧರ್ಮವು ಅನ್ಯಧರ್ಮಗಳಿಗಿಂತ ಹೆಚ್ಚು ಸತ್ಯವೆನ್ನುವುದಕ್ಕೆ ಅದು ಆಕ್ರಮಣಕಾರಿಯಲ್ಲದಿರುವುದೇ ಕಾರಣ. ಅದು ಮತ್ತೊಬ್ಬರನ್ನು ಗೆಲ್ಲಲಿಲ್ಲ, ರಕ್ತಪಾತ ಮಾಡಲಿಲ್ಲ, ಅದರ ಬಾಯಿಯಿಂದ ಯಾವಾಗಲೂ ಆಶೀರ್ವಾದ ಪ್ರೀತಿ, ಶಾಂತಿ, ಸಹಾನುಭೂತಿಯ ವಾಣಿಗಳೇ ಬರುತ್ತಿದ್ದುವು. ಇಲ್ಲಿ ಮಾತ್ರ ಮೊಟ್ಟಮೊದಲಿಗೆ ಅನ್ಯಧರ್ಮಸಹಿಷ್ಣುತೆಯ ಭಾವವು ಬೋಧಿಸಲ್ಪಟ್ಟಿತು. ಇಲ್ಲಿ ಮಾತ್ರ ದಯೆ ಸೌಹಾರ್ದಗಳು ಅನುಷ್ಠಾನದಲ್ಲಿದ್ದುವು. ಉಳಿದ ದೇಶಗಳಲ್ಲೆಲ್ಲಾ ಅದು ಬರಿಯ ಬಾಯಿಮಾತಾಗಿತ್ತು. ಇಲ್ಲಿ ಮಾತ್ರ ಹಿಂದೂವು ಮಹಮ್ಮದೀಯರಾಗಿ ಮಸೀದಿ ಕಟ್ಟುವನು, ಕ್ರೈಸ್ತರಿಗಾಗಿ ಚರ್ಚುಗಳನ್ನು ಕಟ್ಟುವನು.

\vskip   4pt

ನೋಡಿ! ನಮ್ಮ ಸಂದೇಶ ಅನೇಕ ವೇಳೆ ನಿಧಾನವಾಗಿ, ಮೌನವಾಗಿ, ಮತ್ತೊಬ್ಬರಿಗೆ ಕಾಣದಂತೆ ಜಗತ್ತಿಗೆ ಹೋಗಿದೆ. ಭಾರತದೆಲ್ಲಾ ವಿಷಯಗಳಂತೆಯೇ ಇದು. ಶಾಂತಿ ಮೌನಗಳೇ ಭಾರತೀಯ ಚಿಂತನೆಯ ಮುಖ್ಯ ಲಕ್ಷಣಗಳು. ಅದರ ಹಿಂದೆ ಇರುವ ಅದ್ಭುತ ಶಕ್ತಿಯು ಎಂದಿಗೂ ಹಿಂಸೆಯ ಮೂಲಕ ವ್ಯಕ್ತವಾಗಿಲ್ಲ. ಯಾವಾಗಲೂ ಭಾರತೀಯ ಭಾವನೆಯು ಮೌನವಾಗಿ ಮತ್ತೊಬ್ಬರನ್ನು ಸಮ್ಮೋಹನಗೊಳಿಸು ವುದು. ಅನ್ಯರು ನಮ್ಮ ಸಾಹಿತ್ಯದ ಅಧ್ಯಯನಕ್ಕೆ ಪ್ರಯತ್ನಿಸಿದರೆ, ಮೊದಲು ಅವರಿಗೆ ಜುಗುಪ್ಸೆಯುಂಟಾಗುವುದು, ತಕ್ಷಣವೆ ಅವರ ಕುತೂಹಲವನ್ನು ಕೆರಳಿಸುವುದಿಲ್ಲ. ಯೂರೋಪಿಯನರ ರುದ್ರನಾಟಕಗಳನ್ನು ನಮ್ಮ ನಾಟಕಗಳೊಂದಿಗೆ ಹೋಲಿಸಿ ನೋಡಿ. ಅವು ತಾತ್ಕಾಲಿಕವಾಗಿ ನಿಮ್ಮನ್ನು ಕೆರಳಿಸುವಂತಹ ಘಟನೆಗಳಿಂದ ತುಂಬಿವೆ.

\newpage

ಆದರೆ ಓದಿಯಾದ ಮೇಲೆ ಪ್ರತಿಕ್ರಿಯೆ ಬರುವುದು, ಎಲ್ಲಾ ಮಾಯವಾಗುವುದು. ಭಾರತೀಯರ ರುದ್ರನಾಟಕ ಇಂದ್ರಜಾಲಿಕನ ಶಕ್ತಿಯಂತೆ, ಮಂದಗತಿಯಿಂದ ಮೌನ\break ವಾಗಿ ಚಲಿಸುವುದು. ಆದರೆ ಅದನ್ನು ಓದುತ್ತಾ ಹೋದರೆ ಪರವಶರಾಗುವಿರಿ. ನೀವು ಚಲಿಸುವುದಕ್ಕೇ ಆಗುವುದಿಲ್ಲ, ಸ್ತಬ್ಧರಾಗುವಿರಿ. ಯಾರು ನಮ್ಮ ಸಾಹಿತ್ಯವನ್ನು ಓದಲು ಮನಸ್ಸು ಮಾಡಿರುವರೋ ಅವರೆಲ್ಲ ಅದರ ಮೋಹಕತೆಗೆ ಒಳಗಾಗಿರುವರು, ಅವರು ಚಿರಕಾಲ ಅದಕ್ಕೆ ಬದ್ಧರಾಗಿರುವರು. ಯಾರಿಗೂ ಕಾಣದೆ ಬೀಳುವ ಹಿಮಮಣಿ ಅತಿ ಸುಂದರವಾದ ಗುಲಾಬಿಯ ಅಲರನ್ನು ಅರಳಿಸುವಂತೆ ಇದೇ ಜಗತ್ತಿನ ಚಿಂತನೆಗೆ ಭಾರತೀಯ ಭಾವನೆಯ ಕಾಣಿಕೆ. ಅದು ಅಜ್ಞಾತವಾಗಿ ಮೌನವಾಗಿ ಆದರೂ ಮಹಾಶಕ್ತಿಯಿಂದ ಜಗತ್ತಿನ ಮೇಲೆ ಅದ್ಭುತ ಪರಿಣಾಮವನ್ನು ಉಂಟುಮಾಡಿದೆ. ಆದರೆ ಎಂದು ಇದು ಹಾಗೆ ಮಾಡಿತೆಂದು ಯಾರಿಗೂ ಗೊತ್ತಿಲ್ಲ. ಭರತಖಂಡದಲ್ಲಿ ಗ್ರಂಥಕರ್ತರ ಹೆಸರುಗಳನ್ನು ತಿಳಿಯುವುದು ಎಷ್ಟು ಕಷ್ಟ ಎಂದು ನನಗೊಬ್ಬರು ಹೇಳಿದರು. ಅದಕ್ಕೆ ನಾನು “ಭಾರತಿಯರ ರೀತಿಯೇ ಅದು” ಎಂದೆ. ಆಧುನಿಕ ಬರಹಗಾರರಂತೆ ತಮ್ಮ ಪುಸ್ತಕದ ಶೇಕಡ ತೊಂಬತ್ತರಷ್ಟು ಭಾವನೆಗಳನ್ನು ಇತರರಿಂದ ಕದ್ದು, ಮುನ್ನುಡಿಯಲ್ಲಿ ಈ ಭಾವನೆಗಳೆಲ್ಲಾ ನನ್ನದೇ ಎಂದು ಹೇಳುವಂತವರಲ್ಲ ಭಾರತೀಯ ಗ್ರಂಥಕರ್ತೃಗಳು. ಮಾನವನ ಮೇಲೆ ಅದ್ಭುತ ಪರಿಣಾಮವನ್ನು ಉಂಟು ಮಾಡಿದ ಗ್ರಂಥಗಳನ್ನು ಬರೆದವರು ತಮ್ಮ ಹೆಸರನ್ನು ಕೂಡ ಹಾಕದೆ ಸಂತುಷ್ಟರಾಗಿ ಮುಂದೆ ಬರುವವರಿಗೆ ಅದನ್ನು ಬಿಟ್ಟು ಮೌನವಾಗಿ ಮರೆಯಾಗುತ್ತಿದ್ದರು. ನಮ್ಮ ತತ್ತ್ವಜ್ಞರ ಹೆಸರು ಯಾರಿಗೆ ಗೊತ್ತಿದೆ? ಅವೆಲ್ಲವೂ ವ್ಯಾಸ, ಕಪಿಲ, ಮುಂತಾದ ಜಾತಿ ಸೂಚಕ ಹೆಸರಿನಲ್ಲೇ ಪ್ರಕಟವಾಗಿದೆ. ಅವರೆಲ್ಲ ನಿಜವಾಗಿ ಗೀತೆಯನ್ನು ಅನುಸರಿಸಿದರು. “ಕರ್ಮ ಮಾಡುವುದಕ್ಕೆ ಮಾತ್ರ ನಿನಗೆ ಅಧಿಕಾರವಿದೆ, ಅದರ ಫಲಕ್ಕೆ ಅಲ್ಲ” ಎಂಬ ಬೋಧನೆಯನ್ನು ಅಕ್ಷರಶಃ ಅನುಷ್ಠಾನಕ್ಕೆ ತಂದರು.

\vskip   4pt

ಭರತಖಂಡವು ಹೀಗೆ ಜಗತ್ತಿನ ಮೇಲೆ ತನ್ನ ಪ್ರಭಾವವನ್ನು ಬೀರುತ್ತಿರುವುದು. ವ್ಯಾಪಾರದ ಸರಕುಗಳಂತೆ ಚಿಂತನೆಗಳು ಕೂಡ ಯಾರೋ ಆಗಲೇ ಮಾಡಿದ ಮಾರ್ಗದ ಮೂಲಕ ಮಾತ್ರ ಸಂಚರಿಸಬಲ್ಲವು. ಚಿಂತನೆಗಳು ಒಂದು ಸ್ಥಳದಿಂದ ಮತ್ತೊಂದು ಸ್ಥಳಕ್ಕೆ ಸಂಚರಿಸಬೇಕಾದರೂ ಒಂದು ಮಾರ್ಗ ಬೇಕಾಗುತ್ತದೆ. ಜಗತ್ತಿನ ಇತಿಹಾಸದಲ್ಲಿ ಯಾವು\-ದಾದರೂ ಒಂದು ಆಕ್ರಮಣಕಾರಿ ಮಹಾ ರಾಷ್ಟ್ರವು ತಲೆ ಎತ್ತಿ ಜಗತ್ತಿನ ಬೇರೆ ಬೇರೆ ಭಾಗಗಳನ್ನು ಒಂದುಗೂಡಿಸಿದಾಗಲೆಲ್ಲ ಭಾರತೀಯ ಚಿಂತನೆಗಳು ಆ ಮಾರ್ಗದಲ್ಲಿ ಸಂಚರಿಸಿ ಪ್ರತಿ ಜನಾಂಗದ ನಾಡಿಗಳನ್ನೂ ಪ್ರವೇಶಿಸಿದೆ. ಬೌದ್ಧಧರ್ಮವು ಅಸ್ತಿತ್ವಕ್ಕೆ ಬರುವ ಮುಂಚೆಯೇ ಹಿಂದೂಭಾವನೆಗಳು ಪ್ರಪಂಚದಲ್ಲೆಲ್ಲಾ ಹರಡಿದ್ದವು ಎನ್ನುವುದಕ್ಕೆ ಸಾಕಾದಷ್ಟು ಪ್ರಮಾಣಗಳು ಹೆಚ್ಚು ಹೆಚ್ಚು ದೊರಕುತ್ತಿವೆ. ಬೌದ್ಧರಿಗೆ ಮುಂಚೆ ವೇದಾಂತದ ಭಾವನೆಗಳು ಚೈನಾ, ಪರ್ಷಿಯಾ, ಜಾವಾ, ಸುಮಾತ್ರ ಮುಂತಾದ ಕಡೆಗಳಲ್ಲಿ ಹಬ್ಬಿದುವು. ಅನಂತರ ಶಕ್ತಿಶಾಲಿ ಗ್ರೀಸ್​ ಪೂರ್ವರಾಷ್ಟ್ರಗಳನ್ನು ಒಂದು ಸೂತ್ರದಲ್ಲಿ ಬಂಧಿಸಿದಾಗ ಭಾರತೀಯ ಭಾವನೆ ವಿಸ್ತರಿಸಿತು. ಶ್ರೇಷ್ಠ ಸಂಸ್ಕೃತಿ ತನ್ನದೆಂದು ಹೆಮ್ಮೆ ಕೊಚ್ಚಿಕೊಳ್ಳುವ ಕ್ರೈಸ್ತ ಧರ್ಮ ಕೂಡ ಹಲವು ಭಾರತೀಯ ಭಾವನೆಯ ಅಂಶಗಳ ಸಮೂಹ. ಪರಮಶ್ರೇಷ್ಠ ಬೌದ್ಧಧರ್ಮವೂ ಕೂಡ ಹಿಂದೂಧರ್ಮದ ತುಂಟ ಶಿಶುವಿನಂತಿದೆ. ಅದರ ಅಸಮರ್ಪಕ ಅನುಕರಣೆಯೇ ಕ್ರೈಸ್ತಧರ್ಮ. ಹಿಂದಿನ ಪರಿಸ್ಥಿತಿಯೇ ಪುನಃ ತಲೆದೋರುತ್ತಿದೆ. ಇಂಗ್ಲೆಂಡಿನ ಪ್ರಚಂಡಶಕ್ತಿ ಜಗತ್ತಿನ ಬೇರೆ ಬೇರೆ ಭಾಗಗಳನ್ನು ಒಂದುಗೂಡಿಸಿದೆ. ಆಂಗ್ಲೇಯ ಮಾರ್ಗಗಳು ರೋಮನ್​ ಮಾರ್ಗಗಳಂತೆ ಕೇವಲ ಭೂಭಾಗಕ್ಕೆ ಮಾತ್ರ ಸೀಮಿತವಾಗಿಲ್ಲ, ಅವು ಎಲ್ಲ ದಿಕ್ಕುಗಳಿಗೂ ಹರಡಿಕೊಂಡಿವೆ. ಇಂಗ್ಲೆಂಡು, ಸಾಗರಗಳ ಮೇಲೆಯೂ ಸಂಚಾರ ಮಾರ್ಗಗಳನ್ನು ತೆರೆದಿದೆ. ಅವು ಜಗತ್ತಿನ ಪ್ರತಿಯೊಂದು ಭಾಗವನ್ನೂ ಮತ್ತೊಂದು ಭಾಗದೊಂದಿಗೆ ಸೇರಿಸಿವೆ. ಹೊಸ ಸಂದೇಶವಾಹಕನಂತೆ ವಿದ್ಯುತ್​ ಶಕ್ತಿ ಅದ್ಭುತ ಕಾರ್ಯಗಳನ್ನು ಮಾಡುತ್ತಿದೆ. ಈ ಸಮಯದಲ್ಲಿ ಭರತಖಂಡವು ಜಾಗ್ರತವಾಗುವುದನ್ನು ನೋಡುತ್ತಿರುವೆವು. ಜಗತ್ತಿನ ಪ್ರಗತಿಗೆ, ಸಂಸ್ಕೃತಿಗೆ ಭರತಖಂಡ ತನ್ನ ಕಾಣಿಕೆಯನ್ನು ಕೊಡಲು ಈಗ ಸಿದ್ಧವಾಗಿರುವುದು. ಇದರ ಪರಿಣಾಮವಾಗಿಯೇ ಪ್ರಕೃತಿಯ ಬಲಾತ್ಕಾರದಿಂದಲೋ ಎಂಬಂತೆ ನಾನು ಅಮೆರಿಕಾ, ಇಂಗ್ಲೆಂಡ್​ ದೇಶಗಳಿಗೆ ಪ್ರಚಾರ ಕಾರ್ಯಕ್ಕೆ ಹೋದುದು. ಸಮಯ ಸನ್ನಿಹಿತ\-ವಾಗಿದೆ ಎಂಬುದು ಎಲ್ಲರಿಗೂ ಗೋಚರವಾಗಿರ ಬಹುದು. ಶುಭ ಚಿಹ್ನೆಗಳೆಲ್ಲ ಕಾಣುತ್ತಿವೆ, ಭಾರತೀಯರ ದರ್ಶನ ಮತ್ತು ಆಧ್ಯಾತ್ಮಿಕ ಭಾವನೆಗಳು ಮತ್ತೊಮ್ಮೆ ಹೊರಗೆ ಹೋಗಿ ಜಗತ್ತನ್ನು ಗೆಲ್ಲಬೇಕು. ನಮ್ಮ ಸಮಸ್ಯೆ ಬರುಬರುತ್ತ ಬೃಹದಾಕಾರವನ್ನು ತಾಳುತ್ತಿದೆ. ನಾವು ನಮ್ಮ ದೇಶವನ್ನು ಮಾತ್ರ ಜಾಗ್ರತಗೊಳಿಸುವುದಲ್ಲ; ಅದೊಂದು ಸಣ್ಣ ಕೆಲಸ. ನಾನು ಕಲ್ಪನಾ ಜೀವಿ. ಹಿಂದೂ ಜನಾಂಗವು ಇಡೀ ಜಗತ್ತನ್ನು ಗೆಲ್ಲಬೇಕೆಂಬುದೇ ನನ್ನ ಗುರಿ.

\vskip   4pt

ಜಗತ್ತಿನಲ್ಲಿ ಅನ್ಯರಾಷ್ಟ್ರಗಳನ್ನು ಗೆದ್ದ ಎಷ್ಟೋ ಪ್ರಖ್ಯಾತ ಜನಾಂಗಗಳಿವೆ. ನಾವೂ ಪ್ರಖ್ಯಾತ ವಿಜಯಿಗಳೇ ಆಗಿದ್ದೆವು. ನಮ್ಮ ವಿಜಯಚರಿತ್ರೆ ಧಾರ್ಮಿಕ ವಿಜಯ, ತಾತ್ತ್ವಿಕ ವಿಜಯ ಎಂಬುದನ್ನು ಅಶೋಕ ಚಕ್ರವರ್ತಿ ತೋರಿಸಿರುವನು. ಮತ್ತೊಮ್ಮೆ ಭಾರತೀಯರು ಜಗತ್ತನ್ನು ಗೆಲ್ಲಬೇಕು. ಇದೇ ನನ್ನ ಜೀವನದ ಕನಸು. ನನ್ನ ಉಪನ್ಯಾಸವನ್ನು ಕೇಳುವ ನಿಮ್ಮಲ್ಲಿ ಪ್ರತಿಯೊಬ್ಬರೂ ಇದೇ ಕನಸನ್ನು ನಿಮ್ಮ ಮನಸ್ಸಿನಲ್ಲೂ ಕಟ್ಟಿ, ಅದು ಸಿದ್ಧಿಸುವ ತನಕ ನಿಲ್ಲಬೇಡಿ ಎಂದು ಆಶಿಸುವೆನು. ನಿಮ್ಮ ದೇಶದ ಕೆಲಸವನ್ನು ಮೊದಲು ನೋಡಿಕೊಳ್ಳಿ, ಅನಂತರ ಹೊರಗೆ ಹೋಗಿ ಎಂದು ಅನೇಕರು ಹೇಳುತ್ತಾರೆ. ಆದರೆ ನಾನು ಖಂಡಿತವಾಗಿ ಹೇಳುತ್ತೇನೆ, ಇತರರಿಗಾಗಿ ನೀವು ದುಡಿದಾಗ ಬಹಳ ಚೆನ್ನಾಗಿ ಕೆಲಸ ಮಾಡುತ್ತೀರಿ. ಇತರರಿಗಾಗಿ ನೀವು ಕೆಲಸ ಮಾಡಿದಾಗ ನಿಮಗೇ ಮಹದುಪಕಾರಮಾಡಿಕೊಂಡಂತೆ. ಸಮುದ್ರಗಳಾಚೆ ಅನ್ಯ ಭಾಷೆಯಲ್ಲಿ ನೀವು ನಿಮ್ಮ ಭಾವನೆಗಳನ್ನು ಪ್ರಚಾರ ಮಾಡುವುದು ನಿಮ್ಮ ಭಾವನೆಯಿಂದ ಅನ್ಯರಿಗೆ ಜ್ಞಾನವನ್ನು ಕೊಡಲು ಪ್ರಯತ್ನಿಸುವುದು, ನಿಮ್ಮ ದೇಶಕ್ಕೆ ಮಹದುಪಕಾರ ಮಾಡಿದಂತೆ ಎನ್ನುವುದಕ್ಕೆ ಇಂದಿನ ಸಭೆಯೇ ಸಾಕ್ಷಿ. ನಾನು ನನ್ನ ಭಾವನೆಗಳನ್ನು ಭರತಖಂಡಕ್ಕೆ ಮಾತ್ರ ಸೀಮಿತಗೊಳಿಸಿದ್ದರೆ, ಇಂಗ್ಲೆಂಡ್​ ಅಮೆರಿಕಾ ದೇಶಗಳಿಗೆ ಹೋಗಿ ಆದ ಪರಿಣಾಮದಲ್ಲಿ ಕಾಲುಪಾಲು ಕೂಡ ಆಗುತ್ತಿರಲಿಲ್ಲ. ನಮ್ಮ ಮುಂದೆ ಇರುವ ಘನ ಆದರ್ಶವಿದು. ಪ್ರತಿಯೊಬ್ಬರೂ ಅದಕ್ಕೆ ಸಿದ್ಧರಾಗಬೇಕು. ಭರತಖಂಡ ಇಡೀ ಜಗತ್ತನ್ನು ಗೆಲ್ಲಬೇಕು. ಅದಕ್ಕಿಂತ ಕಡಿಮೆ ಅಲ್ಲ. ನಾವೆಲ್ಲರೂ ಅದಕ್ಕೆ ಸಿದ್ಧರಾಗಬೇಕು, ಸರ್ವ ಪ್ರಯತ್ನವನ್ನೂ ಮಾಡಬೇಕು. ಹೊರಗಿನವರು ಬಂದು ನಮ್ಮ ದೇಶವನ್ನು ಸೈನ್ಯದಿಂದ ತುಂಬಿದರೆ\- ಚಿಂತೆ ಇಲ್ಲ. ಹೇ ಭರತಖಂಡವೇ, ಜಾಗೃತವಾಗು! ನಿನ್ನ ಅಧ್ಯಾತ್ಮದಿಂದ ಪ್ರಪಂಚವನ್ನು ಗೆಲ್ಲು. ಮೊದಲೇ ನಮ್ಮ ದೇಶದಲ್ಲಿ ಸಾರಿದಂತೆ, ಪ್ರೇಮವು ದ್ವೇಷವನ್ನು ಗೆಲ್ಲಬೇಕು, ದ್ವೇಷವು ದ್ವೇಷವನ್ನು ಗೆಲ್ಲಲಾರದು. ಜಡವಾದವನ್ನೂ ಅದರ ಪರಿಣಾಮವಾಗಿ ಬರುವ ದುರ್ಗತಿಗಳಾವುದನ್ನೂ ಜಡವಾದವು ಪರಿಹರಿಸಲಾರದು. ಸೈನ್ಯಗಳನ್ನು ಇತರ ಸೈನ್ಯಗಳನ್ನು ಗೆಲ್ಲಲು ಪ್ರಯತ್ನಿಸಿದಾಗಲೆಲ್ಲಾ ಹೋರಾಟವು ಮತ್ತೂ ವೃದ್ಧಿಯಾಗಿ ಮಾನವನನ್ನು ಮೃಗಗಳ ಸಮಾನ ಮಾಡುವುವು. ಅಧ್ಯಾತ್ಮವು ಪಾಶ್ಚಾತ್ಯ ದೇಶಗಳನ್ನು ಗೆಲ್ಲಬೇಕು. ತಾವು ಒಂದು ರಾಷ್ಟ್ರವಾಗಿ ಉಳಿಯಬೇಕಾದರೆ ಅತ್ಯಾವಶ್ಯಕವಾಗಿರುವುದು ಅಧ್ಯಾತ್ಮವೆಂಬುದನ್ನು ಪಾಶ್ಯಾತ್ಯ ದೇಶದವರು ಕ್ರಮೇಣ ಮನಗಾಣುತ್ತಿರುವರು. ಅವರು ಅದಕ್ಕೆ ಕಾಯುತ್ತಿರುವರು. ಉತ್ಸಾಹಿಗಳಾಗಿರುವರು. ಇದನ್ನು ಒದಗಿಸುವವರು ಯಾರು? ಭರತವರ್ಷದ ಮಹಾಪುರುಷರ ಸಂದೇಶವನ್ನು ಜಗತ್ತಿಗೆಲ್ಲ ಸಾರಲು ಸಿದ್ಧರಾಗಿರುವ ವ್ಯಕ್ತಿಗಳೆಲ್ಲಿ? ಈ ಸಂದೇಶವನ್ನು ಪ್ರಪಂಚದ ಮೂಲೆ ಮೂಲೆಗಳಿಗೂ ಪ್ರಚಾರ ಮಾಡುವುದಕ್ಕಾಗಿ ಸರ್ವಸ್ವವನ್ನೂ ತ್ಯಾಗಮಾಡಬಲ್ಲ ವ್ಯಕ್ತಿಗಳೆಲ್ಲಿ? ಸತ್ಯ ಪ್ರಚಾರಕ್ಕೆ ಇಂತಹ ಧೀರ ಪುರುಷರು ಬೇಕಾಗಿದ್ದಾರೆ. ದೇಶದ ಹೊರಗೆ ಹೋಗಿ ವೇದಾಂತದ ಮಹಾ ಸತ್ಯವನ್ನು ಸಾರುವುದಕ್ಕೆ ಇಂತಹ ಧೀರ ಪ್ರಚಾರಕರು ಬೇಕಾಗಿದ್ದಾರೆ. ಜಗತ್ತಿಗೆ ವೇದಾಂತ ಬೇಕಾಗಿದೆ. ಇಲ್ಲದೆ ಇದ್ದರೆ ಜಗತ್ತು ನಾಶವಾಗುವುದು. ಪಾಶ್ಚಾತ್ಯ ಜಗತ್ತು ಒಂದು ಜ್ವಾಲಾಮುಖಿಯ ಮೇಲಿದೆ. ಅದು ನಾಳೆ ಸಿಡಿಯಬಹುದು, ಚೂರುಚೂರಾಗಬಹುದು. ಪ್ರಪಂಚದ ಮೂಲೆಮೂಲೆಗಳನ್ನು ಅವರು ಹುಡುಕಿರುವರು, ಆದರೂ ಶಾಂತಿ ದೊರಕಲಿಲ್ಲ. ಭೋಗದ ಪರಮಾವಧಿಯನ್ನು ಮುಟ್ಟಿರುವರು, ಅದರಲ್ಲಿ ಹುರುಳಿಲ್ಲವೆಂಬುದನ್ನು ಮನಗಂಡಿರುವರು. ಪಾಶ್ಚಾತ್ಯ ದೇಶಗಳ ಅಂತರಾಳಕ್ಕೆ ಭಾರತೀಯ ಭಾವನೆ ಮುಟ್ಟುವಂತೆ ಪ್ರಯತ್ನಿಸುವುದಕ್ಕೆ ಇದೇ ಸಕಾಲ. ಆದ ಕಾರಣವೇ ನನ್ನ ಮದ್ರಾಸಿನ ಯುವಕರೇ, ಇದನ್ನು ಗಮನದಲ್ಲಿಡಿ ಎಂದು ಪ್ರತ್ಯೇಕವಾಗಿ ಹೇಳುತ್ತೇನೆ. ನಾವು ಹೊರಗೆ ಹೋಗಬೇಕು. ನಮ್ಮ ತತ್ತ್ವದಿಂದ ಮತ್ತು ಆಧ್ಯಾತ್ಮ ವಿದ್ಯೆಯಿಂದ ಜಗತ್ತನ್ನು ಗೆಲ್ಲಬೇಕು. ಬೇರೆ ಮಾರ್ಗವೇ ಇಲ್ಲ. ಮಾಡಬೇಕು ಇಲ್ಲವೆ ಮಡಿಯಬೇಕು. ಜಾಗ್ರತವಾದ ಮತ್ತು ತೇಜಸ್ವಿಯಾದ ರಾಷ್ಟ್ರೀಯ ಜೀವನಕ್ಕೆ ಅತ್ಯಾವಶ್ಯಕವಾದುದೆ ಭಾರತೀಯ ಚಿಂತನೆಗಳಿಂದ ಜಗತ್ತನ್ನು ಗೆಲ್ಲುವುದು.

\vskip   4pt

ಆಧ್ಯಾತ್ಮಿಕ ಭಾವನೆಯಿಂದ ಜಗತ್ತನ್ನು ಗೆಲ್ಲುವುದೆಂದರೆ ಜೀವದಾನ ಮಾಡುವಂತಹ ಮಹಾತತ್ತ್ವಗಳನ್ನು ಹೊರಗೆ ಕಳುಹಿಸಬೇಕೆಂಬುದೇ ನನ್ನ ಅಭಿಪ್ರಾಯ. ಶತಮಾನಗಳಿಂದ ನಾವು ಎದೆಗಪ್ಪಿಕೊಂಡಿರುವ ಕೆಲಸಕ್ಕೆ ಬಾರದ ಮೂಢನಂಬಿಕೆಗಳನ್ನು ರವಾನಿಸುವುದಲ್ಲ. ಇಂತಹ ಕಳೆಯನ್ನು ನಮ್ಮ ದೇಶದಿಂದಲೇ ಕಿತ್ತು ಆಚೆಗೆ ಒಗೆಯಬೇಕು. ಅವು ಎಂದೆಂದಿಗೂ ನಾಶವಾಗಲಿ, ಅವೇ ಜನಾಂಗದ ಅಧೋಗತಿಗೆ ಕಾರಣ. ಅವು ನಮ್ಮ ಮಿದುಳನ್ನು ಮಂದ\break ಗೊಳಿಸುವುವು. ಯಾವ ಮಿದುಳು ಉದಾತ್ತ ಉಚ್ಚಭಾವನೆಯನ್ನೇ ಆಲೋಚಿಸಲಾರದೋ, ತನ್ನ ಸ್ವಂತಿಕೆಯನ್ನೆಲ್ಲ ಕಳೆದುಕೊಂಡಿದೆಯೋ, ಶಕ್ತಿಯನ್ನೆಲ್ಲಾ ಕಳೆದುಕೊಂಡಿದೆಯೋ,\break ಧರ್ಮದ ಹೆಸರಿನಲ್ಲಿರುವ ಕೆಲಸಕ್ಕೆ ಬಾರದ ಮೂಢನಂಬಿಕೆಗಳನ್ನೆಲ್ಲಾ ಸ್ವೀಕರಿಸುವುದರ ಮೂಲಕ ತನಗೆ ತಾನೆ ವಿಷವಿಕ್ಕುತ್ತಿದೆಯೋ ಅಂತಹ ಮಿದುಳುಳ್ಳವರ ವಿಷಯದಲ್ಲಿ ನಾವು ಬಹಳ ಜಾಗರೂಕರಾಗಿರಬೇಕು. ಭರತಖಂಡದಲ್ಲಿ ನಮ್ಮ ಕಣ್ಣೆದುರಿಗೆ ಕಾಣುವಂತೆಯೇ ಹಲವು ಅಪಾಯಗಳಿವೆ. ಇತ್ತ ಬಂದರೆ ಹುಲಿ, ಅತ್ತ ಸರಿದರೆ ದರಿ ಎಂಬಂತೆ ಇರುವ ಎರಡು ದೋಷಗಳೇ–ಶುದ್ಧ ಜಡವಾದ ಮತ್ತು ಅದಕ್ಕೆ ವಿರುದ್ಧವಾದ ಅಸಂಬದ್ಧ ಮೂಢನಂಬಿಕೆಗಳು. ಇವುಗಳಿಂದ ನಾವು ಪಾರಾಗಬೇಕು. ಪಾಶ್ಚಾತ್ಯ ಜ್ಞಾನಸುರೆಯನ್ನು ಹೀರಿ ಸರ್ವಜ್ಞ ಎನ್ನುವ ವ್ಯಕ್ತಿ ಒಂದು ಕಡೆ ಇರುವನು. ಅವನು ಹಿಂದಿನ ಋಷಿಗಳನ್ನು ಅಣಕಿಸುವನು. ಅವನ ಪಾಲಿಗೆ ಭಾರತೀಯ ಭಾವನೆಯೆಲ್ಲ ಕೆಲಸಕ್ಕೆ ಬಾರದುದು, ತತ್ತ್ವವು ಕೇವಲ ಹಸುಳೆಗಳ ಹರಟೆ, ಧರ್ಮ ಮೂಢರ ನಂಬಿಕೆಯಾಗಿದೆ. ಮತ್ತೊಂದು ಅತಿರೇಕಕ್ಕೆ ಹೋಗುವ ಕೃತಕ ವಿದ್ಯಾವಂತನಿರುವನು. ಅವನು ಪ್ರತಿಯೊಂದು ಶಕುನವನ್ನೂ ಶಾಸ್ತ್ರೀಯವಾಗಿ ವಿವರಿಸಲು ಯತ್ನಿಸುವನು. ತನ್ನ ವಿಚಿತ್ರ ಜಾತಿಗೆ, ದೇವತೆಗಳಿಗೆ ಹಳ್ಳಿಗೆ ಸೇರಿದ ಪ್ರತಿಯೊಂದು ಮೂಢನಂಬಿಕೆಗೂ ದಾರ್ಶನಿಕ, ತಾತ್ತ್ವಿಕ ಮತ್ತು ಇನ್ನು ಎಂತೆಂತಹದೋ ಬಾಲಿಶವಾದ ವಿವರಣೆ ಕೊಡಲು ಯತ್ನಿಸುವನು. ಅವನಿಗೆ ಪ್ರತಿಯೊಂದು ಗ್ರಾಮದ ಆಚಾರವೂ ಒಂದು ವೇದ ವಾಕ್ಯ. ಅದನ್ನು ಜಾರಿಗೆ ತರುವುದರ ಮೇಲೆಯೇ ನಮ್ಮ ರಾಷ್ಟ್ರೀಯ ಜೀವನ ನಿಂತಿದೆ ಎಂದು ಭಾವಿಸುವನು. ನೀವು ಇದರ ವಿಷಯದಲ್ಲಿ ತುಂಬಾ ಜೋಪಾನವಾಗಿರಬೇಕು. ನೀವೆಲ್ಲಾ ಇಂತಹ ಮೂರ್ಖರಾಗಿರುವುದಕ್ಕಿಂತ ಶುದ್ಧ ನಾಸ್ತಿಕರಾಗಿರುವುದು ಮೇಲು. ಏಕೆಂದರೆ ನಾಸ್ತಿಕ ಜೀವಂತನಾಗಿರುವನು, ಅವನನ್ನು ಉತ್ತಮಪಡಿಸುವುದು ಸಾಧ್ಯ. ಮೂಢನಂಬಿಕೆ ಪ್ರವೇಶಿಸಿದರೆ ತಲೆ ಕೆಡುವುದು, ವ್ಯಕ್ತಿ ಹುಚ್ಚನಾಗುವನು, ಅವನತಿ ಪ್ರಾರಂಭವಾಗುವುದು. ಇವೆರಡರಿಂದ ಪಾರಾಗಿ. ನಮಗೆ ಬೇಕಾಗಿರುವುದು ನಿರ್ಭೀತ ಸಾಹಸಿಗಳು. ನಮಗೆ ಇಂದು ಬೇಕಾಗಿರುವುದು, ರಕ್ತದಲ್ಲಿ ಪುಷ್ಟಿ, ನರಗಳಲ್ಲಿ ಶಕ್ತಿ, ಕಬ್ಬಿಣದಂತಹ ಮಾಂಸಖಂಡಗಳು, ಉಕ್ಕಿನಂತಹ ನರಗಳು, ಕೆಲಸಕ್ಕೆ ಬಾರದ ಜೊಳ್ಳು ಭಾವನೆಗಳಲ್ಲ. ಇವುಗಳನ್ನು ನಿರಾಕರಿಸಿ; ಎಲ್ಲಾ ರಹಸ್ಯಗಳನ್ನು ನಿರಾಕರಿಸಿ. ಧರ್ಮದಲ್ಲಿ ರಹಸ್ಯವಿಲ್ಲ. ವೇದ, ವೇದಾಂತ, ಸಂಹಿತೆ, ಪುರಾಣಗಳಲ್ಲಿ ಏನಾದರೂ ರಹಸ್ಯವಿದೆಯೇ? ಹಿಂದಿನ ಕಾಲದ ಋಷಿಗಳು ತಮ್ಮ ಧರ್ಮಪ್ರಚಾರಕ್ಕೆ ರಹಸ್ಯ ಸಂಸ್ಥೆಗಳನ್ನು ಸ್ಥಾಪಿಸಿದರೇನು? ತಮ್ಮ ಅಮೋಘ ಭಾವನೆಯನ್ನು ಜನರಿಗೆ ಪ್ರಚಾರ ಮಾಡುವುದಕ್ಕೆ ಅವರೇನಾದರೂ ಮಾಯಮಂತ್ರಗಳನ್ನು ಮಾಡಿದರೆಂದು ಎಲ್ಲಿಯಾದರೂ ಹೇಳಿದೆಯೇ? ರಹಸ್ಯಾಚರಣೆ ಮತ್ತು ಮೂಢನಂಬಿಕೆ ಯಾವಾಗಲೂ ದುರ್ಬಲತೆಯ ಚಿಹ್ನೆ, ಅವನತಿಯ ಮತ್ತು ಮೃತ್ಯುವಿನ ಚಿಹ್ನೆ, ಜೋಪಾನವಾಗಿರಿ, ಧೀರರಾಗಿ ನಿಮ್ಮ ಕಾಲಿನ ಮೇಲೆ ನಿಲ್ಲಿ. ಶ್ರೇಷ್ಠ ವಿಷಯಗಳಿವೆ, ಪರಮಾದ್ಭುತ ವಿಷಯಗಳಿವೆ. ನಮ್ಮ ಪ್ರಾಕೃತಿಕ ಭಾವನೆಗಳ ದೃಷ್ಟಿಯಿಂದ ಅವುಗಳನ್ನು ಅತಿ ಪ್ರಾಕೃತವೆಂದು ಕರೆಯಬಹುದು. ಆದರೆ ಅವುಗಳಲ್ಲಿ ಒಂದು ರಹಸ್ಯವೇನಿಲ್ಲ. ಧಾರ್ಮಿಕ ಸತ್ಯ ರಹಸ್ಯವೆಂದಾಗಲಿ, ಅಥವಾ ಹಿಮಾಲಯದ ಮೇಲೆ ಇರುವ ಕೆಲವು ರಹಸ್ಯ ಸಂಸ್ಥೆಗಳಿಗೆ ಆ ಸತ್ಯ ಮೀಸಲು ಎಂದಾಗಲೀ, ನಮ್ಮ ಧರ್ಮದಲ್ಲಿ ಎಂದೂ ಬೋಧಿಸಿಲ್ಲ. ನಾನು ಹಿಮಾಲಯದಲ್ಲಿದ್ದೆ. ನೀವು ಅಲ್ಲಿಗೆ ಹೋಗಿಲ್ಲ. ಇಲ್ಲಿಂದ ನೂರಾರು ಮೈಲುಗಳು ಅಲ್ಲಿಗೆ. ನಾನು ಸಂನ್ಯಾಸಿ. ಕಳೆದ ಹದಿನಾಲ್ಕು ವರುಷಗಳಿಂದಲೂ ಸಂಚರಿಸುತ್ತಿರುವೆನು. ಈ ರಹಸ್ಯಸಂಸ್ಥೆಗಳು ಎಲ್ಲಿಯೂ ಇಲ್ಲ. ಈ ಮೂಢನಂಬಿಕೆಗಳ ಹಿಂದೆ ಓಡಬೇಡಿ. ಅದಕ್ಕಿಂತ ನೀವು ಶುದ್ಧ ನಾಸ್ತಿಕರಾಗುವುದು ನಿಮಗೆ ಮತ್ತು ದೇಶಕ್ಕೆ ಒಳ್ಳೆಯದು. ನಾಸ್ತಿಕತೆಯಲ್ಲಾದರೂ ಸ್ವಲ್ಪ ಶಕ್ತಿ ಇದೆ. ಆದರೆ ಮೂಢನಂಬಿಕೆಯಲ್ಲಿ ಅವನತಿ ಮತ್ತು ಮರಣ ಮಾತ್ರ ಇರುವುದು. ಬಲಾಢ್ಯರು ಇಂತಹ ಮೂಢನಂಬಿಕೆಗಳಲ್ಲಿಯೇ ಕಾಲ ಕಳೆಯುವುದು, ಕೆಲಸಕ್ಕೆ ಬಾರದ ಕುಲಗೆಟ್ಟ ಆಚಾರಗಳನ್ನೆಲ್ಲಾ ವಿವರಿಸುವುದಕ್ಕೆ ಉಪಕಥೆಗಳನ್ನು ಕಲ್ಪಿಸುವುದು, ಮಾನವಕೋಟಿಗೆ ನಾಚಿಕೆಗೇಡು. ಧೈರ್ಯಶಾಲಿಗಳಾಗಿ, ಎಲ್ಲವನ್ನೂ ಆ ರೀತಿ ವಿವರಿಸಲು ಪ್ರಯತ್ನಿಸಬೇಡಿ. ನಿಜವಾಗಿ ನಮ್ಮಲ್ಲಿ ಹಲವು ಮೂಢನಂಬಿಕೆಗಳಿವೆ. ದೇಹದಲ್ಲಿ ಎಷ್ಟೋ ವ್ರಣಗಳಿವೆ; ಅವನ್ನು ಕತ್ತರಿಸಬೇಕು; ನಾಶಮಾಡಬೇಕು. ಅದರಿಂದ ನಮ್ಮ ಸಂಸ್ಕೃತಿ, ಧರ್ಮ, ಅಧ್ಯಾತ್ಮ ನಾಶವಾಗುವುದಿಲ್ಲ. ಧರ್ಮದ ಮುಖ್ಯ ತತ್ತ್ವಗಳೆಲ್ಲ ಸುರಕ್ಷಿತವಾಗಿರುವುವು. ಕೆಲಸಕ್ಕೆ ಬಾರದ ಈ ವ್ರಣಗಳನ್ನು ಎಷ್ಟು ಬೇಗ ತೊಡೆದು ಹಾಕಿದರೆ ಅಷ್ಟು ಒಳ್ಳೆಯದು, ಅಷ್ಟು ಹೆಚ್ಚಾಗಿ ಆಧ್ಯಾತ್ಮಿಕ ತತ್ತ್ವಗಳು ಪ್ರಕಾಶಿಸುವುವು. ಅವುಗಳನ್ನು ಆಶ್ರಯಿಸಿ ನಿಲ್ಲಿ.

ಪ್ರತಿಯೊಂದು ಧರ್ಮವೂ ತಾನೇ ಜಗತ್ತಿನ ವಿಶ್ವಧರ್ಮ ಎಂದು ಹೇಳಿಕೊಳ್ಳುವುದನ್ನು ನೀವು ಕೇಳಿರುವಿರಿ. ಇಂತಹ ಒಂದು ಧರ್ಮ ಎಂದಿಗೂ ಬಹುಶಃ ಇರಲಾರದೆಂದು ಹೇಳುತ್ತೇನೆ. ಒಂದು ವೇಳೆ ಅದು ಸಾಧ್ಯವಾದರೆ ಆ ಸ್ಥಾನ ನಮ್ಮ ಧರ್ಮಕ್ಕೆ ಮಾತ್ರ ಸೇರಿದ್ದು, ಮತ್ತಾವುದಕ್ಕೂ ಅಲ್ಲ. ಉಳಿದ ಧರ್ಮಗಳೆಲ್ಲ ಯಾವುದಾದರೊಂದು ವ್ಯಕ್ತಿಯ ಅಥವಾ ಹಲವು ವ್ಯಕ್ತಿಗಳ ಮೇಲೆ ನಿಂತಿವೆ. ಉಳಿದ ಧರ್ಮಗಳೆಲ್ಲಾ ಅವರು ಚಾರಿತ್ರಿಕ ಎಂದು ಭಾವಿಸುವ ಒಂದು ವ್ಯಕ್ತಿಯ ಜೀವನದ ತಳಹದಿಯ ಮೇಲೆ ನಿಂತಿವೆ. ಯಾವುದನ್ನು ತಮ್ಮ ಧಾರ್ಮಿಕಶಕ್ತಿ ಎಂದು ಭಾವಿಸುವರೋ ಅದೇ ನಿಜವಾಗಿ ದುರ್ಬಲವಾಗಿದೆ. ಏಕೆಂದರೆ ಆ ವ್ಯಕ್ತಿ ಚಾರಿತ್ರಿಕವಲ್ಲ ಎಂದು ಸಪ್ರಮಾಣವಾಗಿ ತೋರಿದರೆ ಇಡೀ ಧರ್ಮಸೌಧ ಕುಸಿದುಹೋಗುವುದು. ಆಗಲೇ ಈ ಧರ್ಮಸಂಸ್ಥಾಪಕರ ಜೀವನದ ಅರ್ಧಭಾಗ ಚೂರು ಚೂರಾಗಿದೆ, ಉಳಿದರ್ಧ ಭಾಗ ತೀವ್ರ ಅನುಮಾನಕ್ಕೆ ಆಸ್ಪದವಾಗಿದೆ. ಯಾವ ಸತ್ಯ ಅವರ ವಾಣಿಯನ್ನು ಅವಲಂಬಿಸಿತ್ತೋ ಅದು ಮಾಯವಾಗುವುದು. ಆದರೆ ನಮ್ಮ ಧರ್ಮದ ಸತ್ಯ ನಮ್ಮಲ್ಲಿ ಎಷ್ಟೋ ವ್ಯಕ್ತಿಗಳಿದ್ದರೂ ಅವರ ಮೇಲೆ ನಿಂತಿಲ್ಲ. ಶ‍್ರೀಕೃಷ್ಣನ ಮಹಿಮೆ ಕೇವಲ ಅವನ ಹೆಸರಿನ ಮೇಲೆ ಇಲ್ಲ, ಅವನೊಬ್ಬ ಪ್ರಖ್ಯಾತ ವೇದಾಂತ ಪ್ರಚಾರಕ ಎನ್ನುವುದರ ಮೇಲಿದೆ. ಹಾಗಿಲ್ಲದೇ ಇದ್ದಿದ್ದರೆ ಬುದ್ಧನ ಹೆಸರು ಮಾಯವಾದಂತೆ ಅವನ ಹೆಸರೂ ಭರತಖಂಡದಿಂದ ಮಾಯವಾಗುತ್ತಿತ್ತು. ಆದ್ದರಿಂದ ನಮ್ಮ ಗೌರವ ಯಾವಾಗಲೂ ತತ್ತ್ವಕ್ಕೆ, ವ್ಯಕ್ತಿಗೆ ಅಲ್ಲ. ವ್ಯಕ್ತಿಗಳು ತತ್ತ್ವದ ಸಾಕಾರ ಮೂರ್ತಿಗಳು, ತತ್ತ್ವಕ್ಕೆ ಉದಾಹರಣೆ, ತತ್ತ್ವವಿದ್ದರೆ ವ್ಯಕ್ತಿಗಳು ಸಹಸ್ರಾರು ಜನರು ಬರುವರು. ತತ್ತ್ವ ಸುರಕ್ಷಿತವಾಗಿದ್ದರೆ ಬುದ್ಧನಂತಹ ವ್ಯಕ್ತಿಗಳು ನೂರಾರು ಸಹಸ್ರಾರು ಮಂದಿ ಹುಟ್ಟುವರು. ಆದರೆ ತತ್ತ್ವ ಮಾಯವಾದರೆ, ಅಳಿಸಿ ಹೋದರೆ, ಇಡಿಯ ಜನಾಂಗ ಯಾವುದಾದರೊಂದು ಚಾರಿತ್ರಿಕ ವ್ಯಕ್ತಿಯ ಮೇಲೆ ನಿಲ್ಲಲು ಯತ್ನಿಸಿದರೆ, ಆ ಧರ್ಮದ ಪಾಡು ಚಿಂತಾಜನಕ, ಆ ಧರ್ಮ ಅಪಾಯದಲ್ಲಿದೆ! ನಮ್ಮ ಧರ್ಮ ಒಂದೇ ಯಾವುದೋ ಒಂದು ಅಥವಾ ಹಲವು ವ್ಯಕ್ತಿಗಳ ಮೇಲೆ ನಿಂತಿಲ್ಲ, ಅದು ತತ್ತ್ವಗಳ ಮೇಲೆ ನಿಂತಿರುವುದು. ಜೊತೆಗೆ ಲಕ್ಷಾಂತರ ವ್ಯಕ್ತಿಗಳಿಗೂ ಅಲ್ಲಿ ಸ್ಥಳವಿದೆ. ವ್ಯಕ್ತಿಗಳಿಗೆ ಇಲ್ಲಿ ಬೇಕಾದಷ್ಟು ಅವಕಾಶವಿದೆ. ಆದರೆ ಪ್ರತಿಯೊಬ್ಬರೂ ಒಂದು ತತ್ತ್ವದ ಉದಾಹರಣೆಯಾಗಿರಬೇಕು; ಇದನ್ನು ಮರೆಯಕೂಡದು. ನಮ್ಮ ಈ ಧಾರ್ಮಿಕ ತತ್ತ್ವಗಳೆಲ್ಲ ಸುರಕ್ಷಿತವಾಗಿವೆ. ಅದರ ಮೇಲೆ ಶತಮಾನಗಳ ಧೂಳು ಕುಳಿತುಕೊಳ್ಳದಂತೆ ಸುರಕ್ಷಿತವಾಗಿಡುವುದು ನಮ್ಮಲ್ಲಿ ಪ್ರತಿಯೊಬ್ಬರ ಕರ್ತವ್ಯ. ಈ ಜನಾಂಗ ಮತ್ತೆ ಮತ್ತೆ ಅವನತಿಗೆ ಒಳಗಾಗಿದ್ದರೂ ವೇದಾಂತ ತತ್ತ್ವಗಳು ಕಾಂತಿಹೀನವಾಗಿಲ್ಲದೆ ಇರುವುದು ಆಶ್ಚರ್ಯ. ಎಷ್ಟೇ ದುಷ್ಟನಾಗಿರಲಿ, ಯಾರೂ ಅವುಗಳನ್ನು ನಿಂದಿಸಲು ಯತ್ನಿಸಿಲ್ಲ. ಜಗತ್ತಿನಲ್ಲಿ ನಮ್ಮ ಶಾಸ್ತ್ರಗಳೇ ಹೆಚ್ಚು ಸುರಕ್ಷಿತವಾಗಿರುವ ಶಾಸ್ತ್ರಗಳು. ಬೇರೆ ಶಾಸ್ತ್ರಗಳಿಗೆ ಹೋಲಿಸಿದರೆ ಇವುಗಳಲ್ಲಿ ಪ್ರಕ್ಷಿಪ್ತ ಭಾಗಗಳು ಕಡಮೆ, ಶಾಸ್ತ್ರಪೀಡನೆಯೂ ಕಡಮೆ, ಮತ್ತು ಇವುಗಳ ಸಾರವೂ ನಾಶವಾಗಿಲ್ಲ. ಇವು ಹಿಂದೆ ಇದ್ದಂತೆಯೇ ಈಗಲೂ ಇರುವುವು, ಹಾಗೂ ಜನರ ಮನಸ್ಸನ್ನು ಆದರ್ಶದೆಡೆಗೆ, ಗುರಿಯೆಡೆಗೆ ಪ್ರಚೋದಿಸುತ್ತಿರುವುವು.

ಈ ಗ್ರಂಥಗಳಿಗೆ ಹಲವು ಭಾಷ್ಯಕಾರರು ಭಾಷ್ಯಗಳನ್ನು ರಚಿಸಿರುವರು. ಶ್ರೇಷ್ಠ ಆಚಾರ್ಯರು ಅವುಗಳನ್ನು ಬೋಧಿಸಿರುವರು, ಮತ್ತು ಅವುಗಳ ಆಧಾರದ ಮೇಲೆ ಹಲವು ಮತಗಳು ಸ್ಥಾಪಿಸಲ್ಪಟ್ಟಿವೆ. ವೇದಗಳಲ್ಲಿರುವ ಹಲವು ಭಾವನೆಗಳು ಪರಸ್ಪರ ವಿರುದ್ಧವಾಗಿರುವಂತೆ ಕಾಣುತ್ತವೆ. ಕೆಲವು ಶಾಸ್ತ್ರವಾಕ್ಯಗಳು ಕೇವಲ ದ್ವೈತಭಾವಕ್ಕೆ ಸೇರಿವೆ, ಮತ್ತೆ ಕೆಲವು ಅದ್ವೈತಕ್ಕೆ ಸೇರಿವೆ. ದ್ವೈತಿಗಳು ಬೇರೆ ಉಪಾಯ ಕಾಣದೆ, ಅದ್ವೈತಪರ ಶ್ರುತಿವಾಕ್ಯಗಳ ಮೇಲೆ ಗದಾಪ್ರಹಾರ ಮಾಡುತ್ತಾರೆ. ದ್ವೈತ ಪ್ರಚಾರಕರು ಮತ್ತು ಪುರೋಹಿತರು ಅದ್ವೈತಕ್ಕೆ ಸಂಬಂಧಿಸಿದ ಶಾಸ್ತ್ರವಾಕ್ಯಗಳನ್ನು ದ್ವೈತದ ಅರ್ಥಬರುವಂತೆ ವಿವರಿಸಲು ಪ್ರಯತ್ನಿಸುವರು. ಹಾಗೆಯೇ ಅದ್ವೈತಿಗಳು ದ್ವೈತಭಾವನೆಯನ್ನು ಕೊಡುವ ಶಾಸ್ತ್ರ ಭಾಗವನ್ನು ಇದೇ ರೀತಿ ಪೀಡಿಸುವರು. ಇದು ವೇದಗಳ ತಪ್ಪಲ್ಲ. ವೇದದಲ್ಲಿ ಇರುವುದೆಲ್ಲಾ ದ್ವೈತವೆಂದು ಸಾರುವುದು ವ್ಯರ್ಥ; ಅದರಂತೆಯೇ ಇರುವುದೆಲ್ಲ ಅದ್ವೈತವೆಂದು ಸಾಧಿಸುವುದೂ ಕೂಡ. ಅವುಗಳಲ್ಲಿ ದ್ವೈತ – ಅದ್ವೈತಗಳೆರಡೂ ಇವೆ. ಹೊಸ ಭಾವನೆಗಳ ಬೆಳಕಿನಲ್ಲಿ ನಾವು ಇಂದು ಅವುಗಳನ್ನು ಚೆನ್ನಾಗಿ ತಿಳಿದುಕೊಳ್ಳುವೆವು. ಪರಮ ಸಿದ್ಧಾಂತಕ್ಕೆ ಒಯ್ಯುವ ಭಿನ್ನ ಭಿನ್ನ ಭಾವನೆಗಳು ಅವು. ಮಾನವ ಮನಸ್ಸಿನ ವಿಕಾಸಕ್ಕೆ, ದ್ವೈತ, ಅದ್ವೈತ ಭಾವನೆಗಳೆರಡೂ ಆವಶ್ಯಕ. ಅದಕ್ಕಾಗಿಯೇ ವೇದ ಅವನ್ನು ಬೋಧಿಸುವುದು.

ಮಾನವ ಕೋಟಿಯ ಮೇಲೆ ಕರುಣೆದೋರಿ, ಮುಕ್ತಿಗೆ ಹಲವು ಮೆಟ್ಟಲುಗಳನ್ನು ವೇದ ತೋರುವುದು. ಅವುಗಳಲ್ಲಿ ಪರಸ್ಪರ ವಿರೋಧವಿಲ್ಲ. ಅವು ಮಕ್ಕಳನ್ನು ಮೋಸಪಡಿಸಲು ಉಪದೇಶಿಸಿದ ವ್ಯರ್ಥಾಲಾಪಗಳಲ್ಲ. ಮಕ್ಕಳಿಗೆ ಮಾತ್ರವಲ್ಲ, ಬೆಳೆದವರಿಗೂ ಅವು ಆವಶ್ಯಕ. ಎಲ್ಲಿಯವರೆಗೆ ನಮಗೆ ದೇಹವಿದೆಯೋ ಎಲ್ಲಿಯವರೆಗೆ ದೇಹವೇ ನಾವು ಎಂಬ ಭ್ರಾಂತಿಗೆ ಒಳಗಾಗಿರುವೆವೋ, ಎಲ್ಲಿಯವರೆಗೆ ನಮಗೆ ಪಂಚೇಂದ್ರಿಯಗಳಿವೆಯೋ, ಎಲ್ಲಿಯವರೆಗೆ ನಾವು ಬಾಹ್ಯ ಪ್ರಪಂಚವನ್ನು ನೋಡುತ್ತಿರುವೆವೋ, ಅಲ್ಲಿಯವರೆಗೆ ನಮಗೆ ಒಬ್ಬ ಸಗುಣದೇವರು ಬೇಕು. ಈ ಭಾವನೆಗಳೆಲ್ಲ ನಮ್ಮಲ್ಲಿ ಇರುವಾಗ ಶ‍್ರೀರಾಮಾನುಜಾಚಾರ್ಯರು ಹೇಳಿರುವಂತೆ, ಜೀವ, ಜಗತ್ತು, ದೇವರಿಗೆ ಸಂಬಂಧಪಟ್ಟ ಭಾವನೆಗಳನ್ನೆಲ್ಲಾ ನಾವು ಸ್ವೀಕರಿಸಬೇಕು. ಒಂದನ್ನು ಸ್ವೀಕರಿಸಿದರೆ ಮೂರನ್ನು ಸ್ವೀಕರಿಸಬೇಕು, ಬೇರೆ ವಿಧಿಯೇ ಇಲ್ಲ. ಎಲ್ಲಿಯವರೆಗೆ ನಾವು ಒಂದು ಬಾಹ್ಯ ಜಗತ್ತನ್ನು ನೋಡುವೆವೋ, ಅಲ್ಲಿಯವರೆಗೂ ಸಗುಣದೇವರನ್ನು ಮತ್ತು ಜೀವವನ್ನು ಅಲ್ಲಗಳೆಯುವುದು ಬರಿಯ ಹುಚ್ಚು ತನ. ಕೆಲವು ವೇಳೆ ಮಹಾವ್ಯಕ್ತಿಗಳು ತಮ್ಮ ಮನಸ್ಸಿನ ಮೇರೆಯನ್ನು ದಾಟಿ, ಪ್ರಕೃತಿಯನ್ನು ಮೀರಿ, ಉಪನಿಷತ್ತು ಸಾರುವಂತೆ – \textbf{“ಯತೋ ವಾಚೋ ನಿವರ್ತಂತೇ ಅಪ್ರಾಪ್ಯ ಮನಸಾ ಸಹ”} –ಯಾವುದರಿಂದ ವಾಕ್ಕುಗಳು (ಅದನ್ನು) ಹೊಂದಲಾರದೆ ಮನಸ್ಸಿನೊಂದಿಗೆ ಹಿಂತಿರುಗುವುವೊ; \textbf{“ನ ತತ್ರ ಚುಕ್ಷುರ್ಗಚ್ಛತಿ, ನ ವಾಗ್​ ಗಚ್ಛತಿ ನೋ ಮನಃ”} —ಅಲ್ಲಿಗೆ ಕಣ್ಣು ಹೋಗಲಾರದು, ಮಾತು ಅದನ್ನು ಮುಟ್ಟಲಾರದು, ಮನಸ್ಸು ಅದನ್ನು ಗ್ರಹಿಸಲಾರದು; \textbf{“ನಾಹಂ ಮನ್ಯೇ ಸುವೇದೇತಿ ನೋ ನ ವೇದೇತಿ ವೇದ ಚ”} —ಅದನ್ನು ಗೊತ್ತಿದೆ ಎಂದೂ ಹೇಳುವುದಕ್ಕೆ ಆಗುವುದಿಲ್ಲ, ಗೊತ್ತಿಲ್ಲವೆಂದೂ ಹೇಳುವುದಕ್ಕೆ ಆಗುವುದಿಲ್ಲ – ಇಂತಹ ಸ್ಥಿತಿಗೆ ಹೋಗಬಹುದು. ಅಲ್ಲಿ ಆತ್ಮ ಎಲ್ಲ ಪರಿಮಿತಿಯನ್ನೂ ಮೀರಿಹೋಗುವುದು. ಆಗ ಮಾತ್ರ “ನಾನು ಮತ್ತು ಪ್ರಪಂಚ ಎರಡೂ ಒಂದೇ,” “ನಾನೇ ಬ್ರಹ್ಮ” ಎಂಬ ಭಾವನೆ ಹೊಳೆಯುವುದು. ಈ ಸಿದ್ಧಾಂತಕ್ಕೆ ಕೇವಲ ಜ್ಞಾನದ ಮತ್ತು ತತ್ತ್ವದ ಮೂಲಕ ಮಾತ್ರ ಬಂದಿರುವುದಲ್ಲ, ಆದರೆ ಅದರ ಕೆಲವು ಅಂಶಗಳನ್ನು ಪ್ರೇಮಶಕ್ತಿಯ ಮೂಲಕವೂ ಮುಟ್ಟಿರುವರು. ಭಾಗವತದಲ್ಲಿ ನಾವು ಇದನ್ನು ಓದುತ್ತೇವೆ: ಕೃಷ್ಣನು ಮಾಯವಾದ ಮೇಲೆ ಗೋಪಿಯರು ವ್ಯಥೆಪಡುವಾಗ, ವಿರಹವ್ಯಥೆ ಮನಸ್ಸಿನಲ್ಲಿ ದಾರುಣವಾಗಿ, ಪ್ರತಿಯೊಬ್ಬರೂ ತಮಗೆ ದೇಹವಿದೆ ಎಂಬುದನ್ನು ಮರೆತು, ತಾವೇ ಕೃಷ್ಣನೆಂದು ಭಾವಿಸಿ ಅವನಂತೆ ಅಲಂಕಾರ ಮಾಡಿಕೊಂಡು ಅವನಂತೆಯೇ ಆಡತೊಡಗಿದರು. ಪ್ರೀತಿಯಿಂದಲೂ ಕೂಡ ಈ ತಾದಾತ್ಮ್ಯಭಾವ ಬರುವುದೆಂದು ನಮಗೆ ಗೊತ್ತಾಗುತ್ತದೆ. ಪರ್ಷಿಯಾದಲ್ಲಿ ಪುರಾತನ ಸೂಫಿ ಕವಿಯೊಬ್ಬನಿದ್ದ. ಅವನ ಒಂದು ಪದ್ಯ ಹೀಗೆ ಇದೆ; ನಾನು ಪ್ರಿಯತಮಳ ಸಮೀಪಕ್ಕೆ ಬಂದೆ. ಅವಳಿರುವ ಕೋಣೆಯ ಬಾಗಿಲು ಹಾಕಿತ್ತು. ನಾನು ಬಾಗಿಲು ತಟ್ಟುವಾಗ ಒಳಗಿನಿಂದ ಒಂದು ಧ್ವನಿ ‘ಯಾರು?’ ಎಂದು ಕೇಳಿತು. ‘ನಾನು’ ಎಂದೆ. ಬಾಗಿಲು ತೆರೆಯಲಿಲ್ಲ. ಎರಡನೆಯ ವೇಳೆ ಬಾಗಿಲನ್ನು ತಟ್ಟಿದೆನು. ಮತ್ತೆ ‘ಯಾರದು’ ಎಂಬ ಧ್ವನಿ ಕೇಳಿಸಿತು. ನಾನು ಇಂತಹವನು ಎಂದೆ. ಆದರೂ ಬಾಗಿಲು ತೆರೆಯಲಿಲ್ಲ. ಮೂರನೆಯ ವೇಳೆ ಬಾಗಿಲು ತಟ್ಟಿದೆನು. ಪುನಃ ‘ಯಾರು’ ಎಂಬ ಧ್ವನಿ ಕೇಳಿಸಿತು. “ಪ್ರಿಯತಮೆ! ನಾನೇ ನೀನು” ಎಂದೆ, ಆಗ ಬಾಗಿಲು ತೆರೆಯಿತು.

\vskip   4pt

ಆದ್ದರಿಂದ ಹಲವು ಅಂತಸ್ತುಗಳಿವೆ. ಅವುಗಳಿಗಾಗಿ ನಾವು ಹೋರಾಡಬೇಕಾಗಿಲ್ಲ. ಹಿಂದಿನ ಭಾಷ್ಯಕಾರರಲ್ಲಿ ಎಷ್ಟೇ ಭಿನ್ನಭಿಪ್ರಾಯವಿದ್ದರೂ, ನಾವು ಅವರನ್ನು ಗೌರವದಿಂದ ಕಾಣಬೇಕು. ಜ್ಞಾನಕ್ಕೆ ಒಂದು ಪರಿಮಿತಿ ಇಲ್ಲ. ಹಿಂದಿನವರಾಗಲೀ ಇಂದಿನವರಾಗಲೀ, ಯಾರೂ ಸರ್ವಜ್ಞರಲ್ಲ. ಹಿಂದಿನ ಕಾಲದಲ್ಲಿ ಋಷಿಗಳು, ತಪಸ್ವಿಗಳು ಇದ್ದರೆ, ಈಗಲೂ ಇರುವರು. ಹಿಂದಿನ ಕಾಲದಲ್ಲಿ ವ್ಯಾಸ, ವಾಲ್ಮೀಕಿ, ಶಂಕರಾಚಾರ್ಯರು ಇದ್ದರೆ ನಿಮ್ಮಲ್ಲಿ ಏತಕ್ಕೆ ಎಲ್ಲರೂ ಶಂಕರಾಚಾರ್ಯರಾಗ ಬಾರದು? ನಮ್ಮ ಧರ್ಮದ ವಿಷಯದಲ್ಲಿ ಯಾವಾಗಲೂ ಜ್ಞಾಪಕದಲ್ಲಿಡಬೇಕಾದ ಮತ್ತೊಂದು ಅಂಶ ಇದು: ಇತರ ಧರ್ಮಶಾಸ್ತ್ರಗಳು ಈಶ್ವರಪ್ರೇರಿತ ಜ್ಞಾನವೇ ಅವಕ್ಕೆ ಪ್ರಮಾಣವೆಂದೂ ಆದರ ಆ ಜ್ಞಾನವು ಎಲ್ಲೋ ಕೆಲವರಿಗೆ ಮಾತ್ರ ಮೀಸಲಾಗಿರುವುದೆಂದೂ ಅವರ ಮೂಲಕ ಸತ್ಯವು ಜನಸಾಮಾನ್ಯರಿಗೆ ಪ್ರಾಪ್ತವಾಗುವುದೆಂದೂ ನಾವುಗಳೆಲ್ಲ ಅದನ್ನು ಒಪ್ಪಬೇಕೆಂದೂ ಕೇಳುತ್ತವೆ. ಸತ್ಯವು ನಜರತ್ತಿನ ಏಸುಕ್ರಿಸ್ತನಿಗೆ ಬಂದುದರಿಂದ ನಾವುಗಳೆಲ್ಲರೂ ಅವನನ್ನು ಅನುಸರಿಸಬೇಕು. ಆದರೆ ಭಾರತದಲ್ಲಿ ಸತ್ಯವು ಮಂತ್ರದ್ರಷ್ಟೃಗಳಾದ ಋಷಿಗಳಿಗೆ ಗೋಚರವಾಯಿತು. ಭವಿಷ್ಯದಲ್ಲಿಯೂ, ಎಲ್ಲಾ ಋಷಿಗಳಿಗೆ ಅದು ಗೋಚರವಾಗುವುದು. ಅದು ಮಾತಾಳಿಗಳಿಗೆ, ಗ್ರಂಥವನ್ನು ಓದುವವರಿಗೆ, ಪಂಡಿತರಿಗೆ, ಭಾಷಾವಿಜ್ಞಾನಿಗಳಿಗೆ ಗೋಚರಿಸುವುದಿಲ್ಲ, ಕೇವಲ ಮಂತ್ರದ್ರಷ್ಟೃಗಳಿಗೆ ಮಾತ್ರ ಗೋಚರಿಸುವುದು.

\begin{verse}
ನಾಯಮಾತ್ಮಾ ಪ್ರವಚನೇನ ಲಭ್ಯೋ~।\\ ನ ಮೇಧಯಾ ನ ಬಹುನಾ ಶ್ರುತೇನ~॥
\end{verse}

“ಪ್ರವಚನದಿಂದ ಆತ್ಮ ಲಭಿಸುವುದಿಲ್ಲ, ಪಾಂಡಿತ್ಯದಿಂದಲೂ ಇಲ್ಲ. ಬಹಳ ಓದುವುದರಿಂದಲೂ ಇಲ್ಲ.” ನಮ್ಮ ಶಾಸ್ತ್ರಗಳೇ ಹೀಗೆ ಹೇಳುತ್ತವೆ. ಶಾಸ್ತ್ರಾಧ್ಯಯನದಿಂದಲೂ ಆತ್ಮನನ್ನು ಪಡೆಯಲಾಗುವುದಿಲ್ಲವೆಂದು ಬೇರೆ ಯಾವುದಾದರೂ ಶಾಸ್ತ್ರವು ಹೇಳುತ್ತದೇನು? ನಿಮ್ಮ ಹೃದಯವನ್ನು ತೆರೆಯಬೇಕು. ಧರ್ಮ ಎಂದರೆ ಕೇವಲ ಚರ್ಚಿಗೆ ಹೋಗುವುದಲ್ಲ, ಹಣೆಯ ಮೇಲೆ ಮತಚಿಹ್ನೆಯನ್ನು ಧರಿಸುವುದಲ್ಲ, ವಿಚಿತ್ರ ರೀತಿಯ ಉಡುಪನ್ನು ಧರಿಸುವುದೂ ಅಲ್ಲ. ಕಾಮನ ಬಿಲ್ಲಿನ ಬಣ್ಣ ಗಳೆಲ್ಲವನ್ನೂ ನೀವು ಬಳಿದುಕೊಳ್ಳಬಹುದು. ನಿಮ್ಮ ಹೃದಯ ತೆರೆಯದೆ ಇದ್ದರೆ, ಅವರನ್ನು ಸಾಕ್ಷಾತ್ಕಾರ ಮಾಡಿಕೊಳ್ಳದೆ ಇದ್ದರೆ, ಎಲ್ಲವೂ ವ್ಯರ್ಥ. ಹೃದಯ ಪರಿಶುದ್ಧವಾಗಿದ್ದರೆ, ಮತ್ತಾವ ಬಣ್ಣವೂ ಬೇಕಾಗಿಲ್ಲ. ಇದೇ ನಿಜವಾದ ಧರ್ಮ ಸಾಕ್ಷಾತ್ಕಾರ. ಎಲ್ಲಿಯವರೆಗೂ ಬಣ್ಣ ಮುಂತಾದವು ನಮಗೆ ಸಹಾಯ ಮಾಡುವುವೋ ಅಲ್ಲಿಯವರೆಗೆ ಅವು ಒಳ್ಳೆಯವು. ಆದರೆ ಅವು ನಮಗೆ ಸಹಾಯ ಮಾಡುವುದರ ಬದಲು, ಪ್ರಗತಿಗೆ ಆತಂಕ ಉಂಟುಮಾಡುವ ಸಂಭವವಿದೆ. ಜನರು ಧರ್ಮವೆಂದರೆ ಕೇವಲ ಬಾಹ್ಯಾಚರಣೆ ಎಂದು ಭಾವಿಸುವರು. ದೇವಸ್ಥಾನಕ್ಕೆ ಹೋಗುವುದೇ ಒಂದು ಧರ್ಮವಾಗಿಬಿಡುವುದು. ಪುರೋಹಿತನಿಗೆ ಏನಾದರೂ ದಾನ ಧರ್ಮ ಮಾಡುವುದೇ ಧರ್ಮವಾಗಿಬಿಡುವುದು. ಇದೇ ಅಪಾಯ. ವಿನಾಶಹೇತು. ಇದನ್ನು ತಕ್ಷಣ ಅಡಗಿಸಬೇಕು. ಬಾಹ್ಯೇಂದ್ರಿಯಗಳ ಜ್ಞಾನಕೂಡ ಧರ್ಮವಲ್ಲವೆಂದು ನಮ್ಮ ಶಾಸ್ತ್ರ ಸಾರುವುದು. ಅವಿಕಾರಿಯಾದುದನ್ನು ಸಾಕ್ಷಾತ್ಕಾರ ಮಾಡಿಕೊಳ್ಳುವುದೇ ಧರ್ಮ. ಅದೇ ಎಲ್ಲರಿಗೂ ಧರ್ಮ. ಯಾರು ಪ್ರಜ್ಞಾತೀತ ಸತ್ಯವನ್ನು ಸಾಕ್ಷಾತ್ಕಾರ ಮಾಡಿಕೊಂಡಿರುವರೋ, ಯಾರು ತಮ್ಮಲ್ಲಿ ಆತ್ಮ ಸಾಕ್ಷಾತ್ಕಾರ ಮಾಡಿಕೊಂಡಿರುವರೊ, ಯಾರು ದೇವರನ್ನು ಸ್ವತಃ ಕಂಡಿರುವರೋ, ಪ್ರತಿಯೊಂದರಲ್ಲಿಯೂ ದೇವರನ್ನು ಮಾತ್ರ ನೋಡುವರೊ, ಅವರೇ ಋಷಿಗಳು. ಋಷಿಗಳಾಗುವವರೆಗೆ ನಿಮಗೆ ಧಾರ್ಮಿಕ ಜೀವನವಿಲ್ಲ. ಆಗ ಮಾತ್ರ ನಿಮಗೆ ನಿಜವಾದ ಧರ್ಮ ಪ್ರಾರಂಭವಾಗುವುದು. ಈಗ ಅದಕ್ಕೆ ಸಿದ್ಧತೆ ಮಾತ್ರ. ಆಗ ಧರ್ಮೋದಯವಾಗುವುದು. ಈಗ ಕೇವಲ ಪಾಂಡಿತ್ಯದ ಕಸರತ್ತು ಮತ್ತು ಶರೀರದಂಡನೆ ಮಾತ್ರ ಆಗುತ್ತಿದೆ.

\vskip   4pt

ಸರಳವಾಗಿ, ಸ್ವಲ್ಪವೂ ಅನುಮಾನವಿಲ್ಲದೆ ನಮ್ಮ ಶಾಸ್ತ್ರವು, ಮುಕ್ತಿಕಾಮಿಗಳೆಲ್ಲಾ ಮಂತ್ರದ್ರಷ್ಟೃಗಳಾಗಬೇಕು, ದೇವರನ್ನು ಕಾಣಬೇಕು ಎಂದು ಸಾರುವುದು. ಅದೇ ಮುಕ್ತಿ. ಅದೇ ನಮ್ಮ ಶಾಸ್ತ್ರ ವಿಧಿಸಿರುವ ನಿಯಮ. ಆಗ ನಾವೇ ಶಾಸ್ತ್ರವನ್ನು ಓದುವುದಕ್ಕೆ ಸುಲಭವಾಗುವುದು, ನಾವೇ ಅರ್ಥವನ್ನು ಗ್ರಹಿಸಬಹುದು, ನಮಗೆ ಇಷ್ಟವಾದಂತೆ ಅದನ್ನು ವಿಶ್ಲೇಷಿಸಬಹುದು ಮತ್ತು ನಾವೇ ಸತ್ಯವನ್ನು ತಿಳಿದುಕೊಳ್ಳಬಹುದು. ನಾವು ಮಾಡಬೇಕಾದ ಕರ್ತವ್ಯವೇ ಇದು. ಹಿಂದಿನ ಕಾಲದ ಮಹಾ ಋಷಿಗಳ ಸೇವೆಯನ್ನೆಲ್ಲ ಪೂಜ್ಯ ದೃಷ್ಟಿಯಿಂದ ಸ್ಮರಿಸಬೇಕು. ಹಿಂದಿನವರು ಮಹಾತ್ಮರು. ಆದರೆ ನಾವು ಅವರಿಗಿಂತ ಹೆಚ್ಚು ಮಹಾಪುರುಷರಾಗಬೇಕು. ಅವರು ಹಿಂದೆ ಮಹಾ ಕೆಲಸವನ್ನು ಸಾಧಿಸಿದರು. ಆದರೆ ನಾವು ಅದಕ್ಕಿಂತ ಉತ್ತಮವಾದುದನ್ನು ಸಾಧಿಸಬೇಕು. ಪ್ರಾಚೀನ ಭರತಖಂಡದಲ್ಲಿ ನೂರಾರು ಮಂದಿ ಋಷಿಗಳಿದ್ದರು. ನಮ್ಮಲ್ಲಿ ಲಕ್ಷಾಂತರ ಋಷಿಗಳು ಬರುವರು. ನಿಮ್ಮಲ್ಲಿ ಪ್ರತಿಯೊಬ್ಬರೂ ಇದನ್ನು ಎಷ್ಟು ಬೇಗ ನಂಬುವಿರೋ, ಅಷ್ಟೂ ಭರತಖಂಡಕ್ಕೆ ಮತ್ತು ಜಗತ್ತಿಗೆ ಒಳ್ಳೆಯದು. ನೀವು ನಂಬಿದಂತೆ ಆಗುವಿರಿ; ಧೀರರೆಂದು ನೀವು ನಂಬಿದರೆ, ಧೀರರಾಗುವಿರಿ. ಮಹಾಪುರುಷರೆಂದು ನಾವು ಭಾವಿಸಿದರೆ, ಮಹಾಪುರುಷರಾಗುವಿರಿ. ನಿಮ್ಮನ್ನು ತಡೆಯುವುದು ಯಾವುದೂ ಇಲ್ಲ. ತೋರಿಕೆಗೆ ಪರಸ್ಪರ ವಿರೋಧಗಳಿಂದ ಕೂಡಿರುವ ಹಲವು ಸಿದ್ಧಾಂತಗಳಲ್ಲಿ ಒಂದು ಸರ್ವಸಾಮಾನ್ಯ ಭಾವನೆ ಇದ್ದರೆ, ಅದೇ, ಎಲ್ಲಾ ಮಹಿಮೆ, ಶಕ್ತಿ ಮತ್ತು ಪರಿಶುದ್ಧತೆ ಆಗಲೇ ಆತ್ಮನಲ್ಲಿ ಇದೆ ಎನ್ನುವುದು. ರಾಮಾನುಜರ ದೃಷ್ಟಿಯಲ್ಲಿ ಜೀವವು ಸಂಕುಚಿತವಾಗುವುದು ಮತ್ತು ವಿಕಾಸವಾಗುವುದು. ಶಂಕರಾಚಾರ್ಯರ ದೃಷ್ಟಿಯಲ್ಲಿ ಜೀವವು ಭ್ರಾಂತಿಗೆ ಒಳಗಾಗುವುದು. ಈ ಭಿನ್ನತೆಯನ್ನು ಗಣನೆಗೆ ತರಬೇಕಾಗಿಲ್ಲ. ಅದು ಸುಪ್ತವಾಗಿರಲಿ ಅಥವಾ ವ್ಯಕ್ತವಾಗಿರಲಿ, ಒಂದು ಶಕ್ತಿಯು ಇದೆ ಎಂಬ ಸತ್ಯವನ್ನು ಎಲ್ಲರೂ ಒಪ್ಪುತ್ತಾರೆ. ನೀವು ಎಷ್ಟು ಬೇಗ ಇದನ್ನು ನಂಬಿದರೆ ಅಷ್ಟೂ ಒಳ್ಳೆಯದು. ಎಲ್ಲ ಶಕ್ತಿಯೂ ನಿಮ್ಮಲ್ಲಿ ಅಂತರ್ಗತವಾಗಿದೆ. ನೀವು ಏನನ್ನು ಬೇಕಾದರೂ ಮಾಡಬಹುದು;\break ಎಲ್ಲವನ್ನೂ ಮಾಡಬಹುದು; ಇದನ್ನು ನಂಬಿ; ನೀವು ದುರ್ಬಲರೆಂದು ನಂಬಬೇಡಿ. ಈಚಿನ ದಿನಗಳಲ್ಲಿ ಕೆಲವರು ಭಾವಿಸುವಂತೆ ನಿಮ್ಮನ್ನು ನೀವು ಅರೆಹುಚ್ಚರೆಂದು ಭಾವಿಸಬೇಡಿ. ಯಾರ ಆಸರೆಯೂ ಇಲ್ಲದೆ ಏನನ್ನು ಬೇಕಾದರೂ, ಎಲ್ಲವನ್ನೂ ನೀವು ಸಾಧಿಸಬಹುದು. ಅನಂತ ಶಕ್ತಿ ನಿಮ್ಮಲ್ಲಿರುವುದು. ನಿಮ್ಮಲ್ಲಿ ಅಂತರ್ಗತವಾಗಿರುವ ಪವಿತ್ರತೆಯನ್ನು ವ್ಯಕ್ತಗೊಳಿಸಿ.

