
\chapter{ಮಧುರೆಯ ಬಿನ್ನವತ್ತಳೆಗೆ ಉತ್ತರ}

ಮಧುರೆಯ ಹಿಂದೂಗಳು ಸ್ವಾಮೀಜಿಯವರಿಗೆ ಈ ಕೆಳಗಿನ ಬಿನ್ನವತ್ತಳೆಯನ್ನು ಅರ್ಪಿಸಿದರು:

\textbf{ಪರಮಪೂಜ್ಯ ಸ್ವಾಮೀಜಿಯವರೇ,}

ಮಧುರೆಯ ಹಿಂದೂಗಳಾದ ನಾವು ಈ ಪುರಾತನ ಮತ್ತು ಪವಿತ್ರ ನಗರಕ್ಕೆ ತಮ್ಮನ್ನು ಹೃತ್ಪೂರ್ವಕವಾಗಿಯೂ ಗೌರವಪೂರ್ವಕವಾಗಿಯೂ ಸ್ವಾಗತಿಸುತ್ತೇವೆ. ಸರ್ವಸಂಗ ಪರಿತ್ಯಾಗಿಯಾಗಿ ಮಾನವಕೋಟಿಯ ಆಧ್ಯಾತ್ಮಿಕ ಕಲ್ಯಾಣಕ್ಕಾಗಿಯೇ ತನ್ನ ಜೀವನವನ್ನು ಮುಡಿಪಾಗಿರಿಸಿದ ಹಿಂದೂ ಸಂನ್ಯಾಸಿಯೊಬ್ಬನ ಜೀವಂತ ನಿದರ್ಶನವಾಗಿ ತಾವು ನಮಗೆ ಕಾಣುತ್ತಿರುವಿರಿ. ಹಿಂದೂಧರ್ಮದ ನಿಜವಾದ ಸಾರವು ಕೇವಲ ನಿಯಮಗಳು ಆಚಾರಗಳು ಮಾತ್ರವಲ್ಲ, ದೀನ ದಲಿತರಿಗೆ ಶಾಂತಿ ಸಮಾಧಾನಗಳನ್ನು ನೀಡುವ ಗಂಭೀರ ತತ್ತ್ವಗಳೂ ಅದರಲ್ಲಿ ಅಡಗಿವೆ ಎಂಬುದನ್ನು ತಮ್ಮ ವೈಯಕ್ತಿಕ ಉದಾಹರಣೆಯ ಮೂಲಕ, ಪ್ರಾಯೋಗಿಕವಾಗಿ ತೋರಿಸಿದ್ದೀರಿ.

ಪ್ರತಿಯೊಬ್ಬ ವ್ಯಕ್ತಿಯೂ ತನ್ನ ಸಾಮರ್ಥ್ಯ ಮತ್ತು ಪರಿಸ್ಥಿತಿಗಳಿಗನುಸಾರವಾಗಿ ಉನ್ನತಿಗೇರುವುದಕ್ಕೆ ಸಹಾಯಮಾಡುವ ದರ್ಶನ ಮತ್ತು ಧರ್ಮಗಳನ್ನು ಅಮೆರಿಕ ಮತ್ತು ಇಂಗ್ಲೆಂಡ್​ ದೇಶದವರು ಮೆಚ್ಚುವ ರೀತಿಯಲ್ಲಿ ತಾವು ಬೋಧಿಸಿದ್ದೀರಿ. ತಾವು ಕಳೆದ ಮೂರು ವರ್ಷಗಳು ವಿದೇಶಗಳಲ್ಲಿ ಬೋಧಿಸುತ್ತಿದ್ದಾಗ್ಯೂ ತಮ್ಮ ಆ ಬೋಧನೆಗಳನ್ನು ಈ ದೇಶದಲ್ಲಿ ಅತ್ಯುತ್ಸಾಹದಿಂದ ಸ್ವೀಕರಿಸಿರುವೆವು ಮತ್ತು ವಿದೇಶದಿಂದ ಆಮದಾದ ಜಡವಾದದ ಪ್ರಗತಿಯನ್ನು ಹಿಮ್ಮೆಟ್ಟಿಸಲು ಅವು ಸಾಕಷ್ಟು ಸಹಾಯ ಮಾಡಿವೆ.

ಭಾರತವು ಇಂದಿಗೂ ಜೀವಂತವಾಗಿದೆ, ಏಕೆಂದರೆ ಜಗತ್ತಿನ ಆಧ್ಯಾತ್ಮಿಕ ಉನ್ನತಿಯನ್ನು ಸಾಧಿಸುವಲ್ಲಿ ಅದು ಹಿರಿದಾದ ಪಾತ್ರವನ್ನು ವಹಿಸಬೇಕಾಗಿದೆ. ಈ ಕಲಿಯುಗದ ಅಂತ್ಯಭಾಗದಲ್ಲಿ ತಮ್ಮಂತಹ ವ್ಯಕ್ತಿಯೊಬ್ಬರ ಆಗಮನವು ಮುಂದೆ ಅನೇಕ ಶ್ರೇಷ್ಠ ಪುರುಷರು ಅವತರಿಸಿ ಮೇಲಿನ ಧ್ಯೇಯೋದ್ದೇಶವನ್ನು ಪೂರ್ತಿಗೊಳಿಸುತ್ತಾರೆ ಎಂಬುದರ ನಿರ್ದಿಷ್ಟ ಸೂಚಕವಾಗಿದೆ.

\newpage

ಸನಾತನ ವಿದ್ಯಾಕೇಂದ್ರವಾದ, ಸುಂದರೇಶ್ವರನ ನೆಚ್ಚಿನ ನಗರಿಯಾದ, ಯೋಗಿಗಳ ಪವಿತ್ರ ದ್ವಾದಶಾಂತಕ್ಷೇತ್ರವಾದ ಮಧುರೆಯು ಭಾರತೀಯ ತತ್ತ್ವಗಳ ತಮ್ಮ ವ್ಯಾಖ್ಯಾನವನ್ನು ಹೃತ್ಪೂರ್ವಕವಾಗಿ ಪ್ರಶಂಸಿಸುವುದರಲ್ಲಾಗಲಿ, ಮಾನವಕೋಟಿಗೆ ತಾವು ಸಲ್ಲಿಸಿರುವ ಅನರ್ಘ್ಯ ಸೇವೆಗೆ ಕೃತಜ್ಞತಾಪೂರ್ವಕವಾಗಿ ಮನ್ನಣೆಯನ್ನು ನೀಡುವುದರಲ್ಲಾಗಲಿ ಭಾರತದ ಬೇರೆ ಯಾವ ನಗರಕ್ಕೂ ಹಿಂದೆ ಬಿದ್ದಿಲ್ಲ.

ಶಕ್ತಿ ಉತ್ಸಾಹ ಉಪಯುಕ್ತತೆಗಳಿಂದ ತುಂಬಿದ ದೀರ್ಘ ಆಯುಷ್ಯವು ತಮಗೆ ಲಭಿಸಲೆಂದು ನಾವು ಭಗವಂತನನ್ನು ಪ್ರಾರ್ಥಿಸುತ್ತೇವೆ.

ಸ್ವಾಮೀಜಿಯವರು ಈ ಕೆಳಗಿನಂತೆ ಉತ್ತರ ನೀಡಿದರು:

ನಿಮ್ಮ ಅಧ್ಯಕ್ಷರು ಸೂಚಿಸಿದಂತೆ ನಿಮ್ಮೊಂದಿಗೆ ಹಲವು ದಿನಗಳಿದ್ದು, ಪಾಶ್ಚಾತ್ಯ ದೇಶಗಳಲ್ಲಿ ನನ್ನ ನಾಲ್ಕು ವರ್ಷಗಳ ಅನುಭವ, ಮತ್ತು ಅಲ್ಲಿ ನನ್ನ ಪರಿಶ್ರಮದ ಫಲ, ಇವುಗಳನ್ನು ಕುರಿತು ಹೇಳಬೇಕೆಂಬ ಆಸೆಯೇನೋ ಇದೆ. ಆದರೆ, ದುರದೃಷ್ಟ ವಶಾತ್​ ಸಂನ್ಯಾಸಿಗೂ ಒಂದು ದೇಹವಿದೆ. ಕಳೆದ ಮೂರು ವಾರಗಳಿಂದ ಬಿಡುವಿಲ್ಲದ ಸಂಚಾರದಿಂದ ಮತ್ತು ಉಪನ್ಯಾಸಗಳಿಂದ ಬಳಲಿರುವೆನು. ಆ ಕಾರಣ ಈ ಸಂಜೆ ನಾನು ದೀರ್ಘ ಉಪನ್ಯಾಸ ನೀಡುವ ಸ್ಥಿತಿಯಲ್ಲಿ ಇಲ್ಲ. ನೀವು ನನಗೆ ನೀಡಿದ ಆದರದ ಸ್ವಾಗತಕ್ಕೆ ನಾನು ಹೃತ್ಪೂರ್ವಕ ಕೃತಜ್ಞತೆಗಳನ್ನು ಅರ್ಪಿಸುವುದರಲ್ಲೇ ಇಂದು ನಾನು ತೃಪ್ತನಾಗುತ್ತೇನೆ. ಮುಂದೆ ಆರೋಗ್ಯ ಸುಧಾರಿಸಿದ ಮೇಲೆ ಬೇರೆ ವಿಷಯಗಳನ್ನು ಕುರಿತು ಮಾತನಾಡುತ್ತೇನೆ. ಈ ಅಲ್ಪಾವಧಿಯಲ್ಲಿ ಆ ಎಲ್ಲಾ ವಿಷಯಗಳನ್ನೂ ಹೇಳಲಾಗುವುದಿಲ್ಲ. ಪ್ರಖ್ಯಾತ ರಾಮನಾಡಿನ ಅತಿಥಿಯಾಗಿ ಮಧುರೆಯಲ್ಲಿರುವ ನನ್ನ ಮನಸ್ಸಿಗೆ ಒಂದು ವಿಷಯ ಪ್ರಮುಖವಾಗಿ ಹೊಳೆಯುತ್ತಿದೆ. ಮೊದಲು ಚಿಕಾಗೋ ನಗರಕ್ಕೆ ಹೋಗುವಂತೆ ಸಲಹೆ ಕೊಟ್ಟವರು ರಾಮ ನಾಡಿನ ಮಹಾರಾಜರು ಎಂಬುದು\break ನಿಮ್ಮಲ್ಲಿ ಹಲವರಿಗೆ ಗೊತ್ತಿರಬಹುದು ಮತ್ತು ಅವರೇ ಈ ವಿಷಯ\-ದಲ್ಲಿ\break ನಿರಂತರವೂ ಪ್ರೋತ್ಸಾಹ ನೀಡಿದವರು. ನಿಮ್ಮ ಬಿನ್ನವತ್ತಳೆಯಲ್ಲಿ ನನ್ನ ಮೇಲೆ ಮಳೆಗರೆದಿರುವ ಪ್ರಶಂಸೆಯ ಬಹುಪಾಲು ದಕ್ಷಿಣ ಭಾರತದ ಆ ಪ್ರಖ್ಯಾತ ವ್ಯಕ್ತಿಗೆ ಸಲ್ಲುವುದು. ಅವರು ರಾಜರಾಗುವುದಕ್ಕಿಂತ ಸಂನ್ಯಾಸಿಗಳಾಗಿದ್ದರೆ ಮೇಲಾಗುತ್ತಿತ್ತೆಂದು ಆಶಿಸುತ್ತೇನೆ. ನಿಜವಾಗಿ ಅವರು ಈ ಸ್ಥಾನಕ್ಕೆ ಪಾತ್ರರು.

ಜಗತ್ತಿನ ಯಾವುದಾದರೊಂದೆಡೆಯಲ್ಲಿ ಏನಾದರೂ ಕೊರತೆಯಿದ್ದರೆ\break ಅದನ್ನು ಪೂರ್ಣ ಮಾಡುವ ವಸ್ತು ಅಲ್ಲಿ ಬಂದು ಸೇರಿ, ನವಜೀವನವನ್ನು ಉಂಟುಮಾಡುತ್ತದೆ. ಇದು ಭೌತಿಕ ಜೀವನದಲ್ಲಿ ಸತ್ಯವಾಗಿರುವಂತೆಯೇ ಆಧ್ಯಾತ್ಮಿಕ ಕ್ಷೇತ್ರದಲ್ಲಿಯೂ ಸತ್ಯ. ಜಗತ್ತಿನ ಒಂದು ಕಡೆ ಆಧ್ಯಾತ್ಮಿಕ ಕೊರತೆ ಇದ್ದು, ಮತ್ತೊಂದು ಕಡೆ ಅದು ವಿಶೇಷವಾಗಿದ್ದರೆ, ನಾವು ಪ್ರಜ್ಞಾಪೂರ್ವಕವಾಗಿ ಪ್ರಯತ್ನಪಡಲಿ, ಪಡದಿರಲಿ, ಅಧ್ಯಾತ್ಮ ಅಲ್ಲಿಗೆ ಹೋಗುವುದು ಮತ್ತು ಆ ಕೊರತೆಯನ್ನು ಹೋಗಲಾಡಿಸುವುದು. ಮಾನವ ಇತಿಹಾಸದಲ್ಲಿ ಒಂದು ವೇಳೆಯಲ್ಲ, ಎರಡು ವೇಳೆಯಲ್ಲ, ಹಲವು ವೇಳೆ ಭರತಖಂಡ ಜಗತ್ತಿಗೆ ಅಧ್ಯಾತ್ಮವನ್ನು ಧಾರೆ ಎರೆದಿದೆ. ದಿಗ್ವಿಜಯದಿಂದಲೋ, ಆರ್ಥಿಕ ಪ್ರಭಾವದಿಂದಲೋ, ಪ್ರಪಂಚದ ಭಿನ್ನ ಭಿನ್ನ ಭಾಗಗಳು ಒಂದುಗೂಡಿದಾಗ, ಒಂದು ಕಡೆಯಿಂದ ಮತ್ತೊಂದು ಕಡೆಗೆ ಹಲವು ವಿಷಯಗಳು ಪ್ರಸಾರವಾಗಿವೆ. ಪ್ರತಿಯೊಂದು ದೇಶವೂ ತನ್ನ ಪಾಲಿನದನ್ನು-ಅದು ರಾಜಕೀಯ, ಸಾಮಾಜಿಕ, ಅಥವಾ ಆಧ್ಯಾತ್ಮಿಕ ವಿಷಯಗಳಾಗಿರಬಹುದು-ಧಾರೆ ಎರೆದಿದೆ. ಮಾನವ ಜ್ಞಾನ ನಿಧಿಗೆ ಭರತಖಂಡದ ದಾನವೇ, ದರ್ಶನ ಮತ್ತು ಆಧ್ಯಾತ್ಮಿಕತೆ. ಪಾರ್ಸಿ ಚಕ್ರಾಧಿಪತ್ಯ ಪ್ರವರ್ಧಮಾನಕ್ಕೆ ಬರುವುದಕ್ಕೆ ಮುಂಚೆಯೇ ಭರತಖಂಡ ಇವನ್ನು ಕೊಟ್ಟಿತು. ಎರಡನೆಯ ವೇಳೆ ಪರ್ಸಿಯಾದ ಚಕ್ರಾಧಿಪತ್ಯದ ಕಾಲದಲ್ಲಿ, ಮೂರನೆಯ ಆವೃತ್ತಿ ಗ್ರೀಕರ ಉಚ್ಛ್ರಾಯ ಕಾಲದಲ್ಲಿ; ಈಗ ಆಂಗ್ಲೇಯರು ಪ್ರಖ್ಯಾತರಾಗಿರುವಾಗ, ನಾಲ್ಕನೆಯ ಬಾರಿ. ಹಿಂದಿನಂತೆ ಭರತಖಂಡ ತನ್ನ ಪಾತ್ರವನ್ನು ಈಗಲೂ ನಿರ್ವಹಿಸುವುದು. ಪಾಶ್ಚಾತ್ಯರ ಸಂಘಟನಾಕ್ರಮ, ಅವರ ಬಾಹ್ಯ ನಾಗರಿಕತೆ, ಹೇಗೆ ನಮ್ಮ ಮೇಲೆ, ನಾವು ಇಚ್ಛಿಸಲಿ ಬಿಡಲಿ, ತಮ್ಮ ಪ್ರಭಾವವನ್ನು ಬೀರು\-ತ್ತಿವೆಯೋ ಹಾಗೆಯೇ ಭರತಖಂಡದ ದರ್ಶನ ಮತ್ತು ಆಧ್ಯಾತ್ಮಿಕತೆ ಇವು ಪಾಶ್ಚಾತ್ಯ ದೇಶದ ಮೇಲೆ ತಮ್ಮ ಪ್ರಭಾವವನ್ನು ಬೀರುತ್ತಿವೆ. ಅದನ್ನು ಯಾರೂ ತಡೆಯಲಾರರು. ನಾವೂ ಕೂಡ ಪಾಶ್ಚಾತ್ಯ ಭೌತಿಕ ನಾಗರಿಕತೆಯ ಪ್ರಭಾವವನ್ನು ತಡೆಯಲಾಗುವುದಿಲ್ಲ. ಬಹುಶಃ ಸ್ವಲ್ಪ ಅವರ ನಾಗರಿಕತೆ ನಮಗೆ ಹಿತಕಾರಿ, ಸ್ವಲ್ಪ ಅಧ್ಯಾತ್ಮ ಅವರಿಗೆ ಹಿತಕಾರಿ. ಹೀಗೆ ಸಮತೋಲನವುಂಟಾಗುತ್ತದೆ. ನಾವು ಪಾಶ್ಚಾತ್ಯರಿಂದ ಎಲ್ಲವನ್ನೂ ಕಲಿತುಕೊಳ್ಳಬೇಕೆಂದೂ ಅಲ್ಲ, ಅಥವಾ ಅವರು ನಮ್ಮಿಂದ ಎಲ್ಲವನ್ನೂ ಕಲಿತುಕೊಳ್ಳಬೇಕೆಂದು ಅಲ್ಲ. ಪ್ರತಿಯೊಬ್ಬರೂ ಮುಂದೆ ಬರುವವರಿಗೆ ತಮ್ಮಲ್ಲಿರುವುದನ್ನು, ಹಲವು ಶತಮಾನಗಳಿಂದ ಕನಸು ಕಾಣುತ್ತಿದ್ದ ಜನಾಂಗದ ಸಾಮರಸ್ಯವೆಂಬ ಆದರ್ಶಜಗತ್ತಿನ ಉದಯಕ್ಕೆ, ಧಾರೆ ಎರೆದು ಕೊಡಬೇಕು. ಆದರ್ಶ ಜಗತ್ತು ಎಂದಾದರೂ ಜಗತ್ತಿನಲ್ಲಿ ಸಾಧ್ಯವೇ, ಎಂಬುದು ನನಗೆ ತಿಳಿಯದು. ಸಮಾಜ ಪೂರ್ಣತೆಯನ್ನು ಮುಟ್ಟುವುದಕ್ಕೆ ಸಾಧ್ಯವೇ ಎಂಬುದರಲ್ಲಿ ನನಗೇನೋ ಸಂದೇಹವಿದೆ. ಅದು ಸಾಧ್ಯವಾಗಲಿ ಆಗದಿರಲಿ, ಅದು ನಾಳೆಯೇ ಬರಬಹುದೆಂದು, ಮತ್ತು ಅದು ನನ್ನ ಕರ್ತವ್ಯದ ಮೇಲೆ ನಿಂತಿರುವುದೆಂದು ತಿಳಿದು ಕಾರ್ಯೋನ್ಮುಖನಾಗಬೇಕು. ಪ್ರತಿಯೊಬ್ಬರೂ ಜಗತ್ತಿನ ಪರಿಪೂರ್ಣತೆಗಾಗಿ ತಮ್ಮ ಪಾಲಿನ ಕರ್ತವ್ಯವನ್ನು ಮಾಡಿರುವರು, ಮಿಕ್ಕಿರುವುದೇ\break ತಮ್ಮ ಪಾಲಿನದು ಎಂದು ಎಲ್ಲರೂ ಭಾವಿಸಬೇಕು. ನಾವು ವಹಿಸಬೇಕಾದ\break ಜವಾಬ್ದಾರಿ ಇದು.

ಈಚೆಗೆ ಭರತಖಂಡದಲ್ಲಿ ಪ್ರಚಂಡ ಧಾರ್ಮಿಕ ಜಾಗೃತಿಯಾಗಿದೆ. ಮುಂದೆ ಇದರಿಂದ ಕೀರ್ತಿ ಬರಬಹುದು. ಆದರೆ ಅದರಲ್ಲಿ ಅಪಾಯವೂ ಇದೆ. ಏಕೆಂದರೆ ಕೆಲವು ವೇಳೆ ಜಾಗೃತಿಯಿಂದ ಮತಭ್ರಾಂತಿ ಜನಿಸುತ್ತದೆ. ಕೆಲವು ವೇಳೆ ಅದು ಮಿತಿ ಮೀರಿ ಹೋಗುತ್ತದೆ. ಅದು ಒಂದು ಮಿತಿಯನ್ನು ಮೀರಿ ಮುಂದೆ ಹೋದ ಮೇಲೆ, ಯಾರು ಈ ಜಾಗೃತಿಗೆ ಕಾರಣಕರ್ತರೋ ಅವರಿಗೂ ಕೂಡ ಅದನ್ನು ನಿಗ್ರಹಿಸಲು ಅಸಾಧ್ಯವಾಗುತ್ತದೆ. ಆದ್ದರಿಂದ ಮುಂಚಿನಿಂದಲೇ ಜೋಪಾನವಾಗಿರುವುದು ಒಳ್ಳೆಯದು. ಪೂರ್ವಾಚಾರ ಪರಾಯಣರು ಒಂದು ಕಡೆ, ಪಾಶ್ಚಾತ್ಯ ಸಂಸ್ಕೃತಿಯ ಅಂತರಾಳಕ್ಕೆ ದಾಳಿ ಇಟ್ಟ ಜಡ ನಿರೀಶ್ವರವಾದ ಒಂದು ಕಡೆ, ಇವುಗಳ ಮಧ್ಯದಲ್ಲಿ ನಾವು ಹೋಗಬೇಕಾಗಿದೆ. ಇವೆರಡರ ವಿಷಯದಲ್ಲಿಯೂ ನಾವು ಎಚ್ಚರಿಕೆ ವಹಿಸಬೇಕಾಗಿದೆ. ನಾವು ಪಾಶ್ಚಾತ್ಯರಾಗಲಾರೆವು. ಆದ್ದರಿಂದ ಪಾಶ್ಚಾತ್ಯರನ್ನು ಅನುಕರಿಸಿ ಪ್ರಯೋಜನವಿಲ್ಲ. ನೀವು ಪಾಶ್ಚಾತ್ಯರನ್ನು ಅನುಕರಿಸುವುದಕ್ಕೆ ಪ್ರಯತ್ನ ಪಟ್ಟೊ ಡನೆಯೇ ನಾಶವಾಗುವಿರಿ, ನೀವು ನಿರ್ಜೀವರಾಗುತ್ತೀರಿ. ಎರಡನೆಯದಾಗಿ ಇದು ಅಸಾಧ್ಯ. ಅನಾದಿಕಾಲದಲ್ಲಿ ನದಿ ಒಂದೆಡೆ ಉಗಮಿಸಿದೆ, ಅದು ಸಹಸ್ರಾರು ವರ್ಷಗಳಿಂದ ಒಂದು ದಿಕ್ಕಿನಲ್ಲಿ ಹರಿಯುತ್ತಿದೆ. ಆ ನದಿಯನ್ನು ಹಿಮಾಲಯದಲ್ಲಿರುವ ಅದರ ಮೂಲಸ್ಥಾನಕ್ಕೆ ಹಿಂತಿರುಗಿಸಬಲ್ಲಿರಾ? ಒಂದು ವೇಳೆ ಅದು ಸಾಧ್ಯವಾದರೂ ನೀವು ಪಾಶ್ಚಾತ್ಯರಾಗ\-ಲಾರಿರಿ. ಪಾಶ್ಚಾತ್ಯ ದೇಶದಲ್ಲಿ ಬರಿಯ ಕೆಲವು ಶತಮಾನಗಳ ಸಂಸ್ಕೃತಿಯನ್ನು ಅವರಿಗೆ ಕಿತ್ತೊಗೆಯಲು ಸಾಧ್ಯವಿಲ್ಲದೆ ಇರುವಾಗ ಸಹಸ್ರಾರು ವರ್ಷಗಳಿಂದ ಬಂದ ಸಂಸ್ಕೃತಿಯನ್ನು ನೀವು ಕಿತ್ತೊಗೆಯಬಲ್ಲಿರಾ? ಇದು ಸಾಧ್ಯವಿಲ್ಲ. ಈಗ ಸಾಧಾರಣವಾಗಿ ಯಾವುದನ್ನು ನಾವು ಧಾರ್ಮಿಕ ಶ್ರದ್ಧೆ ಎನ್ನುವೆವೋ ಅದು ಪ್ರತಿಯೊಂದು ಗ್ರಾಮದೇವತಾ ಪೂಜೆ ಮತ್ತು ಮೂಢಾಚಾರಗಳ ರೂಪವನ್ನು ಪಡೆದಿದೆ ಎಂಬುದನ್ನು ನಾವು ನೆನಪಿನಲ್ಲಿಡಬೇಕು. ಆದರೆ ಲೋಕಾಚಾರಗಳು ಹಲವು ಇವೆ, ಅವು ಪರಸ್ಪರ ವಿರುದ್ಧವಾಗಿವೆ. ಯಾವುದನ್ನು ಅನುಸರಿಸಬೇಕು? ಯಾವುದನ್ನು ವಿರೋಧಿಸಬೇಕು? ದಕ್ಷಿಣ ದೇಶದ ಬ್ರಾಹ್ಮಣ ಮತ್ತೊಬ್ಬ ಬ್ರಾಹ್ಮಣ ಮಾಂಸ ತಿನ್ನುವುದನ್ನು ನೋಡಿದರೆ ಜುಗುಪ್ಸೆಯಿಂದ ಮುಖ ತಿರುಗಿಸುವನು. ಔತ್ತರೇಯ ಬ್ರಾಹ್ಮಣ ಇದು ಪರಮ ಪವಿತ್ರವೆಂದು ಭಾವಿಸಿ, ನೂರಾರು ಆಡುಗಳನ್ನು ಬಲಿ ಕೊಡುವನು. ನೀವು ನಿಮ್ಮ ಆಚಾರವನ್ನು ತಂದರೆ ಅವರು ಅವರ ಆಚಾರವನ್ನು ತರುವರು. ಭರತಖಂಡದಲ್ಲಿ ಎಷ್ಟೋ ಬಗೆಯ ಲೋಕಾಚಾರಗಳಿವೆ. ಆದರೆ ಇವೆಲ್ಲಾ ಬರಿಯ ಲೋಕಾಚಾರಗಳು. ದೊಡ್ಡ ತಪ್ಪು ಯಾವುದೆಂದರೆ ಅಜ್ಞಾನಿಗಳು ಲೋಕಾಚಾರವನ್ನೇ ಧರ್ಮದ ಸಾರವೆಂದು ಭಾವಿಸುವುದು.

ಆದರೆ ಇದರ ಆಚೆ ಇನ್ನೂ ಹೆಚ್ಚು ಕಷ್ಟವಿದೆ. ಎರಡು ಬಗೆಯ ಸತ್ಯಗಳು ನಮ್ಮ ಶಾಸ್ತ್ರದಲ್ಲಿವೆ. ಒಂದು ಮನುಷ್ಯನ ನಿಜಸ್ವರೂಪಕ್ಕೆ ಸಂಬಂಧಪಟ್ಟದ್ದು. ಇದು ದೇವರು, ಆತ್ಮ, ಪ್ರಕೃತಿ ಇವುಗಳ ಪರಸ್ಪರ ಸಂಬಂಧವನ್ನು ವಿವರಿಸುತ್ತದೆ. ಎರಡನೆಯದು ದೇಶ-ಕಾಲ-ಅವಸ್ಥಾ ಭೇದಗಳ ಮೇಲೆ ನಿಂತ ಲೋಕಾಚಾರಗಳು. ಮೊದಲನೆಯದು ನಮ್ಮ ಶಾಸ್ತ್ರವಾದ ವೇದಗಳಲ್ಲಿ ಮುಖ್ಯವಾಗಿ ಅಡಕವಾಗಿದೆ. ಎರಡನೆಯದು ನಮ್ಮ ಸ್ಮೃತಿ ಮತ್ತು ಪುರಾಣಗಳಲ್ಲಿದೆ. ಎಲ್ಲಾ ಕಾಲಕ್ಕೂ ವೇದಗಳೇ ಪರಮ ಗುರಿ ಮತ್ತು ಪ್ರಮಾಣವೆಂಬುದನ್ನು ನಾವು ನೆನಪಿನಲ್ಲಿಡಬೇಕು. ಪುರಾಣದ ಯಾವುದೇ ಭಾಗವು ವೇದಕ್ಕೆ ವಿರುದ್ಧವಾಗಿದ್ದರೆ ಆ ಭಾಗವನ್ನು ನಿರ್ದಾಕ್ಷಿಣ್ಯವಾಗಿ ತಿರಸ್ಕರಿಸಬೇಕು. ಮತ್ತೆ ಬೇರೆ ಬೇರೆ ಸ್ಮೃತಿಗಳಲ್ಲಿಯೂ ಬೋಧನೆಗಳು ಬೇರೆ ಬೇರೆಯಾಗಿರುವುದನ್ನು ಕಾಣುತ್ತೇವೆ. ಒಂದು ಸ್ಮೃತಿಯು ಈ ಕಾಲದ ಆಚಾರ ಇದೇ ಎಂದೂ ಇದನ್ನೇ ಅನುಸರಿಸಬೇಕೆಂದೂ ಹೇಳುತ್ತದೆ. ಇನ್ನೊಂದು ಸ್ಮೃತಿಯು ಇನ್ನೊಂದು ಬಗೆಯ ಆಚಾರವನ್ನು ಹೀಗೆಯೇ ಪ್ರತಿಪಾದಿಸುತ್ತದೆ. ಒಂದು ಸತ್ಯಯುಗದ ಆಚಾರ; ಮತ್ತೊಂದು ಕಲಿಯುಗದ ಆಚಾರ ಎನ್ನುವುದು. ಆದರೆ ಮಾನವನ ನೈಜ ಸ್ವಭಾವದ ಮೇಲೆ ನಿಂತ ನಿತ್ಯಸತ್ಯಗಳು ಎಂದಿಗೂ ಬದಲಾಗುವುದಿಲ್ಲವೆನ್ನುವುದು ಪ್ರಖ್ಯಾತ ಸಿದ್ಧಾಂತ. ಅವು ಎಲ್ಲಾ ದೇಶಗಳಿಗೂ ಅನ್ವಯಿಸುವ ಸಾರ್ವಕಾಲಿಕ ಸನಾತನ ತತ್ತ್ವಗಳು. ಸ್ಮೃತಿಗಳು ಆಯಾ ಕಾಲ-ದೇಶಗಳಿಗೆ ಅನ್ವಯಿಸುವ ಆಚಾರಗಳನ್ನು ಮಾತ್ರ ಹೇಳುತ್ತವೆ. ಅವು ಕಾಲಕಾಲಕ್ಕೆ ಬದಲಾಗುತ್ತವೆ. ಇದನ್ನು ನೀವು ಯಾವಾಗಲೂ ಜ್ಞಾಪಕದಲ್ಲಿಟ್ಟಿರಬೇಕು. ಏಕೆಂದರೆ ಸಮಾಜಕ್ಕೆ ಅನ್ವಯಿಸುವ ಕೆಲವು ಆಚಾರಗಳು ಬದಲಾದರೆ ನಿಮ್ಮ ಧರ್ಮವೇನೂ ಹಾಳಾಗುವುದಿಲ್ಲ. ಈ ಆಚಾರಗಳು ಈಗಾಗಲೇ ಎಷ್ಟೋ ಬದಲಾಗಿವೆ ಎಂಬುದನ್ನು ಗಮನದಲ್ಲಿಡಿ. ದನದ ಮಾಂಸವನ್ನು ತಿನ್ನದೆ ಯಾವ ಬ್ರಾಹ್ಮಣನೂ ಬ್ರಾಹ್ಮಣನಾಗಿ ಉಳಿಯಲಾಗದ ಕಾಲವೊಂದಿತ್ತು. ಭಾರತದಲ್ಲಿ ಸಂನ್ಯಾಸಿ, ರಾಜ, ಅಥವಾ ಶ್ರೇಷ್ಠ ವ್ಯಕ್ತಿಯೊಬ್ಬನು ಮನೆಗೆ ಬಂದರೆ ಅವನ ಔತಣಕ್ಕಾಗಿ ಅತ್ಯುತ್ತಮವಾದ ಎತ್ತನ್ನು ಕೊಲ್ಲುತ್ತಿದ್ದರು ಎಂಬುದನ್ನು ನಾವು ವೇದದಲ್ಲಿ ಓದುತ್ತೇವೆ. ಆದರೆ ಕಾಲಕ್ರಮೇಣ, ನಮ್ಮದು ಕೃಷಿಪ್ರಧಾನ ಜೀವನವಾಗಿದ್ದರಿಂದ, ಅತ್ಯುತ್ತಮ ಎತ್ತುಗಳನ್ನು ಕೊಂದರೆ ಕ್ರಮೇಣ ದನದ ಕುಲವೇ ನಾಶವಾಗುವುದೆಂದು ತಿಳಿದು ಗೋಹತ್ಯೆ ಮಹಾಪಾಪವೆಂದು ಪರಿಗಣಿಸಲು ಪ್ರಾರಂಭಿಸಿದರು. ಈಗ ಮಹಾಪಾಪವೆಂದು ಪರಿಗಣಿಸಿದ ಕೆಲವು ಪದ್ಧತಿಗಳು ಹಿಂದೆ ಧಾರ್ಮಿಕವಾಗಿದ್ದವು. ಕಾಲಕ್ರಮೇಣ ಬೇರೆ ಆಚಾರಗಳು ರೂಢಿಗೆ ಬಂದವು. ಕೆಲವು ಕಾಲದ ಮೇಲೆ ಇವೂ ಹೋಗಿ ಬೇರೆ ಸ್ಮೃತಿಗಳು ಬರುವುವು. ವೇದಗಳು ಮಾತ್ರ ನಿತ್ಯವಾದುದರಿಂದ ಎಲ್ಲಾ ಕಾಲಕ್ಕೂ ಒಂದೇ ಸಮನಾಗಿರುವುವು ಎಂಬ ಅಂಶವನ್ನು ನಾವು ಕಲಿಯಬೇಕಾಗಿದೆ. ಆದರೆ ಸ್ಮೃತಿಗಳು ಕಾಲಕಾಲಕ್ಕೆ ಬದಲಾಗುವುವು. ಕಾಲಕಳೆದಂತೆ ಹೆಚ್ಚು ಹೆಚ್ಚು ಹಿಂದಿನ ಸ್ಮೃತಿಯ ಭಾಗಗಳು ಲುಪ್ತವಾಗುತ್ತವೆ. ಋಷಿಗಳು ಬಂದು ಆಯಾಯ ಕಾಲಕ್ಕೆ ತಕ್ಕ ಬದಲಾವಣೆ ಮಾಡಿ ಹೊಸ ಮಾರ್ಗವನ್ನು ಮತ್ತು ಕರ್ತವ್ಯವನ್ನು ಜನರಿಗೆ ಬೋಧಿಸಿ ಸಮಾಜವನ್ನು ಉತ್ತಮ ಮಾರ್ಗದಲ್ಲಿ ನಡೆಸಿಕೊಂಡು ಹೋಗುತ್ತಾರೆ. ಹೀಗಾಗದೆ ಇದ್ದರೆ ಸಮಾಜ ಉಳಿಯುವುದು ಸಾಧ್ಯವಿಲ್ಲ. ಹೀಗೆ ಈ ಎರಡು ಅಪಾಯಗಳಿಂದ ಪಾರಾಗಿ ನಾವು ಮುಂದುವರಿಯಬೇಕು. ಯಾವುದನ್ನೂ ಬಹಿಷ್ಕರಿ\-ಸದೆ ಎಲ್ಲವನ್ನೂ ಸ್ವೀಕರಿಸುವ ವಿಶಾಲಮತಿಯೂ ಶ್ರದ್ಧೆಯೂ ನಿಮ್ಮೆಲ್ಲರಲ್ಲೂ\- ಇರುವುದೆಂದು ನಾನು ಊಹಿಸುತ್ತೇನೆ. ಮತಭ್ರಾಂತನ ಉದ್ವೇಗದೊಂದಿಗೆ\break ಜಡವಾದಿಯ ವೈಶಾಲ್ಯ ಇರಲಿ. ಸಾಗರದಷ್ಟು ಆಳವಾಗಿ, ಆಗಸದಷ್ಟು ವಿಶಾಲವಾಗಿರಬೇಕು ನಮ್ಮ ಹೃದಯ. ಇತರ ರಾಷ್ಟ್ರಗಳಂತೆಯೇ ನಾವು ಪ್ರಗತಿಗಾಮಿಗಳಾಗಿರೋಣ. ಆದರೆ ಅದರ ಜೊತೆಗೆ ಹಿಂದೂಗಳಿಗೆ ಮಾತ್ರ ಸಾಧ್ಯವಾದ ಸಂಪ್ರದಾಯನಿಷ್ಠೆಯನ್ನೂ ಕಾಪಾಡಿಕೊಳ್ಳೋಣ.

ಪ್ರತಿಯೊಂದರಲ್ಲಿಯೂ ಮುಖ್ಯ ಯಾವುದು ಅಮುಖ್ಯ ಯಾವುದು ಎಂಬುದನ್ನು ನಾವು ಅರಿಯಬೇಕು. ಮುಖ್ಯವಾದುವು ಎಲ್ಲಾ ಕಾಲಕ್ಕೂ ಅನ್ವಯಿಸುವುವು. ಅಮುಖ್ಯವಾದುವು ಕೆಲವು ಕಾಲಕ್ಕೆ ಮಾತ್ರ ಅನ್ವಯಿಸುವುವು. ಅನಂತರ ಅವು ಬದಲಾಗದೆ ಇದ್ದರೆ ಅವುಗಳಿಂದ ಅಪಾಯವೇ ಸಿದ್ಧ. ಹಿಂದಿನ ಎಲ್ಲ ಆಚಾರಗಳನ್ನೂ ಸಂಸ್ಥೆಗಳನ್ನೂ ನೀವು ಅಲ್ಲಗಳೆಯಬೇಕೆಂಬುದು ನನ್ನ ಅಭಿಪ್ರಾಯವಲ್ಲ. ಯಾವುದನ್ನೂ ಅಲ್ಲಗಳೆಯಬೇಡಿ, ಅತಿ ನೀಚವಾದುದನ್ನೂ ಅಲ್ಲಗಳೆಯಕೂಡದು. ಯಾವುದನ್ನೂ ನಿಂದಿಸಬೇಡಿ. ಯಾವುದು ಈಗ ದೋಷಪೂರ್ಣವಾಗಿ ಕಾಣುತ್ತಿದೆಯೋ ಅದು ಹಿಂದೆ ಒಂದಾನೊಂದು ಕಾಲದಲ್ಲಿ ಜೀವನಪ್ರದವಾಗಿತ್ತು. ಈಗ ನಾವು ಅದನ್ನು ತೆಗೆದು ಹಾಕಬೇಕಾದರೆ, ಅದನ್ನು ನಿಂದಿಸಿ ಆ ಕೆಲಸ ಮಾಡಬೇಕಾಗಿಲ್ಲ. ನಮ್ಮ ಜನಾಂಗದ ರಕ್ಷಣೆಗೆ ಅದು ಮಾಡಿದ ಉಪಕಾರವನ್ನು ಸ್ಮರಿಸಿ ಕೃತಜ್ಞತೆಯಿಂದ ಆ ಕೆಲಸ ಮಾಡಬೇಕು. ಅದನ್ನು ನಿಂದಿಸಿ ಕಾರ್ಯ ಸಾಧಿಸದೆ ಇರುವುದಕ್ಕಿಂತ ಅದನ್ನು ಗೌರವಿಸಿ ಮುಂದುವರಿಯೋಣ. ನಮ್ಮ ಸಮಾಜದ ಮುಂದಾಳುಗಳು ಯಾವಾಗಲೂ ರಾಜರಾಗಿರಲಿಲ್ಲ. ಯೋಧರಾಗಿರಲಿಲ್ಲ, ಋಷಿಗಳಾಗಿದ್ದರು. ಈ ಋಷಿಗಳು ಯಾರು? ಉಪನಿಷತ್ತು ಯಾರನ್ನು ಋಷಿ ಎಂದು ಕರೆಯುವುದೋ ಅವನು ಸಾಮಾನ್ಯ ವ್ಯಕ್ತಿಯಲ್ಲ, ಅವನು ಮಂತ್ರದ್ರಷ್ಟಾ, ಅವನು ಧರ್ಮವನ್ನು ನೋಡುವವನು. ಅವನಿಗೆ ಧರ್ಮವೆಂದರೆ ಪಾಂಡಿತ್ಯವಲ್ಲ, ತರ್ಕವಲ್ಲ, ಊಹೆಯಲ್ಲ, ಬರಿಯ ಮಾತೂ ಅಲ್ಲ. ಅದು ಸತ್ಯ ಸಾಕ್ಷಾತ್ಕಾರ, ಇಂದ್ರಿಯಾತೀತ ಆತ್ಮಾನುಭೂತಿ. ಇಂತಹ ಅನುಭವವನ್ನು ಪಡೆದವರೇ ಋಷಿವರ್ಯರು. ಇವರು ಯಾವುದೋ ಕಾಲ-ದೇಶ-ಪಂಗಡಗಳಿಗೆ ಸೇರಿದವರಲ್ಲ. ಸತ್ಯವನ್ನು ಸಾಕ್ಷಾತ್ಕರಿಸಿಕೊಳ್ಳ ಬೇಕೆಂದು\break ವಾತ್ಸ್ಯಾಯನ ಹೇಳುತ್ತಾನೆ. ನಾವು ನೀವು ಎಲ್ಲರೂ ಋಷಿಗಳಾಗಬೇಕು. ನಮ್ಮಲ್ಲಿ ಆತ್ಮಶ್ರದ್ಧೆ ಇರಬೇಕು. ನಾವು ಜಗತ್ತನ್ನೇ ಚಾಲನೆಗೊಳಿಸುವಂತಹವ\break ರಾಗಬೇಕು. ಏಕೆಂದರೆ ಅನಂತಶಕ್ತಿ ನಮ್ಮಲ್ಲಿದೆ. ನಾವು ಧರ್ಮವನ್ನು\break ಅನುಭವಿಸಬೇಕು, ಸಾಕ್ಷಾತ್ಕಾರ ಮಾಡಿಕೊಳ್ಳಬೇಕು. ಅದರ ಬಗೆಗಿನ ಸಂದೇಹವನ್ನು ಹೀಗೆ ಕಳೆದುಕೊಳ್ಳಬೇಕು. ಋಷಿದರ್ಶನವನ್ನು ಆಧಾರವಾಗಿ ಸ್ವೀಕರಿಸಿ ನಿಂತ ನಮ್ಮಲ್ಲಿ ಪ್ರತಿಯೊಬ್ಬರೂ ಮಹಾವ್ಯಕ್ತಿಗಳಾಗುವೆವು. ಆಗ ನಮ್ಮ ಬಾಯಿಂದ ಹೊರಬರುವ ಪ್ರತಿ ನುಡಿಯೂ ಸಂರಕ್ಷಣೆಯ ಅನಂತ ಭರವಸೆಯನ್ನು ಹೊತ್ತಿರುತ್ತದೆ. ಆಗ ಯಾರನ್ನೂ ಶಪಿಸುವ ಆವಶ್ಯಕತೆಯಿಲ್ಲದೆ ಇರುವುದರಿಂದ, ಯಾವ ನಿಂದನೆಯ ಕಾದಾಟದ ಆವಶ್ಯಕತೆಯೂ ಇಲ್ಲದೆ ಇರುವುದರಿಂದ, ಪಾಪವು ತನಗೆ ತಾನೇ ಮಾಯವಾಗುವುದು. ನಮ್ಮ ಮತ್ತು ಇತರರ ಉದ್ಧಾರಕ್ಕಾಗಿ ನಾವು ಋಷಿಗಳಾಗುವಂತೆ ಭಗವಂತ ನಮ್ಮೆಲ್ಲರನ್ನೂ ಆಶೀರ್ವದಿಸಲಿ!

