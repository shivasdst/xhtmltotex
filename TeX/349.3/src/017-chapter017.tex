
\chapter{ಕಲ್ಕತ್ತೆಯ ಬಿನ್ನವತ್ತಳೆಗೆ ಉತ್ತರ}

ಸ್ವಾಮಿ ವಿವೇಕಾನಂದರು ಕಲಕತ್ತೆಗೆ ಬಂದಕೂಡಲೇ ಅವರಿಗೆ ಅತ್ಯಂತ ಉತ್ಸಾಹಪೂರ್ಣವಾದ ಸ್ವಾಗತವನ್ನು ನೀಡಲಾಯಿತು. ಅಲಂಕಾರಗೊಂಡ ನಗರದ ರಸ್ತೆಗಳ ಮೂಲಕ ಅವರು ಸಾಗಿಹೋಗುವುದೇ ಕಷ್ಟವಾಗುವಷ್ಟು ಭಾರೀ ಜನ ಸಮೂಹವು ಅವರ ದರ್ಶನವನ್ನು ಪಡೆಯಲೆಂದು ಕಿಕ್ಕಿರಿದು ತುಂಬಿತ್ತು. ಅವರಿಗೆ ನೀಡಬೇಕಾಗಿದ್ದ ಅಧಿಕೃತ ಸ್ವಾಗತವು ಒಂದು ವಾರದ ಅನಂತರ ಶೋಭಾ ಬಜಾರಿನಲ್ಲಿರುವ ರಾಜಾ ರಾಧಾಕಾಂತ ದೇವಬಹದ್ದೂರ್​ ಅವರ ಮನೆಯಲ್ಲಿ ರಾಜಾ ವಿನಯಕೃಷ್ಣ ದೇವಬಹದ್ದೂರ್​ ಅವರ ಅಧ್ಯಕ್ಷತೆಯಲ್ಲಿ ನೆರ\-ವೇರಿತು. ಅಧ್ಯಕ್ಷರು ಪರಿಚಯಾತ್ಮಕವಾಗಿ ಕೆಲವು ಮಾತುಗಳನ್ನು ಆಡಿದ ಮೇಲೆ ಈ ಕೆಳಗಿನ ಬಿನ್ನವತ್ತಳೆಯನ್ನು ಬೆಳ್ಳಿಯ ಸಂಪುಟದಲ್ಲಿಟ್ಟು ಸ್ವಾಮೀಜಿಯವರಿಗೆ ಅರ್ಪಿಸಿದರು.

\textbf{ಶ‍್ರೀಮತ್​ ಸ್ವಾಮಿ ವಿವೇಕಾನಂದರಿಗೆ}

\textbf{ಪ್ರಿಯ ಸೋದರರೆ,}

ಕಲಕತ್ತೆಯ ಮತ್ತು ಬಂಗಾಳದ ಇತರ ಕೆಲವು ಪ್ರದೇಶಗಳ ಹಿಂದೂನಿವಾಸಿಗಳು ತಮ್ಮ ಈ ಜನ್ಮ ಸ್ಥಳಕ್ಕೆ ತಮ್ಮನ್ನು ಹೃತ್ಪೂರ್ವಕವಾಗಿ ಸ್ವಾಗತಿಸುತ್ತೇವೆ. ಈ ಸಂದರ್ಭದಲ್ಲಿ ನಮಗೆ ಹೆಮ್ಮೆ ಮತ್ತು ಕೃತಜ್ಞತಾಭಾವಗಳೆರಡೂ ಉಕ್ಕಿ ಹರಿಯುತ್ತಿವೆ. ಏಕೆಂದರೆ ತಾವು ಜಗತ್ತಿನ ವಿವಿಧ ಭಾಗಗಳಲ್ಲಿ ಮಾಡಿದ ಕಾರ್ಯಗಳಿಂದಲೂ, ನಡೆಸಿದ ಆದರ್ಶಮಯ ಜೀವನದಿಂದಲೂ ನಮ್ಮ ಧರ್ಮಕ್ಕೆ ಮಾತ್ರವಲ್ಲದೆ ನಮ್ಮ ರಾಷ್ಟ್ರಕ್ಕೂ ಅದರಲ್ಲಿಯೂ ನಮ್ಮ ಪ್ರಾಂತಕ್ಕೂ ಗೌರವವನ್ನು ತಂದಿದ್ದೀರಿ.

ವಿಶ್ವಮೇಳದ ಅಂಗವಾಗಿ ಚಿಕಾಗೊ ಪಟ್ಟಣದಲ್ಲಿ ೧೮೯೩ರಲ್ಲಿ ನಡೆದ ಸರ್ವಧರ್ಮ ಸಮ್ಮೇಳನದಲ್ಲಿ ತಾವು ಆರ್ಯಧರ್ಮದ ತತ್ತ್ವಗಳನ್ನು ನಿವೇದಿಸಿದಿರಿ. ಅಲ್ಲಿ ನೆರೆದಿದ್ದ ಶ್ರೋತೃಗಳಲ್ಲಿ ಬಹು ಮಂದಿಗೆ ತಾವು ನೀಡಿದ ಭಾಷಣದ ವಿಷಯವು ಹೊಸದಾಗಿತ್ತು. ತಮ್ಮ ಭಾಷಣದ ಶೈಲಿಯಂತೂ ತನ್ನ ಶಕ್ತಿ ಗಾಂಭೀರ್ಯಗಳಿಂದ ಶ್ರೋತೃಗಳ ಮನಸ್ಸನ್ನು\break ಸೆರೆಹಿಡಿಯುವುದಾಗಿತ್ತು. ಕೆಲವರು ತಮ್ಮ ಉಪನ್ಯಾಸವನ್ನು ಪ್ರಶ್ನಿಸುವ ಮನೋಭಾವದಿಂದಲೇ ಕೇಳಿರಬಹುದು. ಕೆಲವರು ಅದನ್ನು ಟೀಕಿಸಿರಲೂ ಬಹುದು. ಆದರೆ ಒಟ್ಟಾರೆ ಅದರ ಪರಿಣಾಮವೆಂದರೆ ಬಹುಪಾಲು ಸುಸಂಸ್ಕೃತ ಅಮೆರಿಕನ್ನರ ಧಾರ್ಮಿಕ ಭಾವನೆಗಳಲ್ಲಿ ಅದು ಒಂದು ಕ್ರಾಂತಿಯನ್ನು ಉಂಟು ಮಾಡಿತು. ಅವರ ಮತಿಗಳಲ್ಲಿ ಒಂದು ಹೊಸ ಬೆಳಕನ್ನು ಬೆಳಗಿತು. ತಮಗೆ ಸಹಜವಾದ ಪ್ರಾಮಾಣಿಕತೆ ಮತ್ತು ಸತ್ಯ ಪ್ರೇಮಗಳಿಂದ ಅವರು ತಮ್ಮ ಉಪನ್ಯಾಸದ ಪೂರ್ಣ ಪ್ರಯೋಜನವನ್ನು ಪಡೆಯಲು ನಿರ್ಧರಿಸಿದರು. ತಮ್ಮ ಅವಕಾಶಗಳ ವ್ಯಾಪ್ತಿಯೂ ಕಾರ್ಯಕ್ಷೇತ್ರವೂ ವಿಸ್ತಾರವಾಯಿತು. ಅಮೆರಿಕಾದ ಅನೇಕ ಪ್ರಾಂತಗಳ, ಅನೇಕ ನಗರಗಳ ಆಹ್ವಾನಗಳನ್ನು ತಾವು ಸ್ವೀಕರಿಸಬೇಕಾಯಿತು. ಅಲ್ಲಿ ಅನೇಕ ಪ್ರಶ್ನೆಗಳಿಗೆ ಉತ್ತರ ಕೊಡಬೇಕಾಯಿತು. ಅನೇಕ ಸಂದೇಹಗಳನ್ನು ನಿವಾರಿಸ\-ಬೇಕಾಯಿತು. ಅನೇಕ ರೀತಿಯ ಕಷ್ಟಗಳಿಗೆ ಪರಿಹಾರಗಳನ್ನು ಸೂಚಿಸಬೇಕಾಯಿತು. ಇವೆಲ್ಲವನ್ನೂ ತಾವು ಶಕ್ತಿ, ಸಾಮರ್ಥ್ಯ ಮತ್ತು ಪ್ರಾಮಾಣಿಕತೆಗಳ ಬಲದಿಂದ ಮಾಡಿದ್ದೀರಿ. ಅದರ ಫಲವಾಗಿ ಬಹಳ ಕಾಲ ನಿಲ್ಲುವ ಪರಿಣಾಮಗಳು ಕಾಣಿಸಿಕೊಂಡಿವೆ. ಅನೇಕ ಜನ ವಿದ್ಯಾವಂತರ ಮೇಲೆ ತಮ್ಮ ಬೋಧನೆಯು ಆಳವಾದ ಪ್ರಭಾವವನ್ನು ಬೀರಿದೆ. ಅಲ್ಲಿಯ ಚಿಂತನೆ ಸಂಶೋಧನೆಗಳು ಹೊಸಹಾದಿಯನ್ನು ಹಿಡಿಯುವಂತೆ ಮಾಡಿದೆ. ಅಲ್ಲದೆ ಅನೇಕ ಸಂದರ್ಭಗಳಲ್ಲಿ ಅಮೆರಿಕನ್ನರು ಹಿಂದೂ ಆದರ್ಶಗಳನ್ನು ಹೆಚ್ಚು ಹೆಚ್ಚಾಗಿ ಮೆಚ್ಚುವಂತೆ ಮಾಡಿದೆ. ವಿವಿಧ ಧರ್ಮಗಳನ್ನು ತುಲನಾತ್ಮಕವಾಗಿ ಅಧ್ಯಯನ ಮಾಡುವುದಕ್ಕೂ, ಧಾರ್ಮಿಕ ಸತ್ಯವನ್ನು ಪರೀಕ್ಷಿಸುವುದಕ್ಕೂ ಸಂಘ ಸಂಸ್ಥೆಗಳು ಹೆಚ್ಚಿನ ಸಂಖ್ಯೆಯಲ್ಲಿ ಸ್ಥಾಪಿತ\-ಗೊಂಡುದು, ತಾವು ಪಶ್ಚಿಮದಲ್ಲಿ ಮಾಡಿದ ಕಾರ್ಯಕ್ಕೆ ಸಾಕ್ಷಿಯಾಗಿದೆ. ಲಂಡನ್ನಿನಲ್ಲಿ ವೇದಾಂತ ಬೋಧನೆಗಾಗಿ ತೆರೆದ ಒಂದು ಕಾಲೇಜಿನ ಸ್ಥಾಪಕರು ತಾವು, ಎಂದು ಭಾವಿಸ ಬಹುದಾಗಿದೆ. ತಾವು ಕ್ಲುಪ್ತವಾಗಿ ನೀಡಿದ ಉಪನ್ಯಾಸಗಳಿಗೆ ಶ್ರೋತೃಗಳು ಸಕಾಲಕ್ಕೆ ಆಗಮಿಸಿ ಅವುಗಳನ್ನು ಬಹುವಾಗಿ ಮೆಚ್ಚಿದ್ದಾರೆ. ಅವುಗಳ ಪ್ರಭಾವವು ಉಪನ್ಯಾಸದ ತರಗತಿಗಳ ಗೋಡೆಗಳ ಪರಿಮಿತಿಯನ್ನು ದಾಟಿ ವ್ಯಾಪಕವಾಗಿದೆ. ತಾವು ಲಂಡನ್ನನ್ನು ಬಿಡುವ ಮುಂಚೆ, ಆ ನಗರದ ವೇದಾಂತ ತತ್ತ್ವದ ವಿದ್ಯಾರ್ಥಿಗಳು ತಮಗೆ ನೀಡಿದ ಬಿನ್ನವತ್ತಳೆಯು ತಮ್ಮ ಬೋಧನೆಯಿಂದ ತಾವು ಗಳಿಸಿಕೊಂಡ ಪ್ರೇಮ ಗೌರವಗಳಿಗೆ ಸಾಕ್ಷಿಯಾಗಿದೆ.

ತಾವು ಗುರುಗಳಾಗಿ ಪಡೆದ ಯಶಸ್ಸಿಗೆ, ಆರ್ಯಧರ್ಮದ ಸತ್ಯಗಳ ವಿಷಯವಾಗಿ ತಮಗಿರುವ ಆಳವೂ ಆತ್ಮೀಯವೂ ಆದ ಜ್ಞಾನವೂ, ಮಾತು ಬರಹಗಳ ಮೂಲಕ ಅದನ್ನು ವಿವರಿಸುವ ಕೌಶಲವೂ ಮಾತ್ರವಲ್ಲದೆ ತಮ್ಮ ವ್ಯಕ್ತಿತ್ವವೂ ಕಾರಣವಾಗಿದೆ. ತಮ್ಮ ಉಪನ್ಯಾಸಗಳು, ಪ್ರಬಂಧಗಳು ಮತ್ತು ಪುಸ್ತಕಗಳು ಉನ್ನತವಾದ ಆಧ್ಯಾತ್ಮಿಕ ಮತ್ತು ಸಾಹಿತ್ಯದ ಗುಣಗಳಿಂದ ಕೂಡಿರುವುದರಿಂದ ಒಳ್ಳೆಯ ಪ್ರಭಾವವನ್ನು ಬೀರಿರುವುದು ಸಹಜವಾಗಿಯೇ ಇದೆ. ಆದರೆ ಅದು ಯಾರೂ ಸಮರ್ಪಕವಾಗಿ ವಿವರಣೆ ನೀಡಲಾರದಷ್ಟು ಮಟ್ಟಿನ ಔನ್ನತ್ಯವನ್ನು ಗಳಿಸಿಕೊಂಡಿರುವುದಕ್ಕೆ ಕಾರಣ ಆದರ್ಶಪ್ರಾಯವಾದ ತಮ್ಮ ಸರಳ, ಪ್ರಾಮಾಣಿಕ, ನಿಸ್ವಾರ್ಥ ಜೀವನ ಮತ್ತು ತಮ್ಮ ವಿನಯ, ಭಕ್ತಿ ಮತ್ತು ಉತ್ಸುಕತೆಗಳು.

ನಮ್ಮ ಧರ್ಮದ ಮಹೋನ್ನತ ಸತ್ಯಗಳನ್ನು ಬೋಧಿಸುವ ಗುರುವಾಗಿ ತಾವು ಸಲ್ಲಿಸಿರುವ ಸೇವೆಗೆ ಕೃತಜ್ಞತೆಯನ್ನು ಸಮರ್ಪಿಸುವ ಸಮಯದಲ್ಲೇ ನಾವು ತಮ್ಮ ಪೂಜ್ಯ ಗುರುಗಳಾದ ಶ‍್ರೀರಾಮಕೃಷ್ಣರ ಸ್ಮೃತಿಗೂ ಗೌರವವನ್ನು ಅರ್ಪಿಸುವುದು ಅಗತ್ಯವೆಂದು ಭಾವಿಸುತ್ತೇವೆ. ತಾವೂ ಕೂಡ ನಮಗೆ ಲಭ್ಯವಾದದ್ದು ಅವರಿಂದಲೇ. ಅಪೂರ್ವವಾದ ಮಾಂತ್ರಿಕ ಅಂತರ್ದೃಷ್ಟಿಯಿಂದ ಅವರು ತಮ್ಮಲ್ಲಿದ್ದ ದಿವ್ಯಜ್ಯೋತಿಯನ್ನು ಬಹುಬೇಗ ಗುರುತಿಸಿದರು. ತಾವು ಮುಂದೆ ಮಾಡಲಿರುವ ಸಾಧನೆಯನ್ನು ಕುರಿತು ಭವಿಷ್ಯ ವಾಣಿಯನ್ನು ನುಡಿದಿದ್ದರು. ಈಗ ಅದು ವಾಸ್ತವವಾಗುತ್ತಿದೆ. ಭಗವಂತನು ತಮಗೆ ಕರುಣಿಸಿರುವ ದರ್ಶನವನ್ನೂ ದಿವ್ಯಶಕ್ತಿಯನ್ನೂ ಅನಾವರಣ ಗೊಳಿಸಿರುವರು ಅವರೇ, ಅಲ್ಲದೆ ತಮ್ಮ ಚಿಂತನೆಗಳಿಗೂ ಆಶೋತ್ತರಗಳಿಗೂ ಅಗತ್ಯವಾದ ದಿವ್ಯಸ್ಪರ್ಶವನ್ನೂ ನೀಡಿದವರು ಅವರೇ. ಇನ್ನೂ ಅದೃಶ್ಯವಾಗಿದ್ದ ವಲಯಗಳಲ್ಲಿ ತಾವು ಕೈಗೊಂಡ ಸಾಹಸಗಳಿಗೆ ಅಗತ್ಯವಾದ ನೆರವು ದೊರೆತದ್ದೂ ಅವರಿಂದಲೇ. ಮುಂದಿನ ಜನಾಂಗಗಳಿಗೆ ಅವರು ನೀಡಿರುವ ಅತ್ಯಮೂಲ್ಯವಾದ ಆಸ್ತಿಯೆಂದರೆ ತಾವೇ. ಓ ದಿವ್ಯ ಚೇತನವೇ, ತಾವು ಆರಿಸಿಕೊಂಡಿರುವ ಪಥದಲ್ಲಿ ದೃಢವಾಗಿ ಧೈರ್ಯವಾಗಿ ಕಾರ್ಯೋನ್ಮುಖರಾಗಿ ಮುಂದುವರಿಯಿರಿ. ತಾವು ಗೆಲ್ಲಬೇಕಾದ ಜಗತ್ತೊಂದಿದೆ. ಅಜ್ಞಾನಿಗಳಿಗೂ ಅಪನಂಬಿಕೆಯುಳ್ಳವರಿಗೂ ಬೇಕೆಂದೇ ಕುರುಡರಾಗಿರುವವರಿಗೂ ಹಿಂದೂಧರ್ಮವನ್ನು ವ್ಯಾಖ್ಯಾನಿಸಿ ಅದರ ಸತ್ಯವನ್ನು ಸ್ಥಾಪಿಸಬೇಕಾಗಿದೆ. ನಮ್ಮ ಮೆಚ್ಚುಗೆಯನ್ನು ಸೆಳೆದುಕೊಳ್ಳುವಂತಹ ಮನೋಭಾವದಿಂದ ತಾವು ಕಾರ್ಯಾರಂಭ ಮಾಡಿದ್ದೀರಿ ಮತ್ತು ಅದರಲ್ಲಿ ತಾವು ಸಾಧಿಸಿರುವ ಈ ಯಶಸ್ಸಿಗೆ ಅನೇಕ ರಾಷ್ಟ್ರಗಳು ಸಾಕ್ಷಿಯಾಗಿವೆ. ಆದರೆ ಇನ್ನೂ ಮಾಡಬೇಕಾದ ಕಾರ್ಯ ಅಪಾರವಾಗಿದೆ. ನಮ್ಮ ದೇಶ, ವಾಸ್ತವವಾಗಿ ತಮ್ಮ ದೇಶವೆಂದೇ ಹೇಳಬೇಕು, ತಮ್ಮ ಸೇವೆಗೆ ಸಿದ್ಧವಾಗಿದೆ. ಹಿಂದೂ ಧರ್ಮದ ಸತ್ಯಗಳನ್ನು ಅಪಾರ ಸಂಖ್ಯೆಯ ಹಿಂದೂಗಳಿಗೇ ವಿವರಿಸಿ ಹೇಳಬೇಕಾಗಿದೆ. ಆದ್ದರಿಂದ ಶ್ರಮಸಾಧ್ಯವಾದ ಆ ಕಾರ್ಯವನ್ನು ಮಾಡಲು ತಾವು ಸಿದ್ಧರಾಗಿ. ತಮ್ಮಲ್ಲೂ ತಾವು ಕೈಗೊಂಡಿರುವ ಕಾರ್ಯವು ಸರಿಯಾದುದ್ದು ಎಂಬ ವಿಷಯದಲ್ಲೂ ನಮಗೆ ನಂಬಿಕೆಯಿದೆ. ನಮ್ಮ ರಾಷ್ಟ್ರೀಯ ಧರ್ಮವು ಯಾವ ಲೌಕಿಕ ಜಯಗಳನ್ನು ಗಳಿಸುವುದರಲ್ಲಿಯೂ ಆಸಕ್ತಿಯನ್ನು ಹೊಂದಿಲ್ಲ. ಅದರ ಉದ್ದೇಶಗಳು ಆಧ್ಯಾತ್ಮಿಕವಾದವು. ಅದರ ಉಪಕರಣವೆಂದರೆ ಲೌಕಿಕ ದೃಷ್ಟಿಗೆ ಮರೆಯಾಗಿರುವ ಸತ್ಯ. ಆ ಸತ್ಯವು ವಿಚಾರಶೀಲ ತರ್ಕಕ್ಕೆ ಮಾತ್ರ ಗೋಚರವಾದುದು. ಜಗತ್ತು ಅಥವಾ ಹಿಂದೂಗಳು ತಮ್ಮ ಒಳಗಣ್ಣನ್ನು ತೆರೆಯುವಂತೆ, ತಮ್ಮ ಇಂದ್ರಿಯಗಳನ್ನು ಮೀರುವಂತೆ, ನಮ್ಮ ಶಾಸ್ತ್ರಗಳನ್ನು ಸರಿಯಾಗಿ ಓದುವಂತೆ, ಪರಮ ಸತ್ಯವನ್ನು ಮುಖಾಮುಖಿಯಾಗಿ ನೋಡುವಂತೆ, ತಮ್ಮ ಸ್ಥಿತಿಗತಿಗಳನ್ನೂ ಗುರಿಯನ್ನೂ ಮಾನವರಾಗಿ ಅರ್ಥಮಾಡಿಕೊಳ್ಳುವಂತೆ ತಾವು ಕರೆ ನೀಡಿ. ಜನರನ್ನು ಜಾಗ್ರತಗೊಳಿಸುವುದಕ್ಕೆ ಅಥವಾ ಅವರಿಗೆ ಕರೆನೀಡುವುದಕ್ಕೆ ತಮಗಿಂತ ಹೆಚ್ಚು ಸಮರ್ಥರಾದವರು ಯಾರೂ ಇಲ್ಲ. ವಿಧಿಯು ನಿರ್ಣಯಿಸಿರುವ ತಮ್ಮ ಮಹತ್ಕಾರ್ಯದಲ್ಲಿ ಅಗತ್ಯವಾದ ಹೃತ್ಪೂರ್ವಕ ಸಹಾನುಭೂತಿಯನ್ನೂ ನಿಷ್ಠಾಪೂರ್ವಕವಾದ ಸಹಕಾರವನ್ನೂ ತಮಗೆ ನಾವು ನೀಡುತ್ತೇವೆ ಎಂದು ಭರವಸೆ ಕೊಡುತ್ತೇವೆ.

\begin{longtable}[r]{@{}c@{}}
ಪ್ರಿಯ ಸೋದರರೇ \\
ತಮ್ಮ ಪ್ರೀತಿಯ ಮಿತ್ರರು ಮತ್ತು ಅಭಿಮಾನಿಗಳು. \\
\end{longtable}

ಸ್ವಾಮೀಜಿ ಹೀಗೆ ಉತ್ತರ ನೀಡಿದರು: ವ್ಯಕ್ತಿಯು ಸಮಷ್ಟಿಯಲ್ಲಿ ಲೀನವಾಗಲು ಬಯಸುತ್ತಾನೆ. ಅವನು ಜಗತ್ತನ್ನು ತ್ಯಜಿಸುತ್ತಾನೆ. ಅಲ್ಲಿಂದ ದೂರಸರಿದು ತನ್ನ ದೇಹಕ್ಕೆ ಮತ್ತು ಹಿಂದಿನ ಜೀವನಕ್ಕೆ ಇರುವ ಸಂಬಂಧಗಳನ್ನೆಲ್ಲಾ ಹರಿದು ಹಾಕಲು ಪ್ರಯತ್ನಿಸುತ್ತಾನೆ. ತಾನು ಮನುಷ್ಯ ಎಂಬುದನ್ನು ಕೂಡ ಮರೆಯಲು ಕಷ್ಟಪಟ್ಟು ಸಾಧನೆ ಮಾಡುವನು. ಆದರೆ ಅವನ ಹೃದಯಾಂತರಾಳದಲ್ಲಿ ಒಂದು ಮೃದು ಅಸ್ಫುಟ ಧ್ವನಿ, ಒಂದು ಸ್ವರ, ಒಂದು ಮೆಲುನುಡಿ – ಪ್ರಾಚ್ಯವೋ ಪಾಶ್ಚಾತ್ಯವೋ – \textbf{“ಜನನೀ ಜನ್ಮಭೂಮಿಶ್ಚ ಸ್ವರ್ಗಾದಪಿ ಗರೀಯಸೀ”} (ತಾಯಿಯೂ ಜನ್ಮಭೂಮಿ ಯಾ ಸ್ವರ್ಗಕ್ಕಿಂತ ಮಿಗಿಲಾದುದು) ಎಂಬುದು ಕೇಳಿಬರುತ್ತದೆ. ಈ ಚಕ್ರಾಧಿಪತ್ಯದ ರಾಜಧಾನಿಯ ಪುರಜನರೇ, ನಿಮ್ಮೆದುರಿಗೆ ನಾನು ಸಂನ್ಯಾಸಿಯಂತೆ ನಿಂತಿಲ್ಲ, ಪ್ರಚಾರಕನಂತೆ ನಿಂತಿಲ್ಲ, ಹಿಂದೆ ನಿಮ್ಮೊಂದಿಗೆ ಮಾತನಾಡುತ್ತಿದ್ದ ಕಲ್ಕತ್ತಾ ನಗರದ ಹುಡುಗನಂತೆ ಬಂದಿರುವೆನು. ನನ್ನ ಸಹೋದರರೇ, ಈ ನಗರದ ದಾರಿಯ ಧೂಳಿನಲ್ಲಿ ಕುಳಿತು ಶಿಶುಸಹಜ ಸ್ವಾತಂತ್ರ್ಯದಿಂದ ನನ್ನ ಮನಸ್ಸಿನ ಭಾವವನ್ನು ನಿಮ್ಮೆದುರಿಗೆ ವ್ಯಕ್ತಪಡಿಸುವೆನು. ನೀವು ನನ್ನನ್ನು ಸಹೋದರ ಎಂಬ ಅಪೂರ್ವ ಪದದಿಂದ ಸಂಬೋಧಿಸಿರುವಿರಿ. ಅದಕ್ಕೆ ನನ್ನ ಹೃತ್ಪೂರ್ವಕ ವಂದನೆಗಳು. ಹೌದು, ನಾನು ನಿಮ್ಮ ಸಹೋದರ, ನೀವು ನನ್ನ ಸಹೋದರರು. ನಾನು ಭರತಖಂಡಕ್ಕೆ ಬರುವುದಕ್ಕೆ ಮುಂಚೆ ಒಬ್ಬ ಆಂಗ್ಲ ಸ್ನೇಹಿತ, “ಸ್ವಾಮೀಜಿ, ನಾಲ್ಕು ವರ್ಷಗಳವರೆಗೆ ವಿಲಾಸದ ಲೀಲಾ ಭೂಮಿಯಾದ, ಗೌರವಶೀಲ ಮಹಾಶಕ್ತಿಮಾನ್​ ಪಾಶ್ಚಾತ್ಯ ದೇಶದಲ್ಲಿ ಸಂಚಾರ ಮಾಡಿದ ಮೇಲೆ ನಿಮ್ಮ ಮಾತೃಭೂಮಿಯನ್ನು ಹೇಗೆ ನೋಡುತ್ತೀರಿ?” ಎಂದು ಕೇಳಿದರು. ಅದಕ್ಕೆ ನಾನು “ಇಲ್ಲಿಗೆ ಬರುವುದಕ್ಕೆ ಮುಂಚೆ ಭರತಖಂಡವನ್ನು ಪ್ರೀತಿಸುತ್ತಿದ್ದೆ. ಈಗ ಭಾರತಭೂಮಿಯ ಧೂಳು ಕೂಡ ಪವಿತ್ರವಾಗಿದೆ, ಬೀಸುವ ಗಾಳಿ ಪವಿತ್ರವಾಗಿದೆ, ಪುಣ್ಯಭೂಮಿ ತೀರ್ಥಕ್ಷೇತ್ರವಾಗಿದೆ” ಎಂದೆನು.

ಹೇ ಕಲ್ಕತ್ತ ಪುರನಿವಾಸಿಗಳೇ, ನನ್ನ ಸಹೋದರರೇ, ನೀವು ನನಗೆ ತೋರಿದ ಪ್ರೀತಿಗೆ ನಿಮಗೆ ಕೃತಜ್ಞತೆಯನ್ನು ಅರ್ಪಿಸಲು ನಾನು ಅಸಮರ್ಥನು. ನಾನು ನಿಮಗೆ ಕೃತಜ್ಞತೆಯನ್ನು ಅರ್ಪಿಸಲೂ ಬಾರದು, ನೀವು ನನಗೆ ಸಹೋದರರು, ನೀವು ಕೇವಲ ಸಹೋದರನ ಕರ್ತವ್ಯವನ್ನು, ಹಿಂದೂ ಸಹೋದರನ ಕರ್ತವ್ಯವನ್ನು ಮಾತ್ರ ಮಾಡಿರುವಿರಿ. ಇಂತಹ ಕೌಟುಂಬಿಕ ವಾತ್ಸಲ್ಯದ ಸಂಬಂಧ, ಇಷ್ಟು ಪ್ರೀತಿ ನಮ್ಮ ಮಾತೃಭೂಮಿಯ ಹೊರಗೆ ಮತ್ತೆಲ್ಲಿಯೂ ಇಲ್ಲ.

ಚಿಕಾಗೊ ವಿಶ್ವಧರ್ಮ ಸಮ್ಮೇಳನ ನಿಸ್ಸಂಶಯವಾಗಿ ಒಂದು ವಿರಾಟ್​ ಅಧಿವೇಶನವೇ ಸರಿ. ಇದನ್ನು ಸಂಘಟಿಸಿದವರಿಗೆ ಭರತವರ್ಷದ ಎಲ್ಲಾ ನಗರಗಳಿಂದಲೂ ಧನ್ಯವಾದವನ್ನು ಅರ್ಪಿಸಿರುವರು. ಅವರು ನಮಗೆ ತೋರಿದ ಪ್ರೀತಿಗೆ, ಧನ್ಯವಾದಕ್ಕೆ ಆರ್ಹರು. ಆದರೂ ವಿಶ್ವಧರ್ಮ ಸಮ್ಮೇಳನ ಏತಕ್ಕೆ ಆಯಿತು ಎಂಬುದನ್ನು ನಿಮಗೆ ತಿಳಿಸಲು ಅನುಮತಿ ಕೊಡಿ. ಅವರಿಗೆ ಒಂದು ಕುದುರೆ ಬೇಕಾಗಿತ್ತು. ಅದರ ಮೇಲೆ ಸವಾರಿ ಮಾಡಬೇಕೆಂದು ಇಚ್ಛಿಸಿದರು. ಕ್ರೈಸ್ತಧರ್ಮದ ಪ್ರತಿಷ್ಠೆಗೆ ಇದನ್ನು ಉಪಯೋಗಿಸಬೇಕೆಂದು ಕೆಲವರು ಯೋಚಿಸಿದ್ದರು. ವಿಧಾತ ಅದನ್ನು ಬೇರೆ ವಿಧವಾಗಿ ಇಚ್ಛಿಸಿದ. ಬೇರೆ ವಿಧಿಯೇ ಇರಲಿಲ್ಲ. ಅಲ್ಲಿ ಹಲವರು ನಮಗೆ ಪ್ರೀತಿ ತೋರಿದರು. ನಾವು ಅವರಿಗೆ ಸಾಕಾದಷ್ಟು ಧನ್ಯವಾದವನ್ನು ಅರ್ಪಿಸಿರುವೆವು.

ನಾನು ಅಮೆರಿಕಾ ದೇಶಕ್ಕೆ ಹೋದದ್ದು ವಿಶ್ವಧರ್ಮ ಸಮ್ಮೇಳನಕ್ಕೆ ಅಲ್ಲ. ಆ ಸಭೆ ಗೌಣ. ಅದು ಕೇವಲ ನಿಮಿತ್ತ; ಅದು ಒಂದು ಅವಕಾಶ ಕಲ್ಪಿಸಿತು. ಅದಕ್ಕೆ ನಾವು ಆ ಧರ್ಮಸಮ್ಮೇಳನದ ಸದಸ್ಯರಿಗೆ ಋಣಿಗಳು. ಆದರೆ ನಿಜವಾಗಿ ನಮ್ಮ ಕೃತಜ್ಞತೆ ಸಲ್ಲಬೇಕಾಗಿರುವುದು ಅಮೆರಿಕಾದ ಮಹಾನ್​ ಜನತೆಗೆ, ಅಮೆರಿಕಾ ಜನಾಂಗಕ್ಕೆ. ಉಳಿದ ಎಲ್ಲಾ\break ಜನಾಂಗಗಳಿಗಿಂತಲೂ ಅಧಿಕಾಂಶದಲ್ಲಿ ಸಹೋದರ ಭಾವನೆ ಅಲ್ಲಿ ಬೆಳೆದಿದೆ. ಅಮೆರಿಕಾ ಜನಾಂಗದವರು ವಿಶ್ವಾಸಪರರು, ಅತಿಥಿ ಸತ್ಕಾರಪರರು. ರೈಲಿನಲ್ಲಿ ಪ್ರಯಾಣ ಮಾಡುವಾಗ ಅಮೆರಿಕಾದವನು ನಿಮ್ಮನ್ನು ಐದು ನಿಮಿಷ ನೋಡುವನು. ನೀವಾಗಲೇ ಅವನ ಸ್ನೇಹಿತರಾಗುವಿರಿ. ಮರುಕ್ಷಣವೇ ಅವನು ನಿಮ್ಮನ್ನು ಅವನ ಮನೆಗೆ ಊಟಕ್ಕೆ ಕರೆಯುವನು. ತನ್ನ ಇಡೀ ಜೀವನದ ರಹಸ್ಯವನ್ನು ನಿಮ್ಮೆದುರಿಗೆ ಇಡುವನು. ಇದು ಅಮೆರಿಕಾ ಜನರ ಲಕ್ಷಣ, ನಾವು ಇದನ್ನು ಹೆಚ್ಚು ಮೆಚ್ಚುತ್ತೇವೆ. ಅವರು ನನ್ನ ಅಪೂರ್ವ ದಯೆಯಿಂದ ನೋಡಿರುವರು. ಅದನ್ನು ವರ್ಣಿಸಲು ಹಲವು ವರ್ಷಗಳು ಹಿಡಿಯುವುವು. ಅಟ್ಲಾಂಟಿಕ್​ ಸಾಗರದ ಆಚೆ ಕಡೆ ಇರುವ ಮತ್ತೊಂದು ದೇಶಕ್ಕೂ ನಮ್ಮ ಕೃತಜ್ಞತೆ ಸಲ್ಲಬೇಕು. ತನ್ನ ಹೃದಯದಲ್ಲಿ ಇಂಗ್ಲಿಷ್​ ಜನಾಂಗದ ಬಗ್ಗೆ ನನಗಿಂತ ಹೆಚ್ಚು ದ್ವೇಷವನ್ನು ಹೊತ್ತು ಆ ನೆಲದ ಮೇಲೆ ಹೆಜ್ಜೆಯಿಟ್ಟಿರುವ ಮತ್ತೊಬ್ಬರಿರಲಾರರು. ಇಂದಿನ ವೇದಿಕೆಯ ಮೇಲೆ ಇದನ್ನು ಸಮರ್ಥಿಸುವ ಆಂಗ್ಲ ಸ್ನೇಹಿತರು ಕೆಲವರು ಇರುವರು. ಆದರೆ ಅವರೊಂದಿಗೆ ಹೆಚ್ಚು ವಾಸಮಾಡಿದ ಮೇಲೆ, ಆಂಗ್ಲ ಜನಾಂಗ ಜೀವನದ ಯಂತ್ರ ಹೇಗೆ ಕೆಲಸಮಾಡುತ್ತಿದೆ ಎಂಬುದನ್ನು ನೋಡಿದ ಮೇಲೆ, ಅವರೊಂದಿಗೆ ನಿಕಟವಾಗಿ ಬೆರೆತ ಮೇಲೆ, ಅವರ ಜನಾಂಗದ ಜೀವಾಳವನ್ನು ಅರಿತ ಮೇಲೆ ಅವರನ್ನು ಅಷ್ಟೂ ಹೆಚ್ಚು ಪ್ರೀತಿಸತೊಡಗಿದೆನು. ನನ್ನ ಸಹೋದರರೇ, ಈ ಸಭೆಯಲ್ಲಿರುವ ಯಾರೂ ಆಂಗ್ಲೇಯರನ್ನು ನನ್ನಷ್ಟು ಪ್ರೀತಿಸಲಾರಿರಿ. ನೀವು ಅಲ್ಲಿ ಏನು ಆಗುತ್ತಿದೆ ಎಂಬುದನ್ನು ನೋಡಬೇಕು, ಅವರೊಡನೆ ಬೆರೆಯಬೇಕು. ನಮ್ಮ ದೇಶದ ವೇದಾಂತ ತತ್ತ್ವವು ನಮ್ಮ ದುಃಖ ದುರ್ಗತಿಗಳಿಗೆಲ್ಲಾ ಅಜ್ಞಾನ ಮೂಲವೆಂದು ಸಾರುವುದು. ಆಂಗ್ಲೇಯರಿಗೂ ನಮಗೂ ಇರುವ ಭಿನ್ನತೆಗೆ ಕಾರಣ ಅಜ್ಞಾನ. ನಾವು ಅವರನ್ನು ತಿಳಿದುಕೊಂಡಿಲ್ಲ. ಅವರು ನಮ್ಮನ್ನು ತಿಳಿದುಕೊಂಡಿಲ್ಲ.

ಪಾಶ್ಚಾತ್ಯರಿಗೆ, ಅಧ್ಯಾತ್ಮ ಮತ್ತು ನೀತಿ ಕೂಡ ಪ್ರಾಪಂಚಿಕ ಉನ್ನತಿಯ ಮೇಲೆ ನಿಂತಿದೆ. ಇದೊಂದು ವಿಷಾದ. ಆಂಗ್ಲೇಯ ಅಥವಾ ಪಾಶ್ಚಾತ್ಯ ಯಾವನಾದರೊಬ್ಬ ಭರತಖಂಡಕ್ಕೆ ಬಂದು ಇಲ್ಲಿರುವ ಬಡತನವನ್ನು ಮತ್ತು ಕಷ್ಟವನ್ನು ನೋಡಿದಾಗ ಇಲ್ಲಿ ಧರ್ಮವಿರಲಾರದು, ನೈತಿಕತೆ ಇರಲಾರದು ಎಂಬ ನಿರ್ಣಯಕ್ಕೆ ಬರುವನು. ಅವನ ಅನುಭವ ಮಾತ್ರ ಅವನಿಗೆ ಸತ್ಯ. ಯೂರೋಪಿನಲ್ಲಿ ಶೈತ್ಯ ಪ್ರಧಾನ ವಾತಾವರಣದ ಮತ್ತು ಇತರ ಸನ್ನಿವೇಶಗಳ ಪರಿಣಾಮವಾಗಿ, ದಾರಿದ್ರ್ಯ ಮತ್ತು ಪಾಪ ಒಟ್ಟಿಗೆ ಹೋಗುವುವು. ಆದರೆ ಭರತಖಂಡದಲ್ಲಿ ಹಾಗಲ್ಲ. ಇಲ್ಲಿ ನನ್ನ ಅನುಭವದ ಪ್ರಕಾರ ಮನುಷ್ಯ ಬಡವನಾಗಿದ್ದಷ್ಟೂ ಹೆಚ್ಚು ಪರಿಶುದ್ಧನಾಗಿರುವನು. ಇದನ್ನು ತಿಳಿದುಕೊಳ್ಳಬೇಕಾದರೆ ಹೆಚ್ಚು ಕಾಲಾವಕಾಶ ಬೇಕು. ಈ ರಾಷ್ಟ್ರದ ಜೀವನದ ರಹಸ್ಯವನ್ನು ತಿಳಿದುಕೊಳ್ಳಲು ಎಷ್ಟು ಮಂದಿ ವಿದೇಶಿಯರು ನಿಲ್ಲುವರು? ಜನಾಂಗವನ್ನು ಕೂಲಂಕಷವಾಗಿ ತಿಳಿದುಕೊಳ್ಳುವಷ್ಟು ತಾಳ್ಮೆ ಎಲ್ಲೋ ಕೆಲವರಲ್ಲಿದೆ. ಈ ದೇಶದಲ್ಲಿ ಇದೊಂದು ದೇಶದಲ್ಲಿ ಮಾತ್ರ ಬಡತನ ಅಪರಾಧವಲ್ಲ, ಪಾಪವಲ್ಲ; ಈ ದೇಶದಲ್ಲಿ ಮಾತ್ರ ಬಡತನ ನಿರ್ದೋಷ ಮಾತ್ರವಲ್ಲ, ಬಡತನವನ್ನು ದೈವೀಗುಣ ಎಂದು ಕರೆದಿರುವರು. ಭಿಕ್ಷುಕನ ಉಡುಗೆ ಶ್ರೇಷ್ಠತಮ ವ್ಯಕ್ತಿಯ ಉಡುಗೆ. ಇದರಂತೆಯೇ ಪಾಶ್ಚಾತ್ಯ\break ಸಂಸ್ಥೆಗಳನ್ನೂ ಕೂಲಂಕಷವಾಗಿ ತಿಳಿದುಕೊಳ್ಳಬೇಕು, ಅವಸರದಿಂದ ಹುಚ್ಚು ಹುಚ್ಚಾದ ನಿರ್ಣಯಕ್ಕೆ ಬರಕೂಡದು. ಅಲ್ಲಿ ಸ್ತ್ರೀಪುರುಷರು ಒಟ್ಟಿಗೆ ಬೆರೆಯುವುದಕ್ಕೆ ಮತ್ತು ಅಲ್ಲಿಯ ಬೇರೆ ಬೇರೆ ಆಚಾರ ವ್ಯವಹಾರಗಳಿಗೆ ಅವುಗಳದೇ ಆದ ಅರ್ಥವಿದೆ. ಅವುಗಳಿಗೆ ಉನ್ನತವಾದ ಮುಖಗಳುಂಟು. ಆದರೆ ನೀವು ತಾಳ್ಮೆಯಿಂದ ಅದನ್ನು ತಿಳಿದುಕೊಳ್ಳಬೇಕು. ನಾವು ಅವರ ಆಚಾರ ವ್ಯವಹಾರಗಳನ್ನು ಅನುಕರಿಸಬೇಕೆಂದಾಗಲೀ, ಅವರು ನಮ್ಮ ಆಚಾರ ವ್ಯವಹಾರಗಳನ್ನು ಅನುಕರಿಸ ಬೇಕೆಂದಾಗಲಿ ಹೇಳುವುದಿಲ್ಲ. ಪ್ರತಿಯೊಂದು ಜನಾಂಗದ ಆಚಾರ ವ್ಯವಹಾರಗಳು ಹಲವು ಶತಮಾನಗಳಿಂದ ಸಾವಧಾನವಾಗಿ ಬೆಳೆದುಬಂದಿದೆ. ಪ್ರತಿಯೊಂದರ ಹಿಂದೆಯೂ ಆಳವಾದ ಅರ್ಥವಿದೆ. ಆದಕಾರಣ ಅವರು ನಮ್ಮ ರೀತಿ ನೀತಿಗಳನ್ನು ಅಲ್ಲಗಳೆಯುವುದಾಗಲೀ, ನಾವು ಅದರ ರೀತಿ ನೀತಿಗಳನ್ನು ಅಲ್ಲಗಳೆಯುವುದಾಗಲೀ ಮಾಡಕೂಡದು.

ಈ ಸಭೆಯ ಮುಂದೆ ಮತ್ತೊಂದು ವಿಷಯವನ್ನು ಹೇಳಬೇಕೆಂದಿರುವೆನು. ನಾನು ಅಮೆರಿಕಾ ದೇಶದಲ್ಲಿ ಮಾಡಿದ ಕೆಲಸಕ್ಕಿಂತ ಇಂಗ್ಲೆಂಡಿನಲ್ಲಿ ಮಾಡಿದುದು ಹೆಚ್ಚು ತೃಪ್ತಿಕರವಾಗಿದೆ. ಧೀರ ಅಚಲ ಸ್ಥಿರಸ್ವಭಾವದ ಆಂಗ್ಲೇಯನಿಗೆ ಹೊಸ ವಿಷಯವನ್ನು ಸ್ವೀಕರಿಸುವುದು ಇತರರಿಗಿಂತ ತಡವಾದರೂ, ಅವನಿಗೆ ಯಾವುದಾದರೂ ಹೊಸ ಭಾವನೆಯನ್ನು ಕೊಟ್ಟರೆ ಆದು ಎಂದಿಗೂ ವ್ಯರ್ಥವಾಗುವುದಿಲ್ಲ. ಆ ಜನಾಂಗದಲ್ಲಿ ಶಕ್ತಿಬಾಹುಳ್ಯ, ಅದ್ಭುತ ವ್ಯಾವಹಾರಿಕ ದೃಷ್ಟಿ ಇರುವುದರಿಂದ ಭಾವನೆಗಳು ಬೇರೂರಿ ಬೇಗ ಫಲವನ್ನು ನೀಡುತ್ತದೆ. ಬೇರೆ ದೇಶಗಳಲ್ಲಿ ಹಾಗಲ್ಲ. ಆ ಜನಾಂಗದಲ್ಲಿರುವ ಅದ್ಭುತ ಶಕ್ತಿ ಮತ್ತು ಕಾರ್ಯತತ್ಪರತೆ ಅನ್ಯಜನಾಂಗಗಳಲ್ಲಿ ಇಲ್ಲ. ಅಲ್ಲಿ ಕಲ್ಪನೆ ಕಡಮೆ, ಕಾರ್ಯ ಹೆಚ್ಚು. ಆಂಗ್ಲೇಯರ ಹೃದಯದ ಮೂಲವನ್ನು ಯಾರು ಬಲ್ಲರು? ಅಲ್ಲಿ ಎಷ್ಟು ಕಲ್ಪನೆ, ಭಾವನೆ ಇದೆ ಎಂಬುದು ಯಾರಿಗೆ ಗೊತ್ತು? ಅವರು ನಿಜವಾದ ಕ್ಷತ್ರಿಯರು, ಧೀರ ಜನಾಂಗ. ಅವರು ತಮ್ಮ ಭಾವನೆಯನ್ನು ವ್ಯಕ್ತಪಡಿಸುವುದಿಲ್ಲ, ಯಾವಾಗಲೂ ಗೋಪ್ಯವಾಗಿಡುವರು. ಅವರ ಶಿಕ್ಷಣ ಹಾಗೆ ಇರುವುದು. ಬಾಲ್ಯಾರಭ್ಯ ಅವರ ಶಿಕ್ಷಣವೇ ಹಾಗೆ. ಆಂಗ್ಲೇಯರಲ್ಲಿ ತಮ್ಮ ಭಾವನೆಯನ್ನು ವ್ಯಕ್ತಪಡಿಸುವವರು ಅಪರೂಪ. ಹೆಂಗಸರು ಕೂಡ ಹಾಗೆಯೇ. ಆಂಗ್ಲೇಯ ಮಹಿಳೆಯರು ಕೆಲಸಮಾಡುವುದನ್ನು ನಾನು ನೋಡಿರುವೆನು. ಅತಿ ಧೀರ ವಂಗ ಯುವಕರಿಗೂ ಆಶ್ಚರ್ಯವಾಗುವಷ್ಟರಮಟ್ಟಿನ ಸಾಹಸ ಕಾರ್ಯಗಳನ್ನು ಅವರು ಮಾಡುವರು. ಇಷ್ಟೊಂದು ಧೀರ ಸ್ವಭಾವದ ಹಿಂದೆ, ಯೋಧನ ಸ್ವಭಾವದ ಹಿಂದೆ, ಆಂಗ್ಲೇಯನ ಹೃದಯದ ಭಾವನಾ ಝರಿ ಇದೆ. ಅದರ ರಹಸ್ಯ ನಿಮಗೆ ಗೊತ್ತಾದರೆ, ನೀವು ಅಲ್ಲಿಗೆ ಹೋದರೆ, ಅವರ ಪರಿಚಯ ಮಾಡಿಕೊಂಡು ನಿಕಟವಾಗಿ ಬೆರೆತರೆ, ಅವರು ತಮ್ಮ ಹೃದಯವನ್ನು ಬಿಚ್ಚುವರು. ಅವರು ಎಂದೆಂದಿಗೂ ನಿಮ್ಮ ಸ್ನೇಹಿತರು, ನೀವು ಹೇಳಿದಂತೆ ಕೇಳುವರು. ಆದಕಾರಣವೆ, ನನ್ನ ದೃಷ್ಟಿಯಲ್ಲಿ ಬೇರೆ ಕಡೆಗಿಂತ ಇಂಗ್ಲೆಂಡಿನಲ್ಲಿ ನಾನು ಮಾಡಿದ ಕೆಲಸ ಹೆಚ್ಚು ತೃಪ್ತಿಕರವಾಗಿದೆ. ನಾಳೆ ನಾನೇ ಕಾಲವಾದರೂ ಇಂಗ್ಲೆಂಡಿನ ಕೆಲಸಕ್ಕೆ ಲೋಪಬರುವುದಿಲ್ಲ. ಅದು ವೃದ್ಧಿಯಾಗುತ್ತಾ ಹೋಗುವುದೆಂದು ನಾನು ದೃಢವಾಗಿ ನಂಬುತ್ತೇನೆ.

ಸಹೋದರರೇ, ನನ್ನ ಹೃದಯದ ಮತ್ತೊಂದು ನಾಡಿಯನ್ನು ನೀವು ಮಿಡಿದಿರುವಿರಿ. ಗಂಭೀರತಮವಾದುದು ಅದು. ಅದೇ ನನ್ನ ಗುರುದೇವ, ನನ್ನ ಆಚಾರ್ಯ, ನನ್ನ ಜೀವನಾದರ್ಶ, ನನ್ನ ಇಷ್ಟ, ನನ್ನ ಪ್ರಾಣ, ನನ್ನ ದೇವತೆಯಾದ ಶ‍್ರೀರಾಮಕೃಷ್ಣ ಪರಮಹಂಸರ ಪವಿತ್ರ ನಾಮೋಚ್ಚಾರಣೆಯನ್ನು ಮಾಡಿರುವಿರಿ. ನಾನು ಮನೋವಾಕ್ಕಾಯವಾಗಿ ಯಾವುದಾದರೂ ಸತ್ಕರ್ಮವನ್ನು ಮಾಡಿದ್ದರೆ, ನನ್ನ ಬಾಯಿಂದ ಯಾರಿಗಾದರೂ ಸಹಾಯವಾಗುವಂತಹ ನುಡಿಯೊಂದು ಹೊರಟಿದ್ದರೆ, ಅದು ನನ್ನದಲ್ಲ, ಅದೆಲ್ಲ ಅವರದು. ಆದರೆ ನನ್ನ ಬಾಯಿಂದ ನಿಂದೆಯ ನುಡಿ ಹೊರ ಬಿದ್ದಿದ್ದರೆ, ದ್ವೇಷ ಭಾವನೆ ಹೊರಬಿದ್ದಿದ್ದರೆ, ಅದೆಲ್ಲ ನನ್ನದು, ಅವರದಲ್ಲ. ದುರ್ಬಲವಾಗಿರುವುದೆಲ್ಲ ನನ್ನದು. ಯಾವುದು ಜೀವನಪ್ರದವೋ, ಬಲಪ್ರದವೋ, ಪರಿಶುದ್ಧವಾದುದೋ ಪವಿತ್ರವಾಗಿರುವುದೋ ಅದೆಲ್ಲ ಅವರ ಶಕ್ತಿ, ಲೀಲೆ, ಅವರ ವಾಣಿ, ಸ್ವಯಂ ಅವರೇ ಆಗಿರುವರು. ಹೌದು ಸ್ನೇಹಿತರೆ, ಜಗತ್ತು ಆ ಮಹಾನುಭಾವರನ್ನು ಇನ್ನೂ ತಿಳಿದುಕೊಳ್ಳಬೇಕಾಗಿದೆ. ಜಗತ್ತಿನ ಇತಿಹಾಸದಲ್ಲಿ ಮಹಾತ್ಮರ ಜೀವನವನ್ನು ಮತ್ತು ಸಂದೇಶಗಳನ್ನು ಕುರಿತು ಓದುತ್ತೇವೆ. ಅವರ ಶಿಷ್ಯರ ಹಲವು ಶತಮಾನಗಳ ಬರವಣಿಗೆಯ ಮೂಲಕ ಇದು ನಮಗೆ ವ್ಯಕ್ತವಾಗುವುದು. ಸಾವಿರಾರು ವರುಷಗಳು ಬದಲಾವಣೆಯ ಚಕ್ರಕ್ಕೆ ಸಿಲುಕಿ ನಯವಾಗಿ ಆ ಮಹಾಪುರುಷರ ಜೀವನ ನಮಗೆ ದೊರಕುವುದು. ಆದರೆ ಅಂತಹ ಯಾವ ಒಂದು ವ್ಯಕ್ತಿಯೂ ನನ್ನ ದೃಷ್ಟಿಯಲ್ಲಿ ನಾನೇ ಕಣ್ಣಾರೆ ಕಂಡ ವ್ಯಕ್ತಿಯಷ್ಟು ಉಜ್ವಲವಾಗಿಲ್ಲ. ಆ ವ್ಯಕ್ತಿಯೇ ಶ‍್ರೀರಾಮಕೃಷ್ಣ ಪರಮಹಂಸರು. ಅವರ ಛಾಯೆಯಲ್ಲಿ ನಾನು ಬೆಳೆದಿರುವೆನು. ಎಲ್ಲವನ್ನೂ ಅವರ ಅಡಿದಾವರೆಯಲ್ಲಿ ಕಲಿತಿರುವೆನು.

ಸಹೋದರರೇ, ನಿಮಗೆಲ್ಲ ಆ ಪ್ರಸಿದ್ಧವಾದ ಗೀತಾವಾಕ್ಯವು ಗೊತ್ತೇ ಇದೆ.

\begin{verse}
\textbf{ಯದಾ ಯದಾ ಹಿ ಧರ್ಮಸ್ಯ ಗ್ಲಾನಿರ್ಭವತಿ ಭಾರತ}\\\textbf{ಅಭ್ಯುತ್ಥಾನಮಧರ್ಮಸ್ಯ ತದಾತ್ಮಾನಂ ಸೃಜಾಮ್ಯಹಮ್​~॥}\\\textbf{ಪರಿತ್ರಾಣಾಯ ಸಾಧೂನಾಂ ವಿನಾಶಾಯ ಚ ದುಷ್ಕೃತಾಮ್​~।}\\\textbf{ಧರ್ಮಸಂಸ್ಥಾಪನಾರ್ಥಾಯ ಸಂಭವಾಮಿ ಯುಗೇ ಯುಗೇ~॥}
\end{verse}

“ಭಾರತ! ಎಂದು ಧರ್ಮಗ್ಲಾನಿಯಾಗುವುದೋ, ಅಧರ್ಮ ಮೇಲೇಳುವುದೋ ಆಗ ನಾನು ಅವತಾರ ಮಾಡುವೆನು. ಸಾಧುಗಳ ರಕ್ಷಣೆಗೆ, ದುಷ್ಟರ ನಾಶಕ್ಕೆ ಧರ್ಮ ಸಂಸ್ಥಾಪನೆ ಮಾಡುವುದಕ್ಕೆ ಯುಗ ಯುಗದಲ್ಲಿಯೂ ನಾನು ಜನ್ಮವೆತ್ತುವೆನು.”

\vskip   4pt

ಇದರೊಂದಿಗೆ ಮತ್ತೊಂದು ವಿಷಯವನ್ನು ತಿಳಿದುಕೊಳ್ಳಬೇಕಾಗಿದೆ. ಇಂತಹ ಒಂದು ಗ್ಲಾನಿ ನಮ್ಮಲ್ಲಿ ಇಂದು ಇದೆ. ಆಧ್ಯಾತ್ಮಿಕ ಮಹಾತರಂಗವೊಂದು ಬರುವುದಕ್ಕೆ ಮುಂಚೆ ಸಮಾಜದಲ್ಲಿ ಹಲವು ಸಣ್ಣಪುಟ್ಟ ಅಲೆಗಳೇಳುತ್ತವೆ. ಅದರಲ್ಲಿ ಒಂದು ನಮ್ಮ ದೃಷ್ಟಿಗೆ ಗೋಚರವಾಗದೆ, ನಮಗೆ ತನ್ನ ಸುಳಿವನ್ನು ಕೊಡದೆ ನಮ್ಮ ಆಲೋಚನೆಗೆ ನಿಲುಕದೆ ಕ್ರಮೇಣ ದೊಡ್ಡದು ದೊಡ್ಡದಾಗಿ ಸಣ್ಣಪುಟ್ಟ ಅಲೆಗಳನ್ನೆಲ್ಲ ಒಳಗೊಂಡು ಅದ್ಭುತವಾಗಿ, ಮಹಾ ಪ್ರವಾಹವಾಗಿ ಯಾರಿಗೂ ಎದುರಿಸಲಾಗದ ಪ್ರಚಂಡ ಶಕ್ತಿಯೊಡನೆ ಸಮಾಜದ ಮೇಲೆ ಬೀಳುವುದು. ಇಂದು ನಮ್ಮ ಕಣ್ಣೆದುರಿಗೆ ಹೀಗೆ ಆಗುತ್ತಿದೆ. ಕಣ್ಣಿದ್ದರೆ ನೋಡುವಿರಿ. ನಿಮ್ಮ ಹೃದಯ ತೆರೆದಿದ್ದರೆ ಅದನ್ನು ಸ್ವೀಕರಿಸುವಿರಿ. ನೀವು ಸತ್ಯಾನ್ವೇಷಕರಾದರೆ ನಿಮಗೆ ಅದು ದೊರಕುವುದು. ಕಾಲಚಿಹ್ನೆಯನ್ನು ಕಾಣದವನು ಅಂಧ, ನಿಜವಾಗಿ ಅಂಧನು. ಬಡವರಾದ ಬ್ರಾಹ್ಮಣ ಮಾತಾಪಿತರಿಗೆ ಜನಿಸಿದ ಈ ಬಾಲಕನು ದೂರದ ಒಂದು ಹಳ್ಳಿಯಲ್ಲಿ ಹುಟ್ಟಿದನು. ಆ ಹಳ್ಳಿಯ ಹೆಸರನ್ನೂ ನಿಮ್ಮಲ್ಲಿ ಅನೇಕರು ಕೇಳಿಲ್ಲ. ಶತಮಾನಗಳಿಂದ ಮೂರ್ತಿ ಪೂಜೆಯನ್ನು ವಿರೋಧಿಸುತ್ತಿರುವ ಪಾಶ್ಚಾತ್ಯ ಜನಾಂಗ ಇಂದು ಅವರನ್ನು ಅಕ್ಷರಶಃ ಪೂಜಿಸುತ್ತಿರುವರು. ಇದು ಯಾರ ಶಕ್ತಿ? ನನ್ನದೆ ಅಥವಾ ನಿಮ್ಮದೆ? ಯಾವ ಶಕ್ತಿ ಶ‍್ರೀರಾಮಕೃಷ್ಣ ಪರಮಹಂಸರ ರೂಪಿನಲ್ಲಿ ಆವಿರ್ಭೂತವಾಯಿತೊ ಅದೇ ಶಕ್ತಿ. ಏಕೆಂದರೆ, ನೀವು ಮತ್ತು ನಾನು, ಸಾಧುಗಳು ಮತ್ತು ಪ್ರವಾದಿಗಳು, ಅವತಾರಪುರುಷರು ಮತ್ತು ಇಡಿಯ ಬ್ರಹ್ಮಾಂಡ – ಇವೆಲ್ಲ ಆ ಶಕ್ತಿಯ ಆವಿರ್ಭಾವ ಮಾತ್ರ. ಆ ಶಕ್ತಿ ಕೆಲವು ವೇಳೆ ಕಡಮೆ, ಕೆಲವು ವೇಳೆ ಹೆಚ್ಚಾಗಿ ಘನೀಭೂತವಾಗುವುದು. ಆ ಸಮಯದಲ್ಲಿ ನಾವು ಆ ಮಹಾ ಶಕ್ತಿಯ ಲೀಲೆಯ ಆರಂಭವನ್ನು ಮಾತ್ರ ನೋಡುತ್ತಿರುವೆವು. ಈಗಿರುವವರು ಕಾಲವಾಗುವುದಕ್ಕೆ ಮುಂಚೆ ಆ ಅದ್ಭುತಶಕ್ತಿಯ ಲೀಲೆಯನ್ನು ನೋಡುವರು. ಭರತವರ್ಷದ ಪುನರುತ್ಥಾನದ ಸಮಯಕ್ಕೆ ಸರಿಯಾಗಿ ಅದು ಬಂದಿದೆ. ಯಾವ ಮೂಲ ಜೀವನ ಶಕ್ತಿ ಭಾರತದಲ್ಲಿ ಸದಾ ಸಂಚಾರ ಮಾಡಬೇಕಾಗಿದೆಯೋ ಅದನ್ನು ಕಾಲಕಾಲಕ್ಕೆ ನಾವು ಮರೆಯುವೆವು.

\vskip   4pt

ಪ್ರತಿಯೊಂದು ದೇಶಕ್ಕೂ ಉದ್ದೇಶ ಸಾಧನೆಗೆ ಬೇರೆ ಬೇರೆ ಮಾರ್ಗಗಳಿವೆ. ಕೆಲವರು ರಾಜನೀತಿಯ ಮೂಲಕ, ಕೆಲವರು ಸಮಾಜ ಸುಧಾರಣೆಯ ಮೂಲಕ, ಮತ್ತೆ ಕೆಲವರು ಇನ್ನೂ ಬೇರೆ ಯಾವುದಾದರೂ ಮಾರ್ಗದ ಮೂಲಕ ಅದನ್ನು ಮಾಡುತ್ತಾರೆ. ನಮಗೆ ಇರುವುದು ಧರ್ಮಮಾರ್ಗ ಒಂದೇ. ಆ ಪಥ ಒಂದರಲ್ಲೇ ನಾವು ನಡೆಯಬೇಕಾಗಿದೆ. ಆಂಗ್ಲೇಯನು ಧರ್ಮವನ್ನು ಕೂಡ ರಾಜಕೀಯದ ಮೂಲಕ ತಿಳಿದುಕೊಳ್ಳಬಲ್ಲ. ಬಹುಶಃ ಅಮೆರಿಕಾದವರು ಸಮಾಜಸುಧಾರಣೆಯ ಮೂಲಕ ಧರ್ಮವನ್ನು ಅರ್ಥಮಾಡಿಕೊಳ್ಳಬಲ್ಲರು. ಆದರೆ ಹಿಂದೂವು ರಾಜಕೀಯಕ್ಕೆ ಕೂಡ ಧರ್ಮದ ಮೂಲಕ ಬರಬೇಕು. ಪ್ರತಿಯೊಂದೂ ಧರ್ಮದ ಮೂಲಕ ಬರಬೇಕು. ಏಕೆಂದರೆ ರಾಷ್ಟ್ರ ಜೀವನದ ಸಂಗೀತಕ್ಕೆ ಇದೇ ಮುಖ್ಯಸ್ವರ. ಉಳಿದುವೆಲ್ಲ ಬದಲಾಗುತ್ತಿರುವ ಗೌಣ ವಿಷಯಗಳು. ಆ ಮೂಲವೇ ಅಪಾಯದಲ್ಲಿತ್ತು. ರಾಷ್ಟ್ರ ಜೀವನದ ಈ ಪ್ರಮುಖ ಸ್ವರವೇ ಬದಲಾಗುವಂತೆ ಇತ್ತು. ನಮ್ಮ ಬಾಳಿನ ಆಧಾರವೇ ಬದಲಾಗುವಂತೆ ಕಂಡಿತು. ಧರ್ಮಕ್ಕೆ ಬದಲಾಗಿ ರಾಜಕೀಯವೇ ಪ್ರಧಾನವಾಗುವಂತೆ ತೋರಿತು. ನಾವೇನಾದರೂ ಹೀಗೆ ಮಾಡಿದ್ದರೆ ನಮ್ಮ ಸರ್ವನಾಶವಾಗುತ್ತಿತ್ತು. ಆದರೆ ಹಾಗೆ ಮಾಡಲಾಗಲಿಲ್ಲ. ಅದಕ್ಕೆ ಈ ಶಕ್ತಿ ಅಭಿವ್ಯಕ್ತಿಗೊಂಡಿತು. ನೀವು ಈ ಮಹಾಪುರುಷರನ್ನು ಯಾವ ದೃಷ್ಟಿಯಲ್ಲಿ ನೋಡಿದರೂ ಚಿಂತೆಯಿಲ್ಲ, ಎಷ್ಟು ಗೌರವ ಕೊಟ್ಟರೂ ಚಿಂತೆಯಿಲ್ಲ. ಆದರೆ ನಿಮ್ಮೆದುರಿಗೆ ಈ ಸವಾಲನ್ನು ಹಾಕುತ್ತೇನೆ. ಭರತಖಂಡದಲ್ಲಿ ಹಲವಾರು ಶತಮಾನಗಳಿಂದ ವ್ಯಕ್ತವಾಗದ ಒಂದು ಪ್ರಚಂಡ ಶಕ್ತಿ ಇಲ್ಲಿ ಅಭಿವ್ಯಕ್ತವಾಗಿದೆ. ಭರತಖಂಡದ ಪುನರುತ್ಥಾನಕ್ಕೆ, ಅದರ ಹಿತಕ್ಕೆ ಮತ್ತು ಇಡೀ ವಿಶ್ವದ ಹಿತಕ್ಕೆ ಈ ಒಂದು ಮಹಾವ್ಯಕ್ತಿ ಏನು ಮಾಡಿತು ಎಂಬುದನ್ನು ತಿಳಿದುಕೊಳ್ಳುವುದು ಹಿಂದೂಗಳಾದ ನಿಮ್ಮ ಕರ್ತವ್ಯ. ಜಗತ್ತಿನ ಇತರ ಕಡೆಗಳಲ್ಲಿ ವಿಶ್ವಧರ್ಮ, ಭಿನ್ನ ಭಿನ್ನ ಧರ್ಮಗಳಿಗೆ ಸಹಾನುಭೂತಿ ಎಂಬ ಭಾವನೆಗಳನ್ನು ಆಲೋಚಿಸುವುದಕ್ಕೆ ಮುಂಚೆಯೇ, ಇಲ್ಲೇ, ನಮ್ಮ ನಗರದಲ್ಲೇ, ಯಾರ ಜೀವನವು ಒಂದು ವಿಶ್ವಧರ್ಮ ಸಮ್ಮೇಳನದಂತೆ ಇತ್ತೋ ಅಂತಹ ಮಹಾಪುರುಷರೊಬ್ಬರು ವಾಸಿಸುತ್ತಿದ್ದರು.

\vskip   4pt

ನಮ್ಮ ಶಾಸ್ತ್ರಗಳು ಪರಮೋಚ್ಚ ಆದರ್ಶ ನಿರ್ಗುಣ ಬ್ರಹ್ಮ. ಈಶ್ವರನ ದಯೆಯಿಂದ ನಾವೆಲ್ಲ ನಿರ್ಗುಣ ಬ್ರಹ್ಮೋಪಾಸಕರಾಗಿದ್ದರೆ ಚೆನ್ನಾಗಿತ್ತು. ಅದು ಸಾಧ್ಯವಿಲ್ಲದೇ ಇರುವಾಗ ಮುಕ್ಕಾಲುಪಾಲು ಮಾನವರಿಗೆ ಒಂದು ಸಗುಣ ಆದರ್ಶ ಅವಶ್ಯವಾಗಿದೆ. ಇಂತಹ ಒಬ್ಬ ಮಹಾಪುರುಷರ ಆದರ್ಶದ ಪತಾಕೆಯ ನೆರಳಿಗೆ ಹಾರ್ದಿಕವಾಗಿ ಬರದೇ ಇದ್ದರೆ ಯಾವ ದೇಶವೂ ಮೇಲೇಳಲಾರದು, ಪ್ರಖ್ಯಾತವಾಗಲಾರದು, ಕೆಲಸ ಮಾಡುವರು. ರಾಜನೀತಿ, ಅದರ ಪ್ರತಿನಿಧಿಗಳು, ಸಮಾಜ, ವಾಣಿಜ್ಯ, ಯಾವುದೂ ಭರತಖಂಡದಲ್ಲಿ ನಡೆಯುವಂತೆ ಇಲ್ಲ. ನಮಗೆ ಆಧ್ಯಾತ್ಮಿಕ ಆದರ್ಶಗಳು ಬೇಕು, ಮಹಾ ಆಧ್ಯಾತ್ಮಿಕ ವ್ಯಕ್ತಿಗಳ ಸುತ್ತ ಉತ್ಸಾಹದಿಂದ ನೆರೆಯಲು ಇಚ್ಛಿಸುವೆವು. ನಮ್ಮ ನಾಯಕರು ಧಾರ್ಮಿಕ ಪುರಷರಾಗಿರಬೇಕು ಅಂತಹ ಧಾರ್ಮಿಕ ಪುರುಷರೊಬ್ಬರು ರಾಮಕೃಷ್ಣ ಪರಮಹಂಸರ ಸ್ವರೂಪದಲ್ಲಿ ನಮಗೆ ಲಭ್ಯವಾಗಿದ್ದಾರೆ. ಈ ದೇಶ ಉದ್ಧಾರವಾಗಬೇಕಾದರೆ, ನನ್ನ ಮಾತನ್ನು ಕೇಳಿ. ಆ ಹೆಸರಿನ ಸುತ್ತ ಉತ್ಸಾಹದಿಂದ ನಾವು ಒಟ್ಟುಗೂಡಬೇಕಾಗುತ್ತದೆ. ಶ‍್ರೀರಾಮಕೃಷ್ಣ ಪರಮಹಂಸರನ್ನು ನಾನು ಪ್ರಚಾರ ಮಾಡುತ್ತೇನೆಯೋ, ನೀವೋ ಅನ್ಯರೋ ಅದರ ಬಗ್ಗೆ ನನಗೆ ಚಿಂತೆಯಿಲ್ಲ. ಆದರೆ ನಾನು ಅವರನ್ನು ನಿಮ್ಮ ಮುಂದಿಡುವೆನು. ನೀವು ಆ ವಿಷಯವನ್ನು ಪರ್ಯಾಲೋಚಿಸಿ. ನಿಮ್ಮ ಜನಾಂಗದ ಹಿತಕ್ಕೆ, ದೇಶದ ಹಿತಕ್ಕೆ, ಆ ಮಹಾ ಆದರ್ಶವನ್ನು ಹೇಗೆ ಉಪಯೋಗಿಸಿಕೊಳ್ಳಬಹುದು ಎಂಬುದನ್ನು ನೀವೇ ನಿಷ್ಕರ್ಷಿಸಿ. ನಾವು ಇದನ್ನು ನೆನಪಿನಲ್ಲಿಡಬೇಕು–ಅವರು ನೀವು ನೋಡಿರುವ ವ್ಯಕ್ತಿಗಳೆಲ್ಲರಿಗಿಂತ ಪವಿತ್ರತಮರು. ಇದನ್ನು ಇನ್ನೂ ಸ್ವಲ್ಪ ವಿಶದವಾಗಿ ಹೇಳುತ್ತೇನೆ. ನೀವು ಯಾರ ವಿಷಯವಾಗಿ ಓದಿ ತಿಳಿದುಕೊಂಡಿರುವಿರೋ ಆ ಎಲ್ಲ ವ್ಯಕ್ತಿಗಳಿಗಿಂತ ಇವರು ಪ್ರವಿತ್ರೋತ್ತಮರು. ನೀವು ಇದುವರೆಗೆ ಓದಿರದ, ಯಾರು ನೋಡಲು ಕೂಡ ದುರ್ಲಭರೋ ಅಂತಹ ಆತ್ಮಶಕ್ತಿಯ ಅದ್ಭುತ ನಮ್ಮ ಕಣ್ಣೆದುರಿಗೆ ಇದೆ. ಅವರು ನಿರ್ವಾಣ ಹೊಂದಿದ ಹತ್ತು ವರುಷದೊಳಗೆ ಆ ಶಕ್ತಿ ಪೃಥ್ವಿಯನ್ನೇ ಆವರಿಸಿರುವುದು. ಈ ಸಂಗತಿ ನಿಮ್ಮ ಕಣ್ಣೆದುರಿಗೆ ಇದೆ. ನಮ್ಮ ಜನಾಂಗದ ಹಿತಕ್ಕೆ, ಧರ್ಮದ ಹಿತಕ್ಕೆ, ಕೇವಲ ಕರ್ತವ್ಯ ದೃಷ್ಟಿಯಿಂದ ಈ ಅತ್ಯದ್ಭುತ ಆಧ್ಯಾತ್ಮಿಕ ವ್ಯಕ್ತಿಯ ಆದರ್ಶವನ್ನು ನಿಮ್ಮ ಮುಂದಿಡುವೆನು. ಅವರನ್ನು ನನ್ನ ಮೂಲಕ ಅಳೆಯಬೇಡಿ. ನಾನೊಂದು ದುರ್ಬಲ ಉಪಕರಣಮಾತ್ರ. ನನ್ನ ಮೂಲಕ ಅವರ ಶೀಲವನ್ನು ನಿಷ್ಕರ್ಷಿಸಬೇಡಿ. ನಾನು ಅಥವಾ ಅವರ ಇತರ ಶಿಷ್ಯರು, ನೂರಾರು ಜನ್ಮಗಳನ್ನು ಸವೆಯಿಸಿದರೂ ಅವರ ಕೋಟಿಯ ಒಂದು ಪಾಲು ಕೂಡ ಆಗಲಾರೆವು. ಅಷ್ಟು ಮಹಾಮಹಿಮರಾಗಿದ್ದರು ಅವರು. ನೀವೇ ಇದನ್ನು ಸ್ವಂತ ವಿಚಾರಿಸಿ, ನಿಮ್ಮ ಹೃದಯದಲ್ಲೇ ಸನಾತನ ಸಾಕ್ಷಿ ಇರುವನು. ಆ ಶ‍್ರೀರಾಮಕೃಷ್ಣ ಪರಮಹಂಸರೇ ನಮ್ಮ ದೇಶದ, ಜನಾಂಗದ, ಇಡೀ ವಿಶ್ವದ ಹಿತಕ್ಕೆ, ನಿಮ್ಮ ಹೃದಯ ವಿಕಾಸವಾಗುವಂತೆ ಮಾಡಲಿ, ನಾನು ಪ್ರಯತ್ನ ಪಡಲಿ, ಪಡದೆ ಇರಲಿ, ನಮ್ಮ ದೇಶದಲ್ಲಿ ಆವಶ್ಯಕವಾಗಿ ಬರಬೇಕಾಗಿರುವ ಮಹಾ ಬದಲಾವಣೆಗೆ ಸೊಂಟ ಕಟ್ಟಿ ಸತ್ಯಸಂಧರಾಗಿ ದುಡಿಯುವಂತೆ ಮಾಡಲಿ. ಭಗವಂತನ ಕೆಲಸ ನಿಮ್ಮ ನಮ್ಮ ಇಚ್ಛೆಗೆ ಕಾದು ಕುಳಿತುಕೊಳ್ಳುವುದಿಲ್ಲ. ಅವನ ಕೆಲಸ ಮಾಡುವುದಕ್ಕೆ ನೂರಾರು ಸಹಸ್ರಾರು ಜನರನ್ನು ಧೂಳಿನಿಂದ ಬೇಕಾದರೆ ಸೃಷ್ಟಿಸುವನು. ಅವನ ಅಧೀನದಲ್ಲಿ ಕೆಲಸ ಮಾಡುವ ಸೌಭಾಗ್ಯ ದೊರಕಿದುದಕ್ಕೆ ನಾವೇ ಧನ್ಯರು.

\vskip   5pt

ಇದರಿಂದ ನಮ್ಮ ಭಾವನೆಯು ವಿಸ್ತಾರವಾಗುವುದು. ನೀವು ನನಗೆ ಸೂಚಿಸಿದಂತೆ ನಾವು ಜಗತ್ತನ್ನು ಗೆಲ್ಲಬೇಕು. ನಾವು ಜಯಿಸಬೇಕು! ಭರತಖಂಡವು ಜಗತ್ತನ್ನು ಗೆಲ್ಲಬೇಕು. ನನ್ನ ಆದರ್ಶ ಅದಕ್ಕಿಂತ ಕಡಮೆಯಿಲ್ಲ. ಅದು ಅಗಾಧವಾದ ಕಾರ್ಯವಾಗಿರಬಹುದು. ನಿಮ್ಮಲ್ಲಿ ಅನೇಕರಿಗೆ ಆಶ್ಚರ್ಯವಾಗಬಹುದು. ಆದರೆ ಅದು ಹಾಗೆಯೇ ಆಗಬೇಕು. ನಾವು ಜಗತ್ತನ್ನು ಗೆಲ್ಲಬೇಕು ಇಲ್ಲವೆ ಮಡಿಯಬೇಕು. ನಮಗೆ ಬೇರೆ ಮಾರ್ಗವೇ ಇಲ್ಲ. ವಿಕಾಸವೇ ಜೀವನದ ಚಿಹ್ನೆ. ನಾವು ಹೊರಗೆ ಹೋಗಬೇಕು, ವಿಕಾಸವಾಗಬೇಕು, ಚಟುವಟಿಕೆಯನ್ನು ತೋರಬೇಕು. ಅಥವಾ ಅಧೋಗತಿಗೆ ಇಳಿದು ಕೊಳೆತು ಸಾಯಬೇಕು. ಬೇರೆ ಮಾರ್ಗವೇ ಇಲ್ಲ. ಎರಡರಲ್ಲಿ ಯಾವುದನ್ನಾದರೂ ಆರಿಸಿಕೊಳ್ಳಿ. ಬದುಕಿ, ಇಲ್ಲವೆ ಸಾಯಿರಿ. ನಮ್ಮ ದೇಶದಲ್ಲಿರುವ ಕ್ಷುದ್ರ ಮನಸ್ತಾಪಗಳು, ಜಗಳಗಳು ಎಲ್ಲರಿಗೂ ಗೊತ್ತಿವೆ. ಎಲ್ಲಾ ಕಡೆಗಳಲ್ಲಿಯೂ ಹೀಗೆಯೇ. ಇದನ್ನು ನಂಬಿ. ಇತರ ರಾಷ್ಟ್ರಗಳಿಗೆ ರಾಜಕೀಯ ಜೀವನದಲ್ಲಿ ವಿದೇಶೀ ನೀತಿ ಇದೆ. ದೇಶದಲ್ಲಿ ಹೆಚ್ಚು ಜಗಳವಿರುವಾಗ ಹೊರಗಡೆ ಜಗಳ ಕಾಯುವುದಕ್ಕೆ ಬೇರೆ ಯಾರನ್ನಾದರೂ ಹುಡುಕುವರು. ಆಗ ದೇಶದಲ್ಲಿ ಅಂತಃಕಲಹ ನಿಲ್ಲುವುದು. ನಮ್ಮಲ್ಲಿ ಈ ಕಲಹಗಳಿವೆ. ಆದರೆ ಅವನ್ನು ನಿಲ್ಲಿಸಲು ವಿದೇಶೀ ನೀತಿ ಇಲ್ಲ. ನಮ್ಮ ಶಾಸ್ತ್ರಗಳ ಸತ್ಯವನ್ನು ವಿದೇಶೀಯರಿಗೆ ಬೋಧಿಸುವುದು ನಮ್ಮ ನಿರಂತರವಾದ ವಿದೇಶೀನೀತಿಯಾಗಬೇಕು. ಇದು ನಮ್ಮನ್ನು ಒಂದು ಗೂಡಿಸುವುದು ಎಂದು ಪ್ರಮಾಣಪಡಿಸುವುದಕ್ಕೆ ಬೇರೆ ಉದಾಹರಣೆ ಬೇಕೆ ಎಂದು ರಾಜಕೀಯ ಭಾವನೆ ಇದ್ದವರನ್ನು ಕೇಳುತ್ತೇನೆ. ಈ ಸಭೆಯೇ ಇದಕ್ಕೆ ಸಾಕಷ್ಟು ಪ್ರಮಾಣವಾಗಿದೆ.

\vskip   5pt

ಎರಡನೆಯದಾಗಿ, ಸ್ವಾರ್ಥಪರವಾದ ಈ ಉದ್ದೇಶಗಳು ಇದ್ದಾಗ್ಯೂ ನಿಃಸ್ವಾರ್ಥ ಪರ ಜೀವಂತ ಉದಾಹರಣೆಗಳು ನಮ್ಮ ಮುಂದೆ ಇವೆ. ದುರ್ದೆಶೆಗೆ, ಅವನತಿಗೆ ಒಂದು ಮುಖ್ಯ ಕಾರಣವು ನಮ್ಮ ಜನಾಂಗವು ಕೂಪಮಂಡೂಕದಂತೆ ಸಂಕುಚಿತಗೊಂಡುದು; ಇತರ ಮಾನವ ಜನಾಂಗಕ್ಕೆ ಅನರ್ಘ್ಯರತ್ನಗಳನ್ನು ಕೊಡಲು ನಿರಾಕರಿಸಿದುದು. ಆಧ್ಯಾತ್ಮಿಕ\break ಪಿಪಾಸೆಯಿಂದ ನರಳುತ್ತಿರುವ ಆರ್ಯೇತರರಿಗೆ ಜೀವಪೋಷಕ ವಾದ ಸತ್ಯಗಳನ್ನು ಧಾರೆ ಎರೆಯದೆ ಹೋದುದು. ನಾವು ಹೊರಗೆ ಹೋಗಲಿಲ್ಲ. ಇತರ ರಾಷ್ಟ್ರಗಳೊಂದಿಗೆ ಹೋಲಿಸಿ ನೋಡಲಿಲ್ಲ. ಇದೊಂದು ನಮ್ಮ ದೇಶದ ಅವನತಿಗೆ ಮುಖ್ಯ ಕಾರಣ. ಭರತಖಂಡದಲ್ಲಿ ಇರುವ ಸ್ವಲ್ಪ ಜಾಗ್ರತಿಗೆ, ಚಟುವಟಿಕೆಗೆ ಕಾರಣ, ರಾಜಾ ರಾಮಮೋಹನರಾಯ್​ ಈ ಸಂಕುಚಿತ ಚಿಪ್ಪಿನಿಂದ ಹೊರಗೆ ಹೋದ ದಿನದಿಂದ ಮೊದಲಾಯಿತೆಂದು ನಿಮಗೆಲ್ಲ ಗೊತ್ತಿದೆ. ಅಂದಿನಿಂದ ನಮ್ಮ ಇತಿಹಾಸ ಬೇರೊಂದು ರೂಪವನ್ನು ತಾಳಿದೆ. ಅತಿ ವೇಗದಿಂದ ಚಲಿಸುತ್ತಿದೆ. ಹಿಂದೆ ಇದ್ದವು ಸಣ್ಣ ಸಣ್ಣ ನದಿಗಳಾದರೆ ಮುಂದೆ ಬರುವುದು ಸಾಗರೋಪಮ ಮಹಾಪ್ರವಾಹ. ಯಾರೂ ಅದನ್ನು ತಡೆಯಲಾರರು.

\vskip   4pt

ನಾವು ಹೊರಗೆ ಹೋಗಬೇಕು. ಕೊಟ್ಟು ತೆಗೆದುಕೊಳ್ಳುವುದರಲ್ಲಿದೆ ರಹಸ್ಯ. ನಾವು ಯಾವಾಗಲೂ ಸ್ವೀಕರಿಸುತ್ತಾ ಪಾಶ್ಚಾತ್ಯರ ಪದತಳದಲ್ಲಿರಬೇಕೆ? ಪ್ರತಿಯೊಂದನ್ನೂ, ಧರ್ಮವನ್ನು ಕೂಡ ಅವರಿಂದ ಕಲಿಯಬೇಕೆ? ಅವರಿಂದ ಯಂತ್ರ ಕುಶಲತೆಯನ್ನು ಕಲಿತುಕೊಳ್ಳಬಹುದು. ಆದರೆ ಅವರಿಗೆ ನಾವು ಏನಾದರೂ ಕಲಿಸಬೇಕು. ಅದು ಯಾವುದೆಂದರೆ ನಮ್ಮ ಧರ್ಮ, ಅಧ್ಯಾತ್ಮವಿದ್ಯೆ. ಪೂರ್ಣ ನಾಗರಿಕತೆಗೆ ಜಗತ್ತು ಭರತಖಂಡದಲ್ಲಿ ಅನರ್ಘ್ಯ ರತ್ನಗಳಿಗಾಗಿ ಕಾಯುತ್ತಿದೆ. ಹಲವು ಶತಮಾನಗಳ ದುಃಖದ ಮತ್ತು ಅವನತಿಯ ಕಾಲದಲ್ಲಿಯೂ ಬಿಡದೆ ಸಂರಕ್ಷಿಸಿಕೊಂಡು ಬಂದ ನಮ್ಮ ಪೂರ್ವಿಕರ ಅತ್ಯಮೋಘ ಆಧ್ಯಾತ್ಮಿಕ ಆಸ್ತಿಗಾಗಿ ಅದು ಕಾಯುತ್ತಿದೆ. ಜಗತ್ತು ಈ ಭಂಡಾರಕ್ಕೆ ಕಾಯುತ್ತಿದೆ. ಭರತಖಂಡದಲ್ಲಿ ಹೊರಗೆ ನಮ್ಮ ಅತ್ಯಮೂಲ್ಯ ಧನಕ್ಕಾಗಿ ಜನರೆಷ್ಟು ಕಾತರಿಸುತ್ತಿರುವರು ಎಂಬುದು ನಿಮಗೆ ಗೊತ್ತಿಲ್ಲ. ನಾವಿಲ್ಲಿ ಬರೀ ಮಾತನಾಡುತ್ತೇವೆ; ಜಗಳ ಕಾಯುತ್ತೇವೆ. ಪವಿತ್ರವಾದುದನ್ನೆಲ್ಲ ಅಣಕಿಸುತ್ತೇವೆ. ಪರಮ ಪೂಜ್ಯ ವಸ್ತುವನ್ನು ಅಣಕಿಸುವುದೊಂದು ಈ ರಾಷ್ಟ್ರದ ಬಾಳಿನ ಚಾಳಿಯಾಗಿ ಹೋಗಿದೆ. ಕೋಟ್ಯಂತರ ಜನರು ಹೊರಗೆ ಭರತಖಂಡದ ಕಡೆಗೆ ಕೈನೀಡಿ ಕಾತರತೆಯಿಂದ ನಮ್ಮ ಪೂರ್ವಜರಿಂದ ಶೇಖರಿಸಲ್ಪಟ್ಟ ಅಮೃತಪಾನಕ್ಕೆ ಕಾದಿರುವುದು ನಮಗೆ ಗೊತ್ತಾಗುವುದೇ ಇಲ್ಲ. ನಾವು ಹೊರಗೆ ಹೋಗಬೇಕು. ಆಧ್ಯಾತ್ಮಿಕ ವಿದ್ಯೆಯನ್ನು ಅವರಿಗೆ ಕೊಟ್ಟು ಅವರ ವಿದ್ಯೆಯನ್ನು ನಾವು ಕಲಿಯೋಣ. ಆಧ್ಯಾತ್ಮಿಕ ಪ್ರಪಂಚದ ಅದ್ಭುತಗಳನ್ನು ಭೌತಿಕ ಪ್ರಪಂಚದ ಅದ್ಭುತಗಳಿಗಾಗಿ ವಿನಿಮಯ ಮಾಡೋಣ. ನಾವು ಯಾವಾಗಲೂ ವಿದ್ಯಾರ್ಥಿಗಳೇ ಆಗಿರುವುದಿಲ್ಲ. ಗುರುಗಳೂ ಆಗುವೆವು. ಸಮಾನತೆಯಿಲ್ಲದೆ ಸ್ನೇಹವಿರಲಾರದು. ಒಬ್ಬರು ಯಾವಾಗಲೂ ಕಲಿಸುತ್ತ ಮತ್ತೊಬ್ಬರು ಯಾವಾಗಲೂ ಅವರ ಪಾದದಡಿ ಕಲಿಯುತ್ತಿದ್ದರೆ ಸಮಾನತೆ ಬರಲಾರದು. ನೀವು ಅಮೆರಿಕಾ ದೇಶದವರೊಂದಿಗೆ ಮತ್ತು ಇಂಗ್ಲಿಷರೊಂದಿಗೆ ಸರಿಸಮರಾಗಬೇಕಾದರೆ ನೀವು ಕಲಿಸಬೇಕು ಮತ್ತು ಕಲಿತುಕೊಳ್ಳಬೇಕು. ಇನ್ನೂ ಹಲವು ಶತಮಾನಗಳು ಕಲಿಸುವಷ್ಟು ನಿಮ್ಮಲ್ಲಿದೆ. ಇದನ್ನು ನಾವು ಮಾಡಬೇಕಾಗಿದೆ. ಶ್ರದ್ಧೆ ಪ್ರೋತ್ಸಾಹ ನಮ್ಮ ರಕ್ತದಲ್ಲಿರಬೇಕು. ವಂಗದೇಶದಲ್ಲಿ ಬೇಕಾದಷ್ಟು ಕಲ್ಪನೆ ಇದೆ ಎನ್ನುವರು. ಅದಿದೆ ಎಂದು ನಾನೂ ನಂಬುತ್ತೇನೆ. ಕಾಲ್ಪನಿಕ ಜೀವಿಗಳು, ಭಾವ ಜೀವಿಗಳು, ಎಂದು ಜನರು ನಮ್ಮನ್ನು ಹಾಸ್ಯ ಮಾಡುವರು. ಬುದ್ಧಿವಂತಿಕೆ ಏನೋ ಮುಖ್ಯ. ಆದರೆ ಅದು ಸ್ವಲ್ಪ ದೂರ ಹೋಗಿ ನಿಲ್ಲುವುದು. ಹೃದಯದಿಂದ ಮಾತ್ರವೇ ಯಾವಾಗಲೂ ಪ್ರತಿಭೆ ಬರಬೇಕು. ಭಾವನೆಯ ಮೂಲಕವೇ ರಹಸ್ಯತಮ ವಿಷಯಗಳನ್ನು ತಿಳಿದುಕೊಳ್ಳಬಹುದು. ಆದಕಾರಣ ಭಾವಜೀವಿಯಾದ ಬಂಗಾಳಿಯೇ ಈ ಕೆಲಸವನ್ನು ಮಾಡಬೇಕಾಗಿದೆ.

\textbf{“ಉತ್ತಿಷ್ಠತ, ಜಾಗ್ರತ, ಪ್ರಾಪ್ಯವರಾನ್ನಿಬೋಧತ.”} ಏಳಿ ಜಾಗ್ರತರಾಗಿ ಗುರಿಯನ್ನು ಸೇರುವವರೆಗೆ ನಿಲ್ಲಬೇಡಿ. ಕಲ್ಕತ್ತೆಯ ಯುವಕರೇ, ಜಾಗ್ರತರಾಗಿ ಏಳಿ. ಸಮಯ ಸನ್ನಿಹಿತವಾಗಿದೆ. ನಿಮ್ಮ ಕಣ್ಣ ಮುಂದೆ ಆಗಲೇ ಎಲ್ಲಾ ವಿಕಾಸವಾಗುತ್ತಿದೆ. ನಿರ್ಭೀತರಾಗಿ, ಅಂಜಬೇಡಿ. ನಮ್ಮ ಶಾಸ್ತ್ರಗಳಲ್ಲಿ ಮಾತ್ರ ಭಗವಂತನಿಗೆ ‘ಅಭೀಃ ಅಭೀಃ’ ಎಂಬ ಗುಣವಾಚಕವನ್ನು ಆರೋಪಿಸುವರು. ನಾವು ನಿರ್ಭೀತರಾಗಬೇಕು. ಆಗ ನಮ್ಮ ಕೆಲಸ ಪರಿಪೂರ್ಣವಾಗುವುದು. ಏಳಿ, ಜಾಗ್ರತರಾಗಿ, ನಮ್ಮ ದೇಶಕ್ಕೆ ಅದ್ಭುತ ತ್ಯಾಗ ಬೇಕಾಗಿದೆ. ಯುವಕರು ಮಾತ್ರ ಇದನ್ನು ಮಾಡಬಲ್ಲರು. ಆಶಿಷ್ಠ, ದ್ರಢಿಷ್ಠ, ಬಲಿಷ್ಠ ಮೇಧಾವಿಗೆ ಈ ಕೆಲಸ ಮೀಸಲಾಗಿದೆ. ಕಲ್ಕತ್ತೆಯಲ್ಲಿ ಇಂತಹ ನೂರಾರು, ಸಹಸ್ರಾರು ಯುವಕರು ಇರುವರು. ನಾನು ಏನನ್ನೋ ಸ್ವಲ್ಪ ಸಾಧಿಸಿದೆ ಎಂದು ನೀವು ಹೇಳಿದರೆ, ನಾನು ಕಲ್ಕತ್ತೆಯಲ್ಲಿ ಕೆಲಸಕ್ಕೆ ಬಾರದೆ ಅಲೆಯುತ್ತಿದ್ದವನು, ನಾನೇ ಅಷ್ಟನ್ನು ಮಾಡಿದರೆ ನೀವು ಇನ್ನೆಷ್ಟು ಮಾಡಬಹುದು! ಇದನ್ನು ಜ್ಞಾಪಿಸಿ ಕೊಳ್ಳಿ. ಏಳಿ ಜಾಗ್ರತರಾಗಿ, ಪ್ರಪಂಚ ನಿಮ್ಮನ್ನು ಕರೆಯುತ್ತಿದೆ. ಭರತಖಂಡದ ಬೇರೆ ಪ್ರದೇಶಗಳಲ್ಲಿ ಬುದ್ಧಿ ಇದೆ. ಪಾಂಡಿತ್ಯವಿದೆ. ಆದರೆ ಉತ್ಸಾಹ ನಮ್ಮ ತಾಯ್ನಾಡಿನಲ್ಲಿ ಮಾತ್ರ ಇರುವುದು. ಅದು ವ್ಯಕ್ತವಾಗಬೇಕು. ಆದಕಾರಣ, ಕಲ್ಕತ್ತೆಯ ಯುವಕರೇ, ಶ್ರದ್ಧೋತ್ಸಾಹಗಳಿಂದ ಜಾಗ್ರತರಾಗಿ; ನೀವು ಬಡವರು, ಸ್ನೇಹಿತರಿಲ್ಲ ಎಂದು ಆಲೋಚಿಸಬೇಡಿ. ಹಣ ಮನುಷ್ಯನನ್ನು ತಯಾರು ಮಾಡುವುದನ್ನು ಯಾರು ನೋಡಿರುವರು? ಮನುಷ್ಯನೇ ಯಾವಾಗಲೂ ಹಣಮಾಡುವುದು. ಇಡೀ ಪ್ರಪಂಚವೇ ಮನುಷ್ಯನ ಶಕ್ತಿ, ಉತ್ಸಾಹ, ಶ್ರದ್ಧೆ ಇವುಗಳಿಂದಾಗಿದೆ.

ಉಪನಿಷತ್ತುಗಳಲ್ಲೆಲ್ಲಾ ಅತಿ ಸುಂದರವಾದ ಕಠೋಪನಿಷತ್ತನ್ನು ಯಾರು ಓದಿರುವರೋ ಅವರಿಗೆ ಈ ವಿಷಯ ಜ್ಞಾಪಕದಲ್ಲಿರಬಹುದು. ಒಬ್ಬ ರಾಜ ದೊಡ್ಡದೊಂದು ಯಾಗವನ್ನು ಮಾಡುತ್ತಿದ್ದನು. ಯೋಗ್ಯವಾದ ವಸ್ತುಗಳನ್ನು ದಾನಕೊಡುವ ಬದಲು ಕೆಲಸಕ್ಕೆ ಬಾರದ ಕುದುರೆಗಳನ್ನು ಮತ್ತು ಹಸುಗಳನ್ನು ಅವನು ಕೊಡುತ್ತಿದ್ದನು. ಆ ಸಮಯದಲ್ಲಿ ನಚಿಕೇತನೆಂಬ ಅವನ ಮಗನಿಗೆ ಶ್ರದ್ಧೆಯುಂಟಾಯಿತು ಎಂದು ಉಪನಿಷತ್ತು ಹೇಳುವುದು. ಶ್ರದ್ಧೆ ಎಂಬ ಪದವನ್ನು ನಾನು ಭಾಷಾಂತರ ಮಾಡುವುದಿಲ್ಲ. ಅದು ಬಹಳ ತಪ್ಪಾಗುವುದು. ತಿಳಿದುಕೊಳ್ಳುವುದಕ್ಕೆ ಅತಿ ಸುಂದರವಾದ ಪದ ಇದು. ಎಲ್ಲಾ ಅದರ ಮೇಲಿದೆ. ಅದು ಹೇಗೆ ಕೆಲಸ ಮಾಡುತ್ತದೆ ಎಂಬುದನ್ನು ನೋಡೋಣ. ತಕ್ಷಣವೇ ಈ ಭಾವನೆ ನಚಿಕೇತನ ಮನಸ್ಸಿನಲ್ಲಿ ಹೊಳೆಯಿತು. “ನಾನು ಹಲವರಿಗೆ ಮೇಲಾಗಿರುವೆನು. ಎಲ್ಲೋ ಕೆಲವರಿಗೆ ಮಧ್ಯದಲ್ಲಿರುವೆನು. ಆದರೆ ಎಲ್ಲೂ ನಾನು ಕೊನೆಯವನಲ್ಲ, ನಾನು ಏನನ್ನಾದರೂ ಮಾಡಬಹುದು.” ಈ ಧೈರ್ಯ ಹೆಚ್ಚಾಯಿತು. ಮನಸ್ಸಿನಲ್ಲಿದ್ದ ಸಾವಿನ ಸಮಸ್ಯೆಯನ್ನು ಬಗೆಹರಿಸಬೇಕೆಂದು ಇಚ್ಛಿಸಿದನು. ಮೃತ್ಯುವಿನ ಮನೆಗೆ ಹೋದರೆ ಮಾತ್ರ ಈ ಸಮಸ್ಯೆಯನ್ನು ಬಗೆಹರಿಸಲು ಸಾಧ್ಯ. ಹುಡುಗ ಅಲ್ಲಿಗೆ ಹೊರಟನು. ಆ ಧೀರ ನಚಿಕೇತ ಮೃತ್ಯುವಿನ ಮನೆಗೆ ಹೋಗಿ ಮೂರು ದಿನಗಳವರೆಗೆ ಕಾದು ಕುಳಿತನು. ಅವನಿಗೆ ಬೇಕಾದುದು ಹೇಗೆ ದೊರಕಿತು ಎಂಬುದು ನಿಮಗೆ ಗೊತ್ತಿದೆ. ನಮಗೆ ಬೇಕಾಗಿರುವುದು ಇಂತಹ ಶ್ರದ್ಧೆ. ದುರದೃಷ್ಟವಶಾತ್​ ಭರತಖಂಡದಿಂದ ಇದು ಮುಕ್ಕಾಲುಪಾಲು ಕಣ್ಮರೆಯಾಗಿದೆ. ಅದಕ್ಕಾಗಿಯೇ ನಾವು ಇಂತಹ ಅಧೋಗತಿಗೆ ಬಂದಿರುವುದು. ಒಬ್ಬನಿಗೂ ಮತ್ತೊಬ್ಬನಿಗೂ ವ್ಯತ್ಯಾಸವೆಂದರೆ ಅವರಲ್ಲಿ ಇರುವ ಶ್ರದ್ಧೆಯ ತರತಮದಲ್ಲಿ, ಅನ್ಯಥಾ ಅಲ್ಲ. ಒಬ್ಬನನ್ನು ಪ್ರಖ್ಯಾತನನ್ನಾಗಿ ಮಾಡುವುದು ಶ್ರದ್ಧೆ. ಅಧೋಗತಿಗೆ ಒಯ್ಯುವುದು ಅಶ್ರದ್ಧೆ. ಯಾರು ತಾನು ದುರ್ಬಲ ಎಂದು ಆಲೋಚಿಸುತ್ತಿರುವನೋ ಅವನು ದುರ್ಬಲನೇ ಆಗುತ್ತಾನೆ ಎಂದು ನನ್ನ ಗುರು ಹೇಳುತ್ತಿದ್ದರು. ಇದು ಸತ್ಯ. ಈ ಶ್ರದ್ಧೆ ನಿಮಗೆ ಬರಬೇಕು. ಪಾಶ್ಚಾತ್ಯ ಜನಾಂಗದಲ್ಲಿ ವ್ಯಕ್ತವಾಗುತ್ತಿರುವ ಭೌತಿಕ ಶಕ್ತಿಯ ಮೂಲವೆಲ್ಲ ಶ್ರದ್ಧೆಯಲ್ಲಿದೆ. ಅವರು ತಮ್ಮ ಮಾಂಸಖಂಡದ ಶಕ್ತಿಯನ್ನು ನಂಬಿದರು. ನೀವು ನಿಮ್ಮ ಆತ್ಮಶಕ್ತಿಯನ್ನು ನಂಬಿದರೆ ಮತ್ತೆ ಎಷ್ಟನ್ನು ಸಾಧಿಸಬಹುದು? ನಿಮ್ಮ ಶಾಸ್ತ್ರಗಳಲ್ಲಿ ಮಹರ್ಷಿಗಳು ಏಕವಾಣಿಯಿಂದ ಆತ್ಮ ಅನಂತವಾದುದು, ಶಕ್ತಿ ಅನಂತವಾದುದು ಎಂದು ಬೋಧಿಸಿರುವರು. ಅದನ್ನು ನಂಬಿ. ಆತ್ಮನನ್ನು ಯಾವುದೂ ನಾಶಮಾಡಲಾರದು. ಅದರಲ್ಲಿ ಅನಂತಶಕ್ತಿ ಇದೆ. ಅದನ್ನು ವ್ಯಕ್ತಪಡಿಸಬೇಕಷ್ಟೆ. ಇಲ್ಲೇ ಭಾರತದ ತತ್ತ್ವಶಾಸ್ತ್ರಗಳಿಗೂ ಇತರ ಶಾಸ್ತ್ರಗಳಿಗೂ ಇರುವ ವ್ಯತ್ಯಾಸ. ದ್ವೈತಿಗಳಾಗಲೀ, ವಿಶಿಷ್ಟಾದ್ವೈತಿಗಳಾಗಲೀ, ಅದ್ವೈತಿಗಳಾಗಲೀ ಎಲ್ಲರೂ ಆತ್ಮನಲ್ಲಿ ಆಗಲೇ ಎಲ್ಲವೂ ಇದೆ ಎಂದು ದೃಢವಾಗಿ ನಂಬುವರು. ಅದು ವ್ಯಕ್ತವಾಗಬೇಕು ಅಷ್ಟೆ. ಈ ಆತ್ಮಶ್ರದ್ಧೆ ಇಂದು ನನಗೆ ಬೇಕಾಗಿದೆ, ನಿಮಗೆಲ್ಲಾ ಬೇಕಾಗಿದೆ. ಇದನ್ನು ಪಡೆದುಕೊಳ್ಳುವ ಮಹಾಸಾಹಸ ನಿಮ್ಮ ಮುಂದೆ ಇದೆ. ಎಲ್ಲವನ್ನೂ ಹಾಸ್ಯಮಾಡುವ ಹುಡುಗಾಟಿಕೆಯ ಸ್ವಭಾವ ನಮ್ಮ ಜನಾಂಗದ ಜೀವನದ ಮೇಲೆ ದಾಳಿ ಇಡುತ್ತಿದೆ. ಅದನ್ನು ತ್ಯಜಿಸಿ. ಧೀರರಾಗಿ, ಶ್ರದ್ಧಾವಂತರಾಗಿ, ಉಳಿದುದೆಲ್ಲಾ ಸ್ವಾಭಾವಿಕವಾಗಿ ಸಿದ್ಧಿಸುವುದು.

ನಾನು ಇನ್ನೂ ಏನನ್ನೂ ಮಾಡಿಲ್ಲ. ನೀವು ಕೆಲಸವನ್ನು ಸಾಧಿಸಬೇಕಾಗಿದೆ. ನಾಳೆ ನಾನು ಸತ್ತರೂ ಕಾರ್ಯ ನಿಲ್ಲುವುದಿಲ್ಲ. ಸಹಸ್ರಾರು ಜನರು ಕೆಲಸಕ್ಕೆ ಕೈಹಾಕಿ ನನ್ನ ಕಲ್ಪನೆಗೂ ಅತೀತವಾಗಿ ಕೆಲಸ ಮಾಡುತ್ತಾ ಹೋಗುವರು ಎಂದು ದೃಢವಾಗಿ ನಂಬುತ್ತೇನೆ. ನನಗೆ ನನ್ನ ದೇಶದಲ್ಲಿ ಶ್ರದ್ಧೆಯಿದೆ. ಅದರಲ್ಲೂ ವಂಗದೇಶದ ಯುವಕರಲ್ಲಿ ನನಗೆ ಹೆಚ್ಚು ಶ್ರದ್ಧೆ ಇದೆ. ಯುವಕರ ಪಾಲಿಗೆ ಬಂದ ಕರ್ತವ್ಯಗಳಲ್ಲಿ ಗುರುತರವಾದುದು ವಂಗಯುವಕರ ಮೇಲಿದೆ. ಕಳೆದ ಹತ್ತು ವರುಷಗಳಿಂದ ಭರತಖಂಡವನ್ನೆಲ್ಲಾ ಸಂಚರಿಸಿರುವೆನು. ಭರತಖಂಡವನ್ನು ಯೋಗ್ಯ ಆಧ್ಯಾತ್ಮಿಕ ಶಿಖರಕ್ಕೆ ಸಾಗಿಸುವ ಶಕ್ತಿ ವಂಗ ಯುವಕರಿಂದ ಬರುವುದೆಂಬುದು ನನ್ನ ದೃಢನಂಬಿಕೆ. ಅನಂತ ಶ್ರದ್ಧೆ, ಉತ್ಸಾಹಗಳಿಂದ ತುಂಬಿ ತುಳುಕಾಡುವ ವಂಗಯುವಕರಿಂದ, ಪೂರ್ವಜರ ಸನಾತನ ತತ್ತ್ವಗಳನ್ನು ಜಗತ್ತಿಗೆಲ್ಲಾ ಪ್ರಚಾರಮಾಡುವವರು ಬರುವರು. ನೀವು ಮಾಡಬೇಕಾದ ಪವಿತ್ರ ಕೆಲಸವಿದು. “ಏಳಿ, ಜಾಗ್ರತರಾಗಿ, ಗುರಿ ಸೇರುವವರೆಗೂ\break ನಿಲ್ಲಬೇಡಿ” ಎಂದು ನಿಮಗೆ ಜ್ಞಾಪಿಸಿ ಮುಕ್ತಾಯಗೊಳಿಸುತ್ತೇನೆ. ಅಂಜಬೇಡಿ, ಮಾನವ ಇತಿಹಾಸದಲ್ಲೆಲ್ಲಾ ಮಹಾಶಕ್ತಿ ಇದ್ದುದು ಜನ ಸಾಮಾನ್ಯರಲ್ಲಿ. ಅವರಲ್ಲಿ ಜಗದ ಮಹಾ ವಿಭೂತಿಗಳೆಲ್ಲ ಜನಿಸಿರುವರು. ಇತಿಹಾಸ ಪುನರಾವೃತ್ತಿಯಾಗಲೇಬೇಕು. ಯಾವುದಕ್ಕೂ ಅಂಜಬೇಡಿ. ಅದ್ಭುತ ಕಾರ್ಯಗಳನ್ನು ನೀವು ಸಾಧಿಸುತ್ತೀರಿ. ಅಂಜಿದೊಡನೆಯೇ ನೀವು ಕೆಲಸಕ್ಕೆ ಬಾರದವರಾಗುತ್ತೀರಿ. ಪ್ರಪಂಚದ ದುಃಖಕ್ಕೆ ಮುಖ್ಯ ಕಾರಣ ಅಂಜಿಕೆ. ಇದೇ ಎಲ್ಲಕ್ಕಿಂತ ದೊಡ್ಡ ಮೂಢನಂಬಿಕೆ. ನಿಮ್ಮ ಕಷ್ಟಕ್ಕೆಲ್ಲಾ ಕಾರಣ ಅಂಜಿಕೆ. ನಿರ್ಭಯತೆ ಈ ಕ್ಷಣವೇ ಸ್ವರ್ಗವನ್ನು ಕೂಡ ಕೊಡುವುದು. ಆದಕಾರಣ “ಏಳಿ, ಜಾಗ್ರತರಾಗಿ, ಗುರಿಯನ್ನು ಸೇರುವವರೆಗೂ ನಿಲ್ಲದಿರಿ.”

ಮಹನೀಯರೇ, ನೀವು ತೋರಿದ ಪ್ರೀತಿಗೆ ಪುನಃ ಧನ್ಯವಾದವನ್ನು ಅರ್ಪಿಸಲು ಅನುಮತಿ ಕೊಡಿ. ನಾನು ಜಗತ್ತಿಗೆ ಸ್ವಲ್ಪವಾದರೂ ಉಪಯೋಗವುಳ್ಳವನಾಗಬೇಕು, ಎಲ್ಲಕ್ಕಿಂತ ಹೆಚ್ಚಾಗಿ ನನ್ನ ದೇಶಕ್ಕೆ, ನನ್ನ ದೇಶದ ಜನರಿಗೆ ನೆರವಾಗಬೇಕು ಎಂಬುದು ನನ್ನ ಇಚ್ಛೆ, ನನ್ನ ಅಂತರಂಗದಲ್ಲಿರುವ ತೀವ್ರ ಆಕಾಂಕ್ಷೆ.

