
\chapter{ಸಂನ್ಯಾಸ-ಅದರ ಆದರ್ಶ ಮತ್ತು ಅನುಷ್ಠಾನ}

(ಸ್ವಾಮಿ ವಿವೇಕಾನಂದರು ಎರಡನೆಯ ಬಾರಿ ಪಶ್ಚಿಮ ದೇಶಗಳಿಗೆ ಹೊರಟಾಗ ಬೇಲೂರು ಮಠದ ಸಂನ್ಯಾಸಿಗಳು, ೧೮೯೯, ಜೂನ್​ ೧೯ರಂದು, ಅವರಿಗೊಂದು ಬೀಳ್ಕೊಡುಗೆಯ ಸಮಾರಂಭವನ್ನು ಏರ್ಪಡಿಸಿದ್ದರು. ಆ ಸಂದರ್ಭದಲ್ಲಿ ಸ್ವಾಮೀಜಿಯವರು ಸಂನ್ಯಾಸಿಗಳನ್ನು ಉದ್ದೇಶಿಸಿ ಭಾಷಣ ಮಾಡಿದರು. ಅದರ ಸಾರಾಂಶವು ಮಠದ ದಿನಚರಿಯಲ್ಲಿ ಈ ಕೆಳಗಿನಂತೆ ದಾಖಲಾಗಿದೆ.)

ದೀರ್ಘ ಉಪನ್ಯಾಸಕ್ಕೆ ಇದು ವೇಳೆಯಲ್ಲ. ನೀವು ಅನುಷ್ಠಾನಕ್ಕೆ ತರಬೇಕಾದ ಕೆಲವು ವಿಷಯಗಳನ್ನು ಮಾತ್ರ ನಿಮಗೆ ಸಂಕ್ಷೇಪವಾಗಿ ಹೇಳಬಯಸುತ್ತೇನೆ. ಮೊದಲು ನಾವು ಆದರ್ಶವನ್ನು ತಿಳಿದುಕೊಳ್ಳಬೇಕು. ಅನಂತರ ಅದನ್ನು ಅನುಷ್ಠಾನಕ್ಕೆ ತರುವ ವಿಧಾನವನ್ನು ತಿಳಿದುಕೊಳ್ಳಬೇಕು. ಯಾರು ಸಂನ್ಯಾಸಿಗಳೋ ಅವರು ಪರರಿಗೆ ಹಿತವನ್ನು ಮಾಡುವುದಕ್ಕೆ ಪ್ರಯತ್ನಿಸಬೇಕು. ಏಕೆಂದರೆ ಸಂನ್ಯಾಸವೆಂದರೆ ಅದೇ ಅರ್ಥ. ತ್ಯಾಗದ ಮೇಲೆ ದೀರ್ಘ ಭಾಷಣ ಮಾಡುವುದಕ್ಕೆ ಸಮಯವಿಲ್ಲ. ಸಂನ್ಯಾಸವೆಂದರೆ ಮೃತ್ಯು ಪ್ರೇಮ ಎಂದು ಸಂಕ್ಷೇಪವಾಗಿ ಹೇಳುತ್ತೇನೆ. ಪ್ರಾಪಂಚಿಕರು ಜೀವನವನ್ನು ಪ್ರೀತಿಸುವರು. ಸಂನ್ಯಾಸಿ ಮೃತ್ಯುವನ್ನು ಪ್ರೀತಿಸಬೇಕು. ಹಾಗಾದರೆ ನಾವು ಆತ್ಮಹತ್ಯೆ ಮಾಡಿಕೊಳ್ಳಬೇಕೆ? ಹಾಗಲ್ಲ. ಆತ್ಮಹತ್ಯೆ ಮಾಡಿಕೊಳ್ಳಬಯಸುವವರು ಮೃತ್ಯು ಪ್ರೇಮಿಗಳಲ್ಲ. ಒಮ್ಮೆ ಅವರು ಆತ್ಮಹತ್ಯೆಗೆ ಪ್ರಯತ್ನಿಸಿ ಸೋತರೆ, ಪುನಃ ಅದನ್ನು ಮಾಡಲಿಚ್ಛಿಸುವುದಿಲ್ಲ. ಹಾಗಾದರೆ ಒಂದು ಆದರ್ಶಕ್ಕಾಗಿ ಸಾಯೋಣ. ನಮ್ಮ ಕ್ರಿಯೆಗಳೆಲ್ಲ - ಊಟಮಾಡುವುದು, ಕುಡಿಯುವುದು, ಪ್ರತಿಯೊಂದು ಆತ್ಮತ್ಯಾಗಕ್ಕೆ ಸಹಾಯಕವಾಗಲಿ. ಆಹಾರದಿಂದ ನಿಮ್ಮ ದೇಹವನ್ನು ಪುಷ್ಟಿಗೊಳಿಸುವಿರಿ. ನಿಮ್ಮ ದೇಹವನ್ನು ಇತರರ ಸೇವೆಗಾಗಿ ಉಪಯೋಗಿಸದೆ ಇದ್ದರೆ ಅದನ್ನು ಪೋಷಿಸಿ ಪ್ರಯೋಜನವೇನು? ಗ್ರಂಥಗಳನ್ನು ಓದಿ ನಿಮ್ಮ ಮನಸ್ಸನ್ನು ಬೆಳಸಿಕೊಳ್ಳುವಿರಿ. ಇದನ್ನು ಪ್ರಪಂಚಕ್ಕೆ ವಿನಿಯೋಗ ಮಾಡದಿದ್ದರೆ ಇದರಿಂದಲೂ ಪ್ರಯೋಜನವಿಲ್ಲ. ಏಕೆಂದರೆ ಇಡೀ ಜಗತ್ತೇ ಒಂದು, ನೀವು ಅದರ ಒಂದು ಸಣ್ಣ ಅಂಶವಷ್ಟೆ. ಆದ್ದರಿಂದ ನಿಮ್ಮ ಕ್ಷುದ್ರ ಅಹಂಕಾರವನ್ನು ವೃದ್ಧಿಪಡಿಸುವುದಕ್ಕಿಂತ ನಿಮ್ಮ ಲಕ್ಷಾಂತರ ಸಹೋದರರ ಸೇವೆ ಮಾಡುವುದು ಮೇಲು.

\begin{verse}
\textbf{ಸರ್ವತಃ ಪಾಣಿಪಾದಂ ತತ್​ ಸರ್ವತೋಽಕ್ಷಿಶಿರೋಮುಖಮ್​~।}\\\textbf{ಸರ್ವತಃ ಶ್ರುತಿಮಲ್ಲೋಕೇ ಸರ್ವಮಾವೃತ್ಯ ತಿಷ್ಠತಿ~॥}
\end{verse}

\vskip 2pt

“ಎಲ್ಲೆಲ್ಲಿಯೂ ಅದರ ಕೈಕಾಲುಗಳು, ಎಲ್ಲೆಲ್ಲಿಯೂ ಅದರ ಕಣ್ಣು ತಲೆ ಮುಖಗಳು, ಎಲ್ಲೆಲ್ಲಿಯೂ ಅದರ ಕಿವಿಗಳು-ಹೀಗೆ ಅದು ಎಲ್ಲವನ್ನೂ ಆವರಿಸಿಕೊಂಡಿದೆ.” ನೀವು ಹೀಗೇ ಕ್ರಮೇಣ ಮೃತ್ಯುಮುಖರಾಗಬೇಕು. ಇಂತಹ ಮರಣದಲ್ಲಿ ಸ್ವರ್ಗವಿದೆ. ಸರ್ವ ಶುಭದ ಆಗರ ಇದು. ಇದಕ್ಕೆ ವಿರೋಧ\-ವಾಗಿರುವುದೇ ನರಕ, ಪಾಪ.

\vskip 2pt

ಅನಂತರ ಈ ಆದರ್ಶವನ್ನು ಅನುಷ್ಠಾನಕ್ಕೆ ತರುವ ಮಾರ್ಗ. ಅನುಷ್ಠಾನಕ್ಕೆ ತರಲು ಅಸಾಧ್ಯವಾದ ಆದರ್ಶವನ್ನು ನಾವು ಇಟ್ಟುಕೊಳ್ಳಬಾರದು - ಇದನ್ನು ತಿಳಿದುಕೊಳ್ಳಬೇಕು. ಸಾಮರ್ಥ್ಯಕ್ಕೆ ನಿಲುಕದ ಆದರ್ಶವಿದ್ದರೆ ಆ ದೇಶ ದುರ್ಬಲವಾಗಿ ಅಧೋಗತಿಗೆ ಇಳಿಯುವುದು. ಈ ಸ್ಥಿತಿ ಬೌದ್ಧ ಮತ್ತು ಜೈನಧರ್ಮಗಳು ಜಾರಿಗೆ ತಂದ ಸುಧಾರಣೆಗಳ ಅನಂತರದ ಕಾಲದಲ್ಲಿ ಪ್ರಾಪ್ತವಾಯಿತು. ಹಾಗೆಯೇ ಅನುಷ್ಠಾನಕ್ಕೆ ಅತಿಯಾದ ಪ್ರಾಧಾನ್ಯವನ್ನು ಕೊಡುವುದೂ ತಪ್ಪು. ನಿಮ್ಮಲ್ಲಿ ಸ್ವಲ್ಪವೂ ಕಲ್ಪನೆ ಇಲ್ಲದೇ ಇದ್ದರೆ, ನಿಮಗೆ ದಾರಿ ತೋರುವುದಕ್ಕೆ ಒಂದು ಆದರ್ಶವು ಇಲ್ಲದೇ ಇದ್ದರೆ ನೀವು ಪ್ರಾಣಿಗಳಂತಾಗುತ್ತೀರಿ. ಆದ್ದರಿಂದ ನಾವು ಆದರ್ಶವನ್ನು ಕೆಳಗೆ ಎಳೆಯಲೂ ಕೂಡದು, ಹಾಗೆಯೇ ಅನುಷ್ಠಾನವನ್ನೂ ಕಡೆಗಣಿಸಬಾರದು. ಈ ಎರಡು ಅತಿಗಳಿಂದಲೂ ಪಾರಾಗಬೇಕು. ನಮ್ಮ ದೇಶದಲ್ಲಿ ಹಿಂದಿನ ಕಾಲದ ಆದರ್ಶವು ಒಂದು ಗುಹೆಯಲ್ಲಿ ಕುಳಿತು ಧ್ಯಾನಮಾಡಿ ಸಾಯುವುದು. ಮುಕ್ತಿಗಾಗಿಯೂ ಇತರರಿಗಿಂತ ಮುಂದೆ ಹೋಗುವುದು ತಪ್ಪು. ತಮ್ಮ ಸಹೋದರರ ಮುಕ್ತಿಗಾಗಿ ನಾವು ಪ್ರಯತ್ನಿಸದೆ ಇದ್ದರೆ ನಮಗೂ ಮುಕ್ತಿ ಇಲ್ಲವೆಂಬುದನ್ನು ಇಂದೋ ನಾಳೆಯೋ ಕಲಿಯಬೇಕಾಗಿದೆ. ಅದ್ಭುತ ಆದರ್ಶ ಮತ್ತು ಅದ್ಭುತ ಅನುಷ್ಠಾನ ಇವೆರಡನ್ನು ನಮ್ಮ ಜೀವನದಲ್ಲಿ ಒಂದು ಗೂಡಿಸಲು ಯತ್ನಿಸಬೇಕು. ಈ ಕ್ಷಣ ಧ್ಯಾನಪರವಶರಾಗಲು ಸಾಧ್ಯವಾಗಿರಬೇಕು. ಮರುಕ್ಷಣ ಎದುರಿಗಿರುವ ಹೊಲವನ್ನು ಉಳಲು ಸಿದ್ಧರಾಗಿರಬೇಕು. ಈಗ ಶಾಸ್ತ್ರದ ಜಟಿಲ ಸಮಸ್ಯೆಯನ್ನು ಬಗೆಹರಿಸಲು ಸಿದ್ಧರಾಗಿರಬೇಕು. ಮರುಕ್ಷಣ ತೋಟದ\break ತರಕಾರಿಯನ್ನು ಅಂಗಡಿಯಲ್ಲಿ ಮಾರಲು ಸಿದ್ಧರಾಗಿರಬೇಕು. ಅತಿ ಸಣ್ಣ\break ಕೆಲಸವನ್ನು ಕೂಡ ಮಾಡಲು ಸಿದ್ಧರಾಗಿರಬೇಕು. ಮಾತ್ರವಲ್ಲ, ಎಲ್ಲಾ ಕಡೆಗಳ\-ಲ್ಲಿಯೂ - ಅದನ್ನು ಮಾಡಲು ಸಿದ್ಧರಾಗಿರಬೇಕು.

\vskip 3pt

ಅನಂತರ ನಾವು ಜ್ಞಾಪಕದಲ್ಲಿಡಬೇಕಾದುದೇ, ಈ ಸಂಘದ ಉದ್ದೇಶ “ಪುರುಷ-ನಿರ್ಮಾಣ” ಎಂಬುದು. ಋಷಿಗಳ ಬೋಧನೆಯನ್ನು ನೀವು ಕಲಿತುಕೊಳ್ಳುವುದು ಮಾತ್ರವಲ್ಲ. ಆ ಋಷಿಗಳೆಲ್ಲಾ ಹೊರಟು ಹೋದರು, ಅವರ ಅಭಿಪ್ರಾಯಗಳೂ ಜಗತ್ತಿನಲ್ಲಿ ಅವರೊಂದಿಗೆ ಹೋದುವು. ನೀವೇ ಋಷಿಗಳಾಗಬೇಕು. ಜಗತ್ತಿನಲ್ಲಿ ಜನಿಸಿದ ಶ್ರೇಷ್ಠತಮ ವ್ಯಕ್ತಿಗಳಂತೆ, ಅವತಾರಗಳಂತೆ, ನೀವು ಕೂಡ ಮನುಷ್ಯರೆ. ಕೇವಲ ಪುಸ್ತಕ ಪಾಂಡಿತ್ಯದಿಂದ ಪ್ರಯೋಜನವೇನು? ಧ್ಯಾನದಿಂದ ತಾನೇ ಏನು ಪ್ರಯೋಜನ? ಮಂತ್ರ ತಂತ್ರಗಳಿಂದ ಏನು ಪ್ರಯೋಜನ? ನೀವು ನಿಮ್ಮ ಕಾಲುಗಳ ಮೇಲೆ ನಿಲ್ಲಬೇಕು. ಪುರುಷ ನಿರ್ಮಾಣವೆಂಬ ಈ ಒಂದು ಹೊಸ ವಿಧಾನ ನಿಮಗೆ ಬೇಕು. ಶಕ್ತಿಯಷ್ಟೇ ಶಕ್ತಿಶಾಲಿಯಾಗಿದ್ದರೂ ಸ್ತ್ರೀಯರಂತೆ ಕೋಮಲ ಹೃದಯವುಳ್ಳವನು ಯಾವನೊ ಅವನೇ ನಿಜವಾದ ಮಾನವ. ನಮ್ಮ ಸುತ್ತಲಿರುವ ಮಾನವ ಕೋಟಿಯ ಮೇಲೆ ಅನುಕಂಪ ತಾಳಬೇಕು. ಆದರೂ ಬಲಾಢ್ಯನಾಗಿರಬೇಕು, ಅದಮ್ಯನಾಗಿರಬೇಕು; ಆದರೂ ವಿಧೇಯತೆ ಇರಬೇಕು. ಇದು ವಿರೋಧಾಭಾಸದಂತೆ ತೋರಬಹುದು. ಆದರೂ ತೋರಿಕೆಗೆ ಪರಸ್ಪರ ವಿರೋಧದಂತೆ ಕಾಣುತ್ತಿರುವ ಈ ಗುಣಗಳು ನಿಮ್ಮಲ್ಲಿ ಇರಬೇಕು. ನಿಮ್ಮ ಹಿರಿಯರು ನದಿಗೆ ಧುಮುಕಿ ಮೊಸಳೆ ಹಿಡಿಯಿರಿ ಎಂದು ನಿಮಗೆ ಆಜ್ಞಾಪಿಸಿದರೂ ಮೊದಲು ಅದನ್ನು ಪಾಲಿಸಿ ಅನಂತರ ಈ ವಿಷಯವಾಗಿ ಅವರೊಡನೆ ಚರ್ಚಿಸಬೇಕು. ಅಪ್ಪಣೆ ತಪ್ಪಾಗಿದ್ದರೂ ಮೊದಲು ಅದನ್ನು ಪರಿಪಾಲಿಸಿ, ಅನಂತರ ವಿರೋಧಿಸಿ. ವಿಶೇಷವಾಗಿ ಬಂಗಾಳದಲ್ಲಿರುವ ಮತಗಳ ಒಂದು ಕೇಡು ಏನೆಂದರೆ, ಯಾವನಲ್ಲಾದರೂ ತನ್ನದೇ ಆದ ಒಂದು ಅಭಿಪ್ರಾಯವಿದ್ದರೆ, ಅವನು ಕೂಡಲೆ ಬೇರೊಂದು ಮತವನ್ನು ಸ್ಥಾಪಿಸುತ್ತಾನೆ. ತಾಳಿಕೊಂಡಿರಲು ಅವನಿಗೆ ಸಹನೆಯಿಲ್ಲ. ಆದ್ದರಿಂದ ಸಂಘದ ಮೇಲೆ ನಿಮಗೆ ಅಪೂರ್ವ ಶ್ರದ್ಧಾ ವಿಶ್ವಾಸಗಳಿರಬೇಕು. ಆಜ್ಞೋಲ್ಲಂಘನೆಗೆ ಇಲ್ಲಿ ಅವಕಾಶವಿಲ್ಲ, ನಿರ್ದಾಕ್ಷಿಣ್ಯವಾಗಿ ಅದನ್ನು ನಿರ್ಮೂಲ ಮಾಡಿ. ಈ ಸಂಘದಲ್ಲಿ ಅವಿಧೇಯರಿಗೆ ಎಡೆ ಇಲ್ಲ. ಅವರನ್ನು ಆಚೆಗೆ ಕಳುಹಿಸಬೇಕು. ಸಂಘದಲ್ಲಿ ದ್ರೋಹಿಗಳಾರೂ ಇರಕೂಡದು, ನೀವು ವಾಯುವಿನಂತೆ ಸ್ವಚ್ಛಂದವಾಗಿರಬೇಕು, ಹಾಗೆಯೇ ಈ ಗಿಡದಂತೆ ಅಥವಾ ನಾಯಿಯಂತೆ ವಿಧೇಯರೂ ಆಗಿರಬೇಕು.

