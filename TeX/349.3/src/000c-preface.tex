
\chapter*{ಮುನ್ನುಡಿ}

ಆಕಸ್ಮಿಕ ಎಂಬುದು ಅಪೂರ್ಣ ಜ್ಞಾನ; ತನ್ನನ್ನು ತಾನು ಮರೆಮಾಡಿಕೊಳ್ಳಲೆಳಸುವ ಒಂದು ವಿಧಾನಕ್ಕೆ ನಾವು ಇಡುವ ಹೆಸರು. ನಾವು ಯಾವುದನ್ನು ಆಕಸ್ಮಿಕ ಎಂದು ಕರೆಯುತ್ತೇವೆಯೋ ಅದು ನಿಜವಾಗಿಯೂ ಆಕಸ್ಮಿಕವಾದುದಲ್ಲ. ಪೂರ್ವಾಪರವನ್ನು ಅಖಂಡವಾಗಿಯೂ ಏಕವಾಗಿಯೂ ಒಳಗೊಳ್ಳುವ ತ್ರಿಕಾಲ ದರ್ಶಿಯಾದ ಪೂರ್ಣ ದೃಷ್ಟಿಗೆ ಆಕಸ್ಮಿಕ ಎಂಬುದಿಲ್ಲ. ನಮ್ಮ ಅಲ್ಪಜ್ಞಾನ ತನಗೆ ಅನಿರೀಕ್ಷಿತವಾದುದ್ದನ್ನು ಹಾಗೆ ಕರೆದ ಮಾತ್ರಕ್ಕೆ ಭೂಮಜ್ಞಾನವೂ ಅದನ್ನು ಹಾಗೆ ಗ್ರಹಿಸಬೇಕಾಗಿಲ್ಲ. ಸರ್ವಜ್ಞವಾದ ಸರ್ವೇಚ್ಛಾಶಕ್ತಿ ಸುರುಳಿಬಿಚ್ಚುವ ತನ್ನ ಲೋಕ ಲೀಲೆಯಲ್ಲಿ ಯಾವುದನ್ನೂ ಅನಿಶ್ಚಯಕ್ಕೆ ಬಿಡುವುದಿಲ್ಲ ಎಂಬ ಪೂರ್ಣ ದೃಷ್ಟಿಯ ಶ್ರದ್ಧೆಯಿಂದ ಸಮನ್ವಿತವಾದ ಪ್ರಜ್ಞೆ ಸಾಮಾನ್ಯ ಬುದ್ಧಿಗೆ ಆಕಸ್ಮಿಕ ಎಂದು ತೋರುವುದ\-ರಲ್ಲಿಯೂ ಅರ್ಥ, ಉದ್ದೇಶ, ವ್ಯೂಹಗಳನ್ನು ಸಂದರ್ಶಿಸುತ್ತದೆ.

ಜನವರಿ ಇಪ್ಪತ್ತಾರನೆ ತೇದಿ ಭಾರತೀಯರಾದ ನಮಗೆ ಈಗ ಒಂದು ಮಹಾ–ಸಂಕೇತದ ದಿನವಾಗಿ ಪರಿಣಮಿಸಿದೆ. ಸ್ವತಂತ್ರ ಭಾರತರಾಷ್ಟ್ರ ತಾನುರಿಪಬ್ಲಿಕ್​ ಎಂದು ಘೋಷಿಸಿಕೊಂಡ ಮಹಾ ಸುದಿನವದು. ಅದೊಂದು ಉತ್ಥಾನದ ಮತ್ತು ಉಜ್ಜೀವನದ ಪ್ರಾರಂಭೋತ್ಸವದ ದಿನ; ನವೋತ್ಸಾಹದ ದಿನ; ನವೀನತಾ ಜೀವನದ ದೀಕ್ಷಾದಿನ. ನವಭಾರತದ ಉದ್ಧಾರೋ\-ನ್ಮುಖವೂ ವಿಕಾಸಶೀಲವೂ ಆಗಿರುವ ಸಂಕಲ್ಪ–ಕುಂಡಲಿನಿ ಮತ್ತೆ ಮತ್ತೆ ಪೊರೆಗಳಚಿ ಪುನಃ ಪುನಃ ಊರ್ಧ್ವಗಾಮಿಯಾಗಲು ತನ್ನ ಸಾವಿರಾರು ಹೆಡೆಗಳನ್ನೂ ಎತ್ತಿ, ವರುಷ ವರುಷವೂ ಹೊಸ ಕಂಕಣ ಕಟ್ಟುವ ಪವಿತ್ರದಿನ. ಸರ್ವೋದಯ ರಾಜ್ಯದ ಸುಪೂಜ್ಯ ದಿನ.

ಆ ದಿನ ಭಾರತದ ಕೋಟಿ ಕೋಟಿ ಹೃದಯಗಳಲ್ಲಿ ಒಂದು ಮಿಂಚು ಸಂಚರಿಸುತ್ತದೆ. ಒಂದು ಶಕ್ತಿ ಸ್ಪಂದಿಸುತ್ತದೆ. ಕುರುಡನಾದರೂ ಚಿಂತೆಯಿಲ್ಲ; ಮೂಗನಾದರೂ ಚಿಂತೆಯಿಲ್ಲ; ಒಮ್ಮೆ ಕಂಬನಿಗರೆದರೂ ಚಿಂತೆಯಿಲ್ಲ; ಒಮ್ಮೆ ಬಿಸುಸುಯ್ದರೂ ಚಿಂತೆಯಿಲ್ಲ; ಯಾವುದೋ ಒಂದು ಹೇರಾಸೆ ಆಬಾಲವೃದ್ಧರ ಪ್ರಾಣ ಕೋಶಗಳಲ್ಲಿ ತುಂಬಿ ತುಳುಕುತ್ತದೆ. ಆ ಚಿನ್ಮಯ ಸ್ಪಂದನದ ಮೂಲವೆಲ್ಲಿ? ಅದು ಎಂದು ಮೊದಲಾಯಿತು? ಯಾರಿಂದ ಮೊದಲಾಯಿತು? ಆ ಶಕ್ತಿಬೀಜವನ್ನು ಭಾರತವರ್ಷದ ಸುಪ್ತಚೇತನದಲ್ಲಿ ಬಿತ್ತಿದವರಾರು? ಅದಕ್ಕೆ ನೀರು ಹೊಯ್ದವರಾರು? ಮೊಳೆಯಿಸಿದವರಾರು? ಆ ಎಳೆ ಸಸಿಯನ್ನು ಆರೈಸಿದವರಾರು?

ಅದನ್ನರಿಯಬೇಕಾದರೆ ನಾವು ಮತ್ತೊಂದು ಜನವರಿ ಇಪ್ಪತ್ತಾರನೆ ತೇದಿಗೆ ಯಾತ್ರೆ ಹೊರಡಬೇಕಾಗುತ್ತದೆ; ೧೮೯೭ನೇ ಇಸವಿಯ ಜನವರಿ ಇಪ್ಪತ್ತಾರನೆಯ ದಿವ್ಯದಿನಕ್ಕೆ.

ಏನು ವಿಶೇಷ ಆ ದಿನದ್ದು! ಕಲ್ಕತ್ತೆಯಲ್ಲಿ ಕೇಳು; ಲಾಹೋರಿನಲ್ಲಿ ಕೇಳು; ರಾಮನಾಡಿನಲ್ಲಿ ಕೇಳು; ಆಲ್ಮೋರದಲ್ಲಿ ಕೇಳು; ಸರ್ವಧರ್ಮ ಸಮನ್ವಯಮೂರ್ತಿ ನವಯುಗಾವತಾರ ಶ‍್ರೀರಾಮಕೃಷ್ಣ ಪರಮಹಂಸರ ಪರಮಶಿಷ್ಯ ಪರಮಪೂಜ್ಯ ಸ್ವಾಮಿ ವಿವೇಕಾನಂದರು ಚಿಕಾಗೋ ಸರ್ವಧರ್ಮ ಸಮ್ಮೇಳನದಲ್ಲಿ ಜಗತ್ತಿನ ಸಕಲ ಮತಗಳ ಪ್ರತಿನಿಧಿಗಳನ್ನೂ ವೇದಾಂತ ಡಿಂಡಿಮದಿಂದ ಬೆರಗುಗೊಳಿಸಿ, ದಿಗ್ವಿಜಯಿಯಾಗಿ, ಜಗದ್ವಿಖ್ಯಾತರಾಗಿ ಸ್ವದೇಶಕ್ಕೆ ಹಿಮ್ಮರಳಿ ಭಾರತದ ಭೂಸ್ಪರ್ಶ ಮಾಡಿದ ಪುಣ್ಯ ದಿನವಲ್ಲವೆ ಆ ದಿವ್ಯದಿನ! ಬ್ರಿಟಿಷ್​ ಚಕ್ರಾಧಿಪತ್ಯದ ಮತ್ತು ಸಾಮ್ರಾಜ್ಯ ಶಾಹಿಯ ಕಲೊಸಸ್ಸಿನ ಕಾಲಡಿಯಲ್ಲಿ ಕಷ್ಟ ಸಂಕಟಗಳಲ್ಲಿ ನರಳಿ, ಬಿಡುಗಡೆಗಾಗಿ ಹೋರಾಡಿ ಹೊರಳಿ, ದಾರಿಗಾಣದೆ ದಿಕ್ಕು ತೋರದೆ ಎದೆಗೆಟ್ಟು ಕೆರಳಿ, ತಮ್ಮ ದೈನ್ಯವನ್ನೂ, ದಾರಿದ್ರ್ಯವನ್ನೂ ಕ್ಲೈಬ್ಯವನ್ನೂ ಪರಿಹರಿಸಿ ಹೃದಯಕ್ಕೆ ಸಿಂಹಧೈರ್ಯವನ್ನು ತುಂಬುವ ಪಾಂಚಜನ್ಯ ಘೋಷಕ್ಕಾಗಿ ಕಾಯುತ್ತಿದ್ದ ಜಾಗೃತ ಭಾರತದ ಕೋಟ್ಯಂತರ ಚೇತನಗಳಲ್ಲಿ ಅಭೀಃ ಅಭೀಃ ಎಂಬ ಪ್ರಚಂಡವಾದ ವೇದೋಪ ನಿಷತ್ತಿನ ವಾಙ್ಮಂತ್ರದಿಂದ ಶಕ್ತಿ ಸಾಗರದ ದುರ್ದಮ್ಯ ತರಂಗಗಳನ್ನು ಎಬ್ಬಿಸಿ ಹುರಿದುಂಬಿಸಿದ ವೇದಾಂತ ಕೇಸರಿಯ ವಿರಾಟ್​ ಗರ್ಜನೆ ವಿಂಧ್ಯ ಸಹ್ಯ ಹಿಮಾಲಯಗಳಿಂದ ಪ್ರತಿಧ್ವನಿತವಾದ ಪುಣ್ಯದಿನವಲ್ಲವೆ ಆ ದಿವ್ಯದಿನ!

ಅಂದಿನಿಂದ ಮೊದಲಾಯಿತು ಭರತವರ್ಷದ ಪುನರುತ್ಥಾನ. ದಕ್ಷಿಣೇಶ್ವರ ದೇವ\-ಮಾನವನ ಚಿತ್​ ತಪಸ್​ ಸ್ವಾಮಿ ವಿವೇಕಾನಂದರ ಸಮುದ್ರಘೋಷಸ್ಪರ್ಧಿಯಾದ ವೀರವಾಣಿ\-ಯಲ್ಲಿ ವಾಙ್ಮಂತ್ರವಾಗಿ ಹೊಮ್ಮಿ, ಭಾರತೀಯರ ಧಮನಿ ಧಮನಿಗಳಲ್ಲಿ ದುಮುದುಮುಕಿ ಹರಿದು ಕ್ಲೈಬ್ಯವನ್ನು ಕೊಚ್ಚಿತು, ಧೈರ್ಯ ಧ್ವಜ–ವನ್ನೆತ್ತಿತು. ಅಲ್ಪವನ್ನು ಕಿತ್ತು ಭೂಮವನ್ನು ನೆಟ್ಟಿತು. ಸಂಕುಚಿತ ಮನೋಭಾವನೆಯ ಗೋಡೆಗಳನ್ನೊಡೆದು ಮನಸ್ಸನ್ನು ಗಗನ ವಿಶಾಲವನ್ನಾಗಿ ಮಾಡಿತು. ಕುರಿಗಳಂತಿದ್ದವರನ್ನು ಸಿಂಹಗಳನ್ನಾಗಿ ಪರಿವರ್ತಿಸಿತು. “ಉತ್ತಿಷ್ಠತ ಜಾಗ್ರತ ಪ್ರಾಪ್ಯವರಾನ್ನಿ ಬೋಧತ” ಎಂಬ ಧೀರಮಂತ್ರ ದಶದಿಕ್ಕುಗಳಿಂದಲೂ ಶಕ್ತಿ ಸಂಚಾರಕವಾಗಿ ಮೊಳಗಿದರೆ ಅಧೈರ್ಯ ಅಶಕ್ತಿಗಳಿಗೆ ಹುದುಗಿಕೊಳ್ಳುವುದಕ್ಕಾದರೂ ಜಾಗವಿರುತ್ತದೆಯೆ?

ಸ್ವಾತಂತ್ರ್ಯ ಪೂರ್ವದ ಯಾವ ವಿಧವಾದ ರಾಷ್ಟ್ರೀಯ ಕಾರ್ಯಗಳಲ್ಲಿ ಭಾಗ ವಹಿಸಿದ ಯಾರನ್ನಾದರೂ ಕೇಳಿ, ಗೊತ್ತಾಗುತ್ತದೆ. ಪ್ರತಿಯೊಬ್ಬರೂ ಹಿರಿಯ ಕಿರಿಯ ಪ್ರಸಿದ್ಧರೆಂಬ ಭೇದವಿಲ್ಲದೆ ಸ್ವಾಮಿ ವಿವೇಕಾನಂದರಿಗೆ ಋಣಿಗಳಾಗಿರುತ್ತಾರೆ. ಒಬ್ಬೊಬ್ಬರೂ ತಮ್ಮದೇ ಆದ ಒಂದಲ್ಲ ಒಂದು ರೀತಿಯಲ್ಲಿ “ಕೊಲಂಬೊ ಇಂದ ಆಲ್ಮೋರಕೆ” ಎಂಬ ಹೊತ್ತಿಗೆಯಲ್ಲಿ ಕ್ರೋಢೀಕೃತವಾಗಿರುವ ವಿದ್ಯುನ್ಮಯ ಭಾಷಣ ಪರಂಪರೆಯ ಪ್ರಭಾವದಿಂದ ತಂತಮ್ಮ ವ್ಯಕ್ತಿತ್ವವನ್ನು ರೂಪುಗೊಳಿಸಿಕೊಂಡವರಾಗಿರುತ್ತಾರೆ. ದೇಶಭಕ್ತಿಯ ಹೃದಯದಲ್ಲಿ, ರಾಜಕೀಯ ವ್ಯಕ್ತಿಗಳ ಸಾಹಸದ ಪ್ರಾಣಸ್ಥಾನದಲ್ಲಿ, ಸಮಾಜ ಸುಧಾರಕರ ವೀರೋದ್ಯಮದ ಅಂತರಾಳದಲ್ಲಿ, ಆರ್ಥಿಕ ಅಭ್ಯುದಯದ ಆಕಾಂಕ್ಷೆಯ ನಾಡಿಯಲ್ಲಿ, ಧರ್ಮೋದ್ಧಾರದ ಪ್ರಯತ್ನದ ಧಮನಿಯಲ್ಲಿ, ಆಧ್ಯಾತ್ಮಿಕ ಅಭೀಪ್ಸೆಯ ಅಂತರತಮ ನಿಗೂಢ ಗಹ್ವರದಲ್ಲಿ, ಕಡೆಗೆ ಕಲಾ ವಿಜ್ಞಾನ ಸಾಹಿತ್ಯಾದಿ ಕ್ಷೇತ್ರಗಳಲ್ಲಿ, ತಪಸ್ವಿಗಳಾದವರ ಮಹತ್​ ಸಾಧನೆಯ ಮೂಲದಲ್ಲಿ ಎಲ್ಲೆಲ್ಲಿಯೂ ಎಲ್ಲದರಲ್ಲಿಯೂ ಸ್ವಾಮಿ ವಿವೇಕಾನಂದರ ಚಿತ್​ ತಪಸ್ಸು, ‘ಸ್ವಧಾ’ ಶಕ್ತಿಯಾಗಿ ‘ಪ್ರಯತಿಃ’ ಶಕ್ತಿಯಾಗಿ, ನೂಕುವ ಪ್ರೇರಣೆಯಾಗಿ ಸೆಳೆಯುವ ಆಕರ್ಷಣೆಯಾಗಿ ಅಧಿಕಾರ ಮಾಡುತ್ತಿರುವುದನ್ನು ಕಾಣುತ್ತೇವೆ. ಅಂದಿನಿಂದ ಜನವರಿ ಇಪ್ಪತ್ತಾರರಲ್ಲಿ ಪಶ್ಚಿಮ ದೇಶಗಳಿಂದ ಭಾರತವರ್ಷಕ್ಕೆ ವಿಜಯಿಯಾಗಿ ಹಿಂತಿರುಗಿದ ಸ್ವಾಮೀಜಿ ಭಾರತಮಾತೆಯ\break ಯಾವ ಪಾದಸ್ಥಳದಲ್ಲಿ ಮೊತ್ತಮೊದಲು ಪದವಿಟ್ಟರೋ ಅಲ್ಲಿ ಅಂಕುರಾರ್ಪಣೆಗೊಂಡ ದಿವ್ಯ ಸಂಕಲ್ಪವು ನಾನಾ ವೀಚಿಗಳಲ್ಲಿ ವಿಕಾಸಗೊಂಡು, ಮುಂಬರಿದು ಇಂದಿನ ಜನವರಿ ಇಪ್ಪತ್ತಾರರಲ್ಲಿ ತನ್ನ ಫಲಸಿದ್ಧಿಯ ಮಾರ್ಗದಲ್ಲಿ ರಾಜಕೀಯ ಸ್ವಾತಂತ್ರ್ಯರೂಪವಾದ ಒಂದು ಬಹು ಮುಖ್ಯ ಸೋಪಾನವನ್ನೇರಿ ನಿಂತಿದೆ! ರಾಮನಾಡಿನ ದೊರೆ ರಾಜಾ ಭಾಸ್ಕರವರ್ಮ ಸೇತುಪತಿ ಅವರು ಅಂದು ನೆಟ್ಟಿರುವ ಸ್ಮೃತಿಸ್ತಂಭದ ಶಾಸನದಲ್ಲಿ ಮೊದಲಾಗುವ “ಸತ್ಯಮೇವ ಜಯತೇ” ಎಂಬ ವೇದೋಕ್ತಿಯೇ ನಮ್ಮ ರಿಪಬ್ಲಿಕ್ಕಿನ ಅಧಿಕಾರ ಮುದ್ರೆಯನ್ನು ಅಲಂಕರಿಸಿದೆ ಎಂದು ನೆನೆದರೆ ನಮ್ಮ ಭಾವಪೂರ್ಣತೆ ಅರ್ಥಪೂರ್ಣವೂ ಆಗುವುದಿಲ್ಲವೆ?

ರಾಮನಾಡಿನ ಬಿನ್ನವತ್ತಳೆಗೆ ಉತ್ತರರೂಪವಾಗಿ ಅವರು ಮಾಡಿದ ಭಾಷಣದ ಆದಿಯಲ್ಲಿಯೇ ವಿನ್ಯಾಸಗೊಂಡಿರುವ ಭಂಗಿ ಭಣತಿಗಳನ್ನು ಗಮನಿಸಿದರೇ ಸಾಕು ಗೊತ್ತಾಗು\-ತ್ತದೆ, ಅಲ್ಲಿ ಪ್ರವಾದಿಯ ಭವಿಷ್ಯವಾಣಿ ಮಾತ್ರವಲ್ಲದೆ ಶಕ್ತಿಪೂರ್ಣ ಮಂತ್ರದ್ರಷ್ಟಾರನ ವರಾನುಗ್ರಹ ಸಾಮರ್ಥ್ಯವೂ ವ್ಯಕ್ತವಾಗುತ್ತದೆ ಎಂದು. ಆ ವಾಣಿ ಆ ರೀತಿ ಮುಂದಾಗುವುದನ್ನು ಹೇಳುತ್ತದೆ ಮಾತ್ರವಲ್ಲ ಅದನ್ನು ಆಗುವಂತೆಯೂ ಮಾಡುವ ಆಶ್ವಾಸನೆಯನ್ನೂ ನೀಡುತ್ತದೆ. ಆಲಿಸಿ; ಎನಿತೊಮ್ಮೆ ಕೇಳಿದರೂ ಮತ್ತೊಮ್ಮೆ ಕೇಳಬೇಕು ಎನಿಸುತ್ತದೆ:

“ಸುದೀರ್ಘರಾತ್ರಿ ಕಡೆಗಿಂದು ಕೊನೆಗಾಣುತ್ತಿದೆ. ಬಹುಕಾಲದ ಶೋಕತಾಪಗಳು ಕಡೆಗಿಂದು ಮಾಯವಾಗುತ್ತಲಿವೆ. ಇದುವರೆಗೆ ಶವದಂತೆ ಬಿದ್ದಿದ್ದ ಶರೀರವಿಂದು ಸಚೇತನವಾಗುತ್ತಿದೆ. ಅದೋ ಕಿವಿಗೊಡಿ, ತೂರ್ಯವಾಣಿಯೊಂದು ಕೇಳಿಬರುತ್ತಿದೆ– ಬಹು ಪುರಾತನಕಾಲದ ಇತಿಹಾಸ ಗರ್ಭದಿಂದ ಹೊಮ್ಮಿ, ಪರ್ವತ ಶಿಖರಗಳಿಂದ ಮರುದನಿಯಾಗಿ ಚಿಮ್ಮಿ, ಅರಣ್ಯಾರಣ್ಯ ಕಂದರ ಕಂದರಗಳಲ್ಲಿ ಸಂಚರಿಸಿ, ಬರುಬರುತ್ತ ಪ್ರಬಲವಾಗಿ, ಬಂದಂತೆಲ್ಲಾ ಅಪ್ರತಿಹತವಾಗಿ, ನಮ್ಮೀ ಪುಣ್ಯಭೂಮಿಯನ್ನು ನಿದ್ದೆಯಿಂದ ಎಬ್ಬಿಸಿ, ಜ್ಞಾನ ಭಕ್ತಿ ಕರ್ಮ ವೈರಾಗ್ಯ ಸೇವಾ ತತ್ತ್ವಗಳನ್ನು ಉಚ್ಚಕಂಠದಿಂದ ಸಾರುವ ತೂರ್ಯವಾಣಿಯೊಂದು ಕೇಳಿಬರುತ್ತಿದೆ. ಹಿಮಾಲಯಗಳಿಂದ ಬೀಸುವ ಪುಣ್ಯ ಸಮೀರಣನಂತೆ ನಿರ್ಜೀವವಾದಂತಿದ್ದ ಅಸ್ಥಿ ಮಾಂಸಗಳಿಗೆ ಜೀವದಾನ ಮಾಡುತ್ತಿದೆ. ಜಡ ನಿದ್ದೆಯನ್ನು ಪರಿಹರಿಸುತ್ತಿದೆ; ಕಾರ್ಯೋತ್ಸಾಹ ಸ್ಥೈರ್ಯ ಧೈರ್ಯಗಳನ್ನು ಉದ್ರೇಕಿಸುತ್ತಿದೆ. ಕುರುಡರಿಗೆ ಕಾಣದು; ಮೂರ್ಖರಿಗೆ ತಿಳಿಯದು. ನಮ್ಮೀ ಭಾರತಭೂಮಿ ಯುಗಯುಗಗಳ ನಿದ್ರೆಯಿಂದ ಮೇಲೇಳುತ್ತಿದೆ. ಆಕೆಯನ್ನು ಇನ್ಯಾರೂ ತಡೆಯಬಲ್ಲವರಿಲ್ಲ; ಇನ್ನಾಕೆ ನಿದ್ದೆ ಮಾಡುವುದಿಲ್ಲ. ಯಾವ ಶಕ್ತಿಯೂ ಆಕೆಯನ್ನು ಬಗ್ಗಿಸಲಾರದು. ಏಕೆಂದರೆ, ಅದೋ ನೋಡಿ! ಮಹಾಕಾಳಿ ಮತ್ತೊಮ್ಮೆ ಎಚ್ಚೆತ್ತು ಮೈಕೊಡವಿ ಉಸಿರೆಳೆದು ನಿಲ್ಲುತ್ತಿದ್ದಾಳೆ....”

ಆಲಿಸಿದಿರಲ್ಲವೆ, ಓದಿ ಮನನಮಾಡಿ. ಪಂಕ್ತಿ ಪಂಕ್ತಿಯನ್ನೂ ಮನನ ಮಾಡಿ, ಅದರಲ್ಲಿ ಯಾವುದಾದರೂ ಅಂಶ ಹುಸಿಯಾಗಿದೆಯೇ? ಅಥವಾ ಹುಸಿಯಾಗುವಂತೆ ತೋರುತ್ತದೆಯೇ? ಇಲ್ಲ. ಹುಸಿಯಾಗಿಲ್ಲ. ಹುಸಿಯಾಗುವುದೂ ಇಲ್ಲ!

ರಾಜಕೀಯ ಸ್ವಾತಂತ್ರ್ಯವೇನೋ ಸಿದ್ಧಿಸಿದೆ. ಅದನ್ನು ಒಂದು ಬಹು ಮುಖ್ಯವಾದ ಸೋಪಾನ ಎಂದು ಮಾತ್ರ ಕರೆದಿದ್ದೇವೆ. ಆದರೆ ಸ್ವಾಮೀಜಿ ನಮಗೆ ಮೆಟ್ಟಿಲಿನ ಮೇಲೆಯೇ ಮನೆ ಕಟ್ಟಲು ಹೇಳಿಲ್ಲ. ಅವರು ಮೊಟ್ಟ ಮೊದಲನೆಯ ಕೊಲಂಬೊ ಭಾಷಣದಲ್ಲಿಯೆ ಈ ವಿಚಾರದಲ್ಲಿ ಸ್ಪಷ್ಟವಾಗಿ ಎಚ್ಚರಿಕೆ ಕೊಟ್ಟಿದ್ದಾರೆ. ಪ್ರಾಚೀನ ಭರತಖಂಡದ ಆಧ್ಯಾತ್ಮಿಕ ತೇಜಸ್ಸು ಭೂಮಂಡಲವನ್ನೆಲ್ಲಾ ಹೇಗೆ ವ್ಯಾಪಿಸಿತ್ತು ಎಂಬ ಪೂರ್ವ ವೈಭವವನ್ನು ನೆನಪಿಗೆ ತಂದುಕೊಡುತ್ತಾ ಹೀಗೆ ಹೇಳುತ್ತಾರೆ:

“ಇಂದು ಜಡನಾಗರಿಕತೆಯ ಪ್ರಪಂಚಕ್ಕೆ ಅಧ್ಯಾತ್ಮವನ್ನು ಧಾರೆ ಎರೆಯುವ ಮಹಾ ಪ್ರವಾಹವೂ ಇಲ್ಲಿಂದ ಉದಿಸಬೇಕಾಗಿದೆ. ಇಲ್ಲಿದೆ ಬಾಳಿಗೆ ಹೊಸಬೆಳಕನ್ನು ಕೊಡುವ ಅಮೃತ ಪ್ರವಾಹ....”

“ಪ್ರಪಂಚದ ಇತರ ರಾಷ್ಟ್ರಗಳಿಗೆ ಧರ್ಮವೆಂಬುದು ಜೀವನದ ಹಲವು ಕಸುಬುಗಳಲ್ಲಿ ಒಂದು. ರಾಜಕೀಯವಿದೆ, ಸಮಾಜದ ಸುಖಭೋಗಗಳಿವೆ, ಐಶ್ವರ್ಯ ಮತ್ತು ಅಧಿಕಾರದಿಂದ ಗಳಿಸುವುದು ಇನ್ನೆಷ್ಟೋ ಇವೆ. ಇಂದ್ರಿಯ ಸುಖಕ್ಕೆ ಬೇಕಾದಷ್ಟು ವಿಷಯವಸ್ತುಗಳಿವೆ. ಜೀವನದ ಇಂತಹ ಹಲವು ವ್ಯವಹಾರಗಳ ನಡುವೆ ಇಂದ್ರಿಯಗಳ ಸುಖಸಂತೃಪ್ತಿಗಾಗಿ ಹಲವಾರು ವಸ್ತುಗಳನ್ನು ಅರಸುವಾಗ ಮಧ್ಯೆ ಸ್ವಲ್ಪ ಧರ್ಮವೂ ಇದೆ. ಆದರೆ ಇಲ್ಲಿ, ಭರತಖಂಡದಲ್ಲಿ, ಧರ್ಮವೇ ಪರಮ ಪುರುಷಾರ್ಥರೂಪವಾದ ಏಕಮಾತ್ರ ವ್ಯವಹಾರ....”

“ಜಗತ್ತಿನಲ್ಲಿ ಪ್ರತಿಯೊಂದು ಜನಾಂಗವೂ ತನ್ನ ಪಾಲಿಗೆ ಬಂದ ವಿಶೇಷ ಕರ್ತವ್ಯವನ್ನು ಸಾಧಿಸಬೇಕಾಗಿದೆ..... ರಾಜಕೀಯ ಮಹತ್ವವಾಗಲಿ, ಸೇನಾ ಶಕ್ತಿಯಾಗಲಿ ನಮ್ಮ ಜನಾಂಗದ ಗುರಿಯಲ್ಲ. ಅದು ಎಂದೂ ಹಿಂದೆ ಆಗಿರಲಿಲ್ಲ, ಹೇಳುತ್ತೇವೆ ಗಮನಿಸಿ, ಅದೆಂದೂ ಮುಂದೆಯೂ ಆಗಲೂ ಆರದು.... ಆಧ್ಯಾತ್ಮಿಕ ಜ್ಞಾನವೇ ಪ್ರಪಂಚಕ್ಕೆ ಭರತಖಂಡ ನೀಡುವ ಬಹುಮಾನ.”

“ಪ್ರತಿಯೊಂದು ಜನಾಂಗಕ್ಕೂ ಪ್ರತಿಯೊಬ್ಬ ವ್ಯಕ್ತಿಗಿರುವಂತೆಯೇ ಒಂದೊಂದು\break ಜೀವನೋದ್ದೇಶವಿದೆ. ಆ ಉದ್ದೇಶವೇ ಅದರ ಹೃದಯ. ಉಳಿದುವೆಲ್ಲವೂ ಗೌಣ. ಯಾವ ಜನಾಂಗವಾದರೂ ಶತಮಾನಗಳಿಂದ ತನ್ನ ನಾಡಿನಲ್ಲಿ ಪ್ರವಹಿಸಿ ಬಂದ ಆದರ್ಶವನ್ನು ಕಿತ್ತೊಗೆಯಿತೆಂದರೆ ಸರ್ವನಾಶವಾಗುತ್ತದೆ. ರಾಜಕೀಯ ಒಂದರ ಹೃದಯ; ಕಲಾಜೀವನ ಮತ್ತೊಂದರದು. ಭರತಖಂಡದ ಹೃದಯ–ವೆಂದರೆ ಧರ್ಮ. ಅದನ್ನು ವರ್ಜಿಸಿದರೆ ನಮ್ಮ ಸಂಸ್ಕೃತಿ ಅಳಿದುಹೋಗುತ್ತದೆ.”

ಸ್ವಾತಂತ್ರ್ಯಪೂರ್ವದ ಮಹೋದ್ದೇಶಗಳನ್ನೂ ಮಹಾಪ್ರತಿಜ್ಞೆಗಳನ್ನೂ ಮಹಾಧ್ಯೇಯಗಳನ್ನೂ ಮರೆತೋ ತಿರಸ್ಕರಿಸಿಯೋ, ರಾಜಕೀಯವೇ ಸರ್ವಸ್ವ ಎಂಬಂತೆ ವರ್ತಿಸುವ ಮನೋಭಾವ ಹೆದರಿಕೆ ಹುಟ್ಟಿಸುವಷ್ಟರ ಮಟ್ಟಿಗೆ ಪ್ರಬಲವಾಗಿರುವ ಈ ಸಂಧಿಸಮಯದಲ್ಲಿ ಸ್ವಾಮೀಜಿಯವರ– “ಕೊಲಂಬೊ ಇಂದ ಆಲ್ಮೋರಕೆ” ಎಂಬ ಈ ಹೊತ್ತಗೆ ನಮ್ಮ ತರುಣರಿಗೆ, ಮುಂದೆ ಮುಂದಾಳಾಗುವ ಇಂದಿನ ತರುಣರಿಗೆ ಮಾರ್ಗದರ್ಶಕವಾಗುವುದರಲ್ಲಿ ಸಂದೇಹವಿಲ್ಲ. ಈ ಮಾರ್ಗ ದರ್ಶನವನ್ನು ನಾವು ತಿರಸ್ಕರಿಸಿದರೆ ದೇಶಕ್ಕೆ ಉಳಿಗಾಲವಿಲ್ಲ.

ಸ್ವಾತಂತ್ರ್ಯಪೂರ್ವದಲ್ಲಿ ರಾಜಕೀಯವೂ ತ್ಯಾಗಶೀಲವಾಗಿತ್ತು. ಸ್ವಾತಂತ್ರ್ಯೋತ್ತರ–ದಲ್ಲಿ ಅದು ಒಂದು ರೀತಿಯ ಭೋಗಾಭಿಲಾಷೆಯಾಗಿ ಪರಿಣಮಿಸುತ್ತಿದೆ. ಅಂದು ರಾಜಕೀಯಕ್ಕೆ\break ಪ್ರವೇಶಿಸುವುದೆಂದರೆ, ಮುಖ್ಯವಾಗಿ, ಸ್ವಾತಂತ್ರ್ಯ ಸಂಗ್ರಾಮದಲ್ಲಿ ಗಾಂಧೀಜಿ ಅಂತಹವರ ನೇತೃತ್ವದಲ್ಲಿ ತಿತಿಕ್ಷಾಜೀವನ ನಡೆಸುವುದಾಗಿತ್ತು. ಕಷ್ಟ, ಸಂಕಟ, ಮನಃಕ್ಲೇಶ, ತ್ಯಾಗ ಕಾರಾಗೃಹವಾಸ, ದೈಹಿಕಶ್ರಮ ಇತ್ಯಾದಿ ರೂಪವಾದ ಸಾಧನಾರಂಗದಲ್ಲಿ ವ್ಯಕ್ತಿ ಸಾಧಕನಾಗಬೇಕಾಗಿತ್ತು. ಇಂದು ರಾಜಕೀಯಕ್ಕೆ ಪ್ರವೇಶಿಸುವುದೆಂದರೆ ಸ್ವಾರ್ಥತೆ, ಸ್ವಪ್ರತಿಷ್ಠೆ, ಸ್ವಜಾತಿ ಸ್ವಪಕ್ಷಗಳ ಉದ್ಧಾರದ ನೆವದಲ್ಲಿ ಪಕ್ಷಪ್ರತಿಪಕ್ಷಗಳನ್ನು ಕಟ್ಟಿ ಎದುರಾಳಿಯನ್ನೂ, ಎದುರು ಪಕ್ಷವನ್ನೂ ಸದೆಬಡಿಸುವುದು, ಏನಕೇನ ಪ್ರಕಾರೇಣ ‘ಅಖಿಲಭಾರತ ಧುರೀಣ’ ನಾಗು\-ವುದು, ಅಧಿಕಾರ ಧನ ಮಾನ ಸಂಪಾದನೆ ಇತ್ಯಾದಿ, ಇತ್ಯಾದಿ.

ನೇರವಾಗಿ ರಾಜಕೀಯಕ್ಕೆ ಇಳಿಯುವವರ ಗತಿ ಅದಾದರೆ ಉಳಿದವರದು ಅದಕ್ಕಿಂತಲೂ ಮೇಲಾಗಿ ತೋರುತ್ತಿಲ್ಲ. ಹಳ್ಳಿಗ ಪಟ್ಟಣಿಗ, ಸಾಕ್ಷರ ನಿರಕ್ಷರ, ದರಿದ್ರ ಶ‍್ರೀಮಂತ, ಬಡವ ಬಂಡವಾಳಗಾರ, ಆಳು ಒಡೆಯ, ಅವರು ಇವರು ಎಂಬ ಭೇದವಿಲ್ಲದೆ ಎಲ್ಲರನ್ನೂ ಆಕ್ರಮಿಸಿದೆ ಈ ರಾಜಕೀಯದ ಜ್ವರ. ದಿನ ಬೆಳಗಾದರೆ ಪತ್ರಿಕೆಗಳು ಈ ಜ್ವರವನ್ನು ಮನೆ ಮನೆಗೂ ಮನ ಮನಕ್ಕೂಹಂಚುವ ಶೀಘ್ರ ಸಾಧನಗಳಾಗಿವೆ. ಆ ಪತ್ರಿಕೆಗಳ ಬಹುಭಾಗವನ್ನೆಲ್ಲಾ ವ್ಯಾಪಿಸಿರುತ್ತದೆ ಈ ರಾಜಕೀಯದ ಸುದ್ದಿ ಮತ್ತು ಬೇಗೆ. ಜೊತೆಗೆ ಚುನಾವಣೆ ಬೇರೆ ಈ ರಾಜಕೀಯ ಜ್ವರ ಒಂದರೆನಿಮಿಷವೂ ಯಾರ ಮಿದುಳನ್ನೂ ಬಿಟ್ಟು ಹೋಗದಂತೆ ನೋಡಿಕೊಳ್ಳುತ್ತಿದೆ. ಮಹಾಚುನಾವಣೆ! ಅದಾಯಿತು ಉಪಚುನಾವಣೆ! ಅಷ್ಟರಲ್ಲಿ ಉಪ ಚುನಾವಣೆಗಳ ಕೇಸುಗಳು, ಅವಿನ್ನೂ ಮುಗಿಯುವುದರೊಳಗೆ ಜಿಲ್ಲಾ ಚುನಾವಣೆ ಪ್ರಾರಂಭ. ಅವುಗಳಿಗೆ ಪುಚ್ಛಭೂತವಾಗಿ ಪೌರಸಭೆಗೆ ಗ್ರಾಮ ಪಂಚಾಯಿತಿಗೆ ಮಣ್ಣಿಗೊ ಮಸಣಕ್ಕೊ ಮತ್ತೆ ಚುನಾವಣೆ! ಸರಿ ಇದೆಲ್ಲಾ ಪೂರೈಸುವುದರೊಳಗೆ ಮತ್ತೆ ಮಹಾಚುನಾವಣೆ ಸಮೀಪಿಸುತ್ತದೆ. ಸರಿ, ಮತ್ತೆ ಅದಕ್ಕೆ ಪೂರ್ವಸಿದ್ಧತೆ ಬೇಡವೆ? ವ್ಯೂಹರಚನೆ ಬೇಡವೆ ಯುದ್ಧಕ್ಕೆ ಮುನ್ನ? ಹೀಗೆ ಒಂದಲ್ಲ ಹಲವು ತರಗಳಲ್ಲಿ ಕಲೆ ಸಂಸ್ಕೃತಿ ಧರ್ಮ ಆಧ್ಯಾತ್ಮ ಇವುಗಳ ಕಡೆಯಿಂದ ರಾಜಕೀಯದ ಕಡೆಗೆ ತಿರುಗುವ ಜನಮನಸ್ಸು ಅದನ್ನೇ ಬಹುಮುಖ್ಯವೆಂದೂ ಪರಮಪುರು ಷಾರ್ಥವೆಂದೂ ಭಾವಿಸುತ್ತದೆ. ಪಾಶ್ಚಾತ್ಯ ದೇಶಗಳಲ್ಲಿ ಆಗಿರುವಂತೆ ರಾಜಕೀಯ ವ್ಯಕ್ತಿಯೇ ಪರಮಗಣ್ಯನೂ ಪೂಜ್ಯನೂ ಆಗುತ್ತಾನೆ. ರಾಜಕೀಯವಲ್ಲದ ಜೀವನ ವ್ಯಾಪಾರಕ್ಕೆ ಪುರುಸೊತ್ತೆಲ್ಲಿ ಎಂಬ ಆತ್ಮವಂಚನೆಯ ಕಾರಣ ಸಿದ್ಧವಾಗುತ್ತದೆ. ಕೊನೆಗೆ ಯಾವ ವಿಶೇಷಲಕ್ಷಣದಿಂದ ನಮ್ಮ ಭಾರತೀಯತೆ ಲೋಕಗೌರವಕ್ಕೆ ಭಾಜನವಾಗಿತ್ತೊ ಆ ಲಕ್ಷಣವೇ ನಮ್ಮ ಜೀವನದಿಂದ ದೂರವಾಗುವ ದುರ್ಗತಿ ನಮಗೊದಗುತ್ತದೆ. ಆತ್ಮಶ‍್ರೀ ಶೂನ್ಯರಾಗುತ್ತೇವೆ, ಸ್ವಾಮೀಜಿಯವರ ಎಚ್ಚರಿಕೆಯನ್ನು ನಾವು ಗಮನಿಸದಿದ್ದರೆ. ಆದ್ದರಿಂದಲೇ ಮುಂದೆ ಮುಂದಾಳಾಗುವನಮ್ಮ ಇಂದಿನ ತರುಣರಿಗೆ “ಕೊಲಂಬೊ ಇಂದ ಆಲ್ಮೋರಕೆ” ಒಂದು ಕೈದೀವಿಗೆಯಾಗುವು ದರಲ್ಲಿ ಸಂದೇಹವಿಲ್ಲ.

ನನ್ನ ತಾರುಣ್ಯ ಸ್ವಾಮಿ ವಿವೇಕಾನಂದರಿಗೆ ಎಷ್ಟು ಋಣಿಯಾಗಿತ್ತೆಂದು ನಾನು ಹೇಳಿ ಪೂರೈಸಲಾರೆ. ಅದನ್ನು ನೆನೆದರೇ ಕೃತಜ್ಞತೆಯ ಹರ್ಷಾಶ್ರು ಹನಿಯುತ್ತದೆ. ಹಿಂದೂಧರ್ಮದ ವೈಶಾಲ್ಯದಮತ್ತು ವೇದಾಂತ ತತ್ತ್ವದ ಗಂಭೀರತೆಯಪರಿಚಯ\break ಸರ್ವಸಾಮಾನ್ಯವೂ ಸುಲಭಗ್ರಾಹ್ಯವೂ ಆಗಬೇಕಾದರೆ ಸ್ವಾಮೀಜಿಯವರ ಉಪನ್ಯಾಸಗಳನ್ನು ಬಿಟ್ಟರೆ ಬೇರಾವುದೂ ಇಲ್ಲ. ಸಂಕುಚಿತ ಮತಭಾವಗಳನ್ನು ಕತ್ತರಿಸಿ ಕೆಡಹಿ, ಮನಸ್ಸಿನಲ್ಲಿ ಉದಾತ್ತ ಸಮನ್ವಯ ದೃಷ್ಟಿಯನ್ನು ಪ್ರಜ್ವಲಿಸುವ ಶಕ್ತಿ ಈ ಉಪನ್ಯಾಸಗಳಲ್ಲಿ ಇರುವಂತೆ ಬೇರೆಲ್ಲಿಯೂ ಇಲ್ಲ. ಪತನ ಸಮಯದಲ್ಲಿ ನಮ್ಮನ್ನು ಕೈಹಿಡಿದೆತ್ತುವ ಔದಾರ್ಯ, ಹೃದಯ ದೌರ್ಬಲ್ಯದ ಸಮಯದಲ್ಲಿ ಕ್ಲೈಬ್ಯವನ್ನು ಕಿತ್ತೊಗೆದು ಕೆಚ್ಚನ್ನು ನೆಡುವ ಸಿಡಿಲಾಳ್ಮೆ ಈ ಭಾಷಣಗಳಲ್ಲಿ ಅನುಭವ ಪ್ರತ್ಯಕ್ಷವಾಗುವಂತೆ ಬೇರೆಲ್ಲಿಯೂ ಆಗುವುದಿಲ್ಲ. ಇಲ್ಲಿ ಬುದ್ಧಿಗೆ ಪುಷ್ಟಿ ಇದೆ; ಹೃದಯಕ್ಕೆ ತುಷ್ಟಿ ಇದೆ. ನಮ್ಮ ವ್ಯಕ್ತಿತ್ವ ಸಮಸ್ತವನ್ನೂ ಸರ್ವಾವಯವ ಸಂಪೂರ್ಣವಾಗಿ ವಿಕಾಸಗೊಳಿಸಿ ಪೂರ್ಣತೆಯ ಕಡೆಗೆ ನಮ್ಮನ್ನು ಕೊಂಡೊಯ್ಯುವ ಪೂರ್ಣ ದೃಷ್ಟಿಯೂ ಇಲ್ಲಿ ಸಿದ್ಧಿಸುತ್ತದೆ. ಇದು ಅಮೃತದ ಮಡು; ಮಿಂದು ಧನ್ಯರಾಗಿ! ಇದು ಜ್ಯೋತಿಯ ಖನಿ; ಹೊಕ್ಕು ಪ್ರಬುದ್ಧರಾಗಿ.

ಹಿಂದೆ ಸಂಸ್ಕೃತ ಹಿಮಾಲಯದ ಗೀರ್ವಾಣ–ಶಿಖರಗಳ ಔನ್ಯತ್ಯದಲ್ಲಿಹೆಪ್ಪು–ಗಟ್ಟಿದ್ದ ದಿವ್ಯಜ್ಞಾನ ರಸಾನಂದವಾಹಿನಿಗಳು ದೇಶಭಾಷಾನದೀಪಾತ್ರಗಳಲ್ಲಿ ಪ್ರವಹಿಸಿದ ಮೇಲೆಯೇ ನಮ್ಮ ಜನತೆಯ ಮನಃಕ್ಷೇತ್ರ ಆರ್ದ್ರವಾಗಿ ಫಲವತ್ತಾಗಿ ನಾಡಿನ ಮೇಲ್ಮೆ ಸಿದ್ಧಿಸಿತು. ಹಾಗೆಯೇ ಇಂಗ್ಲಿಷ್​ ಭಾಷೆಯಲ್ಲಿ ಭದ್ರವಾಗಿದ್ದು, ಅದನ್ನರಿತ ಎಲ್ಲಿಯೋ ಕೆಲವೇ ವ್ಯಕ್ತಿಗಳಿಗೆ ಲಭ್ಯವಾಗಿದ್ದರೂ ಅದರ ಪರಿಣಾಮ ಒಟ್ಟು ದೇಶದ ಮೇಲೆ ಮಹತ್ತಾಗಿತ್ತೆಂಬುದನ್ನು ಅರಿತಿದ್ದೇವೆ. ಇದುವರೆಗೆ ಇಂಗ್ಲಿಷ್​ ತಿಳಿದ ಒಬ್ಬನಿಗೆ ತಿಳಿಯುತ್ತಿದ್ದುದು ಇನ್ನುಮೇಲೆ ಕನ್ನಡ ತಿಳಿದ ನೂರು ಜನಕ್ಕೂ ಲಭಿಸುತ್ತದೆ ಎಂದ ಮೇಲೆ ಅದರ ಪರಿಣಾಮ ಮತ್ತೆಷ್ಟು ಮಹತ್ತರವಾಗುತ್ತದೆ ಎಂಬುದನ್ನು ನಾವು ಊಹಿಸಿಯೇ ಹಿಗ್ಗಬೇಕು. ಆ ಹಿಗ್ಗಿಗಾಗಿ ಕನ್ನಡ ಜನತೆ ಶ‍್ರೀ ಸ್ವಾಮಿ ಸೋಮನಾಥಾ ನಂದರಿಗೆ ಹೃತ್ಪೂರ್ವಕವಾಗಿ ಕೃತಜ್ಞವಾಗುತ್ತದೆ.

ಸ್ವಾಮಿ ವಿವೇಕಾನಂದರು ಒಂದೆಡೆ ಕೂಗಿ ಕರೆದು ಹೇಳುತ್ತಾರೆ, ಆ ಶಕ್ತಿಪೂರ್ಣವಾದ ದಿವ್ಯವಾಣಿಯಲ್ಲಿಯೇ ನಾವು ಏಕಕಂಠರಾಗಿ ನಿಮ್ಮನ್ನು ಕೂಗುತ್ತೇವೆ, ತರುಣರಿರಾ, ಓಗೊಳ್ಳಿ:

“ಓ ಮಾನವ, ಅತೀತದ ಪೂಜೆಯಿಂದ ಪ್ರತ್ಯಕ್ಷದ ಪೂಜೆಗೆ ನಿನಗಿದೋ ಆಹ್ವಾನ!\\ಹೋದ ದುಃಖದಿಂದ ಬರುವ ಸುಖಕ್ಕೆ ನಿನಗಿದೋ ಆಹ್ವಾನ!\\ಗತಾನುಶೋಚನೆಯಿಂದ ಆಧುನಿಕ ನವಪ್ರಯತ್ನಕ್ಕೆ ನಿನಗಿದೋ ಆಹ್ವಾನ! ಲುಪ್ತ ಪಂಥತ ಪುನರುದ್ಧಾರ ಸಾಹಸದ ವೃಥಾ ಶಕ್ತಿಕ್ಷಯದಿಂದಸದ್ಯೋನಿರ್ಮಿತ ವಿಶಾಲ ಸನ್ನಿಕಟ ಪಥಕ್ಕೆ ನಿನಗಿದೋ ಆಹ್ವಾನ!”

\bigskip

\noindent
ಮೈಸೂರು\hfill \textbf{ಕುವೆಂಪು}

\noindent
೬$–$೨$–$೧೯೫೩

