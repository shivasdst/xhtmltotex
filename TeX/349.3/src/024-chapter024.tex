
\chapter{ವೇದಾಂತ ತತ್ತ್ವ}

ಸ್ವಾಮಿ ವಿವೇಕಾನಂದರು ತಮ್ಮ ಶಿಷ್ಯರೊಡನೆ ಖೇತ್ರಿಯಲ್ಲಿ ಮಹಾರಾಜರ ಬಂಗಲೆಯಲ್ಲಿ ತಂಗಿದ್ದರು. ೧೮೯೭ರ ಡಿಸೆಂಬರ್​ ೨೦ರಂದು ಅಲ್ಲಿಯ ಸಭಾಂಗಣದಲ್ಲಿ ಅವರು ವೇದಾಂತತತ್ತ್ವದ ಮೇಲೆ ಒಂದು ಉಪನ್ಯಾಸವನ್ನು ನೀಡಿದರು. ಸಭಾಧ್ಯಕ್ಷರಾಗಿದ್ದ ಮಹಾರಾಜರು ಸ್ವಾಮೀಜಿಯವರನ್ನು ಪರಿಚಯಿಸಿದರು. ಸ್ವಾಮೀಜಿಯವರು ಒಂದುವರೆ ಗಂಟೆಗೂ ಹೆಚ್ಚು ಹೊತ್ತು ಅತ್ಯುತ್ಸಾಹದಿಂದ ಮಾತನಾಡಿದರು. ಆದರೆ ಇಂತಹ ಆಸಕ್ತಿಪೂರ್ಣ ಉಪನ್ಯಾಸವನ್ನು ಶೀಘ್ರ ಲಿಪಿಯಲ್ಲಿ ಬರೆದುಕೊಳ್ಳಲು ಯಾರೂ ಇಲ್ಲದೆ ಇದ್ದುದು ದುರದೃಷ್ಟಕರ. ಆ ಉಪನ್ಯಾಸದ ಸಾರಾಂಶವನ್ನು ಮಾತ್ರ ಕೆಳಗೆ ಕೊಟ್ಟಿದೆ:

\vskip 5pt

ಪುರಾತನಕಾಲದ ಗ್ರೀಕರು ಮತ್ತು ಆರ್ಯರು ಬೇರೆ ಬೇರೆ ವಾತಾವರಣ ಮತ್ತು ಪರಿಸರದಲ್ಲಿ ಬೆಳೆದರು. ಪ್ರಕೃತಿಯಲ್ಲಿ ಸುಂದರವಾದ, ಮಧುರವಾದ, ಪ್ರಲೋಭನಕಾರಿಯಾದ ಎಲ್ಲ ಅಂಶಗಳಿಂದಲೂ ಗ್ರೀಕರು ಸುತ್ತುವರಿದಿದ್ದರು. ಅಲ್ಲಿಯ ಹವಾಗುಣ ತುಂಬ ಉತ್ಸಾಹದಾಯಕವಾಗಿತ್ತು. ಆರ್ಯರು ಭವ್ಯವಾದ ಪ್ರಾಕೃತಿಕ ದೃಶ್ಯಗಳಿಂದ ಸುತ್ತುವರಿದಿದ್ದರು. ಅಲ್ಲಿಯ ಹವಾಗುಣ ದೈಹಿಕ ಪರಿಶ್ರಮಕ್ಕೆ ಹೆಚ್ಚು ಅನುಕೂಲವಾಗಿರಲಿಲ್ಲ. ಈ ಕಾರಣಗಳಿಂದಾಗಿ ಈ ಎರಡು ಜನಾಂಗಗಳು ಬೇರೆ ಬೇರೆ ವಿಧದ ನಾಗರಿಕತೆ\break ಗಳನ್ನು ಸೃಷ್ಟಿಸಿದರು. ಗ್ರೀಕರು ಬಾಹ್ಯ ಅನಂತವನ್ನೂ ಆರ್ಯರು ಆಂತರಿಕ ಅನಂತವನ್ನೂ ತಮ್ಮ ಅಧ್ಯಯನದ ವಿಷಯವಾಗಿ ಆರಿಸಿಕೊಂಡರು. ಒಬ್ಬರು ಬ್ರಹ್ಮಾಂಡವನ್ನು ತಿಳಿಯಲು ಪ್ರಯತ್ನಿಸಿದರು. ಇನ್ನೊಬ್ಬರು ಪಿಂಡಾಂಡವನ್ನು ತಿಳಿಯಲು ಪ್ರಯತ್ನಿಸಿದರು. ಪ್ರಪಂಚದ ನಾಗರಿಕತೆಯ ವಿಕಾಸದಲ್ಲಿ ಈ ಎರಡು ಜನಾಂಗಗಳು ತಮ್ಮದೇ ವಿಶಿಷ್ಟ ಪಾತ್ರಗಳನ್ನು ವಹಿಸುತ್ತವೆ. ಒಬ್ಬರು ಮತ್ತೊಬ್ಬರಿಂದ ಎರವಲಾಗಿ ಪಡೆಯಬೇಕಾಗಿಲ್ಲ, ಆದರೆ ಪರಸ್ಪರ ಭಾವವಿನಿಮಯದಿಂದ ಇಬ್ಬರಿಗೂ ಲಾಭವಾಗುತ್ತದೆ. ಆರ್ಯರು ಸ್ವಾಭಾವಿಕವಾಗಿ ವಿಶ್ಲೇಷಣಾತ್ಮಕ ಮನೋಭಾವದವರು. ಅವರು ಗಣಿತಶಾಸ್ತ್ರ ಮತ್ತು ವ್ಯಾಕರಣ ಶಾಸ್ತ್ರಗಳಲ್ಲಿ ಅದ್ಭುತ ಸಾಧನೆ ಮಾಡಿರುವರು, ಮತ್ತು ಮನಸ್ಸಿನ ವಿಶ್ಲೇಷಣೆಯ ಮೂಲಕ ಜ್ಞಾನದ ಪರಿಪೂರ್ಣತೆಯನ್ನು ಮುಟ್ಟಿರುತ್ತಾರೆ. ಪೈಥಾಗೊರಸ್​, ಸಾಕ್ರಟೀಸ್​, ಪ್ಲೇಟೋ ಮತ್ತು ಈಜಿಪ್ಟಿನ ನಿಯೋಪ್ಲೇಯೊನಿಸ್ಟರು ಇವರಲ್ಲಿ ಭಾರತೀಯ ಚಿಂತನೆಗಳ ಅಂಶಗಳನ್ನು ನಾವು ಕಾಣಬಹುದು.

\vskip 5pt

ಅನಂತರ ಸ್ವಾಮೀಜಿಯವರು ಯೂರೋಪಿನ ಮೇಲೆ ಭಾರತೀಯ ಚಿಂತನೆಗಳ ಪ್ರಭಾವವನ್ನು ಕುರಿತು ಸುದೀರ್ಘ ವಿವರಣೆ ನೀಡಿದರು. ಬೇರೆ ಬೇರೆ ಕಾಲಗಳಲ್ಲಿ ಸ್ಪೆಯಿನ್​, ಜರ್ಮನಿ ಮತ್ತು ಬೇರೆ ಐರೋಪ್ಯ ರಾಷ್ಟ್ರಗಳು ಹೇಗೆ ಭಾರತೀಯ ಪ್ರಭಾವಕ್ಕೆ ಒಳಗಾಗಿದ್ದವು\break ಎಂಬುದನ್ನು ತೋರಿಸಿದರು. ಭಾರತೀಯ ರಾಜಕುಮಾರ ದಾರಾಶುಕೂ ಉಪನಿಷತ್ತುಗಳನ್ನು ಪಾರ್ಸಿ ಭಾಷೆಗೆ ಅನುವಾದಿಸಿದನು. ಇದರ ಲ್ಯಾಟಿನ್​ ಅನುವಾದವು ಯೂರೋಪಿನ ಪ್ರಸಿದ್ಧ ತತ್ತ್ವಶಾಸ್ತ್ರಜ್ಞನಾದ ಶೋಪೆನ್ನೇರ್​ಗೆ ದೊರಕಿತು. ಅವನ ತತ್ತ್ವಶಾಸ್ತ್ರದ ಮೇಲೆ ಉಪನಿಷತ್ತಿನ ಪ್ರಭಾವ ಬಹಳವಾಗಿರುವುದು ಕಂಡುಬರುತ್ತದೆ. ಮುಂದೆ ಕ್ಯಾಂಟನ ತತ್ತ್ವಶಾಸ್ತ್ರದಲ್ಲಿಯೂ ಉಪನಿಷತ್ತಿನ ಬೋಧನೆಯ ಕುರುಹುಗಳು ಕಂಡುಬರುತ್ತವೆ. ಯೂರೋಪಿನಲ್ಲಿ ತುಲನಾತ್ಮಕ ಭಾಷಾಶಾಸ್ತ್ರದಲ್ಲಿರುವ ಆಸಕ್ತಿಯಿಂದಾಗಿ ವಿದ್ವಾಂಸರು ಸಂಸ್ಕೃತದ ಕಡೆಗೆ ಆಕರ್ಷಿತರಾಗುತ್ತಾರೆ. ಆದರೆ ಡಾಯ್ಸನ್​ರಂತಹ ಕೆಲವರು ತತ್ತ್ವಶಾಸ್ತ್ರದಲ್ಲಿ ಆಸಕ್ತಿಯುಳ್ಳವರಾಗಿದ್ದಾರೆ. ಮುಂದೆ ಸಂಸ್ಕೃತ ಅಧ್ಯಯನಕ್ಕೆ ಹೆಚ್ಚು ಗಮನ ಕೊಡಬಹುದೆಂದು ಸ್ವಾಮೀಜಿ ಆಶಿಸಿದರು. ಅನಂತರ ಅವರು ‘ಹಿಂದೂ’ ಎಂಬ ಪದವು ಹಿಂದೆ ಹೆಚ್ಚು ಅರ್ಥಪೂರ್ಣವಾಗಿತ್ತು ಎಂಬುದನ್ನು ತೋರಿಸಿದರು. ಹಿಂದೆ ಈ ಪದವು ಸಿಂಧೂ ನದಿಯ ಆಚೆಗೆ ಇರುವ (ಅಂದರೆ ಪೂರ್ವಕ್ಕೆ ಇರುವ) ಜನರನ್ನು ಸೂಚಿಸುತ್ತಿತ್ತು. ಈಗ ಇದು ಆ ಅರ್ಥವನ್ನು ಕಳೆದುಕೊಂಡಿದೆ. ಅದು ಆ ದೇಶವನ್ನಾಗಲಿ ಅವರ ಧರ್ಮವನ್ನಾಗಲಿ ಪ್ರತಿನಿಧಿಸುವುದಿಲ್ಲ. ಏಕೆಂದರೆ ಈಗ ಸಿಂಧೂ ನದಿಯ ಪೂರ್ವ ಭಾಗದಲ್ಲಿ ಬೇರೆ ಬೇರೆ ಧರ್ಮಾನುಯಾಯಿಗಳಾದ ವಿವಿಧ ಜನಾಂಗಗಳು ವಾಸಿಸುತ್ತಾರೆ.

ಅನಂತರ ಸ್ವಾಮೀಜಿ ವೇದಗಳ ಬಗ್ಗೆ ವಿಸ್ತಾರವಾಗಿ ಮಾತನಾಡಿದರು. ಅವು ಯಾವುದೊ ವ್ಯಕ್ತಿಯಿಂದ ಹೇಳಲ್ಪಟ್ಟವುಗಳಲ್ಲ. ಭಾವನೆಗಳು ನಿಧಾನವಾಗಿ ವಿಕಾಸವಾಗುತ್ತ ಬಂದು ಕೊನೆಯಲ್ಲಿ ಅವು ಪುಸ್ತಕ ರೂಪವನ್ನು ಪಡೆದವು. ಅನಂತರ ಆ ಪುಸ್ತಕವು ಪ್ರಮಾಣ ಗ್ರಂಥವಾಯಿತು. ಪುಸ್ತಕಗಳು ಅನೇಕ ಧರ್ಮಗಳ ವಿಷಯವನ್ನು ಹೇಳುತ್ತವೆ. ಪುಸ್ತಕಶಕ್ತಿ ಅನಂತವಾಗಿರುವಂತೆ ತೋರುತ್ತದೆ – ಎಂದು ಅವರು ಹೇಳಿದರು. ಹಿಂದೂಗಳಿಗೆ ವೇದಗಳಿವೆ ಮತ್ತು ಇನ್ನೂ ಸಾವಿರಾರು ವರ್ಷಗಳು ಅವರು ಅವುಗಳನ್ನು ಪ್ರಮಾಣ ಗ್ರಂಥಗಳನ್ನಾಗಿ ಇಟ್ಟುಕೊಳ್ಳಬೇಕಾಗಿದೆ. ಆದರೆ ಅವುಗಳ ಬಗೆಗಿನ ಭಾವನೆ ಬದಲಾಗಬೇಕು ಮತ್ತು ಅಭೇದ್ಯವಾದ ತಳಹದಿಯ ಮೇಲೆ ಹೊಸ ಭಾವನೆಗಳು ನಿರ್ಮಾಣವಾಗಬೇಕು. ವೇದಗಳು ಒಂದು ಅಪಾರ ಸಾಹಿತ್ಯ ರಾಶಿ. ಅವುಗಳಲ್ಲಿ ಶೇಕಡ ತೊಂಬತ್ತು ಭಾಗ ಕಳೆದುಹೋಗಿವೆ. ವೇದಭಾಗಗಳನ್ನು ಕೆಲವು ವಂಶದವರು ನೋಡಿಕೊಳ್ಳುತ್ತಿದ್ದರು. ಆ ವಂಶಸ್ಥರು ಕಣ್ಮರೆಯಾಗುವುದರೊಂದಿಗೆ ಆ ವೇದಭಾಗಗಳೂ ನಶಿಸಿಹೋದವು. ಆದರೆ ಈಗ ಉಳಿದಿರುವಷ್ಟನ್ನು ಇಡುವುದಕ್ಕೂ ಈ ದೊಡ್ಡ ಸಭಾಂಗಣ ಸಾಲದು. ಹಳೆಯ ಕಾಲದ ಸರಳ ಭಾಷೆಯಲ್ಲಿ ವೇದಗಳು ಬರೆಯಲ್ಪಟ್ಟವು. ಅವುಗಳ ವ್ಯಾಕರಣ ಎಷ್ಟು ಪ್ರಾಕೃತವಾದುದೆಂದರೆ ವೇದಗಳ ಕೆಲವು ಭಾಗಕ್ಕೆ ಯಾವ ಅರ್ಥವೂ ಇಲ್ಲವೆಂದು ಹೇಳುತ್ತಾರೆ.

ಅನಂತರ ಅವರು ವೇದಭಾಗಗಳಾದ ಕರ್ಮಕಾಂಡ ಹಾಗೂ ಜ್ಞಾನಕಾಂಡಗಳ ಬಗ್ಗೆ ಹೇಳಿದರು. ಕರ್ಮಕಾಂಡದಲ್ಲಿ ಸಂಹಿತೆ ಮತ್ತು ಬ್ರಾಹ್ಮಣಗಳು ಬರುತ್ತವೆ. ಬ್ರಾಹ್ಮಣಗಳು ಯಜ್ಞಗಳನ್ನು ಕುರಿತು ವಿವರಿಸುತ್ತವೆ. ಸಂಹಿತೆಗಳಲ್ಲಿ ಅನುಷ್ಟುಪ್​, ತ್ರಿಷ್ಟುಪ್​, ಜಗತೀ ಮುಂತಾದ ಛಂದಸ್ಸುಗಳಲ್ಲಿ ರಚಿತವಾದ, ಇಂದ್ರ ವರುಣ ಮುಂತಾದ ದೇವತೆಗಳನ್ನು ಕುರಿತಾದ ಸ್ತೋತ್ರಗಳು ಬರುತ್ತವೆ. ಈಗ ಈ ದೇವತೆಗಳು ಯಾರು ಎಂಬ ಪ್ರಶ್ನೆ ಉಂಟಾಗುತ್ತದೆ. ಇದನ್ನು ವಿವರಿಸುವುದಕ್ಕೆ ಯಾವುದೇ ಸಿದ್ಧಾಂತವನ್ನು ರೂಪಿಸಿದರೂ ಅದನ್ನು ಖಂಡಿಸುವುದಕ್ಕೆ ಇನ್ನೊಂದು ಸಿದ್ಧಾಂತ ತಲೆಯೆತ್ತುತ್ತಿತ್ತು. ಹೀಗೆಯೇ ಇದು ಮುಂದುವರಿಯಿತು.

ಅನಂತರ ಸ್ವಾಮೀಜಿಯವರು ವಿವಿಧ ಪೂಜಾ ಭಾವನೆಗಳ ಬಗ್ಗೆ ಹೇಳಿದರು. ಪುರಾತನ ಬ್ಯಾಬಿಲೋನಿಯನರ ದೃಷ್ಟಿಯಲ್ಲಿ ಜೀವಾತ್ಮವು ದೇಹದ ಒಂದು ಪ್ರತಿರೂಪ. ಅದಕ್ಕೆ ತನ್ನದೇ ಆದ ವೈಯುಕ್ತಿಕತೆ ಇಲ್ಲ ಮತ್ತು ದೇಹದೊಡನೆ ತನ್ನ ಸಂಬಂಧವನ್ನು ಅದು ಕಳಚಿಕೊಳ್ಳಲಾರದು. ಈ ಪ್ರತಿರೂಪಕ್ಕೂ ಹಸಿವು ಬಾಯಾರಿಕೆಗಳು ವೇದನೆ ಭಾವೋದ್ವೇಗಗಳು ಇವೆ ಎಂದು ಅವರು ನಂಬಿದ್ದರು. ಈ ದೇಹಕ್ಕೆ ನೋವಾದರೆ ಅದಕ್ಕೂ ನೋವಾಗುತ್ತದೆ, ಇದು ನಾಶವಾದರೆ ಅದೂ ನಾಶವಾಗುತ್ತದೆ. ಆದ್ದರಿಂದಲೆ ಈ ಭೌತಿಕ ದೇಹವನ್ನು ಸುರಕ್ಷಿತವಾಗಿ ಇರಿಸುವ ಪ್ರವೃತ್ತಿ ಬೆಳೆಯಿತು. ಇದರ ಪರಿಣಾಮವೇ ಮಮ್ಮಿ ಗೋರಿ ಮುಂತಾದವುಗಳು. ಈಜಿಪ್ಟಿಯನರು, ಬ್ಯಾಬಿಲೋನಿಯನರು ಮತ್ತು ಯಹೂದಿಗಳು ಈ ಪ್ರತಿರೂಪ ಭಾವನೆಯನ್ನು ಮೀರಿ ಹೋಗಲೇ ಇಲ್ಲ, ದೇಹವನ್ನು ಮೀರಿದ ಆತ್ಮನ ಭಾವನೆಯನ್ನು ಅವರು ಮುಟ್ಟಲಿಲ್ಲ.

ಮ್ಯಾಕ್ಸ್ ಮುಲ್ಲರರ ಅಭಿಪ್ರಾಯದ ಪ್ರಕಾರ ಋಗ್ವೇದದಲ್ಲಿ ಪಿತೃಪೂಜೆಯ ಯಾವ ಕುರುಹೂ ಇಲ್ಲ. ಶೂನ್ಯ ದೃಷ್ಟಿಯಿಂದ ನಮ್ಮನ್ನು ದಿಟ್ಟಿಸಿ ನೋಡುವ ಮಮ್ಮಿಗಳ ವಿಕಾರ ದೃಶ್ಯ ನಮಗಲ್ಲಿ ಕಾಣಿಸುವುದಿಲ್ಲ. ಅಲ್ಲಿ ದೇವತೆಗಳು ಮನುಷ್ಯನ ಸಖರಾಗಿದ್ದರು. ಪೂಜ್ಯ ಪೂಜಕರ ನಡುವಿನ ಸಂಬಂಧ ಆರೋಗ್ಯಕರವಾಗಿತ್ತು. ಅಲ್ಲಿ ಅಳುಮೋರೆಯಿಲ್ಲ, ಸರಳ ಆನಂದಕ್ಕೆ ಅಭಾವವಿರಲಿಲ್ಲ, ಮಂದಹಾಸಕ್ಕೆ ಕೊರತೆಯಿರಲಿಲ್ಲ. ಕಣ್ಣುಗಳು ಕಾಂತಿಯುಕ್ತವಾಗಿದ್ದವು. ಸ್ವಾಮೀಜಿಯವರು ವೇದವನ್ನು ಕುರಿತು ಆಲೋಚಿಸುವಾಗ ದೇವತೆಗಳ ನಗು ಕೂಡ ಕೇಳುವಂತೆ ತೋರುತ್ತದೆ ಎಂದರು. ವೈದಿಕ ಋಷಿಗಳು ತಮ್ಮ ಭಾವನೆಗಳನ್ನು ನಾಜೂಕಾಗಿ ವ್ಯಕ್ತಪಡಿಸದೆ ಇರಬಹುದು. ಆದರೆ ಅವರು ಸುಸಂಸ್ಕೃತರು ಮತ್ತು ಹೃದಯವಂತರು. ಅವರೊಂದಿಗೆ ಹೋಲಿಸಿದರೆ ನಾವು ಅತ್ಯಂತ ಅನಾಗರೀಕರು. ತಮ್ಮ ಈ ಭಾವನೆಯ ಸಮರ್ಥನೆಗಾಗಿ ಸ್ವಾಮೀಜಿಯವರು ಕೆಲವು ಮಂತ್ರಗಳನ್ನು ಪಠಿಸಿದರು: “ಪಿತೃಗಳು ವಾಸಿಸುವ, ದುಃಖವಿಲ್ಲದ ಸ್ಥಳಕ್ಕೆ ಅವನನ್ನು ಕರೆದೊಯ್ಯಿ.....” ಇತ್ಯಾದಿ. ಹೀಗೆ ಸತ್ತ ದೇಹವನ್ನು ಎಷ್ಟು ಬೇಗ ದಹಿಸಿದರೆ ಅಷ್ಟು ಒಳ್ಳೆಯದೆಂಬ ಭಾವನೆ ಉದಿಸಿತು. ಸೂಕ್ಷ್ಮದೇಹವೊಂದಿದೆ, ಅದು ದುಃಖವಿಲ್ಲದೆ ಕೇವಲ ಸುಖಮಯವಾದ ಸ್ಥಾನಕ್ಕೆ ಹೋಗುತ್ತದೆ ಎಂಬ ಜ್ಞಾನ ಅವರಿಗೆ ಕ್ರಮೇಣ ಪ್ರಾಪ್ತವಾಯಿತು. ಸೆಮೆಟಿಕ್​ ಧರ್ಮಗಳಲ್ಲಿ ದುಃಖ ಮತ್ತು ಭಯ ಪ್ರಧಾನ, ಮನುಷ್ಯನು ದೇವರನ್ನು ಕಂಡರೆ ಸಾಯುತ್ತಾನೆ ಎಂಬ ಭಾವನೆಯಿತ್ತು. ಆದರೆ ಋಗ್ವೇದದ ಪ್ರಕಾರ ಮನುಷ್ಯನು ದೇವರನ್ನು ಪ್ರತ್ಯಕ್ಷ ಕಂಡಾಗಲೇ ಅವನ ನಿಜವಾದ ಜೀವನ ಪ್ರಾರಂಭವಾಗುತ್ತದೆ.

\newpage

ಈಗ ಈ ದೇವತೆಗಳು ಯಾರು, ಎಂಬ ಪ್ರಶ್ನೆ ಏಳುತ್ತದೆ. ಕೆಲವೊಮ್ಮೆ ಇಂದ್ರನು ಬಂದು ಮನುಷ್ಯನಿಗೆ ಸಹಾಯ ಮಾಡುತ್ತಾನೆ, ಕೆಲವೊಮ್ಮೆ ಇಂದ್ರನು ಅತಿಯಾಗಿ ಸೋಮರಸವನ್ನು ಕುಡಿಯುತ್ತಾನೆ. ಆಗಾಗ ಸರ್ವಶಕ್ತ ಸರ್ವವ್ಯಾಪ್ತಿ ಎಂಬ ಗುಣಗಳನ್ನು ಅವನ ಮೇಲೆ ಆರೋಪಿಸುತ್ತಾರೆ. ವರುಣನಿಗೂ ಇದೇ ಮಾತು ಅನ್ವಯಿಸುತ್ತದೆ. ಈ ದೇವತೆಗಳ ಲಕ್ಷಣಗಳನ್ನು ಚಿತ್ರಿಸುವ ಕೆಲವು ಮಂತ್ರಗಳು ಅದ್ಭುತವಾಗಿವೆ. ಅವುಗಳ ಭಾಷೆ ತುಂಬ ಭವ್ಯವಾಗಿವೆ. ಉಪನ್ಯಾಸಕರು ಇಲ್ಲಿ ಪ್ರಳಯ ಸ್ಥಿತಿಯನ್ನು ಚಿತ್ರಿಸುವ ‘ನಾಸದೀಯ ಸೂಕ್ತ’ ವನ್ನು ಪಠಿಸಿದರು. ಇದರಲ್ಲಿಯೆ ‘ತಮಸ್ಸು ತಮಸ್ಸನ್ನು ಆವರಿಸಿದಂತಿತ್ತು’ ಎಂಬ ಭಾವನೆ ಬರುವುದು. ಇಂತಹ ಉದಾತ್ತ ಭಾವನೆಗಳನ್ನು ಇಷ್ಟೊಂದು ಕಾವ್ಯಮಯವಾಗಿ ವಿವರಿಸುವ ಜನರನ್ನು ಅನಾಗರೀಕರು ಅಸಂಸ್ಕೃತರು ಎಂದು ಕರೆಯುವುದಾದರೆ ನಮ್ಮನ್ನು ಏನೆಂದು ಕರೆದುಕೊಳ್ಳಬೇಕು – ಎಂದು ಅವರು ಕೇಳಿದರು. ಋಷಿಗಳನ್ನು ಅಥವಾ, ಇಂದ್ರ, ವರುಣ ಮುಂತಾದ ದೇವತೆಗಳನ್ನು ದೂಷಿಸಲು ತಾವು ಇಷ್ಟಪಡುವುದಿಲ್ಲವೆಂದು ಸ್ವಾಮೀಜಿ ಹೇಳಿದರು. ಇದೆಲ್ಲ ಒಂದು ವಿಶಾಲ ದೃಶ್ಯ ಪರಂಪರೆ – ಒಂದಾದ ಮೇಲೆ ಒಂದು ದೃಶ್ಯ ವಿಕಾಸವಾಗುತ್ತ ಬರುತ್ತದೆ, ಮತ್ತು ಏಕಂ ಸದ್ವಿಪ್ರಾ ಬಹುಧಾ ವದಂತಿ – “ಸತ್ಯವೊಂದೇ, ಆದರೂ ಜ್ಞಾನಿಗಳು ಅದನ್ನು ಬೇರೆ ಬೇರೆ ಹೆಸರಿನಿಂದ ಕರೆಯುತ್ತಾರೆ” – ಎಂಬ ಮಹಾನ್​ ಸತ್ಯ ಇದೆಲ್ಲಕ್ಕೂ ಹಿನ್ನೆಲೆಯಾಗಿದೆ. ಇದೆಲ್ಲವೂ ಅತ್ಯಂತ ರಹಸ್ಯಾತ್ಮಕವೂ ಅದ್ಭುತವೂ ಅತ್ಯಂತ ಸುಂದರವೂ ಆಗಿದೆ. ಇದು ನಮ್ಮ ತಿಳುವಳಿಕೆಗೆ ನಿಲುಕದಂತೆ ಕಂಡರೂ ತೆರೆಯು ಎಷ್ಟು ತೆಳ್ಳಗಿದೆಯೆಂದರೆ ಸ್ವಲ್ಪ ಮುಟ್ಟಿದರೆ ಸಾಕು ಅದು ಹರಿದು ಮರೀಚಿಕೆಯಂತೆ ಮಾಯವಾಗುತ್ತದೆ.

ಮುಂದುವರಿಸುತ್ತ ಅವರು, ಆರ್ಯರೂ ಕೂಡ ಗ್ರೀಕರಂತೆ ಸಮಸ್ಯೆಯ ಪರಿಹಾರಕ್ಕೆ ಹೊರ ಪ್ರಕೃತಿಯನ್ನು ಆಶ್ರಯಿಸಿದ್ದರು ಎಂದರು. ಬಾಹ್ಯ ಪ್ರಕೃತಿಯ ಪ್ರಲೋಭನೆಗೆ ಅವರೂ ಒಳಗಾದರು. ಬಾಹ್ಯ ಜಗತ್ತಿನಲ್ಲಿ ಸುಂದರವಾದುದರ, ಉತ್ತಮವಾದುದರ ಕಡೆಗೆ ಕ್ರಮೇಣ ಅವರ ಮನಸ್ಸು ಹರಿಯಿತು. ಆದರೆ ಭಾರತದಲ್ಲಿ ಯಾವುದು ಭವ್ಯವಾಗಿಲ್ಲವೊ ಅದಕ್ಕೆ ಪ್ರಾಶಸ್ತ್ಯವೇ ಇಲ್ಲ. ಮೃತ್ಯುವಿನ ನಂತರದ ರಹಸ್ಯವನ್ನು ಅರಿಯಬೇಕೆಂದು ಗ್ರೀಕರಿಗೆ ಹೊಳೆಯಲೇ ಇಲ್ಲ. ಆದರೆ ಇಲ್ಲಿ ಪ್ರಾರಂಭದಿಂದಲೂ ಈ ಪ್ರಶ್ನೆಯನ್ನು ಮತ್ತೆ ಮತ್ತೆ ಕೇಳಿದರು: “ನಾನಾರು? ಸತ್ತ ನಂತರ ನಾನೇನಾಗುತ್ತೇನೆ?” ಮನುಷ್ಯನು ಸತ್ತನಂತರ ಸ್ವರ್ಗಕ್ಕೆ ಹೋಗುತ್ತಾನೆ ಎಂದು ಗ್ರೀಕರು ಭಾವಿಸಿದರು. ಸ್ವರ್ಗಕ್ಕೆ ಹೋಗುವುದೆಂದರೇನು? ಪ್ರತಿಯೊಂದರಿಂದ ಹೊರಗೆ ಹೋಗುವುದು ಎಂದರ್ಥ. ಅವರ ದೃಷ್ಟಿಯಲ್ಲಿ ಆಂತರಿಕವೆಂಬುದೇನೂ ಇಲ್ಲವೇ ಇಲ್ಲ, ಇರುವುದೆಲ್ಲ ಹೊರಗಡೆಯೆ. ಗ್ರೀಕನ ಅನ್ವೇಷಣೆಯೆಲ್ಲ ಹೊರಗಡೆಯೆ, ಮಾತ್ರವಲ್ಲ, ಅವನೇ ತನ್ನ ಹೊರಗೆ ಇರುವಂತಿತ್ತು. ಇದೇ ಪ್ರಪಂಚದಂತಿರುವ, ಆದರೆ ದುಃಖವೊಂದಿಲ್ಲದ ಸ್ಥಳವನ್ನು ಸೇರಿದರೆ ಎಲ್ಲವನ್ನೂ ಪಡೆದಂತಾಯಿತೆಂದು ತೃಪ್ತಿಪಡುತ್ತಿದ್ದನು. ಅಲ್ಲಿಗೆ ಧರ್ಮದ ಭಾವನೆಯೆಲ್ಲ ಕೊನೆಗೊಳ್ಳುತ್ತಿತ್ತು. ಆದರೆ ಇದು ಹಿಂದೂ ಮನಸ್ಸಿಗೆ ತೃಪ್ತಿಯನ್ನೀಯಲಿಲ್ಲ. ಅವನ ವಿಶ್ಲೇಷಣಾತ್ಮಕ ದೃಷ್ಟಿಯಲ್ಲಿ ಈ ಎಲ್ಲ ಸ್ವರ್ಗಗಳು ಭೌತಿಕ ಜಗತ್ತಿನ ಒಳಗೇ ಬರುತ್ತವೆ. ‘ಸಂಘಾತದಿಂದ ಉಂಟಾದುದೆಲ್ಲ ನಶಿಸಲೇಬೇಕು’ ಎನ್ನುತ್ತಾರೆ ಹಿಂದೂಗಳು. ಅವರು ಬಾಹ್ಯಪ್ರಕೃತಿಯನ್ನು ‘ಆತ್ಮವೆಂದರೇನೆಂಬುದು ನಿನಗೆ ಗೊತ್ತೆ’ ಎಂದು ಕೇಳಿದರು. ಅದು ಇಲ್ಲವೆಂದಿತು. ‘ದೇವರಿರುವನೆ?’ ಎಂದು ಕೇಳಿದಾಗಲೂ ಅದು ‘ನನಗೆ ತಿಳಿಯದು’ ಎಂದಿತು. ಆಗ ಅವರು ಪ್ರಕೃತಿಯಿಂದ ವಿಮುಖರಾದರು. ಬಾಹ್ಯ ಪ್ರಕೃತಿ ಎಷ್ಟೇ ಶ್ರೇಷ್ಠವಾದರೂ ಭವ್ಯವಾದರೂ, ಅದು ದೇಶಕಾಲಗಳ ಮಿತಿಯೊಳಗಿದೆ, ಎಂದು ಅವರು ಅರಿತರು. ಆಗ ಇನ್ನೊಂದು ಧ್ವನಿ ಕೇಳಿಸಿತು; ಹೊಸ ಭವ್ಯ ಭಾವನೆಗಳು ಅವರ ಮನಸ್ಸಿನಲ್ಲಿ ಉದಿಸಿದವು. ಆ ಧ್ವನಿಯೇ “ನೇತಿ ನೇತಿ” ಇದಲ್ಲ, ಇದಲ್ಲ– ಎಂಬುದು. ಎಲ್ಲ ಬೇರೆ ಬೇರೆ ದೇವತೆಗಳನ್ನು ಒಂದು ತತ್ತ್ವಕ್ಕೆ ಇಳಿಸಿದರು. ಸೂರ್ಯ ಚಂದ್ರ ನಕ್ಷತ್ರಗಳು, ಅಷ್ಟೇ ಅಲ್ಲ, ಇಡೀ ವಿಶ್ವವೇ ಒಂದು. ಈ ಹೊಸ ಆದರ್ಶದ ಮೇಲೆ ಧರ್ಮದ ಆಧ್ಯಾತ್ಮಿಕ ತಳಹದಿಯನ್ನು ಕಟ್ಟಲಾಯಿತು.
\begin{verse}
\textbf{ನ ತತ್ರ ಸೂರ್ಯೋ ಭಾತಿ ನ ಚಂದ್ರತಾರಕಂ}\\\textbf{ನೇಮಾ ವಿದ್ಯುತೋ ಭಾಂತಿ ಕುತೋಽಯಮಗ್ನಿಃ~।}\\\textbf{ತಮೇವ ಭಾಂತಮನುಭಾತಿ ಸರ್ವಂ}\\\textbf{ತಸ್ಯ ಭಾಸಾ ಸರ್ವಮಿದಂ ವಿಭಾತಿ~॥}
\end{verse}

– “ಸೂರ್ಯ ಅಲ್ಲಿ ಬೆಳಗುವುದಿಲ್ಲ, ಚಂದ್ರ ತಾರೆ ಮಿಂಚೂ ಕೂಡ ಅಲ್ಲಿ ಬೆಳಗುವುದಿಲ್ಲ. ಇನ್ನು ಬೆಂಕಿಯ ಮಾತೇನು? ಅವನು ಬೆಳಗಿದರೆ ಎಲ್ಲವೂ ಬೆಳಗುವುವು. ಅವನಿಂದಲೇ ಎಲ್ಲವೂ ಬೆಳಗುತ್ತಿರುವುದು.” ಸಾಂತ, ಅಪಕ್ವ ಸಗುಣ–ಸಾಕಾರ ಭಾವನೆ ಇನ್ನಿಲ್ಲ. ನ್ಯಾಯಾಧಿಪತಿಯಾದ ದೇವರ ಸಾಮಾನ್ಯ ಕಲ್ಪನೆ ಇನ್ನಿಲ್ಲ. ಬಾಹ್ಯ ಅನ್ವೇಷಣೆ ಇನ್ನಿಲ್ಲ. ಇಲ್ಲಿಂದ ಮುಂದೆ ಅನ್ವೇಷಣೆಯೆಲ್ಲ ಒಳಮುಖವೇ. ಹೀಗೆ ಉಪನಿಷತ್ತುಗಳು ಭಾರತದ ಬೈಬಲ್​ ಆದುವು. ಇವು ಒಂದು ವಿಶಾಲ ಸಾಹಿತ್ಯ ರಾಶಿ. ಭಾರತದ ಎಲ್ಲ ಮತದವರೂ ಉಪನಿಷತ್ತುಗಳ ಆಧಾರದ ಮೇಲೆಯೇ ತಮ್ಮ ಮತವನ್ನು ಪ್ರತಿಪ್ಠಾಪಿಸುತ್ತಾರೆ.

ಸ್ವಾಮೀಜಿಯವರು ದ್ವೈತ ವಿಶಿಷ್ಟಾದ್ವೈತ, ಮತ್ತು ಅದ್ವೈತ ಸಿದ್ಧಾಂತಗಳ ಬಗ್ಗೆ ಮಾತನಾಡಿದರು. ಇವು ಒಂದಾದ ಮೇಲೆ ಒಂದರಂತೆ ಏರಿ ಹೋಗಬೇಕಾದ ಮೆಟ್ಟಲುಗಳು ಎಂದು ಹೇಳಿ ಇವುಗಳನ್ನು ಸಮರಸಗೊಳಿಸಿದರು. ಅದ್ವೈತವೇ ವಿಕಾಸದ ಕೊನೆಯ ಹಂತ ಮತ್ತು “ತತ್ತ್ವಮಸಿ” ಯೇ ಕೊನೆಯ ಮೆಟ್ಟಲು. ಶಂಕರ ರಾಮಾನುಜ ಮಧ್ವರಂತ ಶ್ರೇಷ್ಠ ಆಚಾರ್ಯರುಗಳು ಕೂಡ ತಪ್ಪು ಮಾಡಿರುವರು ಎಂಬುದನ್ನು ಸೂಚಿಸಿದರು. ಇವರಲ್ಲಿ ಪ್ರತಿಯೊಬ್ಬರಿಗೆ ಉಪನಿಷತ್ತೇ ಪರಮ ಪ್ರಮಾಣ. ಆದರೆ ಅದು ಒಂದೇ ವಿಷಯವನ್ನು, ಒಂದೇ ಮಾರ್ಗವನ್ನು ಬೋಧಿಸುತ್ತದೆ ಎಂದು ಅವರು ಭಾವಿಸಿದರು. ಶಂಕರಾಚಾರ್ಯರು ಇಡೀ ಉಪನಿಷತ್ತು ಕೇವಲ ಅದ್ವೈತವನ್ನು ಬೋಧಿಸುತ್ತದೆ, ಮತ್ತೇನೂ ಅಲ್ಲ ಎಂದು ತಪ್ಪಾಗಿ ದ್ವೈತ ಭಾವನೆಯಿರುವ ಹೇಳಿಕೆಗಳನ್ನೆಲ್ಲ ತಿರುಚಿ ಅವಕ್ಕೆ ಅದ್ವೈತದ ಅರ್ಥ ಬರುವಂತೆ ಮಾಡಿದರು. ಇದೇ ರೀತಿ ರಾಮಾನುಜ ಮತ್ತು ಮಧ್ವಾಚಾರ್ಯರು ಅದ್ವೈತ ಪರ ಹೇಳಿಕೆಗಳ ಅರ್ಥವನ್ನು ಬದಲಾಯಿಸಿದರು. ಉಪನಿಷತ್ತುಗಳು ಒಂದೇ ತತ್ತ್ವವನ್ನು ಬೋಧಿಸುತ್ತವೆ ಎಂಬುದು ನಿಜ. ಆದರೆ ಅದನ್ನು ಹಂತ ಹಂತವಾಗಿ ಬೋಧಿಸುತ್ತವೆ. ಆಧುನಿಕ ಭಾರತದಲ್ಲಿ ಧರ್ಮ ಸತ್ವಹೀನವಾಗಿದೆ, ಕೇವಲ ಅದರ ಬಾಹ್ಯಾಚಾರಗಳು ಉಳಿದುಕೊಂಡಿವೆ – ಎಂದು ಸ್ವಾಮೀಜಿ ವಿಷಾದಿಸಿದರು. ಜನರು ಹಿಂದೂಗಳೂ ಅಲ್ಲ, ವೇದಾಂತಿಗಳೂ ಅಲ್ಲ. ಅವರು ಕೇವಲ ಮುಟ್ಟಬೇಡಿ ಎನ್ನುವವರಾಗಿದ್ದಾರೆ. ಅಡುಗೆ ಮನೆಯೇ ಅವರ ದೇವಸ್ಥಾನ ಮತ್ತು ಅಡುಗೆ ಪಾತ್ರೆಗಳೇ ಅವರ ದೇವತೆಗಳು. ಈ ಪರಿಸ್ಥಿತಿ ಬದಲಾಗಬೇಕು. ಇದು ಎಷ್ಟು ಬೇಗ ಬದಲಾದರೆ ಅಷ್ಟು ನಮ್ಮ ಧರ್ಮಕ್ಕೆ ಒಳ್ಳೆಯದು. ಅದರ ಜಾಗದಲ್ಲಿ ಉಪನಿಷತ್ತುಗಳು ರಾರಾಜಿಸಲಿ, ಮತ್ತು ಬೇರೆ ಬೇರೆ ಮತಗಳ ನಡುವೆ ಕಲಹವಿಲ್ಲದಿರಲಿ.

ಸ್ವಾಮೀಜಿಯವರ ಆರೋಗ್ಯ ಚೆನ್ನಾಗಿಲ್ಲದ ಕಾರಣ ಇಷ್ಟು ಹೊತ್ತು ಮಾತನಾಡುವುದರಲ್ಲಿ ಅವರಿಗೆ ಸುಸ್ತಾಯಿತು. ಅವರು ಅರ್ಧ ಗಂಟೆ ವಿಶ್ರಾಂತಿ ಪಡೆದರು. ಆ ಸಮಯದಲ್ಲಿ ಸಭಿಕರು ಉಳಿದ ಉಪನ್ಯಾಸವನ್ನು ಕೇಳುವುದಕ್ಕೆ ತಾಳ್ಮೆಯಿಂದ ಕಾಯುತ್ತಿದ್ದರು. ಸ್ವಾಮೀಜಿಯವರು ಮತ್ತೆ ಹೊರಗೆ ಬಂದು ಮತ್ತೆ ಅರ್ಧಗಂಟೆ ಮಾತನಾಡಿದರು. ವೈವಿಧ್ಯದಲ್ಲಿ ಏಕತೆಯನ್ನು ಕಂಡುಹಿಡಿಯುವುದೇ ಜ್ಞಾನವೆಂಬುದನ್ನು ಅವರು ವಿವರಿಸಿದರು. ವೈವಿಧ್ಯಗಳ ಹಿಂದಿರುವ ಏಕತೆಯನ್ನು ಕಂಡುಕೊಂಡಾಗ ಪ್ರತಿಯೊಂದು ವಿಜ್ಞಾನವೂ ತನ್ನ ಚರಮಾವಸ್ಥೆಯನ್ನು ಮುಟ್ಟಿದಂತಾಗುತ್ತದೆ. ಇದು ಭೌತಿಕ ವಿಜ್ಞಾನದಲ್ಲಿ ಎಷ್ಟು ಸತ್ಯವೊ ಅಧ್ಯಾತ್ಮದಲ್ಲಿಯೂ ಅಷ್ಟೇ ಸತ್ಯ.

