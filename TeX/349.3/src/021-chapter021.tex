
\chapter{ಹಿಂದೂಧರ್ಮದ ಸಾಮಾನ್ಯ ತತ್ತ್ವಗಳು}

\begin{center}
(ಲಾಹೋರಿನಲ್ಲಿ ನೀಡಿದ ಉಪನ್ಯಾಸ)
\end{center}

ಪವಿತ್ರ ಆರ್ಯಾವರ್ತದಲ್ಲಿ ಪವಿತ್ರತಮ ಪುಣ್ಯಭೂಮಿ ಎಂದು ಯಾವುದು ಖ್ಯಾತಿಗೊಂಡಿದೆಯೋ ಅದೇ ಇದು. ಮನುವು ಯಾವುದನ್ನು ಬ್ರಹ್ಮಾವರ್ತವೆಂದು ಸಾರಿದನೋ ಆ ಪುಣ್ಯದೇಶವೇ ಇದು. ಚರಿತ್ರೆಯು ಸಾರುತ್ತಿರುವಂತೆ ಕಾಲಕ್ರಮೇಣ ಈ ಜಗತ್ತನ್ನೆಲ್ಲಾ ಆವರಿಸಿದ, ಅಗಾಧ ಆಧ್ಯಾತ್ಮಿಕ ಆಕಾಂಕ್ಷೆಯ ತರಂಗ ಹುಟ್ಟಿದುದು ಇಲ್ಲಿ. ಜಗತ್ತಿನ ಮೂಲೆ ಮೂಲೆಗಳಿಗೆ ನುಗ್ಗಿ, ಮೇಘಗರ್ಜನೆಯಿಂದ ಉದ್ಘೋಷಿಸುತ್ತಿರುವ ಈ ಆಧ್ಯಾತ್ಮಿಕ ಆಶಾಪ್ರವಾಹಗಳು ಹುಟ್ಟಿ ಒಂದುಗೂಡಿದುದು ಇಲ್ಲಿ. ಭಾರತದೇಶಕ್ಕೆ ತಂಡೋಪತಂಡವಾಗಿ ದಂಡೆತ್ತಿ ಬಂದ ಜನರ ಹಾವಳಿಯನ್ನು ತಡೆದು, ಆರ್ಯಾವರ್ತಕ್ಕೆ ನುಗ್ಗಿದ ಅನಾಗರಿಕ ಜನರಿಗೆ ತನ್ನ ವಿಶಾಲ ವಕ್ಷಸ್ಥಳವನ್ನು ಮೊದಲು ಒಡ್ಡಿದ ಭೂಮಿ ಇದೇ ಅಲ್ಲವೆ? ಎಷ್ಟು ಕಷ್ಟಗಳನ್ನನುಭವಿಸಿದರೂ, ತನ್ನ ಮಹಿಮೆಯನ್ನು ಶಕ್ತಿಯನ್ನು ಸಂಪೂರ್ಣವಾಗಿ ಕಳೆದುಕೊಳ್ಳದಿರುವ ನಾಡೇ ಇದು. ಇತ್ತೀಚೆಗೆ ಸಾಧು ನಾನಕನು ತನ್ನ ಅದ್ಭುತವಾದ ವಿಶ್ವಪ್ರೇಮವನ್ನು ಬೋಧಿಸಿದುದು ಇಲ್ಲಿ. ಇಲ್ಲೇ ಅಲ್ಲವೇ ಆತನ ವಿಶಾಲ ಹೃದಯವು ವಿಕಸಿತವಾದುದು? ಆತನ ಬಾಹುಗಳು ಹಿಂದೂ ಮತ್ತು ಮಹಮ್ಮದೀಯ ಜನಾಂಗಗಳನ್ನು, ಏಕೆ ಇಡೀ ಜಗತ್ತನ್ನು ಆಲಿಂಗಿಸಲು ಹೊರಚಾಚಿದುದು ಇಲ್ಲೇ ಅಲ್ಲವೆ? ನಮ್ಮ ಜನಾಂಗದ ಕಟ್ಟಕಡೆಯ ಮಹಾತ್ಮರಲ್ಲಿ ಒಬ್ಬನೆನ್ನಬಹುದಾದ ಗುರು ಗೋವಿಂದಸಿಂಗನು ಧರ್ಮಕ್ಕಾಗಿ ತನ್ನ ಮತ್ತು ತನ್ನಿಷ್ಟಮಿತ್ರ ಕಳತ್ರರ ರಕ್ತವನ್ನು ಚೆಲ್ಲಿ, ಯಾರಿಗೋಸ್ಕರ ಕಾದಾಡಿದನೋ ಅವರೇ ತನ್ನನ್ನು ದೂರ ನೂಕಲು, ತನ್ನ ಮಾತೃ ಭೂಮಿಯನ್ನು ಶಪಿಸದೆ, ಅವರ ವಿಚಾರದಲ್ಲಿ ಸ್ವಲ್ಪವೂ ಗೊಣಗುಟ್ಟದೆ, ಗಾಯ ಹೊಂದಿದ ಮೃಗರಾಜನಂತೆ ತನ್ನ ಪ್ರಾಣವನ್ನು ಬಿಡಲು ದಕ್ಷಿಣದೇಶಕ್ಕೆ ಬಂದು ಬಿಟ್ಟುದು ಈ ನಾಡಿನಿಂದ ಅಲ್ಲವೆ?

ಪಂಚನದಿಗಳ ನಾಡಿನಲ್ಲಿ, ಈ ನಮ್ಮ ಪುರಾತನ ದೇಶದಲ್ಲಿ, ನಾನು ನಿಮ್ಮ ಮುಂದೆ ಗುರುವಿನಂತೆ ನಿಂತಿಲ್ಲ; ಏಕೆಂದರೆ ನಾನು ಇತರರಿಗೆ ಬೋಧಿಸುವಷ್ಟು ಪ್ರಾಜ್ಞನಲ್ಲ. ನಮ್ಮ ಪಶ್ಚಿಮ ದಿಕ್ಕಿನ ಭ್ರಾತೃಗಳಲ್ಲಿ ಕುಶಲಪ್ರಶ್ನೆಯನ್ನು ಮಾಡಲು, ನಮ್ಮ ಅಭಿಪ್ರಾಯಗಳನ್ನು ಹೋಲಿಸಿ ನೋಡಲು, ನಮ್ಮ ದೇಶದ ಪೂರ್ವಭಾಗದಿಂದ ಬಂದು ನಿಂತಿದ್ದೇನೆ. ನಮ್ಮಲ್ಲಿರುವ ಭೇದಭಾವನೆಗಳನ್ನಲ್ಲ, ಸಮಾನ ಭಾವನೆಗಳನ್ನು ಮಾತ್ರ ಕಂಡುಕೊಳ್ಳಲು ಇಲ್ಲಿಗೆ ಬಂದಿರುತ್ತೇನೆ. ಯಾವ ತತ್ತ್ವದ ಆಧಾರದ ಮೇಲೆ ನಾವು ಸರ್ವದಾ ಅಣ್ಣತಮ್ಮಂದಿರಂತಿರಬಹುದೋ, ಯಾವುದರ ಆಧಾರದ ಮೇಲೆ ಅನಾದಿಯಾಗಿರುವ ವಾಣಿಯು ವೃದ್ಧಿಯಾದ ಹಾಗೆಲ್ಲ ಬಲವಾಗ ಬಹುದೋ, ಅದನ್ನು ತಿಳಿದುಕೊಳ್ಳಲೆತ್ನಿಸಲು ಇಲ್ಲಿಗೆ ಬಂದಿರುವೆನು. ನಾಶಕರವಲ್ಲ, ರಚನಾತ್ಮಕ ಕಾರ್ಯಕ್ರಮವನ್ನು ನಿಮ್ಮ ಮುಂದಿಡುವ ಪ್ರಯತ್ನ ಮಾಡಲು ಇಲ್ಲಿ ನಿಂತಿರುವೆನು. ನಿಂದೆಯ ಕಾಲ ಮುಗಿದು ನಾವು ರಚನಾತ್ಮಕ ಕಾರ್ಯಗಳ ಪ್ರವರ್ತನೆಯನ್ನು ನಿರೀಕ್ಷಿಸುತ್ತಿದ್ದೇವೆ. ಕೆಲವು ವೇಳೆ ಈ ಪ್ರಪಂಚಕ್ಕೆ ನಿಂದೆ – ಪ್ರಾಯಶಃ ಈಗಿನ ನಿಂದೆಗಿಂತ ಬಲವಾದ ನಿಂದೆ–ಅವಶ್ಯಕ. ಆದರೆ ಅವು ಸ್ವಲ್ಪ ಕಾಲದ ಮಟ್ಟಿಗೆ ಮಾತ್ರ ಬೇಕಾಗಿರುವುದು. ನಿರಂತರವೂ ಬೇಕಾಗಿರುವುದು ವರ್ಧಿಸುವ, ರಚನಾ ಶಕ್ತಿಯುಳ್ಳ ಕಾರ್ಯಕಲಾಪಗಳು, ನಿಂದೆ ಮತ್ತು ನಾಶಕರ ಕಾರ್ಯಕಲಾಪಗಳಲ್ಲ. ನೂರು ವರ್ಷಗಳಿಂದ ನಮ್ಮ ದೇಶದಲ್ಲಿ ನಿಂದೆ ಎಂಬೀ ಪ್ರವಾಹವು ಎಲ್ಲೆಲ್ಲಿಯೂ ತುಂಬಿ ಹರಿಯುತ್ತಿದೆ. ಪಾಶ್ಚಾತ್ಯ ವಿಜ್ಞಾನಶಾಸ್ತ್ರವು ದೇಶದ ಅಂಧಕಾರಾವೃತ ಪ್ರದೇಶಗಳಲ್ಲೆಲ್ಲಾ ಸಂಪೂರ್ಣವಾಗಿ ಪಸರಿಸಿದ ಫಲವಾಗಿ, ಎಲ್ಲದಕ್ಕಿಂತಲೂ ಹೆಚ್ಚಾಗಿ ಸಂದುಗೊಂದುಗಳೂ ಮೂಲೆಮುಡುಕುಗಳು ಎದ್ದು ಕಾಣುತ್ತಿವೆ. ಸಹಜವಾಗಿಯೇ ನಮ್ಮ ದೇಶದಲ್ಲಿ ಎಲ್ಲೆಡೆಯಲ್ಲಿಯೂ ಸತ್ಯಪ್ರೇಮಿಗಳೂ ನ್ಯಾಯಪಕ್ಷಪಾತಿಗಳೂ ವಿಶಾಲಮತಿಗಳೂ ಆದ ಮಹಾಮಹಿಮರು ಜನ್ಮವೆತ್ತಿದರು. ಇವರಲ್ಲಿ ದೇಶ ವಾತ್ಸಲ್ಯ, ಉಚ್ಚತರ ಧರ್ಮಾಭಿಮಾನ, ದೈವಭಕ್ತಿ, ಇವು ತುಂಬಿ ತುಳುಕುತ್ತಿದ್ದುವು. ಈ ಮಹನೀಯರಿಗೆ ಅಷ್ಟು ಪ್ರೇಮವಿದ್ದುದರಿಂದಲೇ, ಇವರು ತಮ್ಮ ಮಾತೃಭೂಮಿಯಲ್ಲಿರುವ ಕುಂದುಕೊರತೆಗಳಿಗಾಗಿ ಅಷ್ಟು ಹಿರಿದಾದ ದುಃಖವನ್ನು ಅನುಭವಿಸುತ್ತಿದ್ದುದರಿಂದಲೇ, ಅವರಿಗೆ ಯಾವು ಯಾವುವು ತಪ್ಪೆಂದು ಕಂಡವೋ ಅವುಗಳನ್ನೆಲ್ಲಾ ಬಲವಾಗಿ ಖಂಡಿಸಿದರು. ಗತಿಸಿಹೋದ ಆ ಮಹಾತ್ಮರಿಗೆ ಜಯವಾಗಲಿ! ಅವರು ಬಹಳ ಒಳ್ಳೆಯದನ್ನು ಮಾಡಿರುತ್ತಾರೆ; ಆದರೆ ವರ್ತಮಾನಕಾಲದ ವಾಣಿಯ “ಸಾಕು” ಎಂಬ ಕೂಗು ನಮಗೆ ಕೇಳಿಬರುತ್ತಿದೆ. ಸಾಕಾದಷ್ಟು ನಿಂದೆ ಆಗಿದೆ. ಸಾಕಾದಷ್ಟು ತಪ್ಪುಗಳು ಹೊರಬಿದ್ದಿವೆ. ರಚನಾತ್ಮಕ ಕಾಲ ಪುನಃ ಸಮೀಪಿಸಿದೆ; ನಮ್ಮಲ್ಲಿ ಚೆದುರಿಹೋಗಿರುವ ಬಲಗಳನ್ನು ಶೇಖರಿಸಿ, ಒಂದು ಕೇಂದ್ರದಲ್ಲಿ ಅವುಗಳನ್ನು ಏಕೀಕರಿಸಿ, ಈಗ ಅನೇಕ ಶತಮಾನಗಳಿಂದ ಮುಂದುವರಿಯದೆ ನಿಂತಿರುವ ನಮ್ಮ ಜನಾಂಗವನ್ನು, ಅವುಗಳ ಮೂಲಕ ಮುಂದುವರಿಯುವಂತೆ ಮಾಡುವ ಸಮಯ ಒದಗಿದೆ. ಗೃಹವನ್ನು ಶುಚಿಮಾಡಿ ಆಗಿದೆ. ಅದರಲ್ಲಿ ಮತ್ತೆ ವಾಸ ಮಾಡುವಂತಾಗಲಿ. ಧರ್ಮಪಥದಲ್ಲಿ ಬಿದ್ದಿದ್ದ ಮುಳ್ಳು ಕಲ್ಲುಗಳು ಹೋಗಿವೆ. ಆರ್ಯಪುತ್ರರೇ, ಮುಂದೆ ಸಾಗಿರಿ!

ಮಹನೀಯರೇ, ನಾನು ನಿಮ್ಮಲ್ಲಿಗೆ ಬಂದಿರುವುದು ಮೇಲೆ ಹೇಳಿದ ಉದ್ದೇಶದಿಂದ. ಪ್ರಾರಂಭದಲ್ಲೇ, ನಾನು ಯಾವ ಪಂಗಡಕ್ಕೂ, ಯಾವ ಸಂಪ್ರದಾಯಕ್ಕೂ ಸೇರಿದವನಲ್ಲವೆಂದು ನಿಮಗೆ ತಿಳಿಸಲಿಚ್ಛಿಸುತ್ತೇನೆ. ನನ್ನ ಮಟ್ಟಿಗೆ, ಇವುಗಳೆಲ್ಲವೂ ಮಹತ್ತರವಾದವು; ಮಹಿಮೆಯುಳ್ಳವು. ನನಗೆ ಅವುಗಳೆಲ್ಲದರಲ್ಲಿಯೂ ಆದರವಿದೆ. ಅವುಗಳಲ್ಲಿ ಅಡಕವಾಗಿರುವ ಗುಣಗಳನ್ನೂ ಸತ್ಯವನ್ನೂ ಕಂಡುಕೊಳ್ಳಲು, ನನ್ನ ಆಯುಷ್ಯವನ್ನೆಲ್ಲ ಉಪಯೋಗಿಸಲು ಪ್ರಯತ್ನಿಸುತ್ತಿದ್ದೇನೆ. ಆದುದರಿಂದ ಇಂದು ನಿಮ್ಮ ಮುಂದೆ ನಾವೆಲ್ಲರೂ ಒಪ್ಪಿರುವ ವಿಷಯಗಳನ್ನಿಟ್ಟು, ಸಾಧ್ಯವಾದರೆ, ನಾವೆಲ್ಲರೂ ಒಪ್ಪಬಹುದಾದ ಹಿಂದೂಧರ್ಮದ ಸಾಮಾನ್ಯ ಆಧಾರಗಳನ್ನು ಕಂಡುಕೊಳ್ಳುವ ಆಲೋಚನೆಯನ್ನು ಮಾಡುತ್ತೇನೆ. ಭಗವಂತನ ಕೃಪೆಯಿಂದ, ಈ ಯೋಚನೆ ಸಾಧುವೂ, ಸಾಧ್ಯವೂ ಎಂದು ಕಂಡುಬಂದರೆ, ಅದಕ್ಕೆ ಕೈಹಾಕಿ ಅದನ್ನು ವ್ಯವಹಾರದಲ್ಲಿ ಅನುಷ್ಠಾನಕ್ಕೆ ತರೋಣ, ನಾವೆಲ್ಲರೂ ಹಿಂದೂಗಳು. “ಹಿಂದೂ” ಎಂಬ ಪದವನ್ನು ನಾನು ಯಾವ ಕೆಟ್ಟ ಅರ್ಥದಲ್ಲೂ ಬಳಸುತ್ತಿಲ್ಲ. ಅದಕ್ಕೆ ಕೆಟ್ಟ ಅರ್ಥ ಇದೆ ಎನ್ನುವವರ ಅಭಿಪ್ರಾಯವನ್ನು ನಾನು ಒಪ್ಪುವುದಿಲ್ಲ. ಹಿಂದೆ, ಹಿಂದೂ ಎಂದರೆ ಸಿಂಧೂ ನದಿಯಿಂದಾಚೆ ವಾಸ ಮಾಡುವವರೆಂದು ಅರ್ಥವಿತ್ತು. ಆದರೆ ನಮ್ಮನ್ನು ದ್ವೇಷಿಸುವವರಲ್ಲಿ ಅನೇಕರು ಆ ಪದಕ್ಕೆ ತಪ್ಪು ವಿವರಣೆಯನ್ನು ನೀಡಿರುವರು. ಆದರೆ ಹೆಸರಿನಲ್ಲೇನಿದೆ? ಹಿಂದೂ ಶಬ್ದವೂ ಭವ್ಯವಾದ ಆಧ್ಯಾತ್ಮಿಕ ವಿಷಯಗಳನ್ನು ಸೂಚಿಸುವುದೇ; ಇಲ್ಲವೇ ನಿಂದಾರ್ಹನೂ, ಪತಿತನೂ, ಅಪ್ರಯೋಜಕನೂ, ಅನಾಗರಿಕನೂ ಆದ ಮಾನವನೆಂದು ಸೂಚಿಸುವುದೇ, ಎಂಬುದನ್ನು ತೋರಿಸುವುದು ನಮ್ಮ ಅಧೀನದಲ್ಲಿದೆ. ಈಗ ಹಿಂದೂ ಎಂಬ ಪದಕ್ಕೆ ಹೀನವಾದ ಅರ್ಥ ಏನಾದರೂ ಇದ್ದರೆ, ಅದನ್ನು ಗಮನಿಸದಿರಿ. ಯಾವುದೇ ಭಾಷೆಯಲ್ಲಾಗಲೀ ಕಂಡುಬರುವ ಶಬ್ದಗಳಲ್ಲಿ, ಈ ಶಬ್ದವು ಅತ್ಯುತ್ಕೃಷ್ಟವಾದುದೆಂದು ಕಾರ್ಯತಃ ತೋರಿಸಲು ಸಿದ್ಧರಾಗೋಣ. ನಾವು ನಮ್ಮ ಪೂರ್ವಿಕರ ಸಲುವಾಗಿ ಅವಮಾನ ಪಡಬೇಕಾಗಿಲ್ಲ ಎಂಬುದು ನನ್ನ ಜೀವನದ ತತ್ತ್ವಗಳಲ್ಲೊಂದಾಗಿದೆ. ಇದುವರೆಗೆ ಜನ್ಮವೆತ್ತಿ ಆತ್ಮ ಗೌರವದಿಂದ ನಡೆದ ಮಾನವರಲ್ಲಿ ನಾನೂ ಒಬ್ಬ. ಆದರೆ, ನಾನು ಹೆಮ್ಮೆಪಡುವುದು ಖಂಡಿತವಾಗಿಯೂ ನನಗಾಗಿ ಅಲ್ಲ; ನನ್ನ ಪೂರ್ವಜರಿಗೋಸ್ಕರವಾಗಿ. ನಾನು ನಮ್ಮವರ ಚರಿತ್ರೆಯನ್ನು ಹೆಚ್ಚು ಹೆಚ್ಚಾಗಿ ಓದಿದ ಹಾಗೆಲ್ಲಾ, ಹೆಚ್ಚು ಹೆಚ್ಚಾಗಿ ಹಿಂದಿರುಗಿ ನೋಡಿದ ಹಾಗೆಲ್ಲಾ, ಹೆಚ್ಚು ಹೆಚ್ಚಾಗಿ ಅವರ ವಿಚಾರದಲ್ಲಿ ನನಗಿರುವ ಹೆಮ್ಮೆ ನನ್ನಲ್ಲಿ ತಲೆದೋರುತ್ತದೆ. ನನ್ನ ದೃಢನಂಬಿಕೆಗನುಸಾರವಾಗಿ ಕೆಲಸ ಮಾಡುವ ಶ್ರದ್ಧೆ ಶಕ್ತಿಗಳನ್ನು ಕೊಟ್ಟು, ಧೂಳಿಗೆ ಸಮನಾಗಿದ್ದ ನನ್ನನ್ನು ಉನ್ನತ ಸ್ಥಿತಿಗೆ ತಂದಿದೆ; ಮಹಾತ್ಮರಾದ ನಮ್ಮ ಹಿರಿಯರು ಹಾಕಿರುವ ಮಾರ್ಗವನ್ನು ವ್ಯವಹಾರದಲ್ಲಿ ಅನುಸರಿಸುವಂತೆ ಪ್ರೇರಿಸಿದೆ. ಪ್ರಾಚೀನ ಆರ್ಯರ ವಂಶಸ್ಥರಾದ ನೀವೂ, ಭಗವಂತನ ಕೃಪೆಯಿಂದ ಆ ಹೆಮ್ಮೆಯನ್ನು ಪಡೆಯಿರಿ, ನಿಮಗೆ ನಿಮ್ಮ ಪೂರ್ವಿಕರಲ್ಲಿರುವ ವಿಶ್ವಾಸವು ನಿಮ್ಮ ರಕ್ತಗತವಾಗಲಿ, ಅದು ನಿಮ್ಮ ಜೀವನದಲ್ಲಿ ಮುಖ್ಯ ಅಂಶವಾಗಲಿ! ಅದು ಇಡಿಯ ಜಗತ್ತಿಗೇ ಕಲ್ಯಾಣವನ್ನುಂಟುಮಾಡಲಿ!

ನಾವು ಯಾವ ವಿಚಾರದಲ್ಲಿ ಏಕಾಭಿಪ್ರಾಯವುಳ್ಳವರಾಗಿರುವೆವು ಎಂಬುದನ್ನೂ, ನಮ್ಮ ಜೀವನದ ಸರ್ವಸಮ್ಮತ ಆಧಾರವಾವುದೆಂಬುದನ್ನೂ ಅರಸುವ ಪ್ರಯತ್ನ ಮಾಡುವುದಕ್ಕೆ ಮೊದಲು, ಇದೊಂದು ವಿಷಯವನ್ನು ನಾವು ಜ್ಞಾಪಕದಲ್ಲಿ ಇಡಬೇಕು. ಪ್ರತಿಯೊಬ್ಬ ಮನುಷ್ಯನಲ್ಲಿರುವಂತೆ, ಪ್ರತಿಯೊಂದು ಜನಾಂಗದಲ್ಲಿಯೂ ವ್ಯಕ್ತಿತ್ವವಿರುವುದು. ಅವರವರ ವಿಶೇಷ ಗುಣಗಳಿಂದಾಗಿ ವ್ಯಕ್ತಿಗಳಲ್ಲಿ ಭೇದವಿರುವಂತೆ, ಜನಾಂಗಗಳಲ್ಲಿಯೂ ಅವುಗಳ ವಿಶಿಷ್ಟ ಲಕ್ಷಣಗಳ ದೆಸೆಯಿಂದ ಪರಸ್ಪರ ಭೇದಗಳಿವೆ. ಪ್ರತಿಯೊಬ್ಬನೂ ತನ್ನ ಕರ್ಮಾನುಸಾರ, ನಿಯಮಬದ್ಧ ವಾದೊಂದು ಮಾರ್ಗದಲ್ಲಿ, ತನ್ನದೊಂದು ಸಂಕಲ್ಪವನ್ನು ನೆರವೇರಿಸಿಕೊಳ್ಳುವುದು ಹೇಗೆ ಅವನ ಕೆಲಸವೋ, ಹಾಗೆಯೇ ಪ್ರತಿಯೊಂದು ರಾಷ್ಟ್ರಕ್ಕೂ ತಾನು ನೆರವೇರಿಸಿಕೊಳ್ಳಬೇಕಾಗಿರುವ ಉದ್ಧೇಶವೊಂದಿರುವುದು, ಅದು ಮಾಡಬೇಕಾದ ಕೆಲಸವೊಂದಿರುವುದು, ಅದು ಅನ್ಯರಿಗೆ ಕೊಡಬೇಕಾದ ಸಂದೇಶವೊಂದಿರುವುದು. ಆದುದರಿಂದ ನಾವು ಪ್ರಾರಂಭದಲ್ಲಿಯೇ, ನಮ್ಮ ಜನಾಂಗವು ಮಾಡಬೇಕಾದ ಕೆಲಸವಾವುದು, ಜಗತ್ತಿನ ರಾಷ್ಟ್ರಗಳ ಪ್ರಗತಿ ಯಾತ್ರೆಯಲ್ಲಿ ಅದು ಪಡೆಯಬೇಕಾದ ಸ್ಥಾನ ಯಾವುದು, ಜನಾಂಗಗಳ ಪರಸ್ವರ ಸೌಹಾರ್ದಕ್ಕೆ ನಮ್ಮ ಜನಾಂಗವು ಒದಗಿಸಿಕೊಡಬೇಕಾಗಿರುವ ಅವಶ್ಯಕವಾದ ಭಾವನೆಗಳು ಯಾವುವು ಎಂಬೀ ವಿಚಾರಗಳನ್ನು ತಿಳಿದುಕೊಳ್ಳಬೇಕು. ಬಾಲಕರಾಗಿರುವಾಗ ನಾವು ಕೆಲವು ಕಥೆಗಳನ್ನು ಕೇಳಿದ್ದೇವೆ. ಸರ್ಪಗಳ ಹೆಡೆಗಳಲ್ಲಿ ರತ್ನಗಳು ಇರುವುವು, ರತ್ನ ಹೆಡೆಯಲ್ಲಿರುವ ತನಕ ಸರ್ಪಕ್ಕೆ ಸಾವಿಲ್ಲ; ಹೀಗೆಯೇ ಕೆಲವು ರಾಕ್ಷಸರ ಜೀವಗಳು ಪಕ್ಷಿಗಳಲ್ಲಿ ಅಡಗಿಕೊಂಡಿರುವುವು; ಆ ಪಕ್ಷಿಗಳನ್ನು ಕೊಂದಲ್ಲದೆ ಆ ರಾಕ್ಷಸರಿಗೆ ಬೇರಾವ ವಿಧದಲ್ಲಿಯೂ ಸಾವಿಲ್ಲ ಮುಂತಾಗಿ. ಜನಾಂಗದ ಜೀವವೂ ಹೀಗೆಯೇ. ಜನಾಂಗಜೀವನದ ಮೂಲ ಒಂದಿದೆ. ಅದನ್ನು ಮುಟ್ಟಿದಲ್ಲದೆ ಆ ಜನಾಂಗಕ್ಕೆ ನಾಶವಿಲ್ಲ. ಈ ದೃಷ್ಟಿಯಿಂದ ನೋಡಿದರೆ, ಚರಿತ್ರೆಯಲ್ಲಿ ಕಂಡುಬರುವ ಅತ್ಯದ್ಭುತ ವಿಷಯವೊಂದು ನಮಗೆ ಸ್ಪಷ್ಟವಾಗುವುದು. ಅನಾಗರಿಕ ಜನಾಂಗಗಳು ತಂಡೋಪತಂಡವಾಗಿ ಈ ನಮ್ಮ ಪವಿತ್ರ ಭೂಮಿಯ ಮೇಲೆ ದಂಡೆತ್ತಿ ಬಂದವು. ‘ಅಲ್ಲಾ ಹೋ ಅಕ್ಬರ್​’, ಎಂಬ ಧ್ವನಿ ನೂರಾರು ವರ್ಷಗಳವರೆಗೆ ಆಕಾಶವನ್ನೆಲ್ಲಾ ನಡುಗಿಸಿತು. ಹಿಂದೂವಿಗೂ ತನಗೆ ಯಾವ ಗಳಿಗೆಗೆ ಸಾವು ಪ್ರಾಪ್ತವಾಗುತ್ತಿತ್ತೆಂಬುದು ಗೊತ್ತಾಗುತ್ತಿರಲಿಲ್ಲ. ಜಗತ್ತಿನ ಐತಿಹಾಸಿಕ ದೇಶಗಳಲ್ಲೆಲ್ಲಾ ಅತಿ ಹಿರಿದಾದ ದುಃಖವನ್ನೂ ಪರಾಧೀನತೆಯನ್ನೂ ನಮ್ಮ ದೇಶವು ಅನುಭವಿಸಿರುವುದು. ಆದರೂ ನಾವು ವಿನಾಶ ಹೊಂದಿಲ್ಲ. ಆವಶ್ಯಕವಾದಲ್ಲಿ, ಪುನಃ ಪುನಃ ಕಷ್ಟಗಳನ್ನು ಸಹಿಸಲು ಸಿದ್ಧರಾಗಿದ್ದೇವೆ. ನಮ್ಮ ಜನಾಂಗ ಏಕರೀತಿಯಾಗಿದೆ. ಇಷ್ಟೇ ಅಲ್ಲ, ನಾವು ಈಗ ಶಕ್ತಿವಂತರಾಗಿರುವುದು ಮಾತ್ರವಲ್ಲ, ಹೊರಗೂ ಹೋಗಲು ಸಿದ್ಧರಾಗಿದ್ದೇವೆ. ಎಲ್ಲ ವಿಸ್ತಾರವೂ ಜೀವಂತಿಕೆಯ ಲಕ್ಷಣ.

ನಮ್ಮ ಭಾವನೆಗಳನ್ನು ತತ್ತ್ವಗಳನ್ನು ಈಗಿನ ಕಾಲದಲ್ಲಿ, ಭಾರತ ದೇಶದ ಮೇರೆಯಲ್ಲಿ ಬಂಧಿಸಿ ನಿಲ್ಲಿಸಲು ನಮ್ಮಿಂದ ಸಾಧ್ಯವಾಗುವುದಿಲ್ಲ. ನಮಗೆ ಅದು ಇಷ್ಟವಾಗಿರಲಿ, ಇಲ್ಲದಿರಲಿ, ಅವು ಹೊರಟಿವೆ, ಇತರ ಜನಾಂಗಗಳು ಸಾಹಿತ್ಯಗಳಲ್ಲಿ ಸೇರಿಹೋಗಿವೆ. ಆ ರಾಷ್ಟ್ರಗಳಲ್ಲಿ ಸಾರ್ವಭೌಮಿಕ ಸ್ಥಾನವನ್ನು ಅವು ಗಳಿಸಿವೆ. ಇದಕ್ಕೆ ಕಾರಣವೇನು? ಮಾನವನ ಮನಸ್ಸನ್ನು ಆವರಿಸಬಲ್ಲ ವಿಷಯಗಳಲ್ಲಿ ಮಹತ್ತಾದ, ಘನವಾದ ಮತ್ತು ಅತ್ಯುತ್ತಮವಾದ ಪ್ರಸಂಗ ಯಾವುದೆಂದರೆ ತತ್ತ್ವ ವಿಚಾರ ಮತ್ತು ಆತ್ಮವಿದ್ಯೆ. ಇವುಗಳನ್ನು ದಾನ ಮಾಡಿ ನಮ್ಮ ಭಾರತಭೂಮಿ ಈ ಜಗತ್ತಿನ ಪ್ರಗತಿಮಾರ್ಗಕ್ಕೆ ಮಹಾಸಹಾಯವನ್ನು ಮಾಡಿರುವುದೇ ಇದಕ್ಕೆ ಕಾರಣ. ನಮ್ಮ ಪೂರ್ವಿಕರು ಇನ್ನೂ ಅನೇಕ ವಿಷಯಗಳನ್ನು ತಿಳಿಯಲು ಪ್ರಯತ್ನಿಸಿದ್ದರೆಂಬುದೂ, ಇತರ ಜನಾಂಗದವರಂತೆ ಅವರು ಬಾಹ್ಯ ಪ್ರಪಂಚದ ಗುಟ್ಟನ್ನು ಹೊರಗೆಡಹಲು ಯತ್ನಿಸಿದ್ದರೆಂಬುದೂ ಎಲ್ಲರಿಗೂ ತಿಳಿದ ವಿಷಯ. ಈ ವಿಷಯದಲ್ಲಿ ಅತ್ಯಂತ ಪ್ರತಿಭಾಶಾಲಿಗಳಾದ ಈ ಅದ್ಭುತ ಜನರು, ಜಗತ್ತು ಹೆಮ್ಮೆಪಡಬಹುದಾದ, ಪವಾಡಸದೃಶವಾದ ಸಂಗತಿಗಳನ್ನು ಪ್ರಕಟಿಸಬಹುದಾಗಿತ್ತು. ಆದರೆ ಇದಕ್ಕೂ ಅತಿಶಯವಾದ ಇನ್ನೊಂದು ವಿಷಯದ ಸಲುವಾಗಿ, ಬಾಹ್ಯ ವಿಷಯಗಳ ಪ್ರಸ್ತಾಪವನ್ನು ಬಿಟ್ಟರು. ಇದಕ್ಕೂ ಅತ್ಯುತ್ತಮವಾದ ವಿಷಯಗಳನ್ನು ನಮ್ಮ ವೇದಗಳು ಸಾರುತ್ತವೆ. “ವಿಕಾರ ರಹಿತವಾದ ಪರಮಾತ್ಮನನ್ನು ನಮಗೆ ಯಾವುದು ತೋರಿಸಿ ಕೊಡುತ್ತದೆಯೋ ಅದೇ ಸರ್ವೋತ್ಕೃಷ್ಟವಾದ ವಿಜ್ಞಾನ” ವೆಂದು ಅವು ಸಾರುತ್ತಿರುವುವು. ಸಾವು, ಕೇಡು, ಸುಖದುಃಖ ಇವುಗಳಿಂದ ತುಂಬಿರುವ, ವಿಕಾರ ಹೊಂದುವ ಬಾಹ್ಯ ಪ್ರಪಂಚದ ಅರಿವನ್ನು ನಮಗೆ ಮಾಡಿಕೊಡುವ ವಿಜ್ಞಾನವು ಮಹತ್ವವಾಗಿರಬಹದು, ನಿಜ; ಆದರೆ ಯಾವನಿಂದ ಶಾಂತಿ ಅಮೃತತ್ವ, ಪರಿಪೂರ್ಣತೆ ಇವು ದೊರಕಿ, ದುಃಖ ಎಂಬುದು ಸಂಪೂರ್ಣವಾಗಿ ಅಳಿಸಿ ಹೋಗುವುದೋ ಅಂತಹ ಅಚ್ಯುತನೂ, ಮಂಗಳಮಯನೂ ಆದ ಪರಬ್ರಹ್ಮನನ್ನು ತೋರಿಸುವ ಜ್ಞಾನವೇ ಅತಿ ಭವ್ಯವಾದುದೆಂದು ನಮ್ಮ ಪ್ರಾಚೀನರು ಅಭಿಪ್ರಾಯಪಟ್ಟರು. ನಮ್ಮವರು ಇಷ್ಟಪಟ್ಟಿದ್ದರೆ ನಮಗೆ ಅನ್ನ ವಸ್ತುಗಳನ್ನೂ, ನಮ್ಮ ನೆರೆಹೊರೆಯವರ ಮೇಲೆ ಅಧಿಕಾರವನ್ನು, ಅನ್ಯರನ್ನು ಜಯಿಸಿ ಅವರನ್ನಾಳುವ ಮರ್ಮವನ್ನೂ ಬಲಿಷ್ಠರು ದುರ್ಬಲರನ್ನು ಆಳುವುದನ್ನೂ ಕಲಿಸಿ ಕೊಡುವ ಶಾಸ್ತ್ರಗಳನ್ನು ರಚಿಸಬಹುದಾಗಿತ್ತು. ಆದರೆ ದೈವಕೃಪೆಯಿಂದ ಅವರು ಪ್ರಾರಂಭದಲ್ಲೇ, ಅತಿ ಮಹತ್ತಾದ, ಅತ್ಯುನ್ನತವಾದ, ಅತಿ ಮಂಗಳದಾಯಕವಾದ ಶಾಂತಿದಾಯಕವಾದ ಬೇರೊಂದು ಮಾರ್ಗವನ್ನು ಹಿಡಿದರು. ಈ ಮಾರ್ಗವು ನಮ್ಮ ರಾಷ್ಟ್ರ ಲಕ್ಷಣವಾಗಿದೆ. ನಮ್ಮಲ್ಲಿ ಈ ಗುಣವು ಸಾವಿರಾರು ವರ್ಷಗಳಿಂದಲೂ ವಂಶಪಾರಂಪರ್ಯವಾಗಿ ಬರುತ್ತಿದೆ. ಅದು ನಮ್ಮ ಮುಖ್ಯಾಂಶವಾಗಿ ಬಿಟ್ಟಿದೆ. ನಮ್ಮ ರಕ್ತನಾಳಗಳಲ್ಲಿ ಹರಿಯುವ ಒಂದೊಂದು ಬಿಂದುವಿನಲ್ಲಿಯೂ ಕಂಪಿಸತೊಡಗಿದೆ. ನಮ್ಮ ಎರಡನೆಯ ಸ್ವಭಾವವಾಗಿ ಪರಿಣಮಿಸಿದೆ. ಧರ್ಮ ಮತ್ತು ಹಿಂದೂ ಎಂಬೀ ಎರಡು ಶಬ್ದಗಳೂ ಒಂದೇ ಅರ್ಥವನ್ನು ಕೊಡುವಂತೆ ಮಾಡಿದೆ. ಇದೇ ಜನಾಂಗದ ಜೀವಕಳೆ. ಇದನ್ನು ಯಾರೂ ಮುಟ್ಟಬಾರದು. ಕೋವಿಯನ್ನು ಮತ್ತು ಕತ್ತಿಯನ್ನೂ, ತಮ್ಮ ಅನಾಗರಿಕ ಧರ್ಮಗಳನ್ನೂ ತಂದ ಬರ್ಬರರು ರಾಷ್ಟ್ರದ ‘ಈ ಸಾರ ವಸ್ತುವನ್ನು’, ಈ ‘ರತ್ನ’ ವನ್ನು ಮುಟ್ಟಲಾಗಲಿಲ್ಲ; ಜನರ ಜೀವಶಕ್ತಿಯನ್ನು ರಕ್ಷಿಸಿ ಇಟ್ಟುಕೊಂಡಿರುವ ಈ ‘ಪಕ್ಷಿ’ ಯನ್ನು ಕೊಲ್ಲಲು ಅವರು ಸಮರ್ಥರಾಗಲಿಲ್ಲ; ಇದೇ ಜನಾಂಗದ ಸತ್ವ. ಈ ಚೈತನ್ಯವಿರುವ ತನಕ ನಮ್ಮ ಜನಾಂಗವನ್ನು ಕೊಲ್ಲುವ ಶಕ್ತಿ ಯಾವುದೂ ಈ ಜಗತ್ತಿನಲ್ಲಿಲ್ಲ. ಈ ಪ್ರಪಂಚದ ಚಿತ್ರಹಿಂಸೆ, ದುಃಖ, ನಮ್ಮನ್ನು ಅಲುಗಿಸದೆ ನಮ್ಮ ಮೇಲೆ ಹಾದು ಹೋಗುವುವು. ನಮಗೆ ಹಿಂದಿನಿಂದ ಬಂದ ಆಸ್ತಿಗಳಲ್ಲಿ ಸರ್ವೋತ್ತಮ ಆಸ್ತಿಯಾದ ಆತ್ಮಸಾಕ್ಷಾತ್ಕಾರ ವಿದ್ಯೆಯನ್ನು ನಾವು ಎಲ್ಲಿಯವರೆಗೆ ಬಿಗಿಯಾಗಿ ಹಿಡಿದಿರುವೆವೋ ಅಲ್ಲಿಯವರೆಗೆ, ನಾವು ಪ್ರಹ್ಲಾದನಂತೆ, ಈ ಹಿಂಸೆ, ದುಃಖಗಳೆಂಬ ಜ್ವಾಲೆಯಿಂದ ಸುಖವಾಗಿ ಪಾರಾಗುವೆವು. ಪಾರಮಾರ್ಥಿಕ ದೃಷ್ಟಿಯಿಲ್ಲದ ಹಿಂದೂವನ್ನು ನಿಜವಾಗಿ ಹಿಂದೂವೆಂದು ನಾನು ಒಪ್ಪುವುದಿಲ್ಲ. ಇತರ ದೇಶಗಳಲ್ಲಿ ಜನರು ಮೊದಲು ರಾಜಕಾರ್ಯಗಳಲ್ಲಿ ಭಾಗವಹಿಸಿ ಅನಂತರ ಧಾರ್ಮಿಕ ವಿಚಾರದಲ್ಲಿ ಸ್ವಲ್ಪ ಶ್ರದ್ಧೆಯನ್ನು ತೋರಬಹುದು. ಆದರೆ ನಮ್ಮ ದೇಶದಲ್ಲಿ ನಮ್ಮ ಜೀವನದ ಪ್ರಥಮ ಕರ್ತವ್ಯ ಪಾರಮಾರ್ಥಿಕ ದೃಷ್ಟಿಯನ್ನು ಪಡೆಯುವುದು. ಅನಂತರ ಸಮಯ ದೊರೆತರೆ ಇತರ ವಿಷಯಗಳಿಗೆ ಗಮನ ಕೊಡಬಹುದು. ಈ ವಿಷಯವನ್ನು ನಾವು ಗ್ರಹಿಸಿದ್ದೇ ಆದರೆ, ನಮ್ಮ ಪೂರ್ವಿಕರು ಏಕೆ ಜನಾಂಗದ ಏಳಿಗೆಗಾಗಿ ತಮ್ಮ ಆಧ್ಯಾತ್ಮಿಕ ಶಕ್ತಿಯನ್ನೆಲ್ಲಾ ಸಂಗ್ರಹಿಸಿಟ್ಟುಕೊಳ್ಳುತ್ತಿದ್ದರೆಂಬುದೂ, ನಾವು ನಮ್ಮ ದೇಶದ ಏಳಿಗೆಯನ್ನು ಬಯಸುವುದಾದರೆ ಹೀಗೆಯೇ ಮಾಡಬೇಕು ಎಂಬುದೂ, ನಮ್ಮ ಪುತ್ರ ಪೌತ್ರರೂ ಹೀಗೆಯೇ ನಡೆಸಿಕೊಂಡು ಬರಬೇಕು ಎಂಬುದೂ, ಸ್ಪಷ್ಟವಾಗುತ್ತದೆ. ನಮ್ಮಲ್ಲಿ ಚದುರಿಹೋಗಿರುವ ಆಧ್ಯಾತ್ಮಿಕ ಶಕ್ತಿಗಳನ್ನು ಕೇಂದ್ರೀಕರಿಸುವುದರಿಂದ ರಾಷ್ಟ್ರೀಯ ಏಕತೆಯು ಸಾಧ್ಯವಾಗುತ್ತದೆ. ಯಾರ ಹೃದಯಗಳು ಒಂದೇ ತೆರನಾದ ಆಧ್ಯಾತ್ಮಿಕ ಶ್ರುತಿಗೆ ಸರಿಯಾಗಿ ಮಿಡಿಯುತ್ತಿವೆಯೋ ಅಂಥವರ ಏಕತೆಯಿಂದ ಮಾತ್ರವೇ ನಮ್ಮ ರಾಷ್ಟ್ರವು ನಿರ್ಮಾಣವಾಗುವುದು ಸಾಧ್ಯ.

ನಮ್ಮ ದೇಶದಲ್ಲಿ ಪಂಥಗಳು ಹೇರಳವಾಗಿವೆ. ಇವು ಈಗಲೂ ಇವೆ, ಮುಂದೆಯೂ ಇರುತ್ತವೆ. ಏಕೆಂದರೆ ಇದೊಂದು ನಮ್ಮ ಧರ್ಮದ ವಿಶೇಷ ಲಕ್ಷಣ. ಮೂಲತತ್ತ್ವಗಳಲ್ಲಿ ವಿಶೇಷ ಸ್ವಾತಂತ್ರ್ಯವಿದೆ. ಕ್ರಮೇಣ ಇವುಗಳ ಆಧಾರದ ಮೇಲೆ ನೂರಾರು ವಿವರಗಳು ಅಸ್ತಿತ್ವಕ್ಕೆ ಬಂದವು. ಆದರೂ ಇವೆಲ್ಲವೂ ನಮ್ಮ ನೆತ್ತಿಯ ಮೇಲಿರುವ ಆಕಾಶದಂತೆ ವಿಶಾಲವಾಗಿಯೂ ಪ್ರಕೃತಿಯಂತೆ ಶಾಶ್ವತವಾಗಿಯೂ ಇರುವ, ಮೂಲತತ್ತ್ವಗಳನ್ನು ದಿನಂಪ್ರತಿ ವ್ಯವಹಾರದಲ್ಲಿ ರೂಢಿಗೆ ತರುವುದಕ್ಕೋಸ್ಕರ ಸೃಷ್ಟಿಯಾದವು. ಆದುದರಿಂದ ಭಿನ್ನ ಭಿನ್ನ ಪಂಥ ಇರಬೇಕಾದುದು ಸಹಜ. ಆದರೆ ಯಾವುದು ಇರಬಾರದೆಂದರೆ ಮತೀಯ ಕಲಹಗಳು. ಮತಗಳಿರಬೇಕೇ ಹೊರತು ಮತೀಯತೆ ಇರಬಾರದು. ಮತಗಳ ಕ್ಷುದ್ರಬುದ್ಧಿಯಿಂದ ಜಗತ್ತು ಯಾವ ರೀತಿಯೂ ಉತ್ತಮವಾಗುವುದಿಲ್ಲ. ಆದರೆ ಭಿನ್ನ ಭಿನ್ನ ಪಂಥಗಳಿಲ್ಲದೆ ಜಗತ್ತು ಮುಂದೆ ಸಾಗುವಂತಿಲ್ಲ. ಒಂದೇ ಪಂಥದವರು ಎಲ್ಲವನ್ನೂ ಕೈಗೊಂಡು ಸಾಧಿಸುವುದು ಅಸಾಧ್ಯ. ಜಗತ್ತಿನಲ್ಲಿ ಅನಂತವಾಗಿರುವ ಶಕ್ತಿ ರಾಶಿಯನ್ನು ಕೇವಲ ಕೆಲವೇ ಜನರು ನಿಯಂತ್ರಿಸುವುದು ಕಷ್ಟ. ಆದ್ದರಿಂದಲೇ, ಈ ಕರ್ಮವಿಭಾಗ – ಈ ವಿಭಿನ್ನ ಪಂಥಗಳು – ಆವಶ್ಯಕವಾದುವು. ಅಧ್ಯಾತ್ಮಶಕ್ತಿಯ ಬಳಕೆಗಾಗಿ ಭಿನ್ನಪಂಥಗಳು ಇರಲಿ. ಆದರೆ ನಮ್ಮ ಪ್ರಾಚೀನ ಗ್ರಂಥಗಳು “ಈ ಭೇದಭಾವಗಳು ನಿಜವಲ್ಲ; ಹಾಗೆ ಭೇದಭಾವಗಳಿದ್ದರೂ, ಇವುಗಳೆಲ್ಲವನ್ನೂ ಏಕತ್ವವೆಂಬ ಎಲ್ಲವನ್ನು ಸರಿಹೊಂದಿಸುವ ಸೂತ್ರ ಅಂದವಾಗಿ ಒಂದು ಗೂಡಿಸುತ್ತದೆ” ಎಂದು ಸಾರುತ್ತಿವೆ. ಹೀಗಿರುವಲ್ಲಿ ನಾವು ನಮ್ಮಲ್ಲೇ ಕಾದಾಡುವ ಅವಶ್ಯಕತೆ ಏನಿದೆ? “ಏಕಂ ಸತ್​ ವಿಪ್ರಾ ಬಹುಧಾ ವದಂತಿ” – ಸತ್ಯ ಒಂದೇ, ಜ್ಞಾನಿಗಳು ಅದನ್ನು ಅನಂತನಾಮಗಳಿಂದ ಕರೆಯುತ್ತಾರೆ ಎಂದು ನಮ್ಮ ಪುರಾತನ ಗ್ರಂಥಗಳು ಸಾರಿ ಹೇಳುತ್ತಿವೆ. ಆದುದರಿಂದ ಎಲ್ಲ ಪಂಥಗಳನ್ನೂ ಎಲ್ಲ ಕಾಲದಲ್ಲಿಯೂ ಗೌರವಿಸುತ್ತಿರುವ ಈ ಭರತಖಂಡದಲ್ಲಿ, ನಾನಾ ಪಂಥಗಳಲ್ಲಿ ಸ್ಪರ್ಧೆಗಳೂ, ಅಂತಃಕಲಹಗಳೂ, ಪರಸ್ಪರ ದ್ವೇಷಾಸೂಯೆಗಳೂ ಇದ್ದುದೇ ಆದರೆ, ಆ ಮಹನೀಯರ ವಂಶೀಯರೆಂದು ಹೆಮ್ಮೆ ಕೊಚ್ಚಿಕೊಳ್ಳುವುದು ನಮಗೆ ಅಪಮಾನಕರವಲ್ಲವೆ?

ನಮ್ಮಲ್ಲಿ ಪ್ರತಿಯೊಬ್ಬನೂ ಶೈವ, ಶಾಕ್ತ, ಇಲ್ಲವೇ ಗಾಣಪತ್ಯ ಇತ್ಯಾದಿ ಸಂಪ್ರದಾಯಗಳಲ್ಲಿ ಒಂದು ಸಂಪ್ರದಾಯಕ್ಕೆ ಸೇರಿದವನಾಗಿರುತ್ತಾನೆ. ಪುರಾತನ ವೇದಾಂತಿಗಳ ಗುಂಪಿಗೋ ಇಲ್ಲವೆ ನವೀನ ವೇದಾಂತಿಗಳ ಗುಂಪಿಗೋ ಸೇರಿರುತ್ತಾನೆ. ಹಿಂದಿನಿಂದ ಬಂದ, ಶಾಸ್ತ್ರ ಬದ್ಧವಾದ ಸಂಪ್ರದಾಯಗಳ ಅಥವಾ ಪರಿಷ್ಕೃತವಾದ ಆಧುನಿಕ ಸಂಪ್ರದಾಯಗಳ ಅನುಯಾಯಿ ಆಗಿರುತ್ತಾನೆ. ಆದರೂ ನಾನು ಭಾವಿಸಿರುವುದೇನೆಂದರೆ, ನಾವೆಲ್ಲರೂ ಕೆಲವು ಶ್ರೇಷ್ಠ ಮೂಲತತ್ತ್ವಗಳ ವಿಷಯದಲ್ಲಿ ಏಕಾಭಿಪ್ರಾಯವುಳ್ಳವರಾಗಿದ್ದೇವೆ. ಅಲ್ಲದೆ ತಾನು ಹಿಂದೂ ಎಂದು ಯಾವನು ಹೇಳಿಕೊಳ್ಳುತ್ತಾನೆಯೋ ಅವನಿಗೆ ಈ ಮೂಲತತ್ತ್ವಗಳಲ್ಲಿ ನಂಬಿಕೆ ಇದ್ದೇ ಇರಬೇಕು. ಈ ತತ್ತ್ವಗಳಿಗೆ ಭಿನ್ನ ಭಿನ್ನ ವ್ಯಾಖ್ಯಾನಗಳೂ, ಭಿನ್ನ ಭಿನ್ನ ಟೀಕೆಗಳೂ ಇರುವುದು ಸರಿ. ಈ ವ್ಯತ್ಯಾಸಗಳು ಇರುವುದು ಅವಶ್ಯಕ. ಅದಕ್ಕೆ ನಾವು ಅವಕಾಶ ಕೊಡಲೇಬೇಕು. ಏಕೆಂದರೆ ಪ್ರತಿಯೊಬ್ಬನೂ ನಮ್ಮ ಸಿದ್ಧಾಂತವನ್ನೊಪ್ಪುವಂತೆ ಬಲಾತ್ಕರಿಸುವುದು ನಮ್ಮ ಅಭಿಪ್ರಾಯವಲ್ಲ; ನಮ್ಮ ವ್ಯಾಖ್ಯಾನುಸಾರ ನಡೆದುಕೊಳ್ಳುವಂತೆಯೂ, ನಮ್ಮ ಆಚಾರ ಪದ್ಧತಿಗಳನ್ನನುಸರಿಸಿ ತನ್ನ ಜೀವನಯಾತ್ರೆಯನ್ನು ನಡೆಸುವಂತೆಯೂ, ಇನ್ನೊಬ್ಬನನ್ನು ಬಲಾತ್ಕರಿಸುವುದು ಪಾಪಕರ. ಬಹುಶಃ ಇಂದು ಇಲ್ಲಿರುವ ಮಹನೀಯರೆಲ್ಲರೂ, ವೇದಗಳು ಧರ್ಮರಹಸ್ಯಗಳನ್ನು ಬೋಧಿಸುವ ಸನಾತನ ಗ್ರಂಥಗಳೆಂಬುದನ್ನು ಒಪ್ಪುವರು. ನಾವೆಲ್ಲರೂ ಈ ಪವಿತ್ರ ಗ್ರಂಥವು ಆದ್ಯಂತರಹಿತವಾದುದೆಂದೂ, ಆದ್ಯಂತರಹಿತವಾದ ಪ್ರಕೃತಿಯ ಸಮಕಾಲೀನವಾದುದೆಂದೂ ನಂಬಿರುತ್ತೇವೆ. ಈ ಪವಿತ್ರ ಗ್ರಂಥದ ಸಮ್ಮುಖದಲ್ಲಿ ನಾನಾ ಸಂಪ್ರದಾಯಗಳ ಒಳಜಗಳಗಳೂ ಹೋರಾಟಗಳು ಮಾಯವಾಗುವುವು. ಆಧ್ಯಾತ್ಮಿಕತೆಗೆ ಸಂಬಂಧಿಸಿದ ಎಲ್ಲ ಭಿನ್ನಾಭಿಪ್ರಾಯಗಳನ್ನು ಬಗೆಹರಿಸುವ ಪರಮೋಚ್ಚ ನ್ಯಾಯಾಲಯ ಎಂದರೆ ವೇದಗಳೇ ಎಂಬುದನ್ನು ನಾವೆಲ್ಲರೂ ಒಪ್ಪಿದ್ದೇವೆ. ವೇದಗಳು ಯಾವುದೆಂಬ ವಿಚಾರದಲ್ಲಿ ನಾವು ಬೇರೆ ಅಭಿಪ್ರಾಯವುಳ್ಳರಾಗಿರಬಹುದು. ಒಂದು ಸಂಪ್ರದಾಯದವರಿಗೆ ವೇದಗಳ ಒಂದು ಭಾಗವು ಇತರ ಭಾಗಗಳಿಗಿಂತ ಹೆಚ್ಚು ಪವಿತ್ರವಾದುದೆಂದು ಕಾಣಬಹುದು. ಆದರೂ ನಾವು ಎಂದಿನವರೆಗೆ ಈ ಪೂಜಾರ್ಹವಾದ ವೇದಗಳ ಪ್ರಾಮಾಣ್ಯವನ್ನೊಪ್ಪಿ, ಆ ಕಾರಣದಿಂದ ನಾವು ಆಣ್ಣತಮ್ಮಂದಿರೆಂಬ ಭಾವನೆಯುಳ್ಳವರಾಗಿರುವೆವೋ, ಎಂದಿನವರೆಗೆ ಈ ಪೂಜಾರ್ಹವಾದ, ನಿತ್ಯ ಅದ್ಭುತವಾದ ಗ್ರಂಥಗಳಿಂದ ನಮ್ಮಲ್ಲಿ ಶ್ರೇಯಸ್ಕರವಾದ, ಪವಿತ್ರವಾದ, ಪರಿಶುದ್ಧವಾದ ವಿಷಯಗಳೇನೇನು ಇರುವುವೋ ಅವುಗಳೆಲ್ಲವೂ ಬಂದಿರುವುದೆಂದು ತಿಳಿದಿರುವೆವೋ, ಅಂದಿನವರೆಗೆ ವೇದಗಳ ವಿಚಾರವಾಗಿ ನಮ್ಮಲ್ಲಿರುವ ಭಿನ್ನ ಭಿನ್ನ ಭಾವನೆಗಳಿಂದ ನಮಗೆ ಯಾವ ತೊಂದರೆಯೂ ಆಗುವುದಿಲ್ಲ. ಆದುದರಿಂದ ಈ ವಿಷಯಗಳಲ್ಲಿ ನಮಗೆ ನಂಬಿಕೆ ಇದ್ದರೆ, ಈ ಮೂಲತತ್ತ್ವಗಳನ್ನು ಭಾರತ ದೇಶದಲ್ಲೆಲ್ಲಾ ಸಾರೋಣ. ನಾನು ಹೇಳಿದುದೆಲ್ಲ ನಿಜವಾದರೆ, ವೇದಗಳಿಗೆ ಸಹಜವಾಗಿ ಸಲ್ಲುವ, ಅವುಗಳಿಗೆ ಸಲ್ಲುವುದು ಸರಿಯೆಂದು ನಾವು ಒಪ್ಪುವ, ಶ್ರೇಷ್ಠತ್ವವನ್ನು ಕೊಡೋಣ. ಆದುದರಿಂದ ನಾವೆಲ್ಲರೂ ಒಪ್ಪಿರುವ ಮೊದಲನೆ ವಿಷಯ, ವೇದಗಳ ಪ್ರಾಮಾಣ್ಯ. ನಾವೆಲ್ಲರೂ ಒಪ್ಪುವ ಎರಡನೆಯ ವಿಷಯವೇ ದೇವರಭಾವನೆ. ಯಾರಲ್ಲಿ ಅದ್ಭುತವಾದ ಈ ವಿಶ್ವವು ಆಗಾಗ ಅಡಗಿ ಮತ್ತೆ ಮತ್ತೆ ಹೊರಬಿದ್ದು ವ್ಯಕ್ತವಾಗುವುದೋ, ಆ ವಿಶ್ವದ ಸೃಷ್ಟಿ ಸಂಹಾರಕರ್ತನಾದ ದೇವರು ಇರುವನೆಂದು ನಾವೆಲ್ಲರೂ ನಂಬುತ್ತೇವೆ. ನಮ್ಮಲ್ಲಿ ದೇವರ ಭಾವನೆಯ ವಿಷಯವಾಗಿ ಭಿನ್ನಾಭಿಪ್ರಾಯಗಳು ಇರಬಹುದು. ಕೆಲವರ ದೃಷ್ಟಿಗೆ ಭಗವಂತನು ಸಗುಣ ಸಾಕಾರನೆಂದೂ, ಮತ್ತೆ ಕೆಲವರಿಗೆ ಸಗುಣನಾದರೂ ನಿರಾಕಾರನೆಂದೂ, ಇನ್ನು ಕೆಲವರಿಗೆ ನಿರ್ಗುಣನೂ ನಿರಾಕಾರವೂ ಎಂದು ತೋರಬಹುದು. ಹೀಗಿದ್ದರೂ ಇವರೆಲ್ಲರೂ ತಮ್ಮ ತಮ್ಮ ಅಭಿಪ್ರಾಯಗಳಿಗೆ ಪ್ರಮಾಣಗಳನ್ನು ವೇದಗಳಲ್ಲೇ ತೋರಿಸುವರು. ಇಷ್ಟು ಭಿನ್ನಾಭಿಪ್ರಾಯಗಳಿದ್ದರೂ ನಾವೆಲ್ಲರೂ ಆಸ್ತಿಕರೂ ದೈವಭಕ್ತರೂ ಆಗಿರುವೆವು. ಎಲ್ಲಿಂದ ಎಲ್ಲವೂ ಉದ್ಭವಿಸಿರುವುದೋ, ಎಲ್ಲಿ ಎಲ್ಲವೂ ಜೀವಿಸುತ್ತಿರುವುದೋ, ಎಲ್ಲಿ ಎಲ್ಲವೂ ಕೊನೆಗಾಣುತ್ತದೆಯೋ, ಅಂತಹ ಅದ್ಭುತವಾದ, ಅನಂತವಾದ ಶಕ್ತಿಯೊಂದರಲ್ಲಿ ನಂಬಿಕೆಯಿಡದವರನ್ನು ಹಿಂದೂ ಎಂದು ಕರೆಯಲಾಗುವುದಿಲ್ಲ. ಈ ಮಾತು ಸತ್ಯವಾದರೆ ಈ ದೇವರ ಭಾವನೆಯನ್ನು ನಮ್ಮ ದೇಶದ ಎಲ್ಲ ಕಡೆಗಳಲ್ಲೂ ತಿಳಿಸಲು ಪ್ರಯತ್ನಿಸೋಣ. ನಿಮಗೆ ಈಶ್ವರನಲ್ಲಿ ಯಾವ ಭಾವನೆ ಇದ್ದರೂ ಚಿಂತೆ ಇಲ್ಲ. ನೀವು ಅದನ್ನೇ ಇತರರಿಗೆ ತಿಳಿಸಿ; ಈ ವಿಷಯವಾಗಿ ನಾವು ಹೋರಾಡಬೇಕಾಗಿಲ್ಲ. ಆದರೆ ಈಶ್ವರನನ್ನು ಬೋಧಿಸಿ: ಅಷ್ಟೇ ನಮಗೆ ಬೇಕಾದುದು. ಈಶ್ವರ ಭಾವನೆಗಳಲ್ಲಿ ಒಂದು ಭಾವನೆಯು ಇನ್ನೊಂದಕ್ಕಿಂತ ಉತ್ತಮವಾಗಿರಬಹುದು. ಆದರೆ ಇವುಗಳಲ್ಲಿ ಯಾವುದೂ ಕೆಟ್ಟದ್ದಲ್ಲವೆಂಬ ಅಂಶವನ್ನು ನೀವು ನೆನಪಿನಲ್ಲಿಡಬೇಕು. ಒಂದು ಶ್ರೇಷ್ಠವಾದುದು, ಇನ್ನೊಂದು ಶ್ರೇಷ್ಠತರವಾದುದು ಮತ್ತೊಂದು ಶ್ರೇಷ್ಠತಮವಾದುದು ಆಗಿರಬಹುದು. ಆದರೆ ನಮ್ಮ ಧರ್ಮದ ಸಿದ್ಧಾಂತಗಳಲ್ಲಿ ಕೆಟ್ಟದ್ದು ಎಂಬ ಶಬ್ದವೇ ಇರುವುದಿಲ್ಲ. ಆದುದರಿಂದ ಈಶ್ವರ ಭಾವವನ್ನು ಯಾವುದೇ ರೂಪದಲ್ಲಿ ಪ್ರಚಾರ ಮಾಡುವವರೆಲ್ಲರೂ ಅವನ ಅನುಗ್ರಹಕ್ಕೆ ಪಾತ್ರರಾಗಲಿ. ಅವನ ಭಾವವನ್ನು ಪ್ರಚಾರಮಾಡಿದಷ್ಟೂ ನಮ್ಮ ಜನಾಂಗಕ್ಕೆ ಶುಭ. ನಮ್ಮ ಪುತ್ರ, ಪೌತ್ರರು ಈಶ್ವರಭಾವದಲ್ಲಿ ಬೆಳೆಯಲಿ. ಈ ಭಾವನೆಯು ದರಿದ್ರರ, ಪಾಮರರ ಜೋಪಡಿಗಳನ್ನು, ಕೋಟ್ಯಧೀಶ್ವರರ, ಉತ್ತಮರ ಗೃಹಗಳನ್ನು ಪ್ರವೇಶಿಸಲಿ.

ಈ ಜಗತ್ತಿನ ಇತರ ಜನಾಂಗಗಳು, ಈ ವಿಶ್ವ ಇಷ್ಟು ಸಾವಿರ ವರ್ಷಗಳ ಹಿಂದೆ ನಿರ್ಮಿತವಾಯಿತು ಎಂದೂ, ಒಂದು ನಿರ್ದಿಷ್ಟವಾದ ದಿನ ನಿರ್ನಾಮವಾಗುತ್ತದೆ ಎಂದೂ, ಈ ವಿಶ್ವವೂ ಮಾನವ ಜೀವನವೂ ಶೂನ್ಯದಿಂದ ಆದುವೆಂದೂ ಹೇಳುವರು. ನಾವು ಯಾರೂ ಈ ಸಿದ್ಧಾಂತವನ್ನು ಒಪ್ಪುವುದಿಲ್ಲ. ಇದೇ ನಾನು ನಿಮಗೆ ತಿಳಿಸಬೇಕೆಂದಿರುವ ಮೂರನೆಯ ವಿಷಯ. ಈ ವಿಷಯವಾಗಿ ನಮ್ಮಲ್ಲಿ ಏಕಾಭಿಪ್ರಾಯವಿರುವುದು. ಪ್ರಕೃತಿಯು ಆದ್ಯಂತ ರಹಿತವಾದುದು. ಬಾಹ್ಯ ಪ್ರಪಂಚದ ಸ್ಥೂಲ ಭೂತಗಳು ಒಂದು ಅವಸ್ಥೆಯಲ್ಲಿ ಸೂಕ್ಷ್ಮಸ್ಥಿತಿಯನ್ನು ಪಡೆಯುವುವು. ಈ ಸ್ಥಿತಿಯು ಸ್ವಲ್ಪ ಕಾಲವಿರುತ್ತದೆ. ಅನಂತರ ಮರಳಿ ವ್ಯಕ್ತವಾಗಿ ಪ್ರಕೃತಿಯೆಂದು ಕರೆಯಲ್ಪಡುವ ಈ ಅನಂತವಾದ ಸುಂದರ ದೃಶ್ಯಗಳಾಗಿ ಪರಿಣಮಿಸುತ್ತವೆ. ಈ ಸೃಷ್ಟಿಚಕ್ರವು ಕಾಲ ಪ್ರಾರಂಭವಾಗುವುದಕ್ಕೆ ಮುಂಚೆಯೂ ಚಲಿಸುತ್ತಿತ್ತು ಮತ್ತು ಮುಂದೆ ಅನಂತ ಕಾಲದವರೆಗೆ ಚಲಿಸುತ್ತಿರುತ್ತದೆ.

ಮಾನವನು ಕೇವಲ ಸ್ಥೂಲ ಭೌತಿಕ ದೇಹವಲ್ಲ, ಇದರ ಹಿಂದೆ ಮನಸ್ಸೆಂದು ಕರೆಯಲ್ಪಡುವ ಸೂಕ್ಷ್ಮ ಶರೀರವಿರುವುದು. ಮಾತ್ರವಲ್ಲದೆ, ಇದಕ್ಕೂ ಹಿಂದೆ ಇದಕ್ಕಿಂತಲೂ ಶ್ರೇಷ್ಠವಾದ ಆತ್ಮನಿರುವನು ಎಂದು ಎಲ್ಲ ಹಿಂದೂಗಳು ನಂಬುತ್ತಾರೆ. ಈ ದೇಹ ಮನಸ್ಸುಗಳು ವಿಕಾರಗಳಿಗೊಳಗಾಗುವುವು. ಇವುಗಳಿಗೂ ಆಚೆ, ಇವುಗಳಿಗಿಂತಲೂ ಉತ್ತಮವಾದ ನಿರ್ವಿಕಾರ ವಸ್ತುವೊಂದು ಇದೆ. ಅದೇ ಸಾವನ್ನರಿಯದ, ತುದಿಮೊದಲಿಲ್ಲದ ಆತ್ಮ. ಬೇರೆ ಜನಾಂಗಗಳೆಲ್ಲದರ ಭಾವನೆಗಿಂತಲೂ ವಿಭಿನ್ನವಾದ ಭಾವನೆಯೊಂದು ನಮ್ಮಲ್ಲಿರುತ್ತದೆ. ಅದೇನೆಂದರೆ ಈ ಆತ್ಮನು ಒಂದಾದಮೇಲೊಂದು ದೇಹವನ್ನು ಅನೇಕ ಬಾರಿ ಹೊಂದುತ್ತಾನೆ, ಎಂದು ಹೀಗೆ ಮಾಡುವುದರ ಅವಶ್ಯಕತೆ ತೋರುವುದಿಲ್ಲವೋ ಅಂದು ಹೊಸ ದೇಹಗಳನ್ನು ತೆಗೆದುಕೊಳ್ಳುವುದನ್ನು ನಿಲ್ಲಿಸಿ ಮುಕ್ತನಾಗುತ್ತಾನೆ. ಇದನ್ನೇ ನಮ್ಮ ಶಾಸ್ತ್ರಗಳು ಸಂಸಾರವೆಂದು ಕರೆಯುವುದು. ಈ ಸಿದ್ಧಾಂತವನ್ನು ಎಲ್ಲ ಸಂಪ್ರದಾಯಸ್ಥರೂ ಒಪ್ಪಿರುವರು. ಜೀವ ಬ್ರಹ್ಮರ ಸಂಬಂಧ ಏನು ಎಂಬ ವಿಷಯದಲ್ಲಿ ಭಿನ್ನಾಭಿಪ್ರಾಯಗಳಿವೆ. ಕೆಲವು ಸಂಪ್ರದಾಯದವರು ಜೀವ ಮತ್ತು ಬ್ರಹ್ಮರು ಸದಾ ಬೇರೆ ಬೇರೆ ಎಂದೂ, ಇನ್ನು ಕೆಲವರು ಜೀವನು ಬ್ರಹ್ಮನೆಂಬ ಬೆಂಕಿಯ ಒಂದು ಕಿಡಿಯೆಂದೂ ಮತ್ತೆ ಕೆಲವರು ಜೀವಬ್ರಹ್ಮರಿಗೆ ಭೇದವೇ ಇಲ್ಲವೆಂದೂ ಹೇಳುವರು. ಆದರೂ, ಈ ಜೀವನು ಅನಂತನು, ಜನನವಿಲ್ಲದ ಕಾರಣ, ಮರಣರಹಿತನು, ಭಿನ್ನ ಭಿನ್ನ ದೇಹಗಳನ್ನು ಹೊಂದಿ, ಕಡೆಗೆ ಮನುಷ್ಯ ದೇಹವನ್ನು ಪಡೆದು ಅದರಲ್ಲಿ ಮುಕ್ತಿಯನ್ನು ಪಡೆಯುವನು ಎಂದು ನಾವೆಲ್ಲರೂ ಒಪ್ಪುತ್ತೇವೆ. ಈ ಸರ್ವಸಮ್ಮತವಾದ ವಿಷಯದಲ್ಲಿ ನಂಬಿಕೆ ನಮ್ಮೆಲ್ಲರಿಗೂ ಎಲ್ಲಿವರೆಗಿರುವುದೋ, ಅಲ್ಲಿಯವರೆಗೆ ಜೀವ ಬ್ರಹ್ಮರ ಸಂಬಂಧದಲ್ಲಿ ಭಿನ್ನಾಭಿಪ್ರಾಯಗಳಿದ್ದರೂ ಬಾಧಕವಿಲ್ಲ. ಪಾಶ್ಚಾತ್ಯವಾದುದೆಲ್ಲವನ್ನೂ ಪ್ರಾಚ್ಯವಾದುದೆಲ್ಲದರಿಂದ ಪ್ರತ್ಯೇಕಿಸುವ ಅತ್ಯಂತ ಮುಖ್ಯವಾದ ವ್ಯತ್ಯಾಸವೊಂದಿದೆ–ಪಾಶ್ಚಾತ್ಯ ಚಿಂತನೆಗಳನ್ನು ಅಧ್ಯಯನ ಮಾಡಿದ ನಿಮ್ಮಲ್ಲಿ ಕೆಲವರಿಗೆ ಇದು ವೇದ್ಯವಾಗಿರಬಹುದು. ಅದಾವುದೆಂದರೆ – ನಮ್ಮಲ್ಲಿ ಶಾಕ್ತರೂ, ಸೌರರೂ, ವೈಷ್ಣವರೂ, ಬೌದ್ಧರೂ, ಜೈನರೂ ಎಲ್ಲರೂ ಆತ್ಮನು ಸ್ವಭಾವತಃ ಶುದ್ಧನು ಮುಕ್ತನು ಸರ್ವಶಕ್ತನು ಮಂಗಳಮಯನು ಎಂದು ದೃಢವಾಗಿ ನಂಬಿರುತ್ತಾರೆ. ಇದು ಆಧ್ಯಾತ್ಮಿಕ ಕ್ಷೇತ್ರದಲ್ಲಿ ಕಂಡುಹಿಡಿಯಲ್ಪಟ್ಟ ಅತ್ಯಂತ ಅದ್ಭುತವಾದ, ಮಹೋನ್ನತವಾದ ಭಾವನೆ. ದ್ವೈತಿಗಳು ಮಾತ್ರ, ಜೀವನ ಸ್ವಾಭಾವಿಕ ಆನಂದವು ತನ್ನ ಹಿಂದಿನ ಕೆಟ್ಟ ಕರ್ಮಗಳ ದೆಸೆಯಿಂದ ಸಂಕುಚಿಸುವುದೆಂದೂ, ದೈವಕೃಪೆಯಿಂದ ಅದು ಪುನಃ ಅರಳಿ ಪರಿಪೂರ್ಣತೆಯನ್ನು ವ್ಯಕ್ತಪಡಿಸುವುದೆಂದೂ ಹೇಳುತ್ತಾರೆ. ಅದ್ವೈತಿಗಳು, ಈ ಸಂಕುಚಿಸುವುದೆಂಬ ಭಾವವೂ ಕೂಡ ಅಸತ್ಯವೆಂದೂ, ಅಜ್ಞಾನಜನ್ಯವೆಂದೂ, ಜೀವನು ಸ್ವಶಕ್ತಿಯನ್ನು ಕಳೆದುಕೊಂಡಿರುವ ಭಾವನೆಯು ಮಾಯೆಯ ಆವರಣದಿಂದುಂಟಾದುದೆಂದೂ, ವಾಸ್ತವವಾಗಿ ಅದರ ಶಕ್ತಿ ಸದಾ ಪೂರ್ಣವಾಗಿ ಪ್ರಕಾಶಿಸುವುದೆಂದೂ ಅಭಿಪ್ರಾಯ ಪಡುತ್ತಾರೆ. ಭಿನ್ನ ಭಾವನೆಗಳೆಂತಾದರೂ ಇರಲಿ, ಈ ಒಂದು ಕೇಂದ್ರ ಭಾವನೆಯು ಪಾಶ್ಚಾತ್ಯರಿಗೂ ನಮಗೂ ಯಾವ ವಿಷಯಗಳಲ್ಲಿಯೂ ಹೊಂದಿಕೆಯಾಗದಂತೆ ಮಾಡಿರುತ್ತದೆ. ಪೂರ್ವದೇಶಗಳಿಗೆ ಸೇರಿದ ನಾವು ಮಹತ್ತಾದ ಶುಭದಾಯಕವಾದ ವಿಷಯಗಳನ್ನು ನಮ್ಮ ಅಂತರಂಗದಲ್ಲಿ ಹುಡುಕುತ್ತೇವೆ. ನಾವು ದೇವತಾರ್ಚನೆಯ ಕಾಲದಲ್ಲಿ, ನಮ್ಮ ಕಣ್ಣುಗಳನ್ನು ಮುಚ್ಚಿ ಈಶ್ವರನನ್ನು ಮನಸ್ಸಿನಲ್ಲಿ ನೋಡಲು ಪ್ರಯತ್ನಿಸುತ್ತೇವೆ. ಪಾಶ್ಚಾತ್ಯನಾದರೋ, ತನ್ನ ದೇವರನ್ನು ಬಾಹ್ಯ ಪ್ರಪಂಚದಲ್ಲಿ ಹುಡಕುತ್ತಾನೆ. ಅವನ ದೃಷ್ಟಿಯಲ್ಲಿ, ಅವನ ಶಾಸ್ತ್ರಗ್ರಂಥಗಳು ಈ ಬಾಹ್ಯಪ್ರಪಂಚದಲ್ಲಿರುವ ದೇವರಿಂದ ಪ್ರೇರಿತವಾದವುಗಳು. ನಮ್ಮ ದೃಷ್ಟಿಯಲ್ಲಿ ನಮ್ಮ ಗ್ರಂಥಗಳೆಲ್ಲವೂ ಅಂತರಾತ್ಮನಿಂದ ಪ್ರೇರಿತವಾದವುಗಳು. ಅವು ಈಶ್ವರನ ಉಸಿರು, ಅವು ಉಸಿರಿನಂತೆ ಹೊರಬಂದವು. ಅವು ಮಂತ್ರದೃಷ್ಟೃಗಳಾದ ಋಷಿಗಳ ಹೃದಯದಿಂದ ಹೊರ ಚಿಮ್ಮಿದವು.

ನನ್ನ ಮಿತ್ರರೇ, ಭ್ರಾತೃಗಳೇ, ನಾವು ತಿಳಿದುಕೊಳ್ಳಬೇಕಾಗಿರುವ ಈ ಒಂದು ಮುಖ್ಯ ವಿಷಯವನ್ನು ನಿಮಗೆ ತಿಳಿಸಲು ಇಷ್ಟಪಡುತ್ತೇನೆ. ಈ ವಿಷಯದಲ್ಲಿ ನನಗೆ ಪೂರ್ಣ ವಿಶ್ವಾಸವುಂಟು, ನೀವೆಲ್ಲರೂ ಈ ವಿಷಯಕ್ಕೆ ಗಮನಕೊಡಬೇಕೆಂದು ಕೇಳಿಕೊಳ್ಳುತ್ತೇನೆ. ಯಾವನು ಅಹೋ ರಾತ್ರಿ ತಾನು ಯಾವ ಕೆಲಸಕ್ಕೂ ಬಾರದವನೆಂದು ಭಾವಿಸುವನೋ, ಅವನಿಂದ ಜಗತ್ತಿಗೆ ಯಾವ ಪ್ರಯೋಜನವೂ ಆಗುವಂತಿಲ್ಲ. ಹಗಲೂ ಇರಳೂ ಒಬ್ಬನು, ತಾನು ದುಃಖಿ, ನೀಚ, ಶೂನ್ಯನೆಂದು ಎಣಿಸಿದ್ದೇ ಆದರೆ, ಅವನು ಹಾಗೆಯೇ ಅಗುವುದರಲ್ಲಿ ಸಂದೇಹವಿಲ್ಲ. ಒಬ್ಬನು “ನಾನು ಹಾಗೆ, ನಾನು ಹೀಗೆ” ಎಂದು ಮನಃಪೂರ್ವಕ ಅಂದುಕೊಂಡರೆ ಅವನು ಹಾಗೆಯೇ ಆಗುವನು. ಹಾಗಲ್ಲದೆ ಹಗಲೂ ಇರಳೂ “ನಾನು ಹಾಗಲ್ಲ, ನಾನು ಹೀಗಲ್ಲ, ನಾನು ಕೆಲಸಕ್ಕೆ ಬಾರದವ” ಎಂದು ಧ್ಯಾನಿಸುತ್ತಿದ್ದರೆ, ಅವನು ಕೆಲಸಕ್ಕೆ ಬಾರದವನಾಗುವುದೇ ಖಂಡಿತ. ಈ ವಿಷಯವನ್ನು ನೀವು ನೆನಪಿನಲ್ಲಿಡಬೇಕು. ನಾವು ಸರ್ವಶಕ್ತನ ಪುತ್ರರೂ, ಅನಂತವಾದ, ದಿವ್ಯವಾದ ಜ್ವಾಲೆಯ ಕಿಡಿಗಳೂ ಆಗಿರುವಾಗ, ನಾವು ಕೆಲಸಕ್ಕೆ ಬಾರದವರಾಗುವುದು ಹೇಗೆ? ನಾವು ಎಲ್ಲವೂ ಆಗಿದ್ದೇವೆ; ಎಲ್ಲವನ್ನು ಸಾಧಿಸಲು ಸಿದ್ಧರಾಗಿದ್ದೇವೆ, ಎಲ್ಲವನ್ನೂ ಸಾಧಿಸಬಲ್ಲೆವು. ನನ್ನ ಅಭಿಪ್ರಾಯದಲ್ಲಿ ಮಾನವನು ಎಲ್ಲವನ್ನೂ ಸಾಧಿಸಲೇಬೇಕು. ನಮ್ಮ ಪೂರ್ವಿಕರಲ್ಲಿ ಅಂತಹ ಆತ್ಮಶ್ರದ್ಧೆಯಿತ್ತು. ಈ ಆತ್ಮಶ್ರದ್ಧೆಯೇ ಅವರಿಗೆ ಶಕ್ತಿಯನ್ನಿತ್ತು, ಅವರನ್ನು ನಾಗರಿಕತೆಯಲ್ಲಿ ಹೆಚ್ಚು ಹೆಚ್ಚು ಮುಂದುವರಿಯುವಂತೆ ಮಾಡಿತು. ನಾವು ಅಧೋಗತಿಗಿಳಿಯುತ್ತಿರುವುದೂ ನಮ್ಮಲ್ಲಿ ಕುಂದುಕೊರತೆಗಳಿರುವುದೂ ನಿಮಗೆ ನಿಜವಾಗಿ ತೋರಿದರೆ, ಇದು ಪ್ರಾರಂಭವಾದುದು ನಾವು ಆತ್ಮಶ್ರದ್ಧೆಯನ್ನು ಕಳೆದುಕೊಂಡದ್ದರಿಂದ ಎಂದು ತಿಳಿಯಿರಿ. ಆತ್ಮಶ್ರದ್ಧೆ ತಪ್ಪಿತೆಂದರೆ ದೇವರ ಶ್ರದ್ಧೆಯೂ ತಪ್ಪಿದ ಹಾಗೆಯೇ. ಅನಂತನು ಮಂಗಳದಾಯಕನು ಆದ ದೇವರು ನಿಮ್ಮಲ್ಲಿ ನಿಂತು, ನಿಮ್ಮ ಮೂಲಕ ಕೆಲಸಮಾಡುತ್ತಿರುವನೆಂದೂ, ಅವನು ಸರ್ವವ್ಯಾಪಿಯೆಂದೂ ಅಂತರ್ಯಾಮಿಯೆಂದೂ ಪ್ರತಿಯೊಂದು ಅಣುರೇಣುವಿನಲ್ಲಿಯೂ ನಮ್ಮ ದೇಹ, ಮನಸ್ಸು, ಜೀವ ಇವುಗಳೆಲ್ಲದರಲ್ಲಿಯೂ ಓತಪ್ರೋತನಾಗಿರುವನೆಂದೂ, ನಾವು ನಂಬಿರುವಾಗ, ನಾವು ಅಧೀರರಾಗುವುದು ತಾನೆ ಹೇಗೆ? ಅಗಾಧವಾದ ಸಮುದ್ರದಲ್ಲಿ ನಾನೊಂದು ಸಣ್ಣ ನೀರುಗುಳ್ಳೆಯಾಗಿಯೂ, ನೀವು ದೊಡ್ಡದೊಂದು ಅಲೆಯಾಗಿಯೂ ಇರಬಹುದು. ಆದರೇನು? ನನ್ನ ಮತ್ತು ನಿಮ್ಮ ಇಬ್ಬರ ಹಿಂದೆಯೂ ಇರುವುದು ಒಂದೇ ವಿಸ್ತಾರವಾದ ಸಾಗರ. ನಮ್ಮ ಈರ್ವರ ಹಿಂದೆಯೂ, ಜೀವಶಕ್ತಿ ಎಂಬ, ಆತ್ಮಚೈತನ್ಯವೆಂಬ ಅನಂತ ಸಾಗರವಿರುವುದು. ನೀವು ಬೆಟ್ಟದಷ್ಟು ಎತ್ತರವಿರಬಹುದು. ಆದರೂ ನಾನೂ ಕೂಡ ನಿಮ್ಮಂತೆಯೇ ಹುಟ್ಟಿದಂದಿನಿಂದಲೂ ಆ ಅನಂತಜೀವ, ಆ ಅನಂತ ಕಲ್ಯಾಣ, ಆ ಅನಂತ ಶಕ್ತಿಯೊಂದಿಗೆ ಸಂಬಂಧಿಸಿರುವೆನು. ಹಾಗಿಲ್ಲದಿದ್ದರೆ ನಾನು ಜೀವಿಸುತ್ತಿರಲಿಲ್ಲ. ಭ್ರಾತೃವರ್ಯರೇ, ನೀವು ಈ ಉದ್ಧರಿಸುವ, ಉತ್ತಮಗೊಳಿಸುವ, ಭವ್ಯವಾದ ಸಿದ್ಧಾಂತವನ್ನು ನಿಮ್ಮ ಪುತ್ರ ಪೌತ್ರರಿಗೆ, ಅವರು ಹುಟ್ಟಿದಂದಿನಿಂದಲೂ ಬೋಧಿಸಿ. ನೀವು ಅವರಿಗೆ ಅದ್ವೈತವನ್ನೇ ಹೇಳಿಕೊಡಬೇಕೆಂದು ನನ್ನ ಅಭಿಪ್ರಾಯವಲ್ಲ. ದ್ವೈತವನ್ನೋ ಇಲ್ಲವೇ ನಿಮಗೆ ಇಷ್ಟಬಂದ ಮತ್ತಾವ ಸಿದ್ಧಾಂತವನ್ನೊ ಹೇಳಿಕೊಡಿ. ಆದರೆ ಸರ್ವಸಮ್ಮತವಾದ ಈ ಅದ್ಭುತವಾದ ಆತ್ಮ ತತ್ತ್ವವನ್ನು, ಆತ್ಮನ ಪೂರ್ಣತೆಯನ್ನು ಬೋಧಿಸಿ. ತತ್ತ್ವದರ್ಶಿ ಕಪಿಲನು ಹೇಳಿರುವಂತೆ ಆತ್ಮನು ಸ್ವಭಾವತಃ ಶುದ್ಧನಾಗಿಲ್ಲದ ಪಕ್ಷದಲ್ಲಿ ಅವನು ಶುದ್ಧನಾಗುವ ಸಂಭವವೇ ಇಲ್ಲ. ಒಂದು ವೇಳೆ ಪೂರ್ಣತೆಯನ್ನು ಪಡೆದರೂ ಅದು ಮತ್ತೆ ಹೊರಟುಹೋಗುವ ಸಂಭವವಿರುವುದು. ಮನುಷ್ಯನ ಸ್ವಭಾವವೇ ಅಶುದ್ಧವಾದರೆ ಅವನು ಸದಾ ಅಶುದ್ಧನಾಗಿಯೇ ಇರಬೇಕಾಗುವುದು. ಕೆಲವು ವೇಳೆ ಶುದ್ಧನಾದರೂ ಉತ್ತರ ಮರುಕ್ಷಣದಲ್ಲಿಯೇ ಅವನ ಮೇಲೆ ತಾತ್ಕಾಲಿಕವಾಗಿ ಬಂದ ಶುದ್ಧತೆ ಕೊಚ್ಚಿಕೊಂಡು ಹೋಗಿ ನೈಜ ಗುಣವಾದ ಅಶುದ್ಧತೆ ಪುನಃ ಪ್ರಕಾಶವಾಗುತ್ತದೆ. ಆದಕಾರಣವೇ ನಮ್ಮ ತತ್ತ್ವದರ್ಶಿಗಳು, ನಮ್ಮ ನೈಜಗುಣವು ಶುದ್ಧತೆ ಮತ್ತು ಪೂರ್ಣತೆ ಎಂದೂ, ಅಶುದ್ಧತೆ ಮತ್ತು ಅಪೂರ್ಣತೆಯಲ್ಲವೆಂದೂ ಹೇಳುವರು. ನಾವಿದನ್ನು ನೆನಪಿನಲ್ಲಿಡಬೇಕು. ಒಬ್ಬ ಮಹರ್ಷಿಯು ತನ್ನ ಅವಸಾನ ಕಾಲದಲ್ಲಿ ತಾನು ಮಾಡಿದ್ದ ಧೀರ ಕೃತ್ಯಗಳನ್ನೂ, ತಾನು ಭಾವಿಸಿದ್ದ ಧೀರ ಭಾವನೆಗಳನ್ನೂ ನೆನಸುವಂತೆ ತನ್ನ ಮನಸ್ಸಿಗೆ ಆಜ್ಞಾಪಿಸಿದನಲ್ಲದೆ, ತಾನು ಮಾಡಿದ ತಪ್ಪುಗಳನ್ನೂ ತಾನು ತೋರಿಸಿದ್ದ ದೌರ್ಬಲ್ಯವನ್ನೂ ನೆನಸಿಕೊಳ್ಳುವಂತೆ ಹಾಗೆ ಹೇಳಲಿಲ್ಲ. ಈತನ ಉದಾಹರಣೆಯನ್ನು ಮನಸ್ಸಿನಲ್ಲಿಡಿ. ನಮ್ಮಲ್ಲಿ ತಪ್ಪುಗಳಿವೆ. ದೌರ್ಬಲ್ಯವಿದ್ದೇ ಇರಬೇಕು. ಇದ್ದರೇನು? ಸರ್ವದಾ ನಿಮ್ಮ ನೈಜಗುಣವನ್ನು ಸ್ಮರಿಸಿಕೊಳ್ಳಿ. ನಿಮ್ಮ ದೌರ್ಬಲ್ಯವನ್ನು ನೀಗಿಕೊಳ್ಳುವುದಕ್ಕೂ ನಿಮ್ಮ ತಪ್ಪುಗಳನ್ನು ತಿದ್ದಿಕೊಳ್ಳುವುದಕ್ಕೂ ಇರುವುದು ಇದೊಂದೇ ಮಾರ್ಗ.

ನಮ್ಮ ದೇಶದಲ್ಲಿ ಪ್ರಚಾರದಲ್ಲಿರುವ ಧಾರ್ಮಿಕ ಪಂಗಡಗಳಿಗೆಲ್ಲ ಈ ಕೆಲವು ಭಾವನೆಗಳು ಸಾಮಾನ್ಯವೆಂದು ತೋರುವುವು. ಈಗ ನಮ್ಮಲ್ಲಿ ಕೆಲವರು ನಮ್ಮ ಹಿರಿಯರ ಧರ್ಮವನ್ನೇ ನಂಬಿರುವರು, ಕೆಲವರು ಈ ಧರ್ಮವನ್ನು ತಿದ್ದಲೆಳಸುವರು. ಕೆಲವರು ಹಳೆಯ ಮಾದರಿಯ ಮತ್ತು ಕೆಲವರು ಹೊಸ ಮಾದರಿಯ ಸಂಪ್ರದಾಯಗಳನ್ನು ಅನುಸರಿಸುತ್ತಿರುವರು. ಇವರೆಲ್ಲರೂ ಕಲೆತು ಸಹೋದರ ಭಾವದಿಂದ ಈ ಸಾಮಾನ್ಯ ವಿಷಯಗಳ ಆಧಾರದ ಮೇಲೆ ನಡೆದುಕೊಳ್ಳುವುದು ಪ್ರಾಯಶಃ ಸಾಧ್ಯ. ಇವೆಲ್ಲಕ್ಕಿಂತಲೂ ಹೆಚ್ಚಾಗಿ ನಾವು ಮನಸ್ಸಿನಲ್ಲಿಡಬೇಕಾದ ವಿಷಯ ಇನ್ನೊಂದಿರುವುದು. ಈ ವಿಷಯವನ್ನು ನಾವು ಪದೇ ಪದೇ ಮರೆಯುತ್ತಿದ್ದೇವೆಂದು ಹೇಳಲು ವಿಷಾದವಾಗು ತ್ತದೆ. ಭರತಖಂಡದಲ್ಲಿ ಧರ್ಮದ ಗುರಿ ಸಾಕ್ಷಾತ್ಕಾರ, ಮುಕ್ತಿ ಸಾಧನೆ; ಮತ್ತಾವುದೂ ಧರ್ಮವಲ್ಲ. “ನೀವು ನಮ್ಮ ಮತವನ್ನು ನಂಬಿ. ನಿಮಗೆ ಯಾವ ತೊಂದರೆಯೂ ಬಾರದಂತೆ ನಮ್ಮ ದೇವರು ನೋಡಿಕೊಳ್ಳುವನು” ಎಂಬ ಅನ್ಯರ ಬೋಧನೆ ನಮ್ಮವರು ಒಪ್ಪುವ ಮಾತಲ್ಲ. ಈ ಮಾತಿನಲ್ಲಿ ನಮಗೆ ವಿಶ್ವಾಸವಿಲ್ಲ. ನಮ್ಮ ಈಗಿನ ಸ್ಥಿತಿಗೆ ನಾವೇ ಕಾರಣರು. ದೈವಕೃಪೆಯಿಂದಲೂ ಸ್ವಪ್ರಯತ್ನದಿಂದಲೂ ನಮಗೆ ನಮ್ಮ ಈಗಿನ ಸ್ಥಿತಿ ಪ್ರಾಪ್ತವಾಗಿರುವುದು. ಕೆಲವು ಸಿದ್ಧಾಂತಗಳನ್ನು ನಂಬುವುದರಿಂದ ಮಾತ್ರವೇ ಏನೂ ಪ್ರಯೋಜನವಾಗುವುದಿಲ್ಲ. ಭರತಖಂಡದ ಅಧ್ಯಾತ್ಮಗಗನದಿಂದ ಮಾತ್ರವೇ ಬಂದ ಪ್ರಬಲವಾದ ಶಬ್ದ ಯಾವುದೆಂದರೆ ಅದು ಅನುಭೂತಿ ಎಂಬುದು. “ದೇವರನ್ನು ಸಾಕ್ಷಾತ್ಕಾರ ಮಾಡಿಕೊಳ್ಳಬೇಕು” ಎಂದು ಮೇಲಿಂದ ಮೇಲೆ ಸಾರುವುದು ನಮ್ಮ ಗ್ರಂಥಗಳು ಮಾತ್ರ. ಸತ್ಯವೂ ಧೀರವೂ ಆದ ಮಾತುಗಳಿವು. ಇವುಗಳಲ್ಲಿ ಪ್ರತಿಯೊಂದು ಮಾತೂ ಪ್ರತಿಯೊಂದು ಧ್ವನಿಯೂ ಸತ್ಯವಾದುದು. ಧಾರ್ಮಿಕ ವಿಷಯಗಳನ್ನು ಕೇಳಿದರೆ ಸಾಲದು, ಅವನ್ನು ಸಾಧಿಸಿಕೊಳ್ಳಬೇಕು. ಗಿಳಿಯ ಹಾಗೆ ಕೆಲವು ಸಿದ್ಧಾಂತಗಳನ್ನು ಕಂಠಪಾಠ ಮಾಡಿದರೆ ಸಾಲದು. ಅವುಗಳನ್ನು ನಮ್ಮ ಬುದ್ಧಿ ಒಪ್ಪಿದರೂ ಸಾಲದು. ಅದು ಏನೇನೂ ಅಲ್ಲ – ಅವು ನಮ್ಮಲ್ಲಿ ಹೊಕ್ಕು ನಮ್ಮದಾಗಬೇಕು. ಹೌದು, ದೇವರಿದ್ದಾನೆಂಬುದಕ್ಕೆ ಸರ್ವೋತ್ತಮವಾದ ಪ್ರಮಾಣವಾವುದು? ಅವನನ್ನು ನಾವು ನಮ್ಮ ಬುದ್ಧಿಯಿಂದ ತಿಳಿಯಬಲ್ಲೆವೆಂಬುದಲ್ಲ; ಈಗಲೂ, ಹಿಂದೆಯೂ, ದೇವರನ್ನು ಕೆಲವರು ನೋಡಿರುವರು ಎಂಬುದು. ನಾವು ಆತ್ಮನಲ್ಲಿ ನಂಬಿರುವುದು ಏಕೆ? ಆತ್ಮನ ಇರುವಿಕೆಗೆ ಯುಕ್ತಿಪೂರ್ಣ ಪ್ರಮಾಣಗಳಿರುವುವು. ಆದರೆ ಇದರಿಂದ ನಮಗೆ ಆತ್ಮನಿರುವಿಕೆಯಲ್ಲಿ ನಂಬಿಕೆ ಹುಟ್ಟುವುದಿಲ್ಲ. ನಮ್ಮ ಭರತಭೂಮಿಯಲ್ಲಿ ಹಿಂದೆ ಸಾವಿರಾರು ಜನರು ಆತ್ಮಸಾಕ್ಷಾತ್ಕಾರ ಮಾಡಿಕೊಂಡರು. ಈಗಲೂ ಅನೇಕರು ಆತ್ಮನನ್ನು ಕಂಡಿರುವರು, ಇನ್ನು ಮುಂದೆಯೂ ಸಾವಿರಾರು ಜನರು ಆತ್ಮ ಸಾಕ್ಷಾತ್ಕಾರವನ್ನು ಪಡೆದು ಆತ್ಮನನ್ನು ನೋಡುವರು, ಎಂಬ ವಿಷಯ ನಮಗೆ ಆತ್ಮನಲ್ಲಿ ನಂಬಿಕೆಯನ್ನು ಕೊಡುವುದು. ಈಶ್ವರನನ್ನು ಕಂಡು ಆತ್ಮಸಾಕ್ಷಾತ್ಕಾರ ಮಾಡಿಕೊಳ್ಳುವವರೆಗೂ ಮನುಷ್ಯನಿಗೆ ಮುಕ್ತಿಯಿಲ್ಲ. ಆದ್ದರಿಂದ ಎಲ್ಲಕ್ಕಿಂತಲೂ ಹೆಚ್ಚಾಗಿ ಇದನ್ನು ಗ್ರಹಿಸಬೇಕು. ನಾವು ಈ ವಿಷಯವನ್ನು ಹೆಚ್ಚಾಗಿ ಗ್ರಹಿಸಿದಷ್ಟೂ, ನಮ್ಮಲ್ಲಿ ಸಂಕುಚಿತ ಮತೀಯ ಭಾವನೆಯು ಕಡಿಮೆಯಾಗುವುದು. ಏಕೆಂದರೆ ಈಶ್ವರಾನುಭೂತಿ ದೊರೆತ ಮಾನವನಿಗೆ ಮಾತ್ರ ಧರ್ಮಿಷ್ಠನೆಂಬ ಮಾತು ಸಲ್ಲುವುದು. ಅವನ ಬಂಧನಗಳು ಮಾತ್ರ ಹರಿದು ಹೋಗಿರುವುವು. ಸಂಶಯಗಳು ನಾಶವಾಗಿರುವುವು. ಅತ್ಯಂತ ಸಮೀಪನೂ ಹಾಗೂ ಅತ್ಯಂತ ದೂರನೂ ಆದ ದೇವರನ್ನು ಕಂಡವನು ಮಾತ್ರ ಕರ್ಮಫಲಗಳ ವಶನಾಗಿರುವುದಿಲ್ಲ. ಆದರೆ ಕೆಲವು ವೇಳೆ ನಾವು ಧರ್ಮವನ್ನು ಕುರಿತ ಬರಿಯ ಹರಟೆಯನ್ನೇ ನಿಜವಾದ ಧರ್ಮವೆಂದೂ, ಚಮತ್ಕಾರವಾದ ಶಬ್ದ ಜಾಲವನ್ನು ಮಹಾ ಆತ್ಮಾನುಭೂತಿಯೆಂದೂ ತಪ್ಪಾಗಿ ತಿಳಿದುಕೊಳ್ಳುವೆವು. ಈ ತಪ್ಪು ತಿಳುವಳಿಕೆಯಿಂದ ಸಂಪ್ರದಾಯಗಳೊಳಗೆ ಸ್ಪರ್ಧೆಗಳೂ ಹೋರಾಟಗಳೂ ಹುಟ್ಟುವುವು. ಹೀಗೆ ಮೋಸಹೋಗದೆ, ಪರಮಾರ್ಥ ಸಾಧನೆಯೇ ಧರ್ಮವೆಂದು ನಾವು ತಿಳಿದಕೂಡಲೇ ನಾವು ನಮ್ಮ ಮನಸ್ಸನ್ನು ಶೋಧಿಸಿ ನೋಡುತ್ತೇವೆ. ನಾವು ಧರ್ಮದ ನಿಜಾಂಶವನ್ನು ಎಷ್ಟರಮಟ್ಟಿಗೆ ನಮ್ಮದನ್ನಾಗಿ ಮಾಡಿಕೊಂಡಿರುವೆವು ಎಂಬುದನ್ನು ಗೊತ್ತು ಮಾಡಿಕೊಳ್ಳುತ್ತೇವೆ. ಆಗ ನಾವು ಕತ್ತಲಲ್ಲಿ ನಡೆದುಕೊಂಡು ಹೋಗುತ್ತಿರುವೆವೆಂಬುದನ್ನು, ಇತರರನ್ನು ಅದೇ ಕತ್ತಲಲ್ಲಿ ನಡೆಸಿಕೊಂಡು ಹೋಗುತ್ತಿದ್ದೇವೆ ಎಂಬುದನ್ನೂ ತಿಳಿದು, ನಮ್ಮ ಕ್ಷುದ್ರಬುದ್ಧಿಯನ್ನು ಬಿಟ್ಟು ಹೊಡೆದಾಟಗಳನ್ನು ನಿಲ್ಲಿಸುವೆವು. ಸಂಪ್ರದಾಯಗಳ ಕಲಹಗಳನ್ನೆಬ್ಬಿಸುವವರನ್ನು ಕಂಡರೆ ಅವರನ್ನು ಕುರಿತು, “ನೀವು ದೇವರನ್ನು, ಆತ್ಮವನ್ನು ಕಂಡಿದ್ದೀರಾ?” ಎಂದು ಕೇಳಿ, “ನೀವು ಕಂಡಿಲ್ಲವಾದರೆ ಈಶ್ವರನ ವಿಷಯವಾಗಿ ಇತರರಿಗೆ ತಿಳಿಸಲು ನಿಮಗೆ ಅಧಿಕಾರ ಇತ್ತವರಾರು? ಕುರುಡನು ಕುರುಡನಿಗೆ ದಾರಿ ತೋರಿದರೆ ಇಬ್ಬರೂ ಹಳ್ಳದಲ್ಲಿ ಬೀಳುವಂತೆ, ಅಂಧಕಾರದಲ್ಲಿ ವರ್ತಿಸುತ್ತಿರುವ ನಿಮಗೆ ನನ್ನನ್ನು ಕರೆದೊಯ್ಯುವ ಅಧಿಕಾರ ಎಲ್ಲಿಂದ ಬಂದಿತು?” ಎಂದು ಕೇಳಿ.

ಆದ್ದರಿಂದ ಇತರರಲ್ಲಿರುವ ದೋಷಗಳನ್ನು ಹುಡುಕಲು ಹೋಗುವ ಮುನ್ನ ಚೆನ್ನಾಗಿ ಆಲೋಚಿಸಿ. ಸತ್ಯವನ್ನು ತಮ್ಮ ಅಂತರಂಗದಲ್ಲಿ ಕಾಣಲು ಯಾರಾದರೂ ಪ್ರಯತ್ನಪಡುತ್ತಿದ್ದರೆ, ಅವರ ಈ ಪ್ರಯತ್ನದಲ್ಲಿ ತಮ್ಮ ಸ್ವಧರ್ಮವನ್ನನುಸರಿಸಿ ಸತ್ಯವನ್ನು ಕಾಣಲು ಅವರಿಗೆ ಅವಕಾಶಕೊಡಿ. ವಿಶಾಲವಾದ ಸ್ಪಷ್ಟವಾದ ಸತ್ಯ ಅವರಿಗೆ ಗೋಚರವಾದಾಗ, ನಮ್ಮ ಭಾರತ ಭೂಮಿಯಲ್ಲಿ ಪ್ರತಿಯೊಬ್ಬ ಋಷಿಯೂ ಸತ್ಯವನ್ನು ಕಂಡು ಶಾಂತಿಯನ್ನು ಪಡೆದ ಹಾಗೆ, ಅವರು ಆಶ್ಚರ್ಯವೂ ಅದ್ಭುತವೂ ಆದ ಶಾಂತಿಯನ್ನು ಪಡೆಯುವರು. ಆಗ ಅವರ ಹೃದಯದಿಂದ ಪ್ರೇಮಪೂರ್ಣ ನುಡಿಗಳು ಮಾತ್ರ ಹೊರಬೀಳುವುವು. ಏಕೆಂದರೆ ಅವರ ಹೃದಯವನ್ನು ಪ್ರೇಮಮಯನಾದ ಈಶ್ವರನು ಸ್ಪರ್ಶಿಸಿರುವನು. ಆಗಲೇ ಎಲ್ಲ ಮತೀಯ ಕಲಹಗಳೂ ಕೊನೆಗಾಣುವುವು. ಆಗ ಸಂಪ್ರದಾಯಗಳಿಗಾಗಲಿ, ಜನರಿಗಾಗಲಿ ಗಮನಕೊಡದೆ, ಹಿಂದೂ ಎಂಬ ಹೆಸರನ್ನು ತಾಳಿದ ಪ್ರತಿಯೊಬ್ಬನ ಮಹತ್ವವನ್ನು ತಿಳಿದು ಅವನನ್ನು ಆಲಿಂಗಿಸಿ ಅವನನ್ನು ಪ್ರೀತಿಸುತ್ತೇವೆ. ಯಾವುದೇ ದೇಶಕ್ಕೆ ಸೇರಿದವನಾಗಲಿ, ಯಾವುದೇ ಭಾಷೆಯನ್ನು ಮಾತಾನಾಡುತ್ತಿರಲಿ, ಹಿಂದೂ ಎಂಬ ಹೆಸರನ್ನು ಹೊತ್ತವನು ನಿಮಗೆ ಅತ್ಯಂತ ನಿಕಟನೂ, ಪ್ರಿಯನೂ ಆದಾಗ ಮಾತ್ರ ನೀವು ಹಿಂದೂಗಳು. ಎಂದು ಹಿಂದೂ ಎಂಬ ಹೆಸರನ್ನು ಕೇಳಿದ ತಕ್ಷಣ ನಮ್ಮಲ್ಲಿ ಶಕ್ತಿ ಉದ್ಭವಿಸುವುದೋ ಎಂದು ಆ ಹೆಸರಿನಿಂದ ಜೀವಿಸುತ್ತಿರುವ ಪ್ರತಿಯೊಬ್ಬನ ಸಂಕಟವೂ ನಿಮ್ಮ ಮನಸ್ಸನ್ನು ತಾಗಿ, ನಿಮ್ಮ ಪುತ್ರನ ಸಂಕಟ ನಿಮಗೆ ಎಷ್ಟು ವ್ಯಥೆಯನ್ನುಂಟು ಮಾಡುತ್ತದೆಯೋ ಅಷ್ಟೇ ವ್ಯಥೆಯನ್ನು ಉಂಟುಮಾಡಿದಾಗ, ನೀವು ನಿಜವಾದ ಹಿಂದೂಗಳು, ಈ ಉಪನ್ಯಾಸದ ಪ್ರಾರಂಭದಲ್ಲಿ ನಾನು ನಿಮಗೆ ಉದಾಹರಿಸಿದ ಆದರ್ಶ ಪುರುಷರಾದ ಗುರು ಗೋವಿಂದಸಿಂಗರಂತೆ, ನೀವು ಅವರಿಗಾಗಿ, ಬಂದ ಕಷ್ಟಗಳನ್ನೆಲ್ಲ ಅನುಭವಿಸಲು ಅಣಿಯಾದಾಗ ನೀವು ಹಿಂದೂಗಳು. ಅವರು ನಿಮ್ಮ ಮಾತೃ ಭೂಮಿಯಾದ ಪಂಜಾಬನ್ನು ಪೀಡಿಸುತ್ತಿದ್ದ ಶತ್ರುಗಳೊಡನೆ ಹೋರಾಡಿ, ಹಿಂದೂ ಧರ್ಮಕ್ಕಾಗಿ ತಮ್ಮ ರಕ್ತವನ್ನು ಚೆಲ್ಲಿದ್ದರು. ರಣರಂಗದಲ್ಲಿ ತಮ್ಮ ವೀರ ಕುಮಾರರು ಹತರಾದುದನ್ನು ಕಣ್ಣಾರೆ ನೋಡಿದರು. ಯಾರಿಗಾಗಿ ತಮ್ಮ ಮತ್ತು ತಮ್ಮ ಆಪ್ತೇಷ್ಟರ ರಕ್ತವನ್ನು ಬಸಿದರೋ ಅವರೂ ಕೂಡ ತಮ್ಮನ್ನು ತೊರೆಯಲು, ಗಾಯಗೊಂಡ ಸಿಂಹದಂತೆ, ಈ ಮಹಾಪುರುಷರು ರಣಭೂಮಿಯನ್ನು ಶಾಂತಿಯಿಂದ ಬಿಟ್ಟು, ತಮ್ಮನ್ನು ದೂರಮಾಡಿದ ಕೃತಘ್ನರನ್ನು ಕೊಂಚವೂ ಶಪಿಸದೆ ಶಾಂತಚಿತ್ತದಿಂದ ದಕ್ಷಿಣದೇಶಕ್ಕೆ ಬಂದು, ಇಹಲೋಕವನ್ನು ಬಿಟ್ಟರು. ಈ ದೇಶದ ಶ್ರೇಯಸ್ಸನ್ನು ನೀವು ಕೋರುವಿರಾದರೆ, ನಾನು ಹೇಳುವುದನ್ನು ಆಲಿಸಿರಿ: ನಿಮ್ಮಲ್ಲಿ ಪ್ರತಿಯೊಬ್ಬನೂ ಒಬ್ಬ ಗೋವಿಂದಸಿಂಗನಾಗಬೇಕು. ನಿಮ್ಮ ದೇಶದ ಜನರಲ್ಲಿರುವ ಸಾವಿರಾರು ಹುಳುಕುಗಳು ನಿಮ್ಮ ಕಣ್ಣಿಗೆ ಬಿದ್ದರೂ, ಅವರಲ್ಲಿ ಹರಿಯುತ್ತಿರುವ ಹಿಂದೂ ರಕ್ತದ ಕಡೆಗೆ ದೃಷ್ಟಿ ಇಡಿ. ಅವರು ನಿಮಗೆ ಎಷ್ಟು ತೊಂದರೆ ಕೊಟ್ಟರೂ, ಅವರು ನಿಮ್ಮ ಗೌರವಕ್ಕೆ ಪಾತ್ರರಾದವರಲ್ಲಿ ಮೊದಲನೆಯವರೆಂಬುದನ್ನು ಮರೆಯದಿರಿ. ಅವರಲ್ಲಿ ಪ್ರತಿಯೊಬ್ಬನೂ ನಿಮ್ಮನ್ನು ಶಪಿಸಿದರೂ ನೀವು ಪ್ರತಿಯಾಗಿ ವಿಶ್ವಾಸಪೂರ್ವಕ ಮಾತುಗಳನ್ನು ಅವರಿಗೆ ಕಳುಹಿಸಿ. ಅವರು ನಿಮ್ಮನ್ನು ದೇಶಭ್ರಷ್ಟರನ್ನಾಗಿ ಮಾಡಿದರೆ, ಆ ಕೇಸರಿ, ಆ ಪ್ರತಾಪಶಾಲಿ ಗೋವಿಂದಸಿಂಗರಂತೆ; ಮೌನವಾಗಿ ದೇಶವನ್ನು ಬಿಟ್ಟು ಹೊರಟುಹೋಗಿ. ಆಗ ನಿಮಗೆ ಹಿಂದೂ ಎಂಬ ಹೆಸರು ಸಲ್ಲುವುದು. ಇಂತಹ ಆದರ್ಶ ಸದಾ ನಿಮ್ಮ ಮುಂದಿರಲಿ. ನಿಮ್ಮ ಶತ್ರುತ್ವವನ್ನೆಲ್ಲ ಮರೆತು, ಎಲ್ಲೆಲ್ಲಿಯೂ ಶ್ರೇಷ್ಠವಾದ ಪ್ರೇಮವೆಂಬ ಪ್ರವಾಹವನ್ನು ಹರಿಸಿರಿ.

ನಮ್ಮ ದೇಶದ ಪುನರುದ್ಧಾರದ ವಿಚಾರವಾಗಿ ಅನೇಕರು ಅನೇಕ ವಿಧವಾಗಿ ಮನಬಂದಂತೆ ಮಾತನಾಡುತ್ತಾರೆ, ಆಡಲಿ. ಹುಟ್ಟಿದಂದಿನಿಂದ ಈ ಕಾರ್ಯದಲ್ಲಿ ಕೈಹಾಕಿರುವ, ಇಲ್ಲವೇ ಕೈಹಾಕಲು ಪ್ರಯತ್ನಿಸುತ್ತಿರುವ ನಾನು, ನಿಮಗೆ ಸಾರಿ ಹೇಳುವುದೇನೆಂದರೆ–ನಮ್ಮ ದೇಶದ ಪುನರುದ್ಧಾರವು ನೀವು ಆಧ್ಯಾತ್ಮಿಕತೆಯನ್ನು ಅಳವಡಿಸಿಕೊಳ್ಳುವವರೆಗೆ ಆಗಲಾರದು. ಇಷ್ಟೇ ಅಲ್ಲ; ಇಡೀ ಜಗತ್ತಿನ ಅಭ್ಯುದಯವೂ ಅದನ್ನೇ–ಎಂದರೆ ನೀವು ಆಧ್ಯಾತ್ಮಿಕತೆಯನ್ನು ಅಳವಡಿಸಿ ಕೊಳ್ಳುವುದನ್ನೇ ಅವಲಂಬಿಸಿದೆ. ಏಕೆಂದರೆ ನಾನು ನೋಡಿದ್ದರಲ್ಲಿ, ಈಗ ಪಾಶ್ಚಾತ್ಯ ನಾಗರಿಕತೆಯ ತಳಹದಿಯೇ ಕುಸಿಯುತ್ತಿದೆ. ಮರಳಿನ ತಳಹದಿಯ ಮೇಲೆ ಕಟ್ಟಿರುವ ಬೃಹತ್ತಾದ ಭವನಗಳು ಕೂಡ ಒಂದು ದಿನ ಕೆಳಗೆ ಉರುಳಲೇ ಬೇಕು. ಜಗತ್ತಿನ ಇತಿಹಾಸವೇ ಇದಕ್ಕೆ ಸಾಕ್ಷಿ. ರಾಷ್ಟ್ರಗಳಾದ ಮೇಲೆ ರಾಷ್ಟ್ರಗಳು ಹುಟ್ಟಿ, ಜಡವಾದದ ಅಸ್ತಿಭಾರದ ಮೇಲೆ ತಮ್ಮ ಮಹತ್ವವನ್ನು ಸ್ಥಾಪಿಸಿದವು, ಮನುಷ್ಯನು ಕೇವಲ ಭೌತವಸ್ತುವೆಂದು ಸಾರಿದವು. ಪಾಶ್ಚಾತ್ಯರು ಯಾವನಾದರೂ ಸತ್ತರೆ ಅವನು ತನ್ನ ಜೀವವನ್ನು ಬಿಟ್ಟನೆನ್ನುವರು. ನಾವಾದರೋ ಸತ್ತವನು ತನ್ನ ದೇಹವನ್ನು ಬಿಟ್ಟನೆಂದು ಹೇಳುವೆವು. ಮನುಷ್ಯನು ಮುಖ್ಯವಾಗಿ ಶರೀರ, ಆ ಶರೀರದಲ್ಲಿ ಜೀವವು ನೆಲಸಿದೆ ಎಂಬುದು ಪಾಶ್ಚಾತ್ಯರ ಭಾವನೆ. ಮನುಷ್ಯನು ಮುಖ್ಯವಾಗಿ ಆತ್ಮ, ಅವನು ಶರೀರವನ್ನು ಧರಿಸಿರುವನು ಎಂಬುದು ನಮ್ಮ ಭಾವನೆ. ಇದೇ ಈ ಎರಡು ನಾಗರಿಕತೆಗಳಿಗಿರುವ ಅಗಾಧವಾದ ವ್ಯತ್ಯಾಸ. ಆದ್ದರಿಂದ ದೇಹಸುಖ ಇತ್ಯಾದಿ ಮರಳಿನ ಆಧಾರದ ಮೇಲೆ ಕಟ್ಟಿದ ಅವರ ನಾಗರಿಕತೆಗಳೆಲ್ಲವೂ ಸ್ವಲ್ಪ ಕಾಲವಿದ್ದು, ಒಂದಾದಮೇಲೊಂದು ಅಳಿಸಿ ಹೋದವು. ಆದರೆ ಭಾರತಭೂಮಿಯ ಮತ್ತು ಅದರ ಪದತಳದಲ್ಲಿ ಕುಳಿತು ಕಲಿತ ಜಪಾನ್​ ಮತ್ತು ಚೀನಾ ದೇಶಗಳ ನಾಗರಿಕತೆಗಳು ಇಂದಿನವರೆಗೂ ಸಜೀವವಾಗಿವೆ. ಅಲ್ಲದೆ ಅವುಗಳು ಪುನಃ ತಲೆಯೆತ್ತುವ ಚಿಹ್ನೆಗಳು ಕಾಣಿಸುತ್ತಿವೆ. ಅವುಗಳ ಜೀವಕಳೆಯು ರಕ್ತಬೀಜನಂತೆ, ಎಷ್ಟು ಸಲ ನೆಲಕ್ಕುರುಳಿದರೂ, ಮಗದೊಮ್ಮೆ ಇನ್ನು ಅಧಿಕವಾದ ಮಹಿಮೆಯಿಂದ ಕೂಡಿ ಮತ್ತೆ ಮತ್ತೆ ಏಳುವುವು. ನಿರೀಶ್ವರವಾದದ ಆಧಾರದ ಮೇಲೆ ನಿಂತ ನಾಗರಿಕತೆಯು ಒಂದು ಸಲ ನಾಶವಾದರೆ, ಪುನಃ ಏಳುವುದೇ ಇಲ್ಲ. ಆ ಭವನವು ಒಂದು ಸಲ ಕೆಳಗೆ ಬಿದ್ದರೆ ಪುಡಿಪುಡಿಯಾಗುವುದು. ಪುನಃ ಅದನ್ನು ಸರಿಪಡಿಸಲು ಅವಕಾಶವೇ ಇರುವುದಿಲ್ಲ. ಆದ್ದರಿಂದ ತಾಳ್ಮೆಯಿರಲಿ, ಸಾವಧಾನದಿಂದ ಇರಿ, ಭವಿಷ್ಯವು ನಮಗಾಗಿ ಕಾದಿದೆ.

ದುಡುಕಬೇಡಿ ಇತರರನ್ನು ಕಣ್ಣುಮುಚ್ಚಿಕೊಂಡು ಅನುಕರಿಸಲು ಹೋಗಬೇಡಿ. ಈ ನೀತಿಯನ್ನು ನೀವು ನೆನಪಿನಲ್ಲಿಡಿ. ಅಂಧಾನುಕರಣೆ ನಾಗರಿಕತೆಯಲ್ಲ. ಒಬ್ಬ ರಾಜನ ಉಡುಗೆಗಳನ್ನು ತೊಟ್ಟುಕೊಂಡ ಮಾತ್ರಕ್ಕೆ ನಾನು ರಾಜನಾಗುವುದು ಅಸಾಧ್ಯ. ಸಿಂಹದ ಚರ್ಮವನ್ನು ಹೊದ್ದುಕೊಂಡ ಕತ್ತೆ ಎಂದಿಗೂ ಸಿಂಹವಾಗಲಾರದು. ಅಂಧ ಅನುಕರಣೆ, ವಿಚಾರಹೀನ ಹೇಡಿಯ ಅನುಕರಣೆ, ಎಂದಿಗೂ ನಮ್ಮ ಏಳಿಗೆಗೆ ಕಾರಣವಾಗಲಾರದು. ಇಷ್ಟು ಮಾತ್ರವಲ್ಲ ಅದು ನಮ್ಮ ಅಧೋ ಗತಿಯ ಹೆಗ್ಗುರುತು. ವ್ಯಕ್ತಿಯು ಯಾವಾಗ ತನ್ನನ್ನು ತಾನೇ ದ್ವೇಷಿಸುತ್ತಾನೆಯೋ, ಆಗ ಆವನನ್ನು ನಾಶಮಾಡುವ ಕೊನೆಯ ಪೆಟ್ಟು ಅವನಿಗೆ ಬಿತ್ತೆಂದು ತಿಳಿಯಿರಿ. ಯಾವನು ತನ್ನ ಪೂರ್ವಿಕರಿಂದ ತನಗೆ ಅವಮಾನವಾಯಿತೆಂದು ಖಿನ್ನಮನಸ್ಕನಾಗು ವನೊ, ಆಗ ಅವನಿಗೆ ಕೊನೆಗಾಲ ಸಮೀಪಿಸಿತೆಂದು ತಿಳಿಯಿರಿ. ಹಿಂದೂ ಜನಾಂಗದಲ್ಲಿ ನಾನೊಬ್ಬ ತೃಣಪ್ರಾಯ, ಆದರೂ ನಿಮ್ಮ ಮುಂದೆ ಧೈರ್ಯವಾಗಿ ನಿಂತಿದ್ದೇನೆ. ನಮ್ಮ ಜನಾಂಗದ ವಿಷಯದಲ್ಲಿಯೂ, ನಮ್ಮ ಪೂರ್ವಿಕರ ವಿಷಯದಲ್ಲಿಯೂ, ನನಗಿರುವ ಹೆಮ್ಮೆಯೇ ಇದಕ್ಕೆ ಕಾರಣ. ನಾನು ಹಿಂದೂ ಎಂದು ಹೇಳಿಕೊಳ್ಳುವುದೇ ನನಗೊಂದು ಹೆಮ್ಮೆ. ನಿಮ್ಮ ಸೇವಕರಲ್ಲಿ ನಾನೊಬ್ಬ ಅನರ್ಹ ಸೇವಕನಾಗಿರುವುದೇ ನನಗೊಂದು ಹೆಮ್ಮೆ. ಜಗತ್ತು ಎಂದೂ ಇನ್ನೆಲ್ಲಿಯೂ ಕಾಣದ ಮಹಾಮಹಿಮ ಋಷಿಗಳ ವಂಶಕ್ಕೆ ಸೇರಿದ ನೀವು ನನ್ನ ಬಾಂಧವರಾಗಿರುವುದೇ ನನಗೊಂದು ಹೆಮ್ಮೆ. ಆದ್ದರಿಂದ ಆತ್ಮಶ್ರದ್ಧೆ ಉಳ್ಳವರಾಗಿ, ನಿಮ್ಮ ಹಿರಿಯರಿಂದ ನಿಮಗೆ ಅಪಕೀರ್ತಿ ತಾಗಿತೆನ್ನದಿರಿ. ಅವರಿಂದ ನಿಮಗೆ ಕೀರ್ತಿ ಬಂದಿತೆಂದು ಹೆಮ್ಮೆಪಡಿ. ಆದರೆ ದಯವಿಟ್ಟು ವಿಚಾರ ಮಾಡದೆ, ಕುರುಡರಂತೆ ಇತರರನ್ನು ಅನುಕರಿಸಬೇಡಿ. ನೀವು ಅನ್ಯರ ಅಧೀನರಾದಂದಿನಿಂದ ನಿಮಗಿದ್ದ ಬಾಹ್ಯ ಸ್ವಾತಂತ್ರ್ಯ ಹೋಯಿತು. ಧಾರ್ಮಿಕ, ಆಧ್ಯಾತ್ಮಿಕ ವಿಷಯದಲ್ಲಿಯೂ ನೀವು ಅವರ ಅಜ್ಞಾಧಾರಕರಾದರೆ ನಿಮ್ಮ ಎಲ್ಲ ಸಹಜಶಕ್ತಿಗಳನ್ನು ಕಳೆದುಕೊಳ್ಳುವಿರಿ. ಶ್ರಮಪಟ್ಟು ನಿಮ್ಮಲ್ಲಿರುವ ಮಹತ್ವವನ್ನು ವ್ಯಕ್ತಗೊಳಿಸಿ, ಅನುಕರಿಸಬೇಡಿ. ಹೀಗೆಂದ ಮಾತ್ರಕ್ಕೆ ಅನ್ಯರ ಸದ್ಗುಣಗಳನ್ನು ನೀವು ಗ್ರಹಿಸಬಾರದೆಂದು ನನ್ನ ಅಭಿಪ್ರಾಯವಲ್ಲ. ಅನ್ಯರಿಂದ ನಾವು ಅನೇಕ ವಿಷಯಗಳನ್ನು ಕಲಿಯಲೇಬೇಕು. ಬೀಜವನ್ನು ನೆಲದಲ್ಲಿ ಬಿತ್ತಿ ಅದು ಚೆನ್ನಾಗಿ ಬೆಳೆಯಲು ಅತ್ಯಾವಶ್ಯಕವಾದ ಮಣ್ಣು ಗಾಳಿ ನೀರು ಅವುಗಳನ್ನು ಯಥೇಚ್ಛವಾಗಿಕೊಟ್ಟರೆ ಆ ಬಿತ್ತಿದ ಬೀಜವು ಗಿಡವಾಗಿ ಅನಂತರ ಹೆಮ್ಮರವಾಗುತ್ತದೆಯಲ್ಲದೆ, ಅದು ಮಣ್ಣು, ಗಾಳಿ, ನೀರು ಇವುಗಳಲ್ಲಿ ಯಾವುದೂ ಆಗುವುದಿಲ್ಲ. ತನಗೆ ಕೊಟ್ಟುದೆಲ್ಲವನ್ನೂ ತನ್ನಲ್ಲಿ ಐಕ್ಯಮಾಡಿಕೊಂಡು ತನ್ನ ಸ್ವಭಾವಾನುಸಾರ ಅದು ಒಂದು ದೊಡ್ಡ ಗಿಡ ಅಥವಾ ಮರವಾಗುವುದು. ನೀವು ಹೀಗೆ ಮಾಡಬೇಕು. ನೀವು ಅನ್ಯರಿಂದ ಕಲಿಯಬೇಕಾದ ವಿಷಯಗಳು ನಿಜವಾಗಿಯೂ ಹೇರಳವಾಗಿವೆ. ಅನ್ಯರಿಂದ ಕಲಿಯಲೊಲ್ಲದವನು ನಿರ್ಜೀವಿ ಎಂದು ತಿಳಿಯಿರಿ. ನಮ್ಮ ಸ್ಮೃತಿಕಾರ ಮನು ಅನ್ಯರಿಂದ ಕಲಿಯುವ ವಿಷಯದಲ್ಲಿ ಹೀಗೆ ಹೇಳಿರುವನು:

\begin{verse}
\textbf{ಆದದೀತ ಪರಾಂ ವಿದ್ಯಾಂ ಪ್ರಯತ್ನಾದವರಾದಪಿ~।}\\\textbf{ಅಂತ್ಯಾದಪಿ ಪರಂ ಧರ್ಮಂ ಸ್ತ್ರೀರತ್ನಂ ದುಷ್ಕುಲಾದಪಿ~॥}
\end{verse}

“ಹೀನಕುಲಜಳಾದರೂ ಸ್ತ್ರೀಯು ರತ್ನಪ್ರಾಯದವಳಾಗಿದ್ದರೆ ಅವಳನ್ನು ಮದುವೆಯಾಗು, ಹೀನಕುಲಜನಾದರೂ ಚಂಡಾಲನಾದರೂ ಸರಿ, ಅವನನ್ನು ಸೇವಿಸಿ ಮೋಕ್ಷ ಧರ್ಮವನ್ನು, ಮುಕ್ತಿ ಮಾರ್ಗವನ್ನು ಕಲಿ.” ಇತರರ ಸದ್ಗುಣಗಳೆಲ್ಲವನ್ನೂ ಅವರಿಂದ ಕಲಿಯಿರಿ, ಆದರೆ ಅವುಗಳನ್ನು ನಿಮ್ಮದೇ ಆದ ರೀತಿಯಲ್ಲಿ ನಿಮ್ಮವನ್ನಾಗಿ ಮಾಡಿಕೊಳ್ಳಿ. ನೀವೇ ಅನ್ಯರಾಗಬೇಡಿ. ಈ ಭಾರತೀಯ ಜೀವನಕ್ರಮದಿಂದ ದೂರವಾಗುವ ಪ್ರಯತ್ನ ಮಾಡಬೇಡಿ. ಭಾರತೀಯರೆಲ್ಲರೂ ಅನ್ಯದೇಶಿಯರ ಉಡುಗೆಯನ್ನುಟ್ಟು, ಅವರಂತೆ ಉಂಡು, ಅವರ ನಡತೆಯನ್ನು ಕಲಿತರೆ, ನಮ್ಮ ದೇಶಕ್ಕೆ ಮಂಗಳವಾಗುವುದೆಂದು ಒಂದು ಗಳಿಗೆಯೂ ಭಾವಿಸದಿರಿ. ಕೆಲವು ವರ್ಷಗಳಿಂದ ಅನುಸರಿಸಿ ಬಂದ ಅಭ್ಯಾಸವನ್ನು ಬಿಡುವುದು ಎಷ್ಟು ಕಷ್ಟ ಎಂಬುದು ಎಲ್ಲರಿಗೂ ಗೊತ್ತು. ನಿಮ್ಮ ರಕ್ತದಲ್ಲಿ ಎಷ್ಟು ಸಾವಿರ ವರ್ಷಗಳ ಅಭ್ಯಾಸ ಹರಿಯುತ್ತಿರುವುದೆಂಬುದು ಜಗದೀಶ್ವರನಿಗೆಯೇ ಗೊತ್ತು. ನಿಮ್ಮ ರಕ್ತದಲ್ಲಿ ಎಷ್ಟು ಸಾವಿರ ವರ್ಷಗಳ ರಾಷ್ಟ್ರೀಯ ವೈಶಿಷ್ಟ್ಯವು ಹರಿಯುತ್ತಿವೆಯೋ ಭಗವಂತನಿಗೇ ಗೊತ್ತು. ಸಮುದ್ರವನ್ನು ಸೇರಲು ಇನ್ನು ಸ್ವಲ್ಪ ದೂರವಿರುವ ನದಿಯು ಹಿಮಾಲಯದ ಮಂಜುಗಡ್ಡೆಗಳಿಗೆ ಹಿಂದಿರುಗಿ ಹೋಗುವುದು ಹೇಗೆ ಅಸಾಧ್ಯವೊ, ಹಾಗೆಯೇ ಶತಮಾನಗಳಿಂದ ಬಂದ ಅಭ್ಯಾಸಕ್ಕೆ ವಿರುದ್ಧವಾಗಿ ಹೋಗುವುದು ನಿಮಗೆ ಅಸಾಧ್ಯ. ಹಾಗೆ ಮಾಡಲು ಹೋದರೆ ನಾಶ ಖಂಡಿತ. ಆದ್ದರಿಂದ ರಾಷ್ಟ್ರ ಪ್ರವಾಹಕ್ಕೆ ಪಾತ್ರವನ್ನು ಕಲ್ಪಿಸಿ. ಈ ಅಗಾಧವಾದ ನದಿಗೆ ಅಡ್ಡಲಾಗಿ ಬಿದ್ದಿರುವ ಅಡಚಣೆಗಳನ್ನು ತೆಗೆದುಹಾಕಿ. ಅದರ ಮಾರ್ಗದಲ್ಲಿ ಶೇಖರವಾಗಿರುವ ಕಶ್ಮಲವನ್ನು ಗುಡಿಸಿ ಎಸೆಯಿರಿ. ಆಗ ರಾಷ್ಟ್ರವೆಂಬ ನದಿಯು ಸ್ವವೇಗದಿಂದ ತಾನಾಗಿಯೇ ಮುಂದೆ ನುಗ್ಗುತ್ತದೆ. ರಾಷ್ಟ್ರವು ಏಳಿಗೆಗೆ ಬಂದು ತನ್ನ ಜೀವನವನ್ನು ಸರಾಗವಾಗಿ ನಡೆಸಿಕೊಂಡು ಹೋಗುತ್ತದೆ.

ನಮ್ಮ ಭರತಖಂಡದಲ್ಲಿ ಅಧ್ಯಾತ್ಮವನ್ನು ಪ್ರಚಾರಮಾಡಲು ನಾವು ಅನುಸರಿಸಬೇಕಾದ ಕೆಲವು ರೀತಿ ನೀತಿಗಳನ್ನು ನಿಮಗೆ ಇಲ್ಲಿ ತಿಳಿಸಿರುತ್ತೇನೆ. ಕಾಲ ಸ್ವಲ್ಪ ಇರುವುದರಿಂದ ನಾವು ಯೋಚಿಸಬೇಕಾದ ಅನೇಕ ದೊಡ್ಡ ವಿಷಯಗಳನ್ನು ನಿಮ್ಮ ಮುಂದಿಡಲು ಅವಕಾಶವಿಲ್ಲ. ಉದಾಹರಣೆಗಾಗಿ ಅಂತಹ ವಿಷಯಗಳಲ್ಲಿ ಒಂದೆರಡನ್ನು ಸೂಚಿಸುತ್ತೇನೆ. ಜಾತಿ ಎಂಬ ಜಟಿಲವಾದ ಸಮಸ್ಯೆಯನ್ನು ನಾವು ಶೀಘ್ರದಲ್ಲಿಯೇ ಬಿಡಿಸಬೇಕಾಗಿದೆ. ನಾನು ಹುಟ್ಟಿದಂದಿನಿಂದಲೂ ಜಾತಿಯ ವಿಷಯವನ್ನು ಎಲ್ಲಾ ದೃಷ್ಟಿಗಳಿಂದಲೂ ವಿಮರ್ಶೆ ಮಾಡುತ್ತಿದ್ದೇನೆ. ಅದರ ಸ್ಥಿತಿಗಳನ್ನು ನಮ್ಮ ದೇಶದ ಎಲ್ಲ ಪ್ರಾಂತ್ಯಗಳಲ್ಲಿಯೂ ಸ್ವತಃ ನೋಡಿ ತಿಳಿದುಕೊಂಡಿದ್ದೇನೆ. ಎಲ್ಲೆಲ್ಲಿಯೂ ಎಲ್ಲ ಜಾತಿಯವರ ಸಂಗಡವೂ ಓಡಾಡಿದ್ದೇನೆ. ಆದರೂ ಅದರ ರಹಸ್ಯವೇನೆಂದು ತಿಳಿಯಲು ಕಷ್ಟವಾಗಿ ನನಗೆ ದಿಗ್ಭ್ರಾಂತಿಯಾಗಿದೆ. ಅದರ ವಿಷಯವಾಗಿ ನಾನು ವಿಚಾರಮಾಡಿದಷ್ಟೂ ನನಗೆ ದಿಗ್ಭ್ರಮೆಯೂ ಹೆಚ್ಚುತ್ತದೆ. ಆದರೂ ಈಗೀಗ ಅದರ ಗುಟ್ಟು ನನ್ನ ಅನುಭವಕ್ಕೆ ಬರುತ್ತಿದೆ. ನಾವು ಯೋಚಿಸಬೇಕಾದ ಇನ್ನೊಂದು ದೊಡ್ಡ ವಿಷಯ ಯಾವುದೆಂದರೆ ಊಟ–ಉಪಚಾರಕ್ಕೆ ಸಂಬಂಧಿಸಿದುದು. ಸಾಮಾನ್ಯವಾಗಿ ಜನರು ಅದನ್ನು ಅಲ್ಪ ವಿಷಯವೆಂದು ಭಾವಿಸುತ್ತಾರೆ. ಆದರೆ ಅದು ಸಣ್ಣ ವಿಷಯವಲ್ಲ. ಈಗ ನಾವು ಊಟ ಉಪಚಾರಗಳಲ್ಲಿ ಸಾಧಿಸುತ್ತಿರುವ ವಿಧಿನಿಷೇಧಗಳು ವಿಚಿತ್ರವೂ ಶಾಸ್ತ್ರ ವಿರುದ್ಧವೂ ಆಗಿರುವುವು. ನಾವು ನಮ್ಮ ಆಹಾರ ಪಾನೀಯಗಳು ಶುದ್ಧವಾಗಿರುವಂತೆ ನೋಡಿಕೊಳ್ಳುತ್ತಿಲ್ಲ. ನಮ್ಮ ಆಚಾರಗಳ ನಿಜಾಂಶವನ್ನು ತಿಳಿದುಕೊಳ್ಳದೆ ಕೇವಲ ಬಾಹ್ಯಾಡಂಬರಗಳಿಗೆ ಕಟ್ಟುಬಿದ್ದು ಕಷ್ಟ ಪರಂಪರೆಗಳಿಗೆ ಗುರಿಯಾಗುತ್ತಿದ್ದೇವೆ.

ಇನ್ನು ಕೆಲವು ಮಹತ್ತಾದ ವಿಷಯಗಳನ್ನು ಚರ್ಚಿಸಿ ಅವುಗಳ ವಿಚಾರದಲ್ಲಿರುವ ಸಂದೇಹಗಳನ್ನು ಯಾವ ರೀತಿಯಲ್ಲಿ ಪರಿಹರಿಸಬೇಕು, ಇವುಗಳನ್ನು ಹೇಗೆ ರೂಢಿಗೆ ತರಬೇಕು ಎಂಬುದನ್ನು ತೋರಿಸಿಕೊಡಬೇಕೆಂಬ ಆಸೆ ನನಗಿರುತ್ತದೆ. ಆದರೆ ಇಂದು ಸಭೆ ಬಹಳ ಹೊತ್ತಿನ ಮೇಲೆ ಪ್ರಾರಂಭವಾಯಿತು; ಇನ್ನೂ ಬಹಳ ಹೊತ್ತಿನ ತನಕ ನಿಮ್ಮನ್ನು ಇಲ್ಲಿ ನಿಲ್ಲಿಸಿಕೊಳ್ಳಲು ನನಗೆ ಇಷ್ಟವಿಲ್ಲ. ಆದ್ದರಿಂದ ಜಾತಿಯೇ ಮೊದಲಾದ ವಿಷಯಗಳಲ್ಲಿ ನನ್ನ ಅಭಿಪ್ರಾಯಗಳನ್ನು ಮತ್ತೆ ಯಾವಾಗಲಾದರೂ ನಿಮಗೆ ತಿಳಿಸಲು ಪ್ರಯತ್ನಿಸುತ್ತೇನೆ.

ಈಗ ಇನ್ನೊಂದು ವಿಷಯವನ್ನು ಮಾತ್ರ ನಿಮಗೆ ತಿಳಿಸಿ ಇಂದಿನ ಆಧ್ಯಾತ್ಮಿಕ ಪ್ರಸಂಗವನ್ನು ಮುಗಿಸುತ್ತೇನೆ. ನಮ್ಮ ದೇಶದಲ್ಲಿ ಬಹಳ ಕಾಲದಿಂದಲೂ ಧರ್ಮ ಸ್ತಬ್ಧವಾಗಿ ನಿಂತಿದೆ. ಅದು ಮುಂದುವರಿಯುವಂತೆ, ಚಲಿಸುವಂತೆ ಮಾಡಬೇಕು. ಶಕ್ತಿಯಿಂದ ತುಂಬಿ ತುಳುಕಾಡುವಂತೆ ಮಾಡಬೇಕು. ಅದು ಪ್ರತಿಯೊಬ್ಬನ ಜೀವನದಲ್ಲಿಯೂ ಮೂಡಬೇಕು. ಹಿಂದೆ ಇದ್ದಂತೆ ಈಗಲೂ ಈ ದೇಶದ ರಾಜರ ಅರಮನೆಗಳಲ್ಲೂ ಕಡುಬಡವರಾದ ರೈತರ ಗುಡಿಸಲುಗಳಲ್ಲೂ ಧರ್ಮವು ನೆಲಸುವಂತೆ ಮಾಡಬೇಕು. ಸರ್ವರ ಆಸ್ತಿಯಾದ, ಜನಾಂಗದ ಸರ್ವಸಾಮಾನ್ಯ ಹಕ್ಕಾದ ಈ ಧರ್ಮವು ಅನಾಯಾಸವಾಗಿ ಸರ್ವರಿಗೂ ದೊರಕುವಂತಾಗಬೇಕು. ಭಗವಂತನು ದಯಪಾಲಿಸಿರುವ ಗಾಳಿಯಂತೆ, ನಮ್ಮ ದೇಶದಲ್ಲಿ ಧರ್ಮ ಎಲ್ಲರಿಗೂ ಸುಲಭವಾಗಿಯೂ, ಅನಾಯಾಸವಾಗಿಯೂ ಸಿಕ್ಕುವ ಹಾಗಾಗಬೇಕು. ನಮ್ಮ ದೇಶದಲ್ಲಿ ನಾವು ಮಾಡಬೇಕಾದ ಕರ್ತವ್ಯ ಇದು; ಬೇರೆ ಬೇರೆ ಪಂಗಡಗಳನ್ನು ಸೃಷ್ಟಿಸಿ ನಮ್ಮ ಭೇದಭಾವಗಳ ಸಾಧನೆಗಾಗಿ, ನಮ್ಮ ನಮ್ಮಲ್ಲೇ ಕಾದಾಡುವಂತೆ ಮಾಡುವುದು ನಮ್ಮ ಕೆಲಸವಲ್ಲ. ಸಮ್ಮತವಾದ ವಿಷಯಗಳನ್ನು ಮಾತ್ರ ಎಲ್ಲರಿಗೂ ಸಾರಿ ತಿಳಿಸೋಣ. ಹೀಗೆ ಮಾಡುವುದರಿಂದ ಭೇದಭಾವಗಳು ತಾವಾಗಿಯೇ ಅಳಿಸಿ ಹೋಗುವುದಕ್ಕೆ ಅವಕಾಶವನ್ನು ಕಲ್ಪಿಸಿಕೊಟ್ಟಂತಾಗುವುದು. ನಮ್ಮ ದೇಶಬಾಂಧವರಿಗೆ ನಾನು ಬಾರಿ ಬಾರಿಗೂ ಹೇಳುವ ವಿಷಯ ಇದು. ಒಂದು ಚಿಕ್ಕಮನೆಯಲ್ಲಿ ಅನೇಕ ಶತಮಾನಗಳಿಂದ ಕತ್ತಲೆ ಕವಿದುಕೊಂಡಿರುವಂತೆ ಭಾವಿಸಿ. ಆ ಚಿಕ್ಕ ಮನೆಯೊಳಗೆ ನೀವು ಹೋಗಿ, “ಅಯ್ಯೋ, ಕತ್ತಲು, ಕತ್ತಲು” ಎಂದು ಗೊಣಗಾಡಿದರೆ, ಆ ಕತ್ತಲು ಹರಿದುಹೋಗುವುದಿಲ್ಲ. ಅಲ್ಲಿಗೆ ನೀವು ಒಂದು ದೀಪವನ್ನು ತಂದೊಡನೆಯೇ ಶತಮಾನಗಳಿಂದ ಅಲ್ಲಿದ್ದ ಕತ್ತಲೆ ತಕ್ಷಣ ಮಾಯವಾಗುವುದು. ಜನರನ್ನು ಉತ್ತಮಗೊಳಿಸಬೇಕಾದರೆ ನಾವು ಈ ಹಾದಿಯನ್ನು ಅನುಸರಿಸಬೇಕು. ಅವರಿಗೆ ಉತ್ತಮವಾದ ವಿಷಯಗಳನ್ನು ಸೂಚಿಸಿ; ಮನುಷ್ಯನ ಸಾಮರ್ಥ್ಯದಲ್ಲಿ ಅವರಿಗೆ ನಂಬಿಕೆಯನ್ನು ಹುಟ್ಟಿಸಿ; ತಾವು ನೀಚರು ಭ್ರಷ್ಟರು ಎಂಬ ಭಾವನೆಯನ್ನು ಹೋಗಲಾಡಿಸಿ. ಒಬ್ಬನು ನೀಚ ಕೃತ್ಯವನ್ನು ಮಾಡುತ್ತಿರುವಾಗಲೂ ಕೂಡ ನನಗೆ ಅವನ ನೈಜ ಗುಣದಲ್ಲಿ ನಂಬಿಕೆ ತಪ್ಪಿರುವುದಿಲ್ಲ. ನನ್ನ ನಂಬಿಕೆಯು ಮೊದಲು ಮೊದಲು ಅಷ್ಟೇನೂ ಆಶಾಜನಕವಾಗಿಲ್ಲದೆ ಇದ್ದರೂ ಅಂತ್ಯದಲ್ಲಿ ನಿಜವಾಗಿ ಆಶಾಜನಕವಾಗಿದೆ. ಮಾನವನ ಸಾಮರ್ಥ್ಯವನ್ನು ನಂಬಿ. ಪ್ರಾಜ್ಞನಲ್ಲಿಯೂ ಅಜ್ಞನಲ್ಲಿಯೂ ಒಂದೇ ಸಮನಾದ ನಂಬಿಕೆಯನ್ನಿಡಿ. ಅವನು ನಮಗೆ ದೇವತೆಯಾಗಿ ಕಂಡರೂ ಸರಿ, ಸ್ವಯಂ ಪಿಶಾಚಿಯಾಗಿ ಕಂಡರೂ ಸರಿ, ಅವನು ಮನುಷ್ಯನಾದ ಮಾತ್ರಕ್ಕೆ ಅವನ ನೈಜಗುಣವಾದ ಪೂರ್ಣತೆಯಲ್ಲಿ ನಂಬಿಕೆಯನ್ನಿಡಿ. ಮುಂದೆ ನಿಮಗೆ ಅವನಲ್ಲಿ ದೋಷಗಳು ಕಂಡುಬಂದರೆ, ಅವನು ಹಾದಿ ತಪ್ಪಿ ನಡೆದರೆ, ಅವನು ಅಸಾಧುವೂ, ನೀಚವೂ ಆದ ತತ್ತ್ವಗಳನ್ನು ಅವಲಂಬಿಸಿದರೆ, ಅವೆಲ್ಲಕ್ಕೂ ಕಾರಣ ಅವನ ನೈಜ ಸ್ವಭಾವವಲ್ಲವೆಂದೂ, ಉತ್ತಮ ಆದರ್ಶಗಳು ಅವನ ಮುಂದಿಲ್ಲದಿರುವುದೇ ಅವೆಲ್ಲಕ್ಕೂ ಕಾರಣವೆಂದೂ ತಿಳಿಯಿರಿ. ಸನ್ಮಾರ್ಗವನ್ನು ಕಾಣಲಾರದ್ದರಿಂದಲ್ಲವೇ ಅವನು ದುರ್ಮಾರ್ಗಗಳನ್ನು ಅವಲಂಬಿಸುವುದು. ಆದ್ದರಿಂದ ತಪ್ಪನ್ನು ತಿದ್ದುವ ರೀತಿ ಇದೊಂದೇ. ದುರ್ಮಾರ್ಗದಲ್ಲಿ ನಡೆಯುತ್ತಿರುವವನಿಗೆ ಸನ್ಮಾರ್ಗವನ್ನು ತೋರಿ. ಅಲ್ಲಿಗೆ ನಿಮ್ಮ ಕೆಲಸ ಕೊನೆಗಾಣುವುದು. ಅವನು ಎರಡು ಮಾರ್ಗಗಳನ್ನು ಹೋಲಿಸಿ ನೋಡಲಿ. ತನ್ನಲ್ಲಿ ಮೊದಲೇ ಇದ್ದ ಅಸತ್ಯವನ್ನೂ ನೀವು ಕೊಟ್ಟ ಸತ್ಯವನ್ನು ತನ್ನ ಮನದಲ್ಲೇ ಸರಿದೂಗಲಿ. ಆಗ ನೀವು ಅವನಿಗೆ ತಿಳಿಸಿರುವುದು ಸತ್ಯವಾಗಿದ್ದರೆ, ಅಸತ್ಯವು ಮಾಯವಾಗಲೇಬೇಕು. ಬೆಳಕು ಕತ್ತಲನ್ನು ಓಡಿಸಲೇ ಬೇಕು. ಸತ್ಯವು ಶ್ರೇಯಸ್ಸನ್ನು ಎಂದಾದರೂ ವ್ಯಕ್ತಪಡಿಸಲೇಬೇಕು. ನೀವು ದೇಶವನ್ನು ಆಧ್ಯಾತ್ಮಿಕ ವಿಷಯದಲ್ಲಿ ಸುಧಾರಿಸಲು ಇದು ಸರಿಯಾದ ಮಾರ್ಗ. ಜಗಳವಾಡುವುದರಿಂದಲೂ, ತಾವು ಹಿಡಿದಿರುವ ಮಾರ್ಗ ಸರಿಯಲ್ಲವೆಂದು ಜನರಿಗೆ ತೋರಿಸುವುದರಿಂದಲೂ ಇದು ಸಾಧ್ಯವಾಗುವುದಿಲ್ಲ. ಶ್ರೇಯಸ್ಸನ್ನು ಕೊಡುವುದು ಯಾವುದುಂಟೋ ಅದನ್ನು ಅವರಿಗೆ ತಿಳಿಸಿ, ಅವರು ಎಷ್ಟು ಆತುರದಿಂದ ಅದನ್ನು ಸ್ವೀಕರಿಸುವರೆಂಬುದು ನಿಮಗೆ ಗೋಚರವಾಗುವುದು. ಅಚ್ಯುತನೂ, ಸರ್ವಾಂತರ್ಗತನೂ ಆದ ದೈವವು ಜಾಗೃತವಾಗಿ, ವ್ಯಕ್ತವಾಗಿ, ಮಂಗಳಕರವಾದ, ಮಹತ್ತಾದ ಸರ್ವಸ್ವವನ್ನು ಸಾಧಿಸಲು ಯತ್ನಿಸುವುದು ನಿಮಗೇ ಕಾಣುವುದು.

ನಮ್ಮ ಪೂರ್ವಿಕರಿಂದ ಶಿವ, ಶಕ್ತಿ, ಗಣಪತಿಗಳೆಂಬ ಅನಂತ ನಾಮಗಳಿಂದ ಕರೆಸಿಕೊಂಡ, ಸಗುಣ, ನಿರ್ಗುಣ, ಸಾಕಾರ ನಿರಾಕಾರನೆಂದು ಪೂಜೆಗೊಂಡ \textbf{“ಏಕಂ ಸದ್ವಿಪ್ರಾ ಬಹುಧಾ ವದಂತಿ”} – “ಸತ್ಯ ಒಂದೇ, ಜ್ಞಾನಿಗಳು ಅದನ್ನು ಅನಂತನಾಮಗಳಿಂದ ಸ್ತುತಿಸುವರು” ಎಂಬ ತಿಳುವಳಿಕೆಯಿಂದ ಕೊಂಡಾಡಿಸಿ ಕೊಂಡ, ಜಗಜ್ಜನ್ಮಾದಿಕಾರಣನು ತನ್ನ ಅನಂತಪ್ರೇಮದಿಂದ ನಮ್ಮಲ್ಲಿ ಪ್ರವೇಶಿಸಲಿ, ನಮ್ಮ ಮೇಲೆ ತನ್ನ ಆಶೀರ್ವಾದದ ಮಳೆಗರೆಯಲಿ, ನಮ್ಮಲ್ಲಿ ಸೌಹಾರ್ದಭಾವವನ್ನು ಬೀರಲಿ, ನಮ್ಮನ್ನು ಸತ್ಯವ್ರತರನ್ನಾಗಿ ಮಾಡಿ ನೈಜ ಪ್ರೇಮದಿಂದ ಒಬ್ಬರಿಗೊಬ್ಬರು ನೆರವಾಗುವಂತೆ ಮಾಡಲಿ, ನಮ್ಮ ಭಾರತಭೂಮಿಯ ಧಾರ್ಮಿಕ ಜೀವನೋದ್ಧಾರದ ಈ ಮಹತ್ಕಾರ್ಯಕ್ರಮದಲ್ಲಿ ಸ್ವಂತ ಕೀರ್ತಿಗಾಗಲೀ ಸ್ವಪ್ರತಿಷ್ಠೆಗಾಗಲೀ, ಅಥವಾ ವೈಯಕ್ತಿಕಲಾಭಕ್ಕಾಗಲೀ ಮಾಡುವ ಆಶೆಯು ಲೇಶಾಂಶವೂ ನಮ್ಮಲ್ಲಿರದಂತೆ ಅನುಗ್ರಹಿಸಲಿ!

