
\chapter{ಪರಮಕುಡಿಯ ಅಭಿನಂದನೆಗೆ ಉತ್ತರ}

ರಾಮನಾಡನ್ನು ಬಿಟ್ಟ ಅನಂತರ ಸ್ವಾಮೀಜಿಯವರು ತಂಗಿದ ಮೊದಲ ಸ್ಥಳ ಪರಮಕುಡಿ. ಅಲ್ಲಿ ನೆರೆದಿದ್ದ ಭಾರಿ ಜನಸ್ತೋಮವು ಸ್ವಾಮೀಜಿಯವರಿಗೆ ಈ ಕೆಳಗಿನ ಮಾನಪತ್ರವನ್ನು ಅರ್ಪಿಸಿತು:

ಶ‍್ರೀಮತ್​ ವಿವೇಕಾನಂದ ಸ್ವಾಮೀ—

ಪರಮಕುಡಿಯ ಪೌರರಾದ ನಾವು, ಹತ್ತಿರ ಹತ್ತಿರ ನಾಲ್ಕು ವರ್ಷಗಳ ಕಾಲ ಪಾಶ್ಚಾತ್ಯ ಜಗತ್ತಿನಲ್ಲಿ ಆಧ್ಯಾತ್ಮಿಕ ಪ್ರಚಾರವನ್ನು ಯಶಸ್ವಿಯಾಗಿ ನಡೆಸಿ ಬಂದ ಪೂಜ್ಯ ಯತಿವರ್ಯರಾದ ತಮ್ಮನ್ನು ಹೃತ್ಪೂರ್ವಕವಾಗಿ ಸ್ವಾಗತಿಸುತ್ತೇವೆ.

ತಾವು ಚಿಕಾಗೋ ನಗರದಲ್ಲಿ ನಡೆದ ಸರ್ವಧರ್ಮ ಸಮ್ಮೇಳನದಲ್ಲಿ ಭಾಗವಹಿಸಿ ಅಲ್ಲಿ ಜಗತ್ತಿನ ವಿವಿಧ ಧರ್ಮಗಳ ಪ್ರತಿನಿಧಿಗಳ ಮುಂದೆ ನಮ್ಮ ಪ್ರಾಚೀನ ಭಾರತದ ಪವಿತ್ರವೂ ಆದರೆ ಗುಪ್ತವೂ ಆದ ಭಂಡಾರವನ್ನು ತೆರೆದಿಟ್ಟಿದ್ದು ಕೇವಲ ತಮ್ಮ ಪರಹಿತ ದೃಷ್ಟಿಯಿಂದ ಮಾತ್ರ. ಆ ಕಾರಣಕ್ಕಾಗಿ ನಮ್ಮ ದೇಶಬಾಂಧವರು ತೋರಿದ ಸಂತೋಷ ಮತ್ತು ಅಭಿಮಾನಗಳಲ್ಲಿ ನಾವೂ ಭಾಗಿಗಳಾಗಿದ್ದೇವೆ. ಪಶ್ಚಿಮ ಜಗತ್ತಿನ ಮೇಧಾವಿಗಳು ನಮ್ಮ ಪ್ರಾಚೀನ ಧರ್ಮದ ವಿಷಯವಾಗಿ ತಳೆದಿದ್ದ ತಪ್ಪು ಕಲ್ಪನೆಗಳನ್ನು ತಾವು ನಮ್ಮ ವೈದಿಕ ಸಾಹಿತ್ಯದಲ್ಲಿ ಹುದುಗಿರುವ ಪವಿತ್ರವಾದ ಸತ್ಯವನ್ನು ಉದಾರವಾಗಿ ವ್ಯಾಖ್ಯಾನಿಸುವುದರ ಮೂಲಕ ಹೋಗಲಾಡಿಸಿದ್ದೀರಿ. ಅಲ್ಲದೆ ಎಲ್ಲ ಕಾಲದ ವಿವಿಧ ವರ್ಗದ ಬುದ್ಧಿಜೀವಿಗಳೂ ನಮ್ಮ ಧರ್ಮದ ವಿಶ್ವವ್ಯಾಪಕತೆಯನ್ನೂ, ಅದನ್ನು ಅನುಸರಿಸಬಹುದು ಎಂಬುದನ್ನೂ ಒಪ್ಪಿಕೊಳ್ಳುವಂತೆ ಮಾಡಿದ್ದೀರಿ.

ನಮ್ಮ ನಡುವೆ ಇರುವ ತಮ್ಮ ಪಾಶ್ಚಾತ್ಯ ಶಿಷ್ಯರು ತಮ್ಮ ಧಾರ್ಮಿಕ ಬೋಧನೆಗಳನ್ನು ಪಶ್ಚಿಮದ ಜನರು ತಾತ್ತ್ವಿಕವಾಗಿ ಒಪ್ಪಿಕೊಂಡಿದ್ದಾರೆ ಎಂಬುದಕ್ಕೆ ಮಾತ್ರವಲ್ಲದೆ, ಆ ಬೋಧನೆಗಳು ಅನುಷ್ಠಾನಯೋಗ್ಯವಾದ ರೀತಿಯಲ್ಲಿ ಫಲಗಳನ್ನು ನೀಡಿವೆ ಎಂಬುದಕ್ಕೂ ಜೀವಂತ ಸಾಕ್ಷಿಯಾಗಿದ್ದಾರೆ. ಚಿತ್ತಾಕರ್ಷಕವಾದ ತಮ್ಮ ಸಾನ್ನಿಧ್ಯವು, ತಮ್ಮ ಆತ್ಮಸಾಕ್ಷಾತ್ಕಾರದಿಂದಲೂ ಆತ್ಮಸಂಯಮದಿಂದಲೂ ಲೋಕ ಗುರುಗಳಾಗಿ ಪರಿಣಮಿಸಿದ ನಮ್ಮ ಪ್ರಾಚೀನ ಋಷಿಗಳನ್ನು ನೆನಪಿಗೆ ತರುತ್ತಿದೆ.

ಕೊನೆಯದಾಗಿ ಪೂಜ್ಯಪಾದರಿಗೆ ದೀರ್ಘಾಯುಷ್ಯವನ್ನು ನೀಡಿ ಅವರು ಇಡೀ ಮಾನವತೆಯನ್ನು ಆಧ್ಯಾತ್ಮಿಕತೆಗೆ ಪರಿವರ್ತಿಸುವಂತೆ ಮಾಡಲು ಬೇಕಾದ ಶಕ್ತಿಯನ್ನು ನೀಡಲಿ ಎಂದು ದಯಾಮಯನಾದ ಭಗವಂತನನ್ನು ಪ್ರಾರ್ಥಿಸುತ್ತೇವೆ.

\begin{longtable}[r]{@{}l@{}}
ಭಕ್ತಿಪೂರ್ವಕವಾಗಿ ತಮ್ಮ ಅತ್ಯಂತ ವಿಧೇಯರಾದ \\
ಭಕ್ತರು ಮತ್ತು ಸೇವಕರು. \\
\end{longtable}

ಇದಕ್ಕೆ ಉತ್ತರವಾಗಿ ಸ್ವಾಮೀಜಿ ಹೀಗೆ ಹೇಳಿದರು:

ನೀವು ಇಷ್ಟು ಸ್ನೇಹದಿಂದ ಹೃತ್ಪೂರ್ವಕವಾಗಿ ಇತ್ತ ಸ್ವಾಗತಕ್ಕೆ ತಕ್ಕ ಧನ್ಯವಾದವನ್ನು ಅರ್ಪಿಸಲು ನನಗೆ ಅಸಾಧ್ಯ. ಜನರು ನನ್ನನ್ನು ಹೃತ್ಪೂರ್ವಕವಾಗಿ ಸ್ವಾಗತಿಸಲಿ ಅಥವಾ ದೇಶದಿಂದ ಉಚ್ಛಾಟಿಸಲಿ, ನನ್ನ ದೇಶ ಬಾಂಧವರ ಮೇಲೆ ಮತ್ತು ನನ್ನ ದೇಶದ ಮೇಲೆ ನನ್ನ ಪ್ರೀತಿ ಹಾಗೆಯೇ ಇರುವುದು. ಏಕೆಂದರೆ ಶ‍್ರೀಕೃಷ್ಣನು ಗೀತೆಯಲ್ಲಿ ಕರ್ಮಕ್ಕೋಸ್ಕರ ಕರ್ಮಮಾಡಬೇಕು, ಪ್ರೀತಿಗೋಸುಗ ಪ್ರೀತಿಸಬೇಕು ಎಂದಿದ್ದಾನಷ್ಟೆ. ನಾನು ಪಾಶ್ಚಾತ್ಯ ದೇಶದಲ್ಲಿ ಮಾಡಿದ ಕೆಲಸ ಅಲ್ಪ. ನಾನು ಮಾಡಿರುವುದಕ್ಕಿಂತ ನೂರರಷ್ಟು ಮಾಡಲಾರದವರು ಇಲ್ಲಿ ಯಾರೂ ಇಲ್ಲ. ಭರತಖಂಡದ ಅರಣ್ಯಗಳಲ್ಲಿ ಜನ್ಮವೆತ್ತ, ಮತ್ತು ಭರತಖಂಡಕ್ಕೆ ಮಾತ್ರ ಸೇರಿದ, ಅಧ್ಯಾತ್ಮದ ಮತ್ತು ತ್ಯಾಗದ ಭಾವನೆಗಳನ್ನು ಬೋಧಿಸುವುದಕ್ಕಾಗಿ ಭರತಖಂಡದಿಂದ ಪ್ರಪಂಚದ ಮೂಲೆಮೂಲೆಗಳಿಗೆ ಹೋಗುವಂತಹ ಮಹಾಋಷಿಗಳು ಮತ್ತು ವಿಭೂತಿಗಳು ಹುಟ್ಟುವ ಸಮಯವನ್ನು ನಾನು ಎದುರು ನೋಡುತ್ತಿರುವೆನು.

ಮಾನವ ಜನಾಂಗದ ಇತಿಹಾಸದಲ್ಲಿ ಒಂದು ಸಮಯ ಬರುವುದು. ಆಗ ಇಡಿಯ ದೇಶಕ್ಕೇ ಬಾಳು ಬೇಜಾರಾದಂತೆ ತೋರುವುದು, ತಮ್ಮ ಯೋಜನೆಗಳಾವುವೂ ಪ್ರಯೋಜನಕ್ಕೆ ಬಾರದಿರುವುದು ತೋರುವುದು, ಸಂಪ್ರದಾಯಬದ್ಧ ಸಂಸ್ಥೆಗಳು ಕುಸಿದುಬಿದ್ದಂತೆ ತೋರುವುದು, ತಮ್ಮ ಭವಿಷ್ಯವೆಲ್ಲ ಮಣ್ಣುಪಾಲಾಗಿ ಹೋಗುವ ಒಂದು ಸಮಯ ಬರುವುದು. ಸಮಾಜ ಜೀವನಕ್ಕೆ ಎರಡು ತಳಹದಿಗಳನ್ನು ಹಾಕಲು ಜನರು ಪ್ರಯತ್ನಗಳನ್ನು ಮಾಡಿರುವರು. ಒಂದು ಧರ್ಮವನ್ನು ಆಧರಿಸಿದ್ದು, ಮತ್ತೊಂದು ಸಾಮಾಜಿಕ ಆವಶ್ಯಕತೆ\-ಯನ್ನು; ಒಂದು ಅಧ್ಯಾತ್ಮವನ್ನು ಆಧರಿಸಿದ್ದು ಮತ್ತೊಂದು ಲೌಕಿಕ ಆವಶ್ಯಕತೆಗಳನ್ನು; ಒಂದು ಅತೀಂದ್ರಿಯ ದರ್ಶನದ ಮೇಲೆ, ಮತ್ತೊಂದು ಇಂದ್ರಿಯ ಸತ್ಯದ ಮೇಲೆ. ಒಂದು ಈ ಅಲ್ಪ ಜಗತ್ತಿನ ಮಿತಿಯನ್ನು ಮೀರಿ ನೋಡುವುದು; ಅದಕ್ಕೆ ಮತ್ತೊಂದರ ಸಹಾಯವಿಲ್ಲದೆ ಅಲ್ಲಿ ಜೀವನವನ್ನು ಪ್ರಾರಂಭಿಸುವ ಧೈರ್ಯ ಇದೆ. ಮತ್ತೊಂದು ಜಗತ್ತಿನ ಯಥಾರ್ಥ ವಸ್ತುಗಳ ಮೇಲೆ ನಿಲ್ಲುವುದರಲ್ಲೇ ತೃಪ್ತಿಯನ್ನು ಕಾಣುತ್ತದೆ. ವಿಚಿತ್ರವೆಂದರೆ ಕೆಲವು ವೇಳೆ ಅಧ್ಯಾತ್ಮವು ಮುಖ್ಯವಾದಂತೆ ತೋರುವುದು. ಮತ್ತೆ ಕೆಲವು ವೇಳೆ ಪ್ರಾಪಂಚಿಕ ದೃಷ್ಟಿ ಮುಖ್ಯವಾದಂತೆ ತೋರುವುದು-ಅಲೆಗಳಂತೆ ಒಂದಾದಮೇಲೊಂದು ಬರುವುದು. ಒಂದೇ ದೇಶದಲ್ಲಿ ಎರಡು ಬಗೆಯ ಅಲೆಗಳಿರುತ್ತವೆ. ಒಂದು ಕಾಲದಲ್ಲಿ ಜಡವಾದವು ಅತ್ಯಂತ ಮುಖ್ಯವಾಗಿರುವುದು. ಅಭಿವೃದ್ಧಿ, ಹೆಚ್ಚು ಸುಖ ಮತ್ತು ಆಹಾರ ಕೊಡುವ ವಿದ್ಯಾಭ್ಯಾಸ, ಇವೇ ಪ್ರಥಮದಲ್ಲಿ ಅತಿ ಮುಖ್ಯವಾಗಿದ್ದು ಕ್ರಮೇಣ ಅವು ಅವನತಿಗೆ ಇಳಿಯುವುವು. ಅವುಗಳ ಅಭಿವೃದ್ಧಿಯ ಜೊತೆಯಲ್ಲೇ ಮಾನವ ಜನಾಂಗದಲ್ಲಿ ಸುಪ್ತವಾಗಿರುವ ದ್ವೇಷ-ಅಸೂಯೆಗಳೆಲ್ಲ ಹೆಚ್ಚು ವ್ಯಕ್ತವಾಗುತ್ತ ಬರುವುವು. ಪೈಪೋಟಿ, ನಿರ್ದಯೆ, ಕ್ರೌರ್ಯ, ಇವೇ ಮೂಲಮಂತ್ರವಾಗುವುವು. ಒಂದು ಆಂಗ್ಲೇಯ ಗಾದೆಯನ್ನು ಹೇಳಬೇಕಾದರೆ, “ಪ್ರತಿಯೊಬ್ಬರೂ ತಮ್ಮ ತಮ್ಮನ್ನು ನೋಡಿಕೊಳ್ಳಲಿ, ಸೈತಾನನು ಹಿಂದೆ ಬಿದ್ದವನನ್ನು ತೆಗೆದು\- ಕೊಂಡು ಹೋಗಲಿ” - ಇದೇ ಜನಾಂಗದ ಪಲ್ಲವಿಯಾಗುವುದು. ಆಗ ಜೀವನಕ್ರಮವೇ ನಿಷ್ಪ್ರಯೋಜಕವೆಂದು ಜನಕ್ಕೆ ತಿಳಿಯುವುದು. ಇಂಥ ಪರಿಸ್ಥಿತಿಯಲ್ಲಿ ಅಧ್ಯಾತ್ಮವು ಸಹಾಯಕ್ಕೆ ಬಾರದೇ ಹೋದರೆ ಜಗತ್ತೇ ನಾಶಹೊಂದುತ್ತದೆ. ಕುಸಿಯುತ್ತಿರುವ ಜಗತ್ತಿಗೆ ಅಧ್ಯಾತ್ಮದ ಸಹಾಯ ದೊರೆತಾಗ ಹೊಸ ಭರವಸೆ ದೊರೆತಂತಾಗುತ್ತದೆ. ಮತ್ತೊಂದು ಧಾರ್ಮಿಕ ಅಲೆ ಏಳುವುದು, ಅದೂ ಕಾಲಕ್ರಮೇಣ ಕ್ಷೀಣಿಸುವುದು. ಅಧ್ಯಾತ್ಮದೊಂದಿಗೆ, ಜಗತ್ತಿನ ವಿಶೇಷ ಶಕ್ತಿಗಳ ಮೇಲೆ ತಮ್ಮದೇ ಆದ ಹಕ್ಕುದಾರಿಕೆಯನ್ನು ಸ್ಥಾಪಿಸುವ ಒಂದು ವರ್ಗದ ಜನರು ಪ್ರಸಿದ್ಧಿಗೆ ಬರುತ್ತಾರೆ. ಅದರ ತಕ್ಷಣದ ಪರಿಣಾಮವೇ ಜಡವಾದದ ವಿರುದ್ಧ ಪ್ರತಿಕ್ರಿಯೆ. ಅಧಿಕಾರ ಶಕ್ತಿಗಳು ತಮಗೇ ಸೇರಿವೆ ಎಂದು ಹಲವರು ಮುಂದೆ ಬರುವರು. ಕಾಲಕ್ರಮೇಣ ಆ ಜನಾಂಗದ ಅಧ್ಯಾತ್ಮ ಶಕ್ತಿ ಮಾತ್ರವಲ್ಲ, ಪ್ರಪಂಚದ ಐಶ್ವರ್ಯಭೋಗಗಳ ಹಕ್ಕುಗಳೆಲ್ಲ ಕೆಲವು ವ್ಯಕ್ತಿಗಳಲ್ಲಿ ಕೇಂದ್ರೀಕೃತವಾಗುವುವು. ಇಂತಹ ಕೆಲವು ವ್ಯಕ್ತಿಗಳು ಜನಸಮೂಹದ ಹೆಗಲಿನ ಮೇಲೆ ನಿಂತು ದಬ್ಬಾಳಿಕೆ ನಡೆಸಲು ಯತ್ನಿಸುವರು. ಆಗ ಸಮಾಜವು ತನ್ನನ್ನು ತಾನು ರಕ್ಷಿಸಿಕೊಳ್ಳಬೇಕಾಗುವುದು. ಜಡವಾದವು ಅದರ ರಕ್ಷಣೆಗೆ ಬರುವುದು.

ನಮ್ಮ ಮಾತೃಭೂಮಿ ಭರತಖಂಡವನ್ನು ನೋಡಿದರೆ ಈಗ ಹೀಗೆ ಆಗುತ್ತಿರುವುದು ಕಾಣುವುದು. ವೇದಾಂತವನ್ನು ಬೋಧಿಸುವುದಕ್ಕಾಗಿ ಪಾಶ್ಚಾತ್ಯ ದೇಶಗಳಿಗೆ ಹೋದವ\- ನೊಬ್ಬನನ್ನು ನೀವಿಂದು ಸ್ವಾಗತಿಸುತ್ತಿದ್ದೀರಿ. ಪಾಶ್ಚಾತ್ಯ ದೇಶಗಳ ಜಡವಾದವು ನಾನು ಅಲ್ಲಿಗೆ ಹೋಗಲು ಅವಕಾಶವನ್ನು ಕಲ್ಪಿಸದೇ ಇದ್ದಿದ್ದರೆ, ಹೀಗೆ ನೀವು ಮಾಡುವುದು ಸಾಧ್ಯವಾಗುತ್ತಿರಲಿಲ್ಲ. ಜಡವಾದವು ಒಂದು ರೀತಿಯಿಂದ ಭರತ ಖಂಡದ ಸಹಾಯಕ್ಕೆ ಬಂದಿದೆ. ಅದು ವರ್ಣಗಳ ಪ್ರತ್ಯೇಕ ಹಕ್ಕು ಬಾಧ್ಯತೆಗಳನ್ನು ಅಲ್ಲಗಳೆದು ಜೀವನವನ್ನು ಎಲ್ಲರಿಗೂ ತೆರೆದಿದೆ. ಎಲ್ಲೋ ಕೆಲವರಲ್ಲಿ ಅನರ್ಘ್ಯ ತತ್ತ್ವರತ್ನಗಳು ಉಳಿದಿದ್ದುವು. ಅವರು ಕೂಡ ಅವುಗಳ ಉಪಯೋಗವನ್ನು ಮರೆಯುತ್ತಿದ್ದರು. ಅದನ್ನು ಇಂದು ಎಲ್ಲರೂ ಚರ್ಚಿಸಬಲ್ಲ ವಿಷಯವನ್ನಾಗಿ ಮಾಡಿದ್ದಾರೆ. ಅರ್ಧ ನಾಶವಾಗಿದೆ, ಉಳಿದರ್ಧ ತಮಗೆ ಪ್ರಯೋಜನವಿಲ್ಲದೆ ಇದ್ದರೂ ಇತರರಿಗೆ ಅದನ್ನು ಕೊಡದೆ ಗೊಂತಿನಲ್ಲಿರುವ ನಾಯಿಯಂತಹ ಜನರ ಕೈಯಲ್ಲಿರುವುದು. ಅವು ಪಾಶ್ಚಾತ್ಯ ದೇಶಗಳಲ್ಲಿ ಹಿಂದಿನಿಂದಲೂ ಜಾರಿಯಲ್ಲಿವೆ. ಅವುಗಳಲ್ಲಿ ಎಷ್ಟೋ ಕುಂದುಕೊರತೆಗಳು ಕಾಣುತ್ತಿವೆ. ರಾಜಕೀಯಕ್ಕೆ ಸಂಬಂಧಪಟ್ಟ ಸಂಸ್ಥೆಗಳು ಮತ್ತು ರೀತಿನೀತಿಗಳೆಲ್ಲ ಪ್ರಯೋಜನವಿಲ್ಲವೆಂದು ಸಾರಿ ತ್ಯಜಿಸಲ್ಪಟ್ಟಿವೆ. ಯೂರೋಪ್​ ಅಶಾಂತವಾಗಿ ಈಗ ಏನು ಮಾಡಬೇಕು, ಎತ್ತ ತಿರುಗಬೇಕು ಎಂದು ತಿಳಿಯದೆ ಪರಿತಪಿಸುತ್ತಿದೆ. ಐಶ್ವರ್ಯ, ಸಂಪತ್ತುಗಳ ಅತ್ಯಾಚಾರ ಅಸಹ್ಯಕರ. ಕೆಲಸಮಾಡದ ಕೆಲವರ ಕೈಯಲ್ಲಿ ದೇಶದ ಐಶ್ವರ್ಯ ಮತ್ತು ಅಧಿಕಾರವೆಲ್ಲ ಕೇಂದ್ರೀಕೃತವಾಗಿದೆ. ಅವರು ದುಡಿಯುವ ಲಕ್ಷಾಂತರ ಮಾನವರ ಮೇಲೆ ತಮ್ಮ ಅಧಿಕಾರವನ್ನು ಚಲಾಯಿಸುವರು. ಈ ಅಧಿಕಾರದ ಬಲದಿಂದ ಅವರು ಜಗತ್ತನ್ನು ರಕ್ತದಲ್ಲಿತೋಯಿಸುವರು. ಧರ್ಮ ಮುಂತಾದುವೆಲ್ಲ ಅವರ ಅಧೀನ. ಅವರೇ ಆಳುವವರು, ಅವರೇ ಸರ್ವಾಧಿಕಾರಿಗಳು. ಪಾಶ್ಚಾತ್ಯ ಪ್ರಪಂಚವನ್ನು ಕೆಲವು ಶ‍್ರೀಮಂತ ವರ್ತಕರು ಆಳುವರು. ಅವರ ಸಂವಿಧಾನಬದ್ಧ ಶಾಸನ, ಸ್ವಾತಂತ್ರ್ಯ, ಪಾರ್ಲಿಮೆಂಟ್​ ಮುಂತಾದುವೆಲ್ಲ ಹಾಸ್ಯಾಸ್ಪದವಾಗಿದೆ.

ಪಾಶ್ಚಾತ್ಯರು ವರ್ತಕರ ದಬ್ಬಾಳಿಕೆಯಲ್ಲಿ ನರಳುತ್ತಿರುವರು. ಪ್ರಾಚ್ಯರು ಪುರೋಹಿತರ ದಬ್ಬಾಳಿಕೆಯಲ್ಲಿ ನರಳುತ್ತಿರುವರು. ಪ್ರತಿಯೊಬ್ಬರೂ ಮತ್ತೊಬ್ಬರನ್ನು ತಡೆಯಬೇಕಾಗಿದೆ. ಇದರಲ್ಲಿ ಯಾರೋ ಒಬ್ಬರು ಮಾತ್ರ ಪ್ರಪಂಚಕ್ಕೆ ಸಹಾಯ ಮಾಡುತ್ತಾರೆ ಎಂಬುದಾಗಿ ಭಾವಿಸಬೇಡಿ. ನಿಷ್ಪಕ್ಷಪಾತಿಯಾದ ದೇವರು ಪ್ರಪಂಚದಲ್ಲಿ ಪ್ರತಿಯೊಂದನ್ನೂ ಸರಿಸಮವಾಗಿ ಸೃಷ್ಟಿಸಿರುವನು. ಅತಿ ಹೀನ ಆಸುರೀ ವೃತ್ತಿಯವನಲ್ಲಿ, ಒಬ್ಬ ಮಹಾತ್ಮ\-ನಲ್ಲಿಯೂ ಇಲ್ಲದ ಕೆಲವು ಸುಗುಣಗಳಿರುತ್ತವೆ. ಅತಿ ಕ್ಷುದ್ರಕೀಟದಲ್ಲಿ ಅತ್ಯುತ್ತಮ ಮಾನವನಲ್ಲಿಯೂ ಇಲ್ಲದ ಸುಗುಣವಿರಬಹುದು. ಒಬ್ಬ ಕೂಲಿಯನ್ನು ತೆಗೆದುಕೊಳ್ಳಿ. ಅವನ ಪಾಲಿಗೆ ದೊರಕುವ ಸುಖ ಅತ್ಯಲ್ಪ. ಅವನಿಗೆ ನಿಮ್ಮಷ್ಟು ಬುದ್ಧಿಯಿಲ್ಲ. ಅವನ ವೇದಾಂತವನ್ನು ತಿಳಿದುಕೊಳ್ಳಲಾರ. ಆದರೆ ನಿಮ್ಮ ದೇಹದೊಂದಿಗೆ ಅವನ ದೇಹವನ್ನು ಹೋಲಿಸಿ ನೋಡಿ. ನಿಮ್ಮ ದೇಹದಷ್ಟು ಸೂಕ್ಷ್ಮವಲ್ಲ ಅವನ ದೇಹ; ಅವನು ಹೆಚ್ಚು ಕಷ್ಟನ್ನು ಸಹಿಸಬಲ್ಲನೆಂದು ಕಾಣುವುದು. ಅವನಿಗೆ ಗಾಯವಾದರೆ ಅದು ನಮ್ಮ ಗಾಯಕ್ಕಿಂತ ಬೇಗ ಮಾಗುವುದು. ಅವನ ಆನಂದ ಪಂಚೇಂದ್ರಿಯಗಳಲ್ಲಿದೆ. ಅವನು ಅವುಗಳಲ್ಲಿ ಆನಂದಿಸುವನು. ಅವನ ಜೀವನದಲ್ಲೂ ಒಂದು ಸಮತೋಲನವಿದೆ. ಒಬ್ಬನು ಜಡ ಪ್ರಪಂಚದ ಕಾರ್ಯಕ್ಷೇತ್ರದಲ್ಲಿರಲಿ, ಬುದ್ಧಿಗೆ ಸಂಬಂಧಪಟ್ಟ ಕ್ಷೇತ್ರದಲ್ಲಿರಲಿ, ಅಧ್ಯಾತ್ಮಕ್ಕೆ ಸಂಬಂಧಪಟ್ಟ ಕ್ಷೇತ್ರದಲ್ಲಿರಲಿ, ದೇವರು ಪಕ್ಷಪಾತವಿಲ್ಲದೆ ಎಲ್ಲರಿಗೂ ಒಂದೇ ಬಗೆಯ ತೃಪ್ತಿಯನ್ನು ಕೊಟ್ಟಿರುವನು. ಆದ ಕಾರಣ ನಾವೇ ಪ್ರಪಂಚವನ್ನು ಉದ್ಧಾರ ಮಾಡುವವರು ಎಂದು ತಿಳಿಯಬೇಕಾಗಿಲ್ಲ. ನಾವು ಪ್ರಪಂಚಕ್ಕೆ ಎಷ್ಟೋ ವಿಷಯಗಳನ್ನು ಕಲಿಸಬಹುದು ಮತ್ತು ಅದರಿಂದ ಎಷ್ಟೋ ವಿಷಯಗಳನ್ನು ಕಲಿಯಬಹುದು. ಆಯಾಯ ದೇಶಕ್ಕೆ ಏನು ಆವಶ್ಯಕವೋ ಅದನ್ನು ಮಾತ್ರ ನಾವು ಕಲಿಸಬಹುದು. ಆಧ್ಯಾತ್ಮಿಕ ತಳಹದಿ ಇಲ್ಲದೆ ಇದ್ದರೆ ಪಾಶ್ಚಾತ್ಯ ಸಂಸ್ಕೃತಿ ಇನ್ನು ಐವತ್ತು ವರ್ಷಗಳಲ್ಲಿ ನಾಶವಾಗುವುದು. ಕೇವಲ ಕತ್ತಿಯ ಬಲದಿಂದ ಜನಾಂಗವನ್ನು ಆಳಿ ಪ್ರಯೋಜನವಿಲ್ಲ. ಬಲಾತ್ಕಾರದಿಂದ ಆಳಬಹುದು ಎಂಬ ಭಾವನೆ ಯಾವ ಕೇಂದ್ರದಿಂದ ಹೊಮ್ಮಿತೋ ಅದೇ ಮೊದಲು ಅವನತಿ ಹೊಂದಿ ನುಚ್ಚು ನೂರಾಗುವುದು. ಬಾಹ್ಯಶಕ್ತಿಯ ಅಭಿವ್ಯಕ್ತಿ ಮೂಲವಾದ ಯೂರೋಪು, ಅಧ್ಯಾತ್ಮವನ್ನು ತನ್ನ ಜೀವನದ ಮುಖ್ಯ ಆಧಾರವಾಗಿ ಮಾಡಿಕೊಳ್ಳದೇ ಹೋದರೆ ಇನ್ನು ಐವತ್ತು ವರ್ಷಗಳಲ್ಲಿ ನಿರ್ನಾಮವಾಗುವುದು. ಯೂರೋಪನ್ನು ಸಂರಕ್ಷಿಸುವುದೇ ಉಪನಿಷತ್ತಿನ ಧರ್ಮ.

ಭಿನ್ನ ಭಿನ್ನ ಪಂಥಗಳಲ್ಲಿ, ಭಿನ್ನ ಭಿನ್ನ ದರ್ಶನಗಳಲ್ಲಿ ಮತ್ತು ಶಾಸ್ತ್ರಗಳಲ್ಲಿ ಒಂದು ಸರ್ವಸಾಮಾನ್ಯ ಸಿದ್ಧಾಂತವಿದೆ. ಅದೇ, ಮಾನವನ ಆತ್ಮನಲ್ಲಿ ನಂಬಿಕೆ. ಇದು ಎಲ್ಲಾ ಪಂಥಗಳಿಗೂ ಸಮಾನವಾದ ಭಾವನೆ. ಇದು ಜಗತ್ತಿನ ಸ್ವಭಾವವನ್ನೇ ಬದಲಾಯಿಸಬಲ್ಲದು. ಹಿಂದೂಗಳು, ಜೈನರು, ಬೌದ್ಧರು, ಹಾಗೆ ನೋಡಿದರೆ ಭಾರತದ ಎಲ್ಲ ಧರ್ಮದವರೂ ಸರ್ವಶಕ್ತಿ ಮೂಲವಾದ ಒಂದು ಆತ್ಮಭಾವನೆಯಲ್ಲಿ ನಂಬಿಕೆಯನ್ನಿಟ್ಟಿದ್ದಾರೆ. ಭರತಖಂಡದ ಯಾವ ತತ್ತ್ವವಾಗಲೀ, ದರ್ಶನವಾಗಲೀ, ಯಾವುದೂ ಶಕ್ತಿ, ಪಾವಿತ್ರ್ಯ, ಪೂರ್ಣತೆ, ಇವನ್ನು ನೀವು ಹೊರಗಿನಿಂದ ಸ್ವೀಕರಿಸಬಲ್ಲಿರಿ ಎಂದು ಸಾರುವುದಿಲ್ಲ ಎಂಬುದು ನಿಮಗೆ ಚೆನ್ನಾಗಿ ಗೊತ್ತು. ಅವೆಲ್ಲ ನಿಮ್ಮ ಸ್ವಭಾವ, ಆಜನ್ಮಸಿದ್ಧ ಹಕ್ಕು ಎಂದು ಘೋಷಿಸುವುವು. ಅಪವಿತ್ರತೆ ಎಂಬುದು ನಿಮ್ಮ ನಿಜವಾದ ಸ್ವಭಾವವನ್ನು ಮರೆಮಾಚುವ ಒಂದು ಅಸತ್ಯವಾದ ಆವರಣ. ಆದರೆ ನಿಜವಾದ ನಿಮ್ಮ ಸ್ವರೂಪವು ಯಾವಾಗಲೂ ಪರಿಪೂರ್ಣವಾದದ್ದು, ಶಕ್ತಿಯುತವಾದದ್ದು. ನಿಮ್ಮನ್ನು ನೀವು ನಿಗ್ರಹಿಸುವುದಕ್ಕೆ ಯಾವ ಸಹಾಯವೂ ಬೇಕಿಲ್ಲ, ನೀವಾಗಲೇ ನಿಮ್ಮನ್ನು ನಿಗ್ರಹಿಸಿಕೊಂಡಿರುವಿರಿ. ಒಂದು ವ್ಯತ್ಯಾಸವೇನೆಂದರೆ ಅದು ನಿಮಗೆ ಗೊತ್ತಿದೆಯೇ ಇಲ್ಲವೇ ಎಂಬುದು. ಈ ತೊಂದರೆಗಳನ್ನೆಲ್ಲಾ ಅವಿದ್ಯೆ ಎಂಬ ಒಂದು ಪದದಲ್ಲಿ ವಿವರಿಸುವರು. ದೇವರಿಗೂ ಮನುಷ್ಯನಿಗೂ, ಪಾಪಿಗೂ, ಯತಿಗೂ ವ್ಯತ್ಯಾಸವನ್ನು ಕಲ್ಪಿಸುವುದು ಯಾವುದು? ಅಜ್ಞಾನ ಒಂದೇ. ಅತಿ ಪವಿತ್ರಾತ್ಮನಾದ ಮಾನವನಿಗೂ ನಿಮ್ಮ ಪಾದದಡಿಯಲ್ಲಿ ಹರಿಯುತ್ತಿರುವ ಕೀಟಕ್ಕೂ ನಡುವೆ ಇರುವ ವ್ಯತ್ಯಾಸವೇನು? ಅಜ್ಞಾನ, ಅದೊಂದೇ ವ್ಯತ್ಯಾಸಕ್ಕೆ ಕಾರಣ. ಹರಿಯುತ್ತಿರುವ ಆ ಕ್ಷುದ್ರ ಕೀಟದ ಅಂತರಾಳದಲ್ಲಿ ಸಾಕ್ಷಾತ್​ ಭಗವಂತನ ಅನಂತ ಶಕ್ತಿ, ಜ್ಞಾನ, ಪಾವಿತ್ರ್ಯಗಳು ಸುಪ್ತವಾಗಿವೆ. ಅವು ವ್ಯಕ್ತವಾಗಬೇಕಷ್ಟೇ.

ಈ ಒಂದು ಮಹಾಸತ್ಯವನ್ನು ಭರತಖಂಡವು ಜಗತ್ತಿಗೆ ಬೋಧಿಸಬೇಕಾಗಿದೆ. ಏಕೆಂದರೆ ಈ ಭಾವನೆ ಮತ್ತೆಲ್ಲಿಯೂ ಇಲ್ಲ. ಇದೇ ಅಧ್ಯಾತ್ಮತತ್ತ್ವ, ಆತ್ಮವಿಜ್ಞಾನ. ಮನುಷ್ಯನನ್ನು ಯಾವುದು ಕಾರ್ಯೋನ್ಮುಖನನ್ನಾಗಿ ಮಾಡುವುದು? ಶಕ್ತಿ. ಶಕ್ತಿಯೇ ಪುಣ್ಯ, ದೌರ್ಬಲ್ಯವೇ ಪಾಪ. ಯಾವುದಾದರೊಂದು ಪದವು ಉಪನಿಷತ್ತುಗಳ ಮಹಾಗಣಿಯಿಂದ ಸಿಡಿಮದ್ದಿನಂತೆ ಹಾರುತ್ತಿದ್ದರೆ, ಸಿಡಿಮದ್ದಿನಂತೆ ಅಜ್ಞಾನಿಗಳ ಮೇಲೆ ಸಿಡಿಯುತ್ತಿದ್ದರೆ, ಅದೇ “ಅಭೀಃ” ಎಂಬ ಪದ. ಬೋಧಿಸಬೇಕಾದ ಒಂದೇ ಒಂದು ಧರ್ಮವೆಂದರೆ “ಅಭೀಃ” ಅಥವಾ ನಿರ್ಭೀತಿ ಎಂಬ ಧರ್ಮ. ಲೌಕಿಕ ಜಗತ್ತಿನಲ್ಲಾಗಲಿ, ಧಾರ್ಮಿಕ ಜಗತ್ತಿನಲ್ಲಾಗಲೀ ಭಯವೇ ಪಾಪ ಮತ್ತು ಅವನತಿಗೆಲ್ಲಾ ಮೂಲಕಾರಣ ಎನ್ನುವುದರಲ್ಲಿ ಯಾವ ಸಂದೇಹವೂ ಇಲ್ಲ. ಭಯವೇ ದುಃಖದ ಮೂಲ, ಸಾವಿನ ಮೂಲ, ಅನಂತ ಪಾಪಗಳ ಮೂಲ. ಈ ಭಯಕ್ಕೆ ಕಾರಣ ಯಾವುದು? ನಮ್ಮ ನಿಜಸ್ವರೂಪದ ವಿಷಯದಲ್ಲಿ ನಮಗಿರುವ ಅಜ್ಞಾನ. ನಾವೆಲ್ಲರೂ\break ರಾಜಾಧಿರಾಜನ ಸಿಂಹಾಸನಕ್ಕೆ ಉತ್ತರಾಧಿಕಾರಿಗಳು. ನಾವೆಲ್ಲರೂ ಭಗವಂತನ ಸತ್ತ್ವದಿಂದಲೇ ಆದವರು, ಅಷ್ಟೇ ಅಲ್ಲ, ಅದ್ವೈತ ಸಿದ್ಧಾಂತದ ಪ್ರಕಾರ ನಾವೇ ಭಗವಂತ. ನಾವು ಕ್ಷುದ್ರಮಾನವರೆಂದು ಭಾವಿಸಿ ಅದನ್ನು ಮರೆತಿದ್ದೇವೆ. ನಾವು ಆ ಉನ್ನತ ಸ್ಥಿತಿಯಿಂದ ಪತಿತರಾಗಿರುವೆವು. ಆದ್ದರಿಂದಲೇ ನಾನು ನಿಮಗಿಂತ ಸ್ವಲ್ಪ ಮೇಲು, ಅಥವಾ ನೀವು ನನಗಿಂತ ಸ್ವಲ್ಪ ಕೀಳು ಎಂಬ ತಾರತಮ್ಯವನ್ನು ನಾವು ಮಾಡುವುದು. ಭರತಖಂಡ ಜಗತ್ತಿಗೆ ಸಾರಬೇಕಾದ ಒಂದು ಮಹಾ ಸಂದೇಶವೇ ಈ ಏಕತ್ವದ ಭಾವನೆ. ಇದನ್ನು ನಾವು ತಿಳಿದುಕೊಂಡರೆ ವಸ್ತುಸ್ಥಿತಿಯು ಸಂಪೂರ್ಣವಾಗಿ ಬದಲಾಗುತ್ತದೆ. ಏಕೆಂದರೆ ನಾವು ಇದುವರೆಗೆ ಯಾವ ದೃಷ್ಟಿಯಿಂದ ಜಗತ್ತನ್ನು ನೋಡುತ್ತಿರುವೆವೋ, ಇನ್ನು ಮೇಲೆ ಆ ದೃಷ್ಟಿಯಿಂದ ನೋಡುವುದಿಲ್ಲ. ಹಾಗಾದಾಗ, ಇನ್ನು ಮುಂದೆ ಜಗತ್ತು ಒಂದು ಚೇತನವು ಮತ್ತೊಂದು ಚೇತನದೊಂದಿಗೆ ಹೋರಾಡಿ, ಬಲಾಢ್ಯರು ಜಯ ಪಡೆದು ದುರ್ಬಲರು ನಾಶವಾಗುವ ಸಮರಭೂಮಿಯಾಗುವುದಿಲ್ಲ. ಅದಕ್ಕೆ ಬದಲು ಭಗವಂತನು ಶಿಶುವಿನಂತೆ ಆಟವಾಡುವ ಮತ್ತು ನಾವೆಲ್ಲರೂ ಅವನ ಆಟದ ಗೆಳೆಯರಾಗಿರುವ ಕ್ರೀಡಾಂಗಣವಾಗುತ್ತದೆ. ಅದೊಂದು ಆಟ. ಎಷ್ಟೇ ಕ್ರೂರವಾಗಿರಲಿ, ಭಯಂಕರವಾಗಿರಲಿ, ಬೀಭತ್ಸವಾಗಿರಲಿ ಅದೊಂದು ಆಟ ಮಾತ್ರ. ಆ ಅಂಶವನ್ನು ನಾವು ತಪ್ಪಾಗಿ ಗ್ರಹಿಸಿರುವೆವು. ಆತ್ಮಸ್ವರೂಪವು ನಮಗೆ ತಿಳಿದ ಮೇಲೆ, ಅತ್ಯಂತ ಹೀನನಿಗೆ, ದುರ್ಬಲನಿಗೆ, ಪಾಪಿಗೆ ಒಂದು ಭರವಸೆ ದೊರಕುವುದು. ಖಂಡಿತವಾಗಿಯೂ ನಿರಾಶರಾಗಬೇಡಿ ಎಂದು ನಮ್ಮ ಶಾಸ್ತ್ರ ಸಾರುತ್ತದೆ. ನೀವು ಏನು ಮಾಡಿದರೂ ನಿಮ್ಮ ಸ್ವಭಾವ ಬದಲಾಗುವಂತೆ ಇಲ್ಲ. ಪ್ರಕೃತಿಯೇ ಪ್ರಕೃತಿಯನ್ನು ನಾಶಮಾಡಲಾರದು. ಪರಿಶುದ್ಧತೆಯೇ ನಿಮ್ಮ ಸ್ವಭಾವ. ಲಕ್ಷಾಂತರ ವರ್ಷಗಳವರೆಗೆ ಅದು ಹುದುಗಿರಬಹುದು. ಕೊನೆಗೆ ಅದು ಜಯಿಸಿ ಮೇಲೆದ್ದೇ ಏಳುವುದು. ಅದ್ವೈತ ಸಿದ್ಧಾಂತವು ಎಲ್ಲರಿಗೂ ಭರವಸೆಯನ್ನು ತರುವುದೇ ಹೊರತು ನಿರಾಶೆಯನ್ನಲ್ಲ. ಅದು ಭಯದ ಮೂಲಕ ಬೋಧಿಸುವುದಿಲ್ಲ. ಕಾಲು ಜಾರಿದಾಗ ನಿಮ್ಮನ್ನು ಮೆಟ್ಟಿಕೊಳ್ಳಲು ಹವಣಿಸುತ್ತಿರುವ ದೆವ್ವಗಳನ್ನು ಅದು ಸೃಷ್ಟಿಸುವುದಿಲ್ಲ. ದೆವ್ವಗಿವ್ವಗಳಿಗೂ ಅದಕ್ಕೂ ಏನೇನೂ ಸಂಬಂಧವಿಲ್ಲ. ನಿಮ್ಮ ಅದೃಷ್ಟ ನಿಮ್ಮ ಕೈಯಲ್ಲಿ ಇರುವುದೆಂದು ಅದು ಸಾರುವುದು. ಈ ದೇಹವನ್ನು ತಯಾರಿಸಿರುವುದು ನಿಮ್ಮ ಕರ್ಮವೇ ಹೊರತು ಮತ್ತಾವುದೂ ಅಲ್ಲ ಎನ್ನುವುದು. ಸರ್ವವ್ಯಾಪಿಯಾದ ಭಗವಂತನು ನಮ್ಮ ಅಜ್ಞಾನದಿಂದಾಗಿ ಮರೆಯಾಗಿರುವನು. ಇದಕ್ಕೆ ನೀವೇ ಕಾರಣ. ನಿಮ್ಮ ಇಚ್ಛೆ ಇಲ್ಲದೇ ನೀವು ಈ ಭಯಾನಕ ಪ್ರಪಂಚಕ್ಕೆ ಬಂದಿದ್ದಲ್ಲ. ಈ ಕ್ಷಣದಲ್ಲಿ ನೀವು ಹೇಗೆ ಮಾಡುತ್ತಿದ್ದೀರೋ ಹಾಗೆಯೇ ನಿಮ್ಮ ದೇಹದ ಅಂಗುಲ ಅಂಗುಲವನ್ನೂ ನೀವೇ ತಯಾರುಮಾಡಿಕೊಂಡಿದ್ದೀರಿ. ಅಂದರೆ ನಿಮಗಾಗಿ ನೀವೇ ಊಟಮಾಡುತ್ತಿರುವಿರಿ, ಇತರರು ನಿಮಗಾಗಿ ಮಾಡುವುದಿಲ್ಲ; ತಿಂದದ್ದನ್ನು ನೀವೇ ಅರಗಿಸಿಕೊಳ್ಳುತ್ತೀರಿ, ಇತರರು ಅಲ್ಲ. ನೀವೇ ಆಹಾರದಿಂದ ರಕ್ತ, ಮಾಂಸ, ದೇಹ ಇವನ್ನು ಮಾಡಿಕೊಳ್ಳುತ್ತೀರಿ. ಇತರರು ಅಲ್ಲ. ಇದನ್ನು ಹಿಂದಿನಿಂದಲೂ ನೀವು ಮಾಡುತ್ತಾ ಬಂದಿದ್ದೀರಿ. ಅನಂತ ಸರಪಳಿಯ ಒಂದು ಕೊಂಡಿಯನ್ನು ತಿಳಿದರೆ ಇಡೀ ಸರಪಳಿಯನ್ನು ತಿಳಿದಂತೆ; ನಿಮ್ಮ ದೇಹಕ್ಕೆ ನೀವೇ ಕಾರಣ ಎಂದು ತಿಳಿದರೆ, ಹಿಂದಿನ ಮತ್ತು ಮುಂದಿನ ದೇಹಗಳಿಗೂ ನೀವೇ ಕಾರಣವೆಂದು ತಿಳಿಯುತ್ತೀರಿ. ಎಲ್ಲಾ ಒಳ್ಳೆಯ ಮತ್ತು ಕೆಟ್ಟದರ ಜವಾಬ್ದಾರಿಯೂ ನಮ್ಮ ಮೇಲೆ ಇರುವುದು ಆಶಾಜನಕ ಸಂಗತಿ. ನಾನು ಏನನ್ನು ಕಟ್ಟಿರುವೆನೋ ಅದನ್ನು ಧ್ವಂಸ ಮಾಡಬಲ್ಲೆ. ಆದರೆ ನಮ್ಮ ಧರ್ಮವು ಭಗವಂತನ ಕೃಪೆಯ ತತ್ತ್ವವನ್ನು ನಿರಾಕರಿಸುವುದಿಲ್ಲ. ಅವನ ಕೃಪೆ ಯಾವಾಗಲೂ ಇದ್ದೇ ಇರುತ್ತದೆ. ಈ ಒಳ್ಳೆಯ ಮತ್ತು ಕೆಟ್ಟ ಮಹಾಪ್ರವಾಹದ ಪಕ್ಕದಲ್ಲೇ ಅವನೂ ಇದ್ದಾನೆ. ಬಂಧನಾತೀತನಾದ, ಅನವರತವೂ ದಯಾಮೂರ್ತಿಯಾದ ಭಗವಂತ ಯಾವಾಗಲೂ ನಮ್ಮನ್ನು ಪಾರು ಗಾಣಿಸಲು ಸಿದ್ಧನಾಗಿರುವನು. ಅವನ ಕೃಪೆ ಅಪಾರವಾದುದು. ಪರಿಶುದ್ಧಾತ್ಮರಿಗೆಲ್ಲಾ ಅದು ಲಭಿಸುವುದು.

ನಿಮ್ಮ ಆಧ್ಯಾತ್ಮಿಕತೆಯು, ಒಂದು ದೃಷ್ಟಿಯಲ್ಲಿ ನೂತನ ಸಮಾಜ ಸ್ವರೂಪಕ್ಕೆ ಆಧಾರಭೂತವಾಗಿರಬೇಕು. ನನಗೆ ಹೆಚ್ಚು ಸಮಯವಿದ್ದಿದ್ದರೆ ಪಾಶ್ಚಾತ್ಯರು ನಮ್ಮ ಅದ್ವೈತ ಸಿದ್ಧಾಂತದ ನಿರ್ಣಯಗಳಿಂದ ಹೇಗೆ ಹಲವು ವಿಷಯಗಳನ್ನು ಇನ್ನೂ ಕಲಿಯಬೇಕಾಗಿದೆ ಎಂಬುದನ್ನು ತೋರಿಸುತ್ತಿದ್ದೆ. ಏಕೆಂದರೆ ವಿಜ್ಞಾನದ ನವನವ ಆವಿಷ್ಕಾರಗಳ ಈ ಯುಗದಲ್ಲಿ\- ಸಗುಣ ದೇವರ ಕಲ್ಪನೆಯಿಂದ ಅವರಿಗೆ ಹೆಚ್ಚು ಪ್ರಯೋಜನವಿಲ್ಲ. ಆದರೆ, ಧರ್ಮದ ವಿಷಯದಲ್ಲಿ ಅಷ್ಟೇನೂ ಪಕ್ವವಲ್ಲದ ಭಾವನೆಗಳನ್ನು ಹೊಂದಿರುವ ವ್ಯಕ್ತಿ ದೇವಸ್ಥಾನ, ದೇವತಾಮೂರ್ತಿಗಳು ಮುಂತಾದುವನ್ನು ಬಯಸಿದರೆ, ಅವುಗಳನ್ನು ಬೇಕಾದಷ್ಟು ಪ್ರಮಾಣದಲ್ಲಿಟ್ಟುಕೊಳ್ಳಬಹುದು. ಜಗತ್ತಿನಲ್ಲಿ ಪ್ರೀತಿಸುವುದಕ್ಕೆ ಒಬ್ಬ ಪರಮೇಶ್ವರ ಬೇಕಾಗಿದ್ದರೆ, ಇನ್ನೂ ಇದುವರೆಗೆ ಕಲ್ಪಿಸಿಕೊಳ್ಳದಂತಹ ಪರಮೋತ್ಕೃಷ್ಟ ಪರಮೇಶ್ವರನ ಭಾವನೆ ಇಲ್ಲಿದೆ. ಒಬ್ಬನು ವಿಚಾರವಾದಿಯಾಗಿ ಯುಕ್ತಿಯನ್ನು ತೃಪ್ತಿಪಡಿಸಿಕೊಳ್ಳಬಯಸಿದರೆ ನಿರ್ಗುಣ ಬ್ರಹ್ಮನ ವಿಚಾರವಾಗಿ ಯುಕ್ತಿಯುಕ್ತ ಭಾವನೆಗಳೂ ಕೂಡ ಇಲ್ಲಿವೆ.

