
\chapter{ಭಕ್ತಿ}

\begin{center}
(ಪಂಜಾಬಿನ ಸಿಯಾಲ್​ಕೋಟೆಯಲ್ಲಿ ನೀಡಿದ ಉಪನ್ಯಾಸ)
\end{center}

ಜಗತ್ತಿನಲ್ಲಿರುವ ಧರ್ಮಗಳೆಲ್ಲ ಉಪಾಸನಾ ದೃಷ್ಟಿಯಿಂದ ವಿಭಿನ್ನವಾದರೂ ಅವುಗಳ ಮೂಲರೂಪ ಒಂದೇ ಆಗಿದೆ. ಕೆಲವು ಕಡೆ ಜನರು ಗುಡಿಗಳನ್ನು ಕಟ್ಟಿ ಅಲ್ಲಿ ಪೂಜಿಸುವರು; ಕೆಲವು ಕಡೆ ಬೆಂಕಿಯನ್ನು ಪೂಜಿಸುವರು; ಕೆಲವು ಕಡೆ ವಿಗ್ರಹಕ್ಕೆ ಅಡ್ಡ ಬೀಳುವರು; ಮತ್ತೆ ಅನೇಕರು ದೇವರನ್ನು ನಂಬುವುದೇ ಇಲ್ಲ. ಈ ಎಲ್ಲ ವಿಧಾನಗಳೂ ಸತ್ಯವೇ. ಏಕೆಂದರೆ ಅವೆಲ್ಲದರ ಮೂಲ ತತ್ತ್ವ, ಮೂಲಭಾವ ಒಂದೇ ಆಗಿರುವುದು. ಕೆಲವು ಧರ್ಮಗಳಲ್ಲಿ ದೇವರನ್ನು ಪೂಜಿಸುವುದಿಲ್ಲ. ಅವನಲ್ಲಿ ನಂಬಿಕೆಯೂ ಇಲ್ಲ. ಆದರೆ ಸತ್ಪುರುಷರನ್ನು ದೇವರಂತೆ ಪೂಜಿಸುವರು. ಇದಕ್ಕೆ ಉದಾಹರಣೆಯಾಗಿ ಬೌದ್ಧಧರ್ಮವನ್ನು ಪ್ರಸ್ತಾಪಿಸಬಹುದು. ದೇವರನ್ನು ಪೂಜಿಸಲಿ, ಅಥವಾ ಸತ್ಪುರುಷರನ್ನು ಪೂಜಿಸಲಿ, ಭಕ್ತಿ ಎಲ್ಲ ಕಡೆಯೂ ಇದೆ. ಭಕ್ತಿ ರೂಪವನ್ನು ತಾಳಿದ ಉಪಾಸನೆ ಎಲ್ಲಾ ಕಡೆಗಳಲ್ಲಿಯೂ ಪ್ರಬಲವಾಗಿದೆ. ಜ್ಞಾನಕ್ಕಿಂತ ಭಕ್ತಿಯನ್ನು ಪಡೆಯುವುದು ಸುಲಭ. ಜ್ಞಾನಕ್ಕೆ ಅನುಗುಣವಾದ ಪರಿಸರ ಬೇಕು. ಕಠಿಣ ಅಭ್ಯಾಸ ಮಾಡಬೇಕು. ಯೋಗವನ್ನು ಅಭ್ಯಾಸ ಮಾಡಬೇಕಾದರೆ ಒಬ್ಬನು ಸಂಪೂರ್ಣವಾಗಿ ಆರೋಗ್ಯವಾಗಿರಬೇಕು ಮತ್ತು ಸಂಸಾರದ ಆಸಕ್ತಿಯಿಂದ ಮುಕ್ತನಾಗಿರಬೇಕು. ವ್ಯಕ್ತಿಯು ಯಾವ ಸ್ಥಿತಿಯಲ್ಲಿದ್ದರೂ ಭಕ್ತಿಯನ್ನು ಬಹಳ ಸುಲಭವಾಗಿ ಅಭ್ಯಾಸ ಮಾಡಬಹುದು. ಭಕ್ತಿಶಾಸ್ತ್ರವನ್ನು ಬರೆದ ಶಾಂಡಿಲ್ಯರು ಭಗವಂತನಲ್ಲಿರುವ ಪರಮ ಪ್ರೇಮವೇ ಭಕ್ತಿ ಎನ್ನುವರು. ಪ್ರಹ್ಲಾದನೂ ಹೀಗೆಯೇ ಹೇಳುವನು. ಒಬ್ಬನಿಗೆ ಒಪ್ಪತ್ತು ಊಟ ಸಿಕ್ಕದೆ ಇದ್ದರೂ ವ್ಯಾಕುಲಪಡುವನು. ಅವನ ಮಗ ಸತ್ತರೆ ಎಷ್ಟು ದಾರುಣ ದುಃಖಪಡುವನು! ನಿಜವಾದ ಭಕ್ತ ಭಗವಂತನಿಗಾಗಿ ಹಂಬಲಿಸುವಾಗ ಹೀಗೆಯೇ ವ್ಯಥೆಪಡುವನು. ಭಕ್ತಿಯ ಒಂದು ದೊಡ್ಡ ಗುಣವೆಂದರೆ ಅದು ಮನಸ್ಸನ್ನು ಪರಿಶುದ್ಧಗೊಳಿಸುತ್ತದೆ. ಪರಮಾತ್ಮನ ಮೇಲೆ ಇಡುವ ದೃಢಭಕ್ತಿಯೊಂದೇ ಸಾಕು ಮನಸ್ಸನ್ನು ಶುದ್ಧ ಮಾಡುವುದಕ್ಕೆ. “ಹೇ ದೇವರೆ, ಅನಂತ ನಾಮಗಳು ನಿನಗಿವೆ. ಪ್ರತಿಯೊಂದು ಹೆಸರಿನಲ್ಲಿಯೂ ನಿನ್ನ ಶಕ್ತಿ ತುಂಬಿದೆ. ಪ್ರತಿಯೊಂದು ಹೆಸರೂ ಅರ್ಥಗರ್ಭಿತವಾಗಿದೆ.” ಭಗವಂತನನ್ನು ನಾವು ಸದಾ ಚಿಂತಿಸಬೇಕು. ಹಾಗೆ ಮಾಡುವುದಕ್ಕೆ ಕಾಲದೇಶಗಳನ್ನು ಪರಿಗಣಿಸಬೇಕಾಗಿಲ್ಲ.

ಸಾಧಕರು ದೇವರನ್ನು ಬೇರೆ ಬೇರೆ ಹೆಸರುಗಳಿಂದ ಪೂಜಿಸಿದರೂ ದೇವರಲ್ಲಿ ಭಿನ್ನತೆಯಿಲ್ಲ. ಒಬ್ಬನು ತಾನು ದೇವರನ್ನು ಪೂಜಿಸುವ ರೀತಿಯೇ ಬಹಳ ಪ್ರಯೋಜನಕಾರಿ ಎಂದು ಭಾವಿಸುವನು. ಮತ್ತೊಬ್ಬನು ಮುಕ್ತಿ ಗಳಿಸುವುದಕ್ಕೆ ತನ್ನ ಮಾರ್ಗವೇ ಅತಿ ಶ್ರೇಷ್ಠವೆಂದು ಭಾವಿಸುವನು. ಮೂಲ ತತ್ತ್ವದ ದೃಷ್ಟಿಯಿಂದ ಎಲ್ಲರೂ ಒಂದೇ. ಶೈವರು ಶಿವನನ್ನು ಅಮಿತ ಶಕ್ತಿಶಾಲಿ ಎನ್ನುವರು. ದೇವಿಯ ಉಪಾಸಕರು ತಮ್ಮ ದೇವಿಯೇ ಪ್ರಪಂಚದಲ್ಲೆಲ್ಲಾ ಸರ್ವಶಕ್ತಳು ಎಂಬ ತಮ್ಮ ಭಾವನೆಯಲ್ಲಿ ಯಾರಿಗೂ ಸೋಲುವ ಹಾಗಿಲ್ಲ. ನಿಮಗೆ ಶಾಶ್ವತ ಭಕ್ತಿ ಬೇಕಾದರೆ ದ್ವೇಷಭಾವನೆಯನ್ನು ತ್ಯಜಿಸಿ. ಭಕ್ತಿಮಾರ್ಗಕ್ಕೆ ದ್ವೇಷ ದೊಡ್ಡ ಆತಂಕ. ಯಾವ ಭಕ್ತ ಯಾರನ್ನೂ ದ್ವೇಷಿಸುವುದಿಲ್ಲವೋ ಅವನು ದೇವರನ್ನು ಸೇರುತ್ತಾನೆ. ಆದರೂ ತನ್ನ ಇಷ್ಟದ ಮೇಲೆ ಭಕ್ತಿ ಅವಶ್ಯಕ. “ವಿಷ್ಣು ರಾಮ ಇಬ್ಬರೂ ಒಂದೇ ಎನ್ನುವುದು ನನಗೆ ಗೊತ್ತಿದೆ. ಆದರೂ ಕಮಲಲೋಚನನಾದ ರಾಮನೇ ನನ್ನ ಸರ್ವಸ್ವ” ಎನ್ನುವನು ಹನುಮಂತ.

ಪ್ರತಿಯೊಬ್ಬರ ಸ್ವಭಾವವೂ ಬೇರೆ ಬೇರೆ. ಈ ಸ್ವಭಾವದ ವೈಶಿಷ್ಟ್ಯ ಅವರಲ್ಲಿರಬೇಕು. ಆದಕಾರಣ ಪ್ರಪಂಚದಲ್ಲಿ ಒಂದೇ ಧರ್ಮ ಇರಲಾರದು. ಸದ್ಯಕ್ಕೆ ದೇವರ ದಯೆಯಿಂದ ಒಂದೇ ಧರ್ಮವಿಲ್ಲದಿರಲಿ. ಒಂದೇ ಧರ್ಮವಿದ್ದರೆ ಸುವ್ಯವಸ್ಥೆಯ ಬದಲು ಅವ್ಯವಸ್ಥೆಯೇ ಇರುತ್ತದೆ. ಪ್ರತಿಯೊಬ್ಬನೂ ತನಗೇ ವಿಶಿಷ್ಟವಾದ ತನ್ನ ಸ್ವಭಾವದಂತೆ ವರ್ತಿಸಬೇಕು. ತನ್ನ ದಾರಿಯಲ್ಲೇ ಮುಂದುವರಿಯಲು ಪ್ರೋತ್ಸಾಹಿಸುವಂತಹ ಗುರು ಸಿಕ್ಕಿದರೆ ಅವನಿಗೆ ಅನುಕೂಲ. ಅವನು ಮುಂದೆ ಸಾಗುವನು. ಪ್ರತಿಯೊಬ್ಬರಿಗೂ ಅವರ ಸ್ವಭಾವಕ್ಕೆ ತಕ್ಕಂತೆ ವರ್ತಿಸಲು ಅವಕಾಶ ಕೊಡೋಣ. ಅವನನ್ನು ಬೇರೆ ಮಾರ್ಗದಲ್ಲಿ ಹೋಗುವಂತೆ ಬಲ್ಕಾತರಿಸಿದರೆ ಆಗಲೆ ಏನನ್ನು ಅವನು ಪಡೆದಿರುವನೋ ಅದನ್ನು ಕಳೆದುಕೊಂಡು ಕೆಲಸಕ್ಕೆ ಬಾರದವನಾಗುವನು. ಒಬ್ಬನ ಮುಖ ಮತ್ತೊಬ್ಬನ ಮುಖದಂತೆ ಹೇಗೆ ಇಲ್ಲವೋ ಹಾಗೆಯೆ ಒಬ್ಬನ ಸ್ವಭಾವ ಮತ್ತೊಬ್ಬನ ಸ್ವಭಾವದಂತೆ ಇಲ್ಲ. ತನ್ನ ವೈವಿಧ್ಯಕ್ಕೆ ತಕ್ಕಂತೆ ಅವನ ಪ್ರಗತಿಗೆ ಅವಕಾಶವನ್ನೇಕೆ ಕೊಡಬಾರದು? ಒಂದು ನದಿ ಒಂದು ದಿಕ್ಕಿನಲ್ಲಿ ಹರಿಯುತ್ತಿದೆ. ಅದೇ ದಿಕ್ಕಿನಲ್ಲಿ ಅದು ಸರಿಯಾಗಿ ಹರಿಯುವಂತೆ ಮಾಡಿದರೆ ಅದರ ವೇಗ ಮತ್ತು ಶಕ್ತಿ ಹೆಚ್ಚುವುವು. ಆದರೆ ಬೇರೊಂದು ಕಡೆಗೆ ತಿರುಗಿಸಲು ಪ್ರಯತ್ನಪಟ್ಟರೆ ಪರಿಣಾಮ ಬೇರೆ ಆಗುವುದು. ಅದರ ಗಾತ್ರ, ವೇಗ ಎರಡೂ ಕಡಮೆಯಾಗುವುವು. ಈ ಜೀವನವು ಅತಿ ಮುಖ್ಯವಾದುದು. ಒಬ್ಬನ ಪ್ರಕೃತಿಗೆ ತಕ್ಕಂತೆ ಅದನ್ನು ವೃದ್ಧಿಗೊಳಿಸಬೇಕು. ಭರತಖಂಡದಲ್ಲಿ ಧರ್ಮದ ಹೆಸರಿನಲ್ಲಿ ದ್ವೇಷವಿರಲಿಲ್ಲ, ಪ್ರತಿಯೊಂದು ಧರ್ಮವೂ ಸುರಕ್ಷಿತವಾಗಿತ್ತು. ಅದಕ್ಕಾಗಿಯೇ ಧರ್ಮವು ಈಗಲೂ ಜೀವಂತವಾಗಿದೆ. ಧರ್ಮದ ಹೆಸರಿನಲ್ಲಿ ವ್ಯಾಜ್ಯಕ್ಕೆ ಮೂಲಕಾರಣ, ಒಂದು ಮಾತ್ರ ಸತ್ಯವಾಗಿರಬೇಕು, ಯಾರು ಅದನ್ನು ನಂಬುವುದಿಲ್ಲವೊ ಅವರು ಮೂರ್ಖರಾಗಬೇಕು, ಇಲ್ಲವೇ ಮಿಥ್ಯಾಚಾರಿಗಳಾಗಿರಬೇಕು, ಹಾಗೆ ಆಗಿಲ್ಲದೆ ಇದ್ದರೆ ಅವರು ತನ್ನನ್ನು ಅನುಸರಿಸುತ್ತಿದ್ದರು-ಎಂಬ ಭಾವನೆ.

ಮನುಷ್ಯ ಒಂದೇ ಧರ್ಮವನ್ನು ಅನುಸರಿಸಬೇಕೆಂದು ದೇವರು ಯೋಚಿಸಿದ್ದರೆ ಇಷ್ಟೊಂದು ಧರ್ಮಗಳು ಏತಕ್ಕೆ ಹುಟ್ಟಿದವು? ಒಂದೇ ಧರ್ಮವನ್ನು ಎಲ್ಲರ ಮೇಲೂ ಬಲ್ಕಾತರಿಸುವುದಕ್ಕೆ ಬೇಕಾದಷ್ಟು ಪ್ರಯತ್ನಗಳು ನಡೆದಿವೆ. ಕತ್ತಿಯ ಬಲದಿಂದ ಎಲ್ಲರೂ ಒಂದೇ ಧರ್ಮವನ್ನು ಅನುಸರಿಸಬೇಕೆಂದು ಬಲಾತ್ಕಾರ ಮಾಡಿದರೂ, ಅಲ್ಲೇ ಹತ್ತಾರು ಧರ್ಮಗಳು ಹುಟ್ಟಿವೆ ಎಂದು ಇತಿಹಾಸ ಸಾರುವುದು. ಒಂದೇ ಧರ್ಮ ಎಲ್ಲರಿಗೂ ಸರಿ ಹೋಗುವುದಿಲ್ಲ. ಮನುಷ್ಯನು ಕ್ರಿಯೆ ಪ್ರತಿಕ್ರಿಯೆ ಇವೆರಡರ ಪರಿಣಾಮ. ಇದರಿಂದಲೇ ಅವನು ಆಲೋಚಿಸುವುದಕ್ಕೆ ಸಾಧ್ಯವಾಗಿದೆ. ಇಂತಹ ಶಕ್ತಿ ಮನುಷ್ಯನ ಬುದ್ಧಿಶಕ್ತಿಯನ್ನು ಪ್ರಚೋದಿಸದೆ ಇದ್ದರೆ ಅವನಿಗೆ ಆಲೋಚಿಸುವುದಕ್ಕೆ ಸಾಧ್ಯವಾಗುತ್ತಿರಲಿಲ್ಲ. ಮಾನವನು ಆಲೋಚನಾ ಜೀವಿ. ಮನಸ್ಸು ಇರುವವನೇ ಮನುಷ್ಯ. ಆಲೋಚನಾ ಶಕ್ತಿಯನ್ನು ಕಳೆದುಕೊಂಡೊಡನೆಯೇ ಅವನು ಪ್ರಾಣಿಗಿಂತ ಮೇಲಾಗಲಾರ. ಯಾರಿಗೆ ಅಂತಹ ಮನುಷ್ಯ ಇಷ್ಟ? ಭರತಖಂಡದ ಜನರು ದೇವರ ದಯೆಯಿಂದ ಸದ್ಯಕ್ಕೆ ಹಾಗೆ ಆಗದಿರಲಿ.

ಏಕತ್ವದಲ್ಲಿ ವೈವಿಧ್ಯದ ಜ್ಞಾನ ಮನುಷ್ಯನನ್ನು ಮನುಷ್ಯನಂತೆ ಇಟ್ಟಿರುವುದಕ್ಕೆ ಆವಶ್ಯಕ. ಪ್ರತಿಯೊಂದರಲ್ಲಿಯೂ ವೈವಿಧ್ಯವನ್ನು ನಾವು ಕಾಪಾಡಬೇಕು. ವೈವಿಧ್ಯ ಇರುವವರೆಗೆ ಸೃಷ್ಟಿ ಇರುವುದು. ವೈವಿಧ್ಯ ಎಂದರೆ ಒಂದು ಸಣ್ಣದು ಮತ್ತೊಂದು ದೊಡ್ಡದು ಎಂದು ಅರ್ಥವಲ್ಲ. ಪ್ರತಿಯೊಬ್ಬನೂ ತನ್ನ ಜೀವನ ಕಾರ್ಯಕ್ಷೇತ್ರದಲ್ಲಿ ತನ್ನ ಪಾಲಿನ ಪಾತ್ರವನ್ನು ಸರಿಯಾಗಿ ಅಭಿನಯಿಸಿದರೆ ವೈವಿಧ್ಯವನ್ನು ಸುರಕ್ಷಿತವಾಗಿ ಕಾಪಾಡಬಹುದು. ಪ್ರತಿಯೊಂದು ಧರ್ಮದಲ್ಲಿಯೂ ಯೋಗ್ಯರು. ಮಹಾ ಪುರುಷರು ಇರುವರು. ಅವರು ತಮ್ಮ ತಮ್ಮ ಧರ್ಮಕ್ಕೆ ಗೌರವವನ್ನು ತರುವರು. ಎಲ್ಲಾ ಧರ್ಮಗಳಲ್ಲಿಯೂ ಇಂತಹವರು ಇರುವುದರಿಂದ ಯಾರೂ ಯಾವ ಧರ್ಮವನ್ನೂ ದ್ವೇಷಿಸಬೇಕಾಗಿಲ್ಲ. ಹಾಗಾದರೆ ಯಾವ ಧರ್ಮವು ದುಷ್ಟತನಕ್ಕೆ ಉತ್ತೇಜನ ಕೊಡುವುದೋ ಅದನ್ನು ಗೌರವಿಸಬೇಕೆ ಎಂಬ ಪ್ರಶ್ನೆ ಏಳುವುದು. ಅದನ್ನು ಗೌರವಿಸಕೂಡದು ಎಂಬುದೇ ಉತ್ತರ. ಅಂತಹ ಧರ್ಮವನ್ನು ತಕ್ಷಣ ಹೊರದೂಡಬೇಕು. ಏಕೆಂದರೆ ಅದು ಅನಾಹುತಕ್ಕೆ ಕಾರಣವಾಗುತ್ತದೆ. ಎಲ್ಲಾ ಧರ್ಮಗಳೂ ನೀತಿಯ ಮೇಲೆ ನಿಂತಿರಬೇಕು. ವ್ಯಕ್ತಿಯ ಶೀಲಶುದ್ಧಿ ಧರ್ಮಕ್ಕಿಂತ ಮೇಲಾದುದು. ಆಚಾರವೆಂದರೆ ಬಾಹ್ಯ ಮತ್ತು ಆಂತರಿಕ ಪರಿಶುದ್ಧಿ ಎಂಬುದನ್ನು ಗಮನಿಸಬೇಕು. ಬಾಹ್ಯ ಶುದ್ಧಿಯನ್ನು ಸ್ನಾನ ಮುಂತಾದವುಗಳಿಂದ ಪಡೆಯಬಹುದು. ಶಾಸ್ತ್ರಗಳಲ್ಲಿ ಇದನ್ನು ಹೇಳಿರುವರು. ಆಂತರಿಕ ಪರಿಶುದ್ಧಿಯನ್ನು ಪಡೆಯಬೇಕಾದರೆ ಸುಳ್ಳು ಹೇಳಕೂಡದು, ಮದ್ಯಪಾನ ಮಾಡಬಾರದು, ಅನೈತಿಕ ಕ್ರಿಯೆಗಳನ್ನು ಮಾಡಕೂಡದು. ಯಾವಾಗಲೂ ಇತರರಿಗೆ ಒಳ್ಳೆಯದನ್ನು ಮಾಡಬೇಕು. ನೀವು ಪಾಪವನ್ನು ಮಾಡದೆ ಇದ್ದರೆ, ಸುಳ್ಳು ಹೇಳದೆ ಇದ್ದರೆ, ಕುಡಿಯದೇ ಇದ್ದರೆ, ಜೂಜನ್ನು ಆಡದೆ ಇದ್ದರೆ, ಕಳ್ಳತನ ಮಾಡದೆ ಇದ್ದರೆ ಅಷ್ಟಕ್ಕೆ ಮಾತ್ರವೇ ನಿಮ್ಮನ್ನು ಹೊಗಳಬೇಕಾಗಿಲ್ಲ. ಏಕೆಂದರೆ ಹಾಗಿರಬೇಕಾದದ್ದು ನಿಮ್ಮ ಕರ್ತವ್ಯ. ಇತರರಿಗೆ ಯಾವುದಾದರೂ ಸೇವೆಯನ್ನು ಮಾಡಬೇಕು. ನಿಮಗೆ ನೀವು ಒಳ್ಳೆಯದನ್ನು ಮಾಡಿಕೊಳ್ಳುವಂತೆ ಇತರರಿಗೂ ಒಳ್ಳೆಯದನ್ನು ಮಾಡಬೇಕು.

ಆಹಾರದ ನಿಯಮವನ್ನು ಕುರಿತು ನಿಮಗೆ ಸ್ವಲ್ಪ ಹೇಳುತ್ತೇನೆ. ಹಿಂದಿನ ಆಚಾರಗಳೆಲ್ಲ ಈಗ ಮಾಯವಾಗಿವೆ. ಇವನೊಂದಿಗೆ ಊಟಮಾಡಬಾರದು, ಅವನೊಂದಿಗೆ ಊಟಮಾಡಬಾರದು ಎನ್ನುವುದು ಮಾತ್ರ ನಮ್ಮ ದೇಶದಲ್ಲಿ ಈಗ ಉಳಿದಿದೆ. ನೂರಾರು ವರ್ಷಗಳ ಹಿಂದೆ ನಮಗೆ ಹೇಳಿದ ಒಳ್ಳೆಯ ವಿಷಯಗಳಲ್ಲಿ ಸ್ಪರ್ಶ ದೋಷವೊಂದೇ ಉಳಿದಿರುವ ಅವಶೇಷ. ಮೂರು ವಿಧ ಆಹಾರವನ್ನು ಶಾಸ್ತ್ರಗಳು ನಿಷೇಧಿಸುತ್ತವೆ. ಮೊದಲನೆಯದು ಜಾತಿ ದೋಷ - ಈರುಳ್ಳಿ ಬೆಳ್ಳುಳ್ಳಿ ಮುಂತಾದವು.

ಇವನ್ನು ಹೆಚ್ಚು ಸೇವಿಸಿದರೆ ಕಾಮೋದ್ರೇಕವಾಗಿ ದೇವರ ಮತ್ತು ಮನುಷ್ಯನ ದೃಷ್ಟಿಯಲ್ಲಿ ಅಯೋಗ್ಯವಾದ ಹಲವು ಹೀನಕೃತ್ಯಗಳನ್ನು ಅವನು ಮಾಡಬಹುದು. ಎರಡನೆಯದು ನಿಮಿತ್ತದೋಷ-ಆಹಾರವು ಬಾಹ್ಯ ವಸ್ತುಗಳಿಂದ ಅಶುದ್ಧವಾಗುವುದು. ಆಹಾರವನ್ನು ಇಡುವುದಕ್ಕೆ ಒಳ್ಳೆಯ ಪರಿಶುದ್ಧವಾದ ಸ್ಥಳವನ್ನು ನೋಡಬೇಕು. ಮೂರನೆಯದೇ, ಆಶ್ರಯದೋಷ. ದುಷ್ಟ ಜನರು ತಂದ ಆಹಾರವನ್ನು ಸೇವಿಸಬಾರದು, ಅದನ್ನು ಸ್ವರ್ಶಿಸಿದರೆ ನಮಗೂ ಹೀನಭಾವನೆ ಬರುವುದು. ಒಬ್ಬ ಬ್ರಾಹ್ಮಣನ ಮಗನಾದರೂ ಅವನು ಸ್ತ್ರೀಲಂಪಟ, ಕುಕರ್ಮಿ ಆದರೆ ಅವನಿಂದ ಆಹಾರವನ್ನು ಸೇವಿಸಬಾರದು.

ಈ ಆಚಾರದ ಅರ್ಥವೆಲ್ಲ ಈಗ ಮಾಯವಾಗಿದೆ. ಉನ್ನತ ಜಾತಿಗೆ ಸೇರಿದವರ ಕೈಯಿಂದ ಮಾತ್ರ ಆಹಾರ ಸ್ವೀಕರಿಸಬೇಕು. ಅವರು ಈ ಕುಲಕ್ಕೆ ಸೇರದೆ ಇದ್ದರೆ, ಎಷ್ಟೇ ಧರ್ಮಿಷ್ಠರಾದರೂ ಯೋಗ್ಯರಾದರೂ ಅವರಿಂದ ಸ್ವೀಕರಿಸಕೂಡದು ಎಂಬ ಭಾವನೆ ಮಾತ್ರ ಉಳಿದಿದೆ. ಹಳೆಯ ನಿಯಮಗಳನ್ನೆಲ್ಲ ವ್ಯಾಪಾರಿಗಳು ಅಲ್ಲಗಳೆದಿರುವರು. ಅವರ ಅಂಗಡಿಗಳಲ್ಲಿ ಮಿಠಾಯಿ ಮೇಲೆಲ್ಲ ನೊಣ ಹಾರಾಡುತ್ತಿರುವುದು ಕಾಣಿಸುವುದು. ದಾರಿಯ ಧೂಳೆಲ್ಲ ಅದರ ಮೇಲೆ ಇರುವುದು. ಆ ಮಿಠಾಯಿ ಮಾರುವವನ ಉಡುಪು ಕೂಡ ಶುಭ್ರವಿಲ್ಲ. ಅಂಗಡಿಯಲ್ಲಿ ಮಿಠಾಯಿಯನ್ನು ಗಾಜಿನ ಪೆಟ್ಟಿಗೆಯಲ್ಲಿಡದಿದ್ದರೆ ತೆಗೆದುಕೊಳ್ಳುವುದಿಲ್ಲ ವೆಂದು ಜನರು ಹಟ ಹಿಡಿಯಬೇಕು. ಇದರಿಂದ ಕಾಲರ ಪ್ಲೇಗು ಮುಂತಾದವುಗಳ ರೋಗಾಣುಗಳು ನೊಣಗಳ ಮೂಲಕ ಹರಡದಂತೆ ಮಾಡಬಹುದು. ನಾವು ಇದನ್ನು ಉತ್ತಮಗೊಳಿಸಬೇಕು. ಆದರೆ ಹಾಗೆ ಮಾಡದೆ ಇನ್ನೂ ಅಧೋಗತಿಗೆ ಬಂದಿರುವೆವು. ಮನುವು ನೀರಿನಲ್ಲಿ ಉಗುಳಬೇಡಿ ಎನ್ನುವನು. ಆದರೆ ನಾವು ನದಿಗಳಲ್ಲಿ ಕಸವನ್ನೆಲ್ಲ ಹಾಕುವೆವು. ಇದನ್ನೆಲ್ಲ ನೋಡಿದರೆ ಬಾಹ್ಯಶುದ್ಧಿ ಬಹಳ ಆವಶ್ಯಕವೆಂದು ಕಾಣುವುದು. ಶಾಸ್ತ್ರಕಾರರಿಗೆ ಇದು ಚೆನ್ನಾಗಿ ಗೊತ್ತಿತ್ತು. ಆದರೆ ಈಗ ಆಹಾರ ನಿಯಮದ ಅರ್ಥವೆಲ್ಲ ಮಾಯವಾಗಿ ಬಾಹ್ಯ ಆಚಾರ ಒಂದು ಉಳಿದಿದೆ. ನಮ್ಮ ಜಾತಿಯಲ್ಲಿ ಕಳ್ಳರು, ಕುಡುಕರು, ಕೊಲೆಪಾತಕರು ಇರಬಹುದು. ಆದರೆ ಒಬ್ಬ ಒಳ್ಳೆ ಮನುಷ್ಯ ತನ್ನ ಜಾತಿಗಿಂತ ಕೆಳಗೆ ಇರುವವರೊಂದಿಗೆ ಅವನು ತನ್ನಷ್ಟೇ ಯೋಗ್ಯನಿದ್ದರೂ ಅವನೊಂದಿಗೆ ಊಟಮಾಡಿದರೆ ಅವನನ್ನು ಶಾಶ್ವತವಾಗಿ ಬಹಿಷ್ಕರಿಸುವರು. ಇದೊಂದು ನಮ್ಮ ದೇಶದ ದೊಡ್ಡ ದೋಷ. ಪಾಪಿಗಳೊಂದಿಗೆ ಬಂದಾಗ ಪಾಪಸೋಂಕುವುದು. ಪುಣ್ಯಾತ್ಮರ ಸಂಗದಿಂದ ಪುಣ್ಯ ಬರುವುದು. ಪಾಪಿಗಳಿಂದ ದೂರವಿರುವುದೇ ಬಾಹ್ಯ ಶುದ್ಧಿ ಎಂಬುದನ್ನು ನಾವು ಗಮನದಲ್ಲಿಡಬೇಕು.

ಅಂತರಂಗಶುದ್ಧಿ ಇದಕ್ಕಿಂತ ಬಹಳ ಕಷ್ಟ. ಸತ್ಯಹೇಳುವುದು, ದೀನರ ಸೇವೆ ಮಾಡುವುದು, ಕಷ್ಟದಲ್ಲಿದ್ದವರಿಗೆ ಸಹಾಯ ಮಾಡುವುದು- ಇದರಲ್ಲಿದೆ ಅಂತರಂಗ ಶುದ್ಧಿ. ನಾವು ಯಾವಾಗಲೂ ಸತ್ಯವನ್ನು ಹೇಳುವೆವೆ? ಅನೇಕ ವೇಳೆ ಹೀಗೆ ಆಗುವುದು. ಶ‍್ರೀಮಂತನ ಮನೆಗೆ ತನ್ನ ಸ್ವಂತ ಕೆಲಸಕ್ಕೆ ಹೋಗಿ ‘ದೀನರಕ್ಷಕ’ ಎಂದು ಅವನನ್ನು ಹೊಗಳುವರು. ಅವನು ತನ್ನ ಮನೆಗೆ ಬರುವ ಬಡವನ ನಾಶಮಾಡುವುದು ಇವರಿಗೆ ಗೊತ್ತಿದ್ದರೂ ಹಾಗೆ ಹೇಳುವರು. ಇದೇನು ಸುಳ್ಳಲ್ಲದೆ ಬೇರೆಯಲ್ಲ. ಇದೇ ಮನಸ್ಸನ್ನು ಅಶುದ್ಧಿಗೊಳಿಸುವುದು. ಹನ್ನೆರಡು ವರುಷಗಳಿಂದ ಯಾರು ಹೀನ ಆಲೋಚನೆ ಮಾಡದೆ ತಮ್ಮ ಮನಸ್ಸನ್ನು ಶುದ್ಧ ಮಾಡಿಕೊಂಡಿರುವರೊ ಅವರು ಆಡಿದ ಮಾತು ಸತ್ಯವಾಗುವುದರಲ್ಲಿ ಸಂದೇಹವಿಲ್ಲ. ಇದೇ ಸತ್ಯ ಶಕ್ತಿ. ಯಾರು ತನ್ನ ಬಾಹ್ಯವನ್ನು ಮತ್ತು ಅಂತರಂಗವನ್ನು ಶುದ್ಧಮಾಡಿಕೊಂಡಿರುವನೊ ಅವನೇ ಭಕ್ತ. ಆದರೆ ವಿಶೇಷವೇನೆಂದರೆ, ಭಕ್ತಿಯೇ ಒಬ್ಬನ ಮನಸ್ಸನ್ನು ಎಷ್ಟೋ ಪಾಲು ಪರಿಶುದ್ಧ ಮಾಡುವುದು. ಯಹೂದ್ಯರು, ಮಹಮ್ಮದೀಯರು, ಕ್ರೈಸ್ತರು ಹಿಂದೂಗಳಷ್ಟು ಬಾಹ್ಯ ಶುದ್ಧಿಗೆ ಪ್ರಾಮುಖ್ಯವನ್ನು ಕೊಡದೆ ಇದ್ದರೂ ಅದು ಒಂದಲ್ಲ ಒಂದು ರೂಪದಲ್ಲಿ ಇದ್ದೇ ಇರುವುದು. ಅದೂ ಕೂಡ ಸ್ವಲ್ಪ ಆವಶ್ಯಕ ಎಂದು ಅವರೂ ಭಾವಿಸುವರು. ಯಹೂದ್ಯರು ವಿಗ್ರಹರಾಧನೆಯನ್ನು ದೂರುವರು. ಆದರೆ ಅವರ ಗುಡಿಯಲ್ಲಿ ಒಂದು ಪೆಟ್ಟಿಗೆ ಇರುತ್ತದೆ. ಅದನ್ನು ಆರ್ಕ್​ ಎನ್ನುತ್ತಾರೆ. ಅದರಲ್ಲಿ ಅವರ ಧರ್ಮದ ನಿಯಮಗಳನ್ನು ಇಟ್ಟಿರುತ್ತಾರೆ. ಆ ಪೆಟ್ಟಿಗೆ ಮೇಲೆ ಇಬ್ಬರು ದೇವತೆಗಳು ಗರಿಗೆದರಿ ನಿಂತಿರುವರು. ಅವರ ಮಧ್ಯೆ ಭಗವಂತನ ಶಕ್ತಿ ಒಂದು ಮೋಡದಂತೆ ಆವಿರ್ಭವಿಸುವುದು ಎಂದು ಅವರು ಭಾವಿಸುವರು. ಆ ದೇವಸ್ಥಾನವನ್ನು ಎಂದೋ ನಾಶಮಾಡಿದರೂ ಹೊಸ ಗುಡಿಯನ್ನು ಅದರಂತೆಯೇ ಕಟ್ಟುತ್ತಾರೆ ಮತ್ತು ಪೆಟ್ಟಿಗೆಯಲ್ಲಿ ಅವರ ಧರ್ಮಗ್ರಂಥವನ್ನು ಇಡುತ್ತಾರೆ. ರೋಮನ್​ ಕ್ಯಾಥೊಲಿಕ್​ ಮತ್ತು ಗ್ರೀಕ್​ ಕ್ರೈಸ್ತರಲ್ಲಿ ಒಂದು ಬಗೆಯ ವಿಗ್ರಹಾರಾಧನೆ ಇದೆ. ಜೀಸಸ್​ ಮತ್ತು ಆವನ ತಾಯಿಯ ವಿಗ್ರಹಗಳನ್ನು ಪೂಜಿಸುವರು. ಪ್ರಾಟೆಸ್ಟೆಂಟರಲ್ಲಿ ವಿಗ್ರಹಾರಾಧನೆ ಇಲ್ಲ. ಆದರೂ ದೇವರನ್ನು ಅವರು ಒಂದು ವ್ಯಕ್ತಿಯಂತೆ ಉಪಾಸಿಸುವರು. ಇದನ್ನು ಬೇರೆ ವಿಧದ ವಿಗ್ರಹರಾಧನೆ ಎನ್ನಬಹುದು. ಪಾರ್ಸಿಗಳಲ್ಲಿ ಮತ್ತು ಇರಾನೀಯರಲ್ಲಿ ಬೆಂಕಿಯ ಪೂಜೆಯನ್ನು ಮಾಡುವುದು ಬಹಳ ರೂಢಿಯಲ್ಲಿದೆ. ಮಹಮ್ಮದೀಯರಲ್ಲಿ ಪ್ರವಾದಿಗಳನ್ನೂ ಇತರ ಮಹಾತ್ಮರನ್ನು ಪೂಜಿಸುವರು. ಪ್ರಾರ್ಥಿಸುವಾಗ ಕಾಬಾದ ಕಡೆ ತಮ್ಮ ಮುಖವನ್ನು ತಿರುಗಿಸುವರು. ಧರ್ಮದ ಆರಂಭದಲ್ಲಿ ಜನರು ಬಾಹ್ಯದಲ್ಲಿ ಯಾವುದನ್ನಾದರೂ ಸ್ವೀಕರಿಸಬೇಕು ಎನ್ನುವುದನ್ನು ಇದು ತೋರುವುದು. ಮನಸ್ಸು ಶುದ್ಧವಾದ ಮೇಲೆ ಸೂಕ್ಷ್ಮ ಭಾವನೆಗಳನ್ನು ಸ್ವೀಕರಿಸಬಹುದು. “ಪೂಜೆಗಳಲ್ಲಿ ಅತ್ಯುತ್ತಮವಾದುದೇ ಜೀವವು ಬ್ರಹ್ಮದೊಡನೆ ಒಂದಾಗುವುದು. ಮಧ್ಯಮವಾದುದೆಂದರೆ ಧ್ಯಾನ, ಅಧಮವಾದುದೆಂದರೆ ಜಪ, ಅಧಮಾಧಮವಾದುದೆಂದರೆ ಬಾಹ್ಯಪೂಜೆ.” ಕೊನೆಯದನ್ನು ಅನುಸರಿಸಿದರೂ ಪಾಪವಿಲ್ಲವೆಂದು ತಿಳಿಯಬೇಕು. ಪ್ರತಿಯೊಬ್ಬರೂ ತಮಗೆ ಸಾಧ್ಯವಾದುದನ್ನು ಮಾಡಬೇಕು. ಅದನ್ನು ತಪ್ಪಿಸಿದರೆ ಬೇರೊಂದು ರೀತಿಯಿಂದ ಅದನ್ನೇ ಮಾಡುವರು. ವಿಗ್ರಹಾರಾಧನೆ ಮಾಡುವವನನ್ನು ನಿಂದಿಸಕೂಡದು. ಅವನು ಬೆಳವಣಿಗೆಯ ಆ ಹಂತದಲ್ಲಿ ಇರುವನು. ಆದುದರಿಂದ ಅವನು ಅದನ್ನು ಇಟ್ಟುಕೊಳ್ಳಲಿ. ಬುದ್ಧಿವಂತರು ಅಂತಹವರಿಗೆ ಸಹಾಯ ಮಾಡಿ ಉತ್ತಮ ಆದರ್ಶದೆಡೆಗೆ ಅವನನ್ನು ಒಯ್ಯಬೇಕು. ಹಲವು ಬಗೆಯ ಪೂಜಾ ವಿಧಾನಗಳನ್ನು ಕುರಿತು ಜಗಳವಾಡಿ ಪ್ರಯೋಜನವಿಲ್ಲ.

ಕೆಲವರು ಐಶ್ವರ್ಯಕ್ಕಾಗಿ ದೇವರನ್ನು ಪ್ರೀತಿಸುವರು. ಮತ್ತೆ ಕೆಲವರು ಮಕ್ಕಳಿಗಾಗಿ ಪೂಜಿಸುವರು. ತಾವೂ ಕೂಡ ಭಕ್ತರು ಎಂದು ಭಾವಿಸುವರು. ಇದು ಭಕ್ತಿಯೂ ಅಲ್ಲ, ಅವರು ನಿಜವಾದ ಭಕ್ತರೂ ಅಲ್ಲ. ಒಬ್ಬ ಸಾಧುವು ಚಿನ್ನ ಮಾಡುತ್ತೇನೆ ಎಂದರೆ ಎಲ್ಲರೂ ಅವನ ಹತ್ತಿರ ಓಡುವರು. ಆದರೂ ತಾವು ಭಕ್ತರು ಎಂದು ಭಾವಿಸುವರು. ಮಕ್ಕಳಿಗಾಗಿ ದೇವರನ್ನು ಪೂಜಿಸುವುದು ಭಕ್ತಿಯಲ್ಲ, ಐಶ್ವರ್ಯಕ್ಕಾಗಿ ದೇವರನ್ನು ಪೂಜಿಸುವುದು ಭಕ್ತಿಯಲ್ಲ, ಸ್ವರ್ಗಕ್ಕಾಗಿ ದೇವರನ್ನು ಪೂಜಿಸುವುದು ಭಕ್ತಿಯಲ್ಲ. ನರಕಾಗ್ನಿಯಿಂದ ಪಾರಾಗುವುದಕ್ಕಾಗಿ ದೇವರನ್ನು ಪೂಜಿಸುವುದು ಭಕ್ತಿಯಲ್ಲ. ಭಯ ಅಥವಾ ಲೋಭದಿಂದ ಹುಟ್ಟುವಂಥದೂ ಭಕ್ತಿಯಲ್ಲ. ನಿಜವಾದ ಭಕ್ತ “ದೇವರೇ, ನನಗೆ ಸುಂದರ ಸತಿ ಬೇಕಾಗಿಲ್ಲ, ಮುಕ್ತಿ ಬೇಕಾಗಿಲ್ಲ. ನಾನು ನೂರಾರು ಸಲ ಹುಟ್ಟಿ ಸತ್ತರೂ ಚಿಂತೆಯಿಲ್ಲ. ಯಾವಾಗಲೂ ನಿನ್ನ ಮೇಲೆ ಅಚಲ ಭಕ್ತಿಯೊಂದನ್ನು ಮಾತ್ರ ಕರುಣಿಸು” ಎಂದು ಪ್ರಾರ್ಥಿಸುವನು. ಎಲ್ಲದರಲ್ಲಿಯೂ ಯಾವಾಗ ಒಬ್ಬನು ದೇವರನ್ನು ಕಾಣುತ್ತಾನೆಯೋ, ಎಲ್ಲವನ್ನೂ ದೇವರಲ್ಲಿ ನೋಡುತ್ತಾನೆಯೋ, ಆಗ ಅವನಿಗೆ ಪೂರ್ಣಭಕ್ತಿಯು ಉದಿಸಿದೆ ಎಂದರ್ಥ. ಆಗ ಅವನಿಗೆ ಕೀಟದಿಂದ ಬ್ರಹ್ಮನವರೆಗೆ ಎಲ್ಲವೂ ವಿಷ್ಣುಮಯವಾಗಿ ಕಾಣುವುದು. ಆಗ ದೇವರಿಲ್ಲದೆ ಇರುವುದು ಯಾವುದೂ ಇಲ್ಲ. ಆಗ ಮಾತ್ರ ತಾನು ಕ್ಷುದ್ರಜೀವಿ ಎಂದು ಭಾವಿಸಿ ನಿಜವಾದ ಭಕ್ತನಂತೆ ದೇವರನ್ನು ಪೂಜಿಸುವನು, ಆಗ ತೀರ್ಥ, ಬಾಹ್ಯ ಪೂಜೆ ಮೊದಲಾದ ಎಲ್ಲವನ್ನೂ ಬಹಳ ಹಿಂದೆ ತೊರೆಯುವನು. ಆಗ ಪ್ರತಿಯೊಂದು ಜೀವಿಯೂ ಭಗವಂತನ ಪರಿಪೂರ್ಣ ದೇವಾಲಯದಂತೆ ಕಾಣುವುದು.

ಶಾಸ್ತ್ರಗಳು ಭಕ್ತಿಯನ್ನು ಹಲವು ರೀತಿಯಲ್ಲಿ ವಿವರಿಸುತ್ತದೆ. ದೇವರನ್ನು ನಮ್ಮ ತಂದೆ ಎನ್ನುವೆವು. ಹಾಗೆಯೇ, ತಾಯಿ ಮುಂತಾಗಿ ಕರೆಯುವೆವು. ನಮ್ಮಲ್ಲಿರುವ ಭಕ್ತಿ ವೃದ್ಧಿಯಾಗುವುದಕ್ಕೆ, ನಾವು ಅವನಿಗೆ ಅವನ ಬಹಳ ಸಮೀಪದ ಬಂಧುಗಳು, ಪ್ರೇಮಕ್ಕೆ ಪಾತ್ರರು, ಎಂಬುದನ್ನು ತಿಳಿಯುವುದಕ್ಕೆ ಈ ಭಾವನೆಯನ್ನು ಆರೋಪಿಸಿಕೊಳ್ಳುತ್ತೇವೆ. ಇದನ್ನು ನಾವು ಒಂದು ರೀತಿಯಲ್ಲಿ ಸಮರ್ಥಿಸಬಹುದು. ನಿಜವಾದ ಭಕ್ತನು ಭಗವಂತನನ್ನು ಪ್ರೀತಿಯಿಂದ ಕರೆಯುವ ಪದಗಳು ಇವು. ರಾಸಲೀಲೆಯಲ್ಲಿ ಬರುವ ರಾಧಾಕೃಷ್ಣರ ಕಥೆಯನ್ನು ತೆಗೆದುಕೊಳ್ಳಿ. ಭಕ್ತನ ನಿಜವಾದ ಸ್ಥಿತಿಯನ್ನು ಇದು ವಿವರಿಸುವುದು. ಪ್ರಪಂಚದಲ್ಲಿ ಯಾವ ಪ್ರೀತಿಯೂ ಗಂಡಸಿಗೂ ಹೆಂಗಸಿಗೂ ಇರುವ ಪ್ರೀತಿಯನ್ನು ಮೀರಿಸಲಾರದು. ಎಲ್ಲಿ ಇಂತಹ ತೀವ್ರ ಪ್ರೇಮವಿದೆಯೋ, ಅಲ್ಲಿ ಅಂಜಿಕೆ ಇಲ್ಲ. ಪ್ರಣಯಿ - ಪ್ರಣಯಿನಿಯರನ್ನು ಎಂದೆಂದಿಗೂ ಅಗಲದಂತೆ ಬಂಧಿಸುವ ಬಂಧನವಲ್ಲದೆ ಮತ್ತಾವುದೂ ಇರುವುದಿಲ್ಲ. ಆದರೆ ನಮಗೆ ತಾಯಿ ತಂದೆಗಳ ಮೇಲೆ ಇರುವ ಪ್ರೀತಿಯಲ್ಲಿ ಅವರ ಮೇಲೆ ಗೌರವದಿಂದ ಸ್ವಲ್ಪ ಅಂಜಿಕೆ ಇರುವುದು. ದೇವರು ಏನನ್ನಾದರೂ ಸೃಷ್ಟಿಸಿದನೋ ಇಲ್ಲವೋ, ಅವನು ನನ್ನನ್ನು ಶಿಕ್ಷಿಸುವನೋ ಇಲ್ಲವೋ ಆ ವಿಷಯವೇ ಬೇಕಾಗಿಲ್ಲ. ಅವನು ನಮ್ಮ ಪ್ರಿಯತಮ. ಎಲ್ಲಾ ಅಂಕೆ ಅಂಜಿಕೆಗಳಿಂದ ಪಾರಾಗಿ ಅವನನ್ನು ಪ್ರೀತಿಸಬೇಕು. ಮತ್ತಾವ ಆಸೆಯೂ ಇಲ್ಲದೆ, ಮತ್ತಾವ ಆಲೋಚನೆಯೂ ಇಲ್ಲದೆ, ದೇವರನ್ನು ಕುರಿತು ಹುಚ್ಚನಾದಾಗ ಮಾತ್ರ ಅವನು ದೇವರನ್ನು ಪ್ರೀತಿಸುವನು. ಪ್ರಿಯನಿಗೆ ಪ್ರಿಯೆಯ ಮೇಲೆ ಇರುವ ಪ್ರೇಮ, ದೇವರ ಮೇಲೆ ಇರಬೇಕಾದ ಪ್ರೇಮಕ್ಕೆ ಉದಾಹರಣೆಯಾಗಿದೆ. ಕೃಷ್ಣನು ದೇವರು. ರಾಧೆ ಅವನನ್ನು ಪ್ರೀತಿಸುವಳು. ಈ ಕಥೆಯನ್ನು ವಿವರಿಸುವ ಪುಸ್ತಕವನ್ನು ಓದಿ. ಆಗ ದೇವರನ್ನು ಹೇಗೆ ಪ್ರೀತಿಸಬೇಕು ಎನ್ನುವುದು ಗೊತ್ತಾಗುವುದು. ಆದರೆ ಎಷ್ಟು ಜನಕ್ಕೆ ಇದು ಅರ್ಥವಾಗುವುದು? ಯಾರು ಪಾಪಾತ್ಮರೋ, ಯಾರಲ್ಲಿ ನೀತಿಯ ಭಾವನೆಯೇ ಇಲ್ಲವೋ, ಅವರು ಇದನ್ನು ಹೇಗೆ ತಿಳಿದುಕೊಳ್ಳಬಲ್ಲರು? ಪ್ರಾಪಂಚಿಕ ಆಲೋಚನೆಯನ್ನೆಲ್ಲ ಮನಸ್ಸಿನಿಂದ ಓಡಿಸಿ, ಶುದ್ಧ ಧಾರ್ಮಿಕ ಆಧ್ಯಾತ್ಮಿಕ ವಾತಾವರಣದಲ್ಲಿದ್ದರೆ ಆಗ ಮಾತ್ರ ಅವರು ಅವಿದ್ಯಾವಂತರಾಗಿದ್ದರೂ, ಇಂತಹ ಅತಿಸೂಕ್ಷ್ಮ ವಿಷಯಗಳನ್ನು ತಿಳಿದುಕೊಳ್ಳಬಲ್ಲರು. ಆದರೆ ಇಂತಹ ಸ್ವಭಾವದವರು ಎಷ್ಟು ಕಡಮೆ. ಮನುಷ್ಯನು ವಿಕೃತಗೊಳಿಸಲಾಗದ ಒಂದೇ ಒಂದು ಧರ್ಮವೂ ಜಗತ್ತಿನಲ್ಲಿಲ್ಲ. ಉದಾಹರಣೆಗೆ, ಆತ್ಮವು ದೇಹದಿಂದ ಸಂಪೂರ್ಣ ಬೇರೆ, ದೇಹದಿಂದ ಪಾಪಮಾಡುತ್ತಿರುವಾಗ ಆತ್ಮನಿಗೆ ಇದು ಸೋಂಕುವುದಿಲ್ಲ ಎಂದು ತಿಳಿಯಬಹುದು. ಧರ್ಮವನ್ನು ನಿಜವಾಗಿ ಅನುಷ್ಠಾನಕ್ಕೆ ತಂದಿದ್ದರೆ, ಹಿಂದೂ, ಮಹಮ್ಮದೀಯ, ಕ್ರೈಸ್ತ, ಇವರಲ್ಲಿ ಯಾರೊಬ್ಬರೂ ಪರಿಶುದ್ಧರಾಗದೆ ಇರುತ್ತಿರಲಿಲ್ಲ. ಆದರೆ ಜನರು ಒಳ್ಳೆಯದೋ, ಕೆಟ್ಟದ್ದೊ, ತಮ್ಮ ಸ್ವಭಾವವನ್ನು ಅನುಸರಿಸುವರು. ಇದನ್ನು ಒಪ್ಪಿಕೊಳ್ಳದೇ ವಿಧಿಯಿಲ್ಲ. ಆದರೂ ಪ್ರಪಂಚದಲ್ಲಿ ಕೆಲವರು ಇರುವರು. ದೇವರ ಹೆಸರನ್ನು ಕೇಳಿದೊಡನೆಯೇ ಭಾವೋನ್ಮತ್ತರಾಗುವರು. ಅವನ ವಿಷಯವನ್ನು ಓದುವಾಗ ಸಂತೋಷದಿಂದ ಕಂಬನಿಗಳನ್ನು ಸುರಿಸುವರು. ಇವರೇ ನಿಜವಾದ ಭಕ್ತರು.

ಧರ್ಮದ ಪ್ರಾರಂಭಾವಸ್ಥೆಯಲ್ಲಿ ಭಕ್ತನು ದೇವರನ್ನು ತನ್ನ ಸ್ವಾಮಿ ಎಂದೂ ತಾನು ಆತನ ಸೇವಕನೆಂದೂ ಯೋಚಿಸುವನು. ತನ್ನ ಜೀವನಕ್ಕೆ ಆಧಾರವಾದುದನ್ನೆಲ್ಲ ಕರುಣಿಸಿದುದಕ್ಕೆ ಅವನಿಗೆ ಚಿರಋಣಿಯಾಗಿರುವನು. ಇಂತಹ ಭಾವನೆಯನ್ನು ಕಿತ್ತೊಗೆಯಿರಿ. ಎಲ್ಲರನ್ನೂ ಆಕರ್ಷಿಸುವ ಶಕ್ತಿಯೊಂದಿದೆ - ಅದೇ ದೇವರು. ಆ ಆಕರ್ಷಣೆಗೆ ತಲೆಬಾಗಿ ಸೂರ್ಯ ಚಂದ್ರಾದಿಗಳು ಸಂಚರಿಸುವರು. ಪ್ರಪಂಚದಲ್ಲಿರುವ ಪ್ರತಿಯೊಂದೂ, ಒಳ್ಳೆಯದು ಕೆಟ್ಟದ್ದು ಎಲ್ಲಾ, ದೇವರಿಗೆ ಸೇರಿದುದು. ಪ್ರಪಂಚದಲ್ಲಿ ನಮಗೆ ಏನಾದರೂ ಆಗಲೀ, ಒಳ್ಳೆಯದು ಕೆಟ್ಟದ್ದು, ಎಲ್ಲವೂ ನಮ್ಮನ್ನು ಅವನ ಹತ್ತಿರ ಕರೆದುಕೊಂಡು ಹೋಗುತ್ತಿವೆ. ಒಬ್ಬ ಮತ್ತೊಬ್ಬನನ್ನು ಕೊಲ್ಲುತ್ತಾನೆ. ಯಾವುದೋ ಸ್ವಾರ್ಥದಿಂದ ಪ್ರೇರೇಪಿತನಾಗುತ್ತಾನೆ. ಆದರೆ ಅದರ ಉದ್ದೇಶ ಪ್ರೀತಿ. ಅದು ತನಗೋ, ಅಥವಾ ಮತ್ತೊಬ್ಬರಿಗೋ ಇರಬಹುದು. ನಾವು ಒಳ್ಳೆಯದನ್ನಾದರೂ ಮಾಡಲಿ ಅಥವಾ ಕೆಟ್ಟದ್ದನ್ನು ಮಾಡಲಿ, ಅದರ ಮೂಲಕಾರಣ ಪ್ರೀತಿ. ಒಂದು ಹುಲಿ ಎಮ್ಮೆಯೊಂದನ್ನು ಕೊಂದರೆ ತನಗೋ ತನ್ನ ಮರಿಗೋ ಹಸಿವಾಗಿದೆ, ಅದಕ್ಕಾಗಿ ಕೊಲ್ಲುತ್ತದೆ.

ದೇವರು ಪ್ರೇಮಸ್ವರೂಪನು. ಎಲ್ಲದರಲ್ಲಿಯೂ ಅವನು ಇರುವನು. ಪ್ರತಿಯೊಬ್ಬರಿಗೂ ಗೊತ್ತಾಗಲಿ, ಗೊತ್ತಾಗದೆ ಇರಲಿ, ಎಲ್ಲರೂ ದೇವರೆಡೆಗೆ ಆಕರ್ಷಿತ ರಾಗುತ್ತಿರುವರು. ಹೆಂಡತಿ ತನ್ನ ಗಂಡನನ್ನು ಪ್ರೀತಿಸಿದರೆ ತನ್ನ ಗಂಡನಲ್ಲಿರುವ ದೈವೀಭಾವನೆ ಈ ಆಕರ್ಷಣೆಗೆ ಕಾರಣ ಎನ್ನುವುದನ್ನು ಅವಳು ಅರಿಯಳು. ಪ್ರೇಮೇಶ್ವರನು ಮಾತ್ರ ಪೂಜಾಯೋಗ್ಯನು. ಎಲ್ಲಿಯವರೆಗೂ ಅವನನ್ನು ಸೃಷ್ಟಿ ಸ್ಥಿತಿ ಪಾಲಕ ಎಂದು ನೋಡುತ್ತಿರುವೆವೋ ಅಲ್ಲಿಯವರೆಗೆ ಬಾಹ್ಯಪೂಜೆಯನ್ನು ಮಾಡಬಹುದು. ಆದರೆ ಎಂದು ಇದನ್ನು ಮೀರಿ ಅವನು ಪ್ರೇಮೇಶ್ವರನೆಂದು ಭಾವಿಸಿ, ಎಲ್ಲಾ ವಸ್ತುಗಳಲ್ಲಿ ಅವನನ್ನು ಕಂಡು ಅವನಲ್ಲಿ ಎಲ್ಲಾ ವಸ್ತುಗಳನ್ನೂ ನೋಡುವೆವೋ ಆಗ ಪರಮಭಕ್ತಿ ಪ್ರಾಪ್ತವಾಗುವುದು.

