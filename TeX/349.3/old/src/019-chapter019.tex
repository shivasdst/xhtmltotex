
\chapter{ಆಲ್ಮೋರದ ಬಿನ್ನವತ್ತಳೆಗೆ ಉತ್ತರ}

ಸ್ವಾಮೀಜಿ ಆಲ್ಮೋರಕ್ಕೆ ಬಂದಾಗ ಅಲ್ಲಿನ ನಾಗರಿಕರು ಅವರಿಗೆ ಹಿಂದಿ ಭಾಷೆಯಲ್ಲಿ ಒಂದು ಬಿನ್ನವತ್ತಳೆಯನ್ನು ಅರ್ಪಿಸಿದರು. ಈ ಕೆಳಗಿನದು ಅದರ ಅನುವಾದ.

\vskip 6pt

\textbf{ಮಹಾನ್​ ಚೇತನವೇ,}

\vskip 6pt

ತಾವು ಪಶ್ಚಿಮದಲ್ಲಿ ಆಧ್ಯಾತ್ಮಿಕ ವಿಜಯವನ್ನು ಸಾಧಿಸಿ, ಇಂಗ್ಲೆಂಡನ್ನು ಬಿಟ್ಟು ಸ್ವದೇಶಾಭಿಮುಖವಾಗಿ ಹೊರಟಿದ್ದೀರಿ ಎಂಬುದನ್ನು ಕೇಳಿದಾಗಿನಿಂದಲೂ ನಾವು ತಮ್ಮ ದರ್ಶನಾಕಾಂಕ್ಷಿಗಳಾಗಿದ್ದೇವೆ. ಆ ಸರ್ವಶಕ್ತನ ಕೃಪೆಯಿಂದ ಈಗ ಅದು ನಮಗೆ ಲಭ್ಯವಾಗಿದೆ. ಭಕ್ತಚಕ್ರವರ್ತಿಯೂ ಮಹಾಕವಿಯೂ ಆದ ತುಲಸೀದಾಸರ, “ಯಾರನ್ನಾದರೂ ಒಬ್ಬನು ತೀವ್ರವಾಗಿ ಪ್ರೀತಿಸಿದರೆ ಅವನನ್ನು ಕಂಡೇ ಕಾಣುತ್ತಾನೆ” ಈ ಮಾತುಗಳು ಇಂದು ಪೂರ್ಣವಾಗಿ ಸತ್ಯವಾಗಿದೆ. ಇಂದು ನಾವಿಲ್ಲಿ ತಮ್ಮನ್ನು ವಿಶ್ವಾಸಪೂರ್ವಕವಾದ ಭಕ್ತಿಯಿಂದ ಸೇರಿದ್ದೇವೆ. ತಾವು ಮತ್ತೆ ಈ ಊರಿಗೆ ಭೇಟಿಕೊಡುವುದರ ಮೂಲಕ ನಮ್ಮ ಮೇಲೆ ಕೃಪೆದೋರಿದ್ದೀರಿ. ತಮ್ಮ ಈ ಕೃಪೆಗಾಗಿ ನಾವು ತಮಗೆ ಎಷ್ಟು ಕೃತಜ್ಞರಾಗಿದ್ದರೂ ಸಾಲದು. ತಾವು ಧನ್ಯರು. ತಮಗೆ ಯೋಗ ದೀಕ್ಷೆಯನ್ನಿತ್ತ ತಮ್ಮ ಗುರುದೇವರೂ ಧನ್ಯರು. ಈ ಭಯಂಕರವಾಗಿರುವ ಕಲಿಯುಗದಲ್ಲೂ ತಮ್ಮಂತಹ ಆರ್ಯಕುಲದ ನಾಯಕರಿರುವ ಭಾರತವು ಧನ್ಯ. ತಾವು ತಮ್ಮ ಸರಳತೆ, ಪ್ರಾಮಾಣಿಕತೆ, ಚಾರಿತ್ರ್ಯ, ಪರೋಪಕಾರ, ಕಠಿಣವಾದ ಸಂಯಮ, ನಡತೆ, ಜ್ಞಾನಬೋಧೆಗಳಿಂದ ಜಗತ್ತಿನಾದ್ಯಂತ ಈ ಸಣ್ಣ ವಯಸ್ಸಿನಲ್ಲಿಯೇ ಇಷ್ಟು ಪರಿಶುದ್ಧವಾದ ಕೀರ್ತಿಯನ್ನು ಸಂಪಾದಿಸಿರುವುದು ನಮ್ಮೆಲ್ಲರಿಗೂ ಹೆಮ್ಮೆಯ ವಿಷಯವಾಗಿದೆ.

\vskip 7pt

ತಾವು ವಾಸ್ತವವಾಗಿ, ಶ‍್ರೀ ಶಂಕರಾಚಾರ್ಯರ ಕಾಲದಿಂದಲೂ ಯಾರೂ ಕೈಗೊಳ್ಳದಿದ್ದ ಅತ್ಯಂತ ಕಠಿಣವಾದ ಕಾರ್ಯವನ್ನು ಕೈಗೊಂಡಿದ್ದೀರಿ. ಆರ್ಯ ವಂಶೋದ್ಭವರೊಬ್ಬರು, ತಮ್ಮ ತಪಶ್ಶಕ್ತಿಯಿಂದ, ಇಂಗ್ಲೆಂಡ್​ ಮತ್ತು ಅಮೆರಿಕ ದೇಶಗಳ ವಿದ್ವಾಂಸರಿಗೆ, ಪ್ರಾಚೀನ ಭಾರತದ ಧರ್ಮವು ಇತರ ದೇಶಗಳ ಧರ್ಮಗಳಿಗಿಂತ ಶ್ರೇಷ್ಠವಾದುದು ಎಂದು ಸಾಧಿಸಿ ತೋರಿಸುತ್ತಾರೆ, ಎಂಬುದನ್ನು ನಮ್ಮಲ್ಲಿ ಯಾರು ತಾನೆ ಕನಸಿನಲ್ಲಾದರೂ ಊಹಿಸುವುದು ಸಾಧ್ಯವಾಗುತ್ತಿತ್ತು! ಚಿಕಾಗೊದಲ್ಲಿ ನಡೆದ ಸರ್ವಧರ್ಮ ಸಮ್ಮೇಳನದಲ್ಲಿ, ವಿವಿಧ ಧರ್ಮಗಳ ಪ್ರತಿನಿಧಿಗಳ ಮುಂದೆ ನಿಂತು, ಭಾರತದ ಪ್ರಾಚೀನ ಧರ್ಮದ ಮೇಲ್ಮೆಯನ್ನು ತಾವು ಸಮರ್ಥವಾಗಿ ಪ್ರತಿಪಾದಿಸಿ, ಆ ಜನರ ಕಣ್ಣುಗಳನ್ನು ತೆರೆಸಿದ್ದೀರಿ. ಆ ಮಹಾಸಭೆಯಲ್ಲಿ ವಿದ್ವಾಂಸರಾದ ಉಪನ್ಯಾಸಕರು ತಮ್ಮ ತಮ್ಮ ಧರ್ಮಗಳನ್ನು ತಮ್ಮದೇ ಆದ ರೀತಿಯಲ್ಲಿ ಸಮರ್ಥಿಸಿಕೊಂಡರು. ತಾವಾದರೋ ಅವರೆಲ್ಲರನ್ನೂ ಮೀರಿಸಿದ್ದೀರಿ. ಯಾವುದೇ ಧರ್ಮವೂ ವೈದಿಕ ಧರ್ಮದೊಂದಿಗೆ ಸ್ಪರ್ಧಿಸಲಾರದು ಎಂಬುದನ್ನು ತಾವು\break ಸಂಪೂರ್ಣವಾಗಿ ಸ್ಥಾಪಿಸಿದ್ದೀರಿ. ಅಷ್ಟೇ ಅಲ್ಲ, ಅಮೆರಿಕ ಮತ್ತು ಯುರೋಪ್​ ಖಂಡಗಳಲ್ಲಿನ ಬೇರೆ ಬೇರೆ ಊರುಗಳಲ್ಲಿ ಈ ಪ್ರಾಚೀನ ಧರ್ಮವನ್ನು ಬೋಧಿಸಿ, ತಾವು ಅಲ್ಲಿನ ಅನೇಕ ವಿದ್ವಾಂಸರ ಗಮನವನ್ನು ಪ್ರಾಚೀನವಾದ ಆರ್ಯಧರ್ಮ ಮತ್ತು ತತ್ತ್ವ ಶಾಸ್ತ್ರಗಳ ಕಡೆಗೆ ಸೆಳೆದಿದ್ದೀರಿ. ಇಂಗ್ಲೆಂಡಿನಲ್ಲಿ ಕೂಡ ತಾವು ನಮ್ಮ ಪ್ರಾಚೀನ ಧರ್ಮದ ಧ್ವಜವನ್ನು ನೆಟ್ಟಿದ್ದೀರಿ. ಅದನ್ನು ತೆಗೆದು ಹಾಕುವುದು ಅಸಾಧ್ಯ.

\vskip 7pt

ಇದುವರೆಗೆ ಆಧುನಿಕ ನಾಗರಿಕತೆಯನ್ನುಳ್ಳ ರಾಷ್ಟ್ರಗಳಾದ ಯೂರೋಪ್​ ಮತ್ತು ಅಮೆರಿಕಗಳು ನಮ್ಮ ಧರ್ಮದ ನಿಜವಾದ ಸ್ವರೂಪದ ವಿಷಯಲ್ಲಿ ಏನೂ ತಿಳುವಳಿಕೆಯನ್ನು ಹೊಂದಿರಲಿಲ್ಲ. ಆದರೆ ತಾವು ತಮ್ಮ ಆಧ್ಯಾತ್ಮಿಕ ಬೋಧನೆಯಿಂದ ಅವರ ಕಣ್ಣುಗಳನ್ನು ತೆರೆದಿದ್ದೀರಿ. ಆದ್ದರಿಂದ ಯಾವ ದೇಶವನ್ನು ಅವರು ತಮ್ಮ ಅಜ್ಞಾನದಿಂದ “ಅಹಂಕಾರದಿಂದ ತುಂಬಿದ ಜನರ ಸೂಕ್ಷ್ಮತೆಗಳಿಂದ ಕೂಡಿದ ಧರ್ಮ ಅಥವಾ ದಡ್ಡರಿಗಾಗಿ ನೀಡಿದ ಬೋಧನೆಗಳ ರಾಶಿ” ಎಂದು ಭಾವಿಸಿದ್ದರೊ ಆ ದೇಶವು ರತ್ನಗಳ ಗಣಿ ಎಂಬುದಾಗಿ ಅರಿತುಕೊಂಡರು. “ಧರ್ಮಶೀಲನೂ ಪ್ರಾಜ್ಞನೂ ಆದ ಒಬ್ಬ ಮಗನಿರುವುದು ನೂರಾರು ಜನ ಮೂರ್ಖರಿರುವುದಕ್ಕಿಂತ ಉತ್ತಮ” ಎಂಬುದು ಸತ್ಯ. “ಎಲ್ಲ ನಕ್ಷತ್ರಗಳನ್ನು ಒಟ್ಟಾರೆ ಹಾಕಿದರೂ ಅವು ಕತ್ತಲನ್ನು ಹೋಗಲಾಡಿಸಲಾರವು. ಚಂದ್ರನು ಮಾತ್ರ ತನ್ನ ಬೆಳಕಿನಿಂದ ಕತ್ತಲನ್ನು ಓಡಿಸಬಲ್ಲ.” ತಮ್ಮಂತಹ ಸತ್ಪುರುಷರಿಂದ ಮಾತ್ರ ಜಗತ್ತಿಗೆ ನಿಜವಾದ ಪ್ರಯೋಜನವು ದೊರಕುತ್ತದೆ. ತನ್ನ ಅವನತಿ ಸ್ಥಿತಿಯಲ್ಲಿರುವ ಭಾರತಮಾತೆಗೆ ತಮ್ಮಂತಹ ಪುತ್ರರಿರುವುದೇ ಒಂದು ದೊಡ್ಡ ಸಂತೈಕೆಯ ವಿಷಯವಾಗಿದೆ. ಎಷ್ಟೋ ಜನರು ಕಡಲನ್ನು ದಾಟಿ ಗೊತ್ತುಗುರಿಗಳಿಲ್ಲದೆ ಅತ್ತ ಇತ್ತ ಅಲೆದಾಡಿದ್ದಾರೆ. ಆದರೆ ತಾವು ಮಾತ್ರ, ತಮ್ಮ ಪೂರ್ವಜನ್ಮದ ಸುಕೃತದಿಂದ, ಕಡಲಿನಾಚೆ ನಮ್ಮ ಧರ್ಮದ ಮಹಿಮೆಯನ್ನು ಸಾರಿದ್ದೀರಿ. ಮನೋವಾಕ್ಕಾಯಗಳಿಂದ ಮಾನವತೆಗೆ ಆಧ್ಯಾತ್ಮಿಕ ಬೋಧನೆಯನ್ನು ನೀಡುವುದನ್ನೇ ತಮ್ಮ ಜೀವನದ ಪರಮಗುರಿಯನ್ನಾಗಿ ತಾವು ಮಾಡಿಕೊಂಡಿದ್ದೀರಿ.

\vskip 6pt

ತಾವು ಇಲ್ಲಿ ಒಂದು ಮಠವನ್ನು ಸ್ಥಾಪಿಸುವ ಉದ್ದೇಶವನ್ನು ಹೊಂದಿದ್ದೀರಿ ಎಂದು ಕೇಳಿ ನಮಗೆ ತುಂಬ ಸಂತೋಷವಾಗಿದೆ. ತಮ್ಮ ಉದ್ದೇಶವು ಸಫಲವಾಗಲಿ ಎಂದು ನಾವು ಹೃತ್ಪೂರ್ವಕ ಪ್ರಾರ್ಥಿಸುತ್ತೇವೆ. ಮಹಾನ್​ ಆಚಾರ್ಯರಾದ ಶ‍್ರೀ ಶಂಕರರು ಕೂಡ ತಮ್ಮ ಧಾರ್ಮಿಕ ದಿಗ್ವಿಜಯದ ಕೊನೆಯಲ್ಲಿ ಸನಾತನ ಧರ್ಮದ ರಕ್ಷಣೆಗಾಗಿ ಹಿಮಾಲಯದ ಬದರಿಕಾಶ್ರಮದಲ್ಲಿ ಮಠವನ್ನು ಸ್ಥಾಪಿಸಿದರು. ಹಾಗೆಯೇ ತಮ್ಮ ಆಸೆಯು ನೆರವೇರಿದರೆ ಭಾರತಕ್ಕೆ ಬಹುವಾಗಿ ಪ್ರಯೋಜನವಾಗುತ್ತದೆ. ಇಲ್ಲಿ ಮಠವು ಸ್ಥಾಪಿತವಾದರೆ ಕುಮಾನಿನ ಜನರಾದ ನಮಗೆ ಬಹಳವಾದ ಆಧ್ಯಾತ್ಮಿಕ ಪ್ರಯೋಜನವಾಗುತ್ತಿದೆ. ಈ ಪ್ರಾಚೀನ ಧರ್ಮವು ಕ್ರಮಕ್ರಮವಾಗಿ ಮಾಯವಾಗುವುದನ್ನು ನಾವು ನೋಡಬೇಕಾಗುವುದಿಲ್ಲ.

\newpage

ಅನಾದಿಕಾಲದಿಂದಲೂ ಭಾರತದ ಈ ಭಾಗವು ತಪೋಭೂಮಿಯಾಗಿದೆ. ಭಾರತದ ಮಹಾನ್​ ಋಷಿಗಳು ಇಲ್ಲಿ ತಪಸ್ಸನ್ನು ಆಚರಿಸಿದ್ದಾರೆ. ಆದರೆ ಅದು ಗತಕಾಲದ\break ಮಾತಾಯಿತು. ಈಗ ತಾವು ಮಠವನ್ನು ಸ್ಥಾಪಿಸಿದರೆ ಈ ನೆಲಕ್ಕೆ ಆ ಹಳೆಯ ಕೀರ್ತಿ ಮತ್ತೆ ಲಭ್ಯವಾಗುತ್ತದೆ. ನಿಜವಾದ ಧರ್ಮ, ಕರ್ಮ, ಸಂಯಮ, ನ್ಯಾಯದೃಷ್ಟಿ ಇವುಗಳ ತವರು ಎಂದು ಈ ಭಾಗವು ಭಾರತಾದ್ಯಂತ ಪ್ರಸಿದ್ಧವಾಗಿತ್ತು. ಆದರೆ ಕಾಲ ಮಹಿಮೆಯಿಂದ ಅವೆಲ್ಲ ನಶಿಸಿಹೋಗುತ್ತಿವೆ. ಈಗ ತಮ್ಮ ಪ್ರಯತ್ನಗಳಿಂದ ಈ ನಾಡು ತನ್ನ ಪ್ರಾಚೀನ ಸ್ಥಿತಿಗೆ ಹಿಂದಿರುಗುತ್ತದೆ ಎಂದು ನಾವು ನಂಬಿದ್ದೇವೆ.

\vskip 4pt

ತಾವು ಇಲ್ಲಿಗೆ ಆಗಮಿಸಿದ್ದರಿಂದ ನಮಗೆ ಉಂಟಾಗಿರುವ ಸಂತೋಷವನ್ನು ಸಮರ್ಪಕವಾಗಿ ವ್ಯಕ್ತಪಡಿಸಲು ನಮಗೆ ಸಾಧ್ಯವಾಗುತ್ತಿಲ್ಲ. ತಾವು ದೀರ್ಘಕಾಲ, ಸಂಪೂರ್ಣ ಆರೋಗ್ಯಶಾಲಿಯಾಗಿ ಪರೋಪಕಾರ ಜೀವನವನ್ನು ನಡೆಸುತ್ತ ಬಾಳುವಂತಾಗಲಿ. ತಮ್ಮ ಆಧ್ಯಾತ್ಮಿಕ ಶಕ್ತಿಯು ವೃದ್ಧಿಗೊಂಡು ತಮ್ಮ ಪ್ರಯತ್ನಗಳ ಮೂಲಕ ಭಾರತದ ದುಃಸ್ಥಿತಿಯು ಬಹುಬೇಗ ಮಾಯವಾಗುವಂತಾಗಲಿ.

\vskip 5pt

ಸ್ವಾಮೀಜಿಯವರಿಗೆ ಇನ್ನೂ ಎರಡು ಬಿನ್ನವತ್ತಳೆಗಳನ್ನು ಅರ್ಪಿಸಲಾಯಿತು. ಅವುಗಳಿಗೆ ಸ್ವಾಮೀಜಿ ಈ ಕೆಳಗಿನ ಸಂಕ್ಷೇಪವಾದ ಉತ್ತರವನ್ನು ನೀಡಿದರು:

\vskip 5pt

ನಮ್ಮ ಪೂರ್ವಜರ ಸ್ವಪ್ನದೇಶವಿದು. ಭಾರತಜನನಿಯಾದ ಪಾರ್ವತಿ ಉದಿಸಿದ ಸ್ಥಳವಿದು. ಭಾರತವರ್ಷದ ಸತ್ಯಪಿಪಾಸುಗಳೆಲ್ಲ ತಮ್ಮ ಜೀವನದ ಕೊನೆಗಾಲವನ್ನು ಕಳೆಯಬೇಕೆಂದು ಬಯಸುವ ಸ್ಥಳ. ಈ ಪವಿತ್ರ ಗಿರಿಶಿಖರಗಳ ಮೇಲೆ, ಸುತ್ತಮುತ್ತಲಿರುವ ಗಹ್ವರಗಳಲ್ಲಿ, ರಭಸದಿಂದ ಸಾಗುತ್ತಿರುವ ನದಿಯ ತೀರದಲ್ಲಿ, ಅತಿ ಗಹನ ವಿಷಯವನ್ನು ಪ್ರಾಚೀನಕಾಲದಲ್ಲಿ ಆಲೋಚಿಸಿದರು. ಅದರ ಒಂದು ಅಲ್ಪಾಂಶವು ಪಾಶ್ಚಾತ್ಯರ ತುಂಬು ಪ್ರಶಂಸೆಗೆ ಪಾತ್ರವಾಗಿದೆ, ಮತ್ತು ಧುರೀಣ ವಿದ್ಯಾವಂತರೂ ಅದನ್ನು ಹೋಲಿಕೆಗೆ ಮೀರಿದ್ದು ಎಂದು ಪರಿಗಣಿಸಿದ್ದಾರೆ. ನಾನು ಬಾಲ್ಯದಿಂದಲೂ ವಾಸಮಾಡಬೇಕೆಂದು ಕನಸು ಕಾಣುತ್ತಿದ್ದ ಸ್ಥಳವೇ ಇದು. ನಾನು ಪುನಃ ಇಲ್ಲಿಗೆ ಬಂದು ವಾಸಿಸುವುದಕ್ಕೆ ಪ್ರಯತ್ನಿಸುತ್ತಿರುವೆನು ಎಂಬುದು ನಿಮಗೆಲ್ಲ ಗೊತ್ತು. ಅದಕ್ಕೆ ಸರಿಯಾದ ಸಮಯ ಒದಗಿಬರಲಿಲ್ಲ. ಮಾಡುವುದಕ್ಕೆ ಕೆಲಸವಿತ್ತು. ಈ ಪವಿತ್ರ ಸ್ಥಳದಿಂದ ಹೊರಗೆ ಹೋಗಬೇಕಾಗಿ ಬಂತು. ಆದರೂ ಋಷಿಗಳ ನೆಲೆಬೀಡಾಗಿದ್ದ, ನಮ್ಮ ತತ್ತ್ವಶಾಸ್ತ್ರಗಳ ಜನ್ಮಸ್ಥಾನ ವಾದ ಗಿರಿರಾಜನ ಎಡೆಯಲ್ಲಿ ನನ್ನ ಜೀವನವನ್ನು ಕಳೆಯಬೇಕೆಂದು ಇಚ್ಛೆ. ಬಹುಶಃ ನಾನು ಹಿಂದೆ ಆಲೋಚಿಸಿದಂತೆ ಮಾಡುವುದಕ್ಕಾಗದೆ ಇರಬಹುದು. ಆ ಮೌನ ಅಜ್ಞಾತವಾಸ ನನಗೆ ದೊರಕುವುದಾದರೆ – ಆದರೂ ದಯಪಾಲಿಸಬೇಕೆಂದು ನಾನು ಆ, ಅದಕ್ಕಾಗಿ ನಾನು ಹೃತ್ಪೂರ್ವಕ ಪ್ರಾರ್ಥಿಸುತ್ತೇನೆ, ಆಶಿಸುತ್ತೇನೆ. ಈ ಸ್ಥಳದಲ್ಲಿಯೇ ನಾನು ನನ್ನ ಕೊನೆಗಾಲವನ್ನು ಕಳೆಯುತ್ತೇನೆ ಎಂದು ನಂಬುವೆನು.

\newpage

ಪಾಶ್ಚಾತ್ಯ ದೇಶಗಳಲ್ಲಿ ನಾನು ಮಾಡಿದ ಅಲ್ಪ ಕೆಲಸವನ್ನು ನೀವು ಶ್ಲಾಘಿಸಿರುವಿರಿ. ಅದಕ್ಕೆ ನನ್ನ ಅನಂತ ಧನ್ಯವಾದ. ಆ ಕಾರ್ಯವನ್ನು ಕುರಿತು ಪಾಶ್ಚಾತ್ಯ ದೇಶದಲ್ಲಾಗಲೀ, ಪ್ರಾಚ್ಯ ದೇಶದಲ್ಲಾಗಲೀ ಮಾತನಾಡುವುದಕ್ಕೆ ಇಚ್ಛೆ ಇಲ್ಲ. ನನ್ನ ಕಣ್ಣಿಗೆ ಈ ಗಿರಿರಾಜನ ಶಿಖರಗಳು ಒಂದಾದಮೇಲೊಂದು ಕಂಡಾಗ ಕೆಲಸಮಾಡುವ ಕುತೂಹಲವೆಲ್ಲ, ಹಲವು ವರುಷಗಳಿಂದ ಕುದಿಯುತ್ತಿದ್ದ ಆ ಇಚ್ಛೆಯೆಲ್ಲ ತಣ್ಣಗಾಯಿತು. ಈ ಹಿಂದೆ ಏನು ಮಾಡಿದೆ, ಮುಂದೆ ಏನು ಮಾಡಬೇಕೆಂದಿರುವೆ ಎಂಬುದನ್ನು ಮಾತನಾಡುವ ಬದಲು, ಹಿಮಾಲಯ ನಮಗೆ ಸನಾತನವಾಗಿ ಯಾವ ಸಂದೇಶವನ್ನು ಬೋಧಿಸುತ್ತದೆಯೋ, ಯಾವ ಸಂದೇಶ ಈ ವಾತಾವರಣದಲ್ಲಿ ಅನುರಣಿತವಾಗುತ್ತಿದೆಯೋ, ಈ ಗಿರಿ ಝರಿಗಳು ಹಾಡುವ ಯಾವ ಪಲ್ಲವಿ ನನ್ನ ಕಿವಿಗೆ ಬೀಳುತ್ತಿದೆಯೋ ಅದರ ಕಡೆಗೆ ಮನಸ್ಸು ಹೋಗುತ್ತಿದೆ. ಆ ಸಂದೇಶವೇ ತ್ಯಾಗ. \textbf{“ಸರ್ವಂ ವಸ್ತು ಭಯಾನ್ವಿತಂ ಭುವಿ ನೃಣಾಂ ವೈರಾಗ್ಯಮೇವಾಭಯಂ,”} ಈ ಪ್ರಪಂಚದಲ್ಲಿ ಎಲ್ಲಾ ಭಯದಿಂದ ಕೂಡಿದೆ, ವೈರಾಗ್ಯವೊಂದೇ ಒಬ್ಬನನ್ನು ನಿರ್ಭೀತನನ್ನಾಗಿ ಮಾಡುವುದು ಎಂಬ ಭಾವನೆಯನ್ನು ಮನಸ್ಸು ಚಿಂತಿಸತೊಡಗಿದೆ. ಇದು ತ್ಯಾಗಭೂಮಿ.

\vskip 5pt

ಇಂದು ಮಾತನಾಡುವುದಕ್ಕೆ ಸಮಯವಿಲ್ಲ. ಅದಕ್ಕೆ ಸರಿಯಾದ ಸನ್ನಿವೇಶವೂ ಇಲ್ಲ. ಹಿಮಾಲಯವು ತ್ಯಾಗದ ಚಿಹ್ನೆಯಾಗಿದೆ. ಈ ತ್ಯಾಗವೊಂದೇ ಪ್ರಪಂಚಕ್ಕೆ ನಾವು ಬೋಧಿಸಬೇಕಾದ ಮಹಾ ಸಂದೇಶ ಎಂದು ಹೇಳಿ ಉಪನ್ಯಾಸವನ್ನು ಪೂರೈಸಬೇಕಾಗಿದೆ. ನಮ್ಮ ಪೂರ್ವಿಕರು ಹೇಗೆ ತಮ್ಮ ಜೀವನದ ಕೊನೆಗಾಲದಲ್ಲಿ ಈ ಗಿರಿ ಪಿತಾಮಹನೆಡೆಗೆ ಆಕರ್ಷಿತರಾಗುತ್ತಿದ್ದರೋ, ಹಾಗೆಯೇ ಮುಂದೆ ಭೂಮಿಯ ಎಲ್ಲ ಮೂಲೆಗಳಿಂದಲೂ ಬಲಾಢ್ಯ ವ್ಯಕ್ತಿಗಳೆಲ್ಲರೂ ಈ ಪರ್ವತದೆಡೆಗೆ ಆಕರ್ಷಿತರಾಗಿ ಬರುತ್ತಾರೆ. ಧರ್ಮದ ಹೆಸರಿನಲ್ಲಿ ಆಗುವ ಕಾದಾಟ ಮನಸ್ತಾಪ ಭಿನ್ನಾಭಿಪ್ರಾಯಗಳು ಮಾಯವಾಗುವುವು. ಸನಾತನ ಧರ್ಮ ಒಂದಿದೆ; ಅದೆಂದರೆ ಪ್ರತಿಯೊಬ್ಬರಲ್ಲಿಯೂ ಪವಿತ್ರತೆಯನ್ನು ಕಾಣುವುದು. ಇದೊಂದೇ ಧರ್ಮ, ಉಳಿದುದೆಲ್ಲ ಬರಿಯ ಬಾಯಿಮಾತು ಎಂದು ಆಗ ಗೊತ್ತಾಗುವುದು. ಆಗ, ಧರ್ಮಪಿಪಾಸುಗಳು ಈಶ್ವರ ಪೂಜೆಯಲ್ಲದೆ ಉಳಿದುದೆಲ್ಲ ನಶ್ವರ, ಪ್ರಯೋಜನವಿಲ್ಲವೆಂದು ತಿಳಿದು ಇಲ್ಲಿಗೆ ಬರುವರು.

\vskip 5pt

ಸ್ನೇಹಿತರೇ, ಹಿಮಾಲಯದಲ್ಲಿ ನಾನೊಂದು ಆಶ್ರಮವನ್ನು ಸ್ಥಾಪಿಸ ಬೇಕೆಂದಿರುವ ವಿಷಯವನ್ನು ನೀವು ಹೃತ್ಪೂರ್ವಕ ಪ್ರಸ್ತಾಪಿಸಿರುವಿರಿ. ವಿಶ್ವ ಧರ್ಮವನ್ನು ಬೋಧಿಸುವುದಕ್ಕೆ ಎಲ್ಲಕ್ಕಿಂತಲೂ ಇದು ಪವಿತ್ರೋತ್ತಮ ಸ್ಥಳ ಎಂಬುದನ್ನು ನಾನಾಗಲೇ ನಿಮಗೆ ವಿವರಿಸಿರುವೆನು. ಈ ಹಿಮಾಲಯ ಶ್ರೇಣಿಯೊಂದಿಗೆ ನಮ್ಮ ಪೂರ್ವಿಕರ ಸ್ಮೃತಿಯ ಸಂಬಂಧವಿದೆ. ಭಾರತದ ಧಾರ್ಮಿಕ ಇತಿಹಾಸದಿಂದ ಹಿಮಾಲಯವನ್ನು ತೆಗೆದುಬಿಟ್ಟರೆ ಉಳಿಯುವುದು ಅತ್ಯಲ್ಪ. ಆದ್ದರಿಂದ ಇಲ್ಲಿ ಸ್ಥಾಪಿಸಲ್ಪಡುವ ಕೇಂದ್ರವು ಕೇವಲ ಕರ್ಮ ಕ್ಷೇತ್ರವಾಗಿರದೆ, ಧ್ಯಾನಕ್ಕೆ ಪ್ರಶಸ್ತವಾದ, ಶಾಂತಿ ಸಮಾಧಾನಗಳನ್ನು ನೀಡುವ ಸಾಧನ ಕ್ಷೇತ್ರವಾಗಬೇಕು. ನನ್ನ ಈ ಬಯಕೆ ಈಡೇರುವುದೆಂದು ಆಶಿಸುತ್ತೇನೆ. ಮತ್ತೊಮ್ಮೆ ನಿಮ್ಮನ್ನು ನೋಡಿ ಮಾತನಾಡುವ ಅವಕಾಶ ಬರುವುದೆಂದೂ ಯೋಚಿಸುವೆನು. ಸದ್ಯಕ್ಕೆ ನೀವು ನನಗೆ ತೋರಿದ ಆದರಕ್ಕೆಲ್ಲ ಅನಂತ ಧನ್ಯವಾದಗಳು. ಇದನ್ನು ನೀವು ಕೇವಲ ನನಗೆ ತೋರಿದ ಗೌರವ ಎಂದು ಭಾವಿಸುವುದಿಲ್ಲ. ಇದು ನಮ್ಮ ಧರ್ಮಕ್ಕೆ ತೋರಿದ ಗೌರವ. ಇದೆಂದಿಗೂ ನಮ್ಮ ಹೃದಯದಿಂದ ಮಾಯದೆ ಇರಲಿ. ನಾವು ಈಗ ಪರಿಶುದ್ಧರಾಗಿರುವಂತೆಯೇ ಆಧ್ಯಾತ್ಮಿಕ ಉತ್ಸಾಹಿಗಳಾಗಿರುವಂತೆಯೇ ಎಂದೆಂದಿಗೂ ಇರೋಣ.

