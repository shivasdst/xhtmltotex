
\chapter{ಕೊಲಂಬೊ}

\begin{center}
\textbf{(ಪೂರ್ವದೇಶದಲ್ಲಿ ನೀಡಿದ ಮೊದಲ ಸಾರ್ವಜನಿಕ ಉಪನ್ಯಾಸ)}
\end{center}

ಪಶ್ಚಿಮದಲ್ಲಿ ಸ್ಮರಣಾರ್ಹವಾದ ಸಾಧನೆಯ ಅನಂತರ ಸ್ವಾಮಿ ವಿವೇಕಾನಂದರು ೧೮೯೭ರ ಜನವರಿ ೧೫ರ ಮಧ್ಯಾಹ್ನ ಕೊಲಂಬೋದಲ್ಲಿ ಹಡಗಿನಿಂದ ಇಳಿದರು. ಅಲ್ಲಿನ ಹಿಂದೂ ಜನಾಂಗವು ಸ್ವಾಮೀಜಿಯವರಿಗೆ ರಾಜೋಚಿತ ಸ್ವಾಗತವನ್ನು ನೀಡಿತು. ಈ ಕೆಳಗಿನ ಸ್ವಾಗತ ಬಿನ್ನವತ್ತಳೆಯನ್ನು ಅವರಿಗೆ ಅರ್ಪಿಸಲಾಯಿತು.

\begin{center}
\textbf{ಶ‍್ರೀಮತ್​ ವಿವೇಕಾನಂದ ಸ್ವಾಮಿ}
\end{center}

\textbf{ಪೂಜ್ಯರೆ,}

ಕೊಲಂಬೋನಗರದ ಹಿಂದೂ ಜನಾಂಗವು ಏರ್ಪಡಿಸಿದ್ದ ಸಭೆಯಲ್ಲಿ ತೀರ್ಮಾನಿಸಿದಂತೆ ತಮ್ಮನ್ನು ಈ ದ್ವೀಪಕ್ಕೆ ಹೃತ್ಪೂರ್ವಕವಾಗಿ ಸ್ವಾಗತಿಸಲು ತಮ್ಮ ಅಪ್ಪಣೆಯನ್ನು ಬೇಡುತ್ತೇವೆ. ಪಶ್ಚಿಮದಲ್ಲಿ ಮಹತ್ಕಾರ್ಯಗಳನ್ನು ಸಾಧಿಸಿ ಹಿಂತಿರುಗುತ್ತಿರುವ ತಮ್ಮನ್ನು ಈ ನಾಡಿಗೆ ಪ್ರಪ್ರಥಮವಾಗಿ ಸ್ವಾಗತಿಸುವ ಸೌಭಾಗ್ಯ ನಮ್ಮದಾಗಿದೆ.

ತಾವು ಸಾಧಿಸಲು ಹೊರಟ ಮಹತ್ಕಾರ್ಯವು ಭಗವತ್ಕೃಪೆಯಿಂದ ಪಡೆದ ಮಹತ್ತಾದ ಯಶಸ್ಸನ್ನು ನಾವು ಕೃತಜ್ಞತಾಪೂರ್ವಕವಾಗಿಯೂ, ಆನಂದದಿಂದಲೂ ಗಮನಿಸುತ್ತಾ ಬಂದಿದ್ದೇವೆ. ವಿಶ್ವಧರ್ಮವನ್ನು ಕುರಿತಂತೆ ಹಿಂದೂಗಳ ಆದರ್ಶವನ್ನು ಯೂರೋಪು ಮತ್ತು ಅಮೇರಿಕಾ ದೇಶಗಳಲ್ಲಿ ತಾವು ಸಾರಿದ್ದೀರಿ. ಈ ವಿಶ್ವಧರ್ಮವು ವಿವಿಧ ಪಂಥಗಳನ್ನು ಸಮನ್ವಯಗೊಳಿಸುವಂಥದು, ಪ್ರತಿಯೊಂದು ಆತ್ಮಕ್ಕೂ ಅದರ ಆವಶ್ಯಕತೆಗೆ ಅನುಗುಣವಾಗಿ ಆಧ್ಯಾತ್ಮಿಕ ಪುಷ್ಟಿಯನ್ನು ನೀಡಿ ಅದನ್ನು ಪ್ರೇಮ ಪೂರ್ವಕವಾಗಿ ಭಗವಂತನ ಕಡೆಗೆ ಒಯ್ಯುವಂಥದು ಎಂಬುದನ್ನು ತಾವು ಆ ದೇಶಗಳಲ್ಲಿ ಸಾರಿದ್ದೀರಿ. ಪ್ರಪ್ರಾಚೀನ ಕಾಲದಿಂದಲೂ ಭಾರತದ ನೆಲವನ್ನು ಪವಿತ್ರಗೊಳಿಸಿದ ಮಹಾನ್​ ಸಂತರ ಪರಂಪರೆಯು ಬೋಧಿಸಿದ ಸತ್ಯವನ್ನೂ ಅದರ ಪಥವನ್ನೂ ತಾವು ಅಲ್ಲಿ ಉಪದೇಶಿಸಿದ್ದೀರಿ. ಆ ಮಹಾನ್​ ಸಂತರ ಅರ್ಥಪೂರ್ಣವಾದ ಸನ್ನಿಧಿ, ಸ್ಫೂರ್ತಿಗಳು ಭಾರತವು ತನ್ನೆಲ್ಲ ಉನ್ನತಿ ಅವನತಿಗಳ ನಡುವೆಯೂ ಜಗತ್ತಿನ ಜ್ಯೋತಿಯಾಗುವಂತೆ ಮಾಡಿದ್ದವು.

ಶ‍್ರೀರಾಮಕೃಷ್ಣ ಪರಮಹಂಸ ದೇವರಂತಹ ಗುರುವರ್ಯರ ಸ್ಫೂರ್ತಿಗೂ, ನಿಮ್ಮ ಸ್ವಾರ್ಥತ್ಯಾಗಪೂರ್ವಕವಾದ ಉತ್ಸಾಹಕ್ಕೂ ಪಶ್ಚಿಮದೇಶಗಳು ಚಿರಋಣಿಗಳಾಗಿರಬೇಕು. ಭಾರತದ ಆಧ್ಯಾತ್ಮಿಕ ಪ್ರತಿಭೆಯ ಜೀವಂತ ಸಂಪರ್ಕವು ಪಶ್ಚಿಮ ದೇಶಗಳಿಗೆ ದೊರಕುವಂತೆ ಮಾಡಲು ತಾವು ಕಾರಣರಾಗಿದ್ದೀರಿ. ಪಾಶ್ಚಾತ್ಯ ನಾಗರಿಕತೆಯ ಪ್ರಲೋಭನೆಗೆ ಒಳಗಾಗಿರುವ ಪೌರಸ್ತ್ಯರಿಗೆ ತಮ್ಮ ಭವ್ಯ ಪರಂಪರೆಯ ಅರಿವನ್ನು ತಾವು ಮೂಡಿಸಿದ್ದೀರಿ.

ತಮ್ಮ ಉದಾತ್ತವಾದ ಸಾಧನೆಯಿಂದಲೂ, ವ್ಯಕ್ತಿತ್ವದ ಆದರ್ಶದಿಂದಲೂ ಇಡೀ ಮಾನವತೆಯು, ಭರಿಸಲಾರದಷ್ಟು ಋಣದಲ್ಲಿ ಇರುವಂತೆ ತಾವು ಮಾಡಿದ್ದೀರಿ. ನಮ್ಮ ಮಾತೃಭೂಮಿಯ ಮೇಲೆ ಹೊಸ ಕಾಂತಿಯನ್ನು ತಾವು ಚೆಲ್ಲಿದ್ದೀರಿ. ತಮಗೂ ತಾವು ಮಾಡಲಿರುವ ಕಾರ್ಯಗಳಿಗೂ ಹೆಚ್ಚಿನ ಯಶಸ್ಸನ್ನು ನೀಡಲೆಂದು ಭಗವಂತನನ್ನು ನಾವು ಪ್ರಾರ್ಥಿಸುತ್ತೇವೆ.

\vskip   4pt

\begin{longtable}[r]{@{}l@{}}
ಕೊಲಂಬೋದ ಹಿಂದೂಗಳ ಪರವಾಗಿ \\
ತಮ್ಮ ವಿಧೇಯ \\
ಪಿ. ಕುಮಾರ ಸ್ವಾಮಿ, \\
ಸಿಲೋನಿನ ವಿಧಾನಪರಿಷತ್ತಿನ ಸದಸ್ಯ, \\
ಈ ಸಭೆಯ ಅಧ್ಯಕ್ಷ \\
ಎ. ಕುಲವೀರ ಸಿಂಘಂ, ಕಾರ್ಯದರ್ಶಿ \\
ಕೊಲಂಬೊ, ಜನವರಿ ೧೮೯೭ \\
\end{longtable}

\vskip   4pt

ಸ್ವಾಮೀಜಿ ನೀಡಿದ ಹ್ರಸ್ವವಾದ ಉತ್ತರದಲ್ಲಿ ತಮಗೆ ನೀಡಿದ ಉದಾರ ಸ್ವಾಗತಕ್ಕಾಗಿ ವಂದನೆಗಳನ್ನು ಅರ್ಪಿಸಿದರು. ಆ ಅವಕಾಶವನ್ನು ಬಳಸಿಕೊಂಡು ಸ್ವಾಮೀಜಿಯವರು ಉತ್ಸಾಹ ತುಂಬಿದ ಈ ಸ್ವಾಗತವನ್ನು ನೀಡಿದುದು ಶ್ರೇಷ್ಠ ರಾಜಕಾರಣಿಗಲ್ಲ, ಮಹಾಯೋಧನಿಗೂ ಅಲ್ಲ, ಕೋಟ್ಯಧೀಶ್ವರನಿಗೂ ಅಲ್ಲ, ಆದರೆ ಕೇವಲ ಭಿಕ್ಷುಕ ಸಂನ್ಯಾಸಿಯೊಬ್ಬನಿಗೆ; ಇದು ಧರ್ಮದ ವಿಷಯದಲ್ಲಿ ಹಿಂದೂ ಮನಸ್ಸಿನ ಪ್ರವೃತ್ತಿಯನ್ನು ತೋರಿಸುತ್ತದೆ ಎಂಬುದನ್ನು ವಿವರಿಸಿದರು. ರಾಷ್ಟ್ರವು ಉಳಿಯಬೇಕಾದರೆ ಧರ್ಮವನ್ನು ತನ್ನ ಬೆನ್ನೆಲುಬನ್ನಾಗಿ ಮಾಡಿಕೊಳ್ಳುವುದರ ಆವಶ್ಯಕತೆ ಎಷ್ಟು ಎಂಬುದನ್ನು ಒತ್ತಿ ಹೇಳಿದರು. ಈ ಗೌರವ ಸಂದದ್ದು ಒಬ್ಬ ವ್ಯಕ್ತಿಗಲ್ಲ, ಒಂದು ತತ್ತ್ವಕ್ಕೆ ಎಂಬುದನ್ನು ಸ್ಪಷ್ಟಪಡಿಸಿದರು.

\vskip   4pt

೧೮೯೭ನೆ ಇಸವಿ ಜನವರಿ ೧೬ರಂದು ಸಂಜೆ ಸ್ವಾಮೀಜಿ ಕೊಲಂಬೊದ ಫ್ಲೋರಲ್​ ಹಾಲಿನಲ್ಲಿ ಈ ಕೆಳಕಂಡ ಸಾರ್ವಜನಿಕ ಉಪನ್ಯಾಸವನ್ನು ನೀಡಿದರು:

\newpage

ಈಗ ಸಾಧಿತವಾಗಿರುವ ಅಲ್ಪ ಕಾರ್ಯವು ಕೇವಲ ನನ್ನಲ್ಲಿರುವ ಶಕ್ತಿಯಿಂದ ಆದದ್ದಲ್ಲ. ನನ್ನ ಪರಮಮಿತ್ರ, ಪ್ರಿಯತಮ ಮಾತೃಭೂಮಿಯಿಂದ ಹೊರಟ ಉತ್ತೇಜನ, ಶುಭಾಶಯ, ಆಶೀರ್ವಾದಗಳು ಅದಕ್ಕೆ ಕಾರಣ. ಪಾಶ್ಚಾತ್ಯ ದೇಶಗಳಲ್ಲಿ ನಿಸ್ಸಂದೇಹವಾಗಿಯೂ ಸ್ವಲ್ಪ ಒಳ್ಳೆಯ ಕಾರ್ಯವೇನೋ ಆಗಿದೆ. ಆದರೆ ಪ್ರಯೋಜನವಾಗಿರುವುದು, ವಿಶೇಷತಃ ನನಗೆ. ಯಾವುದು ಹಿಂದೆ ಕೇವಲ ಭಾವಪೂರ್ಣವಾಗಿತ್ತೊ ಅದು ಈಗ ಪ್ರಮಾಣಸಿದ್ಧ ಸತ್ಯದಂತೆ ತೋರುತ್ತಿದೆ. ಅದನ್ನು ಈಗ ಕಾರ್ಯರೂಪಕ್ಕೆ ತರುವುದು ಸಾಧ್ಯ ಎಂಬುದು ಪ್ರಮಾಣಗೊಂಡಿದೆ. ಹಿಂದೆ ನಾನು, ಎಲ್ಲಾ ಹಿಂದೂಗಳು ಭಾವಿಸುವಂತೆ, ನಿಮ್ಮ ಘನ ಅಧ್ಯಕ್ಷರು ಈಗ ತಾನೆ ಹೇಳಿದಂತೆ, ಭರತಖಂಡವು ಪುಣ್ಯಭೂಮಿ, ಕರ್ಮಭೂಮಿ ಎಂದು ಭಾವಿಸಿದ್ದೆ. ಇಂದು ನಾನು ನಿಮ್ಮ ಎದುರಿಗೆ ನಿಂತು ಅದು ಸತ್ಯ ಎಂದು ಘಂಟಾಘೋಷವಾಗಿ ಸಾರುತ್ತೇನೆ. ಜಗತ್ತಿನಲ್ಲಿ ಯಾವುದಾದರೂ ಒಂದು ದೇಶವು ಪುಣ್ಯಭೂಮಿಯೆಂದು ಕರೆಯಿಸಿಕೊಳ್ಳಲು ಅರ್ಹವಾಗಿದ್ದರೆ, ಜೀವಿಗಳು ತಮ್ಮ ಬಾಳಿನ ಕೊನೆಯ ಕರ್ಮವನ್ನು ಸವೆಸಲು\break ಬರಬೇಕಾದ ಸ್ಥಳವೊಂದಿದ್ದರೆ, ಭಗವಂತನೆಡೆಗೆ ಸಂಚರಿಸುತ್ತಿರುವ ಪ್ರತಿಯೊಂದು\break ಜೀವಿಯೂ ತನ್ನ ಕೊನೆಯ ಯಾತ್ರೆಯನ್ನು ಪೂರೈಸುವುದಕ್ಕೆ ಒಂದು ಕರ್ಮಭೂಮಿಗೆ ಬರಬೇಕಾಗಿದ್ದರೆ, ಯಾವುದಾದರೂ ದೇಶದಲ್ಲಿ ಮಾನವ ಕೋಟಿಯು ಮಾಧುರ್ಯ, ಔದಾರ್ಯ, ಪಾವಿತ್ರ್ಯ, ಶಾಂತಿ–ಇವುಗಳಲ್ಲಿ ಮತ್ತು ಎಲ್ಲಕ್ಕಿಂತ ಹೆಚ್ಚಾಗಿ ಧ್ಯಾನದಲ್ಲಿ ಮತ್ತು ಅಂತರ್ಮುಖ–ಜೀವನದಲ್ಲಿ ತನ್ನ ಪರಾಕಾಷ್ಠೆಯನ್ನು ಮುಟ್ಟಿದ್ದರೆ, ಅದು ಈ ಭರತಖಂಡವೇ ಆಗಿದೆ. ಅನಾದಿ ಕಾಲದಿಂದಲೂ ಇಲ್ಲಿಂದ ಧರ್ಮಸಂಸ್ಥಾಪನಾಚಾರ್ಯರು ಮತ್ತೆ ಮತ್ತೆ, ಪೃಥ್ವಿಯನ್ನೆಲ್ಲಾ, ತಮ್ಮ ಪವಿತ್ರವಾದ, ಎಂದೆಂದಿಗೂ ಬತ್ತದ, ಆಧ್ಯಾತ್ಮಿಕ ಮಹಾಸತ್ಯದ ಪ್ರವಾಹದಿಂದ ತೋಯಿಸಿರುವರು. ಇಲ್ಲಿಂದ ಎದ್ದ ದರ್ಶನಗಳ ಮಹಾಪ್ರವಾಹದ ಅಲೆಗಳು ಪೂರ್ವ ಪಶ್ಚಿಮ ಉತ್ತರ ದಕ್ಷಿಣಗಳೆನ್ನದೆ ಭೂಮಿಯನ್ನು ಆವರಿಸಿವೆ. ಇಂದು ಜಡನಾಗರಿಕತೆಯ ಪ್ರಪಂಚಕ್ಕೆ ಅಧ್ಯಾತ್ಮವನ್ನು ಧಾರೆ ಎರೆಯುವ ಮಹಾಪ್ರವಾಹವೂ ಇಲ್ಲಿಂದ ಉದಯಿಸಬೇಕಾಗಿದೆ. ಇಲ್ಲಿದೆ ಬಾಳಿಗೆ ಹೊಸ ಬೆಳಕನ್ನು ಕೊಡುವ ಅದ್ಭುತಪ್ರವಾಹ; ಅನ್ಯದೇಶಗಳಲ್ಲಿ ಕೋಟ್ಯಂತರ ಜೀವಿಗಳ ಎದೆಯನ್ನು ದಹಿಸುತ್ತಿರುವ ಜಡವಾದದ ದಳ್ಳುರಿಯ ಶಮನಕ್ಕೆ ಅಮೃತ ಪ್ರವಾಹವಿಲ್ಲಿದೆ. ಸ್ನೇಹಿತರೇ, ನೆಚ್ಚಿ, ಈ ಕಾರ್ಯ ಕೈಗೂಡುವುದು.

ನಾನೆಷ್ಟೋ ನೋಡಿರುವೆನು. ಜನಾಂಗಗಳ ಜೀವನಚರಿತ್ರೆಯನ್ನು ಅಧ್ಯಯನ ಮಾಡಿರುವ ನಿಮ್ಮಲ್ಲಿ ಕೆಲವರಿಗೆ ಆಗಲೇ ಅದು ವೇದ್ಯವಾಗಿದೆ. ನಮ್ಮ ತಾಯ್ನಾಡಿಗೆ ಅನ್ಯರಾಷ್ಟ್ರಗಳು ಸಲ್ಲಿಸಬೇಕಾಗಿರುವ ಋಣ ಮಹತ್ತರವಾದುದು. ಒಂದಾದಮೇಲೊಂದು ದೇಶವನ್ನು ನೋಡಿದರೆ “ಸಹಿಷ್ಣು ಹಿಂದೂ” “ದೀನ ಹಿಂದೂ” ಗಳಿಗೆ ಋಣವನ್ನು ಸಲ್ಲಿಸದೇ ಇರಬೇಕಾದ ಜನಾಂಗವು ಯಾವುದೊಂದೂ ಇಲ್ಲ ಎಂಬುದು ತಿಳಿಯುತ್ತದೆ. “ದೀನ ಹಿಂದೂ” ಎಂಬ ಪದವನ್ನು ನಿಂದೆಯ ಅರ್ಥದಲ್ಲಿ ಬಳಸಿರುವರು. ನಿಂದೆಯ ಮಾತುಗಳಲ್ಲಿ ಏನಾದರೂ ಅದ್ಭುತ ಸತ್ಯ ಹುದುಗಿದ್ದರೆ ಅದೇ “ದೀನ ಹಿಂದೂ” ಎಂಬ ಪದದಲ್ಲಿದೆ. ಅವನು ಯಾವಾಗಲೂ ಧನ್ಯನು, ಅವನು ಭಗವಂತನ ಶಿಶುವಾಗಿರುವನು. ಜಗತ್ತಿನ ಎಷ್ಟೋ ಬಲಾಢ್ಯ ಮತ್ತು ಪ್ರಖ್ಯಾತ ರಾಷ್ಟ್ರಗಳಿಂದ ಗಹನ ಭಾವನೆಗಳು ಉದಯಿಸಿವೆ. ಗತಕಾಲದಲ್ಲಿ ಮತ್ತು ವರ್ತಮಾನ ಕಾಲದಲ್ಲಿ ಒಂದು ಜನಾಂಗದಿಂದ ಮತ್ತೊಂದು ಜನಾಂಗಕ್ಕೆ ಅದ್ಭುತ ಭಾವನೆಗಳು ಹರಡಿವೆ. ಪ್ರಾಚೀನ ಮತ್ತು ಆಧುನಿಕ ಕಾಲಗಳಲ್ಲಿ ಜನಜಾಗೃತಿಯ ಮಹಾ ಪ್ರವಾಹವು ಉದಾತ್ತ ಸತ್ಯಗಳ ಮತ್ತು ಶಕ್ತಿಯ ಬೀಜವನ್ನು ಪ್ರಪಂಚಕ್ಕೆ ಬಿತ್ತಿದೆ. ಆದರೆ ನನ್ನ ಸ್ನೇಹಿತರೇ, ಇದನ್ನು ಗಮನಿಸಿ: ಅದನ್ನು ಸಾರಿದ್ದು ಸಮರ ಕಹಳೆಯ ಧ್ವನಿಯಿಂದ, ದಿಗ್​ಭಿತ್ತಿ ಬಿರಿಯುವ ಪ್ರಚಂಡ ಸೇನಾ ಸಮೂಹದ ಚಲನದ ಪ್ರಭಾವದಿಂದ; ಪ್ರತಿಯೊಂದು ಭಾವನೆಯನ್ನೂ ರಕ್ತಪ್ರವಾಹದಲ್ಲಿ ತೋಯಿಸಬೇಕಾಯಿತು. ಪ್ರತಿಯೊಂದು ಭಾವನೆಯೂ ಕೋಟ್ಯಂತರ ಜನರ ರಕ್ತದ ಮೂಲಕ ಸಾಗಿಹೋಗಬೇಕಾಯಿತು ಪ್ರತಿಯೊಂದು ಶಕ್ತಿದಾಯಕ ಪದದ ಮುಂದೆಯೂ ಲಕ್ಷಾಂತರ ಜನರ ನಿಟ್ಟುಸಿರು, ಅನಾಥರ, ವಿಧವೆಯರ ಕಣ್ಣೀರಿನ ಕೋಡಿ ಹರಿಯಬೇಕಾಯಿತು. ಅನ್ಯರಾಷ್ಟ್ರಗಳು ಇತರರಿಗೆ ಬೋಧಿಸಿರುವುದು ಹೀಗೆ. ಆದರೆ ಭರತಖಂಡವು ಸಹಸ್ರಾರು ವರ್ಷಗಳಿಂದ ಶಾಂತವಾಗಿ ಬಾಳಿದೆ. ಗ್ರೀಸ್​ ಇನ್ನೂ ಹುಟ್ಟದೆ ಇರುವಾಗ, ರೋಮ್​ ದೇಶದ ಕಲ್ಪನೆಯೇ ಇರದಿದ್ದಾಗ, ಆಧುನಿಕ ಪಾಶ್ಚಾತ್ಯ ಜನಾಂಗದ ಪೂರ್ವಿಕರು ಕಾನನಾಂತರಗಳಲ್ಲಿ ನೆಲಸಿ, ಕಾಡುಮನುಷ್ಯರಂತೆ ಮೈಗೆ ಬಣ್ಣ ಬಳಿದುಕೊಳ್ಳುತ್ತಿದ್ದ ಕಾಲದಲ್ಲೇ ನಮ್ಮ ನಾಡು ಪ್ರಬುದ್ಧವಾಗಿತ್ತು. ಇತಿಹಾಸದ ದಾಖಲೆಗೆ ಕೂಡ ನಿಲುಕದ ಪುರಾತನಕಾಲದಿಂದಲೂ, ನಮ್ಮ ಆಚಾರ ವ್ಯವಹಾರಗಳ ಬಗೆಗೂ ನಿಲುಕದ ಗತಕಾಲದಿಂದಲೂ, ಅನಾದಿ ಕಾಲದಿಂದ ಇಂದಿನವರೆಗೂ ಮಹೋನ್ನತ ಭಾವನಾಪರಂಪರೆ ಇಲ್ಲಿಂದ ಹೊರಹೊಮ್ಮಿದೆ. ಈ ದೇಶದ ಜನರು ಆಡಿದ ಪ್ರತಿಯೊಂದು ಮಾತಿನ ಹಿಂದೆಯೂ ಆಶೀರ್ವಾದದ ಬಲವಿದೆ. ಮುಂದೆ ಶಾಂತಿ ಇದೆ. ಪ್ರಪಂಚದಲ್ಲಿ ನಾವು ಇತರ ರಾಷ್ಟ್ರಗಳಂತೆ ಮತ್ತೊಬ್ಬರನ್ನು ಗೆಲ್ಲಲಿಲ್ಲ. ಈ ಆಶೀರ್ವಾದ ನಮ್ಮ ಮೇಲಿರುವುದು, ಅದಕ್ಕೆಯೇ ನಾವಿನ್ನೂ ಜೀವಿಸಿರುವೆವು.

ಒಂದು ಕಾಲವಿತ್ತು. ಆಗ ಯವನ ಸೇನಾಸಮೂಹದ ಪದಾಘಾತಕ್ಕೆ ತಿರೆ ತಲ್ಲಣಿಸುತ್ತಿತ್ತು. ಈಗ ಅದು ತಿರೆಯ ರಂಗಭೂಮಿಯಿಂದ ಮಂಗಮಾಯವಾಗಿದೆ. ಅದರ ಕಥೆಯನ್ನು ಕೂಡ ಈಗ ಹೇಳುವವರಿಲ್ಲ. ಆ ಪುರಾತನ ಯವನದೇಶವು ಗತಕಾಲದಲ್ಲಿ ಗತಿಸಿ ಹೋಯಿತು. ಜಗತ್ತಿನ ಪ್ರಖ್ಯಾತ ಸ್ಥಳಗಳಲ್ಲೆಲ್ಲಾ ರೋಮನ್ನರ ಪತಾಕೆ ಹಾರಾಡುತ್ತಿದ್ದ ಕಾಲ ಒಂದಿತ್ತು. ರೋಮನ್​ ಚಕ್ರಾಧಿಪತ್ಯದ ಶಕ್ತಿಯನ್ನು ಎಲ್ಲರೂ ಮನಗಂಡರು, ಎಲ್ಲರೂ ಅದರ ಪ್ರಭಾವಕ್ಕೆ ಮಣಿಯಬೇಕಾಯಿತು. ‘ರೋಮನ್​’ ಪದವನ್ನು ಕೇಳಿದೊಡನೆಯೇ ಜಗತ್ತು ತಲ್ಲಣಿಸುತ್ತಿತ್ತು. ಆದರೆ ಈಗ ಅವರ ರಾಜಧಾನಿಯು ಭಗ್ನಾವಶೇಷಗಳಿಂದ ತುಂಬಿದೆ. ಸೀಸರ್​ ಚಕ್ರವರ್ತಿಗಳು ಆಳಿದೆಡೆಯಲ್ಲಿ ಜೇಡ ತನ್ನ ಬಲೆಯನ್ನು ಕಟ್ಟುತ್ತಿರುವುದು. ಅನ್ಯ ಜನಾಂಗಗಳು ಇದರಂತೆಯೇ ವೈಭವದ ಶಿಖರದಲ್ಲಿ ಕೆಲವು ಗಂಟೆಗಳ ಕಾಲ ಕಾಂತಿಯಿಂದ ಕೋರೈಸಿ, ಹೀನ ಬಾಳಿನಲ್ಲಿ ನಿರತವಾಗಿ, ನೀರಿನ ಮೇಲಿನ ಕಿರುದೆರೆಯಂತೆ ನಿರ್ನಾಮವಾಗಿ ಹೋಗಿವೆ. ಈ ರಾಷ್ಟ್ರಗಳು ಮಾನವ ಕೋಟಿಯ ಮೇಲೆ ತಮ್ಮ ಪ್ರಭಾವವನ್ನು ಬೀರಿದ್ದು ಹೀಗೆ. ಆದರೆ ನಾವಿನ್ನೂ ಜೀವಂತವಾಗಿರುವೆವು. ಮನು ಏನಾದರೂ ಈಗ ಬಂದರೆ ಅವನಿಗೆ ಆಶ್ಚರ್ಯವಾಗುವುದಿಲ್ಲ, ಪರದೇಶದಲ್ಲಿ ಇರುವೆನೆಂಬ ಭಾವನೆ ಬರುವುದಿಲ್ಲ. ಹಲವು ಸಹಸ್ರ ವರ್ಷಗಳಿಂದ ಆಲೋಚಿಸಿ ಆಚರಣೆಗೆ ತಂದ ಅಂದಿನ ನಿಯಮಗಳೇ ಇಂದೂ ಇವೆ. ಹಲವು ಶತಮಾನಗಳ ಅನುಭವದಿಂದ ಜನಿಸಿದ, ಶಾಶ್ವತವಾಗಿರುವಂತೆ ತೋರುವ ಆಚಾರ\break ವ್ಯವಹಾರಗಳೇ ಇಂದೂ ಇವೆ. ದಿನಗಳು ಕಳೆದಂತೆಲ್ಲಾ ಹಿಂದೂಗಳ ಮೇಲೆ ದುರದೃಷ್ಟದ ಪೆಟ್ಟಿನ ಸುರಿಮಳೆ ಬಿದ್ದಿದೆ; ಆದರೆ ಇದರಿಂದ ಒಂದು ಮಾತ್ರ ಸಿದ್ಧಿಸಿದಂತೆ ಆಗಿದೆ–ಅದೇ ಇವರನ್ನು ಇನ್ನೂ ದೃಢಮನಸ್ಕರನ್ನಾಗಿ ಮಾಡಿರುವುದು, ಬಲಾಢ್ಯರನ್ನಾಗಿ ಮಾಡಿರುವುದು. ಜಗತ್ತನ್ನು ನಾನು ನೋಡಿ, ಪಡೆದ ಅನುಭವದಿಂದ ಹೇಳುತ್ತಿದ್ದೇನೆ. ಆದುದರಿಂದ, ನನ್ನ ಈ ಮಾತುಗಳನ್ನು ನಂಬಿ. ಇಷ್ಟೆಲ್ಲಾ ಆಗಿಯೂ ಈ ದೇಶವು ಉಳಿದಿರುವುದಕ್ಕೆ ಕಾರಣವೆಂದರೆ, ಧರ್ಮವೇ ರಾಷ್ಟ್ರಜೀವನದ ಸ್ಫೂರ್ತಿ ಚಿಲುಮೆಯಾಗಿರುವುದು, ಧರ್ಮವೆಂಬ ಹೃದಯದಿಂದಲೇ ಈ ರಾಷ್ಟ್ರದ ನಾಡಿಗಳಲ್ಲಿ ರಕ್ತವು ಸಂಚರಿಸುತ್ತಿರುವುದು.

ಜಗತ್ತಿನ ಇತರ ರಾಷ್ಟ್ರಗಳಿಗೆ ಧರ್ಮವೆಂಬುದು ಜೀವನದ ಹಲವು ಕಸುಬುಗಳಲ್ಲಿ ಒಂದು. ರಾಜಕೀಯವಿದೆ, ಸಮಾಜದ ಸುಖಭೋಗಗಳಿವೆ, ಐಶ್ವರ್ಯದಿಂದ ಮತ್ತು ಅಧಿಕಾರದಿಂದ ಗಳಿಸುವುದು ಎಷ್ಟೋ ಇವೆ. ಇಂದ್ರಿಯ ಸುಖಕ್ಕೆ ಬೇಕಾದಷ್ಟು ವಿಷಯವಸ್ತುಗಳಿವೆ. ಜೀವನದ ಇಂತಹ ಹಲವು ವ್ಯವಹಾರಗಳ ಮಧ್ಯೆ, ಕ್ಷಣಿಕ ಇಂದ್ರಿಯ ಸುಖ ತೃಪ್ತಿಗಾಗಿ ಹಲವು ವಸ್ತುಗಳನ್ನು ಅರಸುವಾಗ, ಅವುಗಳ ಮಧ್ಯೆ ಸ್ವಲ್ಪ ಧರ್ಮವೂ ಇದೆ. ಆದರೆ ಇಲ್ಲಿ, ಭರತಖಂಡದಲ್ಲಿ ಧರ್ಮವು ಏಕಮಾತ್ರ ವ್ಯವಹಾರ. ನಿಮ್ಮಲ್ಲಿ ಎಷ್ಟು ಜನರಿಗೆ ಚೀನಾ–ಜಪಾನಿನ ಯುದ್ಧ ಗೊತ್ತಿದೆ? ಯಾರಿಗಾದರೂ ಏನಾದರೂ ಗೊತ್ತಿದ್ದರೆ ಅದು ಬಹಳ ಕಡಮೆ. ಪಶ್ಚಿಮ ದೇಶಗಳಲ್ಲಿ ಸಮಾಜವನ್ನು ಪರಿವರ್ತಿಸುವ ಉದ್ದೇಶದಿಂದ ಅನೇಕ ರಾಜಕೀಯ ಮತ್ತು ಸಾಮಾಜಿಕ ಚಳುವಳಿಗಳು ನಡೆಯುತ್ತಿವೆ ಎಂಬುದು ಎಷ್ಟು ಜನಕ್ಕೆ ತಾನೆ ಗೊತ್ತು? ಯಾರಿಗಾದರೂ ಗೊತ್ತಿದ್ದರೆ ಅದು ಬಹಳ ಕಡಮೆ ಜನಕ್ಕೆ ಮಾತ್ರ. ಆದರೆ ಅಮೆರಿಕಾದಲ್ಲಿ ವಿಶ್ವಧರ್ಮ ಸಮ್ಮೇಳನ ಒಂದಾಯಿತು, ಅಲ್ಲಿಗೆ ಒಬ್ಬ ಹಿಂದೂ ಸಂನ್ಯಾಸಿಯನ್ನು ಕಳುಹಿಸಿದ್ದರು, ಎಂಬ ವರ್ತಮಾನವು ಒಬ್ಬ ಕೂಲಿಗೂ ತಿಳಿದಿರುವುದು ಆಶ್ಚರ್ಯ. ಇದು ಗಾಳಿ ಎತ್ತ ಬೀಸುತ್ತಿದೆ ಎಂಬುದನ್ನು ತೋರಿಸುತ್ತದೆ. ಅಲ್ಲದೆ ರಾಷ್ಟ್ರೀಯ ಜೀವನ\break ಇರುವುದೆಲ್ಲಿ ಎಂಬುದನ್ನು ಇದು ತಿಳಿಸುತ್ತದೆ. ವಿದೇಶೀ ಪ್ರವಾಸಿಗಳು ಬರೆದ ಪುಸ್ತಕಗಳನ್ನು ಓದುತ್ತಿದ್ದೆ. ಪ್ರಾಚ್ಯ ಜನಾಂಗದ ಮೌಢ್ಯವನ್ನು ನೋಡಿ ಅವರು ಅಲ್ಲಿ ವ್ಯಥೆಪಟ್ಟಿದ್ದಾರೆ. ಆದರೆ ಇದು ಸ್ವಲ್ಪ ಮಟ್ಟಿಗೆ ಸತ್ಯ, ಸ್ವಲ್ಪಮಟ್ಟಿಗೆ ಅಸತ್ಯವೆಂದು ನನಗೆ ತಿಳಿಯಿತು. ಇಂಗ್ಲೆಂಡ್​, ಅಮೆರಿಕಾ, ಫ್ರಾನ್ಸ್, ಜರ್ಮನಿ–ಈ ದೇಶಗಳ ರೈತರನ್ನು ನೀವು ಯಾವ ರಾಜಕೀಯ ಪಂಗಡಕ್ಕೆ ಸೇರಿದವರೆಂದು ಪ್ರಶ್ನಿಸಿದರೆ ಅವರು, ರ‌್ಯಾಡಿಕಲ್​ ಅಥವಾ ಕನ್​ಸರ್​ವೇಟಿವ್​ ಪಂಗಡಕ್ಕೆ ಸೇರಿದವರೆಂಬುದನ್ನೂ ಮತ್ತು ತಾವು ಯಾರಿಗೆ ಓಟು ಕೊಡುತ್ತೇವೆ ಎಂಬುದನ್ನೂ ಹೇಳುವರು. ಅಮೆರಿಕಾ ದೇಶದಲ್ಲಿ ತಾವು ರಿಪಬ್ಲಿಕನ್​ ಅಥವಾ ಡೆಮೊಕ್ರಾಟ್​ ಎಂದು ಹೇಳುವರು. ಬೆಳ್ಳಿಯ ಸಮಸ್ಯೆಯ ವಿಷಯವಾಗಿಯೂ ಅವರಿಗೆ ಸ್ವಲ್ಪ ಅನುಭವವಿದೆ. ಅವರನ್ನು ಧರ್ಮದ ವಿಷಯವಾಗಿ ಪ್ರಶ್ನಿಸಿದರೆ, ಅವರು ತಾವು ಚರ್ಚಿಗೆ ಹೋಗುತ್ತೇವೆ, ಯಾವುದೋ ಒಂದು ಪಂಗಡಕ್ಕೆ ಸೇರಿರುವೆವು ಎಂದು ಹೇಳುವರು. ಅವರಿಗೆ ಗೊತ್ತಿರುವುದೇ ಇಷ್ಟು. ಇಷ್ಟೇ ಸಾಕೆಂದು ಅವರು ಭಾವಿಸುವರು.

ನೀವು ಭರತಖಂಡದಲ್ಲಿ ಯಾರಾದರೂ ರೈತರನ್ನು ನಿಮಗೆ ರಾಜಕೀಯದ ವಿಷಯ ಏನಾದರೂ ಗೊತ್ತಿದೆಯೇ? ಎಂದು ಪ್ರಶ್ನಿಸಿದರೆ, ಅದೇನು? ಎನ್ನುವರು. ಅವರಿಗೆ ಸಮತಾವಾದಿಗಳ ಚಳುವಳಿ ಗೊತ್ತಿಲ್ಲ, ಬಂಡವಾಳಗಾರರಿಗೆ ಮತ್ತು ಕೂಲಿಗಾರರಿಗೆ ಇರುವ ಸಂಬಂಧ ಗೊತ್ತಿಲ್ಲ. ಅವರು ಈ ವಿಷಯವನ್ನೇ ಜೀವನದಲ್ಲಿ ಕೇಳಿಲ್ಲ. ಕಷ್ಟಪಟ್ಟು ದುಡಿದು\break ಸಂಪಾದಿಸುವರು. “ನಿನ್ನ ಧರ್ಮ ಯಾವುದು?” ಎಂದು ಪ್ರಶ್ನಿಸಿದರೆ, ಆತ “ನೋಡು ಮಿತ್ರನೆ, ಅದನ್ನು ಹಣೆಯ ಮೇಲೆ ಬರೆದಿರುವೆನು” ಎನ್ನುವನು. ಧರ್ಮಕ್ಕೆ ಸಂಬಂಧಪಟ್ಟ ಒಂದೆರಡು ವಿಷಯಗಳನ್ನು ಅವನು ಹೇಳಬಲ್ಲ. ಇದೇ ನನ್ನ ಅನುಭವ. ಇದೇ ನಮ್ಮ ರಾಷ್ಟ್ರ ಜೀವನ.

ವ್ಯಕ್ತಿಗಳಲ್ಲಿ ಎಷ್ಟೋ ವೈಶಿಷ್ಟ್ಯಗಳಿವೆ, ಪ್ರತಿಯೊಬ್ಬನೂ ತನ್ನ ಜೀವನದ ವಿಕಾಸದ ನಿಯಮವನ್ನು ಅನುಸರಿಸಿ ಬೆಳೆಯುವನು. ಅದನ್ನು ಅವನ ಹಿಂದಿನ ಕರ್ಮ ನಿರ್ಧರಿಸು\-ತ್ತದೆ ಎಂದು ಹಿಂದೂ ಹೇಳುವನು. ಈ ಪ್ರಪಂಚಕ್ಕೆ ಬರುವಾಗ ತನ್ನ ಹಿಂದಿನ ಕರ್ಮಗಳನ್ನೆಲ್ಲಾ ತೆಗೆದುಕೊಂಡು ಬರುವನು. ಅವನ ಅನಂತ ಭೂತಕಾಲ ಈ ವರ್ತಮಾನಕಾಲವನ್ನು ನಿರ್ಧರಿಸುವುದು; ಅದನ್ನು ಉಪಯೋಗಿಸಿಕೊಳ್ಳುವುದರ ಮೇಲೆ ಅವನ ಭವಿಷ್ಯ ನಿಂತಿದೆ. ಪ್ರಪಂಚದಲ್ಲಿ ಹುಟ್ಟಿದ ಪ್ರತಿಯೊಬ್ಬನಿಗೂ ಯಾವುದೋ ಒಂದು ಸಂಸ್ಕಾರವಿದೆ; ಅವನು ಯಾವುದೋ ಒಂದು ರೀತಿಯಲ್ಲಿ ಬಾಳಬೇಕಾಗಿದೆ. ಇದು ಒಂದು ವ್ಯಕ್ತಿಗೆ ಹೇಗೆ ಅನ್ವಯಿಸುವುದೋ ಹಾಗೆಯೇ ಒಂದು ಜನಾಂಗಕ್ಕೂ ಅನ್ವಯಿಸುವುದು. ಪ್ರತಿಯೊಂದು ಜನಾಂಗಕ್ಕೂ, ಹೀಗೆ ಒಂದು ವಿಧದ ಸಂಸ್ಕಾರವಿದೆ. ಪ್ರತಿಯೊಂದು ಜನಾಂಗವೂ ಒಂದು ರೀತಿಯಲ್ಲಿ ಆಲೋಚಿಸುವುದು. ಜಗತ್ತಿನಲ್ಲಿ ಪ್ರತಿಯೊಂದು ಜನಾಂಗವೂ ತನ್ನ ಪಾಲಿಗೆ ಬಂದ ವಿಶೇಷ ಕರ್ತವ್ಯವನ್ನು ಸಾಧಿಸಬೇಕಾಗಿದೆ. ಪ್ರತಿಯೊಂದು ಜನಾಂಗವೂ ತನ್ನ ಉದ್ದೇಶ ಸಾಧನೆಗೆ ತಾನೇ ಶ್ರಮಪಡಬೇಕಾಗಿದೆ. ರಾಜಕೀಯ ಮಹತ್ವವಾಗಲಿ, ಸೇನಾಶಕ್ತಿಯಾಗಲೀ ನಮ್ಮ ಜನಾಂಗದ ಗುರಿಯಲ್ಲ. ಅದು ಎಂದೂ ಹಿಂದೆ ಆಗಿರಲಿಲ್ಲ. ಗಮನದಲ್ಲಿಡಿ, ಅದೆಂದೂ ಮುಂದೆ ಆಗಲಾರದು. ಆದರೆ ನಮ್ಮ ಪಾಲಿಗೆ ಬಂದ ಕರ್ತವ್ಯ ಬೇರೆ ಇದೆ. ಜನಾಂಗದ ಆಧ್ಯಾತ್ಮಿಕ ಶಕ್ತಿಯನ್ನೆಲ್ಲಾ ಒಟ್ಟುಗೂಡಿಸಿ, ಕೇಂದ್ರೀಕರಿಸಿ, ಸಮಯ ಬಂದಾಗ ಆ ಮಹಾಪ್ರವಾಹದಿಂದ ಜಗತ್ತೆಲ್ಲವನ್ನು ಮುಳುಗಿಸುವುದು ನಮ್ಮ ಧ್ಯೇಯ. ಪಾರ್ಸಿಗಳು, ಯವನರು, ರೋಮನ್ನರು, ಅರಬ್ಬಿಯರು, ಆಂಗ್ಲೇಯರು ತಮ್ಮ ಸೇನಾಸಮೂಹದಿಂದ ಪ್ರಪಂಚವನ್ನು ಗೆದ್ದು ಬೇರೆ ಬೇರೆ ರಾಷ್ಟ್ರಗಳನ್ನು ಒಟ್ಟುಗೂಡಿಸಲಿ. ಆಗ ಈ ಹೊಸ ಮಾರ್ಗದ ಮೂಲಕ ಅನ್ಯಜನಾಂಗದ ನಾಡಿಯಲ್ಲಿ ಪ್ರವಹಿಸುವುದಕ್ಕೆ ಭಾರತೀಯ ತತ್ತ್ವ ಮತ್ತು ಆಧ್ಯಾತ್ಮಿಕ ಭಾವನೆಗಳು ಸದಾ ಅಣಿಯಾಗಿರುವುವು. ಮಾನವಕೋಟಿಯ ಪ್ರಗತಿಗೆ, ಶಾಂತಿಪರಾಯಣ ಭಾರತೀಯನು ತನ್ನ ಪಾಲಿನದನ್ನೂ ನೀಡಬೇಕಾಗಿದೆ. ಅಧ್ಯಾತ್ಮಜೀವನವೇ ಪ್ರಪಂಚಕ್ಕೆ ಭರತಖಂಡ ಕೊಡುವ ಕಾಣಿಕೆ.

ಇತಿಹಾಸವನ್ನು ಓದಿದರೆ, ಯಾವ ಯಾವಾಗ ಒಂದು ಪ್ರಬಲ ದಿಗ್ವಿಜಯೀ ರಾಷ್ಟ್ರವು ಅನ್ಯ ಜನಾಂಗಗಳನ್ನು ಗೆದ್ದು, ಅವನ್ನು ಸೂತ್ರದಲ್ಲಿ ಬಂಧಿಸಿ ಅವುಗಳೊಡನೆ ಪ್ರತ್ಯೇಕತಾ ಪ್ರಿಯ ಭರತಖಂಡವನ್ನು ಕೂಡ ಸೇರಿಸಿತೋ, ಆವಾಗಾವಾಗ ಅದರ ಫಲರೂಪವಾಗಿ ಸಮಗ್ರ ಜಗತ್ತಿಗೆ ಭಾರತೀಯ ಆಧ್ಯಾತ್ಮಿಕ ತರಂಗ ವ್ಯಾಪಿಸಿರುವುದು ಕಾಣಿಸುತ್ತದೆ. ಪಾರ್ಸೀ ಭಾಷೆಗೆ ಅನುವಾದವಾಗಿದ್ದ ಉಪನಿಷತ್ತುಗಳನ್ನು ಫ್ರಾನ್ಸಿನ ತರುಣನೊಬ್ಬ ಲ್ಯಾಟಿನ್​ ಭಾಷೆಗೆ ಅನುವಾದ ಮಾಡಿದ್ದನು. ಆ ಅಸಮರ್ಪಕ ಭಾಷಾಂತರವನ್ನು ಈ ಶತಮಾನದ ಆದಿಯಲ್ಲಿ ಪ್ರಖ್ಯಾತ ಜರ್ಮನ್​ ತತ್ತ್ವಜ್ಞಾನಿ ಷೋಫೆನ್ಹೇರ್​ ಎಂಬುವನು ಓದಿ, “ಪ್ರಪಂಚದಲ್ಲೆಲ್ಲಾ ಉಪನಿಷತ್ತಿನ ಅಧ್ಯಯನದಷ್ಟು ಪ್ರಯೋಜನಕಾರಿ ಮತ್ತೊಂದು ಇಲ್ಲ. ಅದು ನನ್ನ ಜೀವನಕ್ಕೆ ಶಾಂತಿಯನ್ನು ನೀಡಿದೆ. ಮೃತ್ಯುಸಮ್ಮುಖದಲ್ಲಿಯೂ ಶಾಂತಿಯ ಭರವಸೆಯನ್ನು ನೀಡುತ್ತದೆ” ಎಂದಿರುವನು. ಆ ಪ್ರಖ್ಯಾತ ಜರ್ಮನ್​ ಭವಿಷ್ಯವಾಣಿ ಇದು: “ಗ್ರೀಕ್​ ಸಾಹಿತ್ಯದ ಪುನುರುದ್ಧಾರಕ್ಕಿಂತ ಪ್ರಬಲವಾಗಿರುವ ಮತ್ತು ವಿಸ್ತಾರವಾಗಿರುವ ಆಲೋಚನಾ ಕ್ರಾಂತಿಯನ್ನು ಜಗತ್ತು ಕಾಣುವುದು.” ಅವನ ಭವಿಷ್ಯವಾಣಿ ಇಂದು ಸಫಲವಾಗುತ್ತಿರುವುದು. ಯಾರು ಕಣ್ಣು ತೆರೆದು ನೋಡುತ್ತಿರುವರೋ, ಪಾಶ್ಚಾತ್ಯ ಜನಾಂಗದ ವಿಭಿನ್ನ ಆಲೋಚನೆ\-ಗಳನ್ನು ಗ್ರಹಿಸಬಲ್ಲರೋ, ಯಾವ ಆಲೋಚನಾಶೀಲರು ಅನ್ಯದೇಶಗಳ ಅಧ್ಯಯನ ಮಾಡುತ್ತಿರುವರೋ, ಅವರಿಗೆ, ಭಾರತೀಯ ಆಲೋಚನೆಯು ಧೀರವಾಗಿ ಅವಿರಾಮವಾಗಿ, ಜಗತ್ತಿನ ಮೇಲೆ ತನ್ನ ಪ್ರಭಾವವನ್ನು ಬೀರುತ್ತಿರುವುದು ಕಾಣುತ್ತದೆ. ಜಗತ್ತಿನ ಭಾವಗತಿಯಲ್ಲಿ, ಅದರ ಸಾಹಿತ್ಯದಲ್ಲಿ ಒಂದು ಹೊಸ ರೀತಿಯು ಮೂಡುತ್ತಿರುವುದನ್ನು ಕಾಣುತ್ತೇವೆ.

\vskip   4pt

ಆದರೆ ಭಾರತವು ತನ್ನ ಪ್ರಭಾವವನ್ನು ಬೀರುತ್ತಿರುವ ರೀತಿಯಲ್ಲಿ ವೈಶಿಷ್ಟ್ಯವಿದೆ. ನಾವು ನಮ್ಮ ಭಾವನೆಗಳನ್ನು ಕತ್ತಿಯ ಅಥವಾ ಕೋವಿಯ ಬಲದಿಂದ ಹರಡಲಿಲ್ಲ. ಪ್ರಪಂಚಕ್ಕೆ ಭರತಖಂಡವು ಕೊಟ್ಟ ಕಾಣಿಕೆಯನ್ನು ಸೂಚಿಸಬಲ್ಲ ಒಂದು ಪದವು ಆಂಗ್ಲ ಭಾಷೆ\- ಯಲ್ಲಿದ್ದರೆ, ಭಾರತೀಯ ಸಾಹಿತ್ಯವು ಮಾನವಕೋಟಿಯ ಮೇಲೆ ಬೀರಿದ ಪರಿಣಾಮವನ್ನು ವ್ಯಕ್ತಪಡಿಸುವ ಒಂದು ಪದವು ಆಂಗ್ಲ ಭಾಷೆಯಲ್ಲಿದ್ದರೆ ಆ ಪದವೇ \enginline{fascination}, ಸಮ್ಮೋಹಿನೀ ಶಕ್ತಿ ಎಂಬುದು. ಹಠಾತ್ತಾಗಿ ಮನುಷ್ಯನನ್ನು ಮುಗ್ಧ ಮಾಡುವಂಥವೆಲ್ಲದಕ್ಕೂ ವಿರುದ್ಧವಾದುದು ಅದು. ತನ್ನ ಸಮ್ಮೋಹನ ಶಕ್ತಿಯನ್ನು ಅದು ಅರಿವಿಗೆ ಬಾರದ ರೀತಿಯಲ್ಲಿ ಬೀರುತ್ತದೆ. ಅನೇಕರಿಗೆ ಭಾರತೀಯ ಚಿಂತನೆ, ಭಾರತೀಯರ ರೀತಿ ನೀತಿಗಳು, ಭಾರತೀಯ ದರ್ಶನ, ಸಾಹಿತ್ಯಗಳು ಮೊದಲ ನೋಟಕ್ಕೆ ಜುಗುಪ್ಸೆ ಹುಟ್ಟಿಸುತ್ತವೆ. ಆದರೆ ಅವರು ಮುಂದುವರಿಯಲಿ, ಅವುಗಳನ್ನು ಓದಲಿ, ಅವರಿಗೆ ಆ ಭಾವನೆಗಳ ಹಿಂದೆ ಇರುವ ತತ್ತ್ವಗಳ ಪರಿಚಯವಾಗಲಿ, ಆಗ ನೂರರಲ್ಲಿ ತೊಂಬತ್ತೊಂಬತ್ತು ಮಂದಿ ಮುಗ್ಧರಾಗುವರು; ಅದರ ಆಕರ್ಷಣೆಗೆ ಸಿಲುಕುವರು. ನಿಧಾನವಾಗಿ, ಮೌನವಾಗಿ, ಮುಂಜಾನೆ\- ಯಲ್ಲಿ ಬೀಳುವ ಹಿಮಮಣಿಯು, ಕೇಳಿಸದೇ ಇದ್ದರೂ, ಕಾಣದೇ ಇದ್ದರೂ, ಅದ್ಭುತ ಪರಿಣಾಮವನ್ನುಂಟುಮಾಡುವಂತೆ ಶಾಂತವಾದ, ಸಹಿಷ್ಣುವಾದ, ಅಧ್ಯಾತ್ಮ ಪ್ರಧಾನವಾದ ಭಾರತೀಯ ಚಿಂತನೆಯು, ಜಗತ್ತಿನ ಚಿಂತನೆಯ ಮೇಲೆ ಪ್ರಭಾವವನ್ನು ಬೀರಿದೆ.

ಪುನಃ ಪ್ರಾಚೀನ ಇತಿಹಾಸದ ಪುನರಭಿನಯ ಪ್ರಾರಂಭವಾಗಿದೆ. ಇಂದು ಆಧುನಿಕ ವಿಜ್ಞಾನದ ಆವಿಷ್ಕಾರಗಳಿಂದಾಗಿ ಕೇವಲ ತೋರಿಕೆಗೆ ಸದೃಢವಾಗಿ ಕಾಣುತ್ತಿದ್ದ ಹಳೆಯ ನಂಬಿಕೆಗಳ ಅಡಿಪಾಯವೇ ಕುಸಿದು ಹೋಗುತ್ತಿದೆ. ವಿಭಿನ್ನ ಸಂಪ್ರದಾಯಗಳು ಮಾನವ ಕೋಟಿಯ ಮೇಲೆ ಸ್ಥಾಪಿಸುತ್ತಿದ್ದ ತಮ್ಮ ವಿಶೇಷ ಅಧಿಕಾರಗಳೆಲ್ಲ ಪುಡಿಪುಡಿಯಾಗಿ ಬೀಳುತ್ತಿವೆ. ಆಧುನಿಕ ಕಾಲದ ಪ್ರಾಚೀನ ವಸ್ತುಗಳ ಭಾವನೆಗಳ ಸಂಶೋಧನೆಗಳೆಂಬ ಪ್ರಬಲ ಮುಸಲಧಾರೆಯ ಎದುರಿಗೆ ಸಂಪ್ರದಾಯನಿಷ್ಠರ ನಂಬಿಕೆಗಳೆಲ್ಲಾ ಗಾಜಿನಂತೆ ಚೂರು ಚೂರಾಗುತ್ತಿವೆ. ಪಾಶ್ಚಾತ್ಯ ಜಗತ್ತಿನಲ್ಲಿ ಧರ್ಮವು ಕೇವಲ ಮೂಢಮತಿಗಳ ಆದರಣೆಗೆ ಮಾತ್ರ ಪಾತ್ರವಾಗಿದೆ. ಕೃತವಿದ್ಯರು ಧರ್ಮಕ್ಕೆ ಸಂಬಂಧಪಟ್ಟ ಅಂಶಗಳನ್ನು ತಾತ್ಸಾರದಿಂದ ನೋಡುತ್ತಿರುವರು. ಇಂತಹ ಪರಿಸ್ಥಿತಿಯಲ್ಲಿ ಭಾರತೀಯ–ದರ್ಶನ ಮತ್ತು ಭಾರತೀಯರು ಅನುಷ್ಠಾನದಲ್ಲಿ ತೋರುವ ಸರ್ವೋಚ್ಚ ಧಾರ್ಮಿಕ ಭಾವನೆಗಳು ಪ್ರಮುಖವಾಗುತ್ತಿವೆ. ಅಸೀಮ ಅನಂತ ಜಗತ್ತಿನಲ್ಲಿ ಏಕತ್ವ, ನಿರ್ಗುಣ ಬ್ರಹ್ಮವಾದ, ಜೀವಾತ್ಮನ ಅನಂತ ಸ್ವರೂಪ, ಶರೀರಗಳಲ್ಲಿ ಅವಿಚ್ಛೇದ ಸಂಕ್ರಮಣರೂಪದಲ್ಲಿರುವ ಅಪೂರ್ವ ತತ್ತ್ವ, ಬ್ರಹ್ಮಾಂಡದ ಅನಂತತ್ವ–ಇವು ನಮ್ಮನ್ನು ಪಾರುಮಾಡಲು ಬರುತ್ತಿವೆ. ಪ್ರಾಚೀನ ಸಂಪ್ರದಾಯ ಬದ್ಧರು ಪ್ರಪಂಚವು ಒಂದು ಮಣ್ಣಿನ ಮುದ್ದೆ ಎಂದೂ ಸೃಷ್ಟಿ ಮೊನ್ನೆ ಮೊನ್ನೆಯಷ್ಟೆ ಆರಂಭವಾಯಿತೆಂದೂ ಭಾವಿಸಿದ್ದರು. ಆದರೆ ಕೇವಲ ನಮ್ಮ ಪ್ರಾಚೀನ ಶಾಸ್ತ್ರಗಳಲ್ಲಿ ಮಾತ್ರ ದೇಶ ಕಾಲ ನಿಮಿತ್ತದ ಅನಂತತೆ ಮತ್ತು ಜೀವಾತ್ಮನ ಅನಂತ ಮಹಿಮೆ ಎಲ್ಲಾ ಧರ್ಮದ ಅನ್ವೇಷಣೆಯ ಹಿಂದೆ ಇರುವ ಆಧಾರವಾಗಿದೆ. ಕ್ರಮ–ವಿಕಾಸವಾದ, ಶಕ್ತಿಸ್ಥಾಯಿತ್ವ\break \enginline{(Conservation of energy)} ಮುಂತಾದ ಆಧುನಿಕ ಅದ್ಭುತ ಸಿದ್ಧಾಂತಗಳು ಕ್ಷುದ್ರ ಧರ್ಮಗಳಿಗೆ ಕುಠಾರ ಪ್ರಾಯವಾಗಿವೆ. ಅದ್ಭುತವಾದ, ಯುಕ್ತಿಪೂರ್ಣವಾದ, ವಿಶಾಲವಾದ, ಭವ್ಯ ಭಾವನೆಗಳು ವೇದಾಂತದಲ್ಲಿ ಮಾತ್ರ ದೊರಕುವುವು. ಮಾನವನ ಅತಿ ಅದ್ಭುತ ಕೃತಿ ಇದು. ಇದೇ ದೇವವಾಣಿ, ಇದೇ ಉಪನಿಷತ್ತು. ಆಧುನಿಕ ಕಾಲದಲ್ಲಿ ಉಪನಿಷತ್ತಿನ ಭಾವನೆಗಳು ಮಾತ್ರ ಸುಸಂಸ್ಕೃತ ಮಾನವಕುಲಕ್ಕೆ ಶ್ರದ್ಧಾಭಕ್ತಿಗಳನ್ನು ಕೊಡಬಲ್ಲವು.

ಪರರಾಷ್ಟ್ರಗಳ ಮೇಲೆ ನಮ್ಮ ಧರ್ಮವು ತನ್ನ ಪ್ರಭಾವವನ್ನು ಬೀರುತ್ತಿದೆ ಎಂದು ಹೇಳಿದಾಗ ಅದು ನಮ್ಮ ಧರ್ಮದಲ್ಲಿ ಮೂಲಭೂತವಾದ, ಹಿನ್ನೆಲೆಯಾದ, ತತ್ತ್ವಗಳಿಗೆ ಮಾತ್ರ ಅನ್ವಯಿಸುತ್ತದೆ ಎಂದು ಹೇಳಬೇಕಾಗಿದೆ. ಹಲವು ಶತಮಾನಗಳಿಂದ ಸಮಾಜದ ಹಿತರಕ್ಷಣೆಯ ಆವಶ್ಯಕತೆಗಾಗಿ ಉದಯಿಸಿದ ಆಚಾರ ವ್ಯವಹಾರಗಳು ಗೌಣ: ಅವು ಧರ್ಮವಲ್ಲ. ನಮ್ಮ ಶಾಸ್ತ್ರದಲ್ಲಿ ಎರಡು ವಿಧದ ಸತ್ಯಗಳನ್ನು ಹೇಳುವರು. ಒಂದು ವಿಧವಾದ ಸತ್ಯಗಳು ಚಿರಕಾಲ ಇರುವಂತಹವುಗಳು. ಇವು ಮಾನವನ ಸ್ವರೂಪ, ಆತ್ಮನ ಸ್ವರೂಪ, ಆತ್ಮ ಪರಮಾತ್ಮರ ಸಂಬಂಧ, ಪರಮಾತ್ಮನ ಸ್ವರೂಪ ಇತ್ಯಾದಿಗಳಿಗೆ ಸಂಬಂಧಿಸಿದವು. ಇವಲ್ಲದೆ ಸೃಷ್ಟಿ ತತ್ತ್ವವಿದೆ. ಈ ಸೃಷ್ಟಿ ಅನಂತವಾದುದು, ಜಗತ್ತು ಪರಮಾತ್ಮನಿಂದ ಆವಿರ್ಭವಿಸಿ\break ಅವನಲ್ಲಿಯೇ ತಿರೋಭಾವಗೊಳ್ಳುವುದು. ಈ ಸೃಷ್ಟಿಚಕ್ರ ನಿರಂತರ ಮುಂದುವರಿಯುತ್ತದೆ. ಇವು ಪ್ರಕೃತಿಯ ಸರ್ವವ್ಯಾಪಿ ನಿಯಮಾವಳಿಯ ಮೇಲೆ ನಿಂತ ಸನಾತನ ತತ್ತ್ವಗಳು. ಎರಡನೆಯ ವಿಧದ ಸತ್ಯಗಳು ಗೌಣ ನಿಯಮಾವಳಿಗಳನ್ನು ಒಳಗೊಳ್ಳುತ್ತವೆ. ಇವು ದೈನಂದಿನ ಕಾರ್ಯನಿರ್ವಹಣೆಯಲ್ಲಿ ಮಾರ್ಗದರ್ಶನ ನೀಡುತ್ತವೆ. ಇವು ಪುರಾಣಕ್ಕೆ ಸ್ಮೃತಿಗೆ ಸೇರಿವೆ, ಶ್ರುತಿಗಳಿಗಲ್ಲ. ಇವಕ್ಕೂ ಆಧ್ಯಾತ್ಮಿಕ ತತ್ತ್ವಕ್ಕೂ ಸಂಬಂಧವಿಲ್ಲ. ನಮ್ಮ ದೇಶದಲ್ಲಿ ಈ ಗೌಣ ನಿಯಮಗಳು ಕಾಲಕಾಲಕ್ಕೆ ಬದಲಾಗುತ್ತಿವೆ. ಒಂದು ಕಾಲದ, ಒಂದು ಯುಗದ ಆಚಾರವ್ಯವಹಾರಗಳು ಮತ್ತೊಂದು ಕಾಲಕ್ಕೆ ಮತ್ತು ಯುಗಕ್ಕೆ ಅನ್ವಯಿಸುವುದಿಲ್ಲ. ಯುಗವಾದ ಮೇಲೆ ಯುಗ ಬರುವುದರಿಂದ ಅವು ಇನ್ನೂ ಬದಲಾಗಬೇಕಾಗುತ್ತವೆ. ಮಹಾಋಷಿಗಳು ಜನಿಸಿ ಆಯಾಯ ದೇಶಕಾಲಗಳಿಗೆ ಉಪಯೋಗವಾಗುವಂತಹ ನವನವ ಆಚಾರ\break ವ್ಯವಹಾರಗಳನ್ನು ಜಾರಿಗೆ ತರುವರು.

ಜೀವಾತ್ಮ, ಪರಮಾತ್ಮ, ಬ್ರಹ್ಮಾಂಡ ಇವುಗಳಿಗೆ ಸಂಬಂಧಪಟ್ಟ ಅಪೂರ್ವ ಅನಂತ ಭವ್ಯ ಮಹಾತತ್ತ್ವಗಳು ಭರತಖಂಡದಲ್ಲಿ ಉತ್ಪನ್ನವಾಗಿವೆ. ಕೇವಲ ಭರತ ಖಂಡದಲ್ಲಿ ಮಾತ್ರ ಜನರು ಒಂದು ಸಣ್ಣ ಕೋಮಿನ ದೇವರ ಹೆಸರಿನಲ್ಲಿ, “ನನ್ನ ದೇವರೇ ಸತ್ಯ, ನಿನ್ನದಲ್ಲ, ಅದನ್ನು ಕಾದಾಡಿ ಬಗೆಹರಿಸೋಣ” ಎಂದು ಹೇಳುವುದಿಲ್ಲ. ಕ್ಷುದ್ರ ದೇವತೆಗಳ ಹೆಸರಿನಲ್ಲಿ ಕಾದಾಟ ಇಲ್ಲಿ ಹುಟ್ಟಲಿಲ್ಲ. ಮಾನವನ ಅನಂತತೆಯ ಮೇಲೆ ನಿಂತ ಈ ಸನಾತನ ತತ್ತ್ವಗಳು ಮಾನವನ ಹಿತಕ್ಕೆ ಹೇಗೆ ಸಾವಿರಾರು ವರ್ಷಗಳ ಹಿಂದೆ ಪ್ರಯೋಜನಕಾರಿಯಾಗಿದ್ದವೋ ಹಾಗೆಯೇ ಈಗಲೂ ಆಗಿವೆ. ಎಲ್ಲಿಯವರೆಗೂ ಸೃಷ್ಟಿ ಇರುತ್ತದೆಯೋ, ಕರ್ಮನಿಯಮಗಳು ಜಾರಿಯಲ್ಲಿರುತ್ತವೆಯೋ, ಎಲ್ಲಿಯವರೆಗೆ ನಾವು ಜೀವಿಗಳಂತೆ ಹುಟ್ಟಿ ನಮ್ಮ ಶಕ್ತಿಯ ದ್ವಾರಾ ನಮ್ಮ ಕರ್ಮಾನುಸಾರವಾಗಿ ನಮ್ಮ ಭವಿಷ್ಯವನ್ನು ಸೃಷ್ಟಿಸಿಕೊಳ್ಳಬೇಕಾಗಿದೆಯೋ ಅಲ್ಲಿಯವರೆಗೂ ಈ ನಿಯಮಗಳು ಚಿರಜಾಗೃತವಾಗಿರುವುವು.

ಎಲ್ಲಕ್ಕಿಂತ ಹೆಚ್ಚಾಗಿ ಭರತಖಂಡವು ಜಗತ್ತಿಗೆ ನೀಡಬೇಕಾದುದು ಇದು: ಭಿನ್ನ ಭಿನ್ನ ಜನಾಂಗಗಳಲ್ಲಿ ಹೇಗೆ ಧರ್ಮಗಳು ಹುಟ್ಟಿದವು, ಪ್ರವರ್ಧಮಾನಕ್ಕೆ ಬಂದವು ಎಂಬುದನ್ನು ನೋಡಿದರೆ, ಪ್ರತಿಯೊಂದು ಬುಡಕಟ್ಟಿನವರಿಗೂ ಮೊದಲು ಒಂದೊಂದು ದೇವತೆ ಇತ್ತು. ಬೇರೆ ಬೇರೆ ಬುಡಕಟ್ಟುಗಳಿಗೆ ಪರಸ್ಪರ ಸಂಬಂಧವಿದ್ದರೆ ಅವರ ದೇವತೆಗಳಿಗೆ, ಬ್ಯಾಬಿಲೋನಿಯಾದ ದೇವತೆಗಳಿಗೆ ಇದ್ದಂತೆ, ಒಂದು ಸಾಮಾನ್ಯ ಹೆಸರು ಇರುತ್ತಿತ್ತು. ಬ್ಯಾಬಿಲೋನಿಯನ್​ ಜನರು ವಿವಿಧ ಜನಾಂಗಗಳಾಗಿ ವಿಭಾಗಗೊಂಡಿದ್ದಾಗ ಅವರ ದೇವತೆಗಳ ಸಾಮಾನ್ಯ ಹೆಸರು ‘ಬಾಲ್​’ ಎಂಬುದಾಗಿತ್ತು. ಹಾಗೆಯೇ ಯಹೂದ್ಯರ ವಿವಿಧ ದೇವತೆಗಳಿದ್ದವು, ಹಾಗೂ ಅವುಗಳ ಸಾಮಾನ್ಯ ಹೆಸರು ‘ಮೊಲಾಕ್​’ ಎಂದು. ಅವರಲ್ಲಿ ಯಾವುದಾದರೂ ಒಂದು ಪಂಗಡವು ಉಳಿದವುಗಳಿಗಿಂತ ಪ್ರಬಲವಾದಾಗ ಆ ಪ್ರಬಲವಾದ ಪಂಗಡದ ರಾಜನು ಉಳಿದ ಪಂಗಡಗಳಿಗೂ ರಾಜನಾಗಬೇಕೆಂದು ಅವರು ಬಯಸುತ್ತಿದ್ದರು. ಹಾಗೆಯೇ ತಮ್ಮ ದೇವತೆಯೇ ಉಳಿದ ಪಂಗಡಗಳ ದೇವತೆಯೂ ಆಗಬೇಕೆಂದು ಬಯಸುತ್ತಿದ್ದುದು ಸಹಜವಾಗಿತ್ತು. ಬ್ಯಾಬಿಲೋನಿಯಾದವರು ‘ಬಾಲ್​ ಮೊರೊಡಾಕ’ ನೇ ಅತ್ಯುತ್ತಮ ದೇವತೆ, ಉಳಿದವರೆಲ್ಲ ಗೌಣ ಎನ್ನುತ್ತಿದ್ದರು. ‘ಮೊಲಾಕ್​–ಯಾವಾ’ ಎಂಬ ದೇವತೆ ಉಳಿದವರೆಲ್ಲರಿಗಿಂತಲೂ ಬಲಾಢ್ಯ ದೇವರು. ಈ ಮೊದಲಾದ ಪ್ರಶ್ನೆಗಳನ್ನು ಯುದ್ಧ ಮಾಡಿ ನಿಷ್ಕರ್ಷಿಸಬೇಕಾಗಿತ್ತು. ಭಾರತದಲ್ಲಿಯೂ ಅದೇ ಹೋರಾಟವಿತ್ತು. ಇಲ್ಲಿಯೂ ಆ ಶ್ರೇಷ್ಠ ಪದವಿಗೆ ಹಲವು ದೇವತೆಗಳು ಸ್ಪರ್ಧಿಸುತ್ತಿದ್ದರು. ಆದರೆ ಈ ದೇಶದ ಮತ್ತು ಪ್ರಪಂಚದ ಒಂದು ಮಹಾಭಾಗ್ಯವೇ ಇಂತಹ ಒಂದು ಗೊಂದಲದ ಮಧ್ಯದಿಂದ “ಏಕಂ ಸತ್​ ವಿಪ್ರಾ ಬಹುಧಾ ವದಂತಿ” ಎಂಬ ಧ್ವನಿಯು ಉದಯಿಸಿದ್ದು. ಶಿವ ವಿಷ್ಣುವಿಗಿಂತ ಮೇಲೆಂದಾಗಲಿ, ವಿಷ್ಣು ಶಿವನಿ\-ಗಿಂತ ಮೇಲೆಂದಾಗಲೀ ಅಲ್ಲ. ನೀವು ಶಿವನೆನ್ನುವುದೂ ಅದೇ, ವಿಷ್ಣುವೆನ್ನುವುದೂ ಅದೇ, ಬೇರೆ ನೂರಾರು ಹೆಸರುಗಳಿಂದ ಕರೆಯುವುದೂ ಅದೇ. ಹೆಸರುಗಳು ಬೇರೆ ಬೇರೆ. ಆದರೂ ಅವೆಲ್ಲಾ ಒಂದೇ ವಸ್ತುವನ್ನು ನಿರ್ದೇಶಿಸುವುವು. ಈ ಕೆಲವು ಪದಗಳಲ್ಲಿ ಭಾರತೀಯ ಇತಿಹಾಸವೆಲ್ಲಾ ಹುದುಗಿದೆ. ಈ ಒಂದು ಮಹಾತತ್ತ್ವವನ್ನು ಓಜಸ್ಸಿನಿಂದ ಕೂಡಿದ ಸವಿಸ್ತಾರವಾದ ಭಾಷೆಯಲ್ಲಿ ಇತಿಹಾಸವು ಪುನಃ ಪುನಃ ತೋರಿಸುತ್ತಿದೆ. ಮತ್ತೆ ಮತ್ತೆ ಈ ಮಹಾಸಂದೇಶವನ್ನು ಸಾರುತ್ತಿದೆ. ದೇಶದ ನಾಡಿನಾಡಿಗಳಲ್ಲಿ ಸಂಚರಿಸುವ ಪ್ರತಿಯೊಂದು ರಕ್ತ ಬಿಂದುವಿನಲ್ಲಿಯೂ, ಜನಜೀವನದಲ್ಲಿಯೂ ಓತಪ್ರೋತವಾಗಿ, ರಾಷ್ಟ್ರಜೀವನದಲ್ಲಿ ಅನುರಣಿತವಾಗುವವರೆಗೆ ಈ ಮಹಾಸತ್ಯವನ್ನು ಸಾರಿರುವರು. ಹೀಗೆ ಈ ದೇಶವು ಅದ್ಭುತ ಅನ್ಯಮತಸಹಿಷ್ಣು ಕ್ಷೇತ್ರವಾಗಿ ಪರಿಣಮಿಸಿದೆ. ಹಲವು ಧರ್ಮಗಳನ್ನು ಮತ್ತು ಪಂಗಡಗಳನ್ನು ಈ ನಮ್ಮ ಪ್ರಾಚೀನ ಮಾತೃಭೂಮಿಗೆ ಸ್ವಾಗತಿಸುವ ಹಕ್ಕು ನಮ್ಮ\-ದಾಯಿತು.

ಹಲವು ಧರ್ಮ ಸಂಪ್ರದಾಯಗಳು ಪರಸ್ಪರ ವಿರುದ್ಧವಾಗಿ ಕಂಡರೂ ಸೌಹಾರ್ದದಿಂದ ಬಾಳುವಂತಹ ಅದ್ಭುತ ನಿದರ್ಶನಕ್ಕೆ ಇಲ್ಲಿ ಮಾತ್ರ ವಿವರಣೆ ಇದೆ. ನೀವು ದ್ವೈತಿಯಾಗಿರ\-ಬಹುದು. ಮತ್ತೊಬ್ಬನು ಅದ್ವೈತಿಯಾಗಿರಬಹುದು, ನೀವು ಭಗವಂತನ ಸೇವಕರೆಂದು ಭಾವಿಸಿರಬಹುದು, ಮತ್ತೊಬ್ಬನು ತಾನೇ ದೇವರೆಂದು ಭಾವಿಸಿರಬಹುದು. ಆದರೆ ಇಬ್ಬರೂ ಒಳ್ಳೆಯ ಹಿಂದೂಗಳೇ. ಇದು ಹೇಗೆ ಸಾಧ್ಯ? ಇದನ್ನು ಓದಿ: “ಇರುವುದೊಂದೇ, ಅದನ್ನು ಋಷಿಗಳು ಬೇರೆ ಬೇರೆ ಹೆಸರುಗಳಿಂದ ಕರೆಯುವರು.” ನನ್ನ ದೇಶಬಾಂಧವರೇ, ಎಲ್ಲಕ್ಕಿಂತ ಹೆಚ್ಚಾಗಿ ಈ ಒಂದು ಮಹಾಸತ್ಯವನ್ನು ಜಗತ್ತಿಗೆ ನಾವು ಸಾರಬೇಕಾಗಿದೆ. ವಿದೇಶಗಳ ವಿದ್ಯಾವಂತರು ಕೂಡ ನಮ್ಮ ಧರ್ಮವನ್ನು ವಿಗ್ರಹಾರಾಧನೆ ಎಂದು ಮೂಗು ಮುರಿಯುವರು. ನಾನು ಇದನ್ನು ನೋಡಿರುವೆನು. ತಮ್ಮಲ್ಲೇ ಎಷ್ಟೊಂದು ಮೂಢನಂಬಿಕೆಗಳಿವೆ ಎಂಬುದನ್ನು ಅವರು ಆಲೋಚಿಸುವುದೇ ಇಲ್ಲ. ಈಗಲೂ ಈ ಮಹಾ ಮತಭ್ರಾಂತಿ, ಕೂಪಮಂಡೂಕ ನೀತಿ, ಹಾಗೆಯೇ ಇದೆ. ಒಬ್ಬನಿಗೆ ಏನು ಇದೆಯೋ ಅದೇ ಪ್ರಪಂಚದಲ್ಲೆಲ್ಲಾ ಸಾರ್ಥಕವಾದುದು. ಐಶ್ವರ್ಯದ, ಅಧಿಕಾರದ ಆರಾಧನೆಯಿಂದ ಕೂಡಿದ ತನ್ನ ಅಲ್ಪ ಬಾಳುವೆಯೆ ಅತ್ಯುತ್ತಮ ಜೀವನ. ಪ್ರಪಂಚದಲ್ಲೆಲ್ಲಾ ಸಂಗ್ರಹ ಯೋಗ್ಯವಾದುದೇ ತನ್ನ ಆಸ್ತಿ, ಬೇರೊಂದು ಯಾವುದೂ ಇಲ್ಲ. ಮಣ್ಣಿನ ಮುದ್ದೆಯಿಂದ ಏನನ್ನಾದರೂ ಮಾಡಿದರೆ ಅಥವಾ ಯಾವುದಾದರೂ ಒಂದು ಯಂತ್ರವನ್ನು ಕಂಡುಹಿಡಿದರೆ, ಅದನ್ನೇ ಪರಮೋಚ್ಚ ವಸ್ತುವೆಂದು ಹೊಗಳಬೇಕು. ಎಷ್ಟೇ ವಿದ್ಯಾವಂತರಾದರೂ ಪ್ರಪಂಚದಲ್ಲೆಲ್ಲಾ ಹೀಗೆಯೇ. ಪ್ರಪಂಚಕ್ಕೆ ಇನ್ನೂ ವಿದ್ಯೆ ಮತ್ತು ನಾಗರಿಕತೆ ಬರಬೇಕಾಗಿದೆ. ಎಲ್ಲಿಯೂ ಸಂಸ್ಕೃತಿ ಇನ್ನೂ ಪ್ರಾರಂಭವಾಗಿಲ್ಲ. ಶೇಕಡ ೯೯.೯ ಜನರು ಅನಾಗರಿಕರು. ಗ್ರಂಥಗಳಲ್ಲಿ ನಾವು ಇದನ್ನು ಓದಬಹುದು. ಧರ್ಮದಲ್ಲಿ ಸಹಿಷ್ಣುತೆ ಎಂಬ ಪದವನ್ನು ಕೇಳಬಹುದು. ಆದರೆ ಅನುಷ್ಠಾನದಲ್ಲಿರುವುದು ಅತ್ಯಲ್ಪ. ನನ್ನ ಅನುಭವವೇ ಅದಕ್ಕೆ ಪ್ರಮಾಣ. ಶೇಕಡ ತೊಂಬತ್ತೊಂಬತ್ತು ಮಂದಿ ಅದನ್ನು ಆಲೋಚಿಸುವುದೂ ಇಲ್ಲ. ನಾನು ಕಂಡ ದೇಶಗಳಲ್ಲೆಲ್ಲ ಧರ್ಮದ ಹೆಸರಿನಲ್ಲಿ ಹಿಂಸೆಯಾಗುತ್ತಿದೆ. ಏನಾದರೂ ಹೊಸದಾಗಿ ಕಲಿಯಬೇಕಾದರೆ ಹಿಂದಿನ ಆಕ್ಷೇಪಣೆಯನ್ನೇ ಎತ್ತುವರು. ಪ್ರಪಂಚದಲ್ಲಿ ಎಲ್ಲಿಯಾದರೂ ಧಾರ್ಮಿಕ ಭಾವನೆಗಳ ವಿಷಯದಲ್ಲಿ ಸ್ವಲ್ಪ ಸಹಿಷ್ಣುತೆ ಇದ್ದರೆ, ಅದು ಈ ಆರ್ಯಭೂಮಿಯೊಂದರಲ್ಲಿ ಮಾತ್ರ. ಇಲ್ಲಿ ಮಾತ್ರ ಭಾರತೀಯರು ಮಹಮ್ಮದೀಯರಿಗೆ, ಕ್ರೈಸ್ತರಿಗೆ ದೇವ ಮಂದಿರಗಳನ್ನು ಕಟ್ಟಿಕೊಡುವರು, ಮತ್ತೆಲ್ಲಿಯೂ ಇಲ್ಲ. ನೀವು ಅನ್ಯ ದೇಶಗಳಿಗೆ ಹೋಗಿ; “ಮಹಮ್ಮದೀಯರನ್ನು ಅಥವಾ\break ಅನ್ಯ ಧರ್ಮಾವಲಂಬಿಗಳನ್ನು ನಮಗಾಗಿ ಒಂದು ಗುಡಿಯನ್ನು ಕಟ್ಟಿಸಿಕೊಡಿ” ಎಂದು ಕೇಳಿದರೆ ಅವರು ನಿಮಗೆ ಹೇಗೆ ಸಹಾಯ ಮಾಡುತ್ತಾರೆ ನೋಡಿ. ನಿಮ್ಮ ಗುಡಿಯನ್ನು ಮತ್ತು ಸಾಧ್ಯವಾದರೆ ನಿಮ್ಮನ್ನೂ ಧ್ವಂಸಮಾಡಲು ಯತ್ನಿಸುವರು. ಜಗತ್ತಿಗೆ ಇಂದು ಅತ್ಯಾವಶ್ಯಕವಾಗಿ ಬೇಕಾಗಿರುವ ಮಹಾನ್ ಭಾವನೆ, ಭರತಖಂಡದಿಂದ ಜಗತ್ತು ಕಲಿಯಬೇಕಾಗಿರುವ ನೀತಿ, ಸಹಿಷ್ಣುತೆ ಮಾತ್ರವಲ್ಲ, ಸಹಾನುಭೂತಿ. ಶಿವಮಹಿಮ್ನ ಸ್ತೋತ್ರದಲ್ಲಿ ಈ ಭಾವನೆ ಸೊಗಸಾಗಿ ವ್ಯಕ್ತವಾಗಿದೆ:

\begin{verse}
\textbf{ತ್ರಯೀ ಸಾಂಖ್ಯಂ ಯೋಗಃ ಪಶುಪತಿಮತಂ ವೈಷ್ಣವಮಿತಿ}\\\textbf{ಪ್ರಭಿನ್ನೇ ಪ್ರಸ್ಥಾನೇ ಪರಮಿದಮದಃ ಪಥ್ಯಮಿತಿ ಚ~।}\\\textbf{ರುಚೀನಾಂ ವೈಚಿತ್ರ್ಯಾದೃಜುಕುಟಿಲ ನಾನಾ ಪಥಜುಷಾಂ}\\\textbf{ನೃಣಾಮೇಕೋ ಗಮ್ಯಸ್ತ್ವಮಸಿ ಪಯಸಾಮರ್ಣವ ಇವ~॥}
\end{verse}

ಹಲವು ನದಿಗಳು ಬೇರೆ ಬೇರೆ ಬೆಟ್ಟಗಳಲ್ಲಿ ಹುಟ್ಟಿ, ನೇರವಾಗಿಯೋ ವಕ್ರವಾಗಿಯೋ ಹರಿದು, ಕೊನೆಗೆ ಸಾಗರವನ್ನು ಸೇರುವಂತೆ, ಹೇ ಶಿವನೆ, ಜನರು ತಮ್ಮ ತಮ್ಮ ಸಂಸ್ಕಾರಗಳಿಗೆ ತಕ್ಕಂತೆ ಹಿಡಿಯುವ ಬೇರೆ ಬೇರೆ ಪಥಗಳೆಲ್ಲಾ ಕೊನೆಗೆ ನಿನ್ನೆಡೆಗೆ ಒಯ್ಯುವುವು. ದಾರಿಗಳು ಬೇರೆ ಬೇರೆ ಆದರೂ, ಅವೆಲ್ಲಾ ಒಂದೇ ಗುರಿಯೆಡೆಗೆ ಹೋಗುತ್ತಿರುವುವು. ಕೆಲವು ವಕ್ರವಾಗಿ ಹರಿಯಬಹುದು. ಆದರೆ ಕೊನೆಗೆ ಎಲ್ಲಾ ಈಶ್ವರನೆಡೆಗೆ ಬರಲೇಬೇಕು. ಶಿವನನ್ನು ಲಿಂಗದಲ್ಲಿ ಮಾತ್ರ ನೋಡದೆ ಎಲ್ಲೆಡೆಯಲ್ಲಿಯೂ ಅವನನ್ನು ನೋಡಿದಾಗ ಮಾತ್ರವೇ ನಿಮ್ಮ ಶಿವಭಕ್ತಿ ಪೂರ್ಣವಾಯಿತು. ಎಲ್ಲೆಲ್ಲಿಯೂ ಯಾರು ಹರಿಯನ್ನು ನೋಡಬಲ್ಲರೋ ಅವನೇ ಸಾಧು, ಅವನೇ ಹರಿಭಕ್ತ. ನೀವು ನಿಜವಾದ ಶಿವಭಕ್ತರಾದರೆ ಎಲ್ಲರಲ್ಲಿಯೂ, ಎಲ್ಲದರಲ್ಲಿಯೂ ಅವನನ್ನು ನೋಡಬೇಕು. ಯಾವ ನಾಮರೂಪಗಳಿಂದ ಅವನನ್ನು ಕರೆಯಲಿ, ಪೂಜೆಯೆಲ್ಲಾ ಅವನಿಗೆ ಸಲ್ಲುವಂತೆ ಮಾಡಬೇಕು. ಯಾರು ಪ್ರಾರ್ಥನಾ ಸಮಯದಲ್ಲಿ ಕಾಬಾ ಕಡೆಗೆ ತಿರುಗಿರುವರೋ, ಕ್ರೈಸ್ತರ ಚರ್ಚಿನಲ್ಲಿ ಮಂಡಿಯೂರಿ ನಿಂತಿರುವರೊ, ಬೌದ್ಧವಿಹಾರದಲ್ಲಿ ಧ್ಯಾನದಲ್ಲಿರುವರೋ ಅವರೆಲ್ಲರೂ, ತಮಗೆ ತಿಳಿಯಲಿ ಅಥವಾ ತಿಳಿಯದೆ ಇರಲಿ ಪರಮೇಶ್ವರನ ಅಡಿದಾವರೆಯಲ್ಲಿ ತಮ್ಮ ಭಕ್ತಿ ಕಾಣಿಕೆಗಳನ್ನು ಅರ್ಪಿಸುತ್ತಿರುವರು. ಅವನು ಎಲ್ಲರ ಪ್ರಭು, ಎಲ್ಲ ಆತ್ಮಗಳ ಆತ್ಮ. ಈ ಪ್ರಪಂಚಕ್ಕೆ ಏನು ಬೇಕೋ ಅದು ನನಗಿಂತ ಮತ್ತು ನಿಮಗಿಂತ, ಅವನಿಗೆ ಚೆನ್ನಾಗಿ ಗೊತ್ತು. ಈ ವೈವಿಧ್ಯ ಮಾಯವಾಗುವುದಿಲ್ಲ, ಅದು ಇರಲೇಬೇಕು. ವೈವಿಧ್ಯ ಇಲ್ಲದೆ ಇದ್ದರೆ ಸೃಷ್ಟಿ ನಾಶವಾದಂತೆ. ಈ ಆಲೋಚನೆಯ ಘರ್ಷಣದಿಂದಲೇ, ಭಿನ್ನತೆಯಿಂದಲೇ ಬೆಳಕು ಉತ್ಪನ್ನವಾಗುವುದು, ಚಲನೆ ಪ್ರಾರಂಭವಾಗುವುದು; ಎಲ್ಲಾ ಸಾಗುವುದು ಇದರಿಂದ. ಅನಂತವಾಗಿ ಒಂದಕ್ಕೊಂದು ವಿರುದ್ಧವಾಗಿ ಕಾಣುವ ವೈವಿಧ್ಯವೆಲ್ಲಾ ಇರಬೇಕು. ಆದರೆ ಇದರಿಂದ ನಾವು ಇನ್ನೊಬ್ಬರನ್ನು ದ್ವೇಷಿಸಬೇಕೆಂದಿಲ್ಲ. ನಾವು ಮತ್ತೊಬ್ಬರೊಂದಿಗೆ ಕಾದಾಡಬೇಕಾದ ಆವಶ್ಯಕತೆ ಇಲ್ಲ. ನಮ್ಮ ಮಾತೃಭೂಮಿಯಲ್ಲಿ ಮಾತ್ರ ಸಾರಿದ ಈ ಮೂಲತತ್ತ್ವವನ್ನು ನಾವು ಪುನಃ ಕಲಿಯಬೇಕು. ಇದನ್ನು ಪುನಃ ಮತ್ತೊಬ್ಬರಿಗೆ ಬೋಧಿಸಬೇಕು. ಏಕೆ? ಇದು ನಮ್ಮ ಶಾಸ್ತ್ರದಲ್ಲಿ ಮಾತ್ರ ಇರುವುದಲ್ಲ; ಇದು ನಮ್ಮ ಸಾಹಿತ್ಯದ ಪ್ರತಿಯೊಂದು ಕ್ಷೇತ್ರದಲ್ಲಿಯೂ ಇದೆ. ಇದು ನಮ್ಮ ಜನಜೀವನದಲ್ಲಿ ಹಾಸುಹೊಕ್ಕಾಗಿದೆ. ಇಲ್ಲಿ, ನಮ್ಮ ದೇಶದಲ್ಲಿ ಮಾತ್ರ ಇದನ್ನು ಅನುದಿನವೂ ಅನುಷ್ಠಾನ ಮಾಡುತ್ತಿರುವರು. ಯಾರು ಕಣ್ಣು ಬಿಟ್ಟು ನೋಡುತ್ತಿರುವರೋ ಅವರಿಗೆ ಇದು ಇಲ್ಲಿ ಮಾತ್ರ ಆಚರಣೆಯಲ್ಲಿರುವುದು ಕಾಣುವುದು. ನಾವು ಧರ್ಮವನ್ನು ಹೀಗೆ ಬೋಧಿಸಬೇಕು. ಭಾರತವು ಜಗತ್ತಿಗೆ ಬೋಧಿಸಬೇಕಾದ, ಇದಕ್ಕೂ ಮಿಗಿಲಾದ ಇನ್ನೂ ಬೇರೆ ವಿಷಯಗಳಿವೆ. ಆದರೆ ಅವು ಕೇವಲ ಪಂಡಿತರಿಗೆ ಮಾತ್ರ. ಶಾಂತಭಾವ, ತಿತಿಕ್ಷೆ, ಅನ್ಯಮತ ಸಹಿಷ್ಣುತೆ, ದಯೆ, ಸಹೋದರತ್ವ ಮುಂತಾದುವನ್ನು ಗಂಡಸಾಗಲಿ, ಹೆಂಗಸಾಗಲಿ, ಪಾಮರನಾಗಲಿ, ಯಾವ ಜಾತಿ ಕುಲಗೋತ್ರಕ್ಕೇ ಸೇರಿರಲಿ, ಎಲ್ಲರೂ ಅನುಷ್ಠಾನಕ್ಕೆ ತರಬೇಕು.

\begin{center}
\textbf{“ಏಕಂ ಸತ್ ವಿಪ್ರಾ ಬಹುಧಾ ವದಂತಿ”}
\end{center}

