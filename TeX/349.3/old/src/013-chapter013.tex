
\chapter{ಭಾರತದ ಮಹಾಪುರುಷರು}

\begin{center}
(ಮದ್ರಾಸಿನ ಉಪನ್ಯಾಸ)
\end{center}

ಭಾರತದ ಮಹಾಪುರುಷರ ವಿಷಯವನ್ನು ಕುರಿತು ಮಾತನಾಡುವಾಗ ನನ್ನ ಮನಸ್ಸು ಇತಿಹಾಸಕ್ಕೆ ನಿಲುಕದ ಗತಕಾಲದ ಅಂಧಕಾರದಿಂದ ರಹಸ್ಯಗಳನ್ನು ಬೆಳಕಿಗೆ ತರಲು\break ಸಂಪ್ರದಾಯಕ್ಕೂ ಸಾಧ್ಯವಾಗದ ಅತ್ಯಂತ ಪುರಾತನ ಕಾಲಕ್ಕೆ ಹೋಗುವುದು. ಭಾರತದಲ್ಲಿ ಮಹಾಪುರುಷರ ಸಂಖ್ಯೆಗೆ ಮಿತಿ ಇಲ್ಲ. ಮಹಾಪುರುಷರಿಗೆ ಜನ್ಮವನ್ನು ನೀಡುವುದಲ್ಲದೆ, ಹಿಂದೂ ರಾಷ್ಟ್ರವು ಸಾವಿರಾರು ವರುಷಗಳಿಂದ ಮತ್ತೇನನ್ನು ಮಾಡುತ್ತಿದೆ? ಅವರಲ್ಲಿ ಕೆಲವು ಸರ್ವಶ್ರೇಷ್ಠ ಯುಗಪುರುಷರ ಚರಿತ್ರೆಯನ್ನೂ, ನಾನು ಅವರನ್ನು ತಿಳಿದುಕೊಂಡಿರುವ ರೀತಿಯನ್ನೂ ನಿಮ್ಮ ಅವಗಾಹನೆಗೆ ತರುವೆನು.

ಮೊದಲು ನಾವು ನಮ್ಮ ಶಾಸ್ತ್ರವನ್ನು ಸ್ವಲ್ಪ ತಿಳಿದುಕೊಳ್ಳಬೇಕು. ನಮ್ಮ ಶಾಸ್ತ್ರದಲ್ಲಿ ನಿತ್ಯದ ಎರಡು ಆದರ್ಶಗಳಿವೆ. ಒಂದು ಸನಾತನವಾದುದು, ಮತ್ತೊಂದು ಅಷ್ಟು ಪ್ರಮಾಣಬದ್ಧವಲ್ಲದೇ ಇದ್ದರೂ ಆಯಾಯ ದೇಶಕಾಲ ಪಾತ್ರಗಳಿಗೆ ಅನ್ವಯಿಸುವಂತಹದು. ಶ್ರುತಿಯಲ್ಲಿ ಅಥವಾ ವೇದದಲ್ಲಿ ಜೀವಾತ್ಮ – ಪರಮಾತ್ಮರ ಸ್ವಭಾವ ಮತ್ತು ಅವರ ಪರಸ್ಪರ ಸಂಬಂಧವನ್ನು ವಿವರಿಸಿರುವರು. ಮನು ಮುಂತಾದವರ ಸ್ಮೃತಿಗಳಲ್ಲಿಯೂ, ಯಾಜ್ಞವಲ್ಕ್ಯಾದಿ ಸಂಹಿತೆಗಳಲ್ಲಿಯೂ, ಪುರಾಣಗಳಲ್ಲಿ ಮತ್ತು ತಂತ್ರಶಾಸ್ತ್ರದಲ್ಲಿಯೂ, ಈ ಎರಡನೆಯ ಬಗೆಯ ಸತ್ಯವಿದೆ. ಎರಡನೆಯ ದರ್ಜೆಯ ಶಾಸ್ತ್ರ ಮತ್ತು ಬೋಧನೆ ಶ್ರುತಿಗೆ ಅಧೀನ. ಎಲ್ಲಿ ಶ್ರುತಿಗೂ ಸ್ಮೃತಿಗೂ ಭಿನ್ನಾಭಿಪ್ರಾಯ ಬರುವುದೋ, ಅಲ್ಲಿ ಶ್ರುತಿಗೆ ಪ್ರಾಧಾನ್ಯ. ಇದೇ ನಿಯಮ. ಶ್ರುತಿಯಲ್ಲಿ ಜೀವಾತ್ಮನ ನಿಯತಿ, ಚರಮ ಲಕ್ಷ್ಯ ಮೊದಲಾದ ವಿಷಯಗಳ ಮುಖ್ಯ ಸಿದ್ಧಾಂತದ ವರ್ಣನೆ ಇದೆ. ಸ್ಮೃತಿಯಲ್ಲಿ ಮತ್ತು ಪುರಾಣಗಳಲ್ಲಿ ಅದರ ವಿವರಣಾ\-ತ್ಮಕ ವರ್ಣನೆ ಇದೆ. ಧಾರ್ಮಿಕ ಜೀವನಕ್ಕೆ ಶ್ರುತಿಯ ಉಪದೇಶ ಸಾಕು; ಹೆಚ್ಚು ಹೇಳಬೇಕಾಗಿಯೂ ಇಲ್ಲ, ತಿಳಿದುಕೊಳ್ಳಬೇಕಾಗಿಯೂ ಇಲ್ಲ. ಮುಖ್ಯವಾಗಿ ಯಾವುದನ್ನು ತಿಳಿದುಕೊಳ್ಳಬೇಕೋ, ಜೀವಿಯ ಮೋಕ್ಷಕ್ಕೆ ಯಾವುದು ಆವಶ್ಯಕವೋ, ಅದನ್ನು ಶ್ರುತಿಯಲ್ಲೇ ಸಮಗ್ರವಾಗಿ ಹೇಳಿರುವರು, ಅವುಗಳಲ್ಲಿ ವಿವರಣೆ ಮಾತ್ರ ಇರಲಿಲ್ಲ. ಇದನ್ನು ಸ್ಮೃತಿಗಳು ಕಾಲಕಾಲಕ್ಕೆ ಒದಗಿಸಿವೆ.

ಶ್ರುತಿಯ ಇನ್ನೊಂದು ವೈಶಿಷ್ಟ್ಯವೆಂದರೆ ಅವುಗಳಲ್ಲಿ ಅಡಕವಾದ ಸತ್ಯಗಳನ್ನು ಕಂಡುಕೊಂಡ ಹಲವು ಋಷಿಗಳ ವಿಷಯ ಅಲ್ಲಿದೆ. ಅವರಲ್ಲಿ ಹಲವರು ಪುರುಷರು, ಮತ್ತೆ ಕೆಲವರು ಸ್ತ್ರೀಯರೂ ಇರುವರು. ಆ ವ್ಯಕ್ತಿಗಳ ಜೀವನ, ಅವರು ಹುಟ್ಟಿದ ಕಾಲ, ಈ ವಿಷಯಗಳು ತಿಳಿದಿರುವುದು ಬಹಳ ಅಲ್ಪ. ಆದರೆ ಅವರ ಅತಿಶ್ರೇಷ್ಠ ಆಲೋಚನೆಗಳು ಮತ್ತು ಆವಿಷ್ಕಾರಗಳೆಲ್ಲ ವೇದಗಳೆಂಬ ಪವಿತ್ರ ಧಾರ್ಮಿಕ ಸಾಹಿತ್ಯದಲ್ಲಿ ಅಡಕವಾಗಿವೆ. ಆದರೆ ಸ್ಮೃತಿಯಲ್ಲಿ ವ್ಯಕ್ತಿಗಳು ಎದ್ದು ಕಾಣುತ್ತಾರೆ. ಅಲ್ಲಿ ಮಹಾ ಶಕ್ತಿಶಾಲಿಗಳು ಮತ್ತು ಜಗತ್​ ಚಾಲಕ ವ್ಯಕ್ತಿಗಳು ಕಾಣುವರು. ಅವರ ಉನ್ನತ ಉಜ್ವಲ ಜೀವನದೆದುರು ಅವರ ಬೋಧನೆ ಕೂಡ ಕುಬ್ಜವಾಗುವುದು.

ನಮ್ಮ ಧರ್ಮವು ನಿರ್ಗುಣ–ಸದ್ಗುಣ ದೇವರನ್ನು ಬೋಧಿಸುತ್ತದೆ. ಇದು ನಾವು ಗಮನಿಸಬೇಕಾದ ನಮ್ಮ ಧರ್ಮ ಬೋಧನೆಯ ವೈಶಿಷ್ಟ್ಯ. ಅದು ವ್ಯಕ್ತಿತ್ವವನ್ನು ಮೀರಿದ ಅನೇಕ ತತ್ತ್ವಗಳನ್ನು ಬೋಧಿಸುವುದರ ಜೊತೆಗೆ ಅನೇಕ ವ್ಯಕ್ತಿಗಳ ವಿಷಯವನ್ನೂ ತಿಳಿಸುತ್ತದೆ. ವೇದ ಅಥವಾ ಶ್ರುತಿಯೇ ನಮ್ಮ ಧರ್ಮಕ್ಕೆ ಮೂಲ. ಅಲ್ಲಿ ವ್ಯಕ್ತಿತ್ವ ಗೌಣವಾದದ್ದು. ಅವತಾರಗಳು, ದೇವತೆಗಳು, ಮಹಾಪುರುಷರು, ಮುಂತಾದವರೆಲ್ಲಾ ಬರುವುದು ಸ್ಮೃತಿಗಳಲ್ಲಿ ಮತ್ತು ಪುರಾಣಗಳಲ್ಲಿ. ನಾವು ಇನ್ನೊಂದು ವಿಷಯವನ್ನು ಗಮನಿಸಬೇಕಾಗಿದೆ. ನಮ್ಮ ಧರ್ಮವನ್ನು ಬಿಟ್ಟರೆ, ಜಗತ್ತಿನ ಉಳಿದ ಎಲ್ಲಾ ಧರ್ಮಗಳೂ ಯಾವುದಾದರು ಒಂದು ವ್ಯಕ್ತಿಯ ಅಥವಾ ಅನೇಕ ವ್ಯಕ್ತಿಗಳ ಜೀವನದ ಆಧಾರದ ಮೇಲೆ ನಿಂತಿವೆ. ಏಸುಕ್ರಿಸ್ತನ ಜೀವನದ ಮೇಲೆ ಕ್ರೈಸ್ತಧರ್ಮ, ಮಹಮ್ಮದನ ಜೀವನದ ಮೇಲೆ ಮಹಮ್ಮದೀಯ ಧರ್ಮ, ಬುದ್ಧನ ಜೀವನದ ಮೇಲೆ ಬೌದ್ಧಧರ್ಮ, ಜಿನನ ಜೀವನದ ಮೇಲೆ ಜೈನಧರ್ಮ ಇತ್ಯಾದಿಗಳು ನಿಂತಿವೆ. ಈ ವ್ಯಕ್ತಿಗಳ ಚಾರಿತ್ರಿಕತೆಯ ವಿಚಾರವಾಗಿ ಬೇಕಾದಷ್ಟು ಕಲ್ಪನೆಗಳಿರಬಹುದೆಂದು ನಾವು ಸ್ವಾಭಾವಿಕವಾಗಿ ಊಹಿಸಬಹುದು. ಯಾವುದಾದರೂ ಒಂದು ಸಮಯದಲ್ಲಿ ಈ ವ್ಯಕ್ತಿಗಳ ಚಾರಿತ್ರಿಕ ಪ್ರಮಾಣ ದುರ್ಬಲವಾದರೆ, ಆ ಧರ್ಮಗಳ ತಳಹದಿ ದುರ್ಬಲವಾಗಿ ಆ ಧರ್ಮಸೌಧ ಕುಸಿದು ಬಿದ್ದು ಚೂರು ಚೂರಾಗುವುದು. ನಾವು ಈ ದುರಂತದಿಂದ ಪಾರಾಗಿದ್ದೇವೆ. ಕಾರಣ ಏನೆಂದರೆ, ನಮ್ಮ ಧರ್ಮ ವ್ಯಕ್ತಿನಿಷ್ಠವಲ್ಲ, ತತ್ತ್ವನಿಷ್ಠ. ಅದು ಒಬ್ಬ ಋಷಿಯ ವಾಣಿ ಅಥವಾ ಅವತಾರದ ವಾಣಿ ಎಂದು ನಾವು ಅದನ್ನು ಅನುಸರಿಸುವುದಿಲ್ಲ. ವೇದಕ್ಕೆ ಪ್ರಮಾಣ ಕೃಷ್ಣನಲ್ಲ; ಕೃಷ್ಣನಿಗೆ ಪ್ರಮಾಣ ವೇದ. ಕೃಷ್ಣನ ಮಹಾತ್ಮೆ ಇರುವುದು, ಅವನು ಸರ್ವ ಶ್ರೇಷ್ಠ ವೇದಪ್ರಚಾರಕ ಎನ್ನುವುದರಲ್ಲಿ. ಹೀಗೆಯೇ ಉಳಿದ ಅವತಾರಗಳು ಮತ್ತು ಋಷಿಗಳು ಕೂಡ. ನಮ್ಮ ಮೊದಲನೆಯ ನಿಯಮವೆ ಮನುಷ್ಯನ ಪೂರ್ಣತೆಗೆ, ಅವನ ಮುಕ್ತಿ ಸಾಧನೆಗೆ, ಏನು ಅವಶ್ಯಕವೋ ಅದೆಲ್ಲ ವೇದದಲ್ಲಿದೆ ಎಂಬುದು. ಇದಕ್ಕಿಂತ ಹೊಸದಾದದ್ದು ಏನೂ ಇರುವುದು ಸಾಧ್ಯವಿಲ್ಲ. ಸಮಸ್ತ ಜ್ಞಾನದ ಚರಮ ಲಕ್ಷ್ಯವಾದ ಪೂರ್ಣ ಏಕತ್ವವನ್ನು ಮೀರಿಹೋಗಲು ಅಸಾಧ್ಯ. “ತತ್ತ್ವಮಸಿ” ಎಂಬ ಮಾತನ್ನು ಕಂಡುಹಿಡಿದಾಗ ಧಾರ್ಮಿಕ ಜ್ಞಾನ ತನ್ನ ಚರಮಗುರಿಯನ್ನು ಮುಟ್ಟಿತು. ಇದು ಆಗಲೇ ವೇದದಲ್ಲಿದೆ. ವಿಭಿನ್ನ ದೇಶ–ಕಾಲ–ಪಾತ್ರಗಳಿಗೆ ಅನುಸಾರವಾಗಿ ಕಾಲ ಕಾಲಕ್ಕೆ ಧರ್ಮವನ್ನು ಜನರಿಗೆ ಬೋಧಿಸುವುದು ಮಾತ್ರ ಉಳಿದಿದೆ, ಅಷ್ಟೆ. ಪ್ರಾಚೀನ ಸನಾತನ ಮಾರ್ಗದಲ್ಲಿ ಜನರಿಗೆ ಮಾರ್ಗದರ್ಶನ ನೀಡಬೇಕಾಗಿದೆ. ಇದಕ್ಕಾಗಿಯೇ ಕಾಲಕಾಲಕ್ಕೆ ಹಲವು ಮಹಾಪುರುಷರು ಜನ್ಮವೆತ್ತುವರು. ಈ ಅಭಿಪ್ರಾಯವನ್ನು ಗೀತೆಯಲ್ಲಿ ಶ‍್ರೀಕೃಷ್ಣನ ಈ ಕೆಳಗಿನ ಹೇಳಿಕೆಯಷ್ಟು ಉತ್ತಮವಾಗಿ ಮತ್ತೆಲ್ಲಿಯೂ ಸ್ಪಷ್ಟಪಡಿಸಿಲ್ಲ:

\begin{verse}
\textbf{ಯದಾ ಯದಾ ಹಿ ಧರ್ಮಸ್ಯ ಗ್ಲಾನಿರ್ಭವತಿ ಭಾರತ~।}\\\textbf{ಅಭ್ಯುತ್ಥಾನಮಧರ್ಮಸ್ಯ ತದಾತ್ಮಾನಂ ಸೃಜಾಮ್ಯಹಮ್​~॥}
\end{verse}

\vskip   -0.5cm

“ಎಲೈ ಅರ್ಜುನ, ಯಾವಾಗ ಧರ್ಮದ ಅವನತಿಯಾಗುವುದೋ, ಅಧರ್ಮದ ಉನ್ನತಿಯಾಗುವುದೋ ಆಗ ನಾನು ಧರ್ಮದ ಉದ್ಧಾರಕ್ಕಾಗಿ ಅವತರಿಸುತ್ತೇನೆ.”

ಇದರ ಪರಿಣಾಮ ಏನಾಯಿತು? ಸನಾತನ ತತ್ತ್ವ ಸ್ವತಃಸಿದ್ಧವಾದುದು. ಅದು ಯಾವ ಯುಕ್ತಿಯ ಆಸರೆಯ ಮೇಲೂ ನಿಂತಿಲ್ಲ ಅಥವಾ ಎಷ್ಟೇ ಮಹಾಪುರುಷರಾದ ಋಷಿಗಳಾಗಲೀ, ಮಹಿಮಾಸಂಪನ್ನ ಅವತಾರಗಳಾಗಲೀ ಅವರ ಪ್ರಮಾಣದ ಮೇಲೂ ನಿಂತಿಲ್ಲ. ಇದು ಭಾರತ ಧರ್ಮದ ಒಂದು ವೈಶಿಷ್ಟ್ಯ. ಆದಕಾರಣವೇ ವೇದಾಂತಕ್ಕೆ ಮಾತ್ರ ವಿಶ್ವಧರ್ಮವಾಗುವ ಹಕ್ಕಿದೆ ಎಂದು ನಾವು ಹೇಳುವುದು. ಈಗಾಗಲೇ ಜಗತ್ತಿನಲ್ಲಿ ಅದು ಅಸ್ತಿತ್ವದಲ್ಲಿರುವ ವಿಶ್ವಧರ್ಮವಾಗಿದೆ. ಏತಕ್ಕೆಂದರೆ ಇದು ವ್ಯಕ್ತಿಯನ್ನು ಬೋಧಿಸುವುದಿಲ್ಲ, ತತ್ತ್ವವನ್ನು ಬೋಧಿಸುವುದು. ಯಾವುದಾದರೂ ಒಂದು ವ್ಯಕ್ತಿಯ ಮೇಲೆ ನಿಂತ ಧರ್ಮವನ್ನು ಜಗತ್ತಿನಲ್ಲಿರುವವರೆಲ್ಲ ಸಾರ್ವತ್ರಿಕವಾಗಿ ಸ್ವೀಕರಿಸಲಾರರು. ನಮ್ಮ ದೇಶದಲ್ಲೇ ಎಷ್ಟೋ ಪ್ರಖ್ಯಾತ ವ್ಯಕ್ತಿಗಳಿರುವರು. ಒಂದು ಸಣ್ಣ ಊರಿನಲ್ಲೇ ಎಷ್ಟೋ ಜನರನ್ನು ಆದರ್ಶ ವ್ಯಕ್ತಿಗಳನ್ನಾಗಿ ಆರಿಸಿಕೊಳ್ಳುವರು. ಇಡಿಯ ವಿಶ್ವಕ್ಕೆ, ಒಬ್ಬ ಮಹಮ್ಮದ, ಕ್ರಿಸ್ತ ಅಥವಾ ಬುದ್ಧ ಒಂದು ಆದರ್ಶವೆಂದು ಹೇಗೆ ಒಪ್ಪಿಕೊಳ್ಳುವುದು? ಆ ಒಂದು ವ್ಯಕ್ತಿಯ ಪ್ರಮಾಣದ ಮೇಲೆ ಮಾತ್ರ ಎಲ್ಲಾ ನೀತಿ ಧರ್ಮ ಅಧ್ಯಾತ್ಮ ಹೇಗೆ ಇರಬಲ್ಲದು? ವೇದಾಂತ ಧರ್ಮಕ್ಕೆ ಅಂತಹ ಯಾವ ವ್ಯಕ್ತಿಯ ಪ್ರಮಾಣವೂ ಅನಾವಶ್ಯಕ. ಮನುಷ್ಯನಲ್ಲಿರುವ ಶಾಶ್ವತ ಸ್ವರೂಪವೇ ಇದಕ್ಕೆ ಪ್ರಮಾಣ. ಇದರ ನೀತಿ ಮಾನವ ಜಾತಿಯ ಸನಾತನ ಆಧ್ಯಾತ್ಮಿಕ ಏಕತ್ವದ ಮೇಲೆಯೇ ನಿಂತಿದೆ. ಈ ಏಕತ್ವ ಹೊಸದಾಗಿ ಪಡೆಯುವಂತಹುದಲ್ಲ, ಇದಾಗಲೇ ನಮ್ಮದಾಗಿದೆ.

ಆದರೆ ನಮ್ಮ ಋಷಿಗಳು, ಅತ್ಯಂತ ಪ್ರಾಚೀನಕಾಲದಿಂದಲೂ, ಮುಕ್ಕಾಲು ಪಾಲು ಮಾನವರಿಗೆ ವ್ಯಕ್ತಿಯ ಆಸರೆ ಆವಶ್ಯಕವೆಂಬುದನ್ನು ಮನಗಂಡರು. ಒಂದಲ್ಲ ಒಂದು ವಿಧದಲ್ಲಿ ಅವರಿಗೆ ಸಾಕಾರದೇವರು ಬೇಕು. ಸಗುಣ ದೇವರಿಗೆ ವಿರೋಧವಾಗಿ ಬೋಧಿಸಿದ ಬುದ್ಧನು ಕಾಲವಾಗಿ ಐವತ್ತು ವರುಷ ಆಗಲಿಲ್ಲ. ಆಗಲೇ ಅವನ ಶಿಷ್ಯರು ಅವನನ್ನೇ ಸಗುಣ ದೇವರನ್ನಾಗಿ ಮಾಡಿದರು. ಈ ಸಗುಣ ದೇವರು ಆವಶ್ಯಕ. ನೂರಕ್ಕೆ ತೊಂಬತ್ತೊಂಬತ್ತು ಕಾಲ್ಪನಿಕ ದೇವತೆಗಳು ಮಾನವನ ಪೂಜೆಗೆ ಅನರ್ಹವಾಗಿವೆ. ಅವುಗಳ ಬದಲು ನಮ್ಮೊಡನೆಯೇ ಬಾಳಿ ಸಂಚರಿಸುತ್ತಿರುವ ಜೀವಂತ ದೇವರುಗಳು ಕಾಲ ಕಾಲಕ್ಕೆ ನಮಗೆ ಸಿಕ್ಕುವರು. ನಮ್ಮ ಭ್ರಾಂತಿಜನಿತ, ಕಾಲ್ಪನಿಕ ದೇವರಿಗಿಂತ ಇವರು ಹೆಚ್ಚು ಪೂಜಾ ಯೋಗ್ಯರು. ನೀವು ಅಥವಾ ನಾವು ಕಲ್ಪಿಸಿಕೊಳ್ಳುವ ಭಗವಂತನ ಭಾವನೆಗಿಂತ ಶ‍್ರೀಕೃಷ್ಣ ಮೇಲು. ನಾವು, ನೀವು ಮನಸ್ಸಿನಲ್ಲಿ ಭಾವಿಸುವ ಉಚ್ಚ ಆದರ್ಶಕ್ಕಿಂತ ಬುದ್ಧದೇವನು ಉಚ್ಚತರ ಆದರ್ಶವಾಗಿರುವನು; ಸಚೇತನ ಪರಮ ಆದರ್ಶವಾಗಿರುವನು. ಆದಕಾರಣವೇ ಕಾಲ್ಪನಿಕ ದೇವರನ್ನೆಲ್ಲಾ ಪದಚ್ಯುತರನ್ನಾಗಿ ಮಾಡಿ, ಇಂತಹ ಮಹಾಮಹಿಮರು ಅವರ ಸ್ಥಳದಲ್ಲಿ ಪೂಜಾರ್ಹರಾಗುತ್ತಾರೆ.

ಇದು ನಮ್ಮ ಋಷಿಗಳಿಗೆ ತಿಳಿದಿತ್ತು. ಇಂತಹ ಅವತಾರ ವ್ಯಕ್ತಿಗಳನ್ನು, ಮಹಾಪುರುಷರನ್ನು ಪೂಜಿಸುವುದಕ್ಕೆ ಭಾರತೀಯರಿಗೆ ಅವರು ಅವಕಾಶವನ್ನು ಕಲ್ಪಿಸಿಕೊಟ್ಟರು. ಇದು ಮಾತ್ರವಲ್ಲ, ನಮ್ಮ ಸರ್ವಶ್ರೇಷ್ಠ ಅವತಾರ ಹೀಗೆ ಹೇಳುವನು:

\begin{verse}
\textbf{ಯದ್ಯದ್ವಿಭೂತಿಮತ್​ ಸತ್ತ್ವಂ ಶ‍್ರೀಮದೂರ್ಜಿತಮೇವ ವಾ~।}\\\textbf{ತತ್ತದೇವಾವಗಚ್ಛ ತ್ವಂ ಮಮ ತೇಜೋಽಂಶಸಂಭವಮ್​~॥}
\end{verse}

“ಎಲ್ಲಿ ಅದ್ಭುತ ಆಧ್ಯಾತ್ಮಿಕ ಶಕ್ತಿ ಪ್ರಕಾಶಿಸುತ್ತಿದೆಯೋ, ಅಲ್ಲೆಲ್ಲ ನಾನಿರುವೆನು ಎಂದು ಭಾವಿಸು; ಅದು ನನ್ನಿಂದ ಪ್ರಕಾಶಿಸುತ್ತಿದೆ ಎಂದು ತಿಳಿ.” ಆದಕಾರಣವೇ ಹಿಂದೂಗಳಿಗೆ ಪ್ರಪಂಚದ ಎಲ್ಲಾ ಅವತಾರಗಳನ್ನೂ ಪೂಜಿಸುವುದಕ್ಕೆ ಸ್ವಾತಂತ್ರ್ಯವಿದೆ. ಹಿಂದೂಗಳು ಯಾವ ದೇಶಕ್ಕೆ ಸೇರಿದ ಸಾಧುವನ್ನಾದರೂ ಋಷಿಯನ್ನಾದರೂ ಪೂಜಿಸಬಹುದು. ಅನೇಕ ವೇಳೆ ನಾವು ಕ್ರೈಸ್ತರ ಚರ್ಚಿಗೆ ಹೋಗುತ್ತೇವೆ, ಮಹಮ್ಮದೀಯರ ಮಸೀದಿಗೆ ಹೋಗುತ್ತೇವೆ. ಇದು ಒಳ್ಳೆಯದು. ನಾವು ಏತಕ್ಕೆ ಹೀಗೆ ಮಾಡಬಾರದು? ನಮ್ಮದು ವಿಶ್ವಧರ್ಮ. ಇದು ಎಲ್ಲವನ್ನು ಅಳವಡಿಸಿಕೊಳ್ಳುವಷ್ಟು ವಿಶಾಲವಾಗಿದೆ. ಇದುವರೆಗೆ ಪ್ರಪಂಚದಲ್ಲಿರುವ ಧಾರ್ಮಿಕ ಆದರ್ಶಗಳನ್ನೆಲ್ಲಾ ಅದರಲ್ಲಿ ಸೇರಿಸಬಹುದು. ಭವಿಷ್ಯದಲ್ಲಿ ಬರುವ ವಿಭಿನ್ನ ಧರ್ಮಗಳಿಗಾಗಿ ಹೀಗೆಯೇ ಸಾವಧಾನದಿಂದ ಕಾಯುವೆವು. ಅವುಗಳನ್ನು ಹೀಗೆಯೇ ಸ್ವಾಗತಿಸುವೆವು. ವೇದಾಂತವು ತನ್ನ ವಿಶಾಲ ಬಾಹುಗಳಿಂದ ಅವುಗಳನ್ನೆಲ್ಲಾ ಆಲಂಗಿಸುವುದು.

ಮಹಾಪುರುಷರ, ಭಗವದವತಾರಗಳ ವಿಷಯವಾಗಿ ಇದು ನಮ್ಮ ಭಾವನೆ. ಇವರಿಗಿಂತ ಕಡಿಮೆ ದರ್ಜೆಯ ಮಹಾಪುರುಷರೂ ಇರುವರು ವೇದದಲ್ಲಿ ಋಷಿ ಎಂಬ ಪದವನ್ನು ಪದೇ ಪದೇ ಬಳಸುವರು. ಈಗ ಅದು ಸಾಮಾನ್ಯ ಪದವಾಗಿದೆ. ಋಷಿ ಎಂದರೆ ಮಹಾಪ್ರಮಾಣ–ಪುರುಷ ಎಂದು ಅರ್ಥ. ಈ ಭಾವನೆಯನ್ನು ನಾವು ತಿಳಿದುಕೊಳ್ಳಬೇಕು. ಋಷಿ ಎಂದರೆ ಮಂತ್ರದ್ರಷ್ಟಾ. ಧರ್ಮಕ್ಕೆ ಪ್ರಮಾಣ ಯಾವುದು? ಪುರಾತನ ಕಾಲದಲ್ಲಿ ಈ ಪ್ರಶ್ನೆಯನ್ನು ಹಾಕಿದರು. ಇಂದ್ರಿಯಗಳ ಜಗತ್ತಿನಲ್ಲಿ ಅದಕ್ಕೆ ಪ್ರಮಾಣ ದೊರಕುವುದಿಲ್ಲ. ಅದಕ್ಕಾಗಿಯೇ \textbf{ಯತೋ ವಾಚೋ ನಿವರ್ತಂತೇ ಅಪ್ರಾಪ್ಯ ಮನಸಾ ಸಹಾ} – “ವಾಕ್ಕುಗಳು ಮನಸ್ಸಿನ ಸಹಿತ ಯಾವುದನ್ನು ಗ್ರಹಿಸದೆ ಹಿಂದಿರುಗುವುವೋ”; \textbf{ನ ತತ್ರ ಚಕ್ಷುರ್ಗಚ್ಛತಿ ನ ವಾಗ್​ಗಚ್ಛತಿ ನೋ ಮನಃ} – “ಅದನ್ನು ಕಣ್ಣು ನೋಡಲಾರದು, ಮಾತು ಗ್ರಹಿಸಲಾರದು, ಮನಸ್ಸು ಗ್ರಹಿಸಲಾರದು” ಎಂದು ಹೇಳಿರುವುದು. ಇದನ್ನೇ ಯುಗ ಯುಗಾಂತರಗಳಿಂದಲೂ ಬೋಧಿಸಿರುವರು. ಆತ್ಮನ ಅಸ್ತಿತ್ವ, ಈಶ್ವರ ಅಸ್ತಿತ್ವ, ಅನಂತ ಜೀವನ, ಮನುಷ್ಯನ\- ಚರಮ ಲಕ್ಷ್ಯ ಮುಂತಾದುವನ್ನು ಬಾಹ್ಯಪ್ರಕೃತಿ ಬಗೆಹರಿಸಲಾರದು. ನಮ್ಮ ಮನಸ್ಸು ಅನವರತ ಬದಲಾಗುತ್ತಿರುತ್ತದೆ. ಅದು ಯಾವಾಗಲೂ ಚಲಿಸುತ್ತಿದೆ. ಅದಕ್ಕೆ ಮಿತಿ ಇದೆ. ಅದು ಚೂರುಚೂರಾಗಿ ಒಡೆದು ಹೋಗಿದೆ. ಪ್ರಕೃತಿಯ ಅನಂತವಾದ, ಅಪರಿವರ್ತನಶೀಲವಾದ, ಅಖಂಡವಾದ, ಅವಿಭಾಜ್ಯ ಸನಾತನ ತತ್ತ್ವವನ್ನು ಹೇಗೆ ವಿವರಿಸಬಲ್ಲುದು? ಇದು ಅದಕ್ಕೆ ಎಂದಿಗೂ ಸಾಧ್ಯವಿಲ್ಲ. ಮಾನವನು, ಚೈತನ್ಯಹೀನ ಜಡಪದಾರ್ಥದಿಂದ ಈ ಪ್ರಶ್ನೆಗೆ ಉತ್ತರವನ್ನು ಪಡೆಯಲು ಯತ್ನಿಸಿದಾಗಲೆಲ್ಲಾ ಪರಿಣಾಮ ಎಷ್ಟು ಭಯಂಕರವಾಗಿದೆ ಎಂಬುದನ್ನು ಚರಿತ್ರೆ ತೋರುವುದು. ವೇದವು ಸಾರುವ ಸಂದೇಶವೆಲ್ಲ ಹೇಗೆ ಬಂದಿತು? ಅದು ಋಷಿಗಳಿಂದ ಬಂದಿತು. ಈ ಜ್ಞಾನ ಇಂದ್ರಿಯಗಳಲ್ಲಿ ಇಲ್ಲ. ಇಂದ್ರಿಯವೇ ಮನುಷ್ಯನ ಸರ್ವಸ್ವವೆ? ಮನುಷ್ಯನಲ್ಲಿ ಇಂದ್ರಿಯಗಳೇ ಪರಮ ಪ್ರಮಾಣವೆಂದು ಹೇಳುವುದಕ್ಕೆ ಯಾರಿಗೆ ಧೈರ್ಯವಿದೆ? ನಮ್ಮ ಜೀವನದಲ್ಲಿ ಮತ್ತು ಪ್ರತಿಯೊಬ್ಬರ ಜೀವನದಲ್ಲಿಯೂ ಒಂದೊಂದು ಸಮಯ ಬರುವುದು. ನಾವು ಪ್ರೀತಿಸಿದ ಒಬ್ಬರು ಮೃತ್ಯುವಿಗೆ ಬಲಿಯಾದಾಗಲೋ, ನಾವು ಯಾವುದಾದರೂ ದುರಂತಕ್ಕೆ ಸಿಕ್ಕಿದಾಗಲೋ, ನಮಗೆ ಅನಂತ ಆನಂದ ಪ್ರಾಪ್ತ ವಾದಾಗಲೋ, ಅಥವಾ, ಇನ್ನೂ ಅನೇಕ ಸನ್ನಿವೇಶಗಳಲ್ಲಿ ಮನಸ್ಸು ನಿಶ್ಚಲವಾಗುವುದು. ಆ ಸಮಯದಲ್ಲಿ ಅದು ತನ್ನ ನೈಜ ಸ್ವಭಾವವನ್ನು ಅರಿಯುವುದು, ಮಾತು ಮತ್ತು ಮನಸ್ಸಿಗೆ ನಿಲುಕದ ಬ್ರಹ್ಮಾನುಭವ ನಮಗೆ ವೇದ್ಯವಾಗುವುದು. ಇದು ಸಾಧಾರಣ ಜೀವನದಲ್ಲಿ ಆಗುವುದು. ಈ ಅವಸ್ಥೆಯನ್ನು ಅಭ್ಯಾಸದ ಮೂಲಕ ಗಾಢವೂ ಸ್ಥಿರವೂ ಪರಿಪೂರ್ಣವೂ ಆಗುವಂತೆ ಮಾಡಬೇಕು. ಸಹಸ್ರಾರು ವರುಷಗಳ ಹಿಂದೆ ಜನರು, ಆತ್ಮವು ಇಂದ್ರಿಯದಿಂದ ಬದ್ಧವಾಗಿಲ್ಲ. ಪ್ರಜ್ಞೆಯಿಂದಲೂ \enginline{(consciousness)} ಬದ್ಧವಾಗಿಲ್ಲ ಎಂಬುದನ್ನು ಕಂಡುಕೊಂಡರು. ಅನಂತವಾಗಿರುವ ಸರಪಳಿಯಲ್ಲಿ ಈ ಪ್ರಜ್ಞೆಯೆಂಬುದು ಒಂದು ಕೊಂಡಿ ಎಂಬುದನ್ನು ನಾವು ಅರಿಯಬೇಕು. ಅಸ್ತಿತ್ವವೇ ಪ್ರಜ್ಞೆಯಲ್ಲ, ಪ್ರಜ್ಞೆಯು ಅಸ್ತಿತ್ವದ ಒಂದು ಅಂಶ. ಧೀರರು ಸತ್ಯವನ್ನು ಪ್ರಜ್ಞೆಯ ಆಚೆ ಹುಡುಕುವರು. ಪ್ರಜ್ಞೆಯು ಇಂದ್ರಿಯದಿಂದ ಬದ್ಧವಾಗಿದೆ. ಆಧ್ಯಾತ್ಮಿಕ ಸತ್ಯವನ್ನು ಪಡೆಯಬೇಕಾದರೆ ವ್ಯಕ್ತಿಯು ಅದನ್ನು ಮೀರಿ ಹೋಗಬೇಕು. ಈಗಲೂ ಕೂಡ ಇಂದ್ರಿಯವನ್ನು ಮೀರಿ ಹೋಗಬಲ್ಲ ವ್ಯಕ್ತಿಗಳು ಇರುವರು. ಇವರನ್ನೇ ಋಷಿಗಳೆನ್ನುವರು. ಆಧ್ಯಾತ್ಮಿಕ ಸತ್ಯವನ್ನು ಅವರು ಪ್ರತ್ಯಕ್ಷ ಕಂಡವರು.

ನನ್ನ ಮುಂದೆ ಇರುವ ಮೇಜಿಗೆ ಪ್ರತ್ಯಕ್ಷ ಎಷ್ಟುಮಟ್ಟಿಗೆ ಪ್ರಮಾಣವೋ, ಹಾಗೆಯೇ ವೇದಕ್ಕೂ ಪ್ರತ್ಯಕ್ಷವೇ ಪ್ರಮಾಣ. ನಾನು ಮೇಜನ್ನು ಇಂದ್ರಿಯಗಳ ಮೂಲಕ ನೋಡುತ್ತೇನೆ. ಹಾಗೆಯೇ ಅಧ್ಯಾತ್ಮ ಸತ್ಯವನ್ನು ಜೀವಿಯ ಪ್ರಜ್ಞಾತೀತ ಅವಸ್ಥೆಯಲ್ಲಿ ಗ್ರಹಿಸುವೆವು. ಈ ಋಷಿತ್ವವು ಕಾಲ, ದೇಶ, ಲಿಂಗಗಳಿಗಾಗಲಿ, ಯಾವುದೇ ಜನಾಂಗಕ್ಕಾಗಲಿ ಸೀಮಿತವಾಗಿಲ್ಲ. ಈ ಋಷಿತ್ವ ಎಂಬುದು ಋಷಿ ಸಂತಾನರು ಆರ್ಯರು ಅನಾರ್ಯರು ಮ್ಲೇಚ್ಛರು ಇವರೆಲ್ಲರಿಗೂ ಸಮಾನವಾದ ಸಂಪತ್ತು ಎಂದು ವಾತ್ಸ್ಯಾಯನನು ನಿರ್ಭಯವಾಗಿ ಸಾರಿದ್ದಾನೆ.

ಇದೇ ವೇದಗಳ ಋಷಿತ್ವ. ಈ ಭಾರತೀಯ ಧರ್ಮದ ಆದರ್ಶವನ್ನು ನಾವು ಯಾವಾ\-ಗಲೂ ಸ್ಮರಿಸಬೇಕು. ಅನ್ಯದೇಶೀಯರೂ ಕೂಡ ಇದನ್ನು ಕಲಿಯಲಿ, ಸ್ಮರಿಸಲಿ. ಇದರಿಂದ ವಿವಿಧ ಧರ್ಮಗಳ ವಾದ ವಿವಾದ ಕಡಮೆಯಾಗುವುದು. ಧರ್ಮವು ಶಾಸ್ತ್ರದಲ್ಲಿ ಇಲ್ಲ, ಸಿದ್ಧಾಂತದಲ್ಲಿ ಇಲ್ಲ, ಮೂಢನಂಬಿಕೆಯಲ್ಲಿ ಇಲ್ಲ, ಮಾತಿನಲ್ಲಿ ಇಲ್ಲ, ತರ್ಕದಲ್ಲಿಯೂ ಇಲ್ಲ. ಆತ್ಮನಂತೆ ಇರುವುದು, ಅದರಂತೆ ಆಗುವುದೇ ಧರ್ಮ. ನನ್ನ ಸ್ನೇಹಿತರೆ, ನಿಮ್ಮಲ್ಲಿ ಪ್ರತಿಯೊಬ್ಬರೂ ಋಷಿಗಳಾಗಿ ಆಧ್ಯಾತ್ಮಿಕ ಸತ್ಯಗಳನ್ನು ಪ್ರತ್ಯಕ್ಷ ಮಾಡಿಕೊಳ್ಳುವವರೆಗೆ ನಿಮ್ಮ ಪಾಲಿಗೆ ಧಾರ್ಮಿಕ ಜೀವನವು ಇನ್ನೂ ಆರಂಭವಾಗಿಲ್ಲ ಎಂದೇ ಅರ್ಥ. ಪ್ರಜ್ಞಾತೀತ ಸ್ಥಿತಿಯು ನಿಮಗೆ ಲಭ್ಯವಾಗುವವರೆಗೆ ಧರ್ಮವು ಬರಿಯ ಮಾತು, ಬರಿಯ ಸಿದ್ಧತೆ ಮಾತ್ರ. ನೀವು ಎರಡನೆಯ ಮೂರನೆಯ ವ್ಯಕ್ತಿಗಳಿಂದ ಕೇಳಿದ ವಿದ್ಯೆಯನ್ನು ಮಾತನಾಡುತ್ತಿರುವಿರಿ. ಇದು ಬ್ರಾಹ್ಮಣರೊಂದಿಗೆ ಮಾತನಾಡುತ್ತಿದ್ದ ಬುದ್ಧನ ಸುಂದರ ಉಪಮಾನವನ್ನು ಹೋಲುವುದು. ಬ್ರಾಹ್ಮಣರು ಬ್ರಹ್ಮನ ವಿಚಾರವನ್ನು ಚರ್ಚಿಸುತ್ತ ಬಂದರು. ಬುದ್ಧ, “ನೀವು ಬ್ರಹ್ಮನನ್ನು ಕಂಡಿರುವಿರಾ?” ಎಂದು ಕೇಳಿದನು. "ಇಲ್ಲ” ಎಂದ ಬ್ರಾಹ್ಮಣ. “ನಿಮ್ಮ ತಂದೆ ಆತನನ್ನು ನೋಡಿದ್ದನೊ?” ಎಂದು ಬುದ್ಧ ಮತ್ತೆ ಕೇಳಿದ. “ಇಲ್ಲ, ಅವನೂ ನೋಡಿಲ್ಲ” ಎಂದ ಬ್ರಾಹ್ಮಣ. “ನಿಮ್ಮ ತಾತ ನೋಡಿದ್ದನೋ?” ಎಂದು ಬುದ್ಧ ಕೇಳಿದ. “ಇಲ್ಲ ಅವನೂ ನೋಡಿರುವಂತೆ ಕಾಣುವುದಿಲ್ಲ” ಎಂದು ಬ್ರಾಹ್ಮಣ ಉತ್ತರಿಸಿದ. ಆಗ ಬುದ್ಧ “ನನ್ನ ಸ್ನೇಹಿತರೆ, ಯಾರನ್ನು ನೀವು, ನಿಮ್ಮ ತಂದೆ, ನಿಮ್ಮ ತಾತ, ಯಾರೂ ನೋಡಿಲ್ಲವೋ, ಅವನ ವಿಚಾರ ವಾದಮಾಡುತ್ತ ಒಬ್ಬರು ಮತ್ತೊಬ್ಬರನ್ನು ಮೂದಲಿಸುತ್ತಿರುವಿರಲ್ಲ?” ಎಂದನು. ಇಡಿಯ ಜಗತ್ತು ವರ್ತಿಸುತ್ತಿರುವುದೇ ಹೀಗೆ. ವೇದಾಂತದ ಭಾಷೆಯಲ್ಲಿ ಹೀಗಿದೆ:

\begin{longtable}{@{}l@{}}
\textbf{ನಾಯಮಾತ್ಮಾ ಪ್ರವಚನೇನ ಲಭ್ಯೋ} \\
\textbf{ನ ಮೇಧಯಾ ನ ಬಹುನಾ ಶ್ರುತೇನ} \\
\end{longtable}

\hfill —ಕಠೋಪನಿಷತ್​

“ಆತ್ಮವು ಪ್ರವಚನದಿಂದಾಗಲಿ ಸಿದ್ಧಿಸದು, ಯುಕ್ತಿಯಿಂದಾಗಲಿ ದೊರಕದು, ವೇದಾಧ್ಯಯನದಿಂದಲೂ ದೊರಕದು.”

ವೇದಗಳ ಭಾಷೆಯನ್ನು ಅನುಸರಿಸಿ, ಜಗತ್ತಿನ ಎಲ್ಲ ರಾಷ್ಟ್ರಗಳನ್ನು ಉದ್ದೇಶಿಸಿ ಹೀಗೆ ಹೇಳೋಣ: ನಿಮ್ಮ ವಾದ ವಿವಾದಗಳು ವ್ಯರ್ಥ. ನೀವು ಯಾವ ದೇವರನ್ನು ಬೋಧಿಸಬೇಕೆಂದಿರುವಿರೋ, ನಿಮಗೆ ಅವನ ಸಾಕ್ಷಾತ್ಕಾರವಾಗಿದೆಯೇ? ಇಲ್ಲದೇ ಇದ್ದರೆ ನಿಮ್ಮ ಬೋಧನೆ ವ್ಯರ್ಥ. ನೀವು ಹೇಳುವುದು ನಿಮಗೇ ತಿಳಿದಿಲ್ಲ. ನಿಮಗೆ ದೇವರ ಸಾಕ್ಷಾತ್ಕಾರವಾಗಿದ್ದರೆ ನೀವು ಜಗಳ ಕಾಯುತ್ತಿರಲಿಲ್ಲ. ನಿಮ್ಮ ಮುಖವೇ ಅದನ್ನು ಸಾರುತ್ತಿತ್ತು. ಉಪನಿಷತ್ತಿನ ಒಬ್ಬ ಪುರಾತನ ಋಷಿ ಬ್ರಹ್ಮನ ವಿಷಯವನ್ನು ತಿಳಿಯುವುದಕ್ಕೆ ತನ್ನ ಮಗನನ್ನು ಕಳುಹಿಸಿದನು. ಮಗ ಹಿಂತಿರುಗಿ ಬಂದ ಮೇಲೆ ತಂದೆ, “ನೀನೇನು ಕಲಿತೆ” ಎಂದು ಕೇಳಿದನು. ಮಗ “ಎಷ್ಟೋ ವಿಷಯಗಳನ್ನು ಕಲಿತೆ” ಎಂದನು. ತಂದೆ, “ಅದರಿಂದ ಪ್ರಯೋಜನವಿಲ್ಲ, ಪುನಃ ಹಿಂತಿರುಗಿ ಹೋಗು” ಎಂದನು. ಹಿಂತಿರುಗಿ ಬಂದ ಮೇಲೆ ತಂದೆ ಪುನಃ ಅದೇ ಪ್ರಶ್ನೆಯನ್ನು ಕೇಳಿದನು. ಮಗ ಹಿಂದೆ ಹೇಳಿದಂತೆ “ಬಗೆ ಬಗೆಯ ವಿದ್ಯೆಗಳನ್ನು ಕಲಿತಿರುವೆ” ಎಂದನು. ಅವನು ಪುನಃ ಹೋಗಬೇಕಾಯಿತು. ಮತ್ತೊಂದು ವೇಳೆ ಹಿಂತಿರುಗಿ ಬಂದ ಮೇಲೆ ಅವನ ಮುಖದಲ್ಲಿ ಕಾಂತಿ ಕೋರೈಸುತ್ತಿತ್ತು. ತಂದೆ ಎದ್ದು ನಿಂತು, “ಹಾ! ಮಗು, ಬ್ರಹ್ಮನನ್ನು ತಿಳಿದವನಂತೆ ನಿನ್ನ ಮುಖ ಕಳೆಯಿಂದ ಕೂಡಿರುವುದು” ಎಂದನು ನೀವು ದೇವರನ್ನು ಕಂಡರೆ ಮುಖ ಬದಲಾಗುವುದು. ಧ್ವನಿ ಬದಲಾಗುವುದು. ಇಡಿಯ ಆಕಾರವೇ\break ಬದಲಾಗುವುದು. ನೀವು ಮಾನವಕೋಟಿಗೆ ಮಹಾಕಲ್ಯಾಣಪ್ರದರಾಗುವಿರಿ. ಇದೇ ಋಷಿತ್ವ, ನಮ್ಮ ಧರ್ಮದ ಆದರ್ಶ. ಈ ಉಪನ್ಯಾಸ, ತತ್ವ, ಯುಕ್ತಿ, ದ್ವೈತ ಅದ್ವೈತ ವೇದಾಂತ –ಇವೆಲ್ಲವೂ ಗೌಣ, ಸಿದ್ಧತೆ ಮಾತ್ರ. ಆದರೆ ಸಾಕ್ಷಾತ್ಕಾರ ಮುಖ್ಯ. ವೇದ, ವ್ಯಾಕರಣ, ಜ್ಯೋತಿಷ್ಯ ಮುಂತಾದವು ಎಲ್ಲಾ ಗೌಣ. ಯಾವುದರ ಮೂಲಕ ಅಕ್ಷರವು ಸಾಕ್ಷಾತ್ಕಾರವಾಗುವುದೋ ಅದೇ ಪರಮಜ್ಞಾನ. ಯಾರು ಸಾಕ್ಷಾತ್ಕಾರವನ್ನು ಪಡೆದಿರುವರೋ ಅವರೇ ವೇದರ್ಷಿಗಳು. ಋಷಿ ಎಂಬುದು ಒಂದು ಅವಸ್ಥೆಯ ಹೆಸರು. ನಮ್ಮಲ್ಲಿ ಪ್ರತಿಯೊಬ್ಬ ನಿಜವಾದ ಹಿಂದೂವೂ ಒಂದಲ್ಲ ಒಂದು ದಿನ ಹಾಗೆ ಆಗಬೇಕು. ಹಾಗೆ ಆಗುವುದನ್ನೇ ಮುಕ್ತಿ ಎನ್ನುವುದು. ಸಿದ್ಧಾಂತಗಳಲ್ಲಿ ನಂಬಿಕೆಯಾಗಲಿ, ಸಹಸ್ರಾರು ದೇವಾಲಯಗಳಿಗೆ ಹೋಗುವುದಾಗಲಿ ಪ್ರಪಂಚದ ನದಿಗಳಲ್ಲೆಲ್ಲಾ ಸ್ನಾನಮಾಡುವುದಾಗಲಿ ಅಲ್ಲ. ಮಂತ್ರದ್ರಷ್ಟನಾಗುವುದರಿಂದ, ಋಷಿಯಾಗುವುದರಿಂದ ಮಾತ್ರ ಮುಕ್ತಿ ಅಥವಾ ಸ್ವಾತಂತ್ರ್ಯ ಲಭಿಸುತ್ತದೆ.

ಇಡೀ ಜಗತ್ತಿನ ಮೇಲೆ ಪ್ರಭಾವವನ್ನು ಬೀರಬಲ್ಲ ಮಹಾಪುರುಷರು ಮತ್ತು ಅವತಾರ ಪುರುಷರು ವೇದಕಾಲದ ಅನಂತರ ಬಂದರು. ಭಾಗವತದ ಪ್ರಕಾರ ಅಂತಹ ವ್ಯಕ್ತಿಗಳು ಅಸಂಖ್ಯಾತ. ಭರತಖಂಡದಲ್ಲಿ ನಾವು ಇಂದು ಹೆಚ್ಚಾಗಿ ಪೂಜಿಸುತ್ತಿರುವ ವ್ಯಕ್ತಿಗಳೇ ರಾಮ ಮತ್ತು ಕೃಷ್ಣ. ಮಹರ್ಷಿ ವಾಲ್ಮೀಕಿಯು ಪ್ರಾಚೀನ ವೀರಯುಗದ ಆದರ್ಶಮೂರ್ತಿಯೂ, ಸಮಗ್ರ ನೀತಿತತ್ತ್ವದ ಸಾಕಾರಮೂರ್ತಿಯೂ ಆದ ಶ‍್ರೀರಾಮಚಂದ್ರನ ಚರಿತ್ರೆಯನ್ನು ಚಿತ್ರಿಸಿರುವನು. ಅವನು ಆದರ್ಶ ತನಯ, ಆದರ್ಶ ಪತಿ, ಆದರ್ಶ ಪಿತ, ಎಲ್ಲಕ್ಕಿಂತ ಹೆಚ್ಚಾಗಿ ಆದರ್ಶ ರಾಜ ಎಂಬುದನ್ನು ತೋರಿಸಿರುವನು. ಮಹಾಕವಿಯು ಯಾವ ಭಾಷೆಯಲ್ಲಿ ರಾಮಚಂದ್ರನ ಚರಿತ್ರೆಯನ್ನು ವರ್ಣಿಸಿರುವನೋ, ಆ ಭಾಷೆಗಿಂತ ಶುದ್ಧವಾದ ಮಧುರವಾದ ಸುಂದರವಾದ ಸರಳವಾದ ಭಾಷೆ ಮತ್ತೊಂದಿಲ್ಲ. ಸೀತೆಯ ವಿಷಯವಾಗಿ ಹೇಳುವುದೇನಿದೆ? ನೀವು ಸಮಸ್ತ ಪ್ರಾಚೀನ ಸಾಹಿತ್ಯವನ್ನೂ, ಅಧ್ಯಯನ ಮಾಡಬಹುದು. ಜಗತ್ತಿನಲ್ಲಿ ಮುಂದೆ ಸೃಷ್ಟಿಯಾಗಲಿರುವ ಸಾಹಿತ್ಯವನ್ನೂ ಅಧ್ಯಯನ ಮಾಡಬಹುದು. ನಾನು ಧೈರ್ಯವಾಗಿ ಹೇಳುತ್ತೇನೆ. ಸೀತೆಗೆ ಸಮಾನವಾದ ಶೀಲವುಳ್ಳ ಮತ್ತೊಂದು ವ್ಯಕ್ತಿ ಸಿಕ್ಕುವುದು ದುರ್ಲಭ. ಸೀತೆಯ ಚಾರಿತ್ರ್ಯ ಅಸಾಧಾರಣ. ಅದು ಎಷ್ಟು ಅಸಾಧಾರಣವೆಂದರೆ, ಅಂತಹ ಒಂದು ಪಾತ್ರವು ಹಿಂದೆ ಸೃಷ್ಟಿಯಾಗಿಲ್ಲ, ಮುಂದೆ ಸೃಷ್ಟಿಯಾಗುವ ಸಂಭವವೂ ಇಲ್ಲ. ಬಹುಶಃ ರಾಮನಂತಹ ಎಷ್ಟೋ ಜನರು ಹಿಂದೆ ಇದ್ದಿರಬಹುದು. ಆದರೆ ಸೀತೆಯಂತಹವರು ಮತ್ತೊಬ್ಬರಿಲ್ಲ. ಸೀತೆಯು ಭಾರತೀಯ ಸ್ತ್ರೀಯರ ಆದರ್ಶ ಮೂರ್ತಿ. ಭಾರತೀಯ ಪರಿಪೂರ್ಣ ನಾರಿಯರ ಆದರ್ಶಗಳೆಲ್ಲವೂ ಸೀತೆಯ ಜೀವನದಿಂದ ಮೂಡಿಬಂದಿದೆ. ಸೀತೆಯು ಸಮಗ್ರ ಆರ್ಯಾವರ್ತದಲ್ಲಿ ಸಹಸ್ರಾರು ವರ್ಷಗಳಿಂದ ಆಬಾಲವೃದ್ಧ ಸ್ತ್ರೀ ಪುರುಷರಿಂದ ಪೂಜಿಸಲ್ಪಡುತ್ತಿರುವಳು. ಮಹಿಮಾಮಯಿ ಸೀತೆಯು ಸ್ವಯಂ ಶುದ್ಧತೆಗಿಂತ ಪರಿಶುದ್ಧಳು. ಸಹಿಷ್ಣುತೆಯ ಪರಮೋಚ್ಚ ಆದರ್ಶ ಅವಳು. ಸೀತೆಯು ಸದಾ ಪೂಜ್ಯಸ್ಥಾನದಲ್ಲಿ ಪ್ರತಿಷ್ಠಿತಳಾಗಿರುವಳು. ಸಹನೆ, ಸಹಿಷ್ಣುತೆಗಳ ಮೂರ್ತ ಸ್ವರೂಪ ಅವಳು. ಅವಳು ಸ್ವಲ್ಪವೂ ಗೊಣಗದೆ ಮಹಾದುಃಖವನ್ನು ಅನುಭವಿಸಿದಳು. ಅವಳು ನಿತ್ಯಸಾಧ್ವಿ. ಸದಾ ಶುದ್ಧ ಸ್ವಭಾವದ ಸೀತೆ, ಆದರ್ಶಪತ್ನಿ. ಸೀತೆ, ಮನುಷ್ಯ ಲೋಕಕ್ಕೆ ಮಾತ್ರವಲ್ಲ, ದೇವಲೋಕಕ್ಕೂ ಆದರ್ಶಮೂರ್ತಿ. ಪುಣ್ಯ ಚರಿತಳಾದ ಸೀತೆಯು ಸದಾ ನಮ್ಮ ಜನಾಂಗದ ದೇವತೆಯಾಗಿಯೇ ಇರುವಳು. ನಮ್ಮಲ್ಲಿ ಪ್ರತಿಯೊಬ್ಬರಿಗೂ ಅವಳ ಚಾರಿತ್ರ್ಯದ ಪರಿಚಯವಿದೆ. ಅದರ ವಿಷಯವಾಗಿ ವಿಶೇಷ ವರ್ಣನೆ ಅನವಶ್ಯಕ. ನಮ್ಮ ಪುರಾಣಗಳೆಲ್ಲಾ ನಷ್ಟವಾಗಬಹುದು. ನಮ್ಮ ವೇದ ಲುಪ್ತವಾಗಿ ಹೋಗಬಹುದು. ಕಾಲ ಕಳೆದಂತೆ ನಮ್ಮ ಸಂಸ್ಕೃತಭಾಷೆ ಮಾಯವಾಗಿ ಹೋಗಬಹುದು. ಆದರೆ ನನ್ನ ಮಾತನ್ನು ಧ್ಯಾನಪೂರ್ವಕ ಆಲಿಸಿ; ಎಲ್ಲಿಯವರೆಗೆ ಭಾರತದಲ್ಲಿ ಕೇವಲ ಗ್ರಾಮ್ಯ ಭಾಷೆಯನ್ನು ಮಾತನಾಡುವ ಐದು ಹಿಂದೂಗಳು ಇರುವರೋ, ಅಲ್ಲಿಯವರೆಗೂ ಸೀತೆಯ ಚರಿತ್ರೆ ಬಳಕೆಯಲ್ಲಿರುವುದು. ಸೀತೆ ನಮ್ಮ ಜನಾಂಗದ ಜೀವನದ ಅಂತರಾಳವನ್ನು ಪ್ರವೇಶಿಸಿರುವಳು, ಹಿಂದೂ ನರನಾರಿಯರ ರಕ್ತದಲ್ಲಿ ಸೀತೆ ವಿರಾಜಮಾನಳಾಗಿರುವಳು. ನಾವೆಲ್ಲಾ ಸೀತೆಯ ಸಂತಾನರು. ಸೀತೆಯ ಆದರ್ಶವನ್ನು ತೊರೆದು ನಮ್ಮ ನಾರಿಯರನ್ನು ಆಧುನಿಕರನ್ನಾಗಿ ಮಾಡುವ ಪ್ರಯತ್ನಗಳೆಲ್ಲಾ ನಿಷ್ಫಲವಾಗುವುದು. ಪ್ರತಿದಿನ ನಾವು ಇದರ ಉದಾಹರಣೆಯನ್ನು ನೋಡುತ್ತಿರುವೆವು. ಭಾರತೀಯ ನಾರಿಯರು ಸೀತೆಯ ಚರಣ ಚಿಹ್ನೆಯನ್ನು ಅನುಸರಿಸಿ ಉನ್ನತಿಗಾಗಿ ಪ್ರಯತ್ನಪಡಬೇಕು. ಇದೇ ಭಾರತೀಯ ನಾರಿಯರ ಉನ್ನತಿಗೆ ಏಕಮಾತ್ರ ಪಥ.

ಇದರ ನಂತರ ಬರುವ ವ್ಯಕ್ತಿಯನ್ನು ನಾನಾ ಭಾವಗಳ ಮೂಲಕ ಭಾರತೀಯರು ಪೂಜಿಸುತ್ತಿರುವರು. ಆಬಾಲವೃದ್ಧ ನರನಾರಿಯರಿಗೆ ಅವನು ಪರಮ ಪ್ರಿಯನಾಗಿರುವನು, ಇಷ್ಟದೇವತೆಯಾಗಿರುವನು. ಅವನೇ ಕೃಷ್ಣ. ಭಾಗವತದ ಕರ್ತೃ ಅವನನ್ನು ಅವತಾರ ಮಾತ್ರವೆಂದು ಹೇಳಿ ತೃಪ್ತನಾಗಲಿಲ್ಲ. \textbf{“ಏತೇ ಚಾಂಶಕಲಾಃ ಪುಂಸಃ ಕೃಷ್ಣಸ್ತು ಭಗವಾನ್​ ಸ್ವಯಂ.”} ಉಳಿದ ಅವತಾರಗಳು ಬರಿಯ ಅಂಶಗಳು, ಆದರೆ ಶ‍್ರೀ ಕೃಷ್ಣನಾದರೋ ಸ್ವಯಂ ಭಗವಂತ ಎಂದು ಹೇಳುತ್ತಾನೆ. ಅವನ ಬಹುಮುಖ ವ್ಯಕ್ತಿತ್ವವನ್ನು ಆಲೋಚಿಸಿದರೆ ಇಂತಹ ಗುಣಗಳನ್ನು ಅವನಲ್ಲಿ ಆರೋಪಿಸುವುದರಲ್ಲಿ ಆಶ್ಚರ್ಯವೇನಿಲ್ಲ ಎನಿಸುತ್ತದೆ. ಅವನು ಏಕಕಾಲದಲ್ಲಿ ಅಪೂರ್ವ ಸಂನ್ಯಾಸಿ, ಮತ್ತು ಅದ್ಭುತ ಗೃಹಸ್ಥ. ಅವನು ಅತ್ಯಂತ ಅದ್ಭುತ ರಜೋಶಕ್ತಿಯನ್ನು ವ್ಯಕ್ತಪಡಿಸುತ್ತಿದ್ದನು. ಅವನಲ್ಲಿ ಅತ್ಯಂತ ಅದ್ಭುತ ತ್ಯಾಗವೂ ಇತ್ತು. ಭಗವದ್ಗೀತೆಯನ್ನು ಓದದೇ ಶ‍್ರೀಕೃಷ್ಣನನ್ನು ಪೂರ್ತಿಯಾಗಿ ಅರ್ಥಮಾಡಿಕೊಳ್ಳುವುದಕ್ಕೆ ಆಗುವುದಿಲ್ಲ. ಅವನು ಗೀತೆಯ ಉಪದೇಶದ ಸಾಕಾರ ಸ್ವರೂಪನಾಗಿರುವನು. ಪ್ರತಿಯೊಬ್ಬ ಅವತಾರ ಪುರುಷನೂ, ಯಾವುದನ್ನು ಪ್ರಚಾರಮಾಡುವುದಕ್ಕೆ ಬಂದರೂ, ಆ ಆದರ್ಶಕ್ಕೆ ವ್ಯಾಖ್ಯಾನ ರೂಪವಾಗಿತ್ತು ಅವರ ಬದುಕು. ಗೀತೋಪದೇಶಕನಾದ ಶ‍್ರೀಕೃಷ್ಣನು ಆ ಉಪದೇಶದ ಸಾಕಾರ ಮೂರ್ತಿಯಾಗಿರುವನು. ಅವನು ಅನಾಸಕ್ತಿಯ ಉಜ್ವಲ ಉದಾಹರಣೆ. ತನ್ನ ಸಿಂಹಾಸನವನ್ನು ಇತರರಿಗೆ ಕೊಡುವುದಕ್ಕೆ ಹಿಂಜರಿಯುವುದೇ ಇಲ್ಲ. ಅವನು ಭರತಖಂಡದ ನಾಯಕ. ಅವನ ಒಂದು ಮಾತಿಗೆ ರಾಜರು ಸಿಂಹಾಸನವನ್ನು ಬಿಟ್ಟು ಕೆಳಗೆ ಬರುತ್ತಿದ್ದರು. ಆದರೂ ಅವನು ಸ್ವಯಂ ರಾಜನಾಗಲು ಇಚ್ಛಿಸುವುದಿಲ್ಲ. ಬಾಲ್ಯಕಾಲದಲ್ಲಿ ಎಂತಹ ಸರಳ ಸ್ವಭಾವದಿಂದ ಗೋಪಿಯರೊಂದಿಗೆ ಕ್ರೀಡೆಯಾಡುತ್ತಿದ್ದನೋ, ಅದೇ ಸರಳ ಸ್ವಭಾವದಿಂದಿರುವನು, ಜೀವನದ ಬೇರೆ ಅವಸ್ಥೆಗಳಲ್ಲಿಯೂ. ಅವನ ಜೀವನದ ಆ ಚಿರಸ್ಮರಣೀಯ ಘಟನೆ ನೆನಪಿಗೆ ಬರುವುದು. ಅದನ್ನು ಗ್ರಹಿಸುವುದು ಬಹಳ ಕಷ್ಟ. ಎಲ್ಲಿಯವರೆಗೂ ನಾವು ಪೂರ್ಣ ಬ್ರಹ್ಮಚಾರಿಗಳಾಗಿ, ಪವಿತ್ರ ಸ್ವಭಾವದವರಾಗಿಲ್ಲವೋ, ಅಲ್ಲಿಯವರೆಗೂ ಅದನ್ನು ತಿಳಿದುಕೊಳ್ಳಲು ಪ್ರಯತ್ನಿಸುವುದು ಉಚಿತವಲ್ಲ. ಅವನ ಪ್ರೇಮದ ಅತ್ಯಂತ ಅದ್ಭುತ ಉದಾಹರಣೆ, ಬೃಂದಾವನದ ಮಧುರ ಲೀಲೆಯ ರೂಪಕ ರೂಪದಲ್ಲಿ ವರ್ಣಿತವಾಗಿದೆ. ಯಾವನು ಪ್ರೇಮರೂಪವಾದ ಮಧುಪಾನದಿಂದ ಉನ್ಮತ್ತನಾಗಿರುವನೋ ಅವನ ವಿನಾ ಮತ್ತಾರೂ ಅದನ್ನು ಗ್ರಹಿಸಲಾರರು. ಗೋಪಿಯರಿಗೆ ವಿರಹದಿಂದ ಉತ್ಪನ್ನವಾದ ಯಾತನೆಯನ್ನು ಯಾರು ಅರಿಯಬಲ್ಲರು? ಪ್ರೇಮದ ಆದರ್ಶರೂಪ ಅದು. ಆ ಪ್ರೇಮ ಮತ್ತಾವುದನ್ನೂ ಆಶಿಸುವುದಿಲ್ಲ. ಆ ಪ್ರೇಮದಲ್ಲಿ ಸ್ವರ್ಗದ ಆಕಾಂಕ್ಷೆಯೂ ಇಲ್ಲ. ಆ ಪ್ರೇಮದಲ್ಲಿ ಇಹಲೋಕದಲ್ಲಿ ಯಾವ ಕಾಮವಾಸನೆಯೂ ಇಲ್ಲ. ನನ್ನ ಸ್ನೇಹಿತರೇ, ಗೋಪೀ ಪ್ರೇಮದ್ವಾರಾ ಸಗುಣ ನಿರ್ಗುಣ ಉಪಾಸನೆಯ ವಿವಾದಕ್ಕೆ ಒಂದು ಪರಿಹಾರವು ದೊರೆತಂತಾಗಿದೆ. ಸಗುಣ ಈಶ್ವರ ಮನುಷ್ಯನ ಉತ್ತಮತಮ ಕಲ್ಪನೆ ಎಂಬುದು ನಮಗೆ ಗೊತ್ತಿದೆ. ಆದರೆ ದಾರ್ಶನಿಕ ದೃಷ್ಟಿಯಿಂದ, ಯಾರಿಂದ ಸಕಲವೂ ವಿಕಾಸವಾಗಿದೆಯೊ ಅಂತಹ ಜಗದ್ವ್ಯಾಪಿ ನಿರ್ಗುಣ ದೇವರಲ್ಲಿ ವಿಶ್ವಾಸವಿಡುವುದು ಯುಕ್ತವಾಗಿದೆ. ಆದರೆ ಗ್ರಹಿಸಲು ಸಾಧ್ಯವಾದುದನ್ನು ನಮ್ಮ ಹೃದಯವೂ ಆಶಿಸುವುದು, ನಮ್ಮ ಕೈಗೆ ನಿಲುಕುವ ವಸ್ತುವಿಗಾಗಿ ಕಾತರಿಸುವುದು. ಯಾರ ಪಾದಗಳಲ್ಲಿ ನಮ್ಮ ಹೃದಯದ ಆವೇಗವನ್ನೆಲ್ಲಾ ಹೊರಗೆಡವಬಹುದೋ ಅಂತಹ ಒಂದು ಆಕಾರಕ್ಕಾಗಿ ತವಕಪಡುವುದು. ಸಗುಣ ಈಶ್ವರ ಮಾನವನ ಉಚ್ಚತಮ ಭಾವನೆ. ಆದರೆ ಯುಕ್ತಿಗೆ ಇದರಿಂದ ತೃಪ್ತಿ ಇಲ್ಲ. ಬ್ರಹ್ಮಸೂತ್ರದಲ್ಲಿ ಚರ್ಚಿಸಲ್ಪಟ್ಟಿರುವ ಹಳೆಯ ವಿಷಯವೇ ಇದು. ದ್ರೌಪದಿಯು ಧರ್ಮರಾಜನನ್ನು ಅರಣ್ಯದಲ್ಲಿ ಕೇಳಿದ ಪ್ರಶ್ನೆಯೇ ಇದು. ಅದೇನೆಂದರೆ ಸಂಪೂರ್ಣ ದಯಾಮಯನಾದ ಸರ್ವಶಕ್ತಿಮಾನ್​ ಸಗುಣ ದೇವರೊಬ್ಬನಿದ್ದರೆ, ಈ ನರಕ ಸದೃಶ ಸಂಸಾರವೇಕೆ ಬಂತು? ಅವನೊಬ್ಬ ಮಹಾ ಪಕ್ಷಪಾತಿ ಎಂಬ ಭಾವನೆ ಬರುವುದು. ಇದಕ್ಕೆ ಉತ್ತರವೇ ಇಲ್ಲ. ನನಗೆ ಹೊಳೆಯುವ ಏಕಮಾತ್ರ ಪರಿಹಾರವು ಗೋಪಿಯರ ಪ್ರೇಮದಲ್ಲಿದೆ. ಕೃಷ್ಣನಿಗೆ ಯಾವ ವಿಶೇಷಣವನ್ನೂ ಆರೋಪಿಸಲು ಅವರು ಇಚ್ಛಿಸಲಿಲ್ಲ. ಶ‍್ರೀಕೃಷ್ಣನು ಸೃಷ್ಟಿಕರ್ತನೆಂದು ತಿಳಿಯಲು ಅವರು ಲೆಕ್ಕಿಸಲಿಲ್ಲ. ಅವನು ಸರ್ವೇಶ್ವರ, ಸರ್ವಶಕ್ತ– ಮುಂತಾದ ಯಾವ ವಿಶೇಷಣವನ್ನೂ ಲೆಕ್ಕಿಸ\-ಲಿಲ್ಲ. ಶ‍್ರೀಕೃಷ್ಣನು ಅನಂತಪ್ರೇಮಸ್ವರೂಪನೆಂಬುದೊಂದೇ ಅವರಿಗೆ ಅರ್ಥವಾಗುತ್ತಿದ್ದುದು. ಇಷ್ಟೇ ಅವರಿಗೆ ಸಾಕಾಗಿತ್ತು. ಬೃಂದಾವನದ ಕೃಷ್ಣ ಎಂಬುದಾಗಿ ಮಾತ್ರ ಗೋಪಿಯರು ಅವನನ್ನು ತಿಳಿದುಕೊಂಡಿದ್ದರು. ಅಸಂಖ್ಯಾತ ಸೇನಾನಾಯಕನಾದ, ರಾಜಾಧಿರಾಜನಾದ ಶ‍್ರೀಕೃಷ್ಣನು, ಗೋಪಿಯರಿಗೆ ಸದಾ ನಿಕಟ ಗೋಪಾಲನಾಗಿಯೇ ಇದ್ದ.

\begin{verse}
\textbf{ನ ಧನಂ ನ ಜನಂ ಸುಂದರೀಂ ಕವಿತಾಂ ವಾ ಜಗದೀಶ ಕಾಮಯೇ~।}\\\textbf{ಮಮ ಜನ್ಮನಿ ಜನ್ಮನೀಶ್ವರೇ ಭವತಾದ್ಭಕ್ತಿರಹೈತುಕೀ ತ್ವಯಿ~॥}
\end{verse}

“ಹೇ ಜಗದೀಶ! ನನಗೆ ದ್ರವ್ಯ ಬೇಕಾಗಿಲ್ಲ. ಜನ, ಕವಿತೆ, ಸುಂದರಿ ಯಾವುದೂ ಬೇಕಾಗಿಲ್ಲ, ಹೇ, ಈಶ್ವರ! ಜನ್ಮ ಜನ್ಮಾಂತರಗಳಲ್ಲಿಯೂ ನನಗೆ ನಿನ್ನ ಮೇಲೆ ಅಹೇತುಕ ಭಕ್ತಿಯನ್ನು ದಯಪಾಲಿಸು.” ಪ್ರೀತಿಗಾಗಿ ಪ್ರೀತಿ, ಕರ್ಮಕ್ಕಾಗಿ ಕರ್ಮ, ಕರ್ತವ್ಯಕ್ಕಾಗಿಯೇ ಕರ್ತವ್ಯ ಎಂಬ ಈ ಆದರ್ಶವು ಧರ್ಮದ ಇತಿಹಾಸದಲ್ಲಿ ಒಂದು ನವ ಅಧ್ಯಾಯವಾಗಿದೆ. ಸೃಷ್ಟಿಯಲ್ಲಿ ಸರ್ವಶ್ರೇಷ್ಠ ಅವತಾರನಾದ ಶ‍್ರೀಕೃಷ್ಣನ ಮೂಲಕ ಪ್ರಥಮ ಬಾರಿ ಭಾರತದಲ್ಲಿ ಈ ತತ್ತ್ವ ಪ್ರಚಾರವಾಯಿತು. ಭಯದ ಧರ್ಮ, ಕಾಮನೆಯ ಧರ್ಮ ಮಾಯವಾಯಿತು. ಮನುಷ್ಯನ ಹೃದಯದಲ್ಲಿರುವ ಸ್ವಾಭಾವಿಕ ನರಕಭಯ, ಸ್ವರ್ಗಸುಖ, ಭೋಗದ ಇಚ್ಛೆ ಇವು ನಾಶವಾಗಿ ಈ ಅಹೇತುಕ ಭಕ್ತಿ ಮತ್ತು ನಿಷ್ಕಾಮ ಕರ್ಮ ಎಂಬ ಶ್ರೇಷ್ಠ ಆದರ್ಶಗಳು ಪ್ರಚಲಿತವಾದುವು.

ಆ ಪ್ರೇಮ ಎಂತಹುದು! ಗೋಪೀ ಪ್ರೇಮವನ್ನು ತಿಳಿಯುವುದು ಬಹಳ ಕಷ್ಟವೆಂದು ಈಗ ತಾನೇ ಹೇಳಿದೆ. ಆ ಅದ್ಭುತ ಘಟನೆಯ ಅಮೋಘವಾದ ಪ್ರಾಮುಖ್ಯವನ್ನು ತಿಳಿದುಕೊಳ್ಳಲಾರದ ಮೂಢರಿಗೆ ನಮ್ಮಲ್ಲಿ ಅಭಾವವಿಲ್ಲ. ನಾನು ಪುನಃ ಹೇಳುತ್ತೇನೆ, ನಮ್ಮ ರಕ್ತ ಸಂಬಂಧಿಗಳಲ್ಲೇ ಅನೇಕ ಅಶುದ್ಧಾತ್ಮರಿರುವರು. ಆ ಮೂರ್ಖರು ಗೋಪೀ ಪ್ರೇಮವನ್ನು ಕೇಳಿದೊಡನೆಯೇ, ಅದು ಅತ್ಯಂತ ಅಪವಿತ್ರವೆಂದು, ಭಯದಿಂದ ದೂರ ಸರಿಯುವರು. ನಾನು ಅವರಿಗೆ ಹೇಳುವುದು ಇಷ್ಟೆ: ಮೊದಲು ಶುದ್ಧರಾಗಿ. ಯಾವನು ಗೊಪೀ ಪ್ರೇಮವನ್ನು ವರ್ಣಿಸಿದನೋ, ಆತನು ಶುಕದೇವನಲ್ಲದೆ ಮತ್ತಾರೂ ಅಲ್ಲವೆಂಬುದನ್ನು ಗಮನದಲ್ಲಿಡಿ. ಯಾವನು ಗೋಪಿಯರ ಅದ್ಭುತ ಪ್ರೇಮವನ್ನು ವರ್ಣಿಸಿದನೋ ಅವನು ಆಜನ್ಮ ಪರಿಶುದ್ಧನಾದ ವ್ಯಾಸತನಯ ಶುಕದೇವ. ಎಲ್ಲಿಯವರೆಗೂ ಹೃದಯದಲ್ಲಿ ಸ್ವಾರ್ಥಪರತೆ ಇದೆಯೋ, ಅಲ್ಲಿಯವರೆಗೂ ಭಗವತ್ಪ್ರೇಮ ಅಸಂಭವ. ಅದು ಕೇವಲ ವ್ಯಾಪಾರವಷ್ಟೆ. “ದೇವರೇ! ನಾನು ನಿನಗೆ ಏನನ್ನಾದರೂ ಕೊಡುತ್ತೇನೆ, ನೀನು ನನಗೆ ಏನನ್ನಾದರೂ ಕೊಡು”. ದೇವರು, “ನೀನು ಹೀಗೆ ಮಾಡದೆ ಇದ್ದರೆ ನಿನ್ನನ್ನು ಮರಣಾನಂತರ ವಿಚಾರಿಸಿಕೊಳ್ಳುತ್ತೇನೆ. ಅನಂತರ ನರಕದಲ್ಲಿ ನಿನ್ನನ್ನು ದಹಿಸುತ್ತೇನೆ” ಎನ್ನುವನು. ಎಲ್ಲಿಯವರೆಗೂ ಇಂತಹ ಭಾವನೆ ನಮ್ಮಲ್ಲಿ ಇರುವುದೋ, ಅಲ್ಲಿಯವರೆಗೆ ಗೋಪಿಯರ ಪ್ರೇಮಜನಿತ ವಿರಹದ ಉನ್ಮತ್ತತೆಯನ್ನು ಹೇಗೆ ಅರಿಯಬಲ್ಲೆವು?

\begin{longtable}{@{}l@{}}
\textbf{ಸುರತವರ್ಧನಂ ಶೋಕನಾಶನಂ ಸ್ವರಿತವೇಣುನಾ ಸುಷ್ಠು ಚುಂಬಿತಂ ।} \\
\textbf{ಇತರರಾಗವಿಸ್ಮಾರಣಂ ನೃಣಾಂ ವಿತರ ವೀರ ನಸ್ತೇಽಧರಾಮೃತಮ್​ ॥} \\
\end{longtable}

“ಒಂದು ಸಲ, ಕೇವಲ ಒಂದೇ ಒಂದು ಸಲ, ನಿನ್ನ ಮಧುರ ಚುಂಬನ! ಯಾರು ನಿನ್ನ ಚುಂಬನವನ್ನು ಪಡೆದಿರುವರೋ, ಅವರಿಗೆ ನಿನ್ನ ಮೇಲಿನ ಪ್ರೇಮ ಮತ್ತೂ ಹೆಚ್ಚುವುದು. ದುಃಖವೆಲ್ಲ ನಾಶವಾಗುವುದು. ಇತರ ವಸ್ತುಗಳ ಮೇಲಿನ ಪ್ರೀತಿಯೆಲ್ಲ ಮರೆತುಹೋಗಿ ನಿನ್ನ ಪ್ರೀತಿಯೊಂದೇ ಉಳಿಯುವುದು”.

ಮೊದಲು ಕಾಂಚನ, ಕೀರ್ತಿ, ಯಶಸ್ಸು ಇವುಗಳ ಮೇಲಿನ ಮತ್ತು ಈ ಕ್ಷುದ್ರ ಮಿಥ್ಯಾಸಂಸಾರದ ಮೇಲಿನ ವ್ಯಾಮೋಹವನ್ನು ತ್ಯಜಿಸಿ. ಆಗ ಮಾತ್ರ ನೀವು ಗೋಪೀ ಪ್ರೇಮವನ್ನು ಗ್ರಹಿಸಬಲ್ಲಿರಿ. ಎಲ್ಲಿಯವರೆಗೆ ಮನಸ್ಸು ಅಶುದ್ಧವಾಗಿದೆಯೋ, ಅಲ್ಲಿಯವರೆಗೆ ಅದನ್ನು ಗ್ರಹಿಸಲು ಯತ್ನಿಸಬೇಡಿ. ಎಲ್ಲಿಯವರೆಗೆ ಆತ್ಮವು ಪೂರ್ಣ ಪವಿತ್ರವಾಗಿಲ್ಲವೋ ಅಲ್ಲಿಯವರೆಗೂ ಅದನ್ನು ತಿಳಿಯಲು ಪ್ರಯತ್ನಪಡುವುದೆಲ್ಲ ವಿಫಲ. ಯಾರ ಹೃದಯದಲ್ಲಿ ಪ್ರತಿಕ್ಷಣ ಕಾಮ ಯಶಸ್ಸು ಧನದಾಸೆ ಇವುಗಳ ಬುದ್ಬುದಗಳು ಏಳುತ್ತಿವೆಯೋ, ಅವರು ಗೋಪೀ ಪ್ರೇಮವನ್ನು ತಿಳಿಯುವುದಕ್ಕೆ ಮತ್ತು ಅದನ್ನು ವಿಮರ್ಶಿಸುವ ಸಾಹಸಕ್ಕೆ ಕೈಹಾಕುವುದು ಎಂಥ ವಿಪರ್ಯಾಸ! ಕೃಷ್ಣಾವತಾರದ ಸಾರವೇ ಇದು. ದರ್ಶನಶಾಸ್ತ್ರಶಿರೋಮಣಿಯಂತೆ ಇರುವ ಗೀತೆ ಕೂಡ, ಈ ಪ್ರೇಮೋನ್ಮಾದಕ್ಕೆ ಸರಿದೂಗಲಾರದು. ಇದಕ್ಕೆ ಕಾರಣ, ಗೀತೆಯು ಸಾಧಕನಿಗೆ ಮುಕ್ತಿಯ ಕಡೆಗೆ ಹೇಗೆ ಕ್ರಮಶಃ ಹೋಗಬೇಕೆಂಬುದನ್ನು ಹೇಳುವುದು. ಗೋಪೀ ಪ್ರೇಮದಲ್ಲಿ ಈಶ್ವರ ರಸಾಸ್ವಾದದ ಉನ್ಮತ್ತತೆ ಇದೆ; ಅದ್ಭುತ ಪ್ರೇಮೋನ್ಮಾದವಿದೆ. ಅಲ್ಲಿ ಗುರು, ಶಿಷ್ಯ, ಶಾಸ್ತ್ರ, ಉಪದೇಶ, ಸ್ವರ್ಗ, ಅಂಜಿಕೆ, ಈಶ್ವರಭಾವನೆ ಈ ಯಾವ ಚಿಹ್ನೆಯೂ ಇಲ್ಲ. ಎಲ್ಲಾ ಮಾಯವಾಗಿದೆ. ಪ್ರೇಮೋನ್ಮಾದ ಒಂದೇ ಉಳಿದಿರುವುದು. ಆ ಸಮಯದಲ್ಲಿ ಕೃಷ್ಣನೊಬ್ಬನ ವಿನಾ ಮತ್ತಾವ ವಸ್ತುವಿನ ಸ್ಮರಣೆಯೂ ಇಲ್ಲ. ಆ ಸಮಯದಲ್ಲಿ ಸಮಸ್ತ ಪ್ರಾಣಿಗಳಲ್ಲಿಯೂ ಕೃಷ್ಣನೊಬ್ಬನನ್ನೇ ನೋಡುವರು. ತಮ್ಮ ಮುಖವು ಕೃಷ್ಣನ ಮುಖದಂತೆ ಕಾಣುವುದು. ಮಹಾನುಭಾವ ಶ‍್ರೀಕೃಷ್ಣನ ವರ್ಣನೆ ಇದು.

ಶ‍್ರೀಕೃಷ್ಣನ ಜೀವನದ ಸಣ್ಣಪುಟ್ಟ ವಿಷಯಗಳಲ್ಲಿ ಕಾಲವನ್ನು ವ್ಯರ್ಥಮಾಡಬೇಡಿ. ಅವನ ಜೀವನದ ಮುಖ್ಯ ಅಂಶವನ್ನು ಮಾತ್ರ ಸ್ವೀಕರಿಸಿ. ಕೃಷ್ಣನ ಜೀವನದಲ್ಲಿ ಐತಿಹಾಸಿಕವಲ್ಲದೆ ಹಲವು ಅಂಶಗಳು ಇರಬಹುದು. ಹಲವು ಪ್ರಕ್ಷಿಪ್ತ ಭಾಗಗಳು ಇರಬಹುದು. ಆದರೆ ಈ ಅದ್ಭುತ ನವಪಂಥದ ಸ್ಥಾಪನೆಗೆ ಒಂದು ಆಧಾರ, ಒಂದು ಬುನಾದಿ ಇದ್ದಿರಲೇಬೇಕು. ಇತರ ಮಹಾಪುರುಷರ ಜೀವನವನ್ನು ಕುರಿತು ಆಲೋಚಿಸಿದರೆ, ಅವರು ಏನನ್ನು ಬೋಧಿಸಿದರೋ ಅದು ಕೇವಲ ಅವರ ಹಿಂದಿನವರು ಹೇಳುತ್ತಿದ್ದ ವಿಷಯದ ಮುಂದುವರಿಕೆಯಾಗಿದೆ. ತನ್ನ ಕಾಲದಲ್ಲಿ ಯಾವ ಭಾವನೆಯನ್ನು ಇತರರು ಪ್ರಚಾರಮಾಡುತ್ತಿದ್ದರೋ ಅದೇ ಭಾವನೆಯನ್ನು ಆ ಮಹಾಪುರುಷನೂ ಉಪದೇಶಿಸಿರುವನು ಎಂಬುದು ಕಂಡುಬರುತ್ತದೆ. ಆ ಮಹಾಪುರುಷ ಇದ್ದನೇ ಎಂಬ ವಿಷಯದಲ್ಲಿ ದೊಡ್ಡ ಸಂದೇಹವಿರಬಹುದು. ಆದರೆ ಕೃಷ್ಣನು ಬೋಧಿಸಿದ ಆದರ್ಶಗಳಾದ ಪ್ರೀತಿಗಾಗಿ ಪ್ರೀತಿ, ಕೆಲಸಕ್ಕಾಗಿ ಕೆಲಸ, ಮತ್ತು ಕರ್ತವ್ಯಕ್ಕಾಗಿ ಕರ್ತವ್ಯ – ಇವು ಕೃಷ್ಣನ ಹೊಸ ಸೃಷ್ಟಿಯಲ್ಲ ಎಂಬುದನ್ನು ನಿಮ್ಮಲ್ಲಿ ಯಾರಾದರೂ ತೋರಿಸಲಿ ಎಂದು ನಾನು ಸವಾಲು ಹಾಕುತ್ತೇನೆ. ಅದು ನವ ಆವಿಷ್ಕಾರವಾದರೆ ಅದಕ್ಕೆ ಯಾರಾದರೂ ಕಾರಣಕರ್ತರು ಇರಬೇಕು. ಇತರರಿಂದ ಎರವಲಾಗಿ ತೆಗೆದುಕೊಂಡಿರಲಿಕ್ಕಿಲ್ಲ. ಕೃಷ್ಣನ ಜನನ ಕಾಲದಲ್ಲಿ ಈ ಭಾವನೆಪ್ರಚಲಿತವಾಗಿರಲಿಲ್ಲ. ಆದರೆ ಭಗವಾನ್​ ಶ‍್ರೀಕೃಷ್ಣ ಇದನ್ನು ಪ್ರಥಮ ಬಾರಿ ಜಗತ್ತಿಗೆ ಬೋಧಿಸಿದನು. ಅವನ ಶಿಷ್ಯ ವೇದವ್ಯಾಸ ಅದನ್ನು ಬರೆದು ಜಗತ್ತಿನಲ್ಲಿ ಪ್ರಚಾರ ಪಡಿಸಿದನು. ಮಾನವ ಜೀವನದಲ್ಲಿ ಇಂತಹ ಶ್ರೇಷ್ಠತಮ ಆದರ್ಶ ಬೇರೊಂದಿಲ್ಲ. ‘ಗೋಪೀಜನವಲ್ಲಭ’ ನಾಗಿರುವ ಚಿತ್ರವೇ ಶ‍್ರೀಕೃಷ್ಣನ ಜೀವನದ ಅತ್ಯಂತ ಶ್ರೇಷ್ಠಮುಖ. ಅಂತಹ ಪ್ರೇಮೋನ್ಮಾದ ನಿಮಗೆ ಪ್ರಾಪ್ತವಾದರೆ, ಭಾಗ್ಯವತಿಯರಾದ ಗೋಪಿಯರ ಭಾವವನ್ನು ನೀವು ಗ್ರಹಿಸಿದರೆ, ಆಗ ಪ್ರೇಮವೆಂದರೆ ಏನೆಂಬುದು ನಿಮಗೆ ಗೊತ್ತಾಗುವುದು. ಯಾವಾಗ ಇಡೀ ಜಗತ್ತು ನಿಮ್ಮ ದೃಷ್ಟಿಗೆ ಅಂತರ್ಧಾನವಾಗುವುದೋ, ನಿಮ್ಮ ಹೃದಯವು ಪೂರ್ಣವಾಗಿ ಶುದ್ಧವಾಗುವುದೋ ಬೇರಾವ ಸತ್ಯಾನುಸಂಧಾನದ ಲಕ್ಷ್ಯವೂ ಕೂಡ ಇಲ್ಲದೇ ಹೋಗುವುದೋ, ಆಗ ಮಾತ್ರ ಗೋಪಿಯರ ಪ್ರೇಮೋನ್ಮಾದ ನಿಮಗೆ ಪ್ರಾಪ್ತವಾಗುವುದು. ಆಗ ಮಾತ್ರ ಗೋಪಿಯರ ಪ್ರೇಮಶಕ್ತಿಯು ನಿಮಗೆ ಪ್ರಾಪ್ತವಾಗುವುದು. ಪ್ರೀತಿಗಾಗಿ ಪ್ರೀತಿ ಎಂಬುದು ಪ್ರಾಪ್ತವಾಗುವುದು. ಅದೇ ಗುರಿ. ಈ ಪ್ರೇಮ ನಿಮಗೆ ದೊರೆತರೆ ಸಕಲವೂ ಪ್ರಾಪ್ತವಾದಂತೆ.

ಈಗ ಒಂದು ಮೆಟ್ಟಿಲು ಕೆಳಗೆ ಬಂದು, ಗೀತೋಪದೇಶಕನಾದ ಕೃಷ್ಣನ ಪರಿಚಯ ಮಾಡಿಕೊಳ್ಳೋಣ. ಭಾರತದಲ್ಲಿ ಈಗ ಕುದುರೆಯ ಮುಂದೆ ಗಾಡಿಯನ್ನು ಕಟ್ಟುವ ಒಂದು ಪ್ರಯತ್ನ ನಡೆದಿದೆ. ಗೋಪೀಜನವಲ್ಲಭ ನಮಗೆ ಹಿಡಿಸುವುದಿಲ್ಲ. ಕಾರಣ, ಕೆಲವು ಯುರೋಪಿಯನ್ನರು ಅದನ್ನು ಮೆಚ್ಚುವುದಿಲ್ಲ! ಇಂತಹ ವಿದ್ವಾಂಸ ಅದನ್ನು ಮೆಚ್ಚುವುದಿಲ್ಲ! ಆಗ ನಿಜವಾಗಿ ಗೋಪಿಯರು ದೂರ ಸರಿಯಬೇಕು. ಯುರೋಪಿಯನ್ನರ ಅನುಮತಿ ಇಲ್ಲದೆ ಶ‍್ರೀಕೃಷ್ಣ ಹೇಗೆ ಜೀವಿಸಬಲ್ಲ? ಇಲ್ಲ, ಅವನು ಬದುಕಿರಲಾರ! ಮಹಾಭಾರತದಲ್ಲಿ ಎಲ್ಲೋ ಒಂದೆರಡು ಕಡೆ ಬಿಟ್ಟರೆ ಗೋಪಿಯರ ಪ್ರಸ್ತಾಪವೇ ಇಲ್ಲ. ಆ ಒಂದೆರಡು ಅಷ್ಟೇನೂ ಮುಖ್ಯವಾದ ಸ್ಥಳಗಳಲ್ಲ. ದ್ರೌಪದಿ ಪ್ರಾರ್ಥಿಸುವಾಗ ಮತ್ತು ಶಿಶುಪಾಲ ವಧಾ ಸಮಯದಲ್ಲಿ ಮಾತ್ರ ಬೃಂದಾವನದ ಗೋಪಿಯರ ಪ್ರಸಂಗವಿದೆ. ಇವೆಲ್ಲ ಪ್ರಕ್ಷಿಪ್ತ ಭಾಗಗಳು! ಯೂರೋಪಿಯನ್ನರು ಯಾವುದನ್ನು ಮೆಚ್ಚುವುದಿಲ್ಲವೋ ಅದನ್ನು ನಾವು ಆಚೆಗೆಸೆಯಬೇಕು. ಗೋಪಿಯರ ಮತ್ತು ಕೃಷ್ಣನ ಪ್ರಸಂಗವೆಲ್ಲ ಪ್ರಕ್ಷಿಪ್ತ ಭಾಗ ಎನ್ನುವ ಜನರಲ್ಲಿ ಘೋರ ವ್ಯಾಪಾರ ಮನೋಭಾವ ಇದೆ. ಅಲ್ಲಿ ಧರ್ಮ ಕೂಡ ವ್ಯಾಪಾರವಾಗಿದೆ. ಇಲ್ಲಿ ಏನನ್ನೋ ಮಾಡಿ ಸ್ವರ್ಗಕ್ಕೆ ಹೋಗಲು ಯತ್ನಿಸುತ್ತಿರುವರು. ವ್ಯಾಪಾರಿಗೆ ಚಕ್ರಬಡ್ಡಿ ಬೇಕು. ಇಲ್ಲಿ ಏನನ್ನಾದರೂ ಕೂಡಿಟ್ಟು ಸ್ವರ್ಗದಲ್ಲಿ ಅನುಭವಿಸಬೇಕು. ನಿಜವಾಗಿ ಇಂತಹ ವ್ಯವಹಾರದಲ್ಲಿ ಗೋಪಿಯರಿಗೆ ಸ್ಥಳವಿಲ್ಲ. ಈ ಭಾವನೆಯ ಶಿಖರದಿಂದ, ಗೀತಾ ಪ್ರಚಾರಕ ಕೃಷ್ಣನ ಸಮೀಪಕ್ಕೆ ಇಳಿಯುವೆವು. ವೇದಗಳ ಮೇಲೆ ಬರೆಯಬಹುದಾದ ಅತ್ಯುತ್ತಮ ಭಾಷ್ಯವೆಂದರೆ ಭಗವದ್ಗೀತೆ. ಶ್ರುತಿಸಾರವಾದ ಉಪನಿಷತ್ತನ್ನು ತಿಳಿದುಕೊಳ್ಳುವುದು ಕಷ್ಟ. ಎಷ್ಟೋ ಮಂದಿ ಭಾಷ್ಯಕಾರರು ಇರುವರು. ಒಬ್ಬೊಬ್ಬರು ಒಂದೊಂದು ವಿಧವಾಗಿ ಅದಕ್ಕೆ ವ್ಯಾಖ್ಯಾನ ಮಾಡುವರು. ಕೊನೆಗೆ ಶ್ರುತಿಯ ಸ್ಫೂರ್ತಿದಾಯಕನೇ ಅದರ ಅರ್ಥವನ್ನು ನಮಗೆ ತೋರುವುದಕ್ಕಾಗಿ ಗೀತಾಚಾರ್ಯನಂತೆ ಬರುವನು. ಇಂದು ಭರತಖಂಡಕ್ಕೆ ಅಲ್ಲಿ ವ್ಯಾಖ್ಯಾನಮಾಡುವ ರೀತಿಗಿಂತ ಉತ್ತಮವಾದುದು ಬೇಕಾಗಿಲ್ಲ. ಇಡಿಯ ಜಗತ್ತಿಗೂ ಬೇಕಾಗಿಲ್ಲ. ಅನಂತರ ಬಂದ ಹಲವು ಭಾಷ್ಯಕಾರರು ಗೀತೆಯ ಮೇಲೆ ವ್ಯಾಖ್ಯಾನಮಾಡುವಾಗಲೂ ಅದರ ಅರ್ಥವನ್ನು ಗ್ರಹಿಸದೆ ಇದ್ದುದು ಆಶ್ಚರ್ಯಕರವಾಗಿದೆ. ಗೀತೆಯಲ್ಲಿ ನಿಮಗೆ ಕಾಣುವುದೇನು? ಆಧುನಿಕ ಭಾಷ್ಯಕಾರರಲ್ಲಿ ಕಾಣುವುದೇನು? ಒಬ್ಬ ಅದ್ವೈತಿ ಉಪನಿಷತ್ತಿಗೆ ಭಾಷ್ಯ ಬರೆಯಲು ಯತ್ನಿಸುವನು. ಅಲ್ಲಿರುವ ದ್ವೈತ ಭಾವನೆಯ ಶ್ಲೋಕಗಳನ್ನು ಹೇಗೋ ಹಿಸುಕಿ ಅದರಲ್ಲಿ ತನ್ನ ಅಭಿಪ್ರಾಯ ಬರುವಂತೆ ಮಾಡುವನು. ದ್ವೈತ ಭಾಷ್ಯಕಾರನು ಅಲ್ಲಿರುವ ಎಷ್ಟೋ ಅದ್ವೈತ ಭಾವನೆಯ ಶ್ಲೋಕಗಳಿಗೆ ದ್ವೈತದ ಅರ್ಥ ಬರುವಂತೆ ಮಾಡುವನು. ಆದರೆ ಗೀತೆಯಲ್ಲಿ ಗ್ರಂಥಪೀಡನೆಯ ಪ್ರಯತ್ನವೆ ಇಲ್ಲ. ಭಗವಂತನು ಅವುಗಳೆಲ್ಲ ಸರಿಯಾಗಿದೆ ಎನ್ನುವನು. ಮಾನವಜೀವಿಯು ಕ್ರಮಕ್ರಮವಾಗಿ, ಮೆಟ್ಟಿಲು ಮೆಟ್ಟಿಲಾಗಿ, ಸ್ಥೂಲದಿಂದ ಸೂಕ್ಷ್ಮಕ್ಕೆ, ಸೂಕ್ಷ್ಮದಿಂದ ಸೂಕ್ಷ್ಮತರಕ್ಕೆ, ನಿರ್ಗುಣ ಬ್ರಹ್ಮನ ಗುರಿ ಸೇರುವವರೆಗೆ ವಿಕಾಸವಾಗುತ್ತಿರುವನು. ಗೀತೆಯಲ್ಲಿರುವುದು ಇದು. ಕೃಷ್ಣನು ಕರ್ಮಕಾಂಡವನ್ನು ಕೂಡ ಸ್ವೀಕರಿಸುವನು. ಪ್ರತ್ಯಕ್ಷವಾಗಿ ಅದು ಮುಕ್ತಿಯನ್ನು ಕೊಡದೇ ಇದ್ದರೂ, ಪರೋಕ್ಷವಾಗಿ ಕೊಡುವುದು. ಅದು ಸತ್ಯ. ವಿಗ್ರಹ, ಆಚಾರ, ಇವುಗಳೆಲ್ಲ ಪರೋಕ್ಷವಾಗಿ ಸತ್ಯ. ಚಿತ್ತ ಮಾತ್ರ ಶುದ್ಧವಾಗಿರಬೇಕು. ಹೃದಯ ಶುದ್ಧವಾಗಿದ್ದರೆ, ನಿಷ್ಕಪಟವಾಗಿದ್ದರೆ, ಪೂಜೆ ಸತ್ಯವಾಗುವುದು, ನಮ್ಮನ್ನು ಗುರಿಯ ಎಡೆಗೆ ಒಯ್ಯುವುದು, ವಿಭಿನ್ನ ಉಪಾಸನಾ ರೀತಿಗಳೆಲ್ಲ ಆವಶ್ಯಕ. ಇಲ್ಲದೇ ಇದ್ದರೆ ಅವೆಲ್ಲ ಏತಕ್ಕೆ ಇರಬೇಕಾಗಿತ್ತು? ನಮ್ಮ ಹಲವು ಆಧುನಿಕರು ಯೋಚಿಸುವಂತೆ, ಧರ್ಮ, ಜಾತಿ, ಮುಂತಾದವುಗಳನ್ನೆಲ್ಲ ಕೆಲವು ದುಷ್ಟರು, ಕಪಟಿಗಳು, ಹಣದಾಸೆಯಿಂದ ಸೃಷ್ಟಿಸಲಿಲ್ಲ. ಈ ವಿವರಣೆ ಎಷ್ಟೇ ಯುಕ್ತಿಯುಕ್ತವಾಗಿ ತೋರಿದರೂ ಇದು ಸತ್ಯವಲ್ಲ. ಇವನ್ನು ಕಂಡುಹಿಡಿದುದು ಹೀಗಲ್ಲ. ಜೀವಿಯ ಸ್ವಾಭಾವಿಕ ಆವಶ್ಯಕತೆಗೆ ಅನುಗುಣವಾಗಿ ಅವು ಹುಟ್ಟಿವೆ. ವಿಭಿನ್ನ ಜನರ ತೃಪ್ತಿಗಾಗಿ ಅವು ಇವೆ. ಅವನ್ನು ವಿರೋಧಿಸಬೇಕಾಗಿಲ್ಲ. ಅವುಗಳ ಆವಶ್ಯಕತೆ ಮುಗಿದ ಮೇಲೆ ಅವು ಮಾಯವಾಗುವುವು. ಎಲ್ಲಿಯವರೆಗೂ ಅವುಗಳ ಆವಶ್ಯಕತೆ ಇದೆಯೋ ಅಲ್ಲಿಯವರೆಗೂ ಅವಕ್ಕೆ ವಿರುದ್ಧವಾಗಿ ನೀವು ಎಷ್ಟೇಪ್ರಚಾರ ಮಾಡಿದರೂ, ಟೀಕಿಸಿದರೂ, ಅವು ಇದ್ದೇ ತೀರಬೇಕು. ಅವಕ್ಕೆ ವಿರುದ್ಧವಾಗಿ ಕೋವಿ ಕತ್ತಿಗಳನ್ನು ತರಬಹುದು. ಜಗತ್ತಿನಲ್ಲಿ ರಕ್ತವನ್ನು ಹರಿಸಬಹುದು. ಎಲ್ಲಿಯವರೆಗೆ ವಿಗ್ರಹದ ಆವಶ್ಯಕತೆ ಇದೆಯೋ ಅಲ್ಲಿಯವರೆಗೂ ಅದು ಇರಲೇಬೇಕು. ಮೂರ್ತಿಪೂಜೆ, ಧರ್ಮದ ಹಲವು ಉಪಾಸನೆಗಳು, ಇವು ಇರಲೇಬೇಕು. ಭಗವಾನ್​ ಶ‍್ರೀಕೃಷ್ಣನಿಂದ ಅವುಗಳ ಆವಶ್ಯಕತೆಗೆ ಕಾರಣವನ್ನು ತಿಳಿಯುತ್ತೇವೆ.

ಶ‍್ರೀಕೃಷ್ಣನ ಅನಂತರ ಭಾರತ ಇತಿಹಾಸದ ದುಃಖಮಯ ಅಧ್ಯಾಯ ಪ್ರಾರಂಭವಾಗುತ್ತದೆ. ಗೀತೆಯಲ್ಲಿ ಭಿನ್ನ ಸಂಪ್ರದಾಯಗಳ ವಿರೋಧದ ಕೋಲಾಹಲದ ದೂರ ಧ್ವನಿ ನಮಗೆ ಕೇಳಿಸುವುದು. ಮಹಾಸಮನ್ವಯಾಚಾರ್ಯನಾದ ಶ‍್ರೀಕೃಷ್ಣನು ಅವುಗಳನ್ನೆಲ್ಲ ಸಮನ್ವಯಗೊಳಿಸಲು ಮಧ್ಯೆ ಬರುವನು. \textbf{“ಮಯಿ ಸರ್ವಮಿದಂ ಪ್ರೋತಂ ಸೂತ್ರೇ ಮಣಿಗಣಾ ಇವ”} – ನನ್ನಲ್ಲಿ ಈ ಭಿನ್ನತೆಗಳೆಲ್ಲ ದಾರದಲ್ಲಿ ಮಣಿಗಳು ಹೆಣೆಯಲ್ಪಟ್ಟಂತೆ ಇವೆ ಎನ್ನುವನು. ವಿಭಿನ್ನ ಅಭಿಪ್ರಾಯಗಳ ತುಮುಲವನ್ನು ನಾವು ಆಗಲೇ ಆಲಿಸಿರುವೆವು. ಅನಂತರ ಕೆಲವುಕಾಲ ಸೌಹಾರ್ದ ಶಾಂತಿ ಇದ್ದುವೆಂದು ತೋರುವುದು. ಬಳಿಕ ವಿರೋಧ ಪುನಃ ಉತ್ಪನ್ನವಾಯಿತು. ಕೇವಲ ಧರ್ಮಕ್ಷೇತ್ರದಲ್ಲಿ ಮಾತ್ರವಲ್ಲ, ಬಹುಶಃ ವರ್ಣದ ಕ್ಷೇತ್ರದಲ್ಲಿಯೂ ಇದು ಪ್ರಾರಂಭವಾಯಿತು. ನಮ್ಮ ಸಮಾಜದ ಎರಡು ಪ್ರಬಲ ಅಂಗಗಳಾದ ಬ್ರಾಹ್ಮಣರಲ್ಲಿ ಮತ್ತು ಕ್ಷತ್ರಿಯರಲ್ಲಿ ಇದು ಪ್ರಾರಂಭವಾಯಿತು. ಒಂದು ಸಾವಿರ ವರ್ಷಗಳವರೆಗೆ ಈ ಸಂಘರ್ಷಗಳ ತರಂಗ ನಮ್ಮ ಭರತಖಂಡವನ್ನು ಮುಳುಗಿಸಿತು. ಅದರ ಸರ್ವೋಚ್ಚ ಶಿಖರದಲ್ಲಿ ಒಬ್ಬ ಮಹಾಮಹಿಮ ಮೂರ್ತಿ ಗೋಚರಿಸಿದನು. ಅವನೇ ನಮ್ಮ ಗೌತಮ ಶಾಕ್ಯಮುನಿ. ಅವನ ಉಪದೇಶ ಮತ್ತು ಪ್ರಚಾರಗಳು ನಿಮ್ಮಲ್ಲಿ ಈಗಾಗಲೇ ಎಷ್ಟೋ ಜನರಿಗೆ ತಿಳಿದಿದೆ. ಅವನನ್ನು ಈಶ್ವರಾವತಾರವೆಂದು ಭಾವಿಸಿ ಪೂಜಿಸುವೆವು. ನೀತಿತತ್ತ್ವವನ್ನು ಇಷ್ಟು ನಿರ್ಭೀತಿಯಿಂದ ಪ್ರಚಾರ ಮಾಡಿದವನು ಪ್ರಪಂಚದಲ್ಲಿ ಮತ್ತಾರು ಇಲ್ಲ. ಅವನು ಕರ್ಮಯೋಗಿಗಳಲ್ಲಿ ಸರ್ವ ಶ್ರೇಷ್ಠ. ಸ್ವಯಂ ಕೃಷ್ಣನೇ ಶಿಷ್ಯರೂಪದಲ್ಲಿ ತನ್ನ ಉಪದೇಶವನ್ನು ಅನುಷ್ಠಾನಕ್ಕೆ ತರುವುದಕ್ಕಾಗಿ ಜನಿಸಿದಂತೆ ತೋರುವುದು. ಗೀತೆಯಲ್ಲಿ ಉಪದೇಶಿಸಿದ ಅದೇ ಧ್ವನಿಯೇ ಮತ್ತೆ ಕೇಳಿಸಿತು:

\begin{verse}
\textbf{“ಸ್ವಲ್ಪಮಪ್ಯಸ್ಯ ಧರ್ಮಸ್ಯ ತ್ರಾಯತೇ ಮಹತೋ ಭಯಾತ್​”}
\end{verse}

“ಸ್ವಲ್ಪ ಧರ್ಮದ ಅನುಷ್ಠಾನದಿಂದಲೂ ಮಹಾಭಯದಿಂದ ಪಾರಾಗುವಿರಿ”

\begin{verse}
\textbf{“ಸ್ತ್ರೀಯೋ ವೈಶ್ಯಾಸ್ತಥಾ ಶೂದ್ರಾಸ್ತೇಽಪಿ ಯಾನ್ತಿ ಪರಾಂ ಗತಿಮ್”​}
\end{verse}

“ಸ್ತ್ರೀ ವೈಶ್ಯ ಶೂದ್ರರಿಗೆಲ್ಲ ಪರಮಗತಿ ಪ್ರಾಪ್ತವಾಗುವುದು” ಎಂಬ ವಾಣಿಯನ್ನು ಮತ್ತೆ ಆಲಿಸುವೆವು.

ಗೀತೆಯ ಈ ಕೆಳಗಿನ ವಾಕ್ಯ, ಶ‍್ರೀಕೃಷ್ಣನ ವಜ್ರಸಮಾನ ಗಂಭೀರವಾಣಿ, ಎಲ್ಲರ ಬಂಧನವನ್ನು ಹರಿದು, ಶೃಂಖಲೆಯನ್ನು ಕತ್ತರಿಸಿ, ಪರಮಪದವನ್ನು ಮುಟ್ಟುವುದಕ್ಕೆ ಎಲ್ಲರಿಗೂ ಅಧಿಕಾರವಿದೆ ಎಂದು ಸಾರಿತು:

\begin{verse}
\textbf{ಇಹೈವ ತೈರ್ಜಿತಃ ಸರ್ಗೋ ಯೇಷಾಂ ಸಾಮ್ಯೇ ಸ್ಥಿತಂ ಮನಃ~।}\\\textbf{ನಿರ್ದೋಷಂ ಹಿ ಸಮಂ ಬ್ರಹ್ಮ ತಸ್ಮಾತ್​ ಬ್ರಹ್ಮಣಿ ತೇ ಸ್ಥಿತಾಃ~॥}
\end{verse}

“ಯಾವನ ಮನಸ್ಸು ಏಕತ್ವದಲ್ಲಿ ಸ್ಥಿರವಾಗಿರುವುದೋ ಅವನು ಈ ಜೀವನದಲ್ಲೇ ಸಂಸಾರದಿಂದ ಪಾರಾಗಿರುವನು. ಏಕೆಂದರೆ ದೇವರು ನಿರ್ದೋಷಿ, ಎಲ್ಲರಿಗೂ ಸಮ. ಅಂತಹವರು ಭಗವಂತನಲ್ಲಿಯೇ ವಾಸಿಸುವರು.”

\begin{verse}
\textbf{ಸಮಂ ಪಶ್ಯನ್​ ಹಿ ಸರ್ವತ್ರ ಸಮವಸ್ಥಿತಮೀಶ್ವರಮ್​~।}\\\textbf{ನ ಹಿನಸ್ತ್ಯಾತ್ಮನಾತ್ಮಾನಂ ತತೋ ಯಾತಿ ಪರಾಂ ಗತಿಮ್​~॥}
\end{verse}

“ಎಲ್ಲಾ ಕಡೆಗಳಲ್ಲಿಯೂ ಭಗವಂತನನ್ನೇ ನೋಡುತ್ತಿರುವ ಜ್ಞಾನಿ, ಆತ್ಮನನ್ನು ಆತ್ಮನಿಂದ ಹಿಂಸಿಸುವುದಿಲ್ಲ. ಅವನು ಪರಮ ಪದವನ್ನು ಸೇರುವನು.”

ಗೀತೆಯ ಉಪದೇಶಕ್ಕೆ ಸಜೀವ ಉದಾಹರಣೆಯನ್ನು ಕೊಡುವುದಕ್ಕೋ ಎಂಬಂತೆ ಅದರ ಅಂಶವನ್ನಾದರೂ ಕಾರ್ಯಕಾರಿಯಾಗಿ ಮಾಡುವುದಕ್ಕೆ ಪ್ರಚಾರಕನೇ ಮತ್ತೊಂದು ರೂಪದಿಂದ ಬಂದನು. ಅವನೇ ಶಾಕ್ಯಮುನಿ. ದೀನ ದುಃಖಿಗಳ ಉಪದೇಶಕ. ಅವನು ಜನಸಾಮಾನ್ಯರ ಭಾಷೆಯಲ್ಲಿ ಮಾತನಾಡುವುದಕ್ಕಾಗಿ, ದೇವಭಾಷೆಯನ್ನು ತೊರೆದನು; ದೀನ–ದರಿದ್ರ, ಪತಿತರೊಂದಿಗೆ ಇರುವುದಕ್ಕಾಗಿ, ರಾಜಸಿಂಹಾಸನವನ್ನೂ ತೊರೆದನು; ಎರಡನೆಯ ರಾಮನಂತೆ ಚಂಡಾಲನನ್ನೂ ಆಲಂಗಿಸಿದನು.

ಅವನ ಮಹಾಚರಿತ್ರೆ, ಅದ್ಭುತ ಪ್ರಚಾರಕಾರ್ಯ ನಿಮಗೆಲ್ಲಾ ತಿಳಿದಿದೆ! ಆದರೆ ಅವನ ಕಾರ್ಯದಲ್ಲಿ ಒಂದು ದೊಡ್ಡ ನ್ಯೂನತೆ ಇತ್ತು. ಅದಕ್ಕಾಗಿ ನಾವು ಇಂದು ವ್ಯಥೆಪಡಬೇಕಾಗಿದೆ. ಭಗವಾನ್​ ಬುದ್ಧನಲ್ಲಿ ಕೊಂಚವೂ ದೋಷವಿಲ್ಲ. ಅವನ ಚಾರಿತ್ರ್ಯ ವಿಶುದ್ಧವಾಗಿದೆ, ಉಜ್ವಲವಾಗಿದೆ. ಖೇದದ ವಿಷಯವೇನೆಂದರೆ, ಆ ಧರ್ಮಕ್ಕೆ ಸೇರಿದ ವಿಭಿನ್ನ ಅಸಭ್ಯ ಅಶಿಕ್ಷಿತ ಜನಾಂಗಗಳಿಗೆ ಈ ಪರಮೋಚ್ಚ ಆದರ್ಶವನ್ನು ಅನುಸರಿಸುವುದು ಅಸಾಧ್ಯವಾಯಿತು. ಆ ಜನಾಂಗದವರು ತಮ್ಮ ಮೂಢನಂಬಿಕೆಗಳು ಮತ್ತು ಬೀಭತ್ಸ ಪೂಜಾ ಪದ್ಧತಿಗಳೊಂದಿಗೆ ಆ ಧರ್ಮದ ಒಳಹೊಕ್ಕರು. ಅವರು ಕೆಲವು ಕಾಲ ಸಭ್ಯರಾದಂತೆಯೇ ಕಂಡರು. ಆದರೆ ಒಂದು ಶತಮಾನ ಕಳೆಯುವುದರೊಳಗೆ ಅವರು ಹಿಂದೆ ಉಪಾಸನೆ ಮಾಡುತ್ತಿದ್ದ ಸರ್ಪಭೂತಾದಿಗಳು ಹೊರಬಂದವು. ಈ ಪ್ರಕಾರವಾಗಿ ಇಡಿಯ ಭರತಖಂಡ, ಕುಸಂಸ್ಕಾರಗಳ ಲೀಲಾಕ್ಷೇತ್ರವಾಯಿತು. ಮೊದಮೊದಲು ಬಂದ ಬೌದ್ಧರು ಪ್ರಾಣಿಹಿಂಸೆಗೆ ವಿರುದ್ಧವಾಗಿ ಹೋರಾಡುವಾಗ ವೇದಗಳಲ್ಲಿ ಬರುವ ಯಜ್ಞವನ್ನು ಖಂಡಿಸಿದ್ದರು. ಈ ಯಜ್ಞವನ್ನು ಪ್ರತಿಯೊಂದು ಮನೆಯಲ್ಲಿಯೂ ನಡೆಸುತ್ತಿದ್ದರು. ಇದಕ್ಕೆ ಬೇಕಾಗಿದ್ದುದು ಉರಿಸುವುದಕ್ಕಷ್ಟು ಬೆಂಕಿ, ಹೆಚ್ಚಿನ ಆಡಂಬರವೇನೂ ಬೇಕಾಗಿರಲಿಲ್ಲ. ಬೌದ್ಧಧರ್ಮ ಪ್ರಚಾರವಾದ ಮೇಲೆ, ಯಜ್ಞಕ್ಕೂ ಲೋಪ ಬಂತು. ಅದರ ಬದಲು ಆಧುನಿಕ ಭಾರತದಲ್ಲಿ ನಾವು ನೋಡುವ ಐಶ್ವರ್ಯಯುಕ್ತಮಂದಿರ, ಆಡಂಬರ, ಪ್ರದರ್ಶನಾಲೋಲ ಪೂಜಾರಿವೃಂದವೆಲ್ಲ ಜಾರಿಗೆ ಬಂದವು. ಹೆಚ್ಚು ತಿಳಿವಳಿಕೆಯಿಲ್ಲದ ಆಧುನಿಕ ಪಂಡಿತವೃಂದವು ಬರೆದಿರುವ ಪುಸ್ತಕಗಳಲ್ಲಿ ಬುದ್ಧನನ್ನು ಬ್ರಾಹ್ಮಣರ ವಿಗ್ರಹ ಭೇದಕನೆಂದು ವರ್ಣಿಸಿರುವುದನ್ನು ಓದಿದಾಗ ನಗು ಬರುತ್ತದೆ. ಬೌದ್ಧಧರ್ಮವೇ ಬ್ರಾಹ್ಮಣ ಧರ್ಮ ಮತ್ತು ಮೂರ್ತಿ ಪೂಜೆಗೆ ಕಾರಣವೆಂಬುದು ಅವರಿಗೆ ಸ್ವಲ್ಪವೂ ತಿಳಿಯದು.

ಒಬ್ಬ ರಷ್ಯಾ ದೇಶದವನು ಒಂದೆರಡು ವರ್ಷಗಳ ಕೆಳಗೆ ಒಂದು ಪುಸ್ತಕವನ್ನು ಬರೆದನು. ಏಸುಕ್ರಿಸ್ತನ ಜನನದ ವಿಷಯದಲ್ಲಿ ಒಂದು ರಹಸ್ಯವನ್ನು ಕಂಡುಹಿಡಿದೆ ಎನ್ನುವನು. ಆ ಪುಸ್ತಕದಲ್ಲಿ ಒಂದು ಕಡೆ ಕ್ರಿಸ್ತ ಜಗನ್ನಾಥ ದೇವಾಲಯಕ್ಕೆ ಬ್ರಾಹ್ಮಣರೊಂದಿಗೆ ವಿದ್ಯೆ ಕಲಿಯಲು ಹೋದನಂತೆ. ಅವರ ವಿಗ್ರಹ ಪೂಜೆಯನ್ನು ಮತ್ತು ಜಾತಿಯ ಸಂಕುಚಿತ ಭಾವನೆಯನ್ನು ಸಹಿಸಲಾರದೆ ಟಿಬೆಟ್ಟಿನಲ್ಲಿರುವ ಲಾಮಾಗಳ ಸಮೀಪಕ್ಕೆ ಹೋಗಿ ಪರಿವರ್ತನೆಯನ್ನು ಪಡೆದು ಹಿಂತಿರುಗಿದ ಎಂದು ಬರೆದಿರುವನು. ಹಿಂದೂ ದೇಶದ ಚರಿತ್ರೆಯ ಪರಿಚಯವಿರುವ ಯಾರಿಗಾದರೂ ಇದು ಸುಳ್ಳು ಎಂಬುದು ಗೊತ್ತಾಗುವುದು. ಏಕೆಂದರೆ ಜಗನ್ನಾಥ ದೇವಾಲಯವು ಮುಂಚೆ ಬೌದ್ಧರ ವಿಹಾರವಾಗಿತ್ತು. ಅದನ್ನೂ ಮತ್ತು ಇತರ ದೇವಾಲಯ\-ಗಳನ್ನೂ ನಾವು ತೆಗೆದುಕೊಂಡು ಹಿಂದೂ ದೇವಾಲಯಗಳನ್ನಾಗಿ ಮಾರ್ಪಡಿಸಿದೆವು. ಹೀಗೆ ಇನ್ನೂ ನಾವು ಎಷ್ಟೋ ಮಾಡಬೇಕಾಗಿರುವುದು. ಈ ಜಗನ್ನಾಥದಲ್ಲಿ ಆಗ ಒಬ್ಬ ಬ್ರಾಹ್ಮಣನೂ ಇರಲಿಲ್ಲ. ಆದರೂ ಕ್ರಿಸ್ತನು ಬ್ರಾಹ್ಮಣರೊಂದಿಗೆ ಓದಲು ಅಲ್ಲಿಗೆ ಬಂದನೆಂದು ಹೇಳಿದೆ. ರಷ್ಯಾದ ನಮ್ಮ ಪ್ರಖ್ಯಾತ ಪ್ರಾಚೀನ ಸಂಶೋಧಕ ಹೇಳುವುದು ಇದು.

ಎಲ್ಲಾ ಭೂತಗಳಿಗೂ ದಯೆ ತೋರಬೇಕೆಂದು ಬೋಧಿಸಿದರೂ, ಎಷ್ಟೋ ಮಹೋನ್ನತ ನೀತಿ ಅವರ ಧರ್ಮದಲ್ಲಿ ಇದ್ದರೂ, ಒಬ್ಬ ಶಾಶ್ವತ ಆತ್ಮನಿರುವನೇ ಇಲ್ಲವೇ ಎಂಬ ವಿಷಯದ ಬಗ್ಗೆ ಕೂದಲು ಸೀಳುವಷ್ಟು ನಿಶಿತ ಚರ್ಚೆ ನಡೆದರೂ ಬೀಭತ್ಸ ಬೌದ್ಧಧರ್ಮದ ಸೌಧವು ಕುಸಿದು ನುಚ್ಚು ನೂರಾಯಿತು. ನಾಶ ಅತಿಯಾಗಿತ್ತು. ಬೌದ್ಧಧರ್ಮದ ಅವನತಿಯ ಸಮಯದಲ್ಲಿ ಇದ್ದ ಹೀನ ಆಚಾರ ವ್ಯವಹಾರಗಳ ವಿಷಯವನ್ನು ಕುರಿತು ಮಾತನಾಡುವುದಕ್ಕೆ ಇಚ್ಛೆಯಾಗಲಿ ಸಮಯವಾಗಲಿ ಇಲ್ಲ. ಅತಿ ಕುತ್ಸಿತ ಅನುಷ್ಠಾನ ಪದ್ಧತಿ, ಅತ್ಯಂತ ಭಯಾನಕ ಮತ್ತು ಅಶ್ಲೀಲ ಗ್ರಂಥಗಳು ಆ ಸಮಯದಲ್ಲಿ ಬಂದವು. ಮನುಷ್ಯ ಇದುವರೆಗೆ ಇಂತಹುದನ್ನು ಬರೆದಿರಲಿಲ್ಲ. ಇಂತಹುದನ್ನು ಕಲ್ಪಿಸಿಕೊಂಡಿರಲಿಲ್ಲ. ಇಂತಹ ಭೀಷಣ ಭಯಾನಕ ಅನುಷ್ಠಾನ ಪದ್ಧತಿ ಧರ್ಮದ ಹೆಸರಿನಲ್ಲಿ ಪ್ರಚಲಿತವಾಯಿತು. ಇವೆಲ್ಲಾ ಬೌದ್ಧಧರ್ಮದ ಕೊಡುಗೆ.

ಆದರೆ ಭರತಖಂಡ ಬದುಕಬೇಕಾಗಿತ್ತು. ಪುನಃ ಭಗವಂತನ ಆವಿರ್ಭಾವ ವಾಯಿತು. “ಧರ್ಮಗ್ಲಾನಿಯಾದಾಗ ನಾನು ಬರುತ್ತೇನೆ” ಎಂದ ಭಗವಂತನು ಈ ವೇಳೆ ದಕ್ಷಿಣದಲ್ಲಿ ಜನ್ಮಧಾರಣೆ ಮಾಡಿದನು. ಆ ಬ್ರಾಹ್ಮಣ ಯುವಕ ತಲೆ ಎತ್ತಿದನು. ಹದಿನಾರನೆಯ ವರುಷದ ಹೊತ್ತಿಗೆ ಅವನು ಗ್ರಂಥರಚನೆಯನ್ನೆಲ್ಲಾ ಪೂರೈಸಿದನು ಎಂದು ಹೇಳುವರು. ಇವರೇ ಅದ್ಭುತ ಶಂಕರಾಚಾರ್ಯರು. ಈ ಹದಿನಾರು ವರುಷದ ಯುವಕನ ಗ್ರಂಥಗಳು ಆಧುನಿಕ ಪ್ರಪಂಚದ ಅದ್ಭುತವಿಸ್ಮಯ. ಹಾಗೆಯೇ ಹುಡುಗನೂ ಅದ್ಭುತ ಪ್ರತಿಭಾಶಾಲಿ. ಭರತಖಂಡವನ್ನು ಪ್ರಾಚೀನ ವಿಶುದ್ಧ ಮಾರ್ಗಕ್ಕೆ ತರಬೇಕೆಂದು ಅವರು ಸಂಕಲ್ಪಮಾಡಿದರು. ಆದರೆ ಆ ಕಾರ್ಯ ಎಷ್ಟು ಕಠಿಣ ಎಂಬುದನ್ನು ಕುರಿತು ಆಲೋಚಿಸಿ ನೋಡಿ. ಭರತಖಂಡದ ಅಂದಿನ ಹೀನಸ್ಥಿತಿಯ ವಿಷಯವಾಗಿ ಕೆಲವು ವಿಷಯಗಳನ್ನು ನಾನು ಹೇಳಿರುವೆನು. ಈಗ ಸುಧಾರಿಸಬೇಕೆಂದಿರುವ ಸಂಸ್ಕಾರಗಳೆಲ್ಲಾ ಬೌದ್ಧಧರ್ಮದ ಅವನತಿಯ ಪರಿಣಾಮ. ಮಾನವಕೋಟಿಯ ಭಯಾನಕ ಜನಾಂಗಗಳಾದ ಬಲೂಚಿ, ಟಾರ್ಟರ್​, ಮುಂತಾದವರು ಭರತಖಂಡಕ್ಕೆ ಬಂದು ಬೌದ್ಧರಾಗಿ ನಮ್ಮೊಂದಿಗೆ ಬೆರೆತರು. ಅವರು ತಮ್ಮ ದೇಶದ ಆಚಾರಗಳನ್ನು ತಮ್ಮೊಡನೆ ತಂದರು. ಇದರ ಪರಿಣಾಮವಾಗಿ ನಮ್ಮ ರಾಷ್ಟ್ರೀಯ ಜೀವನವು ಅತ್ಯಂತ ಭಯಾನಕ ಪಾಶವಿಕ ಆಚಾರಗಳ ಕ್ಷೇತ್ರವಾಗಿ ಪರಿಣಮಿಸಿತು. ಆ ಯುವಕನಿಗೆ ಬೌದ್ಧರಿಂದ ಸಿಕ್ಕಿದ ಪೂರ್ವಿಕರ ಆಸ್ತಿಯೇ ಇದು. ಅಂದಿನಿಂದ ಇಂದಿನವರೆಗೆ ಭರತಖಂಡದಲ್ಲಿ ನಡೆದ ಕಾರ್ಯವೆಲ್ಲಾ ಬೌದ್ಧಧರ್ಮದಿಂದ ಅವನತಿ ಹೊಂದಿದ ದೇಶವನ್ನು ವೇದಾಂತದ ಮೂಲಕ ಸುಧಾರಿಸುವುದಾಗಿದೆ. ಇದು ಇನ್ನೂ ಆಗುತ್ತಿದೆ, ಪೂರ್ಣವಾಗಿಲ್ಲ. ಮಹಾ ತತ್ತ್ವಜ್ಞಾನಿ\break ಶಂಕರರು ಬಂದು, ಬೌದ್ಧ ಧರ್ಮದಲ್ಲಿ ಮತ್ತು ವೇದಾಂತ ಧರ್ಮದಲ್ಲಿ ಅಷ್ಟೇನೂ ವ್ಯತ್ಯಾಸವಿಲ್ಲ, ಆದರೆ ಬುದ್ಧನ ಶಿಷ್ಯರು ಗುರುವನ್ನು ಸರಿಯಾಗಿ ತಿಳಿದುಕೊಳ್ಳದೆ ಅವನತಿಗೆ ಇಳಿದು, ಆತ್ಮದ ಮತ್ತು ದೇವರ ಅಸ್ತಿತ್ವವನ್ನು ಅಲ್ಲಗಳೆದು ನಾಸ್ತಿಕರಾದರು ಎಂದು ತೋರಿಸಿದರು. ಬೌದ್ಧರೆಲ್ಲಾ ತಮ್ಮ ಹಳೆಯ ಧರ್ಮಕ್ಕೆ ಹಿಂತಿರುಗಿ ಬಂದರು. ಆದರೆ ಅವರ ಅನುಷ್ಠಾನದಲ್ಲಿ ಎಷ್ಟೋ ಆಚಾರಗಳು ರೂಢಿಗೆ ಬಂದಿದ್ದವು. ಅವನ್ನು ಏನು ಮಾಡಬೇಕು?

ಆಗ ಪ್ರತಿಭಾಶಾಲಿಗಳಾದ ರಾಮಾನುಜರು ಜನ್ಮವೆತ್ತಿದರು. ಶಂಕರಾಚಾರ್ಯರಲ್ಲಿ ಪ್ರತಿಭೆಗೆ ತಕ್ಕ ಉದಾರ ಹೃದಯವಿರಲಿಲ್ಲವೆಂದು ಹೇಳಬೇಕಾಗಿದೆ. ರಾಮಾನುಜಾ\break ಚಾರ್ಯರ ಹೃದಯ ಬಹು ಉದಾರವಾಗಿತ್ತು. ಪತಿತರ ದುಃಖದಲ್ಲಿ ಭಾಗಿಯಾದರು, ಸಹಾನುಭೂತಿ ತೋರಿದರು. ಆ ಸಮಯದಲ್ಲಿ ಪ್ರಚಲಿತವಿದ್ದ ಅನುಷ್ಠಾನ ಪದ್ಧತಿಗಳನ್ನು ತೆಗೆದುಕೊಂಡು ಸಾಧ್ಯವಾದಷ್ಟು ಪರಿಶುದ್ಧ ಮಾಡಿದರು. ಜನರಿಗೆ ಆವಶ್ಯಕವಾದ\break ಹೊಸ ಆಚಾರ, ಪೂಜಾ ವಿಧಾನ, ಇವನ್ನು ಜಾರಿಗೆ ತಂದರು. ಜೊತೆಗೇ ಸರ್ವೋಚ್ಚ ಆಧ್ಯಾತ್ಮಿಕ ಉಪಾಸನೆಯ ಬಾಗಿಲನ್ನು ಬ್ರಾಹ್ಮಣನಿಂದ ಚಂಡಾಲನವರೆಗೆ ತೆರೆದರು. ಇದೇ ರಾಮಾನುಜಾಚಾರ್ಯರ ಕಾರ್ಯ. ಈ ಕಾರ್ಯಹೀಗೇ ಸಾಗಿತು. ಕಾಲಾನಂತರ ಉತ್ತರ ದೇಶಕ್ಕೂ ಹರಡಿತು. ಅಲ್ಲಿ ಕೆಲವರು ಮಹಾವ್ಯಕ್ತಿಗಳು ಇದನ್ನು ಸ್ವೀಕರಿಸಿದರು. ಆದರೆ ಅದು ಮಹಮ್ಮದೀಯರ ಕಾಲದಲ್ಲಿ. ಸ್ವಲ್ಪ ಇತ್ತೀಚಿನವರಾದ ಅದೇ ಪಂಗಡಕ್ಕೆ ಸೇರಿದ ಆಚಾರ್ಯರಲ್ಲಿ ಚೈತನ್ಯರು ಸರ್ವಶ್ರೇಷ್ಠರು.

ರಾಮಾನುಜರ ಕಾಲದಿಂದಲೂ ನೀವು ಒಂದನ್ನು ಗಮನಿಸಬಹುದು. ಅದೇ ಪ್ರತಿಯೊಬ್ಬರಿಗೂ ಆಧ್ಯಾತ್ಮಿಕ ವಿದ್ಯೆಯ ಅಧಿಕಾರವನ್ನು ನೀಡಿದುದು. ಶಂಕರಾಚಾರ್ಯರಿಗಿಂತ ಮುಂಚಿನ ಆಚಾರ್ಯರ ಮುಖ್ಯ ಪಲ್ಲವಿಯೂ ಇದೇ ಆಗಿದ್ದಿತು. ಹಾಗೆಯೇ ರಾಮಾನುಜಾ\-ಚಾರ್ಯರ ಅನಂತರ ಬಂದ ಆಚಾರ್ಯರ ಮುಖ್ಯ ಪಲ್ಲವಿಯೂ ಅದೇ ಆಗಿದೆ. ಶಂಕರಾಚಾರ್ಯರು ಸಂಕುಚಿತ ಮನೋಧರ್ಮದವರು ಎಂದು ಏತಕ್ಕೆ ಹೇಳ\-ಲಾಗಿದೆಯೋ ಗೊತ್ತಿಲ್ಲ. ಅವರ ಗ್ರಂಥದಲ್ಲಿ ಅದಕ್ಕೆ ಆಧಾರ ದೊರಕುವುದಿಲ್ಲ. ಭಗವಾನ್​ ಬುದ್ಧದೇವನ ಸಂದೇಶದ ವಿಷಯದಲ್ಲಿ ಆದಂತೆಯೇ ಇಲ್ಲಿಯೂ ಕೂಡ ಭಗವಾನ್​ ಶಂಕರಾಚಾರ್ಯರಿಗೆ ಆರೋಪಿಸುವ ಈ ಪ್ರತ್ಯೇಕತೆಯ ನೀತಿ ಅವರ ಉಪದೇಶದಲ್ಲಿ ಸೇರಿಲ್ಲ; ಅವರ ಶಿಷ್ಯರು ಸರಿಯಾಗಿ ಅರ್ಥಮಾಡಿಕೊಳ್ಳಲಿಲ್ಲ ಎನ್ನಬೇಕಾಗಿದೆ. ಉತ್ತರ ದೇಶದ ಭಕ್ತ ಚೈತನ್ಯ ಗೋಪಿಯರ ಪ್ರೇಮೋನ್ಮಾದಕ್ಕೆ ಆದರ್ಶವಾಗಿರುವನು. ಚೈತನ್ಯದೇವನೇ ಸ್ವಯಂ ಬ್ರಾಹ್ಮಣ. ದೊಡ್ಡ ವಿದ್ವಾಂಸರ ವಂಶದಲ್ಲಿ ಜನಿಸಿದವನು. ಅವನು ನ್ಯಾಯವನ್ನು ಬೋಧಿಸು\- ತ್ತಿದ್ದನು. ಹಾಗೂ ತರ್ಕದಲ್ಲಿ ಎಲ್ಲರನ್ನೂ ಸೋಲಿಸುತ್ತಿದ್ದನು. ಇದೇ ಜೀವನದ ಉಚ್ಚತಮ ಆದರ್ಶವೆಂದು ಭಾವಿಸಿದ್ದನು. ಒಬ್ಬ ಮಹಾಪುರುಷರ ದಯೆಯಿಂದ ಇವನ ಜೀವನವೇ ಬೇರೆ ರೂಪ ತಾಳಿತು. ವಾದ, ವಿವಾದ, ತರ್ಕ, ನ್ಯಾಯ, ಅಧ್ಯಾಪನ, ಮುಂತಾದುವನ್ನೆಲ್ಲ ತ್ಯಜಿಸಿದನು. ಪ್ರೇಮೋನ್ಮತ್ತ ಚೈತನ್ಯ ಪ್ರಪಂಚದಲ್ಲೆಲ್ಲಾ ಸರ್ವಶ್ರೇಷ್ಠ ಭಕ್ತನಾದನು. ಅವನ ಭಕ್ತಿ ತರಂಗ ಭೂಮಿಯ ಮೇಲೆಲ್ಲಾ ಹರಿಯಿತು. ಎಲ್ಲರಿಗೂ ಶಾಂತಿ ದೊರಕಿತು. ಅವನ ಪ್ರೇಮಕ್ಕೆ ಮೇರೆ ಇರಲಿಲ್ಲ. ಸಾಧು, ಅಸಾಧು, ಹಿಂದೂ, ಮುಸಲ್ಮಾನ, ಪವಿತ್ರ, ಅಪವಿತ್ರ, ವೇಶ್ಯೆ, ಪತಿತ – ಎಲ್ಲರೂ ಅವನ ಪ್ರೇಮಕ್ಕೆ ಭಾಗಿಗಳಾಗಿದ್ದರು. ದಯೆಗೆ ಭಾಗಿಗಳಾಗಿದ್ದರು. ಅವನಿಂದ ಚಾಲಿತವಾದ ಸಂಪ್ರದಾಯವು, ಕಾಲಕ್ರಮೇಣ ಎಲ್ಲವೂ ಅವನತಿಗೆ ಬರುವಂತೆ ಅವನತಿಯನ್ನು ಮುಟ್ಟಿತು. ಆದರೂ ಇಂದಿಗೂ ದರಿದ್ರ, ದುರ್ಬಲ, ಜಾತಿಚ್ಯುತ, ಪತಿತ, ಮತ್ತು ಸಮಾಜದ ಬಾಹಿರರಿಗೆ ಅದು ಆಶ್ರಯವಾಗಿದೆ. ವಿಚಾರ ಪರರಾದ ದಾರ್ಶನಿಕರಲ್ಲಿ ಅದ್ಭುತ ಔದಾರ್ಯವಿದೆ ಎಂದು ನಾನು ಹೇಳಬೇಕಾಗಿದೆ. ಶಂಕರಾಚಾರ್ಯರ ಅನುಯಾಯಿಗಳಲ್ಲಿ ಯಾರೂ ಭರತಖಂಡದಲ್ಲಿರುವ ವಿಭಿನ್ನ ಸಂಪ್ರದಾಯಗಳು ನಿಜವಾಗಿಯೂ ಬೇರೆ ಬೇರೆ ಎಂದು ಹೇಳುವುದಿಲ್ಲ. ಆದರೆ ಜಾತಿಯ ದೃಷ್ಟಿಯಿಂದ ಅವರು ಅತಿಪ್ರತ್ಯೇಕ ಭಾವದವರು. ಆದರೆ ವೈಷ್ಣವಾಚಾರ್ಯರ ಮತದಲ್ಲಿ, ಜಾತಿಯ ವಿಷಯದಲ್ಲಿ ಅದ್ಭುತ ಔದಾರ್ಯ, ಧಾರ್ಮಿಕ ಭಾವನೆಯಲ್ಲಿ ಅತ್ಯಂತ ಪ್ರತ್ಯೇಕತೆಯನ್ನು ನೋಡುವೆವು.

ಒಬ್ಬರದು ಅದ್ಭುತ ಪ್ರತಿಭೆ. ಮತ್ತೊಬ್ಬರದು ವಿಶಾಲ ಹೃದಯ. ಅದ್ಭುತ ಪ್ರತಿಭೆ ಮತ್ತು ವಿಶಾಲ ಹೃದಯ ಇವೆರಡರಿಂದ ಕೂಡಿದ ಒಬ್ಬ ಅದ್ಭುತ ಪುರುಷನ ಜನ್ಮಕ್ಕೆ ಸಮಯ ಸನ್ನಿಹಿತ\-ವಾಯಿತು. ಶಂಕರಾಚಾರ್ಯರ ಅದ್ಭುತ ಪ್ರತಿಭೆಯ ಜೊತೆಗೆ ಚೈತನ್ಯರ ಅದ್ಭುತ ವಿಶಾಲ ಅನಂತ ಹೃದಯ ಇರಬೇಕಾಗಿತ್ತು. ಯಾರು ಪ್ರತಿ ಸಂಪ್ರದಾಯದಲ್ಲಿಯೂ ಒಬ್ಬನೇ ಈಶ್ವರ ಕೆಲಸಮಾಡುತ್ತಿರುವುದನ್ನು ನೋಡಬಲ್ಲರೋ, ಯಾರ ಹೃದಯ ದೀನರಿಗೆ, ದುರ್ಬಲರಿಗೆ, ಜಾತಿ ಬಾಹಿರರಿಗೆ, ಪದದಲಿತರಿಗೆ– ಅವರು ಭರತಖಂಡದಲ್ಲಿರಲಿ, ಪ್ರಪಂಚದಲ್ಲಿ ಬೇರೆಲ್ಲಿ ಬೇಕಾದರೂ ಇರಲಿ– ಮರುಗುತ್ತಿತ್ತೋ ಜೊತೆಗೆ ಯಾರ ಅದ್ಭುತ ಪ್ರಚಂಡ ಶಕ್ತಿ ಭರತಖಂಡದಲ್ಲಿ ಮತ್ತು ಹೊರಗೆ ಇರುವ ಭಿನ್ನ ಭಿನ್ನ ಸಂಪ್ರದಾಯಗಳಲ್ಲಿ ಸಮನ್ವಯತೆಯನ್ನು ಸ್ಥಾಪಿಸುವ ಉದಾತ್ತ ವಿಚಾರವನ್ನು ಚಿಂತಿಸಿತೊ, ಯಾರು ಹೃದಯ ಮತ್ತು ಬುದ್ಧಿಗಳ ಸಾಮರಸ್ಯವನ್ನು ಸಾಧಿಸುವ ವಿಶ್ವಧರ್ಮವನ್ನು ಸ್ಥಾಪಿಸಬಲ್ಲವರಾಗಿದ್ದರೋ ಅಂತಹ ವ್ಯಕ್ತಿ ಜನ್ಮವೆತ್ತಿದನು. ಹಲವು ವರುಷಗಳವರೆಗೆ ಅವರ ಪದತಳದಲ್ಲಿ ಕುಳಿತುಕೊಳ್ಳುವ ಸೌಭಾಗ್ಯ ನನ್ನದಾಗಿತ್ತು. ಅಂತಹ ವ್ಯಕ್ತಿಯ ಜನನಕ್ಕೆ ಸಮಯ ಸನ್ನಿಹಿತವಾಗಿತ್ತು. ಅಂತಹ ವ್ಯಕ್ತಿಯ ಜನನ ಆವಶ್ಯಕವಾಗಿತ್ತು. ಅವರು ಜನಿಸಿದರು. ಎಲ್ಲಕ್ಕಿಂತ ಅಧಿಕ ಆಶ್ಚರ್ಯದ ವಿಷಯವೇನೆಂದರೆ ಅವರ ಸಮಗ್ರ ಜೀವನ ಕಾರ್ಯವು ನಗರದ ಸಮೀಪದಲ್ಲಿತ್ತು. ಆ ನಗರವು ಪಾಶ್ಚಾತ್ಯ\break ಭಾವದಿಂದ ತುಂಬಿ ತುಳುಕಾಡಿ, ಅದರಲ್ಲಿ ಉನ್ಮತ್ತವಾಗಿ, ಭರತಖಂಡದ ಉಳಿದ ಎಲ್ಲಾ ನಗರಗಳಿಗಿಂತಲೂ ಅಧಿಕತರವಾಗಿ ವಿದೇಶೀ ಭಾವದಲ್ಲಿ ತನ್ಮಯವಾಗಿತ್ತು. ಅವರು ಅಲ್ಲಿ ವಾಸವಾಗಿದ್ದರೂ ಅವರಲ್ಲಿ ಪುಸ್ತಕ ಪಾಂಡಿತ್ಯವೇ ಇರಲಿಲ್ಲ. ಅಂತಹ ಮಹಾಪ್ರತಿಭಾ\break ಸಂಪನ್ನರಾದರೂ ತಮ್ಮ ಹೆಸರನ್ನು ಕೂಡ ಅವರಿಗೆ ಬರೆಯಲು ಬರುತ್ತಿರಲಿಲ್ಲ. ಆದರೆ\break ನಮ್ಮ ವಿಶ್ವವಿದ್ಯಾನಿಲಯದ ಪ್ರಚಂಡ ಪ್ರತಿಭಾಸಂಪನ್ನ ಪದವೀಧರರು ಅವರಲ್ಲಿ ಮೇರುದಂಡ ಪ್ರತಿಭೆಯನ್ನು ಕಂಡರು. ಅವರೇ ಶ‍್ರೀರಾಮಕೃಷ್ಣ ಪರಮಹಂಸರು, ಒಬ್ಬ ಅಲೌಕಿಕ ಮಹಾಪುರುಷರು. ಇದೊಂದು ಬೃಹತ್ಕಥೆ. ಅದನ್ನು ಹೇಳಲು ನನಗಿಂದು ಸಮಯವಿಲ್ಲ. ಭಾರತೀಯ ಮಹಾಪುರುಷರ ಪೂರ್ಣಪ್ರಕಾಶ ಸ್ವರೂಪರು, ಯುಗಾಚಾರ್ಯ ಶ‍್ರೀರಾಮಕೃಷ್ಣ ಪರಮಹಂಸರು. ಅವರ ಉಪದೇಶ ಇಂದಿಗೆ ವಿಶೇಷ ಕಲ್ಯಾಣಕಾರಿ. ಅವರಲ್ಲಿ ಈಶ್ವರೀಶಕ್ತಿ ಕೆಲಸ ಮಾಡುತ್ತಿರುವುದನ್ನು ಗಮನಿಸಿ. ಒಬ್ಬ ದರಿದ್ರ ಬ್ರಾಹ್ಮಣನ ಮಗುವಾಗಿ ಜನಿಸಿದರು. ಅವರ ಜನ್ಮ ವಂಗದೇಶದಲ್ಲಿ ಸುದೂರ ಅಜ್ಞಾತ ಅಪರಿಚಿತ ಗ್ರಾಮ ಒಂದರಲ್ಲಿ. ಇಂದು ಯೂರೋಪ್​ ಅಮೆರಿಕಾ ದೇಶಗಳಲ್ಲಿ ಸಹಸ್ರಾರು ಜನರು ವಾಸ್ತವಿಕವಾಗಿ ಅವರನ್ನು ಪೂಜಿಸುತ್ತಿರುವರು. ಭವಿಷ್ಯದಲ್ಲಿ ಮತ್ತೂ ಸಹಸ್ರಾರು ಜನರು ಅವರನ್ನು ಪೂಜಿಸುವರು. ಈಶ್ವರನ ಲೀಲೆಯನ್ನು ಅರಿಯಬಲ್ಲರು?

ಸಹೋದರರೆ, ಯಾರು ಇದರಲ್ಲಿ ವಿಧಾತನ ಕರವನ್ನು ನೋಡಲಾರರೋ ಅವರು ಅಂಧರು, ಜನ್ಮಾಂಧರು. ಮುಂದೆ ಸಮಯ ದೊರೆತರೆ ಅವಕಾಶ ದೊರೆತರೆ ಅವರ ವಿಷಯವನ್ನು ವಿಸ್ತಾರವಾಗಿ ವಿವರಿಸುವೆನು. ಸದ್ಯಕ್ಕೆ ಇಷ್ಟನ್ನು ಮಾತ್ರನಿಮಗೆ ಹೇಳುತ್ತೇನೆ. ನಾನೇನಾದರೂ ಒಂದು ಸತ್ಯವಾಕ್ಯವನ್ನು ನುಡಿದಿದ್ದರೆ ಅದೆಲ್ಲಾ ಅವರ ವಾಕ್ಯ. ಯಾವುದು ಅಸತ್ಯವಾಗಿರುವುದೋ, ಭ್ರಮೆಯಿಂದ ಕೂಡಿರುವುದೋ ಅಥವಾ ಮಾನವನ ಜಾತಿಗೆ ಹಿತಕಾರಿಯಲ್ಲವೋ ಅದೆಲ್ಲಾ ನನ್ನದು. ಅದರ ಜವಾಬ್ದಾರಿ ನನ್ನದು.

