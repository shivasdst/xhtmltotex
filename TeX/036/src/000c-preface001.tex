
\chapter*{ಅರಿಕೆ}

ನಮ್ಮ ರಾಜ್ಯದ ಜನರಲ್ಲಿ ವಿಜ್ಞಾನವನ್ನು ಪ್ರಚಾರ ಮಾಡಿ ವೈಜ್ಞಾನಿಕ ಮನೋಭಾವನೆ ಬೆಳವಣಿಗೆಗೆ ಉತ್ತೇಜನ ನೀಡುವುದು ಕನಾಟಕ ರಾಜ್ಯ ವಿಜ್ಞಾನ ಪರಿಷತ್ತಿನ ಮುಖ್ಯ ಧ್ಯೇಯ. ರಾಜ್ಯದ ವಿವಿಧ ಸ್ಥಳಗಳಲ್ಲಿ ಸ್ವಯಂ ಪ್ರೇರಣೆಯಿಂದ ರೂಪುಗೊಂಡಿರುವ ಪರಿಷತ್ತಿನ ಘಟಕಗಳು ಹಾಗೂ ಜಿಲ್ಲಾ ಸಮಿತಿಗಳ ಮುಖೇನ ಸ್ಥಳೀಯವಾಗಿ ಈ ಕೆಲಸಗಳನ್ನು ಮಾಡುವಲ್ಲಿ ನಿರತವಾಗಿವೆ.

ಉಪನ್ಯಾಸಗಳು, ವಿಚಾರ ಸಂಕಿರಣಗಳು, ವೈಜ್ಞಾನಿಕ ಪ್ರದರ್ಶನಗಳು ಮುಂತಾದವು\-ಗಳನ್ನು ಏರ್ಪಡಿಸುವ ಮೂಲಕ ದಿನ ನಿತ್ಯದ ಸಮಸ್ಯೆಗಳಿಗೆ ವೈಜ್ಞಾನಿಕ ಪರಿಹಾರಗಳನ್ನು\break ಹುಡುಕುವಲ್ಲಿ ಜನತೆಗೆ ನೆರವು ನೀಡುವ ಮುಖೇನ ಪರಿಷತ್ತಿನ ಧ್ಯೇಯಗಳನ್ನು ಸಫಲ\-ಗೊಳಿಸುವ ಪ್ರಯತ್ನ ನಡೆದಿದೆ. ಪರಿಷತ್ತು ಪ್ರಕಟಿಸುವ ನಿಯತಕಾಲಿಕೆಗಳು, ಕಿರುಹೊತ್ತಿಗೆ\-ಗಳು ಈ ಪ್ರಯತ್ನಕ್ಕೆ ಬೆಂಬಲ ನೀಡುತ್ತಲಿವೆ. ಈಗಾಗಲೇ \enginline{40} ವರ್ಷಕ್ಕೆ ಕಾಲಿಟ್ಟಿರುವ “ಬಾಲವಿಜ್ಞಾನ” ಮಾಸಪತ್ರಿಕೆ ಈ ದಿಶೆಯಲ್ಲಿ ಸಾಕಷ್ಟು ಯಶಸ್ಸು ಗಳಿಸಿ ಜನಪ್ರಿಯವಾಗಿದೆ. ವಿಜ್ಞಾನ ವಿಷಯ\-ಗಳನ್ನು ಕಿರುಹೊತ್ತಿಗೆಗಳ ಮುಖೇನ ಪ್ರಕಟಿಸುವ ಕಾರ್ಯ ಪರಿಷತ್ತು ಕೈಗೆತ್ತಿಕೊಂಡು ಈಗಾಗಲೇ \enginline{200}ಕ್ಕೂ ಹೆಚ್ಚು ಪುಸ್ತಕಗಳನ್ನು ಪ್ರಕಟಿಸಿದೆ. ಭಾರತದಲ್ಲಿ ಟ್ರಿಗನಮಿಟ್ರಿಕಲ್​ ಸರ್ವೇ ಬೆಳೆದು ಬಂದ ಹಾದಿಯನ್ನು ಲೇಖಕರು ಮನೋಜ್ಞವಾಗಿ ವಿವರಿಸಿದ್ದಾರೆ, ಈ ಮೂಲಕ ಓದುಗರು ವಿಷಯ ಅರಿಯಲು ಸಹಕಾರಿಯಾಗಿದೆ.

\bigskip

\begin{flushright}
\textbf{ಎಸ್. ವಿ. ಸಂಕನೂರ}\\ ಅಧ್ಯಕ್ಷರು, ಕರಾವಿಪ
\end{flushright}

\vskip -1.6cm

\noindent
ಎಸ್. ಎಂ. ಕೊಟ್ರುಸ್ವಾಮಿ \\ ಅಧ್ಯಕ್ಷರು, ಪುಸ್ತಕ ಪ್ರಕಟಣಾ ಸಮಿತಿ \\ ಕರಾವಿಪ 

\medskip

\noindent
ಬೆಂಗಳೂರು \\ ಜನವರಿ \enginline{2019}

