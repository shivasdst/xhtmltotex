
\chapter{ಥಿಯಡೋಲೈಟ್​ ದೇವಗೋಪುರದಿಂದ ನೆಲಕ್ಕೆ ಅಪ್ಪಳಿಸಿತು}

ಈ ಮಹೊದ್ದೇಶದ ಗ್ರೇಟ್​ ಟ್ರಿಗನಮಿಟ್ರಿಕಲ್​ ಸರ್ವೇಯಲ್ಲಿ ಅನೇಕ ಐತಿಹ್ಯಗಳು, ಘಟನಾವಳಿಗಳು ದಾಖಲಾಗಿವೆ. \enginline{1809}ರಲ್ಲಿ ಗ್ರೇಟ್​ ಆರ್ಕ್ ಕಾರ್ಯವನ್ನು ಕನ್ಯಾಕುಮಾರಿವರೆಗೆ ಯಶಸ್ವಿಯಾಗಿ ಮಾಡಿ ಮುಗಿಸಲಾಯಿತು. ಈ ಮಹಾ ಯಶಸ್ಸಿನ ಸಂಭ್ರಮಾಚರಣೆಯ ಬದಲು ದೊಡ್ಡ ಅವಘಡವೇ ಸಂಭವಿಸಿತು. ಟ್ರಿಗನಮಿಟ್ರಿಕಲ್​ ಸರ್ವೇಯಲ್ಲಿ ಮುಂದಿನ ಸ್ಟೇಷನ್​ನ್ನು ನಿಂತು ವೀಕ್ಷಣೆ ಮಾಡಲು, ಎತ್ತರವಾದ ಬೆಟ್ಟ, ದುರ್ಗಗಳು ಬೇಕು. ಆದರೆ, ತಮಿಳುನಾಡಿನ ಕಾವೇರಿ ಮುಖಜ ಭೂಮಿಯು, ಬೆಟ್ಟ ದುರ್ಗಗಳು ಇಲ್ಲದ, ಸಮತಟ್ಟಾದ ತೆಂಗಿನ ತೋಟದಿಂದ ಕೂಡಿದ ಪ್ರದೇಶ. ಸ್ಟೇಷನ್​ಗಳ ವೀಕ್ಷಣೆಗೆ ಈ ತೆಂಗಿನ ಮರಗಳು ಬಾರೀ ಅಡಚಣೆ. ಭಾರೀ ಮರ ಕಡಿತವೂ ಸಹ ಅಸಾಧ್ಯ. ಗೋಪುರ ನಿರ್ಮಾಣವೂ ಸಹ ದುಬಾರಿ. ಆದರೂ ಈ ತೆಂಗಿನ ಮರಗಳ ಅಡಚಣೆಯನ್ನು ಮೀರಿ ಮೇಲೇರಿ, ಅಲ್ಲಿ ಥಿಯಡೋಲೈಟ್​ ಉಪಕರಣ ಸ್ಥಾಪಿಸಿ, ನಿಂತು ನೋಡುವ ಎತ್ತರದ ಪ್ಲಾಟ್​ಫಾರಂ ಬೇಕು.

ತಮಿಳುನಾಡು ದೇಗುಲಗಳಿಗೆ ಹೆಸರುವಾಸಿ. ಅಲ್ಲಿನ ದೇಗುಲಗಳ ಗೋಪುರಗಳು ಬಹಳ ಎತ್ತರಕ್ಕೆ ಇವೆ. ತಂಜಾವೂರಿನ ಬೃಹದೀಶ್ವರ ದೇವಾಲಯ ಮಹಾಶಿವನ ದೇಗುಲ. \enginline{11}ನೇ ಶತಮಾನದ ಪ್ರಸಿದ್ಧ ಚೋಳ ರಾಜರಾಜ ಅದರ ನಿರ್ಮಾತೃ. ಅದರ ಮುಖ್ಯ ಗುಡಿಯ ಗೋಪುರದ ಎತ್ತರವು \enginline{217} ಅಡಿ ಇದೆ. ಇಡೀ ಭಾರತದಲ್ಲೇ ಭವ್ಯವಾದ ದೇವಗೋಪುರ ಅದು. ಲ್ಯಾಂಬ್​ಟನ್​ರವರಿಗೆ, ಅವರ ಟ್ರೈಯಾಂಗ್ಯುಲೇಷನ್​ ಸ್ಟೇಷನ್​ಗೆ ಸರಿಹೊಂದುವಂತೆ ಈ ಗುಡಿಯ ಗೋಪುರವು ಸೂಕ್ತ ಸ್ಥಾನದಲ್ಲಿದೆ. ತೆಂಗಿನ ತೋಟ, ದಟ್ಟ ಮರಗಳ ಮರೆಯನ್ನು ಮೀರಿ, ಮೇಲೇರಿ ನಿಂತು ನೋಡಲು ಅವರು ತಂಜಾವೂರಿನ ಈ ಪ್ರಸಿದ್ಧ ದೇವಾಲಯದ ಗೋಪುರವನ್ನು ಆಯ್ಕೆ ಮಾಡಿಕೊಂಡರು. ಪ್ಯಾರೀಸ್​ನಲ್ಲಿ ಚರ್ಚ್ ಗೋಪುರ ಮತ್ತು ಲಂಡನ್ನಿನಲ್ಲಿ ಚರ್ಚ್ ಗೋಪುರಗಳನ್ನು ಅಲ್ಲಿನ ಸರ್ವೇಯರುಗಳು ತಮ್ಮ ಟ್ರೈಯಾಂಗ್ಯುಲೇಷನ್​ ಸ್ಟೇಷನ್​ ವೀಕ್ಷಣೆಗಾಗಿ ಬಳಸಿಕೊಂಡ ಪೂರ್ವ ನಿದರ್ಶನಗಳು ಲ್ಯಾಂಬ್​ಟನ್​ರವರಿಗೆ ಇತ್ತು.

ತಂಜಾವೂರಿನ ದೇವಸ್ಥಾನದ ಅರ್ಚಕರನ್ನು ಬಹು ಗೌರವಾದರ ಎಚ್ಚರಿಕೆಗಳಿಂದ ಮನವೊಲಿಸಲಾಯಿತು. ಸರಿ ಎಲ್ಲವೂ ಸಿದ್ಧವಾಯಿತು. ಗ್ರೇಟ್​ ಥಿಯಡೋಲೈಟ್​ ಉಪಕರಣವು ಗೋಪುರದ ಮೇಲೆ ಹಗ್ಗ, ರಾಟೆಗಳ ಮೂಲಕ ಮೇಲೇರತೊಡಗಿತು. ಇನ್ನೇನು ಅದನ್ನು ಗೋಪುರದ ಮೇಲೆ ಎಚ್ಚರಿಕೆಯಿಂದ ಇರಿಸಬೇಕು. ಆದರೆ ಅಷ್ಟರಲ್ಲೇ ಅನಾಹುತ ಜರುಗಿಬಿಟ್ಟಿತು. ರಾಟೆಯ ಹಗ್ಗ ಹರಿದು, ಪೆಟ್ಟಿಗೆ ಸಮೇತ ಉಪಕರಣವು ಬಾರೀ ಶಬ್ದದೊಂದಿಗೆ ನೆಲಕ್ಕೆ ಅಪ್ಪಳಿಸಿತು. ಪೆಟ್ಟಿಗೆ ನಜ್ಜುಗುಜ್ಜಾಯಿತು. ಅತ್ಯಂತ ಪ್ರೀತಿಯಿಂದ ನಾಜೂಕಾಗಿ ಕಾಪಾಡಿಕೊಂಡು ಬಂದಿದ್ದ ಲ್ಯಾಂಬ್​ಟನ್​ರವರ ಥಿಯಡೋಲೈಟ್​ ಮತ್ತು ಅದರ ಅತಿ ಸೂಕ್ಷ್ಮವಾಗಿ ಕ್ಯಾಲಿಬ್ರೇಟ್​ ಮಾಡಿದ್ದ ವರ್ತುಲಾಕಾರದ ಕೋನಮಾಪಿ ಪ್ಲೇಟ್​ಗಳು, ಟ್ಯಾಂಜೆಂಟ್​ ಸ್ಕ್ರೂ, ಕ್ಲಾಂಪ್​ ಸ್ಕ್ರೂ, ಲೆವಲಿಂಗ್​ ಸ್ಕ್ರೂ, ಬೇಸ್​ ಪ್ಲೇಟ್​ ಇವೆಲ್ಲವೂ ವಿರೂಪಗೊಂಡವು.

ಯಾರಾದರೂ ಸಾಮಾನ್ಯರಾಗಿದ್ದರೆ, ಇಂಥ ದುರ್ಘಟನೆಯಿಂದ ಅಧೈರ್ಯಗೊಳ್ಳುತ್ತಿದ್ದರು. ಆದರೆ ಲ್ಯಾಂಬ್​ಟನ್​ರವರು ಅಧಮ್ಯ ಶಕ್ತಿ ಸಾಮರ್ಥ್ಯ ಇದ್ದಂತವರು. ಹಾನಿಗೊಂಡ ಉಪಕರಣದೊಂದಿಗೆ ಬೆಂಗಳೂರಿಗೆ ತಕ್ಷಣ ವಾಪಸ್​ ಆದರು. ಕಂಪನಿ ಆಡಳಿತಕ್ಕೆ ಬೆಂಗಳೂರಲ್ಲಿ ಆ ಕಾಲದ ಒಂದು ಮಿಲಿಟರಿ ಯಂತ್ರಾಗಾರ ಇತ್ತು. ಅಲ್ಲಿ ತಮ್ಮ ಶಿಬಿರದೊಳಗೆ ಸೇರಿ ಹಾನಿಯಾದ ಉಪಕರಣದ ರಿಪೇರಿಗೆ ತಾವೇ ತೊಡಗಿದರು. ಸಹಾಯಕ್ಕೆ ಕೆಲವೇ ಆಪ್ತ ಸಹಾಯಕರನ್ನು ಬಿಟ್ಟು, ಉಳಿದ ಯಾರನ್ನೂ ತಮ್ಮ ಶಿಬಿರದೊಳಗೆೆ ಸೇರಿಸಿಕೊಳ್ಳಲಿಲ್ಲ. ಇಡೀ ಉಪಕರಣದ ಸ್ಕ್ರೂ, ನಟ್ಟು, ಬೋಲ್ಟು, ವೆಡ್ಜ್​ಗಳನ್ನು ಬಿಚ್ಚಿ, ಬಿಡಿ ಭಾಗಗಳನ್ನು ಬೇರ್ಪಡಿಸಿದರು. ಹಾನಿಯಾದ ಭಾಗಗಳನ್ನು ತಟ್ಟಿ, ನೇರ ಮಾಡಿ ಸರಿಪಡಿಸಿದರು. ಬಿಡಿ ಭಾಗಗಳನ್ನು ಎಚ್ಚರಿಕೆಯಿಂದ ಮರುಜೋಡಿಸಿದರು. \enginline{6} ವಾರಗಳಲ್ಲಿ ಹೆಚ್ಚು ಕಡಿಮೆ ಅದರ ಮೊದಲ ರೂಪಕ್ಕೆ ತಂದರು. ನಂತರ \enginline{1830} ರವರೆಗೂ, ಆ ಅವಧಿವರೆಗಿನ ಎಲ್ಲಾ ವೀಕ್ಷಣೆಗೂ, ಇದೇ ಉಪಕರಣವನ್ನು ಬಳಸಲಾಯಿತು. ಈ ಥಿಯಡೊಲೈಟನ್ನು ಡೆಹರಾಡೂನ್​ನಲ್ಲಿರುವ ಸರ್ವೆ ಆಫ್​ ಇಂಡಿಯಾದ ಕೇಂದ್ರ ಕಛೇರಿಯಲ್ಲಿ ಈಗಲೂ ಸಂರಕ್ಷಿಸಿಡಲಾಗಿದೆ.

ಲ್ಯಾಂಬ್​ಟನ್​ರವರು ಈ ದುರ್ಘಟನೆಯ ಪೂರ್ಣ ಜವಾಬ್ದಾರಿಯನ್ನು ತಾವೇ ಹೊತ್ತರು. ಉಪಕರಣದ ಆಗಿನ ಬೆಲೆ \enginline{650} ಬ್ರಿಟೀಷ್​ ಪೌಂಡುಗಳು. ಈಗಿನ ಬೆಲೆ ಸುಮಾರು \enginline{65000} ಬ್ರಿಟೀಷ್​ ಪೌಂಡುಗಳಷ್ಟಾಗಬಹುದು. ಲ್ಯಾಂಬ್​ಟನ್​ರವರು ಬದಲಿ ಉಪಕರಣವನ್ನು ತಮ್ಮ ಸ್ವಂತ ಖರ್ಚಿನಲ್ಲಿಯೇ ಇಂಗ್ಲೆಂಡಿನಿಂದ ತರಿಸಿಕೊಂಡರು. ಆದ ಈ ಅನಾಹುತದ ವಿವರವನ್ನು ಸರಕಾರದ ಗಮನಕ್ಕೆ ಅಧಿಕೃತವಾಗಿ ತರಲೇ ಇಲ್ಲ.

\enginline{1811}ರಲ್ಲಿ ಬೆಂಗಳೂರು – ಕನ್ಯಾಕುಮಾರಿ ಸರಣಿ ಟ್ರಾಂಗ್ಯುಲೇಷನ್​ ಕಾರ್ಯ\break ಮುಗಿಯಿತು. ಗ್ರೇಟ್​ ಆರ್ಕ್ನ ಅಂತ್ಯ ಬಿಂದು ಕನ್ಯಾಕುಮಾರಿಯಿಂದ \enginline{8} ಮೈಲು ಈಶಾನ್ಯಕ್ಕೆ, ಕಡಲ ತಡಿಯಿಂದ \enginline{700} ಗಜ ದೂರದಲ್ಲಿದೆ. ಇದರ ಅಕ್ಷಾಂಶವನ್ನು ಖಗೋಳ ವೀಕ್ಷಣೆ ಮಾಡಿ, ನಿಗಧಿಮಾಡಲು ಲ್ಯಾಂಬ್​ಟನ್​ರವರು \enginline{28} ರಾತ್ರಿಗಳನ್ನು ತೆಗೆದುಕೊಂಡರು. ಒಟ್ಟು \enginline{236} ಬಾರಿ ಖಗೋಳ ವೀಕ್ಷಣೆಯನ್ನು ನಡೆಸಿದರು.

