
\chapter{ಮಧ್ಯ ಭಾರತದ ಕಾಡೆಂದರೆ ಸರ್ವೇ ತಂಡಕ್ಕೆ ಅದು ಖಚಿತ ಮರಣ ದಂಡನೆಯ ಶಿಕ್ಷೆ}

ಉತ್ತರದ ಗೋದಾವರಿ ಪ್ರದೇಶದಲ್ಲಿ ತಮ್ಮ ಸೆಕೆಂಡರೀ ಟ್ರೈಯಾಂಗ್ಯುಲೇಷನ್​ ಕಾರ್ಯದಲ್ಲಿ ಎವರೆಸ್ಟ್​ರವರು ತೊಡಗಿದರು. ಮುಂದಿನ ಟ್ರ್ಯಾಂಗ್ಯುಲೇಶನ್​ ಪಾಯಿಂಟ್​ ಇರುವ ಬೆಟ್ಟದ ಶಿಖರಕ್ಕೆ ಸಿಗ್ನಲ್​ ತಂಡವನ್ನು ರವಾನಿಸಿದರು. ಸಿಗ್ನಲ್​ ತಂಡದ ಕೆಲಸ ದುರ್ಗಮ ಕಾಡಿನ, ಅರಿಯದ ದಾರಿಯಲ್ಲಿ ನಡೆಯಬೇಕು. ಕಾಡಿನಲ್ಲಿ ಹುಲಿ, ಹಾವು, ಹಂದಿ, ಕರಡಿ, ಕಾಡಾನೆ, ಕಾಡೆಮ್ಮೆ, ಮನುಷ್ಯ ಗಾತ್ರದ ಜೇಡರ ಬಲೆ, ಈ ಎಲ್ಲಾ ಅಪಾಯಗಳನ್ನು ದಾಟಿ ಮುಂದುವರೆಯಬೇಕು. ಇವನ್ನೆಲ್ಲಾ ದಾಟಿ, ಬೆಟ್ಟ ಬಂಡೆ ಏರಿ, ಸರ್ವೆ ಬಿಂದು ತಲುಪಬೇಕು. ತಲುಪಿದ ಮೇಲೆ ಅದರ ಕುರುಹುವಾಗಿ ಬೆಂಕಿ ಹಾಕಿ ಹೊಗೆ ಎಬ್ಬಿಸಬೇಕು. ನೋಟ ರೇಖೆಗೆ ಅಡ್ಡ ಬರುವ ಕಾಡು ಕಡಿದು, ಬೆಟ್ಟ ಬಗೆದು ನೇರ ಮಾಡಿ ಅಲ್ಲಿ ಸಿಗ್ನಲ್​ ಸ್ಥಾಪಿಸಬೇಕು. ಇಷ್ಟೆಲ್ಲಾ ಮಾಡಲು ಸಿಗ್ನಲ್​ ತಂಡಕ್ಕೆ ಕನಿಷ್ಠ \enginline{3} ವಾರವಾದರೂ ಬೇಕು. ಇತ್ತ ಎವರೆಸ್ಟ್​ರವರು ಥಿಯಡೊಲೈಟ್​ ಸಿದ್ಧಗೊಳಿಸಿಕೊಂಡು, ಕೋನ ಓದಲು, ದೂರದ ದಿಗಂತದತ್ತ ಟೆಲಿಸ್ಕೋಪ್​ ತಿರುಗಿಸಿ, ಸಿಗ್ನಲ್​ ತಂಡವು ನೀಡುವ ಸಿಗ್ನಲ್​ಗಾಗಿ ದೃಷ್ಟಿ ನೆಟ್ಟು, ಕಾದು ಕುಳಿತರು. ಆದರೆ ಎವರೆಸ್ಟ್​ರವರು ಅತ್ತ ರವಾನಸಿದ ಸಿಗ್ನಲ್​ ತಂಡ ಹೊರಟು \enginline{3–4} ವಾರವಾದರೂ ದೂರದ ಕಾಡಿನಾಚೆಯ ಬೆಟ್ಟದಿಂದ ಹೊಗೆ ಹಾಕಿದ ಯಾವ ಲಕ್ಷಣಗಳೂ ಬರಲಿಲ್ಲ. ಎವರೆಸ್ಟ್​ರವರದು ಕೋಪಿಷ್ಠ ಸ್ವಭಾವ. ಸಿಗ್ನಲ್​ ತಂಡವು ಕೆಲಸ ತೊರೆದು, ಓಡಿ ಹೋಗಿರಬೇಕು ಎಂದುಕೊಂಡರು. ಸಿಟ್ಟಿನಲ್ಲಿಯೇ ಎರಡನೇ ತಂಡವನ್ನು ಒಂದನೇ ತಂಡ ಹೋದಲ್ಲಿಗೆ ಕಳುಹಿಸಿದರು. ಎರಡನೇ ತಂಡದಿಂದಲೂ ತಲುಪಿದ ಬಗ್ಗೆ, ಹೊಗೆಯ ಯಾವ ಲಕ್ಷಣಗಳೂ ಕಂಡು ಬರಲಿಲ್ಲ. ಮತ್ತೆ ಪುನಹ ಸಿಟ್ಟಿನಲ್ಲಿಯೇ ಮೂರನೇ ತಂಡವನ್ನು ಅಲ್ಲಿಗೆ ರವಾನಿಸಿದರು. ಮೂರನೇ ತಂಡದಿಂದಲೂ ಹೊಗೆಯ ಯಾವ ಲಕ್ಷಣಗಳೂ ಬರಲಿಲ್ಲ. ಸಹನೆ ಕಳೆದುಕೊಂಡು, ಕೊನೆಗೆ ತಮ್ಮ ಬ್ರಿಟೀಷ್​ ಮುಖ್ಯ ಸಹಾಯಕ ಜೋಸೆಫ್​ ಆಲೀವರ್​ ಅವರನ್ನು ಅಲ್ಲಿಗೆ ಕಳುಹಿಸಿದರು. ಅಂತಿಮವಾಗಿ ಜೋಸೆಫ್​ ಆಲೀವರ್​\break ಅಲ್ಲಿಗೆ ತಲುಪಿ, ಫ್ಲಾಗನ್ನು ಹಾರಿಸಲು ಯಶಸ್ವೀಯಾದರು. ಎವರೆಸ್ಟ್​ರವರಿಗೆ ಅಲ್ಲಿನ ಹೃದಯ ವಿದ್ರಾವಕ ಸುದ್ದಿಯನ್ನು ಆಲೀವರ್​ ತಲುಪಿಸಿದರು. ಆ ಹೊಸ ತಾಣದ ಹೆಸರು ಯಲ್ಲಾಪುರಮ್. ಯಲ್ಲಾಪುರಮ್‌ನಲ್ಲಿ ಹಿಂದಿನ ಮೂರೂ ಸಿಗ್ನಲ್​ ತಂಡದ ಅಷ್ಟೂ ಜನರು, ಭಯಾನಕ ಜ್ವರಕ್ಕೆ ತುತ್ತಾಗಿ ನೆಲಕ್ಕೊರಗಿ, ಔಷಧಿ ಚಿಕಿತ್ಸೆ ಒಂದೂ ಇಲ್ಲದೇ ಚಿಂತಾಜನಕ ಸ್ಥಿತಿಯಲ್ಲಿ, ಸಾವಿನ ಸ್ಥಿತಿಯಲ್ಲಿದ್ದರು.

ಎವರೆಸ್ಟ್​ರವರಿಗೆ ತಾವೇನಾದರೂ ಅಲ್ಲಿಗೆ ಹೋದರೆ, ಖಂಡಿತವಾಗಿ ತಾವೂ ಆ ಭಯಾನಕ ಜ್ವರಕ್ಕೆ ತುತ್ತಾಗುವ ಅಪಾಯವಿದೆ ಎಂದು ತಿಳಿದಿತ್ತು. ಆದರೂ ಸಹ ಜ್ವರಕ್ಕೆ ಹಿಂಜರಿಯದೇ, ಈಗಾಗಲೇ ಆಯಾಸಗೊಂಡಿದ್ದರೂ, ತಮ್ಮ ಇನ್ನೊಬ್ಬ ಸಹಾಯಕ, ತಂಡದ ವೈದ್ಯಾಧಿಕಾರಿ, ಡಾಕ್ಟರ್​ ವಾಯ್ಸೆಯವರ ಜೊತೆಗೆ ಕಠಿಣ ಮಾರ್ಗದ ಯಲ್ಲಾಪುರಮ್‌ಗೆ ಹೊರಟರು. ಆದರೆ, ಅಲ್ಲಿಗೆ ತಲುಪಿದಾಗ, ಸುರಿಯುತ್ತಿದ್ದ ಮಳೆ ನಿಂತು, ಆಕಾಶ ಶುಭ್ರವಾಗಿ, ವೀಕ್ಷಣಾ ಕಾರ್ಯಕ್ಕೆ ಅನುಕೂಲಕರ ವಾತಾವರಣ ಏರ್ಪಟ್ಟಿತು. ಸುತ್ತಲೂ ಇನ್ನೂ ಮೂರು ಬೆಟ್ಟಗಳು ಮುಂದಿನ ಟ್ರೈಯಾಂಗ್ಯುಲೇಶನ್​ ಪಾಯಿಂಟುಗಳಿಗೆ ಸೂಕ್ತವೆನ್ನುವಂತಹ ದಿಕ್ಕಿನಲ್ಲಿ ಗೋಚರಿಸಿದವು. ಎವರೆಸ್ಟ್​ರವರು ತಮ್ಮ ಬೆನ್ನನ್ನು ತಾವೇ ತಟ್ಟಿಕೊಳ್ಳುವಂತೆ ಖುಷಿಪಟ್ಟರು. ಆ ಹೊಸ ಪಾಯಿಂಟ್​ಗಳಿಗೆ ಸಿಗ್ನಲ್​ ತಂಡವನ್ನು ಖುಷಿಯಲ್ಲಿಯೇ ರವಾನಿಸಿದರು. ಆದರೆ, ಅವರ ಆ ಸಂತೋಷ ತಾತ್ಕಾಲಿಕವಾಗಿಬಿಟ್ಟಿತು. ಟ್ರೈಪಾಡ್​ ಇಟ್ಟು, ಥಿಯಡೊಲೈಟನ್ನು ಏರಿಸಿ, ಪ್ಲಮೆಟ್​ ಪಾಯಿಂಟ್​ ಮೇಲೆ ಸೆಂಟರಿಂಗ್​ ಲೆವಲಿಂಗ್​ ಮಾಡಿ, ಟೆಲಿಸ್ಕೋಪನ್ನು ತಿರುಗಿಸುವಷ್ಟರಲ್ಲಿ ಸ್ವತಃ ಎವರೆಸ್ಟ್​ರವರಿಗೇ ಜ್ವರ ಬಡಿಯಿತು.

ಯಲ್ಲಾಪುರಮ್‌ನಲ್ಲಿ ಇಡೀ ಸರ್ವೇ ತಂಡ ಜ್ವರದಿಂದ ತತ್ತರಿಸಿತು. ಸರ್ವೇ ತಂಡದ ವೈದ್ಯಾಧಿಕಾರಿ ಡಾಕ್ಟರ್​ ವಾಯ್ಸೆರವರಿಗೂ ಸಹ ತೀವ್ರವಾದ ಮಲೇರಿಯಾ ಜ್ವರ ಬಂದಿತು. ಮುಂದಿನ ಐದು ದಿನದಲ್ಲೇ, ಎವರೆಸ್ಟ್​ರವರ ಜೊತೆಗಿದ್ದ ಸರ್ವೇ ತಂಡದ ಉಳಿದ \enginline{150} ಜನರೂ, ಬೆಂಗಾವಲು ಪಡೆ, ಸಿಗ್ನಲ್​ ಜನ, ಸರಕು ಹೊರುವ ಕೂಲಿಗಳು, ಮಾವುತರು ಇವರೆಲ್ಲರೂ ಗಂಬೀರ ಮಲೇರಿಯಾ ಜ್ವರದಿಂದ ಪೀಡಿತರಾದರು. ಟ್ರೈಯಾಂಗ್ಯುಲೇಶನ್​ ಕೆಲಸವನ್ನು ಮುಂದುವರೆಸಲಾಗದೇ, ಸುಮಾರು \enginline{200} ಮೈಲು ದೂರದ ಹೈದರಾಬಾದಿನ ಮುಖ್ಯ ಕೇಂದ್ರಕ್ಕೆ, ಖಾಯಿಲೆ ಬಿದ್ದ ಇಡೀ ಸರ್ವೇ ತಂಡವು ಮರಳಿತು. ಸರ್ವೇ ಕ್ಯಾಂಪಿನಿಂದ ಹೈದರಾಬಾದಿಗೆ ವಾಪಸ್ಸಾಗಲು ತಂಡಕ್ಕೆ ಮೂರು ವಾರಗಳ ಕಾಲ ಹಿಡಿಯಿತು. \enginline{150} ಜನರಲ್ಲಿ \enginline{15} ಮಂದಿ ಮಾರ್ಗ ಮಧ್ಯದಲ್ಲಿಯೇ ಜ್ವರದಿಂದ ಮರಣಿಸಿದರು. ಉಳಿದವರ ಸ್ಥಿತಿಯು, ಈಗ ತಾನೆ ಗೋರಿಯಿಂದ ಎದ್ದುಬಂದ ಜೀವಂತ ಶವಗಳಂತೆ ಆಗಿದ್ದರು.

ಹೈದರಾಬಾದಿನ ಸರ್ವೇಯ ಮುಖ್ಯ ಬಿಡಾರವನ್ನು ತಲುಪಿದ ಮೇಲೂ ಎವರೆಸ್ಟ್​ರವರು ಸಂಪೂರ್ಣ ಗುಣಮುಖರಾಗಲಿಲ್ಲ. ಅವರಿಗೆ ಮೊದಲಿನ ಯೌವ್ವನದ ಪೂರ್ಣಶಕ್ತಿ ಮತ್ತೆ ಬರಲಿಲ್ಲ. ಆದರೂ ಯಲ್ಲಾಪುರಮ್‌ಗೆ ಪುನಃ ಹೊರಟರು. ಅಲ್ಲಿ ಪುನಃ ಅದೇ ಜ್ವರದಿಂದ ಪೀಡಿತರಾದರು. ಹಿಡಿದ ಆ ಟ್ರ್ಯಾಂಗ್ಯುಲೇಶನ್​ ಕಾರ್ಯವನ್ನು ಎವರೆಸ್ಟ್​ರವರಿಗೆ ಮುಂದುವರೆಸಲು ಆಗಲಿಲ್ಲ. ಅನಂತರ ಆ ಕಾರ್ಯವನ್ನು ಅವರ ನಂಬಿಕಸ್ಥ ಸಹಾಯಕ ಜೋಸೆಫ್​ ಆಲೀವರ್​ ಪೂರ್ಣಗೊಳಿಸಿದರು. ಎವರೆಸ್ಟ್​ರವರು ವಿಶ್ರಾಂತಿಯನ್ನು ಪಡೆಯಲೇ ಬೇಕಾಯಿತು. ಒಂದು ವರ್ಷದ ಸಿಕ್​ ಲೀವ್​ ಪಡೆದು, ಗುಣಮುಖರಾಗಲು ಗುಡ್​ಹೋಪ್​ ಭೂಷಿರಕ್ಕೆ ತೆರಳಿದರು. \enginline{1822}ರಲ್ಲಿ ಚೇತರಿಸಿಕೊಂಡು ಅಲ್ಲಿಂದ ಡ್ಯೂಟಿಗೆ ಮತ್ತೆ ವಾಪಸ್ಸಾದರು. ಅನಂತರ ಅವರನ್ನು ಪ್ರೈಮರಿ ಟ್ರಾಂಗ್ಯುಲೇಶನ್​ ಕಾರ್ಯಕ್ಕೆ ಕಳುಹಿಸಲಾಯಿತು. ಗ್ರೇಟ್​ ಆರ್ಕ್ ಸರಣಿಯಿಂದ ಪಶ್ಚಿಮದ ಬಾಂಬೆಯತ್ತ ಕವಲೊಡೆದ ಪ್ರೈಮರಿ ಟ್ರಾಂಗ್ಯುಲೇಶನ್​ ಕಾರ್ಯ ಅದು.

ಇತ್ತ ಲ್ಯಾಂಬ್​ಟನ್​ರವರು, ಗ್ರೇಟ್​ ಆರ್ಕ್‌ನ ಟ್ರ್ಯಾಂಗ್ಯುಲೇಷನ್​ ಕಾರ್ಯದಲ್ಲಿ ಅವಿಶ್ರಾಂತವಾಗಿ ತೊಡಗಿಸಿಕೊಂಡಿದ್ದರು. ಗ್ರೇಟ್​ ಆರ್ಕನ್ನು ಉತ್ತರದ ಹಿಮಾಲಯದತ್ತ ಕೊಂಡೊಯ್ಯುವ ಸಲುವಾಗಿ, ಸರ್ವೇಯ ಕೇಂದ್ರ ಸ್ಥಾನವನ್ನು ಈ ಮೊದಲಿದ್ದ ಹೈದರಾಬಾದ್​ನಿಂದ ನಾಗಪುರಕ್ಕೆ ಬದಲಾಯಿಸಲು ಯೋಚಿಸಿದರು. ಏಕೆಂದರೆ, ಮುಂದಿನ ಹಂತದ ಸರ್ವೇ ಕಾರ್ಯಕ್ಕೆ ನಾಗಪುರವು ಎಲ್ಲಾ ರೀತಿಯಲ್ಲೂ ಸೂಕ್ತವಾದ ಕೇಂದ್ರ ಸ್ಥಾನವಾಗಿತ್ತು.

\enginline{1823}ರ ಜನವರಿ ತಿಂಗಳ ಮೊದಲ ವಾರ. ಲ್ಯಾಂಬ್​ಟನ್​ರವರು ಹೈದರಾಬಾದ್​ನ್ನು ಬಿಟ್ಟು ನಾಗ್ಪುರಕ್ಕೆ ಹೊರಟರು. ದೇಹಾರೋಗ್ಯ ಹದಗೆಟ್ಟಿತು. ಒಂದೇ ಸಮನೆ ಹಿಡಿದ ಕೆಮ್ಮಿನಿಂದ ನರಳಿದರು. ನಾಗ್ಪುರಕ್ಕೆ ಹೋಗುವ ರಸ್ತೆಯಲ್ಲಿ ಹಿಂಗನ್​ ಘಾಟ್​ ಸಿಗುತ್ತದೆ. ಜನವರಿ \enginline{19} ರಂದು, ಹಿಂಗನ್​ಘಾಟ್​ ಕ್ಯಾಂಪ್​ನಲ್ಲಿ ಇದೇ ಅನಾರೋಗ್ಯದಲ್ಲಿ ತಂಗಿದರು. ಮಾರನೇ ದಿನ, ಅಂದರೆ \enginline{1823} ಜನವರಿ \enginline{20} ರಂದು, ಬೆಳಿಗ್ಗೆ ಲ್ಯಾಂಬ್​ಟನ್​ರವರು ಹಾಸಿಗೆಯಿಂದ ಏಳುವ ಸಮಯ ಮೀರಿತ್ತು. ಆದರೂ ಇನ್ನೂ ಎದ್ದು ಹೊರ ಬಂದಿರಲಿಲ್ಲ. ಇನ್ನೂ ಎದ್ದಿಲ್ಲವಲ್ಲ ಏಕೆ ಎಂದು ತಿಳಿಯದೇ, ಅವರ ಸಹಾಯಕ ಬಂದು, ಎದ್ದೇಳಿ ಎಂದು ವಿನಯದಿಂದ ಕರೆದ ಘಟನೆಯೂ ನಡೆಯಿತು. ಯಾರೊಂದಿಗೂ ಏನನ್ನೂ ಹೆಚ್ಚು ಮಾತಾಡದ, ಸದಾ ಮೌನದಲ್ಲಿ, ತಮ್ಮ ಕಾಯಕದಲ್ಲಿ ಮುಳುಗಿರುತ್ತಿದ್ದ ಕಾಯಕ ತಪಸ್ವಿ ಇಂದು ಕೂಡ ಯಾರೊಂದಿಗೂ ಎನನ್ನೂ ಮಾತನ್ನಾಡದೇ, ಚಿರಮೌನಕ್ಕೆ ಜಾರಿದ್ದರು. ಇಂಗ್ಲೆಂಡ್​ನಲ್ಲಿ ಹುಟ್ಟಿ, ಕೆನಡಾ ಅಮೇರಿಕಾಗಳಲ್ಲಿ ಸರ್ವೇ ಕಾರ್ಯ ಮಾಡಿ, ನಾಲ್ಕನೆ ಮೈಸೂರು ಯುದ್ಧದಲ್ಲಿ ಹೋರಾಡಿ, ಭಾರತೀಯ ಟ್ರಿಗನಮಿಟ್ರಿಕಲ್​ ಮಹಾ ಸರ್ವೇಯ ಹೊಣೆ ಹೊತ್ತು, ದಕ್ಷಿಣ ತುದಿಯಿಂದ ಅದನ್ನು ಮಧ್ಯ ಭಾರತದವರೆಗೆ ಯಶಸ್ವೀಯಾಗಿ ಕೊಂಡೊಯ್ದು, ಈಗ, ಹಿಂಗನ್​ಘಾಟ್​ನಲ್ಲಿ ಅಂತಿಮವಾಗಿ ವಿರಮಿಸಿದ್ದರು. ಆಗ ಲ್ಯಾಂಬ್​ಟನ್​ರವರಿಗೆ \enginline{70} ವರ್ಷ. ಅಲ್ಲೇ ಹಿಂಗನ್​ಘಾಟ್​ನಲ್ಲಿ ಅವರಿಗೆ ಒಂದು ಸಾದಾರಣವಾದ ಗೋರಿಯನ್ನು ಕಟ್ಟಲಾಯಿತು. ಸರ್ವೇಯ ಅಂತಹ ಮಾರಣಾಂತಿಕ ಹವಾಮಾನ ಪರಿಸರದಲ್ಲಿ, ನಿರಂತರ ಕಠಿಣ ದುಡಿಮೆಯಲ್ಲಿನ \enginline{70} ವರ್ಷದ ಬದುಕು ಎಂದರೆ ಅದು ಶತಾಯುಷಿ ಆದಂತೆಯೇ ಸರಿ.

ಲ್ಯಾಂಬ್​ಟನ್​ರವರು ಮೃದು ಮನಸ್ಸಿನ ವಿನಯವಂತ. ಮಿತಭಾಷೆಯ ಸರಳ ಸಜ್ಜನಿ. ವಿಜ್ಞಾನವೇ ತಮ್ಮ ಉಸಿರಾಗಿರುವ ಮಹಾ ಮೇಧಾವಿ. ಎರಡು ದಶಕಕ್ಕೂ ಮೀರಿದ ಮಹಾ ಸರ್ವೇಯ ಸಮಯದಲ್ಲಿ ಲ್ಯಾಂಬಟನ್​ರವರು ಯಾರನ್ನೂ ಬೈದಿರಲಿಲ್ಲ, ಗದರಿಸಿರಲಿಲ್ಲ. ಅಂತಹ ಪ್ರಸಂಗವೇ ಅವರಿಗೆ ಬಂದಿರಲಿಲ್ಲ. ಸ್ಥಳೀಯರಾದ ಭಾರತೀಯ ಕೆಲಸಗಾರರ ಬಗ್ಗೆ ಲ್ಯಾಂಬ್​ಟನ್​ರವರಿಗೆ ಸದಾ ಅನುಕಂಪ. ಇವನು ಬ್ರಿಟೀಷ್​, ಇವನು ಭಾರತೀಯ ಎಂಬ ತಾರತಮ್ಯವೇ ಅವರಲ್ಲಿ ಇರಲಿಲ್ಲ. ಕೆಲಸಗಾರರ ಪಾಲಿಗೆ ಅವರು ಆರಾಧ್ಯ ದೈವವೇ ಆಗಿದ್ದರು. ಲ್ಯಾಂಬ್​ಟನ್​ರವರ ಈ ಗುಣ ಬ್ರಿಟೀಷ್​ ಅಧಿಕಾರಿ ವರ್ಗದವರಲ್ಲಿ ಬಹು ಅಪರೂಪದ್ದು. ಅವರು ಸಹನಾ ಮೂರ್ತಿ. ಕಾಯಕ ತಪಸ್ವಿ. ಅವರ ಕೆಲಸವೊಂದನ್ನು ಬಿಟ್ಟು ಉಳಿದದ್ದೇನೂ ಅವರಿಗೆ ಕಾಣುತ್ತಿರಲಿಲ್ಲ. ಲ್ಯಾಂಬ್​ಟನ್​ರವರಿಗೆ ಏನೇ ತೊಂದರೆಯಾದರೂ, ಜ್ವರ ಮಲೇರಿಯಾ, ಬೇಧಿ ಖಾಯಿಲೆಗಳಿಂದ ನರಳಿದರೂ, ಇದು ಯಾವುದನ್ನು ಅವರು ದಾಖಲಿಸುತ್ತಿರಲಿಲ್ಲ. ತಂಡದ ಸಿಗ್ನಲ್​ ಜನರನ್ನು, ಸಹಾಯಕರನ್ನು ಕಳೆದುಕೊಂಡರೂ, ಆನೆ, ಹುಲಿ, ಕರಡಿ, ಕಾಡುಪ್ರಾಣಿಗಳಿಂದ ತೊಂದರೆಪಟ್ಟರೂ, ಇದು ಯಾವುದನ್ನು ಅವರು ತಮ್ಮ ಡೈರಿಯಲ್ಲಿ ದಾಖಲಿಸುತ್ತಿರಲಿಲ್ಲ. ವಾಸ್ತವತೆಗೆ ಜಾಹೀರಾತು ಬೇಕಿರಲಿಲ್ಲ ಅವರಿಗೆ. ಲ್ಯಾಂಬಟನ್​ರವರ ಮಹಾಶ್ರಮ, ಅರ್ಪಣಾ ಭಾವದ ದುಡಿಮೆ, ಅವರ ಮಾರ್ಗದರ್ಶನ, ಇವುಗಳಿಂದ ಗ್ರೇಟ್​ ಆರ್ಕ್ ಸರ್ವೇ ಕಾರ್ಯವು ಮಧ್ಯ ಭಾರತದವರೆಗೂ ಮುಂದುವರೆದಿತ್ತು ಮತ್ತು ಅವರ ನಂತರವೂ ಯಶಸ್ವೀಯಾಗಿ ಉತ್ತರದತ್ತ ಮುಂದುವರೆಯಿತು.

ಲ್ಯಾಂಬ್​ಟನ್​ರವರಿಗೆ ಒಂದು ಕಾಲದಲ್ಲಿ ಸಹಾಯಕರಾಗಿದ್ದು, ಸರ್ವೇಯ ಕ್ಷೇತ್ರ ಕಾರ್ಯದಿಂದ ಖಾಯಿಲೆ ಬಿದ್ದು, ದುರ್ಬಲತೆಯಿಂದ ಸರ್ವೇ ಕಾರ್ಯಕ್ಕೆ ಅನರ್ಹರಾಗಿ, ಇಂಗ್ಲೆಂಡಿಗೆ ವಾಪಾಸು ಹೋಗಿದ್ದವರು ಪ್ರಸಿದ್ಧ ಭೌತ ವಿಜ್ಞಾನಿ ಹೆನ್ರಿ ಕೇಟರ್​ರವರು. ಅವರು ಇಂಗ್ಲೆಂಡಿನಲ್ಲಿ, ಹೊಸದಾಗಿ ಒಂದು ಪುಟ್ಟ ಗಾತ್ರದ ಹಗುರ ಥಿಯಡೋಲೈಟ್​ನ್ನು ವಿನ್ಯಾಸಗೊಳಿಸಿದ್ದರು. ಅದು ಲ್ಯಾಂಬ್​ಟನ್​ರವರು ಬಳಸುತ್ತಿರುವ ಅರ್ಧಟನ್​ ಭಾರದ, ಮಹಾ ಥಿಯಡೋಲೈಟ್​\-ನಷ್ಟೇ ನಿಖರತೆಯದು. ಅದನ್ನು ಪೆಟ್ಟಿಗೆಯಲ್ಲಿಟ್ಟು ಒಬ್ಬ ಮನುಷ್ಯ ಸುಲಭವಾಗಿ ಹಿಡಿದು ನಡೆಯಬಹುದಾಗಿತ್ತು. ಪುಟ್ಟ ಗಾತ್ರದ ಬಹು ಹಗುರವಾದ ಥಿಯಡೋಲೈಟ್​ ಅದು. ಕಾಡಿನಲ್ಲಿ, ಗುಡ್ಡ ಬೆಟ್ಟಗಳಲ್ಲಿ ಒಂದು ಡಜನ್​ ಜನ ಹೊರಬೇಕಾದ ಅರ್ಧ ಟನ್​ ಭಾರವೆಲ್ಲಿ, ಒಬ್ಬ ಕೆಲಸಗಾರ ಸುಲಭವಾಗಿ ಹಿಡಿದು ನಡೆಯುವ ಉಪಕರಣವೆಲ್ಲಿ? ಇಂಗ್ಲೆಂಡಿನಿಂದ ತಮ್ಮ ಪ್ರೀತಿಯ ಗುರು, ಭಾರತದಲ್ಲಿ ಟ್ರಿಗನಮಿಟ್ರಿಕಲ್​ ಸರ್ವೇಯ ಜವಾಬ್ದಾರಿ ಹೊತ್ತಿದ್ದ\break ಲ್ಯಾಂಬ್​ಟನ್​ರವರಿಗೆ, \enginline{1823}ರಲ್ಲಿ ಹೆನ್ರಿ ಕೇಟರ್​ರವರು ಆ ಉಪಕರಣದ ಬಗ್ಗೆ ಸಂತೋಷದ ಸುದ್ದಿಯನ್ನು ಕಾಗದದಲ್ಲಿ ಬರೆದರು. ಆದರೆ ಆ ಪುಟ್ಟ ನಿಖರ ಉಪಕರಣದ ಸಂತೋಷದ ಸುದ್ದಿ ಬರುವ ಮೊದಲೇ ಲ್ಯಾಂಬ್​ಟನ್​ರವರು ಟ್ರಿಗನಮಿಟ್ರಿಕಲ್​ ಸರ್ವೇಯ, ತಮ್ಮ ತಪೋ ಕಾರ್ಯವನ್ನು ನಿಲ್ಲಿಸಿ, ತಾವೇ ಪೆಟ್ಟಿಗೆ ಸೇರಿ, ಹಿಂಗನ್​ಘಾಟ್​ ನೆಲದಲ್ಲಿ ಚಿರನಿದ್ರೆಗೆ ಜಾರಿದ್ದರು.

ಎವರೆಸ್ಟ್​ರವರು \enginline{1823}ರಲ್ಲಿ ಲ್ಯಾಂಬ್​ಟನ್​ರವರ ಮರಣದ ನಂತರ ಜಿ.ಟಿ.ಎಸ್​ನ\break ಸೂಪರಿನ್​ಟೆಂಡೆಂಟ್​ ಆಗಿ ನೇಕಗೊಂಡರು. ಪುನಃ ಅವರು ತಮ್ಮ ಕಠಿಣ ಸರ್ವೇ ಕಾರ್ಯದಲ್ಲಿ ತೊಡಗಿದರು. ಒಂದು ಕಡೆ, ಕೃಷ್ಣಾ ಗೋದಾವರಿ ನದಿಗಳ ನಡುವೆ ಗ್ರೇಟ್​ ಆರ್ಕ್‌ನ ಪೂರ್ವಕ್ಕೆ, ಯಲ್ಲಾಪುರಮ್ನಲ್ಲಿ ಎವರೆಸ್ಟ್​ರವರ ಸರ್ವೇ ತಂಡದ ದುರಂತಮಯ ಕಥೆ. ಅದೇ ಸಮಯದಲ್ಲಿ, ಗ್ರೇಟ್​ ಆರ್ಕ್‌ನ ಪಶ್ಚಿಮಕ್ಕೆ, ಟ್ರೈಯಾಂಗ್ಯುಲೇಷನ್​ ಕಾರ್ಯದಲ್ಲಿ ತೊಡಗಿಸಿಕೊಂಡ ಇನ್ನೊಂದು ಸರ್ವೇ ತಂಡದ ದುರಂತಮಯ ಕಥೆಯೂ ದಾಖಲಾಗಿದೆ.

\enginline{1816}ರಲ್ಲಿ ಲೆಫ್ಟಿನೆಂಟ್​ ಜೇಮ್ಸ ಗಾರ್ಲಿಂಗ್​ರವರ ನೇತೃತ್ವದಲ್ಲಿ, ಗ್ರೇಟ್​ ಆರ್ಕ್‌ನ ಪಶ್ಚಿಮಕ್ಕೆ, ಸೆಕೆಂಡರಿ ಟ್ರ್ಯಾಂಗ್ಯುಲೇಷನ್​ ಕಾರ್ಯದಲ್ಲಿ ತೊಡಗಿತ್ತು. ಸರ್ವೇ ಕಾರ್ಯದ ಪ್ರಗತಿ ಉತ್ತಮವಾಗಿಯೇ ಇತ್ತು. \enginline{1819}ರಲ್ಲಿ ಗಾರ್ಲಿಂಗ್​ರವರ ಸಹಾಯಕ, ಗುರುತು ಹಿಡಿಯದ ಜ್ವರದಿಂದ ಮಡಿದರು. ಮುಂದಿನ ವರ್ಷ ಲೆಫ್ಟಿನೆಂಟ್​ ಜೇಮ್ಸ ಗಾರ್ಲಿಂಗ್​ರವರು ಸಹ ಖಾಯಿಲೆಯಿಂದ ಮಡಿದರು. ಟ್ರಾವಂಕೂರ್​ನಿಂದ ಶ‍್ರೀಯುತ ಕಾನರ್​ರವರು, ಆ ಸೆಕೆಂಡರಿ ಟ್ರ್ಯಾಂಗ್ಯುಲೇಷನ್​ ಕಾರ್ಯವನ್ನು ಮುಂದುವರೆಸುವ ಕಾರ್ಯಕ್ಕೆ ನೇಮಕವಾಗಿ ಬಂದರು. ಅವರೂ ಹೈದರಾಬಾದ್​ ತಲುಪಿದ ಒಂದು ತಿಂಗಳಿನಲ್ಲೇ ಮಡಿದರು. ಡಿಸೆಂಬರ್​ \enginline{1821}ರಲ್ಲಿ ರಾಬರ್ಟ್‌ಯಂಗ್​ರವರು ಸರ್ವೇ ಕಾರ್ಯದ ಹೊಣೆಯನ್ನು ವಹಿಸಿಕೊಂಡರು. ಅವರೂ ಖಾಯಿಲೆಗೆ ತುತ್ತಾಗಿ \enginline{1823}ರಲ್ಲಿ ಜುಲೈನಲ್ಲಿ ಮರಣಿಸಿದರು. ನಂತರ \enginline{1827}ರಲ್ಲಿ,\break ಕ್ರಿಸ್​ರವರಿಗೆ ಆ ಟ್ರ್ಯಾಂಗ್ಯುಲೇಷನ್​ ಕಾರ್ಯದ ಜವಾಬ್ದಾರಿಯನ್ನು ಹೊರಿಸಲಾಯಿತು.\break ಕ್ರಿಸ್​ರವರು ತಮ್ಮ ಜವಾಬ್ದಾರಿಯನ್ನು ವೆಬ್​ರವರಿಗೆ ಹೊರಸಿದರು. ವೆಬ್​ರವರು \enginline{1829}ರಲ್ಲಿ ಖಾಯಿಲೆಗೆ ತುತ್ತಾದರು.

ಇದು ಸರ್ವೇ ಕಾರ್ಯದಲ್ಲಿ ತೊಡಗಿಸಿಕೊಂಡ ಮೇಲಧಿಕಾರಿ ವರ್ಗದವರ, ಅಧಿಕೃತ ಸಾವುನೋವುಗಳ ಸಣ್ಣ ಅವಧಿಯ ಭೀಕರವಾದ ಅಂಕಿ ಅಂಶ. ಸರ್ವೇ ತಂಡದಲ್ಲಿ ದುಡಿಯುತ್ತಿದ್ದ ಅಧೀನವರ್ಗದ, ಸ್ಥಳೀಯ ಜನರ, ದಾಖಲಾಗದ, ಮಣ್ಣುಪಾಲಾದ ದುರಂತಮಯ ಸರಣಿ ಕಥೆಗಳು ಇನ್ನೂ ಎಷ್ಟಿವೆಯೋ? ಮಧ್ಯ ಭಾರತದ ಕಾಡುಗಳೆಂದರೆ ಸರ್ವೇ ತಂಡಕ್ಕೆ ಖಚಿತ ಮರಣ ದಂಡನೆಯ ಶಿಕ್ಷೆಗಳು.

ಡಾಕ್ಟರ್​ ವಾಯ್ಸೆರವರದ್ದು ಮತ್ತೊಂದು ದುರಂತಮಯ ಕಥೆ. ಡಾಕ್ಟರ್​ ವಾಯ್ಸೆರವರು ಸರ್ವೇ ತಂಡಕ್ಕೆ ಸರ್ಜನ್​ ಮತ್ತು ಜಿಯಾಲಜಿಸ್ಟ್​ ಆಗಿ ನೇಮಕವಾದವರು. ಡೆಕ್ಕನ್​ ಶಿಲೆಗಳು, ದಕ್ಷಿಣ ಭಾರತದ ವಜ್ರ ನಿಕ್ಷೇಪಗಳು, ಆಗ್ರಾದ ಕಟ್ಟಡ ಶಿಲೆಗಳು ಈ ಬಗ್ಗೆ ಬರೆದ ಮೊದಲ ಲೇಖಕ ಎಂಬ ಹೆಸರು ಇವರದ್ದು. ಎವರೆಸ್ಟ್​ರವರ ಜೊತೆಯಲ್ಲಿಯೇ \enginline{1818}ರಲ್ಲಿ ಇವರೂ ಸಹ ಟ್ರಿಗನಮಿಟ್ರಿಕಲ್​ ಸರ್ವೇಗೆ ನೇಮಕವಾದವರು. ಟ್ರೈಯಾಂಗ್ಯುಲೇಷನ್​ ಅಳತೆ\break ಕಾರ್ಯದಲ್ಲಿ, ಟ್ರ್ಯಾಂಗ್ಯುಲೇಷನ್​ ಬಿಂದುಗಳ ಸ್ಥಾನಗಳು ಮ್ಯಾಪಿಂಗ್​ ಉದ್ದೇಶಕ್ಕೆ ಸಹಕಾರಿ ಆಗುವಂತಿರಬೇಕು, ಆದ್ದರಿಂದ, ಮುಖ್ಯ ಸರ್ವೇ ತಂಡಕ್ಕಿಂತ ಒಂದು ತಂಡವು ಮುಂದಾಗಿ ಹೋಗಿ, ಸ್ಥಳ ಪರಿಶೀಲನೆ ಮಾಡಿ, ಯೋಜಿತ ನೇರದಲ್ಲಿ, ಅನುಕೂಲ ಸ್ಥಳದಲ್ಲಿ, ಸ್ಟೇಷನ್​ ಆಯ್ಕೆ ಮಾಡುವ, ಪೂರ್ವ ಸಿದ್ಧತೆಯ ಪ್ರಮುಖ ಕಾರ್ಯ ಮಾಡಬೇಕಿರುತ್ತದೆ. ಇದನ್ನು ಸ್ಥಳಾನ್ವೇಷಣಾ ಸರ್ವೇ ಅಥವಾ ರೆಕಾನಾಯ್ಸನ್ಸ್​ ಸರ್ವೇ ಎನ್ನುತ್ತಾರೆ. ಉತ್ತರದ ಆಗ್ರಾದವರೆಗೂ ಈ ಸ್ಥಳಾನ್ವೇಷಣಾ ಸರ್ವೇ ಕಾರ್ಯ ಮಾಡಲು ಲ್ಯಾಂಬ್​ಟನ್​ರವರು,\break ಡಾ. ವಾಯ್ಸೆರವರನ್ನು ಕಳುಹಿಸಿದ್ದರು. ಇದಕ್ಕೂ ಮೊದಲು, ಸರ್​ ಜಾರ್ಜ್ ಎವರೆಸ್ಟ್​ರವರು \enginline{1818}ರಲ್ಲಿ ಕರ್ನಲ್​ ಲ್ಯಾಂಬ್​ಟನ್​ರವರನ್ನು ಟ್ರಿಗನಮಿಟ್ರಿಕಲ್​ ಸರ್ವೇಯಲ್ಲಿ ಜೊತೆಗೂಡಿದಾಗ, ಎವರೆಸ್ಟ್​ರವರ ಸಹೋದ್ಯೋಗಿಯಾಗಿ ಡಾ. ವಾಯ್ಸೆರವರೂ ಜೊತೆಯಲ್ಲಿಯೇ ಇದ್ದರು. ಯಲ್ಲಾಪುರಮ್‌ನಲ್ಲಿ, ಎವರೆಸ್ಟ್​ರವರ ಜೊತೆಅಗೆ, ಇಡೀ ಸರ್ವೇ ತಂಡವು ಮಲೇರಿಯಾ ಕಾಯಿಲೆಗೆ ಒಳಗಾದಾಗ, ಡಾಕ್ಟರ್​ ವಾಯ್ಸೆರವರೂ ಸಹ ತೀವ್ರವಾದ\break ಮಲೇರಿಯಾ ಜ್ವರದಿಂದ ಪೀಡಿತರಾಗಿದ್ದರು.

\enginline{1823}ರಲ್ಲಿ ಗ್ರೇಟ್​ ಆರ್ಕ್‌ನ ಪಿತಾಮಹಾ, ಸರ್ವೇ ತಂಡದ ಭೀಷ್ಮ, ಕರ್ನಲ್​\break ಲ್ಯಾಂಬ್​ಟನ್​ರವರು ಕಾಲವಾದರು. ಅದಕ್ಕಿಂತ ಮೊದಲು ಅವರು ಡಾಕ್ಟರ್​ ವಾಯ್ಸೆರವರನ್ನು ಮುಂಬಡ್ತಿ ನೀಡಿ, ಟ್ರಿಗನಮಿಟ್ರಿಕಲ್​ ಸರ್ವೇಯ ಸಹಾಯಕ ಅಧೀಕ್ಷಕ ಹುದ್ದೆಗೆ ನೇಮಿಸಬೇಕೆಂದು ಒತ್ತಾಯಿಸಿ ಸರ್ಕಾರಕ್ಕೆ ಕಾಗದ ಬರೆದಿದ್ದರು. ವಾಯ್ಸೆರವರು ಸಹ ಮುಂಬಡ್ತಿಯನ್ನು ನಿರೀಕ್ಷಿಸಿದ್ದರು. ಇವರಿಗೆ ಬಡ್ತಿ ಮಂಜೂರಾಗಲಿಲ್ಲ. ಬದಲಿಗೆ, ಇವರ ಜತೆಗಾರ ಎವರೆಸ್ಟ್​ರವರು ಲ್ಯಾಂಬ್​ಟನ್​ರವರ ಉತ್ತರಾಧಿಕಾರಿಯಾಗಿ, ಡಾಕ್ಟರ್​ ವಾಯ್ಸೆರವರು ನಿರೀಕ್ಷಿಸಿದ್ದ ಹುದ್ದೆಗೆ ಬಂದರು.

ಡಾಕ್ಟರ್​ ವಾಯ್ಸೆರವರಿಗೆ, \enginline{1819}ರ ಯಲ್ಲಾಪುರಂ ಜ್ವರ ಪೂರ್ಣ ಗುಣವಾಗುವುದೇ ಇಲ್ಲ. ಆದರೂ ಕರ್ತವ್ಯಕ್ಕೆ ಮರಳಿದರು. ಜೊತೆಗೆ ನಂತರದ ಆಗ್ರಾದ ದಕ್ಷಿಣದಲ್ಲಿ, ರೆಕಾನಾಯಿಸನ್ಸ್​ ಸರ್ವೇಯ ಸಂದರ್ಭದಲ್ಲಿ, ಕ್ರೂರ ಹುಲಿಯ ದಾಳಿಯಿಂದ ಜರ್ಜರಿತರಾಗಿದ್ದರು. ಭಯಂಕರವಾದ ಒಂದು ಹುಲಿಯಂತೂ ಐದು ಜನರನ್ನು ಎಳೆದೊಯ್ದಿತ್ತು. ಅಷ್ಟೇ ಅಲ್ಲ, ತಮ್ಮ ಕೈ ಹಿಡಿಯುವವಳಿದ್ದ, ಜೊತೆಗಿದ್ದ ಪ್ರಿಯತಮೆಯನ್ನೇ ಆ ಹುಲಿ ಮಿಂಚಿನ ವೇಗದಲ್ಲಿ ದಾಳಿಮಾಡಿ, ನೋಡು ನೋಡುತ್ತಿದ್ದಂತೆಯೇ ಇವರ ಕಣ್ಣು ಮುಂದೆಯೇ ಅವಳನ್ನು ಸಾಯಿಸಿತ್ತು. ಕಾಡುವ ಯಲ್ಲಾಪುರಂ ಜ್ವರ, ಕ್ರೂರ ಹುಲಿಯ ಭಯಂಕರ ದಾಳಿ, ಇವುಗಳಿಂದ ಡಾಕ್ಟರ್​ ವಾಯ್ಸೆರವರು ತೀವ್ರ ಆಘಾತಕ್ಕೆ ಒಳಗಾಗಿದ್ದರು. ಇನ್ನೇನೂ ಚೇತರಿಸಿಕೊಳ್ಳಲಾಗದೆ ಇದ್ದಾಗ ತಮ್ಮ ಹುದ್ದೆ ತ್ಯಜಿಸಿ ಇಂಗ್ಲೆಂಡಿಗೆ ಮರಳಲು ತೀರ್ಮಾನಿಸಿ, ಕೊಲ್ಕಾತ್ತಾಗೆ ಹೊರಟರು. ಕೊಲ್ಕಾತ್ತಾಗೆ ಬರುತ್ತಿರುವ ದಾರಿಯಲ್ಲೇ ದೀರ್ಘ ಕಾಲದ ಅನಾರೋಗ್ಯವು ವಿಷಮಿಸಿ ದುರಂತ ಸಾವನ್ನು ಅಪ್ಪಿದರು.

