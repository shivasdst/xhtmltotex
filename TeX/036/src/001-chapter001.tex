
\chapter{ಭೂಮಾಪನಕ್ಕೆ ಭದ್ರ ಬುನಾದಿ}

ಪ್ರತಿ ವರ್ಷ ಏಪ್ರಿಲ್ ​\enginline{10}, ಭಾರತಕ್ಕೆ ಒಂದು ರಾಷ್ಟ್ರೀಯ ಮಹತ್ವದ ದಿನ. ಈ ದಿನ ವಿಜ್ಞಾನದ ಇತಿಹಾಸದಲ್ಲಿಯೇ ಒಂದು ಮೈಲಿಗಲ್ಲು. ಭೂಮಾಪನದ ಇತಿಹಾಸದಲ್ಲಿ ಸುವರ್ಣಾಕ್ಷರದ ದಿವಸ. ಭಾರತದಲ್ಲಿ, ಕರ್ನಲ್​ ಲ್ಯಾಂಬ್​ಟನ್​ ಎಂಬ ಮಿಲಿಟರಿ ಇಂಜಿನಿಯರು ಮೊತ್ತಮೊದಲ ವೈಜ್ಞಾನಿಕ ಸರ್ವೇ, ‘ದಿ ಗ್ರೇಟ್​ ಟ್ರಿಗನಮಿಟ್ರಿಕಲ್​ ಸರ್ವೇ ಆಫ್​ ಇಂಡಿಯಾ’ ಎಂಬ ಮ್ಯಾಪಿಂಗ್​ ಮಹಾ ಕಾರ್ಯವನ್ನು ಪ್ರಾರಂಭಿಸಿದ ಸ್ಮರಣೀಯ ದಿನ.

ಸರ್ವೇ ಮತ್ತು ಮ್ಯಾಪಿಂಗ್​ ಕಾರ್ಯಾಚರಣೆಯ ಉಗಮ, ವಿಕಾಸ, ಇತಿಹಾಸ ಬಹು ಕುತೂಹಲಕಾರಿಯಾದದ್ದು. ಅದು ವಿಜ್ಞಾನದ ಇತಿಹಾಸದಲ್ಲಿ ಕುತೂಹಲ ಮತ್ತು ರೋಚಕವಾದ ಭಾಗವಾಗಿದೆ. ನಮ್ಮ ಭಾರತದಲ್ಲಿ ಬ್ರಿಟೀಷ್​ ಸರ್ವೇಯರುಗಳು ಈ ದೇಶದ ಬಗ್ಗೆ ಅರಿವನ್ನು ಹೆಚ್ಚಿಸುವ ಮ್ಯಾಪ್​ಗಳ ತಯಾರಿಕೆ ಮತ್ತು ಮ್ಯಾಪ್​ಗಳ ತಯಾರಿಕೆಗೆ ಪೂರ್ವಬಾವಿ ಕಾರ್ಯವಾದ ಸರ್ವೇಯಿಂಗ್​ ಕ್ಷೇತ್ರಗಳಲ್ಲಿ ಗಣನೀಯ ಕೊಡುಗೆಯನ್ನು ನೀಡಿದ್ದಾರೆ. ಅವರು ಭಾರತದ ಭೂಮಾಪನಾ ಜ್ಞಾನ ಶಾಖೆಯನ್ನು ಭದ್ರ ಬುನಾದಿಯ ಮೇಲೆ ನಿಲ್ಲಿಸಿದ್ದಾರೆ. ಈಸ್ಟ್​ ಇಂಡಿಯಾ ಕಂಪನಿಯ ಇಂಗ್ಲೀಷ್​ ಆಡಳಿತಗಾರಿಗೆ ಅವರ ಸಮರ ತಂತ್ರಕ್ಕೆ, ದೈನಂದಿನ ಆಡಳಿತಕ್ಕೆ, ನಿಖರ ಭೌಗೋಳಿಕ ಮಾಹಿತಿಯ ಅವಶ್ಯಕತೆ ಇತ್ತು. ಬ್ರಿಟೀಷರಿಗೆ ಯಾವಾಗಲೂ ಹಾಗೆ. ಅವರ ಸೈನಿಕ ಕಾರ್ಯಕ್ಕೆ, ಸೈನಿಕರಿಗಿಂತ ಮೊದಲು ಅಥವಾ ಸೈನಿಕರ ಜೊತೆಗೆ ಅಥವಾ ಸೈನಿಕರ ಆನಂತರ ಸರ್ವೇಯರುಗಳು ಇದ್ದೇ ಇರುತ್ತಿದ್ದರು. ಬ್ರಿಟಿಷರು ಯುದ್ಧಕ್ಕೆ ಹೊರಡಲಿ, ಆಡಳಿತಕ್ಕೆ ಕಾಲಿಡಲಿ, ಆದಾಯದ ಲೆಕ್ಕಾಚಾರವೇ ಮಾಡಲಿ, ಮೊದಲು ಅವರಿಗೆ ಮ್ಯಾಪ್​ ಬೇಕು. ಇದು ಬ್ರಿಟೀಷರ ನೀತಿ. ಬ್ರಿಟೀಷರ ಯುದ್ದದ ಜಯ, ಆಡಳಿತದ ಇತಿಹಾಸ ಅವರ ಸರ್ವೇಯ ಇತಿಹಾಸದ ಜೊತೆ ಜೊತೆಗೇ ಸೇರಿಕೊಂಡಿವೆ. ಅಜ್ಞಾತವಾಗಿದ್ದ ಭಾರತದ ಭೌಗೋಳಿಕ ರೂಪವನ್ನು ಕೆತ್ತಿ ಮ್ಯಾಪಿನಲ್ಲಿ ದೃಶ್ಯರೂಪದಲ್ಲಿ ಮೂಡಿಸಲು ಬಹು ಸಾಹಸದ ಗ್ರೇಟ್​ ಟ್ರಿಗನಮಿಟ್ರಿಕಲ್​ ಸರ್ವೇಯನ್ನು ಅವರು ಪ್ರಾರಂಭಿಸಿದರು. ಈ ಗ್ರೇಟ್​ ಟ್ರಿಗನಮಿಟ್ರಿಕಲ್​ ಸರ್ವೇಯು ವೈಜ್ಞಾನಿಕ ಮಹಾ ಕಾರ್ಯ ಆಗಿದೆ. ಗಾತ್ರ ಗುಣಗಳೆರಡರಲ್ಲೂ ಭೂವಿಜ್ಞಾನದ ಮತ್ತು ಮ್ಯಾಪಿಂಗ್​ ಕಾರ್ಯದ ಅಸಾಮಾನ್ಯ ಸಾಹಸ ಕಾರ್ಯವಾಗಿದೆ.

ಬ್ರಿಟಿಷರಿಗಿಂತ ಪೂರ್ವದಲ್ಲಿ ಮ್ಯಾಪಿಂಗ್​ ವಿಜ್ಞಾನದಲ್ಲಿ, ಭಾರತವು ಬಹುಪಾಲು ಅಜ್ಞಾತವಾಗಿಯೇ ಇದ್ದ ದೇಶ. ಆಗಿನ ಕಾಲ ಈಗಿನಂತಿರಲಿಲ್ಲ. ಜನ ನಕ್ಷೆಯ ಬಳಕೆಯನ್ನು ಕಂಡಿರಲಿಲ್ಲ, ಕೇಳಿರಲಿಲ್ಲ. ವ್ಯಾಪಾರಕ್ಕೆಂದು ಬಂದ ಬ್ರಿಟೀಷರು ಈ ದೇಶದ ಆಡಳಿತ ಕ್ಷೇತ್ರಕ್ಕೆ ಕಾಲಿಟ್ಟರು. ಅವರಿಗೆ ತಮ್ಮ ಆಡಳಿತದ ಎಲ್ಲೆಯನ್ನು ಗುರುತಿಸಿ, ಗೊತ್ತುಪಡಿಸಿ, ಅದನ್ನು ಭದ್ರವಾಗಿ ರಕ್ಷಿಸಿ, ಕಾಪಾಡಿಕೊಳ್ಳಬೇಕಿತ್ತು. ಜತೆಗೆ, ಅವರ ಭೂಪ್ರದೇಶದ ಆಡಳಿತವನ್ನು ಲಾಭಕರವಾಗಿ ಬಳಸಿಕೊಳ್ಳಲು ಅವರಿಗೆ ಸರ್ವೇಯು ಅನಿವಾರ್ಯ ಅವಶ್ಯಕತೆಯಾಗಿತ್ತು. ಮ್ಯಾಪುಗಳು ಭೂಮಾಹಿತಿಯ ಭಂಡಾರ. ‘ಮ್ಯಾಪೊಂದು ಸಾವಿರ ಮಾತಿಗಿಂತ ಮಿಗಿಲು’ ಎಂಬ ಮಾತಿದೆ. ಮ್ಯಾಪು ಭೌಗೋಳಿಕ ಸತ್ಯವನ್ನೆಲ್ಲಾ ಹಿಡಿದು ಕಣ್ಣ ಮುಂದೆ ಕಟ್ಟಿಕೊಡುವ ಸಾಧನ. ದೇಶ ಆಳಬೇಕಾದರೆ ಮೊದಲು ಅದನ್ನು ತಿಳಿದಿರಬೇಕೆಂಬುದು ಬ್ರಿಟೀಷರ ನೀತಿ. ಆದ್ದರಿಂದಲೇ ಭಾರತದ ಭೌಗೋಳಿಕ ಅರಿವನ್ನು ಪಡೆಯಲು, ಮೇಧಾವಿ ಬ್ರಿಟೀಷ್​ ಅಧಿಕಾರಿಗಳು ಮ್ಯಾಪಿಂಗ್​ ಮಹಾ ಕಾರ್ಯವನ್ನು ಇಲ್ಲಿ ಪ್ರಾರಂಭಿಸಿದರು. ಅಂದಿನ ಆ ಪ್ರಯತ್ನದ ಫಲವೇ ಇಂದು ನಮಗೆ, ನಮ್ಮ ದೇಶದ ಬಗ್ಗೆ ಅರಿವನ್ನು ಶ‍್ರೀಮಂತಗೊಳಿಸಿಕೊಳ್ಳಲು ಲಭ್ಯ ಇರುವ, ಭೂಮಾಹಿತಿ ವಿಷಯಗಳು. ಅಂದು ಅವರು ಮೊಟ್ಟ ಮೊದಲಿಗೆ ಕೈಗೊಂಡ, ಟ್ರಿಗನಮಿಟ್ರಿಕಲ್​ ಸರ್ವೇ, ಟೋಪೋಗ್ರಫಿಕಲ್​ ಸರ್ವೇ ಮತ್ತು ಕೆಡಸ್ಟ್ರಲ್​ ಸರ್ವೇಗಳು ನಮ್ಮ ದೇಶದ ಬಗ್ಗೆ ಸಾವಿರಾರು ಪುಟಗಳ ವೈಜ್ಞಾನಿಕ ಮಾಹಿತಿಯನ್ನು ಒದಗಿಸಿವೆ. ಇಂದಿಗೂ ನಮಗೆ ಆಯಾ ಕ್ಷೇತ್ರದ ಅಭಿವೃದ್ಧಿಗೆ ಆಧಾರವಾಗುತ್ತಿವೆ. ಮ್ಯಾಪಿಂಗ್​ಗಾಗಿಯೇ ಬ್ರಿಟೀಷರು ಭಾರತದಲ್ಲಿ ‘ಸರ್ವೇ ಆಫ್​ ಇಂಡಿಯಾ’ ಎಂಬ ಮ್ಯಾಪಿಂಗ್​ ಸಂಸ್ಥೆಯನ್ನು ಸ್ಥಾಪಿಸಿದರು. ಈ ಸಂಸ್ಥೆಯು ಇಲ್ಲಿಯವರೆಗೆ ಯಾರೂ ಮ್ಯಾಪು ಮಾಡಿರದ ಪ್ರದೇಶವನ್ನು ಮ್ಯಾಪು ಮಾಡಿದೆ. ಅಷ್ಟೇ ಅಲ್ಲ, ಯಾರೂ ನೋಡಿರದ ಪ್ರದೇಶವನ್ನು ಸಹ ಮ್ಯಾಪು ಮಾಡಿದೆ. ಬ್ರಿಟೀಷರು ಮ್ಯಾಪ್​ನಲ್ಲಿ ಬ್ರಿಟೀಷ್​ ಇಂಡಿಯಾದ ಚಿತ್ರಣ ಹಾಗೂ ವಿಸ್ತೀರ್ಣಗಳನ್ನು ಮೂಡಿಸಿರಬಹುದು. ಆದರೆ ಅದರಲ್ಲಿ ಬ್ರಿಟೀಷರ ಗುಣಸಾಮರ್ಥ್ಯ, ಕಾರ್ಯ ಕ್ಷಮತೆ, ಮ್ಯಾಪ್​ ಅವಲಂಬನಾ ನೀತಿ ಇವುಗಳ ಛಾಪು ಮೂಡಿದೆ. ಅಂದಿನ ಆ ವೈಜ್ಞಾನಿಕ ಸರ್ವೇ ಮತ್ತು ಮ್ಯಾಪಿಂಗ್​ ಮಹಾ ಕಾರ್ಯದ ಫಲ ಇಂದು ನಮಗೆ ಬಹು ದೊಡ್ಡ ವರದಾನವೇ ಆಗಿದೆ.

ಮ್ಯಾಪಿಂಗ್​ ವಿಜ್ಞಾನದಲ್ಲಿ ಬ್ರಿಟೀಷರ ಅರಿವು ಜಾಗತಿಕ ಮಟ್ಟದ್ದಾಗಿತ್ತು. ಏಕೆಂದರೆ ಅವರು ಅನೇಕರಂತೆ ಅಲ್ಲೇ ಹುಟ್ಟಿ, ಅಲ್ಲೇ ಬೆಳೆದು, ಅಲ್ಲಿಯೇ ತಮ್ಮ ಜೀವನ ಮುಗಿಸಿಕೊಂಡವರಲ್ಲ. ಅವರು ಸಾಹಸಿಗರು, ಅನೇಕ ದೇಶಗಳನ್ನು ನೋಡಿ, ಸುತ್ತಾಡಿ, ವಿಶ್ವಪರ್ಯಟನೆ\break ಮಾಡಿದಂತಹವರು. ಇದರಿಂದ ಬ್ರಿಟೀಷರಿಗೆ ಇಂಡಿಯಾ ಮಾತ್ರವಲ್ಲ, ವಿಸ್ತಾರ ಪ್ರಪಂಚದ ಪರಿಚಯ ಸಹಜವಾಗಿಯೇ ಇತ್ತು. ಪ್ರಪಂಚದ ಅನೇಕ ಕಡೆ ನಡೆಯುತ್ತಿದ್ದ ಮ್ಯಾಪಿಂಗ್​ ಕಾರ್ಯದ ವಿವಿಧ ವಿಧಾನಗಳ ಪರಿಕಲ್ಪನೆ ಅವರಿಗೆ ಸ್ಪಷ್ಟವಾಗಿ ಇತ್ತು. ಮ್ಯಾಪಿಂಗ್​ ಕಾರ್ಯಕ್ಕೆ ಬೇಕಾದ ಉನ್ನತ ಕಲ್ಪನೆ, ಆಧುನಿಕ ವಿಜ್ಞಾನ, ಸಲಕರಣೆ, ವಿಧಾನ, ಅದರ ಹೊಸ ಹೊಸ ಬೆಳವಣಿಗೆ ಇವುಗಳೆಲ್ಲದರ ಅರಿವು ಅಂದಿನ ಕಾಲಘಟ್ಟದಲ್ಲಿ ಬ್ರಿಟೀಷರಿಗೆ ಚೆನ್ನಾಗಿ ಇತ್ತು. ಆದರೆ ಅಂದಿನ ಭಾರತೀಯ ಸ್ಥಳೀಯ ರಾಜರುಗಳಿಗೆ ಇಂತಹ ಮ್ಯಾಪಿಂಗ್​ ಸಾಕ್ಷರತೆಯ ಕೊರತೆಯ ಜೊತೆಗೆ ಒಟ್ಟು ದೇಶದ, ಇಡೀ ಉಪಖಂಡದ ಕಲ್ಪನೆಯು ಸಹ ಕಷ್ಟವಾಗಿತ್ತು ಎನ್ನಬಹುದು.

ಬ್ರಿಟೀಷ್​ ಆಡಳಿತಗಾರರು ಭಾರತ ದೇಶದ ವಿಜ್ಞಾನದಲ್ಲಿ ಆಸಕ್ತರಾಗಿ, ಇಲ್ಲಿ ದೇಶದ ಉದ್ದಾರಕ್ಕಾಗಿ ಈ ಮ್ಯಾಪಿಂಗ್​ ಕಾರ್ಯವನ್ನು ಮಾಡಿದ್ದಲ್ಲ. ಅವರಿಗೆ ದೇಶವನ್ನಾಳಲು ಅಂಗೈ ಮೇಲೆ ಅವರ ಆಡಳಿತ ಕ್ಷೇತ್ರದ ಅರಿವನ್ನು ನೀಡುವ, ಆಡಳಿತವನ್ನು ನಿಯಂತ್ರಿಸುವ ಸಾಧನವಾದ ಮ್ಯಾಪು ಬೇಕಾಗಿತ್ತು. ಅದಕ್ಕಾಗಿ ಅವರು ಕರ್ನಲ್​ ಲ್ಯಾಂಬ್​ಟನ್​ರಂತಹ ಮ್ಯಾಪಿಂಗ್​ ಕ್ಷೇತ್ರದ ಅನೇಕ ಸಮರ್ಥ ವಿಜ್ಞಾನಿಗಳ ಪ್ರತಿಭೆಯನ್ನು ಮ್ಯಾಪಿಂಗಿಗಾಗಿ ಬಳಸಿಕೊಂಡರು. ಆ ವಿಜ್ಞಾನಿಗಳು ಲಭ್ಯ ವಿಜ್ಞಾನವನ್ನು ಬಳಸಿಕೊಂಡು ಪ್ರತಿಭೆ ಪರಿಶ್ರಮಗಳಿಂದ ಬ್ರಿಟೀಷ್​ ಹಿತಾಸಕ್ತಿಯನ್ನು ಬಲಗೊಳಿಸುವ, ಮ್ಯಾಪಿಂಗ್​ ಕಾರ್ಯ ಮಾಡಿದರು. ಬ್ರಿಟೀಷರಿಗೆ ವಾಸ್ತವದ ಅಗತ್ಯಕ್ಕಾಗಿ, ಇಲ್ಲಿ ಯಾವುದೇ ವಿಜ್ಞಾನ ಶಾಖೆಯ ಅಗತ್ಯಬಿದ್ದಾಗ, ಅದನ್ನು ಕಟ್ಟಿ ಬೆಳೆಸಿದ್ದಾರೆ. ಬ್ರಿಟೀಷ್​ ಸಾಮ್ರಾಜ್ಯವನ್ನು ಕಟ್ಟುತ್ತಲೇ, ಭಾರತವನ್ನು ಆಧುನಿಕ ವಿಜ್ಞಾನದ ತಳಹದಿಯ ಮೇಲೆ ತಂದು ಗಟ್ಟಿಯಾಗಿ ಕೂರಿಸಿದ್ದಾರೆ. ಅದರಲ್ಲಿ ಭೂಮಾಪನಾ ವಿಜ್ಞಾನವೂ ಸಹ ಒಂದು.

ನಾಲ್ಕನೇ ಮೈಸೂರು ಯುದ್ಧದಲ್ಲಿ, ಟಿಪ್ಪುರವರು ಸೋತು, ಶ‍್ರೀರಂಗಪಟ್ಟಣವು ಬ್ರಿಟೀಷರ ಕೈವಶವಾಗಿದ್ದು ಈಗ ಇತಿಹಾಸ. ಟಿಪ್ಪು ವಿರುದ್ಧದ ಈ ನಾಲ್ಕನೇ ಮೈಸೂರು ಯುದ್ದದ ಹೊಣೆ ಹೊತ್ತಿದ್ದವನು ಕರ್ನಲ್​ ಅರ್ಥರ್​ ವೆಲ್ಲೆಸ್ಲಿ. ಆಗ ಅರ್ಥರ್​ ವೆಲ್ಲೆಸ್ಲಿ \enginline{27}ರ ಹರೆಯದ ತರುಣ. ಈ ಯುದ್ದದ ಹೊಣೆ ಹೊತ್ತು ಹಡಗಿನಲ್ಲಿ ಇಂಗ್ಲೆಂಡಿನಿಂದ ಕೊಲ್ಕೊತ್ತಾಗೆ ಬಂದು, ಅಲ್ಲಿಂದ ಮೈಸೂರತ್ತ ಹೊರಟ ಸಮಯ. ಅದೇ ಸಮಯದಲ್ಲಿ ಈಸ್ಟ್​ ಇಂಡಿಯಾ ಸೈನ್ಯ ಸೇರಿ, ಕರ್ನಲ್​ ಅರ್ಥರ್​ ಇದ್ದ ಹಡಗಿನಲ್ಲಿ ಮದರಾಸಿನತ್ತ ಹೊರಟ ಮೇಜರ್​ ಲ್ಯಾಂಬ್​ಟನ್​. ಇವರಿಬ್ಬರ ಪ್ರಥಮ ಭೇಟಿ ಹಡಗಿನಲ್ಲೇ ಆಗಿ, ಪರಸ್ಪರ ಮಿತ್ರರಾದರು. ಲ್ಯಾಂಬ್​ಟನ್​ರವರು ಮಿಲಿಟರಿ ಇಂಜಿನಿಯರು. ಈ ಮೊದಲು ಅಮೇರಿಕಾ–ಕೆನಡಾ ಗಡಿ ಅಳತೆ ಕಾರ್ಯದಲ್ಲಿ ಪಾಲ್ಗೊಂಡಿದ್ದವರು. ಅಮೇರಿಕಾ ಸ್ವತಂತ್ರ ಸಂಗ್ರಾಮ (\enginline{1775–81})ದ ಅನಂತರ ಅಲ್ಲಿನ ಯುದ್ದ ನಿರಾಶ್ರಿತರಿಗೆ, ವಸತಿ ಕಾರ್ಯದ ಸರ್ವೇಯಲ್ಲಿ ಇದ್ದವರು. ಆದರೆ ಅಲ್ಲಿಯ ಆಗಿನ ಆರ್ಥಿಕ ಸಂಕಷ್ಟದ ಕಾರಣ ಕೆಲಸ ಕಳೆದುಕೊಂಡಿದ್ದರು. ಈ ಕಾರಣ ಲ್ಯಾಂಬ್​ಟನ್​ರವರು ಭಾರತದ ಈಸ್ಟ್​ ಇಂಡಿಯಾ ಕಂಪನಿ ಸೇರಿ ಭಾರತಕ್ಕೆ ಹೊರಟಿದ್ದರು. ಈ ಅರ್ಥರ್​ ವೆಲ್ಲೆಸ್ಲಿ, ಭಾರತದಲ್ಲಿ ಗವರ್ನರ್​ ಜನರಲ್​ ಆಗಿದ್ದ ರಿಚರ್ಡ್ ವೆಲ್ಲಸ್ಲಿಯವರ\break ಸಹೋದರ. ನಂತರ \enginline{1815} ರಲ್ಲಿ ಪ್ರಸಿದ್ದ ವಾಟರ್​ಲೂ ಕದನದಲ್ಲಿ ನೆಪೋಲಿಯನ್‌ನ್ನು ಸೋಲಿಸಿದವ. ಮುಂದೆ \enginline{1828–30} ರ ಅವಧಿಯಲ್ಲಿ ಬ್ರಿಟನ್ನಿನ ಪ್ರಧಾನಿಯೂ ಆದವನು.

ನಾಲ್ಕನೆ ಮೈಸೂರು ಯುದ್ಧ ನಡೆಯುತ್ತಿದ್ದ ಸಮಯ. ಬ್ರಿಟೀಷ್​ ಸೇನೆ ಶ‍್ರೀರಂಗಪಟ್ಟಣ ಕೋಟೆಯನ್ನು ಗುರಿಯಾಗಿಸಿಕೊಂಡು ನಾಲ್ಕೂ ದಿಕ್ಕಿನಿಂದ ಮುನ್ನುಗ್ಗುತ್ತಿದೆ. ಟಿಪ್ಪುವಿನ ಸೈನ್ಯ, ಕೋಟೆಯ ದಕ್ಷಿಣ ದಿಕ್ಕಿನ ಸುಲ್ತಾನ್​ಪೇಟೆ ತೋಪಿನ ಆಯಕಟ್ಟಿನ ಜಾಗ ಹಿಡಿದು ಬೀಡುಬಿಟ್ಟಿದೆ. ಬ್ರಿಟೀಷ್​ ಸೈನ್ಯಕ್ಕೆ ಕೋಟೆಯತ್ತ ಮುನ್ನುಗ್ಗಲು, ಸುಲ್ತಾನ್​ಪೇಟೆ ತೋಪನ್ನು ತೆರವುಗೊಳಿಸಬೇಕಾಗಿದೆ. ಈ ಕಾರ್ಯಕ್ಕೆ ಕರ್ನಲ್​ ವೆಲ್ಲೆಸ್ಲಿಗೆ ಸೂಚನೆ ಸಿಗುತ್ತದೆ. ಸುರಕ್ಷಿತ ಜಾಗ ಹಿಡಿದು, ಅಲ್ಲಿಂದ ಮಾರನೆ ದಿನ ಬೆಳಗ್ಗೆ ಮೈಸೂರು ಸೈನ್ಯದ ಮೇಲೆ ದಿಡೀರ್​ ದಾಳಿ ಮಾಡಬೇಕು. ಇದಕ್ಕಾಗಿ ಯೋಜಿತ ರೀತಿಯಲ್ಲಿ ರಾತ್ರಿ ಸಮಯದಲ್ಲಿ ಮುಂದುವರೆಯಬೇಕು. ಇದು ಯುದ್ಧ ತಂತ್ರ. ಆದರೆ ಕರ್ನಲ್​ ವೆಲ್ಲೆಸ್ಲಿಗೆ, ಸರಿಯಾದ ಮ್ಯಾಪು ಇಲ್ಲದೇ, ದಿಕ್ಕು ತಿಳಿಯದೇ ಸುರಕ್ಷಿತ ತಾಣವೆಂದು ರಾತ್ರಿ ಸಮಯ ಪ್ರಯಾಣ ಮಾಡಿದ್ದು ತಪ್ಪಾಗಿ ಹೋಗಿತ್ತು. ದಾರಿತಪ್ಪಿ, ಅಲ್ಲೇ ಬೀಡು ಬಿಟ್ಟಿದ್ದ ಟಿಪ್ಪು ಸೈನ್ಯದೊಳಗೇ ಹೋಗಿಬಿಟ್ಟಿದ್ದ. ಅನಾಹುತ ತಿಳಿದು, ತಕ್ಷಣ ಪಾರಾಗುವ ಅವಸರದಲ್ಲಿ ಕುದುರೆ ಏರಿ, ಆ ಗಡಿಬಿಡಿಯಲ್ಲಿ ಕೆಳಗೆ ಬಿದ್ದು, ಮೊಣಕಾಲಿಗೆ ಗಾಯ ಮಾಡಿಕೊಂಡು ಆ ಗೊಂದಲದಲ್ಲಿ ಟಿಪ್ಪು ಸೈನ್ಯಕ್ಕೆ ಸೆರೆ ಸಿಕ್ಕುವವನಿದ್ದ. ಆದರೆ, ಆ ಸಮಯದಲ್ಲಿ, ಅರ್ಥರ್​ನ ಜತೆಗಿದ್ದವರು, ಅನುಭವಿ ಮಿತ್ರ ಲ್ಯಾಂಬ್​ಟನ್​ರವರು. ತಿಳಿದವರಿಗೆ ರಾತ್ರಿ ಸಮಯದ ಆಕಾಶವು ಸಹ, ದೂರವನ್ನು, ದಿಕ್ಕನ್ನು ತಿಳಿಸುವ ವಿಶಾಲ ಮ್ಯಾಪು ಆಗಿರುತ್ತದೆ. ನಕ್ಷತ್ರ ವೀಕ್ಷಣೆ ಮಾಡಿ, ದಿಕ್ಕು ದೂರ ತಿಳಿಯುವ ಅನುಭವ ವಿದ್ಯೆ ಇದ್ದವರು\break ಲ್ಯಾಂಬ್​ಟನ್​ರವರು. ಇವರು ತಕ್ಷಣ ಅರ್ಥರ್​ನನ್ನು, ಅವನ ಪಡೆಯನ್ನು, ಉತ್ತರದ ಸುರಕ್ಷಿತ ದಿಕ್ಕಿಗೆ ತಿರುಗಿಸಿ, ರಕ್ಷಿಸಿದರು. ಸುಲ್ತಾನ್​ಪೇಟೆಯ ಈ ಘಟನೆಯು ಭವಿಷ್ಯದ ಮ್ಯಾಪಿಂಗ್​ ಕಾರ್ಯಕ್ಕೆ ಸಹಾಕಾರಿಯಾಗಿ ನಿಂತಿತು. ಅರ್ಥರ್​ರವರು ಅನಂತರದ ಬ್ರಿಟೀಷ್​ ಆಡಳಿತದಲ್ಲಿ ಭಾರೀ ಪ್ರಭಾವಿ ವ್ಯಕ್ತಿಯಾಗಿ ಹೊರಹೊಮ್ಮಿದರು. ಲ್ಯಾಂಬ್​ಟನ್​ರವರು ಆರಂಭಿಸಿದ ಭಾರತದ ಮ್ಯಾಪಿಂಗ್​ ಕಾರ್ಯಕ್ಕೆ ನಿರಂತರ ಬೆಂಬಲವಾಗಿ ನಿಂತರು.

ಕರ್ನಲ್​ ಲ್ಯಾಂಬಟನ್​ರವರ ಮ್ಯಾಪಿಂಗ್​ ಮಹಾ ಕಾರ್ಯದ ಚಿಂತನೆಯ ಬೀಜವು ಮೊದಲಿಗೆ ಭೂಮಿಗೆ ಬಿದ್ದು ಮೊಳಕೆಯೊಡೆದಿದ್ದು ಇಲ್ಲಿಯೇ ಕನ್ನಡ ನಾಡಿನಲ್ಲಿ. ಬ್ರಿಟೀಷರಿಗೆ ಭಾರತದಲ್ಲಿ, ಅವರ ರಾಜ್ಯ ವಿಸ್ತರಣೆಯ ದಾಹಕ್ಕೆ ದೊಡ್ಡ ಅಡ್ಡಿಯಾಗಿದ್ದುದು ಶ‍್ರೀರಂಗಪಟ್ಟಣ. ಬಲಿಷ್ಠ ಶ‍್ರೀರಂಗಪಟ್ಟಣವೇ ಪತನಹೊಂದಿದ ಅನಂತರ, ಇಡೀ ಭಾರತವನ್ನು ತನ್ನ ತೆಕ್ಕೆಗೆ ತೆಗೆದುಕೊಳ್ಳುವ ಅವರ ಕನಸಿಗೆ ಇನ್ಯಾವ ದೊಡ್ಡ ಅಡ್ಡಿ ಆತಂಕಗಳು ಕಾಣಲಿಲ್ಲ. ಆದ್ದರಿಂದ, ಈಸ್ಟ್​ ಇಂಡಿಯಾ ಕಂಪನಿಯು ಇತ್ತೀಚೆಗೆ ಗೆದ್ದ ಹೊಸ ಭೂಪ್ರದೇಶದ ಸರ್ವೇಗೆ ಬಾರೀ ಯೋಜನೆಯನ್ನು ಲಾರ್ಡ್ ವೆಲ್ಲೆಸ್ಲಿಯವರು ಆರಂಭಿಸಿದರು. ಡಾಕ್ಟರ್​ ಬುಖನಾನ್​ರವರನ್ನು ಮೈಸೂರು ಮತ್ತು ಮಲಬಾರ್​ ಪ್ರದೇಶದ ವ್ಯವಸಾಯೋತ್ಪನ್ನ ಕ್ಷೇತ್ರದ ಸರ್ವೇಗೆ\break ನೇಮಿಸಲಾಯಿತು. ಕರ್ನಲ್​ ಕೊಲಿನ್​ ಮೆಕೆಂಜಿರವರನ್ನು ಟೋಪೋಗ್ರಫಿಕಲ್​ ಸರ್ವೇಗೆ ನೇಮಿಸಲಾಯಿತು.

ಕರ್ನಲ್​ ಕೋಲಿನ್​ ಮೆಕೆಂಜಿರವರು (\enginline{1753–1821}) ಗಣಿತ ಶಾಸ್ರಕ್ಕೆ ಸಂಬಂಧಿಸಿದ ಮಾಹಿತಿ ಸಂಗ್ರಹಕ್ಕೆಂದು ಭಾರತಕ್ಕೆ ಬಂದು, ನಂತರ ಈಸ್ಟ್​ ಇಂಡಿಯಾ ಕಂಪನಿ ಸೇರಿ, ಕರ್ನಲ್​ ಆಗಿ ದುಡಿದವರು. ಇವರು ಸಹ ಮೂರನೇ ಮತ್ತು ನಾಲ್ಕನೇ ಮೈಸೂರು ಯುದ್ದದಲ್ಲಿ ಪಾಲ್ಗೊಂಡಿದ್ದವರು. ಕರ್ನಲ್​ ಮೆಕೆಂಜಿಯವರು \enginline{1799–1809}ರ ಅವಧಿಯಲ್ಲಿ ಮೈಸೂರು ಪ್ರಾಂತ್ಯದ ಮೊದಲ ಮ್ಯಾಪನ್ನು ರಚಿಸಿದರು. ಇವರ ಸರ್ವೇಯು ಆಗಿನ ಮೈಸೂರು ರಾಜ್ಯದ ಗಡಿಯನ್ನು ನಿಖರವಾಗಿ ಗುರ್ತಿಸಿದೆ. ಜೊತೆಗೆ ಪಟ್ಟಣ, ಕೋಟೆ, ಹಳ್ಳ, ನದಿ, ರಸ್ತೆ, ಪರ್ವತ ಇತರೆ ಪ್ರಮುಖ ವಿವರಗಳ ಸ್ಥಾನಗಳನ್ನು ನಿಖರವಾಗಿ ಸೂಚಿಸಿದೆ. ಇವರ ಈ ಸರ್ವೇಯಿಂಗ್​ ಮತ್ತು ಮ್ಯಾಪಿಂಗ್​ ಕಾರ್ಯವು, ಆಗಿನ ಕಾಲದ ಮ್ಯಾಪಿಂಗ್​ ಕಾರ್ಯದ ನಿಖರತೆಗೆ ಒಂದು ಮಾದರಿಯೇ ಆಗಿತ್ತು. ಮೆಕೆಂಜಿರವರ ಪ್ರಯತ್ನದಿಂದ ಆಗಿನ ಮೈಸೂರು ರಾಜ್ಯವು ದೇಶದಲ್ಲೇ ಅತ್ಯುತ್ತಮ ನಕಾಶೆ ಹೊಂದಿದ ಪ್ರಾಂತ್ಯವೆಂಬ ಹೆಗ್ಗಳಿಕೆ ಪಡೆಯುವಂತಾಯಿತು. ಅನಂತರ ಇವರು \enginline{1815–1821}ರ ಅವಧಿಯಲ್ಲಿ, ಏಕೀಕೃತ ಸರ್ವೆ ಆಫ್​ ಇಂಡಿಯಾದ ಮೊದಲ ಸರ್ವೆಯರ್​ ಜನರಲ್​ ಆಗುತ್ತಾರೆ.

ಯುದ್ದ ಕ್ಷೇತ್ರ ಮತ್ತು ಮ್ಯಾಪಿಂಗ್​ ಕ್ಷೇತ್ರಗಳ ಜೊತೆಗೆ ಕರ್ನಲ್​ ಮೆಕೆಂಜಿಯವರು ಕನ್ನಡದ ಪ್ರಸಿದ್ದ ಜಾನಪದ ವಿದ್ವಾಂಸರಾಗಿದ್ದವರು. ದಕ್ಷಿಣ ಭಾರತ ಜಾನಪದ ಸಾಹಿತ್ಯ ಕ್ಷೇತ್ರದಲ್ಲಿ ಕೆಲಸ ಮಾಡಿದ ಮೊದಲ ವಿದೇಶಿ ವಿದ್ವಾಂಸ ಎಂದೇ ಹೆಸರಾದವರು ಇವರು. ಪುರಾತತ್ವ ಕ್ಷೇತ್ರದಲ್ಲಿಯೂ ಇವರಿಗೆ ಬಹು ಆಸಕ್ತಿ. ಮೈಸೂರು ಸರ್ವೇಯ ಕಾಲದಲ್ಲಿ ಮತ್ತು ಅವರ ಇತರ ಪ್ರವಾಸ ಕಾರ್ಯದಲ್ಲಿ ನೂರಾರು ಪ್ರಾಚೀನ ಹಸ್ತ ಪ್ರತಿಗಳನ್ನು, ನಾಣ್ಯಗಳನ್ನು, ಶಾಸನಗಳನ್ನು ಸಂಗ್ರಹಿಸಿದ್ದಾರೆ. ಈ ಕಾರಣಕ್ಕೆ ಕನ್ನಡ ಸಾಹಿತ್ಯ ಲೋಕದಲ್ಲೂ ಇವರು ಬಹು ಪ್ರಸಿದ್ಧರು ಎನ್ನುವುದು ವಿಶೇಷ.

ಕಂಪನಿ ಸರಕಾರವು ಬ್ರಿಟೀಷ್​ ಭಾರತದ ಸರ್ವೇಗೆ ಭಾರೀ ಯೋಜನೆಯನ್ನು ಕೈಗೊಂಡಾಗ, ನಾಲ್ಕನೇ ಮೈಸೂರು ಯುದ್ಧದಲ್ಲಿ ಪಾಲ್ಗೊಂಡಿದ್ದ ಲ್ಯಾಂಬ್​ಟನ್​ರವರು \enginline{1799}ರ ನವೆಂಬರ್​\break ನಲ್ಲಿ ತಮ್ಮ ‘ಮೆರಿಡಿಯನಲ್​ ಆರ್ಕ್ ಮೆಜರ್​ಮೆಂಟ್​ ಮತ್ತು ಟ್ರಿಗನಮಿಟ್ರಿಕಲ್​ ಸರ್ವೇ ಯೋಜನೆ’ ಗಾಗಿ ಪ್ರಸ್ತಾವನೆಯನ್ನು ಆಗಿನ ಮದರಾಸ್​ ಸರಕಾರಕ್ಕೆ ಸಲ್ಲಿಸಿದರು. ಕರ್ನಲ್\break ಲ್ಯಾಂಬ್​ಟನ್​ರವರನ್ನು ಭಾರತದ ಟ್ರಿಗನಮಿಟ್ರಿಕಲ್​ ಸರ್ವೇಯ ಮೂಲಪುರುಷ ಎನ್ನಬಹುದು. ಮಹಾ ಮೇದಾವಿಯಾದ ಅವರು ಅಸಾಧಾರಣ ಪ್ರತಿಭೆಯ ಸ್ವಯಂ ನಿರ್ಮಿತ ವ್ಯಕ್ತಿಯಾಗಿದ್ದರು. ಅವರ ಮಹದಾಸೆಯ ಮ್ಯಾಪಿಂಗ್​ ಪರಿಕಲ್ಪನೆಯೇ ‘ಗ್ರೇಟ್​ ಆರ್ಕ್’ ಮೆಸರ್​\break ಮೆಂಟ್​ ಕಾರ್ಯ. ಭಾರತದಲ್ಲಿನ ಈ ಟ್ರಿಗನಮಿಟ್ರಿಕಲ್​ ಸರ್ವೇಯು ಲ್ಯಾಂಬಟನ್​ರವರ ಚಿಂತನೆಯಲ್ಲಿ ಅರಳಿದ ವಿಜ್ಞಾನದ ಮಹಾಕಾರ್ಯ. ಲ್ಯಾಂಬ್​ಟನ್​ರವರ ಈ ಉದ್ದೇಶಿತ ಸರ್ವೆಯು ವೈಜ್ಞಾನಿಕ ಮತ್ತು ಮ್ಯಾಪಿಂಗ್​ ಈ ಎರಡೂ ಕ್ಷೇತ್ರಗಳಿಗೆ ಬಹು ಮಹತ್ವದ್ದಾಗಿತ್ತು. ಈ ಯೋಜನೆಯ ಪ್ರಾಯೋಗಿಕ ಸಂಶೋಧನಾ ಕಾರ್ಯದಿಂದ ಸಿಗುವ ನೂತನ ಫಲಿತಾಂಶವು, ಭೂಮಿಯ ನಿಖರ ಗಾತ್ರದ ಸ್ಪಷ್ಟ ಚಿತ್ರಣವನ್ನು ನೀಡುವಂತಹದ್ದು. ಹಾಗು ಆವರೆಗೆ ಚಾಲ್ತಿಯಲ್ಲಿದ್ದ ಅನೇಕ ಸಮುದ್ರ ಯಾನದ ಕೋಷ್ಠಕಗಳನ್ನು ತಿದ್ದುಪಡಿ ಮಾಡುವಂತಹದ್ದಾಗಿತ್ತು. ಆದಕ್ಕಿಂತ ಮುಖ್ಯವಾಗಿ, ಈ ಸರ್ವೇ ಕಾರ್ಯದಲ್ಲಿ ಟ್ರೈಯಾಂಗ್ಯುಲೇಷನ್​ನ ಬಿಂದುಗಳು ನಿಖರವಾಗಿ ನಿರ್ಧರಿಸಲ್ಪಡುತ್ತವೆ. ಈ ಬಿಂದುಗಳು, ಟೋಪೋಗ್ರಫಿಕಲ್​ ಮತ್ತು ಕೆಡಸ್ಟ್ರಲ್​ ಮ್ಯಾಪಿಂಗ್​ ಕಾರ್ಯಕ್ಕೆ ಅಗತ್ಯವಿರುವ ನಿಯಂತ್ರಣ ಬಿಂದುಗಳ ಭದ್ರ ಬುನಾದಿಯನ್ನು, ಗಣಿತಬದ್ಧ ಸುಸ್ಥಿರ ಚೌಕಟ್ಟನ್ನು ನೀಡುತ್ತವೆ.

