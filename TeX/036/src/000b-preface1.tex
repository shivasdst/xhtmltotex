
\chapter*{ಲೇಖಕನ ಮಾತು}

ಮ್ಯಾಪು, ಭೂಮಾಹಿತಿಯ ಭಂಡಾರ. ನಿಜಕ್ಕೆ ಹಿಡಿದ ಕನ್ನಡಿ. ಅದು ಭೌಗೋಳಿಕ ಸತ್ಯಾಂಶಗಳನ್ನು ಹಿಡಿದು ಕಣ್ಣಿನ ಮುಂದೆ ಕಟ್ಟಿ ಕೊಡುವ ಸಾಧನ. ಅದು ಎಲ್ಲರಿಗೂ ಅರ್ಥವಾಗುವ ದೃಶ್ಯ ಭಾಷೆ. ಮ್ಯಾಪು ಬೇಕೆಂದರೆ ಸರ್ವೇ ಆಗಬೇಕು. ವೀಕ್ಷಣೆ ಮತ್ತು ಅಳತೆ ಇವುಗಳಿಂದ ದೊರಕಿದ ಮಾಹಿತಿಯೇ ಮ್ಯಾಪಿಗೆ ಆಧಾರ. ಅಂದರೆ, ಸರ್ವೇಯ ಉತ್ಪನ್ನವೇ ಮ್ಯಾಪು. ಮ್ಯಾಪಿಂಗ್​ ಮತ್ತು ಸರ್ವೇಯಿಂಗ್​ ಚಟುವಟಿಕೆ ಇರದ ಕಾಲಘಟ್ಟದಲ್ಲಿ ಭಾರತದಲ್ಲಿ ಬ್ರಿಟೀಷರ ಆಗಮನವಾಯಿತು. ಬ್ರಿಟೀಷರಿಗೆ ಯಾವಾಗಲೂ ಅವರ ಕೈಯಲ್ಲಿ ಮ್ಯಾಪು ಇರಲೇಬೇಕು. ಇದು ಅವರ ನೀತಿ. ವ್ಯಾಪಾರಕ್ಕೆಂದು ಭಾರತಕ್ಕೆ ಬಂದವರು ಅವರು. ನಂತರ ಆಡಳಿತ ಕ್ಷೇತ್ರಕ್ಕೆ ಕಾಲಿಟ್ಟವರು. ಅವರ ರಾಜ್ಯ ವಿಸ್ತರಣೆಗೊಂಡಂತೆ, ಅವರಿಗೆ ಮ್ಯಾಪಿಂಗ್​ ಅಭಾವ ಎದುರಾಯಿತು. ಸರಿ, ಮ್ಯಾಪಿಂಗ್​ ಕಾರ್ಯವನ್ನು ಕಂಪನಿ ಸರ್ಕಾರ ಆರಂಭಿಸಿತು.

ಸರ್ವೇಯಲ್ಲಿ ಅನೇಕ ಕ್ಷೇತ್ರಗಳಿವೆ. ಸರ್ವೇ ಎಂಬುದು ಬಹು ವಿಸ್ತೃತ ಕ್ಷೇತ್ರ. ಆರ್ಥಿಕ ಆಡಳಿತ ಮತ್ತು ರೆವಿನ್ಯೂ ಆಡಳಿತ ಇವುಗಳಿಗೆ ತಳಹದಿ ರೆವಿನ್ಯೂ ಸರ್ವೇ. ಇಡೀ ಪ್ರದೇಶದ ಸ್ಥಳ ಸ್ವರೂಪವನ್ನು ಒಂದೇ ನೋಟದಲ್ಲಿ ಕಣ್ಮುಂದೆ ನೀಡುವುದು ಟೋಪೋಗ್ರಫಿಕಲ್​ ಸರ್ವೇ. ಈ ಸರ್ವೇಗಳಿಗಾಗಿ ಕಂಪನಿ ಸರ್ಕಾರವು ಯೋಜನೆಯನ್ನು ರೂಪಿಸಿತು. ಯಾವುದೇ ಸರ್ವೇಗೆ ಮೊದಲು ಫ್ರೇಮ್ ವರ್ಕ್ ರಚಿಸಿಕೊಳ್ಳಬೇಕು. ನಂತರ ಆ ಫ್ರೇಮ್ವರ್ಕಿನೊಳಗೆ ಟೋಪೋಗ್ರಫಿಕಲ್​ ವಿವರ ಮತ್ತು ರೆವಿನ್ಯೂ ವಿವರಗಳನ್ನು ಸರ್ವೇ ಮಾಡಬೇಕು. ಸರ್ವೇಯಲ್ಲಿ ಇದು ಮುಖ್ಯ ಮೂಲ ತತ್ವ. ಮೊದಲು ಸರ್ವೇಗೆ ಅಗತ್ಯವಾದ ಗಟ್ಟಿ ಚೌಕಟ್ಟನ್ನು ಒದಗಿಸಬೇಕು. \enginline{1802} ರಲ್ಲಿ ಕರ್ನಲ್​ ವಿಲಿಯಮ್ ಲ್ಯಾಂಬಟನ್​ ಎಂಬ ಮಿಲಿಟರಿ ಇಂಜಿನಿಯರ್​ ಭಾರತದ ಟ್ರಿಗನಮಿಟ್ರಿಕಲ್​ ಸರ್ವೇಯನ್ನು ಆರಂಭಿಸಿದರು. ಮ್ಯಾಪಿಂಗ್​ ಕಾರ್ಯದ ನಿಖರತೆಯ ಬೇರು ಇರುವುದೇ ಟ್ರಿಗನಮಿಟ್ರಿಕ್​ ಸರ್ವೇಯಲ್ಲಿ. ಇದು ಲ್ಯಾಂಬಟನ್​ರವರು ಆಲೋಚಿಸಿದ ಮ್ಯಾಪಿಂಗ್​ ರೀತಿ.

ಟ್ರಿಗನಮಿಟ್ರಿಕಲ್​ ಸರ್ವೇಯು ಒಂದು ವೈಜ್ಞಾನಿಕ ಮಹಾ ಕಾರ್ಯವಾಗಿದೆ. ಈ ವೈಜ್ಞಾನಿಕ ಸರ್ವೇ ಮಹಾ ಕಾರ್ಯವನ್ನು ಆರಂಭಿಸಿದ ಲ್ಯಾಂಬಟನ್​ರವರ ಮನದಲ್ಲಿ ಇದ್ದದ್ದು ಎರಡು ಪ್ರಮುಖ ಉದ್ದೇಶಗಳು. ಒಂದು, ಭೂಮಿಯ ಮೇಲೆ ಸರಪಳಿ ಚೆಲ್ಲಿ, ಅಳತೆ ಮಾಡಿ, ಬಂದ ಫಲಿತಾಂಶದಿಂದ ಭೂಮಿಯ ನಿಖರ ಗಾತ್ರವನ್ನು ನಿರೂಪಿಸುವುದು. ಎರಡನೇ ಉದ್ದೇಶ, ಈ ಸಂದರ್ಭದಲ್ಲಿ ರಚಿತವಾಗುವ ಭೌಗೋಳಿಕ ಬಿಂದುಗಳನ್ನು ಮತ್ತು ತ್ರಿಭುಜಗಳ ಜಾಲವನ್ನು ನಿಖರ ಮ್ಯಾಪ್​ ತಯಾರಿಕೆಯ ಫ್ರೇಮ್ವರ್ಕ್ ಆಗಿ ಬಳಸಿಕೊಳ್ಳುವುದು.

ಟ್ರಿಗನಮಿಟ್ರಿಕಲ್​ ಸರ್ವೇಯಲ್ಲಿ ತ್ರಿಭುಜಗಳು ರಚನೆಯಾಗುತ್ತವೆ. ದಕ್ಷಿಣದ ಕನ್ಯಾಕುಮಾರಿಯಿಂದ ಉತ್ತರದ ಮಸೂರಿ ವರೆಗೆ, ದಕ್ಷಿಣೋತ್ತರವಾಗಿ ರಚಿತವಾದ ಪ್ರಮುಖವಾದ ತ್ರಿಭುಜಗಳ ಸರಣಿಯ ರೇಖಾಂಶ ನೇರವೇ ‘ಗ್ರೇಟ್​ ಆರ್ಕ್’. ಈ ಗ್ರೇಟ್​ ಆರ್ಕ್ ಅಳತೆಯ ಪ್ರಾಯೋಗಿಕ ಫಲಿತಾಂಶವು ಭೂಮಿಯ ನಿಖರ ಗಾತ್ರವನ್ನು ಗಣಿತಾತ್ಮಕವಾಗಿ ನೀಡಿದೆ. ಈ ಗ್ರೇಟ್​ ಆರ್ಕ್ ಸರಣಿಯನ್ನು ಆಧರಿಸಿ, ಇಕ್ಕೆಲಗಳಲ್ಲೂ ದೇಶದ ಉದ್ದಗಲಕ್ಕೂ ರಚಿತವಾಗಿರುವ ತ್ರಿಭುಜಗಳ ಜಾಲವು ಟೋಪೋಗ್ರಫಿಕಲ್​ ಮತ್ತು ರೆವಿನ್ಯೂ ಸರ್ವೇಗಳಿಗೆ ಗಟ್ಟಿ ಆಧಾರವನ್ನು ಒದಗಿಸಿದೆ. ಹಿಮಾಲಯದ ಎತ್ತರವನ್ನೂ ನೀಡಿದೆ. ಗ್ರೇಟ್​ ಆರ್ಕ್ ಮತ್ತು ತ್ರಿಭುಜಗಳ ಜಾಲ ರಚನಾ ಕಾರ್ಯಕ್ಕೆ ಜೀಯೋಡೆಸಿ, ಅಸ್ಟ್ರನಮಿ, ಸರ್ವೇಯಿಂಗ್​ ಮತ್ತು ಮೆಥಮೆಟಿಕ್ಸ್​ ಇವುಗಳ ಆಳ ಅನ್ವಯಿಕ ಜ್ಞಾನ ಅಗತ್ಯ. ಈ ವೈಜ್ಞಾನಿಕ ಪ್ರಯೋಗ ಕಾರ್ಯವು ನಡೆದದ್ದು ನಾಲ್ಕು ಗೋಡೆಗಳ ನಡುವಿನ ಪ್ರಯೋಗಾಲಯದಲ್ಲಿ ಅಲ್ಲ. ಆ ಕಾರ್ಯಾಚರಣೆ ನಡೆದದ್ದು ಹೊರಗಿನ ತೆರೆದ ವಿಶಾಲ ಬಯಲಿನಲ್ಲಿ. ದಟ್ಟ ಕಾಡಿನಲ್ಲಿ. ಸುರಿಯುವ ಮಳೆಯಲ್ಲಿ. ಸುಡುವ ಬಿಸಿಲಿನಲ್ಲಿ. ಆದ್ದರಿಂದ ಈ ವೈಜ್ಞಾನಿಕ ಮಹಾ ಕಾರ್ಯದಲ್ಲಿ ಆದ ಕಷ್ಟ ನಷ್ಟ, ಜೀವ ಹಾನಿ ಅಪಾರ.

ಭಾರತದಲ್ಲಿ ಟ್ರಿಗನಮಿಟ್ರಿಕಲ್​ ಸರ್ವೇಯು ಬೆಳೆದು ಬಂದ ಕಥನವನ್ನು ನೀಡುವುದು ಈ ಕಿರು ಹೊತ್ತಿಗೆಯ ಉದ್ದೇಶ. ಈ ನಿಖರ ಗಣಿತ ಸರ್ವೇಯಲ್ಲಿ ದುಡಿದವರ ಪ್ರತಿಭೆ, ಪರಿಶ್ರಮ, ತ್ಯಾಗ, ಸಾಹಸದ ವಿವರಗಳನ್ನು ಕನ್ನಡದಲ್ಲಿ ದಾಖಲಿಸುವ ಪುಟ್ಟ ಪ್ರಯತ್ನ ಮಾಡಿದ್ದೇನೆ. ಸಾವಿರಾರು ಮಂದಿ ಏಕ ಕಾಲದಲ್ಲಿ ದುಡಿದ, ಬಹು ದೊಡ್ಡ ಗಾತ್ರದ ವಿಜ್ಞಾನದ ಮಹಾ ಕಾರ್ಯ ಇದು. ಧೀರ್ಘ ಕಾಲದವರೆಗೆ, ಅರ್ಧ ಶತಮಾನಕ್ಕೂ ಮೀರಿ, ಸಾಗಿದ ಸರ್ವೇಯ ಮಹಾ ಯಾತ್ರೆ ಇದು. ಭಾರತದಲ್ಲಿ ನಡೆದ ಈ ಟ್ರಿಗನಮಿಟ್ರಿಕಲ್​ ಸರ್ವೇ ಕಾರ್ಯಾಚರಣೆಯಲ್ಲಿ ಆಸಕ್ತಿ ಕುತೂಹಲವುಳ್ಳವರಿಗೆ, ಈ ಕಿರು ಹೊತ್ತಿಗೆ ಇಷ್ಟವಾಗಬಹುದೆಂದು ನನ್ನ ಆಶಯ.

\begin{flushright}
\textbf{ ಸರಡವಳ್ಳಿ. ಜಿ. ರಮೇಶ}
\end{flushright}

