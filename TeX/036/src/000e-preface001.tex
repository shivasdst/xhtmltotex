
\chapter*{ಲೇಖಕರ ಮಾತು}

ಮ್ಯಾಪು, ಭೂಮಾಹಿತಿಯ ಭಂಡಾರ. ನಿಜಕ್ಕೆ ಹಿಡಿದ ಕನ್ನಡಿ. ಅದು ಭೌಗೋಳಿಕ ಸತ್ಯಾಂಶಗಳನ್ನೆಲ್ಲಾ ಹಿಡಿದು ಕಣ್ಣಿನ ಮುಂದೆ ಕಟ್ಟಿಕೊಡುವ ಸಾಧನ. ಅದು ಎಲ್ಲರಿಗೂ ಅರ್ಥವಾಗುವ ದೃಶ್ಯ ಭಾಷೆ. ಮ್ಯಾಪು ಬೇಕೆಂದರೆ ಸರ್ವೇ ಆಗಬೇಕು. ವೀಕ್ಷಣೆ ಮತ್ತು ಅಳತೆ ಇವುಗಳಿಂದ ದೊರಕಿದ ಮಾಹಿತಿಯೇ ಮ್ಯಾಪಿಗೆ ಆಧಾರ. ಅಂದರೆ, ಸರ್ವೇಯ ಉತ್ಪನ್ನವೇ ಮ್ಯಾಪು.

ಮ್ಯಾಪಿಂಗ್​ ಮತ್ತು ಸರ್ವೇಯಿಂಗ್​ ಚಟುವಟಿಕೆ ಇರದ ಕಾಲಘಟ್ಟದಲ್ಲಿ ಭಾರತದಲ್ಲಿ ಬ್ರಿಟೀಷರ ಆಗಮನವಾಯಿತು. ಬ್ರಿಟೀಷರಿಗೆ ಯಾವಾಗಲೂ ಅವರ ಕೈಯಲ್ಲಿ ಮ್ಯಾಪು ಇರಲೇಬೇಕು. ಇದು ಅವರ ಸೈನಿಕ ಮತ್ತು ಆಡಳಿತ ನೀತಿ. ಅವರ ರಾಜ್ಯ ವಿಸ್ತರಣೆಗೊಂಡಂತೆ, ಅವರಿಗೆ ಮ್ಯಾಪಿಂಗ್​ ಕೊರತೆ ಎದುರಾಯಿತು. ಸರಿ, ಮ್ಯಾಪಿಂಗ್​ಕಾರ್ಯವನ್ನು ಕಂಪನಿ ಸರ್ಕಾರ ಆರಂಭಿಸಿತು.

ಸರ್ವೇಯಲ್ಲಿ ಅನೇಕ ಕ್ಷೇತ್ರಗಳಿವೆ. ಸರ್ವೇ ಎಂಬುದು ಬಹು ವಿಸ್ತೃತ ಕ್ಷೇತ್ರ. ಆರ್ಥಿಕ ಆಡಳಿತ ಮತ್ತು ರೆವಿನ್ಯೂ ಆಡಳಿತ ಇವುಗಳಿಗೆ ತಳಹದಿ ರೆವಿನ್ಯೂ ಸರ್ವೇ. ಇಡೀ ಪ್ರದೇಶದ ಸ್ಥಳ ಸ್ವರೂಪವನ್ನು ಒಂದೇ ನೋಟದಲ್ಲಿ ಕಣ್ಮುಂದೆ ನೀಡುವುದು ಟೋಪೋಗ್ರಫಿಕಲ್​ ಸರ್ವೇ. ಈ ಸರ್ವೇಗಳಿಗಾಗಿ ಕಂಪನಿ ಸರ್ಕಾರವು ಯೋಜನೆಯನ್ನು ರೂಪಿಸಿತು. ಯಾವುದೇ ಸರ್ವೇಗೆ ಮೊದಲು ಫ್ರೇಮ್‌ವರ್ಕ್ ರಚಿಸಿಕೊಳ್ಳಬೇಕು. ನಂತರ ಆ ಫ್ರೇಮ್‌ವರ್ಕಿನೊಳಗೆ ಟೋಪೋಗ್ರಫಿಕಲ್​ ವಿವರ ಮತ್ತು ರೆವಿನ್ಯೂ ವಿವರಗಳನ್ನು ಸರ್ವೇ ಮಾಡಬೇಕು. ಸರ್ವೇಯಲ್ಲಿ ಇದು ಮುಖ್ಯ ಮೂಲ ತತ್ವ. ಸರ್ವೇಯ ಈ ಮೂಲ ತತ್ವದಂತೆ ಸರ್ವೇಗೆ ಮೊದಲು ಅಗತ್ಯವಾದ ಗಟ್ಟಿ ಚೌಕಟ್ಟನ್ನು ರಚಿಸಿಕೊಳ್ಳಲು \enginline{1802} ರಲ್ಲಿ ಕರ್ನಲ್​ ವಿಲಿಯಮ್ ಲ್ಯಾಂಬಟನ್​ ಎಂಬ ಮಿಲಿಟರಿ ಇಂಜಿನಿಯರು ಭಾರತದ ಟ್ರಿಗನಮಿಟ್ರಿಕಲ್​ ಸರ್ವೇಯನ್ನು ಆರಂಭಿಸಿದರು.

ಈ ವೈಜ್ಞಾನಿಕ ಸರ್ವೇ ಕಾರ್ಯವನ್ನು ಆರಂಭಿಸಿದ ಲ್ಯಾಂಬಟನ್​ರವರ ಮನದಲ್ಲಿ ಇದ್ದದ್ದು ಎರಡು ಪ್ರಮುಖ ಉದ್ದೇಶಗಳು. ಒಂದು, ಭೂಮಿಯ ಮೇಲೆ ಸರಪಳಿ ಚೆಲ್ಲಿ, ಅಳತೆ ಮಾಡಿ, ಬಂದ ಫಲಿತಾಂಶದಿಂದ ಭೂಮಿಯ ನಿಖರಗಾತ್ರವನ್ನು ನಿರೂಪಿಸುವುದು. ಎರಡನೇ ಉದ್ದೇಶ, ಈ ಸಂದರ್ಭದಲ್ಲಿ ರಚಿತವಾಗುವ ಭೌಗೋಳಿಕ ಬಿಂದುಗಳನ್ನು ಮತ್ತು ತ್ರಿಭುಜಗಳ ಜಾಲವನ್ನು ನಿಖರ ಮ್ಯಾಪು ತಯಾರಿಕೆಯ ಫ್ರೇಮ್‌ವರ್ಕ್ ಆಗಿ ಬಳಸಿಕೊಳ್ಳುವುದು.

ಈ ಟ್ರಿಗನಮಿಟ್ರಿಕಲ್​ ಸರ್ವೇಯಲ್ಲಿ ದಕ್ಷಿಣದ ಕನ್ಯಾಕುಮಾರಿಯಿಂದ ಉತ್ತರದ\break ಮಸೂರಿವರೆಗೆ, ರೇಖಾಂಶ ನೇರದಲ್ಲಿ ದಕ್ಷಿಣೋತ್ತರವಾಗಿ ರಚಿತವಾದ ತ್ರಿಭುಜಗಳ ಸರಣಿಯೇ ‘ಗ್ರೇಟ್​ಆರ್ಕ್’. ಈ ಗ್ರೇಟ್​ಆರ್ಕ್ ಅಳತೆಯ ಪ್ರಾಯೋಗಿಕ ಫಲಿತಾಂಶವು ಭೂಮಿಯ ನಿಖರ\-ಗಾತ್ರವನ್ನು ಗಣಿತಾತ್ಮಕವಾಗಿ ನೀಡಿದೆ. ಈ ಗ್ರೇಟ್​ಆರ್ಕ್ ಸರಣಿಯನ್ನು ಆಧರಿಸಿ, ಇಕ್ಕೆಲಗಳಲ್ಲೂ ದೇಶದ ಉದ್ದಗಲಕ್ಕೂ ರಚಿತವಾಗಿರುವ ತ್ರಿಭುಜಗಳ ಜಾಲವು ಟೋಪೋಗ್ರಫಿಕಲ್​ ಮತ್ತು ರೆವಿನ್ಯೂ ಸರ್ವೇಗಳಿಗೆ ಗಟ್ಟಿ ಆಧಾರವನ್ನು ಒದಗಿಸಿದೆ. ಹಿಮಾಲಯದ ಎತ್ತರವನ್ನೂ ನೀಡಿದೆ. ಗ್ರೇಟ್​ಆರ್ಕ್ ಮತ್ತು ತ್ರಿಭುಜಗಳ ಜಾಲ ರಚನಾಕಾರ್ಯಕ್ಕೆ ಜೀಯೋಡೆಸಿ, ಅಸ್ಟ್ರನಮಿ, ಸರ್ವೇಯಿಂಗ್​ ಮತ್ತು ಮೆಥಮೆಟಿಕ್ಸ್​ ಇವುಗಳ ಆಳ ಅನ್ವಯಿಕಜ್ಞಾನ ಅಗತ್ಯವಿದೆ. ಈ ಕಾರಣಕ್ಕೆ ಟ್ರಿಗನಮಿಟ್ರಿಕಲ್​ ಸರ್ವೇಯು ಒಂದು ವೈಜ್ಞಾನಿಕ ಮಹಾ ಕಾರ್ಯವಾಗಿದೆ.

ಈ ವೈಜ್ಞಾನಿಕ ಮಹಾ ಕಾರ್ಯವು ನಡೆದದ್ದು ನಾಲ್ಕು ಗೋಡೆಗಳ ನಡುವಿನ ಪ್ರಯೋ\-ಗಾಲಯದಲ್ಲಿ ಅಲ್ಲ. ಆ ಕಾರ್ಯಾಚರಣೆ ನಡೆದದ್ದು ಹೊರಗಿನ ತೆರೆದ ವಿಶಾಲ ಬಯಲಿನಲ್ಲಿ. ದಟ್ಟಕಾಡಿನಲ್ಲಿ. ಸುರಿಯುವ ಮಳೆಯಲ್ಲಿ. ಸುಡುವ ಬಿಸಿಲಿನಲ್ಲಿ. ಆದ್ದರಿಂದ ಈ ವೈಜ್ಞಾನಿಕ ಮಹಾ ಕಾರ್ಯದಲ್ಲಿ ಆದ ಕಷ್ಟ ನಷ್ಟ, ಜೀವ ಹಾನಿ ಅಪಾರ. ಈ ನಿಖರ ಗಣಿತ ಸರ್ವೇಯಲ್ಲಿ ದುಡಿದ ಮೂಲ ಪುರುಷರ ಪ್ರತಿಭೆ, ಪರಿಶ್ರಮ, ತ್ಯಾಗ, ಸಾಹಸಗಳನ್ನು ಓದಿದಾಗ ಆ ಕ್ಷೇತ್ರದಲ್ಲಿ ತೊಡಗಿಸಿಕೊಂಡ ಈಗಿನ ಯಾರಿಗೇ ಆಗಲಿ, ಆ ಕಾರ್ಯಗಳು ಸ್ಫೂರ್ತಿದಾಯಕ ಮತ್ತು ಆದರ್ಶ ಸಾಧನೆಗಳು ಆಗಿವೆ.

ಭಾರತದಲ್ಲಿ ಟ್ರಿಗನಮಿಟ್ರಿಕಲ್​ ಸರ್ವೇಯು ಐತಿಹಾಸಿಕವಾಗಿ ಬೆಳೆದು ಬಂದ ವಿಜ್ಞಾನದ ಹಾದಿಯನ್ನು ಮತ್ತು ಆ ಹಾದಿಯಲ್ಲಿ ನಡೆದ ಸಾಧಕರ ಹೆಜ್ಜೆಗುರುತುಗಳನ್ನು ನೀಡುವುದು ಈ ಕಿರು ಹೊತ್ತಿಗೆಯ ಉದ್ದೇಶ. ಈ ನಿಖರ ಗಣಿತ ಸರ್ವೇಯಲ್ಲಿ ದುಡಿದವರ ಪ್ರತಿಭೆ ಪರಿಶ್ರಮಗಳ ಚಿತ್ರಣವನ್ನು ಕನ್ನಡದಲ್ಲಿ ದಾಖಲಿಸುವ ಪುಟ್ಟ ಪ್ರಯತ್ನವು ಇಲ್ಲಿದೆ. ಸಾವಿರಾರು ಮಂದಿ ಏಕಕಾಲದಲ್ಲಿ ದುಡಿದ, ಬಹು ದೊಡ್ಡಗಾತ್ರದ ವಿಜ್ಞಾನದ ಮಹಾ ಕಾರ್ಯ ಇದು. ಧೀರ್ಘ\-ಕಾಲದವರೆಗೆ, ಅರ್ಧ ಶತಮಾನಕ್ಕೂ ಮೀರಿ, ಸಾಗಿದ ಸರ್ವೇಯ ಮಹಾ ಯಾತ್ರೆ ಇದು. ಭಾರತದಲ್ಲಿ ನಡೆದ ಈ ಟ್ರಿಗನಮಿಟ್ರಿಕಲ್​ ಸರ್ವೇಕಾರ್ಯಾಚರಣೆಯಲ್ಲಿ ಆಸಕ್ತಿ ಕುತೂಹಲವುಳ್ಳವರಿಗೆ, ವಿಜ್ಞಾನ ಸಾಹಿತ್ಯಾಸಕ್ತರಿಗೆ ಈ ಕಿರು ಹೊತ್ತಿಗೆ ಇಷ್ಟವಾಗಬಹುದೆಂದು ನನ್ನ ಆಶಯ. 

ಕರ್ನಾಟಕ ರಾಜ್ಯ ವಿಜ್ಞಾನ ಪರಿಷತ್​ ಸಂಸ್ಥೆಯು ವಿಜ್ಞಾನದ ಬೆಳವಣಿಗೆಯನ್ನು\break ಬರವಣಿಗೆಯನ್ನು ಸದಾ ಪ್ರೋತ್ಸಾಹಿಸಿಕೊಂಡು ಬರುತ್ತಿರುವ ನಾಡಿನ ಹೆಮ್ಮೆಯ ಸಂಸ್ಥೆಯಾಗಿದೆ. ವಿಜ್ಞಾನದ ಪ್ರಚಾರ ಮತ್ತು ಪ್ರಸಾರಕ್ಕಾಗಿ ಅಪಾರವಾಗಿ ಶ್ರಮಿಸುತ್ತಿರುವ ಮಹತ್ವದ ಸಂಸ್ಥೆಯಾಗಿದೆ. ಸಾಮಾನ್ಯ ಜನರಿಗೆ ವಿಜ್ಞಾನ ವಿಷಯಗಳನ್ನು ತಿಳಿಸುವ ನಿಟ್ಟಿನಲ್ಲಿ ಜನಪ್ರಿಯ\break ವಿಜ್ಞಾನ ಪುಸ್ತಕಗಳ ಪ್ರಕಟಣೆಯನ್ನೂ ಸಹ ಮಾಡಿಕೊಂಡು ಬರುತ್ತಿದೆ. ರೆವಿನ್ಯೂ ಆಡಳಿತಕ್ಕೆ ಮತ್ತು ಆರ್ಥಿಕ ಆಡಳಿತಕ್ಕೆ ತಳಹದಿಯಾಗಿರುವ ಸರ್ವೇಯಿಂಗ್​ ಇಲಾಖೆಯ ತಾಂತ್ರಿಕ ಕಾರ್ಯವು ಮತ್ತು ಅದರ ವಿಜ್ಞಾನ ಸಾಹಿತ್ಯವು ಜನಸಾಮಾನ್ಯರ ಮತ್ತು ವಿಜ್ಞಾನಾಸಕ್ತರ ಮಧ್ಯದಲ್ಲಿ ಬಹಳಷ್ಟು ಅಜ್ಞಾತವಾಗಿಯೇ ಉಳಿದಿರುವ ವಿಷಯವಾಗಿದೆ. ಈ ದಿಕ್ಕಿನಲ್ಲಿ, ಭೂಮಾಪನ ಇಲಾಖೆಯಲ್ಲಿ ಕಾರ್ಯನಿರತನಾಗಿರುವ ನಾನು ಬರೆದಿರುವ ಟ್ರಿಗನಮಿಟ್ರಿಕಲ್​ ಸರ್ವೇ ಬರೆಹವನ್ನು ಗುರುತಿಸಿ, ಕರ್ನಾಟಕ ರಾಜ್ಯ ವಿಜ್ಞಾನ ಪರಿಷತ್ತು ಈ ಕಿರುಹೊತ್ತಿಗೆಯನ್ನು ಪುಸ್ತಕ ರೂಪದಲ್ಲಿ ಹೊರತರುತ್ತಿರುವುದು ನನಗೆ ಸಂತೋಷದ ವಿಷಯವಾಗಿದೆ. ಈ ಕಾರಣಕ್ಕೆ, ಕರ್ನಾಟಕ ರಾಜ್ಯ ವಿಜ್ಞಾನ ಪರಿಷತ್ತಿನ ಕಾರ್ಯಕಾರಿ ಸಮಿತಿಯ ಸನ್ಮಾನ್ಯ ಅಧ್ಯಕ್ಷರು ಮತ್ತು ಶಾಸಕರು, ವಿಧಾನ ಪರಿಷತ್ತು ಆಗಿರುವ ಶ‍್ರೀ ಎಸ್​.ವಿ. ಸಂಕನೂರ ಅವರಿಗೆ ಮತ್ತು ಸಮಿತಿಯ ಮಾನ್ಯ\break ಸದಸ್ಯರುಗಳಿಗೆ ಹೃತ್ಪೂರ್ವಕ ಕೃತಜ್ಞತೆಗಳನ್ನು ಸಲ್ಲಿಸುತ್ತೇನೆ. ಇದಲ್ಲದೆ ಕರ್ನಾಟಕ ರಾಜ್ಯ ವಿಜ್ಞಾನ ಪರಿಷತ್ತಿನ ಪುಸ್ತಕ ಪ್ರಕಟಣಾ ಸಮಿತಿ ಅಧ್ಯಕ್ಷರಾದ ಸನ್ಮಾನ್ಯ ಶ‍್ರೀ ಕೊಟ್ರುಸ್ವಾಮಿ ಎಸ್​. ಎಂ., ಮತ್ತು ಸಮಿತಿಯ ಮಾನ್ಯ ಸದಸ್ಯರುಗಳಿಗೂ ನನ್ನ ಹೃತ್ಪೂರ್ವಕ ಧನ್ಯವಾದ\-ಗಳನ್ನು ಅರ್ಪಿಸುತ್ತೇನೆ.

ಈ ಪುಸ್ತಕಕ್ಕೆ ಅಮೂಲ್ಯವಾದ ಮುನ್ನುಡಿಯನ್ನು ಬರೆದುಕೊಟ್ಟು ನನ್ನನ್ನು ಅಭಿಮಾನದಿಂದ ಪ್ರೋತ್ಸಾಹಿಸುತ್ತಾ ಬಂದಿರುವ ಮೈಸೂರು ಸೈನ್ಸ್​ ಫೌಂಡೇಷನ್ನಿನ ಮಾನ್ಯ ಅಧ್ಯಕ್ಷರು,\break ಭೌತಶಾಸ್ತ್ರದ ಉಪನ್ಯಾಸಕರು, ರಾಷ್ಟ್ರೀಯ ಮಕ್ಕಳ ವಿಜ್ಞಾನ ಸಮಾವೇಶದ ರಾಜ್ಯ ಶೈಕ್ಷಣಿಕ\break ಸಂಯೋಜಕರು ಮತ್ತು ಕರಾವಿಪ ಕಾರ್ಯಕಾರಿ ಸಮಿತಿಯ ಸದಸ್ಯರೂ ಆದ ಶ‍್ರೀ\break ಸಿ. ಕೃಷ್ಣೇಗೌಡರವರಿಗೆ ಹೃತ್ಪೂರ್ವಕ ಕೃತಜ್ಞತೆಗಳನ್ನು ಅರ್ಪಿಸುತ್ತೇನೆ. ಕರಾವಿಪ ಕಾರ್ಯಕಾರಿ ಸಮಿತಿ ಮತ್ತು ಪುಸ್ತಕ ಪ್ರಕಟಣಾ ಸಮಿತಿ ಸದಸ್ಯರಾದ ಮಾನ್ಯ ಶ‍್ರೀ ಎನ್​. ಆರ್​. ಮಂಜುನಾಥ\-ರವರಿಗೂ ಅವರ ಅಮೂಲ್ಯ ಸಲಹೆ ಸಹಕಾರಕ್ಕಾಗಿ ನನ್ನ ಹೃತ್ಪೂರ್ವಕ ನಮನಗಳನ್ನು ಅರ್ಪಿಸುತ್ತೇನೆ.

ಆರಂಭದಲ್ಲಿ ನನ್ನ ಈ ಪುಸ್ತಕದ ಕರಡು ಪ್ರತಿಯನ್ನು ಓದಿ ಅಮೂಲ್ಯ ಸಲಹೆ ಸೂಚನೆ\-ಗಳನ್ನು ನೀಡಿದ ಕರಾವಿಪ ಪುಸ್ತಕ ಪ್ರಕಟಣಾ ಸಮಿತಿಯ ಸದಸ್ಯರು ಮತ್ತು ಗುರು ಸಮಾನರೂ ಆದ ಮಾನ್ಯ ಶ‍್ರೀ ವಿ.ಎಸ್​.ಎಸ್​. ಶಾಸ್ತ್ರಿಸಾರ್​ರವರಿಗೂ ಹೃತ್ಪೂರ್ವಕ ಕೃತಜ್ಞತೆಗಳನ್ನು\break ಅರ್ಪಿಸುತ್ತೇನೆ. ಅಗತ್ಯ ಸಾಫ್ಟ್​ವೇರ್​ ಬಳಸಿ ಸುಂದರವಾದ ಪುಸ್ತಕವನ್ನು ಮಾಡಿಕೊಟ್ಟ ಶ‍್ರೀರಂಗ ಡಿಜಿಟಲ್​ರವರಿಗೂ ವಂದನೆಗಳು. ಪುಸ್ತಕಕ್ಕೆ ಅಂತಿಮರೂಪ ಕೊಡಲು ಶ್ರಮಿಸಿದ\break ಕರಾವಿಪ ಕಚೇರಿ ಅಧಿಕಾರಿ ಮತ್ತು ಸಿಬ್ಬಂದಿ ವರ್ಗದವರಿಗೂ ವಂದನೆಗಳು. ಮೈಸೂರಿನ\break ಹೆಸಾರಂತ ಕಲಾಸುರುಚಿರಂಗಮನೆಯ ಹಿರಿಯರಾದ ಕವಿಗಳೂ ನಾಟಕಕಾರರೂ ಆದ ಡಾ. ಭದ್ರಪ್ಪ ಶಿ.ಹೆನ್ಲಿ ರವರಿಗೆ ಅವರ ಅಮೂಲ್ಯ ಸಲಹೆ ಸೂಚನೆ ಮತ್ತು ಪ್ರೋತ್ಸಾಹಗಳಿಗಾಗಿ\break ವಂದನೆಗಳನ್ನು ಸಲ್ಲಿಸುತ್ತೇನೆ.

ನನ್ನಲ್ಲಿ ಭೂಮಾಪನ ಅರಿವನ್ನು ತುಂಬಿ ಬರೆಯಲು ಪ್ರೇರಣೆ ಪ್ರೋತ್ಸಾಹ ಮತ್ತು ಅವಕಾಶ ನೀಡಿದ ಭೂಮಾಪನ ಇಲಾಖೆಯ ಹಿರಿಯರಿಗೂ ಗೆಳೆಯರಿಗೂ ಹೃತ್ಪೂರ್ವಕ ವಂದನೆ\-ಗಳನ್ನು ಸಲ್ಲಿಸುತ್ತೇನೆ. ಈ ಕೃತಿಯನ್ನು ಓದುತ್ತಿರುವ ಮಾನ್ಯ ಓದುಗರಿಗೂ ಧನ್ಯವಾದಗಳನ್ನು\break ಅರ್ಪಿಸುತ್ತಾ ಈ ಪುಸ್ತಕದ ಬಗ್ಗೆ ನಿಮ್ಮಗಳ ಅಮೂಲ್ಯವಾದ ಅನಿಸಿಕೆ ಅಭಿಪ್ರಾಯಗಳನ್ನು\break ಸ್ವಾಗತಿಸುತ್ತೇನೆ.

\bigskip

\begin{flushright}
\textbf{ಸರಡವಳ್ಳಿ ಜಿ ರಮೇಶ}\\\enginline{269, 2}ನೆ ಹಂತ, ಕೆ.ಹೆಚ್​.ಬಿ.,ಕುವೆಂಪುನಗರ, \\ ಮೈಸೂರು–\enginline{570023}, ದೂರವಾಣಿ: \enginline{9342150519}
\end{flushright}

\vskip -2.1cm

\noindent
\enginline{14} ಡಿಸೆಂಬರ್ \enginline{2018}

