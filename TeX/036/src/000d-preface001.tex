
\chapter*{ಮುನ್ನುಡಿ}

‘ದಿ ಗ್ರೇಟ್​ ಟ್ರಿಗಮಿಟ್ರಿಕಲ್​ ಸರ್ವೇ ಆಫ್​ ಇಂಡಿಯಾ’ ಕೃತಿಯು ಟ್ರಿಗನಮಿಟ್ರಿಕಲ್​ ಸರ್ವೇ ವಿಷಯದ ಮೇಲೆ ಕನ್ನಡ ಭಾಷೆಯಲ್ಲಿ ಬರೆದ ಮೊದಲ ಪುಸ್ತಕ ಎನ್ನಬಹುದು. ಸರಡವಳ್ಳಿ ಜಿ ರಮೇಶ್​ರವರು ಸರ್ವೇ ಡಿಪಾರ್ಟ್ಮೆಂಟ್​ನಲ್ಲಿ ಬೋಧಕರಾಗಿ ಮತ್ತು ಪ್ರತಿಭಾವಂತ ಸಿವಿಲ್​\break ಇಂಜಿನೀಯರ್​ ಆಗಿ ಮೂರು ದಶಕಗಳ ಕಾಲ ಸೇವೆ ಸಲ್ಲಿಸಿದ್ದಾರೆ. ಈ ಅನುಭವದಿಂದ, ಭಾರತದಲ್ಲಿ ಟ್ರಿಗನಮಿಟ್ರಿಕಲ್​ ಸರ್ವೇಯು ಬೆಳೆದು ಬಂದ ಹಾದಿಯನ್ನು ಕನ್ನಡ ಓದುಗರಿಗೆ ಸುಲಭವಾಗಿ ಲಭ್ಯವಾಗಲಿ ಎಂಬ ಉದ್ದೇಶದಿಂದ ಸರಳವಾಗಿ, ರಸವತ್ತಾಗಿ, ಅರ್ಥಪೂರ್ಣವಾಗಿ ಈ ಪುಸ್ತಕ ರೂಪದಲ್ಲಿ ಚಿತ್ರಿಸಿದ್ದಾರೆ.

ನಮ್ಮ ಕರ್ನಾಟಕ ರಾಜ್ಯ ವಿಜ್ಞಾನ ಪರಿಷತ್ತು ಬೆಂಗಳೂರು ಇಂತಹ ಕಿರು ಹೊತ್ತಿಗೆಯನ್ನು ಪ್ರಕಟಿಸಿರುವುದು ಸಾರ್ಥಕ ಕಾರ್ಯವನ್ನು ಮಾಡಿದಂತಾಗಿದೆ. ಶ‍್ರೀ ಸರಡವಳ್ಳಿ ಜಿ\break ರಮೇಶ್​ರವರು ನಿರಂತರವಾಗಿ ವಿಜ್ಞಾನ ಸಾಹಿತ್ಯ ಕೃಷಿಯನ್ನು ಮಾಡಲಿ. ಈ ಪುಸ್ತಕವು ಶ‍್ರೀ ಸರಡವಳ್ಳಿ ಜಿ ರಮೇಶ್​ರವರ ಮೊದಲ ಕೃತಿಯಾಗಿದೆ. ಈ ಕೃತಿಯು ಸರ್ವೇ ಇಲಾಖೆಯಿಂದ ತರಬೇತಿ ಪಡೆಯುವ ಸೇವಾ ನಿರತ ಮೋಜಿಣಿದಾರರಿಗೆ, ಇಂಜಿನಿಯರಿಂಗ್​ ವಿದ್ಯಾರ್ಥಿ\-ಗಳಿಗೆ ಮತ್ತು ಈ ವಿಷಯದಲ್ಲಿ ಕುತೂಹಲವುಳ್ಳ ವಿಜ್ಞಾನ ಸಾಹಿತ್ಯಾಸಕ್ತರಿಗೆ ಅತ್ಯಾವಶ್ಯಕವಾದ ಮಾಹಿತಿಯನ್ನು ಒದಗಿಸುತ್ತದೆ. ಇಂತಹ ಕೃತಿಯನ್ನು ರಚಿಸಿದ ಲೇಖಕರಿಗೆ ನನ್ನ ಹೃತ್ಪೂರ್ವಕ\break ಅಭಿವಂದನೆಗಳನ್ನು ಸಲ್ಲಿಸುತ್ತೇನೆ.

\bigskip

\begin{flushright}
\textbf{ಸಿ. ಕೃಷ್ಣೇಗೌಡ}\\ ಕರಾವಿಪ ಸದಸ್ಯರು ಹಾಗೂ \\ ಅಧ್ಯಕ್ಷರು, ಮೈಸೂರು ಸೈನ್ಸ್ ಫೌಂಡೇಷನ್, ಮೈಸೂರು
\end{flushright}

