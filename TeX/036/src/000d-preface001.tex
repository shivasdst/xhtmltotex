
\chapter*{ಮುನ್ನುಡಿ}

ದಿ ಗ್ರೇಟ್​ ಟ್ರಿಗನಮಿಟ್ರಿಕಲ್​ ಸರ್ವೇ ಆಫ್​ ಇಂಡಿಯಾ ಕನ್ನಡ ಭಾಷೆಯಲ್ಲಿ ಮೊದಲ ಪುಸ್ತಕ\-ವಿರಬಹುದು. ಸರಡವಳ್ಳಿ ಜಿ ರಮೇಶರವರು ಸರ್ವೇ ಡಿಪಾರ್ಟ್ಮೆಂಟ್​ನಲ್ಲಿ ಮೂರು ದಶಕಗಳ ಕಾಲ ಬೋಧಕರಾಗಿ, ಪ್ರತಿಭಾವಂತ ಸಿವಿಲ್​ ಇಂಜಿನಿಯರ್​ ಆಗಿ ಸೇವೆ ಸಲ್ಲಿಸಿದ ಅನುಭವದಿಂದ ಭಾರತದಲ್ಲಿ ಟ್ರಿಗನಮಿಟ್ರಿಕಲ್​ ಸರ್ವೇ ಬೆಳೆದು ಬಂದ ಹಾದಿಯನ್ನು ಸರಳವಾಗಿ, ರಸ\-ವತ್ತಾಗಿ, ಅರ್ಥಪೂರ್ಣವಾಗಿ ಓದುಗರಿಗೆ ಸುಲಭವಾಗಿ ಲಭ್ಯವಾಗಲಿ ಎಂಬ ಉದ್ದೇಶದಿಂದ ಈ ಪುಸ್ತಕವನ್ನು ಬರೆದಿದ್ದಾರೆ. ನಮ್ಮ ಕರ್ನಾಟಕ ರಾಜ್ಯ ವಿಜ್ಞಾನ ಪರಿಷತ್ತು, ಬೆಂಗಳೂರು ಇಂತಹ ಕಿರುಹೊತ್ತಿಗೆಯನ್ನು ಪ್ರಕಟಿಸಿರುವುದು ಸಾರ್ಥಕ ಕಾರ್ಯವನ್ನು ಮಾಡಿದಂತಾಗಿದೆ. ಶ‍್ರೀ ಸರಡವಳ್ಳಿ ಜಿ ರಮೇಶರವರು ನಿರಂತರವಾಗಿ ವಿಜ್ಞಾನ ಸಾಹಿತ್ಯ ಕೃಷಿಯನ್ನು ಮಾಡಲಿ. ಈ ಪುಸ್ತಕವು ಶ‍್ರೀಯುತರ ಮೊದಲ ಕೃತಿಯಾಗಿದೆ. ಈ ಕೃತಿಯು ಸರ್ವೇ ಇಲಾಖೆಯಿಂದ ತರಬೇತಿ ಪಡೆಯುವ ಸೇವಾನಿರತ ಮೋಜಿಣಿದಾರರಿಗೆ ಹಾಗೂ ಇಂಜಿನಿಯರಿಂಗ್​ ವಿದ್ಯಾರ್ಥಿಗಳಿಗೆ ಅತ್ಯಾವಶ್ಯಕವಾದ ಮಾಹಿತಿಯನ್ನು ಒದಗಿಸುತ್ತದೆ. ಇಂತಹ ಕೃತಿಯನ್ನು ರಚಿಸಿದ ಲೇಖಕ\-ರಿಗೆ ನನ್ನ ಹೃತ್ಪೂರ್ವಕವಾದ ಅಭಿನಂದನೆಗಳನ್ನು ಸಲ್ಲಿಸುತ್ತೇನೆ. 

\bigskip

\begin{flushright}
\textbf{ಸಿ. ಕೃಷ್ಣೇಗೌಡ}\\ ಕರಾವಿಪ ಸದಸ್ಯರು ಹಾಗೂ \\ ಅಧ್ಯಕ್ಷರು, ಮೈಸೂರು ಸೈನ್ಸ್ ಫೌಂಡೇಷನ್, ಮೈಸೂರು
\end{flushright}

