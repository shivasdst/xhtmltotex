
\chapter{ಕಾಯಿಲೆಬಿದ್ದ ಎವರೆಸ್ಟ್​ರು ಸಿಕ್​ ಲೀವ್​ ಪಡೆದು ಇಂಗ್ಲೆಂಡಿಗೆ ಹೋದರು}

ಇತ್ತ ಎವರೆಸ್ಟ್​ರವರು ಟ್ರ್ಯಾಂಗ್ಯುಲೇಷನ್​ನನ್ನು ಉತ್ತರದ ನಾಗಾಪುರದತ್ತ ಕೊಂಡೊಯ್ಯುವ ಯೋಜನೆಯಲ್ಲಿದ್ದರು. ಹೈದರಾಬಾದನ್ನು ಬಿಟ್ಟು ಹೊರಡಬೇಕೆನ್ನುವಷ್ಟರಲ್ಲಿ ಸುರಿಮಳೆ ಪ್ರಾರಂಭವಾಯಿತು. ಎವರೆಸ್ಟ್​ರವರಿಗೆ ಸುಡು ಜ್ವರ, ಅಸಾಧ್ಯ ನಡುನೋವು ಶುರುವಾಯಿತು. ವಾರದಲ್ಲೇ ಕಾಲುಗಳು ಲಕ್ವ ಹೊಡೆದವರಂತೆ ನಿಷ್ಕ್ರಿಯಗೊಂಡವು. ಚರ್ಮ ಸುಲಿಯಲು ಶುರುವಾಯಿತು. ನಿದ್ರಾಹೀನತೆಯಿಂದ ನರಳಿ ನಲುಗಿದರು. ಹಿಂದಿನ ಯಲ್ಲಾಪುರಂನ ಜ್ವರ ಪ್ರತಿಕಾರಕ್ಕಾಗಿ ಪುನಃ ಹಿಂತಿರುಗಿ ಬಂದಿತ್ತು. ವೈದ್ಯರು ಮದ್ರಾಸಿನ ಹವೆ ಸೂಕ್ತ ಎಂದರು. ವಿಶ್ರಾಂತಿ ಅವಶ್ಯಕ ಎಂದು ಎಚ್ಚರಿಸಿದರು. ಆದರೆ ಎವರೆಸ್ಟ್​ರವರದು ಅಚಲ ನಿರ್ಧಾರ. ಗ್ರೇಟ್​ ಆರ್ಕ್ನ್ನು ಮುಂದುವರೆಸಬೇಕಾ ಅಥವಾ ಈ ಬೇರಾರ್​ ಕಣಿವೆ ಸ್ಥಳದಲ್ಲೇ, ಗ್ರೇಟ್​ ಆರ್ಕ್ ಕಾರ್ಯವನ್ನು ಅವಮಾನಕರವಾಗಿ ನಿಲ್ಲಿಸಬೇಕಾ? ಇದು ಎವರೆಸ್ಟ್​ರವರು ನಿರ್ಧರಿಸಬೇಕಾದ ವಿಚಾರವಾಗಿತ್ತು. ಅಷ್ಟೇ ಅಲ್ಲದೇ, ಖಾಯಿಲೆಯೆಂದು ನಿಲ್ಲಿಸಿಬಿಟ್ಟರೆ, ಈ ಸರ್ವೇ ಕುಟುಂಬ, ಅಂದರೆ ಸರ್ವೇ ಮಹಾ ಸಂಸ್ಥೆಯು ದಿಕ್ಕಾಪಾಲಾಗಿ ಹೋಗುತ್ತದೆ. ಇಷ್ಟೊಂದು ತರಬೇತಿ ಪಡೆದ ವ್ಯಕ್ತಿಗಳನ್ನು ಕಳೆದುಕೊಳ್ಳಬೇಕಾಗುತ್ತದೆ. ನುರಿತ ಮತ್ತೊಬ್ಬರನ್ನು ಅವರ ಜಾಗದಲ್ಲಿ ಪುನಹ ತರುವುದು ಕಷ್ಠಕರ ಎಂದು ಯೋಚಿಸಿದರು. ಎವರೆಸ್ಟ್​ರವರಿಗೆ ಬೇರೆ ಮಾರ್ಗವೇ ಉಳಿದಿರಲಿಲ್ಲ. ಜ್ವರದೊಂದಿಗೆ ಸೆಣಸುತ್ತಾ ಅಕ್ಟೋಬರ್​ \enginline{1823}ರಲ್ಲಿ ಉತ್ತರದ ನಾಗಪುರ ಮತ್ತು ಯಲಿಚಪುರ ಬೇಸ್​ ಲೈನ್​ನತ್ತ ಹೊರಟರು.

\newpage

ಆದರೆ ಅದು ಎವರೆಸ್ಟ್​ರವರ ಹತಾಶೆಯ ತೀರ್ಮಾನವಾಗಿತ್ತು. ಕೈ ಕಾಲುಗಳು ಲಕ್ವ ಹೊಡೆದವರಂತೆ ನಿತ್ರಾಣವಾಗಿದ್ದವು. ಜೆನಿತ್​ ಸೆಕ್ಟರ್​ನಲ್ಲಿ ಕೋನ ಓದುವಾಗ, ಕೂತ ಕುರ್ಚಿಯಿಂದ ಮೇಲೆತ್ತಿ ನಿಲ್ಲಿಸಲು ಮತ್ತು ಖುರ್ಚಿಯಲ್ಲಿ ಪುನಃ ಕೂರಿಸಲು ಎವರೆಸ್ಟ್​ರವರಿಗೆ ಇಬ್ಬರು ಸಹಾಯಕರು ಸದಾ ಸಹಾಯಕ್ಕೆ ಜತೆಗಿರಲೇಬೇಕಾಗಿತ್ತು. ಥಿಯಡೊಲೈಟಿನ ವರ್ಟಿಕಲ್​ ಸರ್ಕಲ್​ ಸ್ಕ್ರೂವನ್ನು ತಿರುಗಿಸಲು, ಎಡತೋಳನ್ನು ಒಬ್ಬರು ಹಿಡಿದು ಎತ್ತಬೇಕಿತ್ತು. ಸುಸ್ತು ಬಳಲಿಕೆ ಹೇಗಿತ್ತು ಎಂದರೆ ನಿಂತುಕೊಳ್ಳಲು ಸಹ ಇಬ್ಬರು ಹಿಡಿದುಕೊಂಡಿರಬೇಕಿತ್ತು. ಆರು ತಿಂಗಳುಗಳ ಕಾಲ ಇದೇ ರೀತಿಯಲ್ಲೇ ಅರೆ ತ್ರಾಣದಲ್ಲಿ, ಅತ್ತಿಂದಿತ್ತ ಸಂಚರಿಸಲು ಆಗದೆ, ಓಡಾಟಕ್ಕೆ ಡೋಲಿಯನ್ನೇ ಅವಲಂಬಿಸಬೇಕಾದ ಸ್ಥಿತಿಯಾಗಿತ್ತು. ಈ ಅವಧಿಯಲ್ಲಿ, ಎಲಿಚಪುರದಿಂದ ಸಿರೋಂಜ್​ ವರೆಗಿನ \enginline{200} ಮೈಲು ಉದ್ದದ ಗ್ರೇಟ್​ ಆರ್ಕ್ ಕಾರ್ಯ ಪೂರೈಸಲು ಸುಮಾರು \enginline{2} ವರ್ಷ ಹಿಡಿಯಿತು. ಗ್ರೇಟ್​ ಆರ್ಕ್ನ ಈ ಭಾಗವು ಉಪಖಂಡದ ಮಧ್ಯ ಭಾಗದಲ್ಲಿದೆ.

ಎವರೆಸ್ಟ್​ರವರು \enginline{1924}ರಲ್ಲಿ ಸಿರೋಂಜ್​ ಬೇಸ್​ಲೈನ್​ ಅಳತೆಯನ್ನು ಅನಿಶ್ಚಿತ ಆರೋಗ್ಯ ಸ್ಥಿತಿಯಲ್ಲಿಯೇ ಪ್ರಾರಂಭಿಸಿದ್ದರು. ಇದು ಅವರ ಮೊದಲ ಬೇಸ್​ ಲೈನ್​ ಅಳತೆ ಕಾರ್ಯ ಆಗಿತ್ತು. ದುರ್ಬಲ ಅನಾರೋಗ್ಯ ಸ್ಥಿತಿಯಲ್ಲಿಯೇ ಈ ಬೇಸ್​ ಲೈನ್​ ಅಳತೆಕಾರ್ಯವನ್ನು ಪೂರೈಸಿದರು. ಆರೋಗ್ಯದ ಸ್ಥಿತಿಯಲ್ಲಿ ಯಾವುದೇ ಸುಧಾರಣೆ ಇಲ್ಲದೆ ದೇಹದ ಹದಗೆಟ್ಟ ಸ್ಥಿತಿ ಇನ್ನೂ ಮುಂದುವರೆದಿತ್ತು. ಏನೂ ಮಾಡಲಾಗದೇ, ಅನಿವಾರ್ಯವಾಗಿ \enginline{1825}ರ ಮಾರ್ಚ್ನಲ್ಲಿ ಸಿಕ್​ ಲೀವ್​ ಮೇಲೆ ಇಂಗ್ಲೆಂಡಿಗೆ ತೆರಳಿದರು. ಮುಂದಿನ ಐದು ವರ್ಷದವರೆಗೆ ಗ್ರೇಟ್​ ಆರ್ಕ್ ಕಾರ್ಯಕ್ಕೆ ಬರಲಿಲ್ಲ. ಐದು ವರ್ಷ ಸಿಕ್​ ಲೀವ್​ ಈಗ ನಮಗೆ ದೀರ್ಘವೆನಿಸುತ್ತದೆ. ಆದರೆ, ಆ ಕಾಲದಲ್ಲಿ ಊರಿಗೆ ಪ್ರಯಾಣವೆಂದರೆ ಅದು ಸಮುದ್ರ ಯಾನ. ಊರಿಗೆ ಹೋಗಲು ಹಡಗಿನ ಆರು ತಿಂಗಳು ಪ್ರಯಾಣ. ಊರಿಂದ ವಾಪಸ್ಸು ಬರಲು ಪುನಃ ಆರು ತಿಂಗಳು ಹಡಗಿನ ಪ್ರಯಾಣ.

ಸಿಕ್​ ಲೀವ್​ ಪಡೆದು, ಇಂಗ್ಲೆಂಡಿಗೆ ಹೋದ ಎವರೆಸ್ಟ್​ರವರು ಅಲ್ಲಿ ಸುಮ್ಮನೆ ಕೂರುವುದಿಲ್ಲ. ಅವರಿಗೆ ತಮ್ಮ ರಜಾ ಸಕಾಲಿಕವಾಗಿತ್ತು, ಸದಾವಕಾಶವಾಗಿತ್ತು. ಅಲ್ಲಿ ಅವರು ಹೊಸ ಹೊಸ ಸರ್ವೇ ಉಪಕರಣಗಳನ್ನು ಅರಸಿದರು. ಉಪಕರಣ ತಯಾರಕರನ್ನು ಭೇಟಿ ಮಾಡಿದರು. ಇಲ್ಲಿಂದ ಹೊರಡುವಾಗ, ಎರಡು ವರ್ಷ ಮೊದಲಿನ ಅವರ ಸರ್ವೇ ಕಾರ್ಯದ ಸಂಪೂರ್ಣ ವಿವರಗಳನ್ನು ಅಲ್ಲಿಗೆ ಕೊಂಡೊಯ್ದಿದ್ದರು. ಆ ವಿವರಗಳ ದಾಖಲೀಕರಣ ಕಾರ್ಯವನ್ನು ಅಲ್ಲಿಯೇ ಮಾಡಿದರು. ಅಲ್ಲಿ ಇಂಗ್ಲೆಂಡಿನಲ್ಲಿದ್ದುಕೊಂಡೂ ಇಲ್ಲಿಯ ಸರ್ವೇಗೆ ತುಂಬಾ ಉಪಯುಕ್ತವಾದ ಚಟುವಟಿಕೆಗಳಲ್ಲಿ, ನಿರಂತರ ಕಾರ್ಯಮಗ್ನರಾಗಿಯೇ ಇದ್ದರು.

ಎವರೆಸ್ಟ್​ರವರು, ಇಂಗ್ಲೆಂಡ್​ನಲ್ಲಿದ್ದಾಗ, ಡಬ್ಲಿನ್​ನ ಲಾರ್ಡ್ ಲೆಫ್ಟಿನೆಂಟ್​ ಆಗಿದ್ದವರು ರಿಚರ್ಡ್ ವೆಲ್ಲಸ್ಲಿಯವರು. ರಿಚರ್ಡ್ ವೆಲ್ಲಸ್ಲಿಯವರ ಪ್ರೇರಣೆಯಿಂದ ಬ್ರಿಟಿಷ್​ ಆರ್ಡ್ನನ್ಸ್​ ಸರ್ವೇಯು ಹೊಸದಾಗಿ, ಐರ್ಲೆಂಡಿನ ಡೀಟೈಲ್​ ಸರ್ವೇಯನ್ನು ಪ್ರಾರಂಭಿಸಿತ್ತು. ಆರ್ಡನನ್ಸ್​ ಸರ್ವೇಯೆಂದರೆ ಅದು ಬ್ರಿಟನ್ನಿನ ರಾಷ್ಟ್ರೀಯ ಮ್ಯಾಪಿಂಗ್​ ಸಂಸ್ಥೆ. ನಮ್ಮ ಭಾರತಕ್ಕೆ ‘ಸರ್ವೇ ಆಫ್​ ಇಂಡಿಯಾ’ ಇದ್ದ ಹಾಗೆ. ಆರ್ಡನನ್ಸ್​ ಸರ್ವೇಯ ಮಾಸ್ಟರ್​ ಜನರಲ್​ ಮತ್ತು ಅಲ್ಲಿನ ಪ್ರೈಮ್ ಮಿನಿಸ್ಟರ್​ ಆಗಿದ್ದವರು ರಿಚರ್ಡ್ ವೆಲ್ಲೆಸ್ಲಿಯವರ ಸಹೋದರ ಅರ್ಥರ್​ ವೆಲ್ಲೆಸ್ಲಿಯವರು. ನಾಲ್ಕನೇ ಮೈಸೂರು ಯುದ್ದದಲ್ಲಿ ಲ್ಯಾಂಬ್​ಟನ್​ರವರ ಜೊತೆಗಿದ್ದು, ಶ‍್ರೀರಂಗಪಟ್ಟಣದ ಯುದ್ಧ ಭೂಮಿಯಲ್ಲಿ ಕುದುರೆಯಿಂದ ಬಿದ್ದು, ಮೊಣಕಾಲು ಗಾಯ ಮಾಡಿಕೊಂಡಿದ್ದರು ಈ ಅರ್ಥರ್​ ವೆಲ್ಲೆಸ್ಲಿಯವರು. ಆ ಸಮಯದಲ್ಲಿ ಅವರು ಇಂಗ್ಲೆಂಡ್​ನಲ್ಲಿ ಪ್ರಧಾನಿಯಾಗಿದ್ದರು. ಅವರೊಡನೆ ಎವರೆಸ್ಟ್​ರವರು ಮೂರು ತಿಂಗಳು ಇದ್ದು, ಐರಿಷ್​ ಸರ್ವೇಯ ಕಾರ್ಯ ವಿಧಾನಗಳನ್ನು ಮತ್ತು ಸಂಸ್ಥೆಯ ಸಂರಚನೆಯನ್ನು ಅಧ್ಯಯನ ಮಾಡಿದರು. ಐರ್ಲೆಂಡ್​ನಲ್ಲಿ ಟ್ರಿಗನಮಿಟ್ರಿಕಲ್​ ಸರ್ವೇಯು ಉಳಿದೆಲ್ಲಾ ಸರ್ವೇಗಳಿಗೆ ಹೇಗೆ ಪೂರ್ವಾಪೇಕ್ಷಿತ ತಳಹದಿ ಆಗಿದೆ ಎಂಬುದನ್ನು ಗಮನಿಸಿದರು. ಐರಿಷ್​ನಲ್ಲಿ ಅಲ್ಲಿಯ ಸರ್ವೇಯರುಗಳಿಗಾಗಿಯೇ ಉಪಕರಣ ತಯಾರಕರು, ದುರಸ್ತಿಗಾರರು, ಉಪಕರಣ ಪರಿಶೀಲಕರು ಇರುವ ವಿಶೇಷ ವ್ಯವಸ್ಥೆ ಇತ್ತು. ಅವುಗಳ ಹೆಚ್ಚುವರಿ ಅನುಕೂಲಗಳನ್ನು ಎವರೆಸ್ಟ್​ರವರು ಗಮನಿಸಿದ್ದರು. ಈ ಎಲ್ಲಾ ಅನುಕೂಲಗಳು ಭಾರತದ ಟ್ರಿಗನಮಿಟ್ರಿಕಲ್​ ಸರ್ವೇಗೂ ಬೇಕೆಂದು ಇಚ್ಛಿಸಿದರು.

ಅದೇ ಸಮಯದಲ್ಲಿ, ಇಂಗ್ಲೆಂಡಿನಲ್ಲಿ ರಾಯಲ್​ ಸೊಸೈಟಿ ಮತ್ತು ರಾಯಲ್​ ಅಸ್ಟ್ರಾನಮಿಕಲ್​ ಸೊಸೈಟಿಯ ಫೆಲೋ ಆಗಿ ಎವರೆಸ್ಟ್​ರವರು ಆಯ್ಕೆಯಾದರು. \enginline{1830}ರಲ್ಲಿ\break ‘ಸೂಪರಿಂಟೆಂಡೆಂಟ್​ ಆಫ್​ ಗ್ರೇಟ್​ ಟ್ರಿಗನಮಿಟ್ರಿಕಲ್​ ಸರ್ವೇ’ಯ ಹುದ್ದೆಯ ಜೊತೆಗೆ, ‘ಸರ್ವೇಯರ್​ ಜನರಲ್​ ಆಫ್​ ಇಂಡಿಯಾ’ದ ಅತ್ಯಂತ ಹಿರಿಯ ಹುದ್ದೆಗೆ ಇಂಗ್ಲೆಂಡ್​ನಲ್ಲಿ\break ದ್ದಾಗಲೇ ನೇಮಕವಾದರು.

