\sethyphenation{kannada}{
ಅಂಕ
ಅಂಕ-ವೇನೋ
ಅಂಕಿ
ಅಂಕಿ-ತ-ಗೊ-ಳಿ-ಸ-ದಿ-ರ-ಲಾರ
ಅಂಕು-ಡೊಂ-ಕಿನ
ಅಂಗ
ಅಂಗಡಿ
ಅಂಗ-ಡಿ-ಗ-ಳ-ವರು
ಅಂಗ-ಡಿ-ಗ-ಳಿಗೆ
ಅಂಗ-ಡಿಗೆ
ಅಂಗ-ಪ್ರ-ತ್ಯಂ-ಗ-ಗಳಲ್ಲಿ
ಅಂಗ-ಪ್ರ-ತ್ಯಂ-ಗವೂ
ಅಂಗ-ಭಂಗಿ
ಅಂಗ-ಭಂ-ಗಿ-ಗಳನ್ನು
ಅಂಗ-ರ-ಕ್ಷಕ
ಅಂಗ-ರ-ಕ್ಷ-ಕ-ನನ್ನೂ
ಅಂಗ-ಲಾ-ಚಿ-ಕೊಂಡು
ಅಂಗ-ಳ-ದಲ್ಲಿ
ಅಂಗ-ವಾಗಿ
ಅಂಗ-ವಾ-ಗಿತ್ತು
ಅಂಗ-ವಾ-ಗಿಯೇ
ಅಂಗ-ವಾದ
ಅಂಗವೂ
ಅಂಗ-ವೆಂದು
ಅಂಗಾಂ-ಗ-ಗಳು
ಅಂಗಾಂ-ಗಳು
ಅಂಗಾಂ-ಗ-ವನ್ನೂ
ಅಂಗಿ
ಅಂಗಿಯ
ಅಂಗೀ-ಕ-ರಿ-ಸ-ಲಾ-ಯಿತು
ಅಂಗೀ-ಕ-ರಿ-ಸಲು
ಅಂಗೀ-ಕ-ರಿಸಿ
ಅಂಗೀ-ಕ-ರಿ-ಸಿತು
ಅಂಗೀ-ಕ-ರಿ-ಸಿ-ದರು
ಅಂಗೀ-ಕ-ರಿ-ಸಿ-ದ್ದರು
ಅಂಗೀ-ಕ-ರಿ-ಸಿ-ದ್ದಾ-ರೆಂದು
ಅಂಗೀ-ಕ-ರಿ-ಸು-ತ್ತಿ-ದ್ದರು
ಅಂಗೀ-ಕ-ರಿ-ಸು-ವು-ದ-ರಲ್ಲಿ
ಅಂಗೀ-ರಿ-ಸ-ಲಾ-ಯಿತು
ಅಂಗುಲ
ಅಂಗೈ-ಯಲ್ಲಿ
ಅಂಚನ್ನು
ಅಂಚಿ-ನಲ್ಲಿ
ಅಂಚೆಯ
ಅಂಜದೆ
ಅಂಜ-ಬೇಡಿ
ಅಂಜಿ-ಕೆ-ಯಿ-ಲ್ಲದೆ
ಅಂಜಿ-ಸ-ದಿ-ರಲಿ
ಅಂಜು-ತ್ತಂ-ಜು-ತ್ತಲೇ
ಅಂಜುವ
ಅಂಜೂ-ರದ
ಅಂಟಿ
ಅಂಟಿ-ಕೊಂ-ಡಿ-ದ್ದಳು
ಅಂಟಿ-ಕೊಂ-ಡಿ-ರಲು
ಅಂಟಿ-ಕೊಂಡು
ಅಂಟಿ-ಕೊ-ಳ್ಳು-ತ್ತಿತ್ತು
ಅಂಟಿ-ಕೊ-ಳ್ಳುವ
ಅಂಟಿ-ಸ-ಕೂ-ಡ-ದೆಂದು
ಅಂಟಿ-ಸಲು
ಅಂಟಿ-ಸಿ-ದಂ-ತಿದ್ದ
ಅಂತ
ಅಂತಃ-ಪು-ರ-ದಿಂದ
ಅಂತರ
ಅಂತ-ರಂಗ
ಅಂತ-ರಂ-ಗಕ್ಕೆ
ಅಂತ-ರಂ-ಗದ
ಅಂತ-ರಂ-ಗ-ದಲ್ಲಿ
ಅಂತ-ರಂ-ಗ-ದೊ-ಳಗೆ
ಅಂತ-ರಂ-ಗ-ವ-ನ್ನಾ-ಗಲಿ
ಅಂತ-ರಂ-ಗ-ವನ್ನು
ಅಂತ-ರ-ಗಂ-ಗೆಯ
ಅಂತ-ರ-ದಂತೆ
ಅಂತ-ರ-ರಾ-ಷ್ಟ್ರೀಯ
ಅಂತ-ರವು
ಅಂತ-ರ-ವೆಂದರೆ
ಅಂತ-ರಾತ್ಮ
ಅಂತ-ರಾ-ತ್ಮವು
ಅಂತ-ರಾರ್ಥ
ಅಂತ-ರಾ-ಳದ
ಅಂತ-ರಾ-ಳ-ದೊ-ಳ-ಗಿಂದ
ಅಂತ-ರ್ಗ-ತ-ಗೊ-ಳಿಸಿ
ಅಂತ-ರ್ಗ-ತ-ವಾದ
ಅಂತ-ರ್ಜ-ಲ-ದಂತೆ
ಅಂತ-ರ್ದೃಷ್ಟಿ
ಅಂತ-ರ್ದೃ-ಷ್ಟಿ-ಇವು
ಅಂತ-ರ್ದೃ-ಷ್ಟಿ-ಇ-ವು-ಗಳು
ಅಂತ-ರ್ಮು-ಖ-ವಾ-ಗು-ತ್ತಿತ್ತು
ಅಂತ-ರ್ಮು-ಖಿ-ಗ-ಳಾ-ಗಿ-ದ್ದು-ಕೊಂಡು
ಅಂತ-ಶ್ಶಕ್ತಿ-ಯಲ್ಲಿ
ಅಂತ-ಸ್ತಿ-ನಲ್ಲಿ
ಅಂತ-ಸ್ಸತ್ತ್ವ-ವನ್ನು
ಅಂತ-ಸ್ಸತ್ವ
ಅಂತ-ಸ್ಸ-ತ್ವದ
ಅಂತ-ಸ್ಸತ್ವ-ವನ್ನೂ
ಅಂತ-ಸ್ಸಾ-ಕ್ಷಿ-ಯಾಗಿ
ಅಂತಹ
ಅಂತ-ಹ-ದನ್ನು
ಅಂತ-ಹ-ವ-ರಿಗೆ
ಅಂತ-ಹ-ವರು
ಅಂತ-ಹ-ವ-ರೊ-ಬ್ಬ-ರನ್ನು
ಅಂತಿಮ
ಅಂತಿ-ಮ-ವಾಗಿ
ಅಂತೂ
ಅಂತೆಯೇ
ಅಂತ್ಯ
ಅಂತ್ಯ-ಕಾಲ
ಅಂತ್ಯಕ್ಕೆ
ಅಂತ್ಯ-ಕ್ಕೆ-ವಿ-ಶ್ವ-ವಿ-ದ್ಯಾ-ಲ-ಯ-ಗಳ
ಅಂತ್ಯ-ಗೊ-ಳಿಸಿ
ಅಂತ್ಯ-ಗೊ-ಳ್ಳು-ತ್ತಿತ್ತು
ಅಂತ್ಯಜ
ಅಂತ್ಯ-ಜ-ನೊ-ಬ್ಬನ
ಅಂತ್ಯದ
ಅಂತ್ಯ-ದ-ವರೆ-ಗಿನ
ಅಂತ್ಯ-ದ-ವ-ರೆಗೂ
ಅಂತ್ಯ-ಭಾ-ಗದ
ಅಂಥ
ಅಂಥದು
ಅಂಥ-ದು-ಅಂ-ತಹ
ಅಂಥದೇ
ಅಂಥ-ದೊಂ-ದನ್ನು
ಅಂಥ-ದೊಂದು
ಅಂಥವನ
ಅಂಥವನು
ಅಂಥವನೇ
ಅಂಥವರ
ಅಂಥವ-ರ-ನ್ನೆಲ್ಲ
ಅಂಥವ-ರಲ್ಲ
ಅಂಥವ-ರಾ-ಗಿ-ದ್ದರು
ಅಂಥವ-ರಿಂದ
ಅಂಥವ-ರಿಗೆ
ಅಂಥವರು
ಅಂಥವು-ಗಳ
ಅಂಥವು-ಗಳನ್ನು
ಅಂಥವು-ಗ-ಳೆ-ಲ್ಲ-ದ-ರಿಂ-ದಲೂ
ಅಂಥವು-ಗ-ಳೊಂ-ದಿಗೆ
ಅಂಥಾ
ಅಂದರೂ
ಅಂದಿಗೆ
ಅಂದಿನ
ಅಂದಿ-ನಿಂದ
ಅಂದಿ-ನಿಂ-ದಲೂ
ಅಂದಿ-ನಿಂ-ದಲೇ
ಅಂದು
ಅಂದು-ಕೊಂ-ಡಂ-ತಹ
ಅಂದು-ಕೊಂ-ಡರು
ಅಂದು-ಕೊಂ-ಡಳು
ಅಂದು-ಕೊಂಡೆ
ಅಂದೇ
ಅಂದ್ರಲ್ಲ
ಅಂಧ-ಕಾರ
ಅಂಬಿ-ಗ-ರನ್ನು
ಅಂಬಿ-ಗರು
ಅಂಬೆ-ಗಾ-ಲಿ-ಡು-ವು-ದ-ರಲ್ಲೇ
ಅಂಶ
ಅಂಶ-ಗಳ
ಅಂಶ-ಗಳನ್ನು
ಅಂಶ-ಗ-ಳ-ನ್ನುಆ
ಅಂಶ-ಗಳಲ್ಲಿ
ಅಂಶ-ಗ-ಳಾದ
ಅಂಶ-ಗ-ಳಿ-ಗಿಂತ
ಅಂಶ-ಗ-ಳಿ-ವೆಯೋ
ಅಂಶ-ಗಳು
ಅಂಶ-ಗ-ಳೆಂದು
ಅಂಶ-ಗಳೇ
ಅಂಶ-ವನ್ನು
ಅಂಶ-ವನ್ನೂ
ಅಂಶ-ವಾ-ಗಿತ್ತು
ಅಂಶ-ವಾ-ಗಿದೆ
ಅಂಶ-ವಾದ
ಅಂಶ-ವಾ-ಯಿತು
ಅಂಶ-ವಿದೆ
ಅಂಶವೂ
ಅಂಶ-ವೆಂದರೆ
ಅಂಶವೇ
ಅಂಶ-ವೇನೆಂದರೆ
ಅಂಶ-ವೊಂ-ದರ
ಅಕ-ಳಂ-ಕ-ವೆಂ-ಬು-ದರ
ಅಕ-ಸ್ಮಾ-ತ್ತಾಗಿ
ಅಕಾ-ಲಿ-ಕ-ವಾಗಿ
ಅಕ್ಕ-ಪಕ್ಕ
ಅಕ್ಕ-ಪಕ್ಕದ
ಅಕ್ಕಿಯ
ಅಕ್ಕಿ-ರೊ-ಟ್ಟಿಯ
ಅಕ್ಟೋ-ಬ-ರಿನ
ಅಕ್ಟೋ-ಬರ್
ಅಕ್ಟೋ-ಬ-ರ್-ನ-ವೆಂ-ಬರ್
ಅಕ್ಬ-ರನ
ಅಕ್ಬ-ರ-ನದು
ಅಕ್ಬ-ರ-ನನ್ನೇ
ಅಕ್ರಮ
ಅಕ್ರ-ಮ-ವಾಗಿ
ಅಕ್ಷ-ಯ-ಕು-ಮಾರ
ಅಕ್ಷ-ಯ-ಕು-ಮಾರ್
ಅಕ್ಷ-ರಶಃ
ಅಕ್ಷ-ರ-ಸ್ಥ-ರಾ-ದ-ವರು
ಅಖಂಡ
ಅಖಂಡಾ
ಅಖಂಡಾನಂದರ
ಅಖಂಡಾನಂದ-ರನ್ನು
ಅಖಂಡಾನಂದ-ರಿಂದ
ಅಖಂಡಾನಂದ-ರಿ-ಗಂತೂ
ಅಖಂಡಾನಂದ-ರಿಗೆ
ಅಖಂಡಾನಂದರು
ಅಗಣ್ಯ
ಅಗತ್ಯ
ಅಗತ್ಯ-ವಿತ್ತು
ಅಗತ್ಯ-ವಿ-ದ್ದರೆ
ಅಗತ್ಯ-ವೆ-ನಿ-ಸಿ-ದಾಗ
ಅಗ-ಲಲೇ
ಅಗ-ಲ-ವಾ-ಗಿದೆ
ಅಗ-ಲಿ-ಸಿ-ಕೊಂಡು
ಅಗಾಧ
ಅಗಾ-ಧ-ತೆಯ
ಅಗಾ-ಧ-ವಾಗಿ
ಅಗಾ-ಧ-ವಾ-ದದ್ದು
ಅಗಾ-ಧ-ವಾ-ದುದು
ಅಗಿ-ಯುತ್ತ
ಅಗುಳಿ
ಅಗೋ-ಚ-ರ-ವಾಗಿ
ಅಗ್ಗಿ-ಷ್ಟಿ-ಕೆಯ
ಅಗ್ಗಿ-ಷ್ಟಿ-ಕೆ-ಯಲ್ಲಿ
ಅಗ್ನಿ
ಅಗ್ನಿ-ಕುಂಡ
ಅಗ್ನಿ-ಗಾ-ಹುತಿ
ಅಗ್ನಿ-ಜ್ವಾ-ಲೆ-ಯಂ-ತಹ
ಅಗ್ನಿ-ಜ್ವಾ-ಲೆ-ಯಂತೆ
ಅಗ್ನಿ-ಪರೀಕ್ಷೆ--ಯನ್ನೂ
ಅಗ್ನಿ-ಪರೀಕ್ಷೆ-ಯೇ
ಅಗ್ನಿಯ
ಅಗ್ನಿ-ಸ್ಪರ್ಶ
ಅಗ್ರ-ಗಣ್ಯ
ಅಗ್ರ-ಗ-ಣ್ಯ-ರಾದ
ಅಗ್ರ-ಗ-ಣ್ಯರು
ಅಗ್ರ-ಗ-ಣ್ಯರೇ
ಅಗ್ರ-ಸ್ಥಾನ
ಅಘಾತ
ಅಘೋ-ರ-ನಾಥ
ಅಚ-ಲ-ನಾ-ಗಿರು
ಅಚ-ಲ-ವಾ-ಗಿತ್ತು
ಅಚಾ-ತು-ರ್ಯ-ಗಳು
ಅಚಾ-ತು-ರ್ಯ-ವೊಂ-ದನ್ನು
ಅಚ್ಚರಿ
ಅಚ್ಚರಿ-ಗೊಂ-ಡರು
ಅಚ್ಚರಿ-ಗೊಂ-ಡರೂ
ಅಚ್ಚರಿ-ಗೊ-ಳ್ಳ-ದ-ವರೇ
ಅಚ್ಚರಿಯ
ಅಚ್ಚರಿ-ಯಿಂದ
ಅಚ್ಚರಿ-ಯಿಲ್ಲ
ಅಚ್ಚರಿ-ಯುಂ-ಟು-ಮಾ-ಡು-ವಷ್ಟು
ಅಚ್ಚರಿ-ಯೇ-ನಲ್ಲ
ಅಚ್ಚರಿ-ಯೇ-ನಿದೆ
ಅಚ್ಚರಿ-ಯೇ-ನಿಲ್ಲ
ಅಚ್ಚರಿ-ಯೇನೂ
ಅಚ್ಚರಿ-ಯೊಂ-ದಿಗೆ
ಅಚ್ಚ-ಳಿ-ಯದ
ಅಚ್ಚ-ಳಿ-ಯದೆ
ಅಚ್ಚಾ-ಗಿ-ಬಿ-ಟ್ಟಿದೆ
ಅಚ್ಚು
ಅಚ್ಚು-ಕ-ಟ್ಟಾಗಿ
ಅಚ್ಚು-ಕ-ಟ್ಟಾದ
ಅಚ್ಚು-ಕಟ್ಟು
ಅಚ್ಚು-ಮೆ-ಚ್ಚಾ-ಗಿತ್ತು
ಅಚ್ಚು-ಹಾ-ಕಿಸಿ
ಅಜ-ಗ-ಜಾಂ-ತರ
ಅಜಿತ್
ಅಜಿ-ತ್ಸಿಂ-ಗನ
ಅಜಿ-ತ್ಸಿಂ-ಗ-ನನ್ನು
ಅಜಿ-ತ್ಸಿಂ-ಗ-ನಿಗೆ
ಅಜಿ-ತ್ಸಿಂ-ಗನೂ
ಅಜಿ-ತ್ಸಿಂ-ಗ-ನೊಂ-ದಿಗೆ
ಅಜಿ-ತ್ಸಿಂ-ಗನ್ನು
ಅಜಿ-ತ್ಸಿಂ-ಗರು
ಅಜಿ-ತ್ಸಿಂಗ್
ಅಜಿ-ತ್ಸಿಂ-ಗ್ನೊಂ-ದಿಗೆ
ಅಜೇಯ
ಅಜ್ಜ
ಅಜ್ಜಿ
ಅಜ್ಞಾತ
ಅಜ್ಞಾ-ತ-ಅ-ನಾ-ಮ-ಧೇ-ಯ-ನಾ-ಗಿದ್ದ
ಅಜ್ಞಾ-ತ-ರಾ-ಗಿ-ರಲು
ಅಜ್ಞಾ-ತ-ವಾಸ
ಅಜ್ಞಾನ
ಅಜ್ಞಾನ
ಅಜ್ಞಾನ-ದಾರಿದ್ರ್ಯ
ಅಜ್ಞಾನ-ದಾರಿ-ದ್ರ್ಯ-ಗ-ಳಿಂ-ದಾಗಿ
ಅಜ್ಞಾನದ
ಅಜ್ಞಾನ-ದಾಚೆ
ಅಜ್ಞಾನ-ದಿಂದ
ಅಜ್ಞಾನ-ದಿಂ-ದಲೇ
ಅಜ್ಞಾನ-ವನ್ನು
ಅಜ್ಞಾನ-ವನ್ನೂ
ಅಜ್ಞಾನಸೇ
ಅಜ್ಞಾನಿ-ಗಳ
ಅಜ್ಞಾನಿ-ಗಳು
ಅಜ್ಞಾನು-ವರ್ತಿ
ಅಜ್ಞೇ-ಯ-ತಾ-ವಾದ
ಅಜ್ಮೀ-ರಕ್ಕೆ
ಅಜ್ಮೀ-ರದ
ಅಜ್ಮೀ-ರ-ದಲ್ಲಿ
ಅಜ್ಮೀ-ರ-ದ-ಲ್ಲಿನ
ಅಜ್ಮೀ-ರ-ದಿಂದ
ಅಜ್ಮೀ-ರಿಗೆ
ಅಜ್ಮೀರ್
ಅಟ್ಟ-ಕ್ಕೇ-ರಿಸಿ
ಅಟ್ಟಿ-ಸಿ-ಕೊಂಡು
ಅಟ್ಲಾಂ-ಟಿಕ್
ಅಡ-ಕ-ವಾಗಿ
ಅಡ-ಕ-ವಾ-ಗಿದ್ದ
ಅಡ-ಕ-ವಾ-ಗಿ-ರುವ
ಅಡ-ಕ-ವಾ-ಗಿವೆ
ಅಡಗಿದ್ದ
ಅಡಗಿ-ರುವ
ಅಡಗಿ-ರು-ವಂ-ಥವು
ಅಡಗಿ-ಸಿ-ಕೊ-ಳ್ಳುವ
ಅಡಗಿ-ಸಿ-ಬಿ-ಟ್ಟರು
ಅಡ-ಚ-ಣೆ-ಗಳನ್ನೆಲ್ಲ
ಅಡ-ಚ-ಣೆ-ಗ-ಳಿಂ-ದಲೇ
ಅಡ-ಚ-ಣೆ-ಗ-ಳಿಂ-ದಾಗಿ
ಅಡ-ಚ-ಣೆ-ಗ-ಳಿಂ-ದಾ-ಗಿ-ಅ-ಥವಾ
ಅಡ-ಚ-ಣೆ-ಗಳು
ಅಡ-ಚ-ಣೆ-ಯನ್ನೂ
ಅಡ-ಚ-ಣೆ-ಯಿ-ರು-ತ್ತಿ-ರ-ಲಿಲ್ಲ
ಅಡಿ
ಅಡಿ-ಗಲ್ಲು
ಅಡಿಗೆ
ಅಡಿ-ಗೆ-ಗಾ-ಗಲಿ
ಅಡಿ-ಗೆ-ಯನ್ನು
ಅಡಿ-ಗೆ-ಯನ್ನೂ
ಅಡಿ-ಗೆ-ಯ-ವ-ನನ್ನೂ
ಅಡಿ-ಗೆ-ಯ-ವ-ನಿಗೆ
ಅಡಿ-ಗೆ-ಯ-ವನು
ಅಡಿ-ಗೆ-ಯ-ವ-ನೊಂ-ದಿಗೆ
ಅಡಿ-ಗೆ-ಯ-ವ-ರು-ಧೋ-ಬಿ-ಗ-ಳ-ವ-ರೆಗೆ
ಅಡಿ-ಗೆ-ಯಾ-ಗಲು
ಅಡಿ-ಗೆ-ಯೂಟ
ಅಡಿ-ಪಾ-ಯದ
ಅಡಿ-ಪಾ-ಯ-ದಂ-ತಿ-ರುವ
ಅಡಿ-ಪಾ-ಯ-ವನ್ನು
ಅಡಿ-ಮೇಲು
ಅಡಿ-ಯಲ್ಲಿ
ಅಡಿ-ಯಾ-ಳಲ್ಲ
ಅಡಿ-ಯಾಳು
ಅಡಿ-ಯಿಂದ
ಅಡಿ-ಯಿ-ಟ್ಟಳು
ಅಡು-ಗೂ-ಲ-ಜ್ಜಿಯ
ಅಡೆ-ತ-ಡೆ-ಗಳನ್ನು
ಅಡೆ-ತ-ಡೆ-ಗಳನ್ನೂ
ಅಡೆ-ತ-ಡೆ-ಗಳನ್ನೆಲ್ಲ
ಅಡೆ-ತ-ಡೆ-ಗಳು
ಅಡೆ-ತ-ಡೆ-ಯಿ-ಲ್ಲ-ದಂತೆ
ಅಡೆ-ತ-ಡೆ-ಯಿ-ಲ್ಲದೆ
ಅಡ್ಡ-ಗೋ-ಡೆ-ಗಳ
ಅಡ್ಡ-ಗೋ-ಡೆಯು
ಅಡ್ಡ-ವಾಗಿ
ಅಡ್ಡಾ-ಡಲು
ಅಡ್ಡಾಡಿ
ಅಡ್ಡಾ-ಡಿ-ಕೊಂ-ಡು-ಬ-ರಲು
ಅಡ್ಡಾ-ಡು-ತ್ತಿದ್ದ
ಅಡ್ಡಾ-ಡು-ತ್ತಿ-ದ್ದರು
ಅಡ್ಡಾ-ಡು-ತ್ತಿ-ದ್ದಾಗ
ಅಡ್ಡಾ-ಡು-ವಾಗ
ಅಡ್ಡಿ
ಅಡ್ಡಿ-ಗಳನ್ನು
ಅಡ್ವ-ರ್ಟೈ-ಸರ್
ಅಣ-ಕ-ವಾಗಿ
ಅಣ-ಕ-ವಾ-ಡುತ್ತ
ಅಣ-ಕ-ವಿದು
ಅಣ-ಕಿ-ಸು-ತ್ತಿ-ದ್ದರು
ಅಣ-ತಿ-ಯಂ-ತೇನೂ
ಅಣಿ-ಯಾ-ಗು-ವಂತೆ
ಅಣಿ-ಯಾ-ದರು
ಅಣ್ಣ
ಅಣ್ಣ-ತಂ-ಗಿ-ಯ-ರಾದ
ಅಣ್ಣ-ತ-ಮ್ಮಂದಿ
ಅಣ್ಣ-ತ-ಮ್ಮಂ-ದಿರು
ಅಣ್ಣ-ನಂತೆ
ಅತಂ-ತ್ರ-ರಾ-ದರು
ಅತ-ರ್ಕ-ವ-ನ್ನಾ-ಗಿಯೂ
ಅತಿ
ಅತಿ-ಕ್ರ-ಮಿಸಿ
ಅತಿ-ಖಾ-ರದ
ಅತಿಥಿ
ಅತಿ-ಥಿ-ಗಳ
ಅತಿ-ಥಿ-ಗಳನ್ನು
ಅತಿ-ಥಿ-ಗ-ಳ-ನ್ನು-ದ್ದೇ-ಶಿಸಿ
ಅತಿ-ಥಿ-ಗ-ಳಾಗಿ
ಅತಿ-ಥಿ-ಗ-ಳಾ-ಗಿದ್ದ
ಅತಿ-ಥಿ-ಗ-ಳಾ-ಗಿ-ದ್ದರು
ಅತಿ-ಥಿ-ಗ-ಳಾ-ಗಿ-ದ್ದಾರೆ
ಅತಿ-ಥಿ-ಗ-ಳಾ-ಗಿ-ರು-ತ್ತಿ-ದ್ದರು
ಅತಿ-ಥಿ-ಗ-ಳಾದ
ಅತಿ-ಥಿ-ಗ-ಳೆ-ಲ್ಲರೂ
ಅತಿ-ಥಿ-ಗ-ಳೊಂ-ದಿಗೆ
ಅತಿ-ಥಿ-ಯನ್ನು
ಅತಿ-ಥಿ-ಯಾಗಿ
ಅತಿ-ಥಿ-ಯಾ-ಗಿದ್ದ
ಅತಿ-ಥಿ-ಯಾ-ಗಿ-ದ್ದರು
ಅತಿ-ಥಿ-ಯಾ-ಗಿ-ದ್ದ-ವರು
ಅತಿ-ಥಿ-ಯಾ-ಗಿ-ದ್ದು-ಕೊಂಡು
ಅತಿ-ಥಿ-ಯಾ-ಗಿ-ದ್ದುದು
ಅತಿ-ಥಿ-ಯಾ-ಗಿ-ರಿ-ಸಿ-ಕೊಂ-ಡಿ-ದ್ದ-ರಿಂದ
ಅತಿ-ಥಿ-ಯಾ-ಗಿ-ರು-ವಂತೆ
ಅತಿ-ಥಿ-ಯಾದ
ಅತಿ-ಥೇ-ಯ-ನಿಗೆ
ಅತಿ-ಥೇ-ಯ-ರಾದ
ಅತಿಥ್ಯ
ಅತಿ-ಥ್ಯ-ದಲ್ಲಿ
ಅತಿ-ದೊಡ್ಡ
ಅತಿ-ಮಾ-ನುಷ
ಅತಿ-ಮುಖ್ಯ
ಅತಿ-ಮು-ಖ್ಯ-ವಾಗಿ
ಅತಿ-ಮೃ-ತ್ಯು-ಮೇತಿ
ಅತಿ-ಯಾದ
ಅತಿ-ಯಾ-ದಾಗ
ಅತಿ-ಯಾ-ಯಿ-ತೇನೋ
ಅತಿ-ರೇ-ಕದ
ಅತಿ-ವೇ-ಗ-ದಿಂದ
ಅತಿ-ಶಯ
ಅತಿ-ಶ-ಯ-ವಾಗಿ
ಅತಿ-ಶ-ಯ-ವಾದ
ಅತಿ-ಶ-ಯ-ವಾ-ದದ್ದು
ಅತಿ-ಶ-ಯೋಕ್ತಿ
ಅತಿ-ಶ-ಯೋ-ಕ್ತಿ-ಪೂ-ರ್ಣ-ವಾದ
ಅತಿ-ಶ-ಯೋ-ಕ್ತಿ-ಯಲ್ಲ
ಅತಿ-ಶ್ರ-ಮ-ದಿಂದ
ಅತಿ-ಶ್ರೇಷ್ಠ
ಅತಿ-ಶ್ರೇ-ಷ್ಠ-ವೆಂದು
ಅತಿ-ಸೂಕ್ಷ್ಮ
ಅತಿ-ಹೆ-ಚ್ಚಿನ
ಅತಿ-ಹೆಚ್ಚು
ಅತೀಂ-ದ್ರಿಯ
ಅತೀಂ-ದ್ರಿ-ಯ-ಜ್ಯೋ-ತಿ-ರ್ಮಯ
ಅತೀ-ತ-ವಾ-ದದ್ದು
ಅತೀವ
ಅತುಲ
ಅತು-ಲ-ಚಂದ್ರ
ಅತೃ-ಪ್ತ-ಮ-ನ-ಸ್ಕ-ರಾ-ಗಿ-ದ್ದರು
ಅತೃ-ಪ್ತಿ-ಯಿಂ-ದಲೇ
ಅತ್ತ
ಅತ್ತದ್ದೂ
ಅತ್ತಿಂ-ದಿತ್ತ
ಅತ್ತು
ಅತ್ತು-ಬಿ-ಟ್ಟರು
ಅತ್ಯಂತ
ಅತ್ಯಂ-ತಿಕ
ಅತ್ಯ-ಗತ್ಯ
ಅತ್ಯ-ದ್ಭುತ
ಅತ್ಯ-ದ್ಭು-ತ-ವಾದ
ಅತ್ಯ-ದ್ಭು-ತ-ವಾ-ದದ್ದು
ಅತ್ಯ-ದ್ಭು-ತ-ವಾ-ದುವು
ಅತ್ಯ-ಧಿ-ಕ-ವಾ-ಗಿ-ರು-ತ್ತಿತ್ತು
ಅತ್ಯ-ಮೂಲ್ಯ
ಅತ್ಯ-ಮೂ-ಲ್ಯ-ವಾ-ಗಿತ್ತು
ಅತ್ಯ-ಮೂ-ಲ್ಯ-ವಾದ
ಅತ್ಯ-ಮೂ-ಲ್ಯ-ವಾ-ದುದೇ
ಅತ್ಯಾ-ಚಾ-ರ-ಅ-ನಾ-ಚಾ-ರ-ಗಳನ್ನು
ಅತ್ಯಾ-ದ-ರ-ದಿಂದ
ಅತ್ಯಾ-ಧು-ನಿಕ
ಅತ್ಯಾ-ಧು-ನಿ-ಕ-ವಾದ
ಅತ್ಯಾ-ನಂ-ದ-ಗೊಂ-ಡಿ-ದ್ದ-ರಲ್ಲಿ
ಅತ್ಯಾ-ನಂ-ದ-ದಿಂದ
ಅತ್ಯಾ-ನಂ-ದ-ವಾ-ಯಿತು
ಅತ್ಯಾ-ವ-ಶ್ಯಕ
ಅತ್ಯಾ-ವ-ಶ್ಯ-ಕ-ವಾಗಿ
ಅತ್ಯಾ-ವ-ಶ್ಯ-ಕ-ವಾ-ಗಿ-ದೆಯೋ
ಅತ್ಯಾ-ವ-ಶ್ಯ-ಕ-ವಾದ
ಅತ್ಯಾ-ಶ್ಚರ್ಯ
ಅತ್ಯಾ-ಶ್ಚ-ರ್ಯ-ಕ-ರ-ವಾ-ಗಿತ್ತು
ಅತ್ಯಾ-ಶ್ಚ-ರ್ಯದ
ಅತ್ಯಾ-ಶ್ಚ-ರ್ಯ-ವಾಗ
ಅತ್ಯಾ-ಶ್ಚ-ರ್ಯ-ವಾ-ಗು-ತ್ತದೆ
ಅತ್ಯಾ-ಶ್ಚ-ರ್ಯ-ವಾ-ಯಿತು
ಅತ್ಯು
ಅತ್ಯು-ಗ್ರ-ವಾದ
ಅತ್ಯು-ಚ್ಚ-ವಾ-ದುದು
ಅತ್ಯು-ಜ್ವಲ
ಅತ್ಯು-ತ್ಕೃಷ್ಟ
ಅತ್ಯು-ತ್ಕೃ-ಷ್ಟ-ರಾದ
ಅತ್ಯು-ತ್ತಮ
ಅತ್ಯು-ತ್ತ-ಮ-ವಾಗಿ
ಅತ್ಯು-ತ್ತ-ಮ-ವಾ-ದದ್ದು
ಅತ್ಯು-ದ್ಭು-ತ-ವಾ-ಗಿತ್ತು
ಅತ್ಯು-ನ್ನತ
ಅತ್ಯು-ನ್ನ-ತ-ಮ-ಟ್ಟದ
ಅತ್ಯು-ನ್ನ-ತ-ವಾದ
ಅಥ-ರ್ವಣ
ಅಥವಾ
ಅದ
ಅದ-ಕ್ಕ-ನು-ಗು-ಣ-ವಾಗಿ
ಅದ-ಕ್ಕ-ನು-ಗು-ಣ-ವಾದ
ಅದ-ಕ್ಕ-ನು-ಸಾ-ರ-ವಾಗಿ
ಅದ-ಕ್ಕ-ವನು
ಅದ-ಕ್ಕಾಗಿ
ಅದ-ಕ್ಕಿಂತ
ಅದ-ಕ್ಕಿಂ-ತಲೂ
ಅದ-ಕ್ಕಿನ್ನೂ
ಅದ-ಕ್ಕು-ತ್ತ-ರ-ವಾಗಿ
ಅದಕ್ಕೂ
ಅದಕ್ಕೆ
ಅದ-ಕ್ಕೆಲ್ಲ
ಅದಕ್ಕೇ
ಅದ-ಕ್ಕೇಕೆ
ಅದ-ಕ್ಕೊಂದು
ಅದ-ಕ್ಕೊ-ಪ್ಪದೆ
ಅದ-ಕ್ಕೊಪ್ಪಿ
ಅದ-ಕ್ಕೊ-ಪ್ಪುವ
ಅದದೇ
ಅದನ್ನ
ಅದ-ನ್ನ-ವರು
ಅದನ್ನು
ಅದ-ನ್ನುಆ
ಅದನ್ನೂ
ಅದ-ನ್ನೂ-ಪ್ರೀ-ತಿ-ಸುವ
ಅದ-ನ್ನೆ-ದು-ರಿ-ಸ-ಲಾ-ರದೆ
ಅದ-ನ್ನೆ-ದು-ರಿಸಿ
ಅದ-ನ್ನೆಲ್ಲ
ಅದನ್ನೇ
ಅದ-ನ್ನೇ-ದಾರೂ
ಅದ-ನ್ನೊಂದು
ಅದ-ನ್ನೋ-ದಿ-ದ-ವ-ರಿಗೆ
ಅದಮ್ಯ
ಅದರ
ಅದ-ರಂ-ತಯೇ
ಅದ-ರಂತೆ
ಅದ-ರಂ-ತೆಯೇ
ಅದ-ರದ್ದೇ
ಅದ-ರ-ರ್ಥ-ವಲ್ಲ
ಅದ-ರ-ಲ್ಲ-ವರು
ಅದ-ರಲ್ಲಿ
ಅದ-ರ-ಲ್ಲಿತ್ತು
ಅದ-ರ-ಲ್ಲಿದೆ
ಅದ-ರಲ್ಲೂ
ಅದ-ರಲ್ಲೇ
ಅದ-ರ-ಲ್ಲೇನೂ
ಅದ-ರಿಂದ
ಅದ-ರಿಂ-ದಲೂ
ಅದ-ರಿಂ-ದಲೇ
ಅದ-ರಿಂ-ದಾಗಿ
ಅದ-ರಿಂ-ದಾದ
ಅದ-ರಿಂ-ದೇ-ನಾ-ಯಿತು
ಅದ-ರಿಂ-ದೇನು
ಅದ-ರಿಂ-ದೇನೂ
ಅದ-ರು-ತ್ತಿತ್ತು
ಅದರೆ
ಅದ-ರೊಂ-ದಿಗೆ
ಅದ-ರೊಂ-ದಿಗೇ
ಅದ-ರೊ-ಳಕ್ಕೆ
ಅದ-ರೊ-ಳಗೆ
ಅದಲ್ಲ
ಅದಲ್ಲಿ
ಅದಲ್ಲೂ
ಅದ-ವನ
ಅದಾ-ಕ-ಡೆ-ಗಿ-ರಲಿ
ಅದಾ-ಗಲೇ
ಅದಾಗಿ
ಅದಾ-ಗಿತ್ತು
ಅದಾದ
ಅದಾ-ದ-ನಂ-ತರ
ಅದಾ-ವು-ದನ್ನೂ
ಅದಿನ್ನು
ಅದಿನ್ನೂ
ಅದಿ-ರಲಿ
ಅದಿ-ರು-ವುದು
ಅದಿ-ಲ್ಲವೋ
ಅದು
ಅದು-ರ-ಲಾ-ರಂ-ಭಿ-ಸಿತು
ಅದು-ವರೆ-ಗಿನ
ಅದು-ಷ್ಟ-ರ-ನ್ನಾ-ಗಿಯೂ
ಅದು-ಷ್ಟ-ರನ್ನು
ಅದೂ
ಅದೃ-ಶ್ಯ-ರಾ-ದರು
ಅದೃಷ್ಟ
ಅದೃ-ಷ್ಟ-ಆ-ನಂದ
ಅದೃ-ಷ್ಟಕ್ಕೆ
ಅದೃ-ಷ್ಟ-ವನ್ನು
ಅದೃ-ಷ್ಟ-ವ-ಶಾತ್
ಅದೃ-ಷ್ಟ-ವಿಲ್ಲ
ಅದೆಂ-ತಹ
ಅದೆಂಥ
ಅದೆಂ-ದಿಗೂ
ಅದೆಂದೂ
ಅದೆಲ್ಲ
ಅದೆ-ಲ್ಲಕ್ಕೂ
ಅದೆ-ಲ್ಲ-ವನ್ನೂ
ಅದೆ-ಲ್ಲವೂ
ಅದೆ-ಲ್ಲ-ವೂಆ
ಅದೆ-ಷ್ಟಿದೆ
ಅದೆಷ್ಟು
ಅದೆಷ್ಟೇ
ಅದೆಷ್ಟೋ
ಅದೇ
ಅದೇಕೆ
ಅದೇಕೋ
ಅದೇನು
ಅದೇನೂ
ಅದೇ-ನೆಂ-ದರೆ
ಅದೇ-ನೆಂ-ಬುದು
ಅದೇನೇ
ಅದೇನೋ
ಅದೊಂ-ದಿ-ದ್ದರೆ
ಅದೊಂದು
ಅದೊಂದೇ
ಅದ್ದ-ರಿಂದ
ಅದ್ದಿ-ಟ್ಟು-ಕೊಂ-ಡಿ-ರು-ತ್ತಿದ್ದೆ
ಅದ್ದೂ-ರಿಯ
ಅದ್ಭುತ
ಅದ್ಭು-ತ-ಅ-ಪೂರ್ವ
ಅದ್ಭು-ತ-ಅ-ಸಾ-ಧಾ-ರಣ
ಅದ್ಭು-ತ-ಕೌ-ತು-ಕದ
ಅದ್ಭು-ತ-ರ-ಮ್ಯ-ವಾ-ಗಿ-ದ್ದುವು
ಅದ್ಭು-ತ-ಸಂ-ಗೀ-ತ-ಮ-ಯ-ಹಾ-ರ್ಪ್
ಅದ್ಭು-ತ-ಸುಂ-ದರ
ಅದ್ಭು-ತ-ಗಳನ್ನು
ಅದ್ಭು-ತ-ಗ-ಳ-ಲ್ಲೊ-ಬ್ಬರು
ಅದ್ಭು-ತ-ವನ್ನು
ಅದ್ಭು-ತ-ವನ್ನೇ
ಅದ್ಭು-ತ-ವ-ಲ್ಲದೆ
ಅದ್ಭು-ತ-ವ-ಲ್ಲವೆ
ಅದ್ಭು-ತ-ವಾಗಿ
ಅದ್ಭು-ತ-ವಾ-ಗಿತ್ತು
ಅದ್ಭು-ತ-ವಾ-ಗಿ-ತ್ತೆಂ-ಬು-ದನ್ನು
ಅದ್ಭು-ತ-ವಾ-ಗಿದೆ
ಅದ್ಭು-ತ-ವಾ-ಗಿ-ರು-ತ್ತಿ-ದ್ದು-ವೆಂದರೆ
ಅದ್ಭು-ತ-ವಾದ
ಅದ್ಭು-ತ-ವಾ-ದ-ದ್ದನ್ನು
ಅದ್ಭು-ತ-ವಾ-ದದ್ದು
ಅದ್ಭು-ತವೂ
ಅದ್ಭು-ತವೇ
ಅದ್ಭು-ತ-ವೊಂದು
ಅದ್ಯಾವ
ಅದ್ವಿ-ತೀಯ
ಅದ್ವಿ-ತೀ-ಯ-ನಾಗಿ
ಅದ್ವಿ-ತೀ-ಯ-ರಾ-ದ-ವರು
ಅದ್ವಿ-ತೀ-ಯ-ವಾ-ದುದು
ಅದ್ವೈತ
ಅದ್ವೈ-ತ-ಗಳಲ್ಲಿ
ಅದ್ವೈ-ತದ
ಅದ್ವೈ-ತ-ವನ್ನು
ಅದ್ವೈ-ತ-ವಾ-ದ-ವನ್ನು
ಅದ್ವೈ-ತ-ವೆಂದರೆ
ಅದ್ವೈ-ತ-ಸಿ-ದ್ಧಾಂ-ತ-ವನ್ನು
ಅದ್ವೈ-ತಾ-ಶ್ರ-ಮ-ವನ್ನು
ಅದ್ವೈತಿ
ಅದ್ವೈ-ತಿಯು
ಅದ್ವೈ-ತಿ-ಯೆಂದು
ಅಧಃ-ಪ-ತ-ನದ
ಅಧಃ-ಪ-ತ-ನ-ದೆ-ಡೆಗೆ
ಅಧಃ-ಪ-ತ-ನವೆ
ಅಧಃ-ಪಾ-ತಾ-ಳಕ್ಕೆ
ಅಧ-ಕ್ಷ-ರಾ-ಗಲಿ
ಅಧಾ-ರ್ಮಿಕ
ಅಧಿಕ
ಅಧಿ-ಕ-ಪ್ರ-ಸಂಗ
ಅಧಿ-ಕ-ಪ್ರ-ಸಂ-ಗದ
ಅಧಿ-ಕ-ವಾಗಿ
ಅಧಿ-ಕ-ವಾ-ಗುತ್ತ
ಅಧಿ-ಕ-ವಾ-ಗು-ತ್ತ-ಬಂತು
ಅಧಿ-ಕಾರ
ಅಧಿ-ಕಾ-ರ-ಸಂ-ಪತ್ತು
ಅಧಿ-ಕಾ-ರ-ಹಣ
ಅಧಿ-ಕಾ-ರ-ಕ್ಕಾಗಿ
ಅಧಿ-ಕಾ-ರ-ದ-ಲ್ಲಿ-ರುವ
ಅಧಿ-ಕಾ-ರ-ಯು-ತ-ವಾದ
ಅಧಿ-ಕಾ-ರ-ವನ್ನೂ
ಅಧಿ-ಕಾ-ರ-ವ-ರ್ಗ-ದಿಂ-ದಲೂ
ಅಧಿ-ಕಾ-ರ-ವಾ-ಣಿ-ಯಲ್ಲಿ
ಅಧಿ-ಕಾ-ರ-ವಾ-ಣಿ-ಯಿಂದ
ಅಧಿ-ಕಾ-ರ-ವಿಲ್ಲ
ಅಧಿ-ಕಾ-ರ-ವುಂಟು
ಅಧಿ-ಕಾರಿ
ಅಧಿ-ಕಾ-ರಿ-ಗಳ
ಅಧಿ-ಕಾ-ರಿ-ಗಳನ್ನು
ಅಧಿ-ಕಾ-ರಿ-ಗಳನ್ನೂ
ಅಧಿ-ಕಾ-ರಿ-ಗ-ಳಾ-ಗಿ-ದ್ದ-ರ-ಲ್ಲದೆ
ಅಧಿ-ಕಾ-ರಿ-ಗ-ಳಿಗೆ
ಅಧಿ-ಕಾ-ರಿ-ಗಳು
ಅಧಿ-ಕಾ-ರಿ-ಗ-ಳು-ಇವ
ಅಧಿ-ಕಾ-ರಿ-ಗ-ಳು-ಮುಖ್ಯ
ಅಧಿ-ಕಾ-ರಿ-ಗಳೂ
ಅಧಿ-ಕಾ-ರಿ-ಗ-ಳೊಂ-ದಿಗೆ
ಅಧಿ-ಕಾ-ರಿಯ
ಅಧಿ-ಕಾ-ರಿ-ಯನ್ನು
ಅಧಿ-ಕಾ-ರಿ-ಯಾ-ಗಿದ್ದ
ಅಧಿ-ಕಾ-ರಿ-ಯಾದ
ಅಧಿ-ಕಾ-ರಿ-ಯೊಬ್ಬ
ಅಧಿ-ಕಾ-ರಿಯೋ
ಅಧಿ-ಕೃತ
ಅಧಿ-ಕೃ-ತ-ವಾಗಿ
ಅಧಿ-ವೇ-ಶ-ನ-ಗಳ
ಅಧಿ-ವೇ-ಶ-ನದ
ಅಧಿ-ವೇ-ಶ-ನ-ದಲ್ಲಿ
ಅಧಿ-ವೇ-ಶ-ನ-ವೊಂ-ದರ
ಅಧೀ-ರ-ರಾ-ಗ-ಲಿಲ್ಲ
ಅಧೀ-ರ-ರಾ-ದರು
ಅಧೈ-ರ್ಯ-ಗೊಂ-ಡಂತೆ
ಅಧೈ-ರ್ಯ-ವನ್ನು
ಅಧ್ಯಕ್ಷ
ಅಧ್ಯ-ಕ್ಷಣಿ
ಅಧ್ಯ-ಕ್ಷ-ತೆ-ಯನ್ನು
ಅಧ್ಯ-ಕ್ಷ-ತೆ-ಯಲ್ಲಿ
ಅಧ್ಯ-ಕ್ಷ-ನ-ನ್ನಾಗಿ
ಅಧ್ಯ-ಕ್ಷ-ನಾಗಿ
ಅಧ್ಯ-ಕ್ಷ-ನಾ-ಗಿದ್ದ
ಅಧ್ಯ-ಕ್ಷ-ನಾದ
ಅಧ್ಯ-ಕ್ಷ-ಭಾ-ಷ-ಣದ
ಅಧ್ಯ-ಕ್ಷರ
ಅಧ್ಯ-ಕ್ಷ-ರ-ನ್ನಾಗಿ
ಅಧ್ಯ-ಕ್ಷ-ರಾಗಿ
ಅಧ್ಯ-ಕ್ಷ-ರಾ-ಗಿದ್ದ
ಅಧ್ಯ-ಕ್ಷ-ರಾ-ಗಿ-ದ್ದರು
ಅಧ್ಯ-ಕ್ಷ-ರಾದ
ಅಧ್ಯ-ಕ್ಷ-ರಿಗೆ
ಅಧ್ಯ-ಕ್ಷರು
ಅಧ್ಯ-ಕ್ಷ-ಸ್ಥಾನ
ಅಧ್ಯ-ಕ್ಷ-ಸ್ಥಾ-ನಕ್ಕೆ
ಅಧ್ಯ-ಕ್ಷ-ಸ್ಥಾ-ನ-ವನ್ನು
ಅಧ್ಯ-ಕ್ಷಿ-ಣಿ-ಯಾ-ಗಿ-ದ್ದಳು
ಅಧ್ಯ-ಕ್ಷಿ-ಣಿ-ಯಾದ
ಅಧ್ಯ-ಯನ
ಅಧ್ಯ-ಯ-ನ
ಅಧ್ಯ-ಯ-ನ-ಚ-ರ್ಚೆ-ಗಳು
ಅಧ್ಯ-ಯ-ನದ
ಅಧ್ಯ-ಯ-ನ-ದಲ್ಲಿ
ಅಧ್ಯ-ಯ-ನ-ದಿಂದ
ಅಧ್ಯ-ಯ-ನ-ಯೋಗ್ಯ
ಅಧ್ಯ-ಯ-ನ-ವನ್ನು
ಅಧ್ಯ-ಯ-ನಾ-ದಿ-ಗಳ
ಅಧ್ಯ-ಯ-ನಾ-ದಿ-ಗಳಲ್ಲಿ
ಅಧ್ಯ-ಯಿ-ಸಿ-ದರು
ಅಧ್ಯಾ-ತ್ಮ-ಸಂ-ಸಾ-ರ-ಶಾ-ಸ್ತ್ರ-ಗಳು
ಅಧ್ಯಾ-ತ್ಮದ
ಅಧ್ಯಾ-ತ್ಮ-ವನ್ನು
ಅಧ್ಯಾ-ತ್ಮ-ವ-ನ್ನು-ಎಂ-ದರೆ
ಅಧ್ಯಾ-ತ್ಮಿಕ
ಅಧ್ಯಾ-ತ್ಮಿ-ಕ-ತ-ತ್ತ್ವದ
ಅಧ್ಯಾ-ಪಕ
ಅಧ್ಯಾ-ಪ-ಕರು
ಅಧ್ಯಾ-ಪ-ಕಳು
ಅಧ್ಯಾ-ಪ-ಕಿ-ಯಾಗಿಯೂ
ಅಧ್ಯಾ-ಪಕ್ಜಿ
ಅಧ್ಯಾಯ
ಅಧ್ಯಾ-ಯಕ್ಕೆ
ಅಧ್ಯಾ-ಯಕ್ಕೇ
ಅಧ್ಯಾ-ಯ-ಗಳು
ಅನಂತ
ಅನಂ-ತ-ತೆಯ
ಅನಂ-ತ-ನಾದ
ಅನಂ-ತರ
ಅನಂ-ತ-ರ-ವಾ-ದರೂ
ಅನಂ-ತ-ರವೂ
ಅನಂ-ತ-ರವೇ
ಅನಂ-ತ-ರವೋ
ಅನಂ-ತ-ವಾ-ಗಿ-ರ-ಬೇ-ಕ-ಲ್ಲದೆ
ಅನಂ-ತ-ಸಿ-ದ್ಧಿಯ
ಅನ-ಕ್ಷ-ರಸ್ಥ
ಅನ-ನು-ಕ-ರ-ಣೀಯ
ಅನ-ನು-ಕೂ-ಲತೆ
ಅನ-ನು-ಕೂ-ಲ-ತೆ-ಗಳು
ಅನ-ನು-ಕೂ-ಲ-ತೆ-ಗಳೂ
ಅನ-ನು-ಕೂ-ಲ-ವಾ-ಯಿ-ತಾ-ದರೂ
ಅನ-ನ್ಯ-ರಾಗಿ
ಅನ-ನ್ಯ-ರೆಂಬ
ಅನರ್ಘ್ಯ
ಅನ-ರ್ಘ್ಯ-ರತ್ನ
ಅನ-ರ್ಘ್ಯ-ವಾದ
ಅನ-ರ್ಘ್ಯವೂ
ಅನ-ರ್ಥಕ್ಕೆ
ಅನ-ವ-ರತ
ಅನ-ವ-ರ-ತವೂ
ಅನಾ-ಗ-ರಿಕ
ಅನಾ-ಗ-ರಿ-ಕನ
ಅನಾ-ಗ-ರಿ-ಕರ
ಅನಾ-ಗ-ರಿ-ಕ-ರನ್ನು
ಅನಾ-ಗ-ರಿ-ಕ-ರ-ನ್ನೆಲ್ಲ
ಅನಾ-ಗ-ರಿ-ಕ-ರಿ-ಗಾಗಿ
ಅನಾ-ಗ-ರಿ-ಕ-ರೆಂದೂ
ಅನಾ-ಗ-ರಿ-ಕ-ವೆಂದು
ಅನಾಥ
ಅನಾ-ದ-ರಿಸಿ
ಅನಾದಿ
ಅನಾ-ದಿ-ಯಾದ
ಅನಾ-ದಿ-ಯಾ-ದುದು
ಅನಾ-ದಿ-ಯಿಂದ
ಅನಾ-ದಿ-ಯಿಂ-ದಲೂ
ಅನಾ-ಮ-ಧೇಯ
ಅನಾ-ಮ-ಧೇ-ಯ-ರಾ-ಗಿದ್ದ
ಅನಾ-ಮ-ಧೇ-ಯ-ರಾ-ಗಿ-ರಲು
ಅನಾ-ಮಿ-ಕ-ರಾ-ಗಿಯೇ
ಅನಾ-ರೋ-ಗ್ಯದ
ಅನಾ-ರೋ-ಗ್ಯ-ವಾ-ಗಿತ್ತು
ಅನಾ-ರೋ-ಗ್ಯ-ವುಂ-ಟಾಗಿ
ಅನಾ-ವ-ಶ್ಯ-ಕ-ವಾಗಿ
ಅನಾ-ಸ-ಕ್ತ-ರಾ-ಗಿ-ದ್ದರೋ
ಅನಾ-ಹುತ
ಅನಿ-ರೀ-ಕ್ಷಿತ
ಅನಿ-ರೀ-ಕ್ಷಿ-ತ-ವಾಗಿ
ಅನಿ-ರೀ-ಕ್ಷಿ-ತ-ವಾ-ಗಿತ್ತು
ಅನಿ-ರೀ-ಕ್ಷಿ-ತ-ವಾ-ಗಿ-ರು-ತ್ತಿತ್ತು
ಅನಿ-ರೀ-ಕ್ಷಿ-ತ-ವಾ-ದುದೇ
ಅನಿ-ರ್ಬಂ-ಧಿತ
ಅನಿ-ರ್ಬಂ-ಧಿ-ತ-ವಾಗಿ
ಅನಿ-ರ್ವ-ಚ-ನೀಯ
ಅನಿ-ವಾರ್ಯ
ಅನಿ-ವಾ-ರ್ಯ-ವಾ-ಗಿತ್ತು
ಅನಿ-ಶ್ಚಿ-ತ-ತೆ-ಯಿಂದ
ಅನಿ-ಷ್ಟ-ಗ-ಳಂತೆ
ಅನಿ-ಸಿಕೆ
ಅನಿ-ಸಿ-ಕೆ-ಗಳನ್ನು
ಅನಿ-ಸಿ-ಕೆ-ಯನ್ನು
ಅನಿ-ಸಿ-ಕೆಯೂ
ಅನು
ಅನು-ಕಂಪೆ
ಅನು-ಕಂ-ಪೆ-ಪ್ರೇ-ಮ-ಗ-ಳಿಗೆ
ಅನು-ಕಂ-ಪೆ-ಇ-ವು-ಗ-ಳೆಲ್ಲ
ಅನು-ಕಂ-ಪೆಯ
ಅನು-ಕಂ-ಪೆ-ಯನ್ನು
ಅನು-ಕಂ-ಪೆ-ಯನ್ನೂ
ಅನು-ಕಂ-ಪೆ-ಯಿಂದ
ಅನು-ಕಂ-ಪೆ-ಯಿ-ಲ್ಲ-ದಿರು
ಅನು-ಕಂ-ಪೆ-ಯೊಂ-ದಿಗೆ
ಅನು-ಕ-ರಣೆ
ಅನು-ಕ-ರಿ-ಸಲು
ಅನು-ಕೂಲ
ಅನು-ಕೂ-ಲ-ಕರ
ಅನು-ಕೂ-ಲ-ಕ-ರ-ವಾಗಿ
ಅನು-ಕೂ-ಲ-ಕ-ರ-ವಾದ
ಅನು-ಕೂ-ಲಕ್ಕೆ
ಅನು-ಕೂ-ಲ-ಗ-ಳಿ-ರ-ಬ-ಹುದು
ಅನು-ಕೂ-ಲತೆ
ಅನು-ಕೂ-ಲ-ತೆ-ಗಳನ್ನು
ಅನು-ಕೂ-ಲ-ತೆ-ಗಳನ್ನೆಲ್ಲ
ಅನು-ಕೂ-ಲ-ತೆ-ಗ-ಳಿ-ದ್ದುವು
ಅನು-ಕೂ-ಲ-ತೆ-ಯಿ-ರ-ಲಿಲ್ಲ
ಅನು-ಕೂ-ಲ-ವಾ-ಗಿತ್ತು
ಅನು-ಕೂ-ಲ-ವಾ-ಗು-ವಂತೆ
ಅನು-ಕೂ-ಲ-ವಾದ
ಅನು-ಕೂ-ಲ-ವಾ-ದದ್ದು
ಅನು-ಕೂ-ಲ-ವಾ-ದರೂ
ಅನು-ಕೂ-ಲ-ವಿ-ರ-ಲಿಲ್ಲ
ಅನು-ಕ್ಷ-ಣವೂ
ಅನು-ಗು-ಣ-ವಾಗಿ
ಅನು-ಗು-ಣ-ವಾ-ಗಿ-ರ-ಬೇಕೇ
ಅನು-ಗು-ಣ-ವಾ-ಗಿ-ಲ್ಲ-ವೆಂದು
ಅನು-ಗು-ಣ-ವಾದ
ಅನು-ಗ್ರ-ಹ-ವಿ-ಲ್ಲದೆ
ಅನು-ಗ್ರ-ಹಿ-ಸ-ಬೇ-ಕೆಂದು
ಅನು-ಗ್ರ-ಹಿಸಿ
ಅನು-ಗ್ರ-ಹಿ-ಸಿ-ದರು
ಅನು-ಚ-ರರು
ಅನು-ಜ್ಞೆ-ಯನ್ನು
ಅನು-ಪಮ
ಅನು-ಪ-ಸ್ಥಿ-ತಿ-ಯಲ್ಲಿ
ಅನು-ಬಂಧ
ಅನು-ಬಂ-ಧ-ಗಳು
ಅನು-ಭವ
ಅನು-ಭ-ವ-ಕ್ಕಾಗಿ
ಅನು-ಭ-ವಕ್ಕೆ
ಅನು-ಭ-ವ-ಗಳ
ಅನು-ಭ-ವ-ಗಳನ್ನು
ಅನು-ಭ-ವ-ಗಳನ್ನೂ
ಅನು-ಭ-ವ-ಗಳನ್ನೆಲ್ಲ
ಅನು-ಭ-ವ-ಗಳಲ್ಲಿ
ಅನು-ಭ-ವ-ಗ-ಳ-ಲ್ಲೊಂದು
ಅನು-ಭ-ವ-ಗ-ಳಾ-ಗಿ-ದ್ದರೆ
ಅನು-ಭ-ವ-ಗ-ಳಾ-ಗಿ-ದ್ದುವು
ಅನು-ಭ-ವ-ಗ-ಳಾ-ಗು-ತ್ತವೆ
ಅನು-ಭ-ವ-ಗ-ಳಾ-ಗು-ತ್ತಿ-ದ್ದವು
ಅನು-ಭ-ವ-ಗ-ಳಾ-ಗು-ತ್ತಿವೆ
ಅನು-ಭ-ವ-ಗ-ಳಾ-ದುವೋ
ಅನು-ಭ-ವ-ಗಳು
ಅನು-ಭ-ವದ
ಅನು-ಭ-ವ-ದಿಂದ
ಅನು-ಭ-ವ-ಪೂರ್ಣ
ಅನು-ಭ-ವ-ವನ್ನು
ಅನು-ಭ-ವ-ವಲ್ಲ
ಅನು-ಭ-ವ-ವಾಗಿ
ಅನು-ಭ-ವ-ವಾ-ಗಿತ್ತು
ಅನು-ಭ-ವ-ವಾ-ಗಿ-ರ-ಬೇಕು
ಅನು-ಭ-ವ-ವಾಗು
ಅನು-ಭ-ವ-ವಾ-ಗು-ತ್ತದೆ
ಅನು-ಭ-ವ-ವಾ-ಗು-ತ್ತಿತ್ತು
ಅನು-ಭ-ವ-ವಾದ
ಅನು-ಭ-ವ-ವಾ-ದ-ಮೇಲೆ
ಅನು-ಭ-ವ-ವಾ-ದರೂ
ಅನು-ಭ-ವ-ವಾ-ಯಿತು
ಅನು-ಭ-ವ-ವಾ-ಯಿ-ತೆಂ-ದರೆ
ಅನು-ಭ-ವ-ವಾ-ಯಿ-ತೆಂದು
ಅನು-ಭ-ವ-ವಿತ್ತು
ಅನು-ಭ-ವ-ವಿ-ರ-ಲಿಲ್ಲ
ಅನು-ಭ-ವ-ವಿ-ಲ್ಲ-ದಂ-ತಹ
ಅನು-ಭ-ವವು
ಅನು-ಭ-ವ-ವೆಂ-ಥದು
ಅನು-ಭ-ವ-ವೆಂದು
ಅನು-ಭ-ವ-ವೆ-ನ್ನ-ಬೇಕು
ಅನು-ಭ-ವವೇ
ಅನು-ಭ-ವಾ-ಗು-ತ್ತಿತ್ತು
ಅನು-ಭ-ವಿ-ಗ-ಳಿಗೂ
ಅನು-ಭ-ವಿ-ಸ-ಬ-ಹು-ದೇನೋ
ಅನು-ಭ-ವಿ-ಸ-ಬೇ-ಕಾ-ಯಿತು
ಅನು-ಭ-ವಿ-ಸಲು
ಅನು-ಭ-ವಿ-ಸಿದೆ
ಅನು-ಭ-ವಿ-ಸಿ-ರ-ದಿ-ದ್ದಂ-ತಹ
ಅನು-ಭ-ವಿಸು
ಅನು-ಭ-ವಿ-ಸು-ತ್ತಿದ್ದ
ಅನು-ಭ-ವಿ-ಸು-ತ್ತಿ-ದ್ದರು
ಅನು-ಭ-ವಿ-ಸು-ತ್ತಿ-ದ್ದರೋ
ಅನು-ಭ-ವಿ-ಸುವ
ಅನು-ಭೋ-ಗಿ-ಸು-ತ್ತಾರೆ
ಅನು-ಮತಿ
ಅನು-ಮ-ತಿ-ಯನ್ನು
ಅನು-ಮ-ತಿ-ಯನ್ನೂ
ಅನು-ಮಾನ
ಅನು-ಮಾ-ನ-ಕ್ಕೆಡೆ
ಅನು-ಮಾ-ನ-ಗಳನ್ನು
ಅನು-ಮಾ-ನ-ಗೊಂಡ
ಅನು-ಮಾ-ನ-ವುಂ-ಟಾ-ಯಿ-ತಾ-ದರೂ
ಅನು-ಮಾನವೂ
ಅನು-ಮಾ-ನ-ವೇ-ಳ-ಬ-ಹುದು
ಅನು-ಮಾ-ನಿ-ಸದೆ
ಅನು-ಮಾ-ನಿಸಿ
ಅನು-ಮೋ-ದ-ನಾ-ಪ-ತ್ರ-ಗಳ್ನು
ಅನು-ಮೋ-ದ-ನೆ-ಯನ್ನು
ಅನು-ಮೋ-ದಿ-ಸ-ಲಿಲ್ಲ
ಅನು-ಮೋ-ದಿ-ಸಿ-ರ-ಬ-ಹುದು
ಅನು-ಮೋ-ದಿ-ಸು-ತ್ತದೆ
ಅನು-ಮೋ-ದಿ-ಸು-ತ್ತವೆ
ಅನು-ಮೋ-ದಿ-ಸು-ತ್ತಾ-ರೆಂಬ
ಅನು-ಮೋ-ದಿ-ಸು-ತ್ತಿ-ರ-ಲಿ-ಲ್ಲ-ವೆಂ-ಬುದು
ಅನು-ಮೋ-ದಿ-ಸು-ವಂತೆ
ಅನು-ಯಾಯಿ
ಅನು-ಯಾ-ಯಿ-ಗಳ
ಅನು-ಯಾ-ಯಿ-ಗಳನ್ನು
ಅನು-ಯಾ-ಯಿ-ಗಳನ್ನೂ
ಅನು-ಯಾ-ಯಿ-ಗ-ಳಾಗಿ
ಅನು-ಯಾ-ಯಿ-ಗ-ಳಾದ
ಅನು-ಯಾ-ಯಿ-ಗ-ಳಾ-ದರು
ಅನು-ಯಾ-ಯಿ-ಗಳಿಂದ
ಅನು-ಯಾ-ಯಿ-ಗ-ಳಿಗೆ
ಅನು-ಯಾ-ಯಿ-ಗ-ಳಿ-ದ್ದಾರೆ
ಅನು-ಯಾ-ಯಿ-ಗಳು
ಅನು-ಯಾ-ಯಿ-ಗಳೂ
ಅನು-ಯಾ-ಯಿ-ಗ-ಳೆ-ಲ್ಲರೂ
ಅನು-ಯಾ-ಯಿ-ಗಳೇ
ಅನು-ಯಾ-ಯಿ-ಯಾ-ಗಿದ್ದು
ಅನು-ಯಾ-ಯಿಯೂ
ಅನು-ರ-ಣಿ-ತ-ವಾ-ಗು-ತ್ತಿತ್ತು
ಅನು-ವ-ರ್ತಿ-ಗ-ಳೊಂ-ದಿಗೆ
ಅನು-ವಾಗಿ
ಅನು-ವಾದ
ಅನು-ವಾ-ದ-ಕಾ-ರ್ಯಕ್ಕೆ
ಅನು-ವಾ-ದಿ-ಸಲು
ಅನು-ವಾ-ದಿಸು
ಅನು-ವಾ-ದಿ-ಸು-ವಲ್ಲಿ
ಅನುವು
ಅನು-ಷ್ಠಾನ
ಅನು-ಷ್ಠಾ-ನಕ್ಕೆ
ಅನು-ಷ್ಠಾ-ನ-ಗೊ-ಳಿ-ಸಲು
ಅನು-ಷ್ಠಾ-ನ-ಗೊ-ಳ್ಳ-ಬೇ-ಕಾ-ಗಿ-ದೆ-ಯೆಂ-ಬು-ದನ್ನು
ಅನು-ಷ್ಠಾ-ನದ
ಅನು-ಷ್ಠಾ-ನ-ದಲ್ಲಿ
ಅನು-ಷ್ಠಾ-ನ-ಯೋ-ಗ್ಯ-ವಾ-ಗಿತ್ತು
ಅನು-ಷ್ಠಾ-ನ-ಸಾ-ಧ್ಯ-ವಾ-ಗು-ತ್ತ-ವೆಯೋ
ಅನು-ಷ್ಠಾ-ನಾ-ತ್ಮಕ
ಅನು-ಷ್ಠಾ-ನಾ-ತ್ಮ-ಕ-ವಾಗಿ
ಅನು-ಸ-ರ-ಣೀಯ
ಅನು-ಸ-ರ-ಣೀ-ಯ-ವಾದ
ಅನು-ಸ-ರಿಸ
ಅನು-ಸ-ರಿ-ಸದ
ಅನು-ಸ-ರಿ-ಸ-ಬೇ-ಕಾದ
ಅನು-ಸ-ರಿ-ಸ-ಬೇಕು
ಅನು-ಸ-ರಿ-ಸಲು
ಅನು-ಸ-ರಿಸಿ
ಅನು-ಸ-ರಿ-ಸಿದ
ಅನು-ಸ-ರಿ-ಸಿ-ದರೆ
ಅನು-ಸ-ರಿ-ಸಿ-ದೆವು
ಅನು-ಸ-ರಿ-ಸಿರಿ
ಅನು-ಸ-ರಿ-ಸು-ತ್ತದೆ
ಅನು-ಸ-ರಿ-ಸು-ತ್ತಾರೆ
ಅನು-ಸ-ರಿ-ಸು-ತ್ತಿ-ರುವ
ಅನು-ಸ-ರಿ-ಸು-ತ್ತಿ-ರು-ವುದು
ಅನು-ಸ-ರಿ-ಸುವ
ಅನು-ಸ-ರಿ-ಸು-ವಂತೆ
ಅನು-ಸ-ರಿ-ಸು-ವಿ-ರಾ-ದರೆ
ಅನು-ಸ-ರಿ-ಸು-ವುವು
ಅನು-ಸಾ-ರ-ವಾಗಿ
ಅನೂ-ಕೂ-ಲ-ಕರ
ಅನೂ-ಕೂ-ಲ-ವಾ-ದಾಗ
ಅನೂ-ರ್ಜಿ-ತ-ಗೊ-ಳಿ-ಸ-ಲಾ-ಗದು
ಅನೇಕ
ಅನೇ-ಕರ
ಅನೇ-ಕ-ರನ್ನು
ಅನೇ-ಕ-ರಿಗೆ
ಅನೇ-ಕ-ರಿ-ದ್ದ-ರಾ-ದರೂ
ಅನೇ-ಕರು
ಅನೇ-ಕಾ-ನೇಕ
ಅನೈ-ತಿಕ
ಅನೌ-ಪ-ಚಾ-ರಿಕ
ಅನ್ನ
ಅನ್ನ-ಬಟ್ಟೆ
ಅನ್ನ-ಬ-ಹುದು
ಅನ್ನ-ವ-ನ್ನಾ-ದರೂ
ಅನ್ನ-ವನ್ನು
ಅನ್ನ-ವನ್ನೂ
ಅನ್ನ-ವನ್ನೋ
ಅನ್ನಾ-ಹಾ-ರ-ಗ-ಳಿ-ಲ್ಲದೆ
ಅನ್ನಿ
ಅನ್ನಿ-ಸ-ತೊ-ಡ-ಗಿತ್ತು
ಅನ್ನಿ-ಸ-ಲಿಲ್ಲ
ಅನ್ನಿಸಿ
ಅನ್ನಿ-ಸಿತು
ಅನ್ನಿ-ಸಿ-ತುಓ
ಅನ್ನಿ-ಸಿತ್ತು
ಅನ್ನಿ-ಸಿ-ದಾಗ
ಅನ್ನಿ-ಸಿ-ದ್ದ-ರಿಂದ
ಅನ್ನಿ-ಸಿ-ಬಿ-ಟ್ಟಿತು
ಅನ್ನಿ-ಸು-ತ್ತದೆ
ಅನ್ನಿ-ಸು-ತ್ತ-ದೆಈ
ಅನ್ನಿ-ಸು-ತ್ತಿತ್ತು
ಅನ್ನಿ-ಸು-ತ್ತಿದೆ
ಅನ್ನಿ-ಸು-ವು-ದಿಲ್ಲ
ಅನ್ನಿ-ಸು-ವು-ದಿ-ಲ್ಲವೆ
ಅನ್ನು
ಅನ್ನೂ
ಅನ್ಯ-ಚಿಂ-ತೆ-ಯಲ್ಲಿ
ಅನ್ಯಥಾ
ಅನ್ಯ-ಮ-ನ-ಸ್ಕ-ರಾಗಿ
ಅನ್ಯಾಯ
ಅನ್ಯಾ-ಯಕ್ಕೆ
ಅನ್ಯಾ-ಯದ
ಅನ್ಯಾ-ಯ-ದಲ್ಲಿ
ಅನ್ಯಾ-ಯ-ವಾಗಿ
ಅನ್ಯೋ-ನ್ಯ-ತೆ-ಯಿಂದ
ಅನ್ವ-ಯ-ಗಳು
ಅನ್ವ-ಯ-ವಾ-ಗ-ಬಲ್ಲ
ಅನ್ವ-ಯ-ವಾ-ಗು-ತ್ತ-ದೆಯೋ
ಅನ್ವ-ಯ-ವಾ-ಗು-ವಂ-ತಿದ್ದ
ಅನ್ವ-ಯಿ-ಸಿ-ಕೊ-ಳ್ಳಲು
ಅನ್ವ-ಯಿ-ಸಿ-ಕೊ-ಳ್ಳು-ತ್ತಾರೆ
ಅನ್ವ-ಯಿ-ಸಿ-ಕೊ-ಳ್ಳುವ
ಅನ್ವ-ಯಿ-ಸಿ-ದರೆ
ಅನ್ವ-ಯಿ-ಸಿ-ದಾಗ
ಅನ್ವೇ-ಷಕ
ಅನ್ವೇ-ಷಣೆ
ಅನ್ವೇ-ಷ-ಣೆ-ಗಳಲ್ಲಿ
ಅನ್ವೇ-ಷ-ಣೆ-ಗ-ಳೆ-ಲ್ಲ-ವನ್ನೂ
ಅನ್ವೇ-ಷಿ-ಸಿದ
ಅನ್ವೇ-ಷಿ-ಸುವ
ಅಪ
ಅಪ-ಕಾ-ರವೇ
ಅಪ-ಖ್ಯಾತಿ
ಅಪ-ಚಾ-ರ-ವಾ-ಗ-ಬ-ಹು-ದೆಂದು
ಅಪ-ಚಾ-ರ-ವೆ-ಸ-ಗಿ-ದ್ದಾ-ರೆಂದು
ಅಪ-ಥ್ಯ-ವಾ-ದವು
ಅಪ-ಪ್ರ-ಚಾರ
ಅಪ-ಪ್ರ-ಚಾ-ರಕ್ಕೆ
ಅಪ-ಪ್ರ-ಚಾ-ರ-ಗ-ಳಿಗೆ
ಅಪ-ಪ್ರ-ಚಾ-ರದ
ಅಪ-ಪ್ರ-ಚಾ-ರ-ದಿಂ-ದಾಗಿ
ಅಪ-ಪ್ರ-ಚಾ-ರ-ವನ್ನು
ಅಪ-ಪ್ರ-ಚಾ-ರ-ವಾ-ಗಲಿ
ಅಪ-ಪ್ರ-ಚಾ-ರ-ವೆಲ್ಲ
ಅಪ-ಮಾ-ನ-ಕ್ಕೀ-ಡಾ-ಗಲು
ಅಪ-ಮಾ-ನ-ವನ್ನು
ಅಪ-ಮಾ-ನ-ವ-ಲ್ಲವೆ
ಅಪ-ಮಾ-ನ-ವುಂ-ಟು-ಮಾ-ಡಲು
ಅಪ-ರಾಧ
ಅಪ-ರಾ-ಧ-ಗ-ಳಿಗೂ
ಅಪ-ರಾಹ್ನ
ಅಪ-ರಿ-ಗ್ರಹ
ಅಪ-ರಿ-ಗ್ರ-ಹ-ಅ-ಸಂ-ಗ್ರಹ
ಅಪ-ರಿ-ಗ್ರ-ಹ-ಗಳನ್ನು
ಅಪ-ರಿ-ಚಿತ
ಅಪ-ರಿ-ಚಿ-ತ-ನಾ-ಡಿಗೆ
ಅಪ-ರಿ-ಚಿ-ತರ
ಅಪ-ರಿ-ಚಿ-ತ-ರನ್ನು
ಅಪ-ರಿ-ಚಿ-ತ-ರೊಂ-ದಿಗೆ
ಅಪ-ರಿ-ಚಿ-ತ-ವಾಗಿ
ಅಪ-ರಿ-ಚಿ-ತ-ವಾ-ದವು
ಅಪ-ರಿ-ಮಿತ
ಅಪ-ರೂಪ
ಅಪ-ರೂ-ಪದ
ಅಪ-ರೂ-ಪವೇ
ಅಪ-ಲಾ-ಪ-ಗಳನ್ನು
ಅಪ-ವಾ-ದ-ನಿಂ-ದೆ-ಗಳ
ಅಪ-ವಾ-ದಕ್ಕೆ
ಅಪ-ವಾ-ದ-ಗಳ
ಅಪ-ವಾ-ದ-ಗಳನ್ನು
ಅಪ-ವಾ-ದದ
ಅಪ-ವಾ-ದ-ವನ್ನು
ಅಪ-ವಿ-ತ್ರ-ಳು-ಎಂಬ
ಅಪ-ಹ-ರಿ-ಸಿ-ಬಿ-ಡು-ತ್ತಿದೆ
ಅಪ-ಹಾಸ್ಯ
ಅಪ-ಹಾ-ಸ್ಯ-ಕ್ಕೀ-ಡಾಗು
ಅಪ-ಹಾ-ಸ್ಯ-ಕ್ಕೀ-ಡಾ-ದುವು
ಅಪಾ-ದನೆ
ಅಪಾ-ದಿತ
ಅಪಾಯ
ಅಪಾ-ಯ-ಕರ
ಅಪಾ-ಯ-ಕಾರಿ
ಅಪಾ-ಯದ
ಅಪಾ-ಯ-ದಿಂದ
ಅಪಾ-ಯ-ವ-ನ್ನೆ-ದು-ರಿ-ಸಿಯೂ
ಅಪಾ-ಯ-ವಿತ್ತು
ಅಪಾ-ಯ-ವಿದೆ
ಅಪಾ-ಯ-ವಿಲ್ಲ
ಅಪಾ-ಯವೂ
ಅಪಾ-ಯ-ವೇನೂ
ಅಪಾರ
ಅಪಾ-ರ-ಜ್ಞಾ-ನ-ವನ್ನು
ಅಪಾ-ರ-ವಾಗಿ
ಅಪಾ-ರ-ವಾದ
ಅಪಾರ್ಥ
ಅಪಾ-ರ್ಥ-ಗೊ-ಳಿ-ಸ-ಲಾಗಿದೆ
ಅಪೀಲ್
ಅಪೂ-ರ್ಣ-ವಾ-ದಂತೆ
ಅಪೂರ್ವ
ಅಪೂ-ರ್ವ-ಕಾಂತಿ
ಅಪೂ-ರ್ವ-ವಾದ
ಅಪೂ-ರ್ವ-ವಾ-ದದ್ದು
ಅಪೂ-ರ್ವ-ವಾ-ದುದು
ಅಪೂ-ರ್ವ-ವಾ-ದುದೇ
ಅಪೇ-ಕ್ಷಿ-ಸಿ-ದ್ದಂತೆ
ಅಪೇ-ಕ್ಷಿ-ಸು-ತ್ತಾನೆ
ಅಪೇ-ಕ್ಷಿ-ಸುವ
ಅಪೇಕ್ಷೆ
ಅಪೇ-ಕ್ಷೆಯ
ಅಪೇ-ಕ್ಷೆ-ಯಂತೆ
ಅಪೇರಾ
ಅಪೌ-ರು-ಷೇ-ಯ-ವೆಂದು
ಅಪ್ಪಟ
ಅಪ್ಪಣೆ
ಅಪ್ಪ-ಣೆ-ಯಂತೆ
ಅಪ್ಪನ
ಅಪ್ಪ-ಬೇ-ಕಾ-ಯಿತು
ಅಪ್ಪ-ಬೇಕು
ಅಪ್ಪ-ಳಿಸಿ
ಅಪ್ಪಾ
ಅಪ್ಪಿ-ಕೊಂ-ಡಿದ್ದ
ಅಪ್ಯಾಯ
ಅಪ್ಯಾ-ಯ-ಮಾ-ನ-ವಾ-ದದ್ದೇ
ಅಪ್ರ-ತಿ-ಭ-ನಾಗಿ
ಅಪ್ರ-ತಿಮ
ಅಪ್ರ-ತಿ-ಮ-ರೆ-ನ್ನಿಸಿ
ಅಪ್ರ-ತಿ-ಮ-ವಾ-ಗಿತ್ತು
ಅಪ್ರ-ತಿ-ಹತ
ಅಪ್ರ-ತಿ-ಹ-ತ-ವಾ-ಗಿಯೇ
ಅಪ್ರ-ಬುದ್ಧ
ಅಪ್ರ-ಯ-ತ್ನ-ವಾಗಿ
ಅಪ್ರಾ-ಯೋ-ಗಿ-ಕ-ವೆಂದೂ
ಅಪ್ಸ-ರ-ಸ್ತ್ರೀ-ಯರು
ಅಬ-ದ್ಧ-ವಾಗಿ
ಅಬಾ-ಧಿ-ತ-ವಾಗಿ
ಅಬು
ಅಬು-ಪ-ರ್ವ-ತ-ದಲ್ಲಿ
ಅಬು-ವಿನ
ಅಬು-ವಿ-ನಲ್ಲಿ
ಅಬು-ವಿ-ನಿಂದ
ಅಬ್ದುಲ್
ಅಬ್ಬ
ಅಬ್ಬರ
ಅಬ್ಬ-ರ-ತದ
ಅಬ್ಬ-ರಿ-ಸಿದ
ಅಭ-ಯಾ-ನಂದ
ಅಭ-ಯಾ-ನಂದಾ
ಅಭಾ-ವ-ದಿಂ-ದಾಗಿ
ಅಭಾ-ವ-ವನ್ನು
ಅಭಿ
ಅಭಿ-ನಂ-ದನಾ
ಅಭಿ-ನಂ-ದ-ನೆ-ಗ-ಳಾದ
ಅಭಿ-ನಂ-ದ-ನೆಗೆ
ಅಭಿ-ನಂ-ದ-ನೆಯ
ಅಭಿ-ನಂ-ದ-ನೆ-ಯನ್ನು
ಅಭಿ-ನಂ-ದ-ನೆ-ಯನ್ನೂ
ಅಭಿ-ನಂ-ದ-ನೆ-ಯಿಂದ
ಅಭಿ-ನಂ-ದಿ-ಸ-ಬೇಕು
ಅಭಿ-ನಂ-ದಿ-ಸಿ-ದರು
ಅಭಿ-ನಂ-ದಿ-ಸಿ-ದರೂ
ಅಭಿ-ನಂ-ದಿ-ಸು-ತ್ತಾರೆ
ಅಭಿ-ನಂ-ದಿ-ಸು-ತ್ತೇನೆ
ಅಭಿ-ನಂ-ಧ-ನೆ-ಯನ್ನೂ
ಅಭಿ-ನ-ಯ-ದಿಂದ
ಅಭಿ-ನ-ಯಿ-ತ-ವಾಗು
ಅಭಿ-ನ-ಯಿ-ಸು-ವು-ದ-ಕ್ಕೆಂದು
ಅಭಿ-ಪ್ರಾಯ
ಅಭಿ-ಪ್ರಾ-ಯ-ಧೋ-ರ-ಣೆ-ಗಳನ್ನು
ಅಭಿ-ಪ್ರಾ-ಯಕ್ಕೆ
ಅಭಿ-ಪ್ರಾ-ಯ-ಗಳ
ಅಭಿ-ಪ್ರಾ-ಯ-ಗಳನ್ನು
ಅಭಿ-ಪ್ರಾ-ಯ-ಗಳಿಂದ
ಅಭಿ-ಪ್ರಾ-ಯ-ಗ-ಳಿಗೆ
ಅಭಿ-ಪ್ರಾ-ಯ-ಗಳು
ಅಭಿ-ಪ್ರಾ-ಯ-ಗ-ಳು-ಇ-ವು-ಗಳನ್ನೆಲ್ಲ
ಅಭಿ-ಪ್ರಾ-ಯ-ದಂತೆ
ಅಭಿ-ಪ್ರಾ-ಯನ್ನು
ಅಭಿ-ಪ್ರಾ-ಯ-ಪಟ್ಟ
ಅಭಿ-ಪ್ರಾ-ಯ-ಪ-ಟ್ಟರು
ಅಭಿ-ಪ್ರಾ-ಯ-ಪ-ಟ್ಟಿ-ದ್ದರು
ಅಭಿ-ಪ್ರಾ-ಯ-ಪ-ಡು-ತ್ತಾರೆ
ಅಭಿ-ಪ್ರಾ-ಯ-ವನ್ನು
ಅಭಿ-ಪ್ರಾ-ಯ-ವನ್ನೂ
ಅಭಿ-ಪ್ರಾ-ಯ-ವನ್ನೇ
ಅಭಿ-ಪ್ರಾ-ಯ-ವಾ-ಗಿತ್ತು
ಅಭಿ-ಪ್ರಾ-ಯ-ವಾ-ಯಿತು
ಅಭಿ-ಪ್ರಾ-ಯ-ವಿ-ನಿ-ಮಯ
ಅಭಿ-ಪ್ರಾ-ಯವು
ಅಭಿ-ಪ್ರಾ-ಯ-ವುಂ-ಟಾ-ಗಲು
ಅಭಿ-ಪ್ರಾ-ಯವೆ
ಅಭಿ-ಪ್ರಾ-ಯವೇ
ಅಭಿ-ಪ್ರಾ-ಯ-ವೇನು
ಅಭಿ-ಮತ
ಅಭಿ-ಮಾನ
ಅಭಿ-ಮಾ-ನದ
ಅಭಿ-ಮಾ-ನ-ವಿ-ದ್ದದ್ದೇ
ಅಭಿ-ಮಾ-ನಿ-ಗ-ಳು-ಅ-ನು-ಯಾ-ಯಿ-ಗಳಿಂದ
ಅಭಿ-ಮಾ-ನಿ-ಗ-ಳೆಲ್ಲ
ಅಭಿ-ಮು-ಖ-ವಾಗಿ
ಅಭಿ-ಮು-ಖ-ವಾ-ಗಿದ್ದು
ಅಭಿ-ರುಚಿ
ಅಭಿ-ರು-ಚಿ-ಗಳು
ಅಭಿ-ರು-ಚಿ-ಯನ್ನೂ
ಅಭಿ-ಲಾಷೆ
ಅಭಿ-ಲಾ-ಷೆ-ಯಾ-ಗಿತ್ತು
ಅಭಿ-ವೃದ್ಧಿ
ಅಭಿ-ವೃ-ದ್ಧಿಗೆ
ಅಭಿ-ವೃ-ದ್ಧಿ-ಗೊಂ-ಡಿವೆ
ಅಭಿ-ವೃ-ದ್ಧಿ-ಪ-ಡಿ-ಸಲು
ಅಭಿ-ವೃ-ದ್ಧಿಯ
ಅಭಿ-ವ್ಯ-ಕ್ತ-ಗೊಂ-ಡಿ-ತು-ಇ-ದಕ್ಕೆ
ಅಭಿ-ವ್ಯ-ಕ್ತ-ವಾ-ಗು-ತ್ತದೆ
ಅಭಿ-ವ್ಯಕ್ತಿ
ಅಭಿ-ವ್ಯ-ಕ್ತಿ-ಗ-ಳೆಂದು
ಅಭಿ-ವ್ಯ-ಕ್ತಿ-ಗೊ-ಳ್ಳುವ
ಅಭಿ-ವ್ಯ-ಕ್ತಿ-ಯನ್ನು
ಅಭೀಷ್ಟ
ಅಭೂತ
ಅಭೂ-ತ-ಪೂರ್ವ
ಅಭೂ-ತ-ಪೂ-ರ್ವ-ವಾ-ದದ್ದು
ಅಭೂ-ತ-ಪೂ-ರ್ವವೂ
ಅಭೇದಾ
ಅಭೇ-ದಾ-ನಂ-ದರ
ಅಭೇ-ದಾ-ನಂ-ದ-ರನ್ನು
ಅಭೇ-ದಾ-ನಂ-ದರಿ
ಅಭೇ-ದಾ-ನಂ-ದ-ರಿ-ಗಾಗಿ
ಅಭೇ-ದಾ-ನಂ-ದ-ರಿಗೆ
ಅಭೇ-ದಾ-ನಂ-ದರು
ಅಭೇ-ದಾ-ನಂ-ದರೂ
ಅಭೇದ್ಯ
ಅಭೇಧ್ಯ
ಅಭ್ಯಂ-ತ-ರ-ವಿಲ್ಲ
ಅಭ್ಯಂ-ತ-ರ-ವೇ-ನಿದೆ
ಅಭ್ಯಂ-ತ-ರ-ವೇನೂ
ಅಭ್ಯ-ಸಿಸಿ
ಅಭ್ಯ-ಸಿ-ಸಿ-ದಂತೆ
ಅಭ್ಯಾಸ
ಅಭ್ಯಾ-ಸ-ಗಳ
ಅಭ್ಯಾ-ಸ-ದಲ್ಲಿ
ಅಭ್ಯಾ-ಸ-ವನ್ನು
ಅಭ್ಯಾ-ಸ-ವಾ-ಗಿ-ರು-ವು-ದ-ರಿಂ-ದಲೇ
ಅಭ್ಯಾ-ಸ-ವಾ-ಗುತ್ತ
ಅಭ್ಯಾ-ಸ-ವಿದ್ದ
ಅಭ್ಯಾ-ಸ-ವಿ-ರು-ವ-ವ-ರಿಗೆ
ಅಭ್ಯಾ-ಸ-ವಿ-ಲ್ಲ-ದ-ವ-ರಿಗೆ
ಅಮರ
ಅಮ-ರ-ಕೋ-ಶ-ವನ್ನು
ಅಮ-ರತ್ವ
ಅಮ-ರ-ತ್ವಕ್ಕೆ
ಅಮ-ರ-ತ್ವದ
ಅಮ-ರ-ತ್ವವೇ
ಅಮ-ರ-ಪ್ರೇ-ಮದ
ಅಮ-ರ-ವ-ನ್ನಾ-ಗಿಯೂ
ಅಮ-ರಾ-ವ-ತಿ-ಯಾದ
ಅಮ-ಲಿ-ನದ್ದು
ಅಮಾ-ನುಷ
ಅಮಿತ
ಅಮುಖ್ಯ
ಅಮೂಲ್ಯ
ಅಮೂ-ಲ್ಯ-ವಾ-ಗಿ-ರು-ವಾಗ
ಅಮೂ-ಲ್ಯ-ವಾ-ಗಿ-ವೆ-ಯೆಂ-ಬುದು
ಅಮೂ-ಲ್ಯ-ವಾದ
ಅಮೃತ
ಅಮೃ-ತ-ಕ್ಕಿಂ-ತಲೂ
ಅಮೃ-ತ-ತ್ವ-ದೆ-ಡೆಗೆ
ಅಮೃ-ತ-ಪು-ತ್ರನೆ
ಅಮೃ-ತ-ಪು-ತ್ರರೇ
ಅಮೃ-ತ-ವನ್ನು
ಅಮೃ-ತವು
ಅಮೃ-ತವೂ
ಅಮೃ-ತ-ಶಿ-ಲೆ-ಯಲ್ಲಿ
ಅಮೃ-ತ-ಶಿ-ಲೆ-ಯಿಂದ
ಅಮೃ-ತ-ಸ-ಮಾನ
ಅಮೃ-ತಸ್ಯ
ಅಮೃ-ತಾ-ನಂ-ದ-ದಲ್ಲಿ
ಅಮೆ-ರಿಕ
ಅಮೆ-ರಿ-ಕ-ಇಂ-ಗ್ಲೆಂ-ಡು-ಗಳ
ಅಮೆ-ರಿ-ಕ-ಇಂ-ಗ್ಲೆಂ-ಡು-ಗ-ಳ-ಲ್ಲಿನ
ಅಮೆ-ರಿ-ಕ-ಇಂ-ಗ್ಲೆಂ-ಡು-ಗ-ಳೆ-ರಡೂ
ಅಮೆ-ರಿ-ಕ-ಕ್ಕಂತೂ
ಅಮೆ-ರಿ-ಕ-ಕ್ಕಿಂತ
ಅಮೆ-ರಿ-ಕಕ್ಕೆ
ಅಮೆ-ರಿ-ಕ-ಕ್ಕೆ-ಉ-ದಾ-ಹ-ರ-ಣೆಗೆ
ಅಮೆ-ರಿ-ಕ-ಗಳಲ್ಲಿ
ಅಮೆ-ರಿ-ಕದ
ಅಮೆ-ರಿ-ಕ-ದಂ-ತಹ
ಅಮೆ-ರಿ-ಕ-ದ-ಲ್ಲಾ-ದರೂ
ಅಮೆ-ರಿ-ಕ-ದಲ್ಲಿ
ಅಮೆ-ರಿ-ಕ-ದ-ಲ್ಲಿದೆ
ಅಮೆ-ರಿ-ಕ-ದ-ಲ್ಲಿದ್ದ
ಅಮೆ-ರಿ-ಕ-ದ-ಲ್ಲಿ-ದ್ದಾಗ
ಅಮೆ-ರಿ-ಕ-ದ-ಲ್ಲಿ-ದ್ದು-ಕೊಂಡೇ
ಅಮೆ-ರಿ-ಕ-ದ-ಲ್ಲಿನ
ಅಮೆ-ರಿ-ಕ-ದ-ಲ್ಲಿ-ಭಾ-ರ-ತದ
ಅಮೆ-ರಿ-ಕ-ದ-ಲ್ಲಿ-ರುವ
ಅಮೆ-ರಿ-ಕ-ದಲ್ಲೆಲ್ಲ
ಅಮೆ-ರಿ-ಕ-ದಲ್ಲೇ
ಅಮೆ-ರಿ-ಕ-ದ-ವನು
ಅಮೆ-ರಿ-ಕ-ದ-ವರು
ಅಮೆ-ರಿ-ಕ-ದ-ವರೂ
ಅಮೆ-ರಿ-ಕ-ದಾ-ದ್ಯಂತ
ಅಮೆ-ರಿ-ಕ-ದಿಂದ
ಅಮೆ-ರಿ-ಕ-ದಿಂ-ದಲೇ
ಅಮೆ-ರಿ-ಕ-ದೊಂ-ದಿಗೆ
ಅಮೆ-ರಿ-ಕ-ನರ
ಅಮೆ-ರಿ-ಕನ್
ಅಮೆ-ರಿ-ಕ-ನ್ನನೂ
ಅಮೆ-ರಿ-ಕ-ನ್ನರ
ಅಮೆ-ರಿ-ಕ-ನ್ನ-ರನ್ನು
ಅಮೆ-ರಿ-ಕ-ನ್ನ-ರನ್ನೂ
ಅಮೆ-ರಿ-ಕ-ನ್ನ-ರಲ್ಲಿ
ಅಮೆ-ರಿ-ಕ-ನ್ನ-ರಷ್ಟು
ಅಮೆ-ರಿ-ಕ-ನ್ನ-ರಿಗೂ
ಅಮೆ-ರಿ-ಕ-ನ್ನ-ರಿಗೆ
ಅಮೆ-ರಿ-ಕ-ನ್ನರು
ಅಮೆ-ರಿ-ಕ-ವನ್ನು
ಅಮೆ-ರಿ-ಕ-ವ-ನ್ನೆಲ್ಲ
ಅಮೆ-ರಿ-ಕ-ವನ್ನೇ
ಅಮೆ-ರಿ-ಕವು
ಅಮೆ-ರಿಕಾ
ಅಮೆ-ರಿ-ಕಾದಿ
ಅಮೆ-ರಿ-ಕೆಗೆ
ಅಮೆ-ರಿ-ಕೆಯ
ಅಮೆ-ರಿ-ಕೆ-ಯಂ-ತಹ
ಅಮೆ-ರಿ-ಕೆ-ಯ-ದಾ-ಯಿತು
ಅಮೆ-ರಿ-ಕೆ-ಯಲ್ಲಿ
ಅಮೆ-ರಿ-ಕೆ-ಯಿಂದ
ಅಮೇ-ರಿ-ಕಕ್ಕೆ
ಅಮೇ-ರಿ-ಕ-ದ-ಲ್ಲಿ-ಲ್ಲ-ದಾಗ
ಅಮೋ-ಘ-ವಾಗಿ
ಅಮ್ಮ
ಅಮ್ಮ-ನ-ವರ
ಅಮ್ಮ-ನಿ-ಗಾಗಿ
ಅಮ್ಮಾ
ಅಯ-ಶ-ಸ್ವಿ-ಯಾಗಿ
ಅಯ-ಸ್ಕಾಂ-ತ-ದಂತೆ
ಅಯ-ಸ್ಕಾಂ-ತೀಯ
ಅಯಾ-ಚಿ-ತ-ವಾಗಿ
ಅಯೋ-ವಾದ
ಅಯ್ಯಂ-ಗಾ-ರರು
ಅಯ್ಯಂ-ಗಾರ್
ಅಯ್ಯಯ್ಯೋ
ಅಯ್ಯ-ರರ
ಅಯ್ಯ-ರ-ರನ್ನು
ಅಯ್ಯ-ರ-ರಿಗೆ
ಅಯ್ಯ-ರರು
ಅಯ್ಯ-ರ-ರೊಂ-ದಿಗೆ
ಅಯ್ಯ-ರಿ-ಗಂತೂ
ಅಯ್ಯರ್
ಅಯ್ಯ-ರ್ರ-ವರ
ಅಯ್ಯ-ರ್ರ-ವರು
ಅಯ್ಯ-ರ್ರಿಗೆ
ಅಯ್ಯಾ
ಅಯ್ಯೊ
ಅಯ್ಯೋ
ಅರ-ಗಿ-ಸಿ-ಕೊಂ-ಡಿ-ದ್ದೀರಿ
ಅರ-ಗಿ-ಸಿ-ಕೊ-ಳ್ಳಲು
ಅರ-ಚಿ-ಕೊ-ಳ್ಳಲಿ
ಅರ-ಚಿತು
ಅರ-ಚಿ-ದರು
ಅರಣ್ಯ
ಅರ-ಣ್ಯಾ-ಧಿ-ಕಾರಿ
ಅರ-ಬ್ಬರೂ
ಅರಬ್ಬೀ
ಅರ-ಮ-ನೆ-ಗಳಲ್ಲಿ
ಅರ-ಮ-ನೆ-ಗಳು
ಅರ-ಮ-ನೆಗೂ
ಅರ-ಮ-ನೆಗೆ
ಅರ-ಮ-ನೆ-ಗೊಮ್ಮೆ
ಅರ-ಮ-ನೆಯ
ಅರ-ಮ-ನೆ-ಯಂ-ತಹ
ಅರ-ಮ-ನೆ-ಯನ್ನೂ
ಅರ-ಮ-ನೆ-ಯಲ್ಲಿ
ಅರ-ಮ-ನೆ-ಯಲ್ಲೇ
ಅರ-ಮ-ನೆಯೇ
ಅರ-ಳು-ಮ-ರಳು
ಅರ-ಳು-ವುದನ್ನು
ಅರಸ
ಅರಸಿ
ಅರ-ಸಿ-ದ್ದರು
ಅರ-ಸಿ-ರಿ-ಎಂದು
ಅರ-ಸು-ತ್ತಲೋ
ಅರ-ಸು-ತ್ತಿದ್ದ
ಅರ-ಸು-ತ್ತಿ-ದ್ದರು
ಅರ-ಸು-ವು-ದಿಲ್ಲ
ಅರ-ಸು-ವುದೂ
ಅರಿ
ಅರಿತ
ಅರಿ-ತರು
ಅರಿ-ತ-ವ-ರಲ್ಲಿ
ಅರಿ-ತ-ವ-ರಾ-ದರೂ
ಅರಿ-ತ-ವ-ರಿಗೆ
ಅರಿ-ತ-ವರು
ಅರಿ-ತ-ವರೂ
ಅರಿ-ತಿ-ದ್ದರು
ಅರಿ-ತಿ-ದ್ದಿ-ರ-ಬೇಕು
ಅರಿ-ತಿ-ದ್ದೇವೆ
ಅರಿ-ತಿ-ರುವ
ಅರಿ-ತಿ-ರು-ವು-ದಾಗಿ
ಅರಿ-ತಿಲ್ಲ
ಅರಿತು
ಅರಿ-ತು-ಕೊಂಡ
ಅರಿ-ತು-ಕೊಂ-ಡಾಗ
ಅರಿ-ತು-ಕೊಂ-ಡಿ-ದ್ದ-ರ-ಲ್ಲದೆ
ಅರಿ-ತು-ಕೊಂ-ಡಿ-ದ್ದರೋ
ಅರಿ-ತು-ಕೊಂ-ಡಿಲ್ಲ
ಅರಿ-ತು-ಕೊಂಡು
ಅರಿ-ತು-ಕೊ-ಳ್ಳ-ದಿ-ದ್ದ-ವ-ರೆಂ-ದರೆ
ಅರಿ-ತು-ಕೊ-ಳ್ಳ-ಬ-ಲ್ಲ-ಇ-ದೊಂದು
ಅರಿ-ತು-ಕೊ-ಳ್ಳ-ಬೇ-ಕಾದ
ಅರಿ-ತು-ಕೊ-ಳ್ಳ-ಬೇ-ಕೆಂದು
ಅರಿ-ತು-ಕೊ-ಳ್ಳ-ಲಾ-ರೆವು
ಅರಿ-ತು-ಕೊ-ಳ್ಳ-ಲಿ-ಲ್ಲ-ವೆಂ-ಬುದು
ಅರಿ-ತು-ಕೊ-ಳ್ಳಲು
ಅರಿ-ತು-ಕೊ-ಳ್ಳು-ವಂತೆ
ಅರಿಯ
ಅರಿ-ಯದ
ಅರಿ-ಯ-ಬ-ಲ್ಲ-ವ-ರಾ-ಗಿದ್ದ
ಅರಿ-ಯ-ಬೇ-ಕಾ-ದರೆ
ಅರಿ-ಯ-ಲಾ-ರರು
ಅರಿ-ಯ-ಲಿ-ದೆ-ಯೆಂದೂ
ಅರಿ-ಯಲು
ಅರಿಯು
ಅರಿ-ಯುತ್ತ
ಅರಿ-ಯು-ತ್ತಾರೆ
ಅರಿ-ಯು-ತ್ತಿ-ದ್ದರು
ಅರಿ-ಯುವ
ಅರಿ-ಯು-ವ-ವ-ರೆಗೂ
ಅರಿ-ಯು-ವಿರಿ
ಅರಿ-ಯು-ವು-ದ-ರಿಂದ
ಅರಿ-ಯು-ವುದೇ
ಅರಿಯೆ
ಅರಿ-ವನ್ನು
ಅರಿ-ವಾ-ಗ-ತೊಡ
ಅರಿ-ವಾ-ಗಿತ್ತು
ಅರಿ-ವಾ-ಗಿ-ತ್ತೆಂ-ದರೆ
ಅರಿ-ವಾ-ಗಿ-ರ-ಬೇಕು
ಅರಿ-ವಾ-ಗು-ತ್ತದೆ
ಅರಿ-ವಾ-ಗು-ತ್ತಿತ್ತು
ಅರಿ-ವಾ-ಗು-ತ್ತಿದೆ
ಅರಿ-ವಾದ
ಅರಿ-ವಾ-ದದ್ದು
ಅರಿ-ವಾ-ದಾಗ
ಅರಿ-ವಾ-ಯಿತು
ಅರಿ-ವಿಗೇ
ಅರಿ-ವಿತ್ತು
ಅರಿ-ವಿ-ದ್ದಿ-ರ-ಲಾ-ರದು
ಅರಿ-ವಿರ
ಅರಿ-ವಿ-ರು-ತ್ತದೆ
ಅರಿ-ವಿ-ಲ್ಲ-ದಂತೆ
ಅರಿ-ವಿ-ಲ್ಲದೆ
ಅರಿವು
ಅರಿ-ವು
ಅರಿ-ವುಂ-ಟಾ-ಗು-ತ್ತಿತ್ತು
ಅರಿ-ವುಂ-ಟು-ಮಾ-ಡಿ-ಸಲು
ಅರಿ-ವೆ-ಯಿ-ದ್ದಂ-ತೆ-ಬೇ-ಕೆಂ-ದಾಗ
ಅರಿವೇ
ಅರು-ಣೋ-ದ-ಯದ
ಅರು-ಹಿದ
ಅರೆ
ಅರೆ-ನಾ-ಗ-ರಿಕ
ಅರೆ-ಬರೆ
ಅರೆ-ಬಿ-ರಿ-ಯುವ
ಅರೆ-ಮ-ನ-ಸ್ಸಿ-ನಿಂ-ದಲೇ
ಅರೆ-ಶಿ-ಕ್ಷಿತ
ಅರೆ-ಹೊ-ಟ್ಟೆ-ಯೂ-ಟ-ದಿಂದ
ಅರ್ಚಕ
ಅರ್ಚ-ಕ-ಕು-ಲದ
ಅರ್ಚ-ಕ-ರಲ್ಲೂ
ಅರ್ಚ-ಕ-ರಿಗೆ
ಅರ್ಚ-ಕರು
ಅರ್ಜುನ
ಅರ್ಣವ
ಅರ್ಥ
ಅರ್ಥ-ಗ-ಳುಂಟು
ಅರ್ಥ-ದ-ಲ್ಲಲ್ಲ
ಅರ್ಥ-ದಲ್ಲಿ
ಅರ್ಥ-ಪ-ಡಿ-ಸಲು
ಅರ್ಥ-ಪೂರ್ಣ
ಅರ್ಥ-ಮಾಡಿ
ಅರ್ಥ-ಮಾ-ಡಿ-ಕೊಂ-ಡರು
ಅರ್ಥ-ಮಾ-ಡಿ-ಕೊಂ-ಡಿದ್ದ
ಅರ್ಥ-ಮಾ-ಡಿ-ಕೊಂ-ಡಿ-ರ-ಲಿಲ್ಲ
ಅರ್ಥ-ಮಾ-ಡಿ-ಕೊಂ-ಡಿರು
ಅರ್ಥ-ಮಾ-ಡಿ-ಕೊಂ-ಡಿ-ರುವ
ಅರ್ಥ-ಮಾ-ಡಿ-ಕೊಂ-ಡಿ-ರು-ವುದು
ಅರ್ಥ-ಮಾ-ಡಿ-ಕೊಂ-ಡಿಲ್ಲ
ಅರ್ಥ-ಮಾ-ಡಿ-ಕೊಂಡು
ಅರ್ಥ-ಮಾ-ಡಿ-ಕೊಂ-ಡು-ಬಿಟ್ಟಿ
ಅರ್ಥ-ಮಾ-ಡಿ-ಕೊ-ಳ್ಳ-ದಿ-ದ್ದ-ವ-ರೆಂ-ದರೆ
ಅರ್ಥ-ಮಾ-ಡಿ-ಕೊ-ಳ್ಳ-ಬ-ಲ್ಲ-ವರು
ಅರ್ಥ-ಮಾ-ಡಿ-ಕೊ-ಳ್ಳ-ಬ-ಲ್ಲೆಯಾ
ಅರ್ಥ-ಮಾ-ಡಿ-ಕೊ-ಳ್ಳ-ಬ-ಲ್ಲೆ-ಯೇನು
ಅರ್ಥ-ಮಾ-ಡಿ-ಕೊ-ಳ್ಳ-ಬ-ಹುದು
ಅರ್ಥ-ಮಾ-ಡಿ-ಕೊ-ಳ್ಳ-ಬೇ-ಕಾ-ಗು-ತ್ತದೆ
ಅರ್ಥ-ಮಾ-ಡಿ-ಕೊ-ಳ್ಳ-ಬೇ-ಕಾ-ದರೆ
ಅರ್ಥ-ಮಾ-ಡಿ-ಕೊ-ಳ್ಳ-ಬೇಕು
ಅರ್ಥ-ಮಾ-ಡಿ-ಕೊ-ಳ್ಳ-ಲಾಗಿದೆ
ಅರ್ಥ-ಮಾ-ಡಿ-ಕೊ-ಳ್ಳ-ಲಾ-ರದ
ಅರ್ಥ-ಮಾ-ಡಿ-ಕೊ-ಳ್ಳ-ಲಾ-ರದೆ
ಅರ್ಥ-ಮಾ-ಡಿ-ಕೊ-ಳ್ಳಲು
ಅರ್ಥ-ಮಾ-ಡಿ-ಕೊ-ಳ್ಳು-ತ್ತಾರೆ
ಅರ್ಥರ್
ಅರ್ಥ-ವನ್ನು
ಅರ್ಥ-ವನ್ನೇ
ಅರ್ಥ-ವಲ್ಲ
ಅರ್ಥ-ವ-ಲ್ಲವೆ
ಅರ್ಥ-ವಾ-ಗ-ದಿ-ರ-ಬ-ಹುದು
ಅರ್ಥ-ವಾ-ಗ-ಬೇ-ಕಲ್ಲ
ಅರ್ಥ-ವಾ-ಗ-ಲಾ-ರದು
ಅರ್ಥ-ವಾ-ಗ-ಲಿಲ್ಲ
ಅರ್ಥ-ವಾ-ಗ-ಲಿ-ಲ್ಲ-ವೇನೋ
ಅರ್ಥ-ವಾ-ಗಲು
ಅರ್ಥ-ವಾ-ಗಲೇ
ಅರ್ಥ-ವಾ-ಗ-ವು-ದಿಲ್ಲ
ಅರ್ಥ-ವಾ-ಗು-ತ್ತದೆ
ಅರ್ಥ-ವಾ-ಗು-ತ್ತಿದೆ
ಅರ್ಥ-ವಾ-ಗು-ತ್ತಿ-ದ್ದರೂ
ಅರ್ಥ-ವಾ-ಗು-ತ್ತಿ-ರ-ಲಿಲ್ಲ
ಅರ್ಥ-ವಾ-ಗು-ತ್ತಿಲ್ಲ
ಅರ್ಥ-ವಾ-ಗು-ವಂ-ತಹ
ಅರ್ಥ-ವಾ-ಗು-ವಂತೆ
ಅರ್ಥ-ವಾ-ಗು-ವು-ದಿಲ್ಲ
ಅರ್ಥ-ವಾ-ಗು-ವು-ದೇ-ನಿ-ದ್ದರೂ
ಅರ್ಥ-ವಾ-ದ-ದ್ದು-ಪ್ರೀ-ತಿ-ಯೆಂ-ದ-ರೇನು
ಅರ್ಥ-ವಾ-ಯಿತು
ಅರ್ಥ-ವಾ-ಯಿತೆ
ಅರ್ಥ-ವಿದೆ
ಅರ್ಥ-ವಿ-ಲ್ಲದ
ಅರ್ಥ-ವಿ-ಲ್ಲ-ದ್ದ-ನ್ನೆ-ಲ್ಲ-ಗ-ಳ-ಹ-ಬೇಡಿ
ಅರ್ಥವೇ
ಅರ್ಥ-ವೇನು
ಅರ್ಥ-ವೇನೆಂದರೆ
ಅರ್ಥ-ಶಾಸ್ತ್ರ
ಅರ್ಥ-ಹೀನ
ಅರ್ಥ-ಹೀ-ನ-ವಾ-ದ-ದ್ದೊಂದೂ
ಅರ್ಥಾತ್
ಅರ್ಥೈ-ಸಿದ
ಅರ್ಧ
ಅರ್ಧ-ಕ್ಕಿಂತ
ಅರ್ಧ-ಗಂಟೆ
ಅರ್ಧ-ಗಂ-ಟೆ-ಯೊ-ಳ-ಗಾಗಿ
ಅರ್ಧ-ದ-ಷ್ಟಿ-ದ್ದಾ-ಗಲೇ
ಅರ್ಧ-ನಿ-ದ್ರೆ-ಯ-ಲ್ಲಿ-ದ್ದಾಗ
ಅರ್ಧ-ಮಾ-ತ್ರವೇ
ಅರ್ಧ-ವೃ-ತ್ತಾ-ಕಾ-ರ-ದಲ್ಲಿ
ಅರ್ಧ-ವೃ-ತ್ತಾ-ಕಾ-ರ-ವಾಗಿ
ಅರ್ಧಾಂ-ಗಿ-ಇಂ-ತಹ
ಅರ್ಪಣೆ
ಅರ್ಪಿ-ತ-ವಾದ
ಅರ್ಪಿ-ಸ-ಬೇಕು
ಅರ್ಪಿ-ಸಲು
ಅರ್ಪಿಸಿ
ಅರ್ಪಿ-ಸಿದ
ಅರ್ಪಿ-ಸಿ-ದರು
ಅರ್ಪಿ-ಸಿ-ದಳು
ಅರ್ಪಿ-ಸಿ-ಬಿ-ಡ-ಬೇಕು
ಅರ್ಪಿ-ಸು-ತ್ತಿ-ರುವ
ಅರ್ಪಿ-ಸು-ತ್ತಿ-ರು-ವು-ದೆಂದು
ಅರ್ಪಿ-ಸುವ
ಅರ್ಪಿ-ಸು-ವಂ-ತಹ
ಅರ್ಪಿ-ಸು-ವಲ್ಲಿ
ಅರ್ಪಿ-ಸು-ವು-ದಾ-ದರೆ
ಅರ್ಪಿ-ಸೋಣ
ಅರ್ಹ-ತಾ-ಪತ್ರ
ಅರ್ಹತೆ
ಅರ್ಹ-ತೆ-ಗ-ಳಿ-ದ್ದರೂ
ಅರ್ಹ-ತೆ-ಗಿಂ-ತಲೂ
ಅರ್ಹ-ತೆಗೆ
ಅರ್ಹ-ತೆ-ಯನ್ನೂ
ಅರ್ಹ-ತೆ-ಯಿಂ-ದಲೇ
ಅರ್ಹ-ರಾ-ಗಿ-ದ್ದರು
ಅಲಂ-ಕ-ರಿ-ಸ-ಲ್ಪ-ಟ್ಟಿದ್ದ
ಅಲಂ-ಕ-ರಿ-ಸಿ-ದ-ವರು
ಅಲಂ-ಕಾ-ರ-ಪೂ-ರ್ಣ-ವಾ-ಗಿ-ದ್ದ-ರಿಂದ
ಅಲಂ-ಕಾ-ರ-ಮಯ
ಅಲಂ-ಕೃ-ತ-ಗೊಂ-ಡಿ-ದ್ದುವು
ಅಲ-ಕ್ಷ್ಯ-ಭಾ-ವ-ದಿಂದ
ಅಲ-ಹಾ-ಬಾ-ದಿನ
ಅಲ-ಹಾ-ಬಾ-ದ್ಗ-ಳಲ್ಲೂ
ಅಲಿ-ಖಾನ್
ಅಲು-ಗಾಡಿ
ಅಲು-ಗಾ-ಡಿ-ಬಿ-ಡು-ತ್ತ-ದಲ್ಲ
ಅಲು-ಗಾ-ಡಿ-ಸದೆ
ಅಲು-ಗಾ-ಡಿ-ಸ-ಬಲ್ಲ
ಅಲು-ಗಾ-ಡಿ-ಸ-ಬಲ್ಲೆ
ಅಲು-ಗಾ-ಡಿ-ಸಲು
ಅಲು-ಗಾ-ಡಿ-ಸಲೂ
ಅಲು-ಗಾ-ಡಿಸಿ
ಅಲು-ಗಾ-ಡಿ-ಸಿ-ಬಿಟ್ಟ
ಅಲು-ಗಾ-ಡಿ-ಸು-ತ್ತಿದ್ದ
ಅಲು-ಗಾ-ಡು-ತ್ತಿದೆ
ಅಲು-ಗಾ-ಡು-ತ್ತಿ-ದೆ-ಯೆಂ-ಬಂತೆ
ಅಲು-ಗಿ-ಸ-ಬಲ್ಲ
ಅಲೆ
ಅಲೆ-ಅ-ಲೆ-ಯಾಗಿ
ಅಲೆ-ಗಳ
ಅಲೆ-ಗ-ಳಂತೆ
ಅಲೆ-ಗಳನ್ನು
ಅಲೆ-ಗ-ಳ-ನ್ನೆ-ಬ್ಬಿ-ಸುತ್ತ
ಅಲೆ-ಗ-ಳ-ನ್ನೆ-ಬ್ಬಿ-ಸು-ತ್ತಿ-ದ್ದರು
ಅಲೆ-ಗಳಲ್ಲಿ
ಅಲೆ-ಗಳು
ಅಲೆ-ಗ-ಳೆ-ದ್ದುವು
ಅಲೆ-ಗ್ಸಾಂ-ಡರ್
ಅಲೆ-ದಾ-ಟ-ಗಳ
ಅಲೆ-ದಾ-ಟದ
ಅಲೆ-ದಾ-ಟ-ವೆಲ್ಲ
ಅಲೆ-ದಾ-ಡಿ-ದೆನೋ
ಅಲೆ-ದಾ-ಡುತ್ತ
ಅಲೆ-ದಾ-ಡು-ತ್ತವೆ
ಅಲೆ-ದಾ-ಡು-ತ್ತಿದ್ದ
ಅಲೆ-ದಾ-ಡು-ತ್ತಿ-ದ್ದ-ವರು
ಅಲೆ-ದಾ-ಡು-ತ್ತಿ-ರು-ವ-ವನೇ
ಅಲೆ-ದಾ-ಡು-ವು-ದೇಕೆ
ಅಲೆ-ಮಾರಿ
ಅಲೆ-ಯನ್ನೇ
ಅಲೆ-ಯಲ್ಲಿ
ಅಲೆಯು
ಅಲೆ-ಯುತ್ತ
ಅಲೆ-ಯುವ
ಅಲೆ-ಯೋ-ಪಾ-ದಿ-ಯಲ್ಲಿ
ಅಲೋಚಿ
ಅಲೌ-ಕಿಕ
ಅಲೌ-ಕಿ-ಕ-ವಾಗಿ
ಅಲೌ-ಕಿ-ಕ-ವಾದ
ಅಲ್ಪ
ಅಲ್ಪ-ವೆಂದು
ಅಲ್ಪ-ವೆಂದೇ
ಅಲ್ಪವೇ
ಅಲ್ಪ-ವ್ಯ-ಕ್ತಿ-ತ್ವ-ವನ್ನು
ಅಲ್ಪ-ಸ್ವಲ್ಪ
ಅಲ್ಪಾ-ಯು-ಸ್ಸಿ-ನಲ್ಲಿ
ಅಲ್ಪಾ-ಯುಸ್ಸು
ಅಲ್ಪಾ-ವಧಿ
ಅಲ್ಪಾ-ವ-ಧಿ-ಯಲ್ಲಿ
ಅಲ್ಪಾ-ವ-ಧಿ-ಯಲ್ಲೇ
ಅಲ್ಲ
ಅಲ್ಲ-ಗ-ಳೆದ
ಅಲ್ಲ-ಗ-ಳೆ-ಯ-ಲಾ-ಗದ
ಅಲ್ಲ-ಗ-ಳೆ-ಯಲು
ಅಲ್ಲ-ಗ-ಳೆ-ಯು-ತ್ತಿಲ್ಲ
ಅಲ್ಲ-ಗ-ಳೆ-ಯು-ವು-ದರ
ಅಲ್ಲದೆ
ಅಲ್ಲಲ್ಲಿ
ಅಲ್ಲ-ವಿ-ಶ್ವದ
ಅಲ್ಲವೆ
ಅಲ್ಲ-ವೆಂದು
ಅಲ್ಲ-ವೆಂಬ
ಅಲ್ಲವೇ
ಅಲ್ಲ-ಸ್ವತಃ
ಅಲ್ಲಾ
ಅಲ್ಲಾ-ಡಿ-ಸ-ತೊ-ಡ-ಗಿತು
ಅಲ್ಲಾ-ಡಿಸಿ
ಅಲ್ಲಾ-ಡಿ-ಸುತ್ತ
ಅಲ್ಲಿ
ಅಲ್ಲಿಂದ
ಅಲ್ಲಿಂ-ದಲೂ
ಅಲ್ಲಿಂ-ದಲೇ
ಅಲ್ಲಿಂ-ದೆದ್ದು
ಅಲ್ಲಿ-ಎ-ಲ್ಲ-ರಿ-ಗಿಂತ
ಅಲ್ಲಿಗೆ
ಅಲ್ಲಿದ್ದ
ಅಲ್ಲಿ-ದ್ದವ
ಅಲ್ಲಿ-ದ್ದ-ವ-ನೊಬ್ಬ
ಅಲ್ಲಿ-ದ್ದ-ವರಿ
ಅಲ್ಲಿ-ದ್ದ-ವ-ರಿಗೆ
ಅಲ್ಲಿ-ದ್ದ-ವ-ರಿ-ಗೆಲ್ಲ
ಅಲ್ಲಿ-ದ್ದ-ವರೆಲ್ಲ
ಅಲ್ಲಿ-ದ್ದ-ವರೆ-ಲ್ಲರ
ಅಲ್ಲಿ-ದ್ದ-ವ-ರೊಂ-ದಿಗೆ
ಅಲ್ಲಿ-ದ್ದ-ವ-ರೊ-ಬ್ಬರು
ಅಲ್ಲಿನ
ಅಲ್ಲಿ-ನ-ವರು
ಅಲ್ಲಿನ್ನು
ಅಲ್ಲಿಯ
ಅಲ್ಲಿ-ಯ-ವ-ರಿ-ಗೊಂದು
ಅಲ್ಲಿ-ಯ-ವ-ರೆಗೂ
ಅಲ್ಲಿ-ಯ-ವ-ರೆಗೆ
ಅಲ್ಲಿಯೂ
ಅಲ್ಲಿಯೇ
ಅಲ್ಲಿ-ರ-ಬ-ಹುದು
ಅಲ್ಲಿ-ರುವ
ಅಲ್ಲಿ-ರು-ವು-ದೆಲ್ಲ
ಅಲ್ಲೂ
ಅಲ್ಲೆಲ್ಲ
ಅಲ್ಲೇ
ಅಲ್ಲೇ-ಜ-ನರು
ಅಲ್ಲೊಂದು
ಅಲ್ಲೊಬ್ಬ
ಅಲ್ಲೋ-ಲ-ಕ-ಲ್ಲೋಲ
ಅಲ್ಲೋ-ಲ-ಕ-ಲ್ಲೋ-ಲ-ಗಳ
ಅಲ್ಲೋ-ಲ-ಕ-ಲ್ಲೋ-ಲ-ಗೊಂ-ಡಿದ್ದು
ಅಲ್ಲೋ-ಲ-ಕ-ಲ್ಲೋ-ಲ-ವನ್ನೇ
ಅಲ್ವ-ರಿನ
ಅಲ್ವ-ರಿ-ನತ್ತ
ಅಲ್ವ-ರಿ-ನಲ್ಲಿ
ಅಲ್ವ-ರಿ-ನಿಂದ
ಅಲ್ವರೀ
ಅಲ್ವರ್
ಅಳ-ತೆಗೇ
ಅಳ-ತೆ-ಗೋ-ಲಿಗೆ
ಅಳ-ಲಾ-ರಂ-ಭಿ-ಸಿದ
ಅಳ-ವ-ಡಿಸಿ
ಅಳ-ವ-ಡಿ-ಸಿ-ಕೊಂ-ಡಿ-ದ್ದುದು
ಅಳ-ವ-ಡಿ-ಸಿ-ಕೊ-ಳ್ಳ-ಬ-ಹುದು
ಅಳ-ವ-ಡಿ-ಸಿ-ಕೊ-ಳ್ಳ-ಬೇಕು
ಅಳ-ವ-ಡಿ-ಸಿ-ಕೊ-ಳ್ಳು-ವಂತೆ
ಅಳ-ವ-ಡಿ-ಸಿ-ಕೊ-ಳ್ಳು-ವು-ದರ
ಅಳ-ಸಿಂಗ
ಅಳ-ಸಿಂ-ಗ-ಇ-ಬ್ಬರೂ
ಅಳ-ಸಿಂ-ಗ-ನಿಗೆ
ಅಳ-ಸಿಂ-ಗರ
ಅಳ-ಸಿಂ-ಗ-ರಿಗೆ
ಅಳಿ-ದು-ಳಿದ
ಅಳಿಯ
ಅಳಿ-ಯು-ತ್ತವೆ
ಅಳಿ-ಸಿ-ಹೋ-ಗಿತ್ತು
ಅಳಿ-ಸಿ-ಹೋ-ಗಿ-ದ್ದಿತು
ಅಳಿ-ಸಿ-ಹೋ-ದರೂ
ಅಳು
ಅಳು-ನ-ಗು-ವಿನ
ಅಳು-ಕಾ-ಗಲಿ
ಅಳು-ಕಿತು
ಅಳು-ಕಿ-ಲ್ಲದೆ
ಅಳುಕು
ಅಳುತ್ತ
ಅಳು-ತ್ತಿತ್ತು
ಅಳು-ತ್ತಿ-ದ್ದೀರಿ
ಅಳು-ತ್ತಿ-ದ್ದೇನೆ
ಅಳು-ತ್ತಿ-ರುವೆ
ಅಳು-ತ್ತಿ-ರು-ವೆ-ಯಪ್ಪ
ಅಳು-ವಿಗೆ
ಅಳು-ವುದು
ಅಳೆದ
ಅಳೆ-ದು-ನೋ-ಡುವ
ಅಳೆದೆ
ಅಳೆ-ಯುವ
ಅವ
ಅವ-ಕಾಶ
ಅವ-ಕಾ-ಶ-ಗಳು
ಅವ-ಕಾ-ಶ-ವನ್ನು
ಅವ-ಕಾ-ಶ-ವನ್ನೂ
ಅವ-ಕಾ-ಶ-ವಾ-ಗಲಿ
ಅವ-ಕಾ-ಶ-ವಾ-ಗ-ಲೆಂದು
ಅವ-ಕಾ-ಶ-ವಾ-ಯಿತು
ಅವ-ಕಾ-ಶ-ವಿತ್ತು
ಅವ-ಕಾ-ಶ-ವಿದೆ
ಅವ-ಕಾ-ಶ-ವಿದ್ದ
ಅವ-ಕಾ-ಶ-ವಿ-ರ-ಬೇಕು
ಅವ-ಕಾ-ಶ-ವಿ-ರುವ
ಅವ-ಕಾ-ಶವೂ
ಅವ-ಕಾ-ಶ-ವೆ-ಲ್ಲಿ-ರು-ತ್ತಿತ್ತು
ಅವ-ಕಾ-ಶವೇ
ಅವ-ಗಢ
ಅವ-ಗುಣ
ಅವ-ಗು-ಣ-ವ-ನೆ-ಣಿ-ಸ-ದಿರು
ಅವಜ್ಞೆ
ಅವ-ಡು-ಗಚ್ಚಿ
ಅವ-ತ-ರ-ಣ-ದಿಂ-ದಾಗಿ
ಅವ-ತ-ರಿಸಿ
ಅವ-ತ-ರಿ-ಸಿ-ದ-ವ-ರಲ್ಲ
ಅವ-ತ-ರಿ-ಸಿ-ರು-ವುದೇ
ಅವ-ತಾರ
ಅವ-ತಾ-ರ-ಗಳ
ಅವ-ತಾ-ರ-ಗಳಲ್ಲಿ
ಅವ-ತಾ-ರ-ಗ-ಳ-ಲ್ಲೊಂ-ದಾದ
ಅವ-ತಾ-ರ-ಗ-ಳಿ-ಗಿಂ-ತಲೂ
ಅವ-ತಾ-ರ-ಗ-ಳಿಗೂ
ಅವ-ತಾ-ರ-ನಾದ
ಅವ-ತಾ-ರ-ಪು-ರು-ಷ-ನೆಂದು
ಅವ-ತಾ-ರ-ಪು-ರು-ಷರು
ಅವ-ತಾ-ರ-ವ-ರಿ-ಷ್ಠ-ರಾದ
ಅವ-ತಾ-ರ-ವಾದ
ಅವ-ತಾ-ರ-ವೆಂದು
ಅವ-ತಾ-ರ-ವೆಂ-ಬು-ದ-ರಲ್ಲಿ
ಅವಧಿ
ಅವ-ಧಿಯ
ಅವ-ಧಿ-ಯಲ್ಲಿ
ಅವ-ಧಿ-ಯಲ್ಲೇ
ಅವ-ಧಿಯೇ
ಅವ-ಧೂತ
ಅವನ
ಅವ-ನಂ-ತೆಯೇ
ಅವ-ನತಿ
ಅವ-ನ-ತಿಗೂ
ಅವ-ನ-ತಿಗೆ
ಅವ-ನ-ತಿಯ
ಅವ-ನ-ತಿ-ಯನ್ನು
ಅವ-ನತ್ತ
ಅವ-ನದು
ಅವ-ನದ್ದು
ಅವ-ನನ್ನು
ಅವ-ನನ್ನೇ
ಅವ-ನ-ಲ್ಲದೆ
ಅವ-ನಲ್ಲಿ
ಅವ-ನ-ಲ್ಲಿಗೆ
ಅವ-ನ-ಲ್ಲೊಂದು
ಅವ-ನ-ವನ
ಅವ-ನ-ವ-ನದೇ
ಅವ-ನ-ಷ್ಟಕ್ಕೆ
ಅವ-ನಾ-ಗಲೇ
ಅವ-ನಾ-ಜ್ಞೆ-ಯನ್ನು
ಅವ-ನಿಂದ
ಅವ-ನಿ-ಗ-ದನ್ನು
ಅವ-ನಿ-ಗ-ನ್ನಿ-ಸಿತು
ಅವ-ನಿ-ಗಿನ್ನು
ಅವ-ನಿ-ಗಿ-ರ-ಲಿಲ್ಲ
ಅವ-ನಿ-ಗಿ-ರುವ
ಅವ-ನಿಗೂ
ಅವ-ನಿಗೆ
ಅವ-ನಿಗೇ
ಅವ-ನಿ-ಗೇ-ನ-ನ್ನಿ-ಸಿತೋ
ಅವ-ನಿ-ಗೊಂದು
ಅವ-ನಿ-ಗೊಂ-ದೆ-ರಡು
ಅವ-ನಿ-ಗೊಬ್ಬ
ಅವ-ನಿಚ್ಛೆ
ಅವ-ನಿ-ಚ್ಛೆ-ಯಂ-ತೆಯೇ
ಅವ-ನಿನ್ನೂ
ಅವ-ನಿಲ್ಲ
ಅವನು
ಅವನೂ
ಅವನೆ
ಅವ-ನೆ-ದೆ-ಯನ್ನು
ಅವ-ನೆ-ನ್ನು-ತ್ತಾನೆ
ಅವ-ನೆಷ್ಟೇ
ಅವನೇ
ಅವ-ನೇ-ನಾ-ದರೂ
ಅವ-ನೊಂ-ದಿಗೆ
ಅವ-ನೊಂದು
ಅವ-ನೊ-ಡನೆ
ಅವ-ನೊಬ್ಬ
ಅವ-ನೊ-ಬ್ಬನೇ
ಅವನ್ನು
ಅವ-ನ್ನೆಲ್ಲ
ಅವ-ಮ-ರ್ಯಾ-ದೆ-ಯಾಗಿ
ಅವ-ಮಾನ
ಅವ-ಮಾ-ನ-ಕರ
ಅವ-ಮಾ-ನ-ದಿಂದ
ಅವ-ಮಾ-ನ-ವಾ-ದಂತಾ
ಅವ-ಮಾ-ನ-ವಾ-ದರೂ
ಅವ-ಮಾ-ನ-ವೆಂದರೆ
ಅವ-ಮಾ-ನಿ-ಸ-ಲಾಗಿದೆ
ಅವ-ಮಾ-ನಿಸಿ
ಅವರ
ಅವ-ರಂ-ತೆಯೇ
ಅವ-ರತ್ತ
ಅವ-ರ-ದನ್ನು
ಅವ-ರ-ದಲ್ಲ
ಅವ-ರ-ದಾ-ಗಿತ್ತು
ಅವ-ರದು
ಅವ-ರದೇ
ಅವ-ರ-ದೊಂದು
ಅವ-ರ-ದ್ದಕ್ಕೆ
ಅವ-ರದ್ದು
ಅವ-ರ-ನ್ನಾ-ವ-ರಿ-ಸಿ-ಬಿ-ಟ್ಟಿತು
ಅವ-ರನ್ನು
ಅವ-ರನ್ನೂ
ಅವ-ರ-ನ್ನೆಲ್ಲ
ಅವ-ರನ್ನೇ
ಅವ-ರ-ನ್ನೇಕೆ
ಅವ-ರ-ಪ್ರೀ-ತಿ-ವಿ-ಶ್ವಾ-ಸ-ಗ-ಳಿ-ಗಾಗಿ
ಅವ-ರ-ಲ್ಲ-ಡ-ಗಿದ್ದ
ಅವ-ರಲ್ಲಿ
ಅವ-ರ-ಲ್ಲಿ-ಇ-ಲ್ಲ-ದಿ-ರು-ವುದು
ಅವ-ರ-ಲ್ಲಿಗೆ
ಅವ-ರ-ಲ್ಲಿತ್ತು
ಅವ-ರ-ಲ್ಲಿದೆ
ಅವ-ರ-ಲ್ಲಿದ್ದ
ಅವ-ರ-ಲ್ಲಿ-ದ್ದರೂ
ಅವ-ರ-ಲ್ಲಿ-ರು-ವುದು
ಅವ-ರ-ಲ್ಲೀಗ
ಅವ-ರ-ಲ್ಲುಂ-ಟಾದ
ಅವ-ರಲ್ಲೂ
ಅವ-ರಲ್ಲೇ
ಅವ-ರ-ಲ್ಲೊಂದು
ಅವ-ರ-ಲ್ಲೊಬ್ಬ
ಅವ-ರ-ಲ್ಲೊ-ಬ್ಬರು
ಅವ-ರ-ವರ
ಅವ-ರ-ವ-ರನ್ನು
ಅವ-ರ-ಷ್ಟಕ್ಕೆ
ಅವ-ರಾ-ಡಿದ
ಅವ-ರಾ-ಡಿದ್ದೇ
ಅವ-ರಾ-ಡು-ತ್ತಿದ್ದ
ಅವ-ರಾ-ಡುವ
ಅವ-ರಾ-ರಿಗೂ
ಅವರಿ
ಅವ-ರಿಂದ
ಅವ-ರಿಂ-ದಲೇ
ಅವ-ರಿಂದೇ
ಅವ-ರಿ-ಗಂತೂ
ಅವ-ರಿ-ಗ-ದರ
ಅವ-ರಿ-ಗದು
ಅವ-ರಿ-ಗ-ನ್ನಿ-ಸಿತು
ಅವ-ರಿ-ಗ-ನ್ನಿಸು
ಅವ-ರಿ-ಗ-ನ್ನಿ-ಸು-ತ್ತಿತ್ತು
ಅವ-ರಿ-ಗ-ರಿ-ವಿ-ಲ್ಲ-ದಂ-ತೆಯೇ
ಅವ-ರಿ-ಗಾ-ಗಲೇ
ಅವ-ರಿ-ಗಾಗಿ
ಅವ-ರಿ-ಗಾದ
ಅವ-ರಿ-ಗಾವ
ಅವ-ರಿ-ಗಿಂತ
ಅವ-ರಿ-ಗಿತ್ತು
ಅವ-ರಿ-ಗಿದ್ದ
ಅವ-ರಿ-ಗಿ-ದ್ದರೂ
ಅವ-ರಿ-ಗಿನ್ನೂ
ಅವ-ರಿ-ಗಿರ
ಅವ-ರಿ-ಗಿ-ರ-ಲಿಲ್ಲ
ಅವ-ರಿ-ಗಿ-ರುವ
ಅವ-ರಿ-ಗೀಗ
ಅವ-ರಿ-ಗುಂ-ಟಾ-ಗಿತ್ತು
ಅವ-ರಿ-ಗುಂ-ಟಾ-ಗು-ತ್ತಿತ್ತು
ಅವ-ರಿಗೂ
ಅವ-ರಿಗೆ
ಅವ-ರಿ-ಗೆಂಥ
ಅವ-ರಿ-ಗೆಂ-ದಿಗೂ
ಅವ-ರಿ-ಗೆಲ್ಲ
ಅವ-ರಿಗೇ
ಅವ-ರಿ-ಗೇನು
ಅವ-ರಿ-ಗೇನೂ
ಅವ-ರಿ-ಗೊಂ-ದಿಷ್ಟು
ಅವ-ರಿ-ಗೊಂದು
ಅವ-ರಿ-ಗೊ-ಬ್ಬರು
ಅವ-ರಿ-ಗೋ-ಸ್ಕರ
ಅವ-ರಿದ್ದ
ಅವ-ರಿ-ದ್ದಂ-ತಹ
ಅವ-ರಿನ್ನು
ಅವ-ರಿನ್ನೂ
ಅವ-ರಿ-ಬ್ಬರ
ಅವ-ರಿ-ಬ್ಬ-ರನ್ನೂ
ಅವ-ರಿ-ಬ್ಬರು
ಅವ-ರಿ-ಬ್ಬರೂ
ಅವ-ರಿ-ರು-ವ-ಲ್ಲಿಗೆ
ಅವ-ರಿಲ್ಲಿ
ಅವ-ರೀಗ
ಅವ-ರೀ-ರ್ವರ
ಅವರು
ಅವ-ರು-ಗಳ
ಅವರೂ
ಅವ-ರೆಂ-ತಹ
ಅವ-ರೆಂ-ದರು
ಅವ-ರೆಂದಾ
ಅವ-ರೆಂದೂ
ಅವ-ರೆಡೂ
ಅವರೆ-ಡೆಗೆ
ಅವರೆ-ಡೆಗೇ
ಅವರೆ-ದು-ರಿಗೆ
ಅವರೆ-ದು-ರಿ-ನಲ್ಲಿ
ಅವರೆ-ದೆ-ಯಲ್ಲಿ
ಅವರೆ-ನ್ನು-ತ್ತಾರೆ
ಅವರೆ-ನ್ನು-ತ್ತಿ-ದ್ದರು
ಅವರೆಲ್ಲ
ಅವರೆ-ಲ್ಲರ
ಅವರೆ-ಲ್ಲ-ರನ್ನೂ
ಅವರೆ-ಲ್ಲ-ರಿಗೂ
ಅವರೆ-ಲ್ಲರೂ
ಅವರೆ-ಲ್ಲ-ರೊಂ-ದಿಗೂ
ಅವರೆಲ್ಲಿ
ಅವರೆಲ್ಲೂ
ಅವರೆ-ಷ್ಟಾ-ದರೂ
ಅವ-ರೆಷ್ಟು
ಅವ-ರೆಷ್ಟೇ
ಅವರೇ
ಅವ-ರೇಕೆ
ಅವ-ರೇ-ನಾ-ದರೂ
ಅವ-ರೇನು
ಅವ-ರೇನೂ
ಅವ-ರೇ-ನೆಂ-ದು-ಕೊಂ-ಡಾರು
ಅವ-ರೇನೋ
ಅವ-ರೊಂ-ದಿ-ಗಿದ್ದ
ಅವ-ರೊಂ-ದಿ-ಗಿನ
ಅವ-ರೊಂ-ದಿಗೆ
ಅವ-ರೊಂ-ದಿಗೇ
ಅವ-ರೊಂದು
ಅವ-ರೊ-ಡನೆ
ಅವ-ರೊಬ್ಬ
ಅವ-ರೊ-ಬ್ಬರು
ಅವ-ರೊ-ಬ್ಬರೇ
ಅವ-ರೊಮ್ಮೆ
ಅವ-ರೊ-ಲ್ಲೊಂದು
ಅವ-ರೊ-ಳಗಿ
ಅವ-ರೊ-ಳ-ಗಿನ
ಅವ-ರೊ-ಳ-ಗಿ-ನಿಂದ
ಅವ-ರೊ-ಳಗೆ
ಅವ-ರೊ-ಳ-ಗೆಲ್ಲ
ಅವರ್ನು
ಅವ-ರ್ಯಾರೂ
ಅವ-ಲಂ-ಬಿ-ಸ-ಬೇ-ಕಾ-ಗಿತ್ತು
ಅವ-ಲಂ-ಬಿಸಿ
ಅವ-ಲಂ-ಬಿ-ಸಿ-ಕೊಂ-ಡಿದ್ದ
ಅವ-ಲಂ-ಬಿ-ಸಿ-ಕೊಂ-ಡಿದ್ದು
ಅವ-ಲಂ-ಬಿ-ಸಿ-ಕೊಂ-ಡಿರ
ಅವ-ಲಂ-ಬಿ-ಸಿ-ಕೊಂ-ಡಿ-ರು-ವು-ದ-ರಿಂದ
ಅವ-ಲಂ-ಬಿ-ಸಿ-ಕೊಂ-ಡಿ-ರು-ವುದು
ಅವ-ಲಂ-ಬಿ-ಸಿ-ಕೊಂ-ಡಿ-ವೆ-ಕ್ರೈ-ಸ್ತ-ಧ-ರ್ಮವು
ಅವ-ಲಂ-ಬಿ-ಸಿ-ಕೊ-ಳ್ಳು-ವು-ದಿಲ್ಲ
ಅವ-ಲಂ-ಬಿ-ಸಿದೆ
ಅವ-ಲಂ-ಬಿ-ಸಿ-ದ್ದರು
ಅವ-ಲಂ-ಬಿ-ಸಿ-ರು-ತ್ತದೆ
ಅವ-ಲಂ-ಬಿ-ಸಿ-ರು-ವುದು
ಅವ-ಲಂ-ಬಿಸು
ಅವ-ಲಾ-ನ್ಷ್
ಅವ-ಲೋ-ಕನ
ಅವ-ಲೋ-ಕಿ-ಸ-ಬ-ಹು-ದಾ-ಗಿದೆ
ಅವ-ಲೋ-ಕಿ-ಸಿ-ದರು
ಅವ-ಲೋ-ಕಿ-ಸಿ-ದರೆ
ಅವಳ
ಅವ-ಳತ್ತ
ಅವ-ಳ-ದನ್ನು
ಅವ-ಳ-ನ್ನಿನ್ನೂ
ಅವ-ಳನ್ನು
ಅವ-ಳ-ಲ್ಲೊಂದು
ಅವ-ಳಷ್ಟು
ಅವಳಿ
ಅವ-ಳಿ-ಗ-ದನ್ನು
ಅವ-ಳಿ-ಗ-ನ್ನಿ-ಸಿತು
ಅವ-ಳಿ-ಗಾದ
ಅವ-ಳಿ-ಗಿನ್ನು
ಅವ-ಳಿಗೆ
ಅವ-ಳಿ-ಗೊಂದು
ಅವಳು
ಅವಳೂ
ಅವ-ಳೆ-ಡೆಗೆ
ಅವ-ಳೆಷ್ಟೋ
ಅವಳೇ
ಅವ-ಳೊಂ-ದಿಗೇ
ಅವ-ಳೊಂದು
ಅವ-ಶೇ-ಷಕ್ಕೆ
ಅವ-ಶೇ-ಷ-ಗಳ
ಅವ-ಶೇ-ಷ-ಗ-ಳಿ-ಗಾಗಿ
ಅವ-ಶ್ಯ-ಕತೆ
ಅವ-ಶ್ಯ-ಕ-ತೆ-ಗ-ಳಿ-ಗ-ನು-ಗು-ಣ-ವಾಗಿ
ಅವ-ಶ್ಯ-ಕ-ತೆ-ಗ-ಳಿ-ಗಾಗಿ
ಅವ-ಶ್ಯ-ಕ-ತೆ-ಗಳು
ಅವ-ಶ್ಯ-ಕ-ತೆ-ಯಿ-ಲ್ಲ-ವೆಂದೂ
ಅವ-ಶ್ಯ-ಕ-ತೆ-ಯೇ-ನಿತ್ತು
ಅವ-ಶ್ಯ-ಕ-ವಾದ
ಅವ-ಶ್ಯ-ವಾ-ಗಿತ್ತು
ಅವ-ಸರ
ಅವ-ಸ-ರ-ಪ-ಡಿ-ಸಿದ
ಅವ-ಸ-ರ-ಪ-ಡಿ-ಸು-ತ್ತಿ-ದ್ದರು
ಅವ-ಸ-ರ-ವ-ಸ-ರ-ವಾಗಿ
ಅವ-ಸ-ರ-ವಿಲ್ಲ
ಅವಸ್ಥೆ
ಅವ-ಸ್ಥೆ-ಗೇ-ರ-ಬೇ-ಕೆಂಬ
ಅವ-ಸ್ಥೆ-ಯನ್ನು
ಅವ-ಸ್ಥೆ-ಯ-ಲ್ಲಿ-ಟ್ಟಿದೆ
ಅವ-ಸ್ಥೆ-ಯಿಂದ
ಅವ-ಹೇ-ಳ-ನ-ಕ-ರ-ವಾದ
ಅವಾ-ಕ್ಕಾ-ಗು-ತ್ತಿ-ದ್ದರು
ಅವಾ-ಕ್ಕಾ-ದರು
ಅವಿ
ಅವಿ-ಚಲ
ಅವಿ-ಚ್ಛಿ-ನ್ನ-ವಾದ
ಅವಿ-ತಿ-ಟ್ಟು-ಕೊಂ-ಡಿ-ದ್ದಾನು
ಅವಿ-ತಿ-ಟ್ಟು-ಕೊಂ-ಡಿ-ರೋ-ಣವೆ
ಅವಿ-ತು-ಕೊ-ಳ್ಳುವ
ಅವಿ-ದ್ಯಾ-ವಂತ
ಅವಿ-ದ್ಯಾ-ವಂ-ತ-ರಿಗೆ
ಅವಿ-ದ್ಯಾ-ವಂ-ತ-ರು-ಹೀಗೆ
ಅವಿ-ದ್ಯಾ-ವಂ-ತರೂ
ಅವಿ-ನ-ಯ-ದಿಂದ
ಅವಿ-ನಾ-ಶಿ-ಯಾದ
ಅವಿ-ನ್ಯೂ-ದಲ್ಲಿ
ಅವಿ-ಭ-ಜಿತ
ಅವಿ-ಭಾಜ್ಯ
ಅವಿ-ರತ
ಅವಿ-ರ-ತ-ವಾಗಿ
ಅವಿ-ರ್ಭ-ವಿಸಿ
ಅವಿ-ರ್ಭಾ-ವ-ವನ್ನು
ಅವಿ-ವಾ-ಹಿತೆ
ಅವಿ-ವೇಕ
ಅವಿ-ವೇ-ಕ-ಗಳೆ
ಅವಿ-ವೇ-ಕದ
ಅವಿ-ವೇ-ಕ-ವನ್ನು
ಅವಿ-ಶ್ರಾಂತ
ಅವಿ-ಶ್ರಾಂ-ತರೂ
ಅವಿ-ಶ್ರಾಂ-ತ-ವಾಗಿ
ಅವಿ-ಸ್ಮ-ರ-ಣೀಯ
ಅವಿ-ಸ್ಮ-ರ-ಣೀ-ಯ-ವಾಗಿ
ಅವು
ಅವು-ಗಳ
ಅವು-ಗಳನ್ನು
ಅವು-ಗಳನ್ನೂ
ಅವು-ಗಳನ್ನೆಲ್ಲ
ಅವು-ಗಳಲ್ಲಿ
ಅವು-ಗ-ಳ-ಲ್ಲಿ-ರುವ
ಅವು-ಗ-ಳ-ಲ್ಲೊಂ-ದನ್ನು
ಅವು-ಗಳಿ
ಅವು-ಗಳಿಂದ
ಅವು-ಗ-ಳಿಂ-ದಲೇ
ಅವು-ಗ-ಳಿಗೂ
ಅವು-ಗ-ಳಿಗೆ
ಅವು-ಗ-ಳಿ-ಗೆಲ್ಲ
ಅವು-ಗಳು
ಅವು-ಗ-ಳೆ-ಡೆಗೆ
ಅವು-ಗ-ಳೆ-ಲ್ಲ-ದರ
ಅವು-ಗ-ಳೊಂ-ದಿಗೆ
ಅವೆ-ನ್ಯೂ-ನಲ್ಲಿ
ಅವೆ-ನ್ಯೂ-ನ-ಲ್ಲಿದ್ದ
ಅವೆ-ರ-ಡನ್ನೂ
ಅವೆಲ್ಲ
ಅವೆ-ಲ್ಲಕ್ಕೂ
ಅವೆ-ಲ್ಲ-ದರ
ಅವೆ-ಲ್ಲ-ದ-ರಿಂ-ದಲೂ
ಅವೆ-ಲ್ಲವೂ
ಅವ್ಯಕ್ತ
ಅವ್ಯ-ಕ್ತನೂ
ಅವ್ಯ-ಕ್ತ-ವಾದ
ಅವ್ಯ-ವ-ಸ್ಥಿತ
ಅವ್ಯ-ವ-ಸ್ಥೆಗೆ
ಅಶ-ರೀ-ರ-ವಾಣಿ
ಅಶ-ರೀ-ರ-ವಾ-ಣಿ-ಯಾ-ಗಿ-ರ-ಬೇ-ಕೆಂ-ಬುದು
ಅಶೋ-ಕನ
ಅಶ್ಚ-ರ್ಯ-ಕ-ರವೂ
ಅಶ್ಮಾನ್
ಅಶ್ರು-ಜ-ಲ-ದಿಂದ
ಅಶ್ರು-ಧಾರೆ
ಅಶ್ರು-ಭ-ರಿತ
ಅಶ್ಲೀಲ
ಅಶ್ಲೀ-ಲ-ವೆಂದು
ಅಶ್ಲೀ-ಲವೇ
ಅಶ್ವಾ-ಸನೆ
ಅಷ್ಟ
ಅಷ್ಟ-ಕಷ್ಟೆ
ಅಷ್ಟಕ್ಕೇ
ಅಷ್ಟನ್ನೇ
ಅಷ್ಟರ
ಅಷ್ಟ-ರ-ಮ-ಟ್ಟಿಗೆ
ಅಷ್ಟ-ರಲ್ಲಿ
ಅಷ್ಟ-ರಲ್ಲೇ
ಅಷ್ಟ-ರೊ-ಳಗೇ
ಅಷ್ಟಾಗಿ
ಅಷ್ಟಾ-ಧ್ಯಾಯಿ
ಅಷ್ಟಾ-ಧ್ಯಾ-ಯಿ-ಯನ್ನು
ಅಷ್ಟಿ-ಷ್ಟಲ್ಲ
ಅಷ್ಟು
ಅಷ್ಟು-ಹೊ-ತ್ತಿಗೆ
ಅಷ್ಟೆ
ಅಷ್ಟೆಲ್ಲ
ಅಷ್ಟೇ
ಅಷ್ಟೇಕೆ
ಅಷ್ಟೇನೂ
ಅಷ್ಟೊಂದು
ಅಸಂಖ್ಯ
ಅಸಂ-ಖ್ಯಾತ
ಅಸಂ-ಬಂದ್ಧ
ಅಸಂ-ಬದ್ಧ
ಅಸಂ-ಬ-ದ್ಧ-ಗಳನ್ನೆಲ್ಲ
ಅಸಂ-ಬ-ದ್ಧತೆ
ಅಸಂ-ಬ-ದ್ಧ-ತೆ-ಅ-ವಿ-ವೇ-ಕ-ಗಳು
ಅಸಂ-ಸ್ಕೃತ
ಅಸಂ-ಸ್ಕೃ-ತ-ಅ-ನಾ-ಗ-ರಿ-ಕರು
ಅಸಂ-ಸ್ಕೃ-ತ-ರೆಂದು
ಅಸಂ-ಸ್ಕೃ-ತ-ವಾ-ಗಿ-ರು-ತ್ತದೆ
ಅಸ-ಡ್ಡೆಯ
ಅಸ-ತ್ಯ-ದಿಂದ
ಅಸ-ದೃ-ಶ-ವಾ-ದದ್ದು
ಅಸ-ಭ್ಯ-ರೇ-ನಲ್ಲ
ಅಸ-ಭ್ಯ-ವಾಗಿ
ಅಸ-ಮ-ತೋ-ಲ-ನದ
ಅಸ-ಮರ್ಥ
ಅಸ-ಮ-ರ್ಥ-ರಾ-ಗಿ-ದ್ದರು
ಅಸ-ಮ-ರ್ಥ-ರಾ-ದಾಗ
ಅಸ-ಮ-ರ್ಪ-ಕ-ವಾಗಿ
ಅಸ-ಮ-ರ್ಪ-ಕ-ವಾದು
ಅಸ-ಮ-ರ್ಪ-ಕ-ವೆಂದು
ಅಸ-ಮಾ-ಧಾನ
ಅಸ-ಮಾ-ಧಾ-ನ-ಗೊಂಡ
ಅಸ-ಮಾ-ಧಾ-ನ-ಗೊಂಡು
ಅಸ-ಮಾ-ಧಾ-ನ-ಪ-ಟ್ಟು-ಕೊಂ-ಡಿಲ್ಲ
ಅಸ-ಮಾ-ಧಾ-ನ-ವನ್ನು
ಅಸ-ಮಾ-ಧಾ-ನ-ವಾ-ಯಿತು
ಅಸ-ಹ-ಜ-ವಾ-ಗಿಯೋ
ಅಸ-ಹ-ನೀಯ
ಅಸ-ಹ-ನೀ-ಯವಾ
ಅಸ-ಹ-ನೀ-ಯ-ವಾ-ಗಿತ್ತು
ಅಸ-ಹ-ನೀ-ಯ-ವಾದ
ಅಸ-ಹನೆ
ಅಸ-ಹ-ನೆ-ಮ-ನ-ಸ್ತಾ-ಪ-ಗ-ಳಿಗೆ
ಅಸ-ಹ-ನೆ-ಗೊಂ-ಡು-ಅಂಥ
ಅಸ-ಹ-ನೆ-ಯನ್ನೂ
ಅಸ-ಹ-ನೆ-ಯಿಂದ
ಅಸ-ಹ-ನೆಯು
ಅಸ-ಹಾ-ಯಕ
ಅಸ-ಹಿ-ಷ್ಣುತೆ
ಅಸ-ಹಿ-ಷ್ಣು-ತೆಯ
ಅಸ-ಹ್ಯ-ವಾ-ಯಿತು
ಅಸ-ಹ್ಯ-ವಾ-ಯಿ-ತೆಂ-ದರೆ
ಅಸಾ-ಧಾ-ರಣ
ಅಸಾ-ಧಾ-ರ-ಣ-ಅ-ಲೌ-ಕಿಕ
ಅಸಾ-ಧಾ-ರ-ಣವೂ
ಅಸಾಧ್ಯ
ಅಸಾ-ಧ್ಯ-ವಾ-ಗಿತ್ತು
ಅಸಾ-ಧ್ಯ-ವಾ-ಗಿದೆ
ಅಸಾ-ಧ್ಯ-ವೆಂ-ಬು-ದನ್ನು
ಅಸಾ-ಮಾನ್ಯ
ಅಸಾ-ಹ-ಯಕ
ಅಸೀಮ
ಅಸು-ರರು
ಅಸು-ರರೂ
ಅಸೂ-ಯಾ-ಪರ
ಅಸೂ-ಯಾ-ಪ-ರರ
ಅಸೂಯೆ
ಅಸೂ-ಯೆ-ಗ-ಳೆಲ್ಲ
ಅಸೂ-ಯೆ-ಪ-ಡ-ಬೇಕು
ಅಸೂ-ಯೆ-ಯಂತೂ
ಅಸೂ-ಯೆ-ಯನ್ನು
ಅಸೂ-ಯೆ-ಯನ್ನೂ
ಅಸೂ-ಯೆ-ಯಿಂ-ದಾಗಿ
ಅಸೋ-ಸಿ-ಯೇ-ಶನ್
ಅಸೋ-ಸಿ-ಯೇ-ಶನ್ನ
ಅಸೋ-ಸಿ-ಯೇ-ಶ-ನ್ನಲ್ಲಿ
ಅಸೋ-ಸಿ-ಯೇ-ಶ-ನ್ನಿನ
ಅಸೋ-ಸಿ-ಯೇ-ಷ-ನ್ನಿನ
ಅಸೋ-ಸಿ-ಯೇ-ಷ-ನ್ನಿ-ನಲ್ಲಿ
ಅಸ್ತ-ವ್ಯಸ್ತ
ಅಸ್ತ-ವ್ಯ-ಸ್ತ-ಗೊಂಡ
ಅಸ್ತ-ವ್ಯ-ಸ್ತ-ಗೊ-ಳಿ-ಸಿ-ಬಿ-ಟ್ಟರು
ಅಸ್ತಿತ್ವ
ಅಸ್ತಿ-ತ್ವಕ್ಕೆ
ಅಸ್ತಿ-ತ್ವ-ದ-ಲ್ಲಿತ್ತಾ
ಅಸ್ತಿ-ತ್ವ-ದ-ಲ್ಲಿ-ದ್ದುದೇ
ಅಸ್ತಿ-ತ್ವ-ವನ್ನೇ
ಅಸ್ತೇಯ
ಅಸ್ತ್ರ-ಗ-ಳೆಲ್ಲ
ಅಸ್ಪಷ್ಟ
ಅಸ್ಪ-ಷ್ಟ-ವಾಗಿ
ಅಸ್ಪ-ಷ್ಟ-ವಾ-ಗಿ-ಯಾ-ದರೂ
ಅಸ್ಪೃ-ಶ್ಯ-ನಂತೆ
ಅಸ್ಪೃ-ಶ್ಯ-ರಂತೆ
ಅಸ್ಪೃ-ಶ್ಯರು
ಅಸ್ಪೃ-ಶ್ಯ-ರೆಂದು
ಅಸ್ಪೃ-ಶ್ಯ-ರೆ-ನಿ-ಸಿ-ಕೊಂ-ಡ-ವರ
ಅಸ್ವ-ಸ್ಥ-ತೆ-ಗ-ಳಿಗೆ
ಅಸ್ವ-ಸ್ಥ-ತೆ-ಯನ್ನು
ಅಸ್ವ-ಸ್ಥ-ತೆ-ಯಲ್ಲ
ಅಸ್ವ-ಸ್ಥ-ತೆ-ಯುಂ-ಟಾ-ಗ-ದಿ-ರಲು
ಅಸ್ವಾ-ದಿ-ಸುತ್ತಿ
ಅಹಂ
ಅಹಂ-ಕಾರ
ಅಹಂ-ಕಾ-ರದ
ಅಹಂ-ಕಾ-ರ-ರಾ-ಹಿತ್ಯ
ಅಹಂ-ಕಾ-ರ-ವನ್ನು
ಅಹಂ-ಕಾ-ರಿ-ಯೆಂದು
ಅಹಂ-ಭಾ-ವದ
ಅಹ-ಮ-ದಾ-ಬಾ-ದಿಗೆ
ಅಹ-ಮ-ದಾ-ಬಾ-ದಿಗೇ
ಅಹ-ಮ-ದಾ-ಬಾ-ದಿನ
ಅಹ-ಮ-ದಾ-ಬಾ-ದಿ-ನಲ್ಲಿ
ಅಹ-ಮಿ-ಕೆ-ಯನ್ನು
ಅಹ-ಮಿ-ಕೆಯು
ಅಹ-ಸ-ನೆಯ
ಅಹಾ
ಅಹಿಂಸೆ
ಅಹಿ-ತ-ಕರ
ಅಹಿ-ತ-ಕ-ರ-ವಾಗಿ
ಅಹಿ-ತ-ಕ-ರ-ವಾದ
ಅಹು-ದ-ಹು-ದೆ-ನ್ನಿ-ಸು-ತ್ತಿತ್ತು
ಅಹುರ
ಅಹೈ-ತುಕ
ಆ
ಆಂಗ್ಲ
ಆಂಗ್ಲರ
ಆಂಗ್ಲ-ರನ್ನು
ಆಂಗ್ಲೀ-ಕ-ರ-ಣ-ಗೊಂಡ
ಆಂಗ್ಲೊ
ಆಂಗ್ಲೋ
ಆಂಟೇ-ರಿಯೋ
ಆಂತ-ರಿಕ
ಆಂತ-ರಿ-ಕ-ವಾ-ದ-ವೆಂದು
ಆಂತ-ರ್ಯ-ದಿಂದ
ಆಂತಿ-ಮ-ವಾಗಿ
ಆಂದೋ-ಳ-ನದ
ಆಂದೋ-ಳ-ನ-ವನ್ನು
ಆಂದೋ-ಳ-ನವು
ಆಕ-ರ್ಷಕ
ಆಕ-ರ್ಷಣ
ಆಕ-ರ್ಷಣೆ
ಆಕ-ರ್ಷ-ಣೆ-ಗ-ಳ-ಲ್ಲೊಂದು
ಆಕ-ರ್ಷ-ಣೆ-ಗ-ಳೆಲ್ಲ
ಆಕ-ರ್ಷ-ಣೆಗೆ
ಆಕ-ರ್ಷ-ಣೆಯ
ಆಕ-ರ್ಷ-ಣೆ-ಯಲ್ಲಿ
ಆಕ-ರ್ಷಿತ
ಆಕ-ರ್ಷಿ-ತ-ನಾಗಿ
ಆಕ-ರ್ಷಿ-ತ-ನಾದ
ಆಕ-ರ್ಷಿ-ತ-ರಾಗಿ
ಆಕ-ರ್ಷಿ-ತ-ರಾ-ಗಿದ್ದ
ಆಕ-ರ್ಷಿ-ತ-ರಾ-ಗಿ-ದ್ದರೂ
ಆಕ-ರ್ಷಿ-ತ-ರಾ-ಗಿ-ದ್ದ-ವರು
ಆಕ-ರ್ಷಿ-ತ-ರಾ-ಗು-ತ್ತಿ-ದ್ದರು
ಆಕ-ರ್ಷಿ-ತ-ರಾದ
ಆಕ-ರ್ಷಿ-ತ-ರಾ-ದರು
ಆಕ-ರ್ಷಿ-ತ-ರಾ-ದ-ವ-ರ-ಲ್ಲೊ-ಬ್ಬ-ರಾದ
ಆಕ-ರ್ಷಿ-ತ-ಳಾ-ಗಿ-ದ್ದರೂ
ಆಕ-ರ್ಷಿ-ತ-ಳಾ-ಗಿ-ದ್ದುದು
ಆಕ-ರ್ಷಿ-ತ-ಳಾದ
ಆಕ-ರ್ಷಿ-ತ-ವಾ-ಗಿತ್ತು
ಆಕ-ರ್ಷಿ-ಸ-ಬ-ಲ್ಲು-ದಾ-ಗಿತ್ತು
ಆಕ-ರ್ಷಿ-ಸಲು
ಆಕ-ರ್ಷಿಸಿ
ಆಕ-ರ್ಷಿ-ಸಿತು
ಆಕ-ರ್ಷಿ-ಸಿ-ದುವು
ಆಕ-ರ್ಷಿ-ಸಿದ್ದು
ಆಕ-ರ್ಷಿ-ಸಿ-ದ್ದೆಂ-ದರೆ
ಆಕ-ರ್ಷಿ-ಸಿ-ಬಿ-ಟ್ಟಿ-ದ್ದರು
ಆಕ-ರ್ಷಿ-ಸು-ವಂ-ಥ-ದಾ-ಗಿತ್ತು
ಆಕ-ರ್ಷಿ-ಸು-ವಲ್ಲಿ
ಆಕ-ಸ್ಮಿ-ಕ-ವಾಗಿ
ಆಕ-ಸ್ಮಿ-ಕ-ವೊಂ-ದ-ರಲ್ಲಿ
ಆಕಾಂಕ್ಷೆ
ಆಕಾಂ-ಕ್ಷೆ-ಇವು
ಆಕಾಂ-ಕ್ಷೆ-ಗಳನ್ನು
ಆಕಾಂ-ಕ್ಷೆ-ಯಿಂದ
ಆಕಾಂ-ಕ್ಷೆ-ಯಿಂ-ದಾ-ಗಲಿ
ಆಕಾರ
ಆಕಾ-ರ-ದಿಂ-ದಲೂ
ಆಕಾ-ರ-ವಿ-ಲ್ಲದ
ಆಕಾಶ
ಆಕಾ-ಶದ
ಆಕಾ-ಶ-ದಲ್ಲಿ
ಆಕಾ-ಶ-ದಿಂ-ದಲೋ
ಆಕಾ-ಶ-ಬುಟ್ಟಿ
ಆಕಾ-ಶ-ಬು-ಟ್ಟಿಯ
ಆಕಾ-ಶ-ಬು-ಟ್ಟಿ-ಯನ್ನು
ಆಕಾ-ಶ-ಬು-ಟ್ಟಿ-ಯಲ್ಲಿ
ಆಕಾ-ಶ-ಬು-ಟ್ಟಿ-ಯಿಂದ
ಆಕಾ-ಶ-ವನ್ನೇ
ಆಕಾ-ಶ-ವಿ-ಹಾ-ರದ
ಆಕಾ-ಶ-ವಿ-ಹಾ-ರ-ದಿಂದ
ಆಕೃತಿ
ಆಕೃ-ತಿಗೆ
ಆಕೃ-ತ್ರಿಮ
ಆಕೆ
ಆಕೆ-ಗಾಗಿ
ಆಕೆ-ಗಾದ
ಆಕೆಗೆ
ಆಕೆಯ
ಆಕೆ-ಯನ್ನು
ಆಕೆಯು
ಆಕೆಯೇ
ಆಕೆ-ಯೇ-ನಾ-ದರೂ
ಆಕ್ರಂ-ದಿ-ಸಿದ
ಆಕ್ರ-ಮಣ
ಆಕ್ರ-ಮ-ಣ-ಕಾ-ರರು
ಆಕ್ರ-ಮ-ಣ-ಕಾರೀ
ಆಕ್ರ-ಮ-ಣ-ದಿಂದ
ಆಕ್ರ-ಮ-ಣ-ದೊಂ-ದಿಗೆ
ಆಕ್ರ-ಮ-ಣ-ವನ್ನು
ಆಕ್ರ-ಮ-ಣ-ವನ್ನೇ
ಆಕ್ರ-ಮಿ-ಸ-ಬೇ-ಕಾ-ಗಿ-ದೆ-ಯೆಂದು
ಆಕ್ರ-ಮಿಸಿ
ಆಕ್ರ-ಮಿ-ಸಿ-ಕೊಂ-ಡಿವೆ
ಆಕ್ರ-ಮಿ-ಸಿದ್ದ
ಆಕ್ಷೇ-ಪ-ಗಳನ್ನೆಲ್ಲ
ಆಕ್ಷೇ-ಪ-ಗಳಿಂದ
ಆಕ್ಷೇ-ಪಣೆ
ಆಕ್ಷೇ-ಪ-ವೇನೂ
ಆಕ್ಷೇ-ಪಿ-ಸ-ಬ-ಹುದು
ಆಕ್ಷೇ-ಪಿ-ಸ-ಬೇಕು
ಆಕ್ಷೇ-ಪಿ-ಸ-ಲಿಲ್ಲ
ಆಕ್ಷೇ-ಪಿ-ಸಿ-ದರು
ಆಕ್ಷೇ-ಪಿ-ಸಿ-ದರೆ
ಆಕ್ಷೇ-ಪಿ-ಸಿ-ದಾಗ
ಆಕ್ಸ್ಫ-ರ್ಡಿನ
ಆಕ್ಸ್ಫ-ರ್ಡ್
ಆಖಾಡ
ಆಗ
ಆಗ-ತಾನೆ
ಆಗ-ದಿ-ದ್ದರೆ
ಆಗದೆ
ಆಗ-ಬ-ಹು-ದಾ-ಗಿದ್ದ
ಆಗ-ಬ-ಹುದು
ಆಗ-ಬೇಕಾ
ಆಗ-ಬೇ-ಕಾ-ಗಿದೆ
ಆಗ-ಬೇ-ಕಾ-ಗಿದ್ದ
ಆಗ-ಬೇ-ಕಾ-ಗಿ-ರು-ವಂ-ತಹ
ಆಗ-ಬೇ-ಕಾ-ಗಿ-ರು-ವುದು
ಆಗ-ಬೇ-ಕಾದ
ಆಗ-ಬೇ-ಕಾ-ದದ್ದು
ಆಗ-ಬೇ-ಕಾ-ದರೂ
ಆಗ-ಬೇ-ಕಾ-ದು-ದ-ಕ್ಕಿಂತ
ಆಗ-ಬೇಕು
ಆಗ-ಮನ
ಆಗ-ಮ-ನದ
ಆಗ-ಮಿ-ಸ-ಬೇ-ಕೆಂದು
ಆಗ-ಮಿ-ಸಿದ
ಆಗ-ಮಿ-ಸಿ-ದಾಗ
ಆಗ-ಮಿ-ಸಿ-ದ್ದರು
ಆಗ-ಮಿ-ಸಿ-ರುವ
ಆಗ-ರ-ದಂತೆ
ಆಗ-ರ-ವಾ-ಗು-ವುದೇ
ಆಗ-ರ-ವೆಂದೂ
ಆಗ-ಲಾ-ರ-ದೆಂದು
ಆಗಲಿ
ಆಗ-ಲಿ-ಇ-ವೆಲ್ಲ
ಆಗ-ಲಿಲ್ಲ
ಆಗ-ಲಿ-ಲ್ಲ-ನೀವು
ಆಗಲು
ಆಗಲೂ
ಆಗಲೇ
ಆಗ-ಸ-ದಲ್ಲಿ
ಆಗ-ಸ-ದೆ-ತ್ತ-ರಕ್ಕೆ
ಆಗ-ಸ-ದೆ-ತ್ತ-ರದ
ಆಗ-ಸ್ಟಿ-ನಲ್ಲಿ
ಆಗ-ಸ್ಟಿ-ನ-ವ-ರೆಗೂ
ಆಗ-ಸ್ಟ್
ಆಗಾಗ
ಆಗಾಧ
ಆಗಿ
ಆಗಿಂ-ದಾಗ್ಗೆ
ಆಗಿತ್ತು
ಆಗಿ-ತ್ತು-ಮಾ-ನ-ವನ
ಆಗಿ-ತ್ತೆಂದು
ಆಗಿದೆ
ಆಗಿ-ದೆ-ಯೆಂ-ದರೆ
ಆಗಿದ್ದ
ಆಗಿ-ದ್ದರು
ಆಗಿ-ದ್ದರೂ
ಆಗಿ-ದ್ದರೆ
ಆಗಿ-ದ್ದಲ್ಲಿ
ಆಗಿ-ದ್ದ-ವರು
ಆಗಿ-ದ್ದಾ-ರೆಂ-ಬುದು
ಆಗಿ-ದ್ದಿ-ರ-ಲೇ-ಬೇಕು
ಆಗಿ-ದ್ದೀಯೆ
ಆಗಿ-ದ್ದೀರಿ
ಆಗಿದ್ದು
ಆಗಿ-ದ್ದು-ದ-ರಿಂದ
ಆಗಿ-ದ್ದುವು
ಆಗಿನ
ಆಗಿ-ನಿಂ-ದಲೂ
ಆಗಿನ್ನೂ
ಆಗಿ-ಬಿ-ಟ್ಟರು
ಆಗಿ-ಬಿ-ಟ್ಟಿ-ದ್ದರು
ಆಗಿ-ಬಿ-ಟ್ಟಿ-ದ್ದುವು
ಆಗಿ-ಬಿ-ಟ್ಟಿವೆ
ಆಗಿ-ಬಿ-ಡು-ತ್ತಿ-ದ್ದರು
ಆಗಿ-ಬಿ-ಡು-ವು-ದೆಂಬ
ಆಗಿಯೂ
ಆಗಿ-ರದೆ
ಆಗಿ-ರ-ಬ-ಹು-ದಾದ
ಆಗಿ-ರ-ಬ-ಹುದು
ಆಗಿ-ರ-ಬೇ-ಕಾ-ಗು-ತ್ತದೆ
ಆಗಿ-ರ-ಬೇಕು
ಆಗಿ-ರ-ಬೇ-ಕು-ಎಂ-ಬು-ದರ
ಆಗಿ-ರಲಿ
ಆಗಿ-ರ-ಲಿಲ್ಲ
ಆಗಿ-ರಲು
ಆಗಿ-ರು-ತ್ತಿ-ದ್ದರು
ಆಗಿ-ರುವ
ಆಗಿ-ರು-ವುದನ್ನು
ಆಗಿ-ರು-ವು-ದಲ್ಲ
ಆಗಿ-ರು-ವು-ದಿಲ್ಲ
ಆಗಿಲ್ಲ
ಆಗಿ-ಲ್ಲ-ದಿ-ರಲಿ
ಆಗಿ-ಲ್ಲವೋ
ಆಗಿವೆ
ಆಗಿ-ಹೋ-ಗಿತ್ತು
ಆಗಿ-ಹೋ-ಯಿತು
ಆಗುತ್ತ
ಆಗು-ತ್ತದೆ
ಆಗು-ತ್ತಿತ್ತು
ಆಗು-ತ್ತಿದೆ
ಆಗು-ತ್ತಿ-ರ-ಲಿಲ್ಲ
ಆಗು-ತ್ತಿಲ್ಲ
ಆಗುವ
ಆಗು-ವಂ-ತಹ
ಆಗು-ವಂ-ತಿ-ರ-ಲಿಲ್ಲ
ಆಗು-ವು-ದ-ರ-ಲ್ಲಿದೆ
ಆಗು-ವು-ದಿಲ್ಲ
ಆಗು-ವುದೂ
ಆಗು-ಹೋ-ಗು-ಗಳ
ಆಗು-ಹೋ-ಗು-ಗಳನ್ನೆಲ್ಲ
ಆಗು-ಹೋ-ಗು-ಗ-ಳಿಗೆ
ಆಗು-ಹೋ-ಗು-ಗ-ಳೆ-ಲ್ಲವೂ
ಆಗೊಮ್ಮೆ
ಆಗೋ
ಆಗ್ಗೆ
ಆಗ್ರಹ
ಆಘಾತ
ಆಘಾ-ತ-ಗೊಂಡು
ಆಘಾ-ತ-ದಿಂದ
ಆಘಾ-ತ-ವನ್ನೇ
ಆಘಾ-ತ-ವಾಗಿ
ಆಘಾ-ತ-ವಾ-ಗಿ-ರ-ಬ-ಹುದು
ಆಘಾ-ತ-ವೇ-ನಾ-ದರೂ
ಆಘಾ-ತ-ವೊಂದು
ಆಚ-ರಣೆ
ಆಚ-ರ-ಣೆ-ಸಂ-ಪ್ರ-ದಾ-ಯ-ಗಳಲ್ಲಿ
ಆಚ-ರ-ಣೆ-ಗಳ
ಆಚ-ರ-ಣೆ-ಗಳನ್ನೂ
ಆಚ-ರ-ಣೆ-ಗಳಲ್ಲಿ
ಆಚ-ರ-ಣೆ-ಗ-ಳಿಗೆ
ಆಚ-ರ-ಣೆ-ಗಳು
ಆಚ-ರ-ಣೆ-ಗ-ಳು-ಇ-ವು-ಗಳ
ಆಚ-ರ-ಣೆ-ಗ-ಳೆ-ರ-ಡ-ರಲ್ಲೂ
ಆಚ-ರ-ಣೆ-ಗ-ಳೆ-ಲ್ಲವೂ
ಆಚ-ರ-ಣೆಗೆ
ಆಚ-ರ-ಣೆ-ಯಲ್ಲಿ
ಆಚ-ರಿ-ಸ-ಬ-ಹುದು
ಆಚ-ರಿ-ಸ-ಬೇ-ಕಾದ
ಆಚ-ರಿ-ಸ-ಬೇ-ಕಾ-ದರೆ
ಆಚ-ರಿ-ಸ-ಲಾ-ಯಿತು
ಆಚ-ರಿ-ಸಲು
ಆಚ-ರಿಸಿ
ಆಚ-ರಿ-ಸಿದ
ಆಚ-ರಿ-ಸಿ-ದರು
ಆಚ-ರಿ-ಸಿದ್ದು
ಆಚ-ರಿ-ಸುವು
ಆಚ-ರಿ-ಸು-ವು-ದ-ರಿಂದ
ಆಚಾ-ರ-ವಿ-ಚಾ-ರ-ಸಂ-ಪ್ರ-ದಾಯ
ಆಚಾ-ರ-ವಿ-ಚಾ-ರ-ಸಂ-ಪ್ರಾ-ಯ-ಗಳ
ಆಚಾ-ರ-ವಿ-ಚಾ-ರ-ಗಳನ್ನು
ಆಚಾ-ರ-ವ್ಯ-ವ-ಹಾ-ರ-ಗಳ
ಆಚಾ-ರ್ಯ-ನೆಂದು
ಆಚಾ-ರ್ಯ-ಪು-ರು-ಷನ
ಆಚೆ
ಆಚೆಗೆ
ಆಚೆಯ
ಆಜಾ-ನು-ಬಾಹು
ಆಜ್ಞಾ-ನ-ದಲ್ಲಿ
ಆಜ್ಞಾ-ನು-ವರ್ತಿ
ಆಜ್ಞಾ-ನು-ವ-ರ್ತಿ-ಗ-ಳಾ-ಗಿ-ದ್ದರು
ಆಜ್ಞಾ-ಪಿ-ಸಿ-ದರು
ಆಜ್ಞಾ-ಪಿ-ಸಿದ್ದು
ಆಜ್ಞೆ
ಆಜ್ಞೆ-ಗಳನ್ನು
ಆಜ್ಞೆ-ಯನ್ನು
ಆಟ
ಆಟ-ಗ-ಳಿಗೆ
ಆಟ-ಗಳು
ಆಟ-ಗಾ-ರ-ನಿಗೆ
ಆಟ-ಗಾ-ರರು
ಆಟದ
ಆಟ-ಪಾ-ಠ-ಗಳಲ್ಲಿ
ಆಟ-ವನ್ನು
ಆಟ-ವಾ-ಗಲಿ
ಆಟ-ವಾ-ಡಲು
ಆಟ-ವಾಡಿ
ಆಟ-ವಾ-ಡು-ತ್ತಿ-ದ್ದರು
ಆಟ-ವಾ-ಡು-ತ್ತಿ-ದ್ದಾನೆ
ಆಟ-ವಾ-ಡು-ವಂತೆ
ಆಟಿ-ಕೆ-ಗ-ಳಾ-ದುವು
ಆಡ
ಆಡಂ-ಬ-ರ-ಗ-ಳೆಲ್ಲ
ಆಡಂ-ಬ-ರದ
ಆಡಂ-ಬ-ರವೂ
ಆಡದೆ
ಆಡಮ್
ಆಡ-ಮ್ನನ್ನು
ಆಡ-ಲಿಲ್ಲ
ಆಡಲು
ಆಡ-ಳಿತ
ಆಡ-ಳಿ-ತಕ್ಕೆ
ಆಡ-ಳಿ-ತ-ಗಾ-ರ-ನೆಂದೂ
ಆಡ-ಳಿ-ತ-ಗಾ-ರ-ರ-ಲ್ಲೊ-ಬ್ಬ-ನೆಂಬ
ಆಡ-ಳಿ-ತ-ಗಾ-ರರು
ಆಡ-ಳಿ-ತ-ಗಾ-ರ-ರೆಂದು
ಆಡ-ಳಿ-ತ-ವನ್ನು
ಆಡಿ
ಆಡಿ-ಕೊ-ಳ್ಳ-ಬ-ಹು-ದ-ಲ್ಲವೆ
ಆಡಿ-ಕೊ-ಳ್ಳು-ತ್ತಾರೆ
ಆಡಿ-ಕೊ-ಳ್ಳುವ
ಆಡಿದ
ಆಡಿ-ದಂತೆ
ಆಡಿ-ದ್ದರು
ಆಡಿಲ್ಲ
ಆಡು-ತ್ತಿದ್ದ
ಆಡುವ
ಆಡು-ವುದನ್ನು
ಆಡೋಣ
ಆತ
ಆತಂಕ
ಆತಂ-ಕ-ಕ-ರವೂ
ಆತಂ-ಕ-ಗಳ
ಆತಂ-ಕ-ಗೊ-ಳ್ಳು-ತ್ತಿ-ದ್ದರು
ಆತಂ-ಕ-ಪ-ಟ್ಟು-ಕೊ-ಳ್ಳ-ಬೇ-ಕಾ-ಗಿಲ್ಲ
ಆತಂ-ಕ-ವಿದ್ದೇ
ಆತಂ-ಕವೂ
ಆತ-ಥೇ-ಯ-ರಾದ
ಆತನ
ಆತ-ನನ್ನು
ಆತ-ನಿಗೂ
ಆತ-ನಿಗೆ
ಆತನು
ಆತ-ನೊ-ಳ-ಗಿ-ರುವ
ಆತಿ
ಆತಿ-ಥಿ-ಗ-ಳಾಗಿ
ಆತಿ-ಥಿ-ಗಳು
ಆತಿ-ಥೇಯ
ಆತಿ-ಥೇ-ಯನ
ಆತಿ-ಥೇ-ಯ-ನಾ-ಗಿದ್ದ
ಆತಿ-ಥೇ-ಯ-ನಾದ
ಆತಿ-ಥೇ-ಯ-ನೊಂ-ದಿಗೆ
ಆತಿ-ಥೇ-ಯರ
ಆತಿ-ಥೇ-ಯ-ರಾ-ಗಿ-ದ್ದ-ವ-ರಿಗೆ
ಆತಿ-ಥೇ-ಯ-ರಾ-ಗಿ-ರುವ
ಆತಿ-ಥೇ-ಯ-ರಾದ
ಆತಿ-ಥೇ-ಯರು
ಆತಿ-ಥೇ-ಯ-ರೊಂ-ದಿಗೆ
ಆತಿ-ಥೇ-ಯಳ
ಆತಿ-ಥೇ-ಯ-ಳಾದ
ಆತಿ-ಥೇ-ಯ-ಳಾ-ದಳು
ಆತಿ-ಥೇ-ಯಳು
ಆತಿಥ್ಯ
ಆತಿ-ಥ್ಯ-ವನ್ನು
ಆತುರ
ಆತು-ರ-ದ-ಲ್ಲಿ-ದ್ದರು
ಆತು-ರ-ಪ-ಟ್ಟಂತೆ
ಆತು-ರಾ-ತು-ರ-ವಾಗಿ
ಆತುಲ
ಆತ್ಮ
ಆತ್ಮ-ಮಾ-ಯೆ-ಜೀ-ವ-ಇ-ವು-ಗಳ
ಆತ್ಮ-ಅ-ನಂ-ತ-ನಾದ
ಆತ್ಮ-ಕ-ಥೆ-ಯಲ್ಲಿ
ಆತ್ಮಕ್ಕೆ
ಆತ್ಮ-ಗಳನ್ನು
ಆತ್ಮ-ಗೌ-ರವ
ಆತ್ಮ-ಗೌ-ರ-ವಕ್ಕೆ
ಆತ್ಮ-ಚ-ರಿ-ತ್ರೆ-ಯಲ್ಲಿ
ಆತ್ಮ-ಜ್ಞಾನ
ಆತ್ಮ-ತ-ತ್ತ್ವದ
ಆತ್ಮ-ತ-ತ್ತ್ವ-ವನ್ನು
ಆತ್ಮದ
ಆತ್ಮ-ದ-ರ್ಶನ
ಆತ್ಮ-ದೊಂ-ದಿಗೆ
ಆತ್ಮನ
ಆತ್ಮ-ನ-ಲ್ಲವೆ
ಆತ್ಮ-ನಿಷ್ಠೆ
ಆತ್ಮನೇ
ಆತ್ಮ-ಬಲ
ಆತ್ಮ-ಬೋಧ
ಆತ್ಮ-ರ-ಕ್ಷಣೆ
ಆತ್ಮ-ವಂ-ಚನೆ
ಆತ್ಮ-ವನ್ನು
ಆತ್ಮ-ವಿ-ಶ್ವಾಸ
ಆತ್ಮ-ವಿ-ಶ್ವಾ-ಸ-ಆತ್ಮ
ಆತ್ಮ-ವಿ-ಶ್ವಾ-ಸ-ಆ-ತ್ಮ-ಶ-ಕ್ತಿ-ಗ-ಳಿ-ರ-ಬೇ-ಕಾ-ಗು-ತ್ತದೆ
ಆತ್ಮ-ವಿ-ಶ್ವಾ-ಸ-ದಿಂದ
ಆತ್ಮ-ವಿ-ಶ್ವಾ-ಸ-ವನ್ನು
ಆತ್ಮ-ವಿ-ಶ್ವಾ-ಸವೂ
ಆತ್ಮವು
ಆತ್ಮವೇ
ಆತ್ಮ-ಶ-ಕ್ತಿ-ಯನ್ನು
ಆತ್ಮ-ಶ-ಕ್ತಿ-ಯನ್ನೂ
ಆತ್ಮ-ಸಂ-ಯಮ
ಆತ್ಮ-ಸಾ-ಕ್ಷಾ-ತ್ಕಾ-ರದ
ಆತ್ಮ-ಸಾ-ಕ್ಷಾ-ತ್ಕಾ-ರ-ದಿಂದ
ಆತ್ಮ-ಸಾಕ್ಷಿ
ಆತ್ಮ-ಸ್ವ-ರೂ-ಪರು
ಆತ್ಮ-ಹ-ತ್ಯೆಗೂ
ಆತ್ಮಾ-ನಂದ
ಆತ್ಮಾ-ನಂ-ದ-ದಲ್ಲಿ
ಆತ್ಮಾ-ಭಿ-ಮಾನ
ಆತ್ಮಾ-ಭಿ-ಮಾ-ನ-ವನ್ನೂ
ಆತ್ಮೀಯ
ಆತ್ಮೀ-ಯ-ಗೌ-ರ-ವಾ-ನ್ವಿತ
ಆತ್ಮೀ-ಯ-ತೆ-ಸೌ-ಹಾ-ರ್ದ-ತೆ-ಗಳು
ಆತ್ಮೀ-ಯ-ರಾ-ಗಿದ್ದ
ಆತ್ಮೀ-ಯ-ರಾ-ಗು-ತ್ತಿ-ದ್ದರು
ಆತ್ಮೀ-ಯ-ರಿಗೆ
ಆತ್ಮೀ-ಯ-ರೆ-ದು-ರಿ-ನಲ್ಲಿ
ಆತ್ಮೀ-ಯ-ವಾದ
ಆತ್ಮೀ-ಯ-ವಾ-ಯಿತು
ಆತ್ಮೋ-ದ್ಧಾರ
ಆದ
ಆದಂ-ತಹ
ಆದ-ಕಾ-ರಣ
ಆದಮ್ಯ
ಆದರ
ಆದ-ರದ
ಆದ-ರ-ದಿಂದ
ಆದ-ರದು
ಆದ-ರ-ಪೂ-ರ್ವಕ
ಆದ-ರ-ಪೂ-ರ್ವ-ಕ-ವಾ-ಗಿ-ರು-ತ್ತಿತ್ತು
ಆದ-ರಲ್ಲಿ
ಆದ-ರ-ವನ್ನೂ
ಆದ-ರಾ-ಭಿ-ಮಾ-ನ-ಗಳು
ಆದ-ರಾ-ಭಿ-ಮಾ-ನ-ವಿತ್ತು
ಆದ-ರಿಂದ
ಆದ-ರಿ-ಸಲು
ಆದ-ರಿಸಿ
ಆದ-ರಿ-ಸಿ-ದುವು
ಆದ-ರಿ-ಸು-ತ್ತಿ-ದ್ದರು
ಆದ-ರಿ-ಸುವ
ಆದರು
ಆದರೂ
ಆದರೆ
ಆದರೇ
ಆದ-ರೇ-ನಂತೆ
ಆದ-ರೇನು
ಆದರ್ಶ
ಆದ-ರ್ಶ-ಮೌ-ಲ್ಯ-ಗ-ಳಿಗೆ
ಆದ-ರ್ಶಕ್ಕೆ
ಆದ-ರ್ಶ-ಗಳ
ಆದ-ರ್ಶ-ಗಳನ್ನು
ಆದ-ರ್ಶ-ಗಳಲ್ಲಿ
ಆದ-ರ್ಶ-ಗ-ಳಾಗ
ಆದ-ರ್ಶ-ಗಳು
ಆದ-ರ್ಶದ
ಆದ-ರ್ಶ-ದಿಂದ
ಆದ-ರ್ಶ-ದೊಂ-ದಿಗೆ
ಆದ-ರ್ಶ-ವನ್ನು
ಆದ-ರ್ಶ-ವ-ನ್ನು-ಉ-ಜ್ವ-ಲ-ವಾಗಿ
ಆದ-ರ್ಶ-ವಾ-ಗಿ-ರು-ತ್ತದೆ
ಆದ-ರ್ಶವು
ಆದಲ್ಲಿ
ಆದಳು
ಆದ-ವ-ರನ್ನು
ಆದ-ವ-ರೆಂ-ದರೆ
ಆದಾ-ಗಲೇ
ಆದಾ-ಗಿ-ತ್ತು-ಅ-ನೇ-ಕರ
ಆದಾಯ
ಆದಾ-ಯದ
ಆದಾ-ಯ-ವಿ-ರಲಿ
ಆದಿ
ಆದಿ-ಅಂ-ತ-ರ-ಹಿತ
ಆದಿ-ತ್ಯ-ವರ್ಣಂ
ಆದು-ದ-ರಿಂದ
ಆದೇಶ
ಆದೇ-ಶ-ಕ್ಕಾಗಿ
ಆದೇ-ಶಕ್ಕೆ
ಆದೇ-ಶ-ಗಳನ್ನು
ಆದೇ-ಶ-ದಂತೆ
ಆದೇ-ಶ-ದಿಂದ
ಆದೇ-ಶ-ವನ್ನೇ
ಆದೇ-ಶ-ವಾ-ಗಿತ್ತು
ಆದೇ-ಶ-ವಿನ್ನೂ
ಆದೇ-ಶ-ವಿ-ಲ್ಲದೆ
ಆದ್ದ
ಆದ್ದ-ರಿಂ
ಆದ್ದ-ರಿಂದ
ಆದ್ದ-ರಿಂ-ದಲೇ
ಆದ್ದ-ರಿಂ-ದಲೋ
ಆದ್ಯ
ಆದ್ಯ-ಕ-ರ್ತವ್ಯ
ಆದ್ಯ-ತೆ-ಗಳನ್ನು
ಆದ್ಯಾ-ಶ-ಕ್ತಿಯ
ಆದ್ಯಾ-ಶ-ಕ್ತಿ-ಯನ್ನು
ಆದ್ರೆ
ಆಧ-ರಿಸಿ
ಆಧ-ರಿ-ಸಿ-ದ್ದರೂ
ಆಧ-ರಿ-ಸಿ-ರು-ವಾಗ
ಆಧ-ರಿ-ಸಿ-ರು-ವುದು
ಆಧಾ-ರ-ಗ್ರಂ-ಥ-ಗ-ಳಾದ್ದ
ಆಧಾ-ರದ
ಆಧಾ-ರ-ದಿಂದ
ಆಧಾ-ರ-ಭೂತ
ಆಧಾ-ರ-ಭೂ-ತ-ವಾಗಿ
ಆಧಾ-ರ-ವಾ-ಗಿ-ದೆಯೋ
ಆಧಾ-ರ-ವಿತ್ತು
ಆಧಾ-ರವೂ
ಆಧಾ-ರ-ವೆಂ-ಬುದು
ಆಧಾ-ರ-ಸ್ತಂ-ಭ-ವಾ-ಗು-ತ್ತಿ-ದ್ದ-ರೆಂಬು
ಆಧಾ-ರ-ಸ್ತಂ-ಭವೇ
ಆಧಾ-ರಿ-ತ-ವಾ-ಗಿದೆ
ಆಧಾ-ರಿ-ತ-ವಾ-ಗಿ-ದ್ದುದು
ಆಧಿ-ಪ-ತ್ಯ-ವನ್ನು
ಆಧಿ-ರಿ-ಸಿ-ದೆಯೇ
ಆಧು-ನಿಕ
ಆಧು-ನಿ-ಕರು
ಆಧು-ನೀ-ಕ-ರ-ಣದ
ಆಧ್ಯಾ
ಆಧ್ಯಾ-ತ್ಮಿಕ
ಆಧ್ಯಾ-ತ್ಮಿ-ಕ-ತ-ವಾದ
ಆಧ್ಯಾ-ತ್ಮಿ-ಕತೆ
ಆಧ್ಯಾ-ತ್ಮಿ-ಕ-ತೆ-ಗಾಗಿ
ಆಧ್ಯಾ-ತ್ಮಿ-ಕ-ತೆಗೂ
ಆಧ್ಯಾ-ತ್ಮಿ-ಕ-ತೆಯ
ಆಧ್ಯಾ-ತ್ಮಿ-ಕ-ತೆ-ಯನ್ನು
ಆಧ್ಯಾ-ತ್ಮಿ-ಕ-ತೆ-ಯಲ್ಲಿ
ಆಧ್ಯಾ-ತ್ಮಿ-ಕ-ತೆ-ಯಿ-ರು-ವುದು
ಆಧ್ಯಾ-ತ್ಮಿ-ಕ-ತೆಯು
ಆಧ್ಯಾ-ತ್ಮಿ-ಕ-ತೆಯೂ
ಆಧ್ಯಾ-ತ್ಮಿ-ಕ-ತೆಯೇ
ಆಧ್ಯಾ-ತ್ಮಿ-ಕ-ವಾಗಿ
ಆಧ್ಯಾ-ತ್ಮಿ-ಕ-ವಾ-ಗಿಯೂ
ಆಧ್ಯಾ-ತ್ಮಿ-ಕ-ಶಕ್ತಿ
ಆನಂದ
ಆನಂ-ದಕ್ಕೆ
ಆನಂ-ದ-ಗೊಂ-ಡಿದ್ದ
ಆನಂ-ದದ
ಆನಂ-ದ-ದಲ್ಲಿ
ಆನಂ-ದ-ದಾ-ಯ-ಕ-ವಾ-ಗಿ-ದ್ದಿತು
ಆನಂ-ದ-ದಾ-ಯ-ಕ-ವಾದ
ಆನಂ-ದ-ದಾ-ಯ-ಕವೂ
ಆನಂ-ದ-ದಿಂದ
ಆನಂ-ದ-ದಿಂ-ದಿ-ದ್ದ-ರೆಂದು
ಆನಂ-ದ-ಭ-ರಿ-ತ-ರಾಗಿ
ಆನಂ-ದ-ಭ-ರಿ-ತ-ರಾ-ಗಿ-ದ್ದೆವು
ಆನಂ-ದ-ಭ-ರಿ-ತ-ರಾ-ದರು
ಆನಂ-ದ-ಮಯ
ಆನಂ-ದ-ವನ್ನು
ಆನಂ-ದ-ವ-ನ್ನುಂಟು
ಆನಂ-ದ-ವಾ-ಗಿದೆ
ಆನಂ-ದ-ವಾ-ಗಿ-ದ್ದುದು
ಆನಂ-ದ-ವಾ-ಗಿ-ರ-ಲಿ-ಕ್ಕಿಲ್ಲ
ಆನಂ-ದ-ವಾಗು
ಆನಂ-ದ-ವಾ-ಯಿತು
ಆನಂ-ದವು
ಆನಂ-ದ-ವುಂ-ಟು-ಮಾ-ಡಿ-ದರು
ಆನಂ-ದವೂ
ಆನಂ-ದ-ವೆಂದು
ಆನಂ-ದ-ವೊಂದು
ಆನಂ-ದಾ-ತಿ-ಶ-ಯ-ದಿಂದ
ಆನಂ-ದಾ-ಯಕ
ಆನಂ-ದಾ-ಶ್ಚರ್ಯ
ಆನಂ-ದಾ-ಶ್ಚ-ರ್ಯ-ಗಳಿಂದ
ಆನಂ-ದಾ-ಶ್ಚ-ರ್ಯ-ಗ-ಳಿಗೆ
ಆನಂ-ದಾ-ಶ್ಚ-ರ್ಯ-ಗೊಂ-ಡರು
ಆನಂ-ದಾಶ್ರು
ಆನಂ-ದಿ-ತ-ನಾ-ದ-ನೆಂದು
ಆನಂ-ದಿ-ತ-ರ-ನ್ನಾ-ಗಿ-ಸು-ತ್ತಿ-ದ್ದರು
ಆನಂ-ದಿ-ತ-ರಾ-ಗಿ-ದ್ದರು
ಆನಂ-ದಿ-ತ-ರಾ-ಗಿ-ದ್ದಾರೆ
ಆನಂ-ದಿ-ತ-ರಾದ
ಆನಂ-ದಿ-ತಾ-ರ-ದರು
ಆನಂ-ದಿ-ಸ-ಬ-ಹು-ದಾ-ಗಿತ್ತು
ಆನಂ-ದಿ-ಸ-ಲಾ-ರಂ-ಭಿ-ಸಿ-ದರು
ಆನಂ-ದಿಸಿ
ಆನಂ-ದಿ-ಸಿ-ದರು
ಆನಂ-ದಿ-ಸು-ತ್ತಿ-ದ್ದರು
ಆನಂ-ದಿ-ಸುವ
ಆನಂ-ದೋ-ತ್ಸಾ-ಹ-ಭ-ರಿ-ತ-ರಾ-ಗಿ-ದ್ದರು
ಆನಂ-ದೋ-ತ್ಸಾ-ಹ-ಭ-ರಿ-ತ-ರಾ-ದರು
ಆನಂ-ದೋ-ದ್ರೇ-ಕ-ಗೊಂ-ಡಿ-ದ್ದರು
ಆನ-ರ-ಬಲ್
ಆನೆ
ಆನ್ನಿ-ಸ್ಕ್ವಾಮ್
ಆಪಾ-ದ-ನೆ-ಗಳನ್ನು
ಆಪಾ-ದ-ನೆ-ಗ-ಳಿ-ಗೆಲ್ಲ
ಆಪಾ-ದ-ನೆಗೆ
ಆಪಾ-ದ-ನೆಯ
ಆಪಾ-ದಿ-ಸ-ಲ್ಪ-ಟ್ಟ-ವ-ರ-ನ್ನು-ಜೀ-ವ-ಸ-ಹಿ-ತ-ವಾಗಿ
ಆಪಾ-ದಿಸಿ
ಆಪಾಯ
ಆಪ್
ಆಪ್ತ
ಆಪ್ತ-ಕಾರ್ಯ
ಆಪ್ತ-ಕಾ-ರ್ಯ-ದ-ರ್ಶಿ-ಯನ್ನೇ
ಆಪ್ತ-ಕಾ-ರ್ಯ-ದ-ರ್ಶಿ-ಯಾದ
ಆಪ್ತರ
ಆಪ್ತ-ರಲ್ಲಿ
ಆಪ್ತ-ರ-ಲ್ಲೊ-ಬ್ಬ-ಳಾದ
ಆಪ್ತ-ರಾ-ಗಿದ್ದ
ಆಪ್ತ-ರಾದ
ಆಪ್ತ-ರಾ-ದುದು
ಆಪ್ತ-ರಿಗೆ
ಆಪ್ತರು
ಆಪ್ತ-ಶಿ-ಷ್ಯರ
ಆಪ್ತ-ಶಿ-ಷ್ಯ-ರಾದ
ಆಪ್ತ-ಶಿ-ಷ್ಯರು
ಆಪ್ತ-ಶಿ-ಷ್ಯೆ-ಯ-ರ-ಲ್ಲೊ-ಬ್ಬ-ಳಾದ
ಆಪ್ತ-ಶಿ-ಷ್ಯೆ-ಯಾದ
ಆಪ್ತ-ಸ್ನೇ-ಹಿ-ತ-ರಾ-ಗಿ-ದ್ದಾರೆ
ಆಪ್ತ-ಸ್ನೇ-ಹಿ-ತ-ರಿಗೂ
ಆಫ್
ಆಬಟ್
ಆಬಾ-ಲ್ಯ-ವೃದ್ಧ
ಆಬು
ಆಬು-ವಿಗೆ
ಆಬು-ವಿನ
ಆಬು-ವಿ-ನಲ್ಲಿ
ಆಭ-ರ-ಣ-ಗಳನ್ನೆಲ್ಲ
ಆಭ-ರ-ಣ-ಗಳು
ಆಭಿ-ಮಾ-ನ-ಪ್ರೀ-ತಿ-ಗೌ-ರ-ವ-ಗಳು
ಆಮಂ-ತ್ರಣ
ಆಮಂ-ತ್ರ-ಣ-ಗಳನ್ನು
ಆಮಂ-ತ್ರ-ಣ-ವನ್ನು
ಆಮಂ-ತ್ರ-ಣ-ವನ್ನೂ
ಆಮಂ-ತ್ರಿಸಿ
ಆಮಂ-ತ್ರಿ-ಸಿ-ದೆವು
ಆಮೂ-ಲಾ-ಗ್ರ-ವಾಗಿ
ಆಮೆ
ಆಮೇಲೆ
ಆಮ್ಯಾಲೆ
ಆಮ್ಸ್ಟ-ರ್ಡ್ಯಾ-ಮಿನ
ಆಮ್ಸ-್್ಯ-ಟರ್
ಆಯ-ಸ್ಕಾಂ-ತೀಯ
ಆಯಾ
ಆಯಾ-ದಿ-ನವೇ
ಆಯಾ-ದೇ-ವ-ತೆ-ಗಳನ್ನು
ಆಯಾಸ
ಆಯಾ-ಸ-ಕರ
ಆಯಾ-ಸ-ಗೊಂ-ಡಿ-ದ್ದರೂ
ಆಯಾ-ಸ-ದಿಂದ
ಆಯಾ-ಸ-ವಾ-ಗಿದೆ
ಆಯಾ-ಸ-ವಾ-ಗಿ-ಬಿ-ಡು-ತ್ತದೆ
ಆಯಾ-ಸ-ವಾ-ಗಿ-ರಲಿ
ಆಯಾ-ಸ-ವಾ-ಗು-ತ್ತಿತ್ತು
ಆಯಾ-ಸ-ವುಂ-ಟು-ಮಾ-ಡು-ತ್ತಿ-ದ್ದುವು
ಆಯಿತು
ಆಯಿ-ತೆಂದೂ
ಆಯು-ಧ-ಗಳನ್ನು
ಆಯು-ರ್ವೇದ
ಆಯುಸ್ಸೂ
ಆಯೇ
ಆಯ್ಕೆ
ಆಯ್ಕೆಯ
ಆಯ್ದ
ಆರಂಭ
ಆರಂ-ಭದ
ಆರಂ-ಭ-ದಿಂ-ದಲೂ
ಆರಂ-ಭ-ದೊಂ-ದಿಗೆ
ಆರಂ-ಭ-ವಾ-ಗಿತ್ತು
ಆರಂ-ಭವು
ಆರಂ-ಭಿಸಿ
ಆರಂ-ಭಿ-ಸಿ-ದಂ-ದಿ-ನಿಂ-ದಲೂ
ಆರಂ-ಭಿ-ಸಿ-ದರು
ಆರಂ-ಭಿ-ಸಿ-ದುದು
ಆರ-ಕ್ಷಕ
ಆರ-ತಿ-ಯನ್ನು
ಆರದು
ಆರ-ನೆಯ
ಆರ-ವ-ತ್ತೆ-ರಡು
ಆರಾ-ಧ-ಕರೇ
ಆರಾ-ಧ-ನೆ-ಗಳಲ್ಲಿ
ಆರಾ-ಧಿ-ಸ-ಲಾ-ರಿರಿ
ಆರಾ-ಧಿಸಿ
ಆರಾ-ಧಿ-ಸಿದ
ಆರಾ-ಧಿ-ಸಿ-ದರು
ಆರಾ-ಧಿಸು
ಆರಾ-ಧಿ-ಸುತ್ತ
ಆರಾ-ಧಿ-ಸು-ತ್ತಿ-ದ್ದಾರೆ
ಆರಾ-ಧಿ-ಸು-ವರೋ
ಆರಾ-ಮ-ದಾ-ಯಕ
ಆರಾ-ಮ-ವಾಗಿ
ಆರಿ-ಸ-ಲಾ-ಯಿತು
ಆರಿ-ಸಿಕೊ
ಆರಿ-ಸಿ-ಕೊಂಡು
ಆರಿ-ಸಿ-ಕೊ-ಳ್ಳು-ತ್ತಿದ್ದ
ಆರಿ-ಸಿ-ಕೊ-ಳ್ಳೋಣ
ಆರಿ-ಸಿದ
ಆರಿ-ಸಿ-ದರೋ
ಆರು
ಆರುಂ-ಧ-ತಿ-ವ-ಸಿ-ಷ್ಠ-ರು-ಗಳ
ಆರು-ನೂ-ರಕ್ಕೂ
ಆರೆಂಟು
ಆರೇಳು
ಆರೋ
ಆರೋಗ್ಯ
ಆರೋ-ಗ್ಯಕ್ಕೆ
ಆರೋ-ಗ್ಯ-ದಿಂ-ದಿ-ದ್ದಾರೆ
ಆರೋ-ಗ್ಯ-ದಿಂ-ದಿ-ರು-ವಂತೆ
ಆರೋ-ಗ್ಯ-ಪೂರ್ಣ
ಆರೋ-ಗ್ಯ-ವನ್ನು
ಆರೋಪ
ಆರೋ-ಪ-ಗಳನ್ನು
ಆರೋ-ಪ-ವನ್ನು
ಆರೋ-ಪಿ-ಸಿ-ದ್ದಾನೆ
ಆರ್
ಆರ್ಗನ್
ಆರ್ಚ್ಬಿ-ಷಪ್
ಆರ್ಚ್ಬಿ-ಷ-ಪ್ನಂ-ತ-ಹ-ವರ
ಆರ್ಟ್
ಆರ್ತ-ನಾದ
ಆರ್ತ-ರನ್ನು
ಆರ್ತ-ರಿ-ಗಾಗಿ
ಆರ್ಥರ್
ಆರ್ಥಿಕ
ಆರ್ಥಿ-ಕ-ವಾಗಿ
ಆರ್ಭ-ಟಿ-ಸಿ-ದರು
ಆರ್ಭ-ಟಿ-ಸಿ-ದರೂ
ಆರ್ಭ-ಟಿ-ಸುತ್ತ
ಆರ್ಯ
ಆರ್ಯ-ಧ-ರ್ಮದ
ಆರ್ಯ-ಸ-ಮಾ-ಜದ
ಆರ್ಯ-ಸ-ಮಾ-ಜ-ವನ್ನೇ
ಆಲಂ-ಬ-ಜಾ-ರಿಗೆ
ಆಲಂ-ಬ-ಜಾ-ರಿನ
ಆಲಂ-ಬ-ಜಾರ್
ಆಲದ
ಆಲ-ದ-ಮ-ರಕ್ಕೆ
ಆಲ-ದ-ಮ-ರದ
ಆಲನ್
ಆಲ-ಯ-ವನ್ನು
ಆಲಸ್ಯ
ಆಲಿ-ಯಾಸ್
ಆಲಿ-ಸಲು
ಆಲಿಸಿ
ಆಲಿ-ಸಿತು
ಆಲಿ-ಸಿದ
ಆಲಿ-ಸಿ-ದರು
ಆಲಿ-ಸಿ-ದ-ವ-ರಿ-ಗೆಲ್ಲ
ಆಲಿ-ಸಿ-ದುವು
ಆಲಿ-ಸುತ್ತ
ಆಲಿ-ಸು-ತ್ತಿದ್ದ
ಆಲಿ-ಸು-ತ್ತಿ-ದ್ದಂತೆ
ಆಲಿ-ಸು-ತ್ತಿ-ದ್ದರು
ಆಲಿ-ಸು-ತ್ತಿ-ದ್ದ-ವ-ರಿಗೆ
ಆಲಿ-ಸು-ತ್ತಿ-ದ್ದಾರೆ
ಆಲಿ-ಸು-ತ್ತಿ-ದ್ದುದು
ಆಲಿ-ಸು-ತ್ತಿ-ದ್ದೆವು
ಆಲಿ-ಸುವ
ಆಲಿ-ಸು-ವಾಗ
ಆಲಿ-ಸು-ವುದನ್ನು
ಆಲೋ-ಚನಾ
ಆಲೋ-ಚ-ನಾ-ಪ-ರ-ನಾಗಿ
ಆಲೋ-ಚ-ನಾ-ಪ-ರ-ರಾಗಿ
ಆಲೋ-ಚ-ನಾ-ಮ-ಗ್ನ-ರಾ-ಗು-ತ್ತಿ-ದ್ದರು
ಆಲೋ-ಚ-ನಾ-ಲ-ಹ-ರಿ-ಗಳಲ್ಲಿ
ಆಲೋ-ಚ-ನಾ-ಲ-ಹ-ರಿಗೆ
ಆಲೋ-ಚ-ನಾ-ವಿ-ಧಾನ
ಆಲೋ-ಚ-ನಾ-ಸ-ರ-ಣಿ-ಯನ್ನು
ಆಲೋ-ಚ-ನಾ-ಸಾ-ಮ-ರ್ಥ್ಯ-ವುಂ-ಟಾ-ದರೆ
ಆಲೋ-ಚನೆ
ಆಲೋ-ಚ-ನೆ-ಕ-ಲ್ಪ-ನೆ-ಗಳಿಂದ
ಆಲೋ-ಚ-ನೆ-ಚಿಂ-ತ-ನೆ-ಗಳನ್ನೂ
ಆಲೋ-ಚ-ನೆ-ಮಾ-ತು-ಕೃ-ತಿ-ಗ-ಳೆ-ಲ್ಲ-ದ-ರಲ್ಲೂ
ಆಲೋ-ಚ-ನೆ-ಗಳ
ಆಲೋ-ಚ-ನೆ-ಗಳನ್ನು
ಆಲೋ-ಚ-ನೆ-ಗಳನ್ನೆಲ್ಲ
ಆಲೋ-ಚ-ನೆ-ಗ-ಳಿಗೆ
ಆಲೋ-ಚ-ನೆ-ಗಳು
ಆಲೋ-ಚ-ನೆ-ಗಳೂ
ಆಲೋ-ಚ-ನೆ-ಗ-ಳೆಲ್ಲ
ಆಲೋ-ಚ-ನೆಯ
ಆಲೋ-ಚ-ನೆ-ಯನ್ನು
ಆಲೋ-ಚ-ನೆ-ಯ-ಲ್ಲಿ-ದ್ದರು
ಆಲೋ-ಚ-ನೆ-ಯಾ-ಗಿತ್ತು
ಆಲೋ-ಚ-ನೆ-ಯಿಂದ
ಆಲೋ-ಚ-ನೆಯೂ
ಆಲೋ-ಚ-ನೆ-ಯೆಂ-ದರೆ
ಆಲೋ-ಚ-ನೆ-ಯೆಂ-ದ-ರೆ-ಭಾ-ರತ
ಆಲೋ-ಚ-ನೆಯೇ
ಆಲೋ-ಚ-ನೆ-ಯೇನೋ
ಆಲೋ-ಚ-ನೆ-ಯೊಂದು
ಆಲೋ-ಚ-ನೆ-ಯೊಂದೇ
ಆಲೋ-ಚಿ-ಸ-ಬಲ್ಲೆ
ಆಲೋ-ಚಿ-ಸ-ಬ-ಹುದು
ಆಲೋ-ಚಿ-ಸ-ಬೇ-ಕಾದ
ಆಲೋ-ಚಿ-ಸ-ಬೇ-ಕಾ-ದರೆ
ಆಲೋ-ಚಿಸಿ
ಆಲೋ-ಚಿ-ಸಿದ
ಆಲೋ-ಚಿ-ಸಿ-ದಂತೆ
ಆಲೋ-ಚಿ-ಸಿ-ದರು
ಆಲೋ-ಚಿ-ಸಿ-ದ-ರು-ಇ-ದೇ-ನಿದು
ಆಲೋ-ಚಿ-ಸಿ-ದು-ದಾ-ಗಿತ್ತು
ಆಲೋ-ಚಿ-ಸಿದೆ
ಆಲೋ-ಚಿ-ಸಿ-ದೆವು
ಆಲೋ-ಚಿ-ಸಿದ್ದ
ಆಲೋ-ಚಿ-ಸಿ-ದ್ದರು
ಆಲೋ-ಚಿ-ಸಿ-ದ್ದ-ರೆಂ-ಬುದೂ
ಆಲೋ-ಚಿ-ಸಿ-ದ್ದ-ವರು
ಆಲೋ-ಚಿ-ಸಿ-ದ್ದಾ-ರೆಂದು
ಆಲೋ-ಚಿ-ಸಿಯೇ
ಆಲೋ-ಚಿ-ಸಿ-ರ-ಬ-ಹುದು
ಆಲೋ-ಚಿ-ಸಿ-ರ-ಲಿಲ್ಲ
ಆಲೋ-ಚಿ-ಸಿ-ರಲೂ
ಆಲೋ-ಚಿ-ಸಿ-ರಲೇ
ಆಲೋ-ಚಿಸು
ಆಲೋ-ಚಿ-ಸುತ್ತ
ಆಲೋ-ಚಿ-ಸುತ್ತಾ
ಆಲೋ-ಚಿ-ಸು-ತ್ತಾರೆ
ಆಲೋ-ಚಿ-ಸು-ತ್ತಿತ್ತು
ಆಲೋ-ಚಿ-ಸು-ತ್ತಿ-ದ್ದಂತೆ
ಆಲೋ-ಚಿ-ಸು-ತ್ತಿ-ದ್ದರು
ಆಲೋ-ಚಿ-ಸು-ತ್ತಿದ್ದಾ
ಆಲೋ-ಚಿ-ಸು-ತ್ತಿ-ದ್ದಾಗ
ಆಲೋ-ಚಿ-ಸು-ವಂ-ತಾ-ಗು-ತ್ತಿತ್ತು
ಆಲೋ-ಚಿ-ಸು-ವಂತೆ
ಆಲೋ-ಚಿ-ಸು-ವ-ಷ್ಟ-ರಲ್ಲಿ
ಆಲೋ-ಚಿ-ಸು-ವಷ್ಟು
ಆಲೋ-ಚಿ-ಸು-ವುದು
ಆಲ್ಪ್ಸ್
ಆಲ್ಬರ್ಟಾ
ಆಲ್ಬ-ರ್ಟಾ-ಳಿಗೆ
ಆಲ್ಮೋ-ರ-ದಲ್ಲಿ
ಆಲ್ಲಿ
ಆಲ್ವಾ-ರೆ-ಸರು
ಆಲ್ವಾ-ರೆ-ಸ-ರೊಂ-ದಿಗೆ
ಆಲ್ವಾ-ರೆಸ್
ಆಳ
ಆಳಕ್ಕೆ
ಆಳಡಿ
ಆಳ-ದಲ್ಲಿ
ಆಳ-ದಿಂದ
ಆಳ-ಬ-ಯ-ಸುವ
ಆಳ-ರ-ಸರ
ಆಳ-ರ-ಸ-ರಾ-ಗಿದ್ದ
ಆಳ-ರ-ಸ-ರಾ-ಗಿ-ದ್ದು-ದ-ರಿಂದ
ಆಳ-ಲೆಂದೇ
ಆಳ-ವನ್ನೂ
ಆಳ-ವಾಗಿ
ಆಳ-ವಾ-ಗಿತ್ತು
ಆಳ-ವಾ-ಗಿ-ತ್ತೆಂ-ಬು-ದಕ್ಕೆ
ಆಳ-ವಾದ
ಆಳ-ವಾ-ದ-ದ್ದೆಂ-ದರೆ
ಆಳ-ವಾ-ದುದು
ಆಳ-ವಾ-ದು-ದೆಂ-ಬು-ದನ್ನು
ಆಳವೋ
ಆಳಿದ
ಆಳು-ತ್ತಿದ್ದ
ಆಳುವ
ಆಳೆ-ತ್ತ-ರದ
ಆಳ್ವಿಕೆ
ಆಳ್ವಿ-ಕೆಯ
ಆಳ್ವಿ-ಕೆ-ಯಿಂದ
ಆವ-ರಿ-ಸಲು
ಆವ-ರಿಸಿ
ಆವ-ರಿ-ಸಿ-ಕೊಂ-ಡಂತೆ
ಆವ-ರಿ-ಸಿತು
ಆವ-ರಿ-ಸಿ-ದರೆ
ಆವ-ರಿ-ಸಿ-ದೆ-ಏ-ಕೆಂದು
ಆವ-ರಿ-ಸಿ-ಬಿ-ಟ್ಟಿದೆ
ಆವ-ರಿ-ಸಿ-ಬಿ-ಡು-ತ್ತಿತ್ತು
ಆವ-ರಿಸು
ಆವಶ್ಯ
ಆವ-ಶ್ಯಕ
ಆವ-ಶ್ಯ-ಕತೆ
ಆವ-ಶ್ಯ-ಕ-ತೆ-ಗಳ
ಆವ-ಶ್ಯ-ಕ-ತೆ-ಗಳನ್ನು
ಆವ-ಶ್ಯ-ಕ-ತೆ-ಗಳನ್ನೂ
ಆವ-ಶ್ಯ-ಕ-ತೆ-ಗಳನ್ನೆಲ್ಲ
ಆವ-ಶ್ಯ-ಕ-ತೆ-ಗ-ಳಿಗೆ
ಆವ-ಶ್ಯ-ಕ-ತೆ-ಗಳು
ಆವ-ಶ್ಯ-ಕ-ತೆ-ಯನ್ನು
ಆವ-ಶ್ಯ-ಕ-ತೆ-ಯ-ಲ್ಲಿ-ದ್ದರೆ
ಆವ-ಶ್ಯ-ಕ-ತೆ-ಯ-ಲ್ಲಿವೆ
ಆವ-ಶ್ಯ-ಕ-ತೆ-ಯಿತ್ತು
ಆವ-ಶ್ಯ-ಕ-ತೆ-ಯಿತ್ತೆ
ಆವ-ಶ್ಯ-ಕ-ತೆ-ಯಿತ್ತೇ
ಆವ-ಶ್ಯ-ಕ-ತೆ-ಯಿದೆ
ಆವ-ಶ್ಯ-ಕ-ತೆ-ಯಿ-ರು-ವು-ದರ
ಆವ-ಶ್ಯ-ಕ-ತೆಯೂ
ಆವ-ಶ್ಯ-ಕ-ತೆ-ಯೆಂ-ದರೆ
ಆವ-ಶ್ಯ-ಕ-ತೆ-ಯೇ-ನಿದೆ
ಆವ-ಶ್ಯ-ಕ-ತೆ-ಯೇ-ನಿ-ದ್ದಿ-ರ-ಬ-ಹುದು
ಆವ-ಶ್ಯ-ಕ-ತೆ-ಯೇನೂ
ಆವ-ಶ್ಯ-ಕ-ವೆಂದು
ಆವಾಗ
ಆವಿ-ಚಲ
ಆವಿ-ರ್ಭ-ವಿಸಿ
ಆವಿ-ರ್ಭ-ವಿ-ಸಿ-ರ-ಬೇಕು
ಆವಿ-ರ್ಭಾ-ವಕ್ಕೆ
ಆವಿ-ರ್ಭಾ-ವ-ಗೊಂ-ಡಿ-ದ್ದಾರೆ
ಆವಿ-ರ್ಭಾ-ವ-ಗೊಂ-ಡಿ-ರು-ವು-ದ-ರಿಂ-ದಲೇ
ಆವಿ-ರ್ಭಾ-ವ-ವೆಂದು
ಆವಿ-ಷ್ಕ-ರಿ-ಸ-ಬೇ-ಕಾ-ಗಿದೆ
ಆವಿ-ಷ್ಕ-ರಿ-ಸಿದ್ದ
ಆವಿ-ಷ್ಕಾ-ರ-ಗಳೂ
ಆವೃ-ತ-ನಾಗಿ
ಆವೃ-ತ-ರಾ-ಗಿದ್ದ
ಆವೃ-ತ-ರಾ-ಗಿ-ದ್ದರೂ
ಆವೃ-ತ-ವಾ-ಗಿತ್ತು
ಆವೃ-ತ-ವಾ-ಗಿದೆ
ಆವೃ-ತ-ವಾದ
ಆವೃ-ತ್ತಿಯ
ಆವೇಗ
ಆವೇ-ಶ-ಭ-ರಿ-ತ-ರಾಗಿ
ಆಶ-ಯ-ವನ್ನು
ಆಶಾ-ದಾ-ಯಕ
ಆಶಾ-ಪುರಿ
ಆಶಾ-ಪು-ರಿಗೆ
ಆಶಿಸಿ
ಆಶಿ-ಸಿ-ದರು
ಆಶಿ-ಸಿ-ದ್ದರು
ಆಶಿ-ಸು-ತ್ತೇನೆ
ಆಶಿ-ಸು-ತ್ತೇವೆ
ಆಶೀ-ರ್ವ-ದಿ-ಸಲಿ
ಆಶೀ-ರ್ವ-ದಿಸಿ
ಆಶೀ-ರ್ವ-ದಿ-ಸು-ವಂತೆ
ಆಶೀ-ರ್ವಾದ
ಆಶೀ-ರ್ವಾ-ದ-ಎ-ಲ್ಲವೂ
ಆಶೀ-ರ್ವಾ-ದ-ಗಳನ್ನು
ಆಶೀ-ರ್ವಾ-ದ-ಗಳು
ಆಶೀ-ರ್ವಾ-ದದ
ಆಶೀ-ರ್ವಾ-ದ-ಪೂ-ರ್ವಕ
ಆಶೀ-ರ್ವಾ-ದ-ಪೂ-ರ್ವ-ಕ-ವಾಗಿ
ಆಶೀ-ರ್ವಾ-ದ-ಫ-ಲ-ದಿಂದ
ಆಶೀ-ರ್ವಾ-ದ-ವನ್ನು
ಆಶೀ-ರ್ವಾ-ದ-ವಿದೆ
ಆಶು-ಭಾ-ಷ-ಣ-ಗಳನ್ನೂ
ಆಶೋ-ತ್ತ-ರ-ಗಳನ್ನು
ಆಶೋ-ತ್ತ-ರ-ಗಳು
ಆಶ್ಚರ್ಯ
ಆಶ್ಚ-ರ್ಯ-ಆ-ನಂದ
ಆಶ್ಚ-ರ್ಯ-ಸಂ-ತೋ-ಷ-ಗಳಿಂದ
ಆಶ್ಚ-ರ್ಯ-ಕರ
ಆಶ್ಚ-ರ್ಯಕ್ಕೆ
ಆಶ್ಚ-ರ್ಯ-ಗ-ಳುಂ-ಟಾ-ದುವು
ಆಶ್ಚ-ರ್ಯ-ಗೊಂಡ
ಆಶ್ಚ-ರ್ಯ-ಗೊಂ-ಡರು
ಆಶ್ಚ-ರ್ಯ-ಗೊಂಡು
ಆಶ್ಚ-ರ್ಯ-ಚ-ಕಿ-ತ-ರಾ-ದರು
ಆಶ್ಚ-ರ್ಯದ
ಆಶ್ಚ-ರ್ಯ-ದಿಂದ
ಆಶ್ಚ-ರ್ಯ-ದೊಂ-ದಿಗೆ
ಆಶ್ಚ-ರ್ಯ-ಪ-ಡ-ಬೇ-ಕಾ-ಗಿಲ್ಲ
ಆಶ್ಚ-ರ್ಯ-ಭ-ರಿ-ತ-ನಾ-ಗಿ-ದ್ದೇನೆ
ಆಶ್ಚ-ರ್ಯ-ವನ್ನು
ಆಶ್ಚ-ರ್ಯ-ವಾ-ಗು-ತ್ತದೆ
ಆಶ್ಚ-ರ್ಯ-ವಾ-ಯಿತು
ಆಶ್ಚ-ರ್ಯ-ವಿ-ರ-ಲಿಲ್ಲ
ಆಶ್ಚ-ರ್ಯ-ವಿಲ್ಲ
ಆಶ್ಚ-ರ್ಯವೂ
ಆಶ್ಚ-ರ್ಯವೇ
ಆಶ್ಚ-ರ್ಯ-ವೇ-ನಿದೆ
ಆಶ್ಚ-ರ್ಯ-ವೇನೂ
ಆಶ್ಚ-ರ್ಯಾ-ಘಾ-ತ-ಗೊಂಡು
ಆಶ್ಚ-ರ್ಯಾ-ಘಾ-ತ-ದಿಂದ
ಆಶ್ರಮ
ಆಶ್ರ-ಮಕ್ಕೆ
ಆಶ್ರ-ಮ-ವೊಂ-ದನ್ನು
ಆಶ್ರಯ
ಆಶ್ರ-ಯ-ದಲ್ಲಿ
ಆಶ್ರ-ಯ-ದ-ಲ್ಲಿ-ದ್ದೆವು
ಆಶ್ರ-ಯ-ವ-ನ್ನಿತ್ತ
ಆಶ್ರ-ಯ-ವ-ನ್ನಿತ್ತು
ಆಶ್ರ-ಯ-ವ-ನ್ನೀ-ಯು-ವಂ-ಥ-ದಾ-ಗಿ-ರ-ಬೇಕು
ಆಶ್ರ-ಯ-ವಿತ್ತ
ಆಶ್ರ-ಯಿಸಿ
ಆಶ್ರ-ಯಿ-ಸಿ-ಕೊಂಡು
ಆಶ್ರು-ಧಾರೆ
ಆಶ್ರು-ಧಾ-ರೆಯ
ಆಶ್ರು-ನ-ಯ-ನ-ರಾಗಿ
ಆಶ್ವಾ-ಸನೆ
ಆಶ್ವಾ-ಸ-ನೆ-ಯಾ-ಗಲಿ
ಆಷಾ-ಢ-ಭೂತಿ
ಆಷಾ-ಢ-ಭೂ-ತಿ-ಗಳು
ಆಷಾ-ಢ-ಭೂ-ತಿ-ತನ
ಆಷಾ-ಢ-ಭೂ-ತಿ-ತ-ನ-ವನ್ನೂ
ಆಷಾ-ಢ-ಭೂ-ತಿ-ಯಾ-ಗಲು
ಆಸಕ್ತ
ಆಸ-ಕ್ತ-ನಾ-ಗ-ಬಲ್ಲ
ಆಸ-ಕ್ತ-ರಾಗಿ
ಆಸ-ಕ್ತ-ರಾ-ಗಿದ್ದ
ಆಸ-ಕ್ತ-ರಾ-ಗಿ-ದ್ದರು
ಆಸ-ಕ್ತ-ರಾದ
ಆಸ-ಕ್ತ-ಳಾ-ದ-ವಳು
ಆಸಕ್ತಿ
ಆಸ-ಕ್ತಿ-ಕರ
ಆಸ-ಕ್ತಿ-ಕ-ರ-ವಾದ
ಆಸ-ಕ್ತಿ-ಗಳನ್ನೂ
ಆಸ-ಕ್ತಿ-ಯನ್ನು
ಆಸ-ಕ್ತಿ-ಯನ್ನೂ
ಆಸ-ಕ್ತಿ-ಯಿಂದ
ಆಸ-ಕ್ತಿ-ಯಿ-ದೆಯೇ
ಆಸ-ಕ್ತಿ-ಯಿ-ದ್ದ-ವಳು
ಆಸ-ಕ್ತಿ-ಯಿ-ದ್ದುದ
ಆಸ-ಕ್ತಿ-ಯಿರು
ಆಸ-ಕ್ತಿ-ಯಿ-ರು-ವು-ದಾ-ಗಿಯೂ
ಆಸ-ಕ್ತಿಯು
ಆಸ-ಕ್ತಿ-ಯುಂ-ಟಾ-ಗಿ-ರು-ವು-ದ-ರಲ್ಲಿ
ಆಸನ
ಆಸ-ನ-ಗಳನ್ನೆಲ್ಲ
ಆಸ-ನ-ದಿಂ-ದೆದ್ದು
ಆಸ-ರೆ-ಅ-ನೂ-ಕ-ಲತೆ
ಆಸಾಮಿ
ಆಸಾ-ಮಿ-ಯೆಂ-ದರೆ
ಆಸೀ-ನ-ನಾ-ಗಿದ್ದ
ಆಸೀ-ನ-ರಾ-ಗು-ವಂತೆ
ಆಸೀ-ನ-ರಾ-ದರು
ಆಸು-ಪಾ-ಸಿನ
ಆಸೆ
ಆಸೆ-ಗ-ಣ್ಣಿ-ನಿಂದ
ಆಸೆ-ಗಳ
ಆಸೆ-ಗಳನ್ನು
ಆಸೆ-ಗಳನ್ನೂ
ಆಸೆ-ಗಳು
ಆಸೆ-ಯನ್ನೂ
ಆಸೆ-ಯಿಂದ
ಆಸೆ-ಯಿ-ಟ್ಟು-ಕೊಳ್ಳ
ಆಸೆ-ಯಿತ್ತು
ಆಸೆಯೂ
ಆಸ್ಟಿನ್
ಆಸ್ಟ್ರಿಯಾ
ಆಸ್ತಿ
ಆಸ್ತಿ-ಕತೆ
ಆಸ್ತಿ-ಪಾ-ಸ್ತಿ-ಗ-ಳೆಂಬ
ಆಸ್ತಿಯೂ
ಆಸ್ಥಾನ
ಆಸ್ಥಾ-ನ-ದಲ್ಲಿ
ಆಸ್ಥಾ-ನಿ-ಕರ
ಆಸ್ಥಾ-ನಿ-ಕ-ರಿಗೂ
ಆಸ್ಥಾ-ನಿ-ಕ-ರೆ-ದುರು
ಆಸ್ಥೆ-ಯಿಂದ
ಆಸ್ಪ-ತ್ರೆಗೆ
ಆಸ್ಪ-ತ್ರೆಯ
ಆಸ್ಪ-ತ್ರೆ-ಯನ್ನು
ಆಸ್ಪ-ತ್ರೆ-ಯೊಂ-ದರ
ಆಸ್ಪದ
ಆಸ್ವಾ-ದಿ-ಸ-ಬೇ-ಕೆಂದು
ಆಸ್ವಾ-ದಿಸಿ
ಆಸ್ವಾ-ದಿಸು
ಆಸ್ವಾ-ದಿ-ಸುತ್ತ
ಆಸ್ವಾ-ದಿ-ಸು-ತ್ತಿತ್ತು
ಆಸ್ವಾ-ದಿ-ಸು-ತ್ತಿ-ದ್ದರು
ಆಸ್ವಾ-ದಿ-ಸುವ
ಆಹಾ
ಆಹಾರ
ಆಹಾ-ರ-ಆ-ಸರೆ
ಆಹಾ-ರ-ವ-ಸತಿ
ಆಹಾ-ರಕ್ಕೆ
ಆಹಾ-ರದ
ಆಹಾ-ರ-ದಿಂದ
ಆಹಾ-ರ-ಪ-ದ್ಧ-ತಿಯ
ಆಹಾ-ರ-ವ-ನ್ನಾ-ದರೂ
ಆಹಾ-ರ-ವನ್ನು
ಆಹಾ-ರ-ವನ್ನೇ
ಆಹಾ-ರ-ವಾದ
ಆಹಾ-ರ-ವಿ-ಲ್ಲದೆ
ಆಹಾ-ರಾ-ದಿ-ಗ-ಳಿಗೆ
ಆಹಾ-ರಾ-ನ್ವೇ-ಷ-ಣೆಗೆ
ಆಹುತಿ
ಆಹ್
ಆಹ್ಲಾ-ದ-ಕರ
ಆಹ್ಲಾ-ದ-ಕ-ರ-ವಾ-ಗಿತ್ತು
ಆಹ್ಲಾ-ದ-ಕ-ರ-ವಾ-ಗಿ-ರು-ವಂತೆ
ಆಹ್ಲಾ-ದ-ಕ-ರ-ವಾದ
ಆಹ್ಲಾ-ದ-ಕ-ರವೂ
ಆಹ್ವಾನ
ಆಹ್ವಾ-ನಕ್ಕೆ
ಆಹ್ವಾ-ನ-ಗಳ
ಆಹ್ವಾ-ನ-ಗಳನ್ನೂ
ಆಹ್ವಾ-ನ-ಗ-ಳಿಗೆ
ಆಹ್ವಾ-ನ-ಗಳು
ಆಹ್ವಾ-ನ-ಗ-ಳೇನೂ
ಆಹ್ವಾ-ನ-ಗ-ಳೊಂ-ದಿಗೆ
ಆಹ್ವಾ-ನದ
ಆಹ್ವಾ-ನ-ದಂತೆ
ಆಹ್ವಾ-ನ-ವನ್ನು
ಆಹ್ವಾ-ನಿತ
ಆಹ್ವಾ-ನಿ-ತ-ರಲ್ಲಿ
ಆಹ್ವಾ-ನಿ-ತ-ರಾ-ಗಿ-ದ್ದ-ವ-ರಲ್ಲಿ
ಆಹ್ವಾ-ನಿ-ತ-ರಾ-ಗಿ-ದ್ದ-ವರು
ಆಹ್ವಾ-ನಿ-ತ-ರಾದ
ಆಹ್ವಾ-ನಿಸ
ಆಹ್ವಾ-ನಿ-ಸ-ಲಾ-ಗಿತ್ತು
ಆಹ್ವಾ-ನಿ-ಸ-ಲಾ-ಯಿತು
ಆಹ್ವಾ-ನಿಸಿ
ಆಹ್ವಾ-ನಿ-ಸಿತು
ಆಹ್ವಾ-ನಿ-ಸಿತ್ತು
ಆಹ್ವಾ-ನಿ-ಸಿದ
ಆಹ್ವಾ-ನಿ-ಸಿ-ದ-ನ-ಲ್ಲದೆ
ಆಹ್ವಾ-ನಿ-ಸಿ-ದರು
ಆಹ್ವಾ-ನಿ-ಸಿ-ದಳು
ಆಹ್ವಾ-ನಿ-ಸಿ-ದಾಗ
ಆಹ್ವಾ-ನಿ-ಸಿ-ದೆವು
ಆಹ್ವಾ-ನಿ-ಸಿದ್ದ
ಆಹ್ವಾ-ನಿ-ಸಿ-ದ್ದ-ರ-ಲ್ಲದೆ
ಆಹ್ವಾ-ನಿ-ಸಿ-ದ್ದರು
ಆಹ್ವಾ-ನಿ-ಸಿ-ದ್ದರೆ
ಆಹ್ವಾ-ನಿ-ಸಿದ್ದೆ
ಆಹ್ವಾ-ನಿ-ಸಿ-ದ್ದೇನೆ
ಆಹ್ವಾ-ನಿ-ಸಿ-ರ-ಲಿಲ್ಲ
ಆಹ್ವಾ-ನಿ-ಸಿರು
ಆಹ್ವಾ-ನಿ-ಸು-ತ್ತಿದ್ದ
ಆಹ್ವಾ-ನಿ-ಸು-ತ್ತಿ-ದ್ದರು
ಆಹ್ವಾ-ನಿ-ಸು-ತ್ತಿ-ದ್ದ-ವ-ರಿಗೆ
ಆಹ್ವಾ-ನಿ-ಸುವ
ಆಹ್ವಾ-ನಿ-ಸು-ವಂತೆ
ಆಹ್ವಾ-ನಿ-ಸು-ವುದು
ಆಹ್ವಾ-ನಿ-ಸೋಣ
ಆ್ಯ-ನಿ-ಸ್ಕ್ಟಾ-ಮ್ನ-ಲ್ಲಿ-ರುವ
ಆ್ಯ-ನಿ-ಸ್ಕ್ವಾಮ್
ಆ್ಯ-ನಿ-ಸ್ಕ್ವಾಮ್ಗೆ
ಆ್ಯ-ನಿ-ಸ್ಕ್ವಾಮ್ನ
ಆ್ಯ-ನಿ-ಸ್ಕ್ವಾ-ಮ್ನಲ್ಲೂ
ಆ್ಯ-ನಿ-ಸ್ಕ್ವಾ-ಮ್ನಲ್ಲೇ
ಆ್ಯ-ನಿ-ಸ್ಕ್ವಾ-ಮ್ನಿಂದ
ಆ್ಯನ್ನಿ
ಆ್ಯ-ನ್ನಿ-ಬೆ-ಸೆಂ-ಟರು
ಆ್ಯ-ನ್ನಿ-ಸ್ಕ್ವಾಮ್
ಆ್ಯ-ನ್ನಿ-ಸ್ಕ್ವಾ-ಮ್ನಿಂದ
ಇ
ಇಂಗ-ರ್ಸಾಲ್
ಇಂಗ-ರ್ಸಾ-ಲ್ನನ್ನು
ಇಂಗಿ-ತ-ವನ್ನು
ಇಂಗಿ-ಹೋ-ಯಿತು
ಇಂಗ್ಲಿ-ಷನ್ನು
ಇಂಗ್ಲಿ-ಷರ
ಇಂಗ್ಲಿ-ಷ-ರಲ್ಲಿ
ಇಂಗ್ಲಿ-ಷ-ರಿಂದ
ಇಂಗ್ಲಿ-ಷ-ರಿಗೆ
ಇಂಗ್ಲಿ-ಷರು
ಇಂಗ್ಲಿ-ಷ-ರೊಂ-ದಿ-ಗಿನ
ಇಂಗ್ಲಿಷಿ
ಇಂಗ್ಲಿ-ಷಿಗೆ
ಇಂಗ್ಲಿ-ಷಿ-ನಲ್ಲಿ
ಇಂಗ್ಲಿಷು
ಇಂಗ್ಲಿಷ್
ಇಂಗ್ಲಿಷ್ನ್ನು
ಇಂಗ್ಲೀ-ಷಿ-ನಲ್ಲಿ
ಇಂಗ್ಲೀಷ್
ಇಂಗ್ಲೆಂ-ಡನ್ನು
ಇಂಗ್ಲೆಂಡಿ
ಇಂಗ್ಲೆಂ-ಡಿಗೆ
ಇಂಗ್ಲೆಂ-ಡಿ-ಗೆ-ಬ-ಳಿಕ
ಇಂಗ್ಲೆಂ-ಡಿನ
ಇಂಗ್ಲೆಂ-ಡಿ-ನಲ್ಲಿ
ಇಂಗ್ಲೆಂ-ಡಿ-ನ-ಲ್ಲಿಯೂ
ಇಂಗ್ಲೆಂ-ಡಿ-ನಲ್ಲೂ
ಇಂಗ್ಲೆಂ-ಡಿ-ನ-ವ-ನಾದ
ಇಂಗ್ಲೆಂ-ಡಿ-ನ-ವ-ರಷ್ಟೇ
ಇಂಗ್ಲೆಂ-ಡಿ-ನಿಂದ
ಇಂಗ್ಲೆಂಡು
ಇಂಗ್ಲೆಂಡ್
ಇಂಗ್ಲೆಂ-ಡ್-ಅ-ಮೆ-ರಿ-ಕ-ಗಳನ್ನು
ಇಂಗ್ಲೆಂ-ಡ್-ಅ-ಮೆ-ರಿ-ಕ-ಗಳಲ್ಲಿ
ಇಂಗ್ಲೆಂ-ಡ್-ಆ-ಸ್ಟ್ರೇ-ಲಿ-ಯ-ಗಳ
ಇಂಗ್ಲೆಂ-ಡ್-ಭಾ-ರ-ತ-ಗಳ
ಇಂಚಿನ
ಇಂಜಿ-ನಿ-ಯ-ರಾದ
ಇಂಟರ್
ಇಂಟೀ-ರಿ-ಯರ್
ಇಂಡಿಯ
ಇಂಡಿ-ಯನ್
ಇಂಡಿ-ಯನ್ನ
ಇಂಡಿಯಾ
ಇಂತಹ
ಇಂತ-ಹ-ದನ್ನು
ಇಂತ-ಹ-ವರ
ಇಂತ-ಹ-ವ-ರನ್ನು
ಇಂತಿಂ-ತಹ
ಇಂತಿದೆ
ಇಂತಿಷ್ಟು
ಇಂಥ
ಇಂಥ-ದ-ನ್ನೆಲ್ಲ
ಇಂಥದು
ಇಂಥದೇ
ಇಂಥ-ದೇ-ನನ್ನೂ
ಇಂಥ-ದೇನೂ
ಇಂಥ-ದೊಂದು
ಇಂಥ-ಲ್ಲಿಗೆ
ಇಂಥವ
ಇಂಥ-ವನು
ಇಂಥ-ವರ
ಇಂಥ-ವ-ರನ್ನು
ಇಂಥ-ವ-ರ-ನ್ನೆಲ್ಲ
ಇಂಥ-ವ-ರಲ್ಲಿ
ಇಂಥ-ವ-ರ-ಲ್ಲೊಬ್ಬ
ಇಂಥ-ವ-ರಿ-ಗ-ಲ್ಲದೆ
ಇಂಥ-ವ-ರಿಗೆ
ಇಂಥ-ವರು
ಇಂಥವು
ಇಂಥಾ
ಇಂದ
ಇಂದಲ್ಲ
ಇಂದಾವ
ಇಂದಿ-ಗಿಂತ
ಇಂದಿಗೂ
ಇಂದಿಗೆ
ಇಂದಿನ
ಇಂದಿ-ನ-ವ-ರೆಗೂ
ಇಂದಿ-ನಿಂ-ದಲೇ
ಇಂದಿಲ್ಲಿ
ಇಂದು
ಇಂದು-ಮತಿ
ಇಂದೇ
ಇಂದೇನೂ
ಇಂದೋ
ಇಂದೋರ್
ಇಂದ್ರ-ಜಾಲ
ಇಂದ್ರ-ಜಾ-ಲದ
ಇಂದ್ರ-ಜಾ-ಲವೂ
ಇಂದ್ರಿಯ
ಇಂದ್ರಿ-ಯ-ಗಳ
ಇಂದ್ರಿ-ಯ-ಗಳನ್ನು
ಇಂದ್ರಿ-ಯ-ಗ್ರಾಹ್ಯ
ಇಂದ್ರಿ-ಯ-ನಿ-ಗ್ರ-ಹದ
ಇಂದ್ರಿ-ಯ-ಸು-ಖ-ವೆಂಬ
ಇಂದ್ರಿ-ಯ-ಸು-ಖವೇ
ಇಂದ್ರಿ-ಯಾ-ತೀ-ತ-ವಾದ
ಇಂದ್ರಿ-ಯಾ-ನು-ಭವ
ಇಕ
ಇಕ್ಕೆ-ಲ-ಗ-ಳಲ್ಲೂ
ಇಗರ್ಜಿ
ಇಗ-ರ್ಜಿ-ಯಲ್ಲಿ
ಇಗೋ
ಇಗ್ನೇ-ಷಿ-ಯಸ್
ಇಚ್ಛಾ-ನು-ಸಾರ
ಇಚ್ಛಾ-ಶ-ಕ್ತಿ-ಯು-ಳ್ಳ-ವ-ರಿಗೆ
ಇಚ್ಛಿ-ಸ-ದಿ-ದ್ದರೆ
ಇಚ್ಛಿ-ಸದೆ
ಇಚ್ಛಿ-ಸ-ಲಾ-ರರು
ಇಚ್ಛಿಸಿ
ಇಚ್ಛಿ-ಸಿದ
ಇಚ್ಛಿ-ಸಿ-ದರು
ಇಚ್ಛಿ-ಸಿ-ದರೆ
ಇಚ್ಛಿ-ಸಿ-ದು-ದನ್ನು
ಇಚ್ಛಿ-ಸಿದೆ
ಇಚ್ಛಿ-ಸಿ-ದ್ದ-ರಿಂದ
ಇಚ್ಛಿ-ಸಿ-ದ್ದೇನೆ
ಇಚ್ಛಿ-ಸು-ತ್ತಾರೆ
ಇಚ್ಛಿ-ಸು-ತ್ತಿ-ದ್ದರು
ಇಚ್ಛಿ-ಸು-ತ್ತೇನೆ
ಇಚ್ಛಿ-ಸುವ
ಇಚ್ಛಿ-ಸು-ವಂ-ತೆಯೇ
ಇಚ್ಛಿ-ಸು-ವು-ದಾ-ದರೆ
ಇಚ್ಛಿ-ಸು-ವುದೂ
ಇಚ್ಛೆ
ಇಚ್ಛೆಗೆ
ಇಚ್ಛೆ-ಪ-ಟ್ಟರೆ
ಇಚ್ಛೆಯ
ಇಚ್ಛೆ-ಯಂ-ತಾ-ಗಲಿ
ಇಚ್ಛೆ-ಯನ್ನು
ಇಚ್ಛೆ-ಯಾ-ಗಿತ್ತು
ಇಚ್ಛೆ-ಯಾ-ದರೆ
ಇಚ್ಛೆ-ಯಾ-ಯಿತು
ಇಚ್ಛೆ-ಯಿಂದ
ಇಚ್ಛೆ-ಯಿತ್ತೋ
ಇಚ್ಛೆ-ಯಿ-ದ್ದಂ-ತಾ-ಗಲಿ
ಇಚ್ಛೆ-ಯಿ-ದ್ದಂ-ತಾ-ಗು-ತ್ತದೆ
ಇಚ್ಛೆ-ಯಿ-ದ್ದರೆ
ಇಚ್ಛೆ-ಯುಂ-ಟಾಗಿ
ಇಚ್ಛೆ-ಯುಂ-ಟಾ-ಯಿತು
ಇಚ್ಛೆ-ಯು-ಳ್ಳ-ವರೆ-ಲ್ಲರೂ
ಇಚ್ಛೆ-ಯೆಂದಾ
ಇಚ್ಛೆ-ಯೆಂದೇ
ಇಚ್ಛೆಯೇ
ಇಚ್ಛೆ-ಯೇ-ನೆಂದು
ಇಚ್ಛೆಯೋ
ಇಟ-ಲಿಯ
ಇಟ್ಟಿ-ದ್ದಂ-ತಹ
ಇಟ್ಟಿ-ದ್ದಳು
ಇಟ್ಟಿ-ದ್ದಾಳೆ
ಇಟ್ಟಿ-ದ್ದೇನೆ
ಇಟ್ಟಿ-ರ-ಲಿಲ್ಲ
ಇಟ್ಟಿ-ರಲು
ಇಟ್ಟಿ-ರು-ತ್ತಾರೆ
ಇಟ್ಟು
ಇಟ್ಟುಕೊ
ಇಟ್ಟು-ಕೊಂ-ಡರೂ
ಇಟ್ಟು-ಕೊಂ-ಡಿ-ದ್ದರು
ಇಟ್ಟು-ಕೊಂಡು
ಇಟ್ಟು-ಕೊ-ಳ್ಳಲು
ಇಟ್ಟು-ಕೊಳ್ಳಿ
ಇಟ್ಟು-ಕೊ-ಳ್ಳು-ವಷ್ಟು
ಇಟ್ಟು-ಕೊ-ಳ್ಳು-ವುದು
ಇಟ್ಟು-ಕೊ-ಳ್ಳೋಣ
ಇಟ್ಟುಕೋ
ಇಟ್ಟು-ಹೋಗಿ
ಇಡ
ಇಡ-ಲಾಗಿದೆ
ಇಡ-ಲಾಗು
ಇಡಿಯ
ಇಡೀ
ಇಡು-ತ್ತಿ-ದ್ದರು
ಇಡುವ
ಇಣಕಿ
ಇಣಿಕಿ
ಇಣುಕು
ಇಣು-ಕು-ನೋಟ
ಇಣು-ಕು-ನೋ-ಟ-ವೊಂ-ದನ್ನು
ಇತರ
ಇತ-ರರ
ಇತ-ರ-ರತ್ತ
ಇತ-ರ-ರನ್ನು
ಇತ-ರ-ರನ್ನೂ
ಇತ-ರ-ರಲ್ಲಿ
ಇತ-ರ-ರಾ-ದರೂ
ಇತ-ರ-ರಿಂದ
ಇತ-ರ-ರಿ-ಗಾ-ಗಲಿ
ಇತ-ರ-ರಿ-ಗಾಗಿ
ಇತ-ರ-ರಿ-ಗಿಂತ
ಇತ-ರ-ರಿಗೂ
ಇತ-ರ-ರಿಗೆ
ಇತ-ರ-ರಿ-ಗೊಂದು
ಇತ-ರರು
ಇತ-ರರೂ
ಇತ-ರ-ರೆಲ್ಲ
ಇತ-ರ-ರೆ-ಲ್ಲ-ರಿ-ಗಿಂತ
ಇತ-ರ-ರೆ-ಲ್ಲ-ರಿಗೂ
ಇತ-ರ-ರೊಂ-ದಿಗೆ
ಇತ-ರೆ-ಡೆ-ಗಳಲ್ಲಿ
ಇತ-ರೆ-ಡೆ-ಗ-ಳಲ್ಲೂ
ಇತ-ರೆಲ್ಲ
ಇತಿ
ಇತಿ-ಮಿ-ತಿ-ಗಳನ್ನು
ಇತಿ-ಮಿ-ತಿ-ಯನ್ನೂ
ಇತಿ-ಹಾಸ
ಇತಿ-ಹಾ-ಸ-ಅ-ರ್ಥ-ಶಾ-ಸ್ತ್ರ-ಗ-ಳಲ್ಲೂ
ಇತಿ-ಹಾ-ಸ-ಕಾ-ರನ
ಇತಿ-ಹಾ-ಸ-ಕಾ-ರ-ನಾಗಿ
ಇತಿ-ಹಾ-ಸ-ಜ್ಞ-ನೊಬ್ಬ
ಇತಿ-ಹಾ-ಸದ
ಇತಿ-ಹಾ-ಸ-ದಲ್ಲಿ
ಇತಿ-ಹಾ-ಸ-ದಲ್ಲೇ
ಇತಿ-ಹಾ-ಸ-ದ-ಲ್ಲೇ-ಅ-ತ್ಯಂತ
ಇತಿ-ಹಾ-ಸ-ಪ್ರಜ್ಞೆ
ಇತಿ-ಹಾ-ಸ-ಪ್ರ-ಜ್ಞೆ-ಯೆಂ-ಥದು
ಇತಿ-ಹಾ-ಸ-ವನ್ನು
ಇತಿ-ಹಾ-ಸ-ವಿ-ರುವ
ಇತಿ-ಹಾ-ಸವು
ಇತಿ-ಹಾ-ಸ-ವೆಲ್ಲ
ಇತಿ-ಹಾ-ಸವೇ
ಇತಿ-ಹಾ-ಸ-ಹೀಗೆ
ಇತ್ತ
ಇತ್ತೀ-ಚಿನ
ಇತ್ತೀ-ಚಿ-ನದು
ಇತ್ತೀ-ಚಿ-ನ-ವರು
ಇತ್ತೀ-ಚೆ-ಗಷ್ಟೇ
ಇತ್ತೀ-ಚೆಗೆ
ಇತ್ತು
ಇತ್ತು-ಇಂ-ತಹ
ಇತ್ತು-ತಮ್ಮ
ಇತ್ತೆಂದು
ಇತ್ತೋ
ಇತ್ಯ-ರ್ಥ-ಗೊ-ಳಿಸಿ
ಇತ್ಯಾದಿ
ಇದ-ಕ್ಕ-ವರು
ಇದ-ಕ್ಕಾಗಿ
ಇದ-ಕ್ಕಾ-ಗಿಯೇ
ಇದ-ಕ್ಕಿಂತ
ಇದ-ಕ್ಕಿಂ-ತರೂ
ಇದ-ಕ್ಕಿ-ದ್ದಂತೆ
ಇದ-ಕ್ಕು-ತ್ತ-ರ-ವಾಗಿ
ಇದಕ್ಕೂ
ಇದಕ್ಕೆ
ಇದ-ಕ್ಕೆಲ್ಲ
ಇದ-ಕ್ಕೇನು
ಇದ-ಕ್ಕೊ-ಪ್ಪದೆ
ಇದ-ಕ್ಕೊಪ್ಪಿ
ಇದನು
ಇದ-ನ್ನ-ರಿ-ಯದೆ
ಇದ-ನ್ನಲ್ಲ
ಇದ-ನ್ನ-ವರು
ಇದ-ನ್ನಾ-ದರೂ
ಇದ-ನ್ನೀಗ
ಇದನ್ನು
ಇದ-ನ್ನು-ಎಂ-ದೆಂ-ದಿಗೂ
ಇದ-ನ್ನು-ಳಿ-ದಂತೆ
ಇದ-ನ್ನೆಲ್ಲ
ಇದನ್ನೇ
ಇದರ
ಇದ-ರತ್ತ
ಇದ-ರಲ್ಲಿ
ಇದ-ರ-ಲ್ಲೆಲ್ಲ
ಇದ-ರಷ್ಟು
ಇದ-ರಾ-ಚೆಯ
ಇದ-ರಿಂದ
ಇದ-ರಿಂ-ದಲೂ
ಇದ-ರಿಂ-ದಲೇ
ಇದ-ರಿಂ-ದಾಗಿ
ಇದ-ರಿಂ-ದಾತ
ಇದ-ರಿಂ-ದಾದ
ಇದ-ರೊಂ-ದಿಗೆ
ಇದಲ್ಲ
ಇದ-ಲ್ಲದೆ
ಇದಾಗಿ
ಇದಾದ
ಇದಾ-ದ-ನಂ-ತರ
ಇದಾ-ವು-ದನ್ನೂ
ಇದಾ-ವುದೂ
ಇದಿ-ರಾಗಿ
ಇದಿ-ರಿ-ಸಿ-ದರು
ಇದಿ-ರಿ-ಸು-ವುದು
ಇದಿರು
ಇದಿ-ರು-ನೋ-ಡಿ-ಅದು
ಇದಿ-ರು-ನೋ-ಡು-ತ್ತಿದೆ
ಇದಿ-ರು-ನೋ-ಡು-ತ್ತಿ-ದ್ದರು
ಇದಿ-ರು-ಹಾ-ಕಿ-ಕೊ-ಳ್ಳು-ವು-ದಿಲ್ಲ
ಇದೀಗ
ಇದು
ಇದು-ವರೆ-ಗಿನ
ಇದು-ವ-ರೆಗೆ
ಇದೂ
ಇದೆ
ಇದೆಂ-ತಹ
ಇದೆಂಥ
ಇದೆ-ಜಾತಿ
ಇದೆ-ಯಾ-ದರೂ
ಇದೆಯೆ
ಇದೆಯೋ
ಇದೆಲ್ಲ
ಇದೆ-ಲ್ಲ-ದರ
ಇದೆ-ಲ್ಲ-ವನ್ನೂ
ಇದೆ-ಲ್ಲವೂ
ಇದೆಷ್ಟು
ಇದೇ
ಇದೇ-ಏ-ನೆಂ-ದರೆ
ಇದೇಕೆ
ಇದೇ-ನಿದು
ಇದೇನು
ಇದೇನೋ
ಇದೇಯೋ
ಇದೊಂದು
ಇದೊಂದೂ
ಇದೊಂದೇ
ಇದ್ದ
ಇದ್ದಂ-ತಹ
ಇದ್ದಂ-ತಿರ
ಇದ್ದಂತೆ
ಇದ್ದಕ್ಕಿ
ಇದ್ದ-ಕ್ಕಿ-ದ್ದಂತೆ
ಇದ್ದ-ದ್ದ-ರಿಂದ
ಇದ್ದದ್ದು
ಇದ್ದರು
ಇದ್ದರೂ
ಇದ್ದರೆ
ಇದ್ದ-ರೆಷ್ಟು
ಇದ್ದ-ವ-ರಿ-ಗೆಲ್ಲ
ಇದ್ದಾನೆ
ಇದ್ದಾರೆ
ಇದ್ದಾ-ರೆಂ-ಬು-ದನ್ನು
ಇದ್ದಾ-ರೆಯೆ
ಇದ್ದಾ-ರೆಯೇ
ಇದ್ದಿ-ತಾ-ದರೂ
ಇದ್ದಿತು
ಇದ್ದಿ-ತೆಂ-ಬುದು
ಇದ್ದಿ-ತ್ತಾ-ದರೂ
ಇದ್ದಿ-ದ್ದರೆ
ಇದ್ದಿರ
ಇದ್ದಿ-ರಲು
ಇದ್ದಿ-ಲಿ-ನಿಂದ
ಇದ್ದೀರೋ
ಇದ್ದು
ಇದ್ದು-ಕೊಂಡು
ಇದ್ದು-ದನ್ನು
ಇದ್ದುದು
ಇದ್ದು-ಬಿ-ಟ್ಟರೆ
ಇದ್ದು-ಬಿ-ಟ್ಟಿ-ದ್ದರು
ಇದ್ದುವು
ಇದ್ದು-ವೆಂ-ಬುದು
ಇದ್ದೆ
ಇದ್ದೇ
ಇದ್ದೇ-ಇತ್ತು
ಇದ್ದೇನೆ
ಇದ್ದೇವೆ
ಇದ್ದೇವೋ
ಇನೋಕ್
ಇನ್ನ-ವರ
ಇನ್ನಷ್ಟು
ಇನ್ನಾ
ಇನ್ನಾ-ರಿಗೂ
ಇನ್ನಾ-ರಿಗೆ
ಇನ್ನಾರು
ಇನ್ನಾರೂ
ಇನ್ನಾವ
ಇನ್ನಾ-ವು-ದಿ-ರ-ಬ-ಲ್ಲುದು
ಇನ್ನಾ-ವುದೇ
ಇನ್ನಿ-ತರ
ಇನ್ನಿ-ತ-ರರ
ಇನ್ನಿ-ತ-ರ-ರನ್ನೂ
ಇನ್ನಿ-ತ-ರರೂ
ಇನ್ನಿ-ಬ್ಬ-ರನ್ನು
ಇನ್ನಿ-ಬ್ಬ-ರೆಂ-ದರೆ
ಇನ್ನಿಲ್ಲ
ಇನ್ನಿ-ಲ್ಲದ
ಇನ್ನಿ-ಲ್ಲ-ದಂತೆ
ಇನ್ನೀಗ
ಇನ್ನು
ಇನ್ನು-ಮುಂದೆ
ಇನ್ನು-ಳಿದ
ಇನ್ನು-ಳಿ-ದ-ದ್ದ-ರಲ್ಲಿ
ಇನ್ನು-ಳಿ-ದ-ವ-ರ-ನ್ನೆಲ್ಲ
ಇನ್ನು-ಳಿ-ದ-ವರು
ಇನ್ನು-ಳಿ-ದ-ವರೆಲ್ಲ
ಇನ್ನು-ಳಿ-ದ-ವು-ಗಳ
ಇನ್ನು-ಳಿ-ದು-ದನ್ನು
ಇನ್ನು-ಳಿ-ದು-ದ-ನ್ನೆಲ್ಲ
ಇನ್ನು-ಳಿ-ದು-ದೆಲ್ಲ
ಇನ್ನೂ
ಇನ್ನೂರು
ಇನ್ನೆಂಥ
ಇನ್ನೆಂ-ಥವೋ
ಇನ್ನೆ-ರಡು
ಇನ್ನೆ-ರಡೇ
ಇನ್ನೆ-ಲ್ಲಾ-ದರೂ
ಇನ್ನೆ-ಲ್ಲಿಗೆ
ಇನ್ನೆ-ಷ್ಟಿ-ದ್ದಿ-ರ-ಬ-ಹುದು
ಇನ್ನೆಷ್ಟು
ಇನ್ನೆಷ್ಟೋ
ಇನ್ನೇ-ನಾ-ದರೂ
ಇನ್ನೇ-ನಿ-ದೆ-ಯೆಂದು
ಇನ್ನೇನು
ಇನ್ನೇನೂ
ಇನ್ನೊಂದ
ಇನ್ನೊಂ-ದಿ-ದೆಯೆ
ಇನ್ನೊಂದು
ಇನ್ನೊಂ-ದೆಡೆ
ಇನ್ನೊಬ್ಬ
ಇನ್ನೊ-ಬ್ಬ-ನನ್ನು
ಇನ್ನೊ-ಬ್ಬರ
ಇನ್ನೊ-ಬ್ಬ-ರನ್ನು
ಇನ್ನೊ-ಬ್ಬರು
ಇನ್ನೊ-ಬ್ಬ-ರೊಂ-ದಿಗೆ
ಇನ್ನೊ-ಬ್ಬ-ಳೆಂ-ದರೆ
ಇನ್ನೊಮ್ಮೆ
ಇನ್ಯಾ-ರಾ-ದರೂ
ಇನ್ಯಾರೋ
ಇನ್ಯಾವ
ಇನ್ಸ್ಟಿ-ಟ್ಯೂಟ್
ಇನ್ಸ್ಟಿ-ಟ್ಯೂ-ಟ್ನಲ್ಲಿ
ಇನ್ಸ್ಪೆ-ಕ್ಟ-ರನ
ಇನ್ಸ್ಪೆ-ಕ್ಟ-ರ-ನಿಗೆ
ಇನ್ಸ್ಪೆ-ಕ್ಟರ್
ಇಪ್ಪ-ತೈದು
ಇಪ್ಪ-ತ್ತ-ನಾಲ್ಕು
ಇಪ್ಪ-ತ್ತ-ನೆಯ
ಇಪ್ಪತ್ತು
ಇಪ್ಪ-ತ್ತೈದು
ಇಪ್ಪ-ತ್ತೊಂದು
ಇಪ್ಪ-ತ್ತೊಂದೂ
ಇಬ್ಬಂ-ದಿ-ಯಲ್ಲಿ
ಇಬ್ಬನಿ
ಇಬ್ಬರ
ಇಬ್ಬ-ರನ್ನೂ
ಇಬ್ಬ-ರಲ್ಲಿ
ಇಬ್ಬ-ರಿಗೂ
ಇಬ್ಬ-ರಿಗೆ
ಇಬ್ಬರು
ಇಬ್ಬರೂ
ಇಬ್ಬರೇ
ಇಬ್ಬಾಯ
ಇಮ್ಮಡಿ
ಇಮ್ಮ-ಡಿ-ಸಿತು
ಇಯರ್
ಇರ
ಇರ-ದಿ-ದ್ದಂ-ಥವು
ಇರ-ದಿ-ರ-ಲಿಲ್ಲ
ಇರದು
ಇರ-ಬ-ಹುದು
ಇರ-ಬ-ಹುದೆ
ಇರ-ಬ-ಹು-ದೆಂದು
ಇರ-ಬ-ಹುದೇ
ಇರ-ಬಾ-ರದೋ
ಇರ-ಬೇ-ಕಾ-ಗಿತ್ತು
ಇರ-ಬೇ-ಕಾ-ಗಿದೆ
ಇರ-ಬೇ-ಕಾ-ಗು-ತ್ತದೆ
ಇರ-ಬೇ-ಕಾ-ದದ್ದು
ಇರ-ಬೇ-ಕಾ-ದರೆ
ಇರ-ಬೇ-ಕಾ-ಯಿತು
ಇರ-ಬೇಕು
ಇರ-ಬೇ-ಕೆಂದು
ಇರ-ಬೇ-ಕೆಂದೇ
ಇರ-ಬೇ-ಕೆಂ-ಬಂತೆ
ಇರ-ಬೇಕೋ
ಇರ-ಲಾ-ರದು
ಇರ-ಲಾ-ರರು
ಇರ-ಲಾ-ರ-ರೆಂದು
ಇರಲಿ
ಇರ-ಲಿಆ
ಇರ-ಲಿಲ್ಲ
ಇರ-ಲಿ-ಲ್ಲ-ವಾ-ದ್ದ-ರಿಂದ
ಇರ-ಲಿ-ಲ್ಲ-ವೆಂ-ದಲ್ಲ
ಇರ-ಲಿ-ಲ್ಲ-ವೇಕೆ
ಇರಲು
ಇರಲೇ
ಇರ-ಲೇ-ಬಾ-ರದು
ಇರ-ವನ್ನು
ಇರ-ವುದೇ
ಇರಿ
ಇರಿ-ಯುವ
ಇರಿಸಿ
ಇರಿ-ಸಿ-ಕೊಂ-ಡಿ-ದ್ದೀರಿ
ಇರಿ-ಸಿ-ಕೊ-ಳ್ಳಲು
ಇರಿ-ಸಿದ್ದ
ಇರು
ಇರು-ತ್ತದೆ
ಇರು-ತ್ತ-ದೆ-ಯೆಂದು
ಇರು-ತ್ತ-ದೆಯೋ
ಇರು-ತ್ತವೆ
ಇರು-ತ್ತಾರೆ
ಇರು-ತ್ತಾ-ರೆ-ನೀವು
ಇರುತ್ತಿ
ಇರು-ತ್ತಿತ್ತು
ಇರು-ತ್ತಿತ್ತೋ
ಇರು-ತ್ತಿ-ದ್ದರು
ಇರು-ತ್ತಿ-ದ್ದ-ರೆಂ-ದಲ್ಲ
ಇರು-ತ್ತಿ-ದ್ದವು
ಇರು-ತ್ತಿ-ರ-ಲಿಲ್ಲ
ಇರು-ತ್ತೀ-ರಲ್ಲ
ಇರು-ತ್ತೀರಿ
ಇರು-ತ್ತೇನೆ
ಇರು-ತ್ತೇವೆ
ಇರುಳೂ
ಇರುವ
ಇರು-ವಂ-ತ-ಹ-ದಲ್ಲ
ಇರು-ವಂ-ತಾ-ಗ-ಬೇ-ಕೆಂ-ಬುದು
ಇರು-ವಂತೆ
ಇರು-ವಂ-ತೆಯೇ
ಇರು-ವಂ-ಥದು
ಇರು-ವ-ನೇನೋ
ಇರು-ವವ
ಇರು-ವ-ವರು
ಇರು-ವಷ್ಟು
ಇರು-ವಾಗ
ಇರು-ವಿ-ಕೆ-ಯಿಂ-ದಾಗಿ
ಇರು-ವು-ದಕ್ಕೆ
ಇರು-ವು-ದ-ನ್ನೆಲ್ಲ
ಇರು-ವು-ದಿಲ್ಲ
ಇರು-ವುದು
ಇರು-ವು-ದೆಂ-ದರೆ
ಇರು-ವು-ದೆಲ್ಲ
ಇಲಾ-ಖೆಯ
ಇಲಾ-ಖೆ-ಯ-ವ-ನಾ-ದರೂ
ಇಲಿ-ಮ-ರಿ-ಗ-ಳಂತೆ
ಇಲಿ-ಯಾನ್
ಇಲಿ-ಯೊಂ-ದಿಗೆ
ಇಲ್ಲ
ಇಲ್ಲದ
ಇಲ್ಲ-ದಂ-ತಹ
ಇಲ್ಲ-ದಂತಾ
ಇಲ್ಲ-ದಂ-ತಾ-ಗಿ-ಬಿ-ಟ್ಟಿತು
ಇಲ್ಲ-ದಂ-ತಾ-ಗು-ತ್ತಿತ್ತು
ಇಲ್ಲ-ದಂ-ತಾ-ಯಿತು
ಇಲ್ಲ-ದಂತೆ
ಇಲ್ಲ-ದಿ-ದ್ದಂ-ತಹ
ಇಲ್ಲ-ದಿ-ದ್ದರೆ
ಇಲ್ಲ-ದಿ-ದ್ದಲ್ಲಿ
ಇಲ್ಲ-ದಿ-ದ್ದು-ದ-ರಿಂದ
ಇಲ್ಲ-ದಿ-ರಲಿ
ಇಲ್ಲ-ದಿ-ರುವ
ಇಲ್ಲ-ದಿ-ರು-ವಂತೆ
ಇಲ್ಲ-ದಿ-ರು-ವುದನ್ನು
ಇಲ್ಲ-ದಿ-ರು-ವು-ದ-ರಿಂ-ದಲೇ
ಇಲ್ಲದೆ
ಇಲ್ಲ-ದೆ-ಅಂ-ತಹ
ಇಲ್ಲ-ದ್ದ-ನ್ನೆಲ್ಲ
ಇಲ್ಲ-ದ್ದ-ರಿಂದ
ಇಲ್ಲಪ್ಪ
ಇಲ್ಲ-ವಲ್ಲ
ಇಲ್ಲ-ವಾ-ಗಿ-ಬಿ-ಡು-ತ್ತದೆ
ಇಲ್ಲ-ವಾ-ಗಿ-ಸು-ವುದು
ಇಲ್ಲವೆ
ಇಲ್ಲ-ವೆಂದು
ಇಲ್ಲ-ವೆಂಬ
ಇಲ್ಲ-ವೆಂ-ಬಂತೆ
ಇಲ್ಲ-ವೆಂ-ಬುದು
ಇಲ್ಲವೇ
ಇಲ್ಲ-ವೇನೋ
ಇಲ್ಲವೊ
ಇಲ್ಲವೋ
ಇಲ್ಲ-ಸ-ಲ್ಲದ
ಇಲ್ಲಾ-ದರೂ
ಇಲ್ಲಿ
ಇಲ್ಲಿಂದ
ಇಲ್ಲಿಗೂ
ಇಲ್ಲಿಗೆ
ಇಲ್ಲಿದೆ
ಇಲ್ಲಿದ್ದ
ಇಲ್ಲಿ-ದ್ದಾನೆ
ಇಲ್ಲಿ-ದ್ದಾರೆ
ಇಲ್ಲಿ-ದ್ದಿ-ದ್ದರೆ
ಇಲ್ಲಿ-ದ್ದೇ-ವೆ-ವಿ-ವೇ-ಕಾ-ನಂ-ದ-ರಿ-ರುವ
ಇಲ್ಲಿನ
ಇಲ್ಲಿ-ನ-ವರ
ಇಲ್ಲಿ-ನ-ವರು
ಇಲ್ಲಿ-ಯ-ವರು
ಇಲ್ಲಿ-ಯ-ವರೆ-ಗಂತೂ
ಇಲ್ಲಿ-ಯ-ವ-ರೆಗೂ
ಇಲ್ಲಿ-ಯ-ವ-ರೆಗೆ
ಇಲ್ಲಿಯೂ
ಇಲ್ಲಿಯೇ
ಇಲ್ಲಿ-ರ-ಬ-ಹು-ದಲ್ಲ
ಇಲ್ಲಿ-ರ-ಬ-ಹುದು
ಇಲ್ಲಿ-ರಲು
ಇಲ್ಲಿ-ರು-ತ್ತೇನೆ
ಇಲ್ಲಿ-ರು-ವ-ವ-ರಲ್ಲಿ
ಇಲ್ಲಿ-ರು-ವ-ವ-ರಿಗೆ
ಇಲ್ಲಿಲ್ಲ
ಇಲ್ಲಿವೆ
ಇಲ್ಲೆ-ಲ್ಲಾ-ದರೂ
ಇಲ್ಲೇ
ಇಲ್ಲೇ-ಈ-ಗಲೇ
ಇಲ್ಲೊಂದು
ಇಳಿದ
ಇಳಿ-ದರು
ಇಳಿ-ದಾಗ
ಇಳಿ-ದಾ-ಗಿದೆ
ಇಳಿದು
ಇಳಿ-ದು-ಕೊಂ-ಡರು
ಇಳಿ-ದು-ಕೊಂಡಿ
ಇಳಿ-ದು-ಕೊಂ-ಡಿದ್ದ
ಇಳಿ-ದು-ಕೊಂ-ಡಿ-ದ್ದರು
ಇಳಿ-ದು-ಕೊಂ-ಡಿ-ದ್ದ-ರೆಂದು
ಇಳಿ-ದು-ಕೊಂ-ಡಿ-ದ್ದರೋ
ಇಳಿ-ದು-ಕೊ-ಳ್ಳಲು
ಇಳಿ-ದು-ಕೊ-ಳ್ಳುವ
ಇಳಿ-ದು-ಕೊ-ಳ್ಳು-ವಂ-ತಹ
ಇಳಿ-ದು-ದ-ರಲ್ಲಿ
ಇಳಿ-ದು-ಬ-ರ-ಬೇ-ಕಾ-ಗು-ತ್ತಿತ್ತು
ಇಳಿ-ದು-ಬ-ರು-ವುದೇ
ಇಳಿ-ಯಯ್ಯಾ
ಇಳಿ-ಯಿತು
ಇಳಿ-ಯಿ-ತೆಂ-ದರೆ
ಇಳೆ-ಯೆ-ಡೆಗೆ
ಇವ
ಇವತ್ತು
ಇವನ
ಇವ-ನಂ-ತಹ
ಇವ-ನದು
ಇವ-ನನ್ನು
ಇವ-ನಿಂದ
ಇವ-ನಿಗೆ
ಇವನು
ಇವನೂ
ಇವನೇ
ಇವ-ನೊಂ-ದಿಗೆ
ಇವ-ನೊಬ್ಬ
ಇವನ್ನು
ಇವ-ನ್ನೆಲ್ಲ
ಇವ-ನ್ಯಾರೋ
ಇವರ
ಇವ-ರಂ-ಥ-ವರು
ಇವ-ರತ್ತ
ಇವ-ರ-ತ್ತಲೇ
ಇವ-ರದೇ
ಇವ-ರನ್ನು
ಇವ-ರ-ಲ್ಲ-ಡ-ಗಿ-ರುವ
ಇವ-ರ-ಲ್ಲದೆ
ಇವ-ರಲ್ಲಿ
ಇವ-ರಲ್ಲೂ
ಇವ-ರ-ಲ್ಲೊಬ್ಬ
ಇವ-ರ-ಲ್ಲೊ-ಬ್ಬಳು
ಇವ-ರಾ-ಗಲೇ
ಇವ-ರಾ-ಗಿ-ರ-ಬ-ಹುದೇ
ಇವ-ರಿಂದ
ಇವ-ರಿ-ಗಾ-ದರೆ
ಇವ-ರಿ-ಗಿದ್ದ
ಇವ-ರಿಗೂ
ಇವ-ರಿಗೆ
ಇವ-ರಿ-ಬ್ಬರ
ಇವ-ರಿ-ಬ್ಬ-ರನ್ನು
ಇವ-ರಿ-ಬ್ಬ-ರನ್ನೂ
ಇವ-ರಿ-ಬ್ಬ-ರಿಗೂ
ಇವ-ರಿ-ಬ್ಬರು
ಇವ-ರಿ-ಬ್ಬರೂ
ಇವ-ರಿ-ಬ್ಬ-ರೊಂ-ದಿಗೆ
ಇವ-ರೀಗ
ಇವರು
ಇವರೂ
ಇವರೆಲ್ಲ
ಇವರೆ-ಲ್ಲರ
ಇವರೆ-ಲ್ಲ-ರೊಂ-ದಿಗೆ
ಇವರೇ
ಇವ-ರೇನು
ಇವ-ರೇನೂ
ಇವ-ರೊಂ-ದಿಗೆ
ಇವ-ರೊಬ್ಬ
ಇವ-ರ್ಯಾರು
ಇವ-ಲ್ಲದೆ
ಇವಳ
ಇವಳು
ಇವಳೇ
ಇವ-ಳೊಬ್ಬ
ಇವು
ಇವು-ಗಳ
ಇವು-ಗಳನ್ನು
ಇವು-ಗಳನ್ನೂ
ಇವು-ಗಳನ್ನೆಲ್ಲ
ಇವು-ಗ-ಳ-ಲ್ಲದೆ
ಇವು-ಗಳಲ್ಲಿ
ಇವು-ಗ-ಳ-ಲ್ಲೊಂ-ದರ
ಇವು-ಗ-ಳ-ಲ್ಲೊಂ-ದೆಂ-ದರೆ
ಇವು-ಗಳಿಂದ
ಇವು-ಗ-ಳಿಂ-ದಾ-ಚೆಗೆ
ಇವು-ಗ-ಳಿಂ-ದೆಲ್ಲ
ಇವು-ಗ-ಳಿ-ಗಾಗಿ
ಇವು-ಗ-ಳಿಗೆ
ಇವು-ಗ-ಳಿ-ಲ್ಲದೆ
ಇವು-ಗಳು
ಇವು-ಗ-ಳೆಲ್ಲ
ಇವು-ಗಳೇ
ಇವು-ಗ-ಳೊಂ-ದಿಗೆ
ಇವು-ಮೊ-ದ-ಲ-ನೆ-ಯ-ದಾಗಿ
ಇವೂ
ಇವೆ
ಇವೆ-ರ-ಡನ್ನೂ
ಇವೆ-ರಡೂ
ಇವೆಲ್ಲ
ಇವೆ-ಲ್ಲ-ಕ್ಕಿಂತ
ಇವೆ-ಲ್ಲ-ದರ
ಇವೆ-ಲ್ಲ-ದ-ರಿಂ-ದಾಗಿ
ಇವೆ-ಲ್ಲ-ವನ್ನೂ
ಇವೆ-ಲ್ಲ-ವು-ಗ-ಳಿ-ಗಿಂತ
ಇವೆ-ಲ್ಲವೂ
ಇವೇ
ಇಷ್ಟ
ಇಷ್ಟ-ದೇ-ವತೆ
ಇಷ್ಟ-ದೇ-ವ-ತೆ-ಯನ್ನೋ
ಇಷ್ಟನ್ನು
ಇಷ್ಟ-ಪಟ್ಟ
ಇಷ್ಟ-ಪ-ಟ್ಟರು
ಇಷ್ಟ-ಪ-ಟ್ಟರೆ
ಇಷ್ಟ-ಪ-ಟ್ಟಿದ್ದ
ಇಷ್ಟ-ಪ-ಡದೆ
ಇಷ್ಟ-ಪ-ಡ-ಲಿಲ್ಲ
ಇಷ್ಟ-ಪಡು
ಇಷ್ಟ-ಪ-ಡು-ತ್ತಾ-ನೆ-ಅ-ವ-ಸ-ರ-ವ-ಸ-ರ-ವಾಗಿ
ಇಷ್ಟ-ಪ-ಡು-ತ್ತಾರೆ
ಇಷ್ಟ-ಪ-ಡು-ತ್ತಿ-ರ-ಲಿಲ್ಲ
ಇಷ್ಟ-ಪ-ಡು-ತ್ತೇನೆ
ಇಷ್ಟ-ಪ-ಡು-ವು-ದಿಲ್ಲ
ಇಷ್ಟ-ಬಂ-ದ-ದ್ದನ್ನು
ಇಷ್ಟ-ಬಂ-ದಷ್ಟು
ಇಷ್ಟ-ರ-ಮ-ಟ್ಟಿಗೆ
ಇಷ್ಟ-ರಲ್ಲಿ
ಇಷ್ಟ-ರಲ್ಲೇ
ಇಷ್ಟ-ವಾ-ಯಿತು
ಇಷ್ಟ-ವಾ-ಯಿತೇ
ಇಷ್ಟ-ವಿದೆ
ಇಷ್ಟ-ವಿ-ದೆಯೆ
ಇಷ್ಟ-ವಿ-ರ-ಲಿಲ್ಲ
ಇಷ್ಟ-ವಿಲ್ಲ
ಇಷ್ಟ-ವಿ-ಲ್ಲ-ದಿ-ದ್ದರೂ
ಇಷ್ಟ-ವಿ-ಲ್ಲ-ದು-ದ-ರಿಂದ
ಇಷ್ಟ-ವಿ-ಲ್ಲ-ವೆಂದು
ಇಷ್ಟವೇ
ಇಷ್ಟ-ವೇನೋ
ಇಷ್ಟಾ
ಇಷ್ಟಾ-ದರೂ
ಇಷ್ಟಾ-ನಿ-ಷ್ಟ-ಗಳನ್ನು
ಇಷ್ಟು
ಇಷ್ಟು-ದಿ-ನವೂ
ಇಷ್ಟೆ
ಇಷ್ಟೆ-ನಿ-ಮಗೆ
ಇಷ್ಟೆಲ್ಲ
ಇಷ್ಟೇ
ಇಷ್ಟೊಂ-ದಾಗಿ
ಇಷ್ಟೊಂದು
ಇಸ-ವಿ-ತಿಂ-ಗ-ಳು-ದಿನ
ಇಸ-ವಿಯ
ಇಸ-ವಿ-ಯಲ್ಲಿ
ಇಸ-ವಿಯು
ಇಸಾ-ಬೆಲ್
ಇಸ್ರೇ-ಲಿ-ಯರ
ಇಸ್ಲಾಂ
ಇಹ
ಇಹ-ಜೀ-ವ-ನ-ವನ್ನು
ಇಹ-ಪ-ರ-ಗ-ಳೆ-ಲ್ಲ-ವನ್ನು
ಇಹ-ಲೋ-ಕ-ದಲ್ಲಿ
ಇಹ-ಲೋ-ಕ-ಯಾ-ತ್ರೆ-ಯನ್ನು
ಈ
ಈಕೆ
ಈಕೆಯ
ಈಕೆಯೂ
ಈಗ
ಈಗ-ತಾನೆ
ಈಗ-ಲಾ-ದರೂ
ಈಗಲೂ
ಈಗಲೇ
ಈಗಾ
ಈಗಾ-ಗಲೇ
ಈಗಿನ
ಈಗಿ-ನಂತೆ
ಈಗಿ-ನ-ದ-ರಲ್ಲಿ
ಈಗಿ-ರುವ
ಈಗಿ-ರು-ವಂತೆ
ಈಗಿ-ರು-ವಷ್ಟೇ
ಈಗಿ-ರು-ವು-ದ-ಕ್ಕಿಂ-ತಲೂ
ಈಗೀಗ
ಈಗೆಲ್ಲ
ಈಗೆ-ಲ್ಲಿ-ದ್ದಾರೆ
ಈಗೆಲ್ಲೂ
ಈಗೇನು
ಈಗೊಮ್ಮೆ
ಈಚಿನ
ಈಚೆಗೆ
ಈಜಿ
ಈಜಿ-ಕೊಂಡು
ಈಜುತ್ತ
ಈಟಿಯ
ಈಟಿ-ಯಂ-ತಹ
ಈಡಿಗ
ಈಡೇ-ರ-ಬೇ-ಕಾ-ದರೆ
ಈಡೇ-ರ-ಲಿ-ಲ್ಲ-ವೆಂದು
ಈಡೇ-ರಿ-ಕೆ-ಗಾಗಿ
ಈಡೇ-ರಿ-ಕೆಗೆ
ಈಡೇ-ರು-ವು-ದೆಂದು
ಈತ
ಈತನ
ಈತ-ನದು
ಈತ-ನನ್ನು
ಈತ-ನಲ್ಲಿ
ಈತ-ನಿಂದ
ಈತ-ನಿಗೆ
ಈತನೂ
ಈತ-ನೊಂ-ದಿಗೆ
ಈವ-ನಿಂಗ್
ಈವರೆ-ಗಿನ
ಈವ-ರೆಗೆ
ಈಶ್ವರ
ಈಸ್ಟ್
ಉಂಟಾ-ಗ-ಲಿಲ್ಲ
ಉಂಟಾ-ಗಿ-ರ-ಬ-ಹುದು
ಉಂಟಾ-ಗಿ-ರುವ
ಉಂಟಾ-ಗು-ತ್ತದೆ
ಉಂಟಾ-ಗುವ
ಉಂಟಾದ
ಉಂಟಾ-ದುದು
ಉಂಟಾ-ದುವು
ಉಂಟಾ-ಯಿತು
ಉಂಟಾ-ಯಿ-ತೆಂ-ದರೆ
ಉಂಟು
ಉಂಟು-ಮಾ-ಡ-ಬ-ಹು-ದಾ-ಗಿದ್ದ
ಉಂಟು-ಮಾ-ಡ-ಬ-ಹು-ದಾದ
ಉಂಟು-ಮಾಡಿ
ಉಂಟು-ಮಾ-ಡಿತು
ಉಂಟು-ಮಾ-ಡಿದ
ಉಂಟು-ಮಾ-ಡಿ-ದರು
ಉಂಟು-ಮಾ-ಡಿ-ದ-ರೆಂ-ದರೆ
ಉಂಟು-ಮಾ-ಡಿ-ದ-ವರು
ಉಂಟು-ಮಾ-ಡಿ-ದುದು
ಉಂಟು-ಮಾ-ಡಿ-ದುವು
ಉಂಟು-ಮಾ-ಡಿ-ದು-ವೆಂ-ದರೂ
ಉಂಟು-ಮಾ-ಡಿದ್ದ
ಉಂಟು-ಮಾ-ಡಿ-ದ್ದರೂ
ಉಂಟು-ಮಾ-ಡು-ತ್ತಿದ್ದ
ಉಂಟು-ಮಾ-ಡು-ತ್ತಿ-ದ್ದುವು
ಉಂಟು-ಮಾ-ಡುವ
ಉಂಟು-ಮಾ-ಡು-ವುದು
ಉಂಟೆ
ಉಂಡರೆ
ಉಕ್ಕಿ
ಉಕ್ಕಿತು
ಉಕ್ಕಿ-ದುವು
ಉಕ್ಕಿನ
ಉಕ್ಕಿ-ಬಂದ
ಉಕ್ಕಿ-ಬಂ-ದ-ದ್ದಲ್ಲ
ಉಕ್ಕಿ-ಬ-ರುವ
ಉಕ್ಕಿ-ಹ-ರಿದು
ಉಕ್ಕಿ-ಹ-ರಿ-ಯು-ತ್ತಿ-ದ್ದುದು
ಉಕ್ಕಿ-ಹ-ರಿ-ಯು-ತ್ತಿ-ದ್ದುವು
ಉಕ್ಕಿ-ಹ-ರಿ-ಯು-ವುದನ್ನು
ಉಕ್ಕೇ-ರಿತು
ಉಕ್ಕೇರು
ಉಗಮ
ಉಗ-ಮ-ಸ್ಥಾನ
ಉಗಿ-ದೋ-ಣಿ-ಗಳ
ಉಗು-ರಿ-ನಿಂದ
ಉಗುಳ
ಉಗು-ಳಲು
ಉಗುಳಿ
ಉಗು-ಳಿ-ದರೆ
ಉಗುಳು
ಉಗು-ಳು-ವುದೆ
ಉಗ್ರ
ಉಗ್ರ-ತಮ
ಉಗ್ರ-ವಾಗಿ
ಉಚಿತ
ಉಚಿ-ತ-ವಾಗಿ
ಉಚ್ಚ
ಉಚ್ಚ-ಕಂ-ಠ-ದಿಂದ
ಉಚ್ಚ-ಕು-ಲ-ದ-ವ-ರೊಂ-ದಿಗೆ
ಉಚ್ಚ-ರಿ-ಸ-ಬೇ-ಕೆಂ-ಬು-ದನ್ನು
ಉಚ್ಚ-ರಿ-ಸಲು
ಉಚ್ಚ-ರಿ-ಸಿಲ್ಲ
ಉಚ್ಚ-ರಿ-ಸು-ತ್ತಿದ್ದ
ಉಚ್ಚ-ರಿ-ಸುವ
ಉಚ್ಚ-ಸ್ವ-ರ-ದಲ್ಲಿ
ಉಚ್ಚಾ-ರಣೆ
ಉಚ್ಚಾ-ರವೂ
ಉಚ್ಛಾ-ಟಿ-ಸಿ-ದರು
ಉಚ್ಛೃಂ-ಖಲ
ಉಚ್ಛ್ರಾ-ಯದ
ಉಜ್ವಲ
ಉಜ್ವ-ಲ-ವಾಗಿ
ಉಜ್ವ-ಲ-ವಾ-ದದ್ದು
ಉಜ್ವ-ಲವೂ
ಉಟೋ-ಕಿಯ
ಉಡಾ-ಳರು
ಉಡಿ-ಗೆ-ತೊ-ಡಿಗೆ
ಉಡಿ-ಗೆ-ತೊ-ಡಿ-ಗೆ-ಗಳನ್ನು
ಉಡಿ-ಗೆ-ತೊ-ಡಿ-ಗೆಯ
ಉಡಿ-ಗೆ-ತೊ-ಡಿ-ಗೆ-ಯನ್ನೂ
ಉಡುಗೆ
ಉಡು-ಗೆ-ಯನ್ನು
ಉಡು-ಗೆ-ಯನ್ನೂ
ಉಡು-ಗೆ-ಯಲ್ಲಿ
ಉಡು-ಗೆ-ಯೇ-ಕೆಂದು
ಉಡು-ಗೊ-ರೆ-ಗಳನ್ನೂ
ಉಡು-ಗೊ-ರೆ-ಯೊಂ-ದನ್ನು
ಉಡು-ಪಿ-ನಿಂ-ದಲೂ
ಉಡುಪು
ಉಣ-ಬ-ಡಿ-ಸ-ಬೇಕು
ಉಣ-ಬ-ಡಿ-ಸಿ-ದರು
ಉಣಿ-ಸ-ಲಾ-ರದ
ಉಣ್ಣೆ-ಬ-ಟ್ಟೆ-ಗಳನ್ನು
ಉಣ್ಣೆ-ಬ-ಟ್ಟೆ-ಗ-ಳಿ-ರ-ಲಿಲ್ಲ
ಉಣ್ಣೆ-ಬ-ಟ್ಟೆ-ಯನ್ನು
ಉಣ್ಣೆಯ
ಉತ್ಕಂ-ಠಿ-ತ-ರಾ-ದರು
ಉತ್ಕ-ಟಾ-ವ-ಸ್ಥೆ-ಯಲ್ಲೇ
ಉತ್ಕ-ಟೇಚ್ಛೆ
ಉತ್ಕ-ಟೇ-ಚ್ಛೆ-ಯಾ-ಗಿತ್ತು
ಉತ್ಕ-ಟೇ-ಚ್ಛೆ-ಯಾ-ಯಿತು
ಉತ್ಕ-ಟೇ-ಚ್ಛೆ-ಯಿಂದ
ಉತ್ಕ-ಟೇ-ಚ್ಛೆ-ಯುಂ-ಟಾಗು
ಉತ್ಕ-ಟೇ-ಚ್ಛೆಯೂ
ಉತ್ಕ-ರ್ಷದ
ಉತ್ಕ-ರ್ಷ-ಪೂರ್ಣ
ಉತ್ಕಾಂ-ಕ್ಷೆ-ಯಾಗಿ
ಉತ್ಕೃ-ಷ್ಟ-ತೆ-ಯನ್ನು
ಉತ್ಕೃ-ಷ್ಟ-ತೆ-ಯನ್ನೂ
ಉತ್ಕೃ-ಷ್ಟ-ವಾದ
ಉತ್ಕ್ರಾಂತಿ
ಉತ್ಕ್ರಾಂ-ತಿ-ಯನ್ನು
ಉತ್ತಮ
ಉತ್ತ-ಮ-ಗೊ-ಳ್ಳು-ತ್ತಾನೆ
ಉತ್ತ-ಮ-ಪ-ಡಿ-ಸ-ಬ-ಲ್ಲಿ-ರೇನು
ಉತ್ತ-ಮ-ಪ-ಡಿ-ಸಿಕೊ
ಉತ್ತ-ಮ-ರಾದ
ಉತ್ತ-ಮ-ರಾ-ದ-ವರು
ಉತ್ತ-ಮರು
ಉತ್ತ-ಮ-ರು-ನೀ-ಗ್ರೋ-ಗಳು
ಉತ್ತ-ಮ-ರೆ-ನಿ-ಸಿ-ಕೊಂ-ಡ-ವರು
ಉತ್ತ-ಮ-ವಾ-ಗಿತ್ತು
ಉತ್ತ-ಮ-ವಾ-ಗಿದ್ದು
ಉತ್ತ-ಮ-ವಾ-ಗು-ತ್ತಿ-ರು-ವಂತೆ
ಉತ್ತ-ಮವೂ
ಉತ್ತರ
ಉತ್ತ-ರಕ್ಕೆ
ಉತ್ತ-ರ-ಗಳ
ಉತ್ತ-ರ-ಗಳನ್ನು
ಉತ್ತ-ರ-ಗಳನ್ನೂ
ಉತ್ತ-ರ-ಗಳಿಂದ
ಉತ್ತ-ರ-ಗಳು
ಉತ್ತ-ರದ
ಉತ್ತ-ರ-ದಲ್ಲಿ
ಉತ್ತ-ರ-ದಿಂದ
ಉತ್ತ-ರ-ಭಾ-ರ-ತಕ್ಕೆ
ಉತ್ತ-ರ-ವ-ನಿನ್ನೂ
ಉತ್ತ-ರ-ವನ್ನು
ಉತ್ತ-ರ-ವನ್ನೂ
ಉತ್ತ-ರ-ವನ್ನೇ
ಉತ್ತ-ರ-ವಾಗಿ
ಉತ್ತ-ರ-ವಾ-ಗಿತ್ತು
ಉತ್ತ-ರವು
ಉತ್ತ-ರ-ವೆಂದರೆ
ಉತ್ತ-ರ-ಸಿದ್ಧ
ಉತ್ತ-ರಾ-ಧಿ-ಕಾರಿ
ಉತ್ತ-ರಾ-ಧಿ-ಕಾ-ರಿ-ಯಾಗಿ
ಉತ್ತ-ರಾ-ಭಿ-ಮು-ಖ-ವಾಗಿ
ಉತ್ತ-ರಿ-ಸದೆ
ಉತ್ತ-ರಿ-ಸ-ಬಲ್ಲ
ಉತ್ತ-ರಿ-ಸ-ಲಿಲ್ಲ
ಉತ್ತ-ರಿ-ಸಲು
ಉತ್ತ-ರಿಸಿ
ಉತ್ತ-ರಿ-ಸಿದ
ಉತ್ತ-ರಿ-ಸಿ-ದರು
ಉತ್ತ-ರಿ-ಸಿ-ದ-ರುಈ
ಉತ್ತ-ರಿ-ಸಿ-ದ-ರು-ಒಂದು
ಉತ್ತ-ರಿ-ಸಿ-ದರೂ
ಉತ್ತ-ರಿ-ಸಿದ್ದ
ಉತ್ತ-ರಿ-ಸಿ-ದ್ದರು
ಉತ್ತ-ರಿ-ಸಿ-ದ್ದಲ್ಲ
ಉತ್ತ-ರಿ-ಸಿ-ದ್ದೇ-ನೆಯೋ
ಉತ್ತ-ರಿಸು
ಉತ್ತ-ರಿ-ಸುತ್ತ
ಉತ್ತ-ರಿ-ಸು-ತ್ತಾರೆ
ಉತ್ತ-ರಿ-ಸುತ್ತಿ
ಉತ್ತ-ರಿ-ಸು-ತ್ತಿದ್ದ
ಉತ್ತ-ರಿ-ಸು-ತ್ತಿ-ದ್ದರು
ಉತ್ತ-ರಿ-ಸು-ತ್ತಿ-ದ್ದಾರೆ
ಉತ್ತ-ರಿ-ಸು-ತ್ತೇನೆ
ಉತ್ತ-ರಿ-ಸುವ
ಉತ್ತ-ರಿ-ಸು-ವಂ-ತೆಯೂ
ಉತ್ತಿ-ಷ್ಠತ
ಉತ್ತೀ-ರ್ಣ-ರಾ-ದ-ದ್ದನ್ನು
ಉತ್ತುಂಗ
ಉತ್ತೇ-ಜನ
ಉತ್ತೇ-ಜ-ನ-ಕಾ-ರಿ-ಯಾದ
ಉತ್ತೇ-ಜಿಸಿ
ಉತ್ತೇ-ಜಿ-ಸಿತು
ಉತ್ತೇ-ಜಿ-ಸು-ತ್ತಿ-ದ್ದರು
ಉತ್ಥಾ-ನದ
ಉತ್ಪನ್ನ
ಉತ್ಪ-ನ್ನ-ವಾದ
ಉತ್ಪ-ನ್ನ-ವಾ-ಯಿತು
ಉತ್ಪಾ-ದನೆ
ಉತ್ಪಾ-ದ-ನೆಯ
ಉತ್ಪಾ-ದ-ನೆಯೂ
ಉತ್ಪಾದಿ
ಉತ್ಪಾ-ದಿಸಿ
ಉತ್ಪಾ-ದಿ-ಸಿದ
ಉತ್ಪ್ರೇ-ಕ್ಷ-ಣೀ-ಯ-ವಾಗಿ
ಉತ್ಪ್ರೇ-ಕ್ಷೆ-ಯಾಗಿ
ಉತ್ಪ್ರೇ-ಕ್ಷೆ-ಯಾ-ಗಿ-ರಲೂ
ಉತ್ಪ್ರೇ-ಕ್ಷೆ-ಯಿಂದ
ಉತ್ಸ
ಉತ್ಸವ
ಉತ್ಸಾಹ
ಉತ್ಸಾ-ಹ-ಆ-ನಂ-ದ-ಉ-ದ್ವೇ-ಗ
ಉತ್ಸಾ-ಹ-ಕಾ-ತ-ರ-ಗಳಿಂದ
ಉತ್ಸಾ-ಹ-ತಾ-ಳ್ಮೆ-ಗ-ಳೆಂದೂ
ಉತ್ಸಾ-ಹ-ಮೆ-ಚ್ಚು-ಗೆ-ಯಿಂದ
ಉತ್ಸಾ-ಹ-ಸ್ಫೂ-ರ್ತಿಯ
ಉತ್ಸಾ-ಹಕ್ಕೆ
ಉತ್ಸಾ-ಹ-ಗಳನ್ನು
ಉತ್ಸಾ-ಹ-ಗಳಿಂದ
ಉತ್ಸಾ-ಹ-ಗೊ-ಳಿ-ಸಲು
ಉತ್ಸಾ-ಹದ
ಉತ್ಸಾ-ಹ-ದಲ್ಲಿ
ಉತ್ಸಾ-ಹ-ದಿಂದ
ಉತ್ಸಾ-ಹ-ದಿಂ-ದಿ-ದ್ದರು
ಉತ್ಸಾ-ಹ-ಪೂರ್ಣ
ಉತ್ಸಾ-ಹ-ಭ-ರಿತ
ಉತ್ಸಾ-ಹ-ಭ-ರಿ-ತ-ರಾ-ಗು-ವಂತೆ
ಉತ್ಸಾ-ಹ-ಯುತ
ಉತ್ಸಾ-ಹ-ಯು-ತ-ರಾದ
ಉತ್ಸಾ-ಹ-ವನ್ನು
ಉತ್ಸಾ-ಹ-ವನ್ನೂ
ಉತ್ಸಾ-ಹವು
ಉತ್ಸಾ-ಹ-ವು-ಕ್ಕಿತು
ಉತ್ಸಾ-ಹ-ವು-ಕ್ಕಿ-ಸುವ
ಉತ್ಸಾ-ಹ-ವೆಂ-ಥದು
ಉತ್ಸಾ-ಹ-ವೆಲ್ಲ
ಉತ್ಸಾ-ಹ-ಶಾಲಿ
ಉತ್ಸಾ-ಹ-ಶಾಲೀ
ಉತ್ಸಾ-ಹ-ಶೀ-ಲ-ರಾದ
ಉತ್ಸಾ-ಹ-ಶೂ-ನ್ಯತೆ
ಉತ್ಸಾಹಿ
ಉತ್ಸಾ-ಹಿ-ಗ-ಳಾದ
ಉತ್ಸಾ-ಹಿ-ಗಳು
ಉತ್ಸಾ-ಹಿ-ಗಳೂ
ಉತ್ಸಾ-ಹಿ-ತ-ನಾ-ಗಿ-ದ್ದೇನೆ
ಉತ್ಸಾ-ಹಿ-ತ-ರಾದ
ಉತ್ಸಾ-ಹಿ-ತ-ರಾ-ದರು
ಉತ್ಸಾಹೀ
ಉತ್ಸುಕ
ಉತ್ಸು-ಕ-ನಾ-ಗಿ-ದ್ದರೂ
ಉತ್ಸು-ಕ-ರಾ-ಗಿ-ದ್ದರು
ಉತ್ಸು-ಕ-ರಾ-ಗಿ-ರ-ಲಿ-ಲ್ಲ-ವೆಂ-ದಲ್ಲ
ಉತ್ಸು-ಕ-ರಾ-ದದ್ದು
ಉತ್ಸು-ಕ-ರಾ-ದರು
ಉದ-ಯ-ರವಿ
ಉದ-ರ-ನಿ-ಮಿತ್ತ
ಉದಾ
ಉದಾತ್ತ
ಉದಾ-ತ್ತ-ಶ್ರೇಷ್ಠ
ಉದಾ-ತ್ತ-ಚ-ರಿ-ತ-ರಾದ
ಉದಾ-ತ್ತ-ಚ-ರಿ-ತರು
ಉದಾ-ತ್ತ-ತೆಗೆ
ಉದಾ-ತ್ತ-ತೆ-ಯನ್ನು
ಉದಾ-ತ್ತ-ಮ-ನ-ಸ್ಕ-ರಾದ
ಉದಾ-ತ್ತ-ವಾಗಿ
ಉದಾ-ತ್ತ-ವಾ-ಗಿ-ದ್ದರೂ
ಉದಾ-ತ್ತ-ವಾ-ಗಿ-ರಲಿ
ಉದಾ-ತ್ತ-ವಾದ
ಉದಾ-ತ್ತವೂ
ಉದಾರ
ಉದಾ-ರ-ಚ-ರಿ-ತ-ರಾದ
ಉದಾ-ರ-ತೆಗೆ
ಉದಾ-ರ-ವಾಗಿ
ಉದಾ-ರ-ವಾ-ದಿ-ಗ-ಳಾದ
ಉದಾ-ರಿ-ಗಳು
ಉದಾ-ರಿ-ಗ-ಳು-ಹೀಗೆ
ಉದಾ-ರಿ-ಗಳೂ
ಉದಾ-ಸೀನ
ಉದಾ-ಹ-ರಣೆ
ಉದಾ-ಹ-ರ-ಣೆ-ಗಳನ್ನು
ಉದಾ-ಹ-ರ-ಣೆ-ಗಳು
ಉದಾ-ಹ-ರ-ಣೆಗೆ
ಉದಾ-ಹ-ರ-ಣೆ-ಯಾಗಿ
ಉದಾ-ಹ-ರ-ಣೆ-ಯಾ-ಗಿದೆ
ಉದಾ-ಹ-ರ-ಣೆ-ಯೆಂ-ದರೆ
ಉದಾ-ಹ-ರ-ಣೆ-ಯೊಂ-ದನ್ನು
ಉದಾ-ಹ-ರಿಸಿ
ಉದಾ-ಹ-ರಿ-ಸಿ-ದರು
ಉದಿಸಿ
ಉದಿ-ಸಿತ್ತು
ಉದಿ-ಸಿದ
ಉದಿ-ಸಿ-ದಷ್ಟೇ
ಉದಿ-ಸಿ-ದಾಗ
ಉದಿ-ಸಿ-ದು-ದಲ್ಲ
ಉದಿ-ಸಿ-ಬ-ರ-ಲಿ-ದ್ದಾರೆ
ಉದಿ-ಸು-ತ್ತಿದೆ
ಉದ್ಗ-ರಿಸಿ
ಉದ್ಗ-ರಿ-ಸಿದ
ಉದ್ಗ-ರಿ-ಸಿ-ದ-ರಂತೆ
ಉದ್ಗ-ರಿ-ಸಿ-ದರು
ಉದ್ಗ-ರಿ-ಸಿ-ದ-ಳು-ಇ-ದ-ನ್ನೆಲ್ಲ
ಉದ್ಗ-ರಿ-ಸಿ-ದ-ಸ್ವಾ-ಮೀಜಿ
ಉದ್ಗ-ರಿ-ಸಿದೆ
ಉದ್ಗ-ರಿ-ಸುತ್ತ
ಉದ್ಗ-ರಿ-ಸು-ತ್ತಾರೆ
ಉದ್ಗ-ರಿ-ಸು-ತ್ತಿದ್ದ
ಉದ್ಗ-ರಿ-ಸು-ತ್ತಿ-ದ್ದರು
ಉದ್ಗ-ಲಕ್ಕೂ
ಉದ್ಗಾ-ರ-ಗಳು
ಉದ್ಗಾ-ರ-ದಂತೆ
ಉದ್ಗಾ-ರ-ದಿಂದ
ಉದ್ಗಾ-ರ-ವಲ್ಲ
ಉದ್ಗ್ರಂ-ಥ-ವನ್ನು
ಉದ್ಘಾ-ಟ-ನೆಯ
ಉದ್ಘಾ-ಟಿ-ಸ-ಲ್ಪ-ಟ್ಟಿದ್ದ
ಉದ್ಘೋ-ಷಿ-ಸಿತು
ಉದ್ಘೋ-ಷಿ-ಸು-ತ್ತಿ-ದ್ದರು
ಉದ್ದಂಡ
ಉದ್ದಕ್ಕೂ
ಉದ್ದ-ಗ-ಲಕ್ಕೂ
ಉದ್ದ-ಗ-ಲ-ವ-ನ್ನೆಲ್ಲ
ಉದ್ದ-ನೆಯ
ಉದ್ದೀ-ಪ-ನ-ಗೊ-ಳಿ-ಸುವ
ಉದ್ದೇಶ
ಉದ್ದೇ-ಶ-ಸಾ-ಧ-ನೆ-ಗಳ
ಉದ್ದೇ-ಶ-ಕ್ಕಾಗಿ
ಉದ್ದೇ-ಶಕ್ಕೆ
ಉದ್ದೇ-ಶ-ಗಳ
ಉದ್ದೇ-ಶ-ಗಳನ್ನು
ಉದ್ದೇ-ಶ-ಗಳಲ್ಲಿ
ಉದ್ದೇ-ಶ-ಗಳಿಂದ
ಉದ್ದೇ-ಶ-ಗ-ಳಿ-ಗಾಗಿ
ಉದ್ದೇ-ಶ-ಗಳು
ಉದ್ದೇ-ಶ-ಗಳೂ
ಉದ್ದೇ-ಶದ
ಉದ್ದೇ-ಶ-ದಿಂದ
ಉದ್ದೇ-ಶ-ದಿಂ-ದಲೇ
ಉದ್ದೇ-ಶ-ನಿ-ಷ್ಠ-ರಾದ
ಉದ್ದೇ-ಶ-ಪೂ-ರ್ವಕ
ಉದ್ದೇ-ಶ-ವ-ಡ-ಗಿ-ರ-ಬೇ-ಕೆಂದು
ಉದ್ದೇ-ಶ-ವ-ನ್ನ-ರಿ-ಯದೆ
ಉದ್ದೇ-ಶ-ವ-ನ್ನಿ-ಟ್ಟು-ಕೊಂ-ಡಿ-ದ್ದರೂ
ಉದ್ದೇ-ಶ-ವನ್ನು
ಉದ್ದೇ-ಶ-ವನ್ನೂ
ಉದ್ದೇ-ಶ-ವಾಗಿ
ಉದ್ದೇ-ಶ-ವಾ-ಗಿತ್ತು
ಉದ್ದೇ-ಶ-ವಾ-ಗಿ-ದ್ದರೆ
ಉದ್ದೇ-ಶ-ವಾ-ಗಿ-ದ್ದು-ದ-ರಿಂದ
ಉದ್ದೇ-ಶ-ವಾ-ದರೆ
ಉದ್ದೇ-ಶ-ವಿತ್ತು
ಉದ್ದೇ-ಶ-ವಿ-ರು-ತ್ತಿತ್ತು
ಉದ್ದೇ-ಶವು
ಉದ್ದೇ-ಶವೂ
ಉದ್ದೇ-ಶ-ವೆಂದರೆ
ಉದ್ದೇ-ಶ-ವೆಂ-ಬು-ದನ್ನು
ಉದ್ದೇ-ಶವೇ
ಉದ್ದೇ-ಶ-ವೇನು
ಉದ್ದೇ-ಶ-ವೇನೆಂದರೆ
ಉದ್ದೇ-ಶ-ಸಿ-ದ್ಧಿ-ಗಾಗಿ
ಉದ್ದೇ-ಶ-ಸಿ-ದ್ಧಿಗೆ
ಉದ್ದೇಶಿ
ಉದ್ದೇ-ಶಿತ
ಉದ್ದೇ-ಶಿಸಿ
ಉದ್ದೇ-ಶಿ-ಸಿದ
ಉದ್ದೇ-ಶಿ-ಸಿ-ದರು
ಉದ್ದೇ-ಶಿ-ಸಿ-ದ್ದರು
ಉದ್ದೇ-ಶಿ-ಸಿ-ದ್ಧಿ-ಗಾಗಿ
ಉದ್ಧ-ಟ-ತ-ನ-ವನ್ನು
ಉದ್ಧ-ರಿಸಿ
ಉದ್ಧ-ರಿ-ಸಿತು
ಉದ್ಧ-ರಿ-ಸಿ-ದರು
ಉದ್ಧ-ರಿ-ಸುತ್ತ
ಉದ್ಧ-ರಿ-ಸುತ್ತಿ
ಉದ್ಧ-ರಿ-ಸು-ವಂತೆ
ಉದ್ಧ-ವಿ-ಸಿತು
ಉದ್ಧಾ-ರ-ಕ-ರಾ-ಗಿ-ದ್ದರು
ಉದ್ಧಾ-ರ-ಕ್ಕಾಗಿ
ಉದ್ಧಾ-ರದ
ಉದ್ಧಾ-ರ-ವಾ-ಗಲು
ಉದ್ಧಾ-ರ-ವಿಲ್ಲ
ಉದ್ಧೃ-ತ-ವಾದ
ಉದ್ಭ-ವಿಸಿ
ಉದ್ಭ-ವಿ-ಸಿತು
ಉದ್ಭ-ವಿ-ಸಿದ
ಉದ್ಭ-ವಿ-ಸಿ-ದ-ನೀತ
ಉದ್ಭ-ವಿ-ಸಿದೆ
ಉದ್ಭ-ವಿ-ಸು-ತ್ತಾರೆ
ಉದ್ಭ-ವಿ-ಸುವ
ಉದ್ಯ-ಮಿ-ಯಾ-ಗಿದ್ದ
ಉದ್ಯಾನ
ಉದ್ಯಾ-ನ-ಗಳು
ಉದ್ಯಾ-ನ-ವೊಂ-ದರ
ಉದ್ಯು-ಕ್ತರಾ
ಉದ್ಯು-ಕ್ತ-ರಾ-ಗ-ತೊ-ಡ-ಗಿ-ದರು
ಉದ್ಯು-ಕ್ತ-ರಾ-ಗಿ-ದ್ದರು
ಉದ್ಯು-ಕ್ತ-ರಾ-ಗಿ-ರುವ
ಉದ್ಯು-ಕ್ತ-ರಾ-ದರು
ಉದ್ಯೋಗ
ಉದ್ಯೋ-ಗ-ದ-ಲ್ಲಿದ್ದ
ಉದ್ಯೋ-ಗ-ದ-ಲ್ಲಿ-ರು-ವ-ವ-ರಿಗೆ
ಉದ್ಯೋ-ಗ-ವ-ನ್ನ-ರಸಿ
ಉದ್ಯೋ-ಗ-ವ-ನ್ನಾಗಿ
ಉದ್ಯೋ-ಗ-ವಿತ್ತು
ಉದ್ಯೋ-ಗವೇ
ಉದ್ಯೋ-ಗ-ವೊಂ-ದನ್ನು
ಉದ್ರೇ-ಕ-ಕಾರಿ
ಉದ್ರೇ-ಕ-ಕಾರೀ
ಉದ್ವಿ-ಗ್ನತೆ
ಉದ್ವಿ-ಗ್ನ-ನಾಗಿ
ಉದ್ವೇ-ಗ-ಕರ
ಉದ್ವೇ-ಗ-ಗೊಂಡ
ಉದ್ವೇ-ಗ-ಗೊಂಡು
ಉದ್ವೇ-ಗ-ದಿಂದ
ಉದ್ವೇ-ಗ-ವುಂ-ಟಾ-ಗು-ವಂತೆ
ಉನ್ನತ
ಉನ್ನ-ತ-ಉ-ದಾತ್ತ
ಉನ್ನ-ತ-ಭಾ-ವ-ದಲ್ಲೇ
ಉನ್ನತಿ
ಉನ್ನ-ತಿಗೆ
ಉನ್ನ-ತಿಯ
ಉನ್ನ-ತ್ತ-ಳಾದ
ಉನ್ಮ-ತ್ತ-ರಂತೆ
ಉಪ
ಉಪ-ಕ-ರ-ಣ-ಗ-ಳಾ-ಗು-ವ-ವರು
ಉಪ-ಕ-ರ-ಣ-ಗಳು
ಉಪ-ಕ-ರ-ಣ-ಗ-ಳೊಂ-ದಿಗೆ
ಉಪ-ಕ-ರ-ಣದ
ಉಪ-ಕ-ರ-ಣ-ವಾ-ಗ-ಬ-ಲ್ಲ-ರೆಂದು
ಉಪ-ಕ-ರ-ಣ-ವಾಗಿ
ಉಪ-ಕ-ರ-ಣ-ವಾ-ಗು-ವಂ-ತಾ-ಗಲಿ
ಉಪ-ಕ-ರ-ಣವೇ
ಉಪ-ಕಾರ
ಉಪ-ಕಾ-ರ-ಕ್ಕಾಗಿ
ಉಪ-ಕಾ-ರ-ಕ್ಕಿಂತ
ಉಪ-ಕಾ-ರಕ್ಕೂ
ಉಪ-ಕಾ-ರ-ವ-ನ್ನೆಂದೂ
ಉಪ-ಕಾ-ರ-ವಾ-ಗ-ಲಿಲ್ಲ
ಉಪ-ಕಾ-ರ-ವಾ-ದುದು
ಉಪ-ಕಾ-ರಿ-ಯನ್ನು
ಉಪ-ಚ-ರಿ-ಸಿ-ದರು
ಉಪ-ಚ-ರಿ-ಸು-ತ್ತಿ-ದ್ದರು
ಉಪ-ಚ-ರಿ-ಸು-ವಷ್ಟು
ಉಪ-ಚಾ-ರ-ವನ್ನು
ಉಪ-ದಿ-ವಾ-ನ-ನಾದ
ಉಪ-ದೇಶ
ಉಪ-ದೇ-ಶ-ಗಳ
ಉಪ-ದೇ-ಶ-ಗಳನ್ನು
ಉಪ-ದೇ-ಶದ
ಉಪ-ದೇ-ಶ-ವೆಂದು
ಉಪ-ದೇ-ಶಾ-ಮೃ-ತ-ವನ್ನು
ಉಪ-ದೇ-ಶಿದ
ಉಪ-ದೇ-ಶಿ-ಸಿದ್ದು
ಉಪ-ದ್ರ-ವಕ್ಕೆ
ಉಪ-ನ-ಗ-ರ-ಗ-ಳಿಗೆ
ಉಪ-ನ-ಗ-ರ-ವಾದ
ಉಪ-ನ-ಯನ
ಉಪ-ನ-ಯ-ನ-ಕ್ಕಾಗಿ
ಉಪ-ನ-ಯ-ನಕ್ಕೆ
ಉಪ-ನ-ಯ-ನದ
ಉಪ-ನ-ಯ-ನ-ವನ್ನು
ಉಪ-ನ-ಯ-ನ-ವಾ-ಗದೆ
ಉಪ-ನಾಮ
ಉಪ-ನಾ-ಮ-ವಿ-ರ-ಬೇ-ಕೆಂದು
ಉಪ-ನಿ-ಷ-ತ್ಕಾ-ಲದ
ಉಪ-ನಿ-ಷ-ತ್ತಿನ
ಉಪ-ನಿ-ಷ-ತ್ತು-ಗಳ
ಉಪ-ನಿ-ಷ-ತ್ತು-ಗಳನ್ನು
ಉಪ-ನಿ-ಷ-ತ್ತು-ಗಳಲ್ಲಿ
ಉಪ-ನಿ-ಷ-ತ್ತು-ಗಳಿಂದ
ಉಪ-ನಿ-ಷ-ತ್ತು-ಗಳು
ಉಪ-ನಿ-ಷತ್ತೂ
ಉಪ-ನಿ-ಷ-ದ್ವಾ-ಕ್ಯ-ಗಳು
ಉಪ-ನ್ಯಾಸ
ಉಪ-ನ್ಯಾ-ಸ-ಬೋ-ಧ-ನೆ-ವಿ-ಹಾ-ರ-ಗಳಲ್ಲಿ
ಉಪ-ನ್ಯಾ-ಸಕ
ಉಪ-ನ್ಯಾ-ಸ-ಕ-ನಂತೆ
ಉಪ-ನ್ಯಾ-ಸ-ಕ-ನಾ-ಗಿ-ಅ-ದ-ರಲ್ಲೂ
ಉಪ-ನ್ಯಾ-ಸ-ಕರ
ಉಪ-ನ್ಯಾ-ಸ-ಕ-ರಾಗಿ
ಉಪ-ನ್ಯಾ-ಸ-ಕ-ರಾ-ಗು-ವಂತೆ
ಉಪ-ನ್ಯಾ-ಸ-ಕ-ರಿಗೂ
ಉಪ-ನ್ಯಾ-ಸ-ಕರು
ಉಪ-ನ್ಯಾ-ಸ-ಕರೂ
ಉಪ-ನ್ಯಾ-ಸ-ಕ್ಕಂತೂ
ಉಪ-ನ್ಯಾ-ಸ-ಕ್ಕಾಗಿ
ಉಪ-ನ್ಯಾ-ಸಕ್ಕೆ
ಉಪ-ನ್ಯಾ-ಸ-ಗಳ
ಉಪ-ನ್ಯಾ-ಸ-ಗಳನ್ನು
ಉಪ-ನ್ಯಾ-ಸ-ಗಳನ್ನೂ
ಉಪ-ನ್ಯಾ-ಸ-ಗ-ಳ-ಲ್ಲದೆ
ಉಪ-ನ್ಯಾ-ಸ-ಗಳಲ್ಲಿ
ಉಪ-ನ್ಯಾ-ಸ-ಗ-ಳಾದ
ಉಪ-ನ್ಯಾ-ಸ-ಗಳಿ
ಉಪ-ನ್ಯಾ-ಸ-ಗಳಿಂದ
ಉಪ-ನ್ಯಾ-ಸ-ಗ-ಳಿಂ-ದಾಗಿ
ಉಪ-ನ್ಯಾ-ಸ-ಗ-ಳಿಗೂ
ಉಪ-ನ್ಯಾ-ಸ-ಗ-ಳಿಗೆ
ಉಪ-ನ್ಯಾ-ಸ-ಗಳು
ಉಪ-ನ್ಯಾ-ಸ-ಗ-ಳು-ಇ-ವು-ಗಳಲ್ಲಿ
ಉಪ-ನ್ಯಾ-ಸ-ಗ-ಳು-ಇ-ವೆಲ್ಲ
ಉಪ-ನ್ಯಾ-ಸ-ಗಳೂ
ಉಪ-ನ್ಯಾ-ಸ-ಗ-ಳೆಲ್ಲ
ಉಪ-ನ್ಯಾ-ಸ-ಗ-ಳೆ-ಲ್ಲವೂ
ಉಪ-ನ್ಯಾ-ಸ-ಗ-ಳೊಂ-ದಿಗೆ
ಉಪ-ನ್ಯಾ-ಸದ
ಉಪ-ನ್ಯಾ-ಸ-ದಲ್ಲಿ
ಉಪ-ನ್ಯಾ-ಸ-ದಲ್ಲೂ
ಉಪ-ನ್ಯಾ-ಸ-ದೊಂ-ದಿಗೂ
ಉಪ-ನ್ಯಾ-ಸ-ದೊಂ-ದಿಗೆ
ಉಪ-ನ್ಯಾ-ಸ-ಭ-ವನ
ಉಪ-ನ್ಯಾ-ಸ-ಮಾಲೆ
ಉಪ-ನ್ಯಾ-ಸ-ಮಾ-ಲೆಯ
ಉಪ-ನ್ಯಾ-ಸ-ವ-ನ್ನಿ-ರಿ-ಸುವು
ಉಪ-ನ್ಯಾ-ಸ-ವನ್ನು
ಉಪ-ನ್ಯಾ-ಸ-ವನ್ನೂ
ಉಪ-ನ್ಯಾ-ಸ-ವಾದ
ಉಪ-ನ್ಯಾ-ಸವು
ಉಪ-ನ್ಯಾ-ಸವೂ
ಉಪ-ನ್ಯಾ-ಸವೇ
ಉಪ-ನ್ಯಾ-ಸ-ವೊಂ-ದ-ರಲ್ಲಿ
ಉಪ-ನ್ಯಾ-ಸ-ಸಂ-ಸ್ಥೆ-ಗಳ
ಉಪ-ನ್ಯಾ-ಸ-ಸಂ-ಸ್ಥೆಯ
ಉಪ-ಪ್ರಾ-ಧ್ಯಾ-ಪಕ
ಉಪಮೆ
ಉಪ-ಮೆ-ಯ-ನ್ನೊ-ದ-ಗಿ-ಸು-ತ್ತಿತ್ತು
ಉಪ-ಯುಕ್ತ
ಉಪ-ಯು-ಕ್ತ-ತೆಯೂ
ಉಪ-ಯು-ಕ್ತ-ವಾಗಿ
ಉಪ-ಯು-ಕ್ತ-ವಾದ
ಉಪ-ಯು-ಕ್ತ-ವಾ-ದಂ-ಥದು
ಉಪ-ಯು-ಕ್ತ-ವಾ-ದ-ದ್ದನ್ನು
ಉಪ-ಯು-ಕ್ತವೂ
ಉಪ-ಯೋ-ಗ-ಕ-ರ-ವಾಗಿ
ಉಪ-ಯೋ-ಗ-ಕ-ರ-ವಾದ
ಉಪ-ಯೋ-ಗ-ಕ್ಕಾಗಿ
ಉಪ-ಯೋ-ಗಕ್ಕೆ
ಉಪ-ಯೋ-ಗ-ವಾ-ಗ-ಬ-ಹು-ದೆಂದು
ಉಪ-ಯೋ-ಗ-ವಾ-ಗು-ವಂ-ತಹ
ಉಪ-ಯೋ-ಗ-ವೇನೂ
ಉಪ-ಯೋ-ಗಿ-ಸ-ಬ-ಹುದು
ಉಪ-ಯೋ-ಗಿ-ಸ-ಬೇಕು
ಉಪ-ಯೋ-ಗಿಸಿ
ಉಪ-ಯೋ-ಗಿ-ಸಿ-ಕೊಂ-ಡರು
ಉಪ-ಯೋ-ಗಿ-ಸಿ-ಕೊಂಡು
ಉಪ-ಯೋ-ಗಿ-ಸಿ-ಕೊ-ಳ್ಳಲು
ಉಪ-ಯೋ-ಗಿ-ಸಿ-ಕೊ-ಳ್ಳು-ತ್ತಿ-ದ್ದರು
ಉಪ-ಯೋ-ಗಿ-ಸು-ತ್ತಿದ್ದ
ಉಪ-ಯೋ-ಗಿ-ಸು-ತ್ತಿ-ರುವ
ಉಪ-ಯೋ-ಗಿ-ಸುವ
ಉಪ-ಯೋ-ಗಿ-ಸು-ವುದು
ಉಪ-ವಾಸ
ಉಪ-ವಾ-ಸ-ವನ
ಉಪ-ವಾ-ಸದ
ಉಪ-ವಾ-ಸ-ದಿಂದ
ಉಪ-ವಾ-ಸ-ವಿ-ದ್ದೀರಿ
ಉಪ-ವಾ-ಸ-ವಿದ್ದು
ಉಪ-ವಾ-ಸ-ವಿ-ದ್ದು-ದ-ರಿಂದ
ಉಪ-ವಾ-ಸ-ವಿ-ರು-ತ್ತಿದ್ದೆ
ಉಪ-ಸಂ-ಹಾರ
ಉಪ-ಸ್ಥಿ-ತ-ರಿದ್ದ
ಉಪ-ಸ್ಥಿ-ತ-ರಿ-ದ್ದರು
ಉಪ-ಸ್ಥಿ-ತ-ರಿ-ದ್ದ-ವರೆಲ್ಲ
ಉಪ-ಸ್ಥಿ-ತ-ರಿದ್ದು
ಉಪ-ಹಾ-ರ-ವನ್ನೂ
ಉಪಾ-ಧ್ಯಾಯ
ಉಪಾ-ಧ್ಯಾ-ಯ-ರಾ-ಗಿ-ದ್ದರು
ಉಪಾ-ಧ್ಯಾ-ಯ-ರು-ಉಪಾ
ಉಪಾಯ
ಉಪಾ-ಯ-ಗಳ
ಉಪಾ-ಯ-ಗಳಿಂದ
ಉಪಾ-ಯ-ಗಳೂ
ಉಪಾ-ಯ-ದಂತೆ
ಉಪಾ-ಯ-ವನ್ನು
ಉಪಾ-ಯವೂ
ಉಪಾ-ಯ-ವೆಂದರೆ
ಉಪಾ-ಯ-ವೊಂ-ದನ್ನು
ಉಪಾ-ಸ-ಕರು
ಉಪಾ-ಸನೆ
ಉಪಾ-ಸ-ನೆ-ಗಳ
ಉಪಾ-ಸ-ನೆ-ಯನ್ನು
ಉಪಾ-ಹಾರ
ಉಪಾ-ಹಾ-ರಕ್ಕೆ
ಉಪಾ-ಹಾ-ರ-ವನ್ನು
ಉಪಾ-ಹಾ-ರ-ವಾದ
ಉಬ್ಬ-ರದ
ಉಬ್ಬಿ-ಸಲು
ಉಬ್ಬಿ-ಹೋ-ಗಲು
ಉಭಯ
ಉಭ-ಯ-ಕು-ಶ-ಲೋ-ಪ-ರಿ-ಯಾದ
ಉಮಿಯಾ
ಉಮೆಯು
ಉಮ್ಮ-ಳಿಸಿ
ಉರಿ
ಉರಿ-ಕೆಂ-ಡದ
ಉರಿದು
ಉರಿ-ಬಿ-ಸಿ-ಲಿನ
ಉರಿಯ
ಉರಿ-ಯಾ-ಗಿ-ರ-ಬೇಕು
ಉರಿ-ಯು-ತ್ತಿತ್ತು
ಉರಿ-ಯು-ತ್ತಿ-ರು-ತ್ತಿತ್ತು
ಉರಿ-ಯು-ತ್ತಿ-ರುವ
ಉರಿ-ಯುವ
ಉರಿವ
ಉರು
ಉರು-ಳ-ಲೇ-ಬೇಕು
ಉರುಳಿ
ಉರು-ಳಿ-ಕೊಂ-ಡರು
ಉರು-ಳಿದ
ಉರು-ಳಿ-ದಂತೆ
ಉರು-ಳಿ-ದ್ದರ
ಉರು-ಳಿ-ಬೀ-ಳಲು
ಉರು-ಳುವ
ಉರ್ದು
ಉಲೇ-ಖಿ-ಸಿ-ದ್ದಾರೆ
ಉಲ್ಲ-ಸಿ-ತ-ಳಾದೆ
ಉಲ್ಲಾ-ಸದ
ಉಲ್ಲೇ-ಖ-ವಿದೆ
ಉಲ್ಲೇ-ಖ-ವುಳ್ಳ
ಉಲ್ಲೇ-ಖಿ-ಸ-ಲ್ಪ-ಟ್ಟಿ-ರುವ
ಉಲ್ಲೇ-ಖಿಸು
ಉಳಿದ
ಉಳಿ-ದವ
ಉಳಿ-ದ-ವರ
ಉಳಿ-ದ-ವರು
ಉಳಿ-ದ-ವರೆಲ್ಲ
ಉಳಿ-ದ-ವರೆ-ಲ್ಲ-ರಿ-ಗಿಂತ
ಉಳಿ-ದ-ವರೆ-ಲ್ಲ-ರಿಗೂ
ಉಳಿ-ದ-ವ-ರೊ-ಬ್ಬರೂ
ಉಳಿದಿ
ಉಳಿ-ದಿತ್ತು
ಉಳಿ-ದಿದೆ
ಉಳಿ-ದಿ-ದೆಯೋ
ಉಳಿ-ದಿ-ದ್ದಾ-ರಲ್ಲ
ಉಳಿ-ದಿ-ರ-ಲಿಲ್ಲ
ಉಳಿ-ದಿ-ರು-ತ್ತದೆ
ಉಳಿ-ದಿ-ರುವ
ಉಳಿ-ದಿಲ್ಲ
ಉಳಿ-ದಿ-ಲ್ಲ-ವಲ್ಲ
ಉಳಿ-ದಿ-ಲ್ಲ-ವೆಂ-ಬಂತೆ
ಉಳಿದು
ಉಳಿ-ದು-ಕೊಂಡ
ಉಳಿ-ದು-ಕೊಂ-ಡ-ದ್ದ-ರಿಂದ
ಉಳಿ-ದು-ಕೊಂ-ಡದ್ದು
ಉಳಿ-ದು-ಕೊಂ-ಡರು
ಉಳಿ-ದು-ಕೊಂ-ಡರೆ
ಉಳಿ-ದು-ಕೊಂಡಿ
ಉಳಿ-ದು-ಕೊಂ-ಡಿದೆ
ಉಳಿ-ದು-ಕೊಂ-ಡಿದ್ದ
ಉಳಿ-ದು-ಕೊಂ-ಡಿ-ದ್ದರು
ಉಳಿ-ದು-ಕೊಂ-ಡಿ-ದ್ದ-ರೆಂದು
ಉಳಿ-ದು-ಕೊಂ-ಡಿದ್ದಾ
ಉಳಿ-ದು-ಕೊಂ-ಡಿ-ದ್ದಾರೆ
ಉಳಿ-ದು-ಕೊಂ-ಡಿದ್ದು
ಉಳಿ-ದು-ಕೊಂಡು
ಉಳಿ-ದು-ಕೊಂ-ಡುವು
ಉಳಿ-ದು-ಕೊ-ಳ್ಳ-ಬ-ಹು-ದಾ-ಗಿತ್ತು
ಉಳಿ-ದು-ಕೊ-ಳ್ಳ-ಬೇ-ಕಾ-ಗಿತ್ತು
ಉಳಿ-ದು-ಕೊ-ಳ್ಳ-ಬೇ-ಕಾ-ದರೆ
ಉಳಿ-ದು-ಕೊ-ಳ್ಳ-ಬೇ-ಕೆಂದು
ಉಳಿ-ದು-ಕೊ-ಳ್ಳ-ಲಾ-ರರು
ಉಳಿ-ದು-ಕೊ-ಳ್ಳ-ಲಾ-ರ-ರೆಂ-ಬು-ದರ
ಉಳಿ-ದು-ಕೊ-ಳ್ಳಲು
ಉಳಿ-ದು-ಕೊಳ್ಳು
ಉಳಿ-ದು-ಕೊ-ಳ್ಳು-ತ್ತಾರೆ
ಉಳಿ-ದು-ಕೊ-ಳ್ಳು-ತ್ತಾ-ರೆಂದು
ಉಳಿ-ದು-ಕೊ-ಳ್ಳು-ತ್ತಿ-ದ್ದರು
ಉಳಿ-ದು-ಕೊ-ಳ್ಳು-ತ್ತಿ-ದ್ದಳೋ
ಉಳಿ-ದು-ಕೊ-ಳ್ಳು-ತ್ತಿ-ದ್ದುದು
ಉಳಿ-ದು-ಕೊ-ಳ್ಳು-ವಂ-ತಹ
ಉಳಿ-ದು-ಕೊ-ಳ್ಳು-ವಂತೆ
ಉಳಿ-ದು-ಕೊ-ಳ್ಳು-ವು-ದರ
ಉಳಿ-ದು-ಕೊ-ಳ್ಳು-ವುದೊ
ಉಳಿ-ದು-ದೆಲ್ಲ
ಉಳಿ-ದು-ದೆ-ಲ್ಲವೂ
ಉಳಿ-ದು-ವೆಲ್ಲ
ಉಳಿ-ದೆಲ್ಲ
ಉಳಿ-ಯದೆ
ಉಳಿ-ಯ-ಬೇ-ಕಾ-ದರೆ
ಉಳಿ-ಯ-ಬೇ-ಕೆಂ-ದಿ-ದ್ದರೆ
ಉಳಿ-ಯ-ಲಿಲ್ಲ
ಉಳಿ-ಯಲು
ಉಳಿ-ಯಿತು
ಉಳಿ-ಯು-ತ್ತದೆ
ಉಳಿ-ಯು-ವು-ದ-ರಿಂದ
ಉಳಿ-ಯು-ವು-ದಿ-ಲ್ಲ-ವೆಂದು
ಉಳಿ-ಯು-ವುದೇ
ಉಳಿ-ವು-ಬೆ-ಳ-ವ-ಣಿಗೆ
ಉಳಿ-ಸಲು
ಉಳಿಸಿ
ಉಳಿ-ಸಿಕೊ
ಉಳಿ-ಸಿ-ಕೊಂ-ಡರು
ಉಳಿ-ಸಿ-ಕೊಂ-ಡಿದೆ
ಉಳಿ-ಸಿ-ಕೊಂಡು
ಉಳಿ-ಸಿ-ಕೊಂ-ಡು-ಬಂ-ದದ್ದು
ಉಳಿ-ಸಿ-ಕೊ-ಳ್ಳ-ಬೇ-ಕಾ-ಗು-ತ್ತದೆ
ಉಳಿ-ಸಿ-ಕೊ-ಳ್ಳಲು
ಉಳಿ-ಸಿ-ಕೊಳ್ಳು
ಉಳಿ-ಸಿ-ಕೊ-ಳ್ಳು-ವು-ದ-ಕ್ಕಾ-ದರೂ
ಉಳಿ-ಸಿದ
ಉಳಿ-ಸಿ-ದ-ವರು
ಉಳಿ-ಸಿ-ದ್ದರು
ಉಷ್ಣತೆ
ಉಷ್ಣ-ದಿಂ-ದಾ-ಗಿಯೋ
ಉಸಿ-ರಾಟ
ಉಸಿ-ರಾ-ಡದೆ
ಉಸಿ-ರಾ-ಡು-ತ್ತಿತ್ತು
ಉಸಿ-ರಾ-ಡು-ವು-ದನ್ನೂ
ಉಸಿರು
ಉಸಿ-ರು-ಬಿ-ಡುತ್ತ
ಉಸಿ-ರೆ-ತ್ತು-ವು-ದ-ರೊ-ಳ-ಗಾಗಿ
ಉಸಿರೇ
ಉಸು-ರು-ತ್ತಿ-ದ್ದುವು
ಉಹಿ-ಸಿ-ದ-ರು-ಪು-ರಾ-ತನ
ಉಹುಂ
ಊ
ಊಟ
ಊಟ-ವ-ಸ-ತಿಯ
ಊಟ-ಕ್ಕಂತೂ
ಊಟಕ್ಕೂ
ಊಟಕ್ಕೆ
ಊಟದ
ಊಟ-ಮಾಡ
ಊಟ-ಮಾಡಿ
ಊಟ-ಮಾ-ಡಿದ
ಊಟ-ಮಾ-ಡಿ-ಬಿ-ಡ-ಬಲ್ಲೆ
ಊಟ-ವನ್ನೇ
ಊಟ-ವಾದ
ಊಟ-ವಾ-ಯಿತು
ಊಟೋ-ಪ-ಚಾ-ರ-ಗಳ
ಊದಿ-ದರೆ
ಊದು-ತ್ತೀರಿ
ಊರ
ಊರಾ-ಚೆಯ
ಊರಾದ
ಊರಿಂ-ದೂ-ರಿಗೆ
ಊರಿಗೆ
ಊರಿನ
ಊರಿ-ನಲ್ಲಿ
ಊರಿ-ನ-ಲ್ಲಿಯೇ
ಊರಿ-ನ-ಲ್ಲಿ-ರ-ಲಿ-ಲ್ಲ-ವಾ-ದ್ದ-ರಿಂದ
ಊರಿ-ನಲ್ಲೂ
ಊರಿ-ನ-ಲ್ಲೆಲ್ಲ
ಊರಿ-ನಲ್ಲೇ
ಊರಿ-ನ-ವ-ನೊಬ್ಬ
ಊರಿ-ನಿಂದ
ಊರಿ-ನಿಂ-ದಲೂ
ಊರು
ಊರು-ಗಳ
ಊರು-ಗ-ಳ-ಲ್ಲೆಲ್ಲ
ಊರು-ಗಳಿಂದ
ಊರು-ಗ-ಳಿಗೆ
ಊರು-ಗೋ-ಲನ್ನು
ಊರು-ಗೋ-ಲಿನ
ಊರುತ್ತ
ಊರ್ಜಿ-ತ-ಗೊ-ಳಿ-ಸು-ವಂ-ತಾ-ಗು-ತ್ತದೆ
ಊಹಾ
ಊಹಾ-ಪೋ-ಹದ
ಊಹಾ-ಪೋ-ಹ-ವಿ-ದ್ದರೂ
ಊಹಿ-ಸ-ಬ-ಲ್ಲೆಯಾ
ಊಹಿ-ಸ-ಬ-ಹು-ದಾ-ಗಿದೆ
ಊಹಿ-ಸ-ಬ-ಹುದು
ಊಹಿ-ಸಲೂ
ಊಹಿಸಿ
ಊಹಿ-ಸಿಕೊ
ಊಹಿ-ಸಿ-ಕೊ-ಳ್ಳ-ಬ-ಹುದು
ಊಹಿ-ಸಿದ
ಊಹಿ-ಸಿ-ದರು
ಊಹಿ-ಸಿ-ದಳು
ಊಹಿ-ಸಿ-ದೆವು
ಊಹಿ-ಸಿ-ದ್ದರು
ಊಹಿ-ಸಿ-ದ್ದಳು
ಊಹಿ-ಸಿಯೂ
ಊಹಿ-ಸಿ-ರ-ದಿದ್ದ
ಊಹಿ-ಸಿ-ರ-ದಿ-ದ್ದಷ್ಟು
ಊಹಿ-ಸಿ-ರ-ಲಿಲ್ಲ
ಊಹಿ-ಸು-ತ್ತಾನೆ
ಊಹಿ-ಸು-ತ್ತಾರೆ
ಊಹಿ-ಸು-ತ್ತಿತ್ತು
ಊಹಿ-ಸು-ವುದು
ಊಹೆ
ಋಜು
ಋಣ
ಎ
ಎಂ
ಎಂಜಿ-ನಿ-ಯ-ರಿಗೆ
ಎಂಜಿ-ನಿ-ಯರ್
ಎಂಟ-ನೆಯ
ಎಂಟಾ-ದರೂ
ಎಂಟು
ಎಂಟೂ-ವರೆ
ಎಂಟೋ
ಎಂಡಿನ
ಎಂತಹ
ಎಂತ-ಹ-ದೆಂ-ದರೆ
ಎಂತ-ಹ-ದೆ-ನ್ನು-ವುದು
ಎಂತ-ಹದೇ
ಎಂತು
ಎಂತೆಂ-ತಹ
ಎಂತೆಂಥಾ
ಎಂಥ
ಎಂಥದು
ಎಂಥ-ವನೂ
ಎಂಥ-ವ-ರನ್ನೂ
ಎಂಥ-ವ-ರಾ-ದರೂ
ಎಂಥ-ವ-ರಿ-ಗಾ-ದರೂ
ಎಂಥ-ವ-ರಿಗೂ
ಎಂಥ-ವರು
ಎಂಥ-ವ-ರೆಂ-ದರೆ
ಎಂಥವು
ಎಂಥ-ವೆಂ-ಬು-ದನ್ನು
ಎಂಥಾ
ಎಂದ
ಎಂದಂ-ತಾ-ಯಿ-ತಲ್ಲ
ಎಂದ-ಮೇಲೆ
ಎಂದ-ರಂತೂ
ಎಂದ-ರಿತ
ಎಂದ-ರಿ-ತಿ-ರುವ
ಎಂದರು
ಎಂದರೆ
ಎಂದ-ರೇನು
ಎಂದ-ರೇ-ನೆಂದು
ಎಂದರ್ಥ
ಎಂದ-ರ್ಥ-ವ-ಲ್ಲವೆ
ಎಂದಲ್ಲ
ಎಂದಳು
ಎಂದ-ವರು
ಎಂದಷ್ಟೇ
ಎಂದಾಗ
ಎಂದಿ-ಗಾ-ದರೂ
ಎಂದಿ-ಗಿಂ-ತಲೂ
ಎಂದಿಗೂ
ಎಂದಿಗೆ
ಎಂದಿ-ಟ್ಟು-ಕೊಳ್ಳಿ
ಎಂದಿ-ದ್ದರೆ
ಎಂದಿ-ದ್ದಾರೆ
ಎಂದಿ-ನಂತೆ
ಎಂದಿ-ನಿಂ-ದಲೂ
ಎಂದಿ-ರಲ್ಲ
ಎಂದು
ಎಂದು-ಕೊ-ಳ್ಳು-ತ್ತೇ-ನೆ-ಬಿ-ಸಿ-ರ-ಕ್ತದ
ಎಂದು-ತ್ತ-ರಿ-ಸಿ-ದರು
ಎಂದು-ದ್ಗ-ರಿ-ಸಿ-ದರು
ಎಂದು-ಬಿಟ್ಟ
ಎಂದು-ಹಾಗೆ
ಎಂದು-ಹೇಳಿ
ಎಂದೂ
ಎಂದೆಂ-ದಿಗೂ
ಎಂದೆಂದೂ
ಎಂದೆ-ನ್ನ-ದಿರಿ
ಎಂದೆಲ್ಲ
ಎಂದೇ
ಎಂದೇ-ನಾ-ದರೂ
ಎಂದೇ-ನಿಲ್ಲ
ಎಂದೋ
ಎಂಪ್ರೆಸ್
ಎಂಬ
ಎಂಬಂ-ತಹ
ಎಂಬಂ-ತಿತ್ತು
ಎಂಬಂ-ತಿದೆ
ಎಂಬಂ-ತಿ-ದ್ದು-ಬಿ-ಟ್ಟರು
ಎಂಬಂತೆ
ಎಂಬತ್ತು
ಎಂಬ-ರ್ಥದ
ಎಂಬಲ್ಲಿ
ಎಂಬ-ಲ್ಲಿಗೆ
ಎಂಬ-ಲ್ಲಿದ್ದ
ಎಂಬ-ಲ್ಲಿನ
ಎಂಬ-ವನ
ಎಂಬ-ವ-ನಿಂದ
ಎಂಬ-ವ-ನಿಗೆ
ಎಂಬ-ವನು
ಎಂಬ-ವನೂ
ಎಂಬ-ವ-ನೊಂ-ದಿಗೆ
ಎಂಬ-ವ-ನೊಬ್ಬ
ಎಂಬ-ವರ
ಎಂಬ-ವ-ರನ್ನು
ಎಂಬ-ವ-ರಿಗೆ
ಎಂಬ-ವರು
ಎಂಬ-ವ-ರೊ-ಬ್ಬರು
ಎಂಬ-ವಳ
ಎಂಬ-ವ-ಳಿಂದ
ಎಂಬ-ವ-ಳಿಗೆ
ಎಂಬ-ವಳು
ಎಂಬಷ್ಟು
ಎಂಬಷ್ಟೇ
ಎಂಬ-ಹುದು
ಎಂಬಾಕೆ
ಎಂಬಾತ
ಎಂಬಾ-ತನ
ಎಂಬಿ-ತ್ಯಾ-ದಿ-ಯಾಗಿ
ಎಂಬಿ-ಬ್ಬರು
ಎಂಬೀ
ಎಂಬು
ಎಂಬುದ
ಎಂಬು-ದ-ಕ್ಕಿಂತ
ಎಂಬು-ದಕ್ಕೆ
ಎಂಬು-ದ-ನನು
ಎಂಬು-ದ-ನ್ನ-ರಿತ
ಎಂಬು-ದ-ನ್ನ-ವರು
ಎಂಬು-ದನ್ನು
ಎಂಬು-ದನ್ನೂ
ಎಂಬು-ದ-ನ್ನೆಲ್ಲ
ಎಂಬು-ದನ್ನೇ
ಎಂಬು-ದರ
ಎಂಬು-ದ-ರಲ್ಲಿ
ಎಂಬು-ದಷ್ಟೇ
ಎಂಬು-ದಿದೆ
ಎಂಬುದು
ಎಂಬುದೂ
ಎಂಬು-ದೆಲ್ಲ
ಎಂಬುದೇ
ಎಂಬು-ದೇ-ನಾ-ದರೂ
ಎಂಬು-ದೊಂ-ದಿದೆ
ಎಂಬು-ದೊಂದು
ಎಂಬುವ
ಎಂಬು-ವ-ನನ್ನು
ಎಂಬು-ವ-ನೊಬ್ಬ
ಎಂಬು-ವರು
ಎಂಬೊಬ್ಬ
ಎಕ್ಸ್
ಎಗ್ಬ-ರ್ಟ್
ಎಗ್ಸಿ-ಕ್ಯು-ಟಿವ್
ಎಚ್ಚ-ತ್ತಿ-ರ-ಬೇ-ಕಾ-ಗು-ತ್ತದೆ
ಎಚ್ಚ-ತ್ತಿ-ರು-ವಂತೆ
ಎಚ್ಚತ್ತು
ಎಚ್ಚ-ತ್ತು-ಕೊಳ್ಳು
ಎಚ್ಚ-ತ್ತು-ಕೊ-ಳ್ಳು-ವಂ-ತಾ-ಗು-ತ್ತದೆ
ಎಚ್ಚರ
ಎಚ್ಚ-ರ-ಗೊಂ-ಡರು
ಎಚ್ಚ-ರ-ಗೊಂ-ಡಿತು
ಎಚ್ಚ-ರ-ಗೊಂಡು
ಎಚ್ಚ-ರ-ಗೊ-ಳಿಸಿ
ಎಚ್ಚ-ರ-ಗೊ-ಳಿ-ಸಿ-ದರೋ
ಎಚ್ಚ-ರ-ಗೊ-ಳಿ-ಸು-ತ್ತಿ-ದ್ದಾನೆ
ಎಚ್ಚ-ರ-ಗೊಳ್ಳಿ
ಎಚ್ಚ-ರ-ಗೊ-ಳ್ಳು-ತ್ತಿ-ದ್ದೇನೆ
ಎಚ್ಚ-ರ-ದಿಂದ
ಎಚ್ಚ-ರ-ದಿಂ-ದಿ-ದ್ದರು
ಎಚ್ಚ-ರ-ದಿಂ-ದಿ-ರ-ಬೇಕು
ಎಚ್ಚ-ರ-ದಿಂ-ದಿ-ರ-ಬೇ-ಕೆಂದು
ಎಚ್ಚ-ರ-ವಾ-ಗು-ತ್ತಿ-ರ-ಲಿಲ್ಲ
ಎಚ್ಚ-ರ-ವಾ-ಯಿತು
ಎಚ್ಚ-ರಿಕೆ
ಎಚ್ಚ-ರಿ-ಕೆಯ
ಎಚ್ಚ-ರಿ-ಕೆ-ಯನ್ನು
ಎಚ್ಚ-ರಿ-ಕೆ-ಯನ್ನೂ
ಎಚ್ಚ-ರಿ-ಕೆ-ಯಿಂದ
ಎಚ್ಚ-ರಿ-ಕೆ-ಯಿಂ-ದಲೇ
ಎಚ್ಚ-ರಿ-ಕೆ-ಯಿಂ-ದಿ-ರ-ಬೇ-ಕಾ-ಗಿತ್ತು
ಎಚ್ಚ-ರಿ-ಕೆ-ಯಿಂ-ದಿ-ರು-ವಂತೆ
ಎಚ್ಚ-ರಿ-ಕೆ-ಯಿ-ರಲಿ
ಎಚ್ಚ-ರಿ-ಕೆ-ಯೆಂ-ದರೆ
ಎಚ್ಚ-ರಿ-ಸ-ಲಾ-ಗಿತ್ತು
ಎಚ್ಚ-ರಿ-ಸಿ-ದರು
ಎಚ್ಚೆತ್ತು
ಎಟು-ಕ-ಲಾರ
ಎಟು-ಕು-ವಂ-ತಹ
ಎಟು-ಕು-ವಂ-ತಿ-ರ-ಲಿಲ್ಲ
ಎಡ-ಗಡೆ
ಎಡ-ಗೈ-ಯನ್ನು
ಎಡ-ಗೈ-ಯಲ್ಲಿ
ಎಡ-ತಾಕಿ
ಎಡ-ದಿಂದ
ಎಡ-ವು-ವಂ-ತಾ-ಗ-ಬಾ-ರ-ದೆಂದು
ಎಡೆ
ಎಡೆ-ಬಿ-ಡದ
ಎಡೆ-ಬಿ-ಡದೆ
ಎಡೆ-ಮಾಡಿ
ಎಡೆ-ಯಿ-ದ್ದರೂ
ಎಡೆಯೇ
ಎಡ್ಗರ್
ಎಡ್ವ-ರ್ಡ್
ಎಣಿ-ಕೆ-ಯಾ-ಗಿತ್ತು
ಎಣಿಸಿ
ಎಣಿ-ಸು-ವುದು
ಎಣ್ಣೆ
ಎಣ್ಣೆ-ಯನ್ನು
ಎತ್ತ
ಎತ್ತ-ಬೇಕು
ಎತ್ತ-ಬೇಕೆ
ಎತ್ತರ
ಎತ್ತ-ರ-ಕ್ಕೆ-ತ್ತಿದೆ
ಎತ್ತ-ರದ
ಎತ್ತ-ರ-ದಲ್ಲಿ
ಎತ್ತ-ರ-ದ-ಲ್ಲಿ-ದೆ-ಯೆಂದೂ
ಎತ್ತ-ರ-ದಿಂದ
ಎತ್ತ-ರ-ವಿ-ದ್ದಾರೆ
ಎತ್ತಲು
ಎತ್ತಲೋ
ಎತ್ತಿ
ಎತ್ತಿ-ಕೊಂಡು
ಎತ್ತಿ-ಟ್ಟು-ಕೊಂ-ಡರು
ಎತ್ತಿ-ತೋ-ರಿತು
ಎತ್ತಿ-ತೋ-ರಿಸಿ
ಎತ್ತಿ-ತೋ-ರಿ-ಸಿದೆ
ಎತ್ತಿ-ತೋ-ರಿ-ಸು-ತ್ತಿ-ದ್ದರು
ಎತ್ತಿ-ತೋ-ರಿ-ಸು-ವಂ-ತಿತ್ತು
ಎತ್ತಿ-ತೋ-ರಿ-ಸು-ವುದೂ
ಎತ್ತಿದ
ಎತ್ತಿ-ದರು
ಎತ್ತಿ-ದಾಗ
ಎತ್ತಿನ
ಎತ್ತಿ-ನ-ಗಾ-ಡಿ-ಯಲ್ಲಿ
ಎತ್ತಿ-ನ-ಗಾ-ಡಿ-ಯಲ್ಲೇ
ಎತ್ತಿ-ಹಿ-ಡಿದ
ಎತ್ತಿ-ಹಿ-ಡಿ-ದಿತ್ತು
ಎತ್ತಿ-ಹಿ-ಡಿ-ದಿ-ರು-ವುದು
ಎತ್ತಿ-ಹಿ-ಡಿದು
ಎತ್ತಿ-ಹಿ-ಡಿ-ಯಲು
ಎತ್ತಿ-ಹಿ-ಡಿ-ಯು-ತ್ತಿ-ದ್ದರು
ಎತ್ತಿ-ಹಿ-ಡಿ-ಯುವ
ಎತ್ತಿ-ಹಿ-ಡಿ-ಯು-ವಂ-ತಿದೆ
ಎತ್ತಿ-ಹಿ-ಡು-ವಂ-ಥ-ದಾ-ಗಿ-ದ್ದರೂ
ಎತ್ತು
ಎಥಿ-ಕಲ
ಎಥಿ-ಕಲ್
ಎದು-ರಾಗಿ
ಎದು-ರಾ-ಗಿತ್ತು
ಎದು-ರಾ-ಗಿದ್ದ
ಎದು-ರಾ-ಗಿಯೇ
ಎದು-ರಾ-ಗಿ-ರ-ಲಿಲ್ಲ
ಎದು-ರಾ-ಗುವ
ಎದು-ರಾದ
ಎದು-ರಾ-ದದ್ದು
ಎದು-ರಾ-ದರೂ
ಎದು-ರಾ-ದಾಗ
ಎದು-ರಾ-ಯಿತು
ಎದು-ರಾ-ಳಿ-ಗ-ಳಾಗಿ
ಎದು-ರಿ-ಗಿ-ದ್ದ-ವ-ರನ್ನು
ಎದು-ರಿ-ಗಿ-ದ್ದ-ವರು
ಎದು-ರಿ-ಗಿ-ರುವ
ಎದು-ರಿ-ಗಿ-ಲ್ಲ-ದಿ-ರು-ವಾಗ
ಎದು-ರಿಗೇ
ಎದು-ರಿ-ನಲ್ಲಿ
ಎದು-ರಿ-ನ-ಲ್ಲಿದ್ದ
ಎದು-ರಿ-ನಿಂದ
ಎದು-ರಿಸ
ಎದು-ರಿ-ಸ-ಬಲ್ಲ
ಎದು-ರಿ-ಸ-ಬ-ಲ್ಲದು
ಎದು-ರಿ-ಸ-ಬೇ-ಕಾ-ಗಿದೆ
ಎದು-ರಿ-ಸ-ಬೇ-ಕಾ-ಗು-ತ್ತದೆ
ಎದು-ರಿ-ಸ-ಬೇ-ಕಾ-ಗು-ತ್ತಿತ್ತು
ಎದು-ರಿ-ಸ-ಬೇ-ಕಾದ
ಎದು-ರಿ-ಸ-ಬೇ-ಕಾ-ಯಿತು
ಎದು-ರಿ-ಸಲು
ಎದು-ರಿಸಿ
ಎದು-ರಿ-ಸಿದ
ಎದು-ರಿ-ಸಿ-ದರು
ಎದು-ರಿ-ಸಿದ್ದು
ಎದು-ರಿಸು
ಎದು-ರಿ-ಸುತ್ತ
ಎದು-ರು-ಗೊ-ಳ್ಳಲು
ಎದು-ರು-ತ್ತರ
ಎದು-ರು-ಹಾ-ಕಿ-ಕೊ-ಳ್ಳ-ದಿ-ರು-ವುದು
ಎದು-ರು-ಹಾ-ಕಿ-ಕೊ-ಳ್ಳ-ಬಾ-ರದು
ಎದು-ರು-ಹಾ-ಕಿ-ಕೊ-ಳ್ಳುವ
ಎದೆ
ಎದೆ-ಕೆಚ್ಚೆ
ಎದೆ-ಗಾ-ರಿ-ಕೆಯ
ಎದೆ-ಗಾ-ರಿ-ಕೆ-ಯನ್ನು
ಎದೆ-ಗಾ-ರಿ-ಕೆ-ಯನ್ನೂ
ಎದೆ-ಗಾ-ರಿ-ಕೆ-ಯು-ಳ್ಳ-ವರು
ಎದೆ-ಗೆಚ್ಚು
ಎದೆಯ
ಎದೆ-ಯನ್ನು
ಎದೆ-ಯಲ್ಲಿ
ಎದೆ-ಯಲ್ಲೂ
ಎದ್ದ
ಎದ್ದಾಗ
ಎದ್ದಿತು
ಎದ್ದು
ಎದ್ದು-ಕಾ-ಣು-ತ್ತದೆ
ಎದ್ದು-ಕಾ-ಣು-ತ್ತಿತ್ತು
ಎದ್ದು-ಕಾ-ಣು-ತ್ತಿದ್ದ
ಎದ್ದು-ಕಾ-ಣು-ತ್ತಿ-ರುವ
ಎದ್ದು-ಕಾ-ಣುವ
ಎದ್ದು-ತೋ-ರುವ
ಎದ್ದು-ನಿಂತ
ಎದ್ದು-ನಿಂ-ತರು
ಎದ್ದು-ನಿಂ-ತಾಗ
ಎದ್ದು-ನಿಂತು
ಎದ್ದು-ನಿ-ಲ್ಲು-ವಾಗ
ಎದ್ದು-ಬಂದ
ಎದ್ದು-ಬಂ-ದಾಗ
ಎದ್ದು-ಬಿ-ಡು-ತ್ತದೆ
ಎದ್ದು-ಬಿದ್ದು
ಎದ್ದು-ಹೋಗಿ
ಎದ್ದೇ-ಳಲೂ
ಎದ್ದೇಳಿ
ಎದ್ದೇಳು
ಎದ್ದೇ-ಳುವ
ಎದ್ದೊ-ಡನೆ
ಎನಿ-ಸು-ತ್ತದೆ
ಎನ್
ಎನ್ನ
ಎನ್ನ-ಬಹು
ಎನ್ನ-ಬ-ಹು-ದಾ-ಗಿತ್ತು
ಎನ್ನ-ಬ-ಹುದು
ಎನ್ನ-ಬೇ-ಕಾ-ದರೆ
ಎನ್ನ-ಬೇಕು
ಎನ್ನ-ಲಾಗಿದೆ
ಎನ್ನ-ಲಾ-ಗು-ವು-ದಿಲ್ಲ
ಎನ್ನ-ಲಿಲ್ಲ
ಎನ್ನಲು
ಎನ್ನಿ-ಸಿ-ಕೊಂ-ಡ-ವರ
ಎನ್ನಿ-ಸಿ-ಕೊ-ಳ್ಳ-ಲಾ-ರದು
ಎನ್ನಿ-ಸಿ-ಕೊ-ಳ್ಳು-ವಂ-ಥ-ದೇ-ನಾ-ದರೂ
ಎನ್ನಿ-ಸಿತು
ಎನ್ನಿ-ಸು-ತ್ತದೆ
ಎನ್ನಿ-ಸು-ತ್ತಿದೆ
ಎನ್ನು
ಎನ್ನುತ್ತ
ಎನ್ನು-ತ್ತದೆ
ಎನ್ನು-ತ್ತವೆ
ಎನ್ನು-ತ್ತಾರೆ
ಎನ್ನು-ತ್ತಿ-ದ್ದಂತೆ
ಎನ್ನು-ತ್ತಿ-ದ್ದರು
ಎನ್ನು-ತ್ತಿ-ದ್ದಾನೆ
ಎನ್ನು-ತ್ತಿ-ದ್ದಾರೆ
ಎನ್ನು-ತ್ತಿ-ದ್ದೀ-ರಲ್ಲ
ಎನ್ನು-ತ್ತೀಯೋ
ಎನ್ನುವ
ಎನ್ನು-ವಂ-ತ-ಹದು
ಎನ್ನು-ವಂ-ತೆಯೂ
ಎನ್ನು-ವ-ಷ್ಟ-ರಲ್ಲಿ
ಎನ್ನು-ವಾಗ
ಎನ್ನುವು
ಎನ್ನು-ವು-ದ-ಕ್ಕಿಂತ
ಎನ್ನು-ವು-ದಕ್ಕೆ
ಎನ್ನು-ವುದನ್ನು
ಎನ್ನು-ವು-ದ-ನ್ನೆಲ್ಲ
ಎನ್ನು-ವು-ದನ್ನೇ
ಎನ್ನು-ವು-ದಾ-ದರೆ
ಎನ್ನು-ವು-ದಿಲ್ಲ
ಎನ್ನು-ವುದು
ಎನ್ನು-ವುದೇ
ಎನ್ನು-ವು-ದೇನೋ
ಎಪ್ಪತ್ತು
ಎಬ್ಬಿ-ಸ-ಬೇಕು
ಎಬ್ಬಿ-ಸಲು
ಎಬ್ಬಿ-ಸಿ-ದು-ವೆಂದು
ಎಬ್ಬಿ-ಸಿ-ಬಿ-ಟ್ಟಿ-ದ್ದುವು
ಎಬ್ಬಿ-ಸೋಣ
ಎಮ-ರ್ಸ-ನ್ನ-ನಿಂದ
ಎಮಿಲಿ
ಎಮಿ-ಲಿ-ಯ-ವರ
ಎಮೆ-ಲಿನ್
ಎಮ್ಮಾ
ಎಮ್ಮಾ-ಕಾಲ್ವೆ
ಎಮ್ಮಾ-ಥರ್ಸ್ಬಿ
ಎರ-ಗಿ-ದರು
ಎರ-ಗು-ತ್ತಿ-ದ್ದರು
ಎರ-ಡನೆ
ಎರ-ಡ-ನೆಯ
ಎರ-ಡ-ನೆ-ಯ-ದಾಗಿ
ಎರ-ಡ-ನೆ-ಯದು
ಎರ-ಡನೇ
ಎರ-ಡನ್ನೂ
ಎರ-ಡ-ರ-ಷ್ಟಾ-ದರೂ
ಎರ-ಡ-ರಷ್ಟು
ಎರ-ಡಲ್ಲ
ಎರ-ಡಿಲ್ಲ
ಎರಡು
ಎರ-ಡು
ಎರ-ಡು-ಮೂರು
ಎರ-ಡು-ವಾರ
ಎರ-ಡು-ಹೊ-ತ್ತಿನ
ಎರಡೂ
ಎರ-ಡೂ-ವರೆ
ಎರ-ಡೆ-ರಡು
ಎರಡೇ
ಎರಿ-ಕಾಳ
ಎರಿಕ್
ಎರ್ನಾ-ಕು-ಲಂನ
ಎರ್ನಾ-ಕು-ಲಂ-ನಲ್ಲಿ
ಎರ್ನಾ-ಕು-ಲಂ-ನ-ಲ್ಲಿ-ದ್ದರು
ಎರ್ನಾ-ಕು-ಲಂ-ನಿಂದ
ಎಲಿ-ಯಟ್
ಎಲಿ-ವೇ-ಟರ್ನ
ಎಲೆ-ಕ್ಟ್ರಿಕ್
ಎಲೆನ್
ಎಲ್ಲ
ಎಲ್ಲ-ಕ್ಕಿಂತ
ಎಲ್ಲ-ಕ್ಕಿತ
ಎಲ್ಲಕ್ಕೂ
ಎಲ್ಲ-ಕ್ಕೂ-ಶು-ಭ-ಕೋರ
ಎಲ್ಲ-ಡೆ-ಯಿಂ-ದಲೂ
ಎಲ್ಲ-ದಕ್ಕೂ
ಎಲ್ಲ-ದ-ಕ್ಕೂ-ಹ-ಸಿವು
ಎಲ್ಲ-ದರ
ಎಲ್ಲ-ದ-ರಲ್ಲೂ
ಎಲ್ಲ-ದ-ರ-ಲ್ಲೂ-ಧರ್ಮ
ಎಲ್ಲ-ದ-ರೊ-ಳ-ಗೊಂ-ದಾಗಿ
ಎಲ್ಲರ
ಎಲ್ಲ-ರನ್ನೂ
ಎಲ್ಲ-ರಲ್ಲೂ
ಎಲ್ಲ-ರಿಂ-ದಲೂ
ಎಲ್ಲ-ರಿ-ಗಿಂತ
ಎಲ್ಲ-ರಿಗೂ
ಎಲ್ಲರೂ
ಎಲ್ಲ-ರೊಂ-ದಿಗೂ
ಎಲ್ಲ-ವನ್ನೂ
ಎಲ್ಲವೂ
ಎಲ್ಲಾ
ಎಲ್ಲಾ-ದರೂ
ಎಲ್ಲಿ
ಎಲ್ಲಿಂದ
ಎಲ್ಲಿಂ-ದ-ಲಾ-ದರೂ
ಎಲ್ಲಿಂ-ದಲೋ
ಎಲ್ಲಿ-ಗಾ-ದರೂ
ಎಲ್ಲಿಗೂ
ಎಲ್ಲಿಗೆ
ಎಲ್ಲಿಗೇ
ಎಲ್ಲಿದೆ
ಎಲ್ಲಿ-ದೆ-ಯೆಂದೂ
ಎಲ್ಲಿ-ದ್ದೇನೆ
ಎಲ್ಲಿಯ
ಎಲ್ಲಿ-ಯ-ವ-ರೆಗೂ
ಎಲ್ಲಿ-ಯ-ವ-ರೆಗೆ
ಎಲ್ಲಿಯೂ
ಎಲ್ಲಿಯೋ
ಎಲ್ಲಿ-ರ-ಬ-ಹುದೋ
ಎಲ್ಲಿ-ಲ್ಲದ
ಎಲ್ಲಿಸ್
ಎಲ್ಲೂ
ಎಲ್ಲೆ-ಗ-ಳೊ-ಳಗೇ
ಎಲ್ಲೆಡೆ
ಎಲ್ಲೆ-ಡೆ-ಗಳಲ್ಲಿ
ಎಲ್ಲೆ-ಡೆ-ಗ-ಳಲ್ಲೂ
ಎಲ್ಲೆ-ಡೆ-ಗಳಿಂದ
ಎಲ್ಲೆ-ಡೆ-ಗ-ಳಿಂ-ದಲೂ
ಎಲ್ಲೆ-ಡೆಗೂ
ಎಲ್ಲೆ-ಡೆಗೆ
ಎಲ್ಲೆ-ಡೆ-ಯಿಂ-ದಲೂ
ಎಲ್ಲೆ-ಡೆಯೂ
ಎಲ್ಲೆ-ಯನ್ನು
ಎಲ್ಲೆಯೇ
ಎಲ್ಲೆಲ್ಲಿ
ಎಲ್ಲೆ-ಲ್ಲಿಗೋ
ಎಲ್ಲೆ-ಲ್ಲಿಯೂ
ಎಲ್ಲೆಲ್ಲೂ
ಎಲ್ಲೆಲ್ಲೋ
ಎಲ್ಲೇ
ಎಲ್ಲೋ
ಎಳಸು
ಎಳೆ
ಎಳೆ-ತಂ-ದದ್ದು
ಎಳೆದ
ಎಳೆದು
ಎಳೆ-ದು-ತಂ-ದಿತು
ಎಳೆ-ದು-ತ-ರಲು
ಎಳೆ-ದು-ತ-ರುವ
ಎಳೆ-ದೆ-ಳೆ-ದು-ಹಾ-ಕುತ್ತ
ಎಳೆ-ದೊ-ಯ್ಯು-ತ್ತವೆ
ಎಳೆ-ನಿಂ-ಬೆ-ಕಾಯಿ
ಎಳೆಯ
ಎಳೆ-ಯಲು
ಎಳೆ-ಯು-ತ್ತದೆ
ಎಳೆ-ಯು-ತ್ತಿತ್ತು
ಎಳ್ಳಷ್ಟೂ
ಎವ-ರೆಟ್
ಎವೆ-ಯಿ-ಕ್ಕದೆ
ಎಷ್ಟನ್ನು
ಎಷ್ಟರ
ಎಷ್ಟ-ರ-ಮ-ಟ್ಟಿಗೆ
ಎಷ್ಟಾ-ದರೂ
ಎಷ್ಟು
ಎಷ್ಟು-ಸಲ
ಎಷ್ಟೆಷ್ಟು
ಎಷ್ಟೆಷ್ಟೋ
ಎಷ್ಟೇ
ಎಷ್ಟೊಂದು
ಎಷ್ಟೋ
ಎಷ್ಟೋ-ಜನ
ಎಸ-ಗಿದ
ಎಸ-ಗಿ-ದ್ದಾನೆ
ಎಸೆ
ಎಸೆ-ದರೂ
ಎಸೆ-ದು-ಬಿ-ಟ್ಟಿ-ದ್ದಳು
ಎಸೆ-ದು-ಬಿ-ಡ-ಬಲ್ಲ
ಎಸೆ-ದು-ಬಿಡಿ
ಎಸೆ-ದು-ಬಿ-ಡು-ವಂತೆ
ಎಸ್
ಏಂಜ-ಲಿ-ಸ್ನಲ್ಲಿ
ಏಂಜೆ-ಲಿ-ಸ್ನಲ್ಲಿ
ಏಕ
ಏಕ-ಕಂ-ಠ-ದಿಂದ
ಏಕ-ಕಾ-ಲಕ್ಕೆ
ಏಕ-ಕಾ-ಲ-ದಲ್ಲಿ
ಏಕತಾ
ಏಕತೆ
ಏಕ-ತೆ-ಗಳ
ಏಕ-ತೆ-ಯನ್ನು
ಏಕ-ತೆ-ಯಲ್ಲಿ
ಏಕ-ತೆಯು
ಏಕ-ತೆಯೇ
ಏಕ-ತೆ-ಯೊಂ-ದಿದೆ
ಏಕ-ತ್ವದ
ಏಕ-ಮಾತ್ರ
ಏಕ-ಮಾ-ತ್ರ-ವಾದ
ಏಕ-ಮು-ಖ-ವಾದ
ಏಕ-ಮೇ-ವಾ-ದ್ವಿ-ತೀ-ಯ-ವಾ-ದುದು
ಏಕ-ರೀ-ತಿಯ
ಏಕ-ರೀ-ತಿ-ಯಲ್ಲಿ
ಏಕ-ರೂ-ಪದ
ಏಕರ್
ಏಕ-ವಾಗಿ
ಏಕಾಂ-ಗಿ-ಯಾಗಿ
ಏಕಾಂ-ಗಿ-ಯಾ-ಗಿದ್ದು
ಏಕಾಂ-ಗಿ-ಯಾ-ಗಿಯೇ
ಏಕಾಂ-ಗಿ-ಯಾ-ಗಿರು
ಏಕಾಂ-ಗಿಯೂ
ಏಕಾಂ-ಗಿ-ವೀ-ರ-ನಾಗಿ
ಏಕಾಂತ
ಏಕಾಂ-ತ-ದಲ್ಲಿ
ಏಕಾಂ-ತ-ದ-ಲ್ಲಿ-ದ್ದು-ಕೊಂಡು
ಏಕಾ-ಕಿ-ಯಾಗಿ
ಏಕಾ-ಕಿ-ಯಾ-ಗಿಯೇ
ಏಕಾ-ಕಿ-ಯಾ-ದರು
ಏಕಾಗ್ರ
ಏಕಾ-ಗ್ರ-ಗೊ-ಳಿ-ಸಲು
ಏಕಾ-ಗ್ರ-ಗೊ-ಳಿ-ಸಿ-ದಾಗ
ಏಕಾ-ಗ್ರ-ಗೊ-ಳಿ-ಸು-ವುದು
ಏಕಾ-ಗ್ರ-ಚಿ-ತ್ತ-ದಿಂದ
ಏಕಾ-ಗ್ರ-ಚಿ-ತ್ತ-ರಾಗಿ
ಏಕಾ-ಗ್ರ-ತೆಯ
ಏಕಾ-ಗ್ರ-ತೆ-ಯನ್ನು
ಏಕಾ-ಗ್ರ-ತೆ-ಯಿಂದ
ಏಕಾ-ಗ್ರ-ಮ-ನ-ಸ್ಕ-ನಾಗಿ
ಏಕಾತ್ಮ
ಏಕಾ-ದರೂ
ಏಕೆ
ಏಕೆಂ
ಏಕೆಂ-ದರ
ಏಕೆಂ-ದರೆ
ಏಕೆಂದು
ಏಕೈಕ
ಏಕೋ
ಏಜೆಂ-ಟನ
ಏಟಿಗೆ
ಏಟು
ಏಟು-ಗ-ಳಿಗೆ
ಏತಕ್ಕೂ
ಏತಕ್ಕೆ
ಏನದು
ಏನ-ದ್ಭುತ
ಏನ-ನ್ನಾ-ದರೂ
ಏನ-ನ್ನಿ-ಸಿತು
ಏನ-ನ್ನಿ-ಸಿತೋ
ಏನನ್ನು
ಏನನ್ನೂ
ಏನನ್ನೋ
ಏನಪ್ಪ
ಏನಪ್ಪಾ
ಏನಯ್ಯ
ಏನ-ವರ
ಏನಾ-ಗ-ಬ-ಹುದು
ಏನಾ-ಗ-ಲಿ-ರು-ವುದೋ
ಏನಾಗಿ
ಏನಾ-ಗಿ-ದ್ದೇ-ನೆಯೋ
ಏನಾ-ಗಿ-ರು-ತ್ತೇ-ವೆಯೋ
ಏನಾ-ಗಿ-ರು-ವೆನೋ
ಏನಾ-ಗು-ತ್ತದೆ
ಏನಾ-ಗು-ತ್ತ-ದೆಯೋ
ಏನಾ-ಗು-ತ್ತಿ-ದೆ-ಯೆಂಬ
ಏನಾ-ಗು-ತ್ತಿ-ದೆ-ಯೆಂ-ಬು-ದನ್ನು
ಏನಾ-ದ-ರಾ-ಗಲಿ
ಏನಾ-ದರೂ
ಏನಾ-ದ-ರೊಂ-ದನ್ನು
ಏನಾ-ದ-ರೊಂದು
ಏನಾ-ದೀತು
ಏನಾ-ಯಿತು
ಏನಾ-ಯಿ-ತೆಂ-ದರೆ
ಏನಾ-ಯಿ-ತೆಂದು
ಏನಾ-ಯಿ-ತೆಂ-ಬು-ದನ್ನು
ಏನಾ-ಶ್ಚರ್ಯ
ಏನಿದು
ಏನಿದೆ
ಏನಿ-ದ್ದರೂ
ಏನಿಲ್ಲ
ಏನಿ-ಲ್ಲ-ವೆಂ-ದರೂ
ಏನು
ಏನೂ
ಏನೆಂ-ದರೆ
ಏನೆಂ-ದಿರಿ
ಏನೆಂದು
ಏನೆಂ-ಬುದು
ಏನೆ-ನ್ನಿ-ಸಿ-ರ-ಬ-ಹುದು
ಏನೆ-ನ್ನಿ-ಸು-ತ್ತದೆ
ಏನೆ-ನ್ನು-ತ್ತಾರೆ
ಏನೆ-ನ್ನು-ತ್ತಾರೋ
ಏನೇ
ಏನೇ-ನಿ-ದೆಯೋ
ಏನೇನು
ಏನೇನೂ
ಏನೇನೋ
ಏನೋ
ಏನ್ನನ್ನು
ಏಪ್ರಿಲ್
ಏಪ್ರಿ-ಲ್ನಲ್ಲಿ
ಏಪ್ರಿ-ಲ್ಲಿ-ನಲ್ಲಿ
ಏಯ್
ಏರ-ಬ-ಲ್ಲ-ವ-ರಾ-ಗಿ-ದ್ದರು
ಏರ-ಬೇ-ಕಾ-ದರೆ
ಏರಿದ
ಏರಿ-ಬಿ-ಟ್ಟರು
ಏರಿ-ಬಿ-ಟ್ಟಿ-ತ್ತೆಂ-ದರೂ
ಏರಿ-ಬಿಡ
ಏರಿ-ರ-ಲಿಲ್ಲ
ಏರಿ-ಳಿ-ತ-ಗಳಿಂದ
ಏರಿ-ಸ-ಬೇ-ಕೆಂಬ
ಏರಿ-ಸ-ಲ್ಪಟ್ಟ
ಏರಿ-ಸಿ-ದರು
ಏರಿ-ಹೋ-ಗಲು
ಏರುತ್ತ
ಏರು-ತ್ತದೆ
ಏರು-ತ್ತಲೇ
ಏರು-ತ್ತಾ-ರೆ-ಎ-ಲ್ಲವೂ
ಏರು-ಪೇ-ರಾ-ದಾಗ
ಏರು-ಪೇ-ರು-ಗಳನ್ನೆಲ್ಲ
ಏರು-ವಂ-ತಾ-ಗ-ಲೆಂದು
ಏರು-ವಂ-ತಾ-ದದ್ದು
ಏರೆ-ತ್ತ-ರದ
ಏರ್ಪಡಿ
ಏರ್ಪ-ಡಿ-ಸ-ಲಾ-ಗಿತ್ತು
ಏರ್ಪ-ಡಿ-ಸ-ಲಾ-ಗಿದ್ದ
ಏರ್ಪ-ಡಿ-ಸ-ಲಾದ
ಏರ್ಪ-ಡಿ-ಸ-ಲಾ-ಯಿತು
ಏರ್ಪ-ಡಿ-ಸಲು
ಏರ್ಪ-ಡಿಸಿ
ಏರ್ಪ-ಡಿ-ಸಿದ
ಏರ್ಪ-ಡಿ-ಸಿ-ದರು
ಏರ್ಪ-ಡಿ-ಸಿದ್ದ
ಏರ್ಪ-ಡಿ-ಸಿ-ದ್ದರು
ಏರ್ಪ-ಡಿ-ಸುವ
ಏರ್ಪ-ಡಿ-ಸು-ವುದು
ಏರ್ಪ-ಡು-ತ್ತದೆ
ಏರ್ಪ-ಡು-ತ್ತಿತ್ತು
ಏರ್ಪ-ಡು-ವಂತೆ
ಏರ್ಪಾ-ಟಾ-ಗು-ತ್ತಿ-ರು-ವುದು
ಏರ್ಪಾಡಾ
ಏರ್ಪಾ-ಡಾ-ಗಿತ್ತು
ಏರ್ಪಾ-ಡಾ-ಗಿದ್ದು
ಏರ್ಪಾ-ಡಾ-ಗಿ-ದ್ದುದು
ಏರ್ಪಾ-ಡಾ-ದುವು
ಏರ್ಪಾಡು
ಏಳನೇ
ಏಳರ
ಏಳ-ರಿಂದ
ಏಳಲು
ಏಳಿ
ಏಳಿ-ಗೆ-ಗಾಗಿ
ಏಳಿ-ಗೆಗೆ
ಏಳಿ-ಗೆ-ಯನ್ನೇ
ಏಳು
ಏಳುವ
ಏಳು-ವಂ-ತೆಯೇ
ಏಳು-ವು-ದಿಲ್ಲ
ಏಳು-ಸಾ-ವಿರ
ಏಳೆಂಟು
ಏಳ್ಗೆ-ಗಾಗಿ
ಏಳ್ಗೆಯ
ಏಷಿ-ಯಾ-ಟಿಕ್
ಏಷಿ-ಯಾದ
ಏಷಿ-ಯಾ-ದಲ್ಲೂ
ಏಷ್ಯ-ನ್ನರ
ಏಸು-ಕ್ರಿಸ್ತ
ಏಸು-ಕ್ರಿ-ಸ್ತನ
ಏಸು-ಕ್ರಿ-ಸ್ತ-ನಂತೆ
ಏಸು-ಕ್ರಿ-ಸ್ತ-ನಿಗೂ
ಏಸು-ಕ್ರಿ-ಸ್ತ-ನಿಗೆ
ಏಸು-ಕ್ರಿ-ಸ್ತನು
ಏಸು-ವನ್ನು
ಐ
ಐಂದ್ರ-ಜಾ-ಲಿಕ
ಐತಿ
ಐತಿ-ಹಾ-ಸಿಕ
ಐತಿ-ಹಾ-ಸಿ-ಕವೂ
ಐದ-ನೆಯ
ಐದಾರು
ಐದು
ಐನೂ-ರಕ್ಕೂ
ಐನೂರು
ಐರಾ-ವತ
ಐರೋಪ್ಯ
ಐರೋ-ಪ್ಯನೂ
ಐರೋ-ಪ್ಯರ
ಐರೋ-ಪ್ಯರು
ಐರೋ-ಪ್ಯರೂ
ಐರೋ-ಪ್ಯರೇ
ಐರ್ಲೆಂ-ಡಿ-ನಲ್ಲಿ
ಐವತ್ತು
ಐವತ್ತೇ
ಐವರು
ಐಶ್ವರ್ಯ
ಐಶ್ವ-ರ್ಯ-ವನ್ನು
ಐಹಿಕ
ಒಂಟಿ-ಯಾಗಿ
ಒಂಟಿ-ಯಾ-ಗಿ-ರಲು
ಒಂಟೆ
ಒಂಟೆಯ
ಒಂದಂಶ
ಒಂದಂ-ಶ-ವಷ್ಟೆ
ಒಂದ-ಕ್ಕಿಂತ
ಒಂದ-ಕ್ಷ-ರ-ವಾ-ದರೂ
ಒಂದ-ಕ್ಷ-ರವೂ
ಒಂದ-ನೊಂ-ದ-ನ್ನಪ್ಪಿ
ಒಂದನ್ನು
ಒಂದರ
ಒಂದ-ರಲ್ಲಿ
ಒಂದಲ್ಲ
ಒಂದಾ-ಗಲು
ಒಂದಾ-ಗ-ಲೆ-ಣಿಸಿ
ಒಂದಾಗಿ
ಒಂದಾ-ಗಿತ್ತು
ಒಂದಾ-ಗಿ-ದ್ದೇನೆ
ಒಂದಾ-ಗಿ-ರು-ವ-ವನು
ಒಂದಾ-ಗಿ-ರು-ವ-ವರು
ಒಂದಾ-ಗುವ
ಒಂದಾದ
ಒಂದಾ-ದರೆ
ಒಂದಾ-ನೊಂದು
ಒಂದಾರು
ಒಂದಿ-ನಿತೂ
ಒಂದಿ-ಪ್ಪತ್ತು
ಒಂದಿಷ್ಟು
ಒಂದಿಷ್ಟೂ
ಒಂದು
ಒಂದು-ಒಂ-ದೂ-ವರೆ
ಒಂದು-ಅ-ವರು
ಒಂದು-ಕಡೆ
ಒಂದು-ಗೂಡಿ
ಒಂದು-ಗೂ-ಡಿ-ಸುವ
ಒಂದು-ಗೂ-ಡಿ-ಸು-ವು-ದಾ-ಗಿತ್ತು
ಒಂದು-ಗೂ-ಡಿ-ಸೋಣ
ಒಂದು-ವೇಳೆ
ಒಂದು-ಸಾ-ವಿರ
ಒಂದೂ-ಕಾಲು
ಒಂದೂ-ರಲ್ಲಿ
ಒಂದೂ-ವರೆ
ಒಂದೆಂ-ದರೆ
ಒಂದೆಂಬ
ಒಂದೆಡೆ
ಒಂದೆ-ರಡು
ಒಂದೇ
ಒಂದೇ-ಮೊ-ದಲು
ಒಂದೇ-ಮೌನ
ಒಂದೇ-ಸ-ಮನೆ
ಒಂದೈದು
ಒಂದೊಂ-ದಾಗಿ
ಒಂದೊಂದು
ಒಂದೊಂದೂ
ಒಂದೋ
ಒಂಬ-ತ್ತ-ರಂದು
ಒಂಬತ್ತು
ಒಂಬತ್ತೇ
ಒಕ್ಕೂ-ಟ-ಗಳಲ್ಲಿ
ಒಕ್ಕೂ-ಟದ
ಒಕ್ಕೂ-ಟ-ವಷ್ಟೇ
ಒಕ್ಕೂ-ಟವು
ಒಕ್ಕೊ-ರ-ಲಿನ
ಒಕ್ಕೊ-ರ-ಳಿ-ನಿಂದ
ಒಗ-ಟಾ-ಗಿಯೇ
ಒಗ್ಗ-ಟ್ಟಾ-ಗ-ಬೇಕು
ಒಗ್ಗ-ಟ್ಟಾಗಿ
ಒಗ್ಗ-ಟ್ಟಿ-ನಿಂದ
ಒಗ್ಗ-ಲಿ-ಲ್ಲ-ವೆಂದು
ಒಗ್ಗಿ-ಕೊ-ಳ್ಳ-ಲಾ-ರಂ-ಭಿ-ಸಿ-ದರು
ಒಗ್ಗಿದೆ
ಒಗ್ಗೂ-ಡಿ-ಸಲು
ಒಗ್ಗೂ-ಡಿ-ಸಿ-ಕೊಂಡು
ಒಗ್ಗೂ-ಡಿ-ಸಿ-ಕೊ-ಳ್ಳ-ಲೇ-ಬೇಕು
ಒಗ್ಗೂ-ಡಿ-ಸು-ವುದೇ
ಒಟ್ಟಾಗಿ
ಒಟ್ಟಾ-ಗಿದ್ದು
ಒಟ್ಟಾರೆ
ಒಟ್ಟಾ-ರೆ-ಯಾಗಿ
ಒಟ್ಟಿಗೆ
ಒಟ್ಟಿ-ನಲ್ಲಿ
ಒಟ್ಟು
ಒಟ್ಟು-ಗೂ-ಡಿಸಿ
ಒಟ್ಟು-ಗೂ-ಡಿ-ಸಿ-ಕೊಂಡು
ಒಟ್ಟು-ಗೂ-ಡಿ-ಸಿ-ದರೆ
ಒಡ
ಒಡಂ-ಬ-ಡ-ಲಿಲ್ಲ
ಒಡಂ-ಬ-ಡಿ-ಕೆಯು
ಒಡಕು
ಒಡ-ಕು-ಗಳ
ಒಡತಿ
ಒಡ-ನಾಟ
ಒಡ-ನಾ-ಟ-ಗಳು
ಒಡ-ನಾ-ಟ-ವ-ನ್ನಿ-ಟ್ಟು-ಕೊಂ-ಡಿದೆ
ಒಡ-ನಾ-ಟ-ವನ್ನು
ಒಡ-ನಾಡಿ
ಒಡ-ನಾ-ಡಿ-ಗ-ಳೊಂ-ದಿಗೆ
ಒಡ-ಮೂ-ಡ-ಬೇಕು
ಒಡ-ಮೂಡಿ
ಒಡ-ಮೂ-ಡಿ-ದ-ವು-ಗ-ಳಾ-ಗಿದ್ದು
ಒಡ-ಮೂ-ಡಿ-ರದ
ಒಡೆ-ತ-ನದ
ಒಡೆ-ತ-ನ-ವನ್ನು
ಒಡೆ-ದರೆ
ಒಡೆ-ದಿ-ರುವ
ಒಡೆದು
ಒಡೆ-ಯ-ರನ್ನು
ಒಡೆ-ಯ-ರಿಗೆ
ಒಡೆ-ಯರು
ಒಡೆ-ಯರ್
ಒಡೆ-ಯು-ತ್ತದೆ
ಒಣ
ಒಣ-ಗಿತು
ಒಣ-ಗಿದ
ಒಣ-ಗಿ-ಸಲು
ಒಣ-ಗಿ-ಹೋ-ಗಿದೆ
ಒಣ-ರೊ-ಟ್ಟಿಯ
ಒತ್ತಡ
ಒತ್ತ-ಡದ
ಒತ್ತ-ಡ-ದಲ್ಲೂ
ಒತ್ತ-ಡ-ವನ್ನು
ಒತ್ತ-ಡ-ವಿದ್ದು
ಒತ್ತ-ರಿಸಿ
ಒತ್ತಾಯ
ಒತ್ತಾ-ಯಕ್ಕೆ
ಒತ್ತಾ-ಯದ
ಒತ್ತಾ-ಯ-ದಿಂದ
ಒತ್ತಾ-ಯ-ದಿಂ-ದಾಗಿ
ಒತ್ತಾ-ಯ-ಪ-ಡಿಸಿ
ಒತ್ತಾ-ಯ-ಪ-ಡಿ-ಸು-ತ್ತಿ-ದ್ದರು
ಒತ್ತಾ-ಯ-ಪೂ-ರ್ವ-ಕ-ವಾಗಿ
ಒತ್ತಾ-ಯ-ಸಿದ
ಒತ್ತಾ-ಯಿ-ಸಲು
ಒತ್ತಾ-ಯಿಸಿ
ಒತ್ತಾ-ಯಿ-ಸಿ-ದರು
ಒತ್ತಾ-ಯಿಸು
ಒತ್ತಾ-ಯಿ-ಸು-ತ್ತಿ-ದ್ದರು
ಒತ್ತಾ-ಯಿ-ಸುವ
ಒತ್ತಿ
ಒತ್ತಿ-ದರು
ಒತ್ತಿ-ಹಿ-ಡಿ-ಯಲು
ಒತ್ತಿ-ಹಿ-ಡಿ-ಯುತ್ತ
ಒತ್ತಿ-ಹೇ-ಳ-ಲಾ-ಗಿತ್ತು
ಒತ್ತಿ-ಹೇಳಿ
ಒತ್ತಿ-ಹೇ-ಳಿ-ದರು
ಒತ್ತಿ-ಹೇ-ಳು-ತ್ತಿದ್ದ
ಒತ್ತಿ-ಹೇ-ಳುವ
ಒತ್ತು
ಒತ್ತು-ಕೊಟ್ಟು
ಒತ್ತು-ತ್ತಲೇ
ಒತ್ತು-ತ್ತಿದ್ದ
ಒದ-ಗ-ಬ-ಹು-ದಾದ
ಒದ-ಗಿತು
ಒದ-ಗಿ-ದರೂ
ಒದ-ಗಿ-ದರೆ
ಒದ-ಗಿ-ದ್ದ-ಕ್ಕಾಗಿ
ಒದ-ಗಿ-ದ್ದರೆ
ಒದ-ಗಿ-ಬಂದ
ಒದ-ಗಿ-ಬಂ-ದ-ದ್ದನ್ನು
ಒದ-ಗಿ-ಬಂ-ದಿತು
ಒದ-ಗಿ-ಬಂ-ದಿದೆ
ಒದ-ಗಿ-ಬಂ-ದಿರ
ಒದ-ಗಿ-ಬಂ-ದಿ-ರ-ಲಿಲ್ಲ
ಒದ-ಗಿ-ಬಂ-ದಿ-ರ-ಲಿ-ಲ್ಲ-ವೆಂದು
ಒದ-ಗಿ-ಬ-ರು-ತ್ತಿತ್ತು
ಒದ-ಗಿಸಿ
ಒದ-ಗಿ-ಸಿ-ಕೊಟ್ಟು
ಒದ-ಗಿ-ಸಿ-ಕೊ-ಡು-ತ್ತಿ-ದ್ದರು
ಒದ-ಗಿ-ಸಿ-ಕೊ-ಡು-ತ್ತಿ-ದ್ದಾನೆ
ಒದ-ಗಿ-ಸಿ-ಕೊ-ಡು-ತ್ತೇನೆ
ಒದ-ಗಿ-ಸಿ-ಕೊ-ಡು-ವಂತೆ
ಒದ-ಗಿ-ಸಿದ
ಒದ-ಗಿ-ಸಿ-ದ-ನಂತೆ
ಒದ-ಗಿ-ಸಿ-ದ-ವರ
ಒದ-ಗಿ-ಸಿದ್ದ
ಒದ-ಗಿ-ಸು-ತ್ತಾರೆ
ಒದ-ಗುವ
ಒದರಿ
ಒದ-ರಿ-ಬಿ-ಟ್ಟಳು
ಒದೆ
ಒದೆ-ಸಿ-ಕೊ-ಳ್ಳು-ತ್ತಿ-ದ್ದೇವೆ
ಒದ್ದಾ-ಡು-ತ್ತಿ-ರು-ತ್ತಾರೆ
ಒದ್ದಾ-ಡು-ವು-ದಕ್ಕೆ
ಒಪ್ಪಂದ
ಒಪ್ಪಂ-ದಕ್ಕೆ
ಒಪ್ಪಂ-ದದ
ಒಪ್ಪಂ-ದ-ವನ್ನು
ಒಪ್ಪಂ-ದ-ವೊಂ-ದಕ್ಕೆ
ಒಪ್ಪ-ದಿ-ದ್ದಾಗ
ಒಪ್ಪ-ದಿ-ದ್ದಿರ
ಒಪ್ಪ-ದಿ-ಹನು
ಒಪ್ಪದೆ
ಒಪ್ಪ-ಬೇಕು
ಒಪ್ಪ-ಲಾ-ರ-ನೆಂದೂ
ಒಪ್ಪ-ಲಿಲ್ಲ
ಒಪ್ಪಲೇ
ಒಪ್ಪ-ಲೇ-ಬೇ-ಕಾ-ಯಿತು
ಒಪ್ಪ-ವಾಗಿ
ಒಪ್ಪಿ
ಒಪ್ಪಿ-ಕೊಂಡ
ಒಪ್ಪಿ-ಕೊಂ-ಡಂ-ತಾ-ಗಿ-ಬಿ-ಡು-ತ್ತಿತ್ತು
ಒಪ್ಪಿ-ಕೊಂ-ಡದ್ದು
ಒಪ್ಪಿ-ಕೊಂ-ಡರು
ಒಪ್ಪಿ-ಕೊಂ-ಡರೆ
ಒಪ್ಪಿ-ಕೊಂ-ಡಿ-ದ್ದರು
ಒಪ್ಪಿ-ಕೊಂಡು
ಒಪ್ಪಿ-ಕೊಂ-ಡು-ಬಿ-ಡ-ಲಿಲ್ಲ
ಒಪ್ಪಿ-ಕೊಂ-ಡು-ಬಿಡು
ಒಪ್ಪಿ-ಕೊ-ಳ್ಳ-ಬೇ-ಕಾ-ಯಿತು
ಒಪ್ಪಿ-ಕೊ-ಳ್ಳ-ಬೇಕು
ಒಪ್ಪಿ-ಕೊ-ಳ್ಳ-ಲಾ-ರರು
ಒಪ್ಪಿ-ಕೊ-ಳ್ಳ-ಲಿಲ್ಲ
ಒಪ್ಪಿ-ಕೊ-ಳ್ಳಲು
ಒಪ್ಪಿ-ಕೊ-ಳ್ಳ-ಲೇ-ಬೇ-ಕಾ-ಯಿತು
ಒಪ್ಪಿ-ಕೊಳ್ಳು
ಒಪ್ಪಿ-ಕೊ-ಳ್ಳು-ತ್ತೇನೆ
ಒಪ್ಪಿ-ಕೊ-ಳ್ಳು-ವಂ-ತಿ-ರ-ಲಿಲ್ಲ
ಒಪ್ಪಿ-ಕೊ-ಳ್ಳುವು
ಒಪ್ಪಿಗೆ
ಒಪ್ಪಿ-ಗೆಯ
ಒಪ್ಪಿ-ಗೆ-ಯನ್ನು
ಒಪ್ಪಿ-ಗೆ-ಯಾ-ಗ-ಲಿಲ್ಲ
ಒಪ್ಪಿ-ಗೆ-ಯಾ-ಗಿದೆ
ಒಪ್ಪಿ-ಗೆ-ಯಾ-ಗಿ-ರ-ಲಿಲ್ಲ
ಒಪ್ಪಿ-ಗೆ-ಯಾ-ಗುವ
ಒಪ್ಪಿ-ಗೆ-ಯಾ-ಗು-ವಂ-ತಹ
ಒಪ್ಪಿ-ಗೆ-ಯಾ-ಗು-ವಂ-ತಿತ್ತು
ಒಪ್ಪಿ-ಗೆ-ಯಾ-ಗು-ವಂತೆ
ಒಪ್ಪಿ-ಗೆ-ಯಾ-ಗು-ವುಂ-ಥ-ದಲ್ಲ
ಒಪ್ಪಿ-ಗೆ-ಯಾ-ಗು-ವು-ದಾ-ದರೆ
ಒಪ್ಪಿ-ಗೆ-ಯಾ-ದು-ವೆಂದರೆ
ಒಪ್ಪಿ-ಗೆ-ಯಾ-ದು-ವೆಂ-ದಲ್ಲ
ಒಪ್ಪಿ-ಗೆಯೇ
ಒಪ್ಪಿ-ಗೆ-ಯೇನೋ
ಒಪ್ಪಿದ
ಒಪ್ಪಿ-ದರು
ಒಪ್ಪಿ-ದರೆ
ಒಪ್ಪಿದ್ದ
ಒಪ್ಪಿ-ದ್ದರು
ಒಪ್ಪಿ-ಯಾರೆ
ಒಪ್ಪಿಸಿ
ಒಪ್ಪಿ-ಸಿದ
ಒಪ್ಪಿ-ಸಿ-ದರು
ಒಪ್ಪಿ-ಸಿ-ದಾಗ
ಒಪ್ಪಿ-ಸಿ-ದ್ದರು
ಒಪ್ಪಿ-ಸಿ-ರ-ಲಿಲ್ಲ
ಒಪ್ಪಿ-ಸು-ತ್ತೇನೆ
ಒಪ್ಪಿ-ಸು-ವು-ದಾ-ದರೆ
ಒಪ್ಪು-ತ್ತಲೇ
ಒಪ್ಪು-ತ್ತಾನೆ
ಒಪ್ಪು-ತ್ತಾ-ರೆಯೆ
ಒಪ್ಪು-ತ್ತಿಲ್ಲ
ಒಪ್ಪು-ತ್ತೇನೆ
ಒಪ್ಪು-ತ್ತೇ-ವಷ್ಟೇ
ಒಪ್ಪು-ವು-ದಾ-ದರೆ
ಒಪ್ಪು-ವು-ದಿಲ್ಲ
ಒಪ್ಪು-ವು-ದಿ-ಲ್ಲ-ವೆಂದೇ
ಒಬ್ಬ
ಒಬ್ಬಂ-ಟಿ-ಗರು
ಒಬ್ಬಂ-ಟಿ-ಗ-ರೆಂಬ
ಒಬ್ಬನ
ಒಬ್ಬ-ನಂತೆ
ಒಬ್ಬ-ನನ್ನು
ಒಬ್ಬ-ನನ್ನೂ
ಒಬ್ಬ-ನಾ-ಗಿದ್ದ
ಒಬ್ಬ-ನಿ-ಗಿಂತ
ಒಬ್ಬನು
ಒಬ್ಬನೂ
ಒಬ್ಬನೇ
ಒಬ್ಬರ
ಒಬ್ಬ-ರಂತೂ
ಒಬ್ಬ-ರಂತೆ
ಒಬ್ಬ-ರನ್ನು
ಒಬ್ಬ-ರಲ್ಲ
ಒಬ್ಬ-ರಾದ
ಒಬ್ಬ-ರಾ-ದ-ಮೇ-ಲೊ-ಬ್ಬರು
ಒಬ್ಬ-ರಾ-ದರೂ
ಒಬ್ಬ-ರಿ-ಗಾ-ದರೂ
ಒಬ್ಬ-ರಿ-ಗಿಂತ
ಒಬ್ಬರು
ಒಬ್ಬರೂ
ಒಬ್ಬ-ರೆಂ-ಬಂತೆ
ಒಬ್ಬರೇ
ಒಬ್ಬಳ
ಒಬ್ಬಳು
ಒಬ್ಬಳೇ
ಒಬ್ಬ-ಸೇ-ವ-ಕ-ನೆಂದು
ಒಬ್ಬಿ-ಬ್ಬರು
ಒಬ್ಬೊ
ಒಬ್ಬೊಬ್ಬ
ಒಬ್ಬೊ-ಬ್ಬರ
ಒಬ್ಬೊ-ಬ್ಬ-ರಾಗಿ
ಒಬ್ಬೊ-ಬ್ಬ-ರಿಗೂ
ಒಬ್ಬೊ-ಬ್ಬರೂ
ಒಬ್ಬೊ-ಬ್ಬರೇ
ಒಬ್ಬೊ-ಬ್ಬಳು
ಒಮ್ಮ-ತದ
ಒಮ್ಮ-ತ-ವಿಲ್ಲ
ಒಮ್ಮ-ನ-ಸ್ಸಿನ
ಒಮ್ಮೆ
ಒಮ್ಮೆಗೇ
ಒಮ್ಮೆ-ಮೇ-ಳ-ದಲ್ಲಿ
ಒಮ್ಮೆ-ಯಂತೂ
ಒಮ್ಮೆಲೇ
ಒಮ್ಮೊಮ್ಮೆ
ಒಯ್ದರೆ
ಒಯ್ದಿವೆ
ಒಯ್ದು
ಒಯ್ಯು-ವುದೇ
ಒರ-ಗಿ-ಕೊಂಡು
ಒರ-ಗಿ-ಕೊ-ಳ್ಳು-ತ್ತಾರೆ
ಒರಗು
ಒರ-ಗು-ಕುರ್ಚಿ
ಒರ-ಟಾಗಿ
ಒರಟು
ಒರ-ಸದ
ಒರ-ಸ-ಲಾ-ರದ
ಒರೆ-ಗ-ಲ್ಲಿಗೆ
ಒರೆಗೆ
ಒರೆ-ಸಿ-ಕೊ-ಳ್ಳುತ್ತ
ಒಲ-ವನ್ನು
ಒಲವು
ಒಲಿ-ದವ
ಒಲಿಸಿ
ಒಲಿ-ಸಿ-ಕೊಂಡು
ಒಲೆ
ಒಲೆ-ಯೊ-ಳಗೆ
ಒಲ್ಲದ
ಒಳಕ್ಕೆ
ಒಳ-ಗ-ಣ್ಣನ್ನು
ಒಳ-ಗಾಗಿ
ಒಳ-ಗಾ-ಗಿ-ದ್ದಾ-ನೆಂ-ಬು-ದನ್ನು
ಒಳ-ಗಾದ
ಒಳ-ಗಾ-ದರು
ಒಳ-ಗಾದೆ
ಒಳ-ಗಿನ
ಒಳ-ಗಿ-ನಿಂದ
ಒಳಗೂ
ಒಳಗೆ
ಒಳ-ಗೇ-ನೇ-ನಿ-ದೆಯೋ
ಒಳ-ಗೊಂ-ಡಿತ್ತು
ಒಳ-ಗೊಂ-ಡಿದೆ
ಒಳ-ಗೊ-ಳ-ಗಿನ
ಒಳ-ಗೊ-ಳಗೇ
ಒಳ-ಗೊ-ಳ್ಳುವ
ಒಳ-ಗೋಡಿ
ಒಳ-ತಿ-ಗಾಗಿ
ಒಳ-ಪ-ಟ್ಟಂ-ತಹ
ಒಳ-ಪ-ಟ್ಟಿ-ರು-ವಂ-ಥದು
ಒಳ-ಪ-ಡಿ-ಸಿ-ಕೊ-ಳ್ಳು-ವುದನ್ನು
ಒಳ-ಪ-ಡಿ-ಸು-ವು-ದಿಲ್ಲ
ಒಳ-ಪ್ರ-ವೇ-ಶಿ-ಸು-ತ್ತಿ-ದ್ದಂತೆ
ಒಳ-ಪ್ರ-ವೇ-ಶಿ-ಸು-ತ್ತಿ-ದ್ದುವು
ಒಳ-ಭಾ-ಗ-ಗ-ಗಳಲ್ಲಿ
ಒಳ-ಭಾ-ಗ-ದಲ್ಲಿ
ಒಳ-ಭಾ-ಗ-ವನ್ನು
ಒಳ-ಹೊ-ಕ್ಕರು
ಒಳ-ಹೊಕ್ಕು
ಒಳ-ಹೊ-ಗ-ಬೇಕು
ಒಳ-ಹೊ-ರ-ಗನ್ನು
ಒಳ-ಹೋ-ಗುವ
ಒಳಿ-ತನ್ನು
ಒಳಿ-ತನ್ನೇ
ಒಳಿ-ತಷ್ಟೇ
ಒಳಿ-ತಾ-ಗ-ಲೇ-ಬೇಕು
ಒಳಿ-ತಾ-ದೀ-ತೆಂಬ
ಒಳಿತಿ
ಒಳಿ-ತಿ-ಗಾ-ಗ-ಲ್ಲದೆ
ಒಳಿ-ತಿ-ಗಾಗಿ
ಒಳಿ-ತಿ-ಗಾ-ಗಿಯೇ
ಒಳಿ-ತಿಗೆ
ಒಳಿ-ತಿನ
ಒಳಿತು
ಒಳ್ಳೆ
ಒಳ್ಳೆಯ
ಒಳ್ಳೆ-ಯ-ದಕ್ಕೇ
ಒಳ್ಳೆ-ಯ-ದನ್ನು
ಒಳ್ಳೆ-ಯ-ದನ್ನೇ
ಒಳ್ಳೆ-ಯ-ದಲ್ಲ
ಒಳ್ಳೆ-ಯ-ದಾಗಿ
ಒಳ್ಳೆ-ಯ-ದಾ-ದರೂ
ಒಳ್ಳೆ-ಯ-ದಿ-ದ್ದರೆ
ಒಳ್ಳೆ-ಯ-ದಿ-ರಲಿ
ಒಳ್ಳೆ-ಯದು
ಒಳ್ಳೆ-ಯದೂ
ಒಳ್ಳೆ-ಯ-ದೆಂದು
ಒಳ್ಳೆ-ಯದೇ
ಒಳ್ಳೆ-ಯ-ದೇ-ಭಾಷೆ
ಒಳ್ಳೆ-ಯ-ವ-ನಾ-ಗಿ-ದ್ದರೆ
ಒಳ್ಳೆ-ಯ-ವನು
ಒಳ್ಳೆ-ಯ-ವ-ರಾ-ಗು-ವುದು
ಒಳ್ಳೆ-ಯ-ವರು
ಒಳ್ಳೆ-ಯ-ವರೇ
ಒಳ್ಳೇ
ಒಸಾಕ
ಓ
ಓಂ
ಓಂಕಾ-ರ-ವನ್ನು
ಓಕ್ಲೆ
ಓಗೊಟ್ಟು
ಓಗೊಟ್ಟೆ
ಓಗೊ-ಡುವ
ಓಗೊ-ಡು-ವ-ವ-ರೆಗೆ
ಓಜಸ್
ಓಜಸ್ಸು
ಓಡ-ಲಾ-ರದೆ
ಓಡ-ಲಾ-ರರು
ಓಡ-ಲಿ-ಕ್ಕಿಲ್ಲ
ಓಡಲು
ಓಡಾಡ
ಓಡಾ-ಡ-ಬೇ-ಕಾದ
ಓಡಾ-ಡಲು
ಓಡಾ-ಡಿ-ಕೊಂ-ಡಿ-ದ್ದರು
ಓಡಾ-ಡಿ-ಕೊಂ-ಡಿ-ದ್ದರೂ
ಓಡಾ-ಡಿ-ಕೊಂ-ಡಿದ್ದು
ಓಡಾ-ಡಿ-ಕೊಂ-ಡಿ-ರು-ತ್ತಿ-ದ್ದಳು
ಓಡಾ-ಡಿ-ದರು
ಓಡಾ-ಡುತ್ತ
ಓಡಾ-ಡು-ತ್ತಲೇ
ಓಡಾ-ಡು-ತ್ತಿ-ದ್ದರು
ಓಡಾ-ಡು-ತ್ತಿ-ದ್ದ-ವರ
ಓಡಾ-ಡು-ತ್ತಿ-ದ್ದಾಗ
ಓಡಾ-ಡು-ತ್ತಿ-ರು-ತ್ತಾಳೆ
ಓಡಾ-ಡು-ವ-ವರೆಲ್ಲ
ಓಡಾ-ಡು-ವಾಗ
ಓಡಾ-ಡು-ವು-ದಿಲ್ಲ
ಓಡಿ
ಓಡಿ-ಓಡಿ
ಓಡಿದ
ಓಡಿ-ದರು
ಓಡಿ-ದಳು
ಓಡಿ-ಬಂದ
ಓಡಿ-ಬಂದು
ಓಡಿ-ಬ-ರ-ಲೇ-ಬೇಕು
ಓಡಿ-ಬ-ರು-ತ್ತಿ-ದ್ದರು
ಓಡಿ-ಬಿ-ಡು-ತ್ತಾರೆ
ಓಡಿ-ಯಾ-ಡು-ತ್ತಿ-ದ್ದರು
ಓಡಿ-ಸಿ-ದರು
ಓಡಿ-ಸಿದ್ದೂ
ಓಡಿಸು
ಓಡಿ-ಹೋಗಿ
ಓಡಿ-ಹೋ-ಗಿ-ರ-ಬೇಕು
ಓಡಿ-ಹೋ-ಗು-ತ್ತಾರೆ
ಓಡಿ-ಹೋದ
ಓಡಿ-ಹೋ-ದ-ವನು
ಓಡು
ಓಡು-ತ್ತದೆ
ಓಡು-ತ್ತಾರೆ
ಓಡು-ತ್ತಿ-ದ್ದಂತೆ
ಓಡು-ತ್ತಿ-ರುವ
ಓಡು-ವು-ದಕ್ಕೆ
ಓಡು-ವುದನ್ನು
ಓದ-ಲಾ-ರಂ-ಭಿ-ಸಿ-ದರು
ಓದಲು
ಓದಿ
ಓದಿ-ಬಿ-ಡು-ವು-ದಲ್ಲ
ಓದಿ-ಕೊಂಡ
ಓದಿ-ಕೊಂ-ಡರು
ಓದಿ-ಕೊಂ-ಡ-ವ-ನಾಗಿ
ಓದಿ-ಕೊಂ-ಡ-ವ-ನಿ-ಗಿಂತ
ಓದಿ-ಕೊಂ-ಡಿ-ದ್ದಳು
ಓದಿ-ಕೊಂಡು
ಓದಿದ
ಓದಿ-ದರು
ಓದಿ-ದಳು
ಓದಿ-ದ-ವ-ರಲ್ಲಿ
ಓದಿ-ದ-ವ-ರಿ-ದ್ದಾರೆ
ಓದಿ-ದಾಗ
ಓದಿ-ದ್ದರು
ಓದಿ-ದ್ದೀರಿ
ಓದಿ-ದ್ದೆವೋ
ಓದಿ-ದ್ದೇಕೆ
ಓದಿ-ನಲ್ಲಿ
ಓದಿ-ಬಿ-ಟ್ಟರೆ
ಓದಿ-ಯಾ-ದರೂ
ಓದಿಯೇ
ಓದಿ-ರ-ಬೇ-ಕೆಂದು
ಓದು
ಓದು-ಗರ
ಓದು-ಗ-ರಿಗೆ
ಓದು-ಗ-ರೆ-ಲ್ಲ-ರಲ್ಲಿ
ಓದು-ಗರೇ
ಓದುತ್ತ
ಓದು-ತ್ತ-ಹೋ-ದಂತೆ
ಓದು-ತ್ತಾ-ರೆಯೋ
ಓದು-ತ್ತಿ-ದ್ದಂತೆ
ಓದು-ತ್ತಿ-ದ್ದಾಗ
ಓದು-ತ್ತಿ-ದ್ದೇನೆ
ಓದು-ತ್ತಿ-ರು-ವಾಗ
ಓದುವ
ಓದು-ವಾಗ
ಓದು-ವು-ದ-ರಿಂದ
ಓದು-ವುದೇ
ಓಪಿ-ಯಮ್
ಓಯ್
ಓರ-ಣ-ವಾ-ಗಿ-ಟ್ಟು-ಕೊ-ಳ್ಳುವ
ಓರ್ವ
ಓಲೆ
ಓಲೇ
ಓಲೇ-ಬುಲ್
ಓಲೇ-ಬು-ಲ್-ಇ-ವ-ರಿ-ಬ್ಬರು
ಓಲ್ಕಾಟ್
ಓಷಿ-ಯನ್
ಓಹಿ-ಯೋಗೆ
ಓಹೊ
ಓಹೋ
ಓಹ್
ಔಚಿ-ತ್ಯ-ಜ್ಞಾ-ನ-ದಿಂದ
ಔತಣ
ಔತ-ಣ-ಕೂ-ಟಕ್ಕೆ
ಔತ-ಣ-ಕೂ-ಟ-ಗಳನ್ನು
ಔತ-ಣ-ಕೂ-ಟ-ಗ-ಳಲ್ಲೂ
ಔತ-ಣ-ಕೂ-ಟ-ಗ-ಳಿಗೆ
ಔತ-ಣ-ಕೂ-ಟ-ವನ್ನು
ಔತ-ಣ-ಕೂ-ಟ-ವನ್ನೂ
ಔತ-ಣ-ಕೂ-ಟ-ವ-ನ್ನೇ-ರ್ಪ-ಡಿಸಿ
ಔತ-ಣ-ಕೂ-ಟ-ವೊಂ-ದ-ರಲ್ಲಿ
ಔತ-ಣ-ಗಳ
ಔತ-ಣದ
ಔದಾರ್ಯ
ಔನ್ನತ್ಯ
ಔನ್ನ-ತ್ಯದ
ಔನ್ನ-ತ್ಯ-ವನ್ನು
ಔಪ-ಚಾ-ರಿಕ
ಔಪ-ಚಾ-ರಿ-ಕ-ತೆ-ಶಿಷ್ಟಾ
ಔಪ-ಚಾ-ರಿ-ಕ-ವಾಗಿ
ಔಷ-ಧ-ಗಳು
ಔಷಧಿ
ಔಷ-ಧಿ-ಯನ್ನು
ಕಂಕಣ
ಕಂಕ-ಣ-ಬ-ದ್ಧ-ರಾಗಿ
ಕಂಕ-ಣ-ಬ-ದ್ಧ-ರಾ-ಗಿ-ದ್ದರು
ಕಂಕ-ಣ-ಬ-ದ್ಧ-ರಾ-ಗಿ-ರು-ವು-ದಾ-ಗಿಯೂ
ಕಂಕ-ಣ-ಬ-ದ್ಧ-ರಾ-ದರು
ಕಂಕ-ಣ-ಬ-ದ್ಧ-ವಾ-ಯಿತು
ಕಂಗಳಲ್ಲಿ
ಕಂಗಳಿಂದ
ಕಂಗ-ಳಿಂ-ದಲೂ
ಕಂಗಳು
ಕಂಗಾ-ಲಾ-ಗಿ-ದ್ದರೆ
ಕಂಗಾ-ಲಾ-ಗಿ-ಬಿ-ಡು-ತ್ತಾರೆ
ಕಂಗಾ-ಲಾದ
ಕಂಗೆಟ್ಟ
ಕಂಗೊ-ಳಿ-ಸಿ-ದರು
ಕಂಗೊ-ಳಿ-ಸು-ತ್ತಿತ್ತು
ಕಂಗೊ-ಳಿ-ಸು-ತ್ತಿದೆ
ಕಂಗೊ-ಳಿ-ಸು-ತ್ತಿದ್ದ
ಕಂಚಿನ
ಕಂಟ-ಕ-ಗ-ಳಾಗಿ
ಕಂಠ
ಕಂಠ-ದಲ್ಲೇ
ಕಂಠ-ದಿಂದ
ಕಂಠ-ಧ್ವನಿ
ಕಂಠ-ಧ್ವ-ನಿಯ
ಕಂಠ-ಧ್ವ-ನಿ-ಯನ್ನು
ಕಂಠ-ಮಾ-ಧುರ್ಯ
ಕಂಠ-ಮಾ-ಧು-ರ್ಯ-ವನ್ನು
ಕಂಠವು
ಕಂಠವೋ
ಕಂಠ-ಶೋ-ಷಣೆ
ಕಂಠ-ಶ್ರೀ-ಯಿಂದ
ಕಂಠ-ಸ್ವರ
ಕಂಠ-ಸ್ವ-ರ-ದಿಂದ
ಕಂಠ-ಸ್ವ-ರ-ವನ್ನು
ಕಂಠ-ಸ್ವ-ರವೂ
ಕಂಡ
ಕಂಡಂತೆ
ಕಂಡ-ಕಂ-ಡ-ವರ
ಕಂಡ-ಕೂ-ಡಲೆ
ಕಂಡ-ಕೂ-ಡಲೇ
ಕಂಡ-ಕ್ಟ-ರ-ನಿಂದ
ಕಂಡ-ದ್ದ-ಕ್ಕಿಂ-ತಲೂ
ಕಂಡ-ದ್ದನ್ನು
ಕಂಡ-ದ್ದರ
ಕಂಡ-ದ್ದಾ-ಯಿತು
ಕಂಡದ್ದು
ಕಂಡದ್ದೇ
ಕಂಡ-ದ್ದೇನು
ಕಂಡ-ರಾ-ದರೂ
ಕಂಡ-ರಿ-ಯದ
ಕಂಡ-ರಿ-ಯ-ದಂ-ಥದು
ಕಂಡರು
ಕಂಡ-ರು-ಭ-ಗ-ವಂ-ತನ
ಕಂಡರೂ
ಕಂಡರೆ
ಕಂಡವ
ಕಂಡ-ವ-ರ-ಲ್ಲವೆ
ಕಂಡ-ವರು
ಕಂಡ-ವರೆಲ್ಲ
ಕಂಡ-ವ-ರೆಷ್ಟೋ
ಕಂಡ-ವರೇ
ಕಂಡ-ವ-ರೊ-ಬ್ಬರು
ಕಂಡಾಗ
ಕಂಡಾ-ಗಲೂ
ಕಂಡಾರು
ಕಂಡಿ
ಕಂಡಿತು
ಕಂಡಿ-ತು-ಆಹ್
ಕಂಡಿ-ತೆಂ-ದರೆ
ಕಂಡಿತ್ತು
ಕಂಡಿದ್ದ
ಕಂಡಿ-ದ್ದರು
ಕಂಡಿ-ದ್ದಳು
ಕಂಡಿ-ದ್ದಾ-ರೆಂ-ಬು-ದನ್ನು
ಕಂಡಿ-ದ್ದೇನೆ
ಕಂಡಿ-ದ್ದೇವೆ
ಕಂಡಿ-ರ-ಬೇಕು
ಕಂಡಿ-ರ-ಲಾ-ರದು
ಕಂಡಿ-ರ-ಲಿಲ್ಲ
ಕಂಡಿರಿ
ಕಂಡಿ-ರು-ವ-ವನೇ
ಕಂಡಿ-ರು-ವು-ದೆಲ್ಲ
ಕಂಡಿ-ರುವೆ
ಕಂಡಿಲ್ಲ
ಕಂಡು
ಕಂಡು-ಅ-ಥವಾ
ಕಂಡು-ಅ-ನೇ-ಕ-ರಿಗೆ
ಕಂಡು-ಕಂಡು
ಕಂಡು-ಕೇ-ಳ-ರಿ-ಯ-ದಿದ್ದ
ಕಂಡು-ಕೊಂಡ
ಕಂಡು-ಕೊಂ-ಡದ್ದು
ಕಂಡು-ಕೊಂ-ಡರು
ಕಂಡು-ಕೊಂ-ಡ-ರುಈ
ಕಂಡು-ಕೊಂ-ಡಳು
ಕಂಡು-ಕೊಂ-ಡ-ವ-ರಲ್ಲಿ
ಕಂಡು-ಕೊಂ-ಡವು
ಕಂಡು-ಕೊಂಡಿ
ಕಂಡು-ಕೊಂ-ಡಿತು
ಕಂಡು-ಕೊಂ-ಡಿದೆ
ಕಂಡು-ಕೊಂ-ಡಿದ್ದ
ಕಂಡು-ಕೊಂ-ಡಿ-ದ್ದರು
ಕಂಡು-ಕೊಂ-ಡಿ-ದ್ದೇನೆ
ಕಂಡು-ಕೊಂ-ಡಿ-ರುವ
ಕಂಡು-ಕೊಂ-ಡಿಲ್ಲ
ಕಂಡು-ಕೊಂಡು
ಕಂಡು-ಕೊಂಡೆ
ಕಂಡು-ಕೊಂ-ಡೆ-ಹೇ-ಗೆಂ-ದರೆ
ಕಂಡು-ಕೊ-ಳ್ಳ-ಬ-ಹುದು
ಕಂಡು-ಕೊ-ಳ್ಳ-ಬೇ-ಕಾಗಿ
ಕಂಡು-ಕೊ-ಳ್ಳ-ಬೇ-ಕಾ-ಗಿದೆ
ಕಂಡು-ಕೊ-ಳ್ಳ-ಬೇ-ಕೆಂದು
ಕಂಡು-ಕೊ-ಳ್ಳಲು
ಕಂಡು-ಕೊಳ್ಳಿ
ಕಂಡು-ಕೊ-ಳ್ಳು-ತ್ತದೆ
ಕಂಡು-ಕೊ-ಳ್ಳು-ತ್ತೀಯೆ
ಕಂಡು-ಕೊ-ಳ್ಳು-ವಂತೆ
ಕಂಡು-ಕೊ-ಳ್ಳು-ವು-ದ-ರಲ್ಲಿ
ಕಂಡು-ಕೊ-ಳ್ಳು-ವುದು
ಕಂಡು-ಬಂದ
ಕಂಡು-ಬಂ-ದ-ದ್ದ-ರಿಂ-ದಲೇ
ಕಂಡು-ಬಂ-ದರು
ಕಂಡು-ಬಂ-ದರೆ
ಕಂಡು-ಬಂ-ದಿತು
ಕಂಡು-ಬ-ರ-ದಿದ್ದ
ಕಂಡು-ಬ-ರ-ಬ-ಹುದು
ಕಂಡು-ಬ-ರ-ಲಾ-ರಂ-ಭಿ-ಸಿ-ದುವು
ಕಂಡು-ಬರು
ಕಂಡು-ಬ-ರು-ತ್ತದೆ
ಕಂಡು-ಬ-ರು-ತ್ತಿತ್ತು
ಕಂಡು-ಬ-ರು-ತ್ತಿತ್ತೋ
ಕಂಡು-ಬ-ರು-ತ್ತಿದೆ
ಕಂಡು-ಬ-ರು-ತ್ತಿದ್ದ
ಕಂಡು-ಬ-ರು-ತ್ತಿ-ದ್ದರೂ
ಕಂಡು-ಬ-ರು-ತ್ತಿ-ದ್ದು-ದನ್ನು
ಕಂಡು-ಬ-ರು-ತ್ತಿ-ರ-ಲಿಲ್ಲ
ಕಂಡು-ಬ-ರು-ತ್ತಿ-ರುವ
ಕಂಡು-ಬ-ರುವ
ಕಂಡು-ಬ-ರು-ವಂ-ತಹ
ಕಂಡು-ಬ-ರು-ವು-ದಿ-ಲ್ಲವೋ
ಕಂಡು-ಬ-ರು-ವುದು
ಕಂಡು-ಬ-ರು-ವು-ದೆ-ಲ್ಲವೂ
ಕಂಡು-ಹಿ-ಡಿ-ದ-ವರು
ಕಂಡು-ಹಿ-ಡಿ-ದಿದ್ದ
ಕಂಡು-ಹಿ-ಡಿ-ದಿ-ದ್ದರ
ಕಂಡು-ಹಿ-ಡಿ-ದ್ದಾರೆ
ಕಂಡು-ಹಿ-ಡಿ-ಯಲು
ಕಂಡು-ಹಿ-ಡಿ-ಯುವ
ಕಂಡು-ಹಿ-ಡಿ-ಯು-ವು-ದ-ಕ್ಕಿಂತ
ಕಂಡು-ಹು-ಡು-ಕು-ತ್ತಿ-ದ್ದರು
ಕಂಡೂ
ಕಂಡೆ
ಕಂಡೆವು
ಕಂಡೊ-ಡನೆ
ಕಂಡೊ-ಡ-ನೆಯೆ
ಕಂಡೊ-ಡ-ನೆಯೇ
ಕಂಡೋ
ಕಂತೆ
ಕಂತೆ-ಗ-ಳನ್ನೇ
ಕಂತೆ-ಗಳು
ಕಂತೆ-ಗ-ಳೆಂದು
ಕಂತೆಯೇ
ಕಂದಾ-ಚಾ-ರ-ಗಳನ್ನೂ
ಕಂದಾ-ಚಾ-ರ-ಗಳಲ್ಲಿ
ಕಂದಾ-ಚಾ-ರ-ಗಳು
ಕಂಪಿ-ಸಿತು
ಕಂಪೆನಿ
ಕಂಪೆ-ನಿ-ಯಿಂದ
ಕಂಬಕ್ಕೆ
ಕಂಬ-ನಿಗೆ
ಕಂಬ-ಳಿ-ಯನ್ನು
ಕಂಬ-ಳಿ-ಯನ್ನೂ
ಕಂಬ-ಳಿ-ಯೊ-ಳ-ಗಿ-ನಿಂದ
ಕಕ್ಷೆ-ಯಿಂದ
ಕಗ್ಗ-ತ್ತ-ಲಲ್ಲಿ
ಕಚ್
ಕಚ್ಚಿ-ಕೊ-ಳ್ಳು-ವಂ-ತಾ-ಗು-ತ್ತಿತ್ತು
ಕಚ್ಚಿ-ದ್ದ-ರಿಂದ
ಕಚ್ಚೀ
ಕಚ್ನ
ಕಛೇರಿ
ಕಛೇ-ರಿಗೆ
ಕಛೇ-ರಿಯ
ಕಛೇ-ರಿ-ಯಲ್ಲಿ
ಕಟಿ-ಬ-ದ್ಧ-ರಾಗಿ
ಕಟಿ-ಲೋ-ಪಾ-ಯ-ಗಳು
ಕಟು
ಕಟುಕ
ಕಟು-ತರ
ಕಟು-ತ-ರ-ವಾ-ದುದು
ಕಟು-ವಾಗಿ
ಕಟು-ವಾ-ಗಿದ್ದೆ
ಕಟು-ವಾದ
ಕಟೆ-ಕಟೆ
ಕಟೆ-ಯ-ಲ್ಪಟ್ಟ
ಕಟ್ಟ-ಕ-ಡೆಗೆ
ಕಟ್ಟಡ
ಕಟ್ಟ-ಡ-ಗಳನ್ನು
ಕಟ್ಟ-ಡ-ಗ-ಳಿವೆ
ಕಟ್ಟ-ಡದ
ಕಟ್ಟ-ಡ-ದಲ್ಲಿ
ಕಟ್ಟ-ಡ-ವನ್ನು
ಕಟ್ಟ-ಡ-ವಲ್ಲ
ಕಟ್ಟ-ಡ-ವಷ್ಟೇ
ಕಟ್ಟ-ಡವು
ಕಟ್ಟ-ದಿ-ದ್ದರೆ
ಕಟ್ಟ-ಬ-ಲ್ಲುದು
ಕಟ್ಟ-ಬೇ-ಕಾ-ಗಿದೆ
ಕಟ್ಟ-ಬೇ-ಕಾ-ದ-ವರು
ಕಟ್ಟ-ಬೇಕು
ಕಟ್ಟಲು
ಕಟ್ಟ-ಲ್ಪ-ಟ್ಟಿದ್ದು
ಕಟ್ಟ-ಳೆ-ಗಳ
ಕಟ್ಟ-ಳೆ-ಗಳನ್ನೆಲ್ಲ
ಕಟ್ಟಾ
ಕಟ್ಟಿ
ಕಟ್ಟಿ-ಕೊಂ-ಡಿ-ರು-ತ್ತಾರೆ
ಕಟ್ಟಿ-ಕೊಂಡು
ಕಟ್ಟಿ-ಕೊ-ಳ್ಳು-ತ್ತಿ-ದ್ದ-ರಂತೆ
ಕಟ್ಟಿ-ಟ್ಟದ್ದು
ಕಟ್ಟಿದ
ಕಟ್ಟಿ-ದಂ-ತಿದೆ
ಕಟ್ಟಿ-ದರು
ಕಟ್ಟಿ-ದುದು
ಕಟ್ಟಿ-ರುವ
ಕಟ್ಟಿ-ರು-ವುದು
ಕಟ್ಟಿ-ಸ-ಲಾ-ಗಿತ್ತು
ಕಟ್ಟಿ-ಸಿ-ಕೊ-ಡು-ವು-ದಕ್ಕೆ
ಕಟ್ಟಿ-ಹಾಕಿ
ಕಟ್ಟಿ-ಹಾ-ಕಿ-ರುವ
ಕಟ್ಟಿ-ಹೋ-ದರೂ
ಕಟ್ಟಿ-ಹೋ-ದವು
ಕಟ್ಟು
ಕಟ್ಟು-ಕ-ಟ್ಟಲೆ
ಕಟ್ಟು-ಕ-ಟ್ಟ-ಲೆ-ಗಳನ್ನೂ
ಕಟ್ಟು-ಕ-ಟ್ಟ-ಳೆ-ಗಳನ್ನು
ಕಟ್ಟು-ಕತೆ
ಕಟ್ಟು-ಗಳನ್ನೆಲ್ಲ
ಕಟ್ಟು-ತ್ತೀ-ದ್ದೀರಿ
ಕಟ್ಟು-ತ್ತೀರಿ
ಕಟ್ಟು-ತ್ತೇನೆ
ಕಟ್ಟು-ನಿ-ಟ್ಟಾಗಿ
ಕಟ್ಟು-ನಿ-ಟ್ಟಾದ
ಕಟ್ಟು-ಪಾಡು
ಕಟ್ಟು-ಪಾ-ಡು-ಗ-ಳಿಗೆ
ಕಟ್ಟು-ಬಿ-ದ್ದ-ವನು
ಕಟ್ಟು-ಮ-ಸ್ತಾದ
ಕಟ್ಟುವ
ಕಟ್ಟು-ವಂತೆ
ಕಟ್ಟು-ವು-ದರ
ಕಟ್ಟೆಯ
ಕಟ್ಟೊಂ-ದನ್ನು
ಕಠಿಣ
ಕಠಿ-ಣ-ತ-ರ-ವಾ-ಗಿತ್ತು
ಕಠಿ-ಣ-ವಾಗಿ
ಕಠಿ-ಣ-ವಾ-ಗಿಯೇ
ಕಠಿ-ಣ-ವಾ-ಗಿ-ರ-ಲಿಲ್ಲ
ಕಠಿ-ಣ-ವಾ-ಗಿ-ರು-ತ್ತಿ-ದ್ದುವು
ಕಠಿ-ಣ-ವಾದ
ಕಠಿ-ಣವೂ
ಕಠೋರ
ಕಡಗೆ
ಕಡಮೆ
ಕಡಲ
ಕಡ-ಲ-ತೀ-ರದ
ಕಡ-ಲನ್ನು
ಕಡ-ಲಾ-ಚೆಯ
ಕಡ-ಲಿನ
ಕಡಲೆ
ಕಡ-ಲೆ-ಯಾ-ಗಿತ್ತು
ಕಡ-ಲ್ಗಾ-ಲು-ವೆ-ಯನ್ನು
ಕಡಿದಾ
ಕಡಿ-ದಾದ
ಕಡಿ-ದಿ-ರ-ಲಿಲ್ಲ
ಕಡಿಮೆ
ಕಡಿ-ಮೆ-ಯಾ-ಗ-ಲಿಲ್ಲ
ಕಡಿ-ಮೆ-ಯಾ-ಗಿದೆ
ಕಡಿ-ಮೆ-ಯಾ-ಗಿಲ್ಲ
ಕಡಿ-ಮೆ-ಯಾ-ಗು-ತ್ತದೆ
ಕಡಿ-ಮೆ-ಯಾ-ಯಿತು
ಕಡಿ-ಮೆ-ಯಿ-ದ್ದರೂ
ಕಡಿ-ಮೆ-ಯಿ-ದ್ದು-ದನ್ನು
ಕಡಿ-ಮೆ-ಯಿರು
ಕಡಿ-ಮೆ-ಯಿ-ರು-ವಲ್ಲಿ
ಕಡಿ-ಮೆಯೇ
ಕಡಿ-ಮೆ-ಯೇ-ನಲ್ಲ
ಕಡಿ-ಮೆ-ಯೇನೂ
ಕಡಿ-ಯು-ವು-ದ-ರಲ್ಲೇ
ಕಡು
ಕಡು-ಗೆಂ-ಪಿನ
ಕಡು-ಬ-ಡವ
ಕಡು-ಬಿನ
ಕಡು-ಬೇ-ಸಿ-ಗೆಯೂ
ಕಡೆ
ಕಡೆ-ಗ-ಣಿಸಿ
ಕಡೆ-ಗ-ಣಿ-ಸಿ-ಬಾ-ರ-ದೆಂದು
ಕಡೆ-ಗ-ಣಿ-ಸಿ-ಬಿ-ಟ್ಟರು
ಕಡೆ-ಗ-ಣಿ-ಸಿ-ಬಿ-ಟ್ಟಿ-ರು-ವೆ-ಯಲ್ಲ
ಕಡೆ-ಗ-ಣಿ-ಸು-ವಂ-ತಿ-ರ-ಲಿಲ್ಲ
ಕಡೆ-ಗ-ಣಿ-ಸು-ವಂ-ತಿಲ್ಲ
ಕಡೆ-ಗ-ಣಿ-ಸು-ವು-ದರ
ಕಡೆ-ಗಲ್ಲ
ಕಡೆ-ಗಳಲ್ಲಿ
ಕಡೆ-ಗ-ಳ-ಲ್ಲೆಲ್ಲ
ಕಡೆ-ಗ-ಳಿಂ-ದಲೂ
ಕಡೆ-ಗಿನ
ಕಡೆ-ಗೀಗ
ಕಡೆಗೂ
ಕಡೆಗೆ
ಕಡೆ-ಗೆಲ್ಲ
ಕಡೆಗೇ
ಕಡೆ-ಗೇ-ಅ-ವನು
ಕಡೆ-ಗೊಂದು
ಕಡೆ-ಗೊಮ್ಮೆ
ಕಡೆಯ
ಕಡೆ-ಯ-ದಾಗಿ
ಕಡೆ-ಯ-ಪಕ್ಷ
ಕಡೆ-ಯ-ಬಾರಿ
ಕಡೆ-ಯ-ಬಾ-ರಿಗೆ
ಕಡೆ-ಯ-ಭಾರಿ
ಕಡೆ-ಯಲ್ಲಿ
ಕಡೆ-ಯಲ್ಲೇ
ಕಡೆ-ಯ-ವ-ರೆಗೂ
ಕಡೆ-ಯಿಂದ
ಕಡೆ-ಯಿಂ-ದೇನೂ
ಕಡೆ-ಯು-ತ್ತಿ-ರು-ವಂ-ತಿದೆ
ಕಡೇ
ಕಡೇ-ಪಕ್ಷ
ಕಡ್ಡಿ
ಕಡ್ಡಿ-ಯನ್ನು
ಕಣ
ಕಣ-ಕ-ಣವೂ
ಕಣಿ-ವೆಯ
ಕಣ್ಣಂ-ಚಿ-ನಲ್ಲಿ
ಕಣ್ಣ-ಗ-ಲಿಸಿ
ಕಣ್ಣಗೆ
ಕಣ್ಣಲ್ಲಿ
ಕಣ್ಣಲ್ಲೇ
ಕಣ್ಣಾರೆ
ಕಣ್ಣಿಂದ
ಕಣ್ಣಿ-ಗಂತೂ
ಕಣ್ಣಿಗೂ
ಕಣ್ಣಿಗೆ
ಕಣ್ಣಿ-ಟ್ಟಿ-ರು-ವಂತೆ
ಕಣ್ಣಿನ
ಕಣ್ಣಿ-ನಿಂದ
ಕಣ್ಣಿ-ರು-ವುದು
ಕಣ್ಣಿ-ಲ್ಲ-ವಾ-ಗಿ-ಬಿ-ಟ್ಟಿದೆ
ಕಣ್ಣೀ-ರನ್ನು
ಕಣ್ಣೀ-ರನ್ನೂ
ಕಣ್ಣೀ-ರಲ್ಲಿ
ಕಣ್ಣೀ-ರಾಗಿ
ಕಣ್ಣೀ-ರಿ-ನಿಂದ
ಕಣ್ಣೀರು
ಕಣ್ಣೀ-ರ್ಗ-ರೆ-ದರು
ಕಣ್ಣೀ-ರ್ಗ-ರೆ-ಯು-ತ್ತಾರೆ
ಕಣ್ಣು
ಕಣ್ಣುಂದೆ
ಕಣ್ಣು-ಗಳ
ಕಣ್ಣು-ಗಳನ್ನು
ಕಣ್ಣು-ಗ-ಳನ್ನೇ
ಕಣ್ಣು-ಗಳಲ್ಲಿ
ಕಣ್ಣು-ಗ-ಳಲ್ಲೂ
ಕಣ್ಣು-ಗಳಿಂದ
ಕಣ್ಣು-ಗ-ಳಿಗೂ
ಕಣ್ಣು-ಗ-ಳಿಗೆ
ಕಣ್ಣು-ಗಳು
ಕಣ್ಣು-ಗಳೂ
ಕಣ್ಣುಚ್ಚಿ
ಕಣ್ಣು-ಬಿಟ್ಟು
ಕಣ್ಣು-ಮು-ಚ್ಚಾ-ಲೆ-ಯಾ-ಟವೇ
ಕಣ್ಣು-ರಿಗೆ
ಕಣ್ಣೆ-ತ್ತಿಯೂ
ಕಣ್ಣೆ-ದು-ರಿ-ಗಿ-ರು-ವುದು
ಕಣ್ಣೆ-ದು-ರಿಗೆ
ಕಣ್ಣೆ-ದು-ರಿಗೇ
ಕಣ್ಣೆ-ದುರು
ಕಣ್ಣೆ-ವೆ-ಗಳು
ಕಣ್ಣೊ-ರೆ-ಸಿ-ಕೊಂಡು
ಕಣ್ಣೊ-ಳಗೆ
ಕಣ್ತುಂಬ
ಕಣ್ತೆ-ರೆದು
ಕಣ್ತೆ-ರೆ-ಸಲು
ಕಣ್ದೆ-ರೆಸಿ
ಕಣ್ದೆ-ರೆ-ಸಿ-ತಾ-ದರೂ
ಕಣ್ಮ-ಣಿ-ಯಾ-ಗಿದ್ದ
ಕಣ್ಮ-ಣಿ-ಯಾ-ಗಿ-ದ್ದಳು
ಕಣ್ಮ-ಣಿ-ಯಾ-ಗಿ-ದ್ದಾರೆ
ಕಣ್ಮರೆ
ಕಣ್ಮ-ರೆ-ಯಾಗಿ
ಕಣ್ಮ-ರೆ-ಯಾ-ಗು-ವವ
ಕಣ್ಮ-ರೆ-ಯಾ-ದುವು
ಕಣ್ಮುಂದೆ
ಕಣ್ಮುಚ್ಚಿ
ಕಣ್ಸೆಳೆ
ಕತೆ
ಕತೆ-ಗಳ
ಕತೆ-ಗಳನ್ನು
ಕತೆ-ಗ-ಳಿಗೆ
ಕತೆಯ
ಕತೆ-ಯನ್ನು
ಕತೆ-ಯಿಲ್ಲ
ಕತೆಯೂ
ಕತ್ತನ್ನು
ಕತ್ತ-ರಿ-ಸ-ಬೇ-ಕಾಗಿ
ಕತ್ತ-ರಿ-ಸುತ್ತಿ
ಕತ್ತ-ರಿ-ಸು-ವು-ದರ
ಕತ್ತಲ
ಕತ್ತ-ಲನ್ನು
ಕತ್ತ-ಲಲ್ಲಿ
ಕತ್ತ-ಲಾ-ಗುತ್ತ
ಕತ್ತ-ಲಾ-ಗು-ತ್ತಲೇ
ಕತ್ತಲು
ಕತ್ತಲೆ
ಕತ್ತ-ಲೆಯ
ಕತ್ತ-ಲೆ-ಯ-ನ್ನ-ಳಿ-ಸ-ಬಲ್ಲ
ಕತ್ತಿ-ಯ-ಲ-ಗಿನ
ಕತ್ತು
ಕಥ-ನ-ವನ್ನು
ಕಥಾ-ನಾ-ಯಕ
ಕಥೆ
ಕಥೆ-ದೃ-ಷ್ಟಾಂ-ತ-ಗಳ
ಕಥೆ-ಕ-ಟ್ಟಲು
ಕಥೆ-ಗಳ
ಕಥೆ-ಗಳನ್ನು
ಕಥೆ-ಗಳಿಂದ
ಕಥೆ-ಗಳು
ಕಥೆ-ಗಳೂ
ಕಥೆ-ಗ-ಳೆಂ-ದರೆ
ಕಥೆಗೆ
ಕಥೆ-ಯನ್ನು
ಕದ-ಡಿದೆ
ಕದ-ಡಿ-ಹೋ-ಗ-ದಿ-ರದು
ಕದ-ಡಿ-ಹೋ-ಯಿತು
ಕದ-ಡು-ತ್ತಿ-ರ-ಲಿಲ್ಲ
ಕದ-ಡುವ
ಕದ-ಲಿ-ಸು-ವಂ-ತಿ-ರ-ಲಿಲ್ಲ
ಕದ-ಲು-ತ್ತಿಲ್ಲ
ಕದಾಪಿ
ಕದಿ-ಯ-ದಿ-ರು-ವುದು
ಕದಿ-ಯಲು
ಕದೀ-ತೇನೆ
ಕದ್ದ
ಕದ್ದು
ಕನ-ಸನ್ನು
ಕನ-ಸಾ-ಯಿತು
ಕನ-ಸಿನ
ಕನ-ಸಿ-ನಲ್ಲಿ
ಕನ-ಸಿ-ನಿಂ-ದೆ-ಚ್ಚರ
ಕನ-ಸಿ-ನಿಂ-ದೆ-ಚ್ಚ-ರ-ಗೊಂ-ಡಂತೆ
ಕನಸು
ಕನ-ಸು-ಗ-ಳ-ಲ್ಲಿಯೂ
ಕನ-ಸು-ಗಳಿಂದ
ಕನ-ಸು-ಮ-ನ-ಸು-ಗ-ಳಲ್ಲೂ
ಕನಿ-ಕ-ರ-ವಾ-ಗು-ತ್ತದೆ
ಕನಿ-ಕ-ರ-ವೆ-ನಿ-ಸು-ತ್ತದೆ
ಕನಿಷ್ಟ
ಕನಿಷ್ಠ
ಕನಿ-ಷ್ಠ-ಪಕ್ಷ
ಕನಿ-ಷ್ಠ-ವಾದ
ಕನಿ-ಷ್ಠ-ವೆಂ-ಬು-ದನ್ನು
ಕನ್ನಡ
ಕನ್ನ-ಡದ
ಕನ್ನ-ಡ-ದಲ್ಲಿ
ಕನ್ನಡಿ
ಕನ್ನ-ಡಿ-ಗ-ರಾದ
ಕನ್ನ-ಡಿಯ
ಕನ್ನ-ಡಿ-ಯಂ-ತಿದೆ
ಕನ್ನ-ಡಿ-ಯನ್ನು
ಕನ್ನ-ಡಿ-ಯಲ್ಲಿ
ಕನ್ನ-ಡಿ-ಯೊಂ-ದಿತ್ತು
ಕನ್ನ-ಡಿ-ಸ-ಲಾ-ಗಿ-ರು-ವು-ದಿಂದ
ಕನ್ಯ-ಕಾ-ಪ-ರ-ಮೇ-ಶ್ವ-ರಿ-ಯಾಗಿ
ಕನ್ಯಾ-ಕು-ಮಾರಿ
ಕನ್ಯಾ-ಕು-ಮಾ-ರಿಗೆ
ಕನ್ಯಾ-ಕು-ಮಾ-ರಿಯ
ಕನ್ಯಾ-ಕು-ಮಾ-ರಿ-ಯನ್ನು
ಕನ್ಯಾ-ಕು-ಮಾ-ರಿ-ಯಲ್ಲಿ
ಕನ್ಯಾ-ಕು-ಮಾ-ರಿ-ಯ-ವ-ರೆಗೆ
ಕನ್ಯಾ-ಕು-ಮಾ-ರಿ-ಯಾಗಿ
ಕನ್ಯಾ-ಕು-ಮಾ-ರಿ-ಯಿಂದ
ಕನ್ಯಾ-ಕು-ಮಾ-ರಿ-ಯೆ-ಡೆಗೆ
ಕನ್ಯೆ
ಕನ್ಹೇರಿ
ಕಪ-ಟ-ಸಂನ್ಯಾಸಿ
ಕಪ-ಟ-ಸಂ-ನ್ಯಾ-ಸಿ-ಗಳು
ಕಪ-ಟಿ-ಯನ್ನು
ಕಪ-ಟಿ-ಯಲ್ಲ
ಕಪಾಟು
ಕಪಿ-ಸೈ-ನ್ಯವು
ಕಪು-ರ್ತಲ
ಕಪು-ರ್ತ-ಲದ
ಕಪೂರ್ಜಿ
ಕಪೋ-ಲ-ಗಳ
ಕಪ್ಪಿನ
ಕಪ್ಪು
ಕಪ್ಪೆ
ಕಪ್ಪೆಗೂ
ಕಪ್ಪೆಯ
ಕಪ್ಪೆ-ಯಂತೆ
ಕಪ್ಪೆ-ಯನ್ನು
ಕಪ್ಪೆ-ಯ-ನ್ನೇನೋ
ಕಪ್ಪೆ-ಯಾ-ಗಿ-ಬಿ-ಡು-ತ್ತಾಳೆ
ಕಬಚ್
ಕಬ-ಳಿಸ
ಕಬ-ಳಿ-ಸಿದೆ
ಕಬ್ಬಿ-ಣದ
ಕಮಂ-ಡ-ಲು-ಧಾ-ರಿ-ಯಾದ
ಕಮ-ರ್ಷಿ-ಯಲ್
ಕಮಲ
ಕಮಲೆ
ಕಮಾನು
ಕಮಿ-ಷ-ನರ್
ಕಮಿ-ಷ-ನಿನ
ಕಮ್ಮಾ-ರನ
ಕರ
ಕರ-ಕು-ಶ-ವ-ಸ್ತು-ಗ-ಳ-ಲ್ಲಿ-ಅ-ದ-ರಲ್ಲೂ
ಕರ-ಗತ
ಕರ-ಗ-ತ-ಗೊ-ಳಿ-ಸಿ-ಕೊ-ಳ್ಳು-ವು-ದೆಂದು
ಕರ-ಗದೆ
ಕರ-ಗಲೀ
ಕರಗಿ
ಕರ-ಗಿತು
ಕರ-ಗಿಸಿ
ಕರ-ಗಿ-ಹೋ-ಯಿತು
ಕರ-ತಲ
ಕರ-ತಾ-ಡನ
ಕರ-ತಾ-ಡ-ನ
ಕರ-ತಾ-ಡ-ನ-ಹ-ರ್ಷೋ-ದ್ಗಾ-ರ-ಗಳ
ಕರ-ತಾ-ಡ-ನ-ಗಳು
ಕರ-ತಾ-ಡ-ನದ
ಕರ-ತಾ-ಡ-ನ-ದಿಂದ
ಕರ-ತಾ-ಡ-ನ-ವನ್ನು
ಕರ-ಪತ್ರ
ಕರ-ಪ-ತ್ರ-ಗಳ
ಕರ-ಪ-ತ್ರ-ಗಳನ್ನು
ಕರ-ಪ-ತ್ರ-ಗಳನ್ನೂ
ಕರ-ಪ-ತ್ರ-ಗ-ಳಲ್ಲೂ
ಕರ-ಪ-ತ್ರ-ಗ-ಳಿಂ-ದಲೂ
ಕರ-ಪ-ತ್ರ-ದಲ್ಲಿ
ಕರ-ಪ-ತ್ರ-ವನ್ನು
ಕರ-ಪ-ತ್ರವೂ
ಕರ-ವ-ಸ್ತ್ರ-ಗಳನ್ನೂ
ಕರ-ವ-ಸ್ತ್ರ-ಗಳು
ಕರ-ವ-ಸ್ತ್ರ-ದಲ್ಲಿ
ಕರ-ವ-ಸ್ತ್ರ-ದಿಂದ
ಕರ-ವಾಗಿ
ಕರ-ವಾ-ಗಿತ್ತು
ಕರ-ವಾ-ಗಿದ್ದು
ಕರಾ-ಚಿ-ಯ-ವ-ರೆಗೆ
ಕರಾ-ಚಿ-ಯಿಂದ
ಕರಾ-ರನ್ನು
ಕರಾ-ರಿನ
ಕರಾರು
ಕರಾ-ವ-ಳಿಯ
ಕರಿ-ಯ-ರಂತೆ
ಕರು
ಕರು-ಣಾ-ಜ-ನ-ಕ-ವಾಗಿ
ಕರು-ಣಾ-ಪೂ-ರಿತ
ಕರು-ಣಾ-ಪೂ-ರಿ-ತವೂ
ಕರು-ಣಾ-ಪೂರ್ಣ
ಕರು-ಣಾ-ಪೂ-ರ್ಣ-ವಾದ
ಕರು-ಣಿ-ಸಲಿ
ಕರುಣೆ
ಕರು-ಣೆಯ
ಕರು-ಣೆ-ಯನ್ನು
ಕರು-ಣೆ-ಯನ್ನೂ
ಕರು-ಣೆ-ಯಿಂದ
ಕರು-ಬುವ
ಕರೆ
ಕರೆ-ಗಂಟೆ
ಕರೆಗೆ
ಕರೆ-ತಂದ
ಕರೆ-ತಂ-ದದ್ದು
ಕರೆ-ತಂ-ದರು
ಕರೆ-ತಂ-ದ-ವನು
ಕರೆ-ತಂ-ದಾಗ
ಕರೆ-ತಂದು
ಕರೆ-ತ-ರ-ಲಾ-ಯಿತು
ಕರೆ-ತ-ರುವ
ಕರೆ-ತ-ರು-ವಂ-ತೆಯೂ
ಕರೆದ
ಕರೆ-ದ-ನೆಂಬ
ಕರೆ-ದರು
ಕರೆ-ದ-ರು-ನೋಡಿ
ಕರೆ-ದರೆ
ಕರೆ-ದಾಗ
ಕರೆ-ದಾ-ಗಲೂ
ಕರೆ-ದಿ-ದ್ದಳು
ಕರೆ-ದಿ-ರಲ್ಲ
ಕರೆದು
ಕರೆ-ದು-ಕೊಂ-ಡರೂ
ಕರೆ-ದು-ಕೊಂಡು
ಕರೆ-ದು-ಕೊಂಡೇ
ಕರೆ-ದು-ಕೊ-ಳ್ಳ-ಲಾ-ರಂ-ಭಿಸಿ
ಕರೆ-ದು-ಕೊ-ಳ್ಳು-ವಂತೆ
ಕರೆ-ದು-ಕೊ-ಳ್ಳು-ವ-ವರ
ಕರೆ-ದುಕೋ
ಕರೆ-ದುವು
ಕರೆ-ದೊ-ಡ-ನೆಯೇ
ಕರೆ-ದೊಯ್ದ
ಕರೆ-ದೊ-ಯ್ದರು
ಕರೆ-ದೊ-ಯ್ದಳು
ಕರೆ-ದೊ-ಯ್ದಿತು
ಕರೆ-ದೊಯ್ದು
ಕರೆ-ದೊ-ಯ್ಯ-ಲಾ-ಗು-ತ್ತಿ-ದ್ದಂತೆ
ಕರೆ-ದೊ-ಯ್ಯ-ಲಾ-ಯಿತು
ಕರೆ-ದೊಯ್ಯು
ಕರೆ-ದೊ-ಯ್ಯು-ತ್ತಿತ್ತು
ಕರೆ-ದೊ-ಯ್ಯು-ತ್ತಿ-ದ್ದರು
ಕರೆ-ದೊ-ಯ್ಯುವ
ಕರೆ-ದೊ-ಯ್ಯು-ವಂತೆ
ಕರೆ-ನೀಡಿ
ಕರೆ-ನೀ-ಡಿ-ದರು
ಕರೆ-ನೀ-ಡು-ತ್ತೇನೆ
ಕರೆ-ಯ-ತೊ-ಡ-ಗಿ-ದ-ವರ
ಕರೆ-ಯನ್ನು
ಕರೆ-ಯ-ಬ-ಹು-ದಾದ
ಕರೆ-ಯ-ಲಾಗಿದೆ
ಕರೆ-ಯ-ಲಾ-ಗು-ತ್ತದೆ
ಕರೆ-ಯಲಿ
ಕರೆ-ಯಲು
ಕರೆ-ಯ-ಲ್ಪ-ಟ್ಟಿದ್ದ
ಕರೆ-ಯ-ಲ್ಪ-ಟ್ಟಿ-ರುವ
ಕರೆ-ಯ-ಲ್ಪಟ್ಟು
ಕರೆ-ಯ-ಲ್ಪ-ಡುವ
ಕರೆ-ಯ-ಲ್ಪ-ಡು-ವ-ವರ
ಕರೆ-ಯಿ-ಸಿ-ಕೊಂ-ಡಿ-ದ್ದಾಳೆ
ಕರೆ-ಯಿ-ಸಿ-ಕೊಂಡು
ಕರೆಯು
ಕರೆ-ಯು-ತ್ತವೆ
ಕರೆ-ಯು-ತ್ತಾರೆ
ಕರೆ-ಯು-ತ್ತಾರೋ
ಕರೆ-ಯು-ತ್ತಿದೆ
ಕರೆ-ಯು-ತ್ತಿ-ದ್ದರು
ಕರೆ-ಯು-ತ್ತಿ-ದ್ದಾರೆ
ಕರೆ-ಯು-ತ್ತಿ-ರ-ಲಿಲ್ಲ
ಕರೆ-ಯು-ತ್ತಿ-ರು-ವುದು
ಕರೆ-ಯು-ತ್ತೀಯೋ
ಕರೆ-ಯು-ತ್ತೇನೆ
ಕರೆ-ಯು-ತ್ತೇವೆ
ಕರೆ-ಯುವ
ಕರೆ-ಯು-ವು-ದಿಲ್ಲ
ಕರೆ-ಯು-ವುದೇ
ಕರೆ-ಯೊಂದು
ಕರೆ-ಯೋಣ
ಕರೆಸಿ
ಕರೆ-ಸಿ-ಕೊಂಡು
ಕರೆ-ಸಿ-ಕೊ-ಳ್ಳ-ಬೇ-ಕಾ-ದರೆ
ಕರೆ-ಸಿ-ಕೊ-ಳ್ಳಲು
ಕರೆ-ಸಿ-ಕೊ-ಳ್ಳುವ
ಕರೆ-ಸಿ-ಕೊ-ಳ್ಳು-ವಂತೆ
ಕರೆ-ಸಿ-ಕೊ-ಳ್ಳು-ವ-ವರ
ಕರೆ-ಸಿ-ಕೊ-ಳ್ಳು-ವು-ದಕ್ಕೇ
ಕರೆ-ಸಿದೆ
ಕರೋ
ಕರೋತಿ
ಕರ್
ಕರ್ಕಶ
ಕರ್ಣಾ-ಕ-ರ್ಣಿತ
ಕರ್ತವ್ಯ
ಕರ್ತ-ವ್ಯ-ಗಳನ್ನೆಲ್ಲ
ಕರ್ತ-ವ್ಯ-ಚ್ಯು-ತಿ-ಯಾದೀ
ಕರ್ತ-ವ್ಯದ
ಕರ್ತ-ವ್ಯ-ಲೋ-ಪ-ವಾ-ಗು-ವು-ದೆಂದು
ಕರ್ತ-ವ್ಯ-ವನ್ನು
ಕರ್ತ-ವ್ಯ-ವ-ಲ್ಲವೆ
ಕರ್ತ-ವ್ಯ-ವಿದೆ
ಕರ್ತ-ವ್ಯ-ವೆಂದು
ಕರ್ತ-ವ್ಯ-ವೆಂದೇ
ಕರ್ತ-ವ್ಯ-ವೇ-ನಿ-ದ್ದರೂ
ಕರ್ತೃ
ಕರ್ತೃ-ಗ-ಳೆಂದೂ
ಕರ್ತೃ-ವನ್ನು
ಕರ್ತೇ
ಕರ್ನಲ್
ಕರ್ನಾ-ಟ-ಕದ
ಕರ್ಮ
ಕರ್ಮ-ಗಿರ್ಮ
ಕರ್ಮಕ್ಕೆ
ಕರ್ಮ-ಕ್ಷೇ-ತ್ರ-ದಲ್ಲೇ
ಕರ್ಮ-ಗಳನ್ನು
ಕರ್ಮ-ಗಳನ್ನೂ
ಕರ್ಮ-ಗಳಿಂದ
ಕರ್ಮ-ತ್ಯಾಗ
ಕರ್ಮದ
ಕರ್ಮ-ಫ-ಲ-ತ್ಯಾ-ಗದ
ಕರ್ಮ-ಮಾ-ಡಲು
ಕರ್ಮ-ಯೋಗ
ಕರ್ಮ-ಯೋ-ಗ-ಗಳ
ಕರ್ಮ-ವಾ-ದರೆ
ಕರ್ಮ-ಸ-ಮು-ದ್ರದ
ಕಲ-ಕಿ-ಬಿ-ಟ್ಟಿತು
ಕಲ-ಕಿ-ಬಿ-ಡೋಣ
ಕಲಾ-ಕಾರ
ಕಲಾ-ಕೃ-ತಿ-ಗಳ
ಕಲಾ-ಕೃ-ತಿ-ಗಳು
ಕಲಾ-ಕೃ-ತಿ-ಯನ್ನೋ
ಕಲಾ-ಕೇಂ-ದ್ರ-ಗಳನ್ನೂ
ಕಲಾ-ಕೌ-ಶ-ಲ-ವನ್ನು
ಕಲಾ-ತ್ಮಿ-ಕ-ತೆಯು
ಕಲಾ-ಪ-ಗಳ
ಕಲಾ-ಮಂ-ದಿ-ರ-ಗಳು
ಕಲಾ-ವಿದ
ಕಲಾ-ವಿ-ದ-ನಿಗೆ
ಕಲಾ-ಶಾ-ಲೆ-ಗಳನ್ನೂ
ಕಲಿ-ಗಾಲ
ಕಲಿತ
ಕಲಿ-ತ-ಪಾ-ಠ-ವನ್ನು
ಕಲಿ-ತರು
ಕಲಿ-ತರೆ
ಕಲಿ-ತಿದ್ದ
ಕಲಿ-ತಿ-ದ್ದರು
ಕಲಿ-ತಿ-ದ್ದೇನೆ
ಕಲಿತು
ಕಲಿ-ತು-ಕೊಂಡೆ
ಕಲಿ-ತು-ಕೊ-ಳ್ಳ-ಬೇ-ಕಾ-ದದ್ದು
ಕಲಿ-ತು-ಕೊಳ್ಳು
ಕಲಿ-ತು-ಬಿ-ಡು-ತ್ತೇನೆ
ಕಲಿತೆ
ಕಲಿ-ತೇವೋ
ಕಲಿ-ಯ-ಬೇ-ಕಾ-ಗಿತ್ತು
ಕಲಿ-ಯ-ಬೇ-ಕಾ-ಗಿದೆ
ಕಲಿ-ಯ-ಬೇ-ಕಾ-ಯಿತೆ
ಕಲಿ-ಯ-ಬೇ-ಕೆಂಬ
ಕಲಿ-ಯಲು
ಕಲಿ-ಯ-ವುದನ್ನು
ಕಲಿ-ಯಿರಿ
ಕಲಿ-ಯುಗ
ಕಲಿ-ಯುವ
ಕಲಿ-ಯು-ವಂ-ತಾ-ಗ-ಬೇಕು
ಕಲಿ-ಯು-ವಂತೆ
ಕಲಿ-ಯು-ವುದನ್ನು
ಕಲಿ-ಯು-ವು-ದರ
ಕಲಿ-ಯು-ವು-ದ-ರಲ್ಲಿ
ಕಲಿ-ಸ-ಬೇ-ಕಾ-ದ-ದ್ದೇನೂ
ಕಲಿ-ಸ-ಲಾ-ಗ-ದಿ-ದ್ದು-ದನ್ನು
ಕಲಿ-ಸಿ-ಕೊ-ಟ್ಟಿ-ರು-ವರೋ
ಕಲಿ-ಸಿ-ಕೊ-ಡಲು
ಕಲಿ-ಸಿ-ದ್ದೀರಿ
ಕಲಿ-ಸಿ-ದ್ದೇನೆ
ಕಲು
ಕಲು-ಷ-ವನು
ಕಲೆ
ಕಲೆ-ವಿ-ಜ್ಞಾ-ನ-ಗಳಲ್ಲಿ
ಕಲೆ-ಸಂ-ಸ್ಕೃ-ತಿ-ಗಳನ್ನು
ಕಲೆ-ಕ್ಟರ್
ಕಲೆ-ಗಳು
ಕಲೆಗೆ
ಕಲೆತ
ಕಲೆತು
ಕಲೆ-ಯಲ್ಲಿ
ಕಲೆ-ಹಾಕಿ
ಕಲೇಸ್
ಕಲ್ಕತ್ತ
ಕಲ್ಕ-ತ್ತಕ್ಕೆ
ಕಲ್ಕ-ತ್ತ-ಗಳಲ್ಲಿ
ಕಲ್ಕ-ತ್ತದ
ಕಲ್ಕ-ತ್ತ-ದಲ್ಲಿ
ಕಲ್ಕ-ತ್ತ-ದಲ್ಲೂ
ಕಲ್ಕ-ತ್ತ-ದಲ್ಲೇ
ಕಲ್ಕ-ತ್ತ-ದ-ಲ್ಲೊಂದು
ಕಲ್ಕ-ತ್ತ-ದಿಂದ
ಕಲ್ಕ-ತ್ತವು
ಕಲ್ಕತ್ತಾ
ಕಲ್ಪ-ತರು
ಕಲ್ಪ-ನಾ-ಶ-ಕ್ತಿಗೂ
ಕಲ್ಪನೆ
ಕಲ್ಪ-ನೆ-ಗಳನ್ನು
ಕಲ್ಪ-ನೆ-ಗ-ಳ-ನ್ನು-ಆ-ದ-ರ್ಶ-ಗಳನ್ನು
ಕಲ್ಪ-ನೆ-ಗಳನ್ನೂ
ಕಲ್ಪ-ನೆ-ಗಳನ್ನೆಲ್ಲ
ಕಲ್ಪ-ನೆ-ಗಳು
ಕಲ್ಪ-ನೆ-ಗಳೇ
ಕಲ್ಪ-ನೆಗೆ
ಕಲ್ಪ-ನೆಯ
ಕಲ್ಪ-ನೆ-ಯನ್ನು
ಕಲ್ಪ-ನೆ-ಯ-ಲ್ಲಿ-ಆತ್ಮ
ಕಲ್ಪ-ನೆ-ಯಾ-ಗು-ತ್ತಿತ್ತು
ಕಲ್ಪ-ನೆ-ಯಾ-ಯಿತು
ಕಲ್ಪ-ನೆ-ಯಿ-ರು-ವಂ-ತಹ
ಕಲ್ಪ-ನೆಯು
ಕಲ್ಪ-ನೆಯೇ
ಕಲ್ಪಿ-ಸ-ಬೇ-ಕೆಂ-ಬುದು
ಕಲ್ಪಿ-ಸಿ-ಕೊಂ-ಡಿ-ದ್ದರು
ಕಲ್ಪಿ-ಸಿ-ಕೊಂ-ಡಿ-ರ-ಲಿ-ಲ್ಲ-ಅ-ಷ್ಟೊಂದು
ಕಲ್ಪಿ-ಸಿ-ಕೊ-ಟ್ಟದ್ದೇ
ಕಲ್ಪಿ-ಸಿ-ಕೊ-ಡು-ವಂತೆ
ಕಲ್ಪಿ-ಸಿ-ಕೊ-ಳ್ಳ-ಲಾರ
ಕಲ್ಪಿ-ಸಿದ
ಕಲ್ಯಾ-ಣ-ಕ್ಕಾಗಿ
ಕಲ್ಯಾ-ಣ-ದೊಂ-ದಿಗೆ
ಕಲ್ಲಿನ
ಕಲ್ಲು
ಕಲ್ಲು-ಗಳಿಂದ
ಕಲ್ಲು-ಮು-ಳ್ಳು-ಗಳಿಂದ
ಕಲ್ಲೆ-ಸೆದ
ಕಲ್ಲೇ
ಕಳಂಕ
ಕಳಂ-ಕ-ರ-ಹಿತ
ಕಳ-ಕಳಿ
ಕಳ-ಕ-ಳಿಯ
ಕಳ-ಕ-ಳಿ-ಯನ್ನು
ಕಳ-ಕ-ಳಿ-ಯಿಂದ
ಕಳಚಿ
ಕಳ-ಚಿತು
ಕಳ-ಚಿ-ಬಿ-ಡ-ಬ-ಹುದು
ಕಳ-ಚಿ-ಹೋಗಿ
ಕಳ-ವ-ಳ-ಗ-ಳಿಂ-ದಾಗಿ
ಕಳ-ವ-ಳ-ಗೊಂ-ಡಿ-ದ್ದರು
ಕಳ-ವ-ಳ-ವಾ-ಯಿತು
ಕಳ-ಸಿ-ಕೊ-ಟ್ಟರು
ಕಳಿ-ಸದೆ
ಕಳಿ-ಸ-ಬ-ಹುದು
ಕಳಿ-ಸ-ಬೇಕು
ಕಳಿ-ಸ-ಲಾ-ಗ-ದಿ-ದ್ದು-ದರ
ಕಳಿ-ಸ-ಲಾ-ಗು-ವುದು
ಕಳಿ-ಸ-ಲಾ-ರೆಯಾ
ಕಳಿ-ಸಲು
ಕಳಿ-ಸ-ಲೆಂದು
ಕಳಿ-ಸ-ಲೇ-ಬೇಕು
ಕಳಿಸಿ
ಕಳಿ-ಸಿ-ಕೊಟ್ಟ
ಕಳಿ-ಸಿ-ಕೊ-ಟ್ಟ-ದ್ದನ್ನು
ಕಳಿ-ಸಿ-ಕೊ-ಟ್ಟ-ರಾ-ದರೂ
ಕಳಿ-ಸಿ-ಕೊ-ಟ್ಟರು
ಕಳಿ-ಸಿ-ಕೊ-ಟ್ಟಿತು
ಕಳಿ-ಸಿ-ಕೊ-ಟ್ಟಿದ್ದ
ಕಳಿ-ಸಿ-ಕೊ-ಟ್ಟಿ-ದ್ದರು
ಕಳಿ-ಸಿ-ಕೊ-ಟ್ಟಿ-ದ್ದಾರೆ
ಕಳಿ-ಸಿ-ಕೊಟ್ಟು
ಕಳಿ-ಸಿ-ಕೊ-ಟ್ಟು-ಬಿ-ಟ್ಟರು
ಕಳಿ-ಸಿ-ಕೊಟ್ಟೇ
ಕಳಿ-ಸಿ-ಕೊಡ
ಕಳಿ-ಸಿ-ಕೊ-ಡ-ಬೇಕು
ಕಳಿ-ಸಿ-ಕೊ-ಡ-ಬೇ-ಕೆಂದು
ಕಳಿ-ಸಿ-ಕೊ-ಡ-ಬೇ-ಕೆಂದೂ
ಕಳಿ-ಸಿ-ಕೊ-ಡ-ಬೇ-ಕೆಂಬ
ಕಳಿ-ಸಿ-ಕೊ-ಡಲು
ಕಳಿ-ಸಿ-ಕೊಡಿ
ಕಳಿ-ಸಿ-ಕೊ-ಡಿ-ಸಾ-ಧ್ಯ-ವಾ-ದಷ್ಟು
ಕಳಿ-ಸಿ-ಕೊ-ಡುತ್ತ
ಕಳಿ-ಸಿ-ಕೊ-ಡು-ತ್ತಾ-ನೆಯೋ
ಕಳಿ-ಸಿ-ಕೊ-ಡು-ತ್ತಾರೆ
ಕಳಿ-ಸಿ-ಕೊ-ಡುತ್ತಿ
ಕಳಿ-ಸಿ-ಕೊ-ಡು-ತ್ತಿ-ದ್ದರು
ಕಳಿ-ಸಿ-ಕೊ-ಡು-ತ್ತೇನೆ
ಕಳಿ-ಸಿ-ಕೊ-ಡು-ತ್ತೇವೆ
ಕಳಿ-ಸಿ-ಕೊ-ಡುವ
ಕಳಿ-ಸಿ-ಕೊ-ಡು-ವಂತೆ
ಕಳಿ-ಸಿ-ಕೊ-ಡು-ವುದು
ಕಳಿ-ಸಿದ
ಕಳಿ-ಸಿ-ದ-ಪ್ರ-ಕಾಂಡ
ಕಳಿ-ಸಿ-ದರು
ಕಳಿ-ಸಿ-ದರೆ
ಕಳಿ-ಸಿ-ದಳು
ಕಳಿ-ಸಿದ್ದ
ಕಳಿ-ಸಿ-ದ್ದರೂ
ಕಳಿ-ಸಿ-ದ್ದ-ರೆಂ-ಬುದು
ಕಳಿ-ಸಿ-ದ್ದ-ವರ
ಕಳಿ-ಸಿ-ದ್ದ-ವರು
ಕಳಿ-ಸಿ-ದ್ದುದು
ಕಳಿ-ಸಿ-ದ್ದೇ-ವೆಯೆ
ಕಳಿ-ಸಿ-ಬಿ-ಟ್ಟಿದ್ದ
ಕಳಿ-ಸಿ-ಬಿ-ಟ್ಟಿ-ದ್ದೇನೆ
ಕಳಿ-ಸಿರ
ಕಳಿ-ಸಿರಿ
ಕಳಿ-ಸಿ-ರು-ವುದು
ಕಳಿ-ಸು-ತ್ತಿ-ದ್ದೀರಿ
ಕಳಿ-ಸು-ತ್ತಿ-ದ್ದೇವೆ
ಕಳಿ-ಸು-ತ್ತೀರಿ
ಕಳಿ-ಸು-ತ್ತೇನೆ
ಕಳಿ-ಸುವ
ಕಳಿ-ಸು-ವುದನ್ನು
ಕಳಿ-ಸು-ವುದು
ಕಳಿ-ಸೋಣ
ಕಳು-ಹಿ-ಸಲಿ
ಕಳು-ಹಿ-ಸಿ-ಕೊ-ಡು-ತ್ತಾ-ನೆಯೋ
ಕಳೆದ
ಕಳೆ-ದಂ-ತೆಲ್ಲ
ಕಳೆ-ದ-ಮೇಲೆ
ಕಳೆ-ದರು
ಕಳೆ-ದರೂ
ಕಳೆ-ದಿತ್ತು
ಕಳೆ-ದಿ-ದ್ದರು
ಕಳೆ-ದಿ-ದ್ದರೂ
ಕಳೆ-ದಿ-ದ್ದಾನೆ
ಕಳೆ-ದಿ-ದ್ದೇನೆ
ಕಳೆ-ದಿ-ರ-ಬ-ಹುದು
ಕಳೆ-ದಿ-ರ-ಬ-ಹುದೆ
ಕಳೆದು
ಕಳೆ-ದು-ಕೊಂಡ
ಕಳೆ-ದು-ಕೊಂ-ಡಂತೆ
ಕಳೆ-ದು-ಕೊಂ-ಡದ್ದು
ಕಳೆ-ದು-ಕೊಂ-ಡರು
ಕಳೆ-ದು-ಕೊಂ-ಡ-ವ-ನಂತೆ
ಕಳೆ-ದು-ಕೊಂ-ಡಿತೋ
ಕಳೆ-ದು-ಕೊಂ-ಡಿ-ದ್ದಳು
ಕಳೆ-ದು-ಕೊಂ-ಡಿ-ದ್ದೀ-ರ-ಲ್ಲ-ನೀವು
ಕಳೆ-ದು-ಕೊಂಡು
ಕಳೆ-ದು-ಕೊಂ-ಡು-ಬಿ-ಟ್ಟಿ-ದ್ದೇವೆ
ಕಳೆ-ದು-ಕೊಂ-ಡು-ಬಿ-ಡು-ತ್ತೇ-ವೆಯೋ
ಕಳೆ-ದು-ಕೊ-ಳ್ಳದ
ಕಳೆ-ದು-ಕೊ-ಳ್ಳದೆ
ಕಳೆ-ದು-ಕೊ-ಳ್ಳ-ಬ-ಹು-ದೆಂದು
ಕಳೆ-ದು-ಕೊ-ಳ್ಳ-ಬಾ-ರದು
ಕಳೆ-ದು-ಕೊ-ಳ್ಳ-ಲಾ-ಗದು
ಕಳೆ-ದು-ಕೊ-ಳ್ಳು-ತ್ತಿ-ರ-ಲಿಲ್ಲ
ಕಳೆ-ದು-ಕೊ-ಳ್ಳು-ವಂ-ತಹ
ಕಳೆ-ದು-ಕೊ-ಳ್ಳು-ವಂ-ತಾ-ಗು-ತ್ತದೆ
ಕಳೆ-ದುವು
ಕಳೆ-ದು-ಹೋ-ದ-ವರ
ಕಳೆ-ದೆವು
ಕಳೆ-ಯನ್ನು
ಕಳೆ-ಯ-ಲಾ-ಗು-ವಂ-ತಹ
ಕಳೆ-ಯ-ಲಾ-ರಂ-ಭಿ-ಸಿದ
ಕಳೆ-ಯಲು
ಕಳೆ-ಯಿತು
ಕಳೆ-ಯುತ್ತ
ಕಳೆ-ಯು-ತ್ತಿ-ದ್ದರು
ಕಳೆ-ಯುವ
ಕಳೆ-ಯು-ವಂತೆ
ಕಳೆ-ಯು-ವ-ವರ
ಕಳೆ-ಯು-ವ-ಷ್ಟ-ರಲ್ಲಿ
ಕಳೆಯೇ
ಕಳೆವ
ಕಳ್ಳ
ಕಳ್ಳ-ಕಾ-ಕರು
ಕಳ್ಳ-ಕೊ-ರ-ಮರು
ಕಳ್ಳನ
ಕಳ್ಳ-ನನ್ನೇ
ಕಳ್ಳರು
ಕಳ್ಳ-ರು-ಸಂ-ನ್ಯಾ-ಸಿ-ಗ-ಳಿ-ಗೆಲ್ಲ
ಕವ-ಚ-ವನ್ನು
ಕವ-ಚ-ವೆಂದರೆ
ಕವ-ಚವೇ
ಕವ-ನ-ಗಳನ್ನು
ಕವ-ನ-ಗಳಲ್ಲಿ
ಕವ-ನದ
ಕವ-ನ-ವನ್ನು
ಕವ-ನ-ವಿತ್ತು
ಕವ-ನವೇ
ಕವ-ನ-ವೊಂ-ದನ್ನು
ಕವ-ಯಿತ್ರಿ
ಕವಿ-ಗಳ
ಕವಿ-ಗಳನ್ನೂ
ಕವಿ-ದಿತ್ತು
ಕವಿ-ಯು-ತ್ತ-ದಲ್ಲ
ಕವಿ-ಯು-ವಂ-ತಾ-ದಾಗ
ಕವಿ-ಯು-ವಂತೆ
ಕಷಾಯ
ಕಷ್ಟ
ಕಷ್ಟ-ಸಂ-ಕ-ಟ-ಗಳನ್ನು
ಕಷ್ಟ-ಕರ
ಕಷ್ಟ-ಕ-ರ-ವಾದ
ಕಷ್ಟ-ಕಾ-ರ್ಪ-ಣ್ಯ-ಗಳನ್ನೆಲ್ಲ
ಕಷ್ಟ-ಕಾ-ರ್ಪ-ಣ್ಯ-ಗ-ಳೆಲ್ಲ
ಕಷ್ಟ-ಕೋ-ಟ-ಲೆ-ಗಳ
ಕಷ್ಟ-ಕ್ಕಿ-ಟ್ಟು-ಕೊಂ-ಡಿತು
ಕಷ್ಟಕ್ಕೆ
ಕಷ್ಟ-ಗಳ
ಕಷ್ಟ-ಗಳನ್ನು
ಕಷ್ಟ-ಗಳನ್ನೂ
ಕಷ್ಟ-ಗಳನ್ನೆಲ್ಲ
ಕಷ್ಟ-ಗ-ಳಿಗೆ
ಕಷ್ಟ-ಗಳು
ಕಷ್ಟ-ಗ-ಳು-ಎ-ದು-ರಿ-ಸಿದ
ಕಷ್ಟ-ಗ-ಳೆಂ-ಥದು
ಕಷ್ಟ-ಗ-ಳೆ-ಲ್ಲ-ವನ್ನೂ
ಕಷ್ಟ-ತೊಂ-ದ-ರೆ-ಗಳ
ಕಷ್ಟದ
ಕಷ್ಟ-ದ-ಲ್ಲಿ-ರು-ವ-ವ-ರಿಗೆ
ಕಷ್ಟ-ದ್ದೆಂ-ಬುದು
ಕಷ್ಟ-ಪ-ಟ್ಟರೂ
ಕಷ್ಟ-ಪ-ಟ್ಟಿ-ದ್ದೇನೆ
ಕಷ್ಟ-ಪಟ್ಟು
ಕಷ್ಟ-ಪ-ಟ್ಟು-ಕೊಂಡು
ಕಷ್ಟ-ಪ-ಡ-ಬೇ-ಕಾ-ಯಿತು
ಕಷ್ಟ-ಪ-ಡುತ್ತ
ಕಷ್ಟ-ಪ-ಡು-ತ್ತಿ-ರು-ವುದನ್ನು
ಕಷ್ಟ-ಪ-ಡು-ವುದೂ
ಕಷ್ಟ-ವ-ನ್ನ-ನು-ಭ-ವಿ-ಸ-ಬೇಕಾ
ಕಷ್ಟ-ವ-ನ್ನಾ-ದರೂ
ಕಷ್ಟ-ವನ್ನು
ಕಷ್ಟ-ವಾ-ಗ-ಬ-ಹು-ದೆಂದು
ಕಷ್ಟ-ವಾಗಿ
ಕಷ್ಟ-ವಾ-ಗಿತ್ತು
ಕಷ್ಟ-ವಾ-ಗಿ-ಬಿ-ಡು-ತ್ತಿತ್ತು
ಕಷ್ಟ-ವಾ-ಗು-ತ್ತಿ-ದ್ದರೆ
ಕಷ್ಟ-ವಾ-ದರೂ
ಕಷ್ಟ-ವಾ-ಯಿತು
ಕಷ್ಟವೆ
ಕಷ್ಟ-ವೆಂ-ಬಂತೆ
ಕಸ
ಕಸ-ರ-ತ್ತಲ್ಲ
ಕಸ-ರ-ತ್ತಿನ
ಕಸ-ರತ್ತು
ಕಸ-ರ-ತ್ತು-ಗಳನ್ನು
ಕಸ-ರ-ತ್ತು-ಗಳು
ಕಸಿ-ದು-ಕೊಂ-ಡರು
ಕಸಿ-ದು-ಕೊ-ಳ್ಳಲು
ಕಸಿ-ಮಾ-ಡುವ
ಕಸಿ-ವಿ-ಸಿ-ನಾ-ಚಿ-ಕೆ-ಯಾ-ಯಿತು
ಕಸಿ-ವಿ-ಸಿ-ಗೊ-ಳ್ಳ-ಲಿಲ್ಲ
ಕಸಿ-ವಿ-ಸಿ-ಯಾ-ಗು-ತ್ತಿತ್ತು
ಕಸಿ-ವಿ-ಸಿ-ಯುಂ-ಟು-ಮಾ-ಡು-ತ್ತಿ-ದ್ದುವು
ಕಸು-ಬನ್ನು
ಕಸುಬು
ಕಹಾ-ವತ
ಕಹಾ-ವತ್
ಕಹಿ
ಕಹಿ-ಭಾ-ವ-ನೆ-ಗಳು
ಕಾಂಕ್ಷಿ-ಗ-ಳ-ಲ್ಲವೋ
ಕಾಂಗ-ರಳ
ಕಾಂಗರ್
ಕಾಂಗ್ರೆಸ್ನ
ಕಾಂಗ್ರೆ-ಸ್ನಲ್ಲಿ
ಕಾಂಚನ
ಕಾಂಚ-ನ-ಮೃ-ಗ-ವಲ್ಲ
ಕಾಂತಿ-ಮಯ
ಕಾಂತಿ-ಯನ್ನು
ಕಾಂತಿ-ಯಿಂದ
ಕಾಕ-ತಾ-ಳೀಯ
ಕಾಕ-ತಾ-ಳೀ-ಯ-ವಲ್ಲ
ಕಾಕ-ತಾ-ಳೀ-ಯ-ವಾಗಿ
ಕಾಗ-ಕ್ಕ
ಕಾಗದ
ಕಾಗ-ದದ
ಕಾಗ-ದ-ವನ್ನು
ಕಾಗೆ-ಯಂತೆ
ಕಾಚಃ-ಕಾಚಃ
ಕಾಟ-ಮಾ-ರನ್
ಕಾಡಿ
ಕಾಡಿಗೆ
ಕಾಡಿ-ದರು
ಕಾಡಿ-ದ್ದರು
ಕಾಡಿನ
ಕಾಡಿ-ನಿಂದ
ಕಾಡು
ಕಾಡು-ಜ-ನರ
ಕಾಡು-ಜ-ನ-ರಿಂದ
ಕಾಡು-ಜ-ನ-ರೆಂಬ
ಕಾಡು-ತ್ತಿತ್ತು
ಕಾಡು-ತ್ತಿ-ದ್ದರು
ಕಾಡು-ತ್ತಿ-ದ್ದಳು
ಕಾಡು-ತ್ತಿ-ದ್ದ-ವಷ್ಟೆ
ಕಾಡು-ತ್ತೀರಿ
ಕಾಡು-ಮೇಡು
ಕಾಡು-ವ-ವರು
ಕಾಡು-ಹ-ರ-ಟೆಯ
ಕಾಡು-ಹ-ರ-ಟೆ-ಯಲ್ಲೇ
ಕಾಣ
ಕಾಣ-ತೊ-ಡ-ಗಿ-ದರು
ಕಾಣದ
ಕಾಣ-ದಷ್ಟು
ಕಾಣ-ದಿ-ದ್ದರೆ
ಕಾಣ-ದಿರು
ಕಾಣ-ದಿ-ರು-ವ-ವ-ರಿಗೇ
ಕಾಣ-ಬಂದ
ಕಾಣ-ಬಲ್ಲ
ಕಾಣ-ಬ-ಲ್ಲ-ವ-ರಾ-ಗಿ-ದ್ದರು
ಕಾಣ-ಬ-ಹು-ದಾ-ಗಿತ್ತು
ಕಾಣ-ಬ-ಹು-ದಾ-ಗಿದೆ
ಕಾಣ-ಬ-ಹುದು
ಕಾಣ-ಬೇಕಾ
ಕಾಣ-ಬೇ-ಕಾ-ದದ್ದು
ಕಾಣ-ಬೇ-ಕಾ-ದರೆ
ಕಾಣ-ಬೇಕು
ಕಾಣ-ಲಾ-ರೆವು
ಕಾಣ-ಲಿ-ದ್ದೇವೆ
ಕಾಣ-ಲಿಲ್ಲ
ಕಾಣ-ಲಿ-ಲ್ಲವೆ
ಕಾಣಲು
ಕಾಣಲೇ
ಕಾಣ-ಸಿ-ಗ-ಬ-ಹುದೊ
ಕಾಣ-ಸಿ-ಗಲಿ
ಕಾಣ-ಸಿ-ಗು-ತ್ತದೆ
ಕಾಣ-ಸಿ-ಗುವ
ಕಾಣಿಕೆ
ಕಾಣಿ-ಕೆಯ
ಕಾಣಿ-ಕೆ-ಯನ್ನು
ಕಾಣಿ-ಕೆ-ಯನ್ನೇ
ಕಾಣಿ-ಕೆ-ಯಾಗಿ
ಕಾಣಿ-ಕೆ-ಯಾ-ಗಿತ್ತ
ಕಾಣಿ-ಕೆ-ಯೊಂ-ದಿಗೆ
ಕಾಣಿ-ಸದೆ
ಕಾಣಿಸಿ
ಕಾಣಿ-ಸಿ-ಕೊಂಡ
ಕಾಣಿ-ಸಿ-ಕೊಂ-ಡರು
ಕಾಣಿ-ಸಿ-ಕೊಂ-ಡಷ್ಟು
ಕಾಣಿ-ಸಿ-ಕೊಂ-ಡಾಗ
ಕಾಣಿ-ಸಿ-ಕೊಂ-ಡಿತು
ಕಾಣಿ-ಸಿ-ಕೊಂಡು
ಕಾಣಿ-ಸಿ-ಕೊ-ಳ್ಳ-ತೊ-ಡ-ಗು-ತ್ತವೆ
ಕಾಣಿ-ಸಿ-ಕೊ-ಳ್ಳ-ದಿ-ರು-ವುದೇ
ಕಾಣಿ-ಸಿ-ಕೊ-ಳ್ಳಲು
ಕಾಣಿ-ಸಿ-ಕೊಳ್ಳು
ಕಾಣಿ-ಸಿ-ಕೊ-ಳ್ಳು-ತ್ತದೆ
ಕಾಣಿ-ಸಿ-ಕೊ-ಳ್ಳು-ತ್ತಾನೆ
ಕಾಣಿ-ಸಿತು
ಕಾಣಿ-ಸಿ-ದರೇ
ಕಾಣಿ-ಸಿ-ರ-ಲಿಲ್ಲ
ಕಾಣಿ-ಸು-ತ್ತದೆ
ಕಾಣಿ-ಸು-ತ್ತಿತ್ತು
ಕಾಣಿ-ಸು-ವಂ-ತಹ
ಕಾಣು
ಕಾಣು-ತ್ತದೆ
ಕಾಣು-ತ್ತ-ದೆ-ಆ-ದ್ದ-ರಿಂದ
ಕಾಣು-ತ್ತವೆ
ಕಾಣು-ತ್ತಾ-ನೆಯೋ
ಕಾಣು-ತ್ತಾರೆ
ಕಾಣು-ತ್ತಾ-ರೆ-ಅದೇ
ಕಾಣು-ತ್ತಾರೋ
ಕಾಣುತ್ತಿ
ಕಾಣು-ತ್ತಿತ್ತು
ಕಾಣು-ತ್ತಿದೆ
ಕಾಣು-ತ್ತಿದ್ದ
ಕಾಣು-ತ್ತಿ-ದ್ದ-ರ-ಲ್ಲದೆ
ಕಾಣು-ತ್ತಿ-ದ್ದರು
ಕಾಣು-ತ್ತಿ-ದ್ದ-ರೆಂ-ಬು-ದನ್ನು
ಕಾಣು-ತ್ತಿ-ದ್ದ-ರೆಂ-ಬುದು
ಕಾಣು-ತ್ತಿ-ದ್ದಾರೆ
ಕಾಣು-ತ್ತಿ-ದ್ದೀರಿ
ಕಾಣು-ತ್ತಿ-ದ್ದು-ದ-ರಿಂದ
ಕಾಣು-ತ್ತಿ-ದ್ದುವು
ಕಾಣು-ತ್ತಿ-ದ್ದೇನೆ
ಕಾಣು-ತ್ತಿ-ದ್ದೇ-ನೆ-ಯಾರು
ಕಾಣು-ತ್ತಿ-ದ್ದೇವೆ
ಕಾಣು-ತ್ತಿ-ದ್ದೇ-ವೆ-ಪಾ-ಕಿ-ಸ್ತಾ-ನದ
ಕಾಣು-ತ್ತಿರ
ಕಾಣು-ತ್ತಿ-ರ-ಲಿಲ್ಲ
ಕಾಣು-ತ್ತಿ-ರುವ
ಕಾಣು-ತ್ತಿ-ರು-ವ-ವ-ರಾ-ದರೂ
ಕಾಣುತ್ತೀ
ಕಾಣು-ತ್ತೀಯೆ
ಕಾಣು-ತ್ತೀರಿ
ಕಾಣು-ತ್ತೇನೆ
ಕಾಣು-ತ್ತೇವೆ
ಕಾಣುವ
ಕಾಣು-ವಂ-ತಾ-ಗಿದೆ
ಕಾಣು-ವಷ್ಟು
ಕಾಣುವು
ಕಾಣು-ವು-ದ-ಕ್ಕಿಂತ
ಕಾಣು-ವು-ದಿಲ್ಲ
ಕಾಣು-ವುದು
ಕಾಣು-ವು-ದೆಂ-ದರೆ
ಕಾಣು-ವು-ದೆಂದು
ಕಾಣು-ವು-ದೇನೋ
ಕಾಣು-ವು-ದೊಂದು
ಕಾಣೆ
ಕಾಣೆ-ಯಾ-ಗಿ-ದ್ದರೆ
ಕಾತರ
ಕಾತ-ರ-ಗಳಿಂದ
ಕಾತ-ರ-ಗೊಂ-ಡಿದೆ
ಕಾತ-ರ-ತೆ-ಗಳಿಂದ
ಕಾತ-ರ-ತೆ-ಯಿಂದ
ಕಾತ-ರದ
ಕಾತ-ರ-ದಿಂದ
ಕಾತ-ರ-ರಾಗಿ
ಕಾತ-ರ-ರಾ-ಗಿ-ದರೂ
ಕಾತ-ರ-ರಾ-ಗಿದ್ದ
ಕಾತ-ರ-ರಾ-ಗಿ-ದ್ದರು
ಕಾತ-ರವು
ಕಾತ-ರ-ಹೃ-ದ-ಯ-ರಾಗಿ
ಕಾತ-ರಿ-ಸಿತು
ಕಾಥೇ-ವಾಡ
ಕಾಥೇ-ವಾ-ಡದ
ಕಾಥೇ-ವಾ-ಡ-ದಲ್ಲಿ
ಕಾದ
ಕಾದಂ-ಬರಿ
ಕಾದ-ಮೇಲೆ
ಕಾದರು
ಕಾದಿತ್ತು
ಕಾದಿ-ದ್ದರು
ಕಾದಿ-ರಿ-ಸ-ಲಾ-ಗಿತ್ತು
ಕಾದಿ-ರಿ-ಸ-ಲಾಗಿದೆ
ಕಾದಿ-ರಿ-ಸಲು
ಕಾದಿ-ರಿ-ಸಿ-ಕೊ-ಳ್ಳ-ಬೇ-ಕಾದ
ಕಾದಿ-ರಿ-ಸಿದ್ದ
ಕಾದಿ-ರು-ವುದು
ಕಾದು
ಕಾದು-ಕಾದು
ಕಾದು-ಕು-ಳಿ-ತಿ-ದ್ದರು
ಕಾದು-ನಿಂ-ತಿದ್ದ
ಕಾದು-ನಿಂ-ತಿ-ದ್ದಾನೆ
ಕಾನಂದ
ಕಾನಂ-ದರ
ಕಾನಂ-ದ-ಹೀಗೆ
ಕಾನೂ-ನನ್ನು
ಕಾನೂ-ನಿನ
ಕಾನೂನು
ಕಾನೂನೂ
ಕಾನ್ಫ-ರೆ-ನ್ಸಸ್
ಕಾನ್ವೆಂಟ್
ಕಾನ್ವೇ
ಕಾಪಾ-ಡ-ಬೇಕು
ಕಾಪಾ-ಡಲಿ
ಕಾಪಾಡಿ
ಕಾಪಾ-ಡಿ-ಕೊಂ-ಡರು
ಕಾಪಾ-ಡಿದ
ಕಾಪಾ-ಡಿ-ದ್ದಾನೆ
ಕಾಪಾ-ಡು-ವಂತೆ
ಕಾಪಾ-ಲಿ-ಕರು
ಕಾಫಿ
ಕಾಫಿಯ
ಕಾಫಿ-ಯನ್ನು
ಕಾಮ-ಧೇನು
ಕಾಮ-ಭಾ-ವನೆ
ಕಾಮ-ಭಾ-ವ-ನೆ-ಯನ್ನು
ಕಾಯ-ಬೇ-ಕಾ-ಗಿತ್ತು
ಕಾಯ-ಬೇ-ಕಿತ್ತು
ಕಾಯಸ್ಥ
ಕಾಯಿದೆ
ಕಾಯಿ-ಲೆ-ಗಳಿಂದ
ಕಾಯಿ-ಲೆಗೆ
ಕಾಯಿ-ಲೆ-ಯನ್ನು
ಕಾಯಿ-ಲೆಯೂ
ಕಾಯಿ-ಲೆ-ಯೆಂದು
ಕಾಯುತ್ತ
ಕಾಯು-ತ್ತಿದ್ದ
ಕಾಯು-ತ್ತಿ-ದ್ದರೋ
ಕಾಯು-ತ್ತಿ-ದ್ದಾರೆ
ಕಾಯು-ತ್ತಿ-ದ್ದೆವು
ಕಾಯು-ತ್ತಿ-ದ್ದೇನೆ
ಕಾಯು-ತ್ತಿ-ದ್ದೇವೆ
ಕಾಯು-ತ್ತಿ-ರ-ಬೇ-ಕಾ-ಗು-ತ್ತದೆ
ಕಾಯು-ತ್ತಿ-ರು-ಇದೇ
ಕಾಯ್ಟ್
ಕಾಯ್ದು-ಕೊ-ಳ್ಳು-ವುದು
ಕಾರಣ
ಕಾರ-ಣ-ಹೂ-ರ-ಣ-ಗ-ಳೊಂದೂ
ಕಾರ-ಣ-ಕೊಟ್ಟು
ಕಾರ-ಣ-ಕ್ಕಲ್ಲ
ಕಾರ-ಣ-ಕ್ಕಾಗಿ
ಕಾರ-ಣ-ಕ್ಕಾ-ಗಿಯೇ
ಕಾರ-ಣ-ಕ್ಕಿಂತ
ಕಾರ-ಣಕ್ಕೆ
ಕಾರ-ಣಕ್ಕೇ
ಕಾರ-ಣಕ್ಕೋ
ಕಾರ-ಣ-ಗಳನ್ನು
ಕಾರ-ಣ-ಗಳಿ
ಕಾರ-ಣ-ಗಳಿಂದ
ಕಾರ-ಣ-ಗ-ಳಿ-ದ್ದುವು
ಕಾರ-ಣ-ಗ-ಳಿ-ದ್ದು-ವು-ರಾ-ಜ-ಕೀಯ
ಕಾರ-ಣ-ದಿಂದ
ಕಾರ-ಣ-ದಿಂ-ದಷ್ಟೇ
ಕಾರ-ಣ-ದೊಂ-ದಿಗೆ
ಕಾರ-ಣ-ನಾ-ಗಿದ್ದ
ಕಾರ-ಣ-ರಾ-ದ-ವರ
ಕಾರ-ಣ-ರಾ-ದ-ವ-ರನ್ನು
ಕಾರ-ಣ-ವನ್ನು
ಕಾರ-ಣ-ವನ್ನೂ
ಕಾರ-ಣ-ವ-ನ್ನೊಡ್ಡಿ
ಕಾರ-ಣ-ವಲ್ಲ
ಕಾರ-ಣ-ವಾ-ಗ-ಬ-ಹುದು
ಕಾರ-ಣ-ವಾ-ಗಿತ್ತು
ಕಾರ-ಣ-ವಾ-ಗಿ-ತ್ತೆಂದೂ
ಕಾರ-ಣ-ವಾ-ಗಿದೆ
ಕಾರ-ಣ-ವಾ-ಗಿ-ದೆಯೋ
ಕಾರ-ಣ-ವಾ-ಗು-ತ್ತಿತ್ತು
ಕಾರ-ಣ-ವಾದ
ಕಾರ-ಣ-ವಾ-ದು-ದ-ರಿಂದ
ಕಾರ-ಣ-ವಾ-ಯಿತು
ಕಾರ-ಣ-ವಾ-ವುದು
ಕಾರ-ಣ-ವಿತ್ತು
ಕಾರ-ಣ-ವಿ-ತ್ತು-ಶ್ರೀ-ರಾ-ಮ-ಕೃ-ಷ್ಣರ
ಕಾರ-ಣ-ವಿದೆ
ಕಾರ-ಣ-ವಿ-ರ-ಬ-ಹುದು
ಕಾರ-ಣ-ವಿಲ್ಲ
ಕಾರ-ಣ-ವಿ-ಷ್ಟೆ-ಸ್ವಾ-ಮೀಜಿ
ಕಾರ-ಣವೂ
ಕಾರ-ಣ-ವೆಂದರೆ
ಕಾರ-ಣ-ವೆಂದು
ಕಾರ-ಣ-ವೆಂಬ
ಕಾರ-ಣ-ವೆಂ-ಬು-ದನ್ನು
ಕಾರ-ಣ-ವೆಂ-ಬು-ದ-ರಲ್ಲಿ
ಕಾರ-ಣವೇ
ಕಾರ-ಣ-ವೇ-ನಿ-ದ್ದಿ-ರ-ಬ-ಹುದು
ಕಾರ-ಣ-ವೇನು
ಕಾರ-ಣ-ವೇನೂ
ಕಾರ-ಣ-ವೇನೆಂದರೆ
ಕಾರ-ಣ-ವೇ-ನೆಂದು
ಕಾರ-ಣ-ವೇ-ನೆಂ-ಬು-ದನ್ನು
ಕಾರ-ಣ-ವೇನೇ
ಕಾರ-ನಿಗೆ
ಕಾರಾ-ಗೃಹ
ಕಾರಾ-ಗೃ-ಹದ
ಕಾರಾ-ಗೃ-ಹ-ವನ್ನು
ಕಾರಿ-ಯಾದ
ಕಾರುತ್ತ
ಕಾರ್ಖಾ-ನೆ-ಗಳು
ಕಾರ್ಡಿ-ನಲ್
ಕಾರ್ನೆ-ಲಿಯಾ
ಕಾರ್ನೆ-ಲಿ-ಯಾಳ
ಕಾರ್ನೆ-ಲಿ-ಯಾ-ಳಿಗೆ
ಕಾರ್ಬಿನ್
ಕಾರ್ಬಿ-ನ್ನಳ
ಕಾರ್ಮಿ-ಕ-ರ-ನ್ನಾ-ಗಿ-ಸು-ತ್ತೇನೆ
ಕಾರ್ಯ
ಕಾರ್ಯ-ಕ-ರ್ತ-ರನ್ನು
ಕಾರ್ಯ-ಕ-ರ್ತ-ರಿಗೆ
ಕಾರ್ಯ-ಕ-ರ್ತರು
ಕಾರ್ಯ-ಕ-ಲಾಪ
ಕಾರ್ಯ-ಕ-ಲಾ-ಪ-ಗಳ
ಕಾರ್ಯ-ಕ-ಲಾ-ಪ-ಗಳನ್ನು
ಕಾರ್ಯ-ಕ-ಲಾ-ಪ-ಗ-ಳಲ್ಲೂ
ಕಾರ್ಯ-ಕಾರೀ
ಕಾರ್ಯ-ಕ್ಕಾಗಿ
ಕಾರ್ಯ-ಕ್ಕಿಂ-ತಲೂ
ಕಾರ್ಯಕ್ಕೂ
ಕಾರ್ಯಕ್ಕೆ
ಕಾರ್ಯ-ಕ್ರಮ
ಕಾರ್ಯ-ಕ್ರ-ಮ-ಕ್ಕಿಂತ
ಕಾರ್ಯ-ಕ್ರ-ಮಕ್ಕೆ
ಕಾರ್ಯ-ಕ್ರ-ಮ-ಗಳನ್ನು
ಕಾರ್ಯ-ಕ್ರ-ಮ-ಗ-ಳಲ್ಲೇ
ಕಾರ್ಯ-ಕ್ರ-ಮ-ಗ-ಳಿಗೂ
ಕಾರ್ಯ-ಕ್ರ-ಮ-ಗ-ಳಿ-ಗೆಲ್ಲ
ಕಾರ್ಯ-ಕ್ರ-ಮ-ಗಳು
ಕಾರ್ಯ-ಕ್ರ-ಮದ
ಕಾರ್ಯ-ಕ್ರ-ಮ-ದಲ್ಲಿ
ಕಾರ್ಯ-ಕ್ರ-ಮ-ವನ್ನು
ಕಾರ್ಯ-ಕ್ರ-ಮ-ವಿತ್ತು
ಕಾರ್ಯ-ಕ್ರ-ಮ-ವಿ-ರು-ತ್ತಿತ್ತು
ಕಾರ್ಯ-ಕ್ರ-ಮವು
ಕಾರ್ಯ-ಕ್ಷೇತ್ರ
ಕಾರ್ಯ-ಕ್ಷೇ-ತ್ರ-ವನ್ನು
ಕಾರ್ಯ-ಕ್ಷೇ-ತ್ರ-ವಾ-ಗಿ-ರ-ಲಿಲ್ಲ
ಕಾರ್ಯ-ಗತ
ಕಾರ್ಯ-ಗ-ತ-ಗೊ-ಳಿ-ಸ-ಬೇ-ಕೆಂದು
ಕಾರ್ಯ-ಗ-ತ-ಗೊ-ಳಿ-ಸ-ಲಾ-ಗುವ
ಕಾರ್ಯ-ಗ-ತ-ಗೊ-ಳಿ-ಸಲು
ಕಾರ್ಯ-ಗ-ತ-ಗೊ-ಳಿ-ಸಿದ
ಕಾರ್ಯ-ಗ-ತ-ಗೊ-ಳಿ-ಸುವ
ಕಾರ್ಯ-ಗ-ತ-ಗೊ-ಳಿ-ಸು-ವಲ್ಲಿ
ಕಾರ್ಯ-ಗ-ತ-ಗೊ-ಳಿ-ಸು-ವು-ದ-ಕ್ಕಾಗಿ
ಕಾರ್ಯ-ಗ-ತ-ವಾ-ಗಲೇ
ಕಾರ್ಯ-ಗ-ತ-ವಾಗಿ
ಕಾರ್ಯ-ಗ-ತ-ವಾ-ಗುವ
ಕಾರ್ಯ-ಗಳ
ಕಾರ್ಯ-ಗಳನ್ನು
ಕಾರ್ಯ-ಗಳನ್ನೂ
ಕಾರ್ಯ-ಗಳಲ್ಲಿ
ಕಾರ್ಯ-ಗ-ಳ-ಲ್ಲಿಯೂ
ಕಾರ್ಯ-ಗ-ಳಾ-ವುವು
ಕಾರ್ಯ-ಗಳಿಂದ
ಕಾರ್ಯ-ಗ-ಳಿಗೆ
ಕಾರ್ಯ-ಗ-ಳಿವೆ
ಕಾರ್ಯ-ಗಳು
ಕಾರ್ಯತಃ
ಕಾರ್ಯದ
ಕಾರ್ಯ-ದ-ಕ್ಷ-ರಾದ
ಕಾರ್ಯ-ದರ್ಶಿ
ಕಾರ್ಯ-ದ-ರ್ಶಿ-ಗ-ಳಾದ
ಕಾರ್ಯ-ದ-ರ್ಶಿ-ಯಾ-ಗಿ-ದ್ದಳು
ಕಾರ್ಯ-ದ-ರ್ಶಿ-ಯಾದ
ಕಾರ್ಯ-ದಲ್ಲಿ
ಕಾರ್ಯ-ದಿಂದ
ಕಾರ್ಯ-ದೆ-ಡೆಗೆ
ಕಾರ್ಯ-ನಿ-ಮ-ಗ್ನ-ರಾ-ದರು
ಕಾರ್ಯ-ನಿ-ಮಿತ್ತ
ಕಾರ್ಯ-ನಿ-ರ-ತ-ರಾ-ಗಲು
ಕಾರ್ಯ-ನಿ-ರ-ತ-ರಾಗಿ
ಕಾರ್ಯ-ನಿ-ರ-ತ-ರಾ-ಗಿ-ದ್ದಾಗ
ಕಾರ್ಯ-ನಿ-ರ್ವ-ಹ-ಣೆಯ
ಕಾರ್ಯ-ನಿ-ರ್ವಾ-ಹಕ
ಕಾರ್ಯ-ಪ್ರ-ವೃತ್ತ
ಕಾರ್ಯ-ಪ್ರ-ವೃ-ತ್ತ-ರಾ-ದರು
ಕಾರ್ಯ-ಮ-ಗ್ನ-ರಾ-ಗ-ಬೇಕು
ಕಾರ್ಯ-ಮ-ಗ್ನ-ರಾ-ಗ-ಲೇ-ಬೇಕು
ಕಾರ್ಯ-ಮ-ಗ್ನ-ರಾಗಿ
ಕಾರ್ಯ-ಮ-ಗ್ನ-ರಾ-ಗಿ-ರು-ವುದನ್ನು
ಕಾರ್ಯ-ಮ-ಗ್ನ-ರಾ-ದರು
ಕಾರ್ಯ-ಯೋ-ಜನೆ
ಕಾರ್ಯ-ಯೋ-ಜ-ನೆ-ಗಳ
ಕಾರ್ಯ-ಯೋ-ಜ-ನೆಗೂ
ಕಾರ್ಯ-ಯೋ-ಜ-ನೆಗೆ
ಕಾರ್ಯ-ಯೋ-ಜ-ನೆಯ
ಕಾರ್ಯ-ಯೋ-ಜ-ನೆ-ಯ-ಲ್ಲೊಂದು
ಕಾರ್ಯ-ಯೋ-ಜ-ನೆ-ಯಾ-ಗಿತ್ತು
ಕಾರ್ಯ-ರಂ-ಗ-ಕ್ಕಿ-ಳಿದ
ಕಾರ್ಯ-ರಂ-ಗ-ಕ್ಕಿ-ಳಿ-ಯಲು
ಕಾರ್ಯ-ರಂ-ಗಕ್ಕೆ
ಕಾರ್ಯ-ರೂ-ಪಕ್ಕೆ
ಕಾರ್ಯ-ವ-ನ್ನಾಗಿ
ಕಾರ್ಯ-ವನ್ನು
ಕಾರ್ಯ-ವನ್ನೂ
ಕಾರ್ಯ-ವಾ-ಗಿ-ಬಿ-ಟ್ಟಿದೆ
ಕಾರ್ಯ-ವಾ-ವುದೂ
ಕಾರ್ಯ-ವಿದೆ
ಕಾರ್ಯ-ವಿ-ಧಾ-ನ-ಗಳನ್ನೂ
ಕಾರ್ಯ-ವಿ-ಧಾ-ನದ
ಕಾರ್ಯ-ವಿ-ಧಾ-ನ-ದಲ್ಲಿ
ಕಾರ್ಯ-ವಿ-ಧಾ-ನ-ವನ್ನು
ಕಾರ್ಯವು
ಕಾರ್ಯ-ವು-ಅ-ವರೇ
ಕಾರ್ಯವೂ
ಕಾರ್ಯ-ವೆಂದರೆ
ಕಾರ್ಯವೇ
ಕಾರ್ಯ-ವೊಂ-ದನ್ನು
ಕಾರ್ಯ-ಶಕ್ತಿ
ಕಾರ್ಯ-ಶೀ-ಲರು
ಕಾರ್ಯ-ಶೀ-ಲರೂ
ಕಾರ್ಯ-ಸ-ನ್ನ-ದ್ಧ-ರಾಗಿ
ಕಾರ್ಯ-ಸಾ-ಧ-ನೆ-ಗಾಗಿ
ಕಾರ್ಯ-ಸಿ-ದ್ಧಿಯ
ಕಾರ್ಯ-ಸ್ಥಾ-ನ-ಗ-ಳ-ಲ್ಲೊಂ-ದಾಗಿ
ಕಾರ್ಯಾಂ-ತ-ರ-ಗ-ಳಿಂ-ದಾಗಿ
ಕಾರ್ಯಾ-ರಂಭ
ಕಾರ್ಯಾ-ರಂ-ಭ-ಗೊ-ಳ್ಳು-ವಂತೆ
ಕಾರ್ಯೋ-ದ್ದೇಶ
ಕಾರ್ಯೋ-ದ್ದೇ-ಶ-ಕ್ಕಾಗಿ
ಕಾರ್ಯೋ-ದ್ದೇ-ಶಕ್ಕೆ
ಕಾರ್ಯೋ-ದ್ದೇ-ಶ-ಗಳ
ಕಾರ್ಯೋ-ದ್ದೇ-ಶ-ಗ-ಳಿ-ಗಾಗಿ
ಕಾರ್ಯೋ-ದ್ದೇ-ಶ-ಗ-ಳಿಗೆ
ಕಾರ್ಯೋ-ದ್ದೇ-ಶದ
ಕಾರ್ಯೋ-ದ್ದೇ-ಶ-ದಲ್ಲಿ
ಕಾರ್ಯೋ-ದ್ದೇ-ಶ-ದೆ-ಡೆಗೆ
ಕಾರ್ಯೋ-ದ್ದೇ-ಶ-ವಂತೂ
ಕಾರ್ಯೋ-ದ್ದೇ-ಶ-ವನ್ನು
ಕಾರ್ಯೋ-ದ್ದೇ-ಶ-ವನ್ನೂ
ಕಾರ್ಯೋ-ದ್ದೇ-ಶ-ವಿದೆ
ಕಾರ್ಯೋ-ದ್ದೇ-ಶ-ವೊಂ-ದರ
ಕಾರ್ಯೋ-ನ್ನು-ಖ-ರಾ-ಗಿ-ಸು-ತ್ತಿ-ದ್ದ-ರಾ-ದರೂ
ಕಾರ್ಯೋ-ನ್ಮು-ಖ-ಗೊ-ಳಿ-ಸಲು
ಕಾರ್ಯೋ-ನ್ಮು-ಖ-ರ-ನ್ನಾ-ಗಿಸಿ
ಕಾರ್ಯೋ-ನ್ಮು-ಖ-ರಾಗಿ
ಕಾರ್ಯೋ-ನ್ಮು-ಖ-ರಾ-ದರು
ಕಾರ್ಯೋ-ನ್ಮು-ಖ-ವಾ-ಗು-ತ್ತದೆ
ಕಾಲ
ಕಾಲಂ-ಗಳ
ಕಾಲ-ಕ-ಸ-ವಾ-ಗಿತ್ತು
ಕಾಲಕ್ಕೂ
ಕಾಲಕ್ಕೆ
ಕಾಲ-ಕ್ರ-ಮ-ದಲ್ಲಿ
ಕಾಲ-ಗ-ತಿಯ
ಕಾಲ-ಗ-ರ್ಭ-ದಲ್ಲಿ
ಕಾಲ-ಗಳ
ಕಾಲ-ಗ-ಳಲ್ಲೂ
ಕಾಲ-ಡಿ-ಯಲ್ಲಿ
ಕಾಲದ
ಕಾಲ-ದಲ್ಲಿ
ಕಾಲ-ದಲ್ಲೂ
ಕಾಲ-ದಲ್ಲೇ
ಕಾಲ-ದ-ವ-ರೆಗೂ
ಕಾಲ-ದ-ವ-ರೆಗೆ
ಕಾಲ-ದಿಂದ
ಕಾಲ-ದಿಂ-ದಲೂ
ಕಾಲ-ವ-ಧಿಯ
ಕಾಲ-ವನ್ನು
ಕಾಲ-ವಾ-ದ್ದ-ರಿಂದ
ಕಾಲ-ವೀಗ
ಕಾಲವೂ
ಕಾಲ-ವೊಂದು
ಕಾಲ-ಹ-ರಣ
ಕಾಲಾಂ-ತ-ರ-ದಲ್ಲಿ
ಕಾಲಾಂ-ತ-ರ-ದಲ್ಲೋ
ಕಾಲಾ-ನು-ಕ್ರ-ಮ-ಣಿ-ಕೆಯೂ
ಕಾಲಾ-ಪಾ-ನಿ-ಯನ್ನು
ಕಾಲಾ-ವ-ಕಾ-ಶ-ವನ್ನು
ಕಾಲಾ-ವ-ಕಾ-ಶ-ವಿತ್ತು
ಕಾಲಿಗೆ
ಕಾಲಿಟ್ಟ
ಕಾಲಿಟ್ಟು
ಕಾಲಿ-ಡಲೂ
ಕಾಲಿ-ಡು-ತ್ತಿ-ದ್ದಂ-ತೆಯೇ
ಕಾಲಿ-ನ-ಡಿ-ಯಲ್ಲಿ
ಕಾಲೀ-ಚ-ರಣ
ಕಾಲು
ಕಾಲು-ಕಿತ್ತ
ಕಾಲು-ಗಳ
ಕಾಲು-ಗಳನ್ನು
ಕಾಲು-ಗಳಿಂದ
ಕಾಲು-ಗ-ಳಿಗೆ
ಕಾಲು-ಗಳು
ಕಾಲು-ದಾರಿ-ಯ-ಲ್ಲಾ-ದರೆ
ಕಾಲು-ದಾರಿ-ಯಲ್ಲೂ
ಕಾಲು-ನ-ಡಿ-ಗೆ-ಯಲ್ಲೇ
ಕಾಲು-ಮ-ಡಿ-ಸಿ-ಕೊಂಡು
ಕಾಲು-ವೆ-ಗ-ಳನ್ನೇ
ಕಾಲು-ವೆ-ಯನ್ನು
ಕಾಲು-ವೆಯು
ಕಾಲೆ-ಳೆ-ದು-ಕೊಂಡು
ಕಾಲೇ
ಕಾಲೇ-ಜಿನ
ಕಾಲೇ-ಜಿ-ನಲ್ಲಿ
ಕಾಲೇಜು
ಕಾಲೇ-ಜು-ಗಳನ್ನು
ಕಾಲೇ-ಜೊಂ-ದನ್ನು
ಕಾಲ್ನ-ಡಿ-ಗೆಯ
ಕಾಲ್ನ-ಡಿ-ಗೆ-ಯಲ್ಲಿ
ಕಾಲ್ನ-ಡಿ-ಗೆ-ಯಲ್ಲೇ
ಕಾಲ್ವೆ
ಕಾಲ್ವೆಗೆ
ಕಾಲ್ವೆಯ
ಕಾಲ್ವೆ-ಯನ್ನು
ಕಾಳ-ಗ-ಗಳು
ಕಾಳ-ಗದ
ಕಾಳಜಿ
ಕಾಳ-ಜಿ-ಯನ್ನು
ಕಾಳ-ಜಿ-ಯಾ-ಗಿತ್ತು
ಕಾಳ-ಜಿ-ಯೆಂ-ದರೆ
ಕಾಳಿ
ಕಾಳಿ-ದಾಸ
ಕಾಳಿ-ನ-ಷ್ಟೂ-ಇಲ್ಲ
ಕಾಳೀ
ಕಾಳೀ-ಘಾ-ಟಿನ
ಕಾಳೀ-ಚ-ರಣ
ಕಾಳೀ-ಪದ
ಕಾಳು
ಕಾಳ್ಗಿ-ಚ್ಚಿ-ನಂತೆ
ಕಾವಲಿ
ಕಾವ-ಲು-ಗಾ-ರ-ನನ್ನು
ಕಾವ-ಲು-ಗಾ-ರರು
ಕಾವಿ
ಕಾವಿ-ಬಟ್ಟೆ
ಕಾವಿ-ಬ-ಟ್ಟೆ-ಗಳು
ಕಾವಿ-ಬ-ಟ್ಟೆ-ಯನ್ನು
ಕಾವೇ
ಕಾವೇ-ರಿತು
ಕಾವೇ-ರಿ-ದಾಗ
ಕಾವ್ಯ-ಸಾ-ಹಿ-ತ್ಯ-ಗಳನ್ನು
ಕಾವ್ಯ-ಮ-ಯ-ವಾ-ಗ-ಬೇಕು
ಕಾವ್ಯ-ಮ-ಯ-ವಾದ
ಕಾವ್ಯ-ಸಂ-ಪು-ಟ-ದಿಂದ
ಕಾವ್ಯ-ಸಂ-ಪು-ಟ-ವನ್ನು
ಕಾವ್ಯಾ-ತ್ಮಕ
ಕಾಶ
ಕಾಶ-ವನ್ನು
ಕಾಶಿ-ನಾಥ
ಕಾಶೀ-ಪು-ರ-ದಲ್ಲಿ
ಕಾಷಾಯ
ಕಾಷಾ-ಯ-ಧಾ-ರಿ-ಗಳು
ಕಾಷಾ-ಯ-ವ-ಸ್ತ್ರದ
ಕಾಷಾ-ಯ-ವ-ಸ್ತ್ರ-ಧಾ-ರಿ-ಯಾಗಿ
ಕಾಸು-ಗಳು
ಕಾಹೆ
ಕಾಹೇ
ಕಿಂಚಿತ್
ಕಿಂಚಿ-ತ್ತಾ-ದರೂ
ಕಿಂಚಿತ್ತೂ
ಕಿಂಡ್ಲಿಗೆ
ಕಿಕ್ಕಿ-ರಿ-ದಿತ್ತು
ಕಿಕ್ಕಿ-ರಿ-ದಿ-ದ್ದರು
ಕಿಕ್ಕಿ-ರಿ-ದಿ-ರು-ತ್ತಿ-ದ್ದರು
ಕಿಕ್ಕಿ-ರಿದು
ಕಿಕ್ಕಿ-ರಿ-ಯ-ಲಾ-ರಂ-ಭಿ-ಸಿ-ದರು
ಕಿಕ್ಕಿ-ರಿ-ಯು-ತ್ತಿ-ದ್ದರು
ಕಿಚ್ಚಿ-ಟ್ಟಂ-ತಾ-ಯಿತು
ಕಿಟ-ಕಿ-ಗಳ
ಕಿಟ-ಕಿ-ಗಳಿಂದ
ಕಿಟ-ಕಿ-ಯಲ್ಲಿ
ಕಿಟ-ಕಿ-ಯಾಚೆ
ಕಿಟ್ಟಣ್ಣ
ಕಿಡಿ
ಕಿಡಿ-ಗ-ಳಂತೆ
ಕಿಡಿ-ಗಳನ್ನು
ಕಿಡಿ-ಗಳು
ಕಿಡಿಯ
ಕಿಡಿ-ಯನ್ನು
ಕಿಡಿಯೇ
ಕಿತ್ತಂ-ತಿದ್ದ
ಕಿತ್ತಲೆ
ಕಿತ್ತ-ಲೆ-ಬ-ಣ್ಣದ
ಕಿತ್ತಳೆ
ಕಿತ್ತಾ-ಡುವ
ಕಿತ್ತು-ಕೊಂಡ
ಕಿತ್ತು-ಕೊಂ-ಡು-ಬಿಟ್ಟ
ಕಿತ್ತು-ಕೊ-ಳ್ಳ-ಬಾ-ರ-ದೆಂದು
ಕಿತ್ತು-ಕೊ-ಳ್ಳ-ಬೇ-ಕಾ-ಗು-ತ್ತದೆ
ಕಿತ್ತು-ಹಾ-ಕು-ವು-ದೆಂ-ದರೆ
ಕಿತ್ತೆ-ಸೆದು
ಕಿತ್ತೆ-ಸೆ-ಯಲು
ಕಿತ್ತೆ-ಸೆ-ಯು-ವಲ್ಲಿ
ಕಿತ್ತೊ-ಗೆದು
ಕಿನಾ-ರೆ-ಯಲ್ಲಿ
ಕಿನ್ನ-ರ-ಲೋ-ಕ-ವೆಂಬ
ಕಿನ್ನ-ರ-ಲೋ-ಕವೇ
ಕಿಯಾಂಗ್
ಕಿರಣ
ಕಿರ-ಣ-ಗಳ
ಕಿರ-ಣ-ಗಳನ್ನು
ಕಿರ-ಣ-ಗಳು
ಕಿರ-ಣ-ವೊಂ-ದನ್ನು
ಕಿರಿ-ಕಿರಿ
ಕಿರಿ-ಕಿ-ರಿ-ಯಾ-ಗು-ವಂ-ತಿ-ದ್ದರೆ
ಕಿರಿ-ಕಿ-ರಿ-ಯುಂ-ಟು-ಮಾ-ಡಿ-ದ್ದೆಂ-ದರೆ
ಕಿರಿ-ಚಿ-ಕೊಳ್ಳ
ಕಿರಿ-ದಾದ
ಕಿರಿಯ
ಕಿರಿ-ಯ-ರಾದ
ಕಿರಿ-ಯರು
ಕಿರೀ-ಟದ
ಕಿರು
ಕಿರು-ಕಾ-ಣಿ-ಕೆ-ಯನ್ನೂ
ಕಿರು-ಗಾ-ಣಿ-ಕೆ-ಯನ್ನು
ಕಿರು-ಚಾ-ಡಿ-ದರೂ
ಕಿರು-ಚಿದ
ಕಿರು-ದೋ-ಣಿಯೆ
ಕಿವಿ
ಕಿವಿ-ಕಣ್ಣು
ಕಿವಿ-ಗಡ
ಕಿವಿ-ಗಳ
ಕಿವಿ-ಗಳನ್ನು
ಕಿವಿ-ಗಳಲ್ಲಿ
ಕಿವಿ-ಗಳಿಂದ
ಕಿವಿ-ಗ-ಳೊಂ-ದಿಗೆ
ಕಿವಿಗೂ
ಕಿವಿಗೆ
ಕಿವಿ-ಗೊಟ್ಟು
ಕಿವಿ-ಗೊ-ಡ-ಲಿಲ್ಲ
ಕಿವಿ-ಗೊ-ಡಲೇ
ಕಿವಿಚಿ
ಕಿವಿಯ
ಕಿವಿ-ಯಲ್ಲಿ
ಕಿವಿ-ಯಾರೆ
ಕಿವಿ-ಯಿಂದ
ಕಿವಿ-ಯೂದಿ
ಕಿವು-ಡಾ-ಗು-ವಂತೆ
ಕಿಶ-ನ್ಘರ್
ಕಿಷ್ಕಿಂ-ಧಾ-ಕಾಂ-ಡದ
ಕಿಸು-ಬಾ-ಯಿ-ದಾ-ಸರೇ
ಕೀಟ-ಲೆಗೆ
ಕೀಟವೆ
ಕೀಯ-ಗಳನ್ನೂ
ಕೀರ್ತ-ನೆ-ಗಳನ್ನೂ
ಕೀರ್ತ-ನೆ-ಯೊಂ-ದನ್ನು
ಕೀರ್ತಿ
ಕೀರ್ತಿ-ಗೌ-ರ-ವ-ಗಳನ್ನೆಲ್ಲ
ಕೀರ್ತಿ-ಗಳ
ಕೀರ್ತಿ-ಗ-ಳ-ಲ್ಲಾ-ಗಲಿ
ಕೀರ್ತಿ-ಗಳು
ಕೀರ್ತಿ-ಗಾಗಿ
ಕೀರ್ತಿ-ಗಾ-ಗಿಯೂ
ಕೀರ್ತಿ-ಗಿಂ-ತಲೂ
ಕೀರ್ತಿಗೆ
ಕೀರ್ತಿ-ಪ-ತಾ-ಕೆ-ಯನ್ನು
ಕೀರ್ತಿಯ
ಕೀರ್ತಿ-ಯಂತೂ
ಕೀರ್ತಿ-ಯನ್ನು
ಕೀರ್ತಿ-ಯನ್ನೂ
ಕೀರ್ತಿ-ಯೆ-ದುರು
ಕೀರ್ತಿ-ವಂ-ತ-ರಾ-ಗ-ಬ-ಲ್ಲಿರಿ
ಕೀರ್ತಿ-ಶ-ರೀರ
ಕೀರ್ತಿ-ಶಿ-ಖ-ರ-ವ-ನ್ನೇ-ರಿ-ದ್ದರು
ಕೀಲಿಕೈ
ಕೀಲ್
ಕೀಲ್ಗೆ
ಕೀಲ್ನಿಂದ
ಕೀಳಲು
ಕೀಳಲ್ಲ
ಕೀಳು
ಕೀಳು-ಜಾ-ತಿ-ಯ-ವನು
ಕೀಳು-ತ್ತಿ-ದ್ದಾಳೆ
ಕೀಳು-ಮ-ಟ್ಟದ
ಕೀಳ್ಗೈ-ಯುವ
ಕುಂಠಿ-ತ-ಗೊ-ಳ್ಳ-ಬ-ಹು-ದೆಂದು
ಕುಂಠಿ-ತ-ವಾ-ಗ-ಬ-ಹು-ದೆಂದು
ಕುಂತ್ಕೊಳ್ಳೊ
ಕುಂದು-ಬ-ರು-ತ್ತಿತ್ತು
ಕುಕ್
ಕುಕ್ಕ-ರಿ-ಸು-ತ್ತಿದ್ದೆ
ಕುಕ್ಕುವ
ಕುಕ್ಳಿಗೆ
ಕುಗ್ಗಿ-ದರೆ
ಕುಗ್ಗುತ್ತ
ಕುಚೇ-ಷ್ಟೆಯ
ಕುಟ-ವಾಗಿ
ಕುಟಿಲ
ಕುಟುಂಬ
ಕುಟುಂ-ಬಕ್ಕೇ
ಕುಟುಂ-ಬ-ಗಳ
ಕುಟುಂ-ಬ-ಗಳನ್ನು
ಕುಟುಂ-ಬದ
ಕುಟುಂ-ಬ-ದಂತೆ
ಕುಟುಂ-ಬ-ದಲ್ಲಿ
ಕುಟುಂ-ಬ-ದ-ವರ
ಕುಟುಂ-ಬ-ದ-ವ-ರಿ-ಗೆಲ್ಲ
ಕುಟುಂ-ಬ-ದ-ವರು
ಕುಟುಂ-ಬ-ದ-ವರೆ-ನ್ನೆಲ್ಲ
ಕುಟುಂ-ಬ-ದ-ವರೆ-ಲ್ಲರ
ಕುಟುಂ-ಬ-ವ-ರ್ಗ-ದ-ವರ
ಕುಟುಂ-ಬ-ವ-ರ್ಗ-ದ-ವ-ರಿಗೂ
ಕುಟುಂ-ಬವೂ
ಕುಟುಂ-ಬ-ವೊಂ-ದರ
ಕುಟುಂ-ಬ-ವೊಂ-ದ-ರಲ್ಲಿ
ಕುಟ್ಟಿ
ಕುಠಾ-ರ-ದೇ-ಟನ್ನು
ಕುಡಿ
ಕುಡಿದ
ಕುಡಿ-ದರು
ಕುಡಿ-ದಿಲ್ಲ
ಕುಡಿ-ದು-ಬಿಟ್ಟ
ಕುಡಿ-ಯ-ಬೇಡ
ಕುಡಿ-ಯಲು
ಕುಡಿ-ಯಲೂ
ಕುಡಿ-ಯು-ತ್ತಲೇ
ಕುಡಿ-ಯು-ತ್ತಿದ್ದ
ಕುಡಿ-ಯು-ತ್ತೇನೆ
ಕುಡಿ-ಯು-ವು-ದಷ್ಟೇ
ಕುಡಿ-ಯು-ವುದು
ಕುಡು-ಕರು
ಕುಣಿ-ದರು
ಕುಣಿ-ದಾಡಿ
ಕುಣಿ-ದಾ-ಡಿದ
ಕುಣಿ-ದಾ-ಡಿ-ದರು
ಕುಣಿ-ಯು-ತ್ತಿ-ವೆಯೋ
ಕುತಂತ್ರ
ಕುತಂ-ತ್ರ-ಗ-ಳಿಗೆ
ಕುತಂ-ತ್ರಿಯೂ
ಕುತ-ರ್ಕ-ವನ್ನು
ಕುತೂ-ಹಲ
ಕುತೂ-ಹ-ಲ
ಕುತೂ-ಹ-ಲ-ಜಿ-ಜ್ಞಾ-ಸೆಯ
ಕುತೂ-ಹ-ಲ-ಕರ
ಕುತೂ-ಹ-ಲ-ಕ-ರ-ವಾ-ಗಿದೆ
ಕುತೂ-ಹ-ಲ-ಕ-ರ-ವಾ-ಗಿ-ರು-ತ್ತಿತ್ತು
ಕುತೂ-ಹ-ಲ-ಕ-ರ-ವಾದ
ಕುತೂ-ಹ-ಲ-ಕಾರಿ
ಕುತೂ-ಹ-ಲ-ಕ್ಕಾಗಿ
ಕುತೂ-ಹ-ಲ-ಗೊಂಡ
ಕುತೂ-ಹ-ಲ-ಗೊಂಡು
ಕುತೂ-ಹ-ಲದ
ಕುತೂ-ಹ-ಲ-ದಿಂದ
ಕುತೂ-ಹ-ಲ-ಭ-ರಿತ
ಕುತೂ-ಹ-ಲ-ವನ್ನು
ಕುತೂ-ಹ-ಲ-ವಿತ್ತು
ಕುತೂ-ಹ-ಲ-ಸ್ವತಃ
ಕುತೂ-ಹ-ಲಾ-ಕಾಂ-ಕ್ಷಿ-ಗಳು
ಕುತೂ-ಹ-ಲಿತ
ಕುತೂ-ಹ-ಲಿ-ತ-ರಾ-ದರು
ಕುತ್ತಿ-ಗೆಯ
ಕುತ್ತಿ-ಗೆ-ಯನ್ನು
ಕುದಿ-ಯು-ತ್ತಿತ್ತು
ಕುದಿ-ಯು-ತ್ತಿದ್ದ
ಕುದುರೆ
ಕುದು-ರೆ-ಗ-ಳನ್ನೇ
ಕುದು-ರೆ-ಗ-ಳ-ನ್ನೇರಿ
ಕುದು-ರೆಗೆ
ಕುದು-ರೆಯ
ಕುದು-ರೆ-ಯನ್ನು
ಕುದು-ರೆ-ಯನ್ನೂ
ಕುದು-ರೆ-ಯ-ನ್ನೇರಿ
ಕುದು-ರೆ-ಲಾ-ಯದ
ಕುನ್ನಿ-ಗ-ಳಿ-ರು-ತ್ತವೆ
ಕುಪಿ-ತ-ರಾ-ದರು
ಕುಪ್ಪ-ಳಿ-ಸು-ತ್ತಾರೆ
ಕುಬೇ-ರನ
ಕುಬ್ಜ-ರಾ-ಗು-ತ್ತಿ-ರು-ವಂತೆ
ಕುಮಾರಿ
ಕುಮ್ಮಕ್ಕು
ಕುಯು-ಕ್ತಿಯ
ಕುಯು-ಕ್ತಿ-ಯಲ್ಲಿ
ಕುಯ್ದು
ಕುಯ್ಯು-ವು-ದರ
ಕುರಾ-ನಿನ
ಕುರಿ-ಗ-ಳೆಂ-ದೆಂಬ
ಕುರಿತ
ಕುರಿ-ತಾಗಿ
ಕುರಿ-ತಾ-ಗಿತ್ತು
ಕುರಿ-ತಾ-ಗಿಯೂ
ಕುರಿ-ತಾ-ಗಿಯೇ
ಕುರಿ-ತಾದ
ಕುರಿ-ತಾ-ದ-ದ್ದಲ್ಲ
ಕುರಿ-ತಾ-ದದ್ದು
ಕುರಿ-ತಾ-ದವು
ಕುರಿತು
ಕುರಿ-ಯಲ್ಲಿ
ಕುರು-ಡು-ನಂ-ಬಿ-ಕೆ-ಯನ್ನೇ
ಕುರು-ಹಾಗಿ
ಕುರ್ಚಿ
ಕುರ್ಚಿ-ಗಳನ್ನೆಲ್ಲ
ಕುರ್ಚಿ-ಗ-ಳಲ್ಲೂ
ಕುರ್ಚಿಯ
ಕುರ್ಚಿ-ಯನ್ನು
ಕುರ್ಚಿ-ಯಲ್ಲಿ
ಕುಲ
ಕುಲ-ಗು-ರು-ವಿಗೆ
ಕುಲ-ದಲ್ಲಿ
ಕುಲ-ಶೈ-ಲ-ದೊಲು
ಕುಲು-ಕ-ಬೇ-ಕೆಂಬ
ಕುಲುಕಿ
ಕುಲು-ಕಿ-ದಳು
ಕುಲು-ಕುತ್ತ
ಕುಲುಮೆ
ಕುಳ-ಗಳನ್ನೆಲ್ಲ
ಕುಳಿ
ಕುಳಿ-ಗ-ಳು-ಇ-ರು-ತ್ತವೆ
ಕುಳಿತ
ಕುಳಿ-ತರು
ಕುಳಿ-ತ-ರೆಂ-ದರೆ
ಕುಳಿ-ತ-ಲ್ಲಿಂದ
ಕುಳಿ-ತಲ್ಲೇ
ಕುಳಿ-ತಳು
ಕುಳಿ-ತ-ವ-ರ-ಲ್ಲ-ವಲ್ಲ
ಕುಳಿ-ತ-ವ-ರಲ್ಲಿ
ಕುಳಿ-ತ-ವರೆಲ್ಲ
ಕುಳಿ-ತಾಗ
ಕುಳಿ-ತಾ-ಗಲೂ
ಕುಳಿತಿ
ಕುಳಿ-ತಿತ್ತು
ಕುಳಿ-ತಿದ್ದ
ಕುಳಿ-ತಿ-ದ್ದಂತೆ
ಕುಳಿ-ತಿ-ದ್ದ-ಅ-ವನ
ಕುಳಿ-ತಿ-ದ್ದರು
ಕುಳಿ-ತಿ-ದ್ದರೆ
ಕುಳಿ-ತಿ-ದ್ದರೊ
ಕುಳಿ-ತಿ-ದ್ದಾಗ
ಕುಳಿ-ತಿ-ದ್ದಾನೆ
ಕುಳಿ-ತಿ-ದ್ದಾರೆ
ಕುಳಿ-ತಿ-ದ್ದಿ-ರ-ಬ-ಹುದೆ
ಕುಳಿ-ತಿ-ದ್ದಿರಿ
ಕುಳಿ-ತಿ-ದ್ದೀ-ಯಲ್ಲ
ಕುಳಿ-ತಿದ್ದು
ಕುಳಿ-ತಿದ್ದೆ
ಕುಳಿ-ತಿರ
ಕುಳಿ-ತಿ-ರ-ಬೇಡಿ
ಕುಳಿ-ತಿ-ರಲು
ಕುಳಿ-ತಿ-ರು-ತ್ತಾನೋ
ಕುಳಿ-ತಿ-ರು-ತ್ತಿ-ದ್ದರು
ಕುಳಿ-ತಿ-ರು-ತ್ತಿ-ದ್ದೆವು
ಕುಳಿ-ತಿ-ರು-ತ್ತೀರೋ
ಕುಳಿ-ತಿ-ರು-ತ್ತೇನೆ
ಕುಳಿ-ತಿ-ರುವ
ಕುಳಿ-ತಿ-ರು-ವ-ವರು
ಕುಳಿ-ತಿ-ರು-ವುದನ್ನು
ಕುಳಿತು
ಕುಳಿ-ತು-ಕೊಂಡ
ಕುಳಿ-ತು-ಕೊಂ-ಡರು
ಕುಳಿ-ತು-ಕೊಂ-ಡಾಗ
ಕುಳಿ-ತು-ಕೊಂ-ಡಿತು
ಕುಳಿ-ತು-ಕೊಂಡು
ಕುಳಿ-ತು-ಕೊಳ್ಳ
ಕುಳಿ-ತು-ಕೊ-ಳ್ಳ-ಬೆಕು
ಕುಳಿ-ತು-ಕೊ-ಳ್ಳ-ಬೇ-ಕಾದ
ಕುಳಿ-ತು-ಕೊ-ಳ್ಳ-ಬೇ-ಕಾ-ದರೆ
ಕುಳಿ-ತು-ಕೊ-ಳ್ಳಲು
ಕುಳಿ-ತು-ಕೊ-ಳ್ಳಲೂ
ಕುಳಿ-ತು-ಕೊಳ್ಳಿ
ಕುಳಿ-ತು-ಕೊಳ್ಳು
ಕುಳಿ-ತು-ಕೊ-ಳ್ಳು-ತ್ತಾರೆ
ಕುಳಿ-ತು-ಕೊ-ಳ್ಳು-ತ್ತಾ-ರೆ-ಜನ
ಕುಳಿ-ತು-ಕೊ-ಳ್ಳು-ತ್ತಿ-ದ್ದಂತೆ
ಕುಳಿ-ತು-ಕೊ-ಳ್ಳು-ತ್ತಿ-ದ್ದಂ-ತೆಯೇ
ಕುಳಿ-ತು-ಕೊ-ಳ್ಳು-ತ್ತಿ-ದ್ದರು
ಕುಳಿ-ತು-ಕೊ-ಳ್ಳು-ತ್ತಿ-ದ್ದರೆ
ಕುಳಿ-ತು-ಕೊ-ಳ್ಳುವ
ಕುಳಿ-ತು-ಕೊ-ಳ್ಳು-ವಂ-ತಹ
ಕುಳಿ-ತು-ಕೊ-ಳ್ಳು-ವಂ-ತಾ-ಯಿತು
ಕುಳಿ-ತು-ಕೊ-ಳ್ಳು-ವಂತೆ
ಕುಳಿ-ತು-ಕೊ-ಳ್ಳು-ವಷ್ಟು
ಕುಳಿ-ತು-ಕೊ-ಳ್ಳು-ವು-ದಿಲ್ಲ
ಕುಳಿ-ತು-ಬಿಟ್ಟ
ಕುಳಿ-ತು-ಬಿ-ಟ್ಟರು
ಕುಳಿ-ತು-ಬಿ-ಟ್ಟೆವು
ಕುಳಿ-ತು-ಬಿ-ಡು-ತ್ತಿ-ದ್ದ-ನಲ್ಲ
ಕುಳಿ-ತು-ಬಿ-ಡು-ತ್ತಿ-ದ್ದರು
ಕುಳಿ-ತು-ಬಿ-ಡು-ತ್ತಿದ್ದೆ
ಕುಳಿ-ತೆವು
ಕುಳಿತೇ
ಕುಳಿ-ಯಲ್ಲಿ
ಕುಳ್ಳಿ-ರಿ-ಸ-ಬೇ-ಕಾ-ಗು-ತ್ತ-ದಲ್ಲ
ಕುಳ್ಳಿ-ರಿಸಿ
ಕುಳ್ಳಿ-ರಿ-ಸಿ-ಕೊಂಡ
ಕುಳ್ಳಿ-ರಿ-ಸಿ-ಕೊಂಡು
ಕುಳ್ಳಿ-ರಿ-ಸಿತು
ಕುಶ-ಲೋ-ಪ-ರಿ-ಗ-ಳಾದ
ಕುಸಿ-ತ-ವುಂ-ಟಾ-ಗಿದ್ದ
ಕುಸಿದು
ಕುಸಿ-ದು-ಬಿತ್ತು
ಕುಸಿ-ದು-ಬೀ-ಳು-ತ್ತ-ದೆ-ಆ-ದ್ದ-ರಿಂದ
ಕುಸಿ-ದು-ಹೋಗು
ಕುಸಿ-ದು-ಹೋ-ಗು-ತ್ತಿ-ದ್ದರು
ಕುಸಿದೇ
ಕುಸಿ-ಯ-ಲೇ-ಬೇಕು
ಕುಸಿ-ಯು-ತ್ತವೆ
ಕುಸಿ-ಯು-ತ್ತಿ-ರುವ
ಕುಸಿ-ಯುವ
ಕುಸ್ತಿ
ಕುಹಕ
ಕುಹ-ಕಿ-ಗ-ಳಿಗೆ
ಕೂಗ-ಲಾ-ರಂ-ಭಿ-ಸಿದ
ಕೂಗಾಟ
ಕೂಗಿ
ಕೂಗಿ-ಕ-ರೆ-ಯು-ತ್ತಿ-ರು-ವಂ-ತಿತ್ತು
ಕೂಗಿ-ಕೊಂಡ
ಕೂಗಿ-ಕೊಳ್ಳು
ಕೂಗಿ-ಕೊ-ಳ್ಳು-ತ್ತೀರಿ
ಕೂಗಿದ
ಕೂಗಿ-ದ-ಇಲ್ಲ
ಕೂಗಿ-ದ-ಬುದ್ಧ
ಕೂಗಿ-ದರು
ಕೂಗಿ-ದಳು
ಕೂಗಿದ್ದು
ಕೂಗಿ-ಹೇ-ಳಿದ
ಕೂಗಿ-ಹೇ-ಳಿ-ದರು
ಕೂಗು
ಕೂಗು-ಗಳು
ಕೂಗುತ್ತ
ಕೂಗು-ತ್ತಿದ್ದ
ಕೂಗು-ತ್ತಿ-ರು-ವುದು
ಕೂಗು-ವು-ದ-ರೊ-ಳ-ಗಾಗಿ
ಕೂಟಕ್ಕೆ
ಕೂಟ-ವೊಂದು
ಕೂಡ
ಕೂಡ-ಅ-ದಿ-ರು-ವುದೇ
ಕೂಡ-ಆ-ಕ-ರ್ಷ-ಣೀ-ಯ-ವಾ-ಗಿ-ದ್ದುವು
ಕೂಡ-ನ-ನಗೆ
ಕೂಡಲೆ
ಕೂಡಲೇ
ಕೂಡಿ
ಕೂಡಿ-ಕೊಂಡ
ಕೂಡಿ-ಕೊಂ-ಡರು
ಕೂಡಿ-ಕೊಂಡು
ಕೂಡಿ-ಕೊ-ಳ್ಳ-ಲಿದ್ದ
ಕೂಡಿ-ಕೊ-ಳ್ಳಲು
ಕೂಡಿ-ಕೊ-ಳ್ಳು-ತ್ತವೆ
ಕೂಡಿ-ಕೊ-ಳ್ಳು-ತ್ತೀಯೆ
ಕೂಡಿ-ಟ್ಟಿದ್ದ
ಕೂಡಿ-ಡ-ಲಿಲ್ಲ
ಕೂಡಿ-ಡುವ
ಕೂಡಿ-ಡು-ವುದೂ
ಕೂಡಿ-ತ್ತಾ-ದರೂ
ಕೂಡಿತ್ತು
ಕೂಡಿದ
ಕೂಡಿ-ದ-ವು-ಗ-ಳಲ್ಲ
ಕೂಡಿ-ದು-ದಾ-ಗಿತ್ತು
ಕೂಡಿ-ದು-ದಾ-ಗಿ-ರು-ತ್ತ-ದೆಂದು
ಕೂಡಿ-ದುದು
ಕೂಡಿದೆ
ಕೂಡಿದ್ದ
ಕೂಡಿ-ದ್ದ-ವ-ರಾ-ದರೂ
ಕೂಡಿ-ದ್ದಾ-ಗಿದ್ದು
ಕೂಡಿದ್ದು
ಕೂಡಿ-ದ್ದುವು
ಕೂಡಿ-ಬಂ-ದು-ದ-ರಿಂದ
ಕೂಡಿ-ರು-ತ್ತಿದ್ದ
ಕೂಡಿ-ರು-ತ್ತಿ-ದ್ದರು
ಕೂಡಿ-ಹಾ-ಕುವ
ಕೂದ-ಲೆ-ಳೆ-ಯಷ್ಟೂ
ಕೂಪ
ಕೂಪಕ್ಕೆ
ಕೂಪ-ದಲ್ಲೇ
ಕೂಪ-ದಿಂದ
ಕೂಪ-ಮಂ-ಡೂಕ
ಕೂಪ-ಮಂ-ಡೂ-ಕ-ಗ-ಳಂತೆ
ಕೂಪ-ಮಂ-ಡೂ-ಕ-ಗ-ಳಿಗೆ
ಕೂಲಂ-ಕ-ಷ-ವಾಗಿ
ಕೂಲಿ
ಕೂಲಿ-ಗಳು
ಕೂಲಿ-ಗ-ಳೊಂ-ದಿಗೆ
ಕೂಲಿ-ಯ-ವ-ನನ್ನು
ಕೂಲಿ-ಯ-ವ-ನಿ-ಗಾದ
ಕೂಲಿ-ಯಾ-ಳೊಬ್ಬ
ಕೃತ-ಕೃ-ತ್ಯ-ತೆಯ
ಕೃತ-ಕೃ-ತ್ಯ-ರಾ-ದರು
ಕೃತಘ್ನ
ಕೃತ-ಘ್ನತೆ
ಕೃತ-ಘ್ನ-ನೆಂದು
ಕೃತ-ಜ್ಞತಾ
ಕೃತ-ಜ್ಞ-ತಾ-ಪೂ-ರ್ವ-ಕ-ವಾಗಿ
ಕೃತ-ಜ್ಞ-ತಾ-ಭಾ-ವ-ದಿಂದ
ಕೃತ-ಜ್ಞ-ತಾ-ಭಾ-ವನ್ನೂ
ಕೃತ-ಜ್ಞತೆ
ಕೃತ-ಜ್ಞ-ತೆ-ಗ-ಳ-ನ್ನ-ರ್ಪಿಸಿ
ಕೃತ-ಜ್ಞ-ತೆ-ಗ-ಳ-ನ್ನ-ರ್ಪಿ-ಸುವ
ಕೃತ-ಜ್ಞ-ತೆ-ಗಳನ್ನು
ಕೃತ-ಜ್ಞ-ತೆ-ಗಳು
ಕೃತ-ಜ್ಞ-ತೆಯ
ಕೃತ-ಜ್ಞ-ತೆ-ಯ-ನ್ನ-ರ್ಪಿ-ಸುವ
ಕೃತ-ಜ್ಞ-ತೆ-ಯ-ನ್ನ-ರ್ಪಿ-ಸು-ವಂ-ತಹ
ಕೃತ-ಜ್ಞ-ತೆ-ಯನ್ನು
ಕೃತ-ಜ್ಞ-ತೆ-ಯನ್ನೂ
ಕೃತ-ಜ್ಞ-ತೆ-ಯಿಂದ
ಕೃತ-ಜ್ಞ-ನಾ-ಗಿ-ದ್ದೇನೆ
ಕೃತ-ಜ್ಞ-ನಾ-ಗಿ-ರು-ವುದು
ಕೃತ-ಜ್ಞ-ರಾ-ಗಿ-ದ್ದರು
ಕೃತ-ಜ್ಞ-ರಾ-ಗಿ-ದ್ದಾರೆ
ಕೃತ-ಜ್ಞ-ರಾ-ಗಿ-ರ-ಬೇಕು
ಕೃತ-ಜ್ಞ-ರಾ-ಗಿ-ರು-ತ್ತಿ-ದ್ದರು
ಕೃತ-ಜ್ಞ-ವಾ-ಗಿದೆ
ಕೃತಾರ್ಥ
ಕೃತಿ-ಗಳ
ಕೃತಿ-ಗಳನ್ನು
ಕೃತಿ-ಗಳಲ್ಲಿ
ಕೃತಿ-ಗಳು
ಕೃತಿ-ಗ-ಳೆ-ಲ್ಲ-ವನ್ನೂ
ಕೃತಿಯ
ಕೃತಿ-ಯಲ್ಲಿ
ಕೃತಿಯು
ಕೃತಿಯೇ
ಕೃತಿ-ಶ್ರೇಣಿ
ಕೃತಿ-ಶ್ರೇ-ಣಿ-ಯಲ್ಲಿ
ಕೃತಿ-ಶ್ರೇ-ಣಿ-ಯಲ್ಲೂ
ಕೃತ್ಯಕ್ಕೂ
ಕೃತ್ಯ-ಗಳ
ಕೃತ್ಯ-ಗಳನ್ನು
ಕೃತ್ಯ-ಗ-ಳಿಗೆ
ಕೃತ್ಯ-ವನ್ನು
ಕೃತ್ಯವು
ಕೃತ್ರಿಮ
ಕೃಪಾ-ದೃಷ್ಟಿ
ಕೃಪಾ-ನಂದ
ಕೃಪಾ-ನಂ-ದರ
ಕೃಪಾ-ನಂ-ದ-ರಿಗೆ
ಕೃಪಾ-ನಂ-ದರು
ಕೃಪಾ-ನಂ-ದ-ರೊಂ-ದಿಗೆ
ಕೃಪಾ-ಯಾ-ಚನೆ
ಕೃಪಾ-ಯಾ-ಚ-ನೆಯ
ಕೃಪಾ-ವೃ-ಷ್ಟಿ-ಯನ್ನು
ಕೃಪಾ-ಸಾ-ಗರ
ಕೃಪೆ
ಕೃಪೆ-ಗಿಂ-ತಲೂ
ಕೃಪೆ-ದೋರಿ
ಕೃಪೆ-ದೋ-ರಿ-ದರೆ
ಕೃಪೆಯ
ಕೃಪೆ-ಯನ್ನು
ಕೃಪೆ-ಯನ್ನೇ
ಕೃಪೆ-ಯಿಂದ
ಕೃಪೆ-ಯಿಂ-ದಲೇ
ಕೃಪೆ-ಯಿಂ-ದಷ್ಟೆ
ಕೃಪೆ-ಯಿ-ರಲಿ
ಕೃಷಿ
ಕೃಷಿ-ಕ-ರಾ-ಗಿ-ದ್ದರು
ಕೃಷಿ-ವಿ-ಜ್ಞಾನ
ಕೃಷ್ಣ
ಕೃಷ್ಣ-ಬು-ದ್ಧ
ಕೃಷ್ಣನ
ಕೃಷ್ಣ-ಪ್ರೇ-ಮ-ದಿಂದ
ಕೃಷ್ಣ-ಭಾ-ವ-ದಿಂದ
ಕೃಷ್ಣರ
ಕೃಷ್ಣ-ರನ್ನು
ಕೃಷ್ಣ-ರಾ-ಜನ
ಕೃಷ್ಣ-ರಾ-ಜ-ನೆಂಬ
ಕೃಷ್ಣರು
ಕೃಷ್ಣರೇ
ಕೃಷ್ಣ-ವರ್ಣ
ಕೃಷ್ಣ-ವರ್ಮ
ಕೆ
ಕೆಂಪು
ಕೆಂಬೂತ
ಕೆಚ್ಚಿ-ದ್ದರೆ
ಕೆಚ್ಚು
ಕೆಚ್ಚೆ-ದೆಯ
ಕೆಟ್ಟ
ಕೆಟ್ಟ-ದಾಗಿ
ಕೆಟ್ಟ-ದ್ದಿ-ರ-ಲಿ-ಹೆ-ದ-ರ-ಬೇಡ
ಕೆಟ್ಟ-ದ್ದೆನ್ನು
ಕೆಟ್ಟದ್ದೇ
ಕೆಟ್ಟು-ಹೋಗಿ
ಕೆಟ್ಟು-ಹೋ-ಗಿದೆ
ಕೆಟ್ಟು-ಹೋ-ಗು-ತ್ತ-ದ-ಲ್ಲವೆ
ಕೆಟ್ಟು-ಹೋ-ಯಿತು
ಕೆಡ-ಹ-ಬೇಕು
ಕೆಡಿ-ಸಿ-ಕೊ-ಳ್ಳದೆ
ಕೆಡಿ-ಸಿ-ಕೊ-ಳ್ಳ-ಬೇಕು
ಕೆಡಿ-ಸಿ-ಕೊ-ಳ್ಳ-ಲಿಲ್ಲ
ಕೆಡಿ-ಸಿ-ಕೊ-ಳ್ಳಲೂ
ಕೆಡಿ-ಸಿ-ದ್ದುದು
ಕೆಡು-ಕನ್ನು
ಕೆಡು-ಕುಂ-ಟಾ-ಗುವ
ಕೆಡು-ತ್ತದೆ
ಕೆಡೆ-ಸಿ-ಕೊ-ಳ್ಳ-ಬೇಡ
ಕೆಣ-ಕಿದ
ಕೆಣ-ಕಿ-ದಂ-ತಾ-ಯಿತು
ಕೆತ್ತನೆ
ಕೆತ್ತ-ನೆಯ
ಕೆನ-ಡಾದ
ಕೆರ-ಳಲು
ಕೆರ-ಳಿತು
ಕೆರ-ಳಿ-ಸಿತು
ಕೆರ-ಳಿ-ಸಿದ
ಕೆರ-ಳಿ-ಸಿ-ರ-ಲಿಲ್ಲ
ಕೆರೆಗೆ
ಕೆಲ
ಕೆಲ-ಕಾಲ
ಕೆಲ-ಕಾ-ಲದ
ಕೆಲ-ಕಾ-ಲ-ದಲ್ಲೇ
ಕೆಲ-ಕಾ-ಲ-ದಿಂ-ದಲೂ
ಕೆಲ-ತಿಂ-ಗಳ
ಕೆಲ-ತಿಂ-ಗ-ಳಾ-ದರೂ
ಕೆಲ-ದಿನ
ಕೆಲ-ದಿ-ನ-ಗಳ
ಕೆಲ-ದಿ-ನ-ಗಳನ್ನು
ಕೆಲ-ದಿ-ನ-ಗಳಲ್ಲಿ
ಕೆಲ-ದಿ-ನ-ಗ-ಳಲ್ಲೇ
ಕೆಲ-ದಿ-ನ-ಗ-ಳಾದ
ಕೆಲ-ದಿ-ನ-ಗ-ಳಿದ್ದು
ಕೆಲ-ದಿ-ನ-ಗಳು
ಕೆಲ-ದಿ-ನ-ವಿದ್ದು
ಕೆಲ-ನಿ-ಮಿ-ಷ-ಗ-ಳಲ್ಲೇ
ಕೆಲ-ಭಾ-ಗ-ಗಳನ್ನು
ಕೆಲ-ಭಾ-ಗ-ವನ್ನು
ಕೆಲ-ಮಂ-ದಿಗೆ
ಕೆಲ-ಮ-ಟ್ಟಿನ
ಕೆಲ-ವಂತೂ
ಕೆಲ-ವಂ-ಶ-ಗಳು
ಕೆಲ-ವನ್ನು
ಕೆಲ-ವರ
ಕೆಲ-ವ-ರನ್ನು
ಕೆಲ-ವ-ರ-ನ್ನು-ದ್ದೇ-ಶಿಸಿ
ಕೆಲ-ವ-ರಲ್ಲಿ
ಕೆಲ-ವ-ರಷ್ಟೆ
ಕೆಲ-ವ-ರಾ-ದರೆ
ಕೆಲ-ವ-ರಿ-ಗಾಗಿ
ಕೆಲ-ವ-ರಿ-ಗಾ-ದರೂ
ಕೆಲ-ವ-ರಿಗೆ
ಕೆಲ-ವರು
ಕೆಲ-ವರೂ
ಕೆಲ-ವ-ರೆಂ-ದರೆ
ಕೆಲ-ವ-ರೊಂ-ದಿಗೆ
ಕೆಲ-ವ-ರ್ಷ-ಗಳ
ಕೆಲ-ವ-ರ್ಷ-ಗಳಲ್ಲಿ
ಕೆಲ-ವಾ-ದರೂ
ಕೆಲ-ವಾ-ರ-ಗಳ
ಕೆಲವು
ಕೆಲವೆ
ಕೆಲ-ವೆಡೆ
ಕೆಲ-ವೆ-ಡೆ-ಗಳಲ್ಲಿ
ಕೆಲ-ವೆಲ್ಲ
ಕೆಲವೇ
ಕೆಲ-ವೊಂದು
ಕೆಲ-ವೊಮ್ಮೆ
ಕೆಲ-ವೊ-ಮ್ಮೆ-ಹಿಂದಿ
ಕೆಲಸ
ಕೆಲ-ಸ-ಅ-ದನ್ನು
ಕೆಲ-ಸ-ಕಾರ್ಯ
ಕೆಲ-ಸ-ಕಾ-ರ್ಯ-ಗಳ
ಕೆಲ-ಸ-ಕಾ-ರ್ಯ-ಗಳನ್ನು
ಕೆಲ-ಸ-ಕಾ-ರ್ಯ-ಗಳಿಂದ
ಕೆಲ-ಸ-ಕಾ-ರ್ಯ-ಗ-ಳಿಂ-ದಾಗಿ
ಕೆಲ-ಸ-ಕಾ-ರ್ಯ-ಗ-ಳಿಗೆ
ಕೆಲ-ಸ-ಕಾ-ರ್ಯ-ಗಳು
ಕೆಲ-ಸ-ಕಾ-ರ್ಯದ
ಕೆಲ-ಸ-ಕ್ಕಾ-ಗಲಿ
ಕೆಲ-ಸ-ಕ್ಕಾಗಿ
ಕೆಲ-ಸಕ್ಕೆ
ಕೆಲ-ಸ-ಗಳ
ಕೆಲ-ಸ-ಗಳನ್ನು
ಕೆಲ-ಸ-ಗಳನ್ನೆಲ್ಲ
ಕೆಲ-ಸ-ಗಳೂ
ಕೆಲ-ಸ-ಗ-ಳೆಲ್ಲ
ಕೆಲ-ಸ-ಗ-ಳೆ-ಲ್ಲವೂ
ಕೆಲ-ಸ-ಗಾರ
ಕೆಲ-ಸದ
ಕೆಲ-ಸ-ದಲ್ಲಿ
ಕೆಲ-ಸ-ದ-ಲ್ಲಿ-ಅ-ವರು
ಕೆಲ-ಸ-ದ-ಲ್ಲಿ-ರು-ವು-ದೇಕೆ
ಕೆಲ-ಸ-ದಿಂದ
ಕೆಲ-ಸ-ದಿಂ-ದಲೇ
ಕೆಲ-ಸ-ಮಾ-ಡ-ಬೇ-ಕಾ-ಗಿಲ್ಲ
ಕೆಲ-ಸ-ಮಾ-ಡಿ-ದರೆ
ಕೆಲ-ಸ-ಮಾ-ಡು-ತ್ತ-ವೆ-ಯೆಂ-ಬು-ದನ್ನು
ಕೆಲ-ಸ-ವನ್ನು
ಕೆಲ-ಸ-ವನ್ನೂ
ಕೆಲ-ಸ-ವನ್ನೇ
ಕೆಲ-ಸ-ವ-ನ್ನೇನೂ
ಕೆಲ-ಸ-ವಲ್ಲ
ಕೆಲ-ಸ-ವಾ-ಗಿತ್ತು
ಕೆಲ-ಸ-ವಿತ್ತು
ಕೆಲ-ಸ-ವಿಲ್ಲ
ಕೆಲ-ಸ-ವಿ-ಲ್ಲದ
ಕೆಲ-ಸವೂ
ಕೆಲ-ಸ-ವೆಂದು
ಕೆಲ-ಸ-ವೆ-ಲ್ಲವೂ
ಕೆಲ-ಸವೇ
ಕೆಲ-ಸ-ವೇನೂ
ಕೆಲ-ಸ-ವೇನೆಂದರೆ
ಕೆಲ-ಸ-ವೇ-ನೆಂದು
ಕೆಳ
ಕೆಳ-ಕೆ-ಳಕ್ಕೆ
ಕೆಳಕ್ಕೆ
ಕೆಳ-ಕ್ಕೆ-ಳೆದು
ಕೆಳ-ಕ್ಕೆ-ಳೆ-ಯು-ವಂತೆ
ಕೆಳ-ಕ್ಕೊ-ತ್ತು-ತ್ತಿ-ರುವ
ಕೆಳ-ಗ-ಡೆಯೂ
ಕೆಳ-ಗಿ-ಟ್ಟು-ಬಿ-ಟ್ಟರು
ಕೆಳ-ಗಿದ್ದ
ಕೆಳ-ಗಿ-ದ್ದೇವೆ
ಕೆಳ-ಗಿನ
ಕೆಳ-ಗಿ-ನ-ವ-ರನ್ನು
ಕೆಳ-ಗಿ-ಳಿದ
ಕೆಳ-ಗಿಳಿ-ದರು
ಕೆಳ-ಗಿ-ಳಿದು
ಕೆಳ-ಗಿ-ಳಿಯ
ಕೆಳ-ಗಿಳಿ-ಯ-ತೊ-ಡ-ಗಿ-ದಾಗ
ಕೆಳ-ಗಿಳಿ-ಯದೇ
ಕೆಳ-ಗಿಳಿ-ಯ-ಬೇ-ಕೆಂ-ಬು-ದಕ್ಕೆ
ಕೆಳ-ಗಿಳಿ-ಸ-ಬೇ-ಕಾ-ಯಿತು
ಕೆಳ-ಗಿಳಿ-ಸಿ-ದಷ್ಟು
ಕೆಳಗೆ
ಕೆಳ-ಗೆ-ಳೆ-ಯಲು
ಕೆಳ-ಗೆ-ಳೆ-ಯು-ತ್ತದೆ
ಕೆಳಗೇ
ಕೆಳ-ಜಾ-ತಿ-ಗಳ
ಕೆಳ-ಜಾ-ತಿಯ
ಕೆಳ-ದ-ರ್ಜೆಯ
ಕೆಳ-ಧು-ಮು-ಕು-ತ್ತಿ-ರುವ
ಕೆಳ-ಮ-ಟ್ಟದ
ಕೆಳ-ಮ-ಟ್ಟ-ದಿಂದ
ಕೆಳ-ವ-ರ್ಗದ
ಕೆಳ-ಸ್ತ-ರಕ್ಕೆ
ಕೆಳೆ-ಕ್ಕೆಳೆ
ಕೆಳೆಗೆ
ಕೆಸ-ರ-ನ್ನೆಲ್ಲ
ಕೇ
ಕೇಂದ್ರ
ಕೇಂದ್ರ-ಗಳನ್ನು
ಕೇಂದ್ರ-ಗ-ಳ-ನ್ನು-ಕ-ಲ್ಕ-ತ್ತ-ದ-ಲ್ಲೊಂದು
ಕೇಂದ್ರ-ಗಳಲ್ಲಿ
ಕೇಂದ್ರ-ಗಳಿಂದ
ಕೇಂದ್ರ-ಗಳು
ಕೇಂದ್ರ-ದಲ್ಲಿ
ಕೇಂದ್ರ-ಬಿಂದು
ಕೇಂದ್ರ-ವನ್ನು
ಕೇಂದ್ರ-ವಾ-ಗಿ-ಟ್ಟು-ಕೊಂಡು
ಕೇಂದ್ರ-ವಾ-ಗಿ-ದ್ದರು
ಕೇಂದ್ರ-ವಾ-ಗಿ-ದ್ದ-ವರು
ಕೇಂದ್ರ-ವಾ-ಗಿ-ರಿ-ಸಿ-ಕೊಂಡ
ಕೇಂದ್ರ-ವಾ-ಗಿ-ರು-ತ್ತದೆ
ಕೇಂದ್ರ-ವಾ-ಗಿ-ಸಿ-ಕೊಂಡು
ಕೇಂದ್ರ-ವಾದ
ಕೇಂದ್ರವು
ಕೇಂದ್ರ-ವೊಂ-ದನ್ನು
ಕೇಂದ್ರ-ಸ್ಥಾನ
ಕೇಂದ್ರ-ಸ್ಥಾ-ನ-ಕ್ಕಾಗಿ
ಕೇಂದ್ರ-ಸ್ಥಾ-ನ-ಗ-ಳಾದ
ಕೇಂದ್ರ-ಸ್ಥಾ-ನ-ವಾಗಿ
ಕೇಂದ್ರ-ಸ್ಥಾ-ನ-ವಾ-ಗಿ-ದ್ದಂತೆ
ಕೇಂದ್ರ-ಸ್ಥಾ-ನ-ವಾದ
ಕೇಂದ್ರ-ಸ್ಥಾ-ನ-ವಾ-ಯಿ-ತೆ-ನ್ನ-ಬ-ಹುದು
ಕೇಂದ್ರೀ-ಕೃ-ತ-ವಾ-ಗಿತ್ತು
ಕೇಂದ್ರೀ-ಕೃ-ತ-ವಾ-ಗಿ-ರ-ಬೇಕು
ಕೇಂಬ್ರಿ-ಡ್ಜಿನ
ಕೇಂಬ್ರಿ-ಡ್ಜಿ-ನಲ್ಲಿ
ಕೇಂಬ್ರಿ-ಡ್ಜ್
ಕೇಂಬ್ರಿಡ್ಜ್ಗೆ
ಕೇಂಬ್ರಿಡ್ಜ್ನ
ಕೇಂಬ್ರಿ-ಡ್ಜ್ನಲ್ಲಿ
ಕೇಂಬ್ರಿ-ಡ್ಜ್ನ-ಲ್ಲಿದ್ದ
ಕೇಟ್
ಕೇಡು
ಕೇಡು-ಗೈ-ಯಲು
ಕೇಡೆ-ಸ-ಗಿ-ದ-ರ-ವನ್ನೇ
ಕೇರಳ
ಕೇರ-ಳದ
ಕೇರ-ಳ-ದಲ್ಲಿ
ಕೇರಿ-ವ-ಠಾ-ರ-ಗ-ಳೊ-ಳಗೇ
ಕೇರೆ
ಕೇರೆ-ಹಾವು
ಕೇಳ
ಕೇಳದ
ಕೇಳದೆ
ಕೇಳ-ಬ-ಹು-ದಾ-ಗಿತ್ತು
ಕೇಳ-ಬ-ಹು-ದಾದ
ಕೇಳ-ಬೇ-ಕಲ್ಲ
ಕೇಳ-ಬೇ-ಕಾ-ಗು-ತ್ತಿತ್ತು
ಕೇಳ-ಬೇಕು
ಕೇಳ-ಬೇ-ಕೆಂಬ
ಕೇಳ-ರಿ-ಯ-ದಂ-ಥವು
ಕೇಳ-ರಿ-ಯ-ದಿದ್ದ
ಕೇಳ-ಲಾ-ರಂ-ಭಿ-ಸಿದ
ಕೇಳ-ಲಾ-ರಂ-ಭಿ-ಸಿ-ದರು
ಕೇಳಲು
ಕೇಳ-ಲೆಂದು
ಕೇಳಿ
ಕೇಳಿ-ಕೇಳಿ
ಕೇಳಿ-ಕೊಂಡ
ಕೇಳಿ-ಕೊಂ-ಡದ್ದೇ
ಕೇಳಿ-ಕೊಂ-ಡರು
ಕೇಳಿ-ಕೊಂ-ಡಳು
ಕೇಳಿ-ಕೊಂ-ಡಿದ್ದ
ಕೇಳಿ-ಕೊಂ-ಡಿ-ದ್ದರು
ಕೇಳಿ-ಕೊಂ-ಡಿ-ರು-ವರು
ಕೇಳಿ-ಕೊಂ-ಡಿ-ರು-ವು-ದರ
ಕೇಳಿ-ಕೊಂಡು
ಕೇಳಿ-ಕೊ-ಳ್ಳ-ಲಾ-ರಂ-ಭಿ-ಸಿ-ದರು
ಕೇಳಿ-ಕೊ-ಳ್ಳು-ತ್ತಾರೆ
ಕೇಳಿ-ಕೊ-ಳ್ಳು-ತ್ತಿದ್ದ
ಕೇಳಿ-ಕೊ-ಳ್ಳು-ತ್ತಿ-ದ್ದರು
ಕೇಳಿ-ಕೊ-ಳ್ಳು-ತ್ತಿದ್ದೆ
ಕೇಳಿ-ಕೊ-ಳ್ಳು-ತ್ತೇನೆ
ಕೇಳಿ-ಕೊ-ಳ್ಳು-ವುದು
ಕೇಳಿ-ಕೊ-ಳ್ಳು-ವು-ದು-ಇ-ದ-ರಷ್ಟು
ಕೇಳಿ-ತಿ-ಳಿದ
ಕೇಳಿತು
ಕೇಳಿದ
ಕೇಳಿ-ದಂತೆ
ಕೇಳಿ-ದಂ-ದಿ-ನಿಂದ
ಕೇಳಿ-ದ-ಮೇಲೆ
ಕೇಳಿ-ದರು
ಕೇಳಿ-ದ-ರು-ಕೆ-ಲವೇ
ಕೇಳಿ-ದರೂ
ಕೇಳಿ-ದರೆ
ಕೇಳಿ-ದರೇ
ಕೇಳಿ-ದರೋ
ಕೇಳಿ-ದ-ಳಾ-ದರೂ
ಕೇಳಿ-ದಳು
ಕೇಳಿ-ದ-ವರ
ಕೇಳಿ-ದ-ವ-ರಿಗೆ
ಕೇಳಿ-ದ-ವರು
ಕೇಳಿ-ದ-ವ-ರೊ-ಬ್ಬರು
ಕೇಳಿ-ದಾಗ
ಕೇಳಿ-ದಾ-ಗಲೂ
ಕೇಳಿ-ದಾ-ಗಿ-ನಿಂದ
ಕೇಳಿ-ದಿರಿ
ಕೇಳಿದೆ
ಕೇಳಿ-ದೊ-ಡ-ನೆಯೇ
ಕೇಳಿದ್ದ
ಕೇಳಿ-ದ್ದ-ರ-ಲ್ಲದೆ
ಕೇಳಿ-ದ್ದ-ರಷ್ಟೇ
ಕೇಳಿ-ದ್ದರು
ಕೇಳಿ-ದ್ದ-ರು-ನನ್ನ
ಕೇಳಿ-ದ್ದರೂ
ಕೇಳಿ-ದ್ದ-ವರೆಲ್ಲ
ಕೇಳಿ-ದ್ದ-ವಳು
ಕೇಳಿ-ದ್ದೀಯೋ
ಕೇಳಿದ್ದು
ಕೇಳಿ-ದ್ದೆವು
ಕೇಳಿ-ದ್ದೆವೋ
ಕೇಳಿ-ದ್ದೇನೆ
ಕೇಳಿ-ದ್ದೇವೆ
ಕೇಳಿ-ಬಂತು
ಕೇಳಿ-ಬಂ-ದ-ದ್ದ-ರಿಂದ
ಕೇಳಿ-ಬಂ-ದಿತು
ಕೇಳಿ-ಬಂ-ದಿಲ್ಲ
ಕೇಳಿ-ಬ-ರ-ದಿ-ದ್ದಾಗ
ಕೇಳಿ-ಬ-ರು-ತ್ತದೆ
ಕೇಳಿ-ಬ-ರು-ತ್ತಿ-ತ್ತಲ್ಲ
ಕೇಳಿ-ಬ-ರು-ತ್ತಿತ್ತು
ಕೇಳಿ-ಬ-ರು-ತ್ತಿ-ದ್ದುವು
ಕೇಳಿಯೇ
ಕೇಳಿ-ಯೇ-ಬಿ-ಟ್ಟರು
ಕೇಳಿ-ಯೇ-ಬಿಟ್ಟೆ
ಕೇಳಿ-ರ-ಬೇ-ಕ-ಲ್ಲವೆ
ಕೇಳಿ-ರ-ಲಿ-ಲ್ಲ-ವೆಂ-ದಲ್ಲ
ಕೇಳಿ-ರು-ವು-ದ-ರಿಂದ
ಕೇಳಿ-ಸ-ಲಿ-ಲ್ಲ-ವೆದು
ಕೇಳಿ-ಸಿ-ಕೊಂಡು
ಕೇಳಿ-ಸಿ-ಕೊ-ಳ್ಳದೆ
ಕೇಳಿ-ಸಿ-ಕೊ-ಳ್ಳುತ್ತಿ
ಕೇಳಿ-ಸಿತು
ಕೇಳಿ-ಸಿ-ಯಾ-ವೆಂಬ
ಕೇಳು
ಕೇಳು-ಗರ
ಕೇಳು-ಗ-ರನ್ನು
ಕೇಳು-ಗರು
ಕೇಳುತ್ತ
ಕೇಳು-ತ್ತಲೇ
ಕೇಳು-ತ್ತಾನೆ
ಕೇಳು-ತ್ತಾರೆ
ಕೇಳು-ತ್ತಾ-ರೆಯೇ
ಕೇಳುತ್ತಿ
ಕೇಳು-ತ್ತಿತ್ತು
ಕೇಳು-ತ್ತಿದ್ದ
ಕೇಳು-ತ್ತಿ-ದ್ದಂತೆ
ಕೇಳು-ತ್ತಿ-ದ್ದಂ-ತೆಯೇ
ಕೇಳು-ತ್ತಿ-ದ್ದರು
ಕೇಳು-ತ್ತಿ-ದ್ದರೆ
ಕೇಳು-ತ್ತಿ-ದ್ದಳು
ಕೇಳು-ತ್ತಿ-ದ್ದ-ವ-ರಿ-ಗೆಲ್ಲ
ಕೇಳು-ತ್ತಿ-ದ್ದಾರೆ
ಕೇಳು-ತ್ತಿ-ದ್ದೀ-ರಲ್ಲ
ಕೇಳು-ತ್ತಿ-ದ್ದು-ದುಂಟು
ಕೇಳು-ತ್ತಿ-ದ್ದೆವೋ
ಕೇಳು-ತ್ತಿ-ರ-ಬೇಕು
ಕೇಳು-ತ್ತೇನೆ
ಕೇಳುವ
ಕೇಳು-ವಂ-ತಹ
ಕೇಳು-ವಂ-ತಿ-ರ-ಲಿಲ್ಲ
ಕೇಳು-ವಲ್ಲಿ
ಕೇಳು-ವ-ವರೆ-ಲ್ಲರ
ಕೇಳು-ವಾಗ
ಕೇಳು-ವು-ದ-ಕ್ಕಾಗಿ
ಕೇಳು-ವು-ದ-ರಲ್ಲಿ
ಕೇಳು-ವು-ದ-ರಿಂ-ದಲೆ
ಕೇಳು-ವುದು
ಕೇಳು-ವು-ದೆಂ-ದರೆ
ಕೇಳೋ-ದಕ್ಕೆ
ಕೇಳ್ತಾ
ಕೇಳ್ತಾ-ಹೋ-ದರೆ
ಕೇವಲ
ಕೇಶದ
ಕೇಶ-ವ-ಚಂದ್ರ
ಕೇಶ-ವ-ಚಂ-ದ್ರ-ಸೇ-ನ-ನ-ನ್ನಂತೂ
ಕೇಶ-ವ-ಚಂ-ದ್ರ-ಸೇ-ನ-ನಲ್ಲಿ
ಕೇಶ-ವ-ಚಂ-ದ್ರ-ಸೇ-ನನೂ
ಕೇಶ-ವ-ಚಂ-ದ್ರ-ಸೇ-ನನೇ
ಕೇಶ-ವ-ಚಂ-ದ್ರ-ಸೇ-ನರು
ಕೇಶ-ವ-ಚಂ-ದ್ರ-ಸೇ-ನ-ರೊ-ಬ್ಬ-ರನ್ನು
ಕೇಶ-ವ-ಸೇ-ನರು
ಕೇಸರಿ
ಕೇಸ-ರಿ-ಹ-ಳದಿ
ಕೇಸ-ರಿ-ಯಾ-ಗು-ತ್ತಿ-ದ್ದರು
ಕೈ
ಕೈಕ-ಟ್ಟಿ-ಕೊಂಡು
ಕೈಕಾಲು
ಕೈಕಾ-ಲು-ಗಳು
ಕೈಕುಲು
ಕೈಕು-ಲು-ಕುವ
ಕೈಕೆ-ಳಗೂ
ಕೈಕೊ-ಟ್ಟ-ದ್ದ-ರಿಂದ
ಕೈಗ-ಳ-ಲ್ಲೆತ್ತಿ
ಕೈಗಳಿಂದ
ಕೈಗಳು
ಕೈಗಾ-ರಿಕಾ
ಕೈಗಾ-ರಿಕೆ
ಕೈಗಾ-ರಿ-ಕೆ-ಗಳ
ಕೈಗಾ-ರಿ-ಕೆ-ಗಳು
ಕೈಗಾ-ರಿ-ಕೆ-ಗಳೂ
ಕೈಗಾ-ರಿ-ಕೋ-ತ್ಪನ್ನ
ಕೈಗಾ-ರಿ-ಕೋ-ತ್ಪ-ನ್ನ-ಗಳನ್ನು
ಕೈಗೂ-ಡಿದ್ದು
ಕೈಗೂ-ಡಿ-ರ-ಲಿಲ್ಲ
ಕೈಗೂಡು
ಕೈಗೂ-ಡು-ತ್ತದೆ
ಕೈಗೂ-ಡು-ವು-ದೆಂಬ
ಕೈಗೆ
ಕೈಗೆ-ತ್ತಿ-ಕೊಂ-ಡರು
ಕೈಗೆ-ತ್ತಿ-ಕೊಂ-ಡರೆ
ಕೈಗೆ-ತ್ತಿ-ಕೊಂ-ಡಿತು
ಕೈಗೆ-ತ್ತಿ-ಕೊಂಡು
ಕೈಗೆ-ತ್ತಿ-ಕೊ-ಳ್ಳಲು
ಕೈಗೆ-ತ್ತಿ-ಕೊ-ಳ್ಳು-ತ್ತಾರೆ
ಕೈಗೆ-ತ್ತಿ-ಕೊ-ಳ್ಳು-ತ್ತಿ-ದ್ದರು
ಕೈಗೆ-ತ್ತಿ-ಕೊ-ಳ್ಳುವ
ಕೈಗೇ
ಕೈಗೊಂಡ
ಕೈಗೊಂ-ಡ-ದ್ದ-ರಿಂದ
ಕೈಗೊಂ-ಡದ್ದು
ಕೈಗೊಂ-ಡ-ಮೇ-ಲಂತೂ
ಕೈಗೊಂ-ಡರು
ಕೈಗೊಂ-ಡಾಗ
ಕೈಗೊಂ-ಡಿದ್ದ
ಕೈಗೊಂ-ಡಿ-ದ್ದ-ವರು
ಕೈಗೊಂ-ಡಿ-ದ್ದಾಗ
ಕೈಗೊಂ-ಡಿ-ದ್ದಾ-ರೆಂ-ಬುದು
ಕೈಗೊಂ-ಡಿ-ರುವ
ಕೈಗೊಂ-ಡಿ-ರು-ವುದು
ಕೈಗೊಂಡು
ಕೈಗೊಂಡೆ
ಕೈಗೊ-ಪ್ಪಿ-ಸಿ-ಬಿ-ಟ್ಟರು
ಕೈಗೊಳ್ಳ
ಕೈಗೊ-ಳ್ಳ-ಬೇ-ಕಾ-ಗಿತ್ತು
ಕೈಗೊ-ಳ್ಳ-ಬೇ-ಕಾದ
ಕೈಗೊ-ಳ್ಳ-ಬೇಕು
ಕೈಗೊ-ಳ್ಳ-ಬೇ-ಕೆಂದು
ಕೈಗೊ-ಳ್ಳ-ಬೇ-ಕೆ-ನ್ನು-ವುದೂ
ಕೈಗೊ-ಳ್ಳ-ಬೇಕೇ
ಕೈಗೊ-ಳ್ಳ-ಲಿ-ರುವ
ಕೈಗೊ-ಳ್ಳಲು
ಕೈಗೊ-ಳ್ಳುವ
ಕೈಗೊ-ಳ್ಳು-ವಾಗ
ಕೈಗೊ-ಳ್ಳು-ವುದು
ಕೈಚ-ಳಕ
ಕೈಚೆಲ್ಲಿ
ಕೈಜೋ-ಡಿಸಿ
ಕೈಜೋ-ಡಿ-ಸಿ-ಕೊಂಡು
ಕೈತುಂಬ
ಕೈತೊ-ಳೆದು
ಕೈತೊ-ಳೆ-ದು-ಕೊ-ಳ್ಳಲು
ಕೈತೊ-ಳೆ-ಯುವ
ಕೈಪೆ-ಟ್ಟಿ-ಗೆ-ಇ-ವೆಲ್ಲ
ಕೈಬಿ-ಟ್ಟಂತೆ
ಕೈಬಿ-ಟ್ಟ-ನೆಂ-ಬುದು
ಕೈಬಿ-ಟ್ಟರು
ಕೈಬಿ-ಟ್ಟರೆ
ಕೈಬಿ-ಟ್ಟಿ-ದ್ದರು
ಕೈಬಿ-ಟ್ಟಿಲ್ಲ
ಕೈಬಿ-ಟ್ಟು-ಹೋ-ಯಿತು
ಕೈಬಿ-ಡ-ಲಾ-ರ-ನೆಂದು
ಕೈಬಿ-ಡಲು
ಕೈಬಿ-ಡು-ವ-ನೇನು
ಕೈಬಿ-ಡು-ವು-ದಿಲ್ಲ
ಕೈಬಿ-ಡು-ವುದೂ
ಕೈಬಿ-ಡು-ವೆಯಾ
ಕೈಬೀಸಿ
ಕೈಬೆ-ರ-ಳನ್ನು
ಕೈಬೆ-ರ-ಳು-ಗಳಿಂದ
ಕೈಮೀರಿ
ಕೈಮು-ಗಿ-ದರು
ಕೈಮು-ಗಿದು
ಕೈಯ-ಡಿಗೆ
ಕೈಯ-ದ್ದದ
ಕೈಯ-ನ್ನಿ-ಟ್ಟು-ಕೊಂಡು
ಕೈಯನ್ನು
ಕೈಯ-ನ್ನೆತ್ತಿ
ಕೈಯ-ನ್ನೊಮ್ಮೆ
ಕೈಯಲ್ಲ
ಕೈಯಲ್ಲಿ
ಕೈಯ-ಲ್ಲಿಟ್ಟ
ಕೈಯ-ಲ್ಲಿ-ಟ್ಟು-ಕೊಂ-ಡರೆ
ಕೈಯ-ಲ್ಲಿ-ಟ್ಟೇ-ಬಿ-ಟ್ಟರು
ಕೈಯ-ಲ್ಲಿ-ಡುತ್ತ
ಕೈಯ-ಲ್ಲಿದ್ದ
ಕೈಯ-ಲ್ಲಿನ
ಕೈಯ-ಲ್ಲಿ-ಸಾ-ಧ್ಯ-ವಾದ
ಕೈಯ-ಲ್ಲೇಕೆ
ಕೈಯ-ಲ್ಲೊಂದು
ಕೈಯಾ-ಸ-ರೆ-ಗಳ
ಕೈಯಿಂದ
ಕೈಯಿ-ಟ್ಟರೆ
ಕೈಯೆತ್ತಿ
ಕೈಯೊ-ಳ-ಗಿ-ದ್ದೇನೆ
ಕೈಲಾ
ಕೈಲಾದ
ಕೈಲಾ-ದಷ್ಟು
ಕೈಲಿ-ರಿ-ಸು-ತ್ತಿ-ದ್ದೇನೆ
ಕೈವಾ-ಡ-ವ-ಲ್ಲದೆ
ಕೈಸಾಲೆ
ಕೈಸಾ-ಲೆಯೇ
ಕೈಸೇ-ರಿತು
ಕೈಸೇ-ರಿ-ದ-ವು-ಗಳು
ಕೈಸೇ-ರಿದ್ದು
ಕೈಸೇ-ರಿಸಿ
ಕೈಸೇ-ರು-ವಲ್ಲಿ
ಕೈಹಾಕಿ
ಕೈಹಿ-ಡಿದು
ಕೊಂಕಿನ
ಕೊಂಡ
ಕೊಂಡಂ-ತಹ
ಕೊಂಡರು
ಕೊಂಡಳು
ಕೊಂಡ-ವರ
ಕೊಂಡಾಗ
ಕೊಂಡಾಡಿ
ಕೊಂಡಾ-ಡಿತು
ಕೊಂಡಾ-ಡಿ-ದ-ನ-ಲ್ಲದೆ
ಕೊಂಡಾ-ಡಿ-ದರು
ಕೊಂಡಾ-ಡಿ-ದರೂ
ಕೊಂಡಾ-ಡಿ-ದುವು
ಕೊಂಡಾ-ಡಿ-ದ್ದನೋ
ಕೊಂಡಾ-ಡುತ್ತ
ಕೊಂಡಾ-ಡು-ತ್ತಾರೆ
ಕೊಂಡಾ-ಡು-ತ್ತಿ-ದ್ದರು
ಕೊಂಡಾ-ಡು-ತ್ತಿ-ದ್ದಾ-ರಲ್ಲ
ಕೊಂಡಾ-ಡು-ತ್ತಿ-ದ್ದು-ದನ್ನು
ಕೊಂಡಾ-ಡು-ವ-ವರು
ಕೊಂಡಿತು
ಕೊಂಡಿ-ತ್ತಾ-ದರೂ
ಕೊಂಡಿದೆ
ಕೊಂಡಿದ್ದ
ಕೊಂಡಿ-ದ್ದರು
ಕೊಂಡಿ-ದ್ದರೋ
ಕೊಂಡಿ-ದ್ದ-ವನು
ಕೊಂಡಿ-ದ್ದೀ-ರಲ್ಲ
ಕೊಂಡಿದ್ದೆ
ಕೊಂಡಿ-ದ್ದೇನೆ
ಕೊಂಡಿ-ರ-ಲಿಲ್ಲ
ಕೊಂಡಿ-ರಲು
ಕೊಂಡಿಲ್ಲ
ಕೊಂಡು
ಕೊಂಡು-ಕೊಂ-ಡಿ-ದ್ದರು
ಕೊಂಡು-ಕೊಂಡು
ಕೊಂಡು-ಕೊ-ಳ್ಳ-ಬೇ-ಕಾ-ಗಿತ್ತು
ಕೊಂಡು-ಕೊ-ಳ್ಳು-ವಂತೆ
ಕೊಂಡು-ಬ-ರು-ತ್ತಿದ್ದ
ಕೊಂಡೇ
ಕೊಂಡೊಯ್ದು
ಕೊಂಡೊ-ಯ್ಯಲು
ಕೊಂದಿದ್ದ
ಕೊಂದು-ಬಿ-ಟ್ಟೆ-ನಲ್ಲಾ
ಕೊಂಬೆ
ಕೊಚ್ಚಿ
ಕೊಚ್ಚಿ-ಕೊಂ-ಡರು
ಕೊಚ್ಚಿ-ಕೊಂಡು
ಕೊಚ್ಚಿ-ಕೊ-ಳ್ಳು-ತ್ತಿ-ದ್ದ-ವನು
ಕೊಚ್ಚಿ-ಕೊ-ಳ್ಳು-ತ್ತೀರಿ
ಕೊಚ್ಚಿನ್
ಕೊಚ್ಚಿ-ನ್ನಿನ
ಕೊಚ್ಚಿ-ಹೋ-ದರು
ಕೊಚ್ಚಿ-ಹೋ-ಯಿತು
ಕೊಚ್ಚುಣ್ಣಿ
ಕೊಚ್ಚೆ-ಗುಂಡಿ
ಕೊಚ್ಚೆಗೆ
ಕೊಟ್ಟ
ಕೊಟ್ಟಂ-ತಾ-ಗು-ತ್ತದೆ
ಕೊಟ್ಟರು
ಕೊಟ್ಟ-ರು-ಈಗ
ಕೊಟ್ಟ-ರು-ಪ್ರ-ಯೋ-ಜ-ನ-ವಿದೆ
ಕೊಟ್ಟರೂ
ಕೊಟ್ಟರೆ
ಕೊಟ್ಟಳು
ಕೊಟ್ಟಾಗ
ಕೊಟ್ಟಾನು
ಕೊಟ್ಟಾ-ನೆಯೆ
ಕೊಟ್ಟಿತು
ಕೊಟ್ಟಿದ್ದ
ಕೊಟ್ಟಿ-ದ್ದರು
ಕೊಟ್ಟಿ-ದ್ದರೆ
ಕೊಟ್ಟಿ-ದ್ದಾರೆ
ಕೊಟ್ಟಿದ್ದು
ಕೊಟ್ಟಿ-ದ್ದೇವೆ
ಕೊಟ್ಟಿ-ರ-ಬೇಕು
ಕೊಟ್ಟಿ-ರ-ಲಾ-ರರು
ಕೊಟ್ಟಿ-ರಲೂ
ಕೊಟ್ಟಿ-ರು-ವ-ನ-ಲ್ಲವೆ
ಕೊಟ್ಟಿ-ರು-ವು-ದೇ-ಕೆಂ-ದರೆ
ಕೊಟ್ಟೀರಾ
ಕೊಟ್ಟು
ಕೊಟ್ಟುದೇ
ಕೊಟ್ಟು-ಬಿ-ಟ್ಟರು
ಕೊಟ್ಟು-ಬಿಟ್ಟಿ
ಕೊಟ್ಟು-ಬಿ-ಡ-ಬೇಕು
ಕೊಟ್ಟು-ಬಿಡು
ಕೊಟ್ಟು-ಬಿ-ಡು-ವುದು
ಕೊಠ-ಡಿ-ಯಲ್ಲಿ
ಕೊಡ-ದಿ-ರ-ಬೇ-ಕಾ-ದರೆ
ಕೊಡದೆ
ಕೊಡ-ಬ-ಲ್ಲಂ-ತಹ
ಕೊಡ-ಬ-ಲ್ಲು-ದಾ-ದರೆ
ಕೊಡ-ಬ-ಹುದು
ಕೊಡ-ಬೇ-ಕಾ-ಗಿ-ದ್ದುದು
ಕೊಡ-ಬೇ-ಕಾ-ಗಿಯೂ
ಕೊಡ-ಬೇ-ಕಾ-ದ-ದ್ದಷ್ಟೇ
ಕೊಡ-ಬೇ-ಕಾ-ದರೆ
ಕೊಡ-ಬೇಕು
ಕೊಡ-ಬೇಕೆ
ಕೊಡ-ಬೇ-ಕೆಂಬ
ಕೊಡ-ಬೇಡ
ಕೊಡ-ಬೇಡಿ
ಕೊಡ-ಲಾ-ಯಿತು
ಕೊಡ-ಲಾ-ರಂ-ಭಿ-ಸಿ-ದರು
ಕೊಡ-ಲಿಲ್ಲ
ಕೊಡಲು
ಕೊಡಲೇ
ಕೊಡ-ಲೇ-ಬೇ-ಕು-ಅದು
ಕೊಡ-ವನ್ನು
ಕೊಡವಿ
ಕೊಡ-ವಿ-ಕೊಂ-ಡು-ಬಿ-ಟ್ಟಿ-ದ್ದರು
ಕೊಡ-ವಿ-ಹಾ-ಕುವ
ಕೊಡಹಿ
ಕೊಡಿ
ಕೊಡಿ-ಸಿದ
ಕೊಡಿ-ಸಿ-ದರು
ಕೊಡಿ-ಸಿ-ದಳು
ಕೊಡಿ-ಸಿ-ದ್ದ-ರಾ-ದರೂ
ಕೊಡಿ-ಸಿ-ರ-ಲಿಲ್ಲ
ಕೊಡಿ-ಸು-ತ್ತೇನೆ
ಕೊಡು
ಕೊಡುಗೆ
ಕೊಡು-ಗೆ-ಯನ್ನು
ಕೊಡು-ಗೆ-ಯಾಗಿ
ಕೊಡು-ಗೆ-ಯೆಂ-ದರೆ
ಕೊಡುತ್ತ
ಕೊಡುತ್ತಾ
ಕೊಡು-ತ್ತಾ-ನಲ್ಲ
ಕೊಡು-ತ್ತಾನೆ
ಕೊಡು-ತ್ತಾ-ರಂತೆ
ಕೊಡು-ತ್ತಾರೆ
ಕೊಡುತ್ತಿ
ಕೊಡು-ತ್ತಿತ್ತು
ಕೊಡು-ತ್ತಿ-ದ್ದರು
ಕೊಡು-ತ್ತಿ-ದ್ದ-ರೇನೋ
ಕೊಡು-ತ್ತಿ-ದ್ದುದು
ಕೊಡು-ತ್ತಿ-ರ-ಲಿಲ್ಲ
ಕೊಡು-ತ್ತಿ-ರುವ
ಕೊಡು-ತ್ತೇನೆ
ಕೊಡುವ
ಕೊಡು-ವಂ-ತಾ-ಗಲಿ
ಕೊಡು-ವಂ-ತಿದೆ
ಕೊಡು-ವಲ್ಲಿ
ಕೊಡು-ವ-ವರು
ಕೊಡು-ವುದನ್ನು
ಕೊಡು-ವು-ದಾಗಿ
ಕೊಡು-ವು-ದಾ-ದರೆ
ಕೊಡು-ವು-ದಿದೆ
ಕೊಡು-ವು-ದಿಲ್ಲ
ಕೊಡು-ವು-ದಿ-ಲ್ಲ-ವೆಂದು
ಕೊಡು-ವುದು
ಕೊಡು-ವುದೇ
ಕೊನೆ
ಕೊನೆ-ಗಾ-ಣ-ಬಾ-ರದು
ಕೊನೆಗೂ
ಕೊನೆಗೆ
ಕೊನೆ-ಗೊಂ-ಡಿತ್ತು
ಕೊನೆ-ಗೊಂದು
ಕೊನೆ-ಗೊ-ಳಿ-ಸಿ-ಬಿ-ಟ್ಟಿ-ರ-ಬೇ-ಕೆಂದು
ಕೊನೆ-ಮೊ-ದ-ಲಿ-ಲ್ಲ-ದಂತೆ
ಕೊನೆಯ
ಕೊನೆ-ಯ-ದಾಗಿ
ಕೊನೆ-ಯದು
ಕೊನೆ-ಯ-ದೆಂ-ಬಂತೆ
ಕೊನೆ-ಯಲ್ಲಿ
ಕೊನೆ-ಯ-ಲ್ಲಿ-ಅ-ಧ್ಯ-ಕ್ಷರ
ಕೊನೆ-ಯ-ವ-ರೆಗೂ
ಕೊನೆ-ಯಿ-ಲ್ಲದ
ಕೊನೆ-ಯು-ಸಿ-ರಿ-ನ-ವ-ರೆಗೂ
ಕೊನೆ-ಯು-ಸಿ-ರಿ-ನ-ವ-ರೆಗೆ
ಕೊಬ್ಬಿನ
ಕೊರ-ಗುತ್ತ
ಕೊರ-ಡಿ-ನಂ-ತಾ-ಗಿ-ಬಿ-ಟ್ಟಿ-ದ್ದುವು
ಕೊರ-ತೆ-ಗಳನ್ನೂ
ಕೊರ-ತೆ-ಯಾ-ಗ-ದಂತೆ
ಕೊರ-ತೆ-ಯಿಂ-ದಾಗಿ
ಕೊರ-ಳನ್ನು
ಕೊರ-ಳಿಗೆ
ಕೊರಾನ್
ಕೊರಾ-ನ್ಗಳ
ಕೊರೆದ
ಕೊರೆ-ಯ-ತೊ-ಡ-ಗಿತು
ಕೊರೆ-ಯು-ತ್ತಿದ್ದ
ಕೊರೆ-ಯುವ
ಕೊಲಂ-ಬಸ್
ಕೊಲಂ-ಬಿಯ
ಕೊಲಂ-ಬಿ-ಯಕ್ಕೆ
ಕೊಲಂ-ಬಿ-ಯನ್
ಕೊಲೊಂಬೊ
ಕೊಲೊಂ-ಬೊದ
ಕೊಲೋನ್
ಕೊಲ್ಲ-ಬ-ಲ್ಲ-ವರು
ಕೊಲ್ಲ-ಬೇಡ
ಕೊಲ್ಲಲು
ಕೊಲ್ಲಾ-ಪು-ರಕ್ಕೆ
ಕೊಲ್ಲಾ-ಪು-ರದ
ಕೊಲ್ಲಾ-ಪು-ರ-ದಲ್ಲಿ
ಕೊಲ್ಲಾ-ಪುರಿ
ಕೊಲ್ಲು-ತ್ತಾರೆ
ಕೊಲ್ಲು-ತ್ತಾರೋ
ಕೊಳ-ಕಾದ
ಕೊಳಕು
ಕೊಳ-ಗ-ದಂ-ತಿತ್ತು
ಕೊಳೆ
ಕೊಳ್ಳದೆ
ಕೊಳ್ಳ-ಬ-ಹು-ದಾ-ಗಿತ್ತು
ಕೊಳ್ಳ-ಬ-ಹುದು
ಕೊಳ್ಳ-ಬಾ-ರ-ದೆಂದು
ಕೊಳ್ಳ-ಬೇ-ಕಾ-ಗಿತ್ತು
ಕೊಳ್ಳ-ಬೇ-ಕಾ-ಗಿದೆ
ಕೊಳ್ಳ-ಬೇ-ಕಾದ
ಕೊಳ್ಳ-ಬೇ-ಕಾ-ದುದು
ಕೊಳ್ಳ-ಬೇಕು
ಕೊಳ್ಳ-ಬೇ-ಕೆಂ-ದರೂ
ಕೊಳ್ಳ-ಲಾ-ಯಿತು
ಕೊಳ್ಳ-ಲಾ-ರಂ-ಭಿ-ಸಿತು
ಕೊಳ್ಳಲಿ
ಕೊಳ್ಳ-ಲಿದ್ದ
ಕೊಳ್ಳಲು
ಕೊಳ್ಳ-ಲೇ-ಬೇ-ಕ-ಲ್ಲವೆ
ಕೊಳ್ಳ-ಲೇ-ಬೇ-ಕಾ-ಗಿತ್ತು
ಕೊಳ್ಳ-ಲೇ-ಬೇಕು
ಕೊಳ್ಳಿ
ಕೊಳ್ಳುತ್ತ
ಕೊಳ್ಳು-ತ್ತದೆ
ಕೊಳ್ಳು-ತ್ತಾನೆ
ಕೊಳ್ಳು-ತ್ತಾರೆ
ಕೊಳ್ಳು-ತ್ತಿದ್ದ
ಕೊಳ್ಳು-ತ್ತಿ-ದ್ದೀಯೆ
ಕೊಳ್ಳು-ತ್ತಿ-ದ್ದುದು
ಕೊಳ್ಳು-ತ್ತೀರೊ
ಕೊಳ್ಳು-ತ್ತೇನೆ
ಕೊಳ್ಳುವ
ಕೊಳ್ಳು-ವಂತೆ
ಕೊಳ್ಳು-ವು-ದ-ಕ್ಕೋ-ಸ್ಕರ
ಕೊಳ್ಳು-ವುದನ್ನು
ಕೊಳ್ಳು-ವು-ದ-ರಲ್ಲೇ
ಕೊಳ್ಳು-ವುದು
ಕೊಸ-ಗುಟ್ಟಿ
ಕೊಸ-ರಾ-ಡು-ತ್ತಿ-ರು-ತ್ತದೆ
ಕೊಸ-ರಾ-ಡು-ವು-ದಕ್ಕೂ
ಕೊಸ-ರಾ-ಡು-ವುದು
ಕೋಟನ್ನು
ಕೋಟಾ-ಪ್ರಾಂ-ತದ
ಕೋಟಿ
ಕೋಟಿ-ಕೋಟಿ
ಕೋಟಿ-ಗಳಿಂದ
ಕೋಟು
ಕೋಟು-ಗಳನ್ನು
ಕೋಟೆ-ಯಂ-ತಿದೆ
ಕೋಟೆ-ಯನ್ನು
ಕೋಟೆಯು
ಕೋಟ್ಯಂ-ತರ
ಕೋಟ್ಯ-ಧಿ-ಪ-ತಿ-ಗಳು
ಕೋಟ್ಯ-ಧಿ-ಪ-ತಿ-ಯಾದ
ಕೋಟ್ಯ-ಧೀ-ಶರ
ಕೋಟ್ಯ-ಧೀ-ಶರು
ಕೋಟ್ಯ-ನು-ಕೋಟಿ
ಕೋಟ್ಯ-ವಧಿ
ಕೋಟ್ಯಾಂ-ತರ
ಕೋಡಂ-ಗ-ಲ್ಲೂರ್
ಕೋಡೋಣ
ಕೋಣೆ
ಕೋಣೆ-ಗಳನ್ನು
ಕೋಣೆ-ಗಳಲ್ಲಿ
ಕೋಣೆ-ಗ-ಳಲ್ಲೂ
ಕೋಣೆ-ಗಳು
ಕೋಣೆಗೆ
ಕೋಣೆಯ
ಕೋಣೆ-ಯನ್ನು
ಕೋಣೆ-ಯಲ್ಲಿ
ಕೋಣೆ-ಯ-ಲ್ಲಿ-ದ್ದರು
ಕೋಣೆ-ಯಲ್ಲೇ
ಕೋಣೆ-ಯಿಂದ
ಕೋಣೆ-ಯಿಂ-ದಾಚೆ
ಕೋಣೆಯೇ
ಕೋಣೆ-ಯೊಂ-ದನ್ನು
ಕೋಣೆ-ಯೊಂ-ದಿದೆ
ಕೋಣೆ-ಯೊ-ಳಕ್ಕೆ
ಕೋಣೆ-ಯೊ-ಳಗೆ
ಕೋಣೆ-ಯೊ-ಳ್ಳಕ್ಕೆ
ಕೋತಿ
ಕೋತಿಯ
ಕೋತಿ-ಯೊಂದು
ಕೋನ-ಗಳ
ಕೋಪ
ಕೋಪದ
ಕೋಪ-ದಿಂದ
ಕೋಪ-ದಿಂ-ದಲೇ
ಕೋಪ-ವಿಲ್ಲ
ಕೋಪಿ-ಸಿ-ಕೊಂ-ಡಿ-ರು-ವುದು
ಕೋಪಿ-ಸಿ-ಕೊ-ಳ್ಳ-ಬೇಡಿ
ಕೋಬೆ
ಕೋಬೆ-ಯಿಂದ
ಕೋಮಿಗೆ
ಕೋಮಿನ
ಕೋಮು-ವಾರು
ಕೋರ-ಲಾ-ಗಿತ್ತು
ಕೋರಿ
ಕೋರಿ-ಕೆ-ಗಳು
ಕೋರಿ-ಕೆಗೆ
ಕೋರಿ-ಕೆಯ
ಕೋರಿ-ಕೆ-ಯಂತೆ
ಕೋರಿ-ಕೆ-ಯನ್ನು
ಕೋರಿ-ಕೆ-ಯಿದೆ
ಕೋರಿ-ದರು
ಕೋರಿ-ದ್ದರು
ಕೋರು-ತ್ತಾರೆ
ಕೋರುವ
ಕೋರೈ-ಸಿ-ದಂತೆ
ಕೋರೈ-ಸುವ
ಕೋರೈ-ಸು-ವಂತೆ
ಕೋಲಾ-ಹಲ
ಕೋಲಾ-ಹ-ಲಕ್ಕೆ
ಕೋಲಾ-ಹ-ಲ-ವನ್ನೇ
ಕೋಲಾ-ಹ-ಲವೇ
ಕೋಲ್ಬ್ರೂಕ್
ಕೋಲ್ಮಿಂಚು
ಕೋವಿ-ಗಳಿಂದ
ಕೋಶ-ಕ್ಕಾಗಿ
ಕೌತು-ಕದ
ಕೌತು-ಕ-ವನ್ನು
ಕೌನ್ಸಿ-ಲ-ರಾದ
ಕೌಪೀನ
ಕೌಪೀ-ನ-ವನ್ನು
ಕೌಬಾಯ್
ಕೌಶಲ
ಕೌಶ-ಲ-ಇವು
ಕೌಶ-ಲ-ದಿಂದ
ಕ್ಕಾಗಿ
ಕ್ಕಾಗಿಯೇ
ಕ್ಕಿಂತ
ಕ್ಕಿಂತಲೂ
ಕ್ಕಿರುವ
ಕ್ಕೆಸೆದು
ಕ್ಕೇರಿ-ಸಿ-ದರು
ಕ್ಕೇರು-ತ್ತಿದ್ದ
ಕ್ಕೊಂದು
ಕ್ಕೊತ್ತುತ್ತ
ಕ್ಯಾಂಟನ್
ಕ್ಯಾಂಟ-ನ್ನಿ-ನಲ್ಲಿ
ಕ್ಯಾಂಟ-ರ್ಬ-ರಿಯ
ಕ್ಯಾಂಪ್
ಕ್ಯಾಂಬೆಲ್
ಕ್ಯಾಥ-ರಿನ್
ಕ್ಯಾಥ-ರೀನ್
ಕ್ಯಾಥೋ-ಲಿ-ಕರ
ಕ್ಯಾಥೋ-ಲಿಕ್
ಕ್ಯಾನನ್
ಕ್ಯಾಪ್ಟನ್
ಕ್ಯಾಮರಾ
ಕ್ಯಾಲಿ-ಫೋ-ರ್ನಿ-ಯ-ದಲ್ಲಿ
ಕ್ಯೋಟೊ
ಕ್ರಂಗ-ನೂ-ರನ್ನು
ಕ್ರಂಗ-ನೂ-ರಿಗೆ
ಕ್ರಂಗ-ನೂ-ರಿನ
ಕ್ರಂದ-ನ-ವನ್ನು
ಕ್ರತು
ಕ್ರಮ
ಕ್ರಮ-ಗಳನ್ನು
ಕ್ರಮ-ಗ-ಳಿ-ಗೆಲ್ಲ
ಕ್ರಮ-ದಂತೆ
ಕ್ರಮ-ದ-ಲ್ಲಿನ
ಕ್ರಮ-ಬದ್ಧ
ಕ್ರಮ-ಬ-ದ್ಧ-ವಾಗಿ
ಕ್ರಮ-ಬ-ದ್ಧ-ವಾದ
ಕ್ರಮ-ಯು-ತ-ವಾದ
ಕ್ರಮ-ವನ್ನು
ಕ್ರಮ-ವಾಗಿ
ಕ್ರಮ-ವಿತ್ತು
ಕ್ರಮ-ವೆಂ-ಥದೋ
ಕ್ರಮಾ-ಗ-ತ-ವಾಗಿ
ಕ್ರಮಿ-ಸ-ಬೇ-ಕಾ-ಗಿತ್ತು
ಕ್ರಮಿಸಿ
ಕ್ರಮಿ-ಸಿ-ದರು
ಕ್ರಮೇಣ
ಕ್ರಯ-ವಿ-ಕ್ರಯ
ಕ್ರಾಂತಿ
ಕ್ರಾಂತಿ-ಕಾ-ರ-ಕ-ವಾ-ದವು
ಕ್ರಾಂತಿ-ಕಾರಿ
ಕ್ರಾಂತಿ-ಕಾ-ರಿ-ಗ-ಳಿಗೂ
ಕ್ರಾಂತಿ-ಕಾ-ರಿ-ಗ-ಳೆಂದು
ಕ್ರಾಂತಿ-ಕಾ-ರಿ-ಪು-ರು-ಷ-ನಾಗಿ
ಕ್ರಾಂತಿ-ಕಾರೀ
ಕ್ರಾಂತಿ-ಯನ್ನು
ಕ್ರಾಂತಿಯು
ಕ್ರಾಂತಿ-ಯುಂಟು
ಕ್ರಾನಿ-ಕಲ್
ಕ್ರಿ
ಕ್ರಿಕೆಟ್
ಕ್ರಿಟಿಕ್
ಕ್ರಿಮಿ
ಕ್ರಿಮಿ-ಗಳೋ
ಕ್ರಿಮಿ-ಯಂತೆ
ಕ್ರಿಮಿಯೋ
ಕ್ರಿಯಾ
ಕ್ರಿಯಾ-ಶೀಲ
ಕ್ರಿಯಾ-ಶೀ-ಲವೂ
ಕ್ರಿಯೆಯು
ಕ್ರಿಶ್ಚಿ-ಯನ್
ಕ್ರಿಸ್
ಕ್ರಿಸ್ಟೀನ
ಕ್ರಿಸ್ಟೀ-ನ-ರಿ-ಗಾದ
ಕ್ರಿಸ್ಟೀ-ನಳ
ಕ್ರಿಸ್ಟೀ-ನ-ಳನ್ನು
ಕ್ರಿಸ್ಟೀನಾ
ಕ್ರಿಸ್ಟೀನ್
ಕ್ರಿಸ್ತ
ಕ್ರಿಸ್ತನ
ಕ್ರಿಸ್ತ-ನಂ-ತಹ
ಕ್ರಿಸ್ತ-ನಂತೆ
ಕ್ರಿಸ್ತ-ನತ್ತ
ಕ್ರಿಸ್ತ-ನನ್ನು
ಕ್ರಿಸ್ತ-ನನ್ನೇ
ಕ್ರಿಸ್ತ-ನ-ಲ್ಲಿ-ರುವ
ಕ್ರಿಸ್ತ-ನಿಂ-ದಲೇ
ಕ್ರಿಸ್ತ-ನಿಗೂ
ಕ್ರಿಸ್ತ-ನಿ-ದ್ದಿ-ದ್ದರೆ
ಕ್ರಿಸ್ತ-ನಿನ್ನೂ
ಕ್ರಿಸ್ತ-ನಿ-ರುವ
ಕ್ರಿಸ್ತ-ನಿ-ಲ್ಲದ
ಕ್ರಿಸ್ತ-ನೆ-ಡೆಗೆ
ಕ್ರಿಸ್ತನೇ
ಕ್ರಿಸ್ತರು
ಕ್ರಿಸ್ಮಸ್
ಕ್ರಿಸ್ಮ-ಸ್ನಲ್ಲಿ
ಕ್ರೀಡಾಂ-ಗ-ಣ-ಗಳು
ಕ್ರೀಡಾಂ-ಗ-ಣ-ವೆಂದೇ
ಕ್ರೀಸ್ಟೀನ
ಕ್ರೂರ
ಕ್ರೈಸ್ಟ್
ಕ್ರೈಸ್ತ
ಕ್ರೈಸ್ತ-ಧರ್ಮ
ಕ್ರೈಸ್ತ-ಧ-ರ್ಮಕ್ಕೂ
ಕ್ರೈಸ್ತ-ಧ-ರ್ಮದ
ಕ್ರೈಸ್ತ-ಧ-ರ್ಮ-ದಲ್ಲಿ
ಕ್ರೈಸ್ತ-ಧ-ರ್ಮ-ದೊಂ-ದಿಗೆ
ಕ್ರೈಸ್ತ-ಧ-ರ್ಮ-ಪ್ರ-ಚಾ-ರ-ಕರು
ಕ್ರೈಸ್ತ-ಧ-ರ್ಮ-ವನ್ನು
ಕ್ರೈಸ್ತ-ಧ-ರ್ಮವು
ಕ್ರೈಸ್ತ-ಧ-ರ್ಮ-ವೊಂದೇ
ಕ್ರೈಸ್ತ-ಧ-ರ್ಮಾ-ಧಿ-ಕಾ-ರಿ-ಗಳ
ಕ್ರೈಸ್ತ-ಧ-ರ್ಮಾ-ಧಿ-ಕಾ-ರಿ-ಗಳು
ಕ್ರೈಸ್ತ-ಧ-ರ್ಮಾ-ಧಿ-ಕಾ-ರಿ-ಗ-ಳೆಲ್ಲ
ಕ್ರೈಸ್ತ-ಧ-ರ್ಮೀ-ಯ-ರಿಂದ
ಕ್ರೈಸ್ತ-ನಾ-ಗ-ಬೇ-ಕಿಲ್ಲ
ಕ್ರೈಸ್ತ-ನಾ-ದ-ವನು
ಕ್ರೈಸ್ತನೂ
ಕ್ರೈಸ್ತ-ನೆಂಬ
ಕ್ರೈಸ್ತ-ಪಂ-ಗ-ಡ-ಗ-ಳಿಗೆ
ಕ್ರೈಸ್ತ-ಪಂ-ಥ-ಗ-ಳಿಗೂ
ಕ್ರೈಸ್ತ-ಮತ
ಕ್ರೈಸ್ತ-ಮಿ-ಷ-ನ-ರಿ-ಗ-ಳಿಗೆ
ಕ್ರೈಸ್ತರ
ಕ್ರೈಸ್ತ-ರ-ದೆಂದು
ಕ್ರೈಸ್ತ-ರ-ನ್ನಾಗಿ
ಕ್ರೈಸ್ತ-ರ-ಲ್ಲ-ದ-ವರು
ಕ್ರೈಸ್ತ-ರಲ್ಲಿ
ಕ್ರೈಸ್ತ-ರಾ-ಗ-ಬೇಕು
ಕ್ರೈಸ್ತ-ರಾದ
ಕ್ರೈಸ್ತ-ರಿ-ಗಂತೂ
ಕ್ರೈಸ್ತ-ರಿ-ಗಾ-ಗಲಿ
ಕ್ರೈಸ್ತರು
ಕ್ರೈಸ್ತ-ರು-ಎ-ಲ್ಲರೂ
ಕ್ರೈಸ್ತರೂ
ಕ್ರೈಸ್ತ-ರೆಂದು
ಕ್ರೈಸ್ತ-ರೆ-ನ್ನಿ-ಸಿ-ಕೊಂ-ಡ-ವರೇ
ಕ್ರೈಸ್ತ-ರೆಲ್ಲ
ಕ್ರೈಸ್ತ-ರೆ-ಲ್ಲರ
ಕ್ರೈಸ್ತರೇ
ಕ್ರೈಸ್ತ-ರೊಂ-ದಿಗೆ
ಕ್ರೈಸ್ತ-ಳಾಗಿ
ಕ್ರೈಸ್ತ-ವಾ-ಗಲಿ
ಕ್ರೈಸ್ತೀ-ಕ-ರಿ-ಸು-ವುದೂ
ಕ್ರೈಸ್ತೇ-ತರ
ಕ್ಲಬ್
ಕ್ಲಬ್ಗ-ಳಿಗೆ
ಕ್ಲಬ್ಗೆ
ಕ್ಲಬ್ನ
ಕ್ಲಬ್ನಲ್ಲಿ
ಕ್ಲಬ್ಬಿನ
ಕ್ಲಬ್ಬಿ-ನಲ್ಲಿ
ಕ್ಲಬ್ಬಿ-ನ-ಲ್ಲಿ-ದ್ದದ್ದು
ಕ್ಲಬ್ಬು-ಗಳಲ್ಲಿ
ಕ್ಲಬ್ಬು-ಗಳು
ಕ್ಲಿಷ್ಟ
ಕ್ಲಿಷ್ಟ-ವಿ-ವಾ-ದಾ-ಸ್ಪದ
ಕ್ಲಿಷ್ಟವೂ
ಕ್ಲಿಷ್ಟ-ವೆಂದು
ಕ್ವೀನ್
ಕ್ಷಣ
ಕ್ಷಣ-ಕಾಲ
ಕ್ಷಣ-ಕಾ-ಲ-ವಾದ
ಕ್ಷಣ-ಕಾ-ಲವೂ
ಕ್ಷಣ-ಕ್ಷ-ಣಕ್ಕೂ
ಕ್ಷಣ-ಕ್ಷ-ಣವೂ
ಕ್ಷಣ-ಗಳನ್ನು
ಕ್ಷಣ-ಗಳಲ್ಲಿ
ಕ್ಷಣ-ಗ-ಳ-ವ-ರೆಗೆ
ಕ್ಷಣದ
ಕ್ಷಣ-ದ-ಲ್ಲಾ-ದರೂ
ಕ್ಷಣ-ದಲ್ಲಿ
ಕ್ಷಣ-ದಲ್ಲೇ
ಕ್ಷಣ-ದಿಂದ
ಕ್ಷಣ-ದಿಂ-ದಲೇ
ಕ್ಷಣ-ಮಾ-ತ್ರ-ದಲ್ಲಿ
ಕ್ಷಣವೂ
ಕ್ಷಣವೇ
ಕ್ಷಣಾ-ರ್ಧ-ದಲ್ಲಿ
ಕ್ಷಣಿಕ
ಕ್ಷಣಿ-ಕ-ವಾ-ದುದು
ಕ್ಷತ್ರಿಯ
ಕ್ಷತ್ರಿ-ಯ-ರಾದ
ಕ್ಷತ್ರಿ-ಯರು
ಕ್ಷಮಾ-ಶೀ-ಲ-ರಾ-ಗಿ-ರು-ತ್ತಿ-ದ್ದ-ರಾ-ದರೂ
ಕ್ಷಮಿ-ಸ-ಬೇಕು
ಕ್ಷಮಿಸಿ
ಕ್ಷಮಿ-ಸಿ-ಬಿಡಿ
ಕ್ಷಮಿಸು
ಕ್ಷಮಿ-ಸುತ್ತೀ
ಕ್ಷಮೆ
ಕ್ಷಮೆ-ಕೋ-ರಿ-ದರು
ಕ್ಷಮೆ-ಯಾ-ಚಿ-ಸಿ-ದರು
ಕ್ಷಾತ್ರ-ವೀ-ರ್ಯ-ಬ್ರ-ಹ್ಮ-ತೇಜ
ಕ್ಷಾಮ
ಕ್ಷಾಮ-ಡಾ-ಮ-ರ-ಗಳ
ಕ್ಷಾಮದ
ಕ್ಷಿತಿಜ
ಕ್ಷಿತಿ-ಜ-ದ-ಲ್ಲು-ದಿ-ಸುವ
ಕ್ಷೀಣ
ಕ್ಷೀರ-ಸಾ-ಗ-ರ-ವನ್ನು
ಕ್ಷುದ್ರ
ಕ್ಷುದ್ರ-ಪ್ರ-ಪಂ-ಚದ
ಕ್ಷುದ್ರ-ವಾದ
ಕ್ಷುಲ್ಲ-ಕ-ವಾ-ದದ್ದು
ಕ್ಷೇತ್ರ
ಕ್ಷೇತ್ರಕ್ಕೆ
ಕ್ಷೇತ್ರ-ಗಳ
ಕ್ಷೇತ್ರ-ಗಳನ್ನು
ಕ್ಷೇತ್ರ-ಗ-ಳಿಗೆ
ಕ್ಷೇತ್ರದ
ಕ್ಷೇತ್ರ-ದತ್ತ
ಕ್ಷೇತ್ರ-ದ-ಲ್ಲಾ-ಗು-ತ್ತಿದ್ದ
ಕ್ಷೇತ್ರ-ದಲ್ಲಿ
ಕ್ಷೇತ್ರ-ದಲ್ಲೂ
ಕ್ಷೇತ್ರ-ವನ್ನು
ಕ್ಷೇತ್ರವೂ
ಕ್ಷೇಮ
ಕ್ಷೇಮದ
ಕ್ಷೇಮ-ವಲ್ಲ
ಕ್ಷೇಮ-ವಾಗಿ
ಕ್ಷೇಮ-ವಾ-ಗಿ-ದ್ದಾರೆ
ಕ್ಷೇಮಾ-ಭ್ಯು-ದ-ಯವೂ
ಕ್ಷೋಭೆ-ಗೊ-ಳ-ಗಾ-ಗಿದ್ದ
ಕ್ಷೌರ-ಮಾ-ಡಿ-ಕೊಂಡ
ಕ್ಷೌರಿಕ
ಕ್ಷೌರಿ-ಕನ
ಕ್ಷೌರಿ-ಕರ
ಖಂಡದ
ಖಂಡವು
ಖಂಡಿತ
ಖಂಡಿ-ತ-ವಾಗಿ
ಖಂಡಿ-ತ-ವಾ-ಗಿಯೂ
ಖಂಡಿ-ಸದೆ
ಖಂಡಿಸಿ
ಖಂಡಿ-ಸಿದ
ಖಂಡಿ-ಸಿ-ದ-ರ-ಲ್ಲದೆ
ಖಂಡಿ-ಸಿ-ದರು
ಖಂಡಿ-ಸಿ-ದ್ದರು
ಖಂಡಿ-ಸಿ-ದ್ದುದೇ
ಖಂಡಿಸು
ಖಂಡಿ-ಸುತ್ತ
ಖಂಡಿ-ಸು-ತ್ತಿ-ದ್ದರು
ಖಗೋ-ಳ-ವಿ-ಜ್ಞಾ-ನ-ಗಳ
ಖಚಿತ
ಖಚಿ-ತ-ವಾ-ಗಿತ್ತು
ಖಚಿ-ತ-ವಾ-ಗಿದೆ
ಖಚಿ-ತ-ವಾ-ಗಿ-ದೆ-ಜ-ಗ-ತ್ತನ್ನೇ
ಖಚಿ-ತ-ವಾ-ಗಿ-ದ್ದು-ದ-ರಿಂದ
ಖಚಿ-ತ-ವಾ-ಗಿ-ರ-ಲಿಲ್ಲ
ಖಚಿ-ತ-ವಾ-ಗು-ತ್ತದೆ
ಖಚಿ-ತ-ವಾದ
ಖಚಿ-ತ-ವಾ-ದಂ-ತಾ-ಯಿತು
ಖಚಿ-ತ-ವಾ-ಯಿತು
ಖಚಿ-ತವೂ
ಖಜಾಂ-ಚಿ-ಗ-ಳಾಗಿ
ಖಡಾ-ಖಂ-ಡಿ-ತ-ವಾಗಿ
ಖಡ್ಗ
ಖಡ್ಗ-ಲೇ-ಖ-ನಿ-ಗಳ
ಖಡ್ಗದ
ಖಡ್ಗ-ದಂತೆ
ಖಡ್ಗವು
ಖನಿ-ಯಾದ
ಖಯಾ-ಲಿ-ಯಾ-ಗ-ದಂತೆ
ಖರೀ-ದಿ-ಸ-ಲಾ-ಗದ
ಖರೀ-ದಿಸಿ
ಖರ್ಚನ್ನು
ಖರ್ಚನ್ನೂ
ಖರ್ಚ-ನ್ನೆಲ್ಲ
ಖರ್ಚಾ-ಗಿ-ದ್ದುವು
ಖರ್ಚಾ-ಗಿ-ಹೋ-ಯಿತೋ
ಖರ್ಚಾಗು
ಖರ್ಚಿ-ಗಾಗಿ
ಖರ್ಚಿನ
ಖರ್ಚಿ-ನಲ್ಲೇ
ಖರ್ಚು
ಖರ್ಚು-ಗಳನ್ನು
ಖರ್ಚು-ಗಳಲ್ಲಿ
ಖರ್ಚು-ಮಾ-ಡ-ಬ-ಹುದೋ
ಖರ್ಚು-ಮಾ-ಡಲು
ಖರ್ಚು-ಮಾ-ಡಿದ
ಖರ್ಚು-ಮಾ-ಡು-ತ್ತಿ-ದ್ದೀಯೆ
ಖರ್ಚು-ಮಾ-ಡು-ತ್ತೀ-ಯೆ-ಆ-ದರೆ
ಖರ್ಚು-ಮಾ-ಡುವ
ಖರ್ಚು-ವೆ-ಚ್ಚ-ಗಳು
ಖಲು
ಖಲ್ವಿದಂ
ಖಾಂಡ್ವಾ
ಖಾಂಡ್ವಾಗೆ
ಖಾಂಡ್ವಾದ
ಖಾಂಡ್ವಾ-ದಲ್ಲಿ
ಖಾಂಡ್ವಾ-ದ-ಲ್ಲಿದ್ದ
ಖಾಂಡ್ವಾ-ದಲ್ಲೂ
ಖಾಂಡ್ವಾ-ದಿಂದ
ಖಾತ್ರಿ
ಖಾನೆ
ಖಾನೆಯ
ಖಾನ್
ಖಾಯಂ
ಖಾರ
ಖಾರದ
ಖಾರ-ವಾಗಿ
ಖಾರ-ವಾ-ಗಿಯೇ
ಖಾರ-ವಾದ
ಖಾರ-ವೆಂದರೆ
ಖಾಲಿ
ಖಾಲಿ-ಯಾದ
ಖಾಸಗಿ
ಖಾಸ-ಗಿ-ಯಾಗಿ
ಖಾಸಗೀ
ಖಾಸ್ಬಾಗ್
ಖುದ್ದಾಗಿ
ಖುರ್ಷಿದ್
ಖುಷಿ-ಯಾಗಿ
ಖುಷಿ-ಯಾ-ಗಿ-ರಲಿ
ಖುಷಿ-ಯಾ-ಗಿ-ರುತ್ತ
ಖುಷಿ-ಯಾ-ಯಿತು
ಖೇತ್ರಿ
ಖೇತ್ರಿಗೆ
ಖೇತ್ರಿಯ
ಖೇತ್ರಿ-ಯನ್ನು
ಖೇತ್ರಿ-ಯಲ್ಲಿ
ಖೇತ್ರಿ-ಯ-ಲ್ಲಿ-ದ್ದಾಗ
ಖೇತ್ರಿ-ಯ-ಲ್ಲಿದ್ದು
ಖೇತ್ರಿ-ಯಿಂದ
ಖೇತ್ರಿಯೇ
ಖೇದ-ವಾ-ಗಿದೆ
ಖೇದವೂ
ಖ್ಯಾತ
ಖ್ಯಾತ-ರಾ-ಗಿ-ದ್ದರು
ಖ್ಯಾತಿ-ವೆ-ತ್ತ-ವರು
ಗಂಗಾ
ಗಂಗಾ-ತೀ-ರ-ದ-ಲ್ಲೊಂದು
ಗಂಗಾ-ಧರ
ಗಂಗಾ-ನಂದಿ
ಗಂಗೂಲಿ
ಗಂಗೆ-ಯಲಿ
ಗಂಜಿ-ಯನ್ನು
ಗಂಜಿ-ಯನ್ನೋ
ಗಂಟನ್ನು
ಗಂಟನ್ನೂ
ಗಂಟ-ಲಲ್ಲಿ
ಗಂಟಲು
ಗಂಟ-ಲೊ-ಳಗೆ
ಗಂಟು
ಗಂಟು-ಕ-ಟ್ಟಿದ್ದ
ಗಂಟು-ಬಿ-ದ್ದಿದ್ದ
ಗಂಟೆ
ಗಂಟೆ-ಗ-ಟ್ಟಲೆ
ಗಂಟೆ-ಗಳ
ಗಂಟೆ-ಗಳಲ್ಲಿ
ಗಂಟೆಗೂ
ಗಂಟೆಗೆ
ಗಂಟೆಯ
ಗಂಟೆ-ಯನ್ನು
ಗಂಟೆ-ಯ-ವ-ರೆಗೂ
ಗಂಟೆ-ಯ-ವ-ರೆಗೆ
ಗಂಟೆ-ಯಾ-ಗುವ
ಗಂಟೆ-ಯಾದ
ಗಂಟೆ-ಯಿಂದ
ಗಂಟೆ-ಯಿಂ-ದಲೇ
ಗಂಡಂ-ದಿ-ರಿ-ರುವ
ಗಂಡಂ-ದಿರು
ಗಂಡಂ-ದಿ-ರೆಲ್ಲ
ಗಂಡ-ನನ್ನು
ಗಂಡ-ಸರೂ
ಗಂಡ-ಸ್ಥಳ
ಗಂಡಾಂ-ತರ
ಗಂಡಾಂ-ತ-ರಕ್ಕೆ
ಗಂಡಿ-ದ್ದರು
ಗಂಡು
ಗಂಡು-ಮಗು
ಗಂಡು-ಮ-ಗು-ವಾಗು
ಗಂಡು-ಮ-ಗು-ವಾ-ಗು-ವಂತೆ
ಗಂಡು-ಸಂ-ತಾ-ನ-ವಾ-ಯಿತು
ಗಂತೂ
ಗಂಧ-ದಿಂದ
ಗಂಧ-ವನ್ನು
ಗಂಧವೇ
ಗಂಪು
ಗಂಭೀರ
ಗಂಭೀ-ರ-ಭಾವ
ಗಂಭೀ-ರ-ಭಾ-ವ-ದ-ಲ್ಲಿ-ದ್ದರೆ
ಗಂಭೀ-ರ-ಭಾ-ವ-ವನ್ನು
ಗಂಭೀ-ರ-ವಾಗಿ
ಗಂಭೀ-ರ-ವಾ-ಗಿಯೇ
ಗಂಭೀ-ರ-ವಾದ
ಗಂಭೀ-ರ-ವಾ-ದವು
ಗಂಭೀ-ರವೂ
ಗಗ-ನ-ದಷ್ಟು
ಗಗ-ನವೇ
ಗಟ್ಟಲೆ
ಗಟ್ಟಿ
ಗಟ್ಟಿ-ಗರೇ
ಗಟ್ಟಿ-ತ-ನಕ್ಕೆ
ಗಟ್ಟಿ-ಮು-ಟ್ಟಾದ
ಗಟ್ಟಿ-ಯಾಗಿ
ಗಟ್ಟಿ-ಯಾ-ಗಿಯೇ
ಗಟ್ಟಿ-ಯಾ-ಗಿ-ರು-ತ್ತದೆ
ಗಟ್ಟಿ-ಯಾದ
ಗಟ್ಟಿ-ವ್ಯ-ಕ್ತಿ-ಯನ್ನು
ಗಡ-ಚಿ-ಕ್ಕಿತು
ಗಡ-ಚಿ-ಕ್ಕುವ
ಗಡಸು
ಗಡಿ-ಬಿಡಿ
ಗಡಿ-ಬಿ-ಡಿ-ಯಲ್ಲಿ
ಗಡಿ-ಬಿ-ಡಿ-ಯ-ಲ್ಲಿ-ದ್ದಾರೆ
ಗಡಿಯ
ಗಡಿ-ಯ-ಲ್ಲಿ-ರುವ
ಗಡಿ-ಯಾ-ರಕ್ಕೆ
ಗಡುವು
ಗಡ್ಡ
ಗಢ-ಗಢ
ಗಣ-ನೀಯ
ಗಣ-ನೆಯೇ
ಗಣಿ-ತದ
ಗಣಿ-ಯನ್ನು
ಗಣಿಯೇ
ಗಣಿ-ಸದೆ
ಗಣ್ಯ
ಗಣ್ಯ-ಕ್ರೈ-ಸ್ತರೂ
ಗಣ್ಯ-ನಾ-ಗ-ರಿ-ಕರು
ಗಣ್ಯರ
ಗಣ್ಯ-ರ-ನ್ನು-ದ್ದೇ-ಶಿಸಿ
ಗಣ್ಯ-ರಲ್ಲಿ
ಗಣ್ಯರು
ಗಣ್ಯ-ರೆಂ-ದರೆ
ಗಣ್ಯ-ವ-ಕ್ತಿ-ಗ-ಳೊಂ-ದಿಗೆ
ಗಣ್ಯ-ವಾ-ಗಿ-ರಲಿ
ಗಣ್ಯ-ವ್ಯಕ್ತಿ
ಗಣ್ಯ-ವ್ಯ-ಕ್ತಿ-ಗಳ
ಗಣ್ಯ-ವ್ಯ-ಕ್ತಿ-ಗಳನ್ನು
ಗಣ್ಯ-ವ್ಯ-ಕ್ತಿ-ಗಳನ್ನೂ
ಗಣ್ಯ-ವ್ಯ-ಕ್ತಿ-ಗಳಿಂದ
ಗಣ್ಯ-ವ್ಯ-ಕ್ತಿ-ಗಳು
ಗಣ್ಯ-ವ್ಯ-ಕ್ತಿ-ಯನ್ನೂ
ಗಣ್ಯ-ವ್ಯ-ಕ್ತಿ-ಯಾದ
ಗತ-ವೈ-ಭ-ವ-ವನ್ನು
ಗತಿ
ಗತಿ-ಯಲ್ಲಿ
ಗತಿ-ಯಿಂದ
ಗತಿ-ಯಿ-ಲ್ಲದೆ
ಗತಿಯೇ
ಗತಿ-ಯೇನು
ಗತಿ-ಯೇನೂ
ಗತ್ಯಂ-ತರ
ಗತ್ಯಂ-ತ-ರ-ವಿ-ಲ್ಲದೆ
ಗದ-ರಿ-ಸಿದ
ಗದ-ರಿ-ಸಿ-ದರು
ಗದ್ಗದ
ಗದ್ಗ-ದ-ವಾಗಿ
ಗದ್ಗ-ದ-ವಾ-ಯಿತು
ಗದ್ಗ-ದಿ-ತ-ವಾಗಿ
ಗದ್ದ-ಲ-ಗಳನ್ನೆಲ್ಲ
ಗನೊ
ಗಮನ
ಗಮ-ನ-ಕೊ-ಡದೆ
ಗಮ-ನ-ಕೊ-ಡ-ಬ-ಲ್ಲ-ವ-ರಾ-ಗಿ-ದ್ದರು
ಗಮ-ನ-ಕೊ-ಡ-ಬೇ-ಕೆಂದು
ಗಮ-ನ-ಕೊ-ಡು-ವುದನ್ನು
ಗಮ-ನ-ಕೊ-ಡು-ವು-ದಿಲ್ಲ
ಗಮ-ನಕ್ಕೆ
ಗಮ-ನಕ್ಕೇ
ಗಮ-ನ-ಗೊಟ್ಟು
ಗಮ-ನ-ವನ್ನು
ಗಮ-ನ-ವನ್ನೂ
ಗಮ-ನ-ವಿ-ಟ್ಟಿ-ರ-ಬೇ-ಕಲ್ಲ
ಗಮ-ನ-ವಿಟ್ಟು
ಗಮ-ನ-ವಿತ್ತೇ
ಗಮ-ನ-ವೆಲ್ಲ
ಗಮ-ನವೇ
ಗಮ-ನ-ಹ-ರಿ-ಸ-ಬೇ-ಕಾದ
ಗಮ-ನ-ಹ-ರಿ-ಸಿದ್ದು
ಗಮ-ನಾರ್ಹ
ಗಮ-ನಾ-ರ್ಹ-ವಾ-ದುದು
ಗಮ-ನಿ-ಸ-ದಿ-ರು-ವಂ-ತಿಲ್ಲ
ಗಮ-ನಿ-ಸ-ಬಹು
ಗಮ-ನಿ-ಸ-ಬ-ಹು-ದಾ-ಗಿದೆ
ಗಮ-ನಿ-ಸ-ಬ-ಹುದು
ಗಮ-ನಿ-ಸ-ಬೇ-ಕಾದ
ಗಮ-ನಿ-ಸ-ಬೇಕು
ಗಮ-ನಿ-ಸಲು
ಗಮ-ನಿ-ಸಲೇ
ಗಮ-ನಿಸಿ
ಗಮ-ನಿ-ಸಿದ
ಗಮ-ನಿ-ಸಿ-ದರು
ಗಮ-ನಿ-ಸಿ-ದಳು
ಗಮ-ನಿ-ಸಿ-ದ-ವ-ರೊ-ಬ್ಬರು
ಗಮ-ನಿ-ಸಿ-ದಾಗ
ಗಮ-ನಿ-ಸಿ-ದುವು
ಗಮ-ನಿ-ಸಿದ್ದ
ಗಮ-ನಿ-ಸಿ-ದ್ದಂತೆ
ಗಮ-ನಿ-ಸಿ-ದ್ದರು
ಗಮ-ನಿ-ಸಿ-ದ್ದಾರೆ
ಗಮ-ನಿ-ಸಿಯೇ
ಗಮ-ನಿಸು
ಗಮ-ನಿ-ಸುತ್ತ
ಗಮ-ನಿ-ಸು-ತ್ತಲೇ
ಗಮ-ನಿ-ಸುತ್ತಿ
ಗಮ-ನಿ-ಸು-ತ್ತಿ-ದ್ದರು
ಗಮ-ನಿ-ಸು-ತ್ತಿ-ದ್ದಳು
ಗಮ-ನಿ-ಸು-ತ್ತಿ-ದ್ದಾರೆ
ಗಮ-ನಿ-ಸು-ವು-ದರ
ಗಮ-ನಿ-ಸು-ವುದೇ
ಗಮ-ನೀ-ಯ-ವಾ-ದದ್ದೇ
ಗಮಿ-ನಿ-ಸ-ಬೇಕು
ಗಮಿ-ಸಿದೆ
ಗಮ್ಯಃ
ಗರ-ಡಿ-ಯಲ್ಲಿ
ಗರ-ಬ-ಡಿ-ದಂತೆ
ಗರಿಮೆ
ಗರಿ-ಮೆ-ಯನ್ನು
ಗರಿ-ಮೆ-ಯಲ್ಲಿ
ಗರು
ಗರುಡ
ಗರು-ಭಾ-ಯಿ-ಗ-ಳಿಗೆ
ಗರ್ಜಿ-ಸಿತು
ಗರ್ಜಿ-ಸಿದ
ಗರ್ನ್ಸೇ
ಗರ್ನ್ಸೇ-ಯ-ವ-ರಿಗೂ
ಗರ್ವಿ-ಷ್ಠ-ರಾದ
ಗರ್ವಿ-ಸು-ತ್ತಿ-ರು-ವ-ವರು
ಗಲ-ಭೆ-ಗಳಿಂದ
ಗಲ-ಭೆ-ಯೆ-ಬ್ಬಿ-ಸುವ
ಗಲಾ-ಟೆ-ಗಳನ್ನು
ಗಲಾ-ಟೆಯ
ಗಲಿ
ಗಲಿ-ಬಿ-ಲಿ-ಗೊಂಡ
ಗಲಿ-ಬಿ-ಲಿ-ಗೊಂಡು
ಗಲು
ಗಲೂ
ಗಲೇ
ಗಲ್ಲಿ-ಯೊಂ-ದ-ರೊ-ಳಗೆ
ಗಳ
ಗಳಂತೆ
ಗಳಂದು
ಗಳ-ಡ-ಗಿವೆ
ಗಳ-ನ್ನಾ-ಗಿ-ಸುತ್ತ
ಗಳನ್ನು
ಗಳನ್ನೂ
ಗಳ-ನ್ನೆಲ್ಲ
ಗಳಲ್ಲ
ಗಳ-ಲ್ಲವೆ
ಗಳಲ್ಲಿ
ಗಳ-ಲ್ಲಿನ
ಗಳಲ್ಲೂ
ಗಳ-ವ-ರೆಗೆ
ಗಳಷ್ಟು
ಗಳಾ-ಗಿದ್ದ
ಗಳಾ-ಗಿ-ದ್ದೀರಿ
ಗಳಾ-ಗಿ-ದ್ದುವು
ಗಳಾ-ಗು-ತ್ತಿ-ದ್ದುವು
ಗಳಾ-ಗು-ವಂತೆ
ಗಳಾದ
ಗಳಿಂದ
ಗಳಿಂ-ದಲೂ
ಗಳಿಂ-ದಲೇ
ಗಳಿಂ-ದಾಗಿ
ಗಳಿ-ಕೆ-ಗಿಳಿ-ಯ-ಬೇ-ಕಾ-ಯಿ-ತಲ್ಲ
ಗಳಿ-ಗಿ-ರುವ
ಗಳಿಗೂ
ಗಳಿಗೆ
ಗಳಿ-ಗೆ-ಯಲ್ಲಿ
ಗಳಿ-ಗೆಯೂ
ಗಳಿ-ತ-ವಾ-ಗಿತ್ತು
ಗಳಿ-ತ-ವಾ-ಗಿ-ಸಿತು
ಗಳಿ-ದ್ದರೂ
ಗಳಿ-ದ್ದುವು
ಗಳಿಸ
ಗಳಿ-ಸ-ಬೇ-ಕೆಂ-ಬುದು
ಗಳಿ-ಸ-ಲಾರ
ಗಳಿ-ಸಲು
ಗಳಿಸಿ
ಗಳಿ-ಸಿ-ಕೊಂ-ಡದ್ದು
ಗಳಿ-ಸಿ-ಕೊಂ-ಡಿದ್ದ
ಗಳಿ-ಸಿ-ಕೊಂ-ಡು-ಬಿಟ್ಟ
ಗಳಿ-ಸಿ-ಕೊಟ್ಟ
ಗಳಿ-ಸಿ-ಕೊ-ಟ್ಟಿತು
ಗಳಿ-ಸಿ-ಕೊ-ಟ್ಟುವು
ಗಳಿ-ಸಿತು
ಗಳಿ-ಸಿದ
ಗಳಿ-ಸಿ-ದಂ-ದಿ-ನಿಂ-ದಲೇ
ಗಳಿ-ಸಿ-ದ-ಮೇಲೆ
ಗಳಿ-ಸಿ-ದ-ವ-ರಲ್ಲ
ಗಳಿ-ಸಿ-ದ-ವ-ರೆಂದು
ಗಳಿ-ಸಿ-ದುವು
ಗಳಿ-ಸಿದ್ದ
ಗಳಿ-ಸಿ-ದ್ದ-ರಿಂದ
ಗಳಿ-ಸಿ-ದ್ದರು
ಗಳಿ-ಸಿ-ದ್ದಳು
ಗಳಿ-ಸಿ-ದ್ದ-ವನು
ಗಳಿ-ಸಿ-ದ್ದ-ವರು
ಗಳಿ-ಸಿ-ದ್ದಾನೆ
ಗಳಿ-ಸಿ-ದ್ದಾರೆ
ಗಳಿ-ಸಿ-ದ್ದು-ದ-ರಿಂದ
ಗಳಿ-ಸಿ-ದ್ದೇನೆ
ಗಳಿ-ಸು-ತ್ತಿತ್ತು
ಗಳಿ-ಸು-ತ್ತಿ-ದ್ದರೂ
ಗಳಿ-ಸುವ
ಗಳಿ-ಸು-ವಂ-ತಾ-ಗಲಿ
ಗಳು
ಗಳು-ಮಂ-ತ್ರಿ-ಗ-ಳು-ಹೀಗೆ
ಗಳು-ಅ-ವನು
ಗಳು-ಇ-ವೆ-ಲ್ಲವೂ
ಗಳು-ಕೆ-ಲ-ವೊಮ್ಮೆ
ಗಳೂ
ಗಳೆಂದು
ಗಳೆಲ್ಲ
ಗಳೆ-ಲ್ಲ-ದರ
ಗಳೆ-ಲ್ಲವೂ
ಗಳೇ
ಗಳೊಂ-ದಿಗೆ
ಗಳೊ-ಬ್ಬರು
ಗವ-ರ್ನರ್
ಗಹ-ನ-ವಾದ
ಗಾಂಗ್
ಗಾಂಧಿ-ಯ-ವರು
ಗಾಂಪ-ನಿ-ರ-ಬೇ-ಕೆಂದು
ಗಾಂಭೀರ್ಯ
ಗಾಂಭೀ-ರ್ಯ-ದಿಂದ
ಗಾಂಭೀ-ರ್ಯವೂ
ಗಾಗಿ
ಗಾಜು
ಗಾಜೇ
ಗಾಡಿ
ಗಾಡಿ-ಗ-ಟ್ಟಲೆ
ಗಾಡಿಯ
ಗಾಡಿ-ಯಲ್ಲಿ
ಗಾಡಿ-ಯಿಂದ
ಗಾಢ
ಗಾಢ-ಧ್ಯಾ-ನ-ದಲ್ಲಿ
ಗಾಢ-ವಾಗಿ
ಗಾಢ-ವಾ-ಗಿತ್ತು
ಗಾಢ-ವಾ-ಗಿ-ತ್ತೆಂ-ದರೆ
ಗಾಢ-ಸ-ಮಾ-ಧಿ-ಯಲ್ಲಿ
ಗಾತ್ರದ
ಗಾತ್ರ-ದಲ್ಲಿ
ಗಾಥಿಕ್
ಗಾದರೂ
ಗಾದೆ
ಗಾದೆ-ಯಂತೆ
ಗಾನ-ವನ್ನು
ಗಾಬರಿ
ಗಾಯ-ಕನ
ಗಾಯ-ಕ-ರಿಂದ
ಗಾಯ-ಕರು
ಗಾಯ-ಕ-ವಾಡ
ಗಾಯಕಿ
ಗಾಯ-ಕಿಗೆ
ಗಾಯ-ಕಿ-ಯಾ-ಗಿದ್ದ
ಗಾಯ-ಕಿ-ಯೊ-ಬ್ಬಳ
ಗಾಯನ
ಗಾಯ-ನ-ಗಳನ್ನು
ಗಾಯ-ನ-ವನ್ನು
ಗಾರನ
ಗಾರ-ನಾಗಿ
ಗಾರನೂ
ಗಾರ-ರ-ಲ್ಲೊ-ಬ್ಬ-ನಾ-ಗಿದ್ದ
ಗಾರ-ರು-ಹೀಗೆ
ಗಾರಿಕೆ
ಗಾರ್ಗಿ-ಯರು
ಗಾರ್ಡನ್
ಗಾರ್ಡ-ನ್ಸ್
ಗಾರ್ನೆ-ರ್ಗ್ರಾಟ್
ಗಾಲ
ಗಾಲದ
ಗಾಲ್ಪ್
ಗಾಲ್ಫ್
ಗಾಳಿ
ಗಾಳಿ-ಮಾ-ತು-ಗಳು
ಗಾಳಿಯ
ಗಾಳಿ-ಯಲ್ಲಿ
ಗಾಳಿಯೂ
ಗಿಂತ
ಗಿಂತಲೂ
ಗಿಜಿ-ಗು-ಟ್ಟು-ತ್ತಿತ್ತು
ಗಿಟ್ಟಿ-ಸಿ-ಕೊಂಡ
ಗಿಟ್ಟಿ-ಸಿ-ಕೊಂಡೆ
ಗಿಡ-ಮ-ರ-ಗಳಿಂದ
ಗಿಡ-ವಾಗಿ
ಗಿಡಿದ
ಗಿತ್ತು
ಗಿದೆ
ಗಿದ್ದ
ಗಿದ್ದು
ಗಿಬ್ಬ-ನ್ಸ್
ಗಿಯೇ
ಗಿರಿಂ
ಗಿರಿ-ಧಾಮ
ಗಿರಿ-ಧಾ-ಮದ
ಗಿರಿ-ನಾರ್
ಗಿರಿ-ನಾ-ರ್ನಲ್ಲಿ
ಗಿರಿ-ನಾ-ರ್ನಿಂದ
ಗಿರಿ-ಶಿ-ಖ-ರ-ಗಳು
ಗಿರೀ-ಶ್ಚಂ-ದ್ರ-ನೊಂ-ದಿಗೆ
ಗಿರು-ವು-ದ-ರಿಂದ
ಗಿಲ್ಲ
ಗಿಳಿ
ಗಿಳಿ-ಯಂತೆ
ಗೀಚ-ಲಾ-ರಂ-ಭಿ-ಸಿದ
ಗೀತಾ
ಗೀತೆ
ಗೀತೆ-ಗಳನ್ನೂ
ಗೀತೆಯ
ಗೀತೆ-ಯನ್ನು
ಗೀತೆಯೂ
ಗೀರಣ
ಗೀಳನ್ನು
ಗುಂಡು-ಗಳನ್ನು
ಗುಂಡು-ಗಳು
ಗುಂಪನ್ನು
ಗುಂಪನ್ನೇ
ಗುಂಪಾಗಿ
ಗುಂಪಿ
ಗುಂಪಿ-ನಲ್ಲಿ
ಗುಂಪಿ-ನ-ವನೇ
ಗುಂಪು
ಗುಂಪು-ಗಟ್ಟಿ
ಗುಂಪು-ಗಳನ್ನು
ಗುಂಪು-ಗಾ-ರಿಕೆ
ಗುಂಪು-ಗಾ-ರಿ-ಕೆ-ಯನ್ನು
ಗುಂಪು-ಗುಂ-ಪಾಗಿ
ಗುಂಪೊಂ-ದನ್ನು
ಗುಂಪೊಂ-ದಿದೆ
ಗುಜ-ರಾ-ತಿನ
ಗುಜ-ರಾ-ತಿ-ನಲ್ಲಿ
ಗುಟುಕು
ಗುಟ್ಟನ್ನು
ಗುಟ್ಟಿನ
ಗುಡಿ-ಸ-ಲಿ-ನಲ್ಲಿ
ಗುಡಿ-ಸಲು
ಗುಡಿ-ಸ-ಲು-ಗಳಲ್ಲಿ
ಗುಡಿ-ಸ-ಲು-ಗ-ಳಿಗೆ
ಗುಡು-ಗಿ-ದರು
ಗುಡು-ಗಿ-ದ-ರು-ನಿ-ಮ್ಮಲ್ಲಿ
ಗುಡುಗು
ಗುಡು-ಗು-ತ್ತಿ-ದ್ದರು
ಗುಡು-ಗುವ
ಗುಡು-ಗು-ವಾಗ
ಗುಡ್
ಗುಡ್-ಇ-ಯರ್
ಗುಡ್ಡ-ಗ-ಳಾ-ಗ-ತೊ-ಡ-ಗಿ-ದುವು
ಗುಡ್ಡ-ಗಾಡು
ಗುಡ್ಡದ
ಗುಡ್ವಿನ್
ಗುಡ್ವಿ-ನ್ನನ
ಗುಡ್ವಿ-ನ್ನ-ನಿಗೆ
ಗುಡ್ವಿ-ನ್ನ-ನಿಗೇ
ಗುಡ್ವಿ-ನ್ನನೂ
ಗುಡ್ವಿ-ನ್ನ-ನೊಂ-ದಿಗೆ
ಗುಡ್ವಿ-ನ್ನನ್ನು
ಗುಡ್ಸ್ವ-್ಯಾ-ಗನ್
ಗುಣ
ಗುಣ-ರೂ-ಪ-ಗಳನ್ನು
ಗುಣ-ಗಳನ್ನು
ಗುಣ-ಗ-ಳಿವೆ
ಗುಣ-ಗಳು
ಗುಣ-ಗಾನ
ಗುಣ-ಟ್ಟದ
ಗುಣ-ಪ-ಡಿ-ಸ-ಲಾ-ರವು
ಗುಣ-ಪ-ಡಿ-ಸುವ
ಗುಣ-ಮ-ಟ್ಟಕ್ಕೆ
ಗುಣ-ಮು-ಟ್ಟಕ್ಕೆ
ಗುಣ-ರೂ-ಪ-ಗ-ಳನ್ನೋ
ಗುಣ-ವಾ-ಗಲಿ
ಗುಣ-ವಾ-ಚ-ಕ-ಗಳನ್ನು
ಗುಣ-ವಾ-ಚ-ಕ-ಗಳಿಂದ
ಗುಣ-ವಾ-ಯಿತು
ಗುಣ-ವೆಂದರೆ
ಗುಣಾ-ವ-ಗು-ಣ-ಗಳನ್ನು
ಗುತ್ತ
ಗುದ್ದು-ಗಳನ್ನೂ
ಗುಪ್ತ
ಗುಪ್ತ-ವಾಗಿ
ಗುಪ್ತ-ವಾ-ಗಿದ್ದ
ಗುಬ್ಬ-ಕ್ಕನ
ಗುಮಾನಿ
ಗುಮಾ-ಸ್ತೆ-ಗಿರಿ
ಗುಮ್ಮಿ
ಗುರ-ಭಾ-ಯಿ-ಗಳ
ಗುರ-ಭಾ-ಯಿ-ಗ-ಳಿಗೆ
ಗುರ-ವಾ-ದ-ವನು
ಗುರಿ
ಗುರಿ-ಪ-ಡಿ-ಸ-ಬಲ್ಲ
ಗುರಿ-ಮುಕ್ತಿ
ಗುರಿ-ಮು-ಟ್ಟಿತು
ಗುರಿ-ಯತ್ತ
ಗುರಿ-ಯನ್ನು
ಗುರಿ-ಯನ್ನೇ
ಗುರಿ-ಯಾ-ಗ-ಬ-ಹು-ದೆಂಬ
ಗುರಿ-ಯಾ-ಗ-ಬೇ-ಕಾ-ಗಿತ್ತು
ಗುರಿ-ಯಾ-ಗಲು
ಗುರಿ-ಯಾಗಿ
ಗುರಿ-ಯಾ-ಗಿ-ಬಿ-ಡು-ತ್ತಾನೆ
ಗುರಿ-ಯಾ-ಗು-ತ್ತಾರೆ
ಗುರಿ-ಯಾ-ದ-ದ್ದನ್ನು
ಗುರಿ-ಯುತ್ತ
ಗುರಿ-ಯೆಂದು
ಗುರಿ-ಯೆಂ-ಬು-ದನ್ನು
ಗುರಿ-ಯೆ-ಡೆಗೆ
ಗುರಿ-ಸೇ-ರು-ವ-ವ-ರೆಗೂ
ಗುರು
ಗುರು-ಶಿ-ಷ್ಯರ
ಗುರು-ಕುಲ
ಗುರು-ಗ-ಳಂ-ಥ-ವ-ರೊ-ಬ್ಬರು
ಗುರು-ಗ-ಳಾ-ಗ-ಬ-ಹು-ದಾದ
ಗುರು-ಗ-ಳಾದ
ಗುರು-ಗಳಿಂದ
ಗುರು-ಗಳು
ಗುರು-ಗಳೂ
ಗುರು-ಗ-ಳೊ-ಬ್ಬ-ರನ್ನು
ಗುರು-ಚ-ರಣ
ಗುರು-ತನ್ನು
ಗುರು-ತ-ನ್ನುಂ-ಟು-ಮಾ-ಡುವ
ಗುರು-ತರ
ಗುರು-ತಾದ
ಗುರುತಿ
ಗುರು-ತಿ-ಸ-ಬ-ಹು-ದಾ-ಗಿತ್ತು
ಗುರು-ತಿ-ಸ-ಬ-ಹುದು
ಗುರು-ತಿ-ಸಲು
ಗುರು-ತಿಸಿ
ಗುರು-ತಿ-ಸಿದ
ಗುರು-ತಿ-ಸಿ-ದರು
ಗುರು-ತಿ-ಸಿ-ದಾಗ
ಗುರು-ತಿ-ಸಿ-ದುವು
ಗುರು-ತಿ-ಸಿದೆ
ಗುರು-ತಿ-ಸಿ-ದ್ದರು
ಗುರು-ತಿ-ಸಿ-ದ್ದು-ದನ್ನು
ಗುರು-ತಿ-ಸಿದ್ದೆ
ಗುರು-ತಿ-ಸಿ-ದ್ದೇವೆ
ಗುರು-ತಿ-ಸಿ-ಬಿಟ್ಟೆ
ಗುರು-ತಿ-ಸುತ್ತಿ
ಗುರು-ತಿ-ಸುವ
ಗುರು-ತಿ-ಸು-ವು-ದ-ರಲ್ಲಿ
ಗುರು-ತಿ-ಸು-ವುದು
ಗುರುತು
ಗುರು-ತು-ಗಳನ್ನು
ಗುರು-ತು-ಮಾ-ಡಿ-ಟ್ಟು-ಕೊಂ-ಡಳು
ಗುರು-ತು-ಹಿ-ಡಿ-ಯಲು
ಗುರು-ತ್ವಾ-ಕ-ರ್ಷ-ಣೆ-ಯನ್ನು
ಗುರು-ತ್ವಾ-ಕ-ರ್ಷ-ಣೆ-ಯಿಂದ
ಗುರು-ತ್ವಾ-ಕ-ರ್ಷ-ಣೆಯು
ಗುರು-ದೇವ
ಗುರು-ದೇ-ವನ
ಗುರು-ದೇ-ವ-ನನ್ನು
ಗುರು-ದೇ-ವ-ನನ್ನೂ
ಗುರು-ದೇ-ವ-ನಾದ
ಗುರು-ದೇ-ವನು
ಗುರು-ದೇ-ವನೇ
ಗುರು-ದೇ-ವ-ನೊ-ಬ್ಬ-ನನ್ನೇ
ಗುರು-ದೇ-ವ-ನ್ನೊ-ಬ್ಬ-ನನ್ನೇ
ಗುರು-ದೇ-ವರ
ಗುರು-ದೇ-ವ-ರಿಗೆ
ಗುರು-ದೇ-ವರು
ಗುರು-ನಾ-ನಕ್
ಗುರು-ಬಾ-ಯಿ-ಗಳನ್ನೂ
ಗುರು-ಭಕ್ತಿ
ಗುರು-ಭ-ಕ್ತಿ-ಯೆಂ-ದ-ರೇ-ನೆಂ-ಬು-ದರ
ಗುರು-ಭಾಯಿ
ಗುರು-ಭಾ-ಯಿ-ಗಳ
ಗುರು-ಭಾ-ಯಿ-ಗಳನ್ನು
ಗುರು-ಭಾ-ಯಿ-ಗಳಲ್ಲಿ
ಗುರು-ಭಾ-ಯಿ-ಗ-ಳಾದ
ಗುರು-ಭಾ-ಯಿ-ಗಳಿ
ಗುರು-ಭಾ-ಯಿ-ಗಳಿಂದ
ಗುರು-ಭಾ-ಯಿ-ಗ-ಳಿಗೂ
ಗುರು-ಭಾ-ಯಿ-ಗ-ಳಿಗೆ
ಗುರು-ಭಾ-ಯಿ-ಗ-ಳಿ-ಬ್ಬರೂ
ಗುರು-ಭಾ-ಯಿ-ಗಳು
ಗುರು-ಭಾ-ಯಿ-ಗ-ಳೊಂ-ದಿಗೂ
ಗುರು-ಭಾ-ಯಿ-ಗ-ಳೊಂ-ದಿಗೆ
ಗುರು-ಭಾ-ಯಿ-ಗ-ಳೊ-ಬ್ಬ-ರನ್ನು
ಗುರು-ಭಾ-ಯಿಗೆ
ಗುರು-ಭಾ-ಯಿಯ
ಗುರು-ಭಾ-ಯಿ-ಯನ್ನು
ಗುರು-ಭಾ-ಯಿ-ಯಾದ
ಗುರು-ಭಾ-ಯಿಯೇ
ಗುರು-ಭಾ-ಯಿ-ಯೊಂ-ದಿಗೆ
ಗುರು-ಮ-ಹಾ-ರಾ-ಜರ
ಗುರು-ಮ-ಹಾ-ರಾ-ಜ-ರನ್ನು
ಗುರು-ಮ-ಹಾ-ರಾ-ಜರು
ಗುರು-ವನ್ನು
ಗುರು-ವಾ-ಗಲು
ಗುರು-ವಾಗಿ
ಗುರು-ವಾದ
ಗುರು-ವಾ-ದರು
ಗುರು-ವಾ-ದ-ವ-ನಿಗೆ
ಗುರು-ವಿಗೆ
ಗುರು-ವಿನ
ಗುರು-ವಿ-ನಂತೆ
ಗುರು-ವಿ-ನಲ್ಲೂ
ಗುರು-ವಿ-ನಷ್ಟೇ
ಗುರು-ವಿ-ನಿಂದ
ಗುರುವು
ಗುರುವೂ
ಗುರುವೆ
ಗುರು-ವೆಂದು
ಗುರು-ವೆಂದೇ
ಗುರು-ವೆಂಬ
ಗುರು-ಶಿ-ಷ್ಯ-ರಿ-ಬ್ಬರೂ
ಗುರು-ಸ್ಥಾ-ನ-ದಿಂದ
ಗುಲಾಬಿ
ಗುಲಾಮ
ಗುಲಾ-ಮ-ಎ-ರಡೂ
ಗುಲಾ-ಮ-ಗಿರಿ
ಗುಲಾ-ಮ-ಗಿ-ರಿಯೂ
ಗುಲಾ-ಮ-ಗಿ-ರಿಯೇ
ಗುಲಾ-ಮನೆ
ಗುಲಾ-ಮ-ನೆಂದೇ
ಗುಲಾ-ಮರ
ಗುಲಾ-ಮ-ರಂತೆ
ಗುಲಾ-ಮ-ರಾಗಿ
ಗುಲಾ-ಮ-ರಾ-ಗಿ-ರು-ವುದು
ಗುಲಾ-ಮ-ರಾ-ಷ್ಟ್ರ-ದ-ವ-ನಾದ
ಗುಲಾ-ಮ-ರಾ-ಷ್ಟ್ರ-ವಾ-ಗಿದ್ದ
ಗುಲಾ-ಮರು
ಗುಲಾ-ಮ-ರೆ-ಲ್ಲರ
ಗುಲ್ಲೆ-ಬ್ಬಿ-ಸು-ತ್ತಿ-ದ್ದರು
ಗುಳಕ್
ಗುಳ್ಳೆ
ಗುಳ್ಳೇ-ನರಿ
ಗುಸು-ಗುಸು
ಗುಹೆ
ಗುಹೆ-ಗಳನ್ನು
ಗುಹೆ-ಗಳೂ
ಗುಹೆಗೆ
ಗುಹೆಯ
ಗುಹೆ-ಯನ್ನು
ಗುಹೆ-ಯಲ್ಲಿ
ಗುಹೆ-ಯೊಂ-ದರ
ಗುಹೆ-ಯೊಂ-ದ-ರಲ್ಲಿ
ಗುಹೆ-ಯೊ-ಳಗೆ
ಗೂಡು-ಕ-ಟ್ಟಿ-ಕೊಂ-ಡಿದ್ದ
ಗೂಡು-ಗಳು
ಗೂಡ್ಸ್
ಗೂಢಾ-ರ್ಥ-ವಿ-ದ್ದಂ-ತಿತ್ತು
ಗೂಳಿ
ಗೂಳಿಗೆ
ಗೃಧ್ರ-ದೃಷ್ಟಿ
ಗೃಹ
ಗೃಹದ
ಗೃಹವು
ಗೃಹ-ಸ್ಥ-ನನ್ನು
ಗೃಹ-ಸ್ಥರ
ಗೃಹ-ಸ್ಥ-ರಾದ
ಗೃಹ-ಸ್ಥ-ರಿಗೆ
ಗೃಹ-ಸ್ಥರು
ಗೃಹ-ಸ್ಥರೂ
ಗೃಹಿ-ಣಿ-ಯ-ರನ್ನು
ಗೃಹೀ
ಗೃಹೀ-ಭ-ಕ್ತರ
ಗೃಹೀ-ಭ-ಕ್ತ-ರಲ್ಲ
ಗೆ
ಗೆಜೆಟ್
ಗೆಡ್ಡೆ-ಕ-ಟ್ಟಿದ
ಗೆದ್ದ
ಗೆದ್ದ-ವರ
ಗೆದ್ದಿ-ದ್ದೇವೆ
ಗೆದ್ದು
ಗೆದ್ದು-ಕೊಂಡ
ಗೆದ್ದು-ಕೊ-ಳ್ಳ-ಬಲ್ಲ
ಗೆದ್ದು-ಕೊ-ಳ್ಳ-ಬೇ-ಕಾ-ದರೆ
ಗೆದ್ದು-ಕೊ-ಳ್ಳ-ಬೇಕು
ಗೆದ್ದು-ಗೊ-ಳ್ಳ-ಬಲ್ಲ
ಗೆದ್ದುವು
ಗೆದ್ದೆ
ಗೆಲು-ವೆಲ್ಲ
ಗೆಲ್ಲ
ಗೆಲ್ಲ-ಬೇ-ಕೆಂದು
ಗೆಲ್ಲ-ಬೇ-ಕೆಂಬ
ಗೆಲ್ಲ-ಬೇ-ಕೆಂ-ಬು-ದ-ಕ್ಕಿಂತ
ಗೆಲ್ಲಲು
ಗೆಲ್ಲ-ಲೇ-ಬೇಕು
ಗೆಲ್ಲು-ತ್ತದೆ
ಗೆಲ್ಲು-ವುದು
ಗೆಳತಿ
ಗೆಳ-ತಿ-ಯ-ರಿಗೆ
ಗೆಳೆಯ
ಗೆಳೆ-ಯ-ರಾದ
ಗೆಳೆ-ಯ-ರಿಂದ
ಗೆಳೆ-ಯರೂ
ಗೆಳೆ-ಯ-ರೆ-ಲ್ಲರ
ಗೆಳೆ-ಯ-ರೊ-ಬ್ಬರು
ಗೇರಿದ
ಗೇಲಿ
ಗೈಯ-ಲೇ-ಬೇಕು
ಗೈರಿ-ಕ-ವ-ಸನ
ಗೊಂಚಲು
ಗೊಂಡ
ಗೊಂಡರು
ಗೊಂಡಳು
ಗೊಂಡ-ವ-ರಂತೆ
ಗೊಂಡವು
ಗೊಂಡಿತು
ಗೊಂಡಿತ್ತು
ಗೊಂಡಿ-ದ್ದರು
ಗೊಂಡು
ಗೊಂದ-ಲಕ್ಕೆ
ಗೊಂದ-ಲ-ವೇ-ರ್ಪ-ಟ್ಟಿತು
ಗೊಂದು
ಗೊಂಬೆ-ಯಂತೆ
ಗೊಡದೆ
ಗೊಡ-ವೆಗೇ
ಗೊಡ್ಡು
ಗೊಣ-ಗಾಟ
ಗೊಣ-ಗುಟ್ಟಿ
ಗೊಣ-ಗು-ಟ್ಟಿ-ಕೊ-ಳ್ಳುತ್ತ
ಗೊತ್ತಾ
ಗೊತ್ತಾ-ಗ-ಬೇ-ಕಾ-ದರೆ
ಗೊತ್ತಾ-ಗ-ಬೇಕು
ಗೊತ್ತಾ-ಗ-ಲಿಲ್ಲ
ಗೊತ್ತಾ-ಗಿ-ತ್ತ-ಲ್ಲವೆ
ಗೊತ್ತಾ-ಗಿತ್ತು
ಗೊತ್ತಾ-ಗಿ-ಬಿ-ಡು-ತ್ತಿತ್ತು
ಗೊತ್ತಾ-ಗಿ-ರ-ಲಿಲ್ಲ
ಗೊತ್ತಾ-ಗಿ-ಹೋ-ಗಿದೆ
ಗೊತ್ತಾ-ಗು-ತ್ತದೆ
ಗೊತ್ತಾ-ಗು-ತ್ತ-ದೆ-ಕ್ರೈಸ್ತ
ಗೊತ್ತಾ-ಗು-ತ್ತ-ದೆ-ತಾವು
ಗೊತ್ತಾ-ಗು-ತ್ತಿತ್ತು
ಗೊತ್ತಾ-ದದ್ದು
ಗೊತ್ತಾ-ದರೆ
ಗೊತ್ತಾ-ದಾಗ
ಗೊತ್ತಾ-ಯಿತು
ಗೊತ್ತಾ-ಯಿ-ತು-ಅಂದು
ಗೊತ್ತಾ-ಯಿ-ತು-ಅದು
ಗೊತ್ತಾ-ಯಿ-ತು-ಇ-ವರು
ಗೊತ್ತಾ-ಯಿ-ತು-ಸ್ವಾ-ಮೀಜಿ
ಗೊತ್ತಾ-ಯಿ-ತು-ಸ್ವಾ-ಮೀ-ಜಿಯ
ಗೊತ್ತಾ-ಯಿ-ತೆಂದು
ಗೊತ್ತಿಗೆ
ಗೊತ್ತಿತ್ತು
ಗೊತ್ತಿದೆ
ಗೊತ್ತಿ-ದೆ-ನಾನು
ಗೊತ್ತಿ-ದೆಯೆ
ಗೊತ್ತಿ-ದ್ದಿ-ರ-ಲಾ-ರದು
ಗೊತ್ತಿ-ದ್ದೀತು
ಗೊತ್ತಿ-ರ-ಬ-ಹುದು
ಗೊತ್ತಿ-ರ-ಲಿಲ್ಲ
ಗೊತ್ತಿ-ರು-ವು-ದಿಲ್ಲ
ಗೊತ್ತಿ-ರು-ವುದು
ಗೊತ್ತಿ-ರು-ವುದೇ
ಗೊತ್ತಿ-ರು-ವು-ದೇ-ನೆಂ-ದರೆ
ಗೊತ್ತಿಲ್ಲ
ಗೊತ್ತಿ-ಲ್ಲದ
ಗೊತ್ತಿ-ಲ್ಲ-ದಿ-ದ್ದರೂ
ಗೊತ್ತಿ-ಲ್ಲವೆ
ಗೊತ್ತಿ-ಲ್ಲವೋ
ಗೊತ್ತು
ಗೊತ್ತು-ಇದು
ಗೊತ್ತು-ಗು-ರಿ-ಯಿ-ಲ್ಲದೆ
ಗೊತ್ತು-ಭ-ಗ-ವಂ-ತನೇ
ಗೊತ್ತು-ಮಾ-ಡಿದ್ದ
ಗೊತ್ತು-ವ-ಳಿಯ
ಗೊತ್ತು-ವ-ಳಿ-ಯನ್ನು
ಗೊತ್ತು-ವ-ಳಿ-ಯೊಂ-ದನ್ನು
ಗೊತ್ತೂ
ಗೊತ್ತೇ
ಗೊತ್ತೇನು
ಗೊಯ್ಯು-ತ್ತಿವೆ
ಗೊಲಾಪ್
ಗೊಳ-ಗಾ-ದಂ-ತೆಯೇ
ಗೊಳಿ-ಸ-ಬ-ಲ್ಲಿ-ರಾ-ದರೆ
ಗೊಳಿ-ಸ-ಲಾ-ಯಿತು
ಗೊಳಿಸಿ
ಗೊಳಿ-ಸಿ-ದ್ದ-ವ-ನ-ಲ್ಲವೆ
ಗೊಳಿ-ಸು-ತ್ತಾರೆ
ಗೊಳಿ-ಸು-ತ್ತಿತ್ತು
ಗೊಳಿ-ಸು-ತ್ತೇನೆ
ಗೊಳಿ-ಸು-ವಂ-ತಹ
ಗೊಳಿ-ಸು-ವು-ದ-ಕ್ಕಾಗಿ
ಗೊಳಿ-ಸು-ವುದೂ
ಗೊಳ್ಳ-ಬ-ಲ್ಲು-ದಷ್ಟೇ
ಗೊಳ್ಳಿ
ಗೊಳ್ಳು-ವ-ವ-ರೆಗೆ
ಗೋಕು-ಲ-ದಾ-ಸನ
ಗೋಕು-ಲ-ದಾಸ್
ಗೋಕು-ಲ್ದಾ-ಸನ
ಗೋಕು-ಲ್ದಾ-ಸ-ನಿಗೆ
ಗೋಕು-ಲ್ದಾಸ್
ಗೋಗ-ರೆ-ದರು
ಗೋಗ-ರೆ-ದರೂ
ಗೋಚ-ರ-ವಾ-ಗದೆ
ಗೋಚ-ರ-ವಾ-ಗ-ಲಿಲ್ಲ
ಗೋಚ-ರ-ವಾ-ಗು-ತ್ತಿತ್ತು
ಗೋಚ-ರ-ವಾ-ಗು-ತ್ತಿ-ತ್ತು-ಪಾ-ಶ್ಚಾತ್ಯ
ಗೋಚ-ರ-ವಾ-ಗುವ
ಗೋಚ-ರ-ವಾದ
ಗೋಚ-ರ-ವಾ-ದುವು
ಗೋಚ-ರ-ವಾ-ಯಿತು
ಗೋಚ-ರಿ-ಸದ
ಗೋಚ-ರಿ-ಸಿತು
ಗೋಚ-ರಿ-ಸಿ-ತು-ತಾವು
ಗೋಚ-ರಿ-ಸಿ-ರ-ಲಿ-ಲ್ಲವೋ
ಗೋಜಿಗೆ
ಗೋಡೆ
ಗೋಡೆ-ಗಳ
ಗೋಡೆಯ
ಗೋಪಿಯ
ಗೋಮು-ಖ-ವ್ಯಾ-ಘ್ರ-ನೆಂದೂ
ಗೋರಿ-ಯನ್ನು
ಗೋಲ್ಕೊಂ-ಡದ
ಗೋಳಾ-ಡು-ತ್ತಿ-ದ್ದಳು
ಗೋಳಿಗೆ
ಗೋಳು
ಗೋವ-ರ್ಧನ
ಗೋವ-ಳ-ರಿಂದ
ಗೋವಾ
ಗೋವಾಕ್ಕೆ
ಗೋವಾಗೆ
ಗೋವಾದ
ಗೋವಾ-ದ-ಲ್ಲಿದ್ದ
ಗೋವಿಂದ
ಗೋವು-ಗಳ
ಗೋಷ್ಠಿ
ಗೌರವ
ಗೌರ-ವ-ಕೃ-ತ-ಜ್ಞ-ತೆ-ಗಳನ್ನು
ಗೌರ-ವ-ಮ-ನ್ನ-ಣೆ-ಗಳ
ಗೌರ-ವ-ಮ-ನ್ನ-ಣೆ-ಗ-ಳಿಗೆ
ಗೌರ-ವಕ್ಕೂ
ಗೌರ-ವಕ್ಕೆ
ಗೌರ-ವ-ಗಳನ್ನು
ಗೌರ-ವ-ಗಳಿಂದ
ಗೌರ-ವ-ದಿಂದ
ಗೌರ-ವ-ಪೂ-ರ್ವಕ
ಗೌರ-ವ-ಭಾ-ವವು
ಗೌರ-ವ-ಯುತ
ಗೌರ-ವ-ಯು-ತವೂ
ಗೌರ-ವ-ರ್ಣ-ದ-ವ-ರಾ-ದರೂ
ಗೌರ-ವ-ವನ್ನು
ಗೌರ-ವ-ವನ್ನೂ
ಗೌರ-ವ-ವಿಲ್ಲ
ಗೌರ-ವವು
ಗೌರ-ವವೇ
ಗೌರ-ವಸ್ಥ
ಗೌರ-ವಾ-ದರ
ಗೌರ-ವಾ-ದ-ರ-ಗಳು
ಗೌರ-ವಾ-ನ್ವಿತ
ಗೌರ-ವಾ-ಭಿ-ಪ್ರಾ-ಯ-ಗಳನ್ನು
ಗೌರ-ವಾ-ರ್ಥ-ವಾಗಿ
ಗೌರ-ವಾ-ಸ-ನದ
ಗೌರ-ವಿಸ
ಗೌರ-ವಿ-ಸ-ದಿ-ರಲು
ಗೌರ-ವಿ-ಸ-ಬಲ್ಲ
ಗೌರ-ವಿ-ಸಲಾ
ಗೌರ-ವಿ-ಸ-ಲಾಗಿದೆ
ಗೌರ-ವಿ-ಸ-ಲಾ-ಗು-ತ್ತದೆ
ಗೌರ-ವಿ-ಸ-ಲಾ-ರನೋ
ಗೌರ-ವಿ-ಸಲು
ಗೌರ-ವಿ-ಸ-ಲ್ಪಟ್ಟ
ಗೌರ-ವಿ-ಸ-ಲ್ಪ-ಟ್ಟದ್ದು
ಗೌರ-ವಿ-ಸ-ಲ್ಪ-ಡು-ವು-ದಿಲ್ಲ
ಗೌರ-ವಿಸಿ
ಗೌರ-ವಿಸು
ಗೌರ-ವಿ-ಸು-ತ್ತಾರೆ
ಗೌರ-ವಿ-ಸು-ತ್ತೇನೆ
ಗೌರ-ವಿ-ಸು-ವ-ವ-ರಿಲ್ಲ
ಗೌರ-ವಿ-ಸು-ವ-ವರೇ
ಗೌರ-ವಿ-ಸುವು
ಗೌರ-ವಿ-ಸು-ವು-ದ-ಕ್ಕಲ್ಲ
ಗೌರ-ವಿ-ಸು-ವು-ದ-ಕ್ಕಾಗಿ
ಗೌರಿ
ಗೌರೀ
ಗೌಲ್ಡ್
ಗ್ಯಾಂಜಿಸ್
ಗ್ಯಾಲ-ರಿ-ಗ-ಳಲ್ಲೂ
ಗ್ರಂಥ
ಗ್ರಂಥ-ಕ-ರ್ತೆ-ಯಾಗಿಯೂ
ಗ್ರಂಥ-ಗಳ
ಗ್ರಂಥ-ಗಳನ್ನು
ಗ್ರಂಥ-ಗಳನ್ನೂ
ಗ್ರಂಥ-ಗಳನ್ನೆಲ್ಲ
ಗ್ರಂಥ-ಗಳಲ್ಲಿ
ಗ್ರಂಥ-ಗ-ಳ-ಲ್ಲಿನ
ಗ್ರಂಥ-ಗಳಿಂದ
ಗ್ರಂಥ-ಗಳು
ಗ್ರಂಥ-ಗ-ಳೆಲ್ಲ
ಗ್ರಂಥದ
ಗ್ರಂಥ-ದಲ್ಲಿ
ಗ್ರಂಥ-ಮಾ-ಲೆಯ
ಗ್ರಂಥ-ರಾ-ಶಿ-ಇವು
ಗ್ರಂಥ-ರಾ-ಶಿ-ಯಿಂದ
ಗ್ರಂಥ-ವನ್ನು
ಗ್ರಂಥ-ವನ್ನೇ
ಗ್ರಂಥ-ವಾದ
ಗ್ರಂಥವು
ಗ್ರಂಥವೇ
ಗ್ರಂಥ-ವೊಂ-ದನ್ನು
ಗ್ರಂಥಾ-ಲ-ಯ-ಗಳು
ಗ್ರಂಥಾ-ಲ-ಯದ
ಗ್ರಂಥಾ-ಲ-ಯ-ದಲ್ಲಿ
ಗ್ರಂಥಾ-ಲ-ಯ-ವನ್ನು
ಗ್ರಂಥಾ-ಲ-ಯ-ವನ್ನೂ
ಗ್ರಹ-ಚಾರ
ಗ್ರಹ-ಚಾ-ರ-ವಪ್ಪ
ಗ್ರಹಣ
ಗ್ರಹ-ಣ-ಸಾ-ಮ-ರ್ಥ್ಯ-ವನ್ನು
ಗ್ರಹವು
ಗ್ರಹಿಕೆ
ಗ್ರಹಿಸ
ಗ್ರಹಿ-ಸ-ಬೇಕು
ಗ್ರಹಿ-ಸಲು
ಗ್ರಹಿಸಿ
ಗ್ರಹಿ-ಸಿದ
ಗ್ರಹಿ-ಸಿದ್ದೂ
ಗ್ರಹಿ-ಸಿ-ಯಾರು
ಗ್ರಹಿ-ಸು-ತ್ತಿದ್ದ
ಗ್ರಹಿ-ಸುವ
ಗ್ರಾಫಿಕ್
ಗ್ರಾಮ-ಗಳಲ್ಲಿ
ಗ್ರಾಮ-ಜೀ-ವ-ನದ
ಗ್ರಾಮ-ದ-ಲ್ಲಿಯೂ
ಗ್ರಾಮ-ವಾದ
ಗ್ರಾಮಾಂ-ತರ
ಗ್ರಾಮಾ-ಫೋನ್
ಗ್ರಾಮಾ-ಭಿ-ವೃ-ದ್ಧಿಯ
ಗ್ರಾಮೀಣ
ಗ್ರಾಹ್ಯ
ಗ್ರಾಹ್ಯ-ವಾ-ಗುವ
ಗ್ರೀಕರ
ಗ್ರೀಕ್
ಗ್ರೀನೇ-ಕರ್
ಗ್ರೀನೇ-ಕ-ರ್ನಲ್ಲಿ
ಗ್ರೀನ್
ಗ್ರೀನ್ಸ್ಟೈ-ಡೆಲ್
ಗ್ರ್ಯಾಜು-ಯೇಟ್
ಗ್ಲೆನ್
ಗ್ಲೆನ್ನ-ಳಿಗೆ
ಗ್ಲೆನ್ನಳೇ
ಗ್ಲೇಷಿ-ಯ-ರ್ನೀ-ರ್ಗ-ಲ್ಲಿನ
ಘಂಟಾ-ಘೋ-ಷ-ವಾಗಿ
ಘಂಟಾ-ನಾ-ದವು
ಘಂಟೆ
ಘಂಟೆ-ಯನ್ನು
ಘಟ-ನಾ-ವ-ಳಿ-ಗಳನ್ನು
ಘಟನೆ
ಘಟ-ನೆ-ಗಳ
ಘಟ-ನೆ-ಗಳನ್ನು
ಘಟ-ನೆ-ಗಳಲ್ಲಿ
ಘಟ-ನೆ-ಗ-ಳ-ಲ್ಲೊಂದು
ಘಟ-ನೆ-ಗಳು
ಘಟ-ನೆ-ಗಳೇ
ಘಟ-ನೆಯ
ಘಟ-ನೆ-ಯನ್ನು
ಘಟ-ನೆ-ಯಾ-ದರೂ
ಘಟ-ನೆ-ಯಿದು
ಘಟ-ನೆಯೂ
ಘಟ-ನೆಯೇ
ಘಟ-ನೆ-ಯೊಂದು
ಘಟಿ
ಘಟಿ-ಸಲು
ಘಟ್ಟ
ಘಟ್ಟಕ್ಕೆ
ಘಟ್ಟ-ವನ್ನು
ಘನ
ಘನ-ಗಂ-ಭೀರ
ಘನ-ಗಂ-ಭೀ-ರ-ವಾಗಿ
ಘನತೆ
ಘನ-ತೆ-ಮ-ಹ-ತ್ವ-ಗಳ
ಘನ-ತೆ-ಮ-ಹ-ತ್ವ-ಗಳನ್ನು
ಘನ-ತೆ-ವೈ-ಭ-ವ-ಗಳನ್ನು
ಘನ-ತೆಗೆ
ಘನ-ತೆಯ
ಘನ-ತೆ-ಯ-ನ್ನ-ರಿ-ಯುವ
ಘನ-ತೆ-ಯನ್ನು
ಘನ-ತೆ-ಯನ್ನೂ
ಘನ-ವಂತ
ಘನ-ವಾಗಿ
ಘನ-ವ್ಯ-ಕ್ತಿ-ಗ-ಳಾದ
ಘನ-ಸಂ-ದೇ-ಶ-ವನ್ನು
ಘನ-ಸ-ತ್ಯ-ವನ್ನು
ಘನೀ-ಭೂ-ತ-ವಾ-ದಂ-ತಿದೆ
ಘಮ-ಘ-ಮಿ-ಸುವ
ಘರ
ಘರ್ಜಿ-ಸಿ-ದರು
ಘಳಿಗೆ
ಘಳಿ-ಗೆ-ಯಲ್ಲಿ
ಘಾಜೀ-ಪುರ
ಘೋರ
ಘೋರ-ಯು-ದ್ಧ-ಗಳನ್ನೂ
ಘೋಷ-ಣೆಯ
ಘೋಷ-ದೊಂ-ದಿಗೆ
ಘೋಷನ
ಘೋಷಿ
ಘೋಷಿತ
ಘೋಷಿ-ಸ-ಬೇ-ಕಾ-ಗಿದೆ
ಘೋಷಿ-ಸ-ಬೇ-ಕೆಂದು
ಘೋಷಿಸಿ
ಘೋಷಿ-ಸಿದ
ಘೋಷಿ-ಸಿ-ದರು
ಘೋಷಿ-ಸು-ತ್ತಿ-ದ್ದರು
ಘೋಷಿ-ಸುವ
ಘೋಷಿ-ಸು-ವುದನ್ನು
ಘೋಷ್
ಘೋಷ್ರ-ವರು
ಚ
ಚಂಚಲ
ಚಂಡ-ಮಾ-ರುತ
ಚಂಡ-ಮಾ-ರು-ತ-ದಂ-ತಿ-ದ್ದರು
ಚಂಡ-ಮಾ-ರು-ತ-ದಂತೆ
ಚಂಡ-ಮಾ-ರು-ತ-ದಲ್ಲಿ
ಚಂಡಾಲ
ಚಂಡಾ-ಲ-ರ-ವ-ರೆಗೂ
ಚಂಡಾ-ಲ-ರಿಂದ
ಚಂಡಿ-ದಾಸ
ಚಂದ
ಚಂದಾ
ಚಂದೂ-ಲಾ-ಲನ್
ಚಂದ್ರ
ಚಂದ್ರಜಿ
ಚಂದ್ರ-ರಿ-ರು-ವ-ವ-ರೆಗೂ
ಚಕಾ-ರ-ವೆ-ತ್ತು-ವಂ-ತಿ-ರ-ಲಿಲ್ಲ
ಚಕಿ-ತ-ರಾಗಿ
ಚಕಿ-ತ-ರಾ-ದರು
ಚಕ್ರ-ಗಳ
ಚಕ್ರ-ವರ್ತಿ
ಚಕ್ರ-ವ-ರ್ತಿಯ
ಚಕ್ರ-ವ-ರ್ತಿ-ಯೊ-ಬ್ಬ-ನಿಗೆ
ಚಕ್ರ-ವ್ಯೂ-ಹ-ದಲ್ಲಿ
ಚಚ್ಚಿ-ಕೊಂ-ಡರೋ
ಚಚ್ಚಿ-ಕೊಂಡು
ಚಟರ್ಜಿ
ಚಟ-ರ್ಜಿ-ಇ-ವರ
ಚಟ-ರ್ಜಿಯ
ಚಟ-ರ್ಜಿ-ಯ-ವ-ರನ್ನು
ಚಟ-ರ್ಜಿ-ಯ-ವರು
ಚಟ-ರ್ಜಿ-ಯ-ವರೂ
ಚಟ-ರ್ಜಿ-ಶ್ರೀ-ಮತಿ
ಚಟಾಕಿ
ಚಟಾ-ಕಿ-ಗಳನ್ನು
ಚಟಾ-ಕಿ-ಗಳು
ಚಟಾಕು
ಚಟು-ವ-ಟಿಕೆ
ಚಟು-ವ-ಟಿ-ಕೆ-ಗಳ
ಚಟು-ವ-ಟಿ-ಕೆ-ಗಳನ್ನು
ಚಟು-ವ-ಟಿ-ಕೆ-ಗಳಲ್ಲಿ
ಚಟು-ವ-ಟಿ-ಕೆ-ಗ-ಳಿಂ-ದಾಗಿ
ಚಟು-ವ-ಟಿ-ಕೆ-ಗ-ಳಿಗೆ
ಚಟು-ವ-ಟಿ-ಕೆ-ಗಳು
ಚಟು-ವ-ಟಿ-ಕೆ-ಗ-ಳೆ-ಲ್ಲಕ್ಕೂ
ಚಟು-ವ-ಟಿ-ಕೆಯ
ಚಟು-ವ-ಟಿ-ಕೆ-ಯದೇ
ಚಟು-ವ-ಟಿ-ಕೆ-ಯನ್ನು
ಚಟು-ವ-ಟಿ-ಕೆ-ಯಲ್ಲಿ
ಚಟು-ವ-ಟಿ-ಕೆ-ಯಿಂದ
ಚಟು-ವ-ಟಿ-ಕೆಯೂ
ಚಟ್ಟಂಬಿ
ಚಟ್ಟಾಂಬಿ
ಚಟ್ಣಿ-ಯನ್ನು
ಚಟ್ನಿ
ಚಟ್ನಿ-ಯನ್ನೂ
ಚಡ-ಪಡಿ
ಚಡ-ಪ-ಡಿ-ಕೆ-ಯನ್ನು
ಚಡ-ಪ-ಡಿ-ಸ-ಲಾ-ರಂ-ಭಿ-ಸಿತು
ಚಡ-ಪ-ಡಿ-ಸಿ-ದರು
ಚಡ-ಪ-ಡಿಸು
ಚಡ-ಪ-ಡಿ-ಸು-ತ್ತಿತ್ತು
ಚಡ-ಪ-ಡಿ-ಸು-ತ್ತಿದ್ದ
ಚಡ-ಪ-ಡಿ-ಸು-ತ್ತಿ-ದ್ದರು
ಚಡ-ಪ-ಡಿ-ಸು-ತ್ತಿ-ದ್ದೇನೆ
ಚಡ-ಪ-ಡಿ-ಸು-ತ್ತಿ-ರು-ವಂತೆ
ಚಣ-ಮಾ-ತ್ರದಿ
ಚತು-ರ-ನಾಗಿ
ಚತು-ರೋ-ಕ್ತಿ-ಗಳನ್ನೂ
ಚತು-ರೋ-ಪಾ-ಯ-ಗಳನ್ನೂ
ಚತು-ರ್ಧಾ-ಮ-ಗಳಲ್ಲಿ
ಚತು-ರ್ಧಾ-ಮ-ಗ-ಳ-ಲ್ಲೊಂ-ದಾದ
ಚತು-ರ್ಧಾ-ಮ-ಗ-ಳಾದ
ಚತು-ರ್ವರ್ಣ
ಚತು-ರ್ವ-ರ್ಣ-ಗಳ
ಚದು-ರಿ-ಹೋ-ಗಿ-ದ್ದರು
ಚಪ-ಲ-ಚಿ-ತ್ತ-ವನ್ನು
ಚಪಾತಿ
ಚಪಾ-ತಿ
ಚಪಾ-ತಿ-ಗಳನ್ನು
ಚಪಾ-ತಿ-ಗಳನ್ನೂ
ಚಪಾ-ತಿ-ಗಳು
ಚಪ್ಪ-ಟೆ-ಭಾಗ
ಚಪ್ಪ-ರಿಸಿ
ಚಪ್ಪ-ರಿ-ಸಿ-ಕೊಂಡು
ಚಪ್ಪ-ರಿ-ಸುತ್ತ
ಚಪ್ಪ-ಲಿ-ಇದು
ಚಪ್ಪಾಳೆ
ಚಮ-ತ್ಕಾರ
ಚಮ್ಮಾರ
ಚಮ್ಮಾ-ರ-ನಿಗೆ
ಚರಣ
ಚರ-ಣ-ಗ-ಳಿಗೆ
ಚರಿ-ತ-ರನ್ನು
ಚರಿ-ತೆಯ
ಚರಿ-ತೆ-ಯಾದ
ಚರಿ-ತ್ರ-ಪ್ರ-ಸಿದ್ಧ
ಚರಿ-ತ್ರಾರ್ಹ
ಚರಿ-ತ್ರಾ-ರ್ಹವೂ
ಚರಿತ್ರೆ
ಚರಿ-ತ್ರೆ-ಎ-ಲ್ಲ-ವನ್ನೂ
ಚರಿ-ತ್ರೆಯ
ಚರಿ-ತ್ರೆ-ಯನ್ನು
ಚರಿ-ತ್ರೆ-ಯನ್ನೂ
ಚರಿ-ತ್ರೆ-ಯಲ್ಲಿ
ಚರಿ-ತ್ರೆ-ಯಲ್ಲೇ
ಚರಿ-ತ್ರೆಯು
ಚರ್ಚನ್ನೂ
ಚರ್ಚಾ-ಕೂ-ಟ-ವಾ-ಯಿ-ತಷ್ಟೆ
ಚರ್ಚಿ
ಚರ್ಚಿಗೆ
ಚರ್ಚಿನ
ಚರ್ಚಿ-ನಲ್ಲಿ
ಚರ್ಚಿ-ಸ-ಬೇಕು
ಚರ್ಚಿ-ಸ-ಲ್ಪ-ಡು-ತ್ತಿತ್ತು
ಚರ್ಚಿಸಿ
ಚರ್ಚಿ-ಸಿದ
ಚರ್ಚಿ-ಸಿ-ದ-ರ-ಲ್ಲದೆ
ಚರ್ಚಿ-ಸಿ-ದರು
ಚರ್ಚಿ-ಸಿ-ದರೂ
ಚರ್ಚಿ-ಸಿದೆ
ಚರ್ಚಿ-ಸಿ-ರ-ಬೇ-ಕೆಂದು
ಚರ್ಚಿ-ಸುತ್ತ
ಚರ್ಚಿ-ಸು-ತ್ತಿದ್ದ
ಚರ್ಚಿ-ಸು-ತ್ತಿ-ದ್ದರು
ಚರ್ಚಿ-ಸು-ತ್ತಿ-ದ್ದೆವು
ಚರ್ಚಿ-ಸು-ತ್ತಿ-ರು-ವಂತೆ
ಚರ್ಚಿ-ಸುವ
ಚರ್ಚು
ಚರ್ಚು-ಗಳ
ಚರ್ಚು-ಗಳಲ್ಲಿ
ಚರ್ಚು-ಗ-ಳಿಗೆ
ಚರ್ಚು-ಗಳು
ಚರ್ಚೆ
ಚರ್ಚೆ-ಸಂ-ಭಾ-ಷಣೆ
ಚರ್ಚೆ-ಗಳ
ಚರ್ಚೆ-ಗಳನ್ನು
ಚರ್ಚೆ-ಗಳನ್ನೂ
ಚರ್ಚೆ-ಗಳಲ್ಲಿ
ಚರ್ಚೆ-ಗಳಿಂದ
ಚರ್ಚೆಗೆ
ಚರ್ಚೆಯ
ಚರ್ಚೆ-ಯಲ್ಲಿ
ಚರ್ಚೆ-ಯೇ-ರ್ಪ-ಟ್ಟಿತು
ಚರ್ಚೊಂ-ದರ
ಚರ್ಚ್
ಚರ್ಚ್ನಲ್ಲಿ
ಚರ್ಚ್ಳಿಗೂ
ಚರ್ಮ
ಚರ್ಮ-ದ-ವನು
ಚರ್ಮ-ರೋ-ಗ-ವಿ-ದ್ದುದ
ಚರ್ಯ
ಚರ್ವಿ-ತ-ಚ-ರ್ವ-ಣ-ವಾದ
ಚಲ-ನ-ರ-ಹಿತ
ಚಲ-ನ-ರ-ಹಿ-ತ-ವಾ-ಗು-ತ್ತಿತ್ತು
ಚಲ-ನ-ವ-ಲ-ನ-ಇದು
ಚಲ-ನ-ಶೀ-ಲ-ವಾ-ಗ-ಬೇ-ಕಾದ
ಚಲ-ನೆ-ಯನ್ನೂ
ಚಲಾ-ಯಿ-ಸಲು
ಚಲಾ-ಯಿ-ಸುವ
ಚಲಾ-ವ-ಣೆ-ಯ-ಲ್ಲಿದೆ
ಚಲಾ-ವ-ಣೆಯು
ಚಲಿ-ಸದೆ
ಚಲಿ-ಸು-ತ್ತಿ-ರುವ
ಚಳ-ವಳಿ
ಚಳ-ವ-ಳಿ-ಗಳ
ಚಳ-ವ-ಳಿ-ಗ-ಳಿಗೆ
ಚಳ-ವ-ಳಿ-ಗಳು
ಚಳ-ವ-ಳಿಯ
ಚಳ-ವ-ಳಿಯು
ಚಳಿ-ಗಳಿಂದ
ಚಳಿ-ಗಾಲ
ಚಳಿ-ಗಾ-ಲಕ್ಕೆ
ಚಳಿ-ಗಾ-ಲದ
ಚಳಿ-ಗಾ-ಲ-ದಲ್ಲಿ
ಚಳಿ-ಗಾ-ಲ-ವನ್ನು
ಚಳಿ-ಗಾ-ಲ-ವಾ-ದ್ದ-ರಿಂದ
ಚಳಿ-ಗಾ-ಲ-ವೆಲ್ಲ
ಚಳಿ-ಗಾಳಿ
ಚಳಿ-ಯನ್ನು
ಚಳಿ-ಯಿಂದ
ಚಳಿ-ಯಿಂ-ದಾಗಿ
ಚಳಿ-ಯಿ-ತ್ತಾ-ದರೂ
ಚಳಿಯೂ
ಚಳು-ವ-ಳಿಯ
ಚಹಾ
ಚಾಕು-ವಿ-ನಿಂದ
ಚಾಚಿದ
ಚಾಟಿ-ಯನ್ನು
ಚಾಟಿ-ಯೇ-ಟಿ-ನಂ-ತಹ
ಚಾಟಿ-ಯೇ-ಟಿ-ನಂತೆ
ಚಾತು-ರ್ಯ-ದಿಂದ
ಚಾತು-ರ್ಯ-ವ-ನ್ನ-ರಿ-ಯ-ದ-ವರು
ಚಾತು-ರ್ಯ-ವನ್ನು
ಚಾತು-ರ್ಯ-ವೆಂ-ಥದು
ಚಾನಲೇ
ಚಾಪೆ
ಚಾಪೆ-ಯನ್ನೂ
ಚಾಮ-ರ-ವನ್ನೂ
ಚಾಮ-ರಾಜ
ಚಾಮ-ರಾ-ಜೇಂದ್ರ
ಚಾರ-ಗಳ
ಚಾರದ
ಚಾರಿ-ತ್ರಿಕ
ಚಾರಿತ್ರ್ಯ
ಚಾರಿ-ತ್ರ್ಯ-ಇ-ವು-ಗಳೇ
ಚಾರಿ-ತ್ರ್ಯದ
ಚಾರಿ-ತ್ರ್ಯ-ವನ್ನೂ
ಚಾರಿ-ತ್ರ್ಯವು
ಚಾರ್ಮಿ-ನಾರ್
ಚಾರ್ಯ
ಚಾರ್ಯರೂ
ಚಾರ್ಲಟ್
ಚಾರ್ಲ್ಸ್
ಚಾಲಕ
ಚಾಲ-ಕ-ಶ-ಕ್ತಿ-ಯನ್ನು
ಚಾಲಾಕಿ
ಚಾವಟಿ
ಚಾವ-ಡಿಯ
ಚಾಹೇ
ಚಿಂತಕ
ಚಿಂತ-ಕರ
ಚಿಂತ-ಕ-ರನ್ನು
ಚಿಂತ-ಕ-ರನ್ನೂ
ಚಿಂತ-ಕ-ರಿಂದ
ಚಿಂತ-ಕ-ರಿ-ದ್ದರು
ಚಿಂತ-ಕರು
ಚಿಂತ-ಕರೂ
ಚಿಂತನ
ಚಿಂತ-ನ-ಮಂ-ಥನ
ಚಿಂತ-ನ-ಧಾರೆ
ಚಿಂತ-ನ-ಪ್ರ-ಪಂ-ಚದ
ಚಿಂತ-ನ-ಪ್ರ-ವಾ-ಹವು
ಚಿಂತ-ನ-ಲೋ-ಕ-ವನ್ನು
ಚಿಂತ-ನ-ಶೀಲ
ಚಿಂತ-ನ-ಶೀ-ಲರು
ಚಿಂತ-ನ-ಶೀ-ಲ-ರೊಂ-ದಿಗೆ
ಚಿಂತನೆ
ಚಿಂತ-ನೆ-ಗಳ
ಚಿಂತ-ನೆ-ಗಳು
ಚಿಂತ-ನೆ-ಗ-ಳು-ಇ-ವು-ಗ-ಳೆಲ್ಲ
ಚಿಂತ-ನೆಗೆ
ಚಿಂತ-ನೆಯ
ಚಿಂತ-ನೆ-ಯಲ್ಲಿ
ಚಿಂತ-ನೆಯು
ಚಿಂತಾ-ಕ್ರಾಂ-ತ-ರಾಗಿ
ಚಿಂತಿಸ
ಚಿಂತಿ-ಸ-ತೊ-ಡ-ಗಿ-ದರು
ಚಿಂತಿ-ಸದೆ
ಚಿಂತಿ-ಸ-ಬಾ-ರ-ದೆಂದು
ಚಿಂತಿ-ಸ-ಬೇ-ಕಾ-ಗಿ-ರ-ಲಿಲ್ಲ
ಚಿಂತಿ-ಸ-ಬೇ-ಕಾದ
ಚಿಂತಿ-ಸ-ಬೇ-ಕಿಲ್ಲ
ಚಿಂತಿ-ಸ-ಬೇಡ
ಚಿಂತಿ-ಸ-ಬೇ-ಡಿ-ಅದು
ಚಿಂತಿ-ಸ-ಲಾ-ರಂ-ಭಿ-ಸಿ-ದ್ದರು
ಚಿಂತಿ-ಸ-ಲಿಲ್ಲ
ಚಿಂತಿ-ಸಲು
ಚಿಂತಿಸಿ
ಚಿಂತಿ-ಸಿತು
ಚಿಂತಿ-ಸಿ-ದರು
ಚಿಂತಿ-ಸಿ-ದ್ದ-ರೆಂ-ಬು-ದನ್ನು
ಚಿಂತಿ-ಸುತ್ತ
ಚಿಂತಿ-ಸು-ತ್ತಿತ್ತು
ಚಿಂತಿ-ಸು-ತ್ತಿ-ದ್ದಂತೆ
ಚಿಂತಿ-ಸು-ತ್ತಿ-ದ್ದರು
ಚಿಂತಿ-ಸು-ತ್ತಿ-ದ್ದರೂ
ಚಿಂತಿ-ಸು-ತ್ತಿ-ದ್ದ-ಸ್ವಾ-ಮೀಜಿ
ಚಿಂತಿ-ಸು-ತ್ತಿ-ದ್ದೆನೋ
ಚಿಂತಿ-ಸು-ತ್ತೇವೆ
ಚಿಂತಿ-ಸು-ವಂ-ತಾ-ಗು-ತ್ತಿತ್ತು
ಚಿಂತಿ-ಸು-ವಂತೆ
ಚಿಂತಿ-ಸು-ವುದೇ
ಚಿಂತೆ
ಚಿಂತೆ-ಗಳನ್ನು
ಚಿಂತೆ-ಗಳನ್ನೂ
ಚಿಂತೆ-ಗ-ಳಾ-ವುವೂ
ಚಿಂತೆ-ಗಿ-ಟ್ಟು-ಕೊಂ-ಡಿತು
ಚಿಂತೆ-ಗೊ-ಳ-ಗಾ-ದರು
ಚಿಂತೆ-ತನ್ನ
ಚಿಂತೆ-ಯನ್ನು
ಚಿಂತೆ-ಯಲ್ಲಿ
ಚಿಂತೆ-ಯಾ-ಗು-ತ್ತಿತ್ತು
ಚಿಂತೆ-ಯಾದ
ಚಿಂತೆ-ಯಿಲ್ಲ
ಚಿಂತೆ-ಯುಂ-ಟಾ-ಯಿತು
ಚಿಂತೆಯೂ
ಚಿಂತೆಯೇ
ಚಿಕ್ಕ
ಚಿಕ್ಕ-ಚಿಕ್ಕ
ಚಿಕ್ಕ-ದಾ-ಗಿ-ದ್ದಷ್ಟೂ
ಚಿಕ್ಕ-ಪುಟ್ಟ
ಚಿಕ್ಕ-ಪ್ಪ-ಇ-ಬ್ಬರೂ
ಚಿಕ್ಕ-ಮ-ಗ-ಳೂ-ರಿನ
ಚಿಕ್ಕಮ್ಮ
ಚಿಕ್ಕ-ಮ್ಮ-ನಿ-ಗಾಗಿ
ಚಿಕ್ಕ-ವ-ಯ-ಸ್ಸಿ-ನಲ್ಲೇ
ಚಿಕ್ಕ-ವರೂ
ಚಿಕ್ಕಾ-ಸನ್ನೂ
ಚಿಕ್ಕು-ವಂತೆ
ಚಿಗು-ರ-ಲಾ-ರಂ-ಭಿ-ಸಿದ್ದ
ಚಿಗುರಿ
ಚಿಟ್ಟು
ಚಿತ
ಚಿತ್ತ-ಭ್ರಾಂ-ತಿ-ಯಂತೆ
ಚಿತ್ರ
ಚಿತ್ರ-ಕಲಾ
ಚಿತ್ರ-ಗಳ
ಚಿತ್ರ-ಗಳನ್ನು
ಚಿತ್ರ-ಗ-ಳಿವೆ
ಚಿತ್ರ-ಗಳು
ಚಿತ್ರ-ಣ-ದಲ್ಲಿ
ಚಿತ್ರ-ಣ-ವನ್ನು
ಚಿತ್ರ-ದಲ್ಲಿ
ಚಿತ್ರ-ದಲ್ಲೂ
ಚಿತ್ರ-ಫ-ಲ-ಕ-ಗ-ಳಿವೆ
ಚಿತ್ರ-ವನ್ನು
ಚಿತ್ರ-ವನ್ನೂ
ಚಿತ್ರ-ವಿ-ಚಿತ್ರ
ಚಿತ್ರ-ವೊಂ-ದನ್ನು
ಚಿತ್ರಿ-ಸಿದ
ಚಿತ್ರಿ-ಸಿ-ದ್ದರು
ಚಿನ್ನ
ಚಿನ್ನದ
ಚಿನ್ನ-ದಂತೆ
ಚಿನ್ನ-ವ-ನ್ನಾ-ಗಿ-ಪುದು
ಚಿನ್ನ-ವನ್ನು
ಚಿನ್ನವೇ
ಚಿನ್ನ-ವೇ-ನಾ-ದರೂ
ಚಿಪ್ಪೊ-ಡೆದು
ಚಿಮ್ಮಿ
ಚಿಮ್ಮಿತು
ಚಿಮ್ಮಿ-ದರು
ಚಿಮ್ಮಿ-ಬ-ರು-ತ್ತಿ-ದ್ದು-ವು-ಇಲ್ಲ
ಚಿಮ್ಮಿ-ಸುವ
ಚಿಮ್ಮು-ತ್ತಿ-ರು-ವಾ-ಗಲೂ
ಚಿಮ್ಮುವ
ಚಿರಂ-ತನ
ಚಿರ-ಪ-ರಿ-ಚಿತ
ಚಿರ-ಪ-ರಿ-ಚಿ-ತ-ರಂತೆ
ಚಿರ-ಪ-ರಿ-ಚಿ-ತ-ವಾದ
ಚಿರ-ಪು-ಣಿ-ಗ-ಳಾ-ಗಿ-ದ್ದೇವೆ
ಚಿರ-ಪು-ಣಿ-ಯಾ-ಗಿ-ರ-ಬೇಕು
ಚಿರ-ಪು-ಣಿ-ಯೇನೋ
ಚಿರ-ಮು-ದ್ರಿ-ತ-ವಾ-ಯಿತು
ಚಿರ-ಮು-ದ್ರೆ-ಯ-ನ್ನೊ-ತ್ತಿ-ದರು
ಚಿರ-ಸ್ಥಾ-ಯಿ-ಯಾ-ಗಿ-ಸಿವೆ
ಚಿರ-ಸ್ಮ-ರ-ಣೀ-ಯ-ವಾ-ದ-ವು-ಗಳು
ಚಿರ-ಸ್ವ-ತಂತ್ರ
ಚಿಲುಮೆ
ಚಿಲು-ಮೆ-ಯನ್ನು
ಚಿಲು-ಮೆ-ಯಿ-ದ್ದಂತೆ
ಚಿಸ್ತೀ
ಚಿಹ್ನೆ
ಚಿಹ್ನೆ-ಗಳು
ಚಿಹ್ನೆ-ಯಾ-ದಾಗ
ಚೀಟಿ
ಚೀಟಿ-ಗಳನ್ನು
ಚೀಟಿ-ಗಳಲ್ಲಿ
ಚೀಟಿ-ಯಲ್ಲಿ
ಚೀನಾ
ಚೀನಾಕ್ಕೂ
ಚೀನಾಕ್ಕೆ
ಚೀನಾದ
ಚೀನಾ-ದಲ್ಲಿ
ಚೀನಿ
ಚೀನೀ
ಚೀನೀ-ಯನ
ಚೀನೀ-ಯ-ನನ್ನು
ಚೀನೀ-ಯರ
ಚೀನೀ-ಯರು
ಚೀಲ
ಚೀಲ-ವನ್ನು
ಚೀಲ-ವೊಂ-ದಿತ್ತು
ಚುಕ್ಕಾಣಿ
ಚುಕ್ಕಾ-ಣಿ-ಗಳು
ಚುಕ್ಕಾ-ಣಿ-ಯನ್ನು
ಚುಚ್ಚಿ
ಚುಟು-ಕಾ-ಗಿ-ದ್ದರೂ
ಚುಟು-ಕಾದ
ಚುನಾ-ಯಿ-ಸ-ಲ್ಪಟ್ಟ
ಚುರು-ಕಲ್ಲ
ಚುರು-ಕಾ-ಗಿ-ದ್ದುವು
ಚುರು-ಕಾದ
ಚುರು-ಕಾ-ದರು
ಚುರು-ಕಿನ
ಚುರುಕು
ಚುರು-ಕು-ಗೊ-ಳಿಸಿ
ಚುರು-ಕು-ತ-ನದ
ಚುರು-ಕು-ತ-ನವೂ
ಚೂರನ್ನು
ಚೂರ-ಲ್ಲವೆ
ಚೂರಿ-ಗಳನ್ನು
ಚೂರಿನ
ಚೂರು
ಚೂರು-ಗಳನ್ನು
ಚೂರು-ಗ-ಳಾಗಿ
ಚೂರು-ಚೂ-ರಾಗಿ
ಚೂರು-ಪಾ-ರು-ಗ-ಳಂತೆ
ಚೂರು-ಪಾ-ರು-ಗ-ಳನ್ನೇ
ಚೆಂಗಲ್
ಚೆಂಡನ್ನು
ಚೆಂಡು
ಚೆಟ್ಟಿ
ಚೆನ್ನಾಗಿ
ಚೆನ್ನಾ-ಗಿ-ಟ್ಟಿ-ರಲಿ
ಚೆನ್ನಾ-ಗಿತ್ತು
ಚೆನ್ನಾ-ಗಿದೆ
ಚೆನ್ನಾ-ಗಿ-ದ್ದೇನೆ
ಚೆನ್ನಾ-ಗಿಯೇ
ಚೆನ್ನಾ-ಗಿ-ರ-ಲಿಲ್ಲ
ಚೆನ್ನಾ-ಗಿ-ರ-ಲೇ-ಬೇಕು
ಚೆನ್ನಾ-ಗಿಲ್ಲ
ಚೆಲ್ಲಾ-ಪಿ-ಲ್ಲಿ-ಯಾ-ಗಿ-ದ್ದಾರೆ
ಚೆಲ್ಲಿ
ಚೆಲ್ಲಿತ್ತು
ಚೆಲ್ಲಿ-ದ್ದಾರೆ
ಚೇತ-ನ-ಗಳ
ಚೇತ-ನ-ಜ್ಯೋತಿ
ಚೇತ-ನ-ದಂ-ತಿ-ದ್ದರು
ಚೇತ-ನ-ದಿಂದ
ಚೇತ-ನ-ವನ್ನು
ಚೇತ-ನವು
ಚೇತ-ನವೇ
ಚೇತ-ನಾ-ಯು-ಕ್ತವೂ
ಚೇತ-ರಿಸಿ
ಚೇತ-ರಿ-ಸಿ-ಕೊಂ-ಡರು
ಚೇತ-ರಿ-ಸಿ-ಕೊ-ಳ್ಳಲು
ಚೇತ-ರಿ-ಸಿ-ಕೊ-ಳ್ಳುತ್ತ
ಚೇರ್ಮ-ನ್ನ-ರಾದ
ಚೇರ್ಮ-ನ್ನರೂ
ಚೇಷ್ಟೆ-ಯನ್ನು
ಚೈತನ್ಯ
ಚೈತ-ನ್ಯದ
ಚೈತ-ನ್ಯ-ಪೂರ್ಣ
ಚೈತ-ನ್ಯ-ಯು-ಕ್ತ-ವಾದ
ಚೈತ-ನ್ಯ-ವನ್ನು
ಚೈತ-ನ್ಯ-ವಿ-ದ್ದರೆ
ಚೈತ-ನ್ಯವೇ
ಚೊಕ್ಕಟ
ಚೊಕ್ಕ-ಟ-ವಾ-ಗಿ-ಡು-ತ್ತಿ-ದ್ದಳು
ಚೌಕ-ದಲ್ಲಿ
ಚೌಧರಿ
ಚ್ಛಕ್ತಿ-ಗ-ಳಿಗೆ
ಛಂಗ-ನೆದ್ದು
ಛತ್ರಿ-ಗ-ಳೊಂ-ದಿಗೆ
ಛತ್ರಿ-ಯನ್ನು
ಛಬಿಲ್
ಛಬಿ-ಲ್ದಾ-ಸನೂ
ಛಬಿ-ಲ್ದಾ-ಸ್ರಿಗೆ
ಛಲ-ದಿಂದ
ಛಲ-ಬಿ-ಡದ
ಛಲ-ವಂತ
ಛಲ-ವಂ-ತಿಕೆ
ಛಲ-ವನ್ನು
ಛಾಯಾ
ಛಾಯಾ-ಚಿತ್ರ
ಛಾಯಾ-ಚಿ-ತ್ರ-ಗಳನ್ನು
ಛಾಯೆ
ಛಾಯೆ-ಯನ್ನು
ಛಿದ್ರ-ಛಿ-ದ್ರ-ವಾಗಿ
ಛೀಮಾರಿ
ಛೀಮಾ-ರಿ-ಯನ್ನೂ
ಛೆ
ಛೇ
ಛೇಡಿ-ಸ-ಲಾ-ರಂಭಿ
ಜಂಜ-ಡ-ಗಳ
ಜಂಜ-ಡ-ಗಳಿಂದ
ಜಂಜ-ಡ-ಗ-ಳೆಲ್ಲ
ಜಂಜ-ಡ-ದಲ್ಲಿ
ಜಂತು-ವಿ-ನಲ್ಲೂ
ಜಂಬ
ಜಂಬ-ದಿಂದ
ಜಗ
ಜಗ-ಕೆಲ್ಲ
ಜಗ-ತ್ಕ-ಲ್ಯಾ-ಣ-ಕಾ-ರ್ಯ-ದಲ್ಲಿ
ಜಗ-ತ್ತನ್ನು
ಜಗ-ತ್ತನ್ನೇ
ಜಗ-ತ್ತ-ನ್ನೊಮ್ಮೆ
ಜಗತ್ತಿ
ಜಗ-ತ್ತಿಗೂ
ಜಗ-ತ್ತಿಗೆ
ಜಗ-ತ್ತಿ-ಗೆಲ್ಲ
ಜಗ-ತ್ತಿಗೇ
ಜಗ-ತ್ತಿ-ಗೇ-ನಾ-ದರೂ
ಜಗ-ತ್ತಿನ
ಜಗ-ತ್ತಿ-ನ-ಲ್ಲಾ-ಗಲಿ
ಜಗ-ತ್ತಿ-ನಲ್ಲಿ
ಜಗ-ತ್ತಿ-ನಲ್ಲೇ
ಜಗ-ತ್ತಿ-ನಾ-ದ್ಯಂತ
ಜಗ-ತ್ತಿ-ನಿಂದ
ಜಗತ್ತು
ಜಗ-ತ್ತು-ಗಳ
ಜಗ-ತ್ತು-ಗಳನ್ನು
ಜಗತ್ತೂ
ಜಗ-ತ್ತೆಂದು
ಜಗ-ತ್ತೆ-ನ್ನು-ವುದೇ
ಜಗತ್ತೇ
ಜಗ-ತ್ಪಿ-ತ-ನಾದ
ಜಗ-ತ್ಪ್ರ-ಸಿದ್ಧ
ಜಗ-ತ್ಪ್ರ-ಸಿ-ದ್ಧ-ರಾ-ದಂ-ದಿನ
ಜಗ-ತ್ಪ್ರ-ಸಿ-ದ್ಧ-ರಾ-ದಾಗ
ಜಗ-ತ್ಪ್ರ-ಸಿ-ದ್ಧ-ವಾ-ಗಿದೆ
ಜಗ-ದಂ-ಬೆಗೆ
ಜಗದಿ
ಜಗ-ದೆ-ಡೆ-ಗೆ-ಜ-ಗದ
ಜಗ-ದೆ-ದು-ರಿಗೆ
ಜಗ-ದೊಳು
ಜಗ-ದ್ಧಿ-ತ-ವನ್ನು
ಜಗ-ದ್ರೂ-ವಾ-ರಿ-ವಿ-ಶ್ವ-ಶಿಲ್ಪಿ
ಜಗ-ನ್ನಾ-ಥನ
ಜಗ-ನ್ಮಾತೆ
ಜಗ-ನ್ಮಾ-ತೆಗೆ
ಜಗ-ನ್ಮಾ-ತೆಯ
ಜಗ-ನ್ಮಾ-ತೆ-ಯನ್ನೂ
ಜಗ-ನ್ಮಾ-ತೆ-ಯಾದ
ಜಗ-ನ್ಮಾ-ತೆಯೆ
ಜಗ-ನ್ಮಾ-ತೆ-ಯೆಂದು
ಜಗ-ನ್ಮಾ-ತೆ-ಯೊಂ-ದಿಗೆ
ಜಗ-ನ್ಮಿಥ್ಯಾ
ಜಗ-ಮೋ-ಹನ
ಜಗ-ಮೋ-ಹ-ನ-ನಿಗೆ
ಜಗ-ಮೋ-ಹ-ನ-ನೊಂ-ದಿಗೆ
ಜಗ-ಮೋ-ಹ-ನ-ಲಾಲ
ಜಗ-ಮೋ-ಹ-ನ-ಲಾ-ಲ-ನನ್ನು
ಜಗ-ಮೋ-ಹ-ನ-ಲಾ-ಲ-ನೊಂ-ದಿಗೆ
ಜಗ-ಮೋ-ಹ-ನ-ಲಾಲ್
ಜಗ-ಮೋ-ಹ-ನ-ಲಾ-ಲ್ನೊಂ-ದಿಗೆ
ಜಗ-ಮೋ-ಹನ್
ಜಗ-ಲಿ-ಗಳು
ಜಗ-ಳ-ವಾ-ಡ-ಲೆಂ-ದಲ್ಲ
ಜಗ-ಳ-ವಾ-ಡು-ವು-ದ-ರಲ್ಲೇ
ಜಗು-ಲಿ-ಯಲ್ಲೂ
ಜಗ್ಗಲು
ಜಜ್ಜಿ-ಹೋ-ಗ-ಬ-ಹುದು
ಜಟಿಲ
ಜಟಿ-ಲ-ವಾ-ಗಿ-ರುವ
ಜಟಿ-ಲವೂ
ಜಟ್ಟಿ
ಜಡ
ಜಡ-ಚೇ-ತನ
ಜಡ-ತ್ವ-ದಾ-ಳಕ್ಕೆ
ಜಡ-ವಾ-ದದ
ಜಡೆ
ಜತೀಂ-ದ್ರ-ನಾಥ
ಜನ
ಜನ-ಇದು
ಜನಕ
ಜನ-ಕ-ರೆಂದೂ
ಜನ-ಕ-ವಾದ
ಜನ-ಕೋ-ಟಿಯ
ಜನಕ್ಕೆ
ಜನ-ಗ-ಣ-ತಿಯ
ಜನ-ಗಳ
ಜನ-ಗಳನ್ನು
ಜನ-ಗ-ಳ-ಲ್ಲ-ಡ-ಗಿ-ರುವ
ಜನ-ಗಳಲ್ಲಿ
ಜನ-ಗಳಿ
ಜನ-ಗ-ಳಿ-ಗಿಂತ
ಜನ-ಗ-ಳಿಗೆ
ಜನ-ಗ-ಳಿಗೇ
ಜನ-ಗಳು
ಜನ-ಗಳೇ
ಜನ-ಜಂ-ಗುಳಿ
ಜನ-ಜಂ-ಗು-ಳಿಯ
ಜನ-ಜಾ-ತ್ರೆಯೇ
ಜನ-ಜೀ-ವ-ನದ
ಜನ-ಜೀ-ವ-ನ-ದಲ್ಲಿ
ಜನ-ಜೀ-ವ-ನ-ವನ್ನು
ಜನ-ಜೀ-ವ-ನ-ವನ್ನೂ
ಜನ-ತೆಗೆ
ಜನ-ತೆಯ
ಜನ-ತೆ-ಯಲ್ಲಿ
ಜನ-ತೆ-ಯಿಂ-ದಲೂ
ಜನ-ದ-ಟ್ಟ-ಣೆ-ಯಿಂದ
ಜನ-ನ-ಮ-ರ-ಣ-ಗಳು
ಜನ-ನಾ-ಯ-ಕರೂ
ಜನ-ನಿಗೆ
ಜನ-ನಿ-ಯೆಂದು
ಜನ-ಪ್ರ-ವಾ-ಹದ
ಜನ-ಪ್ರಿಯ
ಜನ-ಪ್ರಿ-ಯತೆ
ಜನ-ಪ್ರಿ-ಯ-ತೆಯ
ಜನ-ಪ್ರಿ-ಯ-ತೆ-ಯನ್ನು
ಜನ-ಪ್ರಿ-ಯ-ತೆ-ಯನ್ನೂ
ಜನ-ಪ್ರಿ-ಯ-ತೆ-ಯೊಂ-ದಿಗೆ
ಜನ-ಪ್ರಿ-ಯ-ನಾ-ಗಿದ್ದ
ಜನ-ಪ್ರಿ-ಯ-ನಾ-ಗಿ-ದ್ದೇನೆ
ಜನ-ಪ್ರಿ-ಯ-ರಾ-ದರು
ಜನ-ಪ್ರಿ-ಯ-ವಾಗಿ
ಜನ-ಪ್ರಿ-ಯ-ವಾ-ಗಿತ್ತು
ಜನ-ಪ್ರಿ-ಯ-ವಾ-ಗಿದ್ದ
ಜನ-ಪ್ರಿ-ಯ-ವಾ-ಗಿ-ರು-ವು-ದ-ಲ್ಲದೆ
ಜನ-ಪ್ರಿ-ಯ-ವಾ-ಗಿ-ಸಿ-ದುವು
ಜನ-ಪ್ರಿ-ಯ-ವಾ-ದಂ-ತೆಲ್ಲ
ಜನ-ಬ-ಲ-ವನ್ನು
ಜನ-ಮ-ನ-ದಲ್ಲಿ
ಜನ-ಮ-ನ-ಮು-ಟ್ಟು-ವಂತೆ
ಜನ-ಮ-ನ-ವನ್ನು
ಜನ-ಮೋ-ಹ-ನ-ಲಾಲ್
ಜನರ
ಜನ-ರಂ-ತೆಯೇ
ಜನ-ರ-ಡಾ-ಲ-ರ್-ಆ-ರಾ-ಧ-ನೆಯ
ಜನ-ರನ್ನು
ಜನ-ರ-ನ್ನು-ದ್ದೇ-ಶಿಸಿ
ಜನ-ರನ್ನೂ
ಜನ-ರ-ಮು-ಖ್ಯ-ವಾಗಿ
ಜನ-ರಲ್ನ
ಜನ-ರ-ಲ್ಲಾ-ದರೆ
ಜನ-ರಲ್ಲಿ
ಜನ-ರ-ಲ್ಲಿತ್ತು
ಜನ-ರ-ಲ್ಲಿದೆ
ಜನ-ರಷ್ಟು
ಜನ-ರಾ-ದರೂ
ಜನ-ರಿಂದ
ಜನ-ರಿ-ಗಾಗಿ
ಜನ-ರಿ-ಗಿಂ-ತಲೂ
ಜನ-ರಿಗೂ
ಜನ-ರಿಗೆ
ಜನ-ರಿ-ಗೆ-ಅ-ದ-ರಲ್ಲೂ
ಜನ-ರಿ-ಗೆಲ್ಲ
ಜನ-ರಿ-ಗೇ-ಎಂ-ದರೆ
ಜನ-ರಿ-ದ್ದರು
ಜನ-ರಿ-ದ್ದರೂ
ಜನ-ರಿ-ದ್ದುದು
ಜನ-ರಿ-ದ್ದೇವೋ
ಜನ-ರಿ-ರುವ
ಜನ-ರಿ-ಲ್ಲ-ದಿ-ದ್ದರೆ
ಜನರು
ಜನ-ರು-ವೈ-ದ್ಯರು
ಜನರೂ
ಜನ-ರೆ-ಡೆ-ಗೆ-ಜ-ನರ
ಜನ-ರೆಲ್ಲ
ಜನ-ರೆ-ಲ್ಲರೂ
ಜನ-ರೇನೋ
ಜನ-ರೊಂ-ದಿಗೂ
ಜನ-ರೊಂ-ದಿಗೆ
ಜನರೋ
ಜನ-ವರಿ
ಜನ-ವ-ರಿ-ಫೆ-ಬ್ರು-ವ-ರಿಯ
ಜನ-ವ-ರಿಯ
ಜನ-ವ-ರಿ-ಯಲ್ಲಿ
ಜನ-ವ-ರ್ಗದ
ಜನ-ವ-ರ್ಗ-ವನ್ನು
ಜನ-ಸಂ-ಖ್ಯೆ-ಯಲ್ಲಿ
ಜನ-ಸಂ-ದಣಿ
ಜನ-ಸಂ-ದ-ಣಿ-ಗ-ಲಭೆ
ಜನ-ಸಂ-ದ-ಣಿಗೆ
ಜನ-ಸಂ-ದ-ಣಿ-ಯನ್ನೂ
ಜನ-ಸ-ಮು-ದಾ-ಯ-ವನ್ನು
ಜನ-ಸ-ಮೂಹ
ಜನ-ಸ-ಮೂ-ಹವು
ಜನ-ಸಾ-ಗ-ರ-ವನ್ನು
ಜನ-ಸಾ-ಮಾನ್ಯ
ಜನ-ಸಾ-ಮಾ-ನ್ಯರ
ಜನ-ಸಾ-ಮಾ-ನ್ಯ-ರನ್ನು
ಜನ-ಸಾ-ಮಾ-ನ್ಯ-ರಿ-ಗಾಗಿ
ಜನ-ಸಾ-ಮಾ-ನ್ಯ-ರಿಗೆ
ಜನ-ಸಾ-ಮಾ-ನ್ಯ-ರಿ-ಗೆಲ್ಲ
ಜನ-ಸಾ-ಮಾ-ನ್ಯರು
ಜನ-ಸ್ತೋ-ಮದ
ಜನ-ಸ್ತೋ-ಮ-ವನ್ನು
ಜನ-ಸ್ತೋ-ಮವು
ಜನ-ಸ್ತೋ-ಮವೇ
ಜನ-ಸ್ತ್ರೀ-ಯರು
ಜನಾಂಗ
ಜನಾಂ-ಗ-ಗಳ
ಜನಾಂ-ಗ-ಗ-ಳಂತೆ
ಜನಾಂ-ಗ-ಗಳನ್ನು
ಜನಾಂ-ಗ-ಗ-ಳಿ-ಗಿಂ-ತಲೂ
ಜನಾಂ-ಗ-ಗ-ಳಿಗೆ
ಜನಾಂ-ಗ-ಗಳು
ಜನಾಂ-ಗದ
ಜನಾಂ-ಗ-ದ-ವರ
ಜನಾಂ-ಗ-ಭೇ-ದ-ವ-ರ್ಣ-ಭೇ-ದ-ಗಳ
ಜನಾಂ-ಗ-ವನ್ನು
ಜನಾಂ-ಗವು
ಜನಾಂ-ಗ-ವೆಂದೂ
ಜನಾಂ-ಗವೇ
ಜನಾ-ದ-ರಣೆ
ಜನಾ-ನು-ರಾ-ಗದ
ಜನಾ-ರ-ಣ್ಯದ
ಜನಿಕ
ಜನಿಸಿ
ಜನಿ-ಸಿತ್ತು
ಜನಿ-ಸಿ-ದರೇ
ಜನಿ-ಸಿ-ದ್ದ-ರಿಂದ
ಜನಿ-ಸಿ-ದ್ದಲ್ಲ
ಜನ್ಮ
ಜನ್ಮ-ಗಳ
ಜನ್ಮ-ಜ-ನ್ಮ-ಗ-ಳಲ್ಲೂ
ಜನ್ಮ-ತ-ಳೆದ
ಜನ್ಮ-ತ-ಳೆದು
ಜನ್ಮ-ತಾ-ಳಿ-ದ್ದಲ್ಲ
ಜನ್ಮದ
ಜನ್ಮ-ದಲ್ಲಿ
ಜನ್ಮ-ದ-ಳೆದು
ಜನ್ಮ-ದಿನ
ಜನ್ಮ-ದಿ-ನದ
ಜನ್ಮ-ದಿ-ನೋ-ತ್ಸ-ವ-ದಂದು
ಜನ್ಮ-ವನ್ನೇ
ಜನ್ಮ-ವಿತ್ತ
ಜನ್ಮ-ವೃ-ತ್ತಾಂ-ಗಳನ್ನು
ಜನ್ಮ-ವೆತ್ತಿ
ಜನ್ಮ-ವೆ-ತ್ತಿ-ದನೋ
ಜನ್ಮ-ವೆ-ತ್ತಿ-ದರೇ
ಜನ್ಮ-ವೆ-ತ್ತಿ-ದ-ವ-ರ-ಲ್ಲವೆ
ಜನ್ಮ-ವೆ-ತ್ತಿ-ದ-ವರು
ಜನ್ಮ-ವೆ-ತ್ತಿ-ದ-ವ-ರೆಂ-ಬು-ದ-ರಲ್ಲಿ
ಜನ್ಮ-ವೆ-ತ್ತು-ತ್ತಾರೆ
ಜನ್ಮ-ಸ್ಥಳ
ಜಪ
ಜಪ-ಧ್ಯಾ-ನ-ಗಳನ್ನು
ಜಪಾ-ನಿಗೆ
ಜಪಾ-ನಿನ
ಜಪಾ-ನಿ-ನಿಂದ
ಜಪಾ-ನೀ-ಯರ
ಜಪಾ-ನೀ-ಯ-ರನ್ನು
ಜಪಾ-ನೀ-ಯರು
ಜಪಾ-ನು-ಗ-ಳಿಗೆ
ಜಪಾನ್
ಜಪಿ-ಸುತ್ತ
ಜಬ
ಜಬ-ರ-ದಸ್ತು
ಜಮ-ಖಾ-ನ-ವನ್ನು
ಜಮ-ಖಾ-ನೆಯ
ಜಮೀ-ನಿ-ದ್ದರೂ
ಜಮೀ-ನ್ದಾರ
ಜಮೀ-ನ್ದಾ-ರ-ರಾದ
ಜಮೀ-ನ್ದಾ-ರರು
ಜಯ
ಜಯ-ಕಾರ
ಜಯ-ಕಾ-ರ-ಗ-ಳೆಲ್ಲ
ಜಯ-ಗಾ-ಥವ
ಜಯ-ಘೋಷ
ಜಯ-ಘೋ-ಷದ
ಜಯ-ನ-ಗರ
ಜಯ-ಭೇರಿ
ಜಯ-ವಾ-ಗಲಿ
ಜಯ-ವೆನ್ನಿ
ಜಯ-ಶಾ-ಲಿ-ಗ-ಳಾ-ಗ-ಬೇ-ಕಾ-ಗಿದೆ
ಜಯ-ಶಾ-ಲಿ-ಗ-ಳಾ-ಗು-ತ್ತೇವೆ
ಜಯಿ-ಸ-ಬ-ಹುದು
ಜಯಿ-ಸ-ಬೇ-ಕಾ-ದ-ದ್ದನ್ನು
ಜಯಿ-ಸಲು
ಜಯಿಸಿ
ಜಯಿಸು
ಜರ-ತುಷ್ಟ್ರ
ಜರ-ತು-ಷ್ಟ್ರೀ-ಯರ
ಜರಾ-ಮ-ರ-ಣ-ರೋ-ಗ-ಗಳಿಂದ
ಜರು-ಗ-ಲಿ-ಲ್ಲ-ವೆಂದೇ
ಜರು-ಗಿತು
ಜರು-ಗಿ-ದುದು
ಜರು-ಗಿ-ದುವು
ಜರು-ಗು-ತ್ತದೆ
ಜರೆ-ದರು
ಜರ್ಜರ
ಜರ್ಜ-ರಿ-ತ-ವಾ-ಗಿತ್ತು
ಜರ್ಜ-ರಿ-ತ-ವಾ-ಗಿದ್ದ
ಜರ್ನಲ್
ಜರ್ಮ-ನರು
ಜರ್ಮ-ನರೇ
ಜರ್ಮ-ನಿಯ
ಜರ್ಮನ್
ಜರ್ಸಿ
ಜಲ-ಪಾ-ತ-ಗಳ
ಜಲ-ಪಾ-ತ-ಗಳೂ
ಜಲ-ಪಾ-ತ-ವನ್ನು
ಜಲ-ರಾಶಿ
ಜಲ-ರಾ-ಶಿ-ಯನ್ನು
ಜಲ-ವೊಂ-ದಿ-ರಲು
ಜಲ-ಸಂ-ಧಿ-ಯನ್ನು
ಜಲಾ-ಶಯ
ಜಲೇ
ಜಳ್ಳನ್ನು
ಜಳ್ಳು
ಜವಾ-ಬು-ದಾರಿ-ಯನ್ನು
ಜವಾ-ಬ್ದಾ-ರ-ರ-ಲ್ಲ-ವೆಂ-ದಾ-ಯಿತು
ಜವಾ-ಬ್ದಾರಿ
ಜವಾ-ಬ್ದಾ-ರಿ-ಗಳ
ಜವಾ-ಬ್ದಾ-ರಿ-ಯನ್ನು
ಜವಾ-ಬ್ದಾ-ರಿ-ಯುತ
ಜಸ-ವಂತ
ಜಸ-ವಂ-ತ-ಸಿಂ-ಗನ
ಜಸ-ವಂ-ತ-ಸಿಂ-ಗ-ನಿಗೆ
ಜಸ-ವಂ-ತ-ಸಿಂಗ್
ಜಾಗ
ಜಾಗಕ್ಕೆ
ಜಾಗ-ತಿಕ
ಜಾಗ-ದಲ್ಲಿ
ಜಾಗ-ದ-ಲ್ಲಿ-ರ-ಬೇ-ಕಾ-ದದ್ದು
ಜಾಗ-ರೂ-ಕ-ತೆ-ಯನ್ನೂ
ಜಾಗ-ರೂ-ಕ-ರಾ-ಗಿ-ದ್ದರು
ಜಾಗ-ವನ್ನು
ಜಾಗ-ವಿಲ್ಲ
ಜಾಗ-ವೊಂದು
ಜಾಗೃತ
ಜಾಗೃ-ತ-ಗೊ-ಳಿ-ಸ-ಬಲ್ಲ
ಜಾಗೃ-ತ-ಗೊ-ಳಿ-ಸಲು
ಜಾಗೃ-ತ-ಗೊ-ಳಿಸಿ
ಜಾಗೃ-ತ-ಗೊ-ಳಿ-ಸಿ-ಕೊ-ಳ್ಳುವ
ಜಾಗೃ-ತ-ಗೊ-ಳಿ-ಸು-ತ್ತಿ-ದ್ದಾನೆ
ಜಾಗೃ-ತ-ಗೊ-ಳಿ-ಸುವ
ಜಾಗೃ-ತ-ಗೊ-ಳಿ-ಸು-ವು-ದ-ಕ್ಕಾ-ಗಿಯೇ
ಜಾಗೃ-ತ-ಗೊ-ಳ್ಳಲು
ಜಾಗೃ-ತ-ಗೊ-ಳ್ಳು-ತ್ತವೆ
ಜಾಗೃ-ತ-ನಾದ
ಜಾಗೃ-ತ-ವಾ-ಗ-ಬೇ-ಕೆಂ-ಬುದು
ಜಾಗೃ-ತ-ವಾಗಿ
ಜಾಗೃ-ತ-ವಾ-ಗಿತ್ತು
ಜಾಗೃ-ತ-ವಾ-ಗಿ-ದೆ-ಯೆಂ-ಬು-ದನ್ನು
ಜಾಗೃ-ತ-ವಾ-ಗು-ತ್ತದೆ
ಜಾಗೃ-ತ-ವಾ-ಗು-ತ್ತವೆ
ಜಾಗೃ-ತ-ವಾ-ಯಿತು
ಜಾಗೃ-ತಾ-ವ-ಸ್ಥೆ-ಯ-ಲ್ಲಿ-ದ್ದಾರೆ
ಜಾಗೃ-ತಿಯ
ಜಾಗೃ-ತಿ-ಯ-ನ್ನುಂ-ಟು-ಮಾ-ಡ-ಬೇ-ಕೆಂಬ
ಜಾಗೃ-ತಿ-ಯುಂ-ಟಾ-ಗು-ತ್ತ-ದೆ-ಯೆಂ-ದರೆ
ಜಾಗೃ-ತಿ-ಯುಂ-ಟು-ಮಾ-ಡು-ವುದು
ಜಾಗ್ರತ
ಜಾಜ್ನ್
ಜಾಜ್ವ-ಲ್ಯ-ಮಾನ
ಜಾಜ್ವ-ಲ್ಯ-ಮಾ-ನ-ವಾ-ಯಿತು
ಜಾಡ-ಮಾ-ಲಿ-ಗಳ
ಜಾಡ-ಮಾ-ಲಿಯ
ಜಾಡಿ
ಜಾಡಿ-ಗಳ
ಜಾಡಿ-ಗಳನ್ನೆಲ್ಲ
ಜಾಡಿ-ಗ-ಳಿವೆ
ಜಾಡಿ-ಗ-ಳೆಲ್ಲ
ಜಾಡಿಗೆ
ಜಾಡಿಯ
ಜಾಡಿ-ಯ-ನ್ನಿ-ಟ್ಟು-ಕೊಂ-ಡ-ವ-ರಿ-ಗೆಲ್ಲ
ಜಾಡ್ಯ
ಜಾತಿ
ಜಾತಿ-ಕು-ಲ-ಐ-ಶ್ವ-ರ್ಯ-ಲಿಂ-ಗದ
ಜಾತಿ-ಮತ
ಜಾತಿ-ಮ-ತ-ಗಳ
ಜಾತಿ-ಗಳ
ಜಾತಿಗೆ
ಜಾತಿ-ಪ-ದ್ಧತಿ
ಜಾತಿ-ಪ-ದ್ಧ-ತಿ-ಯನ್ನು
ಜಾತಿ-ಪ-ದ್ಧ-ತಿ-ಯನ್ನೂ
ಜಾತಿ-ಭಾ-ವ-ನೆ-ಯನ್ನು
ಜಾತಿ-ಭೇದ
ಜಾತಿ-ಭೇ-ದದ
ಜಾತಿ-ಮ-ತ-ಗಳ
ಜಾತಿ-ಮ-ತ-ಗ-ಳಿಗೆ
ಜಾತಿಯ
ಜಾತಿ-ಯ-ವನೇ
ಜಾತಿ-ಯ-ವ-ನೊಬ್ಬ
ಜಾತಿ-ಯ-ವರು
ಜಾತಿ-ಯಿಂದ
ಜಾತಿ-ಯೆಂ-ಬುದು
ಜಾತೀ-ಯತೆ
ಜಾತೀ-ಯ-ತೆಯ
ಜಾತೀ-ಯ-ತೆಯೇ
ಜಾತ್ಯಂ-ಧ-ರಾದ
ಜಾನ್
ಜಾನ್ಸನ್
ಜಾರ-ದಂತೆ
ಜಾರಿಗೆ
ಜಾರಿತ್ತು
ಜಾರಿ-ದ-ರಾ-ದರೂ
ಜಾರುತ್ತ
ಜಾರುವ
ಜಾರು-ವಂ-ತಿ-ದ್ದು-ದ-ರಿಂದ
ಜಾರ್ಜ್
ಜಾರ್ಜ್ರ-ವರ
ಜಾಲ-ದಿಂ-ದಈ
ಜಾವ
ಜಾಹೀ-ರಾ-ತಾಗಿ
ಜಾಹೀ-ರಾ-ತಿನ
ಜಾಹೀ-ರಾತು
ಜಿ
ಜಿಂಕೆ-ಯನ್ನು
ಜಿಂಕೆ-ಯೆಂದು
ಜಿಗಿ-ಯು-ತ್ತಿ-ರು-ತ್ತಾಳೆ
ಜಿಜ್ಞಾ-ಸು-ಗಳ
ಜಿಜ್ಞಾ-ಸು-ಗ-ಳಾದ
ಜಿಜ್ಞಾ-ಸೆ-ಗಳನ್ನು
ಜಿಜ್ಞಾ-ಸೆ-ಯಿಂ-ದಾಗಿ
ಜಿಜ್ಞಾ-ಸೆಯೇ
ಜಿನ-ಮಾತಾ
ಜಿನೀವಾ
ಜಿನೀ-ವಾದ
ಜಿನೀ-ವಾ-ದಲ್ಲಿ
ಜಿನೀ-ವಾ-ದಿಂದ
ಜಿನೀ-ವಾ-ವ-ರೆಗೆ
ಜಿಪು-ಣರು
ಜಿಲ್ಲಾ
ಜೀರ್ಣ-ವಾದ
ಜೀರ್ಣ-ವಾ-ದೀತೆ
ಜೀರ್ಣಿಸಿ
ಜೀರ್ಣಿ-ಸಿ-ಕೊ-ಳ್ಳು-ವಲ್ಲಿ
ಜೀವ
ಜೀವ-ಈ-ಶ್ವರ
ಜೀವ-ಬ್ರ-ಹ್ಮ-ಗ-ಳೊ-ಳಗೆ
ಜೀವಂತ
ಜೀವಂ-ತ-ವಾ-ಗ-ಬೇಕು
ಜೀವಂ-ತ-ವಾಗಿ
ಜೀವಂ-ತ-ವಾ-ಗಿ-ದೆ-ಯೆಂಬ
ಜೀವಂ-ತ-ವಾ-ಗಿದ್ದು
ಜೀವಂ-ತ-ವಾ-ಗಿ-ದ್ದುವು
ಜೀವಂ-ತ-ವಾ-ಗಿ-ರಿ-ಸು-ತ್ತದೆ
ಜೀವ-ಕೋ-ಟಿ-ಯನ್ನೂ
ಜೀವಕ್ಕೆ
ಜೀವ-ಗಳ
ಜೀವದ
ಜೀವ-ದಾ-ಳ-ಕ್ಕಿ-ಳಿದು
ಜೀವನ
ಜೀವ-ನ
ಜೀವ-ನ-ಧ್ಯೇ-ಯ-ವ-ನ್ನಾಗಿ
ಜೀವ-ನ-ಬೋ-ಧ-ನೆ-ಗಳ
ಜೀವ-ನ-ಸಂ-ದೇ-ಶ-ಗಳ
ಜೀವ-ನ-ಕಥೆ
ಜೀವ-ನ-ಕ-ರ್ತ-ವ್ಯ-ಗಳನ್ನು
ಜೀವ-ನ-ಕ್ಕಿಂತ
ಜೀವ-ನಕ್ಕೂ
ಜೀವ-ನಕ್ಕೆ
ಜೀವ-ನ-ಕ್ರಮ
ಜೀವ-ನ-ಕ್ರ-ಮ-ದಲ್ಲೂ
ಜೀವ-ನ-ಕ್ರ-ಮ-ವನ್ನು
ಜೀವ-ನ-ಕ್ರ-ಮವೇ
ಜೀವ-ನ-ಗಂಗಾ
ಜೀವ-ನ-ಗಾ-ಥೆಯ
ಜೀವ-ನ-ಚ-ರಿತ್ರೆ
ಜೀವ-ನ-ಚ-ರಿ-ತ್ರೆ-ಯನ್ನು
ಜೀವ-ನ-ತ-ತ್ತ್ವದ
ಜೀವ-ನದ
ಜೀವ-ನ-ದತ್ತ
ಜೀವ-ನ-ದಲ್ಲಿ
ಜೀವ-ನ-ದ-ಲ್ಲಿ-ಅ-ಷ್ಟೇಕೆ
ಜೀವ-ನ-ದಲ್ಲೆಲ್ಲ
ಜೀವ-ನ-ದಲ್ಲೇ
ಜೀವ-ನ-ದಿಂದ
ಜೀವ-ನ-ದಿ-ಯನ್ನು
ಜೀವ-ನ-ದು-ದ್ದಕ್ಕೂ
ಜೀವ-ನ-ದೃ-ಷ್ಟಿ-ಗಳೇ
ಜೀವ-ನ-ದೆ-ಡೆಗೆ
ಜೀವ-ನ-ಪ-ರ್ಯಂತ
ಜೀವ-ನ-ಮ-ಟ್ಟ-ವನ್ನು
ಜೀವ-ನ-ಮ-ತ್ತೇನೂ
ಜೀವ-ನ-ಮೌ-ಲ್ಯ-ಗಳು
ಜೀವ-ನ-ಯಾ-ಪನೆ
ಜೀವ-ನ-ಯಾ-ಪ-ನೆ-ಗಾಗಿ
ಜೀವ-ನ-ವನ್ನ
ಜೀವ-ನ-ವನ್ನು
ಜೀವ-ನ-ವನ್ನೇ
ಜೀವ-ನ-ವಿಡೀ
ಜೀವ-ನವು
ಜೀವ-ನವೂ
ಜೀವ-ನ-ವೆಂದರೆ
ಜೀವ-ನ-ವೆಂ-ದ-ರೇನು
ಜೀವ-ನ-ವೆಂ-ಬುದು
ಜೀವ-ನ-ವೆಲ್ಲ
ಜೀವ-ನವೇ
ಜೀವ-ನಾಡಿ
ಜೀವ-ನಾ-ಡಿಯ
ಜೀವ-ನಾ-ಡಿ-ಯನ್ನು
ಜೀವ-ನಾ-ಡಿ-ಯಾದ
ಜೀವ-ನಾ-ದರ್ಶ
ಜೀವ-ನಾ-ದ-ರ್ಶಕ್ಕೆ
ಜೀವ-ನಾ-ದ-ರ್ಶ-ಗಳನ್ನು
ಜೀವ-ನಾ-ದ-ರ್ಶ-ಗ-ಳಾದ
ಜೀವ-ನಾ-ದ-ರ್ಶ-ವನ್ನು
ಜೀವ-ನಾ-ದ-ರ್ಶವೇ
ಜೀವ-ನಾ-ಧಾ-ರಕ್ಕೆ
ಜೀವ-ನಾ-ವ-ಧಿಯ
ಜೀವ-ನಾ-ವ-ಶ್ಯ-ಕ-ತೆ-ಗಳ
ಜೀವನೋ
ಜೀವ-ನೋ-ದ್ದೇಶ
ಜೀವ-ನೋ-ದ್ದೇ-ಶ-ಗಳ
ಜೀವ-ನೋ-ದ್ದೇ-ಶದ
ಜೀವ-ನೋ-ದ್ದೇ-ಶ-ವ-ನ್ನಿನ್ನೂ
ಜೀವ-ನೋ-ದ್ದೇ-ಶ-ವೆಂದು
ಜೀವ-ನೋ-ಪಾ-ಯ-ಕ್ಕಾಗಿ
ಜೀವ-ನ್ಮ-ರ-ಣ-ಗಳ
ಜೀವ-ನ್ಮ-ರ-ಣ-ಗಳನ್ನು
ಜೀವ-ನ್ಮ-ರ-ಣ-ಗಳನ್ನೂ
ಜೀವ-ನ್ಮು-ಕ್ತ-ರಾದ
ಜೀವ-ಮಾನ
ಜೀವ-ಮಾ-ನ-ದ-ಲ್ಲಾದ
ಜೀವ-ಮಾ-ನ-ವಿಡೀ
ಜೀವ-ರನ್ನು
ಜೀವ-ವನ್ನು
ಜೀವ-ವನ್ನೇ
ಜೀವ-ವೆಂದು
ಜೀವವೇ
ಜೀವ-ಶಾಸ್ತ್ರ
ಜೀವ-ಸ-ಹಿತ
ಜೀವಿ-ಗಳೂ
ಜೀವಿ-ತವು
ಜೀವಿ-ತಾ-ವಧಿ
ಜೀವಿ-ತಾ-ವ-ಧಿ-ಯಲ್ಲಿ
ಜೀವಿ-ತಾ-ವ-ಧಿ-ಯಲ್ಲೇ
ಜೀವಿ-ತಾ-ವ-ಧಿ-ಯ-ಲ್ಲೇ-ಹ-ತ್ತೊಂ-ಬ-ತ್ತನೇ
ಜೀವಿಯ
ಜೀವಿ-ಯ-ಲ್ಲೂ-ಇ-ರು-ವುದು
ಜೀವಿಯೂ
ಜೀವಿ-ಸಲು
ಜೀವಿಸಿ
ಜೀವಿ-ಸಿ-ದರೂ
ಜೀವಿ-ಸಿ-ದ್ದೆವೋ
ಜೀವಿ-ಸಿ-ದ್ದೇ-ವೆಯೋ
ಜೀವಿ-ಸಿ-ರು-ವ-ವ-ರೆಗೂ
ಜೀವಿ-ಸಿ-ರು-ವು-ದ-ರಲ್ಲೇ
ಜೀವಿಸು
ಜೀವಿ-ಸುತ್ತ
ಜೀವಿ-ಸು-ತ್ತಿ-ದ್ದಾರೆ
ಜೀವಿ-ಸು-ತ್ತಿ-ರು-ವಾಗ
ಜೀವಿ-ಸು-ತ್ತಿ-ರು-ವುದನ್ನು
ಜೀವಿ-ಸುವ
ಜೀವಿ-ಸು-ವು-ದ-ಕ್ಕಿಂತ
ಜುಗುಪ್ಸೆ
ಜುಗು-ಪ್ಸೆ-ಗೊಂಡು
ಜುಗು-ಪ್ಸೆ-ಗೊ-ಳಿ-ಸು-ವಷ್ಟು
ಜುಗು-ಪ್ಸೆಯ
ಜುಗು-ಪ್ಸೆ-ಯಾಗಿ
ಜುಗು-ಪ್ಸೆ-ಯಾ-ಗಿ-ದೆ-ಯಂತೆ
ಜುಗು-ಪ್ಸೆ-ಯಾ-ಗಿ-ಬಿ-ಟ್ಟಿದೆ
ಜುಗು-ಪ್ಸೆ-ಯಾ-ಗಿ-ರ-ಲಿ-ದೈ-ವೇ-ಚ್ಛೆ-ಯೆಂ-ಬುದು
ಜುಗು-ಪ್ಸೆ-ಯಾ-ಗು-ತ್ತಿದೆ
ಜುಗು-ಪ್ಸೆ-ಯಾ-ಯಿತು
ಜುಗು-ಪ್ಸೆ-ಯಿತ್ತು
ಜುಗು-ಪ್ಸೆ-ಯುಂ-ಟಾ-ಯಿತು
ಜುನಾ-ಗಢ
ಜುನಾ-ಗ-ಢಕ್ಕೆ
ಜುನಾ-ಗ-ಢದ
ಜುನಾ-ಗ-ಢ-ದಲ್ಲಿ
ಜುನಾ-ಗ-ಢ-ವನ್ನು
ಜುಲೈ
ಜುಲೈ-ನಲ್ಲಿ
ಜುಳು-ಜುಳು
ಜುಷಾಂ
ಜೂನ್
ಜೂಲಿಯಾ
ಜೆ
ಜೆಮ್ಶೆಟ್ಜಿ
ಜೆಹೋ-ವನೊ
ಜೇ
ಜೇನು
ಜೇನ್ಸ್
ಜೇನ್ಸ್ರ
ಜೇನ್ಸ್ರ-ವರು
ಜೇಬಿ-ಗಿಳಿ-ಸು-ತ್ತಾನೆ
ಜೇಬಿಗೆ
ಜೇಬಿ-ನ-ಲ್ಲಿದ್ದ
ಜೇಬಿ-ನಿಂದ
ಜೇಮ್ಸರ
ಜೇಮ್ಸ-ರನ್ನು
ಜೇಮ್ಸರು
ಜೇಮ್ಸರೇ
ಜೇಮ್ಸ್
ಜೇಮ್ಸ್ರನ್ನು
ಜೇಮ್ಸ್ರ-ವರ
ಜೈ
ಜೈಕಾ-ರ-ಹ-ರ್ಷೋ-ದ್ಗಾ-ರ-ಗಳ
ಜೈತ್ರ-ಯಾ-ತ್ರೆಗೆ
ಜೈನ
ಜೈನ-ಧ-ರ್ಮ-ಗಳೂ
ಜೈನ-ಧ-ರ್ಮದ
ಜೈನರ
ಜೈಪು-ರಕ್ಕೆ
ಜೈಪು-ರ-ದತ್ತ
ಜೈಪು-ರ-ದಲ್ಲಿ
ಜೈಪು-ರ-ದ-ಲ್ಲಿ-ದ್ದಾಗ
ಜೈಪು-ರ-ದ-ವ-ರೆಗೂ
ಜೈಪು-ರ-ದ-ವ-ರೆಗೆ
ಜೈಪು-ರ-ದಿಂದ
ಜೈಪು-ರ-ವನ್ನು
ಜೈಲಿಗೆ
ಜೈಲಿನ
ಜೈಲಿ-ನ-ಲ್ಲಾ-ದರೆ
ಜೈಸಿಂಗ್
ಜೊತೆ
ಜೊತೆ-ಕೊ-ಡು-ತ್ತಿ-ದ್ದರು
ಜೊತೆ-ಗಾತಿ
ಜೊತೆ-ಗಾದ
ಜೊತೆ-ಗಾ-ರ-ರನ್ನು
ಜೊತೆ-ಗಾ-ರರು
ಜೊತೆ-ಗಿದ್ದ
ಜೊತೆಗೂ
ಜೊತೆ-ಗೂಡಿ
ಜೊತೆ-ಗೂ-ಡಿ-ಕೊಂಡು
ಜೊತೆ-ಗೂ-ಡಿದ
ಜೊತೆಗೆ
ಜೊತೆಗೇ
ಜೊತೆ-ಗೊ-ಡು-ತ್ತಿದ್ದ
ಜೊತೆ-ಜೊ-ತೆಗೇ
ಜೊತೆ-ಯಲ್ಲಿ
ಜೊತೆ-ಯ-ಲ್ಲಿ-ದ್ದ-ವ-ರಿಗೂ
ಜೊತೆ-ಯ-ಲ್ಲಿ-ದ್ದ-ವರು
ಜೊತೆ-ಯ-ಲ್ಲಿಯೇ
ಜೊತೆ-ಯ-ಲ್ಲಿ-ಲ್ಲ-ದಿ-ದ್ದರೆ
ಜೊತೆ-ಯಲ್ಲೂ
ಜೊತೆ-ಯಲ್ಲೇ
ಜೊತೆ-ಯಾಗಿ
ಜೊಸೆ-ಫಿನ್
ಜೊಸೈಯಾ
ಜೋ
ಜೋಜೋ
ಜೋಡಣೆ
ಜೋಡಿಸಿ
ಜೋಡಿ-ಸಿ-ಕೊಂ-ಡಿ-ದ್ದರು
ಜೋಡಿ-ಸಿ-ರ-ಬೇಕು
ಜೋಡಿ-ಸು-ವುದು
ಜೋತಾ-ಡು-ತ್ತಿ-ರು-ತ್ತವೆ
ಜೋತು
ಜೋಪ-ಡಿ-ಗಳಲ್ಲಿ
ಜೋಪ-ಡಿ-ಯಲ್ಲಿ
ಜೋಪಾ-ನ-ವಾಗಿ
ಜೋಪಾ-ನ-ವಾ-ಗಿ-ದ್ದರೂ
ಜೋರಾಗಿ
ಜೋರಾ-ಗಿ-ತ್ತೆಂ-ದರೆ
ಜೋರಾ-ಗಿಯೇ
ಜೋಲು
ಜೋಶಿ
ಜೋಶಿ-ಯ-ವರ
ಜೋಶಿ-ಯ-ವರು
ಜೋಸೆ-ಫಿನ್
ಜ್ಞಾನ
ಜ್ಞಾನ
ಜ್ಞಾನ-ಭ-ಕ್ತಿ-ಗಳಲ್ಲಿ
ಜ್ಞಾನಕ್ಕೂ
ಜ್ಞಾನಕ್ಕೆ
ಜ್ಞಾನದ
ಜ್ಞಾನ-ದಾ-ಳ-ವನ್ನು
ಜ್ಞಾನ-ದಾಹ
ಜ್ಞಾನ-ದಿಂ-ದಲೂ
ಜ್ಞಾನ-ಪ್ರದ
ಜ್ಞಾನ-ಭಂ-ಡಾ-ರ-ವನ್ನು
ಜ್ಞಾನ-ಮೊ-ದಲು
ಜ್ಞಾನ-ಯೋಗ
ಜ್ಞಾನ-ಯೋ-ಗಕ್ಕೆ
ಜ್ಞಾನ-ಯೋ-ಗ-ಗಳ
ಜ್ಞಾನ-ಯೋ-ಗದ
ಜ್ಞಾನ-ವ-ನ್ನ-ವರು
ಜ್ಞಾನ-ವನ್ನು
ಜ್ಞಾನ-ವ-ಲ್ಲದೆ
ಜ್ಞಾನ-ವಾ-ಗಲಿ
ಜ್ಞಾನ-ವಿ-ರು-ವ-ವ-ರ-ಲ್ಲವೆ
ಜ್ಞಾನ-ವಿ-ರು-ವುದನ್ನು
ಜ್ಞಾನವು
ಜ್ಞಾನ-ವೃ-ದ್ಧನ
ಜ್ಞಾನ-ವೃ-ದ್ಧರು
ಜ್ಞಾನ-ವೆ-ನ್ನು-ವು-ದಾ-ದರೆ
ಜ್ಞಾನ-ಶಾ-ಖೆ-ಗಳನ್ನು
ಜ್ಞಾನ-ಸಾ-ಗ-ರ-ವನ್ನು
ಜ್ಞಾನಿ
ಜ್ಞಾನಿ-ಗ-ಳಾ-ದ-ವರು
ಜ್ಞಾನಿ-ಗಳು
ಜ್ಞಾನಿ-ಯೆಂದೂ
ಜ್ಞಾನೋ-ದಯ
ಜ್ಞಾನೋ-ದ-ಯ-ವಾಗಿ
ಜ್ಞಾಪ-ಕ-ಶ-ಕ್ತಿಯ
ಜ್ಞಾಪ-ಕ-ಶ-ಕ್ತಿ-ಯನ್ನು
ಜ್ಯೋತಿ
ಜ್ಯೋತಿ-ಕಿ-ರ-ಣ-ವಾ-ಗಿದೆ
ಜ್ಯೋತಿ-ಯಂತೆ
ಜ್ಯೋತಿ-ಯನ್ನು
ಜ್ಯೋತಿ-ರ್ಮಯ
ಜ್ಯೋತಿ-ರ್ವ-ಲ-ಯ-ದ-ಲ್ಲಿ-ದ್ದೇ-ವೆ-ಯೆಂ-ಬುದು
ಜ್ವರ
ಜ್ವರ-ದಿಂದ
ಜ್ವಲಂತ
ಜ್ವಲಂ-ತ-ವಾ-ಗಿರು
ಜ್ವಾಲಾ-ಮುಖಿ
ಜ್ವಾಲಾ-ಮು-ಖಿ-ಯಂತೆ
ಜ್ವಾಲೆ
ಜ್ವಾಲೆಯ
ಜ್ವಾಲೆ-ಯಾ-ಗಿತ್ತು
ಝಗರೋ
ಝರಿ-ಗ-ಳು-ಇ-ವು-ಗಳಿಂದ
ಝಳ-ಪಿ-ನಂತೆ
ಝಳ-ಪಿ-ಸುತ್ತ
ಝಾ
ಝಾರ-ವರು
ಟನ್
ಟನ್ನಲ್ಲಿ
ಟನ್ನಾಟ್
ಟರು
ಟಾಗ
ಟಾಟಾ-ರ-ವ-ರಿಗೆ
ಟಾಟಾ-ರ-ವರು
ಟಾಟಾ-ರ-ವರೂ
ಟಾನಿ-ಯ-ವರು
ಟಿ
ಟಿಕೆ-ಟನ್ನು
ಟಿಕೆಟ್
ಟಿಕೆ-ಟ್ಟನ್ನು
ಟಿಕೆಟ್ಟು
ಟಿಕೆ-ಟ್ಟು-ಗ-ಳೆಲ್ಲ
ಟಿಕೇ-ಟನ್ನು
ಟಿಪ್ಪಣಿ
ಟಿಪ್ಪ-ಣಿ-ಗಳ
ಟಿಪ್ಪ-ಣಿ-ಗಳನ್ನು
ಟಿಪ್ಪ-ಣಿ-ಗಳು
ಟಿಪ್ಪು-ಸು-ಲ್ತಾ-ನನು
ಟಿಬೆ-ಟಿ-ಗ-ರಲ್ಲಿ
ಟಿಬೆಟೀ
ಟಿಬೆಟ್
ಟೀಕಿ-ಸ-ದಿ-ರು-ವುದು
ಟೀಕಿ-ಸ-ಬ-ಹು-ಲ್ಲವೆ
ಟೀಕಿ-ಸಲು
ಟೀಕಿಸಿ
ಟೀಕಿ-ಸಿ-ದರು
ಟೀಕಿ-ಸಿ-ದರೂ
ಟೀಕಿ-ಸಿ-ದರೆ
ಟೀಕಿ-ಸಿ-ದು-ದಕ್ಕೂ
ಟೀಕಿ-ಸಿ-ದ್ದರು
ಟೀಕಿ-ಸಿದ್ದೂ
ಟೀಕಿ-ಸಿ-ಬಿಟ್ಟೆ
ಟೀಕಿ-ಸುತ್ತ
ಟೀಕಿ-ಸು-ತ್ತಿ-ದ್ದರು
ಟೀಕಿ-ಸು-ತ್ತಿ-ರುವ
ಟೀಕಿ-ಸುವ
ಟೀಕಿ-ಸು-ವಾ-ಗಲೂ
ಟೀಕೆ
ಟೀಕೆ-ಗಳ
ಟೀಕೆ-ಗಳನ್ನು
ಟೀಕೆ-ಗಳಿಂದ
ಟೀಕೆ-ಗ-ಳಿಗೂ
ಟೀಕೆ-ಗಳು
ಟೀಕೆ-ಗಾ-ಗಲಿ
ಟೀಕೆ-ಗಾ-ರನ
ಟೀಕೆ-ಗಾ-ರನೂ
ಟೀಕೆ-ಗಾ-ರರ
ಟೀಕೆ-ಗಾ-ರ-ರನ್ನೂ
ಟೀಕೆಯ
ಟೀಕೆ-ಯನ್ನು
ಟೀಕೆ-ಯೆಂದು
ಟೆಕೆ-ಟ್ಟು-ಗಳನ್ನು
ಟೆನ್ನಿಸ್
ಟೆನ್ನಿಸ್ಸೀ
ಟೆಲಿ-ಗ್ರಾಮ್
ಟೆಲ್ಲ-ನಿಗೆ
ಟೆಸ್ಟ್
ಟೆಸ್ಲಾ
ಟೈಪ್
ಟೊಂಕ
ಟೊಂಕ-ಕ-ಟ್ಟಿ-ನಿಂತ
ಟೋಕಿಯೊ
ಟೋಟನ್
ಟ್ಯೂಬ್ಲೈ-ಟನ್ನು
ಟ್ಯೂರಿನ್
ಟ್ರಸ್ಟ್ಗಳನ್ನು
ಟ್ರಾನ್
ಟ್ರಾಮಿನ
ಟ್ರಾಮಿ-ನಲ್ಲಿ
ಟ್ರಿಪ್ಲಿ-ಕೇ-ನಿನ
ಟ್ರಿಪ್ಲಿ-ಕೇನ್
ಟ್ರೈನಿನ
ಟ್ರೈನಿ-ನಲ್ಲಿ
ಟ್ರೈನಿ-ನಲ್ಲೇ
ಟ್ರೈನಿ-ನಿಂ-ದಿ-ಳಿದು
ಟ್ರೈನು
ಟ್ರೋಜ-ನರ
ಟ್ವೆಂಟಿ-ಯತ್
ಠಕ್ಕ
ಠಕ್ಕು-ಕ-ಪ-ಟ-ವಂ-ಚ-ನೆ-ಗಳು
ಠಕ್ಕು-ಮೂ-ಢ-ನಂ-ಬಿ-ಕೆ-ಮ-ತಾಂ-ಧ-ತೆ-ಗಳ
ಠಕ್ಕು-ವಂ-ಚ-ನೆ-ಯೇ-ನಿಲ್ಲ
ಠರಾ
ಠರಾ-ವನ್ನು
ಠರಾವು
ಠರಾ-ವು-ಗಳ
ಠರಾ-ವು-ಗಳು
ಠಾಕು-ಠೀ-ಕಾ-ಗಿ-ರ-ಬೇಕು
ಠಾಕೂ-ರರ
ಠಾಕೂರ್
ಠೀವಿ-ಯಿಂದ
ಡಕಾಯಿ
ಡಕಾ-ಯಿತ
ಡಕಾ-ಯಿ-ತರ
ಡಕಾ-ಯಿ-ತ-ರನ್ನು
ಡಕಾ-ಯಿ-ತ-ರಿ-ಗೆಲ್ಲ
ಡಕಾ-ಯಿ-ತರು
ಡಕಾ-ಯಿ-ತರೂ
ಡಕಾ-ಯಿ-ತ-ರೊಂ-ದಿ-ಗಿನ
ಡಚ-ರಳ
ಡಚರ್
ಡಚ-ರ್ಳಿಗೆ
ಡನೆ
ಡಬ್ಬಕ್ಕೆ
ಡಬ್ಬಿ-ಯಿ-ರು-ವುದನ್ನು
ಡಲು
ಡವ-ಡವ
ಡಾ
ಡಾಕೀ
ಡಾಕ್ಟರು
ಡಾಕ್ಟ-ರು-ಗಳು
ಡಾಕ್ಟರ್
ಡಾಯ-ನ್ಸ್
ಡಾಯ್ಸ
ಡಾಯ್ಸನ್
ಡಾಯ್ಸನ್ನ
ಡಾಯ್ಸ-ನ್ನರ
ಡಾಯ್ಸ-ನ್ನ-ರನ್ನು
ಡಾಯ್ಸ-ನ್ನ-ರಿಂದ
ಡಾಯ್ಸ-ನ್ನ-ರಿಗೆ
ಡಾಯ್ಸ-ನ್ನರು
ಡಾಯ್ಸ-ನ್ನ-ರೊಂ-ದಿಗೆ
ಡಾಲ-ರು-ಗಳನ್ನೂ
ಡಾಲರ್
ಡಾಲ-ರ್ಗಳನ್ನು
ಡಾಲ-ರ್ಗ-ಳಿ-ಗಿಂ-ತಲೂ
ಡಾಲರ್ಗೂ
ಡಾಲ-ರ್ನಷ್ಟು
ಡಾಲ-ರ್ವ-ರೆಗೂ
ಡಿ
ಡಿಕ್ಕಿ
ಡಿಗ್ರಿ
ಡಿಯ-ರ್ಬಾ-ರ್ನ್
ಡಿವಿ-ನಿಟಿ
ಡಿಸೆಂ-ಬ-ರಿ-ನಲ್ಲಿ
ಡಿಸೆಂ-ಬರ್
ಡಿಸೆಂ-ಬ-ರ್ನಲ್ಲಿ
ಡೆಕ್ಕನ್
ಡೆಟ್ರಾ-ಯಿ-ಟ್ಟಿ-ನ-ಲ್ಲಿ-ದ್ದಾಗ
ಡೆಟ್ರಾಯ್ಟಿ
ಡೆಟ್ರಾ-ಯ್ಟಿಗೆ
ಡೆಟ್ರಾ-ಯ್ಟಿ-ನಲ್ಲಿ
ಡೆಟ್ರಾ-ಯ್ಟಿ-ನಿಂದ
ಡೆಟ್ರಾ-ಯ್ಟ್
ಡೆಟ್ರಾಯ್ಟ್ಗೆ
ಡೆಟ್ರಾಯ್ಟ್ನ
ಡೆಟ್ರಾ-ಯ್ಟ್ನಲ್ಲಿ
ಡೆಟ್ರಾ-ಯ್ಟ್ನ-ಲ್ಲಿ-ದ್ದಾ-ಗಲೇ
ಡೆಟ್ರಾ-ಯ್ಟ್ನ-ಲ್ಲಿನ
ಡೆಟ್ರಾ-ಯ್ಟ್ನ-ಲ್ಲಿ-ರುವ
ಡೆಟ್ರಾ-ಯ್ಟ್ನಲ್ಲೂ
ಡೆಟ್ರಾ-ಯ್ಟ್ನಿಂದ
ಡೆಸ್
ಡೇ
ಡೈಲಿ
ಡೋಲಾ-ಯ-ಮಾ-ನ-ವಾ-ಯಿತು
ಡೋವರ್
ಡೋವ-ರ್ನಲ್ಲಿ
ಡ್ಯಾಮಿಗೆ
ಡ್ರಾಫ್ಟ್ನೊಂ-ದಿಗೆ
ಡ್ರೈವ್
ಢಣಾರ್
ತಂಗ-ಬೇ-ಕಾ-ಗು-ತ್ತಿತ್ತು
ತಂಗಾ-ಳಿಯ
ತಂಗಾ-ಳಿ-ಯಲ್ಲಿ
ತಂಗಿದ್ದ
ತಂಗಿ-ದ್ದರು
ತಂಗಿ-ದ್ದ-ಲ್ಲಿಗೆ
ತಂಗಿ-ಯಿ-ದ್ದಳು
ತಂಟೆ
ತಂಟೆ-ಕೋರ
ತಂಟೆಗೆ
ತಂಡ
ತಂಡ-ಗಳು
ತಂಡ-ದ-ವ-ರಿಗೆ
ತಂಡ-ದ-ವರು
ತಂಡ-ದ-ವರೂ
ತಂಡ-ವಾಗಿ
ತಂಡವೇ
ತಂಡ-ವೊಂ-ದನ್ನು
ತಂತಮ್ಮ
ತಂತಿ
ತಂತಿ-ಯನ್ನು
ತಂತಿ-ಯೊಂದು
ತಂತ್ರ
ತಂತ್ರ-ಜ್ಞ-ನನ್ನೇ
ತಂತ್ರ-ಜ್ಞ-ನೊ-ಬ್ಬ-ನನ್ನು
ತಂತ್ರ-ಜ್ಞಾ-ನದ
ತಂತ್ರ-ಜ್ಞಾ-ನ-ವನ್ನೂ
ತಂತ್ರ-ವನ್ನೂ
ತಂದ
ತಂದಂ-ತಿತ್ತು
ತಂದ-ದ್ದನ್ನು
ತಂದ-ದ್ದ-ನ್ನೆಲ್ಲ
ತಂದರು
ತಂದಿ-ಟ್ಟಿದ್ದ
ತಂದಿಟ್ಟು
ತಂದಿತು
ತಂದಿದ್ದ
ತಂದಿ-ದ್ದನು
ತಂದಿ-ದ್ದರು
ತಂದಿ-ರ-ಬೇ-ಕಾ-ಗಿತ್ತು
ತಂದಿ-ರುವ
ತಂದಿ-ರು-ವುದು
ತಂದು
ತಂದುಕೊ
ತಂದು-ಕೊಂ-ಡರು
ತಂದು-ಕೊಂಡು
ತಂದು-ಕೊಟ್ಟ
ತಂದು-ಕೊ-ಟ್ಟರು
ತಂದು-ಕೊ-ಟ್ಟಿತು
ತಂದು-ಕೊ-ಟ್ಟಿ-ದ್ದುವು
ತಂದು-ಕೊ-ಡದ
ತಂದು-ಕೊ-ಡ-ಬ-ಲ್ಲಂ-ತಹ
ತಂದು-ಕೊ-ಡ-ಬ-ಲ್ಲದೋ
ತಂದು-ಕೊ-ಡ-ಬಲ್ಲೆ
ತಂದು-ಕೊ-ಡಲು
ತಂದು-ಕೊಡು
ತಂದು-ಕೊ-ಡು-ತ್ತದೆ
ತಂದು-ಕೊ-ಡು-ತ್ತಿದ್ದ
ತಂದು-ಕೊ-ಡು-ತ್ತೇನೆ
ತಂದು-ಕೊ-ಡು-ವಂತೆ
ತಂದು-ಕೊ-ಡು-ವು-ದಿ-ಲ್ಲವೋ
ತಂದು-ಕೊ-ಡು-ವುದೇ
ತಂದು-ಕೊ-ಳ್ಳು-ವು-ದ-ರಿಂದ
ತಂದು-ಬಿ-ಟ್ಟ-ನಪ್ಪ
ತಂದೆ
ತಂದೆಗೆ
ತಂದೆಯ
ತಂದೆ-ಯಂ-ತಿ-ದ್ದರು
ತಂದೆ-ಯಂ-ತೆಯೇ
ತಂದೆ-ಯ-ವರ
ತಂದೆ-ಯಾದ
ತಂದೆಯೊ
ತಂದೆವೋ
ತಂದೊ-ಡ್ಡ-ಬ-ಹುದು
ತಂದೊ-ಡ್ಡಿ-ದರೂ
ತಂಪ-ನ್ನುಂ-ಟು-ಮಾ-ಡು-ವಂ-ಥದು
ತಂಪ-ನ್ನೆ-ರೆದು
ತಂಪಾ-ಗಿ-ರು-ವು-ದ-ರಿಂದ
ತಂಪು
ತಂಪು-ಗಾಳಿ
ತಂಬು-ರಾನ್
ತಕ್ಕ
ತಕ್ಕಂ-ತಹ
ತಕ್ಕಂ-ತಿ-ದೆಯೆ
ತಕ್ಕಂ-ತಿ-ದ್ದುವು
ತಕ್ಕಂತೆ
ತಕ್ಕಂ-ತೆಯೇ
ತಕ್ಕ-ವರು
ತಕ್ಕು-ದಾದ
ತಕ್ಷಣ
ತಕ್ಷ-ಣಕ್ಕೆ
ತಕ್ಷ-ಣದ
ತಕ್ಷ-ಣ-ದಲ್ಲೋ
ತಕ್ಷ-ಣವೇ
ತಗಡು
ತಗ-ನೆ-ದು-ರಾ-ದ-ದ್ದ-ನ್ನೆಲ್ಲ
ತಗ-ಲಿದ
ತಗ-ಲಿ-ಹಾ-ಕಿದ್ದು
ತಗ್ಗಿಸಿ
ತಗ್ಗಿ-ಸಿದ
ತಗ್ಗಿ-ಸಿ-ದರು
ತಟ-ಕ್ಕನೆ
ತಟಸ್ಥ
ತಟ್ಟದೆ
ತಟ್ಟಿ
ತಟ್ಟಿ-ಕೊ-ಳ್ಳು-ವಷ್ಟು
ತಟ್ಟಿ-ತೆಂ-ದರೆ
ತಟ್ಟಿ-ದ್ದರು
ತಟ್ಟಿ-ಹಾ-ರಿ-ಸು-ವಂತೆ
ತಡ
ತಡ-ನಿ-ಧಾನ
ತಡ-ಕಾಡಿ
ತಡ-ಕಾ-ಡು-ತ್ತಿ-ರು-ವ-ವ-ರಿಗೆ
ತಡ-ಮಾ-ಡದೆ
ತಡ-ಮಾ-ಡಿರ
ತಡ-ವ-ರಿ-ಸಿದ
ತಡ-ವಾಗಿ
ತಡ-ವಾ-ಗಿ-ಯಾ-ದರೂ
ತಡ-ವಾ-ಗಿಯೇ
ತಡ-ವಾ-ಯಿ-ತಾ-ದರೂ
ತಡ-ವಾ-ಯಿತು
ತಡೆ-ಗ-ಟ್ಟದೆ
ತಡೆ-ಗ-ಟ್ಟ-ಬಲ್ಲ
ತಡೆ-ಗಟ್ಟಿ
ತಡೆ-ಗ-ಟ್ಟು-ತ್ತಿ-ದ್ದರು
ತಡೆ-ಗ-ಟ್ಟುವ
ತಡೆ-ದಳು
ತಡೆದು
ತಡೆ-ದು-ಕೊ-ಳ್ಳ-ಬೇಕಾ
ತಡೆ-ದು-ಕೊ-ಳ್ಳ-ಬೇ-ಕಾ-ದರೆ
ತಡೆ-ದು-ಕೊ-ಳ್ಳಲು
ತಡೆ-ದು-ಕೊ-ಳ್ಳು-ವುದು
ತಡೆಯ
ತಡೆ-ಯ-ಲಾ-ಗ-ಲಿಲ್ಲ
ತಡೆ-ಯ-ಲಾ-ರದ
ತಡೆ-ಯ-ಲಾ-ರದೆ
ತಡೆ-ಯಲು
ತಡೆ-ಯುಂ-ಟಾ-ಗು-ತ್ತ-ದೆಯೋ
ತಡೆ-ಯುಂ-ಟು-ಮಾ-ಡು-ತ್ತಿ-ದ್ದರು
ತಡೆ-ಯು-ತ್ತಿ-ದ್ದರು
ತಡೆ-ಯು-ತ್ತಿ-ದ್ದರೂ
ತಡೆ-ಯು-ವ-ರಾ-ರಿ-ದ್ದಾರೆ
ತಡೆ-ಹಿ-ಡಿ-ಯ-ಲಾ-ರರು
ತಡೆ-ಹಿ-ಡಿ-ಯಲೇ
ತಣಿ-ಸ-ಲೆಂ-ದಲ್ಲ
ತಣಿ-ಸಿ-ಕೊಂ-ಡಿ-ರ-ಬ-ಹುದು
ತಣ್ಣ-ಗಾಗಿ
ತಣ್ಣ-ಗಾ-ಗಿ-ಬಿ-ಟ್ಟಿ-ದೆಯೋ
ತತ್ಕಾ-ಲ-ಕ್ಕಾಗಿ
ತತ್ಕಾ-ಲಕ್ಕೆ
ತತ್ಕಾ-ಲದ
ತತ್ತ-ರಿಸಿ
ತತ್ತ್ವ
ತತ್ತ್ವ-ಕ್ಕಾಗಿ
ತತ್ತ್ವಕ್ಕೆ
ತತ್ತ್ವ-ಗಳ
ತತ್ತ್ವ-ಗಳನ್ನು
ತತ್ತ್ವ-ಗಳನ್ನೂ
ತತ್ತ್ವ-ಗಳಿಂದ
ತತ್ತ್ವ-ಗ-ಳಿ-ಗಿಂತ
ತತ್ತ್ವ-ಗ-ಳಿಗೆ
ತತ್ತ್ವ-ಗಳು
ತತ್ತ್ವ-ಗ-ಳೆ-ಡೆಗೆ
ತತ್ತ್ವ-ಗ-ಳೆಲ್ಲ
ತತ್ತ್ವ-ಗ್ರ-ಹಣ
ತತ್ತ್ವ-ಜ್ಞಾನ
ತತ್ತ್ವ-ಜ್ಞಾ-ನ-ಗ-ಳಿ-ಗಿಂತ
ತತ್ತ್ವ-ಜ್ಞಾ-ನದ
ತತ್ತ್ವ-ಜ್ಞಾ-ನ-ದಿಂದ
ತತ್ತ್ವ-ಜ್ಞಾ-ನ-ವನ್ನು
ತತ್ತ್ವ-ಜ್ಞಾ-ನ-ವನ್ನೂ
ತತ್ತ್ವ-ಜ್ಞಾ-ನ-ವನ್ನೇ
ತತ್ತ್ವ-ಜ್ಞಾ-ನಿ-ಯಾ-ಗಿ-ರ-ಲಿಲ್ಲ
ತತ್ತ್ವ-ಜ್ಞಾ-ನಿಯೇ
ತತ್ತ್ವದ
ತತ್ತ್ವ-ದೊಂ-ದಿಗೆ
ತತ್ತ್ವ-ಬೋ-ಧ-ಕ-ರಾದ
ತತ್ತ್ವ-ಮಸಿ
ತತ್ತ್ವ-ರಂ-ಗ-ದಲ್ಲಿ
ತತ್ತ್ವ-ವನ್ನು
ತತ್ತ್ವ-ವ-ನ್ನು-ತ-ತ್ತ್ವದ
ತತ್ತ್ವ-ವಾ-ಗಲಿ
ತತ್ತ್ವ-ವಾ-ಗಲು
ತತ್ತ್ವ-ವಾ-ಗಿತ್ತು
ತತ್ತ್ವ-ವಾ-ದ-ಗಳ
ತತ್ತ್ವ-ವಾ-ದದ
ತತ್ತ್ವ-ವಾ-ದ-ದಲ್ಲಿ
ತತ್ತ್ವ-ವಾ-ದ-ವನ್ನು
ತತ್ತ್ವ-ವಾ-ದಿ-ಗ-ಳಿಗೆ
ತತ್ತ್ವ-ವಿ-ರ-ಬೇಕು
ತತ್ತ್ವ-ವಿ-ಲ್ಲದೆ
ತತ್ತ್ವವು
ತತ್ತ್ವವೇ
ತತ್ತ್ವ-ವೇ-ನಿ-ರ-ಬ-ಹು-ದು-ಎಂದು
ತತ್ತ್ವ-ಶಾಸ್ತ್ರ
ತತ್ತ್ವ-ಶಾ-ಸ್ತ್ರ-ಗಳ
ತತ್ತ್ವ-ಶಾ-ಸ್ತ್ರ-ಜ್ಞನ
ತತ್ತ್ವ-ಶಾ-ಸ್ತ್ರ-ಜ್ಞ-ನಾದ
ತತ್ತ್ವ-ಶಾ-ಸ್ತ್ರ-ಜ್ಞ-ನೊಬ್ಬ
ತತ್ತ್ವ-ಶಾ-ಸ್ತ್ರ-ಜ್ಞ-ರಿಗೆ
ತತ್ತ್ವ-ಶಾ-ಸ್ತ್ರ-ಜ್ಞರು
ತತ್ತ್ವ-ಶಾ-ಸ್ತ್ರಜ್ಞೆ
ತತ್ತ್ವ-ಶಾ-ಸ್ತ್ರದ
ತತ್ತ್ವ-ಶಾ-ಸ್ತ್ರ-ದಲ್ಲಿ
ತತ್ತ್ವ-ಶಾ-ಸ್ತ್ರ-ವಾಗಿ
ತತ್ತ್ವ-ಶಾ-ಸ್ತ್ರ-ವಿ-ಭಾ-ಗದ
ತತ್ತ್ವ-ಶಾ-ಸ್ತ್ರವೇ
ತತ್ಪ-ರ-ತೆ-ಯಿಂದ
ತತ್ಪ-ರ-ರಾ-ಗಿ-ರು-ತ್ತಿ-ದ್ದರು
ತತ್ಪ-ರಿ-ಣಾ-ಮ-ವಾಗಿ
ತತ್ಫ-ಲ-ವಾಗಿ
ತತ್ವ-ಗ-ಳಿಗೆ
ತತ್ವ-ಜ್ಞಾ-ನಿಯ
ತತ್ವ-ವನ್ನು
ತತ್ವ-ಶಾ-ಸ್ತ್ರದ
ತಥ್ಯ-ವನ್ನು
ತದ-ನಂ-ತ-ರದ
ತದೇ-ಕ-ಚಿ-ತ್ತ-ತೆ-ಯುಂ-ಟಾ-ಗು-ತ್ತದೆ
ತದ್ವತ್
ತದ್ವಿ-ರುದ್ಧ
ತದ್ವಿ-ರು-ದ್ಧ-ವಾಗಿ
ತದ್ವಿ-ರು-ದ್ಧ-ವಾ-ಗಿ-ದೆ-ಯೆಂ-ಬುದು
ತದ್ವಿ-ರು-ದ್ಧ-ವಾದ
ತದ್ವಿ-ರು-ದ್ಧ-ವಾ-ದವು
ತದ್ವಿ-ರು-ದ್ಧ-ವಾ-ದುದು
ತದ್ವಿ-ರು-ದ್ಧಾ-ರ್ಥದ
ತನ-ಗದು
ತನ-ಗ-ರಿ-ವಿ-ಲ್ಲ-ದಂ-ತೆಯೇ
ತನ-ಗ-ಲ್ಲದೆ
ತನ-ಗಾದ
ತನ-ಗಿಂತ
ತನ-ಗಿದ್ದ
ತನ-ಗಿ-ರುವ
ತನ-ಗಿ-ಷ್ಟ-ಬಂ-ದಷ್ಟು
ತನಗೂ
ತನಗೆ
ತನ-ಗೆ-ದು-ರಾ-ದು-ದನ್ನು
ತನಗೇ
ತನ-ಗೇ-ನಾ-ದರೂ
ತನ-ಗೊಬ್ಬ
ತನ-ವನ್ನೇ
ತನಿಖಾ
ತನಿಖೆ
ತನು
ತನು-ಮ-ನ-ಗ-ಳ-ನ್ನ-ರ್ಪಿ-ಸಿದ
ತನು-ಮ-ನ-ಗಳನ್ನು
ತನು-ಮ-ನ-ಗ-ಳೆ-ರಡೂ
ತನ್ನ
ತನ್ನಂ-ತಹ
ತನ್ನಂ-ತೆಯೇ
ತನ್ನಂ-ಥ-ವ-ನಿಗೆ
ತನ್ನ-ತ-ನ-ದೊಂ-ದಿಗೆ
ತನ್ನ-ತ-ನ-ವನ್ನು
ತನ್ನ-ತ-ನ-ವನ್ನೇ
ತನ್ನ-ದೆ-ನ್ನು-ವು-ದ-ನ್ನೆಲ್ಲ
ತನ್ನದೇ
ತನ್ನನ್ನು
ತನ್ನನ್ನೇ
ತನ್ನ-ಲ್ಲ-ಡ-ಗಿ-ಸಿ-ಕೊಂ-ಡಿ-ರುವ
ತನ್ನಲ್ಲಿ
ತನ್ನ-ಲ್ಲಿದೆ
ತನ್ನ-ಲ್ಲಿ-ರುವ
ತನ್ನ-ಲ್ಲಿ-ಲ್ಲ-ವಲ್ಲ
ತನ್ನಲ್ಲೇ
ತನ್ನ-ವ-ರಿಂ-ದಲೇ
ತನ್ನ-ಷ್ಟಕ್ಕೆ
ತನ್ನ-ಷ್ಟಕ್ಕೇ
ತನ್ನಿ
ತನ್ನಿಂದ
ತನ್ನಿಂ-ದಾದ
ತನ್ನಿಂ-ದಾ-ದು-ದ-ನ್ನೆಲ್ಲ
ತನ್ನಿ-ಚ್ಛೆ-ಯಂತೆ
ತನ್ನೆಲ್ಲ
ತನ್ನೊಂ-ದಿ-ಗಿ-ರು-ವು-ದಾ-ಗಿಯೂ
ತನ್ನೊಂ-ದಿಗೆ
ತನ್ನೊಂ-ದಿಗೇ
ತನ್ನೊ-ಡನೆ
ತನ್ನೊಬ್ಬ
ತನ್ನೊ-ಳ-ಗ-ಡ-ಗಿ-ರುವ
ತನ್ನೊ-ಳ-ಗಿ-ನಿಂದ
ತನ್ನೊ-ಳಗೆ
ತನ್ಮಯ
ತನ್ಮ-ಯ-ಗೊಂ-ಡಿ-ದ್ದಿರ
ತನ್ಮ-ಯತೆ
ತನ್ಮ-ಯ-ತೆ-ಯಿಂದ
ತನ್ಮ-ಯ-ರಾ-ಗಿ-ಬಿಟ್ಟಿ
ತನ್ಮ-ಯ-ರಾ-ಗಿ-ಬಿ-ಡು-ತ್ತಾರೆ
ತನ್ಮೂ-ಲಕ
ತಪ-ಗಳಲ್ಲಿ
ತಪ-ಸ್ವಿ-ಗ-ಳಿ-ದ್ದಾರೆ
ತಪ-ಸ್ವಿ-ಯೋ-ರ್ವ-ನನ್ನು
ತಪ-ಸ್ಸ-ನ್ನಾ-ಗಲಿ
ತಪ-ಸ್ಸ-ನ್ನಾ-ಚ-ರಿ-ಸಲು
ತಪ-ಸ್ಸನ್ನು
ತಪ-ಸ್ಸಾ-ಗು-ತ್ತಿತ್ತು
ತಪ-ಸ್ಸಿನ
ತಪ-ಸ್ಸಿ-ನಿಂದ
ತಪಸ್ಸು
ತಪಾ-ಸಣೆ
ತಪೋ-ಮ-ಹಿ-ಮ-ರೆಂ-ಬು-ದನ್ನು
ತಪ್ಪದೆ
ತಪ್ಪ-ನ್ನು-ಇ-ತ-ರರ
ತಪ್ಪ-ನ್ನು-ಮಾ-ಡಿದೆ
ತಪ್ಪ-ಲಿ-ನ-ಲ್ಲಿ-ರುವ
ತಪ್ಪಲ್ಲ
ತಪ್ಪ-ಲ್ಲವೆ
ತಪ್ಪಾಗಿ
ತಪ್ಪಾ-ಗಿ-ಬಿ-ಟ್ಟರೆ
ತಪ್ಪಾ-ಗಿ-ರು-ವಾಗ
ತಪ್ಪಾ-ಗಿ-ಹೋ-ಯಿ-ತಂತೆ
ತಪ್ಪಾ-ದ್ದ-ರಿಂದ
ತಪ್ಪಿ
ತಪ್ಪಿ-ದರೆ
ತಪ್ಪಿ-ದಾಗ
ತಪ್ಪಿ-ದ್ದ-ಕ್ಕಾಗಿ
ತಪ್ಪಿ-ದ್ದಾನೆ
ತಪ್ಪಿನ
ತಪ್ಪಿ-ಲ್ಲದೆ
ತಪ್ಪಿ-ಸಲು
ತಪ್ಪಿಸಿ
ತಪ್ಪಿ-ಸಿ-ಕೊಂ-ಡರು
ತಪ್ಪಿ-ಸಿ-ಕೊಂಡು
ತಪ್ಪಿ-ಸಿ-ಕೊ-ಳ್ಳ-ಲಾ-ರರು
ತಪ್ಪಿ-ಸಿ-ಕೊ-ಳ್ಳ-ಲಾ-ರಿರಿ
ತಪ್ಪಿ-ಸಿ-ಕೊ-ಳ್ಳಲು
ತಪ್ಪಿ-ಸಿ-ಕೊ-ಳ್ಳ-ಲೆ-ನ್ನಿ-ಸು-ತ್ತಿದ್ದ
ತಪ್ಪಿ-ಸಿ-ಕೊ-ಳ್ಳು-ತ್ತಿ-ದ್ದರು
ತಪ್ಪಿ-ಸಿ-ಕೊ-ಳ್ಳುವ
ತಪ್ಪಿ-ಸಿ-ಕೊ-ಳ್ಳು-ವು-ದ-ಕ್ಕಾ-ಗಿಯೇ
ತಪ್ಪಿ-ಸಿ-ಕೊ-ಳ್ಳು-ವು-ದ-ಕ್ಕೋ-ಸ್ಕರ
ತಪ್ಪಿ-ಸು-ತ್ತಿ-ರ-ಲಿಲ್ಲ
ತಪ್ಪಿ-ಹೋ-ಗಿತ್ತು
ತಪ್ಪಿ-ಹೋ-ಗು-ತ್ತಿತ್ತು
ತಪ್ಪಿ-ಹೋ-ಯಿ-ತಲ್ಲ
ತಪ್ಪು
ತಪ್ಪು-ಕ-ಲ್ಪ-ನೆ-ಗಳನ್ನು
ತಪ್ಪು-ಗಳಲ್ಲಿ
ತಪ್ಪು-ಗಳು
ತಪ್ಪು-ಗ್ರ-ಹಿ-ಕೆ-ಯಾ-ಗಿತ್ತು
ತಪ್ಪು-ತ್ತಿದ್ದ
ತಪ್ಪು-ತ್ತಿ-ರ-ಲಿಲ್ಲ
ತಪ್ಪು-ತ್ತಿ-ರು-ವರೋ
ತಪ್ಪು-ದಾರಿ
ತಪ್ಪೂ
ತಪ್ಪೇನೂ
ತಪ್ಪೋ
ತಬಲ
ತಬಲಾ
ತಬ್ಬ-ಲಿ-ಗಳ
ತಬ್ಬಿ-ಬ್ಬಾಗಿ
ತಬ್ಬಿ-ಬ್ಬಾ-ಗಿ-ಸಿದೆ
ತಬ್ಬಿ-ಬ್ಬಾ-ಗಿ-ಹೋ-ದರು
ತಬ್ಬಿ-ಬ್ಬಾದ
ತಬ್ಬಿ-ಬ್ಬಾ-ದರು
ತಬ್ಬಿ-ಬ್ಬು-ಗೊ-ಳಿ-ಸುವ
ತಮ
ತಮ-ಗದು
ತಮ-ಗರಿ
ತಮ-ಗ-ರಿ-ವಿ-ಲ್ಲದ
ತಮ-ಗ-ರಿ-ವಿ-ಲ್ಲ-ದಂ-ತೆಯೇ
ತಮ-ಗಾ-ಗಲಿ
ತಮ-ಗಾಗಿ
ತಮ-ಗಾ-ಗಿಯೇ
ತಮ-ಗಾದ
ತಮ-ಗಾವ
ತಮ-ಗಿಂತ
ತಮ-ಗಿದ್ದ
ತಮ-ಗಿ-ರುವ
ತಮ-ಗಿ-ಲ್ಲ-ವೆಂದು
ತಮ-ಗಿಷ್ಟ
ತಮ-ಗಿ-ಷ್ಟ-ಬಂದ
ತಮ-ಗಿ-ಷ್ಟ-ವಿ-ಲ್ಲ-ವೆಂದು
ತಮಗೆ
ತಮ-ಗೆ-ದು-ರಾದ
ತಮಗೇ
ತಮ-ಗೇ-ನಾ-ದರೂ
ತಮ-ಗೇನೂ
ತಮ-ಗೊಂದು
ತಮ-ಗೊ-ಪ್ಪಿ-ಸಿದ
ತಮಸಃ
ತಮಸ್ಸು
ತಮ-ಸ್ಸು-ಭ್ರಾಂ-ತಿ-ಗ-ಳಾಚೆ
ತಮಾಷೆ
ತಮಾ-ಷೆಗೆ
ತಮಾ-ಷೆಯ
ತಮಾ-ಷೆ-ಯ-ಲ್ಲವೆ
ತಮಾ-ಷೆ-ಯ-ವನು
ತಮಾ-ಷೆ-ಯಾಗಿ
ತಮಾ-ಷೆ-ಯಾ-ಗಿಯೇ
ತಮಾ-ಷೆ-ಯಾ-ಗಿ-ರು-ತ್ತದೆ
ತಮಾ-ಷೆ-ಯಾದ
ತಮಾ-ಷೆ-ಯಿಂದ
ತಮಾ-ಷೆ-ಯೆಂದು
ತಮಾ-ಷೆಯೋ
ತಮಿ-ಳಿ-ನಲ್ಲಿ
ತಮಿಳು
ತಮಿ-ಳು-ಪ-ದ-ಗಳನ್ನು
ತಮೇವ
ತಮೊಬ್ಬ
ತಮ್ಮ
ತಮ್ಮಂ-ತಹ
ತಮ್ಮಂ-ತೆಯೇ
ತಮ್ಮಂದಿ
ತಮ್ಮಂ-ದಿ-ರು-ಇ-ವರೆ-ಲ್ಲ-ರಿಗೂ
ತಮ್ಮ-ಗಿಷ್ಟ
ತಮ್ಮ-ತ-ನವೇ
ತಮ್ಮ-ತಮ್ಮ
ತಮ್ಮ-ತ-ಮ್ಮಲ್ಲೇ
ತಮ್ಮ-ತ-ಮ್ಮೊ-ಳಗೇ
ತಮ್ಮತ್ತ
ತಮ್ಮ-ದಾ-ಗಿಸಿ
ತಮ್ಮದು
ತಮ್ಮದೇ
ತಮ್ಮನೇ
ತಮ್ಮ-ನ್ನ-ರಸಿ
ತಮ್ಮನ್ನು
ತಮ್ಮ-ನ್ನು-ಒಬ್ಬ
ತಮ್ಮನ್ನೂ
ತಮ್ಮ-ನ್ನೆಂದೂ
ತಮ್ಮನ್ನೇ
ತಮ್ಮಲ್ಲಿ
ತಮ್ಮ-ಲ್ಲಿಗೆ
ತಮ್ಮ-ಲ್ಲಿಗೇ
ತಮ್ಮ-ಲ್ಲಿ-ದೆಯೆ
ತಮ್ಮ-ಲ್ಲಿ-ರುವ
ತಮ್ಮ-ಲ್ಲಿ-ಲ್ಲ-ವೆಂ-ಬು-ದನ್ನು
ತಮ್ಮಲ್ಲೇ
ತಮ್ಮ-ಲ್ಲೇ-ನಿದೆ
ತಮ್ಮ-ಲ್ಲೊ-ಬ್ಬ-ರೆಂದು
ತಮ್ಮವ
ತಮ್ಮ-ವ-ನ-ನ್ನಾ-ಗಿ-ಸಿ-ಕೊಂ-ಡಿ-ದ್ದರು
ತಮ್ಮ-ವ-ರನ್ನು
ತಮ್ಮ-ವರು
ತಮ್ಮ-ಷ್ಟಕ್ಕೆ
ತಮ್ಮ-ಷ್ಟಕ್ಕೇ
ತಮ್ಮಷ್ಟೇ
ತಮ್ಮಿಂದ
ತಮ್ಮಿಂ-ದಾ-ದ-ದ್ದ-ನ್ನೆಲ್ಲ
ತಮ್ಮಿಂ-ದಾ-ದು-ದ-ನ್ನೆಲ್ಲ
ತಮ್ಮಿಷ್ಟ
ತಮ್ಮೆ-ದು-ರಿಗೆ
ತಮ್ಮೆಲ್ಲ
ತಮ್ಮೆ-ಲ್ಲರ
ತಮ್ಮೆ-ಲ್ಲ-ರನ್ನೂ
ತಮ್ಮೊಂ-ದಿ-ಗಿನ
ತಮ್ಮೊಂ-ದಿ-ಗಿ-ರುವ
ತಮ್ಮೊಂ-ದಿಗೆ
ತಮ್ಮೊ-ಡನೆ
ತಮ್ಮೊಬ್ಬ
ತಮ್ಮೊ-ಳ-ಗಿದ್ದ
ತಮ್ಮೊ-ಳ-ಗಿನ
ತಮ್ಮೊ-ಳಗೆ
ತಯಾ
ತಯಾ-ರಾ-ಗು-ತ್ತಿದ್ದ
ತಯಾರಿ
ತಯಾ-ರಿ-ಯನ್ನೂ
ತಯಾ-ರಿಯೂ
ತಯಾ-ರಿ-ಸ-ಬೇಕು
ತಯಾ-ರಿಸಿ
ತಯಾ-ರಿ-ಸು-ತ್ತಿ-ದ್ದರು
ತಯಾ-ರಿ-ಸುವ
ತಯಾರು
ತರಂ-ಗ-ಗಳನ್ನು
ತರಂ-ಗ-ಗಳು
ತರಂ-ಗ-ತ-ರಂಗ
ತರಂ-ಗವೇ
ತರ-ಗತಿ
ತರ-ಗ-ತಿ-ಗಳ
ತರ-ಗ-ತಿ-ಗ-ಳಂತೆ
ತರ-ಗ-ತಿ-ಗ-ಳ-ನ್ನಷ್ಟೇ
ತರ-ಗ-ತಿ-ಗಳನ್ನು
ತರ-ಗ-ತಿ-ಗಳನ್ನೂ
ತರ-ಗ-ತಿ-ಗ-ಳ-ಲ್ಲದೆ
ತರ-ಗ-ತಿ-ಗಳಲ್ಲಿ
ತರ-ಗ-ತಿ-ಗಳಿಂದ
ತರ-ಗ-ತಿ-ಗ-ಳಿ-ಗಾಗಿ
ತರ-ಗ-ತಿ-ಗ-ಳಿ-ಗಾ-ಗಿಯೇ
ತರ-ಗ-ತಿ-ಗ-ಳಿಗೂ
ತರ-ಗ-ತಿ-ಗ-ಳಿಗೆ
ತರ-ಗ-ತಿ-ಗಳು
ತರ-ಗ-ತಿ-ಗಳೂ
ತರ-ಗ-ತಿಗೆ
ತರ-ಗ-ತಿಯ
ತರ-ಗ-ತಿ-ಯನ್ನೂ
ತರ-ಗ-ತಿ-ಯಲ್ಲಿ
ತರ-ಗ-ತಿ-ಯಲ್ಲೂ
ತರ-ಗ-ತಿಯು
ತರ-ಗ-ತಿ-ಯೊಂ-ದನ್ನು
ತರ-ಚಿ-ಕೊಂಡು
ತರದ
ತರ-ಬೇ-ಕಾ-ಗಿದೆ
ತರ-ಬೇಕು
ತರ-ಬೇತಿ
ತರ-ಬೇ-ತು-ಗೊಂ-ಡ-ವ-ರಿಗೆ
ತರ-ಬೇ-ತು-ಗೊ-ಳಿಸ
ತರ-ಬೇ-ತು-ಗೊ-ಳಿ-ಸಿ-ದರು
ತರ-ಬೇ-ತು-ಗೊ-ಳಿ-ಸು-ವು-ದ-ಕ್ಕಿಂತ
ತರ-ಲಾ-ರ-ದೇನು
ತರಲು
ತರ-ವಲ್ಲ
ತರಹ
ತರ-ಹದ
ತರಾ-ಟೆಗೆ
ತರಿ-ಸಿ-ಕೊಂಡ
ತರಿ-ಸಿ-ಕೊಂ-ಡರು
ತರಿ-ಸಿ-ಕೊಟ್ಟು
ತರಿ-ಸಿ-ಕೊ-ಳ್ಳು-ತ್ತಿ-ರು-ವಾಗ
ತರು
ತರುಣ
ತರು-ಣನ
ತರು-ಣ-ನನ್ನೇ
ತರು-ಣ-ನಿಗೆ
ತರು-ಣ-ರಿಗೆ
ತರು-ಣರು
ತರು-ಣ-ರೆಲ್ಲ
ತರು-ಣ-ಸಂ-ನ್ಯಾ-ಸಿ-ಯನ್ನು
ತರು-ಣಿಯ
ತರು-ತಲ
ತರು-ತ್ತಿದೆ
ತರು-ತ್ತಿ-ದ್ದರು
ತರು-ತ್ತಿದ್ದೆ
ತರು-ತ್ತೀಯೋ
ತರು-ತ್ತೀರಿ
ತರುವ
ತರು-ವಂ-ತಹ
ತರು-ವಂ-ಥದು
ತರು-ವತ್ತ
ತರು-ವಲ್ಲಿ
ತರು-ವ-ವ-ರಾ-ಗು-ತ್ತೀ-ರೆಂದು
ತರು-ವ-ಷ್ಟ-ರ-ಮ-ಟ್ಟಿಗೆ
ತರು-ವಾಯ
ತರು-ವು-ದ-ರಲ್ಲಿ
ತರು-ವು-ದ-ರಿಂದ
ತರ್ಕ
ತರ್ಕಕ್ಕೆ
ತರ್ಕದ
ತರ್ಕ-ದಿಂದ
ತರ್ಕ-ಬದ್ಧ
ತರ್ಕ-ಬ-ದ್ಧ-ವಾಗಿ
ತರ್ಕ-ವನ್ನು
ತರ್ಕ-ಶಾ-ಸ್ತ್ರ-ವನ್ನು
ತಲ-ಪು-ತ್ತಿದೆ
ತಲ-ಪು-ವಲ್ಲಿ
ತಲಾ
ತಲು
ತಲುಪ
ತಲು-ಪ-ದಿ-ದ್ದುದು
ತಲು-ಪ-ದಿ-ರು-ವುದು
ತಲು-ಪಲು
ತಲು-ಪಲೇ
ತಲುಪಿ
ತಲು-ಪಿತು
ತಲು-ಪಿದ
ತಲು-ಪಿ-ದರು
ತಲು-ಪಿ-ದಾಗ
ತಲು-ಪಿ-ದು-ವಾ-ದರೂ
ತಲು-ಪಿದ್ದು
ತಲು-ಪಿ-ಬಿ-ಡು-ತ್ತೇನೆ
ತಲು-ಪಿ-ರ-ಬ-ಹುದು
ತಲು-ಪಿ-ರುವ
ತಲು-ಪಿ-ರು-ವುದನ್ನು
ತಲು-ಪಿಸ
ತಲು-ಪಿ-ಸಿದ
ತಲು-ಪಿ-ಸು-ವಂತೆ
ತಲು-ಪಿ-ಸು-ವು-ದಾಗಿ
ತಲುಪು
ತಲು-ಪು-ತ್ತಾ-ರೆಂದು
ತಲು-ಪು-ತ್ತಿ-ದ್ದಂ-ತೆಯೇ
ತಲು-ಪು-ತ್ತಿ-ದ್ದೇ-ನೆಂ-ದರೆ
ತಲು-ಪು-ತ್ತೇನೆ
ತಲು-ಪುವ
ತಲು-ಪು-ವ-ವ-ರೆಗೂ
ತಲು-ಪುವು
ತಲೆ
ತಲೆ-ಮೈ
ತಲೆ-ಕೆ-ಳ-ಗಾದ
ತಲೆ-ಕೆ-ಳಗು
ತಲೆ-ಗಳನ್ನು
ತಲೆ-ಗಳಲ್ಲಿ
ತಲೆಗೆ
ತಲೆ-ಗೊಂ-ದೊಂದು
ತಲೆ-ತ-ಗ್ಗಿ-ಸಿದ
ತಲೆ-ತ-ಗ್ಗಿ-ಸಿ-ದರು
ತಲೆ-ತ-ಗ್ಗಿ-ಸು-ವಂ-ತಾ-ಯಿತು
ತಲೆ-ತ-ಲಾಂ-ತ-ರ-ಗಳ
ತಲೆ-ದೂ-ಗುತ್ತ
ತಲೆ-ದೂ-ಗು-ವಂತೆ
ತಲೆ-ಬಾಗಿ
ತಲೆ-ಬಾ-ಗಿ-ದ್ದರು
ತಲೆ-ಬಾ-ಗು-ತ್ತೇನೆ
ತಲೆ-ಬಾ-ಗು-ವಂತೆ
ತಲೆ-ಬಾ-ಗು-ವುದನ್ನು
ತಲೆ-ಬು-ಡ-ವಿ-ಲ್ಲದ
ತಲೆ-ಬೇ-ನೆಯೇ
ತಲೆ-ಮಾ-ರನ್ನೇ
ತಲೆ-ಮಾರು
ತಲೆ-ಮಾ-ರು-ಗಳ
ತಲೆಯ
ತಲೆ-ಯನ್ನು
ತಲೆ-ಯಲ್ಲಿ
ತಲೆ-ಯಾ-ಡಿ-ಸುತ್ತ
ತಲೆ-ಯಿಂದ
ತಲೆಯು
ತಲೆ-ಯೆ-ತ್ತ-ಲಾ-ರರು
ತಲೆ-ಯೆ-ತ್ತಲು
ತಲೆ-ಯೆತ್ತಿ
ತಲೆ-ಯೆ-ತ್ತಿಯೂ
ತಲೆ-ಯೆ-ತ್ತು-ತ್ತದೆ
ತಲೆಯೇ
ತಲೆ-ಯೊ-ಳಕ್ಕೆ
ತಲೆ-ಹಾ-ಕಲು
ತಲೆ-ಹಾ-ಕುವ
ತಲ್ಲ-ಣ-ಗೊ-ಳಿ-ಸು-ವಂ-ತಿ-ರು-ವುದನ್ನು
ತಲ್ಲ-ಣಿಸಿ
ತಲ್ಲ-ಣಿ-ಸು-ತ್ತದೆ
ತಲ್ಲೀ-ನ-ರಾ-ಗಿ-ಬಿ-ಟ್ಟರು
ತಳ-ದ-ಲ್ಲಿ-ರುವ
ತಳ-ಪಾ-ಯವೇ
ತಳ-ವೂ-ರಿದ್ದ
ತಳ-ಹ-ದಿಯ
ತಳ-ಹ-ದಿ-ಯಾ-ಗಿತ್ತು
ತಳಿಯ
ತಳೆದು
ತಳೆ-ಯ-ದಿ-ರಲು
ತಳ್ಳಿ
ತಳ್ಳಿವೆ
ತಳ್ಳಿ-ಹಾ-ಕ-ಲಾ-ಗದು
ತಳ್ಳಿ-ಹಾ-ಕಲು
ತಳ್ಳಿ-ಹಾಕಿ
ತಳ್ಳಿ-ಹಾ-ಕಿ-ದರು
ತಳ್ಳಿ-ಹಾ-ಕಿ-ಬಿ-ಟ್ಟರು
ತಳ್ಳಿ-ಹಾಕು
ತಳ್ಳಿ-ಹಾ-ಕು-ತ್ತಾರೆ
ತಳ್ಳಿ-ಹಾ-ಕುವ
ತಳ್ಳಿ-ಹಾ-ಕು-ವುದನ್ನು
ತಳ್ಳು-ತ್ತಿ-ರು-ತ್ತಾಳೆ
ತಳ್ಳು-ವಂ-ತಹ
ತಳ್ಳು-ವಂಥ
ತವ-ಕಿ-ಸು-ತ್ತಿತ್ತು
ತವ-ರೂ-ರಾದ
ತವ-ರೂರು
ತವ-ರೂ-ರೆ-ನಿ-ಸಿದ
ತಸ್ತುಃ
ತಾ
ತಾಂಡ-ವ-ವಾ-ಡು-ತ್ತಿದ್ದ
ತಾಂತ್ರಿಕ
ತಾಂಸ್ತ-ಥೈವ
ತಾಕ-ತ್ತಿ-ದ್ದರೆ
ತಾಕತ್ತು
ತಾಡ-ನಕ್ಕೆ
ತಾಣ
ತಾಣ-ಗ-ಳ-ಲ್ಲೊಂದು
ತಾಣ-ದಲ್ಲಿ
ತಾತಂ-ದಿರ
ತಾತಂ-ದಿರೂ
ತಾತ್ಕಾ-ಲಿಕ
ತಾತ್ಕಾ-ಲಿ-ಕ-ವಾದ
ತಾತ್ತ್ವಿಕ
ತಾತ್ವಿಕ
ತಾತ್ವಿ-ಕ-ತೆಯ
ತಾತ್ವಿ-ಕ-ತೆ-ಯನ್ನು
ತಾತ್ವಿ-ಕ-ವಾಗಿ
ತಾದಾತ್ಮ್ಯ
ತಾದಾ-ತ್ಮ್ಯ-ಗೊ-ಳಿ-ಸಿ-ಕೊಂ-ಡು-ಬಿಟ್ಟಿ
ತಾನ-ರಿತು
ತಾನಾಗಿ
ತಾನಾ-ಗಿಯೇ
ತಾನಾ-ಗಿ-ರುವ
ತಾನಾ-ಯಿತು
ತಾನಿದ್ದು
ತಾನಿನ್ನು
ತಾನಿ-ರಲು
ತಾನು
ತಾನೂ
ತಾನೆ
ತಾನೆಂದು
ತಾನೆಂ-ದು-ಕೊಂ-ಡಂ-ತೆಯೇ
ತಾನೆಂಬ
ತಾನೆ-ಲ್ಲಿ-ದ್ದೇನೆ
ತಾನೆಲ್ಲೇ
ತಾನೇ
ತಾನೇ-ತಾ-ನಾಗಿ
ತಾನೊಬ್ಬ
ತಾನೊ-ಬ್ಬನೇ
ತಾಮುಂ-ದೆಂದು
ತಾಮ್ರ
ತಾಯ
ತಾಯಂ-ದಿರ
ತಾಯಂ-ದಿರೇ
ತಾಯಂ-ದಿ-ರೊ-ಡ-ಗೂಡಿ
ತಾಯಂ-ದಿ-ರೊ-ಡನೆ
ತಾಯಿ
ತಾಯಿ-ಮ-ಕ್ಕಳ
ತಾಯಿಗೆ
ತಾಯಿತ
ತಾಯಿಯ
ತಾಯಿ-ಯಂ-ತಿ-ದ್ದರು
ತಾಯಿ-ಯ-ವ-ರಿಗೆ
ತಾಯಿ-ಯ-ವರು
ತಾಯಿ-ಯಿಂದ
ತಾಯ್ತ-ನದ
ತಾಯ್ನಾ-ಡನ್ನು
ತಾಯ್ನಾ-ಡ-ವರ
ತಾಯ್ನಾಡಿ
ತಾಯ್ನಾ-ಡಿಗೆ
ತಾಯ್ನಾ-ಡಿ-ಗೆ-ಹೊ-ರ-ಟರು
ತಾಯ್ನಾ-ಡಿನ
ತಾಯ್ನಾಡೇ
ತಾರಕ
ತಾರ-ಕ-ಕ್ಕೇ-ರಿ-ದಾ-ಗಲೂ
ತಾರ-ಕ-ಮಂ-ತ್ರ-ವನ್ನು
ತಾರ-ದಿ-ರು-ವುದು
ತಾರ-ಸಿಯ
ತಾರ-ಸಿ-ಯ-ವ-ರೆಗೂ
ತಾರ-ಸಿಯು
ತಾರಾ
ತಾರಿ-ಘಾಟ್
ತಾರೀ-ಕಿ-ನಂದು
ತಾರೀಕು
ತಾರೀಖು
ತಾರು-ಣ್ಯ-ದ-ಲ್ಲಿಯೇ
ತಾರು-ಣ್ಯ-ದಲ್ಲೇ
ತಾರೆ-ಯ-ನ್ನು-ಸ್ವಯಂ
ತಾಳ
ತಾಳ-ದ-ವರು
ತಾಳದೆ
ತಾಳ-ಬ-ಲ್ಲನೋ
ತಾಳ-ಬ-ಹುದು
ತಾಳ-ಬೇ-ಕಾ-ಗಿಲ್ಲ
ತಾಳ-ಬೇ-ಕೆಂ-ದರೆ
ತಾಳ-ಲಾ-ರದೆ
ತಾಳ-ವನು
ತಾಳಿ
ತಾಳಿ-ಕೊಂ-ಡಂತೆ
ತಾಳಿ-ಕೊಳ್ಳ
ತಾಳಿ-ಕೊ-ಳ್ಳು-ವುದೆ
ತಾಳಿತು
ತಾಳಿದ
ತಾಳಿ-ದರು
ತಾಳಿ-ದ-ವ-ರಲ್ಲ
ತಾಳಿ-ದ-ವರು
ತಾಳಿ-ದು-ದನ್ನು
ತಾಳಿದ್ದ
ತಾಳಿ-ದ್ದ-ರಿಂದ
ತಾಳಿ-ದ್ದರು
ತಾಳಿ-ದ್ದ-ರೆಂ-ದರೆ
ತಾಳಿ-ದ್ದ-ವರೂ
ತಾಳಿ-ದ್ದಾರೆ
ತಾಳಿಯೇ
ತಾಳಿ-ರುವು
ತಾಳು-ತ್ತಿ-ದ್ದರು
ತಾಳು-ತ್ತಿ-ದ್ದ-ವರು
ತಾಳುವ
ತಾಳು-ವಂತೆ
ತಾಳು-ವುದು
ತಾಳ್ಮೆ
ತಾಳ್ಮೆ-ಯಿಂದ
ತಾಳ್ಮೆಯೂ
ತಾವಾ
ತಾವಾ-ಗಿಯೇ
ತಾವಾ-ಡುವ
ತಾವಿಟ್ಟ
ತಾವಿದ್ದ
ತಾವಿ-ದ್ದಂ-ತಹ
ತಾವಿನ್ನು
ತಾವಿ-ಬ್ಬರೂ
ತಾವಿ-ರ-ಬೇ-ಕೆಂದು
ತಾವಿ-ಲ್ಲಿಗೆ
ತಾವೀಗ
ತಾವು
ತಾವೂ
ತಾವೆ-ರಡು
ತಾವೇ
ತಾವೇಕೆ
ತಾವೇನು
ತಾವೊಂದು
ತಾವೊಬ್ಬ
ತಾವೊ-ಬ್ಬರು
ತಾವೊ-ಬ್ಬರೇ
ತಾಹ್ಲಾ
ತಾಹ್ಲಾ-ದಲ್ಲಿ
ತಾಹ್ಲಾ-ವನ್ನು
ತಿಂಗಳ
ತಿಂಗ-ಳಂತೂ
ತಿಂಗಳನ್ನು
ತಿಂಗಳಲ್ಲಿ
ತಿಂಗ-ಳಲ್ಲೇ
ತಿಂಗ-ಳ-ವ-ರೆಗೆ
ತಿಂಗ-ಳಷ್ಟು
ತಿಂಗ-ಳಾ-ಗಿತ್ತು
ತಿಂಗ-ಳಾ-ದರೂ
ತಿಂಗ-ಳಾ-ದ-ರೂ-ಬೇಡ
ತಿಂಗಳಿ
ತಿಂಗ-ಳಿಗೂ
ತಿಂಗ-ಳಿಗೆ
ತಿಂಗ-ಳಿ-ನಲ್ಲಿ
ತಿಂಗ-ಳಿ-ನಿಂ-ದಲೂ
ತಿಂಗಳು
ತಿಂಗ-ಳು-ಗ-ಟ್ಟಲೆ
ತಿಂಗ-ಳು-ಗಳ
ತಿಂಗ-ಳು-ಗಳನ್ನು
ತಿಂಗ-ಳು-ಗಳಲ್ಲಿ
ತಿಂಗ-ಳು-ಗಳೇ
ತಿಂಗಳೂ
ತಿಂಗ-ಳೆಂ-ದರೆ
ತಿಂಗಳೇ
ತಿಂಗ-ಳೊಂ-ದಿಗೆ
ತಿಂಡಿ-ಗಳನ್ನು
ತಿಂಡಿ-ಗಳನ್ನೆಲ್ಲ
ತಿಂಡಿ-ತೀ-ರ್ಥ-ಗಳ
ತಿಂಡಿಯ
ತಿಂತೇನೆ
ತಿಂದ
ತಿಂದ-ದ್ದ-ರಿಂದ
ತಿಂದದ್ದೇ
ತಿಂದರು
ತಿಂದು
ತಿಂದು-ನೋ-ಡು-ವಂ-ತೆಯೂ
ತಿಂದೇ
ತಿಕ್ಕಿ
ತಿಕ್ಕಿ-ಕೊಳ್ಳು
ತಿಕ್ಕಿ-ಕೊ-ಳ್ಳುತ್ತ
ತಿಕ್ಕಿ-ಕೊ-ಳ್ಳು-ತ್ತಲೇ
ತಿಕ್ಕಿ-ಕೊ-ಳ್ಳು-ವು-ದ-ರಿಂದ
ತಿಕ್ಕಿ-ಕೊ-ಳ್ಳು-ವುದೂ
ತಿದ್ದ
ತಿದ್ದಿ
ತಿದ್ದಿ-ಕೊಂ-ಡರು
ತಿದ್ದಿ-ಕೊಂ-ಡಾನು
ತಿದ್ದಿ-ದರು
ತಿದ್ದುವ
ತಿದ್ದು-ವಂ-ತಾ-ದರೆ
ತಿನ್ನ-ಬಾ-ರ-ದ್ದ-ನ್ನೆಲ್ಲ
ತಿನ್ನ-ಲಾ-ದೀತು
ತಿನ್ನಲು
ತಿನ್ನಿ-ಸ-ಬ-ಹುದು
ತಿನ್ನು
ತಿನ್ನುತ್ತ
ತಿನ್ನು-ತ್ತಲೇ
ತಿನ್ನು-ತ್ತಿ-ದ್ದರು
ತಿನ್ನು-ತ್ತಿ-ದ್ದಾರೆ
ತಿನ್ನು-ತ್ತೇನೆ
ತಿನ್ನುವ
ತಿನ್ನು-ವು-ದಕ್ಕೆ
ತಿನ್ನು-ವು-ದ-ನ್ನಾ-ದರೂ
ತಿನ್ನು-ವು-ದಿಲ್ಲ
ತಿನ್ನು-ವುದು
ತಿನ್ನು-ವು-ದು-ಕುಡಿ
ತಿನ್ನು-ವುದೂ
ತಿನ್ನು-ವುದೆ
ತಿರ
ತಿರ-ಸ್ಕ-ರಿ-ಸ-ಲಾ-ರ-ದ-ವ-ನಾದೆ
ತಿರ-ಸ್ಕ-ರಿಸಿ
ತಿರ-ಸ್ಕ-ರಿ-ಸಿ-ದ್ದ-ರಿಂದ
ತಿರ-ಸ್ಕ-ರಿ-ಸಿದ್ದು
ತಿರ-ಸ್ಕ-ರಿ-ಸಿ-ಬಿ-ಡು-ತ್ತಿದ್ದ
ತಿರ-ಸ್ಕ-ರಿ-ಸು-ತ್ತಾನೆ
ತಿರ-ಸ್ಕ-ರಿ-ಸು-ವುದೂ
ತಿರ-ಸ್ಕಾರ
ತಿರ-ಸ್ಕಾ-ರ-ಅ-ಪ-ಹಾ-ಸ್ಯ-ಪೂ-ರಿತ
ತಿರ-ಸ್ಕಾ-ರದ
ತಿರ-ಸ್ಕಾ-ರ-ದಿಂದ
ತಿರ-ಸ್ಕಾ-ರ-ವನ್ನು
ತಿರು
ತಿರು-ಕ-ನಿ-ಗೇನು
ತಿರು-ಗ-ಬ-ಹು-ದೆಂದು
ತಿರು-ಗ-ಬೇಡಿ
ತಿರು-ಗಲಿ
ತಿರು-ಗಾ-ಡಲು
ತಿರು-ಗಾ-ಡಿ-ಕೊಂ-ಡಿ-ದ್ದೇವೆ
ತಿರು-ಗಾ-ಡಿ-ಕೊಂಡು
ತಿರು-ಗಾ-ಡಿ-ಕೊಂ-ಡು-ಬ-ರಲು
ತಿರು-ಗಾ-ಡುತ್ತ
ತಿರು-ಗಾ-ಡು-ತ್ತಿ-ದ್ದಾಗ
ತಿರುಗಿ
ತಿರು-ಗಿ-ಕೊಂ-ಡಿತು
ತಿರು-ಗಿ-ಕೊಂ-ಡಿತ್ತು
ತಿರು-ಗಿ-ಕೊ-ಳ್ಳು-ತ್ತಿತ್ತು
ತಿರು-ಗಿತು
ತಿರು-ಗಿ-ತೆಂ-ದರೆ
ತಿರು-ಗಿದ
ತಿರು-ಗಿ-ದರು
ತಿರು-ಗಿ-ದ್ದ-ರಿಂದ
ತಿರು-ಗಿ-ದ್ದರು
ತಿರು-ಗಿ-ನೋ-ಡ-ಬೇಡಿ
ತಿರು-ಗಿ-ಬಂ-ದಿ-ದ್ದೇನೆ
ತಿರು-ಗಿ-ಸಲು
ತಿರು-ಗಿಸಿ
ತಿರು-ಗಿ-ಸಿ-ಕೊ-ಳ್ಳು-ತ್ತಿ-ದ್ದ-ರು-ಇದು
ತಿರು-ಗಿ-ಸಿದ
ತಿರು-ಗಿ-ಸಿ-ದರೆ
ತಿರು-ಗಿ-ಹೋ-ಗಿ-ಬಿ-ಟ್ಟಿದೆ
ತಿರು-ಗು-ತ್ತಿ-ರು-ತ್ತಾರೆ
ತಿರು-ಗು-ವ-ವರೇ
ತಿರು-ಗು-ವುದು
ತಿರು-ಪೆ-ಯೆತ್ತಿ
ತಿರು-ಳನ್ನು
ತಿರು-ಳ-ಲ್ಲದೆ
ತಿರು-ಳಾದ
ತಿರು-ಳಿ-ರು-ವುದು
ತಿರು-ಳಿಲ್ಲ
ತಿರುಳು
ತಿರು-ವ-ನಂ-ತ-ಪುರ
ತಿರು-ವ-ನಂ-ತ-ಪು-ರಕ್ಕೆ
ತಿರು-ವ-ನಂ-ತ-ಪು-ರದ
ತಿರು-ವ-ನಂ-ತ-ಪು-ರ-ದಲ್ಲಿ
ತಿರು-ವ-ನಂ-ತ-ಪು-ರ-ದಿಂದ
ತಿರು-ವನ್ನು
ತಿರು-ವಾಂ-ಕೂ-ರಿನ
ತಿರು-ವು-ಗಳಿಂದ
ತಿಲ-ಕರ
ತಿಲ-ಕ-ರಿಗೂ
ತಿಲ-ಕ-ರಿಗೆ
ತಿಲ-ಕರು
ತಿಲ-ಕರೂ
ತಿಲ-ಕರೇ
ತಿಲ-ಕ-ರೊಂ-ದಿಗೆ
ತಿಳಿದ
ತಿಳಿ-ದಂ-ತಿತ್ತು
ತಿಳಿ-ದರು
ತಿಳಿ-ದ-ವನು
ತಿಳಿ-ದ-ವ-ನೆಂಬ
ತಿಳಿ-ದ-ವರು
ತಿಳಿ-ದ-ವ-ರೆಂದೂ
ತಿಳಿ-ದ-ವರೇ
ತಿಳಿ-ದಾಗ
ತಿಳಿ-ದಿತ್ತು
ತಿಳಿ-ದಿದೆ
ತಿಳಿ-ದಿ-ದೆ-ಯೆಂದು
ತಿಳಿ-ದಿದ್ದ
ತಿಳಿ-ದಿ-ದ್ದರು
ತಿಳಿ-ದಿ-ದ್ದರೂ
ತಿಳಿ-ದಿ-ದ್ದರೆ
ತಿಳಿ-ದಿ-ದ್ದಳು
ತಿಳಿ-ದಿ-ದ್ದೀಯೆ
ತಿಳಿ-ದಿ-ದ್ದೀರಿ
ತಿಳಿ-ದಿ-ದ್ದು-ದ-ಕ್ಕಿಂತ
ತಿಳಿ-ದಿ-ದ್ದು-ದೇ-ನೆಂ-ದರೆ
ತಿಳಿ-ದಿದ್ದೆ
ತಿಳಿ-ದಿ-ರ-ದಿದ್ದ
ತಿಳಿ-ದಿ-ರ-ಬ-ಹುದು
ತಿಳಿ-ದಿ-ರ-ಬೇ-ಕಾ-ಗು-ತ್ತದೆ
ತಿಳಿ-ದಿ-ರ-ಬೇ-ಡವೆ
ತಿಳಿ-ದಿ-ರ-ಲಾ-ರದು
ತಿಳಿ-ದಿ-ರ-ಲಿಲ್ಲ
ತಿಳಿ-ದಿ-ರ-ಲಿ-ಲ್ಲ-ವಾ-ದರೂ
ತಿಳಿ-ದಿ-ರ-ಲಿ-ಲ್ಲ-ವೆಂದು
ತಿಳಿ-ದಿ-ರ-ಲಿ-ಲ್ಲ-ವೆಂ-ದು-ಕೊಂ-ಡಿ-ದ್ದೆನೋ
ತಿಳಿ-ದಿ-ರಲು
ತಿಳಿ-ದಿ-ರಲೇ
ತಿಳಿ-ದಿ-ರು-ತ್ತದೆ
ತಿಳಿ-ದಿ-ರುವ
ತಿಳಿ-ದಿ-ರು-ವಂತೆ
ತಿಳಿ-ದಿ-ರು-ವಂ-ಥವೇ
ತಿಳಿ-ದಿ-ರು-ವ-ವರು
ತಿಳಿ-ದಿ-ರು-ವು-ದ-ನ್ನೆಲ್ಲ
ತಿಳಿ-ದಿಲ್ಲ
ತಿಳಿ-ದಿ-ಲ್ಲ-ದಿ-ರು-ವುದು
ತಿಳಿ-ದಿ-ಲ್ಲ-ನೀ-ವೀಗ
ತಿಳಿದು
ತಿಳಿ-ದುಕೊ
ತಿಳಿ-ದು-ಕೊಂಡ
ತಿಳಿ-ದು-ಕೊಂ-ಡರು
ತಿಳಿ-ದು-ಕೊಂ-ಡಿ-ದ್ದಾರೆ
ತಿಳಿ-ದು-ಕೊಂ-ಡಿ-ದ್ದೀರೋ
ತಿಳಿ-ದು-ಕೊಂ-ಡಿರು
ತಿಳಿ-ದು-ಕೊಂ-ಡಿ-ರು-ತ್ತಾರೆ
ತಿಳಿ-ದು-ಕೊಂಡು
ತಿಳಿ-ದು-ಕೊ-ಳ್ಳ-ಬ-ಹುದು
ತಿಳಿ-ದು-ಕೊ-ಳ್ಳ-ಬೇ-ಕಾ-ಗಿತ್ತು
ತಿಳಿ-ದು-ಕೊ-ಳ್ಳ-ಬೇಕು
ತಿಳಿ-ದು-ಕೊ-ಳ್ಳಲಿ
ತಿಳಿ-ದು-ಕೊ-ಳ್ಳ-ಲಿಲ್ಲ
ತಿಳಿ-ದು-ಕೊ-ಳ್ಳಲು
ತಿಳಿ-ದು-ಕೊ-ಳ್ಳ-ಲೆಂದು
ತಿಳಿ-ದು-ಕೊಳ್ಳು
ತಿಳಿ-ದು-ಕೊ-ಳ್ಳು-ತ್ತೇ-ನೆಈ
ತಿಳಿ-ದು-ಕೊ-ಳ್ಳು-ವಂ-ತಾ-ಯಿ-ತೆಂ-ದರೆ
ತಿಳಿ-ದು-ಕೊ-ಳ್ಳು-ವುದು
ತಿಳಿ-ದು-ಕೋ-ಯಾರು
ತಿಳಿ-ದು-ಬಂತು
ತಿಳಿ-ದು-ಬಂ-ತು-ಕೆ-ಲ-ದಿ-ನ-ಗಳ
ತಿಳಿ-ದು-ಬಂ-ತು-ಮ-ಸ್ಸಾ-ಚು-ಸೆ-ಟ್ಸ್
ತಿಳಿ-ದು-ಬಂ-ದ-ದ್ದೇ-ನೆಂ-ದರೆ
ತಿಳಿ-ದು-ಬಂ-ದಾಗ
ತಿಳಿ-ದು-ಬಂ-ದಾ-ಗಿ-ನಿಂ-ದಲೂ
ತಿಳಿ-ದು-ಬಂ-ದಿತು
ತಿಳಿ-ದು-ಬಂ-ದಿ-ತು-ಮಾ-ವಿ-ನ-ಫ-ಸಲು
ತಿಳಿ-ದು-ಬಂ-ದಿ-ತು-ಸ್ವಾ-ಮೀಜಿ
ತಿಳಿ-ದು-ಬಂ-ದಿತ್ತು
ತಿಳಿ-ದು-ಬಂ-ದಿದೆ
ತಿಳಿ-ದು-ಬಂ-ದಿ-ರು-ವುದು
ತಿಳಿ-ದು-ಬಂ-ದಿಲ್ಲ
ತಿಳಿ-ದು-ಬಂ-ದಿ-ಲ್ಲ-ವಾ-ದರೂ
ತಿಳಿ-ದು-ಬಂ-ದಿವೆ
ತಿಳಿ-ದು-ಬರು
ತಿಳಿ-ದು-ಬ-ರು-ತ್ತದೆ
ತಿಳಿ-ದು-ಬ-ರು-ತ್ತವೆ
ತಿಳಿದೂ
ತಿಳಿ-ದೆಯಾ
ತಿಳಿದೇ
ತಿಳಿ-ನೀರ
ತಿಳಿ-ನೀ-ರನ್ನು
ತಿಳಿ-ನೀ-ರಿ-ನಂತೆ
ತಿಳಿ-ನೀರು
ತಿಳಿಯ
ತಿಳಿ-ಯ-ದಂತೆ
ತಿಳಿ-ಯ-ದ-ವನೆ
ತಿಳಿ-ಯ-ದಿದ್ದ
ತಿಳಿ-ಯ-ದಿ-ರು-ತ್ತ-ದೆಯೆ
ತಿಳಿ-ಯದು
ತಿಳಿ-ಯದೆ
ತಿಳಿ-ಯ-ದೆಂದು
ತಿಳಿ-ಯ-ಬ-ಲ್ಲ-ವ-ರಾ-ಗಿ-ರ-ಬೇ-ಕಿ-ತ್ತ-ಲ್ಲವೆ
ತಿಳಿ-ಯ-ಬ-ಹು-ದಾ-ದದ್ದೇ
ತಿಳಿ-ಯ-ಬೇ-ಕಾ-ಗಿಲ್ಲ
ತಿಳಿ-ಯ-ಬೇ-ಕಾ-ಗು-ತ್ತದೆ
ತಿಳಿ-ಯ-ಬೇ-ಕಾ-ದರೆ
ತಿಳಿ-ಯ-ಬೇಕು
ತಿಳಿ-ಯ-ಬೇಡ
ತಿಳಿ-ಯ-ಬೇಡಿ
ತಿಳಿ-ಯರು
ತಿಳಿ-ಯ-ಲಿಲ್ಲ
ತಿಳಿ-ಯ-ಲಿ-ಲ್ಲ-ವೆಂದರೆ
ತಿಳಿ-ಯಲು
ತಿಳಿ-ಯಲೇ
ತಿಳಿ-ಯ-ಹೇ-ಳ-ಬೇಕು
ತಿಳಿ-ಯ-ಹೇಳಿ
ತಿಳಿ-ಯ-ಹೇ-ಳಿ-ದರು
ತಿಳಿ-ಯ-ಹೇ-ಳಿ-ದರೆ
ತಿಳಿ-ಯಾ-ಗಿದ್ದು
ತಿಳಿ-ಯಾದ
ತಿಳಿ-ಯಿತು
ತಿಳಿಯು
ತಿಳಿ-ಯು-ತ್ತದೆ
ತಿಳಿ-ಯು-ತ್ತಾರೆ
ತಿಳಿ-ಯು-ತ್ತಿತ್ತು
ತಿಳಿ-ಯು-ತ್ತಿ-ದ್ದರು
ತಿಳಿ-ಯು-ತ್ತೇನೆ
ತಿಳಿ-ಯುವ
ತಿಳಿ-ಯು-ವಂ-ತಹ
ತಿಳಿ-ಯು-ವಂತಾ
ತಿಳಿ-ಯು-ವಂ-ತಾ-ಯಿತು
ತಿಳಿ-ಯು-ವಂ-ತಿ-ರ-ಲಿಲ್ಲ
ತಿಳಿ-ಯು-ವಂತೆ
ತಿಳಿ-ಯು-ವ-ವ-ರೆಗೂ
ತಿಳಿ-ಯು-ವುದು
ತಿಳಿ-ವ-ಳಿ-ಕ-ಸ್ಥ-ನೆಂದೂ
ತಿಳಿ-ವ-ಳಿ-ಕ-ಸ್ಥ-ರಾ-ದ-ವ-ರಿಗೆ
ತಿಳಿ-ವ-ಳಿಕೆ
ತಿಳಿ-ವ-ಳಿ-ಕೆ-ಗಳು
ತಿಳಿ-ವ-ಳಿ-ಕೆ-ಯನ್ನು
ತಿಳಿ-ವ-ಳಿ-ಕೆಯೂ
ತಿಳಿಸ
ತಿಳಿ-ಸ-ಬಾ-ರ-ದೆಂದು
ತಿಳಿ-ಸ-ಬೇಕು
ತಿಳಿ-ಸ-ಬೇ-ಕೆಂದು
ತಿಳಿ-ಸ-ಲಾ-ಗಿತ್ತು
ತಿಳಿ-ಸ-ಲಾಗಿದೆ
ತಿಳಿ-ಸ-ಲಿಲ್ಲ
ತಿಳಿ-ಸಲು
ತಿಳಿ-ಸಲೂ
ತಿಳಿಸಿ
ತಿಳಿ-ಸಿ-ಕೊಟ್ಟ
ತಿಳಿ-ಸಿ-ಕೊ-ಟ್ಟ-ದ್ದ-ರಲ್ಲಿ
ತಿಳಿ-ಸಿ-ಕೊ-ಟ್ಟರೆ
ತಿಳಿ-ಸಿ-ಕೊಟ್ಟು
ತಿಳಿ-ಸಿ-ಕೊ-ಡ-ಬ-ಲ್ಲುದು
ತಿಳಿ-ಸಿ-ಕೊ-ಡ-ಬೇ-ಕಿತ್ತು
ತಿಳಿ-ಸಿ-ಕೊ-ಡಲು
ತಿಳಿ-ಸಿ-ಕೊಡಿ
ತಿಳಿ-ಸಿ-ಕೊಡು
ತಿಳಿ-ಸಿ-ಕೊ-ಡು-ತ್ತಿ-ದ್ದರು
ತಿಳಿ-ಸಿ-ಕೊ-ಡು-ವು-ದರ
ತಿಳಿ-ಸಿ-ಕೊ-ಡು-ವುದು
ತಿಳಿ-ಸಿದ
ತಿಳಿ-ಸಿ-ದಂತೆ
ತಿಳಿ-ಸಿ-ದ-ರಾ-ದರೂ
ತಿಳಿ-ಸಿ-ದರು
ತಿಳಿ-ಸಿ-ದ-ರು-ನಾ-ನಿನ್ನೂ
ತಿಳಿ-ಸಿ-ದ-ರು-ನಾ-ನಿಲ್ಲಿ
ತಿಳಿ-ಸಿ-ದ-ರು-ನಾನು
ತಿಳಿ-ಸಿ-ದ-ರು-ಸ್ವಾಮಿ
ತಿಳಿ-ಸಿ-ದರೆ
ತಿಳಿ-ಸಿ-ದಾಗ
ತಿಳಿ-ಸಿ-ದುದು
ತಿಳಿ-ಸಿ-ದೆ-ನೆಂ-ದರೆ
ತಿಳಿ-ಸಿದ್ದ
ತಿಳಿ-ಸಿ-ದ್ದರು
ತಿಳಿ-ಸಿ-ದ್ದ-ರು-ನಾನು
ತಿಳಿ-ಸಿ-ದ್ದ-ರು-ವಿ-ಷ-ಯ-ವೇನೆಂದರೆ
ತಿಳಿ-ಸಿ-ದ್ದರೂ
ತಿಳಿ-ಸಿದ್ದೆ
ತಿಳಿ-ಸಿ-ದ್ದೇನೆ
ತಿಳಿ-ಸಿ-ರ-ದಿ-ದ್ದುದು
ತಿಳಿ-ಸಿ-ರು-ತ್ತಿ-ದ್ದರು
ತಿಳಿ-ಸಿ-ರುವ
ತಿಳಿ-ಸಿಲ್ಲ
ತಿಳಿಸು
ತಿಳಿ-ಸುತ್ತ
ತಿಳಿ-ಸು-ತ್ತಾನೆ
ತಿಳಿ-ಸು-ತ್ತಾರೆ
ತಿಳಿ-ಸು-ತ್ತಿ-ದ್ದರು
ತಿಳಿ-ಸು-ತ್ತೇನೆ
ತಿಳಿ-ಸುವ
ತಿಳಿ-ಸು-ವಂತೆ
ತಿಳಿ-ಸು-ವಂ-ತೆಯೂ
ತಿಳಿ-ಸು-ವು-ದ-ಕ್ಕಾಗಿ
ತಿಳಿ-ಸು-ವು-ದಾಗಿ
ತಿಳಿ-ಸು-ವುದು
ತಿಳು-ವ-ಳಿ-ಕಸ್ಥ
ತಿಹಾರೋ
ತೀಕ್ಷ
ತೀಕ್ಷ್ಣ
ತೀಕ್ಷ್ಣ-ದೃಷ್ಟಿ
ತೀಕ್ಷ್ಣ-ದೃ-ಷ್ಟಿ-ಯನ್ನು
ತೀಕ್ಷ್ಣ-ವಾಗಿ
ತೀಕ್ಷ್ಣ-ವಾದ
ತೀಕ್ಷ್ಣ-ವಿ-ಮ-ರ್ಶ-ಕ-ರಾದ
ತೀರ
ತೀರಕ್ಕೆ
ತೀರದ
ತೀರ-ದಲ್ಲಿ
ತೀರ-ದ-ಲ್ಲಿದ್ದ
ತೀರ-ದ-ಲ್ಲಿನ
ತೀರ-ದ-ಲ್ಲಿ-ರುವ
ತೀರಾ
ತೀರಿ
ತೀರಿ-ಕೊಂ-ಡಂತೆ
ತೀರಿ-ಕೊಂ-ಡದ್ದೂ
ತೀರಿ-ಕೊಂ-ಡರು
ತೀರಿ-ಸಲು
ತೀರು-ವ-ವರು
ತೀರು-ವ-ವ-ರೆಗೆ
ತೀರ್ಥ
ತೀರ್ಥ-ಕ್ಷೇತ್ರ
ತೀರ್ಥ-ಕ್ಷೇ-ತ್ರಕ್ಕೆ
ತೀರ್ಥ-ಕ್ಷೇ-ತ್ರ-ಗಳ
ತೀರ್ಥ-ಕ್ಷೇ-ತ್ರ-ಗಳನ್ನು
ತೀರ್ಥ-ಕ್ಷೇ-ತ್ರ-ಗಳು
ತೀರ್ಥ-ಕ್ಷೇ-ತ್ರ-ದ-ಲ್ಲಿ-ದ್ದಾಗ
ತೀರ್ಥ-ಕ್ಷೇ-ತ್ರ-ವಾದ
ತೀರ್ಥ-ಯಾತ್ರೆ
ತೀರ್ಥ-ಯಾ-ತ್ರೆಯ
ತೀರ್ಥ-ರಾ-ದರು
ತೀರ್ಥ-ಸ್ಥ-ಳ-ಗಳನ್ನೆಲ್ಲ
ತೀರ್ಪನ್ನು
ತೀರ್ಪು-ಕೊ-ಟ್ಟೇ-ಬಿಟ್ಟ
ತೀರ್ಮಾನ
ತೀರ್ಮಾ-ನಕ್ಕೂ
ತೀರ್ಮಾ-ನಕ್ಕೆ
ತೀರ್ಮಾ-ನ-ಗಳು
ತೀರ್ಮಾ-ನ-ವನ್ನು
ತೀರ್ಮಾ-ನ-ವಾ-ಗಿತ್ತು
ತೀರ್ಮಾ-ನ-ವಾ-ಯಿತು
ತೀರ್ಮಾ-ನಿಸಿ
ತೀರ್ಮಾ-ನಿ-ಸಿ-ದ-ರಲ್ಲ
ತೀರ್ಮಾ-ನಿ-ಸಿ-ದರು
ತೀರ್ಮಾ-ನಿ-ಸಿ-ಬಿ-ಟ್ಟರು
ತೀರ್ಮಾ-ನಿ-ಸಿ-ಬಿ-ಟ್ಟಿ-ದ್ದರು
ತೀರ್ಮಾ-ನಿ-ಸಿ-ಬಿಡ
ತೀರ್ಮಾ-ನಿ-ಸಿ-ಬಿ-ಡು-ತ್ತಿ-ದ್ದ-ರೇನೋ
ತೀರ್ಮಾ-ನಿ-ಸಿ-ಬಿ-ಡುವ
ತೀರ್ಮಾ-ನಿ-ಸು-ತ್ತೀರಿ
ತೀವ್ರ
ತೀವ್ರ-ಗ-ತಿ-ಯಲ್ಲಿ
ತೀವ್ರ-ಗೊ-ಳಿಸಿ
ತೀವ್ರ-ತರ
ತೀವ್ರ-ತ-ರ-ವಾ-ಗಿ-ರು-ತ್ತಿ-ದ್ದುವು
ತೀವ್ರ-ತೆ-ಗಳೇ
ತೀವ್ರ-ತೆ-ಯನ್ನು
ತೀವ್ರ-ತೆ-ಯಿಂದ
ತೀವ್ರ-ತೆ-ಯು-ಳ್ಳ-ವರು
ತೀವ್ರ-ವಾಗಿ
ತೀವ್ರ-ವಾ-ಗಿತ್ತು
ತೀವ್ರ-ವಾದ
ತುಂಡು
ತುಂಬ
ತುಂಬ-ತುಂಬ
ತುಂಬ-ಲಾ-ರಂ-ಭಿ-ಸಿದ
ತುಂಬಲು
ತುಂಬಲೇ
ತುಂಬಾ
ತುಂಬಿ
ತುಂಬಿ-ಕೊಂಡ
ತುಂಬಿ-ಕೊಂ-ಡಂ-ತಿತ್ತು
ತುಂಬಿ-ಕೊಂ-ಡಿತ್ತು
ತುಂಬಿ-ಕೊಂ-ಡಿ-ದ್ದರು
ತುಂಬಿ-ಕೊಂ-ಡಿ-ದ್ದಾರೆ
ತುಂಬಿ-ಕೊಂ-ಡಿ-ದ್ದುವು
ತುಂಬಿ-ಕೊಂ-ಡಿ-ರುವ
ತುಂಬಿ-ಕೊಂಡು
ತುಂಬಿ-ಕೊ-ಡು-ತ್ತಾರೆ
ತುಂಬಿತು
ತುಂಬಿ-ತು-ಳು-ಕು-ತ್ತಿ-ದ್ದುವು
ತುಂಬಿತ್ತು
ತುಂಬಿದ
ತುಂಬಿ-ದರು
ತುಂಬಿ-ದು-ದನ್ನು
ತುಂಬಿದೆ
ತುಂಬಿ-ದೆ-ಯೆಂ-ದರೆ
ತುಂಬಿದ್ದ
ತುಂಬಿ-ದ್ದಂತೆ
ತುಂಬಿ-ದ್ದನ್ನು
ತುಂಬಿ-ದ್ದರು
ತುಂಬಿ-ದ್ದಾ-ರೆಂದು
ತುಂಬಿ-ದ್ದು-ದನ್ನು
ತುಂಬಿ-ದ್ದುವು
ತುಂಬಿ-ಬಂತು
ತುಂಬಿ-ಬಂ-ದಿತು
ತುಂಬಿ-ಬ-ರು-ತ್ತಿತ್ತು
ತುಂಬಿರು
ತುಂಬಿ-ರು-ತ್ತದೆ
ತುಂಬಿ-ರುವ
ತುಂಬಿ-ರು-ವುದು
ತುಂಬಿ-ಸಲು
ತುಂಬಿ-ಸಿಕೊ
ತುಂಬಿ-ಸಿದೆ
ತುಂಬಿ-ಸಿ-ಬಿ-ಡ-ಬೇ-ಕಲ್ಲ
ತುಂಬಿ-ಹೋ-ಗಿತ್ತು
ತುಂಬಿ-ಹೋ-ಗಿ-ದ್ದರು
ತುಂಬಿ-ಹೋ-ದುವು
ತುಂಬು
ತುಂಬು-ಕಂ-ಠ-ದಿಂದ
ತುಂಬು-ತ್ತದೆ
ತುಂಬು-ತ್ತಿ-ದೆಯೆ
ತುಂಬುವ
ತುಂಬು-ವ-ವ-ರೆಗೂ
ತುಂಬು-ವು-ದರ
ತುಂಬೆಲ್ಲ
ತುಚ್ಛ-ವಾಗಿ
ತುಚ್ಛ-ವೆಂದು
ತುಚ್ಛೀ-ಕರಿ
ತುಚ್ಛೀ-ಕ-ರಿಸಿ
ತುಚ್ಛೀ-ಕ-ರಿ-ಸು-ತ್ತಿ-ರುವ
ತುಚ್ಛೀ-ಕ-ರಿ-ಸುವ
ತುಟಿ
ತುಟಿ-ಪಿ-ಟ-ಕ್ಕೆ-ನ್ನದೆ
ತುಟಿ-ಪಿ-ಟಿ-ಕ್ಕೆ-ನ್ನದೆ
ತುಟಿ-ಯಲ್ಲಿ
ತುಡಿ-ಯು-ವಂತೆ
ತುಣು-ಕಾ-ದರೂ
ತುಣುಕು
ತುಣು-ಕು-ಗಳ
ತುಣು-ಕು-ಗಳನ್ನು
ತುಣು-ಕು-ಗಳನ್ನೂ
ತುಣು-ಕು-ಗಳು
ತುತ್ತ-ತು-ದಿಗೆ
ತುತ್ತ-ತು-ದಿ-ಗೇರಿ
ತುತ್ತ-ತು-ದಿಯ
ತುತ್ತ-ತು-ದಿ-ಯ-ಲ್ಲಿದ್ದೆ
ತುತ್ತನ್ನೂ
ತುತ್ತಾಗಿ
ತುತ್ತಾ-ಗು-ವು-ದ-ರಿಂದ
ತುತ್ತಿನ
ತುತ್ತು
ತುದಿ
ತುದಿ-ಯನ್ನು
ತುದಿ-ಯಲ್ಲಿ
ತುದಿ-ಯಾದ
ತುದಿ-ಯಿಂದ
ತುದಿ-ಯುಳ್ಳ
ತುಪಾ-ಕಿ-ಯಿಂದ
ತುಪ್ಪ
ತುಮು-ಲ-ಗಳನ್ನು
ತುಮು-ಲ-ವನ್ನೂ
ತುಮ್
ತುಯ್ದಾ-ಡು-ತ್ತಿ-ರುವ
ತುರೀ-ಯಾ-ನಂದ
ತುರೀ-ಯಾ-ನಂ-ದ-ರನ್ನು
ತುರೀ-ಯಾ-ನಂ-ದ-ರಿಗೆ
ತುರೀ-ಯಾ-ನಂ-ದ-ರಿಗೇ
ತುರೀ-ಯಾ-ನಂ-ದರು
ತುರೀ-ಯಾ-ನಂ-ದರೂ
ತುರು-ಕಲು
ತುರು-ಕು-ವು-ದಲ್ಲ
ತುರ್ತಾಗಿ
ತುರ್ತಿನ
ತುಲ-ನಾ-ತ್ಮಕ
ತುಲ-ಸೀ-ದಾಸ
ತುಳಿ-ತಕ್ಕೆ
ತುಳು-ಕಾ-ಡಿತು
ತುಷಾ-ರಾ-ವೃತ
ತೂಕ
ತೂಕದ
ತೂಕ-ವಾದ
ತೂಗದೆ
ತೂಗಾ-ಡುತ್ತ
ತೂಗಿ-ನೋ-ಡಿ-ದಳು
ತೂಗು-ತ್ತಿತ್ತು
ತೂಗು-ಹಾ-ಕಲು
ತೂರಿ
ತೂರು-ವುದು
ತೂರು-ವುದೂ
ತೂಲಿ-ಕಾ-ತ-ಲ್ಪ-ದಂ-ತಹ
ತೃಣ
ತೃಣಕ್ಕೆ
ತೃಣ-ಮಾ-ತ್ರ-ವಾ-ದರೂ
ತೃಪ್ತ-ನಾ-ಗಿ-ದ್ದೇನೆ
ತೃಪ್ತ-ರಾ-ಗಲು
ತೃಪ್ತ-ರಾಗಿ
ತೃಪ್ತ-ರಾ-ದರು
ತೃಪ್ತಿ-ಕರ
ತೃಪ್ತಿ-ಕ-ರ-ವಾಗಿ
ತೃಪ್ತಿ-ಕ-ರ-ವಾ-ಗಿತ್ತು
ತೃಪ್ತಿ-ಕ-ರ-ವಾ-ಗಿವೆ
ತೃಪ್ತಿ-ದಾ-ಯ-ಕ-ವಾಗಿ
ತೃಪ್ತಿ-ಪ-ಡಿ-ಸ-ಬಲ್ಲ
ತೃಪ್ತಿ-ಪ-ಡಿ-ಸ-ಲಾ-ಗದ
ತೃಪ್ತಿ-ಪ-ಡಿ-ಸಲು
ತೃಪ್ತಿ-ಪ-ಡಿ-ಸು-ತ್ತಿ-ದ್ದರು
ತೃಪ್ತಿ-ಪ-ಡಿ-ಸು-ವಂ-ತಿ-ರ-ಬೇ-ಕಾ-ಗು-ತ್ತದೆ
ತೃಪ್ತಿ-ಯನ್ನೂ
ತೃಪ್ತಿ-ಯಾ-ಗ-ಲಿಲ್ಲ
ತೃಪ್ತಿ-ಯಾ-ಗಿತ್ತು
ತೃಪ್ತಿ-ಯಾ-ಗಿದೆ
ತೃಪ್ತಿ-ಯಾ-ಗಿ-ರ-ಬೇಕು
ತೃಪ್ತಿ-ಯಾ-ಗು-ವಂ-ಥ-ದ್ದೇನು
ತೃಪ್ತಿ-ಯಿದೆ
ತೃಪ್ತಿ-ಯಿ-ರು-ವು-ದಿಲ್ಲ
ತೃಷೆ-ಯನ್ನು
ತೃಷ್ಣೆ-ಗಳನ್ನು
ತೆಂದರೆ
ತೆಂದು
ತೆಗ-ಳಿ-ಕೆ-ಯಿಂ-ದಲ್ಲ
ತೆಗ-ಳು-ವುದು
ತೆಗೆ-ದರು
ತೆಗೆ-ದರೆ
ತೆಗೆ-ದಿ-ಟ್ಟರು
ತೆಗೆದು
ತೆಗೆ-ದುಕೊ
ತೆಗೆ-ದು-ಕೊಂಡ
ತೆಗೆ-ದು-ಕೊಂ-ಡಂತೆ
ತೆಗೆ-ದು-ಕೊಂ-ಡ-ರ-ಲ್ಲದೆ
ತೆಗೆ-ದು-ಕೊಂ-ಡರು
ತೆಗೆ-ದು-ಕೊಂ-ಡ-ವ-ರಂತೆ
ತೆಗೆ-ದು-ಕೊಂ-ಡಿತು
ತೆಗೆ-ದು-ಕೊಂ-ಡಿ-ದ್ದ-ರಿಂದ
ತೆಗೆ-ದು-ಕೊಂ-ಡಿ-ದ್ದರು
ತೆಗೆ-ದು-ಕೊಂ-ಡಿ-ರ-ಲಿಲ್ಲ
ತೆಗೆ-ದು-ಕೊಂ-ಡಿರು
ತೆಗೆ-ದು-ಕೊಂ-ಡಿ-ರು-ತ್ತದೆ
ತೆಗೆ-ದು-ಕೊಂಡು
ತೆಗೆ-ದು-ಕೊಂ-ಡು-ಹೋಗಿ
ತೆಗೆ-ದು-ಕೊಟ್ಟ
ತೆಗೆ-ದು-ಕೊಳ್ಳ
ತೆಗೆ-ದು-ಕೊ-ಳ್ಳ-ಬೇ-ಕಾ-ಗಿಲ್ಲ
ತೆಗೆ-ದು-ಕೊ-ಳ್ಳ-ಬೇಕು
ತೆಗೆ-ದು-ಕೊ-ಳ್ಳ-ಬೇ-ಕೆಂದು
ತೆಗೆ-ದು-ಕೊ-ಳ್ಳ-ಲಿ-ದ್ದು-ದರ
ತೆಗೆ-ದು-ಕೊ-ಳ್ಳ-ಲಿ-ಲ್ಲ-ವಲ್ಲ
ತೆಗೆ-ದು-ಕೊ-ಳ್ಳಲು
ತೆಗೆ-ದು-ಕೊಳ್ಳಿ
ತೆಗೆ-ದು-ಕೊ-ಳ್ಳು-ತ್ತಿದ್ದ
ತೆಗೆ-ದು-ಕೊ-ಳ್ಳು-ತ್ತಿ-ದ್ದರು
ತೆಗೆ-ದು-ಕೊ-ಳ್ಳು-ತ್ತಿ-ದ್ದು-ದ-ರಿಂದ
ತೆಗೆ-ದು-ಕೊ-ಳ್ಳು-ತ್ತಿದ್ದೆ
ತೆಗೆ-ದು-ಕೊ-ಳ್ಳು-ತ್ತೀರಿ
ತೆಗೆ-ದು-ಕೊ-ಳ್ಳು-ತ್ತೇನೆ
ತೆಗೆ-ದು-ಕೊ-ಳ್ಳುವ
ತೆಗೆ-ದು-ಕೊ-ಳ್ಳು-ವಂತೆ
ತೆಗೆ-ದು-ಕೊ-ಳ್ಳು-ವು-ದ-ಲ್ಲದೆ
ತೆಗೆ-ದು-ಕೊ-ಳ್ಳು-ವು-ದಾಗಿ
ತೆಗೆ-ದು-ಕೊ-ಳ್ಳು-ವು-ದಾ-ದರೆ
ತೆಗೆ-ದು-ಕೊ-ಳ್ಳು-ವು-ದಿಲ್ಲ
ತೆಗೆ-ದು-ಕೊ-ಳ್ಳು-ವುದು
ತೆಗೆ-ದುಕೋ
ತೆಗೆ-ದು-ಹಾ-ಕ-ಬೇಕು
ತೆಗೆ-ದು-ಹಾ-ಕಿ-ಸಿದ
ತೆಗೆ-ಯ-ಬ-ಲ್ಲಿರಿ
ತೆಗೆ-ಯ-ಬಲ್ಲೆ
ತೆಗೆ-ಯ-ಬೇ-ಕಾ-ಯಿತು
ತೆಗೆ-ಯ-ಲಾ-ರಂ-ಭಿ-ಸಿ-ದ-ಗೊಂ-ಚಲು
ತೆಗೆ-ಯ-ಲ್ಪಟ್ಟ
ತೆಗೆ-ಸಿ-ಕೊಂಡ
ತೆಗೆ-ಸಿ-ಕೊಂ-ಡರು
ತೆಗೆ-ಸಿ-ಕೊಂ-ಡಿ-ದ್ದರು
ತೆಗೆ-ಸಿ-ಕೊಟ್ಟ
ತೆಗೆ-ಸಿ-ಕೊಡಿ
ತೆತ್ತಿ-ದ್ದೇನೆ
ತೆತ್ತು
ತೆಪ್ಪಗೆ
ತೆರ-ನಾ-ದದ್ದು
ತೆರ-ಬೇಕಾ
ತೆರ-ಳ-ಲಿ-ದ್ದರು
ತೆರ-ಳ-ಲಿ-ದ್ದೀರಿ
ತೆರ-ಳ-ಲಿ-ರು-ವು-ದ-ರಿಂದ
ತೆರಳಿ
ತೆರ-ಳಿದ
ತೆರ-ಳಿ-ದರು
ತೆರ-ಳಿ-ದ-ರೆಂದು
ತೆರ-ಳಿ-ದ್ದರು
ತೆರ-ಳು-ವು-ದಾಗಿ
ತೆರ-ಳು-ವು-ದಾ-ದರೂ
ತೆರ-ವಾ-ಗಿದ್ದ
ತೆರಿಗೆ
ತೆರು-ತ್ತಿ-ರುವ
ತೆರು-ವು-ದ-ರಿಂದ
ತೆರೆ
ತೆರೆ-ಗಳನ್ನು
ತೆರೆದ
ತೆರೆ-ದರು
ತೆರೆ-ದ-ವರು
ತೆರೆ-ದಿತ್ತು
ತೆರೆ-ದಿದೆ
ತೆರೆ-ದಿ-ದ್ದರೂ
ತೆರೆ-ದಿ-ರ-ಲಿಲ್ಲ
ತೆರೆದು
ತೆರೆ-ದು-ಕೊಂ-ಡಿತು
ತೆರೆ-ದು-ಕೊಂಡು
ತೆರೆ-ದು-ಕೊಂ-ಡುವು
ತೆರೆ-ದು-ಕೊ-ಳ್ಳು-ತ್ತಿ-ರು-ವುದನ್ನು
ತೆರೆ-ದು-ತೋ-ರಿತು
ತೆರೆ-ದು-ತೋ-ರು-ತ್ತದೆ
ತೆರೆ-ದುವು
ತೆರೆ-ಮ-ನ-ಸ್ಸಿನ
ತೆರೆಯ
ತೆರೆ-ಯ-ದಿ-ದ್ದರೂ
ತೆರೆ-ಯ-ಬ-ಹುದೊ
ತೆರೆ-ಯಲು
ತೆರೆ-ಯಾಗಿ
ತೆರೆ-ಯಿತು
ತೆರೆ-ಯಿ-ತೆಂದು
ತೆರೆ-ಯಿ-ರೆ-ನ-ಗೆ-ತೆ-ರೆ-ಯ-ಲೇ-ಬೇ-ಕೆಂ-ದಿಗೆ
ತೆರೆಯು
ತೆರೆ-ಯುತ್ತ
ತೆರೆ-ಯು-ತ್ತಿತ್ತು
ತೆರೆ-ಯು-ತ್ತಿದೆ
ತೆರೆ-ಯು-ತ್ತಿದ್ದೆ
ತೆರೆ-ಯು-ವುದು
ತೆರೆ-ಯೋಣ
ತೆರೆಸಿ
ತೆರೆ-ಸಿದ
ತೆರೆ-ಸಿ-ದಿರಿ
ತೆಲುಗು
ತೇಜಃ-ಪುಂಜ
ತೇಜಃ-ಪುಂ-ಜ-ವಾ-ಗಿ-ರು-ತ್ತಿತ್ತು
ತೇಜಸ್ವಿ
ತೇಜ-ಸ್ಸಿ-ನಿಂದ
ತೇಜಸ್ಸು
ತೇಜೋ-ನ್ವಿತ
ತೇಜೋ-ಮಯ
ತೇಜೋ-ವಧೆ
ತೇಜೋ-ವ-ಲ-ಯ-ದ-ಲ್ಲಿ-ದ್ದ-ವರು
ತೇಪೆ-ಗಳ
ತೇಯುವ
ತೇಲಾ-ಡುವ
ತೇಲಿ
ತೇಲಿತು
ತೇಲಿ-ಬಂದು
ತೇಲಿ-ಬಿ-ಟ್ಟಾ-ಗಿದೆ
ತೇಲುತ್ತ
ತೇಲು-ತ್ತಿ-ದ್ದರು
ತೇಲು-ತ್ತಿ-ರೋ-ಣ-ವೆಂದು
ತೇಲುವ
ತೊಂದರೆ
ತೊಂದ-ರೆ-ಗ-ಳಿಗೂ
ತೊಂದ-ರೆ-ಗ-ಳಿಗೆ
ತೊಂದ-ರೆ-ಗಳು
ತೊಂದ-ರೆ-ಯನ್ನು
ತೊಂದ-ರೆ-ಯಾ-ಗ-ದಂತೆ
ತೊಂದ-ರೆ-ಯಾ-ಗದು
ತೊಂದ-ರೆ-ಯಾ-ಗು-ತ್ತಿತ್ತು
ತೊಂದ-ರೆ-ಯಾ-ಗು-ವು-ದಿ-ಲ್ಲ-ವೆಂದು
ತೊಂದ-ರೆ-ಯಿ-ತ್ತು-ಶಿ-ಕಾ-ಗೋ-ದಲ್ಲಿ
ತೊಂದ-ರೆಯೂ
ತೊಂಬತ್ತು
ತೊಗ-ಟೆ-ಯ-ನ್ನೆಲ್ಲ
ತೊಟ್ಟ
ತೊಟ್ಟ-ವರು
ತೊಟ್ಟಿದ್ದ
ತೊಟ್ಟಿ-ಲು-ಎಂದು
ತೊಟ್ಟು
ತೊಟ್ಟು-ಕೊಂ-ಡದ್ದೇ
ತೊಟ್ಟು-ಕೊಂಡು
ತೊಡ-ಕಿನ
ತೊಡ-ಗ-ಬಾ-ರದು
ತೊಡಗಿ
ತೊಡ-ಗಿದ
ತೊಡ-ಗಿ-ದಂತೆ
ತೊಡ-ಗಿ-ದರು
ತೊಡ-ಗಿ-ದಳು
ತೊಡ-ಗಿ-ದಾಗ
ತೊಡ-ಗಿ-ದಾ-ಗ-ಲೆಲ್ಲ
ತೊಡ-ಗಿ-ದಾಗಿ
ತೊಡ-ಗಿ-ದುವು
ತೊಡ-ಗಿದೆ
ತೊಡ-ಗಿ-ದೆವು
ತೊಡ-ಗಿದ್ದ
ತೊಡ-ಗಿ-ದ್ದಳು
ತೊಡ-ಗಿ-ದ್ದಾರೆ
ತೊಡ-ಗಿ-ದ್ದುವು
ತೊಡ-ಗಿ-ದ್ದೇನೆ
ತೊಡ-ಗಿ-ರ-ಬೇ-ಕಾದ
ತೊಡ-ಗಿ-ರು-ತ್ತಾರೆ
ತೊಡ-ಗಿ-ರು-ತ್ತಿದ್ದ
ತೊಡ-ಗಿ-ರು-ತ್ತಿ-ದ್ದರು
ತೊಡ-ಗಿ-ರು-ವುದನ್ನು
ತೊಡ-ಗಿ-ಸಿ-ಕೊಳ್ಳಿ
ತೊಡ-ಗಿ-ಸಿ-ಕೊ-ಳ್ಳು-ತ್ತಿ-ದ್ದಾರೆ
ತೊಡ-ಗಿಸು
ತೊಡ-ಗಿ-ಸು-ವುದು
ತೊಡ-ಗು-ತ್ತಾ-ರೆಯೋ
ತೊಡ-ಗುತ್ತಿ
ತೊಡ-ಗು-ತ್ತಿತ್ತು
ತೊಡ-ಗು-ತ್ತಿದ್ದ
ತೊಡ-ಗು-ತ್ತಿ-ದ್ದರು
ತೊಡ-ಗು-ವು-ದ-ರಿಂದ
ತೊಡ-ಗು-ವುದೆ
ತೊಡ-ಬ-ಹುದು
ತೊಡ-ರಿ-ಸಲು
ತೊಡೆ-ದು-ಹಾ-ಕಿ-ದ್ದ-ರೆಂ-ದರೆ
ತೊಡೆಯ
ತೊದ-ಲ-ಲೇ-ಬೇಕು
ತೊದ-ಲಿತು
ತೊಯ್ದಿತ್ತು
ತೊಯ್ದು-ಹೋ-ದುವು
ತೊರೆ-ದಿ-ದ್ದರು
ತೊರೆದು
ತೊರೆ-ಯು-ವು-ದಿಲ್ಲ
ತೊಲ-ಗಿತು
ತೊಲ-ಗಿ-ಹೋಗಿ
ತೊಳಲ
ತೊಳ-ಲಾ-ಟದ
ತೊಳಲಿ
ತೊಳ-ಲು-ತ್ತಿ-ದ್ದಾಗ
ತೊಳ-ಲು-ತ್ತಿ-ದ್ದು-ದರ
ತೊಳ-ಲು-ತ್ತಿವೆ
ತೊಳೆದು
ತೊಳೆ-ಯ-ಬೇ-ಕಾ-ದ-ವರು
ತೊಳೆ-ಯಲು
ತೊಳೆ-ಯುವ
ತೊಳೆಸಿ
ತೋ
ತೋಚದೆ
ತೋಚ-ಬೇ-ಡವೆ
ತೋಚ-ಲಿಲ್ಲ
ತೋಚಿ-ದಂತೆ
ತೋಟ
ತೋಟ-ಗಳು
ತೋಟದ
ತೋಡಿ
ತೋಡಿ-ಕೊಂ-ಡರು
ತೋಡಿ-ಕೊ-ಳ್ಳು-ತ್ತಿ-ದ್ದರು
ತೋಡುವ
ತೋಡು-ವು-ದೇಕೆ
ತೋಯಿ-ಸಿತು
ತೋಯಿ-ಸಿವೆ
ತೋರ
ತೋರದು
ತೋರದೆ
ತೋರ-ಬ-ಲ್ಲ-ವನು
ತೋರ-ಬ-ಹುದು
ತೋರ-ಬೇ-ಕಾ-ಗಿದೆ
ತೋರ-ಬೇ-ಕಾ-ದ-ವನೂ
ತೋರ-ಬೇಕು
ತೋರ-ಬೇ-ಕೆಂದು
ತೋರ-ಲಾ-ರ-ರೆಂದು
ತೋರಲಿ
ತೋರ-ಲಿಲ್ಲ
ತೋರ-ಲಿ-ಲ್ಲ-ವಲ್ಲ
ತೋರಾ-ಣಿ-ಕೆ-ಯ-ದಾಗಿ
ತೋರಾ-ಣಿ-ಕೆ-ಯ-ದಾ-ಗಿಯೋ
ತೋರಾ-ಣಿ-ಕೆ-ಯದು
ತೋರಿ
ತೋರಿ-ಕೆಯ
ತೋರಿ-ಕೆಯು
ತೋರಿ-ತಾ-ದರೂ
ತೋರಿತು
ತೋರಿ-ತು-ಅ-ವರು
ತೋರಿದ
ತೋರಿ-ದರು
ತೋರಿ-ದರೂ
ತೋರಿ-ದ-ವರ
ತೋರಿ-ದ-ವ-ರಿ-ಗೆಲ್ಲ
ತೋರಿ-ದ್ದರೂ
ತೋರಿ-ದ್ದವು
ತೋರಿ-ದ್ದುವು
ತೋರಿ-ರ-ಲೇ-ಬೇಕು
ತೋರಿ-ರುವ
ತೋರಿ-ಸ-ಬ-ಲ್ಲೆ-ನಾ-ದರೆ
ತೋರಿ-ಸ-ಬಾ-ರದು
ತೋರಿ-ಸ-ಲಾ-ಯಿತು
ತೋರಿ-ಸಲು
ತೋರಿಸಿ
ತೋರಿ-ಸಿ-ಕೊಟ್ಟ
ತೋರಿ-ಸಿ-ಕೊ-ಟ್ಟರು
ತೋರಿ-ಸಿ-ಕೊ-ಟ್ಟರೆ
ತೋರಿ-ಸಿ-ಕೊ-ಟ್ಟ-ವ-ರಲ್ಲಿ
ತೋರಿ-ಸಿ-ಕೊ-ಟ್ಟಾಗ
ತೋರಿ-ಸಿ-ಕೊ-ಟ್ಟಿತು
ತೋರಿ-ಸಿ-ಕೊ-ಟ್ಟಿ-ದ್ದರೆ
ತೋರಿ-ಸಿ-ಕೊ-ಟ್ಟಿ-ದ್ದಾನೆ
ತೋರಿ-ಸಿ-ಕೊ-ಟ್ಟಿ-ದ್ದಾರೆ
ತೋರಿ-ಸಿ-ಕೊಟ್ಟು
ತೋರಿ-ಸಿ-ಕೊಡ
ತೋರಿ-ಸಿ-ಕೊ-ಡಲು
ತೋರಿ-ಸಿ-ಕೊಡು
ತೋರಿ-ಸಿ-ಕೊ-ಡು-ತ್ತಿ-ದ್ದಾರೆ
ತೋರಿ-ಸಿ-ಕೊ-ಡು-ತ್ತೇನೆ
ತೋರಿ-ಸಿ-ಕೊ-ಡು-ವುದನ್ನು
ತೋರಿ-ಸಿ-ಕೊ-ಡು-ವು-ದ-ಲ್ಲದೆ
ತೋರಿ-ಸಿ-ಕೊ-ಡು-ವು-ದಾ-ಗಿತ್ತು
ತೋರಿ-ಸಿ-ಕೊ-ಳ್ಳು-ತ್ತಾರೆ
ತೋರಿ-ಸಿ-ಕೊ-ಳ್ಳು-ತ್ತಿ-ದ್ದರೂ
ತೋರಿ-ಸಿತು
ತೋರಿ-ಸಿದ
ತೋರಿ-ಸಿ-ದಂ-ತಹ
ತೋರಿ-ಸಿ-ದರು
ತೋರಿ-ಸಿ-ದರೆ
ತೋರಿ-ಸಿ-ದ-ವನು
ತೋರಿ-ಸಿ-ದಾಗ
ತೋರಿ-ಸುತ್ತ
ತೋರಿ-ಸು-ತ್ತದೆ
ತೋರಿ-ಸು-ತ್ತಿ-ದ್ದಾರೆ
ತೋರಿ-ಸು-ತ್ತಿ-ದ್ದೇನೆ
ತೋರಿ-ಸು-ತ್ತೇನೆ
ತೋರಿ-ಸುವ
ತೋರಿ-ಸು-ವಂತೆ
ತೋರಿ-ಸು-ವಂ-ತೆಯೂ
ತೋರಿ-ಸು-ವು-ದ-ಕ್ಕಿಂತ
ತೋರಿ-ಸು-ವು-ದರ
ತೋರಿ-ಸು-ವು-ದ-ರಲ್ಲೇ
ತೋರಿ-ಸು-ವು-ದಿಲ್ಲ
ತೋರಿ-ಸು-ವುದೇ
ತೋರು
ತೋರು-ತ್ತದೆ
ತೋರು-ತ್ತಾನೆ
ತೋರು-ತ್ತಾರೆ
ತೋರು-ತ್ತಾ-ರೆಂದು
ತೋರು-ತ್ತಿತ್ತು
ತೋರು-ತ್ತಿದೆ
ತೋರು-ತ್ತಿ-ರ-ಲಿಲ್ಲ
ತೋರು-ತ್ತಿ-ರುವ
ತೋರು-ತ್ತಿವೆ
ತೋರುವ
ತೋರು-ವಂತೆ
ತೋರು-ವು-ದಿಲ್ಲ
ತೋರ್ಪ-ಡಿಸಿ
ತೋರ್ಪ-ಡಿ-ಸಿ-ಕೊ-ಳ್ಳದೆ
ತೋರ್ಪ-ಡಿ-ಸು-ತ್ತಿ-ದ್ದರು
ತೋಳನ್ನು
ತೋಳಿನ
ತ್ತದೆ
ತ್ತದೆ-ಅಲ್ಲಿ
ತ್ತದೆಯೆ
ತ್ತದೆಯೋ
ತ್ತಮ
ತ್ತಲೂ
ತ್ತಲ್ಲ
ತ್ತವೆ
ತ್ತವೆ-ಯೆಂ-ಬು-ದನ್ನು
ತ್ತಾನೆ
ತ್ತಾರೆ
ತ್ತಾರೆಂ-ದರೆ
ತ್ತಾರೆಂದು
ತ್ತಾರೆ-ಅ-ದೊಂದು
ತ್ತಾರೆಯೋ
ತ್ತಾರೆ-ರಾ-ಜರ
ತ್ತಾರೋ
ತ್ತಾಳೆ
ತ್ತಿತ್ತು
ತ್ತಿತ್ತು-ಅ-ಮೆ-ರಿ-ಕೆ-ಯಂ-ತಹ
ತ್ತಿದೆ
ತ್ತಿದೆಯೆ
ತ್ತಿದ್ದ
ತ್ತಿದ್ದಂತೆ
ತ್ತಿದ್ದನೋ
ತ್ತಿದ್ದರು
ತ್ತಿದ್ದರೂ
ತ್ತಿದ್ದರೆ
ತ್ತಿದ್ದರೋ
ತ್ತಿದ್ದಳು
ತ್ತಿದ್ದ-ವ-ರಿಗೆ
ತ್ತಿದ್ದಷ್ಟೇ
ತ್ತಿದ್ದಾರೆ
ತ್ತಿದ್ದಾ-ರೆ-ಮೊ-ದ-ಲ-ನೆ-ಯ-ದಾಗಿ
ತ್ತಿದ್ದಾ-ರೆಯೋ
ತ್ತಿದ್ದಾಳೋ
ತ್ತಿದ್ದಿರಿ
ತ್ತಿದ್ದು
ತ್ತಿದ್ದುದು
ತ್ತಿದ್ದುವು
ತ್ತಿದ್ದೆವು
ತ್ತಿದ್ದೇವೆ
ತ್ತಿರ-ಬೇಕು
ತ್ತಿರ-ಲಿಲ್ಲ
ತ್ತಿರು-ತ್ತದೆ
ತ್ತಿರು-ತ್ತಾ-ನೆಯೋ
ತ್ತಿರುವ
ತ್ತಿರು-ವಂತೆ
ತ್ತಿರು-ವ-ವರು
ತ್ತಿರು-ವುದನ್ನು
ತ್ತಿರು-ವು-ದಾಗಿ
ತ್ತಿಲ್ಲ
ತ್ತೀಯೆ
ತ್ತೀರಾ
ತ್ತೀರಿ
ತ್ತೇನೆ
ತ್ತೇನೆ-ಮೊ-ದಲು
ತ್ತೇನೆಯೇ
ತ್ತೇವೆ
ತ್ಥಾನದ
ತ್ಮಕ-ವಾದ
ತ್ಮಿಕ-ತೆಯು
ತ್ಯಜಿ-ಸ-ಬೇಕು
ತ್ಯಜಿ-ಸ-ಲೇ-ಬೇಕು
ತ್ಯಜಿಸಿ
ತ್ಯಜಿ-ಸಿ-ದಂ-ತಲ್ಲ
ತ್ಯಜಿ-ಸಿ-ದರೂ
ತ್ಯಜಿ-ಸಿ-ದ್ದ-ರಿಂ-ದಲೇ
ತ್ಯಜಿ-ಸಿ-ಬಿ-ಡ-ಬ-ಹುದು
ತ್ಯಜಿ-ಸಿ-ಬಿಡು
ತ್ಯಜಿ-ಸಿ-ಬಿ-ಡು-ತ್ತಾ-ರೆಯೋ
ತ್ಯಜಿ-ಸಿ-ಬಿ-ಡು-ವಂ-ತೆಯೂ
ತ್ಯಜಿಸು
ತ್ಯಜಿ-ಸು-ವುದು
ತ್ಯಾಗ
ತ್ಯಾಗ-ಯೋ-ಗ-ಗಳನ್ನು
ತ್ಯಾಗ-ವೈ-ರಾ-ಗ್ಯ-ಶ-ರ-ಣಾ-ಗ-ತಿ-ಗಳು
ತ್ಯಾಗ-ವೈ-ರಾ-ಗ್ಯಕ್ಕೆ
ತ್ಯಾಗ-ವೈ-ರಾ-ಗ್ಯ-ಗಳ
ತ್ಯಾಗ-ವೈ-ರಾ-ಗ್ಯದ
ತ್ಯಾಗ-ಕ್ಕಾಗಿ
ತ್ಯಾಗಕ್ಕೆ
ತ್ಯಾಗ-ಜೀ-ವಿ-ಗ-ಳಿ-ಗಾಗಿ
ತ್ಯಾಗದ
ತ್ಯಾಗ-ಬು-ದ್ಧಿಯ
ತ್ಯಾಗ-ಭೂ-ಮಿ-ಯಿಂದ
ತ್ಯಾಗ-ಮ-ನೋ-ಭಾವ
ತ್ಯಾಗ-ಮಾಡಿ
ತ್ಯಾಗ-ವನ್ನೇ
ತ್ಯಾಗ-ವಿ-ಲ್ಲದೆ
ತ್ಯಾಗ-ವೆಂದರೆ
ತ್ಯಾಗ-ವೆಲ್ಲ
ತ್ಯಾಗೀ
ತ್ರಯೀ
ತ್ರಾಣ-ವಿ-ರ-ಲಿಲ್ಲ
ತ್ರಿಗು-ಣಾ-ತೀತಾ
ತ್ರಿಗು-ಣಾ-ತೀ-ತಾ-ನಂ-ದ-ರನ್ನು
ತ್ರಿಗು-ಣಾ-ತೀ-ತಾ-ನಂ-ದರು
ತ್ರಿಗು-ಣಾ-ತೀ-ತಾ-ನಂ-ದ-ರುಈ
ತ್ರಿಚೂ-ರಿನ
ತ್ರಿಚೂ-ರಿ-ನಲ್ಲಿ
ತ್ರಿಪಾಠಿ
ತ್ರಿಪಾ-ಠಿಗೆ
ತ್ರಿಪಾ-ಠಿಯ
ತ್ರಿಪಾ-ಠಿ-ಯೊಂ-ದಿಗೆ
ತ್ರಿಪು-ರಾಂ-ತ-ಕ-ರಾ-ದರು
ತ್ರಿಲೋ-ಕ-ಗ-ಳ-ಲ್ಲಿಯೂ
ತ್ರಿವಿ-ಕ್ರ-ಮ-ನಂತೆ
ತ್ರಿವಿ-ಕ್ರ-ಮಾ-ಕಾ-ರ-ವಾಗಿ
ತ್ರಿಶಂಕು
ತ್ವಮಸಿ
ತ್ವರಿ-ತದ
ಥಂಡಿ
ಥರ್ಸ್ಬಿ
ಥಳ-ಥ-ಳನೆ
ಥಳು-ಕಿಗೆ
ಥಳು-ಕಿನ
ಥಳು-ಕು-ಭಿ-ನ್ನಾ-ಣ-ಗಳು
ಥಾಮಸ್
ಥಿಯ-ಸಾ-ಫಿ-ಕಲ್
ಥಿಯ-ಸೊ-ಫಿ-ಸ್ಟ-ರಿಗೆ
ಥಿಯ-ಸೊ-ಫಿ-ಸ್ಟ-ರಿ-ಗೆಲ್ಲ
ಥಿಯಾ-ಸೊ-ಫಿ-ಸ್ಟ-ನಾ-ಗಿದ್ದ
ಥಿಯಾ-ಸೊ-ಫಿ-ಸ್ಟರು
ಥಿಯಾ-ಸೊ-ಫಿ-ಸ್ಟರೂ
ಥಿಯೇ-ಟರ್
ಥಿಯೇ-ಟರ್ನ
ಥಿಯೊ-ಸಾ-ಫಿ-ಕಲ್
ಥೂ
ಥೆರೆ-ಸಾ-ಹೀಗೆ
ಥೇಮ್ಸ್
ಥೇಯರೂ
ಥೈಲಿ-ಯಿಂ-ದಲ್ಲ
ಥ್ಯಾಂಕ್
ದಂಗಾಗಿ
ದಂಗಾ-ಗು-ತ್ತಿ-ದ್ದರು
ದಂಗಾದ
ದಂಗಾ-ದರು
ದಂಗು
ದಂಗು-ಬ-ಡಿದು
ದಂಗು-ಬ-ಡಿ-ದು-ಹೋದ
ದಂಗು-ಬಿ-ಡಿದು
ದಂಗೆ-ಯನ್ನು
ದಂಗೆ-ಯೆ-ದ್ದರು
ದಂಡ
ದಂಡ-ಕ-ಮಂ-ಡಲು
ದಂಡ-ಕ-ಮಂ-ಡ-ಲು-ಗಳನ್ನೂ
ದಂಡ-ನಾ-ಯ-ಕ-ನಾದ
ದಂಡ-ವನ್ನು
ದಂಡಿ-ಮೇ-ನೆ-ಗಳನ್ನು
ದಂತ-ಪಂಕ್ತಿ
ದಂತಾಗಿ
ದಂತಾ-ಗು-ತ್ತದೆ
ದಂತಾ-ಯಿತು
ದಂತೆ
ದಂತೆಯೇ
ದಂತೆಲ್ಲ
ದಂಥ-ವರೂ
ದಂಪತಿ
ದಂಪ-ತಿ-ಗಳ
ದಂಪ-ತಿ-ಗ-ಳಂತೂ
ದಂಪ-ತಿ-ಗಳನ್ನು
ದಂಪ-ತಿ-ಗ-ಳಿಗೆ
ದಂಪ-ತಿ-ಗ-ಳಿ-ಬ್ಬ-ರಿಗೂ
ದಂಪ-ತಿ-ಗ-ಳಿ-ಬ್ಬರೂ
ದಂಪ-ತಿ-ಗಳು
ದಂಪ-ತಿ-ಗ-ಳು-ಕ್ಯಾ-ಪ್ಟನ್
ದಂಪ-ತಿ-ಗಳೂ
ದಂಪ-ತಿ-ಗ-ಳೊಂ-ದಿಗೆ
ದಕ್ಕಲ್ಲ
ದಕ್ಕಾ-ಗಿಯೇ
ದಕ್ಕಿ-ಸಿ-ಕೊ-ಡು-ತ್ತಿ-ದ್ದುವು
ದಕ್ಕೆ
ದಕ್ಕೇ
ದಕ್ಷ
ದಕ್ಷಿಣ
ದಕ್ಷಿ-ಣ-ತು-ದಿ-ಯಲ್ಲಿ
ದಕ್ಷಿ-ಣದ
ದಕ್ಷಿ-ಣ-ಭಾ-ರ-ತದ
ದಕ್ಷಿ-ಣೇ-ಶ್ವ-ರದ
ದಕ್ಷಿ-ಣೇ-ಶ್ವ-ರ-ದಂ-ತಹ
ದಕ್ಷಿ-ಣೇ-ಶ್ವ-ರ-ದಂ-ತಿತ್ತು
ದಕ್ಷಿ-ಣೇ-ಶ್ವ-ರ-ದಲ್ಲಿ
ದಗಾ-ಖೋ-ರ-ನಾಗಿ
ದಟ್ಟ
ದಟ್ಟ-ವಾ-ತಾ-ವ-ರ-ಣದ
ದಟ್ಟ-ವಾದ
ದಡಕ್ಕೆ
ದಡದ
ದಡ-ದಿಂದ
ದಢಾ-ರನೆ
ದಣಿದ
ದಣಿ-ದಿ-ತ್ತೆಂ-ದರೆ
ದಣಿ-ದಿ-ದ್ದರು
ದಣಿ-ದಿ-ದ್ದಾರೆ
ದಣಿದು
ದಣಿ-ವಿನ
ದಣಿ-ವಿ-ನಿಂ
ದಣಿ-ವಿಲ್ಲ
ದಣಿವು
ದತ್ತ
ದತ್ತ-ನಷ್ಟೆ
ದತ್ತು
ದತ್ತು-ಪು-ತ್ರ-ನಾದ
ದದ್ದು
ದದ್ದೇ
ದನಿ
ದನಿ-ಗಳು
ದನಿ-ಗಳೂ
ದನಿ-ಯನ್ನೇ
ದನಿ-ಯಲ್ಲಿ
ದನಿ-ಯಾ-ಗಲಿ
ದನಿ-ಯಾ-ಗಿತ್ತು
ದನಿಯು
ದನಿಯೂ
ದನ್ನು
ದನ್ನೂ
ದನ್ನೇ
ದಪ್ಪ-ಕ್ಷ-ರ-ಗಳಲ್ಲಿ
ದಪ್ಪ-ನೆಯ
ದಬಾ-ಯಿ-ಸಿ-ದಾಗ
ದಬ್ಬಾ-ಳಿಕೆ
ದಬ್ಬಾ-ಳಿ-ಕೆಗೆ
ದಬ್ಬಾ-ಳಿ-ಕೆ-ಯನ್ನೂ
ದಬ್ಬಿ-ಬಿ-ಡೋ-ಣವೆ
ದಮ್ಮಯ್ಯಾ
ದಯ
ದಯ-ಮಾಡಿ
ದಯ-ಮಾ-ಡಿ-ಸು-ತ್ತಿ-ದ್ದಾರೆ
ದಯ-ವಿಟ್ಟು
ದಯಾ-ದಾ-ಕ್ಷಿ-ಣ್ಯ-ವಿ-ಲ್ಲದ
ದಯಾ-ಪ-ರತೆ
ದಯಾ-ಪೂ-ರಿತ
ದಯಾ-ಳು-ವಾದ
ದಯಾ-ವಂತ
ದಯೆ
ದಯೆಯೇ
ದರ
ದರಲ್ಲಿ
ದರಿದ್ರ
ದರಿ-ದ್ರ-ದ-ವ-ರೆಂದು
ದರಿ-ದ್ರ-ನಾ-ಗಿ-ರ-ಬ-ಹು-ದು-ಇ-ಗ-ರ್ಜಿಗೆ
ದರಿ-ದ್ರ-ನಾ-ರಾ-ಯ-ಣರ
ದರಿ-ದ್ರ-ನಾ-ರಾ-ಯ-ಣ-ರನ್ನು
ದರಿ-ದ್ರ-ರನ್ನು
ದರಿ-ದ್ರರು
ದರು
ದರು-ಭ-ವ್ಯ-ವಾದ
ದರೂ
ದರೆ
ದರೋ
ದರೋ-ಡೆ-ಕೋ-ರ-ರಾಗಿ
ದರೋ-ಡೆ-ಯನ್ನು
ದರ್ಗಾ-ವನ್ನೂ
ದರ್ಜೆಯ
ದರ್ಬಾ-ರಿನ
ದರ್ವಿಶ್
ದರ್ಶನ
ದರ್ಶ-ನ-ಗಳ
ದರ್ಶ-ನ-ಗಳನ್ನು
ದರ್ಶ-ನದ
ದರ್ಶ-ನ-ದಲ್ಲಿ
ದರ್ಶ-ನ-ಭಾಗ್ಯ
ದರ್ಶ-ನ-ವನ್ನು
ದರ್ಶ-ನ-ವಾ-ಗಿತ್ತು
ದರ್ಶ-ನ-ವಾ-ಗು-ತ್ತಿದೆ
ದರ್ಶ-ನ-ವಾ-ಯಿತು
ದರ್ಶ-ನಾ-ರ್ಥಿ-ಗ-ಳಾಗಿ
ದರ್ಶಿ
ದರ್ಶಿಯ
ದರ್ಶಿ-ಯಿಂದ
ದರ್ಶಿ-ಸಲು
ದಲಿತ
ದಲಿ-ತ-ವರ್ಗ
ದಲೇ
ದಲ್ಲಂತೂ
ದಲ್ಲಿ
ದಲ್ಲಿದ್ದ
ದಲ್ಲಿ-ದ್ದಂ-ತಿತ್ತು
ದಲ್ಲಿ-ದ್ದಾ-ಗಲೇ
ದಲ್ಲಿ-ರು-ವು-ದ-ಕ್ಕಿಂತ
ದಲ್ಲೂ
ದಲ್ಲೆಲ್ಲ
ದಲ್ಲೇ
ದಳವೂ
ದಳು
ದಳ್ಳಾ-ಳಿ-ಗಳು
ದವ-ಡೆ-ಯನ್ನು
ದವ-ಡೆ-ಯಲ್ಲಿ
ದವ-ರಲ್ಲಿ
ದವ-ರಿಗೆ
ದವರು
ದವಾ-ಖಾ-ನೆಯ
ದವಾ-ಖಾ-ನೆ-ಯಿಂದ
ದವು-ಗಳನ್ನು
ದಶ-ಕ-ಗಳೇ
ದಶ-ದಿ-ಶೆಗೂ
ದಷ್ಟು
ದಷ್ಟೂ
ದಹಿ-ಸುತ್ತ
ದಹಿ-ಸು-ತ್ತಿದೆ
ದಹಿ-ಸು-ತ್ತಿ-ರು-ವಾಗ
ದಾಂಡಿ-ನಿಂದ
ದಾಂಡು
ದಾಂಡು-ಚೆಂ-ಡು-ಗಳನ್ನು
ದಾಂಪ-ತ್ಯ-ಭಾವ
ದಾಖ-ಲಾ-ಗದೆ
ದಾಖ-ಲಿ-ಸಿ-ದುವು
ದಾಖಲು
ದಾಖ-ಲೆ-ಗಳನ್ನು
ದಾಖ-ಲೆ-ಯಾಗಿ
ದಾಗ
ದಾಗಲೂ
ದಾಗಿ
ದಾಗಿತ್ತು
ದಾಚೆ-ಗಿನ
ದಾಚೆಗೆ
ದಾಟ-ದಿ-ದ್ದರೆ
ದಾಟ-ಬಾ-ರದು
ದಾಟಲು
ದಾಟ-ಲೇ-ಬೇ-ಕೆಂ-ದು-ಕೊಂ-ಡರು
ದಾಟಿ
ದಾಟಿ-ಕೊಂಡು
ದಾಟಿದ
ದಾಟಿ-ದರು
ದಾಟಿ-ದರೆ
ದಾಟಿ-ದೊ-ಡ-ನೆಯೇ
ದಾಟಿದ್ದೂ
ದಾಟಿಯೇ
ದಾಟಿ-ಸು-ವ-ವನು
ದಾತೃ-ಗಳ
ದಾದ
ದಾದರೂ
ದಾದರೆ
ದಾದಿ-ಯರು
ದಾದ್ಯಂತ
ದಾನ
ದಾನ-ವಾಗಿ
ದಾಪು-ಗಾಲು
ದಾಯ-ಗಳ
ದಾಯ-ಶ-ರ-ಣ-ತೆಯ
ದಾಯ-ಸ್ಥರ
ದಾಯ-ಸ್ಥರು
ದಾಯ-ಹೀಗೆ
ದಾಯಿತ್ವ
ದಾರಿ
ದಾರಿ-ಖ-ರ್ಚನ್ನು
ದಾರಿ-ಖ-ರ್ಚಿಗೆ
ದಾರಿ-ಗಡ್ಡ
ದಾರಿ-ಗ-ಡ್ಡ-ವಾಗಿ
ದಾರಿ-ಗ-ಳೆ-ಲ್ಲವೂ
ದಾರಿಗೆ
ದಾರಿ-ಗೆ-ಳೆ-ಯ-ಬೇಡಿ
ದಾರಿ-ತ-ಪ್ಪಿ-ಸಿ-ಕೊಂ-ಡರೂ
ದಾರಿ-ದೀ-ಪ-ವಾ-ಗ-ಬ-ಲ್ಲುದು
ದಾರಿದ್ರ್ಯ
ದಾರಿ-ದ್ರ್ಯ-ಸಂ-ಕ-ಟ-ಗಳನ್ನು
ದಾರಿ-ದ್ರ್ಯವೇ
ದಾರಿಯ
ದಾರಿ-ಯನ್ನು
ದಾರಿ-ಯನ್ನೂ
ದಾರಿ-ಯಲ್ಲಿ
ದಾರಿ-ಯ-ಲ್ಲಿ-ದ್ದಾ-ಗಲೇ
ದಾರಿ-ಯ-ಲ್ಲಿ-ರುವ
ದಾರಿ-ಯಲ್ಲೇ
ದಾರಿ-ಯಾಗಿ
ದಾರಿ-ಯಾ-ಗು-ತ್ತಿತ್ತು
ದಾರಿ-ಯಿಲ್ಲ
ದಾರಿಯು
ದಾರಿ-ಯು-ದ್ದಕ್ಕೂ
ದಾರಿ-ಯೆಂದು
ದಾರಿಯೇ
ದಾರಿ-ಹಿ-ಡಿ-ದಾ-ಗಿತ್ತು
ದಾರಿ-ಹೋ-ಕ-ರನ್ನು
ದಾರುಣ
ದಾರ್ಢ್ಯ
ದಾರ್ಶ-ನಿಕ
ದಾರ್ಶ-ನಿ-ಕನ
ದಾರ್ಶ-ನಿ-ಕರು
ದಾವಾ
ದಾಸ
ದಾಸ-ನಂತೆ
ದಾಸ-ನ-ನ್ನೆಂ-ದಿಗೂ
ದಾಸ-ನಾ-ದರೆ
ದಾಸರ
ದಾಸ-ರಾ-ಗ-ಬೇ-ಕಾ-ಗಿ-ರ-ಲಿಲ್ಲ
ದಾಸ-ರಾ-ಷ್ಟ್ರ-ವಾದ
ದಾಸರು
ದಾಸ-ಳಾ-ಗಲು
ದಾಸಾ-ನು-ದಾಸ
ದಾಸ್ಗೆ
ದಾಸ್ಯ
ದಾಸ್ಯ-ದಲ್ಲಿ
ದಾಹ-ವಾ-ಗಿತ್ತು
ದಾಹವೂ
ದಿ
ದಿಂದ
ದಿಂದಲೂ
ದಿಂದಲೇ
ದಿಂದಾ-ಗಲಿ
ದಿಂದಾಗಿ
ದಿಂಬನ್ನು
ದಿಕ್ಕನ್ನು
ದಿಕ್ಕನ್ನೇ
ದಿಕ್ಕಾ-ಪಾ-ಲಾಗಿ
ದಿಕ್ಕಿಗೆ
ದಿಕ್ಕಿ-ನಲ್ಲಿ
ದಿಕ್ಕು
ದಿಕ್ಕು-ಗಳಲ್ಲಿ
ದಿಕ್ಕು-ಗಾ-ಣದೆ
ದಿಕ್ಕು-ದಿ-ವಾ-ಣ-ವಿ-ಲ್ಲ-ದವ
ದಿಕ್ಸೂ-ಚಿ-ಯಾ-ದುವು
ದಿಗಂತ
ದಿಗಂ-ತಕ್ಕೆ
ದಿಗಂ-ತ-ದಲ್ಲಿ
ದಿಗಂ-ತ-ದ-ವ-ರೆಗೂ
ದಿಗಂ-ತ-ದಾ-ಚೆಗೆ
ದಿಗಂ-ತ-ವನ್ನು
ದಿಗಂ-ತ-ವನ್ನೇ
ದಿಗಂ-ತ-ವೊಂ-ದನ್ನು
ದಿಗಿ-ಲಾ-ಗ-ದಿ-ರದು
ದಿಗಿ-ಲಾ-ಯಿತು
ದಿಗಿಲು
ದಿಗ್ಗ-ಜ-ಗಳು
ದಿಗ್ದಂ-ತಿ-ಗಳ
ದಿಗ್ಭ್ರಮೆ
ದಿಗ್ಭ್ರ-ಮೆ-ಗೊ-ಳಿ-ಸಿತು
ದಿಗ್ಭ್ರ-ಮೆ-ಗೊ-ಳಿ-ಸು-ವಂ-ತಹ
ದಿಗ್ಭ್ರ-ಮೆ-ಗೊ-ಳಿ-ಸು-ವಂ-ತಿ-ರುವ
ದಿಗ್ಭ್ರ-ಮೆ-ಗೊ-ಳಿ-ಸು-ವಷ್ಟು
ದಿಗ್ಭ್ರ-ಮೆ-ಯಾ-ಯಿತು
ದಿಗ್ಭ್ರಾಂ-ತ-ನಾದ
ದಿಗ್ಭ್ರಾಂ-ತ-ರಾಗಿ
ದಿಗ್ಭ್ರಾಂ-ತ-ರಾ-ದದ್ದು
ದಿಗ್ಭ್ರಾಂ-ತ-ರಾ-ದರು
ದಿಗ್ಭ್ರಾಂ-ತ-ಳಾಗಿ
ದಿಗ್ಭ್ರಾಂತಿ
ದಿಗ್ಭ್ರಾಂ-ತಿ-ಯನ್ನೂ
ದಿಗ್ವಿ-ಜ-ಯದ
ದಿಚ್ಛೆಗೆ
ದಿಟ್ಟ
ದಿಟ್ಟ-ತನ
ದಿಟ್ಟ-ಹೆ-ಜ್ಜೆ-ಗ-ಳ-ನ್ನಿ-ಡುತ್ತ
ದಿಟ್ಟಿ-ದ್ದೇನೆ
ದಿಟ್ಟಿಸಿ
ದಿಟ್ಟಿ-ಸಿದ
ದಿಟ್ಟಿ-ಸಿ-ದರು
ದಿಟ್ಟಿ-ಸುತ್ತ
ದಿಟ್ಟಿ-ಸು-ತ್ತಿ-ದ್ದರು
ದಿಟ್ಟಿ-ಸು-ತ್ತಿ-ದ್ದವು
ದಿದ್ದರೂ
ದಿನ
ದಿನ-ಕಳೆ
ದಿನ-ಕ-ಳೆ-ದಂತೆ
ದಿನಕ್ಕೆ
ದಿನ-ಕ್ಕೊ-ಮ್ಮೆಯೋ
ದಿನ-ಗ-ಟ್ಟಲೆ
ದಿನ-ಗ-ಟ್ಟ-ಲೆ-ಯಿಂದ
ದಿನ-ಗಳ
ದಿನ-ಗಳನ್ನು
ದಿನ-ಗಳಲ್ಲಿ
ದಿನ-ಗ-ಳ-ಲ್ಲಿ-ಕಂ-ಡು-ಬ-ರುವ
ದಿನ-ಗ-ಳಲ್ಲೇ
ದಿನ-ಗ-ಳ-ಲ್ಲೊಮ್ಮೆ
ದಿನ-ಗ-ಳ-ವ-ರೆಗೂ
ದಿನ-ಗ-ಳ-ವ-ರೆಗೆ
ದಿನ-ಗಳಾ
ದಿನ-ಗ-ಳಾ-ಗು-ವ-ಷ್ಟ-ರಲ್ಲಿ
ದಿನ-ಗ-ಳಾದ
ದಿನ-ಗ-ಳಾ-ದರೂ
ದಿನ-ಗಳಿಂದ
ದಿನ-ಗ-ಳಿಂ-ದಲೂ
ದಿನ-ಗ-ಳಿಂ-ದಲೇ
ದಿನ-ಗ-ಳಿ-ಗಂತೂ
ದಿನ-ಗ-ಳಿ-ರು-ವಾಗ
ದಿನ-ಗಳು
ದಿನ-ಗ-ಳು-ರುಳಿ
ದಿನ-ಗ-ಳು-ರು-ಳಿ-ದಂತೆ
ದಿನ-ಗ-ಳು-ರು-ಳಿ-ದವು
ದಿನ-ಗಳೇ
ದಿನ-ಚರಿ
ದಿನದ
ದಿನ-ದಂದು
ದಿನ-ದಲ್ಲಿ
ದಿನ-ದ-ವ-ರೆಗೂ
ದಿನ-ದಿಂದ
ದಿನ-ದಿಂ-ದಲೂ
ದಿನ-ದಿ-ನಕ್ಕೂ
ದಿನ-ದಿ-ನಕ್ಕೆ
ದಿನ-ದಿ-ನವೂ
ದಿನ-ನಿ-ತ್ಯದ
ದಿನ-ಪ-ತ್ರಿಕೆ
ದಿನ-ಪ-ತ್ರಿ-ಕೆ-ಗಳಲ್ಲಿ
ದಿನ-ಪ-ತ್ರಿ-ಕೆ-ಗಳಿಂದ
ದಿನ-ಪ-ತ್ರಿ-ಕೆ-ಯನ್ನು
ದಿನ-ವಂತೂ
ದಿನ-ವನ್ನು
ದಿನ-ವನ್ನೂ
ದಿನ-ವಷ್ಟೇ
ದಿನ-ವಾ-ಗಿತ್ತು
ದಿನ-ವಾ-ದರೂ
ದಿನ-ವಿಡೀ
ದಿನ-ವಿ-ದ್ದರು
ದಿನವು
ದಿನವೂ
ದಿನ-ವೆಂದೋ
ದಿನವೇ
ದಿನ-ವೊಂ-ದರ
ದಿನೇ-ದಿನೇ
ದಿರದು
ದಿರ-ಬ-ಹುದೊ
ದಿರಿ-ಸನ್ನು
ದಿರು-ವು-ದಕ್ಕೆ
ದಿಲ್ಲ
ದಿಲ್ಲವೆ
ದಿಲ್ಲವೋ
ದಿವಂ-ಗತ
ದಿವಾನ
ದಿವಾ-ನನ
ದಿವಾ-ನ-ನನ್ನು
ದಿವಾ-ನ-ನನ್ನೂ
ದಿವಾ-ನ-ನಾದ
ದಿವಾ-ನ-ನಿಗೆ
ದಿವಾ-ನನು
ದಿವಾ-ನನೂ
ದಿವಾ-ನ-ನೊಂ-ದಿಗೆ
ದಿವಾ-ನರ
ದಿವಾ-ನ-ರಂತೂ
ದಿವಾ-ನ-ರನ್ನು
ದಿವಾ-ನ-ರಾದ
ದಿವಾ-ನ-ರಿಂದ
ದಿವಾ-ನ-ರಿಗೂ
ದಿವಾ-ನ-ರಿಗೆ
ದಿವಾ-ನರು
ದಿವಾ-ನರೇ
ದಿವಾ-ನ-ರೊಂ-ದಿಗೆ
ದಿವಾ-ನ-ರೊ-ಬ್ಬರು
ದಿವಾನ್
ದಿವಾ-ನ್ಖಾ-ನೆ-ಗಳು
ದಿವಾನ್ಜಿ
ದಿವಾ-ನ್ಜಿಗೆ
ದಿವಾ-ನ್ಜಿಯ
ದಿವಾ-ನ್ಜಿ-ಯ-ವ-ರಿಂದ
ದಿವಾ-ನ್ಜಿ-ಯ-ವರು
ದಿವ್ಯ
ದಿವ್ಯ-ಕಳೆ
ದಿವ್ಯ-ಕ್ಷ-ಣ-ಗಳನ್ನು
ದಿವ್ಯ-ಜ್ಯೋ-ತಿಯ
ದಿವ್ಯತೆ
ದಿವ್ಯ-ತೇ-ಜ-ಸ್ಸಿ-ನಿಂದ
ದಿವ್ಯ-ದ-ರ್ಶ-ನದ
ದಿವ್ಯ-ಧಾ-ಮದ
ದಿವ್ಯ-ಪ್ರೇ-ಮದ
ದಿವ್ಯ-ಪ್ರೇ-ಮ-ವನ್ನು
ದಿವ್ಯ-ಬೋ-ಧ-ನೆ-ಗ-ಳೆಂ-ಥವು
ದಿವ್ಯ-ಮಾ-ನು-ಷ-ಮೂ-ರ್ತಿಯ
ದಿವ್ಯ-ವಾ-ಣಿಯ
ದಿವ್ಯ-ವಾದ
ದಿವ್ಯ-ಸ-ತ್ಯ-ವನ್ನು
ದಿವ್ಯ-ಸ-ನ್ನಿ-ಧಿ-ಯಲ್ಲಿ
ದಿವ್ಯ-ಸ್ವ-ರೂ-ಪ-ವನ್ನು
ದಿವ್ಯಾ
ದಿವ್ಯಾ-ದ-ರ್ಶ-ದಿಂದ
ದಿವ್ಯಾ-ದ್ಭುತ
ದಿವ್ಯಾನಿ
ದಿವ್ಯಾ-ನು-ಭ-ವ-ಗ-ಳನ್ನೇ
ದಿವ್ಯಾ-ನು-ಭ-ವ-ವನ್ನೇ
ದಿವ್ಯಾ-ಮೃ-ತ-ದಲ್ಲಿ
ದಿವ್ಯೋ-ದ್ದೇ-ಶ-ವೊಂ-ದರ
ದಿಸೆ
ದಿಸೆ-ಯಲ್ಲಿ
ದೀಕ್ಷಾ-ಪ್ರ-ದಾನ
ದೀಕ್ಷಾ-ಬ-ದ್ಧ-ರ-ನ್ನಾಗಿ
ದೀಕ್ಷಾ-ರ್ಥಿ-ಯಿಂ-ದಲೂ
ದೀಕ್ಷೆ
ದೀಕ್ಷೆ-ಯನ್ನು
ದೀಕ್ಷೆ-ಯನ್ನೂ
ದೀನ
ದೀನ-ಆರ್ತ
ದೀನ-ದ-ರಿ-ದ್ರ-ರನ್ನೂ
ದೀನ-ದ-ಲಿ-ತ-ದ-ರಿ-ದ್ರ-ರನ್ನು
ದೀನ-ದ-ರಿ-ದ್ರರ
ದೀನ-ದ-ಲಿ-ತರ
ದೀನ-ದ-ಲಿ-ತ-ರಿ-ಗಾಗಿ
ದೀನರ
ದೀನ-ರನ್ನು
ದೀನರು
ದೀನಾರ್ತ
ದೀನಾ-ರ್ತ-ರಿಗೆ
ದೀಪ-ಗಳ
ದೀಪ-ಗಳಿಂದ
ದೀಪ-ಧಾ-ರಿಯೇ
ದೀಪವೇ
ದೀರ್ಘ
ದೀರ್ಘ-ನಿ-ರಂ-ತರ
ದೀರ್ಘ-ಶುಷ್ಕ
ದೀರ್ಘ-ಕಾಲ
ದೀರ್ಘ-ಕಾ-ಲದ
ದೀರ್ಘ-ಕಾ-ಲ-ದಿಂದ
ದೀರ್ಘ-ತಿ-ಕ್ಕಾಟ
ದೀರ್ಘ-ಧ್ಯಾ-ನ-ದಿಂದ
ದೀರ್ಘ-ಪ-ತ್ರ-ಗಳನ್ನು
ದೀರ್ಘ-ಪ-ತ್ರ-ವೊಂ-ದನ್ನು
ದೀರ್ಘ-ವಾಗಿ
ದೀರ್ಘ-ವಾದ
ದೀರ್ಘಾ-ಲೋ-ಚ-ನೆ-ಯಲ್ಲಿ
ದೀರ್ಘಾ-ವ-ಧಿ-ಯನ್ನು
ದುಂಡ-ನೆಯ
ದುಂದುಭಿ
ದುಂದು-ವೆ-ಚ್ಚ-ದಿಂ-ದಲೂ
ದುಂಬಾಲು
ದುಃಖ
ದುಃಖ-ದು-ಮ್ಮಾ-ನದ
ದುಃಖ-ನಿ-ರಾ-ಶೆ-ಯಾ-ಗು-ತ್ತದೆ
ದುಃಖ-ತ-ಪ್ತ-ರಾಗಿ
ದುಃಖದ
ದುಃಖ-ದಾರಿ-ದ್ರ್ಯ-ಗಳು
ದುಃಖ-ದಿಂದ
ದುಃಖ-ದಿಂ-ದಾಗಿ
ದುಃಖ-ದೊಂ-ದಿಗೆ
ದುಃಖ-ನಿ-ವಾ-ರ-ಣೆ-ಗಾಗಿ
ದುಃಖ-ವನ್ನು
ದುಃಖ-ವನ್ನೂ
ದುಃಖ-ವಾ-ಗು-ತ್ತಿದೆ
ದುಃಖ-ವಾ-ಯಿ-ತಾ-ದರೂ
ದುಃಖ-ವಾ-ಯಿತು
ದುಃಖ-ವಿ-ಲ್ಲದ
ದುಃಖ-ವುಂ-ಟಾ-ಗಿದೆ
ದುಃಖವೂ
ದುಃಖವೇ
ದುಃಖಿ-ಗಳನ್ನು
ದುಃಖಿ-ಗಳು
ದುಃಖಿ-ಸಿ-ದ-ನಂತೆ
ದುಃಖಿ-ಸು-ವು-ದಿಲ್ಲ
ದುಃಸ್ಥಿ-ತಿ-ಗಾಗಿ
ದುಃಸ್ಥಿ-ತಿಗೆ
ದುಃಸ್ಥಿ-ತಿಯ
ದುಃಸ್ಥಿ-ತಿ-ಯನ್ನು
ದುಃಸ್ಥಿ-ತಿ-ಯಿಂದ
ದುಃಸ್ಥಿ-ತಿಯೇ
ದುಡಿ
ದುಡಿ-ದರು
ದುಡಿದು
ದುಡಿ-ದು-ದರ
ದುಡಿ-ಮೆಯ
ದುಡಿ-ಯ-ಬೇ-ಕೆಂ-ದರೆ
ದುಡಿ-ಯ-ಲಿ-ಚ್ಛಿ-ಸುವ
ದುಡಿ-ಯಲು
ದುಡಿ-ಯು-ತ್ತಾನೆ
ದುಡಿ-ಯು-ತ್ತಿ-ದ್ದ-ರಾ-ದರೂ
ದುಡಿ-ಯು-ತ್ತಿ-ದ್ದರು
ದುಡಿ-ಯು-ತ್ತಿ-ರುವ
ದುಡಿ-ಯು-ವಂ-ತಹ
ದುಡಿ-ಯು-ವಂತೆ
ದುಡಿ-ಯು-ವು-ದೆಂ-ದರೆ
ದುಡ್ಡಿ-ನಿಂದ
ದುಡ್ಡಿ-ರ-ಲಿಲ್ಲ
ದುಡ್ಡು
ದುಡ್ಡು-ಕಾಸು
ದುದು
ದುದ್ದಕ್ಕೂ
ದುಬಾರಿ
ದುಬಾ-ರಿ-ಯಾ-ಗಿತ್ತು
ದುಬಾ-ರಿ-ಯಾ-ದು-ದೆಂದು
ದುಭಾಷಿ
ದುಭಾ-ಷಿಯ
ದುಭಾ-ಷಿ-ಯನ್ನು
ದುಭಾ-ಷಿಯೂ
ದುರ-ದೃಷ್ಟ
ದುರ-ದೃ-ಷ್ಟವ
ದುರ-ದೃ-ಷ್ಟ-ವ-ಶಾತ್
ದುರ-ದೃ-ಷ್ಟ-ಶಾತ್
ದುರ-ವ-ಸ್ಥೆಗೂ
ದುರ-ವ-ಸ್ಥೆಗೆ
ದುರಸ್ತಿ
ದುರಾ-ಚಾ-ರ-ಗ-ಳಿಂ-ದಾದ
ದುರಾ-ಚಾ-ರಿ-ಗಳ
ದುರಾ-ಚಾ-ರಿ-ಗ-ಳಿಗೂ
ದುರಾಶೆ
ದುರಾ-ಸೆ-ಯುಂ-ಟಾ-ಗು-ತ್ತದೆ
ದುರಾ-ಸೆಯೂ
ದುರು-ಗು-ಟ್ಟಿ-ಕೊಂಡು
ದುರು-ಗು-ಟ್ಟಿ-ದರು
ದುರು-ದ್ದೇ-ಶ-ದಿಂದ
ದುರು-ದ್ದೇ-ಶ-ಪೂ-ರಿತ
ದುರು-ದ್ದೇ-ಶವೂ
ದುರು-ಪ-ಯೋಗ
ದುರ್ಗ-ಣ-ವನ್ನು
ದುರ್ಗ-ತಿ-ಗಿ-ಳಿದ
ದುರ್ಗ-ತಿ-ಗೀ-ಡಾಗು
ದುರ್ಗ-ತಿಗೆ
ದುರ್ಗ-ವನು
ದುರ್ಗೆ-ಯನ್ನು
ದುರ್ಗೆ-ಯಾದ
ದುರ್ನ-ಡ-ತೆ-ಯನ್ನು
ದುರ್ಬಲ
ದುರ್ಬ-ಲ-ಗೊಂ-ಡಿ-ದೆ-ಯಲ್ಲ
ದುರ್ಬ-ಲ-ಗೊ-ಳಿ-ಸು-ತ್ತ-ದೆ-ಯಷ್ಟೆ
ದುರ್ಬ-ಲತೆ
ದುರ್ಬ-ಲರು
ದುರ್ಬ-ಲ-ರೆಲ್ಲ
ದುರ್ಬ-ಲ-ವಾ-ಗ-ತೊ-ಡ-ಗಿತ್ತು
ದುರ್ಬು-ದ್ಧಿ-ಯಿಂ-ದಾಗಿ
ದುರ್ಭರ
ದುರ್ಭಾ-ವ-ನೆ-ಗಳನ್ನು
ದುರ್ಮ-ರ-ಣ-ಕ್ಕೀ-ಡಾ-ಗಿ-ದ್ದಳು
ದುರ್ಲ-ಭ-ವಾದ
ದುರ್ವ-ರ್ತ-ನೆ-ಯನ್ನು
ದುರ್ವಾ-ಸ-ನೆ-ಯಿಂದ
ದುವು
ದುಷ್ಕೃ-ತ್ಯ-ಕ್ಕಾಗಿ
ದುಷ್ಟ
ದುಷ್ಟ-ತನ
ದುಷ್ಟ-ತ-ನ-ವನ್ನು
ದುಷ್ಟ-ತ-ನ-ವೆಂ-ಥದು
ದುಷ್ಟ-ನಾಗಿ
ದುಷ್ಟರ
ದುಷ್ಟ-ರನ್ನು
ದುಷ್ಟ-ರಿಂದ
ದುಷ್ಟ-ರಿಗೆ
ದುಷ್ಟ-ಶಿ-ಕ್ಷಣ
ದುಷ್ಪ-ರಿ-ಣಾ-ಮ-ವನ್ನು
ದುಸ್ಥಿತಿ
ದುಸ್ಥಿ-ತಿ-ಗ-ಳಿಗೆ
ದುಸ್ಸಾ-ಧ್ಯದ
ದುಸ್ಸಾ-ಧ್ಯ-ವಾದ
ದುಸ್ಸಾ-ಧ್ಯ-ವಾ-ದು-ದೆಂ-ಬು-ದನ್ನು
ದುಹು
ದೂಡು-ತ್ತದೆ
ದೂತ
ದೂತನೂ
ದೂರ
ದೂರ-ಸು-ದೂರ
ದೂರ-ಕ್ಕಟ್ಟಿ
ದೂರಕ್ಕೆ
ದೂರ-ಕ್ಕೆ-ಸೆದು
ದೂರ-ಕ್ಕೆ-ಸೆ-ಯುವ
ದೂರ-ತೊ-ಡ-ಗಿತು
ದೂರದ
ದೂರ-ದ-ರ್ಶ-ಕದ
ದೂರ-ದ-ರ್ಶ-ಕ-ವನ್ನು
ದೂರ-ದಲ್ಲಿ
ದೂರ-ದ-ಲ್ಲಿಟ್ಟ
ದೂರ-ದ-ಲ್ಲಿ-ದೆ-ಯೆಂ-ಬುದು
ದೂರ-ದ-ಲ್ಲಿದ್ದ
ದೂರ-ದ-ಲ್ಲಿ-ದ್ದಾನೆ
ದೂರ-ದ-ಲ್ಲಿ-ರುವ
ದೂರ-ದಲ್ಲೇ
ದೂರ-ದ-ಲ್ಲೊಂದು
ದೂರ-ದಿಂದ
ದೂರ-ದಿಂ-ದ-ಲಾ-ದರೂ
ದೂರ-ದಿಂ-ದಲೇ
ದೂರ-ದೂರ
ದೂರ-ದೂ-ರಕ್ಕೆ
ದೂರ-ದೂ-ರದ
ದೂರ-ದೂ-ರ-ದ-ಲ್ಲಿ-ರುವ
ದೂರ-ದೃ-ಷ್ಟಿ-ಯವ
ದೂರ-ದೃ-ಷ್ಟಿಯೂ
ದೂರ-ದೇ-ಶಕ್ಕೆ
ದೂರ-ದೇ-ಶ-ಗ-ಳಿಗೆ
ದೂರ-ಪ್ರ-ದೇ-ಶ-ಗ-ಳಿಗೂ
ದೂರ-ಬೇ-ಕಾ-ದದ್ದು
ದೂರ-ಮಾ-ಡು-ವುದು
ದೂರ-ವನ್ನು
ದೂರ-ವಾ-ಗ-ಲಿ-ದ್ದಾ-ರೆಂಬ
ದೂರ-ವಾಗಿ
ದೂರ-ವಾ-ಗಿದ್ದ
ದೂರ-ವಾ-ಗಿರ
ದೂರ-ವಾ-ಗಿ-ರ-ಲಿಲ್ಲ
ದೂರ-ವಾ-ಗಿ-ಲ್ಲ-ವೆಂ-ಬಂತೆ
ದೂರ-ವಾಗು
ದೂರ-ವಾದ
ದೂರ-ವಾ-ದರು
ದೂರ-ವಿ-ಟ್ಟಿ-ದ್ದೇವೆ
ದೂರ-ವಿ-ಟ್ಟಿರು
ದೂರ-ವಿ-ಡ-ಬೇಕು
ದೂರ-ವಿಡು
ದೂರ-ವಿ-ಡು-ತ್ತಿದ್ದ
ದೂರ-ವಿ-ಡು-ವು-ದಕ್ಕೆ
ದೂರ-ವಿ-ರುವ
ದೂರ-ವು-ದ-ಕ್ಕಾಗಿ
ದೂರ-ವು-ದೇ-ತಕ್ಕೆ
ದೂರವೂ
ದೂರವೇ
ದೂರವೋ
ದೂರ-ಸಂ-ಪರ್ಕ
ದೂರಾ-ಲೋ-ಚ-ನೆ-ಗ-ಳಿ-ಗಾಗಿ
ದೂರುವ
ದೂರು-ಸಂ-ಧ್ಯಾ-ವಂ-ದನೆ
ದೂಷಣೆ
ದೂಷಿ-ಸದ
ದೂಷಿ-ಸ-ಲಾ-ರಂ-ಭಿ-ಸಿದ
ದೂಷಿಸಿ
ದೂಷಿ-ಸಿ-ದರು
ದೂಷಿ-ಸಿ-ದ-ವ-ರಲ್ಲ
ದೂಷಿ-ಸು-ವ-ವರ
ದೃಗ್ಗೋ-ಚ-ರ-ಳಾ-ಗು-ತ್ತಿ-ರುವ
ದೃಢ
ದೃಢ-ಗೊಂಡ
ದೃಢ-ತೆ-ಯನ್ನು
ದೃಢ-ತೆ-ಯಿತ್ತು
ದೃಢ-ತೆ-ಯುಂ-ಟಾ-ಯಿ-ತಷ್ಟೇ
ದೃಢ-ಪ-ಡಿಸಿ
ದೃಢ-ಪ-ಡಿ-ಸಿ-ಕೊ-ಳ್ಳಲು
ದೃಢ-ಪ-ಡಿ-ಸಿ-ದರು
ದೃಢ-ವಾಗಿ
ದೃಢ-ವಾ-ಗಿ-ಟ್ಟು-ಕೊಂ-ಡಾಗ
ದೃಢ-ವಾ-ಗಿತ್ತು
ದೃಢ-ವಾ-ಗಿದೆ
ದೃಢ-ವಾ-ಗಿ-ಬಿ-ಡು-ತ್ತಿತ್ತು
ದೃಢ-ವಾ-ಗಿ-ರಿ-ಸು-ವಲ್ಲಿ
ದೃಢ-ವಾ-ಗು-ತ್ತದೆ
ದೃಢ-ವಾ-ಗು-ತ್ತಿತ್ತು
ದೃಢ-ವಾದ
ದೃಢ-ವಾ-ಯಿತು
ದೃಢ-ವಾ-ಯಿ-ತು-ಅ-ವರ
ದೃಢ-ವಿ-ಶ್ವಾಸ
ದೃಢ-ವಿ-ಶ್ವಾ-ಸ-ವಿತ್ತು
ದೃಢೀ-ಕ-ರಿಸಿ
ದೃಶ್ಯ
ದೃಶ್ಯಕ್ಕೆ
ದೃಶ್ಯ-ಗಳನ್ನು
ದೃಶ್ಯ-ಗ-ಳಿ-ಗಿಂ-ತಲೂ
ದೃಶ್ಯ-ಗಳು
ದೃಶ್ಯ-ದಂತೆ
ದೃಶ್ಯ-ದಲ್ಲಿ
ದೃಶ್ಯ-ವನ್ನು
ದೃಶ್ಯ-ವನ್ನೋ
ದೃಶ್ಯ-ವಾ-ಗಿತ್ತು
ದೃಶ್ಯವು
ದೃಶ್ಯವೂ
ದೃಶ್ಯವೇ
ದೃಶ್ಯ-ವೇ-ರ್ಪ-ಟ್ಟಿತು
ದೃಷ್ಟಾಂ-ತ-ಕ-ಥೆ-ಗಳ
ದೃಷ್ಟಾಂ-ತ-ಪೂ-ರ್ವ-ಕ-ವಾದ
ದೃಷ್ಟಾಂ-ತ-ವನ್ನು
ದೃಷ್ಟಿ
ದೃಷ್ಟಿ-ಕೋನ
ದೃಷ್ಟಿ-ಕೋ-ನ-ಕ್ಕಿಂತ
ದೃಷ್ಟಿ-ಕೋ-ನ-ಗ-ಳಿಂ-ದಲೇ
ದೃಷ್ಟಿ-ಕೋ-ನದ
ದೃಷ್ಟಿ-ಕೋ-ನ-ದಲ್ಲಿ
ದೃಷ್ಟಿ-ಕೋ-ನ-ದಲ್ಲೇ
ದೃಷ್ಟಿ-ಕೋ-ನ-ದಿಂದ
ದೃಷ್ಟಿ-ಕೋ-ನ-ವನ್ನು
ದೃಷ್ಟಿ-ಕೋ-ನವೇ
ದೃಷ್ಟಿಗೆ
ದೃಷ್ಟಿಯ
ದೃಷ್ಟಿ-ಯನ್ನು
ದೃಷ್ಟಿ-ಯನ್ನೇ
ದೃಷ್ಟಿ-ಯಲ್ಲಿ
ದೃಷ್ಟಿ-ಯ-ಲ್ಲಿಟ್ಟು
ದೃಷ್ಟಿ-ಯ-ಲ್ಲಿ-ಟ್ಟು-ಕೊಂಡು
ದೃಷ್ಟಿ-ಯಿಂದ
ದೃಷ್ಟಿ-ಯಿಂ-ದಲೂ
ದೃಷ್ಟಿ-ಯಿಂ-ದಲೇ
ದೃಷ್ಟಿ-ಯುದ್ಧ
ದೃಷ್ಟಿ-ಸು-ತ್ತಿ-ದ್ದಂತೆ
ದೆಂದರೆ
ದೆಂದು
ದೆಂಬಂತೆ
ದೆಂಬು-ದನ್ನು
ದೆಂಬುದು
ದೆಡೆಗೆ
ದೆಡೆ-ಗೆ-ಶ್ರೇ-ಯ-ಸ್ಸಾ-ಧ-ನೆಯ
ದೆಯೇ
ದೆಲ್ಲ
ದೆವೋ
ದೆವ್ವ-ಕ್ಕಿಂ-ತಲೂ
ದೆವ್ವ-ಗಳು
ದೆವ್ವದ
ದೆವ್ವ-ವನ್ನು
ದೆಸೆ
ದೆಸೆ-ಯಿಂದ
ದೆಸೆ-ಯಿಂ-ದಾಗಿ
ದೆಹ-ಲಿ-ಆ-ಗ್ರಾ-ಗ-ಳಂತೆ
ದೆಹ-ಲಿ-ಯಲ್ಲಿ
ದೇಗುಲ
ದೇದೀ-ಪ್ಯ-ಮಾನ
ದೇನೋ-ಅ-ವರ
ದೇವ
ದೇವತೆ
ದೇವ-ತೆ-ಗಳು
ದೇವ-ತೆ-ಗ-ಳೇಕೆ
ದೇವ-ತೆಯೋ
ದೇವ-ದೇ-ವ-ತೆ-ಗಳು
ದೇವ-ದೇ-ವಿ-ಯರ
ದೇವನ
ದೇವ-ನನ್ನು
ದೇವ-ನಾ-ಗ-ರಿ-ಯಲ್ಲಿ
ದೇವ-ಮಾತಾ
ದೇವ-ಮಾ-ನ-ವನ
ದೇವರ
ದೇವ-ರ-ದೇವ
ದೇವ-ರನ್ನು
ದೇವ-ರ-ಲ್ಲವೆ
ದೇವ-ರಲ್ಲಿ
ದೇವ-ರ-ಲ್ಲಿ-ನಂ-ಬಿ-ಕೆ-ಯಿ-ಡಲೇ
ದೇವ-ರಲ್ಲೂ
ದೇವ-ರಾ-ಗಿ-ರು-ವಾಗ
ದೇವ-ರಿಗೆ
ದೇವರು
ದೇವ-ರು-ದಿಂ-ಡರು
ದೇವ-ರು-ಧರ್ಮ
ದೇವ-ರು-ಧ-ರ್ಮ-ತತ್ತ್ವ
ದೇವ-ರು-ಧ-ರ್ಮದ
ದೇವ-ರು-ಗ-ಳೆಲ್ಲ
ದೇವರೆ
ದೇವ-ರೆಂದ
ದೇವ-ರೆಂದು
ದೇವರೇ
ದೇವ-ಲೋ-ಕದ
ದೇವ-ಸ-ನ್ನಿ-ಧಿ-ಯಲ್ಲಿ
ದೇವ-ಸ್ಥಾನ
ದೇವ-ಸ್ಥಾ-ನಕ್ಕೂ
ದೇವ-ಸ್ಥಾ-ನಕ್ಕೆ
ದೇವ-ಸ್ಥಾ-ನ-ಗಳ
ದೇವ-ಸ್ಥಾ-ನ-ಗಳನ್ನು
ದೇವ-ಸ್ಥಾ-ನ-ಗಳನ್ನೂ
ದೇವ-ಸ್ಥಾ-ನದ
ದೇವ-ಸ್ಥಾ-ನ-ದಲ್ಲಿ
ದೇವ-ಸ್ಥಾ-ನ-ದೊ-ಳಕ್ಕೆ
ದೇವ-ಸ್ಥಾ-ನ-ದೊ-ಳಗೆ
ದೇವಾ-ಲಯ
ದೇವಾ-ಲ-ಯಕ್ಕೆ
ದೇವಾ-ಲ-ಯ-ಗಳ
ದೇವಾ-ಲ-ಯ-ಗಳನ್ನು
ದೇವಾ-ಲ-ಯ-ಗಳು
ದೇವಾ-ಲ-ಯದ
ದೇವಾ-ಲ-ಯ-ದಲ್ಲಿ
ದೇವಾ-ಲ-ಯ-ದಲ್ಲೂ
ದೇವಾ-ಲ-ಯ-ದಲ್ಲೋ
ದೇವಾ-ಲ-ಯ-ವನ್ನು
ದೇವಾ-ಲ-ಯ-ವನ್ನೂ
ದೇವಾ-ಲ-ಯ-ವಿತ್ತು
ದೇವಾ-ಲ-ಯ-ವಿದೆ
ದೇವಾ-ಲ-ಯವು
ದೇವಾ-ಲ-ಯ-ವೊಂ-ದನ್ನು
ದೇವಾ-ಸು-ರರು
ದೇವಿ
ದೇವಿಯ
ದೇಶ
ದೇಶ
ದೇಶ-ಕಾ-ಲ-ಪಾ-ತ್ರ-ಗ-ಳಿ-ಗ-ನು-ಗು-ಣ-ವಾಗಿ
ದೇಶಕ್ಕೂ
ದೇಶಕ್ಕೆ
ದೇಶ-ಗಳ
ದೇಶ-ಗಳಲ್ಲಿ
ದೇಶ-ಗ-ಳ-ಲ್ಲಿನ
ದೇಶ-ಗ-ಳ-ಲ್ಲೆಲ್ಲ
ದೇಶ-ಗ-ಳ-ವರೂ
ದೇಶ-ಗಳಿಂದ
ದೇಶ-ಗ-ಳಿಗೆ
ದೇಶದ
ದೇಶ-ದಲ್ಲಿ
ದೇಶ-ದ-ಲ್ಲಿ-ರು-ವ-ವರೆಲ್ಲ
ದೇಶ-ದಲ್ಲೇ
ದೇಶ-ದವ
ದೇಶ-ದ-ವರ
ದೇಶ-ದ-ವ-ರಿನ್ನೂ
ದೇಶದಾ
ದೇಶ-ದಾ-ದ್ಯಂತ
ದೇಶ-ದಿಂದ
ದೇಶ-ದೇ-ಶ-ಗ-ಳನ್ನೇ
ದೇಶ-ಪ್ರೇಮ
ದೇಶ-ಪ್ರೇ-ಮ-ದಲ್ಲಿ
ದೇಶ-ಪ್ರೇ-ಮ-ದಿಂದ
ದೇಶ-ಪ್ರೇ-ಮ-ವನ್ನು
ದೇಶ-ಬಾಂ-ಧ-ವರ
ದೇಶ-ಬಾಂ-ಧ-ವರು
ದೇಶ-ಭಕ್ತ
ದೇಶ-ಭ-ಕ್ತ-ಸಂ-ತ-ರಾ-ದರು
ದೇಶ-ಭ-ಕ್ತಿ-ಯೆಂ-ದರೆ
ದೇಶ-ವ-ನಿಗೇ
ದೇಶ-ವನ್ನು
ದೇಶ-ವಾ-ಗಿತ್ತು
ದೇಶವು
ದೇಶವೂ
ದೇಶ-ವೆಂದರೆ
ದೇಶ-ವೊಂ-ದ-ರಲ್ಲಿ
ದೇಶಾ-ಚಾ-ರ-ಗಳನ್ನು
ದೇಶೀಯ
ದೇಸಾಯಿ
ದೇಸಾ-ಯಿಗೆ
ದೇಸಾ-ಯಿ-ಯ-ವ-ರಿಗೆ
ದೇಸಾ-ಯಿ-ಯ-ವರು
ದೇಸಾ-ಯಿ-ವ-ರಿಗೆ
ದೇಹ
ದೇಹಕ್ಕೆ
ದೇಹ-ಗ-ಳಿಗೆ
ದೇಹ-ಗ-ಳೆಂದು
ದೇಹ-ತ್ಯಾಗ
ದೇಹದ
ದೇಹ-ದಿಂದ
ದೇಹ-ದೊ-ಳ-ಗಿನ
ದೇಹ-ಬು-ದ್ಧಿ-ಯನ್ನು
ದೇಹ-ವನ್ನು
ದೇಹ-ವಲ್ಲ
ದೇಹವು
ದೇಹ-ಶ್ರ-ಮ-ವನ್ನೂ
ದೇಹಾ
ದೇಹಾ-ರೋಗ್ಯ
ದೇಹಾ-ರೋ-ಗ್ಯವು
ದೇಹಿ
ದೈನಂ-ದಿನ
ದೈನ್ಯ-ದಿಂದ
ದೈವ-ತ್ವ-ಗಳನ್ನು
ದೈವ-ತ್ವದ
ದೈವ-ತ್ವ-ದೆ-ಡೆಗೆ
ದೈವ-ತ್ವ-ವನ್ನು
ದೈವ-ತ್ವವು
ದೈವ-ತ್ವವೇ
ದೈವ-ದತ್ತ
ದೈವ-ದ-ತ್ತ-ವಾ-ಗ್ಮಿಈ
ದೈವ-ದ-ತ್ತ-ವಾ-ಗ್ಮಿ-ಯೆಂಬ
ದೈವ-ನಿ-ಯಮ
ದೈವ-ನಿ-ಯಾ-ಮ-ಕ-ದಂತೆ
ದೈವ-ಸ-ಹಾ-ಯ-ವನ್ನು
ದೈವಾಂಶ
ದೈವಾಂ-ಶ-ಸಂ-ಭೂತ
ದೈವಿಕ
ದೈವಿ-ಕತೆ
ದೈವೀ
ದೈವೀ-ಉ-ನ್ಮಾದ
ದೈವೀ-ಶ-ಕ್ತಿಗೆ
ದೈವೇಚ್ಛೆ
ದೈವೇ-ಚ್ಛೆ-ಯ-ಲ್ಲದೆ
ದೈವೇ-ಚ್ಛೆಯೇ
ದೈಹಿಕ
ದೈಹಿ-ಕ-ಮಾ-ನ-ಸಿಕ
ದೈಹಿ-ಕ-ವಾ-ಗಿಯೂ
ದೊಂದಿಗೆ
ದೊಂದು
ದೊಡ್ಡ
ದೊಡ್ಡ-ತಪ್ಪು
ದೊಡ್ಡ-ದಲ್ಲ
ದೊಡ್ಡ-ದಾಗಿ
ದೊಡ್ಡ-ದಾ-ಗಿ-ತ್ತೆಂ-ದರೆ
ದೊಡ್ಡ-ದಾ-ಗಿ-ರಲಿ
ದೊಡ್ಡ-ದಾ-ಗಿ-ರು-ತ್ತಿತ್ತು
ದೊಡ್ಡ-ದಾ-ಗೇನೂ
ದೊಡ್ಡ-ದಾದ
ದೊಡ್ಡದು
ದೊಡ್ಡ-ದೆಂ-ದರೆ
ದೊಡ್ಡ-ದೊಂದು
ದೊಡ್ಡ-ದೊಡ್ಡ
ದೊಡ್ಡ-ಪ್ಪ-ನನ್ನು
ದೊಡ್ಡ-ಮ-ನುಷ್ಯ
ದೊಡ್ಡವ
ದೊಡ್ಡ-ವ-ನಾದ
ದೊಡ್ಡ-ವನು
ದೊಡ್ಡ-ವರ
ದೊಡ್ಡ-ವರು
ದೊಡ್ಡ-ವರೂ
ದೊಡ್ಡ-ವ-ಳಾದ
ದೊಡ್ಡವು
ದೊಡ್ಡ-ಸ್ತಿ-ಕೆ-ಯನ್ನು
ದೊಡ್ಡಾ-ಗಿ-ರುವ
ದೊಣ್ಣೆ
ದೊಣ್ಣೆ-ಗಳನ್ನು
ದೊಣ್ಣೆಯ
ದೊಯ್ದಳು
ದೊರ-ಕ-ಬ-ಹುದು
ದೊರ-ಕ-ಬೇಕು
ದೊರ-ಕಿತು
ದೊರ-ಕಿತೇ
ದೊರ-ಕಿತೋ
ದೊರ-ಕಿದ
ದೊರ-ಕಿ-ದರೆ
ದೊರ-ಕಿದೆ
ದೊರ-ಕಿದ್ದ
ದೊರ-ಕಿ-ದ್ದನ್ನು
ದೊರ-ಕಿವೆ
ದೊರ-ಕಿ-ಸಿ-ಕೊಂಡು
ದೊರ-ಕಿ-ಸಿ-ಕೊಡ
ದೊರ-ಕಿ-ಸಿ-ಕೊ-ಡು-ವುದು
ದೊರ-ಕು-ತ್ತದೆ
ದೊರ-ಕು-ತ್ತ-ದೆಯೋ
ದೊರ-ಕು-ತ್ತಿ-ತ್ತಾ-ದರೂ
ದೊರ-ಕು-ತ್ತಿತ್ತು
ದೊರ-ಕು-ತ್ತಿದ್ದ
ದೊರ-ಕು-ವಂ-ತಾ-ಗ-ಬೇಕು
ದೊರ-ಕು-ವಂತೆ
ದೊರ-ಕು-ವುದು
ದೊರು-ಕು-ತ್ತಿದ್ದ
ದೊರೆತ
ದೊರೆ-ತ-ದುದು
ದೊರೆ-ತ-ದ್ದ-ರಿಂದ
ದೊರೆ-ತದ್ದು
ದೊರೆ-ತರೆ
ದೊರೆ-ತಿದೆ
ದೊರೆ-ತಿ-ರು-ವುದು
ದೊರೆ-ತಿಲ್ಲ
ದೊರೆ-ತುದೂ
ದೊರೆಯ
ದೊರೆ-ಯ-ದಿ-ದ್ದಾಗ
ದೊರೆ-ಯ-ಬ-ಹು-ದಾದ
ದೊರೆಯು
ದೊರೆ-ಯು-ತ್ತದೆ
ದೊರೆ-ಯು-ವಂ-ತಿ-ದ್ದಲ್ಲಿ
ದೊರೆ-ಯು-ವಂತೆ
ದೋಚು-ವುದೇ
ದೋಣಿ
ದೋಣಿ-ಗಳ
ದೋಣಿ-ಗ-ಳ-ಲ್ಲಂತೂ
ದೋಣಿ-ಗ-ಳಷ್ಟೇ
ದೋಣಿ-ಗ-ಳಿ-ಗೆಲ್ಲ
ದೋಣಿ-ಗಳು
ದೋಣಿಗೆ
ದೋಣಿ-ಯನ್ನು
ದೋಣಿ-ಯಲ್ಲಿ
ದೋಣಿ-ಯಲ್ಲೇ
ದೋಣಿ-ಯ-ವನು
ದೋಣಿ-ಯಿಂದ
ದೋಣಿ-ಯೊಂದು
ದೋನೋ
ದೋರು-ವ-ವನು
ದೋಷ
ದೋಷ-ಗ-ಳಿ-ರು-ತ್ತವೆ
ದೋಷ-ವನ್ನೋ
ದೋಷ-ವಿದ್ದೇ
ದೋಷ-ವಿ-ರು-ವುದು
ದೋಷ-ವೆಂದರೆ
ದೋಷಾ-ರೋ-ಪ-ಣೆ-ಗಳನ್ನು
ದೌರ್ಜ-ನ್ಯದ
ದೌರ್ಜ-ನ್ಯ-ದಿಂ-ದಾಗಿ
ದೌರ್ಜ-ನ್ಯ-ವ-ನ್ನಾ-ಗಲಿ
ದೌರ್ಬ-ಲ್ಯ-ಗಳನ್ನೂ
ದೌರ್ಬ-ಲ್ಯ-ಗಳು
ದೌರ್ಬ-ಲ್ಯ-ದಿಂದ
ದೌರ್ಬ-ಲ್ಯ-ವನ್ನು
ದೌರ್ಭಾಗ್ಯ
ದೌರ್ಭಾ-ಗ್ಯದ
ದ್ದಂತಿದೆ
ದ್ದಂತೆ
ದ್ದಂತೆಯೇ
ದ್ದನ್ನು
ದ್ದರಿಂದ
ದ್ದರು
ದ್ದರೂ
ದ್ದರೆ
ದ್ದಲ್ಲದೆ
ದ್ದಳು
ದ್ದವ-ರ-ಲ್ಲೊ-ಬ್ಬನು
ದ್ದವರು
ದ್ದವಳು
ದ್ದಾಗ
ದ್ದಾಗಿತ್ತು
ದ್ದಾನೆ
ದ್ದಾರಷ್ಟೇ
ದ್ದಾರೆ
ದ್ದೀಯೆ
ದ್ದೀರಿ
ದ್ದುದ-ರಿಂದ
ದ್ದುದು
ದ್ದುವು
ದ್ದೂರ್
ದ್ದೆಂದರೆ
ದ್ದೇನು
ದ್ದೇನೂ
ದ್ದೇನೆ
ದ್ದೇನೆಯೋ
ದ್ದೇವೆ
ದ್ಯಂತ
ದ್ರವ್ಯಾ-ರ್ಜನೆ
ದ್ರಾಕ್ಷಿ
ದ್ರೋಹ-ವೆ-ಸ-ಗು-ತ್ತಾನೆ
ದ್ರೌಪ-ದಿಯ
ದ್ವಂದ್ವಕ್ಕೆ
ದ್ವಂದ್ವಿ-ಗಳ
ದ್ವಾರಕಾ
ದ್ವಾರಕೆ
ದ್ವಾರ-ಕೆಗೆ
ದ್ವಾರ-ಕೆಯ
ದ್ವಾರ-ಕೆ-ಯತ್ತ
ದ್ವಾರ-ಕೆ-ಯಲ್ಲಿ
ದ್ವಾರ-ಕೆ-ಯಿಂದ
ದ್ವಾರ-ವನ್ನು
ದ್ವಾರ-ವೊಂದು
ದ್ವಿತೀಯ
ದ್ವಿಧಾ
ದ್ವೀಪ
ದ್ವೀಪಕ್ಕೆ
ದ್ವೀಪ-ಗ-ಳನ್ನೋ
ದ್ವೀಪ-ಗಳಲ್ಲಿ
ದ್ವೀಪ-ಗ-ಳಿ-ರು-ವುದು
ದ್ವೀಪ-ಗಳೂ
ದ್ವೀಪದ
ದ್ವೀಪ-ದಲ್ಲಿ
ದ್ವೀಪ-ದ-ಲ್ಲಿವೆ
ದ್ವೀಪ-ವಾದ
ದ್ವೀಪ-ಸ-ಮು-ದಾ-ಯದ
ದ್ವೇಷ
ದ್ವೇಷಕ್ಕೆ
ದ್ವೇಷ-ರ-ಹಿ-ತ-ರಾಗಿ
ದ್ವೇಷ-ವಿ-ಟ್ಟು-ಕೊಂಡು
ದ್ವೇಷಾ-ಗ್ನಿಗೆ
ದ್ವೇಷಾ-ಸೂ-ಯೆ-ಗಳನ್ನು
ದ್ವೇಷಾ-ಸೂ-ಯೆಯ
ದ್ವೇಷಿ-ಸ-ಲಿ-ಎ-ಲ್ಲಕ್ಕೂ
ದ್ವೇಷಿ-ಸು-ತ್ತಿ-ದ್ದರು
ದ್ವೇಷಿ-ಸು-ತ್ತೇನೆ
ದ್ವೇಷಿ-ಸು-ವ-ವ-ರಲ್ಲ
ದ್ವೇಷಿ-ಸು-ವು-ದಿ-ಲ್ಲವೋ
ದ್ವೈತ
ದ್ವೈತ-ವಿ-ಶಿ-ಷ್ಟಾ-ದ್ವೈ-ತ
ದ್ವೈತ-ವಿ-ಶಿ-ಷ್ಟಾ-ದ್ವೈ-ತ-ಅ-ದ್ವೈ-ತ-ಗಳಲ್ಲಿ
ದ್ವೈತ-ಗಳ
ದ್ವೈತ-ದಿಂ-ದಲೂ
ದ್ವೈತ-ವನ್ನು
ಧಕ್ಕೆ
ಧಕ್ಕೆ-ಯುಂ-ಟಾ-ಗ-ದಂತೆ
ಧಕ್ಕೆ-ಯುಂ-ಟು-ಮಾ-ಡ-ಲಿಲ್ಲ
ಧಗೆ-ಯನ್ನು
ಧತೆಯ
ಧತೆಯು
ಧನ
ಧನ-ಗ-ಳ-ನ್ನ-ರ್ಪಿ-ಸಲು
ಧನ-ವ-ನ್ನ-ರಸಿ
ಧನ-ವನ್ನು
ಧನ-ಸಂ-ಗ್ರ-ಹಣೆ
ಧನ-ಸಂ-ಗ್ರ-ಹ-ಣೆಯ
ಧನ-ಸಂ-ಗ್ರ-ಹ-ಣೆಯೂ
ಧನ-ಸಂ-ಪಾ-ದ-ನೆ-ಗಾ-ಗಿಯೇ
ಧನ-ಸಂ-ಪಾ-ದ-ನೆಯ
ಧನ-ಸ-ಹಾಯ
ಧನ-ಸ-ಹಾ-ಯ-ವನ್ನು
ಧನಾ-ರ್ಜ-ನೆ-ಗಾಗಿ
ಧನಿ-ಕ-ನಾ-ಗಿದ್ದ
ಧನಿ-ಕ-ರಲ್ಲಿ
ಧನು-ಷ್ಕೋ-ಟಿ-ಯಿಂ-ದಲೇ
ಧನ್ಯ
ಧನ್ಯ-ನಾದ
ಧನ್ಯ-ನಾ-ದ-ನೆಂದು
ಧನ್ಯ-ನೆಂ-ದು-ಕೊ-ಳ್ಳು-ತ್ತೇನೆ
ಧನ್ಯ-ರಾ-ಗ-ಬೇ-ಕೆಂದೂ
ಧನ್ಯ-ರಾ-ಗಲು
ಧನ್ಯರು
ಧನ್ಯರೇ
ಧನ್ಯ-ವಾ-ಗಿ-ಸಿತೋ
ಧನ್ಯ-ವಾದ
ಧನ್ಯ-ವಾ-ದ-ಗ-ಳ-ನ್ನ-ರ್ಪಿ-ಸಲು
ಧನ್ಯ-ವಾ-ದ-ಗ-ಳ-ನ್ನ-ರ್ಪಿ-ಸುವ
ಧನ್ಯ-ವಾ-ದ-ಗಳನ್ನು
ಧನ್ಯ-ವಾ-ದ-ಗಳು
ಧರಿ-ಸ-ಲಾ-ರಂ-ಭಿ-ಸಿ-ದರು
ಧರಿ-ಸಲು
ಧರಿಸಿ
ಧರಿ-ಸಿದ
ಧರಿ-ಸಿದ್ದ
ಧರಿ-ಸಿ-ದ್ದ-ರಿಂದ
ಧರಿ-ಸಿದ್ದು
ಧರಿ-ಸಿ-ರುವ
ಧರಿ-ಸಿ-ರು-ವುದು
ಧರಿ-ಸು-ವಂತೆ
ಧರಿ-ಸು-ವಿ-ಕೆ-ಯನ್ನೂ
ಧರೆ-ಗು-ರು-ಳಿ-ದ್ದರು
ಧರೆಗೆ
ಧರೋ
ಧರ್ಮ
ಧರ್ಮ-ಅ-ಧ್ಯಾತ್ಮ
ಧರ್ಮ-ದೇ-ವರು
ಧರ್ಮ-ಸಂ-ಸ್ಕೃ-ತಿ
ಧರ್ಮ-ಸಂ-ಸ್ಕೃ-ತಿ-ತ-ತ್ವ-ಗಳನ್ನು
ಧರ್ಮ-ಸಂ-ಸ್ಕೃ-ತಿ-ಗಳ
ಧರ್ಮ-ಸಂ-ಸ್ಕೃ-ತಿ-ಗ-ಳೆ-ಡೆಗೆ
ಧರ್ಮ-ಸಂ-ಸ್ಕೃ-ತಿ-ಗ-ಳೊಂ-ದಿಗೆ
ಧರ್ಮ-ಇ-ವು-ಗಳಲ್ಲಿ
ಧರ್ಮ-ಇವೇ
ಧರ್ಮ-ಎಂ-ತಹ
ಧರ್ಮ-ಕ್ಕಲ್ಲ
ಧರ್ಮ-ಕ್ಕಾ-ಗಲಿ
ಧರ್ಮ-ಕ್ಕಿಂ-ತಲೂ
ಧರ್ಮಕ್ಕೂ
ಧರ್ಮಕ್ಕೆ
ಧರ್ಮ-ಗಳ
ಧರ್ಮ-ಗಳನ್ನೂ
ಧರ್ಮ-ಗ-ಳ-ಲ್ಲಿ-ರುವ
ಧರ್ಮ-ಗಳಿಂದ
ಧರ್ಮ-ಗ-ಳಿ-ಗಿಂ-ತಲೂ
ಧರ್ಮ-ಗ-ಳಿಗೂ
ಧರ್ಮ-ಗ-ಳಿಗೆ
ಧರ್ಮ-ಗಳು
ಧರ್ಮ-ಗಳೂ
ಧರ್ಮ-ಗ-ಳೆಲ್ಲ
ಧರ್ಮ-ಗುರು
ಧರ್ಮ-ಗು-ರು-ಗ-ಳಾದ
ಧರ್ಮ-ಗು-ರು-ಗ-ಳಿಗೇ
ಧರ್ಮ-ಗ್ರಂ-ಥ-ಗಳನ್ನು
ಧರ್ಮ-ಗ್ರಂ-ಥ-ಗ-ಳ-ಲ್ಲಿ-ರುವ
ಧರ್ಮ-ಗ್ರಂ-ಥ-ದಲ್ಲಿ
ಧರ್ಮದ
ಧರ್ಮ-ದ-ಲ್ಲಲ್ಲ
ಧರ್ಮ-ದಲ್ಲಿ
ಧರ್ಮ-ದ-ಲ್ಲಿ-ರುವ
ಧರ್ಮ-ದಲ್ಲೂ
ಧರ್ಮ-ದಲ್ಲೇ
ಧರ್ಮ-ದಿಂ-ದೇನು
ಧರ್ಮ-ದೊಂ-ದಿಗೂ
ಧರ್ಮ-ದೊ-ಳಕ್ಕೆ
ಧರ್ಮ-ದ್ರೋ-ಹಿ-ಗಳು
ಧರ್ಮ-ದ್ವೇ-ಷಿ-ಗ-ಳೆಂದೂ
ಧರ್ಮ-ನಿಷ್ಠೆ
ಧರ್ಮ-ಪಾಲ
ಧರ್ಮ-ಪಾ-ಲ-ಇ-ವ-ರಿಗೂ
ಧರ್ಮ-ಪಾ-ಲರು
ಧರ್ಮ-ಪ್ರ-ಚಾರ
ಧರ್ಮ-ಪ್ರ-ಚಾ-ರಕ
ಧರ್ಮ-ಪ್ರ-ಚಾ-ರ-ಕ-ನಾಗಿ
ಧರ್ಮ-ಪ್ರ-ಚಾ-ರ-ಕ-ನೊಬ್ಬ
ಧರ್ಮ-ಪ್ರ-ಚಾ-ರ-ಕ-ನೊ-ಬ್ಬ-ನಿಗೆ
ಧರ್ಮ-ಪ್ರ-ಚಾ-ರ-ಕ-ರನ್ನು
ಧರ್ಮ-ಪ್ರ-ಚಾ-ರ-ಕ-ರಿಂ-ದಲೂ
ಧರ್ಮ-ಪ್ರ-ಚಾ-ರ-ಕರು
ಧರ್ಮ-ಪ್ರ-ಚಾ-ರ-ಕಾ-ರ್ಯ-ಕ್ಕಾಗಿ
ಧರ್ಮ-ಪ್ರ-ಚಾ-ರ-ಕ್ಕಾಗಿ
ಧರ್ಮ-ಪ್ರ-ಸಾರ
ಧರ್ಮ-ಪ್ರ-ಸಾ-ರ-ಕನ
ಧರ್ಮ-ಪ್ರ-ಸಾ-ರ-ಕ-ನನ್ನು
ಧರ್ಮ-ಪ್ರ-ಸಾ-ರ-ಕಾರ್ಯ
ಧರ್ಮ-ಪ್ರ-ಸಾ-ರ-ಕಾ-ರ್ಯ-ದಲ್ಲಿ
ಧರ್ಮ-ಪ್ರ-ಸಾ-ರ-ಕ್ಕಾ-ಗಿಯೇ
ಧರ್ಮ-ಪ್ರ-ಸಾ-ರದ
ಧರ್ಮ-ಪ್ರ-ಸಾ-ರ-ವನ್ನು
ಧರ್ಮ-ಬೋ-ಧ-ಕ-ನಾಗಿ
ಧರ್ಮ-ಬೋ-ಧ-ಕರ
ಧರ್ಮ-ಬೋ-ಧ-ಕ-ರಾಗಿ
ಧರ್ಮ-ಬೋ-ಧ-ಕರು
ಧರ್ಮ-ಬೋ-ಧನೆ
ಧರ್ಮ-ಬೋ-ಧ-ನೆ-ಯನ್ನು
ಧರ್ಮ-ಬೋ-ಧೆ-ಯಲ್ಲ
ಧರ್ಮ-ಭಾ-ಸ್ಕ-ರನ
ಧರ್ಮ-ಭೋ-ದ-ನೆಗೆ
ಧರ್ಮ-ಮಂ-ಡಲ
ಧರ್ಮ-ಮಹಾ
ಧರ್ಮ-ವ-ನ್ನಲ್ಲ
ಧರ್ಮ-ವನ್ನು
ಧರ್ಮ-ವನ್ನೂ
ಧರ್ಮ-ವ-ನ್ನೆಂದೂ
ಧರ್ಮ-ವನ್ನೇ
ಧರ್ಮ-ವ-ನ್ನೇಕೆ
ಧರ್ಮ-ವಲ್ಲ
ಧರ್ಮ-ವ-ಲ್ಲವೆ
ಧರ್ಮ-ವಾಗಿ
ಧರ್ಮ-ವಾ-ಗು-ತ್ತದೆ
ಧರ್ಮ-ವಾದ
ಧರ್ಮವು
ಧರ್ಮವೂ
ಧರ್ಮ-ವೆಂದರೆ
ಧರ್ಮ-ವೆಂದು
ಧರ್ಮ-ವೆಂದೂ
ಧರ್ಮ-ವೆಂ-ಬುದು
ಧರ್ಮ-ವೆ-ನ್ನು-ವುದು
ಧರ್ಮವೇ
ಧರ್ಮ-ವೊಂ-ದನ್ನು
ಧರ್ಮ-ವೊಂ-ದರ
ಧರ್ಮ-ವೊಂದೇ
ಧರ್ಮವೋ
ಧರ್ಮ-ಶಾ-ಲೆ-ಯೊಂ-ದ-ರಲ್ಲಿ
ಧರ್ಮ-ಶಾ-ಸ್ತ್ರ-ಗಳನ್ನು
ಧರ್ಮ-ಸಂ-ಸ್ಕೃ-ತಿ-ಸಂ-ಪ್ರ-ದಾಯ
ಧರ್ಮ-ಸಂ-ಸ್ಕೃ-ತಿ-ಗಳ
ಧರ್ಮ-ಸಂ-ಸ್ಥಾ-ಪನೆ
ಧರ್ಮ-ಸಂ-ಸ್ಥಾ-ಪ-ನೆ-ಯನ್ನು
ಧರ್ಮ-ಸಂ-ಸ್ಥೆ-ಗಳ
ಧರ್ಮ-ಸ-ಮ್ಮೇಳ
ಧರ್ಮ-ಸ-ಮ್ಮೇ-ಳನ
ಧರ್ಮ-ಸ-ಮ್ಮೇ-ಳ-ನಕ್ಕೆ
ಧರ್ಮ-ಸ-ಮ್ಮೇ-ಳ-ನದ
ಧರ್ಮ-ಸ-ಮ್ಮೇ-ಳ-ನ-ದಲ್ಲಿ
ಧರ್ಮ-ಸ-ಮ್ಮೇ-ಳ-ನ-ವೊಂ-ದರ
ಧರ್ಮ-ಸಾ-ಧನಂ
ಧರ್ಮಾಂ
ಧರ್ಮಾಂಧ
ಧರ್ಮಾಂ-ಧ-ತೆ-ಗಳು
ಧರ್ಮಾಂ-ಧ-ತೆಗೆ
ಧರ್ಮಾಂ-ಧ-ತೆಯು
ಧರ್ಮಾ-ಧಿ-ಕಾರಿ
ಧರ್ಮಾ-ಧಿ-ಕಾ-ರಿ-ಗಳ
ಧರ್ಮಾ-ಧಿ-ಕಾ-ರಿ-ಗ-ಳಿಗೆ
ಧರ್ಮಾ-ಧಿ-ಕಾ-ರಿ-ಗ-ಳಿ-ದ್ದರು
ಧರ್ಮಾ-ಧಿ-ಕಾ-ರಿ-ಗಳು
ಧರ್ಮಾ-ಧಿ-ಕಾ-ರಿ-ಗ-ಳೆಲ್ಲ
ಧರ್ಮಾ-ಧಿ-ಕಾ-ರಿ-ಯಾದ
ಧರ್ಮಾ-ನು-ಷ್ಠಾ-ನವೂ
ಧರ್ಮೀ-ಯರೂ
ಧವಳ
ಧಸ-ಕ್ಕೆಂ-ದಿತು
ಧಾಟಿ
ಧಾಟಿ-ಯನ್ನು
ಧಾಟಿ-ಯಲ್ಲಿ
ಧಾಟಿ-ಯಿಂ-ದಲೋ
ಧಾನ್ಯ-ಗ-ಳಿಗೆ
ಧಾಮ-ವನ್ನು
ಧಾಮ-ವಾದ
ಧಾಮಾನಿ
ಧಾರ-ಣ-ಶ-ಕ್ತಿಯೇ
ಧಾರ-ದಿಂದ
ಧಾರ-ವಾ-ಡದ
ಧಾರಾ-ಕಾ-ರ-ವಾಗಿ
ಧಾರಾ-ಳ-ವಾಗಿ
ಧಾರಾ-ಳ-ವಾದ
ಧಾರಿ
ಧಾರೆ-ಯೆ-ರೆ-ದಿ-ದ್ದರು
ಧಾರೆ-ಯೆ-ರೆ-ದು-ಕೊಟ್ಟ
ಧಾರ್ಮಿಕ
ಧಾರ್ಮಿ-ಕ-ಆ-ಧ್ಯಾ-ತ್ಮಿಕ
ಧಾರ್ಮಿ-ಕತೆ
ಧಾರ್ಮಿ-ಕ-ತೆಯ
ಧಾರ್ಮಿ-ಕ-ನಿ-ಗಿಂತ
ಧಾರ್ಮಿ-ಕವೇ
ಧಾರ್ಮಿ-ಕ-ಶ್ರ-ದ್ಧೆಗೆ
ಧಾವಿಸಿ
ಧಾವಿ-ಸಿತು
ಧಾವಿ-ಸಿ-ದರು
ಧಾವಿ-ಸಿ-ಬ-ರು-ತ್ತಿ-ದ್ದರು
ಧಾವಿ-ಸು-ತ್ತಿ-ರು-ವುದನ್ನು
ಧಿಕ್ಕಾ-ರ-ವಿ-ರಲಿ
ಧಿಗ್ಗ-ನೆದ್ದು
ಧಿಸಿ-ದರು
ಧೀರ
ಧೀರ-ಮ-ನ-ಮೋ-ಹಕ
ಧೀರ-ಗಂ-ಭೀರ
ಧೀರ-ತ-ನ-ವನ್ನು
ಧೀರ-ತರ
ಧೀರತೆ
ಧೀರ-ತೆಯ
ಧೀರ-ನಾಗು
ಧೀರ-ಪುತ್ರ
ಧೀರರ
ಧೀರ-ರಾಗಿ
ಧೀರರು
ಧೀರರೂ
ಧೀರ-ಹೃ-ದ-ಯಿ-ಗಳೇ
ಧೀರ್ಘ
ಧೀಶ-ಕ್ತಿ-ಯೊಂ-ದಿಗೆ
ಧೀಶರು
ಧುಮು-ಕಿ-ದರು
ಧುಮು-ಕಿ-ಬಿ-ಡ-ಬೇಕು
ಧುಮು-ಕಿಯೇ
ಧುಮು-ಕಿ-ಯೇ-ಬಿ-ಟ್ಟರು
ಧುರೀ-ಣರೂ
ಧುರೀ-ಣ-ರೆಂದು
ಧೂರ್ತ-ರಿಗೆ
ಧೂರ್ತರು
ಧೂಳಿನ
ಧೈರ್ಯ
ಧೈರ್ಯಕ್ಕೆ
ಧೈರ್ಯ-ಗೊಂಡು
ಧೈರ್ಯ-ತಾಳಿ
ಧೈರ್ಯ-ದಿಂದ
ಧೈರ್ಯ-ಮಾ-ಡು-ತ್ತೇನೆ
ಧೈರ್ಯ-ವನ್ನು
ಧೈರ್ಯ-ವಾಗಿ
ಧೈರ್ಯ-ವಾ-ಗಿರು
ಧೈರ್ಯ-ವಿ-ರ-ಲಿಲ್ಲ
ಧೈರ್ಯ-ಶಾ-ಲಿ-ಗ-ಳಾದ
ಧೈರ್ಯ-ಶಾ-ಲಿ-ಗಳು
ಧೈರ್ಯ-ಶಾ-ಲಿ-ಯಾದ
ಧೈರ್ಯ-ಶಾ-ಲಿ-ಯಾ-ದ-ವ-ನೊಬ್ಬ
ಧೋರಣೆ
ಧೋರ-ಣೆ-ಗಳನ್ನು
ಧೋರ-ಣೆ-ಯನ್ನು
ಧೋರ-ಣೆ-ಯಿಂದ
ಧೋರ-ಣೆ-ಯಿಂ-ದಿ-ರಲು
ಧೋರ-ಣೆ-ಯೆಂದು
ಧ್ಯಾನ
ಧ್ಯಾನ-ಜ-ಪ-ಅ-ಧ್ಯ-ಯ-ನಾದಿ
ಧ್ಯಾನ-ತ-ರ-ಗ-ತಿ-ಉ-ಪ-ನ್ಯಾ-ಸ-ಮಾ-ತು-ಕತೆ
ಧ್ಯಾನ-ಪ್ರಾ-ರ್ಥ-ನೆ-ಗಳಲ್ಲಿ
ಧ್ಯಾನ-ಶಾಂ-ತಿ-ಗಳ
ಧ್ಯಾನ-ಕ್ಕೆಂದೇ
ಧ್ಯಾನ-ಜೀ-ವ-ನ-ವನ್ನು
ಧ್ಯಾನದ
ಧ್ಯಾನ-ದಲ್ಲಿ
ಧ್ಯಾನ-ದ-ಲ್ಲಿ-ರು-ವಾಗ
ಧ್ಯಾನ-ದಿಂ-ದೆದ್ದ
ಧ್ಯಾನ-ದಿಂ-ದೆದ್ದು
ಧ್ಯಾನ-ನಿ-ರತ
ಧ್ಯಾನ-ನಿ-ರ-ತ-ನಾ-ಗಿದ್ದೆ
ಧ್ಯಾನ-ನಿ-ರ-ತ-ರಾ-ಗಿ-ಬಿ-ಡು-ತ್ತಿ-ದ್ದರು
ಧ್ಯಾನ-ಭಂ-ಗಿ-ಯಲ್ಲಿ
ಧ್ಯಾನ-ಭಾ-ವ-ರಂ-ಜಿ-ತ-ರಾ-ಗಿ-ರು-ತ್ತಿ-ದ್ದರು
ಧ್ಯಾನ-ಮಗ್ನ
ಧ್ಯಾನ-ಮ-ಗ್ನ-ನಾಗಿ
ಧ್ಯಾನ-ಮ-ಗ್ನ-ನಾ-ಗಿರ
ಧ್ಯಾನ-ಮ-ಗ್ನ-ನಾ-ದಾಗ
ಧ್ಯಾನ-ಮ-ಗ್ನ-ರಾಗಿ
ಧ್ಯಾನ-ಮ-ಗ್ನ-ರಾ-ಗಿ-ಬಿ-ಡು-ತ್ತಿ-ದ್ದರು
ಧ್ಯಾನ-ಮ-ಗ್ನ-ರಾ-ಗು-ತ್ತಿ-ದ್ದರು
ಧ್ಯಾನ-ಮ-ಗ್ನ-ರಾ-ದರು
ಧ್ಯಾನ-ಮಾಡಿ
ಧ್ಯಾನ-ಮಾ-ಡು-ತ್ತಿ-ದ್ದಾಗ
ಧ್ಯಾನ-ಲೀ-ನ-ರಾ-ಗಿಯೇ
ಧ್ಯಾನ-ಲೀ-ನ-ರಾ-ಗು-ತ್ತಿದ್ದ
ಧ್ಯಾನವೂ
ಧ್ಯಾನವೇ
ಧ್ಯಾನ-ಶೀ-ಲ-ತೆ-ಇ-ವು-ಗಳು
ಧ್ಯಾನ-ಸಿ-ದ್ಧನ
ಧ್ಯಾನ-ಸ್ಥ-ರಾ-ಗಿ-ದ್ದು-ದನ್ನು
ಧ್ಯಾನ-ಸ್ಥಿ-ತಿಗೆ
ಧ್ಯಾನ-ಸ್ಥಿ-ತಿ-ಗೇ-ರ-ದಂತೆ
ಧ್ಯಾನ-ಸ್ಥಿ-ತಿ-ಯನ್ನು
ಧ್ಯಾನಾ-ದಿ-ಗ-ಳಿ-ಗಾಗಿ
ಧ್ಯಾನಾ-ನಂ-ದ-ದಲ್ಲಿ
ಧ್ಯಾನಾ-ನಂ-ದಲ್ಲಿ
ಧ್ಯಾನಾ-ಭ್ಯಾ-ಸದ
ಧ್ಯಾನಾ-ವ-ಸ್ಥೆ-ಯಲ್ಲಿ
ಧ್ಯಾನಾ-ವ-ಸ್ಥೆ-ಯಿಂ-ದ-ಸ-ಮಾ-ಧಿಯ
ಧ್ಯಾನಿ-ಸುತ್ತ
ಧ್ಯಾನಿ-ಸು-ತ್ತ-ಅ-ವು-ಗ-ಳಿಗೆ
ಧ್ಯಾಯಿ-ನಿ-ಯರು
ಧ್ಯೇಯ-ಗಳನ್ನು
ಧ್ಯೇಯ-ಗ-ಳಿ-ಗಾಗಿ
ಧ್ಯೇಯ-ಮಂತ್ರ
ಧ್ಯೇಯ-ವಾಕ್ಯ
ಧ್ಯೇಯ-ವಾ-ಕ್ಯ-ವನ್ನು
ಧ್ಯೇಯ-ವೆಂದರೆ
ಧ್ಯೇಯ-ವೆಂದು
ಧ್ಯೇಯಾ-ದ-ರ್ಶ-ಗಳನ್ನು
ಧ್ರುವ-ದ-ವರೆ-ಗೆ-ಜ-ಗ-ತ್ತಿ-ನಾ-ದ್ಯಂತ
ಧ್ರುವ-ದಿಂದ
ಧ್ವಂಸ
ಧ್ವಂಸ-ಗೊ-ಳಿ-ಸಲು
ಧ್ವಜ-ಗಳಿಂದ
ಧ್ವಜದ
ಧ್ವಜವ
ಧ್ವಜ-ವನ್ನು
ಧ್ವನಿ
ಧ್ವನಿ-ಮು-ದ್ರಣ
ಧ್ವನಿ-ಯನ್ನು
ಧ್ವನಿ-ಯಲ್ಲಿ
ಧ್ವನಿ-ಯೆಂದೇ
ಧ್ವನಿ-ಯೊಂ-ದನ್ನು
ಧ್ವನಿ-ಯೊಂದು
ಧ್ವನಿ-ಸ-ಲ್ಪ-ಟ್ಟುವು
ನ
ನಂಜುಂ-ಡ-ರಾವ್
ನಂಜುಂ-ಡ-ರಾವ್ಗೆ
ನಂತರ
ನಂತ-ರದ
ನಂತಹ
ನಂತಿದ್ದ
ನಂತೆ
ನಂತೆಯೇ
ನಂದ
ನಂದದ
ನಂದರ
ನಂದ-ರತ್ತ
ನಂದ-ರ-ದಾ-ಗಿತ್ತು
ನಂದ-ರದು
ನಂದ-ರನ್ನು
ನಂದ-ರಾಗಿ
ನಂದ-ರಿಗೆ
ನಂದರು
ನಂದರೂ
ನಂದಿ-ಸ-ಬೇ-ಕೆಂಬ
ನಂಬ-ಬೇ-ಕಾ-ದರೆ
ನಂಬರೂ
ನಂಬ-ಲ-ಸಾ-ಧ್ಯ-ವೆಂ-ಬಂತೆ
ನಂಬ-ಲಾ-ರಂ-ಭಿ-ಸಿ-ದರು
ನಂಬ-ಲಿಲ್ಲ
ನಂಬಲು
ನಂಬಲೇ
ನಂಬಿ
ನಂಬಿ-ಕ-ಸ್ಥರು
ನಂಬಿಕೆ
ನಂಬಿ-ಕೆ-ಗಳ
ನಂಬಿ-ಕೆ-ಗಳನ್ನು
ನಂಬಿ-ಕೆ-ಗ-ಳ-ನ್ನು-ಸಂ-ದೇ-ಹ-ಗಳನ್ನು
ನಂಬಿ-ಕೆ-ಗಳನ್ನೆಲ್ಲ
ನಂಬಿ-ಕೆ-ಗಳಲ್ಲಿ
ನಂಬಿ-ಕೆ-ಗ-ಳಿಗೆ
ನಂಬಿ-ಕೆ-ಗಳು
ನಂಬಿ-ಕೆ-ಗಳೇ
ನಂಬಿ-ಕೆಗೆ
ನಂಬಿ-ಕೆ-ಯನ್ನು
ನಂಬಿ-ಕೆ-ಯನ್ನೂ
ನಂಬಿ-ಕೆ-ಯಿಡು
ನಂಬಿ-ಕೆ-ಯಿದೆ
ನಂಬಿ-ಕೆ-ಯಿ-ರ-ಬೇ-ಕೆಂ-ದರೆ
ನಂಬಿ-ಕೆ-ಯಿ-ರ-ಲಿಲ್ಲ
ನಂಬಿ-ಕೆ-ಯಿಲ್ಲ
ನಂಬಿ-ಕೆ-ಯಿ-ಲ್ಲ-ವೆಂದೂ
ನಂಬಿ-ಕೆಯು
ನಂಬಿ-ಕೆ-ಯುಂ-ಟಾ-ಗಿತ್ತು
ನಂಬಿ-ಕೆ-ಯುಂ-ಟಾ-ಗಿ-ರ-ಲಿಲ್ಲ
ನಂಬಿ-ಕೆಯೂ
ನಂಬಿ-ಕೆ-ಯೆಂದು
ನಂಬಿ-ಕೆ-ಯೆಂ-ಬುದು
ನಂಬಿ-ಕೆಯೇ
ನಂಬಿ-ಕೆ-ಯೇ-ನೆಂ-ದರೆ
ನಂಬಿ-ಕೊಂ-ಡಿ-ರು-ವ-ವ-ರೆಗೆ
ನಂಬಿದ
ನಂಬಿ-ದರೆ
ನಂಬಿ-ದ-ವ-ರನ್ನು
ನಂಬಿ-ದ-ವ-ರಲ್ಲ
ನಂಬಿದ್ದ
ನಂಬಿ-ದ್ದರು
ನಂಬಿ-ದ್ದಳು
ನಂಬಿ-ದ್ದ-ವಳು
ನಂಬಿ-ದ್ದಾನೆ
ನಂಬಿ-ದ್ದಾರೋ
ನಂಬಿದ್ದೇ
ನಂಬಿ-ದ್ದೇನೆ
ನಂಬಿ-ರು-ತ್ತಾನೆ
ನಂಬಿ-ರು-ತ್ತೇವೆ
ನಂಬಿ-ರುವ
ನಂಬಿ-ರು-ವು-ದಾಗಿ
ನಂಬಿ-ಸು-ವಲ್ಲಿ
ನಂಬು
ನಂಬು-ತ್ತೇನೆ
ನಂಬು-ತ್ತೇವೆ
ನಂಬುವ
ನಂಬು-ವಂತೆ
ನಂಬು-ವ-ದ-ರ-ಲ್ಲಿಲ್ಲ
ನಂಬು-ವು-ದಾ-ದರೂ
ನಂಬು-ವುದು
ನಕ್ಕರು
ನಕ್ಕರೋ
ನಕ್ಕು
ನಕ್ಕು-ನಕ್ಕು
ನಕ್ಕು-ಬಿಡು
ನಕ್ಕೆ
ನಕ್ಷ-ತ್ರ-ಗಳನ್ನು
ನಕ್ಷ-ತ್ರ-ವು-ದಿ-ಸಿತು
ನಗ
ನಗರ
ನಗ-ರಕ್ಕೆ
ನಗ-ರ-ಗಳ
ನಗ-ರ-ಗಳನ್ನು
ನಗ-ರ-ಗಳಲ್ಲಿ
ನಗ-ರ-ಗ-ಳ-ಲ್ಲಿನ
ನಗ-ರ-ಗ-ಳ-ಲ್ಲೆಲ್ಲ
ನಗ-ರ-ಗ-ಳಾದ
ನಗ-ರ-ಗ-ಳಿಗೆ
ನಗ-ರ-ಗಳು
ನಗ-ರದ
ನಗ-ರ-ದ-ರ್ಶ-ನಕ್ಕೆ
ನಗ-ರ-ದ-ರ್ಶ-ನ-ವಾ-ಗ-ಬ-ಹುದು
ನಗ-ರ-ದಲ್ಲಿ
ನಗ-ರ-ದ-ಲ್ಲಿದೆ
ನಗ-ರ-ದ-ಲ್ಲಿ-ದ್ದಾಗ
ನಗ-ರ-ದ-ಲ್ಲಿ-ರುವ
ನಗ-ರ-ದಲ್ಲೂ
ನಗ-ರ-ದಲ್ಲೆಲ್ಲ
ನಗ-ರ-ದ-ವ-ರೆಗೂ
ನಗ-ರ-ದಿಂದ
ನಗ-ರ-ದಿಂ-ದಲೇ
ನಗ-ರ-ದೆ-ಡೆಗೆ
ನಗ-ರ-ವನ್ನು
ನಗ-ರ-ವಾ-ಗಿದ್ದ
ನಗ-ರ-ವಾದ
ನಗ-ರ-ವಾ-ಸಿ-ಗಳಲ್ಲಿ
ನಗ-ರವು
ನಗ-ರವೇ
ನಗಿಸು
ನಗಿ-ಸು-ತ್ತಿ-ದ್ದರು
ನಗಿ-ಸುವ
ನಗು
ನಗುತ್ತ
ನಗು-ತ್ತಿದ್ದ
ನಗು-ತ್ತಿ-ದ್ದರು
ನಗು-ತ್ತಿ-ದ್ದಾಗ
ನಗು-ನಗು
ನಗು-ನ-ಗುತ್ತ
ನಗು-ನ-ಗು-ತ್ತಿ-ದ್ದರು
ನಗು-ಮೊ-ಗ-ದ-ವ-ಳಾದೆ
ನಗು-ವಿನ
ನಗು-ವಿ-ನಲ್ಲಿ
ನಗುವು
ನಗೆ
ನಗೆ-ಪಾ-ಟ-ಲಿ-ಗೀ-ಡಾಗಿ
ನಗೆಯ
ನಟನೆ
ನಟ-ರಾ-ಜನ್
ನಟಿ-ಗಾ-ಯಕಿ
ನಟಿ-ಯಾದ
ನಟ್ಟ
ನಡ-ವ-ಳಿಕೆ
ನಡ-ವ-ಳಿ-ಕೆ-ಗಳನ್ನು
ನಡ-ವ-ಳಿ-ಕೆ-ಗಳಲ್ಲಿ
ನಡ-ವ-ಳಿ-ಕೆ-ಗಳು
ನಡ-ವ-ಳಿ-ಕೆ-ಯಿಂದ
ನಡ-ವ-ಳಿ-ಕೆಯೂ
ನಡ-ಸು-ವಂ-ತಾ-ಗು-ತ್ತದೆ
ನಡಿ
ನಡಿಗೆ
ನಡು
ನಡು-ಕ-ವುಂ-ಟಾ-ಗಲು
ನಡು-ಗ-ಡ್ಡೆ-ಗ-ಳನ್ನೋ
ನಡು-ಗಿತ್ತು
ನಡು-ಗಿದ
ನಡು-ಗಿ-ದರು
ನಡು-ಗಿ-ದಳು
ನಡು-ಗಿ-ಸ-ಲಿ-ದ್ದಾರೆ
ನಡು-ಗಿ-ಸು-ವಂ-ತಹ
ನಡು-ಗು-ತ್ತಲೇ
ನಡು-ಭಾ-ಗ-ದಲ್ಲಿ
ನಡು-ರಾ-ತ್ರಿಯ
ನಡು-ವಣ
ನಡು-ವಿನ
ನಡು-ವಿ-ನಲ್ಲಿ
ನಡುವೆ
ನಡು-ವೆಯೂ
ನಡು-ವೆಯೇ
ನಡೆ
ನಡೆದ
ನಡೆ-ದಂ-ತ-ಹದೇ
ನಡೆ-ದಂತೆ
ನಡೆ-ದ-ಅ-ವನ
ನಡೆ-ದ-ದ್ದರ
ನಡೆ-ದದ್ದು
ನಡೆ-ದರು
ನಡೆ-ದರೆ
ನಡೆ-ದಾಗ
ನಡೆ-ದಾ-ಡಿದ
ನಡೆ-ದಾ-ಡುತ್ತ
ನಡೆ-ದಾ-ಡು-ತ್ತಿ-ದ್ದರು
ನಡೆ-ದಾ-ಡು-ವ-ವರು
ನಡೆ-ದಾ-ಡು-ವಾ-ಗಲೇ
ನಡೆ-ದಿತ್ತು
ನಡೆ-ದಿದೆ
ನಡೆ-ದಿದ್ದ
ನಡೆ-ದಿ-ದ್ದುವು
ನಡೆ-ದಿ-ರ-ಬೇ-ಕೆಂದು
ನಡೆದು
ನಡೆ-ದು-ಕೊಂಡ
ನಡೆ-ದು-ಕೊಂ-ಡರು
ನಡೆ-ದು-ಕೊಂ-ಡರೂ
ನಡೆ-ದು-ಕೊಂ-ಡಿರಿ
ನಡೆ-ದು-ಕೊಂಡು
ನಡೆ-ದು-ಕೊ-ಳ್ಳ-ಬೇ-ಕಾ-ದರೆ
ನಡೆ-ದು-ಕೊ-ಳ್ಳ-ಬೇ-ಕೆಂದು
ನಡೆ-ದು-ಕೊ-ಳ್ಳಲು
ನಡೆ-ದು-ಕೊಳ್ಳು
ನಡೆ-ದು-ಕೊ-ಳ್ಳು-ತ್ತಾರೆ
ನಡೆ-ದು-ಕೊ-ಳ್ಳು-ತ್ತಿ-ದ್ದ-ರೆಂ-ದಲ್ಲ
ನಡೆ-ದು-ಕೊ-ಳ್ಳು-ವಂತೆ
ನಡೆ-ದು-ಕೊ-ಳ್ಳು-ವುದು
ನಡೆ-ದುದು
ನಡೆ-ದು-ಬಂತು
ನಡೆ-ದು-ಬಂ-ದರು
ನಡೆ-ದು-ಬಂದು
ನಡೆ-ದು-ಬ-ರು-ತ್ತಿ-ದ್ದರು
ನಡೆ-ದು-ಬ-ರು-ವಾಗ
ನಡೆ-ದು-ಬ-ರು-ವುದನ್ನು
ನಡೆ-ದು-ಬಿ-ಟ್ಟರು
ನಡೆ-ದುವು
ನಡೆ-ದು-ಹೋ-ಗಿತ್ತು
ನಡೆ-ದು-ಹೋ-ಗುವ
ನಡೆ-ದು-ಹೋದ
ನಡೆದೇ
ನಡೆ-ದೇ-ಬಿ-ಟ್ಟರು
ನಡೆ-ನು-ಡಿ-ಗಳ
ನಡೆ-ನು-ಡಿ-ಗಳನ್ನು
ನಡೆ-ನು-ಡಿ-ಯ-ನ್ನೆಲ್ಲ
ನಡೆ-ನು-ಡಿ-ಯಲ್ಲೂ
ನಡೆಯ
ನಡೆ-ಯ-ಬೇಕಾ
ನಡೆ-ಯ-ಬೇ-ಕಾ-ಗಿತ್ತು
ನಡೆ-ಯ-ಬೇ-ಕಾ-ಗಿದೆ
ನಡೆ-ಯ-ಬೇಕು
ನಡೆ-ಯ-ಬೇ-ಕು-ಅ-ವರು
ನಡೆ-ಯ-ಬೇ-ಕೆಂದು
ನಡೆ-ಯ-ಬೇ-ಕೆಂಬ
ನಡೆ-ಯ-ಲಿದೆ
ನಡೆ-ಯ-ಲಿ-ದೆ-ಯೆಂಬ
ನಡೆ-ಯ-ಲಿ-ದ್ದುದು
ನಡೆ-ಯ-ಲಿ-ರುವ
ನಡೆ-ಯ-ಲಿಲ್ಲ
ನಡೆ-ಯಲು
ನಡೆ-ಯಿತು
ನಡೆ-ಯಿರಿ
ನಡೆಯು
ನಡೆ-ಯುತ್ತ
ನಡೆ-ಯು-ತ್ತದೆ
ನಡೆ-ಯು-ತ್ತವೆ
ನಡೆ-ಯು-ತ್ತಾರೆ
ನಡೆ-ಯು-ತ್ತಿತ್ತು
ನಡೆ-ಯು-ತ್ತಿದೆ
ನಡೆ-ಯು-ತ್ತಿ-ದೆ-ಯಾ-ದರೂ
ನಡೆ-ಯು-ತ್ತಿ-ದೆ-ಯೆಂದು
ನಡೆ-ಯು-ತ್ತಿ-ದೆ-ಯೆಂಬ
ನಡೆ-ಯು-ತ್ತಿ-ದೆಯೋ
ನಡೆ-ಯು-ತ್ತಿದ್ದ
ನಡೆ-ಯು-ತ್ತಿ-ದ್ದಂ-ತಿತ್ತು
ನಡೆ-ಯು-ತ್ತಿ-ದ್ದಂತೆ
ನಡೆ-ಯು-ತ್ತಿ-ದ್ದರು
ನಡೆ-ಯು-ತ್ತಿ-ದ್ದರೂ
ನಡೆ-ಯು-ತ್ತಿದ್ದು
ನಡೆ-ಯು-ತ್ತಿ-ದ್ದುದು
ನಡೆ-ಯು-ತ್ತಿ-ದ್ದುವು
ನಡೆ-ಯು-ತ್ತಿ-ದ್ದೇ-ನೆಯೆ
ನಡೆ-ಯು-ತ್ತಿ-ರ-ಬ-ಹುದು
ನಡೆ-ಯು-ತ್ತಿ-ರ-ಲಿಲ್ಲ
ನಡೆ-ಯು-ತ್ತಿ-ರುವ
ನಡೆ-ಯು-ತ್ತಿ-ರು-ವಂ-ತಿತ್ತು
ನಡೆ-ಯು-ತ್ತೇನೆ
ನಡೆ-ಯುವ
ನಡೆ-ಯು-ವಂ-ತಹ
ನಡೆ-ಯು-ವಂತೆ
ನಡೆ-ಯುವು
ನಡೆ-ಯು-ವು-ದಾಗಿ
ನಡೆವ
ನಡೆ-ವ-ಳಿ-ಕೆ-ಗಳನ್ನೂ
ನಡೆ-ವ-ಳಿ-ಗಳ
ನಡೆಸ
ನಡೆ-ಸ-ಬೇ-ಕಾದ
ನಡೆ-ಸ-ಬೇ-ಕೆಂದು
ನಡೆ-ಸ-ಬೇ-ಕೆಂಬ
ನಡೆ-ಸ-ಬೇ-ಕೆ-ನ್ನು-ವ-ವರೆಲ್ಲ
ನಡೆ-ಸ-ಲಾ-ಯಿತು
ನಡೆ-ಸ-ಲಾ-ರಂ-ಭಿ-ಸಿ-ದಳು
ನಡೆ-ಸ-ಲಿ-ದ್ದರೋ
ನಡೆ-ಸ-ಲಿ-ದ್ದಾ-ರೆಂಬ
ನಡೆ-ಸಲು
ನಡೆ-ಸಲೂ
ನಡೆ-ಸ-ಲ್ಪ-ಟ್ಟಿ-ರುವ
ನಡೆಸಿ
ನಡೆ-ಸಿ-ಕೊಂಡು
ನಡೆ-ಸಿ-ಕೊಟ್ಟ
ನಡೆ-ಸಿದ
ನಡೆ-ಸಿ-ದಂತೆ
ನಡೆ-ಸಿ-ದತ್ತ
ನಡೆ-ಸಿ-ದರು
ನಡೆ-ಸಿ-ದರೆ
ನಡೆ-ಸಿ-ದೆವು
ನಡೆ-ಸಿದ್ದ
ನಡೆ-ಸಿ-ದ್ದರು
ನಡೆ-ಸಿದ್ದು
ನಡೆ-ಸಿಯೂ
ನಡೆಸು
ನಡೆ-ಸುತ್ತ
ನಡೆ-ಸು-ತ್ತಿದ್ದ
ನಡೆ-ಸು-ತ್ತಿ-ದ್ದನೋ
ನಡೆ-ಸು-ತ್ತಿ-ದ್ದರು
ನಡೆ-ಸು-ತ್ತಿ-ದ್ದರೋ
ನಡೆ-ಸು-ತ್ತಿ-ದ್ದಾಗ
ನಡೆ-ಸು-ತ್ತಿ-ದ್ದಾನೆ
ನಡೆ-ಸು-ತ್ತಿ-ದ್ದೀರಿ
ನಡೆ-ಸು-ತ್ತಿ-ದ್ದುದು
ನಡೆ-ಸು-ತ್ತಿ-ರು-ವಂತೆ
ನಡೆ-ಸು-ತ್ತಿ-ರು-ವ-ನೆಂಬ
ನಡೆ-ಸುವ
ನಡೆ-ಸು-ವಂ-ತಾ-ಗು-ತ್ತದೆ
ನಡೆ-ಸು-ವಂತೆ
ನಡೆ-ಸು-ವಾಗ
ನಡೆ-ಸುವು
ನಡೆ-ಸು-ವು-ಕ್ಕಿಂತ
ನಡೆ-ಸು-ವು-ದರ
ನದಿ
ನದಿ-ಗ-ಗಳೂ
ನದಿ-ಗಳನ್ನು
ನದಿ-ಗಳಿಂದ
ನದಿ-ಗ-ಳೆ-ಲ್ಲವೂ
ನದಿಯ
ನದಿ-ಯನ್ನು
ನದಿ-ಯಲ್ಲಿ
ನದಿಯಾ
ನದಿ-ಯಾದ್
ನದಿ-ಯಾದ್ಗೆ
ನದಿ-ಯಾ-ದ್ನಿಂದ
ನದಿಯು
ನದಿಯೂ
ನದಿ-ಯೆಂ-ದರೆ
ನದಿ-ಯೊಂದು
ನದೀ-ತೀ-ರದ
ನನ-ಗಂತೂ
ನನ-ಗ-ದ-ರಿಂ-ದೇನು
ನನ-ಗದು
ನನ-ಗ-ನ್ನಿ-ಸಿತು
ನನ-ಗ-ನ್ನಿ-ಸಿ-ತು-ಇ-ವೆಲ್ಲ
ನನ-ಗ-ನ್ನಿ-ಸು-ತ್ತದೆ
ನನ-ಗ-ನ್ನಿ-ಸು-ತ್ತ-ದೆ-ಏ-ಕೆಂ-ದರೆ
ನನ-ಗ-ನ್ನಿ-ಸು-ತ್ತ-ದೆಯೋ
ನನ-ಗ-ನ್ನಿ-ಸು-ತ್ತಿದೆ
ನನ-ಗ-ನ್ನಿ-ಸು-ತ್ತಿ-ದೆ-ಏ-ಕೆಂ-ದರೆ
ನನ-ಗಾಗಿ
ನನ-ಗಾ-ದರೂ
ನನ-ಗಾವ
ನನ-ಗಿಂತ
ನನ-ಗಿತ್ತು
ನನ-ಗಿದೆ
ನನ-ಗಿ-ದ್ದರೆ
ನನ-ಗಿನ್ನೂ
ನನ-ಗಿ-ರ-ಲಿಲ್ಲ
ನನ-ಗಿ-ರುವ
ನನ-ಗಿಲ್ಲ
ನನ-ಗಿಲ್ಲಿ
ನನ-ಗಿಷ್ಟ
ನನಗೀ
ನನ-ಗೀಗ
ನನಗೂ
ನನಗೆ
ನನ-ಗೆಂ-ದಿಗೂ
ನನ-ಗೆಂದೂ
ನನ-ಗೆ-ದು-ರಾ-ದರೂ
ನನ-ಗೆಲ್ಲೂ
ನನ-ಗೆಷ್ಟು
ನನಗೇ
ನನ-ಗೇಕೆ
ನನ-ಗೇಕೋ
ನನ-ಗೇ-ನಾ-ದರೂ
ನನ-ಗೇನು
ನನ-ಗೇನೂ
ನನ-ಗೇನೋ
ನನ-ಗೊಂದು
ನನ-ಗೊ-ಪ್ಪಿಸಿ
ನನ-ಗೊಬ್ಬ
ನನ-ಗೋ-ಸ್ಕರ
ನನಲ್ಲಿ
ನನ-ಸಾ-ಗಿ-ಸು-ವಲ್ಲಿ
ನನ-ಸಾ-ಗು-ತ್ತದೆ
ನನ-ಸಾ-ಯಿತು
ನನ್ನ
ನನ್ನ-ತ-ನ-ಗೊ-ಳಿ-ಸು-ತ್ತೇನೆ
ನನ್ನಂ-ತೆಯೇ
ನನ್ನಂ-ಥ-ವ-ನಿಗೆ
ನನ್ನಂ-ಥ-ವರು
ನನ್ನತ್ತ
ನನ್ನ-ದಾ-ಗಿತ್ತು
ನನ್ನದು
ನನ್ನದೂ
ನನ್ನದೇ
ನನ್ನ-ದೊಂದು
ನನ್ನನ್ನು
ನನ್ನ-ನ್ನು-ನಾ-ನೀಗ
ನನ್ನನ್ನೂ
ನನ್ನ-ನ್ನೆಂ-ದಿಗೂ
ನನ್ನನ್ನೇ
ನನ್ನ-ನ್ನೇಕೆ
ನನ್ನಲ್ಲಿ
ನನ್ನ-ಲ್ಲಿಗೆ
ನನ್ನ-ಲ್ಲಿದೆ
ನನ್ನ-ಲ್ಲಿ-ದ್ದರೆ
ನನ್ನ-ಲ್ಲಿ-ರುವ
ನನ್ನ-ಲ್ಲಿಲ್ಲ
ನನ್ನ-ಲ್ಲೊಂದು
ನನ್ನ-ವ-ನಾದ
ನನ್ನ-ವ-ರಿ-ಗ-ಲ್ಲದೆ
ನನ್ನ-ಷ್ಟಕ್ಕೆ
ನನ್ನಷ್ಟು
ನನ್ನಾತ್ಮ
ನನ್ನಿಂದ
ನನ್ನಿಂ-ದಾ-ದಷ್ಟು
ನನ್ನಿಂ-ದೇ-ನಾ-ದರೂ
ನನ್ನಿಚ್ಛೆ
ನನ್ನಿ-ಷ್ಟ-ದಂತೆ
ನನ್ನೀ
ನನ್ನು
ನನ್ನೂ
ನನ್ನೆ-ದೆ-ಯನ್ನು
ನನ್ನೆ-ದೆ-ಯಲ್ಲಿ
ನನ್ನೇ
ನನ್ನೊಂ-ದಿಗೆ
ನನ್ನೊಂದು
ನನ್ನೊ-ಡನೆ
ನನ್ನೊ-ಳಗೆ
ನನ್ನೋ
ನಮ
ನಮ-ಗಂತೂ
ನಮ-ಗ-ದನ್ನು
ನಮ-ಗ-ನ್ನಿ-ಸ-ಲಿಲ್ಲ
ನಮ-ಗ-ನ್ನಿ-ಸು-ತ್ತದೆ
ನಮ-ಗ-ರಿ-ವಿ-ಲ್ಲ-ದಂ-ತೆಯೇ
ನಮ-ಗ-ವರು
ನಮ-ಗಾಗಿ
ನಮ-ಗಾ-ಗು-ತ್ತಿದ್ದ
ನಮ-ಗಾವ
ನಮ-ಗಿಂತ
ನಮ-ಗಿಂದು
ನಮ-ಗಿ-ರುವ
ನಮ-ಗೀಗ
ನಮಗೆ
ನಮ-ಗೆಲ್ಲ
ನಮ-ಗೇಕೆ
ನಮ-ಗೇ-ನಾ-ದರೂ
ನಮ-ಗೇನು
ನಮ-ಗೊಂದು
ನಮ-ಸ್ಕ-ರಿಸಿ
ನಮ-ಸ್ಕ-ರಿ-ಸಿದ
ನಮ-ಸ್ಕ-ರಿ-ಸಿ-ದ-ನಂತೆ
ನಮ-ಸ್ಕ-ರಿ-ಸಿ-ದರು
ನಮ-ಸ್ಕ-ರಿ-ಸಿ-ದ-ವ-ನಲ್ಲ
ನಮ-ಸ್ಕ-ರಿ-ಸಿ-ದಾಗ
ನಮ-ಸ್ಕಾರ
ನಮಿ-ಸ-ದಿ-ರಲು
ನಮಿಸಿ
ನಮಿ-ಸಿತು
ನಮಿ-ಸಿ-ದರು
ನಮಿ-ಸುತ್ತ
ನಮೂನೆ
ನಮ್ಮ
ನಮ್ಮ-ತ-ನ-ವನ್ನು
ನಮ್ಮ-ತ-ನ-ವನ್ನೇ
ನಮ್ಮ-ದೆಂದೇ
ನಮ್ಮನ್ನು
ನಮ್ಮ-ನ್ನು-ದ್ದೇ-ಶಿಸಿ
ನಮ್ಮನ್ನೂ
ನಮ್ಮ-ನ್ನೆಲ್ಲ
ನಮ್ಮ-ರಾ-ಷ್ಟ್ರದ
ನಮ್ಮಲ್ಲಿ
ನಮ್ಮ-ಲ್ಲಿಗೆ
ನಮ್ಮ-ಲ್ಲಿನ
ನಮ್ಮಲ್ಲೂ
ನಮ್ಮಲ್ಲೇ
ನಮ್ಮ-ಲ್ಲೇಕೆ
ನಮ್ಮ-ಲ್ಲೊಂದು
ನಮ್ಮವ
ನಮ್ಮ-ವರ
ನಮ್ಮ-ವ-ರಾ-ದಂ-ತೆಯೇ
ನಮ್ಮ-ವ-ರಿಂ-ದಲೇ
ನಮ್ಮ-ಸ್ವಾ-ಮೀಜಿ
ನಮ್ಮಿಂದ
ನಮ್ಮಿ-ಬ್ಬ-ರಲ್ಲಿ
ನಮ್ಮಿ-ಬ್ಬ-ರೊಂ-ದಿಗೆ
ನಮ್ಮೂ-ರಿನ
ನಮ್ಮೆ-ದೆಯ
ನಮ್ಮೆಲ್ಲ
ನಮ್ಮೆ-ಲ್ಲರ
ನಮ್ಮೊಂ-ದಿ-ಗಿದ್ದ
ನಮ್ಮೊಂ-ದಿ-ಗಿ-ರು-ವಂ-ತಾ-ದರೆ
ನಮ್ಮೊಂ-ದಿಗೆ
ನಮ್ಮೊ-ಡನೆ
ನಮ್ಮೊ-ಳಗೆ
ನಮ್ರ
ನಮ್ರ-ಗೊ-ಳಿಸಿ
ನಮ್ರತೆ
ನಯ-ನ-ಗಳು
ನಯ-ನ-ದ್ವ-ಯವು
ನಯ-ನ-ಮ-ನೋ-ಹ-ರ-ರಾಕಿ
ನಯ-ನ-ಮ-ನೋ-ಹ-ರ-ವಾ-ಗಿದೆ
ನಯ-ವಾಗಿ
ನಯ-ವಾದ
ನಯ-ವಾ-ದುದು
ನರ-ಕ-ಕ್ಕಾ-ದರೂ
ನರ-ಕಕ್ಕೂ
ನರ-ಕಕ್ಕೆ
ನರ-ಕ-ಕ್ಕೆಈ
ನರ-ಕದ
ನರ-ಕವೋ
ನರ-ಗಳು
ನರ-ಗ-ಳೆಲ್ಲ
ನರ-ದೌ-ರ್ಬ-ಲ್ಯದ
ನರ-ಪೇ-ತಲ
ನರ-ಭ-ಕ್ಷ-ಕರು
ನರ-ಮಂ-ಡಲ
ನರ-ಮಂ-ಡ-ಲದ
ನರ-ಳು-ತ್ತಿದ್ದ
ನರ-ಳು-ತ್ತಿ-ದ್ದರು
ನರ-ಳು-ತ್ತಿ-ರುವ
ನರ-ಳುವ
ನರ-ಳು-ವುದು
ನರವೂ
ನರ-ಸಿಂಹ
ನರ-ಸಿಂ-ಹಾ-ಚಾರಿ
ನರ-ಸಿಂ-ಹಾ-ಚಾ-ರ್ಯರು
ನರೇಂದ್ರ
ನರೇಂ-ದ್ರ-ಕೃಷ್ಣ
ನರೇಂ-ದ್ರನ
ನರೇಂ-ದ್ರ-ನದೇ
ನರೇಂ-ದ್ರ-ನನ್ನು
ನರೇಂ-ದ್ರ-ನನ್ನೇ
ನರೇಂ-ದ್ರ-ನಲ್ಲ
ನರೇಂ-ದ್ರ-ನಾಥ
ನರೇಂ-ದ್ರ-ನಾ-ಥ-ದತ್ತ
ನರೇಂ-ದ್ರ-ನಿಗೂ
ನರೇಂ-ದ್ರ-ನಿಗೆ
ನರೇಂ-ದ್ರ-ನಿಗೇ
ನರೇಂ-ದ್ರನೇ
ನರೇಂ-ದ್ರ-ಮು-ನಿ-ಯನ್ನು
ನರೇನ್
ನರೇ-ನ್ಭಾಯಿ
ನರ್ತ-ಕಿಗೆ
ನರ್ತ-ಕಿಯ
ನರ್ತ-ನ-ಕೂ-ಟ-ಗಳ
ನರ್ಸರಿ
ನಲ-ದಂತೆ
ನಲ-ವತ್ತು
ನಲಿದು
ನಲಿ-ಯುತ್ತ
ನಲಿ-ಸುವ
ನಲು-ಗಿಲ್ಲ
ನಲ್ಲ
ನಲ್ಲದೆ
ನಲ್ಲಾ-ಗಲಿ
ನಲ್ಲಿ
ನಲ್ಲಿ-ದ್ದಾರೆ
ನಲ್ಲೂ
ನಲ್ಲೇ
ನವ
ನವ-ಚೇ-ತನ
ನವ-ಚೇ-ತ-ನ-ವ-ನ್ನೀ-ಯುವ
ನವ-ಚೇ-ತ-ನ-ವನ್ನು
ನವ-ಚೈ-ತನ್ಯ
ನವ-ಜಾ-ಗೃ-ತಿಗೆ
ನವ-ಜಾತ
ನವ-ಜೀ-ವ-ನ-ವ-ನ್ನೀ-ಯುವ
ನವ-ಜೀ-ವ-ನ-ವನ್ನು
ನವ-ನಿ-ರ್ಮಾ-ಣದ
ನವನು
ನವ-ನೂ-ತನ
ನವ-ಯು-ಗದ
ನವ-ಯು-ವಕ
ನವ-ಯು-ವ-ಕರು
ನವ-ಯು-ವ-ಕ-ರೆಲ್ಲ
ನವ-ಯು-ವ-ತಿ-ಯ-ರನ್ನು
ನವ-ರಾ-ತ್ರಿ-ಗಳನ್ನು
ನವ-ವಿ-ಧಾನ
ನವ-ವಿ-ಧಾ-ನದ
ನವ-ವೃಂ-ದಾ-ವನ
ನವ-ಸ್ಫೂರ್ತಿ
ನವ-ಸ್ಫೂ-ರ್ತಿ-ಯನ್ನು
ನವ-ಸ್ಫೂ-ರ್ತಿ-ಯ-ನ್ನು-ಕ್ಕಿ-ಸಿತು
ನವ-ಸ್ವಾ-ತಂತ್ರ್ಯ
ನವಾಬ
ನವಾ-ಬನ
ನವಾ-ಬ-ನಿಂದ
ನವಾ-ಬ-ನಿಗೆ
ನವಾಬ್
ನವಿ-ಲು-ಗ-ರಿಯ
ನವೀನ
ನವು
ನವೆಂ-ಬ-ರಿ-ನಲ್ಲಿ
ನವೆಂ-ಬರ್
ನವೆಂ-ಬ-ರ್-ಡಿ-ಸೆಂ-ಬರ್
ನವೆಂ-ಬ-ರ್ವ-ರೆಗೂ
ನವೋ-ತ್ಸಾಹ
ನವೋ-ತ್ಸಾ-ಹ-ದಿಂದ
ನವೋ-ತ್ಸಾ-ಹ-ವನ್ನು
ನವೋ-ತ್ಸಾ-ಹ-ವನ್ನೂ
ನಷ್ಟಕ್ಕೆ
ನಷ್ಟ-ವಂತೂ
ನಷ್ಟ-ವಾಗಿ
ನಷ್ಟ-ವಾದ
ನಷ್ಟ-ವಾ-ದರೆ
ನಷ್ಟ-ವಾ-ಯಿ-ತಲ್ಲ
ನಷ್ಟ-ವಾ-ಯಿತು
ನಷ್ಟ-ವೇನೂ
ನಸು-ಗೋ-ಪ-ದಿಂದ
ನಸು-ನಕ್ಕು
ನಸು-ನ-ಗುತ್ತ
ನಸು-ನ-ಗೆ-ಯಿಂದ
ನಸು-ಬೆ-ಳ-ಕಿ-ನೊಂ-ದಿಗೆ
ನಸುವೇ
ನಹೀ
ನಾ
ನಾಂದಿ
ನಾಂದಿ-ಯಾ-ಯಿತು
ನಾಗ-ರ-ಕೋ-ಯಿಲ್
ನಾಗ-ರ-ಹಾ-ವಿನ
ನಾಗ-ರ-ಹಾ-ವಿ-ನಂತೆ
ನಾಗ-ರ-ಹಾ-ವಿ-ನಂ-ಥ-ವರು
ನಾಗ-ರ-ಹಾವು
ನಾಗ-ರ-ಹಾ-ವೇ-ನಾ-ದರೂ
ನಾಗ-ರಿಕ
ನಾಗ-ರಿ-ಕ-ಸು-ಸಂ-ಸ್ಕೃತ
ನಾಗ-ರಿ-ಕತೆ
ನಾಗ-ರಿ-ಕ-ತೆ-ಗಳನ್ನು
ನಾಗ-ರಿ-ಕ-ತೆ-ಗಳು
ನಾಗ-ರಿ-ಕ-ತೆಗೆ
ನಾಗ-ರಿ-ಕ-ತೆಯ
ನಾಗ-ರಿ-ಕ-ತೆ-ಯೆಂ-ಬುದು
ನಾಗ-ರಿ-ಕ-ತೆಯೇ
ನಾಗ-ರಿ-ಕರ
ನಾಗ-ರಿ-ಕ-ರಲ್ಲಿ
ನಾಗ-ರಿ-ಕ-ರ-ಲ್ಲೊಬ್ಬ
ನಾಗ-ರಿ-ಕ-ರಾದ
ನಾಗ-ರಿ-ಕ-ರಿಂದ
ನಾಗ-ರಿ-ಕ-ರಿಗೆ
ನಾಗ-ರಿ-ಕರು
ನಾಗ-ರಿ-ಕಳೂ
ನಾಗ-ರಿ-ಕ-ವಾ-ದದ್ದು
ನಾಗಲು
ನಾಗ-ಸಾ-ಕಿ-ಯನ್ನು
ನಾಗಿ
ನಾಗಿ-ದ್ದ-ವನು
ನಾಗಿ-ರ-ಲಿಲ್ಲ
ನಾಗಿ-ರಲು
ನಾಗೇಂ-ದ್ರ-ನಾಥ
ನಾಗೋ-ರಿಗೆ
ನಾಚಿ
ನಾಚಿಕೆ
ನಾಚಿ-ಕೆ-ದುಃಖ
ನಾಚಿ-ಕೆ-ಗೇ-ಡಿನ
ನಾಚಿ-ಕೆ-ಯಾ-ಗಲಿ
ನಾಚಿ-ಕೆ-ಯಾ-ಗು-ವು-ದಿ-ಲ್ಲವೆ
ನಾಚಿ-ಕೆ-ಯಿಂದ
ನಾಚಿ-ಕೆಯೂ
ನಾಚಿ-ಕೊ-ಳ್ಳ-ಲಿಲ್ಲ
ನಾಚಿ-ಕೊ-ಳ್ಳು-ತ್ತಿ-ದ್ದರು
ನಾಚಿ-ದರು
ನಾಜೂ-ಕಾಗಿ
ನಾಜೂಕು
ನಾಜೂ-ಕು-ಗಾ-ರ-ರಾ-ಗು-ವುದು
ನಾಟಕ
ನಾಟ-ಕ-ಗಳಲ್ಲಿ
ನಾಟ-ಕ-ದಲ್ಲಿ
ನಾಟ-ಕ-ದೋ-ಪಾ-ದಿ-ಯಲ್ಲಿ
ನಾಟ-ಕೀ-ಯ-ತೆಯೂ
ನಾಟ-ಕೀ-ಯ-ವಾಗಿ
ನಾಟ-ಕೀ-ಯ-ವೆನ್ನ
ನಾಟು-ವಂ-ತೆ-ತಿ-ಳಿ-ಯ-ಪ-ಡಿ-ಸಿ-ದ್ದ-ರಿಂದ
ನಾಟ್ಯ-ವಾ-ಡು-ತ್ತಿ-ದ್ದವು
ನಾಡದೆ
ನಾಡಲು
ನಾಡಾದ
ನಾಡಿ
ನಾಡಿಗೆ
ನಾಡಿದ
ನಾಡಿ-ದರು
ನಾಡಿ-ದರೂ
ನಾಡಿ-ದಾ-ಗಲೂ
ನಾಡಿದ್ದು
ನಾಡಿನ
ನಾಡಿ-ನಲ್ಲಿ
ನಾಡಿ-ನ-ವನು
ನಾಡಿ-ನಿಂದ
ನಾಡಿ-ಸಿದ
ನಾಡಿ-ಸಿ-ದರು
ನಾಡು
ನಾಡುತ್ತ
ನಾಡು-ತ್ತಿದ್ದ
ನಾಡು-ತ್ತಿ-ದ್ದರು
ನಾಡುವ
ನಾಡು-ವಾಗ
ನಾಡೇ
ನಾಡೊಂ-ದ-ರಿಂದ
ನಾಣ್ಯ-ಗ-ಳ-ನ್ನಾ-ಗಿಸಿ
ನಾಣ್ಯದ
ನಾದ
ನಾದರೂ
ನಾದವು
ನಾದೆ
ನಾನಂತೂ
ನಾನ-ದಕ್ಕೆ
ನಾನ-ದನ್ನು
ನಾನ-ದ-ನ್ನೆಲ್ಲ
ನಾನಲ್ಲ
ನಾನ-ಲ್ಲಿಗೆ
ನಾನ-ವನ
ನಾನ-ವ-ನಿಗೆ
ನಾನ-ವರ
ನಾನ-ವ-ರನ್ನು
ನಾನ-ವ-ರಿ-ಗಾಗಿ
ನಾನ-ವ-ರಿಗೆ
ನಾನ-ವ-ಳಿಗೆ
ನಾನ-ವು-ಗಳ
ನಾನ-ವು-ಗಳನ್ನು
ನಾನಾ
ನಾನಾಗಿ
ನಾನಾ-ಗಿಯೇ
ನಾನಾ-ದರೋ
ನಾನಿ-ದ-ಕ್ಕೆಲ್ಲ
ನಾನಿ-ದನ್ನು
ನಾನಿ-ದರ
ನಾನಿನ್ನು
ನಾನಿನ್ನೂ
ನಾನಿ-ನ್ನೆಂ-ದಿಗೂ
ನಾನಿ-ನ್ನೇನು
ನಾನಿ-ರ-ಬ-ಹು-ದಾ-ಗಿದೆ
ನಾನಿ-ರು-ವ-ವ-ರೆಗೆ
ನಾನಿಲ್ಲಿ
ನಾನಿ-ಲ್ಲಿಗೆ
ನಾನಿ-ಲ್ಲಿ-ರು-ವ-ವ-ರೆಗೂ
ನಾನಿ-ಷ್ಟೆಲ್ಲ
ನಾನೀಗ
ನಾನೀ-ದೇ-ಶ-ದಲ್ಲಿ
ನಾನು
ನಾನು-ಅವೂ
ನಾನೂ
ನಾನೆಂಥ
ನಾನೆಂ-ದಿಗೂ
ನಾನೆಂದು
ನಾನೆಂದೂ
ನಾನೆಂಬ
ನಾನೆಲ್ಲೂ
ನಾನೆ-ಷ್ಟೆಷ್ಟೋ
ನಾನೇ
ನಾನೇಕೆ
ನಾನೇ-ನಾ-ದರೂ
ನಾನೇನು
ನಾನೇನೂ
ನಾನೇನೋ
ನಾನೊಂದು
ನಾನೊಬ್ಬ
ನಾನೊ-ಬ್ಬನೇ
ನಾನ್ಯಃ
ನಾನ್ಯಾರು
ನಾಪಿತ
ನಾಪಿ-ತನ
ನಾಪಿ-ತ-ನಿಗೆ
ನಾಮ
ನಾಮ-ಕ-ರಣ
ನಾಮ-ರೂ-ಪವ
ನಾಮ-ವನ್ನು
ನಾಮಾ-ವ-ಶೇಷ
ನಾಮಾ-ವ-ಶೇ-ಷ-ವಾ-ಗ-ಬ-ಹುದು
ನಾಮುಂದು
ನಾಯಕ
ನಾಯ-ಕ-ತ್ವ-ದಲ್ಲಿ
ನಾಯ-ಕನ
ನಾಯ-ಕ-ನ-ನ್ನಾಗಿ
ನಾಯ-ಕ-ನೆಂದು
ನಾಯ-ಕ-ನೆಂಬ
ನಾಯ-ಕರ
ನಾಯ-ಕ-ರಾದ
ನಾಯ-ಕರು
ನಾಯ-ಕ-ರೆ-ನ್ನಿ-ಸಿ-ಕೊಂ-ಡ-ವರ
ನಾಯಕ್
ನಾಯಿ
ನಾಯಿ-ಕು-ನ್ನಿ-ಗಳ
ನಾಯಿ-ಯನ್ನು
ನಾಯ್ಕರು
ನಾಯ್ಡು-ರ-ವರ
ನಾರ
ನಾರದ
ನಾರ-ದೀಯ
ನಾರಾ
ನಾರಾ-ಯಣ
ನಾರಾ-ಯ-ಣ-ದಾಸ್
ನಾರಾ-ಯ-ಣ-ನನ್ನು
ನಾರಾ-ಯ-ಣರ
ನಾರಾ-ಯ-ಣರು
ನಾರಾ-ಯಣಿ
ನಾರಿ-ಯ-ರಿಗೆ
ನಾರಿ-ಯರು
ನಾರೀ-ತ್ವದ
ನಾರ್ತ್
ನಾರ್ಥಾಂ-ಟನ್
ನಾರ್ಥಾಂ-ಟನ್ನ
ನಾರ್ಥಾಂ-ಟ-ನ್ನಿಂದ
ನಾರ್ವೇ
ನಾಲಿಗೆ
ನಾಲಿ-ಗೆ-ಲೇ-ಖನಿ
ನಾಲಿ-ಗೆಗೂ
ನಾಲಿ-ಗೆ-ಯಿಂದ
ನಾಲ್ಕ-ಕ್ಷ-ರ-ದಿಂದ
ನಾಲ್ಕ-ನೆಯ
ನಾಲ್ಕ-ರಂದು
ನಾಲ್ಕಾರು
ನಾಲ್ಕು
ನಾಲ್ಕೇ
ನಾಲ್ಕೈದು
ನಾಲ್ದೆ-ಸೆ-ಗ-ಳಲ್ಲೂ
ನಾಲ್ವ-ರನ್ನು
ನಾಲ್ವರು
ನಾಳಿನ
ನಾಳೆ
ನಾಳೆ-ದಿನ
ನಾಳೆಯ
ನಾಳೆಯೋ
ನಾವಂತೂ
ನಾವ-ದನ್ನು
ನಾವ-ಲ್ಲಿಗೆ
ನಾವ-ವ-ರನ್ನು
ನಾವಿಂದು
ನಾವಿನ್ನು
ನಾವಿನ್ನೂ
ನಾವಿ-ಬ್ಬರು
ನಾವಿ-ಬ್ಬರೂ
ನಾವಿಲ್ಲಿ
ನಾವೀಗ
ನಾವು
ನಾವು-ಹಿಂ-ದೂ-ಗ-ಳು-ಇನ್ನೂ
ನಾವೂ
ನಾವೆಂ-ದಾ-ದರೂ
ನಾವೆಂ-ದಿಗೂ
ನಾವೆಂದೂ
ನಾವೆಲ್ಲ
ನಾವೆ-ಲ್ಲರೂ
ನಾವೆಷ್ಟೋ
ನಾವೇ
ನಾವೇಕೆ
ನಾವೊಂದು
ನಾಶ
ನಾಶಕ್ಕೆ
ನಾಶ-ಗೈ-ಯು-ವು-ದ-ಕ್ಕಾಗಿ
ನಾಶ-ಗೊ-ಳಿ-ಸಿ-ರು-ವಾಗ
ನಾಶ-ಗೊ-ಳಿ-ಸು-ವು-ದ-ರಲ್ಲಿ
ನಾಶ-ಪ-ಡಿ-ಸುವ
ನಾಶ-ಮಾ-ಡಲಿ
ನಾಶ-ಮಾ-ಡಿವೆ
ನಾಶ-ವಲ್ಲ
ನಾಶ-ವಾ-ಗ-ಲಾ-ರ-ದೆಂಬ
ನಾಶ-ವಾ-ಗಲು
ನಾಶ-ವಾಗಿ
ನಾಶ-ವಾ-ಗು-ತ್ತದೆ
ನಾಶ-ವಾ-ಗು-ತ್ತವೆ
ನಾಸಿಕ್
ನಾಸ್ತಿಕ
ನಾಸ್ತಿ-ಕ-ಹೇ-ಡಿ-ಗಳಲ್ಲಿ
ನಾಸ್ತಿ-ಕತೆ
ನಾಸ್ತಿ-ಕ-ತೆ-ಯನ್ನೂ
ನಾಸ್ತಿ-ಕ-ನಾ-ದರೂ
ನಾಸ್ತಿ-ಕ-ರಾದ
ನಾಸ್ತಿ-ಕ-ರೆಂದೂ
ನಾಸ್ತಿ-ಕರೇ
ನಾಸ್ತಿ-ಕ್ಯ-ವಾ-ದಿಯೂ
ನಿಂಡೆ
ನಿಂಡೆ-ಯ-ವ-ರನ್ನೂ
ನಿಂತ
ನಿಂತಂತೆ
ನಿಂತ-ದ್ದನ್ನು
ನಿಂತರ
ನಿಂತರು
ನಿಂತರೆ
ನಿಂತ-ರೆಂ-ಬು-ದನ್ನು
ನಿಂತ-ವನು
ನಿಂತ-ವರ
ನಿಂತ-ವ-ರಿಗೆ
ನಿಂತ-ವರು
ನಿಂತವು
ನಿಂತಾಗ
ನಿಂತಿತು
ನಿಂತಿತ್ತು
ನಿಂತಿದೆ
ನಿಂತಿದ್ದ
ನಿಂತಿ-ದ್ದಂ-ತೆಯೇ
ನಿಂತಿ-ದ್ದರು
ನಿಂತಿ-ದ್ದಳು
ನಿಂತಿ-ದ್ದಾಗ
ನಿಂತಿ-ದ್ದಾನೆ
ನಿಂತಿ-ದ್ದಾರೆ
ನಿಂತಿ-ದ್ದುದು
ನಿಂತಿ-ದ್ದೇನೆ
ನಿಂತಿ-ದ್ದೇವೆ
ನಿಂತಿ-ರ-ಬೇಕೆ
ನಿಂತಿರು
ನಿಂತಿ-ರು-ತ್ತಿದ್ದ
ನಿಂತಿ-ರು-ತ್ತಿ-ದ್ದರು
ನಿಂತಿ-ರುವ
ನಿಂತಿ-ರು-ವಾಗ
ನಿಂತಿ-ರು-ವುದನ್ನು
ನಿಂತಿ-ರು-ವು-ದ-ರಿಂದ
ನಿಂತಿ-ರು-ವುದು
ನಿಂತಿವೆ
ನಿಂತಿ-ವೆ-ಯೆಂ-ಬು-ದನ್ನು
ನಿಂತು
ನಿಂತು-ಕೊ-ಳಲು
ನಿಂತು-ಕೊ-ಳ್ಳುವ
ನಿಂತು-ಬಿಟ್ಟ
ನಿಂತು-ಬಿ-ಟ್ಟರು
ನಿಂತು-ಬಿ-ಟ್ಟಳು
ನಿಂತು-ಬಿ-ಟ್ಟಿತು
ನಿಂತು-ಬಿ-ಟ್ಟಿ-ರಲ್ಲ
ನಿಂತು-ಬಿಟ್ಟು
ನಿಂತು-ಬಿ-ಡು-ತ್ತಿ-ದ್ದರು
ನಿಂತು-ಹೋಗಿ
ನಿಂತು-ಹೋ-ಗಿದೆ
ನಿಂತು-ಹೋ-ಯಿತು
ನಿಂತೆ
ನಿಂತೇ
ನಿಂತೇ-ಹೋ-ಗು-ವಂ-ತಿತ್ತು
ನಿಂದ
ನಿಂದ-ಕರ
ನಿಂದ-ಕ-ರನ್ನು
ನಿಂದ-ಕ-ರಿ-ಗೆಲ್ಲ
ನಿಂದ-ಕ-ವರ್ಗ
ನಿಂದ-ನೆಯ
ನಿಂದಲೇ
ನಿಂದಿಸಿ
ನಿಂದಿ-ಸಿ-ದರು
ನಿಂದಿ-ಸಿ-ದರೆ
ನಿಂದಿ-ಸು-ತ್ತಿದ್ದ
ನಿಂದಿ-ಸುವ
ನಿಂದಿ-ಸು-ವಷ್ಟು
ನಿಂದೆ-ಯನ್ನೂ
ನಿಂದೆ-ಯ-ನ್ನೆಲ್ಲ
ನಿಂಬೆ
ನಿಃಸ್ವಾರ್ಥ
ನಿಃಸ್ವಾ-ರ್ಥ-ಭಾ-ವ-ದಿಂದ
ನಿಃಸ್ವಾರ್ಥಿ
ನಿಃಸ್ವಾ-ರ್ಥಿ-ಯ-ನ್ನಾಗಿ
ನಿಃಸ್ವಾ-ರ್ಥಿಯೆ
ನಿಕಟ
ನಿಕ-ಟ-ವ-ರ್ತಿ-ಗಳು
ನಿಕ-ಟ-ವಾ-ಗಿತ್ತು
ನಿಕ-ಟ-ವಾ-ದಂತೆ
ನಿಕ-ಟ-ವಾ-ದಾಗ
ನಿಕ-ಟ-ಸಂ-ಪ-ರ್ಕಕ್ಕೆ
ನಿಕೃ-ಷ್ಟ-ರಾ-ದ-ವರು
ನಿಕೃ-ಷ್ಟ-ವೆಂ-ಬು-ದನ್ನೂ
ನಿಕೋಲಾ
ನಿಖ-ರ-ನಿ-ಶ್ಚಿತ
ನಿಖ-ರತೆ
ನಿಖ-ರ-ತೆಯೂ
ನಿಖ-ರ-ವಾಗಿ
ನಿಖ-ರ-ವಾ-ಗಿ-ರು-ತ್ತಿತ್ತು
ನಿಖ-ರ-ವಾ-ಗಿಲ್ಲ
ನಿಖ-ರವೂ
ನಿಗ-ದಿತ
ನಿಗ-ದಿ-ತ-ವಾ-ಗಿ-ರುವ
ನಿಗ-ದಿ-ತ-ವಾದ
ನಿಗೂ-ಢ-ವ್ಯ-ಕ್ತಿ-ಯಿಂದ
ನಿಗೆ
ನಿಗೇ
ನಿಗ್ರ-ಹಿ-ಸಿ-ಕೊ-ಳ್ಳು-ವುದು
ನಿಗ್ರ-ಹಿ-ಸು-ವು-ದಾ-ದರೂ
ನಿಗ್ರೋ-ಗಳು
ನಿಜ
ನಿಜಕ್ಕೂ
ನಿಜ-ಪ್ರ-ತಿಭೆ
ನಿಜ-ವ-ನ-ರಿ-ತ-ವ-ನಲ್ಲಿ
ನಿಜ-ವನ್ನು
ನಿಜ-ವಾಗಿ
ನಿಜ-ವಾ-ಗಿತ್ತು
ನಿಜ-ವಾ-ಗಿ-ದ್ದರೆ
ನಿಜ-ವಾ-ಗಿ-ರಲು
ನಿಜ-ವಾ-ಗಿ-ರು-ತ್ತಿತ್ತು
ನಿಜ-ವಾದ
ನಿಜ-ವಾ-ದದ್ದು
ನಿಜ-ವಾ-ಯಿತು
ನಿಜವೆ
ನಿಜ-ವೆಂದು
ನಿಜ-ವೆಂದೇ
ನಿಜವೇ
ನಿಜ-ವ್ಯ-ಕ್ತಿ-ತ್ವದ
ನಿಜ-ಶ-ಕ್ತಿ-ಯನ್ನು
ನಿಜ-ಸಂ-ಗತಿ
ನಿಜ-ಸಂ-ಗ-ತಿಯ
ನಿಜ-ಸ್ಥಿತಿ
ನಿಜ-ಸ್ಥಿ-ತಿ-ಯನ್ನು
ನಿಜ-ಸ್ವ-ಭಾ-ವವು
ನಿಜ-ಸ್ವ-ರೂಪ
ನಿಜ-ಸ್ವ-ರೂ-ಪ-ವನ್ನು
ನಿಜ-ಸ್ವ-ರೂ-ಪವು
ನಿಜ-ಸ್ವ-ರೂ-ಪವೇ
ನಿಜಾ-ಮರ
ನಿಜಾಮ್
ನಿಜಾ-ರ್ಥ-ದಲ್ಲಿ
ನಿಜಾ-ರ್ಥ-ವನ್ನು
ನಿಜಾ-ರ್ಥ-ವೇನು
ನಿಟ್ಟಾಗಿ
ನಿಟ್ಟಿ-ನಲ್ಲಿ
ನಿಟ್ಟಿ-ಸುತ್ತ
ನಿಟ್ಟು-ಸಿರು
ನಿಟ್ಟು-ಸಿ-ರೆ-ಳೆ-ದರು
ನಿತ್ಯ
ನಿತ್ಯ-ಕ-ರ್ಮ-ಗ-ಳಿಗೆ
ನಿತ್ಯದ
ನಿತ್ಯ-ನೂ-ತ-ನ-ವಾ-ಗಿದೆ
ನಿತ್ಯ-ವೆಂಬ
ನಿತ್ಯ-ಹ-ರಿ-ದ್ವ-ರ್ಣದ
ನಿತ್ಯೋ-ತ್ಸವ
ನಿತ್ರಾ-ಣ-ರಾ-ಗಿ-ದ್ದದ್ದು
ನಿದ-ರ್ಶ-ನ-ಗಳ
ನಿದ-ರ್ಶ-ನ-ಗ-ಳಿ-ದ್ದರೂ
ನಿದ-ರ್ಶ-ನ-ಗ-ಳಿವೆ
ನಿದ-ರ್ಶ-ನ-ರೂ-ಪ-ವಾಗಿ
ನಿದ-ರ್ಶ-ನ-ವನ್ನು
ನಿದ-ರ್ಶ-ನವೇ
ನಿದ್ರಿ-ಸ-ಬ-ಲ್ಲಿರಿ
ನಿದ್ರಿ-ಸಿತು
ನಿದ್ರಿ-ಸಿ-ದರು
ನಿದ್ರಿ-ಸು-ವಂತೆ
ನಿದ್ರೆ
ನಿದ್ರೆ-ಯಿ-ಲ್ಲ-ವೆಂದರೆ
ನಿದ್ರೆಯೇ
ನಿಧ-ನದ
ನಿಧಾನ
ನಿಧಾ-ನ-ವಾ-ಗ-ಬ-ಹುದು
ನಿಧಾ-ನ-ವಾಗಿ
ನಿಧಾ-ನ-ವಾ-ಗಿ-ಯಾ-ದರೂ
ನಿಧಾ-ನ-ವಾ-ದುವು
ನಿಧಿ
ನಿಧಿ-ಗಾಗಿ
ನಿಧಿಗೆ
ನಿಧಿ-ಯಂತೆ
ನಿಧಿ-ಯನ್ನು
ನಿಧಿ-ಯಾದ
ನಿಧಿ-ಸಂ-ಗ್ರ-ಹಣೆ
ನಿಧಿ-ಸಂ-ಗ್ರ-ಹ-ಣೆಗೆ
ನಿಧಿ-ಸಂ-ಗ್ರ-ಹ-ಣೆ-ಯಲ್ಲಿ
ನಿನ-ಗದು
ನಿನ-ಗ-ನಿ-ಸು-ತ್ತ-ದೆಯೆ
ನಿನ-ಗ-ನ್ನಿ-ಸು-ತ್ತ-ದೆಯೆ
ನಿನಗಾ
ನಿನ-ಗಾಗಿ
ನಿನ-ಗಿದು
ನಿನ-ಗಿದೆ
ನಿನ-ಗಿ-ದೆ-ಯಲ್ಲ
ನಿನ-ಗಿ-ದೆಲ್ಲ
ನಿನಗೆ
ನಿನ-ಗೆಂ-ದಿಗೂ
ನಿನ-ಗೇಕೆ
ನಿನ-ಗೇನು
ನಿನ-ಗೋ-ಸ್ಕರ
ನಿನಾ-ದ-ಗೈ-ಯುತ್ತ
ನಿನ್ನ
ನಿನ್ನಂ-ತಹ
ನಿನ್ನಂತೆ
ನಿನ್ನಂ-ಥ-ವ-ರಿಗೆ
ನಿನ್ನ-ಡಿಯ
ನಿನ್ನ-ದಾ-ಗ-ಲೆಂದು
ನಿನ್ನದು
ನಿನ್ನನ್ನು
ನಿನ್ನನ್ನೂ
ನಿನ್ನನ್ನೇ
ನಿನ್ನಲ್ಲಿ
ನಿನ್ನ-ಲ್ಲಿಗೆ
ನಿನ್ನವ
ನಿನ್ನ-ವನು
ನಿನ್ನಿಂದ
ನಿನ್ನಿ-ಚ್ಛೆ-ಯೊಲು
ನಿನ್ನೆ
ನಿನ್ನೆಯೋ
ನಿನ್ನೊಂ-ದಿ-ಗಿ-ದ್ದಾನೆ
ನಿನ್ನೊಂ-ದಿಗೆ
ನಿನ್ನೊ-ಳಗೆ
ನಿಬಿ-ಡಾಂ-ಧ-ಕಾ-ರ-ದಲ್ಲಿ
ನಿಬ್ಬೆ-ರ-ಗಾಗಿ
ನಿಬ್ಬೆ-ರ-ಗಾ-ಗು-ತ್ತಿ-ದ್ದರು
ನಿಬ್ಬೆ-ರ-ಗಾ-ದರು
ನಿಭಾ-ಯಿ-ಸಿದ
ನಿಭಾ-ಯಿ-ಸುತ್ತೀ
ನಿಮ-ಗದು
ನಿಮ-ಗ-ನ್ನಿ-ಸು-ವುದಾ
ನಿಮ-ಗಾಗಿ
ನಿಮ-ಗಿಂತ
ನಿಮ-ಗಿದು
ನಿಮ-ಗಿಲ್ಲ
ನಿಮ-ಗಿ-ಲ್ಲವೋ
ನಿಮಗೆ
ನಿಮ-ಗೆ-ದು-ರಾಗಿ
ನಿಮ-ಗೆ-ದು-ರಾ-ದರೂ
ನಿಮ-ಗೆಲ್ಲ
ನಿಮ-ಗೆ-ಲ್ಲ-ರಿಗೂ
ನಿಮಗೇ
ನಿಮ-ಗೇ-ನಾ-ದರೂ
ನಿಮ-ಗೇನು
ನಿಮ-ಗೇನೂ
ನಿಮ-ಗೇ-ನೆ-ನ್ನಿ-ಸಿತು
ನಿಮ-ಗೊಂದು
ನಿಮ-ಗೊ-ಪ್ಪಿ-ಸಿ-ಹೋಗು
ನಿಮಿ-ತ್ತ-ವಾಗಿ
ನಿಮಿಷ
ನಿಮಿ-ಷ-ಕ್ಕಿಂತ
ನಿಮಿ-ಷ-ಗಳ
ನಿಮಿ-ಷ-ಗ-ಳಾ-ಗಿ-ರ-ಬ-ಹುದು
ನಿಮಿ-ಷ-ದಲ್ಲಿ
ನಿಮಿ-ಷ-ವಾ-ಗಿ-ರ-ಬೇಕು
ನಿಮಿ-ಷ-ವಾ-ದರೂ
ನಿಮಿ-ಷ-ವಿ-ರು-ವಾ-ಗಲೇ
ನಿಮಿ-ಷವೂ
ನಿಮ್ನ
ನಿಮ್ನ-ವ-ರ್ಗ-ದ-ವ-ರನ್ನು
ನಿಮ್ಮ
ನಿಮ್ಮಂ-ತಹ
ನಿಮ್ಮಂ-ಥ-ವರು
ನಿಮ್ಮ-ಗಳ
ನಿಮ್ಮ-ದಾ-ಗಿ-ಬಿ-ಡು-ತ್ತದೆ
ನಿಮ್ಮದು
ನಿಮ್ಮದೇ
ನಿಮ್ಮದ್ದು
ನಿಮ್ಮ-ನಿ-ಮ್ಮಲ್ಲಿ
ನಿಮ್ಮನ್ನು
ನಿಮ್ಮನ್ನೂ
ನಿಮ್ಮ-ನ್ನೆಲ್ಲ
ನಿಮ್ಮನ್ನೇ
ನಿಮ್ಮಲ್ಲಿ
ನಿಮ್ಮ-ಲ್ಲಿಗೆ
ನಿಮ್ಮ-ಲ್ಲಿ-ಬ-ಡ-ವ-ರಾ-ದರೂ
ನಿಮ್ಮ-ಲ್ಲಿ-ರುವ
ನಿಮ್ಮ-ಲ್ಲಿ-ಲ್ಲವೆ
ನಿಮ್ಮಲ್ಲೇ
ನಿಮ್ಮ-ಲ್ಲೇ-ನಾ-ದರೂ
ನಿಮ್ಮ-ಷ್ಟಕ್ಕೆ
ನಿಮ್ಮಷ್ಟು
ನಿಮ್ಮಿಂದ
ನಿಮ್ಮಿಂ-ದಾಗಿ
ನಿಮ್ಮಿಂ-ದಾ-ದೀತೆ
ನಿಮ್ಮೆಲ್ಲ
ನಿಮ್ಮೆ-ಲ್ಲರ
ನಿಮ್ಮೆ-ಲ್ಲ-ರನ್ನೂ
ನಿಮ್ಮೆ-ಲ್ಲ-ರಿಂ-ದಲೂ
ನಿಮ್ಮೆ-ಲ್ಲ-ರಿಗೂ
ನಿಮ್ಮೆ-ಲ್ಲ-ರೆ-ಡೆಗೂ
ನಿಮ್ಮೊಂ-ದಿಗೆ
ನಿಮ್ಮೊ-ಡ-ನಿ-ದ್ದಾಳೆ
ನಿಮ್ಮೊ-ಳಗೇ
ನಿಯಂ-ತ್ರ-ಣದ
ನಿಯಂ-ತ್ರಿ-ಸ-ಬಲ್ಲ
ನಿಯಂ-ತ್ರಿ-ಸ-ಬೇ-ಕಾ-ಗು-ತ್ತಿದ್ದ
ನಿಯಂ-ತ್ರಿಸಿ
ನಿಯಂ-ತ್ರಿ-ಸು-ತ್ತಾಳೆ
ನಿಯಂ-ತ್ರಿ-ಸು-ತ್ತಿ-ರುವ
ನಿಯ-ತ-ಕಾ-ಲಿಕ
ನಿಯ-ತ-ಕಾ-ಲಿ-ಕ-ಗಳ
ನಿಯ-ತ-ಕಾ-ಲಿ-ಕ-ಗಳು
ನಿಯ-ತ-ವಾಗಿ
ನಿಯಮ
ನಿಯ-ಮ-ಗಳ
ನಿಯ-ಮ-ಗಳನ್ನು
ನಿಯ-ಮ-ಗಳನ್ನೂ
ನಿಯ-ಮ-ಗಳನ್ನೆಲ್ಲ
ನಿಯ-ಮ-ಗಳಿಂದ
ನಿಯ-ಮ-ಗ-ಳಿ-ಗ-ನು-ಸಾ-ರ-ವಾಗಿ
ನಿಯ-ಮ-ಗ-ಳಿಗೆ
ನಿಯ-ಮ-ಗಳು
ನಿಯ-ಮದ
ನಿಯ-ಮ-ಪ-ರಿ-ಪಾ-ಲ-ನೆಯ
ನಿಯ-ಮ-ವೆಂದರೆ
ನಿಯ-ಮವೇ
ನಿಯ-ಮಾ-ನು-ಸಾ-ರ-ವಾಗಿ
ನಿಯ-ಮಾ-ವ-ಳಿ-ಗ-ಳಾ-ಗಲಿ
ನಿಯ-ಮಿ-ತ-ವಾಗಿ
ನಿಯ-ಮಿ-ಸುವ
ನಿಯೋ-ಗದ
ನಿಯೋ-ಜಿತ
ನಿಯೋ-ಜಿ-ತ-ಗೊ-ಳಿ-ಸಿ-ದರು
ನಿರಂ-ಕು-ಶ-ಮ-ತಿ-ಯಾದ
ನಿರಂ-ತರ
ನಿರಂ-ತ-ರ-ವಾಗಿ
ನಿರಂ-ತ-ರವೂ
ನಿರಂ-ತ-ರ-ಸ್ಫೂ-ರ್ತಿಯ
ನಿರ-ಕ್ಷ-ರ-ಕು-ಕ್ಷಿ-ಗಳೂ
ನಿರ-ಕ್ಷ-ರ-ಸ್ಥ-ರು-ವಿ-ದ್ಯಾ-ವಂ-ತ-ರು-ಹೀಗೆ
ನಿರ-ತ-ರಾ-ಗಿ-ದ್ದರು
ನಿರ-ತ-ರಾ-ಗಿ-ದ್ದಾರೆ
ನಿರ-ತ-ರಾ-ಗಿ-ದ್ದೇವೆ
ನಿರ-ತ-ರಾ-ಗಿ-ರು-ತ್ತಿ-ದ್ದರು
ನಿರ-ತ-ರಾ-ಗಿ-ರು-ವುದೇ
ನಿರ-ತ-ರಾ-ದರು
ನಿರ-ಪೇಕ್ಷ
ನಿರ-ಪೇ-ಕ್ಷ-ವಾ-ದದ್ದು
ನಿರ-ಪೇ-ಕ್ಷ-ವಾ-ದದ್ದೇ
ನಿರ-ರ್ಗಳ
ನಿರ-ರ್ಗ-ಳತೆ
ನಿರ-ರ್ಗ-ಳ-ವಾಗಿ
ನಿರ-ರ್ಥ-ಕ-ವ-ಲ್ಲವೆ
ನಿರ-ರ್ಥ-ಕ-ವಾ-ಗಿ-ಲ್ಲ-ವೆಂದು
ನಿರ-ರ್ಥ-ಕವೋ
ನಿರಾ
ನಿರಾ-ಕ-ರಿ-ಸ-ಬ-ಹು-ದಾ-ಗಿತ್ತು
ನಿರಾ-ಕ-ರಿ-ಸ-ಲಾ-ಗದೆ
ನಿರಾ-ಕ-ರಿಸಿ
ನಿರಾ-ಕ-ರಿ-ಸಿ-ದರು
ನಿರಾ-ಕ-ರಿ-ಸಿ-ದಳು
ನಿರಾ-ಕ-ರಿ-ಸಿ-ಬಿ-ಟ್ಟರು
ನಿರಾ-ಕ-ರಿ-ಸು-ತ್ತಲೇ
ನಿರಾ-ಕ-ರಿ-ಸು-ತ್ತಾನೆ
ನಿರಾ-ಕ-ರಿ-ಸು-ತ್ತಿ-ದ್ದರೂ
ನಿರಾ-ಕ-ರಿ-ಸು-ವು-ದ-ರಲ್ಲಿ
ನಿರಾ-ಕಾರ
ನಿರಾ-ಕಾ-ರದ
ನಿರಾ-ಕಾ-ರವೂ
ನಿರಾ-ತಂ-ಕ-ವಾಗಿ
ನಿರಾ-ಳ-ವಾ-ಗಿತ್ತು
ನಿರಾ-ಳ-ವಾ-ಗಿ-ರು-ತ್ತದೆ
ನಿರಾ-ಳ-ವಾದ
ನಿರಾ-ಶ-ನಾ-ಗಿಲ್ಲ
ನಿರಾ-ಶ-ರಾ-ಗದೆ
ನಿರಾ-ಶ-ರಾ-ಗ-ಬೇ-ಕಾ-ಯಿತು
ನಿರಾ-ಶ-ರಾ-ಗ-ಲಿಲ್ಲ
ನಿರಾ-ಶ-ರಾಗಿ
ನಿರಾ-ಶ-ರಾ-ಗಿ-ದ್ದರು
ನಿರಾ-ಶ-ರಾ-ಗಿ-ದ್ದರೋ
ನಿರಾ-ಶ-ರಾ-ಗಿಯೇ
ನಿರಾ-ಶಾ-ದಾ-ಯಕ
ನಿರಾ-ಶಾ-ದಾ-ಯ-ಕ-ವಾ-ಗಿಯೇ
ನಿರಾ-ಶಾ-ದಾ-ಯ-ಕ-ವಾ-ಗೇನೂ
ನಿರಾ-ಶಾ-ಭಾ-ವಕ್ಕೆ
ನಿರಾಶೆ
ನಿರಾ-ಶೆಯ
ನಿರಾ-ಶೆ-ಯಾ-ಯಿತು
ನಿರಾ-ಶೆ-ಯಿಂದ
ನಿರಾ-ಶೆ-ಯುಂ-ಟು-ಮಾ-ಡ-ಲಾ-ರದೆ
ನಿರಾ-ಶೆ-ಯುಂ-ಟು-ಮಾಡಿ
ನಿರಾ-ಶೆಯೂ
ನಿರಾ-ಶೆಯೇ
ನಿರಾ-ಶ್ರಿ-ತ-ರಿಗೆ
ನಿರಾ-ಸಕ್ತ
ನಿರಾ-ಸಕ್ತಿ
ನಿರಾ-ಸೆ-ಯಾ-ಯಿತು
ನಿರೀ-ಕ್ಷ-ಣೆ-ಯನ್ನೂ
ನಿರೀಕ್ಷಿ
ನಿರೀ-ಕ್ಷಿಗೆ
ನಿರೀ-ಕ್ಷಿತ
ನಿರೀ-ಕ್ಷಿಸ
ನಿರೀ-ಕ್ಷಿ-ಸ-ತೊ-ಡ-ಗಿ-ದರು
ನಿರೀ-ಕ್ಷಿ-ಸ-ಬೇಡ
ನಿರೀ-ಕ್ಷಿ-ಸ-ಬೇಡಿ
ನಿರೀ-ಕ್ಷಿ-ಸ-ಲೇ-ಬೇ-ಡಿರಿ
ನಿರೀ-ಕ್ಷಿಸಿ
ನಿರೀ-ಕ್ಷಿ-ಸಿ-ಕೊಂಡು
ನಿರೀ-ಕ್ಷಿ-ಸಿದ
ನಿರೀ-ಕ್ಷಿ-ಸಿ-ದರು
ನಿರೀ-ಕ್ಷಿ-ಸಿದ್ದ
ನಿರೀ-ಕ್ಷಿ-ಸಿ-ದ್ದಂತೆ
ನಿರೀ-ಕ್ಷಿ-ಸಿ-ದ್ದಂ-ತೆಯೇ
ನಿರೀ-ಕ್ಷಿ-ಸಿ-ದ್ದಕ್ಕೆ
ನಿರೀ-ಕ್ಷಿ-ಸಿ-ದ್ದರು
ನಿರೀ-ಕ್ಷಿ-ಸಿ-ದ್ದಳು
ನಿರೀ-ಕ್ಷಿ-ಸಿ-ದ್ದ-ಷ್ಟೇನೂ
ನಿರೀ-ಕ್ಷಿ-ಸಿ-ದ್ದಿ-ರ-ಬೇಕು
ನಿರೀ-ಕ್ಷಿ-ಸಿ-ರ-ಬೇಕು
ನಿರೀ-ಕ್ಷಿ-ಸಿ-ರ-ಲಿಲ್ಲ
ನಿರೀ-ಕ್ಷಿಸು
ನಿರೀ-ಕ್ಷಿ-ಸುತ್ತ
ನಿರೀ-ಕ್ಷಿ-ಸು-ತ್ತಿದೆ
ನಿರೀ-ಕ್ಷಿ-ಸು-ತ್ತಿದ್ದ
ನಿರೀ-ಕ್ಷಿ-ಸು-ತ್ತಿ-ದ್ದರು
ನಿರೀ-ಕ್ಷಿ-ಸು-ತ್ತಿ-ದ್ದೇವೆ
ನಿರೀ-ಕ್ಷಿ-ಸು-ತ್ತೇನೆ
ನಿರೀ-ಕ್ಷಿ-ಸು-ವಂ-ತಿಲ್ಲ
ನಿರೀ-ಕ್ಷಿ-ಸು-ವು-ದಾಗಿ
ನಿರೀ-ಕ್ಷಿ-ಸು-ವುದು
ನಿರೀ-ಕ್ಷಿ-ಸು-ವು-ದೇ-ನೆಂ-ದರೆ
ನಿರೀಕ್ಷೆ
ನಿರೀ-ಕ್ಷೆ-ಗ-ಳಿಗೂ
ನಿರೀ-ಕ್ಷೆಗೂ
ನಿರೀ-ಕ್ಷೆಗೆ
ನಿರೀ-ಕ್ಷೆ-ಯನ್ನೂ
ನಿರೀ-ಕ್ಷೆ-ಯೇ-ನಿಲ್ಲ
ನಿರು-ತ್ಸಾಹ
ನಿರು-ದ್ಯೋಗಿ
ನಿರೂ-ಪಣಾ
ನಿರೂ-ಪಿ-ಸ-ದಿ-ರು-ವು-ದಿಲ್ಲ
ನಿರೂ-ಪಿ-ಸಿದ
ನಿರೂ-ಪಿ-ಸಿ-ರುವ
ನಿರೂ-ಪಿ-ಸು-ತ್ತಾ-ರೆಯೋ
ನಿರೂ-ಪಿ-ಸುವ
ನಿರೂ-ಪಿ-ಸು-ವಂ-ತಹ
ನಿರ್ಗ-ತಿಕ
ನಿರ್ಗ-ತಿ-ಕರ
ನಿರ್ಗ-ತಿ-ಕ-ರಂತೆ
ನಿರ್ಗ-ಮ-ನದ
ನಿರ್ಗ-ಮ-ನ-ದಿಂದ
ನಿರ್ಗ-ಮಿ-ಸಿದ
ನಿರ್ಗ-ಮಿ-ಸಿ-ಬಿ-ಟ್ಟರೆ
ನಿರ್ಗ-ಮಿ-ಸು-ತ್ತಿ-ರು-ವುದು
ನಿರ್ಗ-ಮಿ-ಸುವ
ನಿರ್ಜನ
ನಿರ್ಜ-ನ-ವಾದ
ನಿರ್ಜೀ-ವ-ವಾಗಿ
ನಿರ್ಣಯ
ನಿರ್ಣ-ಯ-ಗಳ
ನಿರ್ಣ-ಯ-ಗಳನ್ನು
ನಿರ್ಣ-ಯ-ಗಳು
ನಿರ್ಣ-ಯ-ಗ-ಳೊಂ-ದಿಗೆ
ನಿರ್ಣ-ಯದ
ನಿರ್ಣ-ಯ-ವನ್ನು
ನಿರ್ದಯ
ನಿರ್ದಯಿ
ನಿರ್ದಾ-ಕ್ಷಿಣ್ಯ
ನಿರ್ದಿಷ್ಟ
ನಿರ್ದಿ-ಷ್ಟ-ವಾಗಿ
ನಿರ್ದೇ-ಶ-ಕ-ನನ್ನು
ನಿರ್ದೇ-ಶ-ಕರ
ನಿರ್ದೇ-ಶ-ಕ-ರ-ಲ್ಲೊ-ಬ್ಬ-ನಿಗೆ
ನಿರ್ಧ-ರಿಸ
ನಿರ್ಧ-ರಿ-ಸ-ಲಾ-ಯಿತು
ನಿರ್ಧ-ರಿ-ಸಲಿ
ನಿರ್ಧ-ರಿಸಿ
ನಿರ್ಧ-ರಿ-ಸಿದ
ನಿರ್ಧ-ರಿ-ಸಿ-ದರು
ನಿರ್ಧ-ರಿ-ಸಿದ್ದ
ನಿರ್ಧ-ರಿ-ಸಿ-ದ್ದರು
ನಿರ್ಧ-ರಿ-ಸಿ-ಬಿಟ್ಟೆ
ನಿರ್ಧ-ರಿ-ಸಿ-ಯೇ-ಬಿ-ಟ್ಟರು
ನಿರ್ಧ-ರಿ-ಸು-ತ್ತಾರೆ
ನಿರ್ಧ-ರಿ-ಸು-ತ್ತಿ-ದ್ದ-ರೇನೋ
ನಿರ್ಧಾರ
ನಿರ್ಧಾ-ರಕ್ಕೂ
ನಿರ್ಧಾ-ರಕ್ಕೆ
ನಿರ್ಧಾ-ರ-ಗಳನ್ನು
ನಿರ್ಧಾ-ರ-ಪೂ-ರ್ವ-ಕ-ವಾಗಿ
ನಿರ್ಧಾ-ರ-ವನ್ನು
ನಿರ್ಧಾ-ರ-ವಾ-ಯಿತು
ನಿರ್ಧಾ-ರ-ವೊಂ-ದನ್ನು
ನಿರ್ಧಾ-ರಿತ
ನಿರ್ಧಾ-ರಿ-ತ-ವಾ-ಗಿದೆ
ನಿರ್ನಾಮ
ನಿರ್ನಾ-ಮ-ಗೊ-ಳಿ-ಸಲು
ನಿರ್ನಾ-ಮ-ಗೊ-ಳಿ-ಸು-ವುದು
ನಿರ್ನಾ-ಮ-ವಾ-ಗು-ತ್ತಿ-ದೆ-ಯೆಂದು
ನಿರ್ಬಂ-ಧ-ಪ-ಡಿ-ಸುವ
ನಿರ್ಭ-ಯತೆ
ನಿರ್ಭ-ಯ-ರಾಗಿ
ನಿರ್ಭಿ-ಡೆ-ಯಿಂದ
ನಿರ್ಭೀತ
ನಿರ್ಭೀ-ತ-ರಾಗಿ
ನಿರ್ಭೀ-ತಿ-ಯಿಂದ
ನಿರ್ಮಲ
ನಿರ್ಮಾಣ
ನಿರ್ಮಾ-ಣ-ಕಾ-ರಿಯೇ
ನಿರ್ಮಾ-ಣ-ಕಾರ್ಯ
ನಿರ್ಮಾ-ಣ-ಕ್ಕಿಂ-ತಲೂ
ನಿರ್ಮಾ-ಣಕ್ಕೆ
ನಿರ್ಮಾ-ಣ-ಗೊ-ಳ್ಳ-ಬೇ-ಕಾ-ದರೆ
ನಿರ್ಮಾ-ಣದ
ನಿರ್ಮಾ-ಣ-ದೊಂ-ದಿಗೆ
ನಿರ್ಮಾ-ಣ-ಮಾ-ಡಿ-ಕೊಂಡು
ನಿರ್ಮಾ-ಣ-ವಾ-ಗ-ಬೇ-ಕಾ-ದರೆ
ನಿರ್ಮಾ-ಣ-ವಾ-ಗಿತ್ತು
ನಿರ್ಮಾ-ಣ-ವಾ-ಗು-ವುದು
ನಿರ್ಮಾ-ಣ-ವಾ-ಯಿತು
ನಿರ್ಮಾ-ಣವೇ
ನಿರ್ಮಿ-ತ-ವಾದ
ನಿರ್ಮಿ-ತಿ-ಗಳನ್ನು
ನಿರ್ಮಿಸ
ನಿರ್ಮಿ-ಸ-ಬೇ-ಕೆಂದು
ನಿರ್ಮಿ-ಸ-ಬೇ-ಕೆಂ-ಬುದು
ನಿರ್ಮಿ-ಸ-ಲಾ-ಯಿತು
ನಿರ್ಮಿ-ಸಲು
ನಿರ್ಮಿ-ಸ-ಲೇ-ಬೇ-ಕೆಂದು
ನಿರ್ಮಿ-ಸ-ಲ್ಪ-ಟ್ಟಿದ್ದ
ನಿರ್ಮಿಸಿ
ನಿರ್ಮಿ-ಸಿ-ಕೊಂಡಿ
ನಿರ್ಮಿ-ಸಿ-ಕೊಂ-ಡಿ-ದ್ದಾರೆ
ನಿರ್ಮಿ-ಸಿ-ಕೊಂಡು
ನಿರ್ಮಿ-ಸಿತು
ನಿರ್ಮಿ-ಸಿದ
ನಿರ್ಮಿ-ಸಿ-ದಂತೆ
ನಿರ್ಮಿ-ಸಿ-ದ-ವರು
ನಿರ್ಮಿ-ಸಿದ್ದು
ನಿರ್ಮಿ-ಸಿಲ್ಲ
ನಿರ್ಮಿ-ಸು-ವು-ದರ
ನಿರ್ಮಿ-ಸು-ವುದು
ನಿರ್ಮಿ-ಸು-ವು-ದೆಂ-ದರೆ
ನಿರ್ಮೂಲ
ನಿರ್ಮೂ-ಲನ
ನಿರ್ಮೂ-ಲ-ನ-ವನ್ನು
ನಿರ್ಲ-ಕ್ಷಿಸಿ
ನಿರ್ಲ-ಕ್ಷಿಸು
ನಿರ್ಲ-ಕ್ಷ್ಯ-ದಿಂದ
ನಿರ್ಲ-ಕ್ಷ್ಯ-ದಿಂ-ದಿದ್ದ
ನಿರ್ಲಿಪ್ತ
ನಿರ್ಲಿ-ಪ್ತ-ಗಂ-ಭೀರ
ನಿರ್ಲಿ-ಪ್ತ-ನಾ-ಗಿ-ರು-ತ್ತಾ-ನೆಯೋ
ನಿರ್ಲಿ-ಪ್ತ-ರಾ-ಗಿದ್ದು
ನಿರ್ವ-ಹಣೆ
ನಿರ್ವ-ಹಿ-ಸ-ಬಲ್ಲ
ನಿರ್ವ-ಹಿ-ಸ-ಬೇಕು
ನಿರ್ವ-ಹಿ-ಸಲು
ನಿರ್ವ-ಹಿಸಿ
ನಿರ್ವ-ಹಿ-ಸು-ತ್ತಿ-ದ್ದರು
ನಿರ್ವಾ-ಣಾ-ಸ-ನ-ದಲ್ಲಿ
ನಿರ್ವಾ-ಹ-ಕರ
ನಿರ್ವಾ-ಹ-ಕ-ರಾ-ದರು
ನಿರ್ವಿ-ಕಲ್ಪ
ನಿರ್ವಿ-ಕ-ಲ್ಪ-ಸ-ಮಾಧಿ
ನಿರ್ವಿ-ಕ-ಲ್ಪ-ಸ-ಮಾ-ಧಿಯ
ನಿರ್ವಿ-ಕಾರ
ನಿರ್ವೀಯ
ನಿರ್ವೀ-ರ್ಯ-ವಾ-ಗಿದ್ದ
ನಿಲ-ಯ-ದಲ್ಲಿ
ನಿಲವು
ನಿಲುಂ-ಗಿ-ಯನ್ನೂ
ನಿಲು-ಕದ್ದು
ನಿಲು-ಕ-ಲಾ-ರ-ದಷ್ಟು
ನಿಲು-ಗಡೆ
ನಿಲು-ವಂಗಿ
ನಿಲು-ವಂ-ಗಿ-ಯನ್ನು
ನಿಲು-ವನ್ನು
ನಿಲು-ವಿನ
ನಿಲು-ವಿ-ನಲ್ಲಿ
ನಿಲು-ವಿ-ನಿಂದ
ನಿಲುವು
ನಿಲು-ವು-ತೇ-ಜ-ಸ್ಸು-ಗಳಿಂದ
ನಿಲುವೇ
ನಿಲ್ದಾ-ಣಕ್ಕೆ
ನಿಲ್ದಾ-ಣದ
ನಿಲ್ದಾ-ಣ-ದಲ್ಲಿ
ನಿಲ್ದಾ-ಣ-ದಲ್ಲೂ
ನಿಲ್ದಾ-ಣ-ದ-ವ-ರೆಗೂ
ನಿಲ್ದಾ-ಣ-ವನ್ನು
ನಿಲ್ಲ-ದಿ-ದ್ದುದೇ
ನಿಲ್ಲ-ದಿ-ರ-ಲಿ-ಎಂದು
ನಿಲ್ಲ-ದಿರು
ನಿಲ್ಲದೆ
ನಿಲ್ಲ-ಬಲ್ಲ
ನಿಲ್ಲ-ಬ-ಲ್ಲದು
ನಿಲ್ಲ-ಬಲ್ಲು
ನಿಲ್ಲ-ಬ-ಲ್ಲೆ-ವೆಂ-ಬು-ದನ್ನು
ನಿಲ್ಲ-ಬ-ಹು-ದಾ-ಗಿತ್ತು
ನಿಲ್ಲ-ಬೇ-ಕಾ-ದರೆ
ನಿಲ್ಲ-ಲಾ-ರದು
ನಿಲ್ಲ-ಲಾ-ರರು
ನಿಲ್ಲಲಿ
ನಿಲ್ಲ-ಲಿಲ್ಲ
ನಿಲ್ಲಲು
ನಿಲ್ಲಲೂ
ನಿಲ್ಲಲೇ
ನಿಲ್ಲಿ
ನಿಲ್ಲಿ-ಸ-ದಿ-ದ್ದು-ದ-ರಿಂದ
ನಿಲ್ಲಿ-ಸ-ಬಾ-ರ-ದೆಂದು
ನಿಲ್ಲಿ-ಸ-ಬೇ-ಕಾ-ಯಿತು
ನಿಲ್ಲಿ-ಸ-ಬೇ-ಕೆಂದು
ನಿಲ್ಲಿ-ಸ-ಬೇಡಿ
ನಿಲ್ಲಿ-ಸ-ಲಾಗಿದೆ
ನಿಲ್ಲಿ-ಸಲು
ನಿಲ್ಲಿ-ಸಲೇ
ನಿಲ್ಲಿಸಿ
ನಿಲ್ಲಿ-ಸಿ-ಕೊಂ-ಡಿ-ದ್ದರು
ನಿಲ್ಲಿ-ಸಿ-ದಾಗ
ನಿಲ್ಲಿ-ಸು-ತ್ತಿತ್ತು
ನಿಲ್ಲಿ-ಸು-ವಂತೆ
ನಿಲ್ಲು
ನಿಲ್ಲು-ತ್ತದೆ
ನಿಲ್ಲು-ತ್ತವೆ
ನಿಲ್ಲು-ತ್ತ-ವೆ-ಯೆಂದು
ನಿಲ್ಲು-ತ್ತಿತ್ತು
ನಿಲ್ಲು-ತ್ತಿ-ದ್ದರು
ನಿಲ್ಲು-ತ್ತಿ-ದ್ದುದು
ನಿಲ್ಲು-ತ್ತಿದ್ದೆ
ನಿಲ್ಲುವ
ನಿಲ್ಲು-ವಂ-ತಾ-ಗ-ಬಾ-ರ-ದೆಂದು
ನಿಲ್ಲು-ವಂತೆ
ನಿಲ್ಲು-ವನು
ನಿಲ್ಲು-ವಲ್ಲಿ
ನಿಲ್ಲು-ವುದನ್ನು
ನಿಲ್ಲು-ವು-ದಿಲ್ಲ
ನಿಲ್ಲು-ವುದು
ನಿಲ್ಲು-ವು-ದು-ಹೀ-ಗೆಯೇ
ನಿಲ್ಲೋ
ನಿವಾ-ರ-ಣೆ-ಗಾಗಿ
ನಿವಾ-ರಿ-ಸ-ಬ-ಲ್ಲುವು
ನಿವಾ-ರಿ-ಸ-ಬೇಕು
ನಿವಾ-ರಿ-ಸಲು
ನಿವಾ-ರಿ-ಸಿ-ಕೊ-ಳ್ಳ-ಬೇ-ಕಾ-ಗು-ತ್ತದೆ
ನಿವಾ-ರಿ-ಸಿದ
ನಿವಾ-ರಿ-ಸು-ತ್ತೇನೆ
ನಿವಾ-ಸಕ್ಕೆ
ನಿವಾ-ಸಿ-ಗಳ
ನಿವಾ-ಸಿ-ಗಳನ್ನು
ನಿವಾ-ಸಿ-ಗ-ಳಾದ
ನಿವಾ-ಸಿ-ಗ-ಳೆಲ್ಲ
ನಿವೃತ್ತ
ನಿವೃ-ತ್ತ-ನಾಗಿ
ನಿವೃ-ತ್ತ-ನಾ-ಗು-ವ-ವ-ನಿ-ದ್ದೇನೆ
ನಿವೃ-ತ್ತ-ನಾದ
ನಿವೃ-ತ್ತ-ರಾ-ಗು-ವು-ದಾ-ಗಲಿ
ನಿವೃ-ತ್ತಿಯೆ
ನಿವೇ-ದನೆ
ನಿವೇ-ದಿತಾ
ನಿವೇ-ದಿತೆ
ನಿವೇ-ದಿ-ತೆಯೇ
ನಿವೇ-ಶ-ನಕ್ಕೆ
ನಿವೇ-ಶ-ನ-ವನ್ನು
ನಿಶಾ-ಚರ
ನಿಶ್ಚಯ
ನಿಶ್ಚ-ಯದ
ನಿಶ್ಚ-ಯ-ವಾಗಿ
ನಿಶ್ಚ-ಯ-ವಾ-ಗಿತ್ತು
ನಿಶ್ಚ-ಯ-ವಾ-ಗಿಯೂ
ನಿಶ್ಚ-ಯ-ವಾ-ಗಿ-ರು-ವಾಗ
ನಿಶ್ಚ-ಯಿಸಿ
ನಿಶ್ಚ-ಯಿ-ಸಿದ
ನಿಶ್ಚ-ಯಿ-ಸಿ-ದರು
ನಿಶ್ಚ-ಯಿ-ಸಿದ್ದ
ನಿಶ್ಚ-ಯಿ-ಸಿ-ದ್ದರು
ನಿಶ್ಚ-ಯಿ-ಸಿ-ರು-ವು-ದಾ-ಗಿಯೂ
ನಿಶ್ಚ-ಯಿ-ಸು-ತ್ತಿ-ದ್ದಾರೆ
ನಿಶ್ಚಲ
ನಿಶ್ಚ-ಲ-ನಾ-ಗಿರು
ನಿಶ್ಚ-ಲ-ವಾಗಿ
ನಿಶ್ಚಿಂತ
ನಿಶ್ಚಿಂ-ತ-ರಾಗಿ
ನಿಶ್ಚಿಂ-ತೆ-ಯಾ-ಯಿತು
ನಿಶ್ಚಿತ
ನಿಶ್ಚಿ-ತತೆ
ನಿಶ್ಚಿ-ತ-ವಾಗಿ
ನಿಶ್ಚಿ-ತ-ವಾ-ಗಿತ್ತು
ನಿಶ್ಚಿ-ತ-ವಾ-ಗಿಯೂ
ನಿಶ್ಶಂ-ಕೆ-ಯಿಂದ
ನಿಶ್ಶ-ಕ್ತ-ರಾಗಿ
ನಿಶ್ಶ-ಬ್ದ-ತೆಯು
ನಿಶ್ಶ-ಬ್ದ-ವಾಗಿ
ನಿಶ್ಶ-ಬ್ದ-ವಾ-ಯಿತು
ನಿಷ-ತ್ತು-ಗಳೇ
ನಿಷಿ-ದ್ಧ-ವಾ-ಗಿತ್ತು
ನಿಷೇಧ
ನಿಷೇ-ಧಿ-ಸಿ-ದ್ದರು
ನಿಷೇ-ಧಿ-ಸುವ
ನಿಷ್ಕ-ಳಂಕ
ನಿಷ್ಕಾಮ
ನಿಷ್ಕಾ-ಮ-ಕರ್ಮ
ನಿಷ್ಕಾ-ಮ-ಕ-ರ್ಮ-ವನ್ನೂ
ನಿಷ್ಕೃ-ಷ್ಟ-ವಾಗಿ
ನಿಷ್ಕೃ-ಷ್ಟ-ವಾದ
ನಿಷ್ಠ
ನಿಷ್ಠ-ರಾ-ಗಿ-ರು-ವಂ-ತೆ-ಎಂ-ದರೆ
ನಿಷ್ಠಾ-ಯು-ತ-ವಾಗಿ
ನಿಷ್ಠಾ-ವಂತ
ನಿಷ್ಠಾ-ವಂ-ತ-ನಾ-ಗಿ-ದ್ದರೂ
ನಿಷ್ಠಾ-ವಂ-ತ-ನಾ-ಗಿರು
ನಿಷ್ಠಾ-ವಂ-ತ-ರಾಗಿ
ನಿಷ್ಠಾ-ವಂ-ತರೂ
ನಿಷ್ಠುರ
ನಿಷ್ಠು-ರದ
ನಿಷ್ಠು-ರ-ನಾ-ಗಿದ್ದೆ
ನಿಷ್ಠು-ರ-ವಾಗಿ
ನಿಷ್ಠೆ-ಯಿಂದ
ನಿಷ್ಠೆ-ಯಿಂ-ದಿ-ರು-ವ-ವ-ರೆಗೆ
ನಿಷ್ಠೆಯೂ
ನಿಷ್ಣಾತ
ನಿಷ್ಣಾ-ತ-ರಾದ
ನಿಷ್ಣಾ-ತರು
ನಿಷ್ಪಂ-ದ-ಗೊ-ಳಿಸಿ
ನಿಷ್ಪ-ಕ್ಷ-ಪಾತ
ನಿಷ್ಪ್ರ-ಯೋ-ಜಕ
ನಿಷ್ಪ್ರ-ಯೋ-ಜ-ಕ-ವಾ-ಗಿದೆ
ನಿಷ್ಫ-ಲ-ಗೊ-ಳಿ-ಸುವ
ನಿಷ್ಫ-ಲ-ಗೊ-ಳಿ-ಸು-ವಂ-ತಾ-ಗ-ಬೇಕು
ನಿಷ್ಫ-ಲ-ವಾಗಿ
ನಿಸ-ರ್ಗದ
ನಿಸ್ತೇ-ಜ-ನಾ-ಗ-ಲೇ-ಬೇ-ಕಾ-ಗಿತ್ತು
ನಿಸ್ತೇ-ಜ-ರಾದ
ನಿಸ್ವಾರ್ಥ
ನಿಸ್ವಾ-ರ್ಥ-ತೆಯ
ನಿಸ್ವಾ-ರ್ಥ-ಸೇವೆ
ನಿಸ್ಸಂ-ಕೋ-ಚ-ವಾಗಿ
ನಿಸ್ಸಂ-ದೇ-ಹ-ವಾಗಿ
ನಿಸ್ಸಂ-ಶ-ಯ-ವಾಗಿ
ನಿಸ್ಸಂ-ಶ-ಯ-ವಾ-ಗಿಯೂ
ನಿಸ್ಸತ್ವ-ಗೊ-ಳಿ-ಸು-ವಂ-ಥದು
ನೀ
ನೀಗ್ರೋ
ನೀಗ್ರೋ-ಗಳ
ನೀಗ್ರೋ-ಗ-ಳ-ಲ್ಲೊಬ್ಬ
ನೀಗ್ರೋ-ಗ-ಳಿ-ಗಿಂತ
ನೀಗ್ರೋ-ಗ-ಳಿ-ಗೆ-ಧ-ರ್ಮ-ಬೋ-ಧನೆ
ನೀಗ್ರೋ-ಗಳು
ನೀಗ್ರೋ-ಗ-ಳೆಂ-ದರೆ
ನೀಚ
ನೀಚ-ತ-ನ-ವೆಂ-ಬುದು
ನೀಚ-ನನ್ನೂ
ನೀಚ-ಬು-ದ್ಧಿಯ
ನೀಡ
ನೀಡ-ಬಲ್ಲ
ನೀಡ-ಬ-ಲ್ಲ-ವ-ರಾ-ಗಿ-ದ್ದರು
ನೀಡ-ಬ-ಲ್ಲಿರಿ
ನೀಡ-ಬ-ಲ್ಲುದು
ನೀಡ-ಬ-ಹು-ದಾದ
ನೀಡ-ಬ-ಹುದು
ನೀಡ-ಬೇಕಾ
ನೀಡ-ಬೇ-ಕಾ-ಗಿದೆ
ನೀಡ-ಬೇ-ಕಾ-ಗಿದ್ದು
ನೀಡ-ಬೇ-ಕಾದ
ನೀಡ-ಬೇ-ಕೆಂದು
ನೀಡ-ಬೇ-ಕೆಂದೂ
ನೀಡ-ಬೇ-ಕೆಂಬ
ನೀಡ-ಬೇ-ಕೆಂ-ಬುದು
ನೀಡ-ಲಾ-ಗಿತ್ತು
ನೀಡ-ಲಾಗಿದೆ
ನೀಡ-ಲಾ-ಗಿ-ರುವ
ನೀಡ-ಲಾ-ಗು-ತ್ತಿತ್ತು
ನೀಡ-ಲಾದ
ನೀಡ-ಲಾ-ಯಿತು
ನೀಡ-ಲಿ-ರುವ
ನೀಡಲು
ನೀಡಲೂ
ನೀಡಿ
ನೀಡಿ-ತ-ಲ್ಲದೆ
ನೀಡಿತು
ನೀಡಿ-ತೆ-ನ್ನ-ಬ-ಹುದು
ನೀಡಿತ್ತು
ನೀಡಿದ
ನೀಡಿ-ದನೋ
ನೀಡಿ-ದ-ರ-ಲ್ಲದೆ
ನೀಡಿ-ದರು
ನೀಡಿ-ದ-ರು-ಬ್ರ-ಹ್ಮ-ಚ-ರ್ಯದ
ನೀಡಿ-ದ-ರು-ಹೆ-ಸರು
ನೀಡಿ-ದರೂ
ನೀಡಿ-ದ-ಳು-ಸ್ವಾ-ಮೀಜಿ
ನೀಡಿ-ದ-ವ-ರಲ್ಲಿ
ನೀಡಿ-ದ-ವರು
ನೀಡಿ-ದಾಗ
ನೀಡಿ-ದುದ
ನೀಡಿ-ದು-ದ-ಕ್ಕಾಗಿ
ನೀಡಿ-ದುದು
ನೀಡಿ-ದುವು
ನೀಡಿದೆ
ನೀಡಿ-ದೆ-ಯೆ-ನ್ನ-ಬೇಕು
ನೀಡಿದ್ದ
ನೀಡಿ-ದ್ದಕ್ಕೆ
ನೀಡಿ-ದ್ದರು
ನೀಡಿ-ದ್ದ-ರೆಂ-ಬು-ದನ್ನು
ನೀಡಿ-ದ್ದಳು
ನೀಡಿ-ದ್ದಾಗ
ನೀಡಿ-ದ್ದಾನೆ
ನೀಡಿ-ದ್ದಾರೆ
ನೀಡಿ-ದ್ದಾಳೆ
ನೀಡಿದ್ದು
ನೀಡಿದ್ದೇ
ನೀಡಿ-ದ್ದೇವೆ
ನೀಡಿ-ರ-ದಿ-ದ್ದರೆ
ನೀಡಿ-ರುವ
ನೀಡು
ನೀಡುತ್ತ
ನೀಡು-ತ್ತದೆ
ನೀಡು-ತ್ತಾರೆ
ನೀಡು-ತ್ತಾ-ರೆಂದು
ನೀಡು-ತ್ತಾ-ರೆ-ಲೌ-ಕಿಕ
ನೀಡು-ತ್ತಾಳೆ
ನೀಡು-ತ್ತಿದ್ದ
ನೀಡು-ತ್ತಿ-ದ್ದ-ರಾ-ದರೂ
ನೀಡು-ತ್ತಿ-ದ್ದರು
ನೀಡು-ತ್ತಿ-ದ್ದಳು
ನೀಡು-ತ್ತಿ-ದ್ದಾರೆ
ನೀಡು-ತ್ತಿ-ದ್ದುದು
ನೀಡು-ತ್ತಿ-ದ್ದು-ದೆಂ-ದರೆ
ನೀಡು-ತ್ತಿ-ರು-ವಾಗ
ನೀಡು-ತ್ತಿ-ರು-ವುದನ್ನು
ನೀಡು-ತ್ತೇನೆ
ನೀಡುವ
ನೀಡು-ವಂ-ತಹ
ನೀಡು-ವಂ-ತಾ-ಗ-ಬೇಕು
ನೀಡು-ವಂತೆ
ನೀಡು-ವ-ವ-ನಾ-ಗು-ತ್ತಾನೆ
ನೀಡು-ವ-ವನು
ನೀಡುವು
ನೀಡು-ವು-ದ-ಕ್ಕಾ-ಗಿಯೇ
ನೀಡು-ವು-ದಕ್ಕೆ
ನೀಡು-ವುದನ್ನು
ನೀಡು-ವು-ದಾಗಿ
ನೀಡು-ವು-ದಾ-ದರೂ
ನೀಡು-ವು-ದಾ-ದರೆ
ನೀಡು-ವುದು
ನೀಡು-ವು-ದು-ಇದು
ನೀಡು-ವುದೇ
ನೀತಿ
ನೀತಿ-ನ-ಡ-ತೆ-ಗಳ
ನೀತಿ-ನಿ-ಯ-ಮ-ಗಳ
ನೀತಿ-ನಿ-ಯ-ಮಾ-ವ-ಳಿ-ಗಳ
ನೀತಿ-ಗೆ-ಟ್ಟ-ವರು
ನೀತಿ-ಗೆ-ಟ್ಟ-ವ್ಯಕ್ತಿ
ನೀತಿ-ನಿ-ಯ-ಮ-ಗಳನ್ನು
ನೀತಿ-ಯ-ನ್ನು-ನೀ-ಚ-ತ-ನ-ವನ್ನು
ನೀತಿ-ವಂತ
ನೀನ-ಗೇನೂ
ನೀನ-ಲ್ಲ-ದಿ-ದ್ದರೂ
ನೀನ-ಲ್ಲವೆ
ನೀನ-ವ-ರಲ್ಲಿ
ನೀನ-ವ-ರಿಗೆ
ನೀನಾಗಿ
ನೀನಾ-ಗಿಯೇ
ನೀನಾ-ದರೂ
ನೀನಾ-ದರೋ
ನೀನಿ-ದನ್ನು
ನೀನಿ-ದ-ರಲ್ಲಿ
ನೀನಿನ್ನು
ನೀನಿನ್ನೂ
ನೀನೀಗ
ನೀನು
ನೀನೂ
ನೀನೆಂ-ದಿಗೂ
ನೀನೆನ್ನ
ನೀನೆ-ಲ್ಲಿಗೆ
ನೀನೇ
ನೀನೇಕೆ
ನೀನೇನು
ನೀನೇನೂ
ನೀನೊಂದು
ನೀನೊಬ್ಬ
ನೀನೊ-ಬ್ಬನೇ
ನೀನೊ-ಬ್ಬಳೇ
ನೀನೋ
ನೀನ್ಯಾ-ರಯ್ಯ
ನೀಯ
ನೀರ
ನೀರ-ನ್ನ-ರ-ಸುತ್ತ
ನೀರನ್ನು
ನೀರನ್ನೂ
ನೀರ-ನ್ನೆ-ರ-ಚಿ-ಸಿ-ಕೊಂಡು
ನೀರವ
ನೀರ-ವ-ಪ್ರ-ಶಾಂತ
ನೀರ-ವತೆ
ನೀರ-ವ-ಧ್ಯಾ-ನ-ದಲ್ಲಿ
ನೀರಸ
ನೀರ-ಸ-ತೆ-ಯನ್ನು
ನೀರ-ಸ-ವಾ-ಗಿ-ತ್ತೆಂದು
ನೀರ-ಸ-ವಾ-ಗಿ-ದ್ದಿ-ರ-ಬ-ಹುದು
ನೀರ-ಸ-ವಾದ
ನೀರಾಗಿ
ನೀರಾ-ಗು-ತ್ತಿತ್ತು
ನೀರಿ-ಗಂತೂ
ನೀರಿ-ಗಾಗಿ
ನೀರಿಗೆ
ನೀರಿನ
ನೀರಿ-ನ-ಲ್ಲಾ-ಗಲಿ
ನೀರಿ-ನಲ್ಲಿ
ನೀರಿ-ನ-ಲ್ಲಿ-ಳಿದು
ನೀರಿ-ನಿಂದ
ನೀರಿ-ರ-ಬೇ-ಕೆಂದು
ನೀರಿ-ಲ್ಲದೆ
ನೀರಿ-ಲ್ಲವೆ
ನೀರು
ನೀರು-ಕೊಟ್ಟ
ನೀರೂ-ರಿ-ಸು-ವಂ-ತಹ
ನೀರ್ಗಲ್ಲ
ನೀರ್ಗ-ಲ್ಲಿ-ನಂತೆ
ನೀಲ-ಕಂಠ
ನೀಲ-ಕಂ-ಠ-ನೆಂಬ
ನೀಲ-ವ-ರ್ಣಕ್ಕೆ
ನೀಲಾ
ನೀಲಾ-ಕಾ-ಶ-ಇ-ವೆಲ್ಲ
ನೀಲಿ
ನೀಳ
ನೀವಂತೂ
ನೀವ-ದಕ್ಕೆ
ನೀವ-ದನ್ನು
ನೀವ-ದ-ರಿಂದ
ನೀವ-ಲ್ಲವೆ
ನೀವಲ್ಲಿ
ನೀವಿನ್ನೂ
ನೀವಿ-ಬ್ಬರು
ನೀವೀಗ
ನೀವು
ನೀವೂ
ನೀವೆಂ-ಥ-ವರು
ನೀವೆಂ-ದಿಗೂ
ನೀವೆಲ್ಲ
ನೀವೆ-ಲ್ಲಿರು
ನೀವೆಷ್ಟು
ನೀವೇ
ನೀವೇಕೆ
ನೀವೇ-ನಾ-ದರೂ
ನೀವೇನು
ನೀವೇನೂ
ನೀವೇನೋ
ನೀವೊಂದು
ನೀವೊಬ್ಬ
ನೀವೊ-ಬ್ಬರು
ನುಂಗಿ
ನುಂಗಿ-ಬಿಟ್ಟೆ
ನುಗ್ಗಲು
ನುಗ್ಗಿ
ನುಗ್ಗಿದ
ನುಗ್ಗಿದ್ದು
ನುಗ್ಗಿ-ಬ-ರು-ತ್ತಿದೆ
ನುಗ್ಗು-ತ್ತಿ-ದ್ದಾರೆ
ನುಗ್ಗು-ತ್ತಿ-ರು-ವುದನ್ನು
ನುಚ್ಚು-ನೂ-ರಾ-ಗಿ-ಬಿ-ಟ್ಟು-ವು-ಅ-ಥವಾ
ನುಚ್ಚು-ನೂ-ರಾ-ದಾಗ
ನುಚ್ಚು-ನೂರು
ನುಡಿ
ನುಡಿ-ಗಳ
ನುಡಿ-ಗಳು
ನುಡಿದ
ನುಡಿ-ದರು
ನುಡಿ-ದಿದ್ದ
ನುಡಿ-ದಿ-ದ್ದರು
ನುಡಿದು
ನುಡಿ-ಯಲು
ನುಡಿ-ಯಿಂದ
ನುಡಿ-ಯು-ವುದನ್ನು
ನುಡಿಯೂ
ನುಡಿಸಿ
ನುಡಿ-ಸಿ-ದಂತೆ
ನುಡಿ-ಸು-ತ್ತಿದ್ದ
ನುಡಿ-ಸು-ವಾಗ
ನುಡಿ-ಸು-ವುದು
ನುಣು-ಚಿ-ಕೊಂಡ
ನುಣು-ಚಿ-ಕೊ-ಳ್ಳುವ
ನುಣು-ಪಾಗಿ
ನುಣುಪು
ನುಣ್ಣಗೆ
ನುಭ-ವ-ದಲ್ಲಿ
ನುರಿತ
ನುರಿ-ತ-ವ-ರಿಂದ
ನುರಿ-ತ-ವರು
ನುಸುಳಿ
ನುಸು-ಳಿವೆ
ನೂಕಿ
ನೂಕು
ನೂತನ
ನೂತ-ನ-ವಾಗಿ
ನೂರ
ನೂರಕ್ಕೂ
ನೂರಕ್ಕೆ
ನೂರನೇ
ನೂರಾರು
ನೂರಿ-ಪ್ಪತ್ತು
ನೂರಿ-ಪ್ಪ-ತ್ತೈದು
ನೂರು
ನೂರು-ನೂರು
ನೂರು-ಸ-ಹಸ್ರ
ನೃಣಾ-ಮೇಕೋ
ನೆಂದರೆ
ನೆಂದು
ನೆಂದೂ
ನೆಚ್ಚಿ-ಕೊಂ-ಡದ್ದು
ನೆಚ್ಚಿ-ಕೊಂ-ಡ-ದ್ದು-ಇದು
ನೆಚ್ಚಿ-ಕೊಂ-ಡಿ-ದ್ದಕ್ಕೆ
ನೆಚ್ಚಿ-ಕೊಂ-ಡಿ-ದ್ದರು
ನೆಚ್ಚಿ-ಕೊಂ-ಡಿ-ದ್ದರೋ
ನೆಚ್ಚಿ-ಕೊಂಡು
ನೆಚ್ಚಿ-ಕೊ-ಳ್ಳ-ಬಹು
ನೆಚ್ಚಿ-ಕೊ-ಳ್ಳುವ
ನೆಚ್ಚಿನ
ನೆಟ್ಟಗೆ
ನೆಟ್ಟಿತು
ನೆಟ್ಟಿದ್ದ
ನೆಟ್ಟಿ-ದ್ದುವು
ನೆಟ್ಟಿ-ರು-ತ್ತಿದ್ದ
ನೆಟ್ಟು
ನೆತ್ತಿಯ
ನೆನ-ಪಾಗಿ
ನೆನ-ಪಾ-ಗು-ತ್ತಿತ್ತು
ನೆನಪಿ
ನೆನ-ಪಿ-ಗಾಗಿ
ನೆನ-ಪಿಗೆ
ನೆನ-ಪಿ-ಟ್ಟಿ-ರಲಿ
ನೆನ-ಪಿಟ್ಟು
ನೆನ-ಪಿ-ಟ್ಟು-ಕೊ-ನಾನು
ನೆನ-ಪಿ-ಟ್ಟು-ಕೊ-ಳ್ಳ-ಬೇಕು
ನೆನ-ಪಿ-ಟ್ಟು-ಕೊ-ಳ್ಳಲು
ನೆನ-ಪಿ-ಟ್ಟು-ಕೊ-ಳ್ಳಿ-ಇ-ದೆಲ್ಲ
ನೆನ-ಪಿ-ಟ್ಟು-ಕೊ-ಶ-ಕ್ತಿಯ
ನೆನ-ಪಿ-ಡ-ಬೇಕು
ನೆನ-ಪಿಡಿ
ನೆನ-ಪಿಡು
ನೆನ-ಪಿತ್ತು
ನೆನ-ಪಿದೆ
ನೆನ-ಪಿ-ದೆ-ತಾನೆ
ನೆನ-ಪಿ-ನಲ್ಲಿ
ನೆನ-ಪಿ-ನಿಂದ
ನೆನ-ಪಿ-ರಲಿ
ನೆನ-ಪಿ-ರು-ತ್ತದೆ
ನೆನ-ಪಿ-ರು-ವು-ದಿಲ್ಲ
ನೆನ-ಪಿಲ್ಲ
ನೆನ-ಪಿ-ಸಿಕೊ
ನೆನ-ಪಿ-ಸಿ-ಕೊಂ-ಡರು
ನೆನ-ಪಿ-ಸಿ-ಕೊಂ-ಡಾಗ
ನೆನ-ಪಿ-ಸಿ-ಕೊಂಡು
ನೆನ-ಪಿ-ಸಿ-ಕೊ-ಟ್ಟವು
ನೆನ-ಪಿ-ಸಿ-ಕೊಟ್ಟು
ನೆನ-ಪಿ-ಸಿ-ಕೊ-ಡಲು
ನೆನ-ಪಿ-ಸಿ-ಕೊ-ಡುವ
ನೆನ-ಪಿ-ಸಿ-ಕೊ-ಳ್ಳ-ಬೇಕು
ನೆನ-ಪಿ-ಸಿ-ಕೊಳ್ಳಿ
ನೆನಪು
ನೆನ-ಪು-ಗಳು
ನೆನ-ಹು-ಗಳನ್ನು
ನೆನೆ-ಸಿ-ಕೊಂ-ಡಾಗ
ನೆನೆ-ಸಿ-ಕೊಂಡು
ನೆಪ-ಗಳನ್ನು
ನೆಪ-ದಲ್ಲಿ
ನೆಪ-ವೊಡ್ಡಿ
ನೆಪೋಲಿ
ನೆಪೋ-ಲಿ-ಯ-ನ್ನನ್ನು
ನೆಮ್ಮದಿ
ನೆಮ್ಮ-ದಿ-ಸ-ಮಾ-ಧಾನ
ನೆಮ್ಮ-ದಿಯ
ನೆಮ್ಮ-ದಿ-ಯಾ-ಗಿದ್ದೆ
ನೆಮ್ಮ-ದಿ-ಯಿಂದ
ನೆಮ್ಮ-ದಿ-ಯಿಂ-ದಿ-ರಲು
ನೆಮ್ಮ-ದಿ-ಯೆ-ನಿ-ಸಿತು
ನೆಯ
ನೆರ
ನೆರ-ದಿದ್ದ
ನೆರ-ಳಲ್ಲಿ
ನೆರ-ಳಿಗೆ
ನೆರ-ಳಿ-ನಂತೆ
ನೆರ-ಳಿ-ನಲ್ಲೋ
ನೆರಳೂ
ನೆರ-ವನ್ನು
ನೆರ-ವನ್ನೂ
ನೆರವಾ
ನೆರ-ವಾಗ
ನೆರ-ವಾ-ಗ-ದಿ-ರು-ತ್ತಿ-ರ-ಲಿಲ್ಲ
ನೆರ-ವಾ-ಗ-ಬ-ಲ್ಲಿರಿ
ನೆರ-ವಾ-ಗ-ಬ-ಲ್ಲುದು
ನೆರ-ವಾ-ಗ-ಬಲ್ಲೆ
ನೆರ-ವಾ-ಗ-ಬ-ಹುದು
ನೆರ-ವಾ-ಗಲಿ
ನೆರ-ವಾ-ಗಲು
ನೆರ-ವಾ-ಗ-ಲೆಂದು
ನೆರ-ವಾ-ಗ-ಲೆಂದೇ
ನೆರ-ವಾಗಿ
ನೆರ-ವಾ-ಗಿದ್ದ
ನೆರ-ವಾ-ಗಿ-ದ್ದರು
ನೆರ-ವಾ-ಗಿ-ದ್ದಳು
ನೆರ-ವಾಗು
ನೆರ-ವಾ-ಗು-ತ್ತಾರೆ
ನೆರ-ವಾ-ಗು-ತ್ತಿದ್ದ
ನೆರ-ವಾ-ಗು-ತ್ತಿ-ದ್ದರೋ
ನೆರ-ವಾ-ಗು-ತ್ತಿ-ದ್ದಳು
ನೆರ-ವಾ-ಗು-ವ-ನೆಂದು
ನೆರ-ವಾ-ಗು-ವು-ದ-ಕ್ಕಾಗಿ
ನೆರ-ವಾ-ಗು-ವು-ದ-ರಲ್ಲಿ
ನೆರ-ವಾ-ಗು-ವುದು
ನೆರ-ವಾದ
ನೆರ-ವಾ-ದರು
ನೆರ-ವಾ-ದರೆ
ನೆರ-ವಾ-ದಳು
ನೆರ-ವಿ-ಗಾ-ಗಿಯೇ
ನೆರ-ವಿಗೆ
ನೆರ-ವಿ-ಗೆಂದು
ನೆರ-ವಿ-ಗೊ-ದ-ಗಿದ
ನೆರ-ವಿ-ನಿಂದ
ನೆರ-ವಿ-ನಿಂ-ದಲೇ
ನೆರ-ವಿ-ಲ್ಲದೆ
ನೆರವು
ನೆರವೂ
ನೆರ-ವೇ-ರ-ದಿ-ದ್ದರೂ
ನೆರ-ವೇ-ರಿತು
ನೆರ-ವೇ-ರಿ-ಸಿ-ದರು
ನೆರ-ವೇ-ರಿ-ಸಿ-ದು-ದ-ಕ್ಕಾಗಿ
ನೆರ-ವೇ-ರಿ-ಸಿ-ದ್ದೀರಿ
ನೆರ-ವೇ-ರಿ-ಸು-ತ್ತಾ-ನೆಯೋ
ನೆರ-ವೇ-ರು-ವಂ-ತಹ
ನೆರ-ವೇ-ರು-ವಂ-ತಾ-ಗಿತ್ತು
ನೆರ-ವೇ-ರು-ವಂ-ತಾ-ಯಿ-ತಲ್ಲ
ನೆರ-ವೇ-ರು-ವಂತೆ
ನೆರೆ
ನೆರೆದಿ
ನೆರೆ-ದಿದ್ದ
ನೆರೆ-ದಿ-ದ್ದರು
ನೆರೆ-ದಿ-ದ್ದ-ವರ
ನೆರೆ-ದಿ-ದ್ದ-ವ-ರಿಗೆ
ನೆರೆ-ದಿ-ದ್ದ-ವರೆ-ಲ್ಲ-ರಿ-ಗಿಂ-ತಲೂ
ನೆರೆ-ದಿ-ದ್ದೇವೆ
ನೆರೆ-ಯ-ವರ
ನೆರೆ-ಯ-ವ-ರನ್ನು
ನೆರೆ-ಯು-ತ್ತಿದ್ದ
ನೆರೆ-ಯುವ
ನೆರೆ-ಯು-ವಂತಾ
ನೆರೆ-ಯು-ವಂ-ತಾ-ಯಿತು
ನೆರೆ-ರಾಜ್ಯ
ನೆರೆ-ರಾ-ಜ್ಯ-ಗಳ
ನೆಲ
ನೆಲಕು
ನೆಲಕ್ಕೆ
ನೆಲದ
ನೆಲ-ದ-ಮೇಲೆ
ನೆಲ-ದಲ್ಲಿ
ನೆಲ-ದ-ವ-ರೆಗೂ
ನೆಲ-ದಷ್ಟೇ
ನೆಲದಿ
ನೆಲ-ದಿಂದ
ನೆಲ-ಮಾ-ರ್ಗ-ವಾಗಿ
ನೆಲ-ವನ್ನು
ನೆಲವೋ
ನೆಲಸ
ನೆಲ-ಸ-ಬೇ-ಕಾದ
ನೆಲ-ಸ-ಬೇ-ಕೆಂದು
ನೆಲ-ಸಮ
ನೆಲಸಿ
ನೆಲ-ಸಿತು
ನೆಲ-ಸಿತ್ತು
ನೆಲ-ಸಿದ
ನೆಲ-ಸಿದ್ದ
ನೆಲ-ಸಿ-ದ್ದಾರೆ
ನೆಲ-ಸಿ-ಬಿ-ಡ-ಬೇ-ಕೆಂದು
ನೆಲಸು
ನೆಲ-ಸುವ
ನೆಲೆ
ನೆಲೆ-ಗೇ-ರಲು
ನೆಲೆ-ಗೊಂ-ಡ-ಮೇಲೆ
ನೆಲೆ-ಗೊ-ಳಿ-ಸಲು
ನೆಲೆ-ಗೊ-ಳಿಸಿ
ನೆಲೆ-ಗೊ-ಳಿ-ಸು-ವಂ-ತಾ-ಗು-ತ್ತದೆ
ನೆಲೆ-ಗೊಳ್ಳು
ನೆಲೆ-ಗೊ-ಳ್ಳು-ವಂತೆ
ನೆಲೆ-ನಿ-ಲ್ಲಿ-ಸಲು
ನೆಲೆ-ನಿ-ಲ್ಲು-ವಂತೆ
ನೆಲೆ-ಯ-ನೆಂ-ದಿಗು
ನೆಲೆ-ಯಾದ
ನೆಲೆ-ಯೂ-ರಲು
ನೆಲೆ-ಯೂರಿ
ನೆಲೆ-ಯೂ-ರಿ-ದ್ದುವು
ನೆಲೆ-ವೀ-ಡಾಗಿ
ನೆಲ್ಕಕೆ
ನೇ
ನೇಗಿಲು
ನೇತೃತ್ವ
ನೇತೃ-ತ್ವ-ದಲ್ಲಿ
ನೇತ್ರ-ಗಳನ್ನು
ನೇತ್ರ-ಗಳಿಂದ
ನೇತ್ರ-ಗಳು
ನೇಪ-ಲ್ಸಿ-ನಲ್ಲಿ
ನೇಪ-ಲ್ಸ್
ನೇಪ-ಲ್ಸ್ನಿಂದ
ನೇಮಕ
ನೇಮಿ-ಸ-ಲಾ-ಯಿತು
ನೇಮಿ-ಸ-ಲ್ಪ-ಟ್ಟಿ-ದ್ದರು
ನೇಮಿ-ಸಿ-ಕೊಂ-ಡರು
ನೇಮಿ-ಸಿ-ಕೊ-ಳ್ಳ-ಬಾ-ರದು
ನೇಮಿ-ಸಿ-ಕೊ-ಳ್ಳ-ಲಾ-ಗಿತ್ತು
ನೇಮಿ-ಸಿ-ಕೊ-ಳ್ಳಲು
ನೇಮಿ-ಸಿದ
ನೇಮಿ-ಸಿ-ದರು
ನೇರ
ನೇರ-ವಾಗಿ
ನೇರ-ವಾ-ಗಿದೆ
ನೇರ-ವಾ-ಗಿ-ರದೆ
ನೇರ-ವಾದ
ನೇರ-ಸ್ವ-ಭಾ-ವದ
ನೇಶನ್
ನೈಚ್ಯಾ-ನು-ಸಂ-ಧಾನ
ನೈಜ
ನೈತಿಕ
ನೈತಿ-ಕ-ಜೀ-ವನ
ನೈತಿ-ಕತೆ
ನೈತಿ-ಕ-ತೆ-ಯನ್ನು
ನೈತಿ-ಕ-ತೆ-ಯಿಂದ
ನೈನ್ಟೀ-ನ್ತ್
ನೈಪುತ್ಯ
ನೊಂದ
ನೊಂದ-ವರು
ನೊಂದಿಗೆ
ನೊಂದು
ನೊಣ-ಗಳೂ
ನೊಬ್ಬ
ನೊಬ್ಬನ
ನೊರೇಂ-ದ್ರ-ನಾಥ್
ನೋಟ
ನೋಟಕ್ಕೇ
ನೋಟ-ದಲ್ಲಿ
ನೋಟವ
ನೋಟ-ವನ್ನು
ನೋಟವೇ
ನೋಟು-ಗಳ
ನೋಟ್
ನೋಡ
ನೋಡ-ತೊ-ಡ-ಗಿದ
ನೋಡ-ದಿರ
ನೋಡ-ದಿರಿ
ನೋಡದೆ
ನೋಡ-ನೋ-ಡು-ತ್ತಿ-ದ್ದಂತೆ
ನೋಡ-ಬಲ್ಲ
ನೋಡ-ಬ-ಹುದು
ನೋಡ-ಬ-ಹು-ದು-ಸ್ವಾಮಿ
ನೋಡ-ಬಾ-ರದು
ನೋಡ-ಬಾ-ರ-ದೇಕೆ
ನೋಡ-ಬೇ-ಕಾ-ಗಿದೆ
ನೋಡ-ಬೇಕು
ನೋಡ-ಬೇ-ಕೆಂಬ
ನೋಡ-ಬೇ-ಕೆಂ-ಬು-ದನ್ನು
ನೋಡ-ಲಾ-ರಂ-ಭಿ-ಸಿದ
ನೋಡ-ಲಾರೆ
ನೋಡ-ಲಿ-ದ್ದೇವೆ
ನೋಡ-ಲಿ-ರು-ವಂತೆ
ನೋಡ-ಲಿಲ್ಲ
ನೋಡಲು
ನೋಡ-ಲೆಂದು
ನೋಡ-ಲೇ-ಬೇಕು
ನೋಡ-ಲೇ-ಬೇ-ಕೆಂದು
ನೋಡಲ್ಲಿ
ನೋಡಿ
ನೋಡಿಕೊ
ನೋಡಿ-ಕೊಂ-ಡರು
ನೋಡಿ-ಕೊಂ-ಡ-ರು-ಎಲಾ
ನೋಡಿ-ಕೊಂ-ಡರೆ
ನೋಡಿ-ಕೊಂ-ಡಳು
ನೋಡಿ-ಕೊಂ-ಡಿ-ದ್ದಾರೆ
ನೋಡಿ-ಕೊಂ-ಡಿರಿ
ನೋಡಿ-ಕೊಂಡು
ನೋಡಿ-ಕೊ-ಳ್ಳ-ಬೇ-ಕಾ-ಗು-ತ್ತ-ದೆಯೋ
ನೋಡಿ-ಕೊ-ಳ್ಳ-ಬೇಕು
ನೋಡಿ-ಕೊ-ಳ್ಳ-ಬೇ-ಕೆಂಬ
ನೋಡಿ-ಕೊ-ಳ್ಳ-ಬೇಕೋ
ನೋಡಿ-ಕೊ-ಳ್ಳ-ಲಾ-ರ-ನೆಂದು
ನೋಡಿ-ಕೊ-ಳ್ಳ-ಲ್ಪ-ಡು-ತ್ತಾರೋ
ನೋಡಿ-ಕೊಳ್ಳು
ನೋಡಿ-ಕೊ-ಳ್ಳುತ್ತ
ನೋಡಿ-ಕೊ-ಳ್ಳು-ತ್ತಾರೆ
ನೋಡಿ-ಕೊ-ಳ್ಳು-ತ್ತಿ-ದ್ದರು
ನೋಡಿ-ಕೊ-ಳ್ಳು-ತ್ತಿ-ದ್ದಾನೆ
ನೋಡಿ-ಕೊ-ಳ್ಳು-ತ್ತಿ-ದ್ದಾರೆ
ನೋಡಿ-ಕೊ-ಳ್ಳು-ತ್ತಿ-ದ್ದುದು
ನೋಡಿ-ಕೊ-ಳ್ಳು-ತ್ತೇನೆ
ನೋಡಿ-ಕೊ-ಳ್ಳು-ತ್ತೇ-ನೆ-ಇದು
ನೋಡಿ-ಕೊ-ಳ್ಳುವ
ನೋಡಿ-ಕೊ-ಳ್ಳು-ವು-ದ-ಕ್ಕಾಗಿ
ನೋಡಿ-ಕೊ-ಳ್ಳು-ವುದು
ನೋಡಿ-ಕೊ-ಳ್ಳು-ವುದೇ
ನೋಡಿಕೋ
ನೋಡಿದ
ನೋಡಿ-ದಂತೆ
ನೋಡಿ-ದ-ರತ್ತ
ನೋಡಿ-ದರು
ನೋಡಿ-ದರೂ
ನೋಡಿ-ದರೆ
ನೋಡಿ-ದರೇ
ನೋಡಿ-ದ-ವ-ರ-ಲ್ಲೆಲ್ಲ
ನೋಡಿ-ದ-ವ-ರಿ-ಗೆಲ್ಲ
ನೋಡಿ-ದ-ವರು
ನೋಡಿ-ದಾಗ
ನೋಡಿ-ದಾ-ಗ-ಲೆಲ್ಲ
ನೋಡಿ-ದಿರಾ
ನೋಡಿದೆ
ನೋಡಿ-ದೆಯಾ
ನೋಡಿ-ದೆವು
ನೋಡಿ-ದೊ-ಡ-ನೆಯೇ
ನೋಡಿದ್ದ
ನೋಡಿ-ದ್ದರ
ನೋಡಿ-ದ್ದಾರೆ
ನೋಡಿ-ದ್ದೀರಾ
ನೋಡಿದ್ದು
ನೋಡಿ-ದ್ದೇವೆ
ನೋಡಿ-ಬಿ-ಡೋಣ
ನೋಡಿ-ಯೇ-ಬಿ-ಡ-ಬೇ-ಕೆಂಬ
ನೋಡಿ-ಯೇ-ಬಿ-ಡು-ವು-ದೆಂದು
ನೋಡಿ-ರ-ಲಿಲ್ಲ
ನೋಡಿ-ರು-ವಂತೆ
ನೋಡಿ-ಲ್ಲವೆ
ನೋಡಿ-ಲ್ಲ-ವೆಂ-ಬುದು
ನೋಡಿಲ್ಲಿ
ನೋಡು
ನೋಡುತ್ತ
ನೋಡು-ತ್ತಲೇ
ನೋಡುತ್ತಾ
ನೋಡು-ತ್ತಾ-ನೆ-ಹೌದು
ನೋಡು-ತ್ತಾರೆ
ನೋಡು-ತ್ತಾ-ರೆಆ
ನೋಡು-ತ್ತಾ-ರೆ-ಎಲ್ಲ
ನೋಡು-ತ್ತಾ-ರೆ-ಕಿ-ತ್ತಳೆ
ನೋಡು-ತ್ತಾ-ರೆ-ತಮ್ಮ
ನೋಡು-ತ್ತಾ-ರೆ-ಸ್ವಾ-ಮೀಜಿ
ನೋಡು-ತ್ತಿದ್ದ
ನೋಡು-ತ್ತಿ-ದ್ದರು
ನೋಡು-ತ್ತಿ-ದ್ದರೂ
ನೋಡು-ತ್ತಿ-ದ್ದಳು
ನೋಡು-ತ್ತಿ-ದ್ದಾನೋ
ನೋಡು-ತ್ತಿ-ದ್ದಾರೆ
ನೋಡು-ತ್ತಿ-ದ್ದಾ-ರೆಂದೂ
ನೋಡು-ತ್ತಿ-ದ್ದೀರಿ
ನೋಡು-ತ್ತಿ-ದ್ದು-ದೇನು
ನೋಡು-ತ್ತಿ-ದ್ದೇನೆ
ನೋಡು-ತ್ತಿರು
ನೋಡು-ತ್ತಿ-ರು-ತ್ತಾನೆ
ನೋಡು-ತ್ತಿ-ರು-ತ್ತೇವೆ
ನೋಡು-ತ್ತಿ-ರು-ವಂತೆ
ನೋಡು-ತ್ತಿ-ರು-ವುದು
ನೋಡು-ತ್ತಿ-ಲ್ಲ-ವೇನು
ನೋಡು-ತ್ತೇನೆ
ನೋಡು-ತ್ತೇ-ನೆ-ಹ-ಲ-ವಾರು
ನೋಡು-ತ್ತೇವೆ
ನೋಡುವ
ನೋಡು-ವಂತೆ
ನೋಡು-ವ-ವರ
ನೋಡು-ವ-ವ-ರಿಗೆ
ನೋಡು-ವ-ವರು
ನೋಡುವು
ನೋಡು-ವು-ದಕ್ಕೂ
ನೋಡು-ವುದು
ನೋಡು-ವೆ-ಯಂತೆ
ನೋಡೋಣ
ನೋಡೋ-ಣ-ಆ-ದರೆ
ನೋಡೋ-ಣ-ವೆಂದರೆ
ನೋಬೆಲ್
ನೋಬೆ-ಲ್ಲ-ಳಿಗೆ
ನೋಬೆ-ಲ್ಲಳು
ನೋವ-ನ್ನುಂ-ಟು-ಮಾ-ಡಲು
ನೋವಾ-ಗು-ತ್ತದೆ
ನೋವಾ-ಗು-ವಂತೆ
ನೋವಾ-ಯಿತು
ನೋವಿನ
ನೋವು
ನೋವು-ನ-ಲಿ-ವು-ಗಳನ್ನು
ನೋವು-ನ-ಲಿವು
ನೌಕ-ರನ
ನೌಕ-ರನೇ
ನೌಕ-ರಿಗೆ
ನೌಕಾ
ನೌಕಾ-ಪ-ಡೆ-ಯನ್ನು
ನ್ನರಿತು
ನ್ನರು
ನ್ನಾಗಿಯೂ
ನ್ನಾಡುವ
ನ್ನಾದರೂ
ನ್ನೆಲ್ಲ
ನ್ನೊಳ-ಗೊಂಡ
ನ್ಮಾತೆಯ
ನ್ಮಾತೆಯೇ
ನ್ಯಾಯ-ನಿ-ಷ್ಠೆ-ಇ-ವೆಲ್ಲ
ನ್ಯಾಯ-ಮೂ-ರ್ತಿ-ಗ-ಳಾದ
ನ್ಯಾಯ-ರ-ತ್ನರ
ನ್ಯಾಯ-ಶಾ-ಸ್ತ್ರದ
ನ್ಯಾಯ-ಸ್ಥಾ-ನದ
ನ್ಯಾಯಾ
ನ್ಯಾಯಾ-ಧೀಶ
ನ್ಯಾಯಾ-ಧೀ-ಶನ
ನ್ಯಾಯಾ-ಧೀ-ಶ-ನಾದ
ನ್ಯಾಯಾ-ಧೀ-ಶರು
ನ್ಯಾಯಾ-ಲಯ
ನ್ಯಾಯಾ-ಲ-ಯದ
ನ್ಯಾಯಾ-ಲ-ಯ-ದಿಂದ
ನ್ಯಾಸದ
ನ್ಯೂ
ನ್ಯೂಟ-ನ್ನನು
ನ್ಯೂಯಾರ್ಕಿ
ನ್ಯೂಯಾ-ರ್ಕಿಗೆ
ನ್ಯೂಯಾ-ರ್ಕಿನ
ನ್ಯೂಯಾ-ರ್ಕಿ-ನಲ್ಲಿ
ನ್ಯೂಯಾ-ರ್ಕಿ-ನ-ಲ್ಲಿ-ದ್ದರು
ನ್ಯೂಯಾ-ರ್ಕಿ-ನ-ಲ್ಲಿ-ದ್ದಾಗ
ನ್ಯೂಯಾ-ರ್ಕಿ-ನ-ಲ್ಲಿ-ರಲು
ನ್ಯೂಯಾ-ರ್ಕಿ-ನಿಂದ
ನ್ಯೂಯಾ-ರ್ಕ್
ಪ
ಪಂಕ್ತಿ-ಯಲ್ಲೇ
ಪಂಗಡ
ಪಂಗ-ಡ-ಗ-ಳಿಗೆ
ಪಂಗ-ಡ-ದ-ವರು
ಪಂಗುಂ
ಪಂಚ-ದಶೀ
ಪಂಚ್
ಪಂಜ-ರಕ್ಕೆ
ಪಂಜ-ರದ
ಪಂಜ-ರ-ದಂತೆ
ಪಂಡಿ
ಪಂಡಿತ
ಪಂಡಿ-ತನ
ಪಂಡಿ-ತ-ನನ್ನು
ಪಂಡಿ-ತ-ನಾ-ದ-ವ-ನೇ-ನಾ-ದರೂ
ಪಂಡಿ-ತ-ನಿಂದ
ಪಂಡಿ-ತ-ನಿಗೆ
ಪಂಡಿ-ತನೂ
ಪಂಡಿ-ತ-ನೊಬ್ಬ
ಪಂಡಿ-ತ-ನೊ-ಬ್ಬನ
ಪಂಡಿ-ತ-ನೊ-ಬ್ಬ-ನೊಂ-ದಿಗೆ
ಪಂಡಿ-ತನೋ
ಪಂಡಿ-ತರ
ಪಂಡಿ-ತ-ರನ್ನು
ಪಂಡಿ-ತ-ರನ್ನೂ
ಪಂಡಿ-ತ-ರನ್ನೋ
ಪಂಡಿ-ತ-ರಾ-ಗಿ-ದ್ದ-ರ-ಲ್ಲದೆ
ಪಂಡಿ-ತ-ರಾದ
ಪಂಡಿ-ತ-ರಿಗೆ
ಪಂಡಿ-ತರು
ಪಂಡಿ-ತರೂ
ಪಂಡಿ-ತ-ರೆಲ್ಲ
ಪಂಡಿ-ತ-ರೆ-ಲ್ಲರ
ಪಂಡಿ-ತ-ರೊಂ-ದಿಗೆ
ಪಂಡಿ-ತ-ವ-ರ್ಗ-ದ-ವರೆಲ್ಲ
ಪಂಡಿತ್ಜಿ
ಪಂಡಿ-ತ್ಜಿಯ
ಪಂತು-ಲು-ರ-ವರು
ಪಂಥ
ಪಂಥಕ್ಕೆ
ಪಂಥ-ಗಳನ್ನೂ
ಪಂಥ-ಗಳು
ಪಂಥದ
ಪಂಥ-ವನ್ನೇ
ಪಂಥ-ವನ್ನೋ
ಪಂಥವೂ
ಪಂಥ-ವೇನು
ಪಂಥಾ
ಪಂದ್ಯ-ದಲ್ಲಿ
ಪಕ್ಕಕ್ಕೆ
ಪಕ್ಕ-ಗಳಲ್ಲಿ
ಪಕ್ಕದ
ಪಕ್ಕ-ದಲ್ಲಿ
ಪಕ್ಕ-ದಲ್ಲೇ
ಪಕ್ಕ-ದ್ಲಲೇ
ಪಕ್ಕ-ವಾದ್ಯ
ಪಕ್ಕಾ
ಪಕ್ಷ
ಪಕ್ಷ-ದಲ್ಲಿ
ಪಕ್ಷ-ಪಾತ
ಪಕ್ಷ-ವನ್ನು
ಪಕ್ಷಿ-ಗ-ಳೆಲ್ಲ
ಪಚ್ಚೈ-ಯಪ್ಪ
ಪಟ್ಟ
ಪಟ್ಟಣ
ಪಟ್ಟ-ಣಕ್ಕೆ
ಪಟ್ಟ-ಣದ
ಪಟ್ಟ-ಣ-ದಲ್ಲಿ
ಪಟ್ಟ-ಣ-ದ-ಲ್ಲಿಯೂ
ಪಟ್ಟ-ಣ-ದೆ-ಡೆಗೆ
ಪಟ್ಟ-ದ-ರಸಿ
ಪಟ್ಟದ್ದು
ಪಟ್ಟನ್ನು
ಪಟ್ಟರು
ಪಟ್ಟ-ವ-ರಿಗೂ
ಪಟ್ಟಾಗಿ
ಪಟ್ಟಿದ್ದು
ಪಟ್ಟಿ-ಯಲ್ಲಿ
ಪಟ್ಟು
ಪಟ್ಟು-ಕೊ-ಳ್ಳ-ಬಾ-ರದು
ಪಟ್ಟು-ಕೊ-ಳ್ಳುವ
ಪಟ್ಟು-ಕೊ-ಳ್ಳು-ವು-ದಿಲ್ಲ
ಪಟ್ಟು-ಬಿ-ಡದೆ
ಪಟ್ಟೆ
ಪಠಿಸಿ
ಪಠಿ-ಸಿ-ದರು
ಪಠಿ-ಸುತ್ತ
ಪಠಿ-ಸು-ತ್ತಿದ್ದ
ಪಠಿ-ಸು-ತ್ತಿ-ದ್ದರು
ಪಠಿ-ಸುವ
ಪಡ-ದಿ-ರು-ವುದನ್ನು
ಪಡದೆ
ಪಡ-ಬಾ-ರದ
ಪಡ-ಬೇ-ಕಾ-ಗ-ಬ-ಹು-ದೆಂದು
ಪಡ-ಬೇ-ಕಾಗಿ
ಪಡ-ಸಾ-ಲೆ-ಯಲ್ಲಿ
ಪಡಿ-ಸ-ಬೇ-ಕೆಂದೂ
ಪಡಿ-ಸಲು
ಪಡಿ-ಸ-ಲೇ-ಬೇಕು
ಪಡಿಸಿ
ಪಡಿ-ಸಿಕೊ
ಪಡಿ-ಸಿ-ಕೊಂಡು
ಪಡಿ-ಸಿ-ದಂ-ತಹ
ಪಡಿ-ಸಿ-ದ-ನ-ಲ್ಲದೆ
ಪಡಿ-ಸಿ-ದ್ದರು
ಪಡಿ-ಸುತ್ತ
ಪಡಿ-ಸು-ತ್ತಿದ್ದ
ಪಡಿ-ಸು-ತ್ತಿ-ದ್ದರು
ಪಡಿ-ಸು-ವಂತೆ
ಪಡಿ-ಸು-ವಲ್ಲಿ
ಪಡಿ-ಸು-ವು-ದಾ-ದರೆ
ಪಡು-ತ್ತಾರೆ
ಪಡು-ತ್ತಿದ್ದ
ಪಡು-ತ್ತೀರಿ
ಪಡು-ತ್ತೇನೆ
ಪಡುವ
ಪಡು-ವಂ-ತಾ-ಯಿತು
ಪಡು-ವಂ-ಥ-ದೇ-ನನ್ನೂ
ಪಡು-ವುದು
ಪಡೆ
ಪಡೆದ
ಪಡೆ-ದ-ದ್ದಾ-ಯಿತು
ಪಡೆ-ದದ್ದು
ಪಡೆ-ದ-ನೆಂ-ದರೆ
ಪಡೆ-ದರು
ಪಡೆ-ದಳು
ಪಡೆ-ದ-ವ-ನಾ-ಗಿ-ರ-ಲಿಲ್ಲ
ಪಡೆ-ದ-ವ-ನಾ-ದರೂ
ಪಡೆ-ದ-ವ-ರಿ-ಗೆಲ್ಲ
ಪಡೆ-ದ-ವರು
ಪಡೆ-ದಿದೆ
ಪಡೆ-ದಿ-ದ್ದರು
ಪಡೆ-ದಿ-ದ್ದೇವೆ
ಪಡೆ-ದಿ-ರ-ಲಿಲ್ಲ
ಪಡೆ-ದಿ-ರು-ವುದನ್ನು
ಪಡೆ-ದಿವೆ
ಪಡೆದು
ಪಡೆ-ದುಕೊ
ಪಡೆ-ದು-ಕೊಂಡ
ಪಡೆ-ದು-ಕೊಂ-ಡ-ರ-ಲ್ಲದೆ
ಪಡೆ-ದು-ಕೊಂ-ಡರು
ಪಡೆ-ದು-ಕೊಂ-ಡ-ವರು
ಪಡೆ-ದು-ಕೊಂ-ಡಿ-ದ್ದರು
ಪಡೆ-ದು-ಕೊಂ-ಡಿ-ದ್ದೇನೆ
ಪಡೆ-ದು-ಕೊಂ-ಡಿ-ರುವ
ಪಡೆ-ದು-ಕೊಂ-ಡಿ-ರು-ವಂತೆ
ಪಡೆ-ದು-ಕೊಂಡು
ಪಡೆ-ದು-ಕೊಂ-ಡು-ಬಿ-ಡ-ಬ-ಹು-ದೆಂದು
ಪಡೆ-ದು-ಕೊ-ಳ್ಳ-ಬ-ಹುದು
ಪಡೆ-ದು-ಕೊ-ಳ್ಳ-ಬ-ಹು-ದೇನೊ
ಪಡೆ-ದು-ಕೊ-ಳ್ಳ-ಬೇಕು
ಪಡೆ-ದು-ಕೊ-ಳ್ಳ-ಲಾ-ರೆವು
ಪಡೆ-ದು-ಕೊ-ಳ್ಳಲು
ಪಡೆ-ದು-ಕೊಳ್ಳು
ಪಡೆ-ದು-ಕೊ-ಳ್ಳುತ್ತ
ಪಡೆ-ದು-ಕೊ-ಳ್ಳು-ತ್ತಿ-ದ್ದರು
ಪಡೆ-ದು-ಕೊ-ಳ್ಳುತ್ತೀ
ಪಡೆ-ದು-ಕೊ-ಳ್ಳು-ತ್ತೀ-ರೆಂದು
ಪಡೆ-ದು-ಕೊ-ಳ್ಳುವ
ಪಡೆ-ದು-ಕೊ-ಳ್ಳು-ವಂತೆ
ಪಡೆ-ದು-ಕೊ-ಳ್ಳು-ವುದು
ಪಡೆ-ದು-ಕೊ-ಳ್ಳು-ವುದೇ
ಪಡೆ-ದು-ಕೊ-ಳ್ಳೋಣ
ಪಡೆ-ದೆವು
ಪಡೆ-ಯನ್ನೇ
ಪಡೆ-ಯ-ಬಲ್ಲ
ಪಡೆ-ಯ-ಬ-ಹುದಾ
ಪಡೆ-ಯ-ಬೇ-ಕಾ-ದರೆ
ಪಡೆ-ಯ-ಬೇ-ಕೆಂಬ
ಪಡೆ-ಯಲಿ
ಪಡೆ-ಯಲು
ಪಡೆ-ಯ-ಲ್ಪ-ಡುವ
ಪಡೆ-ಯಿತು
ಪಡೆ-ಯುತ್ತ
ಪಡೆ-ಯು-ತ್ತಾರೆ
ಪಡೆ-ಯು-ತ್ತಿದ್ದ
ಪಡೆ-ಯು-ತ್ತಿ-ದ್ದರು
ಪಡೆ-ಯು-ತ್ತಿ-ದ್ದ-ರೆಂ-ಬುದು
ಪಡೆ-ಯು-ತ್ತೀರಿ
ಪಡೆ-ಯು-ತ್ತೇನೆ
ಪಡೆ-ಯುವ
ಪಡೆ-ಯು-ವಂ-ತಾ-ಗಲು
ಪಡೆ-ಯು-ವಲ್ಲಿ
ಪಡೆ-ಯು-ವ-ವ-ರೆಗೂ
ಪಡೆ-ಯು-ವಿರಿ
ಪಡೆ-ಯು-ವು-ದಾ-ದರೆ
ಪಡೆ-ಯು-ವುದು
ಪಡೆ-ಯೋ-ಣ-ವೆಂದು
ಪಣ-ತೊ-ಟ್ಟು-ನಿಂತ
ಪತಂಗ
ಪತಂ-ಗ-ಗ-ಳಂತೆ
ಪತಂ-ಜಲಿ
ಪತಂ-ಜ-ಲಿಯ
ಪತಿ
ಪತಿ-ಪ-ತ್ನಿ-ಯರ
ಪತಿಯ
ಪತಿ-ಯನ್ನು
ಪತಿ-ಯ-ವರು
ಪತ್ತೆ
ಪತ್ತೆ-ಹ-ಚ್ಚಿಯೇ
ಪತ್ತೆ-ಹ-ಚ್ಚಿ-ರು-ತ್ತಾರೆ
ಪತ್ತೇ-ದಾರಿಯೇ
ಪತ್ತೇ-ದಾರೀ
ಪತ್ನಿ
ಪತ್ನಿಗೂ
ಪತ್ನಿಗೆ
ಪತ್ನಿ-ಯನ್ನು
ಪತ್ನಿ-ಯರು
ಪತ್ನಿಯು
ಪತ್ನಿಯೂ
ಪತ್ರ
ಪತ್ರ-ಕ-ರ್ತ-ನೊಬ್ಬ
ಪತ್ರ-ಕ-ರ್ತರ
ಪತ್ರ-ಕ-ರ್ತ-ರಂತೂ
ಪತ್ರ-ಕ-ರ್ತರು
ಪತ್ರ-ಕ-ರ್ತರೂ
ಪತ್ರಕ್ಕೂ
ಪತ್ರಕ್ಕೆ
ಪತ್ರ-ಕ್ಕೇನೋ
ಪತ್ರ-ಕ್ಕೊಂದು
ಪತ್ರ-ಗಳ
ಪತ್ರ-ಗಳನ್ನು
ಪತ್ರ-ಗಳನ್ನೂ
ಪತ್ರ-ಗಳನ್ನೆಲ್ಲ
ಪತ್ರ-ಗಳಲ್ಲಿ
ಪತ್ರ-ಗ-ಳಲ್ಲೂ
ಪತ್ರ-ಗಳಿಂದ
ಪತ್ರ-ಗಳು
ಪತ್ರ-ಗ-ಳು-ಲೇ-ಖ-ನ-ಗಳು
ಪತ್ರ-ಗಳೂ
ಪತ್ರ-ಗ-ಳೆಲ್ಲ
ಪತ್ರ-ಗಳೇ
ಪತ್ರದ
ಪತ್ರ-ದಲ್ಲಿ
ಪತ್ರ-ದಿಂದ
ಪತ್ರ-ದೊಂ-ದಿಗೆ
ಪತ್ರ-ಮು-ಖೇನ
ಪತ್ರ-ವ-ನ್ನಾ-ದರೂ
ಪತ್ರ-ವನ್ನು
ಪತ್ರ-ವನ್ನೂ
ಪತ್ರ-ವನ್ನೇ
ಪತ್ರವು
ಪತ್ರವೂ
ಪತ್ರ-ವೊಂ-ದನ್ನು
ಪತ್ರ-ವೊಂ-ದರ
ಪತ್ರ-ವೊಂ-ದ-ರಲ್ಲಿ
ಪತ್ರ-ವೊಂ-ದ-ರಿಂದ
ಪತ್ರ-ವೊಂದು
ಪತ್ರ-ವ್ಯ-ವ-ಹಾ-ರದ
ಪತ್ರ-ವ್ಯ-ವ-ಹಾ-ರ-ವ-ನ್ನಿ-ಟ್ಟು-ಕೊಂ-ಡಿ-ರ-ಲಿಲ್ಲ
ಪತ್ರ-ವ್ಯ-ವ-ಹಾ-ರ-ವನ್ನು
ಪತ್ರ-ವ್ಯ-ವ-ಹಾ-ರವೂ
ಪತ್ರಿಕಾ
ಪತ್ರಿ-ಕಾ-ವ-ಲ-ಯ-ಗಳಲ್ಲಿ
ಪತ್ರಿಕೆ
ಪತ್ರಿ-ಕೆ-ಗಳ
ಪತ್ರಿ-ಕೆ-ಗ-ಳಂತೂ
ಪತ್ರಿ-ಕೆ-ಗಳನ್ನು
ಪತ್ರಿ-ಕೆ-ಗಳಲ್ಲಿ
ಪತ್ರಿ-ಕೆ-ಗ-ಳಲ್ಲೂ
ಪತ್ರಿ-ಕೆ-ಗ-ಳ-ಲ್ಲೆಲ್ಲ
ಪತ್ರಿ-ಕೆ-ಗ-ಳಾದ
ಪತ್ರಿ-ಕೆ-ಗಳಿಂದ
ಪತ್ರಿ-ಕೆ-ಗ-ಳಿಗೆ
ಪತ್ರಿ-ಕೆ-ಗಳು
ಪತ್ರಿ-ಕೆ-ಗಾಗಿ
ಪತ್ರಿ-ಕೆಗೆ
ಪತ್ರಿ-ಕೆಯ
ಪತ್ರಿ-ಕೆ-ಯನ್ನು
ಪತ್ರಿ-ಕೆ-ಯಲ್ಲಿ
ಪತ್ರಿ-ಕೆ-ಯಾ-ಗಿದೆ
ಪತ್ರಿ-ಕೆ-ಯಾದ
ಪತ್ರಿ-ಕೆಯು
ಪತ್ರಿ-ಕೆ-ಯೊಂ-ದಕ್ಕೆ
ಪತ್ರಿ-ಕೆ-ಯೊಂ-ದನ್ನು
ಪತ್ರಿ-ಕೆ-ಯೊಂ-ದರ
ಪತ್ರಿ-ಕೆ-ಯೊಂ-ದ-ರಲ್ಲಿ
ಪತ್ರಿ-ಕೆ-ಯೊಂದು
ಪತ್ರಿ-ಕೋ-ದ್ಯ-ಮಿ-ಯಾಗಿ
ಪತ್ರಿ-ಕೋ-ದ್ಯ-ಮಿ-ಯೆಂದು
ಪತ್ರಿ-ಯೊ-ಬ್ಬರೂ
ಪಥ
ಪಥಕ್ಕೆ
ಪಥ-ಗಳು
ಪಥದ
ಪಥ-ವನ್ನೂ
ಪಥ-ವನ್ನೇ
ಪಥವು
ಪಥಾ-ವ-ಲಂ-ಬಿ-ಗಾ-ಳಾದ
ಪಥಿ-ಕ-ರೊ-ಳ-ಗಿನ
ಪಥ್ಯ-ಮಿತಿ
ಪದ
ಪದಕ
ಪದ-ಗಳು
ಪದ-ತ-ಲ-ದಲ್ಲಿ
ಪದ-ದ-ಡಿಗೆ
ಪದ-ದ-ಡಿ-ಯಲ್ಲಿ
ಪದ-ದಲ್ಲಿ
ಪದ-ಪ್ರ-ಯೋ-ಗ-ಗಳೂ
ಪದರ
ಪದ-ರ-ಗ-ಳ-ಡಿ-ಯಲ್ಲಿ
ಪದ-ರ-ಗಳಿಂದ
ಪದ-ರ-ಗಳು
ಪದ-ರ-ದಲ್ಲಿ
ಪದ-ರ-ಪ-ದ-ರ-ವಾಗಿ
ಪದ-ರ-ವನ್ನು
ಪದ-ವನ್ನು
ಪದ-ವನ್ನೇ
ಪದವಿ
ಪದ-ವಿ-ಗಳನ್ನೂ
ಪದ-ವಿಗೆ
ಪದ-ವಿ-ಗೇ-ರಿ-ದ್ದರು
ಪದ-ವೀ-ಧರ
ಪದ-ವೀ-ಧ-ರ-ನಂತೆ
ಪದ-ವೀ-ಧ-ರ-ರಾ-ಗಿದ್ದು
ಪದ-ವೀ-ಧ-ರ-ರಾದ
ಪದ-ವೀ-ಧ-ರ-ರಾ-ದರೂ
ಪದ-ವೀ-ಧ-ರರು
ಪದ-ವೀ-ಧ-ರರೂ
ಪದವೂ
ಪದಾ-ಧಿ-ಕಾ-ರಿ-ಗಳು
ಪದಾ-ರ್ಥಕ್ಕೆ
ಪದಾ-ರ್ಥ-ಗಳನ್ನು
ಪದಾ-ರ್ಥ-ಗಳಿಂದ
ಪದ್ಧತಿ
ಪದ್ಧ-ತಿ-ಗ-ನು-ಸಾ-ರ-ವಾಗಿ
ಪದ್ಧ-ತಿ-ಗಳ
ಪದ್ಧ-ತಿ-ಗಳು
ಪದ್ಧ-ತಿಗೆ
ಪದ್ಧ-ತಿಯ
ಪದ್ಧ-ತಿ-ಯನ್ನು
ಪದ್ಧ-ತಿಯು
ಪದ್ಮಾ-ಸನ
ಪದ್ಯದ
ಪಯ-ಣಿ-ಸ-ಲಿತ್ತು
ಪಯ-ಣಿಸಿ
ಪಯ-ಣಿ-ಸಿತು
ಪಯ-ಣಿ-ಸಿ-ದರು
ಪಯ-ಣಿ-ಸು-ತ್ತಿರು
ಪಯ-ಣಿ-ಸು-ತ್ತಿ-ರು-ವಾಗ
ಪಯ-ಸಾಂ
ಪಯೊ-ನಿ-ಯರ್
ಪರ
ಪರಂಗಿ
ಪರಂ-ಪರೆ
ಪರಂ-ಪ-ರೆ-ಯ-ನ್ನುಳ್ಳ
ಪರಂ-ಪ-ರೆಯೇ
ಪರಂ-ಪ-ರೆ-ಯೊಂ-ದನ್ನು
ಪರ-ಕೀಯ
ಪರ-ಕೀ-ಯ-ರಲ್ಲ
ಪರ-ಕೀ-ಯ-ರಿಗೇ
ಪರ-ಕೀ-ಯರು
ಪರ-ತತ್ತ್ವ
ಪರ-ತ-ತ್ತ್ವ-ವಾ-ಗಿ-ದ್ದೇನೆ
ಪರ-ತ-ತ್ತ್ವವೇ
ಪರ-ದೇ-ಶದ
ಪರ-ದೇ-ಶ-ದಲ್ಲಿ
ಪರ-ದೇ-ಶ-ವೊಂ-ದ-ರಲ್ಲಿ
ಪರ-ದೇಶಿ
ಪರ-ಧರ್ಮ
ಪರ-ಬ್ರಹ್ಮ
ಪರ-ಬ್ರ-ಹ್ಮನ
ಪರ-ಬ್ರ-ಹ್ಮ-ಭಾ-ವ-ದಲ್ಲಿ
ಪರ-ಬ್ರ-ಹ್ಮ-ವನ್ನು
ಪರ-ಬ್ರ-ಹ್ಮವೇ
ಪರ-ಭಾ-ಷೆ-ಯಲ್ಲಿ
ಪರಮ
ಪರ-ಮ-ಕ-ರು-ಣಾ-ಮೂ-ರ್ತಿ-ಯಾದ
ಪರ-ಮ-ಗತಿ
ಪರ-ಮ-ಗು-ರಿ-ಯಾ-ಗಿ-ದೆಯೋ
ಪರ-ಮ-ಗು-ರಿ-ಯಾದ
ಪರ-ಮ-ಗುರು
ಪರ-ಮತ
ಪರ-ಮ-ತ-ತ್ತ್ವದ
ಪರ-ಮ-ದ-ರ್ಶ-ನ-ವನ್ನು
ಪರ-ಮ-ಧ-ರ್ಮದ
ಪರ-ಮ-ಪ-ವಿತ್ರ
ಪರ-ಮ-ಪೂಜ್ಯ
ಪರ-ಮ-ಪ್ರಿಯ
ಪರ-ಮ-ಭಿ-ಕಾ-ರಿಯ
ಪರ-ಮ-ಮಂ-ಗ-ಳ-ತೀ-ರ್ಥ-ವ-ಹು-ದ-ಲ್ಲವೆ
ಪರ-ಮ-ಸ-ತ್ಯದ
ಪರ-ಮ-ಸ-ತ್ಯ-ದೊಂ-ದಿಗೆ
ಪರ-ಮ-ಸುಖ
ಪರ-ಮ-ಹಂಸ
ಪರ-ಮ-ಹಂ-ಸರ
ಪರ-ಮ-ಹಂ-ಸ-ರಿಂದ
ಪರ-ಮ-ಹಂ-ಸರು
ಪರ-ಮ-ಹಂ-ಸ-ರೊ-ಬ್ಬರು
ಪರ-ಮ-ಹಂ-ಸಾ-ಚ-ಲದ
ಪರ-ಮಾ-ತ್ಮ-ನನ್ನೇ
ಪರ-ಮಾ-ತ್ಮ-ನಲ್ಲಿ
ಪರ-ಮಾ-ತ್ಮ-ನೊಂ-ದಿಗೆ
ಪರ-ಮಾ-ತ್ಮ-ರಂ-ಗದ
ಪರ-ಮಾ-ದರ್ಶ
ಪರ-ಮಾ-ದ-ರ್ಶ-ವನ್ನು
ಪರ-ಮಾ-ದ-ರ್ಶ-ಸಾ-ಧ-ನೆಗೆ
ಪರ-ಮಾ-ದ್ಭುತ
ಪರ-ಮಾ-ನಂ-ದ-ಗೊಂಡು
ಪರ-ಮಾರ್ಥ
ಪರ-ಮಾ-ವ-ಧಿ-ಯನ್ನು
ಪರ-ಮಾ-ವ-ಧಿ-ಯ-ಲ್ಲಿದ್ದ
ಪರ-ಮಾ-ಶ್ಚರ್ಯ
ಪರ-ಮಾ-ಶ್ಚ-ರ್ಯ-ಗೊಂ-ಡರು
ಪರ-ಮಾ-ಶ್ಚ-ರ್ಯ-ಗೊ-ಳ್ಳು-ತ್ತಿ-ದ್ದರು
ಪರ-ಮಿ-ದ-ಮದಃ
ಪರ-ಮೇ-ಶ್ವ-ರ-ನನ್ನು
ಪರ-ಮೋಚ್ಚ
ಪರ-ಮೋ-ತ್ಕೃಷ್ಟ
ಪರ-ಮೋ-ದಾತ್ತ
ಪರ-ಮೋ-ದ್ದೇಶ
ಪರ-ಮೋ-ದ್ದೇ-ಶ-ವಾ-ಗುಳ್ಳ
ಪರ-ರಾ-ಷ್ಟ್ರ-ಗಳ
ಪರ-ರಾ-ಷ್ಟ್ರ-ಗ-ಳ-ಲ್ಲಿ-ಅ-ದ-ರಲ್ಲೂ
ಪರ-ರಾ-ಷ್ಟ್ರ-ಗ-ಳೊಂ-ದಿಗೆ
ಪರ-ರಾ-ಷ್ಟ್ರದ
ಪರ-ರಾ-ಷ್ಟ್ರ-ವೊಂ-ದರ
ಪರ-ರಾ-ಷ್ಟ್ರ-ವೊಂದು
ಪರವಾ
ಪರ-ವಾಗಿ
ಪರ-ವಾ-ಗಿಯೇ
ಪರ-ವಾ-ಗಿಲ್ಲ
ಪರ-ವಾದ
ಪರ-ಸಂ-ಸ್ಕೃ-ತಿಯ
ಪರ-ಸ-ಮಾ-ಜ-ಗಳ
ಪರ-ಸ್ತಾತ್
ಪರ-ಸ್ಪರ
ಪರ-ಸ್ಪ-ರರ
ಪರ-ಸ್ಪ-ರ-ರನ್ನು
ಪರ-ಸ್ಪ-ರ-ರಿಂದ
ಪರ-ಹರಿ
ಪರಾ-ಕಾ-ಷ್ಠೆ-ಗೇ-ರಿ-ಬಿ-ಟ್ಟಿತ್ತು
ಪರಾ-ನು-ಕ-ರ-ಣೆಗೆ
ಪರಾ-ಭ-ವ-ಗೊ-ಳಿ-ಸಿ-ದ-ವರು
ಪರಾ-ಮ-ರ್ಶಿ-ಸಿ-ದರು
ಪರಾ-ರಿ-ಯಾಗಿ
ಪರಾ-ವ-ಲಂಬೀ
ಪರಿ
ಪರಿ-ಗ-ಣಿಸ
ಪರಿ-ಗ-ಣಿ-ಸ-ಬೇ-ಕು-ಆಗ
ಪರಿ-ಗ-ಣಿ-ಸ-ಲಾ-ಗದು
ಪರಿ-ಗ-ಣಿ-ಸ-ಲಿಲ್ಲ
ಪರಿ-ಗ-ಣಿ-ಸ-ಲ್ಪ-ಡು-ತ್ತದೆ
ಪರಿ-ಗ-ಣಿ-ಸ-ಲ್ಪ-ಡು-ತ್ತಿ-ದ್ದುವು
ಪರಿ-ಗ-ಣಿ-ಸ-ಲ್ಪ-ಡುವ
ಪರಿ-ಗ-ಣಿಸಿ
ಪರಿ-ಗ-ಣಿ-ಸಿ-ದರೂ
ಪರಿ-ಗ-ಣಿ-ಸಿ-ದರೆ
ಪರಿ-ಗ-ಣಿ-ಸಿ-ದ-ರೆಂ-ಬುದು
ಪರಿ-ಗ-ಣಿ-ಸಿ-ದ್ದರು
ಪರಿ-ಗ-ಣಿ-ಸಿ-ದ್ದೇವೆ
ಪರಿ-ಗ-ಣಿ-ಸು-ತ್ತಾರೆ
ಪರಿ-ಗ-ಣಿ-ಸು-ತ್ತಿ-ರ-ಲಿಲ್ಲ
ಪರಿ-ಗ-ಣಿ-ಸು-ತ್ತೇನೆ
ಪರಿ-ಚ-ತ-ರಿ-ಲ್ಲ-ವೆಂ-ಬುದು
ಪರಿ-ಚಯ
ಪರಿ-ಚ-ಯ-ದಿಂದ
ಪರಿ-ಚ-ಯ-ಪತ್ರ
ಪರಿ-ಚ-ಯ-ಪ-ತ್ರ-ಗಳನ್ನು
ಪರಿ-ಚ-ಯ-ಪ-ತ್ರದ
ಪರಿ-ಚ-ಯ-ಪ-ತ್ರ-ವನ್ನು
ಪರಿ-ಚ-ಯ-ವಾ-ಗ-ಲಿಲ್ಲ
ಪರಿ-ಚ-ಯ-ವಾಗಿ
ಪರಿ-ಚ-ಯ-ವಾ-ಗಿತ್ತು
ಪರಿ-ಚ-ಯ-ವಾ-ಗಿದ್ದ
ಪರಿ-ಚ-ಯ-ವಾ-ಗು-ತ್ತಿತ್ತು
ಪರಿ-ಚ-ಯ-ವಾ-ಗು-ವಂತೆ
ಪರಿ-ಚ-ಯ-ವಾದ
ಪರಿ-ಚ-ಯ-ವಾ-ದದ್ದು
ಪರಿ-ಚ-ಯ-ವಾ-ದಾಗ
ಪರಿ-ಚ-ಯ-ವಾ-ಯಿತು
ಪರಿ-ಚ-ಯ-ವಿತ್ತು
ಪರಿ-ಚ-ಯ-ವಿ-ರ-ಲಿಲ್ಲ
ಪರಿ-ಚ-ಯ-ವಿ-ರ-ಲಿ-ಲ್ಲ-ವಾ-ದರೂ
ಪರಿ-ಚ-ಯ-ವಿ-ರುವ
ಪರಿ-ಚ-ಯವೇ
ಪರಿ-ಚ-ಯ-ಸ್ಥ-ರನ್ನು
ಪರಿ-ಚ-ಯ-ಸ್ಥ-ರಾದ
ಪರಿ-ಚಯಿ
ಪರಿ-ಚ-ಯಿ-ಸಲು
ಪರಿ-ಚ-ಯಿ-ಸಲೇ
ಪರಿ-ಚ-ಯಿಸಿ
ಪರಿ-ಚ-ಯಿ-ಸಿ-ಕೊಂ-ಡರು
ಪರಿ-ಚ-ಯಿ-ಸಿ-ಕೊಟ್ಟ
ಪರಿ-ಚ-ಯಿ-ಸಿ-ಕೊ-ಟ್ಟರು
ಪರಿ-ಚ-ಯಿ-ಸಿ-ಕೊಟ್ಟು
ಪರಿ-ಚ-ಯಿ-ಸಿ-ಕೊ-ಡುತ್ತ
ಪರಿ-ಚ-ಯಿ-ಸಿ-ಕೊ-ಡು-ತ್ತಿ-ದ್ದಂತೆ
ಪರಿ-ಚ-ಯಿ-ಸಿದ
ಪರಿ-ಚ-ಯಿ-ಸಿ-ದರು
ಪರಿ-ಚ-ಯಿ-ಸು-ತ್ತಾನೆ
ಪರಿ-ಚ-ಯಿ-ಸುವ
ಪರಿ-ಚಾ-ರಿ-ಕೆಯ
ಪರಿ-ಚಿ-ತ-ನಾ-ಗಿದ್ದ
ಪರಿ-ಚಿ-ತನೂ
ಪರಿ-ಚಿ-ತ-ರಲ್ಲಿ
ಪರಿ-ಚಿ-ತ-ರಾ-ಗಿದ್ದ
ಪರಿ-ಚಿ-ತ-ರಾ-ಗಿ-ದ್ದರೂ
ಪರಿ-ಚಿ-ತ-ರಾ-ಗಿ-ದ್ದ-ವರು
ಪರಿ-ಚಿ-ತ-ರಾ-ಗಿ-ರ-ಲಿಲ್ಲ
ಪರಿ-ಚಿ-ತ-ರಾ-ಗು-ವಂ-ತಾ-ಗಲೂ
ಪರಿ-ಚಿ-ತ-ರಾದ
ಪರಿ-ಚಿ-ತ-ರಾ-ದ-ವ-ರೊ-ಬ್ಬರೂ
ಪರಿ-ಚಿ-ತರಿ
ಪರಿ-ಚಿ-ತ-ರಿಗೆ
ಪರಿ-ಚಿ-ತರು
ಪರಿ-ಚಿ-ತ-ಳಾ-ಗಿದ್ದ
ಪರಿ-ಚಿ-ತ-ವಾದ
ಪರಿ-ಜ್ಞಾ-ನ-ವನ್ನು
ಪರಿ-ಜ್ಞಾ-ನ-ವನ್ನೂ
ಪರಿ-ಜ್ಞಾ-ನವೂ
ಪರಿ-ಜ್ಞಾ-ನವೇ
ಪರಿ-ಣತ
ಪರಿ-ಣ-ತ-ನಾ-ಗಿದ್ದ
ಪರಿ-ಣ-ತ-ನಾದ
ಪರಿ-ಣ-ತ-ರಾ-ಗು-ವಂತೆ
ಪರಿ-ಣ-ತ-ರಿ-ರ-ಬ-ಹು-ದು-ಇ-ಲ್ಲಿನ
ಪರಿ-ಣ-ತರೂ
ಪರಿ-ಣ-ತ-ರೆಂಬು
ಪರಿ-ಣ-ತ-ರೊಂ-ದಿಗೆ
ಪರಿ-ಣತಿ
ಪರಿ-ಣ-ತಿ-ಯನ್ನು
ಪರಿ-ಣ-ಮಿ-ಸಿತು
ಪರಿ-ಣ-ಮಿ-ಸಿತ್ತು
ಪರಿ-ಣ-ಮಿ-ಸಿ-ದುವು
ಪರಿ-ಣ-ಮಿ-ಸಿದ್ದ
ಪರಿ-ಣ-ಮಿ-ಸಿ-ರು-ವುದನ್ನು
ಪರಿ-ಣ-ಮಿ-ಸು-ತ್ತದೆ
ಪರಿ-ಣ-ಮಿ-ಸು-ತ್ತಿ-ದ್ದುವು
ಪರಿ-ಣ-ಮಿ-ಸುವ
ಪರಿ-ಣಾಮ
ಪರಿ-ಣಾ-ಮ-ಇ-ವು-ಗ-ಳಿಂ-ದಾಗಿ
ಪರಿ-ಣಾ-ಮ-ಕಾರಿ
ಪರಿ-ಣಾ-ಮ-ಕಾ-ರಿ-ಯಾಗಿ
ಪರಿ-ಣಾ-ಮ-ಕಾ-ರಿ-ಯಾ-ಗಿತ್ತು
ಪರಿ-ಣಾ-ಮ-ಕಾ-ರಿ-ಯಾ-ಗಿ-ತ್ತು-ಎ-ಷ್ಟರ
ಪರಿ-ಣಾ-ಮ-ಕಾ-ರಿ-ಯಾ-ಗಿ-ತ್ತೆಂ-ದರೆ
ಪರಿ-ಣಾ-ಮ-ಕಾ-ರಿ-ಯಾ-ಗಿ-ರ-ಲಾ-ರ-ದೆಂದು
ಪರಿ-ಣಾ-ಮ-ಕಾ-ರಿ-ಯಾ-ಗು-ತ್ತದೆ
ಪರಿ-ಣಾ-ಮ-ಕಾ-ರಿ-ಯಾ-ಗು-ತ್ತಿದೆ
ಪರಿ-ಣಾ-ಮ-ಕಾ-ರಿ-ಯಾದ
ಪರಿ-ಣಾ-ಮ-ಕಾ-ರಿಯೂ
ಪರಿ-ಣಾ-ಮ-ಗ-ಳೇನೇ
ಪರಿ-ಣಾ-ಮದ
ಪರಿ-ಣಾ-ಮ-ವನ್ನು
ಪರಿ-ಣಾ-ಮ-ವ-ನ್ನುಂ-ಟು-ಮಾ-ಡದ
ಪರಿ-ಣಾ-ಮ-ವ-ನ್ನುಂ-ಟು-ಮಾ-ಡ-ದಿ-ದ್ದು-ದನ್ನು
ಪರಿ-ಣಾ-ಮ-ವ-ನ್ನುಂ-ಟು-ಮಾ-ಡದೆ
ಪರಿ-ಣಾ-ಮ-ವ-ನ್ನುಂ-ಟು-ಮಾ-ಡ-ಬಲ್ಲ
ಪರಿ-ಣಾ-ಮ-ವ-ನ್ನುಂ-ಟು-ಮಾ-ಡಿ-ದು-ವೆಂದರೆ
ಪರಿ-ಣಾ-ಮ-ವ-ನ್ನುಂ-ಟು-ಮಾ-ಡು-ತ್ತಿದೆ
ಪರಿ-ಣಾ-ಮ-ವಾಗಿ
ಪರಿ-ಣಾ-ಮ-ವಾದ
ಪರಿ-ಣಾ-ಮ-ವಾ-ಯಿ-ತೆಂದೂ
ಪರಿ-ಣಾ-ಮವೂ
ಪರಿ-ಣಾ-ಮವೇ
ಪರಿ-ಣಾ-ಮ-ವೇ-ನಾ-ಯಿತು
ಪರಿ-ಣಿ-ಮಿ-ಸಿ-ದ್ದರೆ
ಪರಿ-ಣಿ-ಸಿತ್ತು
ಪರಿ-ತ-ಪಿ-ಸಿತು
ಪರಿ-ತ-ಪಿ-ಸು-ವುದನ್ನು
ಪರಿ-ತ್ಯಾಗ
ಪರಿ-ತ್ಯಾ-ಗಿ-ಗ-ಳಾದ
ಪರಿ-ತ್ಯಾ-ಗಿ-ಗ-ಳೆ-ನಿ-ಸಿ-ಕೊಂಡ
ಪರಿ-ತ್ಯಾ-ಗಿ-ಯಾದ
ಪರಿ-ಧಿಯು
ಪರಿ-ಪರಿ
ಪರಿ-ಪ-ರಿ-ಯಾಗಿ
ಪರಿ-ಪ-ರಿ-ಯಾದ
ಪರಿ-ಪಾಠ
ಪರಿ-ಪಾ-ಲನೆ
ಪರಿ-ಪಾ-ಲಿ-ಸು-ತ್ತಿ-ದ್ದರು
ಪರಿ-ಪಾ-ಲಿ-ಸು-ವ-ವರು
ಪರಿ-ಪಾ-ಲಿ-ಸು-ವು-ದಷ್ಟೇ
ಪರಿ-ಪೂರ್ಣ
ಪರಿ-ಪೂ-ರ್ಣ-ಗೊಂಡ
ಪರಿ-ಪೂ-ರ್ಣತೆ
ಪರಿ-ಪೂ-ರ್ಣ-ತೆಗೆ
ಪರಿ-ಪೂ-ರ್ಣ-ತೆ-ಯನ್ನು
ಪರಿ-ಪೂ-ರ್ಣ-ತೆ-ಯನ್ನೇ
ಪರಿ-ಪೂ-ರ್ಣ-ಪ್ರ-ವಾ-ಹ-ವನ್ನೇ
ಪರಿ-ಪೂ-ರ್ಣ-ರಾ-ಗಲು
ಪರಿ-ಪೂ-ರ್ಣ-ರಾ-ಗಿ-ಸಲು
ಪರಿ-ಪೂ-ರ್ಣ-ರಾ-ಗು-ವಂತೆ
ಪರಿ-ಪೂ-ರ್ಣರು
ಪರಿ-ಪೂ-ರ್ಣ-ವಾಗಿ
ಪರಿ-ಪೂ-ರ್ಣ-ವಾ-ಗಿದೆ
ಪರಿ-ಪೂ-ರ್ಣ-ವಾ-ಗಿದ್ದ
ಪರಿ-ಪೂ-ರ್ಣ-ವಾ-ಯಿ-ತೆಂದು
ಪರಿ-ಮಾ-ಣದ
ಪರಿ-ಮಾ-ಣ-ದ-ಲ್ಲಿವೆ
ಪರಿ-ಯನ್ನು
ಪರಿಯೇ
ಪರಿ-ವ-ರ್ತನೆ
ಪರಿ-ವ-ರ್ತ-ನೆ-ಯನ್ನು
ಪರಿ-ವ-ರ್ತಿ-ತ-ಳಾ-ಗಿದ್ದ
ಪರಿ-ವ-ರ್ತಿ-ತ-ವಾಗಿ
ಪರಿ-ವ-ರ್ತಿ-ತ-ವಾದ
ಪರಿ-ವ-ರ್ತಿ-ತ-ವಾ-ದಂತೆ
ಪರಿ-ವ-ರ್ತಿ-ಸ-ಬಲ್ಲ
ಪರಿ-ವ-ರ್ತಿ-ಸ-ಬೇಕು
ಪರಿ-ವ-ರ್ತಿ-ಸಲು
ಪರಿ-ವ-ರ್ತಿಸಿ
ಪರಿ-ವ-ರ್ತಿ-ಸುತ್ತ
ಪರಿ-ವ-ರ್ತಿ-ಸುತ್ತಿ
ಪರಿ-ವಾ-ರ-ದ-ವರು
ಪರಿ-ವೆ-ಯಿಲ್ಲ
ಪರಿ-ವೆ-ಯಿ-ಲ್ಲದೆ
ಪರಿ-ವೆಯೂ
ಪರಿ-ವೆಯೇ
ಪರಿ-ವ್ರ-ಜ-ನದ
ಪರಿ-ವ್ರಾ-ಜಕ
ಪರಿ-ವ್ರಾ-ಜ-ಕ
ಪರಿ-ವ್ರಾ-ಜ-ಕ-ರಾಗಿ
ಪರಿ-ವ್ರಾ-ಜ-ಕ-ರಾ-ಗಿ-ದ್ದಾ-ಗಲೇ
ಪರಿ-ಶೀ-ಲಿಸಿ
ಪರಿ-ಶೀ-ಲಿ-ಸು-ವುದು
ಪರಿ-ಶೀ-ಲಿ-ಸು-ವುದೇ
ಪರಿ-ಶುದ್ಧ
ಪರಿ-ಶು-ದ್ಧ-ಪ-ವಿ-ತ್ರ-ತೇ-ಜೋ-ಮಯ
ಪರಿ-ಶು-ದ್ಧತೆ
ಪರಿ-ಶು-ದ್ಧ-ತೆ-ಯನ್ನು
ಪರಿ-ಶು-ದ್ಧರೂ
ಪರಿ-ಶು-ದ್ಧವೂ
ಪರಿ-ಶು-ದ್ಧಾ-ತ್ಮರೂ
ಪರಿ-ಶ್ರಮ
ಪರಿ-ಶ್ರ-ಮ-ದಿಂದ
ಪರಿ-ಶ್ರ-ಮ-ದಿಂ-ದಾಗಿ
ಪರಿ-ಸರ
ಪರಿ-ಸ-ರ-ಪ-ರಿ-ಸ್ಥಿ-ತಿ-ಗಳಲ್ಲಿ
ಪರಿ-ಸ-ರ-ಪ-ರಿ-ಸ್ಥಿ-ತಿ-ಗಳು
ಪರಿ-ಸ-ರದ
ಪರಿ-ಸ-ರ-ದಲ್ಲಿ
ಪರಿ-ಸ-ರ-ವನ್ನು
ಪರಿ-ಸ-ರವೇ
ಪರಿ-ಸ-ರ-ವೊಂ-ದನ್ನು
ಪರಿ-ಸ್ಥಿತಿ
ಪರಿ-ಸ್ಥಿ-ತಿ-ಗಳನ್ನು
ಪರಿ-ಸ್ಥಿ-ತಿ-ಗ-ಳಲ್ಲೂ
ಪರಿ-ಸ್ಥಿ-ತಿ-ಗ-ಳಿಗೆ
ಪರಿ-ಸ್ಥಿ-ತಿ-ಗಳು
ಪರಿ-ಸ್ಥಿ-ತಿ-ಗಾಗಿ
ಪರಿ-ಸ್ಥಿ-ತಿ-ಗಿಂತ
ಪರಿ-ಸ್ಥಿ-ತಿಯ
ಪರಿ-ಸ್ಥಿ-ತಿ-ಯನ್ನು
ಪರಿ-ಸ್ಥಿ-ತಿ-ಯನ್ನೂ
ಪರಿ-ಸ್ಥಿ-ತಿ-ಯಲ್ಲಿ
ಪರಿ-ಸ್ಥಿ-ತಿ-ಯ-ಲ್ಲಿ-ದ್ದೀ-ಯೆಂದು
ಪರಿ-ಸ್ಥಿ-ತಿ-ಯಿಂದ
ಪರಿ-ಸ್ಥಿ-ತಿಯು
ಪರಿ-ಸ್ಥಿ-ತಿ-ಯೇನು
ಪರಿ-ಸ್ಥಿ-ತಿ-ಯೇನೂ
ಪರಿ-ಸ್ಥಿ-ತಿ-ಯೊಂ-ದಿಗೆ
ಪರಿ-ಸ್ಥಿ-ತಿ-ಯೊ-ದ-ಗ-ಬೇ-ಕಾ-ದರೆ
ಪರಿ-ಸ್ಥಿ-ತಿ-ಯೊ-ದ-ಗಿದೆ
ಪರಿ-ಹ-ರಿಸಿ
ಪರಿ-ಹ-ರಿ-ಸಿ-ಕೊಂಡ
ಪರಿ-ಹ-ರಿ-ಸಿ-ಕೊ-ಳ್ಳಲು
ಪರಿ-ಹ-ರಿ-ಸು-ತ್ತಿ-ದ್ದರು
ಪರಿ-ಹ-ರಿ-ಸುವ
ಪರಿ-ಹಾರ
ಪರಿ-ಹಾ-ರಕ್ಕೆ
ಪರಿ-ಹಾ-ರ-ಗಳನ್ನು
ಪರಿ-ಹಾ-ರ-ವನ್ನು
ಪರಿ-ಹಾ-ರ-ವನ್ನೂ
ಪರಿ-ಹಾ-ರ-ವನ್ನೇ
ಪರಿ-ಹಾ-ರ-ವಾ-ಗ-ಲಾ-ರಂ-ಭಿ-ಸಿ-ದುವು
ಪರಿ-ಹಾ-ರ-ವಾ-ಗಿ-ರ-ಲಿಲ್ಲ
ಪರಿ-ಹಾ-ರ-ವಾ-ಗು-ತ್ತವೆ
ಪರಿ-ಹಾ-ರ-ವಾ-ದಂತೆ
ಪರಿ-ಹಾ-ರ-ವಾ-ದುವು
ಪರಿ-ಹಾ-ರ-ವೇನು
ಪರಿ-ಹಾ-ರೋ-ಪಾಯ
ಪರಿ-ಹಾಸ
ಪರಿ-ಹಾ-ಸ್ಯ-ದಿಂದ
ಪರೀಕ್ಷಾ
ಪರೀ-ಕ್ಷಿ-ಸ-ಬೇ-ಕೆಂದು
ಪರೀ-ಕ್ಷಿ-ಸಲು
ಪರೀ-ಕ್ಷಿಸಿ
ಪರೀ-ಕ್ಷಿ-ಸಿ-ನೋ-ಡ-ಲೆಂದು
ಪರೀ-ಕ್ಷಿ-ಸು-ತ್ತಾನೆ
ಪರೀ-ಕ್ಷಿ-ಸುವ
ಪರೀ-ಕ್ಷಿ-ಸು-ವಂ-ತೆಯೇ
ಪರೀಕ್ಷೆ
ಪರೀ-ಕ್ಷೆ-ಗೊ-ಳಾ-ಗು-ತ್ತದೆ
ಪರೀ-ಕ್ಷೆ-ಮಾಡಿ
ಪರೀ-ಕ್ಷೆ-ಯಲ್ಲಿ
ಪರೀ-ಕ್ಷೆ-ಯಾ-ವನು
ಪರೀ-ಕ್ಷೆ-ಯಿಂದ
ಪರು-ಷ-ಮಣಿ
ಪರು-ಷೋ-ಕ್ತಿ-ಗಳ
ಪರೆ-ಗಳನ್ನೆಲ್ಲ
ಪರೋ
ಪರೋಕ್ಷ
ಪರೋ-ಕ್ಷ-ವಾಗಿ
ಪರೋ-ಪ-ಕಾ-ರವೇ
ಪರ್ಣಿ-ಯ-ವ-ರೆಗೆ
ಪರ್ಯಂತ
ಪರ್ಯ-ಟನೆ
ಪರ್ಯ-ವ-ಸಾ-ನ-ಗೊ-ಳ್ಳು-ವು-ದರ
ಪರ್ಯ-ವ-ಸಾ-ನ-ವಾ-ಗು-ತ್ತದೆ
ಪರ್ಯಾಯ
ಪರ್ವತ
ಪರ್ವ-ತಕ್ಕೆ
ಪರ್ವ-ತ-ಗಳ
ಪರ್ವ-ತ-ಗ-ಳನ್ನೇ
ಪರ್ವ-ತ-ಗ-ಳಷ್ಟು
ಪರ್ವ-ತ-ಗಳಿಂದ
ಪರ್ವ-ತದ
ಪರ್ವ-ತ-ನ್ನೇ-ರುವ
ಪರ್ವ-ತ-ಪ್ರ-ದೇ-ಶದ
ಪರ್ವ-ತ-ವಿದೆ
ಪರ್ವ-ತ-ಶಿ-ಖ-ರ-ಗಳಿಂದ
ಪರ್ವ-ತ-ಶಿ-ಖ-ರದ
ಪರ್ವ-ತ-ಶ್ರೇಣಿ
ಪರ್ವ-ತಾ-ರೋ-ಹ-ಣಕ್ಕೆ
ಪರ್ವ-ತಾ-ರೋ-ಹ-ಣದ
ಪರ್ವ-ತಾ-ರೋ-ಹಿ-ಗಳು
ಪರ್ವತೇ
ಪರ್ವ-ತೋ-ಪ-ಮ-ವಾಗಿ
ಪರ್ಶಿ-ಯನ್
ಪರ್ಸಿ
ಪರ್ಸಿಗೆ
ಪರ್ಸಿ-ಯಲ್ಲಿ
ಪರ್ಸಿ-ಯಿಂದ
ಪರ್ಸ್
ಪಲಾ-ಯನ
ಪಲಾ-ಯ-ನ-ದಿಂ-ದಲೋ
ಪಲ್ಪು
ಪಲ್ಪು-ರ-ವ-ರೊಂ-ದಿಗೆ
ಪಲ್ಲ-ಟ-ಗೊ-ಳಿ-ಸದ
ಪಲ್ಲವಿ
ಪಲ್ಲ-ವಿ-ಯನ್ನು
ಪಲ್ಲ-ವಿ-ಯನ್ನೇ
ಪಲ್ಲ-ವಿ-ಯಾ-ಗಿ-ತ್ತ-ಲ್ಲವೆ
ಪಳ-ಗಿದ
ಪಳ-ಗಿ-ದ-ವನೂ
ಪವ-ಹಾರಿ
ಪವಾಡ
ಪವಾ-ಡ-ಗ-ಳಂತೂ
ಪವಾ-ಡ-ಗಳನ್ನು
ಪವಾ-ಡ-ಗ-ಳ-ನ್ನೇ-ನಾ-ದರೂ
ಪವಾ-ಡ-ಗ-ಳಿಗೂ
ಪವಾ-ಡ-ಗಳು
ಪವಾ-ಡ-ಗ-ಳೆಂದು
ಪವಾ-ಡ-ಪು-ರು-ಷ-ರನ್ನು
ಪವಾ-ಡ-ವೆಂ-ದಲ್ಲ
ಪವಾ-ಡವೇ
ಪವಾ-ಡ-ಶ-ಕ್ತಿ-ಗಳ
ಪವಾ-ಹಾರಿ
ಪವಿತ್ರ
ಪವಿ-ತ್ರ-ತಮ
ಪವಿ-ತ್ರತೆ
ಪವಿ-ತ್ರ-ತೆಯ
ಪವಿ-ತ್ರ-ತೆ-ಯನ್ನು
ಪವಿ-ತ್ರ-ತೆಯೂ
ಪವಿ-ತ್ರ-ದಿನ
ಪವಿ-ತ್ರರು
ಪವಿ-ತ್ರ-ವಾ-ಗಿದೆ
ಪವಿ-ತ್ರ-ವಾದ
ಪವಿ-ತ್ರೀ-ಕ-ರಿ-ಸಿದ
ಪಶು-ಗ-ಳಂತೆ
ಪಶು-ಪ-ಕ್ಷಿ-ಗ-ಳೆಲ್ಲ
ಪಶು-ಪ-ತಿ-ಮತಂ
ಪಶು-ಸ-ದೃಶ
ಪಶ್ಚಾ-ತ್ತಾಪ
ಪಶ್ಚಾ-ತ್ತಾ-ಪ-ದಿಂದ
ಪಶ್ಚಿಮ
ಪಶ್ಚಿ-ಮಕ್ಕೆ
ಪಶ್ಚಿ-ಮದ
ಪಶ್ಚಿ-ಮ-ದತ್ತ
ಪಶ್ಚಿ-ಮ-ದಲ್ಲಿ
ಪಶ್ಚಿ-ಮ-ದೆ-ಡೆಗೆ
ಪಶ್ಚಿ-ಮ-ವೆ-ರಡು
ಪಾಂಡ-ವರ
ಪಾಂಡಿ-ಚೆ-ರಿಗೆ
ಪಾಂಡಿ-ಚೆ-ರಿಯ
ಪಾಂಡಿ-ಚೆ-ರಿ-ಯ-ಲ್ಲಿ-ದ್ದಾಗ
ಪಾಂಡಿ-ಚೆ-ರಿ-ಯಿಂದ
ಪಾಂಡಿತ್ಯ
ಪಾಂಡಿ-ತ್ಯಕ್ಕೂ
ಪಾಂಡಿ-ತ್ಯದ
ಪಾಂಡಿ-ತ್ಯ-ಪೂರ್ಣ
ಪಾಂಡಿ-ತ್ಯ-ವನ್ನು
ಪಾಂಡಿ-ತ್ಯ-ವನ್ನೂ
ಪಾಂಡಿ-ತ್ಯ-ವಿ-ದ್ದ-ವನು
ಪಾಂಡಿ-ತ್ಯ-ವಿ-ರುವ
ಪಾಂಡು-ಪೋ-ಲಿ-ನ-ಲ್ಲಿ-ರುವ
ಪಾಂಡು-ಪೋ-ಲಿ-ನ-ವರೆ-ಗಾ-ದರೂ
ಪಾಂಡು-ರಂಗ
ಪಾಂಡು-ರಂ-ಗನ
ಪಾಂಡ್ಯ
ಪಾಂಡ್ಯನ
ಪಾಕ-ಶಾ-ಲೆ-ಯಲ್ಲಿ
ಪಾಕ-ಶಾ-ಸ್ತ್ರ-ಜ್ಞ-ರಾ-ಗಿದ್ದ
ಪಾಕಿ-ಸ್ತಾ-ನ-ದ-ಲ್ಲಿ-ರುವ
ಪಾಕ್
ಪಾಕ್ಷಿಕ
ಪಾಚಿಯ
ಪಾಟರ್
ಪಾಠ
ಪಾಠ-ಗಳನ್ನು
ಪಾಠ-ಗಳನ್ನೆಲ್ಲ
ಪಾಠದ
ಪಾಠ-ವನ್ನು
ಪಾಠ-ವಾ-ಗಿತ್ತು
ಪಾಠ-ವೊಂ-ದನ್ನು
ಪಾಠ-ಶಾ-ಲೆಗೂ
ಪಾಡಿಗೆ
ಪಾಣ-ನಿಯ
ಪಾಣಿ-ನಿಯ
ಪಾತಂ-ಜಲ
ಪಾತಾಳ
ಪಾತಾ-ಳಕ್ಕೆ
ಪಾತ್ರ
ಪಾತ್ರ-ನಾ-ಗು-ವು-ದೆಂ-ದ-ರೇನು
ಪಾತ್ರ-ನಾದ
ಪಾತ್ರ-ರಾ-ಗ-ಬಲ್ಲ
ಪಾತ್ರ-ರಾ-ಗಿ-ದ್ದರು
ಪಾತ್ರ-ರಾ-ದರು
ಪಾತ್ರ-ಳಾ-ದಳು
ಪಾತ್ರ-ವನ್ನು
ಪಾತ್ರ-ವ-ಹಿ-ಸ-ಲಿ-ದ್ದರು
ಪಾತ್ರ-ವ-ಹಿ-ಸಿದ
ಪಾತ್ರ-ವಾ-ಗಿ-ರು-ವಂ-ತಹ
ಪಾತ್ರವೂ
ಪಾತ್ರ-ವೆಂ-ಥದು
ಪಾತ್ರೆ-ಪ-ಡಗ
ಪಾತ್ರೆ-ಗಳ
ಪಾತ್ರೆ-ಗಳನ್ನು
ಪಾತ್ರೆ-ಗಳನ್ನೂ
ಪಾತ್ರೆ-ಗಳನ್ನೆಲ್ಲ
ಪಾತ್ರೆ-ಗಳಲ್ಲಿ
ಪಾತ್ರೆಯೂ
ಪಾದ-ಕ-ರಾದ
ಪಾದಕ್ಕೆ
ಪಾದ-ಗಳನ್ನು
ಪಾದ-ಗಳಲ್ಲಿ
ಪಾದ-ಗ-ಳಿಗೆ
ಪಾದ-ಗಳು
ಪಾದ-ದಂ-ತೆಯೇ
ಪಾದ-ಪ-ದ್ಮ-ಗಳನ್ನು
ಪಾದ-ಪೂಜೆ
ಪಾದ-ರ-ಕ್ಷೆ-ಗಳನ್ನು
ಪಾದ-ರ-ಸ-ದಂ-ತಹ
ಪಾದ-ರಿಗೆ
ಪಾದಿ-ಸುವ
ಪಾದ್ರಿ
ಪಾದ್ರಿ-ಗಳ
ಪಾದ್ರಿ-ಗ-ಳಂ-ತೆಯೇ
ಪಾದ್ರಿ-ಗ-ಳ-ನನ್ನು
ಪಾದ್ರಿ-ಗಳನ್ನು
ಪಾದ್ರಿ-ಗ-ಳನ್ನೇ
ಪಾದ್ರಿ-ಗಳಲ್ಲಿ
ಪಾದ್ರಿ-ಗ-ಳಿಗೆ
ಪಾದ್ರಿ-ಗಳು
ಪಾದ್ರಿ-ಗಳೂ
ಪಾದ್ರಿ-ಗ-ಳೊಂ-ದಿ-ಗಿನ
ಪಾದ್ರಿಗೆ
ಪಾದ್ರಿ-ಯನ್ನು
ಪಾದ್ರಿ-ಯೊಬ್ಬ
ಪಾದ್ರಿ-ಯೊ-ಬ್ಬರು
ಪಾನ-ಮಾ-ಡಿಲ್ಲ
ಪಾಪ
ಪಾಪ-ಕಾ-ರ್ಯ-ವಲ್ಲ
ಪಾಪ-ಕೃ-ತ್ಯಕ್ಕೆ
ಪಾಪ-ಕೃ-ತ್ಯ-ಗಳನ್ನೂ
ಪಾಪ-ಕೃ-ತ್ಯ-ದಿಂ-ದಾಗಿ
ಪಾಪ-ದೃ-ಷ್ಟಿ-ಯಿಂದ
ಪಾಪ-ವನ್ನೂ
ಪಾಪ-ವಿ-ಮೋ-ಚ-ನೆಯ
ಪಾಪಿ
ಪಾಪಿ-ಗಳ
ಪಾಪಿ-ಗ-ಳಿ-ಗಾಗಿ
ಪಾಪಿ-ಗ-ಳಿಗೆ
ಪಾಪಿ-ಗಳು
ಪಾಪಿ-ಗಳೆ
ಪಾಪಿ-ಗ-ಳೆಂದು
ಪಾಪಿ-ಗ-ಳೆಂಬ
ಪಾಪಿ-ಯನ್ನೂ
ಪಾಪಿ-ಯ-ಲ್ಲಿ-ಎ-ಲ್ಲ-ರಲ್ಲೂ
ಪಾಪಿ-ಯೆ-ನ್ನ-ಬೇಡ
ಪಾಪಿ-ಯೆ-ನ್ನು-ವುದೇ
ಪಾಮರ್
ಪಾಮ-ರ್ರ-ವರ
ಪಾಮ-ರ್ರ-ವ-ರಿಗೆ
ಪಾಮ-ರ್ರ-ವರು
ಪಾಯ-ಕ್ಕಾಗಿ
ಪಾಯದ
ಪಾರ
ಪಾರಂ-ಗ-ತರು
ಪಾರ-ತಂ-ತ್ರ್ಯದ
ಪಾರ-ಮಾ-ರ್ಥಿಕ
ಪಾರವೇ
ಪಾರಸ
ಪಾರಾ-ಗ-ಬೇ-ಕಾ-ಗು-ತ್ತದೆ
ಪಾರಾ-ಗಲು
ಪಾರಾ-ಗಿ-ದ್ದರು
ಪಾರಾ-ಗು-ತ್ತಾರೆ
ಪಾರಾ-ದಂ-ತೆಯೇ
ಪಾರಾ-ದ-ದ್ದೊಂದು
ಪಾರಾ-ದ-ರಲ್ಲ
ಪಾರಾದೆ
ಪಾರಾ-ಯಣ
ಪಾರಿ-ಜಾತ
ಪಾರು
ಪಾರುಗೈ
ಪಾರು-ಮಾ-ಡಲಿ
ಪಾರು-ಮಾ-ಡಿದ
ಪಾರ್ಥ
ಪಾರ್ಥ-ಸಾ-ರ-ಥಿಯೇ
ಪಾರ್ಲಿ-ಮೆಂ-ಟಿನ
ಪಾಲನು
ಪಾಲ-ನೆ-ಯಿ-ರು-ತ್ತ-ದೆಯೋ
ಪಾಲರು
ಪಾಲಿ-ಗಂತೂ
ಪಾಲಿ-ಗಿನ್ನೂ
ಪಾಲಿಗೂ
ಪಾಲಿಗೆ
ಪಾಲಿ-ಗೇನೋ
ಪಾಲಿ-ಗೊಂದು
ಪಾಲಿ-ತಾನ
ಪಾಲಿ-ತಾ-ನ-ದಲ್ಲಿ
ಪಾಲಿನ
ಪಾಲಿ-ನ-ದಾ-ಯಿತು
ಪಾಲಿಸಿ
ಪಾಲಿ-ಸಿ-ಕೊಂಡು
ಪಾಲಿ-ಸು-ತ್ತೇನೆ
ಪಾಲಿ-ಸು-ವಲ್ಲಿ
ಪಾಲಿ-ಸು-ವು-ದ-ರಲ್ಲೇ
ಪಾಲು
ಪಾಲ್
ಪಾಲ್ಗೊಂ-ಡರು
ಪಾಲ್ಗೊಂ-ಡ-ರೆಂದೂ
ಪಾಲ್ಗೊಂ-ಡ-ವನು
ಪಾಲ್ಗೊಂ-ಡಾಗ
ಪಾಲ್ಗೊಂ-ಡಿತು
ಪಾಲ್ಗೊಂ-ಡಿದ್ದ
ಪಾಲ್ಗೊ-ಳ್ಳ-ದೆಯೂ
ಪಾಲ್ಗೊ-ಳ್ಳುವ
ಪಾಲ್ಗೊ-ಳ್ಳು-ವಂತೆ
ಪಾಲ್ಗೊ-ಳ್ಳು-ವ-ರೆಂದು
ಪಾಳೀ
ಪಾಳು-ಬಿದ್ದ
ಪಾವ-ನ-ಕಾ-ರಿ-ಯಾ-ದದ್ದು
ಪಾವ-ನ-ಗೈದ
ಪಾವಿತ್ರ್ಯ
ಪಾವಿ-ತ್ರ್ಯದ
ಪಾವಿ-ತ್ರ್ಯ-ದಿಂದ
ಪಾವಿ-ತ್ರ್ಯ-ವನ್ನು
ಪಾಶ್ಚಾತ್ಯ
ಪಾಶ್ಚಾ-ತ್ಯಈ
ಪಾಶ್ಚಾ-ತ್ಯನು
ಪಾಶ್ಚಾ-ತ್ಯರ
ಪಾಶ್ಚಾ-ತ್ಯ-ರನ್ನು
ಪಾಶ್ಚಾ-ತ್ಯ-ರಲ್ಲಿ
ಪಾಶ್ಚಾ-ತ್ಯ-ರಾ-ಗಿದ್ದು
ಪಾಶ್ಚಾ-ತ್ಯ-ರಾ-ಷ್ಟ್ರ-ಗಳು
ಪಾಶ್ಚಾ-ತ್ಯ-ರಿಗೂ
ಪಾಶ್ಚಾ-ತ್ಯ-ರಿಗೆ
ಪಾಶ್ಚಾ-ತ್ಯರು
ಪಾಶ್ಚಾ-ತ್ಯ-ರು-ಅ-ದ-ರಲ್ಲೂ
ಪಾಶ್ಚಾ-ತ್ಯರೂ
ಪಾಶ್ಚಾ-ತ್ಯ-ರೆಂ-ದರೆ
ಪಾಶ್ಚಾ-ತ್ಯ-ಶಿ-ಷ್ಯ-ರಿಗೆ
ಪಾಶ್ಯಾತ್ಯ
ಪಾಷಂ-ಡ-ತ-ನವೇ
ಪಾಷಂ-ಡ-ವಾ-ದ-ವನ್ನು
ಪಾಸು
ಪಿ
ಪಿಂಗಾಣಿ
ಪಿಂಡಾಂ-ಡ-ವಾಗಿ
ಪಿಕಾ-ಡಿ-ಲಿಯ
ಪಿಕ್ವಿಕ್
ಪಿಟೀಲು
ಪಿಡು-ಗನ್ನು
ಪಿತು
ಪಿತೂರಿ
ಪಿತ್ರಾ-ರ್ಜಿತ
ಪಿನ
ಪಿಯ-ವ-ರಾದ
ಪಿಳ್ಳೆ
ಪಿಳ್ಳೆ-ಯ-ವ-ರಿಗೆ
ಪಿಳ್ಳೆ-ಯ-ವರು
ಪಿಶಾ-ಚ-ನ-ರ್ತ-ನವೋ
ಪಿಶಾ-ಚಿ-ಗಳನ್ನು
ಪಿಶಾ-ಚಿ-ಗಳು
ಪಿಶಾ-ಚಿ-ಯನ್ನು
ಪಿಸ್ಕಾ-ಟಾಕ್ವಾ
ಪೀಠ-ದಲ್ಲಿ
ಪೀಠ-ವನ್ನು
ಪೀಠಿ-ಕೆ-ಯೊಂ-ದಿಗೆ
ಪೀಠೋ-ಪ-ಕ-ರ-ಣ-ಗಳು
ಪೀಡಿತ
ಪೀಡಿ-ತ-ರಾ-ಗಿದ್ದ
ಪೀಡಿ-ತ-ವ-ಲ್ಲದ
ಪೀಡೆ-ಗ-ಳಿಗೂ
ಪೀತ-ದಾರು
ಪೀತ-ವ-ರ್ಣದ
ಪೀತಾ-ರುಣ
ಪೀಪಾಯಿ
ಪೀಪಿ-ಯಂ-ತಹ
ಪುಂಡರ
ಪುಕ್ಕಟೆ
ಪುಗ್ವೇ-ದದ
ಪುಜುತ್ವ
ಪುಜು-ತ್ವ-ವನ್ನು
ಪುಟ
ಪುಟ-ಗ-ಟ್ಟಲೆ
ಪುಟ-ಗಳ
ಪುಟ-ಗಳನ್ನು
ಪುಟ-ಗಳಲ್ಲಿ
ಪುಟಿ-ದೆದ್ದು
ಪುಟಿ-ದೇ-ಳು-ತ್ತಿ-ದೆ-ಯೆಂ-ದರೆ
ಪುಟಿ-ದೇ-ಳು-ವುದನ್ನು
ಪುಟ್ಟ
ಪುಟ್ಟ-ದೊಂದು
ಪುಟ್ಟ-ಬಾಲೆ
ಪುಡಿ
ಪುಡಿ-ಗೈ-ಯು-ತ್ತಿ-ರುವ
ಪುಡಿ-ಗೈ-ಯೋಣ
ಪುಡಿ-ಪುಡಿ
ಪುಣಿ-ಯಾ-ಗಿ-ರು-ತ್ತದೆ
ಪುಣ್ಯ
ಪುಣ್ಯ-ಕ್ಷೇ-ತ್ರ-ಗ-ಳಲ್ಲೂ
ಪುಣ್ಯ-ಕ್ಷೇ-ತ್ರ-ಗ-ಳಿಗೆ
ಪುಣ್ಯದ
ಪುಣ್ಯ-ದಿನ
ಪುಣ್ಯ-ಭೂಮಿ
ಪುಣ್ಯ-ವಂ-ತರ
ಪುಣ್ಯ-ವಂ-ತ-ರಿಗೆ
ಪುಣ್ಯ-ವಿ-ಶೇಷ
ಪುತ-ವೆಂದು
ಪುತು
ಪುತು-ವಿನ
ಪುತು-ವಿ-ನಲ್ಲಿ
ಪುತ್ರ
ಪುತ್ರ-ನಂತೆ
ಪುತ್ರ-ನನ್ನು
ಪುತ್ರ-ನಾಗಿ
ಪುತ್ರ-ನಿ-ಗಿದ್ದ
ಪುತ್ರ-ನಿಗೆ
ಪುತ್ರನೂ
ಪುತ್ರ-ನೆಂ-ಬಂತೆ
ಪುತ್ರನೋ
ಪುತ್ರ-ನೋ-ರ್ವ-ನನ್ನು
ಪುತ್ರ-ರಾದ
ಪುತ್ರ-ರಿರಾ
ಪುತ್ರರು
ಪುತ್ರರೆ
ಪುತ್ರ-ರೆಂದು
ಪುತ್ರ-ಸಂ-ತಾ-ನ-ವಿ-ರ-ಲಿಲ್ಲ
ಪುತ್ರಾಃ
ಪುತ್ರಿ
ಪುತ್ರಿಯ
ಪುತ್ರಿ-ಯ-ರಾದ
ಪುನಃ
ಪುನಃ-ಸ್ಥಾ-ಪಿ-ಸು-ವ-ವ-ರೆಗೂ
ಪುನ-ರಾ-ರಂ-ಭಿ-ಸಿ-ದರು
ಪುನ-ರಾ-ರಂ-ಭಿ-ಸು-ತ್ತಿ-ದ್ದರು
ಪುನ-ರಾ-ರಂ-ಭಿ-ಸುವ
ಪುನ-ರಾ-ವ-ರ್ತನೆ
ಪುನ-ರಾ-ವ-ರ್ತ-ನೆ-ಗೊ-ಳ್ಳ-ಲಿಲ್ಲ
ಪುನ-ರಾ-ವ-ರ್ತ-ನೆ-ಯಾ-ಗು-ವಂತೆ
ಪುನ-ರಾ-ವಿ-ಷ್ಕಾ-ರ-ಗ-ಳಷ್ಟೆ
ಪುನರು
ಪುನ-ರು-ಜ್ಜೀ-ವನ
ಪುನ-ರು-ತ್ಥಾನ
ಪುನ-ರು-ತ್ಥಾ-ನ-ಕ್ಕಾಗಿ
ಪುನ-ರು-ತ್ಥಾ-ನಕ್ಕೆ
ಪುನ-ರು-ತ್ಥಾ-ನದ
ಪುನ-ರು-ತ್ಥಾ-ನವು
ಪುನ-ರು-ದ್ಧಾರ
ಪುನ-ರು-ದ್ಧಾ-ರ-ಕ-ನಾಗಿ
ಪುನ-ರು-ದ್ಧಾ-ರ-ಕ್ಕಾಗಿ
ಪುನ-ರು-ದ್ಧಾ-ರಕ್ಕೆ
ಪುನ-ರು-ದ್ಧಾ-ರದ
ಪುನ-ರು-ದ್ಧಾ-ರವೇ
ಪುನರ್
ಪುನ-ರ್ಜನ್ಮ
ಪುನ-ರ್ಜ-ನ್ಮದ
ಪುನ-ರ್ಜಾ-ಗೃ-ತ-ವಾಗಿ
ಪುನ-ರ್ಜಾ-ಗೃ-ತಿಯ
ಪುನ-ರ್ಜಾ-ಗೃ-ತಿ-ಯಿಂದ
ಪುನ-ರ್ನಿ-ರೂ-ಪಿ-ಸ-ಬೇ-ಕಾದ
ಪುನ-ರ್ನಿ-ರ್ಮಾ-ಣದ
ಪುನ-ರ್ನಿ-ರ್ಮಾ-ಣ-ವಾ-ಗ-ಬೇ-ಕೆಂ-ದರೆ
ಪುನ-ರ್ಮು-ದ್ರಿತ
ಪುನ-ರ್ಮು-ದ್ರಿ-ಸ-ಲಾ-ರಂ-ಭಿ-ಸಿ-ದುವು
ಪುನ-ರ್ಮೌ-ಲ್ಯೀ-ಕ-ರ-ಣ-ಕ್ಕೊ-ಳ-ಪ-ಡಿಸಿ
ಪುನ-ರ್ವ್ಯ-ವ-ಸ್ಥೆ-ಗೊ-ಳಿಸಿ
ಪುನ-ಶ್ಚೇ-ತನ
ಪುನ-ಶ್ಚೇ-ತ-ನ-ಗೊ-ಳಿ-ಸಲು
ಪುನ-ಸ್ಸಂ-ಸ್ಥಾ-ಪ-ನೆ-ಯಲ್ಲೇ
ಪುನಾ-ರ-ವ-ರ್ತ-ನೆ-ಗೊ-ಳಿಸು
ಪುನೀತ
ಪುನೀ-ತ-ಗೊಂ-ಡದ್ದು
ಪುನೀ-ತ-ವಾದ
ಪುರ-ಭ-ವ-ನ-ದಲ್ಲಿ
ಪುರ-ಸ್ಕ-ರಿ-ಸಿ-ಯಾರೆ
ಪುರಾಣ
ಪುರಾ-ಣ-ಗಳ
ಪುರಾ-ಣ-ಗಳನ್ನು
ಪುರಾ-ಣ-ಗಳಲ್ಲಿ
ಪುರಾ-ಣ-ಗಳಿಂದ
ಪುರಾ-ಣ-ಗಳು
ಪುರಾ-ಣದ
ಪುರಾ-ಣ-ದಲ್ಲಿ
ಪುರಾ-ಣ-ಪ್ರ-ಸಿದ್ಧ
ಪುರಾ-ತನ
ಪುರಾ-ತ-ನವೂ
ಪುರಾ-ವೆಯೂ
ಪುರೀ
ಪುರೀ-ಕ್ಷೇ-ತ್ರದ
ಪುರುಷ
ಪುರುಷಂ
ಪುರು-ಷತ್ವ
ಪುರು-ಷನ
ಪುರು-ಷ-ನನ್ನು
ಪುರು-ಷರು
ಪುರು-ಷ-ರು-ನ-ಮಗೆ
ಪುರು-ಷ-ಸಿಂಹ
ಪುರು-ಷ-ಸಿಂ-ಹ-ರ-ನ್ನಾ-ಗಿ-ಸ-ಬಲ್ಲ
ಪುರು-ಷ-ಸಿಂ-ಹ-ರಾಗಿ
ಪುರು-ಷಾ-ರ್ಥ-ಕ್ಕಾ-ಗಿ-ಯೇನು
ಪುರು-ಷೋ-ತ್ತ-ಮಾ-ನಂದ
ಪುರು-ಷೋ-ತ್ತ-ಮಾ-ನಂ-ದ-ರಿಂದ
ಪುರೋ-ಗಾ-ಮಿ-ಗ-ಳಿಗೂ
ಪುರೋ-ಗಾ-ಮಿಯೂ
ಪುರೋ-ಗಾಮೀ
ಪುರೋ-ಭಿ-ವೃ-ದ್ಧಿ-ಗಾಗಿ
ಪುರೋ-ಹಿ-ತರು
ಪುರೋ-ಹಿ-ತ-ವ-ರ್ಗ-ದ-ವರು
ಪುರೋ-ಹಿ-ತ-ಶಾ-ಹಿಯ
ಪುಳ-ಕಿ-ತ-ರಾ-ದರು
ಪುಷಿ
ಪುಷಿ-ಮು-ನಿ-ಗಳ
ಪುಷಿ-ಗ-ಣ್ಣಿಗೆ
ಪುಷಿ-ಗಳ
ಪುಷಿ-ಗ-ಳಿಗೆ
ಪುಷಿ-ಗಳು
ಪುಷಿ-ಗ-ಳೆಲ್ಲ
ಪುಷಿ-ಪ-ರಂ-ಪ-ರೆ-ಯನ್ನೂ
ಪುಷಿ-ಮುನಿ
ಪುಷಿ-ಮು-ನಿ-ಗಳ
ಪುಷಿ-ಮು-ನಿ-ಗಳು
ಪುಷಿಯು
ಪುಷಿಯೇ
ಪುಷ್ಕ-ರ-ತೀ-ರ್ಥ-ದ-ಲ್ಲಿ-ರುವ
ಪುಷ್ಪ-ಗಳನ್ನು
ಪುಷ್ಪ-ಗಳು
ಪುಷ್ಪ-ವೊಂದು
ಪುಷ್ಪಾಂ-ಜ-ಲಿ-ಯನ್ನು
ಪುಷ್ಯಾ-ಶ್ರ-ಮ-ಗಳ
ಪುಸು-ಕಲು
ಪುಸ್ತಕ
ಪುಸ್ತ-ಕ-ಗೌನು
ಪುಸ್ತ-ಕ-ಗಳ
ಪುಸ್ತ-ಕ-ಗಳನ್ನು
ಪುಸ್ತ-ಕ-ಗಳನ್ನೂ
ಪುಸ್ತ-ಕ-ಗಳನ್ನೆಲ್ಲ
ಪುಸ್ತ-ಕ-ಗಳಲ್ಲಿ
ಪುಸ್ತ-ಕ-ಗ-ಳಲ್ಲೂ
ಪುಸ್ತ-ಕ-ಗ-ಳಿಂ-ದೇನೂ
ಪುಸ್ತ-ಕ-ಗ-ಳಿಗೆ
ಪುಸ್ತ-ಕ-ಗಳು
ಪುಸ್ತ-ಕ-ಗ-ಳು-ಎ-ಲ್ಲ-ವನ್ನೂ
ಪುಸ್ತ-ಕ-ಗಳೂ
ಪುಸ್ತ-ಕದ
ಪುಸ್ತ-ಕ-ದಂತೆ
ಪುಸ್ತ-ಕ-ದಲ್ಲಿ
ಪುಸ್ತ-ಕ-ದಲ್ಲೇ
ಪುಸ್ತ-ಕ-ದಿಂದ
ಪುಸ್ತ-ಕ-ರೂ-ಪ-ದಲ್ಲಿ
ಪುಸ್ತ-ಕ-ವನ್ನು
ಪುಸ್ತ-ಕವೂ
ಪುಸ್ತ-ಕವೇ
ಪುಸ್ತ-ಕ-ವೊಂ-ದನ್ನು
ಪುಸ್ತಿ-ಕೆಗೆ
ಪುಸ್ತಿ-ಕೆಯ
ಪೂಜಾ
ಪೂಜಾ-ಕ್ರ-ಮ-ಗಳನ್ನೂ
ಪೂಜಾ-ದಿ-ಗ-ಳೆಲ್ಲ
ಪೂಜಿ-ಸ-ಬಾ-ರದು
ಪೂಜಿ-ಸಲು
ಪೂಜಿ-ಸ-ಲ್ಪ-ಟ್ಟಿ-ದ್ದಾರೆ
ಪೂಜಿಸಿ
ಪೂಜಿ-ಸಿದ್ದು
ಪೂಜಿ-ಸು-ತ್ತಾರೆ
ಪೂಜಿ-ಸು-ತ್ತಿ-ದ್ದಾರೆ
ಪೂಜಿ-ಸು-ತ್ತೇನೆ
ಪೂಜಿ-ಸು-ತ್ತೇ-ನೆಯೋ
ಪೂಜಿ-ಸುವ
ಪೂಜಿ-ಸು-ವು-ದಿಲ್ಲ
ಪೂಜಿ-ಸು-ವುದು
ಪೂಜೆ
ಪೂಜೆ-ಗೀ-ಜೆಯ
ಪೂಜೆ-ಗೀ-ರ್ತಿ-ಪೂ-ಜೆ-ಯೆಲ್ಲ
ಪೂಜೆ-ಗಾಗಿ
ಪೂಜೆಯ
ಪೂಜೆ-ಯನ್ನು
ಪೂಜೆಯು
ಪೂಜೆ-ಯೆಂ-ದರೆ
ಪೂಜ್ಯ
ಪೂಜ್ಯ-ಗೌ-ರವ
ಪೂಜ್ಯ-ತೆ-ಗೌ-ರ-ವ-ಗ-ಳಿಗೆ
ಪೂಜ್ಯ-ಭಾ-ವ-ದಿಂದ
ಪೂಜ್ಯ-ಭಾ-ವ-ವ-ನ್ನಿ-ಟ್ಟು-ಕೊಂ-ಡಿದ್ದ
ಪೂಜ್ಯ-ವಾದ
ಪೂನಾ
ಪೂನಾಕ್ಕೆ
ಪೂನಾಗೆ
ಪೂನಾದ
ಪೂನಾ-ದಲ್ಲಿ
ಪೂನಾ-ದಿಂದ
ಪೂರಕ
ಪೂರ-ಕ-ಪ್ರೇ-ರಕ
ಪೂರ-ಕ-ವಾದ
ಪೂರಿ-ಪಲ್ಯ
ಪೂರಿ-ಗಳು
ಪೂರೈ-ಸ-ಬಂದ
ಪೂರೈ-ಸ-ಬೇ-ಕಾ-ಗಿ-ದೆ-ಯೆಂದು
ಪೂರೈ-ಸಲು
ಪೂರೈ-ಸಿ-ಕೊಟ್ಟು
ಪೂರೈ-ಸಿ-ಕೊ-ಳ್ಳಲು
ಪೂರೈ-ಸಿ-ದರು
ಪೂರೈ-ಸು-ತ್ತಿ-ದ್ದರು
ಪೂರೈ-ಸು-ವಂ-ತಾ-ದದ್ದು
ಪೂರೈ-ಸು-ವು-ದ-ಕ್ಕಾಗಿ
ಪೂರ್ಣ
ಪೂರ್ಣ-ಗೊಂ-ಡಿತು
ಪೂರ್ಣ-ಗೊ-ಳಿಸ
ಪೂರ್ಣ-ಗೊ-ಳಿ-ಸಲಿ
ಪೂರ್ಣ-ಗೊ-ಳಿ-ಸಲು
ಪೂರ್ಣ-ಗೊ-ಳಿ-ಸು-ವ-ವರು
ಪೂರ್ಣ-ಗೊ-ಳಿ-ಸು-ವು-ದಾಗಿ
ಪೂರ್ಣ-ಚಂದ್ರ
ಪೂರ್ಣ-ಪಾ-ಠ-ವನ್ನು
ಪೂರ್ಣ-ಪ್ರ-ಮಾ-ಣ-ದಲ್ಲಿ
ಪೂರ್ಣ-ರ-ಭ-ಸ-ದಿಂದ
ಪೂರ್ಣ-ವಾಗಿ
ಪೂರ್ಣ-ವಾ-ಗಿ-ದ್ದುವು
ಪೂರ್ಣ-ವಾ-ಗು-ವಂ-ತಿ-ರ-ಲಿಲ್ಲ
ಪೂರ್ಣ-ವಾದ
ಪೂರ್ಣ-ವಾ-ದ-ವು-ಗ-ಳ-ಲ್ಲೊಂದು
ಪೂರ್ಣವೂ
ಪೂರ್ತಿ
ಪೂರ್ತಿ-ಯಾಗಿ
ಪೂರ್ವ
ಪೂರ್ವ-ಕ-ವಾಗಿ
ಪೂರ್ವ-ಕ-ವಾದ
ಪೂರ್ವ-ಕಾ-ಲದ
ಪೂರ್ವ-ಗ್ರಹ
ಪೂರ್ವ-ಗ್ರ-ಹದ
ಪೂರ್ವ-ಗ್ರ-ಹ-ದಿಂ-ದಲೂ
ಪೂರ್ವ-ಗ್ರ-ಹ-ಪೀ-ಡಿತ
ಪೂರ್ವ-ಜ-ರನ್ನು
ಪೂರ್ವ-ಜ-ರ-ನ್ನೆಲ್ಲ
ಪೂರ್ವ-ಜ-ರಿ-ಗಾ-ಗಲಿ
ಪೂರ್ವ-ತ-ಯಾ-ರಿ-ಯನ್ನು
ಪೂರ್ವ-ತ-ಯಾ-ರಿ-ಯಿ-ಲ್ಲದೆ
ಪೂರ್ವದ
ಪೂರ್ವ-ಪ-ರಿ-ಚಯ
ಪೂರ್ವ-ವಾದ
ಪೂರ್ವ-ವೃ-ತಾಂ-ತ-ಗಳ
ಪೂರ್ವ-ಷ-ರ-ತ್ತು-ಗಳ
ಪೂರ್ವಾ
ಪೂರ್ವಾ-ಗ್ರಹ
ಪೂರ್ವಾ-ಗ್ರ-ಹ-ಗಳಿಂದ
ಪೂರ್ವಾ-ಗ್ರ-ಹ-ಪೀ-ಡಿ-ತ-ವಾ-ಗು-ವಂತೆ
ಪೂರ್ವಾ-ಗ್ರ-ಹ-ವನ್ನು
ಪೂರ್ವಾ-ಗ್ರ-ಹವೇ
ಪೂರ್ವಾ-ಪ-ರ-ಗಳ
ಪೂರ್ವಾ-ಪ-ರ-ಗ-ಳಾ-ಗಲಿ
ಪೂರ್ವಾ-ಪ-ರ-ಗ-ಳಾ-ವುವೂ
ಪೂರ್ವಾ-ಭಿ-ಮು-ಖ-ವಾಗಿ
ಪೂರ್ವಾ-ಶ್ರ-ಮದ
ಪೂರ್ವಿ-ಕರ
ಪೂರ್ವಿ-ಕ-ರಂತೆ
ಪೂರ್ವಿ-ಕರೂ
ಪೃಥ್ವಿ-ಯನ್ನು
ಪೆಚ್ಚಾಗಿ
ಪೆಟ್ಟನ್ನು
ಪೆಟ್ಟಾ-ಗ-ಲಿಲ್ಲ
ಪೆಟ್ಟಿಗೆ
ಪೆಟ್ಟಿ-ಗೆ-ಗಳ
ಪೆಟ್ಟಿ-ಗೆ-ಗಳನ್ನು
ಪೆಟ್ಟಿ-ಗೆ-ಯನ್ನೂ
ಪೆಟ್ಟಿನ
ಪೆಟ್ಟು
ಪೆಟ್ಟು-ಕೊಟ್ಟು
ಪೆಟ್ರೋ-ಲಿಯಂ
ಪೆನಾಂ-ಗ್ನಲ್ಲಿ
ಪೆನಾಂ-ಗ್ನಿಂದ
ಪೆನಿ-ನ್ಸು-ಲಾರ್
ಪೆನ್ಸಿ-ಲಿ-ನಿಂದ
ಪೆರ-ಮಾ-ಳ-ರಿಗೆ
ಪೆರ-ಮಾ-ಳರು
ಪೆರು
ಪೆರು-ಮಾ-ಳರ
ಪೆರು-ಮಾ-ಳ-ರಿಗೆ
ಪೆರು-ಮಾಳ್
ಪೆರು-ಮಾಳ್ಗೆ
ಪೆಸಿ-ಫಿಕ್
ಪೇಟ
ಪೇಟ-ಇ-ವು-ಗಳನ್ನೆಲ್ಲ
ಪೇಟ-ಗಳನ್ನು
ಪೇಟದ
ಪೇಟ-ವನ್ನೂ
ಪೇಟಾ-ಗ-ಳಿಗೆ
ಪೇಟೆ-ಯಲ್ಲಿ
ಪೇತ-ವಾದ
ಪೇದೆ-ಗಳು
ಪೇಪ-ರ್ಸ್
ಪೇರಿ-ಕೊಂ-ಡಿ-ರುವ
ಪೈಕಿ
ಪೈನ್
ಪೈಲ-ವಾ-ನನ
ಪೈಲ್ವಾನ್
ಪೈಶಾ-ಚಿಕ
ಪೊರೆ-ಯನ್ನು
ಪೊಳ್ಳಾದ
ಪೊಳ್ಳು
ಪೊಳ್ಳು-ಗು-ಳ್ಳೆ-ಗ-ಳೆ-ಲ್ಲ-ವೂ
ಪೊಳ್ಳು-ಜ-ನ-ನ-ವು-ಮ-ರ-ಣವು
ಪೊಳ್ಳು-ತ-ನ-ವನ್ನು
ಪೊಳ್ಳು-ನಾ-ಮವು
ಪೊಳ್ಳು-ಮಾ-ತು-ಗಳನ್ನು
ಪೊಳ್ಳು-ವ್ಯ-ವ-ಹಾ-ರ-ಗಳನ್ನು
ಪೋನ್
ಪೋರ್
ಪೋರ್ಡಿ-ನಲ್ಲಿ
ಪೋರ್ಬಂ-ದ-ರಿಗೆ
ಪೋರ್ಬಂ-ದ-ರಿ-ನಲ್ಲಿ
ಪೋರ್ಬಂ-ದ-ರಿ-ನ-ಲ್ಲಿ-ದ್ದೀಯೆ
ಪೋರ್ಬಂ-ದ-ರಿ-ನಲ್ಲೇ
ಪೋರ್ಬಂ-ದರ್
ಪೋರ್ಬಂ-ದ-ರ್ನಲ್ಲಿ
ಪೋರ್ಬಂ-ದ-ರ್ನಲ್ಲೇ
ಪೋಲಿಷ್
ಪೋಲಿ-ಸನ
ಪೋಲಿ-ಸರ
ಪೋಲಿ-ಸರು
ಪೋಲಿ-ಸ-ರೊಂ-ದಿ-ಗಿನ
ಪೋಲಿಸ್
ಪೋಲೀ-ಸ-ನಿಗೆ
ಪೋಲೀ-ಸ-ರಿಗೆ
ಪೋಲೀ-ಸರು
ಪೋಲೀಸ್
ಪೋಲು
ಪೋಷ-ಕ-ಳೆಂದು
ಪೋಷಾ-ಕಿ-ನಿಂ-ದಲೂ
ಪೋಷಾ-ಕು-ಗಳನ್ನು
ಪೋಷಿ-ಸಿ-ಕೊಂ-ಡು-ಬ-ರಲು
ಪೋಷಿಸು
ಪೋಹ-ಗಳು
ಪೌಂಡು-ಗ-ಳಷ್ಟು
ಪೌಂಡು-ಗಳು
ಪೌರಾ-ಣಿಕ
ಪೌರುಷ
ಪೌರು-ಷ-ಪೂರ್ಣ
ಪೌರು-ಷ-ವಂ-ತನೂ
ಪೌರು-ಷ-ವಂ-ತ-ರಾಗಿ
ಪೌರು-ಷ-ವನ್ನು
ಪೌರು-ಷ-ವಿ-ರು-ವು-ದಿ-ಲ್ಲವೋ
ಪೌರು-ಷವೇ
ಪೌರ್ವಾತ್ಯ
ಪೌರ್ವಾ-ತ್ಯ-ಧಾ-ರ್ಮಿಕ
ಪೌರ್ವಾ-ತ್ಯರ
ಪೌರ್ವಾ-ತ್ಯ-ರಾದ
ಪೌರ್ವಾ-ತ್ಯರು
ಪೌರ್ವಾ-ತ್ಯ-ಶಾ-ಸ್ತ್ರ-ವಿ-ಶಾ-ರ-ದ-ರಾ-ಗಿದ್ದು
ಪ್ಯಾರಿ-ಲಾಲ್
ಪ್ಯಾರಿಸಿ
ಪ್ಯಾರಿ-ಸಿಗೆ
ಪ್ಯಾರಿ-ಸಿನ
ಪ್ಯಾರಿ-ಸಿ-ನಲ್ಲಿ
ಪ್ಯಾರಿ-ಸಿ-ನಿಂದ
ಪ್ಯಾರಿಸ್
ಪ್ಯಾರಿ-ಸ್ಸನ್ನು
ಪ್ಯಾರಿ-ಸ್ಸಿ-ನಲ್ಲಿ
ಪ್ಯಾರೀ
ಪ್ಯಾರೀ-ಮೋ-ಹನ್
ಪ್ರಕಟ
ಪ್ರಕ-ಟ-ಗೊಂಡ
ಪ್ರಕ-ಟ-ಗೊಂ-ಡಿದ್ದ
ಪ್ರಕ-ಟ-ಗೊಂ-ಡಿ-ದ್ದುವು
ಪ್ರಕ-ಟ-ಗೊಂ-ಡಿ-ರು-ತ್ತದೆ
ಪ್ರಕ-ಟ-ಗೊಂ-ಡಿ-ರುವ
ಪ್ರಕ-ಟ-ಗೊಂಡು
ಪ್ರಕ-ಟ-ಗೊಂ-ಡು-ವು-ನ್ಯೂ-ಯಾ-ರ್ಕ್
ಪ್ರಕ-ಟ-ಗೊ-ಳಿ-ಸ-ಬ-ಲ್ಲುದು
ಪ್ರಕ-ಟ-ಗೊ-ಳ್ಳು-ತ್ತಿದ್ದ
ಪ್ರಕ-ಟ-ಗೊ-ಳ್ಳು-ತ್ತಿದ್ದು
ಪ್ರಕ-ಟಣೆ
ಪ್ರಕ-ಟ-ಣೆ-ಗಾಗಿ
ಪ್ರಕ-ಟ-ಣೆಗೆ
ಪ್ರಕ-ಟ-ಣೆಯ
ಪ್ರಕ-ಟ-ಪ-ಡಿ-ಸಿ-ದರು
ಪ್ರಕ-ಟ-ವಾ-ಗ-ತೊ-ಡ-ಗಿ-ದುವು
ಪ್ರಕ-ಟ-ವಾ-ಗ-ಬಲ್ಲ
ಪ್ರಕ-ಟ-ವಾ-ಗ-ಬಾ-ರದು
ಪ್ರಕ-ಟ-ವಾ-ಗ-ಲಿಲ್ಲ
ಪ್ರಕ-ಟ-ವಾಗಿ
ಪ್ರಕ-ಟ-ವಾ-ಗಿತ್ತು
ಪ್ರಕ-ಟ-ವಾ-ಗಿದ್ದ
ಪ್ರಕ-ಟ-ವಾ-ಗಿವೆ
ಪ್ರಕ-ಟ-ವಾಗು
ಪ್ರಕ-ಟ-ವಾ-ಗು-ತ್ತಿತ್ತು
ಪ್ರಕ-ಟ-ವಾ-ಗು-ತ್ತಿ-ದ್ದುವು
ಪ್ರಕ-ಟ-ವಾ-ಗು-ತ್ತಿ-ರು-ವುದನ್ನು
ಪ್ರಕ-ಟ-ವಾ-ಗುವ
ಪ್ರಕ-ಟ-ವಾ-ಗು-ವು-ದ-ರ-ಲ್ಲಿತ್ತು
ಪ್ರಕ-ಟ-ವಾದ
ಪ್ರಕ-ಟ-ವಾ-ದಾಗ
ಪ್ರಕ-ಟ-ವಾ-ದುವು
ಪ್ರಕ-ಟ-ವಾ-ಯಿತು
ಪ್ರಕಟಿ
ಪ್ರಕ-ಟಿಸ
ಪ್ರಕ-ಟಿ-ಸ-ಗೊ-ಡ-ಲಿಲ್ಲ
ಪ್ರಕ-ಟಿ-ಸ-ಬಾ-ರದು
ಪ್ರಕ-ಟಿ-ಸ-ಬಾ-ರ-ದೆಂದು
ಪ್ರಕ-ಟಿ-ಸ-ಲಾಗಿದೆ
ಪ್ರಕ-ಟಿ-ಸ-ಲಾ-ಯಿತು
ಪ್ರಕ-ಟಿ-ಸ-ಲಿದ್ದ
ಪ್ರಕ-ಟಿ-ಸಲು
ಪ್ರಕ-ಟಿ-ಸ-ಲ್ಪ-ಟ್ಟಿತು
ಪ್ರಕ-ಟಿಸಿ
ಪ್ರಕ-ಟಿ-ಸಿತು
ಪ್ರಕ-ಟಿ-ಸಿದ
ಪ್ರಕ-ಟಿ-ಸಿ-ದರು
ಪ್ರಕ-ಟಿ-ಸಿ-ದರೆ
ಪ್ರಕ-ಟಿ-ಸಿ-ದುದು
ಪ್ರಕ-ಟಿ-ಸಿ-ದುವು
ಪ್ರಕ-ಟಿ-ಸಿದೆ
ಪ್ರಕ-ಟಿ-ಸಿದ್ದ
ಪ್ರಕ-ಟಿ-ಸಿ-ದ್ದಂತೆ
ಪ್ರಕ-ಟಿಸು
ಪ್ರಕ-ಟಿ-ಸುವ
ಪ್ರಕಾಂಡ
ಪ್ರಕಾರ
ಪ್ರಕಾ-ರ-ಗಳು
ಪ್ರಕಾ-ರದ
ಪ್ರಕಾ-ರ-ವಾಗಿ
ಪ್ರಕಾ-ರವೂ
ಪ್ರಕಾ-ರವೇ
ಪ್ರಕಾ-ಶ-ಕ-ರಾದ
ಪ್ರಕಾ-ಶನ
ಪ್ರಕಾ-ಶ-ಮಾನ
ಪ್ರಕಾ-ಶ-ಮಾ-ನ-ವಾ-ಗಿತ್ತು
ಪ್ರಕಾ-ಶ-ಯು-ಕ್ತ-ವಾಗಿ
ಪ್ರಕೃ-ತ-ವಾ-ಗಿವೆ
ಪ್ರಕೃತಿ
ಪ್ರಕೃ-ತಿಯ
ಪ್ರಕೃ-ತಿ-ಯನ್ನೂ
ಪ್ರಕೃ-ತಿ-ಯಲ್ಲ
ಪ್ರಕೃ-ತಿಯೇ
ಪ್ರಕೃ-ತಿ-ಸ್ಥ-ರಾಗಿ
ಪ್ರಕ್ರಿಯೆ
ಪ್ರಕ್ಷುಬ್ದ
ಪ್ರಕ್ಷುಬ್ಧ
ಪ್ರಕ್ಷು-ಬ್ಧ-ಗೊಂಡು
ಪ್ರಕ್ಷು-ಬ್ಧ-ವಾ-ಗಿ-ರು-ತ್ತ-ದಾ-ದರೂ
ಪ್ರಖರ
ಪ್ರಖ-ರ-ಜೀ-ವಂತ
ಪ್ರಖ-ರ-ತೆಯು
ಪ್ರಖ-ರ-ವಾಗಿ
ಪ್ರಖ-ರ-ವಾ-ಗುತ್ತ
ಪ್ರಖ-ರವೂ
ಪ್ರಖ್ಯಾತ
ಪ್ರಖ್ಯಾ-ತ-ರಾದ
ಪ್ರಖ್ಯಾ-ತ-ರಾ-ದ-ವರೂ
ಪ್ರಖ್ಯಾ-ತಳು
ಪ್ರಖ್ಯಾ-ತ-ವಾ-ಗಿತ್ತು
ಪ್ರಖ್ಯಾ-ತಿಗೆ
ಪ್ರಖ್ಯಾ-ತಿ-ಯನ್ನು
ಪ್ರಖ್ಯಾ-ತಿಯು
ಪ್ರಗತಿ
ಪ್ರಗ-ತಿ-ಇದೇ
ಪ್ರಗ-ತಿ-ಗಾಗಿ
ಪ್ರಗ-ತಿಗೆ
ಪ್ರಗ-ತಿ-ಪ-ಥ-ದ-ಲ್ಲಿವೆ
ಪ್ರಗ-ತಿ-ಪರ
ಪ್ರಗ-ತಿ-ಪ-ರ-ರಾದ
ಪ್ರಗ-ತಿ-ಪ-ರ-ರೆ-ನ್ನಿ-ಸಿ-ಕೊಂ-ಡ-ವರು
ಪ್ರಗ-ತಿಯ
ಪ್ರಗ-ತಿ-ಯತ್ತ
ಪ್ರಗ-ತಿ-ಯನ್ನು
ಪ್ರಗ-ತಿ-ಯನ್ನೂ
ಪ್ರಗ-ತಿ-ಯಲ್ಲಿ
ಪ್ರಗ-ತಿ-ಯೊಂ-ದಿಗೆ
ಪ್ರಚಂಡ
ಪ್ರಚಂ-ಡ-ರ-ಲ್ಲೊ-ಬ್ಬರು
ಪ್ರಚಂ-ಡ-ರಾದ
ಪ್ರಚಂ-ಡ-ರಾ-ದ-ವ-ರಿಂದ
ಪ್ರಚಂ-ಡ-ರಿಂದ
ಪ್ರಚಂ-ಡ-ವಾಗಿ
ಪ್ರಚಂ-ಡ-ವಾ-ದದ್ದು
ಪ್ರಚ-ಲಿ-ತ-ವಾ-ಗಿ-ದೆ-ಯೆಂ-ದರೆ
ಪ್ರಚ-ಲಿ-ತ-ವಾ-ಗಿ-ದ್ದು-ದನ್ನು
ಪ್ರಚ-ಲಿ-ತ-ವಿತ್ತು
ಪ್ರಚ-ಲಿ-ತ-ವಿದ್ದ
ಪ್ರಚ-ಲಿ-ತ-ವಿ-ರುವ
ಪ್ರಚ-ಲಿ-ತ-ವಿ-ರು-ವುದು
ಪ್ರಚಾರ
ಪ್ರಚಾ-ರಕ
ಪ್ರಚಾ-ರ-ಕ-ನಾ-ಗಿದ್ದ
ಪ್ರಚಾ-ರ-ಕರ
ಪ್ರಚಾ-ರ-ಕ-ರನ್ನು
ಪ್ರಚಾ-ರ-ಕ-ರಲ್ಲಿ
ಪ್ರಚಾ-ರ-ಕ-ರಿ-ಗಂತೂ
ಪ್ರಚಾ-ರ-ಕ-ರಿ-ಗಾ-ಗಲಿ
ಪ್ರಚಾ-ರ-ಕ-ರಿಗೆ
ಪ್ರಚಾ-ರ-ಕರು
ಪ್ರಚಾ-ರ-ಕರೂ
ಪ್ರಚಾ-ರ-ಕಾ-ರ್ಯ-ಕ್ಕಾಗಿ
ಪ್ರಚಾ-ರ-ಕಾ-ರ್ಯವು
ಪ್ರಚಾ-ರ-ಕ್ಕಾಗಿ
ಪ್ರಚಾ-ರದ
ಪ್ರಚಾ-ರ-ದ-ಲ್ಲಿತ್ತು
ಪ್ರಚಾ-ರ-ಮಾ-ಡುತ್ತ
ಪ್ರಚಾ-ರ-ಯಾ-ತ್ರೆಯ
ಪ್ರಚಾ-ರ-ವನ್ನು
ಪ್ರಚಾ-ರ-ವಾ-ಗು-ತ್ತಿತ್ತು
ಪ್ರಚಾ-ರ-ವಾ-ದ-ದ್ದಲ್ಲ
ಪ್ರಚಾ-ರ-ವಾ-ಯಿತು
ಪ್ರಚೋ-ದನೆ
ಪ್ರಚೋ-ದ-ನೆ-ಯುಂ-ಟಾ-ಗಿ-ರ-ಬ-ಹುದು
ಪ್ರಚೋ-ದ-ನೆಯೂ
ಪ್ರಚೋ-ದಿ-ತ-ಗೊಂ-ಡಿ-ರ-ಬೇ-ಕ-ಲ್ಲವೆ
ಪ್ರಚೋ-ದಿಸಿ
ಪ್ರಚೋ-ದಿ-ಸು-ವುದು
ಪ್ರಜಾ-ಕಂ-ಟ-ಕ-ತ-ನ-ವನ್ನು
ಪ್ರಜಾ-ಪೀ-ಡಕ
ಪ್ರಜಾ-ಪೀ-ಡ-ಕ-ರಿರಾ
ಪ್ರಜಾ-ಪೀ-ಡ-ನೆಯೂ
ಪ್ರಜೆ-ಒಬ್ಬ
ಪ್ರಜೆ-ಗಳ
ಪ್ರಜೆ-ಗಳನ್ನು
ಪ್ರಜೆ-ಗ-ಳೆ-ಲ್ಲ-ರಿಗೂ
ಪ್ರಜೆ-ಯ-ನ್ನಾ-ಗಿ-ಸ-ಲಿ-ದ್ದಾರೆ
ಪ್ರಜೆ-ಯಾಗಿ
ಪ್ರಜೆಯೂ
ಪ್ರಜ್ಞಾ-ಪೂ-ರ್ವ-ಕ-ವಾಗಿ
ಪ್ರಜ್ಞಾ-ವ-ಸ್ಥೆ-ಯ-ಲ್ಲಿ-ರುವ
ಪ್ರಜ್ಞೆ
ಪ್ರಜ್ಞೆಯ
ಪ್ರಜ್ಞೆ-ಯನ್ನು
ಪ್ರಜ್ಞೆ-ಯಿಂದ
ಪ್ರಜ್ಞೆಯು
ಪ್ರಜ್ಞೆ-ಯು-ಳ್ಳ-ವ-ರು-ಕೆ-ಲ-ಸಕ್ಕೆ
ಪ್ರಜ್ಞೆಯೂ
ಪ್ರಜ್ವ-ಲಿ-ಸು-ತ್ತಿ-ರುವ
ಪ್ರಜ್ವ-ಲಿ-ಸುವ
ಪ್ರಜ್ವ-ಲಿ-ಸು-ವಂತೆ
ಪ್ರಣಾಮ
ಪ್ರಣಾ-ಳಿಯ
ಪ್ರತಾ-ಪ್ಚಂದ್ರ
ಪ್ರತಿ
ಪ್ರತಿ-ಕೂ-ಲ-ವಾ-ಗಿತ್ತು
ಪ್ರತಿ-ಕೂ-ಲ-ವಾ-ಗಿ-ರ-ದಿ-ದ್ದರೂ
ಪ್ರತಿ-ಕೂ-ಲ-ವಾದ
ಪ್ರತಿ-ಕೆ-ಗ-ಳಿಗೆ
ಪ್ರತಿ-ಕೆ-ಯಿಂ-ದಲೂ
ಪ್ರತಿ-ಕ್ರಿ-ಯಿ-ಸಲು
ಪ್ರತಿ-ಕ್ರಿ-ಯಿ-ಸಿದ್ದ
ಪ್ರತಿ-ಕ್ರಿ-ಯಿ-ಸಿ-ದ್ದೇಕೆ
ಪ್ರತಿ-ಕ್ರಿ-ಯಿ-ಸು-ತ್ತಾರೆ
ಪ್ರತಿ-ಕ್ರಿ-ಯಿ-ಸು-ತ್ತಿ-ದ್ದರು
ಪ್ರತಿ-ಕ್ರಿ-ಯಿ-ಸುವ
ಪ್ರತಿ-ಕ್ರಿಯೆ
ಪ್ರತಿ-ಕ್ರಿ-ಯೆ-ಇ-ದನ್ನೇ
ಪ್ರತಿ-ಕ್ರಿ-ಯೆ-ಗ-ಳಿಗೆ
ಪ್ರತಿ-ಕ್ರಿ-ಯೆಗೆ
ಪ್ರತಿ-ಕ್ರಿ-ಯೆ-ಯನ್ನೂ
ಪ್ರತಿ-ಕ್ರಿ-ಯೆ-ಯಾಗಿ
ಪ್ರತಿ-ಕ್ರಿ-ಯೆಯು
ಪ್ರತಿ-ಕ್ರಿ-ಯೆಯೂ
ಪ್ರತಿ-ಕ್ರಿ-ಯೆಯೋ
ಪ್ರತಿ-ಕ್ಷ-ಣ-ದಲ್ಲೂ
ಪ್ರತಿ-ಕ್ಷ-ಣವೂ
ಪ್ರತಿ-ಗಳನ್ನು
ಪ್ರತಿ-ಗಳು
ಪ್ರತಿ-ಗಳೂ
ಪ್ರತಿ-ತಿಂ-ಗಳೂ
ಪ್ರತಿ-ದಿನ
ಪ್ರತಿ-ದಿ-ನವೂ
ಪ್ರತಿ-ದಿ-ನ-ವೆಂ-ಬಂತೆ
ಪ್ರತಿ-ಧ್ವ-ನಿ-ಗೊ-ಳ್ಳು-ವಂ-ತಾ-ಗಿತ್ತು
ಪ್ರತಿ-ಧ್ವ-ನಿ-ತ-ವಾ-ಗು-ತ್ತಿ-ದ್ದುವು
ಪ್ರತಿ-ಧ್ವ-ನಿ-ತ-ವಾ-ದ-ದ್ದ-ರಿಂದ
ಪ್ರತಿ-ಧ್ವ-ನಿ-ಯಂತೆ
ಪ್ರತಿ-ಧ್ವ-ನಿಸಿ
ಪ್ರತಿ-ಧ್ವ-ನಿ-ಸಿತು
ಪ್ರತಿ-ಧ್ವ-ನಿ-ಸು-ತ್ತಿದೆ
ಪ್ರತಿ-ನ-ಮ-ಸ್ಕ-ರಿ-ಸಿ-ದರು
ಪ್ರತಿ-ನ-ಮ-ಸ್ಕ-ರಿಸು
ಪ್ರತಿ-ನಿಧಿ
ಪ್ರತಿ-ನಿ-ಧಿ-ಎಂ-ದರೆ
ಪ್ರತಿ-ನಿ-ಧಿ-ಗಳ
ಪ್ರತಿ-ನಿ-ಧಿ-ಗ-ಳಂತೆ
ಪ್ರತಿ-ನಿ-ಧಿ-ಗಳನ್ನು
ಪ್ರತಿ-ನಿ-ಧಿ-ಗ-ಳನ್ನೇ
ಪ್ರತಿ-ನಿ-ಧಿ-ಗಳಲ್ಲಿ
ಪ್ರತಿ-ನಿ-ಧಿ-ಗ-ಳ-ಲ್ಲೊ-ಬ್ಬ-ರಾದ
ಪ್ರತಿ-ನಿ-ಧಿ-ಗ-ಳಾ-ದ-ವರು
ಪ್ರತಿ-ನಿ-ಧಿ-ಗಳಿಂದ
ಪ್ರತಿ-ನಿ-ಧಿ-ಗ-ಳಿ-ಗಾಗಿ
ಪ್ರತಿ-ನಿ-ಧಿ-ಗ-ಳಿ-ಗಿಂ-ತಲೂ
ಪ್ರತಿ-ನಿ-ಧಿ-ಗ-ಳಿಗೆ
ಪ್ರತಿ-ನಿ-ಧಿ-ಗಳು
ಪ್ರತಿ-ನಿ-ಧಿ-ಗ-ಳೆಲ್ಲ
ಪ್ರತಿ-ನಿ-ಧಿ-ಗ-ಳೆ-ಲ್ಲರೂ
ಪ್ರತಿ-ನಿ-ಧಿ-ಯನ್ನು
ಪ್ರತಿ-ನಿ-ಧಿ-ಯಾ-ಗಲು
ಪ್ರತಿ-ನಿ-ಧಿ-ಯಾಗಿ
ಪ್ರತಿ-ನಿ-ಧಿ-ಯಾ-ಗಿದ್ದ
ಪ್ರತಿ-ನಿ-ಧಿ-ಯಾ-ಗಿ-ದ್ದ-ರೆಂ-ಬು-ದ-ರಲ್ಲಿ
ಪ್ರತಿ-ನಿ-ಧಿ-ಯಾ-ಗಿ-ರ-ಬೇ-ಕಿತ್ತು
ಪ್ರತಿ-ನಿ-ಧಿ-ಯಾಗು
ಪ್ರತಿ-ನಿ-ಧಿ-ಯಾದ
ಪ್ರತಿ-ನಿ-ಧಿಯು
ಪ್ರತಿ-ನಿ-ಧಿ-ಯೆಂದು
ಪ್ರತಿ-ನಿ-ಧಿಯೇ
ಪ್ರತಿ-ನಿ-ಧಿ-ಯೊಬ್ಬ
ಪ್ರತಿ-ನಿ-ಧಿ-ಯೊ-ಬ್ಬ-ರನ್ನು
ಪ್ರತಿ-ನಿ-ಧಿ-ಯೊ-ಬ್ಬರು
ಪ್ರತಿ-ನಿ-ಧಿಸ
ಪ್ರತಿ-ನಿ-ಧಿ-ಸದೆ
ಪ್ರತಿ-ನಿ-ಧಿ-ಸ-ಲಿಲ್ಲ
ಪ್ರತಿ-ನಿ-ಧಿ-ಸಿ-ದು-ದ-ಕ್ಕಾಗಿ
ಪ್ರತಿ-ನಿ-ಧಿ-ಸಿ-ದ್ದರು
ಪ್ರತಿ-ನಿ-ಧಿ-ಸಿ-ದ್ದ-ವರು
ಪ್ರತಿ-ನಿ-ಧಿ-ಸು-ತ್ತಿದ್ದ
ಪ್ರತಿ-ನಿ-ಧಿ-ಸು-ತ್ತಿ-ದ್ದ-ನಷ್ಟೆ
ಪ್ರತಿ-ನಿ-ಧಿ-ಸು-ತ್ತಿ-ದ್ದುವು
ಪ್ರತಿ-ನಿ-ಧಿ-ಸು-ತ್ತಿ-ರು-ವುದನ್ನು
ಪ್ರತಿ-ನಿ-ಧಿ-ಸು-ತ್ತೇನೆ
ಪ್ರತಿ-ನಿ-ಧಿ-ಸುವ
ಪ್ರತಿ-ನಿ-ಧಿ-ಸು-ವು-ದಿ-ಲ್ಲ-ವೆಂದು
ಪ್ರತಿ-ನಿ-ಧೀ-ಕೃ-ತ-ರಾ-ದ-ವರೆಲ್ಲ
ಪ್ರತಿ-ಪಾ-ದ-ಕ-ನಾ-ಗಿದ್ದ
ಪ್ರತಿ-ಪಾ-ದ-ಕನೂ
ಪ್ರತಿ-ಪಾ-ದ-ಕ-ರಾ-ಗಿದ್ದ
ಪ್ರತಿ-ಪಾ-ದ-ಕ-ರಾದ
ಪ್ರತಿ-ಪಾ-ದ-ಕರು
ಪ್ರತಿ-ಪಾ-ದ-ಕರೂ
ಪ್ರತಿ-ಪಾ-ದ-ನೆ-ಯಾ-ಗಿತ್ತು
ಪ್ರತಿ-ಪಾ-ದಿ-ತ-ವಾ-ಗಿ-ರುವ
ಪ್ರತಿ-ಪಾ-ದಿ-ತ-ವಾದ
ಪ್ರತಿ-ಪಾ-ದಿಸಿ
ಪ್ರತಿ-ಪಾ-ದಿ-ಸಿತು
ಪ್ರತಿ-ಪಾ-ದಿ-ಸಿ-ದರು
ಪ್ರತಿ-ಪಾ-ದಿ-ಸಿ-ದ-ವರು
ಪ್ರತಿ-ಪಾ-ದಿ-ಸಿ-ದುದು
ಪ್ರತಿ-ಪಾ-ದಿ-ಸುವ
ಪ್ರತಿ-ಪಾ-ದಿ-ಸು-ವಂ-ತಹ
ಪ್ರತಿ-ಪಾ-ದಿ-ಸು-ವ-ವರು
ಪ್ರತಿ-ಫ-ಲದ
ಪ್ರತಿ-ಫ-ಲ-ವನ್ನು
ಪ್ರತಿ-ಫ-ಲಾ-ಪ-ಯೇ-ಕ್ಷೆ-ಯಿ-ಲ್ಲದೆ
ಪ್ರತಿ-ಫ-ಲಿಸಿ
ಪ್ರತಿ-ಬಂ-ಧ-ಕ-ಗಳನ್ನು
ಪ್ರತಿ-ಬಂ-ಧ-ಕ-ಗ-ಳಾಗಿ
ಪ್ರತಿ-ಬಂ-ಧ-ಕ-ಗಳು
ಪ್ರತಿ-ಬಿಂ-ಬ-ವನ್ನು
ಪ್ರತಿ-ಬಿಂ-ಬವೂ
ಪ್ರತಿ-ಬಿಂ-ಬ-ವೊಂದು
ಪ್ರತಿ-ಬಿಂಬಿ
ಪ್ರತಿ-ಬಿಂ-ಬಿತ
ಪ್ರತಿ-ಬಿಂ-ಬಿ-ತ-ವಾ-ಗು-ತ್ತಿದ್ದ
ಪ್ರತಿ-ಬಿಂ-ಬಿ-ಸು-ತ್ತ-ದೆ-ಯೆಂ-ಬು-ದ-ನ್ನಂತೂ
ಪ್ರತಿ-ಬಿಂ-ಬಿ-ಸು-ತ್ತಿತ್ತು
ಪ್ರತಿ-ಬಿಂ-ಬಿ-ಸು-ವಂ-ತಹ
ಪ್ರತಿ-ಬಿಂ-ಬಿ-ಸು-ವಂ-ಥ-ದೇನೂ
ಪ್ರತಿ-ಭ-ಟನೆ
ಪ್ರತಿ-ಭ-ಟ-ನೆಯ
ಪ್ರತಿ-ಭ-ಟ-ನೆಯೆ
ಪ್ರತಿ-ಭ-ಟಿ-ಸದೆ
ಪ್ರತಿ-ಭ-ಟಿಸಿ
ಪ್ರತಿ-ಭ-ಟಿ-ಸಿದ
ಪ್ರತಿ-ಭ-ಟಿ-ಸಿ-ದರು
ಪ್ರತಿ-ಭ-ಟಿ-ಸಿ-ದರೂ
ಪ್ರತಿ-ಭ-ಟಿ-ಸಿ-ದ್ದರು
ಪ್ರತಿ-ಭ-ಟಿ-ಸುತ್ತ
ಪ್ರತಿ-ಭ-ಟಿ-ಸು-ತ್ತಿ-ದ್ದರು
ಪ್ರತಿ-ಭ-ಟಿ-ಸು-ವಂ-ತಹ
ಪ್ರತಿ-ಭ-ಟಿ-ಸು-ವಂತೆ
ಪ್ರತಿ-ಭಾ-ವಂ-ತನೂ
ಪ್ರತಿ-ಭಾ-ವಂ-ತ-ರಾದ
ಪ್ರತಿ-ಭಾ-ಶಾಲಿ
ಪ್ರತಿ-ಭಾ-ಶಾ-ಲಿಯ
ಪ್ರತಿಭೆ
ಪ್ರತಿ-ಭೆಯ
ಪ್ರತಿ-ಭೆ-ಯನ್ನು
ಪ್ರತಿ-ಯನ್ನು
ಪ್ರತಿ-ಯಾಗಿ
ಪ್ರತಿಯೊ
ಪ್ರತಿ-ಯೊಂ-ದಕ್ಕೂ
ಪ್ರತಿ-ಯೊಂ-ದನ್ನು
ಪ್ರತಿ-ಯೊಂ-ದನ್ನೂ
ಪ್ರತಿ-ಯೊಂದು
ಪ್ರತಿ-ಯೊಂದೂ
ಪ್ರತಿ-ಯೊಬ್ಬ
ಪ್ರತಿ-ಯೊ-ಬ್ಬನ
ಪ್ರತಿ-ಯೊ-ಬ್ಬ-ನಿಗೂ
ಪ್ರತಿ-ಯೊ-ಬ್ಬನೂ
ಪ್ರತಿ-ಯೊ-ಬ್ಬರ
ಪ್ರತಿ-ಯೊ-ಬ್ಬ-ರನ್ನೂ
ಪ್ರತಿ-ಯೊ-ಬ್ಬ-ರಿಗೂ
ಪ್ರತಿ-ಯೊ-ಬ್ಬ-ರಿ-ಗೂ-ಎ-ಲ್ಲ-ರಿಗೂ
ಪ್ರತಿ-ಯೊ-ಬ್ಬರೂ
ಪ್ರತಿ-ವಂ-ದಿ-ಸ-ಬ-ಹು-ದಂತೆ
ಪ್ರತಿ-ವಂ-ದಿ-ಸಿ-ದುದು
ಪ್ರತಿ-ವಂ-ದಿ-ಸುವ
ಪ್ರತಿ-ವ-ರ್ಷವೂ
ಪ್ರತಿ-ವಾ-ದಿ-ಗಳ
ಪ್ರತಿ-ಷ್ಠಾ-ಪಿ-ತ-ವಾ-ಯಿತು
ಪ್ರತಿ-ಷ್ಠಾ-ಪಿ-ಸುವ
ಪ್ರತಿ-ಷ್ಠಿತ
ಪ್ರತಿ-ಷ್ಠಿ-ತರ
ಪ್ರತಿ-ಷ್ಠಿ-ತ-ರ-ಲ್ಲೊ-ಬ್ಬ-ರಾದ
ಪ್ರತಿ-ಷ್ಠಿ-ತರು
ಪ್ರತಿ-ಷ್ಠಿ-ತ-ರೆಲ್ಲ
ಪ್ರತಿ-ಷ್ಠಿಸು
ಪ್ರತಿ-ಷ್ಠೆ-ಯ-ವ-ರಂತೆ
ಪ್ರತಿ-ಸಲ
ಪ್ರತಿ-ಸ-ಲವೂ
ಪ್ರತಿ-ಸ್ಪಂ-ದಿ-ಸು-ತ್ತಿ-ದ್ದರು
ಪ್ರತಿ-ಸ್ಪರ್ಧಿ
ಪ್ರತೀಕ
ಪ್ರತೀಕ್ಷೆ
ಪ್ರತೀತಿ
ಪ್ರತೀ-ತಿ-ಯಿತ್ತು
ಪ್ರತ್ಯಕ್ಷ
ಪ್ರತ್ಯ-ಕ್ಷ-ದ-ರ್ಶಿ-ಗಳ
ಪ್ರತ್ಯ-ಕ್ಷ-ವಾಗಿ
ಪ್ರತ್ಯ-ಕ್ಷಾ-ನು-ಭ-ವ-ದಿಂದ
ಪ್ರತ್ಯಾ-ಹಾ-ರ-ಧಾ-ರ-ಣ-ಗಳ
ಪ್ರತ್ಯು-ತ್ತರ
ಪ್ರತ್ಯು-ತ್ತ-ರ-ಗಳನ್ನು
ಪ್ರತ್ಯು-ತ್ತ-ರವೇ
ಪ್ರತ್ಯು-ತ್ಪ-ನ್ನ-ಮ-ತಿ-ಯು-ಳ್ಳ-ವರು
ಪ್ರತ್ಯೇಕ
ಪ್ರತ್ಯೇ-ಕಿ-ಸುವ
ಪ್ರಥಮ
ಪ್ರಥ-ಮತಃ
ಪ್ರದ-ಕ್ಷಿಣೆ
ಪ್ರದ-ರ್ಶನ
ಪ್ರದ-ರ್ಶ-ನ-ಗಳಲ್ಲಿ
ಪ್ರದ-ರ್ಶ-ನದ
ಪ್ರದ-ರ್ಶ-ನ-ವನ್ನು
ಪ್ರದ-ರ್ಶ-ನವು
ಪ್ರದ-ರ್ಶ-ನ-ವೊಂದು
ಪ್ರದ-ರ್ಶ-ನಾ-ಲ-ಯ-ಗ-ಳಿಗೆ
ಪ್ರದ-ರ್ಶ-ನಾ-ಲ-ಯ-ಗಳು
ಪ್ರದ-ರ್ಶಿತ
ಪ್ರದ-ರ್ಶಿ-ಸ-ಲಾ-ಗಿತ್ತು
ಪ್ರದ-ರ್ಶಿ-ಸಿ-ಕೊಂ-ಡಿದೆ
ಪ್ರದ-ರ್ಶಿ-ಸಿಲು
ಪ್ರದ-ರ್ಶಿಸು
ಪ್ರದ-ರ್ಶಿ-ಸುತ್ತ
ಪ್ರದ-ರ್ಶಿ-ಸು-ತ್ತದೆ
ಪ್ರದ-ರ್ಶಿ-ಸುವ
ಪ್ರದ-ರ್ಶಿ-ಸು-ವಂತೆ
ಪ್ರದ-ರ್ಶಿ-ಸು-ವು-ದಾಗಿ
ಪ್ರದ-ರ್ಶಿ-ಸು-ವುದೇ
ಪ್ರದೇಶ
ಪ್ರದೇ-ಶ-ಗ-ಳಿಗೆ
ಪ್ರದೇ-ಶದ
ಪ್ರದೇ-ಶ-ದಲ್ಲಿ
ಪ್ರದೇ-ಶ-ದಲ್ಲೇ
ಪ್ರಧಾನ
ಪ್ರಧಾ-ನ-ಮಂ-ತ್ರಿ-ಯಾದ
ಪ್ರಧಾ-ನ-ವಾದ
ಪ್ರಪಂಚ
ಪ್ರಪಂ-ಚಕ್ಕೆ
ಪ್ರಪಂ-ಚದ
ಪ್ರಪಂ-ಚ-ದಲ್ಲಿ
ಪ್ರಪಂ-ಚ-ದ-ಲ್ಲಿ-ದ್ದು-ಕೊಂಡು
ಪ್ರಪಂ-ಚ-ದ-ಲ್ಲಿನ
ಪ್ರಪಂ-ಚ-ದ-ಲ್ಲಿಯೂ
ಪ್ರಪಂ-ಚ-ದಲ್ಲೆಲ್ಲ
ಪ್ರಪಂ-ಚ-ದಲ್ಲೇ
ಪ್ರಪಂ-ಚ-ದ-ವರ
ಪ್ರಪಂ-ಚ-ದಿಂ-ದ-ಶೀ-ಘ್ರವೇ
ಪ್ರಪಂ-ಚ-ವನ್ನು
ಪ್ರಪಂ-ಚ-ವನ್ನೇ
ಪ್ರಪಂ-ಚ-ವೆಂಬ
ಪ್ರಪಂ-ಚ-ವೆಂ-ಬುದು
ಪ್ರಪ-ದ್ಯಂತೇ
ಪ್ರಪು-ಲ್ಲವೂ
ಪ್ರಪ್ರ-ಥಮ
ಪ್ರಬಂ-ಧ-ವಾದ
ಪ್ರಬಂ-ಧ-ವೊಂ-ದನ್ನು
ಪ್ರಬಲ
ಪ್ರಬ-ಲ-ವಾಗಿ
ಪ್ರಬ-ಲ-ವಾ-ಗಿತ್ತು
ಪ್ರಬ-ಲ-ವಾ-ಗಿ-ತ್ತೆಂ-ದರೆ
ಪ್ರಬ-ಲ-ವಾ-ಗಿ-ದ್ದಿ-ರ-ಬೇಕು
ಪ್ರಬ-ಲ-ವಾದ
ಪ್ರಬ-ಲ-ವಾ-ದ-ದ್ದೆಂದು
ಪ್ರಬ-ಲವೂ
ಪ್ರಬುದ್ಧ
ಪ್ರಭಾವ
ಪ್ರಭಾ-ವ-ಕ್ಕಿಂತ
ಪ್ರಭಾ-ವಕ್ಕೆ
ಪ್ರಭಾ-ವ-ಕ್ಕೊ-ಳ-ಗಾಗಿ
ಪ್ರಭಾ-ವ-ಕ್ಕೊ-ಳ-ಗಾದ
ಪ್ರಭಾ-ವ-ಗ-ಳಿಗೆ
ಪ್ರಭಾ-ವದ
ಪ್ರಭಾ-ವ-ಪೂ-ರ್ಣ-ವಾ-ಗಿದೆ
ಪ್ರಭಾ-ವ-ಪೂ-ರ್ಣವೂ
ಪ್ರಭಾ-ವ-ಲ-ಯವು
ಪ್ರಭಾ-ವ-ವನ್ನು
ಪ್ರಭಾ-ವ-ವ-ನ್ನೆಲ್ಲ
ಪ್ರಭಾ-ವ-ವಿ-ರ-ಬ-ಹುದೆ
ಪ್ರಭಾ-ವವು
ಪ್ರಭಾ-ವ-ವೆಂ-ಥದು
ಪ್ರಭಾ-ವ-ಶಾಲಿ
ಪ್ರಭಾ-ವ-ಶಾ-ಲಿ-ಗ-ಳಾಗಿ
ಪ್ರಭಾ-ವ-ಶಾ-ಲಿ-ಗ-ಳಾ-ಗಿ-ದ್ದರು
ಪ್ರಭಾ-ವ-ಶಾ-ಲಿ-ಯಾಗಿ
ಪ್ರಭಾ-ವ-ಶಾ-ಲಿಯೂ
ಪ್ರಭಾ-ವ-ಶಾಲೀ
ಪ್ರಭಾ-ವಿತ
ಪ್ರಭಾ-ವಿ-ತ-ನಾಗಿ
ಪ್ರಭಾ-ವಿ-ತ-ನಾ-ಗಿದ್ದ
ಪ್ರಭಾ-ವಿ-ತ-ನಾ-ಗಿ-ದ್ದ-ನೆಂ-ಬುದು
ಪ್ರಭಾ-ವಿ-ತ-ನಾ-ಗಿ-ದ್ದನೋ
ಪ್ರಭಾ-ವಿ-ತ-ನಾ-ಗಿ-ದ್ದ-ವನು
ಪ್ರಭಾ-ವಿ-ತ-ನಾ-ಗಿ-ದ್ದೇನೆ
ಪ್ರಭಾ-ವಿ-ತ-ನಾದ
ಪ್ರಭಾ-ವಿ-ತ-ನಾ-ದ-ವ-ನ-ಲ್ಲವೆ
ಪ್ರಭಾ-ವಿ-ತ-ನಾದೆ
ಪ್ರಭಾ-ವಿ-ತ-ರಾಗಿ
ಪ್ರಭಾ-ವಿ-ತ-ರಾ-ಗಿದ್ದ
ಪ್ರಭಾ-ವಿ-ತ-ರಾ-ಗಿ-ದ್ದರು
ಪ್ರಭಾ-ವಿ-ತ-ರಾ-ಗಿ-ದ್ದ-ರೆಂ-ದರೆ
ಪ್ರಭಾ-ವಿ-ತ-ರಾ-ಗಿ-ದ್ದಾರೆ
ಪ್ರಭಾ-ವಿ-ತ-ರಾ-ಗು-ತ್ತಿ-ದ್ದಾರೆ
ಪ್ರಭಾ-ವಿ-ತ-ರಾದ
ಪ್ರಭಾ-ವಿ-ತ-ರಾ-ದ-ರೆಂ-ದರೆ
ಪ್ರಭಾ-ವಿ-ತ-ರಾ-ದೆವು
ಪ್ರಭಾ-ವಿ-ತ-ಳಾಗಿ
ಪ್ರಭಾ-ವಿ-ತ-ಳಾದ
ಪ್ರಭಾ-ವಿ-ತ-ಳಾ-ದಳು
ಪ್ರಭಾ-ವಿ-ತ-ವಾಗಿ
ಪ್ರಭಾ-ವಿ-ತ-ವಾ-ಯಿತು
ಪ್ರಭಿನ್ನೇ
ಪ್ರಭು
ಪ್ರಭು-ಗ-ಳಾ-ಗಿ-ರು-ವಾಗ
ಪ್ರಭುತ್ವ
ಪ್ರಭು-ತ್ವ-ವನ್ನು
ಪ್ರಭು-ತ್ವವು
ಪ್ರಭು-ರಾ-ಷ್ಟ್ರದ
ಪ್ರಭುವೆ
ಪ್ರಭೆ-ಯನ್ನು
ಪ್ರಭೇ-ದ-ಗ-ಳಲ್ಲೂ
ಪ್ರಮಾಣ
ಪ್ರಮಾ-ಣ-ಗ್ರಂ-ಥ-ಗಳು
ಪ್ರಮಾ-ಣದ
ಪ್ರಮಾ-ಣ-ದಲ್ಲಿ
ಪ್ರಮಾ-ಣ-ಪ-ತ್ರ-ಗಳ
ಪ್ರಮಾ-ಣ-ವನ್ನು
ಪ್ರಮಾ-ಣವು
ಪ್ರಮಾ-ಣವೂ
ಪ್ರಮಾ-ಣಿ-ಕನೆ
ಪ್ರಮಾ-ಣಿ-ಕ-ವಾದ
ಪ್ರಮಾ-ಣೀ-ಕ-ರಿ-ಸ-ಬ-ಲ್ಲರು
ಪ್ರಮಾ-ದ-ವಾ-ಗ-ದಿ-ರದು
ಪ್ರಮುಖ
ಪ್ರಮು-ಖರ
ಪ್ರಮು-ಖ-ರನ್ನೂ
ಪ್ರಮು-ಖ-ರಿಗೆ
ಪ್ರಮು-ಖ-ರಿ-ದ್ದರು
ಪ್ರಮು-ಖರು
ಪ್ರಮು-ಖರೂ
ಪ್ರಮೇ-ಯ-ಗಳ
ಪ್ರಮೇ-ಯ-ವನ್ನು
ಪ್ರಯತ್ನ
ಪ್ರಯ-ತ್ನಕ್ಕೆ
ಪ್ರಯ-ತ್ನ-ಗಳನ್ನು
ಪ್ರಯ-ತ್ನ-ಗಳು
ಪ್ರಯ-ತ್ನದ
ಪ್ರಯ-ತ್ನ-ದಲ್ಲಿ
ಪ್ರಯ-ತ್ನ-ದ-ಲ್ಲಿದ್ದ
ಪ್ರಯ-ತ್ನ-ದಲ್ಲೇ
ಪ್ರಯ-ತ್ನ-ಪಡು
ಪ್ರಯ-ತ್ನ-ಪೂ-ರ್ವ-ಕ-ವಾಗಿ
ಪ್ರಯ-ತ್ನ-ಮಾ-ಡ-ಬೇಕು
ಪ್ರಯ-ತ್ನ-ಮಾಡಿ
ಪ್ರಯ-ತ್ನ-ಮಾ-ಡಿ-ದ್ದೇನೆ
ಪ್ರಯ-ತ್ನ-ವ-ನ್ನಂತೂ
ಪ್ರಯ-ತ್ನ-ವ-ನ್ನಿನ್ನೂ
ಪ್ರಯ-ತ್ನ-ವನ್ನು
ಪ್ರಯ-ತ್ನ-ವನ್ನೂ
ಪ್ರಯ-ತ್ನ-ವನ್ನೇ
ಪ್ರಯ-ತ್ನವೂ
ಪ್ರಯ-ತ್ನವೇ
ಪ್ರಯ-ತ್ನವೋ
ಪ್ರಯತ್ನಿ
ಪ್ರಯ-ತ್ನಿ-ಸದ
ಪ್ರಯ-ತ್ನಿ-ಸ-ಬ-ಹುದು
ಪ್ರಯ-ತ್ನಿ-ಸ-ಬೇಕು
ಪ್ರಯ-ತ್ನಿ-ಸ-ಬೇಡಿ
ಪ್ರಯ-ತ್ನಿ-ಸಲಿ
ಪ್ರಯ-ತ್ನಿ-ಸ-ಲಿ-ರು-ವಂತೆ
ಪ್ರಯ-ತ್ನಿಸಿ
ಪ್ರಯ-ತ್ನಿ-ಸಿದ
ಪ್ರಯ-ತ್ನಿ-ಸಿ-ದರು
ಪ್ರಯ-ತ್ನಿ-ಸಿ-ದರೂ
ಪ್ರಯ-ತ್ನಿ-ಸಿ-ದರೆ
ಪ್ರಯ-ತ್ನಿ-ಸಿ-ದಳು
ಪ್ರಯ-ತ್ನಿ-ಸಿ-ದ-ವರ
ಪ್ರಯ-ತ್ನಿ-ಸಿದೆ
ಪ್ರಯ-ತ್ನಿ-ಸಿ-ದ್ದರು
ಪ್ರಯ-ತ್ನಿ-ಸಿ-ದ್ದಾನೆ
ಪ್ರಯ-ತ್ನಿ-ಸಿ-ದ್ದೇ-ನೆಂಬ
ಪ್ರಯ-ತ್ನಿಸು
ಪ್ರಯ-ತ್ನಿ-ಸು-ತ್ತಾರೆ
ಪ್ರಯ-ತ್ನಿ-ಸು-ತ್ತಿ-ದ್ದರು
ಪ್ರಯ-ತ್ನಿ-ಸು-ತ್ತಿ-ದ್ದರೆ
ಪ್ರಯ-ತ್ನಿ-ಸು-ತ್ತಿದ್ದೇ
ಪ್ರಯ-ತ್ನಿ-ಸು-ತ್ತಿ-ದ್ದೇನೆ
ಪ್ರಯ-ತ್ನಿ-ಸು-ತ್ತಿರು
ಪ್ರಯ-ತ್ನಿ-ಸು-ತ್ತಿ-ರು-ತ್ತಾರೆ
ಪ್ರಯ-ತ್ನಿ-ಸು-ತ್ತಿ-ರು-ವುದನ್ನು
ಪ್ರಯ-ತ್ನಿ-ಸು-ತ್ತೇನೆ
ಪ್ರಯ-ತ್ನಿ-ಸು-ವ-ವರು
ಪ್ರಯ-ತ್ನಿ-ಸು-ವು-ದಿಲ್ಲ
ಪ್ರಯ-ತ್ನಿ-ಸು-ವೆ-ನೆಂದು
ಪ್ರಯ-ತ್ನಿ-ಸೋಣ
ಪ್ರಯಾಣ
ಪ್ರಯಾ-ಣ-ಕಾ-ಲ-ದಲ್ಲಿ
ಪ್ರಯಾ-ಣ-ಕ್ಕಾಗಿ
ಪ್ರಯಾ-ಣಕ್ಕೆ
ಪ್ರಯಾ-ಣ-ಗಳ
ಪ್ರಯಾ-ಣದ
ಪ್ರಯಾ-ಣ-ದಿಂದ
ಪ್ರಯಾ-ಣ-ದಿಂ-ದಾಗಿ
ಪ್ರಯಾ-ಣ-ದು-ದ್ದಕ್ಕೂ
ಪ್ರಯಾ-ಣ-ವನ್ನು
ಪ್ರಯಾ-ಣವು
ಪ್ರಯಾ-ಣವೇ
ಪ್ರಯಾ-ಣಿಕ
ಪ್ರಯಾ-ಣಿ-ಕ-ನಿಗೆ
ಪ್ರಯಾ-ಣಿ-ಕರ
ಪ್ರಯಾ-ಣಿ-ಕ-ರಿಗೆ
ಪ್ರಯಾ-ಣಿ-ಕ-ರೊಂ-ದಿಗೆ
ಪ್ರಯಾ-ಣಿ-ಸ-ಬೇಕು
ಪ್ರಯಾ-ಸ-ಕರ
ಪ್ರಯಾ-ಸ-ಕ-ರವೂ
ಪ್ರಯೋ
ಪ್ರಯೋಗ
ಪ್ರಯೋ-ಗ-ಶಾ-ಲೆ-ಯೊಂ-ದನ್ನು
ಪ್ರಯೋಗಿ
ಪ್ರಯೋ-ಗಿ-ಸ-ಬ-ಹು-ದಾ-ದಂ-ತಹ
ಪ್ರಯೋ-ಗಿ-ಸಿದ್ದು
ಪ್ರಯೋ-ಜನ
ಪ್ರಯೋ-ಜ-ನಕ್ಕೆ
ಪ್ರಯೋ-ಜ-ನ-ಗ-ಳಾ-ಗಿ-ದ್ದುವು
ಪ್ರಯೋ-ಜ-ನ-ವನ್ನು
ಪ್ರಯೋ-ಜ-ನ-ವಾ-ಗದೆ
ಪ್ರಯೋ-ಜ-ನ-ವಾ-ಗ-ಲಿಲ್ಲ
ಪ್ರಯೋ-ಜ-ನ-ವಾ-ಗಿದೆ
ಪ್ರಯೋ-ಜ-ನ-ವಾ-ಗು-ತ್ತಿಲ್ಲ
ಪ್ರಯೋ-ಜ-ನ-ವಾ-ದೀತು
ಪ್ರಯೋ-ಜ-ನ-ವಾ-ಯಿತು
ಪ್ರಯೋ-ಜ-ನ-ವಿ-ದೆಯೆ
ಪ್ರಯೋ-ಜ-ನ-ವಿ-ರ-ಲಿಲ್ಲ
ಪ್ರಯೋ-ಜ-ನ-ವಿಲ್ಲ
ಪ್ರಯೋ-ಜ-ನ-ವಿ-ಲ್ಲ-ಅ-ಥವಾ
ಪ್ರಯೋ-ಜ-ನ-ವಿ-ಲ್ಲ-ದಂ-ತಾ-ಯಿ-ತಲ್ಲ
ಪ್ರಯೋ-ಜ-ನ-ವಿ-ಲ್ಲ-ವೆಂದು
ಪ್ರಯೋ-ಜ-ನವೂ
ಪ್ರಯೋ-ಜ-ವಾ-ಗು-ವಂ-ತಿಲ್ಲ
ಪ್ರಲಾಪ
ಪ್ರಲೋ-ಭ-ನೆಗೆ
ಪ್ರಲೋ-ಭ-ನೆ-ಗೊ-ಳಿ-ಸಲೂ
ಪ್ರಲೋ-ಭ-ನೆ-ಗೊ-ಳಿಸಿ
ಪ್ರವ-ಚನ
ಪ್ರವ-ಚ-ನ-ಗಳ
ಪ್ರವ-ಚ-ನ-ಗಳನ್ನು
ಪ್ರವ-ಚ-ನ-ಗಳಲ್ಲಿ
ಪ್ರವ-ಚ-ನ-ವೊಂ-ದನ್ನು
ಪ್ರವ-ಚ-ನಾ-ದಿ-ಗಳ
ಪ್ರವ-ರ-ವನ್ನು
ಪ್ರವ-ರ್ತ-ನಾ-ಚಾ-ರ್ಯ-ನಾಗಿ
ಪ್ರವ-ರ್ತ-ನಾ-ಚಾ-ರ್ಯ-ರೆಂದು
ಪ್ರವ-ಹಿ-ಸ-ಬೇ-ಕಾ-ದರೆ
ಪ್ರವ-ಹಿ-ಸಿತು
ಪ್ರವ-ಹಿಸು
ಪ್ರವಾದಿ
ಪ್ರವಾ-ದಿ-ಗಳ
ಪ್ರವಾ-ದಿ-ಗ-ಳಿಗೆ
ಪ್ರವಾ-ದಿಯ
ಪ್ರವಾ-ದಿ-ಯಂತೆ
ಪ್ರವಾ-ದಿ-ಯನ್ನು
ಪ್ರವಾ-ದಿ-ಯಾಗಿ
ಪ್ರವಾ-ದಿ-ಯಾ-ಗಿ-ದ್ದರು
ಪ್ರವಾ-ದಿ-ಯೆಂದು
ಪ್ರವಾ-ದಿ-ಯೊ-ಬ್ಬನು
ಪ್ರವಾಸ
ಪ್ರವಾ-ಸಕ್ಕೆ
ಪ್ರವಾ-ಸ-ಗಳು
ಪ್ರವಾ-ಸದ
ಪ್ರವಾ-ಸ-ದಿಂ-ದಾಗಿ
ಪ್ರವಾ-ಸ-ವನ್ನು
ಪ್ರವಾ-ಸ-ವಾ-ಗ-ಬ-ಹುದು
ಪ್ರವಾ-ಸ-ವಾಗಿ
ಪ್ರವಾ-ಸವು
ಪ್ರವಾ-ಸವೂ
ಪ್ರವಾ-ಸಿ-ಗರ
ಪ್ರವಾ-ಸಿ-ಗರು
ಪ್ರವಾಸೀ
ಪ್ರವಾಹ
ಪ್ರವಾ-ಹ-ವನ್ನು
ಪ್ರವಾ-ಹ-ವನ್ನೂ
ಪ್ರವಾ-ಹ-ವೊಂದು
ಪ್ರವೃ-ತ್ತ-ವಾ-ಗಿ-ಸಲು
ಪ್ರವೃತ್ತಿ
ಪ್ರವೃ-ತ್ತಿ-ಗಳ
ಪ್ರವೃ-ತ್ತಿ-ಗಳನ್ನು
ಪ್ರವೃ-ತ್ತಿಯ
ಪ್ರವೃ-ತ್ತಿ-ಯನ್ನು
ಪ್ರವೃ-ತ್ತಿ-ಯಲ್ಲಿ
ಪ್ರವೃ-ತ್ತಿ-ಯ-ವರೂ
ಪ್ರವೃ-ತ್ತಿ-ಯಿ-ಲ್ಲ-ದಿ-ರು-ವು-ದು-ಇದೇ
ಪ್ರವೃ-ತ್ತಿಯು
ಪ್ರವೃ-ತ್ತಿಯೂ
ಪ್ರವೇಶ
ಪ್ರವೇ-ಶಕ್ಕೆ
ಪ್ರವೇ-ಶ-ದ್ವಾ-ರದ
ಪ್ರವೇ-ಶ-ಧ-ನ-ವನ್ನು
ಪ್ರವೇ-ಶ-ವನ್ನು
ಪ್ರವೇ-ಶ-ವಿ-ರಲು
ಪ್ರವೇ-ಶ-ವಿಲ್ಲ
ಪ್ರವೇ-ಶಿ-ಸ-ದಂತೆ
ಪ್ರವೇ-ಶಿ-ಸಲು
ಪ್ರವೇ-ಶಿಸಿ
ಪ್ರವೇ-ಶಿ-ಸಿದ
ಪ್ರವೇ-ಶಿ-ಸಿ-ದಂತೆ
ಪ್ರವೇ-ಶಿ-ಸಿ-ದರು
ಪ್ರವೇ-ಶಿ-ಸಿ-ದ-ವರು
ಪ್ರವೇ-ಶಿ-ಸಿದೆ
ಪ್ರವೇ-ಶಿ-ಸಿ-ದ್ದರೂ
ಪ್ರವೇ-ಶಿ-ಸು-ತ್ತ-ವೆಯೋ
ಪ್ರವೇ-ಶಿ-ಸು-ತ್ತಿ-ರು-ವಂತೆ
ಪ್ರವೇ-ಶಿ-ಸು-ತ್ತಿ-ರು-ವ-ವ-ರಿ-ಗೆ-ಅಷ್ಟು
ಪ್ರವೇ-ಶಿ-ಸುವ
ಪ್ರವೇ-ಶಿ-ಸು-ವಂ-ತೆಯೇ
ಪ್ರವೇ-ಶಿ-ಸು-ವುದ
ಪ್ರಶಂ-ಶೆ-ಯನ್ನು
ಪ್ರಶಂ-ಸ-ನೀ-ಯವೇ
ಪ್ರಶಂ-ಸಾ-ಪತ್ರ
ಪ್ರಶಂ-ಸಿ-ಸ-ಲಾ-ಗು-ತ್ತಿದೆ
ಪ್ರಶಂ-ಸಿ-ಸ-ಲಾ-ಯಿ-ತು-ಒಂದು
ಪ್ರಶಂ-ಸಿ-ಸಲು
ಪ್ರಶಂ-ಸಿಸಿ
ಪ್ರಶಂ-ಸಿ-ಸಿತು
ಪ್ರಶಂ-ಸಿ-ಸಿ-ದರು
ಪ್ರಶಂ-ಸಿ-ಸಿ-ದುವು
ಪ್ರಶಂ-ಸಿ-ಸಿ-ರು-ವುದನ್ನು
ಪ್ರಶಂ-ಸಿ-ಸು-ತ್ತಿ-ದ್ದರು
ಪ್ರಶಂ-ಸೆ-ಗಳನ್ನು
ಪ್ರಶಂ-ಸೆ-ಗಾ-ಗಲಿ
ಪ್ರಶಂ-ಸೆಗೆ
ಪ್ರಶಂ-ಸೆಯ
ಪ್ರಶಂ-ಸೆ-ಯನ್ನು
ಪ್ರಶಂ-ಸೆ-ಯನ್ನೂ
ಪ್ರಶಂ-ಸೆಯೂ
ಪ್ರಶ-ಸ್ತ-ವಾದ
ಪ್ರಶಾಂತ
ಪ್ರಶಾಂ-ತ
ಪ್ರಶಾಂ-ತ-ಏ-ಕಾಂತ
ಪ್ರಶಾಂ-ತ-ಕ-ರು-ಣಾ-ಪೂರ್ಣ
ಪ್ರಶಾಂ-ತ-ತೆಗೆ
ಪ್ರಶಾಂ-ತ-ಮೂರ್ತಿ
ಪ್ರಶಾಂ-ತವೂ
ಪ್ರಶ್ನಾ-ತೀ-ತ-ವಾ-ಗಿ-ರು-ತ್ತಿ-ದ್ದು-ವೆಂದರೆ
ಪ್ರಶ್ನಿ-ಸ-ತೊ-ಡ-ಗಿ-ದರು
ಪ್ರಶ್ನಿ-ಸ-ಲಾಗಿ
ಪ್ರಶ್ನಿ-ಸ-ಲಾ-ಗಿತ್ತು
ಪ್ರಶ್ನಿ-ಸ-ಲಿಲ್ಲ
ಪ್ರಶ್ನಿ-ಸಲು
ಪ್ರಶ್ನಿ-ಸಿದ
ಪ್ರಶ್ನಿ-ಸಿ-ದರು
ಪ್ರಶ್ನಿ-ಸಿ-ದಾಗ
ಪ್ರಶ್ನಿ-ಸಿ-ದುದು
ಪ್ರಶ್ನಿ-ಸಿ-ದ್ದರು
ಪ್ರಶ್ನಿಸು
ಪ್ರಶ್ನಿ-ಸು-ತ್ತಿ-ದ್ದರು
ಪ್ರಶ್ನಿ-ಸು-ತ್ತಿ-ದ್ದಳು
ಪ್ರಶ್ನಿ-ಸು-ತ್ತಿ-ದ್ದಾನೆ
ಪ್ರಶ್ನಿ-ಸು-ವಂ-ತಿ-ರ-ಲಿಲ್ಲ
ಪ್ರಶ್ನಿ-ಸು-ವಂ-ತಿಲ್ಲ
ಪ್ರಶ್ನಿ-ಸು-ವು-ದೆಂ-ದರೆ
ಪ್ರಶ್ನೆ
ಪ್ರಶ್ನೆ-ಗಳ
ಪ್ರಶ್ನೆ-ಗಳನ್ನು
ಪ್ರಶ್ನೆ-ಗಳನ್ನೆಲ್ಲ
ಪ್ರಶ್ನೆ-ಗಳಲ್ಲಿ
ಪ್ರಶ್ನೆ-ಗಳಿಂದ
ಪ್ರಶ್ನೆ-ಗ-ಳಿಗೂ
ಪ್ರಶ್ನೆ-ಗ-ಳಿಗೆ
ಪ್ರಶ್ನೆ-ಗ-ಳಿ-ಗೆಲ್ಲ
ಪ್ರಶ್ನೆ-ಗಳು
ಪ್ರಶ್ನೆ-ಗ-ಳೆಲ್ಲ
ಪ್ರಶ್ನೆ-ಗಳೇ
ಪ್ರಶ್ನೆಗೂ
ಪ್ರಶ್ನೆಗೆ
ಪ್ರಶ್ನೆಯ
ಪ್ರಶ್ನೆ-ಯ-ನ್ನಾ-ದರೂ
ಪ್ರಶ್ನೆ-ಯನ್ನು
ಪ್ರಶ್ನೆ-ಯ-ನ್ನೇನೋ
ಪ್ರಶ್ನೆಯೇ
ಪ್ರಶ್ನೆ-ಯೇನು
ಪ್ರಶ್ನೆ-ಯೇ-ನೆಂ-ದರೆ
ಪ್ರಶ್ನೆ-ಯೇ-ಳ-ಬ-ಹುದು
ಪ್ರಶ್ನೆ-ಯೇ-ಳು-ತ್ತದೆ
ಪ್ರಶ್ನೆ-ಯೊಂ-ದಕ್ಕೆ
ಪ್ರಶ್ನೋ-ತ್ತರ
ಪ್ರಶ್ನೋ-ತ್ತ-ರ-ಗಳ
ಪ್ರಶ್ನೋ-ತ್ತ-ರ-ಗಳು
ಪ್ರಶ್ನೋ-ತ್ತ-ರದ
ಪ್ರಸಂ-ಗ-ಗಳ
ಪ್ರಸಂ-ಗ-ಗ-ಳಿಂ-ದಾಗಿ
ಪ್ರಸಂ-ಗ-ಗ-ಳೊ-ದಗಿ
ಪ್ರಸಂ-ಗ-ವೊಂ-ದನ್ನು
ಪ್ರಸನ್ನ
ಪ್ರಸ-ರಣ
ಪ್ರಸಾರ
ಪ್ರಸಾ-ರ-ಕ-ನಾಗಿ
ಪ್ರಸಾ-ರ-ಕಾ-ರ್ಯ-ಕ್ಕಾಗಿ
ಪ್ರಸಾ-ರ-ಕಾ-ರ್ಯ-ದಲ್ಲಿ
ಪ್ರಸಾ-ರ-ಕ್ಕಾಗಿ
ಪ್ರಸಾ-ರಕ್ಕೆ
ಪ್ರಸಾ-ರ-ಗೈ-ಯ-ಬೇ-ಕಾದ
ಪ್ರಸಾ-ರ-ಗೊಂಡು
ಪ್ರಸಾ-ರ-ಗೊ-ಳ್ಳಲು
ಪ್ರಸಾ-ರ-ಗೊ-ಳ್ಳು-ತ್ತದೆ
ಪ್ರಸಾ-ರದ
ಪ್ರಸಾ-ರ-ದಲ್ಲಿ
ಪ್ರಸಾ-ರ-ಮಾಡಿ
ಪ್ರಸಾ-ರ-ವಾ-ಗ-ಬೇ-ಕೆಂ
ಪ್ರಸಾ-ರ-ವಾಗಿ
ಪ್ರಸಾ-ರ-ವಾ-ಗಿತ್ತು
ಪ್ರಸಾ-ರ-ವಾ-ಗಿ-ರ-ಲಿಲ್ಲ
ಪ್ರಸಾ-ರ-ವಾ-ಗು-ವಂತೆ
ಪ್ರಸಾ-ರ-ವಾ-ಗು-ವು-ದ-ಕ್ಕಾಗಿ
ಪ್ರಸಾ-ರ-ವಾ-ಯಿತು
ಪ್ರಸಾ-ರ-ವಿನ್ನೂ
ಪ್ರಸಿದ್ಧ
ಪ್ರಸಿ-ದ್ಧ-ನಾ-ಗಿದ್ದ
ಪ್ರಸಿ-ದ್ಧ-ನಾ-ದ-ನ-ಲ್ಲದೆ
ಪ್ರಸಿ-ದ್ಧ-ನಾ-ದ-ವನು
ಪ್ರಸಿ-ದ್ಧನೂ
ಪ್ರಸಿ-ದ್ಧ-ರಾ-ಗ-ಬ-ಲ್ಲ-ರೆಂದು
ಪ್ರಸಿ-ದ್ಧ-ರಾ-ಗು-ತ್ತಿ-ದ್ದರೋ
ಪ್ರಸಿ-ದ್ಧ-ರಾದ
ಪ್ರಸಿ-ದ್ಧ-ರಾ-ದರು
ಪ್ರಸಿ-ದ್ಧ-ಳಾ-ದಳು
ಪ್ರಸಿ-ದ್ಧ-ವಾ-ಗಿತ್ತು
ಪ್ರಸಿ-ದ್ಧ-ವಾ-ಗಿದೆ
ಪ್ರಸಿ-ದ್ಧ-ವಾ-ಗಿದ್ದ
ಪ್ರಸಿ-ದ್ಧ-ವಾದ
ಪ್ರಸಿ-ದ್ಧವೂ
ಪ್ರಸ್ತಾಪ
ಪ್ರಸ್ತಾ-ಪ-ಗೊ-ಳ್ಳು-ತ್ತಿ-ದ್ದುವು
ಪ್ರಸ್ತಾ-ಪವೂ
ಪ್ರಸ್ತಾಪಿ
ಪ್ರಸ್ತಾ-ಪಿಸ
ಪ್ರಸ್ತಾ-ಪಿ-ಸ-ದಿದ್ದ
ಪ್ರಸ್ತಾ-ಪಿ-ಸ-ಲಾಗಿದೆ
ಪ್ರಸ್ತಾ-ಪಿಸಿ
ಪ್ರಸ್ತಾ-ಪಿ-ಸಿದ
ಪ್ರಸ್ತಾ-ಪಿ-ಸಿ-ದರು
ಪ್ರಸ್ತಾ-ಪಿ-ಸಿ-ದರೂ
ಪ್ರಸ್ತಾ-ಪಿ-ಸಿ-ದರೆ
ಪ್ರಸ್ತಾ-ಪಿ-ಸಿ-ದಾಗ
ಪ್ರಸ್ತಾ-ಪಿ-ಸಿ-ದ್ದ-ರು-ಮುಂ-ದಿನ
ಪ್ರಸ್ತಾ-ಪಿಸು
ಪ್ರಸ್ತಾ-ಪಿ-ಸುತ್ತ
ಪ್ರಸ್ತಾ-ಪಿ-ಸು-ತ್ತಾರೆ
ಪ್ರಸ್ತಾ-ಪಿ-ಸು-ತ್ತಿ-ದ್ದ-ರಾ-ದರೂ
ಪ್ರಸ್ತಾ-ವನಾ
ಪ್ರಸ್ತಾ-ವ-ನೆ-ಯಲ್ಲಿ
ಪ್ರಸ್ತುತ
ಪ್ರಸ್ಥಾನೇ
ಪ್ರಹಾ-ರ-ಗಳನ್ನು
ಪ್ರಹಾ-ರ-ಗಳು
ಪ್ರಾಂತ-ಗಳ
ಪ್ರಾಂತ-ಗ-ಳಾಗಿ
ಪ್ರಾಂತದ
ಪ್ರಾಂತ-ದತ್ತ
ಪ್ರಾಂತ-ದಲ್ಲಿ
ಪ್ರಾಂತ-ದ-ಲ್ಲಿ-ರುವ
ಪ್ರಾಂತ-ದಲ್ಲೆಲ್ಲ
ಪ್ರಾಂತ-ದಲ್ಲೇ
ಪ್ರಾಂತ್ಯ-ಗಳ
ಪ್ರಾಂತ್ಯ-ಗಳಲ್ಲಿ
ಪ್ರಾಂತ್ಯ-ಗಳಿಂದ
ಪ್ರಾಂತ್ಯದ
ಪ್ರಾಂತ್ಯ-ದಲ್ಲಿ
ಪ್ರಾಂತ್ಯ-ದಲ್ಲೂ
ಪ್ರಾಂತ್ಯ-ದಲ್ಲೇ
ಪ್ರಾಚಾ-ರ್ಯರು
ಪ್ರಾಚೀನ
ಪ್ರಾಚೀ-ನ-ತ-ಮ-ಜೀ-ವಂತ
ಪ್ರಾಚೀ-ನ-ವಾದ
ಪ್ರಾಚ್ಯ
ಪ್ರಾಚ್ಯ
ಪ್ರಾಚ್ಯ-ಪಾ-ಶ್ಚಾತ್ಯ
ಪ್ರಾಚ್ಯ-ಪಾ-ಶ್ಚಾ-ತ್ಯ-ಗಳ
ಪ್ರಾಟೆ-ಸ್ಟೆಂ-ಟರು
ಪ್ರಾಟೆ-ಸ್ಟೆಂಟ್
ಪ್ರಾಣ
ಪ್ರಾಣ-ತ್ಯಾಗ
ಪ್ರಾಣ-ಪಕ್ಷಿ
ಪ್ರಾಣ-ಪ್ರಭು
ಪ್ರಾಣ-ಪ್ರಾ-ಯ-ವಾ-ದುದು
ಪ್ರಾಣ-ವನ್ನು
ಪ್ರಾಣ-ವನ್ನೇ
ಪ್ರಾಣವು
ಪ್ರಾಣವೇ
ಪ್ರಾಣ-ಸಖ
ಪ್ರಾಣಿ
ಪ್ರಾಣಿ-ಗಳ
ಪ್ರಾಣಿ-ಗಳಿಂದ
ಪ್ರಾಣಿ-ಗಳು
ಪ್ರಾಣಿ-ವನ
ಪ್ರಾತಃ-ಕಾಲ
ಪ್ರಾತಿ-ನಿ-ಧಿಕ
ಪ್ರಾತಿ-ನಿಧ್ಯ
ಪ್ರಾಥ-ಮಿಕ
ಪ್ರಾದೇ-ಶಿಕ
ಪ್ರಾಧಾ-ನ್ಯ-ವನ್ನು
ಪ್ರಾಧ್ಯಾ-ಪ-ಕ-ರಾ-ಗಿದ್ದ
ಪ್ರಾಧ್ಯಾ-ಪ-ಕರು
ಪ್ರಾಧ್ಯಾ-ಪ-ಕರೂ
ಪ್ರಾಪಂ-ಚಿಕ
ಪ್ರಾಪಂ-ಚಿ-ಕರ
ಪ್ರಾಪಂ-ಚಿ-ಕರು
ಪ್ರಾಪ್ತ
ಪ್ರಾಪ್ತ-ವ-ಯ-ಸ್ಕ-ನಾಗಿ
ಪ್ರಾಮಾ-ಣಿಕ
ಪ್ರಾಮಾ-ಣಿ-ಕ-ವಿ-ಶ್ವಾ-ಸ-ಪಾತ್ರ
ಪ್ರಾಮಾ-ಣಿ-ಕತೆ
ಪ್ರಾಮಾ-ಣಿ-ಕ-ತೆ-ಗಳನ್ನು
ಪ್ರಾಮಾ-ಣಿ-ಕ-ತೆಯ
ಪ್ರಾಮಾ-ಣಿ-ಕ-ತೆ-ಯನ್ನು
ಪ್ರಾಮಾ-ಣಿ-ಕ-ತೆ-ಯನ್ನೂ
ಪ್ರಾಮಾ-ಣಿ-ಕ-ತೆ-ಯಿಂದ
ಪ್ರಾಮಾ-ಣಿ-ಕ-ನಾ-ಗಿ-ದ್ದಂತೆ
ಪ್ರಾಮಾ-ಣಿ-ಕನೂ
ಪ್ರಾಮಾ-ಣಿ-ಕ-ನೆಂ-ಬು-ದನ್ನು
ಪ್ರಾಮಾ-ಣಿ-ಕ-ರಾ-ಗಿ-ರುವ
ಪ್ರಾಮಾ-ಣಿ-ಕ-ರಾದ
ಪ್ರಾಮಾ-ಣಿ-ಕರು
ಪ್ರಾಮಾ-ಣಿ-ಕಳು
ಪ್ರಾಮಾ-ಣಿ-ಕ-ವಾಗಿ
ಪ್ರಾಮಾ-ಣಿ-ಕ-ವಾ-ಗಿಯೇ
ಪ್ರಾಮಾ-ಣಿ-ಕ-ವಾದ
ಪ್ರಾಮುಖ್ಯ
ಪ್ರಾಮು-ಖ್ಯತೆ
ಪ್ರಾಮು-ಖ್ಯ-ತೆಯ
ಪ್ರಾಮು-ಖ್ಯ-ದ್ದಾಗಿ
ಪ್ರಾಮು-ಖ್ಯ-ವಾದ
ಪ್ರಾಯ-ಗಳು
ಪ್ರಾಯಶಃ
ಪ್ರಾಯ-ಶ್ಚಿ-ತ್ತಾ-ರ್ಥ-ವಾಗಿ
ಪ್ರಾಯೋ-ಗಿ-ಕ-ವಾಗಿ
ಪ್ರಾಯೋ-ಪ-ವೇಶ
ಪ್ರಾರಂಭ
ಪ್ರಾರಂ-ಭ-ಗೊಂ-ಡಿತು
ಪ್ರಾರಂ-ಭದ
ಪ್ರಾರಂ-ಭ-ದಲ್ಲಿ
ಪ್ರಾರಂ-ಭ-ದಿಂ-ದಲೂ
ಪ್ರಾರಂ-ಭ-ದಿಂ-ದೀ-ಚೆಗೆ
ಪ್ರಾರಂ-ಭ-ಮಾ-ಡಿದ್ದಾ
ಪ್ರಾರಂ-ಭ-ವನ್ನು
ಪ್ರಾರಂ-ಭ-ವಾ-ಗ-ಬೇ-ಕಾ-ಗಿತ್ತು
ಪ್ರಾರಂ-ಭ-ವಾಗಿ
ಪ್ರಾರಂ-ಭ-ವಾ-ಗಿತ್ತು
ಪ್ರಾರಂ-ಭ-ವಾ-ಗಿ-ರ-ಲಿಲ್ಲ
ಪ್ರಾರಂ-ಭ-ವಾ-ಗಿಲ್ಲ
ಪ್ರಾರಂ-ಭ-ವಾ-ಗು-ತ್ತದೆ
ಪ್ರಾರಂ-ಭ-ವಾ-ಗು-ತ್ತಿತ್ತು
ಪ್ರಾರಂ-ಭ-ವಾ-ಗು-ತ್ತಿ-ದ್ದಂ-ತೆಯೇ
ಪ್ರಾರಂ-ಭ-ವಾ-ಗು-ವುದು
ಪ್ರಾರಂ-ಭ-ವಾ-ಗು-ವು-ದೆಂದು
ಪ್ರಾರಂ-ಭ-ವಾದ
ಪ್ರಾರಂ-ಭ-ವಾ-ದದ್ದು
ಪ್ರಾರಂ-ಭ-ವಾ-ದವು
ಪ್ರಾರಂ-ಭ-ವಾ-ದಾಗ
ಪ್ರಾರಂ-ಭ-ವಾ-ದೀ-ತೆಂದು
ಪ್ರಾರಂ-ಭ-ವಾ-ದುವು
ಪ್ರಾರಂ-ಭ-ವಾ-ಯಿತು
ಪ್ರಾರಂಭಿ
ಪ್ರಾರಂ-ಭಿ-ಸ-ಬಾ-ರದು
ಪ್ರಾರಂ-ಭಿ-ಸ-ಲಾ-ಗಿತ್ತು
ಪ್ರಾರಂ-ಭಿ-ಸ-ಲಾ-ಯಿತು
ಪ್ರಾರಂ-ಭಿ-ಸಲು
ಪ್ರಾರಂ-ಭಿಸಿ
ಪ್ರಾರಂ-ಭಿ-ಸಿತು
ಪ್ರಾರಂ-ಭಿ-ಸಿದ
ಪ್ರಾರಂ-ಭಿ-ಸಿ-ದಂ-ತಹ
ಪ್ರಾರಂ-ಭಿ-ಸಿ-ದ-ರ-ಲ್ಲದೆ
ಪ್ರಾರಂ-ಭಿ-ಸಿ-ದರು
ಪ್ರಾರಂ-ಭಿ-ಸಿ-ದಾಗ
ಪ್ರಾರಂ-ಭಿ-ಸಿದ್ದ
ಪ್ರಾರಂ-ಭಿ-ಸಿ-ದ್ದರು
ಪ್ರಾರಂ-ಭಿ-ಸಿ-ದ್ದಾರೆ
ಪ್ರಾರಂ-ಭಿ-ಸಿ-ಬಿ-ಟ್ಟಿ-ದ್ದಾರೆ
ಪ್ರಾರಂ-ಭಿ-ಸಿ-ಬಿಡು
ಪ್ರಾರಂ-ಭಿ-ಸಿ-ರ-ಲಿ-ಲ್ಲ-ಭ-ಗ-ವಂತ
ಪ್ರಾರಂ-ಭಿ-ಸಿ-ರುವ
ಪ್ರಾರಂ-ಭಿ-ಸು-ತ್ತಾರೆ
ಪ್ರಾರಂ-ಭಿ-ಸು-ತ್ತಿತ್ತು
ಪ್ರಾರಂ-ಭಿ-ಸು-ತ್ತೇನೆ
ಪ್ರಾರಂ-ಭಿ-ಸುವ
ಪ್ರಾರಂ-ಭಿ-ಸು-ವಂತೆ
ಪ್ರಾರಂ-ಭಿ-ಸು-ವುದನ್ನು
ಪ್ರಾರಂ-ಭಿ-ಸೋಣ
ಪ್ರಾರ್ಥನಾ
ಪ್ರಾರ್ಥನೆ
ಪ್ರಾರ್ಥ-ನೆ-ಧ್ಯಾ-ನ-ಗಳ
ಪ್ರಾರ್ಥ-ನೆ-ಗ-ಳೊಂ-ದಿಗೆ
ಪ್ರಾರ್ಥ-ನೆಗೆ
ಪ್ರಾರ್ಥ-ನೆ-ಯನ್ನು
ಪ್ರಾರ್ಥ-ನೆ-ಯೊಂದು
ಪ್ರಾರ್ಥಿ-ಸ-ಲಾ-ರಂ-ಭಿ-ಸಿ-ದರು
ಪ್ರಾರ್ಥಿಸಿ
ಪ್ರಾರ್ಥಿ-ಸಿಕೊ
ಪ್ರಾರ್ಥಿ-ಸಿ-ಕೊಂಡ
ಪ್ರಾರ್ಥಿ-ಸಿ-ಕೊಂ-ಡರು
ಪ್ರಾರ್ಥಿ-ಸಿ-ಕೊಂ-ಡ-ವರು
ಪ್ರಾರ್ಥಿ-ಸಿ-ಕೊಂಡಿ
ಪ್ರಾರ್ಥಿ-ಸಿ-ಕೊಂ-ಡಿದ್ದ
ಪ್ರಾರ್ಥಿ-ಸಿ-ಕೊಂಡು
ಪ್ರಾರ್ಥಿ-ಸಿ-ಕೊ-ಳ್ಳು-ತ್ತಲೇ
ಪ್ರಾರ್ಥಿ-ಸಿ-ಕೊ-ಳ್ಳು-ತ್ತೇವೆ
ಪ್ರಾರ್ಥಿ-ಸಿ-ಕೊ-ಳ್ಳು-ವಂತೆ
ಪ್ರಾರ್ಥಿ-ಸಿ-ದರು
ಪ್ರಾರ್ಥಿ-ಸು-ತ್ತಿ-ರು-ತ್ತೇನೆ
ಪ್ರಾರ್ಥಿ-ಸು-ತ್ತೇನೆ
ಪ್ರಾರ್ಥಿ-ಸುವ
ಪ್ರಾಶಸ್ತ್ಯ
ಪ್ರಾಶ್ನಿ-ಕನ
ಪ್ರಾಶ್ನಿ-ಕ-ನನ್ನೇ
ಪ್ರಾಶ್ನಿ-ಕನೂ
ಪ್ರಾಶ್ನಿ-ಕರ
ಪ್ರಿನ್ಸಸ್
ಪ್ರಿನ್ಸಿ-ಪಾ-ಲರಾ
ಪ್ರಿನ್ಸಿ-ಪಾ-ಲ-ಳಾ-ಗಿ-ದ್ದಳು
ಪ್ರಿನ್ಸ್
ಪ್ರಿಯ
ಪ್ರಿಯ-ತ-ಮಈ
ಪ್ರಿಯ-ತ-ಮ-ನಾದ
ಪ್ರಿಯ-ತೆಯ
ಪ್ರಿಯ-ನಾ-ದದ್ದು
ಪ್ರಿಯ-ನಾ-ದ-ವನು
ಪ್ರಿಯ-ವಾಗಿ
ಪ್ರಿಯ-ವಾ-ಗಿತ್ತು
ಪ್ರಿಯ-ವಾ-ಗಿ-ರು-ತ್ತಿತ್ತು
ಪ್ರಿಯ-ವಾ-ಗಿ-ಸಿತು
ಪ್ರಿಯ-ವಾ-ಗು-ವಂ-ತಹ
ಪ್ರಿಯ-ವಾದ
ಪ್ರಿಯ-ವಾ-ದದ್ದೇ
ಪ್ರಿಯವೂ
ಪ್ರಿಯ-ಶಿಷ್ಯ
ಪ್ರಿಸ್ಟಿ-ಟೇ-ರಿ-ಯ-ನ್ನರು
ಪ್ರೀತಿ
ಪ್ರೀತಿ-ಗೌ-ರವ
ಪ್ರೀತಿ-ಸ-ಹಾ-ನು-ಭೂತಿ
ಪ್ರೀತಿ-ಸ-ಹಾ-ನು-ಭೂ-ತಿ-ಗಳು
ಪ್ರೀತಿಗೆ
ಪ್ರೀತಿ-ಪಾ-ತ್ರ-ರ-ನ್ನಾ-ಗಿ-ಸಿತ್ತು
ಪ್ರೀತಿ-ಪಾ-ತ್ರ-ರಾದ
ಪ್ರೀತಿ-ಪಾ-ತ್ರ-ವಾದ
ಪ್ರೀತಿ-ಪೂ-ರ್ವಕ
ಪ್ರೀತಿ-ಪೂ-ರ್ವ-ಕ-ವಾಗಿ
ಪ್ರೀತಿ-ಭಾ-ವವೂ
ಪ್ರೀತಿಯ
ಪ್ರೀತಿ-ಯನ್ನು
ಪ್ರೀತಿ-ಯಾ-ದರೂ
ಪ್ರೀತಿ-ಯಿಂದ
ಪ್ರೀತಿ-ಯಿಂ-ದಾಗಿ
ಪ್ರೀತಿ-ಯಿ-ದೆಯೆ
ಪ್ರೀತಿ-ಯಿ-ರಲಿ
ಪ್ರೀತಿಯು
ಪ್ರೀತಿ-ಯು-ತ-ರಾ-ಗಿ-ರು-ತ್ತಿ-ದ್ದರು
ಪ್ರೀತಿ-ಯು-ಳ್ಳ-ವನೆ
ಪ್ರೀತಿ-ಯು-ಳ್ಳ-ವರು
ಪ್ರೀತಿ-ಯೆಂದೂ
ಪ್ರೀತಿ-ಯೇನೂ
ಪ್ರೀತಿ-ವಿ-ಶ್ವಾಸ
ಪ್ರೀತಿ-ವಿ-ಶ್ವಾ-ಸ-ಗಳು
ಪ್ರೀತಿ-ಸ-ದಿ-ರಲು
ಪ್ರೀತಿ-ಸ-ಬಲ್ಲೆ
ಪ್ರೀತಿ-ಸ-ಲಾ-ರಂ-ಭಿ-ಸಿದೆ
ಪ್ರೀತಿ-ಸಲಿ
ಪ್ರೀತಿ-ಸಲು
ಪ್ರೀತಿಸಿ
ಪ್ರೀತಿಸು
ಪ್ರೀತಿ-ಸು-ತ್ತಾರೆ
ಪ್ರೀತಿ-ಸು-ತ್ತಿ-ದ್ದರು
ಪ್ರೀತಿ-ಸು-ತ್ತಿದ್ದೆ
ಪ್ರೀತಿ-ಸು-ತ್ತೇನೆ
ಪ್ರೀತಿ-ಸು-ವ-ನೋ-ಅ-ವನ
ಪ್ರೀತಿ-ಸು-ವ-ವರು
ಪ್ರೀತಿ-ಸು-ವಿರಾ
ಪ್ರೀತಿ-ಸು-ವೆಯಾ
ಪ್ರೀತ್ಯ-ರ್ಥ-ವಾದ
ಪ್ರೀತ್ಯಾ-ದರ
ಪ್ರೀತ್ಯಾ-ದ-ರ-ಗಳಿಂದ
ಪ್ರೀತ್ಯಾ-ದ-ರ-ವನ್ನು
ಪ್ರೆಸಿ-ಡೆಂಟ್
ಪ್ರೆಸ್
ಪ್ರೆಸ್ಬಿ-ಟೀ-ರಿ-ಯನ್
ಪ್ರೇಕ್ಷ-ಕ-ರಾಗಿ
ಪ್ರೇಕ್ಷ-ಕ-ರಿಗೆ
ಪ್ರೇಕ್ಷ-ಕರು
ಪ್ರೇಕ್ಷ-ಕ-ರೆ-ಲ್ಲರ
ಪ್ರೇಕ್ಷ-ಣೀಯ
ಪ್ರೇತ-ಗಳ
ಪ್ರೇತ-ಗಳನ್ನು
ಪ್ರೇತ-ಗಳು
ಪ್ರೇತ-ಗ-ಳೆಲ್ಲ
ಪ್ರೇತ-ವನ್ನು
ಪ್ರೇತ-ವಿ-ದ್ಯೆಯ
ಪ್ರೇಮ
ಪ್ರೇಮ-ಇ-ವೆ-ರ-ಡನ್ನು
ಪ್ರೇಮದ
ಪ್ರೇಮ-ದೆ-ಡೆಗೆ
ಪ್ರೇಮ-ಮ-ಯ-ವಾ-ಗಿ-ರು-ವಂ-ತಹ
ಪ್ರೇಮ-ವನ್ನು
ಪ್ರೇಮ-ವನ್ನೂ
ಪ್ರೇಮವು
ಪ್ರೇಮವೇ
ಪ್ರೇಮ-ಸ್ವ-ರೂ-ಪ-ನಾದ
ಪ್ರೇರಿ-ತ-ನಾ-ದವ
ಪ್ರೇರಿ-ತ-ರಾಗಿ
ಪ್ರೇರಿ-ತ-ರಾ-ಗು-ತ್ತಿ-ದ್ದರು
ಪ್ರೇರೇ-ಪಿ-ಸು-ತ್ತದೆ
ಪ್ರೇರೇ-ಪಿ-ಸು-ತ್ತಿದೆ
ಪ್ರೇರೇ-ಪಿ-ಸುವ
ಪ್ರೇರೇ-ಪಿ-ಸು-ವುದೇ
ಪ್ರೊ
ಪ್ರೊಫೆ-ಸ-ರರ
ಪ್ರೊಫೆ-ಸ-ರ-ರಾ-ಗಿ-ದ್ದರು
ಪ್ರೊಫೆ-ಸ-ರ-ರಾದ
ಪ್ರೊಫೆ-ಸ-ರ-ರಿ-ಗಿತ್ತು
ಪ್ರೊಫೆ-ಸ-ರ-ರಿಗೆ
ಪ್ರೊಫೆ-ಸ-ರರು
ಪ್ರೊಫೆ-ಸ-ರರೇ
ಪ್ರೊಫೆ-ಸ-ರ-ರೊ-ಬ್ಬರ
ಪ್ರೊಫೆ-ಸ-ರ-ರೊ-ಬ್ಬರು
ಪ್ರೊಫೆ-ಸ-ರು-ಗಳ
ಪ್ರೊಫೆ-ಸ-ರು-ಗಳನ್ನೆಲ್ಲ
ಪ್ರೊಫೆ-ಸ-ರು-ಗಳು
ಪ್ರೊಫೆ-ಸರ್
ಪ್ರೋತ್ಸಾಹ
ಪ್ರೋತ್ಸಾ-ಹಕ
ಪ್ರೋತ್ಸಾ-ಹ-ದಿಂ-ದಲೇ
ಪ್ರೋತ್ಸಾ-ಹ-ವನ್ನು
ಪ್ರೋತ್ಸಾ-ಹ-ವನ್ನೂ
ಪ್ರೋತ್ಸಾಹಿ
ಪ್ರೋತ್ಸಾ-ಹಿಸಿ
ಪ್ರೋತ್ಸಾ-ಹಿ-ಸಿ-ದರು
ಪ್ರೋತ್ಸಾ-ಹಿ-ಸು-ತ್ತಿರ
ಪ್ರೋತ್ಸಾ-ಹಿ-ಸು-ವು-ದರ
ಪ್ರೌಢ
ಪ್ರೌಢವೂ
ಪ್ರೌಢ-ಶಾಲೆ
ಪ್ರೌಢಿ-ಮೆ-ಯನ್ನು
ಪ್ಲೇಟಿ-ನಲ್ಲಿ
ಪ್ಲೈಮತ್
ಫಂಕೆ
ಫಂಕೆ-ಇ-ವ-ರನ್ನು
ಫಂಕೆಯ
ಫಕೀ-ರನ
ಫಕೀ-ರ-ನಂತೆ
ಫಕೀ-ರ-ನಲ್ಲ
ಫಕೀ-ರರು
ಫಣಿ-ಯಾ-ದರು
ಫತೇ-ಹ್ಸಿಂ-ಗರೂ
ಫರ್ಲಾಂಗು
ಫರ್ಲಾಂಗ್
ಫಲ
ಫಲ-ಕಾ-ರಿ-ಯಾ-ಗ-ಲಿಲ್ಲ
ಫಲ-ಕ್ಕಲ್ಲ
ಫಲ-ಗ-ಳ-ಲ್ಲೊಂ-ದೆಂ-ದ-ರೆ-ನಿ-ವೇ-ದಿತಾ
ಫಲ-ಪ್ರದ
ಫಲ-ಪ್ರ-ದ-ವಾ-ಗುವ
ಫಲ-ವ-ತ್ತಾದ
ಫಲ-ವ-ತ್ತಾ-ದದ್ದು
ಫಲ-ವನ್ನು
ಫಲ-ವ-ನ್ನು-ಣ್ಣ-ಬೇ-ಕಾ-ಗು-ತ್ತದೆ
ಫಲ-ವ-ನ್ನೆಲ್ಲ
ಫಲ-ವಾಗಿ
ಫಲ-ವಾ-ಗಿಯೇ
ಫಲ-ವಾದ
ಫಲಿ-ತಾಂಶ
ಫಲಿ-ಸಿತು
ಫಲಿ-ಸು-ವು-ದೆಂಬ
ಫಾಕ್ಸ-ನಿಗೆ
ಫಾಕ್ಸ್
ಫಾಕ್ಸ್ನನ್ನೂ
ಫಾಕ್ಸ್ನಿಗೆ
ಫಾಜೀ-ಪು-ರದ
ಫಾರ್ಮ-ರಳ
ಫಾರ್ಮರ್
ಫಿಂಕೆ
ಫಿಲಾ-ಸ-ಫಿ-ಕಲ್
ಫಿಲಿ-ಪ್ಸ್
ಫಿಲಿ-ಪ್ಸ್ಳನ್ನು
ಫೀನ-ಲ್ಲಿನ
ಫೆಡ್ರಿಕ್
ಫೆಬ್ರ-ವರಿ
ಫೆಬ್ರ-ವ-ರಿ-ಯಲ್ಲಿ
ಫೆಬ್ರು-ವರಿ
ಫೆಬ್ರು-ವ-ರಿಯ
ಫೈಯಾಸ್
ಫೈಯಾ-ಸ್-ಅ-ಲಿ-ಖಾನ್
ಫೋನ್
ಫ್ಯಾಷನ್
ಫ್ರಾಂಕ್ಲಿನ್
ಫ್ರಾನ್ಸಿನ
ಫ್ರಾನ್ಸಿಸ್
ಫ್ರಾನ್ಸಿ-ಸ್ಸ-ನದು
ಫ್ರಾನ್ಸೆಸ್
ಫ್ರಾನ್ಸ್
ಫ್ರಾನ್ಸ್ನಲ್ಲಿ
ಫ್ರೀ
ಫ್ರೀಯರ್
ಫ್ರೀಯ-ರ್ರಂ-ಥ-ವರ
ಫ್ರೆಂಚ್
ಫ್ರೆಡ್ರಿಕ್
ಫ್ಲಾರೆ-ನ್ಸ್
ಫ್ಲೈಮ-ತ್ನಿಂದ
ಬಂಗಲೆ
ಬಂಗ-ಲೆ-ಗ-ಳಿಗೆ
ಬಂಗ-ಲೆಗೇ
ಬಂಗ-ಲೆಯ
ಬಂಗ-ಲೆ-ಯಲ್ಲಿ
ಬಂಗ-ಲೆ-ಯೊಂ-ದ-ರಲ್ಲಿ
ಬಂಗ-ಲೆ-ಯೊ-ಳಗೆ
ಬಂಗಳೀ
ಬಂಗಾಳ
ಬಂಗಾ-ಳಕ್ಕೂ
ಬಂಗಾ-ಳಕ್ಕೆ
ಬಂಗಾ-ಳದ
ಬಂಗಾ-ಳ-ದಲ್ಲಿ
ಬಂಗಾ-ಳ-ದ-ಲ್ಲಿ-ರುವ
ಬಂಗಾಳಿ
ಬಂಗಾ-ಳಿ-ಗಳ
ಬಂಗಾ-ಳಿ-ಗ-ಳಾ-ದರೋ
ಬಂಗಾ-ಳಿ-ಗ-ಳಿ-ಗಿಂತ
ಬಂಗಾ-ಳಿ-ಗ-ಳಿ-ಗೆಲ್ಲ
ಬಂಗಾ-ಳಿ-ಗಳು
ಬಂಗಾ-ಳಿ-ಗ-ಳೆಂದು
ಬಂಗಾ-ಳಿ-ಗ-ಳೊ-ಬ್ಬರು
ಬಂಗಾ-ಳಿಯ
ಬಂಗಾ-ಳಿ-ಯಲ್ಲಿ
ಬಂಗಾ-ಳಿ-ಯಲ್ಲೇ
ಬಂಗಾ-ಳಿಯೂ
ಬಂಗಾಳೀ
ಬಂಟ
ಬಂಡ-ವಾ-ಳ-ವ-ನ್ನಾಗಿ
ಬಂಡಾ-ಯದ
ಬಂಡಿ-ಗ-ಟ್ಟಲೆ
ಬಂಡಿ-ಯಲ್ಲಿ
ಬಂಡು-ಗಾರ
ಬಂಡೆ
ಬಂಡೆ-ಗಳ
ಬಂಡೆಗೆ
ಬಂಡೆಯ
ಬಂಡೆ-ಯನ್ನು
ಬಂಡೆ-ಯ-ನ್ನೇರಿ
ಬಂಡೆ-ಯೆ-ಡೆಗೆ
ಬಂಡೆ-ಯೊಂ-ದರ
ಬಂತು
ಬಂದ
ಬಂದಂ-ತಹ
ಬಂದಂತಾ
ಬಂದಂ-ತಾ-ಗಿತ್ತು
ಬಂದಂ-ತಾ-ಗಿದೆ
ಬಂದಂ-ತಾ-ಯಿತು
ಬಂದಂತೆ
ಬಂದಂ-ತೆಯೇ
ಬಂದ-ಅದೇ
ಬಂದ-ಕೂ-ಡಲೇ
ಬಂದ-ದ್ದನ್ನು
ಬಂದ-ದ್ದರ
ಬಂದ-ದ್ದ-ರಿಂದ
ಬಂದ-ದ್ದಾ-ಯಿತು
ಬಂದದ್ದು
ಬಂದ-ದ್ದೆಂದು
ಬಂದ-ದ್ದೆಲ್ಲಾ
ಬಂದದ್ದೇ
ಬಂದನೆ
ಬಂದ-ಮೇಲೆ
ಬಂದ-ರನ್ನು
ಬಂದ-ರ-ಲ್ಲದೆ
ಬಂದ-ರಿನ
ಬಂದ-ರಿ-ನಲ್ಲಿ
ಬಂದರು
ಬಂದ-ರುಆ
ಬಂದ-ರು-ಏ-ನೆಂ-ದರೆ
ಬಂದ-ರು-ಏನೇ
ಬಂದರೂ
ಬಂದರೆ
ಬಂದರೇ
ಬಂದರೋ
ಬಂದರ್ಗೆ
ಬಂದ-ರ್ನಿಂದ
ಬಂದಲ್ಲಿ
ಬಂದಳು
ಬಂದ-ವನೆ
ಬಂದ-ವನೇ
ಬಂದ-ವರ
ಬಂದ-ವ-ರಲ್ಲ
ಬಂದ-ವ-ರ-ಲ್ಲವೆ
ಬಂದ-ವ-ರಲ್ಲಿ
ಬಂದ-ವ-ರಿಗೆ
ಬಂದ-ವ-ರಿ-ಗೆಲ್ಲ
ಬಂದ-ವರು
ಬಂದ-ವರೆಲ್ಲ
ಬಂದ-ವರೆ-ಲ್ಲರೂ
ಬಂದ-ವ-ರೊ-ಡನೆ
ಬಂದ-ವಳು
ಬಂದವು
ಬಂದಷ್ಟು
ಬಂದಾಗ
ಬಂದಾ-ಗಲೂ
ಬಂದಾ-ಗ-ಲೆಲ್ಲ
ಬಂದಾ-ಗಿ-ನಿಂದ
ಬಂದಾ-ಗಿ-ನಿಂ-ದಲೂ
ಬಂದಾ-ರೆಂದು
ಬಂದಿ
ಬಂದಿ-ಕುಯಿ
ಬಂದಿ-ತಾ-ದರೂ
ಬಂದಿತು
ಬಂದಿ-ತೆಂ-ದರೆ
ಬಂದಿತ್ತು
ಬಂದಿ-ತ್ತೆ-ನ್ನ-ಬ-ಹುದು
ಬಂದಿದೆ
ಬಂದಿದ್ದ
ಬಂದಿ-ದ್ದನೋ
ಬಂದಿ-ದ್ದ-ರಲ್ಲ
ಬಂದಿ-ದ್ದರು
ಬಂದಿ-ದ್ದರೂ
ಬಂದಿ-ದ್ದಳು
ಬಂದಿ-ದ್ದವ
ಬಂದಿ-ದ್ದ-ವನು
ಬಂದಿ-ದ್ದ-ವ-ರಿಗೆ
ಬಂದಿ-ದ್ದ-ವರು
ಬಂದಿ-ದ್ದವೋ
ಬಂದಿ-ದ್ದಾಗ
ಬಂದಿ-ದ್ದಾ-ಗಲೇ
ಬಂದಿ-ದ್ದಾನೆ
ಬಂದಿ-ದ್ದಾರೆ
ಬಂದಿ-ದ್ದಾ-ರೆಂದೂ
ಬಂದಿ-ದ್ದಾ-ರೆಂ-ಬು-ದನ್ನು
ಬಂದಿ-ದ್ದಾರೋ
ಬಂದಿ-ದ್ದಾ-ಳೆ-ಚಿ-ನ್ನ-ದಂಥ
ಬಂದಿದ್ದು
ಬಂದಿ-ದ್ದುವು
ಬಂದಿ-ದ್ದೇನೆ
ಬಂದಿ-ದ್ದೇವೆ
ಬಂದಿರ
ಬಂದಿ-ರ-ಬ-ಹುದು
ಬಂದಿ-ರ-ಬ-ಹುದೆ
ಬಂದಿ-ರ-ಬೇಕು
ಬಂದಿ-ರ-ಲಿಲ್ಲ
ಬಂದಿ-ರಲು
ಬಂದಿ-ರಲೇ
ಬಂದಿರಿ
ಬಂದಿ-ರುವ
ಬಂದಿ-ರು-ವಂತೆ
ಬಂದಿ-ರು-ವಂ-ಥದು
ಬಂದಿ-ರು-ವ-ವರೇ
ಬಂದಿ-ರು-ವು-ದಕ್ಕೆ
ಬಂದಿ-ರು-ವುದನ್ನು
ಬಂದಿ-ರು-ವು-ದಾಗಿ
ಬಂದಿ-ರು-ವು-ದಾ-ಗಿಯೂ
ಬಂದಿ-ರು-ವುದು
ಬಂದಿಲ್ಲ
ಬಂದಿ-ಲ್ಲ-ವೆಂದು
ಬಂದಿ-ಲ್ಲ-ವೆಂಬ
ಬಂದಿ-ಳಿದ
ಬಂದಿ-ಳಿ-ದರು
ಬಂದಿ-ಳಿ-ದಾಗ
ಬಂದಿವೆ
ಬಂದೀತು
ಬಂದು
ಬಂದು-ದಕ್ಕೆ
ಬಂದು-ದನ್ನು
ಬಂದುದು
ಬಂದು-ಬಿ-ಟ್ಟರು
ಬಂದು-ಬಿ-ಟ್ಟಿತು
ಬಂದು-ಬಿ-ಟ್ಟಿ-ತ್ತೆಂ-ದರೆ
ಬಂದು-ಬಿ-ಟ್ಟಿ-ದ್ದರು
ಬಂದು-ಬಿಟ್ಟೆ
ಬಂದು-ಬಿ-ಟ್ಟೆ-ನಲ್ಲ
ಬಂದು-ಬಿ-ಟ್ಟೆ-ವಲ್ಲ
ಬಂದು-ಬಿ-ಡಲಿ
ಬಂದು-ಬಿಡಿ
ಬಂದುವು
ಬಂದು-ಹೋ-ಗ-ಬೇಕು
ಬಂದು-ಹೋ-ಗು-ತ್ತಿ-ದ್ದರು
ಬಂದೆ
ಬಂದೆ-ಯೇಕೆ
ಬಂದೆವು
ಬಂದೇ
ಬಂದೇ-ಬ-ರು-ತ್ತದೆ
ಬಂದೇ-ಬಿ-ಟ್ಟಿತು
ಬಂದೊ
ಬಂದೊ-ಡನೆ
ಬಂದೊ-ದ-ಗಿತ್ತು
ಬಂದೊ-ದ-ಗಿ-ರುವ
ಬಂದೊ-ದ-ಗು-ತ್ತಿತ್ತು
ಬಂದೊ-ದ-ಗು-ತ್ತಿದ್ದ
ಬಂದ್ದದೇ
ಬಂಧ-ದಿಂದ
ಬಂಧ-ನ-ಕ್ಕಿಂ-ತಲೂ
ಬಂಧ-ನ-ಗ-ಳಿಂ-ದಲೂ
ಬಂಧ-ನ-ಗ-ಳಿಲ್ಲ
ಬಂಧ-ನ-ಗಳು
ಬಂಧ-ನ-ದಲ್ಲಿ
ಬಂಧ-ನ-ದ-ಲ್ಲಿ-ಟ್ಟಿ-ರು-ತ್ತವೆ
ಬಂಧ-ನ-ದಿಂದ
ಬಂಧ-ನ-ವನ್ನೇ
ಬಂಧ-ನ-ವಾ-ಗಿತ್ತು
ಬಂಧ-ನ-ವೆಂ-ಬಂತೆ
ಬಂಧಿ-ತ-ರನ್ನು
ಬಂಧಿ-ತ-ರಾ-ಗು-ವು-ದೇನೋ
ಬಂಧಿ-ಸಿ-ಬಿ-ಟ್ಟುವು
ಬಂಧಿ-ಸಿ-ರುವ
ಬಂಧು-ಭ-ಗಿ-ನಿ-ಯ-ರಿ-ಗಾಗಿ
ಬಂಧು-ಗಳು
ಬಂಧು-ಗಳೇ
ಬಂಧು-ಬ-ಳ-ಗ-ವನ್ನು
ಬಂಧು-ಮಿ-ತ್ರರು
ಬಕಾರ
ಬಗ-ಲ-ಲ್ಲೊಂದು
ಬಗೆ
ಬಗೆ-ಇ-ವು-ಗಳ
ಬಗೆ-ಗಳಲ್ಲಿ
ಬಗೆ-ಗಿನ
ಬಗೆಗೂ
ಬಗೆಗೆ
ಬಗೆ-ಗೆಲ್ಲ
ಬಗೆಗೇ
ಬಗೆಗೋ
ಬಗೆ-ದರು
ಬಗೆ-ಬ-ಗೆಯ
ಬಗೆ-ಬ-ಗೆ-ಯಾಗಿ
ಬಗೆಯ
ಬಗೆ-ಯನ್ನು
ಬಗೆ-ಯಲ್ಲಿ
ಬಗೆ-ಯ-ವರೂ
ಬಗೆ-ಹ-ರಿ-ಸು-ವಲ್ಲಿ
ಬಗ್ಗಿ
ಬಗ್ಗೆ
ಬಗ್ಗೆ-ಮು-ಖ್ಯ-ವಾಗಿ
ಬಗ್ಗೆ-ಯಂತೂ
ಬಗ್ಗೆಯೂ
ಬಗ್ಗೆಯೇ
ಬಗ್ಗೆ-ಹೀಗೆ
ಬಚ್ಚಿ-ಟ್ಟು-ಕೊಂ-ಡಿರು
ಬಜಾ-ರಿಗೆ
ಬಟ್ಟ
ಬಟ್ಟ-ಲನ್ನು
ಬಟ್ಟಲು
ಬಟ್ಟೆ-ಗಳನ್ನು
ಬಟ್ಟೆ-ಗ-ಳಿಂ-ದಲೇ
ಬಟ್ಟೆಗೆ
ಬಟ್ಟೆ-ಬ-ರೆಯೇ
ಬಟ್ಟೆಯ
ಬಟ್ಟೆ-ಯನ್ನು
ಬಟ್ಟೆ-ಯನ್ನೂ
ಬಟ್ಟೆ-ಯ-ನ್ನೆಲ್ಲ
ಬಟ್ಟೆ-ಯೊಂ-ದ-ರಲ್ಲಿ
ಬಡ
ಬಡ-ಕಲು
ಬಡ-ಕು-ಟೀ-ರಕ್ಕೇ
ಬಡ-ಗು-ಡಿ-ಸ-ಲು-ಗಳಲ್ಲಿ
ಬಡ-ಜ-ನ-ತೆ-ಗಾಗಿ
ಬಡ-ಜ-ನರ
ಬಡ-ಜ-ನ-ರಿ-ಗಾ-ದರೂ
ಬಡ-ಜ-ನರು
ಬಡ-ತನ
ಬಡ-ತ-ನ-ದಾರಿ-ದ್ರ್ಯ-ಗಳನ್ನು
ಬಡ-ತ-ನ-ದಲ್ಲಿ
ಬಡ-ತ-ನ-ದಲ್ಲೇ
ಬಡ-ತ-ನವೇ
ಬಡ-ಬ-ಗ್ಗ-ರಿಗೆ
ಬಡ-ಭಾ-ರ-ತ-ದಿಂದ
ಬಡ-ಭಾ-ರ-ತೀ-ಯ-ರಿಗೆ
ಬಡ-ಮ-ನೆ-ಯನ್ನು
ಬಡವ
ಬಡ-ವ-ದೀ-ನ-ದ-ಲಿ-ತರ
ಬಡ-ವ-ನಿಗೆ
ಬಡ-ವರ
ಬಡ-ವ-ರನ್ನು
ಬಡ-ವ-ರಿ-ಗಾಗಿ
ಬಡ-ವ-ರಿಗೆ
ಬಡ-ವ-ರಿ-ಗೆಲ್ಲ
ಬಡ-ವರು
ಬಡ-ವ-ರು
ಬಡ-ವ-ರು-ಅ-ವಿ-ದ್ಯಾ-ವಂತ
ಬಡ-ವ-ರು-ದು-ರ್ಬ-ಲರು
ಬಡ-ವ-ರು-ಶ್ರೀ-ಮಂ-ತರು
ಬಡ-ವರೂ
ಬಡ-ವರೆಲ್ಲ
ಬಡ-ವರೋ
ಬಡ-ವಿ-ದ್ಯಾ-ರ್ಥಿ-ಗಳನ್ನು
ಬಡ-ಸಂ-ಸ್ಯಾ-ಸಿ-ಗೇನು
ಬಡ-ಸೇ-ವ-ಕಿಯ
ಬಡಾ-ವ-ಣೆ-ಗ-ಳಿಂ-ದಲೂ
ಬಡಿ
ಬಡಿಕೆ
ಬಡಿ-ತವು
ಬಡಿ-ದಂತೆ
ಬಡಿ-ದಾಗ
ಬಡಿ-ದಾಡಿ
ಬಡಿ-ದಿತ್ತು
ಬಡಿದು
ಬಡಿ-ದು-ಕೊಂಡು
ಬಡಿ-ದೆ-ಬ್ಬಿ-ಸಲು
ಬಡಿ-ದೆ-ಬ್ಬಿಸಿ
ಬಡಿ-ದೆ-ಬ್ಬಿ-ಸಿ-ದ್ದರು
ಬಡಿ-ಯಿತು
ಬಡಿ-ಯು-ತ್ತಿದ್ದ
ಬಡಿ-ಯುವ
ಬಡಿಸಿ
ಬಣ್ಣ-ಕೊ-ಡುವ
ಬಣ್ಣದ
ಬಣ್ಣ-ಸಿ-ದ್ದುವು
ಬಣ್ಣಿಸ
ಬಣ್ಣಿ-ಸ-ಲಾ-ಗಿ-ದ್ದಿತು
ಬಣ್ಣಿ-ಸ-ಲಾ-ರಂ-ಭಿ-ಸಿ-ದರು
ಬಣ್ಣಿ-ಸಲು
ಬಣ್ಣಿ-ಸ-ಲ್ಪಟ್ಟ
ಬಣ್ಣಿಸಿ
ಬಣ್ಣಿ-ಸಿ-ದರು
ಬಣ್ಣಿ-ಸಿ-ದ-ರು-ಓಹ್
ಬಣ್ಣಿ-ಸಿ-ದ್ದರೂ
ಬಣ್ಣಿಸು
ಬಣ್ಣಿ-ಸುತ್ತ
ಬಣ್ಣಿ-ಸು-ತ್ತಾರೆ
ಬಣ್ಣಿ-ಸು-ತ್ತಾಳೆ
ಬಣ್ಣಿ-ಸುತ್ತಿ
ಬಣ್ಣಿ-ಸು-ತ್ತಿ-ದ್ದಂತೆ
ಬಣ್ಣಿ-ಸು-ತ್ತಿ-ದ್ದರು
ಬಣ್ಣಿ-ಸು-ತ್ತಿ-ದ್ದ-ವರು
ಬಣ್ಣಿ-ಸುವ
ಬಣ್ಣಿ-ಸು-ವುದು
ಬಣ್ಣೀ-ಸು-ತ್ತಾರೆ
ಬಣ್ಣೀ-ಸುವ
ಬತ್ತ-ಲಿಲ್ಲ
ಬತ್ತ-ಳಿ-ಕೆ-ಯ-ಲ್ಲಿದ್ದ
ಬತ್ತಿ-ಗ-ಳಾ-ದರು
ಬದರೀ
ಬದ-ರೀ-ನಾ-ಥ-ನನ್ನು
ಬದಲಾ
ಬದ-ಲಾಗ
ಬದ-ಲಾ-ಗ-ಲಾ-ರದು
ಬದ-ಲಾ-ಗಲು
ಬದ-ಲಾಗಿ
ಬದ-ಲಾಗಿದೆ
ಬದ-ಲಾ-ಗಿ-ಹೋ-ಯಿತು
ಬದ-ಲಾ-ಗು-ತ್ತಿತ್ತು
ಬದ-ಲಾ-ಗು-ವುದನ್ನು
ಬದ-ಲಾದ
ಬದ-ಲಾ-ಯಿತು
ಬದ-ಲಾ-ಯಿ-ಸ-ಬ-ಲ್ಲ-ವ-ರಾ-ಗಿ-ದ್ದರು
ಬದ-ಲಾ-ಯಿ-ಸ-ಬೇ-ಕಿ-ತ್ತೆಂದು
ಬದ-ಲಾ-ಯಿ-ಸ-ಬೇ-ಕಿ-ರ-ಲಿಲ್ಲ
ಬದ-ಲಾ-ಯಿ-ಸ-ಲಿಲ್ಲ
ಬದ-ಲಾ-ಯಿಸಿ
ಬದ-ಲಾ-ಯಿ-ಸಿ-ಕೊಂ-ಡಿ-ದ್ದರು
ಬದ-ಲಾ-ಯಿ-ಸಿ-ಕೊ-ಳ್ಳ-ಬೇ-ಕಾ-ಯಿತು
ಬದ-ಲಾ-ಯಿ-ಸಿ-ಕೊಳ್ಳಿ
ಬದ-ಲಾ-ಯಿ-ಸಿತು
ಬದ-ಲಾ-ಯಿ-ಸಿ-ದಂ-ತಹ
ಬದ-ಲಾ-ಯಿ-ಸಿ-ದರು
ಬದ-ಲಾ-ವಣೆ
ಬದ-ಲಾ-ವ-ಣೆ-ಗಳನ್ನು
ಬದ-ಲಾ-ವ-ಣೆ-ಗ-ಳುಂ-ಟಾ-ಗು-ತ್ತವೆ
ಬದ-ಲಾ-ವ-ಣೆ-ಯನ್ನು
ಬದ-ಲಾ-ವ-ಣೆ-ಯ-ನ್ನುಂ-ಟು-ಮಾ-ಡಿ-ದುವು
ಬದ-ಲಾ-ವ-ಣೆ-ಯನ್ನೂ
ಬದ-ಲಾ-ವ-ಣೆ-ಯಾ-ಯಿತು
ಬದ-ಲಾ-ವ-ಣೆ-ಯುಂ-ಟಾ-ಯಿತು
ಬದ-ಲಿಗೆ
ಬದ-ಲಿ-ಸದೆ
ಬದ-ಲಿ-ಸ-ಲಿಲ್ಲ
ಬದ-ಲಿಸಿ
ಬದ-ಲಿ-ಸಿ-ಕೊಂಡು
ಬದ-ಲಿ-ಸಿ-ಕೊ-ಳ್ಳು-ತ್ತಿ-ದ್ದರು
ಬದ-ಲಿ-ಸಿ-ಕೊ-ಳ್ಳು-ತ್ತಿ-ರ-ಲಿಲ್ಲ
ಬದ-ಲಿ-ಸಿ-ಕೊ-ಳ್ಳು-ವಂತೆ
ಬದ-ಲಿ-ಸಿತು
ಬದ-ಲಿ-ಸಿ-ಬಿಟ್ಟ
ಬದಲು
ಬದಿ-ಗಿಟ್ಟು
ಬದಿ-ಗೊ-ತ್ತ-ಬೇಕು
ಬದಿ-ಗೊತ್ತಿ
ಬದಿ-ಯ-ಲ್ಲೊಂದು
ಬದು-ಕನ್ನು
ಬದು-ಕ-ಬೇ-ಕಾ-ದರೆ
ಬದು-ಕ-ಬೇ-ಕಾ-ಯಿತು
ಬದುಕಿ
ಬದು-ಕಿ-ಕೊ-ಳ್ಳ-ಬೇ-ಕಾ-ದರೆ
ಬದು-ಕಿಕೋ
ಬದು-ಕಿ-ದ್ದರೂ
ಬದು-ಕಿ-ದ್ದಾರೆ
ಬದು-ಕಿ-ದ್ದಿ-ದ್ದರೆ
ಬದು-ಕಿನ
ಬದು-ಕಿ-ರಲು
ಬದು-ಕಿ-ರು-ತ್ತೇ-ನೆಯೋ
ಬದು-ಕಿ-ರುವ
ಬದು-ಕಿ-ಸಲು
ಬದುಕು
ಬದು-ಕು-ತ್ತಾ-ರೆಯೋ
ಬದು-ಕು-ವಂ-ತಹ
ಬದ್ದನ್ನು
ಬದ್ದಿಗೆ
ಬದ್ದಿ-ನೊ-ಳಕ್ಕೆ
ಬದ್ದಿ-ನೊ-ಳಗೆ
ಬದ್ದು
ಬದ್ದು-ಗಳು
ಬದ್ದು-ಗ-ಳೊ-ಳಗೆ
ಬದ್ಧ
ಬದ್ಧ-ದ್ವೇಷ
ಬದ್ಧ-ನಾ-ಗಿ-ದ್ದೇನೆ
ಬದ್ಧ-ನಾ-ಗಿ-ರ-ಬೇ-ಕೆಂದು
ಬದ್ಧ-ರ-ನ್ನಾ-ಗಿ-ಸಿತು
ಬದ್ಧ-ರಾ-ಗಿ-ದ್ದರು
ಬದ್ಧ-ರಾ-ಗಿ-ರ-ಲೇ-ಬೇ-ಕೆಂದು
ಬದ್ಧ-ರಾ-ಗಿರಿ
ಬದ್ಧ-ವಾ-ಗಿದ್ದು
ಬದ್ಧ-ವಿ-ರೋ-ಧಿ-ಗ-ಳಾ-ಗಿ-ದ್ದಾರೆ
ಬದ್ಮಾ-ಶ್ಗಳೇ
ಬನಿಯಾ
ಬನ್ನಿ
ಬಫೆಲೊ
ಬಯಕೆ
ಬಯ-ಕೆಯು
ಬಯ-ಲಾ-ಗು-ತ್ತಿ-ರ-ಲಿ-ಲ್ಲ-ವೇನೋ
ಬಯ-ಲಾ-ದುವು
ಬಯ-ಲಿಗೆ
ಬಯ-ಲಿ-ಗೆ-ಳೆ-ದರು
ಬಯ-ಲಿ-ನಲ್ಲಿ
ಬಯಲು
ಬಯ-ಲು-ಗೊ-ಳಿ-ಸು-ತ್ತದೆ
ಬಯ-ಲು-ಸೀ-ಮೆ-ಯಲ್ಲೂ
ಬಯ-ಸದೆ
ಬಯಸಿ
ಬಯ-ಸಿದ
ಬಯ-ಸಿ-ದರು
ಬಯ-ಸಿ-ದರೂ
ಬಯ-ಸಿ-ದರೆ
ಬಯ-ಸಿ-ದಳು
ಬಯ-ಸಿ-ದಾಗ
ಬಯ-ಸಿ-ದು-ದ-ನ್ನೆಲ್ಲ
ಬಯ-ಸಿ-ದ್ದರು
ಬಯ-ಸಿ-ದ್ದುದು
ಬಯ-ಸಿ-ಬಂ-ದಳು
ಬಯ-ಸು-ತ್ತಾನೆ
ಬಯ-ಸು-ತ್ತಿ-ದ್ದರು
ಬಯ-ಸು-ತ್ತೇನೆ
ಬಯ-ಸುವ
ಬಯ-ಸು-ವಂ-ಥ-ದಾ-ದರೂ
ಬಯ-ಸು-ವ-ವನು
ಬಯ-ಸು-ವು-ದಿ-ಲ್ಲವೋ
ಬಯ್ಯ-ಲಾಗಿದೆ
ಬಯ್ಯು-ವು-ದ-ನ್ನಾ-ದರೂ
ಬಯ್ಯು-ವುದನ್ನು
ಬರ
ಬರ-ಗಾಲ
ಬರ-ಗಾ-ಲ-ದಲ್ಲಿ
ಬರ-ತೊ-ಡ-ಗಿ-ದುವು
ಬರ-ದಿ-ದ್ದ-ರಿಂದ
ಬರ-ದಿ-ದ್ದರೆ
ಬರ-ದಿ-ರಲು
ಬರ-ದಿ-ರು-ವಂತೆ
ಬರ-ದೆ-ಯಿ-ದ್ದರೆ
ಬರ-ಬ-ರುತ್ತ
ಬರ-ಬಲ್ಲ
ಬರ-ಬ-ಲ್ಲಿರೋ
ಬರ-ಬ-ಹು-ದಾ-ಗಿ-ತ್ತು-ಸ್ವಲ್ಪ
ಬರ-ಬ-ಹು-ದಾದ
ಬರ-ಬ-ಹು-ದಾ-ದ್ದ-ರಿಂದ
ಬರ-ಬ-ಹುದು
ಬರ-ಬ-ಹು-ದೆಂದು
ಬರ-ಬಾ-ರ-ದೆಂದು
ಬರ-ಬೇ-ಕಾ-ಗಿದೆ
ಬರ-ಬೇ-ಕಾ-ಗು-ತ್ತದೆ
ಬರ-ಬೇ-ಕಾ-ದರೆ
ಬರ-ಬೇಕು
ಬರ-ಬೇ-ಕೆಂದು
ಬರ-ಮಾಡಿ
ಬರ-ಮಾ-ಡಿ-ಕೊಂಡ
ಬರ-ಮಾ-ಡಿ-ಕೊಂ-ಡರು
ಬರ-ಮಾ-ಡಿ-ಕೊಂ-ಡಳು
ಬರ-ಮಾ-ಡಿ-ಕೊಂ-ಡಿ-ದ್ದಳು
ಬರ-ಮಾ-ಡಿ-ಕೊಂಡು
ಬರ-ಮಾ-ಡಿ-ಕೊ-ಳ್ಳಲು
ಬರ-ಮಾ-ಡಿ-ಕೊಳ್ಳಿ
ಬರ-ಲಾ-ಗ-ದಿದ್ದ
ಬರ-ಲಾ-ರಂ-ಭಿ-ಸಿತ್ತು
ಬರ-ಲಾ-ರಂ-ಭಿ-ಸಿದ
ಬರ-ಲಾ-ರಂ-ಭಿ-ಸಿ-ದರು
ಬರ-ಲಾ-ರಂ-ಭಿ-ಸಿ-ದ್ದರು
ಬರ-ಲಾ-ರಂ-ಭಿ-ಸಿರು
ಬರಲಿ
ಬರ-ಲಿ-ಆ-ತನ
ಬರ-ಲಿದೆ
ಬರ-ಲಿ-ದ್ದಳು
ಬರ-ಲಿಲ್ಲ
ಬರ-ಲಿ-ಲ್ಲ-ವೆಂದು
ಬರಲು
ಬರಲೇ
ಬರ-ಲೇ-ಬೇ-ಕು-ಕ-ಡೆಗೆ
ಬರ-ವ-ಣಿ-ಗೆ-ಗಳ
ಬರ-ವ-ಣಿ-ಗೆಗೆ
ಬರ-ವ-ಣಿ-ಗೆ-ಯಿಂದ
ಬರ-ವನ್ನೇ
ಬರ-ವಿ-ಗಾಗಿ
ಬರ-ಸಿ-ಡಿ-ಲಿ-ನಂತೆ
ಬರ-ಸಿ-ಡಿ-ಲಿ-ನಂ-ತೆ-ರ-ಗಿ-ದರು
ಬರ-ಸಿ-ಡಿಲೇ
ಬರಹ
ಬರ-ಹ-ಗಳ
ಬರ-ಹ-ಗಳನ್ನು
ಬರಿ-ಗಾಲ
ಬರಿ-ಗೈಯ
ಬರಿ-ಗೈ-ಯಲ್ಲಿ
ಬರಿ-ದಾ-ಗು-ತ್ತಿತ್ತು
ಬರಿ-ದಾದ
ಬರಿ-ದಾ-ದಂತೆ
ಬರಿದೆ
ಬರಿಯ
ಬರಿ-ಯೋ-ಟವು
ಬರು
ಬರು-ತ್ತದೆ
ಬರು-ತ್ತ-ದೆಂದು
ಬರು-ತ್ತ-ದೆಯೆ
ಬರು-ತ್ತದೋ
ಬರು-ತ್ತಲೂ
ಬರು-ತ್ತಲೇ
ಬರು-ತ್ತವೆ
ಬರು-ತ್ತಾನೆ
ಬರು-ತ್ತಾರೆ
ಬರು-ತ್ತಾ-ರೆಂದು
ಬರು-ತ್ತಾರೋ
ಬರುತ್ತಿ
ಬರು-ತ್ತಿತ್ತು
ಬರು-ತ್ತಿ-ತ್ತು-ಅದು
ಬರು-ತ್ತಿದೆ
ಬರು-ತ್ತಿ-ದೆ-ಯೆಂದೂ
ಬರು-ತ್ತಿದ್ದ
ಬರು-ತ್ತಿ-ದ್ದಂತೆ
ಬರು-ತ್ತಿ-ದ್ದಂ-ತೆಯೇ
ಬರು-ತ್ತಿ-ದ್ದರು
ಬರು-ತ್ತಿ-ದ್ದ-ವರ
ಬರು-ತ್ತಿ-ದ್ದ-ವ-ರಲ್ಲಿ
ಬರು-ತ್ತಿ-ದ್ದ-ವ-ರ-ಲ್ಲೆಷ್ಟೋ
ಬರು-ತ್ತಿ-ದ್ದ-ವರು
ಬರು-ತ್ತಿ-ದ್ದ-ವರೆಲ್ಲ
ಬರು-ತ್ತಿ-ದ್ದಾಗ
ಬರು-ತ್ತಿ-ದ್ದಾರೆ
ಬರು-ತ್ತಿ-ದ್ದು-ದನ್ನು
ಬರು-ತ್ತಿ-ದ್ದುದು
ಬರು-ತ್ತಿ-ದ್ದು-ವಾ-ದರೂ
ಬರು-ತ್ತಿ-ದ್ದುವು
ಬರು-ತ್ತಿದ್ದೆ
ಬರು-ತ್ತಿ-ದ್ದೇವೆ
ಬರು-ತ್ತಿ-ರ-ಲಿಲ್ಲ
ಬರು-ತ್ತಿ-ರ-ಲಿ-ಲ್ಲ-ವಾಗಿ
ಬರು-ತ್ತಿ-ರ-ಲಿ-ಲ್ಲ-ವಾ-ದ್ದ-ರಿಂದ
ಬರು-ತ್ತಿ-ರುವ
ಬರು-ತ್ತಿ-ರು-ವ-ವ-ರೆಗೆ
ಬರು-ತ್ತಿ-ರು-ವಾಗ
ಬರು-ತ್ತಿ-ರು-ವುದನ್ನು
ಬರು-ತ್ತಿಲ್ಲ
ಬರು-ತ್ತಿವೆ
ಬರುತ್ತೀ
ಬರು-ತ್ತೇನೆ
ಬರು-ತ್ತೇವೆ
ಬರುವ
ಬರು-ವಂ-ತಹ
ಬರು-ವಂ-ತಾ-ಗ-ಬೇಕು
ಬರು-ವಂ-ತಾ-ಗಲಿ
ಬರು-ವಂ-ತಾ-ಗ-ಲೆಂಬ
ಬರು-ವಂ-ತಾ-ದದ್ದು
ಬರು-ವಂ-ತಾ-ದುದು
ಬರು-ವಂ-ತಾ-ಯಿತು
ಬರು-ವಂ-ತಿಲ್ಲ
ಬರು-ವಂತೆ
ಬರು-ವಂ-ಥದೂ
ಬರು-ವಂ-ಥದೇ
ಬರು-ವ-ರೆಂಬ
ಬರು-ವರೋ
ಬರು-ವ-ವರ
ಬರು-ವ-ವರು
ಬರು-ವ-ವ-ರೆಗೂ
ಬರು-ವ-ಷ್ಟ-ರಲ್ಲಿ
ಬರು-ವಾಗ
ಬರು-ವಾ-ಗಲೇ
ಬರು-ವಿರಿ
ಬರುವು
ಬರು-ವು-ದ-ಕ್ಕಿಂತ
ಬರು-ವು-ದಕ್ಕೂ
ಬರು-ವು-ದಕ್ಕೆ
ಬರು-ವು-ದನ್ನೇ
ಬರು-ವು-ದಾಗಿ
ಬರು-ವು-ದಿಲ್ಲ
ಬರು-ವು-ದಿ-ಲ್ಲವೆ
ಬರು-ವು-ದಿ-ಲ್ಲ-ವೆಂದು
ಬರು-ವು-ದಿ-ಲ್ಲ-ವೆಂದೂ
ಬರು-ವು-ದಿ-ಲ್ಲ-ವೆಂದೇ
ಬರು-ವು-ದೆಂದು
ಬರು-ವುದೇ
ಬರೆ
ಬರೆದ
ಬರೆ-ದಂ-ತಹ
ಬರೆ-ದಂತೆ
ಬರೆ-ದ-ದ್ದ-ನ್ನೆಲ್ಲ
ಬರೆ-ದದ್ದು
ಬರೆ-ದದ್ದೂ
ಬರೆ-ದರು
ಬರೆ-ದ-ರು-ಅ-ವರು
ಬರೆ-ದ-ರು-ಇಡೀ
ಬರೆ-ದ-ರು-ಧೈ-ರ್ಯ-ವಾಗಿ
ಬರೆ-ದ-ರು-ನಾನು
ಬರೆ-ದ-ರು-ಭ-ಗ-ವಂ-ತನ
ಬರೆ-ದ-ರು-ಹೆ-ಚ್ಚಿನ
ಬರೆ-ದರೆ
ಬರೆ-ದಳು
ಬರೆ-ದ-ವರ
ಬರೆ-ದ-ವ-ರಾ-ರೆಂ-ಬುದೂ
ಬರೆ-ದ-ವರು
ಬರೆ-ದಿಟ್ಟ
ಬರೆ-ದಿ-ಟ್ಟರು
ಬರೆ-ದಿ-ಟ್ಟಿದ್ದು
ಬರೆ-ದಿ-ಟ್ಟು-ಕೊಂ-ಡ-ದ್ದ-ರಿಂದ
ಬರೆ-ದಿ-ಟ್ಟು-ಕೊಂ-ಡದ್ದು
ಬರೆ-ದಿ-ಟ್ಟು-ಕೊಂ-ಡಿ-ದ್ದಳು
ಬರೆ-ದಿ-ಟ್ಟು-ಕೊ-ಳ್ಳಲು
ಬರೆ-ದಿ-ಟ್ಟು-ಕೊ-ಳ್ಳುವ
ಬರೆ-ದಿ-ಡ-ಬೇಕು
ಬರೆ-ದಿ-ಡಲು
ಬರೆ-ದಿ-ಡ-ಲ್ಪ-ಟ್ಟಿವೆ
ಬರೆ-ದಿತ್ತು
ಬರೆ-ದಿದ್ದ
ಬರೆ-ದಿ-ದ್ದರು
ಬರೆ-ದಿ-ದ್ದ-ರು-ಇಂ-ಗ್ಲೆಂ-ಡಿನ
ಬರೆ-ದಿ-ದ್ದ-ಸು-ಳ್ಳಾ-ಡುವ
ಬರೆ-ದಿ-ದ್ದಾನೆ
ಬರೆ-ದಿ-ದ್ದಾರೆ
ಬರೆ-ದಿ-ದ್ದಾಳೆ
ಬರೆ-ದಿ-ದ್ದೀಯೆ
ಬರೆ-ದಿದ್ದು
ಬರೆ-ದಿದ್ದೆ
ಬರೆ-ದಿ-ದ್ದೇನೆ
ಬರೆ-ದಿರ
ಬರೆ-ದಿ-ರಲಿ
ಬರೆ-ದಿ-ರ-ಲಿ-ಯು-ವ-ಸಂ-ನ್ಯಾ-ಸಿ-ಯೊಬ್ಬ
ಬರೆ-ದಿ-ರ-ಲಿಲ್ಲ
ಬರೆ-ದಿ-ರುವ
ಬರೆ-ದಿ-ರು-ವುದನ್ನು
ಬರೆದು
ಬರೆ-ದು-ಕೊಂಡ
ಬರೆ-ದು-ಕೊಂ-ಡಳು
ಬರೆ-ದು-ಕೊಂಡು
ಬರೆ-ದು-ಕೊಟ್ಟ
ಬರೆ-ದು-ಕೊ-ಟ್ಟರು
ಬರೆ-ದು-ಕೊ-ಟ್ಟರೆ
ಬರೆ-ದು-ಕೊ-ಟ್ಟಾರು
ಬರೆ-ದು-ಕೊ-ಟ್ಟಿದ್ದ
ಬರೆ-ದು-ಕೊಟ್ಟು
ಬರೆ-ದು-ಕೊ-ಡುತ್ತ
ಬರೆ-ದು-ಕೊ-ಳ್ಳಲು
ಬರೆ-ದು-ಕೊ-ಳ್ಳುತ್ತ
ಬರೆ-ದು-ಕೊ-ಳ್ಳು-ತ್ತಿ-ದ್ದಳು
ಬರೆ-ದು-ಕೊ-ಳ್ಳು-ವಂತೆ
ಬರೆ-ದು-ಕೊ-ಳ್ಳು-ವುದು
ಬರೆ-ದು-ಬಿ-ಟ್ಟರು
ಬರೆ-ದುವು
ಬರೆದೆ
ಬರೆಯ
ಬರೆ-ಯ-ಬೇ-ಕಾ-ದರೆ
ಬರೆ-ಯ-ಬೇ-ಕಾ-ದ-ವರು
ಬರೆ-ಯ-ಬೇಕು
ಬರೆ-ಯ-ಬೇ-ಕೆಂ-ದರೆ
ಬರೆ-ಯ-ಬೇ-ಕೆಂ-ದಿ-ದ್ದೇನೆ
ಬರೆ-ಯ-ಲಾ-ಗಿತ್ತು
ಬರೆ-ಯ-ಲಾ-ಗಿ-ತ್ತು-ನಿಮ್ಮ
ಬರೆ-ಯ-ಲಾ-ಗಿ-ತ್ತು-ವಿ-ವೇ-ಕಾ-ನಂ-ದರ
ಬರೆ-ಯ-ಲಾ-ಗಿ-ತ್ತು-ಸಂ-ಭಾ-ಷ-ಣೆ-ಯಲ್ಲಿ
ಬರೆ-ಯ-ಲಾ-ಗಿ-ದೆಯೋ
ಬರೆ-ಯ-ಲಾ-ಗಿ-ದ್ದಿತು
ಬರೆ-ಯ-ಲಾ-ಗಿದ್ದು
ಬರೆ-ಯ-ಲಾ-ಗು-ತ್ತದೆ
ಬರೆ-ಯ-ಲಾರೆ
ಬರೆ-ಯ-ಲಿಲ್ಲ
ಬರೆ-ಯಲು
ಬರೆ-ಯ-ಲ್ಪಟ್ಟ
ಬರೆ-ಯ-ಲ್ಪ-ಟ್ಟಿ-ರು-ವುದು
ಬರೆ-ಯಿತು
ಬರೆ-ಯಿ-ತು-ಅ-ವರ
ಬರೆ-ಯಿ-ತು-ಜ-ನ-ಪ್ರಿಯ
ಬರೆ-ಯಿಸಿ
ಬರೆಯು
ಬರೆ-ಯುತ್ತ
ಬರೆ-ಯು-ತ್ತಲೇ
ಬರೆ-ಯು-ತ್ತಾನೆ
ಬರೆ-ಯು-ತ್ತಾರೆ
ಬರೆ-ಯು-ತ್ತಾ-ರೆ-ಭಾ-ರ-ತ-ದಲ್ಲಿ
ಬರೆ-ಯು-ತ್ತಾಳೆ
ಬರೆ-ಯುತ್ತಿ
ಬರೆ-ಯು-ತ್ತಿದ್ದ
ಬರೆ-ಯು-ತ್ತಿ-ದ್ದಂ-ತಹ
ಬರೆ-ಯು-ತ್ತಿ-ದ್ದರು
ಬರೆ-ಯು-ತ್ತಿ-ದ್ದುವು
ಬರೆ-ಯು-ತ್ತಿ-ದ್ದೇ-ನಷ್ಟೆ
ಬರೆ-ಯು-ತ್ತಿ-ದ್ದೇನೆ
ಬರೆ-ಯು-ತ್ತಿ-ದ್ದೇ-ನೆ-ನಾನು
ಬರೆ-ಯು-ತ್ತಿ-ರು-ವು-ದರ
ಬರೆ-ಯು-ತ್ತಿಲ್ಲ
ಬರೆ-ಯು-ತ್ತೇನೆ
ಬರೆ-ಯುವ
ಬರೆ-ಯು-ವಂತೆ
ಬರೆ-ಯು-ವಲ್ಲಿ
ಬರೆ-ಯು-ವ-ವ-ರಿದ್ದು
ಬರೆ-ಯು-ವಾಗ
ಬರೆ-ಯು-ವುದನ್ನು
ಬರೆ-ಯು-ವು-ದ-ರಲ್ಲಿ
ಬರೆ-ಸು-ತ್ತಿ-ದ್ದರು
ಬರೆ-ಸು-ತ್ತಿದ್ದಾ
ಬರೋಡ
ಬರೋ-ಡಕ್ಕೆ
ಬರೋ-ಡದ
ಬರೋ-ಡ-ದಲ್ಲಿ
ಬರೋಸ್
ಬರೋ-ಸ್ರಂ-ತಹ
ಬರೋ-ಸ್ರ-ವರ
ಬರೋ-ಸ್ರ-ವ-ರಿಗೂ
ಬರೋ-ಸ್ರ-ವ-ರಿಗೆ
ಬರೋ-ಸ್ರ-ವ-ರಿ-ಗೆ-ಕ-ಳಿಸಿ
ಬರೋ-ಸ್ರ-ವರು
ಬರೋ-ಸ್ರಿಗೂ
ಬರ್ಕ್
ಬರ್ಗ್
ಬರ್ಗ್ನೊಂ-ದಿಗೆ
ಬರ್ಚ್
ಬರ್ನಾಫ್
ಬರ್ನಾ-ರ್ಡ್
ಬರ್ನ್ಹಾ-ರ್ಟ್
ಬರ್ಮಾ
ಬರ್ಲಿ-ನ್ನಿಗೆ
ಬರ್ಲಿ-ನ್ನಿ-ನಿಂದ
ಬಲ
ಬಲ-ಗಡೆ
ಬಲಗೈ
ಬಲ-ಗೈ-ಗಿಂತ
ಬಲ-ಗೈ-ಯನ್ನು
ಬಲ-ಗೈ-ಯಲ್ಲಿ
ಬಲ-ಗೈಯೇ
ಬಲ-ದಿಂದ
ಬಲ-ಪ-ಡಿ-ಸು-ವಂ-ತಹ
ಬಲ-ಪ್ರ-ಯೋಗ
ಬಲ-ವಂತ
ಬಲ-ವಂ-ತ-ದಿಂದ
ಬಲ-ವಂ-ತ-ವಾಗಿ
ಬಲ-ವನ್ನು
ಬಲ-ವಾಗಿ
ಬಲ-ವಾದ
ಬಲ-ವಾ-ದು-ದೆಂದು
ಬಲ-ವಾ-ಯಿತು
ಬಲ-ಶಾ-ಲಿ-ಗಳು
ಬಲಾ-ಢ್ಯನೂ
ಬಲಾ-ತ್ಕಾ-ರ-ದಿಂದ
ಬಲಾ-ತ್ಕಾ-ರ-ವಾಗಿ
ಬಲಿ-ಕೊಟ್ಟು
ಬಲಿ-ಕೊ-ಡು-ವು-ದರ
ಬಲಿ-ತೆತ್ತು
ಬಲಿ-ದಾ-ನ-ವಾ-ಗ-ಬಲ್ಲ
ಬಲಿ-ದಾ-ನ-ವಾ-ಗ-ಲೆಂದೇ
ಬಲಿ-ಯಾ-ಗು-ತ್ತಿ-ದ್ದಾರೆ
ಬಲಿ-ಯಾ-ದೆ-ನಲ್ಲ
ಬಲಿ-ವೇ-ದಿ-ಕೆಯ
ಬಲಿಷ್ಠ
ಬಲು
ಬಲು-ದೂರ
ಬಲೂನ್
ಬಲೆಗೆ
ಬಲೆ-ಯಲ್ಲಿ
ಬಲೆ-ಯಿಂದ
ಬಲೇ-ಶ್ವ-ರಕ್ಕೆ
ಬಲ್ಲ
ಬಲ್ಲನೋ
ಬಲ್ಲರು
ಬಲ್ಲವ
ಬಲ್ಲ-ವ-ನಾ-ಗಿದ್ದ
ಬಲ್ಲ-ವನು
ಬಲ್ಲ-ವರಾ
ಬಲ್ಲ-ವ-ರಾರು
ಬಲ್ಲ-ವ-ರಿಗೆ
ಬಲ್ಲ-ವ-ರೆಂದೂ
ಬಲ್ಲಿರಾ
ಬಲ್ಲುದು
ಬಲ್ಲೆ
ಬಲ್ಲೆ-ನಿಮ್ಮ
ಬಲ್ಲೆಯ
ಬಲ್ಲೆಯಾ
ಬಳಲಿ
ಬಳ-ಲಿ-ಕೆ-ಯ-ನ್ನುಂಟು
ಬಳ-ಲಿ-ಕೆ-ಯಿಂ-ದಾಗಿ
ಬಳ-ಲಿದ
ಬಳ-ಲಿ-ದರು
ಬಳ-ಲಿ-ದಾಗ
ಬಳ-ಲಿದ್ದ
ಬಳ-ಲಿ-ದ್ದರು
ಬಳ-ಲಿ-ದ್ದೇನೆ
ಬಳ-ಸ-ಬ-ಹುದು
ಬಳ-ಸ-ಬಾ-ರೆಂ-ಬು-ದನ್ನು
ಬಳ-ಸ-ಬೇಕು
ಬಳ-ಸ-ಬೇಡ
ಬಳ-ಸ-ಲಾ-ಗುವ
ಬಳ-ಸ-ಲಿಲ್ಲ
ಬಳಸಿ
ಬಳ-ಸಿ-ಕೊಂಡ
ಬಳ-ಸಿ-ಕೊಂ-ಡರೂ
ಬಳ-ಸಿ-ಕೊ-ಳ್ಳು-ತ್ತಿ-ದ್ದರು
ಬಳ-ಸಿ-ಕೊ-ಳ್ಳು-ವಂತೆ
ಬಳ-ಸಿತ್ತು
ಬಳ-ಸು-ತ್ತಾ-ರಾ-ದ್ದ-ರಿಂದ
ಬಳ-ಸುವ
ಬಳ-ಸು-ವಂ-ತಹ
ಬಳ-ಸುವು
ಬಳ-ಸು-ವು-ದಿಲ್ಲ
ಬಳ-ಸು-ವುದು
ಬಳಿ
ಬಳಿ-ಅದೂ
ಬಳಿಕ
ಬಳಿ-ಕವೇ
ಬಳಿಗೆ
ಬಳಿ-ಗೋ-ಡುವ
ಬಳಿಯ
ಬಳಿ-ಯಲು
ಬಳಿ-ಯಲ್ಲಿ
ಬಳಿ-ಯ-ಲ್ಲಿದ್ದ
ಬಳಿ-ಯಿದ್ದ
ಬಳಿ-ಯಿ-ರು-ವಂ-ಥ-ದೇ-ನಾ-ದ-ರ-ದನ್ನು
ಬಳಿ-ಯೀಗ
ಬಳಿಯೇ
ಬಳಿ-ಸಿ-ಕೊಂಡು
ಬಳಿ-ಸು-ಮ್ಮ-ನಿ-ದ್ದೀತೆ
ಬವಳಿ
ಬಷೀರ್
ಬಸ-ವ-ಳಿದು
ಬಸಿ-ಯು-ತ್ತಿ-ದ್ದರೋ
ಬಹಳ
ಬಹ-ಳ-ಮ-ಟ್ಟಿಗೆ
ಬಹ-ಳ-ವಾಗಿ
ಬಹ-ಳ-ವಾ-ಗಿಯೇ
ಬಹ-ಳ-ವಿದೆ
ಬಹ-ಳವೇ
ಬಹ-ಳ-ಷ್ಟನ್ನು
ಬಹ-ಳಷ್ಟು
ಬಹಾ
ಬಹಾ-ದೂರ್
ಬಹಾ-ದ್ದೂರ್
ಬಹಾ-ದ್ದೂ-ರ್-ಇ-ವ-ರು-ಗಳನ್ನು
ಬಹಿ-ರಂಗ
ಬಹಿ-ರಂ-ಗ-ಪ-ಡಿ-ಸಲು
ಬಹಿ-ರಂ-ಗ-ಪ-ಡಿ-ಸಿ-ಕೊ-ಳ್ಳ-ದಿ-ರು-ವುದು
ಬಹಿ-ರಂ-ಗ-ಪ-ಡಿ-ಸಿ-ದರು
ಬಹಿ-ರಂ-ಗ-ಪ-ಡಿ-ಸಿ-ದಾಗ
ಬಹಿ-ರಂ-ಗ-ವಾಗಿ
ಬಹಿ-ರ್ಮುಖ
ಬಹಿ-ರ್ಮು-ಖ-ರಾಗಿ
ಬಹಿ-ರ್ಮು-ಖ-ರಾದ
ಬಹಿ-ಷ್ಕ-ರಿ-ಸ-ಬೇ-ಕೆಂದೂ
ಬಹಿ-ಷ್ಕಾರ
ಬಹಿ-ಷ್ಕೃ-ತ-ನಾ-ಗ-ಬೇ-ಕಾದ
ಬಹಿ-ಷ್ಕೃ-ತ-ವಾದ
ಬಹು
ಬಹು-ಕಾ-ಲದ
ಬಹು-ಕಾ-ಲ-ದಿಂದ
ಬಹು-ಜ-ನ-ಸು-ಖ-ಗಳು
ಬಹು-ಜ-ನ-ಹಿತ
ಬಹು-ತೇಕ
ಬಹು-ದಾ-ಗಿತ್ತು
ಬಹು-ದಾ-ಗಿದೆ
ಬಹು-ದಾ-ದಷ್ಟು
ಬಹುದು
ಬಹು-ದು-ಅ-ವ-ನಿಗೆ
ಬಹು-ದೂರ
ಬಹು-ದೂ-ರದ
ಬಹು-ದೇ-ವತಾ
ಬಹು-ದೊಡ್ಡ
ಬಹು-ಪ-ತಿತ್ವ
ಬಹು-ಪ-ತಿ-ತ್ವ-ವನ್ನು
ಬಹು-ಪಾ-ಲನ್ನು
ಬಹು-ಪಾಲು
ಬಹು-ಬೇಗ
ಬಹು-ಬೇ-ಗನೆ
ಬಹು-ಭಾ-ಗ-ವನ್ನು
ಬಹು-ಮ-ಟ್ಟಿಗೆ
ಬಹು-ಮಾ-ನ-ವನ್ನೋ
ಬಹು-ಮಾ-ನ-ವೊಂ-ದನ್ನು
ಬಹು-ಮು-ಖ-ವಾ-ದುದು
ಬಹು-ಮುಖ್ಯ
ಬಹು-ವಾಗಿ
ಬಹುಶಃ
ಬಾ
ಬಾಂಧ-ವರ
ಬಾಂಧ-ವ್ಯಕ್ಕೆ
ಬಾಂಧ-ವ್ಯ-ವನ್ನು
ಬಾಂಧ-ವ್ಯವು
ಬಾಕಿ
ಬಾಗಲು
ಬಾಗ-ಲೇ-ಬೇಕು
ಬಾಗಿ
ಬಾಗಿತ್ತು
ಬಾಗಿ-ಬಾಗಿ
ಬಾಗಿ-ಲನ್ನು
ಬಾಗಿ-ಲಲ್ಲಿ
ಬಾಗಿ-ಲಿಂದ
ಬಾಗಿ-ಲಿಗೆ
ಬಾಗಿ-ಲಿನ್ನೂ
ಬಾಗಿಲು
ಬಾಗಿ-ಲು-ಗಳನ್ನು
ಬಾಗಿ-ಲು-ಗ-ಳಾದ
ಬಾಗಿ-ಲು-ಗ-ಳಿ-ಲ್ಲ-ದಿ-ರು-ವು-ದ-ರಿಂದ
ಬಾಗಿ-ಲು-ಗಳೇ
ಬಾಗಿ-ಲೊಂ-ದ-ರಿಂದ
ಬಾಗಿ-ಸು-ವು-ದಾ-ಗಿತ್ತು
ಬಾಗ್
ಬಾಚಿ
ಬಾಚಿ-ಕೊಂ-ಡರು
ಬಾಡಿ-ಗೆಗೆ
ಬಾಡಿ-ಗೆಯ
ಬಾಡಿ-ಹೋ-ಯಿತು
ಬಾಣ
ಬಾಣ-ದಂತೆ
ಬಾಣ-ದಿಂದ
ಬಾಣ-ಪ್ರ-ಯೋಗ
ಬಾತ್
ಬಾದ್ಹಾಗೂ
ಬಾಧೆ-ಕೊ-ಟ್ಟಿತು
ಬಾಧೆ-ಪ-ಡು-ತ್ತಿ-ದ್ದರು
ಬಾನಿ
ಬಾನಿ-ಯನ್ನು
ಬಾನಿ-ಯ-ವರು
ಬಾನೆಟ್
ಬಾಬ-ತ್ತನು
ಬಾಬಾ
ಬಾಬಾಜಿ
ಬಾಬಾ-ಜಿ-ಯ-ವರು
ಬಾಬಾರ
ಬಾಬಾ-ರಿಗೆ
ಬಾಬು
ಬಾಬು-ಗಳ
ಬಾಬು-ಗ-ಳೊಂ-ದಿಗೆ
ಬಾಬು-ವಿಗೆ
ಬಾಬು-ವಿನ
ಬಾಬುವೇ
ಬಾಯ
ಬಾಯಲ್ಲಿ
ಬಾಯಾರಿ
ಬಾಯಾ-ರಿಕೆ
ಬಾಯಾ-ರಿದ
ಬಾಯಿ
ಬಾಯಿಂದ
ಬಾಯಿ-ಗಳು
ಬಾಯಿ-ಗಿಟ್ಟು
ಬಾಯಿಗೆ
ಬಾಯಿ-ಪಾಠ
ಬಾಯಿ-ಬ-ಡು-ಕ-ರಿಂ-ದಲ್ಲ
ಬಾಯಿ-ಬಿ-ಚ್ಚ-ಲಿಲ್ಲ
ಬಾಯಿ-ಬಿಟ್ಟು
ಬಾಯಿ-ಮು-ಚ್ಚಿ-ಕೊಂ-ಡಿರಿ
ಬಾಯಿ-ಮು-ಚ್ಚಿಸಿ
ಬಾಯಿಯ
ಬಾಯಿ-ಯಿಂದ
ಬಾಯ್ತುಂಬ
ಬಾಯ್ಮು-ಚ್ಚಿ-ಕೊಂಡು
ಬಾರದ
ಬಾರ-ದಂತೆ
ಬಾರ-ದ-ವ-ರನ್ನು
ಬಾರ-ದಿದ್ದ
ಬಾರ-ದಿ-ದ್ದರೂ
ಬಾರ-ದಿ-ದ್ದಾಗ
ಬಾರ-ದಿ-ರುವು
ಬಾರದು
ಬಾರ-ದು-ಆಗ
ಬಾರ-ದೆಂ-ಬುದು
ಬಾರಾ
ಬಾರಾ-ನಾ-ಗೋ-ರನ್ನು
ಬಾರಾ-ನಾ-ಗೋರ್
ಬಾರಿ
ಬಾರಿಗೆ
ಬಾರಿ-ಸ-ಬೇಕೆ
ಬಾರಿ-ಸಿತು
ಬಾರಿ-ಸಿ-ದ-ನಂತೆ
ಬಾರಿ-ಸು-ತ್ತೀರಿ
ಬಾರ್ನ್ಳಂತೆ
ಬಾರ್ಲಿ
ಬಾಲ
ಬಾಲಕ
ಬಾಲ-ಕನ
ಬಾಲ-ಕ-ನಂತೆ
ಬಾಲ-ಕ-ನನ್ನು
ಬಾಲ-ಕ-ನಿಗೆ
ಬಾಲ-ಕನೂ
ಬಾಲ-ಕ-ನೊ-ಬ್ಬನ
ಬಾಲ-ಕ-ನೊ-ಬ್ಬ-ನಿಗೆ
ಬಾಲ-ಕ-ರಂ-ತಿ-ರು-ತ್ತಾರೆ
ಬಾಲ-ಕರು
ಬಾಲ-ಕಿಯ
ಬಾಲ-ಕಿ-ಯ-ರ-ನ್ನು-ದ್ದೇ-ಶಿಸಿ
ಬಾಲ-ಕ್ರಿ-ಸ್ತನ
ಬಾಲ-ಗಂ-ಗಾ-ಧರ
ಬಾಲ-ಬು-ದ್ಧಿ-ಯು-ಳ್ಳ-ವರು
ಬಾಲ-ವನ್ನು
ಬಾಲ-ವಿ-ಧ-ವೆ-ಯರ
ಬಾಲ-ವಿ-ಧ-ವೆ-ಯ-ರಿಗೆ
ಬಾಲಾ-ಜಿಗೆ
ಬಾಲಾ-ಜಿ-ರಾವ್
ಬಾಲ್ಟಿಕ್
ಬಾಲ್ಟಿ-ಮೋರ್ಗೆ
ಬಾಲ್ಟಿ-ಮೋ-ರ್ನಿಂದ
ಬಾಲ್ಯ
ಬಾಲ್ಯ-ದಿಂ-ದಲೂ
ಬಾಲ್ಯ-ವಿ-ವಾಹ
ಬಾಲ್ಯಾ-ರ-ಭ್ಯ-ದಿಂದ
ಬಾಳ-ಲಾ-ರದು
ಬಾಳ-ಲಾ-ರವು
ಬಾಳಲಿ
ಬಾಳಿಕೆ
ಬಾಳಿ-ನೊಂ-ದಿಗೆ
ಬಾವ-ಲಿ-ಗ-ಳಂತೆ
ಬಾವಿ
ಬಾವಿ-ಗಿಂ-ತಲೂ
ಬಾವಿಯ
ಬಾವಿ-ಯಲ್ಲೇ
ಬಾವಿ-ಯಿಂ-ದಾಚೆ
ಬಾವುಟ
ಬಾವು-ಟ-ಗ-ಳಿ-ರು-ತ್ತವೆ
ಬಾವು-ಟದ
ಬಾವು-ಟ-ವನ್ನು
ಬಾಸ್ಟನ್
ಬಾಸ್ಟನ್ಗೆ
ಬಾಸ್ಟನ್ನ
ಬಾಸ್ಟ-ನ್ನಿಗೆ
ಬಾಸ್ಟ-ನ್ನಿನ
ಬಾಸ್ಟ-ನ್ನಿ-ನಲ್ಲಿ
ಬಾಸ್ಟ-ನ್ನಿ-ನ-ಲ್ಲಿದ್ದ
ಬಾಸ್ಟ-ನ್ನಿ-ನ-ಲ್ಲಿ-ದ್ದಾಗ
ಬಾಸ್ಟ-ನ್ನಿ-ನಿಂದ
ಬಾಸ್ವಾ
ಬಾಹು-ಗಳಿಂದ
ಬಾಹ್ಯ
ಬಾಹ್ಯ-ಜ-ಗ-ತ್ತಿನ
ಬಾಹ್ಯ-ನೋ-ಟವೇ
ಬಾಹ್ಯ-ಪ-ರಿ-ಸ-ರಕ್ಕೆ
ಬಾಹ್ಯ-ಪೂ-ಜೆ-ಗಳನ್ನು
ಬಾಹ್ಯ-ಪೂ-ಜೆ-ಯನ್ನು
ಬಾಹ್ಯ-ವ್ಯ-ಕ್ತಿ-ತ್ವ-ವನ್ನು
ಬಾಹ್ಯಾ-ಚ-ರ-ಣೆ-ಗ-ಳಿಗೆ
ಬಿ
ಬಿಕ-ರಿ-ಯಾ-ಗದ
ಬಿಕ್ಕ-ಲಾ-ರಂ-ಭಿ-ಸಿದ
ಬಿಗಿದು
ಬಿಗಿ-ಯ-ಲ್ಪಟ್ಟ
ಬಿಗಿ-ಯ-ಲ್ಪ-ಟ್ಟಿ-ದೆ-ಯೆಂ-ಬಂತೆ
ಬಿಗಿ-ಯ-ಲ್ಪ-ಟ್ಟಿದ್ದ
ಬಿಗಿ-ಯಾಗಿ
ಬಿಗಿ-ಹಿ-ಡಿದು
ಬಿಗು-ಮಾನ
ಬಿಗು-ಮಾ-ನ-ವ-ನ್ನೆಲ್ಲ
ಬಿಗು-ಮಾನವೂ
ಬಿಚ್ಚಿ
ಬಿಚ್ಚಿ-ಕೊ-ಡು-ವಂತೆ
ಬಿಚ್ಚಿ-ಡು-ತ್ತಿ-ದ್ದರು
ಬಿಚ್ಚಿ-ಯಾ-ರೆಂದು
ಬಿಚ್ಚು
ಬಿಚ್ಚು-ಕಂ-ಠ-ದಿಂದ
ಬಿಚ್ಚು-ಮ-ನ-ಸ್ಸಿನ
ಬಿಚ್ಚು-ಮ-ನ-ಸ್ಸಿ-ನಿಂದ
ಬಿಚ್ಚು-ಮಾತಿ
ಬಿಚ್ಚು-ಮಾ-ತು-ಗಳನ್ನು
ಬಿಚ್ಚೆ-ದೆಯ
ಬಿಟ್ಟ
ಬಿಟ್ಟ-ಮೇಲೆ
ಬಿಟ್ಟರು
ಬಿಟ್ಟರೆ
ಬಿಟ್ಟ-ರೆಂ-ದರೆ
ಬಿಟ್ಟ-ರೇನು
ಬಿಟ್ಟ-ವನು
ಬಿಟ್ಟಿತು
ಬಿಟ್ಟಿತ್ತು
ಬಿಟ್ಟಿದೆ
ಬಿಟ್ಟಿ-ದೆ-ಮು-ದು-ಕರ
ಬಿಟ್ಟಿ-ದ್ದರು
ಬಿಟ್ಟಿ-ದ್ದೇನೆ
ಬಿಟ್ಟಿ-ರುವ
ಬಿಟ್ಟಿ-ರು-ವಂತೆ
ಬಿಟ್ಟು
ಬಿಟ್ಟು-ಕೊ-ಟ್ಟರು
ಬಿಟ್ಟು-ಕೊ-ಟ್ಟಿದ್ದ
ಬಿಟ್ಟು-ಕೊ-ಟ್ಟಿ-ದ್ದೇವೆ
ಬಿಟ್ಟು-ಕೊಟ್ಟು
ಬಿಟ್ಟು-ಕೊ-ಡ-ಬಾ-ರದು
ಬಿಟ್ಟು-ಕೊ-ಡ-ಲಿಲ್ಲ
ಬಿಟ್ಟು-ಕೊ-ಡಲು
ಬಿಟ್ಟು-ಕೊ-ಡಲೂ
ಬಿಟ್ಟು-ಬಿಟ್ಟ
ಬಿಟ್ಟು-ಬಿ-ಟ್ಟರೂ
ಬಿಟ್ಟು-ಬಿ-ಟ್ಟಿ-ದ್ದರು
ಬಿಟ್ಟು-ಬಿ-ಟ್ಟಿ-ದ್ದೇನೆ
ಬಿಟ್ಟು-ಬಿ-ಟ್ಟಿ-ರು-ವಂತೆ
ಬಿಟ್ಟು-ಬಿಟ್ಟು
ಬಿಟ್ಟು-ಬಿಡಿ
ಬಿಟ್ಟು-ಬಿಡು
ಬಿಟ್ಟು-ಬಿ-ಡು-ತ್ತಿ-ದ್ದರು
ಬಿಟ್ಟು-ಬಿ-ಡುವ
ಬಿಟ್ಟು-ಬಿ-ಡು-ವಂತೆ
ಬಿಟ್ಟು-ಬಿ-ಡು-ವು-ದಕ್ಕೇ
ಬಿಟ್ಟು-ಬಿ-ಡು-ವು-ದಾಗಿ
ಬಿಟ್ಟು-ಬಿ-ಡು-ವುದೇ
ಬಿಟ್ಟು-ಹೋ-ಗಲು
ಬಿಟ್ಟು-ಹೋ-ಗಿ-ರುವ
ಬಿಟ್ಟು-ಹೋ-ಗಿಲ್ಲ
ಬಿಟ್ಟು-ಹೋ-ಗು-ವುದು
ಬಿಟ್ಟೇ-ಬಿಟ್ಟ
ಬಿಟ್ಬಿ-ಡ್ರಪ್ಪೋ
ಬಿಡ-ದಂತೆ
ಬಿಡ-ದಿ-ದ್ದರೂ
ಬಿಡ-ದಿ-ದ್ದರೆ
ಬಿಡ-ದಿ-ರಲು
ಬಿಡ-ದಿ-ರುವ
ಬಿಡದೆ
ಬಿಡ-ಬಾ-ರದು
ಬಿಡ-ಬಾ-ರ-ದೆಂ-ಬುದು
ಬಿಡ-ಬೇ-ಕಲ್ಲ
ಬಿಡ-ಬೇಕಾ
ಬಿಡ-ಬೇ-ಕಾ-ಯಿತು
ಬಿಡ-ಬೇಕು
ಬಿಡ-ಬೇಡ
ಬಿಡ-ಲಾ-ರರು
ಬಿಡ-ಲಿ-ದ್ದೇನೆ
ಬಿಡ-ಲಿಲ್ಲ
ಬಿಡಲು
ಬಿಡಲೂ
ಬಿಡಲೇ
ಬಿಡ-ಲೊ-ಲ್ಲರು
ಬಿಡ-ಲೊ-ಲ್ಲವು
ಬಿಡಾ-ರಕ್ಕೆ
ಬಿಡಿ
ಬಿಡಿ-ಗಾ-ಸಾ-ದರೂ
ಬಿಡಿ-ಗಾ-ಸಿ-ಲ್ಲದ
ಬಿಡಿ-ಗಾಸೂ
ಬಿಡಿ-ಸಲು
ಬಿಡಿ-ಸಿ-ಕೊಂ-ಡಿ-ದ್ದರೂ
ಬಿಡಿ-ಸಿ-ಕೊಂಡು
ಬಿಡಿ-ಸಿ-ಕೊ-ಳ್ಳ-ಬೇಕು
ಬಿಡಿ-ಸಿ-ಕೊ-ಳ್ಳ-ಬೇ-ಕೆಂದು
ಬಿಡಿ-ಸಿ-ಕೊ-ಳ್ಳಲು
ಬಿಡಿ-ಸಿದ
ಬಿಡಿ-ಸು-ವುದು
ಬಿಡು
ಬಿಡು-ಗ-ಡೆ-ಮುಕ್ತಿ
ಬಿಡು-ಗ-ಡೆ-ಗಾಗಿ
ಬಿಡು-ಗ-ಡೆ-ಯನ್ನೂ
ಬಿಡು-ಗ-ಡೆ-ಯಾದ
ಬಿಡು-ಗ-ಡೆಯೇ
ಬಿಡುತ್ತ
ಬಿಡು-ತ್ತದೆ
ಬಿಡು-ತ್ತದೋ
ಬಿಡು-ತ್ತಲೂ
ಬಿಡು-ತ್ತವೆ
ಬಿಡು-ತ್ತಾರೆ
ಬಿಡು-ತ್ತಿ-ದ್ದರು
ಬಿಡು-ತ್ತಿ-ರ-ಲಿಲ್ಲ
ಬಿಡುವ
ಬಿಡು-ವಂತೆ
ಬಿಡು-ವರೋ
ಬಿಡು-ವ-ವ-ನಲ್ಲ
ಬಿಡು-ವಿನ
ಬಿಡು-ವಿ-ಲ್ಲದ
ಬಿಡುವು
ಬಿಡು-ವು-ದಾ-ಗಲಿ
ಬಿಡು-ವು-ದಿಲ್ಲ
ಬಿಡು-ವುದು
ಬಿಡು-ವುದೇ
ಬಿಡು-ವೆಯಾ
ಬಿಡೆ
ಬಿತ್ತಲು
ಬಿತ್ತ-ಲ್ಪ-ಡು-ತ್ತ-ದೆಯೋ
ಬಿತ್ತಿ
ಬಿತ್ತಿ-ದರು
ಬಿತ್ತಿ-ದು-ದ-ರಿಂದ
ಬಿತ್ತಿ-ದ್ದೇನೆ
ಬಿತ್ತಿ-ಯಾ-ಗಿತ್ತು
ಬಿತ್ತು
ಬಿದ್ದ
ಬಿದ್ದಂ-ತಾಗಿ
ಬಿದ್ದದ್ದೇ
ಬಿದ್ದರು
ಬಿದ್ದರೂ
ಬಿದ್ದರೆ
ಬಿದ್ದ-ರೆಂದು
ಬಿದ್ದರೇ
ಬಿದ್ದ-ವ-ನೇ-ನಲ್ಲ
ಬಿದ್ದ-ವರ
ಬಿದ್ದ-ವ-ರಿ-ಗಂತೂ
ಬಿದ್ದಾಗ
ಬಿದ್ದಿತು
ಬಿದ್ದಿತ್ತು
ಬಿದ್ದಿದೆ
ಬಿದ್ದಿದ್ದ
ಬಿದ್ದಿ-ದ್ದೆವು
ಬಿದ್ದಿ-ರ-ಲಿಲ್ಲ
ಬಿದ್ದು
ಬಿದ್ದು-ಬಿ-ಟ್ಟರು
ಬಿದ್ದು-ಬಿ-ಡಲು
ಬಿದ್ದು-ಬಿದ್ದು
ಬಿದ್ದುವು
ಬಿದ್ದು-ಹೋ-ಗ-ಬ-ಹು-ದಾ-ಗಿತ್ತು
ಬಿದ್ದು-ಹೋ-ಗಿತ್ತು
ಬಿದ್ದು-ಹೋ-ಗುವ
ಬಿದ್ದು-ಹೋ-ಯಿತು
ಬಿದ್ದೇ-ಬಿ-ಟ್ಟಿದ್ದೆ
ಬಿನ್ನ-ವ-ತ್ತ-ಳೆ-ಯನ್ನು
ಬಿನ್ನ-ವ-ತ್ತೆ-ಳೆ-ಯೊಂ-ದನ್ನು
ಬಿನ್ನ-ವಿ-ಸಿ-ಕೊಂಡ
ಬಿಯಾಲ್
ಬಿಯಾ-ವ-ರ್ನಲ್ಲಿ
ಬಿರಿ-ದು-ಕೊಂಡು
ಬಿರಿ-ಯು-ವಂತೆ
ಬಿರು-ಕು-ಗಳ
ಬಿರು-ಗಾ-ಳಿ-ಅ-ಲೆ-ಗಳ
ಬಿರು-ಗಾ-ಳಿ-ಯೇಳು
ಬಿರುದು
ಬಿರು-ದು-ಗ-ಳಿ-ಗಿಂತ
ಬಿಲ-ಗಳಿಂದ
ಬಿಲ್ವ-ಮಂ-ಗಲ
ಬಿಲ್ವ-ಮಂ-ಗ-ಲನ
ಬಿಳಿ
ಬಿಳಿ-ಗೂ-ದ-ಲಿನ
ಬಿಳಿ-ಬಟ್ಟೆ
ಬಿಳಿಯ
ಬಿಳಿ-ಯರ
ಬಿಳಿ-ಯ-ರಿ-ಗಿಂತ
ಬಿಳಿ-ಯ-ರಿಗೆ
ಬಿಳಿ-ಯರು
ಬಿಳಿ-ಯರೂ
ಬಿಷಪ್
ಬಿಸಾಕಿ
ಬಿಸಾಡಿ
ಬಿಸಾ-ಡಿದ
ಬಿಸಿ
ಬಿಸಿ-ಗಾಳಿ
ಬಿಸಿ-ಬಿಸಿ
ಬಿಸಿ-ಬಿ-ಸಿ-ಯಾ-ಗಿರು
ಬಿಸಿ-ಯೇ-ರು-ತ್ತಿವೆ
ಬಿಸಿ-ಲಿನ
ಬಿಸಿ-ಲಿ-ನಲ್ಲಿ
ಬಿಸಿ-ಲಿ-ನಲ್ಲೇ
ಬಿಸಿಲು
ಬಿಸಿ-ಲ್ಗು-ದುರೆ
ಬಿಸಿ-ಲ್ಗು-ದು-ರೆಯ
ಬಿಸಿ-ಲ್ಗು-ದು-ರೆ-ಯನ್ನು
ಬಿಹಾರ್
ಬೀಗ
ಬೀಜ
ಬೀಜ-ಗ-ಣಿ-ತದ
ಬೀಜ-ಗಳು
ಬೀಜ-ವನ್ನು
ಬೀಜವು
ಬೀಜಾಂ-ಕುರ
ಬೀಟೆ
ಬೀಡಾದ
ಬೀಡು-ಬಿ-ಟ್ಟಿದ್ದ
ಬೀದಿ-ಗಳಲ್ಲಿ
ಬೀದಿಯ
ಬೀದಿ-ಯಲ್ಲಿ
ಬೀರ-ಬ-ಹು-ದೆಂದು
ಬೀರ-ಲಿವೆ
ಬೀರಲು
ಬೀರಿ
ಬೀರಿತು
ಬೀರಿ-ತೆಂ-ದರೆ
ಬೀರಿತ್ತು
ಬೀರಿ-ತ್ತೆಂ-ದರೆ
ಬೀರಿದ
ಬೀರಿ-ದರು
ಬೀರಿ-ದುವು
ಬೀರಿ-ದೆ-ಯೆಂದೇ
ಬೀರಿದ್ದ
ಬೀರಿ-ದ್ದರೋ
ಬೀರಿದ್ದು
ಬೀರಿದ್ದೂ
ಬೀರಿ-ದ್ದೆಂ-ದರೆ
ಬೀರಿವೆ
ಬೀರುತ್ತ
ಬೀರು-ತ್ತಿದೆ
ಬೀರು-ತ್ತಿದ್ದ
ಬೀರು-ತ್ತಿ-ದ್ದ-ರೆಂ-ಬು-ದನ್ನು
ಬೀರುವ
ಬೀರು-ವಂ-ತಹ
ಬೀರು-ವು-ದರ
ಬೀರು-ವುದೂ
ಬೀರೋ-ಣವೆ
ಬೀಳ-ತೊ-ಡ-ಗಿ-ದುವು
ಬೀಳ-ದಂತೆ
ಬೀಳ-ದಿ-ದ್ದರೆ
ಬೀಳದೆ
ಬೀಳ-ಬೇ-ಕಾ-ಗು-ತ್ತದೆ
ಬೀಳ-ಬೇಕು
ಬೀಳಲಿ
ಬೀಳು-ತ್ತದೆ
ಬೀಳು-ತ್ತ-ದೆಯೋ
ಬೀಳು-ತ್ತಾನೆ
ಬೀಳು-ತ್ತಿತ್ತು
ಬೀಳು-ತ್ತಿತ್ತೋ
ಬೀಳು-ತ್ತಿ-ದ್ದರು
ಬೀಳು-ತ್ತಿವೆ
ಬೀಳು-ತ್ತೇನೆ
ಬೀಳುವ
ಬೀಳು-ವಂತೆ
ಬೀಳು-ವುದು
ಬೀಳು-ವುದೇ
ಬೀಳ್ಕೊಂ-ಡರು
ಬೀಳ್ಕೊಂ-ಡಾಗ
ಬೀಳ್ಕೊಂಡು
ಬೀಳ್ಕೊ-ಟ್ಟರು
ಬೀಳ್ಕೊ-ಡಲು
ಬೀಳ್ಕೊ-ಡು-ಗೆಯ
ಬೀಳ್ಕೊ-ಡುವ
ಬೀಳ್ಕೊ-ಳ್ಳ-ಬೇ-ಕಾದ
ಬೀಳ್ಕೊ-ಳ್ಳ-ಲಿ-ದ್ದಾರೆ
ಬೀಳ್ಕೊ-ಳ್ಳು-ವಾಗ
ಬೀಳ್ಗೊಂ-ಡರು
ಬೀಳ್ಗೊಂಡು
ಬೀಳ್ಗೊಂ-ಡೇ-ಬಿ-ಟ್ಟರು
ಬೀಳ್ಗೊ-ಡಲು
ಬೀಳ್ಗೊ-ಡು-ವು-ದ-ಕ್ಕಾಗಿ
ಬೀಸ-ಣಿಗೆ
ಬೀಸ-ಲಾ-ರಂ-ಭಿ-ಸಿತು
ಬೀಸಲು
ಬೀಸಿ
ಬೀಸಿ-ದರು
ಬೀಸಿ-ದಾಗ
ಬೀಸಿದೆ
ಬೀಸು-ತ್ತಿ-ದ್ದಂತೆ
ಬೀಸು-ತ್ತಿ-ದ್ದರು
ಬೀಸು-ತ್ತಿ-ರುವ
ಬೀಸುವ
ಬುಟ್ಟಿ
ಬುಟ್ಟಿ-ಯಲ್ಲಿ
ಬುಡಕ್ಕೆ
ಬುಡ-ದಲ್ಲಿ
ಬುಡ-ಮೇಲು
ಬುತ್ತಿ-ಯನ್ನೂ
ಬುದು
ಬುದ್ದ-ದೇ-ವನ
ಬುದ್ದನ
ಬುದ್ದಿ-ಜೀ-ವಿ-ಗಳ
ಬುದ್ದಿ-ವಾದ
ಬುದ್ಧ
ಬುದ್ಧ-ಘೋ-ಷ-ನೆಂಬ
ಬುದ್ಧನ
ಬುದ್ಧ-ನಂತೆ
ಬುದ್ಧ-ನದೇ
ಬುದ್ಧ-ನನ್ನು
ಬುದ್ಧ-ನಿಂದ
ಬುದ್ಧನು
ಬುದ್ಧನೊ
ಬುದ್ಧ-ಭ-ಗ-ವಂ-ತನ
ಬುದ್ಧರು
ಬುದ್ಧಿ
ಬುದ್ಧಿ
ಬುದ್ಧಿ-ಅ-ನು-ಭ-ವ-ಗಳ
ಬುದ್ಧಿ-ಮ-ನ-ಸ್ಸು-ಗಳು
ಬುದ್ಧಿ-ಹೃ-ದ-ಯ-ಗ-ಳೊಂ-ದಿಗೆ
ಬುದ್ಧಿ-ಕೆಟ್ಟ
ಬುದ್ಧಿ-ಗಿನ್ನೂ
ಬುದ್ಧಿಗೂ
ಬುದ್ಧಿಗೆ
ಬುದ್ಧಿಗೇ
ಬುದ್ಧಿ-ಗೇಡಿ
ಬುದ್ಧಿ-ಗೇ-ಡಿ-ತ-ನ-ದಿಂ-ದಾ-ಗಿಯೇ
ಬುದ್ಧಿ-ಜೀ-ವಿ-ಗಳ
ಬುದ್ಧಿ-ಜೀ-ವಿ-ಗಳನ್ನು
ಬುದ್ಧಿ-ಜೀ-ವಿ-ಗಳು
ಬುದ್ಧಿ-ಜೀ-ವಿ-ಗಳೂ
ಬುದ್ಧಿ-ಜೀ-ವಿಯೂ
ಬುದ್ಧಿ-ಪ-ಲ್ಲ-ಪ-ಟ-ಗೊ-ಳಿ-ಸು-ವು-ದ-ಕ್ಕಾಗಿ
ಬುದ್ಧಿ-ಪೂ-ರ್ವ-ಕ-ವಾ-ಗಿದ್ದು
ಬುದ್ಧಿ-ಬಲ
ಬುದ್ಧಿ-ಭ್ರ-ಮ-ಣೆಯೂ
ಬುದ್ಧಿ-ಮ-ಟ್ಟಕ್ಕೆ
ಬುದ್ಧಿ-ಮತ್ತೆ
ಬುದ್ಧಿ-ಮ-ತ್ತೆ-ಗಳು
ಬುದ್ಧಿ-ಮ-ತ್ತೆಯ
ಬುದ್ಧಿ-ಮ-ತ್ತೆ-ಯನ್ನು
ಬುದ್ಧಿ-ಮ-ತ್ತೆ-ಯನ್ನೂ
ಬುದ್ಧಿ-ಮಾ-ತು-ಗ-ಳೊಂ-ದಿಗೆ
ಬುದ್ಧಿ-ಮಾನ್
ಬುದ್ಧಿಯ
ಬುದ್ಧಿ-ಯನ್ನು
ಬುದ್ಧಿ-ಯನ್ನೂ
ಬುದ್ಧಿ-ಯಲ್ಲಿ
ಬುದ್ಧಿ-ಯಲ್ಲೂ
ಬುದ್ಧಿ-ಯ-ವ-ರಾದ
ಬುದ್ಧಿ-ಯ-ವರು
ಬುದ್ಧಿ-ಯಿಂದ
ಬುದ್ಧಿ-ಯಿಂ-ದ-ರಿ-ಯ-ಲಾ-ಗದ
ಬುದ್ಧಿ-ಯಿಂ-ದುಂ-ಟಾದ
ಬುದ್ಧಿ-ಯಿ-ಲ್ಲದೆ
ಬುದ್ಧಿಯು
ಬುದ್ಧಿ-ಯು-ಳ್ಳ-ವನ
ಬುದ್ಧಿಯೂ
ಬುದ್ಧಿಯೇ
ಬುದ್ಧಿ-ವಂತ
ಬುದ್ಧಿ-ವಂ-ತ-ನಂತೆ
ಬುದ್ಧಿ-ವಂ-ತ-ನಾದ
ಬುದ್ಧಿ-ವಂ-ತನೂ
ಬುದ್ಧಿ-ವಂ-ತರ
ಬುದ್ಧಿ-ವಂ-ತ-ರಾದ
ಬುದ್ಧಿ-ವಂ-ತರು
ಬುದ್ಧಿ-ವಂ-ತ-ರು-ನಾನೋ
ಬುದ್ಧಿ-ವಂ-ತರೂ
ಬುದ್ಧಿ-ವಂ-ತರೇ
ಬುದ್ಧಿ-ವಂ-ತಿಕೆ
ಬುದ್ಧಿ-ವಂ-ತಿ-ಕೆ-ಯಲ್ಲ
ಬುದ್ಧಿ-ವಂತೆ
ಬುದ್ಧಿ-ಶಕ್ತಿ
ಬುದ್ಧಿ-ಶ-ಕ್ತಿ-ಗ-ಳಿಗೆ
ಬುದ್ಧಿ-ಶ-ಕ್ತಿಗೂ
ಬುದ್ಧಿ-ಶ-ಕ್ತಿಗೆ
ಬುದ್ಧಿ-ಶ-ಕ್ತಿಯ
ಬುದ್ಧಿ-ಶ-ಕ್ತಿ-ಯನ್ನು
ಬುದ್ಧಿ-ಶ-ಕ್ತಿ-ಯಿಂದ
ಬುದ್ಧಿ-ಶ-ಕ್ತಿ-ಯಿಂ-ದಲೂ
ಬುದ್ಧಿ-ಶ-ಕ್ತಿ-ಯಿಂ-ದು-ದಿ-ಸಿದ
ಬುದ್ಧಿ-ಶಾ-ಲಿ-ಗಳು
ಬುದ್ಧಿ-ಶಾ-ಲಿ-ಗಳೂ
ಬುದ್ಧಿ-ಸಾ-ಮ-ರ್ಥ್ಯದ
ಬುದ್ಧಿ-ಸೂ-ಕ್ಷ್ಮ-ತೆ-ಯನ್ನು
ಬುಧ-ವಾರ
ಬುನಾ-ದಿ-ಯನ್ನು
ಬುಲ್
ಬುಲ್ಗೆ
ಬುಲ್ಲಳ
ಬುಲ್ಲ-ಳಿಗೆ
ಬುಲ್ಳ
ಬುಲ್ಳಿಂದ
ಬೂಟಾ-ಟಿ-ಕೆ-ಯ-ಲ್ಲಿಲ್ಲ
ಬೂಟಾ-ಟಿ-ಕೆಯೋ
ಬೂಟು-ಗಳನ್ನು
ಬೂಟ್ಗಳನ್ನು
ಬೂಟ್ಸ್
ಬೂದಿಯ
ಬೂದಿ-ಯಾ-ಗಲಿ
ಬೃಂದಾ-ವನ
ಬೃಂದಾ-ವ-ನ-ದಲ್ಲಿ
ಬೃಹತ್
ಬೃಹ-ತ್ತ-ರ-ವಾದ
ಬೃಹ-ತ್ತಾದ
ಬೃಹ-ತ್ಪ್ರ-ಮಾ-ಣ-ದಲ್ಲಿ
ಬೃಹ-ತ್ಸ-ಭೆ-ಯೊಂ-ದನ್ನು
ಬೃಹ-ತ್ಸ-ಭೆ-ಯೊಂದು
ಬೃಹ್ಮ-ವಾ-ದಿನ್ಗೆ
ಬೆಂಕಿ
ಬೆಂಕಿ-ಕ-ಡ್ಡಿಯ
ಬೆಂಕಿ-ಯಂ-ತಹ
ಬೆಂಕಿ-ಯನ್ನೇ
ಬೆಂಕಿ-ಯ-ಲ್ಲಾ-ಗಲಿ
ಬೆಂಗ-ಳೂ-ರಿಗೆ
ಬೆಂಗ-ಳೂ-ರಿನ
ಬೆಂಗ-ಳೂ-ರಿ-ನಲ್ಲಿ
ಬೆಂಗ-ಳೂ-ರಿ-ನ-ವ-ರಾದ
ಬೆಂಗ-ಳೂರು
ಬೆಂಗ-ಳೂ-ರು-ಗಳಲ್ಲಿ
ಬೆಂಚು-ಗಳನ್ನು
ಬೆಂಜ-ಮಿ-ನ್ಸ್ಯಾ-ನ್ಬಾ-ರ್ನ್
ಬೆಂಡಾ-ಗಿದ್ದ
ಬೆಂದ
ಬೆಂದ-ನೆಂ-ದರೆ
ಬೆಂದು
ಬೆಂಬಲ
ಬೆಂಬ-ಲದ
ಬೆಂಬ-ಲ-ವನ್ನು
ಬೆಂಬ-ಲ-ವನ್ನೂ
ಬೆಂಬ-ಲ-ವಾಗಿ
ಬೆಂಬ-ಲ-ವಾ-ಗಿದೆ
ಬೆಂಬ-ಲ-ವಾ-ಗಿದ್ದ
ಬೆಂಬ-ಲ-ವಾ-ಗಿ-ದ್ದೇವೆ
ಬೆಂಬ-ಲ-ವಾದ
ಬೆಂಬ-ಲ-ವಿತ್ತು
ಬೆಂಬ-ಲ-ವಿದೆ
ಬೆಂಬ-ಲ-ವಿ-ದೆಯೆ
ಬೆಂಬ-ಲ-ವಿ-ರ-ಲಿಲ್ಲ
ಬೆಂಬ-ಲವು
ಬೆಂಬಲಿ
ಬೆಂಬ-ಲಿಗ
ಬೆಂಬ-ಲಿ-ಗ-ನಾಗಿ
ಬೆಂಬ-ಲಿ-ಗ-ರಲ್ಲಿ
ಬೆಂಬ-ಲಿ-ಗ-ರಾಗಿ
ಬೆಂಬ-ಲಿ-ಗ-ರಾ-ಗಿದ್ದ
ಬೆಂಬ-ಲಿ-ಗ-ರಾದ
ಬೆಂಬ-ಲಿ-ಗ-ರಿ-ಗಂತೂ
ಬೆಂಬ-ಲಿ-ಗರು
ಬೆಂಬ-ಲಿ-ಗಳೂ
ಬೆಂಬ-ಲಿಸಿ
ಬೆಂಬ-ಲಿ-ಸಿದ
ಬೆಂಬ-ಲಿ-ಸಿ-ದರು
ಬೆಂಬ-ಲಿ-ಸಿ-ದ-ವರೂ
ಬೆಂಬ-ಲಿ-ಸು-ವ-ವನು
ಬೆಂಬ-ಲಿ-ಸು-ವ-ವನೇ
ಬೆಂಬಿ-ಡದ
ಬೆಂಬಿ-ಸು-ವ-ವರು
ಬೆಕ್ಕ-ಸ-ಬೆ-ರ-ಗಾ-ದರು
ಬೆಕ್ಕು
ಬೆಚ್ಚ-ನೆಯ
ಬೆಟ್ಟ
ಬೆಟ್ಟ-ಗಳ
ಬೆಟ್ಟ-ಗಳಿಂದ
ಬೆಟ್ಟ-ಗ-ಳಿಗೂ
ಬೆಟ್ಟ-ಗಳು
ಬೆಟ್ಟ-ಗು-ಡ್ಡ-ಗಳ
ಬೆಟ್ಟದ
ಬೆಟ್ಟನ್ನೂ
ಬೆಟ್ಟ-ವ-ನ್ನೇರಿ
ಬೆಟ್ಟ-ವಿದೆ
ಬೆಟ್ಟ-ವಿ-ರು-ತ್ತದೆ
ಬೆಟ್ಟವು
ಬೆದರಿ
ಬೆದ-ರಿ-ಕೆ-ಯನ್ನೂ
ಬೆದ-ರಿ-ಸ-ಬ-ಹುದು
ಬೆದ-ರಿ-ಸ-ಲಾ-ಗದ
ಬೆದ-ರಿಸಿ
ಬೆದ-ರಿ-ಸು-ತ್ತೇನೆ
ಬೆನ್ನ
ಬೆನ್ನ-ಟ್ಟಲಿ
ಬೆನ್ನ-ಟ್ಟ-ಲಿ-ಲ್ಲವೆ
ಬೆನ್ನಟ್ಟಿ
ಬೆನ್ನ-ಟ್ಟಿದ
ಬೆನ್ನ-ಟ್ಟಿ-ರುವ
ಬೆನ್ನ-ಟ್ಟು-ವ-ವರು
ಬೆನ್ನ-ಟ್ಟು-ವು-ದ-ರಲ್ಲೇ
ಬೆನ್ನ-ಮೇಲೆ
ಬೆನ್ನ-ಹಿಂದೆ
ಬೆನ್ನಿಗೆ
ಬೆನ್ನಿ-ನ-ಲ್ಲಲ್ಲ
ಬೆನ್ನು
ಬೆನ್ನು-ಹ-ತ್ತಿ-ದ್ದರು
ಬೆಪ್ಪಾಗಿ
ಬೆಪ್ಪು-ಗಟ್ಟಿ
ಬೆರ-ಗಾಗಿ
ಬೆರ-ಗಾ-ಗಿ-ದ್ದ-ರ-ಲ್ಲದೆ
ಬೆರ-ಗಾ-ಗಿ-ಸು-ವಂ-ತಹ
ಬೆರ-ಗಾದ
ಬೆರ-ಗಾ-ದರು
ಬೆರ-ಗಾ-ದಳು
ಬೆರ-ಗಾದೆ
ಬೆರ-ಗು-ಗೊ-ಳಿ-ಸಿತು
ಬೆರ-ಗು-ಗೊ-ಳಿ-ಸಿತ್ತು
ಬೆರ-ಗು-ಗೊ-ಳಿ-ಸಿ-ದರೆ
ಬೆರ-ಗು-ಗೊ-ಳಿ-ಸಿ-ಬಿ-ಟ್ಟಿತು
ಬೆರ-ಗು-ಗೊ-ಳಿ-ಸು-ತ್ತಾರೆ
ಬೆರ-ಗು-ಗೊ-ಳಿ-ಸುವ
ಬೆರ-ಗು-ಗೊ-ಳಿ-ಸು-ವಂ-ತಿದ್ದು
ಬೆರ-ಗು-ಗೊ-ಳಿ-ಸು-ವಂ-ಥದೂ
ಬೆರತು
ಬೆರಳ
ಬೆರ-ಳ-ಚ್ಚಿ-ನಲ್ಲಿ
ಬೆರ-ಳಚ್ಚು
ಬೆರ-ಳ-ತು-ದಿ-ಗಳಲ್ಲಿ
ಬೆರ-ಳು-ಮಾಡಿ
ಬೆರ-ಳೆ-ಣಿ-ಕೆಯ
ಬೆರೆ-ತ-ನಂತೆ
ಬೆರೆ-ತಾಗ
ಬೆರೆ-ತಿ-ದ್ದೇನೆ
ಬೆರೆತು
ಬೆರೆ-ಯ-ಬೇಕು
ಬೆರೆಯು
ಬೆರೆ-ಯು-ತ್ತಿ-ದ್ದರು
ಬೆರೆ-ಯು-ತ್ತಿ-ದ್ದರೋ
ಬೆರೆ-ಸಿದ
ಬೆರೆ-ಸಿ-ದರು
ಬೆಲೆ
ಬೆಲೆ-ಗಳು
ಬೆಲೆಗೇ
ಬೆಲೆ-ಬಾ-ಳುವ
ಬೆಲೆಯ
ಬೆಲೆ-ಯ-ನ್ನಾ-ದರೂ
ಬೆಲೆ-ಯನ್ನು
ಬೆಲ್
ಬೆಲ್ಹೇ-ಲ್ಳಿಗೆ
ಬೆಳ-ಕನ್ನು
ಬೆಳ-ಕಾದ
ಬೆಳ-ಕಿಗೆ
ಬೆಳ-ಕಿನ
ಬೆಳ-ಕಿ-ನತ್ತ
ಬೆಳ-ಕಿ-ನಲ್ಲಿ
ಬೆಳ-ಕಿ-ನಲ್ಲೇ
ಬೆಳ-ಕಿ-ನಿಂದ
ಬೆಳ-ಕಿ-ನೆ-ಡೆಗೆ
ಬೆಳಕು
ಬೆಳ-ಕು-ಕ-ತ್ತ-ಲೆ-ಗಳ
ಬೆಳ-ಕೊಂದು
ಬೆಳ-ಗ-ಬಲ್ಲ
ಬೆಳ-ಗಲು
ಬೆಳ-ಗಾ-ಗು-ತ್ತಿ-ದ್ದಂತೆ
ಬೆಳ-ಗಾ-ದರೆ
ಬೆಳ-ಗಾ-ವಿಗೆ
ಬೆಳ-ಗಾ-ವಿಯ
ಬೆಳ-ಗಾ-ವಿ-ಯನ್ನು
ಬೆಳ-ಗಾ-ವಿ-ಯಲ್ಲಿ
ಬೆಳ-ಗಾ-ವಿ-ಯ-ಲ್ಲಿ-ದ್ದಾ-ಗಲೇ
ಬೆಳ-ಗಾ-ವಿ-ಯಿಂದ
ಬೆಳಗಿ
ಬೆಳ-ಗಿತು
ಬೆಳ-ಗಿ-ದರು
ಬೆಳ-ಗಿನ
ಬೆಳ-ಗಿ-ನಿಂದ
ಬೆಳ-ಗಿ-ನಿಂ-ದಲೂ
ಬೆಳ-ಗಿ-ಸ-ದಿ-ರು-ವು-ದಿಲ್ಲ
ಬೆಳ-ಗಿ-ಸಿದ
ಬೆಳ-ಗು-ತ್ತಿತ್ತು
ಬೆಳ-ಗು-ತ್ತಿದ್ದ
ಬೆಳ-ಗು-ತ್ತಿ-ದ್ದುವು
ಬೆಳ-ಗು-ತ್ತಿ-ರು-ವಂತೆ
ಬೆಳ-ಗು-ತ್ತಿ-ರು-ವುದು
ಬೆಳ-ಗು-ತ್ತೀಯೆ
ಬೆಳಗ್ಗೆ
ಬೆಳ-ಗ್ಗೆ-ಸಂಜೆ
ಬೆಳ-ವ-ಣಿಗೆ
ಬೆಳ-ವ-ಣಿ-ಗೆಗೆ
ಬೆಳ-ವ-ಣಿ-ಗೆ-ಗೊಂ-ಡಿದ್ದು
ಬೆಳ-ವ-ಣಿ-ಗೆಯ
ಬೆಳ-ವ-ಣಿ-ಗೆ-ಯನ್ನು
ಬೆಳ-ವ-ಣಿ-ಗೆ-ಯಲ್ಲಿ
ಬೆಳ-ವ-ಣಿ-ಗೆಯು
ಬೆಳ-ಸಿ-ದರು
ಬೆಳಿಗ್ಗೆ
ಬೆಳಿ-ಗ್ಗೆ-ಸಂಜೆ
ಬೆಳೆದ
ಬೆಳೆ-ದಾಗ
ಬೆಳೆ-ದಿತ್ತು
ಬೆಳೆ-ದಿ-ತ್ತೆಂ-ದರೆ
ಬೆಳೆ-ದಿದೆ
ಬೆಳೆ-ದಿ-ರುವ
ಬೆಳೆ-ದಿ-ರು-ವುದನ್ನು
ಬೆಳೆದು
ಬೆಳೆ-ದು-ಕೊಂಡು
ಬೆಳೆ-ದು-ನಿಂ-ತರು
ಬೆಳೆ-ದು-ಬಂ-ದ-ವು-ಗ-ಳ-ಲ್ಲವೆ
ಬೆಳೆ-ದು-ಬಂ-ದಿದ್ದ
ಬೆಳೆ-ಯ-ದಂತೆ
ಬೆಳೆ-ಯನ್ನು
ಬೆಳೆ-ಯ-ಬೇ-ಕಾ-ದರೆ
ಬೆಳೆ-ಯ-ಬೇಕು
ಬೆಳೆ-ಯಲು
ಬೆಳೆಯಿ
ಬೆಳೆ-ಯಿತು
ಬೆಳೆ-ಯು-ತ್ತದೆ
ಬೆಳೆ-ಯು-ತ್ತಿದ್ದ
ಬೆಳೆ-ಯು-ತ್ತಿ-ರುವ
ಬೆಳೆ-ಯು-ವಂತೆ
ಬೆಳೆ-ಸ-ಲಾ-ಗುವ
ಬೆಳೆ-ಸಲು
ಬೆಳೆಸಿ
ಬೆಳೆ-ಸಿ-ಕೊಂಡ
ಬೆಳೆ-ಸಿ-ಕೊಂ-ಡರು
ಬೆಳೆ-ಸಿ-ಕೊಂ-ಡ-ವ-ರಿಗೆ
ಬೆಳೆ-ಸಿ-ಕೊಂಡು
ಬೆಳೆ-ಸಿ-ಕೊ-ಳ್ಳು-ತ್ತಿ-ದ್ದರು
ಬೆಳೆ-ಸಿದ
ಬೆಳೆ-ಸಿ-ದರು
ಬೆಳೆ-ಸಿ-ದರೆ
ಬೆಳೆ-ಸು-ವಂತೆ
ಬೆಳ್ಳಿ-ನಾ-ಣ್ಯದ
ಬೆಳ್ಳಿಯ
ಬೆವ-ರಿ-ಳಿಸಿ
ಬೆಸೆಂ-ಟರ
ಬೆಸೆ-ಯಲು
ಬೆಸೆ-ಯ-ಲ್ಪ-ಟ್ಟಿ-ರಲಿ
ಬೆಸ್ತ-ಮ-ಕ್ಕಳು
ಬೆಸ್ತರೇ
ಬೆಸ್ಸಿ
ಬೆಸ್ಸಿ-ಮೆ-ಕ್ಲಾ-ಡಳ
ಬೆಸ್ಸಿಯ
ಬೇಕಂತೆ
ಬೇಕಲ್ಲ
ಬೇಕಾ
ಬೇಕಾ-ಗ-ಬ-ಹು-ದಾದ
ಬೇಕಾ-ಗ-ಬ-ಹು-ದಾ-ದರೂ
ಬೇಕಾಗಿ
ಬೇಕಾ-ಗಿತ್ತು
ಬೇಕಾ-ಗಿದೆ
ಬೇಕಾ-ಗಿ-ದ್ದದ್ದು
ಬೇಕಾ-ಗಿ-ದ್ದ-ರಿಂದ
ಬೇಕಾ-ಗಿ-ದ್ದ-ವನು
ಬೇಕಾ-ಗಿ-ದ್ದಾರೆ
ಬೇಕಾ-ಗಿ-ದ್ದುವು
ಬೇಕಾ-ಗಿ-ದ್ದು-ವೇನೋ
ಬೇಕಾ-ಗಿ-ರು-ವುದು
ಬೇಕಾ-ಗಿ-ರು-ವು-ದೇ-ನೆಂ-ದರೆ
ಬೇಕಾ-ಗಿಲ್ಲ
ಬೇಕಾ-ಗು-ತ್ತದೆ
ಬೇಕಾ-ಗು-ತ್ತಿತ್ತು
ಬೇಕಾ-ಗು-ವಂತೆ
ಬೇಕಾದ
ಬೇಕಾ-ದಂ-ತಹ
ಬೇಕಾ-ದಂತೆ
ಬೇಕಾ-ದದ್ದು
ಬೇಕಾ-ದರೂ
ಬೇಕಾ-ದರೆ
ಬೇಕಾ-ದಷ್ಟು
ಬೇಕಾ-ದಾಗ
ಬೇಕಾ-ದು-ದ-ನ್ನೆಲ್ಲ
ಬೇಕಾ-ದು-ದ-ರಿಂದ
ಬೇಕಾ-ದುದು
ಬೇಕಾ-ದು-ದೆಂ-ದರೆ
ಬೇಕಾ-ದು-ದೆ-ಲ್ಲ-ವನ್ನೂ
ಬೇಕಾ-ದುವು
ಬೇಕಾ-ದ್ದ-ಕ್ಕಿಂತ
ಬೇಕಾ-ಯಿತು
ಬೇಕಾ-ಯಿ-ತೆಂ-ಬು-ದನ್ನು
ಬೇಕಿತ್ತು
ಬೇಕಿ-ರ-ಲಿಲ್ಲ
ಬೇಕಿಲ್ಲ
ಬೇಕು
ಬೇಕು-ಅದು
ಬೇಕು-ಗ-ಳಿಂ-ದಲೂ
ಬೇಕು-ಜಾ-ಜ್ವ-ಲ್ಯ-ಮಾ-ನ-ರಾದ
ಬೇಕು-ಸ್ತ್ರೀ-ಪು-ರು-ಷ-ರಿ-ಬ್ಬರೂ
ಬೇಕೆ
ಬೇಕೆಂ-ದ-ರದು
ಬೇಕೆಂ-ದರೆ
ಬೇಕೆಂದು
ಬೇಕೆಂ-ದು-ಕೊಂ-ಡಿ-ದ್ದೇನೆ
ಬೇಕೆಂದೂ
ಬೇಕೆಂಬ
ಬೇಕೆಂ-ಬಂತೆ
ಬೇಕೆಂ-ಬಷ್ಟು
ಬೇಕೆಂ-ಬು-ದರ
ಬೇಕೆಂ-ಬುದು
ಬೇಕೇ
ಬೇಕೇನು
ಬೇಕೇ-ಬೇಕು
ಬೇಕೋ
ಬೇಗ
ಬೇಗಂ-ಬ-ಜಾ-ರಿನ
ಬೇಗನೆ
ಬೇಗ-ನೆಯೇ
ಬೇಗ-ಬೇಗ
ಬೇಗ-ಬೇ-ಗನೆ
ಬೇಗೆ
ಬೇಗೆ-ಯಲ್ಲಿ
ಬೇಟೆ-ಯಾ-ಡುತ್ತ
ಬೇಟೆ-ಯಾ-ಡುವ
ಬೇಡ
ಬೇಡ-ಬಾ-ರದು
ಬೇಡಲೂ
ಬೇಡಲೇ
ಬೇಡವೆ
ಬೇಡ-ವೆಂ-ದಾಗ
ಬೇಡವೇ
ಬೇಡವೋ
ಬೇಡಿ
ಬೇಡಿಕೆ
ಬೇಡಿ-ಕೆಗೆ
ಬೇಡಿ-ಕೆಯ
ಬೇಡಿ-ಕೆ-ಯನ್ನು
ಬೇಡಿ-ಕೆಯು
ಬೇಡಿ-ಕೊಂಡ
ಬೇಡಿ-ಕೊಂ-ಡರು
ಬೇಡಿ-ಕೊಂ-ಡಳು
ಬೇಡಿ-ಕೊಂ-ಡಿದ್ದ
ಬೇಡಿ-ಕೊಂಡು
ಬೇಡಿ-ಕೊ-ಳ್ಳು-ತ್ತೇ-ನೆ-ಪೊ-ಳ್ಳು-ಧ-ರ್ಮ-ಗ-ಳೊಂ-ದಿಗೆ
ಬೇಡಿದ
ಬೇಡಿ-ದರು
ಬೇಡಿ-ದ-ವ-ರಿಗೆ
ಬೇಡುತ್ತ
ಬೇಡು-ತ್ತಾರೆ
ಬೇಡು-ವು-ದಿಲ್ಲ
ಬೇಡು-ವುದು
ಬೇಡು-ವುದೇ
ಬೇಡ್ರಪ್ಪೋ
ಬೇನೆ
ಬೇಯಿ-ಸಿ-ಕೊ-ಳ್ಳಲು
ಬೇಯಿ-ಸಿದ
ಬೇರಲ್ಲ
ಬೇರಾ-ಗ-ಬೇ-ಕಾ-ದರೆ
ಬೇರಾ-ರಿಗೂ
ಬೇರಾರು
ಬೇರಾರೂ
ಬೇರಾವ
ಬೇರಾ-ವು-ದಕ್ಕೂ
ಬೇರಾ-ವು-ದರ
ಬೇರಾ-ವು-ದ-ರಿಂದ
ಬೇರಾ-ವು-ದಾ-ದರೂ
ಬೇರಾ-ವುದೇ
ಬೇರಾ-ವುದೋ
ಬೇರು
ಬೇರು-ಗಳು
ಬೇರೂರಿ
ಬೇರೂ-ರಿದ್ದು
ಬೇರೆ
ಬೇರೆ-ಡೆಗೆ
ಬೇರೆ-ಬೇರೆ
ಬೇರೆ-ಬೇ-ರೆ-ಯಾಗಿ
ಬೇರೆ-ಯ-ವ-ರಲ್ಲ
ಬೇರೆ-ಯಾ-ಗಲು
ಬೇರೆ-ಯಾಗಿ
ಬೇರೆ-ಯಾ-ಗಿ-ದ್ದುವು
ಬೇರೆ-ಯಾ-ಗಿ-ರಲು
ಬೇರೆಯೇ
ಬೇರೆಲ್ಲ
ಬೇರೆಲ್ಲೂ
ಬೇರೇ-ನನ್ನೂ
ಬೇರೇನೂ
ಬೇರೊಂದು
ಬೇರೊಬ್ಬ
ಬೇರ್ಪಟ್ಟು
ಬೇಲಿ
ಬೇಲಿಯ
ಬೇಲೂ-ರಿನ
ಬೇಳೆ
ಬೇಳೆ-ಯನ್ನು
ಬೇಸತ್ತ
ಬೇಸತ್ತು
ಬೇಸರ
ಬೇಸ-ರ-ಸಂ-ಕಟ
ಬೇಸ-ರ-ಗೊಂ-ಡಿದೆ
ಬೇಸ-ರ-ಗೊಂಡು
ಬೇಸ-ರದ
ಬೇಸ-ರ-ದಿಂದ
ಬೇಸ-ರ-ವಾ-ಗಿ-ರ-ಬೇ-ಕ-ಲ್ಲವೆ
ಬೇಸ-ರ-ವಾ-ಯಿ-ತು-ಅ-ಪ-ಪ್ರ-ಚಾರ
ಬೇಸ-ರವೂ
ಬೇಸ-ರಿ-ಸಿ-ಕೊಂ-ಡ-ವ-ರಲ್ಲ
ಬೇಸ-ರಿ-ಸಿ-ಕೊ-ಳ್ಳದೆ
ಬೇಸ-ರಿ-ಸಿ-ಕೊ-ಳ್ಳು-ತ್ತಿ-ರ-ಲಿಲ್ಲ
ಬೇಸ-ರಿ-ಸು-ತ್ತಿರ
ಬೇಸಾ-ಯ-ದಲ್ಲಿ
ಬೇಸಿಗೆ
ಬೇಸಿ-ಗೆ-ಮ-ನೆಗೆ
ಬೇಸಿ-ಗೆ-ಮ-ನೆ-ಯಲ್ಲಿ
ಬೇಸಿ-ಗೆಯ
ಬೇಸಿ-ಗೆ-ಯನ್ನು
ಬೇಸಿ-ಗೆ-ಯ-ಲ್ಲಂತೂ
ಬೇಸಿ-ಗೆ-ಯಲ್ಲಿ
ಬೇಸಿ-ಗೆ-ಯಲ್ಲೇ
ಬೇಸಿ-ಗೆಯು
ಬೈಗಳ
ಬೈಗು-ಳ-ಗಳನ್ನು
ಬೈಗು-ಳ-ವ-ನ್ನೆಲ್ಲ
ಬೈಠಕ್
ಬೈಠ-ಕ್ಖಾ-ನೆಯ
ಬೈಠ-ಕ್ಖಾ-ನೆ-ಯಲ್ಲಿ
ಬೈದರೆ
ಬೈದು
ಬೈಬ-ಲಿನ
ಬೈಬ-ಲಿ-ನಲ್ಲಿ
ಬೈಬ-ಲಿ-ನಿಂದ
ಬೈಬಲ್
ಬೈಯ
ಬೈಯಲಿ
ಬೈಯಲು
ಬೈಯು-ತ್ತೀರಿ
ಬೈಯು-ವಾ-ಗ-ಲೆಲ್ಲ
ಬೈಯೋ-ಣ-ವೆಂದರೆ
ಬೈರ-ನ್ನರ
ಬೈರಾ-ಗಿ-ಗ-ಳ-ಲ್ಲೊ-ಬ್ಬ-ರಷ್ಟೆ
ಬೈಸಿ-ಕೊಂ-ಡರೂ
ಬೊಂಬೆ-ಗ-ಳಂತೆ
ಬೊಗ-ಳು-ತ್ತದೆ
ಬೊಗ-ಳು-ವ-ವರು
ಬೊಗಳೆ
ಬೊಟಾ-ನಿ-ಕಲ್
ಬೊಬ್ಬಿ-ಡುವ
ಬೊಬ್ಬೆ
ಬೋಗಿಯ
ಬೋಗಿ-ಯಲ್ಲಿ
ಬೋಗಿ-ಯಲ್ಲೇ
ಬೋಗಿ-ಯೊ-ಳಗೇ
ಬೋಧ-ಕ-ನನ್ನೂ
ಬೋಧ-ಕ-ರಾಗಿ
ಬೋಧನಾ
ಬೋಧ-ನಾ-ಕಾ-ರ್ಯ-ಗಳನ್ನು
ಬೋಧ-ನಾ-ಕೇಂ-ದ್ರ-ವನ್ನು
ಬೋಧನೆ
ಬೋಧ-ನೆ-ಗಳ
ಬೋಧ-ನೆ-ಗ-ಳ-ನ್ನ-ಲ್ಲದೆ
ಬೋಧ-ನೆ-ಗಳನ್ನು
ಬೋಧ-ನೆ-ಗಳನ್ನೂ
ಬೋಧ-ನೆ-ಗಳನ್ನೆಲ್ಲ
ಬೋಧ-ನೆ-ಗಳಲ್ಲಿ
ಬೋಧ-ನೆ-ಗಳಿಂದ
ಬೋಧ-ನೆ-ಗ-ಳಿ-ಗಿಂತ
ಬೋಧ-ನೆ-ಗ-ಳಿಗೆ
ಬೋಧ-ನೆ-ಗಳು
ಬೋಧ-ನೆ-ಗಳೂ
ಬೋಧ-ನೆ-ಗ-ಳೆಂ-ಥವು
ಬೋಧ-ನೆ-ಗ-ಳೆಲ್ಲ
ಬೋಧ-ನೆ-ಗ-ಳೆ-ಲ್ಲವೂ
ಬೋಧ-ನೆ-ಗಳೇ
ಬೋಧ-ನೆಗೆ
ಬೋಧ-ನೆಯ
ಬೋಧ-ನೆ-ಯನ್ನು
ಬೋಧ-ನೆ-ಯಾ-ಗಲಿ
ಬೋಧ-ನೆಯು
ಬೋಧ-ನೆಯೂ
ಬೋಧ-ನೆಯೇ
ಬೋಧಿ
ಬೋಧಿ-ವೃ-ಕ್ಷಕ್ಕೆ
ಬೋಧಿ-ವೃ-ಕ್ಷದ
ಬೋಧಿ-ಸ-ತ್ವ-ನಂತೆ
ಬೋಧಿ-ಸ-ಬೇ-ಕಾ-ದ-ದ್ದಿದೆ
ಬೋಧಿ-ಸ-ಬೇಕು
ಬೋಧಿ-ಸ-ಬೇ-ಕೆಂದು
ಬೋಧಿ-ಸ-ಲಿಲ್ಲ
ಬೋಧಿ-ಸಲು
ಬೋಧಿಸಿ
ಬೋಧಿ-ಸಿದ
ಬೋಧಿ-ಸಿ-ದರು
ಬೋಧಿ-ಸಿದ್ದ
ಬೋಧಿ-ಸಿ-ದ್ದರೆ
ಬೋಧಿ-ಸಿ-ರುವ
ಬೋಧಿಸು
ಬೋಧಿ-ಸುತ್ತ
ಬೋಧಿ-ಸು-ತ್ತವೆ
ಬೋಧಿ-ಸು-ತ್ತಾನೆ
ಬೋಧಿ-ಸು-ತ್ತಾ-ರಾದ್ದ
ಬೋಧಿ-ಸು-ತ್ತಾರೆ
ಬೋಧಿ-ಸು-ತ್ತಿದ್ದ
ಬೋಧಿ-ಸು-ತ್ತಿ-ದ್ದರು
ಬೋಧಿ-ಸು-ತ್ತಿ-ದ್ದ-ರೆಂ-ಬು-ದನ್ನು
ಬೋಧಿ-ಸು-ತ್ತಿ-ದ್ದಾರೆ
ಬೋಧಿ-ಸು-ತ್ತಿ-ರುವ
ಬೋಧಿ-ಸು-ತ್ತೀ-ದ್ದೀ-ರಲ್ಲ
ಬೋಧಿ-ಸು-ತ್ತೇನೆ
ಬೋಧಿ-ಸುವ
ಬೋಧಿ-ಸು-ವಂತೆ
ಬೋಧಿ-ಸು-ವಲ್ಲಿ
ಬೋಧಿ-ಸು-ವಾಗ
ಬೋಧಿ-ಸು-ವಾ-ಗಲೂ
ಬೋಧಿ-ಸು-ವು-ದ-ಕ್ಕಲ್ಲ
ಬೋಧಿ-ಸು-ವು-ದ-ಕ್ಕಾಗಿ
ಬೋಧಿ-ಸು-ವು-ದ-ಕ್ಕಾ-ಗಿಆ
ಬೋಧಿ-ಸು-ವು-ದನ್ನೂ
ಬೋಧಿ-ಸು-ವು-ದ-ರೊಂ-ದಿಗೆ
ಬೋಧಿ-ಸು-ವು-ದಾ-ಗಿತ್ತು
ಬೋಧಿ-ಸು-ವು-ದಿಲ್ಲ
ಬೋಧಿ-ಸು-ವುದು
ಬೋರಾಲು
ಬೋಳು
ಬೌದ್ಧ
ಬೌದ್ಧ-ಕಾಲ
ಬೌದ್ಧ-ಕಾ-ಲದ
ಬೌದ್ಧ-ಧರ್ಮ
ಬೌದ್ಧ-ಧ-ರ್ಮದ
ಬೌದ್ಧ-ಧ-ರ್ಮ-ಭಾ-ರ-ತೀ-ಯರ
ಬೌದ್ಧ-ಧ-ರ್ಮ-ವ-ನ್ನೀಗ
ಬೌದ್ಧ-ಧ-ರ್ಮ-ವನ್ನು
ಬೌದ್ಧ-ಧ-ರ್ಮ-ವಿ-ಲ್ಲದೆ
ಬೌದ್ಧ-ಧ-ರ್ಮವು
ಬೌದ್ಧ-ಧ-ರ್ಮವೂ
ಬೌದ್ಧ-ನಲ್ಲ
ಬೌದ್ಧ-ನೆಂದು
ಬೌದ್ಧನೋ
ಬೌದ್ಧರ
ಬೌದ್ಧ-ರನ್ನು
ಬೌದ್ಧ-ರಿಗೂ
ಬೌದ್ಧರು
ಬೌದ್ಧ-ವಾ-ಗಲಿ
ಬೌದ್ಧ-ಸಂ-ನ್ಯಾ-ಸಿ-ಗಳು
ಬೌದ್ಧಿಕ
ಬೌದ್ಧಿ-ಕ-ತೆಯ
ಬೌದ್ಧಿ-ಕ-ತೆಯೂ
ಬೌದ್ಧಿ-ಕ-ವಾಗಿ
ಬೌದ್ಧಿ-ಕ-ವಾ-ಗಿಯೂ
ಬ್ಬರು
ಬ್ಬರೂ
ಬ್ಯಾಂಕ-ರು-ಗಳು
ಬ್ಯಾಂಕಿ-ನಲ್ಲಿ
ಬ್ಯಾಂಕ್
ಬ್ಯಾಗ್ಲಿ
ಬ್ಯಾಗ್ಲೀ
ಬ್ಯಾಗ್ಲೀಯ
ಬ್ಯಾಗ್ಲೀ-ಯ-ವರ
ಬ್ಯಾಗ್ಲೀ-ಯ-ವ-ರಂ-ತಹ
ಬ್ಯಾಗ್ಲೀ-ಯ-ವ-ರಿಗೆ
ಬ್ಯಾನರ್ಜಿ
ಬ್ಯಾನ-ರ್ಜಿ-ಯ-ವರು
ಬ್ಯಾರಿ-ಸ್ಟರ್
ಬ್ಯೂರೋ
ಬ್ರದರ್
ಬ್ರಹ್ಮ
ಬ್ರಹ್ಮಂ
ಬ್ರಹ್ಮ-ಚರ್ಯ
ಬ್ರಹ್ಮ-ಚ-ರ್ಯ-ಪ-ರಿ-ಶು-ದ್ಧತೆ
ಬ್ರಹ್ಮ-ಚ-ರ್ಯಕ್ಕೂ
ಬ್ರಹ್ಮ-ಚ-ರ್ಯ-ದೀಕ್ಷೆ
ಬ್ರಹ್ಮ-ಚ-ರ್ಯ-ಪಾ-ಲನೆ
ಬ್ರಹ್ಮ-ಚ-ರ್ಯ-ವನ್ನು
ಬ್ರಹ್ಮ-ಚಾರಿ
ಬ್ರಹ್ಮ-ಚಾ-ರಿ-ಗ-ಳಿಗೆ
ಬ್ರಹ್ಮ-ಚಾ-ರಿಗೆ
ಬ್ರಹ್ಮ-ಚಾ-ರಿ-ಣಿ-ಯ-ರ-ನ್ನಾಗಿ
ಬ್ರಹ್ಮ-ಚಾ-ರಿ-ಣಿ-ಯ-ರನ್ನು
ಬ್ರಹ್ಮ-ಚಾ-ರಿ-ಣಿ-ಯಾ-ದಳು
ಬ್ರಹ್ಮ-ಚಾ-ರಿಯ
ಬ್ರಹ್ಮ-ಚಾ-ರಿ-ಯನ್ನು
ಬ್ರಹ್ಮ-ಜ್ಞಾ-ನಿ-ಗಳಿಂದ
ಬ್ರಹ್ಮದ
ಬ್ರಹ್ಮ-ದಲ್ಲಿ
ಬ್ರಹ್ಮನ
ಬ್ರಹ್ಮ-ರ್ಷಿ-ರಾ-ಜ-ರ್ಷಿ-ಗಳ
ಬ್ರಹ್ಮ-ವನ್ನೇ
ಬ್ರಹ್ಮ-ವ-ಲ್ಲದೆ
ಬ್ರಹ್ಮ-ವಾ-ಗಿ-ದ್ದೇನೆ
ಬ್ರಹ್ಮ-ವಾ-ದಿನ್
ಬ್ರಹ್ಮ-ವಾ-ದಿನ್ಗೆ
ಬ್ರಹ್ಮ-ವಾ-ದಿ-ನ್ಪ-ತ್ರಿ-ಕೆಗೆ
ಬ್ರಹ್ಮ-ವೆಂದು
ಬ್ರಹ್ಮ-ವೆಂದೇ
ಬ್ರಹ್ಮವೇ
ಬ್ರಹ್ಮವೊ
ಬ್ರಹ್ಮ-ಸ-ಕ-ಲವೂ
ಬ್ರಹ್ಮ-ಸೂತ್ರ
ಬ್ರಹ್ಮಾ
ಬ್ರಹ್ಮಾಂ-ಡ-ವಾಗಿ
ಬ್ರಹ್ಮಾ-ನಂ-ದರ
ಬ್ರಹ್ಮಾ-ನಂ-ದ-ರಿಗೆ
ಬ್ರಹ್ಮಾ-ನಂ-ದರು
ಬ್ರಹ್ಮಾ-ನಂ-ದರೂ
ಬ್ರಹ್ಮಾ-ನು-ಭವಿ
ಬ್ರಹ್ಮಾಸ್ತ್ರ
ಬ್ರಹ್ಮಾಸ್ಮಿ
ಬ್ರಾಹ್ಮಣ
ಬ್ರಾಹ್ಮ-ಣ-ಕು-ಟುಂ-ಬ-ದಲ್ಲಿ
ಬ್ರಾಹ್ಮ-ಣನ
ಬ್ರಾಹ್ಮ-ಣ-ನಾ-ದ-ವನು
ಬ್ರಾಹ್ಮ-ಣ-ಬಾ-ಲ-ಕನ
ಬ್ರಾಹ್ಮ-ಣರ
ಬ್ರಾಹ್ಮ-ಣ-ರ-ಪಾ-ಲಿಗೆ
ಬ್ರಾಹ್ಮ-ಣ-ರಲ್ಲ
ಬ್ರಾಹ್ಮ-ಣ-ರಾ-ದದ್ದು
ಬ್ರಾಹ್ಮ-ಣ-ರಿಂ-ದಲೇ
ಬ್ರಾಹ್ಮ-ಣರು
ಬ್ರಾಹ್ಮ-ಣ-ಶಿ-ಷ್ಯರು
ಬ್ರಾಹ್ಮ-ಣ-ಸಂ-ನ್ಯಾ-ಸಿಗೆ
ಬ್ರಾಹ್ಮ-ಣ-ಸಂ-ನ್ಯಾ-ಸಿ-ಯ-ಲ್ಲ-ವೆಂಬ
ಬ್ರಾಹ್ಮ-ಣೇ-ತ-ರ-ನಾಗಿ
ಬ್ರಾಹ್ಮ-ಣ್ಯ-ದಲ್ಲಿ
ಬ್ರಾಹ್ಮ-ಸ-ಮಾ-ಜ-ಗ-ಳ-ಲ್ಲೊಂ-ದ-ರಲ್ಲಿ
ಬ್ರಾಹ್ಮ-ಸ-ಮಾ-ಜದ
ಬ್ರಾಹ್ಮ-ಸ-ಮಾ-ಜ-ದ-ವರ
ಬ್ರಾಹ್ಮ-ಸ-ಮಾ-ಜ-ದ-ವರು
ಬ್ರಾಹ್ಮ-ಸ-ಮಾ-ಜ-ವನ್ನು
ಬ್ರಾಹ್ಮ-ಸ-ಮಾಜೀ
ಬ್ರಾಹ್ಮ-ಸ-ಮಾ-ಜೀ-ಯರೂ
ಬ್ರಾಹ್ಮೀ
ಬ್ರಿಟನ್
ಬ್ರಿಟಿಷ
ಬ್ರಿಟಿ-ಷರ
ಬ್ರಿಟಿ-ಷ-ರನ್ನು
ಬ್ರಿಟಿ-ಷರು
ಬ್ರಿಟಿಷ್
ಬ್ರೀಡ್
ಬ್ರೀಸಿ
ಬ್ರೂಕ್ಲಿನ್
ಬ್ರೂಕ್ಲಿನ್ಗೆ
ಬ್ರೂಕ್ಲಿ-ನ್ನಲ್ಲಿ
ಬ್ರೂಕ್ಲಿ-ನ್ನಿಗೆ
ಬ್ರೂಕ್ಲಿ-ನ್ನಿ-ನಲ್ಲಿ
ಬ್ರೆಡ್ಡನ್ನು
ಬ್ರೌನ್
ಬ್ಲಾಂಕ್
ಬ್ಲಾಜೆಟ್
ಬ್ಲೂಕ್ಲಿ-ನ್ನಿ-ನಲ್ಲಿ
ಭಂಗ
ಭಂಗ-ಮಾ-ಡ-ಲಿ-ದ್ದೇವೆ
ಭಂಗಿ
ಭಂಗಿ-ಗಳ
ಭಂಡ-ಧೈರ್ಯ
ಭಂಡರ
ಭಂಡರು
ಭಂಡಾ-ರ-ವನ್ನು
ಭಕ್ತ
ಭಕ್ತ-ಶಿ-ಷ್ಯ-ವ-ರ್ಗ-ದ-ವರೂ
ಭಕ್ತ-ಶಿ-ಷ್ಯ-ವೃಂದ
ಭಕ್ತ-ಜನ
ಭಕ್ತ-ಜ-ನರು
ಭಕ್ತ-ನಾದ
ಭಕ್ತನೂ
ಭಕ್ತರ
ಭಕ್ತ-ರನ್ನು
ಭಕ್ತ-ರಲ್ಲಿ
ಭಕ್ತ-ರಾದ
ಭಕ್ತ-ರಾ-ದರು
ಭಕ್ತ-ರಿಂದ
ಭಕ್ತ-ರಿಗೂ
ಭಕ್ತ-ರಿಗೆ
ಭಕ್ತರು
ಭಕ್ತ-ರು-ವಿ-ಶ್ವಾ-ಸಿ-ಗ-ರೆಲ್ಲ
ಭಕ್ತ-ರು-ವಿ-ಶ್ವಾ-ಸಿ-ಗ-ಳೆ-ಲ್ಲ-ರನ್ನೂ
ಭಕ್ತ-ರು-ಇ-ವೆಲ್ಲ
ಭಕ್ತರೂ
ಭಕ್ತ-ರೆಲ್ಲ
ಭಕ್ತರೇ
ಭಕ್ತ-ರೊಂ-ದಿಗೆ
ಭಕ್ತ-ಳಾ-ದಳು
ಭಕ್ತಾ-ಗ್ರ-ಣಿ-ಗಳೂ
ಭಕ್ತಾ-ದಿ-ಗಳ
ಭಕ್ತಿ
ಭಕ್ತಿ-ಕೃ-ತ-ಜ್ಞ-ತೆ-ಗಳ
ಭಕ್ತಿ-ಗೌ-ರವ
ಭಕ್ತಿ-ಗೌ-ರ-ವ-ಪ್ರೀ-ತಿ-ಯನ್ನು
ಭಕ್ತಿ-ಪ್ರೇ-ಮ-ಗ-ಳ-ನ್ನಿ-ರಿ-ಸಿ-ದ್ದಂ-ತೆಯೇ
ಭಕ್ತಿ-ವಿ-ಶ್ವಾಸ
ಭಕ್ತಿ-ಗೌ-ರ-ವ-ಗಳಿಂದ
ಭಕ್ತಿ-ಗೌ-ರ-ವ-ಗಳು
ಭಕ್ತಿ-ಪ್ರ-ದ-ವಾ-ಗಿಯೂ
ಭಕ್ತಿ-ಭಾ-ವ-ದಲ್ಲಿ
ಭಕ್ತಿ-ಭಾ-ವ-ದಿಂದ
ಭಕ್ತಿ-ಭಾ-ವೋ-ನ್ಮ-ತ್ತ-ರಾ-ಗಿ-ದ್ದು-ದನ್ನು
ಭಕ್ತಿ-ಮಾ-ರ್ಗದ
ಭಕ್ತಿ-ಯಿಂದ
ಭಕ್ತಿಯು
ಭಕ್ತಿಯೂ
ಭಕ್ತಿಯೇ
ಭಕ್ತಿ-ಯೋಗ
ಭಕ್ತಿ-ಯೋ-ಗದ
ಭಕ್ತಿ-ಯೋ-ಗವು
ಭಕ್ತಿ-ಸೂ-ತ್ರ-ಗಳ
ಭಕ್ತಿ-ಸೂ-ತ್ರ-ಗಳನ್ನು
ಭಕ್ತಿ-ಸೂ-ತ್ರವೇ
ಭಕ್ತೆ-ಯರು
ಭಕ್ತೆಯೇ
ಭಗವ
ಭಗ-ವಂತ
ಭಗ-ವಂ-ತ-ಪ್ರೇ-ಮ-ಸ-ತ್ಯ-ಗಳನ್ನು
ಭಗ-ವಂ-ತನ
ಭಗ-ವಂ-ತ-ನ-ಆ-ಶ್ವಾ-ಸನೆ
ಭಗ-ವಂ-ತ-ನದು
ಭಗ-ವಂ-ತ-ನ-ನ್ನಾಗಿ
ಭಗ-ವಂ-ತ-ನನ್ನು
ಭಗ-ವಂ-ತ-ನನ್ನೂ
ಭಗ-ವಂ-ತ-ನನ್ನೇ
ಭಗ-ವಂ-ತ-ನ-ಲ್ಲದೆ
ಭಗ-ವಂ-ತ-ನಲ್ಲಿ
ಭಗ-ವಂ-ತ-ನ-ಲ್ಲಿನ
ಭಗ-ವಂ-ತ-ನಿಂದ
ಭಗ-ವಂ-ತ-ನಿಂ-ದಲೇ
ಭಗ-ವಂ-ತ-ನಿಗೂ
ಭಗ-ವಂ-ತ-ನಿಗೆ
ಭಗ-ವಂ-ತನು
ಭಗ-ವಂ-ತನೇ
ಭಗ-ವಂ-ತ-ನೊಂ-ದಿಗೆ
ಭಗ-ವಂ-ತ-ನೊ-ಡನೆ
ಭಗ-ವಂ-ತ-ನೊಬ್ಬ
ಭಗ-ವಂ-ತ-ನೊ-ಬ್ಬನೇ
ಭಗ-ವಂತಾ
ಭಗ-ವ-ಚ್ಛ-ಕ್ತಿಗೆ
ಭಗ-ವತ್
ಭಗ-ವ-ತ್ಕಾ-ರ್ಯ-ಕ್ಕಾಗಿ
ಭಗ-ವ-ತ್ಕೃ-ಪೆ-ಯಿಂದ
ಭಗ-ವ-ತ್ಪ್ರೇಮ
ಭಗ-ವ-ತ್ಸಾ-ಕ್ಷಾ-ತ್ಕಾ-ರ-ವನ್ನು
ಭಗ-ವ-ದಾ-ಕಾಂ-ಕ್ಷಿ-ಗಳೋ
ಭಗ-ವ-ದಾ-ನಂ-ದ-ಸುಧೆ
ಭಗ-ವ-ದಿ-ಚ್ಛೆಗೆ
ಭಗ-ವ-ದಿ-ಚ್ಛೆಯ
ಭಗ-ವ-ದಿ-ಚ್ಛೆ-ಯಂತೆ
ಭಗ-ವ-ದಿ-ಚ್ಛೆಯೇ
ಭಗ-ವದ್
ಭಗ-ವ-ದ್ಗೀತೆ
ಭಗ-ವ-ದ್ಗೀ-ತೆಯ
ಭಗ-ವ-ದ್ಗೀ-ತೆ-ಯನ್ನೂ
ಭಗ-ವ-ದ್ಗೀ-ತೆಯು
ಭಗ-ವ-ದ್ಜ್ಞಾ-ನ-ಕ್ಕಿಂತ
ಭಗ-ವ-ದ್ದ-ರ್ಶನ
ಭಗ-ವ-ನ್ನಾ-ಮ-ವ-ನ್ನು-ಚ್ಚ-ರಿ-ಸ-ಲಾ-ರಂ-ಭಿ-ಸಿ-ದರು
ಭಗ-ವಾನ್
ಭಗೀ-ರಥ
ಭಜನೆ
ಭಜ-ನೆ-ಸಂ-ಭಾ-ಷ-ಣೆ-ಗಳು
ಭಜ-ನೆ-ಗಳನ್ನು
ಭಜ-ನೆ-ಗಳನ್ನೂ
ಭಜ-ನೆಯ
ಭಜಾ-ಮ್ಯಹಂ
ಭಟಿ
ಭಟ್ಟನ್
ಭಟ್ಟಾ-ಚಾರ್ಯ
ಭಟ್ಟಾ-ಚಾ-ರ್ಯರ
ಭಟ್ಟಾ-ಚಾ-ರ್ಯ-ರನ್ನು
ಭಟ್ಟಾ-ಚಾ-ರ್ಯರು
ಭದ್ರ
ಭದ್ರ-ವಾ-ಗಿ-ದೆ-ಯ-ಲ್ಲವೆ
ಭದ್ರ-ವಾ-ಗಿ-ರಲೇ
ಭದ್ರ-ವಾದ
ಭಯ
ಭಯ-ಭ-ಕ್ತಿ-ಗೌ-ರ-ವದ
ಭಯಂ-ಕರ
ಭಯಂ-ಕ-ರ-ವಾಗಿ
ಭಯಂ-ಕ-ರ-ವಾ-ಗಿಯೇ
ಭಯಂ-ಕ-ರ-ವಾದ
ಭಯ-ದಿಂದ
ಭಯ-ದಿಂ-ದಾ-ಗಲಿ
ಭಯ-ಪ-ಡು-ವುದನ್ನು
ಭಯ-ಭ-ಕ್ತಿ-ಮಿ-ಶ್ರಿ-ತ-ವಾದ
ಭಯ-ವನ್ನು
ಭಯ-ವಾ-ಯಿತು
ಭಯ-ವಿತ್ತು
ಭಯ-ವಿ-ದ್ದುದ
ಭಯ-ವಿ-ರ-ಲಿಲ್ಲ
ಭಯ-ವಿಲ್ಲ
ಭಯವೂ
ಭಯವೆ
ಭಯವೇ
ಭಯ-ಹು-ಟ್ಟಿ-ಕೊಂ-ಡಿತು
ಭಯಾ-ನಕ
ಭಯಾ-ಶ್ಚ-ರ್ಯ-ಗಳಿಂದ
ಭಯಾ-ಶ್ಚ-ರ್ಯ-ದಿಂದ
ಭಯೇ
ಭರ
ಭರತ
ಭರ-ತ-ಖಂಡ
ಭರ-ತ-ಖಂ-ಡದ
ಭರ-ತ-ಖಂ-ಡ-ದಲ್ಲಿ
ಭರ-ತ-ಭೂ-ಮಿ-ಯನ್ನೇ
ಭರ-ದಿಂದ
ಭರ-ವಸೆ
ಭರ-ವ-ಸೆ-ವಿ-ಶ್ವಾ-ಸ-ಗಳು
ಭರ-ವ-ಸೆ-ಸ-ಮಾ-ಧಾ-ನ-ಗಳ
ಭರ-ವ-ಸೆ-ಸೌಂ-ದ-ರ್ಯ-ಗಳ
ಭರ-ವ-ಸೆ-ಕೊ-ಡುತ್ತ
ಭರ-ವ-ಸೆ-ಗ-ಳ-ಲ್ಲೊಬ್ಬ
ಭರ-ವ-ಸೆ-ಗಳು
ಭರ-ವ-ಸೆಯ
ಭರ-ವ-ಸೆ-ಯ-ನ್ನಿತ್ತು
ಭರ-ವ-ಸೆ-ಯನ್ನು
ಭರ-ವ-ಸೆ-ಯ-ನ್ನುಂ-ಟು-ಮಾಡಿ
ಭರ-ವ-ಸೆ-ಯನ್ನೂ
ಭರ-ವ-ಸೆ-ಯಿಂದ
ಭರ-ವ-ಸೆ-ಯಿಂ-ದ-ಸ-ಮ್ಮೇ-ಳ-ನಕ್ಕೆ
ಭರ-ವ-ಸೆ-ಯಿ-ಟ್ಟು-ಕೊಂ-ಡಿ-ದ್ದರು
ಭರ-ವ-ಸೆ-ಯಿತ್ತು
ಭರ-ವ-ಸೆ-ಯಿ-ದ್ದು-ದ-ರಿಂದ
ಭರ-ವ-ಸೆ-ಯಿ-ರು-ವುದು
ಭರ-ವ-ಸೆಯು
ಭರ-ವ-ಸೆಯೂ
ಭರ-ವ-ಸೆಯೇ
ಭರಾ-ಟೆ-ಯಲ್ಲಿ
ಭರಿತ
ಭರಿ-ತ-ರಾಗಿ
ಭರಿ-ತ-ರಾ-ಗಿ-ಬಿ-ಡು-ತ್ತಿ-ದ್ದರು
ಭರಿ-ತ-ರಾ-ದರು
ಭರಿ-ತ-ವಾಗಿ
ಭರಿ-ತ-ವಾದ
ಭರಿ-ಸು-ವು-ದ-ಲ್ಲದೆ
ಭರೋ
ಭರ್ಜರಿ
ಭರ್ಜ-ರಿ-ಯಾ-ಗಿ-ರು-ವಂತೆ
ಭರ್ತಿ
ಭರ್ತಿ-ಯಾ-ಗಿತ್ತು
ಭರ್ತಿ-ಯಿ-ರದೆ
ಭರ್ತೃ-ಹರಿ
ಭಲೆ
ಭಲೇ
ಭವ
ಭವ-ಗ-ಳೆಲ್ಲ
ಭವ-ತಾ-ರಿಣಿ
ಭವ-ನ-ಗ-ಳಿಗೆ
ಭವ-ನದ
ಭವ-ನ-ದಲ್ಲಿ
ಭವ-ನ-ದಲ್ಲೂ
ಭವ-ನ-ವೊಂ-ದನ್ನು
ಭವ-ಭೂ-ತಿ-ಯರ
ಭವವೂ
ಭವಿ-ತ-ವ್ಯ-ದಲ್ಲಿ
ಭವಿ-ಷ್ಯ-ಗಳು
ಭವಿ-ಷ್ಯ-ತ್ತಿನ
ಭವಿ-ಷ್ಯದ
ಭವಿ-ಷ್ಯ-ದ-ರ್ಶನ
ಭವಿ-ಷ್ಯ-ದಾ-ಳ-ವನ್ನು
ಭವಿ-ಷ್ಯ-ಭಾ-ರ-ತದ
ಭವಿ-ಷ್ಯ-ವ-ಡ-ಗಿ-ರು-ವುದು
ಭವಿ-ಷ್ಯ-ವಾಣಿ
ಭವಿ-ಷ್ಯ-ವಾ-ಣಿ-ಯಲ್ಲಿ
ಭವಿ-ಷ್ಯ-ವಾ-ಣಿ-ಯಾ-ಗಿತ್ತು
ಭವಿ-ಷ್ಯ-ವಿದೆ
ಭವಿ-ಷ್ಯವು
ಭವ್ಯ
ಭವ್ಯ-ತೆ-ಯನ್ನು
ಭವ್ಯ-ಪಂ-ಕ್ತಿಗೆ
ಭವ್ಯ-ಭಾ-ರತ
ಭವ್ಯ-ವಾಗಿ
ಭವ್ಯ-ವಾದ
ಭವ್ಯ-ವ್ಯ-ಕ್ತಿ-ತ್ವ-ದಿಂದ
ಭವ್ಯ-ಸ್ಮಾ-ರಕ
ಭಾಗ
ಭಾಗ-ಗಳ
ಭಾಗ-ಗಳನ್ನು
ಭಾಗ-ಗಳನ್ನೂ
ಭಾಗ-ಗಳಲ್ಲಿ
ಭಾಗ-ಗ-ಳಿಗೂ
ಭಾಗದ
ಭಾಗ-ದಲ್ಲಿ
ಭಾಗ-ದ-ವರು
ಭಾಗ-ವತ
ಭಾಗ-ವನ್ನು
ಭಾಗ-ವ-ನ್ನೆಲ್ಲ
ಭಾಗ-ವಹಿ
ಭಾಗ-ವ-ಹಿಸ
ಭಾಗ-ವ-ಹಿ-ಸ-ದಿ-ದ್ದ-ವರೂ
ಭಾಗ-ವ-ಹಿ-ಸ-ಬಹು
ಭಾಗ-ವ-ಹಿ-ಸ-ಬೇ-ಕೆಂದು
ಭಾಗ-ವ-ಹಿ-ಸ-ಬೇ-ಕೆಂಬ
ಭಾಗ-ವ-ಹಿ-ಸಲು
ಭಾಗ-ವ-ಹಿ-ಸ-ಲೇ-ಬೇಕು
ಭಾಗ-ವ-ಹಿ-ಸ-ಲೇ-ಬೇ-ಕೆಂದು
ಭಾಗ-ವ-ಹಿಸಿ
ಭಾಗ-ವ-ಹಿ-ಸಿದ
ಭಾಗ-ವ-ಹಿ-ಸಿ-ದರು
ಭಾಗ-ವ-ಹಿ-ಸಿ-ದ್ದ-ನೆಂ-ಬುದು
ಭಾಗ-ವ-ಹಿ-ಸಿ-ದ್ದರು
ಭಾಗ-ವ-ಹಿ-ಸಿ-ದ್ದೇನೆ
ಭಾಗ-ವ-ಹಿ-ಸು-ತ್ತಿ-ದ್ದ-ರಾ-ದರೂ
ಭಾಗ-ವ-ಹಿ-ಸು-ತ್ತಿ-ರುವ
ಭಾಗ-ವ-ಹಿ-ಸುವ
ಭಾಗ-ವ-ಹಿ-ಸು-ವಂತೆ
ಭಾಗ-ವ-ಹಿ-ಸು-ವು-ದ-ಕ್ಕಾಗಿ
ಭಾಗ-ವ-ಹಿ-ಸು-ವು-ದ-ಕ್ಕಾ-ಗಿಯೂ
ಭಾಗ-ವಾದ
ಭಾಗ-ವಾ-ದರೂ
ಭಾಗ-ವೊಂ-ದರ
ಭಾಗಿ-ಗ-ಳಾಗಿ
ಭಾಗಿ-ಗ-ಳಾ-ದರು
ಭಾಗಿ-ಗಳು
ಭಾಗ್ಯ
ಭಾಗ್ಯದ
ಭಾಗ್ಯ-ದಿಂದ
ಭಾಗ್ಯ-ದೇ-ವ-ತೆಯು
ಭಾಗ್ಯ-ವಂ-ತರ
ಭಾಗ್ಯ-ವಂತೆ
ಭಾಗ್ಯ-ವನ್ನು
ಭಾಗ್ಯ-ವ-ನ್ನೆಲ್ಲ
ಭಾಗ್ಯ-ವಿ-ಶೇ-ಷ-ವೆಂದು
ಭಾಗ್ಯ-ವೆಂದೇ
ಭಾಗ್ಯವೇ
ಭಾಗ್ಯ-ಶಾಲಿ
ಭಾಗ್ಯ-ಶಾ-ಲಿ-ಗಳ
ಭಾಗ್ಯ-ಶಾ-ಲಿ-ಗ-ಳೆಂದು
ಭಾಗ್ಯ-ಶಾ-ಲಿ-ಯಾ-ಗಿ-ರ-ಬೇಕು
ಭಾಗ್ಯ-ಶಾಲೀ
ಭಾಟೆ
ಭಾಟೆ-ಯ-ವರ
ಭಾಟೆ-ಯ-ವ-ರನ್ನು
ಭಾಟೆ-ಯ-ವರು
ಭಾನು
ಭಾನು-ವಾರ
ಭಾನು-ವಾ-ರ-ಗ-ಳಂದು
ಭಾನು-ವಾ-ರದ
ಭಾನು-ವಾ-ರ-ದಂದು
ಭಾಯಿಯೇ
ಭಾರ
ಭಾರತ
ಭಾರ-ತ-ಇಂ-ಗ್ಲೆಂ-ಡ್-ಅ-ಮೆ-ರಿ-ಕ-ಗ-ಳಿ-ಗಷ್ಟೇ
ಭಾರ-ತ-ಕ್ಕಾಗಿ
ಭಾರ-ತ-ಕ್ಕಿಂದು
ಭಾರ-ತ-ಕ್ಕೂ-ಆತ
ಭಾರ-ತಕ್ಕೆ
ಭಾರ-ತದ
ಭಾರ-ತ-ದ-ಭೂ-ತ-ಭ-ವಿ-ಷ್ಯ-ತ್-ವ-ರ್ತ-ಮಾ-ನ-ಗ-ಳೆಲ್ಲ
ಭಾರ-ತ-ದತ್ತ
ಭಾರ-ತ-ದ-ರ್ಶ-ನ-ವಾದ
ಭಾರ-ತ-ದಲ್ಲಿ
ಭಾರ-ತ-ದ-ಲ್ಲಿದ್ದ
ಭಾರ-ತ-ದ-ಲ್ಲಿ-ದ್ದಾಗ
ಭಾರ-ತ-ದ-ಲ್ಲಿನ
ಭಾರ-ತ-ದಲ್ಲೂ
ಭಾರ-ತ-ದಲ್ಲೆಲ್ಲ
ಭಾರ-ತ-ದಲ್ಲೇ
ಭಾರ-ತ-ದಾ-ದ್ಯಂತ
ಭಾರ-ತ-ದಿಂದ
ಭಾರ-ತ-ದಿಂ-ದಲೂ
ಭಾರ-ತ-ದೆ-ಡೆಗೆ
ಭಾರ-ತ-ದೊ-ಡನೆ
ಭಾರ-ತ-ಮಾ-ತೆಯ
ಭಾರ-ತ-ವನ್ನು
ಭಾರ-ತ-ವನ್ನೇ
ಭಾರ-ತ-ವ-ಲ್ಲ-ವೆಂ-ಬುದು
ಭಾರ-ತ-ವಾ-ದ್ದ-ರಿಂದ
ಭಾರ-ತ-ವಿ-ರು-ವುದು
ಭಾರ-ತವು
ಭಾರ-ತವೆ
ಭಾರ-ತ-ವೆಂದರೆ
ಭಾರ-ತವೇ
ಭಾರ-ತ-ವೊಂದು
ಭಾರ-ತಾಂ-ಬೆಯ
ಭಾರ-ತಾಂ-ಬೆ-ಯಾಗಿ
ಭಾರತೀ
ಭಾರ-ತೀಯ
ಭಾರ-ತೀ-ಯ-ತೆಯ
ಭಾರ-ತೀ-ಯ-ತೆ-ಯಲ್ಲಿ
ಭಾರ-ತೀ-ಯನ
ಭಾರ-ತೀ-ಯ-ನನ್ನು
ಭಾರ-ತೀ-ಯನು
ಭಾರ-ತೀ-ಯರ
ಭಾರ-ತೀ-ಯ-ರನ್ನು
ಭಾರ-ತೀ-ಯ-ರಲ್ಲಿ
ಭಾರ-ತೀ-ಯ-ರಾದ
ಭಾರ-ತೀ-ಯರಿ
ಭಾರ-ತೀ-ಯ-ರಿಂದ
ಭಾರ-ತೀ-ಯ-ರಿ-ಗಾಗಿ
ಭಾರ-ತೀ-ಯ-ರಿ-ಗಾದ
ಭಾರ-ತೀ-ಯ-ರಿಗೆ
ಭಾರ-ತೀ-ಯ-ರಿ-ಗೆ-ಅ-ದರ
ಭಾರ-ತೀ-ಯ-ರಿ-ಗೊಂದು
ಭಾರ-ತೀ-ಯರು
ಭಾರ-ತೀ-ಯ-ರೆಂ-ದರೆ
ಭಾರ-ತೀ-ಯ-ರೆಲ್ಲ
ಭಾರ-ತೀ-ಯ-ರೆ-ಲ್ಲರೂ
ಭಾರ-ತೀ-ಯರೇ
ಭಾರ-ತೀ-ಯ-ರೇ-ನಲ್ಲ
ಭಾರ-ತೀ-ಯ-ರೊಂ-ದಿಗೆ
ಭಾರ-ತೀ-ಯ-ವಾಗಿ
ಭಾರದ
ಭಾರ-ವನ್ನು
ಭಾರ-ವನ್ನೂ
ಭಾರ-ವಾಗಿ
ಭಾರ-ವಾ-ಗಿ-ದ್ದುವು
ಭಾರ-ವಾದ
ಭಾರೀ
ಭಾವ
ಭಾವ-ಭಾ-ರ-ತೀ-ಯರ
ಭಾವಕ್ಕೆ
ಭಾವ-ಗಳನ್ನು
ಭಾವ-ಗಳನ್ನೂ
ಭಾವ-ಗಳಲ್ಲಿ
ಭಾವ-ಗ-ಳ-ಲ್ಲಿ-ದ್ದಾಗ
ಭಾವ-ಗ-ಳ-ಲ್ಲಿ-ರು-ವಂತೆ
ಭಾವ-ಗ-ಳಿ-ಗೇ-ರುತ್ತಿ
ಭಾವ-ಗಳು
ಭಾವ-ಗ-ಳೆಲ್ಲ
ಭಾವ-ಚಿತ್ರ
ಭಾವ-ಚಿ-ತ್ರ-ಗಳನ್ನು
ಭಾವ-ಚಿ-ತ್ರ-ಗಳು
ಭಾವ-ಚಿ-ತ್ರದ
ಭಾವ-ಚಿ-ತ್ರ-ದತ್ತ
ಭಾವ-ಚಿ-ತ್ರ-ದಲ್ಲಿ
ಭಾವ-ಚಿ-ತ್ರ-ವನ್ನು
ಭಾವ-ಚಿ-ತ್ರ-ವನ್ನೂ
ಭಾವ-ತ-ರಂ-ಗ-ಗಳು
ಭಾವ-ತ-ರಂ-ಗ-ವ-ನ್ನೆ-ಬ್ಬಿ-ಸಿ-ದುವು
ಭಾವದ
ಭಾವ-ದ-ಲೆ-ಗಳ
ಭಾವ-ದಲ್ಲಿ
ಭಾವ-ದ-ಲ್ಲಿ-ದ್ದಾಗ
ಭಾವ-ದಲ್ಲೇ
ಭಾವ-ದಾ-ಳ-ದಿಂದ
ಭಾವ-ದಾ-ಳ-ವನ್ನು
ಭಾವ-ದಿಂದ
ಭಾವ-ದಿಂ-ದಿದ್ದ
ಭಾವ-ದೀ-ಪ್ತಿ-ಯನ್ನು
ಭಾವ-ನ-ಗರ
ಭಾವ-ನ-ಗ-ರಕ್ಕೆ
ಭಾವನಾ
ಭಾವ-ನಾ-ತ-ರಂ-ಗ-ಗಳು
ಭಾವ-ನಾ-ತ್ಮಕ
ಭಾವ-ನಾ-ತ್ಮ-ಕ-ವಾ-ಗಿಯೂ
ಭಾವನೆ
ಭಾವ-ನೆ-ಗಳ
ಭಾವ-ನೆ-ಗಳನ್ನು
ಭಾವ-ನೆ-ಗಳನ್ನೂ
ಭಾವ-ನೆ-ಗಳನ್ನೆಲ್ಲ
ಭಾವ-ನೆ-ಗ-ಳನ್ನೇ
ಭಾವ-ನೆ-ಗಳಲ್ಲಿ
ಭಾವ-ನೆ-ಗ-ಳಿಂ-ದಲೂ
ಭಾವ-ನೆ-ಗ-ಳಿ-ಗ-ನು-ಗುಣ
ಭಾವ-ನೆ-ಗ-ಳಿ-ಗ-ನು-ಸಾ-ರ-ವಾಗಿ
ಭಾವ-ನೆ-ಗ-ಳಿಗೆ
ಭಾವ-ನೆ-ಗಳು
ಭಾವ-ನೆ-ಗಳೂ
ಭಾವ-ನೆ-ಗ-ಳೆಲ್ಲ
ಭಾವ-ನೆ-ಗಳೇ
ಭಾವ-ನೆ-ಯ-ನ್ನಾ-ದರೂ
ಭಾವ-ನೆ-ಯನ್ನು
ಭಾವ-ನೆ-ಯ-ನ್ನುಂ-ಟು-ಮಾ-ಡು-ವುದು
ಭಾವ-ನೆ-ಯಿಂದ
ಭಾವ-ನೆ-ಯಿಂ-ದಲೇ
ಭಾವ-ನೆ-ಯಿಂ-ದ-ಲ್ಲವೆ
ಭಾವ-ನೆ-ಯಿತ್ತು
ಭಾವ-ನೆಯೂ
ಭಾವ-ನೆ-ಯೆಂ-ದರೆ
ಭಾವ-ನೆ-ಯೆಲ್ಲಿ
ಭಾವ-ನೆ-ಯೇ-ನೆಂದು
ಭಾವ-ಪ-ರ-ವಶ
ಭಾವ-ಪ-ರ-ವ-ಶ-ತೆ-ಯಲ್ಲಿ
ಭಾವ-ಪ-ರ-ವ-ಶ-ರ-ನ್ನಾಗಿ
ಭಾವ-ಪ-ರ-ವ-ಶ-ರಾ-ಗಿ-ದ್ದರು
ಭಾವ-ಪೂರ್ಣ
ಭಾವ-ಪೂ-ರ್ಣ-ವಾ-ಗಿ-ರು-ತ್ತಿತ್ತು
ಭಾವ-ಪ್ರ-ಕಾ-ಶವು
ಭಾವ-ಪ್ರ-ಪಂ-ಚ-ದ-ಲ್ಲೊಂದು
ಭಾವ-ಪ್ರ-ವಾಹ
ಭಾವ-ಭ-ರಿತ
ಭಾವ-ಭ-ರಿ-ತ-ರಾಗಿ
ಭಾವ-ಭಾ-ರ-ತ-ನಾಗಿ
ಭಾವ-ರಂ-ಜಿ-ತ-ರಾದ
ಭಾವ-ರೂ-ಪದ
ಭಾವ-ಲ-ಹರಿ
ಭಾವ-ಲ-ಹ-ರಿ-ಯನ್ನು
ಭಾವ-ವ-ನ್ನಿ-ಟ್ಟು-ಕೊಂ-ಡಿ-ರು-ವುದನ್ನು
ಭಾವ-ವನ್ನು
ಭಾವ-ವಿಷ್ಟೆ
ಭಾವವು
ಭಾವ-ವುಕ್ಕಿ
ಭಾವ-ವು-ಮ್ಮ-ಳಿ-ಸಿ-ಬಂದು
ಭಾವವೂ
ಭಾವವೇ
ಭಾವ-ಸ-ಮುದ್ರ
ಭಾವ-ಸ್ಥ-ರಾಗಿ
ಭಾವ-ಸ್ಥ-ರಾ-ಗುತ್ತ
ಭಾವಾ-ತಿ-ಶ-ಯ-ದಿಂದ
ಭಾವಾರ್ಥ
ಭಾವಾ-ವಸ್ಥೆ
ಭಾವಾ-ವ-ಸ್ಥೆಗೆ
ಭಾವಾ-ವ-ಸ್ಥೆ-ಗೇ-ರಿ-ದ-ರೆಂ-ದರೆ
ಭಾವಾ-ವ-ಸ್ಥೆ-ಯ-ಲ್ಲಿರು
ಭಾವಾ-ವ-ಸ್ಥೆಯು
ಭಾವಾ-ವೇ-ಶಕ್ಕೆ
ಭಾವಾ-ವೇ-ಶದ
ಭಾವಾ-ವೇ-ಶ-ಭ-ರಿ-ತ-ರಾಗಿ
ಭಾವಾ-ವೇ-ಶ-ವನ್ನು
ಭಾವಿ-ಸ-ತೊ-ಡ-ಗಿ-ದ್ದರು
ಭಾವಿ-ಸ-ತೊ-ಡ-ಗು-ತ್ತೇನೆ
ಭಾವಿ-ಸದೆ
ಭಾವಿ-ಸ-ಬಲ್ಲೆ
ಭಾವಿ-ಸ-ಬೇಡಿ
ಭಾವಿ-ಸ-ಲಾ-ರಂ-ಭಿ-ಸಿದ್ದ
ಭಾವಿ-ಸಲು
ಭಾವಿಸಿ
ಭಾವಿ-ಸಿ-ಕೊಂಡು
ಭಾವಿ-ಸಿ-ಕೊ-ಳ್ಳದೆ
ಭಾವಿ-ಸಿ-ಕೊ-ಳ್ಳ-ಬೇಕು
ಭಾವಿ-ಸಿತು
ಭಾವಿ-ಸಿದ
ಭಾವಿ-ಸಿ-ದ-ನೇನೋ
ಭಾವಿ-ಸಿ-ದ-ರಂತೆ
ಭಾವಿ-ಸಿ-ದರು
ಭಾವಿ-ಸಿ-ದರೂ
ಭಾವಿ-ಸಿ-ದರೆ
ಭಾವಿ-ಸಿ-ದಳು
ಭಾವಿ-ಸಿ-ದಾಗ
ಭಾವಿ-ಸಿ-ದೆಯಾ
ಭಾವಿ-ಸಿದ್ದ
ಭಾವಿ-ಸಿ-ದ್ದಂತೆ
ಭಾವಿ-ಸಿ-ದ್ದರು
ಭಾವಿ-ಸಿ-ದ್ದ-ರು-ಒಂದು
ಭಾವಿ-ಸಿ-ದ್ದರೆ
ಭಾವಿ-ಸಿ-ದ್ದಳು
ಭಾವಿ-ಸಿ-ದ್ದಷ್ಟು
ಭಾವಿ-ಸಿ-ದ್ದಾನೆ
ಭಾವಿ-ಸಿದ್ದೆ
ಭಾವಿ-ಸಿ-ದ್ದೆ-ನಲ್ಲ
ಭಾವಿ-ಸಿ-ನೋಡಿ
ಭಾವಿ-ಸಿಯೇ
ಭಾವಿ-ಸಿ-ರ-ಬೇಕು
ಭಾವಿ-ಸಿ-ರ-ಲಿಲ್ಲ
ಭಾವಿ-ಸಿ-ರುವ
ಭಾವಿ-ಸಿ-ರು-ವು-ದಾಗಿ
ಭಾವಿಸು
ಭಾವಿ-ಸು-ತ್ತಾರೆ
ಭಾವಿ-ಸು-ತ್ತಾರೋ
ಭಾವಿ-ಸು-ತ್ತಿ-ದ್ದಂತೆ
ಭಾವಿ-ಸು-ತ್ತಿ-ದ್ದರು
ಭಾವಿ-ಸು-ತ್ತಿ-ದ್ದಳು
ಭಾವಿ-ಸು-ತ್ತಿ-ರ-ಲಿಲ್ಲ
ಭಾವಿ-ಸುತ್ತೀ
ಭಾವಿ-ಸು-ತ್ತೀಯೋ
ಭಾವಿ-ಸು-ತ್ತೀರಿ
ಭಾವಿ-ಸು-ತ್ತೇನೆ
ಭಾವಿ-ಸು-ತ್ತೇವೆ
ಭಾವಿ-ಸು-ತ್ತೇ-ವೆಯೋ
ಭಾವಿ-ಸು-ವಂತೆ
ಭಾವಿ-ಸು-ವ-ವರು
ಭಾವಿ-ಸು-ವುದನ್ನು
ಭಾವಿ-ಸು-ವು-ದಾ-ದರೆ
ಭಾವಿ-ಸು-ವುದೂ
ಭಾವಿ-ಸು-ವುದೇ
ಭಾವು-ಕ-ತೆ-ಯಲ್ಲ
ಭಾವೋ-ದ್ದೀ-ಪ-ಕ-ವಾದ
ಭಾವೋ-ದ್ರೇಕ
ಭಾವೋ-ದ್ವೇಗ
ಭಾವೋ-ನ್ಮ-ತ್ತ-ರಾಗಿ
ಭಾವೋ-ನ್ಮ-ತ್ತ-ರಾ-ದಂತೆ
ಭಾಷಣ
ಭಾಷ-ಣ-ಕ-ರ್ತನ
ಭಾಷ-ಣ-ಕ-ರ್ತರೂ
ಭಾಷ-ಣ-ಕ-ಲೆಯ
ಭಾಷ-ಣ-ಕ-ಲೆ-ಯನ್ನು
ಭಾಷ-ಣ-ಕಾ-ರನ
ಭಾಷ-ಣ-ಕಾ-ರ-ನನ್ನು
ಭಾಷ-ಣ-ಕಾ-ರ-ನಾಗಿ
ಭಾಷ-ಣ-ಕಾ-ರ-ನಿಗೂ
ಭಾಷ-ಣ-ಕಾ-ರರ
ಭಾಷ-ಣ-ಕಾ-ರ-ರಲ್ಲಿ
ಭಾಷ-ಣ-ಕಾ-ರ-ರಾದ
ಭಾಷ-ಣ-ಕಾ-ರ-ರಿ-ಗಿಂತ
ಭಾಷ-ಣ-ಕಾ-ರ-ರಿಗೂ
ಭಾಷ-ಣ-ಕಾ-ರ-ರಿಗೆ
ಭಾಷ-ಣ-ಕಾ-ರರು
ಭಾಷ-ಣ-ಕೊಡು
ಭಾಷ-ಣ-ಕ್ಕಾಗಿ
ಭಾಷ-ಣಕ್ಕೂ
ಭಾಷ-ಣಕ್ಕೆ
ಭಾಷ-ಣ-ಗಳ
ಭಾಷ-ಣ-ಗಳನ್ನು
ಭಾಷ-ಣ-ಗಳನ್ನೂ
ಭಾಷ-ಣ-ಗಳನ್ನೆಲ್ಲ
ಭಾಷ-ಣ-ಗ-ಳ-ನ್ನೇನೂ
ಭಾಷ-ಣ-ಗ-ಳ-ನ್ನೊ-ಳ-ಗೊಂಡ
ಭಾಷ-ಣ-ಗ-ಳ-ಲ್ಲದೆ
ಭಾಷ-ಣ-ಗಳಲ್ಲಿ
ಭಾಷ-ಣ-ಗ-ಳ-ಲ್ಲೆಲ್ಲ
ಭಾಷ-ಣ-ಗಳಿಂದ
ಭಾಷ-ಣ-ಗ-ಳಿಂ-ದಾಗಿ
ಭಾಷ-ಣ-ಗ-ಳಿ-ಗಿಂ-ತಲೂ
ಭಾಷ-ಣ-ಗ-ಳಿಗೂ
ಭಾಷ-ಣ-ಗ-ಳಿಗೆ
ಭಾಷ-ಣ-ಗಳು
ಭಾಷ-ಣದ
ಭಾಷ-ಣ-ದಲ್ಲಿ
ಭಾಷ-ಣ-ದಿಂದ
ಭಾಷ-ಣ-ದಿಂ-ದಲೇ
ಭಾಷ-ಣ-ಪ್ರ-ವಾಸ
ಭಾಷ-ಣ-ಪ್ರ-ವಾ-ಸ-ದಲ್ಲಿ
ಭಾಷ-ಣ-ಮಾಡಿ
ಭಾಷ-ಣ-ವನ್ನು
ಭಾಷ-ಣ-ವನ್ನೂ
ಭಾಷ-ಣ-ವನ್ನೇ
ಭಾಷ-ಣ-ವಾ-ಗಿತ್ತು
ಭಾಷ-ಣ-ವಾ-ಗು-ತ್ತಲೇ
ಭಾಷ-ಣವು
ಭಾಷ-ಣವೇ
ಭಾಷ-ಣ-ವೊಂ-ದನ್ನು
ಭಾಷ-ಣ-ವೊಂ-ದ-ರಲ್ಲಿ
ಭಾಷಾ
ಭಾಷಾಂ-ತರ
ಭಾಷಾಂ-ತ-ರ-ಗಳಲ್ಲಿ
ಭಾಷಾಂ-ತ-ರಿಸಿ
ಭಾಷಾ-ಪಂ-ಡಿ-ತರು
ಭಾಷಾ-ಪ್ರ-ವಾ-ಹ-ದೊಂ-ದಿಗೆ
ಭಾಷಾ-ಮಾ-ಧ್ಯ-ಮ-ದಲ್ಲಿ
ಭಾಷಾ-ಶಾ-ಸ್ತ್ರ-ಜ್ಞ-ನ-ನ್ನಲ್ಲ
ಭಾಷೆ
ಭಾಷೆ-ಗಳ
ಭಾಷೆ-ಗಳಲ್ಲಿ
ಭಾಷೆ-ಗ-ಳಲ್ಲೂ
ಭಾಷೆಗೆ
ಭಾಷೆಯ
ಭಾಷೆ-ಯನ್ನು
ಭಾಷೆ-ಯನ್ನೂ
ಭಾಷೆ-ಯಲ್ಲಿ
ಭಾಷೆ-ಯ-ಲ್ಲಿ-ರುವ
ಭಾಷೆ-ಯಲ್ಲೇ
ಭಾಷೆ-ಯಾ-ಗಿತ್ತು
ಭಾಷೆ-ಯಾ-ಗಿ-ರ-ಲಿಲ್ಲ
ಭಾಷೆಯೂ
ಭಾಷೆಯೇ
ಭಾಷ್ಯ-ಕಾ-ರರ
ಭಾಷ್ಯ-ಗಳ
ಭಾಷ್ಯ-ಗ-ಳ-ನ್ನೊ-ಳ-ಗೊಂ-ಡಿದ್ದ
ಭಾಷ್ಯ-ಗ್ರಂಥ
ಭಾಷ್ಯದ
ಭಾಷ್ಯ-ವನ್ನು
ಭಾಷ್ಯ-ವಾ-ಗಿ-ದ್ದರು
ಭಾಷ್ಯವೇ
ಭಾಷ್ಯಾ-ರ್ಥ-ವನ್ನು
ಭಾಸ-ವಾ-ಗು-ತ್ತದೆ
ಭಾಸ-ವಾ-ಗು-ತ್ತಿತ್ತು
ಭಾಸ-ವಾ-ಗು-ತ್ತಿದೆ
ಭಾಸ-ವಾ-ಯಿತು
ಭಾಸ-ವಾ-ಯಿ-ತೆಂದು
ಭಾಸ್ಕರ
ಭಾಸ್ಟ-ನ್ನಿನ
ಭಿಕಾರಿ
ಭಿಕಾ-ರಿ-ಯಂತೆ
ಭಿಕ್ಷಾ
ಭಿಕ್ಷಾ-ನ್ನ-ವ-ನ್ನ-ವ-ಲಂ-ಬಿಸಿ
ಭಿಕ್ಷು-ಕನ
ಭಿಕ್ಷು-ಕ-ನಾ-ಗಿ-ರಲು
ಭಿಕ್ಷು-ಕ-ನಾ-ಗಿ-ರು-ವು-ದ-ರಿಂದ
ಭಿಕ್ಷು-ಕ-ನೆಂದೇ
ಭಿಕ್ಷು-ಕರ
ಭಿಕ್ಷು-ಕ-ರಿಗೆ
ಭಿಕ್ಷು-ಕ-ರಿ-ರು-ವುದು
ಭಿಕ್ಷು-ಕರು
ಭಿಕ್ಷು-ಕ-ರೆಂದು
ಭಿಕ್ಷು-ಗಳ
ಭಿಕ್ಷು-ಗ-ಳಿಗೆ
ಭಿಕ್ಷು-ಗಳು
ಭಿಕ್ಷೆ
ಭಿಕ್ಷೆ-ಗಾ-ಗಿಯೋ
ಭಿಕ್ಷೆಯ
ಭಿಕ್ಷೆ-ಯನ್ನೇ
ಭಿನ್ನ-ಭಿನ್ನ
ಭಿನ್ನ-ರಾ-ದೆವು
ಭಿನ್ನ-ವಾಗಿ
ಭಿನ್ನ-ವಾ-ಗಿ-ರ-ಲಿಲ್ಲ
ಭಿನ್ನ-ವೆಂ-ಬು-ದನ್ನು
ಭಿನ್ನಾ-ಭಿ-ಪ್ರಾ-ಯ-ಗ-ಳಿ-ದ್ದರೂ
ಭಿನ್ನಾ-ಭಿ-ಪ್ರಾ-ಯ-ಗ-ಳಿ-ದ್ದು-ದನ್ನು
ಭಿನ್ನಾ-ಭಿ-ಪ್ರಾ-ಯವೆ
ಭೀಕರ
ಭೀಕ-ರ-ತೆ-ಯನ್ನೂ
ಭೀಕ-ರ-ವಾಗಿ
ಭೀಕ-ರ-ವಾ-ಗಿ-ತ್ತೆಂಬು
ಭೀಮ-ಬಂ-ಧ-ನ-ದಲ್ಲಿ
ಭೀಮಾ-ಕಾ-ರದ
ಭುಗಿ-ಲೆ-ದ್ದಿತು
ಭುಗಿ-ಲೆದ್ದು
ಭುಜ-ಗಳ
ಭುವ-ನೇ-ಶ್ವ-ರಿ-ದೇ-ವಿಗೆ
ಭೂಗೋಳ
ಭೂಗೋ-ಳ-ಗಳ
ಭೂಜ್ಗೆ
ಭೂಜ್ನಲ್ಲೂ
ಭೂಜ್ನಿಂದ
ಭೂತ
ಭೂತ
ಭೂತ-ಭ-ವಿ-ಷ್ಯ-ಗಳನ್ನೂ
ಭೂತ-ವ-ರ್ತ-ಮಾ-ನ-ಭ-ವಿ-ಷ್ಯತ್
ಭೂತ-ಗ-ಣ-ಗಳು
ಭೂತ-ದ-ರ್ಶ-ನ-ವಾ-ದಂತೆ
ಭೂತ-ಪ್ರೇ-ತ-ಗಳ
ಭೂತಾ-ಕಾ-ರ-ವಾಗಿ
ಭೂತಿ
ಭೂತಿ-ಕ-ಡೆ-ಯು-ಸಿ-ರಿ-ನ-ವ-ರೆಗೂ
ಭೂದೇ-ವಿ-ಯಂತೆ
ಭೂಪಟ
ಭೂಪೇಂ-ದ್ರ-ನಾ-ಥರ
ಭೂಮಂ-ಡ-ಲ-ವನ್ನು
ಭೂಮಿ
ಭೂಮಿಗೆ
ಭೂಮಿಯ
ಭೂಮಿ-ಯನ್ನು
ಭೂಮಿ-ಯಲ್ಲಿ
ಭೂಮಿ-ಯಾದ
ಭೂಮಿ-ಯಿಂದ
ಭೂಮಿಯು
ಭೂರ್ಜ
ಭೃಗು
ಭೆಟ್
ಭೆಟ್ದ್ವಾ-ರ-ಕೆಗೆ
ಭೇಟಿ
ಭೇಟಿ-ಕೊ-ಡು-ವಂತೆ
ಭೇಟಿ-ಗಾಗಿ
ಭೇಟಿ-ನೀ-ಡಿ-ದರು
ಭೇಟಿ-ಮಾ-ಡಲು
ಭೇಟಿ-ಮಾಡಿ
ಭೇಟಿ-ಮಾ-ಡಿದ
ಭೇಟಿ-ಮಾ-ಡಿ-ದರು
ಭೇಟಿ-ಮಾ-ಡುವ
ಭೇಟಿಯ
ಭೇಟಿ-ಯನ್ನು
ಭೇಟಿ-ಯಲ್ಲೆ
ಭೇಟಿ-ಯಲ್ಲೇ
ಭೇಟಿ-ಯಾ-ಗಲು
ಭೇಟಿ-ಯಾ-ಗಲೂ
ಭೇಟಿ-ಯಾಗಿ
ಭೇಟಿ-ಯಾ-ಗಿ-ದ್ದರು
ಭೇಟಿ-ಯಾ-ಗಿ-ದ್ದಳು
ಭೇಟಿ-ಯಾ-ಗಿದ್ದೆ
ಭೇಟಿ-ಯಾ-ಗಿ-ದ್ದೇನೆ
ಭೇಟಿ-ಯಾ-ಗಿ-ರದೆ
ಭೇಟಿ-ಯಾ-ಗಿ-ರು-ವು-ದಾಗಿ
ಭೇಟಿ-ಯಾಗು
ಭೇಟಿ-ಯಾ-ಗು-ತ್ತಿ-ದ್ದರು
ಭೇಟಿ-ಯಾ-ಗು-ತ್ತೇನೆ
ಭೇಟಿ-ಯಾ-ಗುವ
ಭೇಟಿ-ಯಾ-ಗು-ವಂತೆ
ಭೇಟಿ-ಯಾ-ಗು-ವುದು
ಭೇಟಿ-ಯಾದ
ಭೇಟಿ-ಯಾ-ದದ್ದು
ಭೇಟಿ-ಯಾ-ದ-ರ-ಲ್ಲದೆ
ಭೇಟಿ-ಯಾ-ದರು
ಭೇಟಿ-ಯಾ-ದವ
ಭೇಟಿ-ಯಾ-ದ-ವ-ರಲ್ಲಿ
ಭೇಟಿ-ಯಾ-ದಾಗ
ಭೇಟಿ-ಯಾದೆ
ಭೇಟಿ-ಯಿಂದ
ಭೇಟಿ-ಯಿ-ತ್ತರು
ಭೇಟಿ-ಯಿ-ತ್ತ-ರು-ಮುಖ್ಯ
ಭೇಟಿ-ಯಿತ್ತು
ಭೇಟಿಯು
ಭೇಟಿ-ಯೆಂದು
ಭೇತಾ-ಳ-ದಂತೆ
ಭೇದ
ಭೇದ-ಗಳ
ಭೇದ-ಗ-ಳಿ-ವೆಯೆ
ಭೇದ-ಬುದ್ಧಿ
ಭೇದ-ಭಾವ
ಭೇದ-ಭಾ-ವ-ಗಳನ್ನೂ
ಭೇದ-ಭಾ-ವ-ಗ-ಳೆಲ್ಲ
ಭೇದ-ವನ್ನು
ಭೇದ-ವಿಲ್ಲ
ಭೇದ-ವಿ-ಹುದೆ
ಭೇದ-ವಿ-ಹು-ದೆ-ನು-ತಿ-ರಲು
ಭೇದವು
ಭೇದ-ವುಂ-ಟಾ-ಗು-ತ್ತದೆ
ಭೇದಿಸ
ಭೇದಿ-ಸ-ಬೇ-ಕಾದ
ಭೇದಿಸಿ
ಭೇದಿ-ಸಿ-ದುವು
ಭೇಷಾದ
ಭೋಗ
ಭೋಗ-ಗಳ
ಭೋಗ-ಗಳು
ಭೋಗದ
ಭೋಗ-ದ-ಲ್ಲಾ-ಗಲಿ
ಭೋಗ-ಭ-ರಿತ
ಭೋಗ-ಭೂ-ಮಿ-ಯೆ-ಡೆಗೆ
ಭೋಗ-ಲಾ-ಲ-ಸೆ-ಗಾಗಿ
ಭೋಗ-ವ-ನ್ನ-ನು-ಭ-ವಿ-ಸುವು
ಭೋಗ-ವನ್ನು
ಭೋಗ-ವನ್ನೂ
ಭೋಗ-ವಲ್ಲ
ಭೋಗ-ವಾ-ದದ
ಭೋಗ-ವಾ-ದ-ವನ್ನು
ಭೋಗ-ವಾ-ದಿ-ಗಳ
ಭೋಗ-ವಿ-ಲಾ-ಸ-ಗ-ಳೆಲ್ಲ
ಭೋಗವೇ
ಭೋಗ-ವೈ-ಭ-ವ-ಗಳನ್ನು
ಭೋಗಾ
ಭೋಗಾ-ದ-ರ್ಶ-ವನ್ನು
ಭೋಗಾ-ರಾ-ಧ-ನೆಯ
ಭೋಗಿ-ಗ-ಳಾ-ದ-ವ-ರನ್ನು
ಭೋಗಿ-ಸು-ತ್ತಿ-ದ್ದಾರೆ
ಭೋಜನ
ಭೋಜ-ನಕ್ಕೆ
ಭೋಜ-ನ-ಗೃ-ಹ-ದಲ್ಲಿ
ಭೋಜ-ನದ
ಭೋಜ-ನಾ-ನಂ-ತರ
ಭೋರ್ಗ-ರೆತ
ಭೋರ್ಗ-ರೆದು
ಭೋರ್ಗ-ರೆ-ದುವು
ಭೋರ್ಗ-ರೆ-ಯು-ತ್ತಿದೆ
ಭೋರ್ಗ-ರೆ-ಯು-ತ್ತಿ-ರುವ
ಭೋರ್ಗ-ರೆ-ಯುವ
ಭೋಳೇ-ಶಂ-ಕ-ರರ
ಭೌತ-ಶಾಸ್ತ್ರ
ಭೌತಿ-ಕ-ವಾ-ದದ
ಭ್ಯಾಸ-ಎಂಬ
ಭ್ರಮಿ-ಸಲು
ಭ್ರಮಿಸಿ
ಭ್ರಮೆ
ಭ್ರಮೆ-ಗೊ-ಳ-ಗಾ-ಗಿ-ದ್ದಾಗ
ಭ್ರಮೆ-ಯಿಂದ
ಭ್ರಮೆ-ಯೇನೂ
ಭ್ರಷ್ಟ-ಗೊ-ಳಿ-ಸಿ-ರು-ವುದು
ಭ್ರಾಂತ-ನೆಂದೋ
ಭ್ರಾಂತಿ
ಭ್ರಾಂತಿ-ಯನ್ನು
ಭ್ರಾಂತಿ-ಯ-ನ್ನೆಲ್ಲ
ಭ್ರಾಂತಿ-ಯೆಂ-ಬು-ದಾಗಿ
ಭ್ರಾತೃ
ಭ್ರಾತೃತ್ವ
ಭ್ರಾತೃ-ತ್ವದ
ಭ್ರಾತೃ-ತ್ವ-ಭಾ-ವ-ನೆಗೆ
ಭ್ರಾತೃ-ಭಾ-ವದ
ಭ್ರಾತೃ-ಭಾ-ವ-ವನ್ನು
ಭ್ರೂಣಾ-ವ-ಸ್ಥೆ-ಯಿಂದ
ಮಂಕು-ಗೊಳಿ
ಮಂಕು-ಬ-ಡಿ-ದಂತೆ
ಮಂಗ-ಮಾಯ
ಮಂಗ-ಳ-ಕರ
ಮಂಗ-ಳ-ಸಿಂ-ಗನ
ಮಂಗ-ಳ-ಸಿಂ-ಗ-ನತ್ತ
ಮಂಗ-ಳ-ಸಿಂ-ಗನೋ
ಮಂಗ-ಳ-ಸಿಂಗ್
ಮಂಚ-ವನ್ನು
ಮಂಜನ್ನು
ಮಂಜಾ
ಮಂಜಾ-ದುವು
ಮಂಜಿನ
ಮಂಜು
ಮಂಜು-ಹಿ-ಮ-ಗ-ಳೆ-ಲ್ಲ-ದರ
ಮಂಟಪ
ಮಂಟ-ಪಕ್ಕೆ
ಮಂಟ-ಪ-ದಲ್ಲಿ
ಮಂಟ-ಪವು
ಮಂಡಿ
ಮಂಡಿಯ
ಮಂಡಿ-ಯು-ದ್ದದ
ಮಂಡಿ-ಯೂರಿ
ಮಂಡಿ-ಸ-ಲಾದ
ಮಂಡಿ-ಸಲು
ಮಂಡಿಸಿ
ಮಂಡಿ-ಸಿದ
ಮಂಡಿ-ಸಿ-ದರು
ಮಂಡಿ-ಸುತ್ತ
ಮಂಡಿ-ಸು-ತ್ತಿ-ರು-ವುದು
ಮಂಡೂಕ
ಮಂಡೆ-ಯನ್ನು
ಮಂತ್ರ
ಮಂತ್ರ-ಗಳ
ಮಂತ್ರ-ಗಳನ್ನು
ಮಂತ್ರ-ಗ-ಳೆಲ್ಲ
ಮಂತ್ರ-ದೀಕ್ಷೆ
ಮಂತ್ರ-ದೀ-ಕ್ಷೆಗೆ
ಮಂತ್ರ-ದೀ-ಕ್ಷೆ-ಯನ್ನು
ಮಂತ್ರ-ದೀ-ಕ್ಷೆ-ಯನ್ನೂ
ಮಂತ್ರ-ಮಾ-ಟ-ಗ-ಳ-ಲ್ಲಿಲ್ಲ
ಮಂತ್ರ-ಮು-ಗ್ಧ-ಗೊ-ಳಿಸು
ಮಂತ್ರ-ಮು-ಗ್ಧ-ರಾಗಿ
ಮಂತ್ರ-ಮು-ಗ್ಧ-ರಾ-ದಂತೆ
ಮಂತ್ರ-ಮು-ಗ್ಧ-ರಾ-ದರು
ಮಂತ್ರ-ಮು-ಗ್ಧ-ವಾಗಿ
ಮಂತ್ರ-ಮು-ಗ್ಧ-ವಾ-ಗಿಸಿ
ಮಂತ್ರ-ಮು-ಗ್ಧ-ವಾದ
ಮಂತ್ರ-ವನ್ನು
ಮಂತ್ರಾ-ಕ್ಷತೆ
ಮಂತ್ರಾ-ಕ್ಷ-ತೆ-ಮ-ಣ್ಣು-ಗಳಿಂದ
ಮಂತ್ರಿ-ಗಳ
ಮಂತ್ರಿ-ಗ-ಳ-ಲ್ಲೊ-ಬ್ಬ-ನಾದ
ಮಂತ್ರಿ-ಗ-ಳಿಗೆ
ಮಂತ್ರಿ-ಗಳು
ಮಂತ್ರಿ-ದೀಕ್ಷೆ
ಮಂತ್ರಿ-ಯಾದ
ಮಂಥನ
ಮಂಥ-ನದ
ಮಂಥ-ನವು
ಮಂದ
ಮಂದ-ಸ್ಮಿ-ತ-ವ-ದ-ನ-ರಾಗಿ
ಮಂದ-ಹಾಸ
ಮಂದ-ಹಾ-ಸ-ಪೂ-ರಿತ
ಮಂದ-ಹಾ-ಸ-ವ-ದ-ನ-ವಾದ
ಮಂದ-ಹಾ-ಸ-ವನ್ನು
ಮಂದಾರ
ಮಂದಿ
ಮಂದಿಗೆ
ಮಂದಿ-ಡು-ತ್ತಾರೆ
ಮಂದಿಯ
ಮಂದಿ-ಯಂ-ತೆಯೇ
ಮಂದಿ-ಯನ್ನು
ಮಂದಿ-ರ-ಗಳ
ಮಕ್ಕಳ
ಮಕ್ಕ-ಳಂತೂ
ಮಕ್ಕ-ಳಂತೆ
ಮಕ್ಕ-ಳನ್ನು
ಮಕ್ಕ-ಳ-ನ್ನೆಂ-ದಿಗೂ
ಮಕ್ಕ-ಳಾ-ಟ-ಕ್ಕೆಂದೇ
ಮಕ್ಕ-ಳಾ-ಟ-ಧ-ರ್ಮ-ಪ್ರ-ಚಾರ
ಮಕ್ಕ-ಳಾ-ಟಿ-ಕೆ-ಯಲ್ಲಿ
ಮಕ್ಕ-ಳಿಗೂ
ಮಕ್ಕ-ಳಿಗೇ
ಮಕ್ಕ-ಳಿ-ದ್ದಾರೆ
ಮಕ್ಕಳು
ಮಕ್ಕ-ಳೆಂ-ಬಂತೆ
ಮಖರ್ಜಿ
ಮಗ
ಮಗ-ದೊಮ್ಮೆ
ಮಗನ
ಮಗ-ನಂತೆ
ಮಗ-ನಂ-ತೆಯೇ
ಮಗನೂ
ಮಗನೇ
ಮಗ-ಳಾದ
ಮಗಳು
ಮಗಳೂ
ಮಗ-ಳೊ-ಬ್ಬ-ಳನ್ನು
ಮಗು
ಮಗು-ಚಿ-ಕೊ-ಳ್ಳು-ವಂ-ತಾಗಿ
ಮಗು-ವನ್ನು
ಮಗು-ವಿಗೂ
ಮಗು-ವಿಗೆ
ಮಗು-ವಿದೆ
ಮಗು-ವಿನ
ಮಗು-ವಿ-ನಂತೆ
ಮಗುವು
ಮಗುವೂ
ಮಗುವೇ
ಮಗು-ವೊಂ-ದರ
ಮಗ್ಗುಲು
ಮಗ್ನ
ಮಗ್ನ-ನಾ-ಗಿ-ರು-ತ್ತಾನೆ
ಮಗ್ನ-ರಾಗಿ
ಮಗ್ನ-ರಾ-ಗಿ-ದ್ದರು
ಮಗ್ನ-ರಾ-ಗಿ-ದ್ದರೂ
ಮಗ್ನ-ರಾ-ಗಿ-ರು-ತ್ತಿ-ದ್ದರು
ಮಗ್ನ-ರಾ-ದರು
ಮಗ್ನ-ವಾಗಿ
ಮಗ್ನ-ವಾ-ಯಿತೋ
ಮಚ್ಚು-ಗತ್ತಿ
ಮಜು-ಮ್ದಾರ
ಮಜು-ಮ್ದಾ-ರನ
ಮಜು-ಮ್ದಾ-ರ-ನನ್ನು
ಮಜು-ಮ್ದಾ-ರನೇ
ಮಜು-ಮ್ದಾ-ರ-ನೇನೂ
ಮಜು-ಮ್ದಾರ್
ಮಜೂರಿ
ಮಟ್ಟ
ಮಟ್ಟಕ್ಕೆ
ಮಟ್ಟಕ್ಕೇ
ಮಟ್ಟದ
ಮಟ್ಟ-ದ-ಲ್ಲಿ-ರ-ಬೇ-ಕೆಂ-ದರೆ
ಮಟ್ಟ-ದ-ಲ್ಲಿ-ರು-ವು-ದಾಗಿ
ಮಟ್ಟ-ದ-ವರೆ-ಗಿನ
ಮಟ್ಟ-ದ್ದಾ-ಗಿ-ದ್ದರೂ
ಮಟ್ಟ-ಹಾಕಿ
ಮಟ್ಟಿ-ಗಾ-ದರೂ
ಮಟ್ಟಿ-ಗಿದೆ
ಮಟ್ಟಿಗೂ
ಮಟ್ಟಿಗೆ
ಮಟ್ಟಿ-ಗೆಂ-ದರೆ
ಮಟ್ಟಿನ
ಮಠ
ಮಠ-ಕ್ಕಾಗಿ
ಮಠಕ್ಕೆ
ಮಠ-ಗ-ಳೊ-ಳಗೆ
ಮಠದ
ಮಠ-ದಲ್ಲಿ
ಮಠ-ದೊ-ಳಗೆ
ಮಠ-ವನ್ನು
ಮಠವು
ಮಠ-ವೊಂ-ದನ್ನು
ಮಠ-ವೊಂದು
ಮಠಾ-ಧಿ-ಪ-ತಿ-ಗ-ಳಾದ
ಮಡ-ಕೆಯ
ಮಡಿ-ಯು-ತ್ತೇನೆ
ಮಡಿ-ಲನ್ನು
ಮಡಿ-ಲಲ್ಲಿ
ಮಡಿ-ಲ-ಲ್ಲಿ-ಟ್ಟು-ಕೊಂ-ಡಿ-ರುವ
ಮಡಿ-ಲ-ಲ್ಲಿ-ಟ್ಟು-ಕೊಂಡು
ಮಡಿ-ಲ-ಲ್ಲಿ-ರುವ
ಮಡಿ-ಲಲ್ಲೇ
ಮಡಿ-ಲಾದ
ಮಡಿ-ಲ್ಲ-ಲಿ-ರುವ
ಮಡಿಸಿ
ಮಡು-ವಿ-ನ-ಲ್ಲಿ-ರು-ತ್ತಾ-ನೆಂ-ದರೆ
ಮಣ-ಗ-ಟ್ಟಲೆ
ಮಣಿದು
ಮಣಿ-ಭಾಯ್
ಮಣಿ-ಯಲೇ
ಮಣಿಯು
ಮಣಿ-ಯು-ತ್ತಿ-ದ್ದರು
ಮಣಿಯೇ
ಮಣಿ-ರ್ಮ-ಣಿಃ
ಮಣ್ಣಿನ
ಮಣ್ಣಿ-ನಿಂದ
ಮಣ್ಣು
ಮತ
ಮತ-ಧ-ರ್ಮ-ಗಳ
ಮತ-ಪಂ-ಥ-ಗಳ
ಮತ-ಪಂ-ಥ-ಗ-ಳಿಗೂ
ಮತ-ಪಂ-ಥ-ವನ್ನು
ಮತಕ್ಕೂ
ಮತಕ್ಕೆ
ಮತ-ಗಳ
ಮತ-ಗ-ಳ-ನ್ನು-ಕು-ತ-ರ್ಕ-ಗಳನ್ನು
ಮತ-ಗ-ಳಿಗೆ
ಮತ-ಗ್ರಂ-ಥ-ವಾದ
ಮತದ
ಮತ-ದಲ್ಲೋ
ಮತ-ಧರ್ಮ
ಮತ-ಧ-ರ್ಮ-ಗಳ
ಮತ-ಪಂ-ಥ-ಗಳ
ಮತ-ಪಂ-ಥ-ಗಳಲ್ಲಿ
ಮತ-ಪಂ-ಥ-ಗ-ಳಾ-ಚೆ-ಗಿನ
ಮತ-ಪಂ-ಥ-ಗಳಿಂದ
ಮತ-ಪಂ-ಥ-ಗ-ಳಿಗೆ
ಮತ-ಭೇ-ದ-ಗಳನ್ನೂ
ಮತ-ಭ್ರಾಂ-ತ-ತೆಯ
ಮತ-ಭ್ರಾಂ-ತ-ರಂತೆ
ಮತ-ವನ್ನು
ಮತವೂ
ಮತ-ವೆಂದು
ಮತವೇ
ಮತವೋ
ಮತ-ಸ್ಥ-ರಿಗೆ
ಮತಾಂ
ಮತಾಂ-ತರ
ಮತಾಂ-ತ-ರ-ಗೊಂ-ಡ-ವಳು
ಮತಾಂ-ತ-ರ-ಗೊ-ಳಿ-ಸಲು
ಮತಾಂ-ತ-ರ-ಗೊ-ಳ್ಳ-ಬೇ-ಕಾ-ಗಿತ್ತು
ಮತಾಂ-ತ-ರ-ಗೊ-ಳ್ಳ-ಬೇ-ಕಾ-ಗಿಲ್ಲ
ಮತಾಂ-ತ-ರಿ-ಸು-ತ್ತಿ-ದ್ದರು
ಮತಾಂ-ತ-ರಿ-ಸುವ
ಮತಾಂಧ
ಮತಾಂ-ಧತೆ
ಮತಾಂ-ಧ-ತೆಗೆ
ಮತಾಂ-ಧ-ತೆಯ
ಮತಾಂ-ಧ-ತೆ-ಯನ್ನು
ಮತಾಂ-ಧ-ನಾ-ಗಿದ್ದ
ಮತಾಂ-ಧ-ನಾ-ದಲ್ಲಿ
ಮತಾಂ-ಧನೂ
ಮತಾಂ-ಧನೇ
ಮತಾಂ-ಧರ
ಮತಾಂ-ಧ-ರಿಂದ
ಮತಾ-ವ-ಲಂಬಿ
ಮತೀಯ
ಮತೀ-ಯರು
ಮತ್ತ-ವರ
ಮತ್ತ-ಷ್ಟನ್ನು
ಮತ್ತಷ್ಟು
ಮತ್ತಾ
ಮತ್ತಾ-ರಿಗೆ
ಮತ್ತಾ-ರಿ-ದ್ದಾರೆ
ಮತ್ತಾರೂ
ಮತ್ತಾವ
ಮತ್ತಾ-ವುದೂ
ಮತ್ತಾ-ವುದೋ
ಮತ್ತಿ-ತರ
ಮತ್ತಿ-ತ-ರರು
ಮತ್ತಿ-ತ-ರಿ-ರಿಗೂ
ಮತ್ತಿ-ನ್ನಾ-ವುದು
ಮತ್ತಿ-ನ್ನೆಷ್ಟೋ
ಮತ್ತು
ಮತ್ತು-ಇನ್ನೂ
ಮತ್ತೂ
ಮತ್ತೆ
ಮತ್ತೆಂ-ದಾ-ದರೂ
ಮತ್ತೆಂದೂ
ಮತ್ತೆ-ಮತ್ತೆ
ಮತ್ತೇನು
ಮತ್ತೊಂ-ದನ್ನು
ಮತ್ತೊಂ-ದ-ರಿಂದ
ಮತ್ತೊಂ-ದಿ-ರ-ಲಾ-ರ-ದೆಂದು
ಮತ್ತೊಂ-ದಿ-ರಲು
ಮತ್ತೊಂ-ದಿಲ್ಲ
ಮತ್ತೊಂ-ದಿ-ಲ್ಲ-ಇದು
ಮತ್ತೊಂದು
ಮತ್ತೊಂ-ದು-ಕಡೆ
ಮತ್ತೊಂ-ದೆಡೆ
ಮತ್ತೊಬ್ಬ
ಮತ್ತೊ-ಬ್ಬ-ನಿಗೆ
ಮತ್ತೊ-ಬ್ಬರ
ಮತ್ತೊ-ಬ್ಬ-ರನ್ನು
ಮತ್ತೊ-ಬ್ಬ-ರಿ-ಗಾಗಿ
ಮತ್ತೊ-ಬ್ಬರು
ಮತ್ತೊ-ಬ್ಬ-ರೆಂ-ದರೆ
ಮತ್ತೊ-ಬ್ಬಳು
ಮತ್ತೊಮ್ಮೆ
ಮತ್ಸರ
ಮತ್ಸ-ರ-ಗೊಂಡ
ಮತ್ಸ-ರ-ತಾಳಿ
ಮತ್ಸ-ರ-ವನ್ನು
ಮತ್ಸ-ರ-ವುಂ-ಟಾ-ದರೆ
ಮಥಿಸಿ
ಮಥಿ-ಸು-ವು-ದರ
ಮದ
ಮದ-ರಾ-ಸನ್ನು
ಮದ-ರಾಸಿ
ಮದ-ರಾ-ಸಿಗೆ
ಮದ-ರಾ-ಸಿನ
ಮದ-ರಾ-ಸಿ-ನಲ್ಲಿ
ಮದ-ರಾ-ಸಿ-ನ-ಲ್ಲಿದ್ದ
ಮದ-ರಾ-ಸಿ-ನ-ಲ್ಲಿ-ದ್ದಾಗ
ಮದ-ರಾ-ಸಿ-ನಿಂದ
ಮದ-ರಾಸೀ
ಮದ-ರಾಸು
ಮದ-ರಾ-ಸು-ಗಳಲ್ಲಿ
ಮದರ್
ಮದುವೆ
ಮದು-ವೆಯ
ಮದು-ವೆ-ಯಾ-ಗಲೂ
ಮದು-ವೆ-ಯಾಗಿ
ಮದು-ವೆ-ಯಾ-ಗಿ-ದ್ದಳು
ಮದು-ವೆ-ಯೆಂ-ದ-ರೇನು
ಮದ್ದಿನ
ಮದ್ದೂ
ಮದ್ರಾಸಿ
ಮದ್ರಾ-ಸಿ-ಗರೇ
ಮದ್ರಾ-ಸಿ-ಗಳ
ಮದ್ರಾ-ಸಿ-ಗ-ಳೆಂದು
ಮದ್ರಾ-ಸಿಗೆ
ಮದ್ರಾ-ಸಿನ
ಮದ್ರಾ-ಸಿ-ನಲ್ಲಿ
ಮದ್ರಾ-ಸಿ-ನ-ಲ್ಲೊಂದು
ಮದ್ರಾ-ಸಿ-ನಿಂದ
ಮದ್ರಾಸೀ
ಮದ್ರಾಸು
ಮದ್ರಾ-ಸು-ಗಳ
ಮಧು-ಕ-ರಿಯ
ಮಧುರ
ಮಧು-ರ-ಗಂ-ಭೀರ
ಮಧು-ರ-ಕಂಠ
ಮಧು-ರ-ನಾ-ಮ-ದಿಂದ
ಮಧು-ರ-ನಾ-ಮ-ದಿಂ-ದ-ಅ-ಮೃ-ತ-ಪು-ತ್ರರು
ಮಧು-ರ-ರಾ-ಗಿ-ದ್ದಾರೆ
ಮಧು-ರ-ವಾ-ಗಿತ್ತು
ಮಧು-ರ-ವಾ-ಗಿದ್ದು
ಮಧು-ರ-ವಾ-ಗಿ-ದ್ದುವು
ಮಧು-ರ-ವಾ-ಗಿ-ರಲು
ಮಧು-ರ-ವಾ-ಗಿ-ರು-ವು-ದ-ಕ್ಕಾಗಿ
ಮಧು-ರ-ವಾ-ಗಿ-ಸಲು
ಮಧು-ರ-ವಾದ
ಮಧು-ರೆ-ಯನ್ನು
ಮಧು-ರೆ-ಯಲ್ಲಿ
ಮಧು-ರೆ-ಯಿಂದ
ಮಧು-ಸೂ-ದನ
ಮಧ್ಯ
ಮಧ್ಯ-ಕಾ-ಲೀನ
ಮಧ್ಯಕ್ಕೆ
ಮಧ್ಯದ
ಮಧ್ಯ-ದ-ಲ್ಲಾ-ಗ-ಲಿ-ನಿ-ಮಗೆ
ಮಧ್ಯ-ದಲ್ಲಿ
ಮಧ್ಯ-ದ-ಲ್ಲಿದ್ದ
ಮಧ್ಯ-ದಲ್ಲೂ
ಮಧ್ಯ-ದಲ್ಲೇ
ಮಧ್ಯ-ದೊ-ಳಗೆ
ಮಧ್ಯ-ಪ-ಶ್ಚಿಮ
ಮಧ್ಯ-ಪ-ಶ್ಚಿ-ಮದ
ಮಧ್ಯ-ಪ್ರಾಂ-ತ-ಗ-ಳಿಗೆ
ಮಧ್ಯ-ಭಾ-ರ-ತ-ದಲ್ಲಿ
ಮಧ್ಯ-ಮ-ವ-ರ್ಗದ
ಮಧ್ಯ-ರಾತ್ರಿ
ಮಧ್ಯ-ರಾ-ತ್ರಿಯ
ಮಧ್ಯ-ರಾ-ತ್ರಿ-ಯ-ವ-ರೆಗೂ
ಮಧ್ಯ-ರಾ-ತ್ರಿ-ಯಾ-ಗಿತ್ತು
ಮಧ್ಯ-ವ-ಯಸ್ಸು
ಮಧ್ಯಾಹ್ನ
ಮಧ್ಯಾ-ಹ್ನದ
ಮಧ್ಯಾ-ಹ್ನ-ದ-ವ-ರೆಗೂ
ಮಧ್ಯಾ-ಹ್ನ-ದಿಂ-ದಲೂ
ಮಧ್ಯಾ-ಹ್ನ-ವಾ-ಯಿತು
ಮಧ್ಯಾ-ಹ್ನವೋ
ಮಧ್ಯೆ
ಮಧ್ವಾ-ಚಾ-ರ್ಯರ
ಮನ
ಮನಃ-ಪ-ಟ-ಲದ
ಮನಃ-ಶಾ-ಸ್ತ್ರ-ಜ್ಞರು
ಮನಃ-ಸ್ಥಿ-ತಿ-ಯಲ್ಲಿ
ಮನ-ಕ-ರ-ಗಿ-ಸಿದ
ಮನ-ಗಂಡ
ಮನ-ಗಂ-ಡರು
ಮನ-ಗಂ-ಡಾಗ
ಮನ-ಗಂ-ಡಿದ್ದ
ಮನ-ಗಂ-ಡಿ-ದ್ದರೂ
ಮನ-ಗಂ-ಡಿ-ರು-ವು-ದಾಗಿ
ಮನ-ಗಂಡು
ಮನ-ಗಾ-ಣ-ಬ-ಹುದು
ಮನ-ಗಾ-ಣಲು
ಮನ-ಗಾಣಿ
ಮನ-ಗಾ-ಣಿ-ಸಲು
ಮನ-ಗಾ-ಣಿ-ಸಿ-ಕೊಟ್ಟು
ಮನ-ಗಾ-ಣಿ-ಸಿ-ದರು
ಮನ-ಗಾ-ಣಿ-ಸಿ-ದ-ರೆಂ-ಬು-ದನ್ನು
ಮನ-ಗಾ-ಣಿ-ಸಿ-ದ್ದಾರೆ
ಮನ-ಗಾ-ಣಿ-ಸು-ತ್ತಿ-ರು-ವಂ-ತಿತ್ತು
ಮನ-ಗಾ-ಣಿ-ಸು-ವು-ದ-ರಲ್ಲಿ
ಮನ-ಗಾ-ಣು-ತ್ತಿ-ದ್ದಾರೆ
ಮನ-ಗಾ-ಣು-ವಂ-ತಾ-ಯಿತು
ಮನ-ದ-ಟ್ಟಾಗಿ
ಮನ-ದ-ಟ್ಟಾ-ಗಿತ್ತು
ಮನ-ದ-ಟ್ಟಾ-ಗು-ತ್ತಿದೆ
ಮನ-ದಟ್ಟು
ಮನ-ದ-ಟ್ಟು-ಮಾ-ಡಿಸ
ಮನ-ದಲ್ಲಿ
ಮನ-ದಲ್ಲೂ
ಮನ-ದಲ್ಲೇ
ಮನ-ದ-ಲ್ಲೊಂದು
ಮನ-ದ-ಲ್ಲೊಂದೇ
ಮನ-ದಾ-ಳದ
ಮನ-ದಾ-ಳ-ದಲ್ಲಿ
ಮನ-ದಾ-ಸೆಗೆ
ಮನ-ದಿಂ-ಗಿ-ತ-ವನ್ನು
ಮನನ
ಮನ-ನೀ-ಯ-ವಾ-ಗಿದೆ
ಮನ-ಬಂ-ದೆ-ಡೆಗೆ
ಮನ-ಬಿಚ್ಚಿ
ಮನ-ಮು-ಟ್ಟಿ-ಸಿ-ದ್ದರು
ಮನ-ಮುಟ್ಟು
ಮನ-ಮು-ಟ್ಟು-ವಂ-ತಹ
ಮನ-ಮು-ಟ್ಟು-ವಂ-ತಿದೆ
ಮನ-ಮು-ಟ್ಟು-ವಂತೆ
ಮನ-ಮೋ-ಹಕ
ಮನ-ಮೋ-ಹ-ಕ-ವಾಗಿ
ಮನ-ಮೋ-ಹ-ಕ-ವಾದ
ಮನ-ರಂ-ಜಕ
ಮನ-ರಂ-ಜನೆ
ಮನ-ರಂ-ಜ-ನೆಯ
ಮನ-ವ-ರಿಕೆ
ಮನ-ವ-ರಿ-ಕೆ-ಯಾಗು
ಮನ-ವ-ರಿ-ಕೆ-ಯಾ-ಗು-ವಂ-ತಿತ್ತು
ಮನ-ವ-ರಿ-ಕೆ-ಯಾ-ಯಿತು
ಮನ-ವ-ರಿ-ಕೆ-ಯಾ-ಯಿ-ತು-ತಮ್ಮ
ಮನ-ವಿದು
ಮನ-ವಿಯು
ಮನ-ವೊ-ಲಿ-ಸ-ಲಾ-ಗಿತ್ತು
ಮನ-ವೊ-ಲಿ-ಸಿ-ದರು
ಮನ-ಶ್ಶಕ್ತಿ
ಮನ-ಶ್ಶಕ್ತಿಯ
ಮನ-ಶ್ಶಕ್ತಿ-ಯನ್ನು
ಮನ-ಶ್ಶಾಂತಿ
ಮನ-ಶ್ಶಾಸ್ತ್ರ
ಮನ-ಶ್ಶಾ-ಸ್ತ್ರ-ಇ-ವು-ಗಳಿಂದ
ಮನ-ಶ್ಶಾ-ಸ್ತ್ರ-ಕ್ಕಿಂತ
ಮನ-ಶ್ಶಾ-ಸ್ತ್ರ-ಜ್ಞ-ರನ್ನು
ಮನ-ಶ್ಶಾ-ಸ್ತ್ರ-ಜ್ಞರೂ
ಮನ-ಶ್ಶಾ-ಸ್ತ್ರದ
ಮನ-ಶ್ಶಾ-ಸ್ತ್ರವು
ಮನ-ಸಾರೆ
ಮನ-ಸು-ಗ-ಳ-ಲ್ಲೆಲ್ಲ
ಮನ-ಸೂರೆ
ಮನ-ಸೂ-ರೆ-ಗೊಂ-ಡಿ-ದ್ದರು
ಮನ-ಸೂ-ರೆ-ಗೊ-ಳ್ಳುವ
ಮನ-ಸೆ-ಳೆ-ದದ್ದು
ಮನ-ಸೋ-ತರು
ಮನ-ಸೋ-ತ-ವ-ನೊಬ್ಬ
ಮನ-ಸ್ಕರು
ಮನ-ಸ್ತಾ-ಪ-ವಿ-ಲ್ಲ-ದಿ-ರು-ವ-ವ-ರೆಗೆ
ಮನ-ಸ್ಥಿತಿ
ಮನ-ಸ್ಥಿ-ತಿಯ
ಮನ-ಸ್ಥಿ-ತಿ-ಯ-ನ್ನ-ರಿತ
ಮನ-ಸ್ಥಿ-ತಿ-ಯನ್ನು
ಮನ-ಸ್ಥಿ-ತಿ-ಯ-ಲ್ಲಿ-ದ್ದರು
ಮನ-ಸ್ಥಿ-ತಿ-ಯೊಂ-ದ-ರ-ಲ್ಲಿ-ದ್ದೇನೆ
ಮನ-ಸ್ಸನ್ನು
ಮನಸ್ಸಾ
ಮನ-ಸ್ಸಾ-ಗ-ಲಿಲ್ಲ
ಮನ-ಸ್ಸಾ-ಗಿತ್ತು
ಮನ-ಸ್ಸಾ-ಗು-ತ್ತದೆ
ಮನ-ಸ್ಸಾ-ದರೆ
ಮನ-ಸ್ಸಾ-ಯಿತು
ಮನಸ್ಸಿ
ಮನ-ಸ್ಸಿಗೂ
ಮನ-ಸ್ಸಿಗೆ
ಮನ-ಸ್ಸಿ-ಗೊಂದು
ಮನ-ಸ್ಸಿತ್ತು
ಮನ-ಸ್ಸಿನ
ಮನ-ಸ್ಸಿ-ನಂ-ತೆಯೇ
ಮನ-ಸ್ಸಿ-ನಲ್ಲಿ
ಮನ-ಸ್ಸಿ-ನ-ಲ್ಲಿ-ಇಡೀ
ಮನ-ಸ್ಸಿ-ನ-ಲ್ಲಿ-ಟ್ಟು-ಕೊಂಡು
ಮನ-ಸ್ಸಿ-ನ-ಲ್ಲಿತ್ತು
ಮನ-ಸ್ಸಿ-ನ-ಲ್ಲಿ-ತ್ತೆಂದು
ಮನ-ಸ್ಸಿ-ನ-ಲ್ಲಿ-ದ್ದುವು
ಮನ-ಸ್ಸಿ-ನ-ಲ್ಲಿದ್ದೇ
ಮನ-ಸ್ಸಿ-ನ-ಲ್ಲಿಯೂ
ಮನ-ಸ್ಸಿ-ನ-ಲ್ಲಿ-ರುವ
ಮನ-ಸ್ಸಿ-ನ-ಲ್ಲಿ-ರು-ವುದನ್ನು
ಮನ-ಸ್ಸಿ-ನಲ್ಲೂ
ಮನ-ಸ್ಸಿ-ನಲ್ಲೇ
ಮನ-ಸ್ಸಿ-ನ-ಲ್ಲೇ-ನಿ-ತ್ತೆಂದು
ಮನ-ಸ್ಸಿ-ನ-ಲ್ಲೇ-ನಿ-ದೆ-ಯೆಂ-ಬುದು
ಮನ-ಸ್ಸಿ-ನ-ಲ್ಲೇ-ನಿ-ದೆಯೋ
ಮನ-ಸ್ಸಿ-ನ-ಲ್ಲೊಂದು
ಮನ-ಸ್ಸಿ-ನಾ-ಳ-ದಿಂದ
ಮನ-ಸ್ಸಿ-ನಿಂದ
ಮನ-ಸ್ಸಿ-ನಿಂ-ದಲೇ
ಮನ-ಸ್ಸಿನ್ನೂ
ಮನ-ಸ್ಸಿ-ರ-ಲಿಲ್ಲ
ಮನ-ಸ್ಸಿಲ್ಲ
ಮನ-ಸ್ಸಿ-ಲ್ಲದೆ
ಮನ-ಸ್ಸೀಗ
ಮನಸ್ಸು
ಮನ-ಸ್ಸು
ಮನ-ಸ್ಸು-ಇಂ-ದ್ರಿ-ಯ-ಗಳ
ಮನ-ಸ್ಸು-ಬುದ್ಧಿ
ಮನ-ಸ್ಸು-ಬು-ದ್ಧಿ-ಗ-ಳ-ನ್ನಾ-ವ-ರಿ-ಸಿತು
ಮನ-ಸ್ಸು-ಇ-ವು-ಗ-ಳದು
ಮನ-ಸ್ಸು-ಗ-ಳಿಗೆ
ಮನ-ಸ್ಸು-ಗ-ಳೊಂ-ದಿಗೆ
ಮನಸ್ಸೂ
ಮನಸ್ಸೇ
ಮನಿ-ಹಹ
ಮನು
ಮನು-ಕು-ಲದ
ಮನುಷ್ಯ
ಮನು-ಷ್ಯ-ತ್ವ-ವನ್ನು
ಮನು-ಷ್ಯ-ತ್ವ-ವು-ಳ್ಳ-ವರು
ಮನು-ಷ್ಯನ
ಮನು-ಷ್ಯ-ನಂತೆ
ಮನು-ಷ್ಯ-ನನ್ನು
ಮನು-ಷ್ಯ-ನ-ವರೆ-ಗೂ-ಅ-ತ್ಯಂತ
ಮನು-ಷ್ಯ-ನಾ-ದರೋ
ಮನು-ಷ್ಯ-ನಿಂದ
ಮನು-ಷ್ಯ-ನಿಗೆ
ಮನು-ಷ್ಯನು
ಮನು-ಷ್ಯ-ನೆಂ-ದಾ-ದರೂ
ಮನು-ಷ್ಯ-ನೆಂದು
ಮನು-ಷ್ಯ-ನೊಬ್ಬ
ಮನು-ಷ್ಯನೋ
ಮನು-ಷ್ಯ-ಮಾ-ತ್ರ-ನಾ-ದ-ವನು
ಮನು-ಷ್ಯ-ಮಾ-ತ್ರ-ರಿಂದ
ಮನು-ಷ್ಯ-ಮಾ-ತ್ರ-ರಿಗೆ
ಮನು-ಷ್ಯರ
ಮನು-ಷ್ಯ-ರ-ನ್ನಾ-ಗಲಿ
ಮನು-ಷ್ಯ-ರ-ನ್ನಾ-ಗಿ-ಸಲು
ಮನು-ಷ್ಯ-ರನ್ನು
ಮನು-ಷ್ಯ-ರ-ನ್ನೆಲ್ಲ
ಮನು-ಷ್ಯ-ರ-ಲ್ಲೆಲ್ಲ
ಮನು-ಷ್ಯ-ರಾ-ಗಿಲ್ಲ
ಮನು-ಷ್ಯ-ರಿಗೆ
ಮನು-ಷ್ಯರು
ಮನು-ಷ್ಯ-ರೇನೋ
ಮನು-ಷ್ಯರೋ
ಮನು-ಷ್ಯಾಃ
ಮನೆ
ಮನೆ-ಗಳ
ಮನೆ-ಗಳಲ್ಲಿ
ಮನೆ-ಗ-ಳಲ್ಲೂ
ಮನೆ-ಗ-ಳಿಂ-ದಲೇ
ಮನೆ-ಗ-ಳಿಗೂ
ಮನೆ-ಗ-ಳಿಗೆ
ಮನೆ-ಗಳು
ಮನೆ-ಗ-ಳು-ದೋ-ಣಿ-ಗ-ಳೆಲ್ಲ
ಮನೆ-ಗಳೂ
ಮನೆಗೂ
ಮನೆಗೆ
ಮನೆ-ಗೆ-ಲ-ಸ-ವೆಲ್ಲ
ಮನೆಗೇ
ಮನೆ-ತ-ನದ
ಮನೆ-ತ-ನ-ಸ್ಥ-ನೊಬ್ಬ
ಮನೆ-ಬಾ-ಗಿ-ಲಿಗೇ
ಮನೆ-ಮಂದಿ
ಮನೆ-ಮಂ-ದಿ-ಯೆಲ್ಲ
ಮನೆ-ಮಾ-ಡಿದ್ದ
ಮನೆ-ಮಾ-ತಾ-ಗಿದ್ದು
ಮನೆ-ಮಾ-ತಾ-ಗಿವೆ
ಮನೆಯ
ಮನೆ-ಯನ್ನು
ಮನೆ-ಯನ್ನೇ
ಮನೆ-ಯಲ್ಲಿ
ಮನೆ-ಯ-ಲ್ಲಿದ್ದ
ಮನೆ-ಯ-ಲ್ಲಿ-ದ್ದರು
ಮನೆ-ಯ-ಲ್ಲಿ-ದ್ದ-ವರು
ಮನೆ-ಯ-ಲ್ಲಿ-ದ್ದಷ್ಟು
ಮನೆ-ಯ-ಲ್ಲಿ-ದ್ದಾಗ
ಮನೆ-ಯ-ಲ್ಲಿ-ರು-ವು-ದಾಗಿ
ಮನೆ-ಯ-ಲ್ಲಿ-ರು-ವುದು
ಮನೆ-ಯಲ್ಲೂ
ಮನೆ-ಯಲ್ಲೇ
ಮನೆ-ಯವ
ಮನೆ-ಯ-ವ-ರನ್ನು
ಮನೆ-ಯ-ವ-ರಾದ
ಮನೆ-ಯ-ವ-ರಿಗೆ
ಮನೆ-ಯ-ವ-ರಿ-ಗೆಲ್ಲ
ಮನೆ-ಯ-ವರು
ಮನೆ-ಯ-ವರೆಲ್ಲ
ಮನೆ-ಯ-ವರೆ-ಲ್ಲರ
ಮನೆ-ಯ-ವರೆ-ಲ್ಲರೂ
ಮನೆ-ಯ-ವ-ರೊ-ಬ್ಬರೂ
ಮನೆ-ಯಾ-ಗಿ-ತ್ತೆಂದೂ
ಮನೆ-ಯಿಂದ
ಮನೆ-ಯಿತ್ತು
ಮನೆ-ಯಿ-ದ್ದದ್ದು
ಮನೆ-ಯಿ-ದ್ದುದು
ಮನೆಯು
ಮನೆ-ಯೊಂ-ದ-ರಲ್ಲಿ
ಮನೆ-ಯೊ-ಳಗೆ
ಮನೆ-ಯೊ-ಳ-ಗೇ-ನೆ-ಲಕ್ಕೆ
ಮನೆ-ವಾ-ರ್ತೆಯ
ಮನೋ-ಗ್ರಾ-ಹ್ಯ-ವಾ-ದಂ-ಥವು
ಮನೋ-ಧ-ರ್ಮ-ಕ್ಕ-ನು-ಗು-ಣ-ವಾದ
ಮನೋ-ನಿ-ಗ್ರಹ
ಮನೋ-ಭಾವ
ಮನೋ-ಭಾ-ವ-ಅ-ಭಿ-ರು-ಚಿ-ಸಾ-ಧ್ಯ-ತೆ-ಗಳೂ
ಮನೋ-ಭಾ-ವದ
ಮನೋ-ಭಾ-ವ-ದಲ್ಲಿ
ಮನೋ-ಭಾ-ವ-ದವ
ಮನೋ-ಭಾ-ವ-ದಿಂದ
ಮನೋ-ಭಾ-ವ-ದಿಂ-ದಾಗಿ
ಮನೋ-ಭಾ-ವ-ವ-ನ್ನಾ-ಗಲಿ
ಮನೋ-ಭಾ-ವ-ವನ್ನು
ಮನೋ-ಭಾ-ವ-ವನ್ನೂ
ಮನೋ-ಭಾ-ವ-ವಿ-ತ್ತಾ-ದರೂ
ಮನೋ-ಭಾ-ವ-ವೆಂ-ಥದು
ಮನೋ-ವಿ-ಜ್ಞಾನ
ಮನೋ-ವಿ-ಜ್ಞಾ-ನದ
ಮನೋ-ವಿ-ಜ್ಞಾ-ನವು
ಮನೋ-ವಿ-ಜ್ಞಾ-ನಿ-ಗ-ಳ-ಮು-ಖ್ಯ-ವಾಗಿ
ಮನೋ-ವಿ-ಜ್ಞಾ-ನಿಯೂ
ಮನೋ-ವೃತ್ತಿ
ಮನೋ-ವೃ-ತ್ತಿ-ಗಳು
ಮನೋ-ವೃ-ತ್ತಿ-ಯಲ್ಲಿ
ಮನೋ-ವೃ-ತ್ತಿ-ಯ-ವಳು
ಮನೋ-ವೃ-ತ್ತಿ-ಯು-ಳ್ಳ-ವ-ನಾ-ಗಿದ್ದು
ಮನೋ-ಹರ
ಮನೋ-ಹ-ರ-ವಾಗಿ
ಮನೋ-ಹ-ರ-ವಾದ
ಮನ್ನಣೆ
ಮನ್ನ-ಣೆ-ಸ್ಥಾ-ನ-ಮಾ-ನ-ಗಳನ್ನು
ಮನ್ನ-ಣೆ-ಗಾಗಿ
ಮನ್ನ-ಣೆಗೆ
ಮನ್ನಿಸಿ
ಮನ್ನಿ-ಸು-ವಿ-ರೆಂದು
ಮನ್ಮ-ಥ-ನಾಥ
ಮನ್ಮ-ಥ-ನಾ-ಥರ
ಮನ್ಮ-ಥ-ನಾ-ಥ-ರಿಗೆ
ಮನ್ಮ-ಥ-ನಾ-ಥರು
ಮನ್ಮ-ಥ-ಬಾಬು
ಮನ್ಮ-ಥ-ಬಾ-ಬು-ಗಳ
ಮನ್ಮ-ಥ-ಬಾ-ಬು-ಗಳು
ಮನ್ಸುಖ
ಮನ್ಸ್
ಮಬ್ಬಾ-ಗು-ತ್ತದೆ
ಮಬ್ಬಿ-ನಿಂದ
ಮಬ್ಬು-ಗ-ತ್ತಲು
ಮಮ
ಮಮತೆ
ಮಮ್ಮಲ
ಮಯ-ವಾ-ದುದು
ಮಯವೂ
ಮರ
ಮರಕ್ಕೆ
ಮರ-ಕ್ಷ-ಣ-ದಲ್ಲಿ
ಮರ-ಗಳ
ಮರ-ಗ-ಳಾ-ಚೆಗೆ
ಮರ-ಗ-ಳಿ-ದ್ದುವು
ಮರ-ಗಳು
ಮರಣ
ಮರ-ಣಕ್ಕೆ
ಮರ-ಣದ
ಮರ-ಣ-ಭ-ಯ-ವಿ-ಲ್ಲ-ವೆಂ-ಬು-ದನ್ನು
ಮರ-ಣ-ವೆಂದರೆ
ಮರದ
ಮರ-ದಡಿ
ಮರಳ
ಮರ-ಳ-ದಿ-ರು-ವಂತೆ
ಮರ-ಳನ್ನು
ಮರ-ಳನ್ನೇ
ಮರ-ಳ-ಬೇ-ಕಾ-ಗಿದೆ
ಮರ-ಳ-ಬೇ-ಕೆಂ-ದಿ-ದ್ದೇನೆ
ಮರ-ಳ-ಲಿ-ದ್ದಾ-ರೆಂದು
ಮರಳಿ
ಮರ-ಳಿ-ಕೊ-ಡ-ಬೇಕು
ಮರ-ಳಿ-ಕೊ-ಡಿ-ಸ-ಬೇಕು
ಮರ-ಳಿತು
ಮರ-ಳಿತ್ತು
ಮರ-ಳಿದ
ಮರ-ಳಿ-ದರು
ಮರ-ಳಿ-ದಾಗ
ಮರಳು
ಮರ-ಳು-ಗ-ಲ್ಲನ್ನು
ಮರ-ಳು-ಗಾ-ಡಿನ
ಮರ-ಳು-ಗಾ-ಡಿ-ನಲ್ಲಿ
ಮರ-ಳು-ತ್ತಿ-ದ್ದರು
ಮರ-ಳು-ತ್ತೇನೆ
ಮರ-ಳುವ
ಮರ-ಳು-ವಂತೆ
ಮರ-ಳು-ವು-ದನ್ನೇ
ಮರ-ಳು-ವು-ದ-ರಿಂ-ದೇನು
ಮರ-ವನ್ನು
ಮರ-ವಾಗಿ
ಮರ-ವಾ-ಗು-ವು-ದೆಂದು
ಮರವೂ
ಮರ-ವೊಂ-ದರ
ಮರಾ-ಠಾ-ಹಾಗೂ
ಮರಾ-ಠಿ-ಗರ
ಮರಾ-ಠಿ-ಗರು
ಮರಾಠೀ
ಮರೀಚಿ
ಮರೀ-ಚಿಕೆ
ಮರೀ-ಚಿ-ಕೆಗೆ
ಮರೀ-ಚಿ-ಕೆ-ಯನ್ನು
ಮರೀ-ಚಿ-ಕೆ-ಯೆಂಬ
ಮರು
ಮರು-ಕ-ಗೊಂಡು
ಮರು-ಕ-ಳಿ-ಸು-ತ್ತಿದ್ದ
ಮರು-ಕ-ಳಿ-ಸುವ
ಮರು-ಕು-ಳಿ-ಸ-ದಂತೆ
ಮರು-ಕ್ಷಣ
ಮರು-ಕ್ಷ-ಣ-ದಲ್ಲಿ
ಮರು-ಕ್ಷ-ಣ-ದಲ್ಲೇ
ಮರು-ಕ್ಷ-ಣವೇ
ಮರು-ಗದೆ
ಮರುಗಿ
ಮರು-ಗಿತು
ಮರು-ಗಿ-ದರು
ಮರು-ಗುತ್ತ
ಮರು-ಗುವ
ಮರು-ಗು-ವುದು
ಮರು-ತಂತಿ
ಮರು-ದಿನ
ಮರು-ದಿ-ನದ
ಮರು-ದಿ-ನ-ದಿಂ-ದಲೇ
ಮರು-ದಿ-ನವೇ
ಮರು-ಭೂ-ಮಿ-ಗ-ಳ-ಲ್ಲಾ-ದರೂ
ಮರು-ಭೂ-ಮಿ-ಗ-ಳ-ಲ್ಲಿ-ಅ-ದ-ರಲ್ಲೂ
ಮರು-ಭೂ-ಮಿಯ
ಮರು-ಭೂ-ಮಿ-ಯಲ್ಲಿ
ಮರು-ಮ-ರೀ-ಚಿ-ಕೆಗೆ
ಮರು-ಮ-ರೀ-ಚಿ-ಕೆಯ
ಮರು-ಮಾತ
ಮರು-ಮಾ-ತ-ನಾ-ಡದೆ
ಮರು-ಮಾ-ತಿ-ಲ್ಲದೆ
ಮರು-ಳಾ-ಗ-ಲಿಲ್ಲ
ಮರು-ಳಾಗಿ
ಮರು-ಳಾ-ದಂತೆ
ಮರು-ಳಾ-ದುದು
ಮರುಳು
ಮರು-ಳು-ಗೊ-ಳಿ-ಸಿ-ಬಿ-ಡು-ತ್ತಾರೋ
ಮರು-ಳು-ಗೊ-ಳಿ-ಸು-ತ್ತಿ-ದ್ದಾನೆ
ಮರು-ಳು-ಗೊ-ಳಿ-ಸುವ
ಮರು-ಳು-ಗೊ-ಳ್ಳ-ದಿ-ರು-ವಂ-ತಾ-ಗಲಿ
ಮರು-ವರ್ಷ
ಮರು-ವ-ರ್ಷದ
ಮರೆ
ಮರೆ-ತದ್ದು
ಮರೆ-ತ-ರೆಂ-ದಲ್ಲ
ಮರೆ-ತಿ-ರ-ಲಿಲ್ಲ
ಮರೆ-ತಿ-ರ-ಲಿ-ಲ್ಲ-ವೆಂ-ಬುದು
ಮರೆತು
ಮರೆ-ತು-ಬಿ-ಟ್ಟಿ-ದ್ದಾರೆ
ಮರೆ-ತು-ಬಿ-ಟ್ಟೆಯಾ
ಮರೆ-ತು-ಬಿ-ಡ-ಬೇಕು
ಮರೆ-ತು-ಬಿಡಿ
ಮರೆ-ತು-ಹೋ-ಗಿತ್ತು
ಮರೆ-ತು-ಹೋ-ಗಿದ್ದ
ಮರೆತೇ
ಮರೆ-ತೇ-ಹೋ-ಗು-ತ್ತದೆ
ಮರೆ-ಮಾಡಿ
ಮರೆ-ಯ-ದಿ-ರೋಣ
ಮರೆ-ಯ-ಲಾ-ಗದ
ಮರೆ-ಯ-ಲಾರೆ
ಮರೆ-ಯ-ಲಾ-ರೆವು
ಮರೆ-ಯ-ಲಿಲ್ಲ
ಮರೆ-ಯಾ-ಗಿ-ದ್ದುವು
ಮರೆ-ಯಾ-ಗಿ-ರಲು
ಮರೆ-ಯಾದ
ಮರೆ-ಯು-ತ್ತಿ-ರ-ಲಿಲ್ಲ
ಮರೆ-ಯು-ವಂ-ತಿಲ್ಲ
ಮರೆ-ಯು-ವಂ-ತಿ-ಲ್ಲ-ಯಾವ
ಮರೆ-ಯು-ವು-ದಿಲ್ಲ
ಮರ್-ಡಿ-ಗ್ಲೇಸ್
ಮರ್ತ್ಯ-ನ-ನ್ನು-ಅದೂ
ಮರ್ತ್ಯ-ವನ್ನು
ಮರ್ಮ
ಮರ್ಮ-ಗೋವಾ
ಮರ್ಮ-ಗೋ-ವಾಗೆ
ಮರ್ಮ-ಗೋ-ವಾ-ದಿಂದ
ಮರ್ಮ-ಭೇ-ದ-ಕ-ವಾದ
ಮರ್ಮ-ವನ್ನು
ಮರ್ಮಾ-ಘಾತ
ಮರ್ಯಾ-ದ-ಸ್ಥ-ರಾ-ಗಿ-ರಲಿ
ಮರ್ಯಾ-ದೆ-ಯಾಗಿ
ಮರ್ಯೋ
ಮರ್ವಿನ್
ಮಲ-ಗ-ಲಾ-ರದೆ
ಮಲ-ಗಲು
ಮಲಗಿ
ಮಲ-ಗಿ-ಕೊಂ-ಡರು
ಮಲ-ಗಿ-ಕೊಂಡು
ಮಲ-ಗಿ-ಕೊ-ಳ್ಳದ
ಮಲ-ಗಿ-ದರು
ಮಲ-ಗಿ-ದ-ವ-ರೆದ್ದು
ಮಲ-ಗಿದ್ದ
ಮಲ-ಗಿ-ದ್ದರು
ಮಲ-ಗಿ-ದ್ದಾಗ
ಮಲ-ಗಿ-ದ್ದಾರೆ
ಮಲ-ಗಿದ್ದೆ
ಮಲ-ಗಿ-ಬಿಟ್ಟೆ
ಮಲ-ಗಿ-ರುವ
ಮಲ-ಗಿ-ಸಿ-ಕೊಂಡು
ಮಲ-ಗುತ್ತ
ಮಲ-ಗು-ತ್ತಿ-ದ್ದುದು
ಮಲಿ-ನ-ರೂ-ಪದಿ
ಮಳೆ
ಮಳೆ-ಗ-ರೆ-ಯು-ತ್ತಿತ್ತು
ಮಳೆ-ಗ-ರೆ-ಯು-ತ್ತಿ-ದ್ದಾನೆ
ಮಳೆ-ಗಾಲ
ಮಳೆ-ಗಾ-ಲ-ದಲ್ಲಿ
ಮಳೆ-ಗಾ-ಲ-ವನ್ನು
ಮಳೆ-ಯಂ-ಗಿ-ಗಳು
ಮಳೆ-ಯಲ್ಲೇ
ಮಳೆಯೂ
ಮಸಕಾ
ಮಸಕು
ಮಸಾ-ಚು-ಸೆ-ಟ್ಸ್
ಮಸಾ-ಚು-ಸೆಟ್ಸ್ನ
ಮಸಾ-ಲೆ-ಗಳಿಂದ
ಮಸಿ
ಮಸಿ-ತುಂ-ಬಿ-ರುವ
ಮಸಿ-ಬ-ಳಿ-ಯುವ
ಮಸೀದಿ
ಮಸೀ-ದಿ-ಗೋ-ರಿ-ಗಳು
ಮಸು-ಕಾ-ಗ-ದಿ-ದ್ದರೆ
ಮಸು-ಕಿ-ನಿಂದ
ಮಸೂದೆ
ಮಸೂ-ದೆ-ಯನ್ನು
ಮಸೂ-ದೆ-ಯೊಂದು
ಮಸ್
ಮಸ್ತಾದ
ಮಸ್ದನೊ
ಮಸ್ಸಾ-ಚು-ಸೆ-ಟ್ಸ್
ಮಸ್ಸಿಗೆ
ಮಹಂ
ಮಹಂ-ತರು
ಮಹಡಿ
ಮಹ-ಡಿ-ಗಳ
ಮಹ-ಡಿ-ಗಳು
ಮಹ-ಡಿಯ
ಮಹ-ಡಿ-ಯಲ್ಲಿ
ಮಹತ್
ಮಹ-ತ್ಕಾ-ರ್ಯ-ಗಳನ್ನು
ಮಹ-ತ್ಕಾ-ರ್ಯ-ಗಳು
ಮಹ-ತ್ಕಾ-ರ್ಯ-ದಲ್ಲಿ
ಮಹ-ತ್ತರ
ಮಹ-ತ್ತ-ರ-ವಾ-ದದ್ದೇ
ಮಹ-ತ್ತಾದ
ಮಹ-ತ್ತ್ವದ
ಮಹ-ತ್ತ್ವ-ಪೂ-ರ್ಣ-ವಾ-ದದ್ದು
ಮಹ-ತ್ಪ-ರಿ-ಣಾ-ಮ-ಕಾರಿ
ಮಹತ್ವ
ಮಹ-ತ್ವ-ಇ-ವು-ಗಳನ್ನು
ಮಹ-ತ್ವದ
ಮಹ-ತ್ವ-ದ್ದಾ-ಗಿದ್ದು
ಮಹ-ತ್ವ-ಪೂರ್ಣ
ಮಹ-ತ್ವ-ಪೂ-ರ್ಣ-ವಾದು
ಮಹ-ತ್ವ-ಪೂ-ರ್ಣ-ವಾ-ದುದು
ಮಹ-ತ್ವ-ವನ್ನು
ಮಹ-ತ್ವ-ವನ್ನೂ
ಮಹ-ತ್ವವು
ಮಹ-ತ್ವ-ವೆಂ-ಥದು
ಮಹ-ತ್ವಾ-ಕಾಂಕ್ಷೆ
ಮಹ-ದಾ-ಕಾಂ-ಕ್ಷೆಯ
ಮಹ-ದಾಸೆ
ಮಹ-ದು-ಪ-ಕಾ-ರ-ವನ್ನು
ಮಹ-ದ್ಭಾ-ವ-ನೆ-ಗಳು
ಮಹ-ನೀ-ಯ-ರನ್ನು
ಮಹ-ನೀ-ಯರು
ಮಹ-ನೀ-ಯರೂ
ಮಹ-ನೀ-ಯರೇ
ಮಹ-ಮ-ದನ
ಮಹ-ಮ-ದನೇ
ಮಹ-ಮ-ದೀ-ಯ-ವಾ-ಗಲಿ
ಮಹ-ಮ್ಮ-ದನ
ಮಹ-ಮ್ಮ-ದ-ನನ್ನು
ಮಹ-ಮ್ಮ-ದನೇ
ಮಹ-ರ್ಷಿ-ಗಳ
ಮಹ-ರ್ಷಿ-ಗಳಿಂದ
ಮಹ-ರ್ಷಿ-ಗಳು
ಮಹ-ರ್ಷಿಯ
ಮಹ-ರ್ಷಿ-ಯಂತೆ
ಮಹ-ರ್ಷಿಯು
ಮಹ-ರ್ಷಿಯೂ
ಮಹ-ಲು-ಗಳು
ಮಹಾ
ಮಹಾಂತ
ಮಹಾಂತಂ
ಮಹಾ-ಕ-ಲಾ-ವಿದ
ಮಹಾ-ಕ-ಳಂಕ
ಮಹಾ-ಕಾರ್ಯ
ಮಹಾ-ಕಾ-ರ್ಯ-ಕ್ಕಾಗಿ
ಮಹಾ-ಕಾ-ರ್ಯಕ್ಕೆ
ಮಹಾ-ಕಾ-ರ್ಯ-ಗಳನ್ನೂ
ಮಹಾ-ಕಾ-ರ್ಯ-ಗ-ಳಿ-ಗಾಗಿ
ಮಹಾ-ಕಾ-ರ್ಯ-ಗಳೂ
ಮಹಾ-ಕಾ-ರ್ಯದ
ಮಹಾ-ಕಾ-ರ್ಯ-ದಲ್ಲಿ
ಮಹಾ-ಕಾ-ರ್ಯ-ವನ್ನು
ಮಹಾ-ಕಾ-ರ್ಯ-ವೆಂದರೆ
ಮಹಾ-ಕಾ-ರ್ಯ-ವೊಂ-ದಿದೆ
ಮಹಾ-ಗು-ರು-ವಾದ
ಮಹಾ-ಗು-ರು-ವಿ-ನಂತೆ
ಮಹಾ-ಗ್ರಂ-ಥವು
ಮಹಾ-ಚ-ಕ್ರದ
ಮಹಾ-ಚಿಂ-ತೆಯ
ಮಹಾ-ಜ-ನ-ತೆಗೆ
ಮಹಾ-ಜ್ಞಾ-ನಿ-ಗಳ
ಮಹಾ-ಜ್ಞಾ-ನಿ-ಗ-ಳಾದ
ಮಹಾ-ಜ್ಞಾ-ನಿ-ಯ-ಎಂ-ದರೆ
ಮಹಾ-ಜ್ಞಾ-ನಿಯು
ಮಹಾ-ಜ್ಯೋ-ತಿ-ಯನ್ನು
ಮಹಾ-ತ-ತ್ತ್ವ-ಆ-ದ-ರ್ಶ-ಗಳು
ಮಹಾ-ತ-ತ್ತ್ವ-ಗಳ
ಮಹಾ-ತ-ಪ-ಸ್ವಿ-ಗಳು
ಮಹಾ-ತೃಪ್ತಿ
ಮಹಾತ್ಮ
ಮಹಾ-ತ್ಮ-ಸಂ-ತ-ತ-ತ್ವ-ಜ್ಞಾನಿ
ಮಹಾ-ತ್ಮ-ನನ್ನು
ಮಹಾ-ತ್ಮ-ನೊಬ್ಬ
ಮಹಾ-ತ್ಮ-ನೊ-ಬ್ಬನ
ಮಹಾ-ತ್ಮರ
ಮಹಾ-ತ್ಮ-ರನ್ನು
ಮಹಾ-ತ್ಮ-ರಾ-ಗಿ-ದ್ದಲ್ಲಿ
ಮಹಾ-ತ್ಮರು
ಮಹಾ-ತ್ಮರೂ
ಮಹಾತ್ಮ್ಯ
ಮಹಾ-ತ್ಯಾ-ಗವೂ
ಮಹಾ-ದೇ-ವ-ರಾವ್
ಮಹಾ-ದೇ-ವಾ-ಲ-ಯದ
ಮಹಾ-ದ್ಭುತ
ಮಹಾ-ನ-ಗ-ರದ
ಮಹಾ-ನ-ಗ-ರ-ದಂ-ತೆ-ಅ-ದ-ರಲ್ಲಿ
ಮಹಾ-ನ-ಗ-ರ-ದಲ್ಲಿ
ಮಹಾ-ನು-ಭಾ-ವ-ನಂತೆ
ಮಹಾನ್
ಮಹಾ-ಪಾಪ
ಮಹಾ-ಪಾ-ಪ-ವೆಂ-ದ-ಮೇಲೆ
ಮಹಾ-ಪು-ರು-ಷನ
ಮಹಾ-ಪು-ರು-ಷ-ನೆಂದು
ಮಹಾ-ಪು-ರು-ಷ-ನೆಂದೋ
ಮಹಾ-ಪು-ರು-ಷನೇ
ಮಹಾ-ಪು-ರು-ಷರ
ಮಹಾ-ಪು-ರು-ಷ-ರಿಗೆ
ಮಹಾ-ಪು-ರು-ಷ-ರಿ-ರ-ಬೇಕು
ಮಹಾ-ಪು-ರು-ಷರು
ಮಹಾ-ಪೂರ
ಮಹಾ-ಪೂ-ರ-ವಾಗಿ
ಮಹಾ-ಪ್ರಭು
ಮಹಾ-ಪ್ರ-ವಾ-ದಿ-ಯಾಗಿ
ಮಹಾ-ಬಂ-ಡೆ-ಗಳು
ಮಹಾ-ಬ-ಲೇ-ಶ್ವ-ರಕ್ಕೆ
ಮಹಾ-ಬ-ಲೇ-ಶ್ವ-ರ-ದಲ್ಲಿ
ಮಹಾ-ಬೋಧಿ
ಮಹಾ-ಭಕ್ತ
ಮಹಾ-ಭಾ-ಷ್ಯ-ವನ್ನು
ಮಹಾ-ಭಾ-ಷ್ಯವು
ಮಹಾ-ಮ-ಹಿಮ
ಮಹಾ-ಮ-ಹಿ-ಮ-ನಿಗೆ
ಮಹಾ-ಮ-ಹಿ-ಮರು
ಮಹಾ-ಮ-ಹಿ-ಮರೂ
ಮಹಾ-ಮ-ಹಿ-ಮೆ-ಯಿ-ರು-ವುದು
ಮಹಾ-ಮಾ-ನ-ವ-ರ-ವ-ರೆಗೆ
ಮಹಾ-ಮು-ಹೂ-ರ್ತ-ದಲ್ಲೇ
ಮಹಾ-ಯೋ-ಜ-ನೆ-ಗಳನ್ನು
ಮಹಾ-ರ-ಣ್ಯ-ದಲ್ಲಿ
ಮಹಾ-ರಾಜ
ಮಹಾ-ರಾ-ಜನ
ಮಹಾ-ರಾ-ಜ-ನಂತೆ
ಮಹಾ-ರಾ-ಜ-ನನ್ನು
ಮಹಾ-ರಾ-ಜ-ನನ್ನೂ
ಮಹಾ-ರಾ-ಜ-ನಾದ
ಮಹಾ-ರಾ-ಜ-ನಿಗೆ
ಮಹಾ-ರಾ-ಜನು
ಮಹಾ-ರಾ-ಜನೂ
ಮಹಾ-ರಾ-ಜ-ನೆಂದೇ
ಮಹಾ-ರಾ-ಜ-ನೊಂ-ದಿಗೆ
ಮಹಾ-ರಾ-ಜರ
ಮಹಾ-ರಾ-ಜ-ರಂತೆ
ಮಹಾ-ರಾ-ಜ-ರಂ-ತೆಯೇ
ಮಹಾ-ರಾ-ಜ-ರದು
ಮಹಾ-ರಾ-ಜ-ರನ್ನು
ಮಹಾ-ರಾ-ಜ-ರನ್ನೂ
ಮಹಾ-ರಾ-ಜ-ರಲ್ಲಿ
ಮಹಾ-ರಾ-ಜ-ರಾದ
ಮಹಾ-ರಾ-ಜ-ರಿಂದ
ಮಹಾ-ರಾ-ಜ-ರಿ-ಗೀಗ
ಮಹಾ-ರಾ-ಜ-ರಿಗೂ
ಮಹಾ-ರಾ-ಜ-ರಿಗೆ
ಮಹಾ-ರಾ-ಜ-ರಿ-ಗೇ-ನಾ-ದರೂ
ಮಹಾ-ರಾ-ಜರು
ಮಹಾ-ರಾ-ಜ-ರೇನೂ
ಮಹಾ-ರಾ-ಜ-ರೊಂ-ದಿಗೂ
ಮಹಾ-ರಾಜಾ
ಮಹಾ-ರಾಜ್
ಮಹಾ-ರಾಣಿ
ಮಹಾ-ರಾಯ
ಮಹಾ-ರಾ-ಯನ
ಮಹಾ-ರಾ-ಷ್ಟ್ರಕ್ಕೆ
ಮಹಾ-ವಾ-ಗ್ಮಿ-ಯಾಗಿ
ಮಹಾ-ವಿ-ದ್ವಾಂ-ಸ-ರೆಂದು
ಮಹಾ-ವೃ-ಕ್ಷದ
ಮಹಾ-ವ್ರ-ತ-ಗ-ಳಾದ
ಮಹಾ-ಶಕ್ತಿ
ಮಹಾ-ಶ-ಕ್ತಿಯು
ಮಹಾ-ಶ-ಕ್ತಿಯೂ
ಮಹಾ-ಶ-ಕ್ತಿಯೇ
ಮಹಾ-ಶ-ಕ್ತಿ-ಶಾಲಿ
ಮಹಾ-ಶಯ
ಮಹಾ-ಶ-ಯ-ನೊಬ್ಬ
ಮಹಾ-ಶ-ಯರು
ಮಹಾ-ಶ-ಯರೆ
ಮಹಾ-ಶ-ಯರೇ
ಮಹಾ-ಶ-ಯ-ರೊ-ಬ್ಬರು
ಮಹಾ-ಶಿ-ಖರ
ಮಹಾ-ಸಂ-ಘ-ವನ್ನು
ಮಹಾ-ಸಂ-ನ್ಯಾ-ಸಿಯ
ಮಹಾ-ಸತ್ಯ
ಮಹಾ-ಸ-ಭಿ-ಕ-ರಿಗೆ
ಮಹಾ-ಸ-ಭೆ-ಯಲ್ಲಿ
ಮಹಾ-ಸ-ಮಾ-ಧಿಯ
ಮಹಾ-ಸಾ-ಗರ
ಮಹಾ-ಸಾ-ಧನೆ
ಮಹಾ-ಸಾಧು
ಮಹಾ-ಸಾ-ಧು-ಗಳು
ಮಹಾ-ಸಾ-ಧು-ಗ-ಳೊ-ಬ್ಬರು
ಮಹಾ-ಸ್ವಾಮಿ
ಮಹಿ-ಮಾ-ನ್ವಿ-ತರೇ
ಮಹಿ-ಮಾ-ನ್ವಿ-ತ-ವಾದ
ಮಹಿಮೆ
ಮಹಿ-ಮೆ-ಯನ್ನು
ಮಹಿ-ಮೆ-ಯು-ಳ್ಳದ್ದೂ
ಮಹಿಳಾ
ಮಹಿ-ಳಾ-ಮ-ಣಿ-ಯರ
ಮಹಿ-ಳಾ-ಸಂ-ಘ-ವೊಂ-ದ-ರಲ್ಲಿ
ಮಹಿಳೆ
ಮಹಿ-ಳೆಗೆ
ಮಹಿ-ಳೆಯ
ಮಹಿ-ಳೆ-ಯರ
ಮಹಿ-ಳೆ-ಯ-ರಿಗೂ
ಮಹಿ-ಳೆ-ಯ-ರಿಗೆ
ಮಹಿ-ಳೆ-ಯರು
ಮಹಿ-ಳೆ-ಯರೂ
ಮಹಿ-ಳೆ-ಯಾ-ಗಿ-ದ್ದಳು
ಮಹಿ-ಳೆಯೂ
ಮಹಿ-ಳೆ-ಯೊ-ಬ್ಬಳು
ಮಹೇಂ-ದ್ರ-ನಾಥ
ಮಹೇಂ-ದ್ರ-ನಾ-ಥ-ಇ-ವರೆಲ್ಲ
ಮಹೇಂ-ದ್ರ-ನಾ-ಥ-ದ-ತ್ತ-ನಿಗೆ
ಮಹೇಂ-ದ್ರ-ನಾ-ಥನ
ಮಹೇಂ-ದ್ರ-ನಾ-ಥ-ನನ್ನು
ಮಹೇಂ-ದ್ರ-ನಾ-ಥನೂ
ಮಹೇ-ಶ-ಚಂದ್ರ
ಮಹೇ-ಶ್ವರ
ಮಹೇ-ಶ್ವ-ರ-ನಂತೆ
ಮಹೇ-ಶ್ವ-ರ-ನಷ್ಟೇ
ಮಹೇ-ಶ್ವ-ರ-ನಿ-ಗೆ-ನೀ-ಲ-ಕಂಠ
ಮಹೇ-ಶ್ವ-ರಾದಿ
ಮಹೋ-ದ್ದೇಶ
ಮಹೋ-ದ್ದೇ-ಶ-ವನ್ನು
ಮಹೋ-ದ್ದೇ-ಶವು
ಮಹೋ-ನ್ನತ
ಮಾ
ಮಾಂ
ಮಾಂಟೆ-ರೋ-ಸಾದ
ಮಾಂಡ-ವಿ-ಗ-ಳಿಗೆ
ಮಾಂಡ-ವಿಗೆ
ಮಾಂಡ-ವಿ-ಯನ್ನು
ಮಾಂಡ-ವಿ-ಯಲ್ಲೇ
ಮಾಂಡ-ವಿ-ಯಿಂದ
ಮಾಂಸ
ಮಾಂಸ-ಖಂ-ಡ-ಗಳು
ಮಾಂಸ-ವನ್ನು
ಮಾಕ್ಸ್
ಮಾಗಿಯ
ಮಾಚದೆ
ಮಾಟ-ಮಂ-ತ್ರದ
ಮಾಟ-ಗಾ-ತಿಯ
ಮಾಟ-ಗಾ-ತಿ-ಯ-ರನ್ನು
ಮಾಟ-ಗಾ-ತಿ-ಯ-ರ-ನ್ನು-ಅ-ಥವಾ
ಮಾಡ
ಮಾಡ-ತೊ-ಡ-ಗಿದೆ
ಮಾಡದ
ಮಾಡ-ದಿದ್ದ
ಮಾಡ-ದಿ-ದ್ದರೂ
ಮಾಡ-ದಿ-ದ್ದರೆ
ಮಾಡ-ದಿ-ರು-ವಂತೆ
ಮಾಡದೆ
ಮಾಡ-ದೆ-ಹೋ-ದರೆ
ಮಾಡ-ಬಂ-ದ-ವ-ರನ್ನೇ
ಮಾಡ-ಬ-ಯ-ಸು-ವ-ವರು
ಮಾಡ-ಬಲ್ಲ
ಮಾಡ-ಬ-ಲ್ಲ-ನೆಂ-ಬುದು
ಮಾಡ-ಬ-ಲ್ಲರು
ಮಾಡ-ಬ-ಲ್ಲರೆ
ಮಾಡ-ಬ-ಲ್ಲ-ವರು
ಮಾಡ-ಬ-ಲ್ಲಿರಾ
ಮಾಡ-ಬ-ಲ್ಲು-ದಾ-ಗಿತ್ತು
ಮಾಡ-ಬ-ಲ್ಲು-ದಾ-ದರೆ
ಮಾಡ-ಬಲ್ಲೆ
ಮಾಡ-ಬಹು
ಮಾಡ-ಬ-ಹು-ದಾದ
ಮಾಡ-ಬ-ಹುದು
ಮಾಡ-ಬಾ-ರದು
ಮಾಡ-ಬಾ-ರ-ದೆಂದು
ಮಾಡ-ಬೇ-ಕಾ-ಗಿತ್ತು
ಮಾಡ-ಬೇ-ಕಾ-ಗಿದೆ
ಮಾಡ-ಬೇ-ಕಾ-ಗಿ-ದೆಯೋ
ಮಾಡ-ಬೇ-ಕಾ-ಗು-ತ್ತದೆ
ಮಾಡ-ಬೇ-ಕಾ-ಗು-ತ್ತಿತ್ತು
ಮಾಡ-ಬೇ-ಕಾದ
ಮಾಡ-ಬೇ-ಕಾ-ದಂ-ತಹ
ಮಾಡ-ಬೇ-ಕಾ-ದದ್ದು
ಮಾಡ-ಬೇ-ಕಾ-ದರೆ
ಮಾಡ-ಬೇ-ಕಾ-ದ-ವನು
ಮಾಡ-ಬೇ-ಕಾ-ದು-ದರ
ಮಾಡ-ಬೇ-ಕಾ-ಯಿತು
ಮಾಡ-ಬೇಕು
ಮಾಡ-ಬೇ-ಕೆಂ-ದಿ-ದ್ದೇನೆ
ಮಾಡ-ಬೇ-ಕೆಂದು
ಮಾಡ-ಬೇ-ಕೆಂಬ
ಮಾಡ-ಬೇ-ಕೆಂಬು
ಮಾಡ-ಬೇ-ಕೆಂ-ಬು-ದನ್ನು
ಮಾಡ-ಬೇ-ಕೆಂ-ಬುದು
ಮಾಡ-ಬೇ-ಕೆ-ನ್ನು-ವ-ವ-ರಿಗೆ
ಮಾಡ-ಲಾ-ಗಿತ್ತು
ಮಾಡ-ಲಾಗಿದೆ
ಮಾಡ-ಲಾ-ಗು-ವಂತೆ
ಮಾಡ-ಲಾದ
ಮಾಡ-ಲಾ-ದಂ-ತಹ
ಮಾಡ-ಲಾ-ದೀತು
ಮಾಡ-ಲಾ-ಯಿತು
ಮಾಡ-ಲಾರ
ಮಾಡ-ಲಾ-ರಂ-ಭಿ-ಸಿ-ದರು
ಮಾಡ-ಲಾರೆ
ಮಾಡಲಿ
ಮಾಡ-ಲಿ-ದ್ದಾಳೆ
ಮಾಡ-ಲಿ-ದ್ದೀರಿ
ಮಾಡ-ಲಿ-ನಾ-ನ-ದನ್ನು
ಮಾಡ-ಲಿ-ರು-ವ-ರೆಂದು
ಮಾಡ-ಲಿಲ್ಲ
ಮಾಡ-ಲಿ-ಲ್ಲವೆ
ಮಾಡ-ಲಿ-ಲ್ಲ-ವೇಕೆ
ಮಾಡಲು
ಮಾಡಲೂ
ಮಾಡ-ಲೆಂದೇ
ಮಾಡ-ಲೇ-ಬಾ-ರದು
ಮಾಡ-ಲೇ-ಬೇ-ಕಿತ್ತು
ಮಾಡ-ಲೇ-ಬೇಕು
ಮಾಡ-ಲ್ಪಟ್ಟ
ಮಾಡ-ಲ್ಪ-ಟ್ಟ-ವು-ಗಳು
ಮಾಡ-ಲ್ಪ-ಟ್ಟಿ-ರು-ವುದು
ಮಾಡಿ
ಮಾಡಿ-ಕೊಂಡ
ಮಾಡಿ-ಕೊಂ-ಡ-ದ್ದಲ್ಲ
ಮಾಡಿ-ಕೊಂ-ಡರು
ಮಾಡಿ-ಕೊಂ-ಡರೆ
ಮಾಡಿ-ಕೊಂ-ಡ-ವನು
ಮಾಡಿ-ಕೊಂ-ಡ-ವ-ರಿಗೆ
ಮಾಡಿ-ಕೊಂ-ಡ-ವರು
ಮಾಡಿ-ಕೊಂ-ಡಾಗ
ಮಾಡಿ-ಕೊಂಡಿ
ಮಾಡಿ-ಕೊಂ-ಡಿದ್ದ
ಮಾಡಿ-ಕೊಂ-ಡಿ-ದ್ದರು
ಮಾಡಿ-ಕೊಂ-ಡಿ-ದ್ದೇವೆ
ಮಾಡಿ-ಕೊಂ-ಡಿರು
ಮಾಡಿ-ಕೊಂ-ಡಿ-ರು-ವ-ವರು
ಮಾಡಿ-ಕೊಂ-ಡಿ-ಲ್ಲ-ವಲ್ಲ
ಮಾಡಿ-ಕೊಂಡು
ಮಾಡಿ-ಕೊಟ್ಟ
ಮಾಡಿ-ಕೊ-ಟ್ಟರು
ಮಾಡಿ-ಕೊ-ಟ್ಟಿ-ರು-ವಾಗ
ಮಾಡಿ-ಕೊಟ್ಟು
ಮಾಡಿ-ಕೊ-ಡ-ಬೇ-ಕೆಂ-ದಿ-ದ್ದಾಳೆ
ಮಾಡಿ-ಕೊ-ಡ-ಲಾ-ಯಿತು
ಮಾಡಿ-ಕೊ-ಡಲು
ಮಾಡಿ-ಕೊ-ಡು-ತ್ತವೆ
ಮಾಡಿ-ಕೊ-ಡು-ತ್ತಾರೆ
ಮಾಡಿ-ಕೊ-ಡು-ತ್ತಿ-ದ್ದುದೂ
ಮಾಡಿ-ಕೊ-ಡು-ತ್ತಿ-ರು-ವಂ-ತಿತ್ತು
ಮಾಡಿ-ಕೊ-ಡು-ವಂತೆ
ಮಾಡಿ-ಕೊ-ಡು-ವ-ವ-ರೆಗೆ
ಮಾಡಿ-ಕೊ-ಡು-ವು-ದ-ಕ್ಕಾಗಿ
ಮಾಡಿ-ಕೊ-ಡು-ವುದು
ಮಾಡಿ-ಕೊಳ್ಳ
ಮಾಡಿ-ಕೊ-ಳ್ಳ-ತೊ-ಡ-ಗಿ-ದರು
ಮಾಡಿ-ಕೊ-ಳ್ಳ-ದಷ್ಟು
ಮಾಡಿ-ಕೊ-ಳ್ಳ-ದಿ-ದ್ದರೆ
ಮಾಡಿ-ಕೊ-ಳ್ಳ-ಬ-ಹುದು
ಮಾಡಿ-ಕೊ-ಳ್ಳ-ಬೇ-ಕಾ-ಗಿದೆ
ಮಾಡಿ-ಕೊ-ಳ್ಳ-ಬೇ-ಕಾ-ದರೆ
ಮಾಡಿ-ಕೊ-ಳ್ಳ-ಬೇಕು
ಮಾಡಿ-ಕೊ-ಳ್ಳ-ಬೇಡಿ
ಮಾಡಿ-ಕೊ-ಳ್ಳ-ಲಾರೆ
ಮಾಡಿ-ಕೊ-ಳ್ಳ-ಲಿಲ್ಲ
ಮಾಡಿ-ಕೊ-ಳ್ಳಲು
ಮಾಡಿ-ಕೊ-ಳ್ಳ-ಲೇ-ಬೇ-ಕಾ-ಗಿತ್ತು
ಮಾಡಿ-ಕೊ-ಳ್ಳಲೋ
ಮಾಡಿ-ಕೊ-ಳ್ಳು-ತ್ತಿ-ದ್ದರು
ಮಾಡಿ-ಕೊ-ಳ್ಳು-ತ್ತಿ-ರ-ಲಿಲ್ಲ
ಮಾಡಿ-ಕೊ-ಳ್ಳು-ತ್ತೇನೆ
ಮಾಡಿ-ಕೊ-ಳ್ಳುವ
ಮಾಡಿ-ಕೊ-ಳ್ಳು-ವು-ದಕ್ಕೆ
ಮಾಡಿ-ಕೊ-ಳ್ಳು-ವು-ದಾ-ದರೂ
ಮಾಡಿ-ಕೊ-ಳ್ಳು-ವು-ದಿಲ್ಲ
ಮಾಡಿ-ಕೊ-ಳ್ಳು-ವು-ದೆಂ-ದರೆ
ಮಾಡಿ-ತಲ್ಲ
ಮಾಡಿತು
ಮಾಡಿತ್ತು
ಮಾಡಿದ
ಮಾಡಿ-ದಂ-ತಾ-ಗು-ತ್ತದೆ
ಮಾಡಿ-ದಂತೆ
ಮಾಡಿ-ದನು
ಮಾಡಿ-ದನೊ
ಮಾಡಿ-ದ-ರ-ಲ್ಲದೆ
ಮಾಡಿ-ದ-ರಷ್ಟೇ
ಮಾಡಿ-ದ-ರಾ-ದರೂ
ಮಾಡಿ-ದರು
ಮಾಡಿ-ದ-ರು-ಈ-ತ-ನನ್ನು
ಮಾಡಿ-ದ-ರು-ಒಮ್ಮೆ
ಮಾಡಿ-ದ-ರು-ಬ-ಹುಶಃ
ಮಾಡಿ-ದರೂ
ಮಾಡಿ-ದರೆ
ಮಾಡಿ-ದ-ರೆಂದು
ಮಾಡಿ-ದಲ್ಲಿ
ಮಾಡಿ-ದಳು
ಮಾಡಿ-ದ-ವ-ನಲ್ಲ
ಮಾಡಿ-ದ-ವನು
ಮಾಡಿ-ದ-ವನೇ
ಮಾಡಿ-ದ-ವರ
ಮಾಡಿ-ದ-ವ-ರಂತೆ
ಮಾಡಿ-ದ-ವ-ರಿಗೆ
ಮಾಡಿ-ದ-ವರು
ಮಾಡಿ-ದ-ವ-ರೆಂ-ದರೆ
ಮಾಡಿ-ದ-ವ-ರ್ಯಾರು
ಮಾಡಿ-ದ-ವಳು
ಮಾಡಿ-ದ-ವು-ಗ-ಳ-ಲ್ಲೆಲ್ಲ
ಮಾಡಿ-ದಷ್ಟು
ಮಾಡಿ-ದಾಗ
ಮಾಡಿ-ದಾ-ಗ-ಲಂತೂ
ಮಾಡಿ-ದಿರಿ
ಮಾಡಿ-ದು-ದನ್ನು
ಮಾಡಿ-ದುದು
ಮಾಡಿ-ದುದೂ
ಮಾಡಿ-ದು-ದೆಂ-ಬುದು
ಮಾಡಿ-ದುವು
ಮಾಡಿದೆ
ಮಾಡಿ-ದೆ-ಅ-ಸ್ಸಿ-ಸಿಯ
ಮಾಡಿ-ದೆ-ನಲ್ಲ
ಮಾಡಿ-ದೆನೊ
ಮಾಡಿ-ದೆನೋ
ಮಾಡಿದ್ದ
ಮಾಡಿ-ದ್ದಂ-ತೆಯೇ
ಮಾಡಿ-ದ್ದ-ಕ್ಕಿಂತ
ಮಾಡಿ-ದ್ದಕ್ಕೆ
ಮಾಡಿ-ದ್ದನ್ನು
ಮಾಡಿ-ದ್ದ-ರಂತೆ
ಮಾಡಿ-ದ್ದರು
ಮಾಡಿ-ದ್ದರೂ
ಮಾಡಿ-ದ್ದರೆ
ಮಾಡಿ-ದ್ದರೋ
ಮಾಡಿ-ದ್ದಲ್ಲಿ
ಮಾಡಿ-ದ್ದ-ವರು
ಮಾಡಿ-ದ್ದಾ-ಗಿದೆ
ಮಾಡಿ-ದ್ದಾನೆ
ಮಾಡಿ-ದ್ದಾರೆ
ಮಾಡಿ-ದ್ದಾ-ರೆಂದು
ಮಾಡಿ-ದ್ದೀಯೆ
ಮಾಡಿ-ದ್ದೀರಿ
ಮಾಡಿದ್ದು
ಮಾಡಿದ್ದೂ
ಮಾಡಿದ್ದೆ
ಮಾಡಿ-ದ್ದೆಲ್ಲಿ
ಮಾಡಿದ್ದೇ
ಮಾಡಿ-ದ್ದೇನೆ
ಮಾಡಿ-ದ್ದೊಂದು
ಮಾಡಿ-ಬ-ರು-ವಂತೆ
ಮಾಡಿ-ಬಿ-ಟ್ಟರು
ಮಾಡಿ-ಬಿ-ಟ್ಟಿತು
ಮಾಡಿ-ಬಿ-ಟ್ಟಿ-ದ್ದರು
ಮಾಡಿ-ಬಿಟ್ಟೆ
ಮಾಡಿ-ಬಿ-ಡ-ಬಲ್ಲೆ
ಮಾಡಿ-ಬಿ-ಡ-ಬ-ಹುದು
ಮಾಡಿ-ಬಿ-ಡ-ಬೇ-ಕೆಂದು
ಮಾಡಿ-ಬಿಡಿ
ಮಾಡಿ-ಬಿ-ಡು-ತ್ತಿ-ದ್ದರು
ಮಾಡಿ-ಬಿ-ಡು-ತ್ತೇನೆ
ಮಾಡಿ-ಬಿ-ಡುವ
ಮಾಡಿ-ಬಿ-ಡ್ತೀಯ
ಮಾಡಿ-ಯಾನು
ಮಾಡಿ-ರ-ಬಹು
ಮಾಡಿ-ರ-ಬ-ಹುದು
ಮಾಡಿ-ರ-ಬೇ-ಕೆಂದು
ಮಾಡಿ-ರ-ಲಿಲ್ಲ
ಮಾಡಿ-ರುವ
ಮಾಡಿ-ರು-ವುದು
ಮಾಡಿ-ರು-ವು-ದೇಕೆ
ಮಾಡಿಲ್ಲ
ಮಾಡಿ-ಲ್ಲ-ವಲ್ಲ
ಮಾಡಿವೆ
ಮಾಡಿಸ
ಮಾಡಿ-ಸಲು
ಮಾಡಿಸಿ
ಮಾಡಿ-ಸಿ-ಕೊಟ್ಟ
ಮಾಡಿ-ಸಿ-ಕೊ-ಟ್ಟ-ದ್ದನ್ನು
ಮಾಡಿ-ಸಿ-ಕೊ-ಟ್ಟಿ-ದ್ದರು
ಮಾಡಿ-ಸಿ-ಕೊಟ್ಟು
ಮಾಡಿ-ಸಿ-ಕೊಡಿ
ಮಾಡಿ-ಸಿ-ಕೊ-ಳ್ಳ-ಬೇಕು
ಮಾಡಿ-ಸಿ-ಕೊ-ಳ್ಳು-ತ್ತಿತ್ತು
ಮಾಡಿ-ಸಿದ
ಮಾಡಿ-ಸಿ-ದಂ-ತಹ
ಮಾಡಿ-ಸಿ-ದಂ-ತಿ-ದ್ದುವು
ಮಾಡಿ-ಸಿ-ದರು
ಮಾಡಿ-ಸಿ-ದರೆ
ಮಾಡಿ-ಸಿ-ದುವು
ಮಾಡಿ-ಸುತ್ತ
ಮಾಡಿ-ಸು-ತ್ತಾರೆ
ಮಾಡಿ-ಸು-ತ್ತಿ-ದ್ದರು
ಮಾಡಿ-ಸು-ತ್ತಿ-ದ್ದಾನೆ
ಮಾಡಿ-ಸು-ತ್ತಿ-ದ್ದಾಳೋ
ಮಾಡಿ-ಸು-ತ್ತಿ-ರುವ
ಮಾಡಿ-ಸು-ತ್ತಿ-ರು-ವ-ವರು
ಮಾಡಿ-ಸು-ತ್ತೇನೆ
ಮಾಡಿ-ಸು-ವು-ದ-ಕ್ಕಾಗಿ
ಮಾಡಿ-ಸು-ವು-ದಾಗಿ
ಮಾಡು
ಮಾಡುತ್ತ
ಮಾಡು-ತ್ತದೆ
ಮಾಡು-ತ್ತ-ದೆಂ-ಬು-ದನ್ನು
ಮಾಡು-ತ್ತಲೇ
ಮಾಡು-ತ್ತವೆ
ಮಾಡುತ್ತಾ
ಮಾಡು-ತ್ತಾನೆ
ಮಾಡು-ತ್ತಾ-ರಲ್ಲ
ಮಾಡು-ತ್ತಾರೆ
ಮಾಡು-ತ್ತಾ-ರೆಯೆ
ಮಾಡು-ತ್ತಾ-ರೆಯೋ
ಮಾಡುತ್ತಿ
ಮಾಡು-ತ್ತಿತ್ತು
ಮಾಡು-ತ್ತಿ-ತ್ತೆಂ-ಬು-ದನ್ನು
ಮಾಡು-ತ್ತಿದ್ದ
ಮಾಡು-ತ್ತಿ-ದ್ದಂತೆ
ಮಾಡು-ತ್ತಿ-ದ್ದರು
ಮಾಡು-ತ್ತಿ-ದ್ದರೂ
ಮಾಡು-ತ್ತಿ-ದ್ದರೆ
ಮಾಡು-ತ್ತಿ-ದ್ದರೋ
ಮಾಡು-ತ್ತಿ-ದ್ದಳು
ಮಾಡು-ತ್ತಿ-ದ್ದಾಗ
ಮಾಡು-ತ್ತಿ-ದ್ದಾ-ನಲ್ಲ
ಮಾಡು-ತ್ತಿ-ದ್ದಾನೆ
ಮಾಡು-ತ್ತಿ-ದ್ದಾರೆ
ಮಾಡು-ತ್ತಿ-ದ್ದಾ-ರೆಂ-ಬು-ದನ್ನು
ಮಾಡು-ತ್ತಿ-ದ್ದಾಳೆ
ಮಾಡು-ತ್ತಿ-ದ್ದೀಯಾ
ಮಾಡು-ತ್ತಿ-ದ್ದೀಯೆ
ಮಾಡು-ತ್ತಿ-ದ್ದೀರಿ
ಮಾಡು-ತ್ತಿ-ದ್ದು-ದನ್ನು
ಮಾಡು-ತ್ತಿ-ದ್ದು-ದ-ರಿಂದ
ಮಾಡು-ತ್ತಿ-ದ್ದುವು
ಮಾಡು-ತ್ತಿದ್ದೆ
ಮಾಡು-ತ್ತಿ-ದ್ದೇ-ನಲ್ಲ
ಮಾಡು-ತ್ತಿ-ದ್ದೇನೆ
ಮಾಡು-ತ್ತಿ-ರ-ಬ-ಹುದೋ
ಮಾಡು-ತ್ತಿ-ರಲಿ
ಮಾಡು-ತ್ತಿ-ರುವ
ಮಾಡು-ತ್ತಿ-ರು-ವಂ-ತೆಯೇ
ಮಾಡು-ತ್ತಿ-ರು-ವ-ವರೆಲ್ಲ
ಮಾಡು-ತ್ತಿ-ರು-ವಾಗ
ಮಾಡು-ತ್ತಿ-ರುವು
ಮಾಡು-ತ್ತಿ-ರು-ವು-ದರ
ಮಾಡು-ತ್ತಿ-ರು-ವು-ದಾ-ದರೂ
ಮಾಡು-ತ್ತಿ-ರು-ವುದು
ಮಾಡು-ತ್ತಿ-ರು-ವು-ದೆಲ್ಲ
ಮಾಡು-ತ್ತಿ-ರು-ವೆನೇ
ಮಾಡು-ತ್ತಿ-ರು-ವೆ-ಯಷ್ಟೆ
ಮಾಡು-ತ್ತೀಯೆ
ಮಾಡು-ತ್ತೀರಾ
ಮಾಡು-ತ್ತೇನೆ
ಮಾಡು-ತ್ತೇವೆ
ಮಾಡುವ
ಮಾಡು-ವಂ-ತಹ
ಮಾಡು-ವಂತಾ
ಮಾಡು-ವಂ-ತಾಗ
ಮಾಡು-ವಂ-ತಾ-ಗಲಿ
ಮಾಡು-ವಂ-ತಿ-ರ-ಲಿಲ್ಲ
ಮಾಡು-ವಂ-ತಿಲ್ಲ
ಮಾಡು-ವಂತೆ
ಮಾಡು-ವಂ-ತೆಯೂ
ಮಾಡು-ವಲ್ಲಿ
ಮಾಡು-ವ-ವ-ರನ್ನು
ಮಾಡು-ವ-ವ-ರಿ-ದ್ದಾರೆ
ಮಾಡು-ವ-ವರು
ಮಾಡು-ವ-ವರೇ
ಮಾಡು-ವಾಗ
ಮಾಡು-ವಾ-ಗಲೂ
ಮಾಡುವು
ಮಾಡು-ವುದ
ಮಾಡು-ವು-ದಕ್ಕಾ
ಮಾಡು-ವು-ದ-ಕ್ಕಾಗಿ
ಮಾಡು-ವು-ದ-ಕ್ಕಾ-ಗಿಯೂ
ಮಾಡು-ವು-ದ-ಕ್ಕಿಂತ
ಮಾಡು-ವುದನ್ನು
ಮಾಡು-ವು-ದನ್ನೇ
ಮಾಡು-ವು-ದರ
ಮಾಡು-ವು-ದ-ರಲ್ಲಿ
ಮಾಡು-ವು-ದ-ರಲ್ಲೇ
ಮಾಡು-ವು-ದ-ರಿಂದ
ಮಾಡು-ವು-ದ-ಲ್ಲದೆ
ಮಾಡು-ವು-ದಷ್ಟೆ
ಮಾಡು-ವು-ದಾಗಿ
ಮಾಡು-ವು-ದಾ-ಗಿತ್ತು
ಮಾಡು-ವು-ದಾ-ದರೆ
ಮಾಡು-ವು-ದಿಲ್ಲ
ಮಾಡು-ವು-ದಿ-ಲ್ಲವೆ
ಮಾಡು-ವುದು
ಮಾಡು-ವು-ದು-ಇ-ದೆಲ್ಲ
ಮಾಡು-ವು-ದು-ಇದೇ
ಮಾಡು-ವುದೂ
ಮಾಡು-ವು-ದೆಂ-ದರೆ
ಮಾಡು-ವು-ದೆಂದು
ಮಾಡು-ವುದೇ
ಮಾಡೋಣ
ಮಾಡೋ-ಣವೆ
ಮಾಡೋ-ಣ-ವೆಂದರೆ
ಮಾಡ್ತೇನೆ
ಮಾತ
ಮಾತ-ನಾ-ಡ-ತೊ-ಡ-ಗಿ-ದರು
ಮಾತ-ನಾ-ಡ-ದಿ-ರಲಿ
ಮಾತ-ನಾ-ಡದೆ
ಮಾತ-ನಾ-ಡ-ಬ-ಲ್ಲ-ವನು
ಮಾತ-ನಾ-ಡ-ಬ-ಲ್ಲಿರಿ
ಮಾತ-ನಾ-ಡ-ಬಲ್ಲೆ
ಮಾತ-ನಾ-ಡ-ಬ-ಹುದು
ಮಾತ-ನಾ-ಡ-ಬೇಕಾ
ಮಾತ-ನಾ-ಡ-ಬೇ-ಕಾಗಿ
ಮಾತ-ನಾ-ಡ-ಬೇ-ಕಾ-ಗಿತ್ತು
ಮಾತ-ನಾ-ಡ-ಬೇ-ಕಾದ
ಮಾತ-ನಾ-ಡ-ಬೇ-ಕಾ-ದರೆ
ಮಾತ-ನಾ-ಡ-ಬೇ-ಕೆಂದು
ಮಾತ-ನಾ-ಡ-ಲಾ-ರಂ-ಭಿ-ಸಿ-ದರು
ಮಾತ-ನಾ-ಡ-ಲಾ-ರರು
ಮಾತ-ನಾ-ಡ-ಲಾರೆ
ಮಾತ-ನಾ-ಡ-ಲಿ-ದ್ದಾ-ರೆಂದು
ಮಾತ-ನಾ-ಡಲು
ಮಾತ-ನಾ-ಡ-ಲೇ-ಬೇ-ಕೆಂದು
ಮಾತ-ನಾಡಿ
ಮಾತ-ನಾ-ಡಿ-ಕೊಂ-ಡ-ದ್ದ-ನ್ನೆಲ್ಲ
ಮಾತ-ನಾ-ಡಿ-ಕೊಂ-ಡರು
ಮಾತ-ನಾ-ಡಿ-ಕೊ-ಳ್ಳುತ್ತ
ಮಾತ-ನಾ-ಡಿ-ಕೊ-ಳ್ಳು-ತ್ತಿ-ದ್ದರು
ಮಾತ-ನಾ-ಡಿದ
ಮಾತ-ನಾ-ಡಿ-ದನು
ಮಾತ-ನಾ-ಡಿ-ದರು
ಮಾತ-ನಾ-ಡಿ-ದ-ರು-ಧರ್ಮ
ಮಾತ-ನಾ-ಡಿ-ದರೂ
ಮಾತ-ನಾ-ಡಿ-ದರೆ
ಮಾತ-ನಾ-ಡಿ-ದ-ರೆಂ-ದರೆ
ಮಾತ-ನಾ-ಡಿ-ದ-ರೆಂದೂ
ಮಾತ-ನಾ-ಡಿ-ದರೋ
ಮಾತ-ನಾ-ಡಿ-ದ-ವ-ರಲ್ಲ
ಮಾತ-ನಾ-ಡಿ-ದಾಗ
ಮಾತ-ನಾ-ಡಿ-ದು-ದಾಗಿ
ಮಾತ-ನಾ-ಡಿ-ದುದು
ಮಾತ-ನಾ-ಡಿದೆ
ಮಾತ-ನಾ-ಡಿ-ದ್ದನ್ನು
ಮಾತ-ನಾ-ಡಿ-ದ್ದೀರಿ
ಮಾತ-ನಾ-ಡಿದ್ದು
ಮಾತ-ನಾ-ಡಿ-ದ್ದೆ-ಲ್ಲ-ವನ್ನೂ
ಮಾತ-ನಾ-ಡಿ-ಬಿಟ್ಟೆ
ಮಾತ-ನಾ-ಡಿ-ರ-ಬ-ಹುದು
ಮಾತ-ನಾ-ಡಿ-ರ-ಬೇಕು
ಮಾತ-ನಾ-ಡಿ-ರಲಿ
ಮಾತ-ನಾ-ಡಿ-ಸ-ಬಾ-ರ-ದೆಂದು
ಮಾತ-ನಾ-ಡಿ-ಸಲು
ಮಾತ-ನಾ-ಡಿಸಿ
ಮಾತ-ನಾ-ಡಿ-ಸಿ-ಕೊಂಡು
ಮಾತ-ನಾ-ಡಿ-ಸಿದ
ಮಾತ-ನಾ-ಡಿ-ಸಿ-ದರು
ಮಾತ-ನಾ-ಡಿ-ಸಿ-ದರೆ
ಮಾತ-ನಾ-ಡಿ-ಸಿ-ದಾಗ
ಮಾತ-ನಾ-ಡಿಸು
ಮಾತ-ನಾ-ಡಿ-ಸು-ತ್ತಿ-ದ್ದರು
ಮಾತ-ನಾ-ಡಿ-ಸು-ವ-ವ-ರೆಗೂ
ಮಾತ-ನಾಡು
ಮಾತ-ನಾ-ಡುತ್ತ
ಮಾತ-ನಾ-ಡು-ತ್ತಲೇ
ಮಾತ-ನಾ-ಡು-ತ್ತಾನೆ
ಮಾತ-ನಾ-ಡು-ತ್ತಾ-ರಂತೆ
ಮಾತ-ನಾ-ಡು-ತ್ತಾರೆ
ಮಾತ-ನಾ-ಡು-ತ್ತಾ-ರೆಂದು
ಮಾತ-ನಾ-ಡು-ತ್ತಾರೋ
ಮಾತ-ನಾ-ಡುತ್ತಿ
ಮಾತ-ನಾ-ಡು-ತ್ತಿದ್ದ
ಮಾತ-ನಾ-ಡು-ತ್ತಿ-ದ್ದಂತೆ
ಮಾತ-ನಾ-ಡು-ತ್ತಿ-ದ್ದಂ-ತೆಯೇ
ಮಾತ-ನಾ-ಡು-ತ್ತಿ-ದ್ದರು
ಮಾತ-ನಾ-ಡು-ತ್ತಿ-ದ್ದರೂ
ಮಾತ-ನಾ-ಡು-ತ್ತಿ-ದ್ದರೆ
ಮಾತ-ನಾ-ಡು-ತ್ತಿ-ದ್ದ-ವ-ರಾ-ದರೂ
ಮಾತ-ನಾ-ಡು-ತ್ತಿ-ದ್ದಾಗ
ಮಾತ-ನಾ-ಡು-ತ್ತಿ-ದ್ದುದು
ಮಾತ-ನಾ-ಡು-ತ್ತಿ-ದ್ದು-ದೆಲ್ಲ
ಮಾತ-ನಾ-ಡು-ತ್ತಿ-ದ್ದು-ದೊಂದು
ಮಾತ-ನಾ-ಡು-ತ್ತಿ-ದ್ದೆವು
ಮಾತ-ನಾ-ಡು-ತ್ತಿ-ದ್ದೇ-ವೆಂದೇ
ಮಾತ-ನಾ-ಡು-ತ್ತಿರಿ
ಮಾತ-ನಾ-ಡು-ತ್ತಿ-ರುವ
ಮಾತ-ನಾ-ಡು-ತ್ತಿ-ರು-ವಂ-ತಿತ್ತು
ಮಾತ-ನಾ-ಡು-ತ್ತಿ-ರು-ವಾಗ
ಮಾತ-ನಾ-ಡು-ತ್ತೇನೆ
ಮಾತ-ನಾ-ಡುವ
ಮಾತ-ನಾ-ಡು-ವಂತೆ
ಮಾತ-ನಾ-ಡು-ವ-ವರೆಲ್ಲ
ಮಾತ-ನಾ-ಡು-ವ-ಷ್ಟ-ರಲ್ಲಿ
ಮಾತ-ನಾ-ಡು-ವಷ್ಟು
ಮಾತ-ನಾ-ಡು-ವಾಗ
ಮಾತ-ನಾ-ಡು-ವಾ-ಗ-ಲಂತೂ
ಮಾತ-ನಾ-ಡು-ವು-ದಕ್ಕೆ
ಮಾತ-ನಾ-ಡು-ವುದನ್ನು
ಮಾತ-ನಾ-ಡು-ವು-ದನ್ನೇ
ಮಾತ-ನಾ-ಡು-ವು-ದಿ-ರಲಿ
ಮಾತ-ನಾ-ಡು-ವು-ದಿಲ್ಲ
ಮಾತ-ನಾ-ಡು-ವುದು
ಮಾತ-ನಾ-ಡು-ವು-ದೆಂ-ದ-ರೇನು
ಮಾತ-ನಾ-ಡು-ವು-ದೇ-ನಿ-ದ್ದರೂ
ಮಾತ-ನ್ನಾ-ಡು-ವ-ವ-ರಲ್ಲ
ಮಾತ-ನ್ನಾ-ಡು-ವ-ವರೂ
ಮಾತನ್ನು
ಮಾತನ್ನೂ
ಮಾತನ್ನೇ
ಮಾತಾ-ಡಲಿ
ಮಾತಾಡಿ
ಮಾತಾ-ಡು-ತ್ತಿ-ದ್ದರೆ
ಮಾತಾ-ನಾ-ಡಿ-ದರೆ
ಮಾತಿ-ಗಾ-ರಂ-ಭಿ-ಸಿದ
ಮಾತಿ-ಗಾ-ರಂ-ಭಿ-ಸಿ-ದರು
ಮಾತಿಗೆ
ಮಾತಿಗೇ
ಮಾತಿನ
ಮಾತಿ-ನಂತೆ
ಮಾತಿ-ನಂ-ತೆಯೇ
ಮಾತಿ-ನಲ್ಲಿ
ಮಾತಿ-ನ-ಲ್ಲಿದ್ದ
ಮಾತಿ-ನಿಂದ
ಮಾತಿಲ್ಲ
ಮಾತಿ-ಲ್ಲ-ದೆಯೇ
ಮಾತು
ಮಾತು-ಭಾ-ವ-ನೆ-ಗಳ
ಮಾತು-ಕತೆ
ಮಾತು-ಕ-ತೆ-ಗಳನ್ನೂ
ಮಾತು-ಕ-ತೆ-ಗಳಲ್ಲಿ
ಮಾತು-ಕ-ತೆ-ಗಳು
ಮಾತು-ಕ-ತೆಯ
ಮಾತು-ಕ-ತೆ-ಯನ್ನು
ಮಾತು-ಕ-ತೆ-ಯಲ್ಲಿ
ಮಾತು-ಕ-ತೆ-ಯಾ-ಡಲು
ಮಾತು-ಕ-ತೆ-ಯಾ-ಡಿ-ಕೊಂಡು
ಮಾತು-ಕ-ತೆ-ಯಾ-ಡುತ್ತ
ಮಾತು-ಕ-ತೆ-ಯಾ-ಡು-ತ್ತಿ-ದ್ದರು
ಮಾತು-ಕ-ತೆ-ಯಾ-ಡು-ತ್ತಿ-ದ್ದಾ-ಗಲೂ
ಮಾತು-ಕ-ತೆ-ಯಾ-ಡು-ವು-ದರ
ಮಾತು-ಕ-ತೆ-ಯಿಂದ
ಮಾತು-ಕ-ತೆ-ಯೇನೂ
ಮಾತು-ಕೊಟ್ಟ
ಮಾತು-ಕೊ-ಟ್ಟಿದ್ದ
ಮಾತು-ಕೊಟ್ಟು
ಮಾತು-ಗಳ
ಮಾತು-ಗ-ಳಂತೂ
ಮಾತು-ಗ-ಳ-ನ್ನಾ-ಡಲು
ಮಾತು-ಗ-ಳ-ನ್ನಾಡಿ
ಮಾತು-ಗ-ಳ-ನ್ನಾ-ಡಿ-ದರು
ಮಾತು-ಗ-ಳ-ನ್ನಾ-ಡಿ-ದ-ವನು
ಮಾತು-ಗ-ಳ-ನ್ನಾ-ಡಿ-ದ್ದೇನೆ
ಮಾತು-ಗ-ಳ-ನ್ನಾ-ಲಿ-ಸಿದ
ಮಾತು-ಗ-ಳ-ನ್ನಾ-ಲಿ-ಸು-ತ್ತಿದ್ದ
ಮಾತು-ಗಳನ್ನು
ಮಾತು-ಗಳನ್ನೂ
ಮಾತು-ಗಳನ್ನೆಲ್ಲ
ಮಾತು-ಗ-ಳ-ಲ್ಲದೆ
ಮಾತು-ಗಳಲ್ಲಿ
ಮಾತು-ಗಳಿಂದ
ಮಾತು-ಗ-ಳಿಂ-ದ-ಇ-ತ-ರರ
ಮಾತು-ಗ-ಳಿಂ-ದಲೂ
ಮಾತು-ಗ-ಳಿಂ-ದಲೇ
ಮಾತು-ಗ-ಳಿ-ಗಿಂತ
ಮಾತು-ಗ-ಳಿಗೂ
ಮಾತು-ಗ-ಳಿಗೆ
ಮಾತು-ಗ-ಳಿಗೇ
ಮಾತು-ಗ-ಳಿ-ರ-ಲಿಲ್ಲ
ಮಾತು-ಗ-ಳಿ-ವು-ಈಗ
ಮಾತು-ಗಳು
ಮಾತು-ಗಳೂ
ಮಾತು-ಗ-ಳೆಂದು
ಮಾತು-ಗ-ಳೆಂ-ಬಂತೆ
ಮಾತು-ಗ-ಳೆಲ್ಲ
ಮಾತು-ಗ-ಳೆ-ಲ್ಲ-ವನ್ನೂ
ಮಾತು-ಗಳೇ
ಮಾತು-ಗ-ಳೊಂ-ದಿಗೆ
ಮಾತು-ಗಳೋ
ಮಾತು-ಗಾ-ರ-ನೆಂದು
ಮಾತು-ಗಾ-ರ-ರಾ-ಗಿ-ದ್ದರು
ಮಾತು-ಗಾ-ರಿಕೆ
ಮಾತು-ಸ್ವಾಮಿ
ಮಾತೂ
ಮಾತೃ
ಮಾತೃ-ಪೂ-ಜೆಯ
ಮಾತೃ-ಪ್ರೇ-ಮವು
ಮಾತೃ-ಭಾ-ಷೆ-ಯನ್ನು
ಮಾತೃ-ಭೂ-ಮಿ-ಗಾಗಿ
ಮಾತೃ-ಭೂ-ಮಿಯ
ಮಾತೃ-ಹೃ-ದಯ
ಮಾತೆ-ತ್ತಲು
ಮಾತೆಯ
ಮಾತೆ-ಯ-ರೆ-ಲ್ಲರ
ಮಾತೆ-ಯಾದ
ಮಾತೆಲ್ಲ
ಮಾತೇ
ಮಾತೊಂ-ದನ್ನು
ಮಾತೊಂ-ದ-ರಿಂದ
ಮಾತ್ರ
ಮಾತ್ರಕ್ಕೆ
ಮಾತ್ರ-ದಿಂ-ದಾ-ಗಿಯೇ
ಮಾತ್ರ-ನಿಂದ
ಮಾತ್ರ-ರಿಂದ
ಮಾತ್ರ-ವಲ್ಲ
ಮಾತ್ರ-ವ-ಲ್ಲದೆ
ಮಾತ್ರ-ವಷ್ಟೇ
ಮಾತ್ರ-ವಾ-ದರೂ
ಮಾತ್ರವೇ
ಮಾತ್ರ-ವೇ-ಅ-ದ-ರಲ್ಲೂ
ಮಾತ್ರ-ವೇ-ಮಾನವ
ಮಾತ್ರ-ವೇ-ಹೊ-ರತು
ಮಾತ್ರವೋ
ಮಾತ್ಸ-ರ್ಯವೇ
ಮಾದರಿ
ಮಾಧ-ವ-ಚಂದ್ರ
ಮಾಧುರ್ಯ
ಮಾಧು-ರ್ಯ-ಭ-ವ್ಯ-ತೆ-ಗ-ಳೆ-ರಡೂ
ಮಾಧ್ಯ-ಮವೂ
ಮಾನ
ಮಾನ-ಗಳ
ಮಾನ-ದಂ-ಡ-ಗಳನ್ನು
ಮಾನ-ದಂ-ಡ-ದಿಂದ
ಮಾನವ
ಮಾನ-ವ-ಕೋ-ಟಿಗೆ
ಮಾನ-ವ-ಕೋ-ಟಿಯೇ
ಮಾನ-ವ-ಜೀ-ವ-ನದ
ಮಾನ-ವ-ಜೀ-ವಿಯೂ
ಮಾನ-ವ-ತೆಗೆ
ಮಾನ-ವ-ತೆಗೇ
ಮಾನ-ವ-ತೆಯ
ಮಾನ-ವ-ತೆ-ಯಷ್ಟು
ಮಾನ-ವ-ತೆಯೂ
ಮಾನ-ವ-ತ್ವದ
ಮಾನ-ವನ
ಮಾನ-ವ-ನನ್ನು
ಮಾನ-ವ-ನಲ್ಲಿ
ಮಾನ-ವ-ನಿ-ಗಾಗಿ
ಮಾನ-ವ-ನಿ-ಗಿಂ-ತಲೂ
ಮಾನ-ವ-ನಿಗೆ
ಮಾನ-ವ-ನಿ-ರ್ಮಿತ
ಮಾನ-ವನು
ಮಾನ-ವ-ನೆಂ-ದರೆ
ಮಾನ-ವನ್ನು
ಮಾನ-ವ-ಬು-ದ್ಧಿಯ
ಮಾನ-ವರ
ಮಾನ-ವ-ರನ್ನು
ಮಾನ-ವ-ರನ್ನೂ
ಮಾನ-ವರು
ಮಾನ-ವರೆ-ಡೆಗೆ
ಮಾನ-ವ-ಸ-ಹ-ಜ-ವಾದ
ಮಾನ-ವಾದ
ಮಾನ-ವೀಯ
ಮಾನ-ವೀ-ಯ-ತೆ-ಯನ್ನೇ
ಮಾನ-ಸಿಕ
ಮಾನ-ಸಿ-ಕ-ವಾಗಿ
ಮಾನ-ಸಿ-ಕ-ವಾ-ಗಿಯೂ
ಮಾನು-ಷ-ಸ-ಹ-ಜ-ವಾದ
ಮಾನ್ಕ್ಯೂರ್
ಮಾನ್ಯತೆ
ಮಾಮೂಲಿ
ಮಾಯ-ವಾ-ಗ-ಬೇ-ಕಾ-ಯಿತು
ಮಾಯ-ವಾಗಿ
ಮಾಯ-ವಾ-ಗಿತ್ತು
ಮಾಯ-ವಾ-ಗಿ-ದ್ದುವು
ಮಾಯ-ವಾ-ಗಿ-ಬಿ-ಡು-ತ್ತದೆ
ಮಾಯ-ವಾ-ಗು-ತ್ತಿ-ದ್ದರು
ಮಾಯ-ವಾ-ಯಿತು
ಮಾಯಾ
ಮಾಯಾ-ಕು-ರ್ಚಿಯೊ
ಮಾಯಾ-ತ-ತ್ತ್ವದ
ಮಾಯಾ-ಮಂ-ತ್ರ-ಗಳ
ಮಾಯಾ-ಮಂ-ತ್ರ-ಗ-ಳಿಗೂ
ಮಾಯೆ
ಮಾಯೆಗೆ
ಮಾಯೆಯ
ಮಾಯೆ-ಯನ್ನು
ಮಾಯೆ-ಯಿಂದ
ಮಾಯೆಯು
ಮಾರ-ನೆಯ
ಮಾರ-ಬೇಡ
ಮಾರಾಟ
ಮಾರಾ-ಟ-ವಾಗ
ಮಾರಿ
ಮಾರೀ-ಚ-ನಂತೆ
ಮಾರೀ-ಚ-ನನ್ನು
ಮಾರೀ-ಚ-ನೆಂಬ
ಮಾರು-ತ್ತಿದ್ದ
ಮಾರು-ವೇ-ಷ-ದಲ್ಲಿ
ಮಾರು-ಹೋ-ಗಿದ್ದ
ಮಾರು-ಹೋ-ಗಿ-ದ್ದ-ರ-ಲ್ಲೇನು
ಮಾರು-ಹೋ-ಗಿ-ದ್ದರು
ಮಾರು-ಹೋದ
ಮಾರು-ಹೋ-ದರು
ಮಾರ್ಗ
ಮಾರ್ಗಕ್ಕೇ
ಮಾರ್ಗ-ಗ-ಳೆಲ್ಲ
ಮಾರ್ಗ-ದ-ರ್ಶ-ಕ-ನಾ-ಗಿರು
ಮಾರ್ಗ-ದ-ರ್ಶ-ಕ-ರಾಗಿ
ಮಾರ್ಗ-ದ-ರ್ಶ-ಕರು
ಮಾರ್ಗ-ದ-ರ್ಶನ
ಮಾರ್ಗ-ದ-ರ್ಶ-ನ-ಪ್ರೋ-ತ್ಸಾ-ಹ-ಗಳನ್ನು
ಮಾರ್ಗ-ದ-ರ್ಶ-ನ-ದಲ್ಲಿ
ಮಾರ್ಗ-ದ-ರ್ಶ-ನ-ವನ್ನು
ಮಾರ್ಗ-ದ-ರ್ಶ-ನ-ವನ್ನೂ
ಮಾರ್ಗ-ದರ್ಶಿ
ಮಾರ್ಗ-ದ-ರ್ಶಿಗೂ
ಮಾರ್ಗ-ದ-ರ್ಶಿ-ಯನ್ನು
ಮಾರ್ಗ-ದ-ರ್ಶಿ-ಯಾಗಿ
ಮಾರ್ಗ-ದಲ್ಲಿ
ಮಾರ್ಗ-ದಿಂದ
ಮಾರ್ಗ-ಪ್ರ-ವ-ರ್ತ-ನ-ಮಾ-ರ್ಗ-ದ-ರ್ಶನ
ಮಾರ್ಗ-ರೆಟ್
ಮಾರ್ಗ-ರೆ-ಟ್ಟಳ
ಮಾರ್ಗ-ರೆ-ಟ್ಟಳು
ಮಾರ್ಗ-ರೇಟ್
ಮಾರ್ಗ-ವ-ನ್ನ-ರಿ-ಸಿಯೇ
ಮಾರ್ಗ-ವನ್ನು
ಮಾರ್ಗ-ವಾಗಿ
ಮಾರ್ಗವು
ಮಾರ್ಗ-ವೆಂದರೆ
ಮಾರ್ಗ-ವೆಂ-ದಾ-ಗಲಿ
ಮಾರ್ಗ-ವೆಂದು
ಮಾರ್ಗ-ವೆಂ-ಬು-ದನ್ನು
ಮಾರ್ಗವೇ
ಮಾರ್ಗ-ವೊಂ-ದನ್ನು
ಮಾರ್ಗೆ-ಸ್ಸನ್
ಮಾರ್ಗೆ-ಸ್ಸ-ನ್ನಳ
ಮಾರ್ಚ್
ಮಾರ್ಟಿನ್
ಮಾರ್ತಾಂಡ
ಮಾರ್ತಾಂ-ಡ-ವರ್ಮ
ಮಾರ್ತಾಂ-ಡ-ವ-ರ್ಮನ
ಮಾರ್ತಾಂ-ಡ-ವ-ರ್ಮ-ನನ್ನು
ಮಾರ್ತಾಂ-ಡ-ವ-ರ್ಮ-ನಿಗೆ
ಮಾರ್ಥಾ
ಮಾರ್ದ-ನಿ-ಸಿ-ದುವು
ಮಾರ್ದ-ನಿ-ಸು-ವಂತೆ
ಮಾರ್ಪ-ಡಿ-ಸಿ-ಕೊ-ಳ್ಳು-ತ್ತಾರೆ
ಮಾರ್ಪಡು
ಮಾರ್ಪಾ-ಡಾ-ಗಿ-ದ್ದುವು
ಮಾರ್ಪಾಡು
ಮಾರ್ಮಿಕ
ಮಾಲಿ-ಕೆ-ಗಳನ್ನು
ಮಾಲಿ-ಕೆ-ಯಲ್ಲಿ
ಮಾಲೀ-ಕರು
ಮಾಲೆ-ಯೊಂ-ದನ್ನು
ಮಾಳರು
ಮಾಳಿ-ಗೆಯ
ಮಾವ-ನ-ಮಾ-ನ-ವರ
ಮಾವ-ನ-ತೆಯ
ಮಾವಿನ
ಮಾಸ
ಮಾಸ-ಪ-ತ್ರಿ-ಕೆಗೆ
ಮಾಸ-ಪ-ತ್ರಿ-ಕೆಯ
ಮಾಸ-ಪ-ತ್ರಿ-ಕೆ-ಯನ್ನು
ಮಾಸ-ಪ-ತ್ರಿ-ಕೆ-ಯೊಂ-ದರ
ಮಾಸ್ಟ-ರರ
ಮಾಹಾ-ತ್ಮ್ಯ-ವನ್ನು
ಮಾಹಾ-ತ್ಮ್ಯ-ವನ್ನೂ
ಮಾಹಿತಿ
ಮಾಹಿ-ತಿ-ಗಳ
ಮಾಹಿ-ತಿ-ಗಳು
ಮಾಹಿ-ತಿ-ಯನ್ನು
ಮಿ
ಮಿಂಚಿ-ದುವು
ಮಿಂಚಿನ
ಮಿಂಚಿ-ನಂ-ತಹ
ಮಿಂಚಿ-ನಂತೆ
ಮಿಂಚಿ-ನಂ-ತೆ-ರ-ಗು-ವಂತೆ
ಮಿಂಚಿ-ರ-ಲಿಲ್ಲ
ಮಿಂಚು
ಮಿಂಚು-ತ್ತಿತ್ತು
ಮಿಕ್ಕರೆ
ಮಿಕ್ಕ-ವರೆಲ್ಲ
ಮಿಗಿ
ಮಿಗಿಲಾ
ಮಿಗಿ-ಲಾಗಿ
ಮಿಗಿ-ಲಾ-ಗಿತ್ತು
ಮಿಗಿ-ಲಾದ
ಮಿಗಿ-ಲಾ-ದ-ದ್ದಲ್ಲ
ಮಿಗಿ-ಲಾ-ದ-ದ್ದೇ-ನನ್ನೂ
ಮಿಗಿ-ಲಾ-ದುದು
ಮಿಗಿ-ಲಾ-ದು-ದೇ-ನಿದೆ
ಮಿಗಿಸಿ
ಮಿಚಿ-ಗನ್
ಮಿಡಿ-ತ-ವನ್ನು
ಮಿಡಿ-ಯಲಿ
ಮಿಡಿ-ಯು-ವುದನ್ನು
ಮಿತ-ವಾಗಿ
ಮಿತಿಗೆ
ಮಿತಿ-ಮೀ-ರು-ವಂ-ತಾ-ದಾಗ
ಮಿತಿ-ಯನ್ನು
ಮಿತಿ-ಯೊ-ಳಗೆ
ಮಿತ್ರ
ಮಿತ್ರನ
ಮಿತ್ರ-ನತ್ತ
ಮಿತ್ರ-ನಿಗೆ
ಮಿತ್ರನೇ
ಮಿತ್ರ-ರ-ನ್ನಾ-ಗಲಿ
ಮಿತ್ರ-ರನ್ನು
ಮಿತ್ರ-ರಾದ
ಮಿತ್ರ-ರಿಂದ
ಮಿತ್ರ-ರಿಗೆ
ಮಿತ್ರರು
ಮಿತ್ರ-ರೆ-ಲ್ಲರೂ
ಮಿಥ್ಯಾ-ಚಾ-ರ-ವನ್ನು
ಮಿಥ್ಯಾ-ಚಾ-ರಿ-ಗ-ಳಲ್ಲ
ಮಿಥ್ಯಾ-ಚಾ-ರಿ-ಗ-ಳಾದ
ಮಿಥ್ಯಾ-ಪ-ವಾ-ದ-ಗಳ
ಮಿಥ್ಯಾ-ಪ-ವಾ-ದ-ಗಳನ್ನು
ಮಿದು-ಳನ್ನೂ
ಮಿದು-ಳಿನ
ಮಿದು-ಳಿ-ನಲ್ಲಿ
ಮಿದು-ಳಿ-ಲ್ಲ-ದ-ವನು
ಮಿನರ್ವ
ಮಿನಿ-ಯಾ-ಪೊ-ಲಿಸ್
ಮಿನಿ-ಯಾ-ಪೊ-ಲಿ-ಸ್ನಲ್ಲಿ
ಮಿನಿ-ಯಾ-ಪೊ-ಲಿ-ಸ್ನಿಂದ
ಮಿನಿ-ಸ್ಟರ್
ಮಿನು-ಗಿತು
ಮಿನು-ಗಿ-ಸುತ್ತ
ಮಿನು-ಗು-ತ್ತಿತ್ತು
ಮಿನು-ಗು-ತ್ತಿದೆ
ಮಿನು-ಗು-ತ್ತಿದ್ದ
ಮಿನು-ಗುವ
ಮಿನೆ-ಸೋಟ
ಮಿರರ್
ಮಿರರ್ಗೆ
ಮಿರರ್ನ
ಮಿರ-ರ್ನಲ್ಲಿ
ಮಿಲಿ
ಮಿಲಿಯ
ಮಿಶ್ರ-ವ-ರ್ಣದ
ಮಿಷ-ನರಿ
ಮಿಷ-ನ-ರಿ-ಗಳ
ಮಿಷ-ನ-ರಿ-ಗಳನ್ನು
ಮಿಷ-ನ-ರಿ-ಗ-ಳಿಗೆ
ಮಿಷ-ನ-ರಿ-ಗಳು
ಮಿಷ-ನ-ರಿ-ಗ-ಳು-ಮಿ-ಷ-ನ-ರಿ-ಗ-ಳ-ಲ್ಲ-ದ-ವರು
ಮಿಷ-ನ-ರಿ-ಗಳೂ
ಮಿಷ-ನ-ರಿ-ಗ-ಳೇ-ನಾ-ದರೂ
ಮಿಷ-ನ-ರಿ-ಗ-ಳೊಂ-ದಿಗೆ
ಮಿಷ-ನ-ರಿ-ಯ-ನ್ನಾ-ದರೂ
ಮಿಷ-ನ-ರಿ-ಯಾ-ಗಿ-ರ-ಲಿಲ್ಲ
ಮಿಷ-ನ್ನಿನ
ಮಿಷಿ-ಗನ್
ಮಿಷಿ-ಗ-ನ್ನಿನ
ಮಿಸು-ಕ-ಲಿಲ್ಲ
ಮಿಸ್
ಮಿಸ್ಹ್ಯಾ-ಮ್ಲಿ-ನ್ನ-ಳಂತೂ
ಮೀಟಿ-ದಂತೆ
ಮೀನಾಕ್ಷಿ
ಮೀನಿ-ನಂ-ತಾ-ಗಿತ್ತು
ಮೀನು
ಮೀನು-ಗಳ
ಮೀನು-ಗಾ-ರರೂ
ಮೀರಿ
ಮೀರಿತ್ತು
ಮೀರಿದ
ಮೀರಿ-ದ-ವ-ರಾ-ದರು
ಮೀರಿ-ದು-ದಾ-ಗಿ-ರು-ತ್ತಿತ್ತು
ಮೀರಿ-ದ್ದೇ-ನೆಂದು
ಮೀರಿ-ನ-ಡೆ-ಯಲು
ಮೀರಿ-ನಿಂತ
ಮೀರಿ-ರ-ಬೇಕು
ಮೀರಿ-ಸ-ಬ-ಲ್ಲ-ವ-ರಾ-ಗಿ-ದ್ದಾರೆ
ಮೀರಿ-ಸಿ-ದ-ರೆಂದ
ಮೀರಿ-ಸಿದ್ದು
ಮೀರಿ-ಸು-ತ್ತವೆ
ಮೀರಿ-ಸುವ
ಮೀರಿ-ಸು-ವಂ-ಥ-ವರು
ಮೀಸ-ಲಾ-ಗಿತ್ತು
ಮೀಸ-ಲಾ-ಗಿ-ರಲಿ
ಮೀಸ-ಲಾ-ಗಿ-ರುವ
ಮೀಸ-ಲಾ-ಗಿಲ್ಲ
ಮೀಸ-ಲಾದ
ಮುಂಗಂಡ
ಮುಂಗಾರು
ಮುಂಗೋಪಿ
ಮುಂಚಿ-ತ-ವಾಗಿ
ಮುಂಚಿನ
ಮುಂಚೂ-ಣಿ-ಯಲ್ಲಿ
ಮುಂಚೆ
ಮುಂಚೆ-ಯಲ್ಲ
ಮುಂಜಾನೆ
ಮುಂಜಾ-ನೆಯ
ಮುಂಡನ
ಮುಂತಾದ
ಮುಂತಾ-ದ-ವರ
ಮುಂತಾ-ದ-ವ-ರಿ-ದ್ದರು
ಮುಂತಾ-ದ-ವು-ಗ-ಳೆ-ಲ್ಲವೂ
ಮುಂತಾ-ದ-ವೆಲ್ಲ
ಮುಂದಕ್ಕೆ
ಮುಂದ-ಡಿ-ಯಿಟ್ಟು
ಮುಂದಾ
ಮುಂದಾ-ಗ-ಬ-ಹು-ದಾ-ದ್ದನ್ನು
ಮುಂದಾ-ಗಿ-ದ್ದರು
ಮುಂದಾ-ಗಿ-ದ್ದಳು
ಮುಂದಾ-ಗಿ-ದ್ದಾನೆ
ಮುಂದಾ-ಗಿಯೇ
ಮುಂದಾ-ಗು-ತ್ತಿ-ದ್ದರು
ಮುಂದಾದ
ಮುಂದಾ-ದರು
ಮುಂದಾ-ದರೂ
ಮುಂದಾ-ದರೆ
ಮುಂದಾ-ದಳು
ಮುಂದಾ-ದವ
ಮುಂದಾ-ಲೋ-ಚ-ನೆ-ಯಾ-ಗಿತ್ತು
ಮುಂದಾ-ಳಾ-ಗಿ-ರಿ-ಸಿ-ಕೊಂಡು
ಮುಂದಾ-ಳು-ವೊ-ಬ್ಬರು
ಮುಂದಿ
ಮುಂದಿಗೂ
ಮುಂದಿಟ್ಟ
ಮುಂದಿ-ಟ್ಟದ್ದು
ಮುಂದಿ-ಟ್ಟರು
ಮುಂದಿ-ಟ್ಟರೋ
ಮುಂದಿ-ಟ್ಟ-ಸ್ವಾ-ಮೀಜಿ
ಮುಂದಿ-ಟ್ಟಾಗ
ಮುಂದಿ-ಟ್ಟಿ-ರುವ
ಮುಂದಿಟ್ಟು
ಮುಂದಿ-ಟ್ಟು-ಕೊ-ಳ್ಳಿ-ಜ-ನ-ಸ-ಮೂ-ಹದ
ಮುಂದಿ-ಡ-ಬೇ-ಕಾ-ಗಿದೆ
ಮುಂದಿ-ಡ-ಲಾ-ಯಿತು
ಮುಂದಿ-ಡ-ಲಾ-ರ-ದ-ವ-ನಾಗಿ
ಮುಂದಿ-ಡಲೂ
ಮುಂದಿ-ಡು-ತ್ತಾರೆ
ಮುಂದಿ-ಡು-ತ್ತಿದ್ದ
ಮುಂದಿ-ಡು-ತ್ತಿ-ದ್ದರು
ಮುಂದಿ-ಡುವ
ಮುಂದಿ-ದ್ದ-ವಳೇ
ಮುಂದಿ-ದ್ದಾ-ರೆಂ-ಬುದು
ಮುಂದಿನ
ಮುಂದಿ-ರುವ
ಮುಂದಿ-ರು-ವ-ವನು
ಮುಂದು
ಮುಂದು-ಮುಂ-ದಕ್ಕೆ
ಮುಂದು-ರಿ-ಸಿ-ಕೊಂಡು
ಮುಂದು-ರಿ-ಸಿ-ಕೊಂ-ಡು-ಹೋ-ಗಲು
ಮುಂದು-ವ-ರಿದ
ಮುಂದು-ವ-ರಿ-ದಂತೆ
ಮುಂದು-ವ-ರಿ-ದರು
ಮುಂದು-ವ-ರಿ-ದರೆ
ಮುಂದು-ವ-ರಿ-ದಿದೆ
ಮುಂದು-ವ-ರಿದು
ಮುಂದು-ವ-ರಿ-ದುವು
ಮುಂದು-ವ-ರಿ-ಯ-ಬೇ-ಕಾ-ಗಿದೆ
ಮುಂದು-ವ-ರಿ-ಯ-ಬೇ-ಕಾ-ದರೆ
ಮುಂದು-ವ-ರಿ-ಯ-ಬೇಕು
ಮುಂದು-ವ-ರಿ-ಯ-ಬೇ-ಕೆಂದು
ಮುಂದು-ವ-ರಿ-ಯ-ಲಾರೆ
ಮುಂದು-ವ-ರಿ-ಯಲು
ಮುಂದು-ವ-ರಿ-ಯಿ-ತ-ಲ್ಲದೆ
ಮುಂದು-ವ-ರಿ-ಯಿತು
ಮುಂದು-ವ-ರಿ-ಯಿರಿ
ಮುಂದು-ವ-ರಿ-ಯುತ್ತ
ಮುಂದು-ವ-ರಿ-ಯು-ತ್ತಲೇ
ಮುಂದು-ವ-ರಿ-ಯು-ತ್ತಿತ್ತು
ಮುಂದು-ವ-ರಿ-ಯು-ತ್ತಿದೆ
ಮುಂದು-ವ-ರಿ-ಯು-ತ್ತಿ-ದ್ದರೆ
ಮುಂದು-ವ-ರಿ-ಯು-ತ್ತಿ-ದ್ದುವು
ಮುಂದು-ವ-ರಿ-ಯು-ವಂತೆ
ಮುಂದು-ವ-ರಿ-ಯು-ವುದನ್ನು
ಮುಂದು-ವ-ರಿ-ಯು-ವುದು
ಮುಂದು-ವ-ರಿ-ಸ-ಬೇ-ಕಾ-ಗಿದೆ
ಮುಂದು-ವ-ರಿ-ಸ-ಬೇಕು
ಮುಂದು-ವ-ರಿ-ಸಲು
ಮುಂದು-ವ-ರಿ-ಸ-ಲೇ-ಬೇಕು
ಮುಂದು-ವ-ರಿಸಿ
ಮುಂದು-ವ-ರಿ-ಸಿ-ಕೊಂಡು
ಮುಂದು-ವ-ರಿ-ಸಿದ
ಮುಂದು-ವ-ರಿ-ಸಿ-ದರು
ಮುಂದು-ವ-ರಿ-ಸಿ-ದ-ರು-ಏ-ಕೆಂ-ದರೆ
ಮುಂದು-ವ-ರಿ-ಸಿರಿ
ಮುಂದು-ವ-ರಿಸು
ಮುಂದು-ವ-ರಿ-ಸುತ್ತ
ಮುಂದು-ವ-ರಿ-ಸು-ತ್ತಲೇ
ಮುಂದು-ವ-ರಿ-ಸು-ತ್ತಾನೆ
ಮುಂದು-ವ-ರಿ-ಸು-ತ್ತೇನೆ
ಮುಂದು-ವ-ರಿ-ಸುವ
ಮುಂದು-ವ-ರಿ-ಸು-ವಂತೆ
ಮುಂದು-ವ-ರಿ-ಸು-ವು-ದಾಗಿ
ಮುಂದು-ವರೆ-ಯಲು
ಮುಂದೂ-ಡಿ-ದ-ರೆಂದೂ
ಮುಂದೆ
ಮುಂದೆಂ-ದಾ-ದರೂ
ಮುಂದೆಂದೂ
ಮುಂದೆ-ಮುಂದೆ
ಮುಂದೆಯೂ
ಮುಂದೆಯೇ
ಮುಂದೆ-ಯೇ-ಇದು
ಮುಂದೇನು
ಮುಂದೊಡ್ಡಿ
ಮುಂದೊಮ್ಮೆ
ಮುಂದೋ-ಡಿದ
ಮುಂಬಯಿ
ಮುಂಬ-ಯಿಗೆ
ಮುಂಬ-ಯಿಯ
ಮುಂಬ-ಯಿ-ಯಲ್ಲಿ
ಮುಂಬ-ಯಿ-ಯ-ಲ್ಲಿ-ರುವ
ಮುಂಬ-ಯಿ-ಯಿಂದ
ಮುಂಬ-ರ-ಲಿ-ರುವ
ಮುಂಬರು
ಮುಂಬ-ರುವ
ಮುಂಬೆ-ಳ-ಕಿ-ನಲ್ಲಿ
ಮುಕುಂ-ದ-ಸಿಂ-ಗರ
ಮುಕು-ಟ-ವಿ-ಟ್ಟಂ-ತಿ-ರುವ
ಮುಕೇ
ಮುಕ್ಕಾಲು
ಮುಕ್ತ
ಮುಕ್ತ-ನಾ-ಗ-ಬಲ್ಲ
ಮುಕ್ತ-ಪು-ರು-ಷ-ರಾದ
ಮುಕ್ತ-ರ-ನ್ನಾ-ಗಿ-ಸಲು
ಮುಕ್ತ-ರಾದ
ಮುಕ್ತರು
ಮುಕ್ತ-ವಾಗಿ
ಮುಕ್ತ-ವಾ-ಗಿ-ಸುವ
ಮುಕ್ತ-ವಾ-ದುದು
ಮುಕ್ತ-ವಾ-ದುವು
ಮುಕ್ತಾಯ
ಮುಕ್ತಾ-ಯ-ಗೊಂ-ಡಿತು
ಮುಕ್ತಾ-ಯ-ಗೊ-ಳಿ-ಸಲು
ಮುಕ್ತಾ-ಯ-ಗೊ-ಳಿಸಿ
ಮುಕ್ತಾ-ಯ-ಗೊ-ಳಿ-ಸಿದ
ಮುಕ್ತಾ-ಯ-ಗೊ-ಳಿ-ಸಿ-ದ್ದರು
ಮುಕ್ತಾ-ಯ-ಗೊ-ಳಿ-ಸುತ್ತ
ಮುಕ್ತಾ-ಯ-ಗೊ-ಳಿ-ಸು-ವಾಗ
ಮುಕ್ತಾ-ಯದ
ಮುಕ್ತಿ
ಮುಕ್ತಿಗೆ
ಮುಕ್ತಿಯ
ಮುಕ್ತಿ-ಯ-ನ್ನ-ರ-ಸು-ವುದೂ
ಮುಕ್ತಿ-ಯನ್ನು
ಮುಕ್ತಿಯೇ
ಮುಕ್ತಿ-ಯೊಂ-ದಿಗೆ
ಮುಖ
ಮುಖಂಡ
ಮುಖಂ-ಡ-ನನ್ನು
ಮುಖಂ-ಡ-ನಾದ
ಮುಖಂ-ಡನೂ
ಮುಖಂ-ಡ-ರ-ಲ್ಲೊ-ಬ್ಬ-ನಾದ
ಮುಖಂ-ಡ-ರಾಗಿ
ಮುಖ-ಕ-ಮ-ಲ-ದಿಂದ
ಮುಖಕ್ಕೆ
ಮುಖ-ಗಳ
ಮುಖ-ಗಳನ್ನು
ಮುಖ-ಗ-ಳ-ನ್ನು-ಕೊ-ಳೆತ
ಮುಖ-ಗಳಲ್ಲಿ
ಮುಖ-ಗ-ಳಿವೆ
ಮುಖ-ಗಳು
ಮುಖ-ಗಳೂ
ಮುಖ-ಗ-ಳೆಂ-ಬು-ದನ್ನು
ಮುಖ-ಚ-ರ್ಯೆಯು
ಮುಖತಃ
ಮುಖದ
ಮುಖ-ದಂತೆ
ಮುಖ-ದಲ್ಲಿ
ಮುಖ-ದಲ್ಲೇ
ಮುಖ-ದೆ-ದು-ರಿಗೇ
ಮುಖ-ಪುಟ
ಮುಖ-ಪು-ಟ-ದಲ್ಲಿ
ಮುಖ-ಭಾವ
ಮುಖ-ಭಾ-ವ-ಗಳನ್ನು
ಮುಖ-ಭಾ-ವ-ವನ್ನು
ಮುಖ-ಮಂ-ಡಲ
ಮುಖ-ಮಂ-ಡ-ಲ-ವನ್ನು
ಮುಖ-ಮಂ-ಡ-ಲವು
ಮುಖ-ಮುಖ
ಮುಖ-ಮುದ್ರೆ
ಮುಖ-ಮು-ದ್ರೆಯ
ಮುಖ-ಮು-ದ್ರೆ-ಯನ್ನು
ಮುಖ-ಮು-ದ್ರೆ-ಯಿಂದ
ಮುಖ-ರ್ಜಿ-ಯ-ವರ
ಮುಖ-ರ್ಜಿ-ಯ-ವರು
ಮುಖ-ವನ್ನು
ಮುಖ-ವನ್ನೂ
ಮುಖ-ವನ್ನೇ
ಮುಖವು
ಮುಖವೂ
ಮುಖಾಂ-ತರ
ಮುಖ್ಯ
ಮುಖ್ಯ-ರಾ-ದವ
ಮುಖ್ಯ-ರಾ-ದ-ವ-ರೆಂ-ದರೆ
ಮುಖ್ಯ-ವಲ್ಲ
ಮುಖ್ಯ-ವಾ-ಗ-ಬೇಕು
ಮುಖ್ಯ-ವಾಗಿ
ಮುಖ್ಯ-ವಾ-ಗಿತ್ತು
ಮುಖ್ಯ-ವಾ-ಗಿತ್ತೇ
ಮುಖ್ಯ-ವಾದ
ಮುಖ್ಯ-ವಾ-ದ-ದ್ದಾ-ಗಿತ್ತು
ಮುಖ್ಯ-ವಾ-ದರೆ
ಮುಖ್ಯ-ವಾ-ದು-ದೆಂ-ದರೆ
ಮುಖ್ಯ-ವಾ-ದು-ವೆಂದರೆ
ಮುಖ್ಯ-ಸು-ದ್ದಿ-ಯಾಗಿ
ಮುಖ್ಯ-ಸ್ಥ-ನಾದ
ಮುಖ್ಯ-ಸ್ಥರ
ಮುಖ್ಯ-ಸ್ಥ-ರಾದ
ಮುಖ್ಯ-ಸ್ಥರು
ಮುಖ್ಯಾಂಶ
ಮುಖ್ಯಾ-ಧಿ-ಕಾ-ರಿಗೂ
ಮುಖ್ಯೋ-ಪಾ-ಧ್ಯಾ-ಯ-ರಾಗಿ
ಮುಗಿದ
ಮುಗಿ-ದಂ-ತೆಯೇ
ಮುಗಿ-ದ-ತ-ಕ್ಷಣ
ಮುಗಿ-ದವು
ಮುಗಿ-ದಾಗ
ಮುಗಿ-ದಿದೆ
ಮುಗಿದು
ಮುಗಿ-ದುವು
ಮುಗಿ-ದು-ಹೋ-ಗಿತ್ತು
ಮುಗಿ-ದು-ಹೋ-ಗು-ತ್ತದೆ
ಮುಗಿ-ಯದ
ಮುಗಿ-ಯ-ದಾ-ಟವ
ಮುಗಿ-ಯಿತು
ಮುಗಿ-ಯಿ-ತೆಂದೇ
ಮುಗಿ-ಯುತ್ತ
ಮುಗಿ-ಯುವ
ಮುಗಿ-ಯು-ವ-ವ-ರೆಗೂ
ಮುಗಿ-ಲಿಗು
ಮುಗಿ-ಲಿ-ಗೇ-ರಿ-ದ-ರಾ-ಯಿತು
ಮುಗಿ-ಲಿ-ನಿಂದ
ಮುಗಿ-ಲೆ-ಡೆಗೆ
ಮುಗಿ-ಲೆ-ತ್ತ-ರಕ್ಕೆ
ಮುಗಿ-ಸ-ಬೇ-ಕೆಂಬ
ಮುಗಿಸಿ
ಮುಗಿ-ಸಿ-ಕೊಂಡು
ಮುಗಿ-ಸಿದ
ಮುಗಿ-ಸಿ-ದರು
ಮುಗಿ-ಸಿ-ದಾಗ
ಮುಗಿ-ಸಿ-ದೆವು
ಮುಗಿ-ಸಿ-ಬಿ-ಟ್ಟಿ-ರು-ತ್ತದೆ
ಮುಗಿ-ಸಿ-ಬಿಡಿ
ಮುಗಿ-ಸಿ-ರ-ಬೇ-ಕೆಂದು
ಮುಗಿ-ಸಿ-ರ-ಲಿಲ್ಲ
ಮುಗಿ-ಸು-ತ್ತಿ-ದ್ದಂತೆ
ಮುಗಿ-ಸು-ತ್ತಿ-ದ್ದರು
ಮುಗಿ-ಸುವ
ಮುಗಿ-ಸು-ವ-ವ-ರೆಗೆ
ಮುಗು-ಳ್ನ-ಕ್ಕರು
ಮುಗು-ಳ್ನಕ್ಕು
ಮುಗು-ಳ್ನಗು
ಮುಗು-ಳ್ನ-ಗುತ್ತ
ಮುಗು-ಳ್ನಗೆ
ಮುಗು-ಳ್ನ-ಗೆ-ಯಿಂ-ದಲೋ
ಮುಗು-ಳ್ನ-ಗೆ-ಯೊಂ-ದಿಗೆ
ಮುಗ್ಧ
ಮುಗ್ಧ-ರಾಗಿ
ಮುಗ್ಧ-ರಾದ
ಮುಗ್ಧರು
ಮುಗ್ಧರೂ
ಮುಗ್ಧ-ವಾ-ಗಿ-ಸಿ-ದ್ದರು
ಮುಚ್ಚಿ
ಮುಚ್ಚಿ-ಕೊಂಡು
ಮುಚ್ಚಿ-ಕೊ-ಳ್ಳಲು
ಮುಚ್ಚಿ-ಕೊ-ಳ್ಳು-ತ್ತಿ-ದ್ದುವು
ಮುಚ್ಚಿತ್ತು
ಮುಚ್ಚಿದ
ಮುಚ್ಚಿಯೇ
ಮುಚ್ಚಿ-ಸ-ಬೇಕು
ಮುಚ್ಚಿ-ಸಲು
ಮುಚ್ಚಿಸಿ
ಮುಚ್ಚಿ-ಸಿ-ಬಿ-ಟ್ಟಿತು
ಮುಚ್ಚಿ-ಸು-ವಂತೆ
ಮುಚ್ಚಿ-ಸು-ವಲ್ಲಿ
ಮುಚ್ಚಿ-ಹೋ-ಯಿತು
ಮುಚ್ಚು-ಮರೆ
ಮುಚ್ಚು-ಮ-ರೆಯ
ಮುಚ್ಚು-ಮ-ರೆ-ಯಿ-ಲ್ಲದ
ಮುಚ್ಚು-ಮ-ರೆ-ಯಿ-ಲ್ಲದೆ
ಮುಜ-ರಾಯಿ
ಮುಟ್ಟದೆ
ಮುಟ್ಟ-ಬ-ಹುದೆ
ಮುಟ್ಟ-ಬಾ-ರದೆ
ಮುಟ್ಟ-ಬಾ-ರ-ದೆಂದು
ಮುಟ್ಟ-ಬೇಡಿ
ಮುಟ್ಟಲಿ
ಮುಟ್ಟಲು
ಮುಟ್ಟಿ
ಮುಟ್ಟಿತು
ಮುಟ್ಟಿದ
ಮುಟ್ಟಿ-ದರು
ಮುಟ್ಟಿ-ದರೆ
ಮುಟ್ಟಿ-ದುವು
ಮುಟ್ಟಿ-ದ್ದನ್ನು
ಮುಟ್ಟಿ-ದ್ದಾರೆ
ಮುಟ್ಟಿದ್ದು
ಮುಟ್ಟಿ-ಬಿ-ಟ್ಟಿದೆ
ಮುಟ್ಟಿ-ಸಿ-ಕೊಂಡು
ಮುಟ್ಟಿ-ಸಿದ
ಮುಟ್ಟಿ-ಸಿ-ದರೆ
ಮುಟ್ಟುವ
ಮುಟ್ಟು-ವಂ-ತೆಯೇ
ಮುಟ್ಟು-ವಲ್ಲಿ
ಮುಟ್ಟು-ವುದೇ
ಮುಡಿ
ಮುಡಿ-ಪಾ-ಗಿ-ಡಲು
ಮುಡಿ-ಪಾ-ಗಿ-ಡು-ತ್ತಾರೆ
ಮುಡಿ-ಪಾ-ಯಿತು
ಮುಡಿಯ
ಮುಡಿ-ಯ-ವ-ರೆಗೂ
ಮುತ್ತ
ಮುತ್ತ-ಲಿನ
ಮುತ್ತಿ
ಮುತ್ತಿ-ಕೊಂಡ
ಮುತ್ತಿ-ಕೊಂ-ಡರು
ಮುತ್ತಿ-ಕೊಂ-ಡಿತು
ಮುತ್ತಿ-ಕೊಂ-ಡಿ-ರು-ತ್ತಿ-ದ್ದರು
ಮುತ್ತಿ-ಕೊಂಡು
ಮುತ್ತಿ-ಕೊಂಡೇ
ಮುತ್ತಿ-ಕೊ-ಳ್ಳ-ಲಾ-ರಂ-ಭಿ-ಸಿತೋ
ಮುತ್ತಿ-ಕೊಳ್ಳು
ಮುತ್ತಿ-ಕೊ-ಳ್ಳು-ತ್ತಿ-ದ್ದರು
ಮುತ್ತಿ-ದರು
ಮುತ್ತಿ-ದಾ-ಗ-ಲೆಲ್ಲ
ಮುತ್ತು
ಮುತ್ತು-ತ್ತಾರೆ
ಮುತ್ತು-ತ್ತಿದ್ದ
ಮುದ-ಲಿ-ಯಾ-ರರು
ಮುದ-ಲಿ-ಯಾರ್
ಮುದಿ
ಮುದುಕ
ಮುದು-ಕನ
ಮುದು-ಕ-ನಿಗೆ
ಮುದು-ಕ-ರಿಗೆ
ಮುದು-ಕರು
ಮುದು-ಡಿ-ಕೊಂಡೇ
ಮುದ್ದೆ-ಗಳನ್ನು
ಮುದ್ರ-ಕರು
ಮುದ್ರಣ
ಮುದ್ರ-ಣ-ವನ್ನು
ಮುದ್ರ-ಣವು
ಮುದ್ರಿತ
ಮುದ್ರಿ-ತ-ವಾ-ಗಿದ್ದ
ಮುದ್ರಿ-ತ-ವಾ-ಯಿತು
ಮುದ್ರಿ-ತ-ವಾ-ಯಿ-ತೆಂ-ದರೆ
ಮುದ್ರಿ-ಸ-ಲಾ-ಗು-ತ್ತಿತ್ತು
ಮುದ್ರಿಸಿ
ಮುದ್ರಿ-ಸಿ-ದರು
ಮುದ್ರೆ-ಯ-ನ್ನೊ-ತ್ತಲು
ಮುದ್ರೆ-ಯ-ನ್ನೊತ್ತಿ
ಮುದ್ರೆ-ಯ-ನ್ನೊ-ತ್ತಿತ್ತು
ಮುದ್ರೆ-ಯ-ನ್ನೊ-ತ್ತಿ-ದ್ದರು
ಮುದ್ರೆ-ಯ-ನ್ನೊ-ತ್ತಿ-ದ್ದುವು
ಮುದ್ರೆ-ಯ-ನ್ನೊ-ತ್ತಿವೆ
ಮುದ್ರೆ-ಯ-ನ್ನೊತ್ತು
ಮುದ್ರೆ-ಯಿಂದ
ಮುದ್ರೆ-ಯೊಂ-ದನ್ನು
ಮುದ್ರೆ-ಯೊ-ತ್ತಿವೆ
ಮುನಿಯ
ಮುನಿ-ಯಂತೆ
ಮುನಿ-ಸಿ-ಕೊ-ಳ್ಳು-ತ್ತೀರಿ
ಮುನ್ನ
ಮುನ್ನಡೆ
ಮುನ್ನ-ಡೆ-ದಂ-ತೆಲ್ಲ
ಮುನ್ನ-ಡೆ-ದರು
ಮುನ್ನ-ಡೆ-ದಾಗ
ಮುನ್ನ-ಡೆದು
ಮುನ್ನ-ಡೆಯ
ಮುನ್ನ-ಡೆ-ಯ-ತೊ-ಡ-ಗಿತ್ತು
ಮುನ್ನ-ಡೆ-ಯನ್ನು
ಮುನ್ನ-ಡೆ-ಯ-ಬೇ-ಕಾ-ದದ್ದು
ಮುನ್ನ-ಡೆ-ಯ-ಬೇಕು
ಮುನ್ನ-ಡೆ-ಯಲು
ಮುನ್ನ-ಡೆ-ಯಲ್ಲಿ
ಮುನ್ನ-ಡೆ-ಯಿರಿ
ಮುನ್ನ-ಡೆ-ಯುತ್ತ
ಮುನ್ನ-ಡೆ-ಯು-ತ್ತಲೇ
ಮುನ್ನ-ಡೆ-ಯು-ತ್ತಿ-ದ್ದರು
ಮುನ್ನ-ಡೆ-ಯು-ತ್ತಿ-ದ್ದಾರೆ
ಮುನ್ನ-ಡೆ-ಯು-ತ್ತಿ-ದ್ದೇವೆ
ಮುನ್ನ-ಡೆ-ಯುವ
ಮುನ್ನ-ಡೆ-ಯು-ವಂತೆ
ಮುನ್ನ-ಡೆ-ಯೋಣ
ಮುನ್ನ-ಡೆವ
ಮುನ್ನ-ಡೆ-ಸಲು
ಮುನ್ನ-ಡೆ-ಸಿ-ದರೆ
ಮುನ್ನು-ಗು-ತ್ತಿ-ದೆ-ಇ-ದೊಂದು
ಮುನ್ನು-ಗ್ಗ-ಬ-ಲ್ಲ-ವರೂ
ಮುನ್ನುಗ್ಗಿ
ಮುನ್ನು-ಗ್ಗಿತು
ಮುನ್ನು-ಗ್ಗು-ತ್ತಾ-ರೆಯೋ
ಮುನ್ನುಡಿ
ಮುನ್ನು-ಡಿ-ಯಲ್ಲಿ
ಮುನ್ನೂರು
ಮುನ್ನೆ-ಚ್ಚ-ರಿಕೆ
ಮುನ್ನೆ-ಚ್ಚ-ರಿ-ಕೆ-ಗಳನ್ನು
ಮುನ್ನೆ-ಚ್ಚ-ರಿ-ಕೆ-ಯನ್ನು
ಮುನ್ನೆ-ಚ್ಚ-ರಿ-ಕೆ-ಯನ್ನೂ
ಮುನ್ನೋಟ
ಮುನ್ಷಿ
ಮುನ್ಷಿಗೆ
ಮುನ್ಸೂ-ಚನೆ
ಮುನ್ಸೂ-ಚ-ನೆಯೂ
ಮುನ್ಸೂ-ಚ-ನೆ-ಯೆಂದು
ಮುನ್ಸೂ-ಚ-ನೆಯೇ
ಮುನ್ಸೂಚಿ
ಮುಮು-ಕ್ಷ-ಗ-ಳಿಗೆ
ಮುಮು-ಕ್ಷು-ಗ-ಳಾದ
ಮುಮು-ಕ್ಷು-ಗಳು
ಮುಯ್ಯಿ
ಮುಯ್ಯಿಗೆ
ಮುರಿ
ಮುರಿ-ದಂತೆ
ಮುರಿದು
ಮುರಿ-ದು-ಬಿತ್ತು
ಮುರಿ-ದು-ಬಿ-ದ್ದಿ-ರು-ತ್ತದೆ
ಮುರಿ-ದು-ಹಾ-ಕ-ಬ-ಹುದು
ಮುರಿ-ಯದೆ
ಮುರಿ-ಯು-ತ್ತಿ-ದ್ದರು
ಮುರಿ-ಯುವ
ಮುಲಾಜೂ
ಮುಲಾಮು
ಮುಲ್ಲ-ರರ
ಮುಲ್ಲ-ರ-ರನ್ನು
ಮುಲ್ಲ-ರ-ರಿಗೆ
ಮುಲ್ಲ-ರರು
ಮುಲ್ಲ-ರರೂ
ಮುಲ್ಲ-ರರೇ
ಮುಲ್ಲ-ರಳ
ಮುಲ್ಲ-ರಳೂ
ಮುಲ್ಲರ್
ಮುಲ್ಲಾರ್
ಮುಳು-ಗ-ತೊ-ಡ-ಗಿತು
ಮುಳುಗಿ
ಮುಳು-ಗಿತ್ತು
ಮುಳು-ಗಿದ
ಮುಳು-ಗಿದ್ದ
ಮುಳು-ಗಿ-ದ್ದ-ರಾ-ದರೂ
ಮುಳು-ಗಿ-ದ್ದರು
ಮುಳು-ಗಿ-ದ್ದರೆ
ಮುಳು-ಗಿ-ದ್ದರೋ
ಮುಳು-ಗಿ-ದ್ದ-ವರು
ಮುಳು-ಗಿ-ದ್ದಾ-ಗಲೂ
ಮುಳು-ಗಿ-ದ್ದಾ-ರೆಂ-ದರೆ
ಮುಳು-ಗಿ-ದ್ದೀ-ರ-ಲ್ಲ-ಎಂಥ
ಮುಳು-ಗಿ-ಬಿ-ಟ್ಟರು
ಮುಳು-ಗಿ-ಬಿ-ಟ್ಟಿ-ದ್ದರು
ಮುಳು-ಗಿ-ಬಿಡು
ಮುಳು-ಗಿ-ಬಿ-ಡುವ
ಮುಳು-ಗಿ-ರು-ತ್ತಿ-ದ್ದರು
ಮುಳು-ಗಿ-ರುವ
ಮುಳು-ಗಿ-ರು-ವ-ವ-ರನ್ನು
ಮುಳು-ಗಿ-ರು-ವ-ವರು
ಮುಳು-ಗಿ-ಸು-ವಷ್ಟು
ಮುಳು-ಗಿ-ಹೋ-ಗಿದ್ದ
ಮುಳು-ಗಿ-ಹೋ-ಗಿ-ದ್ದು-ದ-ರಿಂದ
ಮುಳು-ಗಿ-ಹೋ-ಗಿ-ರುವ
ಮುಳು-ಗಿ-ಹೋ-ದಳು
ಮುಳ್ಳು
ಮುಷ್ಕರ
ಮುಸ-ಲ್ಮಾನ
ಮುಸ-ಲ್ಮಾ-ನ-ನಾ-ಗಿದ್ದ
ಮುಸ-ಲ್ಮಾ-ನ-ನೊ-ಬ್ಬನ
ಮುಸ-ಲ್ಮಾ-ನರ
ಮುಸ-ಲ್ಮಾ-ನ-ರನ್ನು
ಮುಸ-ಲ್ಮಾ-ನರು
ಮುಸ-ಲ್ಮಾ-ನರೇ
ಮುಸ್ಲಿಂ
ಮುಸ್ಲಿ-ಮರ
ಮುಸ್ಲಿ-ಮರೂ
ಮುಸ್ಲಿಮ್
ಮುಹೂ-ರ್ತ-ದಿಂದ
ಮೂಕಂ
ಮೂಕ-ನನ್ನು
ಮೂಕ-ರಾಗಿ
ಮೂಕ-ರಾ-ಗಿ-ಬಿ-ಟ್ಟಿ-ದ್ದರು
ಮೂಕ-ವಿ-ಸ್ಮ-ತ-ರಾಗಿ
ಮೂಕ-ವಿ-ಸ್ಮಿ-ತ-ರಾಗಿ
ಮೂಗಿನ
ಮೂಗಿ-ನ-ವರೆ-ಲ್ಲರೂ
ಮೂಗು
ಮೂಗು-ಗಳನ್ನು
ಮೂಡ-ನಂ-ಬಿ-ಕೆ-ಗಳ
ಮೂಡಿ
ಮೂಡಿತು
ಮೂಡಿ-ತುಈ
ಮೂಡಿ-ತ್ತಾ-ದರೂ
ಮೂಡಿದ
ಮೂಡಿದ್ದ
ಮೂಡಿ-ದ್ದರೆ
ಮೂಡಿ-ಬಂತು
ಮೂಡಿ-ಬಂ-ದಿದೆ
ಮೂಡಿ-ಬಂ-ದುವು
ಮೂಡಿ-ಬ-ರು-ತ್ತಿ-ರುವು
ಮೂಡಿ-ರ-ಲಿಲ್ಲ
ಮೂಡಿ-ರುವ
ಮೂಡಿ-ಸಲು
ಮೂಡಿ-ಸಿದೆ
ಮೂಡಿ-ಸಿದ್ದ
ಮೂಡಿ-ಸಿ-ದ್ದುವು
ಮೂಡಿ-ಸಿ-ರು-ವುದು
ಮೂಡಿ-ಸು-ತ್ತಿ-ದ್ದರು
ಮೂಡಿ-ಸು-ತ್ತಿ-ರುವ
ಮೂಡಿ-ಸು-ವು-ದ-ರಲ್ಲಿ
ಮೂಡು-ತ್ತಿ-ದ್ದಂ-ತೆಯೇ
ಮೂಡು-ತ್ತಿ-ದ್ದಿ-ರ-ಬೇಕು
ಮೂಡು-ತ್ತಿ-ದ್ದುವು
ಮೂಡು-ತ್ತಿ-ರು-ವುದನ್ನು
ಮೂಡು-ವಂತೆ
ಮೂಢ
ಮೂಢ-ನಂ-ಬಿಕೆ
ಮೂಢ-ನಂ-ಬಿ-ಕೆ-ಕಂ-ದಾ-ಚಾರ
ಮೂಢ-ನಂ-ಬಿ-ಕೆ-ಮ-ತಾಂ-ಧ-ತೆ-ಗಳ
ಮೂಢ-ನಂ-ಬಿ-ಕೆ-ಗ-ಳಿ-ರು-ವು-ದಾ-ದರೆ
ಮೂಢ-ನಂ-ಬಿ-ಕೆ-ಗ-ಳಿವೆ
ಮೂಢ-ನಂ-ಬಿ-ಕೆಯ
ಮೂಢ-ರು-ಅ-ನಾ-ಗ-ರಿ-ಕ-ರೆಂದು
ಮೂರ-ನೆಯ
ಮೂರ-ನೆ-ಯ-ದಾಗಿ
ಮೂರ-ನೆ-ಯದು
ಮೂರನೇ
ಮೂರ-ರಂದು
ಮೂರು
ಮೂರು
ಮೂರು-ನಾಲ್ಕು
ಮೂರು-ದಿ-ನ-ಗಳ
ಮೂರು-ಸ-ರ್ವ-ಧ-ರ್ಮ-ಗಳ
ಮೂರೇ
ಮೂರ್ಖ
ಮೂರ್ಖ-ತನ
ಮೂರ್ಖ-ತ-ನ-ಕ್ಕಾಗಿ
ಮೂರ್ಖ-ತ-ನ-ಗಳ
ಮೂರ್ಖ-ತ-ನದ
ಮೂರ್ಖ-ತ-ನವೇ
ಮೂರ್ಖ-ರನ್ನು
ಮೂರ್ಖ-ರನ್ನೂ
ಮೂರ್ಖ-ರಲ್ಲ
ಮೂರ್ಖ-ರಿಂ-ದಲ್ಲ
ಮೂರ್ಖರು
ಮೂರ್ಖ-ವ-ರ್ತ-ನೆ-ಯನ್ನು
ಮೂರ್ಛಿತ
ಮೂರ್ತ-ರೂಪ
ಮೂರ್ತ-ರೂ-ಪ-ವೆಂ-ಬಂತೆ
ಮೂರ್ತ-ರೂ-ಪವೇ
ಮೂರ್ತಿ
ಮೂರ್ತಿ-ಗಳು
ಮೂರ್ತಿ-ಗೊಂ-ಡಿ-ಳೆ-ಗಿ-ಳಿದ
ಮೂರ್ತಿ-ಪೂ-ಜ-ಕ-ರಾ-ಗಲು
ಮೂರ್ತಿ-ಪೂ-ಜ-ಕ-ರಾ-ಗಿಯೇ
ಮೂರ್ತಿ-ಪೂ-ಜೆಯ
ಮೂರ್ತಿ-ಪೂ-ಜೆ-ಯನ್ನು
ಮೂರ್ತಿ-ಪೂ-ಜೆ-ಯಲ್ಲಿ
ಮೂರ್ತಿ-ಭಂ-ಜ-ಕರೂ
ಮೂರ್ತಿ-ಯಿದೆ
ಮೂರ್ತಿ-ವೆ-ತ್ತಂ-ತಿ-ದ್ದರು
ಮೂರ್ನಾಲ್ಕು
ಮೂಲ
ಮೂಲಕ
ಮೂಲ-ಕ-ವಲ್ಲ
ಮೂಲ-ಕವೂ
ಮೂಲ-ಕವೇ
ಮೂಲ-ಕಾರಣ-ವನ್ನೂ
ಮೂಲ-ಕಾರಣ-ವಾದ
ಮೂಲ-ಕಾರಣ-ವೆಂದು
ಮೂಲಕ್ಕೆ
ಮೂಲಕ್ಕೇ
ಮೂಲ-ಗಳು
ಮೂಲ-ತ-ತ್ತ್ವ-ಗಳನ್ನು
ಮೂಲದ
ಮೂಲ-ದ-ಲ್ಲಿ-ರು-ವುದು
ಮೂಲ-ದಲ್ಲೇ
ಮೂಲ-ದಿಂದ
ಮೂಲ-ಭೂತ
ಮೂಲ-ಭೂ-ತ-ವಾಗಿ
ಮೂಲ-ಭೂ-ತ-ವಾದ
ಮೂಲ-ಮಂತ್ರ
ಮೂಲ-ವನ್ನೂ
ಮೂಲ-ವಾದ
ಮೂಲ-ವೆಂದು
ಮೂಲ-ವೊಂ-ದನ್ನು
ಮೂಲಾ
ಮೂಲೆ
ಮೂಲೆ-ಗಳಲ್ಲಿ
ಮೂಲೆಗೆ
ಮೂಲೆ-ಮೂ-ಲೆಗೆ
ಮೂಲೆ-ಮೂ-ಲೆಯ
ಮೂಲೆಯ
ಮೂಲೆ-ಯಲ್ಲಿ
ಮೂಲೆ-ಯ-ಲ್ಲಿದ್ದ
ಮೂಲೆ-ಯಿಂದ
ಮೂಲೋ-ದ್ದೇಶ
ಮೂಳೆ
ಮೂಳೆ-ಗಳನ್ನು
ಮೂಳೆ-ಗಳು
ಮೂವ-ತ್ತ-ಮೂ-ರ-ನೆಯ
ಮೂವ-ತ್ತಾರು
ಮೂವತ್ತು
ಮೂವ-ತ್ತೆ-ರಡು
ಮೂವ-ತ್ತೈದು
ಮೂವ-ತ್ತೊಂ-ದನೇ
ಮೂವ-ರನ್ನು
ಮೂವ-ರನ್ನೂ
ಮೂವ-ರಲ್ಲಿ
ಮೂವರು
ಮೃಗ
ಮೃಗ-ಗ-ಳಂತೆ
ಮೃಗ-ಗ-ಳ-ನ್ನಾ-ಗಿಯೇ
ಮೃಗ-ಗಳನ್ನು
ಮೃಗ-ಗ-ಳಲ್ಲ
ಮೃಗ-ಗಳು
ಮೃಗ-ಜಲ
ಮೃಗ-ಜ-ಲದ
ಮೃಗ-ದಂತೆ
ಮೃಗ-ವನ್ನು
ಮೃಗವೋ
ಮೃಗ-ಸ-ದೃ-ಶ-ರಾ-ಗಿ-ರು-ವ-ವ-ರನ್ನು
ಮೃತ-ನಾ-ದಾಗ
ಮೃತ-ಪ್ರಾ-ಯ-ರಾ-ಗಿ-ದ್ದಾ-ರೆಂ-ದರೆ
ಮೃತ-ಪ್ರಾ-ಯ-ರಾ-ಗಿ-ದ್ದೇವೆ
ಮೃತ್ಯು-ಕೂ-ಪ-ದಲ್ಲಿ
ಮೃತ್ಯು-ಘಾ-ತ-ವ-ನ್ನೀ-ಯು-ವು-ದೆಂದು
ಮೃತ್ಯು-ನ-ರ್ತ-ನ-ವನ್ನು
ಮೃತ್ಯು-ಭೀ-ತಿ-ಪೀ-ಡಿ-ತ-ರಾದ
ಮೃತ್ಯು-ಭೀ-ತಿ-ಯಿಂದ
ಮೃತ್ಯು-ವ-ಶ-ರಾ-ದರು
ಮೃತ್ಯು-ವಿಗೂ
ಮೃತ್ಯು-ವಿನ
ಮೃತ್ಯು-ವಿ-ನಿಂದ
ಮೃತ್ಯು-ವೆಂ-ಬುದೂ
ಮೃತ್ಯು-ಸ್ವ-ರೂ-ಪ-ವಾ-ಗಿದ್ದು
ಮೃದು
ಮೃದು-ವಾಗಿ
ಮೃದು-ವಾ-ಗಿ-ಸೂ-ಕ್ಷ್ಮ-ವಾ-ಗಿ-ರುವ
ಮೃದು-ವಾ-ಗಿ-ದ್ದಾರೆ
ಮೃದು-ವಾ-ಗಿ-ರುತ್ತಿ
ಮೃದು-ವಾ-ಗುತ್ತ
ಮೃದು-ವಾ-ದ-ಸೂ-ಕ್ಷ್ಮ-ವಾದ
ಮೃದು-ಹೃ-ದಯ
ಮೆಂಫಿಸ್
ಮೆಂಫಿ-ಸ್ನಲ್ಲಿ
ಮೆಂಫಿ-ಸ್ನಿಂದ
ಮೆಕ-ಲ್ಲಾ-ಡಳೂ
ಮೆಕಾ-ಲಾಡ್
ಮೆಕ್
ಮೆಕ್ಕಾ
ಮೆಕ್ಕಿಂಡ್ಲಿ
ಮೆಕ್ಕಿಂ-ಡ್ಲಿ-ಇ-ವರು
ಮೆಕ್ಕಿಂ-ಡ್ಲಿಯ
ಮೆಕ್ಲಾ-ಡಳ
ಮೆಕ್ಲಾ-ಡ-ಳನ್ನು
ಮೆಕ್ಲಾ-ಡ-ಳಿಗೆ
ಮೆಕ್ಲಾ-ಡಳು
ಮೆಕ್ಲಾಡ್
ಮೆಕ್ಲಾಡ್ನ್ನೂ
ಮೆಕ್ಸಿಕೋ
ಮೆಚ್ಚ-ಬ-ಲ್ಲರೆ
ಮೆಚ್ಚಿ
ಮೆಚ್ಚಿ-ಕೊಂಡ
ಮೆಚ್ಚಿ-ಕೊಂ-ಡರು
ಮೆಚ್ಚಿ-ಕೊಂ-ಡಾರು
ಮೆಚ್ಚಿ-ಕೊಂ-ಡಿತು
ಮೆಚ್ಚಿ-ಕೊಂ-ಡಿದ್ದ
ಮೆಚ್ಚಿ-ಕೊಂ-ಡಿ-ದ್ದರು
ಮೆಚ್ಚಿ-ಕೊಂಡು
ಮೆಚ್ಚಿ-ಕೊಂಡೆ
ಮೆಚ್ಚಿ-ಕೊ-ಳ್ಳ-ದಿ-ರಲು
ಮೆಚ್ಚಿ-ಕೊ-ಳ್ಳು-ತ್ತಾರೆ
ಮೆಚ್ಚಿ-ಕೊ-ಳ್ಳು-ತ್ತಿ-ದ್ದಾರೆ
ಮೆಚ್ಚಿ-ಕೊ-ಳ್ಳು-ತ್ತೀಯೋ
ಮೆಚ್ಚಿ-ಕೊ-ಳ್ಳುವ
ಮೆಚ್ಚಿ-ಕೊ-ಳ್ಳು-ವಂತೆ
ಮೆಚ್ಚಿ-ಕೊ-ಳ್ಳು-ವು-ದ-ರಿಂದ
ಮೆಚ್ಚಿ-ಕೊ-ಳ್ಳು-ವುದು
ಮೆಚ್ಚಿ-ಗೆ-ಯನ್ನು
ಮೆಚ್ಚಿ-ದರು
ಮೆಚ್ಚಿ-ದ್ದರು
ಮೆಚ್ಚಿನ
ಮೆಚ್ಚಿ-ನ-ವನು
ಮೆಚ್ಚಿ-ಸ-ಬಲ್ಲ
ಮೆಚ್ಚುಗೆ
ಮೆಚ್ಚು-ಗೆ-ಗಳನ್ನು
ಮೆಚ್ಚು-ಗೆ-ಗಿಂತ
ಮೆಚ್ಚು-ಗೆಗೆ
ಮೆಚ್ಚು-ಗೆಯ
ಮೆಚ್ಚು-ಗೆ-ಯನ್ನು
ಮೆಚ್ಚು-ಗೆ-ಯನ್ನೂ
ಮೆಚ್ಚು-ಗೆ-ಯಲ್ಲಿ
ಮೆಚ್ಚು-ಗೆ-ಯಾ-ದ್ದ-ದೇ-ಕೆಂದು
ಮೆಚ್ಚು-ಗೆ-ಯಿಂದ
ಮೆಚ್ಚು-ಗೆ-ಯಿತ್ತು
ಮೆಚ್ಚು-ಗೆಯೂ
ಮೆಟಾ-ಫಿ-ಸಿ-ಕಲ್
ಮೆಟ್ಕಾಫ್
ಮೆಟ್ಟಲು
ಮೆಟ್ಟ-ಲು-ಗ-ಳ-ನ್ನೇರಿ
ಮೆಟ್ಟಿ
ಮೆಟ್ಟಿ-ದ್ದಾರೆ
ಮೆಟ್ಟಿ-ಮು-ರಿ-ಯಲು
ಮೆಟ್ಟಿಲ
ಮೆಟ್ಟಿ-ಲಷ್ಟೆ
ಮೆಟ್ಟಿ-ಲಿ-ಳಿದು
ಮೆಟ್ಟಿ-ಲು-ಗಳ
ಮೆಟ್ಟಿ-ಲು-ಗಳು
ಮೆಟ್ಟಿ-ಲು-ಮೆ-ಟ್ಟಿ-ಲಾಗಿ
ಮೆಡೋಸ್
ಮೆಡೋ-ಸ್ನಲ್ಲಿ
ಮೆಣ-ಸಿ-ನ-ಕಾಯಿ
ಮೆಣ-ಸಿ-ನ-ಕಾ-ಯಿ-ಗಳು
ಮೆತ್ತಿ
ಮೆದು-ಳನ್ನು
ಮೆದು-ಳಿ-ನಲ್ಲಿ
ಮೆದು-ಳಿ-ನಲ್ಲೂ
ಮೆದು-ಳಿ-ನಿಂದ
ಮೆದು-ಳಿ-ನೊ-ಳಕ್ಕೆ
ಮೆದುಳು
ಮೆದು-ಳು-ನ-ರ-ಗ-ಳಿಗೆ
ಮೆದು-ಳು-ನ-ರ-ಮಂ-ಡ-ಲ-ಮ-ನ-ಸ್ಸು-ಗಳ
ಮೆನನ್
ಮೆನ-ನ್ನ-ರಿಗೆ
ಮೆನ-ನ್ನರು
ಮೆರ-ವ-ಣಿಗೆ
ಮೆರ-ವ-ಣಿ-ಗೆ-ಗಳನ್ನೂ
ಮೆರ-ವ-ಣಿ-ಗೆ-ಯಲ್ಲಿ
ಮೆರ-ಸ-ಬೇ-ಕಾ-ಗಿದೆ
ಮೆರಿ
ಮೆರು-ಗನ್ನು
ಮೆರೆ-ದ-ವಳು
ಮೆರೆ-ಯಿ-ಸು-ವುದು
ಮೆರೆ-ಯು-ತ್ತಿ-ರುವ
ಮೆರೆ-ಸಿದ
ಮೆರೆಸು
ಮೆರೆ-ಸುತ್ತ
ಮೆಲಕು
ಮೆಲ-ಕು-ಹಾ-ಕು-ವುದು
ಮೆಲುಕು
ಮೆಲು-ಕು-ಹಾ-ಕುತ್ತ
ಮೆಲು-ದ-ನಿ-ಯಲ್ಲಿ
ಮೆಲ್ಲನೆ
ಮೆಹ-ಬೂಬ್
ಮೇ
ಮೇಘ-ಗಳ
ಮೇಘ-ನಾದ
ಮೇಜರ್
ಮೇಜಿನ
ಮೇಜು
ಮೇಡಂ
ಮೇಡ-ನ್ಹೆಡ್
ಮೇಡಮ್
ಮೇಣದ
ಮೇಧಾವಿ
ಮೇಧಾ-ವಿ-ಗಳನ್ನು
ಮೇಧಾ-ವಿ-ಗ-ಳಾದ
ಮೇಧಾ-ವಿ-ಗಳಿ
ಮೇಧಾ-ವಿ-ಗ-ಳಿ-ಗಿಂ-ತಲೂ
ಮೇಧಾ-ಶಕ್ತಿ
ಮೇಧಾ-ಶ-ಕ್ತಿ-ಯನ್ನು
ಮೇಯರ್
ಮೇರಿ
ಮೇರಿಗೆ
ಮೇರಿ-ಫಂಕೆ
ಮೇರಿಯ
ಮೇರೆ
ಮೇರೆ-ಗಳನ್ನೂ
ಮೇರೆಗೆ
ಮೇರೆಗೇ
ಮೇರೆ-ಯನ್ನೇ
ಮೇರೇ
ಮೇಲಂತೂ
ಮೇಲಕ್ಕೆ
ಮೇಲಣ
ಮೇಲ-ಧಿ-ಕಾರಿ
ಮೇಲ-ಧಿ-ಕಾ-ರಿ-ಗ-ಳಿಗೆ
ಮೇಲ-ಧಿ-ಕಾ-ರಿ-ಗಳು
ಮೇಲಲ್ಲ
ಮೇಲ-ಲ್ಲವೆ
ಮೇಲಷ್ಟೇ
ಮೇಲಾ-ಗಲಿ
ಮೇಲಾ-ದರೂ
ಮೇಲಿಂದ
ಮೇಲಿ-ಕಾ-ದವ್ನು
ಮೇಲಿ-ಟ್ಟಿದ್ದ
ಮೇಲಿತ್ತು
ಮೇಲಿದ್ದ
ಮೇಲಿನ
ಮೇಲಿ-ನ-ವ-ರನ್ನು
ಮೇಲಿ-ನಿಂದ
ಮೇಲಿ-ರಿ-ಸಿ-ರುವ
ಮೇಲಿ-ರುವ
ಮೇಲು
ಮೇಲು-ನಾವು
ಮೇಲೂ
ಮೇಲೂ-ಅ-ದ-ರಲ್ಲೂ
ಮೇಲೆ
ಮೇಲೆ-ಅದೇ
ಮೇಲೆ-ಎ-ಲ್ಲರ
ಮೇಲೆ-ತೋ-ರುವ
ಮೇಲೆ-ತ್ತ-ಬಲ್ಲ
ಮೇಲೆ-ತ್ತ-ಬ-ಲ್ಲಿರಾ
ಮೇಲೆ-ತ್ತ-ಬೇಕು
ಮೇಲೆ-ತ್ತ-ಬೇ-ಕೆಂಬ
ಮೇಲೆ-ತ್ತ-ಬೇ-ಕೆಂ-ಬುದು
ಮೇಲೆ-ತ್ತಲು
ಮೇಲೆತ್ತಿ
ಮೇಲೆ-ತ್ತಿ-ದರು
ಮೇಲೆ-ತ್ತುವ
ಮೇಲೆ-ತ್ತು-ವುದು
ಮೇಲೆ-ದ್ದಿತು
ಮೇಲೆದ್ದು
ಮೇಲೆ-ಬ್ಬಿ-ಸಿ-ಇ-ದೊಂದೇ
ಮೇಲೆ-ಬ್ಬಿ-ಸಿದ
ಮೇಲೆ-ಮೇ-ಲೇರಿ
ಮೇಲೆಯೂ
ಮೇಲೆಯೇ
ಮೇಲೆಲ್ಲ
ಮೇಲೇ
ಮೇಲೇನೂ
ಮೇಲೇರ
ಮೇಲೇ-ರಲು
ಮೇಲೇರಿ
ಮೇಲೇ-ರಿ-ದಂತೆ
ಮೇಲೇ-ಳಲು
ಮೇಲೇಳು
ಮೇಲೊಂ-ದ-ರಂತೆ
ಮೇಲೊಂದು
ಮೇಲ್ಜಾತಿ
ಮೇಲ್ಜಾ-ತಿಯ
ಮೇಲ್ಜಾ-ತಿ-ಯ-ವರ
ಮೇಲ್ನೋ-ಟಕ್ಕೆ
ಮೇಲ್ನೋ-ಟದ
ಮೇಲ್ಪಂಕ್ತಿ
ಮೇಲ್ಪಂ-ಕ್ತಿ-ಯನ್ನು
ಮೇಲ್ಭಾಗ
ಮೇಲ್ಭಾ-ಗವೂ
ಮೇಲ್ಮೇಲೆ
ಮೇಲ್ಮೈ
ಮೇಲ್ಮೈ-ನೋ-ಟ-ವನ್ನೂ
ಮೇಲ್ಮೈಯ
ಮೇಲ್ಮೈ-ಯನ್ನು
ಮೇಲ್ವರ್ಗ
ಮೇಲ್ವ-ರ್ಗಕ್ಕೆ
ಮೇಲ್ವಿ-ಚಾ-ರ-ಕ-ನಾದ
ಮೇಲ್ವಿ-ಚಾ-ರ-ಕ-ರಷ್ಟೆ
ಮೇಲ್ವಿ-ಚಾ-ರ-ಕರು
ಮೇಲ್ವಿ-ಚಾ-ರ-ಕಿ-ಯಾದ
ಮೇಲ್ವಿ-ಚಾ-ರ-ಣೆ-ಯೊಂ-ದಿಗೆ
ಮೇಲ್ವಿ-ಚಾ-ರಿ-ಕಿಯೂ
ಮೇಳಕ್ಕೆ
ಮೇಳದ
ಮೇಳ-ದಲ್ಲಿ
ಮೇಳ-ವನ್ನು
ಮೇಳ-ವೊಂ-ದನ್ನು
ಮೇಳ-ವೊಂದು
ಮೇವು
ಮೈಕ-ಟ್ಟನ್ನೂ
ಮೈಕ-ಟ್ಟಿನ
ಮೈಕಟ್ಟು
ಮೈಕೈ
ಮೈಗೂ-ಡಿಸಿ
ಮೈಗೂ-ಡಿ-ಸಿ-ಕೊಂ-ಡಂತೆ
ಮೈಗೂ-ಡಿ-ಸಿ-ಕೊಂಡು
ಮೈಗೆ
ಮೈತಿ-ಳಿ-ದೆದ್ದ
ಮೈತಿ-ಳಿ-ದೆ-ದ್ದಾಗ
ಮೈತಿ-ಳಿ-ಯು-ವಂತೆ
ಮೈತ್ರೇ-ಯಿ-ಯರು
ಮೈದ-ಳೆದ
ಮೈದಾನ
ಮೈದಾ-ನ-ದಲ್ಲಿ
ಮೈದಾ-ನ-ವೆಂದರೆ
ಮೈದಾಳಿ
ಮೈದಾ-ಳಿ-ರುವ
ಮೈದಾ-ಳು-ವುದನ್ನು
ಮೈನು-ದ್ದೀನ್
ಮೈಬಣ್ಣ
ಮೈಬ-ಣ್ಣ-ದಲ್ಲಿ
ಮೈಮ-ನ-ಗಳು
ಮೈಮ-ರೆ-ತರು
ಮೈಮ-ರೆ-ತಿದ್ದ
ಮೈಮ-ರೆತು
ಮೈಮ-ರೆ-ಯುತ್ತ
ಮೈಮ-ರೆ-ಯುತ್ತಿ
ಮೈಮ-ರೆ-ಯು-ವುದನ್ನು
ಮೈಮೇಲೆ
ಮೈಯ-ರ್ಸ್
ಮೈಯೆಲ್ಲ
ಮೈಲಿ
ಮೈಲಿ-ಗ-ಟ್ಟಲೆ
ಮೈಲಿ-ಗಳ
ಮೈಲಿ-ಗಳನ್ನು
ಮೈಲಿ-ಗ-ಳ-ವರೆ-ಗಿನ
ಮೈಲಿ-ಗ-ಳಷ್ಟು
ಮೈಲಿ-ಗಳು
ಮೈಲಿಗೂ
ಮೈಲಿ-ಗೆ-ಯಾ-ಗಲು
ಮೈಲಿಯ
ಮೈಲಿ-ಯ-ವರೆ-ಗಾ-ದರೂ
ಮೈಲಿ-ಯ-ವ-ರೆಗೆ
ಮೈಲೋ
ಮೈಸೂ-ರಿನ
ಮೈಸೂ-ರಿ-ನಲ್ಲಿ
ಮೈಸೂ-ರಿ-ನ-ಲ್ಲಿ-ದ್ದಾ-ಗಲೇ
ಮೈಸೂ-ರಿ-ನಿಂದ
ಮೈಸೂರು
ಮೊಂಡು
ಮೊಂಡು-ತ-ನ-ವನ್ನು
ಮೊಕ-ದ್ದಮೆ
ಮೊಗ-ದ-ಲ್ಲೊಂದು
ಮೊಘಲ್
ಮೊಟ್ಟ
ಮೊಟ್ಟ-ಮೊ-ದಲ
ಮೊಟ್ಟ-ಮೊ-ದ-ಲ-ನೆ-ಯ-ದಾಗಿ
ಮೊಟ್ಟ-ಮೊ-ದ-ಲ-ಬಾ-ರಿಗೆ
ಮೊಟ್ಟ-ಮೊ-ದ-ಲಿಗೆ
ಮೊಟ್ಟ-ಮೊ-ದಲು
ಮೊತ್ತ
ಮೊತ್ತ-ಮೊ-ದಲು
ಮೊತ್ತ-ವನ್ನು
ಮೊತ್ತವೇ
ಮೊತ್ತ-ವೊಂ-ದನ್ನು
ಮೊದ
ಮೊದ-ಮೊ-ದಲ
ಮೊದ-ಮೊ-ದ-ಲಿಗೆ
ಮೊದ-ಮೊ-ದಲು
ಮೊದಲ
ಮೊದ-ಲ-ನೆಯ
ಮೊದ-ಲ-ನೆ-ಯ-ದಾಗಿ
ಮೊದ-ಲ-ನೆ-ಯದು
ಮೊದ-ಲ-ನೆ-ಯಾ-ದಗಿ
ಮೊದ-ಲನೇ
ಮೊದ-ಲ-ಬಾರಿ
ಮೊದ-ಲ-ಬಾ-ರಿಗೆ
ಮೊದ-ಲ-ಸಲ
ಮೊದ-ಲಾದ
ಮೊದ-ಲಾ-ದ-ವನ್ನು
ಮೊದ-ಲಾ-ದ-ವರ
ಮೊದ-ಲಾ-ದ-ವ-ರಿ-ದ್ದರು
ಮೊದ-ಲಾ-ದ-ವರು
ಮೊದ-ಲಾ-ದ-ವರೂ
ಮೊದ-ಲಾ-ದ-ವ-ರೊಂ-ದಿ-ಗಿನ
ಮೊದ-ಲಾ-ದ-ವ-ರೊಂ-ದಿಗೆ
ಮೊದ-ಲಾ-ದ-ವು-ಗಳನ್ನು
ಮೊದ-ಲಾ-ದು-ವನ್ನು
ಮೊದ-ಲಾ-ದು-ವು-ಗಳನ್ನೆಲ್ಲ
ಮೊದ-ಲಾ-ದು-ವು-ಗಳಲ್ಲಿ
ಮೊದ-ಲಿ-ಗ-ರಲ್ಲಿ
ಮೊದ-ಲಿ-ಗರು
ಮೊದ-ಲಿ-ಗ-ರೇ-ನಾ-ಗ-ರ-ಲಿಲ್ಲ
ಮೊದ-ಲಿ-ಗಿಂತ
ಮೊದ-ಲಿ-ಗಿಂ-ತಲೂ
ಮೊದ-ಲಿಗೆ
ಮೊದ-ಲಿ-ಗೇನೋ
ಮೊದ-ಲಿನ
ಮೊದ-ಲಿ-ನಿಂದ
ಮೊದ-ಲಿ-ನಿಂ-ದಲೂ
ಮೊದ-ಲಿ-ಯಾರ್
ಮೊದಲು
ಮೊದಲೂ
ಮೊದಲೇ
ಮೊದ-ಲೇ-ತಾವು
ಮೊದ-ಲ್ಗೊಂಡು
ಮೊನೆ-ಯಂ-ತಹ
ಮೊನ್ನೆ-ಮೊನ್ನೆ
ಮೊನ್ನೆ-ಯಿಂದ
ಮೊಮ್ಮ-ಗಳ
ಮೊಮ್ಮ-ಗಳೂ
ಮೊಯಿನ್ಸ್ಗೆ
ಮೊಯಿ-ನ್ಸ್ನಲ್ಲಿ
ಮೊರಾರ್ಜಿ
ಮೊರೆ
ಮೊರೆ-ಯಿ-ಟ್ಟುವು
ಮೊರೆ-ಯಿ-ಡು-ತ್ತಿ-ದ್ದರು
ಮೊರೆ-ಯು-ತ್ತಿ-ದ್ದಂತೆ
ಮೊಳ-ಕಾ-ಲಿ-ನ-ವ-ರೆಗೆ
ಮೊಳ-ಕೆ-ಯೊ-ಡೆದು
ಮೊಳ-ಗ-ಬೇಕು
ಮೊಳ-ಗ-ಲಿ-ರುವ
ಮೊಳಗಿ
ಮೊಳ-ಗಿತು
ಮೊಳ-ಗಿದ
ಮೊಳ-ಗಿ-ದರು
ಮೊಳ-ಗಿ-ದುವು
ಮೊಳಗು
ಮೊಳ-ಗು-ತ್ತಾನೆ
ಮೊಳ-ಗು-ತ್ತಿದ್ದ
ಮೊಳಗೆ
ಮೊಳೆತು
ಮೊಳೆ-ಯಲು
ಮೊಸಳೆ
ಮೊಸ-ಳೆಯ
ಮೊಹುವಾ
ಮೋಕ್ಷ
ಮೋಕ್ಷದ
ಮೋಜೆ-ನಿ-ಸಿ-ದರೂ
ಮೋಡಿಗೆ
ಮೋಡಿ-ಗೊ-ಳ-ಗಾಗಿ
ಮೋಡಿ-ಗೊ-ಳ-ಗಾದ
ಮೋಡಿ-ಗೊ-ಳಿ-ಸುವ
ಮೋಡಿ-ಯಿಂದ
ಮೋಡಿ-ಯಿ-ದ್ದಿ-ರ-ಬೇಕು
ಮೋತೀ-ಲಾ-ಲರ
ಮೋದಿ-ಸು-ತ್ತ-ದಲ್ಲ
ಮೋರೆ
ಮೋಸ
ಮೋಸ-ಮೂ-ಢ-ನಂ-ಬಿ-ಕೆ-ಗಳು
ಮೋಸ-ವಂ-ಚನೆ
ಮೋಸ-ಸೋ-ಗು-ವಂ-ಚ-ನೆ-ಗಳ
ಮೋಸಕ್ಕೆ
ಮೋಸ-ಗಾ-ರ-ನಲ್ಲ
ಮೋಸ-ಗಾ-ರ-ನ-ಲ್ಲ-ವೆಂ-ಬು-ದನ್ನು
ಮೋಸ-ಗಾ-ರರು
ಮೋಸ-ಗೊ-ಳಿ-ಸು-ವು-ದ-ರಿಂ-ದಲೂ
ಮೋಸ-ಗೊ-ಳಿ-ಸು-ವುದು
ಮೋಸದ
ಮೋಸ-ದಿಂದ
ಮೋಸ-ದಿಂ-ದಾ-ಗಲಿ
ಮೋಸ-ಮಾ-ಡಲು
ಮೋಸ-ವಾ-ಗಲಿ
ಮೋಸ-ವಿಲ್ಲ
ಮೋಸ-ಹೋ-ಗ-ಲಾರೆ
ಮೋಹ-ದಾಚೆ
ಮೋಹನ
ಮೋಹನ್
ಮೋಹವು
ಮೋಹಿ-ತ-ರಾಗಿ
ಮೌಂಟೆನ್ಸ್ನ
ಮೌಂಟ್
ಮೌಢ್ಯ-ಕ್ಕಾಗಿ
ಮೌಢ್ಯಕ್ಕೆ
ಮೌಢ್ಯ-ಗಳನ್ನೆಲ್ಲ
ಮೌಢ್ಯ-ವನ್ನು
ಮೌನ
ಮೌನ-ಬಿ-ಗು-ಮಾ-ನ-ಅ-ರೆ-ಲೆ-ಕ್ಕಾ-ಚಾ-ರದ
ಮೌನ-ದಲ್ಲಿ
ಮೌನ-ದಾ-ಳ-ದಲ್ಲಿ
ಮೌನ-ದಿಂ-ದಾಗಿ
ಮೌನ-ರಾ-ದರು
ಮೌನ-ವನ್ನು
ಮೌನ-ವ-ವಾಗಿ
ಮೌನ-ವಾಗಿ
ಮೌನ-ವಾ-ಗಿದ್ದ
ಮೌನ-ವಾ-ಗಿದ್ದು
ಮೌನ-ವಾ-ಗಿ-ದ್ದು-ಬಿ-ಟ್ಟರು
ಮೌನ-ವಾ-ಗಿ-ದ್ದು-ಬಿ-ಡು-ತ್ತಿ-ದ್ದರು
ಮೌನ-ವಾ-ಗಿ-ರ-ಬ-ಹುದು
ಮೌನ-ವೊಂದೇ
ಮೌಲ್ಯ-ಗಳನ್ನು
ಮೌಲ್ಯ-ಗಳನ್ನೂ
ಮೌಲ್ಯ-ಗಳು
ಮೌಲ್ಯ-ವನ್ನು
ಮೌಲ್ವಿ
ಮೌಲ್ವಿಗೆ
ಮೌಲ್ವಿಯ
ಮೌಲ್ವಿ-ಯನ್ನು
ಮೌಲ್ವೀ
ಮೌಲ್ವೀ-ಸಾ-ಹೇಬ
ಮೌಲ್ವೀ-ಸಾ-ಹೇ-ಬ-ನನ್ನು
ಮೌಲ್ವೀ-ಸಾ-ಹೇ-ಬರೇ
ಮ್ಯಾಕ್ಸಿಮ್
ಮ್ಯಾಕ್ಸ್
ಮ್ಯಾಕ್ಸ್ಮಿ-ಲನ್
ಮ್ಯಾಕ್ಸ್ಮು-ಲ್ಲರ್
ಮ್ಯಾಕ್ಸ್ಮ-್ಯು-ಲ್ಲರ್
ಮ್ಯಾಗ್ನೋ
ಮ್ಯಾಗ್ನೋ-ಲಿಯಾ
ಮ್ಯಾಡಿ-ಸನ್
ಮ್ಯಾನ-ರಿಗೆ
ಮ್ಯಾನ-ರಿ-ನಲ್ಲಿ
ಮ್ಯಾನ-ರಿ-ನಿಂದ
ಮ್ಯಾನರ್
ಮ್ಯಾನರ್ಗೆ
ಮ್ಯಾನೇ-ಜ-ರನೂ
ಮ್ಯಾನೇ-ಜ-ರನ್ನು
ಮ್ಯಾನೇ-ಜ-ರಿ-ನಿಂ-ದಲೂ
ಮ್ಯಾನೇ-ಜರ್
ಮ್ಯಾಲೆ
ಮ್ಯು
ಮ್ಲಾನ-ವ-ದ-ನ-ರಾಗಿ
ಮ್ಲೇಚ್ಛ
ಮ್ಲೇಚ್ಛನು
ಮ್ಲೇಚ್ಛರ
ಮ್ಲೇಚ್ಛರು
ಮ್ಲೇಚ್ಛ-ರೆಂದು
ಯಂತಹ
ಯಂತೂ
ಯಂತೆ
ಯಂತೆಯೇ
ಯಂತ್ರ
ಯಂತ್ರ-ಗಳು
ಯಂತ್ರದ
ಯಂತ್ರೋ-ದ್ಯಮ
ಯಂಥ-ವ-ರಿ-ಗಾಗಿ
ಯಃಕ-ಶ್ಚಿತ್
ಯಃಕಿ-ಶ್ಚಿತ್
ಯಕ್ಷ
ಯಕ್ಷನ
ಯಕ್ಷ-ನಿದ್ದ
ಯಜ-ಮಾ-ನ-ನೆಂ-ದಲ್ಲ
ಯಜ-ಮಾ-ನಿಕೆ
ಯಜ-ಮಾ-ನಿ-ಯರು
ಯಜ-ಮಾನ್ರೆ
ಯಣ
ಯಣ್ಣ
ಯತಿ-ಮಾತಾ
ಯತ್ನಿ-ಸಿ-ದರು
ಯತ್ನಿ-ಸಿ-ದರೂ
ಯತ್ನಿ-ಸಿ-ದಳು
ಯಥಾಂ
ಯಥಾ-ಪ್ರ-ಕಾರ
ಯಥಾ-ರ್ಥ-ವಾಗಿ
ಯಥಾ-ವ-ತ್ತಾಗಿ
ಯಥಾ-ಶಕ್ತಿ
ಯಥಾ-ಸ್ಥಾ-ನ-ದಲ್ಲಿ
ಯಥೇ-ಚ್ಛ-ವಾಗಿ
ಯದಾಗಿ
ಯದು
ಯನ
ಯನ್ನನ
ಯನ್ನ-ರಿ-ಯಲು
ಯನ್ನಷ್ಟೇ
ಯನ್ನಾ-ಗಿ-ಸು-ತ್ತದೆ
ಯನ್ನು
ಯನ್ನು-ಅ-ದರ
ಯನ್ನುಂ-ಟು-ಮಾ-ಡಿದೆ
ಯನ್ನೂ
ಯನ್ನೇ
ಯನ್ನೇನೂ
ಯನ್ನೊ-ದ-ಗಿ-ಸು-ತ್ತಿ-ದ್ದುವು
ಯನ್ನೋ
ಯರ
ಯರಲ್ಲಿ
ಯರಿಗೆ
ಯರಿತು
ಯರು
ಯರ್ಳನ್ನು
ಯಲು
ಯಲ್ಲಿ
ಯಲ್ಲಿದೆ
ಯಲ್ಲಿದ್ದ
ಯಲ್ಲಿ-ದ್ದರೆ
ಯಲ್ಲಿ-ದ್ದು-ವೆಂ-ಬು-ದನ್ನು
ಯಲ್ಲೂ
ಯಲ್ಲೇ
ಯವ-ನನ್ನು
ಯವನು
ಯವರ
ಯಶಸ್ವಿ
ಯಶ-ಸ್ವಿ-ಗ-ಳಾಗಿ
ಯಶ-ಸ್ವಿ-ಗ-ಳಾ-ಗಿ-ದ್ದರು
ಯಶ-ಸ್ವಿ-ಗ-ಳಾ-ಗು-ತ್ತಿ-ದ್ದ-ರೆಂದೂ
ಯಶ-ಸ್ವಿ-ಗ-ಳಾ-ಗು-ವುದ
ಯಶ-ಸ್ವಿ-ಗ-ಳಾ-ದರೆ
ಯಶ-ಸ್ವಿ-ಗೊ-ಳಿ-ಸ-ಬೇ-ಕಾ-ಗಿದೆ
ಯಶ-ಸ್ವಿ-ಯಾ-ಗ-ಬೇ-ಕಾ-ದರೆ
ಯಶ-ಸ್ವಿ-ಯಾ-ಗ-ಲಿಲ್ಲ
ಯಶ-ಸ್ವಿ-ಯಾ-ಗಲು
ಯಶ-ಸ್ವಿ-ಯಾಗಿ
ಯಶ-ಸ್ವಿ-ಯಾ-ಗಿತ್ತು
ಯಶ-ಸ್ವಿ-ಯಾ-ಗಿದೆ
ಯಶ-ಸ್ವಿ-ಯಾ-ಗಿ-ದ್ದರು
ಯಶ-ಸ್ವಿ-ಯಾ-ಗಿದ್ದು
ಯಶ-ಸ್ವಿ-ಯಾ-ಗಿಯೇ
ಯಶ-ಸ್ವಿ-ಯಾ-ಗಿಲ್ಲ
ಯಶ-ಸ್ವಿ-ಯಾ-ಗಿ-ಸಿ-ದು-ದನ್ನು
ಯಶ-ಸ್ವಿ-ಯಾಗು
ಯಶ-ಸ್ವಿ-ಯಾ-ಗು-ತ್ತದೆ
ಯಶ-ಸ್ವಿ-ಯಾ-ಗು-ತ್ತೀಯೋ
ಯಶ-ಸ್ವಿ-ಯಾ-ಗು-ತ್ತೇ-ನೆಯೋ
ಯಶ-ಸ್ವಿ-ಯಾ-ಗು-ತ್ತೇವೆ
ಯಶ-ಸ್ವಿ-ಯಾ-ಗು-ವು-ದ-ರಲ್ಲಿ
ಯಶ-ಸ್ವಿ-ಯಾ-ಗು-ವುದು
ಯಶ-ಸ್ವಿ-ಯಾದ
ಯಶ-ಸ್ವಿ-ಯಾ-ದರು
ಯಶ-ಸ್ವಿ-ಯಾ-ದರೆ
ಯಶ-ಸ್ವಿ-ಯಾ-ದ-ರೆಂದು
ಯಶ-ಸ್ವಿ-ಯಾ-ದ-ರೆಂ-ಬುದು
ಯಶ-ಸ್ವಿ-ಯಾ-ದುವು
ಯಶ-ಸ್ವಿ-ಯಾ-ದೇನೋ
ಯಶ-ಸ್ವಿ-ಯಾ-ಯಿತು
ಯಶ-ಸ್ವಿ-ಯಾ-ಯಿತೋ
ಯಶಸ್ವೀ
ಯಶ-ಸ್ಸನ್ನು
ಯಶ-ಸ್ಸನ್ನೂ
ಯಶ-ಸ್ಸಿ-ಗಿಂತ
ಯಶ-ಸ್ಸಿಗೆ
ಯಶ-ಸ್ಸಿನ
ಯಶ-ಸ್ಸಿ-ನಲ್ಲಿ
ಯಶ-ಸ್ಸಿ-ನಿಂದ
ಯಶಸ್ಸು
ಯಶ-ಸ್ಸು-ಕಾ-ರ್ಯ-ಸಿ-ದ್ಧಿ-ಗ-ಳೆಲ್ಲ
ಯಶ-ಸ್ಸು-ಕೀ-ರ್ತಿ-ಗಳ
ಯಶಸ್ಸೂ
ಯಶೋ-ದ-ಬಾ-ಯಿಗೆ
ಯಸ್ಥ
ಯಹೂ-ದ್ಯ-ಧ-ರ್ಮಕ್ಕೂ
ಯಹೂ-ದ್ಯ-ನಾ-ಗಿ-ದ್ದನು
ಯಹೂ-ದ್ಯನೆ
ಯಹೂ-ದ್ಯರ
ಯಹೂ-ದ್ಯರು
ಯಾಂತ್ರಿ-ಕ-ವಾಗಿ
ಯಾಂತ್ರೀಯ
ಯಾಕಾ-ದರೂ
ಯಾಕೆ
ಯಾಕೊಂದು
ಯಾಗ-ಲಾ-ರದು
ಯಾಗ-ಲಿದೆ
ಯಾಗ-ಲಿಲ್ಲ
ಯಾಗಲು
ಯಾಗಿ
ಯಾಗಿತ್ತು
ಯಾಗಿ-ದೆಯೇ
ಯಾಗಿದ್ದ
ಯಾಗಿ-ದ್ದರು
ಯಾಗಿ-ದ್ದರೂ
ಯಾಗಿ-ದ್ದಾನೆ
ಯಾಗಿದ್ದು
ಯಾಗಿ-ದ್ದು-ಕೊಂಡು
ಯಾಗಿ-ರು-ವಂತೆ
ಯಾಗು-ತ್ತದೆ
ಯಾಗು-ವಂತೆ
ಯಾಚ-ನೆಯ
ಯಾಚಿಸಿ
ಯಾಚಿ-ಸಿ-ದರು
ಯಾಚಿಸು
ಯಾಚಿ-ಸು-ತ್ತಿ-ದ್ದರು
ಯಾಜ್ಞ-ವ-ಲ್ಕ್ಯ-ರಿಂದ
ಯಾಜ್ಞ-ವ-ಲ್ಕ್ಯರು
ಯಾಡಿದ
ಯಾತನೆ
ಯಾತ-ನೆ-ಗ-ಳ-ನ್ನ-ನು-ಭ-ವಿಸು
ಯಾತ-ನೆ-ಯಾ-ಗು-ತ್ತಿತ್ತೋ
ಯಾತ್ರಾ-ರ್ಥಿ-ಗಳು
ಯಾತ್ರಾ-ರ್ಥಿ-ಯಾಗಿ
ಯಾತ್ರಾ-ಸ್ಥ-ಳಈ
ಯಾತ್ರಾ-ಸ್ಥ-ಳ-ಗಳನ್ನು
ಯಾತ್ರಾ-ಸ್ಥ-ಳ-ವಾಗಿ
ಯಾತ್ರಾ-ಸ್ಥ-ಳ-ವಾದ
ಯಾತ್ರಿ-ಕ-ನೊಬ್ಬ
ಯಾತ್ರೆ
ಯಾತ್ರೆಗೆ
ಯಾತ್ರೆಯ
ಯಾತ್ರೆ-ಯನ್ನು
ಯಾತ್ರೆ-ಯೆಂದೇ
ಯಾತ್ರೆ-ಯೊಂ-ದಿಗೆ
ಯಾದ
ಯಾದರು
ಯಾದರೂ
ಯಾದ-ರೆಂ-ಬುದು
ಯಾದ-ಳ-ಲ್ಲದೆ
ಯಾದಳು
ಯಾದ-ವ-ಗಿರಿ
ಯಾದ-ವರು
ಯಾದಷ್ಟೂ
ಯಾದಾ-ಗಲೂ
ಯಾದೀತು
ಯಾದೆ
ಯಾನ
ಯಾಯಿತು
ಯಾಯಿ-ತುಛೇ
ಯಾರ
ಯಾರ-ದ-ರೊ-ಬ್ಬರೂ
ಯಾರ-ನ್ನಾ-ದರೂ
ಯಾರನ್ನು
ಯಾರನ್ನೂ
ಯಾರನ್ನೋ
ಯಾರಪ್ಪ
ಯಾರಲ್ಲಿ
ಯಾರ-ವರು
ಯಾರಾ
ಯಾರಾದ
ಯಾರಾ-ದ-ರಿ-ದ್ದರೆ
ಯಾರಾ-ದರೂ
ಯಾರಾ-ದ-ರೇ-ನಂತೆ
ಯಾರಾ-ದ-ರೊ-ಬ್ಬರ
ಯಾರಾ-ದ-ರೊ-ಬ್ಬರು
ಯಾರಿಂದ
ಯಾರಿಂ-ದ-ಲಾ-ದರೂ
ಯಾರಿಂ-ದಲೂ
ಯಾರಿಂ-ದಲೋ
ಯಾರಿ-ಗಾ-ದರೂ
ಯಾರಿ-ಗಿ-ದ್ದೀತು
ಯಾರಿಗೂ
ಯಾರಿಗೆ
ಯಾರಿಗೇ
ಯಾರಿ-ಗೇ-ಕೆ-ಸ್ವತಃ
ಯಾರಿ-ಗೇನು
ಯಾರಿ-ಗೋ-ಸ್ಕರ
ಯಾರಿ-ದ್ದಾರು
ಯಾರಿ-ದ್ದಾರೆ
ಯಾರಿ-ರ-ಬ-ಹುದು
ಯಾರಿ-ರ-ಬ-ಹುದೋ
ಯಾರು
ಯಾರು-ಯಾರು
ಯಾರು-ಯಾ-ರೆಂ-ಬುದು
ಯಾರೂ
ಯಾರೆಂ-ದರೆ
ಯಾರೆಂದು
ಯಾರೆಂ-ಬುದು
ಯಾರೆಷ್ಟೇ
ಯಾರೇ
ಯಾರೊಂ-ದಿಗೂ
ಯಾರೊಂ-ದಿಗೆ
ಯಾರೊಂ-ದಿಗೋ
ಯಾರೊ-ಬ್ಬರೂ
ಯಾರೋ
ಯಾರ್ಕಿಗೆ
ಯಾರ್ಯಾರ
ಯಾರ್ಯಾ-ರಿಗೆ
ಯಾರ್ಯಾರು
ಯಾವ
ಯಾವನ
ಯಾವನಾ
ಯಾವ-ನಿಗೆ
ಯಾವನು
ಯಾವನೂ
ಯಾವನೋ
ಯಾವ-ಯಾವ
ಯಾವಾ
ಯಾವಾಗ
ಯಾವಾ-ಗ-ಬೇ-ಕಾ-ದರೂ
ಯಾವಾ-ಗಲಾ
ಯಾವಾ-ಗ-ಲಾ-ದರೂ
ಯಾವಾ-ಗ-ಲಾ-ದ-ರೊಮ್ಮೆ
ಯಾವಾ-ಗಲೂ
ಯಾವಾ-ಗೆಂ-ದರೆ
ಯಾವು
ಯಾವು-ದಕ್ಕೂ
ಯಾವು-ದಕ್ಕೆ
ಯಾವು-ದಕ್ಕೋ
ಯಾವುದನ್ನು
ಯಾವು-ದನ್ನೂ
ಯಾವು-ದನ್ನೇ
ಯಾವು-ದರ
ಯಾವು-ದ-ರಲ್ಲೂ
ಯಾವು-ದ-ರಿಂ-ದಲೂ
ಯಾವು-ದ-ರೊಂ-ದಿಗೂ
ಯಾವುದಾ
ಯಾವು-ದಾದ
ಯಾವು-ದಾ-ದರೂ
ಯಾವು-ದಾ-ದ-ರೊಂ-ದರ
ಯಾವು-ದಾ-ದ-ರೊಂದು
ಯಾವು-ದಿ-ರ-ಬ-ಹುದು
ಯಾವುದು
ಯಾವುದೂ
ಯಾವು-ದೆಂ-ದರೆ
ಯಾವು-ದೆಂದು
ಯಾವುದೇ
ಯಾವುದೋ
ಯಾವುವು
ಯಾವುವೂ
ಯಾವೊಂದು
ಯಾವೊಬ್ಬ
ಯಾಸ-ವಾಗಿ
ಯಿಂದ
ಯಿಂದಲೂ
ಯಿಂದಲೇ
ಯಿಂದಾಗಿ
ಯಿತು
ಯಿರಿ
ಯಿರಿಸಿ
ಯಿಲ್ಲ
ಯಿಲ್ಲದೆ
ಯು
ಯುಂಟಾ-ಗಿತ್ತು
ಯುಂಟಾ-ಯಿತು
ಯುಂಟು-ಮಾ-ಡ-ದೆಯೇ
ಯುಂಟು-ಮಾ-ಡಿ-ದ್ದರೆ
ಯುಕ್ತ
ಯುಕ್ತಾ-ಯು-ಕ್ತತೆ
ಯುಕ್ತಾ-ಯು-ಕ್ತ-ತೆ-ಗ-ಳೇನೇ
ಯುಗ
ಯುಗಕ್ಕೆ
ಯುಗ-ಗಳ
ಯುಗದ
ಯುಗ-ದಲ್ಲಿ
ಯುಗ-ಪ್ರ-ವ-ರ್ತ-ಕರು
ಯುಗ-ಯುಗ
ಯುಗ-ಯು-ಗ-ಗ-ಳಷ್ಟು
ಯುಗ-ಯು-ಗ-ಗ-ಳಿಂ-ದಲೂ
ಯುಗ-ಯು-ಗ-ಳಿಂ-ದಲೂ
ಯುಗ-ವನ್ನು
ಯುಗಾ-ವ-ತಾರ
ಯುತ
ಯುತ-ವಾಗಿ
ಯುತವೂ
ಯುತ್ತ
ಯುತ್ತಲೇ
ಯುತ್ತಿದ್ದ
ಯುದ್ಧ-ದಲ್ಲಿ
ಯುದ್ಧ-ನೌ-ಕೆ-ಗಳು
ಯುದ್ಧ-ವನ್ನು
ಯುದ್ಧ-ವನ್ನೇ
ಯುದ್ಧ-ವಲ್ಲ
ಯುದ್ಧ-ವಿ-ರೋ-ಧಿ-ಗ-ಳಾ-ಗಿದ್ದು
ಯುನಿ-ಟೇ-ರಿ-ಯನ್
ಯುವ
ಯುವಕ
ಯುವ-ಕ-ಯು-ವ-ತಿ-ಯ-ರಿಗೆ
ಯುವ-ಕನ
ಯುವ-ಕನು
ಯುವ-ಕ-ನೊಬ್ಬ
ಯುವ-ಕರ
ಯುವ-ಕ-ರನ್ನು
ಯುವ-ಕ-ರಲ್ಲಿ
ಯುವ-ಕ-ರ-ಲ್ಲೊಬ್ಬ
ಯುವ-ಕ-ರಿಂದ
ಯುವ-ಕ-ರಿ-ಗಾಗಿ
ಯುವ-ಕ-ರಿಗೆ
ಯುವ-ಕರು
ಯುವ-ಕ-ರು-ಮು-ದು-ಕರು
ಯುವ-ಕ-ರೆಲ್ಲ
ಯುವ-ಕರೇ
ಯುವ-ಜ-ನರ
ಯುವ-ಜ-ನರೆ
ಯುವ-ತಿ-ಯ-ರಿಗೆ
ಯುವ-ತಿ-ಯರು
ಯುವ-ತಿ-ಯರೇ
ಯುವ-ತಿ-ಯೆಂದು
ಯುವ-ಧ-ರ್ಮಾ-ಧಿ-ಕಾ-ರಿ-ಗಳ
ಯುವ-ಪು-ತ್ರರ
ಯುವ-ಭ-ಕ್ತ-ರಿಗೆ
ಯುವ-ಮಾ-ತೆ-ಯರ
ಯುವ-ವರೇ
ಯುವ-ಶಿ-ಷ್ಯ-ರ-ನ್ನೆಲ್ಲ
ಯುವ-ಸಂನ್ಯಾಸಿ
ಯುವ-ಸಂ-ನ್ಯಾ-ಸಿ-ಅ-ವರು
ಯುವ-ಸಂ-ನ್ಯಾ-ಸಿ-ಗ-ಳಿಗೆ
ಯುವ-ಸಂ-ನ್ಯಾ-ಸಿಯ
ಯುವ-ಸಂ-ನ್ಯಾ-ಸಿ-ಯನ್ನು
ಯುವ-ಸಂ-ನ್ಯಾ-ಸಿ-ಯಾದ
ಯುವುದು
ಯೂನಿ-ಟೇ-ರಿ-ಯನ್
ಯೂನಿ-ಯನ್
ಯೂರೋ
ಯೂರೋ-ಪಿಗೆ
ಯೂರೋ-ಪಿನ
ಯೂರೋ-ಪಿ-ನ-ಲ್ಲಿದ್ದ
ಯೂರೋ-ಪಿ-ನಲ್ಲೂ
ಯೂರೋ-ಪಿ-ನಲ್ಲೇ
ಯೂರೋಪು
ಯೂರೋ-ಪು-ಗಳಲ್ಲಿ
ಯೂರೋಪ್
ಯೆಂದರೆ
ಯೆಂದಾ-ಗಲಿ
ಯೆಂದು
ಯೆಂದೂ
ಯೆತ್ತಿ
ಯೆದ್ದಿತು
ಯೆಲ್ಲ
ಯೇ
ಯೇನೂ
ಯೇನೆಂ-ಬುದು
ಯೊಂದನ್ನು
ಯೊಂದರ
ಯೊಂದ-ರಲ್ಲಿ
ಯೊಂದಿಗೆ
ಯೊಂದಿ-ಗೇನೂ
ಯೊಂದು
ಯೊಕೊ-ಹಾಮ
ಯೊಕೊ-ಹಾ-ಮಕ್ಕೆ
ಯೊಕೊ-ಹಾ-ಮ-ದಲ್ಲಿ
ಯೊಕೊ-ಹಾ-ಮ-ದಲ್ಲೇ
ಯೊಕೊ-ಹಾ-ಮ-ದಿಂದ
ಯೊದ-ಗಿ-ದುದು
ಯೊಬ್ಬ
ಯೊಬ್ಬನ
ಯೊಬ್ಬ-ನಿಗೆ
ಯೊಬ್ಬರ
ಯೊಬ್ಬ-ರನ್ನು
ಯೊಬ್ಬ-ರನ್ನೂ
ಯೊಬ್ಬರೂ
ಯೊಳಗೆ
ಯೋಕೋ
ಯೋಗ
ಯೋಗಃ
ಯೋಗಕ್ಕೂ
ಯೋಗ-ಕ್ಷೇ-ಮದ
ಯೋಗ-ಕ್ಷೇ-ಮ-ವನ್ನು
ಯೋಗ-ಗಳ
ಯೋಗ-ಗ-ಳಿ-ದ್ದಿ-ದ್ದರೆ
ಯೋಗ-ಗ-ಳಿ-ವೆ-ಯಲ್ಲ
ಯೋಗದ
ಯೋಗ-ದಲ್ಲಿ
ಯೋಗ-ದೃ-ಷ್ಟಿ-ಯಲ್ಲಿ
ಯೋಗ-ಧ-ನು-ವಿಗೆ
ಯೋಗ-ಪು-ರುಷ
ಯೋಗ-ಮು-ದ್ರೆ-ಯಲ್ಲಿ
ಯೋಗ-ವನ್ನು
ಯೋಗವು
ಯೋಗ-ಶ-ಕ್ತಿ-ಗಳ
ಯೋಗ-ಶ-ಕ್ತಿ-ಗಳು
ಯೋಗ-ಶ-ಕ್ತಿ-ಯನ್ನು
ಯೋಗ-ಶ-ಕ್ತಿಯು
ಯೋಗ-ಸಿ-ದ್ಧಾಂ-ತ-ದಿಂದ
ಯೋಗ-ಸೂ-ತ್ರ-ಗಳ
ಯೋಗ-ಸೂ-ತ್ರ-ಗ-ಳಿಗೆ
ಯೋಗಾ-ನಂದ
ಯೋಗಾ-ಭ್ಯಾ-ಸ-ಗಳನ್ನು
ಯೋಗಾ-ಭ್ಯಾ-ಸ-ದಲ್ಲಿ
ಯೋಗಿ
ಯೋಗಿ-ಗಳು
ಯೋಗಿ-ಗಳೇ
ಯೋಗಿನ್
ಯೋಗಿಯ
ಯೋಗಿ-ಯನ್ನು
ಯೋಗಿ-ಯಾ-ಗಿ-ರು-ವು-ದಲ್ಲ
ಯೋಗಿ-ಯೊ-ಬ್ಬರು
ಯೋಗಿ-ಸಿ-ಕೊಂಡು
ಯೋಗಿಸು
ಯೋಗೀ-ಶ್ವ-ರ-ನಾದ
ಯೋಗೀ-ಶ್ವ-ರನು
ಯೋಗ್ಯ
ಯೋಗ್ಯತಾ
ಯೋಗ್ಯ-ತಾ-ಪ-ತ್ರ-ಗಳಲ್ಲಿ
ಯೋಗ್ಯ-ತಾ-ಪ-ತ್ರ-ಗಳು
ಯೋಗ್ಯತೆ
ಯೋಗ್ಯ-ತೆ-ಯ-ನ್ನಾ-ಗಲಿ
ಯೋಗ್ಯ-ತೆ-ಯನ್ನು
ಯೋಗ್ಯ-ತೆ-ಯನ್ನೂ
ಯೋಗ್ಯ-ತೆ-ಯಿಲ್ಲ
ಯೋಗ್ಯ-ತೆ-ಯಿ-ಲ್ಲದೆ
ಯೋಗ್ಯ-ತೆಯೂ
ಯೋಗ್ಯ-ತೆ-ಯೇ-ನೆಂ-ಬು-ದನ್ನು
ಯೋಗ್ಯ-ತೆ-ಯೇ-ನೆಂ-ಬುದು
ಯೋಗ್ಯ-ರಾ-ಗಿ-ರು-ತ್ತಿ-ರ-ಲಿಲ್ಲ
ಯೋಗ್ಯ-ರಾದ
ಯೋಗ್ಯರು
ಯೋಗ್ಯ-ವಲ್ಲ
ಯೋಗ್ಯ-ವಾ-ಗಿವೆ
ಯೋಗ್ಯ-ವಾದ
ಯೋಗ್ಯ-ವಾ-ದಂ-ತಹ
ಯೋಗ್ಯವೂ
ಯೋಚನೆ
ಯೋಚಿ-ಸಿ-ದರು
ಯೋಚಿ-ಸಿ-ದ-ರು-ಅ-ಮೆ-ರಿ-ಕದ
ಯೋಚಿ-ಸಿದ್ದ
ಯೋಚಿ-ಸಿ-ನೋಡಿ
ಯೋಚಿ-ಸು-ತ್ತಿ-ದ್ದಿರಿ
ಯೋಜ-ಕರು
ಯೋಜನೆ
ಯೋಜ-ನೆ-ಗಳ
ಯೋಜ-ನೆ-ಗಳನ್ನು
ಯೋಜ-ನೆ-ಗಳನ್ನೂ
ಯೋಜ-ನೆ-ಗಳನ್ನೆಲ್ಲ
ಯೋಜ-ನೆ-ಗಳಲ್ಲಿ
ಯೋಜ-ನೆ-ಗ-ಳ-ಲ್ಲೊಂ-ದೆಂ-ದರೆ
ಯೋಜ-ನೆ-ಗಳು
ಯೋಜ-ನೆ-ಗ-ಳೆಲ್ಲ
ಯೋಜ-ನೆಗೆ
ಯೋಜ-ನೆಯ
ಯೋಜ-ನೆ-ಯನ್ನು
ಯೋಜ-ನೆ-ಯಲ್ಲಿ
ಯೋಜ-ನೆ-ಯಾ-ಗಿತ್ತು
ಯೋಜ-ನೆ-ಯಾದ
ಯೋಜ-ನೆ-ಯಿ-ಲ್ಲದೆ
ಯೋಜ-ನೆಯು
ಯೋಜ-ನೆ-ಯೆಂ-ಬುದೇ
ಯೋಜ-ನೆ-ಯೊಂ-ದನ್ನು
ಯೋಜ-ನೆ-ಯೊಂದು
ಯೋಜಿ-ತ-ಗೊಂ-ಡ-ದ್ದಾ-ದರೂ
ಯೋಜಿ-ಸಿದ್ದ
ಯೋಜಿ-ಸಿ-ದ್ದರು
ಯೋಜಿ-ಸಿ-ರು-ವು-ದಾಗಿ
ಯೋಧ
ಯೋಧ-ಸಂ-ನ್ಯಾ-ಸಿಯೇ
ಯೋಧ-ನಂ-ತೆಯೇ
ಯೋಧ-ನಾಗಿ
ಯೋಧ-ಯೋಗಿ
ಯೋಪಾ-ದಿ-ಯಲ್ಲಿ
ಯೌವ-ನದ
ರ
ರಂಗ-ಗ-ಳಿಗೂ
ರಂಗ-ಗ-ಳಿಗೆ
ರಂಗ-ದಲ್ಲಿ
ರಂಗ-ದಲ್ಲೂ
ರಂಗ-ಮಂ-ಟ-ಪದ
ರಂಗ-ವನ್ನು
ರಂಗಾ-ಚಾರ್ಯ
ರಂಜನೆ
ರಂಜ-ನೆ-ಗಳು
ರಂಜಿ-ತ-ಗೊ-ಳಿ-ಸು-ವ-ವ-ರೆಗೂ
ರಂಜಿ-ತ-ರಾಗಿ
ರಂಜಿ-ಸಿ-ದರು
ರಂಜಿ-ಸುತ್ತ
ರಂತಹ
ರಂತೆ
ರಂದು
ರಂದು-ಬೆಂ-ಗ-ಳೂ-ರಿ-ನಲ್ಲೂ
ರಂದೂ
ರಕ್ತ
ರಕ್ತ-ಇವು
ರಕ್ತ-ಗ-ತ-ವಾ-ಗಿತ್ತು
ರಕ್ತ-ದಲ್ಲಿ
ರಕ್ತ-ದಿಂದ
ರಕ್ತ-ದೊಂ-ದಿಗೇ
ರಕ್ತ-ಬ-ಸಿವ
ರಕ್ತ-ಮಾಂಸ
ರಕ್ತ-ವನ್ನು
ರಕ್ತ-ವನ್ನೇ
ರಕ್ತವೂ
ರಕ್ತ-ಹೆ-ಪ್ಪು-ಗ-ಟ್ಟಿ-ದಂ-ತಹ
ರಕ್ಷಣೆ
ರಕ್ಷ-ಣೆ-ಗಿಂ-ತಲೂ
ರಕ್ಷ-ಣೆಯ
ರಕ್ಷ-ಣೆ-ಯಲ್ಲಿ
ರಕ್ಷ-ಣೆ-ಯಿಂದ
ರಕ್ಷಿ-ಸದೆ
ರಕ್ಷಿ-ಸಲು
ರಕ್ಷಿಸಿ
ರಕ್ಷಿ-ಸಿ-ಕೊ-ಳ್ಳು-ವುದು
ರಕ್ಷಿಸು
ರಕ್ಷಿ-ಸುವ
ರಘ-ನಾ-ಥ-ರಾ-ವ್ರ-ವರು
ರಚ-ನಾ-ತ್ಮಕ
ರಚ-ನಾ-ತ್ಮ-ಕ-ವಾ-ಗಿ-ರ-ಬೇ-ಕ-ಲ್ಲದೆ
ರಚನೆ
ರಚ-ನೆ-ಗಳನ್ನೂ
ರಚ-ನೆ-ಯಲ್ಲಿ
ರಚಿತ
ರಚಿ-ತ-ವಾದ
ರಚಿ-ಸ-ಲಾಗಿದೆ
ರಚಿ-ಸಲು
ರಚಿ-ಸ-ಲ್ಪಟ್ಟ
ರಚಿಸಿ
ರಚಿ-ಸಿದ
ರಚಿ-ಸಿ-ದರು
ರಚಿ-ಸಿ-ದ್ದರು
ರಜತ
ರಜದ
ರಜ-ಪು-ತಾನ
ರಜ-ಪು-ತಾ-ನಕ್ಕೆ
ರಜ-ಪು-ತಾ-ನದ
ರಜ-ಪು-ತಾ-ನ-ದಲ್ಲಿ
ರಜ-ಪು-ತಾ-ನ-ದಲ್ಲೇ
ರಜ-ಪೂ-ತರು
ರಜಾ
ರಜಾ-ದಿ-ನ-ಗಳನ್ನು
ರಜಾ-ದಿ-ನ-ಗಳಲ್ಲಿ
ರಜೆ-ಯನ್ನು
ರಟ್ಟೆ-ಯನ್ನು
ರಣ
ರಣ-ಕ-ಹ-ಳೆಯ
ರಣ-ಜಿತ್
ರತ-ನ್ಲಾಲ್
ರತ್ನ
ರತ್ನ-ಗಳ
ರತ್ನ-ಗಳನ್ನು
ರತ್ನ-ಗ-ಳಾದ
ರತ್ನ-ಗಳು
ರತ್ನ-ಗಳೇ
ರತ್ನ-ಪ್ರಾ-ಯ-ರಾಗಿ
ರತ್ನ-ಪ್ರಾ-ಯ-ವಾದ
ರತ್ನ-ವ-ನ್ನು-ನಿ-ಮ್ಮೊ-ಳಗೇ
ರತ್ನ-ವೆಂ-ಬಂತೆ
ರತ್ನ-ವ್ಯಾ-ಪಾರಿ
ರಥದ
ರಥ-ವನ್ನು
ರದ್ದಾ-ದ-ದ್ದ-ರಿಂದ
ರದ್ದು
ರದ್ದು-ಗೊ-ಳಿ-ಸಿ-ದರು
ರದ್ದು-ಪ-ಡಿ-ಸಿ-ಬಿ-ಟ್ಟಿ-ದ್ದಾ-ರೆಂದು
ರದ್ದು-ಪ-ಡಿ-ಸುವ
ರನ್ನು
ರನ್ನು-ಳಿದು
ರಭಸ
ರಭ-ಸ-ದಿಂದ
ರಭ-ಸ-ಪೂರ್ಣ
ರಭ-ಸ-ವನ್ನು
ರಭ-ಸವು
ರಮ-ಣೀಯ
ರಮಾ-ಬಾಯಿ
ರಮಾ-ಬಾ-ಯಿಯ
ರಮಾ-ಬಾ-ಯಿಯೂ
ರಮ್ಯ
ರಮ್ಯ-ವಾದ
ರಮ್ಯವೂ
ರಲು
ರಲೆ
ರಲೇ
ರಲ್ಲ
ರಲ್ಲ-ಎ-ಳೆಯ
ರಲ್ಲ-ದ-ವ-ರಾ-ದ್ದ-ರಿಂದ
ರಲ್ಲಿ
ರಲ್ಲಿ-ಆಗ
ರಲ್ಲಿ-ಎಂ-ದರೆ
ರಲ್ಲಿಯೂ
ರಲ್ಲೂ
ರಲ್ಲೆಲ್ಲ
ರಲ್ಲೇ
ರಲ್ಲೊ-ಬ್ಬನು
ರಳಿಂದ
ರವರ
ರವ-ರದು
ರವ-ರೆಗೂ
ರವ-ರೆಗೆ
ರವಿ-ಕಿ-ರ-ಣಾ-ವೃ-ತ-ವಾದ
ರವಿ-ವ-ರ್ಮನ
ರಷ್ಯಾ
ರಸ
ರಸ-ಗುಲ್ಲಾ
ರಸ-ದೌ-ತ-ಣ-ವ-ನ್ನೊ-ದ-ಗಿ-ಸಿತು
ರಸ-ಭ-ರಿತ
ರಸ-ಭ-ರಿ-ತ-ವ-ನ್ನಾ-ಗಿ-ಸ-ಬಲ್ಲ
ರಸ-ಭ-ರಿ-ತ-ವಾದ
ರಸ-ವ-ತ್ತಾಗಿ
ರಸ-ವ-ತ್ತಾದ
ರಸ-ವನ್ನು
ರಸ-ವನ್ನೂ
ರಸಾ-ನು-ಭವ
ರಸಾ-ಯ-ನ-ಶಾಸ್ತ್ರ
ರಸ್ತೆ
ರಸ್ತೆ-ಗಳ
ರಸ್ತೆ-ಗ-ಳಂತೆ
ರಸ್ತೆ-ಗಳಲ್ಲಿ
ರಸ್ತೆ-ಗ-ಳು-ಎಲ್ಲ
ರಸ್ತೆಗೆ
ರಸ್ತೆಯ
ರಸ್ತೆ-ಯಲ್ಲಿ
ರಸ್ತೆ-ಯ-ಲ್ಲೆಲ್ಲ
ರಸ್ತೆ-ಯಲ್ಲೇ
ರಸ್ತೆಯೂ
ರಹತ
ರಹ-ಮಾನ್
ರಹಸ್ಯ
ರಹ-ಸ್ಯ-ಗಳನ್ನು
ರಹ-ಸ್ಯ-ಗ-ಳೆಲ್ಲ
ರಹ-ಸ್ಯ-ವನ್ನು
ರಹ-ಸ್ಯ-ವೆಂದರೆ
ರಹ-ಸ್ಯವೇ
ರಾಕ್ಫೆ-ಲ್ಲರ್
ರಾಕ್ಫೆ-ಲ್ಲರ್ನ
ರಾಕ್ಫೆ-ಲ್ಲ-ರ್ನಿಗೆ
ರಾಕ್ಫೆ-ಲ್ಲರ್ನ್ನು
ರಾಕ್ಷಸ
ರಾಕ್ಷ-ಸ-ತ-ನದ
ರಾಕ್ಷ-ಸ-ತ-ನ-ವಿ-ಲ್ಲ-ದಿ-ದ್ದಲ್ಲಿ
ರಾಕ್ಷ-ಸ-ರಂತೆ
ರಾಕ್ಷ-ಸಾ-ಕಾರ
ರಾಖತ
ರಾಖಾ-ಲ-ನನ್ನು
ರಾಗ
ರಾಗಲು
ರಾಗಿ
ರಾಗಿದ್ದ
ರಾಗಿ-ದ್ದರು
ರಾಗಿ-ದ್ದರೂ
ರಾಗಿ-ದ್ದ-ವರು
ರಾಗಿ-ದ್ದ-ವ-ರೊ-ಬ್ಬರು
ರಾಗಿ-ದ್ದಾರೆ
ರಾಗಿ-ದ್ದು-ಬಿ-ಟ್ಟರು
ರಾಗಿ-ಬಿ-ಡು-ತ್ತಿ-ದ್ದರು
ರಾಗಿ-ರ-ಲಿಲ್ಲ
ರಾಗಿ-ರು-ವುದನ್ನು
ರಾಗಿ-ಸಲು
ರಾಗು-ತ್ತಿ-ದ್ದರು
ರಾಗೋಣ
ರಾಚು-ವಂತೆ
ರಾಚೋಲ್
ರಾಜ
ರಾಜಂ
ರಾಜ-ಕಾರಣದ
ರಾಜ-ಕಾರಣ-ದಲ್ಲಿ
ರಾಜ-ಕಾರಣ-ವೆಂ-ಬುದು
ರಾಜ-ಕಾರಣವೇ
ರಾಜ-ಕಾರಣಿ-ಗಳು
ರಾಜ-ಕಾರಣಿ-ಯೊ-ಬ್ಬರ
ರಾಜ-ಕೀಯ
ರಾಜ-ಕೀ-ಯದ
ರಾಜ-ಕೀ-ಯ-ದಲ್ಲೂ
ರಾಜ-ಕೀ-ಯ-ದೊಂ-ದಿಗೆ
ರಾಜ-ಕೀ-ಯ-ವನ್ನು
ರಾಜ-ಕು-ಮಾರ
ರಾಜ-ಕು-ಮಾ-ರ-ನನ್ನು
ರಾಜ-ಕು-ಮಾ-ರ-ನಿಗೆ
ರಾಜ-ಕು-ಮಾ-ರರ
ರಾಜ-ಕು-ಮಾ-ರ-ರಿಗೆ
ರಾಜ-ಕು-ಮಾ-ರ-ರಿ-ಬ್ಬರೂ
ರಾಜ-ಕು-ಮಾ-ರರು
ರಾಜ-ಕು-ಮಾ-ರರೂ
ರಾಜ-ಗಾಂ-ಭೀ-ರ್ಯ-ದಿಂದ
ರಾಜ-ಠಾ-ಕೂರ್
ರಾಜ-ಠೀವಿ
ರಾಜ-ಧಾನಿ
ರಾಜ-ಧಾ-ನಿ-ಯಾದ
ರಾಜನ
ರಾಜ-ನನ್ನು
ರಾಜ-ನನ್ನೋ
ರಾಜ-ನಲ್ಲ
ರಾಜ-ನಾ-ಗಿ-ಬಿ-ಟ್ಟ-ನೆಂ-ದರೆ
ರಾಜ-ನಾ-ಗು-ವುದು
ರಾಜ-ನಾದ
ರಾಜ-ನಿಂದ
ರಾಜ-ನಿಗೆ
ರಾಜನು
ರಾಜನೂ
ರಾಜ-ನೆಂದು
ರಾಜನೇ
ರಾಜ-ನೊಂ-ದಿಗೆ
ರಾಜ-ನೊಬ್ಬ
ರಾಜ-ಮ-ನೆ-ತ-ನ-ಗ-ಳ-ವರ
ರಾಜ-ಮ-ನೆ-ತ-ನದ
ರಾಜ-ಮಹಾ
ರಾಜ-ಮ-ಹಾ-ರಾ-ಜರ
ರಾಜ-ಮ-ಹಾ-ರಾ-ಜ-ರಿಗೆ
ರಾಜ-ಮಾ-ರ್ಗ-ದಲ್ಲಿ
ರಾಜ-ಯೋಗ
ರಾಜ-ಯೋ-ಗ-ಜ್ಞಾ-ನ-ಯೋಗ
ರಾಜ-ಯೋ-ಗ-ಜ್ಞಾ-ನ-ಯೋ-ಗ-ಗಳನ್ನು
ರಾಜ-ಯೋ-ಗ-ಗಳ
ರಾಜ-ಯೋ-ಗ-ಗ-ಳಿ-ಗಿ-ರುವ
ರಾಜ-ಯೋ-ಗದ
ರಾಜ-ಯೋ-ಗ-ವನ್ನು
ರಾಜ-ಯೋ-ಗವು
ರಾಜ-ಯೋಗ್ಯ
ರಾಜರ
ರಾಜ-ರ-ನ್ನೆಲ್ಲ
ರಾಜ-ರನ್ನೇ
ರಾಜ-ರ-ಲ್ಲೆಲ್ಲ
ರಾಜ-ರಿಗೆ
ರಾಜರು
ರಾಜ-ರು-ದಿ-ವಾ-ನ-ರೊಂ-ದಿಗೆ
ರಾಜ-ರು-ಪಂ-ಡಿ-ರೆಲ್ಲ
ರಾಜರೂ
ರಾಜ-ರೊಂ-ದಿಗೆ
ರಾಜ-ವಂ-ಶ-ಗಳ
ರಾಜ-ವಾ-ಡೆ-ಯ-ವ-ರಿಂದ
ರಾಜ-ಸ್ತಾ-ನಿ-ಗ-ಳೆಲ್ಲ
ರಾಜ-ಸ್ಥಾ-ನದ
ರಾಜ-ಸ್ಥಾ-ನ-ದ-ಲ್ಲಿ-ರು-ವ-ರೆಂ-ಬುದು
ರಾಜಾ
ರಾಜಾ-ಸ್ಥಾ-ನದ
ರಾಜಿ
ರಾಜಿ-ಮಾ-ಡಿ-ಕೊ-ಳ್ಳ-ದಿ-ರುವ
ರಾಜಿ-ಮಾ-ಡಿ-ಕೊಳ್ಳು
ರಾಜಿ-ಯನ್ನೂ
ರಾಜೀ-ನಾಮೆ
ರಾಜೋ-ಪ-ಚಾ-ರ-ವನ್ನು
ರಾಜ್ಯ
ರಾಜ್ಯಕ್ಕೆ
ರಾಜ್ಯ-ಗಳ
ರಾಜ್ಯ-ಗಳಲ್ಲಿ
ರಾಜ್ಯದ
ರಾಜ್ಯ-ದಲ್ಲಿ
ರಾಜ್ಯ-ದಿಂದ
ರಾಜ್ಯ-ದೆ-ಡೆಗೆ
ರಾಜ್ಯ-ಪಾ-ಲ-ರಾ-ಗಿದ್ದ
ರಾಜ್ಯ-ವನ್ನು
ರಾಜ್ಯವು
ರಾಜ್ಯ-ವೆಂಬ
ರಾಜ್ಯಾಂ-ಗದ
ರಾಜ್ಯಾ-ಡ-ಳಿತ
ರಾಜ್ಯಾ-ಡ-ಳಿ-ತ-ವನ್ನು
ರಾಣಿ-ಯಂತೆ
ರಾತ್ರಿ
ರಾತ್ರಿ-ಗ-ಳ-ವ-ರೆಗೆ
ರಾತ್ರಿಯ
ರಾತ್ರಿ-ಯನ್ನು
ರಾತ್ರಿ-ಯಲ್ಲಿ
ರಾತ್ರಿ-ಯಲ್ಲೇ
ರಾತ್ರಿ-ಯ-ವ-ರೆಗೂ
ರಾತ್ರಿ-ಯ-ವ-ರೆಗೆ
ರಾತ್ರಿ-ಯಾ-ಗಿತ್ತು
ರಾತ್ರಿ-ಯಿಡೀ
ರಾತ್ರಿಯೂ
ರಾತ್ರಿ-ಯೆಲ್ಲ
ರಾದ
ರಾದರು
ರಾದರೂ
ರಾದರೆ
ರಾದ-ರೆ-ನ್ನು-ವುದು
ರಾದರೋ
ರಾಧಿಯೂ
ರಾಫೇಲ್
ರಾಬ-ರ್ಟ್
ರಾಮ
ರಾಮ-ಕೃಷ್ಣ
ರಾಮ-ಕೃ-ಷ್ಣರ
ರಾಮ-ಕೃ-ಷ್ಣರು
ರಾಮ-ಕೃಷ್ಣಾ
ರಾಮ-ಕೃ-ಷ್ಣಾ-ನಂ-ದ-ರನ್ನು
ರಾಮ-ಕೃ-ಷ್ಣಾ-ನಂ-ದ-ರಿಗೆ
ರಾಮ-ಕೃ-ಷ್ಣಾ-ನಂ-ದ-ರಿ-ಗೊಂದು
ರಾಮ-ಕೃ-ಷ್ಣಾ-ನಂ-ದರೂ
ರಾಮ-ಚಂದ್ರ
ರಾಮ-ಚಂ-ದ್ರಜಿ
ರಾಮ-ದಾಸ್
ರಾಮ-ನಾ-ಡಿಗೆ
ರಾಮ-ನಾ-ಡಿನ
ರಾಮ-ನಾ-ಡು-ಗ-ಳಂ-ತಹ
ರಾಮನೇ
ರಾಮ-ಪ್ರ-ಸಾ-ದನೇ
ರಾಮ-ಮಂತ್ರ
ರಾಮ-ಮೋ-ಹನ
ರಾಮ-ಮೋ-ಹ-ನ-ರಾ-ಯ-ರ-ವ-ರೆಗೆ
ರಾಮಯ್ಯ
ರಾಮ-ಸ್ವಾಮಿ
ರಾಮ-ಸ್ವಾ-ಮಿಗೆ
ರಾಮಾ-ನು-ಜಾ-ಚಾರ್ಯ
ರಾಮಾ-ನು-ಜಾ-ಚಾ-ರ್ಯರ
ರಾಮಾ-ಯ-ಣ-ವನ್ನು
ರಾಮಾ-ಯ-ಯ-ಣ-ಮ-ಹಾ-ಭಾ-ರ-ತ-ಗಳನ್ನೂ
ರಾಮೇ-ಶ್ವರ
ರಾಮೇ-ಶ್ವ-ರಕ್ಕೆ
ರಾಮೇ-ಶ್ವ-ರದ
ರಾಮೇ-ಶ್ವ-ರ-ದಿಂದ
ರಾಮೇ-ಶ್ವ-ರ-ವನ್ನು
ರಾಮ್
ರಾಮ್ಕೆಸ್ಟೊ
ರಾಯ-ಭಾರಿ
ರಾಯ-ಭಾ-ರಿ-ಗಳ
ರಾಯ-ಭಾ-ರಿಯೂ
ರಾಯರ
ರಾಯಲ್
ರಾಯ್
ರಾರಾ-ಜಿ-ಸ-ಬ-ಹು-ದಾ-ಗಿದ್ದ
ರಾಲ್ಫ್
ರಾಳದ
ರಾಳ-ದಿಂದ
ರಾವ-ಣ-ನನ್ನು
ರಾವ್
ರಾಶಿಯ
ರಾಶಿಯೇ
ರಾಶಿ-ರಾಶಿ
ರಾಷ್ಟ್ರ
ರಾಷ್ಟ್ರ-ಎ-ಲ್ಲವೂ
ರಾಷ್ಟ್ರ-ಕ-ನಲ್ಲೂ
ರಾಷ್ಟ್ರ-ಕ್ಕಂ-ಟಿದ
ರಾಷ್ಟ್ರ-ಕ್ಕಾಗಿ
ರಾಷ್ಟ್ರ-ಕ್ಕಿತ್ತ
ರಾಷ್ಟ್ರಕ್ಕೂ
ರಾಷ್ಟ್ರಕ್ಕೆ
ರಾಷ್ಟ್ರ-ಗಳ
ರಾಷ್ಟ್ರ-ಗಳಲ್ಲಿ
ರಾಷ್ಟ್ರ-ಗ-ಳ-ಲ್ಲಿನ
ರಾಷ್ಟ್ರ-ಗ-ಳಲ್ಲೂ
ರಾಷ್ಟ್ರ-ಗ-ಳಾ-ದ್ದ-ರಿಂದ
ರಾಷ್ಟ್ರ-ಗ-ಳಿ-ಗಿಂತ
ರಾಷ್ಟ್ರ-ಗ-ಳಿಗೂ
ರಾಷ್ಟ್ರ-ಗ-ಳಿಗೆ
ರಾಷ್ಟ್ರ-ಗಳು
ರಾಷ್ಟ್ರ-ಗ-ಳೆ-ಲ್ಲವೂ
ರಾಷ್ಟ್ರದ
ರಾಷ್ಟ್ರ-ದಲ್ಲಿ
ರಾಷ್ಟ್ರ-ದ-ಲ್ಲೇ-ಉ-ಳಿ-ದು-ಕೊಳ್ಳ
ರಾಷ್ಟ್ರ-ದಾ-ದ್ಯಂತ
ರಾಷ್ಟ್ರ-ನಿ-ರ್ಮಾ-ಣ-ಕಾ-ರಿ-ಯಾದ
ರಾಷ್ಟ್ರ-ನಿ-ರ್ಮಾ-ಣಕ್ಕೆ
ರಾಷ್ಟ್ರ-ನಿ-ರ್ಮಾ-ಣದ
ರಾಷ್ಟ್ರ-ನಿ-ರ್ಮಾ-ಪಕ
ರಾಷ್ಟ್ರ-ನಿ-ರ್ಮಾ-ಪ-ಕ-ನಾಗಿ
ರಾಷ್ಟ್ರ-ಪ್ರಜ್ಞೆ
ರಾಷ್ಟ್ರ-ಪ್ರ-ಜ್ಞೆಯೂ
ರಾಷ್ಟ್ರ-ಪ್ರೇಮ
ರಾಷ್ಟ್ರ-ಪ್ರೇ-ಮವು
ರಾಷ್ಟ್ರ-ಭಾವ
ರಾಷ್ಟ್ರ-ಮ-ಟ್ಟದ
ರಾಷ್ಟ್ರ-ವನ್ನು
ರಾಷ್ಟ್ರ-ವನ್ನೂ
ರಾಷ್ಟ್ರ-ವನ್ನೇ
ರಾಷ್ಟ್ರ-ವ-ಲ್ಲವೆ
ರಾಷ್ಟ್ರ-ವಾಗಿ
ರಾಷ್ಟ್ರ-ವಾ-ಗಿದ್ದ
ರಾಷ್ಟ್ರ-ವಾದ
ರಾಷ್ಟ್ರ-ವಿ-ಜೇತ
ರಾಷ್ಟ್ರವು
ರಾಷ್ಟ್ರ-ವೆಂದರೆ
ರಾಷ್ಟ್ರ-ವೆಂದು
ರಾಷ್ಟ್ರವೇ
ರಾಷ್ಟ್ರವೋ
ರಾಷ್ಟ್ರಾ-ಭಿ-ಮಾ-ನ-ಸ್ವ-ಜ-ನಾ-ಭಿ-ಮಾ-ನ-ಗಳು
ರಾಷ್ಟ್ರಾ-ಭಿ-ಮಾ-ನ-ವನ್ನೂ
ರಾಷ್ಟ್ರೀಯ
ರಾಷ್ಟ್ರೀ-ಯ-ತೆಯ
ರಾಷ್ಟ್ರೀ-ಯರು
ರಾಷ್ಟ್ರೋ-ನ್ನ-ತಿಯ
ರಾಸಿನ
ರಿಂದ
ರಿಂದಲೂ
ರಿಂದಲೇ
ರಿಗಾಗಿ
ರಿಗಿ
ರಿಗೂ
ರಿಗೆ
ರಿಡ್ಜ್ಲಿ
ರಿಡ್ಜ್ಲಿ-ಯಲ್ಲಿ
ರಿಯಸ್
ರಿಲಿ-ಜಸ್
ರಿಸಿ-ಕೊಂಡು
ರೀಡಿಂ-ಗಿ-ನಲ್ಲಿ
ರೀಡಿಂ-ಗಿ-ನಿಂದ
ರೀಡಿಂಗ್
ರೀಡಿಂಗ್ಗೆ
ರೀತಿ
ರೀತಿ-ರಿ-ವಾಜು
ರೀತಿ-ನೀ-ತಿ-ಗಳ
ರೀತಿ-ನೀ-ತಿ-ಗಳನ್ನು
ರೀತಿ-ನೀ-ತಿ-ಗಳನ್ನೂ
ರೀತಿ-ನೀ-ತಿ-ಗಳನ್ನೆಲ್ಲ
ರೀತಿ-ನೀ-ತಿ-ಗಳಿಂದ
ರೀತಿ-ನೀ-ತಿ-ಗ-ಳಿಗೆ
ರೀತಿ-ನೀ-ತಿ-ಗ-ಳಿ-ರು-ತ್ತವೆ
ರೀತಿ-ನೀ-ತಿ-ಗಳು
ರೀತಿಯ
ರೀತಿ-ಯಂ-ತೆ-ಯೇ-ಅ-ವರ
ರೀತಿ-ಯನ್ನು
ರೀತಿ-ಯ-ಲ್ಲಾ-ದರೂ
ರೀತಿ-ಯಲ್ಲಿ
ರೀತಿ-ಯಲ್ಲೂ
ರೀತಿ-ಯಲ್ಲೇ
ರೀತಿ-ಯಾಗಿ
ರೀತಿ-ಯಿಂದ
ರೀತಿಯು
ರುಚಿ
ರುಚಿ-ಕ-ಟ್ಟಾದ
ರುಚಿ-ಕರ
ರುಚಿ-ಕ-ರ-ವಾ-ಗೇನೋ
ರುಚಿ-ಕ-ರ-ವಾದ
ರುಚಿಗೆ
ರುಚಿ-ಯ-ನ್ನೊಮ್ಮೆ
ರುಚಿ-ಯಾ-ಗಿ-ದ್ದರು
ರುಚಿ-ಸದು
ರುಚಿ-ಸದೆ
ರುಚಿ-ಸು-ವಂ-ತಿ-ರ-ಲಿಲ್ಲ
ರುಚೀ-ನಾಂ
ರುಜು-ವಾತು
ರುಜು-ವಾ-ತು-ಪ-ತ್ರ-ವಿ-ಲ್ಲದೆ
ರುತ್ತ
ರುಮಾ-ಲನ್ನು
ರುಮಾಲು
ರುವ
ರುವು-ದ-ರಿಂದ
ರೂ
ರೂಢ-ಮೂ-ಲ-ವಾದ
ರೂಢಿ-ಯಾ-ಗಿವೆ
ರೂತ್
ರೂಪ
ರೂಪ-ಕೊಟ್ಟು
ರೂಪಕ್ಕೆ
ರೂಪ-ಗ-ಳಷ್ಟೆ
ರೂಪ-ಗಳು
ರೂಪ-ದಲ್ಲಿ
ರೂಪ-ದ-ಲ್ಲಿ-ಮ-ಹಾ-ರಾಜ
ರೂಪ-ದಿಂದ
ರೂಪ-ರೇ-ಷೆ-ಗಳನ್ನು
ರೂಪ-ವನ್ನು
ರೂಪ-ವಾದ
ರೂಪವು
ರೂಪ-ವೊಂ-ದನ್ನು
ರೂಪಾಯಿ
ರೂಪಾ-ಯಿ-ಗಳ
ರೂಪಾ-ಯಿ-ಗಳನ್ನು
ರೂಪಾ-ಯಿ-ಗಳು
ರೂಪಾ-ಯಿ-ಯ-ಷ್ಟಾ-ಗಿ-ತ್ತೆಂದು
ರೂಪಿ-ಸ-ಲ್ಪಟ್ಟ
ರೂಪಿಸಿ
ರೂಪಿ-ಸಿ-ಕೊ-ಳ್ಳಲು
ರೂಪಿ-ಸಿ-ಕೊ-ಳ್ಳು-ತ್ತಾರೆ
ರೂಪಿ-ಸಿ-ದುದು
ರೂಪಿ-ಸಿ-ದ್ದರು
ರೂಪಿ-ಸು-ತ್ತಿ-ದ್ದರು
ರೂಪಿ-ಸು-ವುದು
ರೂಪು
ರೂಪು-ಗೊಂಡ
ರೂಪು-ಗೊ-ಳಿ-ಸಿ-ದ್ದ-ರಲ್ಲಿ
ರೂಪು-ಗೊ-ಳ್ಳ-ಬೇ-ಕಾ-ದರೆ
ರೂಪು-ರೇಷೆ
ರೂಪು-ರೇ-ಷೆ-ಗಳನ್ನು
ರೂಪು-ರೇ-ಷೆ-ಗ-ಳೆಂ-ಥವು
ರೂಮನ್
ರೂವಾ-ರಿ-ಯೊ-ಬ್ಬನ
ರೆ
ರೆಂಡ್
ರೆಂದರೆ
ರೆಂದಾ-ದರೂ
ರೆಂದು
ರೆಂದೇ
ರೆಂಬ
ರೆಂಬು-ದಕ್ಕೆ
ರೆಕ್ಕೆ-ಗಳ
ರೆಗೂ
ರೆನ್ನಿ-ಸಿ-ಕೊಂ-ಡ-ವರ
ರೆನ್ನು-ತ್ತಾರೆ
ರೆಪ್ಪೆ-ಗಳು
ರೆಯೋ
ರೆಲ್ಲ
ರೆಲ್ಲರ
ರೆಲ್ಲರೂ
ರೆವ
ರೆವ-ರೆಂಡ್
ರೆಷ್ಟು
ರೇಖಾ-ಚಿತ್ರ
ರೇಖೆ
ರೇಖೆಯೂ
ರೇಡಿಯೋ
ರೇವಿಗೂ
ರೇವು-ಗ-ಳ-ಲ್ಲೊಂ-ದಾದ
ರೇಷ್ಮೆ
ರೇಷ್ಮೆಯ
ರೈಟರ
ರೈಟ-ರನ್ನು
ರೈಟ-ರ-ಲ್ಲದೆ
ರೈಟ-ರಿಗೆ
ರೈಟರು
ರೈಟ-ರೊಂ-ದಿಗೆ
ರೈಟ್
ರೈಟ್ರ-ವರ
ರೈಟ್ರ-ವರು
ರೈತನ
ರೈತ-ನಂತೆ
ರೈತ-ನನ್ನು
ರೈತ-ನೊಬ್ಬ
ರೈತ-ರಂತೆ
ರೈತ-ರಾ-ಗ-ಬೇಕೆ
ರೈತ-ರಿ-ದ್ದಾರೆ
ರೈತರು
ರೈತಾ-ಪಿ-ಗ-ಳೊಂ-ದಿಗೆ
ರೈನ್
ರೈಲಿ-ನಲ್ಲಿ
ರೈಲು
ರೈಲು-ಟ್ರಾ-ಮು-ಗಳಿಂದ
ರೈಲು-ದಾರಿ-ಗಳ
ರೈಲು-ದಾರಿ-ಯಿದೆ
ರೈಲು-ನಿ-ಲ್ದಾಣ
ರೈಲು-ನಿ-ಲ್ದಾ-ಣದ
ರೈಲು-ನಿ-ಲ್ದಾ-ಣ-ದಲ್ಲಿ
ರೈಲು-ನಿ-ಲ್ದಾ-ಣ-ದ-ಲ್ಲಿ-ದ್ದಾ-ಗಲೇ
ರೈಲು-ಪ್ರ-ಯಾ-ಣ-ಗಳೂ
ರೈಲ್ವೆ
ರೈಲ್ವೆ-ಯ-ವನೇ
ರೈಲ್ವೇ
ರೊಂದಿ-ಗಿ-ರಲಿ
ರೊಂದಿಗೂ
ರೊಂದಿಗೆ
ರೊಂದಿಗೇ
ರೊಂದು
ರೊಚ್ಚಿ-ಗೆ-ಬ್ಬಿ-ಸಿತ್ತು
ರೊಟ್ಟಿಗೂ
ರೊಟ್ಟಿ-ಯನ್ನು
ರೊಬ್ಬ-ರನ್ನು
ರೊಬ್ಬರು
ರೋಗ
ರೋಗಕ್ಕೆ
ರೋಗದ
ರೋಗ-ದಂತೆ
ರೋಗ-ರು-ಜಿ-ನ-ಗಳು
ರೋಗ-ವನ್ನೇ
ರೋಗವು
ರೋಗಿ-ಗಳನ್ನು
ರೋಗ್ಯಕ್ಕೂ
ರೋಮ-ಕೂ-ಪ-ಗ-ಳ-ಲ್ಲೆಲ್ಲ
ರೋಮನ್
ರೋಮ-ನ್ನರ
ರೋಮ-ನ್ನರೂ
ರೋಮಾಂ-ಚಕ
ರೋಮಾಂ-ಚ-ಕರ
ರೋಮಾಂ-ಚ-ಕಾರಿ
ರೋಮಾಂ-ಚ-ಕಾ-ರಿ-ಯಾಗಿ
ರೋಮಾಂ-ಚ-ಕಾ-ರಿ-ಯಾ-ಗಿ-ರು-ತ್ತಿತ್ತು
ರೋಮಾಂ-ಚ-ಕಾ-ರಿ-ಯಾದ
ರೋಮಾಂ-ಚ-ಕಾರೀ
ರೋಮಾಂ-ಚ-ನ-ಗೊಂಡ
ರೋಮಾಂ-ಚ-ವಿ-ರು-ತ್ತಿತ್ತು
ರೋಮಾಂ-ಚಿ-ತ-ರಾ-ದರು
ರೋಲರ್
ರೋಷ
ರ್ಯಾರೂ
ರ್ಹೋನ್
ಲಂಕಾ-ದ್ವೀ-ಪಕ್ಕೆ
ಲಂಗರು
ಲಂಘ-ಯತೇ
ಲಂಚ
ಲಂಡನ್
ಲಂಡನ್ನಿ
ಲಂಡ-ನ್ನಿಗೆ
ಲಂಡ-ನ್ನಿನ
ಲಂಡ-ನ್ನಿ-ನಂಥ
ಲಂಡ-ನ್ನಿ-ನಲ್ಲಿ
ಲಂಡ-ನ್ನಿ-ನ-ಲ್ಲಿದ್ದ
ಲಂಡ-ನ್ನಿ-ನ-ಲ್ಲಿ-ದ್ದಾಗ
ಲಂಡ-ನ್ನಿ-ನ-ಲ್ಲಿ-ದ್ದುದು
ಲಂಡ-ನ್ನಿ-ನ-ಲ್ಲಿನ
ಲಂಡ-ನ್ನಿ-ನಲ್ಲೇ
ಲಂಡ-ನ್ನಿ-ನಿಂದ
ಲಂಡ-ನ್ನಿ-ನಿಂ-ದಾಚೆ
ಲಕೋ-ಟೆ-ಯನ್ನು
ಲಕ್ಷ
ಲಕ್ಷಕ್ಕೂ
ಲಕ್ಷ-ಗ-ಟ್ಟಲೆ
ಲಕ್ಷಣ
ಲಕ್ಷ-ಣ-ಗಳನ್ನು
ಲಕ್ಷ-ಣ-ಗಳನ್ನೂ
ಲಕ್ಷ-ಣ-ಗಳನ್ನೆಲ್ಲ
ಲಕ್ಷ-ಣ-ಗಳು
ಲಕ್ಷ-ಣ-ಗ-ಳೆಂ-ಬು-ದನ್ನು
ಲಕ್ಷ-ಣವು
ಲಕ್ಷ-ಣವೂ
ಲಕ್ಷ-ಣ-ವೆಂದು
ಲಕ್ಷ-ಣವೇ
ಲಕ್ಷಾಂ-ತರ
ಲಕ್ಷಾ-ವಧಿ
ಲಕ್ಷಿ-ಸ-ಬೇಡಿ
ಲಕ್ಷಿ-ಸ-ಲಿಲ್ಲ
ಲಕ್ಷಿ-ಸು-ತ್ತಾರೆ
ಲಕ್ಷಿ-ಸು-ತ್ತೇ-ನೆಂ-ದಾ-ಗಲಿ
ಲಕ್ಷಿ-ಸು-ವು-ದಿಲ್ಲ
ಲಕ್ಷೋ-ಪ-ಲಕ್ಷ
ಲಕ್ಷ್ಮಿಯ
ಲಕ್ಷ್ಮೀ-ಬಾಯಿ
ಲಕ್ಷ್ಮೀ-ಬಾ-ಯಿಗೆ
ಲಕ್ಷ್ಯ
ಲಕ್ಷ್ಯವೇ
ಲಗ್ಗೆ
ಲಘು
ಲಭಿ-ಸಿತ್ತು
ಲಭಿ-ಸು-ತ್ತವೆ
ಲಭ್ಯ-ವಾ-ಗಿದೆ
ಲಭ್ಯ-ವಾ-ಗಿ-ರ-ಲಿಲ್ಲ
ಲಭ್ಯ-ವಾ-ದು-ವೆಂದು
ಲಭ್ಯ-ವಿದ್ದ
ಲಭ್ಯ-ವಿ-ರ-ಲಿಲ್ಲ
ಲಭ್ಯ-ವಿ-ರುವ
ಲಭ್ಯ-ವಿಲ್ಲ
ಲಭ್ಯವೋ
ಲಯೊಲಾ
ಲಲನೆ
ಲವ-ಲ-ವಿಕೆ
ಲವ-ಲೇ-ಶ-ವಾ-ದರೂ
ಲವ-ಲೇ-ಶವೂ
ಲಸ್ಕರ್
ಲಹರಿ
ಲಾಂಗ್
ಲಾಗಿ
ಲಾಗಿತ್ತು
ಲಾಗಿದೆ
ಲಾಗು-ತ್ತಿತ್ತು
ಲಾಗು-ತ್ತಿದ್ದ
ಲಾಟೀನು
ಲಾಠಿ
ಲಾಡು-ಗಳು
ಲಾಡ್
ಲಾದ
ಲಾದರೂ
ಲಾದುವು
ಲಾಭ
ಲಾಭ-ಕ್ಕಾಗಿ
ಲಾಭ-ಗಳನ್ನೂ
ಲಾಭ-ಗಳಲ್ಲಿ
ಲಾಭ-ಪ್ರ-ದ-ವಾ-ದೀ-ತೆಂದು
ಲಾಭ-ವಾ-ಗುತ್ತ
ಲಾಭ-ವಾ-ಯಿತು
ಲಾಭವೇ
ಲಾಯ-ವನ್ನು
ಲಾರಂ-ಭಿ-ಸಿದ
ಲಾರಂ-ಭಿ-ಸಿ-ದರು
ಲಾರ-ದಷ್ಟು
ಲಾರದೆ
ಲಾರರು
ಲಾರಾ
ಲಾರಾ-ಗ್ಲೆನ್
ಲಾರೆ-ನ್ಸ್
ಲಾರೆ-ಹೆಂ-ಗ-ಸಿ-ನಂತೆ
ಲಾಲಿ
ಲಾಲ್
ಲಾಲ್ಶಂ-ಕರ್
ಲಾಸ್
ಲಾಸ್ಯ-ವಾ-ಡುವ
ಲಿಂಗ-ಭೇ-ದ-ವಿ-ಲ್ಲದೆ
ಲಿಂಗ-ವನ್ನು
ಲಿಂಬ್ಡಿಗೆ
ಲಿಂಬ್ಡಿಯ
ಲಿಂಬ್ಡಿ-ಯನ್ನು
ಲಿಂಬ್ಡಿ-ಯಲ್ಲಿ
ಲಿಂಬ್ಡಿ-ಯಿಂದ
ಲಿಖಿತ
ಲಿಟ್ಲ್
ಲಿದ್ದ
ಲಿದ್ದರು
ಲಿದ್ದ-ವರು
ಲಿನ್
ಲಿನ್ನಿಂದ
ಲಿಪಿ
ಲಿಪಿ-ಕಾರ
ಲಿಪಿ-ಯಲ್ಲಿ
ಲಿಯಾ-ದಲ್ಲಿ
ಲಿಯಾ-ನರ
ಲಿಯಾ-ನರು
ಲಿಯಾನ್
ಲಿಯಾನ್ರ
ಲಿಯಾ-ನ್ರ-ನ್ನು-ಳಿದು
ಲಿಯಾ-ನ್ರ-ವರು
ಲಿರುವ
ಲಿಲ್ಲ
ಲಿವ-ರ್ಪೂ-ಲಿಗೆ
ಲಿವ-ರ್ಪೂಲ್
ಲೀಗ್
ಲೀನ-ರಾ-ಗಿ-ದ್ದರು
ಲೀನ-ವಾ-ಗಿ-ರು-ವುದನ್ನು
ಲೀನ-ವಾ-ಗಿ-ಸಿ-ಕೊಂ-ಡ-ವರು
ಲೀಮನ್
ಲೀಲಾ-ಜಾಲ
ಲೀಲಾ-ಜಾ-ಲ-ವಾಗಿ
ಲೀಲಾ-ಮ-ಯ-ನಾದ
ಲೀಲೆ
ಲೀಲೆ-ಗಳ
ಲೀಲೆಯ
ಲೂಟಿ
ಲೂಯಿಸ್
ಲೂಯಿ-ಸ್ಳಿಗೂ
ಲೆಕ್ಕ
ಲೆಕ್ಕ-ಪ-ತ್ರ-ಗಳ
ಲೆಕ್ಕ-ವಿಲ್ಲ
ಲೆಕ್ಕ-ವಿ-ಲ್ಲ-ದಷ್ಟು
ಲೆಕ್ಕವೇ
ಲೆಕ್ಕಾ
ಲೆಕ್ಕಾ-ಚಾರ
ಲೆಕ್ಕಾ-ಚಾ-ರದ
ಲೆಕ್ಕಿ-ಸ-ದಿ-ದ್ದ-ವರು
ಲೆಕ್ಕಿ-ಸದೆ
ಲೆಕ್ಕಿ-ಸ-ಬೇಡಿ
ಲೆಕ್ಕಿ-ಸ-ಲಿಲ್ಲ
ಲೆಕ್ಕಿ-ಸು-ವು-ದಿಲ್ಲ
ಲೆಕ್ಕಿ-ಸು-ವುದೇ
ಲೆಕ್ಚರ್
ಲೆಗ-ಟ್ಟ-ರನ್ನು
ಲೆಗೆಟ್
ಲೆಗೆ-ಟ್ಟರ
ಲೆಗೆ-ಟ್ಟ-ರನ್ನು
ಲೆಗೆ-ಟ್ಟ-ರಿಗೆ
ಲೆಗೆ-ಟ್ಟರು
ಲೆಗೆ-ಟ್ಟ-ರೊಂ-ದಿಗೆ
ಲೆಗೆ-ಟ್ರನ್ನು
ಲೆತ್ನಿ-ಸ-ಬ-ಹುದು
ಲೆಲ್ಲ
ಲೇ
ಲೇಕ್
ಲೇಖ-ಕ-ಪ-ತ್ರಿ-ಕೋ-ದ್ಯ-ಮಿ-ಯಾ-ಗಿದ್ದ
ಲೇಖ-ಕನೂ
ಲೇಖ-ಕರ
ಲೇಖಕಿ
ಲೇಖನ
ಲೇಖ-ನ-ಗಳ
ಲೇಖ-ನ-ಗಳನ್ನು
ಲೇಖ-ನ-ಗಳನ್ನೂ
ಲೇಖ-ನ-ಗಳನ್ನೆಲ್ಲ
ಲೇಖ-ನ-ಗ-ಳಿಂ-ದಲೂ
ಲೇಖ-ನ-ಗಳು
ಲೇಖ-ನ-ಗಳೂ
ಲೇಖ-ನದ
ಲೇಖ-ನ-ದಲ್ಲಿ
ಲೇಖ-ನ-ದಿಂದ
ಲೇಖ-ನ-ವನ್ನು
ಲೇಖ-ನ-ವನ್ನೂ
ಲೇಖ-ನವು
ಲೇಖ-ನ-ವೊಂ-ದನ್ನು
ಲೇಖ-ನ-ವೊಂದು
ಲೇಖನಿ
ಲೇಖ-ನಿಯ
ಲೇಖ-ನಿ-ಯನ್ನು
ಲೇಡಿ
ಲೇವಡಿ
ಲೇವ-ಡಿ-ಮಾ-ಡಿದ
ಲೇಶವೂ
ಲೈಂಗಿಕ
ಲೈಂಗಿ-ಕ-ತೆ-ಯಲ್ಲಿ
ಲೈಮನ್
ಲೈಸಿ-ಯಮ್
ಲೋಕ
ಲೋಕಕ್ಕೆ
ಲೋಕಕ್ಕೇ
ಲೋಕ-ಗು-ರು-ವಾದ
ಲೋಕ-ಗು-ರು-ವಿನ
ಲೋಕದ
ಲೋಕ-ದಲ್ಲಿ
ಲೋಕ-ದ-ಲ್ಲಿ-ದ್ದೆವು
ಲೋಕ-ದ್ದಾ-ಗಿ-ರ-ಲಿಲ್ಲ
ಲೋಕ-ಮಾ-ನ್ಯ-ರೆಂದು
ಲೋಕ-ವೊಂ-ದನ್ನು
ಲೋಕ-ಹಿತ
ಲೋಕ-ಹಿ-ತ-ಕಾ-ರ್ಯ-ಗಳನ್ನು
ಲೋಕಾ-ತೀ-ತನ
ಲೋಕಾ-ನು-ಭವ
ಲೋಕಾ-ನು-ಭ-ವ-ಗಳ
ಲೋಕಾ-ಭಿ-ರಾ-ಮದ
ಲೋಕಾ-ಭಿ-ರಾ-ಮ-ವಾಗಿ
ಲೋಕೋ-ತ್ತರ
ಲೋಪ
ಲೋಪ-ದೋಷ
ಲೋಪ-ದೋ-ಷ-ಗಳ
ಲೋಪ-ದೋ-ಷ-ಗಳನ್ನು
ಲೋಪ-ದೋ-ಷ-ಗಳನ್ನೂ
ಲೋಪ-ದೋ-ಷ-ಗಳೂ
ಲೋಪ-ವನ್ನೋ
ಲೋಲ-ನೆಂದೂ
ಲೋಹ
ಲೋಹ-ಗಳನ್ನು
ಲೋಹದ
ಲೋಹ-ವಿರೆ
ಲೋಹವೇ
ಲೋಹ-ವೊಂ-ದಿ-ರು-ತಿ-ರಲು
ಲೌಕಿಕ
ಲೌಕಿ-ಕ-ರಲ್ಲ
ಲೌಕಿ-ಕಾ-ನು-ಭ-ವ-ಗ-ಳೆ-ಲ್ಲ-ದ-ರ-ಲ್ಲೂ-ಅ-ವ-ನಿ-ಗಿಂತ
ಲ್ಪಟ್ಟದ್ದು
ಲ್ಪಡು-ತ್ತದೆ
ಲ್ಯಾಂಡನ್ನು
ಲ್ಯಾಂಡ್ನಲ್ಲಿ
ಲ್ಯಾಂಡ್ಸ್
ಲ್ಯಾಂಡ್ಸ್ಬ-ರ್ಗ್
ಲ್ಯಾಟಿನ್
ಲ್ಯಾಟೀನ್
ಲ್ಯೂಸ-ರ್ನಿ-ನಲ್ಲಿ
ಲ್ಯೂಸ-ರ್ನಿ-ನಿಂದ
ಲ್ಯೂಸ-ರ್ನ್
ಲ್ಯೂಸ-ರ್ನ್ನಲ್ಲಿ
ಲ್ಲದೆ
ಲ್ಲವೆ
ಲ್ಲಿದ್ದ
ಲ್ಲೆಲ್ಲ
ಲ್ಲೊಂದಾ-ಗಿತ್ತು
ಲ್ಲೊಂದೆಂ-ದರೆ
ಲ್ಲೊಬ್ಬ-ನ-ನ್ನಾಗಿ
ಲ್ಲೊಬ್ಬರ
ಲ್ಲೊಬ್ಬರು
ಳಾದರೂ
ಳಿಗೆ
ಳೆಂದರೆ
ಳೊಂದಿ-ಗಿನ
ಳೊಂದಿಗೆ
ಳೊಬ್ಬಳು
ವಂಚಿ-ತ-ನಾ-ಗಿದ್ದ
ವಂಚಿ-ತ-ಳಾ-ದಳು
ವಂತ-ನನ್ನು
ವಂತ-ರಿಂದ
ವಂತರು
ವಂತ-ವಾಗಿ
ವಂತಹ
ವಂತಾ-ಗ-ಬೇಕು
ವಂತಾ-ಗಲಿ
ವಂತಾ-ಯಿ-ತಲ್ಲ
ವಂತಾ-ಯಿತು
ವಂತಿ-ಕೆ-ಯಿದೆ
ವಂತಿಗೆ
ವಂತಿ-ಗೆ-ಯನ್ನು
ವಂತಿ-ಗೆ-ಯಾಗಿ
ವಂತಿ-ಗೆ-ಯಿಂದ
ವಂತಿತ್ತು
ವಂತಿ-ರ-ಲಿಲ್ಲ
ವಂತಿ-ಲ್ಲ-ಇ-ದೊಂದೇ
ವಂತೆ
ವಂಶ-ಜರ
ವಂಶದ
ವಕ-ವಾಗಿ
ವಕಾ-ರಕ್ಕೆ
ವಕೀಲ
ವಕೀ-ಲ-ಲೇ-ಖಕ
ವಕೀ-ಲ-ನಾ-ಗು-ವು-ದು-ಇದೇ
ವಕೀ-ಲ-ನಾದ
ವಕೀ-ಲ-ನಿಗೆ
ವಕೀ-ಲ-ನೊಬ್ಬ
ವಕೀ-ಲರ
ವಕೀ-ಲ-ರಿಗೆ
ವಕೀ-ಲರು
ವಕೀ-ಲರೂ
ವಕ್ಕ-ರಿ-ಸು-ತ್ತದೆ
ವಕ್ರ
ವಕ್ರ-ಗೊಂಡ
ವಕ್ರೀ-ಕ-ರ-ಣ-ಗೊಂಡ
ವಕ್ರೀ-ಕ-ರ-ಣ-ಗೊ-ಳ್ಳು-ತ್ತವೆ
ವಕ್ಷ-ಗ-ಳಿಂ-ದಾ-ವೃ-ತ-ವಾದ
ವಚ-ನ-ಮಧು
ವಚ-ನ-ಸತ್ಯ
ವಜ್ರ-ದಂತೆ
ವಟ-ಗು-ಟ್ಟಿ-ಕೊಂ-ಡರು
ವಟ-ವೃ-ಕ್ಷ-ದಂತೆ
ವತಿ-ಯಿಂದ
ವದ
ವದಂತಿ
ವದನ
ವದ-ನ-ದಲ್ಲಿ
ವದ-ನ-ರಾಗಿ
ವನ
ವನ-ದಲ್ಲಿ
ವನ-ರಾ-ಜ-ನಂತೆ
ವನ-ರಾಜಿ
ವನ-ರಾ-ಶಿ-ಯಿಂದ
ವನ-ಸಿರಿ
ವನು
ವನ್ನಾ-ದರೂ
ವನ್ನಿಟ್ಟು
ವನ್ನು
ವನ್ನುಂ-ಟು-ಮಾ-ಡಿದೆ
ವನ್ನೂ
ವನ್ನೇ
ವನ್ನೇ-ರು-ವಂತೆ
ವಯ-ಸ್ಕ-ರಾದ
ವಯ-ಸ್ಕ-ಳಾದ
ವಯ-ಸ್ಸನ್ನು
ವಯ-ಸ್ಸಷ್ಟೆ
ವಯ-ಸ್ಸಾ-ಗಿತ್ತು
ವಯ-ಸ್ಸಾ-ಗಿದೆ
ವಯ-ಸ್ಸಿನ
ವಯ-ಸ್ಸಿ-ನಲ್ಲಿ
ವಯ-ಸ್ಸಿ-ನ-ಲ್ಲಿಯೇ
ವಯ-ಸ್ಸಿ-ನಲ್ಲೇ
ವಯಸ್ಸು
ವಯ-ಸ್ಸೊಂ-ದನ್ನು
ವಯೋ-ಮಿ-ತಿ-ಯನ್ನು
ವಯೋ-ವೃದ್ಧ
ವಯೋ-ವೃ-ದ್ಧ-ಜ್ಞಾ-ನ-ವೃ-ದ್ಧ-ನೊಬ್ಬ
ವಯೋ-ವೃ-ದ್ಧ-ರನ್ನೋ
ವರ
ವರಣ
ವರ-ದ-ರಾವ್
ವರದಿ
ವರ-ದಿ-ಗಳನ್ನು
ವರ-ದಿ-ಗಳನ್ನೆಲ್ಲ
ವರ-ದಿ-ಗಳಿಂದ
ವರ-ದಿ-ಗಳು
ವರ-ದಿ-ಗ-ಳೊಂ-ದಿಗೆ
ವರ-ದಿ-ಗಾ-ರನ
ವರ-ದಿ-ಗಾ-ರ-ನೊಬ್ಬ
ವರ-ದಿ-ಗಾ-ರರು
ವರ-ದಿ-ಮಾ-ಡಿತು
ವರ-ದಿ-ಯನ್ನು
ವರ-ದಿ-ಯಲ್ಲಿ
ವರ-ದಿ-ಯಲ್ಲೇ
ವರ-ದಿ-ಯಾ-ಗಿತ್ತು
ವರ-ದಿ-ಯಿಂದ
ವರ-ದಿ-ಯಿಲ್ಲ
ವರ-ದಿಯು
ವರ-ದಿಯೂ
ವರ-ದಿ-ಯೊಂದು
ವರನ್ನು
ವರ-ವಾಗಿ
ವರಿಗೆ
ವರಿ-ಯು-ತ್ತಿತ್ತು
ವರಿ-ಸ-ಲಿಲ್ಲ
ವರು
ವರೂ
ವರೆ
ವರೆಂ-ದರೆ
ವರೆ-ಗಿದು
ವರೆಗೂ
ವರೆಗೆ
ವರೇನೋ
ವರ್ಗ
ವರ್ಗಕ್ಕೆ
ವರ್ಗಕ್ಕೇ
ವರ್ಗ-ಗ-ಗಳ
ವರ್ಗ-ಗಳ
ವರ್ಗ-ಗ-ಳಲ್ಲೂ
ವರ್ಗ-ಗಳಿಂದ
ವರ್ಗದ
ವರ್ಗ-ದಲ್ಲಿ
ವರ್ಗ-ದಲ್ಲೂ
ವರ್ಗ-ದ-ವರ
ವರ್ಗ-ದ-ವ-ರನ್ನು
ವರ್ಗ-ದ-ವ-ರು-ಪು-ಳ-ಕಿ-ತ-ರಾ-ದರು
ವರ್ಗ-ದ-ವರೂ
ವರ್ಗ-ಭೇ-ದ-ಗಳನ್ನು
ವರ್ಗಾ-ವಣೆ
ವರ್ಗೀ-ಕ-ರ-ಣ-ವೆಲ್ಲ
ವರ್ಗೀ-ಕ-ರಿಸಿ
ವರ್ಜಿ-ತ-ವಾ-ಗಿ-ತ್ತೆಂ-ಬು-ದನ್ನು
ವರ್ಜ್ಯ
ವರ್ಡಿ-ಯರ್
ವರ್ಡಿ-ಯರ್ಳ
ವರ್ಣ
ವರ್ಣ-ಚಿ-ತ್ರ-ಗಳನ್ನು
ವರ್ಣ-ಚಿ-ತ್ರ-ವನ್ನು
ವರ್ಣ-ನಾ-ತೀತ
ವರ್ಣನೆ
ವರ್ಣ-ನೆ-ಯನ್ನು
ವರ್ಣ-ನೆ-ಯಿಂದ
ವರ್ಣ-ಭೇದ
ವರ್ಣ-ವಿ-ನ್ಯಾ-ಸ-ಗಳನ್ನು
ವರ್ಣಾ-ಶ್ರ-ಮ-ಧ-ರ್ಮ-ಗಳ
ವರ್ಣಿ-ಸ-ಲ-ದ-ಳ-ವಾ-ದದ್ದು
ವರ್ಣಿ-ಸಲು
ವರ್ಣಿಸಿ
ವರ್ಣಿ-ಸಿತು
ವರ್ಣಿ-ಸಿದ
ವರ್ಣಿ-ಸಿ-ದರು
ವರ್ಣಿ-ಸಿ-ರುವ
ವರ್ಣಿಸು
ವರ್ಣಿ-ಸುತ್ತ
ವರ್ಣಿ-ಸು-ವಂತೆ
ವರ್ತಕ
ವರ್ತ-ಕ-ರಿಂದ
ವರ್ತ-ಕರು
ವರ್ತನೆ
ವರ್ತ-ನೆ-ಗಾಗಿ
ವರ್ತ-ನೆ-ಯನ್ನು
ವರ್ತ-ನೆ-ಯಲ್ಲಿ
ವರ್ತ-ನೆ-ಯಿಂದ
ವರ್ತ-ನೆಯು
ವರ್ತ-ನೆಯೂ
ವರ್ತ-ನೆ-ಯೆಂದು
ವರ್ತ-ಮಾನ
ವರ್ತ-ಮಾನವೂ
ವರ್ತಿ-ಸ-ಬೇಕೋ
ವರ್ತಿ-ಸ-ಲಾ-ರಂ-ಭಿ-ಸಿದ
ವರ್ತಿಸಿ
ವರ್ತಿ-ಸಿ-ದರೂ
ವರ್ತಿ-ಸಿ-ದರೆ
ವರ್ತಿ-ಸು-ತ್ತಾರೆ
ವರ್ತಿ-ಸು-ತ್ತಿ-ದ್ದಾರೆ
ವರ್ತಿ-ಸು-ತ್ತಿ-ರು-ವ-ವ-ರೆಗೆ
ವರ್ತಿ-ಸು-ವು-ದಿಲ್ಲ
ವರ್ತಿ-ಸು-ವುದು
ವರ್ತ್ಯಾ-ನು-ವ-ರ್ತಂತೇ
ವರ್ಮ-ನೊಂ-ದಿಗೆ
ವರ್ಲ್ಡ್
ವರ್ಷ
ವರ್ಷಕ್ಕೆ
ವರ್ಷ-ಗಳ
ವರ್ಷ-ಗಳನ್ನು
ವರ್ಷ-ಗ-ಳ-ಲ್ಲಂತೂ
ವರ್ಷ-ಗಳಲ್ಲಿ
ವರ್ಷ-ಗ-ಳ-ವ-ರೆಗೆ
ವರ್ಷ-ಗ-ಳಾ-ದ-ಮೇಲೆ
ವರ್ಷ-ಗಳಿಂದ
ವರ್ಷ-ಗ-ಳಿಂ-ದಲೂ
ವರ್ಷ-ಗ-ಳಿಗೂ
ವರ್ಷ-ಗ-ಳಿಗೆ
ವರ್ಷ-ಗ-ಳಿ-ರು-ವಾ-ಗ-ಲೇ-ಸ್ವಾ-ಮೀಜಿ
ವರ್ಷ-ಗಳು
ವರ್ಷ-ಗಳೂ
ವರ್ಷ-ಗಳೇ
ವರ್ಷದ
ವರ್ಷ-ದಲ್ಲಿ
ವರ್ಷ-ದಲ್ಲೇ
ವರ್ಷ-ದ-ವ-ಳಾ-ಗಿದ್ದ
ವರ್ಷ-ದಷ್ಟು
ವರ್ಷ-ದಿಂದ
ವರ್ಷ-ದಿಂ-ದಲೂ
ವರ್ಷ-ದೊ-ಳ-ಗಾಗಿ
ವರ್ಷ-ವಾ-ಗ-ಲಿದೆ
ವರ್ಷ-ವಾ-ಗಿತ್ತು
ವರ್ಷ-ವಾ-ಗಿ-ತ್ತೆ-ನ್ನ-ಬ-ಹುದು
ವರ್ಷ-ವಿಡೀ
ವರ್ಷವೂ
ವರ್ಷವೋ
ವರ್ಷಾಂ-ತ-ರ-ಗಳಿಂದ
ವರ್ಷಿ-ಸಲು
ವರ್ಷಿ-ಸಿ-ದರು
ವಲಂ-ಬಿ-ಸು-ವಂತೆ
ವಲ-ಯಕ್ಕೆ
ವಲ-ಯ-ಗಳಲ್ಲಿ
ವಲ-ಯ-ದಲ್ಲಿ
ವಲಸೆ
ವಲ್
ವಲ್ಲ
ವಲ್ಲದೆ
ವಲ್ಲಿ
ವವ-ನಲ್ಲ
ವವ-ನಿಗೆ
ವವ-ರನ್ನೇ
ವವ-ರೆಗೆ
ವಶಕ್ಕೆ
ವಶ-ಪ-ಡಿ-ಸಿ-ಕೊಂ-ಡಿ-ದ್ದಾ-ನೆಂದು
ವಶೀ-ಕ-ರಣ
ವಷ್ಟು
ವಸ-ತಿ-ಗೃ-ಹಕ್ಕೆ
ವಸ-ತಿ-ಗೃ-ಹ-ದಲ್ಲಿ
ವಸ-ತಿ-ಗೃ-ಹ-ವೊಂ-ದ-ರಲ್ಲಿ
ವಸ-ತಿಗೆ
ವಸ-ತಿಯ
ವಸ-ತಿ-ಯನ್ನು
ವಸಿ-ಷ್ಠ-ರಿಂದ
ವಸಿ-ಷ್ಠರು
ವಸೂಲಿ
ವಸ್ತು
ವಸ್ತು-ಗಳ
ವಸ್ತು-ಗ-ಳ-ನ್ನಾ-ಗಲಿ
ವಸ್ತು-ಗಳನ್ನು
ವಸ್ತು-ಗಳನ್ನೂ
ವಸ್ತು-ಗಳನ್ನೆಲ್ಲ
ವಸ್ತು-ಗಳಲ್ಲಿ
ವಸ್ತು-ಗ-ಳಷ್ಟೇ
ವಸ್ತು-ಗ-ಳಾ-ವು-ದ-ರಿಂ-ದಲೂ
ವಸ್ತು-ಗಳು
ವಸ್ತು-ಗಳೂ
ವಸ್ತು-ಗ-ಳೊಂ-ದಿಗೆ
ವಸ್ತು-ಪ್ರ-ದ-ರ್ಶ-ನ-ದಲ್ಲಿ
ವಸ್ತು-ಪ್ರ-ದ-ರ್ಶ-ನ-ದಲ್ಲೇ
ವಸ್ತು-ಪ್ರ-ದ-ರ್ಶ-ನ-ವೊಂದು
ವಸ್ತು-ಪ್ರ-ದ-ರ್ಶ-ನಾ-ಲ-ಯ-ವನ್ನೂ
ವಸ್ತು-ವನ್ನು
ವಸ್ತು-ವನ್ನೂ
ವಸ್ತು-ವಾ-ಗಲು
ವಸ್ತು-ವಿಗೂ
ವಸ್ತು-ವಿನ
ವಸ್ತು-ವಿ-ನಂತೆ
ವಸ್ತು-ವಿ-ನಿಂದ
ವಸ್ತು-ವಿ-ಶೇ-ಷ-ಗಳು
ವಸ್ತುವೂ
ವಸ್ತು-ವೆಂದರೆ
ವಸ್ತುವೋ
ವಸ್ತು-ಸಂ-ಗ್ರ-ಹಾ-ಲಯ
ವಸ್ತು-ಸಂ-ಗ್ರ-ಹಾ-ಲ-ಯ-ಗಳನ್ನೂ
ವಸ್ತು-ಸಂ-ಗ್ರ-ಹಾ-ಲ-ಯ-ಗ-ಳಿಗೆ
ವಸ್ತು-ಸಂ-ಗ್ರ-ಹಾ-ಲ-ಯ-ವನ್ನೂ
ವಸ್ತು-ಸ್ಥಿತಿ
ವಸ್ತು-ಸ್ಥಿ-ತಿ-ಯನ್ನು
ವಸ್ತ್ರ
ವಸ್ತ್ರ-ಗಳನ್ನು
ವಸ್ತ್ರದ
ವಸ್ತ್ರ-ಧಾ-ರಿ-ಯಾಗಿ
ವಸ್ತ್ರ-ವನ್ನು
ವಸ್ತ್ರ-ವಿ-ರು-ವುದು
ವಸ್ತ್ರವು
ವಸ್ತ್ರ-ವೊಂ-ದನ್ನು
ವಹಿ-ಸ-ದಿ-ದ್ದರೆ
ವಹಿ-ಸದೆ
ವಹಿ-ಸ-ಬ-ಲ್ಲು-ದೆಂಬು
ವಹಿ-ಸ-ಬ-ಲ್ಲು-ದೆಂ-ಬುದು
ವಹಿ-ಸ-ಬ-ಹು-ದೆಂದು
ವಹಿ-ಸ-ಬೇ-ಕಾ-ಗು-ತ್ತದೆ
ವಹಿ-ಸ-ಬೇಕು
ವಹಿ-ಸ-ಲಿದೆ
ವಹಿಸಿ
ವಹಿ-ಸಿ-ಕೊಂಡ
ವಹಿ-ಸಿ-ಕೊಂ-ಡಳು
ವಹಿ-ಸಿ-ಕೊಂ-ಡ-ವರು
ವಹಿ-ಸಿ-ಕೊಂಡು
ವಹಿ-ಸಿ-ಕೊಳ್ಳ
ವಹಿ-ಸಿ-ಕೊ-ಳ್ಳು-ವು-ದಾಗಿ
ವಹಿ-ಸಿದ
ವಹಿ-ಸಿ-ದರು
ವಹಿ-ಸಿ-ದು-ದಕ್ಕೆ
ವಹಿ-ಸಿದ್ದ
ವಹಿ-ಸಿ-ದ್ದ-ರಿಂದ
ವಹಿ-ಸಿ-ದ್ದರು
ವಹಿ-ಸು-ತ್ತಿ-ದ್ದರು
ವಹಿ-ಸು-ತ್ತಿ-ದ್ದಳು
ವಹಿ-ಸು-ತ್ತಿ-ದ್ದಾರೆ
ವಹಿ-ಸು-ವಂತೆ
ವಹಿ-ಸು-ವಲ್ಲಿ
ವಹಿ-ಸು-ವುದನ್ನು
ವಹಿ-ಸು-ವುದು
ವಾಂಛೀ-ಶ್ವರ
ವಾಕ್
ವಾಕ್ಪ-ಟುತ್ವ
ವಾಕ್ಪ-ಟು-ತ್ವ-ಇವು
ವಾಕ್ಪ-ಟು-ತ್ವ-ಗಳು
ವಾಕ್ಪ್ರ-ವಾ-ಹ-ದಲ್ಲಿ
ವಾಕ್ಪ್ರ-ವಾ-ಹ-ವನ್ನು
ವಾಕ್ಯ-ಗ-ಳಂತೂ
ವಾಕ್ಯ-ಗಳನ್ನೆಲ್ಲ
ವಾಕ್ಯ-ಗಳು
ವಾಕ್ಯ-ದಲ್ಲಿ
ವಾಕ್ಯ-ವನ್ನು
ವಾಕ್ಲ-ಹ-ರಿ-ಯಿಂದ
ವಾಕ್ಶ-ಕ್ತಿ-ಯಿಂದ
ವಾಕ್ಸ-ರ-ಣಿಯ
ವಾಕ್ಸಾ-ಮ-ರ್ಥ್ಯ-ವನ್ನು
ವಾಕ್ಸಿ-ದ್ಧಿಯ
ವಾಕ್ಸ್
ವಾಗ
ವಾಗ-ದಿ-ದ್ದರೆ
ವಾಗ-ಬಲ್ಲ
ವಾಗ-ಬ-ಲ್ಲುದು
ವಾಗ-ಬ-ಹು-ದು-ಎಂದು
ವಾಗ-ಲಿಲ್ಲ
ವಾಗಲು
ವಾಗಲೂ
ವಾಗಲೇ
ವಾಗಿ
ವಾಗಿ-ಆ-ಧ್ಯಾ-ತ್ಮಿ-ಕ-ವಾಗಿ
ವಾಗಿತ್ತು
ವಾಗಿ-ತ್ತೆಂ-ಬು-ದನ್ನು
ವಾಗಿ-ದೆಯೋ
ವಾಗಿದ್ದ
ವಾಗಿದ್ದು
ವಾಗಿ-ದ್ದುವು
ವಾಗಿ-ಬಿ-ಟ್ಟರೆ
ವಾಗಿಯೂ
ವಾಗಿಯೇ
ವಾಗಿ-ರಲಿ
ವಾಗಿ-ರ-ಲಿಲ್ಲ
ವಾಗಿ-ರು-ತ್ತಿ-ದ್ದು-ದ-ರಿಂದ
ವಾಗಿ-ರು-ವುದು
ವಾಗಿಲ್ಲ
ವಾಗಿ-ಸಿ-ಕೊ-ಳ್ಳುವ
ವಾಗುತ್ತ
ವಾಗು-ತ್ತದೆ
ವಾಗು-ತ್ತಿತ್ತು
ವಾಗು-ತ್ತಿ-ರ-ಲಿಲ್ಲ
ವಾಗು-ತ್ತಿ-ರು-ವಂತೆ
ವಾಗು-ತ್ತಿವೆ
ವಾಗುವ
ವಾಗು-ವಂತೆ
ವಾಗು-ವ-ಷ್ಟರ
ವಾಗು-ವುದು
ವಾಗ್ಜಾಲ
ವಾಗ್ಝರಿ
ವಾಗ್ಝ-ರಿಯ
ವಾಗ್ಝ-ರಿ-ಯನ್ನು
ವಾಗ್ಝ-ರಿ-ಯಲ್ಲಿ
ವಾಗ್ಝ-ರಿ-ಯಿಂದ
ವಾಗ್ಝ-ರಿಯು
ವಾಗ್ದಾನ
ವಾಗ್ದೇ-ವಿ-ಯನ್ನು
ವಾಗ್ಧಾ-ರೆ-ಯಿಂದ
ವಾಗ್ಮಿ
ವಾಗ್ಮಿ-ಗಳೂ
ವಾಗ್ಮಿತೆ
ವಾಗ್ಮಿ-ತೆಯ
ವಾಗ್ಮಿ-ತೆ-ಯಿಂದ
ವಾಗ್ಮಿ-ಯಾಗಿ
ವಾಗ್ಮಿಯೂ
ವಾಗ್ವಾದ
ವಾಗ್ವಾ-ದ-ವೆಂ-ಥದು
ವಾಗ್ವೈ-ಖರಿ
ವಾಗ್ವೈ-ಖ-ರಿ-ಯನ್ನು
ವಾಗ್ವೈ-ಖ-ರಿ-ಯಿಂದ
ವಾಚಕ
ವಾಚ-ನಾ-ಲಯ
ವಾಚ-ನಾ-ಲ-ಯಕ್ಕೆ
ವಾಚಾ-ಮ-ಗೋ-ಚ-ರ-ವಾಗಿ
ವಾಚಾಲಂ
ವಾಚಾ-ಳಿ-ಯ-ನ್ನಾಗಿ
ವಾಡಿ-ಕೊಂಡು
ವಾಣಿ
ವಾಣಿ-ಯಲ್ಲಿ
ವಾಣಿ-ಯಿಂ-ದಲೇ
ವಾಣಿಯು
ವಾತ-ವ-ರಣ
ವಾತಾ
ವಾತಾ-ವ-ರಣ
ವಾತಾ-ವ-ರ-ಣಕ್ಕೆ
ವಾತಾ-ವ-ರ-ಣದ
ವಾತಾ-ವ-ರ-ಣ-ದಲ್ಲಿ
ವಾತಾ-ವ-ರ-ಣ-ದ-ಲ್ಲಿ-ದ್ದೆವು
ವಾತಾ-ವ-ರ-ಣ-ದ-ಲ್ಲಿ-ದ್ದೇನೆ
ವಾತಾ-ವ-ರ-ಣ-ದಿಂದ
ವಾತಾ-ವ-ರ-ಣ-ವನ್ನು
ವಾತಾ-ವ-ರ-ಣ-ವ-ನ್ನುಂಟು
ವಾತಾ-ವ-ರ-ಣ-ವ-ನ್ನೆಲ್ಲ
ವಾತಾ-ವ-ರ-ಣ-ವಿ-ದೆಯೇ
ವಾತಾ-ವ-ರ-ಣ-ವಿದ್ದ
ವಾತಾ-ವ-ರ-ಣ-ವಿ-ರು-ತ್ತಿತ್ತು
ವಾತಾ-ವ-ರ-ಣವು
ವಾತಾ-ವ-ರ-ಣ-ವೇ-ರ್ಪ-ಟ್ಟಿತು
ವಾದ
ವಾದ-ವಿ-ವಾ-ದ-ಗಳು
ವಾದಂ-ತಾ-ಯಿತು
ವಾದಂ-ತಿತ್ತು
ವಾದಕ
ವಾದ-ಕ-ನಾ-ಗಿದ್ದ
ವಾದ-ಕ-ರನ್ನೂ
ವಾದಕ್ಕೆ
ವಾದದ
ವಾದ-ದಲ್ಲಿ
ವಾದ-ದೊಂ-ದಿಗೆ
ವಾದ-ದ್ದನ್ನು
ವಾದ-ದ್ದಾ-ಗಿತ್ತು
ವಾದ-ದ್ದಾ-ಗಿ-ರ-ಬ-ಹುದು
ವಾದದ್ದು
ವಾದ-ದ್ದೆಂದು
ವಾದರೂ
ವಾದರೆ
ವಾದ-ರೆ-ನೀವು
ವಾದ-ವನ್ನು
ವಾದ-ವನ್ನೂ
ವಾದ-ವಿ-ವಾ-ದದ
ವಾದವು
ವಾದ-ವು-ಗಳು
ವಾದ-ವೆಂದು
ವಾದಷ್ಟು
ವಾದಷ್ಟೂ
ವಾದ-ಸ-ರ-ಣಿ-ಯನ್ನು
ವಾದ-ಸ-ರ-ಣಿ-ಯನ್ನೇ
ವಾದಾಗ
ವಾದಿ
ವಾದಿನ್
ವಾದಿ-ಯೆಂದೂ
ವಾದಿಸಿ
ವಾದಿ-ಸಿ-ದರು
ವಾದುದು
ವಾದ್ಯದ
ವಾದ್ಯ-ವನ್ನು
ವಾಧ್ವಾನ್
ವಾಧ್ವಾನ್ಗೆ
ವಾನ-ಪ್ರ-ಸ್ಥಾ-ಶ್ರ-ಮ-ವನ್ನು
ವಾನ-ಪ್ರ-ಸ್ಥಾ-ಶ್ರ-ಮಿ-ಗಳ
ವಾನ್ಹಾ-ಗೆನ್
ವಾಪಸು
ವಾಪಸ್ಸು
ವಾಪಿಂ-ಗ-ನ-ದಲ್ಲಿ
ವಾಮಾ-ಚಾರ
ವಾಮಾ-ಚಾ-ರಿ-ಗಳು
ವಾಯಿತು
ವಾಯಿತೋ
ವಾಯಿದೆ
ವಾಯು
ವಾಯು-ವಿ-ಹಾ-ರಕ್ಕೆ
ವಾರ
ವಾರ-ಕ್ಕಿಂ-ತಲೂ
ವಾರಕ್ಕೆ
ವಾರ-ಕ್ಕೊಂ-ದ-ರಂತೆ
ವಾರ-ಕ್ಕೊಮ್ಮೆ
ವಾರ-ಗ-ಟ್ಟ-ಲೆಯ
ವಾರ-ಗಳ
ವಾರ-ಗ-ಳಂತೂ
ವಾರ-ಗಳನ್ನು
ವಾರ-ಗಳಲ್ಲಿ
ವಾರ-ಗ-ಳ-ವ-ರೆಗೆ
ವಾರ-ಗ-ಳಷ್ಟು
ವಾರ-ಗ-ಳಾ-ಗಿತ್ತು
ವಾರ-ಗ-ಳಾ-ದವು
ವಾರ-ಗ-ಳಿ-ಗಿಂ-ತಲೂ
ವಾರ-ಗ-ಳಿಗೂ
ವಾರ-ಗ-ಳೊ-ಳ-ಗಾಗಿ
ವಾರದ
ವಾರ-ದ-ನಂ-ತರ
ವಾರ-ದಲ್ಲಿ
ವಾರ-ಪ-ತ್ರಿ-ಕೆ-ಯಲ್ಲಿ
ವಾರ-ವಾದ
ವಾರ-ವಿಡೀ
ವಾರ-ವಿದ್ದು
ವಾರವೇ
ವಾರ-ಸು-ದಾ-ರರು
ವಾರಾಂ-ಗ-ನೆ-ಯ-ರನ್ನು
ವಾರಾಂ-ತ್ಯಕ್ಕೆ
ವಾರಿ-ಧಿ-ಯಲ್ಲಿ
ವಾರಿ-ಯನ್ನು
ವಾರು
ವಾರ್ಡ್
ವಾರ್ತೆ-ಯನ್ನು
ವಾಲ-ತೊಡ
ವಾಲ್ಟರ್
ವಾಲ್ಡೊ
ವಾಲ್ಡೊಳ
ವಾಲ್ಡೊ-ಳಿ-ಗಿದ್ದ
ವಾಲ್ಡೊ-ಳಿಗೆ
ವಾಲ್ಡೋ
ವಾಲ್ಮೀಕಿ
ವಾವು-ದಿದೆ
ವಾಷಿಂಗ್
ವಾಷಿಂ-ಗ್ಟನ್
ವಾಷಿಂ-ಗ್ಟನ್ಗೆ
ವಾಷಿಂ-ಗ್ಟ-ನ್ನಲ್ಲಿ
ವಾಸ
ವಾಸ-ಕ್ಕಾ-ಗಿಯೂ
ವಾಸಕ್ಕೆ
ವಾಸ-ಗಳ
ವಾಸ-ಗೃ-ಹ-ದಲ್ಲೇ
ವಾಸದ
ವಾಸ-ದಿಂದ
ವಾಸ-ದಿಂ-ದಾಗಿ
ವಾಸನೆ
ವಾಸ-ಮಾ-ಡು-ತ್ತಿ-ದ್ದಾರೆ
ವಾಸ-ವಾಗಿ
ವಾಸ-ವಾ-ಗಿದ್ದ
ವಾಸ-ವಾ-ಗಿ-ದ್ದ-ನೆಂದು
ವಾಸ-ವಾ-ಗಿ-ದ್ದರು
ವಾಸ-ವಾ-ಗಿ-ದ್ದರೂ
ವಾಸ-ವಾ-ಗಿ-ದ್ದಾಗ
ವಾಸ-ವಾ-ಗಿದ್ದು
ವಾಸ-ವಾ-ಗಿ-ದ್ದು-ದಾಗಿ
ವಾಸ-ವಾ-ಗಿದ್ದೆ
ವಾಸ-ವಾ-ಗಿ-ದ್ದೇನೆ
ವಾಸ-ವಾ-ಗಿ-ರ-ದಿ-ದ್ದರೂ
ವಾಸ-ವಾ-ಗಿ-ರ-ಬ-ಲ್ಲೆ-ಅ-ವರ
ವಾಸ-ವಾ-ಗಿ-ರು-ತ್ತಿ-ದ್ದಳು
ವಾಸ-ವಾ-ಗಿ-ರುವ
ವಾಸ-ವಾ-ಗಿ-ರು-ವು-ದಾಗಿ
ವಾಸಿ
ವಾಸಿ-ಯಾ-ಗು-ತ್ತದೆ
ವಾಸಿ-ಯಾ-ಗು-ತ್ತ-ದೆಂಬ
ವಾಸಿ-ಸ-ಲಾ-ರಂ-ಭಿ-ಸಿ-ದರು
ವಾಸಿ-ಸ-ಲಿ-ದ್ದಾಳೆ
ವಾಸಿ-ಸ-ಲಿ-ರು-ವ-ವರು
ವಾಸಿ-ಸಿದ
ವಾಸಿ-ಸಿ-ದರು
ವಾಸಿ-ಸಿ-ದ್ದೇನೆ
ವಾಸಿ-ಸುತ್ತ
ವಾಸಿ-ಸು-ತ್ತಿದ್ದ
ವಾಸಿ-ಸು-ತ್ತಿ-ದ್ದ-ವರೆಲ್ಲ
ವಾಸಿ-ಸು-ವ-ವರು
ವಾಸಿ-ಸು-ವು-ದ-ಕ್ಕಿಂತ
ವಾಸಿ-ಸು-ವು-ದ-ರಿಂದ
ವಾಸ್ತ-ವ-ವಾ-ಗಿಯೂ
ವಾಸ್ತ-ವಾಂಶ
ವಾಸ್ತ-ವಾಂ-ಶವೇ
ವಾಸ್ತ-ವಿಕ
ವಾಸ್ತ-ವಿ-ಕತೆ
ವಾಸ್ತ-ವಿ-ಕ-ವಾಗಿ
ವಾಸ್ತ-ವಿ-ಕ-ವಾ-ಗಿಯೂ
ವಾಸ್ತವ್ಯ
ವಾಸ್ತ-ವ್ಯದ
ವಾಸ್ತು
ವಾಸ್ತು-ಶಿ-ಲ್ಪದ
ವಾಸ್ತು-ಶಿ-ಲ್ಪ-ವನ್ನೂ
ವಿ
ವಿಂಗ-ಡಿಸಿ
ವಿಂಬ-ಲ್ಡ-ನ್ನಿ-ನಲ್ಲಿ
ವಿಕಟ
ವಿಕ-ಸನ
ವಿಕ-ಸ-ನ-ಮು-ನ್ನಡೆ
ವಿಕ-ಸಿತ
ವಿಕಾಸ
ವಿಕಾ-ಸ-ಗೊ-ಳ್ಳುವ
ವಿಕಾ-ಸದ
ವಿಕಾ-ಸ-ದಲ್ಲಿ
ವಿಕಾ-ಸ-ವಾ-ಗುತ್ತ
ವಿಕಾ-ಸ-ವಾ-ದ-ಗಳು
ವಿಕೃತ
ವಿಕೃ-ತ-ಗೊ-ಳ್ಳು-ತ್ತದೆ
ವಿಕ್ಟೋ-ರಿಯಾ
ವಿಖ್ಯಾತ
ವಿಖ್ಯಾ-ತ-ರಾಗಿ
ವಿಖ್ಯಾ-ತ-ರೊಂ-ದಿಗೆ
ವಿಖ್ಯಾ-ತ-ಳಾದ
ವಿಖ್ಯಾತಿ
ವಿಗಳನ್ನು
ವಿಗ್ರಹ
ವಿಗ್ರ-ಹ-ಗಳನ್ನು
ವಿಗ್ರ-ಹ-ಗ-ಳಿವೆ
ವಿಗ್ರ-ಹದ
ವಿಗ್ರ-ಹ-ದಂತೆ
ವಿಗ್ರ-ಹ-ದಲ್ಲಿ
ವಿಗ್ರ-ಹ-ವನ್ನು
ವಿಗ್ರ-ಹ-ವಿದೆ
ವಿಗ್ರ-ಹವು
ವಿಗ್ರ-ಹಾ-ರಾ-ಧ-ಕರೆ
ವಿಗ್ರ-ಹಾ-ರಾ-ಧನೆ
ವಿಗ್ರ-ಹಾ-ರಾ-ಧ-ನೆ-ಯನ್ನೂ
ವಿಗ್ರ-ಹಾ-ರಾ-ಧ-ನೆ-ಯೆಂ-ದ-ರೇನು
ವಿಚ-ಲಿ-ತ-ಗೊ-ಳಿ-ಸದೆ
ವಿಚ-ಲಿ-ತ-ಗೊ-ಳಿ-ಸು-ವಂ-ತಹ
ವಿಚ-ಲಿ-ತ-ನಾ-ಗದೆ
ವಿಚ-ಲಿ-ತ-ರಾ-ಗದೆ
ವಿಚ-ಲಿ-ತ-ರಾ-ಗಲು
ವಿಚ-ಲಿ-ತ-ರಾ-ಗಿ-ದ್ದರೂ
ವಿಚ-ಲಿ-ತ-ರಾ-ಗು-ತ್ತಿ-ರ-ಲಿಲ್ಲ
ವಿಚ-ಲಿ-ತ-ವಾ-ಗು-ತ್ತಿತ್ತು
ವಿಚಾರ
ವಿಚಾ-ರ-ಗಳ
ವಿಚಾ-ರ-ಗಳನ್ನು
ವಿಚಾ-ರ-ಗಳನ್ನೆಲ್ಲ
ವಿಚಾ-ರ-ಗ-ಳ-ನ್ನೇನೂ
ವಿಚಾ-ರ-ಗಳಲ್ಲಿ
ವಿಚಾ-ರ-ಗ-ಳಾ-ದ್ದ-ರಿಂದ
ವಿಚಾ-ರ-ಗ-ಳಿಗೆ
ವಿಚಾ-ರ-ಗಳು
ವಿಚಾ-ರ-ಗಳೂ
ವಿಚಾ-ರ-ಗ-ಳೆ-ಡೆಗೆ
ವಿಚಾ-ರ-ಗ-ಳೆಲ್ಲ
ವಿಚಾ-ರಣಾ
ವಿಚಾ-ರ-ಣೆ-ಯಿ-ಲ್ಲದೆ
ವಿಚಾ-ರದ
ವಿಚಾ-ರ-ದಲ್ಲಿ
ವಿಚಾ-ರ-ದಲ್ಲೂ
ವಿಚಾ-ರ-ಧಾರೆ
ವಿಚಾ-ರ-ಧಾ-ರೆ-ಇ-ವೆಲ್ಲ
ವಿಚಾ-ರ-ಧಾ-ರೆಗೆ
ವಿಚಾ-ರ-ಧಾ-ರೆಯ
ವಿಚಾ-ರ-ಧಾ-ರೆ-ಯನ್ನು
ವಿಚಾ-ರ-ಧಾ-ರೆ-ಯನ್ನೂ
ವಿಚಾ-ರ-ಧಾ-ರೆ-ಯಿಂದ
ವಿಚಾ-ರ-ಧಾ-ರೆಯು
ವಿಚಾ-ರ-ಪೂ-ರ್ಣ-ವಾದ
ವಿಚಾ-ರ-ಪೂ-ರ್ಣವೂ
ವಿಚಾ-ರ-ಮಾಡಿ
ವಿಚಾ-ರ-ವಂತ
ವಿಚಾ-ರ-ವಂ-ತಹ
ವಿಚಾ-ರ-ವನ್ನು
ವಿಚಾ-ರ-ವಲ್ಲ
ವಿಚಾ-ರ-ವಾಗಿ
ವಿಚಾ-ರ-ವಾ-ಗಿ-ರ-ಬೇಕು
ವಿಚಾ-ರ-ವಿಲ್ಲ
ವಿಚಾ-ರವು
ವಿಚಾ-ರ-ವೆಂದೂ
ವಿಚಾ-ರ-ವೇನೆಂದರೆ
ವಿಚಾ-ರ-ವೊಂ-ದನ್ನು
ವಿಚಾ-ರವೋ
ವಿಚಾ-ರ-ಶೀಲ
ವಿಚಾ-ರಿಸಿ
ವಿಚಾ-ರಿ-ಸಿ-ಕೊಂ-ಡರು
ವಿಚಾ-ರಿ-ಸಿ-ಕೊ-ಳ್ಳು-ತ್ತಾನೆ
ವಿಚಾ-ರಿ-ಸಿದ
ವಿಚಾ-ರಿ-ಸಿ-ದರು
ವಿಚಾ-ರಿ-ಸಿ-ದರೆ
ವಿಚಾ-ರಿ-ಸಿ-ದಾಗ
ವಿಚಾ-ರಿ-ಸು-ವಾಗ
ವಿಚಿತ್ರ
ವಿಚಿ-ತ್ರ-ವಿ-ಕೃ-ತ-ವಾ-ದ-ದ್ದಾ-ಗಿ-ರ-ಲಿ-ಸಂ-ಪೂ-ರ್ಣ-ವಾಗಿ
ವಿಚಿ-ತ್ರ-ವ-ನ್ನೆಲ್ಲ
ವಿಚಿ-ತ್ರ-ವಾಗಿ
ವಿಚಿ-ತ್ರ-ವಾದ
ವಿಚ್ಛಿ-ದ್ರ-ಕಾರಿ
ವಿಜ-ಯದ
ವಿಜ-ಯ-ದ-ಶ-ಮಿಯ
ವಿಜ-ಯವು
ವಿಜ-ಯಿ-ಗ-ಳಾಗಿ
ವಿಜ-ಯಿ-ಯಾ-ಗಲು
ವಿಜ-ಯಿ-ಯಾ-ಗು-ವುದು
ವಿಜ-ಯೋ-ತ್ಸ-ವ-ಗಳನ್ನೂ
ವಿಜಾ-ಪು-ರ-ಕರ್
ವಿಜಿ-ಗೀಷು
ವಿಜೃಂ-ಭ-ಣೆ-ಯಿಂದ
ವಿಜ್ಞಾನ
ವಿಜ್ಞಾ-ನ-ತಂ-ತ್ರ-ಜ್ಞಾ-ನ-ಗಳನ್ನು
ವಿಜ್ಞಾ-ನ-ಮ-ನ-ಶ್ಶಾ-ಸ್ತ್ರ-ಗಳ
ವಿಜ್ಞಾ-ನ-ಗ-ಳೆ-ರಡೂ
ವಿಜ್ಞಾ-ನದ
ವಿಜ್ಞಾ-ನ-ದಲ್ಲೂ
ವಿಜ್ಞಾ-ನ-ದೊಂ-ದಿಗೆ
ವಿಜ್ಞಾ-ನ-ವನ್ನು
ವಿಜ್ಞಾ-ನ-ವನ್ನೂ
ವಿಜ್ಞಾ-ನವು
ವಿಜ್ಞಾ-ನಾ-ನಂ-ದರ
ವಿಜ್ಞಾನಿ
ವಿಜ್ಞಾ-ನಿ-ಗಳು
ವಿಜ್ಞಾ-ನಿ-ಗ-ಳು-ತ-ತ್ವ-ಶಾ-ಸ್ತ್ರ-ಜ್ಞರು
ವಿಜ್ಞಾ-ನಿ-ಗಳೂ
ವಿಟ್ಟು
ವಿಡಂ-ಬನೆ
ವಿಡಂ-ಬ-ನೆಯ
ವಿಡಂ-ಬ-ನೆ-ಯಲ್ಲಿ
ವಿತ-ರಣೆ
ವಿತ-ರ-ಣೆ-ಗೈಯು
ವಿತ್ತು
ವಿದಾಯ
ವಿದಿ-ತ-ವಾ-ಗು-ತ್ತದೆ
ವಿದಿತ್ವಾ
ವಿದೆ
ವಿದೆಯೆ
ವಿದೇ-ಶ-ಗಳಲ್ಲಿ
ವಿದೇಶಿ
ವಿದೇ-ಶಿ-ಗ-ಳಿಗೆ
ವಿದೇ-ಶಿಯ
ವಿದೇ-ಶಿ-ಯ-ನಿಗೂ
ವಿದೇ-ಶೀ-ಯ-ರಿಗೆ
ವಿದೇ-ಶೀ-ಯರು
ವಿದ್ದರೆ
ವಿದ್ಯ-ತೇ-ಯ-ನಾಯ
ವಿದ್ಯಾ
ವಿದ್ಯಾ-ಕ್ಷೇ-ತ್ರ-ದಲ್ಲಿ
ವಿದ್ಯಾ-ಪತಿ
ವಿದ್ಯಾ-ಪಿ-ಪಾ-ಸು-ಗಳೂ
ವಿದ್ಯಾ-ಭ್ಯಾಸ
ವಿದ್ಯಾ-ಭ್ಯಾ-ಸ-ಕ್ಕಾಗಿ
ವಿದ್ಯಾ-ಭ್ಯಾ-ಸಕ್ಕೂ
ವಿದ್ಯಾ-ಭ್ಯಾ-ಸ-ಗಳನ್ನು
ವಿದ್ಯಾ-ಭ್ಯಾ-ಸದ
ವಿದ್ಯಾ-ಭ್ಯಾ-ಸ-ವನ್ನು
ವಿದ್ಯಾ-ಭ್ಯಾ-ಸ-ವನ್ನೂ
ವಿದ್ಯಾ-ಭ್ಯಾ-ಸ-ವೆಂದರೆ
ವಿದ್ಯಾ-ಭ್ಯಾ-ಸ-ವೇನೂ
ವಿದ್ಯಾರ್ಥಿ
ವಿದ್ಯಾ-ರ್ಥಿ-ಗಳ
ವಿದ್ಯಾ-ರ್ಥಿ-ಗಳನ್ನು
ವಿದ್ಯಾ-ರ್ಥಿ-ಗ-ಳ-ನ್ನು-ದ್ದೇ-ಶಿಸಿ
ವಿದ್ಯಾ-ರ್ಥಿ-ಗ-ಳಲ್ಲೇ
ವಿದ್ಯಾ-ರ್ಥಿ-ಗ-ಳಾದ
ವಿದ್ಯಾ-ರ್ಥಿ-ಗ-ಳಿ-ಗಾಗಿ
ವಿದ್ಯಾ-ರ್ಥಿ-ಗ-ಳಿ-ಗಿಂತ
ವಿದ್ಯಾ-ರ್ಥಿ-ಗ-ಳಿಗೆ
ವಿದ್ಯಾ-ರ್ಥಿ-ಗಳು
ವಿದ್ಯಾ-ರ್ಥಿ-ಗ-ಳು-ಉ-ಪಾ-ಧ್ಯಾ-ಯರು
ವಿದ್ಯಾ-ರ್ಥಿ-ಗಳೂ
ವಿದ್ಯಾ-ರ್ಥಿ-ಗ-ಳೆಲ್ಲ
ವಿದ್ಯಾ-ರ್ಥಿ-ಗ-ಳೆ-ಲ್ಲರ
ವಿದ್ಯಾ-ರ್ಥಿ-ಗಳೇ
ವಿದ್ಯಾ-ರ್ಥಿ-ಗಾಗಿ
ವಿದ್ಯಾ-ರ್ಥಿನಿ
ವಿದ್ಯಾ-ರ್ಥಿ-ಯಾ-ಗಿದ್ದ
ವಿದ್ಯಾ-ರ್ಥಿ-ಯಾ-ಗಿ-ದ್ದಾರೆ
ವಿದ್ಯಾ-ಲ-ಯದ
ವಿದ್ಯಾ-ವಂತ
ವಿದ್ಯಾ-ವಂ-ತನೂ
ವಿದ್ಯಾ-ವಂ-ತ-ನೊಬ್ಬ
ವಿದ್ಯಾ-ವಂ-ತ-ರಾ-ಗಿ-ರ-ಬೇಕು
ವಿದ್ಯಾ-ವಂ-ತರು
ವಿದ್ಯಾ-ವಂ-ತ-ರೆ-ನ್ನಿ-ಸಿ-ಕೊಂ-ಡ-ವ-ರಿಂದ
ವಿದ್ಯಾ-ಸಂ-ಸ್ಥೆ-ಗ-ಳಿಗೆ
ವಿದ್ಯಾ-ಸಂ-ಸ್ಥೆ-ಗ-ಳು-ಹೀಗೆ
ವಿದ್ಯಾ-ಸಂ-ಸ್ಥೆ-ಯೊಂ-ದರ
ವಿದ್ಯು
ವಿದ್ಯು-ಚ್ಛ-ಕ್ತಿಯ
ವಿದ್ಯು-ಚ್ಛ-ಕ್ತಿ-ಯನ್ನು
ವಿದ್ಯುತ್
ವಿದ್ಯು-ತ್ತಿ-ನಂಥ
ವಿದ್ಯು-ದೀ-ಕ-ರ-ಣ-ದಲ್ಲಿ
ವಿದ್ಯು-ದ್ದೀ-ಪದ
ವಿದ್ಯೆ
ವಿದ್ಯೆ-ಸ್ಥಾ-ನ-ಮಾ-ನ-ಗಳ
ವಿದ್ಯೆ-ಯನ್ನು
ವಿದ್ಯೆ-ಯಲ್ಲಿ
ವಿದ್ವ-ಜ್ಜ-ನರ
ವಿದ್ವ-ಜ್ಜೀ-ವನ
ವಿದ್ವ-ತ್ತನ್ನು
ವಿದ್ವ-ತ್ತಿ-ಗಾಗಿ
ವಿದ್ವ-ತ್ತಿನ
ವಿದ್ವ-ತ್ತಿ-ನಿಂದ
ವಿದ್ವತ್ತು
ವಿದ್ವ-ತ್ತು-ಗಳಿಂದ
ವಿದ್ವ-ತ್ಪೂರ್ಣ
ವಿದ್ವ-ತ್ಪೂ-ರ್ಣವೂ
ವಿದ್ವ-ತ್ಸ-ಭೆ-ಯ-ನ್ನು-ದ್ದೇಶಿ
ವಿದ್ವ-ದ್ಗೋ-ಷ್ಠಿ-ಯೊಂದು
ವಿದ್ವ-ನ್ಮ-ಣಿ-ಯಾದ
ವಿದ್ವಾಂ-ಸ-ನನ್ನೂ
ವಿದ್ವಾಂ-ಸ-ನಾ-ದರೂ
ವಿದ್ವಾಂ-ಸರ
ವಿದ್ವಾಂ-ಸ-ರನ್ನು
ವಿದ್ವಾಂ-ಸ-ರನ್ನೂ
ವಿದ್ವಾಂ-ಸ-ರಾ-ಗಿ-ದ್ದರೂ
ವಿದ್ವಾಂ-ಸರು
ವಿದ್ವಾಂ-ಸರೂ
ವಿದ್ವಾಂ-ಸ-ರೆಲ್ಲ
ವಿದ್ವಾಂ-ಸ-ರೊ-ಬ್ಬರ
ವಿಧದ
ವಿಧ-ದಲ್ಲಿ
ವಿಧರ್ಮಿ
ವಿಧ-ರ್ಮಿ-ಯನ್ನು
ವಿಧ-ರ್ಮಿ-ಯೆಂದು
ವಿಧ-ರ್ಮೀಯ
ವಿಧ-ರ್ಮೀ-ಯ
ವಿಧ-ರ್ಮೀ-ಯ-ನೊ-ಬ್ಬ-ನನ್ನು
ವಿಧ-ವಾಗಿ
ವಿಧ-ವಾ-ವಿ-ವಾಹ
ವಿಧ-ವಿ-ಧದ
ವಿಧವೆ
ವಿಧ-ವೆ-ಪತ್ನಿ
ವಿಧ-ವೆಯ
ವಿಧ-ವೆ-ಯನ್ನು
ವಿಧ-ವೆ-ಯರ
ವಿಧ-ವೆ-ಯ-ರಿಗೆ
ವಿಧ-ವೆ-ಯಾಗಿ
ವಿಧ-ವೆ-ಯಾ-ದಳು
ವಿಧಾನ
ವಿಧಾ-ನ-ಗಳನ್ನು
ವಿಧಾ-ನ-ಗಳನ್ನೂ
ವಿಧಾ-ನ-ಗಳಲ್ಲಿ
ವಿಧಾ-ನ-ಗಳಿಂದ
ವಿಧಾ-ನ-ಗಳು
ವಿಧಾ-ನದ
ವಿಧಾ-ನ-ದಲ್ಲಿ
ವಿಧಾ-ನ-ವನ್ನು
ವಿಧಾ-ನ-ವೆಂದರೆ
ವಿಧಿ
ವಿಧಿ-ಗಳ
ವಿಧಿ-ಗಳನ್ನೂ
ವಿಧಿ-ನಿ-ಯ-ಮ-ಗಳು
ವಿಧಿಯ
ವಿಧಿ-ಯಿ-ಲ್ಲದೆ
ವಿಧಿ-ಯಿ-ಲ್ಲ-ದ್ದ-ರಿಂದ
ವಿಧಿ-ಯಿ-ಲ್ಲ-ವೆ-ನಿ-ಸಿ-ದಾಗ
ವಿಧಿ-ಸಿ-ದ್ದರು
ವಿಧಿ-ಸಿ-ರು-ವು-ದಕ್ಕೆ
ವಿಧಿ-ಸು-ವು-ದಕ್ಕೆ
ವಿಧೇಯ
ವಿಧೇ-ಯ-ತೆಯೇ
ವಿಧೇ-ಯ-ನಾ-ಗಿರು
ವಿಧ್ಯು-ಕ್ತ-ವಾಗಿ
ವಿಧ್ವಂ-ಸಕ
ವಿಧ್ವಂ-ಸ-ಕ-ವಾ-ಗಿ-ರ-ಬಾ-ರದು
ವಿನಂತಿ
ವಿನಂ-ತಿಸಿ
ವಿನಂ-ತಿ-ಸಿ-ಕೊಂಡ
ವಿನಂ-ತಿ-ಸಿ-ಕೊಂ-ಡರು
ವಿನಂ-ತಿ-ಸಿ-ಕೊಂ-ಡಿದ್ದ
ವಿನಂ-ತಿ-ಸಿ-ಕೊ-ಳ್ಳ-ಲಾ-ಯಿತು
ವಿನಂ-ತಿ-ಸಿ-ಕೊ-ಳ್ಳು-ತ್ತಿ-ದ್ದಾ-ರೆಯೋ
ವಿನಂತೆ
ವಿನಮ್ರ
ವಿನ-ಮ್ರ-ಕಾ-ರ್ಯ-ವನ್ನು
ವಿನ-ಮ್ರ-ತೆ-ಯನ್ನೂ
ವಿನ-ಯ-ದಿಂದ
ವಿನ-ಯ-ದಿಂ-ದಲೇ
ವಿನ-ಯ-ಪೂ-ರ್ಣ-ವ-ರ್ತನೆ
ವಿನ-ಯ-ಪೂ-ರ್ವ-ಕ-ವಾಗಿ
ವಿನ-ಯ-ವಂ-ತ-ರಾ-ಗಿ-ದ್ದರೂ
ವಿನ-ಯ-ವನ್ನು
ವಿನ-ಯ-ವಾಗಿ
ವಿನ-ಯ-ವಿತ್ತು
ವಿನ-ಯ-ಶಾಲಿ
ವಿನ-ಯ-ಶೀಲ
ವಿನಾ-ಶ-ಕಾ-ರಿ-ಯಲ್ಲ
ವಿನಾ-ಶ-ಕಾರೀ
ವಿನಾ-ಶಕ್ಕೆ
ವಿನಾ-ಶ-ವಿ-ರು-ವು-ದ-ರಿಂದ
ವಿನಿ-ಮಯ
ವಿನಿ-ಮ-ಯ-ವಾ-ಗ-ಬೇ-ಕೆಂಬ
ವಿನಿ-ಯೋ-ಗ-ವಾ-ಗ-ಬೇ-ಕೆಂಬ
ವಿನಿ-ಯೋಗಿ
ವಿನಿ-ಯೋ-ಗಿಸ
ವಿನಿ-ಯೋ-ಗಿ-ಸಲು
ವಿನಿ-ಯೋ-ಗಿ-ಸಿ-ದರು
ವಿನೀತ
ವಿನೀ-ತ-ರಾಗಿ
ವಿನೂ-ತನ
ವಿನ್ನಿ-ಪೆಗ್ಗೆ
ವಿನ್ಯಾ-ಸ-ಗಳಿಂದ
ವಿಪ-ತ್ತಿನ
ವಿಪ-ರೀತ
ವಿಪ-ರೀ-ತ-ವಾಗಿ
ವಿಪುಲ
ವಿಫ-ಲ-ಗೊಂ-ಡಿದೆ
ವಿಫ-ಲ-ನಾದೆ
ವಿಫ-ಲ-ರಾ-ಗಿದ್ದ
ವಿಫ-ಲ-ರಾ-ಗಿ-ದ್ದೇವೆ
ವಿಫ-ಲ-ರಾ-ದ-ರೆಂದು
ವಿಫ-ಲ-ಳಾಗಿ
ವಿಭ-ಜ-ನೆ-ಯಾ-ಗ-ಬೇಕು
ವಿಭಾ-ಗದ
ವಿಭಾ-ಗ-ದಲ್ಲಿ
ವಿಭಾ-ಗ-ದಲ್ಲೂ
ವಿಭಿನ್ನ
ವಿಭಿ-ನ್ನ-ನಾಗಿ
ವಿಭಿ-ನ್ನ-ಮಯ
ವಿಭಿ-ನ್ನ-ಳೆಂ-ಬು-ದನ್ನು
ವಿಭಿ-ನ್ನ-ವ-ಲ್ಲ-ವೆಂದು
ವಿಭಿ-ನ್ನ-ವಾಗಿ
ವಿಭಿ-ನ್ನ-ವಾ-ಗಿ-ತ್ತೆಂ-ಬು-ದರ
ವಿಭಿ-ನ್ನ-ವಾ-ಗಿ-ರ-ಬ-ಹು-ದಾ-ದರೂ
ವಿಭಿ-ನ್ನ-ವಾದ
ವಿಭಿ-ನ್ನ-ವಾ-ದದ್ದು
ವಿಭಿ-ನ್ನ-ವಾ-ದುದು
ವಿಮ-ರ್ಶ-ಕನ
ವಿಮ-ರ್ಶ-ಕ-ರಾದ
ವಿಮ-ರ್ಶ-ಕ-ರಿಂದ
ವಿಮರ್ಶಾ
ವಿಮ-ರ್ಶಾ-ತ್ಮಕ
ವಿಮ-ರ್ಶಿ-ಸ-ಲಾ-ಯಿತು
ವಿಮ-ರ್ಶಿಸಿ
ವಿಮ-ರ್ಶಿ-ಸಿತು
ವಿಮ-ರ್ಶಿ-ಸು-ತ್ತೇನೆ
ವಿಮ-ರ್ಶೆ-ಗಳನ್ನು
ವಿಮ-ರ್ಶೆ-ಗಳನ್ನೂ
ವಿಮ-ರ್ಶೆ-ಗಳು
ವಿಮ-ರ್ಶೆ-ಯಲ್ಲಿ
ವಿಮಾ
ವಿಮು-ಕ್ತ-ರ-ನ್ನಾ-ಗಿ-ಸು-ತ್ತಾನೆ
ವಿಮುಕ್ತಿ
ವಿಮು-ಖ-ರಾ-ಗು-ತ್ತಿ-ದ್ದರು
ವಿಮು-ಖ-ರಾ-ಗು-ತ್ತಿ-ದ್ದಾರೆ
ವಿಮೋ-ಚ-ನೆಯ
ವಿರಕ್ತ
ವಿರ-ಬೇ-ಕೆ-ನ್ನು-ವುದು
ವಿರ-ಮಿ-ಸ-ಲಾ-ರಿರಿ
ವಿರ-ಮಿ-ಸ-ಲಾರೆ
ವಿರ-ಮಿ-ಸಲು
ವಿರ-ಮಿ-ಸಿ-ದ-ನಂ-ತರ
ವಿರ-ಮಿ-ಸಿದ್ದು
ವಿರ-ಮಿ-ಸು-ವು-ದಿಲ್ಲ
ವಿರ-ಲಿಲ್ಲ
ವಿರಳ
ವಿರಾ-ಜ-ಮಾ-ನ-ರಾದ
ವಿರಾ-ಜಿ-ಸು-ತ್ತಿ-ದ್ದಾಳೆ
ವಿರಾ-ಜಿ-ಸುವ
ವಿರಾ-ಟ್ವ-ದ-ನ-ದಲ್ಲೇ
ವಿರಾ-ಮವೇ
ವಿರುದ್ಧ
ವಿರು-ದ್ಧ-ವಾಗಿ
ವಿರು-ದ್ಧ-ವಾ-ಗಿ-ದ್ದರು
ವಿರು-ದ್ಧ-ವಾ-ಗಿ-ದ್ದಾರೆ
ವಿರು-ದ್ಧ-ವಾ-ಗಿ-ದ್ದು-ದ-ರಲ್ಲಿ
ವಿರು-ದ್ಧ-ವಾ-ಗಿಯೇ
ವಿರು-ದ್ಧ-ವಾ-ಗಿ-ಲ್ಲ-ವೆಂ-ಬು-ದನ್ನು
ವಿರು-ದ್ಧ-ವಾದ
ವಿರು-ದ್ಧ-ವಾ-ದದ್ದು
ವಿರು-ದ್ಧ-ವಾ-ದ-ವು-ಗಳೇ
ವಿರು-ದ್ಧ-ವಾ-ದು-ದಾ-ವುದೂ
ವಿರು-ದ್ಧ-ವಾ-ದುದು
ವಿರು-ದ್ಧ-ವೆಂಬ
ವಿರೋಧ
ವಿರೋ-ಧದ
ವಿರೋ-ಧ-ವನ್ನೂ
ವಿರೋ-ಧ-ವಾಗಿ
ವಿರೋ-ಧ-ವಾ-ಗಿಲ್ಲ
ವಿರೋ-ಧ-ವೇ-ಳ-ಬ-ಹು-ದೆಂದು
ವಿರೋ-ಧಾ-ಭಾ-ಸ-ವನ್ನು
ವಿರೋ-ಧಿ-ಗಳ
ವಿರೋ-ಧಿ-ಗ-ಳಾದ
ವಿರೋ-ಧಿ-ಗ-ಳಿಗೆ
ವಿರೋ-ಧಿ-ಗಳು
ವಿರೋ-ಧಿ-ಗ-ಳೆ-ದುರು
ವಿರೋ-ಧಿ-ಗ-ಳೊಂ-ದಿ-ಗಿನ
ವಿರೋ-ಧಿ-ಯ-ಲ್ಲ-ವೆಂ-ಬು-ದನ್ನು
ವಿರೋ-ಧಿ-ಸ-ಲಾ-ಗ-ದಂತೆ
ವಿರೋ-ಧಿಸಿ
ವಿರೋ-ಧಿ-ಸಿ-ದರೂ
ವಿರೋ-ಧಿ-ಸಿ-ದ-ವರು
ವಿರೋ-ಧಿ-ಸಿ-ದಷ್ಟೂ
ವಿರೋ-ಧಿ-ಸಿ-ದ್ದರು
ವಿರೋ-ಧಿ-ಸು-ತ್ತಿದ್ದ
ವಿರೋ-ಧಿ-ಸು-ತ್ತಿ-ರು-ವು-ದಾಗಿ
ವಿರೋ-ಧಿ-ಸು-ತ್ತೇನೆ
ವಿರೋ-ಧಿ-ಸುವ
ವಿರೋಧೀ
ವಿಲ-ಕ್ಷಣ
ವಿಲಾ-ಸದ
ವಿಲಾ-ಸ-ದಲ್ಲಿ
ವಿಲಾ-ಸ-ಪೂ-ರ್ಣ-ವಾದ
ವಿಲಾ-ಸ-ಯುತ
ವಿಲಿಯಂ
ವಿಲ್ಕಾ-ಕ್ಸ್
ವಿಲ್ಲ
ವಿಲ್ಲದ
ವಿಲ್ಲ-ದಂ-ತೆಯೇ
ವಿಲ್ಲದೆ
ವಿಳಂಬ
ವಿಳಂ-ಬಕ್ಕೆ
ವಿಳಾಸ
ವಿಳಾ-ಸಕ್ಕೆ
ವಿಳಾ-ಸದ
ವಿಳಾ-ಸ-ವನ್ನು
ವಿಳಾ-ಸ-ವನ್ನೂ
ವಿಳಾ-ಸ-ವಾ-ಗಲಿ
ವಿವರ
ವಿವ-ರ-ಗಳ
ವಿವ-ರ-ಗಳನ್ನು
ವಿವ-ರ-ಗಳನ್ನೆಲ್ಲ
ವಿವ-ರ-ಗಳಲ್ಲಿ
ವಿವ-ರ-ಗ-ಳಿಂ-ದಲೂ
ವಿವ-ರ-ಗ-ಳಿ-ದ್ದುವು
ವಿವ-ರ-ಗಳು
ವಿವ-ರ-ಗ-ಳೆಲ್ಲ
ವಿವ-ರ-ಗ-ಳೇ-ನಾ-ದರೂ
ವಿವ-ರಣಾ
ವಿವ-ರಣೆ
ವಿವ-ರ-ಣೆ-ಗ-ಳಿಗೆ
ವಿವ-ರ-ಣೆ-ಗಳು
ವಿವ-ರ-ಣೆ-ಗ-ಳೆಲ್ಲ
ವಿವ-ರ-ಣೆ-ಗ-ಳೇನೇ
ವಿವ-ರ-ಣೆ-ಯನ್ನು
ವಿವ-ರ-ಣೆ-ಯಲ್ಲಿ
ವಿವ-ರ-ಣೆ-ಯಿಂದ
ವಿವ-ರ-ಣೆ-ಯಿದೆ
ವಿವ-ರ-ಪೂರ್ಣ
ವಿವ-ರ-ಪೂ-ರ್ಣ-ವಾದ
ವಿವ-ರ-ಪೂ-ರ್ಣ-ವಾ-ದದ್ದು
ವಿವ-ರ-ಪೂ-ರ್ಣವೂ
ವಿವ-ರ-ವನ್ನು
ವಿವ-ರ-ವಾಗಿ
ವಿವ-ರ-ವಾದ
ವಿವರಿ
ವಿವ-ರಿ-ಸ-ತೊ-ಡ-ಗಿ-ದಂತೆ
ವಿವ-ರಿ-ಸ-ತೊ-ಡ-ಗಿ-ದರೆ
ವಿವ-ರಿ-ಸ-ತೊ-ಡ-ಗಿ-ದ-ರೆಂ-ದರೆ
ವಿವ-ರಿ-ಸ-ತೊ-ಡ-ಗು-ತ್ತಿ-ದ್ದರು
ವಿವ-ರಿ-ಸ-ಬೇಕಾ
ವಿವ-ರಿ-ಸ-ಬೇ-ಕಾ-ಗು-ತ್ತದೆ
ವಿವ-ರಿ-ಸ-ಬೇ-ಕೆಂದು
ವಿವ-ರಿ-ಸ-ಲಾ-ದೀತು
ವಿವ-ರಿ-ಸ-ಲಾ-ರಂ-ಭಿಸಿ
ವಿವ-ರಿ-ಸ-ಲಿಲ್ಲ
ವಿವ-ರಿ-ಸಲು
ವಿವ-ರಿ-ಸ-ಲ್ಪ-ಟ್ಟದ್ದು
ವಿವ-ರಿ-ಸ-ಹೊ-ರ-ಟಾಗ
ವಿವ-ರಿಸಿ
ವಿವ-ರಿ-ಸಿ-ಕೊಂಡು
ವಿವ-ರಿ-ಸಿದ
ವಿವ-ರಿ-ಸಿ-ದ-ರ-ಲ್ಲದೆ
ವಿವ-ರಿ-ಸಿ-ದರು
ವಿವ-ರಿ-ಸಿ-ದ-ರು-ನಾನು
ವಿವ-ರಿ-ಸಿ-ದ-ರು-ನೋಡಿ
ವಿವ-ರಿ-ಸಿ-ದರೆ
ವಿವ-ರಿ-ಸಿ-ದ-ರೆಂ-ದರೆ
ವಿವ-ರಿ-ಸಿ-ದಾಗ
ವಿವ-ರಿ-ಸಿ-ದುದು
ವಿವ-ರಿ-ಸಿ-ದ್ದರು
ವಿವ-ರಿಸು
ವಿವ-ರಿ-ಸುತ್ತ
ವಿವ-ರಿ-ಸು-ತ್ತದೆ
ವಿವ-ರಿ-ಸು-ತ್ತಾನೆ
ವಿವ-ರಿ-ಸು-ತ್ತಾರೆ
ವಿವ-ರಿ-ಸು-ತ್ತಾಳೆ
ವಿವ-ರಿ-ಸು-ತ್ತಿದ್ದ
ವಿವ-ರಿ-ಸು-ತ್ತಿ-ದ್ದರು
ವಿವ-ರಿ-ಸು-ತ್ತಿ-ದ್ದರೆ
ವಿವ-ರಿ-ಸು-ತ್ತಿ-ದ್ದಾಗ
ವಿವ-ರಿ-ಸು-ತ್ತಿ-ದ್ದಾರೆ
ವಿವ-ರಿ-ಸು-ತ್ತಿ-ದ್ದುದು
ವಿವ-ರಿ-ಸು-ವಂತೆ
ವಿವ-ರಿ-ಸು-ವಲ್ಲಿ
ವಿವ-ರಿ-ಸು-ವಾಗ
ವಿವ-ರಿ-ಸು-ವಾ-ಗ-ಲಂತೂ
ವಿವ-ರಿ-ಸು-ವುದನ್ನು
ವಿವ-ರಿ-ಸು-ವು-ದರ
ವಿವಾ
ವಿವಾ-ಕ-ನಂ-ದರ
ವಿವಾದ
ವಿವಾ-ದ-ಕ್ಕೆ-ಳೆ-ಯಲು
ವಿವಾ-ದದ
ವಿವಾ-ದ-ವ-ನ್ನೆ-ಬ್ಬಿ-ಸಿತ್ತು
ವಿವಾ-ದಾ-ತ್ಮಕ
ವಿವಾ-ದಾ-ಸ್ಪ-ದವೂ
ವಿವಾ-ಹ-ವನ್ನು
ವಿವಾ-ಹ-ವಾ-ಗ-ಲಿ-ದ್ದರು
ವಿವಾ-ಹ-ವಾ-ಗಲು
ವಿವಾ-ಹ-ವೆಂದು
ವಿವಿ-ದಿ-ಶಾ-ನಂದ
ವಿವಿಧ
ವಿವಿ-ಧ-ತೆಯ
ವಿವಿ-ಧ-ತೆ-ಯಲ್ಲಿ
ವಿವಿ-ಧ-ತೆಯೇ
ವಿವೇ
ವಿವೇ
ವಿವೇಕ
ವಿವೇ-ಕ-ಪ್ರಜ್ಞೆ
ವಿವೇ-ಕ-ವನ್ನು
ವಿವೇ-ಕ-ವೆಂಬ
ವಿವೇಕಾ
ವಿವೇ-ಕಾಂ-ದ-ರಿಗೆ
ವಿವೇ-ಕಾಂ-ದರು
ವಿವೇಕಾನಂದ
ವಿವೇಕಾನಂದ-ನ-ದಲ್ಲ
ವಿವೇಕಾನಂದನೇ
ವಿವೇಕಾನಂದರ
ವಿವೇಕಾನಂದ-ರಂ-ತಹ
ವಿವೇಕಾನಂದ-ರಂ-ಥ-ವರು
ವಿವೇಕಾನಂದ-ರದೇ
ವಿವೇಕಾನಂದ-ರನ್ನು
ವಿವೇಕಾನಂದ-ರನ್ನೂ
ವಿವೇಕಾನಂದ-ರನ್ನೇ
ವಿವೇಕಾನಂದ-ರಲ್ಲಿ
ವಿವೇಕಾನಂದ-ರಾಗಿ
ವಿವೇಕಾನಂದ-ರಿ-ಗಿಂತ
ವಿವೇಕಾನಂದ-ರಿಗೂ
ವಿವೇಕಾನಂದ-ರಿಗೆ
ವಿವೇಕಾನಂದ-ರಿಗೇ
ವಿವೇಕಾನಂದರು
ವಿವೇಕಾನಂದ-ರೆಂಬ
ವಿವೇಕಾನಂದರೇ
ವಿವೇಕಾನಂದ-ರೊಂ-ದಿಗೆ
ವಿವೇಕಾನಂದ-ರೊಬ್ಬ
ವಿವೇ-ಕಿ-ಗಳ
ವಿವೇ-ಕಿ-ಗ-ಳಾದ
ವಿವೇ-ಕಿ-ಗಳಿಂದ
ವಿವೇ-ಕಿ-ಗಳು
ವಿವೇ-ಕಿ-ಗಳೂ
ವಿವೇಕೀ
ವಿವೇ-ಖಾ-ನಂ-ದ-ನೆಂಬ
ವಿವೇ-ಖಾ-ನಂ-ದ-ರನ್ನು
ವಿವೇ-ಚಿ-ಸು-ತ್ತಿ-ದ್ದರು
ವಿಶ-ದ-ಪ-ಡಿ-ಸಲು
ವಿಶ-ದ-ಪ-ಡಿ-ಸಿ-ದರು
ವಿಶ-ದ-ವಾಗಿ
ವಿಶ-ದೀ-ಕ-ರಿ-ಸು-ತ್ತಾರೆ
ವಿಶಾಲ
ವಿಶಾ-ಲ-ಗೊಂಡ
ವಿಶಾ-ಲ-ಗೊಂ-ಡಿದೆ
ವಿಶಾ-ಲ-ವಾ-ಗಿತ್ತು
ವಿಶಾ-ಲ-ವಾ-ಗಿದ್ದು
ವಿಶಾ-ಲ-ವಾದ
ವಿಶಾ-ಲ-ವಾ-ಯಿ-ತೆಂ-ದರೆ
ವಿಶಿಷ್ಟ
ವಿಶಿ-ಷ್ಟ-ಸುಂ-ದ-ರ-ವಾದ
ವಿಶಿ-ಷ್ಟ-ವಾಗಿ
ವಿಶಿ-ಷ್ಟ-ವಾದ
ವಿಶಿ-ಷ್ಟ-ವಾ-ದುದು
ವಿಶಿ-ಷ್ಟವೂ
ವಿಶಿ-ಷ್ಟಾ-ದ್ವೈತ
ವಿಶೇಷ
ವಿಶೇ-ಷ-ಗ-ಳೆಂ-ದರೆ
ವಿಶೇ-ಷ-ವಾಗಿ
ವಿಶೇ-ಷ-ವಾದ
ವಿಶೇ-ಷವೂ
ವಿಶೇ-ಷ-ವೆಂದರೆ
ವಿಶೇ-ಷ-ವೇನೂ
ವಿಶ್ರ-ಮಿ-ಸ-ಲಿ-ತ್ತಾ-ದ್ದ-ರಿಂದ
ವಿಶ್ರ-ಮಿ-ಸಲು
ವಿಶ್ರ-ಮಿಸಿ
ವಿಶ್ರ-ಮಿ-ಸಿ-ಕೊ-ಳ್ಳಲು
ವಿಶ್ರ-ಮಿ-ಸಿ-ಕೊ-ಳ್ಳೋ-ಣ-ವೆಂದರೆ
ವಿಶ್ರ-ಮಿ-ಸಿ-ಬೇ-ಕೆಂದು
ವಿಶ್ರ-ಮಿಸು
ವಿಶ್ರ-ಮಿ-ಸುವ
ವಿಶ್ರ-ಮಿ-ಸು-ವಂ-ತೆಯೇ
ವಿಶ್ರಾಂತಿ
ವಿಶ್ರಾಂ-ತಿ-ಗಾಗಿ
ವಿಶ್ರಾಂ-ತಿ-ಗೃ-ಹಕ್ಕೆ
ವಿಶ್ರಾಂ-ತಿ-ಗೃ-ಹ-ದಲ್ಲಿ
ವಿಶ್ರಾಂ-ತಿ-ಧಾಮ
ವಿಶ್ರಾಂ-ತಿ-ಧಾ-ಮಕ್ಕೆ
ವಿಶ್ರಾಂ-ತಿ-ಧಾ-ಮ-ಗಳು
ವಿಶ್ರಾಂ-ತಿ-ಧಾ-ಮ-ವಾದ
ವಿಶ್ರಾಂ-ತಿ-ಧಾ-ಮವು
ವಿಶ್ರಾಂ-ತಿಯ
ವಿಶ್ರಾಂ-ತಿ-ಯೆಲ್ಲಿ
ವಿಶ್ರಾಂ-ತಿಯೇ
ವಿಶ್ಲೇ-ಷಣಾ
ವಿಶ್ಲೇ-ಷ-ಣೆ-ವಿ-ವ-ರಣೆ
ವಿಶ್ಲೇ-ಷಿಸಿ
ವಿಶ್ಲೇ-ಷಿ-ಸಿ-ದ-ರ-ಲ್ಲದೆ
ವಿಶ್ಲೇ-ಷಿ-ಸಿ-ದರು
ವಿಶ್ಲೇ-ಷಿ-ಸಿ-ದ್ದರು
ವಿಶ್ಲೇ-ಷಿ-ಸು-ತ್ತಿ-ದ್ದರು
ವಿಶ್ವ
ವಿಶ್ವದ
ವಿಶ್ವ-ದಂ-ತ-ಸ್ಸತ್ವ
ವಿಶ್ವ-ದಲ್ಲಿ
ವಿಶ್ವ-ದಾ-ದ್ಯಂತ
ವಿಶ್ವ-ದೊಂ-ದಿಗೆ
ವಿಶ್ವ-ಧರ್ಮ
ವಿಶ್ವ-ಧ-ರ್ಮದ
ವಿಶ್ವ-ಧ-ರ್ಮ-ವಾ-ಗುವ
ವಿಶ್ವ-ಧ-ರ್ಮವು
ವಿಶ್ವ-ಧ-ರ್ಮ-ವೆಂದು
ವಿಶ್ವ-ಧ-ರ್ಮ-ವೊಂ-ದರ
ವಿಶ್ವ-ಧ-ರ್ಮ-ಸ-ಮ್ಮೇ-ಳನ
ವಿಶ್ವ-ಧ-ರ್ಮ-ಸ-ಮ್ಮೇ-ಳ-ನದ
ವಿಶ್ವ-ಧ-ರ್ಮ-ಸ-ಮ್ಮೇ-ಳ-ನವೇ
ವಿಶ್ವ-ನಾಥ
ವಿಶ್ವ-ನಿ-ಯ-ಮ-ಗಳ
ವಿಶ್ವ-ಪ್ರೇ-ಮವು
ವಿಶ್ವ-ಭಾ-ತೃ-ತ್ವದ
ವಿಶ್ವ-ಭ್ರಾ-ತೃತ್ವ
ವಿಶ್ವ-ಭ್ರಾ-ತೃ-ತ್ವದ
ವಿಶ್ವ-ಮಾನವ
ವಿಶ್ವ-ಮಾ-ನ-ವ-ನಾಗಿ
ವಿಶ್ವ-ವನ್ನು
ವಿಶ್ವ-ವನ್ನೂ
ವಿಶ್ವ-ವನ್ನೇ
ವಿಶ್ವ-ವಿ-ಖ್ಯಾ-ತ-ರಾ-ದರು
ವಿಶ್ವ-ವಿ-ಖ್ಯಾ-ತ-ರಾ-ದುದು
ವಿಶ್ವ-ವಿ-ಜೇತ
ವಿಶ್ವ-ವಿ-ಜೇ-ತ-ರಾಗಿ
ವಿಶ್ವ-ವಿ-ಜೇ-ತ-ರಾ-ಗು-ವುದು
ವಿಶ್ವ-ವಿ-ಜೇ-ತ-ರಾದ
ವಿಶ್ವ-ವಿ-ಜೇ-ತ-ರಾ-ದುದು
ವಿಶ್ವ-ವಿದ್ಯಾ
ವಿಶ್ವ-ವಿ-ದ್ಯಾ-ನಿ-ಲಯ
ವಿಶ್ವ-ವಿ-ದ್ಯಾ-ನಿ-ಲ-ಯಕ್ಕೆ
ವಿಶ್ವ-ವಿ-ದ್ಯಾ-ನಿ-ಲ-ಯ-ಗಳು
ವಿಶ್ವ-ವಿ-ದ್ಯಾ-ನಿ-ಲ-ಯದ
ವಿಶ್ವ-ವಿ-ದ್ಯಾ-ನಿ-ಲ-ಯ-ದಲ್ಲಿ
ವಿಶ್ವ-ವಿ-ದ್ಯಾ-ನಿ-ಲ-ಯಲ್ಲಿ
ವಿಶ್ವ-ವಿ-ದ್ಯಾ-ನಿ-ಲ-ಯ-ವನ್ನು
ವಿಶ್ವ-ವಿ-ದ್ಯಾ-ನಿ-ಲ-ಯವು
ವಿಶ್ವ-ವಿ-ದ್ಯಾ-ಲ-ಯ-ಗಳ
ವಿಶ್ವ-ವಿ-ದ್ಯಾ-ಲ-ಯ-ಗಳು
ವಿಶ್ವ-ವಿ-ದ್ಯಾ-ಲ-ಯದ
ವಿಶ್ವ-ವಿ-ದ್ಯಾ-ಲ-ಯ-ದಲ್ಲಿ
ವಿಶ್ವ-ವೇ-ದಿ-ಕೆ-ವಿ-ಜ-ಯ-ಪ-ತಾಕೆ
ವಿಶ್ವ-ವೇ-ದಿ-ಕೆಯ
ವಿಶ್ವ-ವೇ-ದಿ-ಕೆ-ಯಲ್ಲಿ
ವಿಶ್ವ-ಶಾಂ-ತಿ-ಗಾಗಿ
ವಿಶ್ವಾ-ತ್ಮಕ
ವಿಶ್ವಾ-ತ್ಮ-ಭಾವ
ವಿಶ್ವಾ-ನು-ಕಂಪ
ವಿಶ್ವಾಸ
ವಿಶ್ವಾ-ಸ-ಗಳನ್ನು
ವಿಶ್ವಾ-ಸ-ಗಳಿಂದ
ವಿಶ್ವಾ-ಸ-ಗೌ-ರ-ವ-ಗ-ಳಿಂ-ದಲೇ
ವಿಶ್ವಾ-ಸದ
ವಿಶ್ವಾ-ಸ-ದಿಂದ
ವಿಶ್ವಾ-ಸ-ದಿಂ-ದಾಗಿ
ವಿಶ್ವಾ-ಸ-ದ್ರೋಹ
ವಿಶ್ವಾ-ಸ-ಪೂ-ರ್ಣ-ವಾದ
ವಿಶ್ವಾ-ಸ-ಯುತ
ವಿಶ್ವಾ-ಸ-ಯು-ತಳು
ವಿಶ್ವಾ-ಸ-ವ-ನ್ನಿ-ರಿ-ಸಿ-ದ್ದ-ರೆಂ-ಬು-ದನ್ನು
ವಿಶ್ವಾ-ಸ-ವನ್ನು
ವಿಶ್ವಾ-ಸ-ವನ್ನೂ
ವಿಶ್ವಾ-ಸ-ವಾ-ಗಿಯೇ
ವಿಶ್ವಾ-ಸ-ವಿಟ್ಟಿ
ವಿಶ್ವಾ-ಸ-ವಿ-ಟ್ಟಿ-ರು-ವು-ದ-ಕ್ಕಾಗಿ
ವಿಶ್ವಾ-ಸ-ವಿಟ್ಟು
ವಿಶ್ವಾ-ಸ-ವಿಡಿ
ವಿಶ್ವಾ-ಸ-ವಿತ್ತು
ವಿಶ್ವಾ-ಸ-ವಿ-ತ್ತೆಂ-ದರೆ
ವಿಶ್ವಾ-ಸ-ವುಂ-ಟಾ-ಗಿ-ರ-ಲಿಲ್ಲ
ವಿಶ್ವಾ-ಸಾ-ರ್ಹ-ತೆ-ಯನ್ನೇ
ವಿಶ್ವಾ-ಸಾ-ರ್ಹರು
ವಿಶ್ವಾ-ಸಿ-ಗ-ರನ್ನು
ವಿಶ್ವಾ-ಸಿ-ಗ-ರಾಗಿ
ವಿಶ್ವಾ-ಸಿ-ಗ-ರಾದ
ವಿಶ್ವಾ-ಸಿ-ಗ-ರಾ-ದ-ವ-ರಲ್ಲಿ
ವಿಶ್ವಾ-ಸಿ-ಗಳ
ವಿಶ್ವಾ-ಸಿ-ಗ-ಳಿಗೆ
ವಿಶ್ವಾ-ಸಿ-ಗಳು
ವಿಶ್ವಾ-ಸಿ-ಗ-ಳು-ಹಿ-ತೈ-ಷಿ-ಗಳು
ವಿಶ್ವಾ-ಸಿ-ಯಾದ
ವಿಶ್ವಾಸೀ
ವಿಶ್ವೇ
ವಿಷ
ವಿಷ-ಮೃ-ತ್ಯು-ಉ-ದ್ಭ-ವಿಸಿ
ವಿಷದ
ವಿಷ-ದಿಂದ
ವಿಷ-ನರಿ
ವಿಷ-ನ-ರಿ-ಗಳ
ವಿಷ-ಮಯ
ವಿಷಯ
ವಿಷ-ಯಕ್ಕೆ
ವಿಷ-ಯ-ಕ್ಕೆಲ್ಲಾ
ವಿಷ-ಯ-ಗಳ
ವಿಷ-ಯ-ಗಳನ್ನು
ವಿಷ-ಯ-ಗಳನ್ನೂ
ವಿಷ-ಯ-ಗಳನ್ನೆಲ್ಲ
ವಿಷ-ಯ-ಗ-ಳನ್ನೇ
ವಿಷ-ಯ-ಗ-ಳ-ಲ್ಲಾ-ದರೂ
ವಿಷ-ಯ-ಗಳಲ್ಲಿ
ವಿಷ-ಯ-ಗ-ಳ-ಲ್ಲೆಲ್ಲ
ವಿಷ-ಯ-ಗ-ಳಾ-ಗಿ-ದ್ದುವು
ವಿಷ-ಯ-ಗಳಿಂದ
ವಿಷ-ಯ-ಗ-ಳಿ-ಗಷ್ಟೇ
ವಿಷ-ಯ-ಗ-ಳಿಗೆ
ವಿಷ-ಯ-ಗ-ಳಿ-ರುತ್ತಿ
ವಿಷ-ಯ-ಗಳು
ವಿಷ-ಯ-ಗ-ಳು-ಹಿಂದೂ
ವಿಷ-ಯ-ಗಳೂ
ವಿಷ-ಯ-ಗ-ಳೆ-ಲ್ಲ-ದರ
ವಿಷ-ಯ-ಗ-ಳೆ-ಲ್ಲವೂ
ವಿಷ-ಯ-ಗ-ಳೊಂದೂ
ವಿಷ-ಯದ
ವಿಷ-ಯ-ದತ್ತ
ವಿಷ-ಯ-ದಲ್ಲಿ
ವಿಷ-ಯ-ದಲ್ಲೂ
ವಿಷ-ಯ-ದಲ್ಲೇ
ವಿಷ-ಯ-ಪ-ಶ್ಚಿ-ಮಕ್ಕೆ
ವಿಷ-ಯ-ಮೇ-ಲ-ಧಿ-ಕಾ-ರಿ-ಗಳು
ವಿಷ-ಯಲ್ಲಿ
ವಿಷ-ಯ-ವನ್ನು
ವಿಷ-ಯ-ವನ್ನೂ
ವಿಷ-ಯ-ವ-ನ್ನೊ-ದ-ಗಿ-ಸು-ತ್ತಿತ್ತು
ವಿಷ-ಯ-ವ-ಲ್ಲವೆ
ವಿಷ-ಯ-ವಾ-ಗಲಿ
ವಿಷ-ಯ-ವಾಗಿ
ವಿಷ-ಯ-ವಾ-ಗಿ-ತಾ-ವಿ-ಬ್ಬರೂ
ವಿಷ-ಯ-ವಾ-ಗಿತ್ತು
ವಿಷ-ಯ-ವಾ-ಗಿಯೂ
ವಿಷ-ಯ-ವಾ-ಗೆಲ್ಲ
ವಿಷ-ಯವು
ವಿಷ-ಯವೂ
ವಿಷ-ಯ-ವೆಂದರೆ
ವಿಷ-ಯ-ವೆಂ-ದಾ-ಗಲಿ
ವಿಷ-ಯ-ವೆಂ-ಬಂತೆ
ವಿಷ-ಯ-ವೆಲ್ಲ
ವಿಷ-ಯವೇ
ವಿಷ-ಯ-ವೇನೆಂದರೆ
ವಿಷ-ಯ-ವೇ-ನೆಂದು
ವಿಷ-ಯ-ವೊಂ-ದನ್ನು
ವಿಷ-ಯವೋ
ವಿಷ-ಯ-ಹಿಂ-ದೂ-ಗಳ
ವಿಷ-ಯಾಗಿ
ವಿಷ-ವನ್ನು
ವಿಷ-ವಿಕ್ಕಿ
ವಿಷ-ವಿ-ರು-ತ್ತಿ-ರ-ಲಿಲ್ಲ
ವಿಷವು
ವಿಷಾದ
ವಿಷಾ-ದ-ದಿಂದ
ವಿಷಾ-ದ-ಪ-ತ್ರ-ದಲ್ಲಿ
ವಿಷಾ-ದ-ವ-ನ್ನುಂ-ಟು-ಮಾ-ಡಿದೆ
ವಿಷಾ-ದ-ವಾ-ಯಿತು
ವಿಷಾ-ದಿ-ಸು-ತ್ತಾಳೆ
ವಿಷ್ಣು
ವಿಷ್ಣು-ಪು-ರಾ-ಣ-ದಲ್ಲಿ
ವಿಷ್ಣು-ವಿನ
ವಿಸ್ಕಾ-ನ್ಸಿನ್
ವಿಸ್ತ-ರಿ-ಸ-ಬೇ-ಕಾ-ಯಿತು
ವಿಸ್ತ-ರಿಸಿ
ವಿಸ್ತ-ರಿ-ಸಿ-ಕೊ-ಳ್ಳುವ
ವಿಸ್ತ-ರಿ-ಸಿ-ದರು
ವಿಸ್ತಾರ
ವಿಸ್ತಾ-ರ-ಗೊ-ಳ್ಳುತ್ತ
ವಿಸ್ತಾ-ರ-ಗೊ-ಳ್ಳು-ತ್ತಿದ್ದು
ವಿಸ್ತಾ-ರ-ವನ್ನು
ವಿಸ್ತಾ-ರ-ವಾಗಿ
ವಿಸ್ತಾ-ರ-ವಾ-ಗಿ-ದ್ದರೂ
ವಿಸ್ತಾ-ರ-ವಾ-ಗಿದ್ದು
ವಿಸ್ತೃತ
ವಿಸ್ತೃ-ತ-ವಾಗಿ
ವಿಸ್ಮಯ
ವಿಸ್ಮ-ಯ-ಮೂ-ಕ-ನಾಗಿ
ವಿಸ್ಮ-ಯ-ಮೂ-ಕ-ನಾ-ಗಿ-ದ್ದೇನೆ
ವಿಸ್ಮ-ಯ-ಮೂ-ಕ-ನಾದ
ವಿಸ್ಮ-ಯ-ಮೂ-ಕ-ರಾ-ಗು-ತ್ತಿ-ದ್ದರು
ವಿಸ್ಮ-ಯ-ಮೂ-ಕ-ರಾ-ದರು
ವಿಸ್ಮ-ಯ-ಮೂ-ಕ-ವಾಗಿ
ವಿಸ್ಮ-ಯ-ವ-ನ್ನುಂಟು
ವಿಸ್ಮ-ಯ-ವನ್ನೋ
ವಿಸ್ಮ-ಯಾ-ನಂದ
ವಿಸ್ಮ-ಯಾ-ನಂ-ದ-ಗೊಂಡ
ವಿಸ್ಮ-ಯಾ-ನಂ-ದಿತ
ವಿಸ್ಮಿ-ತ-ರಾ-ದರು
ವಿಹಂ-ಗಮ
ವಿಹ-ರಿ-ಸಲು
ವಿಹ-ರಿ-ಸಿ-ದಂ-ತಾ-ಯಿ-ತಲ್ಲ
ವಿಹ-ರಿ-ಸಿ-ದ್ದಾರೆ
ವಿಹ-ರಿ-ಸು-ತ್ತಿ-ರು-ವಂ-ತಿದೆ
ವಿಹ-ರಿ-ಸು-ವುದು
ವಿಹಾರ
ವಿಹಾ-ರ-ಗಳು
ವಿಹಾ-ರ-ಧಾ-ಮ-ಗ-ಳಿಗೆ
ವಿಹಾ-ರಿ-ದಾಸ್
ವೀಕ್ಷ-ಕ-ನಿಗೂ
ವೀಕ್ಷಿಸಿ
ವೀಕ್ಷಿ-ಸಿತು
ವೀಕ್ಷಿ-ಸಿ-ದಂತೆ
ವೀಕ್ಷಿ-ಸಿ-ದರು
ವೀಕ್ಷಿಸು
ವೀಕ್ಷಿ-ಸುತ್ತ
ವೀಕ್ಷಿ-ಸು-ತ್ತಿ-ದ್ದಂತೆ
ವೀರ
ವೀರ-ಕೇ-ಸ-ರಿ-ಗಳೇ
ವೀರ-ಚಂದ್
ವೀರ-ನನ್ನು
ವೀರ-ನಾ-ಡಿ-ಯನ್ನು
ವೀರ-ವಾ-ಣಿ-ಯಲ್ಲಿ
ವೀರ-ಸಂನ್ಯಾಸಿ
ವೀರ-ಸಂ-ನ್ಯಾ-ಸಿ-ಯ-ಲ್ಲದೆ
ವೀರ-ಸಂ-ನ್ಯಾ-ಸಿ-ಯಾಗಿ
ವೀರ-ಸಂ-ನ್ಯಾ-ಸಿ-ಯೆಂದ
ವೀರ-ಸಂ-ನ್ಯಾ-ಸಿ-ಯೊ-ಬ್ಬನ
ವೀರಾ-ಧಿ-ವೀ-ರ-ನಾದ
ವೀರ್ಯ-ವಂ-ತನೂ
ವೀಲರ್
ವೀವೇ-ಕಾ-ನಂ-ದರು
ವುಡ್ಸ್
ವುಡ್ಸ್ಳ
ವುದಂತೂ
ವುದ-ಕ್ಕಲ್ಲ
ವುದ-ಕ್ಕಾಗಿ
ವುದ-ಕ್ಕಾ-ಗಿಯೇ
ವುದಕ್ಕೆ
ವುದನ್ನು
ವುದನ್ನೇ
ವುದರ
ವುದ-ರಿಂದ
ವುದ-ರೆಂ-ದ-ರೇನು
ವುದಾಗಿ
ವುದಾ-ದರೆ
ವುದಿಲ್ಲ
ವುದಿ-ಲ್ಲ-ವಲ್ಲ
ವುದು
ವುದು-ಇ-ವು-ಗ-ಳಿಗೆ
ವುದೂ
ವುದೆಂ-ಬು-ದನ್ನು
ವುದೆಲ್ಲ
ವುದೆ-ಲ್ಲ-ವನ್ನು
ವುದೇ
ವುದೇ-ನಿ-ದ್ದರೂ
ವೃಕ್ಷ-ಗಳು
ವೃಕ್ಷ-ವೊಂ-ದರ
ವೃತ್ತ-ಪ-ತ್ರಿಕೆ
ವೃತ್ತ-ಪ-ತ್ರಿ-ಕೆ-ಗಳ
ವೃತ್ತ-ಪ-ತ್ರಿ-ಕೆ-ಗಳನ್ನು
ವೃತ್ತ-ಪ-ತ್ರಿ-ಕೆ-ಗಳಲ್ಲಿ
ವೃತ್ತ-ಪ-ತ್ರಿ-ಕೆ-ಗ-ಳ-ಲ್ಲೆಲ್ಲ
ವೃತ್ತ-ಪ-ತ್ರಿ-ಕೆ-ಗ-ಳಿಗೆ
ವೃತ್ತ-ಪ-ತ್ರಿ-ಕೆ-ಗಳು
ವೃತ್ತ-ಪ-ತ್ರಿ-ಕೆಯ
ವೃತ್ತ-ಪ-ತ್ರಿ-ಕೆ-ಯನ್ನು
ವೃತ್ತ-ಪ-ತ್ರಿ-ಕೆ-ಯೊಂದು
ವೃತ್ತಿ
ವೃತ್ತಿ-ಗಳ
ವೃತ್ತಿ-ನಿ-ರತ
ವೃತ್ತಿ-ಯನ್ನು
ವೃತ್ತಿ-ಯನ್ನೇ
ವೃತ್ತಿ-ಯಿಂದ
ವೃಥಾ
ವೃದ್ಧ
ವೃದ್ಧನ
ವೃದ್ಧನೂ
ವೃದ್ಧಾ-ಪ್ಯ-ದಲ್ಲಿ
ವೃದ್ಧಿ-ಗೊ-ಳಿ-ಸಿ-ಕೊ-ಳ್ಳುವ
ವೃದ್ಧಿ-ಗೊ-ಳ್ಳು-ತ್ತಿದೆ
ವೃದ್ಧಿ-ಯಾ-ಗಲು
ವೃದ್ಧಿ-ಯಾ-ಗು-ತ್ತದೆ
ವೃದ್ಧಿ-ಯಾ-ಯಿತು
ವೆಂಕ-ಟ-ರಂ-ಗ-ರಾವ್
ವೆಂಥದು
ವೆಂದರೂ
ವೆಂದರೆ
ವೆಂದ-ರೇ-ನೆಂದೇ
ವೆಂದಲ್ಲ
ವೆಂದಿಗೂ
ವೆಂದು
ವೆಂದೇ
ವೆಚ್ಚ
ವೆಚ್ಚಕ್ಕೂ
ವೆಚ್ಚ-ವನ್ನು
ವೆನ್ನಿ-ಸ-ಬ-ಹುದು
ವೆಬ್ಸ್ಟರ್ನ
ವೆರಾ
ವೆರಾ-ವಲ್
ವೆರಾ-ವೆಲ್
ವೆಲ್ಲ
ವೆಲ್ಲಿ-ಯದು
ವೆಲ್ಲೆಸ್ಲೀ
ವೆಸ್ಟ್
ವೆಸ್ಟ್ಮಿ-ನಿ-ಸ್ಟರ್ನ
ವೇಕೆ
ವೇಗ-ದಿಂದ
ವೇಗ-ವಾಗಿ
ವೇದ
ವೇದ-ವೇ-ದಾಂ-ತ-ಪು-ರಾ-ಣ-ಗಳ
ವೇದ-ಕಾ-ಲದ
ವೇದ-ಗಳ
ವೇದ-ಗಳನ್ನು
ವೇದ-ಗಳಲ್ಲಿ
ವೇದ-ಗ-ಳ-ಲ್ಲಿನ
ವೇದ-ಗಳಿಂದ
ವೇದ-ಗಳು
ವೇದ-ಗ-ಳೆಂಬ
ವೇದ-ಧ-ರ್ಮಕ್ಕೂ
ವೇದ-ಧ-ರ್ಮಕ್ಕೆ
ವೇದ-ಧ-ರ್ಮದ
ವೇದ-ಧ-ರ್ಮ-ವನ್ನು
ವೇದ-ವನ್ನು
ವೇದವು
ವೇದ-ವೇ-ದಾಂ-ತ-ಗಳ
ವೇದ-ವೇ-ದಾಂ-ತ-ಗ-ಳೆ-ಲ್ಲ-ದ-ಕ್ಕಿಂತ
ವೇದ-ಶಾ-ಸ್ತ್ರ-ಗಳಿಂದ
ವೇದ-ಶಾ-ಸ್ತ್ರ-ಗ-ಳೆಲ್ಲ
ವೇದಾಂತ
ವೇದಾಂ-ತ-ಅ-ದರ
ವೇದಾಂ-ತ-ಕ-ಡಿ-ಯ-ಲಾ-ಗದ
ವೇದಾಂ-ತ-ಕೇಂ-ದ್ರದ
ವೇದಾಂ-ತ-ತತ್ತ್ವ
ವೇದಾಂ-ತ-ತ-ತ್ತ್ವ-ಗಳನ್ನು
ವೇದಾಂ-ತ-ತ-ತ್ತ್ವ-ಗಳಲ್ಲಿ
ವೇದಾಂ-ತ-ತ್ತ್ವ-ಗಳನ್ನು
ವೇದಾಂ-ತದ
ವೇದಾಂ-ತ-ದತ್ತ
ವೇದಾಂ-ತ-ದ-ಲ್ಲ-ಡ-ಗಿ-ರುವ
ವೇದಾಂ-ತ-ದಲ್ಲಿ
ವೇದಾಂ-ತ-ಧ-ರ್ಮದ
ವೇದಾಂ-ತ-ಪ್ರ-ಸಾರ
ವೇದಾಂ-ತ-ವನ್ನು
ವೇದಾಂ-ತ-ವಾ-ಣಿಯ
ವೇದಾಂ-ತವು
ವೇದಾಂ-ತ-ವೆಂ-ಬುದು
ವೇದಾಂ-ತವೇ
ವೇದಾಂತಿ
ವೇದಾಂ-ತಿ-ಕು-ಲದ
ವೇದಾಂ-ತಿ-ಗಳಲ್ಲಿ
ವೇದಾಂ-ತಿ-ಗ-ಳಾದ
ವೇದಾಂ-ತಿ-ಗಳು
ವೇದಾಂ-ತಿಯೇ
ವೇದಾ-ಭ್ಯಾಸ
ವೇದಾ-ಹ-ಮೇತಂ
ವೇದಿಕೆ
ವೇದಿ-ಕೆ-ಗಳ
ವೇದಿ-ಕೆಯ
ವೇದಿ-ಕೆ-ಯನ್ನು
ವೇದಿ-ಕೆ-ಯ-ನ್ನೇರಿ
ವೇದಿ-ಕೆ-ಯ-ನ್ನೇ-ರಿ-ದರು
ವೇದಿ-ಕೆ-ಯ-ನ್ನೇರು
ವೇದಿ-ಕೆಯಿ
ವೇದೋ-ಕ್ತಿ-ಯನ್ನು
ವೇದೋ-ತ್ತರ
ವೇದೋ-ಪ-ನಿ-ಷ-ತ್ತು-ಗಳಿಂದ
ವೇನಾ-ಗಿ-ತ್ತೆಂ-ದರೆ
ವೇನೆಂ-ದರೆ
ವೇನೆಂ-ಬುದು
ವೇನೋ
ವೇರ್ಪ-ಡು-ತ್ತಿ-ದ್ದುದು
ವೇಳಾ-ಯಾಂ
ವೇಳೆ
ವೇಳೆಗ
ವೇಳೆ-ಗಳಲ್ಲಿ
ವೇಳೆ-ಗ-ಳಲ್ಲೂ
ವೇಳೆ-ಗಾ-ಗಲೇ
ವೇಳೆಗೆ
ವೇಳೆ-ಯನ್ನು
ವೇಳೆ-ಯಲ್ಲಿ
ವೇಳೆ-ಯಲ್ಲೂ
ವೇಳೇ-ಗಾ-ಗಲೇ
ವೇಶ್ಯಾ
ವೇಶ್ಯೆ
ವೇಶ್ಯೆಯ
ವೇಶ್ಯೆ-ಯರ
ವೇಶ್ಯೆ-ಯರು
ವೇಶ್ಯೆ-ಯ-ರೊಂ-ದಿಗೆ
ವೇಷ
ವೇಷ-ಭೂ-ಷ-ಣ-ಗಳಲ್ಲಿ
ವೇಷ-ಭೂ-ಷ-ಣ-ಗಳು
ವೇಷ-ಭೂ-ಷ-ಣ-ಧಾ-ರಿ-ಗ-ಳಾದ
ವೇಷ-ಭೂ-ಷ-ಣ-ಧಾ-ರಿ-ಯಾದ
ವೇಷ-ವನ್ನು
ವೈಖ-ರಿ-ಯನ್ನು
ವೈಖ-ರಿಯು
ವೈಚಾ-ರಿಕ
ವೈಚಾ-ರಿ-ಕವೂ
ವೈಚಿತ್ರ್ಯ
ವೈಚಿ-ತ್ರ್ಯ-ವನ್ನು
ವೈಚಿ-ತ್ರ್ಯ-ವನ್ನೂ
ವೈಚಿ-ತ್ರ್ಯಾತ್
ವೈಜ್ಞಾ-ನಿಕ
ವೈಜ್ಞಾ-ನಿ-ಕ-ವಾಗಿ
ವೈಜ್ಞಾ-ನಿ-ಕವೂ
ವೈಟ್
ವೈದಿಕ
ವೈದಿ-ಕ-ರಾದ
ವೈದಿ-ಕ-ವಾ-ಗಲಿ
ವೈದಿ-ಕ-ವಿ-ದ್ವಾಂ-ಸ-ನೆಂದೂ
ವೈದ್ಯ
ವೈದ್ಯ-ಕೀಯ
ವೈದ್ಯರು
ವೈದ್ಯಾ-ಧಿ-ಕಾ-ರಿ-ಯಾ-ಗಿದ್ದ
ವೈಪ-ರೀ-ತ್ಯ-ಗಳನ್ನು
ವೈಭವ
ವೈಭ-ವ-ಆ-ಡಂ-ಬ-ರ-ಗಳು
ವೈಭ-ವ-ಮೌ-ಲ್ಯ-ಗ-ಳೊಂ-ದಿಗೆ
ವೈಭ-ವ-ಗಳನ್ನು
ವೈಭ-ವದ
ವೈಭ-ವ-ದಿಂದ
ವೈಭ-ವ-ದಿಂ-ದಲೂ
ವೈಭ-ವ-ಪೂರ್ಣ
ವೈಭ-ವ-ಪೂ-ರ್ಣ-ವಾ-ಗಿತ್ತು
ವೈಭ-ವ-ಯುತ
ವೈಭ-ವ-ಯು-ತ-ವಾಗಿ
ವೈಭ-ವ-ಯು-ತವೂ
ವೈಭ-ವ-ವನ್ನು
ವೈಭವೋ
ವೈಭ-ವೋ-ಪೇತ
ವೈಭ-ವೋ-ಪೇ-ತ-ವಾದ
ವೈಭೋಗ
ವೈಮ-ನ-ಸ್ಯ-ವಲ್ಲ
ವೈಮನ್
ವೈಯ-ಕ್ತಿಕ
ವೈಯ-ಕ್ತಿ-ಕತೆ
ವೈಯ-ಕ್ತಿ-ಕ-ತೆ-ಯನ್ನು
ವೈಯ-ಕ್ತಿ-ಕ-ವಾಗಿ
ವೈಯಾ-ಕ-ರ-ಣಿ-ಯೆಂದು
ವೈರಾಗ್ಯ
ವೈರಾ-ಗ್ಯಕ್ಕೂ
ವೈರಾ-ಗ್ಯದ
ವೈರಾ-ಗ್ಯ-ಬು-ದ್ಧಿ-ಯನ್ನು
ವೈರಿ
ವೈರಿ-ಗಳ
ವೈರಿ-ಗಳನ್ನು
ವೈರಿ-ಗಳು
ವೈರಿ-ಯಾಗಿ
ವೈವಾ-ಹಿಕ
ವೈವಿ-ಧ್ಯ-ಗಳು
ವೈವಿ-ಧ್ಯದ
ವೈವಿ-ಧ್ಯ-ಮಯ
ವೈವಿ-ಧ್ಯ-ಮ-ಯ-ವಾ-ದುದು
ವೈವಿ-ಧ್ಯ-ವನ್ನು
ವೈಶಾಲ್ಯ
ವೈಶಾ-ಲ್ಯ-ವೆಷ್ಟು
ವೈಶಿಷ್ಟ್ಯ
ವೈಶಿ-ಷ್ಟ್ಯ-ಗಳನ್ನು
ವೈಶಿ-ಷ್ಟ್ಯ-ಪೂರ್ಣ
ವೈಶಿ-ಷ್ಟ್ಯ-ಪೂ-ರ್ಣ-ವಾ-ಗಿತ್ತು
ವೈಶಿ-ಷ್ಟ್ಯ-ವಿದೆ
ವೈಷ್ಣ
ವೈಷ್ಣ-ಮತಿ
ವೈಷ್ಣವ
ವೈಷ್ಣ-ವರು
ವೊಂದನ್ನು
ವೊಂದರ
ವೊಂದ-ರಲ್ಲಿ
ವೊಂದಿದೆ
ವೊಂದು
ವೊಮ್ಮೆ
ವ್ಯಂಗ್ಯ
ವ್ಯಂಗ್ಯದ
ವ್ಯಂಗ್ಯ-ದಿಂದ
ವ್ಯಂಗ್ಯ-ವಾಗಿ
ವ್ಯಕ್ತ
ವ್ಯಕ್ತ-ಗೊಂ-ಡಿತು
ವ್ಯಕ್ತ-ಗೊ-ಳ್ಳು-ವುದನ್ನು
ವ್ಯಕ್ತ-ಜ-ಗ-ತ್ತಿ-ನಲ್ಲಿ
ವ್ಯಕ್ತ-ಪ-ಡಿ-ಸ-ದಿ-ದ್ದಲ್ಲಿ
ವ್ಯಕ್ತ-ಪ-ಡಿ-ಸ-ಬ-ಹುದು
ವ್ಯಕ್ತ-ಪ-ಡಿ-ಸ-ಬೇಕು
ವ್ಯಕ್ತ-ಪ-ಡಿ-ಸ-ಲಿ-ಅ-ದಕ್ಕೆ
ವ್ಯಕ್ತ-ಪ-ಡಿಸಿ
ವ್ಯಕ್ತ-ಪ-ಡಿ-ಸಿದ
ವ್ಯಕ್ತ-ಪ-ಡಿ-ಸಿ-ದಂ-ತಹ
ವ್ಯಕ್ತ-ಪ-ಡಿ-ಸಿ-ದ-ರ-ಲ್ಲದೆ
ವ್ಯಕ್ತ-ಪ-ಡಿ-ಸಿ-ದರು
ವ್ಯಕ್ತ-ಪ-ಡಿ-ಸಿ-ದ-ರು-ಇಂ-ಗ್ಲಿ-ಷರ
ವ್ಯಕ್ತ-ಪ-ಡಿ-ಸಿ-ದಾಗ
ವ್ಯಕ್ತ-ಪ-ಡಿ-ಸಿ-ದು-ದನ್ನು
ವ್ಯಕ್ತ-ಪ-ಡಿ-ಸಿದ್ದ
ವ್ಯಕ್ತ-ಪ-ಡಿ-ಸಿ-ದ್ದರು
ವ್ಯಕ್ತ-ಪ-ಡಿ-ಸಿ-ದ್ದಳು
ವ್ಯಕ್ತ-ಪ-ಡಿ-ಸಿ-ದ್ದಾರೆ
ವ್ಯಕ್ತ-ಪ-ಡಿ-ಸಿದ್ದು
ವ್ಯಕ್ತ-ಪ-ಡಿ-ಸಿ-ದ್ದುಂಟು
ವ್ಯಕ್ತ-ಪ-ಡಿ-ಸಿಯೂ
ವ್ಯಕ್ತ-ಪ-ಡಿ-ಸುತ್ತ
ವ್ಯಕ್ತ-ಪ-ಡಿ-ಸು-ತ್ತವೆ
ವ್ಯಕ್ತ-ಪ-ಡಿ-ಸು-ತ್ತಾರೆ
ವ್ಯಕ್ತ-ಪ-ಡಿ-ಸು-ತ್ತಿತ್ತು
ವ್ಯಕ್ತ-ಪ-ಡಿ-ಸು-ತ್ತಿ-ದ್ದರು
ವ್ಯಕ್ತ-ಪ-ಡಿ-ಸು-ವಂ-ತಹ
ವ್ಯಕ್ತ-ವಾ-ಗ-ಬ-ಹು-ದಾ-ಗಿತ್ತು
ವ್ಯಕ್ತ-ವಾ-ಗ-ಬೇ-ಕೆಂದು
ವ್ಯಕ್ತ-ವಾ-ಗಿತ್ತು
ವ್ಯಕ್ತ-ವಾ-ಗಿದೆ
ವ್ಯಕ್ತ-ವಾ-ಗಿದ್ದ
ವ್ಯಕ್ತ-ವಾ-ಗಿವೆ
ವ್ಯಕ್ತ-ವಾ-ಗು-ತ್ತದೆ
ವ್ಯಕ್ತ-ವಾ-ಗು-ತ್ತವೆ
ವ್ಯಕ್ತ-ವಾ-ಗು-ತ್ತಿತ್ತು
ವ್ಯಕ್ತ-ವಾ-ಗು-ತ್ತಿದೆ
ವ್ಯಕ್ತ-ವಾ-ಗು-ತ್ತಿದ್ದ
ವ್ಯಕ್ತ-ವಾ-ಗು-ವಂತೆ
ವ್ಯಕ್ತ-ವಾ-ಗು-ವು-ದನ್ನೂ
ವ್ಯಕ್ತ-ವಾ-ಗು-ವುದು
ವ್ಯಕ್ತ-ವಾದ
ವ್ಯಕ್ತ-ವಾ-ದಂ-ತಹ
ವ್ಯಕ್ತ-ವಾ-ದ-ದ್ದನ್ನು
ವ್ಯಕ್ತ-ವಾ-ದುವು
ವ್ಯಕ್ತ-ವಾ-ಯಿತು
ವ್ಯಕ್ತಿ
ವ್ಯಕ್ತಿಈ
ವ್ಯಕ್ತಿ-ಒಬ್ಬ
ವ್ಯಕ್ತಿ-ಗಳ
ವ್ಯಕ್ತಿ-ಗ-ಳ-ನ್ನಾ-ಗಿಸಿ
ವ್ಯಕ್ತಿ-ಗಳನ್ನು
ವ್ಯಕ್ತಿ-ಗಳನ್ನೂ
ವ್ಯಕ್ತಿ-ಗಳಲ್ಲಿ
ವ್ಯಕ್ತಿ-ಗ-ಳಾ-ಗಿ-ರ-ಬೇಕು
ವ್ಯಕ್ತಿ-ಗ-ಳಾ-ಗು-ತ್ತಾರೆ
ವ್ಯಕ್ತಿ-ಗಳಿಂದ
ವ್ಯಕ್ತಿ-ಗ-ಳಿಗೆ
ವ್ಯಕ್ತಿ-ಗ-ಳಿ-ದ್ದರು
ವ್ಯಕ್ತಿ-ಗ-ಳಿ-ದ್ದಾರೆ
ವ್ಯಕ್ತಿ-ಗಳು
ವ್ಯಕ್ತಿ-ಗಳೂ
ವ್ಯಕ್ತಿ-ಗ-ಳೆಲ್ಲ
ವ್ಯಕ್ತಿ-ಗ-ಳೊಂ-ದಿಗೆ
ವ್ಯಕ್ತಿ-ಗಾಗಿ
ವ್ಯಕ್ತಿ-ಗಿಂತ
ವ್ಯಕ್ತಿಗೆ
ವ್ಯಕ್ತಿ-ಚಿ-ತ್ರ-ಣ-ವನ್ನು
ವ್ಯಕ್ತಿತ್ವ
ವ್ಯಕ್ತಿ-ತ್ವ-ಜೀ-ವ-ನ-ಗಳನ್ನು
ವ್ಯಕ್ತಿ-ತ್ವ-ವಿ-ಚಾ-ರ-ಗಳ
ವ್ಯಕ್ತಿ-ತ್ವ-ಇವು
ವ್ಯಕ್ತಿ-ತ್ವಈ
ವ್ಯಕ್ತಿ-ತ್ವ-ಕ್ಕಿಂ-ತಲೂ
ವ್ಯಕ್ತಿ-ತ್ವಕ್ಕೆ
ವ್ಯಕ್ತಿ-ತ್ವದ
ವ್ಯಕ್ತಿ-ತ್ವ-ದಲ್ಲಿ
ವ್ಯಕ್ತಿ-ತ್ವ-ದ-ಲ್ಲೊಂದು
ವ್ಯಕ್ತಿ-ತ್ವ-ದಾ-ಳ-ದಿಂದ
ವ್ಯಕ್ತಿ-ತ್ವ-ದಿಂದ
ವ್ಯಕ್ತಿ-ತ್ವ-ದಿಂ-ದಲೂ
ವ್ಯಕ್ತಿ-ತ್ವ-ದಿಂ-ದಾಗಿ
ವ್ಯಕ್ತಿ-ತ್ವ-ದಿಂ-ದೊ-ಡ-ಗೂ-ಡಿ-ದಾಗ
ವ್ಯಕ್ತಿ-ತ್ವ-ವನ್ನು
ವ್ಯಕ್ತಿ-ತ್ವ-ವನ್ನೂ
ವ್ಯಕ್ತಿ-ತ್ವ-ವನ್ನೇ
ವ್ಯಕ್ತಿ-ತ್ವವು
ವ್ಯಕ್ತಿ-ತ್ವವೂ
ವ್ಯಕ್ತಿ-ತ್ವ-ವೆಂ-ಬುದು
ವ್ಯಕ್ತಿ-ತ್ವವೇ
ವ್ಯಕ್ತಿ-ತ್ವವೋ
ವ್ಯಕ್ತಿ-ನಿ-ರ್ಮಾಣ
ವ್ಯಕ್ತಿ-ಮಿಸ್
ವ್ಯಕ್ತಿಯ
ವ್ಯಕ್ತಿ-ಯಂತೆ
ವ್ಯಕ್ತಿ-ಯ-ನ್ನಲ್ಲ
ವ್ಯಕ್ತಿ-ಯನ್ನು
ವ್ಯಕ್ತಿ-ಯನ್ನೂ
ವ್ಯಕ್ತಿ-ಯನ್ನೋ
ವ್ಯಕ್ತಿ-ಯಲ್ಲ
ವ್ಯಕ್ತಿ-ಯ-ಲ್ಲವೆ
ವ್ಯಕ್ತಿ-ಯ-ಲ್ಲೂ-ಭಗ
ವ್ಯಕ್ತಿ-ಯಾಗಿ
ವ್ಯಕ್ತಿ-ಯಾ-ಗಿದ್ದ
ವ್ಯಕ್ತಿ-ಯಾ-ಗಿ-ದ್ದರು
ವ್ಯಕ್ತಿ-ಯಾ-ಗಿ-ಬಿ-ಡು-ತ್ತಿ-ದ್ದರು
ವ್ಯಕ್ತಿ-ಯಾ-ಗಿಯೇ
ವ್ಯಕ್ತಿ-ಯಾ-ಗಿ-ರ-ಲಿಲ್ಲ
ವ್ಯಕ್ತಿ-ಯಾದ
ವ್ಯಕ್ತಿ-ಯಾ-ದ-ವನು
ವ್ಯಕ್ತಿ-ಯಿ-ರು-ವ-ಲ್ಲಿಗೆ
ವ್ಯಕ್ತಿಯೂ
ವ್ಯಕ್ತಿ-ಯೆಂ-ದರೆ
ವ್ಯಕ್ತಿ-ಯೆಂದು
ವ್ಯಕ್ತಿ-ಯೆಂದೋ
ವ್ಯಕ್ತಿ-ಯೆಂ-ಬು-ದರ
ವ್ಯಕ್ತಿ-ಯೆಂ-ಬುದು
ವ್ಯಕ್ತಿ-ಯೆ-ನ್ನು-ವುದು
ವ್ಯಕ್ತಿಯೇ
ವ್ಯಕ್ತಿ-ಯೊಂ-ದಿಗೆ
ವ್ಯಕ್ತಿ-ಯೊಬ್ಬ
ವ್ಯಕ್ತಿ-ಯೊ-ಬ್ಬ-ನನ್ನು
ವ್ಯಕ್ತಿ-ಯೊ-ಬ್ಬನು
ವ್ಯಕ್ತಿ-ಯೊ-ಬ್ಬರು
ವ್ಯಕ್ತಿ-ಯೊ-ಳ-ಗಿನ
ವ್ಯತ್ಯಾಸ
ವ್ಯತ್ಯಾ-ಸಈ
ವ್ಯತ್ಯಾ-ಸ-ವನ್ನೂ
ವ್ಯಥೆ-ಯಿಂದ
ವ್ಯಥೆ-ಯಿ-ತ್ತು-ತನ್ನ
ವ್ಯಭಿ-ಚಾ-ರ-ದ-ವ-ರೆಗೆ
ವ್ಯಭಿ-ಚಾ-ರಿ-ಣಿ-ಯೆಂದು
ವ್ಯರ್ಥ
ವ್ಯರ್ಥ-ವಾ-ಗ-ದಂತೆ
ವ್ಯರ್ಥ-ವಾ-ಗ-ಬಾ-ರ-ದೆಂಬ
ವ್ಯರ್ಥ-ವಾ-ಗ-ಲಿಲ್ಲ
ವ್ಯರ್ಥ-ವಾಗಿ
ವ್ಯರ್ಥ-ವಾ-ಗಿ-ಲ್ಲ-ವೆಂಬ
ವ್ಯರ್ಥವೇ
ವ್ಯವ-ಧಾ-ನ-ವಿಲ್ಲ
ವ್ಯವ-ಧಾ-ನ-ವಿ-ಲ್ಲ-ದಿದ್ದ
ವ್ಯವ-ಸಾಯ
ವ್ಯವ-ಸಾ-ಯಕ್ಕೆ
ವ್ಯವ-ಸಾ-ಯ-ದಲ್ಲಿ
ವ್ಯವ-ಸಾ-ಯ-ವನ್ನು
ವ್ಯವ-ಸ್ಥಾ-ಪಕ
ವ್ಯವ-ಸ್ಥಾ-ಪ-ಕರು
ವ್ಯವ-ಸ್ಥಿತ
ವ್ಯವ-ಸ್ಥಿ-ತ-ಗೊ-ಳಿಸಿ
ವ್ಯವ-ಸ್ಥಿ-ತ-ಗೊ-ಳಿ-ಸಿ-ಕೊ-ಳ್ಳಲು
ವ್ಯವ-ಸ್ಥಿ-ತ-ವಾಗಿ
ವ್ಯವಸ್ಥೆ
ವ್ಯವ-ಸ್ಥೆ-ಗಳನ್ನು
ವ್ಯವ-ಸ್ಥೆ-ಗಳನ್ನೂ
ವ್ಯವ-ಸ್ಥೆಗೆ
ವ್ಯವ-ಸ್ಥೆ-ಗೊ-ಳಿ-ಸ-ಲಾ-ಯಿತು
ವ್ಯವ-ಸ್ಥೆ-ಗೊ-ಳಿ-ಸು-ತ್ತೇನೆ
ವ್ಯವ-ಸ್ಥೆ-ಗೊ-ಳಿ-ಸುವ
ವ್ಯವ-ಸ್ಥೆ-ಯನ್ನು
ವ್ಯವ-ಸ್ಥೆ-ಯನ್ನೂ
ವ್ಯವ-ಸ್ಥೆ-ಯಾ-ಗಿತ್ತು
ವ್ಯವ-ಸ್ಥೆಯೂ
ವ್ಯವ-ಹ-ರಿ-ಸು-ತ್ತಿ-ದ್ದರು
ವ್ಯವ-ಹ-ರಿ-ಸು-ವ-ವರು
ವ್ಯವ-ಹ-ರಿ-ಸು-ವಾಗ
ವ್ಯವ-ಹಾ-ರಕ್ಕೆ
ವ್ಯವ-ಹಾ-ರ-ಗಳನ್ನೆಲ್ಲ
ವ್ಯವ-ಹಾ-ರ-ಗ-ಳಾದ
ವ್ಯವ-ಹಾ-ರ-ಗಳು
ವ್ಯವ-ಹಾ-ರದ
ವ್ಯವ-ಹಾ-ರ-ದಲ್ಲಿ
ವ್ಯವ-ಹಾ-ರ-ವನ್ನು
ವ್ಯಾಂಕೋ-ವ-ರನ್ನು
ವ್ಯಾಂಕೋ-ವ-ರಿ-ನಲ್ಲಿ
ವ್ಯಾಂಕೋ-ವರ್
ವ್ಯಾಂಕೋ-ವ-ರ್ವ-ರೆಗೆ
ವ್ಯಾಕ
ವ್ಯಾಕ-ರಣ
ವ್ಯಾಕ-ರ-ಣಕ್ಕೆ
ವ್ಯಾಕ-ರ-ಣದ
ವ್ಯಾಕ-ರ-ಣ-ದೋ-ಷ-ವೊಂ-ದನ್ನು
ವ್ಯಾಕ-ರ-ಣ-ಭ್ಯಾ-ಸ-ವನ್ನು
ವ್ಯಾಕ-ರ-ಣ-ವನ್ನು
ವ್ಯಾಕ-ರ-ಣವೇ
ವ್ಯಾಕು-ಲ-ತೆಯ
ವ್ಯಾಕು-ಲ-ತೆ-ಯೆಂ-ಬು-ದೇ-ನಿ-ದ್ದರೂ
ವ್ಯಾಕು-ಲ-ತೆಯೇ
ವ್ಯಾಕು-ಲ-ದಿಂದ
ವ್ಯಾಕು-ಲಿ-ತ-ರಾ-ದರು
ವ್ಯಾಖ್ಯಾನ
ವ್ಯಾಖ್ಯಾ-ನ-ಕ್ಕಾಗಿ
ವ್ಯಾಖ್ಯಾ-ನ-ಗಳ
ವ್ಯಾಖ್ಯಾ-ನ-ಗಳನ್ನು
ವ್ಯಾಖ್ಯಾ-ನ-ಗಳು
ವ್ಯಾಖ್ಯಾ-ನ-ದಲ್ಲಿ
ವ್ಯಾಖ್ಯಾ-ನ-ದಿಂದ
ವ್ಯಾಖ್ಯಾ-ನ-ದೊಂ-ದಿಗೆ
ವ್ಯಾಖ್ಯಾ-ನ-ವನ್ನು
ವ್ಯಾಖ್ಯಾ-ನ-ವನ್ನೂ
ವ್ಯಾಖ್ಯಾ-ನವೂ
ವ್ಯಾಖ್ಯಾ-ನಿಸಿ
ವ್ಯಾಖ್ಯಾ-ನಿ-ಸಿ-ದು-ದ-ಕ್ಕಾಗಿ
ವ್ಯಾಖ್ಯಾ-ನಿ-ಸು-ವು-ದ-ರಲ್ಲಿ
ವ್ಯಾಖ್ಯೆ
ವ್ಯಾಖ್ಯೆ-ಯನ್ನು
ವ್ಯಾಗ-ನ್ನಲ್ಲಿ
ವ್ಯಾಗ-ನ್ನಿ-ನಲ್ಲಿ
ವ್ಯಾಧ
ವ್ಯಾಧ-ಗೃ-ಹದಿ
ವ್ಯಾಪಕ
ವ್ಯಾಪ-ಕ-ವಾಗಿ
ವ್ಯಾಪ-ಕ-ವಾ-ಗಿದ್ದ
ವ್ಯಾಪ-ಕ-ವಾ-ಗಿದ್ದು
ವ್ಯಾಪಾರ
ವ್ಯಾಪಾ-ರ-ವ್ಯ-ವ-ಹಾರ
ವ್ಯಾಪಾ-ರ-ಕ್ಕಾಗಿ
ವ್ಯಾಪಾ-ರ-ಗಳ
ವ್ಯಾಪಾ-ರದ
ವ್ಯಾಪಾ-ರ-ವಾಗಿ
ವ್ಯಾಪಾ-ರಸ್ಥ
ವ್ಯಾಪಾರಿ
ವ್ಯಾಪಾ-ರಿ-ಗಳು
ವ್ಯಾಪಾ-ರಿಗೂ
ವ್ಯಾಪಾರೀ
ವ್ಯಾಪಿ-ಯಾದ
ವ್ಯಾಪಿಸಿ
ವ್ಯಾಪಿ-ಸಿ-ಕೊಂಡು
ವ್ಯಾಪಿ-ಸು-ತ್ತಿದೆ
ವ್ಯಾಪ್ತಿಯು
ವ್ಯಾಮೋ-ಹ-ವನ್ನು
ವ್ಯಾಮೋ-ಹ-ವಿ-ತ್ತು-ಅದು
ವ್ಯಾಯಾಮ
ವ್ಯಾವ-ಹಾ-ರಿಕ
ವ್ಯಾಸಂ-ಗಕ್ಕೆ
ವ್ಯಾಸಂ-ಗ-ಮಾ-ಡಲು
ವ್ಯಾಸ-ರಾವ್
ವ್ರತ
ವ್ರತ-ತೊ-ಟ್ಟಿ-ದ್ದೇನೆ
ವ್ರತ-ಧಾ-ರಣೆ
ವ್ರತ-ವನ್ನು
ಶ
ಶಂಕರ
ಶಂಕ-ರನು
ಶಂಕ-ರ-ಪಾಂ-ಡು-ರಂ-ಗನ
ಶಂಕ-ರಯ್ಯ
ಶಂಕ-ರ-ಯ್ಯ-ನ-ವ-ರನ್ನು
ಶಂಕ-ರ-ಲಾಲ್
ಶಂಕರಾ
ಶಂಕ-ರಾ-ಚಾ-ರ್ಯ-ರಾ-ಮಾ-ನು-ಜಾ-ಚಾ-ರ್ಯರ
ಶಂಕ-ರಾ-ಚಾ-ರ್ಯರ
ಶಂಕ-ರಾ-ಚಾ-ರ್ಯ-ರಂತೆ
ಶಂಕಿ-ಸಿ-ದ-ನಲ್ಲ
ಶಂಕಿ-ಸಿ-ದರು
ಶಂಕಿ-ಸಿ-ದ್ದರು
ಶಂಕಿ-ಸು-ತ್ತಾ-ರೆಂ-ದರೆ
ಶಂಕೆ
ಶಂಕೆ-ಯಿಂದ
ಶಂಕೆ-ಯುಂ-ಟಾ-ಗ-ಬ-ಹುದು
ಶಂಕೆಯೂ
ಶಂಖ
ಶಂಭು-ನಾಥ
ಶಂಭು-ನಾ-ಥ-ಜಿಯ
ಶಕ್ತ
ಶಕ್ತ-ನ-ನ್ನಾಗಿ
ಶಕ್ತರೂ
ಶಕ್ತಿ
ಶಕ್ತಿ
ಶಕ್ತಿ-ಇ-ವು-ಗಳ
ಶಕ್ತಿ-ಉ-ತ್ಸಾ-ಹ
ಶಕ್ತಿ-ಉ-ತ್ಸಾ-ಹ-ಗಳನ್ನು
ಶಕ್ತಿ-ಉ-ತ್ಸಾ-ಹದ
ಶಕ್ತಿ-ಉ-ತ್ಸಾ-ಹ-ದಿಂದ
ಶಕ್ತಿ-ಜ್ಞಾ-ನ-ಗಳನ್ನು
ಶಕ್ತಿ-ಇ-ವು-ಗಳನ್ನು
ಶಕ್ತಿ-ಇ-ವೆ-ರ-ಡರ
ಶಕ್ತಿ-ಗಳ
ಶಕ್ತಿ-ಗ-ಳಂತೆ
ಶಕ್ತಿ-ಗಳನ್ನು
ಶಕ್ತಿ-ಗಳನ್ನೂ
ಶಕ್ತಿ-ಗಳಿಂದ
ಶಕ್ತಿ-ಗಳು
ಶಕ್ತಿ-ಗಳೂ
ಶಕ್ತಿಗೆ
ಶಕ್ತಿ-ಚೈ-ತನ್ಯ
ಶಕ್ತಿ-ದಾ-ಯ-ಕವೂ
ಶಕ್ತಿ-ಪ್ರ-ದ-ರ್ಶನ
ಶಕ್ತಿ-ಪ್ರ-ದ-ರ್ಶ-ನ-ಗಳನ್ನು
ಶಕ್ತಿ-ಪ್ರ-ವಾ-ಹ-ದಲ್ಲಿ
ಶಕ್ತಿ-ಮೀರಿ
ಶಕ್ತಿಯ
ಶಕ್ತಿ-ಯಂ-ತಹ
ಶಕ್ತಿ-ಯ-ನ್ನೀ-ಯಲಿ
ಶಕ್ತಿ-ಯನ್ನು
ಶಕ್ತಿ-ಯನ್ನೂ
ಶಕ್ತಿ-ಯ-ನ್ನೆಲ್ಲ
ಶಕ್ತಿ-ಯ-ನ್ನೊ-ಳ-ಗೊಂಡ
ಶಕ್ತಿ-ಯಲ್ಲಿ
ಶಕ್ತಿ-ಯಾಗಿ
ಶಕ್ತಿ-ಯಾ-ಗಿದೆ
ಶಕ್ತಿ-ಯಾದ
ಶಕ್ತಿ-ಯಿಂದ
ಶಕ್ತಿ-ಯಿಂ-ದಲೂ
ಶಕ್ತಿ-ಯಿಂ-ದಲೇ
ಶಕ್ತಿ-ಯಿ-ದೆ-ಆ-ದರೆ
ಶಕ್ತಿ-ಯಿ-ರ-ಲಿಲ್ಲ
ಶಕ್ತಿ-ಯಿ-ರು-ವುದು
ಶಕ್ತಿಯು
ಶಕ್ತಿ-ಯುತ
ಶಕ್ತಿ-ಯು-ತ-ವಾಗಿ
ಶಕ್ತಿ-ಯುಳ್ಳ
ಶಕ್ತಿಯೂ
ಶಕ್ತಿ-ಯೆಂದು
ಶಕ್ತಿಯೇ
ಶಕ್ತಿ-ಯೊಂದು
ಶಕ್ತಿ-ವಂ-ತರು
ಶಕ್ತಿ-ಶಾಲಿ
ಶಕ್ತಿ-ಶಾ-ಲಿ-ಗ-ಳಾಗಿ
ಶಕ್ತಿ-ಶಾ-ಲಿ-ಗ-ಳಾದ
ಶಕ್ತಿ-ಶಾ-ಲಿ-ಗಳೂ
ಶಕ್ತಿ-ಶಾ-ಲಿ-ಯಾ-ಗ-ಬೇಕು
ಶಕ್ತಿ-ಶಾ-ಲಿ-ಯಾದ
ಶಕ್ತಿ-ಶಾ-ಲಿಯೂ
ಶಕ್ತಿ-ಸಾ-ಮರ್ಥ್ಯ
ಶಕ್ತಿ-ಹ್ರಾಸ
ಶಕ್ತ್ಯು-ತ್ಸಾ-ಹ-ಗಳನ್ನು
ಶಕ್ತ್ಯು-ತ್ಸಾ-ಹದ
ಶತ-ಮಾ-ನ-ಗಳಲ್ಲಿ
ಶತ-ಮಾ-ನ-ಗ-ಳಿಂ-ದಲೂ
ಶತ-ಮಾ-ನದ
ಶತ-ಮಾ-ನ-ದಲ್ಲಿ
ಶತ-ಮಾ-ನ-ದ-ವ-ರೆಗೂ
ಶತ-ಮಾ-ನ-ದಿಂದ
ಶತ-ಮಾನವೇ
ಶತ-ಶ-ತ-ಮಾ-ನ-ಗಳ
ಶತ-ಶ-ತ-ಮಾ-ನ-ಗ-ಳ-ವ-ರೆಗೆ
ಶತ-ಶ-ತ-ಮಾ-ನ-ಗ-ಳಿಂ-ದಲೂ
ಶತ್ರುಂ-ಜ-ಯ-ವೆಂಬ
ಶತ್ರು-ಗಳ
ಶತ್ರು-ಗ-ಳ-ನ್ನಾ-ಗಲಿ
ಶತ್ರು-ಮಧ್ಯೇ
ಶಪಿ-ಸ-ಲೂ-ಬ-ಹು-ದು-ಇ-ವ-ನ್ಯಾಕೆ
ಶಪಿ-ಸುತ್ತ
ಶಪಿ-ಸು-ವು-ದ-ಕ್ಕಾಗಿ
ಶಬ್ದ
ಶಬ್ದಕ್ಕೆ
ಶಬ್ದ-ಗಳಿಂದ
ಶಬ್ದ-ಗ-ಳಿಗೆ
ಶಬ್ದ-ಗ-ಳಿಗೇ
ಶಬ್ದ-ಗಳು
ಶಬ್ದ-ಗಳೇ
ಶಬ್ದ-ತ-ರಂ-ಗದ
ಶಬ್ದ-ವನ್ನು
ಶಬ್ದ-ವನ್ನೂ
ಶಬ್ದ-ವಾ-ಯಿತು
ಶಬ್ದವೂ
ಶಬ್ದವೇ
ಶಮನ
ಶಯ-ಗಳನ್ನು
ಶಯ್ಯೆಯ
ಶರ-ಣ-ರಾದ
ಶರ-ಣಾ-ಗ-ತ-ನಾ-ಗಿ-ದ್ದೇನೆ
ಶರ-ಣಾ-ಗ-ತ-ಳಾ-ದದ್ದು
ಶರ-ಣಾ-ಗತಿ
ಶರ-ಣಾ-ಗ-ತಿ-ಭಾ-ವ-ಗಳು
ಶರ-ಣಾ-ಗ-ತಿ-ಭಾ-ವವು
ಶರ-ಣಾಗಿ
ಶರ-ಣಾ-ಗಿರಿ
ಶರ-ಣಾ-ದಾಗ
ಶರತ್
ಶರ-ತ್ಕಾ-ಲ-ಗಳು
ಶರ-ತ್ಕಾ-ಲದ
ಶರ-ಶ-ಯ್ಯೆ-ಯಂತೆ
ಶರೀರ
ಶರೀ-ರ-ಇಂ-ದ್ರಿಯ
ಶರೀ-ರಕ್ಕೆ
ಶರೀ-ರ-ಗಳನ್ನೂ
ಶರೀ-ರದ
ಶರೀ-ರ-ದ-ಲ್ಲಿ-ರುವ
ಶರೀ-ರ-ದಲ್ಲೂ
ಶರೀ-ರ-ದ-ವ-ರಾ-ಗಿ-ರ-ಬೇಕು
ಶರೀ-ರ-ದಿಂದ
ಶರೀ-ರ-ಪ್ರ-ಜ್ಞೆ-ಯನ್ನು
ಶರೀ-ರ-ಪ್ರ-ಜ್ಞೆ-ಯನ್ನೂ
ಶರೀ-ರ-ಮಾದ್ಯಂ
ಶರೀ-ರ-ರ-ಚನೆ
ಶರೀ-ರ-ಲ-ಕ್ಷ-ಣ-ದಲ್ಲಿ
ಶರೀ-ರ-ವನ್ನು
ಶರೀ-ರ-ವಲ್ಲ
ಶರೀ-ರವು
ಶರೀ-ರವೂ
ಶರೀ-ರ-ವೆಂಬ
ಶರೀ-ರವೇ
ಶರೀ-ರ-ಶಾ-ಸ್ತ್ರ-ಜ್ಞ-ರನ್ನು
ಶಶಿ-ಪದ
ಶಾಂಕ-ರ-ಭಾ-ಷ್ಯ-ದಿಂ-ದೊ-ಡ-ಗೂ-ಡಿದ
ಶಾಂತ
ಶಾಂತ-ಏ-ಕಾಂತ
ಶಾಂತ-ಗಂ-ಭೀರ
ಶಾಂತ-ಮ-ಧುರ
ಶಾಂತ-ಗೊಂಡು
ಶಾಂತ-ಗೊ-ಳಿ-ಸುವ
ಶಾಂತ-ಭಾ-ವ-ದಿಂದ
ಶಾಂತ-ರಾ-ಗಿದ್ದು
ಶಾಂತ-ವಾಗಿ
ಶಾಂತ-ವಾ-ಗಿ-ದ್ದುದ
ಶಾಂತ-ವಾ-ಗಿ-ದ್ದು-ಬಿ-ಟ್ಟರು
ಶಾಂತ-ವಾ-ಗಿಯೇ
ಶಾಂತ-ವಾ-ಗಿ-ರು-ತ್ತಾನೆ
ಶಾಂತ-ವಾ-ಗಿ-ಸಲು
ಶಾಂತ-ವಾಗು
ಶಾಂತ-ವಾದ
ಶಾಂತ-ವಾ-ದು-ದನ್ನು
ಶಾಂತ-ಸಾ-ಗ-ರ-ದಲ್ಲಿ
ಶಾಂತಿ
ಶಾಂತಿ-ಆ-ನಂ-ದ-ಗಳಿಂದ
ಶಾಂತಿ-ಸ-ಮ-ನ್ವಯ
ಶಾಂತಿ-ಸ-ಮಾ-ಧಾನ
ಶಾಂತಿ-ದೂ-ತರೇ
ಶಾಂತಿಯ
ಶಾಂತಿ-ಯ-ನ್ನೀಯ
ಶಾಂತಿ-ಯನ್ನು
ಶಾಂತಿ-ಯಿಂದ
ಶಾಂತಿ-ಯಿಂ-ದಿ-ರು-ವಂತೆ
ಶಾಂತಿ-ಯುತ
ಶಾಂತಿ-ವಾ-ದಿಯೂ
ಶಾಕ್ತರ
ಶಾಕ್ತರು
ಶಾಕ್ಯ-ಮುನಿ
ಶಾಕ್ಯ-ಮು-ನಿ-ಯನ್ನು
ಶಾಕ್ಯ-ಮು-ನಿಯು
ಶಾಖ-ದಿಂ-ದಾಗಿ
ಶಾಖೆ-ಗಳ
ಶಾಖೆ-ಗಳು
ಶಾಖೆ-ಯಲ್ಲಿ
ಶಾತ್
ಶಾಪ
ಶಾಫ್
ಶಾಯಿ-ಯಲ್ಲಿ
ಶಾರದಾ
ಶಾರ-ದಾ-ದೇ-ವಿ-ಯರ
ಶಾರ-ದಾ-ದೇ-ವಿ-ಯ-ರಿ-ಗಾಗಿ
ಶಾರ-ದಾ-ದೇ-ವಿ-ಯ-ವರ
ಶಾರ-ದಾ-ದೇ-ವಿ-ಯ-ವ-ರನ್ನು
ಶಾರ-ದಾ-ದೇ-ವಿ-ಯ-ವ-ರಿಗೆ
ಶಾರ-ದಾ-ನಂದ
ಶಾರ-ದಾ-ನಂ-ದರ
ಶಾರ-ದಾ-ನಂ-ದ-ರನ್ನು
ಶಾರ-ದಾ-ನಂ-ದ-ರಿ-ಗಾಗಿ
ಶಾರ-ದಾ-ನಂ-ದ-ರಿ-ಗಾದ
ಶಾರ-ದಾ-ನಂ-ದರು
ಶಾರ-ದಾ-ನಂ-ದರೂ
ಶಾರ-ದಾ-ನಂ-ದ-ರೊ-ಡನೆ
ಶಾರ-ದಾ-ಮ-ಠ-ವನ್ನು
ಶಾರೀ-ರಿಕ
ಶಾಲಾ
ಶಾಲಾ-ಕಾ-ಲೇ-ಜು-ಗಳ
ಶಾಲಿಯೂ
ಶಾಲೀ
ಶಾಲೆ
ಶಾಲೆ-ಗಳನ್ನು
ಶಾಲೆ-ಗ-ಳಿಗೂ
ಶಾಲೆ-ಗಳು
ಶಾಲೆಯ
ಶಾಲೆ-ಯಲ್ಲಿ
ಶಾಶ್ವತ
ಶಾಶ್ವ-ತ-ವಾಗಿ
ಶಾಶ್ವ-ತ-ವಾದ
ಶಾಸ-ಕ-ನಾದ
ಶಾಸನ
ಶಾಸ್ತ್ರ
ಶಾಸ್ತ್ರ-ಗಳ
ಶಾಸ್ತ್ರ-ಗಳನ್ನು
ಶಾಸ್ತ್ರ-ಗಳಲ್ಲಿ
ಶಾಸ್ತ್ರ-ಗ-ಳಾ-ಗಿ-ದ್ದುವು
ಶಾಸ್ತ್ರ-ಗ-ಳಿಗೂ
ಶಾಸ್ತ್ರ-ಗಳು
ಶಾಸ್ತ್ರ-ಗ-ಳೆಲ್ಲ
ಶಾಸ್ತ್ರ-ಗ್ರಂಥ
ಶಾಸ್ತ್ರ-ಗ್ರಂ-ಥ-ಗಳ
ಶಾಸ್ತ್ರ-ಗ್ರಂ-ಥ-ಗಳನ್ನು
ಶಾಸ್ತ್ರ-ಜ್ಞನೂ
ಶಾಸ್ತ್ರ-ಜ್ಞರ
ಶಾಸ್ತ್ರ-ಜ್ಞ-ರನ್ನೂ
ಶಾಸ್ತ್ರ-ಜ್ಞ-ರಾದ
ಶಾಸ್ತ್ರ-ಜ್ಞಾನ
ಶಾಸ್ತ್ರದ
ಶಾಸ್ತ್ರ-ವ-ನ್ನೆಲ್ಲ
ಶಾಸ್ತ್ರ-ವಾ-ಕ್ಯ-ಗಳ
ಶಾಸ್ತ್ರ-ವಿ-ಚಾ-ರ-ಗಳ
ಶಾಸ್ತ್ರ-ವಿ-ಚಾ-ರ-ಗಳನ್ನು
ಶಾಸ್ತ್ರಾ-ಧ್ಯ-ಯನ
ಶಾಸ್ತ್ರಿ-ಗಳ
ಶಾಸ್ತ್ರಿ-ಗಳು
ಶಾಸ್ತ್ರೀಯ
ಶಾಸ್ತ್ರೀ-ಯ-ವಾಗಿ
ಶಾಸ್ತ್ರೀ-ಯ-ವಾದ
ಶಿಕಾಗೋ
ಶಿಕಾ-ಗೋ-ಗಿಂತ
ಶಿಕಾ-ಗೋಗೆ
ಶಿಕಾ-ಗೋದ
ಶಿಕಾ-ಗೋ-ದಲ್ಲಿ
ಶಿಕಾ-ಗೋ-ದ-ಲ್ಲಿದ್ದ
ಶಿಕಾ-ಗೋ-ದ-ಲ್ಲಿದ್ದು
ಶಿಕಾ-ಗೋ-ದ-ಲ್ಲಿನ
ಶಿಕಾ-ಗೋ-ದ-ಲ್ಲಿ-ರು-ತ್ತಿದ್ದ
ಶಿಕಾ-ಗೋ-ದಿಂದ
ಶಿಕಾ-ಗೋ-ವನ್ನು
ಶಿಕಾ-ಗೋ-ವರೆ-ಗಿನ
ಶಿಕ್ಷ-ಕ-ನಾಗಿ
ಶಿಕ್ಷಣ
ಶಿಕ್ಷ-ಣ-ಕ್ಕಾಗಿ
ಶಿಕ್ಷ-ಣಕ್ಕೆ
ಶಿಕ್ಷ-ಣದ
ಶಿಕ್ಷ-ಣ-ದಿಂ-ದಾಗಿ
ಶಿಕ್ಷ-ಣ-ವನ್ನು
ಶಿಕ್ಷ-ಣವು
ಶಿಕ್ಷ-ಣ-ವೆಂದರೆ
ಶಿಕ್ಷೆ
ಶಿಕ್ಷೆಗೆ
ಶಿಕ್ಷೆಯ
ಶಿಕ್ಷೆ-ಯಲ್ಲ
ಶಿಕ್ಷೆ-ಯಲ್ಲೇ
ಶಿಖರ
ಶಿಖ-ರ-ಕ್ಕೇ-ರು-ತ್ತ-ದೆ-ಯೆಂದು
ಶಿಖ-ರ-ಗಳ
ಶಿಖ-ರ-ಗಳಲ್ಲಿ
ಶಿಖ-ರ-ಗ-ಳಿವೆ
ಶಿಖ-ರ-ಗಳು
ಶಿಖ-ರದ
ಶಿಖ-ರ-ದ-ಲ್ಲಿ-ದ್ದರು
ಶಿಖ-ರ-ದಿಂದ
ಶಿಖ-ರ-ದೆ-ಡೆಗೆ
ಶಿಖ-ರ-ವನ್ನು
ಶಿಖ-ರ-ವನ್ನೇ
ಶಿಖ-ರ-ವ-ನ್ನೇ-ರ-ಲಾ-ರಂ-ಭಿ-ಸಿ-ದ್ದರು
ಶಿಖ-ರ-ವ-ನ್ನೇ-ರಲು
ಶಿಖ-ರ-ವ-ನ್ನೇ-ರಿ-ದರು
ಶಿಖ-ರ-ವ-ನ್ನೇ-ರಿ-ದಾಗ
ಶಿಖ-ರ-ವ-ನ್ನೇ-ರು-ವುದು
ಶಿಖ-ರವೂ
ಶಿಥಿಲ
ಶಿರ
ಶಿರ-ಗಾಂವ್
ಶಿರ-ಗಾಂ-ವ್ಕರ್
ಶಿರದ
ಶಿರ-ಬಾ-ಗಲಿ
ಶಿರ-ಬಾಗಿ
ಶಿರ-ಮೆಟ್ಟಿ
ಶಿಲಾ-ಪೀ-ಠದ
ಶಿಲಾ-ಸ್ಮಾ-ರಕ
ಶಿಲು-ಬೆಗೆ
ಶಿಲು-ಬೆ-ಗೇ-ರಿ-ಸಿ-ದರು
ಶಿಲು-ಬೆಯ
ಶಿಲೆ
ಶಿಲೆಗೆ
ಶಿಲೆಯ
ಶಿಲೆ-ಯನ್ನೇ
ಶಿಲೆಯೂ
ಶಿಲ್ಪ
ಶಿಲ್ಪ-ಗಳು
ಶಿವ
ಶಿವ-ಗಂ-ಗೆ-ಯನು
ಶಿವ-ನನ್ನು
ಶಿವನು
ಶಿವ-ಪೂ-ಜೆ-ಯನ್ನು
ಶಿವ-ರಾಜ್
ಶಿವ-ರು-ದ್ರಪ್ಪ
ಶಿವ-ಲಿಂ-ಗ-ವನ್ನು
ಶಿವಾ
ಶಿವಾ-ನಂದ
ಶಿವಾ-ನಂ-ದರ
ಶಿವಾ-ನಂ-ದ-ರಿಗೆ
ಶಿವೋ
ಶಿಶಿರ
ಶಿಶು
ಶಿಶು-ಗಳನ್ನು
ಶಿಶು-ಪಾ-ಲನ
ಶಿಶು-ವಾದ
ಶಿಶು-ವಿಗೆ
ಶಿಶು-ವಿ-ನಂತೆ
ಶಿಶುವೋ
ಶಿಶು-ಸ-ಹಜ
ಶಿಷ್ಟ
ಶಿಷ್ಟರ
ಶಿಷ್ಟ-ರ-ಕ್ಷ-ಣೆ-ಯನ್ನು
ಶಿಷ್ಟಾ-ಚಾರ
ಶಿಷ್ಟಾ-ಚಾ-ರ-ಗಳನ್ನೂ
ಶಿಷ್ಯ
ಶಿಷ್ಯ-ತ್ವ-ವನ್ನು
ಶಿಷ್ಯನ
ಶಿಷ್ಯ-ನನ್ನು
ಶಿಷ್ಯ-ನಾದ
ಶಿಷ್ಯ-ನಿಗೆ
ಶಿಷ್ಯ-ನೊಬ್ಬ
ಶಿಷ್ಯ-ನೊ-ಬ್ಬನ
ಶಿಷ್ಯರ
ಶಿಷ್ಯ-ರಂತೆ
ಶಿಷ್ಯ-ರ-ನ್ನಾಗಿ
ಶಿಷ್ಯ-ರನ್ನು
ಶಿಷ್ಯ-ರಲ್ಲಿ
ಶಿಷ್ಯ-ರ-ಲ್ಲೊ-ಬ್ಬ-ರಾದ
ಶಿಷ್ಯ-ರ-ಲ್ಲೊ-ಬ್ಬರು
ಶಿಷ್ಯ-ರಷ್ಟೇ
ಶಿಷ್ಯ-ರಾ-ಗಲು
ಶಿಷ್ಯ-ರಾ-ಗಿ-ದ್ದ-ವ-ರನ್ನೂ
ಶಿಷ್ಯ-ರಾದ
ಶಿಷ್ಯ-ರಾ-ದರು
ಶಿಷ್ಯ-ರಾ-ದ-ವ-ರಲ್ಲಿ
ಶಿಷ್ಯ-ರಾ-ದೆವು
ಶಿಷ್ಯ-ರಿಂದ
ಶಿಷ್ಯ-ರಿ-ಗಂತೂ
ಶಿಷ್ಯ-ರಿ-ಗಾದ
ಶಿಷ್ಯ-ರಿ-ಗಾ-ದರೂ
ಶಿಷ್ಯ-ರಿಗೂ
ಶಿಷ್ಯ-ರಿಗೆ
ಶಿಷ್ಯ-ರಿ-ಗೆಲ್ಲ
ಶಿಷ್ಯ-ರಿ-ದ್ದ-ರೆಂ-ಬು-ದನ್ನು
ಶಿಷ್ಯರು
ಶಿಷ್ಯ-ರು
ಶಿಷ್ಯ-ರು-ಆ-ಪ್ತರು
ಶಿಷ್ಯ-ರು-ಬೆಂ-ಬ-ಲಿ-ಗ-ರೆ-ನ್ನಿ-ಸಿ-ಕೊಂ-ಡ-ವರು
ಶಿಷ್ಯ-ರು-ಭ-ಕ್ತ-ರನ್ನು
ಶಿಷ್ಯರೂ
ಶಿಷ್ಯ-ರೆಂ-ದರೆ
ಶಿಷ್ಯ-ರೆಂದು
ಶಿಷ್ಯ-ರೆಂದೂ
ಶಿಷ್ಯ-ರೆ-ದುರು
ಶಿಷ್ಯ-ರೆಲ್ಲ
ಶಿಷ್ಯ-ರೆ-ಲ್ಲರ
ಶಿಷ್ಯರೇ
ಶಿಷ್ಯ-ರೊಂ-ದಿಗೆ
ಶಿಷ್ಯ-ರೊಬ್ಬ
ಶಿಷ್ಯ-ರೊ-ಬ್ಬರ
ಶಿಷ್ಯ-ರೊ-ಬ್ಬ-ರಿಗೆ
ಶಿಷ್ಯ-ರೊ-ಬ್ಬರು
ಶಿಷ್ಯ-ಳ-ನ್ನಾಗಿ
ಶಿಷ್ಯ-ವ-ರ್ಗಕ್ಕೆ
ಶಿಷ್ಯ-ವ-ರ್ಗ-ಕ್ಕೆಲ್ಲ
ಶಿಷ್ಯ-ವೃಂ-ದಕ್ಕೆ
ಶಿಷ್ಯ-ವೃಂ-ದದ
ಶಿಷ್ಯ-ವೃಂ-ದ-ದಲ್ಲಿ
ಶಿಷ್ಯಾ-ಗ್ರಣಿ
ಶಿಷ್ಯಾ-ಗ್ರ-ಣಿ-ಯಾದ
ಶಿಷ್ಯೆ
ಶಿಷ್ಯೆ-ಯ-ರ-ಲ್ಲೊ-ಬ್ಬ-ಳಾದ
ಶಿಷ್ಯೆ-ಯ-ರಾದ
ಶಿಷ್ಯೆ-ಯಾಗಿ
ಶಿಷ್ಯೆ-ಯಾ-ಗಿದ್ದ
ಶಿಷ್ಯೆ-ಯಾದ
ಶಿಷ್ಯೆ-ಯಾ-ದ-ಳ-ಲ್ಲದೆ
ಶಿಷ್ಯೆ-ಯಾದೆ
ಶಿಷ್ಯೆಯೂ
ಶಿಷ್ಯೆ-ಯೊ-ಬ್ಬ-ಳಿಗೆ
ಶಿಸ್ತಿನ
ಶಿಸ್ತು-ಬದ್ಧ
ಶಿಸ್ತು-ಬ-ದ್ಧ-ವಾಗಿ
ಶೀಘ್ರ
ಶೀಘ್ರ-ದಲ್ಲಿ
ಶೀಘ್ರ-ದಲ್ಲೇ
ಶೀಘ್ರ-ಲಿಪಿ
ಶೀಘ್ರ-ಲಿ-ಪಿ-ಕಾ-ರ-ನಾ-ಗಿ-ದ್ದ-ನಷ್ಟೇ
ಶೀಘ್ರ-ಲಿ-ಪಿ-ಕಾ-ರ-ನಾ-ಗಿ-ರ-ಲಿಲ್ಲ
ಶೀಘ್ರ-ಲಿ-ಪಿ-ಕಾ-ರ-ನಾದ
ಶೀಘ್ರ-ಲಿ-ಪಿ-ಕಾ-ರ-ನೊ-ಬ್ಬ-ನನ್ನು
ಶೀಘ್ರ-ಲಿ-ಪಿ-ಯಿಂದ
ಶೀಘ್ರ-ವಾಗಿ
ಶೀಘ್ರವೂ
ಶೀರ್ಷಿ-ಕೆ-ಯ-ಡಿ-ಯಲ್ಲಿ
ಶೀರ್ಷಿ-ಕೆ-ಯನ್ನು
ಶೀರ್ಷಿ-ಕೆ-ಯಲ್ಲಿ
ಶೀಲ
ಶೀಲ-ಗು-ಣ-ಗಳ
ಶೀಲಕ್ಕೆ
ಶೀಲದ
ಶೀಲ-ನಿ-ರ್ಮಾ-ಣದ
ಶೀಲ-ಬ-ಲ-ವನ್ನು
ಶೀಲ-ಭ್ರ-ಷ್ಟ-ನಾ-ಗಿದ್ದೀ
ಶೀಲ-ವಂತ
ಶೀಲ-ವನ್ನು
ಶೀಲ-ವೊಂದೇ
ಶೀಲ-ಶಕ್ತಿ
ಶೀಲ-ಸಂ-ಪ-ನ್ನ-ರಾದ
ಶೀಲ-ಸಂ-ಪ-ನ್ನ-ವಾದ
ಶುಚಿ-ಗೊ-ಳಿ-ಸುವ
ಶುಚಿ-ಯಾ-ದ-ವರು
ಶುದ್ಧ
ಶುದ್ಧ-ಬು-ದ್ಧ-ಮುಕ್ತ
ಶುದ್ಧ-ಜ್ಞಾ-ನ-ಸ್ವ-ರೂಪಿ
ಶುದ್ಧ-ಪ-ಡಿ-ಸು-ತ್ತೇನೆ
ಶುದ್ಧ-ವಾ-ಗು-ತ್ತದೆ
ಶುದ್ಧಿ-ಗೊ-ಳಿ-ಸಲು
ಶುಭ
ಶುಭ-ವಾ-ಗಲಿ
ಶುಭಾ
ಶುಭಾ-ಶ-ಯ-ಗಳನ್ನು
ಶುಭಾ-ಶೀ-ರ್ವಾ-ದ-ವನ್ನು
ಶುಭ್ರ
ಶುಭ್ರ-ವಾದ
ಶುರು-ವಾ-ಗ-ಬೇ-ಕಷ್ಟೆ
ಶುಲ್ಕ
ಶುಲ್ಕ-ವ-ನ್ನಿ-ಟ್ಟಿದ್ದು
ಶುಲ್ಕ-ವನ್ನು
ಶುಲ್ಕ-ವನ್ನೂ
ಶುಲ್ಕ-ವಿ-ರ-ಲಿಲ್ಲ
ಶೂದ್ರನು
ಶೂದ್ರ-ರಿಗೆ
ಶೂದ್ರರು
ಶೂದ್ರಾ-ದಿ-ಗಳು
ಶೂನ್ಯ
ಶೂನ್ಯ-ದಲ್ಲಿ
ಶೂನ್ಯ-ರಾ-ಗಿಲ್ಲ
ಶೂನ್ಯ-ವ-ನ್ನಾ-ಗಿಯೂ
ಶೂನ್ಯ-ವಾ-ಗಿದೆ
ಶೃಂಖ-ಲೆ-ಯಲ್ಲಿ
ಶೃಣ್ವಂತು
ಶೇಷಾದ್ರಿ
ಶೈಕ್ಷ-ಣಿಕ
ಶೈಕ್ಷ-ಣಿ-ಕ-ವಾಗಿ
ಶೈಲಿ
ಶೈಲಿಯ
ಶೈಲಿ-ಯಲ್ಲಿ
ಶೈವರು
ಶೈಶ-ವ-ದಲ್ಲಿ
ಶೈಶ-ವಾ-ವ-ಸ್ಥೆ-ಯ-ಲ್ಲಿ-ದ್ದು-ದ-ರಿಂದ
ಶೋಕ-ವನ್ನು
ಶೋಕ-ಸಾ-ಗ-ರ-ದಲ್ಲಿ
ಶೋಕಿ-ಗ-ನು-ಸಾ-ರ-ವಾಗಿ
ಶೋಚ
ಶೋಚ-ನೀ-ಯ-ವಾಗಿ
ಶೋಧ-ನೆ-ಗಳು
ಶೋಧಿ-ಸು-ತ್ತಿ-ದ್ದರು
ಶೋಪೆ-ನ್ಹಾರ್ನ
ಶೋಭ-ನೀಯ
ಶೋಭಾ-ಯ-ಮಾ-ನ-ವಾದ
ಶೋರ-ನೂರು
ಶೋರ್
ಶೋಷಿ-ಸಲು
ಶೋಷಿ-ಸು-ವುದನ್ನು
ಶೌಚ
ಶ್ಯಾಮ್ಜಿ
ಶ್ರದ್ಧಾಂ-ಜ-ಲಿ-ಯ-ನ್ನ-ರ್ಪಿ-ಸಿದೆ
ಶ್ರದ್ಧಾ-ಳು-ಗಳೂ
ಶ್ರದ್ಧಾ-ವಂತ
ಶ್ರದ್ಧಾ-ವಂ-ತ-ರಾ-ಗಿ-ದ್ದಷ್ಟೇ
ಶ್ರದ್ಧಾ-ವಂ-ತ-ರಾದ
ಶ್ರದ್ಧಾ-ವಂ-ತರೂ
ಶ್ರದ್ಧಾ-ವಂತೆ
ಶ್ರದ್ಧೆ
ಶ್ರದ್ಧೆ-ಉ-ತ್ಸಾ-ಹ-ಗಳಲ್ಲಿ
ಶ್ರದ್ಧೆ-ವಿ-ಶ್ವಾಸ
ಶ್ರದ್ಧೆ-ಗಳನ್ನು
ಶ್ರದ್ಧೆಗೆ
ಶ್ರದ್ಧೆಯ
ಶ್ರದ್ಧೆ-ಯನ್ನು
ಶ್ರದ್ಧೆ-ಯ-ನ್ನುಂ-ಟು-ಮಾ-ಡುವ
ಶ್ರದ್ಧೆ-ಯಿಂದ
ಶ್ರದ್ಧೆ-ಯಿಟ್ಟು
ಶ್ರದ್ಧೆ-ಯಿಡಿ
ಶ್ರದ್ಧೆ-ಯಿಡು
ಶ್ರದ್ಧೆ-ಯಿತ್ತು
ಶ್ರದ್ಧೆ-ಯಿ-ದೆಯೋ
ಶ್ರದ್ಧೆ-ಯಿಲ್ಲ
ಶ್ರದ್ಧೆಯು
ಶ್ರದ್ಧೆ-ಯುಂ-ಟಾ-ಗಲು
ಶ್ರದ್ಧೆ-ಯುಂ-ಟಾ-ದಾಗ
ಶ್ರದ್ಧೆ-ಯು-ಳ್ಳ-ವ-ರಾ-ಗಿ-ದ್ದರು
ಶ್ರದ್ಧೆ-ಯೆಂ-ತ-ಹದು
ಶ್ರದ್ಧೆಯೇ
ಶ್ರದ್ಧೆ-ಯೊಂ-ದಿ-ದ್ದರೆ
ಶ್ರದ್ಧೆ-ಯೊಂದು
ಶ್ರಮ
ಶ್ರಮದ
ಶ್ರಮ-ದಾ-ಯಕ
ಶ್ರಮ-ದಾ-ಯ-ಕ-ವಾ-ಗಿ-ದ್ದರೂ
ಶ್ರಮ-ದಿಂದ
ಶ್ರಮ-ದಿಂ-ದಾಗಿ
ಶ್ರಮ-ವನ್ನು
ಶ್ರಮ-ವ-ಹಿಸಿ
ಶ್ರಮ-ವಿತ್ತು
ಶ್ರಮ-ವೆಂ-ಥದು
ಶ್ರಮ-ವೆಲ್ಲ
ಶ್ರಮಿ-ಸ-ಬೇ-ಕಾದ
ಶ್ರಮಿ-ಸ-ಬೇ-ಕು-ಇ-ಲ್ಲವೆ
ಶ್ರಮಿ-ಸಲು
ಶ್ರಮಿಸಿ
ಶ್ರಮಿ-ಸಿ-ದರು
ಶ್ರಮಿ-ಸಿ-ದ-ವ-ರಾದ
ಶ್ರಮಿ-ಸಿ-ದುವು
ಶ್ರಮಿ-ಸಿದ್ದು
ಶ್ರಮಿ-ಸು-ತ್ತಾ-ನೆಯೋ
ಶ್ರಮಿ-ಸು-ತ್ತಾ-ರೆಯೋ
ಶ್ರಮಿ-ಸುತ್ತಿ
ಶ್ರಮಿ-ಸು-ತ್ತಿ-ದ್ದರು
ಶ್ರಮಿ-ಸು-ತ್ತಿ-ರುವ
ಶ್ರಮಿ-ಸು-ವು-ದ-ರಲ್ಲೇ
ಶ್ರಮಿ-ಸು-ವುದು
ಶ್ರಮಿ-ಸು-ವು-ದು-ಇವೇ
ಶ್ರಮಿ-ಸೋಣ
ಶ್ರಾದ್ಧ-ಕರ್ಮ
ಶ್ರೀ
ಶ್ರೀಕೃಷ್ಣ
ಶ್ರೀಕೃ-ಷ್ಣನ
ಶ್ರೀಕೃ-ಷ್ಣ-ಭ-ಕ್ತರು
ಶ್ರೀಗು-ರು
ಶ್ರೀಗು-ರು-ಮ-ಹಾ-ರಾ-ಜರು
ಶ್ರೀಘ್ರ
ಶ್ರೀಘ್ರ-ಲಿಪಿ
ಶ್ರೀನಿ-ವಾ-ಸಾ-ಚಾ-ರ್ಲು-ಇ-ವ-ರು-ಗಳು
ಶ್ರೀಪಾದ
ಶ್ರೀಪಾ-ರ್ಥ-ಸಾ-ರಥಿ
ಶ್ರೀಮಂತ
ಶ್ರೀಮಂ-ತನ
ಶ್ರೀಮಂ-ತ-ನಾ-ಗಲು
ಶ್ರೀಮಂ-ತ-ನಾ-ಗು-ತ್ತೇನೆ
ಶ್ರೀಮಂ-ತನು
ಶ್ರೀಮಂ-ತ-ನೊಬ್ಬ
ಶ್ರೀಮಂ-ತ-ನೊ-ಬ್ಬನ
ಶ್ರೀಮಂ-ತರ
ಶ್ರೀಮಂ-ತ-ರನ್ನು
ಶ್ರೀಮಂ-ತ-ರ-ಲ್ಲೊ-ಬ್ಬ-ನಾದ
ಶ್ರೀಮಂ-ತ-ರಿಂದ
ಶ್ರೀಮಂ-ತರು
ಶ್ರೀಮಂ-ತ-ರು-ವಿ-ದ್ಯಾ-ವಂ-ತ-ರಿಗೆ
ಶ್ರೀಮಂ-ತರೂ
ಶ್ರೀಮಂ-ತ-ರೆ-ನ್ನಿ-ಸಿ-ಕೊಂ-ಡ-ವ-ರಿಂದ
ಶ್ರೀಮಂ-ತರೋ
ಶ್ರೀಮಂ-ತ-ಳಾದ
ಶ್ರೀಮತಿ
ಶ್ರೀಮದ್
ಶ್ರೀಮಾತಾ
ಶ್ರೀಮಾತೆ
ಶ್ರೀಮಾ-ತೆ-ಯರ
ಶ್ರೀಮಾ-ತೆ-ಯ-ವರ
ಶ್ರೀಮಾ-ತೆ-ಯ-ವ-ರನ್ನು
ಶ್ರೀಮಾ-ತೆ-ಯ-ವ-ರಿಗೆ
ಶ್ರೀಮಾ-ತೆ-ಯ-ವ-ರಿ-ಗೊಂದು
ಶ್ರೀಮಾ-ತೆ-ಯ-ವರು
ಶ್ರೀಯುತ
ಶ್ರೀರಾಮ
ಶ್ರೀರಾ-ಮ-ಕೃಷ್ಣ
ಶ್ರೀರಾ-ಮ-ಕೃ-ಷ್ಣನ
ಶ್ರೀರಾ-ಮ-ಕೃ-ಷ್ಣ-ಪ-ರ-ಮ-ಹಂ-ಸ-ರನ್ನೂ
ಶ್ರೀರಾ-ಮ-ಕೃ-ಷ್ಣರ
ಶ್ರೀರಾ-ಮ-ಕೃ-ಷ್ಣ-ರನ್ನು
ಶ್ರೀರಾ-ಮ-ಕೃ-ಷ್ಣ-ರನ್ನೂ
ಶ್ರೀರಾ-ಮ-ಕೃ-ಷ್ಣ-ರನ್ನೇ
ಶ್ರೀರಾ-ಮ-ಕೃ-ಷ್ಣ-ರ-ಲ್ಲದೆ
ಶ್ರೀರಾ-ಮ-ಕೃ-ಷ್ಣ-ರಲ್ಲಿ
ಶ್ರೀರಾ-ಮ-ಕೃ-ಷ್ಣ-ರಾ-ದರೂ
ಶ್ರೀರಾ-ಮ-ಕೃ-ಷ್ಣ-ರಿಂದ
ಶ್ರೀರಾ-ಮ-ಕೃ-ಷ್ಣ-ರಿಗೆ
ಶ್ರೀರಾ-ಮ-ಕೃ-ಷ್ಣರು
ಶ್ರೀರಾ-ಮ-ಕೃ-ಷ್ಣ-ರೆಂ-ದರೆ
ಶ್ರೀರಾ-ಮ-ಕೃ-ಷ್ಣರೇ
ಶ್ರೀರಾ-ಮ-ಕೃ-ಷ್ಣ-ರೊಂ-ದಿ-ಗಿದ್ದು
ಶ್ರೀರಾ-ಮ-ಕೃ-ಷ್ಣ-ರೊಂ-ದಿ-ಗಿನ
ಶ್ರೀರಾ-ಮ-ಕೃ-ಷ್ಣಾ-ಶ್ರಮ
ಶ್ರೀರಾ-ಮ-ಚಂದ್ರ
ಶ್ರೀರಾ-ಮ-ಚಂ-ದ್ರನು
ಶ್ರೀರಾ-ಮನ
ಶ್ರೀರಾ-ಮ-ನಂ-ಥ-ವರು
ಶ್ರೀರಾ-ಮೇ-ಶ್ವರ
ಶ್ರೀಲಂ-ಕಾದ
ಶ್ರೀಶಂ-ಕರಾ
ಶ್ರೀಶಂ-ಕ-ರಾ-ಚ-ರ್ಯ-ರಿಗೂ
ಶ್ರೀಶಂ-ಕ-ರಾ-ಚಾ-ರ್ಯ-ರಿಂದ
ಶ್ರೀಶಾ-ರ-ದಾ-ದೇವಿ
ಶ್ರೀಶಾ-ರ-ದಾ-ದೇ-ವಿ-ಯರು
ಶ್ರೀಶಾ-ರ-ದಾ-ದೇ-ವಿ-ಯ-ವರ
ಶ್ರೀಶಾ-ರ-ದಾ-ದೇ-ವಿ-ಯ-ವ-ರನ್ನು
ಶ್ರೀಶಾ-ರ-ದಾ-ದೇ-ವಿ-ಯ-ವ-ರಿ-ಗಾಗಿ
ಶ್ರೀಶಾ-ರ-ದಾ-ದೇ-ವಿ-ಯ-ವರು
ಶ್ರೀಶಾ-ರ-ದಾ-ದೇವೀ
ಶ್ರೀಸಾ-ಮಾ-ನ್ಯ-ರೊಂ-ದಿಗೂ
ಶ್ರೇಣಿ-ಯಲ್ಲಿ
ಶ್ರೇಯ
ಶ್ರೇಯ-ಸ್ಸಿ-ಗಾಗಿ
ಶ್ರೇಯಸ್ಸು
ಶ್ರೇಯಾಂಕ
ಶ್ರೇಯಾಂ-ಕ-ಗಳು
ಶ್ರೇಷ್ಠ
ಶ್ರೇಷ್ಠ-ಉ-ನ್ನತ
ಶ್ರೇಷ್ಠ-ಕ-ಲಾ-ಕೃ-ತಿ-ಗಳು
ಶ್ರೇಷ್ಠ-ಕೃ-ತಿ-ಗಳ
ಶ್ರೇಷ್ಠ-ತಮ
ಶ್ರೇಷ್ಠ-ತ-ಮ-ರಾದ
ಶ್ರೇಷ್ಠ-ತ-ಮ-ರಾ-ದ-ವರು
ಶ್ರೇಷ್ಠ-ತ-ಮ-ವಾದ
ಶ್ರೇಷ್ಠತೆ
ಶ್ರೇಷ್ಠ-ತೆ-ಯನ್ನು
ಶ್ರೇಷ್ಠ-ರಾದ
ಶ್ರೇಷ್ಠ-ರಾ-ದ-ವರು
ಶ್ರೇಷ್ಠ-ವ-ರ್ಗದ
ಶ್ರೇಷ್ಠ-ವಾದ
ಶ್ರೇಷ್ಠ-ವಾ-ದ-ದ್ದಾ-ಗಿ-ದ್ದರೂ
ಶ್ರೇಷ್ಠ-ವಾ-ದ್ದನ್ನೇ
ಶ್ರೇಷ್ಠವೂ
ಶ್ರೇಷ್ಠ-ವೆಂ-ಬು-ದನ್ನು
ಶ್ರೋತೃ
ಶ್ರೋತೃ-ಗಳ
ಶ್ರೋತೃ-ಗಳನ್ನು
ಶ್ರೋತೃ-ಗಳಲ್ಲಿ
ಶ್ರೋತೃ-ಗಳಿಂದ
ಶ್ರೋತೃ-ಗ-ಳಿ-ಗಿಂತ
ಶ್ರೋತೃ-ಗ-ಳಿಗೂ
ಶ್ರೋತೃ-ಗ-ಳಿಗೆ
ಶ್ರೋತೃ-ಗ-ಳಿಗೇ
ಶ್ರೋತೃ-ಗಳು
ಶ್ರೋತೃ-ಗ-ಳೊ-ಳಗೂ
ಶ್ರೋತೃ-ವರ್ಗ
ಶ್ರೋತೃ-ವ-ರ್ಗದ
ಶ್ರೋತೃ-ವ-ರ್ಗ-ವನ್ನು
ಶ್ಲಾಘ-ನೆಯ
ಶ್ಲಾಘಿ-ಸಿ-ದ್ದರು
ಶ್ಲಾಘ್ಯ-ವಾದ
ಶ್ಲೋಕ-ಗಳ
ಶ್ಲೋಕ-ಗಳನ್ನು
ಶ್ಲೋಕ-ವನ್ನು
ಶ್ಲೋಕ-ವೊಂ-ದನ್ನು
ಶ್ಲೋಕ-ವೊಂ-ದ-ರಿಂದ
ಶ್ವಾಸ
ಶ್ವಾಸೋ-ಚ್ಛ್ವಾಸ
ಷಂಡ-ತನ
ಷಂಡ-ತ-ನ-ವನ್ನು
ಷಡ್ಯಂ-ತ್ರ-ಗಳನ್ನು
ಷಡ್ರ-ಸೋ-ಪೇತ
ಷಮೋ-ನಿ-ಕ್ಸ್
ಷಮೋ-ನಿ-ಕ್ಸ್ನಿಂದ
ಷರ-ತ್ತಿನ
ಷರತ್ತು
ಷರಾ-ಯಿಯ
ಷರು
ಷಿಯಾ-ಗಳು
ಷೆಡ್ಡಿ-ನಲ್ಲಿ
ಷೇಕ್ಸ್ಪಿ-ಯರ್
ಷ್ಟಕ್ಕೆ
ಷ್ಠಾನ
ಷ್ಠಾಪಿಸಿ
ಸಂಕಟ
ಸಂಕ-ಟಕ್ಕೆ
ಸಂಕ-ಟ-ಗಳ
ಸಂಕ-ಟ-ಗಳನ್ನು
ಸಂಕ-ಟ-ಗಳನ್ನೂ
ಸಂಕ-ಟ-ಗ-ಳಿ-ಗೂ-ಸಿ-ದ್ಧ-ನಾ-ಗಿಯೇ
ಸಂಕ-ಟ-ಗಳೇ
ಸಂಕ-ಟ-ದಿಂದ
ಸಂಕ-ಟ-ದಿಂ-ದಾಗಿ
ಸಂಕ-ಟ-ಪ-ಡು-ವುದು
ಸಂಕ-ಟ-ವನ್ನು
ಸಂಕ-ಟ-ವಾ-ಗಿತ್ತು
ಸಂಕ-ಟ-ವಾ-ಗಿ-ದ್ದರೂ
ಸಂಕ-ಟ-ವಾ-ಗು-ತ್ತಿದೆ
ಸಂಕ-ಟ-ವಾ-ಯಿತು
ಸಂಕಲ್ಪ
ಸಂಕ-ಲ್ಪ-ಶ-ಕ್ತಿಯ
ಸಂಕ-ಷ್ಟಕ್ಕೆ
ಸಂಕ-ಷ್ಟ-ಗಳ
ಸಂಕೀ-ರ್ತನೆ
ಸಂಕೀ-ರ್ತ-ನೆ-ಯಲ್ಲಿ
ಸಂಕು-ಚಿತ
ಸಂಕು-ಚಿ-ತ-ಗೊಂ-ಡಿ-ರ-ಬೇ-ಕೆಂದು
ಸಂಕು-ಚಿ-ತ-ತೆ-ಯನ್ನೂ
ಸಂಕು-ಚಿ-ತ-ವಾ-ಗಿದೆ
ಸಂಕೇ-ತ-ವಾ-ಗಿ-ರ-ಬ-ಹುದೆ
ಸಂಕೇ-ತ-ವಾದ
ಸಂಕೋಚ
ಸಂಕೋ-ಚ-ದಿಂದ
ಸಂಕೋ-ಚವೂ
ಸಂಕೋ-ಚ-ವೆ-ನಿ-ಸು-ತ್ತಿತ್ತು
ಸಂಕೋ-ಲೆ-ಯಲ್ಲಿ
ಸಂಕೋ-ಲೆ-ಯಾ-ಗಿತ್ತು
ಸಂಕೋ-ಲೆ-ಯಿಂದ
ಸಂಕ್ರ-ಮಣ
ಸಂಕ್ಷಿ-ಪ್ತ-ವಾಗಿ
ಸಂಕ್ಷೇ-ಪ-ವಾಗಿ
ಸಂಖ್ಯೆ
ಸಂಖ್ಯೆ-ಗಿಂತ
ಸಂಖ್ಯೆಯ
ಸಂಖ್ಯೆ-ಯಲ್ಲಿ
ಸಂಖ್ಯೆ-ಯ-ಲ್ಲಿದ್ದ
ಸಂಗ
ಸಂಗ-ಡಿಗ
ಸಂಗ-ಡಿ-ಗ-ನೊಂ-ದಿಗೆ
ಸಂಗ-ಡಿ-ಗರ
ಸಂಗ-ಡಿ-ಗ-ರಿಗೂ
ಸಂಗ-ಡಿ-ಗ-ರಿಗೆ
ಸಂಗ-ಡಿ-ಗರು
ಸಂಗ-ಡಿ-ಗ-ರೊಂ-ದಿಗೆ
ಸಂಗತಿ
ಸಂಗ-ತಿ-ಗಳ
ಸಂಗ-ತಿ-ಗಳು
ಸಂಗ-ತಿ-ಯನ್ನು
ಸಂಗ-ತಿ-ಯಲ್ಲ
ಸಂಗ-ತಿ-ಯಾ-ಗಿತ್ತು
ಸಂಗ-ತಿ-ಯೆಂ-ದರೆ
ಸಂಗ-ತಿಯೇ
ಸಂಗ-ತಿ-ಯೇ-ನೆಂ-ದರೆ
ಸಂಗ-ತಿ-ಯೊಂ-ದನ್ನು
ಸಂಗಾತಿ
ಸಂಗಾ-ತಿ-ಗ-ಳೊಂ-ದಿಗೆ
ಸಂಗಾ-ತಿ-ಯಾಗಿ
ಸಂಗಾ-ತಿ-ಯೊಂ-ದಿಗೆ
ಸಂಗೀತ
ಸಂಗೀ-ತ
ಸಂಗೀ-ತ-ನೃ-ತ್ಯ
ಸಂಗೀ-ತ-ಇ-ವು-ಗಳು
ಸಂಗೀ-ತ-ಗಾ-ರ-ನೊ-ಬ್ಬ-ನನ್ನು
ಸಂಗೀ-ತ-ಗಾ-ರ-ರನ್ನೂ
ಸಂಗೀ-ತ-ಗಾ-ರರು
ಸಂಗೀ-ತ-ಗಾರ್ತಿ
ಸಂಗೀ-ತದ
ಸಂಗೀ-ತ-ದಂತೆ
ಸಂಗೀ-ತ-ನಾ-ಟಕ
ಸಂಗೀ-ತ-ಪ್ರ-ಧಾನ
ಸಂಗೀ-ತ-ಮಯ
ಸಂಗೀ-ತವೂ
ಸಂಗ್ರಹ
ಸಂಗ್ರ-ಹಣೆ
ಸಂಗ್ರ-ಹ-ಣೆಯ
ಸಂಗ್ರ-ಹ-ಣೆ-ಯಲ್ಲ
ಸಂಗ್ರ-ಹ-ಣೆ-ಯ-ಲ್ಲವೆ
ಸಂಗ್ರ-ಹ-ಯೋ-ಗ್ಯ-ವಾದ
ಸಂಗ್ರ-ಹ-ವಾಗಿ
ಸಂಗ್ರ-ಹ-ವಾ-ಗು-ತ್ತದೆ
ಸಂಗ್ರ-ಹ-ವಾ-ಗು-ತ್ತಿದ್ದ
ಸಂಗ್ರ-ಹ-ವಾದ
ಸಂಗ್ರ-ಹ-ವಾ-ದರೆ
ಸಂಗ್ರ-ಹ-ವಾ-ಯಿತು
ಸಂಗ್ರ-ಹವೇ
ಸಂಗ್ರಹಿ
ಸಂಗ್ರ-ಹಿ-ಸ-ಲಿಲ್ಲ
ಸಂಗ್ರ-ಹಿ-ಸಲು
ಸಂಗ್ರ-ಹಿ-ಸ-ಲೆಂದು
ಸಂಗ್ರ-ಹಿಸಿ
ಸಂಗ್ರ-ಹಿ-ಸಿ-ಕೊಂಡು
ಸಂಗ್ರ-ಹಿ-ಸಿದ
ಸಂಗ್ರ-ಹಿ-ಸಿ-ದರು
ಸಂಗ್ರ-ಹಿ-ಸಿ-ದ-ವರು
ಸಂಗ್ರ-ಹಿ-ಸಿದ್ದ
ಸಂಗ್ರ-ಹಿ-ಸಿ-ರುವ
ಸಂಗ್ರ-ಹಿಸು
ಸಂಗ್ರ-ಹಿ-ಸು-ತ್ತಿದ್ದ
ಸಂಗ್ರ-ಹಿ-ಸು-ತ್ತಿ-ದ್ದಳು
ಸಂಗ್ರ-ಹಿ-ಸು-ವಂತೆ
ಸಂಗ್ರ-ಹಿ-ಸು-ವುದು
ಸಂಘ
ಸಂಘ-ಸಂ-ಸ್ಥೆ-ಗಳನ್ನು
ಸಂಘ-ಸಂ-ಸ್ಥೆಯ
ಸಂಘ-ಒಂದು
ಸಂಘಕ್ಕೆ
ಸಂಘ-ಗಳ
ಸಂಘ-ಗಳನ್ನು
ಸಂಘ-ಗಳಲ್ಲಿ
ಸಂಘ-ಗಳು
ಸಂಘ-ಜೀ-ವ-ನ-ದಲ್ಲಿ
ಸಂಘ-ಟ-ಕ-ರ-ಲ್ಲೊ-ಬ್ಬರೂ
ಸಂಘ-ಟ-ಕ-ರಿಂದ
ಸಂಘ-ಟ-ಕರು
ಸಂಘ-ಟನಾ
ಸಂಘ-ಟನೆ
ಸಂಘ-ಟ-ನೆಯ
ಸಂಘ-ಟ-ನೆ-ಯಲ್ಲಿ
ಸಂಘ-ಟ-ನೆಯೇ
ಸಂಘ-ಟಿ-ತ-ರಾಗಿ
ಸಂಘ-ಟಿ-ತ-ವಾ-ಗಿಲ್ಲ
ಸಂಘ-ಟಿ-ಸಲು
ಸಂಘ-ಟಿ-ಸ-ಲೆಂದೇ
ಸಂಘ-ಟಿ-ಸಿ-ದರು
ಸಂಘದ
ಸಂಘ-ದಂ-ತೆಯೇ
ಸಂಘ-ದಲ್ಲಿ
ಸಂಘ-ದ-ವರು
ಸಂಘ-ವಂತೂ
ಸಂಘ-ವನ್ನು
ಸಂಘ-ವಾಗಿ
ಸಂಘ-ವಾದ
ಸಂಘವು
ಸಂಘ-ವೆಂದ
ಸಂಘ-ವೊಂ-ದನ್ನು
ಸಂಘ-ವೊಂ-ದ-ರಲ್ಲಿ
ಸಂಘ-ವೊಂದು
ಸಂಘ-ಸಂಸ್ಥೆ
ಸಂಚ-ರಿ-ಸ-ಬ-ಹು-ದೆಂದು
ಸಂಚ-ರಿಸಿ
ಸಂಚ-ರಿ-ಸಿದ
ಸಂಚ-ರಿ-ಸಿ-ದರು
ಸಂಚ-ರಿ-ಸಿ-ದ-ವ-ರ-ಲ್ಲವೆ
ಸಂಚ-ರಿ-ಸಿ-ದ್ದೇನೆ
ಸಂಚ-ರಿ-ಸುತ್ತ
ಸಂಚ-ರಿ-ಸು-ತ್ತಿದ್ದ
ಸಂಚ-ರಿ-ಸು-ತ್ತಿ-ದ್ದಾಗ
ಸಂಚ-ರಿ-ಸು-ತ್ತಿ-ದ್ದೇನೆ
ಸಂಚ-ರಿ-ಸು-ತ್ತಿ-ರ-ಬೇಕು
ಸಂಚ-ರಿ-ಸು-ವಂ-ತಾ-ಗ-ಬೇಕು
ಸಂಚ-ರಿ-ಸು-ವುದೇ
ಸಂಚಾರ
ಸಂಚಾ-ರ-ಗೊ-ಳಿ-ಸಿ-ದ್ದಂತೆ
ಸಂಚಾ-ರದ
ಸಂಚಾ-ರ-ವನ್ನೇ
ಸಂಚಾ-ರ-ವಾ-ಗು-ತ್ತಿದೆ
ಸಂಚಾ-ರ-ವಾ-ಗು-ವಂತೆ
ಸಂಚಾ-ರವೂ
ಸಂಚಾ-ಲ-ಕ-ರಾದ
ಸಂಚಾ-ಲ-ಕ-ರಿಗೆ
ಸಂಚಿ-ಕೆ-ಗಳನ್ನು
ಸಂಚಿ-ಕೆ-ಗಳಲ್ಲಿ
ಸಂಚಿ-ಕೆ-ಯಲ್ಲಿ
ಸಂಜೆ
ಸಂಜೆ-ಗ-ತ್ತಲು
ಸಂಜೆಯ
ಸಂಜೆ-ಯ-ವ-ರೆಗೂ
ಸಂಜೆ-ಯಾ-ಗು-ತ್ತಿ-ದ್ದಂತೆ
ಸಂಜೆ-ಯಾ-ದರೂ
ಸಂಜೆ-ಯಾ-ಯಿತು
ಸಂಜೆಯೂ
ಸಂತ
ಸಂತ-ದೇ-ಶ-ಭ-ಕ್ತ-ರಾ-ದರು
ಸಂತ-ನ-ಲ್ಲವೆ
ಸಂತ-ನಲ್ಲಿ
ಸಂತನೂ
ಸಂತ-ನೊ-ಬ್ಬನ
ಸಂತ-ಯಾನ
ಸಂತ-ರನ್ನು
ಸಂತ-ರಾ-ಗುವ
ಸಂತ-ರಿಗೆ
ಸಂತರೇ
ಸಂತಸ
ಸಂತ-ಸ-ಗೊಂಡು
ಸಂತ-ಸದ
ಸಂತ-ಸ-ವನ್ನು
ಸಂತ-ಸ-ವುಂ-ಟು-ಮಾ-ಡಿತು
ಸಂತಾನ
ಸಂತು-ಷ್ಟ-ನಾಗಿ
ಸಂತು-ಷ್ಟ-ರಾಗಿ
ಸಂತು-ಷ್ಟಿಯ
ಸಂತೃ-ಪ್ತ-ನಾದ
ಸಂತೃಪ್ತಿ
ಸಂತೃ-ಪ್ತಿ-ಭಾ-ವ-ವನ್ನು
ಸಂತೃ-ಪ್ತಿ-ಯಿಂದ
ಸಂತೆ-ಯಲ್ಲಿ
ಸಂತೈ-ಸಿ-ದರು
ಸಂತೈ-ಸುವ
ಸಂತೈ-ಸು-ವುದೇ
ಸಂತೋಷ
ಸಂತೋ-ಷ-ಸ-ಮಾ-ಧಾ-ನ-ಗ-ಳ-ನ್ನುಂ-ಟು-ಮಾ-ಡಿ-ದುವು
ಸಂತೋ-ಷ-ಸ-ಮಾ-ಧಾ-ನ-ಗ-ಳುಂ-ಟಾ-ದವು
ಸಂತೋ-ಷ-ಕ-ರ-ವಾದ
ಸಂತೋ-ಷ-ಕೂಟ
ಸಂತೋ-ಷ-ಕೂ-ಟ-ದಲ್ಲಿ
ಸಂತೋ-ಷ-ಕೂ-ಟ-ದಲ್ಲೂ
ಸಂತೋ-ಷ-ಕೂ-ಟ-ವೊಂ-ದನ್ನು
ಸಂತೋ-ಷ-ಗಳನ್ನು
ಸಂತೋ-ಷ-ಗೊಂಡ
ಸಂತೋ-ಷ-ಗೊಂ-ಡರು
ಸಂತೋ-ಷ-ಗೊಂ-ಡಿ-ದ್ದರು
ಸಂತೋ-ಷ-ಗೊಂ-ಡಿ-ದ್ದೇನೆ
ಸಂತೋ-ಷ-ಗೊ-ಳಿ-ಸಿದೆ
ಸಂತೋ-ಷ-ದಿಂದ
ಸಂತೋ-ಷ-ದಿಂ-ದಿ-ರು-ತ್ತಾರೋ
ಸಂತೋ-ಷ-ಪ-ಟ್ಟರು
ಸಂತೋ-ಷ-ಪ-ಟ್ಟ-ವರು
ಸಂತೋ-ಷ-ಪ-ಟ್ಟು-ಕೊಳ್ಳು
ಸಂತೋ-ಷ-ಪ-ಟ್ಟು-ಕೊ-ಳ್ಳು-ತ್ತೇನೆ
ಸಂತೋ-ಷ-ಪ-ಡಿ-ಸಿ-ಕೊಂಡು
ಸಂತೋ-ಷ-ಪಡು
ಸಂತೋ-ಷ-ಪ-ಡು-ತ್ತಾರೆ
ಸಂತೋ-ಷ-ಪ-ಡು-ತ್ತಿ-ದ್ದರು
ಸಂತೋ-ಷ-ವನ್ನು
ಸಂತೋ-ಷ-ವ-ನ್ನುಂಟು
ಸಂತೋ-ಷ-ವ-ನ್ನುಂ-ಟು-ಮಾ-ಡಿ-ದುವು
ಸಂತೋ-ಷ-ವನ್ನೂ
ಸಂತೋ-ಷ-ವಾ-ಗ-ಲಿಲ್ಲ
ಸಂತೋ-ಷ-ವಾಗಿ
ಸಂತೋ-ಷ-ವಾ-ಗಿ-ಡಲು
ಸಂತೋ-ಷ-ವಾ-ಗಿದೆ
ಸಂತೋ-ಷ-ವಾ-ಗಿ-ರ-ಬೇಕು
ಸಂತೋ-ಷ-ವಾಗು
ಸಂತೋ-ಷ-ವಾ-ಯಿತು
ಸಂತೋ-ಷ-ವುಂಟು
ಸಂತೋ-ಷ-ವುಂ-ಟು-ಮಾ-ಡಿತ್ತು
ಸಂತೋ-ಷ-ವುಂ-ಟು-ಮಾ-ಡಿದೆ
ಸಂತೋ-ಷ-ವುಂ-ಟು-ಮಾ-ಡುವ
ಸಂತೋ-ಷವೂ
ಸಂತೋ-ಷ-ವೆ-ನಿ-ಸು-ತ್ತದೆ
ಸಂತೋ-ಷವೇ
ಸಂದ
ಸಂದಂತೆ
ಸಂದ-ಣಿಯೂ
ಸಂದದ್ದು
ಸಂದದ್ದೋ
ಸಂದರ್ಭ
ಸಂದ-ರ್ಭಕ್ಕೆ
ಸಂದ-ರ್ಭ-ಗಳಲ್ಲಿ
ಸಂದ-ರ್ಭದ
ಸಂದ-ರ್ಭ-ದಲ್ಲಿ
ಸಂದ-ರ್ಭ-ದ-ಲ್ಲಿಯೂ
ಸಂದ-ರ್ಭ-ದಲ್ಲೇ
ಸಂದ-ರ್ಭ-ದ-ಲ್ಲೊಂದು
ಸಂದ-ರ್ಭ-ವನ್ನು
ಸಂದ-ರ್ಭವು
ಸಂದ-ರ್ಭ-ವುಂ-ಟಾ-ದಾ-ಗ-ಲೆಲ್ಲ
ಸಂದ-ರ್ಭವೂ
ಸಂದ-ರ್ಭ-ವೊ-ದಗಿ
ಸಂದ-ರ್ಭ-ವೊ-ದ-ಗಿತ್ತು
ಸಂದ-ರ್ಭ-ವೊ-ದ-ಗಿ-ದಾಗ
ಸಂದ-ರ್ಭ-ವೊ-ದ-ಗಿ-ಬಂ-ದಾ-ಗ-ಲೆಲ್ಲ
ಸಂದ-ರ್ಭ-ವೊ-ದ-ಗು-ತ್ತಿತ್ತು
ಸಂದ-ರ್ಭಾ-ನು-ಸಾ-ರ-ವಾಗಿ
ಸಂದ-ರ್ಶ-ಕರು
ಸಂದ-ರ್ಶನ
ಸಂದ-ರ್ಶ-ನಕ್ಕೆ
ಸಂದ-ರ್ಶ-ನ-ಕ್ಕೆಂದು
ಸಂದ-ರ್ಶ-ನ-ಗಳನ್ನು
ಸಂದ-ರ್ಶ-ನ-ಗಳೂ
ಸಂದ-ರ್ಶ-ನದ
ಸಂದ-ರ್ಶ-ನ-ವನ್ನು
ಸಂದ-ರ್ಶ-ನ-ವಾ-ಗು-ತ್ತದೆ
ಸಂದ-ರ್ಶ-ನ-ವೊಂ-ದ-ರಲ್ಲಿ
ಸಂದರ್ಶಿ
ಸಂದ-ರ್ಶಿಸ
ಸಂದ-ರ್ಶಿ-ಸ-ಬೇ-ಕೆಂಬ
ಸಂದ-ರ್ಶಿ-ಸ-ಲಾ-ರಂ-ಭಿ-ಸಿ-ದರು
ಸಂದ-ರ್ಶಿ-ಸಲು
ಸಂದ-ರ್ಶಿ-ಸ-ಲೆಂದು
ಸಂದ-ರ್ಶಿಸಿ
ಸಂದ-ರ್ಶಿ-ಸಿದ
ಸಂದ-ರ್ಶಿ-ಸಿ-ದ-ರ-ಲ್ಲದೆ
ಸಂದ-ರ್ಶಿ-ಸಿ-ದರು
ಸಂದ-ರ್ಶಿ-ಸಿದ್ದ
ಸಂದ-ರ್ಶಿ-ಸಿದ್ದು
ಸಂದ-ರ್ಶಿ-ಸಿ-ಯಾ-ಗಿತ್ತು
ಸಂದ-ರ್ಶಿ-ಸಿ-ರ-ಲಿ-ಲ್ಲವೋ
ಸಂದ-ರ್ಶಿ-ಸುತ್ತ
ಸಂದ-ರ್ಶಿ-ಸುವ
ಸಂದ-ರ್ಶಿ-ಸು-ವುದು
ಸಂದಿಗ್ಧ
ಸಂದಿ-ತಲ್ಲ
ಸಂದಿದೆ
ಸಂದಿ-ದ್ದುವು
ಸಂದಿ-ಯಲ್ಲಿ
ಸಂದಿವೆ
ಸಂದೇಶ
ಸಂದೇ-ಶ-ಗಳ
ಸಂದೇ-ಶ-ಗಳನ್ನು
ಸಂದೇ-ಶ-ಗಳನ್ನೂ
ಸಂದೇ-ಶ-ಗಳಲ್ಲಿ
ಸಂದೇ-ಶ-ಗ-ಳಿಂ-ದಲೂ
ಸಂದೇ-ಶ-ಗಳು
ಸಂದೇ-ಶದ
ಸಂದೇ-ಶ-ದೊಂ-ದಿಗೆ
ಸಂದೇ-ಶ-ಪ್ರ-ಸಾರ
ಸಂದೇ-ಶ-ಪ್ರ-ಸಾ-ರದ
ಸಂದೇ-ಶ-ವ-ನ್ನಾಗಿ
ಸಂದೇ-ಶ-ವನ್ನು
ಸಂದೇ-ಶ-ವನ್ನೂ
ಸಂದೇ-ಶ-ವನ್ನೇ
ಸಂದೇ-ಶ-ವಿದು
ಸಂದೇ-ಶ-ವಿದೆ
ಸಂದೇ-ಶವು
ಸಂದೇ-ಶ-ವೆಂದರೆ
ಸಂದೇ-ಶವೇ
ಸಂದೇ-ಹ-ಕ್ಕೆ-ಡೆ-ಯಿಲ್ಲ
ಸಂದೇ-ಹ-ಕ್ಕೆ-ಡೆ-ಯಿ-ಲ್ಲ-ದಂತೆ
ಸಂದೇ-ಹ-ಗಳನ್ನೂ
ಸಂದೇ-ಹ-ಗಳನ್ನೆಲ್ಲ
ಸಂದೇ-ಹ-ಗ-ಳಿಗೆ
ಸಂದೇ-ಹದ
ಸಂದೇ-ಹ-ವಾದಿ
ಸಂದೇ-ಹ-ವಿ-ರ-ಲಿಲ್ಲ
ಸಂದೇ-ಹ-ವಿಲ್ಲ
ಸಂದೇ-ಹವೇ
ಸಂದೇ-ಹಾ-ಸ್ಪದ
ಸಂಧಿ-ಕಾಲ
ಸಂಧಿ-ಕಾ-ಲದ
ಸಂಧಿ-ಸದೆ
ಸಂಧಿಸಿ
ಸಂಧಿ-ಸಿದ
ಸಂಧಿ-ಸಿ-ದರು
ಸಂಧಿ-ಸಿ-ದಾಗ
ಸಂಧಿ-ಸಿದೆ
ಸಂಧಿ-ಸಿ-ದ್ದರು
ಸಂಧಿ-ಸಿದ್ದು
ಸಂಧಿ-ಸುವ
ಸಂಧ್ಯಾ-ಕಾ-ಲದ
ಸಂಧ್ಯಾ-ವಂ-ದ-ನಾದಿ
ಸಂಧ್ಯಾ-ವಂ-ದ-ನೆಗೆ
ಸಂಧ್ಯಾ-ವಂ-ದ-ನೆಯ
ಸಂಧ್ಯಾ-ವಂ-ದ-ನೆ-ಯನ್ನು
ಸಂಧ್ಯೆ
ಸಂನ್ಯಾಸ
ಸಂನ್ಯಾ-ಸ-ಕ್ಕಾಗಿ
ಸಂನ್ಯಾ-ಸ-ಜೀ-ವ-ನದ
ಸಂನ್ಯಾ-ಸದ
ಸಂನ್ಯಾ-ಸ-ದೀಕ್ಷೆ
ಸಂನ್ಯಾ-ಸ-ಧರ್ಮ
ಸಂನ್ಯಾ-ಸ-ಧ-ರ್ಮ-ಕ್ಕನು
ಸಂನ್ಯಾ-ಸ-ಧ-ರ್ಮಕ್ಕೆ
ಸಂನ್ಯಾ-ಸ-ಧ-ರ್ಮಕ್ಕೇ
ಸಂನ್ಯಾ-ಸ-ಧ-ರ್ಮ-ವನ್ನೂ
ಸಂನ್ಯಾ-ಸ-ವ-ನ್ನ-ರಸಿ
ಸಂನ್ಯಾ-ಸ-ವನ್ನು
ಸಂನ್ಯಾ-ಸ-ವ್ರ-ತ-ವನ್ನು
ಸಂನ್ಯಾಸಿ
ಸಂನ್ಯಾ-ಸಿ-ಗಳ
ಸಂನ್ಯಾ-ಸಿ-ಗ-ಳಂ-ತ-ಲ್ಲದ
ಸಂನ್ಯಾ-ಸಿ-ಗ-ಳ-ನ್ನಾಗಿ
ಸಂನ್ಯಾ-ಸಿ-ಗಳನ್ನು
ಸಂನ್ಯಾ-ಸಿ-ಗ-ಳ-ಲ್ಲ-ವೆಂಬ
ಸಂನ್ಯಾ-ಸಿ-ಗಳಲ್ಲಿ
ಸಂನ್ಯಾ-ಸಿ-ಗ-ಳಲ್ಲೇ
ಸಂನ್ಯಾ-ಸಿ-ಗ-ಳ-ಲ್ಲೊ-ಬ್ಬ-ರನ್ನು
ಸಂನ್ಯಾ-ಸಿ-ಗ-ಳಾಗಿ
ಸಂನ್ಯಾ-ಸಿ-ಗ-ಳಾ-ಗಿ-ದ್ದರೆ
ಸಂನ್ಯಾ-ಸಿ-ಗ-ಳಾದ
ಸಂನ್ಯಾ-ಸಿ-ಗ-ಳಾ-ದ್ದ-ರಿಂದ
ಸಂನ್ಯಾ-ಸಿ-ಗಳಿಂದ
ಸಂನ್ಯಾ-ಸಿ-ಗ-ಳಿ-ಗಾ-ಗಿಯೇ
ಸಂನ್ಯಾ-ಸಿ-ಗ-ಳಿ-ಗಿಂತ
ಸಂನ್ಯಾ-ಸಿ-ಗ-ಳಿಗೆ
ಸಂನ್ಯಾ-ಸಿ-ಗ-ಳಿ-ರಲು
ಸಂನ್ಯಾ-ಸಿ-ಗಳು
ಸಂನ್ಯಾ-ಸಿ-ಗಳೂ
ಸಂನ್ಯಾ-ಸಿ-ಗ-ಳೆಂ-ದರೆ
ಸಂನ್ಯಾ-ಸಿ-ಗ-ಳೆಂದು
ಸಂನ್ಯಾ-ಸಿ-ಗ-ಳೆಲ್ಲ
ಸಂನ್ಯಾ-ಸಿ-ಗ-ಳೊಂ-ದಿಗೆ
ಸಂನ್ಯಾ-ಸಿ-ಗ-ಳೊ-ಬ್ಬರ
ಸಂನ್ಯಾ-ಸಿ-ಗ-ಳೊ-ಬ್ಬರು
ಸಂನ್ಯಾ-ಸಿ-ಗೀತೆ
ಸಂನ್ಯಾ-ಸಿಗೆ
ಸಂನ್ಯಾ-ಸಿನಿ
ಸಂನ್ಯಾ-ಸಿ-ನಿ-ಯರ
ಸಂನ್ಯಾ-ಸಿಯ
ಸಂನ್ಯಾ-ಸಿ-ಯಂ-ತಿ-ರು-ವಂತೆ
ಸಂನ್ಯಾ-ಸಿ-ಯಂತೆ
ಸಂನ್ಯಾ-ಸಿ-ಯನ್ನು
ಸಂನ್ಯಾ-ಸಿ-ಯ-ನ್ನು-ಬ್ರಿ-ಟಿಷ್
ಸಂನ್ಯಾ-ಸಿ-ಯಷ್ಟೆ
ಸಂನ್ಯಾ-ಸಿ-ಯಾಗಿ
ಸಂನ್ಯಾ-ಸಿ-ಯಾ-ಗಿದ್ದ
ಸಂನ್ಯಾ-ಸಿ-ಯಾ-ಗಿ-ದ್ದ-ರಂತೂ
ಸಂನ್ಯಾ-ಸಿ-ಯಾ-ಗಿ-ದ್ದರು
ಸಂನ್ಯಾ-ಸಿ-ಯಾ-ಗಿ-ದ್ದಾರೆ
ಸಂನ್ಯಾ-ಸಿ-ಯಾ-ಗಿದ್ದು
ಸಂನ್ಯಾ-ಸಿ-ಯಾ-ಗಿ-ರ-ಬ-ಹುದು
ಸಂನ್ಯಾ-ಸಿ-ಯಾಗು
ಸಂನ್ಯಾ-ಸಿ-ಯಾ-ಗು-ತ್ತೇನೆ
ಸಂನ್ಯಾ-ಸಿ-ಯಾ-ಗು-ವಂತೆ
ಸಂನ್ಯಾ-ಸಿ-ಯಾದ
ಸಂನ್ಯಾ-ಸಿ-ಯಾ-ದದ್ದೇ
ಸಂನ್ಯಾ-ಸಿ-ಯಾ-ದ-ವನು
ಸಂನ್ಯಾ-ಸಿ-ಯಾ-ದ-ವನೂ
ಸಂನ್ಯಾ-ಸಿ-ಯಿಂದ
ಸಂನ್ಯಾ-ಸಿಯು
ಸಂನ್ಯಾ-ಸಿಯೂ
ಸಂನ್ಯಾ-ಸಿ-ಯೆಂ-ದರಿ
ಸಂನ್ಯಾ-ಸಿ-ಯೆಂದು
ಸಂನ್ಯಾ-ಸಿ-ಯೆಂದೂ
ಸಂನ್ಯಾ-ಸಿ-ಯೆಂಬ
ಸಂನ್ಯಾ-ಸಿಯೇ
ಸಂನ್ಯಾ-ಸಿ-ಯೊಬ್ಬ
ಸಂನ್ಯಾ-ಸಿ-ಯೊ-ಬ್ಬನ
ಸಂನ್ಯಾ-ಸಿ-ಯೊ-ಬ್ಬ-ನಿದ್ದ
ಸಂನ್ಯಾ-ಸಿ-ಯೊ-ಬ್ಬನು
ಸಂನ್ಯಾ-ಸಿ-ಯೊ-ಬ್ಬ-ರನ್ನು
ಸಂನ್ಯಾ-ಸಿ-ಯೊ-ಬ್ಬ-ರಿಗೆ
ಸಂನ್ಯಾ-ಸಿ-ಯೊ-ಬ್ಬರು
ಸಂನ್ಯಾ-ಸಿ-ಯೋ-ರ್ವನ
ಸಂನ್ಯಾಸೀ
ಸಂನ್ಯಾ-ಸೀ-ಶಿ-ಷ್ಯ-ರಾದ
ಸಂನ್ಯಾ-ಸೀ-ಶಿ-ಷ್ಯರು
ಸಂಪ-ತ್ತನ್ನು
ಸಂಪ-ತ್ತಿ-ನ-ಲ್ಲಾ-ಗಲಿ
ಸಂಪ-ತ್ತಿ-ನೊಂ-ದಿಗೆ
ಸಂಪತ್ತು
ಸಂಪ-ದಾ-ಕ-ರಿಗೆ
ಸಂಪ-ದಾ-ಕೀಯ
ಸಂಪ-ದ್ಭ-ರಿತ
ಸಂಪ-ದ್ಭ-ರಿ-ತವೂ
ಸಂಪ-ನ್ನನೂ
ಸಂಪರ್ಕ
ಸಂಪ-ರ್ಕ-ಸ-ಹ-ವಾ-ಸ-ದಲ್ಲೇ
ಸಂಪ-ರ್ಕಕ್ಕೆ
ಸಂಪ-ರ್ಕದ
ಸಂಪ-ರ್ಕ-ದಲ್ಲಿ
ಸಂಪ-ರ್ಕ-ದಿಂದ
ಸಂಪ-ರ್ಕ-ವ-ನ್ನಿ-ಟ್ಟು-ಕೊಂ-ಡಿ-ದ್ದರು
ಸಂಪ-ರ್ಕ-ವನ್ನು
ಸಂಪ-ರ್ಕ-ವಿತ್ತು
ಸಂಪ-ರ್ಕಿಸಿ
ಸಂಪ-ರ್ಕಿ-ಸಿ-ದರು
ಸಂಪ-ರ್ಕಿ-ಸಿ-ದಾಗ
ಸಂಪ-ರ್ಕಿ-ಸಿದ್ದು
ಸಂಪಾದ
ಸಂಪಾ-ದಕ
ಸಂಪಾ-ದ-ಕನೂ
ಸಂಪಾ-ದ-ಕರ
ಸಂಪಾ-ದ-ಕ-ರಾ-ಗಿ-ದ್ದರು
ಸಂಪಾ-ದ-ಕ-ರಾ-ಗಿಯೂ
ಸಂಪಾ-ದ-ಕ-ರಾದ
ಸಂಪಾ-ದ-ಕರು
ಸಂಪಾ-ದ-ಕರೂ
ಸಂಪಾ-ದ-ಕೀಯ
ಸಂಪಾ-ದ-ಕೀ-ಯ-ಗಳನ್ನು
ಸಂಪಾ-ದ-ಕೀ-ಯ-ಗಳು
ಸಂಪಾ-ದ-ಕೀ-ಯದ
ಸಂಪಾ-ದನೆ
ಸಂಪಾ-ದ-ನೆ-ಯಾ-ದೀ-ತೆಂಬ
ಸಂಪಾ-ದಿ-ಸ-ಬ-ಹು-ದಾ-ಗಿತ್ತು
ಸಂಪಾ-ದಿ-ಸ-ಬ-ಹುದು
ಸಂಪಾ-ದಿ-ಸ-ಬೇ-ಕೆಂ-ದಿ-ದ್ದೇನೆ
ಸಂಪಾ-ದಿ-ಸಲು
ಸಂಪಾ-ದಿಸಿ
ಸಂಪಾ-ದಿ-ಸಿ-ಕೊಂ-ಡರು
ಸಂಪಾ-ದಿ-ಸಿ-ಕೊಂ-ಡಿ-ದ್ದರು
ಸಂಪಾ-ದಿ-ಸಿ-ಕೊಂ-ಡಿರಿ
ಸಂಪಾ-ದಿ-ಸಿ-ಕೊಂಡು
ಸಂಪಾ-ದಿ-ಸಿ-ಕೊ-ಳ್ಳು-ತ್ತದೆ
ಸಂಪಾ-ದಿ-ಸಿ-ಕೊ-ಳ್ಳು-ವತ್ತ
ಸಂಪಾ-ದಿ-ಸಿದ್ದ
ಸಂಪಾ-ದಿ-ಸು-ತ್ತಿ-ರುವ
ಸಂಪಾ-ದಿ-ಸು-ತ್ತೇನೆ
ಸಂಪಾ-ದಿ-ಸುವ
ಸಂಪುಟ
ಸಂಪು-ಟ-ಗಳ
ಸಂಪು-ಟ-ಗಳಲ್ಲಿ
ಸಂಪು-ಟ-ಗಳು
ಸಂಪು-ಟ-ದಲ್ಲಿ
ಸಂಪೂರ್ಣ
ಸಂಪೂ-ರ್ಣ-ವಾಗಿ
ಸಂಪ್ರ
ಸಂಪ್ರದಾ
ಸಂಪ್ರ-ದಾಯ
ಸಂಪ್ರ-ದಾ-ಯ-ನ-ಡ-ವ-ಳಿಕೆ
ಸಂಪ್ರ-ದಾ-ಯಕ್ಕೆ
ಸಂಪ್ರ-ದಾ-ಯ-ಗಳ
ಸಂಪ್ರ-ದಾ-ಯ-ಗಳನ್ನು
ಸಂಪ್ರ-ದಾ-ಯ-ಗಳನ್ನೂ
ಸಂಪ್ರ-ದಾ-ಯ-ಗಳಲ್ಲಿ
ಸಂಪ್ರ-ದಾ-ಯ-ಗ-ಳಿಗೆ
ಸಂಪ್ರ-ದಾ-ಯ-ಗಳು
ಸಂಪ್ರ-ದಾ-ಯದ
ಸಂಪ್ರ-ದಾ-ಯ-ದಂತೆ
ಸಂಪ್ರ-ದಾ-ಯ-ದಲ್ಲಿ
ಸಂಪ್ರ-ದಾ-ಯ-ನಿಷ್ಠ
ಸಂಪ್ರ-ದಾ-ಯ-ನಿ-ಷ್ಠ-ಳಾದ
ಸಂಪ್ರ-ದಾ-ಯ-ನಿ-ಷ್ಠವೂ
ಸಂಪ್ರ-ದಾ-ಯ-ಬದ್ಧ
ಸಂಪ್ರ-ದಾ-ಯ-ಬ-ದ್ಧ-ರ-ಲ್ಲದ
ಸಂಪ್ರ-ದಾ-ಯ-ವನ್ನು
ಸಂಪ್ರ-ದಾ-ಯಸ್ಥ
ಸಂಪ್ರ-ದಾ-ಯ-ಸ್ಥರ
ಸಂಪ್ರ-ದಾ-ಯ-ಸ್ಥ-ರನ್ನು
ಸಂಪ್ರ-ದಾ-ಯ-ಸ್ಥ-ರಾದ
ಸಂಪ್ರ-ದಾ-ಯ-ಸ್ಥ-ರಿಗೂ
ಸಂಪ್ರ-ದಾ-ಯ-ಸ್ಥರು
ಸಂಪ್ರ-ದಾ-ಯ-ಸ್ಥರೂ
ಸಂಪ್ರಾ-ಯ-ಗ-ಳಿಗೆ
ಸಂಪ್ರೀ-ತ-ರಾದ
ಸಂಬಂಧ
ಸಂಬಂ-ಧದ
ಸಂಬಂ-ಧ-ದಂತೆ
ಸಂಬಂ-ಧ-ಪಟ್ಟ
ಸಂಬಂ-ಧ-ಪ-ಟ್ಟಂತೆ
ಸಂಬಂ-ಧ-ವನ್ನು
ಸಂಬಂ-ಧ-ವನ್ನೂ
ಸಂಬಂ-ಧ-ವಾಗಿ
ಸಂಬಂ-ಧ-ವಿದೆ
ಸಂಬಂ-ಧ-ವಿಲ್ಲ
ಸಂಬಂ-ಧ-ವಿ-ಲ್ಲ-ದಿ-ದ್ದರೂ
ಸಂಬಂ-ಧವು
ಸಂಬಂ-ಧ-ವು-ಳ್ಳ-ವು-ಗ-ಳಾ-ಗಿವೆ
ಸಂಬಂ-ಧವೇ
ಸಂಬಂ-ಧ-ವೇ-ರ್ಪ-ಟ್ಟಿ-ರ-ಬ-ಹುದು
ಸಂಬಂ-ಧ-ಸ್ಪಂ-ದ-ನ-ಗ-ಳ-ನ್ನು-ಳ್ಳವು
ಸಂಬಂ-ಧಿ-ಕ-ರಿಂದ
ಸಂಬಂ-ಧಿ-ಕ-ರೆ-ಲ್ಲರೂ
ಸಂಬಂ-ಧಿ-ಯಾದ
ಸಂಬಂ-ಧಿಯೂ
ಸಂಬಂ-ಧಿಸಿ
ಸಂಬಂ-ಧಿ-ಸಿದ
ಸಂಬಂ-ಧಿ-ಸಿ-ದಂತೆ
ಸಂಬಂ-ಧಿ-ಸಿ-ದವ
ಸಂಬಂ-ಧಿ-ಸಿ-ದ-ವ-ರಲ್ಲೇ
ಸಂಬಂ-ಧಿ-ಸಿ-ದು-ದಲ್ಲ
ಸಂಬಂ-ಧಿ-ಸಿ-ದುದು
ಸಂಬಂ-ಧಿ-ಸಿ-ದುದೇ
ಸಂಬಳ
ಸಂಬ-ಳ-ಕ್ಕಾಗಿ
ಸಂಬ-ಳ-ಕ್ಕೋ-ಸ್ಕರ
ಸಂಬ-ಳದ
ಸಂಬ-ಳ-ದಲ್ಲಿ
ಸಂಬ-ಳ-ವನ್ನು
ಸಂಬ-ಳ-ವನ್ನೂ
ಸಂಬ-ಳ-ವನ್ನೇ
ಸಂಬೋ
ಸಂಬೋ-ಧನೆ
ಸಂಬೋ-ಧ-ನೆಯು
ಸಂಬೋ-ಧಿ-ಸ-ಲಾ-ಗಿತ್ತು
ಸಂಬೋ-ಧಿ-ಸಲು
ಸಂಬೋ-ಧಿ-ಸಿ-ದರು
ಸಂಬೋ-ಧಿ-ಸಿ-ದಾಗ
ಸಂಬೋ-ಧಿ-ಸಿ-ದ್ದ-ರ-ಷ್ಟ-ರಿಂ-ದಲೇ
ಸಂಬೋ-ಧಿ-ಸಿ-ದ್ದರು
ಸಂಬೋ-ಧಿ-ಸು-ತ್ತಿ-ದ್ದರು
ಸಂಬೋ-ಧಿ-ಸುವ
ಸಂಬೋ-ಧಿ-ಸು-ವಾಗ
ಸಂಭವ
ಸಂಭ-ವ-ವಿತ್ತು
ಸಂಭ-ವ-ವಿದೆ
ಸಂಭ-ವ-ವಿರು
ಸಂಭ-ವಿ-ಸ-ದಿ-ರು-ವಂತೆ
ಸಂಭ-ವಿ-ಸ-ಲಿ-ದೆ-ಯೆಂದು
ಸಂಭ-ವಿ-ಸ-ಲಿಲ್ಲ
ಸಂಭ-ವಿ-ಸಿ-ದಂ-ದಿ-ನಿಂದ
ಸಂಭ-ವಿ-ಸುವ
ಸಂಭ-ವಿ-ಸು-ವುದೋ
ಸಂಭಾ-ವ-ನೆಯ
ಸಂಭಾ-ವ-ನೆ-ಯನ್ನು
ಸಂಭಾ-ವಿತ
ಸಂಭಾ-ವಿ-ತ-ನಾಗಿ
ಸಂಭಾ-ಷಣ
ಸಂಭಾ-ಷಣೆ
ಸಂಭಾ-ಷ-ಣೆ-ಪ್ರ-ವ-ಚ-ನ-ಗಳನ್ನು
ಸಂಭಾ-ಷ-ಣೆ-ಗಳ
ಸಂಭಾ-ಷ-ಣೆ-ಗಳನ್ನು
ಸಂಭಾ-ಷ-ಣೆ-ಗಳಲ್ಲಿ
ಸಂಭಾ-ಷ-ಣೆ-ಗಳು
ಸಂಭಾ-ಷ-ಣೆಯ
ಸಂಭಾ-ಷ-ಣೆ-ಯ-ನ್ನಾ-ರಂ-ಭಿ-ಸಿ-ದರು
ಸಂಭಾ-ಷ-ಣೆ-ಯನ್ನು
ಸಂಭಾ-ಷ-ಣೆ-ಯಲ್ಲಿ
ಸಂಭಾ-ಷ-ಣೆ-ಯಿಂ-ದಲೇ
ಸಂಭಾ-ಷ-ಣೆಯು
ಸಂಭಾ-ಷ-ಣೆಯೇ
ಸಂಭಾಷಿ
ಸಂಭಾ-ಷಿ-ಸಲು
ಸಂಭಾ-ಷಿಸಿ
ಸಂಭಾ-ಷಿ-ಸಿ-ದರು
ಸಂಭಾ-ಷಿ-ಸಿ-ದ್ದೇನೋ
ಸಂಭಾ-ಷಿ-ಸು-ತ್ತಿದ್ದ
ಸಂಭಾ-ಷಿ-ಸು-ತ್ತಿ-ದ್ದರು
ಸಂಭಾ-ಷಿ-ಸು-ವಾಗ
ಸಂಭೂತ
ಸಂಭೋ-ಧಿ-ಸಿ-ದನೋ
ಸಂಭ್ರ-ಮದ
ಸಂಭ್ರ-ಮ-ದಿಂದ
ಸಂಮೋ-ಹ-ಕತೆ
ಸಂಮೋ-ಹ-ನ-ಗೊ-ಳಿ-ಸಿ-ಬಿ-ಡು-ತ್ತಿ-ದ್ದುವು
ಸಂಮೋ-ಹಿ-ನಿಗೆ
ಸಂಮೋ-ಹಿ-ನಿ-ಯನ್ನು
ಸಂಯ-ಮದ
ಸಂಯುಕ್ತ
ಸಂರ-ಕ್ಷ-ಣೆ-ಯ-ನ್ನೇಕೆ
ಸಂರ-ಕ್ಷಿ-ಸ-ಲ್ಪ-ಟ್ಟಿವೆ
ಸಂರ-ಕ್ಷಿ-ಸಿ-ಡ-ಲಾಗಿದೆ
ಸಂರ-ಕ್ಷಿ-ಸಿ-ರು-ವುದು
ಸಂರ್ಪಕ್ಕೆ
ಸಂವ-ತ್ಸರ
ಸಂವಿ-ಧಾ-ನ-ವನ್ನು
ಸಂಶಯ
ಸಂಶ-ಯ-ಗಳನ್ನು
ಸಂಶ-ಯ-ಗ-ಳೆಲ್ಲ
ಸಂಶ-ಯದ
ಸಂಶ-ಯ-ಪ-ಟ್ಟರು
ಸಂಶ-ಯ-ಪೂ-ರ್ಣ-ದ್ದಾ-ಗಿ-ರು-ತ್ತದೆ
ಸಂಶ-ಯ-ವಿ-ರ-ಲಿಲ್ಲ
ಸಂಶ-ಯ-ವಿಲ್ಲ
ಸಂಶ-ಯ-ವೇ-ಳ-ತೊಡ
ಸಂಶ-ಯಾ-ತ್ಮರ
ಸಂಶೋ-ಧ-ನಾ-ಕಾ-ರ್ಯ-ದಲ್ಲಿ
ಸಂಶೋ-ಧನೆ
ಸಂಶೋ-ಧ-ನೆ-ಗಳು
ಸಂಶೋ-ಧ-ನೆಯ
ಸಂಶೋ-ಧ-ನೆ-ಯನ್ನು
ಸಂಶೋ-ಧಿ-ತ-ವಾ-ದ-ದ್ದಾ-ಗಿ-ರು-ತ್ತದೆ
ಸಂಸಾರ
ಸಂಸಾ-ರವೇ
ಸಂಸಾ-ರ-ಸ್ಥರು
ಸಂಸ್ಕ-ರ-ಣ-ಗೊಂ-ಡದ್ದು
ಸಂಸ್ಕ-ರಿಸಿ
ಸಂಸ್ಕೃತ
ಸಂಸ್ಕೃ-ತ-ಇಂ-ಗ್ಲಿಷ್
ಸಂಸ್ಕೃ-ತ-ಇಂ-ಗ್ಲಿ-ಷ್ಗಳಲ್ಲಿ
ಸಂಸ್ಕೃ-ತಕ್ಕೆ
ಸಂಸ್ಕೃ-ತದ
ಸಂಸ್ಕೃ-ತ-ದಲ್ಲಿ
ಸಂಸ್ಕೃ-ತ-ದ-ಲ್ಲಿವೆ
ಸಂಸ್ಕೃ-ತ-ವನ್ನು
ಸಂಸ್ಕೃ-ತವು
ಸಂಸ್ಕೃ-ತಾ-ಭ್ಯಾ-ಸ-ವನ್ನು
ಸಂಸ್ಕೃತಿ
ಸಂಸ್ಕೃ-ತಿ
ಸಂಸ್ಕೃ-ತಿ-ಧರ್ಮ
ಸಂಸ್ಕೃ-ತಿ-ಸಂ-ಪ್ರ-ದಾ-ಯದ
ಸಂಸ್ಕೃ-ತಿ-ಇವು
ಸಂಸ್ಕೃ-ತಿ-ಗಳ
ಸಂಸ್ಕೃ-ತಿಯ
ಸಂಸ್ಕೃ-ತಿ-ಯನ್ನು
ಸಂಸ್ಕೃ-ತಿ-ಯನ್ನೇ
ಸಂಸ್ಕೃ-ತಿ-ಯಲ್ಲಿ
ಸಂಸ್ಕೃ-ತಿ-ಯಿಂದ
ಸಂಸ್ಕೃ-ತಿ-ಯಿದೆ
ಸಂಸ್ಕೃ-ತಿಯು
ಸಂಸ್ಥಾ-ನದ
ಸಂಸ್ಥಾ-ಪ-ಕರೂ
ಸಂಸ್ಥಾ-ಪಕಿ
ಸಂಸ್ಥಾ-ಪ-ನೆಯು
ಸಂಸ್ಥಾ-ಪಿ-ತ-ವಾದ
ಸಂಸ್ಥಾ-ಪಿಸಿ
ಸಂಸ್ಥಾ-ಪಿ-ಸು-ವಲ್ಲಿ
ಸಂಸ್ಥಾ-ಪಿ-ಸು-ವು-ದ-ಕ್ಕಾಗಿ
ಸಂಸ್ಥೆ
ಸಂಸ್ಥೆ-ಗಳ
ಸಂಸ್ಥೆ-ಗಳನ್ನು
ಸಂಸ್ಥೆ-ಗಳಿಂದ
ಸಂಸ್ಥೆ-ಗ-ಳಿಂ-ದಲೂ
ಸಂಸ್ಥೆ-ಗ-ಳಿಗೆ
ಸಂಸ್ಥೆ-ಗಳು
ಸಂಸ್ಥೆಗೂ
ಸಂಸ್ಥೆಗೆ
ಸಂಸ್ಥೆಯ
ಸಂಸ್ಥೆ-ಯನ್ನು
ಸಂಸ್ಥೆ-ಯಲ್ಲಿ
ಸಂಸ್ಥೆ-ಯ-ವರು
ಸಂಸ್ಥೆ-ಯಿಂದ
ಸಂಸ್ಥೆ-ಯಿ-ಲ್ಲದೆ
ಸಂಸ್ಥೆಯು
ಸಂಸ್ಥೆಯೂ
ಸಂಸ್ಥೆ-ಯೆಂ-ದರೆ
ಸಂಸ್ಥೆಯೇ
ಸಂಸ್ಥೆ-ಯೊಂ-ದಕ್ಕೆ
ಸಂಸ್ಥೆ-ಯೊಂ-ದನ್ನು
ಸಂಸ್ಥೆ-ಯೊಂ-ದರ
ಸಂಸ್ಥೆ-ಯೊಂ-ದಿಗೆ
ಸಂಸ್ಥೆ-ಯೊಂದು
ಸಂಸ್ಪ-ರ್ಶ-ದಿಂದ
ಸಂಸ್ಯಾ-ಸಿ-ಗಳ
ಸಂಹ-ರಿಸಿ
ಸಕಲ
ಸಕ-ಲ-ರಿಗೂ
ಸಕ-ಲರೂ
ಸಕಾ-ಲ-ದಲ್ಲಿ
ಸಕಾ-ಲ-ವೊ-ದ-ಗಿಲ್ಲ
ಸಕ್ರಿಯ
ಸಕ್ರಿ-ಯ-ಉ-ಪ-ಯುಕ್ತ
ಸಗು-ಣ-ಸಾ-ಕಾರ
ಸಚೇ-ತನ
ಸಚೇ-ತ-ನ-ಗೊ-ಳಿ-ಸ-ಬೇಕು
ಸಚೇ-ತ-ನ-ಗೊ-ಳಿಸಿ
ಸಚೇ-ತ-ನ-ವಾ-ಗಿ-ರ-ಲಿಲ್ಲ
ಸಚ್ಚಾ-ರಿ-ತ್ರ್ಯ-ಕ್ಕಾಗಿ
ಸಚ್ಚಿ-ದಾ-ನಂದ
ಸಚ್ಚಿ-ದಾ-ನಂ-ದದ
ಸಚ್ಚಿ-ದಾ-ನಂ-ದರು
ಸಚ್ಛಿ-ದಾ-ನಂದ
ಸಜ್ಜಾದ
ಸಡ-ಗ-ರ-ದಲ್ಲಿ
ಸಡಿಲ
ಸಡಿ-ಲಿ-ಸಲು
ಸಡಿ-ಲಿ-ಸಿ-ಕೊ-ಳ್ಳೋ-ಣವೇ
ಸಣ್ಣ
ಸಣ್ಣ-ಪುಟ್ಟ
ಸಣ್ಣ-ಸಣ್ಣ
ಸತ-ತ-ವಾಗಿ
ಸತಿ
ಸತ್-ಚಿ-ತ್-ಆ-ನಂ-ದವೇ
ಸತ್ಕ-ರಿ-ಸಿ-ದರು
ಸತ್ಕ-ರಿ-ಸಿ-ರ-ಲಿ-ಲ್ಲವೆ
ಸತ್ಕ-ರ್ಮ-ದು-ಷ್ಕ-ರ್ಮ-ಗಳ
ಸತ್ಕಾ-ರ-ಕೂ-ಟ-ಗ-ಳಲ್ಲೂ
ಸತ್ಕಾ-ರ-ಕೂ-ಟ-ವನ್ನು
ಸತ್ಕಾರ್ಯ
ಸತ್ಕಾ-ರ್ಯ-ವನ್ನು
ಸತ್ತ
ಸತ್ತಂ-ತೆಯೇ
ಸತ್ತರು
ಸತ್ತು-ಹೋದ
ಸತ್ತೇ
ಸತ್ತ್ವ
ಸತ್ತ್ವ-ಚೈ-ತ-ನ್ಯ-ಗಳನ್ನು
ಸತ್ತ್ವ-ಪರೀಕ್ಷೆ-ಯ
ಸತ್ತ್ವ-ವನ್ನು
ಸತ್ತ್ವ-ವಿ-ರ-ದಿದ್ದ
ಸತ್ತ್ವವೇ
ಸತ್ತ್ವ-ಹೀ-ನ-ತೆ-ಯಿಂ-ದಾಗಿ
ಸತ್ಪ-ರಿ-ಣಾಮ
ಸತ್ಪ-ರಿ-ಣಾ-ಮ-ಗ-ಳುಂ-ಟಾ-ಗಿ-ದ್ದುವು
ಸತ್ಪ-ರಿ-ಣಾ-ಮ-ವನ್ನು
ಸತ್ಪ-ರಿ-ಣಾ-ಮ-ವುಂ-ಟಾ-ಗು-ತ್ತದೆ
ಸತ್ಪ-ರಿ-ಣಾ-ಮವೂ
ಸತ್ಪು-ರು-ಷ-ರ-ನ್ನಾ-ಗಿಯೂ
ಸತ್ಯ
ಸತ್ಯ
ಸತ್ಯಂ
ಸತ್ಯಕ್ಕೆ
ಸತ್ಯ-ಗಳ
ಸತ್ಯ-ಗಳನ್ನು
ಸತ್ಯ-ಗಳು
ಸತ್ಯ-ತೆ-ಯನ್ನು
ಸತ್ಯ-ತೆಯು
ಸತ್ಯದ
ಸತ್ಯ-ದಲ್ಲಿ
ಸತ್ಯ-ದಿಂದ
ಸತ್ಯ-ದೂ-ತನ
ಸತ್ಯ-ದೆ-ಡೆ-ಗಲ್ಲ
ಸತ್ಯ-ದೆ-ಡೆಗೆ
ಸತ್ಯ-ಯುಗ
ಸತ್ಯ-ವ-ಚನ
ಸತ್ಯ-ವ-ನ್ನ-ರ-ಸಲು
ಸತ್ಯ-ವ-ನ್ನ-ರಸಿ
ಸತ್ಯ-ವ-ನ್ನ-ರಿತು
ಸತ್ಯ-ವನ್ನು
ಸತ್ಯ-ವನ್ನೂ
ಸತ್ಯ-ವನ್ನೇ
ಸತ್ಯ-ವ-ಲ್ಲ-ದಿ-ದ್ದರೆ
ಸತ್ಯ-ವಾ-ಗಿದ್ದು
ಸತ್ಯ-ವಾದ
ಸತ್ಯ-ವಾ-ದರೆ
ಸತ್ಯ-ವಾ-ದು-ದನ್ನು
ಸತ್ಯ-ವಾ-ಯಿತು
ಸತ್ಯ-ವಿ-ದ್ದರೆ
ಸತ್ಯವು
ಸತ್ಯ-ವೆಂದು
ಸತ್ಯ-ವೆಂದೇ
ಸತ್ಯ-ವೆಂ-ಬಂ-ತಾ-ಯಿತು
ಸತ್ಯ-ವೆ-ನಿ-ಸಿ-ಕೊ-ಳ್ಳು-ತ್ತದೆ
ಸತ್ಯವೇ
ಸತ್ಯ-ಶೋ-ಧ-ನೆಯ
ಸತ್ಯ-ಸಂ-ಗ-ತಿ-ಯನ್ನು
ಸತ್ಯ-ಸಂ-ಧ-ತೆ-ಪ್ರಾ-ಮಾ-ಣಿ-ಕ-ತೆ-ಧೈ-ರ್ಯ-ಪ-ವಿ-ತ್ರ-ತೆ-ನಿ-ಷ್ಕಾಮ
ಸತ್ಯ-ಸಂ-ಧ-ತೆ-ಪ್ರಾ-ಮಾ-ಣಿ-ಕ-ತೆ-ಗಳನ್ನು
ಸತ್ಯ-ಸಾ-ಕ್ಷಾ-ತ್ಕಾರ
ಸತ್ಯ-ಸಾ-ಕ್ಷಾ-ತ್ಕಾ-ರ-ಕ್ಕಾಗಿ
ಸತ್ಯ-ಸಾ-ಕ್ಷಾ-ತ್ಕಾ-ರದ
ಸತ್ಯ-ಸಾ-ಕ್ಷಾ-ತ್ಕಾ-ರ-ದಿಂದ
ಸತ್ಯ-ಸಾ-ಕ್ಷಾ-ತ್ಕಾ-ರ-ವಾ-ಗು-ತ್ತದೆ
ಸತ್ಯ-ಸ್ವ-ರೂ-ಪ-ನಾದ
ಸತ್ಯಾಂ-ಶ-ವನ್ನು
ಸತ್ಯಾಂ-ಶ-ವಾ-ವು-ದನ್ನೂ
ಸತ್ಯಾಂ-ಶ-ವೇನೆಂದರೆ
ಸತ್ಯಾ-ನ್ವೇ-ಷ-ಣೆ-ಯಲ್ಲಿ
ಸತ್ಯಾ-ನ್ವೇ-ಷಿ-ಗಳ
ಸತ್ಯಾ-ನ್ವೇ-ಷಿ-ಗ-ಳಾದ
ಸತ್ವ
ಸತ್ವ-ಪ-ರೀಕ್ಷೆ
ಸತ್ವ-ಪೂರ್ಣ
ಸತ್ವ-ಪೂ-ರ್ಣ-ವಾದ
ಸತ್ವ-ರ-ಹಿತ
ಸತ್ವ-ವನ್ನು
ಸತ್ವ-ಹೀ-ನ-ವಾಗಿ
ಸತ್ಸಂ-ಗ-ದಲ್ಲಿ
ಸತ್ಸಂ-ಗ-ವನ್ನು
ಸದ-ನ-ದಲ್ಲಿ
ಸದ-ವ-ಕಾ-ಶ-ವನ್ನು
ಸದ-ವ-ಕಾ-ಶ-ವೆಂದು
ಸದ-ಸ್ಯ-ತ್ವ-ವನ್ನು
ಸದ-ಸ್ಯ-ನಾಗಿ
ಸದ-ಸ್ಯನೂ
ಸದ-ಸ್ಯರ
ಸದ-ಸ್ಯ-ರನ್ನು
ಸದ-ಸ್ಯ-ರಾ-ಗಲು
ಸದ-ಸ್ಯ-ರಾ-ಗಿ-ದ್ದರು
ಸದ-ಸ್ಯ-ರಾ-ಗಿ-ದ್ದ-ವರು
ಸದ-ಸ್ಯ-ರಿ-ಗಾಗಿ
ಸದ-ಸ್ಯರು
ಸದ-ಸ್ಯರೂ
ಸದ-ಸ್ಯೆ-ಯರು
ಸದಾ
ಸದಾ-ಚಾ-ರದ
ಸದಾ-ಚಾ-ರ-ದಿಂ-ದಲೂ
ಸದಾ-ನಂ-ದರು
ಸದಾ-ಶಿವಂ
ಸದು-ದ್ದೇಶ
ಸದು-ದ್ದೇ-ಶದ
ಸದು-ದ್ದೇ-ಶ-ದಿಂ-ದ-ನೀವು
ಸದು-ದ್ದೇ-ಶ-ದಿಂ-ದಲೇ
ಸದು-ದ್ದೇ-ಶ-ವನ್ನು
ಸದು-ಪ-ಯೋ-ಗ-ಪ-ಡಿ-ಸಿ-ಕೊಂ-ಡರು
ಸದು-ಪ-ಯೋ-ಗ-ಪ-ಡಿ-ಸಿ-ಕೊಂಡು
ಸದೃಶ
ಸದೃ-ಶ-ವಾ-ಗಿ-ಸಿ-ದೆಯೋ
ಸದೆ
ಸದೆ-ಬ-ಡಿ-ಯಲು
ಸದ್ಗುಣ
ಸದ್ಗು-ಣ-ಗಳನ್ನು
ಸದ್ಗು-ಣ-ಗ-ಳೆಲ್ಲ
ಸದ್ದಾ-ಗು-ತ್ತಿತ್ತು
ಸದ್ದಿ-ಲ್ಲ-ದಂತೆ
ಸದ್ದು
ಸದ್ದು-ಗ-ದ್ದ-ಲದ
ಸದ್ದು-ಗ-ದ್ದ-ಲ-ದಲ್ಲಿ
ಸದ್ಭಾ-ವ-ಕ್ಕಾಗಿ
ಸದ್ಭಾ-ವನೆ
ಸದ್ಯ
ಸದ್ಯಕ್ಕೆ
ಸದ್ಯದ
ಸದ್ಯ-ದಲ್ಲೇ
ಸದ್ವ-ರ್ತ-ನೆ-ಯಿಂದ
ಸದ್ವಿ-ನಿ-ಯೋ-ಗ-ವಾ-ಗ-ಲೇ-ಬೇಕು
ಸದ್ವಿ-ಮ-ರ್ಶೆಗೆ
ಸನಾ-ತನ
ಸನಾ-ತ-ನ-ಧ-ರ್ಮದ
ಸನಾ-ತ-ನ-ಧ-ರ್ಮ-ವನ್ನು
ಸನಾ-ತ-ನ-ವಾ-ದುದು
ಸನಿ-ಹಕ್ಕೆ
ಸನ್
ಸನ್ನ-ದ್ಧ-ರಾ-ದರು
ಸನ್ನಾಹ
ಸನ್ನಿ-ವೇ-ಶ-ಗಳಲ್ಲಿ
ಸನ್ನಿ-ವೇ-ಶ-ದಲ್ಲಿ
ಸನ್ನಿ-ವೇ-ಶವೇ
ಸನ್ನಿ-ಹಿತ
ಸನ್ನಿ-ಹಿ-ತ-ವಾ-ಗಿದೆ
ಸನ್ನಿ-ಹಿ-ತ-ವಾ-ಗು-ತ್ತಿ-ರು-ವುದನ್ನು
ಸನ್ಮಂ-ಗ-ಳ-ವನ್ನು
ಸನ್ಮಾನ
ಸನ್ಮಾ-ನಾ-ರ್ಥ-ವಾಗಿ
ಸನ್ಯಾ-ಸಿ-ಯ-ಲ್ಲ-ಡ-ಗಿ-ರುವ
ಸಪ್ಪೆ
ಸಪ್ಪೆ-ಯಾ-ಗ-ತೊ-ಡ-ಗಿವೆ
ಸಪ್ಪೆ-ಯಾಗಿ
ಸಭಾಂ-ಗಣ
ಸಭಾಂ-ಗ-ಣ-ಗಳಲ್ಲಿ
ಸಭಾಂ-ಗ-ಣದ
ಸಭಾಂ-ಗ-ಣ-ದಲ್ಲಿ
ಸಭಾಂ-ಗ-ಣ-ವನ್ನು
ಸಭಾಂ-ಗ-ಣವು
ಸಭಾಂ-ಗ-ಣವೂ
ಸಭಾ-ಧ್ಯ-ಕ್ಷರ
ಸಭಾ-ಧ್ಯ-ಕ್ಷ-ರಾದ
ಸಭಾ-ಧ್ಯ-ಕ್ಷರು
ಸಭಾ-ಭ-ವನ
ಸಭಾ-ಭ-ವ-ನ-ಗಳಲ್ಲಿ
ಸಭಾ-ಭ-ವ-ನ-ದಲ್ಲಿ
ಸಭಾ-ಭ-ವ-ನ-ದಿಂ-ದಾಚೆ
ಸಭಾ-ಸ-ದರ
ಸಭಾ-ಸ-ದ-ರಿಂದ
ಸಭಿ-ಕರ
ಸಭಿ-ಕ-ರನ್ನು
ಸಭಿ-ಕ-ರ-ನ್ನು-ದ್ದೇ-ಶಿಸಿ
ಸಭಿ-ಕ-ರಲ್ಲಿ
ಸಭಿ-ಕ-ರಾ-ಗಿ-ದ್ದರು
ಸಭಿ-ಕ-ರಿಂದ
ಸಭಿ-ಕ-ರಿಗೆ
ಸಭಿ-ಕ-ರಿ-ಗೆಲ್ಲ
ಸಭಿ-ಕರು
ಸಭಿ-ಕರೂ
ಸಭಿ-ಕ-ರೆ-ದು-ರಿಗೇ
ಸಭಿ-ಕ-ರೆಲ್ಲ
ಸಭಿ-ಕ-ರೆ-ಲ್ಲರ
ಸಭಿ-ಕ-ವೃಂದ
ಸಭೆ
ಸಭೆ-ಗಳ
ಸಭೆ-ಗಳನ್ನು
ಸಭೆ-ಗ-ಳ-ನ್ನು-ದ್ದೇ-ಶಿಸಿ
ಸಭೆ-ಗಳನ್ನೂ
ಸಭೆ-ಗಳಲ್ಲಿ
ಸಭೆ-ಗ-ಳ-ಲ್ಲಿನ
ಸಭೆ-ಗ-ಳಲ್ಲೂ
ಸಭೆ-ಗ-ಳ-ಲ್ಲೆಲ್ಲ
ಸಭೆ-ಗಳಿಂದ
ಸಭೆ-ಗಳು
ಸಭೆಗೆ
ಸಭೆಯ
ಸಭೆ-ಯನ್ನು
ಸಭೆ-ಯ-ನ್ನು-ದ್ದೇ-ಶಿಸಿ
ಸಭೆ-ಯ-ನ್ನೆ-ದು-ರಿ-ಸಲು
ಸಭೆ-ಯನ್ನೇ
ಸಭೆ-ಯಲ್ಲಿ
ಸಭೆಯು
ಸಭೆಯೂ
ಸಭೆಯೇ
ಸಭೆ-ಯೊಂ-ದನ್ನು
ಸಭೆ-ಯೊಂ-ದರ
ಸಭೆ-ಯೊಂ-ದ-ರಲ್ಲಿ
ಸಭ್ಯ
ಸಭ್ಯ-ವಿ-ನ-ಯ-ಪೂರ್ಣ
ಸಭ್ಯ-ಸ-ಜ್ಜ-ನರು
ಸಭ್ಯ-ಸ-ಜ್ಜ-ನ-ರೊಂ-ದಿಗೆ
ಸಭ್ಯತೆ
ಸಭ್ಯ-ತೆ-ಸೌ-ಜ-ನ್ಯ-ಗಳು
ಸಭ್ಯ-ತೆ-ಯನ್ನೂ
ಸಭ್ಯ-ತೆ-ಯಿಂದ
ಸಭ್ಯ-ನಂತೆ
ಸಭ್ಯ-ನೊಂ-ದಿಗೆ
ಸಭ್ಯ-ವಾದ
ಸಭ್ಯ-ವ್ಯಕ್ತಿ
ಸಭ್ಯ-ವ್ಯ-ಕ್ತಿಗೇ
ಸಭ್ಯ-ಸ್ಥ-ನೆ-ನ್ನಿ-ಸಿ-ಕೊಂಡ
ಸಭ್ಯ-ಸ್ಥರು
ಸಮ
ಸಮಂ-ಜಸ
ಸಮ-ಕಾ-ಲೀನ
ಸಮ-ಕ್ಷ-ಮ-ದಲ್ಲಿ
ಸಮಗ್ರ
ಸಮ-ಗ್ರ-ವಾಗಿ
ಸಮ-ಗ್ರೀ-ಕ-ರ-ಣ-ಗೊ-ಳಿಸಿ
ಸಮ-ಗ್ರೀ-ಕ-ರ-ಣ-ವಾ-ಗಿ-ರ-ಬೇಕು
ಸಮ-ತ-ಟ್ಟಾ-ಗಿದೆ
ಸಮ-ತಾ-ಭಾ-ವ-ದಿಂದ
ಸಮ-ತೋ-ಲ-ವನ್ನು
ಸಮ-ತೋ-ಲವು
ಸಮ-ದ-ರಶೀ
ಸಮ-ದರ್ಶಿ
ಸಮ-ದ-ರ್ಶಿ-ಯಾದ
ಸಮ-ದ-ರ್ಶಿ-ಯೆಂ-ದಿ-ಹುದು
ಸಮ-ದರ್ಶೀ
ಸಮ-ದೃ-ಷ್ಟಿ-ಯ-ವನೂ
ಸಮ-ಧುರ
ಸಮ-ನಾಗಿ
ಸಮ-ನಾ-ಗಿ-ವ್ಯ-ವ-ಹ-ರಿ-ಸುತ್ತಿ
ಸಮ-ನಾ-ದದ್ದು
ಸಮನೆ
ಸಮ-ನ್ವಯ
ಸಮ-ನ್ವ-ಯ-ಗೊ-ಳಿ-ಸು-ವು-ದರ
ಸಮ-ನ್ವ-ಯದ
ಸಮ-ನ್ವ-ಯ-ವನ್ನು
ಸಮ-ನ್ವ-ಯವು
ಸಮಯ
ಸಮ-ಯಕ್ಕೆ
ಸಮ-ಯಕ್ಕೇ
ಸಮ-ಯ-ಗಳಲ್ಲಿ
ಸಮ-ಯದ
ಸಮ-ಯ-ದ-ನಂ-ತರ
ಸಮ-ಯ-ದಲ್ಲಿ
ಸಮ-ಯ-ದ-ಲ್ಲಿಯೂ
ಸಮ-ಯ-ದಲ್ಲೆಲ್ಲ
ಸಮ-ಯ-ದಲ್ಲೇ
ಸಮ-ಯ-ಪ್ರ-ಜ್ಞೆಯ
ಸಮ-ಯ-ವನ್ನು
ಸಮ-ಯ-ವನ್ನೂ
ಸಮ-ಯ-ವ-ನ್ನೆಲ್ಲ
ಸಮ-ಯ-ವಾ-ಯಿತು
ಸಮ-ಯ-ವಿ-ತ್ತಂತೆ
ಸಮ-ಯ-ವಿತ್ತು
ಸಮ-ಯ-ವಿಲ್ಲ
ಸಮ-ಯ-ವಿ-ಲ್ಲವೆ
ಸಮ-ಯವೂ
ಸಮ-ಯವೇ
ಸಮರ
ಸಮ-ರ-ಸ-ವಾಗಿ
ಸಮರ್ಥ
ಸಮ-ರ್ಥ-ನಾ-ಗಿದ್ದ
ಸಮ-ರ್ಥ-ನಾ-ಗಿ-ದ್ದಾನೆ
ಸಮ-ರ್ಥ-ನಾ-ಗು-ತ್ತಾನೆ
ಸಮ-ರ್ಥ-ನಾ-ದನು
ಸಮ-ರ್ಥನೆ
ಸಮ-ರ್ಥ-ನೆಂದೂ
ಸಮ-ರ್ಥ-ನೆ-ಗಾಗಿ
ಸಮ-ರ್ಥ-ನೆಗೆ
ಸಮ-ರ್ಥ-ನೆ-ಯನ್ನು
ಸಮ-ರ್ಥ-ರಾ-ಗ-ಬ-ಲ್ಲರು
ಸಮ-ರ್ಥ-ರಾ-ಗಿದ್ದ
ಸಮ-ರ್ಥ-ರಾ-ಗಿ-ದ್ದರು
ಸಮ-ರ್ಥ-ರಾ-ಗಿ-ದ್ದಾರೆ
ಸಮ-ರ್ಥ-ರಾ-ಗಿ-ದ್ದೀರಿ
ಸಮ-ರ್ಥ-ರಾ-ಗಿ-ದ್ದೇವೆ
ಸಮ-ರ್ಥ-ರಾ-ಗಿ-ರು-ವಂತೆ
ಸಮ-ರ್ಥ-ರಾ-ಗಿ-ರು-ವಾಗ
ಸಮ-ರ್ಥ-ರಾ-ಗಿ-ರು-ವುದು
ಸಮ-ರ್ಥ-ರಾ-ಗು-ತ್ತಾರೆ
ಸಮ-ರ್ಥ-ರಾ-ಗು-ತ್ತೇವೆ
ಸಮ-ರ್ಥ-ರಾದ
ಸಮ-ರ್ಥ-ರಾ-ದರು
ಸಮ-ರ್ಥ-ರಾ-ದರೆ
ಸಮ-ರ್ಥ-ರಾ-ದ-ರೆಂ-ಬುದೇ
ಸಮ-ರ್ಥ-ರಾ-ದಾ-ಗಿ-ನಿಂ-ದಲೂ
ಸಮ-ರ್ಥ-ರಾರೂ
ಸಮ-ರ್ಥರು
ಸಮ-ರ್ಥರೂ
ಸಮ-ರ್ಥ-ರೆಂ-ಬು-ದನ್ನು
ಸಮ-ರ್ಥರೇ
ಸಮ-ರ್ಥ-ಳಾ-ದಳು
ಸಮ-ರ್ಥ-ವಾಗಿ
ಸಮ-ರ್ಥ-ವಾ-ಗಿದೆ
ಸಮ-ರ್ಥ-ಶಿ-ಕ್ಷಣ
ಸಮ-ರ್ಥಿ-ಸಲು
ಸಮ-ರ್ಥಿಸಿ
ಸಮ-ರ್ಥಿ-ಸಿ-ಕೊ-ಳ್ಳಲು
ಸಮ-ರ್ಥಿ-ಸಿ-ಕೊ-ಳ್ಳುವ
ಸಮ-ರ್ಥಿ-ಸಿ-ಕೊ-ಳ್ಳು-ವಾಗ
ಸಮ-ರ್ಥಿ-ಸಿ-ಕೊ-ಳ್ಳು-ವು-ದ-ಕ್ಕಾಗಿ
ಸಮ-ರ್ಥಿ-ಸಿ-ದರು
ಸಮ-ರ್ಥಿ-ಸುವ
ಸಮ-ರ್ಥಿ-ಸು-ವುದು
ಸಮ-ರ್ಪಕ
ಸಮ-ರ್ಪ-ಕ-ವಾಗಿ
ಸಮ-ರ್ಪ-ಕ-ವಾ-ಗಿ-ದೆ-ಯೆಂದು
ಸಮ-ರ್ಪಿ-ಸಲು
ಸಮ-ರ್ಪಿ-ಸಿ-ಕೊಂ-ಡಾ-ಗಿತ್ತು
ಸಮ-ರ್ಪಿ-ಸಿ-ಕೊಂ-ಡು-ಬಿ-ಟ್ಟಿ-ದ್ದಾರೆ
ಸಮ-ರ್ಪಿ-ಸಿ-ಕೊ-ಳ್ಳ-ಬೇ-ಕೆಂಬ
ಸಮ-ರ್ಪಿ-ಸಿ-ಕೊ-ಳ್ಳಲು
ಸಮ-ರ್ಪಿ-ಸಿ-ಕೊ-ಳ್ಳು-ವುದೇ
ಸಮ-ರ್ಪಿ-ಸಿ-ಬಿ-ಟ್ಟಿ-ದ್ದೇನೆ
ಸಮ-ರ್ಪಿ-ಸಿ-ಬಿ-ಡ-ಬೇಕು
ಸಮ-ವಾಗಿ
ಸಮ-ವಾ-ಗಿತ್ತು
ಸಮಷ್ಟಿ
ಸಮಸ್ತ
ಸಮಸ್ಯೆ
ಸಮ-ಸ್ಯೆ-ಗಳ
ಸಮ-ಸ್ಯೆ-ಗಳನ್ನು
ಸಮ-ಸ್ಯೆ-ಗ-ಳಿಗೆ
ಸಮ-ಸ್ಯೆ-ಗಳು
ಸಮ-ಸ್ಯೆ-ಗಳೂ
ಸಮ-ಸ್ಯೆ-ಗಳೇ
ಸಮ-ಸ್ಯೆಗೆ
ಸಮ-ಸ್ಯೆಯ
ಸಮ-ಸ್ಯೆ-ಯಂ-ತೆಯೂ
ಸಮ-ಸ್ಯೆ-ಯನ್ನು
ಸಮ-ಸ್ಯೆ-ಯಾ-ದರೆ
ಸಮಾ-ಚಾರ
ಸಮಾ-ಚಾ-ರ-ವನ್ನು
ಸಮಾ-ಚಾ-ರವೇ
ಸಮಾಜ
ಸಮಾ-ಜ-ವಿ-ಜ್ಞಾನ
ಸಮಾ-ಜಕ್ಕೆ
ಸಮಾ-ಜ-ಗ-ಳಲ್ಲೂ
ಸಮಾ-ಜದ
ಸಮಾ-ಜ-ದಲ್ಲಿ
ಸಮಾ-ಜ-ದಿಂದ
ಸಮಾ-ಜ-ದೊ-ಳಗೆ
ಸಮಾ-ಜ-ನಿ-ರ್ಮಾ-ಣ-ಕಾ-ರ್ಯ-ಗಳಲ್ಲಿ
ಸಮಾ-ಜ-ನಿಷ್ಠೆ
ಸಮಾ-ಜ-ವನ್ನು
ಸಮಾ-ಜ-ವನ್ನೂ
ಸಮಾ-ಜ-ವನ್ನೋ
ಸಮಾ-ಜ-ವಾ-ಗಿ-ರದೆ
ಸಮಾ-ಜವು
ಸಮಾ-ಜ-ವೆಂಬ
ಸಮಾ-ಜ-ಸು-ಧಾ-ರಣಾ
ಸಮಾ-ಧಾನ
ಸಮಾ-ಧಾ-ನಕ್ಕೆ
ಸಮಾ-ಧಾ-ನ-ಗೊ-ಳಿ-ಸ-ಬೇಕು
ಸಮಾ-ಧಾ-ನ-ಗೊ-ಳಿಸಿ
ಸಮಾ-ಧಾ-ನ-ಗೊ-ಳಿ-ಸಿ-ದರು
ಸಮಾ-ಧಾ-ನ-ಗೊ-ಳಿ-ಸು-ತ್ತಿ-ದ್ದರು
ಸಮಾ-ಧಾ-ನ-ಚಿ-ತ್ತ-ದಿಂದ
ಸಮಾ-ಧಾ-ನದ
ಸಮಾ-ಧಾ-ನ-ಪ-ಡಿ-ಸ-ಲೆ-ತ್ನಿ-ಸು-ವುದೆ
ಸಮಾ-ಧಾ-ನ-ವನ್ನು
ಸಮಾ-ಧಾ-ನ-ವನ್ನೂ
ಸಮಾ-ಧಾ-ನ-ವಾ-ಗಿತ್ತು
ಸಮಾ-ಧಾ-ನ-ವಾ-ಗಿ-ರ-ಲೇ-ಬೇಕು
ಸಮಾ-ಧಾ-ನ-ವಾ-ಗು-ತ್ತದೆ
ಸಮಾ-ಧಾ-ನ-ವಾ-ಗು-ವಂ-ತಿ-ರ-ಲಿಲ್ಲ
ಸಮಾ-ಧಾ-ನ-ವಾ-ಯಿತು
ಸಮಾ-ಧಾ-ನ-ವಿಲ್ಲ
ಸಮಾಧಿ
ಸಮಾ-ಧಿ-ಗೇ-ರಲು
ಸಮಾ-ಧಿಯ
ಸಮಾ-ಧಿ-ಯನ್ನೂ
ಸಮಾ-ಧಿ-ಸ್ಥಿ-ತಿಗೆ
ಸಮಾ-ಧಿ-ಸ್ಥಿ-ತಿ-ಗೇ-ರಲು
ಸಮಾ-ಧಿ-ಸ್ಥಿ-ತಿ-ಯ-ಲ್ಲಿ-ದ್ದಾರೆ
ಸಮಾನ
ಸಮಾ-ನತಾ
ಸಮಾ-ನ-ತೆಯ
ಸಮಾ-ನ-ಭಾ-ವ-ದಿಂದ
ಸಮಾ-ನ-ವಾಗಿ
ಸಮಾ-ನ-ಸ್ಕಂ-ಧ-ರನ್ನು
ಸಮಾ-ನಾ-ಭಿ-ಪ್ರಾ-ಯ-ವಿ-ರ-ಲಿಲ್ಲ
ಸಮಾ-ನೋ-ದ್ದೇ-ಶ-ಗ-ಳತ್ತ
ಸಮಾ-ಪ್ತಿ-ಗೊ-ಳಿಸಿ
ಸಮಾ-ರಂಭ
ಸಮಾ-ರಂ-ಭಕ್ಕೆ
ಸಮಾ-ರಂ-ಭ-ಗಳಲ್ಲಿ
ಸಮಾ-ರಂ-ಭ-ಗಳು
ಸಮಾ-ರಂ-ಭದ
ಸಮಾ-ರಂ-ಭ-ದಲ್ಲಿ
ಸಮಾ-ರಂ-ಭ-ವನ್ನು
ಸಮಾ-ರಂ-ಭ-ವನ್ನೇ
ಸಮಾ-ರಂ-ಭವು
ಸಮಾ-ರಂ-ಭ-ವೆಂ-ಬು-ದ-ರಲ್ಲಿ
ಸಮಾ-ರಂ-ಭ-ವೊಂ-ದನ್ನು
ಸಮಾ-ಲೋ-ಚಿ-ಸುತ್ತ
ಸಮಿತಿ
ಸಮಿ-ತಿಯ
ಸಮಿ-ತಿ-ಯ-ವ-ರನ್ನು
ಸಮಿ-ತಿ-ಯ-ವರು
ಸಮಿ-ತಿ-ಯಾ-ಯಿತು
ಸಮಿ-ತಿ-ಯೊಂ-ದನ್ನು
ಸಮಿ-ತಿ-ಯೊಂದು
ಸಮೀಪ
ಸಮೀ-ಪದ
ಸಮೀ-ಪ-ದಲ್ಲಿ
ಸಮೀ-ಪ-ದ-ಲ್ಲಿದ್ದ
ಸಮೀ-ಪ-ದಲ್ಲೇ
ಸಮೀ-ಪ-ವಾ-ದಂತೆ
ಸಮೀ-ಪಿ-ಸ-ಬೇಕು
ಸಮೀ-ಪಿ-ಸಿತು
ಸಮೀ-ಪಿ-ಸಿದೆ
ಸಮೀ-ಪಿ-ಸು-ತ್ತಿದೆ
ಸಮೀ-ಪಿ-ಸು-ತ್ತಿ-ರು-ವು-ದರ
ಸಮು-ದಾಯ
ಸಮು-ದಾ-ಯದ
ಸಮು-ದಾ-ಯ-ದಲ್ಲಿ
ಸಮುದ್ರ
ಸಮು-ದ್ರಕ್ಕೆ
ಸಮು-ದ್ರ-ಜಾ-ಡ್ಯ-ದಿಂದ
ಸಮು-ದ್ರ-ತೀ-ರಕ್ಕೆ
ಸಮು-ದ್ರ-ತೀ-ರದ
ಸಮು-ದ್ರ-ತೀ-ರ-ದಲ್ಲಿ
ಸಮು-ದ್ರದ
ಸಮು-ದ್ರ-ದಲ್ಲಿ
ಸಮು-ದ್ರ-ದಾ-ಳ-ದ-ಲ್ಲಿಯೂ
ಸಮು-ದ್ರ-ಮ-ಟ್ಟ-ಕ್ಕಿಂತ
ಸಮು-ದ್ರ-ಮಾ-ರ್ಗ-ವಾಗಿ
ಸಮು-ದ್ರ-ಯಾನ
ಸಮು-ದ್ರ-ಯಾ-ನ-ವನ್ನು
ಸಮು-ದ್ರ-ಯಾ-ನವು
ಸಮು-ದ್ರ-ವನ್ನು
ಸಮು-ದ್ರ-ವ-ನ್ನು-ದಾ-ಟು-ವುದು
ಸಮು-ದ್ರ-ವನ್ನೇ
ಸಮು-ದ್ರ-ಸ್ನಾ-ನ-ಕ್ಕೆಂದು
ಸಮು-ದ್ರ-ಸ್ನಾ-ನ-ವನ್ನು
ಸಮೂಹ
ಸಮೂ-ಹ-ದೊಂ-ದಿಗೆ
ಸಮೃದ್ಧಿ
ಸಮೃ-ದ್ಧಿ-ಸು-ವ್ಯ-ವ-ಸ್ಥೆ-ಗ-ಳಿಗೆ
ಸಮೃ-ದ್ಧಿ-ಯನ್ನೂ
ಸಮೇ-ತ-ನಾಗಿ
ಸಮೇ-ತ-ರಾಗಿ
ಸಮೇ-ತ-ವಾಗಿ
ಸಮ್ಮ-ತಿ-ಯನ್ನು
ಸಮ್ಮ-ತಿ-ಯಿ-ರು-ವು-ದಾ-ದಲ್ಲಿ
ಸಮ್ಮ-ತಿಸ
ಸಮ್ಮ-ತಿಸಿ
ಸಮ್ಮ-ತಿ-ಸಿ-ದರು
ಸಮ್ಮನೆ
ಸಮ್ಮರ್
ಸಮ್ಮಿ-ಳಿ-ತ-ಗೊಂ-ಡಿ-ದ್ದುವು
ಸಮ್ಮಿ-ಳಿ-ತ-ವಾ-ಗಿ-ದ್ದುವು
ಸಮ್ಮು-ಖ-ದಲ್ಲಿ
ಸಮ್ಮೇ-ಳನ
ಸಮ್ಮೇ-ಳ-ನಕ್ಕೆ
ಸಮ್ಮೇ-ಳ-ನ-ಗಳ
ಸಮ್ಮೇ-ಳ-ನ-ಗ-ಳ-ಲ್ಲೊಂ-ದಾದ
ಸಮ್ಮೇ-ಳ-ನ-ಗ-ಳಿಗೆ
ಸಮ್ಮೇ-ಳ-ನ-ಗಳು
ಸಮ್ಮೇ-ಳ-ನದ
ಸಮ್ಮೇ-ಳ-ನ-ದಲ್ಲಿ
ಸಮ್ಮೇ-ಳ-ನ-ದ-ಲ್ಲಿನ
ಸಮ್ಮೇ-ಳ-ನ-ದಿಂದ
ಸಮ್ಮೇ-ಳ-ನ-ದಿಂ-ದಾಗಿ
ಸಮ್ಮೇ-ಳ-ನ-ವನ್ನು
ಸಮ್ಮೇ-ಳ-ನ-ವಾ-ಗಿತ್ತು
ಸಮ್ಮೇ-ಳ-ನವು
ಸಮ್ಮೇ-ಳ-ನವೂ
ಸಮ್ಮೇ-ಳ-ನ-ವೆಂ-ಬುದು
ಸಮ್ಮೇ-ಳ-ನ-ವೊಂ-ದರ
ಸಮ್ಮೇ-ಳ-ನಾ-ಧ್ಯ-ಕ್ಷ-ರಾದ
ಸಯ-ಮ-ದಲ್ಲಿ
ಸಯಾ-ಮ್-ಇ-ವೆಲ್ಲ
ಸಯ್ಯಾ-ಜಿ-ರಾವ್
ಸಯ್ಯಾ-ಜಿ-ರಾ-ವ್ಗಾ-ಯ-ಕ-ವಾ-ಡ-ನನ್ನೂ
ಸರಂ-ಜಾ-ಮು-ಗಳು
ಸರ-ಕಲ್ಲ
ಸರ-ಕಾರ
ಸರ-ಕಾ-ರಕ್ಕೆ
ಸರ-ಕಾ-ರದ
ಸರ-ಕಾ-ರ-ದಿಂ-ದಲೂ
ಸರ-ಕಾ-ರವು
ಸರ-ಕಾರೀ
ಸರ-ಕೆಂ-ದರೆ
ಸರ-ಟೋಗ
ಸರ-ಟೋ-ಗಕ್ಕೆ
ಸರ-ಣಿ-ಗಳನ್ನು
ಸರ-ಣಿಗೆ
ಸರ-ಣಿಯ
ಸರ-ಣಿ-ಯನ್ನು
ಸರ-ಣಿ-ಯಲ್ಲಿ
ಸರ-ಣಿ-ಯೊಂ-ದನ್ನು
ಸರದಿ
ಸರ-ದಿ-ಯಲ್ಲಿ
ಸರ-ಪಳಿ
ಸರಳ
ಸರ-ಳ-ನಿ-ರಾ-ತಂಕ
ಸರ-ಳ-ಪ್ರಾ-ಮಾ-ಣಿಕ
ಸರ-ಳ-ಸಾ-ತ್ವಿಕ
ಸರ-ಳ-ತ-ನ-ವನ್ನು
ಸರ-ಳತೆ
ಸರ-ಳ-ತೆ-ಪಾ-ವಿತ್ರ್ಯ
ಸರ-ಳ-ತೆಯ
ಸರ-ಳ-ತೆ-ಯನ್ನೂ
ಸರ-ಳ-ತೆ-ಯಿಂ-ದಾಗಿ
ಸರ-ಳರು
ಸರ-ಳ-ವಾಗಿ
ಸರ-ಳ-ವಾದ
ಸರ-ಳ-ವಾ-ದರೂ
ಸರ-ಳವೂ
ಸರಸ
ಸರ-ಸ-ರನೆ
ಸರ-ಸ್ವ-ತಿಯ
ಸರಾ-ಗ-ವಾಗಿ
ಸರಾ-ಸರಿ
ಸರಿ
ಸರಿ-ಈ-ಗಿನ
ಸರಿ-ಕಾ-ಣು-ತ್ತ-ದೆಯೋ
ಸರಿ-ಗ-ಟ್ಟಲು
ಸರಿ-ದಾರಿಗೆ
ಸರಿ-ದೂ-ಗುವ
ಸರಿ-ದೊ-ಡ-ನೆಯೇ
ಸರಿ-ಪ-ಡಿ-ಸ-ಬೇ-ಕೆಂಬ
ಸರಿ-ಪ-ಡಿ-ಸಲು
ಸರಿ-ಪ-ಡಿಸಿ
ಸರಿ-ಪ-ಡಿ-ಸಿ-ಕೊಂಡ
ಸರಿ-ಪ-ಡಿ-ಸಿ-ಕೊ-ಳ್ಳಲು
ಸರಿ-ಬಂದ
ಸರಿ-ಬ-ರ-ಲಿಲ್ಲ
ಸರಿ-ಯ-ಬೇ-ಕಾ-ದರೆ
ಸರಿ-ಯಲು
ಸರಿ-ಯಲ್ಲ
ಸರಿ-ಯ-ಲ್ಲ-ದ-ವರು
ಸರಿ-ಯಾಗಿ
ಸರಿ-ಯಾ-ಗಿ-ಟ್ಟು-ಕೊಂಡು
ಸರಿ-ಯಾ-ಗಿದೆ
ಸರಿ-ಯಾ-ಗಿ-ದ್ದರೆ
ಸರಿ-ಯಾ-ಗು-ತ್ತದೆ
ಸರಿ-ಯಾದ
ಸರಿ-ಯಾ-ದದ್ದೂ
ಸರಿ-ಯಾ-ದದ್ದೇ
ಸರಿ-ಯಾ-ದ-ವರು
ಸರಿ-ಯಾ-ದೀತು
ಸರಿ-ಯಾ-ದು-ದು-ಶ್ರೇ-ಷ್ಠ-ವಾ-ದುದು
ಸರಿ-ಯಿತು
ಸರಿ-ಯಿ-ರ-ಲಿಲ್ಲ
ಸರಿ-ಯಿ-ಲ್ಲ-ದ್ದ-ರಿಂದ
ಸರಿ-ಯು-ತ್ತಿ-ರು-ವು-ದರ
ಸರಿಯೆ
ಸರಿ-ಯೆಂದು
ಸರಿ-ಯೆಂಬ
ಸರಿ-ಯೆ-ಅ-ವರ
ಸರಿ-ಯೆ-ನೇ-ರ-ವಾಗಿ
ಸರಿಯೇ
ಸರಿಯೋ
ಸರಿ-ಸ-ಮ-ನಾ-ದದ್ದು
ಸರಿ-ಸ-ರ್ವ-ಸಂ-ಗ-ಪ-ರಿ-ತ್ಯಾ-ಗಿ-ಗಳೇ
ಸರಿ-ಸಾ-ಟಿ-ಯಾಗಿ
ಸರಿ-ಸಾ-ಟಿ-ಯಿ-ಲ್ಲದ್ದು
ಸರಿಸಿ
ಸರಿ-ಸಿ-ದರು
ಸರಿ-ಸುತ್ತ
ಸರಿ-ಹೊಂ-ದಿ-ಸಿ-ಕೊ-ಳ್ಳ-ಬೇಕೆ
ಸರಿ-ಹೋ-ಯಿತು
ಸರೋ
ಸರೋ-ಜಿನಿ
ಸರೋ-ವರ
ಸರೋ-ವ-ರಕ್ಕೆ
ಸರೋ-ವ-ರದ
ಸರೋ-ವ-ರ-ದಲ್ಲಿ
ಸರೋ-ವ-ರ-ವನ್ನು
ಸರೋ-ವ-ರ-ವಲ್ಲ
ಸರೋ-ವ-ರ-ವೊಂ-ದರ
ಸರ್
ಸರ್ಕಾರ
ಸರ್ಕಾ-ರದ
ಸರ್ಕಾ-ರ-ದಲ್ಲಿ
ಸರ್ಕಾರಿ
ಸರ್ಕಾರೀ
ಸರ್ದಾರ್
ಸರ್ಪಿ-ಸಿ-ಕೊ-ಳ್ಳಲು
ಸರ್ವ
ಸರ್ವಂ
ಸರ್ವ-ಜ-ನ-ಗ್ರಾ-ಹ್ಯವೂ
ಸರ್ವತೋ
ಸರ್ವ-ತೋ-ಮುಖ
ಸರ್ವ-ಧರ್ಮ
ಸರ್ವ-ಧ-ರ್ಮ-ಗಳ
ಸರ್ವ-ಧ-ರ್ಮ-ಗ-ಳಲ್ಲೂ
ಸರ್ವ-ಧ-ರ್ಮ-ಸ-ಮ್ಮೇ-ಳ-ನದ
ಸರ್ವ-ಧ-ರ್ಮ-ಸ-ಮ್ಮೇ-ಳ-ನ-ದಲ್ಲಿ
ಸರ್ವ-ಧ-ರ್ಮ-ಸ-ಮ್ಮೇ-ಳನಾ
ಸರ್ವ-ಧ-ರ್ಮ-ಸ್ಥಾ-ಪ-ಕಾ-ಚಾ-ರ್ಯರೂ
ಸರ್ವ-ಧ-ರ್ಮ-ಸ್ವ-ರೂ-ಪಿ-ಗಳೂ
ಸರ್ವ-ನಾ-ಶ-ದೆ-ಡೆಗೆ
ಸರ್ವ-ನಾ-ಶ-ವಾ-ಗು-ತ್ತದೆ
ಸರ್ವ-ನಾ-ಶವು
ಸರ್ವ-ಮಂ-ಗಳ
ಸರ್ವ-ಮತ
ಸರ್ವ-ರಿಗೂ
ಸರ್ವರೂ
ಸರ್ವರೆ-ಡೆಗೂ
ಸರ್ವ-ವನ್ನೂ
ಸರ್ವ-ವಿ-ಧ-ದಿಂ-ದಲೂ
ಸರ್ವ-ವ್ಯಾ-ಪಕ
ಸರ್ವ-ವ್ಯಾ-ಪ-ಕ-ತೆಯ
ಸರ್ವ-ವ್ಯಾ-ಪ-ಕ-ತ್ವ-ದಲ್ಲಿ
ಸರ್ವಶಃ
ಸರ್ವ-ಶಕ್ತ
ಸರ್ವ-ಶ-ಕ್ತ-ನಾದ
ಸರ್ವ-ಶ-ಕ್ತಿ-ಸ್ವ-ರೂ-ಪ-ನಾದ
ಸರ್ವ-ಶ್ರೇ-ಷ್ಠ-ನೆಂದು
ಸರ್ವ-ಶ್ರೇ-ಷ್ಠ-ವೆಂ-ಬು-ದನ್ನು
ಸರ್ವ-ಸಂಗ
ಸರ್ವ-ಸ-ಮ-ದ-ರ್ಶಿತ್ವ
ಸರ್ವ-ಸಿ-ದ್ಧ-ತೆ-ಗಳನ್ನೂ
ಸರ್ವಸ್ವ
ಸರ್ವ-ಸ್ವ-ವನ್ನು
ಸರ್ವ-ಸ್ವ-ವನ್ನೂ
ಸರ್ವಾಂ-ಗೀಣ
ಸರ್ವಾ-ಜ-ನಿಕ
ಸರ್ವಾ-ನು-ಮ-ತ-ದಿಂದ
ಸರ್ವಾ-ರ್ಪಣ
ಸರ್ವಿ-ಸಿ-ನಲ್ಲೇ
ಸರ್ವೇ-ಸಾ-ಧಾ-ರಣ
ಸರ್ವೇ-ಸಾ-ಮಾ-ನ್ಯ-ವಾದ
ಸರ್ವೋ-ತ್ಕೃ-ಷ್ಟ-ವಾದ
ಸಲ
ಸಲ-ಕ್ಕಿಂ-ತಲೂ
ಸಲಕ್ಕೆ
ಸಲ-ದಂತೆ
ಸಲಭ
ಸಲ-ವಂತೂ
ಸಲ-ವಾ-ದರೂ
ಸಲವೂ
ಸಲ-ವೇ-ನ-ಲ್ಲ-ವಲ್ಲ
ಸಲಹಿ
ಸಲಹೆ
ಸಲ-ಹೆ-ಸೂ-ಚ-ನೆ-ಗಳನ್ನು
ಸಲ-ಹೆ-ಗಳ
ಸಲ-ಹೆ-ಗಳನ್ನು
ಸಲ-ಹೆ-ಗ-ಳಾ-ಗಲಿ
ಸಲ-ಹೆ-ಗ-ಳಾ-ಗಿ-ದ್ದು-ವಷ್ಟೇ
ಸಲ-ಹೆ-ಗಳು
ಸಲ-ಹೆಗೆ
ಸಲ-ಹೆಯ
ಸಲ-ಹೆ-ಯಂತೆ
ಸಲ-ಹೆ-ಯಂ-ತೆಯೇ
ಸಲ-ಹೆ-ಯನ್ನು
ಸಲಿಗೆ
ಸಲಿ-ಗೆ-ಯಿಂದ
ಸಲು
ಸಲು-ವಾಗಿ
ಸಲು-ವಾ-ಗಿ-ಯಾ-ದರೂ
ಸಲು-ವಾದ
ಸಲೂ
ಸಲ್ಲ-ದಿ-ದ್ದಂ-ತಹ
ಸಲ್ಲ-ಬೇಕು
ಸಲ್ಲ-ಲಾ-ರದು
ಸಲ್ಲಿ-ಸ-ದಿ-ದ್ದರೆ
ಸಲ್ಲಿ-ಸ-ಬೇ-ಕಾ-ದ-ವರು
ಸಲ್ಲಿ-ಸ-ಬೇಕು
ಸಲ್ಲಿ-ಸ-ಬೇ-ಕೆಂದು
ಸಲ್ಲಿ-ಸಲು
ಸಲ್ಲಿ-ಸಲೇ
ಸಲ್ಲಿಸಿ
ಸಲ್ಲಿ-ಸಿದ
ಸಲ್ಲಿ-ಸಿ-ದರು
ಸಲ್ಲಿ-ಸಿ-ದರೋ
ಸಲ್ಲಿ-ಸಿ-ದಳು
ಸಲ್ಲಿ-ಸಿ-ದಾಗ
ಸಲ್ಲಿ-ಸಿದ್ದ
ಸಲ್ಲಿ-ಸಿ-ದ್ದರು
ಸಲ್ಲಿ-ಸಿ-ರ-ಲಿಲ್ಲ
ಸಲ್ಲಿ-ಸಿ-ರುವ
ಸಲ್ಲಿ-ಸು-ತ್ತಿ-ರುವ
ಸಲ್ಲಿ-ಸು-ತ್ತಿ-ರು-ವುದು
ಸಲ್ಲಿ-ಸು-ತ್ತೇನೆ
ಸಲ್ಲಿ-ಸು-ವಂ-ತಹ
ಸಲ್ಲಿ-ಸು-ವ-ವ-ನೊ-ಬ್ಬ-ನಿಗೆ
ಸಲ್ಲಿ-ಸು-ವು-ದ-ಕ್ಕಾಗಿ
ಸಲ್ಲುತ್ತ
ಸಲ್ಲು-ತ್ತದೆ
ಸಲ್ಲು-ತ್ತವೆ
ಸಳನ್ನು
ಸವ-ರಿ-ಸಿ-ಕೊ-ಳ್ಳುತ್ತ
ಸವಾ-ರಿ-ಸಿ-ಕೊಂಡು
ಸವಾ-ಲನ್ನು
ಸವಾಲು
ಸವಾ-ಲೊ-ಡ್ಡಿದ
ಸವಿದು
ಸವಿ-ನೆ-ನ-ಪೊಂದೇ
ಸವಿ-ಯನ್ನು
ಸವಿ-ಯ-ನ್ನು-ಣಿ-ಸಿ-ದರು
ಸವಿ-ಯನ್ನೂ
ಸವಿ-ಯ-ಬಲ್ಲ
ಸವಿ-ಯಾಗಿ
ಸವಿ-ಯು-ತ್ತಿ-ದ್ದರು
ಸವಿ-ಯು-ವ-ವ-ರೆಗೂ
ಸವಿ-ವ-ರ-ವಾಗಿ
ಸವಿ-ಸ್ತಾ-ರ-ವಾಗಿ
ಸವೆ-ದಿ-ರ-ಲಿಲ್ಲ
ಸವೆಸಿ
ಸವೆ-ಸಿದ
ಸವೆ-ಸಿ-ಬಿ-ಟ್ಟಿ-ದ್ದರು
ಸವೆ-ಸಿ-ರು-ವೆ-ನೆಂದು
ಸಶ-ರೀ-ರಿ-ಯಾಗಿ
ಸಶ-ರೀ-ರಿ-ಯಾ-ಗಿ-ದ್ದಾ-ಗಿ-ನಿಂ-ದಲೂ
ಸಸಿಗೆ
ಸಸ್ಯ
ಸಹ
ಸಹ-ಕ-ರಿ-ಸ-ದಿ-ದ್ದರೆ
ಸಹ-ಕ-ರಿ-ಸಲು
ಸಹ-ಕ-ರಿ-ಸು-ತ್ತಿ-ದ್ದರು
ಸಹ-ಕಾರ
ಸಹ-ಕಾ-ರದ
ಸಹ-ಕಾ-ರ-ದಿಂ-ದಲೇ
ಸಹ-ಕಾ-ರ-ದೊಂ-ದಿಗೆ
ಸಹ-ಗ-ಮನ
ಸಹ-ಚ-ರ-ರಿ-ಗಾಗಿ
ಸಹ-ಚ-ರರು
ಸಹಜ
ಸಹ-ಜ-ಮ-ಧುರ
ಸಹ-ಜ-ಸ-ರಳ
ಸಹ-ಜ-ಧಾ-ಮಕ್ಕೆ
ಸಹ-ಜ-ನರ
ಸಹ-ಜ-ಭಾ-ವ-ದಿಂದ
ಸಹ-ಜ-ವಾಗಿ
ಸಹ-ಜ-ವಾ-ಗಿಯೇ
ಸಹ-ಜ-ವಾ-ಗಿ-ಸಿ-ಕೊ-ಳ್ಳು-ವು-ದನ್ನೇ
ಸಹ-ಜ-ವಾದ
ಸಹ-ಜವೇ
ಸಹ-ಜೀ-ವನ
ಸಹ-ಜೀ-ವ-ನವೇ
ಸಹ-ನಾ-ಗರಿ
ಸಹ-ನೀ-ಯ-ವಾ-ಗ-ಲಿಲ್ಲ
ಸಹನೆ
ಸಹ-ನೆಯ
ಸಹ-ನೆ-ಯಿಂದ
ಸಹ-ನೆ-ಯಿಂ-ದಲೂ
ಸಹ-ಪಾ-ಠಿ-ಗ-ಳೊಂ-ದಿಗೆ
ಸಹ-ಪ್ರ-ತಿ-ನಿ-ಧಿ-ಗ-ಳೊಂ-ದಿಗೆ
ಸಹ-ಪ್ರ-ಯಾ-ಣಿ-ಕನ
ಸಹ-ಪ್ರ-ಯಾ-ಣಿ-ಕ-ನಾ-ಗಿದ್ದ
ಸಹ-ಪ್ರ-ಯಾ-ಣಿ-ಕರ
ಸಹ-ಪ್ರ-ಯಾ-ಣಿ-ಕ-ರಿಗೂ
ಸಹ-ಪ್ರ-ಯಾ-ಣಿ-ಕ-ರೊಂ-ದಿಗೆ
ಸಹ-ಬಾಳ್ವೆ
ಸಹ-ಭಾ-ಗಿ-ಗ-ಳಾ-ಗು-ತ್ತಾ-ರೆಂದು
ಸಹ-ಭಾ-ಗಿ-ಗ-ಳಾ-ದರು
ಸಹ-ಭಾ-ಗಿನಿ
ಸಹ-ಭಾ-ಗಿ-ಯಾ-ಗ-ಲಿದೆ
ಸಹ-ಮಾ-ನ-ವರ
ಸಹ-ಮಾ-ನ-ವ-ರನ್ನು
ಸಹ-ವಾಸ
ಸಹ-ವಾ-ಸ-ದ-ಲ್ಲಿ-ರು-ವುದು
ಸಹಸ್ರ
ಸಹ-ಸ್ರಕ್ಕೂ
ಸಹ-ಸ್ರ-ದ್ವೀ-ಪೋ-ದ್ಯಾನ
ಸಹ-ಸ್ರ-ದ್ವೀ-ಪೋ-ದ್ಯಾ-ನಕ್ಕೆ
ಸಹ-ಸ್ರ-ದ್ವೀ-ಪೋ-ದ್ಯಾ-ನದ
ಸಹ-ಸ್ರ-ದ್ವೀ-ಪೋ-ದ್ಯಾ-ನ-ದಲ್ಲಿ
ಸಹ-ಸ್ರ-ದ್ವೀ-ಪೋ-ದ್ಯಾ-ನವೇ
ಸಹ-ಸ್ರ-ಪಾಲು
ಸಹ-ಸ್ರಾ-ರ-ದ-ವ-ರೆಗೂ
ಸಹ-ಸ್ರಾರು
ಸಹಾನು
ಸಹಾ-ನು-ಭೂತಿ
ಸಹಾ-ನು-ಭೂ-ತಿ-ಇವು
ಸಹಾ-ನು-ಭೂ-ತಿ-ಗಳೇ
ಸಹಾ-ನು-ಭೂ-ತಿ-ಪೂ-ರ್ವ-ಕ-ವಾದ
ಸಹಾ-ನು-ಭೂ-ತಿ-ಯನ್ನು
ಸಹಾ-ನು-ಭೂ-ತಿ-ಯನ್ನೂ
ಸಹಾ-ನು-ಭೂ-ತಿ-ಯಲ್ಲಿ
ಸಹಾ-ನು-ಭೂ-ತಿ-ಯಿಂದ
ಸಹಾ-ನು-ಭೂ-ತಿ-ಯಿ-ದ್ದ-ವನು
ಸಹಾಯ
ಸಹಾ-ಯಕ
ಸಹಾ-ಯ-ಕ-ನಾ-ಗಿದ್ದ
ಸಹಾ-ಯ-ಕ-ಳಾದ
ಸಹಾ-ಯ-ಕ-ವಾ-ಗಿವೆ
ಸಹಾ-ಯ-ಕಾರಿ
ಸಹಾ-ಯ-ಕಾ-ರಿ-ಯಾ-ಗ-ಬ-ಲ್ಲದು
ಸಹಾ-ಯ-ಕಾ-ರಿ-ಯಾ-ಯಿತು
ಸಹಾ-ಯಕಿ
ಸಹಾ-ಯ-ಕ್ಕಾಗಿ
ಸಹಾ-ಯಕ್ಕೆ
ಸಹಾ-ಯ-ಕ್ಕೊ-ದ-ಗುವ
ಸಹಾ-ಯದ
ಸಹಾ-ಯ-ದಿಂದ
ಸಹಾ-ಯ-ದಿಂ-ದಾಗಿ
ಸಹಾ-ಯ-ಮಾಡಿ
ಸಹಾ-ಯ-ಮಾ-ಡಿ-ಕೊ-ಡು-ತ್ತಿದ್ದ
ಸಹಾ-ಯ-ಮಾ-ಡು-ತ್ತದೆ
ಸಹಾ-ಯ-ವ-ನ್ನ-ರಿಸಿ
ಸಹಾ-ಯ-ವನ್ನು
ಸಹಾ-ಯ-ವನ್ನೂ
ಸಹಾ-ಯ-ವ-ನ್ನೆಲ್ಲ
ಸಹಾ-ಯ-ವಾ-ಗ-ಬಹು
ಸಹಾ-ಯ-ವಾ-ಗ-ಬ-ಹುದು
ಸಹಾ-ಯ-ವಾ-ಗ-ಲಿ-ಲ್ಲವೊ
ಸಹಾ-ಯ-ವಾಗಿ
ಸಹಾ-ಯ-ವಾ-ಗಿ-ದ್ದರೆ
ಸಹಾ-ಯ-ವಾ-ಗು-ತ್ತದೆ
ಸಹಾ-ಯ-ವಾ-ದರು
ಸಹಾ-ಯ-ವಾ-ಯಿತು
ಸಹಾ-ಯ-ವಿಲ್ಲ
ಸಹಾ-ಯ-ವಿ-ಲ್ಲದೆ
ಸಹಾ-ಯವು
ಸಹಾ-ಯವೂ
ಸಹಾ-ಯ-ಹಸ್ತ
ಸಹಾ-ಯಾರ್ಥ
ಸಹಾ-ಯಾ-ರ್ಥ-ವಾಗಿ
ಸಹಾಯ್
ಸಹಿ
ಸಹಿತ
ಸಹಿ-ತ-ವಾಗಿ
ಸಹಿ-ಷ್ಣುತಾ
ಸಹಿ-ಷ್ಣುತೆ
ಸಹಿ-ಷ್ಣು-ತೆ-ಒ-ಟ್ಟಿ-ನಲ್ಲಿ
ಸಹಿ-ಷ್ಣು-ತೆಗೆ
ಸಹಿ-ಷ್ಣು-ತೆಯ
ಸಹಿ-ಷ್ಣು-ತೆ-ಯನ್ನು
ಸಹಿ-ಷ್ಣು-ತೆ-ಯನ್ನೂ
ಸಹಿ-ಷ್ಣು-ತೆಯೇ
ಸಹಿ-ಸದೆ
ಸಹಿ-ಸ-ಲಾ-ರದೆ
ಸಹಿ-ಸಲು
ಸಹಿಸಿ
ಸಹಿ-ಸಿ-ಕೊಂ-ಡರು
ಸಹಿ-ಸಿ-ಕೊಂ-ಡಿ-ದ್ದರು
ಸಹಿ-ಸಿ-ಕೊ-ಳ್ಳ-ಲಾ-ಗ-ಲಿಲ್ಲ
ಸಹಿ-ಸಿ-ಕೊ-ಳ್ಳಲು
ಸಹಿ-ಸು-ತ್ತಿ-ರ-ಲಿಲ್ಲ
ಸಹೃ-ದತೆ
ಸಹೃ-ದಯ
ಸಹೃ-ದ-ಯ-ತೆ-ಯನ್ನು
ಸಹೃ-ದ-ಯ-ತೆ-ಯನ್ನೂ
ಸಹೃ-ದ-ಯರ
ಸಹೋ-ದರ
ಸಹೋ-ದ-ರಿ-ಯ-ರ-ಲ್ಲೊ-ಬ್ಬ-ಳಾದ
ಸಹೋ-ದ-ರಿ-ಯ-ರಾದ
ಸಹ್ಯ-ವಾ-ಗ-ಲಿಲ್ಲ
ಸಾಂಕೇ-ತಿಕ
ಸಾಂಕ್ರಾ-ಮಿಕ
ಸಾಂಖ್ಯ
ಸಾಂಖ್ಯಂ
ಸಾಂಖ್ಯ-ತ-ತ್ತ್ವದ
ಸಾಂಖ್ಯ-ದ-ರ್ಶ-ನದ
ಸಾಂಖ್ಯವೇ
ಸಾಂಗ್ಲಿಯ
ಸಾಂತ
ಸಾಂತ-ರೂಪ
ಸಾಂತ್ವನ
ಸಾಂತ್ವ-ನ-ಕ್ಕಾಗಿ
ಸಾಂದ-ರ್ಭಿ-ಕ-ವಾಗಿ
ಸಾಂದ್ರ-ತೆ-ಗಳಿಂದ
ಸಾಂದ್ರ-ತೆ-ಯಲ್ಲಿ
ಸಾಂಪ್ರ-ದಾ-ಯಿಕ
ಸಾಂಪ್ರ-ದಾ-ಯಿ-ಕ-ವಾಗಿ
ಸಾಂಸ್ಕೃ-ತಿಕ
ಸಾಂಸ್ಕೃ-ತಿ-ಕ-ಆ-ಧ್ಯಾ-ತ್ಮಿಕ
ಸಾಕಪ್ಪ
ಸಾಕ-ಷ್ಟಿ-ಲ್ಲ-ವಲ್ಲ
ಸಾಕಷ್ಟು
ಸಾಕಾ-ಗ-ಲಾ-ರದು
ಸಾಕಾಗಿ
ಸಾಕಾ-ಗಿತ್ತು
ಸಾಕಾ-ಗಿ-ಹೋ-ಗಿತ್ತು
ಸಾಕಾ-ಗಿ-ಹೋ-ಗಿದೆ
ಸಾಕಾ-ಗಿ-ಹೋ-ಗಿ-ದ್ದುವು
ಸಾಕಾ-ಗು-ವು-ದಿಲ್ಲ
ಸಾಕಾದ
ಸಾಕಾರ
ಸಾಕಾ-ರ-ವಾ-ದ-ದಲ್ಲಿ
ಸಾಕಾ-ರ-ಸ್ಥಿ-ತಿಗೆ
ಸಾಕಿದ್ದ
ಸಾಕು
ಸಾಕು-ಅದೂ
ಸಾಕು-ತ್ತಿ-ದ್ದರು
ಸಾಕುವ
ಸಾಕು-ಸಾ-ಕಾ-ಗು-ವಷ್ಟು
ಸಾಕು-ಸಾ-ಕೆ-ನಿ-ಸಿತ್ತು
ಸಾಕು-ಸಾ-ಕೆ-ನಿ-ಸು-ವಷ್ಟು
ಸಾಕ್ಷಾತ್
ಸಾಕ್ಷಾ-ತ್ಕ-ರಿ-ಸಿ-ಕೊಂಡ
ಸಾಕ್ಷಾ-ತ್ಕ-ರಿ-ಸಿ-ಕೊಂ-ಡರು
ಸಾಕ್ಷಾ-ತ್ಕ-ರಿ-ಸಿ-ಕೊಂ-ಡ-ವ-ರಿಗೆ
ಸಾಕ್ಷಾ-ತ್ಕ-ರಿ-ಸಿ-ಕೊಂ-ಡಿ-ದ್ದರು
ಸಾಕ್ಷಾ-ತ್ಕ-ರಿ-ಸಿ-ಕೊಂ-ಡಿ-ರುವ
ಸಾಕ್ಷಾ-ತ್ಕ-ರಿ-ಸಿ-ಕೊಂ-ಡಿ-ರು-ವು-ದನ್ನೂ
ಸಾಕ್ಷಾ-ತ್ಕ-ರಿ-ಸಿ-ಕೊಂಡು
ಸಾಕ್ಷಾ-ತ್ಕ-ರಿ-ಸಿ-ಕೊ-ಳ್ಳ-ಬ-ಹುದು
ಸಾಕ್ಷಾ-ತ್ಕ-ರಿ-ಸಿ-ಕೊ-ಳ್ಳ-ಬೇ-ಕಾ-ದರೆ
ಸಾಕ್ಷಾ-ತ್ಕ-ರಿ-ಸಿ-ಕೊ-ಳ್ಳಲು
ಸಾಕ್ಷಾ-ತ್ಕ-ರಿ-ಸಿ-ಕೊ-ಳ್ಳು-ವಲ್ಲಿ
ಸಾಕ್ಷಾ-ತ್ಕ-ರಿ-ಸಿ-ಕೊ-ಳ್ಳು-ವುದೇ
ಸಾಕ್ಷಾ-ತ್ಕಾರ
ಸಾಕ್ಷಾ-ತ್ಕಾ-ರಕ್ಕೆ
ಸಾಕ್ಷಾ-ತ್ಕಾ-ರದ
ಸಾಕ್ಷಾ-ತ್ಕಾ-ರ-ವನ್ನು
ಸಾಕ್ಷಾ-ತ್ಕಾ-ರ-ವನ್ನೇ
ಸಾಕ್ಷಿ
ಸಾಕ್ಷಿ-ಯಾ-ಗ-ಬ-ಲ್ಲರು
ಸಾಕ್ಷಿ-ಯಾಗಿ
ಸಾಕ್ಷಿ-ಯಾ-ಗಿದೆ
ಸಾಕ್ಷಿ-ಯಾದ
ಸಾಕ್ಷ್ಯ
ಸಾಕ್ಷ್ಯ-ವಾ-ಗಿತ್ತು
ಸಾಕ್ಷ್ಯ-ವಾ-ಗಿದೆ
ಸಾಕ್ಷ್ಯ-ವಾ-ಗಿ-ದ್ದುವು
ಸಾಗ-ಬೇ-ಕಾ-ಗಿದೆ
ಸಾಗ-ಬೇ-ಕಾ-ದರೆ
ಸಾಗ-ಬೇಕು
ಸಾಗ-ಬೇ-ಕೆಂದು
ಸಾಗ-ರ-ಗಳು
ಸಾಗ-ರದ
ಸಾಗ-ರ-ದ-ಲ್ಲಿನ
ಸಾಗ-ರ-ದಷ್ಟು
ಸಾಗ-ರ-ದಿಂದ
ಸಾಗ-ರ-ದೊ-ಳಕ್ಕೆ
ಸಾಗ-ರ-ದೊ-ಳ-ಗೊಂ-ದಾ-ಗು-ವಂತೆ
ಸಾಗ-ರ-ವನ್ನು
ಸಾಗ-ರ-ವಾದ
ಸಾಗ-ರವೇ
ಸಾಗಿ
ಸಾಗಿ-ತಾ-ದರೂ
ಸಾಗಿತು
ಸಾಗಿತ್ತು
ಸಾಗಿ-ದರು
ಸಾಗಿ-ದರೆ
ಸಾಗಿ-ದ್ದಾಗ
ಸಾಗಿರಿ
ಸಾಗಿ-ಸುವ
ಸಾಗು
ಸಾಗುತ್ತ
ಸಾಗುತ್ತಾ
ಸಾಗು-ತ್ತಾನೆ
ಸಾಗು-ತ್ತಾ-ನೆಯೋ
ಸಾಗುತ್ತಿ
ಸಾಗು-ತ್ತಿತ್ತು
ಸಾಗು-ತ್ತಿ-ದೆ-ಯೆಂ-ಬು-ದನ್ನು
ಸಾಗು-ತ್ತಿದ್ದ
ಸಾಗು-ತ್ತಿ-ದ್ದರು
ಸಾಗು-ತ್ತಿ-ದ್ದಾಗ
ಸಾಗು-ತ್ತಿ-ದ್ದೆವು
ಸಾಗು-ತ್ತಿ-ರುವ
ಸಾಗು-ತ್ತಿ-ರು-ವುದು
ಸಾಗು-ತ್ತಿ-ಲ್ಲ-ವೆಂಬ
ಸಾಗುವ
ಸಾಗು-ವಂತೆ
ಸಾಗು-ವ-ವ-ಳ-ಲ್ಲ-ವೆಂ-ಬುದು
ಸಾಗು-ವಷ್ಟು
ಸಾಗು-ವು-ದೆಂಬ
ಸಾಚಾ
ಸಾತ್ತ್ವಿಕ
ಸಾತ್ವಿಕ
ಸಾತ್ವಿ-ಕ-ಭಾ-ವ-ದಿಂದ
ಸಾದ್ಯವೋ
ಸಾಧ-ಕನು
ಸಾಧ-ಕರ
ಸಾಧ-ಕ-ರಾಗಿ
ಸಾಧ-ಕ-ರಿಗೆ
ಸಾಧ-ನ-ಗ-ಳಷ್ಟೆ
ಸಾಧ-ನಾ-ಪ-ಥ-ಗಳನ್ನೂ
ಸಾಧ-ನಾ-ವಿ-ಧಾ-ನ-ಗಳನ್ನು
ಸಾಧನೆ
ಸಾಧ-ನೆ-ಗಳ
ಸಾಧ-ನೆ-ಗ-ಳ-ನ್ನಾ-ಗಲಿ
ಸಾಧ-ನೆ-ಗಳನ್ನು
ಸಾಧ-ನೆ-ಗಳನ್ನೂ
ಸಾಧ-ನೆ-ಗಳಲ್ಲಿ
ಸಾಧ-ನೆ-ಗಳು
ಸಾಧ-ನೆ-ಗ-ಳೆ-ಲ್ಲ-ವನ್ನೂ
ಸಾಧ-ನೆ-ಗಾಗಿ
ಸಾಧ-ನೆ-ಗಿಳಿ-ಸ-ಬೇ-ಕಾ-ದರೆ
ಸಾಧ-ನೆಗೆ
ಸಾಧ-ನೆಯ
ಸಾಧ-ನೆ-ಯನ್ನು
ಸಾಧ-ನೆ-ಯಲ್ಲ
ಸಾಧ-ನೆ-ಯಲ್ಲಿ
ಸಾಧ-ನೆ-ಯಿಂದ
ಸಾಧ-ನೆಯು
ಸಾಧ-ನೆಯೂ
ಸಾಧ-ನೆ-ಯೆಲ್ಲ
ಸಾಧಾ-ರಣ
ಸಾಧಾ-ರ-ಣದ
ಸಾಧಾ-ರ-ಣ-ಮ-ಟ್ಟದ್ದು
ಸಾಧಾ-ರ-ಣ-ವಾಗಿ
ಸಾಧಿ-ಸ-ಬಲ್ಲ
ಸಾಧಿ-ಸ-ಬ-ಹುದು
ಸಾಧಿ-ಸ-ಬ-ಹು-ದು-ಬೇ-ಕಾ-ದರೆ
ಸಾಧಿ-ಸ-ಬ-ಹು-ದೆಂ-ಬುದು
ಸಾಧಿ-ಸ-ಬ-ಹುದೋ
ಸಾಧಿ-ಸ-ಬೇ-ಕಾ-ಗಿದೆ
ಸಾಧಿ-ಸ-ಬೇ-ಕಾ-ಗಿದ್ದ
ಸಾಧಿ-ಸ-ಬೇ-ಕಾದ
ಸಾಧಿ-ಸ-ಬೇ-ಕಾ-ದು-ದೇನು
ಸಾಧಿ-ಸ-ಬೇ-ಕೆಂಬ
ಸಾಧಿ-ಸ-ಲಾ-ಗಿ-ಲ್ಲ-ವೆಂಬ
ಸಾಧಿ-ಸ-ಲಾ-ರಿರಿ
ಸಾಧಿ-ಸಲು
ಸಾಧಿಸಿ
ಸಾಧಿ-ಸಿ-ಕೊಂ-ಡ-ವ-ನಿಗೆ
ಸಾಧಿ-ಸಿ-ಕೊ-ಳ್ಳ-ಲಿ-ಕ್ಕಿಲ್ಲ
ಸಾಧಿ-ಸಿ-ಕೊ-ಳ್ಳಲು
ಸಾಧಿ-ಸಿ-ಕೊ-ಳ್ಳು-ತ್ತಾನೆ
ಸಾಧಿ-ಸಿದ
ಸಾಧಿ-ಸಿ-ದುದು
ಸಾಧಿ-ಸಿ-ದೆನೋ
ಸಾಧಿ-ಸಿದ್ದ
ಸಾಧಿ-ಸಿದ್ದು
ಸಾಧಿ-ಸಿ-ಬಿಟ್ಟ
ಸಾಧಿ-ಸಿ-ಬಿ-ಡ-ಬ-ಲ್ಲುವು
ಸಾಧಿ-ಸಿ-ಬಿ-ಡು-ತ್ತೇ-ನೆಂದು
ಸಾಧಿ-ಸಿಲ್ಲ
ಸಾಧಿ-ಸು-ತ್ತೇನೆ
ಸಾಧಿ-ಸುವ
ಸಾಧಿ-ಸು-ವತ್ತ
ಸಾಧಿ-ಸು-ವ-ವ-ರೆಗೂ
ಸಾಧಿ-ಸು-ವುದ
ಸಾಧಿ-ಸು-ವು-ದರ
ಸಾಧಿ-ಸು-ವು-ದ-ರಲ್ಲಿ
ಸಾಧಿ-ಸು-ವುದು
ಸಾಧು
ಸಾಧು-ಸಂ-ನ್ಯಾ-ಸಿ-ಗ-ಳೆಂ-ದರೆ
ಸಾಧು-ಗಳ
ಸಾಧು-ಗ-ಳಂತೆ
ಸಾಧು-ಗಳನ್ನೆಲ್ಲ
ಸಾಧು-ಗ-ಳಾ-ಗಿದ್ದು
ಸಾಧು-ಗ-ಳಿಗೆ
ಸಾಧು-ಗಳು
ಸಾಧು-ಗ-ಳು-ಸಂ-ನ್ಯಾ-ಸಿ-ಗ-ಳೆಲ್ಲ
ಸಾಧು-ಗಳೂ
ಸಾಧು-ಗ-ಳೆ-ನ್ನಿ-ಸಿ-ಕೊಂ-ಡ-ವರೆಲ್ಲ
ಸಾಧು-ಗ-ಳೆಲ್ಲ
ಸಾಧು-ಗಳೇ
ಸಾಧು-ಗ-ಳೊ-ಬ್ಬರ
ಸಾಧು-ಗ-ಳೊ-ಬ್ಬರು
ಸಾಧು-ಜೀ-ವನ
ಸಾಧು-ಜೀ-ವ-ನ-ವನ್ನು
ಸಾಧು-ನಾನು
ಸಾಧು-ಬೈ-ರಾ-ಗಿ-ಗಳು
ಸಾಧು-ವನ್ನು
ಸಾಧು-ವಾ-ಗಿದ್ದ
ಸಾಧು-ವಿನ
ಸಾಧು-ವಿ-ನಂತೆ
ಸಾಧು-ವಿ-ರ-ಬೇಕು
ಸಾಧು-ವೊಬ್ಬ
ಸಾಧು-ಸಂ-ತರು
ಸಾಧು-ಸಂ-ನ್ಯಾ-ಸಿ-ಗಳನ್ನೂ
ಸಾಧು-ಸಂ-ನ್ಯಾ-ಸಿ-ಗ-ಳೊಂ-ದಿಗೆ
ಸಾಧು-ಸ್ವ-ಭಾ-ವ-ದ-ವನು
ಸಾಧ್ಯ
ಸಾಧ್ಯತೆ
ಸಾಧ್ಯ-ತೆ-ಗ-ಳ-ಡ-ಗಿ-ರುವ
ಸಾಧ್ಯ-ತೆ-ಗಳು
ಸಾಧ್ಯ-ತೆ-ಯ-ಡ-ಗಿ-ರು-ವ-ವರು
ಸಾಧ್ಯ-ತೆ-ಯಿದೆ
ಸಾಧ್ಯ-ವಾಗ
ಸಾಧ್ಯ-ವಾ-ಗ-ದಿ-ದ್ದರೆ
ಸಾಧ್ಯ-ವಾ-ಗ-ದಿ-ದ್ದ-ವ-ರಿಗೆ
ಸಾಧ್ಯ-ವಾ-ಗ-ದಿ-ದ್ದುದು
ಸಾಧ್ಯ-ವಾ-ಗದೆ
ಸಾಧ್ಯ-ವಾ-ಗ-ಬ-ಹು-ದೆಂದು
ಸಾಧ್ಯ-ವಾ-ಗ-ಬೇ-ಕಾ-ದರೆ
ಸಾಧ್ಯ-ವಾ-ಗ-ಬೇಕು
ಸಾಧ್ಯ-ವಾ-ಗ-ಲಿಲ್ಲ
ಸಾಧ್ಯ-ವಾ-ಗ-ಲಿವೆ
ಸಾಧ್ಯ-ವಾ-ಗಿದೆ
ಸಾಧ್ಯ-ವಾ-ಗಿ-ದೆಯೆ
ಸಾಧ್ಯ-ವಾ-ಗಿ-ದ್ದರೆ
ಸಾಧ್ಯ-ವಾ-ಗಿ-ದ್ದುದು
ಸಾಧ್ಯ-ವಾ-ಗಿ-ರ-ಲಿಲ್ಲ
ಸಾಧ್ಯ-ವಾ-ಗಿ-ರು-ವು-ದ-ಕ್ಕಾಗಿ
ಸಾಧ್ಯ-ವಾ-ಗಿ-ರು-ವು-ದ-ರಿಂದ
ಸಾಧ್ಯ-ವಾ-ಗಿಲ್ಲ
ಸಾಧ್ಯ-ವಾ-ಗಿ-ಸು-ವು-ದ-ಕ್ಕಾಗಿ
ಸಾಧ್ಯ-ವಾಗು
ಸಾಧ್ಯ-ವಾ-ಗುತ್ತ
ಸಾಧ್ಯ-ವಾ-ಗು-ತ್ತದೆ
ಸಾಧ್ಯ-ವಾ-ಗು-ತ್ತ-ದೆಂದು
ಸಾಧ್ಯ-ವಾ-ಗು-ತ್ತಿ-ರ-ಲಿಲ್ಲ
ಸಾಧ್ಯ-ವಾ-ಗು-ತ್ತಿ-ರ-ಲಿ-ಲ್ಲವೋ
ಸಾಧ್ಯ-ವಾ-ಗು-ತ್ತಿಲ್ಲ
ಸಾಧ್ಯ-ವಾ-ಗು-ವಂ-ತಹ
ಸಾಧ್ಯ-ವಾ-ಗು-ವಂ-ತಿ-ರ-ಲಿಲ್ಲ
ಸಾಧ್ಯ-ವಾ-ಗು-ವಂತೆ
ಸಾಧ್ಯ-ವಾ-ಗು-ವಂ-ಥ-ದಲ್ಲ
ಸಾಧ್ಯ-ವಾ-ಗುವು
ಸಾಧ್ಯ-ವಾ-ಗು-ವು-ದಾ-ದರೆ
ಸಾಧ್ಯ-ವಾ-ಗು-ವು-ದಿಲ್ಲ
ಸಾಧ್ಯ-ವಾ-ಗು-ವು-ದಿ-ಲ್ಲ-ವೆಂದು
ಸಾಧ್ಯ-ವಾ-ಗು-ವುದು
ಸಾಧ್ಯ-ವಾ-ಗು-ವು-ದೆಂದು
ಸಾಧ್ಯ-ವಾ-ಗು-ವುದೇ
ಸಾಧ್ಯ-ವಾದ
ಸಾಧ್ಯ-ವಾ-ದದ್ದು
ಸಾಧ್ಯ-ವಾ-ದದ್ದೂ
ಸಾಧ್ಯ-ವಾ-ದರೆ
ಸಾಧ್ಯ-ವಾ-ದಷ್ಟು
ಸಾಧ್ಯ-ವಾ-ದಷ್ಟೂ
ಸಾಧ್ಯ-ವಾ-ದೀತು
ಸಾಧ್ಯ-ವಾ-ದೀತೆ
ಸಾಧ್ಯ-ವಾ-ದುದು
ಸಾಧ್ಯ-ವಾ-ಯಿತು
ಸಾಧ್ಯ-ವಾ-ಯಿ-ತೆಂ-ದರೆ
ಸಾಧ್ಯ-ವಾ-ಯಿ-ತೆಂ-ಬು-ದನ್ನು
ಸಾಧ್ಯ-ವಿತ್ತು
ಸಾಧ್ಯ-ವಿದೆ
ಸಾಧ್ಯ-ವಿ-ದೆಯೆ
ಸಾಧ್ಯ-ವಿ-ದ್ದರೆ
ಸಾಧ್ಯ-ವಿರ
ಸಾಧ್ಯ-ವಿ-ರ-ಲಿಲ್ಲ
ಸಾಧ್ಯ-ವಿ-ರುವ
ಸಾಧ್ಯ-ವಿ-ರು-ವಂತೆ
ಸಾಧ್ಯ-ವಿಲ್ಲ
ಸಾಧ್ಯ-ವಿ-ಲ್ಲ-ದಂ-ತಹ
ಸಾಧ್ಯ-ವಿ-ಲ್ಲ-ದಂತೆ
ಸಾಧ್ಯ-ವಿ-ಲ್ಲ-ದ-ವ-ರಿಗೆ
ಸಾಧ್ಯ-ವಿ-ಲ್ಲ-ದಾಗ
ಸಾಧ್ಯ-ವಿ-ಲ್ಲ-ದಿ-ರು-ವಂ-ತಹ
ಸಾಧ್ಯ-ವಿ-ಲ್ಲ-ದು-ದನ್ನು
ಸಾಧ್ಯ-ವಿ-ಲ್ಲ-ದುದು
ಸಾಧ್ಯ-ವಿ-ಲ್ಲ-ವಲ್ಲ
ಸಾಧ್ಯ-ವಿ-ಲ್ಲವೆ
ಸಾಧ್ಯ-ವಿ-ಲ್ಲ-ವೆಂದು
ಸಾಧ್ಯ-ವಿ-ಲ್ಲವೋ
ಸಾಧ್ಯವೂ
ಸಾಧ್ಯವೆ
ಸಾಧ್ಯ-ವೆಂದು
ಸಾಧ್ಯ-ವೆಂಬ
ಸಾಧ್ಯ-ವೆಂಬು
ಸಾಧ್ಯ-ವೆಂ-ಬು-ದನ್ನು
ಸಾಧ್ಯವೇ
ಸಾಧ್ಯ-ವೇನು
ಸಾಧ್ಯವೋ
ಸಾಧ್ವೀ-ಮಣಿ
ಸಾನ್ನಿಧ್ಯ
ಸಾನ್ನಿ-ಧ್ಯದ
ಸಾನ್ನಿ-ಧ್ಯ-ದಲ್ಲಿ
ಸಾನ್ನಿ-ಧ್ಯ-ದ-ಲ್ಲಿಯೂ
ಸಾನ್ನಿ-ಧ್ಯ-ದ-ಲ್ಲಿ-ರು-ವ-ವರು
ಸಾನ್ನಿ-ಧ್ಯ-ದಿಂ-ದಾಗಿ
ಸಾನ್ನಿ-ಧ್ಯವು
ಸಾನ್ನಿ-ಧ್ಯವೇ
ಸಾಪೇಕ್ಷ
ಸಾಪ್ತಾ-ಹಿಕ
ಸಾಬೀ-ತಾ-ಗಿತ್ತು
ಸಾಬೀ-ತಾ-ಗಿ-ದ್ದರೂ
ಸಾಬೀ-ತಾದ
ಸಾಬೀ-ತಾ-ಯಿತು
ಸಾಬೀತು
ಸಾಬೀ-ತು-ಗೊ-ಳಿಸು
ಸಾಬೀ-ತು-ಪ-ಡಿ-ಸ-ಬಲ್ಲೆ
ಸಾಬೀ-ತು-ಪ-ಡಿ-ಸಲು
ಸಾಬೀ-ತು-ಪ-ಡಿಸು
ಸಾಮ-ಗ್ರಿ-ಗಳು
ಸಾಮ-ರ-ಸ್ಯ-ಗಳ
ಸಾಮ-ರ-ಸ್ಯದ
ಸಾಮ-ರ-ಸ್ಯ-ದಿಂದ
ಸಾಮ-ರ-ಸ್ಯ-ದಿಂ-ದಿ-ರ-ಲಾರೆ
ಸಾಮರ್ಥ್ಯ
ಸಾಮ-ರ್ಥ್ಯ-ಆ-ಧ್ಯಾ-ತ್ಮಿಕ
ಸಾಮ-ರ್ಥ್ಯ-ಅ-ರ್ಥ-ವಾ-ಯಿತೆ
ಸಾಮ-ರ್ಥ್ಯ-ಕ್ಕ-ನು-ಗು-ಣ-ವಾಗಿ
ಸಾಮ-ರ್ಥ್ಯ-ಕ್ಕಿಂ-ತಲೂ
ಸಾಮ-ರ್ಥ್ಯದ
ಸಾಮ-ರ್ಥ್ಯ-ದಿಂದ
ಸಾಮ-ರ್ಥ್ಯ-ದಿಂ-ದಲೂ
ಸಾಮ-ರ್ಥ್ಯ-ವನ್ನು
ಸಾಮ-ರ್ಥ್ಯ-ವನ್ನೂ
ಸಾಮ-ರ್ಥ್ಯ-ವಿದೆ
ಸಾಮ-ರ್ಥ್ಯ-ವಿ-ರುವ
ಸಾಮ-ರ್ಥ್ಯ-ವಿ-ಲ್ಲದ
ಸಾಮ-ರ್ಥ್ಯವೂ
ಸಾಮ-ರ್ಥ್ಯ-ವೇನು
ಸಾಮ-ರ್ಥ್ಯ-ಶಾ-ಲಿ-ಗಳು
ಸಾಮ-ರ್ಥ್ಯ-ಶಾ-ಲಿ-ಯಾ-ಗಿ-ರ-ಬೇಕು
ಸಾಮ-ರ್ಥ್ಯ-ಶಾಲೀ
ಸಾಮಾ-ಜಿಕ
ಸಾಮಾ-ಜಿ-ಕರ
ಸಾಮಾ-ಜಿ-ಕ-ವಾಗಿ
ಸಾಮಾನು
ಸಾಮಾ-ನು-ಸ-ರಂ-ಜಾ-ಮು-ಗಳು
ಸಾಮಾ-ನು-ಗಳ
ಸಾಮಾ-ನು-ಗಳನ್ನು
ಸಾಮಾ-ನು-ಗ-ಳೆಂ-ದರೆ
ಸಾಮಾನ್ಯ
ಸಾಮಾ-ನ್ಯ-ರಂತೆ
ಸಾಮಾ-ನ್ಯ-ರಂ-ತೆಯೇ
ಸಾಮಾ-ನ್ಯ-ರನ್ನು
ಸಾಮಾ-ನ್ಯ-ರಿಂದ
ಸಾಮಾ-ನ್ಯರು
ಸಾಮಾ-ನ್ಯರೂ
ಸಾಮಾ-ನ್ಯ-ವಾಗಿ
ಸಾಮಾ-ನ್ಯ-ವಾದ
ಸಾಮಾ-ನ್ಯ-ವಾ-ದ-ದ್ದಾ-ದರೂ
ಸಾಮೀ-ಪ್ಯ-ದ-ಲ್ಲಿ-ದ್ದಂತೆ
ಸಾಮು-ದ್ರಿಕ
ಸಾಮ್ಯ-ಭೇ-ದ-ಗಳನ್ನು
ಸಾಮ್ಯ-ಭೇ-ದ-ಗಳು
ಸಾಮ್ಯ-ವನ್ನೂ
ಸಾಮ್ರಾ-ಜ್ಯ-ಗಳ
ಸಾಮ್ರಾ-ಜ್ಯ-ಗಳನ್ನು
ಸಾಮ್ರಾ-ಜ್ಯ-ವನ್ನು
ಸಾಮ್ರಾ-ಜ್ಯ-ಶಾಹೀ
ಸಾಮ್ರಾಟ್
ಸಾಯಂ-ಕಾ-ಲದ
ಸಾಯಂ-ಕಾ-ಲ-ದಲ್ಲಿ
ಸಾಯ-ಣಾ-ಚಾರ್ಯ
ಸಾಯ-ಬ-ಹುದು
ಸಾಯ-ಬೇ-ಕಾ-ಗಿತ್ತು
ಸಾಯ-ಲಾರೆ
ಸಾಯ-ಲೇ-ಬೇಕು
ಸಾಯ-ವುದೇ
ಸಾಯು-ತ್ತಾ-ರಂ-ತಲ್ಲ
ಸಾಯು-ತ್ತಿ-ರುವ
ಸಾಯು-ತ್ತೇ-ನೆಯೋ
ಸಾಯು-ವ-ವರ
ಸಾಯು-ವು-ದ-ಕ್ಕಾಗಿ
ಸಾಯು-ವುದೂ
ಸಾರ
ಸಾರ-ಥಿ-ಯಾ-ಗಿ-ರಲು
ಸಾರದಾ
ಸಾರ-ದಾನ
ಸಾರ-ಬಲ್ಲ
ಸಾರ-ಬೇ-ಕಾ-ದರೆ
ಸಾರ-ಬೇಕು
ಸಾರ-ರೂ-ಪದ
ಸಾರ-ಲಿದ್ದ
ಸಾರ-ಲಿಲ್ಲ
ಸಾರಲು
ಸಾರ-ಲ್ಪಟ್ಟ
ಸಾರ-ವ-ಡ-ಗಿ-ರು-ವುದು
ಸಾರ-ವನ್ನು
ಸಾರ-ವನ್ನೇ
ಸಾರ-ವಾಗಿ
ಸಾರ-ವೆಂದರೆ
ಸಾರ-ಸ-ರ್ವಸ್ವ
ಸಾರ-ಸ-ರ್ವ-ಸ್ವ-ವಾದ
ಸಾರ-ಸ-ರ್ವ-ಸ್ವ-ವೆಂದೂ
ಸಾರಾ
ಸಾರಾಂಶ
ಸಾರಾಂ-ಶ-ವನ್ನು
ಸಾರಾಂ-ಶ-ವನ್ನೂ
ಸಾರಾಂ-ಶ-ವೇನೆಂದರೆ
ಸಾರಾ-ಎ-ಲೆ-ನ್ಮಾ-ಲ್ಡೊ-ಳಿಗೆ
ಸಾರಾದಾ
ಸಾರಾ-ಬುಲ್ಗೆ
ಸಾರಾ-ಬು-ಲ್ಲ-ಳಿಗೆ
ಸಾರಾ-ಳಿಗೆ
ಸಾರಿ
ಸಾರಿತು
ಸಾರಿದ
ಸಾರಿ-ದರು
ಸಾರಿದ್ದ
ಸಾರಿ-ದ್ದರು
ಸಾರಿ-ಹೇ-ಳು-ತ್ತದೆ
ಸಾರುತ್ತ
ಸಾರು-ತ್ತದೆ
ಸಾರು-ತ್ತ-ದೆ-ಎಂದು
ಸಾರು-ತ್ತ-ದೆ-ಸರ್ವಂ
ಸಾರು-ತ್ತಾರೆ
ಸಾರು-ತ್ತಿತ್ತು
ಸಾರು-ತ್ತಿ-ದ್ದರು
ಸಾರುವ
ಸಾರು-ವ-ವ-ರಿ-ದ್ದಾರೆ
ಸಾರು-ವಷ್ಟು
ಸಾರುವು
ಸಾರೋ-ಟಿ-ನಲ್ಲಿ
ಸಾರೋಟು
ಸಾರೋ-ಣವೆ
ಸಾರ್ಥಕ
ಸಾರ್ಥ-ಕ-ವ-ಲ್ಲವೆ
ಸಾರ್ವ
ಸಾರ್ವ-ಜ-ನಿಕ
ಸಾರ್ವ-ಜ-ನಿ-ಕರ
ಸಾರ್ವ-ಜ-ನಿ-ಕ-ರ-ನ್ನು-ದ್ದೇ-ಶಿಸಿ
ಸಾರ್ವ-ಜ-ನಿ-ಕ-ರಿಗೆ
ಸಾರ್ವ-ಜ-ನಿ-ಕರು
ಸಾರ್ವ-ಜ-ನಿ-ಕ-ವಾಗಿ
ಸಾರ್ವ-ಜ-ನಿ-ಕ-ವಾ-ಗಿ-ಬಿ-ಟ್ಟಿ-ದ್ದೇನೆ
ಸಾರ್ವ-ಜ-ನಿ-ಕ-ವಾ-ಗಿಯೇ
ಸಾರ್ವ-ತ್ರಿಕ
ಸಾರ್ವ-ತ್ರಿ-ಕ-ವಾಗಿ
ಸಾರ್ವ-ತ್ರಿ-ಕ-ವಾದ
ಸಾರ್ವ-ತ್ರೀ-ಕ-ರಿ-ಸ-ಬ-ಹು-ದಾದ
ಸಾರ್ವ-ತ್ರೀ-ಕ-ರಿಸಿ
ಸಾಲ
ಸಾಲಂ-ಕೃತ
ಸಾಲ-ದಕ್ಕೆ
ಸಾಲದು
ಸಾಲ-ದು-ಧ-ರ್ಮ-ಸ್ವ-ರೂ-ಪಿ-ಗಳೇ
ಸಾಲದೆ
ಸಾಲ-ದ್ದಕ್ಕೆ
ಸಾಲ-ಲಿಲ್ಲ
ಸಾಲಾ-ರ್ಜಂಗ್
ಸಾಲಿಗೆ
ಸಾಲಿನ
ಸಾಲು
ಸಾಲು-ಗಳನ್ನು
ಸಾಲು-ಗಳು
ಸಾಲು-ತ್ತಿ-ತ್ತಷ್ಟೆ
ಸಾಲು-ತ್ತಿ-ರ-ಲಿಲ್ಲ
ಸಾಲು-ಮ-ನೆ-ಗಳಿಂದ
ಸಾಲು-ಮ-ನೆ-ಗಳು
ಸಾಲ್ಸೆತ್ತೆ
ಸಾವ-ಕಾ-ಶ-ವಾಗಿ
ಸಾವ-ಧಾ-ನ-ಗೊಂಡು
ಸಾವ-ನ್ನ-ಪ್ಪಲು
ಸಾವ-ರಿ-ಸಿ-ಕೊಂಡು
ಸಾವ-ರಿ-ಸಿ-ಕೊಂ-ಡೆದ್ದು
ಸಾವಿನ
ಸಾವಿ-ನಿಂದ
ಸಾವಿರ
ಸಾವಿ-ರ-ಕ್ಕಿಂ-ತಲೂ
ಸಾವಿ-ರಕ್ಕೂ
ಸಾವಿ-ರದ
ಸಾವಿ-ರ-ದ-ಲ್ಲೊಂ-ದು-ಪಾಲು
ಸಾವಿ-ರಾರು
ಸಾವು
ಸಾವೂ
ಸಾವೆಂದರೆ
ಸಾಷ್ಟಾಂಗ
ಸಾಸಿ-ರ-ಮಡಿ
ಸಾಸಿ-ರ-ಮ-ಡಿ-ಯಾಗಿ
ಸಾಸಿವೆ
ಸಾಸ್
ಸಾಸ್-ಫೀ
ಸಾಸ್ಫೀ
ಸಾಸ್ಫೀಗೆ
ಸಾಸ್ಫೀ-ಯಲ್ಲಿ
ಸಾಹಸ
ಸಾಹ-ಸದ
ಸಾಹ-ಸ-ಮಯ
ಸಾಹ-ಸ-ವನ್ನು
ಸಾಹ-ಸವೇ
ಸಾಹಸಿ
ಸಾಹ-ಸಿ-ಗಳೂ
ಸಾಹಿತ್ಯ
ಸಾಹಿ-ತ್ಯ-ವಿ-ಜ್ಞಾ-ನ-ಇ-ತಿ-ಹಾ-ಸ-ಧ-ರ್ಮ-ಅ-ಧ್ಯಾ-ತ್ಮ-ಗ-ಳಿಗೆ
ಸಾಹಿ-ತ್ಯ-ಕ್ಕಾಗಿ
ಸಾಹಿ-ತ್ಯ-ಗಳ
ಸಾಹಿ-ತ್ಯ-ಗ-ಳ-ಲ್ಲಲ್ಲ
ಸಾಹಿ-ತ್ಯ-ಗಳಲ್ಲಿ
ಸಾಹಿ-ತ್ಯದ
ಸಾಹಿ-ತ್ಯ-ದಿಂದ
ಸಾಹಿ-ತ್ಯ-ವನ್ನು
ಸಾಹು-ಕಾ-ರರು
ಸಾಹೇಬ
ಸಾಹೇ-ಬ-ನಿಗೆ
ಸಾಹೇ-ಬನೂ
ಸಾಹೇ-ಬ-ನೊಂ-ದಿಗೆ
ಸಾಹೇ-ಬರ
ಸಾಹೇ-ಬ-ರಿಗೆ
ಸಾಹೇ-ಬರು
ಸಾಹೇ-ಬರೇ
ಸಾಹೇಬ್
ಸಿ
ಸಿಂಗನ
ಸಿಂಗ-ಪು-ರ-ವನ್ನು
ಸಿಂಗ-ರಿ-ಸಿ-ಕೊಂಡು
ಸಿಂಗಾ-ಲ-ವೇಲು
ಸಿಂಗ್
ಸಿಂಹದ
ಸಿಂಹ-ದಂತೆ
ಸಿಂಹ-ಳದ
ಸಿಂಹ-ವನ್ನು
ಸಿಂಹ-ವಾ-ದರು
ಸಿಂಹ-ಸ-ದೃಶ
ಸಿಂಹ-ಸ್ಮಾ-ರಕ
ಸಿಂಹಾ-ಸ-ನದ
ಸಿಕಂ-ದ-ರಾ-ಬಾ-ದಿನ
ಸಿಕ್ಕ
ಸಿಕ್ಕರೆ
ಸಿಕ್ಕಲ್ಲಿ
ಸಿಕ್ಕ-ಸಿಕ್ಕ
ಸಿಕ್ಕಾಗ
ಸಿಕ್ಕಾ-ಯಿತು
ಸಿಕ್ಕಿ
ಸಿಕ್ಕಿ-ಕೊಂ-ಡ-ವರು
ಸಿಕ್ಕಿ-ಕೊಂ-ಡಾ-ಗಿತ್ತು
ಸಿಕ್ಕಿ-ಕೊಂಡು
ಸಿಕ್ಕಿ-ಕೊಂ-ಡು-ಬಿ-ಡು-ತ್ತಾರೋ
ಸಿಕ್ಕಿ-ಕೊ-ಳ್ಳುತ್ತಿ
ಸಿಕ್ಕಿ-ಕೊ-ಳ್ಳು-ತ್ತಿ-ರ-ಲಿಲ್ಲ
ಸಿಕ್ಕಿ-ಕೊ-ಳ್ಳು-ವಂ-ತಹ
ಸಿಕ್ಕಿ-ಕೊ-ಳ್ಳು-ವುದು
ಸಿಕ್ಕಿ-ತಲ್ಲ
ಸಿಕ್ಕಿತು
ಸಿಕ್ಕಿ-ತು-ಅ-ಲ್ಲಿದ್ದ
ಸಿಕ್ಕಿತೆ
ಸಿಕ್ಕಿದ
ಸಿಕ್ಕಿ-ದಂ-ತಾ-ಗು-ತ್ತಿತ್ತು
ಸಿಕ್ಕಿ-ದಂತೆ
ಸಿಕ್ಕಿ-ದ-ರಲ್ಲಿ
ಸಿಕ್ಕಿ-ದರೂ
ಸಿಕ್ಕಿ-ದರೆ
ಸಿಕ್ಕಿ-ದ-ರೆಂದು
ಸಿಕ್ಕಿ-ದ-ವರ
ಸಿಕ್ಕಿ-ದು-ದ-ನ್ನೆಲ್ಲ
ಸಿಕ್ಕಿ-ದುದು
ಸಿಕ್ಕಿದೆ
ಸಿಕ್ಕಿ-ದ್ದನ್ನು
ಸಿಕ್ಕಿ-ದ್ದ-ರಿಂದ
ಸಿಕ್ಕಿ-ದ್ದರೆ
ಸಿಕ್ಕಿದ್ದು
ಸಿಕ್ಕಿ-ದ್ದೆಂ-ದರೆ
ಸಿಕ್ಕಿದ್ದೇ
ಸಿಕ್ಕಿ-ಬಿ-ಡು-ತ್ತದೆ
ಸಿಕ್ಕಿ-ಬಿದ್ದ
ಸಿಕ್ಕಿ-ಬಿ-ದ್ದಿತ್ತು
ಸಿಕ್ಕಿ-ರ-ಲಾ-ರದು
ಸಿಕ್ಕಿ-ರ-ಲಿಲ್ಲ
ಸಿಕ್ಕಿಲ್ಲ
ಸಿಕ್ಕಿವೆ
ಸಿಕ್ಕಿ-ಸಿ-ಕೊಳ್ಳ
ಸಿಕ್ಕಿ-ಹಾ-ಕಿ-ಕೊಂ-ಡಿ-ದ್ದೇನೆ
ಸಿಕ್ಕಿ-ಹಾ-ಕಿ-ಕೊ-ಳ್ಳ-ಬೇ-ಕಾ-ಗಿತ್ತು
ಸಿಕ್ಕೀತು
ಸಿಕ್ಕೇ
ಸಿಖರ್
ಸಿಗ-ದಿ-ರು-ವು-ದ-ರಿಂದ
ಸಿಗ-ದಿ-ರು-ವುದು
ಸಿಗದೆ
ಸಿಗ-ಬ-ಹುದು
ಸಿಗ-ಬ-ಹುದೆ
ಸಿಗ-ಲಾ-ರದು
ಸಿಗ-ಲಿಲ್ಲ
ಸಿಗ-ಲಿ-ಲ್ಲ-ವಾ-ದ್ದ-ರಿಂದ
ಸಿಗಾ-ರು-ಗಳನ್ನು
ಸಿಗು
ಸಿಗು-ತ್ತದೆ
ಸಿಗು-ತ್ತಾರೆ
ಸಿಗು-ತ್ತಾ-ರೆಂ-ಬು-ದ-ರ-ಮೇ-ಲಲ್ಲ
ಸಿಗು-ತ್ತಾರೋ
ಸಿಗು-ತ್ತಿತ್ತು
ಸಿಗು-ತ್ತಿ-ತ್ತು-ದಿ-ನ-ಕ್ಕೊ-ಮ್ಮೆಯೋ
ಸಿಗು-ತ್ತಿದೆ
ಸಿಗು-ತ್ತಿದ್ದ
ಸಿಗು-ತ್ತಿ-ದ್ದುದು
ಸಿಗು-ತ್ತಿ-ರ-ಲಿಲ್ಲ
ಸಿಗುವ
ಸಿಗು-ವಂ-ತಾ-ಗಿದೆ
ಸಿಗು-ವಂತೆ
ಸಿಗು-ವುದೇ
ಸಿಟ್ಟಾ-ಗಲು
ಸಿಟ್ಟಾ-ದ್ದ-ರಿಂದ
ಸಿಟ್ಟಿಗೆ
ಸಿಟ್ಟಿ-ಗೆ-ದ್ದರು
ಸಿಟ್ಟಿ-ಗೆದ್ದು
ಸಿಟ್ಟಿ-ಗೆ-ಬ್ಬಿ-ಸಿತು
ಸಿಟ್ಟಿ-ಗೇಳು
ಸಿಟ್ಟಿನ
ಸಿಟ್ಟಿ-ನಿಂದ
ಸಿಟ್ಟು
ಸಿಟ್ಟೇ-ರಿತು
ಸಿಡಿ
ಸಿಡಿದ
ಸಿಡಿ-ದು-ನಿಂ-ತಾಗ
ಸಿಡಿ-ದು-ಬಿ-ಡು-ತ್ತೇನೋ
ಸಿಡಿ-ದೆ-ದ್ದರು
ಸಿಡಿ-ಯು-ತ್ತಿದ್ದ
ಸಿಡಿ-ಯು-ತ್ತಿ-ದ್ದುವು
ಸಿಡಿ-ಲ-ನ್ನಾ-ಗಿ-ಸು-ತ್ತದೆ
ಸಿಡಿ-ಲನ್ನು
ಸಿಡಿ-ಲಿಗೂ
ಸಿಡಿ-ಲಿ-ನಂತೆ
ಸಿಡಿ-ಲಿ-ನಂ-ಥದೇ
ಸಿಡಿಲು
ಸಿಡಿ-ಲು-ಗ-ಳೊಂ-ದಿಗೆ
ಸಿಡಿ-ಲೆ-ದೆಗೂ
ಸಿತು
ಸಿತು-ಯಾವ
ಸಿದ
ಸಿದರು
ಸಿದಳು
ಸಿದಾಗ
ಸಿದುವು
ಸಿದೆ
ಸಿದ್ದ-ನಿ-ದ್ದೇನೆ
ಸಿದ್ದಾಂ-ತ-ವನ್ನು
ಸಿದ್ಧ
ಸಿದ್ಧ-ಗೊಂ-ಡಿದ್ದು
ಸಿದ್ಧ-ಗೊ-ಳಿ-ಸು-ತ್ತಿ-ದ್ದರು
ಸಿದ್ಧ-ಗೊ-ಳಿ-ಸು-ವುದು
ಸಿದ್ಧ-ತೆ-ಗಳ
ಸಿದ್ಧ-ತೆ-ಗಳನ್ನು
ಸಿದ್ಧ-ತೆ-ಗ-ಳಾ-ಗ-ಬೇ-ಕಿದೆ
ಸಿದ್ಧ-ತೆ-ಗಳು
ಸಿದ್ಧ-ತೆ-ಗಳೂ
ಸಿದ್ಧ-ತೆ-ಯನ್ನು
ಸಿದ್ಧ-ತೆ-ಯಿ-ಲ್ಲದೆ
ಸಿದ್ಧ-ನಾ-ಗ-ತೊ-ಡ-ಗಿ-ದ್ದೇನೆ
ಸಿದ್ಧ-ನಾ-ಗಿ-ದ್ದೇನೆ
ಸಿದ್ಧ-ನಾ-ಗು-ತ್ತಿದ್ದ
ಸಿದ್ಧ-ನಾದ
ಸಿದ್ಧ-ನಿ-ದ್ದಾನೆ
ಸಿದ್ಧ-ನಿ-ದ್ದೇನೆ
ಸಿದ್ಧ-ನಿರ
ಸಿದ್ಧ-ನಿ-ರ-ಲಿಲ್ಲ
ಸಿದ್ಧ-ಪ-ಡಿ-ಸಲಾ
ಸಿದ್ಧ-ಪ-ಡಿ-ಸಿ-ಕೊಂ-ಡರು
ಸಿದ್ಧ-ಪ-ಡಿ-ಸಿ-ಕೊಂ-ಡಿ-ರಲೂ
ಸಿದ್ಧ-ಪ-ಡಿ-ಸಿ-ಕೊಂಡು
ಸಿದ್ಧ-ಪ-ಡಿ-ಸಿ-ಟ್ಟು-ಕೊಂ-ಡಿ-ದ್ದರು
ಸಿದ್ಧ-ಪ-ಡಿ-ಸು-ವುದು
ಸಿದ್ಧ-ಪು-ರು-ಷರು
ಸಿದ್ಧ-ರಾ-ಗ-ತೊ-ಡ-ಗಿ-ದರು
ಸಿದ್ಧ-ರಾಗಿ
ಸಿದ್ಧ-ರಾ-ಗಿ-ದ್ದರು
ಸಿದ್ಧ-ರಾ-ಗಿ-ದ್ದಾರೆ
ಸಿದ್ಧ-ರಾ-ಗಿಯೇ
ಸಿದ್ಧ-ರಾ-ಗಿ-ರ-ಬೇಕು
ಸಿದ್ಧ-ರಾ-ಗಿರು
ಸಿದ್ಧ-ರಾ-ಗು-ತ್ತಾರೆ
ಸಿದ್ಧ-ರಾ-ಗು-ವಂತೆ
ಸಿದ್ಧ-ರಾ-ಗು-ವ-ವ-ರೆಗೂ
ಸಿದ್ಧ-ರಾ-ದರು
ಸಿದ್ಧ-ರಾ-ದ-ವರು
ಸಿದ್ಧ-ರಿ-ದ್ದಂತೆ
ಸಿದ್ಧ-ರಿ-ದ್ದರು
ಸಿದ್ಧ-ರಿ-ದ್ದಾ-ರೆಯೆ
ಸಿದ್ಧ-ರಿ-ದ್ದೀರಿ
ಸಿದ್ಧ-ರಿರ
ಸಿದ್ಧ-ರಿ-ರ-ಲಿಲ್ಲ
ಸಿದ್ಧ-ರಿ-ರು-ತ್ತಿ-ದ್ದರು
ಸಿದ್ಧ-ರಿ-ರು-ತ್ತಿ-ದ್ದರೆ
ಸಿದ್ಧ-ಳಾ-ಗ-ತೊ-ಡ-ಗಿ-ದ್ದರೂ
ಸಿದ್ಧ-ಳಾ-ದಳು
ಸಿದ್ಧ-ಳಿ-ದ್ದೇನೆ
ಸಿದ್ಧ-ಳಿ-ರ-ಲಿಲ್ಲ
ಸಿದ್ಧ-ವಾ-ಗಿ-ದ್ದಾರೆ
ಸಿದ್ಧ-ವಾ-ಗಿದ್ದು
ಸಿದ್ಧ-ವಾ-ಗಿಯೇ
ಸಿದ್ಧ-ವಾ-ಗಿ-ರಲು
ಸಿದ್ಧ-ವಾ-ಗಿ-ರು-ತ್ತ-ವಲ್ಲ
ಸಿದ್ಧ-ವಾ-ಗಿ-ರು-ತ್ತಿತ್ತು
ಸಿದ್ಧ-ವಾ-ಗಿವೆ
ಸಿದ್ಧ-ವಾ-ದಂ-ತಿತ್ತು
ಸಿದ್ಧ-ಹಸ್ತ
ಸಿದ್ಧ-ಹ-ಸ್ತ-ರಾದ
ಸಿದ್ಧಾಂತ
ಸಿದ್ಧಾಂ-ತ-ಗಳ
ಸಿದ್ಧಾಂ-ತ-ಗಳನ್ನು
ಸಿದ್ಧಾಂ-ತ-ಗಳನ್ನೂ
ಸಿದ್ಧಾಂ-ತ-ಗ-ಳ-ಲ್ಲಲ್ಲ
ಸಿದ್ಧಾಂ-ತ-ಗಳಿ
ಸಿದ್ಧಾಂ-ತ-ಗ-ಳಿಗೆ
ಸಿದ್ಧಾಂ-ತ-ಗಳು
ಸಿದ್ಧಾಂ-ತದ
ಸಿದ್ಧಾಂ-ತ-ದಲ್ಲಿ
ಸಿದ್ಧಾಂ-ತ-ವನ್ನು
ಸಿದ್ಧಾಂ-ತವು
ಸಿದ್ಧಿ
ಸಿದ್ಧಿ-ಗಳನ್ನು
ಸಿದ್ಧಿ-ಗಳು
ಸಿದ್ಧಿಯೂ
ಸಿದ್ಧಿ-ಸಿ-ಕೊಂ-ಡ-ವರು
ಸಿದ್ಧಿ-ಸಿ-ಕೊ-ಳ್ಳು-ತ್ತಾರೆ
ಸಿದ್ಧಿ-ಸಿ-ರ-ಲಿಲ್ಲ
ಸಿದ್ಧಿ-ಸು-ತ್ತದೆ
ಸಿದ್ಧಿ-ಸು-ತ್ತವೆ
ಸಿದ್ಧಿ-ಸು-ವುದು
ಸಿನಿ-ಮಾ-ಟಿ
ಸಿಪಾಯಿ
ಸಿಪಾ-ಯಿ-ಗ-ಳಾದ
ಸಿಪಾ-ಯಿ-ದಂ-ಗೆಯ
ಸಿಪಾ-ಯಿ-ಯೊಂ-ದಿಗೆ
ಸಿಮಿ-ಟಿಕ್
ಸಿರ-ಬ-ಹುದೆ
ಸಿರಿ-ರಕ್ಷೆ
ಸಿಲುಕಿ
ಸಿಲು-ಕಿ-ಕೊಂಡ
ಸಿಲು-ಕಿ-ಕೊಂ-ಡರು
ಸಿಲು-ಕಿ-ಕೊಂ-ಡಿದ್ದ
ಸಿಲು-ಕಿ-ಕೊಂಡು
ಸಿಲು-ಕಿ-ಕೊಂ-ಡು-ಬಿ-ಟ್ಟಿ-ದ್ದರು
ಸಿಲು-ಕಿದ
ಸಿಲು-ಕಿ-ರುವ
ಸಿಲು-ಕಿ-ರು-ವುದೇ
ಸಿಲು-ಕು-ವಂ-ತಾ-ದದ್ದು
ಸಿವಾನಿ
ಸಿವಿಲ್
ಸಿಹಿ-ತಿಂಡಿ
ಸಿಹಿ-ತಿಂ-ಡಿ-ಯನ್ನೂ
ಸಿಹಿ-ತಿ-ನಿ-ಸು-ಗಳ
ಸಿಹೋ-ರ್ಗಳನ್ನೂ
ಸೀತೆಯ
ಸೀತೆ-ಯೊಂ-ದಿಗೆ
ಸೀದಾ
ಸೀಮಿತ
ಸೀಮಿ-ತ-ಗೊ-ಳಿ-ಸು-ತ್ತಿ-ರು-ವಿ-ರಲ್ಲ
ಸೀಮಿ-ತ-ವಾ-ಗಿರ
ಸೀಮಿ-ತ-ವಾ-ಗಿ-ರ-ಲಿಲ್ಲ
ಸೀಮಿ-ತ-ವಾ-ಗಿ-ರು-ತ್ತ-ದೆಯೋ
ಸೀಳಿ
ಸೀಳಿ-ಕೊಂಡು
ಸೀಸರ್
ಸುಂಕ-ವಿಲ್ಲ
ಸುಂಟ-ರ-ಗಾಳಿ
ಸುಂಟ-ರ-ಗಾ-ಳಿಯ
ಸುಂಟ-ರ-ಗಾಳೀ
ಸುಂದರ
ಸುಂದ-ರ-ಪ್ರ-ತಿ-ಭಾ-ನ್ವಿತ
ಸುಂದರಂ
ಸುಂದ-ರ-ಕ-ಥೆಯ
ಸುಂದ-ರ-ರಾಮ
ಸುಂದ-ರ-ವಾಗಿ
ಸುಂದ-ರ-ವಾ-ಗಿದೆ
ಸುಂದ-ರ-ವಾದ
ಸುಂದ-ರಾ-ಕಾರ
ಸುಂದ-ರಿ-ಯಾ-ದ-ವಳೂ
ಸುಂಯ್ಗು-ಡತ್ತ
ಸುಂಯ್ಗು-ಡುವ
ಸುಂಸ್ಕೃ-ತ-ರಾ-ದ-ವರ
ಸುಕೃ-ತವೋ
ಸುಕೋ-ಮ-ಲ-ವಾ-ಗಿ-ರುವ
ಸುಕೋ-ಮ-ಲ-ವಾದ
ಸುಖ
ಸುಖ
ಸುಖ-ಕ-ರ-ವಾ-ಗಿತ್ತು
ಸುಖ-ಕ-ರ-ವಾ-ಗಿ-ರ-ಲಿಲ್ಲ
ಸುಖ-ಕ-ರ-ವಾ-ಗಿ-ರು-ತ್ತದೆ
ಸುಖ-ಕ-ರ-ವಾ-ದದ್ದು
ಸುಖ-ಜೀ-ವ-ನಕ್ಕೆ
ಸುಖ-ಭೋ-ಗ-ಕ್ಕಲ್ಲ
ಸುಖ-ವನ್ನು
ಸುಖ-ವನ್ನೂ
ಸುಖ-ವನ್ನೇ
ಸುಖ-ವಾಗಿ
ಸುಖ-ವಾ-ಗಿದ್ದೆ
ಸುಖ-ವಾ-ಗಿ-ರ-ಬಾ-ರ-ದಿತ್ತು
ಸುಖ-ವಿಲ್ಲ
ಸುಖ-ಸ್ವ-ರೂ-ಪ-ವಾದ
ಸುಗ-ಮ-ವಾಗಿ
ಸುಗ-ಮ-ವಾ-ಗು-ತ್ತದೆ
ಸುಟ್ಟು-ಬಿ-ಟ್ಟರು
ಸುಟ್ಟು-ಹೋ-ಗಿ-ದ್ದುವು
ಸುಟ್ಟೇ
ಸುಡ-ಲಿಲ್ಲ
ಸುಡು-ತ್ತದೆ
ಸುಡು-ತ್ತ-ದೆ-ಮೃದು
ಸುಡು-ತ್ತಾ-ರಂತೆ
ಸುಡು-ಬಿ-ಸಿ-ಲಿನ
ಸುಡುವ
ಸುಡು-ವುದು
ಸುತನೆ
ಸುತ್ತ
ಸುತ್ತ-ಮುತ್ತ
ಸುತ್ತ-ಮು-ತ್ತಲ
ಸುತ್ತ-ಮು-ತ್ತ-ಲಿ-ದ್ದ-ವ-ರಿ-ಗೆಲ್ಲ
ಸುತ್ತ-ಮು-ತ್ತ-ಲಿನ
ಸುತ್ತ-ಲಿದ್ದ
ಸುತ್ತ-ಲಿ-ದ್ದ-ವರ
ಸುತ್ತ-ಲಿ-ದ್ದ-ವರು
ಸುತ್ತ-ಲಿ-ದ್ದ-ವರೆಲ್ಲ
ಸುತ್ತ-ಲಿನ
ಸುತ್ತ-ಲಿ-ರು-ವ-ವರೆಲ್ಲ
ಸುತ್ತಲೂ
ಸುತ್ತಲೇ
ಸುತ್ತ-ಲ್ಲಿ-ದ್ದ-ವರೂ
ಸುತ್ತ-ಳತೆ
ಸುತ್ತಾಡಿ
ಸುತ್ತಾ-ಡಿ-ದು-ದರ
ಸುತ್ತಾ-ಡಿದೆ
ಸುತ್ತಾ-ಡಿ-ದ್ದೇನೆ
ಸುತ್ತಾ-ಡು-ತ್ತಿದ್ದ
ಸುತ್ತಾರೆ
ಸುತ್ತಿ
ಸುತ್ತಿ-ಕೊಂಡು
ಸುತ್ತಿ-ಕೊ-ಳ್ಳ-ಬೇ-ಕಾದ
ಸುತ್ತಿ-ಗೆಯ
ಸುತ್ತಿ-ಗೆ-ಯೇಟು
ಸುತ್ತಿತ್ತು
ಸುತ್ತಿದ್ದ
ಸುತ್ತಿ-ದ್ದರು
ಸುತ್ತಿದ್ದು
ಸುತ್ತಿದ್ದೆ
ಸುತ್ತಿ-ರುವ
ಸುತ್ತು
ಸುತ್ತು-ತ್ತಿ-ದ್ದರೂ
ಸುತ್ತು-ತ್ತಿ-ದ್ದೇನೆ
ಸುತ್ತು-ಬ-ಳಸು
ಸುತ್ತು-ಮು-ತ್ತಲ
ಸುತ್ತು-ಮು-ತ್ತ-ಲಿನ
ಸುತ್ತು-ಮು-ತ್ತೆಲ್ಲ
ಸುತ್ತು-ವ-ರಿದ
ಸುತ್ತು-ವ-ರಿ-ದಿ-ದ್ದಾರೆ
ಸುತ್ತು-ವ-ರಿದು
ಸುತ್ತು-ವ-ರಿ-ಯ-ಲ್ಪಟ್ಟು
ಸುತ್ತು-ಹೊ-ಡೆ-ದದ್ದೂ
ಸುದಾಮ
ಸುದಾ-ಮ-ಪುರಿ
ಸುದೀರ್ಘ
ಸುದೀ-ರ್ಘ-ದು-ರ್ಗಮ
ಸುದೀ-ರ್ಘ-ವಾಗಿ
ಸುದೀ-ರ್ಘ-ವಾದ
ಸುದೂರ
ಸುದೃಢ
ಸುದ್ದಿ
ಸುದ್ದಿ-ಗಳನ್ನು
ಸುದ್ದಿ-ಗಳನ್ನೆಲ್ಲ
ಸುದ್ದಿ-ಗಳು
ಸುದ್ದಿ-ಗಾಗಿ
ಸುದ್ದಿ-ಗೆ-ಳೆದ
ಸುದ್ದಿಯ
ಸುದ್ದಿ-ಯನ್ನು
ಸುದ್ದಿ-ಯಿನ್ನೂ
ಸುದ್ದಿ-ಯೊಂ-ದನ್ನು
ಸುದ್ಧಿ
ಸುಧಾ-ರಕ
ಸುಧಾ-ರ-ಕರು
ಸುಧಾ-ರ-ಕ-ರೆ-ನ್ನಿ-ಸಿ-ಕೊಂ-ಡ-ವರ
ಸುಧಾ-ರಕಿ
ಸುಧಾ-ರಣಾ
ಸುಧಾ-ರ-ಣಾ-ಗೃಹ
ಸುಧಾ-ರಣೆ
ಸುಧಾ-ರ-ಣೆ-ಗಳನ್ನು
ಸುಧಾ-ರ-ಣೆ-ಗ-ಳೆಲ್ಲ
ಸುಧಾ-ರ-ಣೆಗೆ
ಸುಧಾ-ರ-ಣೆ-ಗೊ-ಳಿ-ಸುವ
ಸುಧಾ-ರ-ಣೆಯ
ಸುಧಾ-ರ-ಣೆ-ಯನ್ನು
ಸುಧಾ-ರ-ಣೆ-ಯನ್ನೂ
ಸುಧಾ-ರ-ಣೆಯು
ಸುಧಾ-ರಿತ
ಸುಧಾ-ರಿ-ಸ-ಬ-ಹು-ದ-ಲ್ಲದೆ
ಸುಧಾ-ರಿ-ಸ-ಬೇ-ಕೆ-ನ್ನು-ವುದು
ಸುಧಾ-ರಿ-ಸಿ-ಕೊಂಡ
ಸುಧಾ-ರಿ-ಸಿ-ಕೊ-ಳ್ಳು-ತ್ತಿ-ದ್ದಂ-ತೆಯೇ
ಸುಧಾ-ರಿಸು
ಸುಧಾ-ರಿ-ಸುವ
ಸುಧಾ-ರಿ-ಸು-ವು-ದರ
ಸುಧಾ-ರಿ-ಸು-ವು-ದಷ್ಟೇ
ಸುನಿ-ಶ್ಚಿತ
ಸುನ್ನಿ-ಗಳು
ಸುಪ-ರಿ-ಚಿತ
ಸುಪ-ರಿ-ಚಿ-ತ-ನಾದ
ಸುಪ-ರಿ-ಚಿ-ತ-ವಾ-ಗು-ತ್ತಿವೆ
ಸುಪು-ತ್ರನ
ಸುಪು-ತ್ರ-ನನ್ನು
ಸುಪು-ತ್ರನೂ
ಸುಪ್ತ
ಸುಪ್ತ-ಮ-ನಸ್ಸು
ಸುಪ್ತ-ವಾ-ಗಿ-ರುವ
ಸುಪ್ತ-ಶ-ಕ್ತಿ-ಗಳ
ಸುಪ್ತ-ಶ-ಕ್ತಿ-ಗಳು
ಸುಪ್ರ-ತಿ-ಷ್ಠಿತ
ಸುಪ್ರ-ಸಿದ್ಧ
ಸುಪ್ರ-ಸಿ-ದ್ಧ-ನಾ-ಗಿದ್ದ
ಸುಪ್ರ-ಸಿ-ದ್ಧ-ನಾ-ಗಿದ್ದು
ಸುಪ್ರ-ಸಿ-ದ್ಧ-ವಾದ
ಸುಪ್ರ-ಸಿದ್ಧಿ
ಸುಬ್ಬ-ಯ್ಯರ್
ಸುಬ್ರ-ಮಣ್ಯ
ಸುಬ್ರ-ಮ-ಣ್ಯ-ನನ್ನು
ಸುಬ್ರ-ಹ್ಮಣ್ಯ
ಸುಬ್ರ-ಹ್ಮ-ಣ್ಯಾ-ನಂದ
ಸುಬ್ರಾಯ
ಸುಬ್ರಾ-ಯ-ನಾ-ಯ್ಕರ
ಸುಬ್ರಾ-ಯ-ನಾ-ಯ್ಕರು
ಸುಭದ್ರ
ಸುಭಿಕ್ಷೆ
ಸುಮ-ಧುರ
ಸುಮ-ಧು-ರ-ವಾ-ದ-ವು-ಗಳು
ಸುಮ-ಹೂ-ರ್ತ-ವನ್ನು
ಸುಮಾತ್ರಾ
ಸುಮಾ-ರಿಗೆ
ಸುಮಾರು
ಸುಮ್ಮ
ಸುಮ್ಮ-ನಾ-ಗ-ಬೇ-ಕಾ-ಯಿತು
ಸುಮ್ಮ-ನಾ-ಗ-ಲಿಲ್ಲ
ಸುಮ್ಮ-ನಾಗಿ
ಸುಮ್ಮ-ನಾ-ಗಿ-ಸಿದ್ದ
ಸುಮ್ಮ-ನಾ-ಗಿ-ಸು-ತ್ತಿದ್ದ
ಸುಮ್ಮ-ನಾ-ಗು-ವು-ದಿಲ್ಲ
ಸುಮ್ಮ-ನಾ-ದರು
ಸುಮ್ಮ-ನಿ-ದ್ದರು
ಸುಮ್ಮ-ನಿದ್ದು
ಸುಮ್ಮ-ನಿ-ದ್ದು-ದಲ್ಲ
ಸುಮ್ಮ-ನಿ-ರ-ಬೇ-ಕಾ-ಗಿ-ಯಿತು
ಸುಮ್ಮ-ನಿ-ರ-ಲಾ-ಗ-ಲಿಲ್ಲ
ಸುಮ್ಮ-ನಿ-ರ-ಲಾ-ರದ
ಸುಮ್ಮನೆ
ಸುಯೋಗ
ಸುಯೋ-ಗ-ವೆಂದು
ಸುರ-ಕ್ಷಿ-ತವೂ
ಸುರ-ನ-ದಿಯ
ಸುರರೂ
ಸುರ-ಸರೀ
ಸುರಿ-ದಂ-ತಹ
ಸುರಿ-ದಂ-ತಾ-ಯಿತು
ಸುರಿ-ದಂತೆ
ಸುರಿ-ದು-ಬಿ-ಡು-ತ್ತಿ-ದ್ದರು
ಸುರಿ-ಮಳೆ
ಸುರಿ-ಮ-ಳೆ-ಯನ್ನೇ
ಸುರಿ-ಮ-ಳೆ-ಯಾ-ಗಲು
ಸುರಿ-ಮ-ಳೆ-ಯಾ-ಗು-ತ್ತಿ-ರು-ವುದನ್ನು
ಸುರಿ-ಯ-ತೊ-ಡ-ಗಿತು
ಸುರಿ-ಯು-ತ್ತಿತ್ತು
ಸುರಿ-ಯು-ತ್ತಿ-ದ್ದರೂ
ಸುರಿ-ಯು-ವ-ವ-ರೆಗೂ
ಸುರಿ-ಯು-ವು-ದರ
ಸುರಿ-ಸುತ್ತ
ಸುರಿ-ಸು-ತ್ತಿ-ದ್ದೇನೆ
ಸುರು-ಟಿ-ಕೊಂಡು
ಸುರೇಂ-ದ್ರ-ನಾಥ
ಸುಲಭ
ಸುಲ-ಭ-ಗ್ರಾ-ಹ್ಯ-ವ-ಲ್ಲದ
ಸುಲ-ಭದ
ಸುಲ-ಭ-ದಲ್ಲಿ
ಸುಲ-ಭ-ದ್ದೇನೂ
ಸುಲ-ಭ-ವ-ದ್ದೇನೂ
ಸುಲ-ಭ-ವಲ್ಲ
ಸುಲ-ಭ-ವಾಗಿ
ಸುಲ-ಭ-ವಾ-ಗಿಯೇ
ಸುಲ-ಭ-ವಾ-ಗಿರ
ಸುಲ-ಭ-ವಾ-ಗಿ-ರು-ತ್ತ-ದೆಂದು
ಸುಲ-ಭವೂ
ಸುಲ-ಭ-ಸಾ-ಧ್ಯ-ವ-ಲ್ಲ-ವೆಂ-ಬುದು
ಸುಲ-ಲಿತ
ಸುಲ-ಲಿ-ತ-ವಾಗಿ
ಸುಳಿದ
ಸುಳಿ-ದರೂ
ಸುಳಿ-ಯ-ತೊ-ಡಗಿ
ಸುಳಿ-ಯಲ್ಲಿ
ಸುಳಿ-ಯಿ-ತು-ನಮ್ಮ
ಸುಳಿಯು
ಸುಳಿ-ಯು-ವ-ಷ್ಟ-ರಲ್ಲೇ
ಸುಳಿ-ವಿ-ರ-ಲಿಲ್ಲ
ಸುಳಿವು
ಸುಳಿವೇ
ಸುಳ್ಳಾ-ದುದು
ಸುಳ್ಳಿ-ನೊಂ-ದಿಗೆ
ಸುಳ್ಳು
ಸುಳ್ಳು-ಸತ್ಯ
ಸುಳ್ಳು-ಗಾರ
ಸುಳ್ಳು-ಗಾ-ರರು
ಸುಳ್ಳು-ಸ-ರೋ-ವ-ರ-ವನ್ನೇ
ಸುಳ್ಳು-ಸು-ಳ್ಳಾಗಿ
ಸುಳ್ಳು-ಸುಳ್ಳು
ಸುಳ್ಳು-ಹಸೀ
ಸುಳ್ಳೆಂದು
ಸುಳ್ಳೇ
ಸುವ
ಸುವರ್ಣ
ಸುವ-ರ್ಣ-ಮು-ದ್ರೆ-ಯೊಂ-ದನ್ನು
ಸುವ-ರ್ಣ-ಶೃಂ-ಖ-ಲೆ-ಯಿಂದ
ಸುವ-ರ್ಣಾ-ವ-ಕಾಶ
ಸುವ-ರ್ಣಾ-ವ-ಕಾ-ಶ-ವನ್ನು
ಸುವಿ-ಸ್ತಾರ
ಸುವು-ದ-ಕ್ಕಾಗಿ
ಸುವು-ದೇಕೆ
ಸುವ್ಯ-ಕ್ತ-ವಾ-ಗು-ತ್ತದೆ
ಸುವ್ಯ-ಕ್ತ-ವಾದ
ಸುವ್ಯ-ವ-ಸ್ಥಿತ
ಸುವ್ಯ-ವ-ಸ್ಥಿ-ತ-ವಾಗಿ
ಸುಶಿ-ಕ್ಷಿತ
ಸುಶಿ-ಕ್ಷಿ-ತರು
ಸುಶಿ-ಕ್ಷಿ-ತರೂ
ಸುಶೋ-ಭಿ-ತ-ವಾದ
ಸುಶ್ರಾ-ವ್ಯ-ಯ-ವಾಗಿ
ಸುಶ್ರಾ-ವ್ಯ-ವಾಗಿ
ಸುಸಂ-ದರ್ಭ
ಸುಸಂ-ಬ-ದ್ಧ-ವಾಗಿ
ಸುಸಂ-ಬ-ದ್ಧ-ವಾ-ಗಿವೆ
ಸುಸಂ-ಸ್ಕೃತ
ಸುಸಂ-ಸ್ಕೃ-ತ-ವಿ-ನಯ
ಸುಸಂ-ಸ್ಕೃ-ತನೂ
ಸುಸಂ-ಸ್ಕೃ-ತರು
ಸುಸಂ-ಸ್ಕೃ-ತವೂ
ಸುಸಂ-ಸ್ಕೃತೆ
ಸುಸ-ಜ್ಜಿತ
ಸುಸೂ-ತ್ರ-ವಾಗಿ
ಸುಸ್ತಾ-ಗಿ-ಬಿ-ಡು-ತ್ತಿ-ದ್ದರು
ಸುಸ್ತಾ-ಗು-ತ್ತಿ-ದ್ದರು
ಸುಸ್ತು
ಸುಸ್ಥಿ-ತಿಗೆ
ಸುಸ್ಥಿ-ರ-ಸ್ಥಾನ
ಸುಸ್ಪ-ಷ್ಟ-ವಾಗಿ
ಸುಸ್ಪ-ಷ್ಟ-ವಾ-ಗಿ-ಕೆ-ಲ-ವೊಮ್ಮೆ
ಸುಸ್ಪ-ಷ್ಟ-ವಾ-ಗಿತ್ತು
ಸುಸ್ಪ-ಷ್ಟ-ವಾ-ಗಿ-ತ್ತೆಂ-ದರೆ
ಸುಸ್ಪ-ಷ್ಟ-ವಾ-ಗಿದೆ
ಸುಸ್ಪ-ಷ್ಟ-ವಾ-ಗಿ-ದ್ದುವು
ಸುಸ್ಪ-ಷ್ಟ-ವಾದ
ಸೂಕ್ತ
ಸೂಕ್ತ-ರಾದ
ಸೂಕ್ತ-ವಾಗಿ
ಸೂಕ್ತ-ವಾ-ಗಿದೆ
ಸೂಕ್ತ-ವಾದ
ಸೂಕ್ತ-ವಾ-ದೀತು
ಸೂಕ್ತ-ವೆಂದು
ಸೂಕ್ಷ್ಮ
ಸೂಕ್ಷ್ಮ-ಗಳ
ಸೂಕ್ಷ್ಮ-ಗಳನ್ನು
ಸೂಕ್ಷ್ಮ-ಗ್ರ-ಹಿಕೆ
ಸೂಕ್ಷ್ಮತೆ
ಸೂಕ್ಷ್ಮ-ದ-ರ್ಶ-ಕ-ವನ್ನೂ
ಸೂಕ್ಷ್ಮ-ಪ-ರಿ-ಚಯ
ಸೂಕ್ಷ್ಮ-ಬು-ದ್ಧಿಗೆ
ಸೂಕ್ಷ್ಮ-ವಾಗಿ
ಸೂಕ್ಷ್ಮ-ವಾದ
ಸೂಕ್ಷ್ಮ-ವಾ-ದದ್ದು
ಸೂಕ್ಷ್ಮವೂ
ಸೂಕ್ಷ್ಮ-ವೆಂ-ಬು-ದರ
ಸೂಚ-ಕ-ವಾದ
ಸೂಚ-ನಾ-ಫ-ಲ-ಕ-ವನ್ನು
ಸೂಚನೆ
ಸೂಚ-ನೆ-ಗ-ಳ-ನ್ನಿ-ತ್ತರು
ಸೂಚ-ನೆ-ಗಳನ್ನು
ಸೂಚ-ನೆ-ಗಳನ್ನೂ
ಸೂಚ-ನೆ-ಗಳು
ಸೂಚ-ನೆಗೆ
ಸೂಚ-ನೆ-ಯನ್ನು
ಸೂಚ-ನೆ-ಯನ್ನೂ
ಸೂಚ-ನೆಯೇ
ಸೂಚಿ
ಸೂಚಿ-ತ-ವಾದ
ಸೂಚಿ-ಸಲು
ಸೂಚಿಸಿ
ಸೂಚಿ-ಸಿದ
ಸೂಚಿ-ಸಿ-ದರು
ಸೂಚಿ-ಸಿ-ದಳು
ಸೂಚಿ-ಸಿ-ದು-ದನ್ನು
ಸೂಚಿ-ಸಿ-ದ್ದೇನೆ
ಸೂಚಿಸು
ಸೂಚಿ-ಸುವ
ಸೂಚಿ-ಸು-ವಂತೆ
ಸೂಚಿ-ಸು-ವುದಾ
ಸೂಜಿ
ಸೂಜಿ-ಗ-ಲ್ಲಿ-ನಂತೆ
ಸೂಜಿಯ
ಸೂಟು
ಸೂತ್ರ
ಸೂತ್ರಕ್ಕೆ
ಸೂತ್ರ-ಗಳ
ಸೂತ್ರ-ಗಳಿಂದ
ಸೂತ್ರ-ಗ-ಳಿನ್ನೂ
ಸೂತ್ರ-ಗಳು
ಸೂತ್ರದ
ಸೂತ್ರ-ದಿಂದ
ಸೂತ್ರ-ಪ್ರಾ-ಯ-ವಾದ
ಸೂತ್ರ-ವನ್ನೂ
ಸೂಪ-ರಿಂ-ಟೆಂ-ಡೆಂ-ಟ-ರಾದ
ಸೂರ-ಜ್ನಾ-ರಾ-ಯಣ್
ಸೂರ-ದಾಸ
ಸೂರ-ದಾ-ಸನ
ಸೂರ-ದಾ-ಸನು
ಸೂರೆ-ಗೊಂಡ
ಸೂರ್ಯ
ಸೂರ್ಯನ
ಸೂರ್ಯ-ನಂತೆ
ಸೂರ್ಯ-ನನ್ನು
ಸೂರ್ಯನು
ಸೂರ್ಯನೂ
ಸೂರ್ಯಾ-ಸ್ತದ
ಸೂರ್ಯಾ-ಸ್ತ-ಮಾನ
ಸೂರ್ಯೋ-ದಯ
ಸೂರ್ಯೋ-ದ-ಯ-ಸೂ-ರ್ಯಾ-ಸ್ತ-ಮಾ-ನ-ಗಳ
ಸೂಸುತ್ತ
ಸೂಸು-ತ್ತಿದ್ದ
ಸೃಷ್ಟಿ-ಕರ್ತ
ಸೃಷ್ಟಿ-ಕ್ರಮ
ಸೃಷ್ಟಿಯ
ಸೃಷ್ಟಿ-ಯಷ್ಟು
ಸೃಷ್ಟಿ-ಯಾ-ಗಿತ್ತು
ಸೃಷ್ಟಿಯೇ
ಸೃಷ್ಟಿ-ಸ-ಬ-ಲ್ಲ-ವ-ನಾ-ಗಿದ್ದ
ಸೃಷ್ಟಿ-ಸ-ಬೇಕು
ಸೃಷ್ಟಿ-ಸ-ಲ್ಪಟ್ಟ
ಸೃಷ್ಟಿ-ಸ-ಲ್ಪ-ಡದ
ಸೃಷ್ಟಿಸಿ
ಸೃಷ್ಟಿ-ಸಿದ
ಸೃಷ್ಟಿ-ಸಿದೆ
ಸೆ
ಸೆಂಚುರಿ
ಸೆಂಟ್
ಸೆಂಟ್ರಲ್
ಸೆಕೆ-ಗಾಲ
ಸೆಕೆ-ಯಿಂದ
ಸೆಕ್ರೆ-ಟರಿ
ಸೆಣ-ಸಾಟ
ಸೆಣ-ಸಾ-ಟ-ಗಳು
ಸೆಣ-ಸಾ-ಡಿ-ದುವು
ಸೆಣಸಿ
ಸೆಪ್ಟೆಂ-ಬ-ರಿ-ನಲ್ಲಿ
ಸೆಪ್ಟೆಂ-ಬರ್
ಸೆಪ್ಟೆಂ-ಬರ್ನ
ಸೆಪ್ಟೆಂ-ಬ-ರ್ನಿಂದ
ಸೆಮಿ-ನ-ರಿ-ನಲ್ಲಿ
ಸೆಮಿ-ನ-ರಿಯ
ಸೆಮಿ-ನ-ರಿ-ಯಿಂದ
ಸೆಮಿ-ನ-ರಿಯು
ಸೆರೆ-ಹಿ-ಡಿ-ದಿ-ಟ್ಟು-ಕೊ-ಳ್ಳ-ಬ-ಲ್ಲ-ವ-ರಾಗಿ
ಸೆರೆ-ಹಿ-ಡಿ-ದಿ-ತ್ತಾ-ದರೂ
ಸೆರೆ-ಹಿ-ಡಿ-ದು-ಬಿ-ಟ್ಟಿ-ತ್ತೆಂ-ದರೆ
ಸೆರ್ಮ್ಯಾಟ್
ಸೆಲೆ-ಯಾ-ಗಿವೆ
ಸೆಲೆಯೂ
ಸೆಳೆ-ತಕ್ಕೆ
ಸೆಳೆದ
ಸೆಳೆ-ದದ್ದು
ಸೆಳೆ-ದಿತ್ತು
ಸೆಳೆ-ದಿದ್ದ
ಸೆಳೆ-ದಿ-ದ್ದರು
ಸೆಳೆ-ದಿ-ರ-ಲಿಲ್ಲ
ಸೆಳೆದು
ಸೆಳೆ-ದು-ಕೊ-ಳ್ಳಲಿ
ಸೆಳೆ-ದುವು
ಸೆಳೆ-ದೊ-ಯ್ಯು-ತ್ತಿ-ರು-ವಂತೆ
ಸೆಳೆ-ಯಲು
ಸೆಳೆ-ಯ-ಲ್ಪ-ಟ್ಟರು
ಸೆಳೆ-ಯ-ಲ್ಪ-ಟ್ಟರೆ
ಸೆಳೆ-ಯ-ಲ್ಪಟ್ಟು
ಸೆಳೆ-ಯ-ಲ್ಪ-ಡು-ತ್ತಾರೆ
ಸೆಳೆ-ಯಿತು
ಸೆಳೆ-ಯು-ತ್ತಿತ್ತು
ಸೆಳೆ-ಯು-ತ್ತಿ-ದ್ದವು
ಸೆಳೆ-ಯು-ತ್ತಿ-ದ್ದುವು
ಸೆಳೆ-ಯು-ವಂ-ತಿತ್ತು
ಸೆಳೆ-ಯು-ವಲ್ಲಿ
ಸೆಳೆ-ಯು-ವು-ದ-ಕ್ಕಾಗಿ
ಸೆಸೇಮ್
ಸೇ
ಸೇಂಟ್ಪಾಲ್ಗೆ
ಸೇಠ್
ಸೇಠ್ಜಿ
ಸೇಠ್ಜಿ-ಯಿಂದ
ಸೇತು-ಪತಿ
ಸೇತು-ಪ-ತಿ-ಯನ್ನು
ಸೇತು-ವೆ-ಗಳು
ಸೇತು-ವೆಯ
ಸೇತು-ವೆ-ಯನ್ನು
ಸೇದುವ
ಸೇದು-ವುದು
ಸೇನ
ಸೇನನೂ
ಸೇನರ
ಸೇನಾ
ಸೇನಾ-ನಿ-ಗಳ
ಸೇರ-ದೆ-ಹೋ-ಗಿ-ದ್ದರೆ
ಸೇರ-ಬ-ಹುದು
ಸೇರ-ಲಿದ್ದ
ಸೇರ-ಲಿ-ದ್ದರು
ಸೇರ-ಲಿ-ದ್ದಾರೆ
ಸೇರಲು
ಸೇರ-ಲೇ-ಬೇಕು
ಸೇರಿ
ಸೇರಿ-ಕೊಂ-ಡರು
ಸೇರಿ-ಕೊಂ-ಡರೆ
ಸೇರಿ-ಕೊಂ-ಡಾಗ
ಸೇರಿ-ಕೊಂಡು
ಸೇರಿ-ಕೊಂಡೆ
ಸೇರಿ-ಕೊ-ಳ್ಳ-ದಂತೆ
ಸೇರಿ-ಕೊಳ್ಳು
ಸೇರಿ-ಕೊ-ಳ್ಳು-ತ್ತೇನೆ
ಸೇರಿ-ಕೊ-ಳ್ಳು-ವಿರಾ
ಸೇರಿತು
ಸೇರಿತ್ತು
ಸೇರಿದ
ಸೇರಿ-ದಂತೆ
ಸೇರಿ-ದರು
ಸೇರಿ-ದ-ವನು
ಸೇರಿ-ದ-ವ-ನೆಂಬ
ಸೇರಿ-ದ-ವ-ನೆಂ-ಬುದು
ಸೇರಿ-ದ-ವನೋ
ಸೇರಿ-ದ-ವ-ರಾ-ಗಿ-ದ್ದರು
ಸೇರಿ-ದ-ವ-ರಾ-ಗಿ-ರ-ಲಿಲ್ಲ
ಸೇರಿ-ದ-ವ-ರಾದ
ಸೇರಿ-ದ-ವ-ರಾ-ದರು
ಸೇರಿ-ದ-ವರು
ಸೇರಿ-ದ-ವ-ರು-ಇ-ಷ್ಟರ
ಸೇರಿ-ದ-ವರೇ
ಸೇರಿ-ದ-ವರೋ
ಸೇರಿ-ದ-ವಳು
ಸೇರಿ-ದವು
ಸೇರಿ-ದ-ವು-ಗಳು
ಸೇರಿ-ದಾ-ಗಿ-ನಿಂದ
ಸೇರಿ-ದುದು
ಸೇರಿದ್ದ
ಸೇರಿ-ದ್ದ-ರಿಂದ
ಸೇರಿ-ದ್ದರು
ಸೇರಿ-ದ್ದ-ವ-ರಲ್ಲಿ
ಸೇರಿದ್ದು
ಸೇರಿ-ರು-ವಾಗ
ಸೇರಿ-ರು-ವುದನ್ನು
ಸೇರಿ-ರು-ವುದು
ಸೇರಿ-ಸ-ಬೇಕು
ಸೇರಿ-ಸ-ಲಾ-ಯಿತು
ಸೇರಿ-ಸಲು
ಸೇರಿ-ಸಲೂ
ಸೇರಿಸಿ
ಸೇರಿ-ಸಿ-ಕೊ-ಳ್ಳಲು
ಸೇರಿ-ಸಿ-ದ-ಬ-ಳಿಕ
ಸೇರಿ-ಸಿ-ದ-ರು-ಆ-ದರೆ
ಸೇರಿ-ಸು-ತ್ತೀರಿ
ಸೇರಿ-ಸುವ
ಸೇರಿ-ಸು-ವಂತೆ
ಸೇರು
ಸೇರು-ತಿರೆ
ಸೇರು-ತ್ತವೆ
ಸೇರು-ತ್ತಾರೆ
ಸೇರು-ತ್ತಿದ್ದ
ಸೇರು-ತ್ತಿ-ದ್ದರು
ಸೇರುವ
ಸೇರು-ವಂ-ತಾ-ಗ-ಬೇಕು
ಸೇರು-ವಲ್ಲಿ
ಸೇಲ-ಮನ್ನು
ಸೇಲಮ್
ಸೇವಕ
ಸೇವ-ಕನ
ಸೇವ-ಕ-ನನ್ನು
ಸೇವ-ಕ-ನನ್ನೂ
ಸೇವ-ಕ-ನಾಗಿ
ಸೇವ-ಕ-ನಿಗೂ
ಸೇವ-ಕ-ರಾದ
ಸೇವ-ಕ-ರಿಗೆ
ಸೇವ-ಕ-ರಿ-ದ್ದರು
ಸೇವ-ಕರು
ಸೇವ-ಕರೇ
ಸೇವಾ
ಸೇವಾ-ಕಾರ್ಯ
ಸೇವಿ
ಸೇವಿ-ಯರ
ಸೇವಿ-ಯರ್
ಸೇವಿ-ಯರ್ರ
ಸೇವಿ-ಯ-ರ್ರನ್ನು
ಸೇವಿ-ಯ-ರ್ರಿಗೂ
ಸೇವಿ-ಸು-ತ್ತಿ-ದ್ದು-ದ-ರಿಂದ
ಸೇವಿ-ಸು-ವುದು
ಸೇವೆ
ಸೇವೆ-ಗಳೇ
ಸೇವೆ-ಗಾಗಿ
ಸೇವೆ-ಗೈ-ಯುವ
ಸೇವೆಯ
ಸೇವೆ-ಯನ್ನು
ಸೇವೆ-ಯಲ್ಲಿ
ಸೇವೆ-ಯಾ-ಗ-ಬೇ-ಕಿ-ದ್ದರೆ
ಸೇವೆ-ಯೇನೋ
ಸೈಂಟ್
ಸೈನಿ-ಕ-ಪ-ಡೆ-ಯನ್ನು
ಸೈನಿ-ಕ-ರನ್ನು
ಸೈನ್ಯಾ-ಧಿ-ಕಾರಿ
ಸೈನ್ಯಾ-ಧಿ-ಕಾ-ರಿ-ಗಳು
ಸೈನ್ಸ್
ಸೈಬೀ-ರಿಯ
ಸೊಂಟ-ಪಟ್ಟಿ
ಸೊಕ್ಕಿದ
ಸೊಗ-ಸಾಗಿ
ಸೊಗ-ಸಾದ
ಸೊಗ-ಸು-ಗಾರ
ಸೊಗ-ಸು-ಗಾ-ರಿ-ಕೆಗೆ
ಸೊತ್ತಾದ
ಸೊನ್ನೆ
ಸೊನ್ನೆ-ಗಿಂ-ತಲೂ
ಸೊಪ್ಪಿ-ನೊಂ-ದಿಗೆ
ಸೊಬ-ಗ-ನ್ನು-ಅ-ದ-ರಲ್ಲೂ
ಸೊಬ-ಗಿದೆ
ಸೊಬಗು
ಸೊಬ-ಗು-ಇ-ವು-ಗ-ಳಿಂ-ದಾಗಿ
ಸೊಬ-ಗೇನು
ಸೊರ-ಗಿದ
ಸೊಳ್ಳೆ-ಗಳೂ
ಸೊಳ್ಳೆಯ
ಸೊಸೈಟಿ
ಸೊಸೈ-ಟಿ-ಗಳು
ಸೊಸೈ-ಟಿಗೆ
ಸೊಸೈ-ಟಿಯ
ಸೊಸೈ-ಟಿ-ಯನ್ನು
ಸೊಸೈ-ಟಿ-ಯಲ್ಲಿ
ಸೊಸೈ-ಟಿ-ಯ-ವರು
ಸೊಸೈ-ಟಿಯು
ಸೋ
ಸೋಂಕಿ-ದುವು
ಸೋಂಕೇ
ಸೋಂಬೇ-ರಿ-ಗಳು
ಸೋಕಲೂ
ಸೋಗಿನ
ಸೋಗು
ಸೋತ
ಸೋತರು
ಸೋತು
ಸೋತು-ಹೋದೆ
ಸೋದರ
ಸೋದ-ರನ
ಸೋದ-ರ-ನನ್ನು
ಸೋದ-ರ-ನಾದ
ಸೋದ-ರ-ನಿಗೆ
ಸೋದ-ರ-ನಿ-ಗೊಂದು
ಸೋದ-ರರ
ಸೋದ-ರ-ರನ್ನು
ಸೋದ-ರ-ರಲ್ಲಿ
ಸೋದ-ರ-ರಾದ
ಸೋದ-ರ-ರಿಗೆ
ಸೋದ-ರರು
ಸೋದ-ರರೆ
ಸೋದ-ರ-ರೆಂಬ
ಸೋದ-ರರೇ
ಸೋದ-ರ-ರೊಂ-ದಿಗೆ
ಸೋದ-ರ-ಸಂ-ನ್ಯಾ-ಸಿ-ಗಳನ್ನು
ಸೋದ-ರ-ಸಂ-ನ್ಯಾ-ಸಿ-ಗ-ಳಲ್ಲೂ
ಸೋದ-ರ-ಸಂ-ನ್ಯಾ-ಸಿ-ಗ-ಳಿಗೂ
ಸೋದ-ರ-ಸಂ-ನ್ಯಾ-ಸಿ-ಗ-ಳಿಗೆ
ಸೋದ-ರ-ಸಂ-ನ್ಯಾ-ಸಿ-ಗಳು
ಸೋದ-ರ-ಸಂ-ನ್ಯಾ-ಸಿ-ಯಾದ
ಸೋದ-ರ-ಸಂ-ನ್ಯಾ-ಸಿ-ಯೊ-ಬ್ಬರು
ಸೋದ-ರ-ಸೊ-ಸೆ-ಯ-ರಾದ
ಸೋದರಿ
ಸೋದ-ರಿ-ಯನ್ನು
ಸೋದ-ರಿ-ಯ-ರಂತೆ
ಸೋದ-ರಿ-ಯ-ರಲ್ಲಿ
ಸೋದ-ರಿ-ಯ-ರಿಗೆ
ಸೋದ-ರಿ-ಯರು
ಸೋದ-ರಿ-ಯರೆ
ಸೋದ-ರಿ-ಯರೇ
ಸೋಫಾದ
ಸೋಮ
ಸೋಮ-ನಾಥ
ಸೋಮ-ನಾ-ಥ-ವನ್ನು
ಸೋಮ-ವಾರ
ಸೋಮಾ-ರಿ-ತನ
ಸೋಮಾ-ರಿ-ತ-ನಕ್ಕೂ
ಸೋಲನ್ನೂ
ಸೋಲನ್ನೇ
ಸೋಲ-ನ್ನೊ-ಪ್ಪಿ-ಕೊಂ-ಡರು
ಸೋಲ-ನ್ನೊ-ಪ್ಪಿ-ಕೊ-ಳ್ಳು-ವುದು
ಸೋಲ-ಲೇ-ಬೇ-ಕಾ-ಯಿತು
ಸೋಲಿ-ಸ-ಬೇ-ಕೆಂದು
ಸೋಲಿ-ಸಲು
ಸೋಲು
ಸೋಲು-ವು-ದಿಲ್ಲ
ಸೋಶಿ-ಯಲ್
ಸೌಂದರ್ಯ
ಸೌಂದ-ರ್ಯ-ಇ-ವು-ಗಳನ್ನೆಲ್ಲ
ಸೌಂದ-ರ್ಯ-ಗಳು
ಸೌಂದ-ರ್ಯದ
ಸೌಂದ-ರ್ಯ-ದಿಂದ
ಸೌಂದ-ರ್ಯ-ರಾ-ಣಿ-ಯರು
ಸೌಂದ-ರ್ಯ-ವನ್ನು
ಸೌಂದ-ರ್ಯವು
ಸೌಕ-ರ್ಯ-ಗಳೂ
ಸೌಕ-ರ್ಯ-ವಿನ್ನೂ
ಸೌಖ್ಯ-ದಿಂ-ದಿ-ರು-ವು-ದಾಗಿ
ಸೌಜನ್ಯ
ಸೌಜ-ನ್ಯ-ಕ್ಕಾಗಿ
ಸೌಜ-ನ್ಯ-ದಿಂದ
ಸೌಟರ್
ಸೌಭಾಗ್ಯ
ಸೌಭಾ-ಗ್ಯ-ವನ್ನು
ಸೌಭಾ-ಗ್ಯ-ವೆಂದು
ಸೌಮ್ಯ-ತೆ-ಯಲ್ಲಿ
ಸೌಮ್ಯ-ವಾಗಿ
ಸೌಮ್ಯ-ವಾ-ಗಿಯೇ
ಸೌಲಭ್ಯ
ಸೌಲ-ಭ್ಯ-ಗಳನ್ನೂ
ಸೌಲ-ಭ್ಯ-ವಿತ್ತು
ಸೌಲ-ಭ್ಯವೂ
ಸೌಹಾ-ರ್ದ-ಗಳಿಂದ
ಸೌಹಾ-ರ್ದ-ದಿಂದ
ಸೌಹಾ-ರ್ದ-ಯುತ
ಸೌಹಾ-ರ್ದ-ವನ್ನು
ಸ್ಕರಿಸಿ
ಸ್ಕೂಲ್
ಸ್ಕೇಟಿಂಗ್
ಸ್ಕ್ರಿಪ್ಟ್
ಸ್ಕ್ವೇರ್
ಸ್ಟರ್ಜಸ್
ಸ್ಟರ್ಜ-ಸ್ಳನ್ನೂ
ಸ್ಟರ್ಡಿ
ಸ್ಟರ್ಡಿಗೆ
ಸ್ಟರ್ಡಿಯ
ಸ್ಟರ್ಡಿ-ಯನ್ನು
ಸ್ಟರ್ಡಿಯೂ
ಸ್ಟರ್ಡಿ-ಯೊಂ-ದಿಗೆ
ಸ್ಟಾಂಪ್ಸ್ಕಾಟ್
ಸ್ಟಿಟ್ಸ-ರ್ಲ್ಯಾಂ-ಡಿನ
ಸ್ಟೀಮ-ರಿ-ನಲ್ಲಿ
ಸ್ಟೀಮರ್
ಸ್ಟೆಲ್ಲಾ
ಸ್ಟೇಶನ್
ಸ್ಟ್ಯಾಂಟನ್
ಸ್ಟ್ಯಾಂಡ-ರ್ಡ್
ಸ್ಟ್ರೀಟಿನ
ಸ್ಟ್ರೀಟ್
ಸ್ಟ್ರೀಟ್ನಿಗೆ
ಸ್ತಂಭಿ-ತ-ಳ-ನ್ನಾಗಿ
ಸ್ತಂಭೀ-ಭೂತ
ಸ್ತಂಭೀ-ಭೂ-ತ-ರಾ-ಗು-ತ್ತಿ-ದ್ದರು
ಸ್ತಂಭೀ-ಭೂ-ತ-ರಾ-ದರು
ಸ್ತಂಭೀ-ಭೂ-ತ-ರಾ-ದೆವು
ಸ್ತಬ್ಧ-ನಾಗಿ
ಸ್ತಬ್ಧ-ನಾದ
ಸ್ತರ-ಕ್ಕಿ-ಳಿ-ಯ-ಬೇ-ಕಾಗಿ
ಸ್ತರಕ್ಕೆ
ಸ್ತರ-ಗಳ
ಸ್ತರ-ಗ-ಳಾದ
ಸ್ತರ-ದಲ್ಲಿ
ಸ್ತರ-ದ-ಲ್ಲಿ-ರಿಸಿ
ಸ್ತರ-ದಿಂದ
ಸ್ತರ-ವನ್ನು
ಸ್ತಿಮಿತ
ಸ್ತಿಮಿ-ತ-ಗೊ-ಳಿ-ಸಿ-ಕೊ-ಳ್ಳಲು
ಸ್ತಿಮಿ-ತ-ವನ್ನು
ಸ್ತೋಮವು
ಸ್ತ್ರೀ
ಸ್ತ್ರೀ-ಪುರು
ಸ್ತ್ರೀ-ಪು-ರು-ಷ-ನನ್ನೂ
ಸ್ತ್ರೀ-ಪು-ರು-ಷ-ರನ್ನು
ಸ್ತ್ರೀ-ಪು-ರು-ಷರು
ಸ್ತ್ರೀ-ಪು-ರು-ಷ-ರೆ-ಲ್ಲರೂ
ಸ್ತ್ರೀಕು-ಲದ
ಸ್ತ್ರೀತ್ವದ
ಸ್ತ್ರೀಪು-ರುಷ
ಸ್ತ್ರೀಪು-ರು-ಷನೂ
ಸ್ತ್ರೀಪು-ರು-ಷರ
ಸ್ತ್ರೀಪು-ರು-ಷ-ರ-ನ್ನಾ-ದರೂ
ಸ್ತ್ರೀಪು-ರು-ಷ-ರನ್ನು
ಸ್ತ್ರೀಪು-ರು-ಷ-ರಿ-ಬ್ಬರೂ
ಸ್ತ್ರೀಪು-ರು-ಷರು
ಸ್ತ್ರೀಪು-ರು-ಷರೂ
ಸ್ತ್ರೀಪು-ರು-ಷ-ರೆ-ದೆ-ಯನ್ನು
ಸ್ತ್ರೀಮ-ಠ-ವನ್ನು
ಸ್ತ್ರೀಯ
ಸ್ತ್ರೀಯನ್ನು
ಸ್ತ್ರೀಯರ
ಸ್ತ್ರೀಯ-ರ-ಸ-ಹಾ-ನು-ಭೂ-ತಿ-ಯನ್ನು
ಸ್ತ್ರೀಯ-ರನ್ನು
ಸ್ತ್ರೀಯ-ರಿ-ಗಾಗಿ
ಸ್ತ್ರೀಯ-ರಿಗೆ
ಸ್ತ್ರೀಯ-ರಿ-ಗೆ-ಪು-ರು-ಷ-ರಿ-ಗೆ-ಮ-ಕ್ಕ-ಳಿಗೆ
ಸ್ತ್ರೀಯರು
ಸ್ತ್ರೀಯ-ರು-ಪು-ರು-ಷರು
ಸ್ತ್ರೀಯರೂ
ಸ್ತ್ರೀಯರೇ
ಸ್ತ್ರೀಯ-ರೊಂ-ದಿಗೆ
ಸ್ತ್ರೀಯೂ
ಸ್ತ್ರೀಯೊಂ-ದಿಗೆ
ಸ್ತ್ರೀಸಂ-ತಾನ
ಸ್ತ್ರೀಸ್ವಾ-ತಂತ್ರ್ಯ
ಸ್ಥಗಿ-ತ-ಗೊ-ಳ್ಳು-ತ್ತ-ದೆಂದೇ
ಸ್ಥಳ
ಸ್ಥಳಕ್ಕೆ
ಸ್ಥಳಕ್ಕೇ
ಸ್ಥಳ-ಗ-ಳಂತೆ
ಸ್ಥಳ-ಗಳನ್ನು
ಸ್ಥಳ-ಗಳನ್ನೂ
ಸ್ಥಳ-ಗಳನ್ನೆಲ್ಲ
ಸ್ಥಳ-ಗಳಲ್ಲಿ
ಸ್ಥಳ-ಗ-ಳಲ್ಲೂ
ಸ್ಥಳ-ಗ-ಳ-ಲ್ಲೊಂ-ದಾದ
ಸ್ಥಳ-ಗ-ಳ-ಲ್ಲೊಂದು
ಸ್ಥಳ-ಗ-ಳಾದ
ಸ್ಥಳ-ಗ-ಳಿಂ-ದ-ಅ-ವ-ರನ್ನು
ಸ್ಥಳ-ಗ-ಳಿಗೆ
ಸ್ಥಳದ
ಸ್ಥಳ-ದಲ್ಲಿ
ಸ್ಥಳ-ದ-ಲ್ಲಿದ್ದ
ಸ್ಥಳ-ದ-ಲ್ಲಿ-ರು-ವಾಗ
ಸ್ಥಳ-ದಲ್ಲೆಲ್ಲ
ಸ್ಥಳ-ವ-ನ್ನಾ-ದರೂ
ಸ್ಥಳ-ವನ್ನು
ಸ್ಥಳ-ವಾ-ಗಿತ್ತು
ಸ್ಥಳ-ವಾದ
ಸ್ಥಳ-ವಿರು
ಸ್ಥಳ-ವಿ-ಲ್ಲ-ದಂತೆ
ಸ್ಥಳ-ವೆಂದರೆ
ಸ್ಥಳ-ವೇ-ನಾ-ದರೂ
ಸ್ಥಳಾ-ವ-ಕಾ-ಶ-ವಿತ್ತು
ಸ್ಥಳಾ-ವ-ಕಾ-ಶ-ವಿ-ಲ್ಲದೆ
ಸ್ಥಳಾ-ವ-ಕಾ-ಶ-ವಿ-ಲ್ಲದ್ದ
ಸ್ಥಳೀಯ
ಸ್ಥಾನ
ಸ್ಥಾನಕ್ಕೆ
ಸ್ಥಾನ-ಗಳನ್ನೂ
ಸ್ಥಾನ-ಗ-ಳಿಂ-ದು-ದಿ-ಸಿದ
ಸ್ಥಾನ-ದೊ-ಳಕ್ಕೆ
ಸ್ಥಾನ-ಮಾನ
ಸ್ಥಾನ-ಮಾ-ನ-ಮ-ಠಾ-ಧಿ-ಪತ್ಯ
ಸ್ಥಾನ-ಮಾ-ನ-ಗಳ
ಸ್ಥಾನ-ಮಾ-ನದ
ಸ್ಥಾನ-ವನ್ನು
ಸ್ಥಾನ-ವನ್ನೂ
ಸ್ಥಾನ-ವಿ-ದೆ-ಯೆಂದು
ಸ್ಥಾನ-ವಿ-ದೆಯೋ
ಸ್ಥಾನ-ವೆಂ-ಥದು
ಸ್ಥಾನವೇ
ಸ್ಥಾಪಿ-ತ-ವಾ-ಗ-ಲಿದ್ದ
ಸ್ಥಾಪಿಸ
ಸ್ಥಾಪಿ-ಸ-ಬೇ-ಕೆಂಬ
ಸ್ಥಾಪಿ-ಸ-ಲಾ-ಯಿತು
ಸ್ಥಾಪಿ-ಸಲು
ಸ್ಥಾಪಿ-ಸ-ಲ್ಪಟ್ಟ
ಸ್ಥಾಪಿಸಿ
ಸ್ಥಾಪಿ-ಸಿದ
ಸ್ಥಾಪಿ-ಸಿ-ದರು
ಸ್ಥಾಪಿಸು
ಸ್ಥಾಪಿ-ಸುವ
ಸ್ಥಾಪಿ-ಸು-ವಲ್ಲಿ
ಸ್ಥಾಪಿ-ಸು-ವು-ದರ
ಸ್ಥಾಪಿ-ಸು-ವು-ದ-ರೊಂ-ದಿಗೆ
ಸ್ಥಾಪಿ-ಸು-ವುದು
ಸ್ಥಿತಿ
ಸ್ಥಿತಿ-ಗತಿ
ಸ್ಥಿತಿ-ಗತಿ
ಸ್ಥಿತಿ-ಗ-ತಿ-ಗಳ
ಸ್ಥಿತಿ-ಗ-ತಿ-ಗಳನ್ನು
ಸ್ಥಿತಿ-ಗ-ತಿಯ
ಸ್ಥಿತಿ-ಗ-ತಿ-ಯನ್ನು
ಸ್ಥಿತಿ-ಗಳನ್ನು
ಸ್ಥಿತಿ-ಗಿಂತ
ಸ್ಥಿತಿಗೆ
ಸ್ಥಿತಿಗೇ
ಸ್ಥಿತಿ-ಗೇ-ರ-ತೊ-ಡ-ಗಿತು
ಸ್ಥಿತಿ-ಗೇ-ರಿ-ದರು
ಸ್ಥಿತಿ-ಗೇ-ರಿ-ದ-ರೆಂಬ
ಸ್ಥಿತಿ-ಗೇ-ರಿ-ದ್ದಾನೆ
ಸ್ಥಿತಿ-ಗೇ-ರು-ತ್ತಿ-ದ್ದರು
ಸ್ಥಿತಿಯ
ಸ್ಥಿತಿ-ಯನ್ನು
ಸ್ಥಿತಿ-ಯಲ್ಲಿ
ಸ್ಥಿತಿ-ಯ-ಲ್ಲಿ-ತ್ತೆಂ-ದರೆ
ಸ್ಥಿತಿ-ಯ-ಲ್ಲಿದ್ದ
ಸ್ಥಿತಿ-ಯ-ಲ್ಲಿ-ದ್ದರು
ಸ್ಥಿತಿ-ಯ-ಲ್ಲಿ-ದ್ದರೂ
ಸ್ಥಿತಿ-ಯ-ಲ್ಲಿ-ದ್ದಾಗ
ಸ್ಥಿತಿ-ಯ-ಲ್ಲಿ-ರ-ಲಿಲ್ಲ
ಸ್ಥಿತಿ-ಯ-ಲ್ಲಿ-ರು-ವ-ವ-ರಿಗೆ
ಸ್ಥಿತಿ-ಯ-ಲ್ಲಿ-ರು-ವಾಗ
ಸ್ಥಿತಿ-ಯಲ್ಲೇ
ಸ್ಥಿತಿ-ಯಿಂದ
ಸ್ಥಿತಿ-ಯಿಂ-ದ-ಇ-ಳಿ-ಸಲು
ಸ್ಥಿತಿಯು
ಸ್ಥಿತಿಯೂ
ಸ್ಥಿತಿ-ವಂತ
ಸ್ಥಿರ-ಗ-ತಿ-ಯಿಂದ
ಸ್ಥಿರ-ಗೊ-ಳಿ-ಸು-ವುದು
ಸ್ಥಿರ-ಗೊ-ಳ್ಳು-ತ್ತಿ-ದ್ದುವು
ಸ್ಥಿರ-ಬು-ದ್ಧಿ-ಯು-ಳ್ಳ-ವ-ನಾ-ಗಿ-ರು-ತ್ತಾನೆ
ಸ್ಥಿರ-ಬು-ದ್ಧಿ-ಯು-ಳ್ಳ-ವರು
ಸ್ಥಿರ-ವಾಗಿ
ಸ್ಥಿರ-ವಾ-ಗಿ-ಬಿ-ಟ್ಟರು
ಸ್ಥಿರ-ವಾ-ಗಿ-ರ-ಬೇ-ಕಾ-ಗು-ತ್ತದೆ
ಸ್ಥಿರ-ವಾ-ಗಿ-ರ-ಬೇ-ಕಾ-ದರೆ
ಸ್ಥಿರ-ವಾದ
ಸ್ಥಿರ-ಶಾಂ-ತ-ಬು-ದ್ಧಿ-ಯ-ವರೂ
ಸ್ಥೂಲ
ಸ್ನಾನ
ಸ್ನಾನಕ್ಕೆ
ಸ್ನೆಲ್
ಸ್ನೆಲ್ರಂ-ತಹ
ಸ್ನೆಲ್ರ-ವರು
ಸ್ನೇಹ
ಸ್ನೇಹ-ಪ್ರೇ-ಮ-ಭ-ಕ್ತಿ-ಆ-ದ-ರ-ಗಳನ್ನು
ಸ್ನೇಹ-ಸಂ-ಬಂ-ಧ-ಗಳು
ಸ್ನೇಹಕ್ಕೆ
ಸ್ನೇಹ-ಗಳನ್ನು
ಸ್ನೇಹದ
ಸ್ನೇಹ-ಪೂರ್ಣ
ಸ್ನೇಹ-ಭಾ-ವ-ದಿಂದ
ಸ್ನೇಹ-ಭಾ-ವ-ದಿಂ-ದಲೇ
ಸ್ನೇಹ-ಮಯ
ಸ್ನೇಹ-ವನ್ನು
ಸ್ನೇಹ-ವಾ-ಯಿತು
ಸ್ನೇಹವು
ಸ್ನೇಹ-ವೇ-ರ್ಪ-ಟ್ಟಿತ್ತು
ಸ್ನೇಹ-ಸಂ-ಬಂ-ಧ-ವನ್ನು
ಸ್ನೇಹಿತ
ಸ್ನೇಹಿ-ತನ
ಸ್ನೇಹಿ-ತ-ನಾದ
ಸ್ನೇಹಿ-ತ-ನಿಗೆ
ಸ್ನೇಹಿ-ತನೂ
ಸ್ನೇಹಿ-ತ-ನೊಂ-ದಿಗೆ
ಸ್ನೇಹಿ-ತರ
ಸ್ನೇಹಿ-ತ-ರ-ಶಿ-ಷ್ಯ-ರೆಲ್ಲ
ಸ್ನೇಹಿ-ತ-ರ-ನ್ನಾಗಿ
ಸ್ನೇಹಿ-ತ-ರನ್ನು
ಸ್ನೇಹಿ-ತ-ರ-ನ್ನು-ದ್ದೇ-ಶಿಸಿ
ಸ್ನೇಹಿ-ತ-ರನ್ನೂ
ಸ್ನೇಹಿ-ತ-ರ-ಲ್ಲೆಲ್ಲ
ಸ್ನೇಹಿ-ತ-ರ-ಲ್ಲೊ-ಬ್ಬ-ರಾಗಿ
ಸ್ನೇಹಿ-ತ-ರಾದ
ಸ್ನೇಹಿ-ತ-ರಾ-ದರು
ಸ್ನೇಹಿ-ತ-ರಾ-ದೆವು
ಸ್ನೇಹಿ-ತ-ರಿಂದ
ಸ್ನೇಹಿ-ತ-ರಿಗೂ
ಸ್ನೇಹಿ-ತ-ರಿಗೆ
ಸ್ನೇಹಿ-ತ-ರಿ-ಗೆಲ್ಲ
ಸ್ನೇಹಿ-ತ-ರಿ-ದ್ದಾರೆ
ಸ್ನೇಹಿ-ತ-ರಿ-ಬ್ಬ-ರಿಗೂ
ಸ್ನೇಹಿ-ತ-ರಿಲ್ಲ
ಸ್ನೇಹಿ-ತರು
ಸ್ನೇಹಿ-ತ-ರು-ಶಿ-ಷ್ಯರು
ಸ್ನೇಹಿ-ತ-ರು-ಶಿ-ಷ್ಯ-ರು-ಭ-ಕ್ತ-ರು
ಸ್ನೇಹಿ-ತರೂ
ಸ್ನೇಹಿ-ತ-ರೆಲ್ಲ
ಸ್ನೇಹಿ-ತ-ರೊಂ-ದಿಗೆ
ಸ್ನೇಹಿ-ತ-ರೊಬ್ಬ
ಸ್ನೇಹಿ-ತ-ರೊ-ಬ್ಬರು
ಸ್ನೇಹಿತೆ
ಸ್ನೇಹಿ-ತೆಗೆ
ಸ್ನೇಹಿ-ತೆ-ಯ-ರಿಗೆ
ಸ್ನೇಹಿ-ತೆಯೂ
ಸ್ನೇಹಿ-ತೆ-ಯೊ-ಬ್ಬ-ಳಿಗೆ
ಸ್ಪಂದ-ನ-ಗೊ-ಳ್ಳು-ತ್ತಿತ್ತು
ಸ್ಪಂದ-ನ-ವನ್ನು
ಸ್ಪಂದಿ-ಸಿ-ದುವು
ಸ್ಪಂದಿ-ಸು-ತ್ತಿದೆ
ಸ್ಪಂದಿ-ಸು-ವಂತೆ
ಸ್ಪರ್ಡಿ-ಯನ್ನೂ
ಸ್ಪರ್ಧಾ-ತ್ಮ-ಕ-ತೆಯ
ಸ್ಪರ್ಧಿ-ಸಿ-ದರು
ಸ್ಪರ್ಧೆ-ಗಳನ್ನು
ಸ್ಪರ್ಧೆ-ಗಾಗಿ
ಸ್ಪರ್ಧೆ-ಯೆಂಬ
ಸ್ಪರ್ಧೆಯೇ
ಸ್ಪರ್ಶ
ಸ್ಪರ್ಶ-ದಿಂದ
ಸ್ಪರ್ಶ-ಮಾಡಿ
ಸ್ಪರ್ಶ-ಮಾ-ತ್ರ-ದಿಂದ
ಸ್ಪರ್ಶ-ಮಾ-ತ್ರ-ದಿಂ-ದ-ಅ-ಷ್ಟೇಕೆ
ಸ್ಪರ್ಶ-ಮಾ-ತ್ರ-ದಿಂ-ದಲೇ
ಸ್ಪರ್ಶಿಯೂ
ಸ್ಪರ್ಶಿಸಿ
ಸ್ಪರ್ಶಿ-ಸಿ-ದರು
ಸ್ಪರ್ಶಿ-ಸಿ-ದರೆ
ಸ್ಪಷ್ಟ
ಸ್ಪಷ್ಟ-ದ-ರ್ಶನ
ಸ್ಪಷ್ಟ-ಪ-ಡಿ-ಸ-ಲೇ-ಬೇ-ಕೆಂದು
ಸ್ಪಷ್ಟ-ಪ-ಡಿ-ಸಿ-ದರು
ಸ್ಪಷ್ಟ-ಪ-ಡಿಸು
ಸ್ಪಷ್ಟ-ಪ-ಡಿ-ಸು-ತ್ತಾರೆ
ಸ್ಪಷ್ಟ-ಪ-ಡಿ-ಸು-ತ್ತಿ-ದ್ದರು
ಸ್ಪಷ್ಟ-ಪ-ಡಿ-ಸು-ವಂ-ತೆಯೂ
ಸ್ಪಷ್ಟ-ಪ-ಡಿ-ಸು-ವು-ದ-ಕ್ಕಾಗಿ
ಸ್ಪಷ್ಟ-ವಾ-ಗಲಿ
ಸ್ಪಷ್ಟ-ವಾ-ಗಲು
ಸ್ಪಷ್ಟ-ವಾಗಿ
ಸ್ಪಷ್ಟ-ವಾ-ಗಿತ್ತು
ಸ್ಪಷ್ಟ-ವಾ-ಗಿ-ತ್ತೆಂ-ದರೆ
ಸ್ಪಷ್ಟ-ವಾ-ಗಿದೆ
ಸ್ಪಷ್ಟ-ವಾ-ಗಿಯೇ
ಸ್ಪಷ್ಟ-ವಾ-ಗಿಲ್ಲ
ಸ್ಪಷ್ಟ-ವಾ-ಗಿ-ಸಿದೆ
ಸ್ಪಷ್ಟ-ವಾ-ಗು-ತ್ತದೆ
ಸ್ಪಷ್ಟ-ವಾ-ಗು-ತ್ತಿದೆ
ಸ್ಪಷ್ಟ-ವಾ-ಗು-ವ-ವ-ರೆಗೂ
ಸ್ಪಷ್ಟ-ವಾದ
ಸ್ಪಷ್ಟ-ವಾ-ಯಿತು
ಸ್ಪಷ್ಟ-ವಾ-ಯಿ-ತು-ಭಾ-ರ-ತವು
ಸ್ಪಷ್ಟ-ವಾ-ಯಿ-ತೆಂ-ದರೆ
ಸ್ಪಷ್ಪವೂ
ಸ್ಪೆನ್ಸ-ರರ
ಸ್ಪೆಯಿ-ನಿನ
ಸ್ಪೆಯಿ-ನಿ-ನಿಂದ
ಸ್ಪೆಯಿನ್
ಸ್ಪ್ರಿಂಗ್ಸ್
ಸ್ಫುಟ-ವಾಗಿ
ಸ್ಫುಟ-ವಾ-ಗಿ-ರು-ತ್ತಿ-ತ್ತಾ-ದರೂ
ಸ್ಫುಟ-ವಾದ
ಸ್ಫುರ-ದ್ರೂ-ಪ-ವನ್ನು
ಸ್ಫುರಿ-ಸಿತು
ಸ್ಫೂರ್ತಿ
ಸ್ಫೂರ್ತಿ-ಗಳಿಂದ
ಸ್ಫೂರ್ತಿ-ಗೊಂಡ
ಸ್ಫೂರ್ತಿ-ಗೊಂ-ಡಿ-ದ್ದರು
ಸ್ಫೂರ್ತಿ-ಗೊಂಡು
ಸ್ಫೂರ್ತಿ-ದಾ-ಯಕ
ಸ್ಫೂರ್ತಿ-ದಾ-ಯ-ಕ-ವಾ-ಗಿದೆ
ಸ್ಫೂರ್ತಿ-ದಾ-ಯ-ಕ-ವಾ-ಗಿ-ದ್ದುವು
ಸ್ಫೂರ್ತಿ-ದಾ-ಯ-ಕ-ವಾದ
ಸ್ಫೂರ್ತಿ-ದಾ-ಯ-ಕವೂ
ಸ್ಫೂರ್ತಿ-ಪ-ಡೆ-ದಿದ್ದ
ಸ್ಫೂರ್ತಿಯ
ಸ್ಫೂರ್ತಿ-ಯನ್ನು
ಸ್ಫೂರ್ತಿ-ಯಿಂದ
ಸ್ಫೂರ್ತಿ-ಯಿಂ-ದು-ದಿ-ಸಿ-ದಂ-ಥವು
ಸ್ಫೂರ್ತಿಯು
ಸ್ಫೂರ್ತಿ-ಯು-ಕ್ಕಿತು
ಸ್ಫೂರ್ತಿ-ಯುತ
ಸ್ಫೂರ್ತಿ-ಯು-ತ-ರಾಗಿ
ಸ್ಫೂರ್ತಿ-ಯು-ತ-ವಾಗಿ
ಸ್ಫೂರ್ತಿ-ಯು-ತ-ವಾ-ಗಿ-ರು-ತ್ತಿತ್ತು
ಸ್ಫೂರ್ತಿ-ಯೆಂ-ದರೆ
ಸ್ಫೂರ್ತಿ-ಯೆಂದು
ಸ್ಫೂರ್ತಿ-ವಾಣಿ
ಸ್ಫೂರ್ತಿ-ವಾ-ಣಿ-ಗ-ಳಿ-ಗಾಗಿ
ಸ್ಫೋಟ-ವಾ-ಯಿತು
ಸ್ಫೋಟಿ-ಸಿತು
ಸ್ಮರ-ಣೀಯ
ಸ್ಮರ-ಣೀ-ಯ-ವಾ-ದ-ವು-ಗ-ಳ-ಲ್ಲೊಂದು
ಸ್ಮರಣೆ
ಸ್ಮರ-ಣೆ-ಗಳ
ಸ್ಮರ-ಣೆ-ಯಲ್ಲಿ
ಸ್ಮರಿ-ಸ-ಬ-ಹುದು
ಸ್ಮರಿಸಿ
ಸ್ಮರಿ-ಸಿ-ಕೊ-ಳ್ಳು-ವಂತೆ
ಸ್ಮರಿ-ಸಿ-ದರು
ಸ್ಮರಿ-ಸುತ್ತ
ಸ್ಮರಿ-ಸು-ತ್ತಾನೆ
ಸ್ಮರಿ-ಸು-ತ್ತಿ-ದ್ದರು
ಸ್ಮರಿ-ಸು-ವಂತೆ
ಸ್ಮಶಾ-ನ-ವೊಂ-ದರ
ಸ್ಮಾರ-ಕ-ಗಳನ್ನು
ಸ್ಮಾರ-ಕ-ಗ-ಳಿವೆ
ಸ್ಮಾರ-ಕವೇ
ಸ್ಮಿತ್
ಸ್ಮೃತಿ
ಸ್ಮೃತಿ-ಗಳನ್ನು
ಸ್ಮೃತಿ-ಚಿ-ತ್ರ-ಣ-ಗಳ
ಸ್ಮೃತಿ-ಚಿ-ತ್ರ-ಣ-ದಲ್ಲಿ
ಸ್ಮೃತಿ-ಚಿ-ತ್ರ-ಣ-ವನ್ನು
ಸ್ಮೃತಿ-ಲೇ-ಖ-ನ-ವನ್ನು
ಸ್ಮೃತಿ-ಲೇ-ಖ-ನ-ವೊಂ-ದ-ರಲ್ಲಿ
ಸ್ಯಾನ್
ಸ್ಯಾನ್ಬಾ-ರ್ನಳು
ಸ್ಯಾನ್ಬಾ-ರ್ನ್
ಸ್ಯಾನ್ಬಾರ್ನ್ಳ
ಸ್ಯಾನ್ಬಾ-ರ್ನ್ಳಿ-ಗಾ-ಗಲಿ
ಸ್ರೋತ-ವನ್ನು
ಸ್ರೋತ-ವನ್ನೇ
ಸ್ಲೇ
ಸ್ಲೇಟನ್
ಸ್ವ-ಸಂ-ಮೋ-ಹಿನಿ
ಸ್ವಂತ
ಸ್ವಂತ-ದ-ವರು
ಸ್ವಂತ-ದ-ವ-ರೆಂದು
ಸ್ವಂತ-ದ-ವ-ರೆಂ-ಬಂತೆ
ಸ್ವಂತ-ದ-ವ-ರೇನೋ
ಸ್ವಂತ-ದ್ದಲ್ಲ
ಸ್ವಂತಿಕೆ
ಸ್ವಂತಿ-ಕೆ-ಯನ್ನು
ಸ್ವಂತಿ-ಕೆ-ಯಿಂದ
ಸ್ವಂತಿ-ಕೆ-ಯು-ಳ್ಳವೂ
ಸ್ವಚ್ಛಂದ
ಸ್ವಚ್ಛ-ವಾದ
ಸ್ವಜ-ನ-ಸ್ವ-ಧ-ರ್ಮ-ಗಳ
ಸ್ವತಂತ್ರ
ಸ್ವತಂ-ತ್ರ-ನಾ-ಗಿ-ರ-ಬೇಕು
ಸ್ವತಂ-ತ್ರ-ರಾ-ದರು
ಸ್ವತಂ-ತ್ರರು
ಸ್ವತಂ-ತ್ರ-ವಾಗಿ
ಸ್ವತಂ-ತ್ರ-ವಾ-ಯಿತು
ಸ್ವತಂ-ತ್ರವೂ
ಸ್ವತಃ
ಸ್ವತ-ಸ್ಸಿ-ದ್ಧ-ವಾ-ಗಿದೆ
ಸ್ವದೇ-ಶಕ್ಕೆ
ಸ್ವದೇ-ಶ-ದಿಂದ
ಸ್ವಧರ್ಮ
ಸ್ವಧ-ರ್ಮಕ್ಕೆ
ಸ್ವಧ-ರ್ಮ-ದಲ್ಲಿ
ಸ್ವಪ್ನ-ಲೋ-ಕವೇ
ಸ್ವಪ್ರ-ತಿಷ್ಠೆ
ಸ್ವಪ್ರ-ತಿ-ಷ್ಠೆ-ಯಿಂದ
ಸ್ವಪ್ರ-ಯತ್ನ
ಸ್ವಪ್ರ-ಯ-ತ್ನ-ದಿಂ-ದಲೇ
ಸ್ವಭಾವ
ಸ್ವಭಾ-ವ-ಇವು
ಸ್ವಭಾ-ವಕ್ಕೆ
ಸ್ವಭಾ-ವಕ್ಕೇ
ಸ್ವಭಾ-ವ-ಗಳಲ್ಲಿ
ಸ್ವಭಾ-ವ-ಗಳು
ಸ್ವಭಾ-ವತಃ
ಸ್ವಭಾ-ವದ
ಸ್ವಭಾ-ವ-ದಲ್ಲಿ
ಸ್ವಭಾ-ವ-ದ-ವ-ನಾ-ಗಿದ್ದ
ಸ್ವಭಾ-ವ-ದ-ವರು
ಸ್ವಭಾ-ವ-ದ-ವಳು
ಸ್ವಭಾ-ವದ್ದು
ಸ್ವಭಾ-ವ-ವನ್ನು
ಸ್ವಭಾ-ವ-ವೆಂ-ತ-ಹ-ದೆಂದು
ಸ್ವಭಾ-ವವೇ
ಸ್ವಭಾ-ವ-ಸ-ಹಜ
ಸ್ವಭಾ-ವ-ಸ-ಹ-ಜ-ವಾ-ದ-ದ್ದ-ರಿಂದ
ಸ್ವಮ-ತಾ-ಭಿ-ಮಾನ
ಸ್ವಯಂ
ಸ್ವಯಂ-ಪೂರ್ಣ
ಸ್ವಯಂ-ಸೇ-ವ-ಕರು
ಸ್ವಯಂ-ಸ್ಫೂರ್ತಿ
ಸ್ವಯಂ-ಸ್ಫೂ-ರ್ತಿ-ಯಿಂದ
ಸ್ವರ
ಸ್ವರ-ದಲ್ಲಿ
ಸ್ವರ-ದಿಂದ
ಸ್ವರೂಪ
ಸ್ವರೂ-ಪದ
ಸ್ವರೂ-ಪ-ದಲ್ಲಿ
ಸ್ವರೂ-ಪ-ದವು
ಸ್ವರೂ-ಪ-ವನ್ನು
ಸ್ವರೂ-ಪವು
ಸ್ವರ್ಗ
ಸ್ವರ್ಗ-ದ-ಲ್ಲಿ-ರುವ
ಸ್ವರ್ಗ-ದ-ಲ್ಲಿ-ರು-ವ-ವರು
ಸ್ವರ್ಗ-ದಿಂದ
ಸ್ವರ್ಜಸ್
ಸ್ವಲ್ಪ
ಸ್ವಲ್ಪ-ಕಾ-ಲ-ದಲ್ಲೇ
ಸ್ವಲ್ಪ-ಮ-ಟ್ಟಿಗೆ
ಸ್ವಲ್ಪ-ಮ-ಟ್ಟಿನ
ಸ್ವಲ್ಪ-ವಾ-ದರೂ
ಸ್ವಲ್ಪ-ವಾದೂ
ಸ್ವಲ್ಪವೂ
ಸ್ವಲ್ಪವೇ
ಸ್ವಲ್ಪ-ಸ್ವಲ್ಪ
ಸ್ವಲ್ಪ-ಸ್ವ-ಲ್ಪ-ವಾಗಿ
ಸ್ವಲ್ಪ-ಹೊ-ತ್ತಿನ
ಸ್ವಲ್ಪ-ಹೊ-ತ್ತಿ-ನಲ್ಲೇ
ಸ್ವಲ್ಪ-ಹೊತ್ತು
ಸ್ವಸಂ-ತೋಷ
ಸ್ವಸಂ-ತೋ-ಷ-ದಿಂದ
ಸ್ವಸ್ತಿ
ಸ್ವಸ್ಥ-ಳ-ಕ್ಕಿಂತ
ಸ್ವಸ್ಥ-ಳ-ದಿಂದ
ಸ್ವಸ್ವ-ರೂ-ಪ-ದಾಳ
ಸ್ವಸ್ವಾ-ಮ-ರ್ಥ್ಯ-ದಿಂ-ದೇನೂ
ಸ್ವಾಗತ
ಸ್ವಾಗ-ತಕ್ಕೆ
ಸ್ವಾಗ-ತ-ಗೀತೆ
ಸ್ವಾಗ-ತ-ಭಾ-ಷ-ಣ-ಗ-ಳಿಗೆ
ಸ್ವಾಗ-ತ-ವನ್ನು
ಸ್ವಾಗ-ತ-ವಿತ್ತು
ಸ್ವಾಗ-ತ-ವಿ-ದೆ-ಯೆಂದು
ಸ್ವಾಗ-ತವು
ಸ್ವಾಗ-ತ-ವೇನೂ
ಸ್ವಾಗ-ತ-ಸ-ಮಿ-ತಿಯ
ಸ್ವಾಗ-ತಹ
ಸ್ವಾಗ-ತಾರ್ಹ
ಸ್ವಾಗ-ತಿ-ಸ-ಲಾ-ಯಿತು
ಸ್ವಾಗ-ತಿ-ಸಲು
ಸ್ವಾಗ-ತಿಸಿ
ಸ್ವಾಗ-ತಿ-ಸಿದ
ಸ್ವಾಗ-ತಿ-ಸಿ-ದರು
ಸ್ವಾಗ-ತಿ-ಸಿ-ದಳು
ಸ್ವಾಗ-ತಿ-ಸಿ-ದು-ವ-ಲ್ಲದೆ
ಸ್ವಾಗ-ತಿ-ಸಿ-ದುವು
ಸ್ವಾಗ-ತಿ-ಸಿ-ರ-ಬ-ಹುದು
ಸ್ವಾಗ-ತಿ-ಸು-ತ್ತದೆ
ಸ್ವಾಗ-ತಿ-ಸು-ತ್ತಾರೆ
ಸ್ವಾಗ-ತಿ-ಸುವ
ಸ್ವಾಗ-ತಿ-ಸು-ವ-ರೆಂದು
ಸ್ವಾಗ-ತಿ-ಸು-ವುದನ್ನು
ಸ್ವಾತಂ-ತ್ರ-ವಿ-ರ-ಬೇಕೋ
ಸ್ವಾತಂತ್ರ್ಯ
ಸ್ವಾತಂ-ತ್ರ್ಯದ
ಸ್ವಾತಂ-ತ್ರ್ಯ-ಪೂ-ರ್ವ-ದಲ್ಲಿ
ಸ್ವಾತಂ-ತ್ರ್ಯ-ಪ್ರಿ-ಯ-ತೆ-ಯೆಂ-ಥದು
ಸ್ವಾತಂ-ತ್ರ್ಯ-ಪ್ರಿ-ಯರು
ಸ್ವಾತಂ-ತ್ರ್ಯ-ವನ್ನು
ಸ್ವಾತಂ-ತ್ರ್ಯ-ವಿತ್ತು
ಸ್ವಾತಂ-ತ್ರ್ಯ-ವೊಂದೇ
ಸ್ವಾಧೀ-ನ-ಪ-ಡಿ-ಸಿ-ಕೊಂ-ಡಿ-ದ್ದಾನೆ
ಸ್ವಾಧೀ-ನ-ಪ-ಡಿ-ಸಿ-ಕೊಂ-ಡು-ಬಿ-ಟ್ಟರು
ಸ್ವಾಭಾ-ವಿ-ಕ-ವಾ-ಗಿಯೇ
ಸ್ವಾಭಿ-ಮಾನಿ
ಸ್ವಾಭಿ-ಮಾನೀ
ಸ್ವಾಮಿ
ಸ್ವಾಮಿ-ಗಳ
ಸ್ವಾಮಿ-ಗ-ಳ-ವ-ರಿಗೆ
ಸ್ವಾಮಿ-ಗ-ಳಿ-ಗಿ-ರುವ
ಸ್ವಾಮಿ-ಗ-ಳಿಗೂ
ಸ್ವಾಮಿ-ಗ-ಳಿಗೆ
ಸ್ವಾಮಿ-ಗ-ಳಿ-ಬ್ಬರೂ
ಸ್ವಾಮಿ-ಗಳು
ಸ್ವಾಮಿ-ಗ-ಳೆಂಬ
ಸ್ವಾಮಿಜಿ
ಸ್ವಾಮಿ-ಜಿಯ
ಸ್ವಾಮಿ-ನಿಷ್ಠೆ
ಸ್ವಾಮಿಯ
ಸ್ವಾಮಿ-ವಿ-ವೇ-ಕಾ-ನಂ-ದರ
ಸ್ವಾಮೀ-ಜ-ಯನ್ನು
ಸ್ವಾಮೀಜಿ
ಸ್ವಾಮೀ-ಜಿ-ಎಂಬ
ಸ್ವಾಮೀ-ಜಿ-ಒಂದು
ಸ್ವಾಮೀ-ಜಿ-ಗಂತೂ
ಸ್ವಾಮೀ-ಜಿ-ಗಾಗಿ
ಸ್ವಾಮೀ-ಜಿ-ಗಾದ
ಸ್ವಾಮೀ-ಜಿ-ಗಿತ್ತು
ಸ್ವಾಮೀ-ಜಿ-ಗಿದ್ದ
ಸ್ವಾಮೀ-ಜಿಗೂ
ಸ್ವಾಮೀ-ಜಿಗೆ
ಸ್ವಾಮೀ-ಜಿಗೇ
ಸ್ವಾಮೀ-ಜಿ-ಗೊಂದು
ಸ್ವಾಮೀ-ಜಿ-ನೀವು
ಸ್ವಾಮೀ-ಜಿಯ
ಸ್ವಾಮೀ-ಜಿ-ಯಂ-ತಹ
ಸ್ವಾಮೀ-ಜಿ-ಯಂ-ಥ-ವರ
ಸ್ವಾಮೀ-ಜಿ-ಯತ್ತ
ಸ್ವಾಮೀ-ಜಿ-ಯ-ದಾ-ಗಿತ್ತು
ಸ್ವಾಮೀ-ಜಿ-ಯದು
ಸ್ವಾಮೀ-ಜಿ-ಯನ್ನ
ಸ್ವಾಮೀ-ಜಿ-ಯನ್ನು
ಸ್ವಾಮೀ-ಜಿ-ಯನ್ನೇ
ಸ್ವಾಮೀ-ಜಿ-ಯ-ನ್ನೊಮ್ಮೆ
ಸ್ವಾಮೀ-ಜಿ-ಯಲ್ಲಿ
ಸ್ವಾಮೀ-ಜಿ-ಯ-ಲ್ಲಿದ್ದ
ಸ್ವಾಮೀ-ಜಿ-ಯ-ಲ್ಲಿ-ಮತ್ತು
ಸ್ವಾಮೀ-ಜಿ-ಯ-ವರ
ಸ್ವಾಮೀ-ಜಿ-ಯ-ವ-ರದು
ಸ್ವಾಮೀ-ಜಿ-ಯ-ವ-ರನ್ನು
ಸ್ವಾಮೀ-ಜಿ-ಯ-ವ-ರನ್ನೂ
ಸ್ವಾಮೀ-ಜಿ-ಯ-ವ-ರಿಂದ
ಸ್ವಾಮೀ-ಜಿ-ಯ-ವ-ರಿಗೆ
ಸ್ವಾಮೀ-ಜಿ-ಯ-ವರು
ಸ್ವಾಮೀ-ಜಿ-ಯ-ವರೆಲ್ಲಿ
ಸ್ವಾಮೀ-ಜಿ-ಯ-ವ-ರೊಂ-ದಿಗೆ
ಸ್ವಾಮೀ-ಜಿ-ಯಾ-ದರೋ
ಸ್ವಾಮೀ-ಜಿ-ಯಿಂದ
ಸ್ವಾಮೀ-ಜಿ-ಯಿದ್ದ
ಸ್ವಾಮೀ-ಜಿ-ಯಿನ್ನೂ
ಸ್ವಾಮೀ-ಜಿಯೂ
ಸ್ವಾಮೀ-ಜಿ-ಯೆಂ-ದರು
ಸ್ವಾಮೀ-ಜಿಯೇ
ಸ್ವಾಮೀ-ಜಿ-ಯೇ-ನಾ-ದರೂ
ಸ್ವಾಮೀ-ಜಿ-ಯೇನೂ
ಸ್ವಾಮೀ-ಜಿ-ಯೊಂ-ದಿ-ಗಿದ್ದ
ಸ್ವಾಮೀ-ಜಿ-ಯೊಂ-ದಿ-ಗಿನ
ಸ್ವಾಮೀ-ಜಿ-ಯೊಂ-ದಿಗೆ
ಸ್ವಾಮೀ-ಜಿ-ಯೊಂ-ದಿಗೇ
ಸ್ವಾಮೀ-ಜಿ-ಯೊ-ಡನೆ
ಸ್ವಾಮೀ-ಜಿ-ಯೊ-ಬ್ಬರ
ಸ್ವಾಮೀಜೀ
ಸ್ವಾರಸ್ಯ
ಸ್ವಾರ-ಸ್ಯ-ಕರ
ಸ್ವಾರ-ಸ್ಯ-ಕ-ರ-ವಾಗಿ
ಸ್ವಾರ-ಸ್ಯ-ಕ-ರ-ವಾ-ಗಿತ್ತು
ಸ್ವಾರ-ಸ್ಯ-ಕ-ರ-ವಾ-ಗಿದೆ
ಸ್ವಾರ-ಸ್ಯ-ಕ-ರ-ವಾದ
ಸ್ವಾರ-ಸ್ಯ-ಕ-ರವೂ
ಸ್ವಾರ-ಸ್ಯದ
ಸ್ವಾರ-ಸ್ಯ-ಮಯ
ಸ್ವಾರ-ಸ್ಯ-ವನ್ನು
ಸ್ವಾರ-ಸ್ಯ-ವಾಗಿ
ಸ್ವಾರ-ಸ್ಯ-ವಾ-ಗಿದೆ
ಸ್ವಾರ್ಥ-ತೆಯ
ಸ್ವಾರ್ಥ-ತೆಯೇ
ಸ್ವಾರ್ಥದ
ಸ್ವಾರ್ಥ-ಪರ
ಸ್ವಾರ್ಥ-ಪ-ರ-ತೆ-ಯನ್ನು
ಸ್ವಾರ್ಥ-ಪ್ರೇ-ರಿತ
ಸ್ವಾರ್ಥ-ಪ್ರೇ-ರಿ-ತ-ರಾಗಿ
ಸ್ವಾರ್ಥ-ಬು-ದ್ಧಿಯ
ಸ್ವಾರ್ಥ-ರ-ಹಿ-ತತೆ
ಸ್ವಾರ್ಥ-ರ-ಹಿ-ತವೂ
ಸ್ವಾರ್ಥ-ವನ್ನು
ಸ್ವಾರ್ಥ-ವನ್ನೂ
ಸ್ವಾರ್ಥವೇ
ಸ್ವಾರ್ಥ-ಸಾ-ಧ-ನೆಯ
ಸ್ವಾರ್ಥಿ
ಸ್ವಾರ್ಥಿ-ಗ-ಳೇಕೆ
ಸ್ವಾರ್ಥಿ-ಯ-ನ್ನಾ-ಗಿಸಿ
ಸ್ವಾರ್ಥೋ-ದ್ದೇ-ಶ-ಗಳನ್ನೂ
ಸ್ವಾರ್ಥೋ-ದ್ದೇ-ಶ-ಗ-ಳಿಗೆ
ಸ್ವಾವ-ಲಂ-ಬ-ನೆಯ
ಸ್ವಾವ-ಲಂಬಿ
ಸ್ವಾವ-ಲಂ-ಬಿ-ಗ-ಳಾ-ಗ-ಬೇ-ಕೆಂಬ
ಸ್ವಾವ-ಲಂ-ಬಿ-ಗ-ಳಾಗಿ
ಸ್ವಿಟ್ಸರ್
ಸ್ವಿಟ್ಸ-ರ್ಲ್ಯಾಂ-ಡಿಗೆ
ಸ್ವಿಟ್ಸ-ರ್ಲ್ಯಾಂ-ಡಿನ
ಸ್ವಿಟ್ಸ-ರ್ಲ್ಯಾಂಡ್
ಸ್ವೀಕರಿ
ಸ್ವೀಕ-ರಿ-ಸದೆ
ಸ್ವೀಕ-ರಿ-ಸ-ಬೇ-ಕಲ್ಲ
ಸ್ವೀಕ-ರಿ-ಸ-ಬೇಕು
ಸ್ವೀಕ-ರಿ-ಸ-ಲಾ-ಗಿತ್ತು
ಸ್ವೀಕ-ರಿ-ಸ-ಲಾ-ರಂ-ಭಿ-ಸಿ-ದರು
ಸ್ವೀಕ-ರಿ-ಸ-ಲಾ-ರ-ನೆಂದು
ಸ್ವೀಕ-ರಿ-ಸಲು
ಸ್ವೀಕ-ರಿ-ಸ-ಲೇ-ಬೇ-ಕೆಂದು
ಸ್ವೀಕ-ರಿ-ಸ-ಲ್ಪಟ್ಟ
ಸ್ವೀಕ-ರಿ-ಸ-ಲ್ಪ-ಡ-ಬೇ-ಕಾ-ದರೆ
ಸ್ವೀಕ-ರಿಸಿ
ಸ್ವೀಕ-ರಿ-ಸಿದ
ಸ್ವೀಕ-ರಿ-ಸಿ-ದ-ರಾ-ದರೂ
ಸ್ವೀಕ-ರಿ-ಸಿ-ದರು
ಸ್ವೀಕ-ರಿ-ಸಿ-ದ-ರೆಂ-ದರೆ
ಸ್ವೀಕ-ರಿ-ಸಿ-ದ-ಳೆಂದು
ಸ್ವೀಕ-ರಿ-ಸಿ-ದಾಗ
ಸ್ವೀಕ-ರಿ-ಸಿ-ದೆವು
ಸ್ವೀಕ-ರಿ-ಸಿ-ದ್ದರು
ಸ್ವೀಕ-ರಿ-ಸಿದ್ದೂ
ಸ್ವೀಕ-ರಿ-ಸಿ-ಯಾರು
ಸ್ವೀಕ-ರಿ-ಸು-ತ್ತಾರೆ
ಸ್ವೀಕ-ರಿ-ಸು-ತ್ತಾ-ರೆಯೋ
ಸ್ವೀಕ-ರಿ-ಸು-ತ್ತಿ-ದ್ದರು
ಸ್ವೀಕ-ರಿ-ಸು-ತ್ತೇನೆ
ಸ್ವೀಕ-ರಿ-ಸುವ
ಸ್ವೀಕ-ರಿ-ಸು-ವಂ-ತಿಲ್ಲ
ಸ್ವೀಕ-ರಿ-ಸು-ವಂತೆ
ಸ್ವೀಕ-ರಿ-ಸು-ವಲ್ಲಿ
ಸ್ವೀಕ-ರಿ-ಸು-ವ-ವರು
ಸ್ವೀಕ-ರಿ-ಸು-ವು-ದನ್ನೂ
ಸ್ವೀಕ-ರಿ-ಸು-ವು-ದಾ-ಗ-ಬೇಕು
ಸ್ವೀಕ-ರಿ-ಸು-ವುದು
ಸ್ವೀಕಾರ
ಸ್ವೀಕಾ-ರ-ಭಾ-ವ-ವನ್ನೂ
ಸ್ವೀಕಾ-ರ-ವಲ್ಲ
ಸ್ವೀಕಾ-ರಾ-ರ್ಹ-ವಾದ
ಸ್ವೀಕೃ-ತ-ರಾ-ಗು-ವುದು
ಸ್ಸನ್ನು
ಹಂ
ಹಂಗಾ-ದರೂ
ಹಂಗಿಗೂ
ಹಂಗಿಸು
ಹಂಚ-ಲಾ-ಯಿತು
ಹಂಚಿಕೆ
ಹಂಚಿ-ಕೊಂ-ಡಂತೆ
ಹಂಚಿ-ಕೊ-ಳ್ಳ-ಬೇ-ಕು-ಹೆಂ-ಡ-ತಿ-ಯನ್ನೂ
ಹಂಚಿ-ಕೊ-ಳ್ಳು-ವಂ-ತಾ-ಗ-ಬೇಕು
ಹಂಚಿ-ಕೊ-ಳ್ಳು-ವುದೂ
ಹಂಚಿ-ದ್ದರು
ಹಂಚಿ-ಬಿಡಿ
ಹಂಚಿ-ಹೋ-ಗು-ತ್ತಿ-ದ್ದರು
ಹಂತ
ಹಂತಕ್ಕೆ
ಹಂತ-ಗಳಲ್ಲಿ
ಹಂತ-ಗಳು
ಹಂತ-ದಲ್ಲಿ
ಹಂತ-ವ-ನ್ನೀಗ
ಹಂತ-ವನ್ನು
ಹಂದಿ
ಹಂದಿ-ಗಳ
ಹಂದಿ-ಗಳನ್ನು
ಹಂದಿ-ಮಾಂ-ಸ-ವನ್ನು
ಹಂದಿಯ
ಹಂಬಲ
ಹಂಬ-ಲ-ಇ-ವು-ಗ-ಳೆ-ಲ್ಲ-ದರ
ಹಂಬ-ಲ-ವಿ-ತ್ತಾ-ದರೂ
ಹಂಬ-ಲವು
ಹಂಬ-ಲವೂ
ಹಂಬ-ಲ-ವೊಂ-ದಿ-ದ್ದರೆ
ಹಂಬ-ಲಿ-ಸಿದ
ಹಂಬ-ಲಿ-ಸು-ತ್ತಾ-ರೆಯೋ
ಹಂಸ
ಹಕ್ಕಿ
ಹಕ್ಕಿದೆ
ಹಕ್ಕಿಯ
ಹಕ್ಕಿಲ್ಲ
ಹಕ್ಕು-ಗಳನ್ನು
ಹಕ್ಕು-ಗಳನ್ನೂ
ಹಕ್ಕು-ಗ-ಳಿವೆ
ಹಕ್ಕು-ದಾ-ರರು
ಹಕ್ಸ್ಲೀ
ಹಗ-ಲಿ-ರುಳು
ಹಗ-ಲಿ-ರುಳೂ
ಹಗ-ಲಿ-ರು-ಳೆ-ನ್ನದೆ
ಹಗಲು
ಹಗ-ಲು
ಹಗ-ಲು-ಮೂರು
ಹಗಲೂ
ಹಗು-ರ-ಗೊ-ಳಿ-ಸಲು
ಹಗು-ರ-ವಾಗಿ
ಹಗ್ಗ-ಗಳಿಂದ
ಹಚ್ಚಿ
ಹಚ್ಚಿ-ಕೊಂ-ಡರು
ಹಚ್ಚಿ-ಕೊಂ-ಡು-ಬಿ-ಟ್ಟಿ-ದ್ದರು
ಹಚ್ಚಿ-ಕೊ-ಳ್ಳ-ಬೇಕೆ
ಹಚ್ಚಿ-ಕೊ-ಳ್ಳ-ಬೇಡ
ಹಚ್ಚುವ
ಹಜಾರ
ಹಜಾ-ರದ
ಹಜಾ-ರ-ದಲ್ಲಿ
ಹಜಾ-ರ-ವನ್ನು
ಹಜಾ-ರ-ವಿ-ದ್ದುದು
ಹಟ
ಹಟ-ತೊ-ಟ್ಟಂ-ತಿದೆ
ಹಟ-ಮಾ-ರಿಯೂ
ಹಟ-ವನ್ನು
ಹಟ-ಹಿ-ಡಿ-ದರೆ
ಹಠ-ಯೋ-ಗ-ವನ್ನು
ಹಡ-ಗನ್ನು
ಹಡ-ಗ-ನ್ನೇರಿ
ಹಡಗಿ
ಹಡ-ಗಿಗೆ
ಹಡ-ಗಿನ
ಹಡ-ಗಿ-ನಲ್ಲಿ
ಹಡಗು
ಹಡ-ಗು-ಗ-ಟ್ಟಲೆ
ಹಡ-ಗು-ಗಳ
ಹಡು-ಗು-ಪ್ರ-ಯಾ-ಣದ
ಹಡ್ಸನ್
ಹಣ
ಹಣ-ಕಾ-ಸಿನ
ಹಣ-ಕಾಸು
ಹಣ-ಕೊಟ್ಟು
ಹಣ-ಕ್ಕಾಗಿ
ಹಣಕ್ಕೆ
ಹಣ-ಗ-ಳಿ-ಸ-ಬೇಕು
ಹಣದ
ಹಣ-ದಲ್ಲಿ
ಹಣ-ದಾಸೆ
ಹಣ-ದಿಂ-ದೇನು
ಹಣ-ದೊಂ-ದಿಗೆ
ಹಣ-ವಂ-ತ-ರಿಂದ
ಹಣ-ವನ್ನು
ಹಣ-ವನ್ನೂ
ಹಣ-ವ-ನ್ನೆಲ್ಲ
ಹಣ-ವನ್ನೇ
ಹಣ-ವಿ-ರುವ
ಹಣವೂ
ಹಣ-ವೆಲ್ಲ
ಹಣ-ವೆ-ಲ್ಲ-ವನ್ನೂ
ಹಣ-ವೆ-ಲ್ಲವೂ
ಹಣ-ವೆ-ಲ್ಲಿಂದ
ಹಣ-ಸಂ-ಗ್ರಹ
ಹಣ-ಸಂ-ಗ್ರ-ಹ-ಣೆಯ
ಹಣ-ಸಂ-ಪಾ-ದನೆ
ಹಣೆ
ಹಣೆ-ಬ-ರ-ಹ-ವನ್ನು
ಹಣೆಯ
ಹಣೆ-ಯಲ್ಲಿ
ಹಣೇ-ಬ-ರಹ
ಹಣ್ಣನ್ನು
ಹಣ್ಣಾಗಿ
ಹಣ್ಣಿ-ನಿಂದ
ಹಣ್ಣು
ಹಣ್ಣು-ಹೂ-ವು-ಗಳ
ಹಣ್ಣು-ಗಳ
ಹಣ್ಣು-ಗಳನ್ನು
ಹಣ್ಣು-ಗಳನ್ನೆಲ್ಲ
ಹಣ್ಣು-ಗಳು
ಹತಾಶ
ಹತಾ-ಶ-ಹೃ-ದ-ಯ-ದಾ-ಳ-ದಿಂದ
ಹತಾ-ಶೆಯ
ಹತಾ-ಶೆ-ಯಾ-ದ್ದ-ರಿಂದ
ಹತಾ-ಶೆಯೇ
ಹತೋ-ಟಿಗೆ
ಹತೋ-ಟಿ-ಯಿ-ಲ್ಲದೆ
ಹತೋ-ಟಿಯೂ
ಹತ್ತ-ಬೇ-ಕಾ-ಗಿ-ದ್ದದ್ದು
ಹತ್ತಲು
ಹತ್ತಾರು
ಹತ್ತಿ
ಹತ್ತಿ-ಕೊಂ-ಡಿತು
ಹತ್ತಿ-ಕ್ಕ-ಲಾ-ರ-ದಂತೆ
ಹತ್ತಿ-ಕ್ಕ-ಲಾ-ರದೆ
ಹತ್ತಿ-ದರು
ಹತ್ತಿಯೇ
ಹತ್ತಿರ
ಹತ್ತಿ-ರಕ್ಕೆ
ಹತ್ತಿ-ರಕ್ಕೇ
ಹತ್ತಿ-ರದ
ಹತ್ತಿ-ರ-ದಲ್ಲೇ
ಹತ್ತಿ-ರ-ದಿಂದ
ಹತ್ತಿ-ರ-ವಾ-ದಂತೆ
ಹತ್ತಿ-ರ-ವೇನೂ
ಹತ್ತು
ಹತ್ತು-ಹ-ದಿ-ನೈದು
ಹತ್ತು-ಹ-ನ್ನೆ-ರಡು
ಹತ್ತು-ತ್ತದೆ
ಹತ್ತು-ವಾಗ
ಹತ್ತು-ವು-ದಿಲ್ಲ
ಹತ್ತು-ಸಾ-ವಿರ
ಹತ್ತೂ-ವರೆ
ಹತ್ತೇ
ಹತ್ತೊಂ-ಬ-ತ್ತ-ನೆಯ
ಹದಕ್ಕೆ
ಹದ-ಗೆ-ಟ್ಟಿತು
ಹದ-ಗೆ-ಟ್ಟಿತ್ತು
ಹದ-ಗೆ-ಟ್ಟಿ-ತ್ತೆಂ-ದರೆ
ಹದ-ಗೆ-ಡಿ-ಸ-ಬ-ಹುದು
ಹದ-ಗೆ-ಡು-ತ್ತಿ-ದ್ದು-ದನ್ನು
ಹದ-ಗೊ-ಳಿ-ಸಿ-ದು-ದನ್ನು
ಹದ-ವಾ-ಗಿತ್ತು
ಹದ-ವಾ-ದಂ-ತಿತ್ತು
ಹದಿ-ನಾರು
ಹದಿ-ನಾಲ್ಕು
ಹದಿ-ನೆಂಟು
ಹದಿ-ನೇಳು
ಹದಿ-ನೇ-ಳೂ-ಮು-ಕ್ಕಾಲು
ಹದಿ-ನೈದು
ಹದಿ-ನೈ-ದು-ಹ-ದಿ-ನಾರು
ಹನಿ
ಹನಿ-ಗಳನ್ನು
ಹನಿ-ಯೊಂದು
ಹನು-ಮಂ-ತನ
ಹನ್ನೆ-ರ-ಡ-ನೆಯ
ಹನ್ನೆ-ರಡು
ಹನ್ನೊಂದು
ಹಬ್ಬ-ವನ್ನು
ಹಬ್ಬ-ವುಂ-ಟು-ಮಾ-ಡು-ವಂ-ಥವು
ಹಬ್ಬಿ
ಹಬ್ಬಿ-ದು-ದರ
ಹಬ್ಬಿಸಿ
ಹರಕೆ
ಹರ-ಟಿ-ಕೊ-ಳ್ಳಲಿ
ಹರ-ಟುತ್ತ
ಹರ-ಟು-ತ್ತಿದ್ದ
ಹರ-ಟು-ತ್ತಿ-ದ್ದರೆ
ಹರ-ಟುವ
ಹರ-ಟು-ವುದು
ಹರ-ಟೆ-ಮ-ಲ್ಲರು
ಹರ-ಡ-ಬಲ್ಲ
ಹರ-ಡ-ಬೇಕು
ಹರ-ಡಲು
ಹರಡಿ
ಹರ-ಡಿ-ಕೊಂ-ಡಿ-ರುವ
ಹರ-ಡಿತು
ಹರ-ಡಿದ
ಹರ-ಡಿ-ದುವು
ಹರ-ಡಿ-ದ್ದನೋ
ಹರ-ಡಿ-ದ್ದಾನೆ
ಹರ-ಡಿ-ರುವ
ಹರ-ಡುತ್ತ
ಹರ-ಡು-ತ್ತಾ-ರೆಂ-ಬು-ದನ್ನು
ಹರ-ಡು-ತ್ತಿ-ದ್ದಂತೆ
ಹರ-ಡು-ತ್ತೀರಿ
ಹರ-ಡುವ
ಹರ-ಡು-ವಂತೆ
ಹರ-ಡು-ವುದು
ಹರ-ಣೆಗೆ
ಹರ-ಣೆ-ಯನ್ನು
ಹರ-ವಿದ
ಹರ-ವಿ-ಲಾಸ್
ಹರ-ಸ-ಬೇ-ಕೆಂದು
ಹರ-ಸಲಿ
ಹರಸಿ
ಹರ-ಸಿ-ದರು
ಹರ-ಸಿ-ದ್ದರು
ಹರ-ಸು-ತ್ತಾರೆ
ಹರ-ಸು-ವಂತೆ
ಹರಿ
ಹರಿ-ಜ-ನ-ರಿಗೂ
ಹರಿ-ತ-ವಾ-ಗಿತ್ತು
ಹರಿ-ತ-ವಾ-ಗಿ-ದ್ದರೂ
ಹರಿ-ತ-ವಾದ
ಹರಿ-ತವೋ
ಹರಿದ
ಹರಿ-ದದ್ದೇ
ಹರಿ-ದಾ-ಡಿ-ದುವು
ಹರಿ-ದಾ-ಡುವ
ಹರಿ-ದಾಸ
ಹರಿ-ದಾಸಿ
ಹರಿ-ದಾಸ್
ಹರಿ-ದಾ-ಸ್ಬಾಬು
ಹರಿ-ದಿವೆ
ಹರಿದು
ಹರಿ-ದು-ಕೊ-ಳ್ಳುವ
ಹರಿ-ದು-ಬಂ-ದಂ-ತಿದೆ
ಹರಿ-ದು-ಬಂ-ದಂತೆ
ಹರಿ-ದು-ಬಂ-ದು-ದನ್ನು
ಹರಿ-ದು-ಬ-ರು-ತ್ತಿದ್ದ
ಹರಿ-ದು-ಬ-ರುವ
ಹರಿ-ಪದ
ಹರಿ-ಪ-ದನ
ಹರಿ-ಪ-ದ-ನಲ್ಲಿ
ಹರಿ-ಪ-ದ-ನಿಗೆ
ಹರಿ-ಪ-ದ-ಬಾಬು
ಹರಿ-ಭಾಯ್
ಹರಿಯ
ಹರಿ-ಯ-ಲಾ-ರಂ-ಭಿ-ಸಿತು
ಹರಿ-ಯಿತು
ಹರಿ-ಯಿ-ಸಲಿ
ಹರಿ-ಯಿ-ಸಲು
ಹರಿ-ಯಿಸಿ
ಹರಿ-ಯಿ-ಸಿದ್ದು
ಹರಿ-ಯಿ-ಸಿ-ಬಿ-ಡು-ತ್ತಿ-ದ್ದರು
ಹರಿ-ಯಿ-ಸು-ತ್ತೇನೆ
ಹರಿ-ಯುತ್ತ
ಹರಿ-ಯು-ತ್ತಿತ್ತು
ಹರಿ-ಯು-ತ್ತಿ-ದ್ದುವು
ಹರಿ-ಯು-ತ್ತಿ-ರುವ
ಹರಿ-ಯುವ
ಹರಿ-ಯು-ವಂ-ತಾ-ಯಿತು
ಹರಿ-ಯು-ವಂತೆ
ಹರಿವ
ಹರಿ-ಸ-ತೊ-ಡ-ಗಿ-ದ-ರೆಂ-ದರೆ
ಹರಿ-ಸ-ಲಾ-ರಂ-ಭಿ-ಸಿ-ದರು
ಹರಿಸಿ
ಹರಿ-ಸಿಂಗ
ಹರಿ-ಸಿಂ-ಗನ
ಹರಿ-ಸಿಂ-ಗ-ನನ್ನು
ಹರಿ-ಸಿಂ-ಗ-ನಿಗೆ
ಹರಿ-ಸಿಂಗ್
ಹರಿ-ಸಿಂ-ಗ್-ಇ-ಬ್ಬರೂ
ಹರಿ-ಸಿ-ದಂ-ತಹ
ಹರಿ-ಸಿ-ದರು
ಹರಿ-ಸಿ-ರುವ
ಹರಿ-ಸು-ತ್ತಿ-ದ್ದರು
ಹರಿ-ಹ-ರಿದು
ಹರ್ಷಾ-ನಂದ
ಹರ್ಷಿ-ಸಿ-ದರು
ಹರ್ಷೋ-ದ್ಗಾರ
ಹರ್ಷೋ-ದ್ಗಾ-ರಗ
ಹರ್ಷೋ-ದ್ಗಾ-ರ-ಗಳ
ಹರ್ಷೋ-ದ್ಗಾ-ರ-ಗಳಿಂದ
ಹರ್ಷೋ-ದ್ಗಾ-ರವೂ
ಹರ್ಷೋ-ದ್ವೇ-ಗ-ಗೊಂಡು
ಹಲ
ಹಲವ
ಹಲ-ವನ್ನು
ಹಲ-ವರ
ಹಲ-ವ-ರಾ-ದರೆ
ಹಲ-ವರು
ಹಲ-ವಾರು
ಹಲವು
ಹಲ-ವೆಡೆ
ಹಲು-ಬಿ-ದರು
ಹಳಿ-ದು-ಕೊಂ-ಡರು
ಹಳೆಯ
ಹಳೆ-ಯದು
ಹಳೇ
ಹಳ್ಳ
ಹಳ್ಳಕ್ಕೆ
ಹಳ್ಳಿ
ಹಳ್ಳಿ-ಪ-ಟ್ಟ-ಣದ
ಹಳ್ಳಿಗ
ಹಳ್ಳಿ-ಗರ
ಹಳ್ಳಿ-ಗ-ರನ್ನು
ಹಳ್ಳಿ-ಗ-ರ-ನ್ನೆಲ್ಲ
ಹಳ್ಳಿ-ಗ-ರಲ್ಲಿ
ಹಳ್ಳಿ-ಗ-ರಿಂದ
ಹಳ್ಳಿ-ಗಳ
ಹಳ್ಳಿ-ಗಳಲ್ಲಿ
ಹಳ್ಳಿ-ಗ-ಳಿಗೂ
ಹಳ್ಳಿ-ಗ-ಳಿಗೆ
ಹಳ್ಳಿ-ಗಳೂ
ಹಳ್ಳಿಗೆ
ಹಳ್ಳಿಯ
ಹಳ್ಳಿ-ಯನ್ನು
ಹಳ್ಳಿ-ಯಲ್ಲಿ
ಹಳ್ಳಿ-ಯ-ವ-ರೆಗೂ
ಹಳ್ಳಿ-ಯಾದ
ಹಳ್ಳಿ-ಯಿಂದ
ಹಳ್ಳಿಯೂ
ಹಳ್ಳಿಯೇ
ಹಳ್ಳಿ-ಹ-ಳ್ಳಿಗೂ
ಹವ-ಣಿ-ಕೆ-ಯ-ಲ್ಲಿ-ದ್ದ-ವರು
ಹವಾ-ಗುಣ
ಹವಿ-ಸ್ಸನ್ನು
ಹಷೋ-ದ್ಗಾರ
ಹಷೋ-ದ್ಗಾ-ರದ
ಹಸ-ನ್ಮುಖ
ಹಸ-ನ್ಮು-ಖ-ರಾಗಿ
ಹಸ-ನ್ಮು-ಖಿ-ಗ-ಳಾ-ಗಿ-ದ್ದಾ-ರೆ-ಎಂದು
ಹಸಿದ
ಹಸಿ-ದಿತ್ತು
ಹಸಿ-ದಿದ್ದ
ಹಸಿ-ದಿ-ದ್ದಾ-ರೆಯೆ
ಹಸಿ-ದಿ-ದ್ದೇನೆ
ಹಸಿ-ದಿ-ದ್ದೇವೆ
ಹಸಿ-ದು-ಕೊಂಡಿ
ಹಸಿ-ರಾ-ಗಿದೆ
ಹಸಿ-ರಾ-ಗಿ-ದ್ದುವು
ಹಸಿರು
ಹಸಿ-ವನ್ನು
ಹಸಿ-ವಾ-ಗು-ತ್ತಿದೆ
ಹಸಿ-ವಾ-ದಾಗ
ಹಸಿವಿ
ಹಸಿ-ವಿ-ನಿಂದ
ಹಸಿವು
ಹಸಿ-ವು-ಅ-ತೃ-ಪ್ತಿ-ಗಳು
ಹಸಿ-ವೆ-ಯಾ-ಗಿ-ರ-ಬೇಕು
ಹಸಿ-ವೆ-ಯಿಂದ
ಹಸಿ-ವೆಯೂ
ಹಸು-ಗಳನ್ನು
ಹಸು-ಗೂ-ಸನ್ನು
ಹಸುರೆ
ಹಸ್ತ-ಕ್ಷೇಪ
ಹಸ್ತ-ಗಳನ್ನು
ಹಸ್ತ-ಪ್ರ-ತಿ-ಗಳನ್ನು
ಹಸ್ತ-ಪ್ರ-ತಿ-ಗ-ಳ-ಲ್ಲಿ-ರುವ
ಹಹ್ಹಹ್ಹ
ಹಾಂಗ್ಕಾಂ-ಗಿನ
ಹಾಂಗ್ಕಾಂಗ್
ಹಾಂಗ್ಕಾಂಗ್ನ
ಹಾಂಗ್ಕಾಂ-ಗ್ನಲ್ಲಿ
ಹಾಂಗ್ಕಾಂ-ಗ್ನಿಂದ
ಹಾಕದೆ
ಹಾಕ-ಬ-ಹು-ದಾ-ಗಿತ್ತು
ಹಾಕ-ಬಾ-ರದು
ಹಾಕ-ಬೇ-ಕಾದ
ಹಾಕ-ಬೇ-ಕೆಂಬ
ಹಾಕಲು
ಹಾಕಿ
ಹಾಕಿ-ಕೊಂಡ
ಹಾಕಿ-ಕೊಂ-ಡರು
ಹಾಕಿ-ಕೊಂ-ಡಿದ್ದ
ಹಾಕಿ-ಕೊಂ-ಡಿ-ದ್ದರೊ
ಹಾಕಿ-ಕೊಂ-ಡಿ-ದ್ದಾರೆ
ಹಾಕಿ-ಕೊಂ-ಡಿ-ರ-ಬೇಕು
ಹಾಕಿ-ಕೊಂಡು
ಹಾಕಿ-ಕೊ-ಟ್ಟರು
ಹಾಕಿ-ಕೊ-ಳ್ಳ-ಲಿಲ್ಲ
ಹಾಕಿತು
ಹಾಕಿತ್ತು
ಹಾಕಿದ
ಹಾಕಿ-ದಂ-ತಾ-ಯಿತು
ಹಾಕಿ-ದರು
ಹಾಕಿ-ದರೂ
ಹಾಕಿ-ದರೆ
ಹಾಕಿ-ದವು
ಹಾಕಿ-ದಾಗ
ಹಾಕಿ-ದ್ದರು
ಹಾಕಿ-ದ್ದಾರೆ
ಹಾಕಿ-ಬಿ-ಟ್ಟರು
ಹಾಕಿ-ಬಿ-ಡು-ತ್ತಿ-ದ್ದರು
ಹಾಕಿ-ರುವ
ಹಾಕಿಸಿ
ಹಾಕಿ-ಸಿ-ಕೊಂಡು
ಹಾಕಿ-ಸಿ-ಕೊ-ಳ್ಳು-ತ್ತಿ-ದ್ದುದೂ
ಹಾಕು
ಹಾಕುತ್ತ
ಹಾಕು-ತ್ತಾರೆ
ಹಾಕು-ತ್ತಾರೋ
ಹಾಕು-ತ್ತಿದ್ದ
ಹಾಕು-ತ್ತಿ-ದ್ದರು
ಹಾಕು-ತ್ತಿದ್ದೆ
ಹಾಕು-ತ್ತಿ-ರು-ವುದನ್ನು
ಹಾಕು-ತ್ತೇನೆ
ಹಾಕು-ತ್ತೇವೆ
ಹಾಕುವ
ಹಾಕು-ವ-ವರು
ಹಾಕು-ವು-ದ-ರಲ್ಲಿ
ಹಾಕು-ವು-ದ-ರಿಂದ
ಹಾಕು-ವುದು
ಹಾಗನ್ನಿ
ಹಾಗ-ನ್ನಿ-ಸಿತು
ಹಾಗಾ-ಗ-ಬಾ-ರ-ದೆಂದು
ಹಾಗಾ-ಗ-ಬೇ-ಕೆಂ-ಬುದು
ಹಾಗಾ-ಗಲಿ
ಹಾಗಾ-ಗ-ಲಿಲ್ಲ
ಹಾಗಾ-ದರೆ
ಹಾಗಾ-ಲಿ-ಲ್ಲ-ವಷ್ಟೇ
ಹಾಗಿ
ಹಾಗಿ-ದ್ದರೆ
ಹಾಗಿ-ದ್ದಲ್ಲಿ
ಹಾಗಿ-ರಲಿ
ಹಾಗಿ-ರಲೇ
ಹಾಗಿ-ರು-ವಲ್ಲಿ
ಹಾಗಿ-ರು-ವ-ಲ್ಲಿನ
ಹಾಗಿ-ರು-ವಾಗ
ಹಾಗಿ-ಲ್ಲದೆ
ಹಾಗೂ
ಹಾಗೆ
ಹಾಗೆಂದು
ಹಾಗೆ-ನಿ-ಸು-ವು-ದಿಲ್ಲ
ಹಾಗೆ-ನ್ನಿ-ಸಿತು
ಹಾಗೆ-ನ್ನಿ-ಸಿ-ದ್ದೇ-ಕೆಂದು
ಹಾಗೆ-ನ್ನಿ-ಸು-ವು-ದಿಲ್ಲ
ಹಾಗೆ-ಮಾ-ನ-ವನ
ಹಾಗೆಯೇ
ಹಾಗೆಲ್ಲ
ಹಾಗೆಲ್ಲಾ
ಹಾಗೆ-ಹಾ-ಗೆಯೇ
ಹಾಗೇ
ಹಾಗೇಕೆ
ಹಾಗೇ-ನಾ-ದರೂ
ಹಾಗೋ
ಹಾಜ-ರಿದ್ದ
ಹಾಜ-ರಿ-ದ್ದರು
ಹಾಜ-ರಿ-ದ್ದಳು
ಹಾಜ-ರಿ-ರು-ತ್ತಿದ್ದ
ಹಾಜ-ರಿ-ರು-ತ್ತಿ-ದ್ದರು
ಹಾಜ-ರಿ-ರು-ತ್ತಿ-ದ್ದಳು
ಹಾಜ-ರು-ಪ-ಡಿ-ಸಿ-ದರು
ಹಾಡ-ತೊ-ಡ-ಗಿ-ದ-ನೆಂ-ದರೆ
ಹಾಡ-ತೊ-ಡ-ಗಿ-ದ-ರೆಂ-ದರೆ
ಹಾಡನ್ನು
ಹಾಡ-ಬೇ-ಕೆಂದು
ಹಾಡ-ಲಾ-ರಂಭಿ
ಹಾಡ-ಲಾ-ರಂ-ಭಿ-ಸಿ-ದಳು
ಹಾಡ-ಲಾ-ರಂ-ಭಿಸು
ಹಾಡ-ಲಿಲ್ಲ
ಹಾಡಲು
ಹಾಡಿ
ಹಾಡಿ-ದರು
ಹಾಡಿ-ದರೆ
ಹಾಡಿ-ದಳು
ಹಾಡಿ-ದವು
ಹಾಡಿನ
ಹಾಡಿ-ಯಾದ
ಹಾಡು
ಹಾಡು-ಕು-ಣಿ-ತ-ನಾ-ಟ-ಕ-ಗ-ಳಲ್ಲೂ
ಹಾಡು-ಗಳನ್ನು
ಹಾಡು-ತ್ತಿ-ದ್ದಂತೆ
ಹಾಡು-ತ್ತಿ-ದ್ದರು
ಹಾಡು-ತ್ತಿ-ದ್ದರೆ
ಹಾಡುವ
ಹಾಡು-ವಾಗ
ಹಾಡೂ
ಹಾಡೊಂ-ದನ್ನು
ಹಾತೊ-ರೆ-ಯು-ತ್ತಾರೆ
ಹಾತೊ-ರೆ-ಯು-ತ್ತಿ-ದ್ದರು
ಹಾತೊ-ರೆ-ಯು-ತ್ತಿ-ದ್ದೇನೆ
ಹಾತೊ-ರೆ-ಯು-ತ್ತಿ-ರು-ತ್ತವೆ
ಹಾದಿ
ಹಾದಿ-ಯಲ್ಲಿ
ಹಾದಿಯೂ
ಹಾದು
ಹಾದು-ಹೋ-ಗ-ಬೇ-ಕಾಗಿ
ಹಾದು-ಹೋ-ಗ-ಲಾ-ರಂ-ಭಿ-ಸಿ-ದುವು
ಹಾದು-ಹೋ-ಗು-ವಂ-ಥ-ದಾ-ದ್ದ-ರಿಂದ
ಹಾದು-ಹೋ-ದುವು
ಹಾನಿ
ಹಾನಿ-ಯನ್ನು
ಹಾನಿ-ಯುಂಟು
ಹಾನಿ-ಯುಂ-ಟು-ಮಾ-ಡ-ಲಿಲ್ಲ
ಹಾನಿ-ಯುಂ-ಟು-ಮಾ-ಡಿ-ದ್ದರೂ
ಹಾಮ-ದಿಂದ
ಹಾಯಿ-ಸಿ-ದರೆ
ಹಾಯಿ-ಸಿ-ದಾ-ಗ-ಲೆಲ್ಲ
ಹಾರಾ-ಡಿ-ದರೂ
ಹಾರಾ-ಡು-ತ್ತಿದ್ದ
ಹಾರಾ-ಡು-ತ್ತಿ-ದ್ದ-ರೆಂದು
ಹಾರಾ-ಡು-ತ್ತಿ-ದ್ದಾರೆ
ಹಾರಾ-ಡು-ವುದನ್ನು
ಹಾರಿ
ಹಾರಿ-ಬಿದ್ದ
ಹಾರಿಸಿ
ಹಾರಿ-ಸಿ-ದ್ದರು
ಹಾರಿ-ಸಿ-ದ್ದೀರಿ
ಹಾರಿ-ಸಿ-ಬಿ-ಡು-ತ್ತಿ-ದ್ದರು
ಹಾರಿ-ಹೋ-ಗಲು
ಹಾರಿ-ಹೋ-ಗು-ತ್ತಿತ್ತು
ಹಾರಿ-ಹೋ-ಗು-ತ್ತೇವೆ
ಹಾರಿ-ಹೋ-ದವು
ಹಾರು-ವುದು
ಹಾರೈಕೆ
ಹಾರೈ-ಕೆ-ಗ-ಳೊಂ-ದಿಗೆ
ಹಾರೈಸಿ
ಹಾರೈ-ಸಿ-ದರು
ಹಾರೈ-ಸುತ್ತ
ಹಾರೈ-ಸು-ತ್ತೇನೆ
ಹಾರ್ಟ್
ಹಾರ್ದಿಕ
ಹಾರ್ದಿ-ಕ-ವಾಗಿ
ಹಾರ್ದಿ-ಕಾಗಿ
ಹಾರ್ಮೋ-ನಿಯಂ
ಹಾರ್ವ-ರ್ಡಿ-ನ-ಲ್ಲಾದ
ಹಾರ್ವ-ರ್ಡ್
ಹಾರ್ವರ್ಡ್ನ
ಹಾರ್ವಿಚ್
ಹಾಲನ್ನು
ಹಾಲಾ-ಹ-ಲ-ವೆಂಬ
ಹಾಲಿ-ಸ್ಟರ್
ಹಾಲಿ-ಸ್ಟ-ರ್-ಇ-ವ-ರಿ-ಬ್ಬರೂ
ಹಾಲಿ-ಸ್ಟ-ರ್ನನ್ನು
ಹಾಲಿ-ಸ್ಟ-ರ್ನಿಗೆ
ಹಾಲಿ-ಸ್ಟ-ರ್ನೊಂ-ದಿಗೆ
ಹಾಲು-ಹ-ಣ್ಣನ್ನು
ಹಾಲೆಂ-ಡಿನ
ಹಾಲೆಂ-ಡಿ-ನಿಂದ
ಹಾಲ್
ಹಾಲ್ನಲ್ಲಿ
ಹಾಳಾ-ಗು-ತ್ತದೆ
ಹಾಳಾ-ಗು-ತ್ತಿ-ತ್ತ-ಲ್ಲದೆ
ಹಾಳು-ಮಾ-ಡಿ-ಕೊ-ಳ್ಳು-ವು-ದಲ್ಲ
ಹಾವ-ಭಾ-ವ-ಗಳ
ಹಾವ-ಭಾ-ವ-ಗ-ಳು-ಪ್ರ-ತಿ-ಯೊಂದೂ
ಹಾವಳಿ
ಹಾವ-ಳಿ-ಇವು
ಹಾವಿಗೂ
ಹಾವಿ-ನಂತೆ
ಹಾವು
ಹಾಸ
ಹಾಸ-ದಿಂದ
ಹಾಸಿ
ಹಾಸಿಕ
ಹಾಸಿಗೆ
ಹಾಸಿ-ಗೆ-ಯ-ನ್ನಾ-ಗಿ-ಸಿ-ಕೊಂಡು
ಹಾಸಿ-ಗೆ-ಯಿಂದ
ಹಾಸಿ-ಗೆಯು
ಹಾಸಿದ
ಹಾಸು
ಹಾಸುಗೆ
ಹಾಸು-ಹೊ-ಕ್ಕಾ-ಗಿ-ರುವ
ಹಾಸು-ಹೊ-ಕ್ಕಾ-ಗಿವೆ
ಹಾಸ್ಯ
ಹಾಸ್ಯ-ನ-ಗೆ-ಚಾ-ಟಿ-ಕೆ-ಗಳಲ್ಲಿ
ಹಾಸ್ಯದ
ಹಾಸ್ಯ-ಪ್ರಜ್ಞೆ
ಹಾಸ್ಯ-ಪ್ರ-ವೃತ್ತಿ
ಹಾಸ್ಯ-ಪ್ರಿ-ಯರು
ಹಾಸ್ಯ-ಭ-ರಿ-ತವೂ
ಹಾಸ್ಯ-ರಸ
ಹಾಸ್ಯ-ರ-ಸ-ವನ್ನು
ಹಾಸ್ಯ-ವಾಗಿ
ಹಾಸ್ಯ-ವಾ-ಗಿಯೇ
ಹಾಸ್ಯ-ಶೀಲ
ಹಾಸ್ಯಾ-ಸ್ಪ-ದ-ವಾ-ದ-ದ್ದುಂಟೆ
ಹಾಸ್ಯಾ-ಸ್ಪ-ದವೇ
ಹಾಹಾ-ಕಾ-ರ-ವೆ-ದ್ದಿತು
ಹಿಂಗ-ಲಾ-ಜಿಗೆ
ಹಿಂಗ-ಲಾಜ್
ಹಿಂಗಿ-ಸ-ಬಲ್ಲ
ಹಿಂಗಿಸಿ
ಹಿಂಜ-ರಿ-ಕೆ-ಯಿನ್ನೂ
ಹಿಂಜ-ರಿ-ದರು
ಹಿಂಜ-ರಿ-ದರೂ
ಹಿಂಜ-ರಿ-ದಿದ್ದ
ಹಿಂಜ-ರಿ-ದು-ದ-ರಲ್ಲಿ
ಹಿಂಜ-ರಿ-ದು-ಬಿ-ಟ್ಟಳು
ಹಿಂಜ-ರಿ-ಯ-ಬೇ-ಕಾ-ಗಿಲ್ಲ
ಹಿಂಜ-ರಿ-ಯು-ತ್ತಲೇ
ಹಿಂಜ-ರಿ-ಯು-ತ್ತಿ-ದ್ದರು
ಹಿಂಜ-ರಿ-ಯು-ತ್ತಿ-ದ್ದೀ-ರಲ್ಲ
ಹಿಂಜ-ರಿ-ಯು-ತ್ತಿ-ರು-ವಂತೆ
ಹಿಂಡ-ಬ-ಲ್ಲಿ-ರಾ-ದರೆ
ಹಿಂಡ-ಬೇ-ಕೆಂ-ಬು-ದನ್ನು
ಹಿಂಡಿ
ಹಿಂಡಿ-ದಂ-ತಾ-ಯಿತು
ಹಿಂಡಿ-ಬಿ-ಡು-ತ್ತದೆ
ಹಿಂಡು-ಹಿಂ-ಡಾಗಿ
ಹಿಂದಕ್ಕೆ
ಹಿಂದಿ
ಹಿಂದಿ-ಗಿಂ-ತಲೂ
ಹಿಂದಿದ್ದ
ಹಿಂದಿ-ದ್ದಾನೆ
ಹಿಂದಿನ
ಹಿಂದಿ-ನಂತೆ
ಹಿಂದಿ-ನಂ-ತೆ-ಯೇ-ಶೋ-ಚ-ನೀ-ಯ-ವಾ-ಗಿ-ಯೇ-ಇತ್ತು
ಹಿಂದಿ-ನ-ಕಾಲ
ಹಿಂದಿ-ನ-ದೆ-ಲ್ಲ-ವನ್ನೂ
ಹಿಂದಿ-ನಿಂದ
ಹಿಂದಿ-ನಿಂ-ದಲೂ
ಹಿಂದಿ-ನಿಂ-ದಲೇ
ಹಿಂದಿ-ಯಲ್ಲಿ
ಹಿಂದಿ-ರು-ಗ-ದಿ-ದ್ದಾಗ
ಹಿಂದಿ-ರು-ಗದೆ
ಹಿಂದಿ-ರು-ಗ-ಬೇ-ಕಾ-ಗಿ-ಬಂತು
ಹಿಂದಿ-ರು-ಗ-ಬೇ-ಕಾ-ಯಿತು
ಹಿಂದಿ-ರು-ಗ-ಬೇ-ಕೆಂ-ದರೆ
ಹಿಂದಿ-ರು-ಗ-ಬೇ-ಕೆಂದು
ಹಿಂದಿ-ರು-ಗ-ಲಾರ
ಹಿಂದಿ-ರು-ಗ-ಲಾರೆ
ಹಿಂದಿ-ರು-ಗ-ಲಿ-ದ್ದಾ-ರೆಂಬ
ಹಿಂದಿ-ರು-ಗಲು
ಹಿಂದಿ-ರುಗಿ
ಹಿಂದಿ-ರು-ಗಿದ
ಹಿಂದಿ-ರು-ಗಿ-ದರು
ಹಿಂದಿ-ರು-ಗಿ-ದರೆ
ಹಿಂದಿ-ರು-ಗಿ-ದ-ವ-ರು-ಪೌ-ರ್ವಾ-ತ್ಯ-ವಾ-ದ-ದ್ದ-ನ್ನೆಲ್ಲ
ಹಿಂದಿ-ರು-ಗಿ-ದಾಗ
ಹಿಂದಿ-ರು-ಗಿದೆ
ಹಿಂದಿ-ರು-ಗಿ-ದೆವು
ಹಿಂದಿ-ರು-ಗಿದ್ದ
ಹಿಂದಿ-ರು-ಗಿ-ದ್ದರು
ಹಿಂದಿ-ರು-ಗಿ-ದ್ದೇನೆ
ಹಿಂದಿ-ರು-ಗಿಯೂ
ಹಿಂದಿ-ರು-ಗಿ-ರ-ಬೇಕು
ಹಿಂದಿ-ರು-ಗಿ-ಸು-ತ್ತಾನೆ
ಹಿಂದಿ-ರುಗು
ಹಿಂದಿ-ರು-ಗು-ತ್ತಾರೆ
ಹಿಂದಿ-ರು-ಗು-ತ್ತಿ-ದ್ದಂತೆ
ಹಿಂದಿ-ರು-ಗು-ತ್ತಿ-ದ್ದರು
ಹಿಂದಿ-ರು-ಗು-ತ್ತಿ-ದ್ದಳು
ಹಿಂದಿ-ರು-ಗು-ತ್ತಿದ್ದೆ
ಹಿಂದಿ-ರು-ಗು-ತ್ತಿ-ರ-ಲಿಲ್ಲ
ಹಿಂದಿ-ರು-ಗು-ತ್ತೇನೆ
ಹಿಂದಿ-ರು-ಗುವ
ಹಿಂದಿ-ರು-ಗು-ವಂ-ತಾ-ದರೆ
ಹಿಂದಿ-ರು-ಗು-ವಂತೆ
ಹಿಂದಿ-ರು-ಗು-ವ-ವ-ರೆಗೂ
ಹಿಂದಿ-ರು-ಗು-ವಾಗ
ಹಿಂದಿ-ರು-ಗು-ವಾ-ಗಲೂ
ಹಿಂದಿ-ರು-ಗು-ವು-ದಿಲ್ಲ
ಹಿಂದಿ-ರು-ಗು-ವು-ದೆಂದು
ಹಿಂದಿ-ರು-ಗು-ವು-ದೆಂದೇ
ಹಿಂದಿ-ರುವ
ಹಿಂದಿ-ರು-ವುದನ್ನು
ಹಿಂದೀ
ಹಿಂದು
ಹಿಂದು-ಗಳ
ಹಿಂದು-ಗ-ಳಿಂ-ದಲೇ
ಹಿಂದು-ಗಳು
ಹಿಂದು-ಗ-ಳೆಂದು
ಹಿಂದು-ತ್ವೀ-ಕ-ರ-ಣದ
ಹಿಂದು-ತ್ವೀ-ಕ-ರಿ-ಸು-ವುದೂ
ಹಿಂದು-ಮುಂದು
ಹಿಂದು-ಳಿದ
ಹಿಂದು-ಳಿ-ದರು
ಹಿಂದು-ಳಿ-ದರೆ
ಹಿಂದು-ಳಿ-ದ-ವರ
ಹಿಂದು-ಳಿ-ದಿ-ದೆ-ಯಲ್ಲ
ಹಿಂದು-ಳಿ-ದಿ-ರು-ತ್ತೀರಿ
ಹಿಂದು-ಳಿ-ದಿ-ರು-ವಂ-ತೆಯೇ
ಹಿಂದು-ಳಿ-ದಿವೆ
ಹಿಂದು-ಳಿಯು
ಹಿಂದು-ವನ್ನೂ
ಹಿಂದು-ವಾಗಿ
ಹಿಂದು-ವಾ-ಗಿದ್ದ
ಹಿಂದು-ವಾ-ಗಿ-ದ್ದನು
ಹಿಂದು-ವಾ-ಗಿ-ರಲು
ಹಿಂದು-ವಾ-ದ-ವನು
ಹಿಂದು-ವಿ-ಗಿಂತ
ಹಿಂದು-ವಿಗೆ
ಹಿಂದು-ವಿ-ನಿಂದ
ಹಿಂದುವು
ಹಿಂದುವೂ
ಹಿಂದುವೇ
ಹಿಂದುವೋ
ಹಿಂದು-ಹಿಂ-ದಕ್ಕೆ
ಹಿಂದೂ
ಹಿಂದೂ-ಮು-ಸ-ಲ್ಮಾನ
ಹಿಂದೂ-ಮು-ಸ-ಲ್ಮಾ-ನರ
ಹಿಂದೂ-ಗಳ
ಹಿಂದೂ-ಗಳನ್ನು
ಹಿಂದೂ-ಗಳಲ್ಲಿ
ಹಿಂದೂ-ಗ-ಳಷ್ಟೇ
ಹಿಂದೂ-ಗ-ಳಾ-ಗಿ-ದ್ದರೂ
ಹಿಂದೂ-ಗಳಿಂದ
ಹಿಂದೂ-ಗ-ಳಿಗೂ
ಹಿಂದೂ-ಗ-ಳಿಗೆ
ಹಿಂದೂ-ಗ-ಳಿಗೇ
ಹಿಂದೂ-ಗಳು
ಹಿಂದೂ-ಗ-ಳೆಂ-ದರೆ
ಹಿಂದೂ-ತ-ತ್ತ್ವ-ಗಳು
ಹಿಂದೂ-ತನ್ನ
ಹಿಂದೂ-ದ-ರ್ಶ-ನ-ಗಳ
ಹಿಂದೂ-ಧರ್ಮ
ಹಿಂದೂ-ಧ-ರ್ಮ
ಹಿಂದೂ-ಧ-ರ್ಮಕ್ಕೆ
ಹಿಂದೂ-ಧ-ರ್ಮದ
ಹಿಂದೂ-ಧ-ರ್ಮ-ದಂತೆ
ಹಿಂದೂ-ಧ-ರ್ಮ-ದಲ್ಲಿ
ಹಿಂದೂ-ಧ-ರ್ಮ-ವನ್ನು
ಹಿಂದೂ-ಧ-ರ್ಮ-ವನ್ನೇ
ಹಿಂದೂ-ಧ-ರ್ಮ-ವಲ್ಲ
ಹಿಂದೂ-ಧ-ರ್ಮವು
ಹಿಂದೂ-ಧ-ರ್ಮವೇ
ಹಿಂದೂ-ಧ-ರ್ಮ-ವೊಂ-ದ-ನ್ನು-ಳಿದು
ಹಿಂದೂ-ಧ-ರ್ಮ-ವೊಂದೇ
ಹಿಂದೂ-ಧ-ರ್ಮೀ-ಯ-ನಿಗೆ
ಹಿಂದೂ-ನ-ನ್ನಂತೆ
ಹಿಂದೂ-ಪು-ನ-ರು-ತ್ಥಾ-ನದ
ಹಿಂದೂ-ಮ-ಹಾ-ಸಾ-ಗ-ರದ
ಹಿಂದೂ-ವಾ-ದರೂ
ಹಿಂದೂ-ವಿಗೆ
ಹಿಂದೂ-ಸಂ-ಸ್ಥೆಯ
ಹಿಂದೂ-ಸ್ತಾನಿ
ಹಿಂದೂ-ಸ್ಥಾ-ನ-ದಲ್ಲಿ
ಹಿಂದೆ
ಹಿಂದೆಂ-ದಿ-ಗಿಂತ
ಹಿಂದೆಂ-ದಿ-ಗಿಂ-ತಲೂ
ಹಿಂದೆಂದೂ
ಹಿಂದೆ-ಗೆ-ದು-ಕೊಂ-ಡರು
ಹಿಂದೆ-ಗೆ-ದು-ಕೊಳ್ಳ
ಹಿಂದೆ-ತಾನೆ
ಹಿಂದೆ-ಮುಂದೆ
ಹಿಂದೆ-ಯಷ್ಟೇ
ಹಿಂದೆಯೂ
ಹಿಂದೆಯೇ
ಹಿಂದೇಟು
ಹಿಂದೊಮ್ಮೆ
ಹಿಂದೋ-ಡು-ವು-ದೆಂ-ದರೆ
ಹಿಂಬಾ-ಲ-ಕ-ರಾ-ದರು
ಹಿಂಬಾ-ಲ-ಕರು
ಹಿಂಬಾ-ಲಿಸಿ
ಹಿಂಬಾ-ಲಿ-ಸಿ-ಕೊಂಡು
ಹಿಂಬಾ-ಲಿ-ಸಿ-ದ್ದರು
ಹಿಂಬಾ-ಲಿ-ಸಿ-ದ್ದೇ-ಕೆಂದು
ಹಿಂಬಾ-ಲಿ-ಸು-ತ್ತಿದ್ದ
ಹಿಂಬಾ-ಲಿ-ಸು-ತ್ತಿ-ದ್ದರು
ಹಿಂಬಾ-ಲಿ-ಸು-ತ್ತಿ-ದ್ದೀ-ಯಲ್ಲ
ಹಿಂಬಾ-ಲಿ-ಸು-ತ್ತಿ-ದ್ದೀಯೆ
ಹಿಂಬಾ-ಲಿ-ಸು-ತ್ತಿಲ್ಲ
ಹಿಂಬಾ-ಲಿ-ಸು-ದ್ದುವು
ಹಿಂಬಾ-ಲಿ-ಸುವ
ಹಿಂಬಾ-ಲಿ-ಸು-ವುದನ್ನು
ಹಿಂಸಾ-ದ್ವೇ-ಷ-ಗ-ಳಿಗೆ
ಹಿಂಸೆ-ಯನ್ನು
ಹಿಂಸೆ-ಯಿಂದ
ಹಿಗ್ಗಿತು
ಹಿಗ್ಗಿ-ದರು
ಹಿಗ್ಗಿ-ನ್ಸನ್
ಹಿಗ್ಗಿ-ನ್ಸ-ರಿ-ಗಂತೂ
ಹಿಗ್ಗಿ-ನ್ಸ್
ಹಿಗ್ಗಿ-ಸಿತು
ಹಿಗ್ಗು-ವಂತೆ
ಹಿಟ್ಟು
ಹಿಡಿ
ಹಿಡಿತ
ಹಿಡಿ-ತ-ದ-ಲ್ಲಿ-ಟ್ಟು-ಕೊಂ-ಡಿ-ದ್ದೇನೆ
ಹಿಡಿ-ತ-ದ-ಲ್ಲಿ-ಟ್ಟು-ಕೊಂಡು
ಹಿಡಿ-ತ-ದ-ಲ್ಲಿ-ಟ್ಟು-ಕೊಳ್ಳು
ಹಿಡಿ-ತ-ದಿಂದ
ಹಿಡಿ-ತ-ವ-ನ್ನಿ-ಟ್ಟು-ಕೊ-ಳ್ಳಲು
ಹಿಡಿ-ತ-ವನ್ನು
ಹಿಡಿ-ತ-ವಿ-ರ-ಬೇಕು
ಹಿಡಿ-ತವೂ
ಹಿಡಿದ
ಹಿಡಿ-ದರು
ಹಿಡಿ-ದರೆ
ಹಿಡಿ-ದಳು
ಹಿಡಿ-ದಾಗ
ಹಿಡಿ-ದಿಟ್ಟ
ಹಿಡಿ-ದಿ-ಟ್ಟಿಲ್ಲ
ಹಿಡಿ-ದಿ-ಟ್ಟು-ಕೊ-ಳ್ಳು-ತ್ತಿ-ದ್ದರು
ಹಿಡಿ-ದಿ-ಡ-ಲಾ-ರರು
ಹಿಡಿ-ದಿ-ಡಲು
ಹಿಡಿ-ದಿ-ಡು-ವುದೇ
ಹಿಡಿ-ದಿದ್ದ
ಹಿಡಿ-ದಿ-ದ್ದಾನೆ
ಹಿಡಿ-ದಿ-ದ್ದೇನೆ
ಹಿಡಿ-ದಿ-ರಲು
ಹಿಡಿ-ದಿ-ರುವ
ಹಿಡಿದು
ಹಿಡಿ-ದುಕೊ
ಹಿಡಿ-ದು-ಕೊಂ-ಡಿದೆ
ಹಿಡಿ-ದು-ಕೊಂ-ಡಿದ್ದ
ಹಿಡಿ-ದು-ಕೊಂ-ಡಿ-ರುವ
ಹಿಡಿ-ದು-ಕೊಂಡು
ಹಿಡಿ-ದು-ಬಿ-ಡು-ತ್ತದೆ
ಹಿಡಿ-ದು-ಹಾಕಿ
ಹಿಡಿ-ದು-ಹಾ-ಕಿ-ದರು
ಹಿಡಿ-ಯ-ಬೇ-ಕಾಗಿ
ಹಿಡಿ-ಯಲು
ಹಿಡಿ-ಯ-ಲ್ಪ-ಟ್ಟಿದ್ದ
ಹಿಡಿ-ಯಿ-ತಾ-ದರೂ
ಹಿಡಿ-ಯಿತು
ಹಿಡಿ-ಯಿರಿ
ಹಿಡಿ-ಯು-ತ್ತಿತ್ತು
ಹಿಡಿ-ಯು-ತ್ತಿ-ದ್ದರು
ಹಿಡಿ-ಯು-ವಂ-ತಾ-ಗಿ-ಬಿ-ಟ್ಟಿತು
ಹಿಡಿ-ಯು-ವ-ವಳು
ಹಿಡಿ-ಯು-ವು-ದ-ಕ್ಕಾಗಿ
ಹಿಡಿ-ಯು-ವುದು
ಹಿಡಿ-ಶಾಪ
ಹಿಡಿಸ
ಹಿಡಿ-ಸ-ಲಾ-ರ-ದಷ್ಟು
ಹಿಡಿ-ಸುವ
ಹಿತ-ಕ್ಕಾಗಿ
ಹಿತ-ಚಿಂ-ತ-ನೆ-ಯಲ್ಲಿ
ಹಿತ-ಚಿಂ-ತ-ನೆ-ಯಲ್ಲೇ
ಹಿತವೂ
ಹಿತ-ವೆ-ನೆ-ನಿ-ಸ-ದಿ-ರ-ಬ-ಹುದು
ಹಿತ-ಸಾ-ಧ-ನೆ-ಗಾಗಿ
ಹಿತ-ಸಾ-ಧ-ನೆಗೆ
ಹಿತಾ-ಕಾಂ-ಕ್ಷಿ-ಗಳ
ಹಿತಾ-ಸ-ಕ್ತಿ-ಗಳನ್ನು
ಹಿತೈ-ಷಿ-ಗಳು
ಹಿತ್ತಾ-ಳೆಯ
ಹಿನ್ನೆ-ಲೆ-ಗ-ಳಾ-ವುವೂ
ಹಿನ್ನೆ-ಲೆ-ಗ-ಳೆಲ್ಲ
ಹಿನ್ನೆ-ಲೆಗೆ
ಹಿನ್ನೆ-ಲೆಯ
ಹಿನ್ನೆ-ಲೆ-ಯನ್ನು
ಹಿನ್ನೆ-ಲೆ-ಯಲ್ಲಿ
ಹಿನ್ನೆ-ಲೆ-ಯಲ್ಲೂ
ಹಿನ್ನೆ-ಲೆ-ಯಲ್ಲೇ
ಹಿನ್ನೆ-ಲೆ-ಯಿಂ-ದಾಗಿ
ಹಿನ್ನೆ-ಲೆ-ಯಿದೆ
ಹಿನ್ನೆ-ಲೆ-ಯಿ-ರು-ತ್ತಿತ್ತು
ಹಿನ್ನೆ-ಲೆ-ಯಿ-ರುವ
ಹಿನ್ನೆ-ಲೆಯೂ
ಹಿಮ
ಹಿಮದ
ಹಿಮ-ಪಾ-ತ-ವಾ-ಯಿತು
ಹಿಮ-ಬಂ-ಡೆ-ಗಳಿಂದ
ಹಿಮ-ಮಯ
ಹಿಮ-ರಾ-ಶಿ-ಯಲ್ಲಿ
ಹಿಮಾ-ಚ್ಛಾ-ದಿತ
ಹಿಮಾ-ಮೃ-ತ-ವಾದ
ಹಿಮಾ-ಲಯ
ಹಿಮಾ-ಲ-ಯಕ್ಕೆ
ಹಿಮಾ-ಲ-ಯದ
ಹಿಮಾ-ಲ-ಯ-ದಲ್ಲಿ
ಹಿಮಾ-ಲ-ಯ-ದ-ಲ್ಲೊಂ-ದು-ಸ್ಥಾ-ಪಿ-ಸ-ಬೇ-ಕೆಂಬ
ಹಿಮಾ-ಲ-ಯ-ದ-ವರೆ-ಗಿನ
ಹಿಮಾ-ಲ-ಯ-ದಿಂದ
ಹಿಮಾ-ವೃತ
ಹಿಮ್ಮೆ-ಟ್ಟಿ-ದರು
ಹಿಯಾ-ಳಿ-ಸು-ತ್ತಿದ್ದ
ಹಿರಮ್
ಹಿರಿ
ಹಿರಿಮೆ
ಹಿರಿ-ಮೆ-ಮ-ಹಿ-ಮೆ-ಗಳನ್ನು
ಹಿರಿ-ಮೆಯ
ಹಿರಿ-ಮೆ-ಯನ್ನು
ಹಿರಿ-ಮೆಯೂ
ಹಿರಿ-ಮೆ-ಯೆಂ-ದರೆ
ಹಿರಿ-ಮೆ-ಯೇ-ನೆಂ-ಬುದು
ಹಿರಿ-ಮೆ-ಯೊಂದೇ
ಹಿರಿಯ
ಹಿರಿ-ಯ-ರನ್ನೇ
ಹಿರಿ-ಯರು
ಹಿಸುಕಿ
ಹೀಗ-ಳೆ-ಯು-ವು-ದರ
ಹೀಗಾಗಿ
ಹೀಗಾ-ದರೆ
ಹೀಗಿತ್ತು
ಹೀಗಿದೆ
ಹೀಗಿದ್ದ
ಹೀಗಿ-ದ್ದರೂ
ಹೀಗಿ-ದ್ದರೆ
ಹೀಗಿ-ದ್ದುವು
ಹೀಗಿದ್ದೂ
ಹೀಗಿ-ರು-ವ-ವನೇ
ಹೀಗಿ-ರು-ವಾಗ
ಹೀಗಿವೆ
ಹೀಗೆ
ಹೀಗೆಂದ
ಹೀಗೆಂ-ದಿತು
ಹೀಗೆಂದು
ಹೀಗೆ-ನ್ನುತ್ತ
ಹೀಗೆ-ನ್ನು-ತ್ತಿ-ದ್ದಂತೆ
ಹೀಗೆಯೆ
ಹೀಗೆಯೇ
ಹೀಗೆ-ಯೇ-ಎ-ಡ-ಗೈ-ಯನ್ನು
ಹೀಗೇ
ಹೀಗೇಕೆ
ಹೀನ
ಹೀನ-ಕಾ-ರ್ಯ-ವನ್ನು
ಹೀನ-ಕೃ-ತ್ಯ-ಗಳಲ್ಲಿ
ಹೀನ-ತೆಗೆ
ಹೀನ-ವಾದ
ಹೀನಾಯ
ಹೀನಾ-ಯ-ವಾಗಿ
ಹೀನಾ-ಯ-ವಾದ
ಹೀರ-ಬ-ಲ್ಲಿರಿ
ಹೀರ-ಬೇಕು
ಹೀರುತ್ತ
ಹುಕುಂ
ಹುಕ್ಕಾ
ಹುಕ್ಕಾ-ಸಿ-ಗಾ-ರು-ಗಳ
ಹುಕ್ಕಾ-ದಿಂದ
ಹುಕ್ಕಾ-ವನ್ನು
ಹುಕ್ಕಾ-ಸೇ-ದುತ್ತ
ಹುಚ್ಚ
ಹುಚ್ಚ-ನಂ-ತಾದ
ಹುಚ್ಚ-ನೆಂದೋ
ಹುಚ್ಚರ
ಹುಚ್ಚ-ರಾ-ಗೋಣ
ಹುಚ್ಚಾ-ಟ-ಗ-ಳಿಗೆ
ಹುಚ್ಚಿ-ಯಂ-ತಾ-ದಳು
ಹುಚ್ಚು
ಹುಚ್ಚು-ಗೂ-ಳಿ-ಯಂತೆ
ಹುಚ್ಚು-ತನ
ಹುಚ್ಚು-ತ-ನ-ಗಳಿಂದ
ಹುಚ್ಚು-ತ-ನ-ದಲ್ಲಿ
ಹುಚ್ಚು-ನಂ-ಬಿ-ಕೆ-ಯಿನ್ನೂ
ಹುಚ್ಚೆದ್ದು
ಹುಚ್ಚೇ
ಹುಟ್ಟ-ಡ-ಗಿ-ಸ-ಬೇ-ಕೆಂದು
ಹುಟ್ಟ-ದಿ-ರ-ಲಾ-ರದೆ
ಹುಟ್ಟಲೂ
ಹುಟ್ಟಿ
ಹುಟ್ಟಿ-ಕೊಂಡ
ಹುಟ್ಟಿ-ಕೊಂ-ಡಿತು
ಹುಟ್ಟಿ-ಕೊಂ-ಡಿ-ದ್ದುವು
ಹುಟ್ಟಿ-ಕೊಂ-ಡುವು
ಹುಟ್ಟಿ-ಕೊ-ಳ್ಳು-ತ್ತಿ-ದ್ದರು
ಹುಟ್ಟಿತು
ಹುಟ್ಟಿದ
ಹುಟ್ಟಿ-ದ-ವ-ನಲ್ಲ
ಹುಟ್ಟಿ-ದ-ವನು
ಹುಟ್ಟಿ-ದ-ವರು
ಹುಟ್ಟಿ-ದ-ಹಬ್ಬ
ಹುಟ್ಟಿ-ದೆ-ನೆಂದು
ಹುಟ್ಟಿ-ದ್ದೀರಿ
ಹುಟ್ಟಿದ್ದು
ಹುಟ್ಟಿ-ನಿಂದ
ಹುಟ್ಟಿ-ನಿಂ-ದಲೂ
ಹುಟ್ಟಿ-ಬೆ-ಳೆ-ದ-ವ-ರಾ-ದರೂ
ಹುಟ್ಟಿ-ರು-ವು-ದ-ರಿಂದ
ಹುಟ್ಟಿ-ಸ-ಬ-ಹುದು
ಹುಟ್ಟಿ-ಸ-ಲಾದ
ಹುಟ್ಟಿ-ಸಲು
ಹುಟ್ಟಿ-ಸಿತು
ಹುಟ್ಟಿ-ಸಿ-ದುವು
ಹುಟ್ಟಿ-ಸಿ-ಬಿ-ಟ್ಟಿದೆ
ಹುಟ್ಟಿ-ಸು-ತ್ತವೆ
ಹುಟ್ಟಿ-ಸು-ವ-ವರು
ಹುಟ್ಟು
ಹುಟ್ಟು-ಸಾ-ವು-ಗಳ
ಹುಟ್ಟು-ಗು-ಣ-ವಾದ
ಹುಟ್ಟುವ
ಹುಟ್ಟು-ವಾ-ಗಲೇ
ಹುಟ್ಟು-ಹಾ-ಕಿದ
ಹುಟ್ಟು-ಹಾ-ಕು-ತ್ತೀರಿ
ಹುಟ್ಟು-ಹಾ-ಕುವ
ಹುಡು-ಕದೆ
ಹುಡು-ಕ-ಬೇಕು
ಹುಡು-ಕಾಟ
ಹುಡುಕಿ
ಹುಡು-ಕಿ-ಕೊಂಡು
ಹುಡು-ಕಿ-ದ-ನೆಂದ
ಹುಡು-ಕಿ-ದರೂ
ಹುಡು-ಕು-ತ್ತಿ-ರು-ತ್ತದೆ
ಹುಡು-ಕು-ತ್ತಿ-ರು-ವು-ದಾಗಿ
ಹುಡು-ಕು-ವಾ-ಗಲೂ
ಹುಡುಗ
ಹುಡು-ಗನ
ಹುಡು-ಗ-ನನ್ನು
ಹುಡು-ಗರ
ಹುಡು-ಗರು
ಹುಡು-ಗರೆ
ಹುಡು-ಗರೇ
ಹುಡು-ಗಾ-ಟಿ-ಕೆಯ
ಹುಡು-ಗಾ-ಟಿ-ಕೆ-ಯಾ-ಡಲು
ಹುಡು-ಗಿಯ
ಹುಡು-ಗಿ-ಯರು
ಹುಣಿ-ಸೇ-ಹಣ್ಣು
ಹುಣ್ಣಾಗಿ
ಹುಣ್ಣಾ-ಗಿ-ಸುವ
ಹುಣ್ಣಾ-ಗು-ವಷ್ಟು
ಹುಣ್ಣು
ಹುದು-ಗಿ-ಕೊಂ-ಡುವು
ಹುದು-ಗಿ-ದ-ವರೆ-ದ್ದೆದ್ದು
ಹುದು-ಗಿ-ರುವ
ಹುದ್ದೆ-ಯ-ಲ್ಲಿದ್ದು
ಹುಮ್ಮ-ಸ್ಸಿ-ನಿಂದ
ಹುರಿ-ದುಂ-ಬಿಸಿ
ಹುರಿ-ದುಂ-ಬಿ-ಸಿದ
ಹುರಿ-ದುಂ-ಬಿ-ಸು-ತ್ತಿ-ದ್ದರು
ಹುಲಿ
ಹುಲಿ-ಗಳ
ಹುಲಿ-ಗಾ-ಹು-ತಿ-ಯಾ-ಗುವ
ಹುಲಿಯ
ಹುಲಿ-ಯಲ್ಲಿ
ಹುಲಿ-ರಾಯ
ಹುಲಿ-ರಾ-ಯನೇ
ಹುಲ್ಲಿಲ್ಲ
ಹುಲ್ಲು
ಹುಲ್ಲು-ಕ-ಡ್ಡಿ-ಯಿಂದ
ಹುಲ್ಲು-ತಿಂದು
ಹುಳ
ಹುಳು-ಕನ್ನು
ಹುಳು-ಹಿ-ಡಿದ
ಹೂಗಳನ್ನು
ಹೂಗ-ಳು-ಇ-ವು-ಗಳು
ಹೂಗ್ಲಿ-ಯಿಂದ
ಹೂಜಿ-ಯನ್ನೂ
ಹೂಡ-ತೊ-ಡ-ಗಿ-ದರು
ಹೂಡಿ
ಹೂಡಿ-ದರು
ಹೂಡಿದ್ದ
ಹೂಡಿ-ದ್ದರು
ಹೂಡಿದ್ದು
ಹೂಡುತ್ತ
ಹೂಡು-ತ್ತಿ-ದ್ದರು
ಹೂಡು-ವ-ವ-ರೆಗೂ
ಹೂಡು-ವು-ದಿಲ್ಲ
ಹೂಮಾ-ಲೆ-ಗಳ
ಹೂವು
ಹೂವು-ಗಳು
ಹೃತ್ಪೂರ್
ಹೃತ್ಪೂ-ರ್ವಕ
ಹೃತ್ಪೂ-ರ್ವ-ಕ-ತೆ-ಯಿತ್ತು
ಹೃತ್ಪೂ-ರ್ವ-ಕ-ವಾಗಿ
ಹೃದಯ
ಹೃದ-ಯ-ನಾ-ಲ-ಗೆ-ಗಳಿಂದ
ಹೃದ-ಯ-ಬು-ದ್ಧಿ-ಗಳ
ಹೃದ-ಯಂ-ಗ-ಮ-ವಾಗಿ
ಹೃದ-ಯ-ಇ-ವು-ಗಳನ್ನು
ಹೃದ-ಯಕ್ಕೂ
ಹೃದ-ಯ-ಗಳ
ಹೃದ-ಯ-ಗಳನ್ನು
ಹೃದ-ಯ-ಗಳಲ್ಲಿ
ಹೃದ-ಯ-ಗ-ಳಿಂ-ದಲೇ
ಹೃದ-ಯ-ಗಳೂ
ಹೃದ-ಯ-ತುಂಬಿ
ಹೃದ-ಯದ
ಹೃದ-ಯ-ದಲ್ಲಿ
ಹೃದ-ಯ-ದಲ್ಲೂ
ಹೃದ-ಯ-ದಾ-ಳದ
ಹೃದ-ಯ-ದಿಂದ
ಹೃದ-ಯ-ದುಂ-ಬಿ-ಬಂ-ದಿತು
ಹೃದ-ಯ-ದೊಂ-ದಿಗೆ
ಹೃದ-ಯ-ದೊ-ಳಕ್ಕೆ
ಹೃದ-ಯ-ರಾಗಿ
ಹೃದ-ಯ-ರಾದ
ಹೃದ-ಯ-ವಂ-ತಿ-ಕೆ-ಯನ್ನು
ಹೃದ-ಯ-ವಂ-ತಿ-ಕೆ-ಯಿ-ಲ್ಲ-ದಿ-ರು-ವುದು
ಹೃದ-ಯ-ವ-ನ್ನಿನ್ನೂ
ಹೃದ-ಯ-ವನ್ನು
ಹೃದ-ಯ-ವನ್ನೂ
ಹೃದ-ಯ-ವಿ-ದ್ರಾ-ವಕ
ಹೃದ-ಯ-ವಿ-ಲ್ಲದೆ
ಹೃದ-ಯವು
ಹೃದ-ಯವೂ
ಹೃದ-ಯವೇ
ಹೃದ-ಯ-ವೇನೋ
ಹೃದ-ಯ-ವೈ-ಶಾ-ಲ್ಯ-ವನ್ನು
ಹೃದ-ಯ-ಶುದ್ಧಿ
ಹೃದ-ಯ-ಶೂ-ನ್ಯ-ರಾದ
ಹೃದ-ಯ-ಸ್ಪರ್ಶಿ
ಹೃದ-ಯ-ಸ್ಪ-ರ್ಶಿ-ಯಾ-ಗಿದೆ
ಹೃದ-ಯ-ಸ್ಪ-ರ್ಶಿ-ಯಾ-ಗಿರು
ಹೃದ-ಯ-ಸ್ಪ-ರ್ಶಿ-ಯಾದ
ಹೃದ-ಯಾಂತ
ಹೃದ-ಯಾಂ-ತ-ರಾಳ
ಹೃದ-ಯಾಂ-ತ-ರಾ-ಳ-ದಿಂದ
ಹೃದ-ಯಾಂ-ತ-ರಾ-ಳ-ವನ್ನು
ಹೃದ-ಯಾ-ಘಾ-ತ-ದಿಂದ
ಹೃದ-ಯಾ-ರ್ಥವ
ಹೃದ-ಯಿ-ಗ-ಳಾದ
ಹೃದ್ಗ-ತ-ಗೊ-ಳಿ-ಸಿ-ಕೊಂ-ಡರು
ಹೃದ್ಗ-ತ-ವಾಗಿ
ಹೃನ್ಮ-ನ-ಗಳನ್ನು
ಹೃನ್ಮ-ನ-ಗ-ಳಿಗೆ
ಹೃಷೀ-ಕೇ-ಶ-ದಲ್ಲಿ
ಹೆಂಗ-ಸರ
ಹೆಂಗ-ಸ-ರಿ-ಗೆ-ಗಂ-ಡ-ಸ-ರಿ-ಗೆ-ಮ-ಕ್ಕ-ಳಿ-ಗೆ-ಎ-ಲ್ಲ-ರಿಗೂ
ಹೆಂಗ-ಸರು
ಹೆಂಗ-ಸರೂ
ಹೆಂಗ-ಸಲ್ಲ
ಹೆಂಗ-ಸಿಗೆ
ಹೆಂಗ-ಸಿನ
ಹೆಂಗಸು
ಹೆಂಗಸೇ
ಹೆಂಡತಿ
ಹೆಂಡ-ತಿ-ಮ-ಕ್ಕ-ಳನ್ನು
ಹೆಂಡ-ತಿ-ಯನ್ನು
ಹೆಂಡ-ತಿಯೇ
ಹೆಂಡಿರ
ಹೆಗ-ಲ-ಮೇ-ಲಿದ್ದ
ಹೆಗಲು
ಹೆಗ್ಗ-ಳಿ-ಕೆ-ಗೋ-ಸ್ಕ-ರ-ವಲ್ಲ
ಹೆಚ್
ಹೆಚ್ಚಾ-ಗಲಿ
ಹೆಚ್ಚಾಗಿ
ಹೆಚ್ಚಾ-ಗಿತ್ತು
ಹೆಚ್ಚಾ-ಗಿ-ತ್ತು-ಅ-ಥವಾ
ಹೆಚ್ಚಾ-ಗಿ-ಬಿ-ಟ್ಟಿ-ತ್ತೆಂ-ದರೆ
ಹೆಚ್ಚಾ-ಗಿ-ರುವ
ಹೆಚ್ಚಾ-ಗುತ್ತ
ಹೆಚ್ಚಾ-ಗು-ತ್ತದೆ
ಹೆಚ್ಚಾ-ಗು-ತ್ತಿತ್ತು
ಹೆಚ್ಚಾ-ಗು-ತ್ತಿದೆ
ಹೆಚ್ಚಾ-ಗು-ತ್ತಿ-ದ್ದು-ವಲ್ಲಾ
ಹೆಚ್ಚಾ-ದಂ-ತೆಲ್ಲ
ಹೆಚ್ಚಾ-ದಾಗ
ಹೆಚ್ಚಾ-ಯಿತು
ಹೆಚ್ಚಾ-ಯಿ-ತೆಂ-ದರೆ
ಹೆಚ್ಚಾ-ಯಿ-ತೆಂದು
ಹೆಚ್ಚಿ
ಹೆಚ್ಚಿ-ಗೇನೂ
ಹೆಚ್ಚಿತು
ಹೆಚ್ಚಿನ
ಹೆಚ್ಚಿ-ನಂ-ಶ-ವನ್ನು
ಹೆಚ್ಚಿ-ನ-ದನ್ನು
ಹೆಚ್ಚಿ-ನದು
ಹೆಚ್ಚಿ-ನದೇ
ಹೆಚ್ಚಿ-ನ-ದೇ-ನನ್ನೂ
ಹೆಚ್ಚಿ-ನ-ವನ್ನು
ಹೆಚ್ಚಿ-ನ-ವ-ರಿಗೆ
ಹೆಚ್ಚಿ-ನ-ವರು
ಹೆಚ್ಚಿ-ಸ-ಬೇ-ಕೆಂದು
ಹೆಚ್ಚಿಸಿ
ಹೆಚ್ಚಿ-ಸಿ-ಕೊಂ-ಡರು
ಹೆಚ್ಚಿ-ಸಿ-ಕೊ-ಳ್ಳ-ಬ-ಹುದು
ಹೆಚ್ಚಿ-ಸಿ-ಕೊ-ಳ್ಳು-ವು-ದಕ್ಕಾ
ಹೆಚ್ಚಿ-ಸಿತು
ಹೆಚ್ಚು
ಹೆಚ್ಚು-ಕ-ಡ-ಮೆ-ಯಾ-ಗು-ವು-ದಿಲ್ಲ
ಹೆಚ್ಚು-ಕ-ಡಿಮೆ
ಹೆಚ್ಚು-ಕ-ಡಿ-ಮೆ-ಯಾ-ಗಿ-ದೆ-ಯೆಂದು
ಹೆಚ್ಚು-ಕಾಲ
ಹೆಚ್ಚುತ್ತ
ಹೆಚ್ಚು-ತ್ತ-ಹೋ-ಯಿತು
ಹೆಚ್ಚು-ತ್ತಿತ್ತು
ಹೆಚ್ಚು-ತ್ತಿ-ರುವ
ಹೆಚ್ಚು-ಹೆ-ಚ್ಚಾಗಿ
ಹೆಚ್ಚು-ಹೆಚ್ಚು
ಹೆಚ್ಚೆಂ-ದರೆ
ಹೆಚ್ಚೆಂದು
ಹೆಚ್ಚೆಚ್ಚು
ಹೆಚ್ಚೇ
ಹೆಚ್ಚೇಕೆ
ಹೆಚ್ಚೇ-ನನ್ನೂ
ಹೆಚ್ಚೇನೂ
ಹೆಜ್ಜೆ
ಹೆಜ್ಜೆ-ಗಳು
ಹೆಜ್ಜೆಗೂ
ಹೆಜ್ಜೆಯ
ಹೆಜ್ಜೆ-ಯ-ನ್ನಿ-ಡು-ವುದೇ
ಹೆಜ್ಜೆ-ಯನ್ನೂ
ಹೆಜ್ಜೆ-ಯಷ್ಟೇ
ಹೆಜ್ಜೆ-ಯಿ-ಟ್ಟಂತೆ
ಹೆಜ್ಜೆ-ಯಿ-ಟ್ಟರೂ
ಹೆಜ್ಜೆ-ಹೆ-ಜ್ಜೆಗೂ
ಹೆಡ್ಡ-ತ-ನದ
ಹೆಣೆ-ದು-ಕೊಂ-ಡಿ-ರುವ
ಹೆಣ್ಣು-ಗಂ-ಡೆಂಬ
ಹೆಣ್ಣು-ಮ-ಕ್ಕಳ
ಹೆಣ್ಣು-ಮ-ಕ್ಕ-ಳಿಗೆ
ಹೆಣ್ಣು-ಮ-ಕ್ಕ-ಳಿ-ದ್ದ-ರಾ-ದರೂ
ಹೆಣ್ಣು-ಮ-ಕ್ಕಳು
ಹೆಣ್ಣು-ಮ-ಕ್ಕ-ಳು-ಇ-ವರೆಲ್ಲ
ಹೆತ್ತ
ಹೆದರ
ಹೆದ-ರ-ದಿರಿ
ಹೆದ-ರ-ಬೇ-ಕಾದ
ಹೆದ-ರ-ಬೇ-ಕಿಲ್ಲ
ಹೆದ-ರ-ಬೇಕು
ಹೆದ-ರ-ಬೇಡ
ಹೆದ-ರ-ಬೇ-ಡಯ್ಯ
ಹೆದ-ರ-ಬೇಡಿ
ಹೆದ-ರಲೇ
ಹೆದರಿ
ಹೆದ-ರಿ-ಕೆ-ಭಯ
ಹೆದ-ರಿ-ಕೆ-ಯಿಲ್ಲ
ಹೆದ-ರಿ-ಕೊ-ಳ್ಳ-ಬೇ-ಕಾಗಿ
ಹೆದ-ರಿ-ದರು
ಹೆದ-ರಿ-ದ್ದರು
ಹೆದ-ರಿದ್ದೆ
ಹೆದ-ರಿಸಿ
ಹೆದ-ರಿ-ಸಿ-ಬಿಟ್ಟೆ
ಹೆದ-ರು-ತ್ತಾ-ನೆಂದು
ಹೆದ-ರು-ತ್ತಿ-ದ್ದರು
ಹೆದ-ರು-ವ-ವ-ನಲ್ಲ
ಹೆದ-ರು-ವು-ದಕ್ಕೆ
ಹೆದ-ರು-ವು-ದಿಲ್ಲ
ಹೆನ್ರಿ
ಹೆನ್ರಿಟಾ
ಹೆನ್ರಿ-ಬ-ರೋ-ಸ್ರ-ವರ
ಹೆಪ್ಪು-ಗ-ಟ್ಟಿದ
ಹೆಬ್ಬುಲಿ
ಹೆಮ್ಮ-ರ-ವಾಗಿ
ಹೆಮ್ಮೆ
ಹೆಮ್ಮೆ-ಗಾಗಿ
ಹೆಮ್ಮೆ-ಪ-ಟ್ಟು-ಕೊ-ಳ್ಳು-ವಂ-ತಹ
ಹೆಮ್ಮೆಯ
ಹೆಮ್ಮೆ-ಯನ್ನು
ಹೆಮ್ಮೆ-ಯಾ-ಗಿ-ಬಿ-ಟ್ಟಿದೆ
ಹೆಮ್ಮೆ-ಯಿಂದ
ಹೆರಾ-ಲ್ಡ್
ಹೆಲೆನ್
ಹೆಳ-ವನು
ಹೆಸ-ರನ್ನು
ಹೆಸ-ರನ್ನೂ
ಹೆಸ-ರನ್ನೇ
ಹೆಸ-ರಾಂತ
ಹೆಸ-ರಾ-ಗಲಿ
ಹೆಸ-ರಾಗಿ
ಹೆಸ-ರಾ-ಗಿ-ದ್ದ-ವನು
ಹೆಸ-ರಾ-ಗಿ-ದ್ದ-ವರು
ಹೆಸ-ರಾದ
ಹೆಸ-ರಾ-ಯಿತು
ಹೆಸ-ರಿ-ಗಾಗಿ
ಹೆಸ-ರಿಗೆ
ಹೆಸ-ರಿಟ್ಟಿ
ಹೆಸ-ರಿನ
ಹೆಸ-ರಿ-ನಲ್ಲಿ
ಹೆಸ-ರಿ-ನಲ್ಲೇ
ಹೆಸ-ರಿ-ನಿಂದ
ಹೆಸ-ರಿ-ನಿಂ-ದಲೇ
ಹೆಸ-ರಿ-ಸ-ಲಾದ
ಹೆಸರು
ಹೆಸ-ರು
ಹೆಸ-ರು-ಕೀರ್ತಿ
ಹೆಸ-ರು-ಕೀ-ರ್ತಿ-ಗಳ
ಹೆಸ-ರು-ಕೀ-ರ್ತಿ-ಗಳಿಂದ
ಹೆಸ-ರು-ಕೀ-ರ್ತಿಯ
ಹೆಸ-ರು-ಕೀ-ರ್ತಿ-ಯೊಂದೂ
ಹೆಸ-ರು-ಗಳನ್ನು
ಹೆಸ-ರು-ಗ-ಳಷ್ಟೇ
ಹೆಸ-ರು-ಗಳಿಂದ
ಹೆಸ-ರು-ಗ-ಳಿಗೆ
ಹೆಸ-ರು-ಗ-ಳಿ-ಸಿ-ದ್ದರು
ಹೆಸ-ರು-ಗಳು
ಹೆಸ-ರು-ಗ-ಳೆಲ್ಲ
ಹೆಸ-ರು-ವಾ-ಸಿ-ಯಾ-ಗಿದ್ದ
ಹೆಸ-ರು-ವಾ-ಸಿ-ಯಾ-ಗಿ-ದ್ದ-ವರು
ಹೆಸ-ರು-ವಾ-ಸಿ-ಯಾ-ದ-ವರು
ಹೆಸರೂ
ಹೆಸ-ರೆ-ತ್ತಿ-ದರೆ
ಹೆಸರೇ
ಹೇ
ಹೇಗದು
ಹೇಗಾ-ಗಿ-ಬಿ-ಟ್ಟಿ-ದೆ-ಯೆಂ-ದರೆ
ಹೇಗಾ-ದರಾ
ಹೇಗಾ-ದರೂ
ಹೇಗಾ-ದೀತು
ಹೇಗಿತ್ತು
ಹೇಗಿ-ತ್ತೆಂ-ದರೆ
ಹೇಗಿತ್ತೋ
ಹೇಗಿದೆ
ಹೇಗಿ-ದೆಯೋ
ಹೇಗಿ-ದ್ದಿ-ರ-ಬ-ಹುದು
ಹೇಗಿ-ದ್ದೇ-ನೆಂ-ಬು-ದನ್ನು
ಹೇಗಿ-ರ-ಬ-ಹುದು
ಹೇಗಿ-ರ-ಬ-ಹು-ದೆಂ-ಬು-ದರ
ಹೇಗಿ-ರ-ಬ-ಹುದೋ
ಹೇಗಿ-ರು-ತ್ತದೆ
ಹೇಗಿ-ರು-ತ್ತಿ-ತ್ತೆಂ-ಬು-ದನ್ನು
ಹೇಗಿ-ರು-ವೆನೋ
ಹೇಗೆ
ಹೇಗೆಂ-ದರೆ
ಹೇಗೆಂದು
ಹೇಗೆಂಬು
ಹೇಗೆಂ-ಬು-ದನ್ನು
ಹೇಗೆಂ-ಬು-ದರ
ಹೇಗೆಂ-ಬುದು
ಹೇಗೇ
ಹೇಗೋ
ಹೇಡಿ-ಗ-ಳೊಂ-ದಿಗೆ
ಹೇಡಿ-ತ-ನ-ವನ್ನು
ಹೇಯ
ಹೇರಿ-ಕೊ-ಳ್ಳು-ವಂ-ತಾ-ಗ-ಬಾ-ರದು
ಹೇರಿತ್ತು
ಹೇರಿದ್ದು
ಹೇರು
ಹೇರು-ವುದನ್ನು
ಹೇಲ್
ಹೇಲ್ಗೆ
ಹೇಲ್ಮೇ-ರಿಯ
ಹೇಲ್ರ
ಹೇಲ್ರ-ವ-ರಿಗೆ
ಹೇಲ್ರ-ವರು
ಹೇಲ್ರಿಗೆ
ಹೇಲ್ರೊಂ-ದಿಗೆ
ಹೇಲ್ಳ
ಹೇಲ್ಳಿಗೆ
ಹೇಲ್ಳೊಂ-ದಿಗೆ
ಹೇಳ
ಹೇಳ-ತೀ-ರದು
ಹೇಳ-ತೊ-ಡ-ಗಿದ
ಹೇಳ-ದಿ-ದ್ದರೂ
ಹೇಳ-ದಿ-ದ್ದು-ದಕ್ಕೆ
ಹೇಳದೆ
ಹೇಳ-ಬ-ಲ್ಲ-ನಂತೆ
ಹೇಳ-ಬ-ಲ್ಲ-ವ-ರಾ-ಗಿ-ದ್ದರು
ಹೇಳ-ಬ-ಲ್ಲಿರಾ
ಹೇಳ-ಬಲ್ಲೆ
ಹೇಳ-ಬ-ಹು-ದಷ್ಟೆ
ಹೇಳ-ಬ-ಹುದು
ಹೇಳ-ಬ-ಹು-ದೇನೋ
ಹೇಳ-ಬಾ-ರದು
ಹೇಳ-ಬೇಕಾ
ಹೇಳ-ಬೇ-ಕಾಗಿ
ಹೇಳ-ಬೇ-ಕಾ-ಗಿತ್ತು
ಹೇಳ-ಬೇ-ಕಾ-ಗಿದೆ
ಹೇಳ-ಬೇ-ಕಾ-ಗಿಯೇ
ಹೇಳ-ಬೇ-ಕಾ-ಗಿಲ್ಲ
ಹೇಳ-ಬೇ-ಕಾ-ದದ್ದು
ಹೇಳ-ಬೇ-ಕಾ-ದು-ದ-ನ್ನೆಲ್ಲ
ಹೇಳ-ಬೇ-ಕಾ-ಯಿತು
ಹೇಳ-ಬೇಕು
ಹೇಳ-ಬೇ-ಕೆಂ-ದರೆ
ಹೇಳ-ಬೇ-ಕೆಂ-ದಿದ್ದ
ಹೇಳ-ಬೇ-ಕೆಂ-ದಿ-ದ್ದು-ದನ್ನು
ಹೇಳ-ಬೇ-ಕೆಂದು
ಹೇಳ-ಬೇ-ಕೆಂ-ದು-ಕೊಂ-ಡಿ-ದ್ದೆಲ್ಲ
ಹೇಳ-ಬೇ-ಕೆನು
ಹೇಳ-ಲಾ-ಗಿ-ರುವ
ಹೇಳ-ಲಾ-ಗಿ-ರು-ವಂತೆ
ಹೇಳ-ಲಾ-ಗು-ತ್ತದೆ
ಹೇಳ-ಲಾ-ಗುವ
ಹೇಳ-ಲಾದ
ಹೇಳ-ಲಾ-ರಂ-ಭಿ-ಸಿ-ದಾಗ
ಹೇಳ-ಲಾ-ರದ
ಹೇಳ-ಲಾ-ರ-ದ-ವ-ರಾಗಿ
ಹೇಳ-ಲಾರೆ
ಹೇಳಲಿ
ಹೇಳ-ಲಿಲ್ಲ
ಹೇಳ-ಲಿ-ಲ್ಲ-ವೆ-ಹ-ಸಿದ
ಹೇಳಲು
ಹೇಳಲೂ
ಹೇಳಲೆ
ಹೇಳ-ಲೇ-ಬೇಕು
ಹೇಳ-ಲ್ಪ-ಟ್ಟಿ-ರುವ
ಹೇಳಿ
ಹೇಳಿ-ಎಲೈ
ಹೇಳಿ-ಕ-ಳಿ-ಸಿ-ದ್ದೇನೆ
ಹೇಳಿ-ಕೆ-ಗಳ
ಹೇಳಿ-ಕೆ-ಗಳು
ಹೇಳಿ-ಕೆ-ಗಳೇ
ಹೇಳಿ-ಕೆಗೆ
ಹೇಳಿ-ಕೆ-ಯನ್ನು
ಹೇಳಿ-ಕೊಂಡ
ಹೇಳಿ-ಕೊಂ-ಡಂತೆ
ಹೇಳಿ-ಕೊಂ-ಡರು
ಹೇಳಿ-ಕೊಂ-ಡ-ರು-ಹೌದು
ಹೇಳಿ-ಕೊಂ-ಡರೂ
ಹೇಳಿ-ಕೊಂ-ಡಳು
ಹೇಳಿ-ಕೊಂ-ಡ-ವ-ನೊಂ-ದಿಗೆ
ಹೇಳಿ-ಕೊಂ-ಡಾಗ
ಹೇಳಿ-ಕೊಂ-ಡಿ-ದ್ದರೂ
ಹೇಳಿ-ಕೊಂ-ಡಿ-ದ್ದರೆ
ಹೇಳಿ-ಕೊಂಡು
ಹೇಳಿ-ಕೊಂಡೆ
ಹೇಳಿ-ಕೊ-ಟ್ಟಿ-ದ್ದ-ಮೇರೆ
ಹೇಳಿ-ಕೊ-ಟ್ಟಿ-ರು-ವಿರೋ
ಹೇಳಿ-ಕೊ-ಡ-ಬೇಕೆ
ಹೇಳಿ-ಕೊ-ಡಲು
ಹೇಳಿ-ಕೊ-ಳ್ಳ-ಬ-ಹು-ದಾ-ಗಿ-ತ್ತಲ್ಲ
ಹೇಳಿ-ಕೊ-ಳ್ಳ-ಲಾ-ರದೆ
ಹೇಳಿ-ಕೊ-ಳ್ಳಲು
ಹೇಳಿ-ಕೊ-ಳ್ಳಲೂ
ಹೇಳಿ-ಕೊ-ಳ್ಳುತ್ತ
ಹೇಳಿ-ಕೊ-ಳ್ಳು-ತ್ತಿದ್ದೆ
ಹೇಳಿ-ಕೊ-ಳ್ಳು-ತ್ತೀರಿ
ಹೇಳಿ-ಕೊ-ಳ್ಳು-ತ್ತೀರೋ
ಹೇಳಿ-ಕೊ-ಳ್ಳುವ
ಹೇಳಿ-ಕೊ-ಳ್ಳು-ವಂ-ತ-ಹದು
ಹೇಳಿ-ಕೊ-ಳ್ಳು-ವಂತೆ
ಹೇಳಿ-ಕೊ-ಳ್ಳು-ವುದು
ಹೇಳಿತು
ಹೇಳಿದ
ಹೇಳಿ-ದಂ-ತಹ
ಹೇಳಿ-ದಂ-ತಾ-ಯಿ-ತ-ಲ್ಲವೆ
ಹೇಳಿ-ದಂ-ತಾ-ಯಿತು
ಹೇಳಿ-ದಂತೆ
ಹೇಳಿ-ದ-ನೆಂ-ದರೆ
ಹೇಳಿ-ದ-ರಾ-ದರೂ
ಹೇಳಿ-ದರು
ಹೇಳಿ-ದ-ರು-ಅ-ತ್ಯು-ನ್ನತ
ಹೇಳಿ-ದ-ರು-ಒಮ್ಮೆ
ಹೇಳಿ-ದ-ರು-ನಿ-ಮಗೆ
ಹೇಳಿ-ದ-ರು-ನೀನು
ಹೇಳಿ-ದ-ರು-ನೋ-ಡಿಲ್ಲಿ
ಹೇಳಿ-ದ-ರು-ಮಾ-ನ-ವ-ನೊ-ಳ-ಗಿ-ರುವ
ಹೇಳಿ-ದ-ರು-ಯೋಗಿ
ಹೇಳಿ-ದರೂ
ಹೇಳಿ-ದರೆ
ಹೇಳಿ-ದರೇ
ಹೇಳಿ-ದಳು
ಹೇಳಿ-ದ-ವು-ಗ-ಳಿ-ಗಿಂತ
ಹೇಳಿ-ದ-ಸ್ವಾ-ಮೀಜಿ
ಹೇಳಿ-ದಾಗ
ಹೇಳಿ-ದಿರಿ
ಹೇಳಿ-ದು-ದ-ಕ್ಕಿಂತ
ಹೇಳಿ-ದು-ದನ್ನು
ಹೇಳಿ-ದು-ದಾಗಿ
ಹೇಳಿ-ದುದೂ
ಹೇಳಿ-ದು-ದೆಲ್ಲ
ಹೇಳಿದೆ
ಹೇಳಿ-ದೆ-ಎಂದು
ಹೇಳಿ-ದೆ-ನೀನು
ಹೇಳಿದ್ದ
ಹೇಳಿ-ದ್ದಂತೆ
ಹೇಳಿ-ದ್ದ-ಎಲ್ಲಿ
ಹೇಳಿ-ದ್ದ-ನಾ-ದರೂ
ಹೇಳಿ-ದ್ದನ್ನು
ಹೇಳಿ-ದ್ದನ್ನೇ
ಹೇಳಿ-ದ್ದ-ರಂತೆ
ಹೇಳಿ-ದ್ದ-ರಿಂದ
ಹೇಳಿ-ದ್ದರು
ಹೇಳಿ-ದ್ದ-ರು-ನಾ-ವೇ-ನಾ-ದರೂ
ಹೇಳಿ-ದ್ದರೂ
ಹೇಳಿ-ದ್ದರೆ
ಹೇಳಿ-ದ್ದ-ರೆಂ-ಬು-ದನ್ನು
ಹೇಳಿ-ದ್ದ-ಲ್ಲ-ವೇನೋ
ಹೇಳಿ-ದ್ದಾನೆ
ಹೇಳಿ-ದ್ದಾರೆ
ಹೇಳಿ-ದ್ದೀರಾ
ಹೇಳಿದ್ದು
ಹೇಳಿ-ದ್ದುಂಟು
ಹೇಳಿ-ದ್ದು-ದ-ರಲ್ಲಿ
ಹೇಳಿ-ದ್ದುದು
ಹೇಳಿದ್ದೆ
ಹೇಳಿ-ದ್ದೆಲ್ಲ
ಹೇಳಿದ್ದೇ
ಹೇಳಿ-ದ್ದೊಂದೂ
ಹೇಳಿ-ಬಿಟ್ಟ
ಹೇಳಿ-ಬಿ-ಟ್ಟರು
ಹೇಳಿ-ಬಿ-ಟ್ಟ-ರು-ನೋಡಿ
ಹೇಳಿ-ಬಿ-ಟ್ಟರೆ
ಹೇಳಿ-ಬಿ-ಟ್ಟಾಗ
ಹೇಳಿ-ಬಿ-ಟ್ಟಿ-ದ್ದರು
ಹೇಳಿ-ಬಿ-ಟ್ಟಿ-ದ್ದಾರೆ
ಹೇಳಿ-ಬಿಟ್ಟೆ
ಹೇಳಿ-ಬಿ-ಟ್ಟೆ-ಯಲ್ಲ
ಹೇಳಿ-ಬಿಡು
ಹೇಳಿ-ಬಿ-ಡು-ತ್ತಿ-ದ್ದರು
ಹೇಳಿ-ಯಾನು
ಹೇಳಿ-ಯಾರು
ಹೇಳಿಯೂ
ಹೇಳಿ-ಯೇ-ಬಿಟ್ಟ
ಹೇಳಿ-ರ-ಲಿಲ್ಲ
ಹೇಳಿ-ರು-ತ್ತಿದ್ದೆ
ಹೇಳಿ-ರು-ವಂತೆ
ಹೇಳಿ-ರು-ವೆ-ನಾ-ದರೆ
ಹೇಳಿಲ್ಲ
ಹೇಳಿ-ಸಿ-ಕೊಳ್ಳಿ
ಹೇಳಿ-ಸಿದ್ದು
ಹೇಳು
ಹೇಳುತ್ತ
ಹೇಳು-ತ್ತದೆ
ಹೇಳು-ತ್ತ-ದೆಂ-ಬುದು
ಹೇಳು-ತ್ತವೆ
ಹೇಳು-ತ್ತ-ವೆ-ಯೆಂದು
ಹೇಳು-ತ್ತಾನೆ
ಹೇಳು-ತ್ತಾರೆ
ಹೇಳು-ತ್ತಾ-ರೆ-ಇ-ವೆಲ್ಲ
ಹೇಳು-ತ್ತಾ-ರೆಓ
ಹೇಳು-ತ್ತಾ-ರೆ-ದೇ-ವರು
ಹೇಳು-ತ್ತಾ-ರೆ-ಧ-ರ್ಮವು
ಹೇಳು-ತ್ತಾ-ರೆ-ಯಾ-ರನ್ನೂ
ಹೇಳು-ತ್ತಾ-ರೆ-ಯಾರೀ
ಹೇಳು-ತ್ತಾರೋ
ಹೇಳು-ತ್ತಾಳೆ
ಹೇಳು-ತ್ತಾ-ಳೆ-ಅ-ವರ
ಹೇಳು-ತ್ತಾ-ಳೆ-ಅ-ವರು
ಹೇಳುತ್ತಿ
ಹೇಳು-ತ್ತಿದ್ದ
ಹೇಳು-ತ್ತಿ-ದ್ದಂತೆ
ಹೇಳು-ತ್ತಿ-ದ್ದ-ರಾ-ದರೂ
ಹೇಳು-ತ್ತಿ-ದ್ದರು
ಹೇಳು-ತ್ತಿ-ದ್ದ-ರು-ಆ-ಲೋ-ಚನೆ
ಹೇಳು-ತ್ತಿ-ದ್ದ-ರುಈ
ಹೇಳು-ತ್ತಿ-ದ್ದ-ರು-ಕೇ-ವಲ
ಹೇಳು-ತ್ತಿ-ದ್ದಾಗ
ಹೇಳು-ತ್ತಿ-ದ್ದಾರೆ
ಹೇಳು-ತ್ತಿ-ದ್ದೀ-ರಿ-ನೀವೂ
ಹೇಳು-ತ್ತಿದ್ದೆ
ಹೇಳು-ತ್ತಿ-ದ್ದೇನೆ
ಹೇಳು-ತ್ತಿ-ರ-ಲಿಲ್ಲ
ಹೇಳು-ತ್ತಿ-ರು-ತ್ತೀರಿ
ಹೇಳು-ತ್ತಿ-ರುವ
ಹೇಳು-ತ್ತಿ-ರು-ವ-ವರು
ಹೇಳು-ತ್ತಿ-ರು-ವಾ-ಗಲೇ
ಹೇಳು-ತ್ತಿ-ರು-ವು-ದ-ನ್ನೆಲ್ಲ
ಹೇಳು-ತ್ತಿ-ರು-ವುದು
ಹೇಳು-ತ್ತಿಲ್ಲ
ಹೇಳು-ತ್ತೀ-ಯಲ್ಲ
ಹೇಳು-ತ್ತೀರಿ
ಹೇಳು-ತ್ತೇನೆ
ಹೇಳು-ನೀವು
ಹೇಳುವ
ಹೇಳು-ವಂತೆ
ಹೇಳು-ವಂ-ತೆ-ಒಂದು
ಹೇಳು-ವ-ವ-ರನ್ನು
ಹೇಳು-ವ-ವರು
ಹೇಳು-ವಾಗ
ಹೇಳುವು
ಹೇಳು-ವುದನ್ನು
ಹೇಳು-ವು-ದ-ನ್ನೆಲ್ಲ
ಹೇಳು-ವು-ದನ್ನೇ
ಹೇಳು-ವು-ದರ
ಹೇಳು-ವು-ದ-ರಲ್ಲಿ
ಹೇಳು-ವು-ದಾ-ದರೆ
ಹೇಳು-ವು-ದಿಷ್ಟೆ
ಹೇಳು-ವು-ದಿ-ಷ್ಟೆ-ಅ-ವೆಲ್ಲ
ಹೇಳು-ವುದು
ಹೇಳು-ವುದೇ
ಹೇಳು-ವು-ದೇ-ನಿದೆ
ಹೇಳು-ವು-ದೇ-ನೆಂ-ದರೆ
ಹೇಳು-ಶಕ್ತಿ
ಹೇವೀಸ್
ಹೇಸ-ರ-ಗ-ತ್ತೆ-ಗಳ
ಹೇಸ-ರ-ಗ-ತ್ತೆ-ಗಳನ್ನು
ಹೇಸು-ವು-ದಿ-ಲ್ಲ-ವಲ್ಲ
ಹೈ
ಹೈಡೆ-ಲ್ಬರ್ಗ್ಗೆ
ಹೈದ-ರ-ಬಾ-ದಿನ
ಹೈದರಾ
ಹೈದ-ರಾ-ಬಾ-ದಿಗೆ
ಹೈದ-ರಾ-ಬಾ-ದಿನ
ಹೈದ-ರಾ-ಬಾ-ದಿ-ನಲ್ಲಿ
ಹೈದ-ರಾ-ಬಾ-ದಿ-ನ-ಲ್ಲಿದ್ದ
ಹೈದ-ರಾ-ಬಾ-ದಿ-ನ-ಲ್ಲಿ-ದ್ದಾಗ
ಹೈದ-ರಾ-ಬಾದ್
ಹೈಸ್ಕೂ-ಲಿನ
ಹೊಂದ
ಹೊಂದ-ಬೇ-ಕೆಂಬ
ಹೊಂದಾ-ಣಿಕೆ
ಹೊಂದಾ-ಣಿ-ಕೆ-ಗ-ಳ-ನ್ನುಂ-ಟು-ಮಾ-ಡು-ವುದು
ಹೊಂದಿ
ಹೊಂದಿ-ಕೆ-ಯಾ-ಗು-ತ್ತವೆ
ಹೊಂದಿ-ಕೆ-ಯಾ-ಗು-ತ್ತಿತ್ತು
ಹೊಂದಿ-ಕೆ-ಯಾ-ಗು-ವಂ-ಥ-ದಲ್ಲ
ಹೊಂದಿ-ಕೊ-ಳ್ಳ-ಬೇಕು
ಹೊಂದಿ-ಕೊ-ಳ್ಳಲು
ಹೊಂದಿ-ಕೊ-ಳ್ಳು-ತ್ತಿ-ದ್ದೇನೆ
ಹೊಂದಿ-ಕೊ-ಳ್ಳು-ವು-ದ-ರ-ಲ್ಲಿ-ದ್ದೇನೆ
ಹೊಂದಿತ್ತು
ಹೊಂದಿದ
ಹೊಂದಿ-ದರು
ಹೊಂದಿ-ದ-ವ-ರಾ-ಗ-ಬೇಕು
ಹೊಂದಿ-ದ-ವ-ರಾದ
ಹೊಂದಿದೆ
ಹೊಂದಿದ್ದ
ಹೊಂದಿ-ದ್ದ-ರಾ-ದರೂ
ಹೊಂದಿ-ದ್ದರು
ಹೊಂದಿ-ದ್ದರೂ
ಹೊಂದಿ-ದ್ದ-ವನು
ಹೊಂದಿ-ದ್ದ-ವರು
ಹೊಂದಿ-ದ್ದಾ-ನೆಂದು
ಹೊಂದಿ-ದ್ದಾರೆ
ಹೊಂದಿ-ದ್ದೀರಿ
ಹೊಂದಿದ್ದೂ
ಹೊಂದಿ-ರ-ಬೇಕು
ಹೊಂದಿ-ರುವ
ಹೊಂದಿ-ರು-ವಂ-ತಹ
ಹೊಂದಿ-ರು-ವ-ವರು
ಹೊಂದಿ-ರು-ವು-ದ-ಲ್ಲದೆ
ಹೊಂದಿ-ರು-ವೆ-ನೆಂ-ದಾ-ಗಲಿ
ಹೊಂದಿ-ಸಿ-ಕೊ-ಡ-ಬ-ಹುದು
ಹೊಂದಿ-ಸಿ-ಕೊ-ಳ್ಳ-ಲಾ-ರ-ದ-ವ-ನಾ-ಗಿ-ದ್ದಾನೆ
ಹೊಂದಿ-ಸಿ-ಕೊ-ಳ್ಳಲಿ
ಹೊಂದುತ್ತ
ಹೊಂದು-ತ್ತವೆ
ಹೊಂದು-ತ್ತಾರೆ
ಹೊಂದು-ತ್ತಿದ್ದ
ಹೊಂದು-ತ್ತಿದ್ದು
ಹೊಂದು-ತ್ತೀರಿ
ಹೊಂದುವ
ಹೊಂದು-ವಂ-ತಹ
ಹೊಂದು-ವಂ-ತಾ-ಗಲಿ
ಹೊಂದು-ವು-ದ-ಕ್ಕಾಗಿ
ಹೊಕ್ಕಾಗ
ಹೊಕ್ಕಿ-ರ-ದಿ-ದ್ದರೆ
ಹೊಕ್ಕಿಲ್ಲ
ಹೊಕ್ಕು
ಹೊಗ-ಳ-ಲಾರೆ
ಹೊಗಳಿ
ಹೊಗ-ಳಿ-ಕೆ-ತೆ-ಗ-ಳಿ-ಕೆ-ಗ-ಳಿಗೆ
ಹೊಗ-ಳಿ-ಕೆ-ಗಳು
ಹೊಗ-ಳಿ-ಕೆಯ
ಹೊಗ-ಳಿ-ಕೆ-ಯನ್ನೂ
ಹೊಗ-ಳಿ-ಕೆ-ಯಿಂದ
ಹೊಗ-ಳಿ-ದರು
ಹೊಗ-ಳಿ-ಸಿ-ಕೊ-ಳ್ಳಲು
ಹೊಗ-ಳುತ್ತ
ಹೊಗ-ಳು-ಭ-ಟರು
ಹೊಗ-ಳು-ಭ-ಟರೇ
ಹೊಗ-ಳು-ವುದು
ಹೊಚ್ಚ-ಹೊಸ
ಹೊಚ್ಚ-ಹೊ-ಸ-ತಾದ
ಹೊಚ್ಚ-ಹೊ-ಸ-ದಾ-ಗಿ-ರು-ತ್ತಿ-ದ್ದುವು
ಹೊಚ್ಚ-ಹೊ-ಸದೂ
ಹೊಟ್ಟೆ
ಹೊಟ್ಟೆ-ಕಿ-ಚ್ಚಿನ
ಹೊಟ್ಟೆ-ಕಿಚ್ಚು
ಹೊಟ್ಟೆ-ಗಿ-ಲ್ಲದೆ
ಹೊಟ್ಟೆಗೆ
ಹೊಟ್ಟೆ-ತುಂಬ
ಹೊಟ್ಟೆಯ
ಹೊಟ್ಟೆ-ಯಲ್ಲಿ
ಹೊಟ್ಟೆ-ಯಿಂದ
ಹೊಟ್ಟೆ-ಯು-ರಿ-ಯನ್ನು
ಹೊಟ್ಟೆ-ಹು-ಣ್ಣಾಗು
ಹೊಡುವ
ಹೊಡೆದ
ಹೊಡೆ-ದರು
ಹೊಡೆ-ದಾಗ
ಹೊಡೆ-ದಾ-ಡಿ-ಕೊಂ-ಡಿದ್ದು
ಹೊಡೆ-ದಾ-ಡಿ-ದುವು
ಹೊಡೆ-ದಾ-ಡು-ತ್ತಾನೆ
ಹೊಡೆ-ದಾ-ಡು-ತ್ತಾ-ರೆ-ಇದೇ
ಹೊಡೆ-ದಾ-ಡುವ
ಹೊಡೆದು
ಹೊಡೆಯು
ಹೊಡೆ-ಯು-ವಂ-ತಿದೆ
ಹೊಣೆ
ಹೊಣೆ-ಗಾ-ರಿ-ಕೆಯ
ಹೊಣೆ-ಗಾ-ರಿ-ಕೆ-ಯನ್ನು
ಹೊಣೆ-ಗಾ-ರಿ-ಕೆ-ಯಿಂದ
ಹೊಣೆ-ಗಾ-ರಿ-ಕೆ-ಯಿ-ದೆ-ಯಲ್ಲ
ಹೊಣೆ-ಗಾ-ರಿ-ಕೆಯು
ಹೊಣೆ-ಯ-ನ್ನಾ-ಗಿ-ಸುವ
ಹೊಣೆ-ಯನ್ನು
ಹೊಣೆ-ಯಾ-ಗಿಸಿ
ಹೊತ್ತ
ಹೊತ್ತ-ಗೆ-ಗಳು
ಹೊತ್ತರು
ಹೊತ್ತಾದ
ಹೊತ್ತಾ-ದರೂ
ಹೊತ್ತಿ
ಹೊತ್ತಿ-ಗಂತೂ
ಹೊತ್ತಿ-ಗಾ-ಗಲೇ
ಹೊತ್ತಿಗೂ
ಹೊತ್ತಿಗೆ
ಹೊತ್ತಿ-ತ-ನಲ್ಲಿ
ಹೊತ್ತಿನ
ಹೊತ್ತಿ-ನಲ್ಲಿ
ಹೊತ್ತಿ-ನಲ್ಲೇ
ಹೊತ್ತಿ-ನ-ವ-ರೆಗೂ
ಹೊತ್ತಿ-ನ-ವ-ರೆಗೆ
ಹೊತ್ತಿ-ನಿಂ-ದಲೇ
ಹೊತ್ತಿ-ಸ-ಬೇ-ಕಾ-ಗಿದೆ
ಹೊತ್ತಿ-ಸಿ-ಯಾ-ಗಿತ್ತು
ಹೊತ್ತು
ಹೊತ್ತು-ಕೊಂ-ಡಿ-ದ್ದ-ನ-ಲ್ಲದೆ
ಹೊತ್ತೂ
ಹೊತ್ತೇ
ಹೊದಿ-ಕೆ-ಯಾ-ಗಿ-ಸಿ-ಕೊಂಡು
ಹೊದಿ-ಸಿ-ದರು
ಹೊದಿಸು
ಹೊದ್ದು-ಕೊ-ಳ್ಳಲು
ಹೊನ-ಲನ್ನೇ
ಹೊನಲೇ
ಹೊನ್ನಾಯ್ತು
ಹೊಮ್ಮಿ
ಹೊಮ್ಮಿತು
ಹೊಮ್ಮಿ-ಬಂದ
ಹೊಮ್ಮಿ-ಸಿದ
ಹೊಮ್ಮಿ-ಸು-ತ್ತಿ-ರುವ
ಹೊಮ್ಮು-ತ್ತಿದ್ದ
ಹೊಮ್ಮು-ತ್ತಿದ್ದು
ಹೊಮ್ಮು-ತ್ತಿ-ರು-ವಂತೆ
ಹೊಮ್ಮು-ವುದೋ
ಹೊರ
ಹೊರಕ್ಕೆ
ಹೊರ-ಗ-ಟ್ಟಿ-ದರು
ಹೊರ-ಗ-ಟ್ಟು-ವಂತೆ
ಹೊರ-ಗ-ಡೆಯ
ಹೊರ-ಗ-ಡೆಯೇ
ಹೊರಗಿ
ಹೊರ-ಗಿ-ಡ-ಬೇ-ಕೆಂದು
ಹೊರ-ಗಿ-ಡು-ವುದು
ಹೊರ-ಗಿನ
ಹೊರ-ಗಿ-ನಿಂದ
ಹೊರಗೂ
ಹೊರಗೆ
ಹೊರ-ಗೆ-ಡವಿ
ಹೊರ-ಗೆ-ಡ-ವಿ-ದರು
ಹೊರ-ಗೆ-ಡ-ವು-ತ್ತಾರೆ
ಹೊರ-ಗೆ-ಡ-ಹಿ-ದರು
ಹೊರ-ಗೆ-ಡ-ಹಿ-ದಾಗ
ಹೊರಗೇ
ಹೊರ-ಚಿ-ಮ್ಮಿ-ದರೆ
ಹೊರ-ಜ-ಗ-ತ್ತಿಗೆ
ಹೊರಟ
ಹೊರ-ಟರು
ಹೊರ-ಟರೆ
ಹೊರ-ಟ-ರೆಂದು
ಹೊರ-ಟ-ರೆಂಬ
ಹೊರ-ಟಲ್ಲಿ
ಹೊರ-ಟಳು
ಹೊರ-ಟಾಗ
ಹೊರ-ಟಾ-ಗಿ-ನಿಂದ
ಹೊರಟಿ
ಹೊರ-ಟಿದ್ದ
ಹೊರ-ಟಿ-ದ್ದರು
ಹೊರ-ಟಿ-ದ್ದ-ವರು
ಹೊರ-ಟಿ-ದ್ದಾಗ
ಹೊರ-ಟಿದ್ದು
ಹೊರ-ಟಿ-ದ್ದೇವೆ
ಹೊರ-ಟಿ-ರುವ
ಹೊರ-ಟಿ-ರು-ವುದು
ಹೊರಟು
ಹೊರ-ಟು-ನಿಂ-ತರು
ಹೊರ-ಟು-ನಿಂ-ತಾಗ
ಹೊರ-ಟು-ಬಂ-ದರು
ಹೊರ-ಟು-ಬ-ರು-ವಂತೆ
ಹೊರ-ಟು-ಬಿಟ್ಟ
ಹೊರ-ಟು-ಬಿ-ಟ್ಟರು
ಹೊರ-ಟು-ಬಿ-ಟ್ಟರೆ
ಹೊರ-ಟು-ಬಿ-ಟ್ಟಿ-ದ್ದರು
ಹೊರ-ಟು-ಬಿ-ಟ್ಟಿ-ದ್ದಾರೆ
ಹೊರ-ಟು-ಬಿ-ಡಲಿ
ಹೊರ-ಟು-ಬಿ-ಡಲು
ಹೊರ-ಟು-ಬಿ-ಡು-ತ್ತಾನೆ
ಹೊರ-ಟು-ಬಿ-ಡು-ತ್ತಾರೆ
ಹೊರ-ಟು-ಬಿ-ಡು-ತ್ತಿ-ದ್ದರು
ಹೊರ-ಟುವು
ಹೊರ-ಟು-ಹೋ-ಗಲು
ಹೊರ-ಟು-ಹೋಗಿ
ಹೊರ-ಟು-ಹೋ-ಗಿದ್ದ
ಹೊರ-ಟು-ಹೋದ
ಹೊರ-ಟು-ಹೋ-ದಂತೆ
ಹೊರ-ಟು-ಹೋ-ದ-ಮೇಲೆ
ಹೊರ-ಟು-ಹೋ-ದರು
ಹೊರ-ಟು-ಹೋ-ದರೂ
ಹೊರ-ಟು-ಹೋ-ಯಿತು
ಹೊರಟೆ
ಹೊರಟೇ
ಹೊರ-ಟೇ-ಬಿ-ಟ್ಟರು
ಹೊರಡ
ಹೊರ-ಡ-ದಾ-ಯಿತು
ಹೊರ-ಡ-ಬ-ಹುದು
ಹೊರ-ಡ-ಬೇ-ಕಾ-ಗಿದೆ
ಹೊರ-ಡ-ಬೇ-ಕಾ-ಗಿ-ರು-ವು-ದ-ರಿಂದ
ಹೊರ-ಡ-ಬೇಕು
ಹೊರ-ಡಲಿ
ಹೊರ-ಡ-ಲಿತ್ತು
ಹೊರ-ಡ-ಲಿ-ದ್ದರು
ಹೊರ-ಡ-ಲಿ-ದ್ದಾ-ರೆಂಬ
ಹೊರ-ಡ-ಲಿಲ್ಲ
ಹೊರ-ಡಲು
ಹೊರ-ಡಲೇ
ಹೊರ-ಡ-ಲೇ-ಬೇಕು
ಹೊರ-ಡಿ-ಸ-ಲಾದ
ಹೊರ-ಡಿ-ಸಲೋ
ಹೊರ-ಡಿಸಿ
ಹೊರ-ಡಿ-ಸಿ-ದರು
ಹೊರ-ಡಿ-ಸಿ-ದಾಗ
ಹೊರ-ಡಿ-ಸು-ವಂತೆ
ಹೊರಡು
ಹೊರ-ಡು-ತ್ತಾನೆ
ಹೊರ-ಡು-ತ್ತಾರೆ
ಹೊರ-ಡುತ್ತಿ
ಹೊರ-ಡು-ತ್ತಿ-ದ್ದರು
ಹೊರ-ಡು-ತ್ತಿ-ರ-ಲಿಲ್ಲ
ಹೊರ-ಡುವ
ಹೊರ-ಡು-ವಂತೆ
ಹೊರ-ಡು-ವಂದು
ಹೊರ-ಡು-ವ-ವ-ರೆಗೂ
ಹೊರ-ಡು-ವಾಗ
ಹೊರ-ಡು-ವು-ದಕ್ಕೆ
ಹೊರ-ಡು-ವು-ದರ
ಹೊರ-ಡು-ವು-ದಾಗಿ
ಹೊರ-ಡು-ವುದು
ಹೊರ-ಡು-ವು-ದೆಂದು
ಹೊರ-ಡು-ವುದೇ
ಹೊರ-ತ-ರಲು
ಹೊರ-ತ-ರು-ತ್ತಿ-ದ್ದೇವೆ
ಹೊರ-ತ-ರುವ
ಹೊರ-ತಾಗಿ
ಹೊರತು
ಹೊರ-ತೆ-ಗೆದು
ಹೊರ-ತೇ-ನಲ್ಲ
ಹೊರ-ದೂ-ಡಿ-ಬಿ-ಟ್ಟಿ-ದೆಯೋ
ಹೊರ-ನೋ-ಟಕ್ಕೆ
ಹೊರ-ಬಂ-ದಂತೆ
ಹೊರ-ಬಂ-ದರು
ಹೊರ-ಬಂ-ದಳು
ಹೊರ-ಬಂದು
ಹೊರ-ಬ-ರು-ತ್ತಿತ್ತು
ಹೊರ-ಬಿಟ್ಟ
ಹೊರ-ಬಿದ್ದ
ಹೊರ-ಬೀ-ಳು-ತ್ತವೆ
ಹೊರ-ಬೇ-ಕು-ಅದೂ
ಹೊರ-ರಾಷ್ಟ್ರ
ಹೊರ-ಳಾ-ಡ-ತೊ-ಡ-ಗಿ-ದರು
ಹೊರ-ಳಿತು
ಹೊರ-ಸೂ-ಸುತ್ತ
ಹೊರ-ಸೂ-ಸು-ತ್ತಿದ್ದ
ಹೊರ-ಹ-ರಿ-ಯ-ಲಾ-ರಂ-ಭಿ-ಸು-ತ್ತಿತ್ತು
ಹೊರ-ಹಾ-ಕಿದ
ಹೊರ-ಹಾ-ಕಿ-ದ-ವ-ರನ್ನು
ಹೊರ-ಹೊ-ಮ್ಮ-ದಂತೆ
ಹೊರ-ಹೊ-ಮ್ಮ-ಬೇಕು
ಹೊರ-ಹೊಮ್ಮಿ
ಹೊರ-ಹೊ-ಮ್ಮಿದ
ಹೊರ-ಹೊ-ಮ್ಮಿ-ಸು-ತ್ತದೆ
ಹೊರ-ಹೊ-ಮ್ಮಿ-ಸುವ
ಹೊರ-ಹೊ-ಮ್ಮು-ತ್ತಿತ್ತು
ಹೊರ-ಹೊ-ಮ್ಮು-ತ್ತಿದ್ದ
ಹೊರ-ಹೊ-ಮ್ಮುವ
ಹೊರ-ಹೊ-ಮ್ಮು-ವಂ-ತಿತ್ತು
ಹೊರ-ಹೊ-ಮ್ಮು-ವಂತೆ
ಹೊರ-ಹೊ-ಮ್ಮು-ವುದನ್ನು
ಹೊರ-ಹೋ-ಗು-ವಂ-ತಿಲ್ಲ
ಹೊರಿಸಿ
ಹೊರಿ-ಸಿದ
ಹೊರಿ-ಸಿದ್ದ
ಹೊರೆ-ಯುವ
ಹೊಲ-ಗ-ದ್ದೆ-ಗಳು
ಹೊಲದ
ಹೊಲ-ದಲ್ಲಿ
ಹೊಲಸು
ಹೊಳಪು
ಹೊಳೆ
ಹೊಳೆ-ದಂ-ತಾ-ಯಿತು
ಹೊಳೆ-ದಾಗ
ಹೊಳೆ-ದಿ-ರ-ಲಿಲ್ಲ
ಹೊಳೆ-ಯ-ಲಿಲ್ಲ
ಹೊಳೆ-ಯಲೇ
ಹೊಳೆ-ಯಿತು
ಹೊಳೆ-ಯಿ-ತು-ನಾವು
ಹೊಳೆ-ಯುತ್ತ
ಹೊಳೆ-ಯು-ತ್ತಿ-ದ್ದುವು
ಹೊಳೆ-ಯುವ
ಹೊಳೆ-ಯು-ವು-ದಿಲ್ಲ
ಹೊಸ
ಹೊಸ-ಕಿ-ಹಾ-ಕಿ-ದ-ವ-ರಿಗೆ
ಹೊಸ-ಕಿ-ಹಾ-ಕಿ-ಬಿಡಿ
ಹೊಸ-ಕು-ತ್ತಿ-ದ್ದೇವೆ
ಹೊಸ-ತನ
ಹೊಸ-ತ-ನ-ದಿಂ-ದೊ-ಡ-ಗೂಡಿ
ಹೊಸ-ತಾ-ಗಿಯೇ
ಹೊಸ-ತಾದ
ಹೊಸ-ತಾ-ದು-ದೇ-ನನ್ನೂ
ಹೊಸತು
ಹೊಸ-ತು-ಭಾ-ರ-ತ-ದಲ್ಲಿ
ಹೊಸ-ತೇನೂ
ಹೊಸ-ದಾಗಿ
ಹೊಸ-ದಿ-ರ-ಬ-ಹುದು
ಹೊಸದು
ಹೊಸದೆ
ಹೊಸ-ದೇ-ನನ್ನೂ
ಹೊಸ-ದೊಂದು
ಹೊಸ-ಬ-ರನ್ನು
ಹೊಸ-ಹೊಸ
ಹೊಸ್ತಿ-ಲ-ಲ್ಲಿ-ರು-ವ-ವ-ರು-ಬೇಕು
ಹೋ
ಹೋಗ
ಹೋಗ-ದಂತೆ
ಹೋಗ-ದಿ-ದ್ದುದು
ಹೋಗ-ಬಲ್ಲ
ಹೋಗ-ಬ-ಹುದು
ಹೋಗ-ಬ-ಹು-ದು-ಆ-ದರೆ
ಹೋಗ-ಬಾ-ರದು
ಹೋಗ-ಬೇ-ಕಾಗಿ
ಹೋಗ-ಬೇ-ಕಾ-ಗಿತ್ತು
ಹೋಗ-ಬೇ-ಕಾ-ಗಿದೆ
ಹೋಗ-ಬೇ-ಕಾ-ಗಿ-ದ್ದುದು
ಹೋಗ-ಬೇ-ಕಾ-ಗಿಲ್ಲ
ಹೋಗ-ಬೇ-ಕಾ-ಗು-ತ್ತದೆ
ಹೋಗ-ಬೇ-ಕಾದ
ಹೋಗ-ಬೇ-ಕಾ-ದರೆ
ಹೋಗ-ಬೇ-ಕಾ-ದು-ದರ
ಹೋಗ-ಬೇ-ಕಾ-ಯಿತು
ಹೋಗ-ಬೇ-ಕಿತ್ತು
ಹೋಗ-ಬೇಕು
ಹೋಗ-ಬೇ-ಕೆಂ-ದರೆ
ಹೋಗ-ಬೇ-ಕೆಂ-ದಿ-ರುವ
ಹೋಗ-ಬೇ-ಕೆಂ-ದಿ-ರು-ವು-ದಾಗಿ
ಹೋಗ-ಬೇ-ಕೆಂದು
ಹೋಗ-ಬೇ-ಕೆಂ-ದು-ಕೊಂ-ಡ-ಮೇಲೆ
ಹೋಗ-ಬೇ-ಕೆಂ-ದು-ಕೊಂ-ಡಿ-ದ್ದೇನೆ
ಹೋಗ-ಬೇ-ಕೆಂಬ
ಹೋಗ-ಬೇ-ಕೆಂ-ಬುದು
ಹೋಗ-ಬೇ-ಕೆ-ನ್ನು-ವುದು
ಹೋಗ-ಬೇಡ
ಹೋಗ-ಬೇ-ಡಪ್ಪಾ
ಹೋಗ-ಬೇಡಿ
ಹೋಗ-ಲಾ-ಡಿ-ಸ-ಬೇ-ಕಾ-ಗಿದೆ
ಹೋಗ-ಲಾ-ಡಿ-ಸ-ಬೇ-ಕಾದ್ದು
ಹೋಗ-ಲಾ-ಡಿ-ಸ-ಬೇ-ಕೆಂದು
ಹೋಗ-ಲಾ-ಡಿ-ಸ-ಲಾ-ಗದೆ
ಹೋಗ-ಲಾ-ಡಿ-ಸಲು
ಹೋಗ-ಲಾ-ಡಿ-ಸಿ-ಕೊಂ-ಡಾಗ
ಹೋಗ-ಲಾ-ಡಿ-ಸಿ-ಕೊ-ಳ್ಳ-ಬೇ-ಕಾ-ಗಿದೆ
ಹೋಗ-ಲಾ-ಡಿ-ಸುವ
ಹೋಗ-ಲಾ-ಡಿ-ಸು-ವುದು
ಹೋಗ-ಲಾರ
ಹೋಗ-ಲಾ-ರ-ದಷ್ಟು
ಹೋಗ-ಲಾರೆ
ಹೋಗಲಿ
ಹೋಗ-ಲಿ-ದ್ದಾರೆ
ಹೋಗ-ಲಿ-ರುವ
ಹೋಗ-ಲಿಲ್ಲ
ಹೋಗಲು
ಹೋಗಲೂ
ಹೋಗ-ಲೇ-ಬೇ-ಕಾ-ಗಿತ್ತು
ಹೋಗ-ಲೇ-ಬೇ-ಕಾದ
ಹೋಗ-ಲೇ-ಬೇಕು
ಹೋಗ-ಲೇ-ಬೇ-ಕೆಂದು
ಹೋಗ-ವು-ದೆಂದು
ಹೋಗಾ-ಲಾ-ಡಿ-ಸುವ
ಹೋಗಿ
ಹೋಗಿತ್ತು
ಹೋಗಿದೆ
ಹೋಗಿದ್ದ
ಹೋಗಿ-ದ್ದ-ರ-ಲ್ಲದೆ
ಹೋಗಿ-ದ್ದರು
ಹೋಗಿ-ದ್ದರೆ
ಹೋಗಿ-ದ್ದರೋ
ಹೋಗಿ-ದ್ದ-ವನು
ಹೋಗಿ-ದ್ದಾಗ
ಹೋಗಿ-ದ್ದಾನೆ
ಹೋಗಿ-ದ್ದಾರೆ
ಹೋಗಿ-ದ್ದಿ-ರ-ಬೇಕು
ಹೋಗಿ-ದ್ದೀರಾ
ಹೋಗಿದ್ದು
ಹೋಗಿ-ದ್ದುವು
ಹೋಗಿದ್ದೆ
ಹೋಗಿ-ದ್ದೇನೆ
ಹೋಗಿ-ಬಂ-ದರು
ಹೋಗಿ-ಬ-ರು-ತ್ತೇನೆ
ಹೋಗಿ-ಬಿಟ್ಟ
ಹೋಗಿ-ಬಿ-ಟ್ಟಿತು
ಹೋಗಿಯೇ
ಹೋಗಿ-ರ-ಬ-ಹುದು
ಹೋಗಿ-ರಲಿ
ಹೋಗಿ-ರ-ಲಿಲ್ಲ
ಹೋಗಿ-ರ-ಲಿ-ಲ್ಲ-ವೆಂದು
ಹೋಗಿ-ರಲು
ಹೋಗಿ-ರುವ
ಹೋಗಿ-ವೆ-ಯೆ-ನ್ನ-ಬೇಕು
ಹೋಗಿ-ಸಂ-ಜೆಯೋ
ಹೋಗು
ಹೋಗುತ್ತ
ಹೋಗು-ತ್ತದೆ
ಹೋಗು-ತ್ತ-ದೆಯೋ
ಹೋಗು-ತ್ತಲೇ
ಹೋಗು-ತ್ತಾ-ನೆಯೋ
ಹೋಗು-ತ್ತಾ-ರಲ್ಲ
ಹೋಗು-ತ್ತಾರೆ
ಹೋಗು-ತ್ತಾ-ರೆಂ-ದರೆ
ಹೋಗು-ತ್ತಾ-ರೆಂದು
ಹೋಗು-ತ್ತಿತ್ತು
ಹೋಗು-ತ್ತಿದ್ದ
ಹೋಗು-ತ್ತಿ-ದ್ದರು
ಹೋಗು-ತ್ತಿ-ದ್ದರೆ
ಹೋಗು-ತ್ತಿ-ದ್ದ-ರೆಂ-ಬು-ದನ್ನು
ಹೋಗು-ತ್ತಿ-ದ್ದರೋ
ಹೋಗು-ತ್ತಿ-ದ್ದ-ವ-ನೊಬ್ಬ
ಹೋಗು-ತ್ತಿ-ದ್ದಾಗ
ಹೋಗು-ತ್ತಿ-ದ್ದಾನೆ
ಹೋಗು-ತ್ತಿ-ದ್ದಾರೆ
ಹೋಗು-ತ್ತಿ-ದ್ದುದು
ಹೋಗು-ತ್ತಿ-ದ್ದು-ವಾ-ದರೂ
ಹೋಗು-ತ್ತಿದ್ದೆ
ಹೋಗು-ತ್ತಿ-ದ್ದೆವು
ಹೋಗು-ತ್ತಿ-ರು-ವಾಗ
ಹೋಗು-ತ್ತಿ-ರು-ವುದನ್ನು
ಹೋಗು-ತ್ತಿ-ರು-ವು-ದಾ-ಗಿಯೂ
ಹೋಗು-ತ್ತಿ-ರು-ವುದು
ಹೋಗು-ತ್ತಿ-ರು-ವುದೇ
ಹೋಗು-ತ್ತೀರಿ
ಹೋಗು-ತ್ತೇನೆ
ಹೋಗುವ
ಹೋಗು-ವಂ-ತಾ-ಗಲು
ಹೋಗು-ವಂ-ತಾ-ಗಿತ್ತು
ಹೋಗು-ವಂ-ತಿತ್ತು
ಹೋಗು-ವಂ-ತಿಲ್ಲ
ಹೋಗು-ವಂತೆ
ಹೋಗು-ವಂಥ
ಹೋಗು-ವ-ವ-ನಿ-ದ್ದೇನೆ
ಹೋಗು-ವ-ವ-ರೆಗೂ
ಹೋಗು-ವ-ವ-ರೆಷ್ಟು
ಹೋಗು-ವ-ವರೇ
ಹೋಗು-ವ-ಷ್ಟ-ರಲ್ಲಿ
ಹೋಗುವು
ಹೋಗು-ವು-ದ-ಕ್ಕಾಗಿ
ಹೋಗು-ವುದನ್ನು
ಹೋಗು-ವು-ದ-ರಿಂದ
ಹೋಗು-ವು-ದಷ್ಟೇ
ಹೋಗು-ವು-ದಾಗಿ
ಹೋಗು-ವು-ದಾ-ದರೆ
ಹೋಗು-ವು-ದಿಲ್ಲ
ಹೋಗು-ವುದು
ಹೋಗು-ವುದೂ
ಹೋಗು-ವು-ದೆಂದು
ಹೋಗು-ವು-ದೇಕೆ
ಹೋಗು-ವುದೊ
ಹೋಗುವೆ
ಹೋಗು-ವೆನೋ
ಹೋಗೋಣ
ಹೋಗೋ-ಣ-ವೆಂದು
ಹೋಟ-ಲಿಗೆ
ಹೋಟೆಲಿ
ಹೋಟೆ-ಲಿ-ಗಾ-ದರೂ
ಹೋಟೆ-ಲಿಗೇ
ಹೋಟೆ-ಲಿನ
ಹೋಟೆ-ಲಿ-ನಲ್ಲಿ
ಹೋಟೆಲು
ಹೋಟೆ-ಲು-ಗಳ
ಹೋಟೆ-ಲೊಂ-ದ-ರಲ್ಲಿ
ಹೋಟೆ-ಲೊ-ಳಗೆ
ಹೋಟೆಲ್
ಹೋಟೆಲ್ಲೇ
ಹೋದ
ಹೋದಂ-ತಿತ್ತು
ಹೋದಂತೆ
ಹೋದ-ದ್ದನ್ನೂ
ಹೋದ-ದ್ದ-ರಿಂದ
ಹೋದ-ನಂತೆ
ಹೋದನೋ
ಹೋದ-ಮೇಲೆ
ಹೋದ-ರಲ್ಲಿ
ಹೋದರು
ಹೋದರೂ
ಹೋದರೆ
ಹೋದ-ರೆಂ-ದರೆ
ಹೋದ-ರೆಂದು
ಹೋದ-ರೆಂ-ಬು-ದನ್ನು
ಹೋದ-ರೆಷ್ಟು
ಹೋದರೋ
ಹೋದಲ್ಲೆಲ್ಲ
ಹೋದಳು
ಹೋದ-ವ-ರಿಗೆ
ಹೋದ-ಹೋ-ದಂತೆ
ಹೋದ-ಹೋ-ದಲ್ಲೆಲ್ಲ
ಹೋದಾಗ
ಹೋದಾ-ಗಿ-ನಿಂ-ದಲೂ
ಹೋದಾರು
ಹೋದೀತು
ಹೋದುದು
ಹೋದೆ
ಹೋದೆ-ಡೆ-ಯ-ಲ್ಲೆಲ್ಲ
ಹೋದೆವು
ಹೋಮ-ಎಂದು
ಹೋಮ-ರ-ನಿಗೆ
ಹೋಮರ್
ಹೋಮ-ವಾ-ಗು-ತ್ತಿತ್ತು
ಹೋಮ್ಗೆ
ಹೋಯಿತು
ಹೋರಾಟ
ಹೋರಾ-ಟ-ಗ-ಳ-ನ್ನ-ಲ್ಲದೆ
ಹೋರಾ-ಟ-ಗಳು
ಹೋರಾ-ಟ-ಗ-ಳೆ-ಲ್ಲವೂ
ಹೋರಾ-ಟದ
ಹೋರಾ-ಟ-ದಲ್ಲಿ
ಹೋರಾ-ಟ-ದಿಂ-ದಲೋ
ಹೋರಾ-ಟ-ಮಯ
ಹೋರಾ-ಟ-ವನ್ನು
ಹೋರಾ-ಟವೂ
ಹೋರಾ-ಟ-ವೆಷ್ಟು
ಹೋರಾಡ
ಹೋರಾ-ಡ-ಬಲ್ಲ
ಹೋರಾ-ಡ-ಬೇ-ಕಾ-ಗಿ-ಬಂ-ದಿತು
ಹೋರಾ-ಡ-ಬೇ-ಕಾ-ಗಿ-ಬ-ರು-ವಾಗ
ಹೋರಾ-ಡ-ಬೇ-ಕಾ-ಗಿಲ್ಲ
ಹೋರಾ-ಡ-ಬೇ-ಕಾ-ಯಿತು
ಹೋರಾ-ಡ-ಬೇಕು
ಹೋರಾ-ಡಿದ
ಹೋರಾಡು
ಹೋರಾ-ಡು-ತ್ತಾರೆ
ಹೋರಾ-ಡು-ವಂತೆ
ಹೋರಾ-ಡು-ವಲ್ಲಿ
ಹೋರಾ-ಡು-ವುದೇ
ಹೋಲಿ-ಕೆಯ
ಹೋಲಿ-ಕೆ-ಯಾ-ದರೂ
ಹೋಲಿಸಿ
ಹೋಲಿ-ಸಿ-ದರೆ
ಹೋಲಿ-ಸಿ-ನೋ-ಡುತ್ತ
ಹೋಲಿ-ಸಿ-ರು-ವು-ದ-ರಲ್ಲಿ
ಹೋಲಿ-ಸು-ತ್ತೇನೆ
ಹೋಲು
ಹೋಲು-ತ್ತದೆ
ಹೋಲು-ತ್ತಿತ್ತು
ಹೋಲು-ವು-ದೆಂಬ
ಹೋವೇ
ಹೋವ್
ಹೌದಾ-ದಲ್ಲಿ
ಹೌದು
ಹೌದೆ
ಹೌದೆಂದು
ಹೌದೆ-ನ್ನುತ್ತ
ಹೌಸ-ನ್ನಿಗೆ
ಹೌಸ್ಗೆ
ಹೌಹಾರಿ
ಹೌಹಾ-ರಿದ
ಹ್ಞೂ
ಹ್ಞೂಂ
ಹ್ಯಾಂಪ್ಶೈರ್
ಹ್ಯಾಂಬ-ರ್ಗಿ-ನಲ್ಲಿ
ಹ್ಯಾಂಬ-ರ್ಗಿ-ನಿಂದ
ಹ್ಯಾಂಬ-ರ್ಗ್
ಹ್ಯಾಟು
ಹ್ಯಾಟು-ಗಳನ್ನೂ
ಹ್ಯಾಮಂ-ಡನ
ಹ್ಯಾಮಂಡ್
ಹ್ಯಾಮ್ಲಿನ್
ಹ್ಯಾರಿ-ಯೆಟ್
ಹ್ಯುಮೇನ
್ಣವಾಗಿ
}
