
\chapter{ಪರಿವ್ರಜನದ ಅನುಭವ ಪರಿಪರಿ}

\noindent

ಸ್ವಾಮಿ ವಿವೇಕಾನಂದರ ಪರಿವ್ರಾಜಕ ಜೀವನ ಅತ್ಯಂತ ವೈವಿಧ್ಯಮಯವಾದುದು; ವೈಚಿತ್ರ್ಯ ಮಯವಾದುದು. ವೀರಸಂನ್ಯಾಸಿ ವಿವೇಕಾನಂದರು ವಿಶ್ವವಿಜೇತರಾದ ಪರಿಯೇ ನಾಟಕೀಯವೆನ್ನ ಬಹುದಾದಷ್ಟು ವೇಗವಾಗಿ ನಡೆದುಹೋದ ಕಥೆ. ಅದು ಅನೇಕ ಅನಿರೀಕ್ಷಿತ ತಿರುವುಗಳಿಂದ ಕೂಡಿದುದು. ಅವರ ಆ ದಿನಗಳ ಬಗ್ಗೆ ನಮಗೆ ತಿಳಿದುಬಂದಿರುವುದು ಬಹಳ ಅಲ್ಪವೆಂದೇ ಹೇಳ ಬೇಕು. ಸ್ವಾಮೀಜಿಯಂತಹ ಅಸಾಧಾರಣ-ಅಲೌಕಿಕ ವ್ಯಕ್ತಿಗೆ ಎಂತೆಂತಹ ಅನುಭವಗಳಾದುವೋ ಯಾರಿಗೆ ಗೊತ್ತು! ಏಕೆಂದರೆ ಅವರು ಆ ಬಗ್ಗೆ ತಾವಾಗಿಯೇ ಹೇಳಿದ್ದು ತೀರ ಅಪರೂಪ. ಅದ ರಲ್ಲೂ ತಮ್ಮ ಹಿರಿಮೆ-ಮಹಿಮೆಗಳನ್ನು ಪ್ರತಿಬಿಂಬಿಸುವಂತಹ ಘಟನೆಗಳನ್ನು ಅವರು ಪ್ರಸ್ತಾಪಿ ಸುತ್ತಲೇ ಇರಲಿಲ್ಲ. ಯಾವಾಗಲಾದರೂ ಪ್ರಸ್ತಾಪಿಸಿದರೂ ಅದೊಂದು ತೀರ ಸಾಮಾನ್ಯ ವಿಷಯವೋ ಎಂಬಂತೆ ಹಾರಿಸಿಬಿಡುತ್ತಿದ್ದರು. ಹೀಗೆ ಅದೃಷ್ಟವಶಾತ್ ನಮಗೆ ತಿಳಿದು ಬಂದಿ ರುವ ಕೆಲವು ಘಟನೆಗಳನ್ನು ಈಗ ನೋಡೋಣ.

ಆ ದಿನಗಳಲ್ಲಿ ಸಾಧುಗಳಂತೆ ವೇಷ ಧರಿಸಿ ಭಾರತದಾದ್ಯಂತ ಅಲೆದಾಡುತ್ತಿದ್ದವರು ಲೆಕ್ಕವಿಲ್ಲ ದಷ್ಟು ಜನ. ಇವರ ಪೈಕಿ ನಿಜವಾದ ಸಾಧುಗಳು ಯಾರು, ಕೃತ್ರಿಮ ಸಾಧುಗಳು ಯಾರು ಎಂದು ಹೇಳುವುದು ಬಹಳ ಕಷ್ಟವಾಗಿತ್ತು. ಎಷ್ಟೋ ಭಿಕ್ಷುಕರಿಗೆ ಕಾಷಾಯ ವಸ್ತ್ರವು ಉದರನಿಮಿತ್ತ ವೇಷ ವಾಗಿತ್ತು. ಅಲ್ಲದೆ ಮತ್ತಿನ್ನೆಷ್ಟೋ ಜನ ದುರಾಚಾರಿಗಳಿಗೂ ಕ್ರಾಂತಿಕಾರಿಗಳಿಗೂ ಇದು ಆಶ್ರಯ ವಾಗಿತ್ತು. ಆದ್ದರಿಂದ ಬ್ರಿಟಿಷ್ ಸರಕಾರಕ್ಕೆ ಸಾಧುಗಳ ಮೇಲೆ ಯಾವಾಗಲೂ ಗೃಧ್ರದೃಷ್ಟಿ. ಸ್ವಾಮೀಜಿ ಬಿಹಾರ್ ರಾಜ್ಯದಲ್ಲಿ ಅಲೆದಾಡುತ್ತಿದ್ದ ಸಮಯ. ಆಗ ಯಾರೋ ಕೆಲವರು ರಾಜ್ಯ ದಾದ್ಯಂತ ಮಾವಿನ ಮರಗಳ ಮೇಲೆ ಮಂತ್ರಾಕ್ಷತೆ ಬೆರೆಸಿದ ಮಣ್ಣಿನ ಮುದ್ದೆಗಳನ್ನು ಮೆತ್ತಿ ದ್ದುದು ಕಂಡು ಬಂದಿತು. ಇದು ಯಾವುದೋ ಘೋರ ವಿಪತ್ತಿನ ಮುನ್ಸೂಚನೆಯೆಂದು ಊಹಿಸಿದ ಸರಕಾರ ನಡುಗಿತ್ತು. ಗುಪ್ತ ಪೋಲಿಸರು ಕೂಡಲೇ ಕಾರ್ಯಪ್ರವೃತ್ತರಾದರು. ಇದು ಬಂಡಾಯದ ರಹಸ್ಯ ಸಂದೇಶವೇ ಇರಬೇಕೆಂದು ಅವರ ಸೂಕ್ಷ್ಮಬುದ್ಧಿಗೆ ಗೋಚರವಾಯಿತು. ಸರಿ, ಹಳ್ಳಿಗರನ್ನೆಲ್ಲ ಹಿಡಿದು ಪ್ರಶ್ನಿಸಿದರು; ಹೆದರಿಸಿ ಬೆದರಿಸಿ ನೋಡಿದರು. ತಮಗೆ ಇದೊಂದೂ ತಿಳಿಯದೆಂದು ಜನರು ಗೋಗರೆದರು. ಈಗ ಪೋಲಿಸರ ಅನುಮಾನ ಊರಿಂದೂರಿಗೆ ಅಲೆಯುವ ಸಾಧುಗಳ ಮೇಲೆ ತಿರುಗಿತು. ಸಿಕ್ಕಸಿಕ್ಕ ಸಾಧುಗಳನ್ನೆಲ್ಲ ಹಿಡಿದುಹಾಕಿದರು. ಕಡೆಗೆ ಯಾವ ಪುರಾವೆಯೂ ಸಿಗಲಿಲ್ಲವಾದ್ದರಿಂದ ಅವರನ್ನೂ ಬಿಡಬೇಕಾಯಿತು.

ಆದರೆ ಇಷ್ಟು ಹೊತ್ತಿಗೆ ಸ್ವಾಮೀಜಿಯೂ ಒಂದು ಸಲ ಪೋಲಿಸರ ಕೈಗೆ ಸಿಕ್ಕಿಕೊಂಡಾಗಿತ್ತು. ಆಹಾರಕ್ಕೆ ಭಿಕ್ಷೆಯನ್ನೇ ಅವಲಂಬಿಸಿಕೊಂಡಿದ್ದ ಅವರು, ಎಷ್ಟೋ ಸಲ ಉಪವಾಸದಿಂದ ತೊಳಲ ಬೇಕಾಗುತ್ತಿತ್ತು. ಜೊತೆಗೆ ಉರಿಯುವ ಬಿಸಿಲು ಬೇರೆ. ಹೀಗಾಗಿ ಪರಿವ್ರಾಜಕ ಜೀವನ ಬಲು ಕಷ್ಟ ಕರವಾಗಿ ಪರಿಣಮಿಸಿತ್ತು. ಹೀಗೇ ಒಂದು ದಿನ ಕಾಲೆಳೆದುಕೊಂಡು ಸಾಗಿದ್ದಾಗ, ಹಿಂದಿನಿಂದ ಕರ್ಕಶ ಧ್ವನಿಯೊಂದು ಅರಚಿತು.

“ಏಯ್! ನಿಲ್ಲೋ ಅಲ್ಲಿ!”

ಹಿಂದಿರುಗಿ ನೋಡಿದರೆ, ಸಕಲ ವೇಷಭೂಷಣಧಾರಿಯಾದ ಪೋಲಿಸ್ ಅಧಿಕಾರಿಯೊಬ್ಬ ಕುದುರೆಯ ಮೇಲೆ ಕುಳಿತಿದ್ದಾನೆ. ಇರಿಯುವ ಕಣ್ಣುಗಳು, ಉದ್ದನೆಯ ಗಡ್ಡ, ಕೈಯಲ್ಲೊಂದು ಚಾವಟಿ. ಹಿಂದೆ ಇನ್ನೂ ಕೆಲವು ಪೋಲೀಸ್ ಪೇದೆಗಳು. ಹತ್ತಿರ ಬಂದು ಅಧಿಕಾರಿ ತನ್ನ ಸಹಜ ‘ಮಧುರ’ ಶೈಲಿಯಲ್ಲಿ ವಿಚಾರಿಸಿದ:

“ಯಾವನೋ ನೀನು?”

“ಖಾನ್ ಸಾಹೇಬರೇ, ನೀವೇ ನೋಡುತ್ತಿರುವಂತೆ ನಾನೊಬ್ಬ ಸಾಧು!” ನಯವಾಗಿ ಉತ್ತರಿಸಿ ದರು ಸ್ವಾಮೀಜಿ.

“ಎಲ್ಲ ಸಾಧುಗಳೂ ಬದ್ಮಾಶ್​ಗಳೇ!”–ಅಧಿಕಾರವಾಣಿಯಲ್ಲಿ ತನ್ನ ತೀರ್ಪುಕೊಟ್ಟೇಬಿಟ್ಟ ಸಾಹೇಬ. “ನಡಿ ನನ್ನ ಹಿಂದೆ, ನಿನ್ನನ್ನು ಜೈಲಿಗೆ ಕಳಿಸುತ್ತೇನೆ!”

“ಜೈಲಿಗೆ! ಎಷ್ಟು ದಿನ?” ಸ್ವಾಮೀಜಿ ಆನಂದಾಶ್ಚರ್ಯಗಳಿಂದ ಪ್ರಶ್ನಿಸಿದರು.

“ಎಷ್ಟು ದಿನವಾದರೂ ಆಗಬಹುದು. ಹದಿನೈದು ದಿನಗಳೇ ಆಗಬಹುದು, ಇಲ್ಲದಿದ್ದರೆ ಒಂದು ತಿಂಗಳೂ ಆಗಬಹುದು.”

ಸ್ವಾಮೀಜಿ ತುಂಬ ನಿರಾಶೆಯಿಂದ, ಯಾಚನೆಯ ದನಿಯಲ್ಲಿ ಕೇಳಿದರು:

“ಅಯ್ಯೊ, ಒಂದೇ ತಿಂಗಳೇ? ಖಾನ್ ಸಾಹೇಬರೇ... ಒಂದಾರು ತಿಂಗಳಾದರೂ–ಬೇಡ ಹೋಗಲಿ, ಕಡೇ ಪಕ್ಷ ಮೂರ್ನಾಲ್ಕು ತಿಂಗಳಾದರೂ ನನ್ನನ್ನು ಜೈಲಿಗೆ ಕಳಿಸಲು ಸಾಧ್ಯವಿಲ್ಲವೆ?”

ಪೋಲಿಸನ ಮನಸ್ಸಿನಲ್ಲೊಂದು ಸಂಶಯ ಹುಟ್ಟಿಕೊಂಡಿತು.

“ಏಯ್, ಜೈಲಿಗೆ ಹೋಗಲು ನಿನಗೇಕೆ ಅಷ್ಟೊಂದು ಇಷ್ಟ?”

ಸ್ವಾಮೀಜಿ ಸತ್ಯವನ್ನೇ ಹೇಳಿದರು:

“ನಮ್ಮ ಈ ಅಲೆದಾಟದ ಜೀವನಕ್ಕಿಂತ ಜೈಲಿನ ವಾಸ ಎಷ್ಟೋ ಸುಖಕರವಾದದ್ದು. ಹೀಗೆ ಬೆಳಗಿನಿಂದ ರಾತ್ರಿಯವರೆಗೆ ಸುತ್ತಾಡಿ ಒದ್ದಾಡುವುದಕ್ಕೆ ಹೋಲಿಸಿದರೆ ಜೈಲಿನ ಕೆಲಸ ಏನೇನೂ ಅಲ್ಲ. ಇಲ್ಲಿ ನಮಗೆ ಭಿಕ್ಷೆ ಸಿಕ್ಕಿದರೆ ಸಿಕ್ಕಿತು. ಇಲ್ಲದಿದ್ದರೆ ಇಲ್ಲ. ಎಷ್ಟೋ ಸಲ ನಾನು ಉಪವಾಸ ಬೀಳಬೇಕಾಗುತ್ತದೆ. ಜೈಲಿನಲ್ಲಾದರೆ ಎರಡುಹೊತ್ತಿನ ಊಟಕ್ಕಂತೂ ಮೋಸವಿಲ್ಲ. ಸಾಹೇಬರೇ, ನನ್ನನ್ನು ಒಂದು ನಾಲ್ಕಾರು ತಿಂಗಳಾದರೂ ಜೈಲಿಗೆ ಕಳಿಸಿದರೆ, ನಿಮ್ಮ ಉಪಕಾರವನ್ನೆಂದೂ ಮರೆಯುವುದಿಲ್ಲ.”

“ಏಯ್, ನೀನು ಮೊದಲು ಇಲ್ಲಿಂದ ಹೊರಡು!” ಎಂದು ಸ್ವಾಮೀಜಿಯನ್ನು ಬೀಳ್ಕೊಂಡು ಸಾಹೇಬ, ಕುದುರೆಯನ್ನು ಹಿಂದಕ್ಕೆ ತಿರುಗಿಸಿ ನಡೆದ.

ಎಷ್ಟೋ ದಿನಗಳಾದ ಮೇಲೆ ವಾಸ್ತವಾಂಶ ತಿಳಿದುಬಂದಿತು–ಮಾವಿನಫಸಲು ಹೆಚ್ಚಾಗಲಿ ಎಂದು ಮರಗಳ ಮಾಲೀಕರು ಮಂತ್ರಾಕ್ಷತೆ-ಮಣ್ಣುಗಳಿಂದ ಈ ಗುರುತುಗಳನ್ನು ಮಾಡಿದ್ದರಂತೆ!

ಮತ್ತೊಮ್ಮೆ ಸ್ವಾಮೀಜಿ, ಹಲವಾರು ಕ್ಷೇತ್ರಗಳ ಸಂದರ್ಶನವನ್ನು ಮುಗಿಸಿಕೊಂಡು ಬಾರಾ ನಾಗೋರಿಗೆ ಹಿಂದಿರುಗಿದ್ದರು; ತಮ್ಮ ಗುರುಭಾಯಿಗಳೊಂದಿಗೆ ಧ್ಯಾನ-ಜಪ-ಅಧ್ಯಯನಾದಿ ಗಳಲ್ಲಿ ಮುಳುಗಿದ್ದರು. ಒಂದು ದಿನ ಅವರು ತಮ್ಮ ಕುಟುಂಬದವರಿಗೆಲ್ಲ ಸುಪರಿಚಿತನಾದ ವ್ಯಕ್ತಿಯೊಬ್ಬನನ್ನು ಭೇಟಿ ಮಾಡಿದರು. ಅವನು ಅಪರಾಧ ತನಿಖಾ ವಿಭಾಗದಲ್ಲಿ ಉನ್ನತ ಅಧಿಕಾರಿ. ತಮ್ಮ ನರೇಂದ್ರನನ್ನು ಕಂಡು ಆತ ಬಹಳ ವಿಶ್ವಾಸದಿಂದ ಮಾತನಾಡಿಸಿ, ಅಂದು ರಾತ್ರಿ ತನ್ನ ಮನೆಗೆ ಊಟಕ್ಕೆ ಬರಬೇಕೆಂದು ಆಹ್ವಾನಿಸಿದ. ಸ್ವಾಮೀಜಿ ಆತನ ಮನೆಗೆ ಹೋದಾಗ ಅಲ್ಲಿ ಇನ್ನೂ ಕೆಲವರು ಕುಳಿತಿದ್ದರು. ಸ್ವಲ್ಪ ಹೊತ್ತಿನಲ್ಲಿ ಎಲ್ಲರೂ ಒಬ್ಬೊಬ್ಬರಾಗಿ ಹೊರಟು ಹೋದರು. ಪೋಲೀಸ್ ಅಧಿಕಾರಿ ಸ್ವಾಮೀಜಿಯೊಂದಿಗೆ ಹಲವಾರು ವಿಷಯಗಳ ಬಗ್ಗೆ ಮಾತ ನಾಡಿದ; ಆದರೆ ಊಟದ ಸೂಚನೆಯೇ ಇಲ್ಲ! ಲೋಕಾಭಿರಾಮವಾಗಿ ಹರಟುತ್ತಿದ್ದ ಅಧಿಕಾರಿಯ ಮುಖಭಾವ ಇದ್ದಕ್ಕಿದ್ದಂತೆ ಬದಲಾಯಿತು. ಸ್ವಾಮೀಜಿಯನ್ನು ಕ್ರೂರ ದೃಷ್ಟಿಯಿಂದ ನೋಡುತ್ತ, ತಗ್ಗಿಸಿದ ದನಿಯಲ್ಲಿ ಮಾತನಾಡಿದ:

“ಈಗ, ಒಳ್ಳೇ ಮಾತಿನಲ್ಲಿ ನಿಜವನ್ನು ಒಪ್ಪಿಕೊಂಡುಬಿಡು. ಸುಮ್ಮನೆ ಕಥೆಕಟ್ಟಲು ನೋಡ ಬೇಡ. ನನಗೆ ನಿನ್ನ ವಿಷಯವೆಲ್ಲ ಗೊತ್ತಿಗೆ. ನೀನೂ ನಿನ್ನ ತಂಡದವರೂ ಸಾಧುಗಳಂತೆ ನಾಟಕ ವಾಡಿಕೊಂಡು ಸರ್ಕಾರದ ವಿರುದ್ಧ ಒಳಗೊಳಗೇ ಪಿತೂರಿ ಮಾಡುತ್ತಿರುವುದು ನನಗೆ ಗೊತ್ತಿಲ್ಲ ವೆಂದು ಭಾವಿಸಿದೆಯಾ?”

“ಏನು, ಏನು ನೀನು ಹೇಳುತ್ತಿರುವುದು! ಎಂಥ ಪಿತೂರಿ?” ಆಶ್ಚರ್ಯಾಘಾತಗೊಂಡು ಸ್ವಾಮೀಜಿ ಉದ್ಗರಿಸಿದರು.

“ಅದನ್ನೇ ನಾನೂ ಕೇಳಿದ್ದು. ನೀವೆಲ್ಲ ಸೇರಿ ಏನೋ ಕುತಂತ್ರ ನಡೆಸುತ್ತಿದ್ದೀರಿ ಎನ್ನುವುದಕ್ಕೆ ನನಗೆ ಖಚಿತವಾದ ಸೂಚನೆಗಳು ಸಿಕ್ಕಿವೆ. ನೀನಾಗಿಯೇ ಈಗ ಎಲ್ಲವನ್ನೂ ಇದ್ದದ್ದು ಇದ್ದಂತೆ ಹೇಳಿಬಿಟ್ಟರೆ ಕಡಿಮೆ ಶಿಕ್ಷೆಯಲ್ಲೇ ಮುಗಿದುಹೋಗುತ್ತದೆ....”

ಹಕ್ಕಿ ಬಿದ್ದರೆ ಬೀಳಲಿ, ಇಲ್ಲದಿದ್ದರೆ ಹೋಗಲಿ ಎಂದು ಆಲೋಚಿಸಿ ಆ ಅಧಿಕಾರಿ ಒಂದು ಕಲ್ಲೆಸೆದ ಎಂದು ತೋರುತ್ತದೆ. ತಾನು ಅನುಮಾನ ಪಟ್ಟದ್ದು ಒಂದು ವೇಳೆ ನಿಜವಾಗಿದ್ದರೆ, ತನಗೆ ಕೀರ್ತಿ, ಪದಕ ಎಲ್ಲ ಸಿಗುತ್ತದೆ. ಇಲ್ಲದಿದ್ದರೆ ನಷ್ಟವಂತೂ ಇಲ್ಲವಲ್ಲ! ಆದರೆ ಸ್ವಾಮೀಜಿಗೆ ಈ ಅಪಮಾನವನ್ನು ಸಹಿಸಿಕೊಳ್ಳಲಾಗಲಿಲ್ಲ.

“ನಿನಗೆ ಅಷ್ಟೊಂದು ವಿಷಯ ತಿಳಿದಿದ್ದ ಮೇಲೆ ನಮ್ಮ ಮೇಲೆ ಮೊಕದ್ದಮೆ ಹೂಡುವುದಿಲ್ಲ ವೇಕೆ? ಅಥವಾ ನಮ್ಮ ಮನೆಯನ್ನೇ ತನಿಖೆ ಮಾಡಲಿಲ್ಲವೇಕೆ?”

ಹೀಗೆನ್ನುತ್ತ ಮೆಲ್ಲನೆ ಕುಳಿತಲ್ಲಿಂದ ಎದ್ದುಹೋಗಿ ಬಾಗಿಲು ಮುಚ್ಚಿ ಅಗುಳಿ ಹಾಕಿದರು. ಸ್ವಾಮೀಜಿಯದು ಪೈಲವಾನನ ಮೈಕಟ್ಟು; ಪೋಲಿಸ್ ಅಧಿಕಾರಿಯೋ ನರಪೇತಲ ಆಸಾಮಿ. ಗುಳ್ಳೇನರಿ ಸಿಂಹವನ್ನು ಕೆಣಕಿದಂತಾಯಿತು. ಹತ್ತಿರ ಬಂದು ನಿಂತರು ಸ್ವಾಮೀಜಿ. ಅಧಿಕಾರಿ ನಡುಗಿದ. ಗಂಟಲು ಒಣಗಿತು. ನಾಲಿಗೆ ತೊದಲಿತು. “ನಾನು... ಅದು... ನಾ... ನಾ...” ಎಂದ. ಅವನೊಂದು ಕೆಟ್ಟ ಕ್ರಿಮಿಯೋ ಎಂಬಂತೆ ನೋಡಿ ಸ್ವಾಮೀಜಿ ಹೇಳಿದರು:

“ನೀನು ಒಂದು ಸುಳ್ಳು ಕಾರಣಕೊಟ್ಟು ನನ್ನನ್ನು ನಿನ್ನ ಮನೆಗೆ ಕರೆಸಿದೆ. ಬಳಿಕ, ನನ್ನ ಮೇಲೂ ನನ್ನ ಸ್ನೇಹಿತರ ಮೇಲೂ ಸುಳ್ಳು ಅಪಾದನೆ ಮಾಡಿದೆ. ಅದೇ ನಿನ್ನ ಬದುಕು. ಆದರೆ ನಾನು, ಅನ್ಯಾಯಕ್ಕೆ ಪ್ರತಿ ಅನ್ಯಾಯ ಮಾಡಬಾರದೆಂದು ಕಲಿತಿದ್ದೇನೆ. ನಾನು ನಿಜಕ್ಕೂ ಒಬ್ಬ ಅಪ ರಾಧಿಯೂ ಕುತಂತ್ರಿಯೂ ಆಗಿದ್ದರೆ, ಈಗ ನೀನು ಉಸಿರೆತ್ತುವುದರೊಳಗಾಗಿ ನಿನ್ನ ಕುತ್ತಿಗೆಯನ್ನು ಹಿಸುಕಿ ಹಾಕಬಹುದಾಗಿತ್ತು. ಹಾಗೆ ಮಾಡಿದರೆ ನನ್ನನ್ನು ತಡೆಯುವರಾರಿದ್ದಾರೆ? ಬಿಟ್ಟಿದ್ದೇನೆ, ಬದುಕಿಕೋ ಹೋಗು!”

ಹೀಗೆಂದು ಸ್ವಾಮೀಜಿ, ಬಾಗಿಲನ್ನು ತೆರೆದು, ಹಿಂದಿರುಗಿ ನೋಡದೆ ದಿಟ್ಟಹೆಜ್ಜೆಗಳನ್ನಿಡುತ್ತ ನೆಟ್ಟಗೆ ಅಲ್ಲಿಂದ ಹೊರಟರು. ಭೂತದರ್ಶನವಾದಂತೆ ಬಿದ್ದಿದ್ದ ಪೋಲೀಸನಿಗೆ ಚೇತರಿಸಿಕೊಳ್ಳಲು ಬಹಳ ಹೊತ್ತೇ ಬೇಕಾಯಿತು. ಮತ್ತೆಂದೂ ಆತ ಸ್ವಾಮೀಜಿ ಹಾಗೂ ಅವರ ಸಂಗಡಿಗರ ತಂಟೆಗೆ ಹೋಗಲಿಲ್ಲ.

ಅಲೌಕಿಕ ಶಕ್ತಿಯೊಂದು ಸ್ವಾಮೀಜಿಯ ಹಿಂದೆ ನಿರಂತರವಾಗಿ ಇದ್ದುಕೊಂಡು ಕೆಲಸ ಮಾಡುತ್ತಿತ್ತೆಂಬುದನ್ನು ತೋರಿಸುವ ಘಟನೆಗಳು ಅವರ ಪರಿವ್ರಾಜಕ ಜೀವನದಲ್ಲಿ ಹಲವಾರು. ಫಾಜೀಪುರದ ಬಳಿ ಇಂಥದೊಂದು ಘಟನೆ ನಡೆಯಿತು. ಯಾವುದೋ ಟ್ರೈನಿನ ಮೂರನೇ ದರ್ಜೆಯ ಬಂಡಿಯಲ್ಲಿ ಸ್ವಾಮೀಜಿ ಪ್ರಯಾಣ ಮಾಡುತ್ತಿದ್ದರು. ಅವರು ಟ್ರೈನು ಹತ್ತುವಾಗ, ಅವರ ಭಕ್ತರು ಸ್ವಲ್ಪ ಹಣವನ್ನೂ ತಿಂಡಿತೀರ್ಥಗಳ ಬುತ್ತಿಯನ್ನೂ ಕೊಡಲು ಮುಂದಾಗಿದ್ದರು. ಆದರೆ ಅಪರಿಗ್ರಹ-ಅಸಂಗ್ರಹ ವ್ರತ ತೊಟ್ಟಿದ್ದ ಸ್ವಾಮೀಜಿ ಅದಾವುದನ್ನೂ ಸ್ವೀಕರಿಸದೆ ಮುಂದಿನ ಪ್ರಯಾಣಕ್ಕೆ ಒಂದು ಟಿಕೆಟನ್ನು ಮಾತ್ರ ಪಡೆದುಕೊಂಡಿದ್ದರು.

ಸ್ವಾಮೀಜಿಯ ಜೊತೆಯಲ್ಲೇ ಒಬ್ಬ ಶ್ರೀಮಂತ ಬನಿಯಾ (ವ್ಯಾಪಾರಿ) ಪ್ರಯಾಣ ಮಾಡು ತ್ತಿದ್ದ. ಇವನಿಗೆ ಸಂನ್ಯಾಸಿಗಳೆಂದರೆ ಆಗುತ್ತಿರಲಿಲ್ಲ. ಈ ಸಂನ್ಯಾಸಿಗಳೆಲ್ಲ ದುಡಿದು ತಿನ್ನಲು ಮನಸ್ಸಿಲ್ಲದೆ ತಿರುಪೆಯೆತ್ತಿ ಜೀವನಯಾಪನೆ ಮಾಡುವ ಸೋಂಬೇರಿಗಳು ಎಂದು ಅವನು ನಂಬಿದ್ದ. ಈಗ ಎದುರಿಗೇ ಒಬ್ಬ ಸಂನ್ಯಾಸಿ ಸಿಕ್ಕಾಗ, ಅವರನ್ನು ಒಂದೇ ಸಮನೆ ಛೇಡಿಸಲಾರಂಭಿ ಸಿದ. ವಿಪರೀತ ಸೆಕೆಯಿಂದ ಸ್ವಾಮೀಜಿಗೆ ತುಂಬ ದಾಹವಾಗಿತ್ತು. ಆದರೆ ಅಲ್ಲಿ ನೀರನ್ನೂ ಹಣ ಕೊಟ್ಟು ಕೊಂಡುಕೊಳ್ಳಬೇಕಾಗಿತ್ತು. ಕೈಯಲ್ಲೊಂದು ಬಿಡಿಗಾಸೂ ಇಲ್ಲದ್ದರಿಂದ ಸ್ವಾಮೀಜಿ ನೀರಿಲ್ಲದೆ ಸುಮ್ಮನಿರಬೇಕಾಗಿಯಿತು. ಇದನ್ನು ಕಂಡ ಬನಿಯಾ ಇನ್ನಷ್ಟು ಸಂತೋಷಗೊಂಡ. ಪ್ರತಿ ನಿಲ್ದಾಣದಲ್ಲೂ ಆತ ಬೇಕಾದಷ್ಟು ನೀರು ಕೊಂಡುಕೊಂಡು ಕುಡಿಯುತ್ತಿದ್ದ. ನೀರು ಕುಡಿ ಯುತ್ತ, ಸ್ವಾಮೀಜಿಯನ್ನು ಇನ್ನಷ್ಟು ಅಪಹಾಸ್ಯ ಮಾಡಿದ: “ಏನೋ, ನೋಡಿಲ್ಲಿ. ಈ ನೀರು ಎಷ್ಟು ಚೆನ್ನಾಗಿದೆ ಗೊತ್ತಾ! ಆಹ್! ಪಾಪ, ನೀನಾದರೋ ಸಂನ್ಯಾಸಿ; ಎಲ್ಲವನ್ನೂ ಬಿಟ್ಟವನು. ಈಗ? ನೀರು ಕುಡಿಯಲೂ ಅದೃಷ್ಟವಿಲ್ಲ! ಅನುಭವಿಸು, ನಿನ್ನ ಕರ್ಮ. ಅದೇ, ನನ್ನನ್ನು ನೋಡು! ಕೈತುಂಬ ಸಂಪಾದಿಸುತ್ತೇನೆ, ಹೊಟ್ಟೆ ತುಂಬ ತಿನ್ನುತ್ತೇನೆ. ಕುಡಿಯುತ್ತೇನೆ! ನೀನೂ ಯಾಕೆ ನನ್ನಂತೆಯೇ ಸಂಪಾದನೆ ಮಾಡಿ ಸುಖವಾಗಿರಬಾರದಿತ್ತು?”

ಹೀಗೇ ಸಾಗಿತು ಅವನ ಉಪದೇಶ. ಅವನು ಹೇಳುತ್ತಿರುವುದು ತಮಗೆ ಅರ್ಥವೇ ಆಗುತ್ತಿಲ್ಲ ವೇನೋ ಎಂಬಂತಿತ್ತು ಸ್ವಾಮೀಜಿಯ ಮುಖಭಾವ. ಆದರೂ ಆ ಬನಿಯಾ ತನ್ನ ಕೆಲಸವನ್ನು ನಿಲ್ಲಿಸಲೇ ಇಲ್ಲ. ರೈಲು ರಾತ್ರಿಯೆಲ್ಲ ಓಡಿ, ಮರುದಿನ ಮಧ್ಯಾಹ್ನ ತಾರಿಘಾಟ್ ನಿಲ್ದಾಣವನ್ನು ಸೇರಿತು. ಸ್ವಾಮೀಜಿಗೆ ಹಿಂದಿನ ದಿನದಿಂದಲೂ ಹೊಟ್ಟೆಗೆ ಏನೇನೂ ಬಿದ್ದಿರಲಿಲ್ಲ. ಸುಡುವ ಬಿಸಿಲು ಬೇರೆ. ಇಲ್ಲಿ ಅವರು ಟ್ರೈನು ಬದಲಾಯಿಸಬೇಕಿತ್ತೆಂದು ಕಾಣುತ್ತದೆ; ಆದ್ದರಿಂದ ಟ್ರೈನಿನಿಂದಿಳಿದು ರೈಲುನಿಲ್ದಾಣದ ಷೆಡ್ಡಿನಲ್ಲಿ ಕುಳಿತುಕೊಂಡರು. ಆದರೆ ಕೂಲಿಯಾಳೊಬ್ಬ ಅವರನ್ನು ಅಲ್ಲಿ ಕುಳಿತುಕೊಳ್ಳಲೂ ಬಿಡಲಿಲ್ಲ. ಸರಿ, ಅಲ್ಲೇ ಒಂದು ಕಂಬಕ್ಕೆ ಒರಗಿಕೊಂಡು ಬಿಸಿಲಿನಲ್ಲೇ ಕುಳಿತರು ಸ್ವಾಮೀಜಿ.

ಬೆಂಬಿಡದ ಭೇತಾಳದಂತೆ ಇಲ್ಲಿಗೂ ಬಂದ ಬನಿಯಾ! ಅಲ್ಲೇ ಒಂದು ಒಳ್ಳೇ ಜಾಗವನ್ನು ಹುಡುಕಿ ತನ್ನ ಜಮಖಾನವನ್ನು ಹಾಸಿದ. ಅದರ ಮೇಲೆ ಕುಳಿತು, ತನ್ನ ಗಂಟನ್ನು ಬಿಚ್ಚಿ ಒಂದೊಂದಾಗಿ ತಿಂಡಿಗಳನ್ನು ತೆಗೆದು ಹರಡಿದ. ರುಚಿಕರವಾದ ತಿಂಡಿಗಳನ್ನು ಚಪ್ಪರಿಸಿ ಚಪ್ಪರಿಸಿ ತಿನ್ನುತ್ತ ಮತ್ತೆ ಪ್ರಾರಂಭಿಸಿದ ತನ್ನ ಉಪದೇಶಾಮೃತವನ್ನು: “ಏನಯ್ಯ? ಹೇಗಿದೆ ನೋಡಿದೆಯಾ ಈ ಪೂರಿಗಳು, ಲಾಡುಗಳು! ಪಾಪ! ನಿನಗೆ ಕಷ್ಟಪಟ್ಟು ದುಡಿದು ಹಣ ಸಂಪಾದನೆ ಮಾಡಲು ಇಷ್ಟವಿಲ್ಲ. ನಿನ್ನಂಥವರಿಗೆ ಇದೇ ಗತಿ. ಹಹ್ಹಹ್ಹ! ಹೋಗಲಿ, ನಾನು ತಿನ್ನುವುದನ್ನಾದರೂ ಕಣ್ತುಂಬ ನೋಡಿ ಹೊಟ್ಟೆ ತುಂಬಿಸಿಕೊ.” ಬನಿಯಾ ಇಷ್ಟು ದುಷ್ಟನಾಗಿ ವರ್ತಿಸಿದರೂ, ಸ್ವಾಮೀಜಿಯ ಮುಖದ ಒಂದು ನರವೂ ಮಿಸುಕಲಿಲ್ಲ.

ಈ ಸಮಯಕ್ಕೆ ಸರಿಯಾಗಿ ಅದೇ ಊರಿನವನೊಬ್ಬ ಅವಸರವಸರವಾಗಿ ಅಲ್ಲಿಗೆ ಬಂದ. ಅವನ ಬಲಗೈಯಲ್ಲಿ ಒಂದು ದೊಡ್ಡ ಗಂಟು, ಒಂದು ಬಟ್ಟಲು. ಎಡಗೈಯಲ್ಲಿ ಮಣ್ಣಿನ ಹೂಜಿಯನ್ನೂ ಬಗಲಲ್ಲೊಂದು ಚಾಪೆಯನ್ನೂ ಹಿಡಿದಿದ್ದ. ಬಂದವನೇ, ಒಂದು ಶುಭ್ರವಾದ ಜಾಗದಲ್ಲಿ ಚಾಪೆ ಯನ್ನು ಹಾಸಿ, ತಾನು ತಂದದ್ದನ್ನೆಲ್ಲ ಅದರ ಮುಂದೆ ಹರಡಿ ಸ್ವಾಮೀಜಿಯನ್ನು ಕರೆದ:

“ಹ್ಞೂಂ, ಬನ್ನಿ, ಬನ್ನಿ ಬಾಬಾಜಿ. ಈ ತಿಂಡಿಗಳನ್ನೆಲ್ಲ ತೆಗೆದುಕೊಳ್ಳಿ. ಬಿಸಿಬಿಸಿಯಾಗಿರು ವಾಗಲೇ ಮುಗಿಸಿಬಿಡಿ.”

ಬನಿಯಾ ಇದನ್ನು ನೋಡಿದ. ಅವನ ಮುಖದ ಕಳೆಯೇ ಇಂಗಿಹೋಯಿತು. ಕಣ್ಣು ಬಾಯಿ ತೆರೆದುಕೊಂಡು ನೋಡಲಾರಂಭಿಸಿದ.

ಸ್ವಾಮೀಜಿಗೂ ಆಶ್ಚರ್ಯವಾಯಿತು. ಯಾರನ್ನು ಈತ ಕರೆಯುತ್ತಿರುವುದು?... ತಮ್ಮನ್ನೇ!

“ಅಪ್ಪಾ, ನೀನು ನನ್ನನ್ನು ಬೇರೆ ಯಾರೋ ಎಂದು ತಪ್ಪು ತಿಳಿದಿದ್ದೀಯೆ. ನಾನು ನಿನ್ನನ್ನು ಈ ಹಿಂದೆ ಎಂದೂ ನೋಡಿದ ನೆನಪಿಲ್ಲ.”

“ಇಲ್ಲ ಇಲ್ಲ ಬಾಬಾಜಿ, ನಾನು ನಿಮ್ಮನ್ನೇ ನೋಡಿದ್ದು. ಹ್ಞೂ, ಬೇಗ ಬನ್ನಿ!”

“ನೀನು ನನ್ನನ್ನು ನೋಡಿದೆಯಾ! ಎಲ್ಲಿ, ಯಾವಾಗ?”

“ಬಾಬಾಜಿ, ನಾನೊಬ್ಬ ಸಿಹಿತಿಂಡಿ ವ್ಯಾಪಾರಿ. ಇವತ್ತು ಬೆಳಿಗ್ಗೆ ಊಟವಾದ ಮೇಲೆ, ಸ್ವಲ್ಪ ನಿದ್ರೆ ಮಾಡೋಣ ಅಂತ ಮಲಗಿದ್ದೆ. ಆಗ ನನಗೆ ಕನಸಿನಲ್ಲಿ ಶ್ರೀರಾಮಚಂದ್ರ ಕಾಣಿಸಿಕೊಂಡು ನಿಮ್ಮನ್ನು ತೋರಿಸಿ, ‘ನೋಡು ನೋಡು, ಈತ ನನ್ನ ಭಕ್ತ. ಮೊನ್ನೆಯಿಂದ ತುಂಬ ಹಸಿದುಕೊಂಡಿ ದ್ದಾನೆ. ನನಗೆ ಬಹಳ ದುಃಖವಾಗುತ್ತಿದೆ. ನೀನು ಈ ಕ್ಷಣವೇ ಪೂರಿ-ಪಲ್ಯ ತಯಾರಿಸಿ, ಸಿಹಿತಿಂಡಿ, ಒಳ್ಳೇ ತಂಪು ನೀರು, ಎಲ್ಲವನ್ನೂ ತೆಗೆದುಕೊಂಡು ರೈಲು ನಿಲ್ದಾಣಕ್ಕೆ ಹೋಗು’ ಅಂತ ಸ್ಪಷ್ಟವಾಗಿ ಹೇಳಿದ. ನನಗೆ ಎಚ್ಚರವಾಯಿತು. ಆದರೆ ಇದೇನೋ ವಿಚಿತ್ರ ಕನಸು ಎಂದು ಭಾವಿಸಿ, ನಾನು ಮಗ್ಗುಲು ಬದಲಾಯಿಸಿ ಮಲಗಿಬಿಟ್ಟೆ. ಆದರೆ, ಬಾಬಾಜಿ, ಪರಮ ದಯಾಳುವಾದ ರಾಮಚಂದ್ರ, ನನ್ನ ಮೇಲೆ ಕೃಪೆದೋರಿ ಮತ್ತೆ ಕಾಣಿಸಿಕೊಂಡ! ಹಾಸಿಗೆಯಿಂದ ನನ್ನನ್ನು ನಿಜಕ್ಕೂ ನೂಕಿ, ‘ಏಳು ಮೇಲೇಳು, ಬೇಗ ನಾನು ಹೇಳಿದ್ದನ್ನು ಮಾಡು’ ಎಂದು ಅವಸರಪಡಿಸಿದ. ನಾನು ತಕ್ಷಣ ಮೇಲೆದ್ದು ಸ್ವಲ್ಪ ಪೂರಿ-ಪಲ್ಯ ತಯಾರಿಸಿ, ಇವತ್ತು ಬೆಳಿಗ್ಗೆ ತಾನೇ ಮಾಡಿದ್ದ ಸಿಹಿತಿಂಡಿಯನ್ನೂ ತೆಗೆದುಕೊಂಡು ಓಡಿ ಬಂದೆ. ದೂರದಿಂದಲೇ ನಿಮ್ಮನ್ನು ಗುರುತಿಸಿಬಿಟ್ಟೆ! ಇನ್ನು ತಡಮಾಡದೆ ಇದನ್ನು ಸ್ವೀಕರಿಸಿ ಬಾಬಾಜಿ. ಪಾಪ, ನಿಮಗೆ ಬಹಳ ಹಸಿವೆಯಾಗಿರಬೇಕು.”

ಸ್ವಾಮೀಜಿಯ ಕಂಗಳಿಂದ ಅಶ್ರುಧಾರೆ ಚಿಮ್ಮಿತು. ಗದ್ಗದ ಸ್ವರದಿಂದ ಸರಳ ಹೃದಯದ ಈ ಅತಿಥೇಯನಿಗೆ ತಮ್ಮ ಧನ್ಯವಾದ ಹೇಳಿದರು. ಆದರೆ ಆತ ಅದನ್ನು ಪ್ರತಿಭಟಿಸುತ್ತ, “ಅಯ್ಯೋ, ನನಗೇಕೆ ಧನ್ಯವಾದ! ಎಲ್ಲವೂ ಶ್ರೀರಾಮನ ಕೃಪೆ!” ಎಂದ. ಇದನ್ನೆಲ್ಲ ನೋಡುತ್ತಿದ್ದ ಬನಿಯಾ ಆಶ್ಚರ್ಯಾಘಾತದಿಂದ ಸ್ತಬ್ಧನಾದ. ಬಳಿಕ ಸ್ವಾಮೀಜಿಯ ಪಾದಕ್ಕೆ ಬಿದ್ದು, “ಕ್ಷಮಿಸಿಬಿಡಿ ಬಾಬಾಜಿ, ನನ್ನನ್ನು ಕ್ಷಮಿಸಿಬಿಡಿ” ಎಂದು ಬೇಡಿಕೊಂಡ.

ಸ್ವಾಮೀಜಿಗೆ ಶ್ರೀರಾಮಕೃಷ್ಣರು ಉಪದೇಶಿದ ಮಂತ್ರ ರಾಮಮಂತ್ರ. ರಾಮನೇ ಅವರ ಇಷ್ಟದೇವತೆ. ಇಂತಹ ಅಸಾಹಯಕ ಪರಿಸ್ಥಿತಿಯಲ್ಲಿ, ಶ್ರೀರಾಮ ತನ್ನ ನಂಬಿದವರನ್ನು ಕಾಪಾಡಿದ ಮಾರ್ಮಿಕ ಘಟನೆಯಿದು.

ಸ್ವಾಮೀಜಿ ಅದೇ ಪ್ರಾಂತದಲ್ಲಿ ಪ್ರಯಾಣ ಮಾಡುತ್ತಿದ್ದಾಗ ನಡೆದ ಇನ್ನೊಂದು ಘಟನೆ. ಒಮ್ಮೆ ಅವರು ಕುಳಿತಿದ್ದ ಬೋಗಿಯಲ್ಲೇ ಇಬ್ಬರು ಬಿಳಿಯರೂ ಇದ್ದರು. ಅವರು ಸ್ವಾಮೀಜಿ ಯನ್ನು ಕಂಡು ಇವನ್ಯಾರೋ ಒಬ್ಬ ಗಾಂಪನಿರಬೇಕೆಂದು ಊಹಿಸಿ ಅವರ ವಿಷಯವಾಗಿ ಇಂಗ್ಲಿಷಿ ನಲ್ಲಿ ಮಾತನಾಡಿಕೊಳ್ಳುತ್ತ ಅಪಹಾಸ್ಯ ಮಾಡಲಾರಂಭಿಸಿದರು. ಅವರ ಮಾತುಗಳ ಒಂದು ಪದ ಕೂಡ ತಮಗೆ ಅರ್ಥವಾಗಲಿಲ್ಲವೇನೋ ಎಂಬಂತೆ ಸುಮ್ಮನೆ ಕುಳಿತಿದ್ದರು ಸ್ವಾಮೀಜಿ. ರೈಲು ಒಂದು ನಿಲ್ದಾಣಕ್ಕೆ ಬಂದು ನಿಂತಿತು. ಸ್ವಾಮೀಜಿಗೆ ಕುಡಿಯಲು ನೀರು ಬೇಕಾಗಿತ್ತು. ಅಲ್ಲೇ ಇದ್ದ ಸ್ಟೇಶನ್ ಮಾಸ್ಟರರ ಹತ್ತಿರ, “ಇಲ್ಲಿ ಕುಡಿಯಲು ನೀರು ಸಿಗಬಹುದೆ?” ಎಂದು ಇಂಗ್ಲಿಷಿನಲ್ಲಿ ಕೇಳಿದರು. ತಕ್ಷಣ ಆ ಇಬ್ಬರು ಬಿಳಿಯರು ಮುಖ ಮುಖ ನೋಡಿಕೊಂಡರು–“ಎಲಾ, ಈತನಿಗೆ ಇಂಗ್ಲಿಷ್ ಬರುತ್ತದೆ!” ಇಲ್ಲಿಯವರೆಗೆ ತಾವು, ಅವರಿಗೆ ಇಂಗ್ಲಿಷ್ ಬರುವುದಿಲ್ಲವೆಂದು ತಿಳಿದು ಕೊಂಡು ಅಪಹಾಸ್ಯ ಮಾಡಿ ಮಾತನಾಡಿಕೊಂಡದ್ದನ್ನೆಲ್ಲ ಅವರು ಅರ್ಥಮಾಡಿಕೊಂಡುಬಿಟ್ಟಿ ದ್ದಾರೆ ಎಂದು ಗೊತ್ತಾದಾಗ ತಬ್ಬಿಬ್ಬಾದರು. ಕ್ಷಮೆ ಕೋರುವ ದನಿಯಲ್ಲಿ ಸ್ವಾಮೀಜಿಯನ್ನು ಕೇಳಿದರು: “ನಮ್ಮ ಮಾತು ನಿಮಗೆ ಅರ್ಥವಾಗುತ್ತಿದ್ದರೂ ನೀವೇಕೆ ಪ್ರತಿಭಟಿಸದೆ ಸುಮ್ಮನೆ ಕುಳಿತಿದ್ದಿರಿ?”

ಕಿಟಕಿಯಾಚೆ ನೋಡುತ್ತ ಸ್ವಾಮೀಜಿ ಶಾಂತವಾಗಿ ಹೇಳಿದರು: “ನಾನು ಮೂರ್ಖರನ್ನು ನೋಡುತ್ತಿರುವುದು ಇದು ಮೊದಲ ಸಲವೇನಲ್ಲವಲ್ಲ.”

ಈ ಖಾರದ ಉತ್ತರವನ್ನು ಕೇಳಿ ಆ ಇಂಗ್ಲಿಷರಿಗೆ ಕೋಪ ಬಂದರೂ ಸ್ವಾಮೀಜಿಯ ಕಟ್ಟು ಮಸ್ತಾದ ಮೈಕಟ್ಟನ್ನೂ ಅವರ ಕಣ್ಣುಗಳಲ್ಲಿ ಸೂಸುತ್ತಿದ್ದ ಆತ್ಮಶಕ್ತಿಯನ್ನೂ ಕಂಡು ತಣ್ಣಗಾಗಿ ಕ್ಷಮೆ ಕೇಳಿದರು.

ಸ್ವಾಮೀಜಿಗೆ ಈ ದೀರ್ಘ ರೈಲು ಪ್ರಯಾಣಗಳ ಅವಧಿಯಲ್ಲಿ ಎಷ್ಟೆಷ್ಟೋ ಹೊಸ ಅನುಭವ ಗಳಾಗುತ್ತಿದ್ದುವು. ಜನರ ವಿಚಿತ್ರ ರೀತಿನೀತಿಗಳು, ಮಾತುಕತೆಗಳು ಅವರಿಗೆ ಮನರಂಜನೆ ಯನ್ನೊದಗಿಸುತ್ತಿದ್ದುವು. ಕೆಲವೊಮ್ಮೆ ಇಂತಹ ಸಂದರ್ಭಗಳಲ್ಲಿ ಅವರು ಬಾಲಕ ನರೇಂದ್ರನೇ ಆಗಿಬಿಡುತ್ತಿದ್ದರು. ಒಂದು ಸಲ ಯಾವುದೋ ದೀರ್ಘ ಪ್ರಯಾಣದ ಸಮಯದಲ್ಲಿ ಒಬ್ಬ ತರುಣ ಅವರಿಗೆ ಗಂಟುಬಿದ್ದಿದ್ದ. ಇವನಿಗೆ ವಾಮಾಚಾರ ಮೊದಲಾದುವುಗಳಲ್ಲಿ ಬಹಳ ಆಸಕ್ತಿ. ಬೋಗಿಯ ಒಂದು ಮೂಲೆಯಲ್ಲಿ ಪದ್ಮಾಸನ ಹಾಕಿಕೊಂಡು ಕುಳಿತಿದ್ದ ಸ್ವಾಮೀಜಿಯನ್ನು ಕಂಡು ಆಕರ್ಷಿತನಾಗಿ, ಹತ್ತಿರ ಬಂದು ಮಾತಿಗಾರಂಭಿಸಿದ. ಅಷ್ಟರಲ್ಲಿ ಅವರು ಈ ತರುಣನ ಹತ್ತಿರ ಒಂದು ದೊಡ್ಡ ಊಟದ ಡಬ್ಬಿಯಿರುವುದನ್ನು ಗಮನಿಸಿದರು. ಬೆಳಗಿನಿಂದ ಅವರಿಗೆ ತಿನ್ನಲು ಏನೂ ಸಿಕ್ಕಿರಲಿಲ್ಲ. ಸಹ ಪ್ರಯಾಣಿಕ ಕೇಳಿದ:

“ಸ್ವಾಮೀಜಿ, ನೀವು ಹಿಮಾಲಯಕ್ಕೆ ಹೋಗಿದ್ದೀರಾ?”

“ಓಹೊ, ಹೋಗಿದ್ದೇನೆ. ಅಷ್ಟೇ ಅಲ್ಲ, ಅಲ್ಲಿ ನಾನು ಬಹಳ ಕಾಲ ವಾಸಿಸಿದ್ದೇನೆ. ಅಲ್ಲಿನ ಒಂದೊಂದು ಶಿಖರವೂ ನನಗೆ ಚೆನ್ನಾಗಿ ಗೊತ್ತು.”

“ಹೌದೆ! ಹಾಗಾದರೆ ನೀವಲ್ಲಿ ಹಲವಾರು ಪವಾಡಪುರುಷರನ್ನು ಕಂಡಿರಬೇಕು! ಅವರು ಮಾಡುವ ಅದ್ಭುತ ಪವಾಡಗಳನ್ನೇನಾದರೂ ನೋಡಿದ್ದೀರಾ? ನನಗಂತೂ ಇದರಲ್ಲೆಲ್ಲ ತುಂಬ ಆಸಕ್ತಿ!”

ನಿಜವಾದ ಆಧ್ಯಾತ್ಮಿಕ ಜೀವನಕ್ಕೂ ಪವಾಡಗಳಿಗೂ ಸಂಬಂಧವಿಲ್ಲ ಎಂಬುದು ಈ ಮುಗ್ಧ ತರುಣನಿಗೆ ತಿಳಿಯದು. ಪವಾಡ ಮಾಡುವವರೇ ಮಹಾಮಹಿಮರು ಎಂದು ನಂಬಿದ್ದಾನೆ. ಈತನಿಗೆ ಅತೀಂದ್ರಿಯ ಶಕ್ತಿಗಳ ಮೇಲಿರುವ ಭ್ರಾಂತಿಯನ್ನು ಹೋಗಲಾಡಿಸಬೇಕೆಂದು ನಿಶ್ಚಯಿಸಿ ದರು ಸ್ವಾಮೀಜಿ. ಆದರೆ ಮೊದಮೊದಲು ಒಳ್ಳೇ ತಮಾಷೆಯಾಗಿ ಆ ತರುಣನ ಪ್ರಶ್ನೆಗಳಿಗೆ ತುಂಬ ರಸವತ್ತಾದ ಉತ್ತರಗಳನ್ನು ಕೊಡಲಾರಂಭಿಸಿದರು.

“ನಾನು ಹಿಮಾಲಯದಲ್ಲಿ ಅನೇಕ ಮಹಾತ್ಮರನ್ನು ಭೇಟಿಯಾಗಿದ್ದೇನೆ. ಏನವರ ಶಕ್ತಿಗಳು! ಏನವರ ತೇಜಸ್ಸು! ಅವರೆಲ್ಲ ಯಾವಾಗಲೂ ಅದ್ಭುತವಾದ ಸಾಧನೆಗಳಲ್ಲಿ ತೊಡಗಿರುತ್ತಾರೆ. ಅವರು ಕಣ್ಣಿಗೆ ಬೀಳುವುದೇ ಕಷ್ಟ. ಆದರೆ ಅವರನ್ನು ಒಲಿಸಿಕೊಂಡು ಮಾತನಾಡಿಸಿದರೆ ಎಷ್ಟೋ ವಿಷಯಗಳನ್ನು ತಿಳಿದುಕೊಳ್ಳಬಹುದು...”

ಸಹಪ್ರಯಾಣಿಕನ ಆನಂದಾಶ್ಚರ್ಯಗಳಿಗೆ ಪಾರವೇ ಇಲ್ಲ. ಅವನ ಕುತೂಹಲ ಭುಗಿಲೆದ್ದು ಬಗೆಬಗೆಯ ಪ್ರಶ್ನೆಗಳನ್ನು ಕೇಳಲಾರಂಭಿಸಿದ:

“ತಾವು ಭೇಟಿ ಮಾಡಿದ ಮಹಾತ್ಮರು ಈ ಯುಗದ ಕಾಲವಧಿಯ ಬಗ್ಗೆ ಏನಾದರೂ ಹೇಳಿದರೆ? ಅವರಿಗೆ ಈ ವಿಷಯವೆಲ್ಲ ಚೆನ್ನಾಗಿ ತಿಳಿದಿರುತ್ತದೆ ಅಲ್ಲವೆ? ಕಲಿಯುಗ ಇನ್ನೇನು ಮುಗಿಯುತ್ತ ಬಂದಿರಬೇಕು. ಹೇಗಿರಬಹುದೋ ಸತ್ಯಯುಗ!”

“ಹೌದು ಹೌದು! ಈ ಬಗ್ಗೆ ನಾನು ಅವರೊಂದಿಗೆ ವಿವರವಾಗಿ ಚರ್ಚಿಸಿದೆ. ಅವರು ಇವುಗಳ ಬಗ್ಗೆ ಬಹಳ ಚೆನ್ನಾಗಿ ಹೇಳಿದರು” ಎಂದು ಸ್ವಾಮೀಜಿ, ಆ ಮಾತುಕತೆಯನ್ನು ಸವಿವರವಾಗಿ ಕಣ್ಣಿಗೆ ಕಟ್ಟುವಂತೆ ಬಣ್ಣಿಸಲಾರಂಭಿಸಿದರು. ಅತ್ತ ರೈಲು ಓಡುತ್ತಿದ್ದಂತೆ, ಸ್ವಾಮೀಜಿಯ ಕಥೆಗಳೂ ಮುಂದುವರಿದುವು. ಅವರು ಹೇಳುವುದನ್ನೇ ಕಿವಿ-ಕಣ್ಣು ಅಗಲಿಸಿಕೊಂಡು ಕೇಳಿದ ತರುಣ ಅವರಾಡಿದ ಪ್ರತಿಯೊಂದು ಮಾತನ್ನೂ ನಂಬಿದ. ಕೊನೆಗೆ ಸ್ವಾಮೀಜಿ ತಮ್ಮ ಮಾತನ್ನು ನಿಲ್ಲಿಸಿದಾಗ ತನಗೆ ಪ್ರಿಯವಾದ ವಿಷಯದ ಬಗ್ಗೆ ಇಷ್ಟೆಲ್ಲ ಹೊಸ ವಿಚಾರಗಳನ್ನು ತಿಳಿಸಿಕೊಟ್ಟ ಅವರ ಬಗ್ಗೆ ತುಂಬ ಸಂತೋಷ ತಾಳಿ, ತನ್ನೊಡನೆ ಅವರನ್ನು ಊಟಕ್ಕೆ ಕರೆದ. ಬೇಡ ಎನ್ನಲಿಲ್ಲ ಸ್ವಾಮೀಜಿ.

ಊಟವಾಯಿತು. ಕೈತೊಳೆದು ಬಂದು ಯಥಾಸ್ಥಾನದಲ್ಲಿ ಕುಳಿತರು ಸ್ವಾಮೀಜಿ. ಈ ಮನುಷ್ಯನ ಮನಸ್ಸು ಒಳ್ಳೆಯದಾದರೂ, ಹಿಂದು ಮುಂದು ನೋಡದೆ ಎಲ್ಲವನ್ನೂ ನಂಬುವ ಪ್ರವೃತ್ತಿ ಯಿಂದಾಗಿ ಈತ ತಪ್ಪುದಾರಿ ಹಿಡಿದಿದ್ದಾನೆ ಎಂದು ಅವರಿಗನ್ನಿಸಿತು. ಇವನಿಗೆ ತಿಳಿಯಹೇಳಿದರೆ ತಿದ್ದಿಕೊಂಡಾನು ಎಂದು ಊಹಿಸಿದರು. ಸ್ವಲ್ಪ ಹೊತ್ತು ಸುಮ್ಮನೆ ಕುಳಿತಿದ್ದು, ತಾವೇ ಸಂಭಾಷಣೆಯನ್ನು ಪುನರಾರಂಭಿಸಿದರು:

“ಅಲ್ಲಾ, ನೀನು ಅಷ್ಟೊಂದು ತಿಳಿದವನು, ಅದು ಗೊತ್ತು-ಇದು ಗೊತ್ತು ಎಂದು ಜಂಬ ಕೊಚ್ಚಿಕೊಳ್ಳುತ್ತಿದ್ದವನು, ಈಗ ನಾನು ಹೇಳಿದ ಈ ತಲೆಬುಡವಿಲ್ಲದ ಕತೆಗಳನ್ನು ಅದು ಹೇಗೆ ಸ್ವಲ್ಪವೂ ಅನುಮಾನ ಪಡದೆ ಗುಳಕ್ ಅಂತ ನುಂಗಿಬಿಟ್ಟೆ?”

ಆ ತರುಣ ದಿಗ್ಭ್ರಾಂತನಾದ.

“ಏನು! ನೀವು ಇಷ್ಟು ಹೊತ್ತೂ ಹೇಳಿದ್ದೆಲ್ಲ ಸುಳ್ಳೇ ಹಾಗಾದರೆ?”

“ನಿನ್ನನ್ನು ನೋಡಿದರೆ ಬುದ್ಧಿವಂತನಂತೆ ಕಾಣುತ್ತೀಯೆ. ನಿನ್ನಂತಹ ವ್ಯಕ್ತಿ ಎಲ್ಲವನ್ನೂ ವಿಚಾರ ಮಾಡಿ, ಬುದ್ಧಿಯ ಒರೆಗಲ್ಲಿಗೆ ತಿಕ್ಕಿ ನೋಡಬೇಕು. ವಿವೇಕವನ್ನು ಉಪಯೋಗಿಸಬೇಕು. ಈ ಪವಾಡಗಳನ್ನು ಪವಾಡಪುರುಷರನ್ನು ನಂಬಿ ಕುಳಿತಿದ್ದೀಯಲ್ಲ, ನೀನು ಏನಾಶ್ಚರ್ಯ!”

ನಾಚಿಕೆಯಿಂದ ತಲೆತಗ್ಗಿಸಿದ ಆ ಯುವಕ. ಸ್ವಾಮೀಜಿ ಮುಂದುವರಿಸಿದರು:

“ನೋಡು, ಈ ಮಾಯಾಮಂತ್ರಗಳಿಗೂ ಆಧ್ಯಾತ್ಮಿಕತೆಗೂ ಏನೇನೂ ಸಂಬಂಧವಿಲ್ಲ. ನಿಜವಾದ ಆಧ್ಯಾತ್ಮಿಕ ಜೀವನದಿಂದ ಮಾತ್ರ ನಿನಗೆ ಶ್ರೇಯಸ್ಸು. ಕೇವಲ ಪವಾಡಗಳನ್ನು ಮಾಡಿ ತೋರಿಸುವುದರಲ್ಲೇ ಮುಳುಗಿರುವವರನ್ನು ಪರೀಕ್ಷೆಮಾಡಿ ನೋಡಿದಾಗ ಗೊತ್ತಾಗುತ್ತದೆ, ಅವ ರೆಲ್ಲ ಆಸೆಗಳ ದಾಸರು, ಹೆಸರು-ಕೀರ್ತಿಗಳ ಬೆನ್ನಟ್ಟುವವರು ಎಂದು. ಶೀಲಬಲವನ್ನು ಕಾಪಾಡಿ ಕೊಳ್ಳುವುದರಲ್ಲೇ ನಿಜವಾದ ಆಧ್ಯಾತ್ಮಿಕತೆಯಿರುವುದು. ಎಲ್ಲ ಆಸೆಗಳನ್ನೂ ಕಾಮಭಾವನೆ ಗಳನ್ನೂ ನಿರ್ಮೂಲನ ಮಾಡುವುದರಲ್ಲೇ ಆಧ್ಯಾತ್ಮಿಕತೆ ಇರುವುದು. ಯಾವ ಈ ಮಂಕುಗೊಳಿ ಸುವ ಪವಾಡಗಳು ನಮ್ಮ ಜೀವನದ ಮಹತ್ತರ ಸಮಸ್ಯೆಗಳನ್ನು ಬಗೆಹರಿಸುವಲ್ಲಿ ಸಂಪೂರ್ಣ ನಿರರ್ಥಕವೋ ಅಂಥವುಗಳ ಹಿಂದೋಡುವುದೆಂದರೆ ಭಯಂಕರ ಶಕ್ತಿಹ್ರಾಸ. ಅವು ಮನುಷ್ಯನನ್ನು ಕಡು ಸ್ವಾರ್ಥಿಯನ್ನಾಗಿಸಿ ಅಧಃಪತನದೆಡೆಗೆ ಎಳೆದೊಯ್ಯುತ್ತವೆ. ನಮ್ಮ ರಾಷ್ಟ್ರವನ್ನು ನೀತಿ ಭ್ರಷ್ಟಗೊಳಿಸಿರುವುದು ಇಂತಿಂತಹ ಅವಿವೇಕಗಳೆ. ನಮಗೀಗ ಬೇಕಾಗಿರುವುದು ಪ್ರಬಲ ವಿವೇಕ ಪ್ರಜ್ಞೆ, ಸಾಮಾಜಿಕ ಪ್ರಜ್ಞೆ ಮತ್ತು ನಮ್ಮನ್ನು ಪುರುಷಸಿಂಹರನ್ನಾಗಿಸಬಲ್ಲ ತತ್ತ್ವಶಾಸ್ತ್ರ ಹಾಗೂ ಧರ್ಮ.”

ಸ್ವಾಮೀಜಿಯ ಮಾತುಗಳನ್ನು ಗಮನವಿಟ್ಟು ಕೇಳಿದ ಆ ತರುಣ ಅವರ ಮಾತಿನ ತಥ್ಯವನ್ನು ಮನಗಂಡ. ಅಲ್ಲದೆ ಇಲ್ಲಿಯವರೆಗೆ ತನ್ನಲ್ಲಿ ಮನೆಮಾಡಿದ್ದ ತಪ್ಪು ತಿಳಿವಳಿಕೆಯನ್ನು ಕಂಡು ಕೊಂಡ. ಇನ್ನುಮುಂದೆ ತಾನು ಅವರ ಮಾತಿನಂತೆಯೇ ನಡೆಯುವುದಾಗಿ ಮಾತುಕೊಟ್ಟ.

ಮುಂದೊಮ್ಮೆ, ತಮ್ಮ ಪರಿವ್ರಾಜಕ ದಿನಗಳ ಬಗ್ಗೆ ಗಿರೀಶ್​ಚಂದ್ರನೊಂದಿಗೆ ಮಾತನಾಡುತ್ತ ಸ್ವಾಮೀಜಿ, ಖೇತ್ರಿ ರಾಜ್ಯದಲ್ಲಿ ನಡೆದ ಒಂದು ಘಟನೆಯನ್ನು ತಿಳಿಸುತ್ತಾರೆ. ಆಗ ಅವರು ಯಾವುದೋ ಗ್ರಾಮಾಂತರ ಪ್ರದೇಶದಲ್ಲಿ ಉಳಿದುಕೊಂಡಿದ್ದರೆಂದು ಕಾಣುತ್ತದೆ. ಅವರನ್ನು ಕಾಣಲು ಜನ ಹಿಂಡುಹಿಂಡಾಗಿ ಬರಲಾರಂಭಿಸಿದ್ದರು. ಒಮ್ಮೆಯಂತೂ ಜನರ ಸಂಖ್ಯೆ ಎಷ್ಟು ಹೆಚ್ಚಾಗಿಬಿಟ್ಟಿತ್ತೆಂದರೆ, ಮೂರು ಹಗಲು-ಮೂರು ರಾತ್ರಿಗಳವರೆಗೆ ಸ್ವಾಮೀಜಿ ಜನರೊಂದಿಗೆ ನಿರಂತರವಾಗಿ ಮಾತನಾಡುತ್ತಲೇ ಇರಬೇಕಾಯಿತು. “ಇದನ್ನೀಗ ನಂಬಲು ಕಷ್ಟವೆಂಬಂತೆ ತೋರುತ್ತದೆ. ಆದರೂ ಅದು ಸತ್ಯ” ಎಂದು ಹೇಳಿದ್ದಾರೆ ಸ್ವಾಮೀಜಿ. ಜನರೇನೋ ಬರುತ್ತಿದ್ದರು; ಅಧ್ಯಾತ್ಮ-ಸಂಸಾರ-ಶಾಸ್ತ್ರಗಳು ಎಂದೆಲ್ಲ ಮಾತನಾಡಿ, ನಮಸ್ಕರಿಸಿ ಹೊರಟುಬಿಡುತ್ತಿದ್ದರು. ಆದರೆ ಅಷ್ಟು ಹೊತ್ತಿಗೆ ಇನ್ನಷ್ಟು ಜನ ಬಂದು ಕುಳಿತಿರುತ್ತಿದ್ದರು. ಹೀಗಾಗಿ ಸ್ವಾಮೀಜಿಗೆ ಒಂದೇ ಒಂದು ನಿಮಿಷವೂ ಬಿಡುವು ಸಿಕ್ಕಿರಲಿಲ್ಲ. ಅಷ್ಟೇ ಅಲ್ಲ, ಅವರು ಹಸಿದಿದ್ದಾರೆಯೆ? ಏನಾದರೂ ಉಂಡರೆ?–ಎಂದು ಒಬ್ಬರಾದರೂ ಕೇಳಬೇಕಲ್ಲ! ಮೂರನೆಯ ದಿನ ರಾತ್ರಿ, ಬಂದಿದ್ದ ಜನರೆಲ್ಲ ಹೊರಟುಹೋದಮೇಲೆ, ಮೂಲೆಯಲ್ಲಿ ನಿಂತಿದ್ದ ಒಬ್ಬ ಮನುಷ್ಯ ಅವರನ್ನು ಮಾತ ನಾಡಿಸಿದ. ಅವನೊಬ್ಬ ಅಂತ್ಯಜ, ಕಡುಬಡವ ಎಂಬುದು ನೋಡಿದರೇ ತಿಳಿಯುತ್ತಿತ್ತು. ಆತ ಹೇಳಿದ:

“ಸ್ವಾಮೀಜಿ, ನಾನು ಮೂರು ದಿನಗಳಿಂದಲೂ ಗಮನಿಸುತ್ತಲೇ ಇದ್ದೇನೆ, ನೀವು ಏನೂ ಆಹಾರವನ್ನೇ ತೆಗೆದುಕೊಂಡಂತೆ ಕಾಣಲಿಲ್ಲ. ಇಲ್ಲಿಗೆ ಬಂದವರೆಲ್ಲ ಮಾತುಕತೆಯಾಡಿಕೊಂಡು ಸುಖವಾಗಿ ಹೊರಟುಬಿಟ್ಟರು. ನೀವು ಮಾತ್ರ ಹೀಗೆ ಉಪವಾಸವಿದ್ದೀರಿ. ನಿಮ್ಮನ್ನು ನೋಡಿದರೆ ನನಗೆ ತುಂಬ ಸಂಕಟವಾಗುತ್ತಿದೆ, ಸ್ವಾಮೀಜಿ,”

ಈ ಬಡವನಿಗೆ ಇರುವ ಕಾಳಜಿಯನ್ನು ಕಂಡು ಸ್ವಾಮೀಜಿಯ ಹೃದಯ ತುಂಬಿ ಬಂದಿತು. ಸಾಕ್ಷಾತ್ ಭಗವಂತನೇ ಈತನ ರೂಪದಲ್ಲಿ ಬಂದಿರಬಹುದೆ!

“ಹೌದು, ನಾನು ತುಂಬ ಹಸಿದಿದ್ದೇನೆ. ನೀನು ನನಗೇನಾದರೂ ತಿನ್ನಲು ತಂದುಕೊಡಲು ಸಾಧ್ಯವೆ?”

ಆತ ಅಂಜುತ್ತಂಜುತ್ತಲೇ ಉತ್ತರಿಸಿದ:

“ನಿಮಗೇನಾದರೂ ತಿನ್ನಲು ಕೊಡಬೇಕು ಎಂಬ ಆಸೆ ನನಗೇನೋ ಇದೆ, ಸ್ವಾಮೀಜಿ, ಆದರೆ ನಾನೊಬ್ಬ ಚಮ್ಮಾರ. ನನ್ನಂಥವರು ಮುಟ್ಟಿದ್ದನ್ನು ನೀವು ಹೇಗೆ ತಿನ್ನಲಾದೀತು? ಆದರೆ ನೀವು ಅನುಮತಿ ನೀಡುವುದಾದರೆ ನಾನು ಹಿಟ್ಟು ಬೇಳೆ ಮೊದಲಾದುವುಗಳನ್ನೆಲ್ಲ ತಂದುಕೊಡುತ್ತೇನೆ. ನೀವೇ ಚಪಾತಿ ಮಾಡಿಕೊಳ್ಳಬಹುದು. ನೀವು ಒಪ್ಪಿಕೊಂಡರೆ ನಾನು ಧನ್ಯ.”

“ನೋಡು, ನಾನೊಬ್ಬ ಸಂನ್ಯಾಸಿ; ಅಗ್ನಿಸ್ಪರ್ಶ ಮಾಡುವಂತಿಲ್ಲ. ಆದ್ದರಿಂದ ನೀನೇನೂ ಚಿಂತಿಸಬೇಡ, ನೀನೇ ಚಪಾತಿಗಳನ್ನು ಮಾಡಿ ತಂದುಕೊಡು. ನಾನು ಸಂತೋಷದಿಂದ ಅವು ಗಳನ್ನು ತೆಗೆದುಕೊಳ್ಳುತ್ತೇನೆ.”

ಈಗ ಚಮ್ಮಾರನಿಗೆ ವಿಪರೀತ ದಿಗಿಲಾಯಿತು.

“ಸ್ವಾಮೀಜಿ, ಅದು ಖಂಡಿತ ಸಾಧ್ಯವಿಲ್ಲ. ನಾನೇನೋ ಇವತ್ತು ನಿಮಗೆ ಚಪಾತಿಗಳನ್ನು ಮಾಡಿ ತಂದುಕೊಡಬಲ್ಲೆ. ಆದರೆ ನಾಳೆ ದಿನ ಈ ವಿಷಯ ಮಹಾರಾಜರಿಗೇನಾದರೂ ಗೊತ್ತಾಯಿತು ಎಂದರೆ, ಆಗ ನನಗೆ ಗ್ರಹಚಾರ ವಕ್ಕರಿಸುತ್ತದೆ.”

“ನೀನೇನೂ ಹೆದರಬೇಡಯ್ಯ! ನೀನಗೇನೂ ತೊಂದರೆಯಾಗದಂತೆ ನಾನು ನೋಡಿಕೊಳ್ಳು ತ್ತೇನೆ! ನಾನೇ ಕೇಳಿ ತರಿಸಿಕೊಳ್ಳುತ್ತಿರುವಾಗ ನಿನಗೇಕೆ ಭಯ?” ಬಳಿಕ ಪ್ರೀತಿಪೂರ್ವಕವಾಗಿ ಒತ್ತಾಯ ಪಡಿಸುತ್ತ ಹೇಳಿದರು, “ಹೋಗಲಿ, ಏನಾದರೂ ತರುತ್ತೀಯೋ ಇಲ್ಲವೋ ಹೇಳು. ನಾನಂತೂ ತುಂಬ ಹಸಿದಿದ್ದೇನೆ. ಅದು ನಿನಗೆ ಕಾಣುತ್ತಿದೆ.”

ಸ್ವಾಮೀಜಿ ಅಷ್ಟು ಹೇಳಿದರೂ ಚಮ್ಮಾರನಿಗೆ ಸಂಪೂರ್ಣ ಭರವಸೆ ಉಂಟಾಗಲಿಲ್ಲ. ಆದರೂ ಅವರ ಪರಿಸ್ಥಿತಿಗಾಗಿ ಮರುಗಿ, ಏನಾದರಾಗಲಿ ಎಂದು ಚಪಾತಿಗಳನ್ನು ಮಾಡಿ ತಂದುಕೊಟ್ಟ. ಆ ಬಡ ಚಮ್ಮಾರ ಮಾಡಿದ ಚಪಾತಿಗಳು ಹೇಗಿದ್ದಿರಬಹುದು! ಆದರೆ ಅದನ್ನು ಬಾಯಿಗಿಟ್ಟು ಕೊಂಡಾಗ ಸ್ವಾಮೀಜಿಗೆ ಅದು ದೇವಲೋಕದ ಅಮೃತಕ್ಕಿಂತಲೂ ಸವಿಯಾಗಿ ತೋರಿತು. ಕೃತಜ್ಞತೆ ಯಿಂದ ಅವರ ಹೃದಯ ಕರಗಿ ಕಣ್ಣೀರಾಗಿ ಹರಿಯಿತು. ‘ಆಹಾ! ಭಾರತದಲ್ಲಿ ಇಂತಹ ವಿಶಾಲ ಹೃದಯಿಗಳಾದ ಲಕ್ಷಾಂತರ ಜನ ಬಡಗುಡಿಸಲುಗಳಲ್ಲಿ ವಾಸಮಾಡುತ್ತಿದ್ದಾರೆ. ಆದರೆ ನಾವು ಅವರನ್ನು ಅಸ್ಪೃಶ್ಯರೆಂದು ಕರೆದು ಅನಿಷ್ಟಗಳಂತೆ ದೂರವಿಟ್ಟಿದ್ದೇವೆ!’ ಎಂದು ಅವರು ಮರುಗಿ ದರು. ಕೆಲದಿನಗಳಾದ ಮೇಲೆ ಸ್ವಾಮೀಜಿ ಖೇತ್ರಿಯ ಮಹಾರಾಜನಿಗೆ ಈ ವಿಷಯವನ್ನು ತಿಳಿಸಿ ದರು. ಮಹಾರಾಜ ಚಮ್ಮಾರನಿಗೆ ಕರೆ ಕಳಿಸಿದ. ತಾನೆಂದುಕೊಂಡಂತೆಯೇ ಆಯಿತು ಎಂದು ಚಮ್ಮಾರ ಭಯದಿಂದ ನಡುಗುತ್ತಲೇ ಬಂದ. ಆದರೆ ಅವನಿಗೆ ದೊರೆತದ್ದು ಶಿಕ್ಷೆಯಲ್ಲ,. ಸಿರಿರಕ್ಷೆ. ಮಹಾರಾಜ ಆತನಿಗೆ ದೊಡ್ಡ ಬಹುಮಾನವೊಂದನ್ನು ಅನುಗ್ರಹಿಸಿ, ಮುಂದೆ ಅವನ ಜೀವನಕ್ಕೆ ಏನೂ ಕೊರತೆಯಾಗದಂತೆ ಮಾಡಿದ.

ಪರಿವ್ರಾಜಕರಾಗಿ ಹಳ್ಳಿಯಿಂದ ಹಳ್ಳಿಗೆ, ಊರಿನಿಂದ ಊರಿಗೆ, ಕಾಡುಮೇಡು ಬೆಟ್ಟಗುಡ್ಡಗಳ ಮೂಲಕ ಬೆಳಗಿನಿಂದ ಸಂಜೆಯವರೆಗೂ ಅಲೆದಾಡುತ್ತಿದ್ದ ಸ್ವಾಮೀಜಿಯ ಮನಸ್ಸಿನಲ್ಲಿ ಕೆಲ ವೊಮ್ಮೆ ತಮ್ಮ ಈ ಅಲೆದಾಟವೆಲ್ಲ ನಿರರ್ಥಕವಲ್ಲವೆ ಎಂಬ ಸಂಶಯದ ಬಿರುಗಾಳಿಯೇಳು ತ್ತಿತ್ತು. ‘ಛೇ! ಇದೇನಿದು? ಹೀಗೆ ಗೊತ್ತುಗುರಿಯಿಲ್ಲದೆ ಅಲೆಯುತ್ತ, ನಾಚಿಕೆಯಾಗಲಿ ಅಳುಕಾಗಲಿ ಇಲ್ಲದೆ ಕಾಗೆಯಂತೆ ಕಂಡಕಂಡವರ ಮನೆಯಲ್ಲಿ ಭಿಕ್ಷೆ ಬೇಡಿ ಹೊಟ್ಟೆ ಹೊರೆಯುವ ಪುರುಷಾರ್ಥಕ್ಕಾಗಿಯೇನು, ನಾನು ಎಲ್ಲವನ್ನೂ ತ್ಯಾಗ ಮಾಡಿ ಬಂದದ್ದು?’ ಎಂಬ ಚಿಂತೆ ಅವರನ್ನು ಆಗಾಗ ಕಾಡುತ್ತಿತ್ತು. ಕಡೆಗೊಂದು ದಿನ ಈ ಭಾವನೆ ಭೂತಾಕಾರವಾಗಿ ಬೆಳೆದು ನಿಂತಿತು. ‘ನಾನಿನ್ನು ಭಿಕ್ಷೆ ಬೇಡುವುದೇ ಇಲ್ಲ. ನನ್ನಂಥವನಿಗೆ ಭಿಕ್ಷೆ ಹಾಕುವುದರಿಂದ ಆ ಬಡಜನರಿಗಾದರೂ ಏನು ಪ್ರಯೋಜನ? ನನಗೆ ಹಾಕುವ ಒಂದು ಹಿಡಿ ಅನ್ನ ಮಿಕ್ಕರೆ ಅವರದನ್ನು ತಮ್ಮ ಮಕ್ಕಳಿಗೇ ತಿನ್ನಿಸಬಹುದು. ಅಲ್ಲದೆ, ಭಗವತ್ಕಾರ್ಯಕ್ಕಾಗಿ ಈ ದೇಹ ಉಪಯೋಗಕ್ಕೆ ಬಾರದಿದ್ದ ಮೇಲೆ ಅದು ಇದ್ದರೆಷ್ಟು ಹೋದರೆಷ್ಟು ’ ಎಂದು ಆಲೋಚಿಸಿ, ಕಾಡಿನ ಮಧ್ಯದಲ್ಲಿ ಕುಳಿತು ಪ್ರಾಯೋಪವೇಶ ಮಾಡಿ ಪ್ರಾಣತ್ಯಾಗ ಮಾಡಿಬಿಡಬೇಕೆಂದು ನಿಶ್ಚಯಿಸಿದರು.

ಈ ಯೋಚನೆ ಮನಸ್ಸಿನಲ್ಲಿ ಬಂದದ್ದೇ ತಡ, ಮತ್ತೆ ಒಂದಿನಿತೂ ಅನುಮಾನಿಸದೆ ಸ್ವಾಮೀಜಿ, ಕಾಡಿನ ಮಧ್ಯಕ್ಕೆ ನಡೆದೇ ಬಿಟ್ಟರು. ಮಧ್ಯಾಹ್ನವಾಯಿತು. ಸಂಜೆಯಾಯಿತು, ಆದರೂ ಎಲ್ಲೂ ನಿಲ್ಲದೆ ಮೈಲಿಗಟ್ಟಲೆ ನಡೆದೇ ನಡೆದರು. ಬೆಳಗಿನಿಂದ ಒಂದು ಗುಟುಕು ನೀರನ್ನೂ ಕುಡಿದಿಲ್ಲ. ಕಡೆಗೆ ಆಯಾಸದಿಂದ ಕಣ್ಣು ಕತ್ತಲೆ ಬಂದು ಬಿದ್ದುಬಿಟ್ಟರು. ಬಳಿಕ ಹಾಗೇ ಸಾವರಿಸಿಕೊಂಡೆದ್ದು ಬಂದು, ಸಮೀಪದ ಮರಕ್ಕೆ ಒರಗಿಕೊಂಡು, ಭಗವಂತನಲ್ಲಿ ಮನಸ್ಸನ್ನು ನೆಲೆಗೊಳಿಸಿ ಶೂನ್ಯ ವನ್ನು ದಿಟ್ಟಿಸುತ್ತ ಕುಳಿತರು.

ಸ್ವಲ್ಪಹೊತ್ತಿನಲ್ಲೇ ಸೂರ್ಯನೂ ಮುಳುಗಿದ. ನಿಧಾನವಾಗಿ ಮಬ್ಬುಗತ್ತಲು ಆವರಿಸಿತು. ನಿಶಾಚರ ಪಶುಪಕ್ಷಿಗಳೆಲ್ಲ ಎದ್ದು ಆಹಾರಾನ್ವೇಷಣೆಗೆ ಹೊರಟುವು. ಹೀಗೆ ಹೊರಟ ಒಂದು ಹುಲಿ ಸ್ವಾಮೀಜಿಯನ್ನು ಕಂಡು ಹತ್ತಿರ ಹತ್ತಿರಕ್ಕೆ ಬಂದಿತು. ಅದನ್ನು ಕಂಡ ಸ್ವಾಮೀಜಿ, ತಮ್ಮ ಉದ್ದೇಶ ಇಷ್ಟು ಬೇಗ ನೆರವೇರುವಂತಾಯಿತಲ್ಲ ಎಂದು ಸಂತೋಷಗೊಂಡರು. ‘ಬಾ ಹುಲಿ ಯಣ್ಣ! ಒಳ್ಳೆ ಸಮಯಕ್ಕೇ ಬಂದೆ. ಇಬ್ಬರೂ ಹಸಿದಿದ್ದೇವೆ. ಈ ನನ್ನ ಶರೀರದಿಂದ ಪ್ರಪಂಚಕ್ಕೆ ಏನೂ ಪ್ರಯೋಜವಾಗುವಂತಿಲ್ಲ. ಹೋಗಲಿ, ನೀನಾದರೂ ನನ್ನನ್ನು ತಿಂದು ಹೊಟ್ಟೆ ತುಂಬಿಸಿಕೊ, ಬಾ’ ಎಂದು ಮನಸ್ಸಿನಲ್ಲೇ ಆಹ್ವಾನಿಸಿದರು. ಆದರೆ ಹುಲಿರಾಯ ಸ್ವಲ್ಪ ಆಲೋಚನೆ ಮಾಡುತ್ತ ಅಲ್ಲೇ ಕುಳಿತುಬಿಟ್ಟ. ಎರಡು ಹಸಿದ ಹುಲಿಗಳ ನಡುವೆ ಸ್ವಲ್ಪ ಹೊತ್ತು ದೃಷ್ಟಿಯುದ್ಧ ನಡೆಯಿತು. ಕಡೆಗೆ ಕಾಡಿನ ಹುಲಿರಾಯನೇ ಸೋತ. ‘ನಿನ್ನನ್ನು ತಿನ್ನುವುದಕ್ಕೆ ನನಗೇನು ಹುಚ್ಚೇ?’ ಎಂಬಂತೆ ನೋಡಿ, ಬಾಲವನ್ನು ಅಲ್ಲಾಡಿಸಿ, ಬಂದ ದಾರಿಯಲ್ಲೇ ಹೊರಟುಬಿಟ್ಟ!

ಸ್ವಾಮೀಜಿಯ ಮುಖದಲ್ಲಿ ಮಂದಹಾಸ ಬೆಳಗಿತು. ಆದರೂ ಅದೂ ಮತ್ತೆ ಬಂದೀತು ಎಂದು ಕಾದರು. ಆದರೆ ಅದರ ಸುಳಿವೇ ಇಲ್ಲ. ರಾತ್ರಿಯೆಲ್ಲ ಸ್ವಾಮೀಜಿ ತಮ್ಮ ಅಂತರಾತ್ಮ ದೊಂದಿಗೆ ಒಂದಾಗಿ, ಅಲ್ಲೇ ಭಾವಸ್ಥರಾಗಿ ಕುಳಿತಿದ್ದರು. ತಮ್ಮಿಷ್ಟ ಬಂದಂತೆ ಮಾಡಲು ಅವರಿಗೆ ಎಲ್ಲಿದೆ ಅಧಿಕಾರ! ಅವರು ರಾಜೀನಾಮೆ ಕೊಟ್ಟರೂ ಶ್ರೀರಾಮಕೃಷ್ಣರು ಸ್ವೀಕರಿಸಬೇಕಲ್ಲ! ಬೆಳಗಾಗುತ್ತಿದ್ದಂತೆ ಸ್ವಾಮೀಜಿ ತಮ್ಮ ದಾರಿಯನ್ನು ಹಿಡಿದು ಮತ್ತೆ ಮುಂದೆ ಸಾಗಿದರು.

ಮತ್ತೊಮ್ಮೆ ಸ್ವಾಮೀಜಿ, ತಾವಾಗಿಯೇ ಭಿಕ್ಷೆ ಬೇಡುವುದಿಲ್ಲ, ಹಿಂದಿರುಗಿಯೂ ನೋಡುವು ದಿಲ್ಲ ಎಂಬ ವ್ರತ ತೊಟ್ಟು, ನಿರಂತರವಾಗಿ ನಡೆಯುತ್ತಿದ್ದರು. ಯಾರಾದರೂ ತಾವಾಗಿಯೇ ಅವರನ್ನು ನಿಲ್ಲಿಸಿ, ಭಿಕ್ಷೆ ನೀಡಲು ಮುಂದಾದರೆ ಮಾತ್ರ ನಿಲ್ಲುತ್ತಿದ್ದರು. ಆದರೆ ಅವರು ಇಂತಹ ವ್ರತವನ್ನು ಕೈಗೊಂಡಿದ್ದಾರೆಂಬುದು ಯಾರಿಗಾದರೂ ಹೇಗೆ ತಿಳಿಯಬೇಕು! ಅಂತೂ ಹೇಗೋ ಭಿಕ್ಷೆ ಸಿಗುತ್ತಿತ್ತು–ದಿನಕ್ಕೊಮ್ಮೆಯೋ, ಎರಡು ದಿನಕ್ಕೊಮ್ಮೆಯೋ. ಒಂದು ಸಂಜೆ ಹಾಗೇ ನಡೆಯುತ್ತ, ಶ್ರೀಮಂತನೊಬ್ಬನ ಕುದುರೆಲಾಯದ ಬಳಿಗೆ ಬಂದರು. ಅವರು ಊಟ ಮಾಡಿ ಆಗಲೇ ಎರಡು ದಿನವಾಗಿತ್ತು. ತುಂಬ ನಿತ್ರಾಣರಾಗಿದ್ದದ್ದು ಕಂಡು ಬರುತ್ತಿತ್ತು. ಆಗ ಅವರನ್ನು ಕಂಡ ಒಬ್ಬ ಸೇವಕ ಅವರನ್ನು ಕರೆದು ನಮಸ್ಕರಿಸಿ ಕೇಳಿದ:

“ಬಾಬಾಜಿ, ಇವತ್ತು ತಮಗೇನಾದರೂ ಭಿಕ್ಷೆ ಸಿಕ್ಕಿತೆ?”

“ಇಲ್ಲಪ್ಪ, ನನಗೇನೂ ಸಿಕ್ಕಿಲ್ಲ.”

ಸೇವಕ ಅವರನ್ನು ಒಳಗೆ ಕರೆದುಕೊಂಡು ಹೋಗಿ ಕೈಕಾಲು ತೊಳೆಯಲು ನೀರುಕೊಟ್ಟ. ಬಳಿಕ ತನ್ನದೇ ಆಹಾರವಾದ ಚಪಾತಿಗಳನ್ನೂ ಚಟ್ನಿಯನ್ನೂ ತಂದು ಮುಂದಿಟ್ಟ. ಸ್ವಾಮೀಜಿ ಚಪಾತಿ- ಚಟ್ನಿ ತಿಂದರು. ಚಟ್ನಿ ವಿಪರೀತ ಖಾರ. ಆದರೆ ಅವರಿಗೆ ಖಾರವೆಂದರೆ ಇಷ್ಟವೇ. ಸಂತೋಷ ದಿಂದಲೇ ಎಲ್ಲವನ್ನೂ ಮುಗಿಸಿದರು. ಆದರೆ ಅದನ್ನು ತಿಂದ ತಕ್ಷಣ ಹೊಟ್ಟೆಯಲ್ಲಿ ಭಯಂಕರ ಉರಿ ಹತ್ತಿಕೊಂಡಿತು. ತಾಳಲಾರದೆ ನೆಲದ ಮೇಲೆ ಹೊರಳಾಡತೊಡಗಿದರು. ಸೇವಕ ತಲ್ಲಣಿಸಿ ಹೋದ. “ಅಯ್ಯಯ್ಯೋ, ಇದೇನು ಮಾಡಿಬಿಟ್ಟೆ ನಾನು? ಸಾಧುವನ್ನು ಕೊಂದುಬಿಟ್ಟೆನಲ್ಲಾ!” ಎಂದು ಎದೆ ಬಡಿದುಕೊಂಡು ಅಳಲಾರಂಭಿಸಿದ. ಖಾಲಿ ಹೊಟ್ಟೆಯಲ್ಲಿ ಅಷ್ಟೊಂದು ಖಾರ ತಿಂದದ್ದರಿಂದ ಅಷ್ಟು ಉರಿಯಾಗಿರಬೇಕು. ಆ ಹೊತ್ತಿಗೆ ಸರಿಯಾಗಿ, ತಲೆಯ ಮೇಲೆ ಬುಟ್ಟಿ ಇಟ್ಟುಕೊಂಡು ಹೋಗುತ್ತಿದ್ದವನೊಬ್ಬ ಆ ದಾರಿಯಾಗಿ ಬಂದ. ಆ ಸೇವಕನ ಕೂಗಾಟ ಕೇಳಿ ಏನು ಸಮಾಚಾರ ಎಂದು ಬಗ್ಗಿ ನೋಡಿದ. ಸ್ವಾಮೀಜಿ ಕೇಳಿದರು, “ಏನದು ಬುಟ್ಟಿಯಲ್ಲಿ?” “ಹುಣಿಸೇಹಣ್ಣು,” “ಆಹ್! ಅದೇ ನನಗೀಗ ಬೇಕಾಗಿದ್ದದ್ದು!” ಎಂದು ಸ್ವಾಮೀಜಿ, ಸ್ವಲ್ಪ ಹುಣಿಸೇಹಣ್ಣು ತೆಗೆದುಕೊಂಡು ನೀರಿನಲ್ಲಿ ಕಿವಿಚಿ ಕುಡಿದರು. ನಿಧಾನವಾಗಿ ಉರಿ ಶಾಂತ ವಾಯಿತು. ಸ್ವಲ್ಪ ಹೊತ್ತು ವಿಶ್ರಮಿಸಿ, ಸೇವಕನಿಗೂ ವ್ಯಾಪಾರಿಗೂ ಕೃತಜ್ಞತೆ ಹೇಳಿ ಅಲ್ಲಿಂದ ಹೊರಟರು.

ಸ್ವಾಮೀಜಿ ಅನ್ನಾಹಾರಗಳಿಲ್ಲದೆ ಬಳಲಿದ ದಿನಗಳು ಲೆಕ್ಕವಿಲ್ಲದಷ್ಟು. ಹೀಗೇ ಒಂದು ದಿನ ಉರಿವ ಸೂರ್ಯನ ಬಿಸಿಲಿನಲ್ಲಿ ನಡೆದುಕೊಂಡು ಹೋಗುತ್ತಿರುವಾಗ, ಸಂಪೂರ್ಣ ನಿಶ್ಶಕ್ತರಾಗಿ ಒಂದು ಮರದ ಬುಡದಲ್ಲಿ ಕುಸಿದು ಕುಳಿತರು. ಕಾಲುಗಳು ಅಲುಗಾಡಿಸಲೂ ಸಾಧ್ಯವಾಗದೆ ಕೊರಡಿನಂತಾಗಿಬಿಟ್ಟಿದ್ದುವು. ಕಣ್ತೆರೆದು ನೋಡಲು ಪ್ರಯತ್ನಿಸಿದರೂ ಕಣ್ಣೆವೆಗಳು ಭಾರವಾಗಿ ಮತ್ತೆಮತ್ತೆ ಮುಚ್ಚಿಕೊಳ್ಳುತ್ತಿದ್ದುವು. ಆಲೋಚನೆ ಮಾಡಲೂ ಸಾಧ್ಯವಿಲ್ಲದಂತೆ, ಮೆದುಳು ಸಂಪೂರ್ಣ ಬರಿದಾದಂತೆ ಅನ್ನಿಸಿತು. ಆಗ ನಿಬಿಡಾಂಧಕಾರದಲ್ಲಿ ಕೋಲ್ಮಿಂಚು ಕೋರೈಸಿದಂತೆ, ಇದ್ದಕ್ಕಿದ್ದಂತೆ ಅವರ ಮನಸ್ಸಿಗೊಂದು ಹೊಸ ಬೆಳಕು ಕಂಡಿತು–‘ಆಹ್! ನಾನು ದುರ್ಬಲ ನಾಗಿರಲು ಹೇಗೆ ಸಾಧ್ಯ? ನಾನು ಸರ್ವಶಕ್ತಿಸ್ವರೂಪನಾದ ಆತ್ಮನಲ್ಲವೆ? ಈ ಶರೀರ-ಇಂದ್ರಿಯ ಗಳು ನನ್ನ ಆತ್ಮದ ಮೇಲೆ ಪ್ರಭುತ್ವವನ್ನು ಸಾಧಿಸಲು ಹೇಗೆ ಸಾಧ್ಯ?’ ಈ ಭಾವನೆ ಅವರ ಮನಸ್ಸಿ ನಲ್ಲಿ ಮೂಡುತ್ತಿದ್ದಂತೆಯೇ ಅಪರಿಮಿತ ಶಕ್ತಿಯೊಂದು ಅವರ ಆಂತರ್ಯದಿಂದ ಚಿಮ್ಮಿತು. ಮನದಾಳದಲ್ಲಿ ಬೆಳಕಿನ ಹೊಳೆ ಹೊಮ್ಮಿ ಹರಿಯಿತು. ಇಂದ್ರಿಯಗಳ ಚೈತನ್ಯ ಸಾಸಿರಮಡಿಯಾಗಿ ಮರಳಿತು. ಈಗ ಸ್ವಾಮೀಜಿ ಧಿಗ್ಗನೆದ್ದು ನಿಂತರು. ಅಡಿ ಮುಂದಿಟ್ಟು ಸರಸರನೆ ನಡೆದೇಬಿಟ್ಟರು.

ಸ್ವಾಮೀಜಿಯ ಪ್ರಯಾಣಕಾಲದಲ್ಲಿ ಅನೇಕ ಸಲ ಇಂತಹ ದುರ್ಭರ ಪ್ರಸಂಗಗಳೊದಗಿ ದ್ದುವು. ಪ್ರತಿಸಲವೂ ಅವರೊಳಗಿನಿಂದ ಯಾವುದೋ ದಿವ್ಯ ಶಕ್ತಿ ಉದ್ಭವಿಸಿ ಅವರ ವ್ಯಕ್ತಿತ್ವದಲ್ಲಿ ಭೋರ್ಗರೆದು ಹರಿದು ಅವರನ್ನು ಮೇಲೆತ್ತಿ ನಿಲ್ಲಿಸುತ್ತಿತ್ತು. ಮುಂದೆ ಕ್ಯಾಲಿಫೋರ್ನಿಯದಲ್ಲಿ ಮಾಡಿದ ಉಪನ್ಯಾಸವೊಂದರಲ್ಲಿ ತಮ್ಮ ಈ ದಿವ್ಯಾನುಭವಗಳನ್ನೇ ಶ್ರೋತೃಗಳಿಗೆ ತಮ್ಮ ಸಂದೇಶವನ್ನಾಗಿ ನೀಡುತ್ತಾರೆ:

“.... ನನ್ನ ಪ್ರಯಾಣಕಾಲದಲ್ಲಿ ನಾನೆಷ್ಟೆಷ್ಟೋ ಸಲ ಮೃತ್ಯುವಿನ ದವಡೆಯಲ್ಲಿ ಸಿಲುಕಿ ಕೊಂಡಿದ್ದೆ. ಹೊಟ್ಟೆಯಲ್ಲಿ ಹಸಿವು; ಪಾದಗಳಲ್ಲಿ ಹುಣ್ಣು; ಅಂಗಪ್ರತ್ಯಂಗಗಳಲ್ಲಿ ನೋವು, ಆಯಾಸ. ದಿನಗಟ್ಟಲೆಯಿಂದ ಆಹಾರವಿಲ್ಲದೆ ಉಪವಾಸವಿರುತ್ತಿದ್ದೆ. ಬಳಲಿಕೆಯಿಂದಾಗಿ ಇನ್ನೊಂದು ಹೆಜ್ಜೆಯನ್ನೂ ಮುಂದಿಡಲಾರದವನಾಗಿ ಒಂದು ಮರದ ಕೆಳಗೆ ಕುಕ್ಕರಿಸುತ್ತಿದ್ದೆ. ಇನ್ನು ನನ್ನ ಕತೆ ಮುಗಿಯಿತು ಎಂದು ನಂಬಿ ಕಣ್ಮುಚ್ಚಿ ಕುಳಿತುಬಿಡುತ್ತಿದ್ದೆ. ಆ ಕ್ಷಣದಲ್ಲಿ ನನ್ನ ಮನಸ್ಸಿನಾಳದಿಂದ ಅಂತರ್ಜಲದಂತೆ ಭಾವನೆಗಳು ಚಿಮ್ಮಿಬರುತ್ತಿದ್ದುವು–‘ಇಲ್ಲ! ನಾನು ಸಾಯಲಾರೆ!ನಾನೆಂದು ಹುಟ್ಟಲೂ ಇಲ್ಲ. ನನಗೆಂದಿಗೂ ಸಾವೂ ಇಲ್ಲ! ನನಗೆ ಹಸಿವೆಯೂ ಇಲ್ಲ, ದಾಹವೂ ಇಲ್ಲ! ಸೋ\eng{s}ಹಂ! ಸೋ\eng{s}ಹಂ! ಆತ್ಮನೇ ನಾನು! ಸಮಸ್ತ ಪ್ರಕೃತಿಯೇ ನನಗೆದುರಾದರೂ ಅದು ನನ್ನನ್ನು ಸೋಕಲೂ ಆರದು. ಅದು ನನ್ನ ಆಜ್ಞಾನುವರ್ತಿ. ಓ ನನ್ನ ಆತ್ಮ! ಸಕಲ ದೇವರದೇವ!ನಿನ್ನ ನಿಜಶಕ್ತಿಯನ್ನು ಮೆರೆಸು! ನಿನ್ನ ನಿಜಸ್ವರೂಪವನ್ನು ಪುನಃ ಪ್ರತಿಷ್ಠಿಸು!ಎದ್ದೇಳು, ಮುಂದೆ ನಡೆ, ನಿಲ್ಲದಿರು!’ ಎಂದು ನನಗೆ ನಾನೇ ಹೇಳಿಕೊಳ್ಳುತ್ತಿದ್ದೆ. ನನ್ನ ದೌರ್ಬಲ್ಯವನ್ನು ಮೆಟ್ಟಿ ನಿಲ್ಲುತ್ತಿದ್ದೆ. ಈಗ ನಾನು ನಿಮ್ಮ ಮುಂದೆ ನಿಂತಿರುವುದನ್ನು ನೀವೇ ನೋಡುತ್ತಿದ್ದೀರಿ! ಆದ್ದರಿಂದ ನೀವೂ ಕೂಡ, ನಿಮ್ಮನ್ನು ಅಜ್ಞಾನದ ಅಂಧಕಾರ ಮುತ್ತಿದಾಗಲೆಲ್ಲ ನೀವು ಆತ್ಮಸ್ವರೂಪರು ಎಂಬುದನ್ನು ನೆನಪಿಸಿಕೊಳ್ಳಿ. ಆಗ ಕಷ್ಟಕಾರ್ಪಣ್ಯಗಳೆಲ್ಲ ಪಲಾಯನ ಗೈಯಲೇಬೇಕು. ಏಕೆಂದರೆ, ಅದೆಷ್ಟೇ ಆದರೂ ಒಂದು ಕನಸು. ಕಷ್ಟಗಳು ಪರ್ವತೋಪಮವಾಗಿ ದಾರಿಗಡ್ಡವಾಗಿ ನಿಂತು ನಿಮ್ಮನ್ನು ಬೆದರಿಸಬಹುದು. ರಾಕ್ಷಸಾಕಾರ ತಳೆದು ನಿಮ್ಮನ್ನು ಕಬಳಿಸ ಲೆತ್ನಿಸಬಹುದು. ಆದರೂ ಅಂಜಬೇಡಿ. ಅದನ್ನು ಎದುರಿಸಿ ನಿಲ್ಲಿ; ಕುಟ್ಟಿ ಪುಡಿಪುಡಿ ಮಾಡಿಬಿಡಿ! ಅದು ಕೇವಲ ಮಾಯೆ; ಕ್ಷಣಾರ್ಧದಲ್ಲಿ ಅದು ಇಲ್ಲವಾಗಿಬಿಡುತ್ತದೆ. ಅದನ್ನು ಕಾಲಡಿಯಲ್ಲಿ ಕ್ರಿಮಿಯಂತೆ ಹೊಸಕಿಹಾಕಿಬಿಡಿ; ಅದು ಅಲ್ಲೇ ನಿರ್ಜೀವವಾಗಿ ಬೀಳುತ್ತದೆ.”

ಇಲ್ಲಿ ಸ್ವಾಮೀಜಿ ತಮ್ಮನ್ನು ಅಲೌಕಿಕ ವ್ಯಕ್ತಿಯೆಂದೋ ಮಹಾಪುರುಷನೆಂದೋ ಬಣ್ಣಿಸಿ ಕೊಳ್ಳದೆ ಅಥವಾ ಭಾವಿಸಿಕೊಳ್ಳದೆ ತಮ್ಮಂತೆಯೇ ಜಗತ್ತಿನ ಪ್ರತಿಯೊಬ್ಬ ವ್ಯಕ್ತಿಯೂ ಕಷ್ಟಗಳ ಸಂಕೋಲೆಯಿಂದ ಮುಕ್ತನಾಗಬಲ್ಲ ಎಂದು ಘೋಷಿಸುವುದನ್ನು ನೋಡುತ್ತೇವೆ. ತಮ್ಮ ದಿವ್ಯಾನುಭವವನ್ನೇ ಸಾರ್ವತ್ರೀಕರಿಸಿ ಬೋಧಿಸುವಲ್ಲಿ ಅವರ ಸಂದೇಶದ ಉದಾತ್ತತೆಯನ್ನು ಮನಗಾಣಬಹುದು.

ಇನ್ನೊಮ್ಮೆ ಸ್ವಾಮೀಜಿ ರಜಪುತಾನದ ಮರುಭೂಮಿಯಲ್ಲಿ ಪ್ರಯಾಣ ಮಾಡುತ್ತಿದ್ದಾಗ ನಡೆದ ಘಟನೆ. ನೆತ್ತಿಯ ಮೇಲೆ ಸೂರ್ಯ ಉರಿಕೆಂಡದ ಮಳೆಗರೆಯುತ್ತಿದ್ದಾನೆ. ನೆಲವೋ ಕಾದ ಕಾವಲಿ. ಜೊತೆಗೆ ಆಗಾಗ ರಭಸದಿಂದ ಬೀಸುವ ಬಿಸಿಗಾಳಿ. ಸ್ವಾಮೀಜಿಯ ಕಂಠ ಒಣಗಿಹೋಗಿದೆ. ಸ್ವಲ್ಪ ವಿಶ್ರಮಿಸಿಕೊಳ್ಳೋಣವೆಂದರೆ ಒಂದೇ ಒಂದು ಮರವೂ ಇಲ್ಲ. ದೃಷ್ಟಿ ಸಾಗುವಷ್ಟು ದೂರವೂ ಮರಳು, ಮರಳು ಮತ್ತು ಮರಳು. ಹಾಗೆ ಕಾಲೆಳೆದುಕೊಂಡು ನಡೆದರು ಸ್ವಾಮೀಜಿ. ಮತ್ತಷ್ಟು ನಡೆದ ಮೇಲೆ ದೂರದಲ್ಲೊಂದು ಸರೋವರ ಕಂಡಿತು. ಜೊತೆಗೆ ವಿಶ್ರಮಿಸಿಕೊಳ್ಳಲು ಮರಗಳ ನೆರಳೂ ಇದೆ! ಸ್ವಾಮೀಜಿಗೆ ಹೋದ ಜೀವ ಬಂದಂತಾಯಿತು. ಸರೋವರದ ದಿಕ್ಕಿನಲ್ಲಿ ಬೇಗಬೇಗ ಹೆಜ್ಜೆ ಹಾಕಿದರು. ಸರೋವರದ ತಿಳಿನೀರು ಥಳಥಳನೆ ಹೊಳೆಯುತ್ತ ಅವರನ್ನೇ ಕೂಗಿಕರೆಯುತ್ತಿರುವಂತಿತ್ತು. ಆದರೆ ಆ ಸ್ಥಳ ಅವರು ಮೊದಲು ಭಾವಿಸಿದ್ದಷ್ಟು ಹತ್ತಿರವೇನೂ ಇರಲಿಲ್ಲ. ಹೋದಹೋದಂತೆ ಅದು ಅಷ್ಟು ದೂರದಲ್ಲಿದೆಯೆಂಬುದು ಮನವರಿಕೆಯಾಯಿತು. ಮತ್ತಷ್ಟು ದೂರ ನಡೆದರು. ಆದರೆ ನೀರಿನ ಸೂಚನೆಯೇ ಇಲ್ಲ. ಸ್ವಾಮೀಜಿ ಆಯಾಸದಿಂದ ಮರಳ ಮೇಲೇ ಕುಳಿತು ಆಲೋಚಿಸಿದರು–ಇದೇನಿದು, ಎಲ್ಲಿಗೆ ಹೋಯಿತು ಆ ತಿಳಿನೀರ ಸರೋ ವರ? ಇದ್ದಕ್ಕಿದ್ದಂತೆ ಅವರ ಬುದ್ಧಿಗೆ ಸತ್ಯ ಗೋಚರಿಸಿತು–ತಾವು ಕಂಡದ್ದು ಸರೋವರವಲ್ಲ, ಮೃಗಜಲ! ಅವರು ಕಂಡದ್ದು ಬಿಸಿಲ್ಗುದುರೆಯನ್ನು, ಮರೀಚಿಕೆಯನ್ನು! ಆಗ ಅವರಿಗೇ ನಾಚಿಕೆ ಯಾಯಿತು–‘ಛೇ, ನಾನು ಅಷ್ಟೆಲ್ಲ ಓದಿಕೊಂಡವನಾಗಿ, ತಿಳಿದೂ ತಿಳಿದೂ ಈ ಬಿಸಿಲ್ಗುದುರೆಯ ಮೋಸಕ್ಕೆ ಬಲಿಯಾದೆನಲ್ಲ!’ ಎಂದು ಮರಕ್ಷಣದಲ್ಲಿ ಅವರಲ್ಲಿ ಮತ್ತೊಂದು ಆಲೋಚನೆ ಮೂಡಿತು–‘ಈ ಜೀವನವೂ ಹೀಗೆಯೇ ಅಲ್ಲವೆ! ಮಾಯೆ! ಜೀವರನ್ನು ಮಾಯೆ ಮೋಸ ಗೊಳಿಸುವುದೂ ಹೀಗೆಯೆ!’ ಈಗ ಅವರು ಮೇಲೆದ್ದು ತಮ್ಮ ಪ್ರಯಾಣವನ್ನು ಮುಂದುವರಿಸಿ ದರು. ಇನ್ನೂ ಒಂದೆರಡು ಹೆಜ್ಜೆ ಇಟ್ಟಿರಲಿಲ್ಲ. ಅಷ್ಟರಲ್ಲೇ ಸರೋವರ ಮತ್ತೆ ಕಾಣಿಸಿಕೊಂಡಿತು! ‘ಬಾ, ಪ್ರಯಾಣಿಕ, ಬಂದು ಈ ತಿಳಿನೀರನ್ನು ಕುಡಿ, ವಿಶ್ರಮಿಸು’ ಎಂದು ಕೈಬೀಸಿ ಆಹ್ವಾನಿಸಿತು. ಆದರೆ ಈಗ ಅವರು ಮರುಮರೀಚಿಕೆಗೆ ಮರುಳಾಗಲಿಲ್ಲ. ಮುಗುಳ್ನಕ್ಕು ಮುಂದೆ ಸಾಗಿದರು. ಮುಂದೆ ಅವರು ಪಾಶ್ಚಾತ್ಯ ರಾಷ್ಟ್ರಗಳಲ್ಲಿ ಮಾಯೆಯ ಕುರಿತಾಗಿ ಉಪನ್ಯಾಸ ಮಾಡುವಾಗ ತಮ್ಮ ಈ ಅನುಭವವನ್ನು ಉಪಯೋಗಿಸಿಕೊಂಡು, ಜೀವನದ ಬಿಸಿಲ್ಗುದುರೆಯ ಸ್ವರೂಪವನ್ನು ವಿವರಿ ಸುತ್ತಾರೆ.\footnote{* ನೋಡಿ: ಅನುಬಂಧ ೩.}

ಸ್ವಾಮೀಜಿ ಮಧ್ಯಭಾರತದಲ್ಲಿ ಸಂಚರಿಸುತ್ತಿದ್ದಾಗ, ಕೆಲವು ಸಲ ಜನ ಅವರಿಗೆ ಆಹಾರ-ವಸತಿ ಗಳನ್ನು ನೀಡಲು ನಿರಾಕರಿಸಿಬಿಟ್ಟರು. ಇದರಿಂದ ಅವರು ಬಹಳವಾಗಿ ಕಷ್ಟಪಡಬೇಕಾಯಿತು. ಈ ದಿನಗಳಲ್ಲೇ ಅವರು ಅತ್ಯಂತ ಹೀನ ವರ್ಗಕ್ಕೆ ಸೇರಿದವರಾದ ಜಾಡಮಾಲಿಗಳ ಮನೆಯಲ್ಲಿ ಉಳಿದುಕೊಂಡದ್ದು; ಅವರ ನಿಕಟ ಸಂಪರ್ಕಕ್ಕೆ ಬರುವಂತಾದದ್ದು. ಸಮಾಜವು, ಅಸ್ಪೃಶ್ಯರು ಎಂದು ತುಚ್ಛೀಕರಿಸಿ ಹೊರಹಾಕಿದವರನ್ನು ಹತ್ತಿರದಿಂದ ಕಂಡಾಗ, ಈ ಜನಗಳಲ್ಲಡಗಿರುವ ಅಪಾರ ಸಾಧ್ಯತೆಗಳು ಸ್ವಾಮೀಜಿಯ ದೃಷ್ಟಿಗೆ ಗೋಚರವಾದುವು. ಅವರಲ್ಲೂ ಎಂತಹ ಹೃದಯ ವಂತಿಕೆಯಿದೆ, ಎಂತಹ ಮಾನವೀಯ ಗುಣಗಳಿವೆ ಎಂಬುದನ್ನು ಮನಗಂಡಾಗ ಸ್ವಾಮೀಜಿಯ ಹೃದಯ ತುಂಬಿಬಂದಿತು. ಅವರನ್ನು ಮೃಗಗಳಂತೆ ಕಂಡು ಮೃಗಗಳನ್ನಾಗಿಯೇ ಪರಿವರ್ತಿಸುತ್ತಿ ರುವ ಮೇಲ್ಜಾತಿಯವರ ಮೇಲೆ ರೋಷ ಉಕ್ಕಿತು. ಎಂದಿನಿಂದಲೂ ಸ್ವಾಮೀಜಿ ಸಮಾಜದ ಆ ಅಸಮತೋಲನದ, ಅನ್ಯಾಯದ ಪರಿಸ್ಥಿತಿಯನ್ನು ಕಂಡವರೇ. ಅದನ್ನು ಸರಿಪಡಿಸಬೇಕೆಂಬ ಆಕಾಂಕ್ಷೆ ಅವರಲ್ಲಿ ಮೊದಲಿನಿಂದಲೂ ಇದ್ದಿತು. ಆದರೆ ಅಸ್ಪೃಶ್ಯರೆನಿಸಿಕೊಂಡವರ ಜೊತೆ ಯಲ್ಲೇ ಜೀವಿಸುವ ಈ ಅನುಭವಗಳು ಅವರಲ್ಲಿ ಹೊಸ ಭಾವತರಂಗವನ್ನೆಬ್ಬಿಸಿದುವು. ಭವ್ಯ ಸಂಸ್ಕೃತಿಯ ಅಪೂರ್ವ ಪರಂಪರೆಯನ್ನುಳ್ಳ ವಿಶಾಲ ಭಾರತದಲ್ಲಿ ಅವರು ಎಲ್ಲೆಲ್ಲೂ ಕಂಡದ್ದು ಕೇವಲ ಬಡತನ ಅಜ್ಞಾನ ಸಂಕಟಗಳನ್ನು. ಈ ದಾರುಣ ದೃಶ್ಯವನ್ನು ಕಂಡು ಅವರ ಹೃದಯ ಅನುಕಂಪೆಯಿಂದ ಕರಗಿತು. ಭಾರತದ ಕೋಟ್ಯಾಂತರ ದೀನದಲಿತರ ಸೇವೆಗಾಗಿ ಅವರನ್ನು ಕಂಕಣ ಬದ್ಧರನ್ನಾಗಿಸಿತು.

ಈ ಎಲ್ಲ ವಿವಿಧ, ವಿಚಿತ್ರ, ವಿಶಾಲ ಅನುಭವಗಳ ನಡುವೆಯೂ ಸ್ವಾಮೀಜಿ ತಮ್ಮ ಅಂತರಂಗ ದಲ್ಲಿ ಒಬ್ಬ ಆದರ್ಶ ಸಂನ್ಯಾಸಿಯೇ. ಆಧ್ಯಾತ್ಮಿಕ ಜೀವನಾದರ್ಶವನ್ನು ಎತ್ತಿಹಿಡಿಯುವ ನಿಷ್ಠುರ ಸಂನ್ಯಾಸಿಯೇ. ಆದ್ದರಿಂದ ದಿನಗಳು ಕಳೆದಂತೆಲ್ಲ, ಆಧುನಿಕ ನಾಗರಿಕತೆಯ ಮೂಲಭೂತ ಆವಶ್ಯಕತೆಯೆಂದರೆ ಧರ್ಮ-ಅಧ್ಯಾತ್ಮ ಎಂದು ಅವರಿಗೆ ಹೆಚ್ಚುಹೆಚ್ಚಾಗಿ ಅನ್ನಿಸತೊಡಗಿತ್ತು. ಆದ್ದರಿಂದ ಅವರು ತಮ್ಮ ವೈಯಕ್ತಿಕ ಜೀವನದಲ್ಲಿ ಸಂನ್ಯಾಸದ ಪರಮಾದರ್ಶವನ್ನು ಕಟ್ಟು ನಿಟ್ಟಾಗಿ ಪಾಲಿಸುವಲ್ಲಿ ಅತಿಹೆಚ್ಚಿನ ಎಚ್ಚರದಿಂದಿದ್ದರು. ಸಂನ್ಯಾಸ ಜೀವನವು ಅವರ ಸ್ವಭಾವವೇ ಆಗಿತ್ತೆಂದು ಹೇಳಬಹುದು. ಆದ್ದರಿಂದಲೇ ಅವರನ್ನು ಭೇಟಿ ಮಾಡಿದ ಇತರ ಪ್ರಾಮಾಣಿಕ ಸಂನ್ಯಾಸಿಗಳು, ಆ ತಾರುಣ್ಯದಲ್ಲೇ ಅವರಲ್ಲಿ ಪರಿಪೂರ್ಣವಾಗಿದ್ದ ಸಂನ್ಯಾಸದ ಪರಮಾದರ್ಶ ವನ್ನು ಕಂಡು ಆಕರ್ಷಿತರಾಗಿ, ಅವರನ್ನು ಪ್ರೀತಿಸಿ ಆದರಿಸುತ್ತಿದ್ದರು. ಉತ್ತಮರಾದವರು ಅವರ ದಿವ್ಯ ತೇಜೋನ್ವಿತ ಮುಖಮಂಡಲವನ್ನು ಕಂಡೊಡನೆಯೆ ಅವರ ಮಹಿಮೆಯನ್ನು ಗುರುತಿಸುತ್ತಿ ದ್ದರು. ಒಮ್ಮೆ ಸ್ವಾಮೀಜಿ ಹಿಮಾಲಯದಲ್ಲಿ ಸಂಚರಿಸುತ್ತಿದ್ದಾಗ, ಎದುರಿನಿಂದ ಬಂದ ಸಂನ್ಯಾಸಿ ಯೊಬ್ಬ ಅವರನ್ನು ಕಂಡು ಇದ್ದಕ್ಕಿದ್ದಂತೆ ಉದ್ವೇಗಗೊಂಡು “ಓ! ಶಿವ, ಶಿವ!” ಎಂದು ಉದ್ಗರಿಸಿ ನಮಸ್ಕರಿಸಿದನಂತೆ!

ಸಂನ್ಯಾಸಿಗಳೆಂದರೆ ಭಾರತದ ಅತ್ಯುನ್ನತ ಆದರ್ಶಗಳನ್ನು ಧರಿಸಿ ಪರಿಪಾಲಿಸುವವರು ಎಂಬ ಭಾವನೆಯಿಂದ ಅವರನ್ನು ಸ್ವಾಮೀಜಿ ಅತ್ಯಂತ ಭಕ್ತಿ ಗೌರವಗಳಿಂದ ಕಾಣುತ್ತಿದ್ದರು. ಈ ದೃಷ್ಟಿ ಯಿಂದ ಅವರು ಅದೆಷ್ಟು ಸಂನ್ಯಾಸಿಗಳಿಗೆ ಸೇವೆ ಸಲ್ಲಿಸಿದರೋ! ಹಿಮಾಲಯ ಪ್ರದೇಶದಲ್ಲಿ ಒಂದು ದಿನ ಅವರು ವೃದ್ಧ ಸಂನ್ಯಾಸಿಯೊಬ್ಬರನ್ನು ಕಂಡರು. ಎಲ್ಲ ಬಗೆಯ ಆಸರೆ-ಅನೂಕಲತೆ ಗಳನ್ನು ನಿರ್ಲಕ್ಷಿಸಿ, ಸಿಂಹದಂತೆ ಜೀವಿಸುವ ವ್ರತ ತೊಟ್ಟವರು ಈ ಸಂನ್ಯಾಸಿ. ಆದರೆ ಈಗ ವೃದ್ಧಾಪ್ಯದಲ್ಲಿ, ವಿಪರೀತ ಚಳಿಯಿಂದಾಗಿ ಅವರಿಗೆ ತೀವ್ರ ಅನಾರೋಗ್ಯವಾಗಿತ್ತು. ಅವರ ದುಸ್ಥಿತಿ ಯನ್ನು ಕಂಡು ಸ್ವಾಮೀಜಿ ತಮ್ಮ ಬಳಿಯಿದ್ದ ಒಂದೇ ಒಂದು ಕಂಬಳಿಯನ್ನು ತೆಗೆದು ಅವರಿಗೆ ಹೊದಿಸಿದರು. ಆಗ ಆ ವೃದ್ಧ ಸಂನ್ಯಾಸಿ ತಲೆಯೆತ್ತಿ ನೋಡಿ ಕೃತಜ್ಞತೆಯ ಮಂದಹಾಸ ಸೂಸುತ್ತ ಉದ್ಗರಿಸಿದರು, “ಮಗು, ನಿನಗೆ ನಾರಾಯಣ ಒಳ್ಳೆಯದು ಮಾಡಲಿ!”

ಸ್ವಾಮೀಜಿಯಿಂದ ಪ್ರಭಾವಿತರಾದ ಕೆಲವು ಸಂನ್ಯಾಸಿಗಳು ಅವರೊಂದಿಗೆ ತುಂಬ ಆತ್ಮೀಯ ಸಂಪರ್ಕವನ್ನು ಬೆಳೆಸಿಕೊಂಡು, ತಮ್ಮ ಅಂತರಂಗವನ್ನು ಅವರ ಮುಂದೆ ಬಿಚ್ಚಿಡುತ್ತಿದ್ದರು. ತಮ್ಮ ಪೂರ್ವಾಶ್ರಮದ ವಿಷಯಗಳನ್ನೂ, ಪಾಪಕೃತ್ಯಗಳನ್ನೂ ಹೇಳಿಕೊಂಡು ಪಶ್ಚಾತ್ತಾಪ ವ್ಯಕ್ತಪಡಿಸುತ್ತಿದ್ದರು. ಹೃಷೀಕೇಶದಲ್ಲಿ ಸ್ವಾಮೀಜಿ, ಆಧ್ಯಾತ್ಮಿಕವಾಗಿ ಬಹಳ ಮುಂದುವರಿದ ಸಂನ್ಯಾಸಿಯೊಬ್ಬರನ್ನು ಭೇಟಿಯಾದರು. ಅವರೊಬ್ಬ ತಪೋಮಹಿಮರೆಂಬುದನ್ನು ಅವರ ಮುಖ ಲಕ್ಷಣವೇ ಸಾರುತ್ತಿತ್ತು. ಸ್ವಾಮೀಜಿಯೊಂದಿಗೆ ಮಾತನಾಡುವ ಸಂದರ್ಭದಲ್ಲಿ ಅವರೊಮ್ಮೆ ಭಾವಾವೇಶಭರಿತರಾಗಿ ಕೇಳಿದರು.

“ನೀವು ಪವಾಹಾರಿ ಬಾಬಾರ ಬಗ್ಗೆ ಕೇಳಿರಬೇಕಲ್ಲವೆ?”

“ಹೌದು ಹೌದು; ಅವರೊಬ್ಬ ಮಹಾತಪಸ್ವಿಗಳು; ಬಹಳ ಅಪೂರ್ವ ವ್ಯಕ್ತಿ.”

“ಹಾಗಾದರೆ, ಅವರ ಆಶ್ರಮಕ್ಕೆ ಕದಿಯಲು ಬಂದಿದ್ದ ಕಳ್ಳನ ಕತೆ ನಿಮಗೆ ಗೊತ್ತಿರಬಹುದು. ಆ ಕಳ್ಳ ತಾನು ಕದ್ದ ವಸ್ತುಗಳೊಂದಿಗೆ ಓಡಿ ಹೋಗುತ್ತಿರುವುದನ್ನು ಕಂಡು, ‘ನಿಲ್ಲು, ನಿಲ್ಲು’ ಎನ್ನುತ್ತ ಬಾಬಾಜಿಯವರು ಅವನನ್ನು ಅಟ್ಟಿಸಿಕೊಂಡು ಹೋದರು. ಕಳ್ಳ ಹೆದರಿ, ತಾನು ಹಿಡಿದಿದ್ದ ಪಾತ್ರೆಗಳನ್ನೆಲ್ಲ ಅಲ್ಲೇ ಬಿಸಾಡಿ ಮುಂದೋಡಿದ. ಆಗ ಬಾಬಾಜಿಯವರು ಅವುಗಳನ್ನು ಎತ್ತಿ ಕೊಂಡು, ಪುನಃ ಅವನ ಬೆನ್ನಟ್ಟಿ ಅವನನ್ನು ಹಿಡಿದು, ‘ಅಯ್ಯಾ, ಇವೆಲ್ಲ ನಿನಗೆ ಸೇರಿದ್ದು; ದಯ ವಿಟ್ಟು ತೆಗೆದುಕೊಳ್ಳಬೇಕು. ನೀನು ಸಾಕ್ಷಾತ್ ನಾರಾಯಣ!’ ಎನ್ನುತ್ತ ಅವುಗಳನ್ನೆಲ್ಲ ಅವನಿಗೇ ಕೊಟ್ಟರು. ಇದೆಲ್ಲ ನಿಮಗೆ ತಿಳಿದಿರಬಹುದು?”

“ಓಹೋ, ಈ ಕತೆ ನನಗೆ ಚೆನ್ನಾಗಿ ನೆನಪಿದೆ. ಪವಹಾರಿ ಬಾಬಾ ನಿಜಕ್ಕೂ ತುಂಬ ಅದ್ಭುತ! ಕಳ್ಳನನ್ನೇ ನಾರಾಯಣ ಎಂದು ತಿಳಿದವರು!”

“ಸ್ವಾಮೀಜಿ, ನಾನೇ... ಆ ಕಳ್ಳ!”

ಆಶ್ಚರ್ಯದಿಂದ ಸ್ವಾಮೀಜಿಯ ಬಾಯಲ್ಲಿ ಮಾತೇ ಹೊರಡಲಿಲ್ಲ. ಆ ಸಾಧುಗಳು ತಮ್ಮ ಕತೆಯನ್ನು ಮುಂದುವರಿಸಿದರು.

“ನಾನು ನನ್ನ ದುಷ್ಟತನವನ್ನು ಕಂಡುಕೊಂಡೆ. ನನ್ನ ಬಗ್ಗೆ ನನಗೇ ತುಂಬ ಅಸಹ್ಯವಾಯಿತು. ಸಣ್ಣಪುಟ್ಟ ವಸ್ತುಗಳನ್ನು ಕದ್ದು ಜೀವಿಸುವುದಕ್ಕಿಂತ ಎಲ್ಲಕ್ಕಿಂತ ಅಮೂಲ್ಯ ನಿಧಿಯಾದ ಭಗವಂತ ನನ್ನೇ ಪಡೆಯೋಣವೆಂದು ನಿರ್ಧರಿಸಿ ನಾನು ಈ ಜೀವನವನ್ನು ಕೈಗೊಂಡೆ.”

ಹೀಗೆ ಗಂಟೆಗಟ್ಟಲೆ ಸಂಭಾಷಣೆ ಮುಂದುವರಿಯಿತು. ತಾವು ಸಾಧುಜೀವನವನ್ನು ಕೈಗೊಂಡ ಮೇಲೆ ಮಾಡಿದ ಸಾಧನೆಗಳನ್ನೂ ಪಡೆದುಕೊಂಡ ಅನುಭವಗಳನ್ನೂ ವಿವರಿಸಿ, ತಮ್ಮ ಅಧ್ಯಾತ್ಮಿಕ ಜ್ಞಾನವನ್ನು ಸ್ವಾಮೀಜಿಯ ಮುಂದಿಟ್ಟರು. ಅಂದು ರಾತ್ರಿ ಇಬ್ಬರೂ ಪರಸ್ಪರ ಬೀಳ್ಕೊಂಡಾಗ ಮಧ್ಯರಾತ್ರಿ ಮೀರಿತ್ತು; ಇಬ್ಬರ ಹೃದಯದಲ್ಲೂ ಅಪೂರ್ವ ಶಾಂತಿ ನೆಲಸಿತ್ತು.

ಈ ಘಟನೆಯನ್ನು ಸ್ವಾಮೀಜಿ ಎಂದೂ ಮರೆಯಲಿಲ್ಲ. ಇದು ಅವರಿಗೊಂದು ಹೊಸ ಅನುಭವ. ಹೊಸ ಪಾಠ. ಬಹಳ ದಿನಗಳವರೆಗೆ ಅವರು ಇದನ್ನೇ ಮೆಲಕು ಹಾಕುತ್ತಿದ್ದರು. ದುಷ್ಟರ ಶಕ್ತಿಯನ್ನು ಸದಾಚಾರದ ಕಡೆಗೆ ತಿರುಗಿಸಿದರೆ, ಅದೇ ಶಕ್ತಿಯೇ ಉಪಯೋಗಕರವಾಗಿ ಮಾರ್ಪಡು ವುದೆಂಬುದನ್ನು ಅವರು ಗುರುತಿಸಿದರು. ‘ಶಕ್ತಿಶಾಲಿಯೂ ಪೌರುಷವಂತನೂ ಆದ ಒಬ್ಬ ನೀಚ ನನ್ನೂ ನಾನು ಗೌರವಿಸುತ್ತೇನೆ’ ಎಂಬ ಅವರ ಮಾತಿನ ಅರ್ಥವನ್ನು ಇಲ್ಲಿ ಕಾಣಬಹುದಾಗಿದೆ. ಆದ್ದರಿಂದ ‘ಪಾಪ’ವನ್ನೂ ‘ಪಾಪಿ’ಯನ್ನೂ ವಾಚಾಮಗೋಚರವಾಗಿ ದೂಷಿಸುವವರ, ತಮ್ಮನ್ನು ಕ್ರೈಸ್ತರೆಂದು ಕರೆದುಕೊಳ್ಳುವವರ ದೇಶದಲ್ಲಿ ಸ್ವಾಮೀಜಿ “ಪಾಪಿಗಳು ಎಂದರೆ ಸಂತರಾಗುವ ಸಾಧ್ಯತೆಯಡಗಿರುವವರು” ಎಂದು ಘೋಷಿಸಿ ತಮ್ಮ ‘ನೂತನ ವೇದಾಂತ’ವನ್ನು ಸಾರುತ್ತಾರೆ.

ಪೋಲಿಸರೊಂದಿಗಿನ ಸ್ವಾಮೀಜಿಯ ಮತ್ತೊಂದು ಅನುಭವ ಹಿಂದೆ ಹೇಳಿದವುಗಳಿಗಿಂತ ಸ್ವಲ್ಪ ವಿಭಿನ್ನವಾದದ್ದು. ಉತ್ತರ ಭಾರತದ ಒಂದು ತೀರ್ಥಕ್ಷೇತ್ರದಲ್ಲಿದ್ದಾಗ ಅವರು ಒಬ್ಬ ಪೋಲೀಸ್ ಇನ್ಸ್​ಪೆಕ್ಟರನ ಅತಿಥಿಯಾಗಿದ್ದರು. ಈತ ಪೋಲೀಸ್ ಇಲಾಖೆಯವನಾದರೂ ಧಾರ್ಮಿಕ ಸ್ವಭಾವದವನಾಗಿದ್ದ. ಪ್ರತಿದಿನ ಶಾಸ್ತ್ರಗ್ರಂಥಗಳ ಪಾರಾಯಣ ಮಾಡುತ್ತಿದ್ದ. ಈತ ಸ್ವಾಮೀಜಿಯನ್ನು ತುಂಬ ಮೆಚ್ಚಿಕೊಂಡು, ಭಕ್ತಿಗೌರವಗಳಿಂದ ಕಾಣುತ್ತಿದ್ದ.

ಕೆಲವು ದಿನಗಳಾಗುವಷ್ಟರಲ್ಲಿ ಸ್ವಾಮೀಜಿ ತಮ್ಮ ಆತಿಥೇಯನ ಮನೆಯಲ್ಲಿ ಒಂದು ವೈಚಿತ್ರ್ಯ ವನ್ನು ಗಮನಿಸಿದರು. ಆ ಇನ್ಸ್​ಪೆಕ್ಟರನಿಗೆ ಬರುತ್ತಿದ್ದ ಸಂಬಳ ತಿಂಗಳಿಗೆ ನೂರಿಪ್ಪತ್ತೈದು ರೂಪಾಯಿ. ಆದರೆ ಅವನು ನಡೆಸುತ್ತಿದ್ದ ಜೀವನವನ್ನು ನೋಡಿದರೆ ಅವನ ಆದಾಯ ಏನಿಲ್ಲ ವೆಂದರೂ ಅದರ ಎರಡರಷ್ಟಾದರೂ ಇರಬೇಕೆಂಬಂತೆ ತೋರುತ್ತಿತ್ತು. ಅವನ ಪರಿಚಯ ಇನ್ನಷ್ಟು ನಿಕಟವಾದಾಗ ಸ್ವಾಮೀಜಿ ಅವನನ್ನು ಕೇಳಿದರು:

“ಏನಯ್ಯ, ನೀನು ಇಷ್ಟೆಲ್ಲ ಖರ್ಚು ಹೇಗೆ ನಿಭಾಯಿಸುತ್ತೀ? ನಿನಗೆ ಬರುವ ಸಂಬಳದಲ್ಲಿ ಅಷ್ಟೆಲ್ಲ ಖರ್ಚು ಮಾಡಲು ಸಾಧ್ಯವೆ?”

ಇನ್ಸ್​ಪೆಕ್ಟರ್ ನಕ್ಕು ತನ್ನ ಯಶಸ್ಸಿನ ಗುಟ್ಟನ್ನು ಅರುಹಿದ:

“ಸ್ವಾಮೀಜಿ, ನಿಮ್ಮಂತಹ ಸಾಧುಗಳೇ ನನಗೆ ಸಹಾಯ ಮಾಡುವವರು!”

ಸಾಧುಗಳು ಸಹಾಯ ಮಾಡುವವರು?! ಸಾಧುಗಳು ಏನಾದರೂ ಆಶೀರ್ವಾದ ಮಾಡ ಬಹುದು. ಅಥವಾ ಹೆಚ್ಚೆಂದರೆ ದೇವರು-ಧರ್ಮ ಎಂದೇನಾದರೂ ಉಪದೇಶ ಮಾಡಬಹುದು. ಆದರೆ ಅವರೇನು ದುಡ್ಡು-ಕಾಸು ಕೊಟ್ಟು ಸಹಾಯ ಮಾಡಬಲ್ಲರೆ?

“ನಿನ್ನ ಮಾತು ಅರ್ಥವಾಗಲಿಲ್ಲ...”

“ನೋಡಿ ಸ್ವಾಮೀಜಿ, ಈ ಸ್ಥಳಕ್ಕೆ ಎಷ್ಟೋ ಜನ ಸಾಧುಬೈರಾಗಿಗಳು ಬರುತ್ತಾರೆ. ಆದರೆ ಎಲ್ಲರೂ ನಿಮ್ಮ ತರಹ ಒಳ್ಳೆಯವರೇ ಆಗಿರುವುದಿಲ್ಲ. ಕಪಟಸಂನ್ಯಾಸಿಗಳು, ಕಳ್ಳಕಾಕರು, ಮೋಸ ಗಾರರು–ಹೀಗೆ ಎಷ್ಟೋ ತರದ ಜನಗಳು ಸಾಧುಗಳ ವೇಷ ಹಾಕಿಕೊಂಡು ತಿರುಗುತ್ತಿರುತ್ತಾರೆ. ನಾನು ಎಲ್ಲರನ್ನೂ ಗಮನಿಸಿ ನೋಡುತ್ತಾ ಇರುತ್ತೇನೆ. ಯಾರ ಮೇಲಾದರೂ ಗುಮಾನಿ ಬಂದರೆ ಅವರನ್ನು ಹಿಡಿದುಹಾಕಿ, ಹತ್ತಿರ ಇರುವುದನ್ನೆಲ್ಲ ತಪಾಸಣೆ ಮಾಡಿ ನೋಡುತ್ತೇನೆ. ಎಷ್ಟೋ ಸಲ ಈ ತೋರಿಕೆಯ ಸಾಧುಗಳು ಬಹಳಷ್ಟು ಹಣವನ್ನು ಎಲ್ಲೆಲ್ಲೋ ವಿಚಿತ್ರವಾಗಿ ಬಚ್ಚಿಟ್ಟುಕೊಂಡಿರು ತ್ತಾರೆ. ಯಾರು ಹಣವನ್ನು ಅಕ್ರಮವಾಗಿ ಹೊಂದಿದ್ದಾರೆ ಅಂತ ನನಗನ್ನಿಸುತ್ತದೆಯೋ ಅಂಥ ವರನ್ನು ಲಾಠಿ ತೋರಿಸಿ ಸ್ವಲ್ಪ ಬೆದರಿಸುತ್ತೇನೆ. ಬೀಸುವ ದೊಣ್ಣೆ ತಪ್ಪಿದರೆ ಸಾಕು ಅಂತ ಅವರು ಹಣವನ್ನೆಲ್ಲ ಬಿಟ್ಟು ಕೈಮುಗಿದು ಓಡಿಹೋಗುತ್ತಾರೆ.”

ಹೀಗೆ ಹೇಳಿದ ಆರಕ್ಷಕ, ಹೆಮ್ಮೆಯಿಂದ ಮತ್ತೊಂದು ವಿಚಾರವನ್ನು ತಿಳಿಸಿದ:

“ಆದರೆ ಸ್ವಾಮೀಜಿ, ನಿಮ್ಮ ಮುಂದೆ ಪ್ರಮಾಣ ಮಾಡಿ ಹೇಳುತ್ತೇನೆ, ನಾನು ಇನ್ಯಾವ ರೀತಿ ಯಲ್ಲೂ ಲಂಚ ತೆಗೆದುಕೊಳ್ಳುವುದಿಲ್ಲ!”

ಹಿಮಾಲಯ ಪ್ರದೇಶದಲ್ಲಿ ಸ್ವಾಮೀಜಿಗಾದ ಹೊಸ ಅನುಭವಗಳಲ್ಲೊಂದು: ಆಗ ಅವರು ಟಿಬೆಟೀ ಕುಟುಂಬವೊಂದರಲ್ಲಿ ಉಳಿದುಕೊಂಡಿದ್ದರು. ಟಿಬೆಟಿಗರಲ್ಲಿ ‘ಬಹುಪತಿತ್ವ’ ಆಚರಣೆ ಯಲ್ಲಿದೆ. (ಬಹುಪತಿತ್ವ ಎಂದರೆ ಒಬ್ಬ ಹೆಂಗಸಿಗೆ ಒಬ್ಬನಿಗಿಂತ ಹೆಚ್ಚಿನ ಗಂಡಂದಿರಿರುವ ಪದ್ಧತಿ. ಸಮಾಜದಲ್ಲಿ ಸ್ತ್ರೀಯರ ಸಂಖ್ಯೆ ಕಡಿಮೆಯಿರುವಲ್ಲಿ ಈ ಪದ್ಧತಿ ಕಂಡುಬರುತ್ತದೆ.) ಸ್ವಾಮೀಜಿ ಈ ಪದ್ಧತಿಯ ಬಗ್ಗೆ ಕೇಳಿದ್ದರೂ ಕಣ್ಣಾರೆ ಕಂಡಿರಲಿಲ್ಲ. ಅವರು ಉಳಿದುಕೊಂಡಿದ್ದ ಕುಟುಂಬದಲ್ಲಿ ಆರು ಜನ ಅಣ್ಣ ತಮ್ಮಂದಿರು–ಇವರೆಲ್ಲರಿಗೂ ಸೇರಿ ಒಬ್ಬಳೇ ಹೆಂಡತಿ. ಇದನ್ನು ಕಂಡು ಸ್ವಾಮೀಜಿಗೆ ಬಹಳ ಜುಗುಪ್ಸೆಯಾಯಿತು. ಆದ್ದರಿಂದ ಆ ಅಣ್ಣತಮ್ಮಂದಿ ರೊಂದಿಗೆ ಸ್ವಲ್ಪ ಸಲಿಗೆ ಬೆಳೆದಾಗ ಬಹುಪತಿತ್ವವನ್ನು ಖಂಡಿಸಿ ಅವರಿಗೆ ತಿಳಿಯಹೇಳಿದರು. ಆದರೆ ಆ ಸೋದರರಿಗೆ ಸ್ವಾಮೀಜಿಯ ವಾದ ಬಹಳ ವಿಚಿತ್ರವಾಗಿ ಕಂಡಿತು. ಅವರಲ್ಲೊಬ್ಬ ಸ್ವಲ್ಪ ಕೋಪದಿಂದಲೇ ಹೇಳಿದ, “ಸ್ವಾಮೀಜಿ, ನೀವೊಬ್ಬರು ಸಂನ್ಯಾಸಿಗಳಾಗಿ, ಜನರಿಗೆ ನೀವೇ ಇಂತಹ ಸ್ವಾರ್ಥಪರತೆಯನ್ನು ಬೋಧಿಸುತ್ತೀದ್ದೀರಲ್ಲ! ‘ಈ ವಸ್ತು ನನಗೆ ಮಾತ್ರ ಸೇರಿದ್ದು, ಬೇರೆ ಯಾರಿಗೂ ಅಲ್ಲ’ ಎಂದು ಭಾವಿಸುವುದೇ ತಪ್ಪಲ್ಲವೆ? ನಾವು ಒಬ್ಬೊಬ್ಬರೂ ಒಬ್ಬೊಬ್ಬಳು ಹೆಂಡತಿಯನ್ನು ಇಟ್ಟುಕೊಳ್ಳುವಷ್ಟು ಸ್ವಾರ್ಥಿಗಳೇಕೆ ಆಗಬೇಕು? ಸ್ವಾಮೀಜಿ, ಅಣ್ಣತಮ್ಮಂದಿರು ತಮ್ಮಲ್ಲಿರುವ ಪ್ರತಿಯೊಂದು ವಸ್ತುವನ್ನೂ ಹಂಚಿಕೊಳ್ಳಬೇಕು–ಹೆಂಡತಿಯನ್ನೂ ಕೂಡ.”

ಈ ವಾದವನ್ನು ಕೇಳಿ ಸ್ವಾಮೀಜಿ ಬೆರಗಾದರು. ಮತ್ತೆ ಅವರು ತಮ್ಮ ಮಾತನ್ನು ಮುಂದು ವರಿಸಲಿಲ್ಲ. ಈ ವಾದದ ಯುಕ್ತಾಯುಕ್ತತೆಗಳೇನೇ ಇರಲಿ; ಆ ಸರಳ ಸ್ವಭಾವದ ಗುಡ್ಡಗಾಡು ಜನರ ಉತ್ತರ, ಅವರನ್ನು ಆಲೋಚಿಸುವಂತೆ ಮಾಡಿತು. ಬಹುಶಃ ಇಂತಹ ಘಟನೆಗಳೇ ಅವರನ್ನು ತಾವು ದೇಶದ ನಾನಾ ಭಾಗಗಳಲ್ಲಿ ಕಂಡ ಬೇರೆ ಬೇರೆ ಜನರ ಸಂಪ್ರದಾಯ-ನಡವಳಿಕೆ ಗಳ ಕುರಿತಾಗಿ ತೀವ್ರವಾಗಿ ಚಿಂತಿಸುವಂತೆ ಮಾಡಿದುವು. ಇದರಿಂದ ಅವರ ದೃಷ್ಟಿಕೋನ ವಿಶಾಲ ಗೊಂಡಿತು. ಅಲ್ಲದೆ ಇದರಿಂದಾಗಿ ಪ್ರಪಂಚದ ಬೇರೆಬೇರೆ ಜನಾಂಗಗಳ ಸಾಮಾಜಿಕ ಹಾಗೂ ನೈತಿಕ ಮಾನದಂಡಗಳನ್ನು, ಆಯಾ ಜನಾಂಗದ ದೃಷ್ಟಿಕೋನಗಳಿಂದಲೇ ನೋಡಲು ಅವರಿಗೆ ಸಾಧ್ಯವಾಯಿತು.

ಹೀಗೆ ಈ ಬಗೆಯ ಹಲವಾರು ಅನುಭವಗಳು ಅವರ ವ್ಯಕ್ತಿತ್ವದಲ್ಲಿ, ಅವರ ಆಲೋಚನಾ ವಿಧಾನದಲ್ಲಿ ಒಂದು ಮಹತ್ತರ ಬದಲಾವಣೆಯನ್ನುಂಟುಮಾಡಿದುವು. ಈ ಐದಾರು ವರ್ಷಗಳ ಪರಿವ್ರಾಜಕ ಜೀವನವು ಅವರ ಪಾಲಿಗೊಂದು ತೀರ್ಥಯಾತ್ರೆ ಮಾತ್ರವಲ್ಲದೆ ಒಂದು ಶೈಕ್ಷಣಿಕ ಪ್ರವಾಸವಾಗಿ ಪರಿಣಮಿಸಿತು. ಅದು ಅವರಿಗೆ ಎಷ್ಟೋ ವಿಚಾರಗಳ ನೇರ ಅನುಭವ ಮಾಡಿಸಿ ಕೊಟ್ಟು, ಅವರ ವ್ಯಕ್ತಿತ್ವವನ್ನು ಮತ್ತಷ್ಟು ಗಳಿತವಾಗಿಸಿತು. ಅವರು ಬಾಲ್ಯದಿಂದಲೂ ಅಧ್ಯಯನಾದಿಗಳ ಮೂಲಕ ಗಳಿಸಿಕೊಂಡಿದ್ದ ಜ್ಞಾನಭಂಡಾರವನ್ನು ಅಪಾರವಾಗಿ ಹಿಗ್ಗಿಸಿತು. ಅಂದು ದಕ್ಷಿಣೇಶ್ವರದಲ್ಲಿ ಶ್ರೀರಾಮಕೃಷ್ಣರಿಂದ ಅವರು ಯಾವ ಅದ್ಭುತ ವಿಚಾರಗಳನ್ನೆಲ್ಲ ಅರಿತುಕೊಂಡಿದ್ದರೋ ಅವು ಎಲ್ಲಿ ಹೇಗೆ ಅನುಷ್ಠಾನಗೊಳ್ಳಬೇಕಾಗಿದೆಯೆಂಬುದನ್ನು ಕ್ರಮೇಣ ಕಂಡುಕೊಂಡರು. ಶ್ರೀರಾಮಕೃಷ್ಣರು ಮಾಡಿಸಿಕೊಟ್ಟ ಅನುಭವಗಳನ್ನೆಲ್ಲ ಯುಕ್ತಾಯುಕ್ತತೆ ಯರಿತು, ದೇಶ-ಕಾಲ-ಪಾತ್ರಗಳಿಗನುಗುಣವಾಗಿ ಬೋಧಿಸಲು ಈ ಪರಿವ್ರಾಜಕ ಜೀವನದ ಅನುಭವಗಳು ದಿಕ್ಸೂಚಿಯಾದುವು.

ಅವರ ಸೋದರಸಂನ್ಯಾಸಿಯೊಬ್ಬರು ಹೇಳುವಂತೆ, ಈ ದಿನಗಳಲ್ಲಿ ಸ್ವಾಮೀಜಿ ಯಾವಾಗಲೂ ಹೊಸ ಹೊಸ ಅನುಭವಗಳನ್ನು ಅರಸುತ್ತಿದ್ದರು. ಸದಾ ಹೊಸ ಹೊಸ ವಿಚಾರಗಳನ್ನು ಸಂಗ್ರಹಿಸಿ ಅವುಗಳ ಪರಸ್ಪರ ಸಾಮ್ಯ-ಭೇದಗಳನ್ನು ವಿಶ್ಲೇಷಿಸುತ್ತಿದ್ದರು. ಭಾರತದ ಪ್ರತಿಯೊಂದು ಪ್ರಾಂತ್ಯದ ಸಾಮಾಜಿಕ ಹಾಗೂ ಧಾರ್ಮಿಕ ವಿಚಾರಗಳ ವಿವರಗಳು ಅವರಿಗೆ ಸಂಪೂರ್ಣ ಕರಗತ ವಾಗಿದ್ದು, ಯಾವುದೇ ಪ್ರದೇಶದ ಜನಗಳ ಆಚಾರ-ವಿಚಾರ-ಸಂಪ್ರಾಯಗಳ ಬಗ್ಗೆ ಸಂಪೂರ್ಣ ಮಾಹಿತಿಯನ್ನು ನೀಡಬಲ್ಲವರಾಗಿದ್ದರು. ವಿವಿಧ ಮತ ಧರ್ಮಗಳ ಅಪೂರ್ವ ಶಾಸ್ತ್ರಗ್ರಂಥಗಳನ್ನು ಅಧ್ಯಯನ ಮಾಡುವ, ಅಥವಾ ಆ ಧರ್ಮಗಳ ಪರಿಣತರೊಂದಿಗೆ ಚರ್ಚಿಸುವ ಯಾವ ಅವಕಾಶವನ್ನೂ ಅವರು ಕಳೆದುಕೊಳ್ಳುತ್ತಿರಲಿಲ್ಲ.

ಭಾರತದ ವಿಭಿನ್ನಮಯ ಆದರ್ಶಗಳ ಜಗತ್ತಿನಲ್ಲಿ ಸೂತ್ರಪ್ರಾಯವಾದ ಏಕತೆ ಯಾವುದು ಎಂದು ಅವರು ನಿರಂತರವಾಗಿ ಶೋಧಿಸುತ್ತಿದ್ದರು. ಕಡೆಗೆ ಅವರು ಕಂಡುಕೊಂಡರು–ಈ ಎಲ್ಲ ವಿಭಿನ್ನ ಆಚರಣೆಗಳ ಹಾಗೂ ಸಂಪ್ರದಾಯಗಳ ಮೂಲದಲ್ಲಿರುವುದು ಅಧ್ಯಾತ್ಮಿಕತತ್ತ್ವದ ಏಕತೆಯೇ, ಎಂದು. ಅಲ್ಲದೆ ಹಿಂದೂ-ಮುಸಲ್ಮಾನರ ನಡುವೆ ಕಂಡುಬರುವ ಅಂತರವು ಬಹುತೇಕ ತೋರಾಣಿಕೆಯದು ಎಂದವರು ನಿಶ್ಚಯಿಸಿದರು. ಮುಸಲ್ಮಾನರನ್ನು ಒಟ್ಟಾರೆಯಾಗಿ ಪರಿಗಣಿಸಿದರೆ, ಅವರು ಹಿಂದೂಗಳಷ್ಟೇ ಉದಾರಿಗಳು, ಮನುಷ್ಯತ್ವವುಳ್ಳವರು ಹಾಗೂ ‘ಭಾರತೀ ಯರು’ ಎಂಬುದನ್ನು ಸ್ವಾಮೀಜಿ ಕಣ್ಣಾರೆ ಕಂಡರು. ಹಿಂದೂ ಧರ್ಮಸಂಸ್ಕೃತಿ-ಸಂಪ್ರದಾಯ ಗಳನ್ನು ಹೃತ್ಪೂರ್ವಕವಾಗಿ ಮೆಚ್ಚಿಕೊಳ್ಳುವ ವಿವೇಕೀ ಮುಸ್ಲಿಮರೂ ಇದ್ದಾರೆಂಬುದನ್ನು ಅವರು ಗುರುತಿಸಿದ್ದರು.

ಬರಬರುತ್ತ ಅವರ ವಿಚಾರದ ದಿಗಂತ ಎಷ್ಟು ವಿಶಾಲವಾಯಿತೆಂದರೆ, ಧರ್ಮದ ಎಲ್ಲ ಪ್ರಭೇದಗಳಲ್ಲೂ ಅಂತರ್ಗತವಾದ ಅದೇ ಪರಿಪೂರ್ಣತೆಯನ್ನೇ ಕಾಣುತ್ತಿದ್ದರು. ಆದ್ದರಿಂದ ಸಕಲ ಮತಧರ್ಮಗಳ ತತ್ತ್ವಗಳನ್ನೂ ಅವುಗಳ ಅನುಯಾಯಿಗಳನ್ನೂ ಸಹಾನುಭೂತಿಯಿಂದ, ಸಮತಾಭಾವದಿಂದ ಕಾಣಬಲ್ಲವರಾಗಿದ್ದರು. ಶ್ರೀರಾಮಕೃಷ್ಣರು ಸಾಕ್ಷಾತ್ಕರಿಸಿಕೊಂಡು ಬೋಧಿಸಿದ್ದ ಧಾರ್ಮಿಕ, ಆಧ್ಯಾತ್ಮಿಕ ತತ್ತ್ವಗಳ ಮಹತ್ವವನ್ನು ಸಾಮಾಜಿಕ ಹಾಗೂ ರಾಜಕೀಯ ದೃಷ್ಟಿಕೋನದಿಂದ ಕಂಡು, ಅವುಗಳಿಗೆ ಹೊಸ ಅರ್ಥವನ್ನು ಕಂಡುಕೊಂಡಿದ್ದರು.

ಸಮಸ್ತ ಭಾರತೀಯರಲ್ಲಿ ಪರಸ್ಪರ ಭ್ರಾತೃಭಾವವನ್ನು ಜಾಗೃತಗೊಳಿಸಲು ಸಾಧ್ಯವಾದರೆ, ಮತ್ತು ಮುಖ್ಯವಾಗಿ ವಿದ್ಯಾವಂತ ವರ್ಗದಲ್ಲಿ ಬಡಜನರ ಬಗೆಗಿನ ಹೊಣೆಗಾರಿಕೆಯನ್ನು ಮನವರಿಕೆ ಮಾಡಿಸಲು ಸಾಧ್ಯವಾದರೆ, ಎಲ್ಲ ಜಾತಿಮತಗಳ ಜನರೂ ವ್ಯವಸ್ಥಿತ ಸಮಾಜ ಜೀವನದಲ್ಲಿ ಸಹಬಾಳ್ವೆ ನಡೆಸುವಂತಾಗುತ್ತದೆ ಎಂಬುದರಲ್ಲಿ ಸ್ವಾಮೀಜಿಗೆ ದೃಢ ವಿಶ್ವಾಸವಿತ್ತು. ಧರ್ಮವೆಂಬುದು ಒಬ್ಬನ ವೈಯಕ್ತಿಕ ವಿಚಾರವಾಗಿರಬೇಕು. ಮತ್ತು ಸರಕಾರವು ಅದರಲ್ಲಿ ಹಸ್ತಕ್ಷೇಪ ಮಾಡಬಾರದು; ಅಲ್ಲದೆ ಧರ್ಮವೂ ಸರಕಾರದ ವ್ಯವಹಾರದಲ್ಲಿ ತಲೆ ಹಾಕಬಾರದು ಎಂಬುದು ಅವರ ಅಭಿಪ್ರಾಯವಾಗಿತ್ತು. ಆ ಕಾಲದ ಅನೇಕ ಧಾರ್ಮಿಕ ಚಳವಳಿಗಳು ಪಾಶ್ಚಾತ್ಯ ಸಂಸ್ಕೃತಿಯಿಂದ ಹಾಗೂ ಪಾಶ್ಚಾತ್ಯ ಚಿಂತಕರ ಅಭಿಪ್ರಾಯಗಳಿಂದ ತೀವ್ರವಾಗಿ ಪ್ರಭಾವಿತವಾಗಿ ದ್ದುವು. ಆದ್ದರಿಂದ ಹಿಂದೂಧರ್ಮವನ್ನು ಮೇಲೆತ್ತಲು ಹೊರಟ ಈ ಚಳವಳಿಗಳು, ಬಹುಶಃ ತಮಗರಿವಿಲ್ಲದಂತೆಯೇ ಭಾರತದ ಈ ಆಧುನಿಕ ಸಂಕ್ರಮಣ ಯುಗವನ್ನು ತ್ರಿಶಂಕು ಸ್ಥಿತಿ ಗೊಯ್ಯುತ್ತಿವೆ ಎಂಬುದನ್ನವರು ಮನಗಂಡರು. ಧಾರ್ಮಿಕ ಆಚರಣೆ-ಸಂಪ್ರದಾಯಗಳಲ್ಲಿ ಕಾಳು ಯಾವುದು ಜಳ್ಳು ಯಾವುದು ಎಂಬುದನ್ನು ಗುರುತಿಸುವ ವಿವೇಕ ಪ್ರಜ್ಞೆಯು ಜನರಲ್ಲಿ ಉದಿಸಿ ದರೆ, ಸಂಕುಚಿತ ಬುದ್ಧಿಯಿಂದುಂಟಾದ ಕಹಿಭಾವನೆಗಳು ತಾವಾಗಿಯೇ ನಾಶವಾಗುತ್ತವೆ ಎಂದು ಅವರು ನಿರೀಕ್ಷಿಸಿದರು. ಆಧುನಿಕ ಅವಶ್ಯಕತೆಗಳಿಗನುಗುಣವಾಗಿ ಭಾರತದ ಪುರಾತನ ಸಂಸ್ಕೃತಿ ಯನ್ನು ಪುನರ್ಮೌಲ್ಯೀಕರಣಕ್ಕೊಳಪಡಿಸಿ ನವಚೇತನ ತುಂಬುವ ಮಹತ್ಕಾರ್ಯದಲ್ಲಿ ಸರ್ವರೂ ಒಂದಾಗುವ ಸುಮಹೂರ್ತವನ್ನು ಕಾತರದ ಕಣ್ಣಿನಿಂದ ಇದಿರು ನೋಡಿದರು.

\delimiter

ಹೀಗೆ, ನಾನಾ ಬಗೆಯ ಭಾವ ತುಮುಲಗಳನ್ನು ಮನದಲ್ಲಿ ಹೊತ್ತು, ನೂರಾರು ಅಮೂಲ್ಯ, ಅನುಭವಗಳ ಸಾರವನ್ನೇ ತುಂಬಿಕೊಂಡು, ಸ್ವಾಮೀಜಿ ರಾಮೇಶ್ವರದಿಂದ ಮುಂದಕ್ಕೆ ಹೊರಟರು. ಇಲ್ಲಿಗೆ ಅವರ ಪರಿವ್ರಾಜಕ ಜೀವನವು ಒಂದು ರೀತಿಯಲ್ಲಿ ಕೊನೆಗೊಂಡಿತ್ತು.

