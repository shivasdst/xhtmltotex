
\chapter{ಇಂಗ್ಲೆಂಡಿನಲ್ಲಿ ಕಾರ್ಯಾರಂಭ}

\noindent

ಸಹಸ್ರದ್ವೀಪೋದ್ಯಾನದಲ್ಲಿ ತಮ್ಮ ಶಿಷ್ಯರಿಗೆ ತರಬೇತಿ ನೀಡಿ ದೀಕ್ಷಾಬದ್ಧರನ್ನಾಗಿ ಮಾಡುವ ಮಹಾಕಾರ್ಯವನ್ನು ಮುಗಿಸಿದ ಮೇಲೆ ಸ್ವಾಮೀಜಿ ನ್ಯೂಯಾರ್ಕಿಗೆ ಮರಳಿದರು. ಇಲ್ಲಿಂದ ಅವರು ಫ್ರಾನ್ಸಿನ ಮೂಲಕ ಇಂಗ್ಲೆಂಡಿಗೆ ಹೋಗಲು ಸಿದ್ಧತೆಗಳನ್ನು ಮಾಡಿಕೊಂಡರು. ಇಂಗ್ಲೆಂಡಿ ನಲ್ಲಿ ತಮ್ಮ ಸಂದೇಶವನ್ನು ಪ್ರಸಾರ ಮಾಡಬೇಕೆಂಬ ಇಚ್ಛೆ ಬಹಳ ದಿನಗಳಿಂದಲೂ ಅವರ ಮನಸ್ಸಿನಲ್ಲಿತ್ತು. ಈಗ ಅವರ ಇಚ್ಛೆ ನೆರವೇರುವಂತಹ ಪರಿಸ್ಥಿತಿ ತಾನಾಗಿಯೇ ಬಂದೊದಗಿತ್ತು. ಇಂಗ್ಲೆಂಡಿಗೆ ಬರುವಂತೆ ಅವರಿಗೆ ಮೊದಲ ಆಹ್ವಾನ ಬಂದದ್ದು ಮಿಸ್ ಹೆನ್ರಿಟಾ ಮುಲ್ಲಾರ್ ಎಂಬವಳಿಂದ. ಈಕೆ ಸರ್ವಧರ್ಮ ಸಮ್ಮೇಳನದ ‘ಥಿಯೊಸಾಫಿಕಲ್ ಕಾಂಗ್ರೆಸ್​’ನಲ್ಲಿ ಮಾತನಾಡಿ ದ್ದಳು. ಆಗ ಇವಳು ಸ್ವಾಮೀಜಿಯನ್ನು ಭೇಟಿಯಾಗಿದ್ದಳು. ಮಿಸ್ ಮುಲ್ಲರಳ ‘ದತ್ತು ಪುತ್ರ’ನೂ ಸ್ವಾಮೀಜಿಯ ಹಳೆಯ ಪರಿಚಿತನೂ ಆದ ಅಕ್ಷಯಕುಮಾರ ಘೋಷ್ ಎಂಬವನು ೧೮೯೪ರ ಆಗಸ್ಟಿನಲ್ಲಿ ಮಿಸ್ ಮುಲ್ಲರಳ ಪರವಾಗಿ ಅವರಿಗೊಂದು ಆಮಂತ್ರಣ ಕಳಿಸಿ, ಲಂಡನ್ನಿಗೆ ಬಂದು ತಮ್ಮ ಅತಿಥಿಯಾಗಿ ಉಳಿದುಕೊಳ್ಳಬೇಕೆಂದು ಕೇಳಿಕೊಂಡಿದ್ದ. ಸ್ವಾಮೀಜಿ ಈ ಆಮಂತ್ರಣವನ್ನು ಅಂಗೀಕರಿಸಿದ್ದರು. ಮರುವರ್ಷದ ಜನವರಿ-ಫೆಬ್ರುವರಿಯ ವೇಳೆಗೆ ತಾವು ಅಲ್ಲಿಗೆ ಹೋಗಲು ಸಾಧ್ಯವಾಗಬಹುದೆಂದು ಅವರು ಭಾವಿಸಿದ್ದರು. ಆದರೆ ೧೮೯೫ರ ಆಗಸ್ಟಿನವರೆಗೂ ಅವರಿಗೆ ಅಮೆರಿಕದಿಂದ ಹೊರಡಲು ಸಾಧ್ಯವಾಗಲಿಲ್ಲ.

ಇಷ್ಟರಲ್ಲಿ ಲಂಡನ್ನಿನ ಇ. ಟಿ. ಸ್ಟರ್ಡಿ ಎಂಬವನಿಂದ ಸ್ವಾಮೀಜಿಗೆ ಮತ್ತೊಂದು ಆಹ್ವಾನ ಬಂದಿತು. ಇವನು ಹಿಂದೆ ಒಬ್ಬ ಥಿಯಾಸೊಫಿಸ್ಟನಾಗಿದ್ದ. ಅಲ್ಲದೆ ಕೆಲಕಾಲ ಭಾರತದಲ್ಲಿ ವಾಸ ವಾಗಿದ್ದು ಹಿಂದೂಧರ್ಮವನ್ನು ಅಧ್ಯಯನ ಮಾಡಿದ್ದ. ಈತ ಹಿಮಾಲಯದ ಆಲ್ಮೋರದಲ್ಲಿ ವಾಸವಾಗಿದ್ದಾಗ ಇವನಿಗೆ ಸ್ವಾಮಿ ವಿವೇಕಾನಂದರ ಸೋದರ ಸಂನ್ಯಾಸಿಗಳಾದ ಸ್ವಾಮಿ ಶಿವಾ ನಂದರ ಪರಿಚಯವಾಗಿತ್ತು. ಸ್ವಾಮಿ ಶಿವಾನಂದರ ಮೂಲಕ ಸ್ವಾಮೀಜಿಯ ಬಗ್ಗೆ ತಿಳಿದ ಸ್ಟರ್ಡಿ, ಇಂಗ್ಲೆಂಡಿಗೆ ಹಿಂದಿರುಗಿದ ಮೇಲೆ ಅವರೊಂದಿಗೆ ಪತ್ರವ್ಯವಹಾರವನ್ನು ಪ್ರಾರಂಭಿಸಿದ. ಆಗ ಅವರು ನ್ಯೂಯಾರ್ಕಿನಲ್ಲಿದ್ದರು. ಮಿಸ್ ಮುಲ್ಲರ್ ಸ್ವಾಮೀಜಿಯನ್ನು ಲಂಡನ್ನಿಗೆ ಆಹ್ವಾನಿಸಿರು ವುದು ತಿಳಿದಾಗ ಸ್ಟರ್ಡಿಯೂ ಅವರಿಗೊಂದು ಆದರದ ಆಹ್ವಾನವನ್ನು ಕಳಿಸಿ, ತನ್ನ ಅತಿಥಿ ಯಾಗಿರುವಂತೆ ಕೇಳಿಕೊಂಡ. ತಮ್ಮ ಸಂದೇಶವನ್ನು ಪ್ರಸಾರ ಮಾಡಲು ಲಂಡನ್ ತುಂಬ ಫಲವತ್ತಾದ ಕ್ಷೇತ್ರ ಎಂದು ಈತ ತಿಳಿಸಿದ. ಇದೇ ವೇಳೆಗೆ, ಸ್ವಾಮೀಜಿಯ ಆಪ್ತಶಿಷ್ಯರಾದ ಫ್ರಾನ್ಸಿಸ್ ಲೆಗೆಟ್ಟರು ಯೂರೋಪಿಗೆ ಹೊರಡಲಿದ್ದರು. ಪ್ಯಾರಿಸಿನಲ್ಲಿ ಅವರು ಬೆಸ್ಸಿ(ಮೆಕ್​ಲಾಡಳ ಸೋದರಿ)ಯನ್ನು ವಿವಾಹವಾಗಲಿದ್ದರು. ತಮ್ಮೊಂದಿಗೆ ಯೂರೋಪಿಗೆ ಬರುವಂತೆ ಅವರು ಸ್ವಾಮೀಜಿಯನ್ನು ಆಹ್ವಾನಿಸಿದರು.

ಹೀಗೆ ಲಂಡನ್ನಿಗೆ ಬರುವಂತೆ ತಮಗೆ ಅನೇಕ ಅನಿರೀಕ್ಷಿತ ಆಹ್ವಾನಗಳು ಬಂದಾಗ ಸ್ವಾಮೀಜಿ ಇಂಗ್ಲೆಂಡಿಗೆ ಹೋಗುವ ಬಗ್ಗೆ ಗಂಭೀರವಾಗಿ ಯೋಚಿಸಿದರು. ಅದು ಭಗವದಿಚ್ಛೆಯೇ ನಿಜ ಎಂಬುದು ಖಚಿತವಾದ ಮೇಲೆ ಅವರು ಹೊರಡಲು ನಿರ್ಧರಿಸಿದರು. ಇಂಗ್ಲೆಂಡಿನಲ್ಲಿ ವೇದಾಂತ ಪ್ರಸಾರ ಮಾಡುವ ಕಾರ್ಯ ನಡೆಯಬೇಕಾಗಿದೆ, ಮತ್ತು ಅದು ನಡೆಯುತ್ತದೆ ಎಂಬುದು ಅವರಿಗೆ ದೃಢವಾಯಿತು. ನ್ಯೂಯಾರ್ಕಿನಿಂದ ಹೊರಡುವ ದಿನವನ್ನು ಸ್ವಾಮೀಜಿ ಕಾತರದಿಂದ ನಿರೀಕ್ಷಿಸ ತೊಡಗಿದರು.

ಆಗಸ್ಟ್ ೧೭ರಂದು ಅವರು ಲೆಗೆಟ್, ಬೆಸ್ಸಿ ಸ್ಟರ್ಜಸ್ ಹಾಗೂ ಮಿಸ್ ಮೆಕ್​ಲಾಡ್​– ಇವರೊಂದಿಗೆ ಎಸ್. ಎಸ್. ಟ್ಯೂರಿನ್ ಎಂಬ ಹಡಗಿನಲ್ಲಿ ನ್ಯೂಯಾರ್ಕಿನಿಂದ ಪ್ಯಾರಿಸಿಗೆ ಹೊರಟು, ಇಪ್ಪತ್ತನಾಲ್ಕು ದಿನಗಳ ಆನಂದಾಯಕ ಪ್ರಯಾಣದ ನಂತರ ಪ್ಯಾರಿಸ್ಸನ್ನು ತಲುಪಿ ದರು. ಈ ಸಮುದ್ರಯಾನ ಅವರ ದಣಿದ ಮೆದುಳು-ನರಗಳಿಗೆ ವಿಶ್ರಾಂತಿ ನೀಡಿತು. ಅವರು ಪ್ಯಾರಿಸಿನಲ್ಲಿ ಎರಡು ವಾರಗಳಿಗೂ ಹೆಚ್ಚು ಕಾಲ ಇದ್ದರು. ಈ ಸಮಯದಲ್ಲಿ ಅವರು ಫ್ರೆಂಚ್ ಸಂಸ್ಕೃತಿಯ ಬಗ್ಗೆ ಸಾಧ್ಯವಾದಷ್ಟು ವಿಷಯಗಳನ್ನು ಕೇಳಿ ತಿಳಿದುಕೊಂಡರು. ಶ್ರೀಮತಿ ಸ್ಟರ್ಜಸ್ ಹಾಗೂ ಆಕೆಯ ಸೋದರಿ ಮೆಕ್​ಲಾಡ್ ಇವರಿಬ್ಬರಿಗೂ ಪ್ಯಾರಿಸ್ ಚಿರಪರಿಚಿತ. ಇವರು ಸ್ವಾಮೀಜಿಯನ್ನು ಪ್ಯಾರಿಸ್ ನಗರದ ವಿವಿಧ ವಸ್ತುಸಂಗ್ರಹಾಲಯಗಳಿಗೆ, ಚರ್ಚುಗಳಿಗೆ ಮತ್ತು ಚಿತ್ರಕಲಾ ಪ್ರದರ್ಶನಾಲಯಗಳಿಗೆ ಕರೆದೊಯ್ದರು. ಫ್ರಾನ್ಸ್ ದೇಶದಲ್ಲಿ ಕಲಾತ್ಮಿಕತೆಯು ಅದ್ಭುತ ವಾಗಿ ಬೆಳೆದಿರುವ ರೀತಿಯನ್ನು ಕಂಡು ಸ್ವಾಮೀಜಿ ಬಹಳ ಸಂತೋಷಗೊಂಡರು. ನೆಪೋಲಿ ಯನ್ನನ ಗೋರಿಯನ್ನು ಹಾಗೂ ಪ್ಯಾರಿಸಿನ ಇತರೆಡೆಗಳಲ್ಲಿ ಕಾಣಸಿಗುವ ಅವನ ಹಲವಾರು ಸ್ಮಾರಕಗಳನ್ನು ಕಂಡಾಗ ಅವರ ಹೃದಯ ಆ ವೀರನನ್ನು ಬಹಳವಾಗಿ ಮೆಚ್ಚಿಕೊಂಡಿತು. ಇಲ್ಲಿ ಅವರ ಜೊತೆಗಾರರು ಅವರಿಗೆ ತಮ್ಮ ಸ್ನೇಹಿತರಾದ ಅನೇಕ ವಿದ್ವಾಂಸರನ್ನು ಪರಿಚಯ ಮಾಡಿಸಿ ಕೊಟ್ಟರು. ಅವರು ಅನೇಕ ವಿದ್ವತ್ಪೂರ್ಣ ವಿಷಯಗಳಿಂದ ಹಿಡಿದು ಅತ್ಯುನ್ನತ ಆಧ್ಯಾತ್ಮಿಕತೆಯ ವರೆಗೆ ಸ್ವಾಮೀಜಿಯೊಂದಿಗೆ ಸಂಭಾಷಣೆ ನಡೆಸಿದರು. ಸ್ವಾಮೀಜಿಯಲ್ಲಿ ಇತಿಹಾಸಕಾರನ ಜ್ಞಾನ ಹಾಗೂ ತತ್ವಜ್ಞಾನಿಯ ವಿದ್ವತ್ತು ಸಮ್ಮಿಳಿತಗೊಂಡಿದ್ದುವು. ಜೊತೆಗೆ ಅವರ ಹಾಸ್ಯಪ್ರವೃತ್ತಿ ಹಾಗೂ ಅವರ ಅದ್ಭುತ ಮಾತುಗಾರಿಕೆ ಎಂಥವರನ್ನೂ ಅವರೆಡೆಗೆ ಆಕರ್ಷಿಸಬಲ್ಲುದಾಗಿತ್ತು. ಇವೆಲ್ಲದರಿಂದಾಗಿ ಈ ವಿದ್ವಾಂಸರೆಲ್ಲ ಸ್ವಾಮೀಜಿಯ ಆಪ್ತ ಸ್ನೇಹಿತರಾದರು.

ಸ್ವಾಮೀಜಿ ಪ್ಯಾರಿಸಿಗೆ ಬಂದುದು ಕೆಲದಿನಗಳ ವಿಶ್ರಾಂತಿ ಪಡೆದುಕೊಳ್ಳುವ ಉದ್ದೇಶದಿಂದ ಮಾತ್ರ. ಇಲ್ಲಿ ಯಾವುದೇ ಬಗೆಯ ಕೆಲಸವನ್ನು ಕೈಗೆತ್ತಿಕೊಳ್ಳುವ ಆಲೋಚನೆ ಅವರಿಗಿರಲಿಲ್ಲ. ಆದರೆ ಪ್ಯಾರಿಸಿನಲ್ಲಿ ಹಲವಾರು ಪ್ರೇಕ್ಷಣೀಯ ಸ್ಥಳಗಳನ್ನು ಸಂದರ್ಶಿಸುತ್ತ, ಅನೇಕ ಗಣ್ಯವ್ಯಕ್ತಿ ಗಳೊಂದಿಗೆ ಚರ್ಚಿಸುತ್ತ ಅವರ ಸ್ನೇಹವನ್ನು ಗಳಿಸುತ್ತಿದ್ದರೂ ಸ್ವಾಮೀಜಿಯ ಮನಸ್ಸಿನಿಂದ ತಮ್ಮ ಕಾರ್ಯಯೋಜನೆಯ ಕುರಿತ ಆಲೋಚನೆ ದೂರವಾಗಿರಲಿಲ್ಲ. ಅಲ್ಲದೆ ಭಾರತದಲ್ಲಿನ ಅವರ ವಿರೋಧಿಗಳ ಚಟುವಟಿಕೆಯೂ ಅವರ ಗಮನವನ್ನು ಬಲಾತ್ಕಾರದಿಂದ ಸೆಳೆಯುತ್ತಿತ್ತು. ಸ್ವಾಮೀಜಿ ನ್ಯೂಯಾರ್ಕಿನಿಂದ ಹೊರಟು ನಿಂತಿದ್ದಾಗ, ಅವರ ಮದ್ರಾಸೀ ಶಿಷ್ಯರಿಂದ ಅವರಿ ಗೊಂದು ಪತ್ರ ಬಂದಿತ್ತು. ಭಾರತದ ಮಿಷನರಿಗಳು ಅವರ ವಿರುದ್ಧ ಪ್ರಾರಂಭಿಸಿದ್ದ ಭಾರೀ ಅಪಪ್ರಚಾರದ ಬಗ್ಗೆ ತಿಳಿಸಿ ಅವರನ್ನು ಆ ಬಗ್ಗೆ ಎಚ್ಚರಿಸಲಾಗಿತ್ತು. ಈ ಮಿಷನರಿಗಳು ಲೇಖನಗಳು ಹಾಗೂ ಕರಪತ್ರಗಳ ಮೂಲಕ ‘ವಿವೇಕಾನಂದರ ನೈತಿಕಜೀವನ ಚೆನ್ನಾಗಿಲ್ಲ; ಅವರ ಬೋಧನೆಗಳೆಲ್ಲ ಕೇವಲ ಬೊಗಳೆ; ಅವರು ಅಮೆರಿಕದಲ್ಲಿ ತಿನ್ನಬಾರದ್ದನ್ನೆಲ್ಲ ತಿನ್ನುತ್ತಿದ್ದಾರೆ’ ಎಂದು ಅಪಪ್ರಚಾರ ಮಾಡುತ್ತಿದ್ದ ವಿಷಯವನ್ನು ಅವರ ಶಿಷ್ಯರು ತಿಳಿಸಿದ್ದರು. ಅಲ್ಲದೆ ಈ ಲೇಖನಗಳಿಂದಲೂ ಕರಪತ್ರಗಳಿಂದಲೂ ಪ್ರಭಾವಿತರಾಗಿ ಕೆಲವು ಸಂಪ್ರದಾಯಸ್ಥ ಹಿಂದೂಗಳು ಸ್ವಾಮೀಜಿಯನ್ನು ಉಗ್ರವಾಗಿ ಟೀಕಿಸುತ್ತ ಅವರ ಬದ್ಧವಿರೋಧಿಗಳಾಗಿದ್ದಾರೆ ಎಂದೂ ಆ ಶಿಷ್ಯರು ತಿಳಿಸಿದ್ದರು. ಇದನ್ನು ಓದಿ ಸ್ವಾಮೀಜಿಯವರಿಗೆ ತುಂಬ ಬೇಸರವಾಯಿತು–ಅಪಪ್ರಚಾರ ಜೋರಾಗಿ ನಡೆಯುತ್ತಿದೆಯೆಂಬ ಕಾರಣಕ್ಕಿಂತ ಹೆಚ್ಚಾಗಿ ತಮ್ಮ ಶಿಷ್ಯರೂ ತಮ್ಮನ್ನು ಸರಿಯಾಗಿ ಅರ್ಥ ಮಾಡಿಕೊಂಡಿಲ್ಲವಲ್ಲ ಎಂದು. ಪ್ಯಾರಿಸಿನಿಂದ ಅವರು ಅಳಸಿಂಗ ಪೆರುಮಾಳರಿಗೆ ಒಂದು ಪತ್ರ ಬರೆದರು:

“ಮಿಷನರಿಗಳ ಪೊಳ್ಳುಮಾತುಗಳನ್ನು ನೀವೆಲ್ಲ ಅಷ್ಟೊಂದು ಗಂಭೀರವಾಗಿ ತೆಗೆದುಕೊಂಡಿರು ವುದನ್ನು ನೋಡಿ ನನಗೆ ಆಶ್ಚರ್ಯವಾಗುತ್ತದೆ. ಭಾರತದ ಜನ ನಾನು ಹಿಂದೂ ಆಹಾರ ಪದ್ಧತಿಗೆ ಕಟ್ಟುನಿಟ್ಟಾಗಿ ಬದ್ಧನಾಗಿರಬೇಕೆಂದು ಇಚ್ಛಿಸುವುದಾದರೆ, ಅವರಿಗೆ ದಯವಿಟ್ಟು ಹೇಳು– ನನಗೊಬ್ಬ ಅಡಿಗೆಯವನನ್ನೂ ಅವನನ್ನು ಇಟ್ಟುಕೊಳ್ಳಲು ಸಾಕಷ್ಟು ಹಣವನ್ನು ಕಳಿಸಿಕೊಡಿ, ಎಂದು. ಕೂದಲೆಳೆಯಷ್ಟೂ ಸಹಾಯ ಮಾಡದ, ಕೆಲಸಕ್ಕೆ ಬಾರದ ಜನ ಬುದ್ದಿವಾದ ಹೇಳುವು ದನ್ನು ಕಂಡರೆ ನಗು ಬರುತ್ತದೆ. ಆದರೆ, ನಾನು ಸಂನ್ಯಾಸದ ಎರಡು ಮಹಾವ್ರತಗಳಾದ ಬ್ರಹ್ಮಚರ್ಯ ಮತ್ತು ಅಪರಿಗ್ರಹಗಳನ್ನು ಮೀರಿದ್ದೇನೆಂದು ಆ ಮಿಷನರಿಗಳೇನಾದರೂ ನಿನಗೆ ಹೇಳಿದರೆ ನೀನವರಿಗೆ ಹೇಳು–ನೀವು ದೊಡ್ಡ ಸುಳ್ಳುಗಾರರು, ಎಂದು. ನೀನು ‘ಮಿಷನರಿ ಹೋಮ್​’ಗೆ ಪತ್ರ ಬರೆದು, ನನ್ನಲ್ಲಿ ಅವರು ಏನು ದುರ್ನಡತೆಯನ್ನು ಕಂಡಿದ್ದಾರೆಂಬುದನ್ನು ಸ್ಪಷ್ಟವಾಗಿ ಬರೆದು ತಿಳಿಸುವಂತೆ ಕೇಳು. ಅಲ್ಲದೆ ಅವರಿಗೆ ಈ ಮಾಹಿತಿಯನ್ನು ಒದಗಿಸಿದವರ ಹೆಸರುಗಳನ್ನು ತಿಳಿಸಿ ಈ ಮಾಹಿತಿಗಳು ಪ್ರತ್ಯಕ್ಷದರ್ಶಿಗಳ ಹೇಳಿಕೆಗಳೇ ಹೇಗೆ ಎಂಬುದನ್ನು ಸ್ಪಷ್ಟಪಡಿಸುವಂತೆಯೂ ಬರೆ. ಅದು ಸಮಸ್ಯೆಯನ್ನು ಇತ್ಯರ್ಥಗೊಳಿಸಿ ಎಲ್ಲವನ್ನೂ ಬಯಲುಗೊಳಿಸುತ್ತದೆ.

“ಇನ್ನು ನನ್ನ ವಿಷಯವಾಗಿ ಹೇಳುವುದಾದರೆ, ನೆನಪಿಟ್ಟುಕೊ–ನಾನು ಯಾರ ಕೈಕೆಳಗೂ ಇಲ್ಲ. ನನ್ನ ಜೀವನದಲ್ಲಿ ನಾನು ಸಾಧಿಸಬೇಕಾದುದೇನು ಎನ್ನುವುದು ನನಗೆ ಗೊತ್ತಿದೆ. ನಾನು ಎಷ್ಟರ ಮಟ್ಟಿಗೆ ಭಾರತಕ್ಕೆ ಸೇರಿದವನೋ ಅಷ್ಟರಮಟ್ಟಿಗೆ ಜಗತ್ತಿಗೂ ಸೇರಿದವನು. ಈ ವಿಷಯದಲ್ಲಿ ಠಕ್ಕು-ವಂಚನೆಯೇನಿಲ್ಲ. ನಾನು ನಿನಗೆ ನನ್ನ ಕೈಲಾದ ಸಹಾಯವನ್ನೆಲ್ಲ ಮಾಡಿದ್ದೇನೆ. ಈಗ ನಿಮಗೆ ನೀವೇ ಸಹಾಯ ಮಾಡಿಕೊಳ್ಳಬೇಕು. ಯಾವೊಂದು ದೇಶಕ್ಕೂ ನನ್ನ ಮೇಲೆ ಯಾವುದೇ ವಿಶೇಷ ಹಕ್ಕಿಲ್ಲ. ನಾನು ಯಾವುದೋ ಒಂದು ದೇಶದ ಗುಲಾಮನೆ? ಸುಮ್ಮನೆ ಅರ್ಥವಿಲ್ಲದ್ದನ್ನೆಲ್ಲಗಳಹಬೇಡಿ.

“ನಾನು ಕಷ್ಟಪಟ್ಟು ಕೆಲಸ ಮಾಡಿ ನನಗೆ ಬಂದ ಹಣವನ್ನೆಲ್ಲ ಕಲ್ಕತ್ತಕ್ಕೆ ಮತ್ತು ಮದರಾಸಿಗೆ ಕಳಿಸಿಬಿಟ್ಟಿದ್ದೇನೆ. ಇದನ್ನೆಲ್ಲ ಮಾಡಿದ ಮೇಲೂ ಅವರ ಮೂರ್ಖ ಆಜ್ಞೆಗಳನ್ನು ತಾಳಿಕೊಳ್ಳುವುದೆ? ನಾಚಿಕೆಯಾಗುವುದಿಲ್ಲವೆ ಅವರಿಗೆ? ನನಗೆ ಅವರ ಹಂಗಾದರೂ ಏನು? ಕೆಲಸಕ್ಕೆ ಬಾರದ ಹೊಗಳಿಕೆ-ತೆಗಳಿಕೆಗಳಿಗೆ ನಾನು ಲಕ್ಷ್ಯ ಕೊಡಬೇಕೆ? ಮಗು, ನಾನೊಬ್ಬ ವಿಲಕ್ಷಣ ವ್ಯಕ್ತಿ. ನೀವು ಕೂಡ ನನ್ನನ್ನು ಅರ್ಥಮಾಡಿಕೊಂಡಿಲ್ಲ. ನಿಮ್ಮ ಕೆಲಸವನ್ನು ನೀವು ಮಾಡಿ; ಮಾಡಲು ಸಾಧ್ಯ ವಾಗದಿದ್ದರೆ ಮಾಡುವುದನ್ನು ನಿಲ್ಲಿಸಿ. ಆದರೆ ನನ್ನ ಮೇಲೆ ಅರ್ಥವಿಲ್ಲದ ಯಜಮಾನಿಕೆ ನಡೆಸಲು ಮಾತ್ರ ಪ್ರಯತ್ನಿಸಬೇಡಿ. ಮಾನವನಿಗಿಂತಲೂ ದೊಡ್ಡದಾದ ಅಥವಾ ದೆವ್ವಕ್ಕಿಂತಲೂ ದೊಡ್ಡ ದಾದ ಒಂದು ಶಕ್ತಿ ನನ್ನ ಬೆನ್ನ ಹಿಂದಿರುವುದನ್ನು ನಾನು ಸ್ಪಷ್ಟವಾಗಿ ನೋಡುತ್ತಿದ್ದೇನೆ. ನನಗೆ ಯಾರ ಸಹಾಯವೂ ಬೇಕಾಗಿಲ್ಲ. ನನ್ನ ಜೀವನವೆಲ್ಲ ನಾನು ಇತರರಿಗೆ ಸಹಾಯ ಮಾಡುತ್ತಲೇ ಕಳೆದಿದ್ದೇನೆ. ತಮ್ಮ ದೇಶ ನಿರ್ಮಿಸಿದ ಅತಿದೊಡ್ಡ ವ್ಯಕ್ತಿಯಾದ ರಾಮಕೃಷ್ಣ ಪರಮಹಂಸರ ಕಾರ್ಯಕ್ಕಾಗಿ ಕೆಲವು ರೂಪಾಯಿಗಳನ್ನು ಸೇರಿಸಲೂ ಅವರಿಂದ ಸಾಧ್ಯವಿಲ್ಲ. ಅಂತಹ ಅವರು, ಯಾವನಿಗೆ (ವಿವೇಕಾನಂದರಿಗೆ) ಅವರಿಂದ ಎಳ್ಳಷ್ಟೂ ಸಹಾಯವಾಗಲಿಲ್ಲವೊ ಯಾವನು ಅವರಿಗೇ ತನ್ನಿಂದಾದ ಸಹಾಯ ಮಾಡಿದನೊ ಅವನ ಕುರಿತಾಗಿ ಅರ್ಥವಿಲ್ಲದ ಮಾತನಾಡುತ್ತಾರೆ! ಮತ್ತು ಅವನ ಮೇಲೆ ಅಧಿಕಾರ ಚಲಾಯಿಸಲು ನೋಡುತ್ತಾರೆ! ಇಂತಹ ಕೃತಘ್ನ ಜಗತ್ತು ಇದು.

“ವಿದ್ಯಾವಂತ ಹಿಂದೂಗಳಲ್ಲಿ ಮಾತ್ರವೇ ಕಂಡುಬರುವಂತಹ ಜಾತ್ಯಂಧರಾದ, ನಿರ್ದಯ ರಾದ, ಮಿಥ್ಯಾಚಾರಿಗಳಾದ ಮತ್ತು ಮೂಢನಂಬಿಕೆಯ ನಾಸ್ತಿಕ-ಹೇಡಿಗಳಲ್ಲಿ ಒಬ್ಬನಂತೆ ಜೀವನ ನಡೆಸಿ ಸಾಯುವುದಕ್ಕಾಗಿ ನಾನು ಹುಟ್ಟಿದೆನೆಂದು ನಿಮ್ಮ ಅಭಿಪ್ರಾಯವೆ? ಹೇಡಿತನವನ್ನು ದ್ವೇಷಿಸುತ್ತೇನೆ ನಾನು. ಹೇಡಿಗಳೊಂದಿಗೆ ಅಥವಾ ಅರ್ಥವಿಲ್ಲದ ರಾಜಕೀಯದೊಂದಿಗೆ ನನಗೆ ಏನೂ ಕೆಲಸವಿಲ್ಲ. ಯಾವ ರಾಜಕೀಯದಲ್ಲೂ ನನಗೆ ನಂಬಿಕೆಯಿಲ್ಲ. ದೇವರು ಮತ್ತು ಸತ್ಯ ಇವೇ ಪ್ರಪಂಚದಲ್ಲಿನ ಏಕಮಾತ್ರ ರಾಜಕೀಯ. ಇನ್ನುಳಿದುದೆಲ್ಲ ಕೆಲಸಕ್ಕೆ ಬಾರದ ಕಸ.”

ಸ್ವಾಮೀಜಿಯ ಸ್ವಾತಂತ್ರ್ಯಪ್ರಿಯತೆಯೆಂಥದು, ಅವರ ನಿರ್ಭೀತ ಮನೋಭಾವವೆಂಥದು, ಅವರ ಸಂನ್ಯಾಸಧರ್ಮ ಎಷ್ಟು ಕಟುತರವಾದುದು ಎಂಬುದೆಲ್ಲ ಈ ಪತ್ರದಲ್ಲಿ ವ್ಯಕ್ತವಾಗುತ್ತವೆ. ತಮ್ಮೊಳಗಿನ ಭಾವಗಳನ್ನು ಒತ್ತಿಹಿಡಿಯಲು ಸ್ವಾಮೀಜಿ ಎಷ್ಟು ಪ್ರಯತ್ನಿಸಿದರೂ ಒಂದೆರಡು ಕಿಡಿಗಳು ಹೊರಹೊಮ್ಮುವುದನ್ನು ನಾವಿಲ್ಲಿ ಕಾಣುತ್ತೇವೆ. ಹಿಂದೂಧರ್ಮವನ್ನು ಪುನರುಜ್ಜೀವನ ಗೊಳಿಸುವುದಕ್ಕಾಗಿ ಜನ್ಮವೆತ್ತಿದವರು ಸ್ವಾಮೀಜಿ. ಅದಾಗಲೇ ಅವರ ಕಾರ್ಯ ಮೊಳಕೆಯೊಡೆದು ಮೇಲೆ ಬರಲಾರಂಭಿಸಿತ್ತು. ಆದರೆ ಅವರ ಯಶಸ್ಸನ್ನು ಕಂಡು ಕರುಬುವ ಜನ ಅವರ ಕಾರ್ಯವನ್ನು ನಿಷ್ಫಲಗೊಳಿಸುವ ಉದ್ದೇಶದಿಂದ ಅವರ ಮೇಲೆ ಮಿಥ್ಯಾಪವಾದಗಳನ್ನು ಹೊರಿಸಿ ಹರಡುವ ಪ್ರಯತ್ನ ಮಾಡಿದರೆ ಸ್ವಾಮೀಜಿ ತಮ್ಮನ್ನೂ ತಮ್ಮ ಕಾರ್ಯೋದ್ದೇಶವನ್ನೂ ರಕ್ಷಿಸಿ ಕೊಳ್ಳಲೇಬೇಕಲ್ಲವೆ? ನಿಜಕ್ಕೂ ಈ ವೇಳೆಗಾಗಲೇ ಭಾರತದಲ್ಲಿ ಕ್ರೈಸ್ತ ಮಿಷನರಿಗಳು ಅವರ ಮೇಲೆ ಮಾಡುತ್ತಿದ್ದ ಅಪಪ್ರಚಾರ ಪರಾಕಾಷ್ಠೆಗೇರಿಬಿಟ್ಟಿತ್ತು. ಆದರೆ ಸ್ವಾಮೀಜಿ ಬಲಶಾಲಿಗಳು; ಆವಶ್ಯಕತೆ ಬಿದ್ದರೆ ಎದುರಾಳಿಗಳಾಗಿ ನಿಲ್ಲಬಲ್ಲ ಸಮರ್ಥರು. ಅಲ್ಲದೆ ಅವರು ಹಾಗಿರಲೇ ಬೇಕಾಗಿತ್ತು. ಹಾಗಿಲ್ಲದೆ ಹೋಗಿದ್ದರೆ ಅವರ ಸನಾತನ ಧರ್ಮಕ್ಕೆ ಅಪಖ್ಯಾತಿ ಅಂಟಿಕೊಳ್ಳುತ್ತಿತ್ತು. ಅವರ ಜನತೆಯ ಗೌರವಕ್ಕೆ ಕುಂದುಬರುತ್ತಿತ್ತು. ಅವರ ಬೋಧನೆಗಳು ಅಪಹಾಸ್ಯಕ್ಕೀಡಾಗು ತ್ತಿದ್ದುವು. ಆದ್ದರಿಂದ ಅವರು ತಾವು ಕೈಗೊಂಡ ಕಾರ್ಯದ ಘನತೆಯನ್ನು, ಮಹತ್ವವನ್ನು ಜನ ಗುರುತಿಸಿ ಅದಕ್ಕೊಂದು ಯೋಗ್ಯ ಮನ್ನಣೆ ಸಿಗುವಂತೆ ಮಾಡುವುದಕ್ಕಾಗಿ ಹೆಜ್ಜೆಹೆಜ್ಜೆಗೂ ಹೋರಾಡಬೇಕಾಗಿಬಂದಿತು. ಅಸೂಯಾಪರ ವ್ಯಕ್ತಿಗಳು ತಮ್ಮ ಶೀಲಕ್ಕೆ ಕಳಂಕ ಹಚ್ಚುವ ಪ್ರಯತ್ನ ಮಾಡಿದಾಗಲಂತೂ ಅವರು ಅತ್ಯಂತ ಉಗ್ರವಾಗಿ ಪ್ರತ್ಯುತ್ತರ ಕೊಟ್ಟರು. ಅವರಿಗೆ ಬೇರೆ ದಾರಿಯೇ ಇರಲಿಲ್ಲ. ಏಕೆಂದರೆ ಆ ನಿಂದಕರ ಮಾತಿನಂತೆ ಸ್ವಾಮೀಜಿಯ ಶೀಲ ಸಮರ್ಪಕ ವಾಗಿಲ್ಲ ಎಂದೇ ಸಾಬೀತಾದ ಪಕ್ಷದಲ್ಲಿ ಇನ್ನವರ ಬೋಧನೆಗಳಿಗೆ ಏನು ಬೆಲೆ ಬಂದೀತು? ಆದರೆ ಇನ್ನು ಕೆಲವು ಸಲ ಸ್ವಾಮೀಜಿ ಜಗನ್ಮಾತೆಯ ರಕ್ಷಣೆ ಕೋರಿ ಏಕಾಂತದಲ್ಲಿ ಕುಳಿತು ಮಗು ವಿನಂತೆ ಅತ್ತದ್ದೂ ಉಂಟು. ಅಮೆರಿಕದಲ್ಲಿದ್ದಾಗ ಒಂದು ದಿನವಂತೂ ಯಾವುದೇ ಬಗೆಯ ಆಧಾರವೂ ಇಲ್ಲದೆ ಬರೆದ ಅವರ ಶೀಲವನ್ನು ನಿಂದಿಸುವ ಒಂದು ಲೇಖನವನ್ನು ಓದಿದಾಗ ಅವರ ಕಣ್ಣಂಚಿನಲ್ಲಿ ನೀರು ತುಂಬಿತ್ತು. ಅದನ್ನು ಕಂಡವರೊಬ್ಬರು, “ಏಕೆ ಅಳುತ್ತಿದ್ದೀರಿ, ಸ್ವಾಮೀಜಿ?” ಎಂದಾಗ ಅವರೆಂದರು, “ಈ ಜಗತ್ತಿನಲ್ಲಿ ನೀಚತನವೆಂಬುದು ಎಷ್ಟರ ಮಟ್ಟಿಗಿದೆ ನೋಡಿ! ಈ ಜನ ಧರ್ಮದ ಹೆಸರಿನಲ್ಲಿ ದುಷ್ಟತನ ಮಾಡಲು ಎಲ್ಲಿಯವರೆಗೂ ಹೋದಾರು! ಅದೇ ಭಗವಂತನ ರಾಜ್ಯದಲ್ಲಿ ಸೇವೆ ಸಲ್ಲಿಸುವವನೊಬ್ಬನಿಗೆ ಕಳಂಕ ಅಂಟಿಸಲು ಏನು ಮಾಡಲೂ ಹೇಸುವುದಿಲ್ಲವಲ್ಲ!”

ಸೆಪ್ಟೆಂಬರ್ ೧ಂರಂದು ಸ್ವಾಮೀಜಿ ಇಂಗ್ಲೆಂಡಿಗೆ ಹೊರಟರು. ‘ಆ ಇಂಗ್ಲಿಷರು ನನ್ನನ್ನು ಯಾವ ರೀತಿ ಸ್ವೀಕರಿಸಿಯಾರು?’ ಎಂದು ಅವರ ಮನಸ್ಸು ಊಹಿಸುತ್ತಿತ್ತು. ‘ಒಂದು ವೇಳೆ ಸರಿಯಾದ ರೀತಿಯಲ್ಲಿ ಸ್ವೀಕರಿಸದೆ ಹೋದರೆ?’ ಎಂಬ ಅನುಮಾನವೂ ಆಗಾಗ ಸುಳಿಯು ತ್ತಿತ್ತು. ಏಕೆಂದರೆ ಎಷ್ಟಾದರೂ ಈ ಇಂಗ್ಲೆಂಡು ಅಂದು ಭಾರತವನ್ನು ಆಳುತ್ತಿದ್ದ ರಾಷ್ಟ್ರವಲ್ಲವೆ? ತಾವು ಈ ರಾಷ್ಟ್ರಕ್ಕೆ ಬಂದು ಇಲ್ಲಿನ ನಗರಗಳಲ್ಲಿ ಧರ್ಮಪ್ರಸಾರಕಾರ್ಯ ಮಾಡಬೇಕೆಂದು ಸ್ವಾಮೀಜಿ ಹಲವು ಸಲ ಕನಸು ಕಂಡಿದ್ದರು. ಆದರೆ ದಾಸರಾಷ್ಟ್ರವಾದ ಭಾರತದಿಂದ ಬರುತ್ತಿರುವ ತಮ್ಮನ್ನು–ಒಬ್ಬ ಹಿಂದೂ ಸಂನ್ಯಾಸಿಯನ್ನು–ಬ್ರಿಟಿಷ್ ಸಾರ್ವಜನಿಕರು ಯಾವ ದೃಷ್ಟಿಯಿಂದ ಕಂಡಾರು ಎಂಬ ಸಂಶಯ ಅವರಲ್ಲಿತ್ತು. ಕೆಲವು ದಯಾದಾಕ್ಷಿಣ್ಯವಿಲ್ಲದ ಟೀಕೆಗಾರರ ಮೂಲಕ ಹಿಂದೂಧರ್ಮದ ಕುರಿತಾಗಿ ಕೇವಲ ತಪ್ಪು ಕಲ್ಪನೆಗಳೇ ತುಂಬಿಕೊಂಡಿದ್ದುವು ಆ ರಾಷ್ಟ್ರದಲ್ಲಿ. ಈಗ ಇಂಥಲ್ಲಿಗೆ ಹಿಂದೂಧರ್ಮವನ್ನು ಪ್ರಸಾರ ಮಾಡಲು ಬಂದಾಗ ಇಲ್ಲಿನ ಜನ ತಮ್ಮನ್ನು ಯಾವ ದೃಷ್ಟಿಯಿಂದ ಕಂಡಾರು ಎಂಬ ಕುತೂಹಲ ಸ್ವಾಮೀಜಿಗೆ. ಆದರೆ ತಾವು ಇಂಗ್ಲೆಂಡನ್ನು ಪ್ರವೇಶಿಸಿದ ಮೇಲೆ ತಮಗೆ ದೊರಕಿದ ಅನುಪಮ ಯಶಸ್ಸನ್ನು ಕಂಡು ಅವರೇ ಆನಂದಾಶ್ಚರ್ಯ ಪಡುವಂತಾಯಿತು. ಲಂಡನ್ ನಗರದ ಪ್ರಥಮ ನೋಟವೇ ಅವರಿಗೊಂದು ಸ್ಫೂರ್ತಿ ತಂದಿತು. ಅಲ್ಲಿ ಅವರ ಸ್ನೇಹಿತರಾದ ಇ. ಟಿ. ಸ್ಟರ್ಡಿ, ಮಿಸ್ ಹೆನ್ರಿಟಾ ಮುಲ್ಲರ್ ಮೊದಲಾದವರು ಅವರಿಗೆ ಆದರದ ಸ್ವಾಗತ ನೀಡಿದರು. ಮೊದಲಿಗೆ ಅವರು ಕೇಂಬ್ರಿಡ್ಜ್​ನಲ್ಲಿದ್ದ ಮಿಸ್ ಮುಲ್ಲರಳ ಮನೆಯಲ್ಲಿ ಉಳಿದುಕೊಂಡರು. ಕೆಲದಿನಗಳ ಬಳಿಕ ಅವರು ಲಂಡನ್ನಿನಿಂದ ಮೂವತ್ತಾರು ಮೈಲಿ ದೂರದಲ್ಲಿರುವ ರೀಡಿಂಗ್ ಎಂಬಲ್ಲಿನ ಸ್ಟರ್ಡಿಯ ಮನೆಗೆ ಬಂದರು.

ಸ್ಟರ್ಡಿಯ ಜೊತೆಯಲ್ಲಿ ಸ್ವಾಮೀಜಿ ಸುಮಾರು ಆರು ವಾರಗಳ ಕಾಲ ವಾಸಿಸಿದರು. ಈತ ಒಬ್ಬ ನಿಷ್ಠಾವಂತ ವೇದಾಂತಿ; ಹಿಂದೂ ಶಾಸ್ತ್ರಗಳಲ್ಲಿ ಆಸಕ್ತ. ರೀಡಿಂಗ್ ಪಟ್ಟಣದಲ್ಲಿ ಸ್ವಾಮೀಜಿ ಎಲ್ಲ ಐತಿಹಾಸಿಕ ಸ್ಥಳಗಳನ್ನೂ ಕಲಾಕೇಂದ್ರಗಳನ್ನೂ ಸಂದರ್ಶಿಸಿದರು. ತಮ್ಮ ಆತಿಥೇಯನಾದ ಸ್ಟರ್ಡಿಯೊಂದಿಗೆ ತತ್ತ್ವಶಾಸ್ತ್ರಗಳ ಮೇಲೆ ದೀರ್ಘಕಾಲ ಸಂಭಾಷಣೆ ನಡೆಸಿದರು; ಅವನಿಗೆ ಸಂಸ್ಕೃತ ಕಲಿಯಲು ನೆರವಾದರು. ಮತ್ತು ನಾರದ ಭಕ್ತಿಸೂತ್ರಗಳನ್ನು ಇಂಗ್ಲಿಷಿಗೆ ಅನುವಾದಿಸು ವಲ್ಲಿ ಮತ್ತು ಅವುಗಳ ಮೇಲೆ ವ್ಯಾಖ್ಯೆ ಬರೆಯುವಲ್ಲಿ ಅವನಿಗೆ ಸಹಾಯವಾದರು.

ಹೀಗೆ ಸ್ವಾಮೀಜಿ ರೀಡಿಂಗ್ ಪಟ್ಟಣದಲ್ಲಿ ಸೆಪ್ಟೆಂಬರ್ ಮತ್ತು ಅಕ್ಟೋಬರ್ ತಿಂಗಳನ್ನು ಶಾಂತವಾಗಿ ಕಳೆದರು. ಈ ವಿಷಯವಾಗಿ ಅವರು ಶ್ರೀಮತಿ ಬುಲ್​ಗೆ ಬರೆದ ಪತ್ರದಲ್ಲಿ ತಿಳಿಸಿದರು–“ನಾನಿನ್ನೂ ಇಲ್ಲಿ ಕಣ್ಣಿಗೆ ಕಾಣಿಸುವಂತಹ ಕೆಲಸವನ್ನೇನೂ ಮಾಡಿಲ್ಲ. ಲಂಡನ್ನಿನ ಅನುಕೂಲಕರ ಪುತು (ಸಾಂಸ್ಕೃತಿಕ ಚಟುವಟಿಕೆಗಳ ಕಾಲ) ಇನ್ನೂ ಪ್ರಾರಂಭವಾಗಿಲ್ಲ. ಅಲ್ಲದೆ ಸ್ಟರ್ಡಿಯೂ ಕೂಡ ಮೆಲ್ಲನೆ ಮುಂದುವರಿಯಲು ಇಷ್ಟಪಡುತ್ತಾನೆ–ಅವಸರವಸರವಾಗಿ ಹೋಗಿ ಕೇವಲ ನಿಷ್ಪ್ರಯೋಜಕ ಸದ್ದುಗದ್ದಲದಲ್ಲಿ ಪರ್ಯವಸಾನಗೊಳ್ಳುವುದರ ಬದಲಾಗಿ ಒಂದು ಬಲ ವಾದ ಅಡಿಪಾಯವನ್ನು ಹಾಕಬೇಕೆಂಬ ಸಲಹೆ ಅವನದ್ದು. ಆದ್ದರಿಂದ ನಾವು ಸಾವಕಾಶವಾಗಿ ಮುನ್ನಡೆಯುತ್ತಿದ್ದೇವೆ. ಇಲ್ಲಿಯವರೆಗೆ ಎಲ್ಲವೂ ಚೆನ್ನಾಗಿದೆ. ಮುಂದೆ ಕೈಗೊಳ್ಳಲಿರುವ ಮಹಾ ಕಾರ್ಯಕ್ಕಾಗಿ ಕಾಯುತ್ತಿದ್ದೇವೆ. ‘ನೀನಾಗಿ ಹುಡುಕಿಕೊಂಡು ಹೋಗಬೇಡ, ಅದಾಗಿ ಬಂದದ್ದನ್ನು ಬಿಡಬೇಡ; ಭಗವಂತ ಏನನ್ನು ಕಳಿಸಿಕೊಡುತ್ತಾನೆಯೋ ಅದಕ್ಕಾಗಿ ಕಾಯುತ್ತಿರು’–ಇದೇ ನನ್ನ ಧ್ಯೇಯಮಂತ್ರ.”

ಇಂಗ್ಲೆಂಡಿಗೆ ಹೋದಾಗಿನಿಂದಲೂ, ಆ ಇಂಗ್ಲಿಷರು ತಮ್ಮನ್ನು ಬರಮಾಡಿಕೊಂಡ ಬಗೆ ಸ್ವಾಮೀಜಿಯನ್ನು ಬೆರಗುಗೊಳಿಸಿತ್ತು. ಇವರು ತಮ್ಮನ್ನು ಇಷ್ಟೊಂದು ಆದರದಿಂದ, ಇಷ್ಟೊಂದು ಹೃತ್ಪೂರ್ವಕವಾಗಿ ಹೇಗೆ ಸ್ವಾಗತಿಸಿರಬಹುದು ಎಂದು ಆಲೋಚಿಸಿದರು. ಮೊದಮೊದಲಿಗೆ ಅಮೆರಿಕದಲ್ಲಿ ತಮಗೆದುರಾದ ತಿರಸ್ಕಾರ-ಅಪಹಾಸ್ಯಪೂರಿತ ದೃಷ್ಟಿಯನ್ನೇ ಇಂಗ್ಲೆಂಡಿನಲ್ಲೂ ಅವರು ನಿರೀಕ್ಷಿಸಿದ್ದರು. ಅಲ್ಲದೆ ಭಾರತದಲ್ಲಿನ ಆಂಗ್ಲ ಅಧಿಕಾರಿಗಳು ಭಾರತೀಯರನ್ನು ಯಾವ ದೃಷ್ಟಿಯಿಂದ ಕಾಣುತ್ತಿದ್ದರೆಂಬುದು ಅವರಿಗೆ ತಿಳಿದ ವಿಷಯವೇ. ಆದ್ದರಿಂದ ಅವರಿಗೆ ಇಂಗ್ಲೆಂಡಿನ ಜನರು ಎಷ್ಟೋ ಮೇಲು ಎಂಬಂತೆ ತೋರಿತು. ರೀಡಿಂಗಿನಿಂದ ಬರೆದ ಒಂದು ಪತ್ರದಲ್ಲಿ ಸ್ವಾಮೀಜಿ ಹೀಗೆ ತಿಳಿಸಿದರು–“ನಾನಿಲ್ಲಿ ಭಾರತದಿಂದ ಹಿಂದಿರುಗಿದ ಕೆಲವು ನಿವೃತ್ತ ಅಧಿಕಾರಿಗಳನ್ನು ಭೇಟಿಯಾದೆ. ಅವರೆಲ್ಲ ನನ್ನೊಂದಿಗೆ ತುಂಬ ಸಭ್ಯತೆಯಿಂದ ವಿನಯವಾಗಿ ನಡೆದುಕೊಂಡರು. ಕಣ್ಣಿಗೆ ಕಂಡ ಕಪ್ಪು ಮನುಷ್ಯರನ್ನೆಲ್ಲ ನೀಗ್ರೋ ಎಂದು ತೀರ್ಮಾನಿಸಿಬಿಡುವ ಅಮೆರಿಕನ್ನರ ಅದ್ಭುತ ಬುದ್ಧಿವಂತಿಕೆ ಇಲ್ಲಿ ಕಂಡು ಬರುವುದಿಲ್ಲ. ಮತ್ತು ರಸ್ತೆಯಲ್ಲಿ ನನ್ನನ್ನೇ ದುರುಗುಟ್ಟಿಕೊಂಡು ನೋಡುವವರು ಇಲ್ಲಿ ಯಾರೂ ಇಲ್ಲ. ನಾನಿಲ್ಲಿ ಇನ್ನಾವುದೇ ಹೊರರಾಷ್ಟ್ರ ದಲ್ಲಿರುವುದಕ್ಕಿಂತ ಹೆಚ್ಚು ಆತ್ಮೀಯವಾದ ವಾತಾವರಣದಲ್ಲಿದ್ದೇನೆ.” ಜನವರಿಯಲ್ಲಿ ಬರೆದ ಮತ್ತೊಂದು ಪತ್ರದಲ್ಲಿ ಹೀಗೆ ಹೇಳಿದರು, “ಈ ಇಂಗ್ಲಿಷು ಜನ ನನ್ನನ್ನು ತೆರೆದ ಬಾಹುಗಳಿಂದ ಬರಮಾಡಿಕೊಂಡರು. ಈ ಜನಾಂಗದ ಬಗ್ಗೆ ನನ್ನ ಅಭಿಪ್ರಾಯವನ್ನು ಎಷ್ಟೋ ಬದಲಾಯಿಸಿ ಕೊಂಡಿದ್ದೇನೆ. ಇಂಗ್ಲಿಷ್ ಚರ್ಚಿನ ಕೆಲವು ಅತ್ಯುತ್ತಮ ವ್ಯಕ್ತಿಗಳು ಮತ್ತು ಅತ್ಯುನ್ನತ ಸ್ಥಾನಮಾನ ಗಳಿಂದ ಕೂಡಿದ ಕೆಲವು ಗಣ್ಯವ್ಯಕ್ತಿಗಳು ನನ್ನ ಆಪ್ತಸ್ನೇಹಿತರಾಗಿದ್ದಾರೆ.”

ಅಕ್ಟೋಬರಿನ ಕೊನೆಯ ವಾರದಲ್ಲಿ ಸ್ವಾಮೀಜಿಯ ಕೆಲವು ಸ್ನೇಹಿತರು ಲಂಡನ್ನಿನ \eng{Prince’s Hall} ಎಂಬ ಪ್ರಸಿದ್ಧ ಸಭಾಂಗಣದಲ್ಲಿ ಅವರ ಸಾರ್ವಜನಿಕ ಉಪನ್ಯಾಸವನ್ನು ಏರ್ಪಡಿಸಿದರು. ಉಪನ್ಯಾಸದ ವಿಷಯ ‘ಆತ್ಮಬೋಧ’. ಉಪನ್ಯಾಸ ಅತ್ಯಂತ ಯಶಸ್ವಿಯಾಯಿತು. ಸ್ವಾಮೀಜಿ ಉಪನ್ಯಾಸ ಮಾಡಲು ಎದ್ದು ನಿಂತಾಗ ನೋಡುತ್ತಾರೆ–ಎಲ್ಲ ಬಗೆಯ ವೃತ್ತಿಗಳ ಜನರಿಂದ, ಲಂಡನ್ನಿನ ಅತ್ಯಂತ ಶ್ರೇಷ್ಠ ಚಿಂತನಶೀಲ ವ್ಯಕ್ತಿಗಳಿಂದ ಕೂಡಿದ ಬೃಹತ್ ಸಭೆ ನಡೆದಿತ್ತು. ಅವರ ಉಪನ್ಯಾಸವನ್ನು ಕೇಳಿದವರೊಬ್ಬರು ಬರೆದರು–“ಅವರು ತಮ್ಮ ಭವ್ಯವಾದ ಹಾಗೂ ಪ್ರಬಲವಾದ ಭಾಷಣ ಸಾಮರ್ಥ್ಯದಿಂದ ಸಭಿಕರಲ್ಲಿ ವಿದ್ಯುತ್ ಸಂಚಾರವನ್ನೇ ಉಂಟುಮಾಡಿ ದರು.” ಮರುದಿನ ಬೆಳಿಗ್ಗೆ ವೃತ್ತಪತ್ರಿಕೆಗಳಲ್ಲೆಲ್ಲ ಭಾಷಣದ ಮೇಲಿನ ಶ್ಲಾಘನೆಯ ಲೇಖನಗಳು ತುಂಬಿದ್ದುವು. ‘ಸ್ಟ್ಯಾಂಡರ್ಡ್​’ ಎಂಬ ಪತ್ರಿಕೆ ಬರೆಯಿತು:

“ರಾಮಮೋಹನ ರಾಯರ ಅನಂತರ ದಿನಗಳಲ್ಲಿ ಕೇಶವಚಂದ್ರಸೇನರೊಬ್ಬರನ್ನು ಬಿಟ್ಟರೆ, ‘ಪ್ರಿನ್ಸಸ್ ಹಾಲ್​’ನಲ್ಲಿ ಮಾತನಾಡಿದ ಈ ಹಿಂದುವಿಗಿಂತ (ವಿವೇಕಾನಂದರಿಗಿಂತ) ಹೆಚ್ಚು ಸ್ವಾರಸ್ಯವಾಗಿ ಮಾತನಾಡುವ ಇನ್ನೊಬ್ಬ ಭಾರತೀಯನು ಇಂಗ್ಲಿಷ್ ವೇದಿಕೆಯ ಮೇಲೆ ಕಾಣಿಸಿ ಕೊಂಡಿಲ್ಲ. ಬುದ್ಧ ಅಥವಾ ಕ್ರಿಸ್ತರು ತಮ್ಮ ಕೇವಲ ನಾಲ್ಕೈದು ಮಾತುಗಳಿಂದ ಸಾಧಿಸಿದ ಒಳಿತಿಗೆ ಹೋಲಿಸಿದರೆ ಇಂದಿನ ಈ ಎಲ್ಲ ಕಾರ್ಖಾನೆಗಳು, ಯಂತ್ರಗಳು, ಇತರ ನವ ಸಂಶೋಧನೆಗಳು ಹಾಗೂ ಗ್ರಂಥರಾಶಿ–ಇವು ಮಾನವನಿಗಾಗಿ ಮಾಡಿರುವ ಉಪಕಾರ ಎಷ್ಟು ಕ್ಷುಲ್ಲಕವಾದದ್ದು ಎಂಬುದನ್ನು ಎತ್ತಿತೋರಿಸಿ ಸ್ವಾಮೀಜಿ ಕುಟವಾಗಿ ಟೀಕಿಸಿದರು. ಅವರ ಉಪನ್ಯಾಸವು ಪೂರ್ವ ಸಿದ್ಧತೆಯಿಲ್ಲದೆ ಸ್ವಯಂಸ್ಫೂರ್ತಿಯಿಂದ ಮಾಡಿದುದೆಂಬುದು ಸ್ಪಷ್ಟವಾಗಿತ್ತು; ಅಲ್ಲದೆ ಯಾವ ಬಗೆಯ ಸಂಕೋಚವೂ ಇಲ್ಲದೆ ಮಧುರವಾದ ಕಂಠಶ್ರೀಯಿಂದ ಕೂಡಿತ್ತು.”

‘ಲಂಡನ್ ಡೈಲಿ ಕ್ರಾನಿಕಲ್​’ ಪತ್ರಿಕೆ ಬರೆಯಿತು–“ಜನಪ್ರಿಯ ಹಿಂದೂ ಸಂನ್ಯಾಸಿಯಾದ ವಿವೇಕಾನಂದರ ಮುಖ ಲಕ್ಷಣವು ಚಿರಪರಿಚಿತವಾದ ಬುದ್ಧನ ಮುಖವನ್ನೇ ಹೋಲುತ್ತದೆ. ಅವರು ನಮ್ಮ ಆರ್ಥಿಕ ಸಮೃದ್ಧಿಯನ್ನೂ ಘೋರಯುದ್ಧಗಳನ್ನೂ ಧಾರ್ಮಿಕ ಅಸಹಿಷ್ಣುತೆ ಯನ್ನೂ ತೀವ್ರವಾಗಿ ಖಂಡಿಸಿದರು. ನಮ್ರ ಸ್ವಭಾವದ ಹಿಂದುವು ಇಂತಹ ಬೆಲೆಯನ್ನು ತೆತ್ತು, ನಾವು ಹೆಮ್ಮೆಪಟ್ಟುಕೊಳ್ಳುವಂತಹ ನಮ್ಮ ಸಂಸ್ಕೃತಿಯನ್ನು ಎಂದಿಗೂ ಸ್ವೀಕರಿಸಲಾರನೆಂದು ವಿವೇಕಾನಂದರು ಘೋಷಿಸಿದರು.”

ಸ್ವಾಮೀಜಿಯೊಂದಿಗೆ ಸಂದರ್ಶನ ನಡೆಸಿದ ‘ವೆಸ್ಟ್ ಮಿನಿಸ್ಟರ್ ಗೆಜೆಟ್​’ ಪತ್ರಿಕೆಯ ಪ್ರತಿನಿಧಿ ಯೊಬ್ಬ ‘ಲಂಡನ್ನಿನಲ್ಲಿ ಒಬ್ಬ ಭಾರತೀಯ ಯೋಗಿ’ ಎಂಬ ಶೀರ್ಷಿಕೆಯಡಿಯಲ್ಲಿ ಬರೆದ– “ಶಾಂತವಾದ ಹಾಗೂ ಕರುಣಾಪೂರಿತ ಮುದ್ರೆಯಿಂದ ಕೂಡಿದ ತಲೆಗೆ ಪೇಟ ಸುತ್ತಿರುವ ವಿವೇಕಾ ನಂದರದು ಪ್ರಖರ ವ್ಯಕ್ತಿತ್ವ. ಅವರ ಮುಖವು ಒಂದು ಮುಗ್ಧ ಮಗುವಿನ ಮುಖದಂತೆ ಮಿಂಚು ತ್ತದೆ–ಅಲ್ಲಿ ಸರಳತೆ, ಪುಜುತ್ವ, ಪ್ರಾಮಾಣಿಕತೆ ಎದ್ದುಕಾಣುತ್ತದೆ.”

ಈ ಸಂದರ್ಶನದ ಸಮಯದಲ್ಲಿ ದೀರ್ಘ ಸಂಭಾಷಣೆಯೇ ನಡೆಯಿತು. ಪ್ರಶ್ನೆಗಳಿಗೆ ಉತ್ತರಿಸುತ್ತ ಸ್ವಾಮೀಜಿ, ತಾವು ಹೇಗೆ ಸಂನ್ಯಾಸದ ದಿವ್ಯಾದರ್ಶದಿಂದ ಪ್ರಭಾವಿತರಾದೆವು ಎಂಬುದನ್ನು ವಿವರಿಸಿದರು. ಬಳಿಕ ಶ್ರೀರಾಮಕೃಷ್ಣರ ಆಗಮನದ ಕುರಿತಾಗಿ ಮಾತನಾಡುತ್ತ, ಅವರು ಯಾವುದೋ ಒಂದು ಮತವನ್ನು ಸ್ಥಾಪಿಸಲು ಬಂದವರಲ್ಲ; ಒಂದು ಧಾರ್ಮಿಕ ಪಂಗಡ ವನ್ನು ಕಟ್ಟಲು ಬಂದವರಲ್ಲ; ಅವರು ವೇದಾಂತದ ಸಾರ್ವತ್ರಿಕವಾದ ಸನಾತನ ಸತ್ಯಗಳನ್ನು ಮತ್ತೊಮ್ಮೆ ಪ್ರಸಾರ ಮಾಡಿ, ಅವುಗಳನ್ನು ಪತ್ರಿಯೊಬ್ಬರೂ ತಮ್ಮದೇ ಆದ ಮೂಲಭೂತ ತತ್ತ್ವಗಳಿಗೆ ಅಳವಡಿಸಿಕೊಳ್ಳುವಂತೆ ಮಾಡಲು ಬಂದವರು, ಎಂದು ಸ್ವಾಮೀಜಿ ವಿವರಿಸಿದರು. “ನಾನು ಯಾವೊಂದು ರಹಸ್ಯ ಸಂಸ್ಥೆಯ ಪ್ರತಿಪಾದಕನೂ ಅಲ್ಲ, ಅಥವಾ ಇಂತಹ ಸಂಸ್ಥೆಗಳಿಂದ ಏನಾದರೂ ಒಳಿತಾದೀತೆಂಬ ನಂಬಿಕೆಯೂ ನನಗಿಲ್ಲ. ಸತ್ಯವು ತನ್ನ ಆಧಾರದ ಮೇಲೆ ತಾನು ನಿಲ್ಲುತ್ತದೆ. ಮತ್ತು ಸತ್ಯವು ಎಂತಹ ಅಗ್ನಿಪರೀಕ್ಷೆಯನ್ನೂ ಎದುರಿಸಬಲ್ಲದು... ನಾನು ಪ್ರತಿ ಪಾದಿಸುವ ತತ್ತ್ವವು ಈ ಜಗತ್ತಿನಲ್ಲಿ ಎಷ್ಟೆಷ್ಟು ಧರ್ಮಗಳು ಇರಲು ಸಾದ್ಯವೋ ಅವೆಲ್ಲಕ್ಕೂ ಆಧಾರಭೂತವಾಗಿ ನಿಲ್ಲಬಲ್ಲದು ಮತ್ತು ಈ ಎಲ್ಲ ಧರ್ಮಗಳ ಬಗ್ಗೆ ನನ್ನ ಧೋರಣೆ ಎಂತಹ ದೆಂದರೆ ಸಂಪೂರ್ಣ ಸಹಾನುಭೂತಿಯಿಂದ ಕೂಡಿದುದು. ನನ್ನ ಬೋಧನೆಯು ಈ ಯಾವುದಕ್ಕೂ ವಿರೋಧವಾಗಿ ನಿಲ್ಲುವುದಿಲ್ಲ. ನನ್ನ ಗಮನವೆಲ್ಲ ಮಾನವರ ಕಡೆಗೆ; ನನ್ನ ಗಮನವೆಲ್ಲ ವ್ಯಕ್ತಿ ಯನ್ನು ಶಕ್ತನನ್ನಾಗಿ ಮಾಡುವುದರ ಕಡೆಗೆ; ಪ್ರತಿಯೊಬ್ಬನೂ ತನ್ನನ್ನು ತಾನು ಆತ್ಮ ಎಂದು ಅರಿತುಕೊಳ್ಳುವಂತೆ ಮಾಡುವುದರ ಕಡೆಗೆ. ಸಮಸ್ತ ಮಾನವತೆಯೂ ತನ್ನೊಳಗಡಗಿರುವ ದಿವ್ಯತೆ ಯನ್ನು ಕಂಡುಕೊಳ್ಳಬೇಕೆಂದು ನಾನು ಕರೆನೀಡುತ್ತೇನೆ” ಎಂದು ಸ್ವಾಮೀಜಿ ನುಡಿದರು. ಪತ್ರಿಕೆಯ ಪ್ರತಿನಿಧಿ ತನ್ನ ಲೇಖನದಲ್ಲಿ ಸ್ವಾಮೀಜಿಯ ಆದರ್ಶಗಳ ಕುರಿತಾಗಿ ಹಾಗೂ ಅಮೆರಿಕದಲ್ಲಿ ಅವರು ಗಳಿಸಿದ ಯಶಸ್ಸಿನ ಕುರಿತಾಗಿ ಬರೆದು ಕೊನೆಯದಾಗಿ ಒಂದು ವಾಕ್ಯವನ್ನು ಸೇರಿಸಿದ–“ಬಳಿಕ ನಾನು, ಮನುಷ್ಯರಲ್ಲೆಲ್ಲ ಅತ್ಯಂತ ಸ್ವಂತಿಕೆಯಿಂದ ಕೂಡಿದ ವ್ಯಕ್ತಿಯೊಬ್ಬನನ್ನು ಭೇಟಿ ಮಾಡಿದ ಹೆಮ್ಮೆಯಿಂದ ಅವರಿಂದ ಬೀಳ್ಕೊಂಡು ಹೊರಟೆ.” ಹೀಗೆ ಲಂಡನ್ನಿನ ಸಾರ್ವಜನಿಕರಿಗೆ ಸ್ವಾಮೀಜಿಯ ಉಪನ್ಯಾಸದ ಪರಿಣಾಮವಾಗಿ ಅವರ ಸಂದೇಶದ ರೂಪುರೇಷೆಗಳೆಂಥವು ಎಂಬು ದರ ಅಭಿಪ್ರಾಯವಾಯಿತು. ಇದರಿಂದ ಆಸಕ್ತರಾದ ಜನ ಅವರಿಂದ ಹೆಚ್ಚಿನ ಬೋಧನೆಗಳನ್ನು ಪಡೆದುಕೊಳ್ಳಲು ಇಲ್ಲವೆ ಅವರ ವೈಯಕ್ತಿಕ ಪರಿಚಯ ಮಾಡಿಕೊಳ್ಳಲು ಗುಂಪುಗುಂಪಾಗಿ ಬರಲಾರಂಭಿಸಿದರು.

ಹೀಗೆ ಈ ಇಂಗ್ಲಿಷು ಜನ ಅತ್ಯಂತ ಉತ್ಸಾಹವನ್ನು ತೋರಿಸುವುದರ ಮೂಲಕ ತಮ್ಮ ಬೋಧನೆಗಳನ್ನು ಹಾಗೂ ತಮ್ಮ ವ್ಯಕ್ತಿತ್ವವನ್ನು ಅರಿತು ಒಪ್ಪಿಗೆ ಸೂಚಿಸಿದುದನ್ನು ಕಂಡು ಸ್ವಾಮೀಜಿಗೆ ತುಂಬ ಸಮಾಧಾನವಾಯಿತು. ಇದು ಅವರ ಪಾಲಿಗೆ ಒಂದು ನವೀನ ಅನುಭವವೇ ಆಯಿತೆಂದೂ ಹೇಳಬಹುದು. ಲಂಡನ್ನಿನ ಪ್ರಿನ್ಸಸ್ ಹಾಲ್​ನಲ್ಲಿ ಭಾಷಣ ಮಾಡಿದ ಮೇಲೆ ಸ್ವಾಮೀಜಿ ಲಂಡನ್ನಿನಲ್ಲಿದ್ದುದು ಕೇವಲ ಒಂದು ತಿಂಗಳ ಕಾಲ ಮಾತ್ರವೇ ಆದರೂ ಆ ಅಲ್ಪಾವಧಿ ಯಲ್ಲೇ ಅವರು ತಮ್ಮ ಮುಂದಿನ ಕಾರ್ಯಕ್ಕೆ ಭದ್ರವಾದ ಅಡಿಪಾಯವನ್ನು ಹಾಕಿದರು. ಮತ್ತು ಈ ಅವಧಿಯಲ್ಲೇ ಅವರು ತಮ್ಮ ಸಂಪರ್ಕಕ್ಕೆ ಬಂದ ಪ್ರತಿಯೊಬ್ಬರ ಮೇಲೂ ಅಚ್ಚಳಿಯದ ಚಿರಮುದ್ರೆಯನ್ನೊತ್ತಿದರು.

ಇದೇ ಸಂದರ್ಭದಲ್ಲಿ ಅವರು ಲಂಡನ್ನಿನಲ್ಲಿ ಹಲವಾರು ಗಣ್ಯವ್ಯಕ್ತಿಗಳ ಮನೆಗಳಲ್ಲಿ ತರಗತಿ ಗಳನ್ನು ತೆಗೆದುಕೊಳ್ಳುತ್ತಿದ್ದರು. ಆ ಸಂಬಂಧವಾಗಿ ಮುಂದೊಮ್ಮೆ ಪತ್ರ ಬರೆಯುತ್ತ ಹೀಗೆ ತಿಳಿಸಿದರು–“ನಾನು ಆ ದಿನಗಳಲ್ಲಿ ಸಾರ್ವಜನಿಕ ಉಪನ್ಯಾಸಗಳಲ್ಲದೆ ವಾರಕ್ಕೆ ಎಂಟು ತರಗತಿ ಗಳನ್ನು ತೆಗೆದುಕೊಳ್ಳುತ್ತಿದ್ದೆ. ಈ ತರಗತಿಗಳಲ್ಲಿ ಜನ ಕಿಕ್ಕಿರಿದಿರುತ್ತಿದ್ದರು. ಸಮಾಜದ ಉನ್ನತ ವರ್ಗದ ಮಹಿಳೆಯರೂ ನೆಲದ ಮೇಲೆ ಕಾಲುಮಡಿಸಿಕೊಂಡು ಕುಳಿತಿರುತ್ತಿದ್ದರು. ಆದರೂ ಆ ಬಗ್ಗೆ ಅವರು ಅನ್ಯಥಾ ಭಾವಿಸುತ್ತಿರಲಿಲ್ಲ.”

ಸ್ವಾಮೀಜಿ ಲಂಡನ್ನಿನಲ್ಲಿ ನೀಡಿದ ಒಂದು ಉಪನ್ಯಾಸದ ಸಾರಾಂಶವನ್ನು ‘ಇಂಡಿಯನ್ ಮಿರರ್​’ ಪತ್ರಿಕೆ ಹೀಗೆ ಪ್ರಕಟಿಸಿತು:

“ಬಲೂನ್ ಸೊಸೈಟಿಯ ವಾರದ ಸಭೆಯೊಂದರಲ್ಲಿ ಸ್ವಾಮಿ ವಿವೇಕಾನಂದರು ‘ವೇದಾಂತದ ಬೆಳಕಿನಲ್ಲಿ ಮಾನವ ಮತ್ತು ಸಮಾಜ’ ಎಂಬ ವಿಷಯವಾಗಿ ಭಾಷಣ ಮಾಡಿದರು. ಸಮಾಜದ ರಚನೆಯಲ್ಲಿ ಧರ್ಮ ಎಂಬುದು ಅತ್ಯದ್ಭುತವಾದ ಒಂದು ಅಂಶ ಎಂದರು ಸ್ವಾಮೀಜಿ. ವಿಜ್ಞಾನ ದಿಂದ ಒದಗುವ ಅತಿದೊಡ್ಡ ಲಾಭವೇ ಜ್ಞಾನವೆನ್ನುವುದಾದರೆ ಧರ್ಮದ ಅಧ್ಯಯನದಿಂದ ಪ್ರಾಪ್ತ ವಾಗುವ ಮನುಷ್ಯನ ಸ್ವರೂಪ ಜ್ಞಾನ, ಆತ್ಮಜ್ಞಾನ ಹಾಗೂ ಭಗವದ್​ಜ್ಞಾನಕ್ಕಿಂತ ಮಿಗಿಲಾದುದು ಇನ್ನಾವುದಿರಬಲ್ಲುದು? (ಇನ್ನೊಂದು ವಿಚಾರವೇನೆಂದರೆ) ಸಮಸ್ತ ಜಗತ್ತಿಗೆಲ್ಲ ಒಂದೇ ಧರ್ಮ ವಿರಬೇಕೆನ್ನುವುದು ಸಾಧ್ಯವೇ ಇಲ್ಲ, ಅಷ್ಟೇ ಅಲ್ಲ, ಅದು ಅಪಾಯಕರ ಕೂಡ. ಸಕಲ ಧಾರ್ಮಿಕ ಭಾವನೆಗಳೂ ಒಂದೇ ಮಟ್ಟದಲ್ಲಿರಬೇಕೆಂದರೆ ಅದು ಧಾರ್ಮಿಕ ಭಾವನೆಗಳ ನಾಶಕ್ಕೆ ನಾಂದಿ ಯಾದೀತು. ವಿವಿಧತೆಯೇ ಅದರ ಲಕ್ಷಣ. ದುರದೃಷ್ಟವಶಾತ್ ಪ್ರತಿಯೊಬ್ಬರೂ ತಮ್ಮತಮ್ಮ ಧರ್ಮದಲ್ಲೇ ಎಷ್ಟು ಮುಳುಗಿದ್ದಾರೆಂದರೆ ಅವರಿಗೆ ಈ ಜಗತ್ತಿನಲ್ಲಿ ಇನ್ನೇನಿದೆಯೆಂದು ನೋಡುವುದಕ್ಕೂ ಕಣ್ಣಿಲ್ಲವಾಗಿಬಿಟ್ಟಿದೆ. ಅಲ್ಲದೆ ಇತರರೂ ತಮ್ಮ ದಾರಿಯಲ್ಲೇ ಬರುವಂತೆ ಮಾಡಲು ಹೋರಾಡುತ್ತಾರೆ. ಇತರ ಎಲ್ಲ ಧರ್ಮಗಳಿಗೂ ಮನ್ನಣೆ ಕೊಡುವ ಧರ್ಮವೇ ಶ್ರೇಷ್ಠ ವಾದ ಧರ್ಮ. ವೇದಾಂತ ಧರ್ಮವು ಇತರ ಎಲ್ಲವನ್ನೂ ತನ್ನೊಳಗೆ ಅಂತರ್ಗತಗೊಳಿಸಿ ಕೊಂಡಿದೆ. ಇಲ್ಲಿ ಪ್ರತಿಯೊಬ್ಬನಿಗೂ ಅವನವನ ಸ್ವಭಾವಕ್ಕೆ ಅನುಗುಣವಾದ ಮಾರ್ಗವನ್ನು ಅನುಸರಿಸಲು ಅವಕಾಶವಿದೆ ಎಂದರು ಸ್ವಾಮೀಜಿ.”

ಸ್ವಾಮೀಜಿ ನ್ಯೂಯಾರ್ಕಿನಲ್ಲಿ ಮಾಡಿದಂತೆ ಇಂಗ್ಲೆಂಡಿನಲ್ಲಿಯೂ ವಿಶ್ರಾಂತಿಯೇ ಇಲ್ಲದೆ ನಿರಂತರವಾಗಿ ಶ್ರಮಿಸಿದರು. ಬಳಿಗೆ ಬಂದವರಿಗೆ ಸಂಪೂರ್ಣ ಗಮನಗೊಟ್ಟು ಹೃತ್ಪೂರ್ವಕವಾಗಿ ಬೋಧಿಸಿದರು. ಅವರ ಪ್ರಭಾವಲಯವು ಸ್ಥಿರವಾಗಿ ವಿಸ್ತಾರಗೊಳ್ಳುತ್ತ ಬಂದಿತು; ಅವರ ಪ್ರಭಾವ ದಿನೇದಿನೇ ಹೆಚ್ಚುತ್ತ ಬಂದಿತು. ಅವರಿಗೆ ಹಲವಾರು ಜನರ ಪರಿಚಯವಾಗುವಂತೆ, ತರಗತಿಗಳು ಏರ್ಪಡುವಂತೆ ಇ. ಟಿ. ಸ್ಟರ್ಡಿ ಅಪಾರ ನೆರವು ನೀಡಿದ. ಅಲ್ಲದೆ ಈ ವೇಳೆಗೆ ಪ್ಯಾರಿಸಿ ನಿಂದ ಲಂಡನ್ನಿಗೆ ಬಂದಿದ್ದ ಮಿಸ್ ಮೆಕ್​ಲಾಡ್ ಹಾಗೂ ಲೆಗೆಟ್ ದಂಪತಿಗಳು ಕೂಡ ಹಲವಾರು ಹೊಸಬರನ್ನು ಕರೆತಂದು ಸ್ವಾಮೀಜಿಗೆ ಪರಿಚಯ ಮಾಡಿಸಿಕೊಟ್ಟು ಅವರ ವೇದಾಂತ ಪ್ರಸಾರ ಕಾರ್ಯಕ್ಕೆ ತುಂಬ ನೆರವಾದರು.

ಸ್ವಾಮೀಜಿ ಇದ್ದಂತಹ ಓಕ್​ಲೆ ಸ್ಟ್ರೀಟಿನ ತರಗತಿಗಳಿಗೆ ಬರುತ್ತಿದ್ದ ಮೊದಲಿಗರಲ್ಲಿ ಲೇಡಿ ಇಸಾಬೆಲ್ ಮಾರ್ಗೆಸ್ಸನ್ ಒಬ್ಬಳು. ಇವಳು ಸಮಾಜ ಸುಧಾರಣೆ ಹಾಗೂ ವಿದ್ಯಾಭ್ಯಾಸ ಸುಧಾರಣೆ ಗಳ ಕಾರ್ಯದಲ್ಲಿ ವಿಶೇಷ ಆಸಕ್ತಿಯಿದ್ದವಳು. ಇವಳು ಸ್ವಾಮೀಜಿಯನ್ನು ತನ್ನ ಮನೆಗೆ ಆಮಂತ್ರಿಸಿ ಅಲ್ಲಿಗೆ ಆಹ್ವಾನಿತರಾದ ತನ್ನ ಕೆಲವು ಸ್ನೇಹಿತರನ್ನುದ್ದೇಶಿಸಿ ಮಾತನಾಡಬೇಕೆಂದು ಕೇಳಿಕೊಂಡಳು. ಅಂದು ಭಾನುವಾರ. ಅಲ್ಲಿ ಆಹ್ವಾನಿತರಾಗಿದ್ದವರಲ್ಲಿ ಮಿಸ್ ಮಾರ್ಗರೆಟ್ ನೋಬೆಲ್ ಒಬ್ಬಳು. ಇವಳೇ ಮುಂದೆ ವಿಖ್ಯಾತಳಾದ ಸೋದರಿ ನಿವೇದಿತಾ. ಸ್ವಾಮೀಜಿಯ ಆಧ್ಯಾತ್ಮಿಕ ಸಂಸ್ಕೃತಿಯ ವಿಸ್ತಾರವನ್ನು ಕಂಡು ಮಿಸ್ ನೋಬೆಲ್ ಬೆರಗಾದಳು. ತಾತ್ತ್ವಿಕ ವಿಚಾರ ಗಳಲ್ಲಿ ಅವರಿಗಿದ್ದ ನಿಖರ-ನಿಶ್ಚಿತ ವಿಪುಲ ಜ್ಞಾನ ಅವಳನ್ನು ದಿಗ್ಭ್ರಮೆಗೊಳಿಸಿತು. ಅಲ್ಲದೆ ಮುಂದೆ ಆಕೆಯೇ ಹೇಳುವಂತೆ, ಆಕೆಗೆ ಸ್ವಾಮೀಜಿಯ ದಿವ್ಯ ಕರೆಯೊಂದು ಕೇಳಿಬಂತು.

ಮಿಸ್ ನೋಬೆಲ್ಲಳು ಶಿಕ್ಷಣ ಕ್ಷೇತ್ರದಲ್ಲಿ ಬಹಳವಾಗಿ ಆಸಕ್ತಳಾದವಳು. ಅವಳು ತಾನೇ ಸ್ಥಾಪಿಸಿದ ವಿದ್ಯಾಸಂಸ್ಥೆಯೊಂದರ ಪ್ರಿನ್ಸಿಪಾಲಳಾಗಿದ್ದಳು. ವಿದ್ಯಾಕ್ಷೇತ್ರದಲ್ಲಿ ಒಂದು ಉನ್ನತ ಮಟ್ಟದ ಸುಧಾರಣೆಯನ್ನು ಮಾಡಬೇಕೆಂಬುದು ಅವಳ ಅಭಿಲಾಷೆಯಾಗಿತ್ತು. ಇವಳು ಯಾವಾ ಗಲೂ ವಿಚಾರವಂತಹ ವಲಯಗಳಲ್ಲಿ ಓಡಾಡಿಕೊಂಡಿರುತ್ತಿದ್ದಳು. ಎಲ್ಲ ಬಗೆಯ ಆಧುನಿಕ ವಿಚಾರಗಳಲ್ಲಿ ಹಾಗೂ ಆಲೋಚನಾಲಹರಿಗಳಲ್ಲಿ ಆಳವಾದ ಆಸಕ್ತಿ ವಹಿಸುತ್ತಿದ್ದಳು. ಸ್ವಾಮೀಜಿಯ ಮಾತುಗಳನ್ನು ಮಾರ್ಗರೆಟ್ ತುಂಬ ಎಚ್ಚರದಿಂದ ತೂಗಿನೋಡಿದಳು. ಮೊದ ಮೊದಲಿಗೆ ಅವರ ಅನೇಕ ವಿಚಾರಗಳು ಅವಳಿಗೆ ಒಪ್ಪಿಗೆಯಾಗಲಿಲ್ಲ. ಆದ್ದರಿಂದ ಅವಳು ಆ ಕುರಿತಾಗಿ ಮತ್ತೆಮತ್ತೆ ಪ್ರಶ್ನಿಸುತ್ತಿದ್ದಳು. ಅವಳ ಮನೋಭಾವವನ್ನು ಗಮನಿಸಿದ ಸ್ವಾಮೀಜಿ ಅದು ಅವಳ ವೈಚಾರಿಕ ಶಕ್ತಿಯ ಲಕ್ಷಣವೆಂದು ತಿಳಿದರು. ಅವಳು ತಮ್ಮ ಮಾತುಗಳನ್ನು ಸ್ವೀಕರಿ ಸಲು ಮೊದಮೊದಲಿಗೆ ಹಿಂಜರಿದರೂ ಒಮ್ಮೆ ತಮ್ಮ ಭಾವನೆಗಳನ್ನೂ ತತ್ತ್ವಗಳನ್ನೂ ಸ್ವೀಕರಿಸಿದ ಳೆಂದರೆ ಆಮೇಲೆ ಅವುಗಳನ್ನು ಅವಳಷ್ಟು ಉಗ್ರವಾಗಿ ನಿಷ್ಠಾಯುತವಾಗಿ ಪ್ರತಿಪಾದಿಸುವವರು ಬೇರಾರೂ ಇರಲಾರರು ಎಂದು ಅರಿತರು. ಅಂತೂ ಸ್ವಾಮೀಜಿಯ ತತ್ತ್ವಗಳನ್ನು ಸಮಗ್ರವಾಗಿ ಸ್ವೀಕರಿಸಲು ನೋಬೆಲ್ಲಳಿಗೆ ಹಲವು ತಿಂಗಳೇ ಬೇಕಾದುವು. ತಾನು ಅವರನ್ನು ಲೇಡಿ ಇಸಾಬೆಲ್ ಮಾರ್ಗೆಸ್ಸನ್ನಳ ಮನೆಯಲ್ಲಿ ಮೊದಲ ಸಲ ನೋಡಿದ್ದರ ಕುರಿತಾಗಿ ತುಂಬ ಸ್ವಾರಸ್ಯವಾಗಿ ಬರೆದಿದ್ದಾಳೆ:

“ನಾನವರನ್ನು ಮೊದಲ ಸಲ ಸಂದರ್ಶಿಸಿದ್ದು ನವೆಂಬರ್ ತಿಂಗಳ ಒಂದು ಭಾನುವಾರದ ಅಪರಾಹ್ನ; ಸ್ಥಳ ವೆಸ್ಟ್ ಎಂಡಿನ ಬೈಠಕ್ ಖಾನೆ. ಅವರು ಅರ್ಧವೃತ್ತಾಕಾರವಾಗಿ ಕುಳಿತು ಆಲಿಸುತ್ತಿದ್ದ ತಮ್ಮ ಶ್ರೋತೃಗಳಿಗೆ ಅಭಿಮುಖವಾಗಿ ಕುಳಿತಿದ್ದರು. ಅವರ ಹಿಂದೆ ಅಗ್ಗಿಷ್ಟಿಕೆಯಲ್ಲಿ ಬೆಂಕಿ ಉರಿಯುತ್ತಿತ್ತು. ಆ ಸಂದರ್ಭವನ್ನು ನೆನಪಿಸಿಕೊಂಡಾಗ ಇಂದಿಗೂ ನನ್ನ ಮನಸ್ಸಿನಲ್ಲಿ ಭಾವನಾತರಂಗಗಳು ಏಳುವಂತೆಯೇ, ಆಗ ಅವರ ಮನಸ್ಸಿನಲ್ಲೂ ರವಿಕಿರಣಾವೃತವಾದ ತಮ್ಮ ತಾಯ್ನಾಡಿನ ಹಲವಾರು ನೆನಪುಗಳು ಮೂಡುತ್ತಿದ್ದಿರಬೇಕು. ಅವರು ಪ್ರಶ್ನೆಗಳ ಮೇಲೆ ಪ್ರಶ್ನೆ ಗಳನ್ನು ಉತ್ತರಿಸುತ್ತ ತಮ್ಮ ಮಾತಿಗೆ ನಿದರ್ಶನರೂಪವಾಗಿ ತಮ್ಮ ಸುಮಧುರ ಕಂಠದಿಂದ ಯಾವುದಾದರೂ ಸಂಸ್ಕೃತ ಶ್ಲೋಕವನ್ನು ಪಠಿಸುತ್ತಿದ್ದರು. ಸಂಜೆಗತ್ತಲು ನಿಧಾನವಾಗಿ ಆವರಿಸು ತ್ತಿದ್ದ ಆ ಸಮಯದಲ್ಲಿ ಅಂದಿನ ದೃಶ್ಯವು ಅವರ ಮನಸ್ಸಿನಲ್ಲಿ ಭರತಖಂಡದ ಉದ್ಯಾನವೊಂದರ ಸಂಧ್ಯಾಕಾಲದ ಸುಂದರ ದೃಶ್ಯವನ್ನೋ ಅಥವಾ ಸಂಜೆಯ ವೇಳೆಗೆ ಹಳ್ಳಿಯ ಹೊರಗಿನ ಮರವೊಂದರ ಕೆಳಗೆ ಸಾಧುವೊಬ್ಬ ಹಳ್ಳಿಗರಿಂದ ಸುತ್ತುವರಿಯಲ್ಪಟ್ಟು ಕುಳಿತ ದೃಶ್ಯವನ್ನೋ ಮೂಡುವಂತೆ ಮಾಡಿರಬಹುದು. ಸ್ವಾಮೀಜಿಯನ್ನು ಒಬ್ಬ ಬೋಧಕರಾಗಿ ಇಷ್ಟು ಸರಳವಾದ ರೀತಿಯಲ್ಲಿ ಮತ್ತೆಂದೂ ನಾನು ಇಂಗ್ಲೆಂಡಿನಲ್ಲಿ ನೋಡಲಿಲ್ಲ. ಇದಾದ ಬಳಿಕ ಅವರು ಯಾವಾಗಲೂ ದೊಡ್ಡ ಸಾರ್ವಜನಿಕ ಸಭೆಗಳಲ್ಲಿ ಉಪನ್ಯಾಸ ಮಾಡುತ್ತಿದ್ದರು; ಇಲ್ಲವೆ ಸಭೆ ಗಳಲ್ಲಿ ಜನ ಔಪಚಾರಿಕವಾಗಿ ಕೇಳಿದ ಪ್ರಶ್ನೆಗಳಿಗೆ ಉತ್ತರಿಸುತ್ತಿದ್ದರು. ಆ ಮೊದಲ ಸಲ ಮಾತ್ರವೇ ನಾವು ಕೇವಲ ಹದಿನೈದು-ಹದಿನಾರು ಜನರಿದ್ದುದು. ಅದರಲ್ಲೂ ಬಹಳಷ್ಟು ಜನ ಅವರ ಆತ್ಮೀಯ ಸ್ನೇಹಿತರು. ಅಲ್ಲದೆ ಅವರು ನಮ್ಮ ನಡುವೆಯೇ ಕುಳಿತಿದ್ದರು. ಅವರು ದೂರದ ನಾಡೊಂದರಿಂದ ಸಂದೇಶ ತಂದಂತಿತ್ತು. ಮಧ್ಯೆ ಮಧ್ಯೆ ‘ಶಿವ ಶಿವ’ ಎನ್ನುವ ವಿಚಿತ್ರ ಸ್ವಭಾವ. ತಮ್ಮ ಹೆಚ್ಚಿನ ಸಮಯವನ್ನೆಲ್ಲ ಧ್ಯಾನದಲ್ಲಿ ಕಳೆಯುವವರ ಮುಖದಲ್ಲಿ ಕಂಡುಬರುವಂತಹ ಮಾಧುರ್ಯ-ಭವ್ಯತೆಗಳೆರಡೂ ಅವರ ಮುಖದಲ್ಲಿ ಸಮ್ಮಿಳಿತವಾಗಿದ್ದುವು. ಅದು ಪ್ರಾಯಶಃ ರಾಫೇಲ್ ಚಿತ್ರಿಸಿದ ಬಾಲಕ್ರಿಸ್ತನ ಮುಖವನ್ನು ಹೋಲುತ್ತಿತ್ತು.”

ಸ್ವಾಮೀಜಿಯ ಬೋಧನೆಗಳಲ್ಲಿ ಕೆಲವನ್ನು ಮಾರ್ಗರೆಟ್ ಕೂಡಲೇ ಒಪ್ಪಿಕೊಂಡು ಸ್ವೀಕರಿಸಿದ ಳಾದರೂ ಇತರ ಹಲವನ್ನು ಆಕೆ ಒಮ್ಮೆಗೇ ಸ್ವೀಕರಿಸಲು ಸಿದ್ಧಳಿರಲಿಲ್ಲ. ಅವುಗಳಲ್ಲಿ ಕೆಲವನ್ನು ಪರೀಕ್ಷಿಸಬೇಕೆಂದು ಆಕೆಗೆ ತೋರಿತು. ಅವರ ಕೆಲವು ಸಿದ್ಧಾಂತಗಳು ಇತರ ಅನೇಕರ ಸಿದ್ಧಾಂತಗಳಿ ಗಿಂತಲೂ ವಿಭಿನ್ನವಲ್ಲವೆಂದು ಅವಳು ಭಾವಿಸಿದ್ದಳು. ಆದ್ದರಿಂದ ಅಂಥವುಗಳನ್ನು ಅನುಸರಿಸಿ ದರೆ ಮುಂದೆ ಪಶ್ಚಾತ್ತಾಪ ಪಡಬೇಕಾಗಬಹುದೆಂದು ಭಾವಿಸಿದಳು. ಆದರೆ ಅವಳ ಸಂಶಯಗಳೆಲ್ಲ ನಿಧಾನವಾಗಿ ಪರಿಹಾರವಾಗಲಾರಂಭಿಸಿದುವು. ಸ್ವಾಮೀಜಿಯ ಸಂಪರ್ಕಕ್ಕೆ ಬಂದ ಮೇಲಿನ ಒಂದು ತಿಂಗಳ ಕಾಲದಲ್ಲೇ ಆಕೆ ಅವರಿಂದ ಸಾಕಷ್ಟು ತೀವ್ರವಾಗಿ ಪ್ರಭಾವಿತಳಾದಳು. ಆ ಅವಧಿಯಲ್ಲೇ ಅವಳು ಸ್ವಾಮೀಜಿಯ ಶಿಷ್ಯತ್ವವನ್ನು ಸ್ವೀಕರಿಸಿದಳೆಂದು ಹೇಳಬೇಕು. ಆ ವಿಷಯ ವಾಗಿ ಅವಳೇ ಮುಂದೆ ‘ನಾ ಕಂಡಂತೆ ನನ್ನ ಗುರುದೇವ\eng{’ (The Master as I saw him)} ಎಂಬ ಪುಸ್ತಕದಲ್ಲಿ ಬಣ್ಣಿಸುತ್ತಾಳೆ:

“ಈ ಸಂಧಿಕಾಲದ ಬಗ್ಗೆ ಸ್ಪಷ್ಟವಾಗಿ ತಿಳಿಸುವುದು ಕಷ್ಟ. ಸ್ವಾಮೀಜಿ ಇಂಗ್ಲೆಂಡಿನಿಂದ ಹೊರಡುವ ಮೊದಲೇ ನಾನವರನ್ನು ‘ಗುರುದೇವ’ ಎಂದು ಸಂಬೋಧಿಸುವ ಕಾಲ ಬಂದಿತು. ಅವರಲ್ಲಿದ್ದ ವೀರನಾಡಿಯನ್ನು ನಾನು ಗುರುತಿಸಿದ್ದೆ. ಅವರಿಗೆ ತಮ್ಮ ದೇಶಬಾಂಧವರ ಮೇಲಿದ್ದ ಪ್ರೀತಿಗೆ ದಾಸಳಾಗಲು ನಾನು ಇಚ್ಛಿಸಿದೆ. ಆದರೆ ನಿಜಕ್ಕೂ ನಾನು ಶರಣಾಗತಳಾದದ್ದು ಅವರ ಶೀಲಕ್ಕೆ. ಒಬ್ಬ ಧರ್ಮಬೋಧಕನಾಗಿ ಅವರು ತಮ್ಮದೇ ಆದ ಒಂದು ಸಿದ್ದಾಂತವನ್ನು ನೀಡಿದರೂ, ಸ್ವೀಕರಿಸುವವರು ಅದಕ್ಕೇ ದಾಸರಾಗಬೇಕಾಗಿರಲಿಲ್ಲ; ಸತ್ಯ ಅವರನ್ನು ಬೇರೆಡೆಗೆ ಒಯ್ದರೆ ಅದನ್ನು ಅನುಸರಿಸಲು ಅವರಿಗೆ ಸ್ವಾತಂತ್ರ್ಯವಿತ್ತು. ಈ ಅಂಶವನ್ನು ನಾನು ಗುರುತಿಸಿದೆ, ಮತ್ತು ಅಷ್ಟರಮಟ್ಟಿಗೆ ನಾನವರ ಶಿಷ್ಯೆಯಾದೆ. ಆದರೆ ಇನ್ನುಳಿದವುಗಳ ವಿಚಾರದಲ್ಲಿ ಹೇಳುವು ದಾದರೆ, ಅವರ ಬೋಧನೆಗಳು ಎಷ್ಟರಮಟ್ಟಿಗೆ ಸಮಂಜಸ ಎಂಬುದನ್ನು ಕಂಡುಕೊಳ್ಳಲು ನಾನವುಗಳನ್ನು ಸಾಕಷ್ಟು ಅಧ್ಯಯನ ಮಾಡತೊಡಗಿದೆ. ಅವರ ಮಾತುಗಳ ಸತ್ಯತೆಯು ನನ್ನ ಅನುಭವಕ್ಕೆ ಬರುವವರೆಗೂ ನಾನವುಗಳ ವಿಷಯದಲ್ಲಿ ಅಂತಿಮ ತೀರ್ಮಾನವನ್ನು ತೆಗೆದುಕೊಳ್ಳ ಲಿಲ್ಲ. ಅಲ್ಲದೆ, ಆಗ ನಾನವರ ವ್ಯಕ್ತಿತ್ವದಿಂದ ತೀವ್ರವಾಗಿ ಆಕರ್ಷಿತಳಾಗಿದ್ದರೂ ನನಗೆ ತಿಳಿದಿದ್ದ ಇತರ ಚಿಂತಕರು ಹಾಗೂ ಮೇಧಾವಿಗಳಿಗಿಂತಲೂ ಅವರು ಎಷ್ಟು ಮುಂದಿದ್ದಾರೆಂಬುದು ನನ್ನ ಬುದ್ಧಿಗಿನ್ನೂ ಗೋಚರವಾಗಲಿಲ್ಲ.”

ಲಂಡನ್ನಿನ ಹಲವಾರು ಕ್ಲಬ್ಬುಗಳಲ್ಲಿ ಹಾಗೂ ಶ್ರೀಮಂತರ ಮನೆಗಳಲ್ಲಿ ನೀಡುತ್ತಿದ್ದ ಅನೌಪಚಾರಿಕ ಭಾಷಣಗಳಲ್ಲಿ ಹಾಗೂ ಚರ್ಚೆಗಳಲ್ಲಿ ಸ್ವಾಮೀಜಿ ಹೆಚ್ಚಾಗಿ ಮಾತನಾಡುತ್ತಿದ್ದುದು ಹಿಂದೂಧರ್ಮದ ಸಿದ್ಧಾಂತಗಳ ಕುರಿತಾಗಿ, ಅದರಲ್ಲೂ ಮುಖ್ಯವಾಗಿ, ಹಿಂದೂಧರ್ಮದ ತಿರುಳಾದ ವೇದಾಂತತತ್ತ್ವಗಳನ್ನು ಅವರು ವಿಶದವಾಗಿ ವಿವರಿಸಿದರು. ಸಹಜವಾಗಿಯೇ ಅವರು ಅನೇಕ ಬಗೆಯ ಜನರಿಂದ ನಾನಾ ಪ್ರಕಾರದ ಪ್ರಶ್ನೆಗಳ ಸುರಿಮಳೆಯನ್ನೇ ಎದುರಿಸಬೇಕಾಯಿತು. ಅತ್ಯಂತ ಸಮರ್ಥ ಗುರುವಿನಂತೆ ಸ್ವಾಮೀಜಿ ಆ ಎಲ್ಲ ಪ್ರಶ್ನೆಗಳಿಗೂ ಅವುಗಳ ಅರ್ಹತೆಗೆ ತಕ್ಕಂತೆ ಸೂಕ್ತವಾದ ಉತ್ತರಗಳನ್ನು ಕೊಟ್ಟು ಸಮಾಧಾನಗೊಳಿಸುತ್ತಿದ್ದರು. ಆದರೆ ಅಮೆರಿಕಕ್ಕಿಂತ ಕೆಲವು ವಿಷಯಗಳಲ್ಲಾದರೂ ಇಂಗ್ಲೆಂಡ್ ಮೇಲು ಎನ್ನಬಹುದಾಗಿತ್ತು. ಏಕೆಂದರೆ ಇಂಗ್ಲಿಷರು ಭಾರತದ ಆಳರಸರಾಗಿದ್ದುದರಿಂದ ಇಂಗ್ಲೆಂಡ್​-ಭಾರತಗಳ ನಡುವಣ ಸಂಬಂಧವು ಹೆಚ್ಚು ನಿಕಟವಾಗಿತ್ತು. ಆದ್ದರಿಂದ ಭಾರತದ ಬಗ್ಗೆ ಇಂಗ್ಲಿಷರ ತಿಳಿವಳಿಕೆಯೂ ಉತ್ತಮವಾಗಿತ್ತು. ಹಿಂದೂಧರ್ಮದ ಬಗೆಗಿನ ಇಂಗ್ಲಿಷರ ಪರಿಜ್ಞಾನವೂ ಉತ್ತಮವಾಗಿದ್ದು ಸ್ವಾಮೀಜಿಯ ಭಾಷಣ ಗಳಿಗೆ ಸಿಗುತ್ತಿದ್ದ ಪ್ರತಿಕ್ರಿಯೆ ಚೆನ್ನಾಗಿತ್ತು. ಅಮೆರಿಕದಲ್ಲಿ ತಾವು ಸಾರಿದ್ದ ಸಂದೇಶವನ್ನೇ ಅವರು ಇಂಗ್ಲೆಂಡಿನಲ್ಲೂ ಬೋಧಿಸಿದರು. ಸರ್ವಧರ್ಮಗಳ ಹಿಂದಿನ ಸಮಾನ ತತ್ತ್ವಗಳನ್ನೂ ಧರ್ಮಾಂ ಧತೆಯ ಮೂಲಕಾರಣವನ್ನೂ ಅವರು ತೋರಿಸಿಕೊಟ್ಟರು. ಸಕಲ ಧರ್ಮಗಳೂ ತ್ಯಾಗವನ್ನೇ ಬೋಧಿಸುತ್ತವೆ, ತ್ಯಾಗವಿಲ್ಲದೆ ಧರ್ಮಾನುಷ್ಠಾನವೂ ಸಿದ್ಧಿಯೂ ಸಾಧ್ಯವೇ ಇಲ್ಲ ಎಂದು ಪ್ರತಿಪಾದಿಸಿದರು.

ಸ್ವಾಮೀಜಿಯ ಪ್ರತಿಯೊಂದು ಉಪನ್ಯಾಸವೂ ಆ ಪಾಶ್ಚಾತ್ಯ ಶ್ರೋತೃಗಳಿಗೆ ಜ್ಞಾನಪ್ರದ ವಾಗಿತ್ತು. ಅವುಗಳಲ್ಲಿ ಕೆಲವಂತೂ ಅತ್ಯಂತ ಸ್ಫೂರ್ತಿದಾಯಕವಾಗಿದ್ದುವು. ಯಾವುದಾದ ರೊಂದು ವಿಷಯವನ್ನು ಅವರು ತನ್ಮಯತೆಯಿಂದ ವಿವರಿಸತೊಡಗಿದರೆಂದರೆ ಅವರ ಮುಖ ದಲ್ಲಿ ವ್ಯಕ್ತವಾಗುತ್ತಿದ್ದ ಭಾವೋದ್ವೇಗ, ಅಥವಾ ಇದ್ದಕ್ಕಿದ್ದಂತೆ ಅವರ ಕಣ್ಣುಗಳಿಂದ ಹೊರ ಹೊಮ್ಮುತ್ತಿದ್ದ ಮಿಂಚು, ಅವರ ಯಾವುದೇ ಒಂದು ಮಾತಿನ ಮೂಲಕ ಹರಿದುಬರುತ್ತಿದ್ದ ಉಜ್ವಲ ಭಾವಪ್ರವಾಹ, ಇಲ್ಲವೆ ಶ್ರೋತೃಗಳ ದೃಷ್ಟಿಯಿಂದ ಅವರ (ಸ್ವಾಮೀಜಿಯ) ಅಸ್ತಿತ್ವ ವನ್ನೇ ಮರೆಮಾಡಿ ಪ್ರಕಟಗೊಳ್ಳುತ್ತಿದ್ದ ಅವರ ಅತೀಂದ್ರಿಯ-ಜ್ಯೋತಿರ್ಮಯ ವ್ಯಕ್ತಿತ್ವ–ಈ ಅಂಶಗಳಲ್ಲಿ ಯಾವುದಾದರೊಂದು ಅಥವಾ ಒಂದಕ್ಕಿಂತ ಹೆಚ್ಚು ಅಂಶಗಳು ಶ್ರೋತೃಗಳನ್ನು ಸಂಮೋಹನಗೊಳಿಸಿಬಿಡುತ್ತಿದ್ದುವು.

ಒಂದು ದಿನ ಲಂಡನ್ ಸಮಾಜದ ಅತ್ಯುನ್ನತ ವರ್ಗ ಮಹಿಳಾಮಣಿಯರ ಗುಂಪೊಂದನ್ನು ಉದ್ದೇಶಿಸಿ ಸ್ವಾಮೀಜಿ ಮಾತನಾಡುತ್ತಿದ್ದರು. ಭಕ್ತಿಮಾರ್ಗದ ಹಿರಿಮೆಯನ್ನು ಅವರು ಬಣ್ಣಿಸು ತ್ತಿದ್ದರು. ಅತ್ಯುನ್ನತವಾದ ಪ್ರೀತಿಯು ವ್ಯಕ್ತಿಯನ್ನು ಹೇಗೆ ಅತ್ಯಂತ ನಿಃಸ್ವಾರ್ಥಿಯನ್ನಾಗಿ ಮಾಡ ಬಲ್ಲುದು, ವ್ಯಕ್ತಿಯೊಳಗಿನ ಆತ್ಮಶಕ್ತಿಯನ್ನು ಹೇಗೆ ಪ್ರಕಟಗೊಳಿಸಬಲ್ಲುದು ಎಂಬುದನ್ನು ಸ್ವಾಮೀಜಿ ಸ್ಫೂರ್ತಿಯುತರಾಗಿ ಬಣ್ಣಿಸುತ್ತ, ಶ್ರೋತೃಗಳ ಮನಮುಟ್ಟುವಂತಹ ಒಂದು ಉದಾ ಹರಣೆಯನ್ನು ಕೊಟ್ಟರು–“ಈಗ ನೀವು ರಸ್ತೆಯಲ್ಲಿ ಹೋಗುತ್ತಿರುವಾಗ ಇದ್ದಕ್ಕಿದ್ದಂತೆ ಒಂದು ಹೆಬ್ಬುಲಿ ನಿಮಗೆದುರಾಗಿ ನಿಂತು ಗರ್ಜಿಸಿತು ಎಂದಿಟ್ಟುಕೊಳ್ಳಿ. ಆಗ ನಿಮಗೆ ಹೇಗಾದೀತು! ನಿಮ್ಮ ಜೀವವನ್ನು ಉಳಿಸಿಕೊಳ್ಳಲು ನೀವು ಹೇಗೆ ಉನ್ಮತ್ತರಂತೆ ಅಲ್ಲಿಂದ ಓಡಲಿಕ್ಕಿಲ್ಲ! ಆದರೆ...” ಹೀಗೆನ್ನುತ್ತಿದ್ದಂತೆ ಸ್ವಾಮೀಜಿಯ ಸ್ವರ ಬದಲಿಸಿತು; ಅವರ ಮುಖ ವಿಶೇಷ ಭಾವದಿಂದ ಜಾಜ್ವಲ್ಯಮಾನವಾಯಿತು. ಆಧ್ಯಾತ್ಮಿಕ ಶಕ್ತಿಯಿಂದ ಮಾತ್ರ ಪ್ರಕಟವಾಗಬಲ್ಲ ತೇಜಸ್ಸು ಮತ್ತು ನಿರ್ಭಯತೆ ಅವರ ವದನದಲ್ಲಿ ವ್ಯಕ್ತವಾದುವು! “ಆದರೆ... ಒಂದು ವೇಳೆ ಆ ಹುಲಿಯ ಹಾಗೂ ನಿಮ್ಮ ನಡುವೆ ನಿಮ್ಮ ಪುಟ್ಟ ಮಗುವಿದೆ ಎಂದು ಭಾವಿಸಿ. ಮರುಕ್ಷಣದಲ್ಲಿ ನೀವೆಲ್ಲಿರು ತ್ತಿದ್ದಿರಿ?–ಹುಲಿಯ ಬಾಯ ಮುಂದೆ! ನಿಮ್ಮಲ್ಲಿ ಯಾರೇ ಆದರೂ ಸರಿಯೆ–ನೇರವಾಗಿ ಹುಲಿಯ ಬಾಯ ಮುಂದೆಯೇ–ಇದು ಖಂಡಿತ!” ಸ್ವಾಮೀಜಿಯ ಮಾತಿನ ರಭಸವು ಆ ಯುವಮಾತೆಯರ ಹೃದಯಗಳನ್ನು ಹೇಗೆ ತಟ್ಟಿತೆಂದರೆ ಅವರೆಲ್ಲ ಸ್ತಂಭೀಭೂತರಾದರು. ಸ್ವಾಮೀಜಿಯ ಮಾತಿನ ಸತ್ಯತೆಯನ್ನು ಯಾರೂ ಪ್ರಶ್ನಿಸುವಂತಿರಲಿಲ್ಲ. ಪ್ರತಿಯೊಬ್ಬರಿಗೂ ಅವರ ವರ ಪ್ರಾಣವು ಅತ್ಯಮೂಲ್ಯವಾದುದೇ ಸರಿ. ಹುಲಿಯ ಬಾಯಿಗೆ ಸಿಕ್ಕಿಕೊಳ್ಳುವಂತಹ ಪರಿಸ್ಥಿತಿ ಒದಗಿದರೆ, ಅಂತಹ ಸಾವಿನಿಂದ ತಪ್ಪಿಸಿಕೊಳ್ಳಲು ಯಾರೇ ಆಗಲಿ ಶಕ್ತಿ ಮೀರಿ ಪ್ರಯತ್ನಿಸುತ್ತಾರೆ. ಅಂತೆಯೇ ಒಬ್ಬಳು ಸ್ತ್ರೀಯೂ ಕೂಡ. ಆದರೆ ಅವಳ ಮಗುವೇ ಹುಲಿಗಾಹುತಿಯಾಗುವ ಸಂದರ್ಭ ಒದಗಿದರೆ ಆಗ ಅವಳ ಮಾತೃಪ್ರೇಮವು ಪೂರ್ಣಪ್ರಮಾಣದಲ್ಲಿ ಪ್ರಕಟಗೊಂಡು, ತನ್ನ ಪ್ರಾಣ ವನ್ನೇ ಬಲಿತೆತ್ತು ಮಗುವನ್ನು ಬದುಕಿಸಲು ಹೋರಾಡುವಂತೆ ಆಕೆಯನ್ನು ಪ್ರೇರೇಪಿಸುತ್ತದೆ. ಈ ಸಂದಿಗ್ಧ ಪರಿಸ್ಥಿತಿಯಲ್ಲಿ ಸ್ವಾರ್ಥದ ಪ್ರಶ್ನೆಯೇ ಏಳುವುದಿಲ್ಲ ಎಂಬುದು ಸ್ವಾಮೀಜಿಯ ಮಾತಿನ ಅರ್ಥ. ಅವರ ಈ ಒಂದು ಮಾತು ಆ ಕ್ಷಣದಲ್ಲಿ ಆ ಮಾತೆಯರೆಲ್ಲರ ಸುಪ್ತ ಆಧ್ಯಾತ್ಮಿಕ ಚೇತನವನ್ನು ಬಡಿದೆಬ್ಬಿಸಿ ವ್ಯಕ್ತವಾಗುವಂತೆ ಮಾಡಿತು.

ಸ್ವಾಮೀಜಿ ಲಂಡನ್ನಿಗೆ ಬಂದದ್ದರಿಂದ ಒಂದು ಪ್ರಾಮುಖ್ಯವಾದ ಕಾರ್ಯವನ್ನು ಸಾಧಿಸಿ ದಂತಾಯಿತು. ಏನೆಂದರೆ ತಾವು ಇಂಗ್ಲೆಂಡಿನಲ್ಲಿ ಮುಂದೆ ಏನೇನು ಕಾರ್ಯಗಳನ್ನು ಕೈಗೊಳ್ಳ ಬೇಕೆಂದು ಯೋಜನೆ ಹಾಕಿಕೊಂಡಿದ್ದರೊ ಅವುಗಳಿಗೆಲ್ಲ ಒಂದು ಭದ್ರವಾದ ಬುನಾದಿಯನ್ನು ಹಾಕಿದಂತಾಯಿತು. ಇಂಗ್ಲೆಂಡಿಗೆ ಬರುವ ಮೊದಲು ಸ್ವಾಮೀಜಿ ಭಾವಿಸಿದ್ದರು, ತಮ್ಮ ಲಂಡನ್ ಭೇಟಿಯು ಕೇವಲ ಒಂದು ಪ್ರವಾಸವಾಗಬಹುದು ಅಥವಾ ನಗರದರ್ಶನವಾಗಬಹುದು ಎಂದು. ಆದರೆ ಅವರು ಲಂಡನ್ನಿಗೆ ಪ್ರವೇಶ ಮಾಡಿದ ಮೇಲೆ ಅವರಿಗೆ ತಿಳಿಯಿತು, ಇಲ್ಲಿ ತಮ್ಮ ಕಾರ್ಯವು ಕೇವಲ ಪ್ರಾಯೋಗಿಕವಾಗಿ ಉಳಿಯಲಿಲ್ಲ; ಬದಲಾಗಿ ನೇರವಾಗಿ ಹಾಗೂ ಶ್ರೀಘ್ರ ವಾಗಿ ತಮ್ಮ ನಿರೀಕ್ಷಣೆಯನ್ನೂ ಮೀರಿ ಯಶಸ್ಸಿನ ಶಿಖರವನ್ನೇ ಮುಟ್ಟಿಬಿಟ್ಟಿದೆ ಎಂದು. ಪತ್ರಿಕೆಗಳಂತೂ ಸ್ವಾಮೀಜಿಯ ಸಂದೇಶಗಳನ್ನು ಸ್ವಾಗತಿಸಿದುವಲ್ಲದೆ ಅವನ್ನು ರಾಷ್ಟ್ರದಾದ್ಯಂತ ಹರಡಿದುವು. ಅತಿ ಪ್ರಸಿದ್ಧವಾದ ಕೆಲವು ಸಂಸ್ಥೆಗಳು ಹಾಗೂ ಅನೇಕ ಕ್ರೈಸ್ತ ಸಂಸ್ಥೆಗಳು ಅವರನ್ನು ಆಮಂತ್ರಿಸಿ ಅವರಿಂದ ಉಪನ್ಯಾಸಗಳನ್ನು ಮಾಡಿಸಿದುವು. ಮತ್ತು ಅವರನ್ನು ಅತ್ಯಂತ ಗೌರವ ದಿಂದ ಆದರಿಸಿದುವು. ಸ್ವಾಮೀಜಿ ಇಂಗ್ಲಿಷ್ ಸಮಾಜದ ಅತ್ಯಂತ ಬುದ್ಧಿವಂತ ಹಾಗೂ ಶ್ರೇಷ್ಠ ಜನರ ವಲಯದಲ್ಲಿ ಓಡಾಡಿದರು. ಉನ್ನತ ವರ್ಗದ ಗೌರವಾನ್ವಿತ ವ್ಯಕ್ತಿಗಳು ಅವರನ್ನು ತಮ್ಮ ಸ್ನೇಹಿತರನ್ನಾಗಿ ಸ್ವೀಕರಿಸಿದರು. ಹೀಗೆ ಇಂಗ್ಲಿಷ್ ಜನರು ತಮ್ಮನ್ನೂ ತಮ್ಮ ಸಂದೇಶಗಳನ್ನೂ ಹೃತ್ಪೂರ್ವಕವಾಗಿ ಸ್ವೀಕರಿಸಿದಾಗ ಅಲ್ಲಿಯವರೆಗೆ ಇಂಗ್ಲಿಷ್ ಜನಾಂಗದ ಮೇಲೆ ತಮಗಿದ್ದ ಅಭಿಪ್ರಾಯವನ್ನು ಅವರು ಬದಲಾಯಿಸಿಕೊಳ್ಳಬೇಕಾಯಿತು. ಅಮೆರಿಕದ ಜನರೇನೋ ತಮ್ಮ ನೂತನ ಭಾವನೆಗಳನ್ನೂ ಸಂದೇಶಗಳನ್ನೂ ಉತ್ಸಾಹದಿಂದ ಸ್ವೀಕರಿಸಿದರಾದರೂ ಸಂಪ್ರದಾಯ ಶರಣರಾದ ಇಂಗ್ಲೆಂಡಿನ ಜನ ತಮ್ಮ ವಿನೂತನ ಬೋಧನೆಗಳನ್ನು ಒಪ್ಪಿಕೊಳ್ಳಲಾರರು ಎಂದು ಸ್ವಾಮೀಜಿ ಶಂಕಿಸಿದ್ದರು. ಆದರೆ ಇಂಗ್ಲೆಂಡಿಗೆ ಬಂದ ಮೇಲೆ ಅವರಿಗೆ ತಿಳಿಯಿತು– ಸಂಪ್ರ ದಾಯಶರಣತೆಯ ಮನೋಭಾವದಿಂದಾಗಿ ಈ ಇಂಗ್ಲಿಷರಿಗೆ ತಮ್ಮ ನೂತನ ಭಾವನೆಗಳನ್ನು ಅರ್ಥ ಮಾಡಿಕೊಳ್ಳಲು ಸ್ವಲ್ಪ ನಿಧಾನವಾಗಬಹುದು; ಆದರೆ ಅವರಿಗೆ ಆ ಅಭಿಪ್ರಾಯಗಳು ಒಮ್ಮೆ ಒಪ್ಪಿಗೆಯಾದುವೆಂದರೆ ಆ ಬೋಧನೆಗಳನ್ನೂ ಬೋಧಕನನ್ನೂ ಹಾರ್ದಿಕವಾಗಿ ಮೆಚ್ಚಿ ಸ್ವೀಕರಿಸುವಲ್ಲಿ ಅಮೆರಿಕನ್ನರನ್ನೂ ಮೀರಿಸಬಲ್ಲವರಾಗಿದ್ದಾರೆ ಎಂದು. ಆದ್ದರಿಂದಲೇ ಅವರು ಲಂಡನ್ನಿನಿಂದ ಅಮೆರಿಕೆಗೆ ಹಿಂದಿರುಗುವ ವೇಳೆಗೆ ಅಲ್ಲಿನ ಅನೇಕ ಸ್ತ್ರೀಪುರುಷರು ತಮ್ಮ ನಿಷ್ಠಾವಂತ ಬೆಂಬಲಿಗರಾಗಿ, ಪ್ರಾಮಾಣಿಕ ವಿಶ್ವಾಸಿಗರಾಗಿ ದೊರಕಿದ್ದನ್ನು ಕಂಡು ತೃಪ್ತರಾದರು. ಈ ಕುರಿತಾಗಿ ಅವರು ಮದ್ರಾಸಿನ ಶಿಷ್ಯರೊಬ್ಬರಿಗೆ ಪತ್ರ ಬರೆದರು:

“ಇಂಗ್ಲೆಂಡಿನಲ್ಲಿ ನನ್ನ ಕಾರ್ಯ ತುಂಬ ಅದ್ಭುತವಾಗಿ ನಡೆದಿದೆ. ಜನರು ತಂಡ ತಂಡವಾಗಿ ನನ್ನ ಬಳಿಗೆ ಬರುತ್ತಿದ್ದಾರೆ. ನನ್ನ ಕೋಣೆಯಲ್ಲಿ ಅವರಿಗೆಲ್ಲ ಸ್ಥಳ ಸಾಲದು. ಆದ್ದರಿಂದ ಸ್ತ್ರೀಯರೂ ಸೇರಿದಂತೆ ಎಷ್ಟೋ ಜನ ನೆಲದ ಮೇಲೆ ಕುಳಿತುಕೊಳ್ಳುತ್ತಾರೆ. ನಾನು ಮುಂದಿನ ವಾರವೇ ಇಲ್ಲಿಂದ ಹೊರಡಬೇಕಾಗಿರುವುದರಿಂದ ಇವರಿಗೆ ಖೇದವಾಗಿದೆ. ನಾನು ಬೇಗ ಇಲ್ಲಿಂದ ಹೊರಟುಬಿಟ್ಟರೆ ನನ್ನ ಕೆಲಸ ಕುಂಠಿತಗೊಳ್ಳಬಹುದೆಂದು ಕೆಲವರು ಅಭಿಪ್ರಾಯಪಡುತ್ತಾರೆ. ಆದರೆ ನನಗೆ ಹಾಗೆನಿಸುವುದಿಲ್ಲ. ನಾನು ಮನುಷ್ಯರನ್ನಾಗಲಿ ವಸ್ತುಗಳನ್ನಾಗಲಿ ಅವಲಂಬಿಸಿ ಕೊಂಡಿಲ್ಲ. ನಾನು ಅವಲಂಬಿಸಿಕೊಂಡಿರುವುದು ಭಗವಂತನನ್ನು ಮಾತ್ರವೇ. ಮತ್ತು ಅವನೇ ನನ್ನ ಮೂಲಕ ತನ್ನ ಕಾರ್ಯವನ್ನು ಸಾಧಿಸಿಕೊಳ್ಳುತ್ತಾನೆ.

“ಇಲ್ಲಿನ ನಿರಂತರ ಕೆಲಸಕಾರ್ಯಗಳಿಂದಾಗಿ ನಾನು ನಿಜಕ್ಕೂ ಬಳಲಿದ್ದೇನೆ. ಇಲ್ಲಿ ಇನ್ನಾವ ಹಿಂದೂವಾದರೂ ನನ್ನಷ್ಟು ಕಷ್ಟಪಟ್ಟು ಕೆಲಸಮಾಡಿದರೆ ಅವನು ಸತ್ತೇ ಹೋಗುತ್ತಿದ್ದ... ನಾನು ದೀರ್ಘ ವಿಶ್ರಾಂತಿಗಾಗಿ ಭಾರತಕ್ಕೆ ಮರಳಬೇಕೆಂದಿದ್ದೇನೆ.”

ಸ್ವಾಮೀಜಿಯ ತರಗತಿಗಳಲ್ಲಿ ಭಾಗವಹಿಸಿದ ಪತ್ರಕರ್ತನೊಬ್ಬ ತನ್ನ ವರದಿಯಲ್ಲಿ ಹೀಗೆ ಬರೆದ:

“ಲಂಡನ್ನಿನ ಕೆಲವು ಅತ್ಯಂತ ಸೊಗಸುಗಾರ ಸ್ತ್ರೀಯರೂ, ಭಾರತದ ಶಿಷ್ಯರಂತೆ ವಿನೀತರಾಗಿ ಕುಳಿತು ತಮ್ಮ ಗುರುವಿನ ಮಾತನ್ನು ಆಲಿಸುತ್ತಿದ್ದುದು ಒಂದು ಅಪೂರ್ವ ದೃಶ್ಯವೇ ಸರಿ. ಇಂಗ್ಲಿಷ್ ಜನಾಂಗದವರ ಮನಸ್ಸಿನಲ್ಲಿ ಭಾರತದ ಕುರಿತಾಗಿ ಸ್ವಾಮೀಜಿ ಮೂಡಿಸುತ್ತಿರುವ ಪ್ರೀತಿ-ಸಹಾನುಭೂತಿಗಳು, ಭಾರತದ ಪ್ರಗತಿಗೆ ಶಕ್ತಿಯ ಆಗರವಾಗುವುದೇ ಖಂಡಿತ.”

ಹೀಗೆ ಸ್ವಾಮೀಜಿ ಲಂಡನ್ನಿನಲ್ಲಿ ತಮ್ಮ ಚಟುವಟಿಕೆಗಳಲ್ಲಿ ಮುಳುಗಿದ್ದರೆ, ಅವರಿಗೆ ಅಮೆರಿಕೆಗೆ ಮರಳುವಂತೆ ಒತ್ತಾಯಿಸುವ ಪತ್ರಗಳು ಬರತೊಡಗಿದುವು. ಅಮೆರಿಕದಲ್ಲಿ ಅವರ ಕಾರ್ಯಕ್ಷೇತ್ರ ವಿಸ್ತಾರಗೊಳ್ಳುತ್ತಿದ್ದು, ತಮ್ಮ ಸಲುವಾಗಿಯಾದರೂ ಬೇಗೆ ಹಿಂದಿರುಗಬೇಕೆಂದು ಅವರ ಶಿಷ್ಯರು ಬೇಡಿಕೊಂಡರು. ಆದರೆ ಇತ್ತ ಅವರ ಇಂಗ್ಲಿಷ್ ಸ್ನೇಹಿತರು, ಲಂಡನ್ನಿನಲ್ಲೇ ಸ್ಥಿರವಾಗಿ ನೆಲಸ ಬೇಕೆಂದು ಅವರನ್ನು ಕೇಳಿಕೊಳ್ಳುತ್ತಿದ್ದರು.

ಈ ಪರಿಸ್ಥಿತಿಯಲ್ಲಿ ಸ್ವಾಮೀಜಿ ತಾವು ಅಮೆರಿಕೆಗೆ ಹಿಂದಿರುಗುವುದೆಂದೇ ನಿಶ್ಚಯಿಸಿದರು. ಏಕೆಂದರೆ ಅವರು ಇಂಗ್ಲೆಂಡಿಗೆ ಬಂದದ್ದೇ ಅಮೆರಿಕದ ತಮ್ಮ ಕಾರ್ಯಗಳಿಂದ ಕೆಲಕಾಲದ ಮಟ್ಟಿಗೆ ಬಿಡುವು ಮಾಡಿಕೊಂಡು, ಅಮೆರಿಕದ ಕಾರ್ಯ ಆಗತಾನೆ ಒಂದು ಹದಕ್ಕೆ ಬರುತ್ತಿದ್ದಾಗ ಅವರು ಅದನ್ನು ಬಿಟ್ಟು ಬಂದಿದ್ದರು. ಇಂಗ್ಲೆಂಡಿನಲ್ಲಿ ವೇದಾಂತ ಪ್ರಸಾರಕ್ಕೆ ಅನುಕೂಲವಾದ ವಾತಾವರಣವಿದೆಯೇ ಎಂಬುದನ್ನು ಪರಿಶೀಲಿಸುವುದೇ ಅವರ ಮುಖ್ಯ ಉದ್ದೇಶವಾಗಿತ್ತು. ಅಲ್ಲದೆ ಅವರು ಅಮೇರಿಕದಲ್ಲಿಲ್ಲದಾಗ ತರಗತಿಗಳು ನಡೆದುಕೊಂಡು ಹೋಗುತ್ತಿದ್ದುವಾದರೂ ಅವರಿಗೆ ಬದಲಾಗಿ ನಿಲ್ಲಬಲ್ಲ ಸಮರ್ಥರಾರೂ ಇರಲಿಲ್ಲ. ಆದ್ದರಿಂದ ಅದು ಯಾವಾಗಲಾದರೂ ಬಿದ್ದುಹೋಗಬಹುದಾಗಿತ್ತು. ಹೀಗೆ ತಮ್ಮ ಶ್ರಮವೆಲ್ಲ ವ್ಯರ್ಥವಾಗಬಾರದೆಂಬ ಉದ್ದೇಶದಿಂದ ಸ್ವಾಮೀಜಿ ಅಮೆರಿಕೆಗೆ ಹೊರಡಲು ನಿರ್ಧರಿಸಿದರು.

ಆದರೆ ಇತ್ತ ಇಂಗ್ಲೆಂಡಿನಲ್ಲೂ ಅವರ ಆವಶ್ಯಕತೆ ಕಡಿಮೆಯೇನೂ ಇರಲಿಲ್ಲ. ಕೆಲಕಾಲದಲ್ಲೇ ಬೇರು ಬಿಟ್ಟು ಚಿಗುರಲಾರಂಭಿಸಿದ್ದ ಅವರ ಕಾರ್ಯೋದ್ದೇಶವನ್ನು ಪೋಷಿಸಿಕೊಂಡುಬರಲು ಯೋಗ್ಯ ವ್ಯಕ್ತಿಗಳು ಇಲ್ಲಿಯೇ ಉಳಿದುಕೊಳ್ಳಬೇಕಾಗಿತ್ತು. ಇದು ಸ್ವಾಮೀಜಿಗೆ ಗೊತ್ತಿರಲಿಲ್ಲ ವೆಂದಲ್ಲ. ಆದರೆ ತಮ್ಮ ಬದಲಾಗಿ ತಮ್ಮ ಸೋದರ ಸಂನ್ಯಾಸಿಗಳಲ್ಲೊಬ್ಬರನ್ನು ಇಂಗ್ಲೆಂಡಿನ ಕಾರ್ಯಕ್ಕೆ ತೊಡಗಿಸುವುದು ಅವರ ಆಲೋಚನೆಯಾಗಿತ್ತು. ಅವರ ಕಾರ್ಯ ಯಶಸ್ವಿಯಾಗಲು ಬಹುಮಟ್ಟಿಗೆ ಕಾರಣನಾಗಿದ್ದ ಇ. ಟಿ. ಸ್ಟರ್ಡಿಯೂ ಈ ಯೋಜನೆಗೆ ಒಪ್ಪಿದ್ದ. ಅಲ್ಲದೆ ಅವನು ಕೈಗೊಂಡಿದ್ದ ನಾರದ ಭಕ್ತಿಸೂತ್ರವೇ ಮೊದಲಾದ ಗ್ರಂಥಗಳ ಅನುವಾದಕಾರ್ಯಕ್ಕೆ ಸಂಸ್ಕೃತ ವನ್ನು ಚೆನ್ನಾಗಿ ಬಲ್ಲ ವಿದ್ವಾಂಸರೊಬ್ಬರ ನೆರವು ಬೇಕಾಗಿತ್ತು. ಆದ್ದರಿಂದ ಸೆಪ್ಟೆಂಬರ್ ತಿಂಗಳಿ ನಲ್ಲೇ ಸ್ವಾಮೀಜಿ ಮಠಕ್ಕೆ ಪತ್ರ ಬರೆದು ಸಮರ್ಥರಾದ ಯಾರಾದರೊಬ್ಬರು ಇಂಗ್ಲೆಂಡಿಗೆ ಬರಲು ಸಿದ್ಧರಾಗುವಂತೆ ಹೇಳಿದ್ದರು. ಶ್ರೀರಾಮಕೃಷ್ಣರ ಅತ್ಯಂತ ನಿಷ್ಠಾವಂತ ಶಿಷ್ಯರೂ ಭಕ್ತಾಗ್ರಣಿಗಳೂ ವಿದ್ವಾಂಸರೂ ಆದ ಸ್ವಾಮಿ ರಾಮಕೃಷ್ಣಾನಂದರನ್ನು ಅವರು ಮೊದಲು ಆಹ್ವಾನಿಸಿದ್ದರು. ಆದರೆ ಆ ಸಮಯದಲ್ಲಿ ರಾಮಕೃಷ್ಣಾನಂದರಿಗೆ ಒಂದು ಬಗೆಯ ತೀವ್ರವಾದ ಚರ್ಮರೋಗವಿದ್ದುದ ರಿಂದ ಪರದೇಶದ ಪ್ರಯಾಣಕ್ಕೆ ವೈದ್ಯರು ಒಡಂಬಡಲಿಲ್ಲ. ಆಗ ಅಭೇದಾನಂದರು, ಶಾರದಾ ನಂದರು ಹಾಗೂ ತ್ರಿಗುಣಾತೀತಾನಂದರು–ಈ ಮೂವರಲ್ಲಿ ಯಾರದರೊಬ್ಬರೂ ಕೂಡಲೇ ಹೊರಟುಬರುವಂತೆ ಸ್ವಾಮೀಜಿ ಮತ್ತೆಮತ್ತೆ ಪತ್ರ ಬರೆದರು, ದಾರಿಯ ಖರ್ಚನ್ನೂ ಕಳಿಸಿದರು. ಆದರೆ ಆ ಮೂವರಲ್ಲಿ ಯಾರೊಬ್ಬರೂ ಬರಲು ಸಿದ್ಧರಿದ್ದಂತೆ ಕಾಣಲಿಲ್ಲ.

ಇದಕ್ಕೆ ಕಾರಣವೂ ಇರದಿರಲಿಲ್ಲ. ಸ್ವಾಮೀಜಿ ಈ ಹಿಂದೆ ಬರೆದ ಪತ್ರಗಳನ್ನು ಓದಿ ಆ ಸೋದರಸಂನ್ಯಾಸಿಗಳು ಸಾಕಷ್ಟು ಹೆದರಿದ್ದರು. ಸ್ವಾಮೀಜಿ ಬ್ರಹ್ಮಾನಂದರಿಗೆ ಬರೆದಿದ್ದರು, “ಇವು ಪ್ರಚಂಡ ವಿದ್ವಾಂಸರ ಹಾಗೂ ಬುದ್ಧಿಜೀವಿಗಳ ರಾಷ್ಟ್ರಗಳು. ಇಂಥ ಜನರನ್ನು ಶಿಷ್ಯರನ್ನಾಗಿ ಮಾಡಿಕೊಳ್ಳುವುದೆಂದರೆ ತಮಾಷೆಯೆಂದು ತಿಳಿದೆಯಾ?” ರಾಮಕೃಷ್ಣಾನಂದರಿಗೆ ಬರೆದ ಪತ್ರ ದಲ್ಲಿ ಹೇಳಿದ್ದರು–“ನಾವೇನಾದರೂ ಇಲ್ಲಿ ದೇವರು-ದಿಂಡರು ಎಂದು ಬೋಧಿಸಲು ಹೊರಟರೆ ಇಲ್ಲಿನ ಜನಸಾಮಾನ್ಯರು ನಮ್ಮನ್ನು ದೂರದಲ್ಲೇ ಇಟ್ಟಿರುತ್ತಾರೆ. ಇವನ್ಯಾರೋ ಇನ್ನೊಬ್ಬ ಪಾದ್ರಿ ಇರಬೇಕೆಂದೇ ಭಾವಿಸುತ್ತಾರೆ.” ಅಮೆರಿಕದಿಂದಲೇ ಬರೆದಿದ್ದ ಮತ್ತೊಂದು ಪತ್ರದಲ್ಲಿ ಹೀಗೆ ತಿಳಿಸಿದ್ದರು–“ವಿಷಯವೇನೆಂದರೆ, ನಾವಿಲ್ಲಿ ಉಳಿದುಕೊಳ್ಳಬೇಕಾದರೆ ನಮ್ಮ ಪ್ರತಿಯೊಂದು ತುತ್ತನ್ನೂ ಮಿಷನರಿ ವಿದ್ವಾಂಸರ ಕೈಯಿಂದ ಕಿತ್ತುಕೊಳ್ಳಬೇಕಾಗುತ್ತದೆ. ಎಂದರೆ ನಮ್ಮ ವಿದ್ವತ್ತಿನ ಶಕ್ತಿಯಿಂದ ಈ ಜನರನ್ನು ಗೆದ್ದುಕೊಳ್ಳಬೇಕು. ಇಲ್ಲವೆ ಅವರು ಒಂದು ಸಲ ಊದಿದರೆ ನಾವು ಹಾರಿಹೋಗುತ್ತೇವೆ. ಈ ಜನಗಳಿಗೆ ನಿಮ್ಮ ‘ಸಾಧುಗಳು-ಸಂನ್ಯಾಸಿಗಳೆ’ಲ್ಲ ಅರ್ಥವಾಗವುದಿಲ್ಲ. ನಿಮ್ಮ ವೈರಾಗ್ಯದ ಮಾತೆಲ್ಲ ಅವರ ತಲೆಗೆ ಹತ್ತುವುದಿಲ್ಲ. ಅವರಿಗೆ ಅರ್ಥವಾಗುವುದೇನಿದ್ದರೂ ವಿಶಾಲ ಜ್ಞಾನ, ಚುರುಕಾದ ವಾಕ್ಪಟುತ್ವ ಮತ್ತು ಪ್ರಚಂಡ ಕಾರ್ಯಶಕ್ತಿ. ಎಲ್ಲಕ್ಕಿಂತ ಹೆಚ್ಚಾಗಿ, ಇಡೀ ರಾಷ್ಟ್ರವೇ ನಿಮ್ಮಲ್ಲಿರುವ ಹುಳುಕನ್ನು ಹುಡುಕುತ್ತಿರುತ್ತದೆ; ಪಾದ್ರಿಗಳು ನಿಮ್ಮನ್ನು ಬಲ ದಿಂದಾಗಲಿ ಮೋಸದಿಂದಾಗಲಿ ಮೆಟ್ಟಿಮುರಿಯಲು ಹಗಲಿರುಳು ಪ್ರಯತ್ನಿಸುತ್ತಿರುತ್ತಾರೆ. ನಿಮ್ಮ ತತ್ತ್ವಗಳನ್ನು ಸಾರಬೇಕಾದರೆ ಆ ಅಡಚಣೆಗಳನ್ನೆಲ್ಲ ನೀವು ನಿವಾರಿಸಿಕೊಳ್ಳಬೇಕಾಗುತ್ತದೆ.”

ಇಂಥದನ್ನೆಲ್ಲ ಕೇಳಿದ ಮೇಲೆ ಅವರ ಸೋದರಸಂನ್ಯಾಸಿಗಳು ಹಿಂಜರಿದುದರಲ್ಲಿ ಆಶ್ಚರ್ಯ ವಿಲ್ಲ. ಆದರೆ ಸ್ವಾಮೀಜಿಗೆ ಮಾತ್ರ ಇದು ಸಹನೀಯವಾಗಲಿಲ್ಲ. ಅವರು ಹಾಗೆ ಹೆದರುವುದಕ್ಕೆ ಕಾರಣವೇನೂ ಕಾಣಲಿಲ್ಲ. ಅವರ ಬರವಿಗಾಗಿ ಕಾದು ಕಾದು ಬೇಸತ್ತ ಸ್ವಾಮೀಜಿ ಶಾರದಾ ನಂದರಿಗೆ ಕಡೆಗೊಂದು ಖಾರವಾದ ಪತ್ರ ಬರೆದರು:

“ನಿನ್ನ ಪತ್ರವನ್ನು ಓದಿ ನನಗೆ ದುಃಖವಾಯಿತು. ಅಷ್ಟೇ. ನೀವು ನಿಮ್ಮೆಲ್ಲ ಉತ್ಸಾಹವನ್ನೂ ಕಳೆದುಕೊಂಡಂತೆ ಕಾಣುತ್ತೀರಿ. ನಾನು ನಿಮ್ಮೆಲ್ಲರನ್ನೂ ಚೆನ್ನಾಗಿ ಬಲ್ಲೆ–ನಿಮ್ಮ ಶಕ್ತಿಯನ್ನು ಹಾಗೂ ಇತಿಮಿತಿಗಳನ್ನು ನಾನು ಬಲ್ಲೆ. ನಿಮ್ಮ ಕೈಯಲ್ಲಿ ಸಾಧ್ಯವಿಲ್ಲದಿರುವಂತಹ ಯಾವುದೇ ಕೆಲಸವನ್ನೂ ಮಾಡಿಸಲು ನಾನು ನಿಮ್ಮನ್ನು ಕರೆಯುತ್ತಿರಲಿಲ್ಲ... ನಾನು ನಿಮ್ಮನ್ನು, ಇಲ್ಲಿ ನೀವು ಕೆಲಸ ಮಾಡಲು ಬೇಕಾಗುವಂತೆ ತಯಾರು ಮಾಡುತ್ತಿದ್ದೆ. ಹಾಗೆ ನೋಡಿದರೆ ಯಾರು ಬೇಕಾ ದರೂ ಬರಬಹುದಾಗಿತ್ತು–ಸ್ವಲ್ಪ ಮಟ್ಟಿನ ಸಂಸ್ಕೃತ ಜ್ಞಾನ ಮಾತ್ರ ತೀರ ಅವಶ್ಯವಾಗಿತ್ತು. ಇರಲಿ, ಎಲ್ಲವೂ ಒಳ್ಳೆಯದಕ್ಕೇ. ಅದು ಭಗವಂತನ ಕೆಲಸವೇ ಆಗಿದ್ದಲ್ಲಿ ಸರಿಯಾದ ವ್ಯಕ್ತಿ ಸರಿಯಾದ ಸಮಯಕ್ಕೆ ಬಂದೇ ಬರುತ್ತಾನೆ. ನೀವು ಯಾರೂ ತೊಂದರೆ ತೆಗೆದುಕೊಳ್ಳಬೇಕಾಗಿಲ್ಲ.

“ಒಟ್ಟಿನಲ್ಲಿ ಸಾರಾಂಶವೇನೆಂದರೆ ಧೀರರೂ ಸಾಹಸಿಗಳೂ ಮುನ್ನುಗ್ಗಬಲ್ಲವರೂ ಆದ ವ್ಯಕ್ತಿಗಳು ನನ್ನ ಸಹಾಯಕ್ಕೆ ಬೇಕು. ಇಲ್ಲದಿದ್ದರೆ ನನ್ನ ಪಾಡಿಗೆ ನಾನೊಬ್ಬನೇ ಎಲ್ಲವನ್ನೂ ಮಾಡಿಕೊಳ್ಳುತ್ತೇನೆ. ನಾನು ಸಾಧಿಸಬೇಕಾದ ಒಂದು ಕಾರ್ಯೋದ್ದೇಶವಿದೆ. ಅದನ್ನು ನಾನು ಏಕಾಂಗಿಯಾಗಿಯೇ ಸಾಧಿಸುತ್ತೇನೆ. ಯಾರು ಬರುವರೋ ಯಾರೂ ಬಿಡುವರೋ ನಾನು ಲೆಕ್ಕಿಸುವುದಿಲ್ಲ... 

“ನಿಮಗೆ ನಮಸ್ಕಾರ! ಇನ್ನು ನಾನು ನಿಮಗೆ ತೊಂದರೆ ಕೊಡುವುದಿಲ್ಲ. ನಿಮಗೆಲ್ಲ ಆಶೀರ್ವಾದಗಳು. ನಿಮಗೆ ನನ್ನಿಂದ ಸ್ವಲ್ಪವಾದೂ ಸಹಾಯವಾಗಿದ್ದರೆ ನನಗೆ ಸಂತೋಷ– ಅದೂ, ನೀವು ಹಾಗೆ ಭಾವಿಸುವುದಾದರೆ... ಹೋಗಲಿ, ನನ್ನ ಗುರುದೇವನು ನನ್ನ ಮೇಲೆ ಹೊರಿಸಿದ ಜವಾಬ್ದಾರಿಯನ್ನು ಕಾರ್ಯಗತಗೊಳಿಸಲು ನಾನು ನನ್ನಿಂದಾದಷ್ಟು ಪ್ರಯತ್ನಿಸುತ್ತಿದ್ದೇ ನಲ್ಲ, ಅಷ್ಟಕ್ಕೇ ಸಂತೋಷಪಟ್ಟುಕೊಳ್ಳುತ್ತೇನೆ. ನಾನು ಮಾಡಿದುದನ್ನು ಚೆನ್ನಾಗಿ ಮಾಡಿದೆನೊ ಇಲ್ಲವೊ, ಆದರೆ ನಾನು ಪ್ರಯತ್ನವನ್ನಂತೂ ಮಾಡಿದೆನಲ್ಲ. ಅದಕ್ಕೇ ಸಂತೋಷಪಟ್ಟುಕೊಳ್ಳು ತ್ತೇನೆ. ಸರಿ, ನಮಸ್ಕಾರ, ನಿಮ್ಮೆಲ್ಲರಿಗೂ ನಮಸ್ಕಾರ. ನನ್ನ ಜೀವನದಲ್ಲಿ ಒಂದು ಅಧ್ಯಾಯ ಮುಗಿಯಿತು... ನಾನಿನ್ನು ಬರುತ್ತೇನೆ. ಎಂದೆಂದಿಗೂ ನಿಮ್ಮ ಮೇಲೆ ಭಗವಂತನ ಕೃಪೆಯಿರಲಿ.”

ಈ ಪತ್ರ ಬರೆದ ಕೆಲದಿನಗಳಲ್ಲೇ ಶಾರದಾನಂದರು ಇಂಗ್ಲೆಂಡಿಗೆ ಹೊರಟರು. ಆದರೆ ಅವರು ಇಂಗ್ಲೆಂಡಿಗೆ ಬಂದದ್ದು ಸ್ವಾಮೀಜಿ ಅಲ್ಲಿಂದ ಹೊರಟು ಅಮೆರಿಕವನ್ನು ಸೇರಿದ ಮೇಲೆಯೇ. ನವೆಂಬರ್ ೧೮ರವರೆಗೂ ಸ್ವಾಮೀಜಿ ಶಾರದಾನಂದರಿಗಾಗಿ ಕಾದಿದ್ದರು. ಆದರೆ ಅವರು ಬಾರದಿದ್ದಾಗ ಲಂಡನ್ನಿನ ತಮ್ಮ ನಿಷ್ಠಾವಂತ ಅನುಯಾಯಿಗಳನ್ನು ಸೇರಿಸಿ, ಕೆಲಸಕಾರ್ಯಗಳನ್ನು ಹೇಗೆ ಮುಂದುವರಿಸಿಕೊಂಡು ಹೋಗಬೇಕು ಎಂಬುದರ ಬಗ್ಗೆ ಸೂಚನೆಗಳನ್ನಿತ್ತರು. ಅವರೆಲ್ಲ ನಿಯತವಾಗಿ ಒಂದೆಡೆ ಸೇರಿ ಭಗವದ್ಗೀತೆಯನ್ನೂ ಹಾಗೂ ಇನ್ನಿತರ ಹಿಂದೂ ಶಾಸ್ತ್ರಗ್ರಂಥ ಗಳನ್ನೂ ಓದಿ ಅಧ್ಯಯನ ಗೋಷ್ಠಿ ನಡೆಸುತ್ತಿರುವಂತೆ ಹೇಳಿದರು. ಅಲ್ಲದೆ ಮುಂದಿನ ಬೇಸಿಗೆ ಯಲ್ಲಿ ತಾವೇ ಖುದ್ದಾಗಿ ಬಂದು ಆ ಕಾರ್ಯವನ್ನು ಮುಂದುವರಿಸುವುದಾಗಿ ಭರವಸೆಯಿತ್ತು ನವೆಂಬರ್ ೨೭ರಂದು ಅಮೆರಿಕೆಗೆ ಹೊರಟರು.

ಇಂಗ್ಲೆಂಡಿನಲ್ಲಿ ತಮ್ಮ ಮೂಲಕ ಭಗವಂತ ಇಷ್ಟನ್ನು ಮಾಡಿಸಿ ಯಶಸ್ವಿಯಾಗಿಸಿದುದನ್ನು ಕಂಡು ಸ್ವಾಮೀಜಿ ತುಂಬ ಸಂತೋಷಗೊಂಡಿದ್ದರು. ತಮ್ಮ ಅಮೆರಿಕದ ಶಿಷ್ಯರಿಗೆ ನೆರವಾಗಲು ಹೊಸ ಉತ್ಸಾಹದಿಂದ ಸನ್ನದ್ಧರಾದರು, ಆಗಿನ್ನೂ ಶಿಶಿರ ಪುತುವಿನ ಪ್ರಾರಂಭ. ಭಾಸ್ಟನ್ನಿನ ಸ್ಥಿತಿವಂತ ಮಹಿಳೆಯೊಬ್ಬಳು ಆ ಚಳಿಗಾಲವೆಲ್ಲ ಅವರ ಕಾರ್ಯದ ಖರ್ಚುಗಳನ್ನು ತಾನು ವಹಿಸಿಕೊಳ್ಳುವುದಾಗಿ ಆಶ್ವಾಸನೆ ನೀಡಿದ್ದಳು. ಒಟ್ಟಿನಲ್ಲಿ ತಮ್ಮ ಸಂದೇಶ ಪ್ರಸರಣ ಕಾರ್ಯ ಹಾಗೂ ಶಿಷ್ಯ ನಿರ್ಮಾಣ ಕಾರ್ಯಗಳು ನಿರಾತಂಕವಾಗಿ ನಡೆಯುತ್ತವೆ ಎಂಬ ಭರವಸೆ ಅವರಿಗುಂಟಾಗುತ್ತಿತ್ತು.

ಸ್ವಾಮೀಜಿಯ ಇಂಗ್ಲೆಂಡ್ ಭೇಟಿಯ ಕುರಿತಾಗಿ ಇ. ಟಿ. ಸ್ಟರ್ಡಿ ಮದ್ರಾಸಿನ ‘ಬ್ರಹ್ಮವಾದಿನ್​’ ಪತ್ರಿಕೆಗೆ ಕಳಿಸಿದ ಲೇಖನವೊಂದು ೧೮೯೬ರ ಫೆಬ್ರವರಿ ಸಂಚಿಕೆಯಲ್ಲಿ ಪ್ರಕಟವಾಯಿತು. ಅದರ ಒಂದು ಭಾಗ ಹೀಗಿತ್ತು:

“ಇಂಗ್ಲೆಂಡಿಗೆ ಸ್ವಾಮಿ ವಿವೇಕಾನಂದರು ನೀಡಿದ ಭೇಟಿಯಿಂದ ಒಂದು ಅಂಶ ಖಚಿತವಾಗಿದೆ. ಏನೆಂದರೆ, ಇಲ್ಲಿ ವಿಚಾರಶೀಲ ಹಾಗೂ ವಿದ್ಯಾವಂತ ಜನಗಳ ಗುಂಪೊಂದಿದೆ. ಮತ್ತು ನವಜೀವನವನ್ನೀಯುವ ಭಾರತೀಯ ವಿಚಾರಧಾರೆಯಿಂದ ಹೆಚ್ಚಿನ ಪ್ರಯೋಜನವನ್ನು ಪಡೆ ಯಲು ಅದು ಸಮರ್ಥವಾಗಿದೆ; ಆದರೆ ಅವರು ಹಾಗೆ ಪ್ರಯೋಜನವನ್ನು ಪಡೆಯುವಂತಾಗಲು ಅವರನ್ನು ಗುರುತಿಸಿ ಸರಿಯಾದ ರೀತಿಯಲ್ಲಿ ಸಮೀಪಿಸಬೇಕು ಅಷ್ಟೆ, ಎಂದು.

“ಅಲ್ಲದೆ ಇಲ್ಲಿನ ಕೆಲವು ತೆರೆಮನಸ್ಸಿನ ಪಾದ್ರಿಗಳೂ ಸ್ವಾಮಿ ವಿವೇಕಾನಂದರ ಮೂಲಕ ತಮ್ಮ ಸ್ವಂತ ಧರ್ಮಕ್ಕೆ ಶುದ್ಧ ವೇದಾಂತ ಸತ್ಯಗಳನ್ನು ಅನ್ವಯಿಸಿಕೊಳ್ಳಲು ಸಮರ್ಥರಾಗಿದ್ದಾರೆ; ಇದು ಆ ಪಾದ್ರಿಗಳು ಚರ್ಚಿನ ವೇದಿಕೆಗಳ ಮೇಲೆ ವಿವೇಕಾನಂದರ ಬೋಧನೆಗಳನ್ನು ಉಲ್ಲೇಖಿಸು ವುದನ್ನು ನೋಡಿದಾಗ ಸ್ಪಷ್ಟವಾಗುತ್ತದೆ. ಸ್ವಾಮಿ ವಿವೇಕಾನಂದರ ತರಗತಿಗಳು ಇಂಗ್ಲಿಷ್ ಸಮಾಜದ ವಿವಿಧ ವರ್ಗಗಳಿಂದ ಗಮನಾರ್ಹ ಸಂಖ್ಯೆಯ ಜನರನ್ನು ಆಕರ್ಷಿಸಿದುವು. ಇವರಲ್ಲಿ ಬಹಳಷ್ಟು ಜನರಿಗೆ ವಿವೇಕಾನಂದರೊಬ್ಬ ಸಮರ್ಥ ಗುರು ಎಂಬುದು ಸ್ಪಷ್ಟವಾಗಿದೆ.”

