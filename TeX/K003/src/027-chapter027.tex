
\chapter{ಅನುಬಂಧಗಳು}

\section{ಅನುಬಂಧ ೧: ಸ್ವಾಮೀಜಿಯ ಮನಕರಗಿಸಿದ ಆ ಹಾಡಿನ ಭಾವಾರ್ಥ}

\begin{center}
(ಪುಟ ೪೧)
\end{center}

\begin{verse}
ನೀನೆನ್ನ ಅವಗುಣವನೆಣಿಸದಿರು ಪ್ರಭುವೆ ॥
\end{verse}

\begin{verse}
ಸಮದರ್ಶಿಯೆಂದಿಹುದು ನಿನ್ನ ಬಿರುದು–\\ನಿನ್ನಿಚ್ಛೆಯೊಲು ಎನ್ನ ಪಾರುಗೈ ಪ್ರಭುವೆ
\end{verse}

\begin{verse}
ಲೋಹವೊಂದಿರುತಿರಲು ದೇವಸನ್ನಿಧಿಯಲ್ಲಿ,\\ಮತ್ತೊಂದು ಲೋಹವಿರೆ ವ್ಯಾಧಗೃಹದಿ;\\ಪರುಷಮಣಿ ಮನದಲ್ಲಿ ಭೇದ ಭಾವನೆಯೆಲ್ಲಿ?–\\ಚಿನ್ನವನ್ನಾಗಿಪುದು ಚಣಮಾತ್ರದಿ!
\end{verse}

\begin{verse}
ಮಲಿನರೂಪದಿ ನೆಲದಿ ಹರಿವ ಜಲವೊಂದಿರಲು,\\ಕಲುಷವನು ಕಳೆವ ನದಿಯೊಂದು ತಾನಿರಲು–\\ಹರಿಹರಿದು ಸುರನದಿಯ ಸೇರುತಿರೆ ತಾವೆರಡು\\ಪರಮಮಂಗಳತೀರ್ಥವಹುದಲ್ಲವೆ?
\end{verse}

\begin{verse}
ಜೀವ-ಬ್ರಹ್ಮಗಳೊಳಗೆ ಭೇದವಿಹುದೆನುತಿರಲು\\ಸೂರದಾಸನು ಇದನು ಒಪ್ಪದಿಹನು;\\ಅಜ್ಞಾನದಿಂದಲೇ ಸಕಲ ಭೇದವು ಜಗದಿ,\\ನಿಜವನರಿತವನಲ್ಲಿ ಭೇದವಿಹುದೆ?
\end{verse}



\begin{center}
ಅನುಬಂಧ ೨: ಯಕ್ಷನ ನಿಧಿ
\end{center}

\begin{center}
(ಪುಟ ೭೨)
\end{center}

ಮಾಯೆಯ ಸ್ವರೂಪವನ್ನು ಮತ್ತು ಮನುಷ್ಯನ ಸ್ವಭಾವವನ್ನು ಬಣ್ಣಿಸಲು ಶ್ರೀರಾಮಕೃಷ್ಣರು ಈ ಸುಂದರಕಥೆಯ ದೃಷ್ಟಾಂತವನ್ನು ಕೊಡುತ್ತಿದ್ದರು:

ಒಂದೂರಲ್ಲಿ ಒಬ್ಬ ರಾಜ. ಅವನಿಗೊಬ್ಬ ಕ್ಷೌರಿಕ. ಒಂದು ದಿನ ಆ ಕ್ಷೌರಿಕ ಒಂದು ಮರದಡಿ ಯಲ್ಲಿ ಸಾಗುತ್ತಿದ್ದಾಗ “ನಿನಗೆ ಏಳು ಜಾಡಿ ಚಿನ್ನ ಬೇಕೆ?” ಎಂಬ ಧ್ವನಿಯೊಂದು ಅವನಿಗೆ ಕೇಳಿಸಿತು. ಅದು ಆ ಮರದ ಮೇಲೆ ವಾಸವಾಗಿದ್ದ ಯಕ್ಷನ ಧ್ವನಿ. ಕ್ಷೌರಿಕ ಸುತ್ತಲೂ ನೋಡಿದ; ಯಾರೂ ಕಾಣಲಿಲ್ಲ. ಆದರೇನಂತೆ? ಚಿನ್ನ ಎಂಬ ಶಬ್ದವೇ ಅವನ ಮನಸ್ಸಿನಲ್ಲಿ ದುರಾಶೆ ಹುಟ್ಟಿಸಿತು. “ಹೌದು! ಏಳು ಜಾಡಿ ಚಿನ್ನ ನನಗೆ ಬೇಕು! ಎಲ್ಲಿದೆ ಅದು?” ಎಂದು ಗಟ್ಟಿಯಾಗಿ ಕೂಗಿದ. ತಕ್ಷಣ ಪ್ರತ್ಯುತ್ತರ ಕೇಳಿಸಿತು, “ಮನೆಗೆ ಹೋಗಿ ನೋಡು. ಅಲ್ಲಿ ನಾನು ಅವುಗಳನ್ನು ನಿನಗೋಸ್ಕರ ಇಟ್ಟಿದ್ದೇನೆ.”

ಕ್ಷೌರಿಕ ಎದ್ದುಬಿದ್ದು ಮನೆಗೆ ಓಡಿದ. ಮನೆಯಲ್ಲಿ ನೋಡುತ್ತಾನೆ–ಹೌದು! ನಿಜಕ್ಕೂ ಅಲ್ಲಿ ಚಿನ್ನ ತುಂಬಿದ ಏಳು ಜಾಡಿಗಳಿವೆ! ಕ್ಷೌರಿಕ ಪರಮಾನಂದಗೊಂಡು ಕುಣಿದಾಡಿದ. ಆದರೆ ಆ ಏಳು ಜಾಡಿಗಳ ಪೈಕಿ ಒಂದರಲ್ಲಿ ಮಾತ್ರ ಚಿನ್ನ ಭರ್ತಿಯಿರದೆ ಅರ್ಧಮಾತ್ರವೇ ಇತ್ತು. ಇದರಿಂದ ಅವನಿಗೆ ತುಂಬ ನಿರಾಸೆಯಾಯಿತು. ‘ಏನಾದರೂ ಮಾಡಿ ಅದನ್ನು ತುಂಬಿಸಿಬಿಡಬೇಕಲ್ಲ!’ ಎಂದು ಆಲೋಚಿಸಿದ. ತನ್ನ ಮನೆಯಲ್ಲಿದ್ದ ಆಭರಣಗಳನ್ನೆಲ್ಲ ಕರಗಿಸಿ ನಾಣ್ಯಗಳನ್ನಾಗಿಸಿ ಅದರಲ್ಲಿ ತುಂಬಿದ. ಆದರೆ ಆ ವಿಚಿತ್ರ ಜಾಡಿ ತುಂಬಲೇ ಇಲ್ಲ. ಎಷ್ಟೋ ದಿನ ಅವನು ತಾನೂ ಉಪವಾಸವಿದ್ದು ಹೆಂಡತಿ ಮಕ್ಕಳಿಗೂ ಉಪವಾಸ ಹಾಕಿ, ಉಳಿಸಿದ ದುಡ್ಡಿನಿಂದ ಅದನ್ನು ತುಂಬಿಸಲು ನೋಡಿದ. ಉಹುಂ! ಏನು ಮಾಡಿದರೂ ಅದು ಅಷ್ಟೇ ಇತ್ತು. ಕಡೆಗೆ ರಾಜನ ಬಳಿ ಹೋಗಿ ತನ್ನ ಸಂಬಳವನ್ನು ಹೆಚ್ಚಿಸಬೇಕೆಂದು ಕೇಳಿಕೊಂಡ. ಅವನು ರಾಜನಿಗೆ ತುಂಬ ಬೇಕಾಗಿದ್ದವನು. ಆದ್ದರಿಂದ ಕೂಡಲೇ ರಾಜ ಅವನ ಸಂಬಳವನ್ನು ಇಮ್ಮಡಿ ಮಾಡಿದ. ಅದರಿಂದಲೂ ಈತ ಚಿನ್ನವನ್ನು ಕೊಂಡು ಜಾಡಿಗೆ ತುಂಬಲಾರಂಭಿಸಿದ. ಆದರೂ ಅದು ಭರ್ತಿ ಯಾಗಲಿಲ್ಲ! ಕಡೆಗೆ ಅವನು ಭಿಕ್ಷೆ ಬೇಡಲೂ ತೊಡಗಿದ. ಹೀಗೆ ಎಷ್ಟೋ ತಿಂಗಳೇ ಕಳೆದು ಹೋದರೂ ಆ ಜಾಡಿ ಮಾತ್ರ ತುಂಬುವ ಲಕ್ಷಣವೇ ಕಾಣಲಿಲ್ಲ. ಆದರೆ ಈ ನಾಪಿತನಿಗೆ ಮಾತ್ರ ಇದೇಕೆ ಹೀಗೆ ಎಂದು ಅರ್ಥವಾಗಲೇ ಇಲ್ಲ. ಬರಬರುತ್ತ ಅವನು ತೀರಾ ದುರ್ಗತಿಗಿಳಿದ. ಹಗಲು ಇರುಳೂ ಆ ಅರ್ಧ ತುಂಬಿದ ಜಾಡಿಯ ಬಗ್ಗೆಯೇ ಆಲೋಚಿಸುತ್ತ ಹುಚ್ಚನಂತಾದ. ಇದನ್ನು ಕಂಡು ಒಂದು ದಿನ ರಾಜ ಕೇಳಿದ, “ಏನೋ! ನಿನ್ನ ಸಂಬಳ ಈಗಿನದರಲ್ಲಿ ಅರ್ಧದಷ್ಟಿದ್ದಾಗಲೇ ಎಷ್ಟೋ ನೆಮ್ಮದಿಯಾಗಿದ್ದೆ, ಸುಖವಾಗಿದ್ದೆ. ನಿನಗೆ ಈಗೇನು ಬಂತು ಕೇಡು? ಆ ಏಳು ಜಾಡಿ ಚಿನ್ನವೇನಾದರೂ ನಿನಗೆ ಸಿಕ್ಕಿತೆ?” ಇದನ್ನು ಕೇಳಿ ನಾಪಿತ ಆಶ್ಚರ್ಯಗೊಂಡು, “ಮಹಾಪ್ರಭು, ನಿಮಗೆ ಹೇಗೆ ಗೊತ್ತಾಯಿತು ಅದರ ಸಮಾಚಾರ?” ಎಂದ. ಆಗ ರಾಜ ಹೇಳಿದ, “ಅಯ್ಯೋ! ಆ ಯಕ್ಷನ ಏಳು ಜಾಡಿ ಸಿಕ್ಕಿದವರ ಗತಿಯೇ ಇಷ್ಟು ಎಂಬುದು ಗೊತ್ತಿಲ್ಲವೆ? ಅವನು ನನಗೂ ಅದನ್ನು ಕೊಡಲು ಬಂದ. ಆಗ ನಾನು ‘ಅದನ್ನು ನಾನು ಖರ್ಚುಮಾಡಬಹುದೋ ಅಥವಾ ಸುಮ್ಮನೆ ನೋಡಿಕೊಳ್ಳಬೇಕೋ?’ ಎಂದು ಕೇಳಿದೆ. ಆಗ ಆ ಯಕ್ಷ ಮತ್ತೆ ಮಾತನ್ನೇ ಆಡಲಿಲ್ಲ! ನೋಡು, ಆ ಜಾಡಿಯನ್ನಿಟ್ಟುಕೊಂಡವರಿಗೆಲ್ಲ ತೃಪ್ತಿಪಡಿಸಲಾಗದ ದುರಾಸೆಯುಂಟಾಗುತ್ತದೆ. ಈ ಕೂಡಲೇ ಹೋಗಿ ಆ ಜಾಡಿಗಳನ್ನೆಲ್ಲ ವಾಪಸ್ಸು ಕೊಟ್ಟು ಬಾ.” ರಾಜನ ಮಾತು ಕೇಳಿ ಕ್ಷೌರಿಕನ ಕಣ್ಣು ತೆರೆಯಿತು. ಆ ಯಕ್ಷನಿದ್ದ ಮರದ ಬಳಿಗೆ ಹೋಗಿ, “ಅಯ್ಯಾ, ನಿನ್ನ ಜಾಡಿಗಳನ್ನೆಲ್ಲ ನೀನೇ ಇಟ್ಟುಕೊ” ಎಂದು ಕೂಗಿದ. “ಆಗಬಹುದು” ಎಂಬ ಉತ್ತರ ಕೇಳಿಸಿತು. ಮನೆಗೆ ಬಂದು ನೋಡಿದರೆ, ಆ ಜಾಡಿಗಳೆಲ್ಲ ಮಾಯವಾಗಿದ್ದುವು. ಅವುಗಳೊಂದಿಗೆ ಅವನು ಅಷ್ಟು ದಿನ ಕೂಡಿಟ್ಟಿದ್ದ ಹಣವೂ ಮಾಯವಾಗಿತ್ತು. ಜೊತೆಗೆ ನಾಪಿತನ ದುರಾಸೆಯೂ ತೊಲಗಿತು.


\begin{center}
ಅನುಬಂಧ ೩: ಮರೀಚಿಕೆ
\end{center}

\begin{center}
(ಪುಟ ೧೧೬)
\end{center}

ಮರೀಚಿಕೆ, ಮೃಗಜಲ, ಬಿಸಿಲ್ಗುದುರೆ ಎಂದೆಲ್ಲ ಕರೆಯಲ್ಪಡುವ ಈ ದೃಶ್ಯವು ಒಂದು ಅದ್ಭುತವಾದ ಸುಳ್ಳು-ಸತ್ಯ; ಅಥವಾ ಸತ್ಯವಾದ ಸುಳ್ಳು. ಮರುಭೂಮಿಗಳಲ್ಲಿ–ಅದರಲ್ಲೂ ವಿಶೇಷವಾಗಿ ಸುಡುಬಿಸಿಲಿನ ದಿನಗಳಲ್ಲಿ–ಕಂಡುಬರುವ ಪ್ರಕೃತಿ ವೈಚಿತ್ರ್ಯ ಇದು. ಸೂರ್ಯನ ಶಾಖದಿಂದಾಗಿ ಭೂಮಿಯ ಮೇಲ್ಮೈ ಕಾದು, ವಾತಾವರಣದ ಗಾಳಿಯ ಸಾಂದ್ರತೆಯಲ್ಲಿ ಬದಲಾವಣೆಗಳುಂಟಾಗುತ್ತವೆ. ಗಾಳಿಯ ವಿವಿಧ ಪದರಗಳು ವಿವಿಧ ಸಾಂದ್ರತೆಗಳಿಂದ ಕೂಡಿ ದ್ದಾಗ, ಅದರಲ್ಲಿ ಹಾದು ಹೋಗುವ ಕಿರಣಗಳು ವಕ್ರೀಕರಣಗೊಳ್ಳುತ್ತವೆ. ಉದಾಹರಣೆಗೆ, ಒಂದು ಅಂಜೂರದ ಮರವನ್ನು ಒಬ್ಬ ಮನುಷ್ಯ ದೂರದಿಂದ ನೋಡಿದರೆ, ಆತ ಅದನ್ನು ವಕ್ರೀಕರಣಗೊಂಡ ಬೆಳಕಿನ ಕಿರಣಗಳ ಸಹಾಯದಿಂದ ನೋಡುತ್ತಿರುತ್ತಾನೆ. ಆ ವಕ್ರಗೊಂಡ ಕಿರಣಗಳು, ಆ ವಸ್ತುವಿನ ತಲೆಕೆಳಗಾದ ಪ್ರತಿಬಿಂಬವೊಂದು ಇರುವಂತೆ ಭ್ರಮೆ ಹುಟ್ಟಿಸುತ್ತವೆ. ಆದ್ದರಿಂದ ಆ ವ್ಯಕ್ತಿ ಆ ಮರದ ಪಕ್ಕದ್ಲಲೇ ನೀರಿರಬೇಕೆಂದು ಊಹಿಸುತ್ತಾನೆ. ಸಾಲದ್ದಕ್ಕೆ, ಗಾಳಿ ಬೀಸಿದಾಗ ಆ ಪ್ರತಿಬಿಂಬವೂ ಅಲುಗಾಡಿ, ನೀರು ಅಲುಗಾಡುತ್ತಿದೆಯೆಂಬಂತೆ ತೋರುತ್ತದೆ. ಹೀಗೆ ಕಂಡ ನೀರನ್ನರಸುತ್ತ ಮನುಷ್ಯನೋ ಮೃಗವೋ ಹತ್ತಿರ ಹೋದಂತೆ, ಆ ದೃಶ್ಯವೂ ಮುಂದೆಮುಂದೆ ಜರುಗುತ್ತದೆ. ಇದೇ ‘ಮೃಗಜಲ.’

ಇದರ ವೈಜ್ಞಾನಿಕ ವಿವರಣೆಗಳೇನೇ ಇರಲಿ, ನೋಡುವವರ ಕಣ್ಣಿಗಂತೂ ಈ ದೃಶ್ಯ ಸತ್ಯ- ಸುಂದರ. ಬಿಸಿಲಿನ ಬೇಗೆಯಲ್ಲಿ ಬೆಂದು ಬಾಯಾರಿದ ಮರಳುಗಾಡಿನ ಮೃಗಗಳು ನೀರಿಗಾಗಿ ಅಲೆದಾಡುತ್ತವೆ. ಆಗ ಕೆಲವೊಮ್ಮೆ ದೂರದಲ್ಲೊಂದು ತಿಳಿನೀರ ಸರೋವರ ಕಾಣಿಸಿಕೊಳ್ಳುತ್ತದೆ. ಆ ಸುಳ್ಳುಸರೋವರವನ್ನೇ ಸತ್ಯವೆಂದು ಭಾವಿಸಿ ಆ ಪ್ರಾಣಿ ಸರೋವರದ ದಿಕ್ಕಿನಲ್ಲಿ ಓಡುತ್ತದೆ. ಆದರೆ ಆ ಸರೋವರದ ಹತ್ತಿರ ಹೋದ ಕೂಡಲೇ ಸರೋವರ ಮಾಯವಾಗಿಬಿಡುತ್ತದೆ. ಅರೆ! ಈಗತಾನೆ ಕಂಡ ಸರೋವರ ಎಲ್ಲಿ ಮಾಯವಾಯಿತು? ಎಂದು ಕಣ್ಣು ಹಾಯಿಸಿದರೆ ಅಲ್ಲೇ ಸ್ವಲ್ಪ ದೂರದಲ್ಲೇ ಕಾಣಿಸಿಕೊಳ್ಳುತ್ತದೆ. ಪ್ರಾಣಿ ಮತ್ತೆ ಓಡುತ್ತದೆ. ಸರೋವರ ಮತ್ತೆ ಮುಂದೆ ಹೋಗುತ್ತದೆ. ಹೀಗೆ ಈ ಮೃಗ ಓಡಿಓಡಿ ಬಾಯಾರಿ ಬಳಲಿ ಬಸವಳಿದು ಕಡೆಗೆ ಬಾಯಾರಿಕೆ ಯಿಂದಲೇ ಬಿದ್ದು ಪ್ರಾಣ ಬಿಡುತ್ತದೆ ಎಂದು ಪ್ರತೀತಿ. ಆದ್ದರಿಂದಲೇ ‘ಮೃಗಜಲ’ ಎಂಬ ಹೆಸರು. ‘ಮರೀಚಿ’ ಎಂಬ ಶಬ್ದಕ್ಕೆ ‘ಬೆಳಕು’, ‘ಕಿರಣ’ ಎಂಬ ಅರ್ಥಗಳುಂಟು. ಬೆಳಕಿನ ಕಣ್ಣುಮುಚ್ಚಾಲೆಯಾಟವೇ ಈ ‘ಮರೀಚಿಕೆ’ಗೆ ಕಾರಣವಾದುದರಿಂದ ಅದಕ್ಕೆ ಆ ಹೆಸರು ಬಂದಿರ ಬೇಕು. ಶ್ರೀರಾಮ ಮಾರೀಚನನ್ನು ಜಿಂಕೆಯೆಂದು ಭ್ರಮಿಸಿ ಅವನ ಬೆನ್ನಟ್ಟಲಿಲ್ಲವೆ? ಅವನು ಆ ಜಿಂಕೆಯನ್ನು ತನ್ನ ಬಾಣದಿಂದ ಹೊಡೆದಾಗ ಗೊತ್ತಾಯಿತು–ಅದು ಕಾಂಚನಮೃಗವಲ್ಲ, ಮಾರೀಚನೆಂಬ ರಾಕ್ಷಸ ಎಂದು. ಮಾರೀಚನಂತೆ ಮೋಸಗೊಳಿಸುವುದರಿಂದಲೂ ಈ ದೃಶ್ಯಕ್ಕೆ ‘ಮರೀಚಿಕೆ’ಯೆಂಬ ಹೆಸರು ಬಂದಿರಬಹುದು. ಸಾಮಾನ್ಯ ಜನರೆಲ್ಲ ಆ ಮರಳುಗಾಡಿನ ಮೃಗ ದಂತೆ ಇಂದ್ರಿಯಸುಖವೆಂಬ ‘ತಿಳಿನೀರ ಸರೋವರ’ದ ಕಡೆಗೆ ಒಂದೇ ಸಮನೆ ಓಡುತ್ತಾರೆ. ಆದರೆ ಸ್ವಾಮೀಜಿಯಂತಹ ಕೆಲವರು ವಿಚಾರ ಮಾಡಿ ಹಿಂದಕ್ಕೆ ಬರುತ್ತಾರೆ, ಮೃಗಜಲದ ಮೋಸದಿಂದ ಪಾರಾಗುತ್ತಾರೆ. ಶ್ರೀರಾಮನಂಥವರು ತಮ್ಮ ವಿವೇಕವೆಂಬ ಬಾಣಪ್ರಯೋಗ ಮಾಡಿ, ಆ ಮಾಯಾ ಜಿಂಕೆಯನ್ನು ಕೊಲ್ಲುತ್ತಾರೆ; ಮೋಸದ ಮಾರೀಚನನ್ನು ಕಾಣುತ್ತಾರೆ. ಇನ್ನುಳಿದವರೆಲ್ಲ ಮರಳುಗಾಡಿನ ಮರಳು ಮೃಗದಂತೆ, ಓಡಿಓಡಿ, ಕೊನೆಗೂ ಅತೃಪ್ತಿಯಿಂದಲೇ ಪ್ರಾಣ ಬಿಡುತ್ತಾರೆ.


\begin{center}
ಅನುಬಂಧ ೪: ‘ಬ್ರಾಹ್ಮಣ ಸಂನ್ಯಾಸಿ’
\end{center}

\begin{center}
(ಪುಟ ೨ಂಂ)
\end{center}

ಸ್ವಾಮಿ ವಿವೇಕಾನಂದರನ್ನು ಈ ಪತ್ರಿಕೆಯ ವರದಿಯಲ್ಲಿ ಒಬ್ಬ 'ಬ್ರಾಹ್ಮಣ ಸಂನ್ಯಾಸಿ'\eng{(Brahmin Monk)} ಎಂದು ಕರೆಯಲಾಗಿದೆ. ಇದನ್ನು ಕಂಡು ನಾವು ಸಂನ್ಯಾಸಿಗಳಲ್ಲಿ ಬ್ರಾಹ್ಮಣ ಸಂನ್ಯಾಸಿ, ಕ್ಷತ್ರಿಯ ಸಂನ್ಯಾಸಿ ಎಂಬ ಭೇದಗಳಿವೆಯೆ?–ಎಂದು ಆಶ್ಚರ್ಯಪಡಬೇಕಾಗಿಲ್ಲ. ಅಥವಾ ಹುಟ್ಟಿನಿಂದ ಕ್ಷತ್ರಿಯರಾದ ವಿವೇಕಾನಂದರು ಬ್ರಾಹ್ಮಣರಾದದ್ದು ಯಾವಾಗ? ಎಂದೂ ಚಿಂತಿಸಬೇಕಿಲ್ಲ. ಏಕೆಂದರೆ ಅವರ ಕುರಿತಾದ ಈ ಸಂಬೋಧನೆಯು ಕೇವಲ ಅಜ್ಞಾನದಿಂದ ಉಂಟಾದುದು. ಆಗಿನ ಕಾಲದಲ್ಲಿ ಪರರಾಷ್ಟ್ರಗಳಲ್ಲಿ–ಅದರಲ್ಲೂ ಅಮೆರಿಕದಲ್ಲಿ–ಭಾರತದ ಬಗ್ಗೆ ಇದ್ದ ತಿಳಿವಳಿಕೆ ತೀರ ಅಲ್ಪ. ಅದರಲ್ಲೂ ತಪ್ಪು ನಂಬಿಕೆಗಳೇ ಹೆಚ್ಚು. ‘ಬ್ರಾಹ್ಮಣ’ ಎಂದರೆ ಸಮಾಜದ ಪೂಜ್ಯತೆ-ಗೌರವಗಳಿಗೆ ಪಾತ್ರನಾದ ವ್ಯಕ್ತಿ ಎಂದು ಅಮೆರಿಕನ್ನರು ಕಲ್ಪಿಸಿಕೊಂಡಿದ್ದರು. ಚತುರ್ವರ್ಣಗಳ ಅರಿವು ಅವರಿಗಿರಲಿಲ್ಲ. ಆದ್ದರಿಂದ ಸ್ವಾಮೀಜಿಯೇನಾದರೂ ತಾವು ಬ್ರಾಹ್ಮಣ ರಲ್ಲ, ಕ್ಷತ್ರಿಯರು ಎಂದು ಹೇಳಿಕೊಂಡಿದ್ದರೂ ಅದು ಹೆಚ್ಚಿನವರಿಗೆ ಅರ್ಥವಾಗುತ್ತಿರಲಿಲ್ಲ. ಬದಲಾಗಿ ಹೆಚ್ಚಿನ ಅನರ್ಥಕ್ಕೆ ಕಾರಣವಾಗುತ್ತಿತ್ತು. ಸ್ವಾಮೀಜಿಯ ವಿರುದ್ಧ ಅಪಪ್ರಚಾರ ಮಾಡು ತ್ತಿದ್ದವರಿಗೆ ಮತ್ತೊಂದು ವಿಷಯ ಸಿಕ್ಕಿದಂತಾಗುತ್ತಿತ್ತು. ‘ನೋಡಿದಿರಾ! ವಿವೇಕಾನಂದರು ಯಾವುದೋ ಕೀಳು ಜಾತಿಗೆ ಸೇರಿದವರು! ಇಂಥವರು ಧರ್ಮಬೋಧನೆ ಮಾಡುತ್ತಿದ್ದಾರೆ!’ ಎಂದು ಗುಲ್ಲೆಬ್ಬಿಸುತ್ತಿದ್ದರು.

ವಿವೇಕಾನಂದರನ್ನು ಅಮೆರಿಕದ ಪತ್ರಿಕೆಗಳು ‘ಬ್ರಾಹ್ಮಣ ಸಂನ್ಯಾಸಿ’ ಎಂದಷ್ಟೇ ಅಲ್ಲದೆ ಇತರ ಹಲವಾರು ವಿಚಿತ್ರ ಹೆಸರುಗಳಿಂದ ಕರೆದುವು. ಅಮೆರಿಕದಲ್ಲಿ ಪ್ರತಿಯೊಬ್ಬರಿಗೂ ಒಂದು ಉಪನಾಮ \eng{(Surname)} ಇದ್ದೇ ಇರುತ್ತದೆ. ಆದ್ದರಿಂದ ‘ಕಾನಂದ’ ಎಂಬುದು ‘ವಿವೇ- ಕಾನಂದ’ರ ಉಪನಾಮವಿರಬೇಕೆಂದು ಬಹಳಷ್ಟು ಜನ ಭಾವಿಸಿದ್ದರು. ಎಲ್ಲೆಲ್ಲೂ ಅವರನ್ನು ಮಿ॥ ಕಾನಂದ ಎಂದೇ ಮುದ್ರಿಸಲಾಗುತ್ತಿತ್ತು! ‘ವಿವೇ ಕಾನಂದ’, ‘ವಿವಾ ಕಾನಂದ’, ‘ಸಿವಾನಿ ವಿವಾ ಕಾನಂದ’–ಹೀಗೆ ಅವರ ಹೆಸರನ್ನು ಬಗೆಬಗೆಯಾಗಿ ಮುದ್ರಿಸಿದರು. ಅಲ್ಲದೆ ‘ಅರ್ಚಕ ಕುಲ ದವರು’, ‘ಉತ್ತಮ ಜಾತಿಯವರು’ ಎಂದೆಲ್ಲ ಬಣ್ಣಿಸಿದರು. ಆದರೆ ಇವರ ಹತ್ತಿರ ತಲೆ ಚಚ್ಚಿಕೊಂಡು ಪ್ರಯೋಜನವಿಲ್ಲವೆಂದು ಸ್ವಾಮೀಜಿಗೆ ಅನುಭವವಾಗಿರಬೇಕು. ಆದ್ದರಿಂದ ಎಷ್ಟೋ ಕಾಲ ಇದು ಹೀಗೆಯೇ ನಡೆಯಿತು. ಇನ್ನೊಂದು ವಿಷಯವೇನೆಂದರೆ, ಅಲ್ಲಿನವರು ವಿವೇಕಾನಂದರನ್ನು ಒಬ್ಬ ಸಂನ್ಯಾಸಿಯೆಂಬ ದೃಷ್ಟಿಯಲ್ಲಿ ಕಾಣುತ್ತಿರಲಿಲ್ಲ. ಬದಲಾಗಿ ಅವರನ್ನು ಒಬ್ಬ ಗೌರವಸ್ಥ, ಸಂಭಾವಿತ ಗಣ್ಯ ಮನುಷ್ಯನಂತೆ ಕಂಡು ಅದರಂತೆ ವ್ಯವಹರಿಸುತ್ತಿದ್ದರು. ಏಕೆಂದರೆ ಆ ಜನಕ್ಕೆ ಪ್ರಥಮತಃ ಸಂನ್ಯಾಸ ಧರ್ಮದ ಪರಿಚಯವೇ ಇರಲಿಲ್ಲ. ಅವರಿಗೆ ತಿಳಿದಿದ್ದುದೇನೆಂದರೆ ತತ್ತ್ವಶಾಸ್ತ್ರಜ್ಞರು, ಪಾದ್ರಿಗಳು, ಧರ್ಮಬೋಧಕರು ಮತ್ತು ಭಾರತದ ಅರ್ಚಕರು! ಆದರೆ ಅಮೆರಿಕದ ಜನ ಕ್ರಮೇಣ ಸ್ವಾಮೀಜಿಯ ಹಾಗೂ ಇತರ ಸಂನ್ಯಾಸಿಗಳ ಸಂಪರ್ಕಕ್ಕೆ ಬಂದಂತೆ, ಅವರನ್ನು ಯಾವ ದೃಷ್ಟಿಯಿಂದ ನೋಡಬೇಕೆಂಬುದನ್ನು ಕಲಿತು ಕೊಂಡರು.


\begin{center}
ಅನುಬಂಧ ೫: ನಾಗರಹಾವು
\end{center}

\begin{center}
(ಪುಟ ೩೩೫)
\end{center}

ಗುರವಾದವನು ಎಂತಹ ಸಾಮರ್ಥ್ಯಶಾಲಿಯಾಗಿರಬೇಕು, ಮತ್ತು ಅವನಿಗೆ ತನ್ನ ಶಿಷ್ಯರ ಮೇಲೆ ಎಂತಹ ಹಿಡಿತವಿರಬೇಕು ಎಂಬುದಕ್ಕೆ ಶ್ರೀರಾಮಕೃಷ್ಣರು ಹಾಸ್ಯವಾಗಿ ನಾಗರಹಾವಿನ ಉಪಮೆ ಯನ್ನು ಕೊಡುತ್ತಾರೆ.

ಅವರು ಹೇಳುತ್ತಿದ್ದರು: ಒಂದು ಕೇರೆಹಾವು ಕಪ್ಪೆಯನ್ನೇನೋ ಹಿಡಿದುಬಿಡುತ್ತದೆ. ಆದರೆ ಕಪ್ಪೆ ಮಾತ್ರ ಅದರ ಗಂಟಲೊಳಗೆ ಸುಲಭವಾಗಿ ತೂರುವುದೂ ಇಲ್ಲ, ಬೇಗೆ ಸಾಯುವುದೂ ಇಲ್ಲ. ಈಗ ಹಾವಿಗೂ ಸಂಕಟ, ಕಪ್ಪೆಗೂ ಸಂಕಟ, ಕಪ್ಪೆ ಒಂದೇ ಸಮನೆ ಬೊಬ್ಬೆ ಹೊಡೆಯು ತ್ತಿರುತ್ತದೆ. ಹಾವು ಕೊಸರಾಡುತ್ತಿರುತ್ತದೆ. ಆದರೆ ಒಂದು ನಾಗರಹಾವೇನಾದರೂ ಕಪ್ಪೆಯನ್ನು ಹಿಡಿದರೆ, ಕಪ್ಪೆ ಮೂರು ಸಲ ಕೂಗುವುದರೊಳಗಾಗಿ ಅದನ್ನು ಮುಗಿಸಿಬಿಟ್ಟಿರುತ್ತದೆ. ಸಮರ್ಥ ಗುರುಗಳು ನಾಗರಹಾವಿನಂತೆ. ಶಿಷ್ಯ ‘ಆ ಊ’ ಎನ್ನುವಷ್ಟರಲ್ಲಿ ಅವನ ಸಂದೇಹಗಳನ್ನೆಲ್ಲ ಪರಿಹಾರ ಮಾಡಿ, ಅವನಿಗೆ ಜ್ಞಾನೋದಯ ಮಾಡಿಸುತ್ತಾರೆ. ಉಳಿದ ಸಾಮಾನ್ಯ ಗುರುಗಳು ಕೇರೆ ಹಾವಿನಂತೆ; ಶಿಷ್ಯನ ಅಜ್ಞಾನವನ್ನೂ ಹೋಗಲಾಡಿಸಲಾಗದೆ ಅವನನ್ನು ಬಿಟ್ಟುಕೊಡಲೂ ಆಗದೆ ಇಬ್ಬರೂ ಒದ್ದಾಡುತ್ತಿರುತ್ತಾರೆ.

ತಾವು ಈ ನಾಗರಹಾವಿನಂಥವರು; ತಮ್ಮ ಕೈಗೆ ಸಿಕ್ಕಿಕೊಂಡವರು ಇನ್ನು ಆ ಹಿಡಿತದಿಂದ ತಪ್ಪಿಸಿಕೊಳ್ಳಲಾರರು ಎಂಬುದು ಇಲ್ಲಿ ಸ್ವಾಮೀಜಿಯ ಅಭಿಪ್ರಾಯ.


\section{ಚಿತ್ರಗಳ ವಿವರಣೆ}

೧. ಪುಟ ೪೪೩\\ಲಂಡನ್ನಿನ ‘ಡೈಲಿ ಗ್ರಾಫಿಕ್​’ ಎಂಬ ಪತ್ರಿಕೆಯಲ್ಲಿ ೧೮೯೫ರ ಅಕ್ಟೋಬರ್ ೨೨ರಂದು ಪ್ರಕಟವಾದ ಒಂದು ವರದಿ.

೨. ಪುಟ ೪೪೪\\೬-೧-೧೮೯೬ರಂದು ‘ನ್ಯೂಯಾರ್ಕ್ ವರ್ಲ್ಡ್ ಟೆಲಿಗ್ರಾಮ್​’ ಎಂಬ ಪತ್ರಿಕೆ ವಿವೇಕಾನಂದರ ಬಗ್ಗೆ ಒಂದು ಲೇಖನವನ್ನು ಪ್ರಕಟಿಸಿತು. ಅದರೊಂದಿಗೆ ಮುದ್ರಿತವಾಗಿದ್ದ ಅವರ ರೇಖಾಚಿತ್ರ ಇದು. ಲೇಖನದಲ್ಲಿ ಅವರನ್ನು “ಎತ್ತರದ ನಿಲುವಿನ, ಸುಂದರ ಮುಖ ಹಾಗೂ ಮೈಕಟ್ಟಿನ ವ್ಯಕ್ತಿ” ಎಂದು ಬಣ್ಣಿಸಿದ್ದರೂ ಪತ್ರಿಕೆಯ ಕಲಾವಿದನಿಗೆ ಸಾಧ್ಯವಾದದ್ದು ಇಷ್ಟೆ!

\chapter{ಗ್ರಂಥ ಋಣ}

\textbf{ಇಂಗ್ಲಿಷ್​}

\selectlanguage{english}

\begin{enumerate}
\item \textit{The Life of Swami Vivekananda (vol. 1 \& 2) by His Eastern \& Westren Disciples}
\begin{flushright}
Advaita Ashrama
\end{flushright}

 \item \textit{A Comprehensive Biography of Swami Vivekananda }
\begin{flushright}
Sailendra Nath Dhar
\end{flushright}

 \item \textit{Swami Vivekananda in the West–New Discoveries\\His Prophetic Mission (Parts 1 \& 2) (1983 \& 84)\\The World Teacher (Part 1) (1985) }
\begin{flushright}
Marie Louise Burke
\end{flushright}

 \item \textit{Swami Vivekananda–A Forgotten Chapter of His life}
\begin{flushright}
Beni Shankar Sharma
\end{flushright}

 \item \textit{Reminiscences of Swami Vivekananda }
\begin{flushright}
Advaita Ashrama
\end{flushright}

 \item \textit{Letters of Swami Vivekananda }
\begin{flushright}
Advait aAshrama
\end{flushright}

 \item \textit{Vivekananda in Indian Newspapers–1893-1902 }
\begin{flushright}
Shankariprasad Basu and Sunil Bihari Ghosh
\end{flushright}

 \item \textit{Service of God in Man }
\begin{flushright}
Swami Akhandananda
\end{flushright}

 \item \textit{Vivekananda–ABiography in Pictures }
\begin{flushright}
Advaita Ashrama
\end{flushright}

 \item \textit{Swami Abhedananda–A Spiritual Biography }
\begin{flushright}
Moni Bagchi
\end{flushright}

\end{enumerate}

\selectlanguage{kannada}

\textbf{ಕನ್ನಡ}

\textit{೧. ಸ್ವಾಮಿ ವಿವೇಕಾನಂದರ ಕೃತಿಶ್ರೇಣಿ }
\begin{flushright}
ಶ್ರೀರಾಮಕೃಷ್ಣಾಶ್ರಮ, ಮೈಸೂರು
\end{flushright}

\chapter{ಇದೇ ಲೇಖಕರ ಇತರ ಕೆಲವು ಕೃತಿಗಳು}

\begin{enumerate}
\item ಯುಗಾವತಾರ ಶ್ರೀರಾಮಕೃಷ್ಣ\\ಶ್ರೀರಾಮಕೃಷ್ಣ ಪರಮಹಂಸರ ವಿಸ್ತೃತ ಜೀವನಚರಿತ್ರೆ\\–ಸುಲಭ ಬೆಲೆಯ ನಾಲ್ಕು ಸಂಪುಟಗಳಲ್ಲಿ

 \item ಶ್ರೀಶಾರದಾದೇವಿ ಜೀವನಗಂಗಾ\\ಶ್ರೀಮಾತೆ ಶಾರದಾದೇವಿಯವರ ವಿವರಪೂರ್ಣ ಜೀವನಕಥೆ

 \item ಬ್ರಹ್ಮಾನುಭವಿ\\–ಸ್ವಾಮಿ ಬ್ರಹ್ಮಾನಂದರ ಜೀವನ ಚರಿತ್ರೆ

 \item ಶ್ರೀಶಾರದಾದೇವೀ ಸಂದೇಶ ಮಂದಾರ

\end{enumerate}

\begin{center}
\textbf{ಕಿರು ಹೊತ್ತಗೆಗಳು}
\end{center}

\begin{enumerate}
\item ಚಿಂತನ-ಮಂಥನ

 \item ಮಿಂಚಿನ ಗೊಂಚಲು

 \item ವಿಶ್ವಮಾನವನಾಗಿ ವಿವೇಕಾನಂದ

 \item ಕಲ್ಪತರು ಶ್ರೀರಾಮಕೃಷ್ಣ

 \item ಶಾಂತಿಯ ಹರಕೆ

 \item ಶ್ರೀಮಾತಾ ವಚನಮಧು

\end{enumerate}

\begin{center}
\textbf{ವಿಶೇಷವಾಗಿ ವಿದ್ಯಾರ್ಥಿಗಳ ಹಾಗೂ ಯುವಕರ ಉಪಯೋಗಕ್ಕಾಗಿ ಬರೆದ ಪುಸ್ತಕಗಳು}
\end{center}

\begin{enumerate}
\item ವೀರ ನರೇಂದ್ರ (ವೀರಸಂನ್ಯಾಸಿ ವಿವೇಕಾನಂದ ಗ್ರಂಥದ ಮೊದಲ ಆರು ಅಧ್ಯಾಯಗಳು)

 \item ಧೀರತೆಯ ದುಂದುಭಿ

 \item ಬಾಲಕ ಸಂಘ–ಒಂದು ವಿನೂತನ ಪ್ರಯೋಗ

 \item ವಿದ್ಯಾರ್ಥಿಗಾಗಿ

 \item \eng{Letter to a Student}

 \item \eng{Secret of Concentration}

\end{enumerate}

