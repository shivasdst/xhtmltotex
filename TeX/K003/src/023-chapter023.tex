
\chapter{ವಿವಿಧ ಭಾವಗಳಲ್ಲಿ}

\noindent

ಎಡೆಬಿಡದ ಚಟುವಟಿಕೆಗಳಿಂದಾಗಿ ಸ್ವಾಮೀಜಿಯ ಶರೀರ ದುರ್ಬಲವಾಗತೊಡಗಿತ್ತು. ಇತ್ತೀಚಿನ ಸುದೀರ್ಘ ಅಮೆರಿಕ ಪ್ರವಾಸದ ಸಂದರ್ಭದಲ್ಲಿ ಅವರು ಶಿಕಾಗೋಗೆ ಬಂದಿದ್ದಾಗಲೇ ಅವರ ಆರೋಗ್ಯ ಸರಿಯಿರಲಿಲ್ಲ. ೧೮೯೬ರ ಏಪ್ರಿಲ್ ತಿಂಗಳಲ್ಲಿ ನ್ಯೂಯಾರ್ಕಿಗೆ ಹಿಂದಿರುಗುವ ಹೊತ್ತಿಗಂತೂ ಅವರು ಸಂಪೂರ್ಣ ಬಳಲಿ ಹೋಗಿದ್ದರು. ತಮ್ಮ ಶಕ್ತಿಯನ್ನೆಲ್ಲ ಅವರು ಕಾರ್ಯೋದ್ದೇಶಕ್ಕಾಗಿ ಸವೆಸಿಬಿಟ್ಟಿದ್ದರು. ನಿಜಕ್ಕೂ ಅವರ ದೇಹ ಎಷ್ಟು ದಣಿದಿತ್ತೆಂದರೆ, ಅವರು ರೈಲು ಪ್ರಯಾಣ ಮಾಡಿ ಬಂದ ಎಷ್ಟೋ ದಿನಗಳವರೆಗೂ ಟ್ರೈನಿನ ಚಕ್ರಗಳ ಗಢಗಢ ಸದ್ದು ಕಿವಿಗಳಲ್ಲಿ ಮೊರೆಯುತ್ತಿದ್ದಂತೆ ಅವರಿಗೆ ಭಾಸವಾಗುತ್ತಿತ್ತು. ಅವರ ತಲೆಯಲ್ಲಿ ಆಲೋಚನೆ ಗಳೆಲ್ಲ ಸ್ಫುಟವಾಗಿರುತ್ತಿತ್ತಾದರೂ ಕೆಲವೊಮ್ಮೆ ಏನೋ ಒಂದು ಬಗೆಯ ಅಳುಕು ಅವರನ್ನು ಆವರಿಸಿಬಿಡುತ್ತಿತ್ತು. ಯೌವನದ ಉತ್ತುಂಗ ಸ್ಥಿತಿಯಲ್ಲಿದ್ದಾಗ ಅವರು ನಡೆಸಿದ ಪರಿವ್ರಾಜಕ ಜೀವನದ ಹಾಗೂ ಆಧ್ಯಾತ್ಮಿಕ ಸಾಧನೆಗಳ ಪರಿಶ್ರಮ, ಅಮೆರಿಕದಲ್ಲಿ ಅವರು ಅವಿಶ್ರಾಂತವಾಗಿ ಮಾಡಿದ ವೇದಾಂತಪ್ರಸಾರ ಕಾರ್ಯದ ದಣಿವು ಮತ್ತು ಅವರ ಮನಸ್ಸಿನಲ್ಲಿ ಅನವರತವೂ ನಡೆಯುತ್ತಿದ್ದ ಮಹಾ ಆಂದೋಳನದ ಪರಿಣಾಮ–ಇವುಗಳಿಂದಾಗಿ ಅವರ ಶರೀರ ಜರ್ಜರ ಗೊಂಡಿತ್ತು. ಜಗತ್ತಿನ ಹಿತಕ್ಕಾಗಿ ಹೇಗೆ ಅವರು ತಮ್ಮನ್ನೇ ಸಮರ್ಪಿಸಿಕೊಂಡುಬಿಟ್ಟಿದ್ದಾರೆ ಎಂಬುದನ್ನು ಗಮನಿಸುತ್ತಲೇ ಇದ್ದ ಅವರ ಶಿಷ್ಯರು-ಆಪ್ತರು ಕಳವಳಗೊಂಡಿದ್ದರು. ತಮ್ಮ ಕಾರ್ಯಸಾಧನೆಗಾಗಿ ಸ್ವಾಮೀಜಿ ತೆರುತ್ತಿರುವ ಬೆಲೆ ತುಂಬ ದುಬಾರಿಯಾದುದೆಂದು ಅವರಿಗೆಲ್ಲ ಅನ್ನಿಸಿತ್ತು. ಡೆಟ್ರಾಯ್ಟಿನಲ್ಲಿ ಶ್ರೀಮತಿ ಫಂಕೆ ಗಮನಿಸಿದ್ದಂತೆ, ಸ್ವಾಮೀಜಿ ಇನ್ನು ಹೆಚ್ಚು ಕಾಲ ಉಳಿದುಕೊಳ್ಳಲಾರರೆಂಬುದರ ಸೂಚನೆ ಆಗಲೇ ಕಂಡುಬರುತ್ತಿತ್ತು. ಆ ಒತ್ತಡವನ್ನು ತಾಳಿಕೊಳ್ಳ ಲಾರದೆ ಅವರ ಸುದೃಢ ಶರೀರವೂ ಕೂಡ, ನಿಧಾನವಾಗಿಯಾದರೂ ನಿಶ್ಚಯವಾಗಿ ಕುಸಿದು ಹೋಗುತ್ತಿದ್ದುದು ಗೋಚರವಾಗುತ್ತಿತ್ತು. ಈ ವಿಷಯ ಸ್ವಾಮೀಜಿಗೂ ಚೆನ್ನಾಗಿಯೇ ಅರಿ ವಾಗಿತ್ತು. ಬಾಸ್ಟನ್ನಿನಿಂದ ಅಳಸಿಂಗರಿಗೆ ಬರೆದ ಒಂದು ಪತ್ರದಲ್ಲಿ ಆ ವಿಷಯವನ್ನು ಅವರು ಪ್ರಸ್ತಾಪಿಸಿದ್ದರು–“ಮುಂದಿನ ತಿಂಗಳು ನಾನು ಮತ್ತೆ ಇಂಗ್ಲೆಂಡಿಗೆ ಹೋಗುವವನಿದ್ದೇನೆ. ನಾನು ಕೆಲಸ ಮಾಡಿದ್ದು ಶರೀರಕ್ಕೆ ಅತಿಯಾಯಿತೇನೋ... ಈ ದೀರ್ಘ-ನಿರಂತರ ಪರಿಶ್ರಮದಿಂದಾಗಿ ನನ್ನ ನರಗಳೆಲ್ಲ ಹೆಚ್ಚುಕಡಿಮೆ ಕುಸಿದೇ ಹೋಗಿವೆಯೆನ್ನಬೇಕು. ನೀನು ನನ್ನ ಬಗ್ಗೆ ಸಹಾನುಭೂತಿ ಯನ್ನೇನೂ ತಾಳಬೇಕಾಗಿಲ್ಲ, ಆದರೆ ಇನ್ನು ನೀನು ನನ್ನಿಂದ ಹೆಚ್ಚಿನದೇನನ್ನೂ ನಿರೀಕ್ಷಿಸುವಂತಿಲ್ಲ ಎಂದು ಸೂಚಿಸಲು ನಾನಿದನ್ನು ಬರೆಯುತ್ತಿದ್ದೇನಷ್ಟೆ. ನಿನ್ನಿಂದ ಸಾಧ್ಯವಾದಷ್ಟು ಚೆನ್ನಾಗಿ ಕೆಲಸ ಮಾಡು. ಮಹತ್ಕಾರ್ಯಗಳನ್ನು ಸಾಧಿಸಲು ಈಗ ನನ್ನಲ್ಲಿರುವ ಭರವಸೆ ತೀರ ಸ್ವಲ್ಪ. ಅದೇನೇ ಇದ್ದರೂ, ಶೀಘ್ರಲಿಪಿಯಿಂದ ನನ್ನ ಉಪನ್ಯಾಸಗಳ ಟಿಪ್ಪಣಿಗಳು ಸಿದ್ಧವಾಗಿದ್ದು, ಸಾಕಷ್ಟು ಹೆಚ್ಚಿನ ಪ್ರಮಾಣದಲ್ಲಿ ಸಾಹಿತ್ಯವನ್ನು ಹೊರತರಲು ಸಾಧ್ಯವಾಗಿರುವುದರಿಂದ ನನಗೆ ಸಂತೋಷವಾಗಿದೆ. ಈಗಾಗಲೇ ನಾಲ್ಕು ಪುಸ್ತಕಗಳು ಸಿದ್ಧವಾಗಿವೆ... ಒಳ್ಳೆಯದು, ನನ್ನ ಕೈಯಲ್ಲಿಸಾಧ್ಯವಾದ ಮಟ್ಟಿಗೂ ಒಳಿತನ್ನು ಮಾಡಲು ಪ್ರಯತ್ನಿಸಿದ್ದೇನೆಂಬ ಸಮಾಧಾನ ನನಗಿದೆ. ನನ್ನ ಕಾರ್ಯದಿಂದ ನಾನು ನಿವೃತ್ತನಾಗಿ ಗುಹೆಯೊಂದರ ಒಳಗೆ ಕುಳಿತುಕೊಂಡಾಗ ನನ್ನ ಆತ್ಮಸಾಕ್ಷಿ ನಿರಾಳವಾಗಿರುತ್ತದೆ.”

ಆದರೆ ಸ್ವಾಮೀಜಿ ಜಗನ್ಮಾತೆಗೆ ಬಲಿದಾನವಾಗಲೆಂದೇ ಜನ್ಮವೆತ್ತಿದವರಲ್ಲವೆ? ತಮ್ಮ ಕಾರ್ಯದಿಂದ ನಿವೃತ್ತರಾಗುವುದಾಗಲಿ ಗುಹೆಯೊಳಗೆ ಶಾಂತವಾಗಿ ಕುಳಿತು ಬಿಡುವುದಾಗಲಿ ಅವರ ಹಣೆಯಲ್ಲಿ ಬರೆದಿರಲಿಲ್ಲ. ನಿಜಕ್ಕೂ ಹೀಗೆ ಹೇಳಿದ ಮೇಲೆ ಅವರು ಮೊದಲಿಗಿಂತಲೂ ಹೆಚ್ಚಿನ ಶ್ರಮವಹಿಸಿ ದುಡಿದರು.

ಸ್ವಾಮೀಜಿ ಅಮೆರಿಕದಲ್ಲಿ ನಿರಂತರ ಚಟುವಟಿಕೆಗಳಲ್ಲಿ ಮುಳುಗಿದ್ದರಾದರೂ ಆಗಾಗ ಸ್ವಲ್ಪ ಬಿಡುವು ಮಾಡಿಕೊಂಡು ತಮ್ಮ ಅನುವರ್ತಿಗಳೊಂದಿಗೆ ಹಾಸ್ಯ ವಿಡಂಬನೆಯಲ್ಲಿ ತೊಡಗುತ್ತಿ ದ್ದರು. ಅಂತಹ ಸಮಯಗಳಲ್ಲಿ ಅವರು ಮುಗ್ಧ ಬಾಲಕನಂತೆ ನಕ್ಕು ನಲಿದು ಇತರರನ್ನೂ ನಗಿಸು ತ್ತಿದ್ದರು. ಆದರೆ ಅವರ ಮನಸ್ಸು ಗಂಭೀರ ವಿಚಾರಗಳೆಡೆಗೆ ತಿರುಗಿತೆಂದರೆ ಅದು ತಕ್ಷಣ ಅಂತರ್ಮುಖವಾಗುತ್ತಿತ್ತು. ಆ ಗಂಭೀರ ಭಾವದಿಂದ ಅವರ ಮನಸ್ಸು ಹಿಂದಿರುಗಬೇಕೆಂದರೆ ಬಹಳ ಹೊತ್ತು ಹಿಡಿಯುತ್ತಿತ್ತು. ಈ ಬಗೆಯ ಚಿಂತನೆ ಅವರಿಗೆ ಸಾಕಷ್ಟು ಬಳಲಿಕೆಯನ್ನುಂಟು ಮಾಡುತ್ತಿತ್ತು. ಆದ್ದರಿಂದ ಅವರ ಜೊತೆಯಲ್ಲಿದ್ದವರು ಅವರನ್ನು ಯಾವಾಗಲೂ ನಗುನಗು ತ್ತಿರುವಂತೆ ಮಾಡಲು ಪ್ರಯತ್ನಿಸುತ್ತಿದ್ದರು.

ಸ್ವತಃ ಸ್ವಾಮೀಜಿ ಹಾಸ್ಯಪ್ರಿಯರು ಮತ್ತು ಅತ್ಯಂತ ಸುಸಂಸ್ಕೃತವೂ ಪರಿಶುದ್ಧವೂ ಆದ ಹಾಸ್ಯ ಪ್ರಜ್ಞೆಯನ್ನು ಹೊಂದಿದ್ದವರು. ಅವರ ಹಾಸ್ಯದ ಶೈಲಿ ತುಂಬ ವಿಶಿಷ್ಟವಾದುದು, ಅನನುಕರಣೀಯ ವಾದುದು. ಒಮ್ಮೆ ಅವರು ಹಾಸ್ಯರಸವನ್ನು ಹರಿಸತೊಡಗಿದರೆಂದರೆ ಅವರ ಸುತ್ತಲಿದ್ದವರೆಲ್ಲ ನಕ್ಕುನಕ್ಕು ಸುಸ್ತಾಗುತ್ತಿದ್ದರು. ಅವರಿಗೆ ಕೆಲವು ಹಾಸ್ಯ ಚಟಾಕಿಗಳು ವಿಶೇಷವಾಗಿ ಪ್ರಿಯವಾಗಿ ದ್ದುವು. ಅವುಗಳನ್ನು ಅವರು ಎಷ್ಟೋ ಸಲ ತಮ್ಮ ಆತ್ಮೀಯರೆದುರಿನಲ್ಲಿ ನಾಟಕೀಯವಾಗಿ ವಿವರಿಸಿ ನಗೆಯ ಅಲೆಗಳನ್ನೆಬ್ಬಿಸುತ್ತಿದ್ದರು.

ಇವುಗಳಲ್ಲಿ ಒಂದೆಂದರೆ ಕ್ರೈಸ್ತ ಪಾದ್ರಿಯೊಬ್ಬ ಧರ್ಮಪ್ರಚಾರಕ್ಕಾಗಿ ಗುಡ್ಡಗಾಡು ಜನರ ದ್ವೀಪಕ್ಕೆ ಹೋದ ಕತೆ. ಈ ದ್ವೀಪದ ಜನ ನರಭಕ್ಷಕರು. ಈ ಜನರಿಗೆ ಬುದ್ಧಿ ಹೇಳಿ, ಕ್ರೈಸ್ತಧರ್ಮದ ಮಹಿಮೆಯನ್ನು ತಿಳಿಸಿಕೊಟ್ಟು ಅವರನ್ನೂ ಕ್ರೈಸ್ತರನ್ನಾಗಿ ಪರಿವರ್ತಿಸಲು ಎಷ್ಟೋ ಧರ್ಮಪ್ರಚಾರ ಕರು ಆ ದ್ವೀಪಕ್ಕೆ ಹೋಗಿದ್ದರು. ಆದರೆ ಅವರ ಕಾರ್ಯ ಯಶಸ್ವಿಯಾಯಿತೋ ಇಲ್ಲವೋ ಎಂಬುದೇ ಗೊತ್ತಾಗಿರಲಿಲ್ಲ. ಏಕೆಂದರೆ ಅವರ್ಯಾರೂ ಬಂದು ಅದರ ಬಗ್ಗೆ ವರದಿ ಸಲ್ಲಿಸಲೇ ಇಲ್ಲ. ಆದ್ದರಿಂದ ಕ್ರೈಸ್ತಧರ್ಮ ಪ್ರಚಾರಕ ಸಂಸ್ಥೆಯೊಂದು ಅಲ್ಲಿಗೆ ಇನ್ನೊಬ್ಬ ಪಾದ್ರಿಯನ್ನು ಕಳಿಸಿಕೊಟ್ಟಿತು. ಈತ ತುಂಬ ಉತ್ಸಾಹಿ. ತುಂಬ ಆತ್ಮವಿಶ್ವಾಸದಿಂದ ಆ ದ್ವೀಪಕ್ಕೆ ಹೋದ. ಇವನನ್ನು ನೋಡುತ್ತಲೇ ಅಲ್ಲಿನ ನಿವಾಸಿಗಳೆಲ್ಲ ಕುತೂಹಲದಿಂದ ಓಡಿಬಂದು ಇವನ ಸುತ್ತಲೂ ಮುತ್ತಿಕೊಂಡರು. ಆ ಜನರ ಮುಖಂಡನೂ ಬಂದ. ಜನರೆಲ್ಲ ತಾವಾಗಿಯೇ ಬಂದದ್ದನ್ನು ಕಂಡು ಈ ಪಾದ್ರಿಗೆ ಬಹಳ ಸಂತೋಷವಾಯಿತು. ತನ್ನ ಪ್ರವರವನ್ನು ಪ್ರಾರಂಭಿಸುವ ಮೊದಲು ಅಲ್ಲಿನ ಮುಖಂಡನನ್ನು ಕೇಳಿದ:

“ಏನಪ್ಪ, ನನಗಿಂತ ಮುಂಚೆ ಇಲ್ಲಿ ಬಂದಿದ್ದರಲ್ಲ ಪಾದ್ರಿಗಳು, ಅವರ ಬಗ್ಗೆ ನಿಮಗೆ ಏನನ್ನಿಸಿತು?”

ಆ ಮುಖಂಡ ಬಹಳ ಸಂತೋಷದಿಂದ ಬಾಯಿ ಚಪ್ಪರಿಸುತ್ತ ಹೇಳಿದ: “ಆಹ್! ಸ್ವಾಮಿ, ಅವರು ತುಂಬ-ತುಂಬ ರುಚಿಯಾಗಿದ್ದರು!”

ಎರಡನೆಯ ಕತೆ, ಒಬ್ಬ ನೀಗ್ರೋ ಪಾದರಿಗೆ ಸಂಬಂಧಿಸಿದುದು. ಅಮೆರಿಕದ ನೀಗ್ರೋಗಳು ಆ ದೇಶದಲ್ಲೇ ಹುಟ್ಟಿಬೆಳೆದವರಾದರೂ ಅವರಲ್ಲಿ ಬಹಳಷ್ಟು ಜನರ ಇಂಗ್ಲಿಷ್ ಉಚ್ಚಾರಣೆ ಮತ್ತು ಭಾಷಾ ಪ್ರಯೋಗ ತುಂಬ ಅಸಂಸ್ಕೃತವಾಗಿರುತ್ತದೆ. ಈ ನೀಗ್ರೋಗಳಲ್ಲೊಬ್ಬ ಧರ್ಮ ಪ್ರಚಾರಕನಾಗಿದ್ದ. ಅವನು ತಮ್ಮ ಜನರಿಗೇ–ಎಂದರೆ ನೀಗ್ರೋಗಳಿಗೆ–ಧರ್ಮಬೋಧನೆ ಮಾಡುತ್ತಿದ್ದ. ಒಮ್ಮೆ ದೇವರ ಮಹಿಮೆಯನ್ನು ಬಣ್ಣಿಸುತ್ತ ಬೈಬಲಿನಲ್ಲಿ ಹೇಳಿರುವಂತೆ ಸೃಷ್ಟಿಯ ಕತೆಯನ್ನು ಉಚ್ಚಸ್ವರದಲ್ಲಿ ಹೇಳತೊಡಗಿದ:

“ಇಲ್ಲಿ ನೋಡಿ, ದೇವರು ಮೊದಲು ಆಡಮ್ (ಮೊಟ್ಟ ಮೊದಲ ಪುರುಷ)ನನ್ನು ಮಾಡಿದ. ಅವನು ಆಡಮ್​ನನ್ನು ಮಣ್ಣಿನಿಂದ ತಯಾರು ಮಾಡಿದ. ಮಾಡಿ ಮುಗಿಸಿದ ಮ್ಯಾಲೆ ಅವನನ್ನು ಒಣಗಿಸಲು ಬೇಲಿಯ ಮ್ಯಾಲೆ ಹರವಿದ. ಆಮ್ಯಾಲೆ... ”

ಅಷ್ಟರಲ್ಲೇ ಸಭಿಕರ ಮಧ್ಯದಲ್ಲಿ ಒಬ್ಬ ಬುದ್ಧಿವಂತ ಎದ್ದುನಿಂತು ಕೂಗಿದ: “ಓಯ್, ಸ್ವಲ್ಪ ನಿಲ್ಲಿ ಸ್ವಲ್ಪ ನಿಲ್ಲಿ ಯಜಮಾನ್ರೆ! ಆ ಬೇಲಿ ಅಂದ್ರಲ್ಲ, ಅದನ್ನ ಮಾಡಿದವರ್ಯಾರು?”

ಈ ಕಪ್ಪು ಪಾದ್ರಿ ಸಿಟ್ಟಿಗೆದ್ದು ಖಾರವಾಗಿ ಹೇಳಿದ:

“ಲೇ ತಮ್ಮ, ಸುಮ್ಮನೆ ಕೇಳ್ತಾ ಕುಂತ್ಕೊಳ್ಳೊ. ಈಗ ಹಾಗೆಲ್ಲ ಪ್ರಶ್ನೆ ಕೇಳೋದಕ್ಕೆ ಹೋಗ ಬೇಡ. ಹಾಗೆಲ್ಲಾ ಕೇಳ್ತಾಹೋದರೆ ನೀನು ನಮ್ಮ ಶಾಸ್ತ್ರವನ್ನೆಲ್ಲ ತಲೆಕೆಳಗು ಮಾಡಿಬಿಡ್ತೀಯ!”

ಸ್ವಾಮೀಜಿ ಬಹಳವಾಗಿ ಇಷ್ಟಪಟ್ಟಿದ್ದ ಇನ್ನೊಂದು ಹಾಸ್ಯ ಚಟಾಕಿ ಹೀಗಿತ್ತು:

ಒಮ್ಮೆ ಒಬ್ಬ ಚೀನೀ ಮೂಲದ ಮನುಷ್ಯನನ್ನು ಹಂದಿಮಾಂಸವನ್ನು ಕದ್ದ ಆಪಾದನೆಯ ಮೇಲೆ ಅಮೆರಿಕದ ಪೋಲೀಸರು ಹಿಡಿದು ನ್ಯಾಯಾಧೀಶನ ಮುಂದೆ ಹಾಜರುಪಡಿಸಿದರು. ನ್ಯಾಯಾಧೀಶ ಈ ಚೀನೀಯನನ್ನು ಕಂಡು ಆಶ್ಚರ್ಯದಿಂದ ಉದ್ಗರಿಸಿದ: “ಅರೆ! ಚೀನೀಯರು ಹಂದಿಯ ಮಾಂಸವನ್ನು ತಿನ್ನುವುದಿಲ್ಲ ಎಂದು ಭಾವಿಸಿದ್ದೆನಲ್ಲ ನಾನು!”

ಆಗ ಆ ಅಪಾದಿತ ವಿವರಣೆ ನೀಡಿದ:

“ಹೌದು ಮಹಾಸ್ವಾಮಿ. ನೀವು ಹೇಳಿದ್ದು ನಿಜವೇ. ಆದ್ರೆ ಈಗ ನಾನು ಮೇಲಿಕಾದವ್ನು (ಅಮೆರಿಕದವನು). ಈಗ ನಾನು ಕದೀತೇನೆ, ಹಂದಿ ಮಾಂಸ ತಿಂತೇನೆ, ಎಲ್ಲಾ ಮಾಡ್ತೇನೆ.”

ಇಂತಹ ಚಟಾಕಿಗಳನ್ನು ಹಾರಿಸಿ ಸ್ವಾಮೀಜಿ ತಾವೂ ನಕ್ಕು ಇತರರನ್ನೂ ಹೊಟ್ಟೆಹುಣ್ಣಾಗು ವಂತೆ ನಗಿಸುತ್ತಿದ್ದರು. ನಗುವು ಸಹಜ ಧರ್ಮವಲ್ಲವೆ? ಮಹಾಪುರುಷರು ಎಲ್ಲ ಹೊತ್ತಿಗೂ ಗಂಭೀರವಾಗಿಯೇ ಇರುವುದಿಲ್ಲ. ಬೌದ್ಧಿಕ ಸಾಮರ್ಥ್ಯ-ಆಧ್ಯಾತ್ಮಿಕ ಶಕ್ತಿಗಳಂತೆ ಹಾಸ್ಯ ಪ್ರವೃತ್ತಿಯೂ ಸ್ವಾಮೀಜಿಯ ವ್ಯಕ್ತಿತ್ವದ ಒಂದು ಅಂಗವಾಗಿತ್ತು. ಜನರು ಮಹಾತ್ಮನೊಬ್ಬನ ಉನ್ನತ ಬೋಧನೆಗಳನ್ನು ಕೇಳಲು ಇಚ್ಛಿಸುವಂತೆಯೇ, ಅವನ ವೈಯಕ್ತಿಕ ಸ್ವಭಾವದ ಕುರಿತಾಗಿ, ಅವನ ನಡೆನುಡಿಗಳ ಕುರಿತಾಗಿ ತಿಳಿದುಕೊಳ್ಳಲು ಇಚ್ಛಿಸಿದರೆ ಅದು ಸಹಜವೇ. ಮಹಾತ್ಮರ ಸಾನ್ನಿಧ್ಯದಲ್ಲಿರುವವರು ಅವರ ಮಾನುಷಸಹಜವಾದ ಗುಣಗಳನ್ನು ಕೂಡ ಕಂಡು ಮೆಚ್ಚಿ ಕೊಳ್ಳುತ್ತಾರೆ. ಅಂತೆಯೇ ಸ್ವಾಮೀಜಿಯ ಆಪ್ತರು, ಅವರ ಸಹಜ-ಮಧುರ ಸ್ವಭಾವವನ್ನು ಮೆಚ್ಚಿ ಆನಂದಿಸುತ್ತಿದ್ದರು; ಅವರ ಮನಸ್ಸು ಸದಾ ಆಹ್ಲಾದಕರವಾಗಿರುವಂತೆ ಮಾಡಲು ಶ್ರಮಿಸುತ್ತಿ ದ್ದರು. ಲೆಗೆಟ್ ಕುಟುಂಬದವರು, ಹೇಲ್ ಕುಟುಂಬದವರು ಮೊದಲಾದ ಅವರ ಶ್ರೀಮಂತ ಸ್ನೇಹಿತರು ಅವರನ್ನು ತಮ್ಮ ಮನೆಗಳಿಗೆ ರಜಾ ದಿನಗಳನ್ನು ಕಳೆಯಲು ಆಹ್ವಾನಿಸುತ್ತಿದ್ದರು. ಆ ಮನೆಗಳಲ್ಲಿ ಸ್ವಾಮೀಜಿಗೆ ಸಂಪೂರ್ಣ ಸ್ವಾತಂತ್ರ್ಯ. ಅಲ್ಲಿ ಅವರು ತಮ್ಮಗಿಷ್ಟ ಬಂದಂತೆ ಓಡಾಡ ಬಹುದಾಗಿತ್ತು. ಅವರಿಗೆ ಸ್ವಲ್ಪ ಹರಟುವ ಮನಸ್ಸಾದರೆ ಮನೆಮಂದಿಯೆಲ್ಲ ಅವರನ್ನು ಮುತ್ತಿ ಕೊಂಡು ಸಂತೋಷದಿಂದ ಜೊತೆಕೊಡುತ್ತಿದ್ದರು. ಯಾವಾಗಲಾದರೂ ಅವರು ಸ್ವಸಂತೋಷ ಕ್ಕಾಗಿ ಹಾಡತೊಡಗಿದರೆಂದರೆ ಇತರರು ಆ ಮಧುರ ಗಾಯನವನ್ನು ಆಲಿಸುತ್ತ ಮೈಮರೆಯುತ್ತಿ ದ್ದರು. ಇಲ್ಲವೆ ಅವರೇನಾದರೂ ಗಂಭೀರಭಾವದಲ್ಲಿದ್ದರೆ ಯಾರೂ ಅವರ ಗಮನವನ್ನು ವಿಚಲಿತಗೊಳಿಸದೆ ಅವರಷ್ಟಕ್ಕೆ ಬಿಟ್ಟುಬಿಡುತ್ತಿದ್ದರು. ಹೀಗೆ ಕೆಲವೊಮ್ಮೆ ಅವರು ದಿನಗಟ್ಟಲೆ ಅದೇ ಭಾವದಲ್ಲಿ ಮುಳುಗಿ, ಹೆಚ್ಚುಕಡಿಮೆ ಯಾವಾಗಲೂ ಮೌನವಾಗಿದ್ದುಬಿಡುತ್ತಿದ್ದರು. ಅಥವಾ ಅವರು ತಮ್ಮ ಕೋಣೆಯಲ್ಲಿ ಕುಳಿತು ಗಂಟೆಗಟ್ಟಲೆ ಕಾಲ ಧ್ಯಾನಮಗ್ನರಾಗುತ್ತಿದ್ದರು. ಸ್ವಾಮೀಜಿ ತಮ್ಮೊಂದಿಗಿರುವ ಪ್ರತಿಯೊಂದು ಕ್ಷಣವೂ ತಮ್ಮ ಪರಮ ಸೌಭಾಗ್ಯವೆಂದು ಆ ಆತಿಥೇಯರು ತಿಳಿಯುತ್ತಿದ್ದರು. ಅವರನ್ನು ಸಂತೋಷವಾಗಿಡಲು ತಮ್ಮಿಂದಾದುದನ್ನೆಲ್ಲ ಮಾಡುತ್ತಿದ್ದರು. ಒಮ್ಮೆ ಸ್ವಾಮೀಜಿ ಶಿಕಾಗೋದಲ್ಲಿ ತಮ್ಮ ಅತ್ಯಂತ ಆಪ್ತರಾದ ಹೇಲ್ ಕುಟುಂಬದವರ ಮನೆಯಲ್ಲಿದ್ದಾಗ ಅವರಿಗೆ ಉರುಳುವ ಸ್ಕೇಟಿಂಗ್ (ರೋಲರ್ ಸ್ಕೇಟಿಂಗ್​) ಆಡುವ ಮನಸ್ಸಾಯಿತು. ಸರಿ, ಸ್ಕೇಟಿಂಗ್ ಬೂಟ್​ಗಳನ್ನು ತೊಟ್ಟುಕೊಂಡು ಮೂರು ದಿನಗಳ ಕಾಲ ಸ್ಕೇಟಿಂಗ್ ಅಭ್ಯಾಸ ಮಾಡಿದರು. ಅವರು ಹಾಗೆ ಅಭ್ಯಾಸ ಮಾಡಿದ್ದೆಲ್ಲಿ ಎಂದರೆ ಆ ವಿಶಾಲವಾದ ಮನೆಯೊಳಗೇ–ನೆಲಕ್ಕೆ ಅಂಟಿಸಿದಂತಿದ್ದ ಅಮೂಲ್ಯವಾದ ಜಮಖಾನೆಯ ಮೇಲೆಯೇ! ಆದರೆ ಮನೆಯವರೊಬ್ಬರೂ ಅದಕ್ಕೆ ಆಕ್ಷೇಪಿಸಲಿಲ್ಲ; ಅಷ್ಟೇ ಅಲ್ಲ, ಅವರು ಜಾರುತ್ತ ಓಡುವುದನ್ನು ಕಂಡು ಮನಸಾರೆ ಸಂತೋಷಪಟ್ಟರು. ಆದರೆ ಮೂರು ದಿನಗಳಾದ ಮೇಲೆ ಸ್ವಾಮೀಜಿಗೆ ಅದರ ಮೇಲೆ ಬೇಸರ ಹುಟ್ಟಿತು. ಮತ್ತೆಂದೂ ಅವರು ಸ್ಕೇಟಿಂಗ್ ಆಡಲಿಲ್ಲ.

ಸ್ವಾಮೀಜಿಯಲ್ಲಿದ್ದ ಒಂದು ಅಪೂರ್ವ ಗುಣವೆಂದರೆ ಸರಳತೆ, ಅಹಂಕಾರರಾಹಿತ್ಯ. ಇದಕ್ಕೆ ಉದಾಹರಣೆಯಾಗಿ, ಅವರ ಆಪ್ತರಲ್ಲೊಬ್ಬಳಾದ ಮಿಸ್ ಎಲೆನ್ ವಾಲ್ಡೊಳ ಅನುಭವಗಳನ್ನು ನೋಡಬಹುದು.

ಸ್ವಾಮೀಜಿಯ ಬಗ್ಗೆ ಮಿಸ್ ವಾಲ್ಡೊಳಿಗಿದ್ದ ಭಕ್ತಿ-ಗೌರವ ಅತಿಶಯವಾದದ್ದು. ಆದರೆ ಅವಳು ಅವರನ್ನು ಪರೀಕ್ಷೆ ಮಾಡದೆ ಸುಮ್ಮನೆ ಒಪ್ಪಿಕೊಂಡುಬಿಡಲಿಲ್ಲ. ಅವರ ವ್ಯಕ್ತಿತ್ವದ ಹಾಗೂ ಬೋಧನೆಯ ಆಕರ್ಷಣೆ ಆಕೆಯನ್ನು ಸೆರೆಹಿಡಿದಿತ್ತಾದರೂ ಅವರನ್ನು ಆಕೆ ಯಾವಾಗಲೂ ಪರೀಕ್ಷಾ ದೃಷ್ಟಿಯಿಂದಲೇ ಗಮನಿಸುತ್ತಿದ್ದಳು. ಅವಳು ಆದಾಗಲೇ ಹಲವಾರು ವರ್ಷಗಳ ಕಾಲ ಅನೇಕ ಧರ್ಮಬೋಧಕರ ಶಿಷ್ಯೆಯಾಗಿ ಕುಳಿತು, ಅವರ ಬೋಧನೆಗಳನ್ನು ಕೇಳಿದ್ದವಳು. ಆದರೆ ಯಾವ ಗುರುಗಳೂ ಯಾವ ಬೋಧನೆಯೂ ಸಂಪೂರ್ಣವಾಗಿ ಒಪ್ಪಿಗೆಯಾಗಿರಲಿಲ್ಲ. ಅಲ್ಲದೆ ಅವಳು ಪ್ರತಿಯೊಬ್ಬ ಗುರುವಿನಲ್ಲೂ ಒಂದಲ್ಲ ಒಂದು ಲೋಪವನ್ನೋ ದೋಷವನ್ನೋ ಕಂಡಿದ್ದಳು. ಸ್ವಾಮೀಜಿಯ ಸಂಪರ್ಕಕ್ಕೆ ಬಂದಾಗ, ಇವರಲ್ಲೂ ಅಂತಹ ಯಾವುದಾದರೊಂದು ಲೋಪ- ದೋಷವಿದ್ದೇ ಇರುತ್ತದೆಯೆಂದು ಅವಳು ನಿರೀಕ್ಷಿಸಿದ್ದಳು.

ಒಂದು ದಿನ ಅಂಥದೊಂದು ದೌರ್ಬಲ್ಯವನ್ನು ಸ್ವಾಮೀಜಿಯಲ್ಲಿ ಆಕೆ ಕಂಡು ಹಿಡಿದಳು. ನ್ಯೂಯಾರ್ಕಿನಲ್ಲಿ ಅವರಿಬ್ಬರೂ ಶಿಷ್ಯರೊಬ್ಬರ ಮನೆಯಲ್ಲಿ ಇಳಿದುಕೊಂಡಿದ್ದರು. ಆ ಮನೆ ಯಲ್ಲಿ ಎರಡು ದೊಡ್ಡ ಕಿಟಕಿಗಳ ಮಧ್ಯೆ, ನೆಲದಿಂದ ತಾರಸಿಯವರೆಗೂ ಕನ್ನಡಿಯೊಂದಿತ್ತು. ಈ ಕನ್ನಡಿಯನ್ನು ಕಂಡು ಸ್ವಾಮೀಜಿಗೆ ಏನನ್ನಿಸಿತೋ ತಿಳಿಯದು. ಮತ್ತೆ ಮತ್ತೆ ಅವರು ಅದರ ಮುಂದೆ ನಿಂತು ತಮ್ಮನ್ನೇ ದಿಟ್ಟಿಸಿ ನೋಡಿಕೊಳ್ಳುತ್ತ ಆಲೋಚನಾಮಗ್ನರಾಗುತ್ತಿದ್ದರು. ಅವರು ತಮ್ಮ ಸ್ಫುರದ್ರೂಪವನ್ನು ಕಂಡು ತಾವೇ ಮರುಳಾದಂತೆ ಕಾಣುತ್ತಿತ್ತು. ಸ್ವಲ್ಪ ಹೊತ್ತು ಹಾಗೇ ತಮ್ಮ ಪ್ರತಿಬಿಂಬವನ್ನು ದಿಟ್ಟಿಸಿ ಈ ಕಡೆಗೆ ತಿರುಗುವುದು, ಬಳಿ ನಾಲ್ಕು ಹೆಜ್ಜೆ ಅಡ್ಡಾಡಿ ಮತ್ತೆ ಕನ್ನಡಿಯ ಮುಂದೆ ನಿಲ್ಲುವುದು–ಹೀಗೆಯೇ ಮಾಡುತ್ತಿದ್ದರು. ಮಿಸ್ ವಾಲ್ಡೊಳ ಕಣ್ಣುಗಳು ಸ್ವಾಮೀಜಿಯನ್ನೇ ಹಿಂಬಾಲಿಸುದ್ದುವು. “ಹ್ಞೂ, ಗಾಳಿಯ ಗುಳ್ಳೆ ಇನ್ನೇನು ಒಡೆಯುತ್ತದೆ. ಅವರಿಗೆ ತುಂಬ ಹೆಮ್ಮೆಯಾಗಿಬಿಟ್ಟಿದೆ” ಎಂದು ಮನಸ್ಸಿನಲ್ಲೇ ಅಂದುಕೊಂಡಳು. ಇದ್ದಕ್ಕಿದ್ದಂತೆ ಸ್ವಾಮೀಜಿ ಅವಳತ್ತ ತಿರುಗಿ ಹೇಳಿದರು, “ಎಲೆನ್, ನಿಜಕ್ಕೂ ಇದೊಂದು ಆಶ್ಚರ್ಯವೇ ಸರಿ. ನಾನು ಹೇಗಿದ್ದೇನೆಂಬುದನ್ನು ನೆನಪಿಟ್ಟುಕೊಳ್ಳಲು ನನಗೆ ಸಾಧ್ಯವೇ ಆಗುತ್ತಿಲ್ಲ! ಕನ್ನಡಿಯಲ್ಲಿ ನಾನು ನನ್ನನ್ನೇ ಮತ್ತೆಮತ್ತೆ ನೋಡಿಕೊಳ್ಳುತ್ತೇನೆ; ಆದರೆ ಇತ್ತ ಕಡೆಗೆ ತಿರುಗಿದ ತಕ್ಷಣವೇ ನಾನು ಹೇಗೆ ಕಾಣುತ್ತೇನೆ ಎಂಬುದು ಮರೆತೇಹೋಗುತ್ತದೆ!”

ಸ್ವಾಮೀಜಿ ತಮ್ಮ ದೇಹಬುದ್ಧಿಯನ್ನು ಎಷ್ಟರಮಟ್ಟಿಗೆ ತೊಡೆದುಹಾಕಿದ್ದರೆಂದರೆ, ತಾವು ಹೇಗೆ ಕಾಣುತ್ತೇವೆ ಎಂಬುದನ್ನೂ ನೆನಪಿಟ್ಟುಕೊಳ್ಳಲು ಅವರಿಗೆ ಸಾಧ್ಯವಿರಲಿಲ್ಲ! ಈ ವಿಷಯ ಅವಳ ಮನಸ್ಸಿಗೆ ತಟಕ್ಕನೆ ಹೊಳೆದಾಗ ಅವಳಿಗಾದ ನಾಚಿಕೆ-ದುಃಖ ಅಷ್ಟಿಷ್ಟಲ್ಲ.

ಮತ್ತೊಂದು ಸಂದರ್ಭ; ಒಮ್ಮೆ ಮಿಸ್ ವಾಲ್ಡೊಳ ಕಣ್ಣುಗಳಲ್ಲಿ ನೀರು ತುಂಬಿದ್ದುದನ್ನು ಸ್ವಾಮೀಜಿ ಗಮನಿಸಿದರು. ಕಾತರದಿಂದ ಅವರು “ಏಕೆ ಅಳುತ್ತಿರುವೆ? ಏನಾಯಿತು?” ಎಂದು ಪ್ರಶ್ನಿಸಿದರು. ಆಗ ಅವಳು ದುಃಖದಿಂದ ಹೇಳಿದಳು, “ಸ್ವಾಮೀಜಿ, ನಿಮಗೆ ಸಂತೋಷವುಂಟು ಮಾಡಲು ನನಗೆ ಸಾಧ್ಯವೇ ಇಲ್ಲವೆಂಬಂತೆ ತೋರುತ್ತದೆ. ನಿಮಗೆ ಇತರರಿಂದ ತೊಂದರೆ ಯಾದಾಗಲೂ ನೀವು ನನಗೇ ಬೈಯುತ್ತೀರಿ.” ಅದನ್ನು ಕೇಳಿ ಸ್ವಾಮೀಜಿ ತುಂಬ ಸಹಜವಾಗಿ ಉತ್ತರಿಸಿದರು, “ಹೌದು; ಆದರೆ ಅವರಿಗೇ ಬೈಯೋಣವೆಂದರೆ ನನಗೆ ಅವರ ಪರಿಚಯ ಸಾಕಷ್ಟಿಲ್ಲವಲ್ಲ! ನಾನು ನನ್ನವರಿಗಲ್ಲದೆ ಇನ್ನಾರಿಗೆ ಬೈಯಲಿ?” ಈ ಮಾತು ಅವಳ ಮೇಲೆ ಗಾಢ ಪರಿಣಾಮವನ್ನು ಬೀರಿತು. ಅಂದಿನಿಂದ ಸ್ವಾಮೀಜಿ ಬೈದರೆ ಆಕೆ ಎಂದೂ ಬೇಸರಿಸುತ್ತಿರ ಲಿಲ್ಲ. ಬದಲಾಗಿ, ಅದು ತನ್ನ ಪಾಲಿನ ವಿಶೇಷ ಭಾಗ್ಯವೆಂದೇ ಭಾವಿಸುತ್ತಿದ್ದಳು.

ವಿವೇಕಾನಂದರಲ್ಲಿ ಕಂಡುಬರುತ್ತಿದ್ದ ಒಂದು ಅತಿ ಮುಖ್ಯ ಅಂಶವೆಂದರೆ ಅವರ ಸ್ವತಂತ್ರ ಮನೋವೃತ್ತಿ. ಅವರ ಆಲೋಚನೆ-ಮಾತು-ಕೃತಿಗಳೆಲ್ಲದರಲ್ಲೂ ಇದು ಎದ್ದು ತೋರುತ್ತಿತ್ತು. ಎಷ್ಟೋ ಸಲ ಇತರರಿಗೆ ಅವರ ವರ್ತನೆ ತುಂಬ ವಿಚಿತ್ರವಾಗಿ ಕಾಣುತ್ತಿತ್ತು. ಆದರೆ ಅವರು ಇತರರಿಗಾಗಿ ತಮ್ಮ ವರ್ತನೆಯನ್ನು ಬದಲಿಸಿಕೊಳ್ಳುತ್ತಿರಲಿಲ್ಲ. ತಮಗೆ ಯಾವುದು ಸರಿಯೆಂದು ಕಂಡುಬರುತ್ತಿತ್ತೋ ಅದರಂತೆಯೇ ಮಾಡುತ್ತಿದ್ದರು. ಎಷ್ಟೋ ಜನರಿಗೆ ಕೆಲವೊಮ್ಮೆ ಇದರಿಂದ ಕಸಿವಿಸಿಯಾಗುತ್ತಿತ್ತು. ಸ್ವಾಮೀಜಿ ಹೀಗೇಕೆ ಮಾಡಿದರು ಎಂದು ಚಿಂತಿಸುವಂತಾಗುತ್ತಿತ್ತು. ಆದರೆ ಅವರ ಪ್ರತಿಯೊಂದು ಕಾರ್ಯದ ಹಿಂದೆಯೂ ಒಂದು ಉದ್ದೇಶವಿರುತ್ತಿತ್ತು. ಈ ಉದ್ದೇಶವನ್ನು ಅರಿತವರಿಗೆ ಮಾತ್ರ ಅವರ ಆಲೋಚನಾವಿಧಾನ ಎಷ್ಟು ವಿಭಿನ್ನ, ಎಷ್ಟು ಭವ್ಯ, ಹಾಗೂ ಅವರ ಅಂತರ್ದೃಷ್ಟಿ ಎಷ್ಟು ಸೂಕ್ಷ್ಮವೆಂಬುದರ ಕಲ್ಪನೆಯಾಗುತ್ತಿತ್ತು.

ಸ್ವಾಮೀಜಿ ಸ್ವಾತಂತ್ರ್ಯಪ್ರಿಯರು. ಅವರು ಪ್ರತಿಯೊಂದು ವಿಚಾರದಲ್ಲೂ ಸ್ವಾತಂತ್ರ್ಯವನ್ನು ಬಯಸುತ್ತಿದ್ದರು. ಯಾವ ವಿಷಯದಲ್ಲಿ ಸ್ವಾತಂತ್ರ್ಯವನ್ನು ಬಯಸಿದರೂ ಅದು ಒಂದುವೇಳೆ ಸಿಗದೆ ಹೋದರೆ ಆ ಯೋಜನೆಯನ್ನು ಅಲ್ಲೇ ಒದರಿ ಬಿಡುತ್ತಿದ್ದರು. ಯಾವುದೇ ಬಗೆಯ ಹಂಗಿಗೂ ಅವರು ಸಿಕ್ಕಿಕೊಳ್ಳುತ್ತಿರಲಿಲ್ಲ. ತಮಗೆ ಅತಿ ಸಣ್ಣ ಉಪಕಾರ ಮಾಡಿದವರ ಬಗ್ಗೆಯೂ ಅವರು ಎಂದೆಂದಿಗೂ ಕೃತಜ್ಞರಾಗಿರುತ್ತಿದ್ದರು. ಆದರೆ ತಮಗೆ ನೆರವಾಗಲೆಂದು ಬಂದು ತಮ್ಮ ಮೇಲೆ ಯಾವುದೇ ಬಗೆಯಲ್ಲಿ ಒತ್ತಡ ಹೇರುವುದನ್ನು ಅವರು ಸಹಿಸುತ್ತಿರಲಿಲ್ಲ. ಎಷ್ಟೇ ಕಷ್ಟ ವಾದರೂ ಸರಿಯೆ, ತಮ್ಮ ಪ್ರತಿಯೊಂದು ಕೆಲಸವನ್ನೂ ತಾವೇ ಮಾಡಿಕೊಳ್ಳಲು ಅವರು ಯಾವಾ ಗಲೂ ಸಿದ್ಧರಿರುತ್ತಿದ್ದರು. ಯಾರು ಸ್ವಸಂತೋಷದಿಂದ, ಅತ್ಯಂತ ನಿಃಸ್ವಾರ್ಥಭಾವದಿಂದ ತಮಗೆ ನೆರವಾಗುತ್ತಿದ್ದರೋ ಅಂಥವರ ಪ್ರೀತಿಗೆ ಮಾತ್ರ ಅವರು ಮಣಿಯುತ್ತಿದ್ದರು. ಒಂದು ಸಲ ಒಬ್ಬಳು ಶ್ರೀಮಂತ ಮಹಿಳೆ, ಅವರ ಕೆಲಸಕಾರ್ಯಗಳ ಸಂಬಂಧವಾಗಿ ಕೆಲವು ವ್ಯವಸ್ಥೆಗಳನ್ನು ಮಾಡಲು ನೆರವಾಗುತ್ತಿದ್ದಳು. ಆದರೆ ಅವುಗಳನ್ನು ಆಕೆ ತನಗೆ ಸರಿಬಂದ ರೀತಿಯಲ್ಲಿ ಮಾಡಲು ತೊಡಗಿದಾಗ ಸ್ವಾಮೀಜಿ ಆ ಯೋಜನೆಗಳನ್ನೆಲ್ಲ ಅಸ್ತವ್ಯಸ್ತಗೊಳಿಸಿಬಿಟ್ಟರು. ಇದರಿಂದ ಅವಳಿಗೆ ಹತಾಶೆಯಾದ್ದರಿಂದ ಮೊದಲಿಗೇನೋ ಸ್ವಲ್ಪ ಅಸಮಾಧಾನವಾಯಿತು. ಆದರೆ ಬಳಿಕ ತನ್ನ ತಪ್ಪಿನ ಅರಿವಾದ ಮೇಲೆ ಆ ಮಹಿಳೆ ನಗುತ್ತ ತನ್ನ ಗೆಳತಿಯರಿಗೆ ಹೇಳುತ್ತಾಳೆ, “ಅವರಿಗೋಸ್ಕರ ನಾನು ಎಲ್ಲ ವ್ಯವಸ್ಥೆಗಳನ್ನು ಮಾಡಿದರೆ, ಕಡೇ ಘಳಿಗೆಯಲ್ಲಿ ಅವರು ಅದನ್ನೆಲ್ಲ ಅಡಿಮೇಲು ಮಾಡಿ ಬಿಡುತ್ತಾರೆ. ಅವರಿಗೆ ಅವರು ಹೇಳಿದ್ದೇ ನಡೆಯಬೇಕು–ಅವರು ಪಿಂಗಾಣಿ ಅಂಗಡಿಗೆ ನುಗ್ಗಿದ ಹುಚ್ಚುಗೂಳಿಯಂತೆ!” ಆದರೆ ಸ್ವಾಮೀಜಿ ಎಲ್ಲರ ವಿಚಾರದಲ್ಲೂ ಯಾವಾಗಲೂ ಹೀಗೆಯೇ ನಡೆದುಕೊಳ್ಳುತ್ತಿದ್ದರೆಂದಲ್ಲ. ಇತರರಿಗೆ ತಮ್ಮಿಂದ ಸಾಧ್ಯವಾದ ಎಲ್ಲ ನೆರವನ್ನೂ ಮಾಡಲು ಅವರು ಸಿದ್ಧರಾಗಿದ್ದರು, ಬದ್ಧರಾಗಿದ್ದರು. ಒಮ್ಮೆ, ಅವರ ತರಗತಿಗಳಿಗೆ ಕೆಲಕಾಲ ಬಂದಿದ್ದವ ಳೊಬ್ಬಳು ಬೇರೊಂದು ಊರಿನಲ್ಲಿ ತಾನೇ ತರಗತಿಗಳನ್ನು ನಡೆಸಲಾರಂಭಿಸಿದಳು. ಸ್ವಾಮೀಜಿ ಯೇನೂ ಅವಳಿಗೆ ಆ ಕೆಲಸವನ್ನು ಒಪ್ಪಿಸಿರಲಿಲ್ಲ, ಅಥವಾ ಅವಳು ಅವರ ಅನುಮತಿಯನ್ನೂ ಪಡೆದಿರಲಿಲ್ಲ. ಆದರೆ ಆ ಹೆಂಗಸು, ತಾನು ಸ್ವಾಮೀಜಿಯ ಪರವಾಗಿ ಈ ತರಗತಿಗಳನ್ನು ನಡೆಸು ತ್ತಿರುವುದಾಗಿ ಸಾರಿ, ಅದಕ್ಕೆ ಪ್ರವೇಶ ಶುಲ್ಕವನ್ನೂ ಇಟ್ಟಿದ್ದಳು! ಈ ವಿಷಯವನ್ನು ಶಿಷ್ಯರೊಬ್ಬರು ಸ್ವಾಮೀಜಿಯ ಗಮನಕ್ಕೆ ತಂದರು. ಸ್ವಾಮೀಜಿ ಸುಮ್ಮನೆ ಮುಗುಳ್ನಕ್ಕು “ಹೋಗಲಿ ಬಿಡಿ, ಪಾಪ!” ಎಂದರು. ಆಗ ಆ ಶಿಷ್ಯರು ಮತ್ತೆ ಹೇಳಿದರು, “ಆದರೆ ಸ್ವಾಮೀಜಿ, ಅವಳು ನಿಮ್ಮ ಹೆಸರಿನಲ್ಲಿ ಕೇವಲ ವ್ಯಾಯಾಮ ಮಾಡಿಸುತ್ತ ದುಡ್ಡು ಕೀಳುತ್ತಿದ್ದಾಳೆ!” ಸ್ವಾಮೀಜಿ ಒಂದು ನಿಮಿಷ ಗಂಭೀರ ವಾಗಿದ್ದು ಬಳಿಕ “ಶಿವ ಶಿವ!” ಎಂದುದ್ಗರಿಸಿದರು. ಅವರು ‘ಶಿವ ಶಿವ’ ಎಂದರೆ ಆ ವಿಷಯ ಅಲ್ಲಿಗೆ ಮುಗಿಯಿತೆಂದೇ ಅರ್ಥ. ಒಮ್ಮೆ ಅವರು ಒಂದು ವಿಷಯವನ್ನು ತಲೆಯಿಂದ ಕೊಡವಿ ಬಿಟ್ಟರೆಂದರೆ, ಅದು ಅವರ ತಲೆಯನ್ನು ಮತ್ತೆ ಪ್ರವೇಶಿಸುವಂತೆಯೇ ಇಲ್ಲ!

ತಮ್ಮ ಶರೀರವೇ ತಮಗೊಂದು ದೊಡ್ಡ ಬಂಧನವೆಂಬಂತೆ ಸ್ವಾಮೀಜಿಗೆ ಕೆಲವೊಮ್ಮೆ ಅನ್ನಿಸುತ್ತಿತ್ತು. “ಆಹ್, ನಾನು ಎಂದೆಂದಿಗೂ ಮರೆಯಾಗಿರಲು ಸಾಧ್ಯವಾಗಿದ್ದರೆ!” ಎಂದು ಅವರು ಯಾವಾಗಲಾದರೊಮ್ಮೆ ಉದ್ಗರಿಸುತ್ತಿದ್ದರು. ಆದರದು ಕೇವಲ ಉದ್ಗಾರವಲ್ಲ. ಅವರ ಆತ್ಮವು ಶರೀರವೆಂಬ ಪಂಜರಕ್ಕೆ ಬಲವಂತವಾಗಿ ಬಿಗಿಯಲ್ಪಟ್ಟಿದೆಯೆಂಬಂತೆ ಅವರ ಸುತ್ತಲಿದ್ದ ವರಿಗೆ ತೋರುತ್ತಿತ್ತು. ಅವರ ಈ ಭಾವವು ‘ಸಂನ್ಯಾಸಿಗೀತೆ’ ಹಾಗೂ ‘ನನ್ನ ಆಟ ಮುಗಿದಿದೆ\eng{’ (My Play is Done)} ಎಂಬ ಕವನಗಳಲ್ಲಿ ಸ್ಪಷ್ಟವಾಗಿ ಮೂಡಿಬಂದಿದೆ. ‘ನನ್ನ ಆಟ ಮುಗಿದಿದೆ’ ಎಂಬ ಪದ್ಯದ ಕೆಲವು ಸಾಲುಗಳು ಹೀಗಿವೆ:

\begin{verse}
ಮುಗಿಯದಾಟವ ಕಂಡು ಮನವಿದು ಬರಿದೆ ಬೇಸರಗೊಂಡಿದೆ\\ತೋರಿಕೆಯು ತಾನಿನ್ನು ರುಚಿಸದು; ಜೀವನದ ಬರಿಯೋಟವು\\ನೆಲೆಯನೆಂದಿಗು ತಲುಪದಿರುವುದು, ದಡದ ಸುಳಿವೇ ತೋರದು!
\end{verse}

\begin{verse}
ಇನ್ನು ವಿರಮಿಸಲಾರೆ ಜಗದೊಳು, ಪೊಳ್ಳುಗುಳ್ಳೆಗಳೆಲ್ಲವೂ-\\ಪೊಳ್ಳುನಾಮವು, ಪೊಳ್ಳು ರೂಪವು, ಪೊಳ್ಳುಜನನವು-ಮರಣವು!\\ನಾಮರೂಪವ ಮೀರಿನಡೆಯಲು ಜೀವ ಕಾತರಗೊಂಡಿದೆ\\ಬಾಗಿಲುಗಳೇ! ತೆರೆಯಿರೆನಗೆ–ತೆರೆಯಲೇಬೇಕೆಂದಿಗೆ!
\end{verse}

ಅವರ ಆ ದಿವ್ಯ ಅಸಹನೆಯು ಅವರ ಕೆಲವು ಪತ್ರಗಳಲ್ಲೂ ವ್ಯಕ್ತವಾಗುತ್ತಿತ್ತು. ಶ್ರೀಮತಿ ಸಾರಾಬುಲ್ಲಳಿಗೆ ಅವರು ಒಂದು ಪತ್ರದಲ್ಲಿ ಹೀಗೆ ಬರೆದರು:

“ಜಗತ್ತಿನಾದ್ಯಂತ ನನ್ನೊಂದಿಗೆ ಸಂಚರಿಸಿದ ಒಂದು ನೋಟ್ ಪುಸ್ತಕ ಬಳಿ ಇದೆ. ಏಳು ವರ್ಷಗಳ ಹಿಂದೆ ಅದರಲ್ಲಿ ನಾನು ಬರೆದ ಮಾತುಗಳಿವು–‘ಈಗ ನಿರ್ಜನವಾದ ಒಂದು ಮೂಲೆ ಯನ್ನು ಹುಡುಕಬೇಕು, ಮತ್ತೆ ಅಲ್ಲಿಯೇ ನಿಶ್ಚಲವಾಗಿ ಕುಳಿತು ಪ್ರಾಣವನ್ನು ತ್ಯಜಿಸಬೇಕು’ ಎಂದು! ಆದರೂ ಈ ಎಲ್ಲ ಕರ್ಮ ಇನ್ನೂ ಉಳಿದಿತ್ತು. ಈಗ ನಾನು ಅವನ್ನೆಲ್ಲ ಸವೆಸಿರುವೆನೆಂದು ನಂಬುತ್ತೇನೆ...

“ನಾನು ‘ಅದನ್ನು ಮಾಡಬೇಕು, ಇದನ್ನು ಮಾಡಬೇಕು’ ಎಂದೆಲ್ಲ ಮಕ್ಕಳಂತೆ ಕನಸು ಕಂಡದ್ದು ಈಗ ನನಗೆ ಚಿತ್ತಭ್ರಾಂತಿಯಂತೆ ತೋರುತ್ತದೆ. ಈಗ ನಾನು ಆ ಕನಸುಗಳಿಂದ ಎಚ್ಚರಗೊಳ್ಳುತ್ತಿದ್ದೇನೆ... ಬಹುಶಃ ನನ್ನನ್ನು ಈ ದೇಶಕ್ಕೆ ಎಳೆದುತರಲು ಆ ಹುಚ್ಚು ಆಸೆಗಳು ಬೇಕಾಗಿದ್ದುವೇನೋ. ಇರಲಿ, ಈ ಅನುಭವಕ್ಕಾಗಿ ನಾನು ಭಗವಂತನಿಗೆ ಕೃತಜ್ಞನಾಗಿದ್ದೇನೆ.”

ಸ್ವಾಮೀಜಿ ಇಂತಹ ಗಂಭೀರ ಭಾವಗಳಲ್ಲಿದ್ದಾಗ ಅವರ ಶಿಷ್ಯರಿಗೆ ಚಿಂತೆಯಾಗುತ್ತಿತ್ತು– ಎಲ್ಲಿ ಅವರು ಇದ್ದಕ್ಕಿದ್ದಂತೆ ತಮ್ಮ ಶರೀರವನ್ನು ತ್ಯಜಿಸಿಬಿಡುತ್ತಾರೆಯೋ ಎಂದು. ಆದ್ದರಿಂ ದಲೇ ಆ ಶಿಷ್ಯರು, ಅವರು ಯಾವಾಗಲೂ ಲಘು ಭಾವಗಳಲ್ಲಿರುವಂತೆ ನೋಡಿಕೊಳ್ಳುತ್ತಿದ್ದುದು.

ಚಳಿಗಾಲದ ಹಿಮಾವೃತ ದಿನಗಳಲ್ಲಿ ಹೊರಗೆ ಓಡಾಡಲು ಸಾಧ್ಯವಿಲ್ಲದಾಗ, ಅಗ್ಗಿಷ್ಟಿಕೆಯ ಮುಂದೆ ಆರಾಮವಾಗಿ ಕುಳಿತು, ತಮ್ಮ ಹಿಂದಿನ ವರ್ಷಗಳ ಸ್ಮೃತಿಗಳನ್ನು ಮೆಲಕುಹಾಕುವುದು ಸ್ವಾಮೀಜಿಗೆ ಅತ್ಯಂತ ಪ್ರಿಯವಾಗಿತ್ತು. ಕೆಲವೊಮ್ಮೆ ಅವರು ಯಾವುದಾದರೂ ವೃತ್ತಪತ್ರಿಕೆ ಯನ್ನು ಹಿಡಿದು ಅದನ್ನು ಆಮೂಲಾಗ್ರವಾಗಿ ಓದಿ ಮುಗಿಸುತ್ತಿದ್ದರು. ಸುದ್ದಿಯ ಶೀರ್ಷಿಕೆಯನ್ನು ಓದಿಯೇ ಆ ವರದಿಗಾರನ ಸ್ವಭಾವವೆಂತಹದೆಂದು ಹೇಳಿಬಿಡುತ್ತಿದ್ದರು. ಇಲ್ಲವೆ ‘ಪಂಚ್​’ ನಂತಹ ಯಾವುದಾದರೂ ಹಾಸ್ಯ ಪ್ರಧಾನವಾದ ಪತ್ರಿಕೆಯನ್ನು ಓದುತ್ತ ಕಣ್ಣಲ್ಲಿ ನೀರು ಸುರಿಯುವವರೆಗೂ ನಗುತ್ತಿದ್ದರು.

ಆದರೆ ಇವೆಲ್ಲ ಸ್ವಾಮೀಜಿಯ ಪಾಲಿಗೆ ಕೇವಲ ತಾತ್ಕಾಲಿಕವಾದ ಬಿಡುವು ಅಷ್ಟೆ. ಅವರ ನಿಜಸ್ವಭಾವವು ಯಾವ ಕ್ಷಣದಲ್ಲಾದರೂ ವ್ಯಕ್ತವಾಗಬಹುದಾಗಿತ್ತು. ಆಗ ಅವರು ಬೇರೆಯೇ ವ್ಯಕ್ತಿಯಾಗಿಬಿಡುತ್ತಿದ್ದರು. ಸ್ವಾಮೀಜಿ ಇಂತಹ ಲಘು ಭಾವಗಳಲ್ಲಿದ್ದಾಗ ಮಾತ್ರ ಅವರನ್ನು ನೋಡಿದ್ದ ಒಬ್ಬ ಶಿಷ್ಯ, ಅವರನ್ನು ಸರಿಯಾಗಿ ಅರ್ಥಮಾಡಿಕೊಂಡಿರಲಿಲ್ಲ. ಬಹುಶಃ ಇವರು ಯಾವಾಗಲೂ ಹೀಗೆಯೇ ತಮಾಷೆ ಮಾಡಿಕೊಂಡಿರುವವರು ಎಂದು ಅವನು ಭಾವಿಸಿರಬೇಕು. ಒಂದು ದಿನ ಸ್ವಾಮೀಜಿ ಹೀಗೆಯೇ ತಮಾಷೆ ಮಾಡುತ್ತ ನಗುತ್ತಿದ್ದಾಗ ಈ ಶಿಷ್ಯ ಸಾಂದರ್ಭಿಕವಾಗಿ ಅಧ್ಯಾತ್ಮದ ಕುರಿತಾಗಿ ಯಾವುದೋ ಪ್ರಶ್ನೆ ಹಾಕಿದ. ಆಗ ಇದ್ದಕ್ಕಿದ್ದಂತೆ ಸ್ವಾಮೀಜಿಯ ಮುಖ ಭಾವವೇ ಬದಲಾಯಿಸಿತು. ತಮಾಷೆ ಮಾಡುತ್ತ ನಗುತ್ತಿದ್ದ ಅವರ ಬಾಹ್ಯ ವ್ಯಕ್ತಿತ್ವ ಕಳಚಿಹೋಗಿ, ಅದರ ಹಿಂದಿನ ಅವರ ನಿಜ ವ್ಯಕ್ತಿತ್ವ ಬೆಳಗುತ್ತಿರುವಂತೆ ಆ ಶಿಷ್ಯನಿಗೆ ಭಾಸವಾಯಿತು. ಹೀಗೆ ಮೇಲ್ನೋಟಕ್ಕೆ ಅವರು ಬೇರೆಲ್ಲ ಜನರಂತೆಯೇ ಮಾತುಕತೆಯಾಡುತ್ತಿದ್ದಾಗಲೂ ಅವರೊಳಗೆ ಅದೇ ಆಧ್ಯಾತ್ಮಿಕ ಜ್ವಾಲೆ ಉರಿಯುತ್ತಿರುತ್ತಿತ್ತು. ಮೇಲಿನ ಆ ಪದರ ಸ್ವಲ್ಪ ಸರಿದೊಡನೆಯೇ, ಒಳಗಿನ ಜ್ವಾಲೆಯ ಬೆಳಕು, ಸುತ್ತಲಿದ್ದವರ ಕಣ್ಣು ಕೋರೈಸುವಂತೆ ಹೊರಬರುತ್ತಿತ್ತು.

ಸ್ವಾಮೀಜಿಯ ವ್ಯಕ್ತಿತ್ವವೆಂಬುದು ಸಮುದ್ರದಲ್ಲಿ ತೇಲುವ ಬಹು ದೊಡ್ಡ ನೀರ್ಗಲ್ಲಿನಂತೆ– ಅದರಲ್ಲಿ ಗೋಚರವಾಗುವ ಅಂಶಗಳಿಗಿಂತ ಗೋಚರಿಸದ ಅಂಶಗಳೇ ಹೆಚ್ಚು. ಅವರಾಡುವ ಒಂದೊಂದು ಮಾತಿನ ಮೂಲಕವೂ, ಅವರ ಸಣ್ಣಪುಟ್ಟ ಕಾರ್ಯಗಳಲ್ಲಿಯೂ ಅವರ ವ್ಯಕ್ತಿತ್ವದ ಒಂದೊಂದು ಹೊಸ ಅಂಶ ಬೆಳಕಿಗೆ ಬರುತ್ತಿತ್ತು. ಆದರೆ ಸ್ವಾಮೀಜಿಯ ವ್ಯಕ್ತಿತ್ವದ ಅಂತರಾಳದ ಎಷ್ಟೋ ಸಂಗತಿಗಳು ಜಗತ್ತಿನ ಪಾಲಿಗೆ ಮುಚ್ಚಿದ ಪುಸ್ತಕವೇ ಸರಿ. ಅವರ ಶಿಷ್ಯರಲ್ಲೊಬ್ಬರು ಹೇಳುತ್ತಾರೆ, “ಪ್ರತಿಯೊಂದು ಗಳಿಗೆಯೂ ಅವರ ವ್ಯಕ್ತಿತ್ವದಿಂದ ಒಂದು ಹೊಸ ಭಾವನೆ, ಒಂದು ಹೊಸ ಮಾಧುರ್ಯ, ಮಾನವನ ನಿಜ ಸ್ವರೂಪದ ಬಗ್ಗೆ ಒಂದು ಹೊಸ ಬೋಧನೆ ಅಥವಾ ಬೆರಗುಗೊಳಿಸುವ ಹೊಸ ಯೋಜನೆಯೊಂದು ಹೊರಹೊಮ್ಮುತ್ತಿತ್ತು.” ಇನ್ನೊಬ್ಬರು ಹೇಳು ತ್ತಾರೆ, “ಇತರರು ಅವರೊಂದಿಗೆ ಸುಮ್ಮನೆ ರಸ್ತೆಗಳಲ್ಲಿ ನಡೆದಾಡುವಾಗಲೇ ಹೊಟ್ಟೆ ಹುಣ್ಣಾಗಿ ಸುವ ತಮಾಷೆಯಿಂದ ಇದ್ದಕ್ಕಿದ್ದಂತೆ ಅತ್ಯದ್ಭುತ ವಿಚಾರಗಳ ಬೇರೊಂದು ಜಗತ್ತಿಗೆ ಏರಿಬಿಡ ಬಹುದಾಗಿತ್ತು.” ಅವರ ಮತ್ತೊಬ್ಬ ಶಿಷ್ಯರು ಹೇಳುತ್ತಾರೆ, “ಶೋಭಾಯಮಾನವಾದ ಅವರ ಬಾಹ್ಯ ನೋಟವೇ ಪ್ರತಿಯೊಬ್ಬರ ಗಮನವನ್ನೂ ತೀವ್ರವಾಗಿ ಆಕರ್ಷಿಸುವಂಥದಾಗಿತ್ತು. ಹಾಗಿ ದ್ದರೂ ಕೂಡ, ತಾವು ಆತ್ಮವೇ ಹೊರತು ಶರೀರವಲ್ಲ ಎಂಬ ಭಾವನೆಯನ್ನು ಅವರು ಇತರರಲ್ಲಿ ಯಾವಾಗಲೂ ಮೂಡಿಸುತ್ತಿದ್ದರು.” ಸ್ವಾಮೀಜಿಯ ಘನ ಗಂಭೀರ ವ್ಯಕ್ತಿತ್ವದ ಬಗ್ಗೆ ಶ್ರೀಮತಿ ಲೆಗೆಟ್ ಹೇಳುತ್ತಾಳೆ, “ತಮ್ಮ ಘನತೆಯನ್ನು ಕಿಂಚಿತ್ತೂ ಕಳೆದುಕೊಳ್ಳದೆ, ಇತರರು ತಮ್ಮ ಸಮ್ಮುಖದಲ್ಲಿ ಅತ್ಯಂತ ಸಹಜಭಾವದಿಂದ ಇರುವಂತೆ ಮಾಡಬಲ್ಲ ಇಬ್ಬರೇ ಇಬ್ಬರು ಮಹಾ ವ್ಯಕ್ತಿಗಳನ್ನು ನಾನು ನನ್ನ ಇಡೀ ಜೀವನದಲ್ಲಿ ಭೇಟಿಯಾಗಿದ್ದೇನೆ; ಅವರಲ್ಲಿ ಒಬ್ಬರು ಜರ್ಮನ್ ಚಕ್ರವರ್ತಿ, ಇನ್ನೊಬ್ಬರು ಸ್ವಾಮಿ ವಿವೇಕಾನಂದರು.”

ಸ್ವಾಮೀಜಿ ತಮ್ಮೊಳಗಿದ್ದ ಪ್ರಚಂಡ ಶಕ್ತಿಯು ಹೊರಹೊಮ್ಮದಂತೆ ಯಾವಾಗಲೂ ಪ್ರಯತ್ನ ಪೂರ್ವಕವಾಗಿ ಹಿಡಿದಿಟ್ಟುಕೊಳ್ಳುತ್ತಿದ್ದರು. ಆದರೆ ಯಾವಾಗಲಾದರೊಮ್ಮೆ ಅವರ ಆ ಶಕ್ತಿಯು ಹೊರಚಿಮ್ಮಿದರೆ ಎದುರಿಗಿದ್ದವರು ಸ್ತಂಭೀಭೂತರಾಗುತ್ತಿದ್ದರು. ಅವರನ್ನು ಇಂತಹ ಭಾವ ಗಳಲ್ಲಿ ಕಂಡಿದ್ದ ಶಿಷ್ಯನೊಬ್ಬ ಹೇಳುತ್ತಾನೆ–

“ಸ್ವಾಮೀಜಿಯ ಸಾನ್ನಿಧ್ಯವು ಉಂಟುಮಾಡಬಹುದಾಗಿದ್ದ ಅಪ್ರತಿಹತ ಪ್ರಭಾವವನ್ನು ಬಣ್ಣಿಸು ವುದು ಅಸಾಧ್ಯ. ಅವರು ಎಲ್ಲರ ಗಮನವನ್ನೂ ತಮ್ಮಲ್ಲೇ ಸೆರೆಹಿಡಿದಿಟ್ಟುಕೊಳ್ಳಬಲ್ಲವರಾಗಿ ದ್ದರು. ಇದು ನಂಬಲಸಾಧ್ಯವೆಂಬಂತೆ ತೋರಬಹುದು; ಆದರೆ ಅವರು ಪೂರ್ಣ ಗಂಭೀರಭಾವ ದಿಂದ ಮಾತನಾಡಿದರೆ ಅವರ ಶ್ರೋತೃಗಳಲ್ಲಿ ಕೆಲವರು ಅಕ್ಷರಶಃ ಸುಸ್ತಾಗಿಬಿಡುತ್ತಿದ್ದರು! ಒಮ್ಮೆಯಂತೂ ಒಬ್ಬ ವ್ಯಕ್ತಿ ಸ್ವಾಮೀಜಿಯೊಡನೆ ನಡೆಸಿದ ಚರ್ಚೆಯ ಪರಿಣಾಮವಾಗಿ ಆತ ಮೂರು ದಿನ ಹಾಸಿಗೆ ಹಿಡಿಯಬೇಕಾಗಿ ಬಂದದ್ದು ನನಗೆ ಗೊತ್ತಿದೆ. ಸ್ವಾಮೀಜಿಯ ವ್ಯಕ್ತಿತ್ವವು ಏಕಕಾಲಕ್ಕೆ ಬೆರಗುಗೊಳಿಸುವಂಥದೂ ಮಹಿಮೆಯುಳ್ಳದ್ದೂ ಆಗಿತ್ತು. ತಾವು ಇಚ್ಛಿಸಿದರೆ ವ್ಯಕ್ತಿ ಯೊಬ್ಬನ ಅಹಮಿಕೆಯನ್ನು ನಿರ್ನಾಮ ಮಾಡಬಲ್ಲ ಸಾಮರ್ಥ್ಯ ಅವರಿಗಿತ್ತು.”

ಸ್ವಾಮೀಜಿ ಅಮೆರಿಕದಲ್ಲಿ ಕಾರ್ಯನಿರತರಾಗಿದ್ದಾಗ ಅವರ ಮನಸ್ಸು ವಿಧವಿಧದ ಕಾರ್ಯ ಯೋಜನೆಗಳ ಬಗ್ಗೆ ಆಲೋಚಿಸುತ್ತಿತ್ತು. ವಿಶಾಲವಾದ ಶ್ರೋತೃವರ್ಗವನ್ನು ಸಂಪಾದಿಸಿಕೊಂಡು ತಮ್ಮ ಕಾರ್ಯಕ್ಕೆ ವಿದ್ವಜ್ಜನರ ಬೆಂಬಲವನ್ನು ಗಳಿಸಲು ಸಮರ್ಥರಾದಾಗಿನಿಂದಲೂ “ವಿಶ್ವಾತ್ಮಕ ದೇವಾಲಯ”ವೊಂದನ್ನು \eng{(Temple Universal)} ಪ್ರತಿಷ್ಠಾಪಿಸುವ ಉದ್ದೇಶ ಅವರಿಗಿತ್ತು. ಆ ದೇವಾಲಯದಲ್ಲಿ ಅಖಂಡ ಸಚ್ಚಿದಾನಂದದ ಸಂಕೇತವಾದ ಓಂಕಾರವನ್ನು ಮಾತ್ರ ಆರಾಧಿಸುತ್ತ, ಜಗತ್ತಿನ ಸಮಸ್ತ ಮತ-ಪಂಥಗಳ ಜನರು ಸಾಮರಸ್ಯದಿಂದ ಸೇರುವಂತಾಗಬೇಕು ಎಂಬುದು ಸ್ವಾಮೀಜಿಯ ಇಚ್ಛೆಯಾಗಿತ್ತು. ಆದರೆ ಅವರ ಪ್ರಥಮ ಯೋಜನೆಯಾದ ವೇದಾಂತ ಪ್ರಸಾರದ ಕಾರ್ಯವೇ ಅವರ ಸಮಯವನ್ನೆಲ್ಲ ತೆಗೆದುಕೊಳ್ಳುತ್ತಿದ್ದುದರಿಂದ ಅದನ್ನು ಬಿಟ್ಟು ಬೇರೊಂದು ಕೆಲಸವನ್ನು ಕೈಗೆತ್ತಿಕೊಳ್ಳಲು ಅವರಿಗೆ ಸಾಧ್ಯವಾಗಲಿಲ್ಲ.

ಬಾಸ್ಟನ್ನಿಗೆ ಭೇಟಿ ನೀಡಿದ್ದ ಸಂದರ್ಭದಲ್ಲಿ ಸ್ವಾಮೀಜಿ, ಹಾರ್ವರ್ಡ್ ವಿಶ್ವವಿದ್ಯಾನಿಲಯದ ಸುಪ್ರತಿಷ್ಠಿತ ಪ್ರೊಫೆಸರರಾದ ವಿಲಿಯಂ ಜೇಮ್ಸರನ್ನು ಭೇಟಿಯಾದರು. ಶ್ರೀಮತಿ ಸಾರಾ ಬುಲ್ಲಳ ಮನೆಯಲ್ಲಿ ಈ ಭೇಟಿ ಏರ್ಪಾಡಾಗಿತ್ತು. ಸಂಜೆಯ ಔತಣದ ನಂತರ ಸ್ವಾಮೀಜಿ ಹಾಗೂ ಪ್ರೊ॥ ಜೇಮ್ಸ್ ಪರಸ್ಪರ ಅನ್ಯೋನ್ಯತೆಯಿಂದ, ವಿಶ್ವಾಸಯುತ ಸಂಭಾಷಣೆಯಲ್ಲಿ ತೊಡಗಿದರು. ಅವರು ಸಂಭಾಷಣೆಯನ್ನು ಮುಗಿಸಿ ಎದ್ದಾಗ ಮಧ್ಯರಾತ್ರಿಯಾಗಿತ್ತು. ಈಗ ಶ್ರೀಮತಿ ಬುಲ್ಲಳಿಗೆ ಒಂದು ಕುತೂಹಲ–ಸ್ವತಃ ಅಸಾಧಾರಣ ಪ್ರತಿಭಾವಂತರಾದ ಸ್ವಾಮೀಜಿಯನ್ನು ವಿದ್ವನ್ಮಣಿಯಾದ ವಿಲಿಯಂ ಜೇಮ್ಸರು ಸಂಧಿಸಿದಾಗ ಅವರಿಗೆ ಪರಸ್ಪರರ ವಿಷಯದಲ್ಲಿ ಏನೆನ್ನಿಸಿರಬಹುದು? ಅವರಿಬ್ಬರ ನಡುವೆ ಎಂತಹ ಸಂಬಂಧವೇರ್ಪಟ್ಟಿರಬಹುದು? ಎಂದು. ಆದ್ದರಿಂದ ಅನಂತರ ಆಕೆ, “ಸ್ವಾಮೀಜಿ, ಪ್ರೊ ॥ ಜೇಮ್ಸರ ವಿಷಯದಲ್ಲಿ ನಿಮಗೇನೆನ್ನಿಸಿತು?” ಎಂದು ಕೇಳಿದಳು. ಅದಕ್ಕೆ ಸ್ವಾಮೀಜಿ, “ಬಹಳ ಒಳ್ಳೆಯ ಮನುಷ್ಯ, ಬಹಳ ಒಳ್ಳೆಯ ಮನುಷ್ಯ” ಎಂದು ಉತ್ತರಿಸಿ ದರು. ಹಾಗೆ ಹೇಳುವಾಗ ‘ಒಳ್ಳೆಯ’ ಎಂಬ ಪದವನ್ನು ಒತ್ತಿ ಹೇಳಿದರು; ಅವರ ಮಾತಿನಲ್ಲಿ ಏನೋ ಗೂಢಾರ್ಥವಿದ್ದಂತಿತ್ತು. ಮರುದಿನ ಸ್ವಾಮೀಜಿ ಒಂದು ಪತ್ರವನ್ನು ಶ್ರೀಮತಿ ಬುಲ್ಲಳಿಗೆ ಕೊಡುತ್ತ “ಬೇಕಾದರೆ ಇದನ್ನು ಓದಿ ನೋಡು” ಎಂದರು. ಸಾರಾ ಬುಲ್ ಪತ್ರವನ್ನು ಓದಿದಳು– ಅವಳ ಆಶ್ಚರ್ಯಕ್ಕೆ ಪಾರವೇ ಇಲ್ಲ. ಆ ಪತ್ರದಲ್ಲಿ ವಿಲಿಯಂ ಜೇಮ್ಸರು ಸ್ವಾಮೀಜಿಯನ್ನು ತಮ್ಮ ಮನೆಗೆ ಭೋಜನಕ್ಕೆ ಆಹ್ವಾನಿಸಿದ್ದರಲ್ಲದೆ, ಅವರನ್ನು ‘ಗುರು’ ಎಂದು ಸಂಬೋಧಿಸಿದ್ದರು! ಪ್ರಥಮ ಭೇಟಿಯಲ್ಲೆ ಆ ಪ್ರತಿಷ್ಠಿತ ವಿದ್ವಾಂಸರ ಮೇಲೆ ಸ್ವಾಮೀಜಿ ಬೀರಿದ್ದ ಪ್ರಭಾವ ಅಷ್ಟು ಗಾಢವಾಗಿತ್ತು. ಪ್ರೊ ॥ ಜೇಮ್ಸರು ಸ್ವಾಮೀಜಿಯ ಮೇಲಿಟ್ಟಿದ್ದ ಗೌರವಭಾವವು ಮುಂದೆ ಅವರು ಬರೆದ ಹಲವಾರು ಕೃತಿಗಳಲ್ಲಿ ವ್ಯಕ್ತವಾಗಿದೆ. \eng{‘Pragmatism’} ಎಂಬ ತಮ್ಮ ಗ್ರಂಥದಲ್ಲಿ ಇವರು ವಿವೇಕಾನಂದರನ್ನು ‘ವೇದಾಂತಿಕುಲದ ಅತ್ಯುತ್ಕೃಷ್ಟ ವ್ಯಕ್ತಿ’ ಎಂದಿದ್ದಾರೆ. \eng{‘Varieties of Religious Experiences’} ಎನ್ನುವ ಗ್ರಂಥದಲ್ಲಿ ಸ್ವಾಮೀಜಿಯ ‘ಜ್ಞಾನಯೋಗ’ ಹಾಗೂ ‘ರಾಜ ಯೋಗ’ ಗ್ರಂಥಗಳಿಂದ ಸಾಕಷ್ಟು ಉಲೇಖಿಸಿದ್ದಾರೆ. ಅಲ್ಲದೆ ತಮ್ಮ ಸುಪ್ರಸಿದ್ಧ ಪ್ರಬಂಧವಾದ \eng{‘The Energies of Man’} ಎಂಬುದರಲ್ಲಿ, ತಮ್ಮ ನರದೌರ್ಬಲ್ಯದ ನಿವಾರಣೆಗಾಗಿ ರಾಜಯೋಗ ವನ್ನು ಅಭ್ಯಾಸ ಮಾಡಿದ ಪ್ರೊಫೆಸರರೊಬ್ಬರ ವಿಷಯವನ್ನು ತಿಳಿಸುತ್ತಾರೆ. ತತ್ಪರಿಣಾಮವಾಗಿ ಆ ಪ್ರೊಫೆಸರರಿಗೆ ನರಮಂಡಲದ ದೃಢತೆಯುಂಟಾಯಿತಷ್ಟೇ ಅಲ್ಲದೆ ಬೌದ್ಧಿಕವಾಗಿಯೂ ಆಧ್ಯಾತ್ಮಿಕವಾಗಿಯೂ ಅವರಿಗೆ ಬಹಳ ಪ್ರಯೋಜನವಾಯಿತು ಎಂಬ ವಿಷಯವನ್ನೂ ಅದರಲ್ಲಿ ವಿವರಿಸುತ್ತಾರೆ. ಗ್ರಂಥದಲ್ಲಿ ಹೇಳಲಾಗಿರುವ ಆ ಪ್ರೊಫೆಸರರು ಬೇರೆ ಯಾರೂ ಅಲ್ಲ–ಸ್ವತಃ ವಿಲಿಯಂ ಜೇಮ್ಸರೇ ಎಂಬುದು ಅನೇಕರ ನಿಶ್ಚಿತ ಅಭಿಪ್ರಾಯವಾಗಿತ್ತು.

ಹೀಗೆ, ಅಮೆರಿಕದಲ್ಲಿ ಸ್ವಾಮೀಜಿ ಸುಮಾರು ಎರಡೂವರೆ ವರ್ಷಗಳ ಕಾಲ ಮಹಾಶಕ್ತಿಶಾಲಿ ಯಾದ ಪ್ರಜ್ವಲಿಸುವ ಜ್ಯೋತಿಯಂತೆ ಬೆಳಗಿದರು. ಅವರು ಮಹಾಗುರುವಿನಂತೆ, ನೂತನ ಪ್ರವಾದಿಯಂತೆ, ಮಹಿಮಾನ್ವಿತವಾದ ಓರ್ವ ಬೋಧಿಸತ್ವನಂತೆ ತಮ್ಮ ಶಿಷ್ಯರ ನಡುವೆ ಸಂಚರಿಸಿ ದರು. ಕೆಲವರು ಅವರನ್ನು ಉಪನಿಷತ್ಕಾಲದ ಮಹರ್ಷಿಯಂತೆ ಕಂಡರು; ಇನ್ನು ಕೆಲವರು ಅವರನ್ನು ಮತ್ತೊಬ್ಬ ಶಂಕರಾಚಾರ್ಯರಂತೆ ಕಂಡರು; ಮತ್ತೆ ಕೆಲವರು ಅವರನ್ನು ಒಬ್ಬ ಬುದ್ಧ ನಂತೆ ಅಥವಾ ಒಬ್ಬ ಕ್ರಿಸ್ತನಂತೆ ಕಂಡರು. ಮತ್ತು ಸರ್ವರೂ ಅವರನ್ನು ಅತ್ಯುನ್ನತ ಆಧ್ಯಾತ್ಮಿಕ ಪ್ರಜ್ಞೆಯ ಮೂರ್ತರೂಪವೆಂಬಂತೆ ಕಂಡರು. ಬಹುಶಃ ಸಾವಿರಾರು ಧರ್ಮಪ್ರಚಾರಕರು ನೂರಾರು ವರ್ಷಗಳಲ್ಲಿ ಸಾಧಿಸಲು ಸಾಧ್ಯವಿಲ್ಲದುದನ್ನು ಸ್ವಾಮೀಜಿ ಏಕಾಂಗಿಯಾಗಿ ಕೇವಲ ಎರಡೂವರೆ ವರ್ಷಗಳಲ್ಲಿ ಸಾಧಿಸಿ, ಅಮೆರಿಕದ ತಮ್ಮ ಪ್ರಥಮ ಹಾಗೂ ಅತಿಮುಖ್ಯ ಭೇಟಿ ಯನ್ನು ಮುಕ್ತಾಯಗೊಳಿಸಿ ಇಂಗ್ಲೆಂಡಿಗೆ–ಬಳಿಕ ತಮ್ಮ ತಾಯ್ನಾಡಿಗೆ–ಹೊರಟರು.

