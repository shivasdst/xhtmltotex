
\chapter{ಜೈತ್ರಯಾತ್ರೆಗೆ ಸನ್ನಾಹ}

\noindent

೧೮೯೩ರ ಜನವರಿ; ದಂಡ ಕಮಂಡಲುಧಾರಿಯಾದ ಸ್ವಾಮಿ ವಿವೇಕಾನಂದರು ರಾಮೇಶ್ವರ ದಿಂದ ಹೊರಟು ಉತ್ತರಾಭಿಮುಖವಾಗಿ ನಡೆದರು. ಇದು ಅವರ ಜೀವನದ ಮೂವತ್ತೊಂದನೇ ಸಂವತ್ಸರ. ಅವರ ಜೀವನದಲ್ಲಿ–ಅಷ್ಟೇಕೆ, ಆಧುನಿಕ ಭಾರತದ ಇತಿಹಾಸದಲ್ಲೇ–ಅತ್ಯಂತ ಮಹತ್ತ್ವದ ಸಂವತ್ಸರ. ವಿವೇಕಾನಂದರು ವಿಶ್ವವಿಜೇತರಾದುದು, ವಿಶ್ವವಿಖ್ಯಾತರಾದುದು ಈ ವರ್ಷದಲ್ಲೇ. ಆದರೆ ಅಂದು ಅವರಿನ್ನೂ ಭಾರತದ ಲಕ್ಷಾಂತರ ಸಾಧು ಬೈರಾಗಿಗಳಲ್ಲೊಬ್ಬರಷ್ಟೆ. ಕಾಲುಗಳು ನಡೆಸಿದತ್ತ ನಡೆದು ಸ್ವಾಮೀಜಿ ಪಾಂಡಿಚೆರಿಗೆ ಬಂದರು.

ಪಾಂಡಿಚೆರಿಯ ಸಮುದ್ರತೀರದಲ್ಲಿ ಅಡ್ಡಾಡುತ್ತಿದ್ದಾಗ ಅಕಸ್ಮಾತ್ತಾಗಿ ತಮ್ಮ ಹಳೆಯ ಪರಿಚಿತರಾದ ಮನ್ಮಥನಾಥ ಭಟ್ಟಾಚಾರ್ಯರನ್ನು ಭೇಟಿಯಾದರು. ಇವರು ತಮ್ಮ ಕೆಲಸದ ನಿಮಿತ್ತವಾಗಿ ಇಲ್ಲಿಗೆ ಬಂದಿದ್ದರು. ಸ್ವಾಮೀಜಿಯನ್ನು ಕಂಡು ಆನಂದಿತರಾದ ಮನ್ಮಥನಾಥರು ಅವರನ್ನು ತಮ್ಮ ಮನೆಗೆ ಕರೆದೊಯ್ದರು. ಅಲ್ಲದೆ ತಾವು ಮದ್ರಾಸಿಗೆ ಹಿಂದಿರುಗುವಾಗ ಅವರೂ ತಮ್ಮೊಡನೆ ಖಂಡಿತ ಬರಬೇಕು ಎಂದು ಒತ್ತಾಯಿಸಿ ಒಪ್ಪಿಸಿದರು.

ಸ್ವಾಮೀಜಿ ಪಾಂಡಿಚೆರಿಯಲ್ಲಿದ್ದಾಗ, ಕಡು ಮತಾಂಧನೂ ಅತಿ ಹಟಮಾರಿಯೂ ಆದ ಸಂಪ್ರದಾಯಸ್ಥ ಪಂಡಿತನೊಬ್ಬನೊಂದಿಗೆ ಹಿಂದೂಧರ್ಮದ ಸುಧಾರಣೆಯ ಬಗ್ಗೆ ಚರ್ಚಿಸುವ ಸಂದರ್ಭವೊದಗಿತ್ತು. ಆಧುನಿಕ ವಿಶ್ವದ ಬದಲಾದ ಪರಿಸರದಲ್ಲಿ ಹಿಂದೂ ಧರ್ಮವು ಉಳಿದು ಬೆಳೆಯಬೇಕಾದರೆ ಹೇಗೆ ಅದು ಜಳ್ಳನ್ನು ತೂರಿ ಸಾರವನ್ನು ಮಾತ್ರ ಉಳಿಸಿಕೊಳ್ಳಬೇಕಾಗುತ್ತದೆ ಎಂಬ ತಮ್ಮ ವಿಚಾರಧಾರೆಯನ್ನು ಮಂಡಿಸಿದರು ಸ್ವಾಮೀಜಿ. ಆದರೆ ಸಂಪ್ರದಾಯಗಳನ್ನು ಸುಧಾರಣೆಗೊಳಿಸುವ ಪ್ರಸ್ತಾಪ ಬಂದಾಗ ಈ ಪಂಡಿತ ಹೌಹಾರಿದ. ಅದೂ ಒಬ್ಬ ಹಿಂದೂ ಸಂನ್ಯಾಸಿಯಿಂದ! ನಮ್ಮ ತಾತಂದಿರೂ, ತಾತಂದಿರ ತಾತಂದಿರೂ ಪಾಲಿಸಿಕೊಂಡು ಬಂದ ಸಂಪ್ರದಾಯದಲ್ಲಿ ತಪ್ಪು ಕಂಡುಹಿಡಿಯುವುದಕ್ಕಿಂತ ಮಹಾಪಾಪ ಇನ್ನೊಂದಿದೆಯೆ?! ಸುಧಾರಣೆಯ ಮಾತನಾಡುವವರೆಲ್ಲ ಧರ್ಮದ್ರೋಹಿಗಳು ಎಂಬುದು ಈ ಪಂಡಿತನ ನಿಶ್ಚಿತ ಅಭಿಮತ. ಸ್ವಾಮೀಜಿ ಆಡಿದ ಪ್ರತಿಯೊಂದು ಮಾತನ್ನೂ ಉಗ್ರವಾಗಿ ಖಂಡಿಸಿದ. ಅವನಿಗೆ ವಿಚಾರ ಮಂಥನ ಮಾಡುವುದಕ್ಕಿಂತ ಜಗಳವಾಡುವುದರಲ್ಲೇ ಹೆಚ್ಚಿನ ಆಸಕ್ತಿ. ಪರಿಣತಿ, ಸ್ವಾಮೀಜಿಯ ವಿಚಾರದ ತಿರುಳು ಹೀಗಿತ್ತು: “ಈಗ ಹಿಂದೂಧರ್ಮವು ತನ್ನನ್ನೇ ತಾನು ಪರೀಕ್ಷಿಸಿ ಕೊಳ್ಳಬೇಕಾದ ಕಾಲ ಸನ್ನಿಹಿತವಾಗಿದೆ. ಪಾಶ್ಚಾತ್ಯ ಸಂಸ್ಕೃತಿಯ ವೈಭವ-ಮೌಲ್ಯಗಳೊಂದಿಗೆ ಅದು ಸೆಣಸಿ ತನ್ನತನವನ್ನು ಮೆರಸಬೇಕಾಗಿದೆ. ಸನಾತನ ಧರ್ಮವು ತನ್ನ ಸಾರಸರ್ವಸ್ವವಾದ ಆಧ್ಯಾತ್ಮಿಕತೆಯನ್ನು ಉಳಿಸಿಕೊಂಡು ಆಧುನಿಕ ಯುಗದ ಆವಶ್ಯಕತೆಗಳಿಗೆ, ಸಮಸ್ಯೆಗಳಿಗೆ ಹೊಂದಿ ಕೊಳ್ಳಬೇಕಾಗಿದೆ.” ಆದರೆ ಆ ಪಂಡಿತ ಸ್ವಾಮೀಜಿಯ ಅಭಿಪ್ರಾಯಗಳನ್ನು ತಳ್ಳಿಹಾಕಿ ಒಂದೇ ಹಟ ಹಿಡಿದು ಕುಳಿತ. ಅವನ ವಾದ ಇಷ್ಟೆ: “ಹಿಂದೂಧರ್ಮಕ್ಕೆ ಯಾವ ಸುಧಾರಣೆಯ ಆವಶ್ಯ ಕತೆಯೂ ಇಲ್ಲ. ಪಾಶ್ಚಾತ್ಯರೆಂದರೆ ಮ್ಲೇಚ್ಛರು. ಆ ಪರಂಗಿ ಜನರ ಸಂಪರ್ಕಕ್ಕೆ ಬಂದ್ದದೇ ಆದರೆ ಹಿಂದೂಗಳ ಬುದ್ಧಿಯೂ ವಿಕೃತಗೊಳ್ಳುತ್ತದೆ; ಸರ್ವನಾಶವಾಗುತ್ತದೆ.” ಪಂಡಿತ ಮಹಾಶಯ ತಾನು ಹುಟ್ಟಿದ ಬಾವಿಯಿಂದಾಚೆ ಇಣಿಕಿ ನೋಡಿಲ್ಲವೆಂಬುದು ಸ್ಪಷ್ಟವಾಗಿತ್ತು. ಸಂಪ್ರದಾಯಸ್ಥ ರೆನ್ನಿಸಿಕೊಂಡವರ ಇಂತಹ ಸಂಕುಚಿತ ಬುದ್ಧಿಯೇ ದೇಶವನ್ನು ಈ ದುರವಸ್ಥೆಗೆ ತಂದಿರುವುದು ಎನ್ನಿಸಿತು ಸ್ವಾಮೀಜಿಗೆ. ಅದನ್ನೇ ಸ್ಪಷ್ಟವಾಗಿ ಹೇಳಿಯೂ ಬಿಟ್ಟರು. ಬರಬರುತ್ತ ಚರ್ಚೆಗೆ ಕಾವೇರಿತು. ಕತ್ತಲೆಯ ಗುಹೆಯಲ್ಲಿ ಕುಳಿತು ಕಣ್ಣುಗಳನ್ನು ಬಿಗಿಯಾಗಿ ಮುಚ್ಚಿಕೊಂಡು “ಇಲ್ಲೇ ನನಗೆ ಬೆಳಕು ಕಾಣುತ್ತಿದೆ” ಎನ್ನುವ ಮೊಂಡುತನವನ್ನು ಕಂಡು ಅವರಿಗೆ ಜುಗುಪ್ಸೆಯಾಯಿತು. ಕಡೆಗೆ ಉದ್ವೇಗಗೊಂಡು ಹೇಳಿದರು, “ಏನು, ಏನು ನೀವು ಹೇಳುತ್ತಿರುವುದು? ಶೂದ್ರಾದಿಗಳು ಮೇಲೆದ್ದು ನಿಂತು ತಮ್ಮ ಹಕ್ಕುಗಳನ್ನು ಚಲಾಯಿಸುವ ಕಾಲ ಈಗ ಸನ್ನಿಹಿತವಾಗಿದೆ. ದಲಿತವರ್ಗ ದವರಿಗೆ ಶಿಕ್ಷಣ ನೀಡಿ, ಅವರಲ್ಲಿ ಸಮಾನತೆಯ ಭಾವನೆಯನ್ನುಂಟುಮಾಡುವುದು ಮತ್ತು ಪುರೋಹಿತಶಾಹಿಯ ದಬ್ಬಾಳಿಕೆಯನ್ನೂ ಜಾತಿಪದ್ಧತಿಯನ್ನೂ ನಿರ್ಮೂಲ ಮಾಡಿ, ಉನ್ನತ ಧಾರ್ಮಿಕ ವಿಚಾರಗಳ ಬಗ್ಗೆ ತಿಳಿವಳಿಕೆ ನೀಡುವುದು–ಇದು ಪ್ರತಿಯೊಬ್ಬ ವಿದ್ಯಾವಂತ ಭಾರತೀ ಯನ ಮೇಲಿರುವ ಹೊಣೆ.” ಇಂಥ ಸಂಪ್ರದಾಯ ವಿರುದ್ಧವಾದ ಮಾತುಗಳನ್ನೆಲ್ಲ ಕೇಳಿ ಸಹಿಸಿ ಕೊಂಡಿರಲು ಪಂಡಿತನಿಗೆ ಸಾಧ್ಯವಾಗಲಿಲ್ಲ. ಸ್ವಾಮೀಜಿ ಮಾತನಾಡುತ್ತಿದ್ದಾಗ ಅವರಿಗೆ ಅವಮಾನ ವಾಗುವಂತೆ ಅಂಗಭಂಗಿಗಳನ್ನು ಮಾಡುತ್ತಿದ್ದ. ಕಡೆಗೆ ಇತರರತ್ತ ತಿರುಗಿ ತಿರಸ್ಕಾರದಿಂದ, “ಒಬ್ಬ ಅಲೆಮಾರಿ ತಿರುಕನಿಗೇನು ಗೊತ್ತಿದ್ದೀತು? ಎಳೆನಿಂಬೆಕಾಯಿ!” ಎಂದು ಹೇಳಿದ. ಅಲ್ಲದೆ ಆಗಾಗ “ಕದಾಪಿ ನ, ಕದಾಪಿ ನ” “ಎಂದಿಗೂ ಇಲ್ಲ, ಎಂದಿಗೂ ಇಲ್ಲ” ಎಂದು ಕೂಗುತ್ತಿದ್ದ.

ಆ ಪಂಡಿತ, ‘ಸುಧಾರಣೆ’ ‘ಬದಲಾವಣೆ’ ಎಂಬ ಮಾತುಗಳಿಗೇ ವಿರುದ್ಧವಾಗಿದ್ದುದರಲ್ಲಿ ಆಶ್ಚರ್ಯವೇನೂ ಇರಲಿಲ್ಲ. ಏಕೆಂದರೆ ಸ್ವತಃ ಅವನೇ ಸಾಮಾಜಿಕ ಕಂದಾಚಾರಗಳಲ್ಲಿ ಸಿಕ್ಕಿಬಿದ್ದ ವನು; ಜಾತಿ-ಮತ ಭೇದಗಳ ಸಂಕೋಲೆಯಲ್ಲಿ ಕಟ್ಟುಬಿದ್ದವನು. ಅವನು ತನ್ನಂತಹ ಸಾವಿರಾರು ಜನರನ್ನು ಪ್ರತಿನಿಧಿಸುತ್ತಿದ್ದನಷ್ಟೆ. ಆದರೆ ಅವನೆಷ್ಟೇ ಪ್ರತಿಭಟಿಸಿದರೂ, ಹೀನಾಯವಾಗಿ ಮಾತ ನಾಡಿದರೂ ಸ್ವಾಮೀಜಿ ಮಾತ್ರ ಅವನನ್ನು ಅಷ್ಟಕ್ಕೇ ಬಿಡಲಿಲ್ಲ. ತಮ್ಮ ವಾದಕ್ಕೆ ಶಾಸ್ತ್ರವಾಕ್ಯಗಳ ಬೆಂಬಲವಿದೆ ಎಂಬುದನ್ನು ಸಂದೇಹಕ್ಕೆಡೆಯಿಲ್ಲದಂತೆ ಸಾಬೀತು ಮಾಡಿ ತೋರಿಸಿದರು. ಜಾತೀಯತೆಯೇ ಮೊದಲಾದ ಸಾಮಾಜಿಕ ಪದ್ಧತಿಗಳು ಹಿಂದೂಧರ್ಮದ ಅವಿಭಾಜ್ಯ ಅಂಗ ಗಳಂತೆ ತೋರಿದರೂ ಅಂತಹ ಅಡ್ಡಗೋಡೆಗಳ ನಿರ್ಮೂಲನವನ್ನು ಸ್ವತಃ ಹಿಂದೂ ಶಾಸ್ತ್ರಗ್ರಂಥ ಗಳೇ ಅನುಮೋದಿಸುತ್ತವೆ ಎಂಬುದನ್ನು ತೋರಿಸಿಕೊಟ್ಟರು. ಕೊಟ್ಟ ಕೊನೆಯಲ್ಲಿ ಅವನು ಸ್ವಾಮೀಜಿಯ ವಾದದ ಸತ್ಯತೆಯನ್ನು ಒಪ್ಪಿಕೊಳ್ಳಬೇಕಾಯಿತು. ಆದರೆ ಹಿಂದೂಗಳಿಂದ ಮ್ಲೇಚ್ಛ ರನ್ನು ಪ್ರತ್ಯೇಕಿಸುವ ಗುರುತಾದ ‘ಕಪ್ಪು ನೀರನ್ನು’ ಎಂದರೆ ಸಾಗರವನ್ನು ಮಾತ್ರ ಸಂನ್ಯಾಸಿಗಳು ದಾಟಬಾರದು ಎಂಬ ಹಟವನ್ನು ಬಿಡಲೇ ಇಲ್ಲ. ಎಲ್ಲಕ್ಕಿಂತ ಹೆಚ್ಚಾಗಿ, ಈ ಎಳೆಯ ಯುವಕನು ತನ್ನಂತಹ ‘ಪ್ರಬುದ್ಧ’, ವಯೋವೃದ್ಧ, ಜ್ಞಾನವೃದ್ಧನ ವಿದ್ವತ್ತು, ನಂಬಿಕೆಗಳನ್ನು ಪ್ರಶ್ನಿಸು ವಂತಾಯಿತಲ್ಲ ಎನ್ನುವುದೇ ಆ ಪಂಡಿತನ ಸಂಕಟವಾಗಿತ್ತು.

ಪಾಂಡಿಚೆರಿಯಿಂದ ಹೊರಡುವುದಕ್ಕೆ ಕೆಲದಿನಗಳ ಮುಂಚೆ ಭಟ್ಟಾಚಾರ್ಯರು ಮೈಸೂರಿನ ತಮ್ಮೊಬ್ಬ ಸ್ನೇಹಿತನಿಗೆ ಬರೆದ ಪತ್ರದಲ್ಲಿ, ಸ್ವಾಮೀಜಿ ತನ್ನೊಂದಿಗಿರುವುದಾಗಿಯೂ, ತಾವಿಬ್ಬರೂ ಮದರಾಸಿಗೆ ಹೋಗುತ್ತಿರುವುದಾಗಿಯೂ ತಿಳಿಸಿದ್ದರು. ಈ ಸ್ನೇಹಿತ ಸ್ವಾಮೀಜಿಯನ್ನು ಕಂಡು ಪ್ರಭಾವಿತನಾಗಿದ್ದವನು. ಅವನು ಮದರಾಸಿನ ತನ್ನ ಸ್ನೇಹಿತರಿಗೆಲ್ಲ ಪತ್ರ ಬರೆದು ಸ್ವಾಮೀಜಿಯ ಬಗ್ಗೆ ತಿಳಿಸಿದ್ದ. ಆದ್ದರಿಂದ ಸ್ವಾಮೀಜಿ ಮದರಾಸಿಗೆ ಬರುತ್ತಿದ್ದಂತೆ ಅವರನ್ನು ಭೇಟಿಯಾಗಲು ಅಳಸಿಂಗ ಪೆರುಮಾಳ್, ವ್ಯಾಸರಾವ್, ವೆಂಕಟರಂಗರಾವ್, ಬಾಲಾಜಿರಾವ್, ನರಸಿಂಹಾಚಾರಿ, ಡಾ ॥ ನಂಜುಂಡರಾವ್ ಮೊದಲಾದ ಹಲವಾರು ಯುವಕರು ಭಟ್ಟಾಚಾರ್ಯರ ಮನೆಗೆ ಧಾವಿಸಿ ದರು. ಇವರೊಬ್ಬ ಅಪೂರ್ವ ಸಂನ್ಯಾಸಿಗಳೆಂದು ಅವರೆಲ್ಲ ಕೇಳಿದ್ದರು. ಪಾಶ್ಚಾತ್ಯ ಸಂಸ್ಕೃತಿ- ಸಾಹಿತ್ಯಗಳಲ್ಲಿ ವಿಶೇಷವಾಗಿ ಆಸಕ್ತರಾಗಿದ್ದ, ಹಾಗೂ ಆಧುನಿಕ ಜನಾಂಗವನ್ನು ಪ್ರತಿನಿಧಿಸುತ್ತಿದ್ದ ಈ ನವಯುವಕರು ಸ್ವಾಮೀಜಿಯ ಪರಿಚಯ ಮಾಡಿಕೊಳ್ಳಲು ತೀವ್ರ ಕುತೂಹಲ ತಾಳಿದ್ದರು. ಅವರನ್ನೆಲ್ಲ ಸ್ವಾಮೀಜಿ ಮುಗುಳ್ನಗುತ್ತ ಸ್ವಾಗತಿಸಿದರು. ಸ್ವಾಮೀಜಿಯ ಮುಖದಲ್ಲಿ ಬೆಳಗುತ್ತಿದ್ದ ಅದ್ಭುತ ಹೊಳಪು ಹಾಗೂ ಕಣ್ಣುಗಳ ಕಾಂತಿಯನ್ನು ಗಮನಿಸಿ ಈ ಸ್ನೇಹಿತರು ಅಚ್ಚರಿಗೊಂಡರು. ಬಳಿಕ ತಮ್ಮನ್ನು ತಾವೇ ಪರಿಚಯಿಸಿಕೊಂಡರು. ಸ್ವಲ್ಪ ಹೊತ್ತು ಲೋಕಾಭಿರಾಮದ ಮಾತುಕತೆ ನಡೆದ ಮೇಲೆ, ಸಾಹಿತ್ಯ-ವಿಜ್ಞಾನ-ಇತಿಹಾಸ-ಧರ್ಮ-ಅಧ್ಯಾತ್ಮಗಳಿಗೆ ಸಂಬಂಧಿಸಿದ ಪ್ರಶ್ನೆಗಳಿಂದ ಸ್ವಾಮೀಜಿಯ ಮೇಲೆ ಆಕ್ರಮಣವನ್ನೇ ಮಾಡಿದರು. ಪ್ರಶ್ನೆಗಳಿಗೆಲ್ಲ ಸ್ವಾಮೀಜಿ ನಗುನಗುತ್ತ, ಲೀಲಾಜಾಲವಾಗಿ ಉತ್ತರಿಸಿದರು. ಅವರ ಉತ್ತರಗಳು ಚುಟುಕಾಗಿದ್ದರೂ ನೇರವಾಗಿ, ಅರ್ಥ ಪೂರ್ಣವಾಗಿ ಹಾಗೂ ಮಧುರವಾಗಿದ್ದುವು. ಸಂದರ್ಭಾನುಸಾರವಾಗಿ ಸಾಹಿತ್ಯ, ವಿಜ್ಞಾನ, ಇತಿ ಹಾಸ, ಧರ್ಮಗಳಿಗೆ ಸಂಬಂಧಿಸಿದ ಶ್ರೇಷ್ಠಕೃತಿಗಳ ಸಾಲುಗಳನ್ನು ಸರಾಗವಾಗಿ ಉದ್ಧರಿಸುತ್ತಿ ದ್ದರು. ತಮ್ಮ ಚುರುಕಾದ ಉತ್ತರಗಳಿಂದ ಸ್ವಾಮೀಜಿ, ಪ್ರಶ್ನೆ ಕೇಳುವವರೆಲ್ಲರ ಬಾಯಿ ಮುಚ್ಚಿಸಿ ಬಿಟ್ಟರು. ಕತ್ತಲಾಗುತ್ತಲೇ ಈ ಸ್ನೇಹಿತರು ಸ್ವಾಮೀಜಿಯಿಂದ ಬೀಳ್ಕೊಂಡು ಹೊರಟರು. ಆ ಯುವಸಂನ್ಯಾಸಿಯ ಪ್ರತಿಭೆ ಹಾಗೂ ವಾದ ಸಾಮರ್ಥ್ಯವನ್ನು ಕಂಡು ಅವರೆಲ್ಲ ವಿಸ್ಮಯ ಮೂಕರಾಗಿಬಿಟ್ಟಿದ್ದರು.

ಸ್ವಾಮೀಜಿಯ ವ್ಯಕ್ತಿತ್ವ-ವಿಚಾರಗಳ ಸೆಳೆತಕ್ಕೆ ಸಿಲುಕಿಕೊಂಡ ಯುವಕರಲ್ಲಿ ಹಲವರು ಮದ ರಾಸಿನ ‘ಟ್ರಿಪ್ಲಿಕೇನ್ ಸಾಹಿತ್ಯ ಸಂಘ’ದ ಸದಸ್ಯರಾಗಿದ್ದರು. ಈ ಸಂಘವು ತೀವ್ರ ಉತ್ಸಾಹಿಗಳೂ ಕಾರ್ಯಶೀಲರೂ ಆದ ಯುವಕರಿಂದ ಕೂಡಿದುದಾಗಿತ್ತು. ಇದರ ಆಶ್ರಯದಲ್ಲಿ ಸಮಾಜದ ಪ್ರಮುಖ ವ್ಯಕ್ತಿಗಳಿಂದ ಉಪನ್ಯಾಸಗಳು ನಡೆಯುತ್ತಿದ್ದುವು. ತಮ್ಮ ಸಂಘದಲ್ಲಿ ಮಾತನಾಡು ವಂತೆ ಈ ಯುವಕರು ಸ್ವಾಮೀಜಿಯನ್ನು ಒಪ್ಪಿಸಿದರು. ಮೊದಲ ಸಲ ಸ್ವಾಮೀಜಿ ಮಾತನಾಡಿ ದಾಗ ಸಭೆ ಬಹಳ ದೊಡ್ಡದಾಗೇನೂ ಇರಲಿಲ್ಲ. ಆದರೆ ಭಾಷಣದ ಪರಿಣಾಮ ಮಾತ್ರ ಅದ್ಭುತ ವಾಗಿತ್ತು. ಇಲ್ಲಿಯವರೆಗೂ ಅವರ ಸಣ್ಣ ವಯಸ್ಸನ್ನು ಕಂಡು ನಿರ್ಲಕ್ಷ್ಯದಿಂದಿದ್ದ ಊರಿನ ಹಿರಿ ಯರು ಇದೀಗ ಎಚ್ಚರಗೊಂಡರು. ಈ ಅದ್ಭುತ ಸನ್ಯಾಸಿಯಲ್ಲಡಗಿರುವ ಅಸಾಧಾರಣ ಬುದ್ಧಿಶಕ್ತಿ, ಪ್ರಕಾಂಡ ಪಾಂಡಿತ್ಯ, ಪ್ರಾಮಾಣಿಕ ದೇಶಪ್ರೇಮ, ತೀಕ್ಷ್ಣ ಹಾಸ್ಯಪ್ರಜ್ಞೆ, ಹಾಗೂ ಎಲ್ಲಕ್ಕಿಂತ ಮಿಗಿ ಲಾಗಿ ಅವಿಚಲ ವೈರಾಗ್ಯಬುದ್ಧಿಯನ್ನು ಕಂಡು, ಇದ್ದಕ್ಕಿದ್ದಂತೆ ಎಲ್ಲಿಂದ ಉದ್ಭವಿಸಿದನೀತ?– ಎಂದು ಆಶ್ಚರ್ಯಚಕಿತರಾದರು. ತಮ್ಮನ್ನು ಸ್ವಾಮಿ ಸಚ್ಚಿದಾನಂದ ಎಂದು ಪರಿಚಯಿಸಿ ಕೊಂಡಿದ್ದ ಸ್ವಾಮೀಜಿಯ ಹೆಸರು ನಗರದಲ್ಲೆಲ್ಲ ಪ್ರಚಾರವಾಯಿತು.

ಅಂದಿನಿಂದಲೇ ಮನ್ಮಥಬಾಬುಗಳ ಮನೆ ಒಂದು ದಿನನಿತ್ಯದ ಯಾತ್ರಾಸ್ಥಳವಾಗಿ ಬಿಟ್ಟಿತು. ಅವರ ಬುದ್ಧಿಸಾಮರ್ಥ್ಯದ ಬಗ್ಗೆ ಕೇಳಿದ ತಿಳಿದ ಸಾರ್ವಜನಿಕರು ಅವರನ್ನು ಕಾಣಲು ಗುಂಪು ಗುಂಪಾಗಿ ಬರಲಾರಂಭಿಸಿದರು. ಪ್ರತಿದಿನ ಸಂಜೆ ನಾಲ್ಕು ಗಂಟೆಯಿಂದ ಹತ್ತು ಗಂಟೆಯವರೆಗೂ ಜನ ಬಂದುಹೋಗುತ್ತಿದ್ದರು. ತಮ್ಮನ್ನು ಕಟ್ಟಾ ನಾಸ್ತಿಕರೆಂದೂ ಧರ್ಮದ್ವೇಷಿಗಳೆಂದೂ ಹೇಳಿ ಕೊಳ್ಳುತ್ತಿದ್ದ ಜನರೂ ಈಗ ಧರ್ಮದ ಬಗ್ಗೆ ವಿಶೇಷ ಒಲವು ತಾಳಿ, ಅವರನ್ನು ನೋಡಲು ಬರಲಾರಂಭಿಸಿದ್ದರು. ಒಮ್ಮೆ ಅವರನ್ನು ನೋಡಿದವರು, ಮತ್ತೆ ಮತ್ತೆ ಬರದಿರಲು ಸಾಧ್ಯವೇ ಇರಲಿಲ್ಲ.

ಹೀಗೆ ಹಲವಾರು ಬಗೆಯ ಜನಗಳು ಅವರ ಭೇಟಿಗಾಗಿ ಬರುತ್ತಿದ್ದರು. ಬಗೆ ಬಗೆಯ ಪ್ರಶ್ನೆ ಗಳನ್ನು ಹಾಕುತ್ತಿದ್ದರು; ಬಗೆಬಗೆಯ ಉತ್ತರಗಳನ್ನು ಪಡೆಯುತ್ತಿದ್ದರು. ಸಂಭಾಷಣೆಯು ಧಾರ್ಮಿಕ-ಆಧ್ಯಾತ್ಮಿಕ ವಿಷಯಗಳಿಗಷ್ಟೇ ಸೀಮಿತವಾಗಿರಲಿಲ್ಲ. ಮನಶ್ಶಾಸ್ತ್ರ, ವಿಜ್ಞಾನ, ಸಾಹಿತ್ಯ, ಕಲೆ, ಇತಿಹಾಸ–ಹೀಗೆ ಮಾನವಜೀವನದ ಸಕಲ ವಿಚಾರಗಳೂ ಪ್ರಸ್ತಾಪಗೊಳ್ಳುತ್ತಿದ್ದುವು. ಸ್ವಾಮೀಜಿಗೆ ಇಂಥ ವಿಷಯ ಗೊತ್ತು ಇಂಥ ವಿಷಯ ಗೊತ್ತಿಲ್ಲ ಎಂಬುದೇ ಇಲ್ಲ! ಅಷ್ಟೇ ಅಲ್ಲ, ಅಲ್ಲಿ ನೆರೆದಿದ್ದವರೆಲ್ಲರಿಗಿಂತಲೂ ಅವರ ತಿಳಿವಳಿಕೆ ನಿಖರವಾಗಿರುತ್ತಿತ್ತು. ಆಳವಾಗಿ ಇರುತ್ತಿತ್ತು. ಇದನ್ನು ಕಂಡು ಅಚ್ಚರಿಗೊಳ್ಳದವರೇ ಇರಲಿಲ್ಲ.

ಒಂದು ದಿನ ಕೆಲವು ಸಂದರ್ಶಕರು ಸ್ವಾಮೀಜಿ ಭಕ್ತಿಭಾವೋನ್ಮತ್ತರಾಗಿದ್ದುದನ್ನು ಗಮನಿಸಿ ಆಶ್ಚರ್ಯಗೊಂಡರು. ಅವರಿಗೆ ಸ್ವಾಮೀಜಿಯ ಈ ಭಾವ ಹೊಸದು. ಆದ್ದರಿಂದ ಅವರಲ್ಲೊಬ್ಬ ಕೇಳಿದ, “ಸ್ವಾಮೀಜಿ, ಹಿಂದೂಗಳಿಗೆ ಅದ್ಭುತ ವೇದಾಂತ ತತ್ತ್ವಗಳ ಹಿನ್ನೆಲೆಯಿದೆ. ಆದರೂ ಅವರು ಮೂರ್ತಿಪೂಜಕರಾಗಿಯೇ ಉಳಿದಿದ್ದಾರಲ್ಲ. ಏಕೆ?” ಸ್ವಾಮೀಜಿಯನ್ನೇ ದೃಷ್ಟಿಯಲ್ಲಿಟ್ಟು ಕೊಂಡು ಕೇಳಿದ ಪ್ರಶ್ನೆ ಇದು. ಆಗ ಅವರ ಕಣ್ಣುಗಳು ಮಿಂಚಿದುವು. ಆ ಪ್ರಾಶ್ನಿಕನನ್ನೇ ದಿಟ್ಟಿಸುತ್ತ ಉತ್ತರಿಸಿದರು: “ಏಕೆಂದರೆ, ನಮ್ಮ ಮುಂದೆ ಹಿಮಾಲಯ ಪರ್ವತವಿದೆ!”

ಸ್ವಾಮೀಜಿಯ ಈ ಮಾತಿನ ಅರ್ಥವೇನು? ಘನಗಂಭೀರವಾಗಿ ನಿಂತಿರುವ ಭವ್ಯ ಸುಂದರ ಹಿಮಾಲಯದ ದರ್ಶನ ಮಾಡಿದವರಿಗೆ ಇದರ ಮರ್ಮ ಅರಿವಾಗುತ್ತದೆ. ಆಗಸದೆತ್ತರಕ್ಕೆ ನಿಂತ ಶುಭ್ರ ಹಿಮಾಚ್ಛಾದಿತ ಶಿಖರಗಳು, ವಿಚಿತ್ರ ವಿನ್ಯಾಸಗಳಿಂದ ಕೂಡಿದ ವಿವಿಧ ವನರಾಜಿ, ಅತಿ ಎತ್ತರದಿಂದ ಕೆಳಧುಮುಕುತ್ತಿರುವ ಜಲಪಾತಗಳ ಭೋರ್ಗರೆತ, ಜುಳುಜುಳು ನಿನಾದಗೈಯುತ್ತ ಹರಿಯುತ್ತಿರುವ ತಿಳಿನೀರ ಝರಿಗಳು–ಇವುಗಳಿಂದ ಕೂಡಿದ ಪ್ರಕೃತಿ ದೇವಿಯ ಮಡಿಲ್ಲಲಿರುವ ವರು ಅದಕ್ಕೆ ಬಾಗಿ ನಮಿಸದಿರಲು ಸಾಧ್ಯವೇ ಇಲ್ಲ. ಆದ್ದರಿಂದಲೇ ಭಾರತೀಯರು ಅತ್ಯುನ್ನತ ವಾದ ವೇದಾಂತ ತತ್ತ್ವಗಳನ್ನು ಅರಿತವರಾದರೂ ಬಾಹ್ಯಪೂಜೆಯನ್ನು ಬಿಡಲಾರರು ಎಂಬುದು ಆ ಉತ್ತರದ ಅರ್ಥ.

ಸ್ವಾಮೀಜಿಯ ವ್ಯಕ್ತಿತ್ವವು ಮದರಾಸಿನ ಬುದ್ಧಿವಂತ ಜನವರ್ಗವನ್ನು ಸೂಜಿಗಲ್ಲಿನಂತೆ ಸೆಳೆಯಿತು. ಅವರ ಕಂಚಿನ ಕಂಠ, ಪ್ರತಿಯೊಂದು ನಡೆನುಡಿಯಲ್ಲೂ ಹೊರಸೂಸುತ್ತಿದ್ದ ಆತ್ಮ ಶಕ್ತಿ ಹಾಗೂ ಬುದ್ಧಿಶಕ್ತಿ ಪ್ರಶ್ನೆಗಳಿಗೆ ಪ್ರತಿಧ್ವನಿಯಂತೆ ತಕ್ಷಣ ಉತ್ತರಿಸುತ್ತಿದ್ದ ರೀತಿ, ನಗು ಚಿಮ್ಮಿಸುವ ಹಾಸ್ಯ, ಮೋಡಿಗೊಳಿಸುವ ವಾಗ್ಝರಿ, ಮಧುರ ಸಂಗೀತ–ಇವುಗಳು ಜನರನ್ನು ವಿಶೇಷವಾಗಿ ಆಕರ್ಷಿಸಿದುವು. ದಿನದಿಂದ ದಿನಕ್ಕೆ ಮನ್ಮಥನಾಥ ಭಟ್ಟಾಚಾರ್ಯರ ಮನೆಗೆ ಬರುವವರ ಸಂಖ್ಯೆ ಅಧಿಕವಾಗುತ್ತ ಬಂದಿತು. ಸ್ವಾಮೀಜಿ ಯಾವ ಶಂಕೆಯೂ ಇಲ್ಲದೆ ದಿಟ್ಟತನ ದಿಂದ ಮಾತನಾಡುತ್ತಿದ್ದವರಾದರೂ, ಅವರ ಮಾತಿನಲ್ಲಿ ವಿನಯವಿತ್ತು. ಜೊತೆಗೆ ತಾವು ಮಂಡಿ ಸುವ ವಿಚಾರವನ್ನು ಅಧಿಕಾರವಾಣಿಯಿಂದ ಹೇಳಬಲ್ಲವರಾಗಿದ್ದರು. ಈ ಕ್ಷಣದಲ್ಲಿ ಅವರು, ತಮ್ಮ ಮಾತನ್ನು ತಳ್ಳಿ ಹಾಕಿ ಅವಜ್ಞೆ ತೋರಿದ ಪಂಡಿತರನ್ನೋ ವಯೋವೃದ್ಧರನ್ನೋ ನಮ್ರತೆ ಯಿಂದಲೇ ಕಂಡು ಮಾತನಾಡಿದರೆ, ಮತ್ತೊಂದು ಕ್ಷಣದಲ್ಲಿ ತಮ್ಮ ವಿಚಾರಧಾರೆಯನ್ನು ಯಾರೂ ವಿರೋಧಿಸಲಾಗದಂತೆ ಚಂಡಮಾರುತದಂತೆ ಎರಗುತ್ತಿದ್ದರು. ಆದರೆ ಅವರ ಈ ಯಾವ ವರ್ತನೆಯೂ ಅಸಹಜವಾಗಿಯೋ ತೋರಾಣಿಕೆಯದಾಗಿಯೋ ಕಂಡುಬರುತ್ತಿರಲಿಲ್ಲ. ಏಕೆಂದರೆ ಅವರ ಮಾತುಗಳಲ್ಲಿ ಪ್ರಾಮಾಣಿಕತೆ ಸ್ಪಷ್ಟವಾಗಿ ಕಾಣುತ್ತಿತ್ತು. ಎಂತಹ ಸಂದರ್ಭ ದಲ್ಲೂ ಸ್ವಾಮೀಜಿ ಯಾರ ಮನಸ್ಸಿಗೂ ನೋವಾಗುವಂತೆ ಮಾತನಾಡಿದವರಲ್ಲ. ಆದರೆ ಯಾರಾ ದರೂ ಮನಸ್ಸಿಗೆ ಬಂದಂತೆ ಅಸಮರ್ಪಕವಾಗಿ ಮಾತಾನಾಡಿದರೆ ಅದನ್ನು ತೀವ್ರವಾಗಿ ಖಂಡಿಸದೆ ಬಿಡುತ್ತಲೂ ಇರಲಿಲ್ಲ. ಉದಾಹರಣೆಗೆ, ಒಂದು ದಿನ ಪಂಡಿತ ಮಹಾಶಯರೊಬ್ಬರು ಕೇಳಿದರು, “ಸ್ವಾಮೀಜಿ ಈಗೆಲ್ಲ ಜನಗಳಿಗೆ ಸಾಕಷ್ಟು ಸಮಯ ಸಿಗದಿರುವುದರಿಂದ ಸಂಧ್ಯಾವಂದನೆಯನ್ನು ಬಿಟ್ಟುಬಿಟ್ಟರೂ ಅದರಲ್ಲಿ ತಪ್ಪೇನೂ ಇಲ್ಲ, ಅಲ್ಲವೆ?”

ನೂರು ವರ್ಷಗಳ ಹಿಂದೆಯೂ ಅದೇ ಗೋಳು, ಅದೇ ದೂರು–‘ಸಂಧ್ಯಾವಂದನೆ ಮಾಡಲು ಸಮಯವಿಲ್ಲ!’ ಅದೂ, ಹಾಗೆ ಹೇಳುತ್ತಿರುವವರು ಸಾಮಾನ್ಯ ‘ಲೌಕಿಕ’ರಲ್ಲ. ವೈದಿಕರಾದ ಪಂಡಿತರು! ಅವರ ಮಾತು ಕೇಳಿ ಸ್ವಾಮೀಜಿ ಘರ್ಜಿಸಿದರು:

“ಏನು! ಸಮಯವಿಲ್ಲವೆ? ನಿಧಾನವಾಗಿ ನಡೆದು ಹೋಗಲೂ ವ್ಯವಧಾನವಿಲ್ಲದಿದ್ದ ಆ ಪುರಾತನ ಪುಷಿಗಳಿಗೆ, ಯಾರ ಮಹಾನ್ ವ್ಯಕ್ತಿತ್ವದ ಮುಂದೆ ನಾವು ಕೇವಲ ಕ್ಷುದ್ರ ಕ್ರಿಮಿಗಳೋ ಅಂತಹವರಿಗೆ, ಸಂಧ್ಯಾವಂದನೆಗೆ ಸಮಯವಿತ್ತಂತೆ ನಿಮಗಿಲ್ಲವೋ?”

ಇಲ್ಲಿ ಸ್ವಾಮೀಜಿ, ಸಂಧ್ಯಾವಂದನಾದಿ ನಿತ್ಯಕರ್ಮಗಳಿಗೆ ಎಷ್ಟು ಪ್ರಾಶಸ್ತ್ಯ ನೀಡುತ್ತಿದ್ದಾರೆ ಎಂಬುದು ಗಮನಿಸಬೇಕಾದ ಅಂಶ.

ಅದೇ ಸಂದರ್ಭದಲ್ಲಿ ಪಾಶ್ಚಾತ್ಯ ನಾಗರಿಕತೆಗೆ ಮನಸೋತವನೊಬ್ಬ ತನ್ನ ಅಭಿಪ್ರಾಯವನ್ನು ಮುಂದಿಟ್ಟು: “ಸ್ವಾಮೀಜಿ, ಆ ಪುಷಿಗಳು ಬರೆದಿರುವ ವೇದಶಾಸ್ತ್ರಗಳೆಲ್ಲ ಕೇವಲ ಗೊಡ್ಡು ಕಂತೆ ಗಳಲ್ಲವೆ? ಅವುಗಳಲ್ಲಿ ಉಪಯುಕ್ತವಾದಂಥದು ಒಂದಕ್ಷರವಾದರೂ ಇರಲಾರದು!” ಸ್ವಾಮೀಜಿ ಸಿಟ್ಟಿಗೆದ್ದು ಬರಸಿಡಿಲಿನಂತೆ ಎರಗಿದರು: “ಏನೆಂದಿರಿ? ಆ ನಿಮ್ಮ ಪರಮಪೂಜ್ಯ ಹಿರಿಯರನ್ನೇ ನಿಂದಿಸುವಷ್ಟು ಎದೆಕೆಚ್ಚೆ ನಿಮಗೆ? ನೀವು ಕಲಿತ ನಾಲ್ಕಕ್ಷರದಿಂದ ನಿಮ್ಮ ತಲೆಯೇ ತಿರುಗಿಹೋಗಿಬಿಟ್ಟಿದೆ. ನೀವು ಆ ಪುಷಿಗಳ ಜ್ಞಾನದಾಳವನ್ನು ಪರೀಕ್ಷಿಸಿ ನೋಡಿದ್ದೀರಾ? ಹೋಗಲಿ, ಆ ವೇದಗಳನ್ನು ಆಮೂಲಾಗ್ರವಾಗಿ ಓದಿಯಾದರೂ ಇದ್ದೀರೋ? ಆ ಗ್ರಂಥಗಳೆಲ್ಲ ಪುಷಿಗಳು ನಿಮ್ಮ ಮುಂದಿಟ್ಟಿರುವ ಸವಾಲು. ನಿಮ್ಮಲ್ಲೇನಾದರೂ ತಾಕತ್ತಿದ್ದರೆ ಅದನ್ನು ಎದುರಿಸಿ ನೋಡೋಣ! ಕೆಚ್ಚಿದ್ದರೆ ಅವರ ಉಪದೇಶಗಳನ್ನು ಒರೆಗೆ ಹಚ್ಚಿ, ನೋಡೋಣ!”

ಸ್ವಾಮೀಜಿಯ ಮಾತುಗಳನ್ನು ಕೇಳಿದ ಮದರಾಸಿನ ಬುದ್ಧಿವಂತ ಜನರು ಬೆಕ್ಕಸಬೆರಗಾದರು. ಅವರು ವೇದಾಂತತತ್ತ್ವಗಳನ್ನು ವೈಜ್ಞಾನಿಕವಾಗಿ ಬೋಧಿಸುವುದನ್ನೂ ಅವುಗಳನ್ನು ಅವರು ಸ್ವತಃ ಸಾಕ್ಷಾತ್ಕರಿಸಿಕೊಂಡಿರುವುದನ್ನೂ ಕಂಡ ಜನರು, ಆ ಮಾತುಗಳಲ್ಲಿ ವಿಶೇಷ ಸತ್ತ್ವ-ಚೈತನ್ಯಗಳನ್ನು ಕಂಡುಕೊಂಡರು. ಮದರಾಸಿನ ಜನರಿಗೆ ಅವರ ಮೂಲಕ ಪ್ರತಿದಿನವೂ ಜ್ಞಾನದ ಹೊಸ ಹೊಸ ಅಂಶ ಸಿಗುತ್ತಿತ್ತು. ಇಂದು ವಾಲ್ಮೀಕಿ, ಕಾಳಿದಾಸ, ಭವಭೂತಿಯರ ಬಗ್ಗೆ ಮತ್ತು ಅಷ್ಟೇ ಅಧಿಕಾರ ಯುತವಾಗಿ ಹೋಮರ್, ಷೇಕ್ಸ್​ಪಿಯರ್, ಬೈರನ್ನರ ಬಗ್ಗೆ ಮಾತನಾಡಿದರೆ, ನಾಳೆ ದಿನ ಟ್ರೋಜನರ ಹಾಗೂ ಪಾಂಡವರ ಬಗ್ಗೆ, ಹೆಲೆನ್ ಹಾಗೂ ದ್ರೌಪದಿಯ ಬಗ್ಗೆ ಹೇಳುತ್ತಾರೆ. ಒಮ್ಮೆ ಗ್ರೀಕರ ಕಲೆಯಲ್ಲಿ ಇಂದ್ರಿಯ ಭೋಗಾದರ್ಶವನ್ನು ಪ್ರತಿಪಾದಿಸುವ ಅಂಶವನ್ನು ವಿವರಿಸಿದರೆ, ಮತ್ತೊಮ್ಮೆ ಆ ಆದರ್ಶಕ್ಕೆ ತದ್ವಿರುದ್ಧವಾದ ಭಾರತೀಯ ಕಲೆ-ಸಂಸ್ಕೃತಿಗಳನ್ನು ಬಣ್ಣಿಸುತ್ತಾರೆ. ಹಿಂದೂ ಮನಶ್ಶಾಸ್ತ್ರದ ಕುರಿತಾಗಿ ಮಾತನಾಡುವಾಗಲಂತೂ ಸ್ವಾಮೀಜಿ, ಕೇಳುಗರನ್ನು ಅವರು ಕಂಡರಿಯದ ಹೊಸ ಲೋಕಕ್ಕೇ ಕರೆದೊಯ್ಯುತ್ತಿದ್ದರು.

ನಿರಂತರ ಜನಪ್ರವಾಹದ ಸಂಪರ್ಕದಿಂದ ಉಂಟಾದ ದೈಹಿಕ-ಮಾನಸಿಕ ದಣಿವಿನ ಪರಿಹಾರ ಕ್ಕಾಗಿ ಸ್ವಾಮೀಜಿ ಆಗಾಗ ಸಮುದ್ರದ ಕಿನಾರೆಯಲ್ಲಿ ತಿರುಗಾಡಿಕೊಂಡು ಬರಲು ಹೋಗುತ್ತಿದ್ದರು. ಒಂದು ದಿನ ಹೀಗೆ ಹೊರಟಿದ್ದಾಗ ಅರೆಹೊಟ್ಟೆಯೂಟದಿಂದ ಸೊರಗಿದ ಬಡ ಬೆಸ್ತಮಕ್ಕಳು, ತಮ್ಮ ತಾಯಂದಿರೊಡನೆ ನೀರಿನಲ್ಲಿಳಿದು ದುಡಿಯುತ್ತಿರುವ ಹೃದಯವಿದ್ರಾವಕ ದೃಶ್ಯವನ್ನು ಕಂಡರು. ಬೇರೆ ಯಾರಿಗೂ ಈ ದೃಶ್ಯದಲ್ಲಿ ವಿಶೇಷವೇನೂ ಕಂಡಿರಲಾರದು. ಆದರೆ ನಗುನಗುತ್ತ ಆಟಪಾಠಗಳಲ್ಲಿ ತೊಡಗಿರಬೇಕಾದ ವಯಸ್ಸಿನಲ್ಲಿ ಈ ಮಕ್ಕಳು, ವಿಧಿಯಿಲ್ಲದೆ ತಮ್ಮ ತಾಯಂದಿರೊಡಗೂಡಿ ತುತ್ತಿನ ಗಳಿಕೆಗಿಳಿಯಬೇಕಾಯಿತಲ್ಲ ಎಂದು ಸ್ವಾಮೀಜಿ ಮರುಗಿದರು. ಅವರಿಗರಿವಿಲ್ಲದಂತೆಯೇ ಕಂಗಳಿಂದ ನೀರು ಚಿಮ್ಮಿತು. ದುಃಖತಪ್ತರಾಗಿ ಉದ್ಗರಿಸಿದರು: “ಹೇ ಭಗವಂತ, ಈ ದುಃಖಿಗಳನ್ನು ಏಕಾದರೂ ಸೃಷ್ಟಿಸಿದೆ ನೀನು? ನಾನಿದನ್ನು ನೋಡಲಾರೆ. ಎಲ್ಲಿಯ ವರೆಗಿದು, ಭಗವಂತಾ, ಎಲ್ಲಿಯವರೆಗೆ?” ಬಡವರನ್ನು ಕಂಡು ಕಣ್ಣೀರ್ಗರೆಯುತ್ತಾರೆ ಬ್ರಹ್ಮ ಜ್ಞಾನಿ! ಅವರ ಕರುಣಾಪೂರ್ಣ ಕ್ರಂದನವನ್ನು ಕಂಡು ಜೊತೆಯಲ್ಲಿದ್ದವರಿಗೂ ಉಕ್ಕಿಬಂದ ಕಣ್ಣೀರನ್ನು ತಡೆಯಲು ಸಾಧ್ಯವಾಗಲಿಲ್ಲ.

ಸ್ವಾಮೀಜಿಯ ಸನ್ಮಾನಾರ್ಥವಾಗಿ ಮದರಾಸಿನ ನಾಗರಿಕರು ಒಂದು ಸಂಜೆ ಸಂತೋಷಕೂಟ ವೊಂದನ್ನು ಏರ್ಪಡಿಸಿದರು. ಸಮಾರಂಭಕ್ಕೆ ಹೆಸರಾಂತ ಮೇಧಾವಿಗಳನ್ನು ಆಹ್ವಾನಿಸಲಾಗಿತ್ತು. ಈ ಸಂದರ್ಭದಲ್ಲಿ ಮಾತನಾಡುವಾಗ ಸ್ವಾಮೀಜಿ ತಾವೊಬ್ಬರು ಅದ್ವೈತಿ ಎಂದು ಧೈರ್ಯವಾಗಿ ಹೇಳಿಕೊಂಡರು. ಅದ್ವೈತವೆಂದರೆ, ಎರಡಿಲ್ಲ; ಎಲ್ಲವೂ ಒಂದೇ ಎಂಬ ಸಿದ್ಧಾಂತ. ಎಂದರೆ, ಸಕಲ ವಸ್ತುಗಳೂ ಬ್ರಹ್ಮವಲ್ಲದೆ ಬೇರಲ್ಲ ಎಂದರ್ಥ. ತಾವು ಅದ್ವೈತಿಯೆಂದು ಹೇಳಿಕೊಂಡ ಮೇಲೆ, ತಮ್ಮನ್ನು ತಾವು ಬ್ರಹ್ಮವೆಂದೇ ನಂಬಿರುವುದಾಗಿ, ಅರಿತಿರುವುದಾಗಿ, ಹೇಳಿದಂತಾಯಿತು. ತಮ್ಮ ಈ ಮಾತಿಗೆ ವಿರೋಧವೇಳಬಹುದೆಂದು ಸ್ವಾಮೀಜಿ ಮೊದಲೇ ಊಹಿಸಿದ್ದರು. ಅದರಂತೆಯೇ, ಬಳಿಕ ಕೆಲವರು ಬುದ್ಧಿಜೀವಿಗಳು ಅವರನ್ನು ಕೇಳಿಯೇಬಿಟ್ಟರು: “ಸ್ವಾಮೀಜಿ, ನೀವು ನಿಮ್ಮನ್ನು ಪರಮಾತ್ಮನಲ್ಲಿ ಒಂದಾಗಿರುವವರು ಎಂದು ಹೇಳಿಕೊಳ್ಳುತ್ತೀರಿ. ಹಾಗಾದರೆ ನೀವು ನಿಮ್ಮ ಯಾವ ಕೃತ್ಯಕ್ಕೂ ಜವಾಬ್ದಾರರಲ್ಲವೆಂದಾಯಿತು! ಎಂದ ಮೇಲೆ, ನೀವೇನಾದರೂ ತಪ್ಪು ಹೆಜ್ಜೆಯಿಟ್ಟರೂ ನಿಮ್ಮನ್ನು ಯಾರೂ ಪ್ರಶ್ನಿಸುವಂತಿಲ್ಲ ಎಂದರ್ಥವಲ್ಲವೆ?” ಇದಕ್ಕೆ ಸ್ವಾಮೀಜಿ ಉತ್ತರಿಸಿದರು, “ನಾನು ಪರಮಾತ್ಮನೊಂದಿಗೆ ಒಂದಾಗಿದ್ದೇನೆ ಎಂದು ಪ್ರಾಮಾಣಿಕ ವಾಗಿ ನಂಬಿದ್ದೇ ಆದರೆ ನನ್ನಿಂದ ಯಾವ ತಪ್ಪೂ ಘಟಿಸಲು ಸಾಧ್ಯವೇ ಇಲ್ಲ. ಆದ್ದರಿಂದ ನನ್ನ ಮೇಲೆ ಯಾವ ಹಿಡಿತವೂ ಬೇಕೇ ಇಲ್ಲ.” ಈ ಉತ್ತರವನ್ನು ಕೇಳಿ ಆ ಪ್ರಾಶ್ನಿಕರ ಬಾಯಿಗಳು ಕಟ್ಟಿಹೋದವು.

ಆದರೆ ಸ್ವಾಮೀಜಿ ತಾವು ವಾದದಲ್ಲಿ ಗೆಲ್ಲಬೇಕೆಂಬ ಉದ್ದೇಶದಿಂದ ಹೀಗೆ ಉತ್ತರಿಸಿದ್ದಲ್ಲ. ಅವರ ಉತ್ತರದಲ್ಲಿ ದೃಢತೆಯಿತ್ತು. ಕೆಲವಾರಗಳ ಹಿಂದೆ ರಾಮನಾಡಿನ ರಾಜನ ಅರಮನೆಗೆ ಹೋಗಿದ್ದಾಗ ಮಾತುಕತೆ ಇದೇ ತರಹ ತಿರುಗಿಕೊಂಡಿತ್ತು. ಮನುಷ್ಯಮಾತ್ರನಾದವನು ಅವ್ಯಕ್ತನೂ ನಿರಾಕಾರವೂ ಆದ ಪರಬ್ರಹ್ಮವನ್ನು ಕಾಣಲು ಸಾಧ್ಯವೆಂಬ ಮಾತು ಹಾಸ್ಯಾಸ್ಪದವೇ ಸರಿ ಎಂದು ಕೆಲವರು ಗೇಲಿ ಮಾಡಿದ್ದರು. ತಮ್ಮಿಂದ ಸಾಧ್ಯವಿಲ್ಲದುದು ಯಾರಿಂದಲೂ ಸಾಧ್ಯ ವಿಲ್ಲ ಎಂದು ತಿಳಿಯುತ್ತಾರೆ ಜನ. ಆಗ ಸ್ವಾಮೀಜಿ ಗಂಭೀರವಾಗಿ ಉತ್ತರಿಸಿದ್ದರು: “ನಾನು ಆ ಅವ್ಯಕ್ತವಾದ ಪರಬ್ರಹ್ಮವನ್ನು ನಿಶ್ಚಯವಾಗಿಯೂ ಕಂಡಿದ್ದೇನೆ.”

ಸ್ವಾಮೀಜಿಯ ಬಳಿಗೆ ಬರುತ್ತಿದ್ದ ಹಲವಾರು ಯುವಕರು ಮದರಾಸಿನಲ್ಲಿ ಸಮಾಜಸುಧಾರಣಾ ಚಳವಳಿಗಳಿಗೆ ಸೇರಿದವರಾಗಿದ್ದರು. ಇವರ ಕಾರ್ಯವಿಧಾನವನ್ನು ಸ್ವಾಮೀಜಿ ಒಪ್ಪಲಿಲ್ಲ. ಇವರು ಸುಧಾರಣೆಯ ಹೆಸರಿನಲ್ಲಿ ಸಾಂಪ್ರದಾಯಿಕ ರೀತಿನೀತಿಗಳನ್ನೆಲ್ಲ ಹಿಂದುಮುಂದು ನೋಡದೆ ಉಗ್ರವಾಗಿ ಖಂಡಿಸಿ, ಪಾಶ್ಚಾತ್ಯ ಸಂಸ್ಕೃತಿಯನ್ನು ನೇರವಾಗಿ ಅನುಕರಿಸಲು ಮುಂದಾಗಿದ್ದರು. ಇದನ್ನು ಕಂಡು ಸ್ವಾಮೀಜಿ, ಅವರಿಗೆ ಮತ್ತೆ ಮತ್ತೆ ಒತ್ತಿಹೇಳಿದರು: “ನೀವು ಪಾಶ್ಚಾತ್ಯ ಆದರ್ಶ ಗಳನ್ನು ಹಿಂಬಾಲಿಸುವ ಮೊದಲು ಅವುಗಳನ್ನು ಚೆನ್ನಾಗಿ ಪರಿಶೀಲಿಸಿ, ವಿಮರ್ಶಿಸಿ ನೋಡಬೇಕು. ಅವರ ವಿಲಾಸಪೂರ್ಣವಾದ ಸಂಸ್ಕೃತಿಯು ನಮ್ಮ ಸಮಾಜದೊಳಗೆ ಸೇರಿಕೊಳ್ಳದಂತೆ ಅತಿ ಹೆಚ್ಚಿನ ಎಚ್ಚರ ವಹಿಸಬೇಕು. ಜೊತೆಗೆ, ನಮ್ಮ ಪುರಾತನ ಭಾರತೀಯ ಸಂಸ್ಕೃತಿಯಲ್ಲಿ ಮಹತ್ವ ಪೂರ್ಣವೂ ವೈಭವಯುತವೂ ಆದ ಯಾವಯಾವ ಅಂಶಗಳಿವೆಯೋ ಅವುಗಳನ್ನೆಲ್ಲ ಕಾಪಾಡಿ ಕೊಳ್ಳುವ ಹಾಗೂ ಎತ್ತಿಹಿಡಿಯುವ ಪ್ರಯತ್ನ ಮಾಡಬೇಕು. ಹಾಗೆ ಮಾಡದೆಹೋದರೆ, ಭಾರತದ ರಾಷ್ಟ್ರೀಯತೆಯ ತಳಪಾಯವೇ ಕುಸಿದು ಹೋದೀತು... ಆದರೆ ನಾನು ಸಮಾಜ ಸುಧಾರಣೆಯ ವಿರೋಧಿಯಲ್ಲವೆಂಬುದನ್ನು ನೀವು ಅರ್ಥಮಾಡಿಕೊಳ್ಳಬೇಕು. ಬದಲಾಗಿ ಸುಧಾರಣೆಯನ್ನು ತೀವ್ರವಾಗಿ ಬೆಂಬಲಿಸುವವನು ನಾನು. ಒಟ್ಟಿನಲ್ಲಿ ನಾನು ಹೇಳುವುದೇನೆಂದರೆ, ಸುಧಾರಣೆಯು ರಾಷ್ಟ್ರದ ಆದರ್ಶ-ಮೌಲ್ಯಗಳಿಗೆ ಅನುಗುಣವಾಗಿರಬೇಕೇ ಹೊರತು ಪರರಾಷ್ಟ್ರವೊಂದರ ಸಂಸ್ಕೃತಿ ಯನ್ನು ಹೇರಿಕೊಳ್ಳುವಂತಾಗಬಾರದು. ಎಂದರೆ, ಯಾವುದೇ ಬಗೆಯ ಸುಧಾರಣೆಯು ರಚನಾತ್ಮಕವಾಗಿರಬೇಕಲ್ಲದೆ ವಿಧ್ವಂಸಕವಾಗಿರಬಾರದು.”

ಟ್ರಿಪ್ಲಿಕೇನಿನ ಸಾಹಿತ್ಯ ಸಂಘದಲ್ಲಿ ಮಾಡಿದ ಮೊದಲ ಉಪನ್ಯಾಸವು ಯಶಸ್ವಿಯಾದ ನಂತರ, ಆ ಸಂಘದ ಆಹ್ವಾನದ ಮೇರೆಗೆ ಸ್ವಾಮೀಜಿ ಹಲವಾರು ಗಣ್ಯವಕ್ತಿಗಳೊಂದಿಗೆ ಅನೇಕ ಸಂಭಾಷಣೆ ಗಳಲ್ಲಿ ಭಾಗವಹಿಸಿದರು. ಮದರಾಸಿನ ಹೆಚ್ಚಿನ ಜನರಿಗೆ ಸ್ವಾಮೀಜಿಯ ಬಗ್ಗೆ ತಿಳಿಯುವಂತಾ ದದ್ದು ಈ ಸಂಘದ ಮೂಲಕವೇ. ಇಲ್ಲಿ ಅವರು ನಡೆಸಿಕೊಟ್ಟ ಇಂತಹ ಸಭೆಗಳಲ್ಲಿ ಒಂದರ ಕುರಿತಾಗಿ ವಿವರಪೂರ್ಣವಾದ ವರದಿಯೊಂದು \eng{Indian Social Reformer} ಎಂಬ ಸಾಪ್ತಾಹಿಕ ಪತ್ರಿಕೆಯಲ್ಲಿ ಪ್ರಕಟವಾಯಿತು. ಈ ಪತ್ರಿಕೆಯ ಸಂಪಾದಕರು, ಅಂದಿನ ಕಾಲದ ಅತ್ಯಂತ ಶಕ್ತಿ ಶಾಲೀ ಹಾಗೂ ಪ್ರಭಾವಶಾಲೀ ಲೇಖಕ-ಪತ್ರಿಕೋದ್ಯಮಿಯಾಗಿದ್ದ ಶ್ರೀ ಕೆ. ನಟರಾಜನ್ ಎಂಬ ವರು. ಸ್ವತಃ ಇವರೂ ಸ್ವಾಮೀಜಿಯವರ ವ್ಯಕ್ತಿತ್ವ ಹಾಗೂ ಅಭಿಪ್ರಾಯಗಳಿಗೆ ಮಾರುಹೋಗಿದ್ದರು.

ಈ ವಾರಪತ್ರಿಕೆಯಲ್ಲಿ ತಿಳಿಸಿರುವ ಪ್ರಕಾರ, ಅಂದಿನ ಸಭೆಯಲ್ಲಿ ನೂರಕ್ಕೂ ಹೆಚ್ಚಿನ ಸುಶಿಕ್ಷಿತ, ಗಣ್ಯನಾಗರಿಕರು ಹಾಜರಿದ್ದರು. ರಾಷ್ಟ್ರಮಟ್ಟದ ಪ್ರಮುಖ ವ್ಯಕ್ತಿಯಾಗಿದ್ದ ದಿವಾನ್ ಬಹಾದ್ದೂರ್ ರಘನಾಥರಾವ್​ರವರು ಅಂದಿನ ಸಭೆಯ ಅಧ್ಯಕ್ಷರು. ಇಂತಹ ಗಣ್ಯರನ್ನುದ್ದೇಶಿಸಿ ಮಾತನಾಡಿದ ಸ್ವಾಮೀಜಿ ಹಿಂದೂಧರ್ಮದ ಹಲವಾರು ಅಂಶಗಳನ್ನು ವಿವರಿಸಿದರು. ವೇದ ಧರ್ಮ, ಹಿಂದೂ ಜೀವನಾದರ್ಶ, ಪಾಶ್ಚಾತ್ಯ ದೃಷ್ಟಿಕೋನದಲ್ಲಿ ಭಾರತೀಯ ಸಂಸ್ಕೃತಿ–ಇವು ಗಳನ್ನು ತಮ್ಮ ವಿಶಿಷ್ಟ ಶೈಲಿಯಲ್ಲಿ ವಿಶ್ಲೇಷಿಸಿದರಲ್ಲದೆ, ಆ ದಿನಗಳಲ್ಲಿ ವಿವಾದಾತ್ಮಕ ವಿಷಯ ಗಳಾಗಿದ್ದ ಶ್ರಾದ್ಧಕರ್ಮ, ಸ್ತ್ರೀ ವಿದ್ಯಾಭ್ಯಾಸ ಹಾಗೂ ಹಿಂದೂಪುನರುತ್ಥಾನದ ಬಗ್ಗೆ ತಮ್ಮ ವಿನೂತನ ಅಭಿಪ್ರಾಯಗಳನ್ನು ವ್ಯಕ್ತಪಡಿಸಿದರು.

ಹೀಗೆ ಕೆಲವಾರಗಳ ಹಿಂದೆಯಷ್ಟೇ ಒಬ್ಬ ಸರ್ವೇಸಾಧಾರಣ ಭಿಕಾರಿ ಸಾಧುವಾಗಿದ್ದ ಸ್ವಾಮೀಜಿ, ಮದರಾಸಿಗೆ ಬರುತ್ತಿದ್ದಂತೆಯೇ ಯಶಸ್ಸು-ಕೀರ್ತಿಗಳ ಶಿಖರವನ್ನೇರಲಾರಂಭಿಸಿದ್ದರು. ಅವರಲ್ಲಿ ವಿಶ್ವವಿಜೇತ ವಿವೇಕಾನಂದರನ್ನು ಸ್ಪಷ್ಟವಾಗಿ ಕಂಡುಕೊಂಡವರಲ್ಲಿ ಮದ್ರಾಸಿಗರೇ ಮೊದಲಿಗರು. ಅತಿ ಹೆಚ್ಚಿನ ಜನರಿಗೆ ಅವರ ಪರಿಚಯವಾದದ್ದು ಈ ಊರಿನಲ್ಲಿಯೇ. ಅಲ್ಲದೆ ಅವರು ಅಮೆರಿಕದ ವಿಶ್ವಧರ್ಮ ಸಮ್ಮೇಳನಕ್ಕೆ ಹೋಗಲು ಸಾಧ್ಯವಾದದ್ದೂ ಇಲ್ಲಿನ ಯುವಕರ ಪ್ರೋತ್ಸಾಹ ಹಾಗೂ ನೆರವಿನಿಂದಲೇ.

ಮದರಾಸಿನಲ್ಲಿ ಸ್ವಾಮೀಜಿಯ ಸಂಪರ್ಕಕ್ಕೆ ಬಂದು, ಕಡೆಯವರೆಗೂ ಅವರ ನಿಷ್ಠಾವಂತ ಅನುಯಾಯಿಗಳಾಗಿ ಉಳಿದುಕೊಂಡ ಈ ಯುವಕರಲ್ಲಿ ಅಗ್ರಗಣ್ಯರೇ ಅಳಸಿಂಗ ಪೆರುಮಾಳ್, ಚಿಕ್ಕಮಗಳೂರಿನ ಕನ್ನಡಿಗರಾದ ಈತ, ಪದವೀಧರರಾದ ಮೇಲೆ ಮದರಾಸು ಪ್ರಾಂತ್ಯದಲ್ಲಿ ಉಪಾಧ್ಯಾಯರಾಗಿದ್ದರು. ಬಳಿಕ ಚಿಕ್ಕ ವಯಸ್ಸಿನಲ್ಲೇ ಮದರಾಸಿನ ಪಚ್ಚೈಯಪ್ಪ ಹೈಸ್ಕೂಲಿನ ಮುಖ್ಯೋಪಾಧ್ಯಾಯರಾಗಿ ನೇಮಿಸಲ್ಪಟ್ಟಿದ್ದರು. ಆದರೆ ಕಡಿಮೆ ಸಂಬಳದಲ್ಲಿ ದೊಡ್ಡ ಸಂಸಾರ ವನ್ನು ಸಾಕುವ ಭಾರ ತಲೆಯ ಮೇಲಿತ್ತು. ಕಷ್ಟಗಳ ಪರಂಪರೆಯೇ ಎದುರಾದರೂ ಅಳಸಿಂಗರ ಉತ್ಸಾಹ ಬತ್ತಲಿಲ್ಲ. ಸ್ವಾಮೀಜಿಯ ಬಗ್ಗೆ ಇವರಿಗಿದ್ದ ಪ್ರೀತಿ, ಸ್ವಾಮಿನಿಷ್ಠೆ ಅದ್ವಿತೀಯವಾದುದು. ಸ್ವಾಮೀಜಿಯೂ ಇವರನ್ನು ಅತ್ಯಂತ ವಿಶ್ವಾಸದಿಂದ ಕಾಣುತ್ತಿದ್ದರಲ್ಲದೆ, ಇವರನ್ನು ವಿಶೇಷವಾಗಿ ನೆಚ್ಚಿಕೊಂಡಿದ್ದರು. ಅಮೆರಿಕದಲ್ಲಿದ್ದಾಗ ಅವರು ಪೆರುಮಾಳರಿಗೆ ಆಗಾಗ ಪತ್ರಗಳನ್ನು ಬರೆ ಯುತ್ತಲೇ ಇದ್ದರು. ಮುಂದೆ ೧೮೯೫ರಲ್ಲಿ ಇವರು ಸ್ವಾಮೀಜಿಯ ಆದೇಶದಂತೆ ಮದರಾಸಿನಲ್ಲಿ ‘ಬ್ರಹ್ಮವಾದಿನ್​’ ಎಂಬ ಆಂಗ್ಲ ಮಾಸಪತ್ರಿಕೆಯನ್ನು ಪ್ರಾರಂಭಿಸಿದರಲ್ಲದೆ ಬಳಿಕ ಅದರ ಸಂಪಾದಕರಾಗಿಯೂ ದುಡಿದರು.

ಒಂದು ದಿನ ಮದರಾಸಿನ ಕ್ರಿಶ್ಚಿಯನ್ ಕಾಲೇಜಿನಲ್ಲಿ ವಿಜ್ಞಾನ ವಿಭಾಗದ ಉಪಪ್ರಾಧ್ಯಾಪಕ ರಾಗಿದ್ದವರೊಬ್ಬರು ಸ್ವಾಮೀಜಿಯನ್ನು ಕಾಣಲು ಬಂದರು. ಇವರ ಹೆಸರು ಸಿಂಗಾಲವೇಲು ಮುದಲಿಯಾರ್. ಇವರು ಹಿಂದೂಗಳಾಗಿದ್ದರೂ, ಕ್ರೈಸ್ತಧರ್ಮವು ಹೆಚ್ಚು ಅನುಷ್ಠಾನಾತ್ಮಕ ವಾದದ್ದೆಂದು ಭಾವಿಸಿ, ಹಿಂದೂಧರ್ಮವನ್ನು ಟೀಕಿಸುತ್ತಿದ್ದರು. ಸ್ವಾಮೀಜಿ ಹಿಂದೂ ಧರ್ಮವನ್ನು ಬಲವಾಗಿ ಸಮರ್ಥಿಸುವ ವಿಚಾರ ತಿಳಿದು, ಅವರೊಂದಿಗೆ ವಾದ ಮಾಡಿ ಸೋಲಿಸಬೇಕೆಂದು ಮುದಲಿಯಾರರು ಬಂದಿದ್ದರು. ಆದರೆ ಚರ್ಚೆ ಹೆಚ್ಚು ಹೊತ್ತು ನಡೆಯಲಿಲ್ಲ. ಸ್ವಲ್ಪ ಹೊತ್ತಿ ನಲ್ಲೇ ಇವರ ಬತ್ತಳಿಕೆಯಲ್ಲಿದ್ದ ಅಸ್ತ್ರಗಳೆಲ್ಲ ನಿಷ್ಫಲವಾಗಿ ಬಿದ್ದುವು. ವಾದ ಹೂಡಿ ಗೆಲ್ಲಬೇಕೆಂದು ಬಂದ ಮುದಲಿಯಾರರು ಸೋತು ಕೈಮುಗಿದರು; ತಮ್ಮ ವಾದದ ಪೊಳ್ಳುತನವನ್ನು ಮನಗಂಡು ಸ್ವಾಮೀಜಿಗೆ ಶರಣಾಗಿ ಅವರ ಶಿಷ್ಯರಾದರು. ಇವರ ಬುದ್ಧಿಮತ್ತೆಯನ್ನೂ ಪ್ರಾಮಾಣಿಕತೆಯನ್ನೂ ಕಂಡು ಸ್ವಾಮೀಜಿ ಬಹಳವಾಗಿ ಪ್ರೀತಿಸುತ್ತಿದ್ದರು. ಅವರನ್ನು ಸಲಿಗೆಯಿಂದ ‘ಕಿಡಿ’ ಎಂದು ಸಂಬೋಧಿಸುತ್ತಿದ್ದರು. (ಕಿಡಿ ಎಂದರೆ ತಮಿಳಿನಲ್ಲಿ ಗಿಳಿ ಎಂದರ್ಥ. ಅವರು ಗಿಳಿಯಂತೆ ತೀರ ಕಡಿಮೆ ಆಹಾರ ಸೇವಿಸುತ್ತಿದ್ದುದರಿಂದ ಅವರಿಗೆ ಆ ಹೆಸರು ಬಂದಿರಬೇಕು.) ಈ ‘ಕಿಡಿ’ಯ ಕುರಿತಾಗಿ ಸ್ವಾಮೀಜಿ ಹೇಳುತ್ತಿದ್ದರು: “ಸೀಸರ್ ಹೇಳಿದ, ‘ನಾ ಬಂದೆ, ನಾ ಕಂಡೆ, ನಾ ಗೆದ್ದೆ’ ಅಂತ. ಆದರೆ ಈ ಕಿಡಿ ಬಂದ, ಕಿಡಿ ಕಂಡ, ಕಿಡಿಯೇ ಸೋತ \eng{!” “caesar said, ‘I came, I saw, I conquered.’ But Kidi came, he saw and HE was conquered!”} ಮುಂದೆ ‘ಪ್ರಬುದ್ಧ ಭಾರತ’ ಎಂಬ ಮಾಸಪತ್ರಿಕೆಯನ್ನು ಹೊರಡಿಸಿದಾಗ ಇವರು ಅದರ ಗೌರವ ಕಾರ್ಯ ನಿರ್ವಾಹಕರಾದರು. ಕಾಲಕ್ರಮದಲ್ಲಿ ಇವರು ಸರ್ವಸಂಗ ತ್ಯಾಗ ಮಾಡಿ ಸಾಧುಜೀವನ ನಡೆಸುತ್ತ ಇಹಲೋಕಯಾತ್ರೆಯನ್ನು ಮುಗಿಸಿದರು.

ಸ್ವಾಮೀಜಿಯ ಮತ್ತೊಬ್ಬ ಆಪ್ತ ಅನುಯಾಯಿ ಸುಬ್ರಮಣ್ಯ ಅಯ್ಯರರು ಅವರನ್ನು ಮೊದಲ ಸಲ ಸಂಧಿಸಿದ ಘಟನೆ ಸ್ವಾರಸ್ಯಕರವಾಗಿದೆ. ಕ್ರೈಸ್ಟ್ ಕಾಲೇಜಿನ ವಿದ್ಯಾರ್ಥಿಯಾಗಿದ್ದ ಇವರಿಗೆ ಹಿಂದೂಧರ್ಮದ ಬಗ್ಗೆ ಒಂದು ಬಗೆಯ ತಿರಸ್ಕಾರ ಬೆಳೆದಿತ್ತು. ಒಂದು ದಿನ ಇವರು ತಮ್ಮ ಕೆಲವು ಸಹಪಾಠಿಗಳೊಂದಿಗೆ, ಸ್ವಲ್ಪ ತಮಾಷೆ ಮಾಡಿ ನೋಡಲೆಂದು ಸ್ವಾಮೀಜಿಯ ಬಳಿಗೆ ಹೋದರು. ಸ್ವಾಮಿಗಳು ಬಹಳ ತಿಳಿದವರೆಂದೂ ಇಂಗ್ಲಿಷ್ ಬಲ್ಲವರೆಂದೂ ಗೊತ್ತಾಗಿತ್ತು. ಅವರನ್ನು ತಬ್ಬಿಬ್ಬುಗೊಳಿಸುವ ಒಂದಿಷ್ಟು ಪ್ರಶ್ನೆಗಳನ್ನು ಸಿದ್ಧಪಡಿಸಿಕೊಂಡು ಹೊರಟಿದ್ದರು.

ಬಂದು ನೋಡುತ್ತಾರೆ–ಸ್ವಾಮೀಜಿ ಕುರ್ಚಿಯಲ್ಲಿ ಒರಗಿಕೊಂಡು ಹುಕ್ಕಾಸೇದುತ್ತ ಅರೆ ಜಾಗೃತಾವಸ್ಥೆಯಲ್ಲಿದ್ದಾರೆ; ಅವರ ಮನಸ್ಸು ಯಾವುದೋ ಉನ್ನತ ಪ್ರಜ್ಞೆಯ ಹಂತದಲ್ಲಿ ದ್ದಂತಿದೆ; ಅವರ ವ್ಯಕ್ತಿತ್ವದಿಂದ ಒಂದು ಬಗೆಯ ಅಲೌಕಿಕ ಶಕ್ತಿ ಹೊಮ್ಮುತ್ತಿರುವಂತೆ ಭಾಸವಾಗುತ್ತಿದೆ! ಇದನ್ನು ಕಂಡು ಈ ತರುಣರು ಏಕೋ ಸ್ವಲ್ಪ ಹಿಮ್ಮೆಟ್ಟಿದರು. ಬಳಿಕ ಅವರ ಪೈಕಿ ಹೆಚ್ಚು ಧೈರ್ಯಶಾಲಿಯಾದವನೊಬ್ಬ, ತನ್ನ ಬಲವನ್ನು ಒಗ್ಗೂಡಿಸಿಕೊಂಡು ಕೇಳಿದ, “ಸ್ವಾಮೀಜಿ, ದೇವರು ಎಂದರೇನು?”

‘ಹ್ಞೂ, ಈಗ ನೋಡು ತಮಾಷೆ’ ಎಂದು ಮನಸ್ಸಿನಲ್ಲೇ ಅಂದುಕೊಂಡರು ತರುಣರು. ಸ್ವಾಮೀಜಿ ಈಗ ಏನು ಉತ್ತರಿಸಿದರೂ ಅವರನ್ನು ಮತ್ತೆ ಪ್ರಶ್ನೆಗಳಿಂದ ಕಾಡಿ, ಸುಸ್ತು ಮಾಡಿಸ ಬಹುದು ಎಂದು ಅವರು ನಿರೀಕ್ಷಿಸಿದರು. ಇದೊಂದು ಒಳ್ಳೆಯ ಪ್ರಶ್ನೆಯೇ. ಸುಲಭವಾಗಿ ಉತ್ತರಿಸಲು ಸಾಧ್ಯವಿಲ್ಲದಂತಹ ಪ್ರಶ್ನೆಯೇ. ಆದರೆ ಸ್ವಾಮೀಜಿಗೆ ಈ ಪ್ರಶ್ನೆಯೇನು ಹೊಸದೆ? ಇಂತಹ ಎಷ್ಟೋ ಪ್ರಶ್ನೆಗಳನ್ನು ಅವರೇ ಕೇಳಿದ್ದರಷ್ಟೇ ಅಲ್ಲ, ಅದಕ್ಕೆ ಉತ್ತರವನ್ನೂ ಕಂಡು ಕೊಂಡಿದ್ದರು.

ಒಂದು ನಿಮಿಷ ಸ್ವಾಮೀಜಿ ಮೌನವನ್ನು ಮುರಿಯದೆ, ಅದೇ ಉನ್ನತ ಭಾವದಲ್ಲಿ ಮುಳುಗಿ ದ್ದರು. ಬಳಿಕ ನಿಧಾನವಾಗಿ ತಮ್ಮ ಪ್ರಖರ ನೇತ್ರಗಳನ್ನು ಅರ್ಧ ತೆರೆದು ಆ ತರುಣನನ್ನೇ ಕೇಳಿದರು:

“ಅದಿರಲಿ, ನೀನು ಹೇಳು–ಶಕ್ತಿ ಎಂದರೇನು?”

ತನಗೆ ತಿಳಿದಿದ್ದ ವಿಜ್ಞಾನದ ಆಧಾರದ ಮೇಲೆ ಅವನು ಉತ್ತರಿಸಿದ. ಆದರೆ ಸ್ವಾಮೀಜಿ ತಮ್ಮ ಆಕ್ಷೇಪಗಳಿಂದ ಅದನ್ನು ಒಮ್ಮೆಗೇ ತಳ್ಳಿಹಾಕಿಬಿಟ್ಟರು. ಆಗ ಅವನ ಸ್ನೇಹಿತರು ಅವನ ಸಹಾಯಕ್ಕೆ ಬಂದು ತಲೆಗೊಂದೊಂದು ವಿವರಣೆ ನೀಡಿದರು. ಸ್ವಾಮೀಜಿ ಈ ಎಲ್ಲ ಉತ್ತರಗಳನ್ನೂ ಅಸಮರ್ಪಕವೆಂದು ಸಾಬೀತು ಮಾಡಿದರು. ಇದನ್ನು ಕಂಡು ಆ ತರುಣರೆಲ್ಲ ಪೆಚ್ಚಾಗಿ ನಿಂತರು. ಈಗ ಸ್ವಾಮೀಜಿ ಹುಕ್ಕಾವನ್ನು ಬದಿಗಿಟ್ಟು, ಪೂರ್ಣ ಬಹಿರ್ಮುಖರಾಗಿ ಕುಳಿತು ಹೇಳಿದರು, “ಏನಪ್ಪ ಇದು? ನೀವು ಪ್ರತಿದಿನವೂ ಉಪಯೋಗಿಸುತ್ತಿರುವ ‘ಶಕ್ತಿ’ಯಂತಹ ಸರಳ ಪದವನ್ನೇ ವಿವರಿಸಲು ನಿಮಗೆ ಸಾಧ್ಯವಿಲ್ಲ. ಹೀಗಿರುವಾಗ, ‘ದೇವರು’ ಎಂಬುದನ್ನು ವಿವರಿಸುವಂತೆ ನನ್ನನ್ನು ಕೇಳುತ್ತಿದ್ದೀರಲ್ಲ!” ಬಳಿಕ ಸ್ವಾಮೀಜಿ, ದೇವರು ಹಾಗೂ ಶಕ್ತಿ–ಇವೆರಡರ ಕಲ್ಪನೆಯನ್ನು ಎಂತಹ ಉನ್ನತ ಸ್ತರದಲ್ಲಿ ವಿವರಿಸಿದರೆಂದರೆ, ತಾವು ಅತಿ ಕುಬ್ಜರಾಗುತ್ತಿರುವಂತೆ ಆ ತರುಣರಿಗೆ ಭಾಸವಾಯಿತು. ಬಳಿಕ ಅವರು ಕೇಳಿದ ಇತರ ಪ್ರಶ್ನೆಗಳಿಗೂ ಇದೇ ಬಗೆಯ ಉತ್ತರ ಸಿಕ್ಕಿತು.

ಈಗ ಈ ಸ್ನೇಹಿತರೆಲ್ಲ ಅಲ್ಲಿಂದ ಹೊರಟರು. ಇವರ ಜೊತೆಯಲ್ಲಿ ಬಂದಿದ್ದ ಯುವಕ ಸುಬ್ರಮಣ್ಯ ಮಾತ್ರ ಸ್ವಾಮೀಜಿಯ ಮಾತು, ವ್ಯಕ್ತಿತ್ವವನ್ನು ಕಂಡು ಬಹಳ ಮೆಚ್ಚಿಕೊಂಡಿದ್ದ. ಆದ್ದರಿಂದ ಅವನು ಅಲ್ಲೇ ಉಳಿದುಕೊಂಡ. ಅಂದು ಸಂಜೆ ಸ್ವಾಮೀಜಿ ಕೆಲವರೊಂದಿಗೆ ಸಮುದ್ರ ತೀರದಲ್ಲಿ ಅಡ್ಡಾಡಿಕೊಂಡುಬರಲು ಹೊರಟಾಗ ತಾನೂ ಅವರೊಂದಿಗೆ ಹೋದ. ದಾರಿಯಲ್ಲಿ ಸ್ವಾಮೀಜಿ, ಯುವಕ ಸುಬ್ರಮಣ್ಯನನ್ನು ಕೇಳಿದರು, “ಏನಪ್ಪ, ನನ್ನ ಜೊತೆಯಲ್ಲಿ ಕುಸ್ತಿ ಆಡ ಬಲ್ಲೆಯಾ?” ಅದಕ್ಕವನು “ಓಹೋ, ಆಗಬಹುದು” ಎಂದು ಉತ್ತರಿಸಿದ. ಆಗ ಸ್ವಾಮೀಜಿ ತಮಾಷೆಯಾಗಿ ಹೇಳಿದರು, “ಹಾಗಿದ್ದರೆ ಬಾ, ಕುಸ್ತಿ ಆಡೋಣ!”

ಸ್ವಾಮೀಜಿಯ ಈ ಬಗೆಯ ಉತ್ಸಾಹವನ್ನು ಕಂಡು ಸುಬ್ರಹ್ಮಣ್ಯ ಅಯ್ಯರ್, ಮುಂದೆ ಅವರನ್ನು ‘ಪೈಲ್ವಾನ್ ಸ್ವಾಮಿ’ ಎಂದು ಕರೆಯುತ್ತಿದ್ದರು.

ಸ್ವಾಮೀಜಿಯವರನ್ನು ಅಮೆರಿಕಕ್ಕೆ ಕಳಿಸಿಕೊಟ್ಟ ಯುವಕರಲ್ಲಿ ಬಾಲಾಜಿರಾವ್ ಒಬ್ಬರು. ತಮ ಗಿಂತ ಎರಡು ವರ್ಷ ಕಿರಿಯರಾದ ಇವರನ್ನು ಸ್ವಾಮೀಜಿ ತುಂಬ ವಿಶ್ವಾಸದಿಂದ ಕಾಣುತ್ತಿದ್ದರು. ಆಗಾಗ ಅವರ ಮನೆಗೂ ಹೋಗಿ ಆತಿಥ್ಯವನ್ನು ಸ್ವೀಕರಿಸುತ್ತಿದ್ದರು. ಒಮ್ಮೆ ಇವರ ಮನೆಯಲ್ಲಿ ದ್ದಾಗ ಸ್ವಾಮೀಜಿ ಹಠಯೋಗವನ್ನು ಪ್ರದರ್ಶಿಸಿಲು ತಮ್ಮ ಕೈಬೆರಳನ್ನು ಚಾಕುವಿನಿಂದ ಕುಯ್ದು ಕೊಂಡಾಗ ಒಂದು ತೊಟ್ಟು ರಕ್ತವೂ ಬರಲಿಲ್ಲವೆಂದು ಇವರ ಮಗ ಹೇಳಿದ್ದಾನೆ. “ಮುಂದಿನ ಜನ್ಮದಲ್ಲಿ ನಿನಗೆ ಮುಕ್ತಿ ದೊರಕುತ್ತದೆ; ಆದರೆ ಈ ಜನ್ಮದಲ್ಲಿ ದೇವರು ನಿನ್ನನ್ನು ಪರಿಪರಿಯಾಗಿ ಪರೀಕ್ಷಿಸುತ್ತಾನೆ” ಎಂದು ಬಾಲಾಜಿಗೆ ಸ್ವಾಮೀಜಿ ಹೇಳಿದ್ದರಂತೆ. ಇದಾಗಿ ಸ್ವಲ್ಪಕಾಲದಲ್ಲೇ ಇವರ ಇಬ್ಬರು ಗಂಡು ಮಕ್ಕಳು ಒಬ್ಬರಾದಮೇಲೊಬ್ಬರು ಮೃತ್ಯುವಶರಾದರು. ಈ ವಿಷಯ ತಿಳಿದಾಗ ಸ್ವಾಮೀಜಿ ಬಾಲಾಜಿಗೆ ಬರೆದ ವಿಷಾದಪತ್ರದಲ್ಲಿ ಅತ್ಯಂತ ಹೃದಯ ಸ್ಪರ್ಶಿಯೂ ಉದಾತ್ತವೂ ಆದ ಭಾವಗಳನ್ನು ವ್ಯಕ್ತಪಡಿಸಿದ್ದಾರೆ. ಮುಂದೆ ಇವರು ಬ್ಯಾಂಕ್ ಒಂದರ ಮ್ಯಾನೇಜರ್ ಆದರು. ಆದರೆ ೧೯ಂ೭ರಲ್ಲಿ ಆಕಸ್ಮಿಕವೊಂದರಲ್ಲಿ ತಮ್ಮ ಸರ್ವಸ್ವವನ್ನೂ ಕಳೆದುಕೊಂಡರು. ಇಷ್ಟಾದರೂ ಸ್ವಾಮೀಜಿಯ ಸಂಪರ್ಕದ ಪರಿಣಾಮವಾಗಿ, ಕಷ್ಟಗಳೆಲ್ಲವನ್ನೂ ಗಟ್ಟಿ ಹೃದಯದಿಂದ ಎದುರಿಸಿದರು.

ಈ ನವಯುವಕರೆಲ್ಲ ಸ್ವಾಮೀಜಿಯ ವ್ಯಕ್ತಿತ್ವದ ಆಕರ್ಷಣೆಗೆ ಸಿಲುಕಿ, ಅವರನ್ನು ಮುತ್ತಿಕೊಂಡ ಕಥೆ ತುಂಬ ರಮ್ಯ. ಹೀಗೆ ಆಕರ್ಷಿತರಾದವರಲ್ಲೊಬ್ಬರಾದ ಕೆ. ವ್ಯಾಸರಾವ್ ಅದನ್ನು ಹೀಗೆ ಬಣ್ಣೀಸುತ್ತಾರೆ:

“ಕಲ್ಕತ್ತ ವಿಶ್ವವಿದ್ಯಾಲಯದ ಒಬ್ಬ ಯುವ ಪದವೀಧರ; ಮುಂಡನ ಮಾಡಿಕೊಂಡ ಶಿರ; ಭಕ್ತಿ-ಗೌರವ ಭಾವವನ್ನು ಉದ್ದೀಪನಗೊಳಿಸುವ ವ್ಯಕ್ತಿತ್ವ; ತ್ಯಾಗ ಸೂಚಕವಾದ ಗೈರಿಕವಸನ ಧಾರಿ; ಸಂಸ್ಕೃತ-ಇಂಗ್ಲಿಷ್ ಭಾಷೆಗಳಲ್ಲಿ ನಿರರ್ಗಳ ವಾಗ್ಜಾಲ! ಪ್ರತ್ಯುತ್ತರ ಕೊಡುವಲ್ಲಿ ಅಸಾಮಾನ್ಯ ಪ್ರತಿಭಾಶಾಲಿ! ಸುಮಧುರ ಧ್ವನಿಯಲ್ಲಿ ಬಿಚ್ಚುಕಂಠದಿಂದ ಹಾಡತೊಡಗಿದನೆಂದರೆ ಪರಮಾತ್ಮನೊಂದಿಗೆ ತಾದಾತ್ಮ್ಯ ಹೊಂದುವಂತಹ ತನ್ಮಯತೆ! ಇಷ್ಟೆಲ್ಲ ಆದರೂ ಅವನೊಬ್ಬ ಪರಿವ್ರಾಜಕ ಸಂನ್ಯಾಸಿಯಷ್ಟೆ! ಆತ ಗಟ್ಟಿಮುಟ್ಟಾದ ಮೈಕಟ್ಟಿನ ಜಟ್ಟಿ; ಮಾತುಗಳಲ್ಲಿ ಹೊಟ್ಟೆ ಹುಣ್ಣಾಗಿಸುವ ಹಾಸ್ಯ. ಪವಾಡ ಮಾಡುವವರನ್ನು ಕಂಡರೆ ಎಲ್ಲಿಲ್ಲದ ತಿರಸ್ಕಾರ. ಆತ ಒಳ್ಳೇ ಅಡಿಗೆಯನ್ನು ಸವಿಯಬಲ್ಲ; ಹುಕ್ಕಾ-ಸಿಗಾರುಗಳ ಸವಿಯನ್ನೂ ಬಲ್ಲವನು; ಆದರೆ ಈ ಕ್ಷಣದಲ್ಲಿ ಎಲ್ಲವನ್ನೂ ಬೆರಳ ತುದಿಯಿಂದ ಎಸೆದುಬಿಡಬಲ್ಲ ತ್ಯಾಗಮನೋಭಾವ! ಎಂಥವರಾದರೂ ತಲೆದೂಗುವಂತೆ ತಲೆಬಾಗುವಂತೆ ಮಾಡುವ ಪ್ರಾಮಾಣಿಕ ವೈರಾಗ್ಯ! ಈ ವೈಪರೀತ್ಯಗಳನ್ನು ಕಂಡು ಎಂತೆಂತಹ ಪದವೀಧರರೂ ತಬ್ಬಿಬ್ಬಾದರು. ಪರಮಾತ್ಮರಂಗದ ದಿವ್ಯಾದ್ಭುತ ಆಖಾಡ ದಲ್ಲಿ, ಈತ ಹೇಗೆ ತನ್ನ ಅಪ್ರತಿಮ ಶಕ್ತಿಪ್ರದರ್ಶನ ಮಾಡುತ್ತ ಆಯುಧಗಳನ್ನು ಝಳಪಿಸುತ್ತ ದೃಢವಾಗಿ ನಿಲ್ಲಬಲ್ಲ ಎಂಬುದನ್ನು ಆ ಯುವಕರೆಲ್ಲ ನೋಡಿದರು. ಹಾಗೆಯೇ, ಅತ್ಯುನ್ನತ ವಿಚಾರ ಗಳ ಸ್ತರದಿಂದ ಕೆಳಗಿಳಿದು ಸರಸ ಸಂಭಾಷಣೆಯಲ್ಲಿ ತೊಡಗಿದಾಗ ಈತ ಹಾಸ್ಯ-ನಗೆಚಾಟಿಕೆಗಳಲ್ಲಿ ಎಷ್ಟು ಆಸಕ್ತನಾಗಬಲ್ಲ ಎಂಬುದನ್ನೂ ಕಂಡರು. ಅವನ ನಿರ್ದಾಕ್ಷಿಣ್ಯ ಮಾತುಗಳನ್ನೂ ಚತುರೋಕ್ತಿಗಳನ್ನೂ ಕೇಳಿ ಮಾರುಹೋದರು. ಆದರೆ ಇವೆಲ್ಲಕ್ಕಿಂತ ಹೆಚ್ಚಾಗಿ ಜನರಿಗೆ ಅವನು ಪ್ರಿಯನಾದದ್ದು ತನ್ನ ಪ್ರಾಮಾಣಿಕ ದೇಶಪ್ರೇಮದಿಂದ. ಈ ಯುವ ಸಂನ್ಯಾಸಿಯು ಪ್ರಪಂಚದ ಎಲ್ಲ ಆಸೆಗಳನ್ನು ತ್ಯಾಗಮಾಡಿ, ಸಕಲ ಬಂಧನಗಳಿಂದಲೂ ಬಿಡಿಸಿಕೊಂಡಿದ್ದರೂ ಅವನಲ್ಲಿ ಒಂದು ವ್ಯಾಮೋಹವಿತ್ತು–ಅದು ತನ್ನ ರಾಷ್ಟ್ರದ ಮೇಲೆ; ಒಂದು ವ್ಯಥೆಯಿತ್ತು–ತನ್ನ ರಾಷ್ಟ್ರದ ಅವನತಿಯನ್ನು ಕಂಡು. ಈ ವ್ಯಥೆಯಿಂದ ನೊಂದು ಆತ ಭಾವಭಾರತನಾಗಿ ಮಾತನಾಡುತ್ತಿದ್ದರೆ ಕೇಳುಗರು ಅಷ್ಟೇ ತನ್ಮಯತೆಯಿಂದ ಮೂಕರಾಗಿ ಕುಳಿತಿರುತ್ತಿದ್ದರು. ಹೂಗ್ಲಿಯಿಂದ ತಾಮ್ರ ಪರ್ಣಿಯವರೆಗೆ ಸಂಚರಿಸಿ ಬಂದಿದ್ದ ಈ ಯುವ ಸಂನ್ಯಾಸಿ ನಮ್ಮರಾಷ್ಟ್ರದ ಯುವಜನರ ಷಂಡತನವನ್ನು ಅತ್ಯಂತ ಕಠಿಣವಾದ ಶಬ್ದಗಳಿಂದ ಖಂಡಿಸಿದ; ರಾಷ್ಟ್ರದ ದುಃಸ್ಥಿತಿಯನ್ನು ಕಂಡು ಆಕ್ರಂದಿಸಿದ. ಅವನ ಮಾತುಗಳು ಮಿಂಚಿನಂತೆ ಹೊಳೆಯುತ್ತಿದ್ದುವು; ಖಡ್ಗದಂತೆ ಕತ್ತರಿಸುತ್ತಿ ದ್ದುವು. ಹೀಗೆ ಆತ ಎಲ್ಲರ ಮನ ಸೂರೆಗೊಂಡ; ಕೆಲವರಲ್ಲಿ ಉತ್ಸಾಹವನ್ನು ತುಂಬಿದ; ಮತ್ತೆ ಕೆಲವು ಆಪ್ತರಲ್ಲಿ ಅವಿನಾಶಿಯಾದ ಶ್ರದ್ಧೆಯ ಜ್ಯೋತಿಯನ್ನು ಬೆಳಗಿಸಿದ.”

ಇನ್ನೊಬ್ಬ ಶಿಷ್ಯ ಅವರ ಕುರಿತಾಗಿ ಬರೆಯುತ್ತಾನೆ:

“ಸ್ವಾಮೀಜಿ ತಮ್ಮ ಅತ್ಯುನ್ನತ ವಿಚಾರಧಾರೆಯನ್ನು ಕೇಳುಗರ ಬುದ್ಧಿಗೆ ಗ್ರಾಹ್ಯವಾಗುವ ರೀತಿಯಲ್ಲಿ ತಿಳಿಸುವುದಕ್ಕಾಗಿ ಮತ್ತೆಮತ್ತೆ ಅವರ ಮಟ್ಟಕ್ಕೆ ಇಳಿದುಬರಬೇಕಾಗುತ್ತಿತ್ತು. ಎಷ್ಟೋ ಸಲ ಅವರು, ತಮ್ಮ ಶಿಷ್ಯರು ಮುಂದೆ ಕೇಳಬಹುದಾದ ಪ್ರಶ್ನೆಗಳನ್ನು ಮೊದಲೇ ಊಹಿಸಿ, ಅವುಗಳಿಗೆ ಸೂಕ್ತ ಉತ್ತರಗಳನ್ನು ನೀಡಿ ತೃಪ್ತಿಪಡಿಸುತ್ತಿದ್ದರು. ಮತ್ತೆ ಕೆಲವೊಮ್ಮೆ, ಅನೇಕರ ಆಲೋಚನೆಗಳನ್ನು ಏಕಕಾಲದಲ್ಲಿ ಗ್ರಹಿಸಿ, ಅವರೆಲ್ಲರ ಸಂದೇಹಗಳನ್ನೂ ಏಕಕಾಲದಲ್ಲಿ ಪರಹರಿ ಸುತ್ತಿದ್ದರು. ‘ಇತರರ ಮನಸ್ಸಿನ ಆಲೋಚನೆಗಳನ್ನು ಅವರು ಹೇಗೆ ಅರಿಯಲು ಸಾಧ್ಯವಾಯಿತು?’ ಎಂದು ಕೇಳಿದರೆ ಅವರು ಉತ್ತರಿಸುತ್ತಿದ್ದರು, ‘ಸಂನ್ಯಾಸಿಗಳು ಮಾನವರ ವೈದ್ಯರು; ಅವರು ಔಷಧಿಯನ್ನು ಸೂಚಿಸುವ ಮೊದಲು ಕಾಯಿಲೆಯನ್ನು ಪತ್ತೆಹಚ್ಚಿರುತ್ತಾರೆ!’

“ಯಾರ ಮೇಲೆ ಅವರ ಕೃಪಾದೃಷ್ಟಿ ಬೀಳುತ್ತಿತ್ತೋ ಅಂಥವರ ವಿಷಯದಲ್ಲಿ ಸ್ವಾಮೀಜಿ ಕ್ಷಮಾಶೀಲರಾಗಿರುತ್ತಿದ್ದರಾದರೂ, ಅವರ ಬಳಿಯಲ್ಲಿ ವಾಸಿಸುವವರು ಮಾತ್ರ ಭಯಂಕರ ಸಿಡಿ ಮದ್ದಿನ ಸಾಮೀಪ್ಯದಲ್ಲಿದ್ದಂತೆ ಸದಾ ಎಚ್ಚರಿಕೆಯಿಂದಿರಬೇಕಾಗಿತ್ತು. ಒಂದು ಕೆಟ್ಟ ಆಲೋಚನೆ ನಮ್ಮ ಮನಸ್ಸಿನಲ್ಲಿ ಸುಳಿದರೂ ಸಾಕು, ಅದು ಅವರಿಗೆ ಗೊತ್ತಾಗಿಬಿಡುತ್ತಿತ್ತು. ಅವರ ತುಟಿಯಲ್ಲಿ ಆಗ ಲಾಸ್ಯವಾಡುವ ವಿಶಿಷ್ಟ ಮುಗುಳ್ನಗೆಯಿಂದಲೋ ಅವರ ಮಾತಿನ ಧಾಟಿಯಿಂದಲೋ ಅದನ್ನು ಗುರುತಿಸಬಹುದಾಗಿತ್ತು.”

ಅಮೆರಿಕಕ್ಕೆ ಹೋಗಬೇಕೆಂಬ ತಮ್ಮ ಆಲೋಚನೆಯನ್ನು ಸ್ವಾಮೀಜಿ ಈ ವೇಳೆಗೆ ಬಹಿರಂಗ ಪಡಿಸಿದ್ದರು. ಮದರಾಸಿನ ತಮ್ಮ ಆಪ್ತ ಶಿಷ್ಯವರ್ಗಕ್ಕೆ ಅವರು ಹೇಳಿದರು: “ನಮ್ಮ ಧರ್ಮವನ್ನು ಪ್ರಸಾರಗೈಯಬೇಕಾದ ಕಾಲವೀಗ ಒದಗಿಬಂದಿದೆ. ಸನಾತನ ಮಹರ್ಷಿಗಳಿಂದ ರೂಪುಗೊಂಡ ಹಿಂದೂ ಧರ್ಮವು ಎಚ್ಚೆತ್ತು ಚಲನಶೀಲವಾಗಬೇಕಾದ ಕಾಲ ಈಗ ಸನ್ನಿಹಿತವಾಗಿದೆ. ಪರ ರಾಷ್ಟ್ರೀಯರು ನಮ್ಮ ಸನಾತನ ಧರ್ಮದ ಕೋಟೆಯನ್ನು ಧ್ವಂಸ ಮಾಡಲು ಪ್ರಯತ್ನಿಸುತ್ತಿರು ವಾಗ ನಾವು ಸುಮ್ಮನೆ ನೋಡುತ್ತ ನಿಂತಿರಬೇಕೆ? ಆ ಕೋಟೆಯು ನಾಶವಾಗಲಾರದೆಂಬ ಹುಚ್ಚುನಂಬಿಕೆಯಿನ್ನೂ ದೂರವಾಗಿಲ್ಲವೆಂಬಂತೆ ಕಾಣುತ್ತಿದೆ. ನಮ್ಮ ಪೂರ್ವಿಕರಂತೆ ನಾವೂ ನಮ್ಮ ಧರ್ಮದ ವೈಭವವನ್ನು ಎಲ್ಲೆಡೆಯೂ ಸಾರೋಣವೆ, ಇಲ್ಲವೆ ನಮ್ಮ ಸಣ್ಣ ಸಣ್ಣ ಊರು- ಕೇರಿ-ವಠಾರಗಳೊಳಗೇ ಕೂಪಮಂಡೂಕಗಳಂತೆ ಅವಿತಿಟ್ಟುಕೊಂಡಿರೋಣವೆ? ಅಥವಾ ತಲೆ ಯೆತ್ತಿ ನಿಂತು, ಜಗತ್ತಿನ ಇತರ ರಾಷ್ಟ್ರಗಳ ಚಿಂತನಲೋಕವನ್ನು ಪ್ರವೇಶಿಸಿ, ಭಾರತದ ಅಭಿವೃದ್ಧಿಗೆ ಅವರು ನೆರವು ನೀಡುವಂತೆ ಪ್ರಭಾವ ಬೀರೋಣವೆ? ಈಗ ಭಾರತವು ಮತ್ತೊಮ್ಮೆ ಉನ್ನತಿಯ ದಾರಿಗೆ ಬರಬೇಕಾದರೆ ಅದು ಒಗ್ಗಟ್ಟಾಗಿ ನಿಂತು, ಶಕ್ತಿಶಾಲಿಯಾಗಬೇಕು. ಅದು ತನ್ನ ಅಳಿದುಳಿದ ಶಕ್ತಿಯನ್ನೆಲ್ಲ ಒಗ್ಗೂಡಿಸಿಕೊಳ್ಳಲೇಬೇಕು.”

ಇಲ್ಲಿ ಸ್ವಾಮೀಜಿ ಕ್ರಾಂತಿಕಾರಿಪುರುಷನಾಗಿ, ರಾಷ್ಟ್ರನಿರ್ಮಾಪಕನಾಗಿ, ಸನಾತನ ಧರ್ಮದ ಪುನರುದ್ಧಾರಕನಾಗಿ ಪ್ರಕಟವಾಗುತ್ತಿರುವುದನ್ನು ಕಾಣುತ್ತೇವೆ. ಪರಕೀಯರು ನಮ್ಮ ಸನಾತನ ಧರ್ಮವನ್ನು ನಾಮಾವಶೇಷ ಮಾಡುವ ಪ್ರಯತ್ನದಲ್ಲಿ ತೊಡಗಿರುವುದನ್ನು ಎತ್ತಿತೋರಿಸಿ, ಅದನ್ನು ಪ್ರತಿಭಟಿಸುವಂತೆ ಸ್ವಾಮೀಜಿ ಕರೆ ನೀಡುತ್ತಿದ್ದಾರೆ. ಸಮಗ್ರ ಭಾರತವು ಆರುನೂರಕ್ಕೂ ಹೆಚ್ಚು ಚಿಕ್ಕಪುಟ್ಟ ಪ್ರಾಂತಗಳಾಗಿ ಒಡೆದು, ಆ ಪ್ರಾಂತಗಳ ಆಡಳಿತಗಾರರು ತಮ್ಮನ್ನು ಚಕ್ರವರ್ತಿ ಗಳೆಂದು ಭಾವಿಸಿಕೊಂಡು, ಯಾವಾಗಲೂ ತಮ್ಮತಮ್ಮೊಳಗೇ ಹೊಡೆದಾಡಿಕೊಂಡಿದ್ದು, ಬ್ರಿಟಿಷ ರಿಗೆ ಮತ್ತಷ್ಟು ಸಹಾಯಮಾಡಿಕೊಡುತ್ತಿದ್ದ ಕಾಲ ಅದು. ‘ಒಡೆದು ಆಳು’ವ ಬ್ರಿಟಿಷರ ಯೋಜನೆಗೆ ಹೇಳಿ ಮಾಡಿಸಿದಂತಹ ಪರಿಸ್ಥಿತಿ ಅದು. ತಮ್ಮದು ಎಂದು ಭಾರತೀಯರು ಹೆಮ್ಮೆ ಯಿಂದ ಹೇಳಿಕೊಳ್ಳುವಂತಹದು ಯಾವುದೂ ಕಂಡುಬರದಿದ್ದ ದುಸ್ಥಿತಿ ಅದು. ಇಂತಹ ಭರತ ಖಂಡವು ವಿಶ್ವದ ಸಮ್ಮುಖದಲ್ಲಿ ಹೆಮ್ಮೆಯಿಂದ ತಲೆಯೆತ್ತಿ ನಿಲ್ಲಬೇಕಾದರೆ ಜನರು ಈ ಎಲ್ಲ ಕಟ್ಟುಕಟ್ಟಲೆಗಳನ್ನೂ ಸಂಕುಚಿತತೆಯನ್ನೂ ಮೀರಿ ನಿಂತು ಒಗ್ಗಟ್ಟಾಗಬೇಕು; ಹಾಗೂ ಭಾರತ ದಲ್ಲಿ ಇನ್ನು ಯಾವಯಾವ ಬಗೆಯ ಶಕ್ತಿ ಉಳಿದಿದೆಯೋ ಅವೆಲ್ಲವೂ ಒಂದುಗೂಡಿ ಮಹಾಶಕ್ತಿ ಯಾಗಿ ವ್ಯಕ್ತವಾಗಬೇಕೆಂದು ಅವರಿಲ್ಲಿ ಕರೆ ನೀಡುತ್ತಿದ್ದಾರೆ. ಅಂದು ಸ್ವಾಮೀಜಿ ಆಡಿದ ಮಾತು ಕೇವಲ ಐವತ್ತೇ ವರ್ಷಗಳಲ್ಲಿ ಕಾರ್ಯಗತವಾಗಿ, ಸ್ವತಂತ್ರ ಹಾಗೂ ಸಮಗ್ರ ಭಾರತದ ನಿರ್ಮಾಣ ವಾದದ್ದನ್ನು ನಾವಿಂದು ಕಾಣುತ್ತಿದ್ದೇವೆ–ಪಾಕಿಸ್ತಾನದ ಅಚಾತುರ್ಯವೊಂದನ್ನು ಬಿಟ್ಟು.

ಹೀಗೆ ಸ್ವಾಮೀಜಿ ನವಜಾಗೃತಿಗೆ ಕರೆ ನೀಡಿದಾಗ ಅದನ್ನು ಆಲಿಸಿದವರಿಗೆಲ್ಲ ಅನ್ನಿಸಿತು– ಇವರು ಧರ್ಮಸಂಸ್ಥಾಪನೆ ಹಾಗೂ ಧರ್ಮಪ್ರಸಾರಕ್ಕಾಗಿಯೇ ಜನ್ಮವೆತ್ತಿದವರು, ಎಂದು. ಆದ್ದರಿಂದ ಪಾಶ್ಚಾತ್ಯ ರಾಷ್ಟ್ರಗಳಿಗೆ ಹೋಗಬೇಕೆಂಬ ಅವರ ಯೋಜನೆಯ ಹಿಂದಿನ ಉದ್ದೇಶ ಜನರಿಗೆ ಅರ್ಥವಾಯಿತು. ಅಷ್ಟೇ ಅಲ್ಲ, ಅವರು ಹೋಗಲೇಬೇಕೆಂದು ಎಲ್ಲರೂ ಒಕ್ಕೊರಳಿನಿಂದ ಒತ್ತಾಯಪೂರ್ವಕವಾಗಿ ಕೇಳಿಕೊಳ್ಳಲಾರಂಭಿಸಿದರು. ಮದರಾಸಿನ ಉತ್ಸಾಹೀ ಜನ, ಮುಖ್ಯವಾಗಿ ಯುವಕರು, “ಸ್ವಾಮೀಜಿ, ಈ ಯೋಜನೆ ಕಾರ್ಯಗತವಾಗಲೇ ಬೇಕು. ಏನೇ ಆಗಲಿ, ನೀವು ವಿಶ್ವ ಧರ್ಮ ಸಮ್ಮೇಳನದಲ್ಲಿ ಭಾಗವಹಿಸಲೇಬೇಕು. ಖಂಡಿತವಾಗಿಯೂ ನೀವು ಯಶಸ್ವಿಯಾಗು ತ್ತೀರಿ. ನೀವೊಂದು ಅದ್ಭುತವನ್ನೇ ಸಾಧಿಸುವುದರಲ್ಲಿ ಸಂಶಯವಿಲ್ಲ. ನಾವೆಲ್ಲರೂ ನಿಮಗೆ ಬೆಂಬಲವಾಗಿದ್ದೇವೆ” ಎಂದು ಪ್ರೋತ್ಸಾಹಿಸಿದರು. ಮತ್ತು ಅಷ್ಟಕ್ಕೇ ಸುಮ್ಮನಾಗಲಿಲ್ಲ; ಅವರ ಪ್ರಯಾಣದ ಖರ್ಚಿಗಾಗಿ ವಂತಿಗೆ ಸಂಗ್ರಹಿಸಲು ಹೊರಟರು.

ಆದರೆ ಸ್ವಾಮೀಜಿ ಇನ್ನೂ ಒಂದು ನಿರ್ಧಾರವನ್ನು ತೆಗೆದುಕೊಂಡಿರಲಿಲ್ಲ. ಶಿಕಾಗೋ ಸಮ್ಮೇಳನದಲ್ಲಿ ಭಾಗವಹಿಸಬೇಕೆಂಬ ಆಲೋಚನೆ ಅವರ ಮನಸ್ಸಿನಲ್ಲಿ ಬಹಳ ದಿನಗಳಿಂದಲೂ ಇದ್ದೇ ಇದ್ದಿತ್ತಾದರೂ ಅದೇ ಸರಿಯಾದ ಮಾರ್ಗವೆಂದಾಗಲಿ, ಅದೇ ಭಗವಂತನ ಇಚ್ಛೆಯೆಂದಾ ಗಲಿ ಅವರಿಗಿನ್ನೂ ಸಂಪೂರ್ಣ ವಿಶ್ವಾಸವುಂಟಾಗಿರಲಿಲ್ಲ. ಯಾವುದೋ ಹುಚ್ಚು ಉತ್ಸಾಹದ ಭರ ದಲ್ಲಿ ಈ ಯುವಕರ ಮಾತು ಕೇಳಿಕೊಂಡು ಅಮೆರಿಕಕ್ಕೆ ಹೋದರೂ, ಏನೂ ಪ್ರಯೋಜನವಾಗದೆ ‘ಬಂದ ದಾರಿಗೆ ಸುಂಕವಿಲ್ಲ’ ಎಂದು ಹಿಂದಿರುಗುವಂತಾದರೆ?–ಎಂಬ ಸಂಶಯವೇಳತೊಡ ಗಿತ್ತು. ಆದ್ದರಿಂದ ಇದೇ ಜಗನ್ಮಾತೆಯ ಇಚ್ಛೆ ಎಂದು ಸ್ಪಷ್ಟವಾಗುವವರೆಗೂ ತಾವು ಅಂತಿಮ ನಿರ್ಧಾರಕ್ಕೆ ಬರಬಾರದೆಂದು ನಿಶ್ಚಯಿಸಿದರು.

ಇತ್ತ ಚಂದಾ ಎತ್ತಲು ಹೋಗಿದ್ದ ಯುವಕರು ಹಲವಾರು ಗಣ್ಯವ್ಯಕ್ತಿಗಳನ್ನು ಸಂಪರ್ಕಿಸಿದರು. ಸ್ವಾಮೀಜಿ ಅಮೆರಿಕಕ್ಕೆ ಹೋಗಬೇಕೆಂಬ ಸಲಹೆಯನ್ನು ಬಹುತೇಕ ಜನ ಸ್ವಾಗತಿಸಿ, ತಮ್ಮತಮ್ಮ ಕೈಲಾದಷ್ಟು ಧನಸಹಾಯವನ್ನು ನೀಡಿದರು. ದಿವಾನರೊಬ್ಬರು ಐನೂರು ರೂಪಾಯಿಗಳನ್ನು ಕೈಯೆತ್ತಿ ಕೊಟ್ಟರು. ದಾತೃಗಳ ಪಟ್ಟಿಯಲ್ಲಿ ಇವರೇ ಅಗ್ರಗಣ್ಯರು. ಆದರೆ ಹತ್ತುಸಾವಿರ ರೂಪಾಯಿ ನೀಡುವುದಾಗಿ ವಾಗ್ದಾನ ಮಾಡಿದ್ದ ರಾಮನಾಡಿನ ಅರಸ ಮಾತ್ರ ಒಂದು ಚಿಕ್ಕಾಸನ್ನೂ ಕೊಡಲಿಲ್ಲ! ವಂತಿಗೆ ಕೇಳಲು ಬಂದ ಯುವಕರಿಗೆ, “ನಾನಿದಕ್ಕೆಲ್ಲ ಹಣ ಕೊಡಲು ಸಾಧ್ಯವಿಲ್ಲ” ಎಂದು ಕಡ್ಡಿ ಮುರಿದಂತೆ ಹೇಳಿ ಕಳಿಸಿಬಿಟ್ಟಿದ್ದ. ಇದ್ದಕ್ಕಿದ್ದಂತೆ ಅವನ ಮನಸ್ಸು ಬದಲಾಗಲು ಕಾರಣವೇನೆಂದರೆ ಅವನ ಆಪ್ತರಾದ ಕೆಲವರು ಅವನ ಕಿವಿಯೂದಿ ಮನಸ್ಸು ಕೆಡಿಸಿದ್ದುದು. ಇವರ ಮಾತು ಕೇಳಿ ಅವನು ಆಲೋಚಿಸಿದ್ದ, ‘ಅಲ್ಲ, ಈತ ಮೊದಲೇ ಬಂಗಾಳಿ, ಜೊತೆಗೆ ಮಹಾ ಪ್ರಚಂಡ. ಸಂನ್ಯಾಸಿಯ ವೇಷ ಧರಿಸಿ ರಾಜ್ಯದಿಂದ ರಾಜ್ಯಕ್ಕೆ ಹೋಗುತ್ತಿದ್ದಾನೆ. ಇವನ ಮನಸ್ಸಿನಲ್ಲೇನಿದೆಯೋ ಯಾರಿಗೆ ಗೊತ್ತು. ಈತ ಮುಂದೆ ಏನಾದರೂ ರಾಜಕೀಯ ಮಾಡಿ ನನಗೇ ಆಪಾಯ ತಂದೊಡ್ಡಿದರೂ ಆಶ್ಚರ್ಯವಿಲ್ಲ!’ ಎಂದು. ಈ ವಿಷಯವೆಲ್ಲ ತಿಳಿದು ಬಂದಾಗ ಸ್ವಾಮೀಜಿಗಾದ ಬೇಸರ-ಸಂಕಟ ಅಷ್ಟಿಷ್ಟಲ್ಲ. ಯಾರಿಂದ ತಾವು ಬಹಳಷ್ಟು ನೆರವನ್ನು ನೆಚ್ಚಿಕೊಂಡಿದ್ದರೋ ಆತ ಕೈಬಿಟ್ಟನೆಂಬುದು ಒಂದು ಕಾರಣ. ಆದರೆ ಅದಕ್ಕಿಂತ ಹೆಚ್ಚಾಗಿ ಆತ ತಮ್ಮ ವಿಶ್ವಾಸಾರ್ಹತೆಯನ್ನೇ ಶಂಕಿಸಿದನಲ್ಲ ಎಂದು ಅವರಿಗೆ ಬಹಳ ನೋವಾಯಿತು.

ಈ ಬೇಸರದ ಜೊತೆಗೆ, ಸಂಗ್ರಹವಾದ ಹಣದ ಮೊತ್ತವನ್ನು ಕಂಡಾಗ ಅವರಿಗೆ ಹತಾಶೆಯೇ ಆಯಿತು. ಎಲ್ಲ ಸೇರಿ ಸುಮಾರು ಒಂದು ಸಾವಿರ ರೂಪಾಯಿಯಷ್ಟಾಗಿತ್ತೆಂದು ತೋರುತ್ತದೆ. ಈ ಮೊತ್ತ ಏತಕ್ಕೂ ಆಗುವಂತಿರಲಿಲ್ಲ. ಆದ್ದರಿಂದ ಅವರ ಮನಸ್ಸಿನಲ್ಲಿ ಮತ್ತೆ ಸಂಶಯದ ಅಲೆಗಳೆದ್ದುವು: ‘ಹಾಗಾದರೆ ನಾನು ನನ್ನಿಷ್ಟದಂತೆ ನಡೆಯುತ್ತಿದ್ದೇನೆಯೆ? ಅಥವಾ ಇದಕ್ಕೆ ಜಗನ್ಮಾತೆಯ ಬೆಂಬಲವಿದೆಯೆ?’ ಈ ತೊಳಲಾಟದ ನಡುವೆಯೇ ಅವರೊಂದು ದಿಟ್ಟ ನಿರ್ಧಾರಕ್ಕೆ ಬಂದರು–ಏನೇ ಆಗಲಿ, ಜಗನ್ಮಾತೆಯ ಇಚ್ಛೆಯೇನೆಂದು ಸ್ಪಷ್ಟವಾಗುವವರೆಗೂ ಮುಂದಿನ ಹೆಜ್ಜೆಯನ್ನಿಡುವುದೇ ಇಲ್ಲ ಎಂದು. ಬಳಿಕ ಹಣವನ್ನು ಸಂಗ್ರಹಿಸಿ ತಂದಿದ್ದ ತಮ್ಮ ಶಿಷ್ಯರಿಗೆ ಹೇಳಿದರು, “ನೋಡಿ, ನಾನೀಗ ಜಗನ್ಮಾತೆ ತನ್ನ ಇಂಗಿತವನ್ನು ಸ್ಪಷ್ಟಪಡಿಸಲೇಬೇಕೆಂದು ಹಟ ಹಿಡಿದಿದ್ದೇನೆ. ನಾನು ಅಮೆರಿಕೆಗೆ ಹೋಗಬೇಕೆಂಬುದು ಆಕೆಯ ಇಚ್ಛೆಯೇ ಆದಲ್ಲಿ ಅವಳದನ್ನು ರುಜುವಾತು ಪಡಿಸಲೇಬೇಕು. ಏಕೆಂದರೆ ಈ ಕೆಲಸ ಕಣ್ಣು ಕಾಣದ ಕಗ್ಗತ್ತಲಲ್ಲಿ ಹೆಜ್ಜೆಯಿಟ್ಟಂತೆ. ಆದ್ದರಿಂದ ಈಗ ನೀವು ಸಂಗ್ರಹಿಸಿರುವ ಹಣವನ್ನೆಲ್ಲ ಬಡಬಗ್ಗರಿಗೆ ಹಂಚಿಬಿಡಿ. ನಾನು ಹೋಗಬೇಕೆಂಬುದು ಆಕೆಯ ಇಚ್ಛೆಯೇ ಆಗಿದ್ದಲ್ಲಿ ಇದಕ್ಕೆ ಬೇಕಾದ ಹಣ ತಾನಾಗಿಯೇ ಬರಬೇಕು, ಬರುತ್ತದೆ.” ಸಂಗ್ರಹಿಸಿದ ಹಣವನ್ನು ಬಡವರಿಗೆ ಹಂಚಿಬಿಡಿ ಎಂದಾಗ ಆ ಶಿಷ್ಯರೆಲ್ಲ ಅಧೀರರಾದರು. ಅಲ್ಲದೆ ಅವರಿಗೆ ಸ್ವಲ್ಪ ಸಂಕಟವಾಗಿದ್ದರೂ ಆಶ್ಚರ್ಯವಿಲ್ಲ. ಆದರೆ ಸ್ವಾಮೀಜಿಯ ಮಾತಿನಲ್ಲಿದ್ದ ದೃಢತೆಯನ್ನು ಕಂಡು ಮರುಮಾತಿಲ್ಲದೆ ಅವರ ಆಜ್ಞೆಯನ್ನು ಪಾಲಿಸಿ ದರು. ಈಗ ಸ್ವಾಮೀಜಿಗೆ ತಮ್ಮ ಹೆಗಲಮೇಲಿದ್ದ ದೊಡ್ಡದೊಂದು ಭಾರವನ್ನು ಕೆಳಗಿಳಿಸಿದಷ್ಟು ನೆಮ್ಮದಿಯೆನಿಸಿತು.

ಈಗ ಅವರು ಮತ್ತೆ ಮಾತುಕತೆ, ಬೋಧನೆ, ಭಜನೆಗಳನ್ನು ಆರಂಭಿಸಿದರು. ಹೊಸಹೊಸ ಜನರನ್ನು ಭೇಟಿಯಾಗುತ್ತಿದ್ದರು. ಅದರ ಜೊತೆಗೆ, ತಮಗೆ ಮಾರ್ಗದರ್ಶನ ನೀಡುವಂತೆ ಜಗನ್ಮಾತೆಯನ್ನೂ ಗುರುದೇವನನ್ನೂ ತೀವ್ರವಾಗಿ ಪ್ರಾರ್ಥಿಸಲಾರಂಭಿಸಿದರು. ಕೆಲವೊಮ್ಮೆ ಗಾಢ ಧ್ಯಾನಮಗ್ನರಾಗುತ್ತಿದ್ದರು. ಮತ್ತೆ ಕೆಲವೊಮ್ಮೆ ದೇವಿಯ ಸಂಕೀರ್ತನೆಯಲ್ಲಿ ತೊಡಗಿ, ಭಾವ ಭರಿತರಾಗಿಬಿಡುತ್ತಿದ್ದರು. ಆಗ ಸುತ್ತಲ್ಲಿದ್ದವರೂ ಅದೇ ಭಾವದ ಸೆಳೆತಕ್ಕೆ ಸಿಕ್ಕಿ ತೇಲುತ್ತಿದ್ದರು.

ಒಂದು ದಿನ ಮನ್ಮಥನಾಥರ ಅಡಿಗೆಯವನು ತಮ್ಮ ಹುಕ್ಕಾವನ್ನು ಆಸೆಗಣ್ಣಿನಿಂದ ನೋಡು ತ್ತಿರುವುದನ್ನು ಸ್ವಾಮೀಜಿ ಗಮನಿಸಿದರು. ಅದು ಮೈಸೂರು ಮಹಾರಾಜರು ಪ್ರೀತಿಪೂರ್ವಕ ಒತ್ತಾಯದಿಂದ ಕೊಟ್ಟಿದ್ದ ಸುಂದರವಾದ, ಬೀಟೆ ಮರದ ಹುಕ್ಕಾ. ಸ್ವಾಮೀಜಿ, “ಏನು, ನಿನಗದು ಬೇಕೇನು?” ಎಂದು ಕೇಳಿದರು. ಅವರು ಸುಮ್ಮನೆ ತಮಾಷೆಗೆ ಕೇಳುತ್ತಿರಬೇಕು ಎಂದು ಅವನು ಭಾವಿಸಿದನೇನೋ, ಅವರ ಕಡೆಗೊಮ್ಮೆ ನೋಡಿ ಮತ್ತೆ ಅದನ್ನೇ ದಿಟ್ಟಿಸಿದ. ಸ್ವಾಮೀಜಿ ಮತ್ತೆ ಕೇಳಿದರು, “ನಿನಗದು ಬೇಕೆ?” ಅವನಿಗೆ ‘ಬೇಕು’ ಎನ್ನಲು ಧೈರ್ಯ ಸಾಲಲಿಲ್ಲ. ಆದರೆ ಅವನ ಕಣ್ಣುಗಳು ಮಾತ್ರ ಹುಕ್ಕಾದಿಂದ ಕದಲುತ್ತಿಲ್ಲ! ತಕ್ಷಣ ಸ್ವಾಮೀಜಿ ಅದನ್ನು ಅವನ ಕೈಯಲ್ಲಿ ಇಟ್ಟು, “ತೆಗೆದುಕೋ, ಇದು ನಿನಗೆ!” ಎಂದರು. ಆ ಅಡಿಗೆಯವನಿಗೆ ಅದನ್ನು ನಂಬಲೇ ಸಾಧ್ಯ ವಾಗಲಿಲ್ಲ. ಇಷ್ಟು ಹೊತ್ತು ತಾನು ಕಣ್ಣಲ್ಲೇ ಹಿಡಿದುಕೊಂಡಿದ್ದ ಹುಕ್ಕಾ ಈಗ ತನ್ನ ಕೈಗೇ ಸೇರಿ ದಾಗ ಅವನ ಆನಂದಾಶ್ಚರ್ಯಗಳಿಗೆ ಪಾರವೇ ಇಲ್ಲದಂತಾಗಿಬಿಟ್ಟಿತು. ಸ್ವಾಮಿ ವಿವೇಕಾನಂದ ರಂತಹ ಬ್ರಹ್ಮಜ್ಞಾನಿಗಳಿಂದ ಮುಮುಕ್ಷುಗಳು ಜ್ಞಾನವನ್ನು ಪಡೆದುಕೊಂಡರು; ವಿವೇಕಿಗಳು ಸದ್ಗುಣಗಳನ್ನು ಪಡೆದುಕೊಂಡರು; ನೊಂದವರು ಶಾಂತಿಯನ್ನು ಪಡೆದುಕೊಂಡರು; ಈ ಅಡಿಗೆ ಯವನು ಹುಕ್ಕಾ ಪಡೆದುಕೊಂಡ! ಯಾರ್ಯಾರಿಗೆ ಏನೇನು ಬೇಕೋ, ಏನೇನು ಲಭ್ಯವೋ, ಅದನ್ನು ಪಡೆದುಕೊಳ್ಳುವುದೇ ಸಹಜ. ಆದರೆ ಸ್ವಾಮೀಜಿಯ ತ್ಯಾಗಬುದ್ಧಿಯ ತೀವ್ರತೆಯನ್ನು ಕಂಡು ಸುತ್ತ ಲಿದ್ದವರು ಬೆರಗಾದರು. ಏಕೆಂದರ ಎಂತಹ ಒತ್ತಡದ ಸಮಯದಲ್ಲಿಯೂ ಅವರಿಗೊಂದಿಷ್ಟು ಸುಖ ಕೊಡುತ್ತಿದ್ದುದು ಅದೊಂದೇ. ಅದನ್ನೇ ಅವರೀಗ ಹಿಂದುಮುಂದು ನೋಡದೆ ಕೊಟ್ಟುಬಿಟ್ಟಿ ದ್ದಾರೆ! ಹೀಗೆ ತಮ್ಮ ಬಳಿ ಏನೇನು ಇರುತ್ತಿತ್ತೋ ಅದನ್ನು ಯಾರಾದರೂ ಬಯಸಿದರೆ ಅವರಿಗೇ ಕೊಟ್ಟುಬಿಡುವುದು ಸ್ವಾಮೀಜಿಯ ಸ್ವಭಾವವೇ ಆಗಿತ್ತು. ಉದಾಹರಣೆಗೆ ಮುಂದೆ ಅವರು ಅಮೆರಿಕೆಯಲ್ಲಿ ಶ್ರೀಮತಿ ವುಡ್ಸ್ ಎಂಬವಳ ಮನೆಯಲ್ಲಿದ್ದಾಗ ಆಕೆಯ ಮಗ ಪ್ರಿನ್ಸ್ ವುಡ್ಸ್, ಅವರ ಬಳಿಯಿದ್ದ ದಂಡವನ್ನು ಬಹಳ ಇಷ್ಟಪಟ್ಟ. ಅವರು ಭಾರತದಲ್ಲಿ ಪರಿವ್ರಾಜಕರಾಗಿ ತಿರುಗಾಡುತ್ತಿದ್ದಾಗ, ಅವರೊಂದಿಗೆ ಸಕಲ ತೀರ್ಥಕ್ಷೇತ್ರಗಳ ಪವಿತ್ರ ಸ್ಪರ್ಶ ಮಾಡಿದ್ದ ದಂಡ ಅದು. ಅವರ ಪಾಲಿಗೆ ಅದೊಂದು ಅಮೂಲ್ಯ ವಸ್ತು. ಆ ಹುಡುಗ ಅದನ್ನು ಬಯಸಿದಾಗ ಸ್ವಾಮೀಜಿ ಕ್ಷಣಕಾಲವೂ ಚಿಂತಿಸದೆ, ಅವನಿಗದನ್ನು ಕೊಡುತ್ತ ಹೇಳಿದರು, “ನೀನು ಯಾವುದನ್ನು ಮೆಚ್ಚಿಕೊಳ್ಳುತ್ತೀಯೋ ಅದಾಗಲೇ ನಿನ್ನದು! ತೆಗೆದುಕೋ.” ಬಳಿಕ ಅವನ ತಾಯಿ ಶ್ರೀಮತಿ ವುಡ್ಸ್, ಅವರ ಪೆಟ್ಟಿಗೆಯನ್ನೂ ಕಂಬಳಿಯನ್ನೂ ಇಟ್ಟುಕೊಳ್ಳಲು ಬಯಸಿದಳು. ಅದನ್ನೂ ಅವರು ತಕ್ಷಣ ಕೊಟ್ಟುಬಿಟ್ಟರು. ಮುಂದೆ ತಮ್ಮ ಕೊನೆಯ ದಿನಗಳಲ್ಲೊಮ್ಮೆ ಅವರು ಕೆಲವು ಯುವಕ ರೊಂದಿಗೆ ತುಂಬ ಸಲಿಗೆಯಿಂದ ಮಾತನಾಡುತ್ತಿದ್ದರು. ಅವರ ಜೇಬಿನಲ್ಲಿದ್ದ ಚಿನ್ನದ ಗಡಿಯಾರಕ್ಕೆ ಸೇರಿದ ಚಿನ್ನದ ಸರಪಳಿ ಅವರ ಕೊರಳನ್ನು ಬಳಸಿತ್ತು. ಅವರ ಸುಂದರ ಮೈಬಣ್ಣ ದೊಂದಿಗೆ ಆ ಸರಪಳಿ ಬಹಳ ಚೆನ್ನಾಗಿ ಹೊಂದಿಕೆಯಾಗುತ್ತಿತ್ತು. ಬಹುಶಃ ಅದನ್ನು ಅವರ ಪಾಶ್ಚಾತ್ಯ ಶಿಷ್ಯರು ಯಾರೋ ಪ್ರೀತಿಯ ಕಾಣಿಕೆಯಾಗಿ ಕೊಟ್ಟಿರಬೇಕು. ಆ ಯುವಕರಲ್ಲೊಬ್ಬ ಅದನ್ನು ಕೈಯಲ್ಲಿ ಮುಟ್ಟಿ “ಅಹಾ, ಎಷ್ಟು ಸುಂದರವಾಗಿದೆ!” ಎಂದು ಉದ್ಗರಿಸಿದ. ತಕ್ಷಣ ಸ್ವಾಮೀಜಿ “ಓ, ನಿನಗದು ಇಷ್ಟವಾಯಿತೇ? ಹಾಗಾದರೆ ಅದು ನಿನ್ನದು! ಆದರೆ, ಮಗು, ಅದನ್ನು ಮಾರಬೇಡ. ಅದನ್ನು ನೆನಪಿಗಾಗಿ ಇಟ್ಟುಕೋ” ಎನ್ನುತ್ತ ಅವನ ಕೈಯಲ್ಲಿಟ್ಟೇಬಿಟ್ಟರು! ಆತ ಅತ್ಯಂತ ಆನಂದಿತನಾದನೆಂದು ಹೇಳಬೇಕಾಗಿಯೇ ಇಲ್ಲ. ಅಂತಹ ಅಮೂಲ್ಯ ವಸ್ತುವನ್ನು ಅವರು ಅಷ್ಟು ಸುಲಭವಾಗಿ ತೆಗೆದುಕೊಟ್ಟ ಪರಿಯನ್ನು ಕಂಡವರು ನಿಬ್ಬೆರಗಾದರು. ಒಮ್ಮೆ ಸ್ವಾಮೀಜಿ ಹೇಳುತ್ತಾರೆ, “ತ್ಯಾಗವೆಂದರೆ ನಮ್ಮ ಬಳಿಯಿರುವಂಥದೇನಾದರದನ್ನು ತ್ಯಾಗ ಮಾಡು ವುದು. ಯಾರು ಎಲ್ಲವನ್ನೂ ಹೊಂದಿದ್ದೂ ಅದರ ಬಗ್ಗೆ ನಿರ್ಲಿಪ್ತನಾಗಿರುತ್ತಾನೆಯೋ ಅಂಥವನೇ ನಿಜವಾದ ವಿರಕ್ತ. ಯಾರ ಬಳಿ ಏನೇನೂ ಇಲ್ಲವೋ ಅವನೊಬ್ಬ ಭಿಕಾರಿ, ಅಷ್ಟೆ. ಅವನು ಯಾರಿಗೇನು ಕೊಟ್ಟಾನು?” ಆದರೆ ಸ್ವಾಮೀಜಿಯಂತಹ ‘ಪರಮಭಿಕಾರಿ’ಯ ಬಳಿಯಿದ್ದ ಅತ್ಯಂತ ಸಾಮಾನ್ಯ ವಸ್ತುವೂ ಮಹತ್ತರ ಪಾವನಕಾರಿಯಾದದ್ದು. ಅದನ್ನು ಪಡೆದವರು ನಿಜಕ್ಕೂ ಧನ್ಯರೇ ಸರಿ.

ಮದರಾಸಿನಲ್ಲಿದ್ದ ಈ ದಿನಗಳಲ್ಲಿ ಸ್ವಾಮೀಜಿಗೊಂದು ವಿಚಿತ್ರ ಅನುಭವವಾಗುತ್ತಿತ್ತು. ಕೆಲವು ಪ್ರೇತಗಳು ಅವರ ಮನಸ್ಸನ್ನು ಕದಡುವ ಪ್ರಯತ್ನ ಮಾಡುತ್ತಿದ್ದುವು. ಇವು ಅವರ ಕಿವಿಯಲ್ಲಿ ಸುಳ್ಳುಸುಳ್ಳು ಸುದ್ದಿಗಳನ್ನು ತಂದು ಉಸುರುತ್ತಿದ್ದುವು. ಈ ಸುದ್ದಿಗಳು ಸುಳ್ಳು ಎಂಬುದು ಪರೀಕ್ಷೆ ಮಾಡಿ ನೋಡಿದ ಮೇಲಷ್ಟೇ ಗೊತ್ತಾಗುತ್ತಿತ್ತು. ಇದರಿಂದ ಅವರು ಗೊಂದಲಕ್ಕೆ ಸಿಕ್ಕಿಕೊಳ್ಳುತ್ತಿ ದ್ದರು. ಇದು ಹೀಗೇ ಕೆಲವು ದಿನ ನಡೆಯಿತು. ಕಡೆಗೆ ಸ್ವಾಮೀಜಿ ಬೇಸತ್ತು, ಅವುಗಳಿಗೆ ಛೀಮಾರಿ ಹಾಕಿ, “ನನ್ನನ್ನೇಕೆ ಸುಮ್ಮನೆ ಹೀಗೆ ಕಾಡುತ್ತೀರಿ?” ಎಂದು ಕೇಳಿದರು. ಆಗ ಅವು ತಮ್ಮ ದಯ ನೀಯ ಸ್ಥಿತಿಯನ್ನು ಹೇಳಿಕೊಂಡು ತಮ್ಮನ್ನು ಉದ್ಧರಿಸುವಂತೆ ಮೊರೆಯಿಟ್ಟುವು. ಅವರ ಗಮನ ವನ್ನು ತಮ್ಮತ್ತ ಸೆಳೆಯುವುದಕ್ಕಾಗಿ ಅವು ಹೀಗೆ ಕಾಡುತ್ತಿದ್ದವಷ್ಟೆ. ಈ ಬಗ್ಗೆ ಸ್ವಾಮೀಜಿ ಅಲೋಚಿ ಸಿದರು. ಅವುಗಳಿಗೆ ಮುಕ್ತಿ ನೀಡಲು ತಾವೇನು ಮಾಡಬೇಕೆಂದು ಅವರಿಗೆ ತಿಳಿಯಲಿಲ್ಲ. ಬಳಿಕ ಒಂದು ದಿನ ಸಮುದ್ರ ತೀರಕ್ಕೆ ಹೋಗಿ, ಕೈಗೆ ಒಂದು ಹಿಡಿ ಮರಳನ್ನು ತೆಗೆದುಕೊಂಡರು. ಧಾನ್ಯಗಳಿಗೆ ಬದಲಾಗಿ ಈ ಮರಳನ್ನೇ ಅರ್ಪಣೆ ಮಾಡಿ, ಆ ಪ್ರೇತಗಳ ಬಿಡುಗಡೆಗಾಗಿ ಹೃತ್ಪೂರ್ವಕವಾಗಿ ಪ್ರಾರ್ಥಿಸಿಕೊಂಡರು. ಅಂದಿನಿಂದಲೇ ಅವು ತೊಂದರೆ ಕೊಡುವುದು ನಿಂತು ಹೋಯಿತು. ಪ್ರೇತಗಳೆಲ್ಲ ಮುಕ್ತವಾದುವು.

ಈ ಮಧ್ಯೆ ಸ್ವಾಮೀಜಿಯ ವಿಷಯವಾಗಿ ತಮ್ಮ ಮದರಾಸೀ ಸ್ನೇಹಿತರಿಂದ ಬಹಳಷ್ಟು ಕೇಳಿ ಆಕರ್ಷಿತರಾಗಿದ್ದ ಹೈದರಾಬಾದಿನ ಕೆಲವರು ತಮ್ಮಲ್ಲಿಗೆ ಕೆಲದಿನಗಳ ಭೇಟಿ ನೀಡಬೇಕೆಂದು ಅವರನ್ನು ಪ್ರಾರ್ಥಿಸಿಕೊಂಡರು. ಅಮೆರಿಕಾ ಯಾತ್ರೆಯ ಬಗೆಗಿನ ಅನಿಶ್ಚಿತತೆಯಿಂದ ತೀವ್ರ ಕ್ಷೋಭೆಗೊಳಗಾಗಿದ್ದ ಸ್ವಾಮೀಜಿಗೆ ಈ ಕರೆ ಅನಿರೀಕ್ಷಿತವಾಗಿತ್ತು. ಆದರೆ ಇದರಲ್ಲಿ ಯಾವುದೋ ಒಂದು ಅವ್ಯಕ್ತವಾದ ಉದ್ದೇಶವಡಗಿರಬೇಕೆಂದು ಅವರಿಗೆ ತೋರಿತು. ಆದ್ದರಿಂದ ಅವರು ತಕ್ಷಣ ಈ ಆಹ್ವಾನವನ್ನು ಅಂಗೀಕರಿಸಿದರು. ಕೂಡಲೇ ಮನ್ಮಥಬಾಬು, ಹೈದರಾಬಾದಿನಲ್ಲಿದ್ದ ಮಧುಸೂದನ ಚಟರ್ಜಿ ಎಂಬ ತಮ್ಮ ಸ್ನೇಹಿತರಿಗೆ ತಂತಿ ಕಳಿಸಿ, ಫೆಬ್ರವರಿ ೧ಂರಂದು ಸ್ವಾಮೀಜಿ ಹೈದರಾಬಾದಿಗೆ ತಲುಪುತ್ತಾರೆಂದು ತಿಳಿಸಿದರು. ಚಟರ್ಜಿಯವರು ನಿಜಾಮ್ ಸರ್ಕಾರದಲ್ಲಿ ಹಿರಿಯ ಎಂಜಿನಿಯರ್ ಆಗಿದ್ದವರು. ಇವರ ನೇತೃತ್ವದಲ್ಲಿ ೯ನೇ ತಾರೀಕಿನಂದು ಹೈದರಾ ಬಾದ್​ಹಾಗೂ ಸಿಕಂದರಾಬಾದಿನ ಪ್ರಮುಖ ನಾಗರಿಕರು ಸಭೆ ಸೇರಿ, ಸ್ವಾಮೀಜಿಗೆ ಉಚಿತ ಸ್ವಾಗತ ವನ್ನು ನೀಡಿ ಬರಮಾಡಿಕೊಳ್ಳಲು ನಿಶ್ಚಯಿಸಿದರು. ಮರುದಿನ ಸ್ವಾಮೀಜಿ ಹೈದರಾಬಾದಿನ ರೈಲು ನಿಲ್ದಾಣದಲ್ಲಿ ಬಂದಿಳಿದಾಗ ಅವರಿಗೊಂದು ದೊಡ್ಡ ಆಶ್ಚರ್ಯ ಕಾದಿತ್ತು. ಸುಮಾರು ೫ಂಂ ಜನರು ಅವರನ್ನು ಸ್ವಾಗತಿಸಲು ಅಲ್ಲಿ ನೆರೆದಿದ್ದರು! ಇವರಲ್ಲಿ ನಿಜಾಮರ ದರ್ಬಾರಿನ ಪ್ರಮುಖ ಅಧಿಕಾರಿಗಳು, ಪಂಡಿತರು, ವಕೀಲರು, ಶ್ರೀಮಂತ ವರ್ತಕರು ಹಾಗೂ ಇತರ ಗಣ್ಯರು ಇದ್ದರು. ಅಲ್ಲದೆ ಸ್ವಾಮೀಜಿಯ ಆತಿಥೇಯರಾದ ಮಧುಸೂದನ ಚಟರ್ಜಿ ಹಾಗೂ ಅವರ ಪುತ್ರ ಕಾಳೀ ಚರಣ ಚಟರ್ಜಿಯವರೂ ಅಲ್ಲಿ ಹಾಜರಿದ್ದರು. ಕಲ್ಕತ್ತದಲ್ಲೇ ಸ್ವಾಮೀಜಿಗೆ ಪರಿಚಿತರಾಗಿದ್ದ ಕಾಳೀಚರಣ ಚಟರ್ಜಿ, ಅವರನ್ನು ಅಲ್ಲಿ ನೆರೆದಿದ್ದವರಿಗೆ ಪರಿಚಯಿಸಿದರು. ಸ್ವಾಮೀಜಿಯ ಕೊರಳಿಗೆ ಹೂಮಾಲೆಗಳ ರಾಶಿಯೇ ಬಿದ್ದಿತು.

ಈ ದೃಶ್ಯವನ್ನು ಕಣ್ಣಾರೆ ಕಂಡವರೊಬ್ಬರು ಹೇಳುವಂತೆ, ಸಂನ್ಯಾಸಿಯೊಬ್ಬರಿಗೆ ಸಿಕ್ಕ ಈ ಭವ್ಯ ಸ್ವಾಗತವು ಹೈದರಬಾದಿನ ಅಷ್ಟೇಕೆ, ಭಾರತದ ಇತಿಹಾಸದಲ್ಲೇ ಅಭೂತಪೂರ್ವವಾದದ್ದು. ಚಕ್ರವರ್ತಿಯೊಬ್ಬನಿಗೆ ನೀಡಬಹುದಾದ ಸ್ವಾಗತಕ್ಕೆ ಅದು ಸಮವಾಗಿತ್ತು. ಅಲ್ಲದೆ ಅಲ್ಲಿ ಸೇರಿದ್ದ ಸಂದಣಿಯೂ ಅಪೂರ್ವವಾದದ್ದು. ಅವರನ್ನು ಸ್ವಾಗತಿಸಲು ನಿಲ್ದಾಣಕ್ಕೆ ಬರಲಾಗದಿದ್ದ ಅನೇಕರು. ಅವರು ಇಳಿದುಕೊಂಡಿದ್ದ ಬಂಗಲೆಗೇ ಬಂದು ಭೇಟಿ ಮಾಡಿದರು.

ಮರುದಿನ ಬೆಳಿಗ್ಗೆ ಸಿಕಂದರಾಬಾದಿನ ಹಿಂದೂಗಳ ಪರವಾಗಿ ನೂರು ಜನರ ಸಮಿತಿಯೊಂದು ಹೂವು ಹಣ್ಣು ಸಿಹಿತಿನಿಸುಗಳ ಕಾಣಿಕೆಯೊಂದಿಗೆ ಸ್ವಾಮೀಜಿಯವರ ದರ್ಶನ ಮಾಡಿ, ತಮ್ಮ ನಗರದಲ್ಲಿ ಅವರೊಂದು ಉಪನ್ಯಾಸ ನೀಡಬೇಕೆಂದು ಕೇಳಿಕೊಂಡರು. ಅವರು ಇದಕ್ಕೆ ಸಮ್ಮತಿಸಿ ದರು. ಸಿಕಂದರಾಬಾದಿನ ಮೆಹಬೂಬ್ ಕಾಲೇಜಿನಲ್ಲಿ ೧೩ನೇ ತಾರೀಕು ಉಪನ್ಯಾಸವನ್ನಿರಿಸುವು ದೆಂದು ತೀರ್ಮಾನವಾಯಿತು. ಬಳಿಕ ಸ್ವಾಮೀಜಿ ಕಾಳೀಚರಣ ಬಾಬುಗಳೊಂದಿಗೆ ಇತಿಹಾಸ ಪ್ರಸಿದ್ಧವಾದ ಗೋಲ್ಕೊಂಡದ ಕೋಟೆಯನ್ನು ಸಂದರ್ಶಿಸಿದರು. ಇಲ್ಲಿಂದ ಹಿಂದಿರುಗುವ ವೇಳೆಗೆ, ನಿಜಾಮರ ಸಂಬಂಧಿಯೂ ನಗರದ ಅಗ್ರಗಣ್ಯ ವ್ಯಕ್ತಿಯೂ ಆದ ನವಾಬ್ ಬಹಾದೂರ್ ಸರ್ ಖುರ್ಷಿದ್ ಝಾ ಎಂಬಾತನ ಆಪ್ತ ಕಾರ್ಯದರ್ಶಿ ಸ್ವಾಮೀಜಿಗಾಗಿ ಕಾಯುತ್ತ ಕುಳಿತಿದ್ದ. “ಖುರ್ಷಿದ್ ಝಾರವರು ತಮ್ಮನ್ನು ನಾಳೆ ದಿನ ಭೇಟಿಯಾಗಲು ಇಚ್ಛಿಸುತ್ತಾರೆ. ಆದ್ದರಿಂದ ತಾವು ನಾಳೆ ಬೆಳಿಗ್ಗೆ ಅರಮನೆಗೆ ಆಗಮಿಸಬೇಕೆಂದು ಕೋರುತ್ತಾರೆ” ಎಂದು ಈತ ತಿಳಿಸಿದ. ಇದಕ್ಕೊಪ್ಪಿ ಸ್ವಾಮೀಜಿ, ಮರುದಿನ ನಿಗದಿತ ಸಮಯಕ್ಕೆ ಕಾಳೀಚರಣ ಬಾಬುಗಳ ಜೊತೆಯಲ್ಲಿ ಅರಮನೆಗೆ ಹೋದರು.

ಅರಮನೆಯಲ್ಲಿ ಅವರನ್ನು ನವಾಬ ಖುರ್ಷಿದ್ ಝಾ ಅತ್ಯಂತ ಗೌರವದಿಂದ ಹಾರ್ದಿಕವಾಗಿ ಬರಮಾಡಿಕೊಂಡ. ಈತ ಪರಮತ ಸಹಿಷ್ಣುತೆಗೆ ಹೆಸರಾಗಿದ್ದವನು. ಇವನು ಕನ್ಯಾಕುಮಾರಿಯಿಂದ ಹಿಮಾಲಯದವರೆಗಿನ ಸಕಲ ಹಿಂದೂ ತೀರ್ಥಕ್ಷೇತ್ರಗಳನ್ನು ಸಂದರ್ಶಿಸಿದ್ದ ಏಕಮಾತ್ರ ಮುಸಲ್ಮಾನನಾಗಿದ್ದ. ಈತ ಸ್ವಾಮೀಜಿಗೆ ನಮಸ್ಕರಿಸಿ, ಅವರಿಗೆ ತನ್ನ ಪಕ್ಕದಲ್ಲೇ ಒಂದು ಆಸನ ವನ್ನು ನೀಡಿದ. ವಿದ್ಯಾವಂತನೂ ಸುಸಂಸ್ಕೃತನೂ ಆದ ಈತನೊಂದಿಗೆ ಸ್ವಾಮೀಜಿ ಸುಮಾರು ಎರಡು ಗಂಟೆಗೂ ಹೆಚ್ಚಿನ ಕಾಲ ಸಂಭಾಷಿಸಿದರು. ಹಿಂದೂಧರ್ಮ ಮಾತ್ರವಲ್ಲದೆ ಇಸ್ಲಾಂ ಹಾಗೂ ಕ್ರೈಸ್ತ ಧರ್ಮದ ಬಗ್ಗೆಯೂ ಅವರು ಆಳವಾಗಿ ಚರ್ಚಿಸಿದರು. ನವಾಬ ತಾನು ನಿರಾಕಾರ ವಾದಿಯೆಂದೂ, ಹಿಂದೂಧರ್ಮವು ಪ್ರತಿಪಾದಿಸುವಂತಹ ಸಾಕಾರವಾದದಲ್ಲಿ ತನಗೆ ಸ್ವಲ್ಪವೂ ನಂಬಿಕೆಯಿಲ್ಲವೆಂದೂ ಹೇಳಿದ. ಆಗ ಸ್ವಾಮೀಜಿ, ದೇವರ ಕಲ್ಪನೆಯು ಹೇಗೆ ಮೆಟ್ಟಿಲುಮೆಟ್ಟಿಲಾಗಿ ವಿಕಾಸ ಹೊಂದುತ್ತ ಬಂದಿತು ಎಂಬುದನ್ನು ವಿವರಿಸಿದರು. ಮನುಷ್ಯನ ಪ್ರತಿಯೊಂದು ಅನು ಭವವೂ ವೈಯಕ್ತಿಕ, ಅಲ್ಲದೆ ಅದು ಮನಸ್ಸು-ಇಂದ್ರಿಯಗಳ ಮಿತಿಗೆ ಒಳಪಟ್ಟಿರುವಂಥದು; ಆದ್ದರಿಂದ ಸಾಕಾರಸ್ಥಿತಿಗೆ ಮಿಗಿಲಾದದ್ದೇನನ್ನೂ ಮೊದಮೊದಲು ಆತ ಕಲ್ಪಿಸಿಕೊಳ್ಳಲಾರ; ಆ ಕಾರಣ, ಅತ್ಯುನ್ನತವಾದ ನಿರಾಕಾರದ ಸ್ಥಿತಿಗೆ ಏರಬೇಕಾದರೆ ಸಗುಣ-ಸಾಕಾರ ದೇವರ ಕಲ್ಪನೆಯು ತುಂಬ ಸಹಾಯಕಾರಿ ಎಂಬುದನ್ನು ತೋರಿಸಿಕೊಟ್ಟರು. ಬಳಿಕ ಹಿಂದೂಧರ್ಮದ ಹಿರಿಮೆಯನ್ನು ನವಾಬನಿಗೆ ಮನಗಾಣಿಸಿದರು. ಹಿಂದೂಧರ್ಮವೊಂದನ್ನುಳಿದು, ಜಗತ್ತಿನ ಪ್ರತಿಯೊಂದು ಧರ್ಮವೂ ಒಬ್ಬೊಬ್ಬ ವ್ಯಕ್ತಿಯನ್ನು ಅವಲಂಬಿಸಿಕೊಂಡಿವೆ–ಕ್ರೈಸ್ತಧರ್ಮವು ಕ್ರಿಸ್ತನನ್ನು, ಇಸ್ಲಾಂ ಧರ್ಮವು ಮಹಮ್ಮದನನ್ನು, ಹೀಗೆ. ಆದರೆ ಹಿಂದೂ ಧರ್ಮ ಮಾತ್ರ ತತ್ತ್ವಗಳನ್ನು ಆಧಿರಿಸಿದೆಯೇ ಹೊರತು ವ್ಯಕ್ತಿಯನ್ನಲ್ಲ. ಹಿಂದೂ ಧರ್ಮದಲ್ಲಿ ಯಾಜ್ಞವಲ್ಕ್ಯರು ಜನಿಸಿದರೇ ಹೊರತು ಯಾಜ್ಞವಲ್ಕ್ಯರಿಂದ ಹಿಂದೂ ಧರ್ಮ ಜನಿಸಿದ್ದಲ್ಲ; ಹಿಂದೂಧರ್ಮದಲ್ಲಿ ವಸಿಷ್ಠರು ಜನ್ಮವೆತ್ತಿದರೇ ಹೊರತು ವಸಿಷ್ಠರಿಂದ ಹಿಂದೂಧರ್ಮ ಜನ್ಮತಾಳಿದ್ದಲ್ಲ. ಆದ್ದರಿಂದ ಹಿಂದೂ ಧರ್ಮವು ಮಾತ್ರ ವಿಶ್ವಧರ್ಮವಾಗುವ ಸಾಮರ್ಥ್ಯವನ್ನೂ ಅರ್ಹತೆಯನ್ನೂ ಪಡೆದಿದೆ ಎಂದು ಸ್ವಾಮೀಜಿ ಹೇಳಿದರು. ಅನಂತರ ಅವರು, ಪ್ರತಿಯೊಂದು ಧಾರ್ಮಿಕ ಭಾವನೆಯೂ ಹೇಗೆ ಮನುಷ್ಯನ ಸತ್ಯಶೋಧನೆಯ ಫಲವಾಗಿ ಉದ್ಭವಿಸಿದೆ ಎಂಬುದನ್ನು ವಿವರಿಸಿ, ಆದಕಾರಣ ಪ್ರತಿಯೊಂದು ಧರ್ಮವೂ ಸತ್ಯ, ಪ್ರತಿಯೊಂದು ನಂಬಿಕೆಯೂ ಸತ್ಯ, ಪ್ರತಿಯೊಂದು ಪಂಥವೂ ಸತ್ಯ ಎಂದು ಒತ್ತಿಹೇಳಿದರು. ಆದ್ದರಿಂದ ಯಾವೊಂದು ಪಂಥವನ್ನೇ ಆಗಲಿ ಅವಲಂಬಿಸಿ ಅತ್ಯಂತ ತೀವ್ರತೆಯಿಂದ ಸಾಧನೆ ಮಾಡಿದಲ್ಲಿ, ಅದರ ಗುರಿಯನ್ನು ಸೇರಬಹುದು; ಹಾಗೂ ತನ್ನಲ್ಲಿರುವ ದೈವತ್ವವನ್ನು ಸಾಕ್ಷಾತ್ಕರಿಸಿಕೊಳ್ಳಬಹುದು ಎಂಬ ತಮ್ಮ ನೂತನ ವಿಚಾರಧಾರೆ ಯನ್ನು ನವಾಬನ ಮುಂದಿಟ್ಟರು. ಕಡೆಯಲ್ಲಿ ಅತ್ಯುನ್ನತವಾದ ಅದ್ವೈತ ವೇದಾಂತದ ತತ್ತ್ವಗಳನ್ನು ಪ್ರಸ್ತಾಪಿಸಿ, ಮಾನವನ ನಿಜಸ್ವರೂಪವು ದೈವತ್ವವೇ ಎಂದು ವಿವರಿಸಿದರು.

ಸ್ವಾಮೀಜಿ ಈ ಅತ್ಯುನ್ನತ ವಿಚಾರಗಳನ್ನೆಲ್ಲ ತಮ್ಮ ಅನನುಕರಣೀಯ ಶೈಲಿಯಲ್ಲಿ ವರ್ಣಿಸು ತ್ತಿದ್ದಂತೆ ಅವರ ವಿಶಾಲವಾದ ಕಂಗಳು ಮತ್ತಷ್ಟು ಪ್ರಖರವಾಗಿ ಬೆಳಗುತ್ತಿದ್ದುವು; ಅವರ ಮುಖಮಂಡಲವು ವಿಶೇಷ ತೇಜಸ್ಸಿನಿಂದ ಕಂಗೊಳಿಸುತ್ತಿತ್ತು. ಅವರ ಇಡೀ ವ್ಯಕ್ತಿತ್ವದಿಂದ ಒಂದು ದಿವ್ಯ ಚೈತನ್ಯ ಹೊರಹೊಮ್ಮುತ್ತಿತ್ತು. ಹೀಗೆ ಮಾತನಾಡುವ ಸಂದರ್ಭದಲ್ಲಿ ತಮಗರಿ ವಿಲ್ಲದಂತೆಯೇ ಅವರು, ತಾವು ಸನಾತನ ಧರ್ಮವನ್ನು ಪ್ರಸಾರ ಮಾಡುವುದಕ್ಕಾಗಿ ವಿಶ್ವಧರ್ಮ ಸಮ್ಮೇಳನದಲ್ಲಿ ಭಾಗವಹಿಸಲು ಅಮೆರಿಕೆಗೆ ಹೋಗಬೇಕೆಂದಿರುವ ವಿಷಯವನ್ನು ಹೊರಗೆಡವಿ ದರು. ಈ ವೇಳೆಗಾಗಲೇ ಅವರ ಮಾತುಗಳಿಂದ ತೀವ್ರವಾಗಿ ಪ್ರಭಾವಿತನಾಗಿದ್ದ ನವಾಬ ತಕ್ಷಣ, “ಸ್ವಾಮೀಜಿ ನಿಮ್ಮ ಈ ಕಾರ್ಯಕ್ಕೆ ಸಹಾಯವಾಗಿ ನಾನು ಒಂದುಸಾವಿರ ರೂಪಾಯಿಗಳನ್ನು ಕೊಡಲು ಸಿದ್ಧನಿದ್ದೇನೆ. ದಯವಿಟ್ಟು ಸ್ವೀಕರಿಸಬೇಕು” ಎಂದ. ಇದೊಂದು ಅದ್ಭುತ ಸನ್ನಿವೇಶವೇ ಸರಿ. ಹಿಂದೂ ಸಂನ್ಯಾಸಿಯೊಬ್ಬನು ಹಿಂದೂಧರ್ಮದ ಪ್ರಸಾರಕ್ಕಾಗಿ ಪಾಶ್ಚಾತ್ಯ ರಾಷ್ಟ್ರಕ್ಕೆ ಹೊರ ಟಾಗ, ವಂತಿಗೆ ಕೊಡುವ ವಿಷಯದಲ್ಲಿ ಹಿಂದೂ ರಾಜನೇ ಹಿಂದುಳಿದರೆ, ಕಟ್ಟಾ ಮುಸಲ್ಮಾನ ರಾಜನೊಬ್ಬ ಧನಸಹಾಯ ಮಾಡಲು ಮುಂದಾಗಿದ್ದಾನೆ! ಆದರೆ ನವಾಬ ಅಷ್ಟು ದೊಡ್ಡ ಮೊತ್ತ ವನ್ನು ನೀಡಲು ಮುಂದಾದರೂ ಸ್ವಾಮೀಜಿ, “ನವಾಬ್ ಸಾಹೇಬ್, ನಾನು ಹೊರಡಲು ಇನ್ನೂ ಸಕಾಲವೊದಗಿಲ್ಲ. ನಾನು ಭಗವಂತನ ಆದೇಶಕ್ಕಾಗಿ ಕಾಯುತ್ತಿದ್ದೇನೆ. ಹೊರಡುವುದು ನಿಶ್ಚಯ ವಾದ ಕೂಡಲೇ ನಿನಗೆ ತಿಳಿಸುತ್ತೇನೆ” ಎಂದು ಅದನ್ನು ನಿರಾಕರಿಸಿದರು. ಬಳಿಕ ಅವನಿಗೆ ಕೃತಜ್ಞತೆಗಳನ್ನರ್ಪಿಸಿ ಬೀಳ್ಗೊಂಡರು.

ತಾವು ಸರ್ವಧರ್ಮ ಸಮ್ಮೇಳನದಲ್ಲಿ ಭಾಗವಹಿಸುವುದಕ್ಕಾಗಿಯೂ ಹಿಂದೂ ಧರ್ಮವನ್ನು ಪ್ರಸಾರ ಮಾಡುವುದಕ್ಕಾಗಿಯೂ ಅಮೆರಿಕೆಗೆ ಹೋಗಬೇಕೆಂದಿರುವುದಾಗಿ ಸ್ವಾಮೀಜಿ ಸಾಮಾನ್ಯ ವಾಗಿ ಹೇಳುತ್ತಿದ್ದರಾದರೂ, ತಮ್ಮ ಪ್ರಯಾಣದ ಖರ್ಚಿಗಾಗಿ ವಂತಿಗೆ ಸಂಗ್ರಹಿಸಲು ಹೋಗಿದ್ದ ಆಪ್ತ ಶಿಷ್ಯರಿಗೆ ಅವರು ತಮ್ಮ ಮನದಿಂಗಿತವನ್ನು ಸ್ಪಷ್ಟವಾಗಿ ತಿಳಿಸಿದ್ದರು–“ನಾನು ಹಿಂದೂ ಧರ್ಮವನ್ನು ವಿಶ್ವಧರ್ಮವೆಂದು ಹೇಳುವುದು, ಬಹುಶಃ ನಿಮ್ಮಲ್ಲಿ ಕೆಲವರು ಭಾವಿಸಿರುವ ಸೀಮಿತ ಅರ್ಥದಲ್ಲಲ್ಲ. ಆ ಸಮ್ಮೇಳನದಲ್ಲಿ ಇತರ ಧರ್ಮಗಳ ಪ್ರತಿನಿಧಿಗಳು ಪ್ರಾಯಶಃ ಪ್ರಯತ್ನಿಸಲಿರುವಂತೆ, ನಾನೂ ನಮ್ಮ ಹಿಂದೂಧರ್ಮವನ್ನು ಹೊಗಳಿ ಅಟ್ಟಕ್ಕೇರಿಸಿ, ಇತರ ಧರ್ಮಗಳ ಮೇಲೆ ಅದರ ಆಧಿಪತ್ಯವನ್ನು ಸ್ಥಾಪಿಸಲು ಪ್ರಯತ್ನಿಸುವೆನೆಂದು ನಿರೀಕ್ಷಿಸಲೇಬೇಡಿರಿ. ಯಾವುದೇ ಬಗೆಯ ಪೂರ್ವಷರತ್ತುಗಳ ಮೇಲೆ ನಾನು ಹೋಗಲಾರೆ. ಅಲ್ಲದೆ, ಕಡೆಗೆ ನಾನು ಆ ಸಮ್ಮೇಳನದಲ್ಲಿ ಪಾಲ್ಗೊಳ್ಳದೆಯೂ ಇರಬಹುದು. ನನಗೆ ಅತ್ಯಂತ ಸೂಕ್ತವೆಂದು ತೋರುವ ಮಾರ್ಗವನ್ನು ಅನುಸರಿಸಲು ನಿಮ್ಮ ಸಮ್ಮತಿಯಿರುವುದಾದಲ್ಲಿ ಮಾತ್ರ ನಾನು ಅಮೆರಿಕೆಗೆ ಹೋಗಲು ಸಿದ್ಧನಿದ್ದೇನೆ.” ಇದಕ್ಕೆ ಆ ಶಿಷ್ಯರು ಸರ್ವಾನುಮತದಿಂದ, ಹೃತ್ಪೂರ್ವಕವಾಗಿ ಒಪ್ಪಿದ್ದರು.

ನವಾಬನಿಂದ ಬೀಳ್ಕೊಂಡು ಹೊರಟ ಸ್ವಾಮೀಜಿ, ಹೈದರಾಬಾದಿನ ಸುಂದರ ಪ್ರೇಕ್ಷಣೀಯ ಸ್ಥಳಗಳಾದ ಚಾರ್​ಮಿನಾರ್, ಮೆಕ್ಕಾ ಮಸೀದಿ, ನಿಜಾಮರ ಅರಮನೆಗಳು, ಸಾಲಾರ್​ಜಂಗ್ ವಸ್ತುಸಂಗ್ರಹಾಲಯ, ಬಷೀರ್ ಬಾಗ್ ಮೊದಲಾದುವನ್ನು ವೀಕ್ಷಿಸಿದರು. ಮರುದಿನ ಬೆಳಿಗ್ಗೆ, ರಾಜ್ಯದ ಪ್ರಧಾನಮಂತ್ರಿಯಾದ ಸರ್ ಅಶ್ಮಾನ್ ಝಾ, ಉಪದಿವಾನನಾದ ನರೇಂದ್ರ ಕೃಷ್ಣ ಬಹಾದ್ದೂರ್ ಹಾಗೂ ಮತ್ತೊಬ್ಬ ಅಧಿಕಾರಿಯಾದ ಶಿವರಾಜ್ ಬಹಾದ್ದೂರ್​–ಇವರುಗಳನ್ನು ಭೇಟಿಯಾದರು. ಈ ಎಲ್ಲ ಗಣನೀಯ ವ್ಯಕ್ತಿಗಳು ಅವರ ಧರ್ಮಪ್ರಸಾರಕಾರ್ಯದಲ್ಲಿ ನೆರವಾಗು ವುದಾಗಿ ವಾಗ್ದಾನ ಮಾಡಿದರು.

ಅದೇ ದಿನ ಮಧ್ಯಾಹ್ನ ಮೆಹಬೂಬ್ ಕಾಲೇಜಿನಲ್ಲಿ ಸ್ವಾಮೀಜಿ “ಪಾಶ್ಚಾತ್ಯ ರಾಷ್ಟ್ರಗಳಿಗೆ ನನ್ನ ಸಂದೇಶ” ಎಂಬ ವಿಷಯವಾಗಿ ಸಾರ್ವಜನಿಕ ಭಾಷಣ ಮಾಡಿದರು. ಪಂಡಿತ ರತನ್​ಲಾಲ್ ಎಂಬವರು ಆ ಸಭೆಯ ಅಧ್ಯಕ್ಷರಾಗಿದ್ದರು. ಆ ಸಭೆಯಲ್ಲಿ ಹಲವಾರು ಮಂದಿ ಐರೋಪ್ಯರೂ ಸೇರಿದಂತೆ ಒಂದು ಸಾವಿರಕ್ಕೂ ಹೆಚ್ಚು ಜನರು ಹಾಜರಿದ್ದರು. ಸ್ವಾಮೀಜಿಯ ಇಂಗ್ಲಿಷ್ ಭಾಷಾ ಪ್ರಭುತ್ವ, ನಿರರ್ಗಳತೆ, ಅಪಾರ ವಿದ್ವತ್ತು, ವಿಷಯ ನಿರೂಪಣಾ ಕೌಶಲ–ಇವು ಅಲ್ಲಿನ ಜನರ ಕಂಡರಿಯದಂಥದು. ತಮ್ಮ ಅಂದಿನ ಭಾಷಣದಲ್ಲಿ ಅವರು ಹಿಂದೂ ಧರ್ಮದ ಶ್ರೇಷ್ಠ ಅಂಶ ಗಳು ಮತ್ತು ಪೂರ್ವಕಾಲದ ಹಿಂದೂ ಸಂಸ್ಕೃತಿ ಹಾಗೂ ಸಮಾಜದ ಹಿರಿಮೆಯ ಬಗ್ಗೆ ಹೇಳಿದರು. ವೇದಕಾಲದ ಹಾಗೂ ವೇದೋತ್ತರ ಕಾಲದ ಚಿಂತನೆಯ ರೂಪುರೇಷೆಗಳನ್ನು ಅವರು ವಿವರಿಸಿ ದರು. ಪುರಾತನ ಭಾರತದ ಮಹರ್ಷಿಗಳ ಬಗ್ಗೆ ಪ್ರಸ್ತಾಪಿಸಿ, ಅವರನ್ನು ಜಗತ್ತಿನ ಶ್ರೇಷ್ಠ ಶಾಸ್ತ್ರ ಗ್ರಂಥಗಳ ಕರ್ತೃಗಳೆಂದೂ ನ್ಯಾಯಶಾಸ್ತ್ರದ ಜನಕರೆಂದೂ ಕೊಂಡಾಡಿದರು. ಬಳಿಕ ‘ಗೊಡ್ಡು ಕಂತೆ’ಗಳೆಂದು ಆಧುನಿಕರು ತುಚ್ಛೀಕರಿಸುವ ಹಿಂದೂ ಪುರಾಣಗಳಲ್ಲಿ ಅಡಕವಾಗಿರುವ ಅತ್ಯುನ್ನತ ನೈತಿಕ ಆದರ್ಶಗಳನ್ನು ತೋರಿಸಿಕೊಟ್ಟರು. ಕಡೆಯಲ್ಲಿ ತಮ್ಮ ಕಾರ್ಯೋದ್ದೇಶದ ಬಗ್ಗೆ ಹೇಳುತ್ತ, ಭಾರತದ ಪುನರುದ್ಧಾರವೇ ತಮ್ಮ ಅಂತಿಮ ಹಾಗೂ ಅತ್ಯುನ್ನತ ಗುರಿಯೆಂದು ಘೋಷಿಸಿದರು. ಅಲ್ಲದೆ, ಹಿಂದೂಧರ್ಮದ ಶ್ರೇಷ್ಠತೆಯನ್ನು, ವೇದವೇದಾಂತಗಳ ಅಪ್ರತಿಮ ವೈಭವವನ್ನು ಸಾರಲು ಒಬ್ಬ ಪ್ರಸಾರಕನಾಗಿ ದೂರದ ಪಶ್ಚಿಮ ದೇಶಗಳಿಗೆ ಹೋಗುವುದನ್ನು ತಮ್ಮ ಆದ್ಯ ಹಾಗೂ ಅತ್ಯಾವಶ್ಯಕ ಕರ್ತವ್ಯವೆಂದು ತಾವು ಭಾವಿಸಿರುವುದಾಗಿ ಸಾರಿದರು.

ಸ್ವಾಮೀಜಿಯ ಭಾಷಣದಿಂದ ಸಮಸ್ತ ಸಭಿಕವೃಂದ ಸಂಪೂರ್ಣ ಪ್ರಭಾವಿತವಾಯಿತು. ಮರು ದಿನ, ಬೇಗಂಬಜಾರಿನ ಪ್ರಸಿದ್ಧ ಬ್ಯಾಂಕರುಗಳು ಸೇಠ್ ಮೋತೀಲಾಲರ ನೇತೃತ್ವದಲ್ಲಿ ಬಂದು ಅವರನ್ನು ಸಂದರ್ಶಿಸಿ ಅವರ ದಾರಿಖರ್ಚನ್ನು ತಾವು ವಹಿಸಿಕೊಳ್ಳುವುದಾಗಿ ಅಶ್ವಾಸನೆ ನೀಡಿದರು. ಇದಲ್ಲದೆ, ಇಲ್ಲಿನ ಥಿಯೊಸಾಫಿಕಲ್ ಸೊಸೈಟಿಯ ಹಾಗೂ ಸಂಸ್ಕೃತ ಧರ್ಮಮಂಡಲ ಸಭೆಯ ಕೆಲವು ಸದಸ್ಯರು ಬಂದು ತಮ್ಮ ಬೆಂಬಲವನ್ನು ಸೂಚಿಸಿದರು. \textit{(ಆದರೆ ಕೊನೆಯಲ್ಲಿ ನಿಜಕ್ಕೂ ನೆರವು ನೀಡಿದವರು ಯಾರುಯಾರೆಂಬುದು ನಿಖರವಾಗಿ ತಿಳಿದುಬಂದಿಲ್ಲ.)} ಫೆಬ್ರವರಿ ೧೫ರಂದು, ಸ್ವಾಮೀಜಿಯನ್ನು ಪೂನಾಕ್ಕೆ ಒಂದು ಭೇಟಿ ನೀಡಬೇಕೆಂದು ಒತ್ತಾಯಿಸುವ ತಂತಿಯೊಂದು ಬಂದಿತು. ಅಲ್ಲಿನ ಹಿಂದೂ ಸಂಘಗಳ ಪರವಾಗಿ ಅನೇಕ ಪ್ರಮುಖ ನಾಗರಿಕರು ಇದಕ್ಕೆ ಸಹಿ ಮಾಡಿದ್ದರು. ಆದರೆ, ತಾವೀಗ ಅಲ್ಲಿಗೆ ಬರಲು ಸಾಧ್ಯವಾಗುವುದಿಲ್ಲ. ಮುಂದೆ ಯಾವಾಗಲಾ ದರೂ ಅನೂಕೂಲವಾದಾಗ ಸಂತೋಷದಿಂದ ಬರುತ್ತೇವೆ ಎಂದು ಸ್ವಾಮೀಜಿ ಪ್ರತ್ಯುತ್ತರ ಕಳಿಸಿದರು.

ಮಾನವನ ಮನಶ್ಶಕ್ತಿಯ ಅಗಾಧತೆಯ ಬಗ್ಗೆ ಸ್ವಾಮೀಜಿ ಆಳವಾಗಿ ಚಿಂತಿಸುವಂತೆ ಮಾಡಿದ ಘಟನೆಯೊಂದು ಅವರು ಹೈದರಾಬಾದಿನಲ್ಲಿದ್ದಾಗ ನಡೆಯಿತು. ಇಲ್ಲಿನ ಒಬ್ಬ ಬ್ರಾಹ್ಮಣ ‘ಎಲ್ಲಿಂದಲೋ’ ಹಲವಾರು ವಸ್ತುಗಳನ್ನು ಸೃಷ್ಟಿಸಬಲ್ಲವನಾಗಿದ್ದ. ಈತ ಒಬ್ಬ ವ್ಯಾಪಾರಸ್ಥ ಸಭ್ಯ ವ್ಯಕ್ತಿ. ಇವನ ಬಗ್ಗೆ ಕೇಳಿತಿಳಿದ ಸ್ವಾಮೀಜಿ ಇವನನ್ನು ಭೇಟಿಯಾಗಿ, ತನ್ನ ವಿದ್ಯೆಯನ್ನು ಪ್ರದರ್ಶಿಸು ವಂತೆ ಕೇಳಿಕೊಂಡರು. ಆದರೆ ಈ ಮನುಷ್ಯ ಆಗ ಜ್ವರದಿಂದ ಹಾಸಿಗೆ ಹಿಡಿದಿದ್ದ. ತನಗೆ ಗುಣ ವಾದ ಮೇಲೆ ತನ್ನ ಇಂದ್ರಜಾಲ ವಿದ್ಯೆಯನ್ನು ಪ್ರದರ್ಶಿಸುವುದಾಗಿ ಅವನು ಮಾತುಕೊಟ್ಟ. ಸಾಧು ಗಳ ಸ್ಪರ್ಶದಿಂದ ರೋಗ ವಾಸಿಯಾಗುತ್ತದೆಂಬ ನಂಬಿಕೆ ಬಹಳ ಜನರಲ್ಲಿದೆ. ಅಂತೆಯೇ ಅವನು, “ನೀವು ದಯವಿಟ್ಟು ನನ್ನ ತಲೆಯ ಮೇಲೆ ಕೈಯಿಟ್ಟರೆ ನನ್ನ ಜ್ವರ ವಾಸಿಯಾಗುತ್ತದೆ” ಎಂದು ಬೇಡಿಕೊಂಡ. ಅದಕ್ಕೊಪ್ಪಿ ಸ್ವಾಮೀಜಿ ಅವನ ತಲೆಯನ್ನು ಸ್ಪರ್ಶಿಸಿದರು. ಜ್ವರ ಬಿಟ್ಟು ಹೋಯಿತು. ಆಮೇಲೆ ಅವನು ತನ್ನ ಮಾತಿನಂತೆ, ತನ್ನ ವಿದ್ಯೆಯನ್ನು ತೋರಿಸಲು ಬಂದ.

ಸರಿ; ಅವನು ತನ್ನ ಕೌಪೀನವನ್ನು ಬಿಟ್ಟು ಉಳಿದೆಲ್ಲ ಬಟ್ಟೆಯನ್ನೂ ತೆಗೆದು ದೂರದಲ್ಲಿಟ್ಟ. ಆಗ ಚಳಿಗಾಲವಾದ್ದರಿಂದ, ಹೊದ್ದುಕೊಳ್ಳಲು ಒಂದು ಕಂಬಳಿಯನ್ನು ಕೇಳಿ ಪಡೆದುಕೊಂಡು ಒಂದು ಮೂಲೆಯಲ್ಲಿ ಕುಳಿತುಕೊಂಡ. ಇಪ್ಪತ್ತೈದು ಜೊತೆ ಕಣ್ಣುಗಳು ಎವೆಯಿಕ್ಕದೆ ಅವನನ್ನೇ ದಿಟ್ಟಿಸುತ್ತಿದ್ದವು. ಆ ಮನುಷ್ಯನ ಬಳಿ ಈಗ ಏನೇನೂ ಇಲ್ಲ ಎಂಬುದು ಎಲ್ಲರಿಗೂ ಖಾತ್ರಿ ಯಾಗಿತ್ತು. ಈಗ ಅವನು “ನಿಮಗೆ ಏನೇನು ವಸ್ತು ಬೇಕೋ ಅದನ್ನೆಲ್ಲ ಚೀಟಿಗಳಲ್ಲಿ ಬರೆದು ಕೊಡಿ” ಎಂದು ಹೇಳಿದ. ಆ ಸ್ಥಳದಲ್ಲಿ ಅತ್ಯಂತ ದುರ್ಲಭವಾದ ಹಣ್ಣು-ಹೂವುಗಳ ಹೆಸರುಗಳನ್ನು ಬರೆದುಕೊಟ್ಟರು. ಏನಾಶ್ಚರ್ಯ! ನೋಡನೋಡುತ್ತಿದ್ದಂತೆ ಆ ಮನುಷ್ಯ ಕಂಬಳಿಯೊಳಗಿನಿಂದ ಇವರು ಬಯಸಿದುದನ್ನೆಲ್ಲ ಒಂದೊಂದಾಗಿ ತೆಗೆಯಲಾರಂಭಿಸಿದ–ಗೊಂಚಲು ಗೊಂಚಲು ದ್ರಾಕ್ಷಿ, ರಸಭರಿತ ಕಿತ್ತಲೆ ಹಣ್ಣುಗಳು, ಘಮಘಮಿಸುವ ಹೂವುಗಳು– ಏನು ಬೇಕೆಂದರದು! ಅವನು ಸೃಷ್ಟಿಸಿದ ಹಣ್ಣುಗಳನ್ನೆಲ್ಲ ಒಟ್ಟುಗೂಡಿಸಿ ನೋಡಿದರೆ ಅದು ಕಡೆಯಪಕ್ಷ ಅವನ ಶರೀರದ ಎರಡರಷ್ಟು ತೂಗುತ್ತಿತ್ತು. ಅಷ್ಟೇಕೆ? ಆ ಹಣ್ಣುಗಳನ್ನು ತಿಂದುನೋಡುವಂತೆಯೂ ಅವನು ಅಲ್ಲಿದ್ದವರಿಗೆ ಹೇಳಿದ. ಆದರೆ ಈ ಇಂದ್ರಜಾಲದ ಹಣ್ಣುಗಳನ್ನು ತಿನ್ನುವುದೆ?! ಎಂದು ಅವರು ಹಿಂದೆಮುಂದೆ ನೋಡಿದರು. ಆಗ ಆ ಬ್ರಾಹ್ಮಣ, ಹಣ್ಣುಗಳನ್ನು ತಾನೇ ತಿಂದು ತೋರಿಸಿದ. ಇದನ್ನು ಕಂಡು ಇತರರೂ ಧೈರ್ಯಗೊಂಡು ಹಣ್ಣುಗಳ ರುಚಿ ನೋಡಿದರು. ಎಲ್ಲವೂ ಅತ್ಯು ತ್ತಮ ದರ್ಜೆಯ ಹಣ್ಣುಗಳು! ಅಲ್ಲದೆ ಅವನು ರುಚಿಕಟ್ಟಾದ ಬಿಸಿ ಬಿಸಿ ಅಡಿಗೆಯನ್ನೂ ಉತ್ಪಾದಿ ಸಿದ!ಕಡೆಯಲ್ಲಿ ರಾಶಿರಾಶಿ ಗುಲಾಬಿ ಹೂಗಳನ್ನು ಸೃಷ್ಟಿಸಿ ತೋರಿಸಿದ. ಒಂದೊಂದೂ ಆಗ ತಾನೆ ಕಿತ್ತಂತಿದ್ದ ಹೊಚ್ಚಹೊಸ ಹೂವು. ಮುಂಜಾನೆಯ ಇಬ್ಬನಿ ಹಾಗೇ ಇದೆ; ಒಂದು ದಳವೂ ನಲುಗಿಲ್ಲ! ಇವುಗಳನ್ನು ಕಂಡು ಸ್ವಾಮೀಜಿ “ನೀನು ಇದನ್ನೆಲ್ಲ ಹೇಗೆ ಮಾಡುತ್ತೀಯೆ?” ಎಂದು ಕೇಳಿದರು. ಅದಕ್ಕವನು “ಇದೆಲ್ಲ ಕೇವಲ ಕೈಚಳಕ, ಅಷ್ಟೆ” ಎಂದುಹೇಳಿ ನುಣುಚಿಕೊಂಡ. ಆದರೆ ಈ ಉತ್ತರ ಅವರಿಗೆ ಒಪ್ಪಿಗೆಯಾಗಲಿಲ್ಲ. ಅಷ್ಟೊಂದು ಬಗೆಯ, ಅಷ್ಟೊಂದು ಪರಿಮಾಣದ ವಸ್ತುಗಳನ್ನು ಅವನು ಎಲ್ಲಿ ಹೇಗೆ ತಾನೆ ಅವಿತಿಟ್ಟುಕೊಂಡಿದ್ದಾನು? ಅದು ಕೇವಲ ಕೈಚಳಕ ವೆಂದರೆ ಯಾರೂ ಒಪ್ಪಿಕೊಳ್ಳುವಂತಿರಲಿಲ್ಲ.

ಸ್ವಾಮೀಜಿ ಈ ಮನುಷ್ಯನನ್ನು ಸೂಕ್ಷ್ಮವಾಗಿ ಪರಾಮರ್ಶಿಸಿದರು. ತಾವು ಕಂಡ ಅದ್ಭುತ ಘಟನೆಯ ಬಗ್ಗೆ ಮತ್ತೆ ಮತ್ತೆ ಆಳವಾಗಿ ಆಲೋಚಿಸಿದರು. ಕಡೆಗೆ ಈ ಬಗ್ಗೆ ಅವರೊಂದು ತೀರ್ಮಾನಕ್ಕೆ ಬಂದರು–ಆ ಮನುಷ್ಯನು ತೋರಿಸಿದ ಇಂದ್ರಜಾಲವೂ ಅವನು ಉತ್ಪಾದಿಸಿದ ವಸ್ತುಗಳೂ ಕೇವಲ ಮನೋಗ್ರಾಹ್ಯವಾದಂಥವು. ಅರ್ಥಾತ್, ಅವು ನಿಜಕ್ಕೂ ಅಲ್ಲಿ ಇರವುದೇ ಇಲ್ಲ. ಆದರೆ ಪ್ರತಿಯೊಬ್ಬ ವ್ಯಕ್ತಿಯ ಮನಸ್ಸೂ ವಿಶ್ವದ ಸಮಷ್ಟಿ ಮನಸ್ಸಿನ ಅಂಗ. ಆದ್ದರಿಂದ ಪ್ರತಿಯೊಂದು ಮನಸ್ಸೂ ಇತರ ಎಲ್ಲ ಮನಸ್ಸುಗಳೊಂದಿಗೆ ಸಂಪರ್ಕವನ್ನು ಹೊಂದಿದೆ. ಈ ಮನಸ್ಸಿನ ಲಕ್ಷಣಗಳನ್ನೂ ಶಕ್ತಿಗಳನ್ನೂ ಶಾಸ್ತ್ರೀಯವಾಗಿ ಅಧ್ಯಯನ ಮಾಡಿ, ಅರಿತು, ಅಸಾಧಾರಣ ಶಕ್ತಿಗಳನ್ನು ಆ ಯೋಗಿ ಗಳಿಸಿದ್ದಾನೆ. ಮತ್ತು ಈ ಸಮಷ್ಟಿ ಮನಸ್ಸನ್ನು ತನ್ನಿಚ್ಛೆಯಂತೆ ನಿಯಂತ್ರಿಸಿ ಇತರರ ಮನಸ್ಸಿನಲ್ಲಿ ಇಲ್ಲದಿರುವುದನ್ನು ಕಾಣುವ, ಕೇಳುವ ಭ್ರಮೆ ಹುಟ್ಟಿಸಲು ಈತ ಸಮರ್ಥನಾಗಿದ್ದಾನೆ.

ಮುಂದೆ ಅಮೆರಿಕದ ಲಾಸ್ ಏಂಜೆಲಿಸ್​ನಲ್ಲಿ ನೀಡಿದ ‘ಮನಸ್ಸಿನ ಶಕ್ತಿಗಳು’ ಎಂಬ ಉಪನ್ಯಾಸದಲ್ಲಿ ಸ್ವಾಮೀಜಿ, ಮನೋನಿಗ್ರಹ ಹಾಗೂ ಏಕಾಗ್ರತೆಯಿಂದ ಎಂತಹ ಅದ್ಭುತಗಳನ್ನು ಸಾಧಿಸಬಹುದು ಎಂದು ವಿವರಿಸುತ್ತ ಈ ತಮ್ಮ ಅನುಭವವನ್ನು ಬಣ್ಣಿಸುತ್ತಾರೆ.

ಹೈದರಾಬಾದಿನ ಅತ್ಯಂತ ಯಶಸ್ವೀ ಹಾಗೂ ಫಲಪ್ರದ ಭೇಟಿಯನ್ನು ಮುಗಿಸಿಕೊಂಡು, ಫೆಬ್ರುವರಿ ೧೭ರಂದು ಸ್ವಾಮೀಜಿ ಮದರಾಸಿಗೆ ಹೊರಟರು. ಅವರಿಗೆ ವಿದಾಯ ಹೇಳಲು ರೈಲು ನಿಲ್ದಾಣದಲ್ಲಿ ಒಂದು ಸಾವಿರಕ್ಕೂ ಹೆಚ್ಚು ಜನರು ಸೇರಿದ್ದರು! ಎಂದರೆ, ಈ ಅಲ್ಪಾವಧಿಯಲ್ಲಿ ಅವರು ಆ ಊರಿನಲ್ಲಿ ಉಂಟುಮಾಡಿದ್ದ ಪರಿಣಾಮವನ್ನು ಊಹಿಸಬಹುದು. ಕೆಲವು ದಿನಗಳ ಹಿಂದೆಯಷ್ಟೇ ತಾವು ಕಂಡು ಕೇಳರಿಯದಿದ್ದ ಅನಾಮಧೇಯ ಪರಿವ್ರಾಜಕ ಸಂನ್ಯಾಸಿಯೊಬ್ಬನ ಮೇಲೆ ಜನ ಇಷ್ಟೊಂದು ಭಕ್ತಿ-ವಿಶ್ವಾಸ ತಾಳಬೇಕೆಂದರೆ ಅವರ ವ್ಯಕ್ತಿತ್ವದ ಆಕರ್ಷಣೆ ಎಷ್ಟು ಪ್ರಬಲವಾಗಿದ್ದಿರಬೇಕು! ಅವರ ಮಾತಿನಲ್ಲಿ ಎಂತಹ ಮೋಡಿಯಿದ್ದಿರಬೇಕು! ಬಾಬು ಕಾಳೀ ಚರಣ ಚಟರ್ಜಿ ಬರೆಯುತ್ತಾರೆ. “ಸ್ವಾಮೀಜಿಯ ಪವಿತ್ರ ಹಾಗೂ ಸರಳ ಸ್ವಭಾವ, ಅವರ ಕಟುತರ ಆತ್ಮಸಂಯಮ ಹಾಗೂ ತೀವ್ರ ಧ್ಯಾನಶೀಲತೆ–ಇವುಗಳು ಹೈದರಾಬಾದಿನ ಜನರ ಮನಸ್ಸಿನ ಮೇಲೆ ಅಚ್ಚಳಿಯದ ಮುದ್ರೆಯನ್ನೊತ್ತಿದ್ದುವು.”

ಸ್ವಾಮೀಜಿ ಮದರಾಸಿಗೆ ಹಿಂದಿರುಗುತ್ತಿದ್ದಂತೆ ಅವರಿಗಾಗಿ ಕಾದುನಿಂತಿದ್ದ ಶಿಷ್ಯರು ಜಯ ಘೋಷದೊಂದಿಗೆ ಸಂಭ್ರಮದಿಂದ ಬರಮಾಡಿಕೊಂಡರು. ಈಗ ಇವರ ಉತ್ಸಾಹ ಇಮ್ಮಡಿ ಯಾಗಿತ್ತು. ಇವರಲ್ಲಿ ಅನೇಕರು ಹೈದರಾಬಾದಿನಲ್ಲಿ ಸ್ವಾಮೀಜಿ ಮಾಡಿದ ಸಾರ್ವಜನಿಕ ಭಾಷಣ ವನ್ನೂ ಕೇಳಿದ್ದರಲ್ಲದೆ, ಅವರು ಗಳಿಸಿದ್ದ ಜನಪ್ರಿಯತೆಯನ್ನೂ ಕಂಡಿದ್ದರು. ಅಲ್ಲದೆ ಈಗ ಸ್ವತಃ ಸ್ವಾಮೀಜಿಯೂ ಹೆಚ್ಚು ಆತ್ಮವಿಶ್ವಾಸದಿಂದ ತುಂಬಿದ್ದರು. ಅವರು ತಮ್ಮ ಭಾಷಣ ಸಾಮರ್ಥ್ಯ ವನ್ನು ಮೆಹಬೂಬ್ ಕಾಲೇಜಿನಲ್ಲಿ ನಡೆದ ಬೃಹತ್ ಸಭೆಯಲ್ಲಿ ಮಾತನಾಡಿ ಪರೀಕ್ಷಿಸಿ ನೋಡಿ ಕೊಂಡಿದ್ದರು. ಹಿಂದೊಮ್ಮೆ ಅವರು ಬೆಳಗಾವಿಯಲ್ಲಿ ಹರಿಪದ ಮಿತ್ರನ ಬಳಿ ಹೇಳಿದ್ದರು– ದೊಡ್ಡ ಸಭೆಯೊಂದರ ಮುಂದೆ ನಿಂತು ಮಾತನಾಡುವಾಗ ಭಾಷಣಕಾರನ ಸುಪ್ತ ಶಕ್ತಿಸಾಮರ್ಥ್ಯ ಗಳೆಲ್ಲ ಹೊರಹೊಮ್ಮಿ ಬರುತ್ತವೆ, ಎಂದು. ಈ ಮಾತು ಅವರ ವಿಚಾರದಲ್ಲಿ ನಿಜವಾಗಿತ್ತು.

ಮದರಾಸಿನಲ್ಲಿ ಸ್ವಾಮೀಜಿ ಮತ್ತೆ ಸಂಭಾಷಣೆ-ಪ್ರವಚನಗಳನ್ನು ಮುಂದುವರಿಸಿದರು. ದಿನಕಳೆ ದಂತೆಲ್ಲ ಹೊಸ ಹೊಸ ಭಕ್ತರು, ಶಿಷ್ಯರು ಅವರ ಸಂಪರ್ಕಕ್ಕೆ ಬರಲಾರಂಭಿಸಿದರು. ಈಗೀಗ ಅವರು ಅಮೆರಿಕೆಗೆ ಹೋಗುವ ವಿಷಯದಲ್ಲೇ ತತ್ಪರರಾಗಿರುತ್ತಿದ್ದರು. ಕೆಲವೊಮ್ಮೆ ಅವರಿಗನ್ನಿಸು ತ್ತಿತ್ತು–ಅಮೆರಿಕೆಯಂತಹ ಭೋಗಭರಿತ ರಾಷ್ಟ್ರಕ್ಕೆ ಹೋಗಿ ಅಲ್ಲಿನ ಜನರ ಮೇಲೆ ಪ್ರಭಾವ ಬೀರಲು ಸಾಧ್ಯವಾದೀತೆ? ಅದಕ್ಕೆ ಬೇಕಾದ ಸಾಮರ್ಥ್ಯ ತಮ್ಮಲ್ಲಿದೆಯೆ? ಎಂದು. ಆದರೆ ಮತ್ತೆ ಕೆಲವು ಸಲ ಅವರ ಅಂತರಂಗಕ್ಕೆ ಗೋಚರವಾಗುತ್ತಿತ್ತು–ಪಾಶ್ಚಾತ್ಯ ದೇಶಗಳಲ್ಲಿ ಹಿಂದೂ ಧರ್ಮದ ಘನತೆಯನ್ನು ಸಾರಲು ಇದೇ ಸುಸಂದರ್ಭ; ಈ ಕೆಲಸದಲ್ಲಿ ತಾವು ಖಂಡಿತವಾಗಿ ಜಯಶಾಲಿಗಳಾಗುತ್ತೇವೆ, ಎಂದು. ಪಾಶ್ಚಾತ್ಯ ರಾಷ್ಟ್ರಗಳಿಗೆ ತಾವು ನೀಡಲಿರುವ ಸಂದೇಶದ ರೂಪರೇಷೆಗಳನ್ನು ಆಗಾಗ ತಮ್ಮ ಶಿಷ್ಯರ ಮುಂದಿಡುತ್ತಿದ್ದರು. ತಾವು ಮಹತ್ತರವಾದದ್ದೇ ನನ್ನೋ ಸಾಧಿಸುವುದು ಸುನಿಶ್ಚಿತ ಎಂದು ಹೇಳಿದರು. ಅವರ ಶಿಷ್ಯರಿಗಂತೂ ಇದರಲ್ಲಿ ಸಂದೇಹವೇ ಇರಲಿಲ್ಲ. “ನರೇಂದ್ರ ಲೋಕಕ್ಕೆ ಬೋಧಿಸುತ್ತಾನೆ” ಎಂದು ಶ್ರೀರಾಮಕೃಷ್ಣರು ಅಂದು ನುಡಿದ ಭವಿಷ್ಯವಾಣಿಯಲ್ಲಿ ನರೇಂದ್ರನಿಗೇ ನಂಬಿಕೆಯಿರಲಿಲ್ಲ. ಆದರೆ ಈಗ ಆ ಕಾಲ ಸನ್ನಿಹಿತವಾಗುತ್ತಿರುವುದನ್ನು ಸ್ವಾಮೀಜಿ ಗಮನಿಸುತ್ತಿದ್ದಾರೆ. ಕೇವಲ ಆರೇಳು ವರ್ಷಗಳಲ್ಲಿ ಅವರ ಅರಿವು ಅನುಭವಗಳಲ್ಲಿ ಎಷ್ಟು ವ್ಯತ್ಯಾಸ! ಎಂತಹ ಮಾರ್ಪಾಡು!

ಅಮೆರಿಕೆಗೆ ಹೊರಡುವ ಬಗ್ಗೆ ಬೇಗ ನಿರ್ಧಾರ ಕೈಗೊಂಡು ಅಣಿಯಾಗುವಂತೆ ಮದರಾಸಿನ ಶಿಷ್ಯರು ಒತ್ತಾಯಪಡಿಸುತ್ತಿದ್ದರು. ಅಲ್ಲದೆ ತಾವೂ ಈಗ ಮತ್ತೊಮ್ಮೆ ನಿಧಿಸಂಗ್ರಹಣೆಗೆ ಹೊರ ಡಲು ಸಿದ್ಧರಾದರು. ಇದಕ್ಕೆ ಸ್ವಾಮೀಜಿ ಅರೆಮನಸ್ಸಿನಿಂದಲೇ ಒಪ್ಪಿಗೆ ಕೊಟ್ಟರು. ನಿಧಿ ಸಂಗ್ರಹಣೆ ಗಾಗಿ ಮದರಾಸಿನ ಶಿಷ್ಯರು ಅಳಸಿಂಗ ಪೆರುಮಾಳರ ನಾಯಕತ್ವದಲ್ಲಿ ಸಮಿತಿಯೊಂದನ್ನು ರಚಿಸಿ ಕೊಂಡು ಅತ್ಯಂತ ಉತ್ಸಾಹದಿಂದ ಕಾರ್ಯಮಗ್ನರಾದರು. ಕೆಲವರು ಈ ಉದ್ದೇಶಕ್ಕಾಗಿ ಮೈಸೂರು, ಬೆಂಗಳೂರು, ಹೈದರಾಬಾದ್, ರಾಮನಾಡುಗಳಂತಹ ದೂರಪ್ರದೇಶಗಳಿಗೂ ಹೋದರು. ಇವರು ಸ್ವಾಮೀಜಿಯ ಶಿಷ್ಯರಾಗಿದ್ದವರನ್ನೂ ಇನ್ನಿತರರನ್ನೂ ಸಂಪರ್ಕಿಸಿ ವಂತಿಗೆ ಸಂಗ್ರಹಿಸಿದರು. ಇವರೊಂದಿಗೆ ಬೆಂಗಳೂರಿನ ನಂಜುಂಡರಾವ್ ಮೊದಲಾದವರೂ ಹೆಗಲು ಕೊಟ್ಟರು. ಶ್ರೀಮಂತರು, ಅಧಿಕಾರಿಗಳು ಮಾತ್ರವಲ್ಲದೆ ಮಧ್ಯಮವರ್ಗದ ಜನಸಾಮಾನ್ಯ ರಿಂದಲೂ ಇವರು ವಂತಿಗೆ ಬೇಡಿದರು. ಏಕೆಂದರೆ ಸ್ವಾಮೀಜಿ ಇವರಿಗೆ ಸ್ಪಷ್ಟವಾಗಿ ಹೇಳಿದ್ದರು: “ನಾನು ಪಾಶ್ಚಾತ್ಯ ರಾಷ್ಟ್ರಗಳಿಗೆ ಹೋಗಬೇಕೆಂಬುದು ಜಗನ್ಮಾತೆಯ ಇಚ್ಛೆಯೇ ಆದಲ್ಲಿ, ಜನ ಗಳಿಂದಲೇ ಹಣ ಬರುವಂತಾಗಲಿ! ಏಕೆಂದರೆ ನಾನು ಅಲ್ಲಿಗೆ ಹೋಗುತ್ತಿರುವುದೇ ಬಡವರಿಗಾಗಿ ಮತ್ತು ಜನಸಾಮಾನ್ಯರಿಗಾಗಿ” ಎಂದು. ಮಾರ್ಚ್ ಏಪ್ರಿಲ್ ತಿಂಗಳಲ್ಲಿ ನಿಧಿಸಂಗ್ರಹಣೆ ಭರ ದಿಂದ ಸಾಗಿತು. ಮಧ್ಯಮವರ್ಗದ ಜನರಿಂದ ಹೆಚ್ಚಿನ ವಂತಿಗೆ ಸಿಕ್ಕಿತು. ಸ್ವತಃ ಸುಬ್ರಮಣ್ಯ ಅಯ್ಯರ್ ಹಾಗೂ ಮನ್ಮಥಬಾಬುಗಳು ತಲಾ ಐನೂರು ರೂಪಾಯಿ ನೀಡಿದರಲ್ಲದೆ, ಇತರ ರಿಂದಲೂ ವಂತಿಗೆ ಸಂಗ್ರಹಿಸಿದರು. ಮೈಸೂರಿನ ಮಹಾರಾಜರು ಧಾರಾಳವಾಗಿ ಧನಸಹಾಯ ನೀಡಿದರು. ಅಲ್ಲದೆ, ಎಲ್ಲ ರಾಜ್ಯಗಳ ಉನ್ನತ ಅಧಿಕಾರಿಗಳು ವಂತಿಗೆ ನೀಡುತ್ತಿರುವುದನ್ನು ಕಂಡು ರಾಮನಾಡಿನ ರಾಜನೂ ಮನ್ಮಥನಾಥರ ಬೇಡಿಕೆಯ ಮೇರೆಗೆ ಐನೂರು ರೂಪಾಯಿ ಕಾಣಿಕೆ ನೀಡಿದ. ಹಿಂದೆ ಈತ ಹತ್ತು ಸಾವಿರ ರೂಪಾಯಿ ಕೊಡುವುದಾಗಿ ಹೇಳಿದ್ದನಾದರೂ, ಬಳಿಕ ಇತರರ ಮಾತು ಕೇಳಿಕೊಂಡು ಹಿಂಜರಿದಿದ್ದ, ಆದರೆ ಈಗ ಇತರರು ಕೊಡುವುದನ್ನು ಕಂಡಾಗ ಅವನಿಗೇನನ್ನಿಸಿತೋ, ತನ್ನ ‘ಕಿರುಕಾಣಿಕೆ’ಯನ್ನೂ ಅರ್ಪಿಸಿದ. ಆದರೆ ಮುಂದೆ ಸ್ವಾಮೀಜಿ ಶಿಕಾಗೋ ಸಮ್ಮೇಳನದಲ್ಲಿ ಯಶಸ್ವಿಗಳಾಗಿ ವಿಶ್ವವಿಜೇತರಾದ ಮೇಲೆ ಇವನ ಕಣ್ಣು ಪೂರ್ತಿಯಾಗಿ ತೆರೆಯಿತೆಂದು ಕಾಣುತ್ತದೆ. ಆದ್ದರಿಂದಲೇ ಅವರು ಭಾರತಕ್ಕೆ ಹಿಂದಿರುಗಿ, ರಾಮನಾಡಿಗೆ ಬಂದಿಳಿದಾಗ ಅವರಿಗೆ ಇವನು ಹೃತ್ಪೂರ್ವಕ ಸ್ವಾಗತ ನೀಡುವುದನ್ನು ನೋಡಲಿದ್ದೇವೆ.

ಮದರಾಸಿನ ಶಿಷ್ಯರೆಲ್ಲ ಸೇರಿ ಏಪ್ರಿಲ್ ಮಧ್ಯದೊಳಗೆ ಸುಮಾರು ನಾಲ್ಕು ಸಾವಿರ ರೂಪಾಯಿ ಗಳನ್ನು ತಂದು ಸ್ವಾಮೀಜಿಗೆ ಅರ್ಪಿಸಿದರು. ಆದರೆ ಇಷ್ಟೆಲ್ಲ ಕೆಲಸ ನಡೆಯುತ್ತಿದ್ದರೂ, ಇದಕ್ಕೆ ಜಗನ್ಮಾತೆಯ ಆಶೀರ್ವಾದವಿದೆ ಎಂದು ಅವರಿಗಿನ್ನೂ ನಂಬಿಕೆಯುಂಟಾಗಿರಲಿಲ್ಲ. ತಮಗೆ ದಾರಿ ತೋರಬೇಕೆಂದು ಜಗನ್ಮಾತೆಯನ್ನೂ ಶ್ರೀರಾಮಕೃಷ್ಣರನ್ನೂ ಪ್ರಾರ್ಥಿಸಿಕೊಳ್ಳುತ್ತಲೇ ಇದ್ದರು. ಆದರೆ ಈಗ ಶಿಷ್ಯರೆಲ್ಲ ಉತ್ಸಾಹದಿಂದ ನಿಧಿ ಸಂಗ್ರಹಿಸಿ ತಂದದ್ದನ್ನು ಕಂಡು ಅವರಿಗೆ ಅನ್ನಿಸಿತು– ‘ಇದೂ ಸರಿಯೆ! ಇವರೆಲ್ಲ ಸೇರಿ ಇಷ್ಟೊಂದು ಹುಮ್ಮಸ್ಸಿನಿಂದ ಕಾರ್ಯಮಗ್ನರಾಗಿರುವುದನ್ನು ನೋಡಿದರೆ, ಇದೇ ಭಗವದಿಚ್ಛೆಯ ಮೊದಲ ಸಂಕೇತವಾಗಿರಬಹುದೆ...?’ ಆದರೆ ಅವರಿಗೆ ತಾವು ಅಮೆರಿಕೆಗೆ ಹೋಗಬೇಕಾದುದರ ಆವಶ್ಯಕತೆಯೂ ಉಪಯುಕ್ತತೆಯೂ ಮನದಟ್ಟಾಗಿ ದ್ದರೂ, ಅತಿ ಉತ್ಸಾಹದಲ್ಲಿ ಎಡವುವಂತಾಗಬಾರದೆಂದು ಆಲೋಚಿಸುತ್ತಿದ್ದರು. ಆದ್ದರಿಂದ ಒಂದು ಪ್ರತ್ಯಕ್ಷ ದರ್ಶನದ ಮೂಲಕ ಶ್ರೀರಾಮಕೃಷ್ಣರು ಅವರ ಸಮ್ಮತಿಯನ್ನು ಸೂಚಿಸುವುದಾ ದರೆ ತಮ್ಮ ಸಂಶಯಗಳೆಲ್ಲ ದೂರವಾಗಿ, ನಿಶ್ಚಿಂತ ಮನಸ್ಸಿನಿಂದ ಹೊರಡಬಹುದು ಎಂದು ಚಿಂತಿಸುತ್ತಿದ್ದರು. ಹೀಗೆ ಇಬ್ಬಂದಿಯಲ್ಲಿ ಸಿಲುಕಿಕೊಂಡಿದ್ದ ಸಮಯದಲ್ಲೇ ಅವರಿಗೊಂದು ದರ್ಶನವಾಯಿತು. ಒಂದು ದಿನ ಅವರು ಒರಗಿಕೊಂಡು ಅರ್ಧನಿದ್ರೆಯಲ್ಲಿದ್ದಾಗ ಇದ್ದಕ್ಕಿದ್ದಂತೆ ಶ್ರೀರಾಮಕೃಷ್ಣರ ಜ್ಯೋತಿರ್ಮಯ ರೂಪವನ್ನು ಕಂಡರು. ಶ್ರೀರಾಮಕೃಷ್ಣರು ಸಮುದ್ರದ ದಡದಿಂದ ಮುನ್ನಡೆದು ಅಲೆಗಳ ಮೇಲೆ ನಡೆದು ಹೋಗುತ್ತ ತಮ್ಮನ್ನು ಅನುಸರಿಸಿ ಬರುವಂತೆ ಸ್ವಾಮೀಜಿಯನ್ನು ಕೈಬೀಸಿ ಕರೆಯುತ್ತಿದ್ದರು. ಈ ದೃಶ್ಯವನ್ನು ಕಂಡು ಸ್ವಾಮೀಜಿಯ ಮೈಮನಗಳು ಅಪೂರ್ವ ಶಾಂತಿ-ಆನಂದಗಳಿಂದ ತುಂಬಿಹೋದುವು. ತಕ್ಷಣ ಅವರು ಎಚ್ಚತ್ತು ಕುಳಿತರು. ಆದೇಶ ಸ್ಪಷ್ಟವಾಗಿತ್ತು. ಇದು ದೈವೇಚ್ಛೆಯೇ ಸರಿ ಎಂದು ಅವರಿಗೆ ದೃಢವಾಯಿತು. ಇದೀಗ ಅವರು ಅಮೆರಿಕೆಗೆ ಹೊರಡಲು ಉದ್ಯುಕ್ತರಾಗತೊಡಗಿದರು.

ಸ್ವಾಮೀಜಿ ಮದರಾಸಿನಲ್ಲಿದ್ದಾಗ ಅಲ್ಲಿನ ಥಿಯೊಸಾಫಿಕಲ್ ಸೊಸೈಟಿಯ ಕೇಂದ್ರ ಕಛೇರಿ ಯಲ್ಲಿ ಕನಿಷ್ಠ ಎರಡು ಉಪನ್ಯಾಸಗಳನ್ನು ನೀಡಿದ್ದರು. ಈ ಸೊಸೈಟಿಯ ಪತ್ರಿಕೆಯಾದ \eng{The Theosophist}ನ ಮಾರ್ಚ್ ೧೮೯೩ರ ಸಂಚಿಕೆಯಲ್ಲಿ ಈ ಕುರಿತಾದ ವರದಿಯೊಂದು ಪ್ರಕಟ ವಾಗಿದ್ದು, ಮದರಾಸಿನ ವಿದ್ಯಾವಂತ ಜನವರ್ಗದ ಮೇಲೆ ಸ್ವಾಮೀಜಿ ಎಂತಹ ಪ್ರಭಾವ ಬೀರುತ್ತಿದ್ದರೆಂಬುದನ್ನು ಇದರಲ್ಲಿ ಕಾಣಬಹುದಾಗಿದೆ.

ಅಮೆರಿಕೆಗೆ ಹೋದಾಗ ಯಾವುದಾದರೂ ಪ್ರಸಿದ್ಧ ಸಂಸ್ಥೆಯಿಂದ ಒಂದು ಪರಿಚಯಪತ್ರ ವಿದ್ದರೆ ತಮಗೆ ಉಪಯೋಗವಾಗಬಹುದೆಂದು ಸ್ವಾಮೀಜಿ ಭಾವಿಸಿದರು. ಆದ್ದರಿಂದ ತಮಗೆ ಈಗಾಗಲೇ ಪರಿಚಿತರಾಗಿದ್ದ ಥಿಯೊಸಾಫಿಕಲ್ ಸೊಸೈಟಿಯ ಅಧ್ಯಕ್ಷರಾಗಿದ್ದ ಕರ್ನಲ್ ಓಲ್ಕಾಟ್ ಎಂಬವರನ್ನು ಭೇಟಿಯಾದರು. ‘ಈತ ಅಮೆರಿಕದವರು, ಭಾರತದ ಬಗ್ಗೆ ಪ್ರೀತಿಯುಳ್ಳವರು; ಆದ್ದರಿಂದ ಇವರು ಅಮೆರಿಕದಲ್ಲಿರುವ ಯಾರಿಗಾದರೂ ತಮ್ಮ ಬಗ್ಗೆ ಒಂದು ಪರಿಚಯಪತ್ರ ವನ್ನು ಬರೆದುಕೊಟ್ಟಾರು’ ಎಂದು ಸ್ವಾಮೀಜಿ ನಿರೀಕ್ಷಿಸಿದ್ದರು. ಅಲ್ಲದೆ ಸ್ವಾಮೀಜಿಯ ಯೋಗ್ಯತೆ ಯೇನೆಂಬುದು ಥಿಯಸೊಫಿಸ್ಟರಿಗೆಲ್ಲ ಚೆನ್ನಾಗಿ ಅರಿವಾಗಿತ್ತು. ಅದಕ್ಕಿಂತ ಹೆಚ್ಚಾಗಿ, ಸ್ವಾಮಿಜಿ ಯಂಥವರಿಗಾಗಿ ಪರಿಚಯ ಪತ್ರವನ್ನು ಕೊಟ್ಟಿದ್ದರೆ ಥಿಯಸೊಫಿಸ್ಟರಿಗೆ ಗೌರವವೇ ಹೆಚ್ಚುತ್ತಿತ್ತು. ಆದರೆ ಅದು ಹಾಗಾಗಲಿಲ್ಲ. ಅಧ್ಯಕ್ಷರು ಕೇಳಿದರು:

“ಕೋಡೋಣ, ಆದರೆ ನೀವು ನಮ್ಮ ಸಂಸ್ಥೆಗೆ ಸೇರಿಕೊಳ್ಳುವಿರಾ?”

“ಇಲ್ಲ, ಅದು ಹೇಗೆ ಸಾಧ್ಯವಾದೀತು? ನಾನು ನಿಮ್ಮ ಎಷ್ಟೋ ತತ್ತ್ವಗಳನ್ನು ಒಪ್ಪಿಕೊಳ್ಳು ವುದಿಲ್ಲವಲ್ಲ! ಹೀಗಿದ್ದ ಮೇಲೆ ನಾನು ನಿಮ್ಮ ಸಂಘವನ್ನು ಸೇರುವ ಪ್ರಶ್ನೆಯೇ ಇಲ್ಲ.”

“ಓ, ಹಾಗೋ... ಹಾಗಿದ್ದರೆ ಕ್ಷಮಿಸಿ; ನಿಮಗಾಗಿ ನಾನೇನೂ ಮಾಡಲಾರೆ.”

ಇಲ್ಲಿ ನಾವು ಸ್ವಾಮೀಜಿಯ ಧೀರತನವನ್ನು, ಎದೆಗಾರಿಕೆಯನ್ನು ನೋಡಬೇಕು. ಎಂತಹ ಅಸಹಾಯಕ ಸ್ಥಿತಿಯಲ್ಲಿದ್ದರೂ ಪರಿಸ್ಥಿತಿಯೊಂದಿಗೆ ರಾಜಿಮಾಡಿಕೊಳ್ಳದಿರುವ ಅವರ ದಾರ್ಢ್ಯ ಇಲ್ಲಿ ವ್ಯಕ್ತವಾಗುತ್ತದೆ.

ಸ್ವಾಮೀಜಿಯನ್ನು ಅಮೆರಿಕೆಗೆ ಕಳಿಸಿಕೊಡಲು ಅವರ ಶಿಷ್ಯರು ಭರದಿಂದ ತಯಾರಿ ನಡೆಸು ತ್ತಿದ್ದ ಸಂದರ್ಭದಲ್ಲಿ ಇದ್ದಕ್ಕಿದ್ದಂತೆ ಒಂದು ತೊಂದರೆ ಉದ್ಭವಿಸಿತು. ಒಂದು ರಾತ್ರಿ, ತಮ್ಮ ತಾಯಿ ತೀರಿಕೊಂಡಂತೆ ಸ್ವಾಮೀಜಿಗೆ ಕನಸಾಯಿತು. ಇದು ಅವರ ಮನಸ್ಸನ್ನು ಕಲಕಿಬಿಟ್ಟಿತು. ಅದನ್ನೊಂದು ಕನಸು ಎಂದು ತಳ್ಳಿಹಾಕಲು ಅವರಿಗೆ ಸಾಧ್ಯವಾಗಲಿಲ್ಲ. ತಮ್ಮ ತಾಯಿಯ ಕ್ಷೇಮದ ಬಗ್ಗೆ ತಿಳಿಯುವವರೆಗೂ ಅವರಿಗೆ ಸಮಾಧಾನವಾಗುವಂತಿರಲಿಲ್ಲ. ತಮ್ಮ ಮನೆಯವ ರೊಂದಿಗಿರಲಿ, ಗುರುಭಾಯಿಗಳೊಂದಿಗೂ ಆಗ ಅವರು ಪತ್ರವ್ಯವಹಾರವನ್ನಿಟ್ಟುಕೊಂಡಿರಲಿಲ್ಲ. ತಮಗಾದ ಕನಸಿನ ಬಗ್ಗೆ ಸ್ವಾಮೀಜಿ, ಮನ್ಮಥನಾಥರಿಗೆ ತಿಳಿಸಿದರು. ತಕ್ಷಣ ಅವರು ನಿಜಸ್ಥಿತಿ ಯನ್ನರಿಯಲು ಕಲ್ಕತ್ತಕ್ಕೆ ತಂತಿ ಕಳಿಸಿದರು. ಆದರೆ ಈ ತಂತಿ ಕಲ್ಕತ್ತ ತಲುಪಿ, ಅಲ್ಲಿಂದ ಪ್ರತ್ಯುತ್ತರ ಬರುವುದಕ್ಕೂ ಸಾಕಷ್ಟು ಕಾಲ ಕಾಯಬೇಕಾಗಿತ್ತು. ಈ ಮಧ್ಯೆ ಮದರಾಸಿನ ಶಿಷ್ಯರು ಸ್ವಾಮೀಜಿಯನ್ನು ಬೇಗ ಕಳಿಸಿಕೊಡಲು ಕಾತರರಾಗಿದ್ದರು. ಆದರೆ ತಮ್ಮ ತಾಯಿಯ ಸುದ್ದಿ ತಿಳಿದ ಹೊರತು ಹೊರಡಲು ಸ್ವಾಮೀಜಿಗೆ ಮನಸ್ಸಿರಲಿಲ್ಲ. ಅವರ ಮನಸ್ಥಿತಿಯನ್ನರಿತ ಮನ್ಮಥಬಾಬು ಗಳು, ಅವರಿಗೊಂದು ಸಲಹೆ ಮಾಡಿದರು: “ಇಲ್ಲೇ ಸಮೀಪದಲ್ಲಿ ಗೋವಿಂದ ಚೆಟ್ಟಿ ಎಂಬುವ ನೊಬ್ಬ ಇದ್ದಾನೆ. ಅವನು ಪ್ರೇತಗಳನ್ನು ವಶಪಡಿಸಿಕೊಂಡಿದ್ದಾನೆಂದು ಹೇಳುತ್ತಾರೆ. ಅವನು ಯಾವುದೇ ವ್ಯಕ್ತಿಯ ಭೂತ-ಭವಿಷ್ಯಗಳನ್ನೂ ಹೇಳಬಲ್ಲನಂತೆ. ನೀವು ಒಪ್ಪುವುದಾದರೆ ನಾವೆಲ್ಲ ಹೋಗಿ ನೋಡಿಕೊಂಡು ಬರಬಹುದು.” ಸ್ವಾಮೀಜಿ ಇದಕ್ಕೆ ಒಪ್ಪಿದರು.

ಮನ್ಮಥಬಾಬು, ಅಳಸಿಂಗ ಹಾಗೂ ಇನ್ನೊಬ್ಬರೊಂದಿಗೆ ಸ್ವಾಮೀಜಿ ಅವನಲ್ಲಿಗೆ ಹೋದರು. ಊರಾಚೆಯ ಸ್ಮಶಾನವೊಂದರ ಬಳಿ ಅವನು ಕುಳಿತಿದ್ದ–ಅವನ ಮುಖವೂ ದೆವ್ವದ ಮುಖ ದಂತೆಯೇ ಇತ್ತು. ಕಾಡಿಗೆ ಕಪ್ಪಿನ ಮೈಬಣ್ಣ! ಅವನ ಸಹಚರರು ಇವರನ್ನು ಬರಮಾಡಿ ಕೊಂಡರು. ಆ ಮನುಷ್ಯ ಭೂತಪ್ರೇತಗಳ ಮೇಲೆ ಸಂಪೂರ್ಣ ಹಿಡಿತ ಹೊಂದಿದ್ದಾನೆಂದು ಅವನ ಶಿಷ್ಯರು ಹೇಳಿದರು. ಆದರೆ ಆ ಮನುಷ್ಯ ಇವರತ್ತ ಕಣ್ಣೆತ್ತಿಯೂ ನೋಡಲಿಲ್ಲ. ಸ್ವಾಮೀಜಿ ಸ್ವಲ್ಪ ಹೊತ್ತು ಕಾದು, ಕಡಗೆ ಬೇಸತ್ತು ಅಲ್ಲಿಂದ ಹೊರಟರು. ಆಗ ಆ ಮನುಷ್ಯ ಇವರಿಗೆ ಕುಳಿತುಕೊಳ್ಳು ವಂತೆ ಹೇಳಿದ. ಅಳಸಿಂಗ ಪೆರುಮಾಳ್ ದುಭಾಷಿಯ ಕೆಲಸ ಮಾಡಿದರು. ಅಷ್ಟು ಹೊತ್ತಿಗೆ ಆ ಮನುಷ್ಯ ಒಂದು ಪೆನ್ಸಿಲಿನಿಂದ ಏನೋ ಚಿತ್ರಗಳನ್ನು ಗೀಚಲಾರಂಭಿಸಿದ. ಹೀಗೆ ಬರೆಯುತ್ತಿ ದ್ದಂತೆಯೇ ಆತ ಏಕಾಗ್ರಮನಸ್ಕನಾಗಿ ಕುಳಿತುಬಿಟ್ಟ. ಈಗ ಅವನು ಸ್ವಾಮೀಜಿಯ ಹೆಸರು, ಅವರ ವಂಶಜರ ಹೆಸರುಗಳು, ವಂಶದ ಚರಿತ್ರೆ–ಎಲ್ಲವನ್ನೂ ಹೇಳತೊಡಗಿದ. ಬಳಿಕ ಹೇಳಿದ, “ನಿಮ್ಮ ಈ ದೀರ್ಘ ಸಂಚಾರದ ಉದ್ದಕ್ಕೂ ಶ್ರೀರಾಮಕೃಷ್ಣರು ನಿಮ್ಮ ಜೊತೆಯಲ್ಲಿಯೇ ಇದ್ದಾರೆ... ನಿಮ್ಮ ತಾಯಿಯವರು ಕ್ಷೇಮವಾಗಿದ್ದಾರೆ, ಆರೋಗ್ಯದಿಂದಿದ್ದಾರೆ, ಏನೂ ಚಿಂತಿಸ ಬೇಕಿಲ್ಲ... ಇಷ್ಟರಲ್ಲೇ ನೀವು ಧರ್ಮ ಪ್ರಚಾರಕಾರ್ಯಕ್ಕಾಗಿ ದೂರ ದೇಶಗಳಿಗೆ ತೆರಳಲಿದ್ದೀರಿ.”

ಹೀಗೆ ತಮ್ಮ ತಾಯಿಯ ಕುರಿತಾದ ಶುಭ ಸಮಾಚಾರವನ್ನು ತಿಳಿದ ಸ್ವಾಮೀಜಿ ತಮ್ಮ ಸಂಗಡಿಗರೊಂದಿಗೆ ಹಿಂದಿರುಗಿದರು. ತರುವಾಯ ಕಲ್ಕತ್ತದಿಂದ ಅವರಿಗೊಂದು ತಂತಿ ಬಂದಿತು. ಅದರಲ್ಲಿ ಅವರ ತಾಯಿ ಸೌಖ್ಯದಿಂದಿರುವುದಾಗಿ ಬರೆದಿತ್ತು. ಸ್ವಾಮೀಜಿಗೆ ಈಗ ತುಂಬ ಸಮಾಧಾನವಾಯಿತು.

ಪರಿವ್ರಾಜಕರಾಗಿ ಕಡೆಯಬಾರಿ ಕಲ್ಕತ್ತದಿಂದ ಹೊರಡುವ ಮೊದಲು ಅವರು, ಶ್ರೀಮಾತೆ ಶಾರದಾದೇವಿಯರ ಶುಭಾಶೀರ್ವಾದವನ್ನು ಪಡೆದುಕೊಂಡಿದ್ದರು. ಈಗ ಅಮೆರಿಕೆಯಂತಹ ದೂರದೇಶಕ್ಕೆ ಪ್ರಯಾಣ ಕೈಗೊಳ್ಳುವ ಮೊದಲು ಮತ್ತೊಮ್ಮೆ ಅವರ ಆಶೀರ್ವಾದವನ್ನು ಪಡೆದು ಕೊಳ್ಳಲು ಸ್ವಾಮೀಜಿಯ ಮನಸ್ಸು ಕಾತರಿಸಿತು. ಅದಕ್ಕನುಸಾರವಾಗಿ, ತಮ್ಮನ್ನು ಹರಸುವಂತೆ ಬೇಡಿಕೊಂಡು ಶ್ರೀಮಾತೆಯವರಿಗೊಂದು ಪತ್ರ ಬರೆದರು. ಆದರೆ ತಾವು ಅಮೆರಿಕೆಗೆ ಹೋಗ ಲಿರುವ ವಿಷಯವನ್ನು ಬೇರಾರಿಗೂ ತಿಳಿಸಬಾರದೆಂದು ಕೇಳಿಕೊಂಡಿದ್ದರು. ಈ ಪತ್ರವನ್ನು ಕಂಡು ಶ್ರೀಮಾತೆಯವರಿಗೆ ಆಗಿರಬಹುದಾದ ಆನಂದವನ್ನು ಭಾವಿಸಿಯೇ ನೋಡಬೇಕು. ಶ್ರೀರಾಮ ಕೃಷ್ಣರ ಶಿಷ್ಯಾಗ್ರಣಿ ಎಂಬ ಕಾರಣದಿಂದಷ್ಟೇ ಅಲ್ಲದೆ ಅವರ ಮೇಲೆ ಶಾರದಾದೇವಿಯವರಿಗೆ ವಿಶೇಷ ಮಮತೆ. ಎಷ್ಟೋ ವರ್ಷಗಳೇ ಕಳೆದ ಮೇಲೆ ತಮ್ಮ ಪ್ರಿಯ ನರೇಂದ್ರನ ಕ್ಷೇಮ ಸಮಾಚಾರ ಮತ್ತು ಅವನು ದೂರದೇಶಗಳಿಗೆ ಧರ್ಮಪ್ರಚಾರಕ್ಕಾಗಿ ಹೊರಟಿರುವ ವಿಷಯ ತಿಳಿದು ಅತ್ಯಾನಂದವಾಯಿತು. ಅಲ್ಲದೆ, ಶ್ರೀರಾಮಕೃಷ್ಣರ ಮಹಾಸಮಾಧಿಯ ಅನಂತರ ಅವರಿಗೂ ಒಂದು ದಿವ್ಯ ದರ್ಶನವಾಗಿತ್ತು. ಅದರಲ್ಲಿ ಅವರು ಶ್ರೀರಾಮಕೃಷ್ಣರ ಚೇತನವು ನರೇಂದ್ರನ ದೇಹವನ್ನು ಪ್ರವೇಶಿಸಿ, ಅವನ ಮೂಲಕ ಜಗತ್ಕಲ್ಯಾಣಕಾರ್ಯದಲ್ಲಿ ತೊಡಗಿದಂತೆ ಕಂಡಿದ್ದರು. ಈಗ ಅವರಿಗೆ ಆ ದೃಶ್ಯ ನೆನಪಿಗೆ ಬಂದಿತು. ಆದರೆ ತಮ್ಮ ಮಗುವನ್ನು ಸಮುದ್ರ ದಾಚೆಗಿನ, ಅಷ್ಟು ದೂರದ ನಾಡಿಗೆ ಕಳಿಸಿಕೊಡಲು, ಮೊದಮೊದಲು ಅವರ ಮಾತೃಹೃದಯ ಒಪ್ಪಲಿಲ್ಲ. ಕಡೆಗೂ ಇದೂ ಸ್ವಯಂ ಶ್ರೀರಾಮಕೃಷ್ಣರ ಕಾರ್ಯವೇ, ಮತ್ತು ಅವರ ಇಚ್ಛೆಯೇ ಎಂದು ಸಮಾಧಾನ ತಂದುಕೊಂಡರು. ಬಳಿಕ ತಮ್ಮ ವೈಯಕ್ತಿಕ ಭಾವನೆಗಳನ್ನು ಬದಿಗೊತ್ತಿ, ಕೆಲವು ಬುದ್ಧಿಮಾತುಗಳೊಂದಿಗೆ ತಮ್ಮ ಆಶೀರ್ವಾದಪೂರ್ವಕ ಅನುಮತಿಯನ್ನು ಬರೆದು ಕಳಿಸಿ ದರು. ಅಲ್ಲದೆ ತಮಗಾದ ಆ ದಿವ್ಯದರ್ಶನದ ವಿಚಾರವನ್ನು ಈ ಪತ್ರದಲ್ಲಿ ತಿಳಿಸಿದರು.

ಶ್ರೀಮಾತೆಯವರ ಪತ್ರವನ್ನು ಓದಿ ಸ್ವಾಮೀಜಿ ಆನಂದಾತಿಶಯದಿಂದ ಒಮ್ಮೆ ಕುಣಿದಾಡಿ ದರು; ಭಾವವುಮ್ಮಳಿಸಿಬಂದು ಅತ್ತುಬಿಟ್ಟರು. ಬಳಿಕ ತಮ್ಮ ಭಾವಾವೇಶವನ್ನು ಮುಚ್ಚಿಕೊಳ್ಳಲು ಕೋಣೆಯೊಳಕ್ಕೆ ಸೇರಿಕೊಂಡರು. ಸ್ವಲ್ಪ ಹೊತ್ತಾದ ಮೇಲೆ ಸಮುದ್ರತೀರಕ್ಕೆ ಹೊರಟರು. ಆಗ ಅವರು ತಮ್ಮಷ್ಟಕ್ಕೆ ಹೇಳಿಕೊಂಡರು, “ಆಹ್, ಈಗ ಎಲ್ಲ ಸರಿಹೋಯಿತು. ಇದೆಲ್ಲ ಜಗ ನ್ಮಾತೆಯ ಇಚ್ಛೆಯೇ!” ಅವರು ಮನೆಗೆ ಹಿಂದಿರುಗಿದಾಗ ಅವರ ಮುಖ ವಿಶೇಷ ಕಾಂತಿಯಿಂದ ಬೆಳಗುತ್ತಿತ್ತು. ಅಷ್ಟು ಹೊತ್ತಿಗಾಗಲೇ ಅಲ್ಲಿ ಅನೇಕ ಶಿಷ್ಯರು ನೆರೆದಿದ್ದರು. ಮನೆಯೊಳಗೆ ಬರುತ್ತಿದ್ದಂತೆ ಸ್ವಾಮೀಜಿ ಉದ್ಗರಿಸಿದರು, “ಹೌದು! ಇನ್ನೀಗ ಪಶ್ಚಿಮಕ್ಕೆ ಹೊರಡುವುದೇ ಕೆಲಸ. ನಾನೀಗ ಸಿದ್ಧನಾಗಿದ್ದೇನೆ. ನಾವೆಲ್ಲ ತಡಮಾಡದೆ ಕಾರ್ಯಮಗ್ನರಾಗಬೇಕು. ಸ್ವಯಂ ಜಗ ನ್ಮಾತೆಯೇ ತನ್ನ ಅನುಮತಿಯನ್ನು ನೀಡಿದ್ದಾಳೆ!” ಇದನ್ನು ಕೇಳಿ ಅಲ್ಲಿದ್ದವರೆಲ್ಲ ಒಂದು ಕ್ಷಣ ವಿಸ್ಮಯಮೂಕರಾದರು. ಬಳಿಕ ನವೋತ್ಸಾಹದಿಂದ ಕೂಡಿ, ಕಾರ್ಯಮಗ್ನರಾದರು. ತಾವು ಈಗಾಗಲೇ ಸಂಗ್ರಹಿಸಿದ್ದ ಹಣದೊಂದಿಗೆ ಮತ್ತಷ್ಟನ್ನು ಸಂಗ್ರಹಿಸಲು ಹೊರಟರು.

