
\chapter{ಮುನ್ನುಡಿ}

\noindent

ನಾವು ಯಾವ ಜಗತ್ತಿನಲ್ಲಿ ಜೀವಿಸಿದ್ದೇವೆಯೋ ಅದರ ಕರ್ತೃವನ್ನು ‘ದೇವರು’ ಎಂದು ಕರೆಯು ತ್ತಾರೆ. ಈ ಜಗತ್ತಿನ ಸ್ಥಿತಿಗಳನ್ನು ನಿಯಮಿಸುವ ಭಗವಚ್ಛಕ್ತಿಗೆ ‘ಪುತ’ವೆಂದು ಹೆಸರು. ವೈಯಕ್ತಿಕ ಸ್ತರದಲ್ಲಿ ಇದು ‘ಸತ್ಯ’ವೆನಿಸಿಕೊಳ್ಳುತ್ತದೆ ಮತ್ತು ಸತ್ಯವಚನ, ವಚನಸತ್ಯ ಎಂಬ ಎರಡು ರೂಪ ಗಳಲ್ಲಿ ಅಭಿವ್ಯಕ್ತವಾಗುತ್ತದೆ. ಸಾಮಾಜಿಕ ಸ್ತರದಲ್ಲಿ, ಮಾವನ-ಮಾನವರ ಸಂಬಂಧದ ಪ್ರಶ್ನೆ ಬಂದಾಗ ಅದೇ ‘ಧರ್ಮ’ವಾಗುತ್ತದೆ. ಇಡೀ ಜಗತ್ತನ್ನು ಜನಾಂಗಗಳನ್ನು ಪ್ರಜೆಗಳನ್ನು ಪರಸ್ಪರ ಪ್ರೀತಿ ಸೌಹಾರ್ದಗಳಿಂದ ನಡೆದುಕೊಳ್ಳುವಂತೆ ಮಾಡುವ ತತ್ತ್ವವೇ ಧರ್ಮ. ನೀತಿ ನಿಯಮಗಳು, ಕರ್ತವ್ಯ, ದಾಯಿತ್ವ, ನ್ಯಾಯನಿಷ್ಠೆ–ಇವೆಲ್ಲ ಧರ್ಮದ ವಿವಿಧ ಮುಖಗಳು. ಧರ್ಮವು ದೃಢವಾಗಿ ದ್ದರೆ ಬಹುಜನಹಿತ ಬಹುಜನಸುಖಗಳು ಸಾಧ್ಯ. ಧರ್ಮವು ಕುಗ್ಗಿದರೆ ಸಮಾಜವು ಅಸ್ತವ್ಯಸ್ತ ವಾಗುತ್ತದೆ; ನಷ್ಟವಾದರೆ ಸರ್ವನಾಶವು ಕಟ್ಟಿಟ್ಟದ್ದು.

ಸಮಾಜದಲ್ಲಿ ಧರ್ಮದ ಸಮತೋಲವು ತಪ್ಪಿದಾಗ, ಏರುಪೇರಾದಾಗ, ಅವತಾರಪುರುಷರು ಆವಿರ್ಭವಿಸಿ ಧರ್ಮಸಂಸ್ಥಾಪನೆಯನ್ನು ಮಾಡುತ್ತಾರೆ. ಧರ್ಮ ಸಂಸ್ಥಾಪನೆಯು ದುಷ್ಟಶಿಕ್ಷಣ ಕ್ಕಿಂತಲೂ ಅಧಿಕವಾಗಿ ಶಿಷ್ಟರಕ್ಷಣೆಯನ್ನು ಅವಲಂಬಿಸಿರುತ್ತದೆ. ಶಿಷ್ಟರ ರಕ್ಷಣೆಗಿಂತಲೂ ಹೆಚ್ಚಾಗಿ, ದುಷ್ಟರನ್ನು ಅದುಷ್ಟರನ್ನಾಗಿಯೂ ಅದುಷ್ಟರನ್ನು ಸತ್ಪುರುಷರನ್ನಾಗಿಯೂ ಮಾಡುವ ಭಾರ ದೊಡ್ಡದು. ಶುದ್ಧ ಸಾತ್ತ್ವಿಕ ಅವತಾರವಾದ ಭಗವಾನ್ ಶ್ರೀರಾಮಕೃಷ್ಣರು ಸಾಧಿಸಿದ್ದು ಇದನ್ನೇ.

ಅವರ ಅದ್ಭುತ ಸಾಧನೆಗೆ ಮೂರು ಮುಖಗಳಿವೆ. ಸನಾತನಧರ್ಮದ ಹಾಗೂ ಇತರ ಧರ್ಮ ಗಳ ಎಲ್ಲ ಸಾಧನಾಪಥಗಳನ್ನೂ ಅನುಸರಿಸಿ ಅವೆಲ್ಲ ಒಂದೇ ಸತ್ಯದ ಸಾಕ್ಷಾತ್ಕಾರಕ್ಕೆ ಕರೆದೊಯ್ಯು ತ್ತವೆಯೆಂಬುದನ್ನು ಅನುಭವದಿಂದ ಕಂಡುಕೊಂಡದ್ದು ಮೊದಲನೆಯದು. ತಮ್ಮ ಅತ್ಯದ್ಭುತ ಅತೀಂದ್ರಿಯ ಅನುಭವಗಳನ್ನೂ ಅವುಗಳೆಡೆಗೆ ಕರೆದೊಯ್ಯುವ ಸಾಧನೆಗಳನ್ನೂ ಆಬಾಲ್ಯವೃದ್ಧ ರಿಗೂ ಅರ್ಥವಾಗುವಂತಹ ಸರಳ ಮಾತುಗಳಲ್ಲಿ ಉಪದೇಶಿಸಿದ್ದು ಎರಡನೆಯದು. ಈ ಪುಣ್ಯ ಕಾರ್ಯವು ಪ್ರಭಾವಶಾಲಿಯಾಗಿ ಶತಶತಮಾನಗಳವರೆಗೆ ಮುಂದುವರೆಯಲು ಬೇಕಾದ ಪವಿತ್ರ ಸಂನ್ಯಾಸೀ ಪರಂಪರೆಯೊಂದನ್ನು ನಿರ್ಮಿಸಿದ್ದು ಮೂರನೆಯದು.

ಈ ಮೂರನೆಯ ಮಹಾಕಾರ್ಯದ ಭಾರವನ್ನು ಅವರು ಹೇರಿದ್ದು ಸ್ವಾಮಿ ವಿವೇಕಾನಂದರ ವಿಶಾಲ ಭುಜಗಳ ಮೇಲೆ. ದೇಹ ಬಲ, ಅತೀವ ಬುದ್ಧಿಬಲ, ಅತ್ಯದ್ಭುತ ಆತ್ಮಬಲ ಇವನ್ನು ಹೊಂದಿದ್ದ ಸ್ವಾಮೀಜಿಯವರು ಈ ಗುರುತರ ಜವಾಬುದಾರಿಯನ್ನು ಅತ್ಯಂತ ಸಮರ್ಪಕವಾಗಿ ಯಶಸ್ವಿಯಾಗಿ ನೆರವೇರಿಸಿದರು.

ಇದರ ಅಂಗವಾಗಿಯೇ ಅವರು ಕಾಲ್ನಡಿಗೆಯಲ್ಲಿ ಇಡೀ ಭಾರತವನ್ನು ಸುತ್ತಿದ್ದು, ಪಡಬಾರದ ಕಷ್ಟಗಳನ್ನು ಪಟ್ಟಿದ್ದು. ಆದರೆ ಹಾಗೆ ಮಾಡುವಾಗ ಅವರು ಅತಿಸೂಕ್ಷ್ಮ ಮತ್ತು ತೀಕ್ಷ್ಣದೃಷ್ಟಿ ಯಿಂದ ಜನಜೀವನವನ್ನು ಪರಿಶೀಲಿಸಿ ನಮ್ಮ ಸಮಾಜದ ಅಂತಸ್ಸತ್ತ್ವವನ್ನು ಗುರುತಿಸಿದರು. ನಮ್ಮ ಸಮಸ್ಯೆಯ ಮೂಲವನ್ನೂ ಅದರ ಪರಿಹಾರವನ್ನೂ ಕಂಡುಕೊಂಡರು. ಪರಿಹಾರವನ್ನು ಕಾರ್ಯಗತ ಮಾಡಲು ಅವರು ಪಾಶ್ಚಾತ್ಯ ದೇಶಗಳಿಗೆ ಹೋಗಬೇಕಾಯಿತು. ದಾಸ್ಯದಲ್ಲಿ ನರಳುತ್ತಿದ್ದ ಭಾರತದಿಂದ ಸ್ವತಂತ್ರವೂ ಸಂಪದ್ಭರಿತವೂ ಚೇತನಾಯುಕ್ತವೂ ಆದ ಅಮೇರಿಕಕ್ಕೆ ಹೋದರೂ ದಾಸನಂತೆ ನೈಚ್ಯಾನುಸಂಧಾನ ಮಾಡದೆ ಚಿರಸ್ವತಂತ್ರ ಸಿಂಹದಂತೆ ಹೋದರು, ಗುಡುಗಿದರು, ವೇದಾಂತವಾಣಿಯ ವಾರಿಯನ್ನು ವರ್ಷಿಸಿದರು.

ಸ್ವಾಮೀಜಿಯವರ ಜೀವನಗಾಥೆಯ ಈ ಭಾಗ (ಸಂಪುಟ ೨) ಅತ್ಯಂತ ರೋಮಾಂಚಕರ ವಾದದ್ದು. ಸ್ವಾಮಿ ಪುರುಷೋತ್ತಮಾನಂದರಿಂದ ರಚಿತವಾದ ಈ ದ್ವಿತೀಯ ಸಂಪುಟದಲ್ಲಿ ಇದರ ಸ್ವಾರಸ್ಯಕರ ಮತ್ತು ವಿಸ್ತೃತ ವಿವರಣೆಯಿದೆ. ಡಿಸೆಂಬರ್ ೧೯೮೬ರಲ್ಲಿ ಇದರ ಪ್ರಥಮ ಮುದ್ರಣವು ಬೆಳಕನ್ನು ಕಂಡಿತು. ಮತ್ತೆ ೧೯೮೯ರಲ್ಲಿ ತಂದ ಸುಲಭ ಬೆಲೆಯ ಜನಪ್ರಿಯ ಆವೃತ್ತಿಯ ಪ್ರತಿಗಳೂ ಮುಗಿದವು. ಬೇಡಿಕೆಯು ಮಾತ್ರ ಮುಂದುವರಿಯುತ್ತಲೇ ಇದೆ. ಆದುದರಿಂದ ಈಗ ಮತ್ತೊಂದು ಮುದ್ರಣವನ್ನು ಹೊರತರುತ್ತಿದ್ದೇವೆ.

ನಮ್ಮ ಮಠದ ಭಕ್ತರೂ ಸುಸಂಸ್ಕೃತ ವಿದ್ಯಾಪಿಪಾಸುಗಳೂ ಇದನ್ನು ಎಂದಿನಂತೆ ಸ್ವಾಗತಿಸುವರೆಂದು ನಂಬಿರುತ್ತೇವೆ.

ಏಪ್ರಿಲ್ ೧೯೯೩

\begin{flushright}
\textbf{ಸ್ವಾಮಿ ಹರ್ಷಾನಂದ}\\ಅಧ್ಯಕ್ಷರು, ರಾಮಕೃಷ್ಣ ಮಠ, ಬೆಂಗಳೂರು
\end{flushright}

\chapter{ಒಂದು ಮಾತು...}

\noindent

“ವಿಶ್ವವಿಜೇತ ವಿವೇಕಾನಂದ” ಗ್ರಂಥವು ಸ್ವಾಮಿ ವಿವೇಕಾನಂದರ ಸಮಗ್ರ ಜೀವನ ಚರಿತೆಯ ದ್ವಿತೀಯ ಸಂಪುಟ. ವಿವೇಕಾನಂದರು ವೀರಸಂನ್ಯಾಸಿಯೆಂದ ಮೇಲೆ ವಿಶ್ವವಿಜೇತರಾಗುವುದು ತೀರ ಸಹಜವೇ! ಧಾರ್ಮಿಕ ಇತಿಹಾಸವನ್ನು ಅವಲೋಕಿಸಿದರೆ ನಮಗೆ ರಾಷ್ಟ್ರವಿಜೇತ ಸಂನ್ಯಾಸಿ ಗಳ ಸಂದರ್ಶನವಾಗುತ್ತದೆ. ಆದರೆ ಈಗ ನಾವು ಬದುಕಿರುವ ಈ ಶತಮಾನದಲ್ಲಿ ವಿಶ್ವವಿಜೇತ ನಾದ ಧೀರ ಸಂನ್ಯಾಸಿಯೋರ್ವನ ಸಮಕ್ಷಮದಲ್ಲಿ ನಿಂತಿದ್ದೇವೆ. ವಿವೇಕಾನಂದರ ಧೀರಗಂಭೀರ ನಿಲುವೇ ಅವರು ವಿಶ್ವವನ್ನು ಜಯಿಸಲು ಸಮರ್ಥರೆಂಬುದನ್ನು ಸಾರಿಹೇಳುತ್ತದೆ. ನಿರಂತರ ಕಾಂತಿಮಯ ಕಿರಣಗಳನ್ನು ಹೊಮ್ಮಿಸುತ್ತಿರುವ ಅವರ ಪ್ರಕಾಶಮಾನ ನಯನದ್ವಯವು ಪ್ರತಿ ದ್ವಂದ್ವಿಗಳ, ಪ್ರತಿವಾದಿಗಳ ಬಾಯಿಮುಚ್ಚಿಸಿ ಅವರ ಅವಿವೇಕವನ್ನು ಅವರಿಗೇ ತೋರಿಸಿ ಕೊಡುವಂತಿದೆ. ಇನ್ನು ಅವರ ಅಂತರಂಗ ವ್ಯಕ್ತಿತ್ವವೋ, ಅದು ಪರಬ್ರಹ್ಮ ಸಂಸ್ಪರ್ಶದಿಂದ ಪುನೀತಗೊಂಡದ್ದು; ಸತ್ಯಸಾಕ್ಷಾತ್ಕಾರದಿಂದ ಸಂಸ್ಕರಣಗೊಂಡದ್ದು. ಅಂತಹ ಗಂಭೀರ ವ್ಯಕ್ತಿತ್ವದಾಳದಿಂದ ಸಿಡಿಯುತ್ತಿದ್ದ ಅವರ ವಾಣಿಯಲ್ಲಿ ಲಕ್ಷ್ಮಿಯ ತೇಜಸ್ಸು, ಸರಸ್ವತಿಯ ಓಜಸ್ಸು. ಇಂತಹ ಒಬ್ಬ ಸಂನ್ಯಾಸಿ ಅಂದಿನ ಗುಲಾಮರಾಷ್ಟ್ರವಾಗಿದ್ದ ಬಡಭಾರತದಿಂದ ಪುಟಿದೆದ್ದು ವಿಶ್ವದ ವೈಭವೋಪೇತ ರಾಷ್ಟ್ರವಾದ ಅಮೆರಿಕೆಗೆ ತೆರಳಿ ಅಲ್ಲಿನ ಜನಮನವನ್ನು ಭಾರತದ ಭವ್ಯ ಸನಾತನ ಧರ್ಮ-ಸಂಸ್ಕೃತಿಗಳೆಡೆಗೆ ಹರಿಯಿಸಿದ್ದು ಒಂದು ರೋಮಾಂಚಕಾರಿ ಘಟನೆಯೇ ಸರಿ.

ಅಮೆರಿಕದ ಪತ್ರಿಕೆಗಳು ಸ್ವಾಮಿ ವಿವೇಕಾನಂದರ ವ್ಯಕ್ತಿತ್ವವನ್ನೂ ಸಂದೇಶಗಳನ್ನೂ ನೂರಾರು ಬಗೆಗಳಲ್ಲಿ ಬಣ್ಣಿಸಿ ಕೊಂಡಾಡಿದುವು. ಒಂದು ಪತ್ರಿಕೆಯ ಸಂಪಾದಕೀಯ ಹೀಗೆಂದಿತು: “ಸ್ವಾಮಿ ವಿವೇಕಾನಂದರನ್ನು ನೋಡುವ, ಅವರ ಭಾಷಣವನ್ನು ಕೇಳುವ ಅವಕಾಶವನ್ನು ಬುದ್ಧಿವಂತನಾದ ಯಾವನೂ ಕಳೆದುಕೊಳ್ಳಬಾರದು... ಪ್ರತಿಯೊಬ್ಬ ಬುದ್ಧಿಜೀವಿಯೂ ಅವರ ವ್ಯಕ್ತಿತ್ವವನ್ನು ಅಧ್ಯಯನ ಮಾಡಬೇಕು.”

ಇದು ಬಹಳ ಮುಖ್ಯವಾದ ಮಾತು–“ಸ್ವಾಮಿ ವಿವೇಕಾನಂದರ ವ್ಯಕ್ತಿತ್ವವನ್ನು ಅಧ್ಯಯನ ಮಾಡಬೇಕು.” ಅವರ ಜೀವನಚರಿತ್ರೆಯನ್ನು ಕೇವಲ ಓದಿ-ಬಿಡುವುದಲ್ಲ. ಅಧ್ಯಯನ ಮಾಡ ಬೇಕು. ವಿವೇಕಾನಂದರನ್ನು ಅಧ್ಯಯನ ಮಾಡುವುದೆಂದರೆ ಸಮಗ್ರ ಭಾರತವನ್ನು ಒಳಹೊಕ್ಕು ವೀಕ್ಷಿಸಿದಂತೆ. ಸಮಗ್ರ ಹಿಂದೂಧರ್ಮವನ್ನು ಅರಿತು ಮೈಗೂಡಿಸಿಕೊಂಡಂತೆ. ಅಷ್ಟೇ ಅಲ್ಲ, ವಿಶ್ವಭ್ರಾತೃತ್ವದ ವಿಶಾಲ ಭಾವವನ್ನು ಅಭ್ಯಸಿಸಿದಂತೆ; ಪರಿಪೂರ್ಣ ಮಾನವತೆಯ ಮರ್ಮವನ್ನು ಮಥಿಸಿ ಮನನ ಮಾಡಿದಂತೆ!

ಈ ಅಧ್ಯಯನದ ದಿಸೆಯಲ್ಲಿ ಈಗಾಗಲೇ ಬಹಳಷ್ಟು ಯಶಸ್ವೀ ಪ್ರಯತ್ನ ನಡೆದಿದೆ. ಕರ್ನಾಟಕದ ಹಲವೆಡೆ ಮೈದಾಳಿರುವ ‘ವಿವೇಕಾನಂದ ವಿಚಾರ ವೇದಿಕೆ’, ‘ವಿವೇಕಾನಂದ ಸೇವಾ ಕೇಂದ್ರ, ’, ‘ವಿವೇಕಾನಂದ ಅಧ್ಯಯನ ಕೇಂದ್ರ’ಗಳು ವಿವೇಕಾನಂದರ ಸಮಗ್ರ ಜೀವನ ಚರಿತ್ರೆಯ ಮೊದಲ ಎರಡು ಸಂಪುಟಗಳ ಮೇಲೆ ಶಾಲಾ-ಕಾಲೇಜುಗಳ ವಿದ್ಯಾರ್ಥಿಗಳಿಗಾಗಿ ಈಗಾಗಲೇ ಹಲವಾರು ಸ್ಪರ್ಧೆಗಳನ್ನು ನಡೆಸಿದ್ದು, ಅವು ಬಹಳಷ್ಟು ಸತ್ಪರಿಣಾಮ ಬೀರಿವೆ. ಸ್ಪರ್ಧೆಗಾಗಿ ಓದುವಾಗ ಸ್ವಲ್ಪವಾದರೂ ಆಳವಾದ ಅಧ್ಯಯನ ನಡೆದೇ ನಡೆಯುತ್ತದೆ. ತನ್ಮೂಲಕ ಸ್ವಾಮಿ ವಿವೇಕಾನಂದರ ಸ್ಫೂರ್ತಿಯ ಕಿಡಿ ಹೃದಯವನ್ನು ಬೆಳಗಿಸದಿರುವುದಿಲ್ಲ.

ವಿದ್ಯಾರ್ಥಿಗಳು-ಉಪಾಧ್ಯಾಯರು, ಸ್ತ್ರೀಯರು-ಪುರುಷರು, ಯುವಕರು-ಮುದುಕರು, ಅಧಿಕಾರಿ ಗಳು-ಮಂತ್ರಿಗಳು–ಹೀಗೆ ಸಮಾಜದ ಎಲ್ಲ ಸ್ತರಗಳ ಜನರೂ ವಿವೇಕಾನಂದರ ಜೀವನ- ಸಂದೇಶಗಳನ್ನು ಅಧ್ಯಯನ ಮಾಡಿದ್ದೇ ಆದರೆ, ರಾಷ್ಟ್ರವು ಅಪೇಕ್ಷಿಸುವ ಶ್ರೇಷ್ಠ ವ್ಯಕ್ತಿಗಳ ನಿರ್ಮಾಣವಾಗುವುದು ಖಂಡಿತ.

ವಿವೇಕಾನಂದರನ್ನು ಅರಿತು ಗುರುತಿಸಿದ ಅನೇಕ ಪಾಶ್ಚಾತ್ಯ ಮಹಿಳೆಯರ ಉಲ್ಲೇಖವಿದೆ ಈ ಗ್ರಂಥದಲ್ಲಿ. ಅಮೆರಿಕದ ಮಹಿಳೆಯೊಬ್ಬಳು ಬರೆಯುತ್ತಾಳೆ:

“ಓಹ್, ವಿವೇಕಾನಂದರ ದಿವ್ಯಬೋಧನೆಗಳೆಂಥವು! ಅಲ್ಲಿ ಅರ್ಥಹೀನವಾದದ್ದೊಂದೂ ಇಲ್ಲ....ಅವುಗಳನ್ನು ಕೇಳಿದ ಮೇಲೆ ಇನ್ನು ನಾನು ಹಳೆಯ ‘ನಾನಾ’ಗಿ ಉಳಿಯುವುದಿಲ್ಲವೆಂದು ನನಗನ್ನಿಸುತ್ತದೆ–ಏಕೆಂದರೆ, ಅವುಗಳಲ್ಲಿ ನಾನು ಪರಮಸತ್ಯದ (ಭಗವಂತನ) ಇಣುಕುನೋಟ ವೊಂದನ್ನು ಪಡೆದುಕೊಂಡಿದ್ದೇನೆ.”

ಲಂಡನ್ನಿನ ಮಹಿಳೆಯೊಬ್ಬಳು ಹೇಳುತ್ತಾಳೆ:

“ಚರ್ಚುಗಳಲ್ಲಿ ನಡೆಯುವ ಎಷ್ಟೋ ಆರಾಧನೆಗಳಲ್ಲಿ ನಾನು ಜೀವನದುದ್ದಕ್ಕೂ ಭಾಗವಹಿಸಿ ದ್ದೇನೆ. ಉಪನ್ಯಾಸಗಳನ್ನು ಕೇಳಿದ್ದೇನೆ. ಆದರೆ ತಮ್ಮ ಸತ್ತ್ವಹೀನತೆಯಿಂದಾಗಿ ಅವೆಲ್ಲ ತೀರ ಸಪ್ಪೆಯಾಗಿ ತೋರಿದ್ದವು. ಇತರರು ಹೋಗುತ್ತಾರಲ್ಲ ಅಂತ ನಾನು ಹೋಗುತ್ತಿದ್ದೆ. ಆದರೆ ವಿವೇಕಾನಂದರ ಮಾತುಗಳನ್ನು ಕೇಳಿದಾಗಿನಿಂದ, ಧರ್ಮದೊಳಕ್ಕೆ ಹೊಸ ಬೆಳಕೊಂದು ಹರಿದು ಬಂದಂತೆ ತೋರುತ್ತಿದೆ. ಧರ್ಮವೆಂಬುದು ಜೀವಂತ, ಅದು ಸತ್ಯ! ಈಗ ಅದಕ್ಕೊಂದು ಹೊಸ ಆಹ್ಲಾದಕರ ಅರ್ಥವಿದೆ; ಮತ್ತು ಧರ್ಮದ ಸ್ವರೂಪವು ನನ್ನ ಪಾಲಿಗೆ ಸಂಪೂರ್ಣ ಪರಿವರ್ತಿತವಾಗಿ ಕಾಣುತ್ತಿದೆ.”

ಹೀಗೆ, ವಿವೇಕಾನಂದರ ಜೀವಿತಾವಧಿಯಲ್ಲಿ ಅವರನ್ನು ಗುರುತಿಸಿ ಅನುಸರಿಸಿದ ಪಾಶ್ಚಾತ್ಯ ಮಹಿಳೆಯರು ಅನೇಕರಿದ್ದರಾದರೂ, ಭಾರತದ ನಾರಿಯರು ಅವರನ್ನು ವಿಶೇಷವಾಗಿ ಕಂಡು ಕೊಂಡ ಉದಾಹರಣೆಗಳು ಹೆಚ್ಚಾಗಿ ಸಿಗದಿರುವುದು ಆಶ್ಚರ್ಯದ ಸಂಗತಿಯೇ ಸರಿ. ಆದರೆ ಇಂದಿನ ಸ್ವತಂತ್ರ ಸುಧಾರಿತ ಭಾರತದ ಸ್ತ್ರೀಯರು ವಿವೇಕಾನಂದರನ್ನು ಗುರುತಿಸಲು ಸಮರ್ಥ ರಾಗಿದ್ದಾರೆ ಎಂಬುದಕ್ಕೆ ಅಲ್ಲಲ್ಲಿ ಕೆಲವು ಸ್ತ್ರೀಯರೇ ಸೇರಿ ಅವರ ಹೆಸರಿನಲ್ಲಿ ಸಂಘ-ಸಂಸ್ಥೆಗಳನ್ನು ಕಟ್ಟಿಕೊಂಡು ವ್ಯಕ್ತಿನಿರ್ಮಾಣ, ಸಮಾಜನಿರ್ಮಾಣಕಾರ್ಯಗಳಲ್ಲಿ ನಿರತರಾಗಿರುವುದೇ ಸಾಕ್ಷಿ.

ನಮ್ಮ ರಾಷ್ಟ್ರವನ್ನು ಕಟ್ಟಬೇಕಾದವರು ನಾವೇ ಹೊರತು ಪರಕೀಯರಲ್ಲ. ಆದರೆ ರಾಷ್ಟ್ರ ನಿರ್ಮಾಣಕಾರ್ಯ ಯಶಸ್ವಿಯಾಗುವುದು ಮೂರ್ಖರಿಂದಲ್ಲ, ಬಾಯಿಬಡುಕರಿಂದಲ್ಲ; ಶೀಲ ವಂತರಿಂದ, ವಿವೇಕಿಗಳಿಂದ. ಅಂತಹ ಶೀಲವಂತ ವಿವೇಕಿಗಳ ನಿರ್ಮಾಣಕ್ಕೆ ಈ ಗ್ರಂಥದ ಅಧ್ಯಯನ ಮಹತ್ಪರಿಣಾಮಕಾರಿ.

ನಮ್ಮ ರಾಷ್ಟ್ರದ ಕೋಟಿಕೋಟಿ ಜನ ಸಿನಿಮಾ-ಟಿ. ವಿ.ಗಳನ್ನು ನೋಡಿಕೊಂಡು ವಿಲಾಸದಲ್ಲಿ ಮೈಮರೆಯುತ್ತ ಖುಷಿಯಾಗಿರಲಿ, ನಷ್ಟವೇನೂ ಇಲ್ಲ. ಆದರೆ ಬುದ್ಧಿವಂತರೂ ಜವಾಬ್ದಾರಿ ಅರಿತವರೂ ವಿವೇಕಿಗಳೂ ಆದ ಚಿಂತನಶೀಲ ಜನರು ಸ್ವಾಮಿ ವಿವೇಕಾನಂದರನ್ನು ಅಧ್ಯಯನ ಮಾಡಿ ರಾಷ್ಟ್ರನಿರ್ಮಾಣಕ್ಕೆ ಟೊಂಕ ಕಟ್ಟದಿದ್ದರೆ ಪ್ರಮಾದವಾಗದಿರದು.

\begin{flushright}
\textbf{ಸ್ವಾಮಿ ಪುರುಷೋತ್ತಮಾನಂದ}
\end{flushright}

