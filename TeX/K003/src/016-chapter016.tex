
\chapter{ವಿರೋಧದ ಚಕ್ರವ್ಯೂಹದಲ್ಲಿ}

\noindent

ಇತ್ತ ಸ್ವಾಮೀಜಿ, ಸರ್ವಧರ್ಮ ಸಮ್ಮೇಳನದ ಆರಂಭದಿಂದಲೂ ಅಮೆರಿಕದಲ್ಲಿ ತಮ್ಮ ಪ್ರಖರ ಭಾಷಣಗಳ ಮೂಲಕ ಹಾಗೂ ವ್ಯಕ್ತಿತ್ವದ ಮೂಲಕ ಉತ್ಕ್ರಾಂತಿ ಮಾಡುತ್ತಿದ್ದರೆ, ಅತ್ತ ಭಾರತ ದಲ್ಲೂ ಅವರ ಸಮಾಚಾರ ಸ್ವಲ್ಪಸ್ವಲ್ಪವಾಗಿ ಬರಲಾರಂಭಿಸಿತ್ತು. ಆದರೆ ಆಗಿನ ದಿನಗಳಲ್ಲಿ ಸಮೂಹ ಮಾಧ್ಯಮವೂ ದೂರಸಂಪರ್ಕ ವ್ಯವಸ್ಥೆಯೂ ಶೈಶವಾವಸ್ಥೆಯಲ್ಲಿದ್ದುದರಿಂದ ಎಲ್ಲವೂ ಬಹಳ ತಡ-ನಿಧಾನ. ಅಮೆರಿಕದಿಂದ ಸ್ವಾಮೀಜಿಯ ದಿಗ್ವಿಜಯದ ಕುರಿತಾದ ವರದಿ ಗಳೆಲ್ಲ ಭಾರತಕ್ಕೆ ತಲುಪಿದಾಗ ಭಾರತದ ಪತ್ರಿಕೆಗಳು ಅವುಗಳನ್ನು ಉತ್ಸಾಹದಿಂದ ವರದಿ ಮಾಡಿ ದ್ದನ್ನು ಈ ಹಿಂದೆಯೇ ನೋಡಿದ್ದೇವೆ. ಇವುಗಳನ್ನು ಓದಿ ತಿಳಿದ ಭಾರತೀಯರೆಲ್ಲರೂ– ಮುಖ್ಯವಾಗಿ ವಿದ್ಯಾವಂತ ವರ್ಗದವರು–ಪುಳಕಿತರಾದರು. ಇಡೀ ಭಾರತಕ್ಕೆ ಇದೊಂದು ಹೊಸ ಅನುಭವ. ಸಮಸ್ತ ಜಗತ್ತಿಗೇ ನೀಡುವಂತಹ ಅಮೂಲ್ಯ ಸಂಪತ್ತು ತನ್ನಲ್ಲಿದೆ ಎಂಬುದರ ಅರಿವೇ ಭಾರತೀಯರಿಗೊಂದು ಹೊಸ ಸ್ಫೂರ್ತಿಯನ್ನು ನೀಡಿತ್ತು. ತನ್ನ ಪ್ರತಿನಿಧಿಯು ವಿಶ್ವವೇದಿಕೆಯಲ್ಲಿ ಸರ್ವಶ್ರೇಷ್ಠನೆಂದು ಗೌರವಿಸಲ್ಪಟ್ಟದ್ದು ಭಾರತಕ್ಕೆ ಅಭಿಮಾನದ ಸಂಗತಿಯಾಗಿತ್ತು.

ಆದರೆ ಸ್ವಾಮೀಜಿಯ ಕಾರ್ಯದ ಮಹತ್ವವನ್ನು ಗುರುತಿಸಿದ ಭಾರತೀಯರು ಅವರನ್ನು ಹೃತ್ಪೂರ್ವಕವಾಗಿ ಅಭಿನಂದಿಸಿದರೂ, ಭಾರತೀಯರ ಪ್ರತಿಕ್ರಿಯೆಯು ಅಮೆರಿಕದ ಜನರನ್ನು ತಲುಪಲೇ ಇಲ್ಲ. ಇದಕ್ಕೆ ಅನೇಕ ಕಾರಣಗಳಿದ್ದುವು. ಮೊದಲನೆಯದಾಗಿ, ಸ್ವಾಮೀಜಿ ಯಾವೊಂದು ಸಂಸ್ಥೆಗೂ ಸೇರಿದವರಾಗಿರಲಿಲ್ಲ. ಅವರ ಶಿಷ್ಯರು-ಬೆಂಬಲಿಗರೆನ್ನಿಸಿಕೊಂಡವರು ಭಾರತದ ಉದ್ಗಲಕ್ಕೂ ಹಲವಾರು ಜನರಿದ್ದರೂ, ಇವರ ನಡುವೆ ಸಂಘಟನೆಯೇ ಇರಲಿಲ್ಲ. ಅತ್ತ ಕಲ್ಕತ್ತದಲ್ಲಿ ಶ್ರೀರಾಮಕೃಷ್ಣ ಸಂಘವಂತೂ ಅಸ್ತಿತ್ವದಲ್ಲಿದ್ದುದೇ ಯಾರಿಗೂ ಗೊತ್ತಿರಲಿಲ್ಲ. ಆ ಸಂಘದ ಸಂನ್ಯಾಸಿಗಳೂ ದಿಕ್ಕಾಪಾಲಾಗಿ ಚದುರಿಹೋಗಿದ್ದರು. ಸ್ವಾಮೀಜಿಯ ಯಶಸ್ಸಿನ ಬಗ್ಗೆ ಕೇಳಿ ಸಂತೋಷ ಪಟ್ಟವರಿಗೂ, ತಮ್ಮ ಮೆಚ್ಚುಗೆಯನ್ನು ಅವರಿಗೆ ತಿಳಿಸಬೇಕೆಂದು ಅನ್ನಿಸಲಿಲ್ಲ. ಈ ರೀತಿಯ ಅಭಿನಂದನೆಯಿಂದ, ಮೆಚ್ಚುಗೆಯಿಂದ ಸ್ವಾಮೀಜಿಗೆ ವೈಯಕ್ತಿಕವಾಗಿ ಆಗಬೇಕಾದ ದ್ದೇನೂ ಇರಲಿಲ್ಲ. ಯಾವುದೇ ಬಗೆಯ ಹೊಗಳಿಕೆಯನ್ನೂ ಕೀರ್ತಿಯನ್ನೂ ಅವರು ಇಷ್ಟಪಡು ತ್ತಲೂ ಇರಲಿಲ್ಲ. ಆದರೂ ಕೂಡ ಭಾರತೀಯರಿಂದ ಅವರಿಗೆ ಒಂದೇ ಒಂದು ಕೃತಜ್ಞತೆಯ ಇಲ್ಲವೆ ಅಭಿನಂದನೆಯ ಪತ್ರವೂ ತಲುಪದಿದ್ದುದು, ಅವರಿಗೆ ಹೇಳಲಾರದ ತೊಂದರೆಯನ್ನು ಉಂಟುಮಾಡಿತು. ಆದರೆ ಅವರ ವಿರೋಧಿಗಳಿಗೆ ಇದೊಂದು ವರವಾಗಿ ಪರಿಣಮಿಸಿತು. ವಿವೇಕಾ ನಂದರು ಭಾರತೀಯರ, ಹಿಂದೂಗಳ ಅಧಿಕೃತ ಪ್ರತಿನಿಧಿಯೇ ಅಲ್ಲವೆಂದು ಅಮೆರಿಕನ್ನರನ್ನು ನಂಬಿಸುವಲ್ಲಿ ಅವರ ವಿರೋಧಿಗಳು ಸಾಕಷ್ಟು ಯಶಸ್ವಿಯಾದರು. ಅತ್ಯಂತ ಸಂಘಟಿತರಾಗಿ ಆಕ್ರಮಣ ಮಾಡಿದ ತಮ್ಮ ವಿರೋಧಿಗಳೆದುರು ಸ್ವಾಮೀಜಿ ಏಕಾಕಿಯಾದರು, ಅತಂತ್ರರಾದರು.

ಸ್ವಾಮೀಜಿ ತಮ್ಮ ಅಪ್ರತಿಮ ವ್ಯಕ್ತಿತ್ವದಿಂದಲೂ ಭಾವನೆಗಳಿಂದಲೂ ಸರ್ವಧರ್ಮ ಸಮ್ಮೇಳನದಲ್ಲಿ ಅಭೂತಪೂರ್ವ ಯಶಸ್ಸನ್ನು ಗಳಿಸಿದಂದಿನಿಂದಲೇ ಅವರಿಗೆ ವೈರಿಗಳು ಜನಿಸಿ ದರು. ದಿನದಿನಕ್ಕೂ ಈ ವೈರಿಗಳ ಸಂಖ್ಯೆ ಏರುತ್ತ ಬಂದಿತು. ಭಾರತದ ಒಬ್ಬ ಅನಾಮಧೇಯ ಸಂನ್ಯಾಸಿ ಅಮೆರಿಕದಂತಹ ರಾಷ್ಟ್ರದಲ್ಲಿ ಸರ್ವಧರ್ಮ ಸಮ್ಮೇಳನದ ಉನ್ನತ ವೇದಿಕೆಯನ್ನೇರು ವುದರೆಂದರೇನು! ಅಲ್ಲಿನ ಪಂಡಿತ ದಿಗ್ದಂತಿಗಳ ಮುಂದೆ ಅತ್ಯುನ್ನತ ವಿಚಾರಗಳ ಬಗ್ಗೆ ಮಾತನಾಡುವುದೆಂದರೇನು! ಆ ಪಾಶ್ಚಾತ್ಯರ ಭಾಷೆಯಲ್ಲೇ ನಿರರ್ಗಳವಾಗಿ, ಅಧಿಕಾರವಾಣಿಯಿಂದ ಮಾತನಾಡಿ, ‘ದೈವದತ್ತವಾಗ್ಮಿ’ಯೆಂಬ ಕೀರ್ತಿಗೆ ಪಾತ್ರನಾಗುವುದೆಂದರೇನು! ಧರ್ಮಾಂಧ ಕ್ರೈಸ್ತರ ಎದೆಯಲ್ಲಿ ನಡುಕವುಂಟಾಗಲು ಇಷ್ಟೇ ಸಾಕಾಗಿತ್ತು. ಇದರೊಂದಿಗೆ ಸ್ವಾಮೀಜಿ, ಭಾರತದ ನಿಜಸ್ಥಿತಿಯನ್ನು ಜನರಿಗೆ ಮನವರಿಕೆ ಮಾಡಿಕೊಡಲು ತೊಡಗಿದಾಗ ಕಂಗಾಲಾದ ಮಿಷನರಿಗಳೂ ಇತರ ಅಸೂಯಾಪರ ವ್ಯಕ್ತಿಗಳೂ ಸ್ವಾಮೀಜಿಯನ್ನು ನಿರ್ನಾಮ ಮಾಡಲು ಕಂಕಣಬದ್ಧರಾದರು. ಇದರೊಂದಿಗೆ ಸ್ವಾಮೀಜಿ ನಡೆಸಿದ ಹೋರಾಟದ ವಿವರಗಳೆಲ್ಲ ಓದುಗರಿಗೆ ಹಿತವೆನೆನಿಸದಿರಬಹುದು. ಆದರೆ ಸ್ವಾಮೀಜಿಯ ಶೀಲಸಂಪನ್ನವಾದ ಧೀರ ಗಂಭೀರ ವ್ಯಕ್ತಿತ್ವದ ಒಂದು ದೃಶ್ಯವು ವ್ಯಕ್ತವಾಗುವುದು ಈ ಸಂದರ್ಭದಲ್ಲೇ.

ಭಾರತೀಯರೆಲ್ಲ ಅರೆನಾಗರಿಕ ಕಾಡುಜನರೆಂಬ ನಂಬಿಕೆಯು ಅಮೆರಿಕದ ಜನಮನದಲ್ಲಿ ದೃಢ ವಾಗಿ ನೆಲೆನಿಲ್ಲುವಂತೆ ಮಾಡಿದ್ದವರು ಅಲ್ಲಿನ ಮಿಷನರಿಗಳು. ಇಲ್ಲಿನ ‘ಬಡ ಅನಾಗರಿಕರನ್ನು ಉದ್ಧಾರದ ದಾರಿಗೆ ಎಳೆದುತರುವ ಮಹಾಕಾರ್ಯ’ಕ್ಕಾಗಿ ಅವರು ಅಪಾರ ಹಣವನ್ನು ಸಂಗ್ರಹಿಸು ತ್ತಿದ್ದರು. ಅವರು ಹೇಳಿದ್ದನ್ನೇ ನಿಜವೆಂದು ನಂಬಿದ ಅಮೆರಿಕದ ಶ್ರೀಮಂತ ಜನರು, ಭಾರತದ ‘ಅನಾಗರಿಕ’ರಿಗಾಗಿ ಮರುಗಿ, ಉದಾರವಾಗಿ ಸಹಾಯ ಮಾಡುತ್ತಿದ್ದರು. ಆದರೆ ಈ ಅನಾಗರಿಕರ ಪ್ರತಿನಿಧಿಯಾದ ವಿವೇಕಾನಂದರನ್ನು ಕಂಡು ಜನ ಆಶ್ಚರ್ಯಚಕಿತರಾದರು. ಅವರ ಸಂದೇಶದ ಔನ್ನತ್ಯವನ್ನು ಕಂಡು, ಇಂಥವರಿಗೆ ಧರ್ಮಬೋಧನೆ ಮಾಡಲು ಪ್ರಚಾರಕರನ್ನು ಕಳಿಸಿಕೊಡುವ ತಮ್ಮ ಮೌಢ್ಯಕ್ಕಾಗಿ ನಾಚಿದರು. ‘ಭಾರತಕ್ಕೆ ಬೇಕಾಗಿರುವುದು ಧರ್ಮವಲ್ಲ, ಅನ್ನ’ ಎಂಬುದನ್ನು ಸ್ವಾಮೀಜಿ ಜನರ ಮನಮುಟ್ಟುವಂತೆ ಸಾರಿದರು.

ನಿಜಕ್ಕೂ ಇದು ಎಷ್ಟು ಪರಿಣಾಮಕಾರಿಯಾಗಿತ್ತೆಂದರೆ, ಕ್ರೈಸ್ತ ಮಿಷನರಿಗಳ ಮೇಲೆ ಅದೊಂದು ಬಲವಾದ ಆಘಾತವಾಗಿ ಕೆಲಸ ಮಾಡಿತು. ಈ ಸಂಸ್ಥೆಗಳ ಆದಾಯ ಇದ್ದಕ್ಕಿದ್ದಂತೆ ಇಳಿದುದರಲ್ಲಿ ಇದು ಸ್ಪಷ್ಟವಾಗಿ ವ್ಯಕ್ತವಾಯಿತು. ಆ ಕುರಿತಾಗಿ ಕ್ರೈಸ್ತ ಮಿಷನರಿಗಳ ಒಂದು ಅಧಿಕೃತ ವರದಿಯಲ್ಲೇ ಹೀಗೆ ಬರೆಯಲಾಗಿತ್ತು–“ವಿವೇಕಾನಂದರ ಯಶಸ್ಸಿನ ಹಾಗೂ ಅವರ ಬೋಧನೆಗಳ ಪರಿಣಾಮವಾಗಿ ನಮ್ಮ ಭಾರತದಲ್ಲಿನ ಚಟುವಟಿಕೆಗಳಿಗೆ ಸಂಗ್ರಹವಾಗುತ್ತಿದ್ದ ಹಣದ ಪ್ರಮಾಣವು ಹತ್ತು ಲಕ್ಷ ಪೌಂಡುಗಳಷ್ಟು ಕಡಿಮೆಯಾಗಿದೆ.” ಅವರ ಪಾಲಿಗೆ ಇದೊಂದು ಬರಸಿಡಿಲೇ ಸರಿ. ಆದರೆ ಸ್ವಾಮೀಜಿಯ ಬೋಧನೆಗಳಲ್ಲಿ ಕ್ರೈಸ್ತ ಮಿಷನರಿಗಳ ಕುರಿತಾಗಿ ನಿಂದನೆಯ ಮಾತುಗಳಿರಲಿಲ್ಲ. ಅವು ಕೇವಲ ಸದುದ್ದೇಶ ಪೂರ್ವಕವಾದ ಸಲಹೆಗಳಾಗಿದ್ದುವಷ್ಟೇ. ಅವರಿಗೆ ಸ್ವಾಮೀಜಿ, ಅವರ ಕಾರ್ಯವಿಧಾನವನ್ನು ಬದಲಿಸಿಕೊಳ್ಳುವಂತೆ ಹೇಳಿದರು. ಆದರೆ ಇದನ್ನೆಲ್ಲ ತಾಳ್ಮೆಯಿಂದ ಆಲಿಸುವ ಸಾಮರ್ಥ್ಯ ಅವರಿಗಿರಲಿಲ್ಲ. ತಮ್ಮ ಕಾರ್ಯವಿಧಾನದಲ್ಲಿ ಸುಧಾರಣೆ ತಂದುಕೊಂಡು, ಹಿಂದೂಗಳ ಮನಸ್ಸನ್ನು ಇನ್ನೂ ಉತ್ತಮ ಮಾರ್ಗದಿಂದ ಒಲಿಸಿ ಕೊಳ್ಳುವ ಪ್ರಯತ್ನ ಮಾಡುವ ಬದಲಿಗೆ, ವಿವೇಕಾನಂದರನ್ನೇ ಧ್ವಂಸ ಮಾಡಿಬಿಡುವ ತಂತ್ರ ಹೂಡತೊಡಗಿದರು. ‘ಅವರೊಬ್ಬ ನೀತಿಗೆಟ್ಟವ್ಯಕ್ತಿ; ಯಾವ ಗೌರವಾನ್ವಿತ ಅಮೆರಿಕನ್ನನೂ ಅವರನ್ನು ಮನೆಗೆ ಆಹ್ವಾನಿಸುವುದು ತರವಲ್ಲ; ಅವರ ಬೋಧನೆಗಳೆಲ್ಲ ಅರ್ಥವಿಲ್ಲದ ಬಡ ಬಡಿಕೆ’ ಎಂದು ಸಾರಿದರು. ಸ್ವಾಮೀಜಿ ತಮ್ಮ ಪ್ರತಿನಿಧಿಯೆಂದು ಯಾವೊಬ್ಬ ಹಿಂದುವೂ ಒಪ್ಪಲಾರನೆಂದೂ ಆದ್ದರಿಂದ ಅಮೆರಿಕದವರೂ ಅವರನ್ನು ಬಹಿಷ್ಕರಿಸಬೇಕೆಂದೂ ಪ್ರಚಾರ ಮಾಡಿದರು. ಈ ಮಾತುಗಳನ್ನು ಪ್ರತಿಭಟಿಸುವಂತಹ ಅಥವಾ ಅವು ಸುಳ್ಳೆಂದು ಸಾಬೀತುಪಡಿಸು ವಂತಹ ಒಂದೇ ಒಂದು ದನಿಯೂ ಕೇಳಿಬರದಿದ್ದಾಗ ಅವರಾಡಿದ್ದೇ ಸತ್ಯವೆಂಬಂತಾಯಿತು!

ಇದರ ಜೊತೆಗೆ ಕ್ರೈಸ್ತಮಿಷನರಿಗಳಿಗೆ ಬೆಂಬಲವಾಗಿದ್ದ ಭಾರತೀಯರ ಪಾತ್ರವೂ ಅಲ್ಪವೇ ನಾಗಿರಲಿಲ್ಲ. ಸ್ವಾಮೀಜಿ ಅಮೆರಿಕೆಗೆ ಹೋಗುವ ಮೊದಲೇ ಅಲ್ಲಿ ತಳವೂರಿದ್ದ ಭಾರತದ ಬ್ರಾಹ್ಮಸಮಾಜದವರು ಹಾಗೂ ಥಿಯೊಸಾಫಿಕಲ್ ಸೊಸೈಟಿಯವರು ಅಮೆರಿಕದಲ್ಲಿ ಸಾಕಷ್ಟು ಪ್ರಭಾವಶಾಲಿಗಳಾಗಿದ್ದರು. ಆದರೆ ಸ್ವಾಮೀಜಿಯ ದೇದೀಪ್ಯಮಾನ ಕೀರ್ತಿಯೆದುರು ಇವರೆಲ್ಲ ಮೇಣದ ಬತ್ತಿಗಳಾದರು. ಇದರಿಂದಾಗಿ ಇವುಗಳ ನಾಯಕರು ಸ್ವಾಮೀಜಿಯ ಬಗ್ಗೆ ಅಸೂಯೆ ತಾಳಿದರು; ಅಸೂಯೆಯನ್ನು ಸಹಿಸಲಾರದೆ ಕ್ರೈಸ್ತ ಮಿಷನರಿಗಳೊಂದಿಗೆ ಕೈಸೇರಿಸಿ ಸ್ವಾಮೀಜಿಯ ವ್ಯಕ್ತಿತ್ವವನ್ನು ನಾಶಗೊಳಿಸುವುದರಲ್ಲಿ ನಿರತರಾದರು. ಇವರ ಪೈಕಿ ಮುಖ್ಯರಾದವರೆಂದರೆ, ಬ್ರಾಹ್ಮಸಮಾಜದ ‘ನವವಿಧಾನ’ ವಿಭಾಗದ ಮುಖ್ಯಸ್ಥನಾದ ಪ್ರತಾಪ್​ಚಂದ್ರ ಮಜುಮ್​ದಾರ್. ಇವನು ಪಾಶ್ಚಾತ್ಯರಿಗೆ ಪ್ರಿಯವಾಗುವಂತಹ ತತ್ತ್ವವಾದವನ್ನು ಬೋಧಿಸಿ ಜನಪ್ರಿಯನಾಗಿದ್ದ. ಇವನದು ಒಂದು ರೀತಿಯ ‘ಭಾರತೀಯತೆಯ ಗಂಧದಿಂದ ಕೂಡಿದ ಕ್ರೈಸ್ತಧರ್ಮ.’ ಮೊದ ಮೊದಲಿಗೆ ಈತ ಸ್ವಾಮೀಜಿಯೊಂದಿಗೆ ಚೆನ್ನಾಗಿಯೇ ನಡೆದುಕೊಂಡ. ಆದರೆ ಅವರು ಇದ್ದಕ್ಕಿ ದ್ದಂತೆ ಗಳಿಸಿದ ಯಶಸ್ಸನ್ನು ಕಂಡು ಮತ್ಸರಗೊಂಡ. ಸ್ವಾಮಿ ರಾಮಕೃಷ್ಣಾನಂದರಿಗೆ ಬರೆದ ಪತ್ರ ವೊಂದರಲ್ಲಿ ಸ್ವಾಮೀಜಿ ಹೇಳುತ್ತಾರೆ, “ಭಗವದಿಚ್ಛೆಯಂತೆ ನಾನಿಲ್ಲಿ ಮಜುಮ್​ದಾರನನ್ನು ಭೇಟಿ ಯಾದೆ. ಮೊದಮೊದಲು ಅವನು ನನ್ನೊಂದಿಗೆ ವಿಶ್ವಾಸವಾಗಿಯೇ ಇದ್ದ. ಆದರೆ ಯಾವಾಗ ಶಿಕಾಗೋ ನಗರದ ಸಮಸ್ತ ಜನಸ್ತೋಮವೇ ನನ್ನನ್ನು ಮುತ್ತಿಕೊಳ್ಳಲಾರಂಭಿಸಿತೋ, ಆಗ ಅವನ ತಲೆಯಲ್ಲಿ ಹುಳ ಕೊರೆಯತೊಡಗಿತು. ಸೋದರ, ಅವನ ವರ್ತನೆಯನ್ನು ಕಂಡು ನನಗೆ ಅತ್ಯಾಶ್ಚರ್ಯವಾಯಿತು. ಅವನು ಸಮ್ಮೇಳನಕ್ಕೆ ಬಂದ ಮಿಷನರಿಗಳ ಹತ್ತಿರ ಇಲ್ಲಸಲ್ಲದ ಅಪವಾದಗಳನ್ನು ಹರಡಿದ್ದಾನೆ. ‘ಇವನೊಬ್ಬ ದಿಕ್ಕುದಿವಾಣವಿಲ್ಲದವ, ಠಕ್ಕ; ಇಲ್ಲಿಗೆ ಬಂದು ಸಂನ್ಯಾಸಿಯಂತೆ ನಟನೆ ಮಾಡುತ್ತಿದ್ದಾನೆ’ ಎಂದು ಆರೋಪಿಸಿದ್ದಾನೆ. ಹೀಗೆ ಅವನು, ಇಲ್ಲಿನವರ ಮನಸ್ಸು ನನ್ನ ವಿರುದ್ಧವಾಗಿ ಪೂರ್ವಾಗ್ರಹಪೀಡಿತವಾಗುವಂತೆ ಮಾಡುವಲ್ಲಿ ಸಾಕಷ್ಟು ಯಶಸ್ವಿ ಯಾಗಿದ್ದಾನೆ. ಹೀಗೆಯೇ ಅವನು ಡಾ ॥ ಬರೋಸ್​ರವರ ಮನಸ್ಸಿನಲ್ಲೂ ಸಂಶಯದ ಬೀಜ ಬಿತ್ತಿದುದರಿಂದ ಅವರು ನನ್ನೊಂದಿಗೆ ಸರಿಯಾಗಿ ಮಾತನ್ನೂ ಆಡಲಿಲ್ಲ. ಅವರ ಪುಸ್ತಕಗಳಲ್ಲೂ ಕರಪತ್ರಗಳಲ್ಲೂ ನನ್ನ ಬಾಯಿ ಮುಚ್ಚಿಸಲು ಪ್ರಯತ್ನ ನಡೆದಿದೆ. ಆದರೆ ನನ್ನ ಗುರುದೇವನೇ ನನ್ನ ಸಹಾಯ. ಈ ಮಜುಮ್​ದಾರ ಏನು ತಾನೆ ಮಾಡಬಲ್ಲ?”

ಈ ಮಜುಮ್​ದಾರನ ಕೆಲಸ ಅಷ್ಟು ಸುಲಭವದ್ದೇನೂ ಆಗಿರಲಿಲ್ಲ. ಏಕೆಂದರೆ ಸ್ವಾಮೀಜಿ ಅದಾಗಲೇ ಅಮೆರಿಕದಲ್ಲಿ ಹಲವಾರು ವಿಶ್ವಾಸಿಗರನ್ನು ಸಂಪಾದಿಸಿಕೊಂಡಿದ್ದರು. ಒಂದು ದಿನ ಮಜುಮ್​ದಾರ ಸಣ್ಣ ಗುಂಪೊಂದನ್ನು ಉದ್ದೇಶಿಸಿ ಮಾತನಾಡುತ್ತ ವಿವೇಕಾನಂದರನ್ನೂ ಅವರ ಗುರುವಾದ ಶ್ರೀರಾಮಕೃಷ್ಣಪರಮಹಂಸರನ್ನೂ ದೂಷಿಸಲಾರಂಭಿಸಿದ. ಇದೇ ಮಜುಮ್​ದಾರ್ ಸುಮಾರು ಹದಿನೈದು ವರ್ಷಗಳ ಹಿಂದೆ, ಶ್ರೀರಾಮಕೃಷ್ಣರನ್ನು ಗುಣಗಾನ ಮಾಡಿ ಒಂದು ಕರಪತ್ರ ವನ್ನು ಬರೆದಿದ್ದ. ಸಮ್ಮೇಳನದ ಪ್ರಾರಂಭದ ದಿನಗಳಲ್ಲಿ ಸ್ವಾಮೀಜಿಯೊಂದಿಗೆ ತುಂಬ ಸ್ನೇಹ ಭಾವದಿಂದಿದ್ದ ಮಜುಮ್​ದಾರ, ಅವರಿಗೆ ತನ್ನ ಈ ಕರಪತ್ರಗಳನ್ನು ಕೊಟ್ಟಿದ್ದ. ತಮ್ಮ ಹಿನ್ನೆಲೆಯ ಬಗ್ಗೆ ತಿಳಿಯಲು ಆಸಕ್ತರಾದ ಕೆಲವರಿಗೆ, ತಮ್ಮ ಗುರುದೇವನ ಒಂದು ಸ್ಥೂಲ ವ್ಯಕ್ತಿಚಿತ್ರಣವನ್ನು ನೀಡಲು ಸ್ವಾಮೀಜಿ ಈ ಕರಪತ್ರಗಳನ್ನು ಹಂಚಿದ್ದರು. ಮಜುಮ್​ದಾರ ಶ್ರೀರಾಮಕೃಷ್ಣರನ್ನೇ ನಿಂದಿಸುತ್ತಿದ್ದ ಆ ಸಂದರ್ಭದಲ್ಲಿ ಅಲ್ಲಿದ್ದವನೊಬ್ಬ ಆ ಕರಪತ್ರವನ್ನು ಅವನ ಮುಂದೆ ಹಿಡಿದು, “ಇದನ್ನು ಬರೆದವರು ನೀವಲ್ಲವೆ?” ಎಂದು ಕೇಳಿದ. ಇದರಿಂದಾತ ತಬ್ಬಿಬ್ಬಾಗಿ ತಡವರಿಸಿದ. ಅವನು ಏನು ತಾನೆ ಹೇಳಲು ಸಾಧ್ಯವಿತ್ತು!

ಆದರೆ ಮಜುಮ್​ದಾರನೇನೂ ತನ್ನ ದುಷ್ಕೃತ್ಯಕ್ಕಾಗಿ ನಾಚಿಕೊಳ್ಳಲಿಲ್ಲ. ಕೆಲದಿನಗಳಲ್ಲೇ ಆತ ಭಾರತಕ್ಕೆ ಮರಳಿ, ಇಲ್ಲಿಯೂ ಅದನ್ನು ಬೇರೊಂದು ರೀತಿಯಲ್ಲಿ ಮುಂದುವರಿಸಿದ. ವಿವೇಕಾ ನಂದರು ಅಮೆರಿಕದಲ್ಲಿ ವ್ಯಭಿಚಾರದವರೆಗೆ ಎಲ್ಲ ಬಗೆಯ ಹೀನಕೃತ್ಯಗಳಲ್ಲಿ ನಿರತರಾಗಿದ್ದಾರೆ; ಸಕಲ ವಿಧದ ವರ್ಜ್ಯ ಪದಾರ್ಥಗಳನ್ನು ತಿಂದು ಭೋಗಿಸುತ್ತಿದ್ದಾರೆ ಎಂದು ಭಾರೀ ಪ್ರಚಾರ ಮಾಡಿದ. ‘ನವವಿಧಾನ’ ಬ್ರಾಹ್ಮಸಮಾಜದ \eng{Unity and the Minister} ಎಂಬ ಪತ್ರಿಕೆಯು ವಿವೇಕಾನಂದರು ನವಿಲುಗರಿಯ ವೇಷ ಧರಿಸಿದ ಕೆಂಬೂತ ಮಾತ್ರವೇ ಎಂಬುದನ್ನು ಸಾಧಿಸಿ ತೋರಿಸಿತು. ಸ್ವಾಮೀಜಿ ಬಾಸ್ಟನ್ನಿನಲ್ಲಿ ಭಾಷಣಗಳನ್ನು ಮಾಡುತ್ತ, ಅವರ ಜನಪ್ರಿಯತೆ ತಾರಕ ಕ್ಕೇರುತ್ತಿದ್ದ ಸಂದರ್ಭದಲ್ಲೇ ಅಲ್ಲಿನ \eng{Boston Daily Adveritser} ಪತ್ರಿಕೆ, ಮೇಲೆ ಹೇಳಿದ ಪತ್ರಿಕೆಯ ವರದಿಯನ್ನು ಉದ್ಧರಿಸಿತು:

\eng{“The Indian Mirror} ಪತ್ರಿಕೆಯು ತನ್ನ ಈಚಿನ ಕೆಲವು ಸಂಚಿಕೆಗಳಲ್ಲಿ ‘ನೂತನ ಹಿಂದು’ \textbf{ಬಾಬು ನೊರೇಂದ್ರನಾಥ್ ದತ್ತ } ಆಲಿಯಾಸ್​ \textbf{ವಿವೇಖಾನಂದ}ರನ್ನು ಪ್ರಶಂಸಿಸಿ ಹಲವಾರು ದೀರ್ಘಪತ್ರಗಳನ್ನು ಪ್ರಕಟಿಸಿದೆ. ಈ ಸಂನ್ಯಾಸಿಯ ಕುರಿತಾಗಿ ಹಾಗೆ ಗುಣಗಾನ ಮಾಡಿ ಪ್ರಕಟಿಸು ವುದರ ಬಗ್ಗೆ ನಮ್ಮ ಆಕ್ಷೇಪವೇನೂ ಇಲ್ಲ; ಆದರೆ ಈತ ‘ನವವೃಂದಾವನ ಥಿಯೇಟರ್​’ನ ರಂಗಮಂಟಪದ ಮೇಲೆ ಅಭಿನಯಿಸುವುದಕ್ಕೆಂದು ನಮ್ಮಲ್ಲಿಗೆ ಬಂದಾಗಿನಿಂದ, ಅಥವಾ ಈ ನಗರದ (ಕಲ್ಕತ್ತದ) ಬ್ರಾಹ್ಮಸಮಾಜಗಳಲ್ಲೊಂದರಲ್ಲಿ ಹಾಡುಗಳನ್ನು ಹಾಡಲು ತೊಡಗಿದಾಗಿ ನಿಂದ ನಮಗೆ ಈತನ ಪರಿಚಯ ಎಷ್ಟು ಚೆನ್ನಾಗಿ ಆಗಿದೆಯೆಂದರೆ, ಈತನ ಶೀಲದ ಬಗೆಗೆ ನಾವು ಹೊಂದಿರುವ ಅಭಿಪ್ರಾಯವು, ವೃತ್ತಪತ್ರಿಕೆಗಳಲ್ಲಿ ಪ್ರಕಟವಾದ ಯಾವುದೇ ಪ್ರಮಾಣದ ಬರಹ ದಿಂದಲೂ ಬದಲಾಗಲಾರದು. ನಮ್ಮ ಈ ಹಳೆಯ ಸ್ನೇಹಿತ, ಇತ್ತೀಚೆಗೆ ತನ್ನ ಭಾಷಣಗಳಿಂದ ಒಳ್ಳೆಯ ಅಭಿಪ್ರಾಯವನ್ನು ಮೂಡಿಸಿರುವುದು ನಮಗೆ ಸಂತೋಷವೇ. ಆದರೆ ಈ ನಮ್ಮ ಸ್ನೇಹಿತ ಪ್ರತಿನಿಧಿಸುವ ‘ನವ ಹಿಂದೂಧರ್ಮವು’ ಸಾಂಪ್ರದಾಯಿಕ ಹಿಂದೂಧರ್ಮವಲ್ಲ ಎಂಬುದರ ಅರಿವು ನಮಗೆ ಚೆನ್ನಾಗಿಯೇ ಇದೆ. ಒಬ್ಬ ಸಂಪ್ರದಾಯಸ್ಥ ಹಿಂದುವು ಮಾಡಲು ಸಾಧ್ಯವಿರುವ ಅತಿ ಹೀನ ಕಾರ್ಯವೆಂದರೆ ಕಾಲಾಪಾನಿಯನ್ನು (ಸಮುದ್ರವನ್ನು)ದಾಟುವುದು, ಮ್ಲೇಚ್ಛರ ಆಹಾರ ವನ್ನು ಸೇವಿಸುವುದು, ಕೊನೆಮೊದಲಿಲ್ಲದಂತೆ ಸಿಗಾರುಗಳನ್ನು ಸೇದುವುದು, ಇಂಥವು. ನಿಜವಾದ ಹಿಂದೂಧರ್ಮೀಯನಿಗೆ ನಾವು ಸಲ್ಲಿಸುವಂತಹ ಗೌರವವು, ಒಬ್ಬ ‘ಆಧುನಿಕ ಹಿಂದು’ವಿಗೆ ಎಂದೂ ಸಲ್ಲಲಾರದು. ಈ ನಮ್ಮ ಪ್ರಸ್ತುತ ವ್ಯಕ್ತಿ, \textbf{ವಿವೇಖಾನಂದ}ನೆಂಬ ಹೆಸರಿನಲ್ಲಿ ಪ್ರಖ್ಯಾತ ನಾಗಲು ಎಷ್ಟಾದರೂ ಪ್ರಯತ್ನಿಸಲಿ; ಆದರೆ ಆತ, ಕಣ್ಣು ಕುಕ್ಕುವ ಅಸಂಬದ್ಧ ವಿಚಾರಗಳನ್ನೆಲ್ಲ ಪ್ರಕಟಿಸಿದರೆ ಮಾತ್ರ ನಾವದನ್ನು ಸಹಿಸಿಕೊಳ್ಳಲು ಸಾಧ್ಯವಿಲ್ಲ.”

ಈ ಲೇಖನದಲ್ಲಿ ಸೂಚಿತವಾದ ಕುಹಕ ಸುಸ್ಪಷ್ಟವಾಗಿದೆ: ಆತ ತನ್ನನ್ನು ‘ವಿವೇಕಾನಂದ’ ನೆಂದು ಕರೆದುಕೊಂಡರೂ, ಸತ್ಯಾಂಶವೇನೆಂದರೆ ಆತ ‘ಬಾಬು ನರೇಂದ್ರನಾಥ ದತ್ತ’ನಷ್ಟೆ. ಸಮುದ್ರವನ್ನು ದಾಟಿದ್ದೂ ಅಲ್ಲದೆ, ಮ್ಲೇಚ್ಛರ ಆಹಾರವನ್ನು ತಿನ್ನುತ್ತ ಸಿಗಾರುಗಳನ್ನು ಸೇದುವ ಈ ವ್ಯಕ್ತಿ, ಸಾಂಪ್ರದಾಯಿಕ ಹಿಂದೂ ಸಮಾಜದಿಂದ ಬಹಿಷ್ಕೃತವಾದ ಕೃತ್ಯಗಳನ್ನು ಎಸಗಿದ್ದಾನೆ. ಅಲ್ಲದೆ ಆತ ಸ್ವರ್ಗದಿಂದ ಬಿದ್ದವನೇನಲ್ಲ; ಕೇವಲ ಒಬ್ಬ ಬಂಗಾಳೀ ಬಾಬು ಅಷ್ಟೆ! ಇನ್ನೂ ಬೇಕೆ ಈತನ ವಿಚಾರ? ಅವನು ಹಾಡು-ಕುಣಿತ-ನಾಟಕಗಳಲ್ಲೂ ಪಾಲ್ಗೊಂಡವನು! ಇಂತಹ ಒಬ್ಬ ಸಾಧಾರಣ ಮರ್ತ್ಯನನ್ನು–ಅದೂ ತನ್ನವರಿಂದಲೇ ಬಹಿಷ್ಕೃತನಾಗಬೇಕಾದ ಕಪಟಿಯನ್ನು –ಜನರು ಒಬ್ಬ ಪ್ರವಾದಿಯೆಂದು ಕೊಂಡಾಡುತ್ತಿದ್ದಾರಲ್ಲ!

ಸ್ವಾಮೀಜಿಯ ಬಗ್ಗೆ ಇಂತಹ ಹೊಸಹೊಸ ವಿಷಯಗಳನ್ನೆಲ್ಲ ಪತ್ತೆ ಹಚ್ಚಿ ವರದಿ ಮಾಡಿದ ಈ ಲೇಖನದಲ್ಲಿ ಮತ್ತೊಂದು ವಿಷಯ ತಪ್ಪಿಹೋಗಿತ್ತು. ಅದೇನೆಂದರೆ, ಬ್ರಾಹ್ಮಸಮಾಜದ ಆಶ್ರಯದಲ್ಲಿ ನಡೆದ ಆ ನಾಟಕದಲ್ಲಿ ನರೇಂದ್ರ ಮಾತ್ರವಲ್ಲ, ಅದರ ಅತಿಮುಖ್ಯ ಹಾಗೂ ಅತಿ ಪ್ರಸಿದ್ಧ ಮುಖಂಡರಲ್ಲೊಬ್ಬನಾದ ಕೇಶವಚಂದ್ರಸೇನನೂ ಭಾಗವಹಿಸಿದ್ದನೆಂಬುದು! ವಾಸ್ತವಿಕ ವಾಗಿ ಅದೊಂದು ಪೌರಾಣಿಕ ನಾಟಕ. ಈ ನಾಟಕದಲ್ಲಿ ಯೋಗಿಯ ಪಾತ್ರವನ್ನು ವಹಿಸುವಂತೆ, ‘ನವವಿಧಾನ’ದ ಪ್ರತಿಸ್ಪರ್ಧಿ ಸಂಘವಾದ ‘ಸಾಧಾರಣ’ ಬ್ರಾಹ್ಮಸಮಾಜದ ಸದಸ್ಯನೂ ಸಂಗೀತ- ನಾಟಕಗಳಲ್ಲಿ ಅತ್ಯಂತ ಪ್ರತಿಭಾವಂತನೂ ಆದ ತರುಣ ನರೇಂದ್ರನನ್ನು ಮನವೊಲಿಸಲಾಗಿತ್ತು. ಕೇಶವಚಂದ್ರಸೇನನೇ ಕಥಾನಾಯಕ. ಆದರೆ ಇದ್ದುದನ್ನು ಇದ್ದಂತೆ ಹೇಳಿದ್ದರೆ ಅಪವಾದಕ್ಕೆ ಅವಕಾಶವೆಲ್ಲಿರುತ್ತಿತ್ತು!

ತಮ್ಮ ವಿರುದ್ಧ ನಡೆಯುತ್ತಿದ್ದ ಈ ಅಪಪ್ರಚಾರವೆಲ್ಲ ಸ್ವಾಮೀಜಿಗೆ ತಿಳಿದೇ ಇತ್ತು. ಆದರೆ ಅವರು ಬಹಿರಂಗವಾಗಿ ತಮ್ಮನ್ನು ಸಮರ್ಥಿಸಿಕೊಳ್ಳುವ ಪ್ರಯತ್ನವನ್ನೇ ಮಾಡಲಿಲ್ಲ. ಇದು ಎಲ್ಲಿಯವರೆಗೆ ಹೋಗುತ್ತದೆಯೋ ನೋಡೋಣ ಎಂಬಂತೆ ಸುಮ್ಮನಿದ್ದು ಬಿಟ್ಟರು. ಅಮೆರಿಕ ದಲ್ಲಿ ಅವರ ವಿರುದ್ಧ ಕ್ರೈಸ್ತ ಮಿಷನರಿಗಳು ನಡೆಸುತ್ತಿದ್ದ ಅಪಪ್ರಚಾರದ ಸುದ್ದಿ ತಿಳಿದ ಭಾರತದ ಅವರ ಶಿಷ್ಯರು ಆ ಬಗ್ಗೆ ಆಶ್ಚರ್ಯವನ್ನು ವ್ಯಕ್ತಪಡಿಸಿದಾಗ, ೧೮೯೪ರ ಜನವರಿ ತಿಂಗಳಲ್ಲಿ ಅವರಿಗೆ ಸ್ವಾಮೀಜಿ ಬರೆದರು:

“ನನ್ನ ಬಗ್ಗೆ ನಿಮಗೆ ಅಷ್ಟೊಂದು ‘ಸಮಾಚಾರ’ ತಲುಪಿರುವುದನ್ನು ಕಂಡು ನಾನು ಬೆರಗಾದೆ. ನೀವು ಹೇಳುವ ಆ \eng{Interior} ಪತ್ರಿಕೆಯ ಟೀಕೆಯನ್ನು ಅಮೆರಿಕದ ಜನರ ಧೋರಣೆಯೆಂದು ತಿಳಿಯಬೇಕಾಗಿಲ್ಲ. ಆ ಪತ್ರಿಕೆಯ ಹೆಸರು ಇಲ್ಲಿ ಯಾರಿಗೂ ಗೊತ್ತೇ ಇಲ್ಲ ಎನ್ನಬಹುದು. ಅದು ‘ನೀಲಿ ಮೂಗಿನ ಪ್ರೆಸ್​ಬಿಟೀರಿಯನ್ ಪತ್ರಿಕೆ’ ಎಂದು ಕರೆಯಲ್ಪಡುವ ಜಾತಿಗೆ ಸೇರಿದ್ದು (ಪ್ರೆಸ್​ಬಿಟೀರಿಯನ್ ಎಂಬುದು ಕ್ರೈಸ್ತರಲ್ಲಿ ಒಂದು ಪಂಗಡ); ಅತ್ಯಂತ ಮತಾಂಧ ಪತ್ರಿಕೆ. ಆದರೆ ‘ನೀಲಿ ಮೂಗಿ’ನವರೆಲ್ಲರೂ ಅಸಭ್ಯರೇನಲ್ಲ. ಅಮೆರಿಕದ ಜನ ಹಾಗೂ ಎಷ್ಟೋ ಪಾದ್ರಿ ಗಳು ನನ್ನನ್ನು ಪ್ರೀತ್ಯಾದರಗಳಿಂದ ನೋಡಿಕೊಂಡಿದ್ದಾರೆ. ಸಮಾಜವು ಎತ್ತಿಹಿಡಿದ ವ್ಯಕ್ತಿಯೊಬ್ಬ ನನ್ನು ಹೀಗಳೆಯುವುದರ ಮೂಲಕ ತಾನು ‘ಪ್ರಖ್ಯಾತಿ’ಯನ್ನು ಗಳಿಸಬೇಕೆಂಬುದು ಅವರ ಉದ್ದೇಶವಾಗಿತ್ತು. ಈ ಉಪಾಯ ಇಲ್ಲಿ ಎಲ್ಲರಿಗೂ ಗೊತ್ತಿರುವುದೇ ಆದ್ದರಿಂದ ಯಾರೂ ಅದನ್ನು ಲೆಕ್ಕಿಸುವುದೇ ಇಲ್ಲ. ಆದರೆ ನಮ್ಮ ಭಾರತದ ಮಿಷನರಿಗಳು ಇದನ್ನೇ ಒಂದು ದೊಡ್ಡ ಬಂಡವಾಳವನ್ನಾಗಿ ಮಾಡಿಕೊಳ್ಳಲು ಪ್ರಯತ್ನಿಸಬಹುದು. ಹಾಗೇನಾದರೂ ಪ್ರಯತ್ನಿಸಿದರೆ ಅವರಿಗೆ ಹೇಳಿ–‘ಎಲೈ ಯಹೂದ್ಯನೆ, ಎಚ್ಚರಿಕೆ; ದೇವರು ನಿನ್ನನ್ನು ವಿಚಾರಿಸಿಕೊಳ್ಳುತ್ತಾನೆ!’ ಎಂದು. ಅವರ ಹಳೆಯ ಕಟ್ಟಡವು ಪಾಯದ ಸಮೇತವಾಗಿ ಅಲುಗಾಡುತ್ತಿದೆ. ಅವರೆಷ್ಟೇ ಕಿರುಚಾಡಿದರೂ ಅದು ಉರುಳಲೇಬೇಕು. ಪೌರ್ವಾತ್ಯಧಾರ್ಮಿಕ ಭಾವನೆಗಳು ಇಲ್ಲಿ ಹಬ್ಬಿದುದರ ಪರಿಣಾಮವಾಗಿ ಭಾರತದಲ್ಲಿನ ಅವರ ಸುಖಜೀವನಕ್ಕೆ ಗಂಡಾಂತರ ಒದಗಿದ್ದರೆ, ಅವರ ವಿಷಯದಲ್ಲಿ ನನಗೆ ಕನಿಕರವೆನಿಸುತ್ತದೆ. ಆದರೆ ಇಲ್ಲಿ ಅವರ ಮುಖ್ಯ ಪಾದ್ರಿಗಳಲ್ಲಿ ಒಬ್ಬರೂ ನನಗೆಂದೂ ವಿರೋಧವಾಗಿಲ್ಲ. ಆಗಲಿ, ನೀರಿಗಂತೂ ಇಳಿದಾಗಿದೆ; ಚೆನ್ನಾಗಿಯೇ ಸ್ನಾನ ಮಾಡಿಬಿಡುತ್ತೇನೆ.”

ಆದರೆ ಸ್ವಾಮೀಜಿಯ ವಿರುದ್ಧ ಸಮರ ಹೂಡಿದ್ದ ಶಕ್ತಿಗಳು, ಅವರನ್ನು ಸದೆಬಡಿಯಲು ಸಕಲ ರೀತಿಯಲ್ಲೂ ಶ್ರಮಿಸಿದುವು. ತಮ್ಮ ಗುರುದೇವನೊಬ್ಬನನ್ನೇ ನೆಚ್ಚಿಕೊಂಡು ಸ್ವಾಮೀಜಿ ಶಾಂತ ರಾಗಿದ್ದುಬಿಟ್ಟರು. ಅವರ ಈ ವಿಶ್ವಾಸವನ್ನು ಬಲಪಡಿಸುವಂತಹ ಘಟನೆಯೊಂದು, ಸ್ವಾಮಿ ವಿಜ್ಞಾನಾನಂದರ ಮೂಲಕ ತಿಳಿದುಬರುತ್ತದೆ: ಡೆಟ್ರಾಯ್ಟ್​ನಲ್ಲಿ ಔತಣಕೂಟವೊಂದರಲ್ಲಿ ಪಾಲ್ಗೊಂಡಿದ್ದ ಸ್ವಾಮೀಜಿ, ತಮಗಾಗಿ ತಂದಿಟ್ಟಿದ್ದ ಕಾಫಿಯ ಬಟ್ಟಲನ್ನು ಕೈಗೆತ್ತಿಕೊಂಡರು. ಇನ್ನೇನು ಅದರಿಂದ ಕಾಫಿಯನ್ನು ಹೀರಬೇಕು, ಅಷ್ಟರಲ್ಲಿ ಅವರು ತಮ್ಮ ಪಕ್ಕದಲ್ಲೇ ಶ್ರೀರಾಮ ಕೃಷ್ಣರನ್ನು ಕಂಡರು. “ಕುಡಿಯಬೇಡ ಅದನ್ನು, ಅದು ವಿಷ! ಅದು ವಿಷ!” ಎಂದು ಶ್ರೀರಾಮ ಕೃಷ್ಣರು ಹೇಳಿದಂತಾಯಿತು. ತಕ್ಷಣ ಸ್ವಾಮೀಜಿ ಬಟ್ಟಲನ್ನು ಕೆಳಗಿಟ್ಟುಬಿಟ್ಟರು. ಎಲ್ಲೆಲ್ಲೂ ದ್ವೇಷಾಸೂಯೆಯ ವಾತಾವರಣವಿದ್ದ ಈ ಸಂದರ್ಭದಲ್ಲಿ, ಈ ಎಚ್ಚರಿಕೆಯನ್ನು ಅವರು ತಳ್ಳಿ ಹಾಕಲು ಸಾಧ್ಯವಿರಲಿಲ್ಲ. ಅವರ ಹೆಸರಿಗೆ ಮಸಿ ಬಳಿಯಲು ಹೊರಟ ನೀಚಬುದ್ಧಿಯ ಜನ, ಅವರ ಜೀವಕ್ಕೆ ಹಾನಿಯುಂಟುಮಾಡಿದ್ದರೂ ಆಶ್ಚರ್ಯವಿರಲಿಲ್ಲ.

ಹೀಗೆಯೇ ಕ್ರಮೇಣ ಸ್ವಾಮೀಜಿಯನ್ನು ಅವರ ಕಾರ್ಯೋದ್ದೇಶದ ಸಹಿತವಾಗಿ ನಾಶಪಡಿಸುವ ಪ್ರಯತ್ನ ಎಷ್ಟು ಬೃಹತ್ ಪ್ರಮಾಣದಲ್ಲಿ ಬೆಳೆದುಕೊಂಡು ಬಂದಿತೆಂದರೆ, ಆಗ ಅವರು ಅದನ್ನು ಪ್ರತಿಭಟಿಸಿ ತಡೆಗಟ್ಟದೆ ಹೋಗಿದ್ದರೆ, ಅವರ ಅದುವರೆಗಿನ ಶ್ರಮವೆಲ್ಲ ನೀರಿನಲ್ಲಿ ಮಾಡಿದ ಹೋಮವಾಗುತ್ತಿತ್ತು. ಇದನ್ನು ಮನಗಂಡ ಸ್ವಾಮೀಜಿ ಕಟ್ಟಕಡೆಗೆ ಈ ಬಗ್ಗೆ ಕ್ರಮ ಕೈಗೊಳ್ಳಲು ನಿಶ್ಚಯಿಸಿದರು. ಆದರೆ ಆಗಲೂ ಅವರು ತಮ್ಮ ಚಾರಿತ್ರ್ಯದ ಮೇಲಿನ ಹಾಗೂ ತಮ್ಮ ಹಿನ್ನೆಲೆಯ ಮೇಲಿನ ಆರೋಪಗಳನ್ನು ಸಾರ್ವಜನಿಕವಾಗಿ ಅಲ್ಲಗಳೆಯಲು ಹೋಗಲಿಲ್ಲ. ಅದರ ಬದಲಾಗಿ, ಅವರು ಯಾರ ಪ್ರತಿನಿಧಿಯಾಗಿ ಅಮೆರಿಕೆಗೆ ಹೋಗಿದ್ದರೋ, ಯಾರಿಗೋಸ್ಕರ ಹೋರಾಡು ತ್ತಿದ್ದರೋ, ಅದೇ ಭಾರತೀಯರೇ ಈಗ ಎದ್ದು ನಿಂತು, ‘ವಿವೇಕಾನಂದರು ನಮ್ಮ ಪ್ರತಿನಿಧಿಯೇ ಹೌದು; ಅವರ ಸಂದೇಶವು ಹಿಂದೂ ತತ್ವಶಾಸ್ತ್ರದ ತಿರುಳಲ್ಲದೆ ಬೇರಲ್ಲ’ ಎಂದು ಬಹಿರಂಗವಾಗಿ ಘೋಷಿಸಬೇಕೆಂದು ನಿರೀಕ್ಷಿಸಿದರು. ಮತ್ತು ಹಾಗೆ ಮಾಡುವಂತೆ ತಮ್ಮ ಶಿಷ್ಯರಿಗೆ ಪತ್ರ ಬರೆದು ಕೇಳಿಕೊಂಡರು.

ನಿಜಕ್ಕೂ ಇದು ಅತ್ಯಂತ ದೌರ್ಭಾಗ್ಯದ ಸಂಗತಿಯೇ ಸರಿ. ಸ್ವಾಮಿ ವಿವೇಕಾನಂದರಂತಹ ಯುಗಪ್ರವರ್ತಕರು, ತಾವು ಕಪಟಿಯಲ್ಲ ಎಂಬುದನ್ನು ಸಾಬೀತು ಮಾಡಬೇಕಾದಂತಹ ಪರಿಸ್ಥಿತಿ ಯೊದಗಿದುದು ಎಂತಹ ವಿಡಂಬನೆ! ತಮ್ಮನ್ನು ತಾವು ಆಕ್ರಮಣದಿಂದ ರಕ್ಷಿಸಿಕೊಳ್ಳುವುದು ಅತ್ಯಂತ ಕೀಳುಮಟ್ಟದ ವರ್ತನೆಯೆಂದು ಸ್ವಾಮೀಜಿ ನಂಬಿದ್ದರು. ಅದು ಸಂನ್ಯಾಸಧರ್ಮಕ್ಕೆ ವಿರುದ್ಧವೆಂಬ ಒಂದೇ ಕಾರಣಕ್ಕಲ್ಲ. ಅದು ಅವರಿಗೆ ಸ್ವಭಾವಸಹಜವಾದದ್ದರಿಂದ. ತರುಣ ನರೇಂದ್ರನನ್ನು ಒಮ್ಮೆ ಶ್ರೀರಾಮಕೃಷ್ಣರು ಪ್ರಶ್ನಿಸಿದ್ದರು, “ಏನಪ್ಪಾ, ಯಾರಾದರೂ ನಿನ್ನನ್ನು ನಿಂದಿಸಿದರೆ ನೀನೇನು ಭಾವಿಸುತ್ತೀ?” ಎಂದು. “ಆನೆ ಹೋಗುತ್ತಿದ್ದರೆ ನಾಯಿ ಬೊಗಳುತ್ತದೆ ಎಂದುಕೊಳ್ಳುತ್ತೇನೆ”–ಬಿಸಿರಕ್ತದ ಸ್ವಾಭಿಮಾನಿ ತರುಣ ಸಹಜವಾಗಿ ಉತ್ತರಿಸಿದ್ದ. ಆಗ ಶ್ರೀ ರಾಮಕೃಷ್ಣರು, “ಓ! ಅಷ್ಟು ದೂರ ಹೋಗಬೇಡಪ್ಪಾ!” ಎಂದು ನಗುತ್ತ ಹೇಳಿದ್ದರು. ನಿಜಕ್ಕೂ ಕೆಲದಿನಗಳಲ್ಲೇ ನರೇಂದ್ರನ ಸ್ವಭಾವದಲ್ಲಿ ಮಹತ್ತರ ಬದಲಾವಣೆಯುಂಟಾಯಿತು. ಅವನು ಪ್ರತಿಯೊಬ್ಬರನ್ನೂ ಅತ್ಯುನ್ನತ ದೃಷ್ಟಿಕೋನದಿಂದ ಸಮಾನವಾಗಿ ನೋಡುವ ಸಮದರ್ಶಿಯಾದ. ಇಂತಹ ಅಲೌಕಿಕ ವ್ಯಕ್ತಿತ್ವದ ಸ್ವಾಮೀಜಿ, ಸಾಮಾನ್ಯರಂತೆ ತಮ್ಮನ್ನು ಸಮರ್ಥಿಸಿಕೊಳ್ಳಲು ಇಚ್ಛಿಸದಿದ್ದರೆ ಅದು ಸಹಜವೇ ಆಗಿತ್ತು. ಅವರನ್ನು ವಿವಾದಕ್ಕೆಳೆಯಲು ಹಲವಾರು ಜನ ಪ್ರಯತ್ನಿಸಿದರು. ಪತ್ರಿಕೆಗಳಿಗೆ ಲೇಖನಗಳನ್ನು, ಪತ್ರಗಳನ್ನು ಬರೆದು ಬಹಿರಂಗವಾಗಿ ಅವರ ಮೇಲೆ ಆಪಾದನೆಗಳನ್ನು ಮಾಡಿದರು. ಅವುಗಳಿಗೆ ಉತ್ತರಿಸುವಂತೆಯೂ ಕೇಳಿಕೊಂಡರು. ಆದರೆ ಸ್ವಾಮೀಜಿ ಇಂತಹ ಕುತಂತ್ರಗಳಿಗೆ ಜಗ್ಗಲು ನಿರಾಕರಿಸಿದರು. ಏನೇ ಆದರೂ ಅವುಗಳಿಗೆ ಉತ್ತರಿಸಲು ಹೋಗಲಿಲ್ಲ. ಆದರೆ ಸ್ವಾರಸ್ಯದ ಸಂಗತಿಯೇನೆಂದರೆ, ಅವರ ವಿರುದ್ಧ ಮಾಡಲಾದ ಆಪಾದನೆಗೆ ಇತರ ಓದುಗರೇ ಸಮರ್ಥವಾಗಿ ಉತ್ತರಿಸಿದರು; ಸ್ವಾಮೀಜಿಯ ವಿರುದ್ಧ ನಡೆಯು ತ್ತಿದ್ದ ಅಪಪ್ರಚಾರವೆಲ್ಲ ಎಂತಹ ದುರುದ್ದೇಶದಿಂದ ಕೂಡಿದುದು, ಇದರ ಹಿಂದಿರುವ ವ್ಯಕ್ತಿಗಳ ದುಷ್ಟತನವೆಂಥದು ಎಂಬುದನ್ನು ಬಹಿರಂಗಪಡಿಸಿದರು. ಆದರೆ ಪರಿಸ್ಥಿತಿ ಕೈಮೀರಿ ಹೋಗು ವಷ್ಟು ಅಪಪ್ರಚಾರ ಪ್ರಾರಂಭವಾದಾಗ, ಏಪ್ರಿಲ್ ೯ರಂದು ಅಳಸಿಂಗ ಪೆರುಮಾಳ್​ಗೆ ಬರೆದ ಪತ್ರವೊಂದರಲ್ಲಿ ಸ್ವಾಮೀಜಿ ತಮ್ಮ ಪರಿಸ್ಥಿತಿಯನ್ನು ವಿವರಿಸಿ, ಬಳಿಕ ತಮ್ಮ ಶಿಷ್ಯರು ಆ ಬಗ್ಗೆ ಏನು ಮಾಡಬೇಕೆಂಬುದನ್ನು ತಿಳಿಸಿದರು:

“ಸಾಧ್ಯವಿದ್ದರೆ ನೀವೊಂದು ಕೆಲಸ ಮಾಡಬೇಕು. ರಾಮನಾಡಿನ ರಾಜನನ್ನೋ ಅಥವಾ ಇನ್ನಾ ರಾದರೋ ಭಾರೀ ವ್ಯಕ್ತಿಯನ್ನೋ ಅಧ್ಯಕ್ಷಸ್ಥಾನಕ್ಕೆ ಹಿಡಿದು, ಮದರಾಸಿನಲ್ಲಿ ಒಂದು ದೊಡ್ಡ ಸಭೆಯನ್ನು ಏರ್ಪಡಿಸಲು ನಿಮ್ಮಿಂದ ಸಾಧ್ಯವೇನು? ನಾನು ಇಲ್ಲಿಗೆ ಬಂದಿರುವುದಕ್ಕೆ ನಿಮ್ಮ ಸಂಪೂರ್ಣ ಅನುಮೋದನೆಯನ್ನು ವ್ಯಕ್ತಪಡಿಸುವಂತಹ ಒಂದು ಗೊತ್ತುವಳಿಯನ್ನು ಹೊರಡಿಸಿ ಅದನ್ನು ಇಲ್ಲಿನ ‘ಶಿಕಾಗೋ ಹೆರಾಲ್ಡ್​’, ‘ಇಂಟರ್ ಓಷಿಯನ್​’, ‘ನ್ಯೂಯಾರ್ಕ್ ಸನ್​’ ಮತ್ತು ಡೆಟ್ರಾಯ್ಟ್​ನ ‘ಕಮರ್ಷಿಯಲ್ ಅಡ್ವರ್​ಟೈಸರ್​’ ಪತ್ರಿಕೆಗಳಿಗೆ ಕಳಿಸಲು ನಿಮ್ಮಿಂದಾದೀತೆ?... ನಿಮ್ಮ ಗೊತ್ತುವಳಿಯ ಪ್ರತಿಗಳನ್ನು ಸಮ್ಮೇಳನಾಧ್ಯಕ್ಷರಾದ ಡಾ॥ ಬರೋಸ್​ರಿಗೂ ಡೆಟ್ರಾಯ್ಟ್​ನ ಶ್ರೀಮತಿ ಜೆ. ಜೆ. ಬ್ಯಾಗ್​ಲೀಯವರಿಗೆ ಕಳಿಸಿಕೊಡಿ.

“ಈ ಸಭೆ ಸಾಧ್ಯವಾದಷ್ಟು ದೊಡ್ಡದಾಗಿರಲಿ. ದೊಡ್ಡದೊಡ್ಡ ಕುಳಗಳನ್ನೆಲ್ಲ ಹಿಡಿಯಿರಿ. ಅವರು ತಮ್ಮ ಧರ್ಮ ಹಾಗೂ ರಾಷ್ಟ್ರಕ್ಕಾಗಿ ಅದರಲ್ಲಿ ಸೇರಲೇಬೇಕು. ಮೈಸೂರಿನ ಮಹಾರಾಜರಿಂದ ಹಾಗೂ ದಿವಾನರಿಂದ ಹಾಗೆಯೇ ಖೇತ್ರಿಯಿಂದ, ಈ ಸಭೆಯ ಬಗೆಗೂ ಅದರ ನಿರ್ಣಯದ ಬಗೆಗೂ ಅನುಮೋದನಾಪತ್ರಗಳ್ನು ಪಡೆಯಲು ಪ್ರಯತ್ನಿಸಿ... ನಿಮ್ಮಿಂದ ಸಾಧ್ಯವಾಗುವು ದಾದರೆ ಇದನ್ನು ಪ್ರಯತ್ನಿಸಿ. ಇದು ಅಂಥಾ ದೊಡ್ಡ ಕೆಲಸವೇನೂ ಅಲ್ಲ. ಎಲ್ಲ ಕಡೆಗಳಿಂದಲೂ ಸಹಾನುಭೂತಿ ಪತ್ರಗಳನ್ನು ಪಡೆದುಕೊಂಡು, ಅವುಗಳನ್ನು ಮುದ್ರಿಸಿ, ಅವುಗಳ ಪ್ರತಿಗಳನ್ನು ಅಮೆರಿಕದ ಪತ್ರಿಕೆಗಳಿಗೆ ಕಳಿಸಿಕೊಡಿ–ಸಾಧ್ಯವಾದಷ್ಟು ಬೇಗ! ಸೋದರರೆ, ಅದರಿಂದ ಬಹಳಷ್ಟು ಸಹಾಯವಾಗುತ್ತದೆ. ಈ ಬ್ರಾಹ್ಮಸಮಾಜದ ಮಂದಿ ಇಲ್ಲಿ ಎಲ್ಲ ಬಗೆಯ ಅಸಂಬದ್ಧ ಪ್ರಲಾಪ ಗಳನ್ನು ಪ್ರಾರಂಭಿಸಿದ್ದಾರೆ. ಸಾಧ್ಯವಾದಷ್ಟು ಬೇಗ ನಾವು ಅವರ ಬಾಯಿ ಮುಚ್ಚಿಸಬೇಕು. ಏಳಿ, ಕಾರ್ಯಸನ್ನದ್ಧರಾಗಿ. ನಿಮಗೆ ಇಷ್ಟು ಮಾಡಲು ಸಾಧ್ಯವಾದರೆ, ಮುಂದೆ ನಾವು ಬಹಳಷ್ಟನ್ನು ಸಾಧಿಸಬಹುದೆಂಬುದು ನಿಶ್ಚಿತ. ಸನಾತನ ಹಿಂದೂಧರ್ಮವೊಂದೇ ಶಾಶ್ವತ! ಏಳಿ, ನನ್ನ ಹುಡುಗರೆ, ನಾವು ಗೆಲ್ಲುವುದು ಖಂಡಿತ!”

ಇದೇ ಪತ್ರದಲ್ಲಿ ಸ್ವಾಮೀಜಿ ಕಲ್ಕತ್ತದಲ್ಲೂ ಅದೇ ತರಹದ ಸಭೆಯನ್ನು ಸೇರಿಸುವಂತೆ ಸೂಚಿಸಿದರು. ನಿಜಕ್ಕೂ ಇದು ಅವರು ಬರೆದಿದ್ದ ತ್ವರಿತದ ಪತ್ರ. ಅಂತೆಯೇ ಇದಕ್ಕವರು ಒಂದು ತ್ವರಿತದ ಉತ್ತರವನ್ನು ನಿರೀಕ್ಷಿಸಿದರು. ಆದರೆ ಜುಲೈ ಎರಡನೇ ವಾರ ಕಳೆದರೂ ಮದ್ರಾಸಿನ ಕಡೆಯಿಂದ ಪ್ರತ್ಯುತ್ತರವೇ ಇಲ್ಲ! ಆದರೆ ಅಳಸಿಂಗ ಪೆರುಮಾಳ್ ಉತ್ತರ ಬರೆದಿರಲಿಲ್ಲ ವೆಂದಲ್ಲ; ಅವರು ಸ್ವಾಮೀಜಿಯ ಯಾವುದೋ ಒಂದು ತಾತ್ಕಾಲಿಕ ವಿಳಾಸಕ್ಕೆ ಬರೆದಿದ್ದರು. ಅದು ಸ್ವಾಮೀಜಿಯ ಕೈಗೆ ಸೇರಿದ್ದು ಮಾತ್ರ ಬಹಳ ನಿಧಾನ. ಈ ಮಧ್ಯೆ ಸ್ವಾಮೀಜಿ ಮದ್ರಾಸಿನ ಇನ್ನೊಬ್ಬ ಶಿಷ್ಯನಿಗೆ ಬರೆದರು:

“ನಾನು ಅಳಸಿಂಗನಿಗೆ ಸಭೆ ಸೇರಿಸುವ ವಿಷಯವಾಗಿ ಎರಡು ತಿಂಗಳಿಗೂ ಹಿಂದೆಯೇ ಒಂದು ಪತ್ರ ಬರೆದಿದ್ದೆ. ಅವನು ಈ ಪತ್ರಕ್ಕೊಂದು ಉತ್ತರ ಕೂಡ ಬರೆಯಲಿಲ್ಲ. ಬಹುಶಃ ಅವನ ಉತ್ಸಾಹವೆಲ್ಲ ತಣ್ಣಗಾಗಿಬಿಟ್ಟಿದೆಯೋ ಏನೋ.”

ಅಂತೂ ಈಗ ಸ್ವಾಮೀಜಿ ಏನು ತಿಳಿದುಕೊಳ್ಳುವಂತಾಯಿತೆಂದರೆ ಈ ಮದ್ರಾಸೀ ಶಿಷ್ಯರೆಲ್ಲ ತಮ್ಮ ಸೂಚನೆಯನ್ನು ಕಡೆಗಣಿಸಿಬಿಟ್ಟರು, ಆದ್ದರಿಂದ ಯಾರೂ ಎಲ್ಲೂ ಸಭೆಯನ್ನು ಸೇರಿಸುವ ಪ್ರಯತ್ನ ಮಾಡಲಿಲ್ಲ, ಎಂದು. ಅಮೆರಿಕಕ್ಕಂತೂ ಈ ಕುರಿತಾಗಿ ಯಾವ ವರ್ತಮಾನವೂ ತಲುಪ ಲಿಲ್ಲ. ಆದ್ದರಿಂದ ಮದ್ರಾಸಿಗಳ ಈ ದೀರ್ಘ ಮೌನದಿಂದಾಗಿ ಅಮೆರಿಕದಲ್ಲಿನ ಸ್ವಾಮೀಜಿಯ ನಿಂದಕವರ್ಗ ತನ್ನ ಹೀನಕಾರ್ಯವನ್ನು ನಿರಾತಂಕವಾಗಿ ನಡೆಸಿಕೊಂಡು ಬರುವಂತಾಯಿತು.

ಸ್ವಾಮೀಜಿಯ ಸುತ್ತಮುತ್ತ ಇಷ್ಟೆಲ್ಲ ನಡೆಯುತ್ತಿದ್ದರೂ ಅವರು ಸಾರ್ವಜನಿಕರ ಮುಂದೆ ತಮ್ಮನ್ನು ಸಮರ್ಥಿಸಿಕೊಳ್ಳುವ ಪ್ರಯತ್ನವನ್ನಿನ್ನೂ ಪ್ರಾರಂಭಿಸಿರಲಿಲ್ಲ–ಭಗವಂತ ನೋಡಿ ಕೊಳ್ಳುತ್ತಾನೆ ಎಂದು ಬಿಟ್ಟುಬಿಟ್ಟಿದ್ದರು. ಆದರೆ ಭಗವಂತ ಸ್ವಾಮೀಜಿಗೂ ಬುದ್ಧಿಯನ್ನು ಕೊಟ್ಟಿರುವನಲ್ಲವೆ? ಅಲ್ಲದೆ ಅವರೊಬ್ಬ ಜವಾಬ್ದಾರಿಯುತ ವ್ಯಕ್ತಿಯಲ್ಲವೆ? ಜೊತೆಗೆ ಅಮೆರಿಕದ ಹಲವಾರು ಸಭ್ಯ-ಸಜ್ಜನರು ಅವರಿಗೆ ತಮ್ಮತಮ್ಮ ಮನೆಗಳಲ್ಲಿ ಹಲವಾರು ತಿಂಗಳು ಆಹಾರ-ಆಸರೆ ಕೊಟ್ಟು ಸತ್ಕರಿಸಿರಲಿಲ್ಲವೆ? ಈಗ ಇಂತಹ ವ್ಯಕ್ತಿಯ ಮೇಲೆ ನಾಲ್ಕು ದಿಸೆ ಗಳಿಂದಲೂ ಅಪವಾದ-ನಿಂದೆಗಳ ಸುರಿಮಳೆಯಾಗುತ್ತಿರುವುದನ್ನು ಕಂಡಾಗ ಆ ಮನೆಯವರಿಗೆಲ್ಲ ಕೊನೆಗೂ ಅವರ ಮೇಲೆ ಸಂಶಯ ಹುಟ್ಟದಿರಲಾರದೆ? ಇದನ್ನು ಹೋಗಲಾಡಿಸಬೇಕಾದ್ದು ಸ್ವಾಮೀಜಿಯ ಕರ್ತವ್ಯವಲ್ಲವೆ? ಆದ್ದರಿಂದ ಈಗ ಸ್ವಾಮೀಜಿ ಯೋಚಿಸಿದರು–ಅಮೆರಿಕದ ತಮ್ಮ ಆತ್ಮೀಯ ಸ್ನೇಹಿತ ವರ್ಗಕ್ಕೆ ಸತ್ಯಸಂಗತಿಯನ್ನು ತಿಳಿಸಿ ಅವರ ಅನುಮಾನಗಳನ್ನು ನಿವಾರಿಸಬೇಕು ಎಂದು. ಅಲ್ಲದೆ ಈಗ ಎಂತಹ ಪರಿಸ್ಥಿತಿ ಬಂದುಬಿಟ್ಟಿತ್ತೆಂದರೆ ಸ್ವಾಮೀಜಿ ಏನಾದರೂ ಸಹನೆಯ ಹೆಸರಿನಲ್ಲಿ ಬಾಯ್ಮುಚ್ಚಿಕೊಂಡು ಮೌನವಾಗಿದ್ದ ಪಕ್ಷದಲ್ಲಿ ಅವರು ಸುಳ್ಳುಗಾರ, ಠಕ್ಕ, ಕಪಟಸಂನ್ಯಾಸಿ ಎನ್ನುವುದನ್ನೆಲ್ಲ ಒಪ್ಪಿಕೊಂಡಂತಾಗಿಬಿಡುತ್ತಿತ್ತು. ಆಗ ಅವರ ಆಪ್ತ ಸ್ನೇಹಿತರಿಗೂ ಇದು ದೃಢವಾಗಿಬಿಡುತ್ತಿತ್ತು. ಆಮೇಲೆ ಸ್ವಾಮೀಜಿಗೆ ಅಮೆರಿಕದಲ್ಲಿ ಉಳಿಯುವುದೇ ಕಷ್ಟವಾಗಿಬಿಡುತ್ತಿತ್ತು. ಅವರು ಅಮೆರಿಕದಲ್ಲಿ ತಮ್ಮ ಕಾರ್ಯ ಮುಂದುವರಿಸು ವುದಕ್ಕಾಗಿ ತಮ್ಮನ್ನು ಸಮರ್ಥನೆ ಮಾಡಿಕೊಳ್ಳದಿದ್ದರೆ ಚಿಂತೆಯಿಲ್ಲ; ಆದರೆ ಅವರು ಅಲ್ಲಿನ ಹಲವು ಸಭ್ಯ-ಸಜ್ಜನರೊಂದಿಗೆ ಬೆಳೆಸಿದ ಆತ್ಮೀಯ ಸ್ನೇಹವನ್ನು ಉಳಿಸಿಕೊಳ್ಳುವುದಕ್ಕಾದರೂ ತಮ್ಮನ್ನು ಸಮರ್ಥನೆ ಮಾಡಿಕೊಳ್ಳಲೇಬೇಕಾಗಿತ್ತು. ಸ್ವಾಮೀಜಿ ಈಗ ಮೊಟ್ಟಮೊದಲನೆಯದಾಗಿ ಗಮನ ಕೊಡಬೇಕಾಗಿದ್ದುದು ಪ್ರೊ ॥ ಜಾನ್ ಹೆನ್ರಿ ರೈಟರ ಕಡೆಗೆ, ಏಕೆಂದರೆ ಯಾವ ಪರಿಚಯ ಪತ್ರವೂ ಇಲ್ಲದೆ ಬರಿಗೈಯಲ್ಲಿ ಬಂದಿದ್ದ ಸ್ವಾಮೀಜಿಯನ್ನು ಸಮ್ಮೇಳನದಲ್ಲಿ ಭಾಗವಹಿಸಲು ಪ್ರತಿನಿಧಿಯಾಗಲು ಸರ್ವವಿಧದಿಂದಲೂ ತಕ್ಕವರು ಎಂದು ಘೋಷಿಸಿ, ಅವರ ಅರ್ಹತೆಗೆ ತಾವೇ ಸಾಕ್ಷಿಯಾಗಿ ನಿಂತು ಸಮ್ಮೇಳನದ ವೇದಿಕೆಯ ಮೇಲೆ ಕಾಣಿಸಿಕೊಳ್ಳಲು ಅನುವು ಮಾಡಿಕೊಟ್ಟ ವರು ಈ ಪ್ರೊಫೆಸರರೇ ಅಲ್ಲವೆ? ಆದ್ದರಿಂದ ಈಗ ಸ್ವಾಮೀಜಿ, ತಮ್ಮ ವ್ಯಕ್ತಿತ್ವವನ್ನು ಸ್ಪಷ್ಟವಾಗಿ ನಿರೂಪಿಸುವಂತಹ ಯೋಗ್ಯತಾಪತ್ರಗಳು ಭಾರತದ ಕಡೆಯಿಂದ ಏನೇನು ಬಂದಿದ್ದವೋ ಅವು ಗಳನ್ನೆಲ್ಲ ಕಲೆಹಾಕಿ ಪ್ರೊ ॥ ರೈಟರಿಗೆ ಕಳಿಸಿದರು. ಆ ಯೋಗ್ಯತಾಪತ್ರಗಳಲ್ಲಿ ಕೆಲವೆಲ್ಲ ಭಾರತ ದಿಂದ ಮೊನ್ನೆಮೊನ್ನೆ ಅವರ ಕೈಸೇರಿದವುಗಳು. ಇವುಗಳನ್ನು ಕಳಿಸಿಕೊಟ್ಟು, ಅನಂತರ ಸ್ವಾಮೀಜಿ ಪ್ರೊಫೆಸರರಿಗೆ ಒಂದು ಪತ್ರ ಬರೆದರು:

“ಈ ವೇಳೆಗೆ ನಿಮಗೆ ಕರಪತ್ರ ಹಾಗೂ ಪತ್ರಗಳು ತಲುಪಿರಬಹುದು. ನೀವು ಇಚ್ಛೆಪಟ್ಟರೆ, ಭಾರತದಿಂದ ಬಂದಿರುವ ಕೆಲವು ರಾಜರ ಹಾಗೂ ಮಂತ್ರಿಗಳ ಪತ್ರಗಳನ್ನೂ ಕಳಿಸಿಕೊಡುತ್ತೇನೆ. ಇವರಲ್ಲಿ ಒಬ್ಬರು ಮಂತ್ರಿಗಳು (ಹರಿದಾಸ್ ವಿಹಾರಿದಾಸ್ ದೇಸಾಯಿ) ಭಾರತದ ರಾಯಲ್ ಕಮಿಷನಿನ ಕೆಳಗಿದ್ದ ಓಪಿಯಮ್ ನಿಯೋಗದ ಸದಸ್ಯರಾಗಿದ್ದವರು. ನೀವು ಇಷ್ಟಪಟ್ಟರೆ ನಾನು ಅವರಿಂದ ನಿಮಗೆ ಪತ್ರ ಬರೆಯಿಸಿ ನಾನು ಮೋಸಗಾರನಲ್ಲ ಎಂಬುದು ನಿಮಗೆ ಮನವರಿಕೆ ಯಾಗುವಂತೆ ಮಾಡುತ್ತೇನೆ. ಆದರೆ ಸೋದರ, ಸಂನ್ಯಾಸಿಗಳಾದ ನಮ್ಮ ಜೀವನಾದರ್ಶವೇ ನಮ್ಮನ್ನು ನಾವು ಬಹಿರಂಗಪಡಿಸಿಕೊಳ್ಳದಿರುವುದು, ನಮ್ಮನ್ನು ನಾವು ನಿಗ್ರಹಿಸಿಕೊಳ್ಳುವುದು, ನಮ್ಮತನವನ್ನು ಇಲ್ಲವಾಗಿಸುವುದು.

“ನಾವು ಮಾಡಬೇಕಾದದ್ದು ತ್ಯಾಗ, ಸ್ವೀಕಾರವಲ್ಲ. ನನ್ನ ತಲೆಯಲ್ಲಿ ಒಂದು ‘ಹುಚ್ಚು’ ಹೊಕ್ಕಿರದಿದ್ದರೆ ನಾನಿಲ್ಲಿಗೆ ಬರುತ್ತಲೇ ಇರಲಿಲ್ಲ. ನನ್ನ ಕಾರ್ಯೋದ್ದೇಶಕ್ಕೆ ಸಹಾಯವಾಗಬಹು ದೆಂದು ನಾನು ವಿಶ್ವಧರ್ಮ ಸಮ್ಮೇಳನವನ್ನು ಸೇರಿಕೊಂಡೆ. ನಮ್ಮ ಜನ ನನಗೆ ಸಮ್ಮೇಳನಕ್ಕೆ ಹೋಗುವಂತೆ ಹೇಳಿದಾಗ ನಾನು ಯಾವಾಗಲೂ ಅದನ್ನು ನಿರಾಕರಿಸುತ್ತಲೇ ಬಂದೆ. ನಾನು ಅವರಿಗೆ ಹೇಳಿದ್ದೆ, ‘ನಾನು ಆ ಸಭೆಯಲ್ಲಿ ಸೇರಿಕೊಂಡರೆ ಸೇರಿಕೊಂಡೆ, ಇಲ್ಲದಿದ್ದರೆ ಇಲ್ಲ; ನಿಮಗಿದು ಒಪ್ಪಿಗೆಯಾಗುವುದಾದರೆ ನೀವು ನನ್ನನ್ನು ಅಲ್ಲಿಗೆ ಕಳಿಸಬಹುದು’ ಎಂದು. ಆಗ ಅವರು ‘ನಿಮಗೆ ತೋಚಿದಂತೆ ಮಾಡಿ’ ಎಂದು ನನ್ನನ್ನು ಕಳಿಸಿಕೊಟ್ಟರು. ಇನ್ನುಳಿದುದನ್ನೆಲ್ಲ ನೀವು ನೋಡಿಕೊಂಡಿರಿ.

“ಓ ನನ್ನ ವಿಶ್ವಾಸೀ ಗೆಳೆಯ, ನನ್ನ ವಿಷಯದಲ್ಲಿ ನಿಮಗೆ, ಸಾಧ್ಯವಾದ ಎಲ್ಲ ಸಮಾಧಾನವನ್ನು ಉಂಟುಮಾಡುವುದು ನನ್ನ ಜವಾಬ್ದಾರಿ. ಆದರೆ ಇನ್ನುಳಿದ ಪ್ರಪಂಚದ ಜನ ಏನು ಹೇಳು ತ್ತಾರೋ ನಾನದನ್ನು ಲೆಕ್ಕಿಸುವುದಿಲ್ಲ. ಆದ್ದರಿಂದ ಆ ಕರಪತ್ರದಲ್ಲಿ ಹಾಗೂ ಪತ್ರಗಳಲ್ಲಿ ಬರೆದಿರುವುದನ್ನು ಪ್ರಕಟಿಸಬಾರದು ಮತ್ತು ಯಾರಿಗೂ ತೋರಿಸಬಾರದು ಎಂದು ನಿಮ್ಮನ್ನು ಕೇಳಿಕೊಳ್ಳುತ್ತೇನೆ. ಮಿಷನರಿಗಳು ಮಾಡುತ್ತಿರುವ ಪ್ರಯತ್ನಗಳನ್ನು ನಾನು ಲಕ್ಷಿಸುವುದಿಲ್ಲ. ಆದರೆ ಮಜುಮ್​ದಾರನನ್ನು ಹಿಡಿದುಕೊಂಡಿರುವ ಹೊಟ್ಟೆಕಿಚ್ಚಿನ ಜ್ವರ ನನಗೆ ದೊಡ್ಡ ಆಘಾತ ವನ್ನುಂಟುಮಾಡಿದೆ. ಅವನಿಗೆ ಒಳ್ಳೆಯ ಬುದ್ಧಿ ಬರಲಿ ಎಂದು ಪ್ರಾರ್ಥಿಸುತ್ತೇನೆ. ಏಕೆಂದರೆ ಅವನೊಬ್ಬ ದೊಡ್ಡಮನುಷ್ಯ ಮತ್ತು ಒಳ್ಳೆಯ ವ್ಯಕ್ತಿ, ಮತ್ತು ತನ್ನ ಜೀವನದಲ್ಲೆಲ್ಲ ಒಳ್ಳೆಯದನ್ನೇ ಮಾಡಲು ಪ್ರಯತ್ನಿಸಿದ್ದಾನೆ. ಆದರೆ ಇದರಿಂದ ನನ್ನ ಗುರುದೇವನ ಒಂದು ಮಾತು ಇನ್ನಷ್ಟು ಖಚಿತವಾಗುತ್ತದೆ: ‘ಮಸಿತುಂಬಿರುವ ಕೋಣೆಯಲ್ಲಿ ಎಷ್ಟೇ ಜೋಪಾನವಾಗಿದ್ದರೂ ಬಟ್ಟೆಗೆ ಸ್ವಲ್ಪವಾದರೂ ಮಸಿ ಹತ್ತಿಯೇ ಹತ್ತುತ್ತದೆ.’

“ನಾನೆಂದಿಗೂ ಮಿಷನರಿಯಾಗಿರಲಿಲ್ಲ. ಮುಂದೆ ಆಗಲು ಸಾಧ್ಯವೂ ಇಲ್ಲ. ನನ್ನ ಸ್ಥಳವಿರು ವುದು ಹಿಮಾಲಯದಲ್ಲಿ. ಇಷ್ಟರಮಟ್ಟಿಗೆ ನನ್ನ ವಿಷಯದಲ್ಲಿ ನಾನು ತೃಪ್ತನಾಗಿದ್ದೇನೆ. ಅದನ್ನೇ ನನ್ನ ಸಂಪೂರ್ಣ ಅಂತಸ್ಸಾಕ್ಷಿಯಾಗಿ ಹೇಳಬಲ್ಲೆ, ‘ಹೇ ಭಗವಂತ, ನಾನು ನನ್ನ ಮಾನವ ಸೋದರರಲ್ಲಿ ಭಯಂಕರ ಸಂಕಟವನ್ನು ಕಂಡೆ; ಅದರಿಂದ ಅವರನ್ನು ಹೊರತರುವ ದಾರಿ ಯನ್ನೂ ಕಂಡುಕೊಂಡೆ; ಪರಿಹಾರವನ್ನು ಕಾರ್ಯರೂಪಕ್ಕೆ ತರಲು ನನ್ನ ಕೈಲಾದಷ್ಟು ಪ್ರಯತ್ನಿ ಸಿದೆ. ಆದರೆ ಸೋತುಹೋದೆ. ಅಂತೂ ನಿನ್ನ ಇಚ್ಛೆಯಿದ್ದಂತಾಗಲಿ’ ಎಂದು.

“ಅವನ ಆಶೀರ್ವಾದ ನಿಮ್ಮ ಮತ್ತು ನಮ್ಮೆಲ್ಲರ ಮೇಲೆ ಎಂದೆಂದಿಗೂ ಇರಲಿ.”

ಸ್ವಾಮೀಜಿ ಪ್ರೊ ॥ ರೈಟರಿಗೆ ಕಳಿಸಿಕೊಟ್ಟಿದ್ದ ಪತ್ರಗಳು ಜುನಾಗಢದ ದಿವಾನರಿಂದ ಬಂದದ್ದು. ಇವುಗಳನ್ನು ದಿವಾನ್​ಜಿಯವರು, ಇಸಾಬೆಲ್ ಮೆಕ್​ಕಿಂಡ್ಲಿಯ (ಹೇಲ್ ಸೋದರಿಯರಲ್ಲಿ ಒಬ್ಬಳು) ವಿಳಾಸಕ್ಕೆ ಕಳಿಸಿಕೊಟ್ಟಿದ್ದರು. ಆಕೆಯ ಮೂಲಕ ಈ ಪತ್ರಗಳನ್ನು ಪಡೆದ ಮರುದಿನ ಸ್ವಾಮೀಜಿ ಅವಳಿಗೆ ಬರೆದರು:

“ನಿನ್ನೆ ನೀನು ಕಳಿಸಿದ ಭಾರತದ ಪತ್ರಗಳೆಲ್ಲ ನಿಜಕ್ಕೂ ಬಹುಕಾಲದ ಅನಂತರ ಬಂದ ಒಂದು ಶುಭ ಸಮಾಚಾರವೇ ಸರಿ. ದಿವಾನ್​ಜಿಯವರಿಂದ ಒಂದು ಸುಂದರ ಪತ್ರ ಬಂದಿದೆ. ಅಲ್ಲದೆ ಕಲ್ಕತ್ತದಲ್ಲಿ ನನ್ನ ಕುರಿತಾಗಿ ಪ್ರಕಟಿಸಿದ ಒಂದು ಪುಟ್ಟ ಕರಪತ್ರವೂ ಅದರಲ್ಲಿದೆ. ಅಂತೂ ಒಬ್ಬ ಪ್ರವಾದಿಯನ್ನು ಅವನ ದೇಶದಲ್ಲೇ ಗೌರವಿಸಲಾಗುತ್ತದೆ ಎಂಬುದನ್ನು ನನ್ನ ಜೀವನದಲ್ಲಿ ಮೊದಲ ಬಾರಿಗೆ ಕಂಡುಕೊಂಡೆ. ಕಲ್ಕತ್ತದ ದಿನಪತ್ರಿಕೆಗಳಿಂದ ಉದ್ಧೃತವಾದ ಬರಹಗಳ ಧಾಟಿ ತೀರ ಅಲಂಕಾರಪೂರ್ಣವಾಗಿದ್ದರಿಂದ, ಅವುಗಳನ್ನು ನಿನಗೆ ಕಳಿಸಲು ನನಗೆ ಇಷ್ಟವಿಲ್ಲ. ಆದರೆ ಅವು ಒಟ್ಟಿನಲ್ಲಿ ತೃಪ್ತಿಕರವಾಗಿವೆ. ಅವು ನನ್ನನ್ನು ‘ಖ್ಯಾತಿವೆತ್ತವರು’ ‘ಅದ್ಭುತ ವ್ಯಕ್ತಿ’ ಎಂದು ಇನ್ನೂ ಏನೇನೋ ಕೆಲಸಕ್ಕೆ ಬಾರದ ಗುಣವಾಚಕಗಳಿಂದ ಕರೆಯುತ್ತವೆ. ಆದರೆ ಅವು, ಇಡೀ ರಾಷ್ಟ್ರ ನನ್ನ ಮೇಲೆ ತೋರಿರುವ ಕೃತಜ್ಞತೆಯನ್ನು ವ್ಯಕ್ತಪಡಿಸುತ್ತವೆ.”

ತಮ್ಮನ್ನು ಮೆಚ್ಚಿ ಗೌರವಿಸಿ ಕೃತಜ್ಞತೆಯನ್ನರ್ಪಿಸುವಂತಹ ಈ ಪತ್ರಗಳು-ಲೇಖನಗಳು ಸ್ವಾಮೀಜಿಗೆ ತಲುಪಿದುವಾದರೂ ಮತ್ತು ಅವರು ಅದನ್ನು ತಮ್ಮ ಕೆಲವು ಆಪ್ತರಿಗೆ ಕಳಿಸಿಕೊಟ್ಟ ರಾದರೂ, ಅಮೆರಿಕದ ಪತ್ರಿಕೆಗಳಲ್ಲಿ ಅವು ಯಾವುವೂ ಪ್ರಕಟವಾಗಲಿಲ್ಲ. ಆದರೆ ಭಾರತದ ಮಿಷನರಿಗಳಿಗೆ ಸೇರಿದ ಪತ್ರಿಕೆಗಳಲ್ಲಿ ಹಾಗೂ ಇತರ ಧರ್ಮಾಂಧ ಪತ್ರಿಕೆಗಳಲ್ಲಿ ಸ್ವಾಮೀಜಿಯನ್ನು ಟೀಕಿಸಿ ನಿಂದಿಸಿ ಬರೆದ ವಿಚಾರಗಳನ್ನೆಲ್ಲ ಅಮೆರಿಕದ ಪತ್ರಿಕೆಗಳು ತಕ್ಷಣ ಪ್ರಕಟಿಸಿದುವು. ಜನರೂ ಅವುಗಳನ್ನು ಕುತೂಹಲದಿಂದ ಓದಿಕೊಂಡರು. ಇಂತಹ ಏಕಮುಖವಾದ ವರದಿಗಳನ್ನೆಲ್ಲ ಓದಿದ ಜನ ಕ್ರಮೇಣ ಅವುಗಳನ್ನು ನಂಬಲಾರಂಭಿಸಿದರು. ವಿವೇಕಾನಂದರನ್ನು ಅವರ ಸ್ವಂತ ತಾಯ್ನಾಡಿ ನಲ್ಲೇ ಗೌರವಿಸುವವರಿಲ್ಲ ಎಂದಮೇಲೆ ಅವರನ್ನು ನಂಬುವುದಾದರೂ ಎಂತು?–ಎಂದು ಶಂಕಿಸಿದರು. ಆದರೂ ಸ್ವಾಮೀಜಿ, ಸಾರ್ವಜನಿಕವಾಗಿ ಆ ಬಗ್ಗೆ ಮಾತನಾಡಲು ಇಷ್ಟಪಡಲಿಲ್ಲ. ಆದರೆ ತಮ್ಮ ಆಪ್ತರಾದ ಕೆಲವರಿಗೆ ಮಾತ್ರ ತಮ್ಮ ಪುಜುತ್ವವನ್ನು ಮನದಟ್ಟುಮಾಡಿಸ ಬೇಕಾದುದು ತಮ್ಮ ಕರ್ತವ್ಯವೆಂದು ಬಗೆದರು. ಪ್ರೊ ॥ ರೈಟರಿಗೆ ತಾವು ಈಗಾಗಲೇ ಕಳಿಸಿ ಕೊಟ್ಟಿದ್ದ ಪತ್ರಗಳನ್ನು ಓದಿ ಅವರು ಇದೆಲ್ಲ ಕೇವಲ ಹೊಗಳಿಕೆಯ ಮಾತುಗಳೆಂದು ನಿರ್ಲಕ್ಷಿಸು ವರೇನೋ ಎಂದು ಸ್ವಾಮೀಜಿ ಶಂಕಿಸಿದರು. ನಿಜಕ್ಕೂ ಪ್ರೊ ॥ರೈಟರಿಗೆ ಈ ಪ್ರಮಾಣಪತ್ರಗಳ ಆವಶ್ಯಕತೆಯಿತ್ತೇ ಎಂಬುದೇ ಸಂದೇಹಾಸ್ಪದ. ಆದರೆ ಅತ್ಯಂತ ವಿಷಮಯ ವಾತಾವರಣದಿಂದ ಆವೃತರಾಗಿದ್ದ ಸ್ವಾಮೀಜಿಗಂತೂ ಹಾಗೆನ್ನಿಸಿತು. ಆದ್ದರಿಂದ ಅವರು, ಖೇತ್ರಿಯ ಮಹಾರಾಜ ಅಜಿತ್​ಸಿಂಗ್ ಹಾಗೂ ಜುನಾಗಢದ ದಿವಾನರಾದ ಹರಿದಾಸ್ ವಿಹಾರಿದಾಸ್ ದೇಸಾಯಿಯವರು ಬರೆದಿದ್ದ ಪತ್ರಗಳನ್ನು ರೈಟರಿಗೆ ಕಳಿಸಿಕೊಡುತ್ತ ಹೀಗೆ ಬರೆದರು:

“ಇವು, ನಾನು ಮೋಸಗಾರನಲ್ಲವೆಂಬುದನ್ನು ನಿಮಗೆ ಮನವರಿಕೆ ಮಾಡಿಕೊಡುತ್ತವೆ ಎಂದು ಆಶಿಸುತ್ತೇನೆ. ನಾನು ಸಾಚಾ ಸಂನ್ಯಾಸಿಯೇ ಎಂಬುದು ಸ್ಪಷ್ಟವಾಗಲು, ಸಾಧ್ಯವಿರುವ ಎಲ್ಲ ಸಮಾಧಾನವನ್ನೂ ನಿಮಗೆ ನೀಡಲು ನಾನು ಬದ್ಧನಾಗಿದ್ದೇನೆ. ಆದರೆ ನಾನು ಹೀಗೆ ನೀಡುವುದು ನಿಮಗೆ ಮಾತ್ರ. ಬೀದಿಯ ಜನ ನನ್ನ ಬಗ್ಗೆ ಏನು ಹೇಳುತ್ತಾರೆ, ಏನು ಭಾವಿಸುತ್ತಾರೆ ಎಂಬುದನ್ನು ನಾನು ಲಕ್ಷಿಸುವುದಿಲ್ಲ.”

ಖೇತ್ರಿಯ ರಾಜ ಸ್ವಾಮೀಜಿಗೆ ಬರೆದ ಪತ್ರ ಹೀಗಿತ್ತು–

“ನನ್ನ ಪ್ರೀತಿಯ ಗುರುವೆ... ನನಗಿಂತ ಬಹಳಷ್ಟು ಹೆಚ್ಚು ತಿಳಿವಳಿಕಸ್ಥರಾದವರಿಗೆ ನಾನು ಸಲಹೆ ನೀಡುವ ಆವಶ್ಯಕತೆಯೇನೂ ಇಲ್ಲವೆಂಬುದು ನನಗೆ ಗೊತ್ತು. ಆದರೂ ನಾನು ಇಷ್ಟನ್ನು ಮಾತ್ರ ಹೇಳುವ ಧೈರ್ಯ ಮಾಡುತ್ತೇನೆ, ಏನೆಂದರೆ, ಬೆನ್ನ ಹಿಂದೆ ಆಡಿಕೊಳ್ಳುವ ನಮ್ಮ ಜನರ ಪ್ರವೃತ್ತಿಯನ್ನು ಕಂಡು ನೀವು ಜುಗುಪ್ಸೆ ಪಟ್ಟುಕೊಳ್ಳಬಾರದು. ಏಕೆಂದರೆ, ನಿಮಗೆ ತಿಳಿದಿರುವಂತೆ, ‘ಕ್ರಯವಿಕ್ರಯ ವೇಳಾಯಾಂ ಕಾಚಃಕಾಚಃ ಮಣಿರ್ಮಣಿಃ’ (ಬೆಲೆ ಕಟ್ಟುವ ವೇಳೆಯಲ್ಲಿ ಗಾಜು ಗಾಜೇ, ಮಣಿಯು ಮಣಿಯೇ.) ಪಾಶ್ಚಾತ್ಯ ದೇಶದ ಉನ್ನತ ಸುಸಂಸ್ಕೃತ ವ್ಯಕ್ತಿಗಳಿಂದ ಸ್ವಲ್ಪ ಸಹಾಯ ಪಡೆದು ನಮ್ಮ ತಾಯ್ನಾಡನ್ನು ಮೇಲೆತ್ತಬೇಕೆಂಬ ನಿಮ್ಮ ದೀರ್ಘಕಾಲದ ಯೋಜನೆ ಯನ್ನು ನೀವೇ ಕೈಬಿಟ್ಟರೆ, ಅದನ್ನು ಪೂರ್ಣಗೊಳಿಸುವವರು ಮತ್ತಾರಿದ್ದಾರೆ? ನಿಮ್ಮ ಶುಭ ದರ್ಶನವನ್ನು ಮಾಡಬೇಕೆಂಬ ಕಾತರವು, ಶೀಘ್ರದಲ್ಲೇ ಹಿಂದಿರುಗುವಂತೆ ನಿಮಗೆ ಬರೆಯುವಂತೆ ನನ್ನನ್ನು ಪ್ರೇರೇಪಿಸುತ್ತಿದೆ. ಆದರೆ ಅದರ ಜೊತೆಗೇ ಬೇರಾವುದೋ ಒಂದು ಶಕ್ತಿ ನನ್ನ ಲೇಖನಿ ಯನ್ನು ನಿಯಂತ್ರಿಸಿ, ನನ್ನ ಮನದಾಸೆಗೆ ವಿರುದ್ಧವಾಗಿ, ಎಂದರೆ ಮತ್ತಷ್ಟು ದಿನ ಅಲ್ಲೇ–ಜನರು ರತ್ನಗಳನ್ನು ಪರೀಕ್ಷಿಸುವಂತೆಯೇ ವ್ಯಕ್ತಿಗಳನ್ನು ಪರೀಕ್ಷಿಸುವ ಆ ರಾಷ್ಟ್ರದಲ್ಲೇ–ಉಳಿದುಕೊಳ್ಳ ಬೇಕೆಂದು ನಿಮ್ಮನ್ನು ಪ್ರಾರ್ಥಿಸಿಕೊಳ್ಳುವಂತೆ ಮಾಡುತ್ತದೆ.”

ಪ್ರೊ ॥ ರೈಟರಲ್ಲದೆ ತಮ್ಮ ಇತರ ಕೆಲವು ಆಪ್ತಸ್ನೇಹಿತರಿಗೂ ತಮ್ಮ ಪುಜುತ್ವವನ್ನು ಸಾಬೀತು ಮಾಡಬೇಕೆಂದು ಸ್ವಾಮೀಜಿ ಭಾವಿಸಿರಬೇಕು. ಆದ್ದರಿಂದ, ಬಹುಶಃ ಅವರ ಅಪೇಕ್ಷೆಯ ಮೇರೆಗೆ ಹರಿದಾಸ್ ದೇಸಾಯಿಯವರು ಸ್ವಾಮೀಜಿಯ ಅತ್ಯಂತ ಗೌರವಾನ್ವಿತ ಸ್ನೇಹಿತರಾದ ಶ್ರೀ ಜಾರ್ಜ್ ಹೇಲ್​ರವರಿಗೆ ಒಂದು ಪತ್ರವನ್ನು ಬರೆದು ಸ್ವಾಮೀಜಿ ಹಿಂದೂಧರ್ಮದ ನಿಷ್ಠಾವಂತ ಪ್ರತಿ ಪಾದಕರಾದ ಶ್ರೇಷ್ಠ ಸಂನ್ಯಾಸಿಗಳು ಎಂಬುದು ತಮಗೆ ನಿಶ್ಚಯವಾಗಿ ತಿಳಿದಿದೆಯೆಂದು ಹೇಳಿ ದರು. ಈ ಪತ್ರವನ್ನು ಶ್ರೀ ಜಾಜ್​ನ್ ಹೇಲ್​ರವರು ಸ್ವಾಮೀಜಿಗೆ ತೋರಿಸಿದರು. ಅದನ್ನು ನೋಡಿ ಸಂತೋಷಗೊಂಡ ಸ್ವಾಮೀಜಿ, ದೇಸಾಯಿಯವರಿಗೆ ತಮ್ಮ ಧನ್ಯವಾದಗಳನ್ನು ತಿಳಿಸಿದರು. ಆದರೆ ಶ್ರೀ ಜಾರ್ಜ್ ಹೇಲ್​ರವರು ಸ್ವಾಮೀಜಿಯ ಬಗ್ಗೆ ಇಟ್ಟಿದ್ದಂತಹ ಭರವಸೆ-ವಿಶ್ವಾಸಗಳು ಎಷ್ಟು ಆಳವಾದದ್ದೆಂದರೆ ಅವರ ಪಾಲಿಗೆ ಈ ಬಗೆಯ ‘ಯೋಗ್ಯತಾ ಪತ್ರ’ ಬೇಕೇ ಇರಲಿಲ್ಲ. ಇದಕ್ಕೆ ಒಂದು ನಿದರ್ಶನವನ್ನು ಕೊಡಬಹುದು. ಒಮ್ಮೆ ಅವರಿಗೆ ಅನಾಮಧೇಯ ಪತ್ರವೊಂದು ಬಂದಿತ್ತು. ಅದರಲ್ಲಿ ಒಂದು ಎಚ್ಚರಿಕೆಯ ಮಾತು ಬರೆಯಲಾಗಿತ್ತು–“ನಿಮ್ಮ ಮನೆಯಲ್ಲಿ ವಿವೇಕಾನಂದರನ್ನು ಇರಿಸಿಕೊಂಡಿದ್ದೀರಿ. ಆದರೆ ನಿಮ್ಮ ಹೆಣ್ಣುಮಕ್ಕಳು ಅವರೊಂದಿಗೆ ಸ್ವಲ್ಪ ಎಚ್ಚರಿಕೆಯಿಂದಿರುವಂತೆ ನೋಡಿಕೊಳ್ಳುವುದು ಕ್ಷೇಮ” ಎಂದು. ಜೊತೆಗೆ ಆ ಪತ್ರದಲ್ಲಿ ಇನ್ನೂ ಏನೇನೋ ಅಪವಾದದ ಮಾತುಗಳು ತುಂಬಿಕೊಂಡಿದ್ದುವು. ಇಂತಹ ಪತ್ರವನ್ನು ಓದಿದಾಗ ಯಾವ ತಂದೆ ತಾಯಂದಿರ ಮನಸ್ಸು ತಾನೆ ಕದಡಿಹೋಗದಿರದು! ಆದರೆ ಹೇಲ್​ರವರು ಅದೊಂದು ಹುಳುಹಿಡಿದ ವಸ್ತುವೋ ಎಂಬಂತೆ ತಕ್ಷಣವೇ ಒಲೆಯೊಳಗೆ ಹಾಕಿ ಸುಟ್ಟುಬಿಟ್ಟರು.

ಆದರೆ ಈ ವೇಳೆಗೆ ತಮ್ಮ ಇನ್ನೊಬ್ಬರು ವಿಶ್ವಾಸಿಗರಾದ ಶ್ರೀಮತಿ ಬ್ಯಾಗ್​ಲೀ ತಮ್ಮನ್ನು ದೂರ ಮಾಡಿದ್ದಾರೆಂದು ಸ್ವಾಮೀಜಿ ಭಾವಿಸಿದರು. ನಿಜಕ್ಕೂ ಅವರು ಸ್ವಾಮೀಜಿಯನ್ನು ದೂರ ಮಾಡಿದ್ದರೋ ಇಲ್ಲವೊ, ಆದರೆ ಸ್ವಾಮೀಜಿಗಂತೂ ಹಾಗನ್ನಿಸಿತು. ಏಕೆಂದರೆ ಬಾಸ್ಟನ್​ನ ಪತ್ರಿಕೆಯೊಂದರಲ್ಲಿ ಅವರ ವಿರುದ್ಧವಾಗಿ ಬರೆದಂತಹ ಲೇಖನದ ಪ್ರತಿಯೊಂದನ್ನು ಆಕೆ ಅವರಿಗೆ ಕಳಿಸಿಕೊಟ್ಟಿದ್ದರು ಮತ್ತು ಅದಾದನಂತರ ಪತ್ರವನ್ನೇ ಬರೆದಿರಲಿಲ್ಲ. ಶ್ರೀಮತಿ ಬ್ಯಾಗ್​ಲೀ ಆ ಲೇಖನದಿಂದ ಪ್ರಭಾವಿತರಾಗಿ ತಮ್ಮೊಂದಿಗಿನ ವ್ಯವಹಾರವನ್ನು ಕೊನೆಗೊಳಿಸಿಬಿಟ್ಟಿರಬೇಕೆಂದು ಸ್ವಾಮೀಜಿ ಊಹಿಸಿದರು. ಅಲ್ಲದೆ ಆ್ಯನಿಸ್ಕ್ವಾಮ್ ಎಂಬಲ್ಲಿನ ತನ್ನ ಸ್ವಂತ ವಿಶ್ರಾಂತಿಗೃಹಕ್ಕೆ ಬರುವಂತೆ ಆಕೆ ಆ ಮೊದಲು ಕಳಿಸಿದ್ದ ಆಹ್ವಾನವನ್ನು ರದ್ದುಪಡಿಸಿಬಿಟ್ಟಿದ್ದಾರೆಂದು ಸ್ವಾಮೀಜಿ ಭಾವಿಸಿದರು. ಆದರೆ ಕೊನೆಗೆ ನೋಡಿದರೆ ಇದೆಲ್ಲವೂ ಸ್ವಾಮೀಜಿಯ ತಪ್ಪುಗ್ರಹಿಕೆಯಾಗಿತ್ತು ಅಷ್ಟೆ. ಏಕೆಂದರೆ ಮುಂದೆ ಕೆಲವೇ ದಿನಗಳಲ್ಲಿ ಆಕೆ ತನ್ನ ಸ್ನೇಹಿತೆಯೊಬ್ಬಳಿಗೆ ಬರೆದ ಪತ್ರದಲ್ಲಿ ಸ್ವಾಮೀಜಿಯನ್ನು ತುಂಬ ಬಲವಾಗಿ ಸಮರ್ಥಿಸುವುದು ಕಂಡು ಬರುತ್ತದೆ. ಆ ಪತ್ರ ಹೀಗಿದೆ:

“ಪ್ರಿಯ ಗೆಳತಿ,

ನೀನು ನನ್ನ ಸ್ನೇಹಿತರಾದ ವಿವೇಕಾನಂದರ ಕುರಿತಾಗಿ ಬರೆದಿದ್ದೀಯೆ. ಅವರ ಶೀಲದ ಕುರಿತಾಗಿ ನನ್ನ ಮೆಚ್ಚುಗೆಯನ್ನು ಸೂಚಿಸಲು ಈ ಅವಕಾಶ ದೊರೆತಿರುವುದು ನನಗೆ ಸಂತೋಷವನ್ನುಂಟು ಮಾಡಿದೆ. ಮತ್ತು ಯಾರಾದರೂ ಅವರ ಶೀಲವನ್ನು ಶಂಕಿಸುತ್ತಾರೆಂದರೆ ನನಗೆ ಕೋಪ ಉಕ್ಕೇರು ತ್ತದೆ. ಅಮೆರಿಕದಲ್ಲಿ ನಮಗವರು ನಾವು ಈ ಹಿಂದೆಂದೂ ಕೇಳರಿಯದಿದ್ದ ಜೀವನಾದರ್ಶಗಳನ್ನು ನೀಡಿದ್ದಾರೆ. ಒಂದು ಹಳೆಯ ಸಂಪ್ರದಾಯನಿಷ್ಠ ನಗರವಾದ ಡೆಟ್ರಾಯ್ಟ್​ನ ಕ್ಲಬ್ಬುಗಳಲ್ಲಿ ಈ ಹಿಂದೆಂದೂ ಇನ್ನಾರಿಗೂ ಸಲ್ಲದಿದ್ದಂತಹ ಗೌರವ ಅವರಿಗೆ ಸಂದಿದೆ. ಅವರ ವಿರುದ್ಧವಾಗಿ ಒಂದೇ ಒಂದು ಮಾತನ್ನಾಡುವವರೂ ಕೂಡ ಅವರ ಘನತೆಯ ವಿಷಯದಲ್ಲಿ ಮತ್ಸರ ತಾಳಿದ್ದಾರೆ ಎಂದಷ್ಟೇ ನನಗನ್ನಿಸುತ್ತದೆ. ಅಲ್ಲದೆ ಜನ ಯಾಕಾದರೂ ಅಸೂಯೆಪಡಬೇಕು? ವಿವೇಕಾನಂದರು ಅಸೂಯೆ ಪಡುವಂಥದೇನನ್ನೂ ಮಾಡಿಲ್ಲವಲ್ಲ!

“ಕ್ರೈಸ್ತರಿಗಂತೂ ವಿವೇಕಾನಂದರು ಒಂದು ಹೊಸ ದೃಷ್ಟಿಯನ್ನೇ ಕೊಟ್ಟಿದ್ದಾರೆ. ಅವರು ನಮಗೆಲ್ಲ ದಿವ್ಯವಾದ ಮತ್ತು ಉದಾತ್ತವಾದ ಜೀವನ ಸಾಧ್ಯವಾಗುವಂತೆ ಮಾಡಿದ್ದಾರೆ. ಒಬ್ಬ ಧರ್ಮಬೋಧಕರಾಗಿ ಮತ್ತು ಪ್ರತಿಯೊಬ್ಬರಿಗೂ ಅನುಸರಣೀಯ ವ್ಯಕ್ತಿಯಾಗಿ ಅವರ ಮಟ್ಟಕ್ಕೆ ಬರಬಲ್ಲ ಮತ್ತೊಬ್ಬರನ್ನು ನಾನು ಕಾಣೆ. ಅವರೊಬ್ಬ ಉಚ್ಛೃಂಖಲ ವ್ಯಕ್ತಿಯೆನ್ನುವುದು ಸರಿಯಲ್ಲ, ಮತ್ತು ಸುಳ್ಳು ಕೂಡ. ದಿನದಿಂದ ದಿನಕ್ಕೆ, ಅವರ ಸಂಪರ್ಕಕ್ಕೆ ಬಂದವರೆಲ್ಲರೂ ಕೂಡ ಅವರ ಪರಿಶುದ್ಧ ಶೀಲ-ಗುಣಗಳ ಕುರಿತಾಗಿ ಅತ್ಯಂತ ಉತ್ಸಾಹದಿಂದ ಮಾತನಾಡುತ್ತಾರೆ. ಡೆಟ್ರಾಯ್ಟ್​ನಲ್ಲಿರುವ ಅತ್ಯಂತ ಕಟು ವಿಮರ್ಶಕರಾದ ಮತ್ತು ಈ ವಿಷಯದಲ್ಲಿ ಯಾರನ್ನೂ ಬಿಡದಿರುವ ಜನ ಕೂಡ ವಿವೇಕಾನಂದರನ್ನು ಮೆಚ್ಚಿಕೊಳ್ಳುತ್ತಾರೆ ಮತ್ತು ಗೌರವಿಸುತ್ತಾರೆ. ಅವರು ಮೂರು ವಾರಗಳಿಗಿಂತಲೂ ಹೆಚ್ಚು ಕಾಲ ನನ್ನ ಅತಿಥಿಯಾಗಿದ್ದರು. ನನ್ನ ಮಕ್ಕಳು, ನನ್ನ ಅಳಿಯ, ನನ್ನ ಇಡೀ ಕುಟುಂಬ ವಿವೇಕಾನಂದರಲ್ಲಿ ಕಂಡುಕೊಂಡದ್ದು ಸದಾ ಅತ್ಯಂತ ಶಿಷ್ಟ ಹಾಗೂ ವಿನಯಶೀಲ ಸಭ್ಯ ವ್ಯಕ್ತಿಯನ್ನು, ಉತ್ಸಾಹವುಕ್ಕಿಸುವ ಜೊತೆಗಾರರನ್ನು ಮತ್ತು ಸದಾ ಸ್ವಾಗತಾರ್ಹ ರಾದ ಅತಿಥಿಯನ್ನು. ನಾನವರನ್ನು ಇಲ್ಲಿ ಆ್ಯನಿಸ್ಕ್ಟಾಮ್​ನಲ್ಲಿರುವ ನನ್ನ ಬೇಸಿಗೆಮನೆಗೆ ಬರುವಂತೆ ಆಹ್ವಾನಿಸಿದ್ದೇನೆ. ನನ್ನ ಕುಟುಂಬದಲ್ಲಿ ಅವರಿಗೆ ಸದಾ ಆದರ ಹಾಗೂ ಸ್ವಾಗತ ಇದೆ. ಅವರ ವಿಷಯದಲ್ಲಿ ಇಲ್ಲದ್ದನ್ನೆಲ್ಲ ಹೇಳುವವರನ್ನು ಕಂಡರೆ ನನಗೆ ನಿಜಕ್ಕೂ, ಕೋಪ ಬರುವುದಕ್ಕಿಂತ ಹೆಚ್ಚಾಗಿ ಕನಿಕರವೆನಿಸುತ್ತದೆ. ಏಕೆಂದರೆ ಅವರಿಗೆ ತಾವು ಯಾರ ಬಗ್ಗೆ ಮಾತನಾಡುತ್ತಿದ್ದೇವೆಂದೇ ಗೊತ್ತಿಲ್ಲ. ಅವರು ಶಿಕಾಗೋದಲ್ಲಿರುತ್ತಿದ್ದ ಹೆಚ್ಚಿನ ಸಮಯದಲ್ಲೆಲ್ಲ ಶಿಕಾಗೋದ ಶ್ರೀ ಮತ್ತು ಶ್ರೀಮತಿ ಹೇಲ್​ರೊಂದಿಗೆ ಇರುತ್ತಿದ್ದರು. ಅದೇ ಅವರ ಮನೆಯಾಗಿತ್ತೆಂದೂ ಅನ್ನಬಹುದು. ಹೇಲ್ ದಂಪತಿಗಳು ಮೊದಲು ಅವರನ್ನು ಅತಿಥಿಯಾಗಿ ಆಹ್ವಾನಿಸಿದರು; ಆದರೆ ಆಮೇಲೆ ಅವರನ್ನು ಬಿಟ್ಟುಬಿಡುವುದಕ್ಕೇ ಇಷ್ಟಪಡಲಿಲ್ಲ. ಹೇಲ್ ದಂಪತಿಗಳು ಪ್ರಿಸ್ಟಿಟೇರಿಯನ್ನರು, ಸುಸಂಸ್ಕೃತರು ಮತ್ತು ನಾಗರಿಕ ಜನ. ಅವರು ವಿವೇಕಾನಂದರನ್ನು ಮೆಚ್ಚಿಕೊಳ್ಳುತ್ತಾರೆ, ಗೌರವಿಸು ತ್ತಾರೆ ಮತ್ತು ಪ್ರೀತಿಸುತ್ತಾರೆ. ವಿವೇಕಾನಂದರು ಶಕ್ತಿವಂತರು, ಉದಾತ್ತಚರಿತರು ಮತ್ತು ಭಗವಂತನೊಂದಿಗೆ ನಡೆದಾಡುವವರು. ಅವರು ಶಿಶುವಿನಂತೆ ಸರಳರು ಮತ್ತು ವಿಶ್ವಾಸಾರ್ಹರು. ಡೆಟ್ರಾಯ್ಟ್​ನಲ್ಲಿ ನಾನವರಿಗಾಗಿ ಒಂದು ಸಂಜೆ ಔತಣಕೂಟವನ್ನೇರ್ಪಡಿಸಿ ಮಹನೀಯರನ್ನು ಆಹ್ವಾನಿಸಿದ್ದೆ. ಎರಡು ವಾರಗಳ ನಂತರ ಅವರು ನನ್ನ ಮನೆಯ ಬೈಠಕ್​ಖಾನೆಯಲ್ಲಿ ಆಹ್ವಾನಿತ ಅತಿಥಿಗಳನ್ನುದ್ದೇಶಿಸಿ ಉಪನ್ಯಾಸ ಮಾಡಿದರು. ಅತಿಥಿಗಳ ಪಟ್ಟಿಯಲ್ಲಿ ವಕೀಲರು, ನ್ಯಾಯಾ ಧೀಶರು, ಮಂತ್ರಿಗಳು, ಸೈನ್ಯಾಧಿಕಾರಿಗಳು, ಡಾಕ್ಟರುಗಳು, ವ್ಯಾಪಾರಿಗಳು ಮತ್ತು ಇವರೆಲ್ಲರ ಪತ್ನಿಯರು ಹಾಗೂ ಹೆಣ್ಣುಮಕ್ಕಳು–ಇವರೆಲ್ಲ ಇದ್ದರು. ವಿವೇಕಾನಂದರು ಎರಡು ಗಂಟೆಗಳ ಕಾಲ “ಪ್ರಾಚೀನ ಹಿಂದೂ ದಾರ್ಶನಿಕರು ಮತ್ತು ಅವರ ಬೋಧನೆಗಳು” ಎಂಬುದರ ಕುರಿತಾಗಿ ಮಾತನಾಡಿದರು. ಎಲ್ಲರೂ ಕಡೆಯವರೆಗೂ ಅತ್ಯಂತ ಆಸಕ್ತಿಯಿಂದ ಕೇಳಿದರು. ಅವರು ಮಾತನಾಡಿದ್ದೆಲ್ಲವನ್ನೂ ಜನ ಅತ್ಯಂತ ಆನಂದದಿಂದ ಕೇಳಿ ಆಶ್ಚರ್ಯದಿಂದ ಹೇಳುತ್ತಾರೆ: ‘ಮನುಷ್ಯನೊಬ್ಬ ಹಾಗೆ ಮಾತನಾಡಿದ್ದನ್ನು ನಾವೆಂದಿಗೂ ಕೇಳಿಯೇ ಇಲ್ಲ’ ಎಂದು. ವಿವೇಕಾ ನಂದರು ಯಾರನ್ನೂ ಇದಿರುಹಾಕಿಕೊಳ್ಳುವುದಿಲ್ಲ. ಬದಲಾಗಿ ಜನರನ್ನು ಉನ್ನತ ಸ್ತರಕ್ಕೆ ಎತ್ತು ತ್ತಾರೆ. ಜನ ಅವರ ಭಾಷಣವನ್ನು ಕೇಳುವಾಗ ಮಾನವನಿರ್ಮಿತ ಮತಪಂಥಗಳಾಚೆಗಿನ ಏನೋ ಒಂದನ್ನು ಕಾಣುತ್ತಿದ್ದರು. ಮತ್ತು ಧಾರ್ಮಿಕ ನಂಬಿಕೆಗಳಲ್ಲಿ ವಿವೇಕಾನಂದರೊಂದಿಗೆ ತಾದಾತ್ಮ್ಯ ವನ್ನು ಅನುಭವಿಸುತ್ತಿದ್ದರು.

“ವಿವೇಕಾನಂದರನ್ನು ಅರಿಯುವುದರಿಂದ ಮತ್ತು ಅವರೊಂದಿಗೆ ವಾಸಿಸುವುದರಿಂದ ಪ್ರತಿ ಯೊಬ್ಬ ಮಾನವಜೀವಿಯೂ ಕೂಡ ಉತ್ತಮಗೊಳ್ಳುತ್ತಾನೆ. ಅಮೆರಿಕದಲ್ಲಿರುವ ಪ್ರತಿಯೊ ಬ್ಬರೂ ವಿವೇಕಾನಂದರನ್ನು ಅರಿತುಕೊಳ್ಳಬೇಕೆಂದು ನಾನು ಬಯಸುತ್ತೇನೆ. ಮತ್ತು ಭಾರತದಲ್ಲಿ ವಿವೇಕಾನಂದರಂಥವರು ಇನ್ನೂ ಯಾರಾದರಿದ್ದರೆ ಅವರನ್ನೂ ಭಾರತ ನಮ್ಮಲ್ಲಿಗೆ ಕಳುಹಿಸಲಿ.”

ಸ್ವಾಮೀಜಿ ತಮ್ಮ ಮೇಲೆ ಬಂದ ಅಪವಾದದ ಮಾತುಗಳನ್ನು ಕೇಳಿ ಒಬ್ಬೊಬ್ಬರಾಗಿ ಕೈಬಿಟ್ಟಂತೆ ಶ್ರೀಮತಿ ಬ್ಯಾಗ್​ಲೀ ಕೂಡ ತಮ್ಮನ್ನು ಕೈಬಿಟ್ಟರು ಎಂದು ಭಾವಿಸಿ ಕುಳಿತಿದ್ದರೆ, ಆಕೆ ಅವರ ಕುರಿತಾಗಿ ತನ್ನ ಸ್ನೇಹಿತೆಗೆ ಬರೆದ ಪತ್ರದ ಧಾಟಿ ಹೀಗಿದೆ! ಪರರಾಷ್ಟ್ರದ, ಪರಸಂಸ್ಕೃತಿಯ ಮಹಿಳೆಯೊಬ್ಬಳು ಸ್ವಾಮೀಜಿಯ ಒಳಹೊರಗನ್ನು ಇಷ್ಟು ಸ್ಪಷ್ಟವಾಗಿ ಅರ್ಥಮಾಡಿಕೊಂಡಿರು ವುದನ್ನು ಕಂಡಾಗ ಆಶ್ಚರ್ಯವಾಗುತ್ತದೆ.

ಅಂತೂ ಸ್ವಾಮೀಜಿಯ ಪಾಲಿಗೆ ಇಷ್ಟೆಲ್ಲ ಮೆಚ್ಚುಗೆಯ ಮಾತುಗಳು ಸಿಕ್ಕಿದರೂ ಅವರ ತಾಯ್ನಾಡಿನ ಜನರಿಂದ ನಾಲ್ಕು ವಾಕ್ಯಗಳು ಅಮೆರಿಕದ ಪತ್ರಿಕೆಗಳಲ್ಲಿ ಕಾಣಿಸದೆ ಹೋದರೆ ಏನೂ ಪ್ರಯೋಜನವಿರಲಿಲ್ಲ. ಆದ್ದರಿಂದ ಅಲ್ಲಿನ ಅವರ ವೈರಿಗಳನ್ನು ಇದಿರಿಸುವುದು ಕಷ್ಟಕರ ವಾಗಿಯೇ ಉಳಿದಿತ್ತು; ಸ್ವಾಮೀಜಿ ತತ್ಕಾಲಕ್ಕೆ ಇನ್ನೂ ನಿರಾಶರಾಗಿಯೇ ಇದ್ದರು. ಅವರು ತಮ್ಮ ಆಗಿನ ಸ್ಥಿತಿಯ ಬಗ್ಗೆ ಮದ್ರಾಸಿನ ಶಿಷ್ಯರೊಬ್ಬರಿಗೆ ಬರೆದರು:

“ಈಗ ಇಲ್ಲಿ ನನ್ನ ಭವಿಷ್ಯದ ಕುರಿತು ಹೇಳುವುದಾದರೆ, ಅದೊಂದು ದೊಡ್ಡ ಸೊನ್ನೆ, ಅಷ್ಟೆ. ಏಕೆ? ನಾನು ಅತ್ಯಂತ ಉನ್ನತ ಉದ್ದೇಶವನ್ನಿಟ್ಟುಕೊಂಡಿದ್ದರೂ ಈ ಎಲ್ಲ ಕಾರಣಗಳಿಂದ ಅದು ನಿಷ್ಪ್ರಯೋಜಕವಾಗಿದೆ, ಶೂನ್ಯವಾಗಿದೆ. ನನಗೆ ಭಾರತದ ಕಡೆಯಿಂದ ಮಾಹಿತಿ ದೊರೆಯು ವುದೇನಿದ್ದರೂ ಅದು ನಿನ್ನ ಪತ್ರಗಳಿಂದ ಮಾತ್ರ. ನಿನ್ನ ಪತ್ರಗಳು, ಹೇಗೆ ನನ್ನನ್ನು ಭಾರತದಲ್ಲಿ ಪ್ರಶಂಸಿಸಲಾಗುತ್ತಿದೆ ಎಂಬುದನ್ನು ಮತ್ತೆ ಮತ್ತೆ ಹೇಳುತ್ತವೆ. ಆದರೆ ಅದೆಲ್ಲ ಏನಿದ್ದರೂ ನನ್ನ ನಿನ್ನ ನಡುವೆ ಮಾತ್ರ. ಏಕೆಂದರೆ ಅಳಸಿಂಗ ನನಗೆ ಕಳಿಸಿದ ಆ ಮೂರು ಇಂಚಿನ ಸುದ್ದಿಯನ್ನು ಬಿಟ್ಟರೆ ಮತ್ತೆ ಒಂದೇ ಒಂದು ಪದ ಕೂಡ ಅಮೆರಿಕಕ್ಕೆ ತಲುಪಲೇ ಇಲ್ಲ. ಇದರಿಂದಾಗಿ ಈ ದೇಶದಲ್ಲಿ ಬಹಳಷ್ಟು ಜನ ನನ್ನನ್ನು ಒಬ್ಬ ಠಕ್ಕ ಎಂದು ತಿಳಿದುಕೊಂಡಿದ್ದಾರೆ. ಈ ಮಿಷನರಿಗಳಿಗೆ ನಮ್ಮ ಹಿಂದುಗಳ ಹೊಟ್ಟೆಕಿಚ್ಚು ಬೆಂಬಲವಾಗಿ ನಿಂತಿರುವಾಗ ಅವರೆದುರಿಗೆ ಹೇಳಲು ನನ್ನಲ್ಲಿ ಒಂದೇ ಒಂದು ಮಾತೂ ಇಲ್ಲ. ಈಗ ನನಗೆ ಅನ್ನಿಸುತ್ತದೆ–ಈ ಮದರಾಸು ಹುಡುಗರ ಮಾತು ಕೇಳಿಕೊಂಡು ಇಲ್ಲಿಗೆ ಬಂದದ್ದು ನನ್ನ ಮೂರ್ಖತನ ಎಂದು. ಎಷ್ಟಾದರೂ ಅವರು ಹುಡುಗರು, ನಾನವರಿಗೆ ಎಂದೆಂದಿಗೂ ಚಿರಪುಣಿಯೇನೋ ನಿಜ. ಆದರೆ ಅವರೆಷ್ಟಾದರೂ ಯಾವುದೇ ಬಗೆಯ ಕಾರ್ಯನಿರ್ವಹಣೆಯ ಸಾಮರ್ಥ್ಯವಿಲ್ಲದ ಯುವಕರು, ಅಷ್ಟೆ. ನಾನು ಭಾವಿಸಿದ್ದೆ– ಮದರಾಸು ಕಲ್ಕತ್ತಗಳಲ್ಲಿ ಕೆಲವು ಗೌರವಾನ್ವಿತ ವ್ಯಕ್ತಿಗಳ ಸಭೆಯೊಂದನ್ನು ಸೇರಿಸಿ, ನನಗೆ ಹಾಗೂ ನನ್ನ ವಿಷಯದಲ್ಲಿ ವಿಶ್ವಾಸವಿಟ್ಟಿರುವುದಕ್ಕಾಗಿ ಅಮೆರಿಕದ ಜನರಿಗೆ ಧನ್ಯವಾದಗಳನ್ನು ಅರ್ಪಿಸುವ ಗೊತ್ತುವಳಿಯೊಂದನ್ನು ಹೊರಡಿಸಿ ಅದನ್ನು ಕಾರ್ಯಕ್ರಮದ ಅಧ್ಯಕ್ಷರ ಮೂಲಕ ಅಧಿಕೃತವಾಗಿ ಅಮೆರಿಕಕ್ಕೆ–ಉದಾಹರಣೆಗೆ, ಡಾ ॥ ಬರೋಸ್​ರವರಿಗೆ–ಕಳಿಸಿ ಅದನ್ನು ಪತ್ರಿಕೆಗಳಲ್ಲಿ ಪ್ರಕಟಿಸು ವಂತೆ ಕೇಳಿಕೊಳ್ಳುವುದು–ಇದರಷ್ಟು ಸುಲಭದ ಕೆಲಸ ಮತ್ತಾವುದೂ ಇರಲಾರದು ಎಂದು. ಆದರೆ ಈಗ ನನಗೆ ಕಂಡುಬರುತ್ತಿದೆ, ಭಾರತಕ್ಕೆ ಇದೇ ಒಂದು ಭಾರವಾದ ಕಾರ್ಯವಾಗಿಬಿಟ್ಟಿದೆ ಎಂದು. ಈ ಒಂದು ವರ್ಷದಲ್ಲಿ ನನ್ನ ಪರವಾಗಿ ಒಂದೇ ಒಂದು ಶಬ್ದವೂ ಇಲ್ಲ. ಇಲ್ಲಿ ಪ್ರತಿ ಯೊಬ್ಬರೂ ನನಗೆ ವಿರುದ್ಧವಾಗಿದ್ದಾರೆ. ಏಕೆಂದರೆ, ನೀವು ನಿಮ್ಮ ಮನೆಗಳಲ್ಲಿ ನನ್ನ ಬಗೆಗೆ ಏನು ಹೇಳಿಕೊಂಡರೂ ಅದರ ವಿಚಾರ ಇಲ್ಲಿರುವವರಿಗೆ ಹೇಗೆ ತಾನೆ ಗೊತ್ತಾಗಬೇಕು?

“ಎರಡು ತಿಂಗಳಿಗೂ ಹಿಂದೆಯೇ ನಾನಿದರ ಕುರಿತಾಗಿ ಅಳಸಿಂಗನಿಗೆ ಬರೆದಿದ್ದೆ. ಆದರೆ ಅವನು ನನ್ನ ಪತ್ರಕ್ಕೆ ಉತ್ತರವನ್ನೇ ಬರೆಯಲಿಲ್ಲ. ಅವನೆದೆಯನ್ನು ಉತ್ಸಾಹಶೂನ್ಯತೆ ಆವರಿಸಿಬಿಟ್ಟಿದೆ ಎಂದು ತೋರುತ್ತದೆ. ಓಹ್, ಸ್ವಲ್ಪ ಬುದ್ಧಿಶಕ್ತಿ ಹಾಗೂ ಸಾಮರ್ಥ್ಯವಿರುವ ಒಬ್ಬ ಮನುಷ್ಯ ನಾದರೂ ನನಗೆ ಭಾರತದಲ್ಲಿ ಬೆಂಬಲಿಗನಾಗಿ ಸಿಕ್ಕಿದ್ದರೆ!ಆದರೆ, ಹೋಗಲಿ ಬಿಡು, ಭಗವಂತನ ಇಚ್ಛೆಯಿದ್ದಂತಾಗುತ್ತದೆ. ಈಗ ನಾನೀದೇಶದಲ್ಲಿ ಒಬ್ಬ ದಗಾಖೋರನಾಗಿ ನಿಂತಿದ್ದೇನೆ. ಯಾವುದೇ ಬಗೆಯ ಯೋಗ್ಯತಾ ಪತ್ರವೂ ಇಲ್ಲದೆ–ಅಂತಹ ಪತ್ರಗಳು ನನಗೆ ಅಲ್ಲಿಯೇ ಸಿಗು ತ್ತವೆ ಎಂಬ ಭರವಸೆಯಿಂದ–ಸಮ್ಮೇಳನಕ್ಕೆ ಬಂದದ್ದು ನನ್ನದೇ ಮೂರ್ಖತನ. ನಾನದನ್ನು ಕ್ರಮೇಣ ಹೋಗಲಾಡಿಸಿಕೊಳ್ಳಬೇಕಾಗಿದೆ. ಸಾಕು, ಇನ್ನು ನಿಮಗೆಲ್ಲ ನಮಸ್ಕಾರ. ನಾನು ಹಿಂದೂ ಗಳನ್ನು ಬೇಕಾದಷ್ಟು ಕಂಡದ್ದಾಯಿತು. ಈಗ ಭಗವಂತನ ಇಚ್ಛೆಯಂತಾಗಲಿ. ನಾನು ನನ್ನ ಕರ್ಮಕ್ಕೆ ತಲೆಬಾಗುತ್ತೇನೆ. ಆದರೆ ನನ್ನನ್ನು ಕೃತಘ್ನನೆಂದು ಭಾವಿಸಬೇಡಿ. ನನಗಾಗಿ ಮದರಾಸಿನ ಜನ ನನ್ನ ಅರ್ಹತೆಗಿಂತಲೂ ಹೆಚ್ಚಾಗಿ, ಮತ್ತು ಅವರ ಸಾಮರ್ಥ್ಯಕ್ಕಿಂತಲೂ ಮಿಗಿಲಾಗಿ ಮಾಡಿದ್ದಾರೆ. ಆದರೆ ನಾವು–ಹಿಂದೂಗಳು–ಇನ್ನೂ ಮನುಷ್ಯರಾಗಿಲ್ಲ ಎಂಬುದನ್ನು ಒಂದು ಕ್ಷಣ ಮರೆತದ್ದು, ಮತ್ತು ಒಂದು ಕ್ಷಣ ನನ್ನ ಸ್ವಾವಲಂಬನೆಯ ಅಭ್ಯಾಸವನ್ನು ಬಿಟ್ಟುಕೊಟ್ಟು ಹಿಂದೂಗಳನ್ನು ನೆಚ್ಚಿಕೊಂಡದ್ದು–ಇದು ನನ್ನ ಮೂರ್ಖತನ. ಇದರಿಂದಲೇ ನಾನು ಸಂಕಟಕ್ಕೆ ಸಿಲುಕುವಂತಾದದ್ದು. ಕ್ಷಣಕ್ಷಣವೂ ನಾನು ಭಾರತದಿಂದ ಏನಾದರೂ ಬರುತ್ತದೆಂದು ನಿರೀಕ್ಷಿ ಸುತ್ತಿದ್ದೆ. ಆದರೆ ಇಲ್ಲ, ಅದೆಂದೂ ಬರಲೇ ಇಲ್ಲ. ಅದರಲ್ಲೂ ಕಳೆದ ಎರಡು ತಿಂಗಳಂತೂ ನಾನು ಪ್ರತಿಕ್ಷಣವೂ ಹಿಂಸೆಯನ್ನು ಅನುಭವಿಸಬೇಕಾಯಿತು. ಇಲ್ಲಿ ನನ್ನ ಸ್ನೇಹಿತರು ಕಾದರು, ತಿಂಗಳುಗಟ್ಟಲೆ ಕಾದರು. ಆದರೆ ಏನೂ ಬರಲಿಲ್ಲ. ಕಡೆಗೆ ಎಷ್ಟೋ ಜನ ಬೇಸರಗೊಂಡು ನನ್ನ ಕೈಬಿಟ್ಟರು. ಆದರೆ ಇದು ಮನುಷ್ಯರನ್ನು ಹಾಗೂ ಮೃಗಗಳನ್ನು ನೆಚ್ಚಿಕೊಂಡಿದ್ದಕ್ಕೆ ದೊರೆತ ಶಿಕ್ಷೆ. ಏಕೆಂದರೆ ನಮ್ಮ ದೇಶದವರಿನ್ನೂ ಮನುಷ್ಯರಾಗಿಲ್ಲ. ಅವರು ಹೊಗಳಿಸಿಕೊಳ್ಳಲು ಸಿದ್ಧ. ಆದರೆ ತಾವು ಒಂದೇ ಒಂದು ಮಾತನಾಡಬೇಕಾಗಿ ಬಂದಾಗ, ಅವರು ಮಂಗಮಾಯ.

“ಮದರಾಸಿನ ಯುವಕರಿಗೆ ಅನಂತ ಧನ್ಯವಾದಗಳು. ದೇವರು ಅವರಿಗೆ ಎಂದೆಂದೂ ಒಳ್ಳೆಯದು ಮಾಡಲಿ... ಜಿ. ಜಿ., ಅಳಸಿಂಗ, ಸೆಕ್ರೆಟರಿ ಹಾಗೂ ಉಳಿದವರೆಲ್ಲರಿಗೂ ನನ್ನ ಅನಂತ ಆಶೀರ್ವಾದಗಳನ್ನು ತಿಳಿಸು. ನಾನು ಸದಾ ಅವರ ಒಳಿತಿಗಾಗಿ ಪ್ರಾರ್ಥಿಸುತ್ತಿರುತ್ತೇನೆ. ನಾನವರ ವಿಷಯದಲ್ಲಿ ಸ್ವಲ್ಪವೂ ಅಸಮಾಧಾನಪಟ್ಟುಕೊಂಡಿಲ್ಲ. ಆದರೆ ನನ್ನ ವಿಷಯದಲ್ಲೇ ನನಗೆ ಸಮಾಧಾನವಿಲ್ಲ. ನಾನು ನನ್ನ ಜೀವನದಲ್ಲಿ ಒಂದು ಬಾರಿ ಭಯಂಕರ ತಪ್ಪನ್ನು–ಇತರರ ಸಹಾಯವನ್ನು ನೆಚ್ಚಿಕೊಳ್ಳುವ ತಪ್ಪನ್ನು–ಮಾಡಿದೆ. ನಾನದಕ್ಕೆ ದಂಡ ತೆತ್ತಿದ್ದೇನೆ. ಅದು ನನ್ನ ತಪ್ಪು, ಅವರದಲ್ಲ. ಮದರಾಸಿನ ಜನರಿಗೆಲ್ಲ ಭಗವಂತ ಒಳ್ಳೆಯದು ಮಾಡಲಿ. ಅವರು ಬಂಗಾಳಿಗಳಿಗಿಂತ ಎಷ್ಟೋ ಪಾಲು ಮೇಲು. ಬಂಗಾಳಿಗಳಾದರೋ ಶುದ್ಧ ಮೂರ್ಖರು. ಅವರಲ್ಲಿ ಜೀವವೇ ಇಲ್ಲ, ಏನೂ ಸತ್ತ್ವವೇ ಇಲ್ಲ... ಇನ್ನು ಸಾಕು, ನಮಸ್ಕಾರ! ದೋಣಿಯನ್ನು ಅಲೆಗಳಲ್ಲಿ ತೇಲಿಬಿಟ್ಟಾಗಿದೆ; ಈಗ ಏನಾಗುತ್ತದೆಯೋ ಆಗಲಿ. ಆದರೆ, ನನ್ನ ಪರುಷೋಕ್ತಿಗಳ ಬಗ್ಗೆ ಹೇಳುವುದಾದರೆ ನಿಜಕ್ಕೂ ನನಗೆ ಅವುಗಳನ್ನು ನುಡಿಯಲು ಅಧಿಕಾರವಿಲ್ಲ. ನನ್ನ ಅರ್ಹತೆ ಗಿಂತಲೂ ಕೋಟಿ ಪಾಲು ಹೆಚ್ಚಿನದನ್ನು ನೀವು ನನಗಾಗಿ ಮಾಡಿದ್ದೀರಿ. ನನ್ನ ಕರ್ಮದ ಭಾರವನ್ನು ನಾನೇ ಹೊರಬೇಕು–ಅದೂ ತುಟಿಪಿಟಿಕ್ಕೆನ್ನದೆ. ಭಗವಂತ ನಿಮಗೆಲ್ಲ ಒಳ್ಳೆಯದು ಮಾಡಲಿ.”

ಈ ಪತ್ರದಲ್ಲಿ ಸ್ವಾಮೀಜಿಯ ಮನಸ್ಸಿನ ತುಮುಲವನ್ನೂ ಅವರು ಸಿಲುಕಿಕೊಂಡಿದ್ದ ಪರಿ ಸ್ಥಿತಿಯ ಭೀಕರತೆಯನ್ನೂ ಕಾಣಬಹುದು. ಒಬ್ಬರೇ ಕುಳಿತು ಧ್ಯಾನಮಾಡಿ ಭಗವಂತನ ಸಾಕ್ಷಾತ್ಕಾರ ಮಾಡಿಕೊಳ್ಳುವುದಾದರೂ ಸುಲಭ. ಆದರೆ ಸಾರ್ವಜನಿಕ ಸೇವೆ ಮಾಡಿ ಯಶಸ್ವಿಯಾಗಬೇಕಾದರೆ, ಬಹಳ ಕಷ್ಟ. ಅಂಥ ಶಕ್ತಿಶಾಲಿಗಳೂ ಮಹಾಮಹಿಮರೂ ಸ್ಥಿರಶಾಂತಬುದ್ಧಿಯವರೂ ಆದ ಸ್ವಾಮಿ ವಿವೇಕಾನಂದರಿಗೇ ಇಂತಹ ಪರಿಸ್ಥಿತಿಯೊದಗಬೇಕಾದರೆ! ಆದರೆ ನಿಜಕ್ಕೂ ಸ್ವಾಮೀಜಿ ಭಾವಿಸಿದ್ದಂತೆ ಈ ಮದರಾಸೀ ಜನ ಸಭೆ ಏರ್ಪಡಿಸುವ ವಿಷಯದಲ್ಲಿ ಅಷ್ಟೇನೂ ತಡಮಾಡಿರ ಲಿಲ್ಲ. ಸ್ವಾಮೀಜಿ ಪತ್ರವನ್ನು ಬರೆದ ಎರಡು ವಾರಗಳೊಳಗಾಗಿ ಬೃಹತ್ಸಭೆಯೊಂದು ಯಶಸ್ವಿ ಯಾಗಿ ನಡೆದುಹೋಗಿತ್ತು. ಇದು ನಡೆದದ್ದು ಆಗಬೇಕಾದುದಕ್ಕಿಂತ ಸಾಕಷ್ಟು ತಡವಾಗಿಯೇ ಆದರೂ, ಆಗಲೂ ಕಾಲ ಮಿಂಚಿರಲಿಲ್ಲ. ಆದರೆ ಈ ಸಭೆಯ ವರದಿ ಸ್ವಾಮೀಜಿಯ ಕೈಗೆ ತಲಪುವಲ್ಲಿ ತೀರ ತಡವಾಯಿತು. ಇದರಿಂದಾದ ಹಾನಿ ಮಾತ್ರ ಅಪಾರ.

