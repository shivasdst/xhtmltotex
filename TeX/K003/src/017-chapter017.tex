
\chapter{ಅಭಿನಂದನೆಯ ಮಹಾಪೂರ}

\noindent

ಸಭೆಯೊಂದನ್ನು ಏರ್ಪಡಿಸುವ ಬಗ್ಗೆ ಸ್ವಾಮೀಜಿ ಮೊತ್ತಮೊದಲು ಅಳಸಿಂಗ ಪೆರುಮಾಳ್​ಗೆ ಬರೆದ ಮರುದಿನವೇ, ಎಂದರೆ ೧೮೯೪ರ ಏಪ್ರಿಲ್ ೧ಂರಂದೇ, ಕಲ್ಕತ್ತದ ಪ್ರಸಿದ್ಧ ದಿನಪತ್ರಿಕೆ ಯಾದ ‘ಇಂಡಿಯನ್ ಮಿರರ್​’ನಲ್ಲಿ ಇಂತಹ ಧನ್ಯವಾದ ಸಭೆಯೊಂದನ್ನು ನಡೆಸಬೇಕಾದ ಆವಶ್ಯಕತೆಯನ್ನು ಒತ್ತಿಹೇಳಲಾಗಿತ್ತು. ಅಲ್ಲದೆ ಇಂತಹ ಕಾರ್ಯಕ್ರಮಗಳನ್ನು ನಡೆಸಬೇಕೆಂಬ ಉತ್ಸಾಹ ಎಲ್ಲೆಲ್ಲೂ ಕಂಡುಬರುತ್ತಿತ್ತು. ಬಹುಶಃ ಕಲ್ಕತ್ತದ ಜನರಿಗೆ ಇಂತಹ ಉತ್ಸಾಹ ಸಹಜ ವಾಗಿಯೇ ಉಂಟಾಗಿರಬಹುದು. ಇಲ್ಲವೆ ಸ್ವಾಮೀಜಿಯೇ ಬರೆದ ಯಾವುದಾದರೂ ಪತ್ರದಿಂದ ಅವರಲ್ಲಿ ಪ್ರಚೋದನೆಯುಂಟಾಗಿರಬಹುದು. ಆ ಕುರಿತಾಗಿ ಪ್ರಕಟವಾದ ಸಂಪಾದಕೀಯ ಹೀಗಿತ್ತು:

“ಪ್ರವಾದಿಯೊಬ್ಬನು ತನ್ನ ದೇಶದಲ್ಲೇ ಗೌರವಿಸಲ್ಪಡುವುದಿಲ್ಲ ಎಂಬ ಮಾತು ನಮ್ಮ ಜೀವನ ದಲ್ಲೇ ಆಗಾಗ ಅನುಭವಕ್ಕೆ ಬರುವ ಸಾಮಾನ್ಯ ಸತ್ಯ. ಸ್ವಾಮಿ ವಿವೇಕಾನಂದರು ಅಮೆರಿಕಕ್ಕೆ ಭೇಟಿ ನೀಡಿರದಿದ್ದರೆ ಇಷ್ಟೊಂದು ಪ್ರಸಿದ್ಧರಾಗುತ್ತಿದ್ದರೋ ಇಲ್ಲವೋ ಅನುಮಾನ. ಶಿಕಾಗೋದ ಸರ್ವ ಧರ್ಮ ಸಮ್ಮೇಳನದಲ್ಲಿ ಹಿಂದೂಧರ್ಮದ ವ್ಯಾಖ್ಯಾನಕ್ಕಾಗಿ ಅವರಿಗೆ ಏನೇನು ಯಶಸ್ಸು ದೊರಕಿತೋ ಅದೆಲ್ಲಕ್ಕೂ ವಿಶಾಲ ಹೃದಯರಾದ ಅಮೆರಿಕನ್ನರನ್ನು ಅಭಿನಂದಿಸಬೇಕು... ಸ್ವಾಮಿ ವಿವೇಕಾನಂದರು ಅಮೆರಿಕದಲ್ಲಿ ಮಾಡಿದ ಭಾಷಣಪ್ರವಾಸದಲ್ಲಿ ಗಳಿಸಿದ ಪ್ರಚಂಡ ಯಶಸ್ಸನ್ನು ಗಮನಿಸಿ ಹಿಂದೂಗಳು ಅವರಿಗೆ, ಮತ್ತು ಯಾರ ಸಹಾಯವಿಲ್ಲದೆ ಅವರಿಗೆ ಅಮೆರಿಕ ದಲ್ಲಿ ಅಷ್ಟು ದೃಢವಾಗಿ ನೆಲೆಯೂರಲು ಸಾಧ್ಯವಾಗುತ್ತಿರಲಿಲ್ಲವೋ ಆ ಸಮ್ಮೇಳನದ ವ್ಯವಸ್ಥಾಪಕ ರಿಗೆ ಒಂದು ಬಿನ್ನವತ್ತಳೆಯನ್ನು ಅರ್ಪಿಸುವುದಾದರೆ, ಆ ಮೂಲಕ ನಮ್ಮ ಕೃತಜ್ಞತೆಯನ್ನರ್ಪಿಸುವ ಕರ್ತವ್ಯವನ್ನು ಮಾಡಿದಂತಾಗುತ್ತದೆ ಎಂದು ನಮಗನ್ನಿಸುತ್ತದೆ. ಈ ಯೋಜನೆಯಲ್ಲಿ ದೇಶ ದಾದ್ಯಂತ ನಮ್ಮ ಹಿಂದೂ ಬಂಧುಗಳು ಹೃತ್ಪೂರ್ವಕವಾಗಿ ಪಾಲ್ಗೊಳ್ಳುವರೆಂದು ಆಶಿಸುತ್ತೇವೆ. ಸ್ವಾಮಿ ವಿವೇಕಾನಂದರು ಇನ್ನೂ ಅಮೆರಿಕದಲ್ಲೇ ಇದ್ದಾರೆ. ತಡಮಾಡದೆ ಬಿನ್ನವತ್ತಳೆಯನ್ನು ಅಲ್ಲಿಗೆ ಕಳಿಸಲೇಬೇಕು. ನಮ್ಮ ಹಿಂದೂ ಸೋದರನಿಗೆ ಸಲ್ಲಿಸಿದ ಮನ್ನಣೆಗಾಗಿ ನಾವು ಕೃತಜ್ಞತಾ ಶೂನ್ಯರಾಗಿಲ್ಲ ಎಂಬುದನ್ನು ನಾವು ನಮ್ಮ ಅಮೆರಿಕದ ಮಿತ್ರರಿಗೆ ತಿಳಿಸಬೇಕು. ಬಿನ್ನವತ್ತಳೆಯನ್ನು ಅರ್ಪಿಸುವಲ್ಲಿ ಇನ್ನು ಹೆಚ್ಚು ವಿಳಂಬ ಮಾಡಬಾರದು. ಮತ್ತು, ಈ ವಿಷಯದಲ್ಲಿ ದೇಶದ ಎಲ್ಲೆಡೆ ಗಳಲ್ಲಿನ ಹಿಂದೂ ಸೋದರರ ಅಭಿಪ್ರಾಯಗಳನ್ನು ನಾವು ಪಡೆದುಕೊಳ್ಳಬೇಕು.”

ಸ್ವಾಮೀಜಿ ಏಪ್ರಿಲ್ ೯ರಂದು ಬರೆದ ಪತ್ರ ಅಳಸಿಂಗ ಪೆರುಮಾಳರ ಕೈಗೆ ತಲುಪುವ ಮೊದಲೇ ಮದರಾಸಿನಲ್ಲಿ ಸಾರ್ವಜನಿಕ ಸಮಾರಂಭವೊಂದನ್ನು ನಡೆಸಲು ಸಿದ್ಧತೆಗಳು ಭರದಿಂದ ನಡೆಯು ತ್ತಿದ್ದುವು. ಏಪ್ರಿಲ್ ೨೮ರಂದು ಮದರಾಸಿನ ಪಚ್ಚೈಯಪ್ಪ ಹಾಲ್​ನಲ್ಲಿ ಸಭೆ ಏರ್ಪಾಡಾಗಿತ್ತು. ನಾಗರಿಕರು ಹಾಗೂ ವಿದ್ಯಾರ್ಥಿಗಳು ಬಹುದೊಡ್ಡ ಸಂಖ್ಯೆಯಲ್ಲಿ ಭಾಗವಹಿಸಿದ್ದರು. ಮದರಾಸಿನ ಪ್ರಮುಖ ನಾಗರಿಕರಾದ ಸರ್ ರಾಮಸ್ವಾಮಿ ಮೊದಲಿಯಾರ್, ದಿವಾನ್ ಬಹಾದ್ದೂರ್ ಸುಬ್ರಮಣ್ಯ ಅಯ್ಯರ್ ಮೊದಲಾದವರು ಉಪಸ್ಥಿತರಿದ್ದರು. ರಾಮನಾಡಿನ ರಾಜ ತನ್ನ ಶುಭಾ ಶಯಗಳನ್ನು ಕಳಿಸಿದ್ದ. ಅಲ್ಲದೆ ಇನ್ನೂ ಅನೇಕ ಗಣ್ಯವ್ಯಕ್ತಿಗಳು ಈ ಸಭೆಯ ಬಗ್ಗೆ ತಮ್ಮ ಸಂಪೂರ್ಣ ಬೆಂಬಲವನ್ನೂ ಸಂತೋಷವನ್ನೂ ವ್ಯಕ್ತಪಡಿಸಿ ಪತ್ರಗಳನ್ನು ಬರೆದಿದ್ದರು. ಸಭೆಯ ಅಧ್ಯಕ್ಷರಾಗಿ ಚುನಾಯಿಸಲ್ಪಟ್ಟ ಸುಬ್ರಮಣ್ಯ ಅಯ್ಯರರು ಮಾತನಾಡಿ, ಅಮೆರಿಕೆಯಲ್ಲಿ ಸ್ವಾಮೀಜಿ ಸಾಧಿಸಿದ ಅದ್ಭುತ ಯಶಸ್ಸು, ಹಿಂದೂಗಳಿಗೂ ಅಮೆರಿಕನ್ನರಿಗೂ ಅತ್ಯಂತ ಮಹತ್ವಪೂರ್ಣವಾದು ದೆಂದು ಹೇಳಿದರು. ಬಳಿಕ ಮಾತನಾಡಿದ ಶ್ರೀಪಾರ್ಥಸಾರಥಿ ಅಯ್ಯಂಗಾರರು, “ಹಿಂದೂ ಧರ್ಮದ ತತ್ತ್ವಗಳ ಪರಿಜ್ಞಾನವೇ ಇಲ್ಲದೆ ಪ್ರತಿಯೊಬ್ಬರೂ ಅದನ್ನು ಟೀಕಿಸುತ್ತಿರುವ ಸಂದರ್ಭ ದಲ್ಲಿ, ಹಿಂದೂಧರ್ಮವು ಇತರ ಯಾವುದೇ ಧರ್ಮದೊಂದಿಗೂ ಸರಿಸಾಟಿಯಾಗಿ ನಿಲ್ಲಬಲ್ಲು ದೆಂಬುದನ್ನು ತೋರಿಸಿ ಕೊಟ್ಟ ಯಶಸ್ಸು ಸ್ವಾಮಿ ವಿವೇಕಾನಂದರಿಗೆ ಹಾಗೂ ಅಮೆರಿಕನ್ನರಿಗೆ ಸೇರಿದುದು” ಎಂದು ಹೇಳಿದರು. ಬಳಿಕ, ‘ಶಿಕಾಗೋದ ಸರ್ವಧರ್ಮ ಸಮ್ಮೇಳನದಲ್ಲಿ ಭಾರತ ವನ್ನು ಪ್ರತಿನಿಧಿಸಿದುದಕ್ಕಾಗಿ ಹಾಗೂ ಅಲ್ಲಿ ತಮ್ಮ ಅಪೂರ್ವ ವಾಗ್ವೈಖರಿಯಿಂದ ಹಿಂದೂ ಧರ್ಮವನ್ನು ವ್ಯಾಖ್ಯಾನಿಸಿದುದಕ್ಕಾಗಿ’ ವಿವೇಕಾನಂದರಿಗೆ ಧನ್ಯವಾದಗಳನ್ನರ್ಪಿಸುವ ಠರಾ ವೊಂದನ್ನು ಸಭೆಯು ಅಂಗೀಕರಿಸಿತು. ಅನಂತರ ‘ಸ್ವಾಮಿ ವಿವೇಕಾಂದರಿಗೆ ಸೌಹಾರ್ದಯುತ ಸ್ವಾಗತವನ್ನು ನೀಡಿದುದಕ್ಕಾಗಿ’ ಅಮೆರಿಕದ ಜನರಿಗೆ ಕೃತಜ್ಞತೆಗಳನ್ನು ಅರ್ಪಿಸುವ ಎರಡನೆಯ ನಿರ್ಣಯವನ್ನು ಮಂಡಿಸಿ ಅಂಗೀಕರಿಸಲಾಯಿತು. ಅಲ್ಲದೆ ಸಭಾಧ್ಯಕ್ಷರು ಈ ಎರಡು ನಿರ್ಣಯ ಗಳ ಪ್ರತಿಗಳನ್ನು ಸ್ವಾಮಿ ವಿವೇಕಾನಂದರಿಗೂ ಡಾ ॥ ಬರೋಸ್​ರವರಿಗೆ ಕಳಿಸಿಕೊಡಬೇಕೆಂದೂ ಸಭೆಯಲ್ಲಿ ನಿರ್ಧರಿಸಲಾಯಿತು.

ಹೀಗೆ ಈ ಸಮಾರಂಭ ಅತ್ಯಂತ ಯಶಸ್ವಿಯಾಗಿ ನಡೆಯಿತು. ಆದರೆ ಈ ಹಿಂದೆಯೇ ತಿಳಿಸಿ ದಂತೆ, ಈ ಸಭೆಯ ವಿಷಯವಾಗಲಿ ಅದರ ನಿರ್ಣಯಗಳ ವಿಷಯವಾಗಲಿ ಮೇ, ಜೂನ್ ತಿಂಗಳು ಕಳೆದರೂ ಸ್ವಾಮೀಜಿಗೆ ತಲುಪಲೇ ಇಲ್ಲ. ಆದ್ದರಿಂದ ಭಾರತದಲ್ಲಿ ತಾವು ನಿರೀಕ್ಷಿಸಿದ ಸಭೆ ಜರುಗಲಿಲ್ಲವೆಂದೇ ಅವರು ತಿಳಿಯುವಂತಾಯಿತು. ಮದರಾಸಿನಲ್ಲಿ ನಡೆದಂತಹದೇ ಮತ್ತೊಂದು ಭಾರೀ ಸಭೆ ಕಲ್ಕತ್ತದಲ್ಲೂ ನಡೆಯಿತು. ಇದರ ಹಿಂದಿನ ಮುಖ್ಯ ಸ್ಫೂರ್ತಿಯೆಂದರೆ ಮಹಾಬೋಧಿ ಸಂಸ್ಥೆಯ ಕಾರ್ಯದರ್ಶಿಗಳಾದ ಧರ್ಮಪಾಲರು. ಇವರು ಸರ್ವಧರ್ಮ ಸಮ್ಮೇಳನದಲ್ಲಿ ಸಿಂಹಳದ ಬೌದ್ಧರನ್ನು ಪ್ರತಿನಿಧಿಸಿದ್ದವರು. ಧರ್ಮಪಾಲರು ತಮ್ಮ ವಾಗ್ಮಿತೆ ಯಿಂದಲೂ ಸಂದೇಶಗಳಿಂದಲೂ ಸ್ವತಃ ಸಮ್ಮೇಳನದಲ್ಲಿ ಮನ್ನಣೆಗೆ ಪಾತ್ರರಾಗಿದ್ದರು. ಆದರೆ ಅವರು ಕೂಡ ಸ್ವಾಮೀಜಿಯವರ ಉಪನ್ಯಾಸಗಳನ್ನು ಕೇಳಿ ಬೆರಗಾಗಿದ್ದರಲ್ಲದೆ, ಅವರ ವ್ಯಕ್ತಿತ್ವಕ್ಕೆ ಮಾರುಹೋಗಿದ್ದರು; ಅವರ ಸಂದೇಶಗಳ ಮಹತ್ವವನ್ನು ಮನಗಂಡು ಸ್ಫೂರ್ತಿಗೊಂಡಿದ್ದರು. ಮಜುಮ್​ದಾರನೇ ಮೊದಲಾದ ಇತರ ಭಾರತೀಯ ಪ್ರತಿನಿಧಿಗಳಂತೆ ಧರ್ಮಪಾಲರು ಸ್ವಾಮೀಜಿ ಯವರ ಬಗ್ಗೆ ಮತ್ಸರ ತಾಳಿದವರಲ್ಲ. ಇವರಿಬ್ಬರ ನಡುವೆ ವಿಶ್ವಾಸಯುತ ಸ್ನೇಹವೇರ್ಪಟ್ಟಿತ್ತು. ಭಾರತಕ್ಕೆ ಮರಳಿದ ಮೇಲೆ ಧರ್ಮಪಾಲರು, ಅಮೆರಿಕದಲ್ಲಿ ಸ್ವಾಮೀಜಿ ಮಾಡಿದ ಕಾರ್ಯವನ್ನು ಬಹಳವಾಗಿ ಪ್ರಶಂಸಿಸಿ ಮಾತನಾಡಿದರು. ಅಲ್ಲದೆ ಕಲ್ಕತ್ತದ ಬಳಿಯ ಆಲಂಬಜಾರಿನ ಮಠಕ್ಕೆ ಹೋಗಿ ಸ್ವಾಮೀಜಿಯ ಗುರುಭಾಯಿಗಳನ್ನು ಭೇಟಿ ಮಾಡಿ, ಅಮೆರಿಕದಲ್ಲಿ ಅವರು ಸಾಧಿಸಿದ ಅದ್ಭುತ ಯಶಸ್ಸನ್ನು ಬಣ್ಣಿಸಿ ಕೊಂಡಾಡಿದರು. ಮೇ ೧೪ ರಂದು ಕಲ್ಕತ್ತದ ಮಿನರ್ವ ಥಿಯೇಟರ್ ಎಂಬ ಸಭಾಂಗಣದಲ್ಲಿ ಬಹಿರಂಗ ಸಭೆಯೊಂದನ್ನು ಏರ್ಪಡಿಸಲಾಯಿತು. ಮಹಾರಾಜಾ ಬಹಾ ದ್ದೂರ್ ಸರ್ ನರೇಂದ್ರಕೃಷ್ಣ ಎಂಬ ಕಲ್ಕತ್ತದ ಪ್ರಮುಖರ ಅಧ್ಯಕ್ಷತೆಯಲ್ಲಿ ನಡೆದ ಅಂದಿನ ಸಭೆಯಲ್ಲಿ ಧರ್ಮಪಾಲರು “ಸ್ವಾಮಿ ವಿವೇಕಾನಂದರು ಮತ್ತು ಅಮೆರಿಕದಲ್ಲಿ ಹಿಂದೂಧರ್ಮ” ಎಂಬ ವಿಷಯವಾಗಿ ಮಾತನಾಡಿದರು. ಜಪಾನಿನ ಬೌದ್ಧ ಮಠಾಧಿಪತಿಗಳಾದ ಉಟೋಕಿಯ ವರೂ ಉಪಸ್ಥಿತರಿದ್ದು, ಕೆಲವು ಮಾತುಗಳನ್ನಾಡಿದರು. ಈ ಸಭೆಯ ಬಗ್ಗೆ ‘ಇಂಡಿಯನ್ ಮಿರರ್​’ ಬರೆಯಿತು:

“ಧರ್ಮಪಾಲರು ವಿಷಯವನ್ನು ನಿರೂಪಿಸಿದ ರೀತಿಯು, ಬೌದ್ಧರು ಹಿಂದೂಗಳ ಬಗ್ಗೆ ಹೊಂದಿರುವ ಸದ್ಭಾವನೆ ಹಾಗೂ ಸ್ನೇಹಕ್ಕೆ ಸಾಕ್ಷ್ಯವಾಗಿತ್ತು. ಅಧಿಕ ಸಂಖ್ಯೆಯಲ್ಲಿ ಸೇರಿದ್ದ ಸಭಿಕರು ತಮ್ಮ ಹರ್ಷೋದ್ಗಾರಗಳ ಮೂಲಕ ಭಾಷಣದ ಬಗೆಗೆ ತಮ್ಮ ಸಂತೋಷವನ್ನು ವ್ಯಕ್ತಪಡಿಸಿದರಲ್ಲದೆ, ಹಿಂದೂಧರ್ಮಕ್ಕೆ ವಿವೇಕಾನಂದರು ಸಲ್ಲಿಸಿದ ಸೇವೆಯನ್ನು ತಾವು ಗುರುತಿಸಿದ್ದೇವೆ ಎಂಬುದನ್ನೂ ತೋರಿಸಿಕೊಟ್ಟರು. ಸರ್ವಧರ್ಮ ಸಮ್ಮೇಳನದಲ್ಲಿ ಮತ್ತು ಅಮೆರಿಕದ ಇತರ ಪ್ರಮುಖ ಕೇಂದ್ರಗಳಲ್ಲಿ ಸ್ವಾಮಿ ವಿವೇಕಾನಂದರು ಸಾಧಿಸಿ ಅತ್ಯುಜ್ವಲ ಕಾರ್ಯಗಳನ್ನು ಕೀಳ್ಗೈಯುವ ಅತ್ಯಂತ ಹೇಯ ಪ್ರಯತ್ನ ಕೆಲವು ವಲಯಗಳಲ್ಲಿ ನಡೆದಿತ್ತು. ಇವು ತಮ್ಮ ಹೀನತೆಗೆ ತಕ್ಕಂತೆಯೇ ಹೀನವಾದ ಸೋಲನ್ನೂ ಅಪ್ಪಬೇಕಾಯಿತು. ಈಗ ವಿವೇಕಾ ನಂದರ ಕುರಿತಾದ ಗೌರವಾಭಿಪ್ರಾಯಗಳನ್ನು ಸಾರ್ವಜನಿಕವಾಗಿ, ಇನ್ನು ಸಂದೇಹಕ್ಕೆಡೆಯಿಲ್ಲ ದಂತೆ ಎತ್ತಿಹಿಡಿದಿರುವುದು ನಮಗೆ ಇನ್ನೂ ಹೆಚ್ಚಿನ ಸಂತೋಷವುಂಟುಮಾಡಿದೆ. ಅಲ್ಲದೆ ಅವರ ಘನತೆಯ ಕುರಿತಾಗಿ ಅಯಾಚಿತವಾಗಿ ಹಾಗೂ ಅನಿರೀಕ್ಷಿತವಾಗಿ ದೊರಕಿದ ಬೌದ್ಧರ ಪ್ರಶಂಸೆಯೂ, ಇನ್ನು ಮುಂದೆ ನಡೆಯುವ ಅಂತಹ ದುಷ್ಟ ಪ್ರಯತ್ನಗಳನ್ನು ನಿಷ್ಫಲಗೊಳಿಸುವಂತಾಗಬೇಕು.”

ಅಂತೂ ತಮ್ಮ ಕುರಿತಾಗಿ ನಾಲ್ಕು ಒಳ್ಳೆಯ ಮಾತುಗಳು ಭಾರತದ ಪತ್ರಿಕೆಗಳಲ್ಲಿ ಪ್ರಕಟವಾಗು ವುದನ್ನೇ ನಿರೀಕ್ಷಿಸಿಕೊಂಡು ಕುಳಿತಿದ್ದ ಸ್ವಾಮೀಜಿಗೆ ಸುಮಾರು ಜುಲೈ ೯ರ ವೇಳೆಗೆ ಅಂದಿನ ಸಾರ್ವಜನಿಕ ಸಭೆಯಲ್ಲಿ ಧರ್ಮಪಾಲರು ಮಾಡಿದ ಭಾಷಣದ ವರದಿ ತಲುಪಿತು. ಇದರಿಂದ ಅವರಿಗೆ ಎಷ್ಟು ನಿಶ್ಚಿಂತೆಯಾಯಿತು, ಎಷ್ಟು ಆನಂದವಾಯಿತು ಎನ್ನುವುದು ಅವರು ಹೇಲ್ ಸೋದರಿಯರಿಗೆ ಬರೆದ ಒಂದು ಪತ್ರದಿಂದ ವಿದಿತವಾಗುತ್ತದೆ:

“ಓ ನನ್ನ ಸೋದರಿಯರೇ,

ಜಗದಂಬೆಗೆ ಜಯವಾಗಲಿ! ನಾನು ಜಯಿಸಬೇಕಾದದ್ದನ್ನು ನನ್ನ ನಿರೀಕ್ಷೆಗೂ ಮೀರಿ ಜಯಿಸಿ ದ್ದೇನೆ. ಪ್ರವಾದಿಯನ್ನು ‘ಹಳೇ ಬಾಕಿ ಸಮೇತವಾಗಿ’ ಗೌರವಿಸಲಾಗಿದೆ. ಭಗವಂತನ ಕರುಣೆಯನ್ನು ಕಂಡು ನಾನು ಮಗುವಿನಂತೆ ಅಳುತ್ತಿದ್ದೇನೆ. ಸೋದರಿಯರೇ, ಭಗವಂತ ತನ್ನ ದಾಸನನ್ನೆಂದಿಗೂ ಕೈಬಿಡುವುದಿಲ್ಲ. ಈಗ ನಾನು ನಿಮಗೆ ಕಳಿಸುವ ಪತ್ರ ಎಲ್ಲವನ್ನೂ ವಿವರಿಸುತ್ತದೆ. ಅಲ್ಲದೆ ನನ್ನ ಪರವಾದ ಮುದ್ರಿತ ವಿಷಯಗಳೆಲ್ಲವೂ ಈಗ ಅಮೆರಿಕೆಗೆ ಬರುತ್ತಿವೆ. ಅವುಗಳಲ್ಲಿರುವ ಹೆಸರು ಗಳೆಲ್ಲ (ಅರ್ಥಾತ್, ಆ ವ್ಯಕ್ತಿಗಳೆಲ್ಲ) ನಮ್ಮ ದೇಶದ ರತ್ನಗಳು.

“ಓ ಸೋದರಿಯರೇ, ನಾನು ಭಗವಂತನ ಕೈಯೊಳಗಿದ್ದೇನೆ ಎಂಬುದನ್ನು ಪ್ರತಿಕ್ಷಣದಲ್ಲೂ ನೋಡುತ್ತಿದ್ದರೂ ಕೂಡ ಕೆಲವೊಮ್ಮೆ ನನ್ನ ಶ್ರದ್ಧೆ ಅಲುಗಾಡಿಬಿಡುತ್ತದಲ್ಲ, ಮನಸ್ಸಿಗೆ ಕೆಲವೊಮ್ಮೆ ನಿರಾಶೆ ಕವಿಯುತ್ತದಲ್ಲ, ಎಂಥ ದುಷ್ಟ ನಾನು! ಸೋದರಿಯರೇ, ಜಗತ್ಪಿತನಾದ ಹಾಗೂ ಜಗನ್ಮಾತೆಯಾದ ಭಗವಂತನೊಬ್ಬ ಇರುವುದು ನಿಜ. ಅವನು ತನ್ನ ಮಕ್ಕಳನ್ನೆಂದಿಗೂ ಕೈಬಿಡುವುದಿಲ್ಲ. ಎಂದಿಗೂ ಎಂದಿಗೂ, ಎಂದೆಂದಿಗೂ ಕೈಬಿಡುವುದಿಲ್ಲ. ಅಸಂಬದ್ಧ ವಿಚಾರಗಳ ನ್ನೆಲ್ಲ ಬದಿಗೊತ್ತಿ ಶಿಶುಸಹಜ ಬುದ್ಧಿಯಿಂದ ಅವನಿಗೆ ಶರಣಾಗಿರಿ. ನಾನು ಹೆಚ್ಚೇನನ್ನೂ ಬರೆಯ ಲಾರೆ–ಹೆಂಗಸಿನಂತೆ ಅಳುತ್ತಿದ್ದೇನೆ ನಾನು.

“ಪ್ರಭು, ನನ್ನ ಪ್ರಾಣಪ್ರಭು! ಧನ್ಯ, ನೀ ಧನ್ಯ!”

ಸ್ವಾಮೀಜಿಯ ಈ ಪತ್ರದಲ್ಲಿ, ಅವರ ಮನಸ್ಸಿನ ಹೋರಾಟಗಳನ್ನಲ್ಲದೆ ಅವರ ಶಿಶುಸಹಜ ಸರಳತೆಯನ್ನೂ ಕಾಣಬಹುದು.

ಕಲ್ಕತ್ತದ ಸಭೆಯ ವರದಿ ಬಂದ ಕೆಲವೇ ದಿನಗಳಲ್ಲಿ ಮದರಾಸಿನ ಸಭೆಯ ವರದಿಯೂ ಸ್ವಾಮೀಜಿಯ ಕೈಸೇರಿತು. ಏಪ್ರಿಲ್ ೨೯ರಂದು ನಡೆದಿದ್ದ ಮದರಾಸಿನ ಸಾರ್ವಜನಿಕ ಸಭೆಯ ವರದಿಯನ್ನು ಅಳಸಿಂಗ ಪೆರಮಾಳರು ಸಾಧ್ಯವಾದಷ್ಟು ಬೇಗನೆಯೇ ಕಳಿಸಿದ್ದರೂ, ಅದು ಅವರ ಕೈಸೇರಿದ್ದು ಜುಲೈ ತಿಂಗಳ ಸುಮಾರಿಗೆ, ಈ ವಿಳಂಬಕ್ಕೆ ಕಾರಣವೇನೆಂದರೆ, ಅವರು ಅದನ್ನು ಸ್ವಾಮೀಜಿಯ ‘ಕೇಂದ್ರಸ್ಥಾನ’ವಾದ ಶಿಕಾಗೋದ ಹೇಲ್ ಕುಟುಂಬದವರ ವಿಳಾಸಕ್ಕೆ ಕಳಿಸದೆ, ದೇಶದ ಮತ್ತಾವುದೋ ಭಾಗದ ಅವರ ತಾತ್ಕಾಲಿಕ ವಿಳಾಸಕ್ಕೆ ಕಳಿಸಿದ್ದುದು. ಆದರೆ ಸ್ವಾಮೀಜಿ ಅಮೆರಿಕದಲ್ಲೆಲ್ಲ ಹೆಸರು ಗಳಿಸಿದ್ದುದರಿಂದ, ಆ ಪತ್ರ ಅಮೆರಿಕವನ್ನೆಲ್ಲ ಸುತ್ತಿ, ಕೊನೆಗೂ ಅವರನ್ನು ಮುಟ್ಟಿತು!

ಇದಾದ ಕೆಲದಿನಗಳಲ್ಲೇ ಮದರಾಸಿನ ಸಭೆಯ ಕಲಾಪಗಳ ಹಾಗೂ ಅದರ ಠರಾವುಗಳ ವಿವರ ಗಳು ಅಮೆರಿಕದ ಪ್ರಮುಖ ಪತ್ರಿಕೆಗಳಲ್ಲಿ ಪ್ರಕಟಗೊಂಡು ವ್ಯಾಪಕ ಪ್ರಚಾರವನ್ನು ಗಳಿಸಿದುವು. ವಿವೇಕಾನಂದರಿಗೆ ಅವರ ದೇಶದಲ್ಲೇ ಗೌರವವಿಲ್ಲ, ಮತ್ತು ಅವರು ಬೋಧಿಸುವುದು ಹಿಂದೂ ಧರ್ಮವೇ ಅಲ್ಲ ಎಂಬ ಕ್ರೈಸ್ತ ಪಾದ್ರಿಗಳ ಹಾಗೂ ಬ್ರಾಹ್ಮಸಮಾಜದವರ ಅಪವಾದದ ಮಾತು ಗಳು ಅಪಹಾಸ್ಯಕ್ಕೀಡಾದುವು. ಅಮೆರಿಕದ ಸಮಸ್ತ ಜನತೆಯ ಮನಸ್ಸಿನಲ್ಲಿ ಮೂಡಿದ್ದ ಪೂರ್ವಾ ಗ್ರಹವು ಮಾಯವಾಗಿ, ಸ್ವಾಮೀಜಿಯವರು ಸನಾತನ ಹಿಂದೂಧರ್ಮದ ಶ್ರೇಷ್ಠತಮ ಪ್ರತಿನಿಧಿ ಎಂಬುದು ಪ್ರತಿಷ್ಠಾಪಿತವಾಯಿತು. ಪತ್ರಿಕೆಗಳಲ್ಲಿ ಪ್ರಕಟವಾದ ವರದಿಯನ್ನು ಓದಿ ಸಂತುಷ್ಟರಾಗಿ ಸ್ವಾಮೀಜಿ ಅಳಸಿಂಗರಿಗೆ ಬರೆದರು:

“ಪ್ರೀತಿಯ ಅಳಸಿಂಗ, ಇಲ್ಲಿಯವರೆಗೆ ನೀನು ಮಾಡಿರುವುದು ಅದ್ಭುತವಾಗಿದೆ. ನಾನು ಉದ್ವೇಗಗೊಂಡ ಕ್ಷಣಗಳಲ್ಲಿ ಆಡಿದ ಮಾತುಗಳನ್ನು ಮಸ್ಸಿಗೆ ಹಚ್ಚಿಕೊಳ್ಳಬೇಡ. ಸ್ವದೇಶದಿಂದ ಹದಿನೈದು ಸಾವಿರ ಮೈಲಿ ದೂರವಿರುವ ದೇಶದಲ್ಲಿ ಏಕಾಂಗಿಯಾಗಿದ್ದು, ಕೇಡುಗೈಯಲು ಉದ್ಯುಕ್ತರಾಗಿರುವ ಧರ್ಮಾಂಧ ಕ್ರೈಸ್ತರೊಂದಿಗೆ ಹೆಜ್ಜೆ ಹೆಜ್ಜೆಗೂ ಹೋರಾಡಬೇಕಾಗಿಬರುವಾಗ, ಕೆಲವೊಮ್ಮೆ ಮನಸ್ಸಿನ ಸ್ತಿಮಿತವನ್ನು ಕಳೆದುಕೊಳ್ಳುವಂತಾಗುತ್ತದೆ. ಆದ್ದರಿಂದ, ನನ್ನ ಧೀರಪುತ್ರ, ನೀನು ಇದನ್ನೆಲ್ಲ ಅರ್ಥಮಾಡಿಕೊಂಡು, ಹಿಡಿದ ಕಾರ್ಯವನ್ನು ಬಿಡದೆ ಮುನ್ನಡೆಯಬೇಕು. ಈಗಾ ಗಲೇ ನಿನಗೆ ಅಷ್ಟೊಂದು ಮಾಡಲು ಸಾಧ್ಯವಾಗಿರುವುದಕ್ಕಾಗಿ ಸಂತೋಷಪಡು. ನಿನಗೆ ನಿರಾಶೆ ಕವಿಯುವಂತಾದಾಗ, ಕಳೆದ ಒಂದು ವರ್ಷದೊಳಗಾಗಿ ಏನೇನು ಸಾಧ್ಯವಾಗಿದೆ ಎಂಬುದನ್ನು ನೆನಪಿಸಿಕೊ. ಏನೇನೂ ಇಲ್ಲದ ಪರಿಸ್ಥಿತಿಯಿಂದ ಪ್ರಾರಂಭಿಸಿ, ನಾವು ಹೇಗೆ ಮೇಲೆ ಬಂದಿದ್ದೇವೆ, ಹೇಗೆ ಜಗತ್ತಿನ ಗಮನವನ್ನು ಸೆಳೆಯುವಲ್ಲಿ ಸಮರ್ಥರಾಗಿದ್ದೇವೆ ಎಂಬುದನ್ನು ಆಲೋಚಿಸು! ಭಾರತದ ಮಾತ್ರವೇ ಅಲ್ಲ, ಹೊರಗಡೆಯ ಜಗತ್ತೂ ಕೂಡ ನಮ್ಮಿಂದ ಮಹತ್ಕಾರ್ಯಗಳನ್ನು ನಿರೀಕ್ಷಿಸುತ್ತಿದೆ. ಮಿಷನರಿಗಳು, ಮಜುಮ್​ದಾರ ಅಥವಾ ಮೂರ್ಖ ಅಧಿಕಾರಿಗಳು–ಇವ ರ್ಯಾರೂ ಸತ್ಯ ಪ್ರೇಮ ಪ್ರಾಮಾಣಿಕತೆಗಳನ್ನು ತಡೆಹಿಡಿಯಲಾರರು. ನೀನು ಪ್ರಮಾಣಿಕನೆ? ಕೊನೆಯುಸಿರಿನವರೆಗೂ ನಿಃಸ್ವಾರ್ಥಿಯೆ? ಮತ್ತು ಪ್ರೀತಿಯುಳ್ಳವನೆ? ಹಾಗಾದರೆ ಹೆದರಬೇಡ. ಮೃತ್ಯುವಿಗೂ ಹೆದರಬೇಡ. ಮುನ್ನಡೆಯಿರಿ, ನನ್ನ ಹುಡುಗರೆ! ಇಡೀ ಜಗತ್ತಿಗೆ ಬೆಳಕು ಬೇಕಾಗಿದೆ. ನನ್ನ ಧೀರ ಪುತ್ರರೆ, ನೀವೆಲ್ಲ ಮಹತ್ಕಾರ್ಯಗಳನ್ನು ಸಾಧಿಸಲು ಜನ್ಮವೆತ್ತಿದವರೆಂಬುದರಲ್ಲಿ ಶ್ರದ್ಧೆಯಿಡಿ! ನಾಯಿಕುನ್ನಿಗಳ ಕೂಗಾಟ ನಿಮ್ಮನ್ನು ಅಂಜಿಸದಿರಲಿ. ಅಷ್ಟೇಕೆ, ಮುಗಿಲಿನಿಂದ ಬೀಳುವ ಸಿಡಿಲಿಗೂ ಹೆದರದಿರಿ. ಎದ್ದುನಿಂತು ಕಾರ್ಯೋನ್ಮುಖರಾಗಿ!”

ವೃತ್ತಪತ್ರಿಕೆಗಳ ಮೂಲಕ, ಅಮೆರಿಕದಲ್ಲಿ ಸ್ವಾಮೀಜಿ ಗಳಿಸಿದ ಯಶಸ್ಸಿನ ವಿಷಯ ದೇಶದಾ ದ್ಯಂತ ಪ್ರಸಾರವಾಯಿತು. ಪ್ರತಿಯೊಂದು ಪಟ್ಟಣದಲ್ಲಿಯೂ ಪ್ರತಿಯೊಂದು ಗ್ರಾಮದಲ್ಲಿಯೂ ವಿವೇಕಾನಂದರ ಹೆಸರು ಪ್ರತಿಧ್ವನಿಸಿತು. ಕಲ್ಕತ್ತ, ಮದರಾಸುಗಳಲ್ಲಿ ನಡೆದ ಸಭೆಗಳನ್ನು ಅನು ಸರಿಸಿ, ದೇಶದ ನಾನಾ ಭಾಗಗಳಲ್ಲಿ ಸಮಾರಂಭಗಳು ಜರುಗಿದುವು. ಸ್ವಾಮೀಜಿಗೆ, ಹಾಗೂ ಅವರು ಸಾಧಿಸಿದ ಕಾರ್ಯಗಳಿಗೆ ಎಲ್ಲೆಲ್ಲಿಯೂ ಮಾನ್ಯತೆ ದೊರಕಿತು.

೧೮೯೪ರ ಆಗಸ್ಟ್ ೨೬ರಂದುಬೆಂಗಳೂರಿನಲ್ಲೂ ಇಂಥದೊಂದು ಸಭೆ ಜರುಗಿತು. ಈ ಸಭೆಯ ಕಾರ್ಯಕಲಾಪಗಳ ವರದಿಯು \eng{Bangalore Spectator} ಎಂಬ ಪತ್ರಿಕೆಯಲ್ಲಿ ಹೀಗೆ ಪ್ರಕಟವಾಯಿತು:

“ಸ್ವಾಮಿ ವಿವೇಕಾನಂದರಿಗೆ ಧನ್ಯವಾದಗಳನ್ನರ್ಪಿಸಲು ಸೆಂಟ್ರಲ್ ಕಾಲೇಜಿನ ಉಪನ್ಯಾಸಭವನ ದಲ್ಲಿ ಏರ್ಪಡಿಸಲಾಗಿದ್ದ ಹಿಂದೂ ನಾಗರಿಕರ ಸಭೆ ಬೆಳಗ್ಗೆ ಎಂಟು ಗಂಟೆಗೆ ಪ್ರಾರಂಭವಾಯಿತು. ಸಭೆಯ ಅಧ್ಯಕ್ಷತೆಯನ್ನು ಮೈಸೂರಿನ ದಿವಾನರಾದ ಸರ್ ಕೆ. ಶೇಷಾದ್ರಿ ಅಯ್ಯರ್​ರವರು ವಹಿಸಿದ್ದರು. ಉತ್ಸಾಹೀ ಶ್ರೋತೃಗಳಿಂದ ಸಭೆ ಕಿಕ್ಕಿರಿದಿತ್ತು. ವೇದಿಕೆಯ ಮೇಲೆ ಉಪಸ್ಥಿತರಿದ್ದ ಇತರ ಗಣ್ಯರೆಂದರೆ ಮೈಸೂರು ಮಹಾರಾಜರ ಕೌನ್ಸಿಲರಾದ ಆನರಬಲ್ ಚೆಂಗಲ್ ರಾವ್ ಪಂತುಲುರವರು, ಮೈಸೂರಿನ ಮುಖ್ಯ ನ್ಯಾಯಸ್ಥಾನದ ನ್ಯಾಯಮೂರ್ತಿಗಳಾದ ರಾಮಚಂದ್ರ ಅಯ್ಯರ್, ಜನಗಣತಿಯ ಕಮಿಷನರ್ ಆದ ಶ್ರೀ ನರಸಿಂಹ ಅಯ್ಯಂಗಾರ್, ಮೈಸೂರಿನ ಐ. ಜಿ. ಪಿ.ಯವರಾದ ಶ್ರೀ ಮಹಾದೇವರಾವ್, ಮುಜರಾಯಿ ಇಲಾಖೆಯ ಸೂಪರಿಂಟೆಂಡೆಂಟರಾದ ಶ್ರೀ ಶ್ರೀನಿವಾಸಾಚಾರ್ಲು–ಇವರುಗಳು.

“ಅಧ್ಯಕ್ಷರು ಸಭೆಯನ್ನುದ್ದೇಶಿಸಿ ಸ್ವಾರಸ್ಯಕರವಾದ ಪ್ರಸ್ತಾವನಾ ಭಾಷಣವನ್ನು ಮಾಡಿದ ಬಳಿಕ ಶ್ರೀ ಜಿ. ಜಿ. ನರಸಿಂಹಾಚಾರ್ಯರು ವಿವೇಕಾನಂದರ ಜೀವನದ ಕುರಿತು ತುಂಬ ಉತ್ಸಾಹಪೂರ್ಣ ವಾಗಿ ಒಂದು ದೀರ್ಘ ಭಾಷಣ ಮಾಡಿದರು. ಬಳಿಕ ಶ್ರೀರಾಮಚಂದ್ರ ಅಯ್ಯರ್​ರವರು ಒಂದು ಪುಟ್ಟ ಭಾಷಣಮಾಡಿ, ವಿವೇಕಾನಂದರಿಗೆ ಧನ್ಯವಾದಗಳನ್ನರ್ಪಿಸುವ ಮೊದಲನೆಯ ಠರಾವನ್ನು ಮಂಡಿಸಿದರು. ಅದನ್ನು ಸಭೆ ಸರ್ವಾನುಮತದಿಂದ ಅಂಗೀಕರಿಸಿತು. ಬಳಿಕ ಇತರ ಗಣ್ಯರು ಭಾಷಣಗಳನ್ನು ಮಾಡಿದರು. ಅನಂತರ ವಿವೇಕಾನಂದರಿಗೆ ಆದರದ ಆತಿಥ್ಯವನ್ನು ನೀಡಿದುದ ಕ್ಕಾಗಿ ಅಮೆರಿಕದ ನಾಗರಿಕರಿಗೆ ಕೃತಜ್ಞತೆಗಳನ್ನರ್ಪಿಸುವ ನಿರ್ಣಯವನ್ನು ಮತ್ತು ಈ ಎರಡು ನಿರ್ಣಯಗಳನ್ನು ಸ್ವಾಮಿ ವಿವೇಕಾನಂದರಿಗೂ ಡಾ ॥ ಬರೋಸ್​ರವರಿಗೂ ಕಳಿಸಿಕೊಡಬೇಕೆಂಬ ಮತ್ತೊಂದು ನಿರ್ಣಯವನ್ನು ಸರ್ವಾನುಮತದಿಂದ ಅಂಗೀರಿಸಲಾಯಿತು.”

ಸ್ವಾಮೀಜಿಯ ಪ್ರಚಂಡ ಯಶಸ್ಸು ಭಾರತದಾದ್ಯಂತ ಉತ್ಸಾಹದ ವಾತಾವರಣವನ್ನುಂಟು ಮಾಡಿತು. ಆದರೆ ಕಲ್ಕತ್ತದಲ್ಲಿ ಮಾತ್ರ ಅದು ಸ್ವಲ್ಪ ಹೆಚ್ಚಾಗಿತ್ತು–ಅಥವಾ ಅದು ಭಾವಾವೇಶಕ್ಕೆ ಏರಿಬಿಟ್ಟಿತ್ತೆಂದರೂ ಅತಿಶಯೋಕ್ತಿಯಲ್ಲ. ಕಲ್ಕತ್ತವು ಸ್ವಾಮಿ ವಿವೇಕಾನಂದರ ಜನ್ಮಸ್ಥಳ, ಶ್ರೀರಾಮಕೃಷ್ಣರ ಶಿಷ್ಯವೃಂದದ ಕೇಂದ್ರಸ್ಥಾನ ಎಂಬ ಕಾರಣದೊಂದಿಗೆ ಮತ್ತೊಂದು ವಿಶೇಷವೂ ಇತ್ತು. ಸ್ವಾಮೀಜಿಯ ಮೇಲಿನ ಅಸೂಯೆಯಿಂದಾಗಿ ಪ್ರತಾಪ್​ಚಂದ್ರ ಮಜುಮ್​ದಾರ ಹಾಗೂ ಅವನ ಸಂಗಡಿಗರು ಅವರ ಮೇಲೆ ಅತ್ಯಂತ ಹೀನವಾದ ದೋಷಾರೋಪಣೆಗಳನ್ನು ಮಾಡಿ ಹೆಸರು ಗಳಿಸುವ ನೀಚ ಪ್ರಯತ್ನ ಮಾಡಿದುದು ಇಲ್ಲಿಯೇ ಮತ್ತು ಅವನ ಅಪಪ್ರಚಾರಕ್ಕೆ ಸ್ವಲ್ಪಮಟ್ಟಿನ ಕುಮ್ಮಕ್ಕು ದೊರೆತುದೂ ಇಲ್ಲಿಯೇ. ಆದ್ದರಿಂದ ಸ್ವಾಮೀಜಿಯ ಗೌರವಾರ್ಥವಾಗಿ ಕಲ್ಕತ್ತದಲ್ಲಿ ನಡೆದ ಸಮಾರಂಭದ ವೈಶಿಷ್ಟ್ಯ ಹೆಚ್ಚು.

ಸೆಪ್ಟೆಂಬರ್ ೫ರಂದು ಈ ಬೃಹತ್ ಕಲ್ಕತ್ತಾ ನಗರದಲ್ಲಿ ಪ್ರಮುಖ ಹಿಂದೂ ನಾಗರಿಕರ ಸಹಕಾರದೊಂದಿಗೆ ಶ್ರೀರಾಮಕೃಷ್ಣರ ಸಂನ್ಯಾಸೀಶಿಷ್ಯರು ಭಾರೀ ಬಹಿರಂಗ ಸಭೆಯನ್ನು ಏರ್ಪಡಿಸಿ ದರು. ನಾನಾ ವರ್ಗಗಳ, ನಾನಾ ಅಭಿಪ್ರಾಯಗಳ ನಾಲ್ಕು ಸಾವಿರ ಜನ ಸೇರಿದ್ದರು. ಈ ಸಭೆಯಲ್ಲಿ ಪಂಡಿತರು, ಶ್ರೀಮಂತ ಜಮೀನ್ದಾರರು, ನ್ಯಾಯಾಧೀಶರು, ವಕೀಲರು, ರಾಜಕಾರಣಿಗಳು, ಪ್ರಾಧ್ಯಾಪಕರು ಹಾಗೂ ಇತರ ಹಲವಾರು ಗಣ್ಯ ನಾಗರಿಕರು ಭಾಗವಹಿಸಿದರು. ಈ ಬೃಹತ್ ಸಭೆಯ ಅಧ್ಯಕ್ಷಸ್ಥಾನವನ್ನು ಅಲಂಕರಿಸಿದವರು ರಾಜಾ ಪ್ಯಾರೀ ಮೋಹನ್ ಮುಖರ್ಜಿಯವರು. ಸ್ವಾಮೀಜಿ ಸಾಧಿಸಿದ ಮಹಾಕಾರ್ಯಗಳಿಗಾಗಿ ಅವರಿಗೆ ಅಭಿನಂಧನೆಯನ್ನೂ ಕೃತಜ್ಞತೆಯನ್ನೂ ಅರ್ಪಿಸುವಂತಹ ಸ್ಫೂರ್ತಿಯುತ ಉಜ್ವಲ ಭಾಷಣಗಳನ್ನು ಸಭೆ ವೀಕ್ಷಿಸಿತು. ವೇದಿಕೆಯ ಮೇಲೆ ಕುಳಿತವರಲ್ಲಿ, ಆ ಕಾಲದ ಸುಪ್ರಸಿದ್ಧ ರಾಷ್ಟ್ರೀಯ ಧುರೀಣರೂ ಪ್ರಚಂಡ ವಾಗ್ಮಿಗಳೂ ಆದ ಸರ್ ಸುರೇಂದ್ರನಾಥ ಬ್ಯಾನರ್ಜಿಯವರು ಮತ್ತು ನಾಗೇಂದ್ರನಾಥ ಘೋಷ್​ರವರು ಇದ್ದರು.

ಅಂದಿನ ಸಭೆಯಲ್ಲಿ ಈ ಹಿಂದೆಯೇ ಹೇಳಲಾದ ರೀತಿಯ ಮೂರು ನಿರ್ಣಯಗಳನ್ನು ಅಂಗೀಕರಿಸಲಾಯಿತು. ಈ ನಿರ್ಣಯಗಳೊಂದಿಗೆ ಈ ಕೆಳಗಿನ ಪತ್ರವನ್ನು ಸ್ವಾಮಿ ವಿವೇಕಾ ನಂದರಿಗೆ ಕಳಿಸಿಕೊಡಬೇಕೆಂದು ತೀರ್ಮಾನವಾಯಿತು–

\textbf{ಶ್ರೀಮದ್ ವಿವೇಕಾನಂದ ಸ್ವಾಮಿಗಳವರಿಗೆ}

ಪ್ರಿಯ ಮಹನೀಯರೇ,

ಸೆಪ್ಟೆಂಬರ್ ೫, ೧೮೯೪ರಂದು ಕಲ್ಕತ್ತದ ಪುರಭವನದಲ್ಲಿ ಏರ್ಪಡಿಸಲಾಗಿದ್ದ, ಕಲ್ಕತ್ತ ಹಾಗೂ ಅದರ ಉಪನಗರಗಳಿಗೆ ಸೇರಿದ ಹಿಂದೂ ನಿವಾಸಿಗಳ ಒಂದು ದೊಡ್ಡ ಪ್ರಾತಿನಿಧಿಕ ಹಾಗೂ ಪ್ರಭಾವಶಾಲೀ ಸಭೆಯ ಅಧ್ಯಕ್ಷನಾಗಿ, ನೀವು ಸೆಪ್ಟೆಂಬರ್ ೧೮೯೩ರಲ್ಲಿ ಶಿಕಾಗೋದಲ್ಲಿ ನಡೆದ ಸರ್ವಧರ್ಮ ಸಮ್ಮೇಳನದಲ್ಲಿ ಹಿಂದೂಧರ್ಮವನ್ನು ಸಮರ್ಥವಾಗಿ ಪ್ರತಿನಿಧಿಸಿದುದಕ್ಕಾಗಿ ಸ್ಥಳೀಯ ಹಿಂದೂ ಸಮುದಾಯದ ಕೃತಜ್ಞತೆಗಳನ್ನು ಅರ್ಪಿಸಲು ನನಗೆ ಸಂತೋಷವೆನಿಸುತ್ತದೆ.

“ಹಿಂದೂಧರ್ಮದ ಪ್ರತಿನಿಧಿಯಾಗಿ ಅಮೆರಿಕೆಗೆ ಭೇಟಿ ನೀಡಿ ನೀವು ಎದುರಿಸಿದ ತೊಂದರೆ ಹಾಗೂ ಮಾಡಿದ ತ್ಯಾಗಕ್ಕಾಗಿ, ನಿಮ್ಮಿಂದ ಪ್ರತಿನಿಧೀಕೃತರಾದವರೆಲ್ಲ ನಿಮ್ಮನ್ನು ಹೃದಯಾಂತ ರಾಳದಿಂದ ಅಭಿನಂದಿಸುತ್ತಾರೆ. ಅಲ್ಲದೆ ಅವರಿಗೆ ಅತ್ಯಂತ ಪೂಜ್ಯವಾದ ಅವರ ಆರ್ಯಧರ್ಮದ ಒಳಿತಿಗಾಗಿ ನೀವು ನಿಮ್ಮ ಭಾಷಣಗಳ ಮೂಲಕ ಮತ್ತು ಜಿಜ್ಞಾಸುಗಳ ಪ್ರಶ್ನೆಗಳಿಗೆ ಕೊಡುವ ಸಿದ್ಧ ಉತ್ತರಗಳ ಮೂಲಕ ಸಲ್ಲಿಸಿದ ಸೇವೆಗಾಗಿ ನಿಮಗೆ ಅವರ ವಿಶೇಷ ಕೃತಜ್ಞತೆಗಳು ಸಲ್ಲುತ್ತವೆ. ಹಿಂದೂಧರ್ಮದ ಮೂಲಭೂತ ತತ್ತ್ವಗಳನ್ನು ಕುರಿತ ಯಾವ ವ್ಯಾಖ್ಯಾನವೂ ಒಂದು ಉಪ ನ್ಯಾಸದ ಮಿತಿಯೊಳಗೆ, ನೀವು ೧೮೯೩ರ ಸೆಪ್ಟೆಂಬರ್ ೧೯ರಂದು ಸರ್ವಧರ್ಮ ಸಮ್ಮೇಳನದಲ್ಲಿ ಮಾಡಿದ ಭಾಷಣದಲ್ಲಿ ಹೇಳಿದುದಕ್ಕಿಂತ ಖಚಿತವೂ ಉಜ್ವಲವೂ ಆಗಿರಲು ಸಾಧ್ಯವಿಲ್ಲ. ಅಲ್ಲದೆ ಮುಂದೆ ಬೇರೆ ಬೇರೆ ಸಂದರ್ಭಗಳಲ್ಲಿ ಇದೇ ವಿಷಯದ ಮೇಲಿನ ನಿಮ್ಮ ಹೇಳಿಕೆಗಳು ಅಷ್ಟೇ ಸ್ಪಷ್ಪವೂ ನಿಖರವೂ ಆಗಿವೆ. ಯುಗಯುಗಗಳಿಂದಲೂ ಹಿಂದೂಧರ್ಮವು ತಪ್ಪಾಗಿ ಗ್ರಹಿಸ ಲ್ಪಟ್ಟದ್ದು ಹಾಗೂ ತಪ್ಪಾಗಿ ವಿವರಿಸಲ್ಪಟ್ಟದ್ದು ಹಿಂದೂಗಳ ದೌರ್ಭಾಗ್ಯ. ಆದ್ದರಿಂದ, ಅಪರಿಚಿತ ನಾಡಿನಲ್ಲಿ, ವಿಭಿನ್ನ ಧರ್ಮವನ್ನು ಅನುಸರಿಸುವ ಅಪರಿಚಿತ ಜನರ ಮಧ್ಯದಲ್ಲಿ, ಹಿಂದೂಧರ್ಮದ ಕುರಿತಾದ ಸತ್ಯವನ್ನು ನುಡಿದು ಭ್ರಾಂತಿಯನ್ನು ಹೋಗಲಾಡಿಸುವ ಎದೆಗಾರಿಕೆಯನ್ನೂ ಸಾಮರ್ಥ್ಯ ವನ್ನೂ ವ್ಯಕ್ತಪಡಿಸಿದ ಒಬ್ಬರ ವಿಚಾರದಲ್ಲಿ ಅವರು ವಿಶೇಷ ಕೃತಜ್ಞತೆಯನ್ನು ತಳೆಯದಿರಲು ಸಾಧ್ಯವೇ ಇಲ್ಲ. ಅಲ್ಲದೆ ನಿಮ್ಮನ್ನು ಆದರಿಂದ ಸ್ವೀಕರಿಸಿ, ಮಾತನಾಡಲು ಅವಕಾಶವನ್ನು ಕೊಟ್ಟು, ನಿಮ್ಮ ಕಾರ್ಯದಲ್ಲಿ ನಿಮ್ಮನ್ನು ಉತ್ತೇಜಿಸಿ, ನಿಮ್ಮ ಮಾತುಗಳನ್ನು ಸಹನೆಯಿಂದಲೂ ಸಹೃದತೆ ಯಿಂದಲೂ ಆಲಿಸಿದ ಶ್ರೋತೃಗಳಿಗೂ ಹಿಂದೂಗಳು ಸಮಾನವಾಗಿ ಕೃತಜ್ಞರಾಗಿದ್ದಾರೆ. ಹಿಂದೂ ಧರ್ಮವು ತನ್ನ ಇತಿಹಾಸದಲ್ಲೇ ಮೊಟ್ಟಮೊದಲಬಾರಿಗೆ ಒಬ್ಬ ಧರ್ಮಪ್ರಸಾರಕನನ್ನು (ಮಿಷನರಿ ಯನ್ನು) ಕಂಡುಕೊಂಡಿದೆ. ಅಲ್ಲದೆ ನಿಮ್ಮಷ್ಟು ಸಮರ್ಥರಾದ ಹಾಗೂ ಸಿದ್ಧಹಸ್ತರಾದ ವ್ಯಕ್ತಿ ಯೊಬ್ಬರನ್ನು ಕಂಡುಕೊಂಡದ್ದು ಹಿಂದೂಧರ್ಮದ ಅಪೂರ್ವ ಭಾಗ್ಯ. ತಮ್ಮ ಪ್ರಾಚೀನ ಧರ್ಮದ ಬಗೆಗಿನ ನಿಜವಾದ ಜ್ಞಾನವನ್ನು ಪ್ರಸಾರ ಮಾಡುವಲ್ಲಿ ನೀವು ವಹಿಸಿದ ಸಕಲ ಶ್ರಮ ಕ್ಕಾಗಿ, ನಿಮಗೆ ತಮ್ಮ ಹೃತ್ಪೂರ್ವಕ ಸಹಾನುಭೂತಿಯನ್ನು ಹಾಗೂ ಕಳಕಳಿಯ ಕೃತಜ್ಞತೆಗಳನ್ನು ಸಲ್ಲಿಸದಿದ್ದರೆ ತಮ್ಮಿಂದ ಕರ್ತವ್ಯಲೋಪವಾಗುವುದೆಂದು ನಿಮ್ಮ ದೇಶಬಾಂಧವರು, ಸಹನಾಗರಿ ಕರು ಹಾಗೂ ಹಿಂದೂ ಬಂಧುಗಳು ಭಾವಿಸುತ್ತಾರೆ. ನೀವು ಪ್ರಾರಂಭಿಸಿರುವ ಸತ್ಕಾರ್ಯವನ್ನು ಮುಂದುರಿಸಿಕೊಂಡುಹೋಗಲು ಭಗವಂತನು ನಿಮಗೆ ಶಕ್ತಿ ಉತ್ಸಾಹಗಳನ್ನು ಕರುಣಿಸಲಿ!

\begin{flushright}
ತಮ್ಮ ವಿಧೇಯ,\\\textbf{ಪ್ಯಾರೀ ಮೋಹನ್ ಮಖರ್ಜಿ}\\ಅಧ್ಯಕ್ಷ.
\end{flushright}

ಅಂದಿನ ಸಭೆಯಲ್ಲಿ ಶ್ರೀ ನಾಗೇಂದ್ರನಾಥ ಘೋಷ್, ಸರ್ ಸುರೇಂದ್ರನಾಥ ಬ್ಯಾನರ್ಜಿ ಮೊದಲಾದವರು ಮಾಡಿದ ಭಾಷಣಗಳು ವಿಶೇಷ ಧಾರ್ಮಿಕ ಭಾವನೆಗಳ ಒಂದು ಅಲೆಯನ್ನೇ ಎಬ್ಬಿಸಿದುವೆಂದು ಹೇಳಬಹುದು. ಸನಾತನ ಧರ್ಮದ ಚೈತನ್ಯವೇ ಅಲ್ಲಿ ಅವಿರ್ಭವಿಸಿ ಅಂದಿನ ಭಾಷಣಕಾರರ ಮಾತುಗಳಲ್ಲಿ ನವಸ್ಫೂರ್ತಿಯನ್ನು ಸಂಚಾರಗೊಳಿಸಿದ್ದಂತೆ ಭಾಸವಾಗುತ್ತಿತ್ತು. ರಾಜಾ ಪ್ಯಾರೀಮೋಹನ್ ಮುಖರ್ಜಿಯವರ ಅಧ್ಯಕ್ಷಭಾಷಣದ ಒಂದು ಭಾಗ ಹೀಗಿತ್ತು:

“ಈ ಸಂಜೆ ನಾವೆಲ್ಲ ಇಲ್ಲಿ ಒಬ್ಬ ವ್ಯಕ್ತಿಗೆ ಕೃತಜ್ಞತೆಗಳನ್ನು ಅರ್ಪಿಸಲು ನೆರೆದಿದ್ದೇವೆ. ಆದರೆ ನಾವು ಕೃತಜ್ಞತೆಗಳನ್ನು ಅರ್ಪಿಸುತ್ತಿರುವ ಆ ವ್ಯಕ್ತಿ, ಸರಕಾರಕ್ಕೆ ಶ್ಲಾಘ್ಯವಾದ ಸೇವೆಯನ್ನು ಸಲ್ಲಿಸಿ ಹೆಸರು ಗಳಿಸಿದವರಲ್ಲ, ಅಥವಾ ರಾಜಕಾರಣದಲ್ಲಿ ವಿಜಯಿಗಳಾಗಿ ಪ್ರಖ್ಯಾತರಾದವರೂ ಅಲ್ಲ. ನಾವು ಈ ಮಹಾಸಭೆಯಲ್ಲಿ ಸೇರಿ ನಮ್ಮ ತುಂಬು ಮೆಚ್ಚುಗೆ ಹಾಗೂ ಹೃತ್ಪೂರ್ವಕ ಕೃತಜ್ಞತೆ ಗಳನ್ನು ಸಲ್ಲಿಸುತ್ತಿರುವುದು ಕೇವಲ ಮೂವತ್ತು ವರ್ಷ ವಯಸ್ಸಿನ ಒಬ್ಬ ಸರಳ ಸಂನ್ಯಾಸಿಗೆ. ಅವರು ಅಮೆರಿಕದ ಘನವಂತ ಜನರಿಗೆ ನಮ್ಮ ಧರ್ಮದ ಸತ್ಯಗಳನ್ನು, ಅತ್ಯುನ್ನತ ಪ್ರಶಂಸೆಗೆ ಪಾತ್ರವಾಗಿರುವಂತಹ ತಮ್ಮ ಸಾಮರ್ಥ್ಯದಿಂದ, ಔಚಿತ್ಯಜ್ಞಾನದಿಂದ ಮತ್ತು ಸೂಕ್ಷ್ಮಗ್ರಹಿಕೆ ಯಿಂದ ವ್ಯಾಖ್ಯಾನಿಸಿ ಹೇಳುತ್ತಿದ್ದಾರೆ. ಸೋದರ ವಿವೇಕಾನಂದರು ಹಿಂದೂಧರ್ಮದ ಘನ ಸತ್ಯ ಗಳನ್ನು ವಿವರಿಸುವುದರ ಮೂಲಕ ನಾಗರಿಕ ಜಗತ್ತಿನ ಪ್ರಮುಖ ಭಾಗವೊಂದರ ಕಣ್ಣು ತೆರೆಸಿ ದ್ದಾರೆ. ಮತ್ತು ಮಾನವಬುದ್ಧಿಯ ಅತ್ಯಂತ ಅನರ್ಘ್ಯ, ನಿರ್ಮಿತಿಗಳನ್ನು ಕಾಣಬೇಕಾದದ್ದು ಪಾಶ್ಚಾತ್ಯ ವಿಜ್ಞಾನ ಹಾಗೂ ಸಾಹಿತ್ಯಗಳಲ್ಲಲ್ಲ, ಬದಲಾಗಿ, ನಮ್ಮ ಪ್ರಾಚೀನ ಶಾಸ್ತ್ರಗಳಲ್ಲಿ ಎಂಬು ದನ್ನು ಅವರಿಗೆ ಮನಗಾಣಿಸಿದ್ದಾರೆ... ನಮ್ಮ ದೇಶದ ಇಂತಹ ಉಪಕಾರಿಯನ್ನು ಗೌರವಿಸಲು ಇಷ್ಟು ದೊಡ್ಡ ಪ್ರಭಾವಶಾಲೀ ಸಮೂಹ ಸೇರಿರುವುದನ್ನು ಕಂಡು ನಾನು ಬಹಳ ಸಂತೋಷಗೊಂಡಿದ್ದೇನೆ.”

‘ಇಂಡಿಯನ್ ಮಿರರ್​’ ಪತ್ರಿಕೆಯ ಸಂಪಾದಕರಾದ ಬಾಬು ನರೇಂದ್ರನಾಥ ಸೇನರ ಭಾಷಣ ಹೀಗಿತ್ತು:

“ಈ ನಗರದಲ್ಲಿ ನಡೆಯುವಂತಹ ಈ ರೀತಿಯ ಸಭೆಗಳಲ್ಲೆಲ್ಲ ಇಂದಿನ ಸಭೆ ಅಪೂರ್ವವಾದುದು. ಏಕೆಂದರೆ ನಾವಿಲ್ಲಿ ಸೇರಿರುವುದು ನಾವು ಸಾಮಾನ್ಯವಾಗಿ ಮಾಡುವಂತೆ ಒಬ್ಬ ಉನ್ನತ ಸರಕಾರೀ ಅಧಿಕಾರಿಯನ್ನು ಗೌರವಿಸುವುದಕ್ಕಲ್ಲ. ಬದಲಾಗಿ, ಕಡಲನ್ನು ದಾಟಿ ಹೋಗಿ, ತನ್ನ ವಾಗ್ವೈಖರಿ ಹಾಗೂ ವಿದ್ವತ್ತುಗಳಿಂದ ಹಿಂದೂಗಳ ಪುರೋಭಿವೃದ್ಧಿಗಾಗಿ ಶ್ರಮಿಸಿ ರುವ ಹಿಂದೂ ತಪಸ್ವಿಯೋರ್ವನನ್ನು ಗೌರವಿಸುವುದಕ್ಕಾಗಿ... ಈ ಯಶಸ್ಸು ಹಿಂದೂ ದೇಶೀಯ ರಿಗೆ ಒಂದು ಹೊಸ ಜನ್ಮವನ್ನೇ ನೀಡಿದೆಯೆನ್ನಬೇಕು. ಹಿಂದೂಧರ್ಮದ ಸಮಕಾಲೀನ ಇತಿಹಾಸದ ಕತ್ತಲೆಯ ಪುಟಗಳಲ್ಲಿ ಈ ಯಶಸ್ಸು ಒಂದು ಪ್ರಖರ ಜ್ಯೋತಿಕಿರಣವಾಗಿದೆ; ಹಿಂದೂಗಳು ಹಿಂದೆಂದೂ ಅನುಭವಿಸಿರದಿದ್ದಂತಹ ನವೋತ್ಸಾಹವನ್ನು ತುಂಬಿದೆ. ಈ ಕೆಲವು ಕಾಲದಲ್ಲಿ ನಮ್ಮ ಪರಿಸ್ಥಿತಿಯು ಎಷ್ಟು ಹದಗೆಟ್ಟಿತ್ತೆಂದರೆ, ಹಿಂದೂ ಧರ್ಮವನ್ನು ಎತ್ತಿಹಿಡಿಯಲು ಅಮೆರಿಕೆಯಲ್ಲಿ ಶ್ರಮಿಸುತ್ತಿರುವ ಈ ಪ್ರತಿಭಾಶಾಲಿಯ ಪರಿಶ್ರಮದಿಂದ ದೊರಕಿದ ಯಶಸ್ಸು ನಮ್ಮೆದೆಯ ಕತ್ತಲನ್ನು ಬೆಳಗಿ ಹೊಸ ಭರವಸೆಯನ್ನು ತುಂಬುವವರೆಗೂ ನಾವು ನಿರಾಶೆಯ ಕೂಪದಲ್ಲೇ ಬಿದ್ದಿದ್ದೆವು... 

“ಈ ದೇಶಕ್ಕೆ ವಿವೇಕಾನಂದರು ಸಲ್ಲಿಸಿರುವ ಸೇವೆಯ ಮಹತ್ವವನ್ನು ಎಷ್ಟು ಕೊಂಡಾಡಿದರೂ ಕಡಿಮೆಯೇ. ಅವರಿಗೆ ನಮ್ಮ ಗೌರವ-ಕೃತಜ್ಞತೆಗಳನ್ನು ಸಲ್ಲಿಸುವುದಕ್ಕಾಗಿ ಈ ಸಂಜೆ ನಾವಿಲ್ಲಿ ಸಭೆ ಸೇರದೆಹೋಗಿದ್ದರೆ, ನಾವು ಹಿಂದುಗಳೆಂದು ಕರೆಸಿಕೊಳ್ಳುವುದಕ್ಕೇ ಯೋಗ್ಯರಾಗಿರುತ್ತಿರಲಿಲ್ಲ.”

ಸ್ವಾಮೀಜಿಯವರ ವ್ಯಕ್ತಿತ್ವವನ್ನು ನಾಶಗೈಯುವುದಕ್ಕಾಗಿ ಪ್ರತಾಪ್​ಚಂದ್ರ ಮಜುಮ್​ದಾರ ಏನೇನು ಮಿಥ್ಯಾಪವಾದಗಳನ್ನು ಹರಡಿದ್ದನೋ ಅವೆಲ್ಲವೂ ಎಂತಹ ದುರುದ್ದೇಶಪೂರಿತ ಸುಳ್ಳು ಗಳು ಎಂಬುದು ಅಂದಿನ ಸಭೆಯಲ್ಲಿ ಮತ್ತೊಮ್ಮೆ ಸಾಬೀತಾಯಿತು. ಸಭೆಯನ್ನುದ್ದೇಶಿಸಿ ಮಾತ ನಾಡಿದ ಗಣ್ಯ ಜಮೀನ್ದಾರರಾದ ರಾಯ್ ಜತೀಂದ್ರನಾಥ ಚೌಧರಿ ಹೇಳಿದರು, “ಸರ್ವಧರ್ಮ ಸಮ್ಮೇಳನದಲ್ಲಿ ಹಿಂದೂಧರ್ಮದ ಪ್ರತಿನಿಧಿಗಳಾದವರು ಅದನ್ನು ಸರಿಯಾದ ರೀತಿಯಲ್ಲಿ ಪ್ರತಿನಿಧಿಸಲಿಲ್ಲ ಎಂಬ ಆರೋಪವನ್ನು ನಾವು ಕೇಳಿದ್ದೆವು. ಆದರೆ ಇಂದು ನಡೆಯುತ್ತಿರುವ ಈ ಸಭೆಯು, ಅದೊಂದು ಸುಳ್ಳು ಎಂಬುದಕ್ಕೆ ಸಾಕ್ಷ್ಯವಾಗಿದೆ.”

ಮತ್ತೊಬ್ಬ ಭಾಷಣಕಾರರಾದ ‘ಇಂಡಿಯನ್ ನೇಶನ್​’ ಪತ್ರಿಕೆಯ ಸಂಪಾದಕ ಎನ್. ಎನ್. ಘೋಷ್ ಹೀಗೆ ಮಾತನಾಡಿದರು:

“ಯಾವ ಯಶಸ್ಸೂ ವಿವೇಕಾನಂದರ ಯಶಸ್ಸಿಗಿಂತ ಹೆಚ್ಚು ಶೀಘ್ರವೂ ಉಜ್ವಲವೂ ಆಗಿರಲು ಸಾಧ್ಯವಿಲ್ಲ. ನಿಜಕ್ಕೂ ವಾಕ್​ಸಿದ್ಧಿಯ ಇತಿಹಾಸದಲ್ಲೇ ಇದಕ್ಕಿಂತ ಗಮನಾರ್ಹವಾದುದು ಮತ್ತಾ ವುದೂ ಇರಲಾರದು. ಅನಾಮಧೇಯ ಹಿಂದೂ ಸಂನ್ಯಾಸಿಯೊಬ್ಬನಿದ್ದ; ಆತ ತನ್ನ ಅರೆ ಪೌರ್ವಾತ್ಯ ಉಡುಗೆಯಲ್ಲಿ ಒಂದು ಸಭೆಯನ್ನುದ್ದೇಶಿಸಿ ಮಾತನಾಡಿದ; ಆ ಸಭೆಯಲ್ಲಿ ಮುಕ್ಕಾಲು ವಾಸಿ ಜನಕ್ಕೆ ಆತನ ಹೆಸರನ್ನೂ ಉಚ್ಚರಿಸಲು ಬರುತ್ತಿರಲಿಲ್ಲ; ಆತ ಮಾತನಾಡಿದ ವಿಚಾರವೋ ಅವರ ಆಲೋಚನೆ-ಕಲ್ಪನೆಗಳಿಂದ ಬಲುದೂರ; ಆದರೆ ಆತ ಒಮ್ಮೆಲೇ ಅವರ ಕರತಾಡನ- ಗೌರವಗಳನ್ನು ಗಳಿಸಿಕೊಂಡುಬಿಟ್ಟ! ಭಾಷಣಕರ್ತನ ಯೋಗ್ಯತೆಯೂ ಸಾಧನೆಯೂ ಖಂಡಿತ ವಾಗಿ ಅದ್ಭುತವೂ ಅಶ್ಚರ್ಯಕರವೂ ಆಗಿದ್ದಿರಲೇಬೇಕು, ನಿಜ. ಆದರೆ ಆ ಯಶಸ್ಸಿನಲ್ಲಿ ಸಮ ಪಾಲು, ಆತನನ್ನು ಮೆಚ್ಚಿಕೊಂಡ, ಹುರಿದುಂಬಿಸಿದ, ಮಾತನಾಡಲು ಮತ್ತೆಮತ್ತೆ ಅವಕಾಶವನ್ನು ಕಲ್ಪಿಸಿದ ಜನರಿಗೂ ಸಲ್ಲುತ್ತದೆ ಎಂಬುದನ್ನು ಮರೆಯದಿರೋಣ... ಪ್ರತಿನಿಧಿಗಳೆಲ್ಲರೂ ನಿಯಮಗಳಿಗನುಸಾರವಾಗಿ ಆಹ್ವಾನಿತರಾಗಿದ್ದವರು. ಆದರೆ ವಿವೇಕಾನಂದರಿಗೆ ಆಹ್ವಾನ ವಿರಲಿಲ್ಲ. ಆದ್ದರಿಂದ ಈ ನಿಯಮಗಳ ಕಾರಣವನ್ನೊಡ್ಡಿ ಬಹಳ ಸುಲಭವಾಗಿ ಅವರಿಗೆ ಸಮ್ಮೇಳನಕ್ಕೆ ಪ್ರವೇಶವನ್ನು ನಿರಾಕರಿಸಬಹುದಾಗಿತ್ತು. ಆದರೆ ಡಾ ॥ ಬರೋಸ್​ರವರು ಒಂದು ವಿಶೇಷ ಸೌಜನ್ಯದಿಂದ ಔಪಚಾರಿಕ ನಿಯಮಗಳನ್ನೆಲ್ಲ ಬದಿಗೊತ್ತಿ ಅವರಿಗೆ ಮಾತನಾಡಲು ಅವಕಾಶ ಕೊಟ್ಟರು. ವಿವೇಕಾನಂದರು ಕ್ರೈಸ್ತಧರ್ಮದ ಬಗ್ಗೆ ಅಂತಹ ನಯವಾದ ಮತ್ತು ಮಧುರವಾದ ವಿಚಾರಗಳನ್ನೇನೂ ಹೇಳಲಿಲ್ಲ. ಆಗಾಗ ಕೆಲವು ಬಲವಾದ ಗುದ್ದುಗಳನ್ನೂ ಕೊಟ್ಟರು. ಆದರೂ ಅಮೆರಿಕದ ಸಭಿಕರು ಅವರ ಮಾತುಗಳನ್ನು ಆಸಕ್ತಿಯಿಂದ ಆಲಿಸಿದರು; ಮತ್ತು ಅವರ ಹಿರಿಮೆಯ ಅಂಶಗಳನ್ನು ಹೃತ್ಪೂರ್ವಕವಾಗಿ ಮೆಚ್ಚಿಕೊಂಡರು... “ಅಮೆರಿಕ ದಲ್ಲಿ ವಿವೇಕಾನಂದರ ಸಾಧನೆಗಳು ನಿಜಕ್ಕೂ ಪ್ರಶಂಸನೀಯವೇ. ಆದರೂ ಅವುಗಳನ್ನು ನಾನು ಸಿದ್ಧಿಗಳು ಎಂಬುದಕ್ಕಿಂತ ಭರವಸೆಗಳು ಎಂದು ಪರಿಗಣಿಸುತ್ತೇನೆ. ಏಕೆಂದರೆ ಅವರ ನಿಜವಾದ ಕಾರ್ಯ ಆಗಬೇಕಾಗಿರುವುದು ಭಾರತದಲ್ಲಿ. ನನಗನ್ನಿಸುತ್ತದೆ, ಭಾರತದ ಪುನರುತ್ಥಾನ ಸಾಧ್ಯ ವಾಗುವುದು ರಾಜಕಾರಣದ ಮೂಲಕವಲ್ಲ, ಧರ್ಮದ ಮೂಲಕ ಎಂದು. ಈ ದೇಶದಲ್ಲಿ ರಾಜಕಾರಣವೆಂಬುದು ಒಂದು ಮೇಲ್ಮೈಯ ಅರಿವೆಯಿದ್ದಂತೆ–ಬೇಕೆಂದಾಗ ತೊಡಬಹುದು, ಬೇಡವೆಂದಾಗ ಕಳಚಿಬಿಡಬಹುದು. ಅದಕ್ಕೂ ಜನರಿಗೂ ದೂರದ ಸಂಬಂಧ. ಆದರೆ ಧರ್ಮ ಅವರಿಗೆ ಪ್ರಾಣಪ್ರಾಯವಾದುದು, ಅವರ ರಕ್ತದೊಂದಿಗೇ ಬಂದಿರುವಂಥದು...

“ಇಂದಿನ ಈ ಸಭೆಯ ಸಂಘಟಕರು, ಸಂಘಟನೆಯ ಚಾತುರ್ಯವನ್ನರಿಯದವರು. ಇವರಲ್ಲಿ ಹೆಚ್ಚಿನವರು ಬರಿಗಾಲ ಸಂನ್ಯಾಸಿಗಳು, ಕಾಷಾಯಧಾರಿಗಳು. ರಾಜಕೀಯ ಪ್ರದರ್ಶನಗಳಲ್ಲಿ ಬಳಸಲಾಗುವ ವಿಧಾನಗಳು ಇವರಿಗೆ ತಿಳಿದಿದ್ದರೆ, ಈಗಾಗಲೇ ದೊಡ್ಡಾಗಿರುವ ಈ ಸಭೆ, ಇದಕ್ಕಿಂತರೂ ಹತ್ತು ಪಾಲು ದೊಡ್ಡದಾಗಿರುತ್ತಿತ್ತು.”

ಕಲ್ಕತ್ತದ ಈ ಸಮಾರಂಭದಲ್ಲಿ ಮಾಡಲಾದಂತಹ ಭಾಷಣಗಳು, ಹಿಂದೂ ಸಮಾಜದ ಒಟ್ಟು ಧೋರಣೆಯನ್ನು ಪ್ರತಿನಿಧಿಸುತ್ತಿದ್ದುವು. ಈ ಎಲ್ಲ ಪ್ರಗತಿಪರರಾದ ಹಾಗೂ ದೂರದೃಷ್ಟಿಯವ ರಾದ ಭಾಷಣಕಾರರ ಹಿಂದೆ, ಮುಂದಡಿಯಿಟ್ಟು ನಡೆಯಲು ಕಾತರರಾಗಿದ್ದ ಸಹಸ್ರಾರು ಜನರ ಬೆಂಬಲವಿತ್ತು. ಎಲ್ಲ ಬಗೆಯ ಬದಲಾವಣೆಯನ್ನೂ ಪ್ರಗತಿಯ ಪ್ರವಾಹವನ್ನೂ ತಡೆಗಟ್ಟಿ ನಿಲ್ಲುವ ಜನ ಎಲ್ಲ ಸಮಾಜಗಳಲ್ಲೂ ಇರುವಂತೆಯೇ ಇಲ್ಲಿಯೂ ಇರಲಿಲ್ಲವೆಂದಲ್ಲ. ಆದರೆ ವಿವೇಕಾನಂದರ ಹೆಸರು ಭಾರತದ ಉದ್ದಗಲಕ್ಕೂ ಮೊಳಗಿ, ಅವರನ್ನು ಈ ಯುಗದ ಆವಶ್ಯಕತೆ ಯನ್ನು ಪೂರೈಸಬಂದ ಮಹಾನ್ ಆಚಾರ್ಯನೆಂದು ಸಾರ್ವತ್ರಿಕವಾಗಿ ಗುರುತಿಸಿ ಗೌರವಿಸಲಾ ಯಿತು. ಜಡತ್ವದಾಳಕ್ಕೆ ಬಿದ್ದಿದ್ದ ಹಿಂದೂ ಸ್ಫೂರ್ತಿಯನ್ನು ಸ್ವಾಮೀಜಿ ಬಡಿದೆಬ್ಬಿಸಿದ್ದರು. ಈ ಎಲ್ಲ ಸಭೆಗಳಲ್ಲಿ ಮುಂಬರಲಿರುವ ಭಾರತದ ನವ ಅರುಣೋದಯದ ಸೂಚನೆಯನ್ನು ಕಾಣ ಬಹುದಾಗಿತ್ತು; ಹಿಂದಿನಂತೆ ಭಾರತ ಮತ್ತೊಮ್ಮೆ ತನ್ನ ಶಕ್ತಿ ಸಾಮರ್ಥ್ಯ ವೈಭವಗಳನ್ನು ಸಂಪಾದಿಸಿಕೊಳ್ಳುವತ್ತ ಹೆಜ್ಜೆ ಹಾಕುತ್ತಿರುವುದನ್ನು ಗುರುತಿಸಬಹುದಾಗಿತ್ತು.

ಕಲ್ಕತ್ತದಲ್ಲಿ ನಡೆದ ಸಭೆಯ ಕಾರ್ಯಕಲಾಪಗಳನ್ನು ಹಾಗೂ ಭಾಷಣಗಳನ್ನೊಳಗೊಂಡ ಎರಡು ಸಾವಿರ ಚಿಕ್ಕ ಪುಸ್ತಕಗಳನ್ನು ‘ನ್ಯೂ ಕಲ್ಕತ್ತ ಪ್ರೆಸ್​’ ಎಂಬ ಪ್ರಕಾಶನ ಸಂಸ್ಥೆ ಪ್ರಕಟಿಸಿತು. ಕಾಲಕ್ರಮದಲ್ಲಿ ಅಮೆರಿಕದ ಪತ್ರಿಕೆಗಳಲ್ಲೂ ಪ್ರಕಟಗೊಂಡ ಈ ವರದಿಗಳು ಸ್ವಾಮೀಜಿಗೂ ಸಂತೋಷವನ್ನುಂಟುಮಾಡಿದುವು. ಸಭೆಯಲ್ಲಿ ಮಂಡಿಸಲಾದ ಠರಾವುಗಳು ಹಾಗೂ ಸಭಾಧ್ಯಕ್ಷರ ಪತ್ರವು ಸ್ವಾಮೀಜಿಯ ಹೃದಯದ ದನಿಯನ್ನೇ ಮಾರ್ದನಿಸಿದುವು. ಅಲ್ಲದೆ ಸಭೆಯ ಈ ಎಲ್ಲ ಯಶಸ್ಸಿನ ಹಿಂದೆ ತಮ್ಮ ಗುರುಭಾಯಿಗಳಾದ ಸ್ವಾಮಿ ಅಭೇದಾನಂದರ, ಸ್ವಾಮಿ ರಾಮಕೃಷ್ಣಾ ನಂದರ ಹಾಗೂ ಇನ್ನಿತರರ ನಿರಂತರ ಶ್ರಮವಿತ್ತು ಎಂಬುದನ್ನು ತಿಳಿದ ಮೇಲಂತೂ ಅವರ ಸಂತೋಷ ಹೇಳತೀರದು. ಜೊತೆಗೆ ಅವರಿಗೆ ಕೃತಜ್ಞತೆಯನ್ನು ಸಲ್ಲಿಸಿ ಹೊರಡಿಸಲಾದ ಠರಾವು ಗಳು, ಅವರ ವಿರುದ್ಧ ಇಲ್ಲಸಲ್ಲದ ಅಪಪ್ರಚಾರ ಮಾಡಿಕೊಂಡು ಓಡಾಡುತ್ತಿದ್ದವರ ಬಾಯಿ ಮುಚ್ಚಿಸುವಲ್ಲಿ ಅತ್ಯಂತ ಯಶಸ್ವಿಯಾದುವು. ಆದರೆ ಸ್ವಾಮೀಜಿ ಅಮೆರಿಕದಲ್ಲಿ ಮಾಡುತ್ತಿರುವು ದೆಲ್ಲ ಕೇವಲ ರಾಜಕೀಯ ಉದ್ದೇಶಗಳಿಂದ ಕೂಡಿದುದು ಎಂಬುದು ಇನ್ನು ಕೆಲವರ ಅಪಪ್ರಚಾರ. ಇದರ ಕುರಿತಾಗಿ ಸ್ವಾಮೀಜಿ ಅಳಸಿಂಗ ಪೆರುಮಾಳರಿಗೆ ಹೀಗೆ ಬರೆದರು:

“ಕಲ್ಕತ್ತದಲ್ಲಿ ಪ್ರಕಟಿಸಿದ ನನ್ನ ಉಪನ್ಯಾಸಗಳ ಮತ್ತು ಹೇಳಿಕೆಗಳ ಪುಸ್ತಕಗಳಲ್ಲಿ ನಾನೊಂದು ವಿಷಯವನ್ನು ಗಮಿಸಿದೆ. ಅವುಗಳಲ್ಲಿ ಕೆಲವನ್ನು ರಾಜಕೀಯದ ಬಣ್ಣಕೊಡುವ ರೀತಿಯಲ್ಲಿ ಪ್ರಕಟಿಸಲಾಗಿದೆ. ಆದರೆ ನಾನು ರಾಜಕೀಯ ವ್ಯಕ್ತಿಯೂ ಅಲ್ಲ ಅಥವಾ ರಾಜಕೀಯ ಚಳವಳಿ ಗಾರನೂ ಅಲ್ಲ. ನನ್ನ ಗಮನವೆಲ್ಲ ಆತ್ಮದ ಕಡೆಗೆ. ಅದೊಂದು ಸರಿಯಾಗಿದ್ದರೆ ಉಳಿದುದೆಲ್ಲವೂ ತನಗೆ ತಾನೆ ಸರಿಯಾಗುತ್ತದೆ. ಆದ್ದರಿಂದ ನನ್ನ ಯಾವುದೇ ಬರವಣಿಗೆಗೆ ಅಥವಾ ಹೇಳಿಕೆಗೆ ಸುಳ್ಳುಸುಳ್ಳಾಗಿ ಯಾವುದೇ ರಾಜಕೀಯ ಮಹತ್ವವನ್ನೂ ಅಂಟಿಸಕೂಡದೆಂದು ನೀನು ಈ ಕಲ್ಕತ್ತದ ಜನಕ್ಕೆ ಎಚ್ಚರಿಕೆ ಕೊಡಬೇಕು. ಎಂಥ ಮೂರ್ಖತನ! ರೆವರೆಂಡ್ ಕಾಲೀಚರಣ ಬ್ಯಾನರ್ಜಿಯವರು ಕ್ರಿಶ್ಚಿಯನ್ ಮಿಷನರಿಗಳನ್ನು ಉದ್ದೇಶಿ ಮಾಡಿದ ಭಾಷಣವೊಂದರಲ್ಲಿ ನಾನೊಬ್ಬ ರಾಜಕೀಯ ಪ್ರತಿನಿಧಿಯೆಂದು ಹೇಳಿದುದಾಗಿ ಕೇಳಿದೆ. ಅದನ್ನೇದಾರೂ ಅವರು ಸಾರ್ವಜನಿಕವಾಗಿ ಹೇಳಿದ್ದರೆ, ಈಗ ಅದನ್ನು ಕಲ್ಕತ್ತದ ಯಾವುದಾದರೊಂದು ಪತ್ರಿಕೆಯಲ್ಲಿ ಬರೆದು ಸಾಬೀತು ಪಡಿಸಬೇಕೆಂದೂ ಇಲ್ಲವೆ ಆ ಬುದ್ಧಿಗೇಡಿ ಹೇಳಿಕೆಯನ್ನು ಹಿಂದೆಗೆದುಕೊಳ್ಳ ಬೇಕೆಂದೂ ನನ್ನ ಪರವಾಗಿ ಅವರನ್ನು ಸಾರ್ವಜನಿಕವಾಗಿಯೇ ಕೇಳು. ಹೇಗಿದೆ ನೋಡು ಅವರ ಉಪಾಯ! ನಾನು ಕೆಲವೊಮ್ಮೆ ಕ್ರೈಸ್ತ ಸರಕಾರ ಬಗ್ಗೆ ಪ್ರಾಮಾಣಿಕ ಟೀಕೆ ಮಾಡಿ ಕೆಲವು ಕಟು ಮಾತುಗಳನ್ನಾಡಿದ್ದೇನೆ, ನಿಜ. ಆದರೆ ನಾನು ರಾಜಕೀಯವನ್ನು ಲಕ್ಷಿಸುತ್ತೇನೆಂದಾಗಲಿ ಅಥವಾ ಅಂಥವುಗಳೊಂದಿಗೆ ಯಾವುದೇ ಸಂಬಂಧವನ್ನು ಹೊಂದಿರುವೆನೆಂದಾಗಲಿ ಅದರರ್ಥವಲ್ಲ. ನನ್ನ ಆ ಉಪನ್ಯಾಸಗಳಿಂದ ಕೆಲವು ಆಯ್ದ ಭಾಗಗಳನ್ನು ಮಾತ್ರ ಪ್ರಕಟಿಸಿ ನಾನೊಬ್ಬ ರಾಜಕೀಯ ಪ್ರಚಾರಕ ಎಂದು ಸಿದ್ಧಪಡಿಸುವುದು ಒಳ್ಳೇ ಭೇಷಾದ ಕೆಲಸವೆಂದು ಯಾರು ಭಾವಿಸುತ್ತಾರೋ ಅವರಿಗೆ ನಾನು ಹೇಳುತ್ತೇನೆ, ‘ಭಗವಂತ, ನನ್ನ ಗೆಳೆಯರಿಂದ ನನ್ನನ್ನು ರಕ್ಷಿಸು’ ಎಂದು.

“ನನ್ನ ನಿಂದಕರಿಗೆಲ್ಲ ನಾನು ಕೊಡುವ ಉತ್ತರವೆಂದರೆ ಏಕರೀತಿಯ ಮೌನವೊಂದೇ ಎಂದು ನನ್ನ ಸ್ನೇಹಿತರಿಗೆ ಹೇಳು. ಅವರ ವಿಷಯದಲ್ಲಿ ನಾನೇನಾದರೂ ಮುಯ್ಯಿಗೆ ಮುಯ್ಯಿ ತೀರಿಸಲು ಹೊರಟಲ್ಲಿ ಅದು ನಮ್ಮನ್ನೂ ಅವರ ಮಟ್ಟಕ್ಕೇ ಎಳೆಯುತ್ತದೆ. ಸತ್ಯವು ತನ್ನನ್ನು ತಾನೇ ನೋಡಿ ಕೊಳ್ಳುತ್ತದೆ. ಮತ್ತು ನನಗಾಗಿ ಅವರು ಯಾರೊಂದಿಗೂ ಹೋರಾಡಬೇಕಾಗಿಲ್ಲ ಎಂದು ಅವರಿಗೆ (ಸ್ನೇಹಿತರಿಗೆ) ಹೇಳು. ಅವರಿನ್ನೂ ಬಹಳ ಕಲಿಯಬೇಕಾಗಿದೆ. ಅವರಿನ್ನೂ ಕೇವಲ ಮಕ್ಕಳು. ಈ ಸಾರ್ವಜನಿಕ ಜೀವನದ ಅಸಂಬದ್ಧತೆ, ಈ ವೃತ್ತಪತ್ರಿಕೆಗಳ ವರ್ಣನೆ ನನಗೆ ಸಂಪೂರ್ಣ ಜುಗುಪ್ಸೆ ಯನ್ನುಂಟುಮಾಡಿದೆ. ಹಿಮಾಲಯದ ಪ್ರಶಾಂತತೆಗೆ ಹಿಂದಿರುಗಬೇಕೆಂದು ನಾನು ಹಾತೊರೆಯುತ್ತಿದ್ದೇನೆ.”

ಮದ್ರಾಸು, ಬೆಂಗಳೂರು, ಕಲ್ಕತ್ತಗಳಲ್ಲಿ ಏರ್ಪಡಿಸಲಾದ ಸಭೆಗಳಿಂದ ಸ್ವಾಮೀಜಿಯವರಿಗೆ ಇನ್ನೊಂದು ರೀತಿಯಲ್ಲಿ ಸಹಾಯವಾಯಿತು. ಅಮೆರಿಕದಲ್ಲಿ ಅವರ ಕಾರ್ಯ ಯಶಸ್ವಿಯಾಗಿದೆ ಎಂಬುದನ್ನು ಅವು ದೃಢೀಕರಿಸಿ ದಾಖಲಿಸಿದುವು. ಹೀಗೆ ಅವು ಸ್ವಾಮೀಜಿ ತಮ್ಮ ಕಾರ್ಯವನ್ನು ಮುಂದುವರಿಸಿಕೊಂಡು ಹೋಗಬೇಕಾದುದರ ಆವಶ್ಯಕತೆಯನ್ನು ಗುರುತಿಸಿದುವು. ಭಾರತದ ಲ್ಲಿದ್ದ ಅವರ ಭಕ್ತರು, ಮಿತ್ರರು ಮತ್ತು ವಿಶ್ವಾಸಿಗಳು ಅವರನ್ನು ತಾಯ್ನಾಡಿಗೆ ಹಿಂದಿರುಗುವಂತೆ ಮತ್ತೆಮತ್ತೆ ಕೇಳಿಕೊಳ್ಳುತ್ತಿದ್ದರು. ಇದಕ್ಕೆ ಸಂಬಂಧಿಸಿದಂತೆ ಸ್ವಾಮೀಜಿ ಅಳಸಿಂಗರಿಗೆ ಒಂದು ಪತ್ರ ಬರೆದರು, “ಸೆಕ್ರೆಟರಿ ಸಾಹೇಬರು ನನ್ನ ಕಾರ್ಯಕ್ಷೇತ್ರ ಭಾರತವಾದ್ದರಿಂದ ನಾನು ಭಾರತಕ್ಕೆ ಹಿಂದಿರುಗಬೇಕೆಂದು ಬರೆಯುತ್ತಾರೆ. ಅದರಲ್ಲೇನೂ ಸಂದೇಹವೇ ಇಲ್ಲ. ಆದರೆ ನನ್ನ ಸೋದರ, ನಾವು ಭಾರತದ ಮೇಲೆಲ್ಲ ಬೆಳಕು ಬೀರುವಂತಹ ಜ್ಯೋತಿಯನ್ನು ಹೊತ್ತಿಸಬೇಕಾಗಿದೆ. ಆದ್ದ ರಿಂದ ನಾವು ಅವಸರ ಮಾಡುವುದು ಬೇಡ. ಭಗವಂತನ ಕೃಪೆಯಿಂದ ಎಲ್ಲವೂ ಸಾಧ್ಯವಾಗುತ್ತದೆ.”

ಹೀಗೆ, ಭಾರತಕ್ಕೆ ಹಿಂದಿರುಗಬೇಕೆಂದು ಅನೇಕ ಕೋರಿಕೆಗಳು ಬಂದರೂ ಅವರು ತಮ್ಮ ನಿರ್ಧಾರವನ್ನು ಬದಲಾಯಿಸಲಿಲ್ಲ. ಅವರ ವಿರೋಧಿಗಳ ಅಪಪ್ರಚಾರವಾಗಲಿ ಹಿತಾಕಾಂಕ್ಷಿಗಳ ಸಲಹೆಗಳಾಗಲಿ ಅವರ ನಿರ್ಧಾರವನ್ನು ಕದಲಿಸುವಂತಿರಲಿಲ್ಲ. ತಮ್ಮ ಯೋಜನೆಗಳನ್ನು ಕಾರ್ಯ ರೂಪಕ್ಕೆ ತರುವಲ್ಲಿ ಆರ್ಥಿಕ ನೆರವು ನೀಡಲು ಬಡಭಾರತೀಯರಿಗೆ ಸಾಧ್ಯವಿಲ್ಲ, ಶ್ರೀಮಂತ ಭಾರತೀಯರಿಗೆ ಮನಸ್ಸಿಲ್ಲ ಎಂಬುದನ್ನು ಸ್ವಾಮೀಜಿ ಚೆನ್ನಾಗಿ ಅರಿತಿದ್ದರು. ಆದರೆ ಅಮೆರಿಕದಲ್ಲೇ ಉಳಿಯುವುದರಿಂದ ಅವರು ತಮ್ಮ ಉದ್ದೇಶಿಸಿದ್ಧಿಗಾಗಿ ಬೇಕಾದ ಹಣವನ್ನು ತಾವೇ ಗಳಿಸ ಬಹುದಾಗಿತ್ತು. ಅಲ್ಲದೆ ಪರಾನುಕರಣೆಗೆ ಸಿಲುಕಿದ ಪರಾವಲಂಬೀ ದುರ್ಬಲ ರಾಷ್ಟ್ರವನ್ನು ಮೇಲೆತ್ತಲು ಆ ರಾಷ್ಟ್ರಕ್ಕೆ (ಭಾರತಕ್ಕೆ) ತನ್ನದೇ ಆದ ಆದರ್ಶಗಳಲ್ಲಿ ಶ್ರದ್ಧೆಯನ್ನುಂಟುಮಾಡುವ ಆವಶ್ಯಕತೆಯಿತ್ತು. ಇದನ್ನು ಸಾಧಿಸುವ ಒಂದು ವಿಧಾನವೆಂದರೆ ಆ ಆದರ್ಶಗಳನ್ನು ಮೆಚ್ಚಿ ಗೌರವಿಸಬಲ್ಲ ಪಾಶ್ಚಾತ್ಯ ರಾಷ್ಟ್ರಗಳಲ್ಲಿ ಅವುಗಳ ಪರಿಚಯ ಮಾಡಿಕೊಡುವುದು. ಈ ಕುರಿತಾಗಿ ಸ್ವಾಮೀಜಿ ಶ್ರೀಮತಿ ಬೆಲ್ ಹೇಲ್​ಳಿಗೆ ಬರೆಯುತ್ತಾರೆ:

“ಪ್ರಿಯ ತಾಯಿ,

ತನ್ನ ತಾಯ್ನಾಡವರ ಹೊಗಳಿಕೆಯಿಂದ ಈ ಸೇವಕ ಉಬ್ಬಿಹೋಗಲು ಭಗವಂತ ಅವಕಾಶ ಕೊಡುವುದಿಲ್ಲವೆಂದು ತಿಳಿಯುತ್ತೇನೆ.

“ಒಬ್ಬ ಮನುಷ್ಯನನ್ನು ಹೊಗಳಿಕೆಯಿಂದ ಸುಧಾರಿಸಬಹುದಲ್ಲದೆ ತೆಗಳಿಕೆಯಿಂದಲ್ಲ. ಅಂತೆಯೇ ರಾಷ್ಟ್ರಗಳ ವಿಚಾರದಲ್ಲೂ ಕೂಡ. ಸುಮ್ಮನೆ ಅನಾವಶ್ಯಕವಾಗಿ ಭಾರತವನ್ನು ಎಷ್ಟೊಂದು ಬಯ್ಯಲಾಗಿದೆ ಎಂಬುದನ್ನು ಆಲೋಚಿಸಿ ನೋಡಿ. ಭಾರತೀಯರು ಕ್ರೈಸ್ತರಿಗಾಗಲಿ ಅವರ ಧರ್ಮಕ್ಕಾಗಲಿ ಅಥವಾ ಅವರ ಪ್ರಚಾರಕರಿಗಾಗಲಿ ಎಂದಿಗೂ ಹಾನಿಯುಂಟುಮಾಡಲಿಲ್ಲ. ಅವರು ಯಾವಾಗಲೂ ಎಲ್ಲರೊಂದಿಗೂ ಸ್ನೇಹಭಾವದಿಂದಲೇ ಇದ್ದಾರೆ.

“ಆದ್ದರಿಂದ ನೋಡಿ ತಾಯಿ, ಪರರಾಷ್ಟ್ರವೊಂದು ಭಾರತದ ಬಗ್ಗೆ ಆಡುವ ಪ್ರತಿಯೊಂದು ಒಳ್ಳೆಯ ಮಾತೂ ಭಾರತದಲ್ಲಿ ಬಹಳಷ್ಟು ಒಳಿತು ಮಾಡುವ ಶಕ್ತಿಯನ್ನು ಹೊಂದಿದೆ. ಅಮೆರಿಕನ್ನರು ನನ್ನ ಈ ವಿನಮ್ರಕಾರ್ಯವನ್ನು ಮೆಚ್ಚಿಕೊಳ್ಳುವುದರಿಂದ ಭಾರತಕ್ಕೆ ನಿಜಕ್ಕೂ ಬಹಳಷ್ಟು ಪ್ರಯೋಜನವಾಗಿದೆ. ತುಳಿತಕ್ಕೆ ಸಿಲುಕಿದ ಲಕ್ಷಾವಧಿ ಭಾರತೀಯರನ್ನು ಹಗಲೂ ರಾತ್ರಿ ಬಯ್ಯುವುದನ್ನು ಬಿಟ್ಟು ಅವರಿಗೆ ಕನಿಷ್ಠಪಕ್ಷ ಒಂದು ಒಳ್ಳೆಯ ಮಾತನ್ನು ಇಲ್ಲವೆ ಒಂದು ಒಳ್ಳೆಯ ಭಾವನೆಯನ್ನಾದರೂ ಕಳಿಸಿರಿ. ಪ್ರತಿಯೊಂದು ರಾಷ್ಟ್ರವನ್ನೂ ನಾನು ಕೇಳಿಕೊಳ್ಳುವುದು ಇಷ್ಟೆ–ನಿಮಗೆ ಸಾಧ್ಯವಾದರೆ ಅವರಿಗೆ ಸಹಾಯಮಾಡಿ. ಆಗದಿದ್ದರೆ ಅವರನ್ನು ಬಯ್ಯುವುದನ್ನಾದರೂ ನಿಲ್ಲಿಸಿ.”

ಸ್ವಾಮೀಜಿಯೇ ಅಪೇಕ್ಷಿಸಿದ್ದಂತೆ ಭಾರತದಿಂದ ಅವರ ಕಾರ್ಯದ ಕುರಿತಾಗಿ ಪ್ರಶಂಸಾಪತ್ರ ಗಳು ತಡವಾಗಿಯಾದರೂ, ಬಂದುವು. ಆದರೆ ಒಮ್ಮೆ ಬರಲು ಪ್ರಾರಂಭವಾದದ್ದು ಮಾತ್ರ ನಿಲ್ಲಲೇ ಇಲ್ಲ. ಈ ವೃತ್ತಪತ್ರಿಕೆಗಳ ಪ್ರಶಂಸೆಗಳನ್ನು ಕಂಡುಕಂಡು ಸ್ವಾಮೀಜಿಗೆ ತಲೆ ಚಿಟ್ಟು ಹಿಡಿಯುವಂತಾಗಿಬಿಟ್ಟಿತು. ಆದ್ದರಿಂದ ಅವರು ಅಳಸಿಂಗ ಪೆರುಮಾಳರಿಗೆ, ಈ ವೃತ್ತಪತ್ರಿಕೆ ಗಳನ್ನು ಕಳಿಸುವುದು ಇನ್ನು ಸಾಕು ಎಂದು ಮತ್ತೆ ಮತ್ತೆ ಪತ್ರ ಬರೆದರು. ಈ ಕುರಿತಾಗಿ ಮೇರಿಗೆ ಬರೆದ ಪತ್ರದಲ್ಲಿ ಹೀಗೆ ಹೇಳಿದರು, “ಭಾರತದಿಂದ ಬಂದ ಬಂಡಿಗಟ್ಟಲೆ ವೃತ್ತಪತ್ರಿಕೆಗಳನ್ನು ಕಂಡು ನನಗೆ ತಲೆಬೇನೆಯೇ ಬಂತು. ಆದ್ದರಿಂದ ‘ಮದರ್ ಚರ್ಚ್​’ಳಿಗೂ (ಶ್ರೀಮತಿ ಬೆಲ್ ಹೇಲ್​–ಮೇರಿಯ ತಾಯಿ) ಶ್ರೀಮತಿ ಗರ್ನ್​ಸೇಯವರಿಗೂ ಒಂದೊಂದು ಗಾಡಿ ಪತ್ರಿಕೆಗಳನ್ನು ಕಳಿಸಿಕೊಟ್ಟ ಮೇಲೆ ಭಾರತದ ಸ್ನೇಹಿತರಿಗೆ ‘ಇನ್ನು ಸಾಕಪ್ಪ’ ಎಂದು ಹೇಳಬೇಕಾಯಿತು. ಭಾರತ ದಲ್ಲಿ ನನಗೆ ಸಾಕುಸಾಕಾಗುವಷ್ಟು ಪ್ರಚಾರ ಸಿಕ್ಕಾಯಿತು. ಅಳಸಿಂಗ ಹೇಳುತ್ತಾನೆ, ಭಾರತದಲ್ಲಿ ಪ್ರತಿಯೊಂದು ಹಳ್ಳಿಯೂ ನನ್ನ ವಿಷಯವಾಗಿ ಕೇಳಿದೆ ಎಂದು. ಅಂತೂ ನನ್ನ ಹಿಂದಿನ ಶಾಂತಿ ಕೈಬಿಟ್ಟುಹೋಯಿತು. ಇನ್ನು ಮೇಲೆ ನನಗೆಲ್ಲೂ ವಿಶ್ರಾಂತಿ ಇಲ್ಲ. ಭಾರತದ ಈ ಪತ್ರಗಳೇ ನನ್ನ ಪಾಲಿನ ಮೃತ್ಯುವೆಂಬುದೂ ಖಂಡಿತ.” ಆದರೂ ಈ ವೃತ್ತಪತ್ರಿಕೆಗಳನ್ನು ಕಳಿಸುವುದನ್ನು ಅವರ ಶಿಷ್ಯರು ನಿಲ್ಲಿಸಲೇ ಇಲ್ಲ. ಆದ್ದರಿಂದ ಇದಾದ ಎರಡು ತಿಂಗಳ ನಂತರ ಅವರು ತಮೊಬ್ಬ ಮದರಾಸೀ ಶಿಷ್ಯನಿಗೆ ಸ್ವಲ್ಪ ಖಾರವಾಗಿಯೇ ಬರೆದರು, “ನಾನು ಹಿಂದೆಯೇ ಬರೆದಿದ್ದೇನೆ, ಮತ್ತು ಈಗ ಮತ್ತೆ ಬರೆಯುತ್ತಿದ್ದೇನೆ–ನಾನು ವೃತ್ತಪತ್ರಿಕೆಗಳ ಯಾವುದೇ ಟೀಕೆಗಾಗಲಿ ಪ್ರಶಂಸೆಗಾಗಲಿ ಸ್ವಲ್ಪವೂ ಗಮನಕೊಡುವುದಿಲ್ಲ. ಅವುಗಳನ್ನು ನಾನು ಅಗ್ನಿಗಾಹುತಿ ಮಾಡುತ್ತೇನೆ. ನೀನೂ ಹಾಗೆಯೇ ಮಾಡು. ಈ ಪತ್ರಿಕೆಗಳ ಅಸಂಬದ್ಧ ಮಾತುಗಳಿಗೂ ಟೀಕೆಗಳಿಗೂ ಸ್ವಲ್ಪವೂ ಗಮನ ಕೊಡಬೇಡ.”

ನಿಜಕ್ಕೂ, ಭಾರತದಲ್ಲಿ ನಡೆದ ಸಾರ್ವಜನಿಕ ಸಭೆಗಳ ಸಮಾಚಾರ, ಅಮೆರಿಕದ ಪತ್ರಿಕೆಗಳಲ್ಲಿ ಪ್ರಕಟವಾಗುವ ಮೊದಲೇ ಸ್ವಾಮೀಜಿಗೆ ಹೊಗಳಿಕೆಗಳು ಸಾಕಾಗಿಹೋಗಿದ್ದುವು. ಆಗಸ್ಟ್ ತಿಂಗಳಲ್ಲಿ ಅವರು ಶ್ರೀಮತಿ ಬೆಲ್ ಹೇಲ್​ಳಿಗೆ ಬರೆದಿದ್ದರು:

“ಭಾರತದಲ್ಲಿ ನಾನು ಭಯಂಕರವಾಗಿ ಸಾರ್ವಜನಿಕವಾಗಿಬಿಟ್ಟಿದ್ದೇನೆ; ಜನಜಂಗುಳಿ ನನ್ನನ್ನು ಹಿಂಬಾಲಿಸಿ ನನ್ನ ಪ್ರಾಣ ಹಿಂಡಿಬಿಡುತ್ತದೆ. ಪ್ರತಿಯೊಂದು ಚಟಾಕು ಕೀರ್ತಿಯನ್ನೂ ಒಂದು ಸೇರು ಶಾಂತಿ -ಪಾವಿತ್ರ್ಯಗಳ ಬೆಲೆಯನ್ನು ತೆರುವುದರಿಂದ ಮಾತ್ರವೇ ಕೊಳ್ಳಲು ಸಾಧ್ಯ. ಈ ಹಿಂದೆಂದೂ ನಾನದನ್ನು ಆಲೋಚಿಸಿರಲೇ ಇಲ್ಲ. ಈ ಪ್ರಶಂಶೆಯನ್ನು ಕೇಳಿಕೇಳಿ ನನಗೆ ಸಂಪೂರ್ಣ ಜುಗುಪ್ಸೆಯಾಗಿಬಿಟ್ಟಿದೆ. ನನ್ನ ಬಗೆಗೇ ನನಗೆ ಜುಗುಪ್ಸೆ. ಭಗವಂತ ನನಗೆ ಶಾಂತಿ -ಪಾವಿತ್ರ್ಯಗಳ ದಾರಿಯನ್ನು ತೋರಲಿ. ತಾಯಿ, ನಾನು ನಿಮ್ಮ ಮುಂದೆ ನಿಜವನ್ನು ಹೇಳುತ್ತೇನೆ– ಸ್ಪರ್ಧೆಯೆಂಬ ಪಿಶಾಚಿಯನ್ನು ಶಾಂತ ಹೃದಯದೊಳಕ್ಕೆ ತಲೆಹಾಕಲು ಬಿಡದೆ, ಯಾವನೂ ಸಾರ್ವಜನಿಕ ಜೀವನದ ವಾತಾವರಣದಲ್ಲಿ ಬದುಕಿರಲು ಸಾಧ್ಯವಿಲ್ಲ. ಧರ್ಮದ ವಿಷಯದಲ್ಲಿ ಕೂಡ ಇದು ಸತ್ಯ. ಆದರೆ ಯಾವುದೋ ಒಂದು ಸಿದ್ಧಾಂತವನ್ನು ಪ್ರಚಾರ ಮಾಡಲೆಂದೇ ತರಬೇತುಗೊಂಡವರಿಗೆ ಇದು ಅನುಭವಕ್ಕೆ ಬರುವುದಿಲ್ಲ. ಏಕೆಂದರೆ ಅವರಿಗೆಂದಿಗೂ ಧರ್ಮ ವೆಂದರೇನೆಂದೇ ಗೊತ್ತಿರುವುದಿಲ್ಲ. ಆದರೆ ಯಾರು ಭಗವದಾಕಾಂಕ್ಷಿಗಳೋ ಮತ್ತು ಭೋಗಾ ಕಾಂಕ್ಷಿಗಳಲ್ಲವೋ ಅಂಥವರಿಗೆ ತಕ್ಷಣ ಗೊತ್ತಾಗುತ್ತದೆ–ತಾವು ಸಂಪಾದಿಸುವ ಹೆಸರು ಕೀರ್ತಿ ಗಳ ಪ್ರತಿಯೊಂದು ತುಣುಕು ಕೂಡ ತಮ್ಮ ಪಾವಿತ್ರ್ಯದ ಆಹುತಿ ತೆಗೆದುಕೊಂಡಿರುತ್ತದೆ, ಎಂದು. ಪರಿಪೂರ್ಣ ನಿಸ್ವಾರ್ಥತೆಯ ಆದರ್ಶದಿಂದ, ಹೆಸರು ಕೀರ್ತಿಗಳ ಕಡೆಗಿನ ಸಂಪೂರ್ಣ ನಿರಾಸಕ್ತಿ ಯಿಂದ ಅದು ಬಹಳ ಬಹಳ ದೂರ.”

ಸ್ವಾಮೀಜಿ ತಮ್ಮ ಕೀರ್ತಿಯ ಬಗ್ಗೆ ಎಷ್ಟು ಅನಾಸಕ್ತರಾಗಿದ್ದರೋ, ಅದನ್ನು ಕಂಡು ಮತ್ಸರತಾಳಿ ಅದಕ್ಕೆ ಮಸಿ ಬಳಿಯಲು ಪ್ರಯತ್ನಿಸಿದವರ ವಿಷಯದಲ್ಲೂ ಅಷ್ಟೇ ದ್ವೇಷರಹಿತರಾಗಿ ದ್ದರು. ಮಜುಮ್​ದಾರನ ದುರ್ಬುದ್ಧಿಯಿಂದಾಗಿ ಅವರು ಎಷ್ಟೆಷ್ಟು ಕಷ್ಟವನ್ನನುಭವಿಸಬೇಕಾ ಯಿತು! ಆದರೂ ಅವರು ಅವನ ವಿಚಾರದಲ್ಲಿ ಬೇಸರಿಸಿಕೊಂಡವರಲ್ಲ. ಇದು ಅವರ ಒಂದು ದೈವೀ ಗುಣ. ಶ್ರೀಮತಿ ಬೆಲ್ ಹೇಲ್​ಳಿಗೆ ಬರೆದ ಒಂದು ಪತ್ರದಲ್ಲಿ ಸಾಂದರ್ಭಿಕವಾಗಿ ಹೇಳು ತ್ತಾರೆ, “ಆ ‘ಇಂಟೀರಿಯರ್​’ ಪತ್ರಿಕೆ (ಕ್ರೈಸ್ತರ ಒಂದು ಪತ್ರಿಕೆ) ನನ್ನ ಕೈ ಸೇರಿತು. ಅದರಲ್ಲಿ ಮಜುಮ್​ದಾರನ ಪುಸ್ತಕವನ್ನು ಬಹಳವಾಗಿ ಪ್ರಶಂಸಿಸಿರುವುದನ್ನು ಕಂಡು ನನಗೆ ತುಂಬ ಸಂತೋಷವಾಯಿತು. ಮಜುಮ್​ದಾರ ಒಬ್ಬ ದೊಡ್ಡ ಮನುಷ್ಯ, ಬಹಳ ಒಳ್ಳೆಯವನು. ಆತ ತನ್ನ ಸಹಚರರಿಗಾಗಿ ಬಹಳಷ್ಟು ಮಾಡಿದ್ದಾನೆ.” ಇದೇ ಮಹಾತ್ಮರ ವೈಶಿಷ್ಟ್ಯ!

