
\chapter{ಪ್ರವಾದಿಯ ಪತ್ರಗಳು}

\noindent

೧೮೯೪ ನೆಯ ಇಸವಿಯು ಪಶ್ಚಿಮದಲ್ಲಿ ಸ್ವಾಮೀಜಿಯ ಜೀವನದ ಅತ್ಯಂತ ಹೋರಾಟಮಯ ವರ್ಷವಾಗಿತ್ತೆನ್ನಬಹುದು. ಅಮೆರಿಕದಲ್ಲಿ ಮಾತ್ರವಲ್ಲದೆ ಭಾರತದಲ್ಲೂ ಅವರ ಕಾರ್ಯ ಯೋಜನೆಯು ಭ್ರೂಣಾವಸ್ಥೆಯಿಂದ ಹೊರಬಂದು ಬೆಳೆದ ಅತಿಮುಖ್ಯ ವರ್ಷವೂ ಅದೇ. ಅಮೆರಿಕದಲ್ಲಿ ಸ್ವಾಮೀಜಿ ಹೇಗೆ ದೃಢವಾಗಿ ನೆಲೆಯೂರಿ ನಿಂತರೆಂಬುದನ್ನು ಮತ್ತು ಇನ್ನೂ ಕೆಲ ಕಾಲ ಭಾರತಕ್ಕೆ ಹಿಂದಿರುಗದೆ ಅಮೆರಿಕದಲ್ಲೇ ತಮ್ಮ ಕಾರ್ಯವನ್ನು ಮುಂದುವರಿಸುವ ನಿರ್ಧಾರ ಮಾಡಿದ್ದನ್ನು ನೋಡಿದ್ದೇವೆ. ಆದರೆ ಭಾರತದಲ್ಲಿ ಆಗಬೇಕಾಗಿದ್ದ ಕಾರ್ಯಗಳನ್ನು ಅವರು ಮರೆತರೆಂದಲ್ಲ, ಅಥವಾ ಮುಂದೂಡಿದರೆಂದೂ ಅಲ್ಲ. ಬದಲಾಗಿ ಅವರು, ಅಮೆರಿಕದಲ್ಲಿ ತಮ್ಮ ಕಾರ್ಯಗಳನ್ನು ಮುಂದುವರಿಸುತ್ತಲೇ, ತಮ್ಮ ಆಲೋಚನೆಗಳ ಹಾಗೂ ಮಾತುಗಳ ಶಕ್ತಿಯಿಂದ ಭಾರತದಲ್ಲಿನ ತಮ್ಮ ಉದ್ದೇಶಿತ ಯೋಜನೆಯು ಕಾರ್ಯಾರಂಭಗೊಳ್ಳುವಂತೆ ಮಾಡಿದರು. ಅದನ್ನವರು ಸಾಧಿಸಿದ ರೀತಿ ತುಂಬ ರೋಮಾಂಚಕ.

೧೮೯೩-೯೪ರಲ್ಲಿ ಸ್ವಾಮೀಜಿ ಬರೆದ ಪತ್ರಗಳಲ್ಲಿ ಅವರ ಆಲೋಚನೆಗಳು ಜ್ವಲಂತವಾಗಿರು ವುದನ್ನು ಕಾಣಬಹುದು. ಅದರಲ್ಲೂ ಮುಖ್ಯವಾಗಿ ಅವರು ತಮ್ಮ ಭಾರತೀಯ ಶಿಷ್ಯರಿಗೆ, ಸ್ನೇಹಿತರಿಗೆ ಹಾಗೂ ಸೋದರಸಂನ್ಯಾಸಿಗಳಿಗೆ ಬರೆಯುವ ಪತ್ರಗಳಲ್ಲಿ ಅವರ ಶಕ್ತಿಯು ಅಕ್ಷರಶಃ ಹರಿದುಬಂದುದನ್ನು ಕಾಣಬಹುದು. ತಮ್ಮನ್ನೂ ತಮ್ಮ ಕಾರ್ಯಗಳನ್ನೂ ಧ್ವಂಸಗೊಳಿಸಲು ಟೊಂಕಕಟ್ಟಿನಿಂತ ವಿಚ್ಛಿದ್ರಕಾರಿ ಶಕ್ತಿಗಳಿಂದ ಆವೃತರಾಗಿದ್ದರೂ, ತಮ್ಮ ಗುರುದೇವರು ತಮಗೆ ವಹಿಸಿದ್ದ ಹೊಣೆಗಾರಿಕೆಯು ಅವರ ಮುಂದೆ ಸ್ಫುಟವಾಗಿ ನಿಂತಿತ್ತು. ಅವರ ಈ ಪತ್ರಗಳನ್ನು ಓದುತ್ತಹೋದಂತೆ ಅವರು ಹೇಗೆ ಒಬ್ಬ ಮಹಾಪ್ರವಾದಿಯಾಗಿ ಬೆಳೆದುನಿಂತರು ಎಂಬುದನ್ನು ಗುರುತಿಸಬಹುದು.

ಸ್ವಾಮೀಜಿ ಅಮೆರಿಕೆಗೆ ಹೋಗುವ ದಾರಿಯಲ್ಲಿದ್ದಾಗಲೇ ಅವರ ಮನಸ್ಸು ತಾವು ಅಮೆರಿಕದಲ್ಲಿ ಮಾಡಬೇಕಾದುದರ ಕುರಿತಾಗಿ ಮಾತ್ರವಲ್ಲದೆ ಭಾರತದ ಪುನರುತ್ಥಾನದ ಕುರಿತಾಗಿಯೂ ಆಳ ವಾಗಿ ಚಿಂತಿಸುತ್ತಿತ್ತು. ಜಪಾನಿನ ಪ್ರಗತಿಯನ್ನು ಕಂಡ ತಕ್ಷಣ ಅವರ ಮನಸ್ಸಿಗೆ ಬಂದದ್ದು ಭಾರತ. ಯೊಕೊಹಾಮದಿಂದ ತಮ್ಮ ಮದರಾಸಿನ ಶಿಷ್ಯರಿಗೆ ಬರೆದರು:

“ಬನ್ನಿ ಮುಂದೆ, ಪುರುಷಸಿಂಹರಾಗಿ! ನಿಮ್ಮ ಸಣ್ಣ ಬಿಲಗಳಿಂದ ಹೊರಬಂದು ಸುತ್ತಲಿನ ಜಗತ್ತನ್ನೊಮ್ಮೆ ಕಣ್ಣುಬಿಟ್ಟು ನೋಡಿ. ಇತರ ರಾಷ್ಟ್ರಗಳು ಹೇಗೆ ಪ್ರಗತಿಪಥದಲ್ಲಿವೆ ಎಂಬುದನ್ನು ನೋಡಿ. ನೀವು ಮಾನವರನ್ನು ಪ್ರೀತಿಸುವಿರಾ? ನೀವು ನಿಮ್ಮ ರಾಷ್ಟ್ರವನ್ನು ಪ್ರೀತಿಸುವಿರಾ? ಹಾಗಾದರೆ ಬನ್ನಿ, ನಾವೆಲ್ಲ ಉನ್ನತ ಹಾಗೂ ಶ್ರೇಷ್ಠ ಧ್ಯೇಯಗಳಿಗಾಗಿ ಶ್ರಮಿಸೋಣ. ಹಿಂದಿರುಗಿ ನೋಡದಿರಿ; ಇಲ್ಲ, ನಿಮ್ಮ ಅತ್ಯಂತ ಹತ್ತಿರದ ಪ್ರಿಯ ಬಂಧುಮಿತ್ರರು ಅತ್ತು ಗೋಗರೆದರೂ ಹಿಂದಿರುಗಿ ನೋಡದಿರಿ. ಸದಾ ಮುನ್ನಡೆಯುತ್ತಲೇ ಇರಿ.

“ಭಾರತಕ್ಕಿಂದು ಆಕೆಗಾಗಿ ಬಲಿದಾನವಾಗಬಲ್ಲ ಕನಿಷ್ಠಪಕ್ಷ ಒಂದು ಸಹಸ್ರ ಯುವಪುತ್ರರ ಆವಶ್ಯಕತೆಯಿದೆ. ಆದರೆ ಎಚ್ಚರಿಕೆ, ಒಂದು ಸಹಸ್ರ ಮನುಷ್ಯರು, ಮೃಗಗಳಲ್ಲ! ಬಡವರಿಗೆ ಅನುಕಂಪೆಯನ್ನೂ ಅನ್ನವನ್ನೂ ಕೊಡಬಲ್ಲಂತಹ, ಜನಸಾಮಾನ್ಯರಿಗೆಲ್ಲ ಬೆಳಕನ್ನು ನೀಡುವಂತಹ ನೂತನ ಪರಿಸರವೊಂದನ್ನು ನಿರ್ಮಿಸಲು ಜೀವನ್ಮರಣಗಳನ್ನೂ ಲೆಕ್ಕಿಸದೆ ದುಡಿಯಲಿಚ್ಛಿಸುವ ಯುವಕರು ಬೇಕು. ನಿಮ್ಮ ಪೂರ್ವಿಕರ ದೌರ್ಜನ್ಯದಿಂದಾಗಿ ಮೃಗಸದೃಶರಾಗಿರುವವರನ್ನು ಪುನಃ ಮನುಷ್ಯರನ್ನಾಗಿಸಲು ಕೊನೆಯುಸಿರಿನವರೆಗೆ ಹೋರಾಡಬಲ್ಲ ವ್ಯಕ್ತಿಗಳು ಬೇಕು. ಇಂತಹ ಎಷ್ಟು ಮಂದಿಯನ್ನು ನಿಮ್ಮ ಮದರಾಸು ನಗರ ನೀಡಬಲ್ಲುದು?”

ಅಮೆರಿಕದಲ್ಲಿನ ತಮ್ಮ ಧರ್ಮಪ್ರಸಾರದ ಕಾರ್ಯದಲ್ಲಿ ಮುಳುಗಿದ್ದಾಗಲೂ, ಅನುಕ್ಷಣವೂ ಅವರು ಭಾರತದ ಸ್ಥಿತಿಗತಿಯನ್ನು ಸುಧಾರಿಸುವ ಉಪಾಯಗಳ ಬಗ್ಗೆಯೇ ಚಿಂತಿಸುತ್ತಿದ್ದರು. ಯಾವ ಒಂದು ಹೊಸ ವಸ್ತುವನ್ನು, ದೃಶ್ಯವನ್ನು ಕಂಡಾಗಲೂ ಭಾರತದಲ್ಲಿ ಅವುಗಳನ್ನು ಹೇಗೆ ಅಳವಡಿಸಿಕೊಳ್ಳಬಹುದು ಎಂದು ಚಿಂತಿಸುತ್ತಿದ್ದರು. ಅಮೆರಿಕದಲ್ಲಿ ಅವರು ಕಂಡ ‘ಸುಧಾರಣಾ ಗೃಹ’ವು \eng{(Reformatory)} ರಾಷ್ಟ್ರ ನಿರ್ಮಾಣದ ಬಗ್ಗೆ ಅವರಲ್ಲಿ ಹೊಸ ಭಾವನೆಗಳನ್ನು ಉತ್ತೇಜಿಸಿತು. ಇವುಗಳಲ್ಲಿ ಕೆಲವನ್ನು ಅವರು ಅಳಸಿಂಗರಿಗೆ ಬರೆದು ತಿಳಿಸಿದರು:

“ನಿನ್ನೆ, ಇಲ್ಲಿನ ಮಹಿಳಾ ಕಾರಾಗೃಹದ ಮೇಲ್ವಿಚಾರಕಿಯಾದ ಶ್ರೀಮತಿ ಜಾನ್ಸನ್ ಇಲ್ಲಿಗೆ ಬಂದಿದ್ದರು. ಇಲ್ಲಿ ಇದನ್ನು ಕಾರಾಗೃಹ ಎಂದು ಕರೆಯುವುದಿಲ್ಲ. ಬದಲಾಗಿ ‘ಸುಧಾರಣಾಗೃಹ’ ಎಂದು ಕರೆಯುತ್ತಾರೆ. ಇದು ನಾನು ಅಮೆರಿಕೆಯಲ್ಲಿ ಕಂಡ ಅತ್ಯಂತ ಭವ್ಯವಾದ ವಿಷಯ. ಇಲ್ಲಿನ ಬಂಧಿತರನ್ನು ಎಷ್ಟು ದಯಾಪೂರಿತ ದೃಷ್ಟಿಯಿಂದ ನೋಡಿಕೊಳ್ಳುತ್ತಾರೆ! ಅವರನ್ನೆಲ್ಲ ಹೇಗೆ ಪರಿವರ್ತಿಸಿ ಸಮಾಜಕ್ಕೆ ಉಪಯುಕ್ತ ವ್ಯಕ್ತಿಗಳನ್ನಾಗಿಸಿ ಕಳಿಸಿಕೊಡುತ್ತಾರೆ! ನಿಜಕ್ಕೂ ಇದೆಷ್ಟು ಭವ್ಯ, ಇದೆಷ್ಟು ಸುಂದರ! ಇದನ್ನು ನಂಬಬೇಕಾದರೆ ನೀನು ಇಲ್ಲಿಗೆ ಬಂದು ನೋಡಬೇಕು. ಅಯ್ಯೊ! ನಮ್ಮ ದೇಶದ ಬಡವರನ್ನು, ನಿಮ್ನ ವರ್ಗದವರನ್ನು ನೆನೆಸಿಕೊಂಡಾಗ ನನಗೆಷ್ಟು ಸಂಕಟವಾಯಿತು ಬಲ್ಲೆಯಾ? ಉದ್ಧಾರವಾಗಲು ಅವರಿಗೆ ಅವಕಾಶವೇ ಇಲ್ಲ. ಮೇಲೇರಲು ಅವರಿಗೆ ಮಾರ್ಗವೇ ಇಲ್ಲ. ಭಾರತದಲ್ಲಿ ಬಡವರಿಗೆ, ಪಾಪಿಗಳಿಗೆ ಸ್ನೇಹಿತರಿಲ್ಲ, ಸಹಾಯವಿಲ್ಲ. ಅವರೆಷ್ಟೇ ಯತ್ನಿಸಿದರೂ ತಲೆಯೆತ್ತಲಾರರು. ಅವರು ದಿನದಿನವೂ ಕೆಳಕೆಳಕ್ಕೆ ಕುಸಿದುಹೋಗು ತ್ತಿದ್ದಾರೆ. ಕ್ರೂರ ಸಮಾಜವು ತಮ್ಮ ಮೇಲೆ ಬೀಸುತ್ತಿರುವ ಪ್ರಹಾರಗಳನ್ನು ಅವರು ತಿನ್ನುತ್ತಲೇ ಇರುತ್ತಾರೆ; ಆದರೆ ಆ ಪ್ರಹಾರಗಳು ಎಲ್ಲಿಂದ ಬೀಳುತ್ತಿವೆ ಎಂಬುದು ಮಾತ್ರ ಅವರಿಗೆ ತಿಳಿಯದು. ತಾವು ಮನುಷ್ಯರು ಎಂಬುದನ್ನೇ ಅವರು ಮರೆತುಬಿಟ್ಟಿದ್ದಾರೆ. ಇದರ ಪರಿಣಾಮವೇ ನಾವು ಕಾಣುವ ದಾಸ್ಯ! ಈಚಿನ ಕೆಲವರ್ಷಗಳಲ್ಲಿ ಚಿಂತನಶೀಲ ವ್ಯಕ್ತಿಗಳು ಇದನ್ನು ಕಂಡುಕೊಂಡಿ ದ್ದಾರೆ. ಆದರೆ ದುರದೃಷ್ಟವಶಾತ್ ಇದಕ್ಕೆಲ್ಲ ಅವರು ಹಿಂದೂಧರ್ಮವನ್ನೇ ಹೊಣೆಯಾಗಿಸಿ ದ್ದಾರೆ. ಈ ಜನರ ದೃಷ್ಟಿಗೆ ಸುಧಾರಣೆಯ ಏಕೈಕ ಮಾರ್ಗವೆಂದರೆ ಜಗತ್ತಿನ ಅತಿ ಶ್ರೇಷ್ಠ ಧರ್ಮ ವನ್ನು (ಹಿಂದೂಧರ್ಮವನ್ನು) ನಿರ್ನಾಮಗೊಳಿಸುವುದು. ಗೆಳೆಯ, ಇಲ್ಲಿ ಕೇಳು, ಭಗವಂತನ ಕೃಪೆಯಿಂದ ನಾನಿದರ ರಹಸ್ಯವನ್ನು ಕಂಡುಕೊಂಡಿದ್ದೇನೆ. ಇದು ಧರ್ಮದ ತಪ್ಪಲ್ಲ, ಹಾಗೆ ನೋಡಿದರೆ ನಿಮ್ಮ ಧರ್ಮ ಹೇಳುತ್ತದೆ, ಪ್ರತಿಯೊಂದು ಜೀವಿಯಲ್ಲೂಇರುವುದು ನೀನೇ ಎಂದು. ತಪ್ಪು ಯಾವುದೆಂದರೆ ಧರ್ಮವನ್ನು ಅನುಷ್ಠಾನಕ್ಕೆ ತಾರದಿರುವುದು, ಅನುಕಂಪೆಯಿಲ್ಲದಿರು ವುದು, ಹೃದಯವಂತಿಕೆಯಿಲ್ಲದಿರುವುದು. ಭಗವಂತನೇ ಬುದ್ಧನ ರೂಪದಲ್ಲಿ ಅವತರಿಸಿ, ಬಡವರಿಗಾಗಿ ಆರ್ತರಿಗಾಗಿ ಪಾಪಿಗಳಿಗಾಗಿ ಮರುಗುವುದು ಹೇಗೆ, ಅನುಕಂಪೆ ತಾಳುವುದು ಹೇಗೆ ಎಂಬುದನ್ನು ಬೋಧಿಸಿದ. ಆದರೆ ನೀವದಕ್ಕೆ ಕಿವಿಗೊಡಲಿಲ್ಲ. ಬದಲಾಗಿ ಬುದ್ಧನು ಅವತರಿಸಿ ದುದು ವಿಕೃತ ತತ್ತ್ವಗಳನ್ನು ಬೋಧಿಸಿ ಜನರ ಬುದ್ಧಿಪಲ್ಲಪಟಗೊಳಿಸುವುದಕ್ಕಾಗಿ ಎಂದು ನಿಮ್ಮ ಪುರೋಹಿತರು ಭಯಂಕರ ಕತೆಗಳನ್ನು ಕಟ್ಟಿದರು! ನಿಜ, ನಿಜ. ಆದರೆ ಅಸುರರು ಯಾರೆಂದರೆ ನಾವೇ ಹೊರತು ಈ ತತ್ತ್ವಗಳನ್ನು ನಂಬಿದವರಲ್ಲ. ಕ್ರಿಸ್ತನನ್ನು ಅಲ್ಲಗಳೆದ ಯಹೂದ್ಯರು ಅಂದಿನಿಂದಲೂ ನಿರ್ಗತಿಕರಂತೆ ಭೂಮಿಯ ಮೇಲೆಲ್ಲ ಅಲೆದಾಡುತ್ತ ಎಲ್ಲರಿಂದಲೂ ದಬ್ಬಾಳಿಕೆ ಗೊಳಗಾದಂತೆಯೇ ನೀವು ಕೂಡ ನಿಮ್ಮನ್ನು ಆಳಬಯಸುವ ಯಾವುದೇ ರಾಷ್ಟ್ರಕ್ಕೆ ಗುಲಾಮರಾಗಿ ದ್ದೀರಿ. ಆಹ್, ಪ್ರಜಾಪೀಡಕರಿರಾ! ಪ್ರಜಾಪೀಡನೆಯೂ ಗುಲಾಮಗಿರಿಯೂ ಒಂದೇ ನಾಣ್ಯದ ಎರಡು ಮುಖಗಳೆಂಬುದನ್ನು ನೀವು ಅರಿತಿಲ್ಲ. ‘ಪ್ರಜಾಪೀಡಕ’ ‘ಗುಲಾಮ’–ಎರಡೂ ಪರ್ಯಾಯ ಪದಗಳು.”

ಭಾರತದ ಪುನರ್ನಿರ್ಮಾಣವಾಗಬೇಕೆಂದರೆ, ಕೆಳವರ್ಗದ ಲಕ್ಷಾಂತರ ಜನರಿಗೆ ಸರ್ವತೋ ಮುಖ ಬೆಳವಣಿಗೆಯನ್ನು ಸಾಧಿಸಲು ಅವಕಾಶವನ್ನೂ ನೆರವನ್ನೂ ಕಲ್ಪಿಸಬೇಕೆಂಬುದು ಸ್ವಾಮೀಜಿಯ ಯೋಜನೆಯಾಗಿತ್ತು. ‘ಬಡವರಿಗಾಗಿ, ಆರ್ತರಿಗಾಗಿ ಮರುಗುವುದು ಹೇಗೆಂಬು ದನ್ನು ಬುದ್ಧನು ತೋರಿಸಿಕೊಟ್ಟ’ ಎಂದು ಹೇಳಿದ ಅವರ ಮೂಲಕವೂ ಅದೇ ಬುದ್ಧನ ಚೇತನವೇ ಕೆಲಸ ಮಾಡುತ್ತಿತ್ತು. ಈ ಇಬ್ಬರು ಪ್ರವಾದಿಗಳ ಪ್ರಕಾರವೂ ಆಧ್ಯಾತ್ಮಿಕ ಮುಕ್ತಿಯೇ ಪ್ರತಿ ಯೊಂದು ಜೀವಿಯ ಅಂತಿಮ ಗುರಿ. ಇಂತಹ ಆಧ್ಯಾತ್ಮಿಕತೆಯು ಭಾರತದಲ್ಲಿ ಪುನರ್ಜಾಗೃತವಾಗಿ ಇಡೀ ಜಗತ್ತಿಗೇ ದಾರಿದೀಪವಾಗಬಲ್ಲುದು, ಆಗುತ್ತದೆ ಎಂದು ಸ್ವಾಮೀಜಿ ನಂಬಿದ್ದರು. ಆದ್ದರಿಂದ ಅವರು ಅದೇ ಪತ್ರದಲ್ಲಿ ಮತ್ತೆ ಬರೆದರು:

“ಪಾವಿತ್ರ್ಯದಿಂದ ಪ್ರಜ್ವಲಿಸುತ್ತಿರುವ, ಭಗವಂತನ ಮೇಲಿನ ಆವಿಚಲ ಶ್ರದ್ಧೆಯಿಂದ ಶಕ್ತ ರಾದ, ಬಡವರ ಹಾಗೂ ಕೆಳಗೆ ಬಿದ್ದವರ ಮೇಲಿನ ಅನುಕಂಪೆಯೊಂದಿಗೆ ಸಿಂಹದ ಧೈರ್ಯವನ್ನು ಹೊಂದಿರುವಂತಹ ಒಂದು ನೂರುಸಹಸ್ರ ಸ್ತ್ರೀ-ಪುರುಷರು ಮುಕ್ತಿಯ ತತ್ತ್ವವನ್ನು ಬೋಧಿಸುತ್ತ, ಪರಸ್ಪರ ಸಹಾಯದ ಹಾಗೂ ಸಾಮಾಜಿಕ ಜಾಗೃತಿಯ ಮತ್ತು ಸಮಾನತೆಯ ತತ್ತ್ವವನ್ನು ತಿಳಿಸಿ ಕೊಡುತ್ತ, ನಾಡಿನ ಉದ್ದಗಲಕ್ಕೂ ಸಂಚರಿಸುವಂತಾಗಬೇಕು.

“ಈ ಜಗತ್ತಿನ ಯಾವುದೇ ಇತರ ಧರ್ಮವೂ ಮಾನವತೆಯ ಹಿರಿಮೆಯನ್ನು ಹಿಂದೂಧರ್ಮ ದಷ್ಟು ಉದಾತ್ತವಾಗಿ ಸ್ಪಷ್ಟವಾಗಿ ಬೋಧಿಸುವುದಿಲ್ಲ. ಮತ್ತು ದೀನದಲಿತರ ಕುತ್ತಿಗೆಯನ್ನು ಮೆಟ್ಟಿ ನಿಲ್ಲುವ ಧರ್ಮವೂ ಹಿಂದೂಧರ್ಮದಂತೆ ಮತ್ತೊಂದಿಲ್ಲ. ಆದರೆ ಇದಾವುದೂ ಧರ್ಮದ ತಪ್ಪಲ್ಲ ಎಂಬುದನ್ನು ಭಗವಂತ ನನಗೆ ತೋರಿಸಿಕೊಟ್ಟಿದ್ದಾನೆ. ಪಾರಮಾರ್ಥಿಕ ಹಾಗೂ ವ್ಯಾವಹಾರಿಕ ತತ್ತ್ವಗಳ ಹೆಸರಿನಲ್ಲಿ ದೌರ್ಜನ್ಯದ ಷಡ್ಯಂತ್ರಗಳನ್ನು ಕಂಡುಹಿಡಿದವರು ಹಿಂದೂ ಧರ್ಮದಲ್ಲಿರುವ ಆಷಾಢಭೂತಿಗಳು.”

ಇದೇ ಪತ್ರದಲ್ಲಿ ಸ್ವಾಮೀಜಿ ತಮ್ಮ ಶಿಷ್ಯರನ್ನು ಕಾರ್ಯೋನ್ಮುಖಗೊಳಿಸಲು ಪ್ರಚೋದನೆ ಯನ್ನೂ ಮಾರ್ಗದರ್ಶನವನ್ನೂ ನೀಡುತ್ತ ಹೀಗೆ ಬರೆದರು:

“ಶ್ರೀಮಂತರೆನ್ನಿಸಿಕೊಂಡವರಿಂದ ಹೆಚ್ಚಿನದೇನನ್ನೂ ನಿರೀಕ್ಷಿಸಬೇಡಿ. ಅವರು ಬದುಕಿದ್ದಾರೆ ಎನ್ನುವುದಕ್ಕಿಂತ ಹೆಚ್ಚಾಗಿ ಮೃತಪ್ರಾಯರಾಗಿದ್ದಾರೆಂದರೆ ಸರಿಯಾದೀತು. ಭರವಸೆಯಿರುವುದು ನಿಮ್ಮಲ್ಲಿ–ಬಡವರಾದರೂ ಪ್ರಾಮಾಣಿಕರಾಗಿರುವ ನಿಮ್ಮಲ್ಲಿ. ಭಗವಂತನಲ್ಲಿ ಶ್ರದ್ಧೆಯಿಡಿ. ಯೋಜನೆಗಳನ್ನು ಹೂಡುತ್ತ ಕುಳಿತಿರಬೇಡಿ, ಅದರಿಂದೇನೂ ಪ್ರಯೋಜನವಿಲ್ಲ. ಆರ್ತರಿಗಾಗಿ ಮರುಗುತ್ತ ಭಗವಂತನ ನೆರವನ್ನು ಇದಿರುನೋಡಿ–ಅದು ಬಂದೇಬರುತ್ತದೆ. ಈ ಭಾರವನ್ನು ಎದೆಯಲ್ಲಿ ಹೊತ್ತು, ಈ ಆಲೋಚನೆಯನ್ನು ತಲೆಯಲ್ಲಿ ಹೊತ್ತು ನಾನು ಹನ್ನೆರಡು ವರ್ಷ ಸುತ್ತಾಡಿದ್ದೇನೆ. ಶ್ರೀಮಂತರು, ದೊಡ್ಡವರು ಎನ್ನಿಸಿಕೊಂಡವರ ಬಾಗಿಲಿಂದ ಬಾಗಿಲಿಗೆ ಎಡತಾಕಿ ದ್ದೇನೆ. ನಾನು ರಕ್ತಬಸಿವ ಹೃದಯದಿಂದ ಅರ್ಧ ಭೂಮಂಡಲವನ್ನು ಸುತ್ತಿ ಈ ಅಪರಿಚಿತನಾಡಿಗೆ ಸಹಾಯವನ್ನರಿಸಿ ಬಂದೆ. ದೇವರು ದೊಡ್ಡವ, ಅವನು ನನಗೆ ನೆರವಾಗುವನೆಂದು ನಾ ಬಲ್ಲೆ. ಹಸಿವು ಚಳಿಗಳಿಂದ ನಾನಿಲ್ಲಿ ಸತ್ತೇ ಹೋಗಬಹುದು; ಆದರೆ ಓ ನನ್ನ ಯುವಜನರೆ, ಈ ನನ್ನ ಅನುಕಂಪೆಯನ್ನು, ಬಡವ-ದೀನ-ದಲಿತರ ಪರವಾದ ಹೋರಾಟವನ್ನು ನಿಮಗೊಪ್ಪಿಸಿಹೋಗು ತ್ತೇನೆ. ದಿನದಿನಕ್ಕೂ ಅಧಃಪಾತಾಳಕ್ಕೆ ಕುಸಿಯುತ್ತಿರುವ ಮುನ್ನೂರು ಮಿಲಿಯ ಮಾನವರ ಉದ್ಧಾರಕ್ಕಾಗಿ ನಿಮ್ಮ ಇಡೀ ಜೀವನವನ್ನು ಮುಡಿಪಾಗಿಡಲು ವ್ರತಧಾರಣೆ ಮಾಡಿ.”

ಭಾರತದಲ್ಲಿನ ತಮ್ಮ ಯೋಜನೆಗಳನ್ನು ಕಾರ್ಯರೂಪಕ್ಕೆ ತಂದು, ಅವು ಗುರಿಸೇರುವವರೆಗೂ ನಿಲ್ಲುವಂತಾಗಬಾರದೆಂದು ಅವರು ಬಯಸಿದ್ದರು. ಆದ್ದರಿಂದ ಅವರು ತಮ್ಮ ಆದೇಶಗಳನ್ನು ಮಧುರವಾಗಿಸಲು ಹೋಗಲಿಲ್ಲ. ಅವರ ಆಜ್ಞೆ ಸಾಕಷ್ಟು ಕಠಿಣವಾಗಿಯೇ ಇತ್ತು. ಏಕೆಂದರೆ ಅವರ ದೃಷ್ಟಿಯಲ್ಲಿ ಭಾರತದ ಪುನರುದ್ಧಾರಕ್ಕೆ ಅರೆಬರೆ ತ್ಯಾಗವೆಲ್ಲ ಸಾಕಾಗಲಾರದು. ಆದ್ದರಿಂದ ಅದೇ ಪತ್ರದಲ್ಲಿ ಮತ್ತೆ ಹೀಗೆ ಬರೆದರು:

“ಅದು ಒಂದು ದಿನದಲ್ಲಿ ಆಗುವ ಕೆಲಸವಲ್ಲ. ಮತ್ತು ಪಥವು ಅತ್ಯಂತ ಭಯಾನಕ ಮುಳ್ಳು ಗಳಿಂದ ತುಂಬಿದೆ. ಆದರೆ ಸ್ವಯಂ ಪಾರ್ಥಸಾರಥಿಯೇ ನಮ್ಮ ಸಾರಥಿಯಾಗಿರಲು ಸಿದ್ಧನಿದ್ದಾನೆ –ಇದು ಖಂಡಿತ. ಅವನಲ್ಲಿ ಅವಿಚಲ ಶ್ರದ್ಧೆಯಿಟ್ಟು ಅವನ ಹೆಸರಿನಲ್ಲಿ, ಯುಗಯುಗಳಿಂದಲೂ ಭಾರತದ ಮೇಲೆ ಪೇರಿಕೊಂಡಿರುವ ಸಂಕಟಗಳ ಪರ್ವತಕ್ಕೆ ಬೆಂಕಿ ಹಚ್ಚಿ, ಅದು ಉರಿದು ಬೂದಿಯಾಗಲಿ. ಬನ್ನಿ ಓ ಸೋದರರೆ, ಎದುರಿಸಿ ಅದನ್ನು. ಅದೊಂದು ಮಹೋನ್ನತ ಕಾರ್ಯ, ಅಲ್ಲದೆ ನಾವು ಬಹಳ ಕೆಳಗಿದ್ದೇವೆ. ಆದರೆ ನಾವು ಭಗವಂತನ ಪುತ್ರರು, ದಿವ್ಯಜ್ಯೋತಿಯ ಕಿಡಿ ಗಳು. ಭಗವಂತನಿಗೆ ಜಯವೆನ್ನಿ, ನಾವು ಯಶಸ್ವಿಯಾಗುತ್ತೇವೆ. ಹೋರಾಟದಲ್ಲಿ ನೂರಾರು ಜನ ಬೀಳಲಿ, ಆದರೇನಂತೆ, ನೂರಾರು ಜನ ಆ ಹೋರಾಟವನ್ನು ಕೈಗೆತ್ತಿಕೊಳ್ಳುತ್ತಾರೆ. ನಾನಿಲ್ಲಿ ಅಯಶಸ್ವಿಯಾಗಿ ಸಾಯಬಹುದು. ಆದರೆ ಮತ್ತೊಬ್ಬ ಅದನ್ನು ಮುಂದುವರಿಸುತ್ತಾನೆ. ರೋಗವು ಏನೆಂಬುದು ನಮಗೆ ತಿಳಿದಿದೆ. ಅದಕ್ಕೆ ಮದ್ದೂ ತಿಳಿದಿದೆ. ಆದರೆ ಅದರಲ್ಲಿ ವಿಶ್ವಾಸ ತಾಳ ಬೇಕಾಗಿದೆ. ದೊಡ್ಡವರು, ಶ್ರೀಮಂತರು ಎಂದು ಕರೆಸಿಕೊಳ್ಳುವವರ ಕಡೆ ತಿರುಗಿನೋಡಬೇಡಿ. ಹೃದಯಶೂನ್ಯರಾದ ಬುದ್ಧಿಜೀವಿಗಳ ಮಾತಿಗೆ ಬೆಲೆ ಕೊಡಬೇಡಿ. ಮತ್ತು ಅವರ ವೃತ್ತಪತ್ರಿಕೆಯ ಬರಹಗಳನ್ನು ಲಕ್ಷಿಸಬೇಡಿ. ಶ್ರದ್ಧೆ! ಅನುಕಂಪೆ! ಜ್ವಲಂತ ಶ್ರದ್ಧೆ! ಮತ್ತು ಜ್ವಲಂತ ಅನುಕಂಪೆ! ಈ ಜೀವನ ಏನೂ ಅಲ್ಲ, ಈ ಮರಣ ಏನೂ ಅಲ್ಲ, ಈ ಹಸಿವು ಏನೂ ಅಲ್ಲ, ಈ ಚಳಿಯೂ ಏನೂ ಅಲ್ಲ. ಜಯವಾಗಲಿ ಭಗವಂತನಿಗೆ! ಮುಂದೆ ಸಾಗಿರಿ. ಭಗವಂತನೇ ನಮ್ಮ ಸೇನಾ ನಾಯಕ. ಯಾರು ಕೆಳಗೆ ಬಿದ್ದರೆಂದು ನೋಡಲು ಹಿಂದೆ ತಿರುಗಬೇಡಿ, ಮುನ್ನಡೆಯಿರಿ, ನುಗ್ಗಿ ನಡೆಯಿರಿ. ಸೋದರರೇ, ಇದೇ ಪ್ರಕಾರವಾಗಿ ನಾವು ಮುನ್ನಡೆಯುತ್ತ ಸಾಗಬೇಕು. ಒಬ್ಬ ಬೀಳುತ್ತಾನೆ, ಮತ್ತೊಬ್ಬ ಅದನ್ನು ಮುಂದುವರಿಸುತ್ತಾನೆ.”

ಭಾರತದ ಇತಿಹಾಸದ ಹಿನ್ನೆಲೆಯಲ್ಲಿ, ಹಾಗೂ ಅದರ ಭವಿಷ್ಯದ ಮುಂಬೆಳಕಿನಲ್ಲಿ, ಅದಕ್ಕೆ ಅತ್ಯಂತ ಸೂಕ್ತವಾದ ಸಂದೇಶವಿದು. ಇಂತಹ ಅದ್ಭುತ ಆಲೋಚನೆಗಳನ್ನು ಚಿಂತಿಸಿ ಅದ್ಭುತ ಮಾತುಗಳನ್ನು ಬರೆಯಬೇಕಾದರೆ, ಸ್ವಾಮೀಜಿ ಭಾರತದ ನಾಡಿ ಮಿಡಿತವನ್ನು ಎಷ್ಟು ಚೆನ್ನಾಗಿ ಅರಿತಿದ್ದಿರಬೇಕು! ಅವರ ಈ ಮಾತುಗಳು ಭಾರತೀಯರ ಹೃದಯವನ್ನು ಸ್ಪಂದಿಸಿದುವು; ಅವರ ಯೋಜನೆಗಳು ಗುರಿಯತ್ತ ಮುನ್ನಡೆಯುವಂತೆ ಮಾಡಿದುವು.

೧೮೯೩ರ ನವೆಂಬರ್ ೨ರಂದು ಅಳಸಿಂಗ ಪೆರುಮಾಳರಿಗೆ ಬರೆದ ಪತ್ರದಲ್ಲಿ ಸ್ವಾಮೀಜಿ, ಮೊದಲ ಬಾರಿಗೆ ಸಮ್ಮೇಳನದಲ್ಲಿನ ತಮ್ಮ ಯಶಸ್ಸಿನ ಬಗ್ಗೆ ಪ್ರಸ್ತಾಪಿಸಿದರು. ಆದರೆ ತಕ್ಷಣ ಪತ್ರದ ಧಾಟಿಯನ್ನು ಬದಲಿಸಿ, ತಮ್ಮ ಮನದಾಳದ ಏಕೈಕ ಚಿಂತೆಯಾದ ಭಾರತದ ಪುನರು ತ್ಥಾನದ ಬಗ್ಗೆ ತಮ್ಮ ಆಲೋಚನೆಗಳನ್ನು ಹರಿಸಲಾರಂಭಿಸಿದರು. ಅದರಲ್ಲಿ ಅವರು ಹೀಗೆ ಬರೆದರು:

“ಹಿಂದುವಾದವನು ತನ್ನ ಧರ್ಮವನ್ನೆಂದೂ ಬಿಡಬಾರದು. ಆದರೆ ಧರ್ಮವನ್ನು ಅದರ ಎಲ್ಲೆಗಳೊಳಗೇ ಇರಿಸಿ, ಸಮಾಜವು ಬೆಳೆಯಲು ಸ್ವಾತಂತ್ರ್ಯ ಕೊಡಬೇಕು. ಭಾರತದ ಸುಧಾರಕ ರೆಲ್ಲರೂ ಪುರೋಹಿತಶಾಹಿಯ ಘೋರ ಅಪರಾಧಗಳಿಗೂ ಸಮಾಜದ ಅವನತಿಗೂ ಧರ್ಮವನ್ನೇ ಹೊಣೆಯನ್ನಾಗಿಸುವ ದೊಡ್ಡ ತಪ್ಪು ಮಾಡಿದರು. ಮತ್ತು ಅವಿಚ್ಛಿನ್ನವಾದ ಧರ್ಮದ ಕಟ್ಟಡ ವನ್ನು ಕಿತ್ತೆಸೆಯಲು ಪ್ರಯತ್ನಿಸಿದರು. ಅದರ ಪರಿಣಾಮ? ಸೋಲು. ಬುದ್ಧನಿಂದ ಹಿಡಿದು ರಾಜಾ ರಾಮಮೋಹನರಾಯರವರೆಗೆ ಪ್ರತಿಯೊಬ್ಬರೂ ಜಾತಿಪದ್ಧತಿಯನ್ನು ಧರ್ಮದ ಒಂದು ಅಂಗವೆಂದು ಪರಿಗಣಿಸಿ, ಅವೆರಡನ್ನೂ ಒಟ್ಟಿಗೆ ನಿರ್ನಾಮಗೊಳಿಸಲು ಪ್ರಯತ್ನಿಸಿ, ಕಡೆಗೆ ಸೋತರು. ಆದರೆ ಈ ಪುರೋಹಿತವರ್ಗದವರು ಎಷ್ಟೇ ಹಾರಾಡಿದರೂ ಎಷ್ಟೇ ಆರ್ಭಟಿಸಿದರೂ, ಜಾತಿಯೆಂಬುದು ಒಂದು ಸಾಮಾಜಿಕ ವ್ಯವಸ್ಥೆ ಮಾತ್ರವೇ. ತನ್ನ ಉದ್ದೇಶಿತ ಸೇವೆಯನ್ನು ಸಲ್ಲಿಸಿದ ಮೇಲೆ ಈಗ ಅದು ಭಾರತದ ವಾತಾವರಣವನ್ನೆಲ್ಲ ತನ್ನ ದುರ್ವಾಸನೆಯಿಂದ ತುಂಬಿಸಿದೆ. ಅದನ್ನು ಹೋಗಾಲಾಡಿಸುವ ಏಕೈಕ ಮಾರ್ಗವೆಂದರೆ, ಜನರಿಗೆ ಅವರು ಕಳೆದುಕೊಂಡ ಸಾಮಾಜಿಕ ವೈಯಕ್ತಿಕತೆಯನ್ನು ಮರಳಿ ದೊರಕಿಸಿಕೊಡುವುದು. ಇಲ್ಲಿ (ಅಮೆರಿಕದಲ್ಲಿ) ಹುಟ್ಟುವ ಪ್ರತಿಯೊಬ್ಬನೂ ತಾನೊಬ್ಬ ಮನುಷ್ಯನೆಂದು ಬಲ್ಲ. ಭಾರತದಲ್ಲಿ ಹುಟ್ಟುವ ಪ್ರತಿಯೊಬ್ಬನೂ ತಾನು ಸಮಾಜದ ಗುಲಾಮನೆಂದೇ ನಂಬಿರುತ್ತಾನೆ. ಬೆಳವಣಿಗೆಯು ಸಾಧ್ಯವಾಗುವುದು ಸ್ವಾತಂತ್ರ್ಯದ ಸ್ಥಿತಿಯಲ್ಲಿ ಮಾತ್ರವೇ. ಆ ಸ್ವಾತಂತ್ರ್ಯವನ್ನು ಕಿತ್ತುಕೊಂಡ ತಕ್ಷಣವೇ ಅವನತಿ ಪ್ರಾರಂಭವಾಗುತ್ತದೆ. ಆಧುನಿಕ ಯುಗದ ಸ್ಪರ್ಧಾತ್ಮಕತೆಯ ಆರಂಭದೊಂದಿಗೆ, ಜಾತಿ ಪದ್ಧತಿಯು ಹೇಗೆ ವೇಗವಾಗಿ ನಿರ್ನಾಮವಾಗುತ್ತಿದೆಯೆಂದು ನೋಡಿ! ಈಗ ಅದನ್ನು ಕೊಲ್ಲಲು ಯಾವ ಉಪಾಯವೂ ಬೇಕಾಗಿಲ್ಲ...”

ಭಾರತೀಯ ಸ್ತ್ರೀಯರ ಸ್ಥಿತಿಗತಿಗಳನ್ನು ಸುಧಾರಿಸಬೇಕೆನ್ನುವುದು, ಭಾರತದ ಪುನರುತ್ಥಾನದ ಬಗೆಗಿನ ಸ್ವಾಮೀಜಿಯ ಯೋಜನೆಯ ಒಂದು ಮುಖ್ಯಾಂಶ. ಅಮೆರಿಕದ ಸಮಾಜದಲ್ಲಿ ಸ್ತ್ರೀಯರು ಹೊಂದಿದ್ದ ಗೌರವಯುತ ಸ್ಥಾನ ಹಾಗೂ ಅವರು ಹೊಂದಿದ್ದ ಪ್ರಭಾವವನ್ನು ಕಂಡು ಸ್ವಾಮೀಜಿ ವಿಸ್ಮಯಮೂಕರಾದರು. ಮಹಿಳೆಯರಿಗೆ ಅಮೆರಿಕನ್ನರು ನೀಡಿರುವ ಸ್ಥಾನವೇ ಅವರ ಸಮಾಜದ ಸುಸ್ಥಿತಿಗೆ ಮುಖ್ಯ ಕಾರಣವೆಂಬುದನ್ನು ಸ್ವಾಮೀಜಿ ಗಮನಿಸಿದರು. ಒಂದು ಪತ್ರದಲ್ಲಿ ಅವರು ಆ ಬಗ್ಗೆ ಬರೆಯುತ್ತಾರೆ:

“ನಿಜವಾದ ‘ಶಕ್ತಿ ಉಪಾಸಕರು’ ಯಾರೆಂದು ಬಲ್ಲೆಯಾ? ಯಾವನು ಭಗವಂತನು ಸರ್ವ ವ್ಯಾಪಿಯಾದ ಶಕ್ತಿಯೆಂದು ಬಲ್ಲನೋ ಮತ್ತು ಸ್ತ್ರೀಯನ್ನು ಆ ಶಕ್ತಿಯ ಆವಿರ್ಭಾವವೆಂದು ಕಾಣುತ್ತಾನೆಯೋ ಅವನು. ಇಲ್ಲಿನ (ಅಮೆರಿಕದ) ಅನೇಕ ಪುರುಷರು ಸ್ತ್ರೀಯರನ್ನು ಆ ದೃಷ್ಟಿ ಯಿಂದ ಕಾಣುತ್ತಾರೆ. ಮನು ಮಹರ್ಷಿಯೂ ಅದನ್ನೇ ಹೇಳಿದ್ದ–ಎಲ್ಲಿ ನಾರಿಯರು ಚೆನ್ನಾಗಿ ನೋಡಿಕೊಳ್ಳಲ್ಪಡುತ್ತಾರೋ ಮತ್ತು ಸಂತೋಷದಿಂದಿರುತ್ತಾರೋ ಅಂತಹ ಕುಟುಂಬಗಳನ್ನು ದೇವತೆಗಳು ಹರಸುತ್ತಾರೆ ಎಂದು. ಇಲ್ಲಿ ಪುರುಷರು ಸ್ತ್ರೀಯರೊಂದಿಗೆ ಅತ್ಯಂತ ಶ್ರೇಷ್ಠವಾದ ರೀತಿಯಲ್ಲಿ ನಡೆದುಕೊಳ್ಳುತ್ತಾರೆ. ಆದ್ದರಿಂದಲೇ ಅವರು ಅಷ್ಟು ಸ್ವತಂತ್ರರು, ಉತ್ಸಾಹಶಾಲಿ ಗಳು, ವಿದ್ಯಾವಂತರು, ಶ್ರೀಮಂತರು. ಹಾಗಾದರೆ ನಾವೇಕೆ ಗುಲಾಮರಂತೆ ಕಷ್ಟಪಡುತ್ತ ಮೃತಪ್ರಾಯರಾಗಿದ್ದೇವೆ? ಉತ್ತರ ಸ್ವತಸ್ಸಿದ್ಧವಾಗಿದೆ. ನೀವು ನಿಮ್ಮ ಸ್ತ್ರೀಯರ ಪರಿಸ್ಥಿತಿಯನ್ನು ಉತ್ತಮಪಡಿಸಬಲ್ಲಿರೇನು? ಹಾಗಾದರೆ ನೀವು ಅಭಿವೃದ್ಧಿ ಹೊಂದುವ ಆಸೆಯಿಟ್ಟುಕೊಳ್ಳ ಬಹುದು. ಇಲ್ಲದಿದ್ದರೆ, ನೀವು ಈಗಿರುವಷ್ಟೇ ಹಿಂದುಳಿದಿರುತ್ತೀರಿ.”

ಸ್ವಾಮೀಜಿಯ ಎಲ್ಲ ಆಲೋಚನೆಗಳೂ ಉಪಾಯಗಳೂ ಆಧಾರಿತವಾಗಿದ್ದುದು ಒಂದು ಆಲೋಚನೆಯ ಮೇಲೆ–ಅದೇ, ಭಾರತೀಯ ಸಮಾಜದಲ್ಲಿ ಆಲೋಚನಾ ಸ್ವಾತಂತ್ರ್ಯದ ಹಾಗೂ ಕ್ರಿಯಾ ಸ್ವಾತಂತ್ರ್ಯದ ವಾತಾವರಣವನ್ನು ನಿರ್ಮಿಸುವುದು. ಈ ಮೂಲಭೂತ ಆವಶ್ಯಕತೆಯನ್ನು ಅವರು, ೧೮೯೪ರ ಜನವರಿಯಲ್ಲಿ ತಮ್ಮ ಮದ್ರಾಸೀ ಶಿಷ್ಯರಿಗೆ ಬರೆದ ಪತ್ರದಲ್ಲಿ ವಿಶದೀಕರಿಸುತ್ತಾರೆ:

“ಭಾರತದಲ್ಲೂ ಜಗತ್ತಿನ ಇತರೆಡೆಗಳಲ್ಲೂ ಮಾನವ ಜನಾಂಗವು ಅಭಿವೃದ್ಧಿ ಪಡಿಸಿದಂತಹ ಉನ್ನತ-ಉದಾತ್ತ ಭಾವನೆಗಳನ್ನು ಭಾರತದ ಅತ್ಯಂತ ನಿರ್ಗತಿಕರ, ಅತ್ಯಂತ ಹಿಂದುಳಿದವರ ಮನೆಬಾಗಿಲಿಗೇ ಕೊಂಡೊಯ್ದು, ಅವರನ್ನು ಆಲೋಚಿಸುವಂತೆ ಮಾಡಬೇಕು. ಜಾತೀಯತೆ ಇರಬೇಕೋ ಬೇಡವೋ, ಸ್ತ್ರೀಯರಿಗೆ ಸಂಪೂರ್ಣ ಸ್ವಾತಂತ್ರವಿರಬೇಕೋ ಇರಬಾರದೋ ಎಂಬು ದೆಲ್ಲ ನನಗೆ ಅಷ್ಟು ಮುಖ್ಯವಲ್ಲ. ಜನರಿಗೆ ಆಲೋಚನಾಸಾಮರ್ಥ್ಯವುಂಟಾದರೆ ಅವರೇ ಅದನ್ನು ನಿರ್ಧರಿಸುತ್ತಾರೆ. ‘ಆಲೋಚನೆ ಮತ್ತು ಕೃತಿಗಳ ಸ್ವಾತಂತ್ರ್ಯವೊಂದೇ ಜೀವನದ, ಬೆಳವಣಿಗೆಯ ಹಾಗೂ ನೆಮ್ಮದಿಯ ಹಿಂದಿನ ಅವಶ್ಯಕತೆ.’ ಎಲ್ಲಿ ಅದಿಲ್ಲವೋ ಅಲ್ಲಿ ಮನುಷ್ಯ, ಅವನ ಕುಲ, ರಾಷ್ಟ್ರ–ಎಲ್ಲವೂ ಕುಸಿಯಲೇಬೇಕು.

“ಜೀವನದಲ್ಲಿ ನನ್ನ ಒಟ್ಟಾರೆ ಉದ್ದೇಶವೇನೆಂದರೆ, ಪ್ರತಿಯೊಬ್ಬರ ಮನೆಬಾಗಿಲಿಗೇ ಉದಾತ್ತ ಆಲೋಚನೆಗಳನ್ನು ತಂದುಕೊಡಬಲ್ಲಂತಹ ಒಂದು ವ್ಯವಸ್ಥೆಯನ್ನು ನಿರ್ಮಾಣ ಮಾಡುವುದು; ಬಳಿಕ ಸ್ತ್ರೀ-ಪುರುಷರೆಲ್ಲರೂ ತಮ್ಮ ತಮ್ಮ ಹಣೆಬರಹವನ್ನು ತಾವೇ ರೂಪಿಸಿಕೊಳ್ಳಲು ಬಿಡು ವುದು. ಜೀವನದ ಅತಿ ಮುಖ್ಯ ವಿಚಾರಗಳ ಬಗ್ಗೆ ನಮ್ಮ ಪೂರ್ವಿಕರೂ ಇತರ ದೇಶಗಳವರೂ ಏನು ಆಲೋಚಿಸಿದ್ದಾರೆಂದು ಅವರು ತಿಳಿದುಕೊಳ್ಳಲಿ. ಅದರಲ್ಲೂ ಮುಖ್ಯವಾಗಿ, ಇತರರು ಈಗೇನು ಮಾಡುತ್ತಿದ್ದಾರೆಂಬುದನ್ನು ತಿಳಿದುಕೊಂಡು, ಬಳಿಕ ನಿರ್ಧರಿಸಲಿ. ಶ್ರಮವಹಿಸಿ ದುಡಿ ಯಿರಿ, ಏಕರೀತಿಯಲ್ಲಿ ಮುಂದುವರಿಯಿರಿ ಮತ್ತು ಭಗವಂತನಲ್ಲಿ ಶ್ರದ್ಧೆಯಿಡಿ.

“ಈ ಧ್ಯೇಯವಾಕ್ಯವನ್ನು ನಿಮ್ಮ ಮುಂದಿಟ್ಟುಕೊಳ್ಳಿ–‘ಜನಸಮೂಹದ ಧರ್ಮಕ್ಕೆ ಹಾನಿ ಯುಂಟುಮಾಡದೆಯೇ, ಅವರನ್ನು ಮೇಲೆತ್ತುವುದು.’”

ಚರಿತ್ರೆಯ ದಿಕ್ಕನ್ನು ಬದಲಾಯಿಸಿದಂತಹ ವ್ಯಕ್ತಿಗಳಲ್ಲಿ, ಜನಗಳ ಧಾರ್ಮಿಕಶ್ರದ್ಧೆಗೆ ಹಾನಿ ಯುಂಟುಮಾಡದೆಯೇ ಅವರ ಸ್ಥಿತಿಗತಿಯನ್ನು ಸುಧಾರಿಸುವುದರ ಆವಶ್ಯಕತೆಯನ್ನು ಅರ್ಥಮಾಡಿ ಕೊಂಡವರ ಸಂಖ್ಯೆ, ಪ್ರಾಯಶಃ ಬಹಳ ಕಡಿಮೆ. ಆಧುನಿಕ ಭಾರತದ ಚಿಂತಕರ ಪೈಕಿ, ಜನಜೀವನದಲ್ಲಿ ಹಾಸುಹೊಕ್ಕಾಗಿರುವ ಧಾರ್ಮಿಕ ಸಂಸ್ಕೃತಿಯನ್ನು ಪಲ್ಲಟಗೊಳಿಸದ ರೀತಿಯಲ್ಲಿ ಅವರನ್ನು ಮೇಲೆತ್ತುವ ಪ್ರಯತ್ನಕ್ಕೆ ಅತಿ ಹೆಚ್ಚಿನ ಒತ್ತು ನೀಡಿದವರಲ್ಲಿ ಸ್ವಾಮಿ ವಿವೇಕಾನಂದರೇ ಪ್ರಥಮ ಹಾಗೂ ಇದುವರೆಗಿನ ಅತಿಮುಖ್ಯ ವ್ಯಕ್ತಿ. ಇಲ್ಲದಿದ್ದಲ್ಲಿ ರಾಷ್ಟ್ರದ ಪುನರುತ್ಥಾನವು ಅಸಾಧ್ಯವೆಂಬುದನ್ನು ಅವರು ಕಂಡುಕೊಂಡಿದ್ದರು. ಅವರು ಅದೇ ಪತ್ರದಲ್ಲಿ ಮುಂದುವರಿದು ಹೀಗೆ ಹೇಳುತ್ತಾರೆ:

“ಭಾರತವಿರುವುದು ಜೋಪಡಿಗಳಲ್ಲಿ ಎಂಬುದನ್ನು ನೆನಪಿಡಿ. ಆದರೆ ಅಯ್ಯೋ! ಅವರಿಗಾಗಿ ಯಾರೂ ಏನೂ ಮಾಡಲಿಲ್ಲ. ನಮ್ಮ ಆಧುನಿಕ ಸಮಾಜ ಸುಧಾರಕರು ವಿಧವಾವಿವಾಹ ಕಾರ್ಯದ ಗಡಿಬಿಡಿಯಲ್ಲಿದ್ದಾರೆ. ನಾನು ಪ್ರತಿಯೊಂದು ಸುಧಾರಣೆಯನ್ನೂ ಬೆಂಬಲಿಸುವವನೇ. ಆದರೆ ಒಂದು ರಾಷ್ಟ್ರದ ಭವಿಷ್ಯವು ಆಧರಿಸಿರುವುದು, ಎಷ್ಟು ಜನ ವಿಧವೆಯರಿಗೆ ಗಂಡಂದಿರು ಸಿಗುತ್ತಾರೆಂಬುದರಮೇಲಲ್ಲ, ಆದರೆ ಜನಸಾಮಾನ್ಯರ ಸ್ಥಿತಿಗತಿಯ ಮೇಲೆ. ಅವರನ್ನು ನೀವು ಮೇಲೆತ್ತಬಲ್ಲಿರಾ? ಅವರು ತಮ್ಮಲ್ಲಿ ಅಂತರ್ಗತವಾದ ಆಧ್ಯಾತ್ಮಿಕತೆಯನ್ನು ಕಳೆದುಕೊಳ್ಳದ ರೀತಿಯಲ್ಲಿ, ಅವರಿಗೆ ಅವರ ಕಳೆದು ಹೋದ ವೈಯಕ್ತಿಕತೆಯನ್ನು ಮರಳಿ ದೊರಕಿಸಿಕೊಡ ಬಲ್ಲಿರಾ? ನಿಮ್ಮ ಕೆಲಸದ ವಿಧಾನಗಳಲ್ಲಿ, ಸ್ವತಂತ್ರ ಮನೋವೃತ್ತಿಯಲ್ಲಿ, ಶಕ್ತಿಯಲ್ಲಿ, ಸಮಾನತಾ ಮನೋಭಾವದಲ್ಲಿ ಪಕ್ಕಾ ಪಾಶ್ಚಾತ್ಯರಾಗಿದ್ದು ಜೊತೆಗೇ ನಿಮ್ಮ ಧಾರ್ಮಿಕ ಸಂಸ್ಕೃತಿ ಹಾಗೂ ಸ್ವಭಾವಗಳಲ್ಲಿ ಅತ್ಯಂತ ನಿಷ್ಠಾವಂತ ಹಿಂದುವಾಗಿರಲು ನಿಮಗೆ ಸಾಧ್ಯವಿದೆಯೆ? ಇದು ಸಾಧ್ಯವಾಗಬೇಕು, ಮತ್ತು ನಾವದನ್ನು ಖಂಡಿತ ಮಾಡುತ್ತೇವೆ. ನೀವೆಲ್ಲ ಅದನ್ನು ಸಾಧಿಸುವುದ ಕ್ಕಾಗಿಯೇ ಹುಟ್ಟಿದ್ದೀರಿ. ನಿಮ್ಮಲ್ಲಿ ನೀವು ಶ್ರದ್ಧೆಯಿಡಿ. ಶ್ರೇಷ್ಠವಾದ ನಂಬಿಕೆಗಳೇ ಶ್ರೇಷ್ಠ ಕಾರ್ಯ ಗಳ ಮೂಲ, ಎಂದೆಂದಿಗೂ ಮುಂದೆ ಸಾಗಿ! ಬಡವರ ಬಗ್ಗೆ ಹಿಂದುಳಿದವರ ಬಗ್ಗೆ ಸಹಾನು ಭೂತಿ–ಕಡೆಯುಸಿರಿನವರೆಗೂ ಇದೇ ನಮ್ಮ ಧ್ಯೇಯವಾಕ್ಯ.”

ಆಗಿನ ದಿನಗಳಲ್ಲಿ ಇವೆಲ್ಲ ಅತ್ಯಂತ ಕ್ರಾಂತಿಕಾರೀ ಭಾವನೆಗಳು. ಪ್ರತಿಯೊಬ್ಬ ವ್ಯಕ್ತಿಯ ಒಳಿತನ್ನೇ ಸಾಧಿಸುವ ಉದ್ದೇಶಗಳು ಇವು. ಆದರೆ ಇತರರ ಎದೆಯ ಮೇಲೆ ನಿಂತವರಿಗೆ ಈ ಮಾತುಗಳು ರುಚಿಸುವಂತಿರಲಿಲ್ಲ. ಕೂಪಮಂಡೂಕಗಳಿಗೆ ಅವರ ವಿಶಾಲ ಭಾವನೆಗಳು ಅರ್ಥ ವಾಗಲು ಸಾಧ್ಯವಿರಲಿಲ್ಲ. ಇದು ಸ್ವಾಮೀಜಿಗೆ ಚೆನ್ನಾಗಿಯೇ ಗೊತ್ತಿತ್ತು. ಆದರೆ ಅವರು ಅದನ್ನು ಲಕ್ಷಿಸಲಿಲ್ಲ. ೧೮೯೪ರ ಜನವರಿಯಲ್ಲಿ ಜುನಾಗಢದ ದಿವಾನರಾದ ಹರಿದಾಸ್ ದೇಸಾಯಿಯವರಿಗೆ ಬರೆದ ಪತ್ರದಲ್ಲಿ ಅವರು ತಮ್ಮ ಆಲೋಚನೆಗಳನ್ನು ಹೊರಗೆಡವುತ್ತಾರೆ–

“ಭಾರತದಲ್ಲಿ ಎಷ್ಟೋ ಜನರಿಗೆ ನನ್ನನ್ನು ಅರ್ಥ ಮಾಡಿಕೊಳ್ಳಲು ಸಾಧ್ಯವಾಗಲಿಲ್ಲ. ನಿಜವೇ, ಪಾಪ, ಅವರಿಗದು ಹೇಗೆ ತಾನೆ ಸಾಧ್ಯವಿತ್ತು? ಪ್ರತಿ ದಿನದ ವ್ಯವಹಾರಗಳಾದ ತಿನ್ನುವುದು-ಕುಡಿ ಯುವುದು ಇವುಗಳಿಂದಾಚೆಗೆ ಅವರ ಆಲೋಚನೆಗಳು ಎಂದೂ ಹರಿದದ್ದೇ ಇಲ್ಲ. ನಿಮ್ಮಂತಹ ಕೆಲವು ಉದಾತ್ತ ವ್ಯಕ್ತಿಗಳು ಮಾತ್ರ ನನ್ನನ್ನು ಮೆಚ್ಚಿಕೊಳ್ಳುವುದು ಎಂಬುದು ನನಗೆ ಗೊತ್ತು. ದೇವರು ನಿಮ್ಮನ್ನು ಹರಸಲಿ. ಆದರೆ ಮೆಚ್ಚುಗೆ ಬರುತ್ತದೋ ಬಿಡುತ್ತದೋ, ನಾನಂತೂ ಯುವಕರನ್ನು ಸಂಘಟಿಸಲೆಂದೇ ಹುಟ್ಟಿದವನು; ಪ್ರತಿಯೊಂದು ನಗರದಲ್ಲೂ ಇನ್ನೂ ನೂರಾರು ಜನ ನನ್ನೊಂದಿಗೆ ಸೇರಲಿದ್ದಾರೆ. ಅವರನ್ನು ಅಪ್ರತಿಹತ ಅಲೆಗಳಂತೆ ಭಾರತದಾದ್ಯಂತ ಕಳಿಸಿ, ನೆಮ್ಮದಿ, ನೈತಿಕತೆ, ಧರ್ಮ, ವಿದ್ಯಾಭ್ಯಾಸಗಳನ್ನು ಅತ್ಯಂತ ದೀನ ದುರ್ಬಲ ಜನರಿಗೆ ತಲುಪಿಸ ಬೇಕೆಂಬುದು ನನ್ನ ಆಕಾಂಕ್ಷೆ. ನಾನಿದನ್ನು ಮಾಡುತ್ತೇನೆ ಇಲ್ಲವೆ ಮಡಿಯುತ್ತೇನೆ.”

ಭಾರತದ ನಿರಾಶಾದಾಯಕ ವಾಸ್ತವಿಕತೆ ಸ್ವಾಮೀಜಿಗೆ ಸ್ಪಷ್ಟವಾಗಿಯೇ ತಿಳಿದಿತ್ತು. ಇದ್ದಕ್ಕಿ ದ್ದಂತೆ ಎಲ್ಲವೂ ಚೆನ್ನಾಗಿ ಆಗಿಬಿಡುವುದೆಂಬ ಭ್ರಮೆಯೇನೂ ಅವರಿಗಿರಲಿಲ್ಲ. ಆದರೆ ಭಾರತದ ಉತ್ತಮ ಭವಿಷ್ಯದ ಬಗೆಗಿನ ಅವರ ವಿಶ್ವಾಸ ಮಾತ್ರ ಅಚಲವಾಗಿತ್ತು. ಭಾರತೀಯರು ಮೇಲೇರ ದಿರುವುದಕ್ಕೆ ಅವರ ಸ್ವಭಾವವೇ ಅತಿ ದೊಡ್ಡ ವೈರಿ ಎಂಬುದನ್ನವರು ವಿಶ್ಲೇಷಿಸಿದ್ದರು. ಈ ಬಗ್ಗೆ ಅವರು ಅದೇ ಪತ್ರದಲ್ಲಿ ಬರೆಯುತ್ತಾರೆ–

“ಭಾರತದಲ್ಲಿ ಮೂರು ಜನ ಒಂದೈದು ನಿಮಿಷವಾದರೂ ಒಗ್ಗಟ್ಟಾಗಿ ಸೇರಿ ಕೆಲಸ ಮಾಡ ಲಾರರು. ಪ್ರತಿಯೊಬ್ಬನೂ ಅಧಿಕಾರಕ್ಕಾಗಿ ಹೊಡೆದಾಡುತ್ತಾನೆ. ಕೊನೆಗೊಂದು ದಿನ ಇಡೀ ಸಂಸ್ಥೆಯೇ ಮುರಿದುಬಿದ್ದಿರುತ್ತದೆ. ಭಗವಂತಾ! ಭಗವಂತಾ!ಹೊಟ್ಟೆಕಿಚ್ಚು ಪಡದಿರುವುದನ್ನು ನಾವು ಎಂದಿಗೆ ಕಲಿತೇವೋ! ಇಂಥಾ ಒಂದು ರಾಷ್ಟ್ರದಲ್ಲಿ, ಅದರಲ್ಲೂ ಬಂಗಾಳದಲ್ಲಿ, ಪರಸ್ಪರ ಭಿನ್ನಾಭಿಪ್ರಾಯಗಳಿದ್ದರೂ ಅಮರಪ್ರೇಮದ ಅಭೇಧ್ಯ ಬಂಧದಿಂದ ಬಿಗಿಯಲ್ಪಟ್ಟ ವ್ಯಕ್ತಿಗಳ ತಂಡವೊಂದನ್ನು ನಿರ್ಮಿಸುವುದೆಂದರೆ ಅದೊಂದು ಅದ್ಭುತವಲ್ಲವೆ? ಈ ತಂಡ ಇನ್ನಷ್ಟು ಬೆಳೆಯುತ್ತದೆ. ಅನಂತ ಶಕ್ತಿ ಹಾಗೂ ಪ್ರಗತಿಯೊಂದಿಗೆ ಜೊತೆಗೂಡಿದ ಅದ್ಭುತ ಔದಾರ್ಯ ಪೂರ್ಣವಾದ ಈ ವಿಚಾರವು ಭಾರತದಾದ್ಯಂತ ಹರಡಬೇಕು. ಅದು ಸಮಸ್ತ ರಾಷ್ಟ್ರವನ್ನೇ ಸಚೇತನಗೊಳಿಸಬೇಕು ಮತ್ತು ಈ ಗುಲಾಮರ ರಾಷ್ಟ್ರದ ಪಿತ್ರಾರ್ಜಿತ ಸೊತ್ತಾದ ಭಯಂಕರ ಅಜ್ಞಾನ, ದ್ವೇಷ, ಜಾತೀಯತೆ, ಹಳೆಯ ಕಂದಾಚಾರಗಳು ಹಾಗೂ ಮತ್ಸರ ಇವುಗಳನ್ನೂ ಭೇದಿಸಿ ಸಮಾಜದ ರೋಮಕೂಪಗಳಲ್ಲೆಲ್ಲ ಒಳಹೊಗಬೇಕು.”

ಸ್ವಾಮೀಜಿ ರಾಷ್ಟ್ರಕ್ಕಿತ್ತ ರಣಕಹಳೆಯ ಮೊಳಗು ಅವರು ತಮ್ಮ ಅನುಯಾಯಿಗಳಿಗೆ ಬರೆದ ಪತ್ರಗಳ ಮೂಲಕ ಭಾರತವನ್ನು ಮುಟ್ಟಿತು. ಅತ್ಯಂತ ಅಚ್ಚರಿಯ ಸಂಗತಿಯೇನೆಂದರೆ, ಇಂತಹ ರಾಷ್ಟ್ರನಿರ್ಮಾಣಕಾರಿಯಾದ ಜಾಜ್ವಲ್ಯಮಾನ ಪತ್ರಗಳನ್ನು ಬರೆಯುವಾಗ ಅವರು ತಮ್ಮ ನಿರಂತರ ಉಪನ್ಯಾಸ ಪ್ರವಾಸದ ಗಡಿಬಿಡಿಯಲ್ಲಿ ಹಾಗೂ ವಿರೋಧಿಗಳೊಂದಿಗಿನ ದೀರ್ಘ ಯುದ್ಧದಲ್ಲಿ ನಿರತರಾಗಿದ್ದರು. ಅಲ್ಲದೆ ಅನರ್ಘ್ಯವೂ ಶಕ್ತಿದಾಯಕವೂ ಆದ ವೇದಾಂತದ ಸಂದೇಶವನ್ನು ಪಶ್ಚಿಮದಲ್ಲಿ ಹರಡುವುದು ಅವರ ಪಾಲಿಗೆ ಅತ್ಯಂತ ಮುಖ್ಯವಾಗಿತ್ತು. ಇಷ್ಟು ಒತ್ತಡದಲ್ಲೂ ಅವರು ದೀರ್ಘಕಾಲದಿಂದ ನಿರ್ವೀರ್ಯವಾಗಿದ್ದ ಭಾರತದ ಬುದ್ಧಿಶಕ್ತಿಯನ್ನು ಚುರುಕುಗೊಳಿಸಿ, ತಮ್ಮ ಪುನರ್ಜಾಗೃತಿಯ ಸಂದೇಶವನ್ನು ಬಿತ್ತಲು ಸಿದ್ಧಗೊಳಿಸುತ್ತಿದ್ದರು.

ಅಮೆರಿಕದಿಂದ ಸ್ವಾಮೀಜಿ ತಮ್ಮ ಶಿಷ್ಯರಿಗೆ ಹಾಗೂ ಸ್ನೇಹಿತರಿಗೆ ಪತ್ರಗಳನ್ನು ಬರೆದು, ಬೀಜ ಬಿತ್ತಲು ನೆಲವನ್ನು ಹದಗೊಳಿಸಿದುದನ್ನು ಈವರೆಗೆ ನೋಡಿದೆವು. ಅವರು ಇದೇ ಸಮಯದಲ್ಲಿ ಕೈಗೊಂಡ ಮತ್ತೊಂದು ಮಹಾಕಾರ್ಯವೆಂದರೆ ತಮ್ಮದೇ ಆದ ಸಂನ್ಯಾಸಿಗಳ ಸಂಘವನ್ನು ಮತ್ತಷ್ಟು ವ್ಯವಸ್ಥಿತಗೊಳಿಸಿ, ಅದರ ಧ್ಯೇಯಾದರ್ಶಗಳನ್ನು ಸ್ಪಷ್ಟವಾಗಿ ರೂಪಿಸಿದುದು. ಈ ಕಾರ್ಯವನ್ನೂ ಅವರು ಸಾಧಿಸಿದುದು ತಮ್ಮ ಪತ್ರಗಳ ಮೂಲಕವೇ. ಅವರ ಪಾಲಿಗೆ ಈ ಸಂನ್ಯಾಸಿಗಳ ಸಂಘದ ನಿರ್ಮಾಣವೇ ಅತ್ಯಂತ ಮುಖ್ಯವಾದದ್ದಾಗಿತ್ತು. ಅವರ ಎಲ್ಲ ಕಾರ್ಯ ಯೋಜನೆಗಳ ಆಧಾರಸ್ತಂಭವೇ ಅದಾಗಿತ್ತು. ಈಗಾಗಲೇ ಅವರ ಈ ಸಂಘವು ಅಸ್ತಿತ್ವದಲ್ಲಿತ್ತಾ ದರೂ ಅದು ಸಚೇತನವಾಗಿರಲಿಲ್ಲ. ಅವರೆಲ್ಲರೂ ಸರ್ವಸಂಗ ಪರಿತ್ಯಾಗ ಮಾಡಿ ಬಂದ ಉತ್ಸಾಹೀ ಯುವಕರೇ ಆದರೂ ಪ್ರತಿಯೊಬ್ಬನ ಮನೋಭಾವ-ಅಭಿರುಚಿ-ಸಾಧ್ಯತೆಗಳೂ ಬೇರೆ ಬೇರೆಯಾಗಿದ್ದುವು. ಅಂಥವರನ್ನೆಲ್ಲ ಒಗ್ಗೂಡಿಸುವುದೇ ಒಂದು ಮಹಾಸಾಧನೆ. ಈಗ ಅವರನ್ನು ಸಮಾನ ತತ್ತ್ವಗಳ ಮೇಲೆ ನಿಂತು ಸಮಾನೋದ್ದೇಶಗಳತ್ತ ಸಾಗುವಂತೆ ಮಾಡುವುದು ಕಠಿಣತರವಾಗಿತ್ತು.

ನಾವು ಈಗಾಗಲೇ ನೋಡಿದಂತೆ, ೧೮೯ಂರಲ್ಲಿ ಕಡೆಯಬಾರಿಗೆ ಬಾರಾನಾಗೋರನ್ನು ಬಿಟ್ಟು ಹೊರಟ ಸ್ವಾಮೀಜಿ, ಮರುವರ್ಷ ಫೆಬ್ರವರಿಯಲ್ಲಿ ತಮ್ಮ ಗುರುಭಾಯಿಗಳಿಂದ ಬೇರ್ಪಟ್ಟು ಏಕಾಂಗಿಯಾಗಿ ಹೊರಟರು. ಇದಾದ ಮೇಲೆ ಅವರು ತಮ್ಮ ಗುರುಭಾಯಿಗಳನ್ನು ಸಂಧಿಸಿದ್ದು ಮತ್ತು ಸಂಪರ್ಕಿಸಿದ್ದು ಆಗೊಮ್ಮೆ ಈಗೊಮ್ಮೆ ಮಾತ್ರ. ಆದ್ದರಿಂದ ಅವರ ಮೆದುಳಿನಲ್ಲಿ ನಡೆಯುತ್ತಿದ್ದ ಮಹಾ ಆಂದೋಳನದ ಸುಳಿವು ಅವರ ಸೋದರ ಸಂನ್ಯಾಸಿಗಳಿಗೆ ಸಿಗಲಿಲ್ಲ. ಅವು ಗಳನ್ನು ತಾವಾಗಿಯೇ ಇವರಿಗೆ ತಿಳಿಸಲೂ ಇಲ್ಲ ಸ್ವಾಮೀಜಿ. ಹೀಗಾಗಿ, ಅಮೆರಿಕೆಯಿಂದ ಪತ್ರಗಳ ಮೂಲಕ ತಮ್ಮ ಯೋಜನೆಗಳನ್ನು ಅವರು ಹೊರಗೆಡಹಿದಾಗ ಅವರ ಸೋದರಸಂನ್ಯಾಸಿಗಳಿಗೆ ಅವು ಆಘಾತವನ್ನೇ ಉಂಟುಮಾಡಿದುವೆಂದರೂ ಆಶ್ಚರ್ಯವಿಲ್ಲ.

ಸರ್ವಧರ್ಮ ಸಮ್ಮೇಳನದಲ್ಲಿ ಅದ್ಭುತ ಯಶಸ್ಸನ್ನು ಗಳಿಸಿ ವಿಖ್ಯಾತರಾಗಿ, ಅವರ ಕೀರ್ತಿ ಭಾರತಕ್ಕೆ ಹರಿದು ಬಂದ ಮೇಲೂ ಸ್ವಾಮೀಜಿ ತಮ್ಮ ಗುರುಭಾಯಿಗಳಿಗೆ ಪತ್ರ ಬರೆದಿರಲಿಲ್ಲ. ಅಮೆರಿಕದಲ್ಲಿನ ಮೊದಲ ಕೆಲ ತಿಂಗಳುಗಳ ಅನುಭವಗಳ ಬಗ್ಗೆ ಅವರು ಬರೆದದ್ದು ತಮ್ಮ ಮದ್ರಾಸೀ ಶಿಷ್ಯರಿಗೆ ಮಾತ್ರ. ಅವರು ತಮ್ಮ ಗುರುಭಾಯಿಗಳಿಗೆ ಪ್ರಥಮ ಪತ್ರವನ್ನು ಬರೆದದ್ದು ೧೮೯೪ರ ಜನವರಿಯಲ್ಲಿ. ಆದರೆ ಇದಾದ ನಂತರ ೧೮೯೭ರ ಜನವರಿಯಲ್ಲಿ ಅವರು ಭಾರತಕ್ಕೆ ಹಿಂದಿರುಗುವವರೆಗೂ ಸೋದರ ಸಂನ್ಯಾಸಿಗಳಿಗೆ ಅವರಿಂದ ನಿರಂತರವಾಗಿ ಪತ್ರಗಳು ಬರುತ್ತಿದ್ದುವು.

ಹೀಗೆ ಪ್ರಾರಂಭವಾದ ವಿಚಾರಧಾರೆಯು ಸ್ವಾಮೀಜಿಯ ಮನಸ್ಸು ಹೃದಯಗಳನ್ನು ಅವರ ಸೋದರ ಸಂನ್ಯಾಸಿಗಳಿಗೆ ಸ್ಪಷ್ಟವಾಗಿ ತೆರೆದುತೋರಿತು. ಇವುಗಳಲ್ಲಿ ಆ ಸಂನ್ಯಾಸಿಗಳು ತಮ್ಮ ಹಿಂದಿನ ಪ್ರಿಯ ನರೇಂದ್ರನನ್ನೇ ಕಂಡರಾದರೂ ತಾವು ಹಿಂದೆ ಕಂಡದ್ದಕ್ಕಿಂತಲೂ ಸಹಸ್ರಪಾಲು ತ್ರಿವಿಕ್ರಮಾಕಾರವಾಗಿ ಅವನು ಬೆಳೆದು ನಿಂತದ್ದನ್ನು ಗುರುತಿಸಿದರು. ಇವರಿಗೆ ಸ್ವಾಮೀಜಿ ಬಂಗಾಳಿಯಲ್ಲಿ ಬರೆದ ಪತ್ರಗಳಲ್ಲಿ, ಒಮ್ಮೆ ಅತ್ಯಂತ ಭಾವಭರಿತರಾಗಿ, ಇನ್ನೊಮ್ಮೆ ಪ್ರವಾದಿ ಯೋಪಾದಿಯಲ್ಲಿ, ಮತ್ತೊಮ್ಮೆ ಪ್ರಬಲ ಪ್ರೋತ್ಸಾಹಕ ಶಕ್ತಿಯಾಗಿ, ಮಗದೊಮ್ಮೆ ಹಾಸ್ಯರಸ ಭರಿತವಾಗಿ ಗಂಭೀರವಾದ ಹಾಗೂ ಭಾವೋದ್ದೀಪಕವಾದ ಸಂದೇಶವನ್ನು ಹರಿಸಿದರು. ಇಂತಹ ಸಂದೇಶವನ್ನು ಅರ್ಥಮಾಡಿಕೊಳ್ಳಲು ಆ ಸೋದರ ಸಂನ್ಯಾಸಿಗಳಿಗೆ ಸಹಜವಾಗಿಯೇ ಸ್ವಲ್ಪ ಕಾಲ ಹಿಡಿಯಿತು.

ಈ ಹಿಂದಿನ ಆರು ತಿಂಗಳಲ್ಲಿ ಅವರು ತಮ್ಮ ಮದರಾಸೀ ಶಿಷ್ಯರ ಹೃದಯದೊಳಕ್ಕೆ ತಮ್ಮ ಭಾವನೆಗಳನ್ನು ತುಂಬಿ ಭಾರತದ ದುಃಸ್ಥಿತಿಯ ಬಗ್ಗೆ ಅವರನ್ನು ಹೇಗೆ ಎಚ್ಚರಗೊಳಿಸಿದರೋ, ಅದರಂತೆಯೇ ಈಗ ತಮ್ಮ ಸೋದರಸಂನ್ಯಾಸಿಗಳಲ್ಲೂ ಆ ಅರಿವನ್ನು ಮೂಡಿಸಲು ಉದ್ಯುಕ್ತರಾ ದರು. ಅವರು ಆಲಂಬಜಾರ್ ಮಠಕ್ಕೆ ಬರೆದ ಮೊದಲ ಪತ್ರದಲ್ಲಿ ತಮ್ಮ ಕಲ್ಪನೆಗಳನ್ನೂ ಯೋಜನೆಗಳನ್ನೂ ಸಂಕ್ಷಿಪ್ತವಾಗಿ ತಿಳಿಸಿದರು:

“ಲಕ್ಷಾಂತರ ಜನರು ಕೇವಲ ಮೊಹುವಾ ಗಿಡಿದ ಹೂಗಳನ್ನು ತಿಂದು ಬದುಕುವಂತಹ ದೇಶ ಅದು; ಇಂತಹ ಬಡಜನರ ರಕ್ತವನ್ನೇ ಹೀರುತ್ತ ಅವರ ಒಳಿತಿಗಾಗಿ ಸ್ವಲ್ಪವೂ ಪ್ರಯತ್ನಿಸದ ಸುಮಾರು ಒಂದು ಕೋಟಿ ಬ್ರಾಹ್ಮಣರು ಹಾಗೂ ಒಂದೆರಡು ಮಿಲಿಯ ಸಾಧುಗಳು ತುಂಬಿರು ವಂತಹ ಆ ದೇಶವು ಒಂದು ರಾಷ್ಟ್ರವೋ ಅಥವಾ ನರಕವೋ? ಇದೇನು ಒಂದು ಧರ್ಮವೋ ಅಥವಾ ಪಿಶಾಚನರ್ತನವೋ?

“ಸೋದರ, ಇವುಗಳನ್ನೆಲ್ಲ, ಅದರಲ್ಲೂ ವಿಶೇಷವಾಗಿ ಬಡತನ ಹಾಗೂ ಅಜ್ಞಾನವನ್ನು ಕಂಡು ನನಗೆ ನಿದ್ರೆಯೇ ಇರಲಿಲ್ಲ. ಕನ್ಯಾಕುಮಾರಿಯಲ್ಲಿ ಜಗನ್ಮಾತೆಯ ದೇವಾಲಯದಲ್ಲೂ ಭಾರತದ ತುತ್ತತುದಿಯ ಬಂಡೆಯ ಮೇಲೂ ಕುಳಿತು ಧ್ಯಾನಮಗ್ನನಾದಾಗ ನನಗೊಂದು ಆಲೋಚನೆ ಹೊಳೆಯಿತು–ನಾವು ಇಷ್ಟೊಂದು ಜನ ಸಂನ್ಯಾಸಿಗಳು ನಾಡಿನ ತುಂಬೆಲ್ಲ ಜನರಿಗೆ ತತ್ತ್ವ ಬೋಧನೆ ಮಾಡುತ್ತ ತಿರುಗಾಡಿಕೊಂಡಿದ್ದೇವೆ. ಆದರೆ ಇದೆಲ್ಲ ಎಂತಹ ಹುಚ್ಚುತನ! ನಮ್ಮ ಗುರುದೇವರು ಹೇಳಲಿಲ್ಲವೆ–‘ಹಸಿದ ಹೊಟ್ಟೆ ಧರ್ಮಭೋದನೆಗೆ ಯೋಗ್ಯವಲ್ಲ’ ಎಂದು? ಅಜ್ಞಾನ-ದಾರಿದ್ರ್ಯಗಳಿಂದಾಗಿ ಬಡಜನರು ಮೃಗಗಳಂತೆ ಜೀವಿಸುತ್ತಿದ್ದಾರೆ. ನಾವು ಯುಗಯುಗ ಗಳಿಂದಲೂ ಅವರ ರಕ್ತವನ್ನು ಹೀರುತ್ತ ಅವರನ್ನು ಕಾಲಿನಡಿಯಲ್ಲಿ ಹೊಸಕುತ್ತಿದ್ದೇವೆ.

“ಇತರ ಒಳಿತಿಗಾಗಿಯೇ ಪಣತೊಟ್ಟುನಿಂತ ಕೆಲವು ನಿಃಸ್ವಾರ್ಥಿ ಸಂನ್ಯಾಸಿಗಳು ಹಳ್ಳಿಹಳ್ಳಿಗೂ ಹೋಗಿ ಈ ಜನರಿಗೆ ವಿದ್ಯಾಭ್ಯಾಸವನ್ನು ನೀಡುತ್ತ ಚಂಡಾಲರವರೆಗೂ ಪ್ರತಿಯೊಬ್ಬರ ಒಳಿತಿನ ಮಾರ್ಗವನ್ನು ಹುಡುಕಬೇಕು; ಭೂಪಟ, ಕ್ಯಾಮರಾ, ಭೂಗೋಳಗಳ ಮೂಲಕ ಎಲ್ಲರಿಗೂ ಬೋಧಿಸಬೇಕು; ಇದು ಕಾಲಕ್ರಮದಲ್ಲಿ ಒಳಿತನ್ನು ತರಲಾರದೇನು ಎಂದು ನಾನು ಆಲೋಚಿಸಿದೆ. ನನ್ನ ಎಲ್ಲ ಯೋಜನೆಗಳನ್ನು ಈ ಚಿಕ್ಕ ಪತ್ರದಲ್ಲಿ ಬರೆಯಲು ಸಾಧ್ಯವಿಲ್ಲ. ಒಟ್ಟಿನಲ್ಲಿ ಸಾರಾಂಶವೇನೆಂದರೆ ‘ಬೆಟ್ಟ ಮಹಮ್ಮದನ ಬಳಿಗೆ ಬರದಿದ್ದರೆ ಮಹಮ್ಮದನೇ ಬೆಟ್ಟದ ಬಳಿಗೆ ಹೋಗಬೇಕು.’ (ಇಂಗ್ಲೀಷ್ ಗಾದೆ.) ನಮ್ಮ ದೇಶದ ಬಡವರೆಲ್ಲ ಪಾಠಶಾಲೆಗೂ ಹೋಗಲಾರ ದಷ್ಟು ಬಡವರು. ಅಲ್ಲದೆ ಕಾವ್ಯ-ಸಾಹಿತ್ಯಗಳನ್ನು ಓದುವುದರಿಂದ ಅವರಿಗೇನೂ ಪ್ರಯೋಜನ ವಿಲ್ಲ. ಅವರಿಗೆ ಅತ್ಯಾವಶ್ಯಕವಾದ ಮೂಲಭೂತ ವಿದ್ಯಾಭ್ಯಾಸವನ್ನು ಕೊಡಬೇಕು. ನಾವು ನಮ್ಮ ತನವನ್ನೇ ಕಳೆದುಕೊಂಡುಬಿಟ್ಟಿದ್ದೇವೆ. ಭಾರತದ ಎಲ್ಲ ಪೀಡೆಗಳಿಗೂ ಅದೇ ಕಾರಣ. ನಾವು ನಮ್ಮ ರಾಷ್ಟ್ರಕ್ಕೆ ಅದು ಕಳೆದುಕೊಂಡ ತನ್ನತನವನ್ನು ಮರಳಿಕೊಡಿಸಬೇಕು; ಮತ್ತು ಜನಸಾಮಾನ್ಯರನ್ನು ಮೇಲೆತ್ತಬೇಕು. ಹಿಂದೂಗಳು, ಮುಸಲ್ಮಾನರು ಮತ್ತು ಕ್ರೈಸ್ತರು–ಎಲ್ಲರೂ ಜನಸಾಮಾನ್ಯ ರನ್ನು ಕಾಲಿನಡಿಯಲ್ಲಿ ಹಾಕಿ ಮೆಟ್ಟಿದ್ದಾರೆ. ಈಗ ಅವರನ್ನು ಮೇಲೆತ್ತುವ ಕಾರ್ಯ ನಮ್ಮವ ರಿಂದಲೇ, ಎಂದರೆ ಸಂಪ್ರದಾಯಸ್ಥ ಹಿಂದುಗಳಿಂದಲೇ ಆಗಬೇಕಾಗಿದೆ. ಪ್ರತಿಯೊಂದು ರಾಷ್ಟ್ರ ದಲ್ಲೂ ದೋಷವಿರುವುದು ಧರ್ಮದಲ್ಲಲ್ಲ, ಜನರಲ್ಲಿ. ಆದ್ದರಿಂದ ದೂರಬೇಕಾದದ್ದು ಧರ್ಮವನ್ನಲ್ಲ, ಜನರನ್ನು.

“ಇದನ್ನು ಸಾಧಿಸಲು ನಮಗೆ ಮೊಟ್ಟಮೊದಲಿಗೆ ಬೇಕಾದುದೆಂದರೆ ಜನ, ಎರಡನೆಯದು ಧನ. ಗುರುದೇವರ ಕೃಪೆಯಿಂದ ಪ್ರತಿಯೊಂದು ಊರಿನಲ್ಲೂ ಹತ್ತು ಹದಿನೈದು ಜನರನ್ನು ಗೆದ್ದು ಕೊಳ್ಳುವ ಭರವಸೆ ನನಗಿತ್ತು. ಬಳಿಕ ನಾನು ಧನವನ್ನರಸಿ ಸುತ್ತಾಡಿದೆ. ಆದರೆ ನಮ್ಮ ಭಾರತದ ಜನ ಹಣ ಬಿಚ್ಚಿಯಾರೆಂದು ನಿನಗನಿಸುತ್ತದೆಯೆ! ಸ್ವಾರ್ಥದ ಮೂರ್ತಿಗಳು! ಇಂಥವರು ಹಣ ಖರ್ಚು ಮಾಡುತ್ತಾರೆಯೆ? ಆದ್ದರಿಂದ ನಾನು ನಾನಾಗಿಯೇ ಹಣ ಗಳಿಸಲು ಅಮೆರಿಕೆಗೆ ಬಂದಿ ದ್ದೇನೆ. ಬಳಿಕ ನಾನು ನನ್ನ ಜೀವನದ ಉಳಿದ ದಿನಗಳನ್ನು ಒಂದು ಉದ್ದೇಶಸಿದ್ಧಿಗಾಗಿ ವಿನಿಯೋಗಿ ಸಲು ಭಾರತಕ್ಕೆ ಮರಳುತ್ತೇನೆ.

“ನಮ್ಮ ದೇಶವು ಸಾಮಾಜಿಕ ನಡವಳಿಕೆಗಳಲ್ಲಿ ಹಿಂದುಳಿದಿರುವಂತೆಯೇ ಈ ದೇಶವು ಆಧ್ಯಾತ್ಮಿಕತೆಯಲ್ಲಿ ಹಿಂದುಳಿದಿವೆ. ಅವರಿಗೆ ನಾನು ಅಧ್ಯಾತ್ಮವನ್ನು ಕೊಡುತ್ತೇನೆ, ಅವರು ನನಗೆ ಹಣ ಕೊಡುತ್ತಾರೆ. ನನ್ನ ಗುರಿಯನ್ನು ಸಾಧಿಸಲು ನನಗೆ ಎಷ್ಟು ಕಾಲ ಬೇಕೋ ತಿಳಿಯದು... ಈ ಜನಗಳು ಮಿಥ್ಯಾಚಾರಿಗಳಲ್ಲ, ಅಸೂಯೆಯಂತೂ ಇಲ್ಲವೇ ಇಲ್ಲ. ಹಿಂದೂಸ್ಥಾನದಲ್ಲಿ ನಾನು ಯಾರನ್ನೂ ಅವಲಂಬಿಸಿಕೊಳ್ಳುವುದಿಲ್ಲ. ನಾನು ನನ್ನ ಉದ್ದೇಶಸಿದ್ಧಿಗೆ ಬೇಕಾದುದನ್ನೆಲ್ಲ ನನ್ನ ಕೈಯಲ್ಲಿ ಸಾಧ್ಯವಾದಷ್ಟು ಪ್ರಯತ್ನಿಸಿ ಪಡೆಯುತ್ತೇನೆ. ಮತ್ತು ನನ್ನ ಯೋಜನೆಗಳನ್ನು ಕಾರ್ಯಗತ ಗೊಳಿಸುತ್ತೇನೆ. ಇಲ್ಲವೆ ಈ ಪ್ರಯತ್ನದಲ್ಲೇ ಮಡಿಯುತ್ತೇನೆ. ‘ಸಾವು ನಿಶ್ಚಯವಾಗಿರುವಾಗ ಒಂದು ಒಳ್ಳೆಯ ಉದ್ದೇಶಕ್ಕಾಗಿ ತನ್ನನ್ನು ಸಮರ್ಪಿಸಿಕೊಳ್ಳುವುದೇ ಶ್ರೇಷ್ಠ.’

“ಇದೆಲ್ಲ ಎಂತಹ ಅಸಂಬದ್ಧ ಪ್ರಲಾಪ ಎಂದು ಬಹುಶಃ ನೀನು ಆಲೋಚಿಸಬಹುದು. ನನ್ನ ಅಂತರಂಗದೊಳಗೆ ಏನಾಗುತ್ತಿದೆಯೆಂಬುದನ್ನು ನೀನೇನು ಬಲ್ಲೆ? ನೀವು ಯಾರಾದರೂ ಈ ನನ್ನ ಯೋಜನೆಗಳಲ್ಲಿ ನನಗೆ ನೆರವಾದರೆ ಸಂತೋಷ. ಇಲ್ಲದಿದ್ದರೆ ಗುರುದೇವನೇ ದಾರಿ ತೋರುತ್ತಾನೆ.”

ಸ್ವಾಮೀಜಿ ಮಠಕ್ಕೆ ಬರೆದ ಮತ್ತೊಂದು ಪತ್ರ ತುಂಬ ಪ್ರಭಾವಪೂರ್ಣವಾಗಿದೆ. ಅದರ ಒಂದಂಶ ಹೀಗಿದೆ.

“ನಾನು ನಿಮಗೊಂದು ಹೊಸ ಉಪಾಯವನ್ನು ತೋರಿಸುತ್ತೇನೆ. ನೀವದನ್ನು ಕಾರ್ಯಗತ ಗೊಳಿಸಬಲ್ಲಿರಾದರೆ, ನಿಜಕ್ಕೂ ನೀವೀಗ ಪುರುಷರು ಹಾಗೂ ಪ್ರಯೋಜನಕ್ಕೆ ಬರುವವರು ಎಂದು ತಿಳಿಯುತ್ತೇನೆ.

“ನಮ್ಮ ದೇಶದ ಹಳ್ಳಿಗಳಲ್ಲಿ ಲಕ್ಷಾಂತರ ಬಡವರು, ಅಜ್ಞಾನಿಗಳು ಇದ್ದಾರೆ. ಅವರ ಗುಡಿಸಲು ಗಳಿಗೆ ಹೋಗಿ–ಸಂಜೆಯೋ ಮಧ್ಯಾಹ್ನವೋ ಯಾವಾಗಲಾದರೂ ಸರಿಯೆ–ಅವರ ಕಣ್ತೆರೆಸಲು ಪ್ರಯತ್ನಿಸಿ. ಪುಸ್ತಕಗಳಿಂದೇನೂ ಪ್ರಯೋಜನವಿಲ್ಲ. ಅವರೊಂದಿಗೆ ಮಾತುಕತೆಯಾಡುತ್ತ ವಿಷಯಗಳನ್ನು ತಿಳಿಸಿಕೊಡಿ. ಬಳಿಕ ನಿಮ್ಮ ಕೇಂದ್ರಗಳನ್ನು ಕ್ರಮೇಣ ವಿಸ್ತರಿಸಿ. ಇದನ್ನೆಲ್ಲ ನೀವು ಮಾಡಬಲ್ಲಿರಾ? ಅಥವಾ ಸಮ್ಮನೆ ಪೂಜೆಯ ಗಂಟೆಯನ್ನು ಅಲ್ಲಾಡಿಸುತ್ತ ಕುಳಿತಿರುತ್ತೀರೋ?...

“ಬನ್ನಿ, ಈ ಕೆಲಸದಲ್ಲಿ ನಿಮ್ಮನ್ನು ಸಂಪೂರ್ಣವಾಗಿ ತೊಡಗಿಸಿಕೊಳ್ಳಿ. ಕಾಡುಹರಟೆಯ ಹಾಗೂ ಪೂಜೆ-ಗೀಜೆಯ ದಿನ ಆಗಿಹೋಯಿತು. ನನ್ನ ಹುಡುಗರೇ, ನೀವೀಗ ಕಾರ್ಯಮಗ್ನರಾಗಲೇಬೇಕು... ಪೂಜಾದಿಗಳೆಲ್ಲ ಗೃಹಸ್ಥರಿಗೆ ಹೇಳಿಸಿದ್ದು. ನಿಮ್ಮ ಕೆಲಸವೇನೆಂದರೆ ಆಲೋಚನಾ ತರಂಗಗಳನ್ನು ಉತ್ಪಾದಿಸಿ ಅವುಗಳನ್ನು ಎಲ್ಲೆಡೆ ಹರಡುವುದು.

“ಯಾವನು ಪ್ರತಿಯೊಂದು ಜೀವಿಯ ಬಗ್ಗೆಯೂ ತೀವ್ರ ಅನುಕಂಪೆ ತಾಳಬಲ್ಲನೋ ಮತ್ತು ತಾನೇ ನರಕಕ್ಕೆ ಹೋಗುವ ಅಪಾಯವನ್ನೆದುರಿಸಿಯೂ ಇತರರಿಗಾಗಿ ಶ್ರಮಿಸುತ್ತಾನೆಯೋ ಅವನು ಮಾತ್ರವೇ ಶ್ರೀರಾಮಕೃಷ್ಣರ ಪುತ್ರ. ಉಳಿದವರೆಲ್ಲ ಭಂಡರು. ಈ ಮಹಾ ಆಧ್ಯಾತ್ಮಿಕ ಸಂಧಿಕಾಲ ದಲ್ಲಿ ಯಾವನು ಧೀರ ಹೃದಯದಿಂದ ಎದ್ದು ನಿಂತು, ಶ್ರೀರಾಮಕೃಷ್ಣರ ಸಂದೇಶವನ್ನು ಸಾರುತ್ತ ಮನೆಯಿಂದ ಮನೆಗೆ, ಹಳ್ಳಿಯಿಂದ ಹಳ್ಳಿಗೆ ಸಾಗುತ್ತಾನೆಯೋ ಅವನು ಮಾತ್ರವೇ ನನ್ನ ಸಹೋದರ; ಅವನು ಮಾತ್ರವೇ ಶ್ರೀರಾಮಕೃಷ್ಣರ ಪುತ್ರ. ಇದೇ ನಿಜವಾದ ಪರೀಕ್ಷೆ–ಯಾವನು ಶ್ರೀರಾಮಕೃಷ್ಣರ ಪುತ್ರನೋ ಅವನು ತನ್ನ ಸ್ವಂತ ಒಳಿತನ್ನು ಅರಸುವುದಿಲ್ಲ.”

ಹೀಗೆ ಅವರ ಪತ್ರಗಳಲ್ಲಿ ಮತ್ತೆಮತ್ತೆ ಮರುಕಳಿಸುತ್ತಿದ್ದ ಸಂದೇಶವೆಂದರೆ, ‘ಹಳ್ಳಿಗಳಿಗೆ ಹೋಗಿ, ಬಡವರ ಗುಡಿಸಲುಗಳಿಗೆ ಹೋಗಿ; ಬಡವರನ್ನು ದೀನರನ್ನು ಆರ್ತರನ್ನು ಎಚ್ಚರಗೊಳಿಸಿ; ಅವರಿಗೆ ನೆರವಾಗಿ, ಅವರನ್ನು ಮೇಲೆಬ್ಬಿಸಿ’–ಇದೊಂದೇ! ತಮ್ಮ ನೆಚ್ಚಿನ ಗುರುಭಾಯಿಯಾದ ಅಖಂಡಾನಂದರಿಗೆ ಒಂದು ಪತ್ರದಲ್ಲಿ ಅದೇ ಆದೇಶವನ್ನೇ ಮತ್ತಷ್ಟು ಸ್ಪಷ್ಟವಾಗಿ ತಿಳಿಸುತ್ತ, ಅದರೊಂದಿಗೆ ಸಂನ್ಯಾಸ ಜೀವನದ ಕುರಿತಾದ ತಮ್ಮ ಕ್ರಾಂತಿಕಾರಿ ಭಾವನೆಗಳನ್ನು ಸೇರಿಸಿ ಬರೆದರು:

“ಖೇತ್ರಿ ನಗರದ ಬಡವರ ಹಾಗೂ ಹಿಂದುಳಿದವರ ಮನೆಯಿಂದ ಮನೆಗೆ ಹೋಗಿ ಅವರಿಗೆ ಬೋಧನೆ ಮಾಡು. ಮತ್ತು ಅದರೊಂದಿಗೇ ಭೂಗೋಳ ಮತ್ತಿತರ ವಿಷಯಗಳನ್ನು ತಿಳಿಸಿಕೊಡು. ಆಗಾಗ ಇತರ ಹಳ್ಳಿಗಳಿಗೂ ಹೋಗು... ಕೆಲಸ, ಸೇವೆ ಮತ್ತು ಜ್ಞಾನ–ಮೊದಲು ಕೆಲಸ; ಅದರಿಂದ ನಿನ್ನ ಮನಸ್ಸು ಶುದ್ಧವಾಗುತ್ತದೆ. ಇಲ್ಲದಿದ್ದರೆ, ಹವಿಸ್ಸನ್ನು ಪವಿತ್ರವಾದ ಅಗ್ನಿಯ ಮೇಲೆ ಸುರಿಯುವುದರ ಬದಲಾಗಿ ಬೂದಿಯ ರಾಶಿಯ ಮೇಲೆ ಸುರಿದಂತೆ ಎಲ್ಲವೂ ವ್ಯರ್ಥ ವಾಗುತ್ತದೆ... ನೀನು ತೆಗೆದುಕೊಳ್ಳುವ ಆಹಾರದ ಬಗ್ಗೆ ಜನರು ಆಕ್ಷೇಪಿಸಿದರೆ, ತಕ್ಷಣ ಅದನ್ನು ಬಿಟ್ಟುಬಿಡು. ಇತರರಿಗೆ ಒಳ್ಳೆಯದು ಮಾಡುವುದಕ್ಕಾಗಿ ಹುಲ್ಲುತಿಂದು ಜೀವಿಸಿದರೂ ಚಿಂತೆ ಇಲ್ಲ. ಕಾಷಾಯ ವಸ್ತ್ರವಿರುವುದು ಸುಖಭೋಗಕ್ಕಲ್ಲ. ಅದು ಧೀರತೆಯ ಗುರುತು...”

ಹಿಂದೂ ಸಂನ್ಯಾಸ ಧರ್ಮದ ಸಂಪ್ರದಾಯಕ್ಕೆ ಹೊಚ್ಚಹೊಸತಾದ ಇಂತಹ ಭಾವನೆಗಳು ಸ್ವಾಮೀಜಿಯ ಲೇಖನಿಯ ಮೂಲಕ ದಿಗ್ಭ್ರಮೆಗೊಳಿಸುವಂತಹ ರಭಸದಿಂದ ಮೂಡಿಬಂದುವು. ಇತರ ಸಂನ್ಯಾಸಿಗಳಾಗಿದ್ದರೆ ಸ್ವಾಮೀಜಿಯನ್ನು ಒಬ್ಬ ಹುಚ್ಚನೆಂದೋ ಭ್ರಾಂತನೆಂದೋ ಕರೆದು ಬಹಿಷ್ಕಾರ ಹಾಕಿಬಿಡುತ್ತಿದ್ದರು. ಈ ವೇಳೆಗಾಗಲೇ, ಬಾಹ್ಯ ಪ್ರಪಂಚವನ್ನೇ ಮರೆತು ತಮ್ಮ ಆಧ್ಯಾತ್ಮಿಕ ಸಾಧನೆಯಲ್ಲಿ ಮುಳುಗಿಹೋಗಿದ್ದ ಅವರ ಸೋದರ ಸಂನ್ಯಾಸಿಗಳೂ ಇಂತಹ ವಿಚಿತ್ರ ಆಲೋಚನೆಗಳನ್ನು ಕೇಳಿ ಹಾಗೆಯೇ ತೀರ್ಮಾನಿಸಿಬಿಡುತ್ತಿದ್ದರೇನೋ. ಆದರೆ ಈ ಸೋದರರು ಸ್ವಾಮೀಜಿಯ ಅದೇ ಪತ್ರಗಳಲ್ಲಿ ಶ್ರೀರಾಮಕೃಷ್ಣರ ಮೇಲಿನ ಅವರ ಪರಿಪೂರ್ಣ ನಿಷ್ಠೆಯೂ ಭಕ್ತಿಯೂ ಕೋರೈಸುವ ಪರಿಯನ್ನು ಕಂಡು ಬೆರಗಾದರು. ಆದ್ದರಿಂದ ಅವರು ಸ್ವಾಮೀಜಿಯ ಈ ಆಲೋಚನೆಗಳೆಲ್ಲ ಪಾಶ್ಚಾತ್ಯ ಚಿಂತನೆಯ ಪ್ರಭಾವ ಎಂದು ತೀರ್ಮಾನಿಸಿ ತಳ್ಳಿಹಾಕಲು ಸಾಧ್ಯ ವಿರಲಿಲ್ಲ. ತಮ್ಮ ಗುರುದೇವನ ಮೇಲೆ ನರೇಂದ್ರನ ಭಕ್ತಿಯು ಸರಿಸಾಟಿಯಿಲ್ಲದ್ದು ಎಂಬುದು ಅವರಿಗೆಲ್ಲ ಸ್ಪಷ್ಟವಾಗಿ ಗೋಚರವಾಯಿತು. ಅಲ್ಲದೆ ನರೇಂದ್ರನ ಎಲ್ಲ ಕಾರ್ಯೋದ್ದೇಶಗಳ ಹಿಂದಿನ ಸ್ಫೂರ್ತಿಯು ಶ್ರೀರಾಮಕೃಷ್ಣರ ಬೋಧನೆಗಳೇ ಎಂಬುದೂ ಅವರಿಗೆ ಅರಿವಾಯಿತು.

ಎಲ್ಲಕ್ಕಿಂತ ಹೆಚ್ಚಾಗಿ, ಶ್ರೀರಾಮಕೃಷ್ಣರೇ ನರೇಂದ್ರನ ಬಗ್ಗೆ ಏನೇನು ಹೇಳಿದ್ದರೆಂಬುದನ್ನು ಅವರ್ಯಾರೂ ಮರೆತಿರಲಿಲ್ಲ. ಶ್ರೀರಾಮಕೃಷ್ಣರು ಸಶರೀರಿಯಾಗಿದ್ದಾಗಿನಿಂದಲೂ ನರೇಂದ್ರನ ಸ್ಥಾನವೆಂಥದು! ಆ ಯುವಶಿಷ್ಯರನ್ನೆಲ್ಲ ತನ್ನ ಅಲೌಕಿಕ ವ್ಯಕ್ತಿತ್ವದಿಂದ, ಶಕ್ತಿಯಿಂದ ಮರುಳು ಗೊಳಿಸಿದ್ದವನಲ್ಲವೆ ಅವನು? ತಮ್ಮ ಬಳಿಕ ನರೇಂದ್ರನೇ ಅವರೆಲ್ಲರ ನಾಯಕನೆಂದು ಶ್ರೀರಾಮ ಕೃಷ್ಣರೇ ಅದೆಷ್ಟು ಸಲ ಸ್ಪಷ್ಟವಾಗಿ ತಿಳಿಸಿಲ್ಲ? ಆದ್ದರಿಂದ ಈ ಸಂನ್ಯಾಸೀ ಸೋದರರು ಸ್ವಾಮೀಜಿಯ ಮಾತುಗಳನ್ನು ಪ್ರಶ್ನಿಸಲು ಸಾಧ್ಯವೇ ಇರಲಿಲ್ಲ. ಆದರೆ ಸ್ವಾಮೀಜಿಯ ಯೋಜನೆ ಗಳೆಲ್ಲ ಅವರಿಗೆ ಒಪ್ಪಿಗೆಯಾದುವೆಂದಲ್ಲ. ಲೋಕಹಿತ ಕಾರ್ಯವನ್ನು ಕೈಗೊಳ್ಳಬೇಕೇ ಬೇಡವೇ ಎಂಬುದರ ಬಗ್ಗೆ ಅವರಲ್ಲಿ ಮೊದಲಿನಿಂದಲೂ ಭಿನ್ನಾಭಿಪ್ರಾಯಗಳಿದ್ದುದನ್ನು ಹಿಂದೆಯೇ ನೋಡಿದ್ದೇವೆ. ಈ ಮುಖ್ಯ ಸಮಸ್ಯೆ ಕೆಲವರ ಪಾಲಿಗಿನ್ನೂ ಪರಿಹಾರವಾಗಿರಲಿಲ್ಲ. ಆದ್ದರಿಂದ ೧೮೯೭ರಲ್ಲಿ ಸ್ವಾಮೀಜಿ ಭಾರತಕ್ಕೆ ಹಿಂದಿರುಗಿದ ಮೇಲೆ, ತಾವು ಅನುಸರಿಸುತ್ತಿರುವ ಮಾರ್ಗವು ಸರಿಯಾದುದು—ಶ್ರೇಷ್ಠವಾದುದು ಎಂಬುದನ್ನು ಮನವರಿಕೆ ಮಾಡಿಕೊಡುವವರೆಗೆ ಕೆಲವರಲ್ಲಿ ಇನ್ನೂ ವಿಶ್ವಾಸ ಮೂಡಿರಲಿಲ್ಲ. ಆದರೆ ಸ್ವಾಮೀಜಿ ಬರೆಯುತ್ತಿದ್ದ ಪತ್ರಗಳನ್ನು ಓದುತ್ತಿದ್ದಂತೆ, ಅವರು ಶ್ರೀರಾಮಕೃಷ್ಣರ ಒಂದು ಉಪಕರಣವೇ ಆಗಿದ್ದಾರೆಂಬುದು ಈ ಸೋದರರಿಗೆ ಮತ್ತೆ ಮತ್ತೆ ದೃಢವಾಗುತ್ತಿತ್ತು. ೧೮೯೪ರ ಚಳಿಗಾಲದಲ್ಲಿ ಸ್ವಾಮಿ ಶಿವಾನಂದರಿಗೆ ಸ್ವಾಮೀಜಿ ಬರೆದ ಈ ಪತ್ರ ನೋಡಿ:

“ನನ್ನ ಪ್ರಿಯ ಸೋದರ, ಶ್ರೀರಾಮಕೃಷ್ಣರು ಭಗವಂತನ ಅವತಾರವೆಂಬುದರಲ್ಲಿ ನನಗೆ ತೃಣ ಮಾತ್ರವಾದರೂ ಸಂಶಯವಿಲ್ಲ. ಆದರೆ ಅವರೇನು ಬೋಧಿಸುತ್ತಿದ್ದರೆಂಬುದನ್ನು ಜನರು ತಾವಾ ಗಿಯೇ ತಿಳಿದುಕೊಳ್ಳಲು ನೀವು ಬಿಡಬೇಕು. ನಮ್ಮ ನಂಬಿಕೆಗಳನ್ನು ನಾವು ಜನರ ಮೇಲೆ ಹೇರು ವಂತಿಲ್ಲ–ಇದೊಂದೇ ನನ್ನ ಆಕ್ಷೇಪಣೆ.

“ಜನರು ತಮ್ಮತಮ್ಮ ನಂಬಿಕೆಗಳನ್ನು ವ್ಯಕ್ತಪಡಿಸಲಿ–ಅದಕ್ಕೆ ನಾವೇಕೆ ಆಕ್ಷೇಪಿಸಬೇಕು? ಮೊದಲು ಶ್ರೀರಾಮಕೃಷ್ಣರನ್ನು ಅಧ್ಯಯನ ಮಾಡದೆ, ವೇದ-ವೇದಾಂತ-ಪುರಾಣಗಳ ನಿಜವಾದ ಅರ್ಥವನ್ನು ಯಾರೂ ಅರಿಯಲಾರರು. ಶ್ರೀರಾಮಕೃಷ್ಣರ ಜೀವನವೆಂಬುದು ಸಮಸ್ತ ಭಾರತೀಯ ಆಧ್ಯಾತ್ಮಿಕ ಚಿಂತನೆಯ ಮೇಲೆ ಹರಿಸಿದಂತಹ ಪ್ರಚಂಡ ಶಕ್ತಿಯ ಬೆಳಕು. ಅವರು ವೇದಗಳ ಹಾಗೂ ಅವುಗಳ ಬೋಧನೆಯ ಬಗ್ಗೆ ಬರೆದ ಜೀವಂತ ಭಾಷ್ಯವಾಗಿದ್ದರು. ಅವರು ತಮ್ಮ ಒಂದೇ ಜೀವನದಲ್ಲಿ, ಭಾರತದ ಧಾರ್ಮಿಕತೆಯ ಸರ್ವಸ್ವವನ್ನೂ ಆಚರಿಸಿ ತೋರಿದರು.

“ಶ್ರೀರಾಮಕೃಷ್ಣರು ಅವತಾರಗಳಲ್ಲಿ ಇತ್ತೀಚಿನವರು, ಮತ್ತು ಶ್ರೇಷ್ಠತಮರಾದವರು; ಜ್ಞಾನ, ಪ್ರೇಮ, ತ್ಯಾಗ, ಧಾರ್ಮಿಕತೆ ಹಾಗೂ ಮಾನವನಿಗೆ ಸೇವೆಗೈಯುವ ಹಂಬಲ–ಇವುಗಳೆಲ್ಲದರ ಪರಿಪೂರ್ಣ ಮೂರ್ತರೂಪ ಅವರು. ಆದ್ದರಿಂದ ಅವರಿಗೆ ಹೋಲಿಕೆಯಾದರೂ ಎಲ್ಲಿದೆ? ಯಾವನು ಶ್ರೀರಾಮಕೃಷ್ಣರನ್ನು ಅರ್ಥಮಾಡಿಕೊಂಡು ಗೌರವಿಸಲಾರನೋ ಅಂಥವನು ಹುಟ್ಟಿದ್ದು ವ್ಯರ್ಥವೇ ಸರಿ! ಜನ್ಮಜನ್ಮಗಳಲ್ಲೂ ನಾನವರ ದಾಸನಾದರೆ ಅದೇ ನನ್ನ ಪರಮ ಸೌಭಾಗ್ಯ. ಅವರ ಒಂದೇ ಒಂದು ಮಾತು ನನ್ನ ಪಾಲಿಗೆ ವೇದವೇದಾಂತಗಳೆಲ್ಲದಕ್ಕಿಂತ ಮಿಗಿಲಾ ದುದು... ಓಹ್, ನಾನು ಅವರ ದಾಸರ ದಾಸರ ದಾಸ. ಆದರೆ ಅತ್ಯಂತ ಕ್ಷುದ್ರವಾದ ಧರ್ಮಾಂಧತೆಯು ಅವರ ತತ್ತ್ವಗಳಿಗೆ ವಿರುದ್ಧವಾದುದು. ಮತ್ತು ಅದೇ ನನ್ನನ್ನು ಸಿಟ್ಟಿಗೇಳು ವಂತೆ ಮಾಡುವುದು. ಶ್ರೀರಾಮಕೃಷ್ಣರ ಹೆಸರೇ ಅಳಿಸಿಹೋದರೂ ಚಿಂತೆಯಿಲ್ಲ. ಅವರ ಬೋಧನೆ ಗಳು ಮಾತ್ರ ಪ್ರಸಾರಗೊಂಡು ಫಲ ಕೊಡುವಂತಾಗಲಿ! ಏಕೆ? ಅವರು ಕೀರ್ತಿಯ ಗುಲಾಮ ರಾಗಿರಲಿಲ್ಲ! ಕೆಲವರು ಮೀನುಗಾರರೂ ನಿರಕ್ಷರಕುಕ್ಷಿಗಳೂ ಏಸುವನ್ನು ದೇವರೆಂದು ಕರೆದರು; ಆದರೆ ಅಕ್ಷರಸ್ಥರಾದವರು ಅವನನ್ನು ಶಿಲುಬೆಗೇರಿಸಿದರು. ಬುದ್ಧನು ತನ್ನ ಜೀವಿತಾವಧಿಯಲ್ಲಿ ಅನೇಕ ವರ್ತಕರಿಂದ, ಗೋವಳರಿಂದ ಗೌರವಿಸಲ್ಪಟ್ಟ. ಆದರೆ ಶ್ರೀರಾಮಕೃಷ್ಣರು ಅವರ ಜೀವಿತಾವಧಿಯಲ್ಲೇ–ಹತ್ತೊಂಬತ್ತನೇ ಶತಮಾನದ ಅಂತ್ಯಕ್ಕೆ–ವಿಶ್ವವಿದ್ಯಾಲಯಗಳ ಮಹಾ ಪ್ರಚಂಡರಿಂದ ಭಗವಂತನ ಅವತಾರವೆಂದು ಕರೆಯಲ್ಪಟ್ಟು ಪೂಜಿಸಲ್ಪಟ್ಟಿದ್ದಾರೆ. ಕೃಷ್ಣ-ಬುದ್ಧ- ಕ್ರಿಸ್ತನೇ ಮೊದಲಾದವರ ಬಗ್ಗೆ ಅವರ ಕಾಲದಲ್ಲೇ ಬರೆಯಲ್ಪಟ್ಟಿರುವುದು ತೀರ ಸ್ವಲ್ಪ. ‘ನಿಜವಾದ ಒಬ್ಬ ಶ್ರೇಷ್ಠ ಗೃಹಸ್ಥನನ್ನು ನಾವಂತೂ ಕಂಡಿಲ್ಲ’ ಎನ್ನುತ್ತದೆ ಹಳೆಯ ಬಂಗಾಳೀ ಗಾದೆ. ಆದರೆ ನಾವು ಹಗಲಿರುಳೂ ಶ್ರೀರಾಮಕೃಷ್ಣರೊಂದಿಗಿದ್ದು ಅವರನ್ನು ಚೆನ್ನಾಗಿ ನೋಡಿದ ಮೇಲೂ ಇತರೆಲ್ಲ ಅವತಾರಗಳಿಗಿಂತಲೂ ಶ್ರೇಷ್ಠರಾದವರು ಇವರೇ ಎಂದು ಪರಿಗಣಿಸಿದ್ದೇವೆ. ಈ ಅದ್ಭುತವನ್ನು ನೀನು ಅರ್ಥಮಾಡಿಕೊಳ್ಳಬಲ್ಲೆಯಾ?”

ಶ್ರೀರಾಮಕೃಷ್ಣರನ್ನು ಸ್ವಾಮೀಜಿ ಯಾವ ದೃಷ್ಟಿಯಲ್ಲಿ ಕಾಣುತ್ತಿದ್ದರೆಂಬುದನ್ನು ಅವರ ಈ ಪತ್ರ ತೆರೆದು ತೋರುತ್ತದೆ. ಅವರ ನಿರ್ಭೀತ, ಅನಿರ್ಬಂಧಿತ ವಾಗ್ಝರಿಯಲ್ಲಿ ಬರೆಯಲ್ಪಟ್ಟ ಈ ಅಪೂರ್ವ ಪತ್ರದಲ್ಲಿ ಅವರ ಅತ್ಯಂತ ವೈಚಾರಿಕ ಬುದ್ಧಿಯನ್ನು ಕಾಣಬಹುದು. ನಿಜಕ್ಕೂ ಹಿಂದೆಯೂ ಶ್ರೀರಾಮಕೃಷ್ಣರ ಮಾಹಾತ್ಮ್ಯವನ್ನು ಅವರ ಸಂನ್ಯಾಸೀ ಶಿಷ್ಯರು ಚೆನ್ನಾಗಿ ಅರ್ಥ ಮಾಡಿಕೊಳ್ಳಲು ಸಾಧ್ಯವಾಗಿದ್ದುದು ನರೇಂದ್ರನ ಮೂಲಕವೇ. ಈಗ ಸ್ವಾಮೀಜಿ ಮತ್ತೊಮ್ಮೆ ಶ್ರೀರಾಮಕೃಷ್ಣರ ಉಪದೇಶಗಳ ನಿಜಾರ್ಥವನ್ನು ವಿವರಿಸುತ್ತ, ತಮ್ಮ ಮುಂದಿನ ಕಾರ್ಯವನ್ನು ಮತ್ತೊಂದು ಪತ್ರದಲ್ಲಿ ಬರೆಯುತ್ತಾರೆ:

“ಕೆಲಸ ಮಾಡಬೇಕು; ನಿರ್ಭಯರಾಗಿ ಕಾರ್ಯಮಗ್ನರಾಗಬೇಕು! ನಮಗಾವ ಭಯ? ನಿಮ್ಮನ್ನು ತಡೆಗಟ್ಟಬಲ್ಲ ಶಕ್ತಿವಂತರು ಯಾರಿದ್ದಾರೆ! ನಾವು ನಕ್ಷತ್ರಗಳನ್ನು ಪುಡಿಗೈಯೋಣ, ವಿಶ್ವವನ್ನೇ ಕಲಕಿಬಿಡೋಣ. ನಾವು ಯಾರೆಂಬುದು ನಿಮಗೆ ಗೊತ್ತಿಲ್ಲವೆ? ನಾವು ಶ್ರೀರಾಮಕೃಷ್ಣರ ದಾಸರು! ಭಯವೆ? ತಮ್ಮನ್ನು ದೇಹಗಳೆಂದು ತಿಳಿಯುವಂತಹ ಮೂರ್ಖರು ಮಾತ್ರವೇ ‘ನಾವು ದುರ್ಬಲರು, ನಾವು ದೀನರು’ ಎಂದು ಕರುಣಾಜನಕವಾಗಿ ಅಳುವುದು. ಇದೆಲ್ಲ ನಾಸ್ತಿಕತೆ. ಭಯವನ್ನು ಮೀರಿದ ಸ್ಥಿತಿಯನ್ನು ತಲುಪಿರುವ ನಾವಿನ್ನು ಭಯಪಡುವುದನ್ನು ನಿಲ್ಲಿಸಿ ಧೀರ ರಾಗೋಣ. ಶ್ರೀರಾಮಕೃಷ್ಣರ ಸೇವಕರಾದ ನಾವು ಅನುಸರಿಸಬೇಕಾದ ಆಸ್ತಿಕತೆ ಇದೇ.

“ಪ್ರಪಂಚದ ಮೇಲಿನ ವ್ಯಾಮೋಹವನ್ನು ತ್ಯಜಿಸಿ, ಅಮರತ್ವದ ಅಮೃತವನ್ನು ನಿರಂತರವಾಗಿ ಹೀರುತ್ತ, ಎಲ್ಲ ತೊಂದರೆಗಳಿಗೂ ಮೂಲವಾದ ಸ್ವಾರ್ಥವನ್ನು ಎಂದೆಂದಿಗೂ ತ್ಯಜಿಸಿ, ಸಕಲ ಸುಖಸ್ವರೂಪವಾದ ಗುರುವಿನ ಪವಿತ್ರ ಪಾದಪದ್ಮಗಳನ್ನು ಅನವರತ ಧ್ಯಾನಿಸುತ್ತ–ಅವುಗಳಿಗೆ ಪುನಃ ಪುನಃ ನಮಿಸುತ್ತ, ಆ ದಿವ್ಯಾಮೃತದಲ್ಲಿ ನಮ್ಮೊಂದಿಗೆ ಪಾಲ್ಗೊಳ್ಳುವಂತೆ ಸಕಲ ಜಗತ್ತನ್ನೇ ಆಹ್ವಾನಿಸೋಣ.

“ವೇದಗಳೆಂಬ ಅನಂತ ಜ್ಞಾನಸಾಗರವನ್ನು ಮಥಿಸುವುದರ ಮೂಲಕ ಉತ್ಪನ್ನವಾದ, ಬ್ರಹ್ಮ ವಿಷ್ಣು ಮಹೇಶ್ವರಾದಿ ದೇವದೇವತೆಗಳು ತಮ್ಮ ಶಕ್ತಿಯನ್ನು ಸುರಿದಂತಹ, ಆ ಅಮೃತವನ್ನು ಶ್ರೀರಾಮಕೃಷ್ಣರು ತಮ್ಮೊಳಗೆ ಸಂಪೂರ್ಣವಾಗಿ ತುಂಬಿಕೊಂಡಿದ್ದಾರೆ.”

ಭಾರತದಲ್ಲಿನ ಆಗುಹೋಗುಗಳ ಬಗ್ಗೆ ತಿಳಿದು ಆ ಕುರಿತಾಗಿ ಮತ್ತೊಂದು ಪತ್ರದಲ್ಲಿ ಹೀಗೆ ಪ್ರತಿಕ್ರಿಯಿಸುತ್ತಾರೆ:

“ಮಜುಮ್​ದಾರನ ಕೃತ್ಯಗಳ ಬಗ್ಗೆ ಕೇಳಿ ನನಗೆ ವಿಷಾದವಾಯಿತು. ಇತರರೆಲ್ಲರಿಗಿಂತ ತಾನು ಮುಂದೆ ಬರಬೇಕೆಂದು ಬಯಸುವವನು ವರ್ತಿಸುವುದು ಹೀಗೆಯೇ. ಮಜುಮ್​ದಾರ ಇಲ್ಲಿಗೆ ಹತ್ತು ವರ್ಷಗಳ ಕೆಳಗೆ ಬಂದ; ಬಹಳ ಕೀರ್ತಿಯನ್ನೂ ಗೌರವವನ್ನೂ ಗಳಿಸಿದ. ಆದರೆ ಈಗ ಇಲ್ಲಿನ ಆಕರ್ಷಣೆಯ ಕೇಂದ್ರ ನಾನು. ಹಾಗಾಗಬೇಕೆಂಬುದು ಗುರುವಿನ ಇಚ್ಛೆ. ಅದಕ್ಕೆ ನಾನೇನು ಮಾಡಲಾದೀತು? ಇಲ್ಲಿನ ಜನಗಳ ವಿವೇಕಪ್ರಜ್ಞೆ ಜಾಗೃತವಾಗಬೇಕೆಂಬುದು ಭಗವಂತನ ಇಚ್ಛೆ. ಅವನು ತನ್ನಿಚ್ಛೆಯಂತೆ ಮಾಡುವುದನ್ನು ತಪ್ಪಿಸಲು ಯಾರಿಗಾದರೂ ಸಾಧ್ಯವೆ? ನನಗೆ ಹೆಸರು-ಕೀರ್ತಿಯೊಂದೂ ಬೇಡ. ನಾನೊಂದು ಅಶರೀರವಾಣಿಯಾಗಿರಬೇಕೆಂಬುದು ನನ್ನಿಚ್ಛೆ. ಯಾರೂ ನನ್ನನ್ನು ಸಮರ್ಥಿಸುವುದು ನನಗೆ ಬೇಕಿಲ್ಲ. ಭಗವಂತನ ಮುನ್ನಡೆಯಲ್ಲಿ ನೆರವಾಗಲು ಅಥವಾ ಅದನ್ನು ತಡೆಯಲು ನಾನ್ಯಾರು? ಅಥವಾ ಇತರರಾದರೂ ಯಾರು? ನಾನೊಬ್ಬ ಯಂತ್ರ, ಅವನು ಚಾಲಕ. ಈ ಉಪಕರಣದ ಮೂಲಕ ಅವನು ಈ ದೂರದ ನಾಡಿನಲ್ಲಿ ಸಾವಿರಾರು ಜನರ ಹೃದಯದಲ್ಲಿ ಧಾರ್ಮಿಕ ಪ್ರಜ್ಞೆಯನ್ನು ಎಚ್ಚರಗೊಳಿಸುತ್ತಿದ್ದಾನೆ. ಇಲ್ಲಿನ ಸಾವಿರಾರು ಸ್ತ್ರೀ-ಪುರು ಷರು ನನ್ನನ್ನು ಪ್ರೀತಿಸುತ್ತಾರೆ, ಪೂಜಿಸುತ್ತಾರೆ. ‘ಮೂಕಂ ಕರೋತಿ ವಾಚಾಲಂ ಪಂಗುಂ ಲಂಘಯತೇ ಗಿರಿಂ’ (ಅವನು ಮೂಕನನ್ನು ವಾಚಾಳಿಯನ್ನಾಗಿ ಮಾಡುತ್ತಾನೆ, ಹೆಳವನು ಬೆಟ್ಟ ವನ್ನೇರುವಂತೆ ಮಾಡುತ್ತಾನೆ.) ನಾನೊಂದು ಅಶರೀರವಾಣಿ ಮಾತ್ರ.

“ಕೈಯಲ್ಲಿ ಬಿಡಿಗಾಸಿಲ್ಲದ ಒಂದು ಆರೆಂಟು ಹುಡುಗರು ಪ್ರಾರಂಭಿಸಿದಂತಹ ಒಂದು ಚಳವಳಿಯು ಈಗ ಭಾರೀ ವೇಗದಿಂದ ಮುನ್ನುಗುತ್ತಿದೆ–ಇದೊಂದು ಕೇವಲ ಬೂಟಾಟಿಕೆಯೋ ಅಥವಾ ಭಗವಂತನ ಇಚ್ಛೆಯೋ?”

ಧರ್ಮಗಳ ಹಾಗೂ ಧಾರ್ಮಿಕ ಚಳವಳಿಗಳ ಚರಿತ್ರೆಯನ್ನು ಆಮೂಲಾಗ್ರವಾಗಿ ಬಲ್ಲವರಾ ಗಿದ್ದ ಸ್ವಾಮೀಜಿ, ಅವುಗಳ ದೌರ್ಬಲ್ಯಗಳನ್ನೂ ಅವುಗಳ ಅವನತಿಗೆ ಕಾರಣವನ್ನೂ ಸ್ಪಷ್ಟವಾಗಿ ಕಂಡುಕೊಂಡಿದ್ದರು. ಅವತಾರವರಿಷ್ಠರಾದ ಶ್ರೀರಾಮಕೃಷ್ಣರಿಂದ ಸಂಸ್ಥಾಪಿತವಾದ ಈ ನವ ಆಂದೋಳನವು ಶಕ್ತಿಯುತವಾಗಿ, ಉಪಯುಕ್ತವಾಗಿ ಮುಂದುವರಿಯಬೇಕಾದರೆ, ಹಿಂದೆ ಘಟಿ ಸಿದ ತಪ್ಪುಗಳು, ಅಚಾತುರ್ಯಗಳು ತಮ್ಮಿಂದ ಮತ್ತೆ ಸಂಭವಿಸದಿರುವಂತೆ ಸ್ವಾಮೀಜಿ ಸದಾ ಜಾಗರೂಕರಾಗಿದ್ದರು. ಇಂತಹ ತಪ್ಪುಗಳಲ್ಲಿ ಅತಿ ಮುಖ್ಯವಾದುದೆಂದರೆ ತತ್ತ್ವವನ್ನು ಬಿಟ್ಟು ವ್ಯಕ್ತಿಯ ಹೆಸರನ್ನೇ ಹಿಡಿದುಕೊಂಡು ಕೊಸರಾಡುವುದು. ಈ ಬಗ್ಗೆ ಸ್ವಾಮೀಜಿ ತಮ್ಮ ಸೋದರ ಸಂನ್ಯಾಸಿಗಳನ್ನು ಮತ್ತೆಮತ್ತೆ ಎಚ್ಚರಿಸಿದರು. ಈ ಕೆಳಗಿನ ಪತ್ರದಲ್ಲಿ ಅದನ್ನು ಸ್ಪಷ್ಟವಾಗಿ ಕಾಣಬಹುದು:

“ಮಹಾತಪಸ್ವಿಗಳು ವಿಶೇಷ ಸಂದೇಶಗಳನ್ನು ಹೊತ್ತು ಭೂಮಿಗೆ ಬರುತ್ತಾರೆ, ಹೆಸರಿಗಾಗಿ ಅಲ್ಲ. ಆದರೆ ಅವರ ಅನುಯಾಯಿಗಳು ಅವರ ಬೋಧನೆಗಳನ್ನು ದೂರಕ್ಕೆಸೆದು, ಆ ವ್ಯಕ್ತಿಯ ಹೆಸರಿನಲ್ಲಿ ಹೊಡೆದಾಡುತ್ತಾರೆ–ಇದೇ ಜಗತ್ತಿನ ಚರಿತ್ರೆ. ಜನಗಳು ಶ್ರೀರಾಮಕೃಷ್ಣರ ಹೆಸರನ್ನು ಸ್ವೀಕರಿಸುತ್ತಾರೆಯೋ ಇಲ್ಲವೋ ಎಂಬುದರ ಬಗ್ಗೆ ನನಗೆ ಚಿಂತೆಯಿಲ್ಲ. ಆದರೆ ಅವರ ಬೋಧನೆ ಗಳು, ಅವರ ಜೀವನ ಹಾಗೂ ಸಂದೇಶಗಳು ಜಗತ್ತಿನ ಎಲ್ಲೆಡೆಗೆ ಪ್ರಸಾರವಾಗುವುದಕ್ಕಾಗಿ ನೆರವಾ ಗಲು ನಾನು ನನ್ನ ಜೀವನವನ್ನು ಸಮರ್ಪಿಸಲು ಸಿದ್ದನಿದ್ದೇನೆ.”

ಅದೇ ಧಾಟಿಯಲ್ಲಿ ಬರೆದ ಮತ್ತೊಂದು ಪತ್ರ:

“ನಾವು ಎಲ್ಲರೊಂದಿಗೂ ಬೆರೆಯಬೇಕು, ಯಾರನ್ನೂ ಎದುರುಹಾಕಿಕೊಳ್ಳಬಾರದು. ಪ್ರತಿ ಯೊಬ್ಬರೂ ನಮ್ಮ ಗುರುವನ್ನು ಸ್ವೀಕರಿಸಲೇಬೇಕೆಂದು ಒತ್ತಾಯಿಸಲು ಹೋಗಬೇಡಿ... ನೀವು ಎಲ್ಲಿಗೇ ಹೋದರೂ ಅಲ್ಲೆಲ್ಲ ಒಂದೊಂದು ಶಾಶ್ವತವಾದ ಬೋಧನಾಕೇಂದ್ರವನ್ನು ತೆರೆಯ ಬೇಕು. ಆಗ ಮಾತ್ರವೇ ಜನರು ಬದಲಾಗಲು ಪ್ರಾರಂಭಿಸುತ್ತಾರೆ. ಶ್ರೀರಾಮಕೃಷ್ಣರು ಜಗತ್ತಿನ ಒಳಿತಿಗಾಗಿ ಬಂದರೇ ಹೊರತು ಹೆಸರು ಕೀರ್ತಿಗಾಗಿ ಅಲ್ಲ ಎಂಬುದನ್ನು ಯಾವಾಗಲೂ ನೆನಪಿಟ್ಟು ಕೊಳ್ಳಿ. ಅವರು ಏನನ್ನು ಬೋಧಿಸಲು ಬಂದರೋ ಅದನ್ನು ಮಾತ್ರ ಪ್ರಸಾರಮಾಡಿ. ಅವರ ಹೆಸರಿನ ಬಗ್ಗೆ ಚಿಂತಿಸಬೇಡಿ–ಅದು ತನ್ನಷ್ಟಕ್ಕೇ ಪ್ರಸಾರಗೊಳ್ಳುತ್ತದೆ. ಪ್ರತಿಯೊಬ್ಬರೂ ನಿಮ್ಮ ಗುರುವನ್ನು ಸ್ವೀಕರಿಸಲೇಬೇಕೆಂದು ಹಟಹಿಡಿದರೆ, ನೀವು ಒಂದು ಮತವನ್ನು ಹುಟ್ಟುಹಾಕುತ್ತೀರಿ; ಆಗ ಎಲ್ಲವೂ ಕುಸಿದುಬೀಳುತ್ತದೆ–ಆದ್ದರಿಂದ ಎಚ್ಚರ! ಪ್ರತಿಯೊಬ್ಬರ ಬಗ್ಗೆಯೂ ಒಳ್ಳೆಯ ದನ್ನೇ ಆಡಿ. ಸಿಟ್ಟು ತೋರಿಸಿದರೆ ಕೆಲಸ ಹಾಳಾಗುತ್ತದೆ. ಜನ ತಮ್ಮ ಮನಸ್ಸಿಗೆ ಬಂದಂತೆ ಆಡಿ ಕೊಳ್ಳಲಿ. ನೀವಂತೂ ನಿಮ್ಮದೇ ಆದ ನಂಬಿಕೆಗೆ ಬದ್ಧರಾಗಿರಿ, ಜಗತ್ತು ನಿಮ್ಮ ಅಜ್ಞಾನುವರ್ತಿ ಯಾಗುತ್ತದೆ. ‘ಅವನನ್ನು ನಂಬು, ಇವನನ್ನು ನಂಬು’ ಎನ್ನುತ್ತಾರೆ ಜನ. ಆದರೆ ನಾನು ಹೇಳು ತ್ತೇನೆ–‘ಮೊದಲು ನಿನ್ನಲ್ಲಿ ನೀನು ನಂಬಿಕೆಯಿಡು’ ಎಂದು. ಅದೊಂದೇ ಮಾರ್ಗ.”

ಶ್ರೀರಾಮಕೃಷ್ಣರ ಕಾರ್ಯವನ್ನು ಯಾವ ರಭಸದಿಂದ ಕೈಗೊಳ್ಳಬೇಕು, ಎಂಥ ಪ್ರಚಂಡ ಉತ್ಸಾಹದಿಂದ ಮುಂದುವರಿಸಬೇಕು ಎಂಬುದರ ಬಗ್ಗೆ ತಮ್ಮ ಕಲ್ಪನೆಗಳನ್ನು ಸ್ವಾಮೀಜಿ ತಮ್ಮ ಸೋದರರ ಮುಂದಿಟ್ಟ ಈ ಪತ್ರಗಳಲ್ಲಿ ಚಿಮ್ಮುವ ಅವರ ಅದಮ್ಯ ಶಕ್ತಿಯನ್ನು ನೋಡಿ:

“ನಾನು ಹಿಂದಿನ ಪತ್ರದಲ್ಲಿ ಹೇಳಿದ್ದು ನೆನಪಿದೆತಾನೆ?– ನಮಗೆ ಸ್ತ್ರೀಪುರುಷರಿಬ್ಬರೂ ಬೇಕು. ಆತ್ಮಕ್ಕೆ ಹೆಣ್ಣುಗಂಡೆಂಬ ಭೇದವಿಲ್ಲ. ಸುಮ್ಮನೆ ಶ್ರೀರಾಮಕೃಷ್ಣರನ್ನು ಅವತಾರವೆಂದು ಕರೆದರೆ ಆಗಲಿಲ್ಲ–ನೀವು ಈ ಶಕ್ತಿಯನ್ನು ವ್ಯಕ್ತಪಡಿಸಬೇಕು. ಗೌರೀ ಮಾ, ಯೋಗಿನ್ ಮಾ, ಗೊಲಾಪ್ ಮಾ (ಶ್ರೀರಾಮಕೃಷ್ಣರ ಭಕ್ತೆಯರು )–ಇವರೆಲ್ಲ ಎಲ್ಲಿ? ಈ ಆಲೋಚನೆಗಳನ್ನು ಹರಡುವಂತೆ ಅವರಿಗೆ ಹೇಳಿ. ಹಿಮಾಲಯದಿಂದ ಕನ್ಯಾಕುಮಾರಿಯವರೆಗೆ, ಉತ್ತರ ಧ್ರುವದಿಂದ ದಕ್ಷಿಣ ಧ್ರುವದವರೆಗೆ–ಜಗತ್ತಿನಾದ್ಯಂತ ಕಾಳ್ಗಿಚ್ಚಿನಂತೆ ಹರಡಬಲ್ಲ ಸಾವಿರಾರು ಸ್ತ್ರೀ ಪುರುಷರು ನಮಗೆ ಬೇಕಾಗಿದ್ದಾರೆ. ಮಕ್ಕಳಾಟಿಕೆಯಲ್ಲಿ ತೊಡಗುವುದರಿಂದ ಪ್ರಯೋಜನವಿಲ್ಲ–ಅಥವಾ ಅದಕ್ಕೆ ಸಮಯವೂ ಇಲ್ಲ. ಮಕ್ಕಳಾಟಕ್ಕೆಂದೇ ಬಂದವರು ಕೂಡಲೇ ಹೊರಟುಬಿಡಲಿ, ಇಲ್ಲದಿದ್ದರೆ ಅವರು ಖಂಡಿತ ಕಷ್ಟಕ್ಕೆ ಗುರಿಯಾಗುತ್ತಾರೆ. ನಮಗೊಂದು ಸಂಘಟನೆ ಬೇಕು. ಸೋಮಾರಿತನ ವನ್ನು ದೂರಕ್ಕಟ್ಟಿ. ಹರಡಿ! ಕಾಳ್ಗಿಚ್ಚಿನಂತೆ ಎಲ್ಲೆಡೆಗೂ ಹರಡಿ. ನನ್ನನ್ನು ಅವಲಂಬಿಸಿಕೊಂಡಿರ ಬೇಡಿ. ನಾನು ಬದುಕಿರುತ್ತೇನೆಯೋ ಸಾಯುತ್ತೇನೆಯೋ, ನೀವಂತೂ ನಿಮ್ಮಷ್ಟಕ್ಕೆ ಕಾರ್ಯವನ್ನು ಮುಂದುವರಿಸಿ.”

೧೮೯೪ರ ಬೇಸಿಗೆಯಲ್ಲಿ ಬರೆದ ಮತ್ತೊಂದು ಪತ್ರದಲ್ಲಿ ಅದನ್ನು ಇನ್ನೂ ಸ್ಪಷ್ಟವಾಗಿ ತಿಳಿಸುತ್ತಾರೆ:

“ನಮಗೀಗ ಬೇಕಾಗಿರುವುದೇನೆಂದರೆ ಸಂಘಟನಾ ಸಾಮರ್ಥ್ಯ–ಅರ್ಥವಾಯಿತೆ?... ಸೋದರರಾದ ತಾರಕ, ಶರತ್ ಮತ್ತು ಹರಿ (ಸ್ವಾಮಿ ಶಿವಾನಂದ, ಸ್ವಾಮಿ ಶಾರದಾನಂದ ಹಾಗೂ ಸ್ವಾಮಿ ತುರೀಯಾನಂದ) ಇವರಿಗೆ ಆ ಸಾಮರ್ಥ್ಯವಿದೆ. –ನಿಗೆ ಸ್ವಂತಿಕೆ ಬಹಳ ಕಡಿಮೆ, ಆದರೆ ಅವನು ಅತ್ಯುತ್ತಮ ಕೆಲಸಗಾರ ಮತ್ತು ಛಲವಂತ, ಇದೊಂದು ಅತಿ ಮುಖ್ಯ ಅಂಶ. ನಮಗೆ ಕೆಲವು ಅನುಯಾಯಿಗಳು ಬೇಕು–ಜಾಜ್ವಲ್ಯಮಾನರಾದ ತರುಣರು, ಅರ್ಥವಾಯಿತೆ? ಬುದ್ಧಿ ವಂತರು, ಧೈರ್ಯಶಾಲಿಗಳು, ಸಾವಿನ ದವಡೆಯನ್ನು ಪ್ರವೇಶಿಸುವ ಎದೆಗಾರಿಕೆಯುಳ್ಳವರು, ಸಮುದ್ರವನ್ನೇ ಈಜಿ ದಾಟಲು ಸಿದ್ಧರಾದವರು. ನಾನು ಹೇಳುವುದು ಅರ್ಥವಾಯಿತೆ? ಅಂತಹ ನೂರಾರು ಜನ–ಸ್ತ್ರೀಯರು ಮತ್ತು ಪುರುಷರು–ನಮಗೆ ಬೇಕು. ಅದೊಂದು ಗುರಿಯನ್ನೇ ಸಾಧಿಸುವತ್ತ ನೀನು ಶಕ್ತಿಮೀರಿ ಪ್ರಯತ್ನಪಡು. ಎಷ್ಟು ಸಾಧ್ಯವೋ ಅಷ್ಟು ಜನರನ್ನು ಸಂನ್ಯಾಸಿ ಗಳನ್ನಾಗಿಸುತ್ತ ಹೋಗು ಮತ್ತು ಅವರನ್ನೆಲ್ಲ ನಮ್ಮ ‘ಪಾವಿತ್ರ್ಯವನ್ನು ತೋಡುವ ಯಂತ್ರ’ದ ಕೆಲಸಕ್ಕೆ ತೊಡಗಿಸು. \eng{(Put them into our purity-drilling machine.)}

“ಶೀಲನಿರ್ಮಾಣದ ಕಾರ್ಯ ನಡೆದ ಮೇಲೆ ನಾನು ನಿಮ್ಮಲ್ಲಿಗೆ ಹಿಂದಿರುಗುತ್ತೇನೆ. ನಿನಗಿದು ಅರ್ಥವಾಯಿತೆ? ನಮಗೆ ಎರಡು ಸಾವಿರ ಸಂನ್ಯಾಸಿಗಳು ಬೇಕು, ಅಲ್ಲ ಹತ್ತು ಸಾವಿರ, ಅಥವಾ ಇಪ್ಪತ್ತು ಸಾವಿರ ಬೇಕು–ಸ್ತ್ರೀಪುರುಷರಿಬ್ಬರೂ. ನಮ್ಮ ಯಜಮಾನಿಯರು ಏನು ಮಾಡುತ್ತಿ ದ್ದಾರೆ? ಎಂತಹ ಕಷ್ಟವಾದರೂ ಸರಿ–ಸರ್ವಸಂಗಪರಿತ್ಯಾಗಿಗಳೇ ನಮಗೆ ಬೇಕು. ಅವರಿಗೆಲ್ಲ ಇದನ್ನು ಹೇಳು, ಮತ್ತು ನೀನು ಪ್ರಯತ್ನಿಸು; ಶಕ್ತಿಮೀರಿ ಪ್ರಯತ್ನಿಸು. ಗೃಹೀಭಕ್ತರಲ್ಲ ನಮಗೆ ಬೇಕಾದುದು, ನೆನಪಿರಲಿ, ನಮಗೆ ಸಂನ್ಯಾಸಿಗಳು ಬೇಕು. ನೀವು ಪ್ರತಿಯೊಬ್ಬರೂ ನೂರುನೂರು ತಲೆಗಳನ್ನು ಮುಂಡನ ಮಾಡಿಸಿ, ನೋಡೋಣ–ಆದರೆ ಸುಶಿಕ್ಷಿತ ತರುಣರು, ಮೂರ್ಖರಲ್ಲ. ಆಗ ನೀವು ನಿಜವಾದ ಧೀರರು. ನಾವೊಂದು ಕೋಲಾಹಲವನ್ನೇ ಎಬ್ಬಿಸಬೇಕು. ನಿಮ್ಮ ನೀರಸ, ನಿರ್ಲಿಪ್ತ ಮನೋಭಾವವನ್ನು ಬಿಟ್ಟು, ಕಟಿಬದ್ಧರಾಗಿ ನಿಲ್ಲಿ. ಕಲ್ಕತ್ತ ಮದ್ರಾಸುಗಳ ನಡುವೆ ನೀವೊಂದು ಪ್ರಬಲ ಆಂದೋಳನವನ್ನು ಪ್ರಾರಂಭಿಸುವುದನ್ನು ನಾನು ನೋಡಬೇಕು. ಬೇರೆಬೇರೆ ಸ್ಥಳಗಳಲ್ಲಿ ಕೇಂದ್ರಗಳನ್ನು ಸ್ಥಾಪಿಸಿ; ಜನರನ್ನು ಪರಿವರ್ತಿಸುತ್ತ ಹೋಗಿ. ಲಿಂಗಭೇದವಿಲ್ಲದೆ, ಯಾರುಯಾರು ಸಂನ್ಯಾಸವನ್ನರಸಿ ಬರುತ್ತಾರೋ ಅವರನ್ನೆಲ್ಲ ಬರಮಾಡಿಕೊಳ್ಳಿ. ಆಗ ನಾನು ಮತ್ತೆ ನಿಮ್ಮಲ್ಲಿಗೆ ಮರಳುತ್ತೇನೆ. ಒಂದು ಮಹಾ ಆಧ್ಯಾತ್ಮಿಕ ಉಬ್ಬರದ ಅಲೆ ನುಗ್ಗಿಬರುತ್ತಿದೆ– ನಿಕೃಷ್ಟರಾದವರು ಉನ್ನತ ವ್ಯಕ್ತಿಗಳಾಗುತ್ತಾರೆ, ಅಜ್ಞಾನಿಗಳು ಮಹಾಜ್ಞಾನಿಗಳ ಗುರುವಿನ ಪದವಿಗೆ ಏರುತ್ತಾರೆ–ಎಲ್ಲವೂ ಆ ಭಗವಂತನ ಕೃಪೆಯಿಂದ.”

ಸಂನ್ಯಾಸಿಗಳ ಒಂದು ಮಹಾಸಂಘವನ್ನು ಸ್ಥಾಪಿಸುವ ಉದ್ದೇಶವನ್ನು ಸ್ವಾಮೀಜಿ ಹೊಂದಿದ್ದ ರಾದರೂ, ಸಂಘದ ಸದಸ್ಯರ ಸಂಖ್ಯೆಗಿಂತ ಅದರ ಗುಣಮುಟ್ಟಕ್ಕೆ ಅವರು ಪ್ರಾಧಾನ್ಯವನ್ನು ನೀಡಿದ್ದರೆಂಬುದನ್ನು ನಾವು ಗಮನಿಸಬೇಕು. ಆ ಬಗ್ಗೆ ಅವರು ತಮ್ಮ ಗುರುಭಾಯಿಗಳಿಗೆ ಮುನ್ನೆಚ್ಚರಿಕೆ ನೀಡಿದ್ದರು:

“ಅಸೂಯೆ, ಒಗ್ಗಟ್ಟಿನಿಂದ ಕೆಲಸ ಮಾಡುವ ಪ್ರವೃತ್ತಿಯಿಲ್ಲದಿರುವುದು–ಇದೇ ಗುಲಾಮ ರಾಷ್ಟ್ರಗಳಲ್ಲಿ ಎದ್ದುಕಾಣುವ ಸ್ವಭಾವ. ಆದರೆ ನಾವದನ್ನು ಕೊಡವಿಹಾಕುವ ಪ್ರಯತ್ನ ಮಾಡಲೇಬೇಕು.

“ಎಷ್ಟೇ ಕಷ್ಟವನ್ನಾದರೂ ಎದುರಿಸಿ ಯಾವುದೇ ಬೆಲೆಯನ್ನಾದರೂ ತೆತ್ತು, ಆ ಅಸೂಯೆ ನಮ್ಮೊಳಗೆ ಪ್ರವೇಶಿಸದಂತೆ ನೋಡಿಕೊಳ್ಳಬೇಕು. ನಾವು ಹತ್ತೇ ಜನರಿದ್ದೇವೋ ಅಥವಾ ಇಬ್ಬರೇ ಇದ್ದೇವೋ, ಲೆಕ್ಕಿಸಬೇಡಿ; ಆದರೆ ಇರುವಷ್ಟು ಜನ ಮಾತ್ರ ನಿಷ್ಕಳಂಕ ವ್ಯಕ್ತಿಗಳಾಗಿರಬೇಕು.”

ತಮ್ಮ ಸೋದರಸಂನ್ಯಾಸಿಗಳಿಗೆ ಬರೆದ ಪತ್ರಗಳಲ್ಲಿ ಸ್ವಾಮೀಜಿ, ಸಂಘದ ಉಳಿವು-ಬೆಳವಣಿಗೆ ಗಳಿಗೆ ಅತಿ ಮುಖ್ಯವಾದ ಅನೇಕ ಅಂಶಗಳನ್ನು ಮತ್ತೆಮತ್ತೆ ವಿವರಿಸಿದರು:

“ನಿಮಗೆ ನಿಮ್ಮನಿಮ್ಮಲ್ಲಿ ಪರಸ್ಪರ ಅತಿಶಯವಾದ ಪ್ರೀತಿಯಿರಲಿ; ಜನರ ಟೀಕೆಗಳ ಬಗ್ಗೆ ನಿರ್ಲಿಪ್ತ ಧೋರಣೆಯಿಂದಿರಲು ಅದೊಂದಿದ್ದರೆ ಸಾಕು. ನಾನು ನಿಮಗೆ ಭರವಸೆ ಕೊಡುತ್ತೇನೆ– ಭಗವಂತನ ಕೃಪೆಯಿಂದ ನಿಮ್ಮನಿಮ್ಮಲ್ಲಿ ಮನಸ್ತಾಪವಿಲ್ಲದಿರುವವರೆಗೆ, ‘ಜಲೇ ಚಾನಲೇ ಪರ್ವತೇ ಶತ್ರುಮಧ್ಯೇ’ (ನೀರಿನಲ್ಲಾಗಲಿ, ಬೆಂಕಿಯಲ್ಲಾಗಲಿ, ಪರ್ವತಶಿಖರದ ಮೇಲಾಗಲಿ, ಶತ್ರುಗಳ ಮಧ್ಯದಲ್ಲಾಗಲಿ)–ನಿಮಗೆ ಯಾವುದೇ ಅಪಾಯವಿಲ್ಲ. ಅಲ್ಲದೆ, ಶ್ರೇಷ್ಠವಾದ ಪ್ರತಿ ಯೊಂದು ಕಾರ್ಯವೂ ಅಡಚಣೆಗಳಿಂದಲೇ ತುಂಬಿರುತ್ತದೆ. ನೀವೆಲ್ಲ ಕಟಿಬದ್ಧರಾಗಿ ನಿಂತು ನನ್ನನ್ನು ಹಿಂಬಾಲಿಸಿ ಬರುತ್ತಿರುವವರೆಗೆ, ಇಡೀ ಜಗತ್ತೇ ನಿಮಗೆದುರಾದರೂ ನಿಮಗೆ ಭಯವಿಲ್ಲ.”

“ನಾನು ನಿಮ್ಮೆಲ್ಲರಿಂದಲೂ ನಿರೀಕ್ಷಿಸುವುದು ಇದನ್ನು–ಎಂದೆಂದಿಗೂ ನೀವು ಸ್ವಪ್ರತಿಷ್ಠೆ ಯನ್ನು, ಮತ್ಸರವನ್ನು ಮತ್ತು ಗುಂಪುಗಾರಿಕೆಯನ್ನು ತ್ಯಜಿಸಬೇಕು. ನೀವು ಭೂದೇವಿಯಂತೆ ಅಪಾರ ಸಹಿಷ್ಣುತೆಯನ್ನು ಹೊಂದಿದವರಾಗಬೇಕು.”

“ಸಂನ್ಯಾಸಿಗಳನ್ನು ಯಾವಾಗಲೂ ಅಂಟಿಕೊಳ್ಳುವ ಒಂದು ದೋಷವೆಂದರೆ, ತಮ್ಮ ಸಂನ್ಯಾಸದ ಬಗ್ಗೆ ಅಭಿಮಾನ ತಾಳುವುದು. ಮೊದಮೊದಲ ಹಂತಗಳಲ್ಲಿ ಅದರಿಂದ ಕೆಲವು ಅನುಕೂಲಗಳಿರಬಹುದು. ಆದರೆ ಅವರು ಸಾಕಷ್ಟು ಬೆಳೆದ ಮೇಲೆ, ಅವರಿಗದರ ಆವಶ್ಯಕತೆ ಯಿಲ್ಲ. ಸಂನ್ಯಾಸಿಯಾದವನು, ಗೃಹಸ್ಥರ ಹಾಗೂ ಸಂನ್ಯಾಸಿಗಳ ಮಧ್ಯೆ ಭೇದಭಾವ ತೋರ ಬಾರದು–ಆಗ ಮಾತ್ರವೇ ಅವನು ನಿಜವಾದ ಸಂನ್ಯಾಸಿ”

ಪ್ರತಿಯೊಂದು ರೀತಿಯಲ್ಲೂ ಪರಿಪೂರ್ಣರಾಗುವಂತೆ ತಮ್ಮ ಸೋದರಸಂನ್ಯಾಸಿಗಳನ್ನು ಸ್ವಾಮೀಜಿ ಮತ್ತೆಮತ್ತೆ ಉತ್ತೇಜಿಸುತ್ತಿದ್ದರು. ಆದರೆ ಶಕ್ತಿಶಾಲಿಗಳಾದ ವ್ಯಕ್ತಿಗಳಿಗೆ ಬರೆದ ಅವರ ಪತ್ರಗಳೂ ಕಠಿಣವಾಗಿರುತ್ತಿದ್ದುವು. ಅವರ ಈಟಿಯ ಮೊನೆಯಂತಹ ಮಾತುಗಳ ರುಚಿಯನ್ನೊಮ್ಮೆ ನೋಡಿ:

“ನಮ್ಮ ದೇಶಕ್ಕೆ ಯಾವ ಭರವಸೆಯ ಲಕ್ಷಣವೂ ಇಲ್ಲ. ಒಬ್ಬನ ಮೆದುಳಿನಲ್ಲೂ ಹೊಸತಾದ ಒಂದು ಸ್ವಂತ ಆಲೋಚನೆ ಹೊಳೆಯುವುದಿಲ್ಲ. ರಾಮಕೃಷ್ಣ ಪರಮಹಂಸರು ಅಂಥವರಾಗಿದ್ದರು -ಇಂಥವರಾಗಿದ್ದರು ಎಂಬ ಅದೇ ಹಳೇ ಪಲ್ಲವಿಯನ್ನೇ ಹಿಡಿದು ಎಳೆದೆಳೆದುಹಾಕುತ್ತ ಕಿತ್ತಾಡುವ ಕಿಸುಬಾಯಿದಾಸರೇ ಎಲ್ಲ. ತಲೆಬುಡವಿಲ್ಲದ ಅಡುಗೂಲಜ್ಜಿಯ ಕತೆಗಳನ್ನು ಹೇಳಿಕೊಂಡು ತಿರುಗುವವರೇ ಎಲ್ಲ. ಅಯ್ಯೋ ದೇವರೆ! ನೀವು ಈ ಸಾಧಾರಣ ಜನಗಳಿಗಿಂತ ಸ್ವಲ್ಪವಾದರೂ ಭಿನ್ನವೆಂಬುದನ್ನು ತೋರಿಸಿಕೊಡಲು ಏನಾದರೂ ಮಾಡುವುದಿಲ್ಲವೆ? ಕೇವಲ ಹುಚ್ಚುತನದಲ್ಲಿ ತೊಡಗುವುದೆ?... ಇವತ್ತು ನೀವು ಘಂಟೆ ಬಾರಿಸುತ್ತೀರಿ. ನಾಳೆ ಒಂದು ಶಂಖ ಊದುತ್ತೀರಿ. ಮತ್ತೆ ನಾಡಿದ್ದು ಒಂದು ಚಾಮರವನ್ನೂ ಸೇರಿಸುತ್ತೀರಿ (ಪೂಜೆಯ ಸಂಬಂಧವಾಗಿ). ಇಲ್ಲದಿದ್ದರೆ ಇವತ್ತು ಒಂದು ಮಂಚವನ್ನು ತರುತ್ತೀರಿ. ನಾಳೆ ಅದರ ಕಾಲುಗಳಿಗೆ ಬೆಳ್ಳಿಯ ತಗಡು ಹೊದಿಸು ತ್ತೀರಿ. ಜನ ಅಕ್ಕಿಯ ಗಂಜಿಯನ್ನು ಕುಡಿಯುತ್ತಲೇ ಇರುತ್ತಾರೆ–ನೀವು ಎರಡು ಸಾವಿರ ಕಾಗಕ್ಕ- ಗುಬ್ಬಕ್ಕನ ಕಥೆಗಳನ್ನು ಹರಡುತ್ತೀರಿ. ಒಟ್ಟಿನಲ್ಲಿ ಕೇವಲ ಬಾಹ್ಯಪೂಜೆಗಳನ್ನು ಮಾಡಿಕೊಂಡಿರು ತ್ತೀರಿ. ಇದಕ್ಕೆ ಇಂಗ್ಲೀಷಿನಲ್ಲಿ \eng{Imbecility (}ಷಂಡತನ) ಎನ್ನುತ್ತಾರೆ. ಯಾರ ತಲೆಯೊಳಕ್ಕೆ ಕೇವಲ ಇಂತಹ ಕ್ಷುದ್ರ ಆಲೋಚನೆಗಳು ಮಾತ್ರ ಪ್ರವೇಶಿಸುತ್ತವೆಯೋ ಅವರನ್ನು ನಿರ್ವೀಯ ರೆನ್ನುತ್ತಾರೆ. ‘ಘಂಟೆಯನ್ನು ಎಡಗಡೆ ಬಾರಿಸಬೇಕೆ, ಬಲಗಡೆ ಬಾರಿಸಬೇಕೆ?’ ‘ಗಂಧವನ್ನು ಹಣೆಯ ಮೇಲೆ ಹಚ್ಚಿಕೊಳ್ಳಬೇಕೆ, ಇನ್ನೆಲ್ಲಾದರೂ ಹಚ್ಚಿಕೊಳ್ಳಬೇಕೆ?’ ‘ಆರತಿಯನ್ನು ಎರಡು ಸಲ ಎತ್ತಬೇಕೆ, ನಾಲ್ಕು ಸಲ ಎತ್ತಬೇಕೆ?’ ಎಂಬಂತಹ ಪ್ರಶ್ನೆಗಳು ಮಾತ್ರ ಯಾರ ತಲೆಗಳಲ್ಲಿ ಹಗಲಿರುಳು ಕುಣಿಯುತ್ತಿವೆಯೋ, ಅಂಥವರು ದರಿದ್ರದವರೆಂದು ಕರೆಸಿಕೊಳ್ಳಲು ಯೋಗ್ಯರು. ಇಂತಹ ಬುದ್ಧಿಗೇಡಿತನದಿಂದಾಗಿಯೇ ಭಾಗ್ಯದೇವತೆಯು ನಮ್ಮನ್ನು ಅಸ್ಪೃಶ್ಯರಂತೆ ದೂರವೇ ಇಟ್ಟಿದ್ದಾಳೆ. ಮತ್ತು ಪಾಶ್ಚಾತ್ಯರು ಇಡೀ ಜಗತ್ತಿನ ಪ್ರಭುಗಳಾಗಿರುವಾಗ ನಾವು ಮಾತ್ರ ಎಲ್ಲ ರಿಂದಲೂ ಛೀಮಾರಿ ಹಾಕಿಸಿಕೊಂಡು ಒದೆಸಿಕೊಳ್ಳುತ್ತಿದ್ದೇವೆ. ಸೋಮಾರಿತನಕ್ಕೂ ವೈರಾಗ್ಯಕ್ಕೂ ಸಾಗರದಷ್ಟು ಅಂತರ.”

ಸ್ವಾಮೀಜಿಯ ಟೀಕೆಗಳು ಇಷ್ಟು ಹರಿತವಾಗಿದ್ದರೂ ಅವರ ಉದ್ದೇಶ ಮಾತ್ರ ತುಂಬ ಒಳ್ಳೆ ಯದು; ಅವರ ನಾಲಿಗೆ-ಲೇಖನಿ ಎಷ್ಟು ಹರಿತವೋ ಅವರ ಹೃದಯ ಅಷ್ಟೇ ನಯವಾದುದು. ತಮ್ಮ ಸೋದರರ ನಿಜವಾದ ಆಧ್ಯಾತ್ಮಿಕ ಯೋಗ್ಯತೆಯೇನೆಂಬುದನ್ನು ಸ್ವಾಮೀಜಿ ಮಾತ್ರ ಬಲ್ಲವ ರಾಗಿದ್ದರು. ತಮ್ಮ ಸೋದರರ ಮೇಲಿನ ಅತಿಶಯ ಪ್ರೀತಿ ಹಾಗೂ ಅವರ ಶ್ರೀರಾಮಕೃಷ್ಣರ ಕಾರ್ಯದಲ್ಲಿ ಉಪಕರಣಗಳಾಗುವವರು ಎಂಬ ನಂಬಿಕೆಯೇ ಸ್ವಾಮೀಜಿ ಇಷ್ಟು ಕಠಿಣವಾಗಿ ಬರೆಯುವಂತೆ ಮಾಡಿದುದು. ಅವರು ತಮ್ಮ ಶ್ರದ್ಧೆ-ಉತ್ಸಾಹಗಳಲ್ಲಿ ಸಹಭಾಗಿಗಳಾಗುತ್ತಾರೆಂದು ಸ್ವಾಮೀಜಿ ದೃಢವಾಗಿ ನಂಬಿದ್ದರು:

“ಉತ್ತಿಷ್ಠತ! ಜಾಗ್ರತ! ಭಗವಂತ ಮಹಾಮಹಿಮ! ಅವನೇ ನಮ್ಮ ಹಿಂದಿದ್ದಾನೆ. ನಾನು ಇದಕ್ಕಿಂತ ಹೆಚ್ಚಿಗೇನೂ ಬರೆಯಲಾರೆ. ಮುಂದೆ ಸಾಗಿ! ನಾನು ಇಷ್ಟು ಮಾತ್ರ ಹೇಳಬಲ್ಲೆ– ಯಾರುಯಾರು ನನ್ನ ಈ ಪತ್ರವನ್ನು ಓದುತ್ತಾರೆಯೋ ಅವರೆಲ್ಲ ನನ್ನ ಶಕ್ತಿಯನ್ನು ಪಡೆದುಕೊಳ್ಳು ತ್ತಾರೆ. ವಿಶ್ವಾಸವಿಡಿ! ಮುನ್ನುಗ್ಗಿ!... ಯಾರೋ ನನ್ನ ಕೈಹಿಡಿದು ಈ ರೀತಿ ಬರೆಸುತ್ತಿದ್ದಾ ರೆಯೋ ಎಂಬಂತೆ ನನಗನ್ನಿಸುತ್ತಿದೆ.”

ತಮ್ಮ ಸಂನ್ಯಾಸೀ ಸಂಘವನ್ನು ನಿರ್ಮಾಣ ಮಾಡಲು, ಅದನ್ನು ಸಚೇತನಗೊಳಿಸಿ ಕಾರ್ಯ ಪ್ರವೃತ್ತವಾಗಿಸಲು ಸ್ವಾಮೀಜಿ ಪಟ್ಟ ಶ್ರಮವೆಂಥದು, ಅವರು ತಾಳಿದ ಉತ್ಸಾಹವೆಂಥದು ಎಂಬು ದನ್ನು ಈವರೆಗೆ ನೋಡಿದ್ದೇವೆ. ಶ್ರೀರಾಮಕೃಷ್ಣರನ್ನು ಕೇಂದ್ರವಾಗಿಟ್ಟುಕೊಂಡು ನಿರ್ಮಿತವಾದ ಈ ಸಂಘದಂತೆಯೇ, ಶ್ರೀಮಾತೆ ಶಾರದಾದೇವಿಯವರನ್ನು ಕೇಂದ್ರವಾಗಿರಿಸಿಕೊಂಡ ಸಂನ್ಯಾಸಿನಿ ಯರ ಸಂಘವೊಂದು ಜನ್ಮದಳೆದು, ಆಧುನಿಕ ಭಾರತದಲ್ಲಿ ಸಂನ್ಯಾಸಿನಿಯರ ಪರಂಪರೆ ಯೊಂದು ಬೆಳೆಯುವಂತೆ ಮಾಡಬೇಕೆಂಬ ಉತ್ಕಟೇಚ್ಛೆ ಅವರಿಗಿತ್ತು. (ಅವರ ಈ ಬಯಕೆಯು ಅವರ ಜೀವಿತಾವಧಿಯಲ್ಲೇ ನೆರವೇರದಿದ್ದರೂ ಮುಂದೆ ಅನೇಕ ವರ್ಷಗಳ ಬಳಿಕ ಅದು ರೂಪ ತಾಳಿತು.) ಶ್ರೀರಾಮಕೃಷ್ಣರ ಶಿಷ್ಯರ ಪೈಕಿ, ಶ್ರೀಮಾತೆಯವರ ಮಾಹಾತ್ಮ್ಯವನ್ನೂ ಭಾರತದ ಹಾಗೂ ಮಾನವತೆಯ ಪುನರುದ್ಧಾರದ ಕಾರ್ಯದಲ್ಲಿ ಇವರ ಮಹತ್ವವನ್ನೂ ಅರಿತವರಲ್ಲಿ ಸ್ವಾಮಿ ವಿವೇಕಾನಂದರೇ ಮೊದಲಿಗರು. ೧೮೯೪ರಲ್ಲಿ–ಎಂದರೆ ಶ್ರೀಶಾರದಾದೇವಿಯರು ಶರೀರ ತ್ಯಾಗ ಮಾಡಲು ಇನ್ನೂ ಸುಮಾರು ಇಪ್ಪತೈದು ವರ್ಷಗಳಿರುವಾಗಲೇ–ಸ್ವಾಮೀಜಿ ತಮ್ಮ ಗುರುಭಾಯಿಗಳಾದ ಶಿವಾನಂದರಿಗೆ ಬರೆದ ಪತ್ರದಲ್ಲಿ ತಮ್ಮ ಅಂತರಾಳದ ಭಾವನೆಗಳನ್ನು ಹೀಗೆ ಹೊರಗೆಡಹಿದರು:

“ಶ್ರೀಮಾತೆಯವರ ಜೀವನದ ಮಹತ್ವವನ್ನು ಇನ್ನೂ ನಿಮ್ಮಲ್ಲಿ ಒಬ್ಬರೂ ಅರಿತುಕೊಂಡಿಲ್ಲ. ಆದರೆ ಕಾಲಕ್ರಮದಲ್ಲಿ ನೀವದನ್ನು ಅರಿಯುವಿರಿ. ಆದ್ಯಾಶಕ್ತಿಯ ಅನುಗ್ರಹವಿಲ್ಲದೆ ಜಗತ್ತಿಗೆ ಉದ್ಧಾರವಿಲ್ಲ. ಇತರ ಎಲ್ಲ ರಾಷ್ಟ್ರಗಳಿಗಿಂತ ನಮ್ಮ ರಾಷ್ಟ್ರ ದುರ್ಬಲಗೊಂಡಿದೆಯಲ್ಲ, ಹಿಂದುಳಿದಿದೆಯಲ್ಲ, ಏಕೆ? ಏಕೆಂದರೆ ಇಲ್ಲಿ ಆದ್ಯಾಶಕ್ತಿಯನ್ನು ಅವಮಾನಿಸಲಾಗಿದೆ. ಈ ಅದ್ಭುತ ಶಕ್ತಿಯನ್ನು ಜಾಗೃತಗೊಳಿಸುವುದಕ್ಕಾಗಿಯೇ ಶ್ರೀಮಾತೆಯವರು ಅವತರಿಸಿ ಬಂದಿದ್ದಾರೆ. ಅವ ರನ್ನು ಕೇಂದ್ರವಾಗಿಸಿಕೊಂಡು ಈ ಜಗತ್ತಿನಲ್ಲಿ ಇನ್ನೊಮ್ಮೆ ಗಾರ್ಗಿಯರು ಮೈತ್ರೇಯಿಯರು ಉದಿಸಿಬರಲಿದ್ದಾರೆ. ಪ್ರಿಯ ಸೋದರ, ನಿನಗಿದೆಲ್ಲ ಈಗ ಅಷ್ಟು ಚೆನ್ನಾಗಿ ಅರ್ಥವಾಗಲಾರದು, ಆದರೆ ಕಾಲಕ್ರಮದಲ್ಲಿ ಎಲ್ಲವೂ ಅರ್ಥವಾಗುತ್ತದೆ. ಅಮೆರಿಕ ಯೂರೋಪುಗಳಲ್ಲಿ ನೀನೇನು ಕಾಣುತ್ತಿ? ಶಕ್ತಿಯ ಪೂಜೆಯನ್ನು! ಆದರೆ ಅವರು ಅಜ್ಞಾನದಿಂದ ಆ ಶಕ್ತಿಯನ್ನು ಇಂದ್ರಿಯ ಭೋಗದ ಮೂಲಕ ಆರಾಧಿಸುತ್ತಿದ್ದಾರೆ. ಹೀಗಿರುವಾಗ, ಯಾರು ಅದೇ ಆದ್ಯಾಶಕ್ತಿಯನ್ನು ತಮ್ಮ ಜನನಿಯೆಂದು ಭಾವಿಸಿ ಸಾತ್ವಿಕಭಾವದಿಂದ, ಪರಿಶುದ್ಧ ಹೃದಯದಿಂದ ಆರಾಧಿಸುವರೋ ಅವರು ಇನ್ನೆಷ್ಟು ಸನ್ಮಂಗಳವನ್ನು ಸಾಧಿಸಿಕೊಳ್ಳಲಿಕ್ಕಿಲ್ಲ!... ದಿನ ಕಳೆದಂತೆಲ್ಲ ನನಗೆ ಈ ವಿಷಯ ಹೆಚ್ಚುಹೆಚ್ಚು ಸ್ಪಷ್ಟವಾಗುತ್ತಿದೆ; ನನ್ನ ಅಂತರ್ದೃಷ್ಟಿ ಹೆಚ್ಚು ಹೆಚ್ಚು ತೆರೆಯುತ್ತಿದೆ. ಆದ್ದರಿಂದ ನಾವು ಮೊಟ್ಟಮೊದಲು ಶ್ರೀಶಾರದಾದೇವಿಯವರ ಹೆಸರಿನಲ್ಲಿ ಮಠವನ್ನು ಕಟ್ಟಬೇಕು. ಮೊದಲು ಶ್ರೀಶಾರದಾದೇವಿಯರು ಹಾಗೂ ಅವರ ಸ್ತ್ರೀಸಂತಾನ; ಆಮೇಲೆ ಶ್ರೀರಾಮಕೃಷ್ಣರು ಹಾಗೂ ಅವರ ಪುರುಷ ಸಂತಾನ. ನೀನಿದನ್ನು ಅರ್ಥಮಾಡಿಕೊಳ್ಳಬಲ್ಲೆಯೇನು? ಶ್ರೀಮಾತೆಯವರ ಕೃಪೆ ನನ್ನ ಪಾಲಿಗೆ ಶ್ರೀರಾಮಕೃಷ್ಣರ ಕೃಪೆಗಿಂತಲೂ ನೂರು ಸಾವಿರ ಪಟ್ಟು ಅಮೂಲ್ಯ. ಮಾತೆಯ ಕೃಪೆ, ಮಾತೆಯ ಆಶೀರ್ವಾದ–ಎಲ್ಲವೂ ನನಗೆ ಪರಮೋತ್ಕೃಷ್ಟ. ಸೋದರ, ನೋಡು, ನನ್ನ ಹೆಸರು ನಿಜಕ್ಕೂ ವಿವೇಕಾನಂದನೇ ಹೌದಾದಲ್ಲಿ, ನಿಮಗೆಲ್ಲ ನಾನು ಜೀವಂತ ದುರ್ಗೆಯನ್ನು ಆರಾಧಿಸಿ ತೋರಿಸಿಕೊಡುತ್ತೇನೆ. ದಯವಿಟ್ಟು ನನ್ನನ್ನು ಕ್ಷಮಿಸು. ಮಾತೆಯ ವಿಚಾರದಲ್ಲಿ ನಾನು ಸ್ವಲ್ಪ ಮತಾಂಧನೇ ಸರಿ. ಆಕೆಯೇನಾದರೂ ಆಜ್ಞೆ ಮಾಡಿದರೆ ಆಕೆಯ ಈ ಭೂತಗಣಗಳು ಏನನ್ನಾದರೂ ಸಾಧಿಸಿಬಿಡಬಲ್ಲುವು. ಶ್ರೀರಾಮಕೃಷ್ಣರ ವಿಚಾರದಲ್ಲಿ, ನೀನು ಅವರನ್ನು ದೇವ ರೆಂದಾದರೂ ಕರೆ, ಮನುಷ್ಯನೆಂದಾದರೂ ಕರೆ, ಅಥವಾ ಇನ್ನೇನು ಬೇಕಾದರೂ ಕರೆದುಕೋ. ಆದರೆ ಶಾರದಾದೇವಿಯವರನ್ನು ಯಾರು ಸಾಕ್ಷಾತ್ ಜಗನ್ಮಾತೆಯೆಂದು ಒಪ್ಪಿ ಗೌರವಿಸುವು ದಿಲ್ಲವೋ ಅವನಿಗೆ ಧಿಕ್ಕಾರವಿರಲಿ!... 

“ಈಗ ನೋಡು, ಶ್ರೀಮಾತೆಯವರ ಮಠಕ್ಕಾಗಿ ಹಣವನ್ನು ಸಂಗ್ರಹಿಸಲೆಂದು ನಾನು ಈ ಭಯಂಕರ ಚಳಿಗಾಲದಲ್ಲಿ ಅನೇಕ ಅಡ್ಡಿಗಳನ್ನು ಎದುರಿಸುತ್ತ ಊರಿಂದೂರಿಗೆ ಸುತ್ತುತ್ತಿದ್ದೇನೆ... ನೀನು ಒಂದು ಜಾಗವನ್ನು ಖರೀದಿಸಿ, ಜೀವಂತ ದುರ್ಗೆಯಾದ ಶ್ರೀಮಾತೆಯವರನ್ನು ಸಂಸ್ಥಾಪಿಸಿ ದಾಗ ನನಗೆ ಸಮಾಧಾನವಾಗುತ್ತದೆ... ”

ಈ ಪತ್ರದಲ್ಲಿ ನೋಡುವಂತೆ, ಶಾರದಾದೇವಿಯರಿಗಾಗಿ ಒಂದು ಸೂಕ್ತ ನಿವೇಶನವನ್ನು ಕೊಂಡುಕೊಳ್ಳುವಂತೆ ಸ್ವಾಮೀಜಿ ತಮ್ಮ ಗುರುಭಾಯಿಗಳಿಗೆ ಮತ್ತೆಮತ್ತೆ ಒತ್ತಾಯಪಡಿಸಿ ಬರೆಯುತ್ತಿದ್ದರು. ಶ್ರೀಮಾತೆಯವರು ನೆಮ್ಮದಿಯಿಂದ ಜೀವಿಸಲು ಸಾಧ್ಯವಾಗುವಂತೆ ಅವರದೇ ಆದ ಜಾಗವೊಂದು ಇರುವಂತಾಗಬೇಕೆಂಬುದು ಸ್ವಾಮೀಜಿಯ ಇಚ್ಛೆಯಾಗಿತ್ತು. ಅಲ್ಲದೆ ಈ ನಿವೇಶನಕ್ಕೆ ಹಣವನ್ನು ಕಳಿಸುವ ಭರವಸೆಯನ್ನೂ ನೀಡಿದ್ದರು. ತಮ್ಮ ಕಾರ್ಯೋದ್ದೇಶಗಳ ಪೈಕಿ, ಶ್ರೀಶಾರದಾದೇವಿಯವರಿಗಾಗಿ ಒಂದು ಮನೆಯನ್ನು ಕಟ್ಟಿಸಿಕೊಡುವುದಕ್ಕೆ ಸ್ವಾಮೀಜಿ ಅಗ್ರಸ್ಥಾನ ವನ್ನು ನೀಡಿದ್ದರು. ಇಡೀ ಭಾರತದ ಅಥವಾ ಇಡೀ ವಿಶ್ವದ ಸ್ತ್ರೀಯರ ಸ್ಥಿತಿಗತಿಗಳನ್ನು ಸುಧಾರಿಸು ವುದರ ಬಗ್ಗೆ ಅವರಿಗಿದ್ದ ಕಾಳಜಿ ಅದೆಷ್ಟು ಅಗಾಧವಾದದ್ದು! ಸ್ತ್ರೀಯರ ಕಲ್ಯಾಣದೊಂದಿಗೆ ಹಿಂದೂ ಸಮಾಜದ ಒಟ್ಟು ಸುಧಾರಣೆಯು ಸ್ವಾಮೀಜಿಯ ಅತಿ ಮುಖ್ಯ ಕಾಳಜಿಯಾಗಿತ್ತು. ಹಿಂದೂ ಸಮಾಜವು ಮುಕ್ತ ಸಮಾಜವಾಗಿರದೆ ಅನೇಕ ಶತಮಾನಗಳಿಂದಲೂ ಸಂಪ್ರದಾಯಗಳ ಕೂಪದಿಂದ ಮೇಲೆದ್ದು ನಿಲ್ಲದಿದ್ದುದೇ ಭಾರತದ ಸಕಲ ಸಮಸ್ಯೆಗಳ ಮೂಲಕಾರಣವೆಂದು ಅವರು ವಿಶ್ಲೇಷಿಸಿದ್ದರು. ಹಿಂದೂ ಸಮಾಜದಲ್ಲಿ ಹೊಸ ರಕ್ತ ಪ್ರವಹಿಸಬೇಕಾದರೆ ಜನರು ಹೆಚ್ಚೆಚ್ಚು ಸಂಖ್ಯೆಯಲ್ಲಿ ಪರರಾಷ್ಟ್ರಗಳ ಹಾಗೂ ಪರಸಮಾಜಗಳ ಮುನ್ನಡೆಯನ್ನು ಕಣ್ಣಾರೆ ಕಾಣ ಬೇಕೆಂದು ಅವರು ಬಯಸಿದರು. ಆದ್ದರಿಂದ ಅವರಿನ್ನೂ ಪರಿವ್ರಾಜಕರಾಗಿದ್ದಾಗಲೇ ಒಂದು ಪತ್ರದಲ್ಲಿ ಹೀಗೆ ಬರೆದಿದ್ದರು: “ನಾವು ಯಾತ್ರೆ ಮಾಡಬೇಕು. ನಾವು ವಿದೇಶಿಗಳಿಗೆ ಹೋಗಬೇಕು. ಇತರ ರಾಷ್ಟ್ರಗಳಲ್ಲಿ ಸಮಾಜದ ಯಂತ್ರ ಹೇಗೆ ಕೆಲಸ ಮಾಡುತ್ತದೆಂಬುದನ್ನು ನೋಡಬೇಕು. ಮತ್ತು ನಾವು ನಿಜಕ್ಕೂ ಮತ್ತೊಮ್ಮೆ ಒಂದು ರಾಷ್ಟ್ರವಾಗಿ ರೂಪುಗೊಳ್ಳಬೇಕಾದರೆ, ಇತರ ರಾಷ್ಟ್ರ ಗಳ ಆಲೋಚನಾ ಲಹರಿ ಯಾವ ದಿಕ್ಕಿನಲ್ಲಿ ಸಾಗುತ್ತಿದೆಯೆಂಬುದನ್ನು ನಿರಂತರವಾಗಿ ಗಮನಿಸು ತ್ತಿರಬೇಕು.” ತಾವು ಅಮೆರಿಕದ ಯಾತ್ರೆಯನ್ನು ಕೈಗೊಂಡಮೇಲಂತೂ ಈ ಆಲೋಚನೆ ಸ್ವಾಮೀಜಿಯವರಿಗೆ ಮತ್ತಷ್ಟು ದೃಢವಾಯಿತು.

ಆದರೆ ಅವರ ಈ ಬಗೆಯ ಸಂಪರ್ಕದ ಉದ್ದೇಶ, ಭಾರತದ ಆರ್ಥಿಕ ಹಾಗೂ ಸಾಮಾಜಿಕ ಸ್ಥಿತಿಯನ್ನು ಸುಧಾರಿಸುವುದಷ್ಟೇ ಆಗಿರಲಿಲ್ಲ. ಬದಲಾಗಿ, ಭಾರತದ ಅನಂತ ಆಧ್ಯಾತ್ಮಿಕ ಐಶ್ವರ್ಯವನ್ನು ಪರರಾಷ್ಟ್ರಗಳೊಂದಿಗೆ ಹಂಚಿಕೊಳ್ಳುವುದೂ ಆಗಿತ್ತು. ಭಾರತವು ಪುನಶ್ಚೇತನ ಗೊಳ್ಳಬಲ್ಲುದಷ್ಟೇ ಅಲ್ಲದೆ ಅದೊಂದು ಪ್ರಬಲ ಶಕ್ತಿಯಾಗಿ ನಿಲ್ಲುವ ಸಾಮರ್ಥ್ಯವನ್ನು ಹೊಂದಿದೆ ಎಂಬುದು ಅಮೆರಿಕದಲ್ಲಿನ ಅವರ ಸಾಧನೆಯಿಂದ ಸಾಬೀತಾಗಿತ್ತು. ಇದರ ಬಗ್ಗೆ ಅವರಿಗಂತೂ ಯಾವುದೇ ಸಂಶಯವಿರಲಿಲ್ಲ. ಅವರು ೧೮೯೪ರ ಸೆಪ್ಟೆಂಬರಿನಲ್ಲಿ ಹರಿದಾಸ್ ದೇಸಾಯಿವರಿಗೆ ಬರೆದ ಪತ್ರದಲ್ಲಿ ಹೀಗೆ ಹೇಳಿದರು:

“ಇಷ್ಟು ಕಾಲವೂ ನಾನು ಈ ದೇಶದಾದ್ಯಂತ ಸಂಚರಿಸುತ್ತ ಪ್ರತಿಯೊಂದನ್ನೂ ಗಮನಿಸುತ್ತಿ ದ್ದೇನೆ. ಕಡೆಗೀಗ ನಾನೊಂದು ತೀರ್ಮಾನಕ್ಕೆ ಬಂದಿದ್ದೇನೆ. ಏನೆಂದರೆ ಈ ಪ್ರಪಂಚದಲ್ಲಿ ಧರ್ಮ ವನ್ನು ನಿಜವಾಗಿ ಅರ್ಥಮಾಡಿಕೊಂಡಿರುವ ಒಂದೇ ಒಂದು ರಾಷ್ಟ್ರವೆಂದರೆ ಅದು ಭಾರತ; ಮತ್ತು, ಅವರ ಎಲ್ಲ ಲೋಪದೋಷಗಳನ್ನೂ ಪರಿಗಣಿಸಿದರೂ, ನೈತಿಕತೆ ಹಾಗೂ ಆಧ್ಯಾತ್ಮಿಕತೆ ಯಲ್ಲಿ ಹಿಂದೂಗಳು ಜಗತ್ತಿನ ಇತರೆಲ್ಲ ಜನರಿಗಿಂತಲೂ ಎಷ್ಟೋ ಪಾಲು ಉತ್ತಮರು. ಮತ್ತು ಭಾರತದ ಎಲ್ಲ ತ್ಯಾಗೀ ಯುವಕರು ಸರಿಯಾದ ರೀತಿಯಲ್ಲಿ ಎಚ್ಚರಿಕೆಯಿಂದ ಶ್ರಮವಹಿಸಿ ದುಡಿದು, ಪಾಶ್ಚಾತ್ಯರ ಶಕ್ತಿಯ ಹಾಗೂ ಧೀರತೆಯ ಅಂಶಗಳನ್ನು ಹಿಂದೂಗಳ ಶಾಂತಿಯ ಅಂಶ ಗಳೊಂದಿಗೆ ಸೇರಿಸಲು ಸಮರ್ಥರಾದರೆ, ಈ ಜಗತ್ತಿನಲ್ಲಿ ಹಿಂದೆ ಇದ್ದ ಎಲ್ಲ ಬಗೆಯ ಜನಗಳಿ ಗಿಂತಲೂ ಶ್ರೇಷ್ಠತಮರಾದ ಜನರು ಇಲ್ಲಿ ಉದ್ಭವಿಸುತ್ತಾರೆ ಎಂಬುದು ನನಗೆ ದೃಢವಾಗಿದೆ.”

ಹೀಗೆ ಮಾನವನ ಭವಿತವ್ಯದಲ್ಲಿ ಹಿಂದೂಧರ್ಮಕ್ಕೆ ಒಂದು ಪ್ರಮುಖ ಸ್ಥಾನವಿದೆಯೆಂದು ಸ್ವಾಮೀಜಿ ಅರಿತಿದ್ದರು. ಭಾರತದಲ್ಲಿ ಮಾತ್ರವಲ್ಲದೆ ಇಡೀ ವಿಶ್ವದ ಆಧುನಿಕ ನಾಗರಿಕತೆಯ ಬೆಳವಣಿಗೆಯಲ್ಲಿ ಧರ್ಮವು ಎಂತಹ ಸಕ್ರಿಯ-ಉಪಯುಕ್ತ ಪಾತ್ರವನ್ನು ವಹಿಸಬಲ್ಲುದೆಂಬು ದನ್ನು ತೋರಿಸಿಕೊಟ್ಟವರಲ್ಲಿ ಸ್ವಾಮಿ ವಿವೇಕಾನಂದರು ಅಗ್ರಗಣ್ಯರು, ಮೊದಲಿಗರು. ಆಧುನಿಕ ಭಾರತದ ಹಾಗೂ ಪ್ರಪಂಚದ ಸಮಕಾಲೀನ ಚಿಂತನೆಗೆ ಇದು ಅವರ ಅತಿ ಮುಖ್ಯ ಕೊಡುಗೆ ಎನ್ನಬಹುದು. ಆದರೆ ಅವರ ಈ ದಿವ್ಯ ಸಂದೇಶದ ಪ್ರಸಾರವಿನ್ನೂ ಆಗಬೇಕಾದ ಪ್ರಮಾಣದಲ್ಲಿ ಆಗಿಲ್ಲ, ಮತ್ತು ಅದರ ಘನತೆಗೆ ತಕ್ಕ ಪ್ರತಿಕ್ರಿಯೆಯೂ ಇನ್ನೂ ಕೇಳಿಬಂದಿಲ್ಲ ನಿಜ. ಆದರೆ ಚಿಂತನಪ್ರಪಂಚದ ಮೇಲೆ ಅದು ನಿಧಾನವಾಗಿಯಾದರೂ ನಿಶ್ಚಯವಾಗಿ ತನ್ನ ಪ್ರಭಾವವನ್ನು ಬೀರುತ್ತಿದೆ. ಧರ್ಮವೆಂದರೆ ತಿರಸ್ಕಾರದಿಂದ ನೋಡುವವರು ಧರ್ಮದ ಬಗೆಗಿನ ಸ್ವಾಮೀಜಿಯ ವಿಚಾರಧಾರೆ ಎಷ್ಟು ವೈಚಾರಿಕ ಪ್ರಜ್ಞೆಯಿಂದ ಕೂಡಿದುದಾಗಿತ್ತು ಎಂಬುದನ್ನು ತಿಳಿಯರು. ಸ್ವಾಮೀಜಿಯ ಕಲ್ಪನೆಯ ಧರ್ಮ ಎಷ್ಟು ಅನುಷ್ಠಾನಯೋಗ್ಯವಾಗಿತ್ತು ಎಂಬುದಕ್ಕೆ, ಅವರು ಅಳಸಿಂಗ ಪೆರುಮಾಳರಿಗೆ ಬರೆದ ಈ ಪತ್ರ ಕನ್ನಡಿಯಂತಿದೆ:

“ನಾನು ದೇವರನ್ನು ನಂಬುತ್ತೇನೆ, ಮತ್ತು ನಾನು ಮನುಷ್ಯನ ಶಕ್ತಿಯನ್ನೂ ನಂಬುತ್ತೇನೆ. ಕಷ್ಟದಲ್ಲಿರುವವರಿಗೆ ನೆರವಾಗುವುದರಲ್ಲಿ ನನಗೆ ನಂಬಿಕೆಯಿದೆ. ಇತರರನ್ನು ಉಳಿಸಲು ನಾನು ನರಕಕ್ಕಾದರೂ ಹೋಗಲು ಸಿದ್ಧನಿದ್ದೇನೆ.

“ನಾನೊಬ್ಬ ತ್ಯಾಗೀ ಸಂನ್ಯಾಸಿ. ಆದರೆ ಒಬ್ಬಳು ವಿಧವೆಯ ಕಣ್ಣೀರನ್ನು ಒರಸಲಾರದ ಅಥವಾ ಅನಾಥ ಶಿಶುವಿಗೆ ಒಂದು ತುತ್ತು ಅನ್ನವನ್ನು ಉಣಿಸಲಾರದ ದೇವರಲ್ಲೂ ಧರ್ಮದಲ್ಲೂ ನನಗೆ ನಂಬಿಕೆಯಿಲ್ಲ. ಬೋಧನೆಗಳು ಎಷ್ಟೇ ಉದಾತ್ತವಾಗಿರಲಿ, ತತ್ತ್ವಗಳು ಎಷ್ಟೇ ಅಚ್ಚುಕಟ್ಟಾಗಿ ಬೆಸೆಯಲ್ಪಟ್ಟಿರಲಿ, ಎಲ್ಲಿಯವರೆಗೆ ಆ ಧರ್ಮವು ಕೇವಲ ಪುಸ್ತಕಗಳಿಗೆ ಮತ್ತು ಸಿದ್ಧಾಂತಗಳಿಗೆ ಸೀಮಿತವಾಗಿರುತ್ತದೆಯೋ ಅಲ್ಲಿಯವರೆಗೆ ಅದನ್ನು ನಾನು ಧರ್ಮವೆಂದು ಕರೆಯುವುದಿಲ್ಲ. ಕಣ್ಣಿರುವುದು ಮುಖದಲ್ಲಿ, ಬೆನ್ನಿನಲ್ಲಲ್ಲ. ಆದ್ದರಿಂದ ನೇರವಾಗಿ ಮುನ್ನಡೆಯಿರಿ. ಮತ್ತು ಯಾವು ದನ್ನು ನೀವು ನಿಮ್ಮ ಧರ್ಮವೆಂದು ತುಂಬ ಜಂಬದಿಂದ ಹೇಳಿಕೊಳ್ಳುತ್ತೀರೋ ಅದನ್ನು ಅನುಷ್ಠಾನಕ್ಕೆ ತನ್ನಿ. ದೇವರು ನಿಮಗೆ ಒಳ್ಳೆಯದು ಮಾಡಲಿ!

“ಪ್ರೀತಿಯೆಂದೂ ಸೋಲುವುದಿಲ್ಲ ಮಗು. ಇಂದೋ ನಾಳೆಯೋ ಅಥವಾ ಯುಗಗಳ ಅನಂತರವೋ, ಸತ್ಯವೇ ಗೆಲ್ಲುವುದು. ಪ್ರೇಮವೇ ಎಂದೂ ವಿಜಯಿಯಾಗುವುದು. ನೀನು ನಿನ್ನ ಸಹಮಾನವರನ್ನು ಪ್ರೀತಿಸುವೆಯಾ? ದೇವರನ್ನು ಹುಡುಕಿಕೊಂಡು ನೀನೆಲ್ಲಿಗೆ ಹೋಗುವೆ– ದುಃಖಿಗಳು, ದರಿದ್ರರು, ದುರ್ಬಲರೆಲ್ಲ ದೇವರಲ್ಲವೆ? ಅವರನ್ನೇಕೆ ಮೊದಲು ಪೂಜಿಸಬಾರದು? ಗಂಗಾ ತೀರದಲ್ಲಿ ಬಾವಿ ತೋಡುವುದೇಕೆ? ಪ್ರೇಮದ ಸರ್ವವ್ಯಾಪಕತ್ವದಲ್ಲಿ ನಂಬಿಕೆಯಿಡು. ಕ್ಷುದ್ರವಾದ ಹೆಸರು ಕೀರ್ತಿಯನ್ನು ಯಾರು ಲಕ್ಷಿಸುತ್ತಾರೆ? ನಾನಂತೂ ಈ ವೃತ್ತಪತ್ರಿಕೆಗಳು ಏನು ಹೇಳುತ್ತವೆಯೆಂದು ಗಮನಿಸುವುದೇ ಇಲ್ಲ. ನಿನ್ನಲ್ಲಿ ಪ್ರೀತಿಯಿದೆಯೆ? ಹಾಗಿದ್ದರೆ ನೀನು ಸರ್ವಶಕ್ತ. ನೀನು ಸಂಪೂರ್ಣ ನಿಃಸ್ವಾರ್ಥಿಯೆ? ಹಾಗಾದರೆ ನೀನು ಅಪ್ರತಿಹತ. ಎಲ್ಲೆಡೆಯೂ ನಿನಗೆ ನೆರವಾಗುವುದು ನಿನ್ನ ಶೀಲವೊಂದೇ. ಕಷ್ಟಗಳ ಸಮುದ್ರದಾಳದಲ್ಲಿಯೂ ಭಗವಂತ ತನ್ನ ಮಕ್ಕಳನ್ನು ರಕ್ಷಿಸದೆ ಬಿಡುವವನಲ್ಲ. ನಿಮ್ಮ ರಾಷ್ಟ್ರಕ್ಕೆ ಧೀರರು ಬೇಕಾಗಿದ್ದಾರೆ; ನೀವು ಧೀರರಾಗಿ! ದೇವರು ನಿಮಗೆ ಒಳ್ಳೆಯದು ಮಾಡಲಿ!”

ಸ್ವಾಮೀಜಿ ಮಾನವನ ಸಮಸ್ಯೆಗಳಿಗೆ ಅವುಗಳ ಮೂಲದಲ್ಲೇ ಪ್ರಯೋಗಿಸಬಹುದಾದಂತಹ ಪರಿಹಾರಗಳನ್ನು ಕಂಡುಹುಡುಕುತ್ತಿದ್ದರು. ಆದರೆ ಸಾಮಾನ್ಯವಾಗಿ, ರೋಗದ ಲಕ್ಷಣಗಳನ್ನು ಮಾತ್ರ ಪರಿಹರಿಸುವ ಔಷಧಗಳು ರೋಗವನ್ನೇ ಗುಣಪಡಿಸಲಾರವು. ಬದಲಾಗಿ ಅವು ರೋಗ ವನ್ನು ಮತ್ತಷ್ಟು ಹದಗೆಡಿಸಬಹುದು. ಇಲ್ಲವೆ ತಕ್ಷಣದಲ್ಲೋ ಕಾಲಾಂತರದಲ್ಲೋ ಇತರ ಸಮಸ್ಯೆ ಗಳನ್ನು ತಂದೊಡ್ಡಬಹುದು. ಅಂತೆಯೇ ಯಾವುದೇ ಸಮಸ್ಯೆಯ ಬಾಹ್ಯ ಲಕ್ಷಣಗಳನ್ನು ಮಾತ್ರ ನಿರ್ಮೂಲ ಮಾಡುವ ಉಪಾಯಗಳಿಂದ, ತಕ್ಷಣಕ್ಕೆ ಆ ಸಮಸ್ಯೆ ಪರಿಹಾರವಾದಂತೆ ಕಂಡು ಬಂದರೂ, ಅದರ ಬೇರುಗಳು ಜೀವಂತವಾಗಿದ್ದು ಮತ್ತೊಮ್ಮೆ ಅದು ತಲೆಯೆತ್ತುತ್ತದೆ. ಈ ವಿಚಾರವು ಎಂದೆಂದಿಗೂ ಸತ್ಯವೆಂದು ಕಂಡುಕೊಂಡಿದ್ದ ಸ್ವಾಮೀಜಿ, ಮಾನವನ ಯಾವುದೇ ಸಮಸ್ಯೆಗೆ ಪರಿಹಾರ ಹುಡುಕುವಾಗಲೂ ಅದರ ಮೂಲಕ್ಕೆ ಹೋಗಿ, ಅದನ್ನು ಶಾಶ್ವತವಾಗಿ ನಿರ್ಮೂಲ ಮಾಡುವಂತಹ ಪರಿಹಾರವನ್ನೇ ರೂಪಿಸುತ್ತಿದ್ದರು.

ಸ್ವಾಮೀಜಿಯ ಪ್ರತಿಯೊಂದು ಕಾರ್ಯಕ್ಕೂ ಅನ್ವಯವಾಗುವಂತಿದ್ದ ಒಂದು ಸಾಮಾನ್ಯ ನಿಯಮವೆಂದರೆ ಪ್ರೀತಿ-ಸಹಾನುಭೂತಿ. ಒಂದು ಪತ್ರದಲ್ಲಿ ಸ್ವಾಮೀಜಿ ಬರೆಯುತ್ತಾರೆ, “ಏಕ ಮಾತ್ರ ಉಪಾಯವೆಂದರೆ ಪ್ರೀತಿ ಮತ್ತು ಸಹಾನುಭೂತಿ. ಏಕಮಾತ್ರ ಪೂಜೆಯೆಂದರೆ ಪ್ರೀತಿ.” ಸಕಲ ಮಾನವರನ್ನೂ ದಿವ್ಯ ಪ್ರೇಮದೆಡೆಗೆ ಒಯ್ಯುವುದೇ ಅವರ ಕಾರ್ಯೋದ್ದೇಶ.

