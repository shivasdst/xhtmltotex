
\chapter{ಭಾರತಾಂಬೆಯ ಪದತಲದಲ್ಲಿ}

\noindent

ತಿರುವನಂತಪುರದಿಂದ ಸ್ವಾಮೀಜಿ ನಾಗರಕೋಯಿಲ್ ಮಾರ್ಗವಾಗಿ ಸಾಗಿ, ಭರತಖಂಡದ ದಕ್ಷಿಣ ತುದಿಯಾದ ಪುರಾಣ ಪ್ರಸಿದ್ಧ ಕನ್ಯಾಕುಮಾರಿಯನ್ನು ತಲುಪಿದರು. ಯಾವ ಜಗನ್ಮಾತೆಯ ಕಾರ್ಯಕ್ಕಾಗಿ ಸ್ವಾಮೀಜಿ ಹೊರಟು ಬಂದಿದ್ದಾರೋ ಅದೇ ದಕ್ಷಿಣೇಶ್ವರದ ಭವತಾರಿಣಿ, ಕಾಳೀಘಾಟಿನ ಕಾಳಿ, ಇಲ್ಲಿ ಕನ್ಯಾಕುಮಾರಿಯಾಗಿ, ಕನ್ಯಕಾಪರಮೇಶ್ವರಿಯಾಗಿ ವಿರಾಜಿಸುತ್ತಿದ್ದಾಳೆ. ಯಾವ ತಾಯಿ ತನ್ನ ಸುಪುತ್ರನ ಕತ್ತು ಬಗ್ಗಿ ಕೆಲಸ ಮಾಡಿಸುತ್ತಿದ್ದಾಳೋ, ಕೈಹಿಡಿದು ನಡೆಸು ತ್ತಿದ್ದಾಳೋ ಅವಳೇ ಈಗ ಅವನನ್ನು ತನ್ನ ಪದದಡಿಗೆ ಕರೆಯಿಸಿಕೊಂಡಿದ್ದಾಳೆ. ಸ್ವಾಮೀಜಿಗೆ ಭಾರತಾಂಬೆಯಾಗಿ ದೃಗ್ಗೋಚರಳಾಗುತ್ತಿರುವ ಆಕೆಯ ಪವಿತ್ರ ಪಾದಗಳನ್ನು ಸ್ಪರ್ಶಿಸಿ ಧನ್ಯ ವಾಗಲು ಮೂರು ಸಾಗರಗಳು ನಾಮುಂದು ತಾಮುಂದೆಂದು ಧಾವಿಸಿ ಬರುತ್ತಿವೆ!

ಸಮಸ್ತ ಹಿಂದೂಗಳಿಗೆ ತೀರ್ಥಕ್ಷೇತ್ರವಾದ ಕನ್ಯಾಕುಮಾರಿಗೆ ಆಗಮಿಸಿದ ತಕ್ಷಣ ಸ್ವಾಮೀಜಿ ನೇರವಾಗಿ ದೇವಿಯ ಆಲಯವನ್ನು ಪ್ರವೇಶಿಸಿದರು. ತಾಯ ಬಳಿಗೋಡುವ ಶಿಶುವಿನಂತೆ ಕಾತರ ರಾಗಿ ಕನ್ಯಾಕುಮಾರಿಯ ದರ್ಶನ ಮಾಡಲು ಧಾವಿಸಿದರು. ದೇವಾಲಯವನ್ನು ತಲುಪಿ ದೇವಿಯ ಮುಂದೆ ಉದ್ದಂಡ ಪ್ರಣಾಮ ಮಾಡಿದಾಗ ಅವರ ಎದೆ ತುಂಬಿ ಬಂತು. ಭಾವ ಉಕ್ಕಿ ಹರಿಯಿತು. ಅವರ ಹೃದಯದಿಂದ ಪ್ರಾರ್ಥನೆಯೊಂದು ತಾನೇ ತಾನಾಗಿ ಹೊಮ್ಮಿತು:

“ಹೇ ಜಗನ್ಮಾತೆ, ನನಗೆ ಸ್ವರ್ಗ ಬೇಡ, ಮುಕ್ತಿ ಬೇಡ. ನನ್ನ ಭಾರತದ ಕೋಟಿಕೋಟಿ ದೀನ-ದಲಿತ-ದರಿದ್ರರನ್ನು ಮೇಲೆತ್ತುವ ಮಾರ್ಗ ತೋರು!”

ಸ್ವಲ್ಪ ಹೊತ್ತಿನ ನಂತರ ಮೈತಿಳಿದೆದ್ದ ಸ್ವಾಮೀಜಿ ದೇವಾಲಯದ ಆಚೆ ಬಂದು ಮಹಾಸಾಗರ ವನ್ನು ದಿಟ್ಟಿಸುತ್ತ ನಿಂತರು. ಸಮುದ್ರದ ಮಧ್ಯೆ ಎರಡು ಮಹಾಬಂಡೆಗಳು ತಲೆಯೆತ್ತಿ ನಿಂತಿವೆ. ಇವುಗಳಲ್ಲಿ ಒಂದು ದೊಡ್ಡದು. ಪುರಾಣಗಳ ಪ್ರಕಾರ ದೇವಿ ಕನ್ಯಾಕುಮಾರಿ ಶಿವನನ್ನು ಒಲಿಸಿ ಕೊಳ್ಳಲು ತಪಸ್ಸನ್ನು ಆಚರಿಸಿದ್ದು ಈ ಬಂಡೆಯ ಮೇಲೆಯೇ. ಇದನ್ನು “ಶ್ರೀಪಾದ ಶಿಲೆ” ಎಂದು ಕರೆಯುತ್ತಾರೆ. ತಪಸ್ಸನ್ನಾಚರಿಸಲು ಇದು ಅತ್ಯಂತ ಪ್ರಶಸ್ತವಾದ ಸ್ಥಳ ಎಂಬುದು ಶಾಕ್ತರ ನಂಬಿಕೆ. ಸ್ವಾಮೀಜಿಯ ದೃಷ್ಟಿ ಆ ಬಂಡೆಯ ಮೇಲೆ ನೆಟ್ಟಿತು. ಅದರ ಮೇಲೆ ಕುಳಿತು ಧ್ಯಾನನಿರತ ರಾಗಲು ಅವರು ಆಶಿಸಿದರು. ಆದರೆ ಆ ಬಂಡೆ ಇರುವುದು ಸಮುದ್ರದ ಮಧ್ಯೆ ಎರಡು ಫರ್ಲಾಂಗ್ ದೂರದಲ್ಲಿ. ಭೋರ್ಗರೆಯುತ್ತಿರುವ ಈ ಜಲರಾಶಿಯನ್ನು ದಾಟಿ ಅಲ್ಲಿಗೆ ತಲುಪುವು ದಾದರೂ ಹೇಗೆ? ಅಲ್ಲಿದ್ದ ಕೆಲವು ಅಂಬಿಗರನ್ನು ಆ ಬಂಡೆಗೆ ತಮ್ಮನ್ನು ಕರೆದೊಯ್ಯುವಂತೆ ಕೇಳಿಕೊಂಡರು. ಅಂಬಿಗರು ತಮ್ಮ ಮಾಮೂಲಿ ಶುಲ್ಕವನ್ನು ಕೇಳಿದರು–ಕೆಲವೇ ಕೆಲವು ಕಾಸುಗಳು! ಆದರೆ ಬರಿಗೈಯ ಸ್ವಾಮೀಜಿ ಬಳಿ ಬಿಡಿಗಾಸಾದರೂ ಎಲ್ಲಿಂದ ಬರಬೇಕು? ಆದರೇ ನಂತೆ, ಹೋಗಬೇಕೆಂದುಕೊಂಡಮೇಲೆ ಹೋಗಿಯೇ ತೀರುವವರು ಸ್ವಾಮೀಜಿ. ಕ್ಷಣಕಾಲವೂ ಅನುಮಾನಿಸದೆ ನೀರಿಗೆ ಧುಮುಕಿಯೇಬಿಟ್ಟರು. ದೂರವೋ ಎರಡು ಫರ್ಲಾಂಗು! ಆಳವೋ ಅಳತೆಗೇ ನಿಲುಕದ್ದು! ಸದಾ ದಡಕ್ಕೆ ತಳ್ಳುವಂಥ ಅಲೆಗಳು! ಭಯಂಕರ ಮೀನುಗಳ ಅಪಾಯ ಬೇರೆ! ಇದಾವುದನ್ನೂ ಲೆಕ್ಕಿಸದೆ ತೆರೆಗಳನ್ನು ಸೀಳಿಕೊಂಡು ಈಜುತ್ತ ಮುನ್ನಡೆದರು ಸ್ವಾಮೀಜಿ. ಸಮುದ್ರಸ್ನಾನಕ್ಕೆಂದು ಬಂದ ಕೆಲವರು ಈ ಅಪರಿಚಿತ ಸಂನ್ಯಾಸಿಯ ಎದೆಗಾರಿಕೆಯನ್ನು ಕಂಡು ದಂಗು ಬಡಿದು ನಿಂತಿದ್ದಂತೆಯೇ ಸ್ವಾಮೀಜಿ ಬಂಡೆಯನ್ನೇರಿ ಅದೃಶ್ಯರಾದರು.

ಸುತ್ತಲೂ ಸಮುದ್ರ ಏರೆತ್ತರದ ಅಲೆಗಳನ್ನೆಬ್ಬಿಸುತ್ತ ಭೋರ್ಗರೆಯುತ್ತಿದೆ. ಆದರೆ ಅವರ ಮನಸ್ಸಿನಲ್ಲಿ ತುಯ್ದಾಡುತ್ತಿರುವ ಬಗೆ ಬಗೆಯ ಭಾವದಲೆಗಳ ಮುಂದೆ ಅದ್ಯಾವ ಲೆಕ್ಕ? ಭಾರತದ ತುತ್ತತುದಿಯ ಬಂಡೆಯ ಮೇಲೆ ಕುಳಿತ ಸ್ವಾಮೀಜಿ ಗಾಢಧ್ಯಾನದಲ್ಲಿ ಮುಳುಗಿಬಿಟ್ಟರು. ಧ್ಯಾನದ ವಸ್ತು ಯಾವುದಿರಬಹುದು? ಭಾರತ! ಭಾರತದ-ಭೂತ-ಭವಿಷ್ಯತ್​-ವರ್ತಮಾನಗಳೆಲ್ಲ ಅವರ ಮನಃಪಟಲದ ಮೇಲೆ ತೆರೆ ತೆರೆಯಾಗಿ ಹಾದುಹೋಗಲಾರಂಭಿಸಿದುವು. ಭಾರತದ ಅವನತಿಗೆ ಕಾರಣವೇನೆಂಬುದನ್ನು ಅವರು ಸ್ಪಷ್ಟವಾಗಿ ಕಂಡರು. ಭವ್ಯ ಭಾರತವು ತನ್ನ ವೈಭವದ ಶಿಖರದಿಂದ ಅಧಃಪತನದ ಪಾತಾಳಕ್ಕೆ ಉರುಳಿಬೀಳಲು ಕಾರಣವಾವುದು ಎಂಬುದು ಅವರ ಪುಷಿಗಣ್ಣಿಗೆ ಕಂಡಿತು. ಆ ಬಿರುಗಾಳಿ-ಅಲೆಗಳ ಪ್ರಕ್ಷುಬ್ದ ತಾಣದಲ್ಲಿ ಕುಳಿತು ಅವರು ಭಾರತದ ಉದ್ದೇಶ-ಸಾಧನೆಗಳ ಕುರಿತಾಗಿ ಚಿಂತಿಸಿದರು. ಸಮಗ್ರ ಭಾರತದ ಹಾಗೂ ಭಾರತದ ಸಮಸ್ತ ಜನಕೋಟಿಯ ಚಿತ್ರ ಅವರ ಕಣ್ಣೆದುರು ಬಂದು ನಿಂತಿತು. ಶತಶತಮಾನಗಳ ಇತಿಹಾಸವೇ ಅವರ ಕಣ್ಮುಂದೆ ಗೋಚರಿಸಿತು. ರೂವಾರಿಯೊಬ್ಬನ ಮನಸ್ಸಿನಲ್ಲಿಇಡೀ ಕಟ್ಟಡದ ರೂಪುರೇಷೆ ಮೂಡಿ ನಿಲ್ಲುವಂತೆ ಭವಿಷ್ಯಭಾರತದ ಅಂಗಪ್ರತ್ಯಂಗವೂ ತಾನೇ ತಾನಾಗಿ ಅವರ ಮನಸ್ಸಿನಲ್ಲಿ ರೂಪು ಗೊಂಡಿತು. ಧರ್ಮವೇ ಸಮಸ್ತ ಭಾರತೀಯರ ಜೀವನಾಡಿ ಎಂಬ ಸತ್ಯ ಅವರಿಗೆ ಸುಸ್ಪಷ್ಟವಾಗಿ ಗೋಚರಿಸಿತು. ಯಾವ ಅತ್ಯುನ್ನತ ಆಧ್ಯಾತ್ಮಿಕತೆಯು ಭಾರತವನ್ನು ಸರ್ವ ರಾಷ್ಟ್ರಗಳಿಗೂ ಮಾತೃ ಸದೃಶವಾಗಿಸಿದೆಯೋ ಅಂತಹ ಆಧ್ಯಾತ್ಮಿಕತೆಯ ಪುನರ್ಜಾಗೃತಿಯಿಂದ ಮಾತ್ರವೇ ಭಾರತ ಮತ್ತೊಮ್ಮೆ ಮೇಲೇಳಲು ಸಾಧ್ಯ; ಮೇಲೆದ್ದು ದಶದಿಶೆಗೂ ತನ್ನ ಪ್ರಭಾವವನ್ನು ಬೀರಲು ಸಾಧ್ಯ ಎನ್ನುವುದನ್ನು ಅವರು ತಮ್ಮ ಹೃದಯದಾಳದ ಮೌನದಲ್ಲಿ ಸಾಕ್ಷಾತ್ಕರಿಸಿಕೊಂಡರು. ಈ ಮೂಲಕ ಅವರು ಭಾರತದ ಗರಿಮೆಯನ್ನು ಮನಗಂಡರು. ಅಂತೆಯೇ, ಯಾವುದರ ಅಭಾವದಿಂದಾಗಿ ಭಾರತವು ತನ್ನತನವನ್ನೇ ಕಳೆದುಕೊಂಡಿತೋ, ಆ ದೌರ್ಬಲ್ಯವನ್ನು ಮನಗಂಡರು. ಪುಷಿಮುನಿ ಗಳ ಸಂಸ್ಕೃತಿಯ ಪುನಸ್ಸಂಸ್ಥಾಪನೆಯಲ್ಲೇ ಭಾರತದ ಭವಿಷ್ಯವಡಗಿರುವುದು ಎಂಬುದನ್ನು ಕಂಡರು. ಭಾರತದ ಅವನತಿಗೆ ಕಾರಣ ಧರ್ಮವಲ್ಲ, ಬದಲಾಗಿ ಆ ಧರ್ಮ ಎಲ್ಲಿಯೂ ಅನುಷ್ಠಾನದಲ್ಲಿ ಕಾಣಿಸಿಕೊಳ್ಳದಿರುವುದೇ ಅವನತಿಯ ಕಾರಣ; ಬಾಳಿನೊಂದಿಗೆ ಒಂದಾಗಿ ಬೆರೆತಾಗ, ಆಮೂಲಾಗ್ರವಾಗಿ ಮಹತ್ವದ ಪರಿಣಾಮವನ್ನುಂಟುಮಾಡಬಲ್ಲ ಮಹಾಶಕ್ತಿ ಧರ್ಮ ದಂತೆ ಮತ್ತೊಂದಿಲ್ಲ–ಇದು ಸ್ವಾಮೀಜಿ ಕಂಡ ಸ್ಪಷ್ಟದರ್ಶನ.

ಇಂತಹ ಸಮಗ್ರ ಭಾರತದರ್ಶನವಾದ ಕ್ಷಣದಲ್ಲೇ ಸ್ವಾಮೀಜಿಯಲ್ಲಿ ಅಡಗಿದ್ದ ಸಮಾಜ ಸುಧಾರಕ ಜನ್ಮತಳೆದ; ಅವರಲ್ಲಡಗಿದ್ದ ರಾಷ್ಟ್ರನಿರ್ಮಾಪಕ ಮೈದಳೆದ; ಅವರಲ್ಲಡಗಿದ್ದ ಜಗದ್ರೂವಾರಿ-ವಿಶ್ವಶಿಲ್ಪಿ ಜಾಗೃತನಾದ. ಭಾರತದ ಬಡತನ-ದಾರಿದ್ರ್ಯಗಳನ್ನು ಸ್ಮರಿಸಿ ಅವರ ಹೃದಯ ಮಮ್ಮಲ ಮರುಗಿತು. ಯಾವ ಧರ್ಮವು ತನ್ನ ಕಕ್ಷೆಯಿಂದ ಲಕ್ಷಾವಧಿ ಜನಸಾಮಾನ್ಯ ರನ್ನು ಹೊರದೂಡಿಬಿಟ್ಟಿದೆಯೋ ಅಂತಹ ಧರ್ಮದಿಂದೇನು ಪ್ರಯೋಜನ? ಎಲ್ಲ ಕಾಲಗಳಲ್ಲೂ ಎಲ್ಲೆಡೆಗಳಲ್ಲೂ ಬಡಜನರು ತುಳಿತಕ್ಕೆ ಸಿಲುಕಿರುವುದೇ ಕಂಡು ಬರುತ್ತದೆ. ಪುರೋಹಿತಶಾಹಿಯ ದಬ್ಬಾಳಿಕೆ ಮತ್ತು ಸಮಾಜವನ್ನು ಚೂರುಚೂರಾಗಿ ಒಡೆದಿರುವ ಜಾತೀಯತೆಯ ಹಾವಳಿ–ಇವು ಭಾರತದ ಪ್ರಗತಿಯ ಹಾದಿಯಲ್ಲಿ ಮೇಲೆದ್ದು ನಿಂತಿರುವ ಅಭೇಧ್ಯ ಕೋಟೆಯಂತಿದೆ. ಇವುಗಳ ನ್ನೆಲ್ಲ ಆ ಸಂನ್ಯಾಸಿಯ ಹೃದಯ ಕಂಡಿತು. ಕಂಡು ಭಾವಿಸಿತು, ಭಾವಿಸಿ ಪರಿತಪಿಸಿತು. ಸ್ವಾಮೀಜಿ ತಮ್ಮ ಭಾವರೂಪದ ಮೂಲಕ ಜನಸಾಮಾನ್ಯರ ಜೀವನವನ್ನು ಪ್ರವೇಶಿಸಿದರು. ಅವರ ಸಂಕಟ ದಲ್ಲಿ ಸಹಭಾಗಿಗಳಾದರು; ಅವರ ಅವನತಿಯನ್ನು ಕಂಡು ತಾವೂ ತಲೆತಗ್ಗಿಸಿದರು; ಅವರ ಕಣ್ಣೀರಲ್ಲಿ ತಮ್ಮ ಕಣ್ಣೀರನ್ನೂ ಬೆರೆಸಿದರು. ತಾವೇ ಧರ್ಮದ ವಾರಸುದಾರರು ಎಂದು ಹೇಳಿ ಕೊಳ್ಳುತ್ತ ಗರ್ವಿಸುತ್ತಿರುವವರು ಶತಶತಮಾನಗಳಿಂದಲೂ ಜನಸಾಮಾನ್ಯರನ್ನು ಹೇಗೆ ಕೆಳ ಕ್ಕೊತ್ತುತ್ತ ಬಂದಿದ್ದಾರೆಂಬುದನ್ನು ಭಾವಿಸಿ ಭಾವಿಸಿ ಅವರ ಹೃದಯ ಹಿಂಡಿದಂತಾಯಿತು.

ಆದರೆ ಇದಕ್ಕೆಲ್ಲ ಪರಿಹಾರವೇನು? ತ್ಯಾಗ ಮತ್ತು ಸೇವೆಗಳೇ ಭಾರತದ ಅವಳಿ ಆದರ್ಶಗಳಾಗ ಬೇಕು ಎಂಬುದನ್ನು ಅವರ ಅಂತರ್ದೃಷ್ಟಿ ಕಂಡುಕೊಂಡಿತು. ರಾಷ್ಟ್ರದ ಜೀವನದಿಯನ್ನು ತೀವ್ರ ಗತಿಯಿಂದ ಈ ದಿಸೆಯಲ್ಲಿ ಹರಿಯಿಸಲು ಸಾಧ್ಯವಾದರೆ ಉಳಿದುದೆಲ್ಲವೂ ತನ್ನಷ್ಟಕ್ಕೆ ಕೈಗೂಡು ತ್ತದೆ ಎಂಬುದು ಅವರಿಗೆ ಗೋಚರಿಸಿತು. ತ್ಯಾಗ ಮಾತ್ರವೇ ಭಾರತದಲ್ಲಿ ಎಲ್ಲ ಕಾಲಕ್ಕೂ ಚಾಲಕಶಕ್ತಿಯನ್ನು ಪಡೆದುಕೊಂಡು ಬಂದಿದೆ. ಆದ್ದರಿಂದ ಸ್ವಾಮೀಜಿ ಈ ಸತ್ತ್ವಪರೀಕ್ಷೆಯ ಸಮಯದಲ್ಲಿ ತ್ಯಾಗಜೀವಿಗಳಿಗಾಗಿ ಕಾತರದ ಕಣ್ಣಿಂದ ನಿರೀಕ್ಷಿಸಿದರು. ಹಾಗೇ ಚಿಂತಿಸುತ್ತಿದ್ದಂತೆ ಅವರಿಗೊಂದು ಉಪಾಯ ಸ್ಫುರಿಸಿತು. ಅದೇನೆಂಬುದು ಮುಂದೊಮ್ಮೆ ಅವರು ಬರೆಯುವ ಪತ್ರವೊಂದರಿಂದ ತಿಳಿದುಬರುತ್ತದೆ:

“... ಕನ್ಯಾಕುಮಾರಿಯಲ್ಲಿ ಜಗನ್ಮಾತೆಯ ದೇವಾಲಯದಲ್ಲೂ, ಅನಂತರ ಸಮುದ್ರ ಮಧ್ಯದ ಬಂಡೆಯ ಮೇಲೂ ಕುಳಿತು ಧ್ಯಾನಮಗ್ನನಾದಾಗ ನನಗೊಂದು ಆಲೋಚನೆ ಹೊಳೆಯಿತು– ನಾವು ಇಷ್ಟೊಂದು ಜನ ಸಂನ್ಯಾಸಿಗಳು ನಾಡಿನ ತುಂಬೆಲ್ಲ ತಿರುಗಾಡುತ್ತ ಜನರಿಗೆ ತತ್ತ್ವ ಬೋಧನೆ ಮಾಡಿಕೊಂಡಿದ್ದೇವೆ. ಆದರೆ ಇದೆಲ್ಲ ಎಂತಹ ಹುಚ್ಚುತನ! ನಮ್ಮ ಗುರುದೇವರು ಹೇಳಲಿಲ್ಲವೆ–‘ಹಸಿದ ಹೊಟ್ಟೆ ಧರ್ಮ ಬೋಧನೆಗೆ ಯೋಗ್ಯವಲ್ಲ’ ಎಂದು? ಅಜ್ಞಾನ-ದಾರಿದ್ರ್ಯ ಗಳಿಂದಾಗಿ ಬಡಜನರು ಮೃಗಗಳಂತೆ ಜೀವಿಸುತ್ತಿದ್ದಾರೆ. ನಾವು ಯುಗಯುಗಗಳಿಂದಲೂ ಅವರ ರಕ್ತ ಹೀರುತ್ತ ಅವರನ್ನು ಕಾಲಿನಡಿಯಲ್ಲಿ ಹೊಸಕುತ್ತಿದ್ದೇವೆ...

“ಇತರರ ಒಳಿತಿಗಾಗಿಯೇ ಪಣತೊಟ್ಟುನಿಂತ ಕೆಲವು ನಿಸ್ವಾರ್ಥ ಸಂನ್ಯಾಸಿಗಳು ಹಳ್ಳಿಹಳ್ಳಿಗೂ ಹೋಗಿ ಈ ಜನರಿಗೆ ವಿದ್ಯಾಭ್ಯಾಸವನ್ನು ಕೊಡುತ್ತ, ಚಂಡಾಲರಿಂದ ಮೊದಲ್ಗೊಂಡು ಪ್ರತಿ ಯೊಬ್ಬರ ಒಳಿತಿನ ಮಾರ್ಗವನ್ನು ಹುಡುಕಬೇಕು; ಭೂಪಟ, ಕ್ಯಾಮರಾ, ಭೂಗೋಳಗಳ ಮೂಲಕ ಎಲ್ಲರಿಗೂ ಬೋಧಿಸಬೇಕು; ಇದು ಕಾಲಕ್ರಮದಲ್ಲಿ ಒಳಿತನ್ನು ತರಲಾರದೇನು?– ಎಂದು ಆಲೋಚಿಸಿದೆ. ನನ್ನ ಎಲ್ಲ ಯೋಜನೆಗಳನ್ನೂ ಈ ಚಿಕ್ಕ ಪತ್ರದಲ್ಲಿ ಬರೆಯಲು ಸಾಧ್ಯ ವಿಲ್ಲ. ಒಟ್ಟಿನಲ್ಲಿ ಸಾರಾಂಶವೇನೆಂದರೆ ‘ಬೆಟ್ಟ ಮಹಮದನ ಬಳಿಗೆ ಬರದಿದ್ದರೆ ಮಹಮದನೇ ಬೆಟ್ಟದ ಬಳಿಗೆ ಹೋಗಬೇಕು’ (ಇಂಗ್ಲಿಷ್ ಗಾದೆ). ನಮ್ಮ ದೇಶದ ಬಡವರೆಲ್ಲ ಶಾಲೆಗಳಿಗೂ ಹೋಗಲಾರದಷ್ಟು ದರಿದ್ರರು. ಅಲ್ಲದೆ, ಕಾವ್ಯ-ಸಾಹಿತ್ಯಗಳನ್ನು ಓದುವುದರಿಂದ ಅವರಿಗೇನೂ ಪ್ರಯೋಜನವಿಲ್ಲ... ಅವರಿಗೆ ಅತ್ಯಾವಶ್ಯಕವಾದ ಮೂಲಭೂತ ವಿದ್ಯಾಭ್ಯಾಸವನ್ನು ಕೊಡಬೇಕು....ನಾವು ನಮ್ಮತನವನ್ನೇ ಕಳೆದುಕೊಂಡುಬಿಟ್ಟಿದ್ದೇವೆ. ಭಾರತದ ಎಲ್ಲ ದುರವಸ್ಥೆಗೂ ಅದೇ ಕಾರಣ. ನಾವು ನಮ್ಮ ರಾಷ್ಟ್ರಕ್ಕೆ ಅದು ಕಳೆದುಕೊಂಡ ತನ್ನತನವನ್ನು ಮರಳಿಕೊಡಬೇಕು; ಮತ್ತು ಜನಸಾಮಾನ್ಯರನ್ನು ಮೇಲೆತ್ತಬೇಕು. ಹಿಂದೂಗಳು, ಮುಸಲ್ಮಾನರು ಮತ್ತು ಕ್ರೈಸ್ತರು– ಎಲ್ಲರೂ ಜನಸಾಮಾನ್ಯರನ್ನು ಕಾಲಿನಡಿಯಲ್ಲಿ ಹಾಕಿ ಮೆಟ್ಟಿದ್ದಾರೆ. ಈಗ ಅವರನ್ನು ಮೇಲೆತ್ತಬಲ್ಲ ಕಾರ್ಯ ನಮ್ಮವರಿಂದಲೇ, ಎಂದರೆ ಸಂಪ್ರದಾಯಸ್ಥ ಹಿಂದುಗಳಿಂದಲೇ ಆಗಬೇಕಾಗಿದೆ. ಎಲ್ಲ ರಾಷ್ಟ್ರಗಳಲ್ಲೂ ದೋಷವಿರುವುದು ಧರ್ಮದಲ್ಲಲ್ಲ, ಜನರಲ್ಲಿ. ಅದ್ದರಿಂದ ದೂರಬೇಕಾದದ್ದು ಧರ್ಮವನ್ನಲ್ಲ, ಜನರನ್ನು.”

ಆದರೆ, ತಾನೊಬ್ಬ ಬಡ ಸಂನ್ಯಾಸಿ. ತಾನೊಬ್ಬ ಏನು ಮಾಡಲು ಸಾಧ್ಯ?... ಈ ಹತಾಶೆಯ ಮಧ್ಯದಲ್ಲೂ ಅವರೊಲ್ಲೊಂದು ಸ್ಫೂರ್ತಿ ಚಿಮ್ಮಿತು. ಅವರು ಭಾರತದ ಉದ್ದಗಲಕ್ಕೂ ಸಂಚರಿಸಿದವರಲ್ಲವೆ? ತಾವು ಭೇಟಿ ನೀಡಿದ್ದ ಪ್ರತಿಯೊಂದು ಊರಿನಿಂದಲೂ ಕನಿಷ್ಠಪಕ್ಷ ಹತ್ತು-ಹನ್ನೆರಡು ಜನರಾದರೂ ತಮ್ಮ ಸಹಾಯಕ್ಕೆ ಬರುವರೆಂಬ ಭರವಸೆ ಅವರಲ್ಲಿ ಉದಿಸಿತ್ತು. ಆದರೆ ಹಣವೆಲ್ಲಿಂದ ಬರಬೇಕು? ಮನುಷ್ಯರೇನೋ ಸಿಗಬಹುದು; ಆದರೆ ಹಣ? ಈಗ ಸ್ವಾಮೀಜಿ ಹತಾಶ ಹೃದಯರಾಗಿ ಕಡಲಾಚೆಯ ದಿಗಂತವನ್ನೇ ನಿಟ್ಟಿಸುತ್ತ ನಿಂತರು. ಇದ್ದಕ್ಕಿದ್ದಂತೆ ಅವರ ಮನದಲ್ಲೊಂದು ಮಿಂಚು ಹೊಳೆದಂತಾಯಿತು –“ಹೌದು! ಭಾರತದ ಕೋಟ್ಯಂತರ ಜನರ ಪರವಾಗಿ ನಾನು ಅಮೆರಿಕೆಗೆ ಹೋಗಬೇಕು. ಅಲ್ಲಿ ನನ್ನ ಮೇಧಾಶಕ್ತಿಯನ್ನು ಉಪಯೋಗಿಸಿ ಹಣಗಳಿಸಬೇಕು. ಬಳಿಕ ಭಾರತಕ್ಕೆ ಹಿಂದಿರುಗಿ ನನ್ನ ದೇಶಬಾಂಧವರ ಪುನರುದ್ಧಾರಕ್ಕಾಗಿ ಅನವರತ ಶ್ರಮಿಸಬೇಕು–ಇಲ್ಲವೆ ಆ ಪ್ರಯತ್ನದಲ್ಲೇ ಶರೀರವನ್ನು ತ್ಯಜಿಸಬೇಕು... ಇಡೀ ಜಗತ್ತಿನಲ್ಲಿ ನನಗೆ ಬೇರೆ ಯಾರೂ ಸಹಾಯ ಮಾಡದಿದ್ದರೂ ಚಿಂತೆಯಿಲ್ಲ. ಒಬ್ಬರಂತೂ ನನ್ನ ಬೆನ್ನ ಹಿಂದೆ ಸದಾ ಇದ್ದೇ ಇದ್ದಾರೆ. ಹೌದು; ಅವರೇ ನನ್ನ ಪರಮ ಗುರು. ಅವರು ನನಗೆ ದಾರಿ ತೋರುತ್ತಾರೆ; ನನ್ನ ಕಾರ್ಯದಲ್ಲಿ ನೆರವಾಗುತ್ತಾರೆ.”

ಭಾರತದ ಜನಕೋಟಿಯ ಬಗ್ಗೆ ಸ್ವಾಮೀಜಿ ಅನೇಕ ವರ್ಷಗಳಿಂದ ಯಾವ ಚಿಂತೆಯಲ್ಲಿ ಮುಳುಗಿದ್ದರೋ, ಆ ಎಲ್ಲ ಚಿಂತನೆಗಳು ಇಂದು ಇಲ್ಲಿ ಕನ್ಯಾಕುಮಾರಿಯಲ್ಲಿ ಉತ್ತುಂಗ ಶಿಖರ ವನ್ನು ಮುಟ್ಟಿದುವು. ದೀನದಲಿತರ ಉದ್ಧಾರಕ್ಕಾಗಿ ಮಾರ್ಗವೊಂದನ್ನು ಕಂಡುಹಿಡಿಯಲು ಅವರು ಇಷ್ಟು ದಿನಗಳಿಂದ ಕಾತರಹೃದಯರಾಗಿ ತೊಳಲುತ್ತಿದ್ದುದರ ಫಲ ಇಂದಿಲ್ಲಿ ಸಿಕ್ಕಿದೆ! ಅವರ ನಯನಗಳು ಅಶ್ರುಜಲದಿಂದ ಮಂಜಾದುವು. ಕೃತಜ್ಞತಾಭಾವದಿಂದ ಅವರ ಶಿರ ಶ್ರೀಗುರು- ಶ್ರೀಮಾತೆಯರ ಪಾದಗಳಿಗೆ ಬಾಗಿಬಾಗಿ ನಮಿಸಿತು. ಈ ಕ್ಷಣದಿಂದ ಅವರ ಜೀವನ ಭಾರತದ ಸೇವೆಗಾಗಿ ಮುಡಿಪಾಯಿತು. ದೀನ ನಾರಾಯಣರ, ಹಸಿದ ನಾರಾಯಣರ, ದಲಿತ ನಾರಾಯಣರ ಸೇವೆಗಾಗಿ ಅವರ ಕೈ ಕಂಕಣಬದ್ಧವಾಯಿತು. ಈಗ ಅವರ ಕನಸುಮನಸುಗಳಲ್ಲೂ ಇದೊಂದೇ ಭಾವ-ಭಾರತೀಯರ ಸರ್ವತೋಮುಖ ಒಳಿತಿಗಾಗಿ ತನ್ನ ಸರ್ವಸ್ವವನ್ನೂ ಸಮರ್ಪಿಸಿಬಿಡಬೇಕು; ಕರ್ಮಸಮುದ್ರದ ಆಳಕ್ಕೆ ಒಮ್ಮೆಗೇ ಧುಮುಕಿಬಿಡಬೇಕು ಎಂದು.

ಆದರೆ.... ಆದರೆ.... ನಿರ್ವಿಕಲ್ಪಸಮಾಧಿ....? ಆತ್ಮಾನಂದ....? ತನ್ನ ತಾನರಿತು ತನ್ನಲ್ಲೇ ತಾ ಬೆರೆತು ತನ್ನತನದೊಂದಿಗೆ ಒಂದಾಗುವ ಆ ಪರಮಸುಖ....? ಬೇಡವೆ....? ಛೆ! ಅದಾಕಡೆಗಿರಲಿ. ಈಗ ಕಣ್ಮುಂದೆ ಕಾಣುತ್ತಿರುವವರಾದರೂ ಯಾರು? ನಾರಾಯಣರು! ದೀನ ನಾರಾಯಣರು! ದರಿದ್ರ ನಾರಾಯಣರು! ಹಸಿದ ನಾರಾಯಣರು! ಸಾಕ್ಷಾತ್ ನಾರಾಯಣರು! ಅವರೇ ನಾರಾಯಣರು, ಅವರೇ ತಾನು! ಅವರು ಮುಕ್ತರಾದ ಮೇಲಲ್ಲವೆ ತನ್ನ ಮುಕ್ತಿ?

ಸ್ವಾಮೀಜಿಯ ಪಾಲಿಗೆ ಧರ್ಮ ಎನ್ನುವುದು ಎಲ್ಲಕ್ಕಿಂತ ಎತ್ತರದಲ್ಲಿ ಉಳಿದು ನಿಂತ ಬೇರೆಯೇ ಒಂದು ಕಾರ್ಯಕ್ಷೇತ್ರವಾಗಿರಲಿಲ್ಲ. ಎಲ್ಲದರೊಳಗೊಂದಾಗಿ ಎಲ್ಲವನ್ನೂ ತನ್ನೊಳಗೆ ಅಡಗಿಸಿಕೊಳ್ಳುವ ಅಪಾರ ಶಕ್ತಿ ಅದಾಗಿತ್ತು. ಅವರ ಪಾಲಿಗೆ ವೇದಗಳು, ಪುಷಿಮುನಿಗಳು, ಧ್ಯಾನ, ತಪಸ್ಸು, ಆತ್ಮದರ್ಶನ, ಜನ, ಅವರ ಬದುಕು, ಅವರ ಆಶೋತ್ತರಗಳು, ಅವರ ನೋವುನಲಿವು ಗಳು–ಇವೆಲ್ಲವೂ ಧರ್ಮದ ವಿರಾಟ್​ವದನದಲ್ಲೇ ಅಡಗಿರುವಂಥವು. ದೀನದಲಿತರ ಕಣ್ಣೀರನ್ನು ಒರಸದ ಧರ್ಮವು ಒಣಗಿದ ತೃಣಕ್ಕೆ ಸಮಾನ ಎಂಬಂತೆ ಅವರಿಗನ್ನಿಸಿತು. ಹೌದು, ನಿಜಕ್ಕೂ ಅವರು ಕನ್ಯಾಕುಮಾರಿಯ ಶಿಲೆಯ ಮೇಲೆ ಕುಳಿತ ಆ ಮಹಾಮುಹೂರ್ತದಲ್ಲೇ ದೇಶಭಕ್ತಸಂತರಾದರು - ಸಂತದೇಶಭಕ್ತರಾದರು!

ಹೀಗೆ ಆ ಶಿಲೆಯ ಮೇಲೆ ಮಾಡಿದ ಧ್ಯಾನದ ಕಾಲದಲ್ಲಿ ಸ್ವಾಮೀಜಿಗೆ ತಾವು ಅಮೆರಿಕಕ್ಕೆ ಹೋಗುವ ಸಂಗತಿ ಖಚಿತವಾದಂತಾಯಿತು; ದೈವೀ ಆದೇಶ ದೃಢವಾಯಿತು. ಸ್ವಪ್ರತಿಷ್ಠೆಯಿಂದ ಮೆರೆಯುತ್ತಿರುವ ಪಾಶ್ಚಾತ್ಯ ದೇಶಗಳ ತಲೆಯನ್ನು ಪ್ರಾಚೀನ ಪುಷಿಮುನಿಗಳ ಮಹತ್ತರ ಅನುಭವಪೂರ್ಣ ಸಂದೇಶದ ಮುಂದೆ ನಮ್ರಗೊಳಿಸಿ ಬಾಗಿಸುವುದಾಗಿತ್ತು ಅವರ ಮಹಾ ಕಾರ್ಯ! ಯಾವ ತಮ್ಮ ಜನಾಂಗದ ಪರಮಾದರ್ಶಸಾಧನೆಗೆ ಭಾರತದ ಸಂನ್ಯಾಸಿಗಳು ಕತ್ತಿಯಲಗಿನ ಪಥದ ಮೇಲೆ ನಡೆದು ಹಗಲಿರುಳು ಶ್ರಮಿಸುತ್ತಾರೆಯೋ ಅಂತಹ ಆದರ್ಶವನ್ನು ಪಾಶ್ಚಾತ್ಯರ ಮುಂದೆ ಭವ್ಯ ನಿಲುವಿನಲ್ಲಿ ಮೆರೆಯಿಸುವುದು ಅವರ ಮಹೋದ್ದೇಶ! ಸಮಸ್ತ ಭಾರತವನ್ನೇ ಬೆಳಗಬಲ್ಲ ಭವ್ಯ ಜ್ಯೋತಿಯನ್ನು ಮುಗಿಲೆತ್ತರಕ್ಕೆ ಎತ್ತಿ ಹಿಡಿಯುವುದು ಅವರ ಮಹಾ ಸಂಕಲ್ಪ! ನಿರ್ವಿಕಲ್ಪ ಸಮಾಧಿಯ ಸುಖವನ್ನು ತೊರೆದು ಸಹಜನರ ಸೌಭಾಗ್ಯ ಸಾಧನೆಗೆ ಶ್ರಮಿಸುವುದರಲ್ಲೇ ಅವರಿಗೆ ಮಹಾತೃಪ್ತಿ!

ಹೀಗೆ ಶ್ರೀರಾಮಕೃಷ್ಣರ ಚೇತನಜ್ಯೋತಿ ನರೇಂದ್ರಮುನಿಯನ್ನು ಕೊಸರಾಡುವುದಕ್ಕೂ ಎಡೆ ಗೊಡದೆ ತನ್ನ ಭೀಮಬಂಧನದಲ್ಲಿ ಬಿಗಿಹಿಡಿದು, ಜಗದೆಡೆಗೆ-ಜಗದ ಜನರೆಡೆಗೆ-ಜನರ ಶ್ರೇಯ ದೆಡೆಗೆ-ಶ್ರೇಯಸ್ಸಾಧನೆಯ ಕಾರ್ಯದೆಡೆಗೆ ಅತಿವೇಗದಿಂದ ಎಳೆದು ತರುತ್ತಿದೆ. ಆ ಚೇತನಜ್ಯೋತಿ ತನ್ನನ್ನು ನಡೆಸಿದಂತೆ ನಡೆದು, ನುಡಿಸಿದಂತೆ ನುಡಿಯುವುದನ್ನು ಹೊರತು ಇನ್ನೇನು ಮಾಡಲೂ ಸಾಧ್ಯವಿಲ್ಲ. ‘ಕೀಲಿಕೈ’ ತನ್ನಲ್ಲಿಲ್ಲವಲ್ಲ!

ಆ ಪವಿತ್ರ ಶಿಲೆಯ ಮೇಲೆ, ಒಂದಲ್ಲ, ಎರಡಲ್ಲ, ಮೂರು ದಿನ ಹೀಗೆ ನಿರಂತರವಾಗಿ ಕುಳಿತು ಬಿಟ್ಟರು ಸ್ವಾಮೀಜಿ! ಅವರು ಕನ್ಯಾಕುಮಾರಿಗೆ ಬಂದಾಗಿನಿಂದಲೂ ಆ ಊರಿನ ಸದಾಶಿವಂ ಪಿಳ್ಳೆ ಎಂಬವರು ಅವರ ವ್ಯಕ್ತಿತ್ವದಿಂದ ಆಕರ್ಷಿತರಾಗಿ, ಅವರ ನಡೆನುಡಿಯನ್ನೆಲ್ಲ ಗಮನಿಸು ತ್ತಿದ್ದರು. ಅವರು ಸಮುದ್ರ ಮಧ್ಯದ ಬಂಡೆಯೆಡೆಗೆ ಈಜಿಕೊಂಡು ಹೋದದ್ದನ್ನೂ ಪಿಳ್ಳೆಯವರು ಗಮನಿಸಿದ್ದರು. ಅವರು ಧ್ಯಾನ ಮಾಡುತ್ತಿದ್ದುದನ್ನು ಕಣ್ಣಾರೆ ಕಂಡ ಮತ್ತೊಬ್ಬರೆಂದರೆ ರಾಮ ಸುಬ್ಬಯ್ಯರ್ ಎಂಬವರು. ಮೊದಲ ದಿನ ಸಂಜೆಯಾದರೂ ಸ್ವಾಮೀಜಿ ಹಿಂದಿರುಗದಿದ್ದಾಗ ಪಿಳ್ಳೆಯವರಿಗೆ ಕಳವಳವಾಯಿತು. ಮರುದಿನ ಬೆಳಿಗ್ಗೆ ಅವರು ಸ್ವಾಮೀಜಿಗಾಗಿ ಆಹಾರ ತೆಗೆದು ಕೊಂಡು ದೋಣಿಯಲ್ಲಿ ಆ ಬಂಡೆಗೆ ಹೋದರು. ಅಲ್ಲಿಗೆ ಹೋಗಿ ನೋಡಿದಾಗ ಸ್ವಾಮೀಜಿ ಧ್ಯಾನಸ್ಥರಾಗಿದ್ದುದನ್ನು ಕಂಡು, ಅವರನ್ನು ಹಿಂದಿರುಗಿ ಬರುವಂತೆ ಕೇಳಿಕೊಂಡರು. ಆದರೆ ಸ್ವಾಮೀಜಿ ನಿರಾಕರಿಸಿದರು. ತಾವು ತಂದಿದ್ದ ಆಹಾರವನ್ನಾದರೂ ಸ್ವೀಕರಿಸುವಂತೆ ಕೇಳಿ ಕೊಂಡಾಗ. ಸ್ವಾಮೀಜಿಯೆಂದರು, “ನೋಡಿ, ದಯವಿಟ್ಟು ಈಗ ತೊಂದರೆ ಕೊಡಬೇಡಿ. ಬೇಕಾ ದರೆ, ನೀವು ತಂದಿರುವ ಆ ಹಾಲು-ಹಣ್ಣನ್ನು ಅಲ್ಲೇ ಯಾವುದಾದರೂ ಕುಳಿಯಲ್ಲಿ ಇಟ್ಟುಹೋಗಿ. ನನಗೆ ಬೇಕಾದಾಗ ತೆಗೆದುಕೊಳ್ಳುತ್ತೇನೆ.” ಬಳಿಗೆ ಬಂದ ಊಟ ತಿಂಡಿಯ ಕಡೆಗೂ ಗಮನ ಕೊಡದಿರಬೇಕಾದರೆ ಅವರ ಮನಸ್ಸು ಭಾರತ ಹಿತಚಿಂತನೆಯಲ್ಲಿ ಅದೆಷ್ಟು ತನ್ಮಯಗೊಂಡಿದ್ದಿರ ಬೇಕು! ಆದರೆ ಯಾವ ಊಟ-ವಸತಿಯ ವ್ಯವಸ್ಥೆಯೂ ಇಲ್ಲದೆ ಅವರು ಅಲ್ಲಿ ಹೇಗೆ ಇದ್ದಿರ ಬಹುದು? ಶಿಲೆಯನ್ನೇ ಹಾಸಿಗೆಯನ್ನಾಗಿಸಿಕೊಂಡು ಆಕಾಶವನ್ನೇ ಹೊದಿಕೆಯಾಗಿಸಿಕೊಂಡು ರಾತ್ರಿಯನ್ನು ಕಳೆದಿರಬಹುದೆ? ಅಥವಾ ಹಗಲಿರುಳೆನ್ನದೆ ನಿರಂತರವಾಗಿ ಧ್ಯಾನಲೀನರಾಗಿಯೇ ಕುಳಿತಿದ್ದಿರಬಹುದೆ?....

ನಾಲ್ಕನೆಯ ದಿನ ಸದಾಶಿವಂ ಪಿಳ್ಳೆ ಹಾಗೂ ಇತರ ಕೆಲವರು ‘ಕಾಟಮಾರನ್​’ ಎಂದು ಕರೆಯಲ್ಪಡುವ ದೋಣಿಯಲ್ಲಿ ಹೋಗಿ ಸ್ವಾಮೀಜಿಯನ್ನು ಕರೆತಂದರು. ತೀರಕ್ಕೆ ಹಿಂದಿರುಗಿದಾಗ ಹಲವಾರು ಜನ ಅವರನ್ನು ಮುತ್ತಿಕೊಂಡು ಬಗೆಬಗೆಯ ಪ್ರಶ್ನೆಗಳನ್ನು ಕೇಳಲಾರಂಭಿಸಿದರು. ತಾವು ಶ್ರೀರಾಮಕೃಷ್ಣ ಪರಮಹಂಸರ ಶಿಷ್ಯರೆಂದೂ ಅವರ ಬಗ್ಗೆ ಇಡೀ ಜಗತ್ತೇ ಶೀಘ್ರದಲ್ಲಿ ಅರಿಯಲಿದೆಯೆಂದೂ ಉತ್ತರಿಸಿದರು ಸ್ವಾಮೀಜಿ. ಅವರ ಈ ತಪಸ್ಸಿನ ಉದ್ದೇಶವೇನು? ಅವರಿಗಾದ ಅನುಭವವೆಂಥದು? ಎಂದು ಜನ ಕೇಳಿದಾಗ ಸ್ವಾಮೀಜಿ, “ನಾನು ಯಾವ ಒಂದು ಉದ್ದೇಶದಿಂದ ಹಲವಾರು ವರ್ಷಗಳ ಕಾಲ ಅಲೆದಾಡಿದೆನೋ, ಯಾವ ಉದ್ದೇಶಸಿದ್ಧಿಗಾಗಿ ಚಿಂತಿಸುತ್ತಿದ್ದೆನೋ, ಅದು ಇಂದಿಲ್ಲಿ ನೆರವೇರಿತು” ಎಂದಷ್ಟೇ ಹೇಳಿ ಸುಮ್ಮನಾದರು.

ಸ್ವಾಮಿ ವಿವೇಕಾನಂದರು ತಮ್ಮ ಅಪೂರ್ವ ತಪಸ್ಸಿನಿಂದ ಪಾವನಗೈದ ಆ ಶಿಲೆ ಇಂದು ‘ವಿವೇಕಾನಂದ ಶಿಲೆ’ ಎಂದು ಜಗತ್ಪ್ರಸಿದ್ಧವಾಗಿದೆ. ಅದರ ಮೇಲೆ ಈಗ ‘ವಿವೇಕಾನಂದ ಮಂಟಪ’ ಎಂಬ ಭವ್ಯ ದೇಗುಲ ತಲೆಯೆತ್ತಿ ನಿಂತಿದೆ. ಭಾರತವು ತನ್ನ ಶ್ರೇಷ್ಠತಮ ಪುತ್ರನಿಗೆ ಈ ಮೂಲಕ ಶ್ರದ್ಧಾಂಜಲಿಯನ್ನರ್ಪಿಸಿದೆ. ಇದನ್ನು ನಿರ್ಮಿಸಿದವರು ‘ವಿವೇಕಾನಂದ ಶಿಲಾಸ್ಮಾರಕ ಸಮಿತಿ’ ಎಂಬ ಸಂಸ್ಥೆಯವರು. ಈ ಮಂಟಪವು ಕೇವಲ ಒಂದು ಸುಂದರ ಕಟ್ಟಡವಷ್ಟೇ ಅಲ್ಲ. ಅದೊಂದು ತೀರ್ಥಕ್ಷೇತ್ರ. ಬುದ್ಧನ ಜೀವನದಲ್ಲಿ ಬೋಧಿವೃಕ್ಷಕ್ಕೆ ಯಾವ ಸ್ಥಾನವಿದೆಯೋ, ಸ್ವಾಮಿ ವಿವೇಕಾನಂದರ ಜೀವನದಲ್ಲಿ ಈ ಶಿಲೆಗೆ ಅದೇ ಸ್ಥಾನ. ‘ವಿವೇಕಾನಂದ ಮಂಟಪ’ಕ್ಕೆ ತಗಲಿದ ಒಟ್ಟು ವೆಚ್ಚ ಸುಮಾರು ಒಂದೂಕಾಲು ಕೋಟಿ ರೂಪಾಯಿಗಳು (೧೯೭ಂರಲ್ಲಿ). ಇದರಲ್ಲಿ ಹೆಚ್ಚಿನಂಶವನ್ನು ವಂತಿಗೆಯಾಗಿ ನೀಡಿದವರು ಭಾರತದ ಲಕ್ಷಾಂತರ ಜನಸಾಮಾನ್ಯರು. ಅದನ್ನು ಸಂಗ್ರಹಿಸಿದವರು ಸಹಸ್ರಾರು ಸ್ವಯಂಸೇವಕರು. ಈ ಮಂಟಪದಲ್ಲಿ, ಆಳೆತ್ತರದ ಶಿಲಾಪೀಠದ ಮೇಲೆ ಪರಿವ್ರಾಜಕ ವಿವೇಕಾನಂದರ ಭವ್ಯವಾದ ಕಂಚಿನ ವಿಗ್ರಹವನ್ನು ಸ್ಥಾಪಿಸ ಲಾಗಿದೆ. ಈ ಮಂಟಪವು ಸ್ವಾಮೀಜಿಗೆ ತಕ್ಕುದಾದ ಸ್ಮಾರಕವೇ ಸರಿ.

ಕನ್ಯಾಕುಮಾರಿಯಲ್ಲಿ ತಮ್ಮ ಜೀವನೋದ್ದೇಶದ ಬಗ್ಗೆ ಹಾಗೂ ತಮ್ಮ ಮುಂದಿನ ಕಾರ್ಯ ಪ್ರಣಾಳಿಯ ಬಗ್ಗೆ ಅತಿಮುಖ್ಯ ನಿರ್ಧಾರವೊಂದನ್ನು ಕೈಗೊಂಡ ಸ್ವಾಮೀಜಿ, ಅಲ್ಲಿಂದ ರಾಮೇಶ್ವರ ದೆಡೆಗೆ ಹೊರಟರು. ದಾರಿಯಲ್ಲಿ ಮಧುರೆಯನ್ನು ಸಂದರ್ಶಿಸಿ, ಮೀನಾಕ್ಷಿ ಅಮ್ಮನವರ ದರ್ಶನ ಮಾಡಿದರು. ಮಧುರೆಯಲ್ಲಿ ಅವರು, ಸುಂದರರಾಮ ಅಯ್ಯರರು ಕೊಟ್ಟಿದ್ದ ಪರಿಚಯಪತ್ರದ ನೆರವಿನಿಂದ ರಾಮನಾಡಿನ ಅರಸ ಭಾಸ್ಕರ ಸೇತುಪತಿಯನ್ನು ಭೇಟಿಮಾಡಿದರು. ದಕ್ಷಿಣ ಭಾರತದ ರಾಜರಲ್ಲೆಲ್ಲ ಈತ ಅತ್ಯಂತ ತಿಳಿವಳಿಕಸ್ಥನೆಂದೂ ಸಮರ್ಥನೆಂದೂ ಹೆಸರು ಗಳಿಸಿದ್ದ. ಮೊದಲ ಭೇಟಿಯಲ್ಲೇ ಅವನು ಸ್ವಾಮೀಜಿಯಿಂದ ತೀವ್ರವಾಗಿ ಆಕರ್ಷಿತನಾಗಿ, ಅವರ ವಿಚಾರಧಾರೆಗೆ ಮಾರುಹೋದ. ಭಾರತೀಯರ ದಾರಿದ್ರ್ಯ ಹಾಗೂ ಇತರ ಸಮಸ್ಯೆಗಳು, ಅವುಗಳ ಪರಿಹಾರೋಪಾಯ, ಭಾರತದ ಸುಪ್ತ ಸಾಂಸ್ಕೃತಿಕ-ಆಧ್ಯಾತ್ಮಿಕ ಶಕ್ತಿ-ಇವುಗಳ ಬಗ್ಗೆ ಸ್ವಾಮೀಜಿ ಉಜ್ವಲವಾಗಿ ಮಾತನಾಡಿದರು. ಅವರು ವಿಶ್ವಧರ್ಮ ಸಮ್ಮೇಳನದಲ್ಲಿ ಭಾಗವಹಿಸುವ ತಮ್ಮ ಆಲೋಚನೆಯನ್ನು ಮುಂದಿಟ್ಟಾಗ, ಮಹಾರಾಜ ಅದನ್ನು ಹೃತ್ಪೂರ್ವಕವಾಗಿ ಬೆಂಬಲಿಸಿದ. ಭಾರತದ ಅಮೂಲ್ಯ ಆಧ್ಯಾತ್ಮಿಕ ತತ್ತ್ವಗಳೆಡೆಗೆ ಸಮಸ್ತ ವಿಶ್ವದ ಗಮನವನ್ನು ಸೆಳೆಯಲು ಇದೊಂದು ಸುವರ್ಣಾವಕಾಶ ಎಂದು ಅವನು ಅಭಿಪ್ರಾಯ ಪಟ್ಟ. ಸ್ವಾಮೀಜಿ ಇದರಲ್ಲಿ ಭಾಗವಹಿಸಲೇಬೇಕೆಂದು ಅವನು ಆಗ್ರಹ ಪಡಿಸಿದನಲ್ಲದೆ, ಅದಕ್ಕೆ ಬೇಕಾದ ಧನಸಹಾಯವನ್ನು ನೀಡಲೂ ಮುಂದಾದ. ಆದರೆ ಸ್ವಾಮೀಜಿ, ತಾವೀಗ ಮೊದಲು ರಾಮೇಶ್ವರ ಯಾತ್ರೆಯನ್ನು ಪೂರ್ಣಗೊಳಿಸುವುದಾಗಿ ಹೇಳಿ, ಅಮೆರಿಕೆಗೆ ಹೋಗುವ ಬಗ್ಗೆ ತಮ್ಮ ನಿರ್ಧಾರವನ್ನು ಸದ್ಯದಲ್ಲೇ ತಿಳಿಸುವುದಾಗಿ ಮಾತುಕೊಟ್ಟು, ಅವನಿಂದ ಬೀಳ್ಕೊಂಡರು.

ಮಧುರೆಯಿಂದ ಸ್ವಾಮೀಜಿ ಕಾಲ್ನಡಿಗೆಯಲ್ಲಿ ಪ್ರಯಾಣ ಮಾಡಿ ರಾಮೇಶ್ವರವನ್ನು ತಲುಪಿ ದರು. ರಾಮಾಯಣವನ್ನು ಬಲ್ಲವರಿಗೆ ರಾಮೇಶ್ವರ ಕ್ಷೇತ್ರ ಸುಪರಿಚಿತ. ಇಲ್ಲಿಗೆ ಸಮೀಪದ ಧನುಷ್ಕೋಟಿಯಿಂದಲೇ ಕಪಿಸೈನ್ಯವು ಲಂಕಾದ್ವೀಪಕ್ಕೆ ಸೇತುವೆಯನ್ನು ಕಟ್ಟಿದುದು; ರಾವಣನನ್ನು ಸಂಹರಿಸಿ ಸೀತೆಯೊಂದಿಗೆ ಮರಳಿದ ಶ್ರೀರಾಮಚಂದ್ರನು ವಿಧ್ಯುಕ್ತವಾಗಿ ಶಿವಲಿಂಗವನ್ನು ಪ್ರತಿ ಷ್ಠಾಪಿಸಿ ಪರಮೇಶ್ವರನನ್ನು ಪೂಜಿಸಿದ್ದು ಇಲ್ಲಿಯೇ. ಪವಿತ್ರ ಸಮುದ್ರಸ್ನಾನವನ್ನು ಮುಗಿಸಿ ಬಂದ ಸ್ವಾಮೀಜಿ, ಶ್ರೀರಾಮೇಶ್ವರ ಲಿಂಗವನ್ನು ಪೂಜಿಸಿ ಕೃತಕೃತ್ಯರಾದರು.

ರಾಮೇಶ್ವರ ಯಾತ್ರೆಯೊಂದಿಗೆ ಸ್ವಾಮೀಜಿಯ ಪರಿವ್ರಾಜಕ ಜೀವನದ ಒಂದು ಬಹುಮುಖ್ಯ ಹಂತ ಪೂರ್ಣಗೊಂಡಿತು. ಭಾರತ ದರ್ಶನದ ಅವಧಿಯಲ್ಲಿ ಪ್ರಮುಖ ತೀರ್ಥಕ್ಷೇತ್ರಗಳನ್ನು, ಅದರಲ್ಲೂ ಮುಖ್ಯವಾಗಿ ಚತುರ್ಧಾಮಗಳಾದ ಬದರೀ, ದ್ವಾರಕಾ, ರಾಮೇಶ್ವರ ಹಾಗೂ ಪುರೀ ಕ್ಷೇತ್ರಗಳನ್ನು ಸಂದರ್ಶಿಸುವುದು ಅವರ ಇಚ್ಛೆಯಾಗಿತ್ತು. ಅವರು ಬದರೀನಾಥನನ್ನು ಸಂದರ್ಶಿ ಸಲು ಶಕ್ತಿ ಮೀರಿ ಪ್ರಯತ್ನಿಸಿದ್ದರು; ಆದರೆ ದೈವೇಚ್ಛೆ ಪ್ರತಿಕೂಲವಾಗಿತ್ತು. ಪುರೀ ಕ್ಷೇತ್ರವನ್ನು ಸಂದರ್ಶಿಸಬೇಕೆಂಬ ಇಚ್ಛೆ ಅವರಿಗಿದ್ದರೂ ಅದು ಕೈಗೂಡಿರಲಿಲ್ಲ. ಇನ್ನುಳಿದ ದ್ವಾರಕಾ ಹಾಗೂ ರಾಮೇಶ್ವರ ಕ್ಷೇತ್ರಗಳಿಗೆ ಭೇಟಿ ನೀಡಲು ಅವರು ಸಮರ್ಥರಾಗಿದ್ದರು. ಆದರೆ ಯಾವ ಯಾವ ಕ್ಷೇತ್ರಗಳನ್ನು ಅವರು ಅಕ್ಷರಶಃ ಸಂದರ್ಶಿಸಿರಲಿಲ್ಲವೋ, ಅವುಗಳನ್ನೂ ಅವರು ಮಾನಸಿಕ ವಾಗಿ-ಆಧ್ಯಾತ್ಮಿಕವಾಗಿ ಸಂದರ್ಶಿಸಿಯಾಗಿತ್ತು. ಭೂತ-ವರ್ತಮಾನ-ಭವಿಷ್ಯತ್ ಕಾಲಗಳ ಭಾರತ ದೊಂದಿಗೆ ಹಾಗೂ ಭಾರತೀಯರೊಂದಿಗೆ ಅವರು ತಮ್ಮನ್ನು ತಾದಾತ್ಮ್ಯಗೊಳಿಸಿಕೊಂಡುಬಿಟ್ಟಿ ದ್ದರು. ಎಲ್ಲಕ್ಕಿಂತ ಹೆಚ್ಚಾಗಿ, ಈ ತೀರ್ಥಯಾತ್ರೆಯ ನೆಪದಲ್ಲಿ ಅವರು ತಮ್ಮ ಜೀವನೋದ್ದೇಶ ವನ್ನು ಕಂಡುಕೊಳ್ಳುವುದರಲ್ಲಿ ಹಾಗೂ ಅದನ್ನು ಸಾಧಿಸಲು ಬೇಕಾದ ಪೂರ್ವತಯಾರಿಯನ್ನು ಮಾಡಿಕೊಂಡು ಕಂಕಣಬದ್ಧರಾಗಿ ನಿಲ್ಲುವಲ್ಲಿ ಯಶಸ್ವಿಯಾಗಿದ್ದರು.

