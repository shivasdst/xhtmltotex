
\chapter{ವಿಖ್ಯಾತರೊಂದಿಗೆ ವಿವೇಕಾನಂದರು}

\noindent

ವಿಶ್ವಧರ್ಮಸಮ್ಮೇಳನ ಕೊನೆಗೊಂಡಿತ್ತು. ಇದೀಗ ಸ್ವಾಮೀಜಿ ಬೆಳೆಯನ್ನು ಬಿತ್ತಲು ಭೂಮಿ ಹದವಾಗಿತ್ತು. ತಮ್ಮ ಜೀವನೋದ್ದೇಶವೆಂದು ಕಂಡುಕೊಂಡಿರುವ ಮಹಾಕಾರ್ಯವನ್ನು ಪ್ರಾರಂಭಿಸಲು ಸ್ವಾಮೀಜಿ ಮಾನಸಿಕವಾಗಿ ಸಿದ್ಧರಾಗತೊಡಗಿದರು. ೧೮೯೩ರ ಅಕ್ಟೋಬರ್ ೨ ರಂದು ಅವರು ತಮ್ಮ ಪ್ರಿಯ ಸ್ನೇಹಿತರಾದ ಪ್ರೊ ॥ ರೈಟರಿಗೆ ಬರೆದ ಪತ್ರದಲ್ಲಿ ಅವರ ಮನಸ್ಥಿತಿ ವ್ಯಕ್ತವಾಗುತ್ತದೆ:

“... ನಾನೀಗ ಇಲ್ಲಿ ನನ್ನ ಜೀವನಕ್ಕೆ ಹೊಂದಿಕೊಳ್ಳುವುದರಲ್ಲಿದ್ದೇನೆ. ನನ್ನ ಜೀವನದುದ್ದಕ್ಕೂ ನನಗೆ ಒದಗಿಬಂದ ಪ್ರತಿಯೊಂದು ಪರಿಸ್ಥಿತಿಯನ್ನೂ ಭಗವಂತನ ಇಚ್ಛೆಯೆಂದೇ ತಿಳಿದು ಅದಕ್ಕೆ ಶಾಂತವಾಗಿ ಹೊಂದಿಕೊಳ್ಳುತ್ತಿದ್ದೇನೆ. ಮೊದಮೊದಲಿಗೆ ಅಮೆರಿಕದಲ್ಲಿ ನನ್ನ ಸ್ಥಿತಿ ನೀರಿನಿಂದ ಹೊರಹಾಕಿದ ಮೀನಿನಂತಾಗಿತ್ತು. ಸದಾ ಭಗವಂತನಿಂದಲೇ ಮಾರ್ಗದರ್ಶನವನ್ನು ಪಡೆಯುವ ನನ್ನ ರೂಢಮೂಲವಾದ ಅಭ್ಯಾಸವನ್ನು ಬಿಟ್ಟುಬಿಟ್ಟು ಎಲ್ಲಿ ನನ್ನ ಯೋಗಕ್ಷೇಮವನ್ನು ನಾನೇ ನೋಡಿಕೊಳ್ಳಬೇಕಾಗುತ್ತದೆಯೋ ಎಂದು ನಾನು ಹೆದರಿದ್ದೆ! ಅದೆಂತಹ ಘೋರ ಕೃತಘ್ನತೆ! ಈಗ ನಾನು ಸ್ಪಷ್ಟವಾಗಿ ಕಾಣುತ್ತಿದ್ದೇನೆ–ಯಾರು ಹಿಮಾಲಯದ ತುಷಾರಾವೃತ ಶಿಖರಗಳ ಮೇಲೂ ಭಾರತದ ಸುಡುವ ಬಯಲುಸೀಮೆಯಲ್ಲೂ ನನ್ನ ಕೈಹಿಡಿದು ನಡೆಸುತ್ತಿದ್ದನೋ ಅವನೇ ಇಂದು ಮಾರ್ಗದರ್ಶನ ಮಾಡಲು ಮತ್ತು ನೆರವಾಗಲು ಇಲ್ಲಿದ್ದಾನೆ. ಆ ಮಹಾಮಹಿಮನಿಗೆ ಜಯವಾಗಲಿ! ಅಂತೂ ನಾನೀಗ ನನ್ನ ಹಳೆಯ ಮಾರ್ಗಕ್ಕೇ ಹಿಂದಿರುಗಿದ್ದೇನೆ. ಒಬ್ಬರಲ್ಲ ಒಬ್ಬರು ನನಗೆ ಆಹಾರ-ಆಸರೆ ಒದಗಿಸುತ್ತಾರೆ. ಮತ್ತೆ ಇನ್ಯಾರಾದರೂ ಬಂದು ಭಗವಂತನ ವಿಷಯವಾಗಿ ಮಾತನಾಡುವಂತೆ ಕೇಳಿಕೊಳ್ಳುತ್ತಾರೆ. ನನಗೆ ಗೊತ್ತು–ಭಗವಂತನೇ ಅವರನ್ನು ಕಳಿಸಿರುವುದು ಎಂದು. ಅವನಾಜ್ಞೆಯನ್ನು ಪರಿಪಾಲಿಸುವುದಷ್ಟೇ ನನ್ನ ಕೆಲಸ. ಅವನೇ ನನ್ನ ಆವಶ್ಯಕತೆಗಳನ್ನೆಲ್ಲ ಒದಗಿಸಿಕೊಡುತ್ತಿದ್ದಾನೆ. ಇನ್ನು ಅವನಿಚ್ಛೆಯಂತೆಯೇ ಆಗಲಿ!

“ ‘ಯಾರು ಅನನ್ಯರಾಗಿ ನನ್ನನ್ನು ಧ್ಯಾನಿಸುತ್ತ ಉಪಾಸನೆ ಮಾಡುತ್ತಾರೆಯೋ ಅಂತಹ ಭಕ್ತರ ಯೋಗಕ್ಷೇಮವನ್ನು ನಾನು ನೋಡಿಕೊಳ್ಳುತ್ತೇನೆ’–ಇದು ಭಗವಂತನ-ಆಶ್ವಾಸನೆ.

“ಏಷಿಯಾದಲ್ಲೂ ಅಷ್ಟೆ, ಯೂರೋಪಿನಲ್ಲೂ ಅಷ್ಟೆ, ಅಮೆರಿಕದಲ್ಲಾದರೂ ಅಷ್ಟೆ, ಭಾರತದ ಮರುಭೂಮಿಗಳಲ್ಲಾದರೂ ಅಷ್ಟೆ. ಅಂತೆಯೇ ಅಮೆರಿಕದ ಜನಜಂಗುಳಿಯ ಸದ್ದುಗದ್ದಲದ ನಡುವೆಯೂ ಅಷ್ಟೆ. ಏಕೆಂದರೆ ಅವನು ಇಲ್ಲಿಯೂ ಇಲ್ಲವೆ? ಒಂದು ವೇಳೆ ಅವನಿಲ್ಲ ಎನ್ನುವು ದಾದರೆ, ನಾನು ಖಂಡಿತವಾಗಿ ತಿಳಿದುಕೊಳ್ಳುತ್ತೇನೆ–ಈ ಮೂರು ನಿಮಿಷಗಳ ಮಣ್ಣಿನ ದೇಹ ವನ್ನು ತ್ಯಜಿಸಬೇಕು ಎಂದು. ಮತ್ತು ಸಂತೋಷವಾಗಿ ಹಾಗೆಯೇ ಮಾಡುತ್ತೇನೆ.”

ಸಂನ್ಯಾಸಿಯ ಜೀವನಾದರ್ಶಗಳಾದ ತ್ಯಾಗ-ವೈರಾಗ್ಯ-ಶರಣಾಗತಿಗಳು ಸ್ವಾಮೀಜಿಯ ಸ್ವಭಾವವೇ ಆಗಿಬಿಟ್ಟಿದ್ದುವು. ಅವರು ಅವುಗಳಿಂದ ಸ್ವಲ್ಪವಾದರೂ ವಿಚಲಿತರಾಗಲು ಸಾಧ್ಯವೇ ಇಲ್ಲದಂತಹ ಸ್ಥಿತಿಯಲ್ಲಿದ್ದರು. ಆದರೆ ಅವರು ಅಮೆರಿಕೆಗೆ ಬಂದ ಉದ್ದೇಶವೇ ದ್ರವ್ಯಾರ್ಜನೆ ಮಾಡುವುದು. ಭೂಮಿಯ ಮೇಲಿನ ಅಮರಾವತಿಯಾದ ಅಮೆರಿಕದ ಜನರಿಗೆ ಅತ್ಯುನ್ನತ ಧಾರ್ಮಿಕ ಬೋಧನೆಗಳನ್ನು ನೀಡಿ, ಶ್ರೇಷ್ಠತಮವಾದ ಆಧ್ಯಾತ್ಮಿಕ ಆದರ್ಶಗಳನ್ನು ಮನವರಿಕೆ ಮಾಡಿಸಿಕೊಟ್ಟು, ಪ್ರತಿಯಾಗಿ ಅವರಿಂದ ಧನವನ್ನು ಸಂಗ್ರಹಿಸಿ ತಂದು, ಭಾರತವನ್ನು ಆರ್ಥಿಕ ವಾಗಿ ಮೇಲೆತ್ತಬೇಕೆಂಬುದು ಅವರ ಕಾರ್ಯಯೋಜನೆಯಾಗಿತ್ತು. ಆ ನಿಟ್ಟಿನಲ್ಲಿ ಅವರು ಕಾರ್ಯಾರಂಭ ಮಾಡಿದರಾದರೂ ಅದು ಅಷ್ಟು ಯಶಸ್ವಿಯಾಗಲಿಲ್ಲ. ಆದ್ದರಿಂದ ಸ್ವಾಮೀಜಿ ತತ್ಕಾಲಕ್ಕೆ ಆ ಬಗೆಯ ಯೋಜನೆಯನ್ನು ಕೈಬಿಟ್ಟರು. ಆ ಸಂಬಂಧವಾಗಿ ಅವರು ಅಕ್ಟೋಬರ್ ೨೬ರಂದು ಮತ್ತೆ ಪ್ರೊ ॥ ರೈಟರಿಗೆ ಬರೆದರು:

“ಪ್ರಿಯ ಅಧ್ಯಾಪಕ್​ಜಿ,

ನಾನಿಲ್ಲಿ ಚೆನ್ನಾಗಿದ್ದೇನೆ. ಎಲ್ಲೋ ಕೆಲವರು ತೀರ ಸಂಪ್ರದಾಯಸ್ಥರನ್ನು ಬಿಟ್ಟರೆ ಉಳಿದವ ರೆಲ್ಲರೂ ನನ್ನನ್ನು ಬಹಳ ಪ್ರೀತಿ ವಿಶ್ವಾಸದಿಂದ ನೋಡಿಕೊಳ್ಳುತ್ತಿದ್ದಾರೆ. ದೂರ ದೂರದ ದೇಶ ಗಳಿಂದ ಬಂದು ಇಲ್ಲಿ ಕಲೆತ ಜನರೆಲ್ಲ ಇಲ್ಲಿ ಕಾರ್ಯಗತಗೊಳಿಸಲು ಅನೇಕ ಯೋಜನೆಗಳನ್ನು ಹಾಕಿಕೊಂಡಿದ್ದಾರೆ. ಏಕೆಂದರೆ ಪ್ರತಿಯೊಂದು ಕೆಲಸವೂ ಯಶಸ್ವಿಯಾಗಲು ಅವಕಾಶವಿರುವ ಏಕೈಕ ಸ್ಥಳವೆಂದರೆ ಅಮೆರಿಕ. ಆದರೆ ನಾನು ಸ್ವಲ್ಪ ಯೋಚಿಸಿನೋಡಿ, ಈಗ ನನ್ನ ಯೋಜನೆಯ ಮಾತನಾಡುವುದನ್ನೇ ಬಿಟ್ಟುಬಿಟ್ಟಿದ್ದೇನೆ. ಏಕೆಂದರೆ, ‘ಈ ಪರದೇಶಿ ಬಂದು ಚೆನ್ನಾಗಿ ಬಾಚಿ ಕೊಂಡು ಹೋದ’ ಎಂದು ಇಲ್ಲಿಯವರು ಆಡಿಕೊಳ್ಳುತ್ತಾರೆ. ಆದ್ದರಿಂದ ನಾನು ನನ್ನ ಯೋಜನೆ ಯನ್ನು ಹಿನ್ನೆಲೆಯಲ್ಲೇ ಇಟ್ಟುಕೊಂಡು, ಯಾವುದೇ ಇತರ ಉಪನ್ಯಾಸಕನಂತೆ ಉಪನ್ಯಾಸಗಳನ್ನು ನೀಡುತ್ತ, ಶ್ರಮವಹಿಸಿ ದುಡಿಯಲು ಇಚ್ಛಿಸಿದ್ದೇನೆ.

“ನನ್ನನ್ನು ಇಲ್ಲಿಗೆ ಕರೆತಂದವನು ಇನ್ನೂ ನನ್ನ ಕೈಬಿಟ್ಟಿಲ್ಲ; ಮತ್ತು ನಾನಿಲ್ಲಿರುವವರೆಗೂ ಕೈಬಿಡುವುದೂ ಇಲ್ಲ. ಧನಸಂಪಾದನೆಯ ವಿಷಯವೊಂದನ್ನು ಬಿಟ್ಟು ಉಳಿದ ಎಲ್ಲ ರೀತಿಯಲ್ಲೂ ನನ್ನ ಕೆಲಸ ಚೆನ್ನಾಗಿ ಮುಂದುವರಿಯುತ್ತಿದೆ. ಇದನ್ನು ಕೇಳಿ ನೀವು ಖಂಡಿತವಾಗಿಯೂ ಸಂತೋಷ ಪಡುತ್ತೀರಿ. ಎಷ್ಟೇ ಆಗಲಿ, ಇಂತಹ ವ್ಯವಹಾರಕ್ಕೆ ನಾನಿನ್ನೂ ತೀರ ಎಳಸು. ಆದರೆ ಶೀಘ್ರದಲ್ಲೇ ಅದನ್ನೂ ಕಲಿತುಬಿಡುತ್ತೇನೆ. ಶಿಕಾಗೋದಲ್ಲಿ ನಾನು ತುಂಬ ಜನಪ್ರಿಯನಾಗಿದ್ದೇನೆ. ಆದ್ದರಿಂದ ನಾನಿಲ್ಲಿ ಕೆಲಕಾಲ ಇದ್ದು, ಆಮೇಲೆ ‘ಹಣ’ ಸಂಪಾದಿಸಬೇಕೆಂದಿದ್ದೇನೆ....”

ಇಲ್ಲಿ ಸ್ವಾಮೀಜಿ ಭಾರತೀಯರಿಗಾಗಿ ಮಾಡುತ್ತಿರುವ ತ್ಯಾಗ ಎಂತಹ ಅದ್ಭುತವಾದದ್ದು ಎಂಬುದನ್ನು ಗಮಿನಿಸಬೇಕು. ತಮ್ಮ ಸಂನ್ಯಾಸಧರ್ಮಕ್ಕೇ ವಿರುದ್ಧವಾದ ಹಣಸಂಗ್ರಹಣೆಯ ಕಾರ್ಯದಲ್ಲಿ ಅವರು ತಮ್ಮನ್ನು ತಾವು ತೊಡಗಿಸಿಕೊಳ್ಳುತ್ತಿದ್ದಾರೆ. ತಮ್ಮ ಭಾರತದ ಕೋಟಿ ಗಟ್ಟಲೆ ದರಿದ್ರನಾರಾಯಣರನ್ನು ಮೇಲೆತ್ತಬೇಕೆಂಬ ಒಂದೇ ಉದ್ದೇಶದಿಂದ, ತಮಗೆ ಸ್ವಭಾವ ಸಹಜವಾದ ಶುದ್ಧ ಸಾತ್ವಿಕ ಸಂನ್ಯಾಸಜೀವನದ ಆದರ್ಶದಿಂದ ಸ್ವಲ್ಪ ವಿಮುಖರಾಗುತ್ತಿದ್ದಾರೆ. ಆದರೆ ಜನರ ಬಾಯಿಗೆ ತುತ್ತಾಗುವುದರಿಂದ ತಪ್ಪಿಸಿಕೊಳ್ಳಲು, ಸಾರ್ವಜನಿಕವಾಗಿ ವಂತಿಗೆ ವಸೂಲಿ ಮಾಡುವುದನ್ನು ಬಿಟ್ಟು, ತಮ್ಮ ಸ್ವಂತ ಶ್ರಮದಿಂದ ಹಣವನ್ನು ಸಂಪಾದಿಸಲು ನಿಶ್ಚಯಿಸುತ್ತಿದ್ದಾರೆ.

ಈ ದಿಸೆಯಲ್ಲಿ ಆಲೋಚಿಸಿ ಸ್ವಾಮೀಜಿ, ‘ಸ್ಲೇಟನ್ ಲೈಸಿಯಮ್ ಲೆಕ್ಚರ್ ಬ್ಯೂರೋ’ ಎಂಬ ಸಂಸ್ಥೆಯೊಂದಿಗೆ ಒಂದು ಒಪ್ಪಂದ ಮಾಡಿಕೊಂಡರು. ಇಂತಹ ಉಪನ್ಯಾಸಸಂಸ್ಥೆಗಳ ಕೆಲಸವೇ ನೆಂದರೆ, ವಿವಿಧ ಸ್ಥಳಗಳಲ್ಲಿ ಪ್ರಸಿದ್ಧ ವ್ಯಕ್ತಿಗಳ ಉಪನ್ಯಾಸಗಳನ್ನು ಏರ್ಪಡಿಸುವುದು; ಅದಕ್ಕೆ ಪ್ರವೇಶಧನವನ್ನು ವಸೂಲಿ ಮಾಡಿ, ಅದರಲ್ಲಿ ನಿಗದಿತ ಪ್ರಮಾಣವನ್ನು ತಾನು ಉಳಿಸಿಕೊಳ್ಳು ವುದು. ಅಮೆರಿಕದಾದ್ಯಂತ ಸ್ವಾಮೀಜಿ ಉಪನ್ಯಾಸಗಳನ್ನು ನೀಡಬೇಕೆಂದು ತೀರ್ಮಾನವಾಯಿತು. ಇದೇ ಅತ್ಯುತ್ತಮ ಮಾರ್ಗವೆಂದು ಸ್ವಾಮೀಜಿ ಭಾವಿಸಿದರು. ಹಿಂದೂಧರ್ಮದ ಹಾಗೂ ತಮ್ಮ ಗುರುದೇವರ ಅದ್ಭುತ ವಿಚಾರಗಳನ್ನೆಲ್ಲ ಪ್ರಸಾರ ಮಾಡುತ್ತ, ಜೊತೆಗೇ ತಮ್ಮ ಭಾರತದ ಕೆಲಸ ಕ್ಕಾಗಿ ಧನ ಸಂಗ್ರಹಣೆ ಮಾಡುತ್ತ, ಇತರ ಚಿಂತೆಗಳಾವುವೂ ಇಲ್ಲದೆ ತಾವು ಎಲ್ಲೆಡೆಯೂ ಸಂಚರಿಸಬಹುದೆಂದು ಅವರು ಆಲೋಚಿಸಿದರು.

ತಮ್ಮ ಈ ಭಾಷಣದ ಕಾರ್ಯಕ್ರಮವನ್ನು ಶಿಕಾಗೋ ನಗರದಿಂದಲೇ ಪ್ರಾರಂಭಿಸಿದರು. ಸಮ್ಮೇಳನದಿಂದಾಗಿ ಇಲ್ಲಿ ಅವರ ಹೆಸರು ಮೊದಲೇ ಸಾಕಷ್ಟು ಪ್ರಸಿದ್ಧವಾಗಿತ್ತು. ಈಗ ಈ ಹೊಸ ಉಪನ್ಯಾಸಗಳಿಂದಾಗಿ ಅವರ ಕೀರ್ತಿ ಮತ್ತಷ್ಟು ಹೆಚ್ಚಾಯಿತು. ಅಲ್ಲಿನ ಒಂದು ಪತ್ರಿಕೆ ಬರೆದಂತೆ, ಸ್ವಾಮೀಜಿಯ ಇರುವಿಕೆಯಿಂದಾಗಿ ಇಡೀ ಶಿಕಾಗೋದಲ್ಲಿ ನಿತ್ಯೋತ್ಸವ ನಡೆಯುತ್ತಿದ್ದಂತಿತ್ತು.

ಸ್ವಾಮೀಜಿ ಶಿಕಾಗೋದಲ್ಲಿದ್ದ ಅವಧಿಯಲ್ಲಿ, ಶ್ರೀಮತಿ ಮತ್ತು ಶ್ರೀ ಜಾನ್ ಬಿ. ಲಿಯಾನ್ ಎಂಬ ಶ್ರೀಮಂತ ದಂಪತಿಗಳ ಮನೆಯಲ್ಲಿ ಇಳಿದುಕೊಂಡಿದ್ದರು. ಇವರ ಮನೆಗೆ ಸ್ವಾಮೀಜಿ ಯಾವಾಗ ಹೋದರು ಎಂಬುದು ಸ್ಪಷ್ಟವಾಗಿಲ್ಲ. ಬಹುಶಃ ಸಮ್ಮೇಳನದ ಆರಂಭದ ಹಿಂದಿನ ದಿನ ಹೋಗಿದ್ದಿರಬೇಕು. ಶ್ರೀ ಲಿಯಾನ್​ರವರು ಶಿಕಾಗೋದ ಅತ್ಯಂತ ಗಣ್ಯ ನಾಗರಿಕರಲ್ಲೊಬ್ಬ ರಾಗಿದ್ದರು. ಇವರು ಸಮ್ಮೇಳನದ ಪ್ರತಿನಿಧಿಯೊಬ್ಬರನ್ನು ತಮ್ಮ ಆತಿಥಿಗಳಾಗಿ ಇರಿಸಿಕೊಳ್ಳಲು ಬಯಸಿ, ವಿಶಾಲ ಮನೋಭಾವದ ಒಬ್ಬರನ್ನು ತಮ್ಮಲ್ಲಿಗೆ ಕಳಿಸಿಕೊಡುವಂತೆ ಸಮ್ಮೇಳನದ ಅಧಿಕಾರಿಗಳನ್ನು ಕೇಳಿಕೊಂಡಿದ್ದರು. ಈ ದಂಪತಿಗಳಿಬ್ಬರೂ ಅತ್ಯಂತ ಉದಾರ ಬುದ್ಧಿಯವರು. ಮತಾಂಧತೆಯನ್ನು ಅವರು ಸಹಿಸುತ್ತಿರಲಿಲ್ಲ. ಆಗ ಶಿಕಾಗೋದಲ್ಲಿ ನಡೆಯುತ್ತಿದ್ದ ವೈಭವಯುತ ವಾದ ಮೇಳವನ್ನು ನೋಡಲು ಬೇರೆಬೇರೆ ಊರುಗಳಿಂದ ಬಂದ ಸಂಬಂಧಿಕರಿಂದ ಅವರ ಮನೆ ತುಂಬಿಹೋಗಿತ್ತು. ಎಲ್ಲರೂ ತಮ್ಮ ಮನೆಗೆ ಬರಬಹುದಾದ ಪ್ರತಿನಿಧಿಯನ್ನು ಕಾಣಲು ಉತ್ಸುಕರಾಗಿದ್ದರು.

ತಮ್ಮ ಮನೆಗೆ ಯಾರಾದರೂ ಬಿಳಿಯ ಕ್ರೈಸ್ತ ಪಾದ್ರಿಗಳು ಬರುತ್ತಾರೆಂದು ಇವರು ನಿರೀಕ್ಷಿಸಿ ದ್ದರು. ಆದರೆ ರಾತ್ರಿ ಎಷ್ಟು ಹೊತ್ತಾದರೂ ನಿರೀಕ್ಷಿತ ಅತಿಥಿ ಬರದಿದ್ದರಿಂದ ಶ್ರೀಮತಿ ಎಮಿಲಿ ಲಿಯಾನ್​ರನ್ನುಳಿದು ಉಳಿದವರೆಲ್ಲ ಮಲಗಿಕೊಂಡರು. ಮಧ್ಯರಾತ್ರಿಯ ನಂತರ ಕರೆಗಂಟೆ ಬಾರಿಸಿತು. ಶ್ರೀಮತಿ ಎಮಿಲಿ ಕಾತರದಿಂದ ಬಾಗಿಲು ತೆರೆದು ನೋಡುತ್ತಾರೆ–ಕಿತ್ತಳೆ ಬಣ್ಣದ ನಿಲುವಂಗಿ, ಕೆಂಪು ಪೇಟಗಳನ್ನು ಧರಿಸಿದ ಮಂದಹಾಸವದನವಾದ ಸುಂದರ ಪೌರ್ವಾತ್ಯ ಯುವಕ ನೊಬ್ಬ ನಿಂತಿದ್ದಾನೆ!ಯಾವ ಜನ್ಮದ ಸುಕೃತವೋ! ಆ ದಂಪತಿಗಳು ವಿಶಾಲ ಮನೋಭಾವದವ ರೊಬ್ಬರನ್ನು ಕಳಿಸಿ ಕೊಡಿ ಎಂದರೆ ಸ್ವಾಮಿ ವಿವೇಕಾನಂದರನ್ನೇ ಕಳಿಸಿಕೊಟ್ಟಿದ್ದಾರೆ! ಆದರೆ ಸ್ವಾಮೀಜಿಯಿನ್ನೂ ಸಮ್ಮೇಳನದಲ್ಲಿ ಭಾಷಣ ಮಾಡಿರಲಿಲ್ಲ. ಆದ್ದರಿಂದ ಅವರಿನ್ನೂ ಶಿಕಾಗೋದ ಮಹಾಜನತೆಗೆ ಪರಿಚಿತರಾಗಿರಲಿಲ್ಲ. ಆದರೂ ಅವರನ್ನು ಕಂಡ ತಕ್ಷಣವೇ ಎಮಿಲಿ ಆಕರ್ಷಿತ ರಾದರು. ಅವರನ್ನು ಆದರದಿಂದ ಸ್ವಾಗತಿಸಿ ಉಪಚರಿಸಿದರು.

ಆದರೆ ಈಗ ಆಕೆಗೆ ಚಿಂತೆ ಹತ್ತಿಕೊಂಡಿತು. ಏಕೆಂದರೆ ಅವರ ಸಂಬಂಧಿಕರೆಲ್ಲರೂ ದಕ್ಷಿಣದ ಪ್ರಾಂತ್ಯಗಳಿಂದ ಬಂದವರು. ಹಿಂದೆ ಗುಲಾಮಗಿರಿ ವ್ಯಾಪಕವಾಗಿದ್ದ ಆ ಪ್ರಾಂತ್ಯಗಳಲ್ಲಿ ವರ್ಣ ಭೇದಬುದ್ಧಿ ಇನ್ನೂ ಹೋಗಿರಲಿಲ್ಲ. ಆದ್ದರಿಂದ ತಮ್ಮ ಆತಿಥಿಗಳು, ಪೌರ್ವಾತ್ಯರಾದ ಈ ಪ್ರತಿನಿಧಿಯನ್ನು ಕಂಡು ಏನೆನ್ನುತ್ತಾರೋ ಎಂಬ ಚಿಂತೆಯುಂಟಾಯಿತು. ಮರುದಿನ ಎದ್ದ ತಕ್ಷಣ ಅವರು ತಮ್ಮ ಚಿಂತೆಯನ್ನು ಪತಿಯ ಮುಂದೆ ತೋಡಿಕೊಂಡರು. ಜಾನ್ ಲಿಯಾನ್ ಆ ಬಗ್ಗೆ ಆಲೋಚಿಸುತ್ತ, ವೃತ್ತಪತ್ರಿಕೆಯನ್ನು ಓದಲು ತಮ್ಮ ಮನೆಯ ವಾಚನಾಲಯಕ್ಕೆ ಹೋದರು. ಅಲ್ಲಿ ಅವರು ಸ್ವಾಮೀಜಿ ಕುಳಿತಿರುವುದನ್ನು ಕಂಡರು. ಪರಸ್ಪರ ಅಭಿನಂದನೆಗಳಾದ ಮೇಲೆ ಇಬ್ಬರೂ ಮಾತನಾಡುತ್ತ ಕುಳಿತರು.

ಅರ್ಧಗಂಟೆಯೊಳಗಾಗಿ ಶ್ರೀ ಲಿಯಾನ್ ಬಂದು ತಮ್ಮ ಪತ್ನಿಗೆ ಹೇಳಿದರು, “ಎಮಿಲಿ, ನಮ್ಮ ಅತಿಥಿಗಳೆಲ್ಲರೂ ಇಲ್ಲಿಂದ ಹೊರಟುಹೋದರೂ ನಾನು ದುಃಖಿಸುವುದಿಲ್ಲ! ನಾನು ಕಂಡವ ರಲ್ಲೆಲ್ಲ ಅತ್ಯಂತ ಮೇಧಾವಿ, ಕುತೂಹಲಕರ ವ್ಯಕ್ತಿ–ಈ ಭಾರತೀಯ. ತನಗಿಷ್ಟಬಂದಷ್ಟು ದಿನ ಆತ ಇಲ್ಲಿರಬಹುದು.” ಈ ಭೇಟಿಯು ಸ್ವಾಮೀಜಿ ಹಾಗೂ ಈ ದಂಪತಿಗಳ ನಡುವಿನ ಸುದೀರ್ಘ ಸುಮಧುರ ಬಾಂಧವ್ಯಕ್ಕೆ ನಾಂದಿಯಾಯಿತು. ಜಾರ್ಜ್ ಹೇಲ್​ರ ಕುಟುಂಬದಂತೆ ಲಿಯಾನ್​ರ ಕುಟುಂಬವೂ ಸ್ವಾಮೀಜಿಗೆ ಅತ್ಯಂತ ಆತ್ಮೀಯವಾಯಿತು. ಶಿಕಾಗೋಗೆ ಬಂದಾಗಲೆಲ್ಲ ಸ್ವಾಮೀಜಿ ಈ ಎರಡು ಕುಟುಂಬಗಳ ಅತಿಥಿಗಳಾಗಿರುತ್ತಿದ್ದರು. ವೃದ್ಧ ಲಿಯಾನ್ ದಂಪತಿಗಳು ಅವರನ್ನು ತಮ್ಮ ಸ್ವಂತ ಮಗನಂತೆಯೇ ಕಾಣುತ್ತಿದ್ದರು.

ಈ ದಂಪತಿಗಳಿಗೆ ಒಬ್ಬಳು ಚಿಕ್ಕ ವಯಸ್ಸಿನ ವಿಧವೆ ಮಗಳೂ, ಆರು ವರ್ಷದ ಮೊಮ್ಮಗಳೂ ಇದ್ದರು. ಈ ಮೊಮ್ಮಗಳ ಹೆಸರು ಕಾರ್ನೆಲಿಯಾ ಕಾಂಗರ್. ಈ ಪುಟ್ಟ ಬಾಲಕಿಯ ಎಳೆಯ ಮನಸ್ಸಿನ ಮೇಲೆ ಸ್ವಾಮೀಜಿಯ ನೆನಪುಗಳು ಎಷ್ಟು ಆಳವಾಗಿ ಮುದ್ರಿತವಾಯಿತೆಂದರೆ, ಅವಳು ಆರವತ್ತೆರಡು ವರ್ಷಗಳ ಅನಂತರ, ಎಂದರೆ ೧೯೫೬ರಲ್ಲಿ, ಅವರನ್ನು ತಾನು ಕಂಡದ್ದು ನಿನ್ನೆಯೋ ಎಂಬಂತೆ, ವಿವರಪೂರ್ಣವಾದ ಸ್ಮೃತಿಲೇಖನವನ್ನು ಬರೆದಳು!

ಪುಟ್ಟ ಕಾರ್ನೆಲಿಯಾಳಿಗೆ ಸ್ವಾಮೀಜಿಯ ಪ್ರತಿಯೊಂದು ಅಂಶವೂ ಒಂದು ಕೌತುಕದ ವಿಷಯ. ಅವರ ಭವ್ಯ ವ್ಯಕ್ತಿತ್ವ, ಕಂಚಿನ ಕಂಠಧ್ವನಿ, ಮಿನುಗುವ ಕಣ್ಣುಗಳು, ಸುಂದರ ಉಡುಗೆ, ಪ್ರತಿ ಸಲವೂ ಸುತ್ತಿಕೊಳ್ಳಬೇಕಾದ ಅವರ ‘ಹ್ಯಾಟು’ (ಪೇಟ)–ಇವುಗಳನ್ನೆಲ್ಲ ಆಕೆ ಅಚ್ಚರಿಯಿಂದ ದಿಟ್ಟಿಸಿ ನೋಡುತ್ತಿದ್ದಳು. ಆಕೆಗೆ ಸ್ವಾಮೀಜಿ ಭಾರತದ ಅನೇಕ ವಿಷಯಗಳನ್ನು ಕಣ್ಣಿಗೆ ಕಟ್ಟುವಂತೆ ಹೇಳುತ್ತಿದ್ದರು. ಅವರು ಎಲ್ಲಿಗಾದರೂ ಹೋಗಿದ್ದು ಮನೆಗೆ ಹಿಂದಿರುಗಿದರೆ ಈ ಪುಟ್ಟಬಾಲೆ ತಕ್ಷಣ ಅವರ ಮಡಿಲನ್ನು ಸೇರಿ, “ನನಗೆ ಇನ್ನೊಂದು ಕಥೆ ಹೇಳಿ ಸ್ವಾಮೀಜಿ” ಎಂದು ಕಾಡುತ್ತಿದ್ದಳು.

ಶಿಕಾಗೋದಲ್ಲಿ ಸ್ವಾಮೀಜಿ ಹಲವಾರು ಕಡೆಗಳಲ್ಲಿ ಭಾಷಣ ಮಾಡಿ, ಸಂಗ್ರಹವಾದ ಹಣವನ್ನು ಮನೆಗೆ ತರುತ್ತಿದ್ದರು. ಅವರ ಬಳಿ ಪರ್ಸ್ ಇರಲಿಲ್ಲವಾದ್ದರಿಂದ ಹಣವನ್ನು ತಮ್ಮ ಕರವಸ್ತ್ರದಲ್ಲಿ ಕಟ್ಟಿಕೊಂಡು ತಂದು, ಬಾಲಕನಂತೆ ಹೆಮ್ಮೆಯಿಂದ ಅದನ್ನು ಶ್ರೀಮತಿ ಎಮಿಲಿಯವರ ಕೈಗೆ ಸುರಿದುಬಿಡುತ್ತಿದ್ದರು. ಆಕೆ ಅದನ್ನು ಎಣಿಸಿ, ಬ್ಯಾಂಕಿನಲ್ಲಿ ಇಡುತ್ತಿದ್ದರು. ಹಣದ ವಿಚಾರದಲ್ಲಿ ಸ್ವಾಮೀಜಿ ತಮ್ಮ ಗುರುವಿನಷ್ಟೇ ಮುಗ್ಧರು. ಶ್ರೀಮತಿ ಲಿಯಾನರು ಹಣವನ್ನು ಹೇಗೆ ಒಪ್ಪವಾಗಿ ಜೋಡಿಸುವುದು, ಹೇಗೆ ಎಣಿಸುವುದು ಎಂಬುದನ್ನೆಲ್ಲ ತೋರಿಸಿಕೊಟ್ಟರು.

ಒಂದು ದಿನ ಸ್ವಾಮೀಜಿ ಎಮಿಲಿ ಲಿಯಾನರ ಬಳಿಗೆ ಬಂದು “ನಾನು ಇಲ್ಲಿ ನನ್ನ ಜೀವನದ ಅತಿ ದೊಡ್ಡ ಪ್ರಲೋಭನೆಗೆ ಒಳಗಾದೆ” ಎಂದು ಉದ್ಗರಿಸಿದರು. ಆಗ ಆ ಅಜ್ಜಿ ಕೀಟಲೆಗೆ ಕೇಳಿದರು, “ಯಾರಿರಬಹುದೋ ಆ ಭಾಗ್ಯವಂತೆ!” ಆಗ ಸ್ವಾಮೀಜಿ ಗಟ್ಟಿಯಾಗಿ ನಗುತ್ತ “ಓ, ಅದು ಹೆಂಗಸಲ್ಲ, ಸಂಘಟನೆ” ಎಂದರು. ಬಳಿಕ ಅವರು ತಮಗೆ ಸಂಘಟನೆಯ ಅಂಶ ಅಷ್ಟೊಂದು ಮೆಚ್ಚುಗೆಯಾದ್ದದೇಕೆಂದು ವಿವರಿಸಿದರು–“ನೋಡಿ, ನಮ್ಮಲ್ಲಿ ಶ್ರೀರಾಮಕೃಷ್ಣರ ಸಂನ್ಯಾಸೀ ಶಿಷ್ಯರೆಲ್ಲ ಚೆಲ್ಲಾಪಿಲ್ಲಿಯಾಗಿದ್ದಾರೆ. ಅವರು ಯಾವುದಾದರೂ ಹಳ್ಳಿಗೆ ಹೋಗುತ್ತಾರೆ; ಅಲ್ಲಿ ಒಂದು ಮರದ ಬುಡದಲ್ಲಿ ಕುಳಿತುಕೊಳ್ಳುತ್ತಾರೆ–ಜನ ತಮ್ಮ ಬಳಿಗೆ ಬಂದು ಏನಾದರೂ ಸಹಾಯವನ್ನು ಪಡೆಯುವುದಾದರೆ ಪಡೆಯಲಿ, ಎಂದು ಹೀಗೆ ಆ ಸಂನ್ಯಾಸಿಗಳಿಂದ ಲೋಕ ಸೇವೆಯೇನೋ ನಡೆಯುತ್ತಿದೆಯಾದರೂ ಅದು ತೀರ ಅಲ್ಪ ಪ್ರಮಾಣದಲ್ಲಿ. ಆದ್ದರಿಂದ, ಇದೇ ಜನಬಲವನ್ನು ಸರಿಯಾಗಿ ಸಂಘಟಿಸಲು ಸಾಧ್ಯವಾದರೆ ಎಷ್ಟೊಂದು ಅದ್ಭುತವಾದ ಕಾರ್ಯ ಗಳನ್ನು ಸಾಧಿಸಬಹುದು ಎಂಬುದನ್ನು ನಾನು ಈ ಅಮೆರಿಕದಲ್ಲಿ ಕಂಡುಕೊಂಡೆ.” ಆದರೆ ಅವರು ಸಂಘಟನೆಯ ಅನುಕೂಲತೆಗಳನ್ನು ಮನಗಂಡಿದ್ದರೂ, ಯಾವ ಬಗೆಯ ಸಂಘವು ಭಾರತೀಯ ಹಿನ್ನೆಲೆಗೆ ಸೂಕ್ತವಾದೀತು ಎಂಬ ಬಗ್ಗೆ ಅವರು ಇನ್ನೂ ನಿರ್ಧಾರಕ್ಕೆ ಬಂದಿರಲಿಲ್ಲ. ಏಕೆಂದರೆ ಸಂಘವೆಂದ ಮೇಲೆ ಕೆಲವು ಅನನುಕೂಲತೆಗಳೂ ಲೋಪದೋಷಗಳೂ ಕಂಡುಬರುವುದು ಅವರಿಗೆ ಚೆನ್ನಾಗಿಯೇ ತಿಳಿದಿತ್ತು. ಆದ್ದರಿಂದ ಸಂಘ ನಿರ್ಮಾಣದೊಂದಿಗೆ ಅಮೆರಿಕದ ಯಾವ ಬಗೆಯ ಒಳ್ಳೆಯ ಅಂಶಗಳನ್ನು ಭಾರತಕ್ಕೆ ಒಯ್ದರೆ ಅದರಿಂದ ಹೆಚ್ಚು ಪ್ರಯೋಜನವಾದೀತು ಎಂಬುದನ್ನು ಸ್ವಾಮೀಜಿ ಅಧ್ಯಯನ ಮಾಡುತ್ತಿದ್ದರು.

ಆದರೆ ಶ್ರೀಮತಿ ಎಮಿಲಿ ಹಾಗೆ ಹಾಸ್ಯವಾಗಿ ಹೇಳಿದರೂ ಅವರಿಗೆ ಸ್ವಲ್ಪ ಆತಂಕವೂ ಇದ್ದಿದ್ದರೆ ಅದರಲ್ಲಿ ಅಚ್ಚರಿಯೇನಿಲ್ಲ. ಏಕೆಂದರೆ ಸ್ವಾಮೀಜಿಯ ಧೀರ-ಮನಮೋಹಕ ವ್ಯಕ್ತಿತ್ವವನ್ನು ಕಂಡು ಪ್ರತಿಯೊಬ್ಬರೂ ಆಕರ್ಷಿತರಾಗುತ್ತಿದ್ದರು. ನೂರಾರು ಜನ ಸ್ತ್ರೀಯರು ಅವರನ್ನು ಒಮ್ಮೆ ಯಾದರೂ ಮಾತನಾಡಿಸಲು ಉತ್ಸಾಹದಿಂದ ಮುಂದಾಗುತ್ತಿದ್ದರು. ಏನಾದರೂ ಮಾಡಿ ಅವರ ಗಮನವನ್ನು ತಮ್ಮತ್ತ ಸೆಳೆಯಲು ನಾನಾ ವಿಧವಾಗಿ ಪ್ರಯತ್ನಿಸುತ್ತಿದ್ದರು. ಆದರೆ ಸ್ವಾಮೀಜಿ ಪ್ರಚಂಡ ಬುದ್ಧಿಶಕ್ತಿಯಿಂದಲೂ ಆಧ್ಯಾತ್ಮಿಕ ಶಕ್ತಿಯಿಂದಲೂ ಕೂಡಿದ್ದವರಾದರೂ ಲೌಕಿಕ ವಿಚಾರಗಳಲ್ಲಿ ಅವರು ಬಹಳ ಹಿಂದೆ. ಅಲ್ಲದೆ ಅವರು ಯೌವನದ ಶಿಖರದಲ್ಲಿದ್ದರು. ಆದ್ದರಿಂದ ಇದನ್ನೆಲ್ಲ ಕಂಡು ಶ್ರೀಮತಿ ಎಮಿಲಿ ಲಿಯಾನರು ತೀವ್ರವಾಗಿ ಆತಂಕಗೊಳ್ಳುತ್ತಿದ್ದರು. ಈ ಲಲನೆ ಯರು ಎಲ್ಲಿ ಸ್ವಾಮೀಜಿಯನ್ನು ಮರುಳುಗೊಳಿಸಿಬಿಡುತ್ತಾರೋ, ಎಲ್ಲಿ ಸ್ವಾಮೀಜಿ ಅವರ ಬಲೆಗೆ ಸಿಕ್ಕಿಕೊಂಡುಬಿಡುತ್ತಾರೋ ಎಂದು ಚಡಪಡಿಸಿದರು. ಒಮ್ಮೆ ಸ್ವಾಮೀಜಿಗೆ ಆ ಬಗ್ಗೆ ಸೂಕ್ಷ್ಮವಾಗಿ ಎಚ್ಚರಿಕೆಯನ್ನೂ ಕೊಟ್ಟರು. ತಮ್ಮ ಬಗ್ಗೆ ಈಕೆ ಅಷ್ಟೊಂದು ಕಾಳಜಿ ವಹಿಸುವುದನ್ನು ಕಂಡು ಅವರಿಗೆ ಮೋಜೆನಿಸಿದರೂ ಇನ್ನೊಂದು ಕಡೆ ಸಂತೋಷವೂ ಆಯಿತು. ಸ್ವಾಮೀಜಿ ವಿಶ್ವಾಸದಿಂದ ಆಕೆಯ ಕೈಯನ್ನು ತಟ್ಟಿ ಹೇಳಿದರು, “ಪ್ರೀತಿಯ ತಾಯಿ, ನನ್ನ ವಿಷಯದಲ್ಲಿ ನೀವೇನೂ ಹೆದರ ಬೇಡಿ. ಭಾರತದಲ್ಲಿ ನಾನು ಬಡ ರೈತನೊಬ್ಬ ಕೊಟ್ಟ ರೊಟ್ಟಿಯನ್ನು ತಿಂದು ಆಲದ ಮರದ ಕೆಳಗೆ ಮಲಗುತ್ತಿದ್ದುದು ನಿಜವೇ. ಆದರೆ ರಾಜಮಹಾರಾಜರ ಅರಮನೆಗಳಲ್ಲಿ ಅವರ ಅತಿಥಿ ಯಾಗಿದ್ದುಕೊಂಡು, ಅವರು ನೇಮಿಸಿದ ಪರಿಚಾರಿಕೆಯ ಕೈಯಿಂದ ರಾತ್ರಿಯಿಡೀ ಬೀಸಣಿಗೆ ಯಿಂದ ಗಾಳಿ ಹಾಕಿಸಿಕೊಳ್ಳುತ್ತಿದ್ದುದೂ ಅಷ್ಟೇ ಸತ್ಯ. ಈ ಬಗೆಯ ಆಕರ್ಷಣೆಗಳೆಲ್ಲ ನನಗೆ ರೂಢಿಯಾಗಿವೆ. ನೀವೇನೂ ಹೆದರಬೇಡಿ.”

ಸ್ವಾಮೀಜಿ ಹೀಗೆ ಹೇಳಿದರಾದರೂ, ಅಮೆರಿಕದ ಎಷ್ಟೋ ಜನರಿಗೆ ಸಂನ್ಯಾಸ ಧರ್ಮದ ವಿಷಯ ಅರ್ಥವಾಗುತ್ತಿರಲಿಲ್ಲ. ತಮ್ಮ ಪಾದ್ರಿಗಳಂತೆಯೇ ಇವರೂ ‘ವೃತ್ತಿನಿರತ’ ಪ್ರಚಾರಕ ರೆಂದು ಅವರು ಭಾವಿಸಿದ್ದರು. ಎಷ್ಟೋ ಜನ ಕುತೂಹಲಕ್ಕಾಗಿ ಅವರನ್ನು “ನಿಮಗೆ ಮದುವೆ ಯಾಗಿದೆಯೇ?” ಎಂದು ಪ್ರಶ್ನಿಸುತ್ತಿದ್ದರು. ಅವರ ಸುಂದರ-ಪ್ರತಿಭಾನ್ವಿತ ವ್ಯಕ್ತಿತ್ವವನ್ನು ಕಂಡು ಮೋಹಿತರಾಗಿ ಕೆಲವು ಮಹಿಳೆಯರು ಅವರನ್ನು ಮದುವೆಯಾಗಲೂ ಮುಂದಾಗಿದ್ದರು. ಒಮ್ಮೆ ಅತ್ಯಂತ ಶ್ರೀಮಂತ ಮಹಿಳೆಯೊಬ್ಬಳು, “ಸ್ವಾಮೀಜಿ, ನಾನು ನನ್ನ ಸಕಲ ಸಂಪತ್ತಿನೊಂದಿಗೆ ನನ್ನನ್ನು ನಿಮ್ಮ ಪಾದಗಳಿಗೆ ಸರ್ಪಿಸಿಕೊಳ್ಳಲು ಸಿದ್ಧಳಿದ್ದೇನೆ” ಎಂದು ಮುಂದಾದಳು. ಆಗ ಸ್ವಾಮೀಜಿ, “ತಾಯಿ, ನಾನೊಬ್ಬ ಸಂನ್ಯಾಸಿ, ಸಂನ್ಯಾಸಿಗೆ ಮದುವೆಯೆಂದರೇನು! ನನ್ನ ಪಾಲಿಗೆ ಸಕಲ ಸ್ತ್ರೀಯರೂ ನನ್ನ ಹೆತ್ತ ತಾಯಿಗೆ ಸಮ” ಎಂದುತ್ತರಿಸಿದರು.

ಕಾರ್ನೆಲಿಯಾ ಕಾಂಗರಳ ಚಿಕ್ಕಮ್ಮ ಹಾಗೂ ಚಿಕ್ಕಪ್ಪ–ಇಬ್ಬರೂ ಸುಶಿಕ್ಷಿತರು, ಬುದ್ಧಿವಂತರು. ಇವರಿಗೆ ಸ್ವಾಮೀಜಿಯ ಪ್ರತಿಭೆಯ ಬಗ್ಗೆ ತುಂಬ ಮೆಚ್ಚುಗೆಯಿತ್ತು. ಒಮ್ಮೆ ಅವರು ತಮ್ಮ ಸ್ನೇಹಿತರಾದ ಕೆಲವು ಪ್ರಾಧ್ಯಾಪಕರು ಹಾಗೂ ಪತ್ರಕರ್ತರ ಮುಂದೆ ಸ್ವಾಮೀಜಿಯನ್ನು ಕೊಂಡಾಡಿ ದರು. ಆಗ ಆ ಸ್ನೇಹಿತರು ನಕ್ಕು, “ಈಗಿನ ಕಾಲದ ವಿಜ್ಞಾನಿಗಳೂ ಮನಶ್ಶಾಸ್ತ್ರಜ್ಞರೂ ಆ ಸ್ವಾಮಿಯ ಧಾರ್ಮಿಕ ನಂಬಿಕೆಗಳನ್ನೆಲ್ಲ ನಿಮಿಷದಲ್ಲಿ ಬುಡಮೇಲು ಮಾಡಬಲ್ಲರು” ಎಂದು ಪರಿಹಾಸ ಮಾಡಿದರು. ಆಗ ಕಾರ್ನೆಲಿಯಾಳ ಚಿಕ್ಕಮ್ಮ ಹೇಳಿದರು, “ನಾನು ಮುಂದಿನ ಭಾನುವಾರ ಇಲ್ಲಿಗೆ ಬರುವಂತೆ ಸ್ವಾಮೀಜಿಯನ್ನು ಒಪ್ಪಿಸುತ್ತೇನೆ. ನೀವೆಲ್ಲ ಆ ದಿನ ಬಂದು ಅವರನ್ನು ಮಾತನಾಡಿಸು ತ್ತೀರಾ?” ಆ ಬುದ್ಧಿವಂತ ಸ್ನೇಹಿತರು ಆ ಯೋಜನೆಗೆ ಒಪ್ಪಿದರು. ಅದರಂತೆ ಮುಂದಿನ ಭಾನು ವಾರ ಎಲ್ಲರನ್ನೂ ಊಟಕ್ಕೆ ಆಹ್ವಾನಿಸಲಾಯಿತು. ಹಲವಾರು ವಿಷಯಗಳು ಚರ್ಚೆಗೆ ಬಂದುವು. ತುಂಬ ಕುತೂಹಲಕರವಾದ ಚರ್ಚೆಯೇರ್ಪಟ್ಟಿತು. ಚರ್ಚೆಯಲ್ಲಿ ಭಾಗವಹಿಸದಿದ್ದವರೂ ಆಸಕ್ತಿ ಯಿಂದ ನೋಡುತ್ತ ಕುಳಿತರು. ಕುಬೇರನ ಭಂಡಾರವನ್ನು ಮೀರಿಸುವ ಸ್ವಾಮೀಜಿಯವರ ಜ್ಞಾನ ಭಂಡಾರವನ್ನು ಕಂಡು ಪ್ರಾಧ್ಯಾಪಕರೂ ಪತ್ರಕರ್ತರೂ ಬೆರಗಾದರು. ಜೊತೆಗೆ ಅವರ ಹಾಸ್ಯ ಪ್ರಜ್ಞೆಯೂ ಪಾದರಸದಂತಹ ಚುರುಕುತನವೂ ಸೇರಿಕೊಂಡು ಸಂಭಾಷಣೆ ಅತ್ಯಂತ ಆಹ್ಲಾದಕರ ವಾಗಿತ್ತು. ಹಲವಾರು ಪೌರ್ವಾತ್ಯ ಧರ್ಮಗಳ ಜ್ಞಾನವಲ್ಲದೆ, ಬೈಬಲ್, ಕೊರಾನ್​ಗಳ ಸಂಪೂರ್ಣ ತಿಳಿವಳಿಕೆಯೂ ಆಧುನಿಕ ವಿಜ್ಞಾನ-ಮನಶ್ಶಾಸ್ತ್ರಗಳ ಮೇಲಿನ ಅವರ ಹತೋಟಿಯೂ ಅತ್ಯಾಶ್ಚರ್ಯ ಕರವಾಗಿತ್ತು. ಕಡೆಗೆ ಈ ಸಂಶಯಾತ್ಮರ ಸಂಶಯಗಳೆಲ್ಲ ಪರಿಹಾರವಾದುವು. ಸ್ವಾಮೀಜಿಯ ಬಗ್ಗೆ ಹೃತ್ಪೂರ್ವಕ ಮೆಚ್ಚುಗೆಯನ್ನೂ ಆದರವನ್ನೂ ಹೊತ್ತು ಅವರು ಬೀಳ್ಕೊಂಡರು.

ಕಾರ್ನೆಲಿಯಾಳ ತಾಯಿ ತಾರುಣ್ಯದಲ್ಲಿಯೇ ಪತಿಯನ್ನು ಕಳೆದುಕೊಂಡಿದ್ದಳು. ಆಕೆಯ ದುಃಖ ವನ್ನು ಗಮನಿಸಿ ಸ್ವಾಮೀಜಿ ಆಕೆಯನ್ನು ಸಂತೈಸಿದರು. ಅದರಿಂದ ಆಕೆ ಬಹುಮಟ್ಟಿಗೆ ಸಮಾಧಾನ ಗೊಂಡಳು. ಆಕೆ ಸ್ವಾಮೀಜಿಯ ಉಪನ್ಯಾಸಗಳನ್ನು ಕೇಳಿದಳಾದರೂ ಅದರಲ್ಲಿ ಎಷ್ಟೋ ವಿಚಾರ ಗಳು ಆಕೆಗೆ ಅರ್ಥವಾಗಲಿಲ್ಲ. ಆದ್ದರಿಂದ ಅವರು ತಮ್ಮ ಮನೆಯಿಂದ ಹೊರಟುಹೋದ ಮೇಲೆ ಅವರ ಪುಸ್ತಕಗಳನ್ನೂ ಇತರ ಹಿಂದೂ ಗ್ರಂಥಗಳನ್ನೂ ಅಧ್ಯಯನ ಮಾಡಿದಳು. ಈ ಸಂದರ್ಭ ದಲ್ಲಿ ಅವಳು ಒಂದು ವೈಚಿತ್ರ್ಯವನ್ನು ಗಮನಿಸಿದಳು. ಯಾರಾದರೂ ಬರೆದ ಪತ್ರವನ್ನು ಹರಿದು, ಚೂರುಗಳನ್ನು ತನ್ನ ಕೈಯಲ್ಲಿಟ್ಟುಕೊಂಡರೆ, ಒಂದು ಕ್ಷಣ ಆಕೆಗೆ ಆ ಪತ್ರವನ್ನು ಬರೆದವರ ವ್ಯಕ್ತಿತ್ವದ ಸ್ಪಷ್ಟ ಅರಿವುಂಟಾಗುತ್ತಿತ್ತು. ಸುಮಾರು ಒಂದು ವರ್ಷದ ಬಳಿಕ ಸ್ವಾಮೀಜಿ ಶಿಕಾಗೋಗೆ ಮರಳಿದಾಗ ಆಕೆ ತನ್ನ ಅನುಭವವನ್ನು ಅವರ ಮುಂದೆ ಹೇಳಿಕೊಂಡಳು. ಆಗ ಸ್ವಾಮೀಜಿ ಹೇಳಿದರು, “ಹಿಂದೆ ನನಗೂ ಇಂಥದೇ ಅನುಭವಾಗುತ್ತಿತ್ತು. ನಾನು ಇದನ್ನು ಪ್ರದರ್ಶಿಸುತ್ತ ತಮಾಷೆ ಮಾಡುತ್ತಿದ್ದೆ. ಇದು ಗೊತ್ತಾದಾಗ ಶ್ರೀರಾಮಕೃಷ್ಣರು ನನ್ನ ಮಂಡಿಯ ಮೇಲೊಂದು ಪೆಟ್ಟುಕೊಟ್ಟು ಹೇಳಿದರು–‘ನೋಡಿಲ್ಲಿ! ನಿನ್ನ ಈ ಶಕ್ತಿಯನ್ನು ಮಾನವತೆಯ ಒಳಿತಿಗಾಗಲ್ಲದೆ ಬೇರಾವುದಕ್ಕೂ ಬಳಸಬೇಡ. ಈ ಅನುಭವಗಳನ್ನು ಪಡೆಯಬಲ್ಲ ಕೈಗಳು ಜನರ ಸಂಕಟಗಳನ್ನೂ ನಿವಾರಿಸಬಲ್ಲುವು. ಅಂತಹ ದುಃಖನಿವಾರಣೆಗಾಗಿ ಈ ಶಕ್ತಿಯನ್ನು ಉಪ ಯೋಗಿಸು!’ ಎಂದು.”

ಶ್ರೀಮತಿ ಎಮಿಲಿ ಲಿಯಾನರು ಶಿಕಾಗೋದ ಮಹಿಳೆಯರ ಆಸ್ಪತ್ರೆಯೊಂದರ ಅಧ್ಯಕ್ಷಣಿ ಯಾಗಿದ್ದರು. ಒಂದು ದಿನ ಸ್ವಾಮೀಜಿ ಆ ಆಸ್ಪತ್ರೆಯನ್ನು ಸಂದರ್ಶಿಸಿದರು. ಆಸ್ಪತ್ರೆಯ ಬಗ್ಗೆ ಅವರೆಷ್ಟು ಆಸಕ್ತಿ ತಾಳಿದ್ದರೆಂದರೆ, ಮಕ್ಕಳ ಮರಣದ ಪ್ರಮಾಣ ಮೊದಲಾದ ಹಲವಾರು ವಿಷಯ ಗಳ ಅಂಕಿ ಅಂಶಗಳನ್ನು ಕೇಳಿ ತಿಳಿದುಕೊಂಡರು. ವೈದ್ಯರು, ದಾದಿಯರು, ರೋಗಿಗಳನ್ನು ಮಾತ್ರ ವಲ್ಲದೆ ಅಡಿಗೆಯವರು-ಧೋಬಿಗಳವರೆಗೆ ಪ್ರತಿಯೊಬ್ಬರನ್ನೂ ಮಾತನಾಡಿಸಿ ನೂರಾರು ಪ್ರಶ್ನೆ ಗಳನ್ನು ಕೇಳಿ ಉತ್ತರ ಪಡೆದರು.

ಇದಲ್ಲದೆ ಅವರು ಶಿಕಾಗೋದಲ್ಲಿ ಪ್ರದರ್ಶನಾಲಯಗಳು, ವಿಶ್ವವಿದ್ಯಾಲಯಗಳು, ಶಾಲೆಗಳು, ಕಲಾಮಂದಿರಗಳು ಹಾಗೂ ಇತರ ಹಲವಾರು ಸಾರ್ವಜನಿಕ ಸ್ಥಳಗಳನ್ನು ಸಂದರ್ಶಿಸಿದರು. ಯಾವುದೋ ಕಲಾಕೃತಿಯನ್ನೋ ತಂತ್ರಜ್ಞಾನದ ವಿಸ್ಮಯವನ್ನೋ ದೃಷ್ಟಿಸುತ್ತಿದ್ದಂತೆ, ಮಾನವನ ಬುದ್ಧಿಶಕ್ತಿಯ ವೈಭವವನ್ನು ಕಂಡು ಅವರ ಮನಸ್ಸು ವಿಶೇಷ ಉತ್ಸಾಹ-ಮೆಚ್ಚುಗೆಯಿಂದ ತುಂಬಿಬರುತ್ತಿತ್ತು. ಅಮೆರಿಕದ ಸಾರ್ವಜನಿಕ ಹಾಗೂ ಸಾಮಾಜಿಕ ಜೀವನವನ್ನು ಅವರು ನಿರಂತರವಾಗಿ ಅಧ್ಯಯನ ಮಾಡುತ್ತಿದ್ದರು. ಕೈಗಾರಿಕೆಗಳ ಬೃಹತ್ ಕಟ್ಟಡಗಳನ್ನು ಕಂಡು ವಿಸ್ಮಯಮೂಕರಾಗುತ್ತಿದ್ದರು. ಅಮೆರಿಕದ ಆರ್ಥಿಕ ಪರಿಸ್ಥಿತಿಯನ್ನು, ತಾಂತ್ರಿಕ ಮುನ್ನಡೆಯನ್ನು, ಯಂತ್ರೋದ್ಯಮ ಕ್ಷೇತ್ರದಲ್ಲಾಗುತ್ತಿದ್ದ ಕ್ರಾಂತಿಯನ್ನು ಗಮನಿಸುತ್ತಿದ್ದರು. ಜೊತೆಗೇ ಅವರಿಗೆ ಭಾರತದ ಸ್ಥಿತಿಗತಿಗಳ ನೆನಪಾಗುತ್ತಿತ್ತು. ಅಮೆರಿಕದೊಂದಿಗೆ ಭಾರತವನ್ನು ಹೋಲಿಸಿನೋಡುತ್ತ ಈ ಎರಡು ದೇಶಗಳ ಒಳ್ಳೆಯ ಅಂಶಗಳು ಯಾವುವು, ಪರಸ್ಪರ ಸಾಮ್ಯ-ಭೇದಗಳು ಎಂಥವು ಎಂಬುದನ್ನು ವಿವೇಚಿಸುತ್ತಿದ್ದರು.

ಹೀಗೆ ಅವರ ಜಾಗೃತ ಮನಸ್ಸು ಸುತ್ತಲಿನ ಅದ್ಭುತಗಳನ್ನು ಗಮನಿಸುತ್ತ ಅದರ ಬಗೆಗೇ ಚಿಂತಿಸುತ್ತಿದ್ದರೂ, ಅವರ ಸುಪ್ತಮನಸ್ಸು ತನ್ನ ಸಹಜ ನೆಲೆಯಾದ ಧ್ಯಾನ ಸಮಾಧಿಗೇರಲು ತವಕಿಸುತ್ತಿತ್ತು. ಇತರರ ದೃಷ್ಟಿಗೆ ಅವರು ಯಾವುದೇ ಬಗೆಯ ತಪಸ್ಸನ್ನಾಗಲಿ ಸಾಧನೆಗಳನ್ನಾಗಲಿ ಮಾಡುವಂತೆ ಕಾಣುತ್ತಿರಲಿಲ್ಲ. ಆದರೆ ಸೋದರಿ ನಿವೇದಿತೆ ಹೇಳುವಂತೆ, ಅವರ ಮನಸ್ಸು ಅನುಕ್ಷಣವೂ ಎಂತಹ ತೀವ್ರ ಏಕಾಗ್ರತೆಯ ಸ್ಥಿತಿಯಲ್ಲಿತ್ತೆಂದರೆ ಬೇರೆ ಯಾರಿಗೇ ಆದರೂ ಅದೊಂದು ಉಗ್ರತಮ ತಪಸ್ಸಾಗುತ್ತಿತ್ತು. ಧ್ಯಾನಸ್ಥಿತಿಗೆ ತಾನೇತಾನಾಗಿ ಏರಿಹೋಗಲು ಚಡಪಡಿ ಸುತ್ತಿದ್ದ ತಮ್ಮ ಮನಸ್ಸನ್ನು ಕೆಳಕ್ಕೆಳೆದು ಹಿಡಿದಿಡುವುದೇ ಅವರ ಪಾಲಿಗೆ ಕಷ್ಟವಾಗಿ ಪರಿಣಿಸಿತ್ತು. ರೈಲು-ಟ್ರಾಮುಗಳಿಂದ ಮತ್ತು ಅತಿಯಾದ ಶಿಷ್ಟಾಚಾರ ಹಾಗೂ ಸಮಯಪ್ರಜ್ಞೆಯ ಜನರಿಂದ ಕೂಡಿದ ಅಮೆರಿಕೆಗೆ ಬಂದಾಗ, ಆ ಜಟಿಲ ವ್ಯವಸ್ಥೆಗೆ ಹೊಂದಿಕೊಳ್ಳಲು ಬಹಳವೇ ಕಷ್ಟವಾಯಿತು. ಎಷ್ಟೋ ಬಾರಿ ಅವರು ಅನ್ಯಮನಸ್ಕರಾಗಿ ಟ್ರಾಮಿನಲ್ಲಿ ಕುಳಿತು, ಮತ್ತೆ ಮತ್ತೆ ಅದೇ ದಾರಿಯಾಗಿ ಎರಡು-ಮೂರು ಸಲ ಸುತ್ತುಹೊಡೆದದ್ದೂ ಉಂಟು. ಆಗಾಗ ಟ್ರಾಮಿನ ಕಂಡಕ್ಟರನಿಂದ ಮಾತ್ರ ಸ್ವಲ್ಪ ಮಟ್ಟಿಗೆ ಅವರ ಗಮನ ವಿಚಲಿತವಾಗುತ್ತಿತ್ತು. ಇಂತಹ ಪ್ರಸಂಗಗಳ ಬಗ್ಗೆ ಅವರು ತುಂಬ ನಾಚಿಕೊಳ್ಳುತ್ತಿದ್ದರು. ಅಂಥದು ಮತ್ತೆ ಮರುಕುಳಿಸದಂತೆ ಎಚ್ಚರ ವಹಿಸುತ್ತಿದ್ದರು. 

ಸರ್ವಧರ್ಮ ಸಮ್ಮೇಳನದ ಪರಿಣಾಮವಾಗಿ ಸ್ವಾಮೀಜಿಗೆ ಆದ ಲಾಭಗಳಲ್ಲಿ ಒಂದೆಂದರೆ ಜಗತ್ತಿನ ಅನೇಕ ಶ್ರೇಷ್ಠ ವಿದ್ವಾಂಸರು ಹಾಗೂ ಚಿಂತನಶೀಲರೊಂದಿಗೆ ಆದ ಪರಸ್ಪರ ಪರಿಚಯ. ಇದೇ ಸಮಯದಲ್ಲಿ ಅವರು ಅಮೆರಿಕದ ಅತ್ಯಂತ ಪ್ರಸಿದ್ಧ ನಾಸ್ತಿಕ್ಯವಾದಿಯೂ ವಾಗ್ಮಿಯೂ ಆದ ರಾಬರ್ಟ್ ಇಂಗರ್​ಸಾಲ್​ನನ್ನು ಭೇಟಿಯಾದರು. ಇವನೊಂದಿಗೆ ಸ್ವಾಮೀಜಿ ಅನೇಕ ಸಲ ಧಾರ್ಮಿಕ ಹಾಗೂ ತಾತ್ತ್ವಿಕ ವಿಷಯಗಳ ಬಗ್ಗೆ ಅಭಿಪ್ರಾಯವಿನಿಮಯ ಮಾಡಿಕೊಂಡರು. ಸಂಪ್ರದಾಯಸ್ಥ ಕ್ರೈಸ್ತರ ಪಾಲಿಗೆ ಇಂಗರ್​ಸಾಲ್ ಒಬ್ಬ ವೈರಿಯಾಗಿ ಪರಿಣಮಿಸಿದ್ದ. ಆದರೆ ಇವರಿಬ್ಬರೂ ಪರಸ್ಪರರನ್ನು ಮೆಚ್ಚಿಕೊಂಡರು. ಏಕೆಂದರೆ ಇಂಗರ್​ಸಾಲ್ ನಾಸ್ತಿಕನಾದರೂ ಪ್ರಾಮಾಣಿಕನೆಂಬುದನ್ನು ಸ್ವಾಮೀಜಿ ಗಮನಿಸಿದರು. ಹಾಗೆಯೇ ತನ್ನಂತೆಯೇ ಸ್ವಾಮೀಜಿಯೂ ಒಬ್ಬ ಕ್ರಾಂತಿಕಾರಿ ಎಂಬುದನ್ನು ಇಂಗರ್​ಸಾಲ್ ಗುರುತಿಸಿದ. ಆದರೆ ತಮ್ಮ ಧಾರ್ಮಿಕ ತತ್ತ್ವ ಗಳನ್ನು ಸ್ವಾಮೀಜಿ ನಿಶ್ಶಂಕೆಯಿಂದ ಸಾರುವ ಪರಿಯನ್ನು ಕಂಡು ಇಂಗರ್​ಸಾಲ್ ಅವರಿಗೆ ಬುದ್ಧಿ ಮಾತೊಂದನ್ನು ಹೇಳಿದ–“ಸ್ವಾಮೀಜಿ, ನೀವು ಇಷ್ಟೊಂದು ಧೈರ್ಯವಾಗಿ ಮುಚ್ಚುಮರೆ ಇಲ್ಲದೆ ಮಾತನಾಡುವುದು ಒಳ್ಳೆಯದಲ್ಲ. ನಿಮ್ಮ ಹೊಸ ಸಿದ್ಧಾಂತಗಳನ್ನು ಇಲ್ಲಿ ಬೋಧಿಸುವಾಗಲೂ ಜನರ ರೀತಿನೀತಿಗಳನ್ನು ಟೀಕಿಸುವಾಗಲೂ ನೀವು ಎಚ್ಚರದಿಂದಿರಬೇಕು.” ಸ್ವಾಮೀಜಿ ಅಚ್ಚರಿ ಯಿಂದ “ಅದೇಕೆ?” ಎಂದು ಕೇಳಿದಾಗ ಅವನು ಹೇಳಿದ, “ನೀವೇನಾದರೂ ಐವತ್ತು ವರ್ಷಗಳ ಹಿಂದೆ ಇಲ್ಲಿಗೆ ಬಂದು ಈ ರೀತಿ ಬೋಧಿಸಿದ್ದರೆ ನಿಮ್ಮನ್ನು ಜನ ಜೀವಸಹಿತ ಬಿಡುತ್ತಿರಲಿಲ್ಲ. ಈಗ ಕೆಲವೇ ವರ್ಷಗಳ ಹಿಂದೆ ಬಂದಿದ್ದರೂ ಜನರು ನಿಮ್ಮನ್ನು ಕಲ್ಲುಗಳಿಂದ ಹೊಡೆದು ಓಡಿಸು ತ್ತಿದ್ದರು. ಈಗಲಾದರೂ ಅಂತಹ ಒಳ್ಳೆಯ ಪರಿಸ್ಥಿತಿಯೇನೂ ಇಲ್ಲ” ಎಂದು. ಆದರೆ ಸ್ವಾಮೀಜಿ ಇದನ್ನು ನಂಬಲಿಲ್ಲ. ಅಮೆರಿಕದಲ್ಲಿ ಅಷ್ಟೊಂದು ಮತಾಂಧತೆ ಪ್ರಚಲಿತವಾಗಿದೆಯೆಂದರೆ ತಾವು ಒಪ್ಪುವುದಿಲ್ಲವೆಂದೇ ಹೇಳಿದರು.

ಇವರಿಬ್ಬರು ನಡುವಿನ ಮಾತುಕತೆ ಸೌಹಾರ್ದದಿಂದ ಕೂಡಿತ್ತಾದರೂ ಇಬ್ಬರಲ್ಲಿ ಯಾವುದೇ ವಿಷಯದ ಬಗ್ಗೆ ಸಮಾನಾಭಿಪ್ರಾಯವಿರಲಿಲ್ಲ. ಮುಖ್ಯವಾಗಿ ಇಬ್ಬರ ಜೀವನದೃಷ್ಟಿಗಳೇ ತದ್ವಿರುದ್ಧವಾದವು. ಜೀವನದ ಬಗ್ಗೆ ತನ್ನ ದೃಷ್ಟಿಕೋನವನ್ನು ಇಂಗರ್​ಸಾಲ್ ಹೀಗೆ ವರ್ಣಿಸಿದ– “ಈ ಪ್ರಪಂಚವೆಂಬುದು ಒಂದು ಕಿತ್ತಳೆ ಹಣ್ಣು. ಇದೊಂದೇ ನಮ್ಮ ಕಣ್ಣೆದುರಿಗಿರುವುದು. ಇದೊಂದೇ ಸತ್ಯ. ಆದ್ದರಿಂದ ಬೇರೆ ಯಾವುದರ ಬಗೆಗೂ ತಲೆ ಕೆಡಿಸಿಕೊಳ್ಳದೆ, ಇದರಿಂದ ರಸ ತೆಗೆದು ಕುಡಿಯುವುದಷ್ಟೇ ನನ್ನ ಉದ್ದೇಶ. ಆದ್ದರಿಂದ ನಾನು ಇದನ್ನು ಚೆನ್ನಾಗಿ ಹಿಂಡಿ, ಹೆಚ್ಚಿನ ರಸವನ್ನು ಪಡೆಯುತ್ತೇನೆ.” ಆಗ ಸ್ವಾಮೀಜಿ ಹೇಳಿದರು, “ಕಿತ್ತಳೆ ಹಣ್ಣಿನಿಂದ ಅತಿ ಹೆಚ್ಚಿನ ರಸ ವನ್ನು ಹಿಂಡಬೇಕೆಂಬುದನ್ನು ನಾನೂ ಒಪ್ಪುತ್ತೇನೆ. ಆದರೆ ನಾನು ನಿಮಗಿಂತ ಹೆಚ್ಚಿನ ರಸವನ್ನು ತೆಗೆಯಬಲ್ಲೆ! ಹೇಗೆ ಗೊತ್ತೇನು? ನನಗೆ ಗೊತ್ತಿದೆ–ನಾನು ಎಂದಿಗೂ ಸಾಯವುದೇ ಇಲ್ಲ. ನಾನು ಅಮರ ಎಂದು. ಆದ್ದರಿಂದ ನನಗೆ ಅವಸರವಿಲ್ಲ. ನನಗೆ ಬೇರಾವುದೇ ಕರ್ತವ್ಯ ಎಂಬು ದಿಲ್ಲ. ನನಗೆ ಹೆಂಡತಿ ಮಕ್ಕಳು ಆಸ್ತಿಪಾಸ್ತಿಗಳೆಂಬ ಬಂಧನಗಳಿಲ್ಲ. ಆದ್ದರಿಂದ ನಾನು ಪ್ರತಿ ಯೊಬ್ಬ ಸ್ತ್ರೀ-ಪುರುಷನನ್ನೂ ಸಮಾನವಾಗಿ ಪ್ರೀತಿಸಬಲ್ಲೆ. ಪ್ರತಿಯೊಬ್ಬರೂ ನನಗೆ ದೇವರೇ. ಮನುಷ್ಯನನ್ನು ಭಗವಂತನನ್ನಾಗಿ ಭಾವಿಸಿದಾಗ ಉಂಟಾಗುವ ಆನಂದವನ್ನು ಭಾವಿಸಿನೋಡಿ! ನೀವು ನಿಮ್ಮ ಕಿತ್ತಳೆ ಹಣ್ಣನ್ನು ಈ ರೀತಿ ಹಿಂಡಬಲ್ಲಿರಾದರೆ ಹತ್ತುಸಾವಿರ ಪಟ್ಟು ಹೆಚ್ಚಿನ ರಸವನ್ನು ತೆಗೆಯಬಲ್ಲಿರಿ! ಪ್ರತಿಯೊಂದು ತುಣುಕು ರಸವನ್ನೂ ಹೀರಬಲ್ಲಿರಿ!”

ಶಿಕಾಗೋದಲ್ಲಿ ಸ್ವಾಮೀಜಿ ಹಲವಾರು ಶಿಷ್ಯರ, ಸ್ನೇಹಿತರ ಅತಿಥಿಗಳಾಗಿದ್ದರು. ಸಮಾಜದ ಅತ್ಯಂತ ಗಣ್ಯರ ಹಾಗೂ ಶ್ರೀಮಂತರ ಮನೆಗಳಲ್ಲಿ ಅವರಿಗೆ ಯಾವಾಗಲೂ ಹೃತ್ಪೂರ್ವಕ ಸ್ವಾಗತ ವಿತ್ತು. ಸ್ವಾಮೀಜಿ ಯಾರ್ಯಾರ ಮನೆಗಳಲ್ಲಿ ಇಳಿದುಕೊಂಡಿದ್ದರೋ ಅವರೆಲ್ಲರೂ ತಮ್ಮನ್ನು ಅತ್ಯಂತ ಭಾಗ್ಯಶಾಲಿಗಳೆಂದು ತಿಳಿದು, ಜೀವನದುದ್ದಕ್ಕೂ ಆ ದಿವ್ಯಕ್ಷಣಗಳನ್ನು ಸ್ಮರಿಸುತ್ತಿದ್ದರು. ಯಾರಿಗೂ ಅವರನ್ನು ಪ್ರೀತಿಸದಿರಲು ಸಾಧ್ಯವೇ ಇರಲಿಲ್ಲ. ಸ್ವಾಮೀಜಿಯ ಪ್ರಚಂಡ ಬುದ್ಧಿಶಕ್ತಿ, ವಾಕ್ಪಟುತ್ವಗಳು ಅವರ ವ್ಯಕ್ತಿತ್ವದ ಒಂದಂಶವಷ್ಟೆ. ಆದರೆ ಅವರ ಶಿಶುಸಹಜ ಆಕೃತ್ರಿಮ ಮಾತುಕತೆ, ಅತ್ಯಂತ ಸಭ್ಯ-ವಿನಯಪೂರ್ಣ ನಡವಳಿಕೆ ಹಾಗೂ ಸದಾ ನಗುನಗುತ್ತ ಇತರರನ್ನೂ ನಗಿಸುವ ಸ್ವಭಾವ–ಇವು ಪ್ರತಿಯೊಬ್ಬರನ್ನೂ ಅವರೆಡೆಗೆ ಸೆಳೆಯುತ್ತಿದ್ದುವು. ಪ್ರತಿಯೊಬ್ಬರ ಪಾಲಿಗೂ ಅವರು ಮಗ, ಸ್ನೇಹಿತ, ಸೋದರ, ಇಲ್ಲವೆ ತಂದೆ. ಇದೊಂದು ಅದ್ಭುತವೇ ಸರಿ. ಏಕೆಂದರೆ ಆಗ ಗುಲಾಮ ರಾಷ್ಟ್ರವಾಗಿದ್ದ ಭಾರತದ ಬಗೆಗೂ ಭಾರತೀಯರ ಬಗೆಗೂ ಪಾಶ್ಚಾತ್ಯ ದೇಶಗಳಲ್ಲಿ ತಿರಸ್ಕಾರದ ಭಾವನೆಯಿತ್ತು. ಅಲ್ಲದೆ ಭಾರತೀಯರು ಅನಾಗರಿಕರೆಂದೂ ಮೂಢ ನಂಬಿಕೆಗಳ ಆಗರವೆಂದೂ ಆ ಜನರು ದೃಢವಾಗಿ ನಂಬಿದ್ದರು. ಹೀಗಿರುವಾಗ ಅಂತಹ ದೇಶದವ ನಾದ ಒಬ್ಬ ‘ವಿಧರ್ಮಿ’ಯನ್ನು ಜನರು ಅಷ್ಟೊಂದು ಪ್ರೀತಿಸಲು, ಅಷ್ಟೊಂದು ಸಲಿಗೆಯಿಂದ ಕಾಣಲು ಹೇಗೆ ಸಾಧ್ಯವಾಯಿತು ಎಂಬುದೊಂದು ಅಚ್ಚರಿಯ ಅಂಶವೇ ಸರಿ. ಅಲ್ಲದೆ ಸ್ವಾಮೀಜಿ ಎಷ್ಟೇ ವಿನಯವಂತರಾಗಿದ್ದರೂ ಆತ್ಮಾಭಿಮಾನವನ್ನೂ ರಾಷ್ಟ್ರಾಭಿಮಾನವನ್ನೂ ಎಂದಿಗೂ ಸಡಿಲಿಸಲು ಸಿದ್ಧರಿರಲಿಲ್ಲ. ಆದ್ದರಿಂದ ಯಾವುದೇ ಬಗೆಯ ಅಹಂಕಾರವನ್ನು ಹೊತ್ತು ಅವರಲ್ಲಿಗೆ ಹೋದರೆ ಸ್ವಾಮೀಜಿ ಕ್ಷಣಾರ್ಧದಲ್ಲಿ ಆ ಅಹಂಕಾರವನ್ನು ನುಚ್ಚುನೂರು ಮಾಡಿಬಿಡುತ್ತಿದ್ದರು. ಇದರಿಂದಾಗಿ ಅಂಥವರಿಗೆ ಅವರು ತೀರ ಪ್ರತಿಷ್ಠೆಯವರಂತೆ ಕಾಣುತ್ತಿದ್ದರು. ಹೀಗಿದ್ದರೂ ಅವರ ಸದ್ಗುಣಗಳನ್ನು ಮೆಚ್ಚಿಕೊಳ್ಳದಿರಲು ಎಂಥವರಿಗೂ ಸಾಧ್ಯವಿರಲಿಲ್ಲ.

ಆ ಕಾಲದ ಅತಿ ಪ್ರಸಿದ್ಧ ಅಪೇರಾ (ಸಂಗೀತಪ್ರಧಾನ ನಾಟಕ) ಗಾಯಕಿಯಾಗಿದ್ದ ಮೇಡಂ ಎಮ್ಮಾ ಕಾಲ್ವೆಯ ಮೇಲೆ ಸ್ವಾಮೀಜಿ ಉಂಟುಮಾಡಿದ ಪ್ರಭಾವ ಪ್ರಚಂಡವಾದದ್ದು. ಆಕೆ ಯೂರೋಪು ಅಮೆರಿಕಗಳಲ್ಲಿ ಲಕ್ಷಾಂತರ ಜನರ ಕಣ್ಮಣಿಯಾಗಿದ್ದಳು. ಅಧಿಕಾರ ಐಶ್ವರ್ಯ ಕೀರ್ತಿಗಳು ಅವಳಿಗೆ ಕಾಲಕಸವಾಗಿತ್ತು. ಆದರೆ ಇಷ್ಟೆಲ್ಲ ಇದ್ದರೂ ಅವಳಿಗೆ ಜೀವನದಲ್ಲಿ ಸುಖ ವಿರಲಿಲ್ಲ. ಏಕೆಂದರೆ ಅವಳ ಮನಸ್ಸೇ ಅಂಥದು–ಅಂತಹ ಅಸಹನೆ, ಉದ್ವಿಗ್ನತೆ, ಆಕೆ ಹೀಗೆಯೇ ಉಳಿದುಕೊಳ್ಳುತ್ತಿದ್ದಳೋ ಏನೋ. ಆದರೆ ಅತಿ ದೊಡ್ಡ ಆಘಾತವೊಂದು ಎಮ್ಮಾ ಕಾಲ್ವೆಯನ್ನು ಸ್ವಾಮೀಜಿಯ ಬಳಿಗೆ ಎಳೆದುತಂದಿತು; ಅವಳಲ್ಲೊಂದು ದೊಡ್ಡ ಪರಿವರ್ತನೆಯನ್ನು ಉಂಟು ಮಾಡಿತು.

ಕಾಲ್ವೆಗೆ ಅತ್ಯಂತ ಪ್ರೀತಿಪಾತ್ರವಾದ ವಸ್ತುವೆಂದರೆ ಆಕೆಯ ಮಗಳು. ಅವಳಿಗೆ ಸ್ವಲ್ಪಮಟ್ಟಿಗೆ ಸಮಾಧಾನ ನೀಡುತ್ತಿದ್ದುದೆಂದರೆ ಅವಳ ಮಗಳು ಮಾತ್ರವೇ. ಒಂದು ದಿನ ಕಾಲ್ವೆ ತನ್ನ ಜನ ಪ್ರಿಯ ಸಂಗೀತನಾಟಕ ಪ್ರದರ್ಶನ ನೀಡುತ್ತಿದ್ದಳು. ಅದೇಕೋ ಆಕೆಗೆ ಅಂದು ತುಂಬ ಆಯಾಸ ವಾದಂತಿತ್ತು. ಮೊದಲನೆಯ ಅಂಕ ಮುಗಿಯುವ ಹೊತ್ತಿಗೆ ಸಾಕಾಗಿಹೋಗಿತ್ತು. ಇಂದು ಪ್ರದರ್ಶನ ಕೆಡುತ್ತದೆ ಎಂದೇ ಅವಳಿಗೆ ಅನ್ನಿಸಿತು. ಆದರೆ ಎರಡನೆಯ ಅಂಕ ಅದ್ಭುತವಾಗಿ ಮೂಡಿಬಂತು. ಇನ್ನು ತನ್ನ ಕೈಯಲ್ಲಿ ಸಾಧ್ಯವೇ ಇಲ್ಲವೆಂದು ಹೇಳಿ, ಪ್ರದರ್ಶನವನ್ನು ರದ್ದು ಪಡಿಸುವಂತೆ ಮ್ಯಾನೇಜರನ್ನು ಕೇಳಿಕೊಂಡಳು. ಆಗ ಅಲ್ಲಿದ್ದವರೆಲ್ಲ ಆಕೆಯನ್ನು ಹುರಿದುಂಬಿಸಿ, ಹೇಗೋ ಒಪ್ಪಿಸಿದರು. ಕಡೆಗೆ ತನಗೆ ಸ್ವಲ್ಪವೂ ಇಷ್ಟವಿಲ್ಲದಿದ್ದರೂ ಕಾಲ್ವೆ, ತನ್ನ ಶಕ್ತಿಯನ್ನೆಲ್ಲ ಒಟ್ಟುಗೂಡಿಸಿಕೊಂಡು ಹಾಡಿದಳು. ಆದರೆ ಅಂದು ಅವಳ ಕಂಠ ಹಿಂದೆಂದಿಗಿಂತಲೂ ಮಧುರ ವಾಗಿತ್ತು. ಪ್ರೇಕ್ಷಕರು ಹುಚ್ಚೆದ್ದು ಕುಣಿದರು. ಎಮ್ಮಾಕಾಲ್ವೆ ಜನರ ಅಭಿನಂದನೆಯನ್ನು ಸ್ವೀಕರಿ ಸಲೂ ನಿಲ್ಲದೆ ಸೀದಾ ತನ್ನ ಕೋಣೆಗೆ ಓಡಿದಳು. ಅವಳಿಗಿನ್ನು ನಿಲ್ಲಲೂ ತ್ರಾಣವಿರಲಿಲ್ಲ. ಆದರೆ ಅಲ್ಲಿ ಮ್ಯಾನೇಜರನೂ ಇನ್ನಿತರರೂ ಮ್ಲಾನವದನರಾಗಿ ನಿಂತಿದ್ದರು. ಅವರನ್ನು ಕಂಡಕೂಡಲೆ, ಏನೋ ಹೆಚ್ಚುಕಡಿಮೆಯಾಗಿದೆಯೆಂದು ಕಾಲ್ವೆ ಊಹಿಸಿದಳು. ಅವಳ ಊಹೆ ನಿಜವಾಗಿತ್ತು... ಅವಳ ಮಗಳು ಅಗ್ನಿ ಆಕಸ್ಮಿಕವೊಂದರಲ್ಲಿ ದುರ್ಮರಣಕ್ಕೀಡಾಗಿದ್ದಳು.

ಮೇಡಂ ಕಾಲ್ವೆ ಶೋಕಸಾಗರದಲ್ಲಿ ಮುಳುಗಿಹೋದಳು. ಅವಳಿಗೆ ಬದುಕು ಅಸಹನೀಯವಾ ಯಿತು. ಆತ್ಮಹತ್ಯೆಗೂ ಯತ್ನಿಸಿದಳು. ಅದರಲ್ಲೂ ವಿಫಲಳಾಗಿ ಹುಚ್ಚಿಯಂತಾದಳು. ಯಾರಿಗೂ ಅವಳನ್ನು ಸಂತೈಸುವ ಶಕ್ತಿಯಿರಲಿಲ್ಲ. ಆಕೆಯ ಒಬ್ಬಳು ಸ್ನೇಹಿತೆ ಮಾತ್ರ ಸದಾ ಅವಳೊಂದಿಗೇ ಇದ್ದು ಸಮಾಧಾನ ಮಾಡುತ್ತಿದ್ದಳು. ಹೋಗಿ ಸ್ವಾಮೀಜಿಯನ್ನು ನೋಡುವಂತೆ ಆಕೆ ಕಾಲ್ವೆಯನ್ನು ಬೇಡಿಕೊಂಡಳು. ಆದರೆ ಕಾಲ್ವೆ ಮತ್ತೆ ಮತ್ತೆ ಅದಕ್ಕೆ ನಿರಾಕರಿಸಿದಳು. ಒಂದು ಕೆರೆಗೆ ಹೋಗಿ ಬಿದ್ದುಬಿಡಲು ಅವಳೆಷ್ಟೋ ಸಲ ಪ್ರಯತ್ನಿಸಿದಳು. ಆದರೆ ಪ್ರತಿಸಲವೂ ಅವಳು ಬಂದು ನಿಲ್ಲುತ್ತಿದ್ದುದು ಸ್ವಾಮೀಜಿಯಿದ್ದ ಮನೆಯ ರಸ್ತೆಯಲ್ಲಿ! ಆಗ ಕನಸಿನಿಂದೆಚ್ಚರಗೊಂಡಂತೆ ಮನೆಗೆ ಹಿಂದಿರುಗುತ್ತಿದ್ದಳು. ಕಡೆಗೊಮ್ಮೆ ಅವಳು ಸ್ವಾಮೀಜಿಯನ್ನು ಕಾಣುವುದೆಂದು ನಿರ್ಧರಿಸಿ ಅವರಿದ್ದ ಮನೆಗೆ ಬಂದಳು. ಆಗ ಏನಾಯಿತೆಂಬುದನ್ನು ಮೇಡಂ ಕಾಲ್ವೆ ತನ್ನ ಆತ್ಮಚರಿತ್ರೆಯಲ್ಲಿ \eng{(My Life)} ಹೀಗೆ ವಿವರಿಸುತ್ತಾಳೆ:

“ನಾನು ಅವರಿದ್ದ ಮನೆಗೆ ಹೋದಾಗ, ಕೂಡಲೇ ನನ್ನನ್ನು ಅವರ ಅಧ್ಯಯನದ ಕೋಣೆಗೆ ಕರೆದೊಯ್ಯಲಾಯಿತು. ಅವರು ತಾವಾಗಿಯೇ ನನ್ನನ್ನು ಮಾತನಾಡಿಸುವವರೆಗೂ ನಾನವರನ್ನು ಮಾತನಾಡಿಸಬಾರದೆಂದು ನನಗೆ ಮೊದಲೇ ತಿಳಿಸಲಾಗಿತ್ತು. ನಾನು ಕೋಣೆಯನ್ನು ಪ್ರವೇಶಿಸಿದೆ. ಒಂದು ಕ್ಷಣ ಮೌನವಾಗಿ ನಿಂತೆ. ಅವರು ಘನಗಂಭೀರ ಧ್ಯಾನಭಂಗಿಯಲ್ಲಿ ಕುಳಿತಿದ್ದರು. ಅವರ ಕೇಸರಿ-ಹಳದಿ ಉಡುಗೆ ನೆಲದವರೆಗೂ ಚೆಲ್ಲಿತ್ತು. ರುಮಾಲು ಸುತ್ತಿದ್ದ ಅವರ ಶಿರ ನಸುವೇ ಮುಂದಕ್ಕೆ ಬಾಗಿತ್ತು. ಅವರ ಕಣ್ಣುಗಳು ನೆಲದಮೇಲೆ ನೆಟ್ಟಿದ್ದುವು. ಕರುಣಾಪೂರ್ಣವಾದ ಅವರ ಮುಖ ನಸುನಗೆಯಿಂದ ಬೆಳಗುತ್ತಿತ್ತು. ಸ್ವಲ್ಪ ಸಮಯದನಂತರ ಅವರು ಮೇಲೆತ್ತಿ ನೋಡ ದೆಯೇ ಮಾತನಾಡತೊಡಗಿದರು. ಅವರು ಹೇಳಿದರು:

‘ಮಗು, ನಿನ್ನ ಸುತ್ತುಮುತ್ತೆಲ್ಲ ಎಂತಹ ಪ್ರಕ್ಷುಬ್ಧ ವಾತಾವರಣವನ್ನು ನಿರ್ಮಿಸಿಕೊಂಡಿ ದ್ದೀಯೆ! ಶಾಂತವಾಗು, ಮಗು; ಅದು ಬಹಳ ಮುಖ್ಯ!’

“ಅನಂತರ, ನನ್ನ ಹೆಸರು ಕೂಡ ಗೊತ್ತಿಲ್ಲದ ಅವರು, ನಿಶ್ಚಿಂತ ಹಾಗೂ ನಿರಾಸಕ್ತ ಭಾವದಿಂದ ಮೆಲುದನಿಯಲ್ಲಿ ನನ್ನ ಖಾಸಗೀ ವಿಷಯಗಳ ಬಗ್ಗೆ ಮಾತನಾಡಿದರು. ನನ್ನ ರಹಸ್ಯ ಸಮಸ್ಯೆಗಳು ಹಾಗೂ ಚಿಂತೆಗಳನ್ನು ಅವರು ಪ್ರಸ್ತಾಪಿಸಿದರು. ಯಾವ ವಿಷಯಗಳು ನನ್ನ ಅತಿ ಹತ್ತಿರದ ಸ್ನೇಹಿತ ರಿಗೂ ತಿಳಿದಿರಲಿಲ್ಲವೆಂದುಕೊಂಡಿದ್ದೆನೋ ಅವುಗಳ ಕುರಿತಾಗಿ ಮಾತನಾಡಿದರು. ನನಗೆ ಅದೊಂದು ಅದ್ಭುತವಾಗಿ ಅಲೌಕಿಕವಾಗಿ ಕಂಡಿತು.

“ಕೊನೆಗೆ ನಾನು ಕೇಳಿಯೇಬಿಟ್ಟೆ, ‘ಇದೆಲ್ಲ ನಿಮಗೆ ಹೇಗೆ ಗೊತ್ತು! ನನ್ನ ವಿಚಾರವಾಗಿ ನಿಮಗೆ ಯಾರು ಹೇಳಿದರು?’ ಎಂದು.

“ಅವರು ಮಂದಹಾಸವನ್ನು ಬೀರುತ್ತ, ಅರ್ಥವಿಲ್ಲದ ಪ್ರಶ್ನೆಯನ್ನು ಕೇಳಿದ ಮಗುವನ್ನು ನೋಡುವಂತೆ ನನ್ನತ್ತ ನೋಡಿದರು. ಬಳಿಕ ನಿಧಾನವಾಗಿ ಹೇಳಿದರು–

‘ಯಾರೂ ನನಗೆ ಹೇಳಲಿಲ್ಲ. ಹಾಗೆ ಹೇಳಬೇಕಾದದ್ದು ಆವಶ್ಯಕವೆಂದು ನಿನಗನ್ನಿಸುತ್ತದೆಯೆ? ನಾನು ನಿನ್ನ ಮನಸ್ಸನ್ನು ತೆರೆದ ಪುಸ್ತಕದಂತೆ ಓದುತ್ತಿದ್ದೇನೆ!’

“ಕೊನೆಗೆ ನಾನು ಹೊರಡುವ ಸಮಯವಾಯಿತು. ನಾನು ಎದ್ದುನಿಲ್ಲುವಾಗ ಅವರು ಮತ್ತೆ ಹೇಳಿದರು, ‘ನೀನು ಹಿಂದಿನದೆಲ್ಲವನ್ನೂ ಮರೆತುಬಿಡಬೇಕು. ಮತ್ತೊಮ್ಮೆ ಲವಲವಿಕೆ ಸಂತೋಷ ಗಳನ್ನು ತಂದುಕೊ. ಆರೋಗ್ಯವನ್ನು ಉತ್ತಮಪಡಿಸಿಕೊ. ನಿನ್ನ ಸಮಸ್ಯೆಗಳ ಬಗೆಗೆ ಚಿಂತಿಸಿ ತಲೆ ಕೆಡೆಸಿಕೊಳ್ಳಬೇಡ. ನಿನ್ನಲ್ಲಿ ಉಕ್ಕಿಬರುವ ಭಾವನೆಗಳಿಗೆ ಯಾವುದಾದರೊಂದು ಬಗೆಯ ಬಾಹ್ಯ ಅಭಿವ್ಯಕ್ತಿಯನ್ನು ಕೊಡು. ನಿನ್ನ ಆಧ್ಯಾತ್ಮಿಕ ಆರೋಗ್ಯಕ್ಕೆ ಅದು ಬೇಕಾಗಿದೆ. ನಿನ್ನ ಕಲೆಗೆ ಅದು ಅತ್ಯಾವಶ್ಯಕ.’

“ನಾನವರ ಮಾತುಗಳಿಂದಲೂ ವ್ಯಕ್ತಿತ್ವದಿಂದಲೂ ಗಾಢವಾಗಿ ಪ್ರಭಾವಿತಳಾಗಿ ಹಿಂದಿರುಗಿದೆ. ಅವರು ನನ್ನ ಮೆದುಳಿನಿಂದ ಎಲ್ಲ ಉದ್ರೇಕಕಾರೀ ಪ್ರವೃತ್ತಿಗಳನ್ನು ಕಿತ್ತೊಗೆದು, ಬದಲಾಗಿ ಅದರ ಜಾಗದಲ್ಲಿ ತಿಳಿಯಾದ ಶಾಂತ ಭಾವನೆಗಳನ್ನು ತುಂಬಿದ್ದಂತೆ ಭಾಸವಾಯಿತು. ಅವರ ಈ ಪ್ರಬಲ ಸಂಕಲ್ಪಶಕ್ತಿಯ ಪರಿಣಾಮವಾಗಿ ನಾನು ಮತ್ತೊಮ್ಮೆ ಉಲ್ಲಸಿತಳಾದೆ. ನಗುಮೊಗದವಳಾದೆ. ಅವರು ಯಾವುದೇ ಬಗೆಯ ವಶೀಕರಣ ತಂತ್ರವನ್ನೂ ಬಳಸಲಿಲ್ಲ. ನನ್ನ ಮನಸ್ಸಿನಲ್ಲಿ ಆ ನಿಶ್ಚಿತತೆ ಯನ್ನು ಉಂಟುಮಾಡಿದುದು ಅವರ ಶೀಲಶಕ್ತಿ ಹಾಗೂ ಅವರು ಸದುದ್ದೇಶದ ತೀವ್ರತೆಗಳೇ ಆಗಿದ್ದುವು. ಅವರು ಇನ್ನೂ ಚೆನ್ನಾಗಿ ಪರಿಚಯವಾದಾಗ ತೋರಿತು–ಅವರು ಯಾರೊಂದಿಗೆ ಮಾತನಾಡುತ್ತಾರೋ ಆ ವ್ಯಕ್ತಿ ತಮ್ಮ ಮಾತುಗಳನ್ನು ಏಕಾಗ್ರಚಿತ್ತದಿಂದ ಆಲಿಸಲು ಸಾಧ್ಯವಾಗು ವಂತೆ ಅವನ ಅಸ್ತವ್ಯಸ್ತಗೊಂಡ ಭಾವನೆಗಳನ್ನು ನಿಷ್ಪಂದಗೊಳಿಸಿ ಅವನ ಮನಸ್ಸನ್ನು ಶಾಂತ ಗೊಳಿಸುತ್ತಾರೆ, ಎಂದು.

“ಅವರು ಬಹುಮಟ್ಟಿಗೆ ದೃಷ್ಟಾಂತಕಥೆಗಳ ಮೂಲಕ ನಮ್ಮ ಪ್ರಶ್ನೆಗಳಿಗೆ ಉತ್ತರಿಸುತ್ತಿದ್ದರು. ಅಥವಾ ಯಾವುದಾದರೊಂದು ಕಾವ್ಯಮಯವಾದ ಹೋಲಿಕೆಯ ಮೂಲಕ ವಿಷಯಗಳನ್ನು ಸ್ಪಷ್ಟ ಪಡಿಸುತ್ತಿದ್ದರು. ಒಂದು ದಿನ ಅಮರತ್ವದ ಬಗ್ಗೆ ಹಾಗೂ ಮನುಷ್ಯರ ವೈಯಕ್ತಿಕ ಲಕ್ಷಣಗಳು ಉಳಿದುಕೊಳ್ಳುವುದರ ಬಗ್ಗೆ ಮಾತುಕತೆ ನಡೆಯುತ್ತಿತ್ತು. ಅವರು ತಮ್ಮ ಬೋಧನೆಯ ಅತ್ಯಂತ ಮೂಲಭೂತ ವಿಷಯಗಳಲ್ಲಿ ಒಂದಾದ ಪುನರ್ಜನ್ಮದ ಸಿದ್ಧಾಂತವನ್ನು ವಿವರಿಸುತ್ತಿದ್ದರು. ಅದನ್ನು ಕೇಳಿ ನಾನು ಉದ್ಗರಿಸಿದೆ:

‘ಈ ಕಲ್ಪನೆಯನ್ನು ನನಗಂತೂ ಸಹಿಸಲು ಸಾಧ್ಯವಿಲ್ಲ! ನನ್ನ ವೈಯಕ್ತಿಕತೆ ಎಷ್ಟೇ ಅಮುಖ್ಯ ವಾದದ್ದಾಗಿರಬಹುದು. ಆದರೆ ನಾನಂತೂ ಅದಕ್ಕೆ ಅಂಟಿಕೊಂಡಿರಲು ಇಷ್ಟಪಡುತ್ತೇನೆ. ನಿತ್ಯ ವಾದ ಏಕತೆಯಲ್ಲಿ ಒಂದಾಗಲು ನನಗೆ ಸ್ವಲ್ಪವೂ ಇಷ್ಟವಿಲ್ಲ. ಅದರ ಆಲೋಚನೆಯೇ ನನಗೆ ಭಯಂಕರವಾಗಿ ತೋರುತ್ತದೆ!’

“ಅದಕ್ಕೆ ಸ್ವಾಮೀಜಿ ಮುಗುಳ್ನಕ್ಕು ಉತ್ತರಿಸಿದರು–‘ಒಂದು ದಿನ ನೀರಿನ ಹನಿಯೊಂದು ವಿಶಾಲ ಸಾಗರದೊಳಕ್ಕೆ ಬಿತ್ತು. ತಾನೆಲ್ಲಿದ್ದೇನೆ ಎಂಬುದು ಗೊತ್ತಾದಾಗ ಅದೂ ಕೂಡ, ಈಗ ನೀನು ಮಾಡುತ್ತಿರುವಂತೆಯೇ ಅಳುತ್ತ ದೂರತೊಡಗಿತು. ಆಗ ಮಹಾಸಾಗರ ನಕ್ಕು ಕೇಳಿತು, ‘ಏಕೆ ಅಳುತ್ತಿರುವೆಯಪ್ಪ? ನಿನ್ನ ಅಳುವಿಗೆ ಕಾರಣವೇನೆಂದು ನನಗೆ ಅರ್ಥವಾಗುತ್ತಿಲ್ಲ. ನೀನು ನನ್ನನ್ನು ಸೇರಿಕೊಂಡಾಗ ನಿನ್ನ ಅಣ್ಣ-ತಂಗಿಯರಾದ ಇತರ ಹನಿಗಳನ್ನು ಕೂಡಿಕೊಳ್ಳುತ್ತೀಯೆ. ಅವರಿಂದಲೇ ನಾನು ಮಾಡಲ್ಪಟ್ಟಿರುವುದು. ನೀನೀಗ ಸಾಗರವೇ ಆಗಿದ್ದೀಯೆ! ನೀನು ನನ್ನಿಂದ ಬೇರಾಗಬೇಕಾದರೆ ಸೂರ್ಯನ ಕಿರಣವೊಂದನ್ನು ಹಿಡಿದು ಮುಗಿಲಿಗೇರಿದರಾಯಿತು; ಅಷ್ಟೇ! ಅಲ್ಲಿಂದ ನೀನು ಪುನಃ ಒಂದು ಹನಿ ನೀರಾಗಿ, ಬಾಯಾರಿದ ಭೂಮಿಗೆ ವರವಾಗಿ ಕೆಳಗಿಳಿಯ ಬಹುದು!’ ಎಂದು.”

ಇದು ಸ್ವಾಮೀಜಿಯೊಂದಿಗೆ ಮೇಡಂ ಕಾಲ್ವೆಯ ಅನುಭವ. ಪ್ರಥಮ ಭೇಟಿಯಲ್ಲೇ ಅವರಿಂದ ಅಲೌಕಿಕ ಶಾಂತಿಯನ್ನು ಪಡೆದು, ಹೊಸ ಜೀವನವನ್ನೇ ಪ್ರಾರಂಭಿಸಿದ ಕಾಲ್ವೆ, ಸ್ವಾಮೀಜಿಯ ಅತ್ಯಂತ ನಿಷ್ಠಾವಂತ ಭಕ್ತಳಾದಳು. ಈ ಪ್ರಥಮ ಭೇಟಿ ನಡೆದದ್ದು ಯಾವ ಇಸವಿಯಲ್ಲಿ ಎಂಬುದು ಸ್ಪಷ್ಟವಾಗಿಲ್ಲ. ಆದರೆ ಮುಂದೆ ೧೯ಂಂರಲ್ಲಿ ಸ್ವಾಮೀಜಿ ಎರಡನೆಯ ಸಲ ಪಾಶ್ಚಾತ್ಯ ರಾಷ್ಟ್ರಗಳ ಪ್ರವಾಸ ಕೈಗೊಂಡಿದ್ದಾಗ, ಮೇಡಂ ಕಾಲ್ವೆ ಮತ್ತೆ ಕೆಲವರೊಂದಿಗೆ ಸ್ವಾಮೀಜಿಯ ಜೊತೆಯಲ್ಲಿ ಪ್ರಯಾಣ ಮಾಡಿ, ಅವರ ದಿವ್ಯ ಸಾನ್ನಿಧ್ಯದ ಆನಂದವನ್ನು ಪಡೆದಳು. ಸ್ವಾಮೀಜಿಯ ಬಗ್ಗೆ ಈ ಶ್ರೇಷ್ಠ ನಟಿ-ಗಾಯಕಿ ತನ್ನ ಆತ್ಮಚರಿತ್ರೆಯಲ್ಲಿ ಬರೆಯುತ್ತಾಳೆ:

“ನಿಜಕ್ಕೂ ‘ಭಗವಂತನೊಡನೆ ನಡೆದಾಡಿದ’ ಒಬ್ಬ ಮಹಾತ್ಮ-ಸಂತ-ತತ್ವಜ್ಞಾನಿ ಹಾಗೂ ನಿಜವಾದ ಸ್ನೇಹಿತನಾದ ವ್ಯಕ್ತಿಯ ಸಂಪರ್ಕಕ್ಕೆ ಬರುವ ಅದೃಷ್ಟ-ಆನಂದ ನನ್ನದಾಗಿತ್ತು. ನನ್ನ ಆಧ್ಯಾತ್ಮಿಕ ಜೀವನದ ಮೇಲೆ ಅವರು ಬೀರಿದ ಪ್ರಭಾವ ಅತ್ಯಂತ ಅಗಾಧ. ಅವರು ನನ್ನ ಧಾರ್ಮಿಕ ಕಲ್ಪನೆಗಳನ್ನು-ಆದರ್ಶಗಳನ್ನು ವಿಸ್ತರಿಸಿ, ಸಮಗ್ರೀಕರಣಗೊಳಿಸಿ, ನನ್ನ ಕಣ್ಮುಂದೆ ಹೊಸ ಕ್ಷಿತಿಜ ವನ್ನೇ ತೆರೆದರು; ಸತ್ಯದ ಕುರಿತು ವಿಶಾಲ ತಿಳಿವಳಿಕೆಯನ್ನು ಬೋಧಿಸಿದರು. ನನ್ನಾತ್ಮ ಅವರಿಗೆ ಎಂದೆಂದಿಗೂ ಪುಣಿಯಾಗಿರುತ್ತದೆ.”

ಸ್ವಾಮೀಜಿಯೊಂದಿಗೆ ನಿಕಟ ಬಾಂಧವ್ಯವನ್ನು ಬೆಳೆಸಿಕೊಂಡ ಮೇಡಂ ಕಾಲ್ವೆ, ಅವರಿಗೆ ಸಂಬಂಧಿಸಿದ ಇತರ ಕೆಲವು ಅಪೂರ್ವ ಸಂಗತಿಗಳ ಬಗ್ಗೆ ತಿಳಿದಿದ್ದಳು. ಇವುಗಳನ್ನು ಈಕೆ ಮುಂದೊಮ್ಮೆ ತನ್ನ ಆಪ್ತ ಸ್ನೇಹಿತೆ ಮೇಡಂ ಪಾಲ್ ವರ್ಡಿಯರ್ ಎಂಬವಳಿಗೆ ಹೇಳಿದಳು. ಈ ಪಾಲ್ ವರ್ಡಿಯರ್ ಕೂಡ ಸ್ವಾಮೀಜಿಯ ನಿಷ್ಠಾವಂತ ಭಕ್ತೆಯೇ. ಎಮ್ಮಾ ಕಾಲ್ವೆ, ಅವಳಿಗೆ ಸ್ವಾಮೀಜಿಯವರು ಅಮೆರಿಕದ ಕೋಟ್ಯಧಿಪತಿಯಾದ ರಾಕ್​ಫೆಲ್ಲರ್​ನ್ನು ಭೇಟಿಯಾದ ಬಗ್ಗೆಯೂ ಹೇಳಿದ್ದು, ಅವಳು ಅದನ್ನು ಬರೆದಿಟ್ಟುಕೊಂಡಿದ್ದಳು. ಪಾಲ್ ವರ್ಡಿಯರ್​ಳ ಟಿಪ್ಪಣಿಗಳ ಮೂಲಕ ಅವರಿಬ್ಬರ ಭೇಟಿಯ ವಿಷಯ ತಿಳಿದುಬರುತ್ತದೆ.

ಜಾನ್ ಡಿ. ರಾಕ್​ಫೆಲ್ಲರ್ ಅಮೆರಿಕದ ಅತ್ಯಂತ ಧನಿಕರಲ್ಲಿ ಅಗ್ರಗಣ್ಯ. ಪೆಟ್ರೋಲಿಯಂ ಉತ್ಪಾದನೆಯ ಮೂಲಕ ಧನಿಕನಾಗಿದ್ದ ಈತನ ಒಟ್ಟು ಆಸ್ತಿ ೧೫ಂಕೋಟಿ ಡಾಲರ್​ಗಳಿಗಿಂತಲೂ ಮಿಗಿಲಾಗಿತ್ತು. ರಾಕ್​ಫೆಲ್ಲರ್​ನ ಒಬ್ಬ ಸ್ನೇಹಿತನ ಮನೆಯಲ್ಲಿ ಆಗ ಸ್ವಾಮೀಜಿ ಇಳಿದುಕೊಂಡಿ ದ್ದರು. ಎಷ್ಟೋ ಸಲ ತನ್ನ ಸ್ನೇಹಿತರು ‘ಅಸಾಧಾರಣ, ಅದ್ಭುತ, ಹಿಂದೂ ಸಂನ್ಯಾಸಿ’ಯೊಬ್ಬನ ಬಗ್ಗೆ ಮಾತನಾಡುವುದನ್ನು ಅವನು ಕೇಳಿದ್ದ. ಅವರನ್ನು ಭೇಟಿಯಾಗುವಂತೆ ರಾಕ್​ಫೆಲ್ಲರ್​ನ ಸ್ನೇಹಿತರು ಅವನನ್ನು ಆಹ್ವಾನಿಸಿದರು. ಆದರೆ ಪ್ರತಿಸಲವೂ ಒಂದಲ್ಲ ಒಂದು ನೆಪವೊಡ್ಡಿ ತಪ್ಪಿಸಿ ಕೊಳ್ಳುತ್ತಿದ್ದ. ಆಗ ಅವನಿನ್ನೂ ತನ್ನ ಭಾಗ್ಯದ ತುತ್ತತುದಿಗೆ ಏರಿರಲಿಲ್ಲ. ಆದರೆ ಅವನಾಗಲೇ ಅತ್ಯಂತ ಬಲಾಢ್ಯನೂ ಹಟಮಾರಿಯೂ ಆಗಿದ್ದ. ಅವನಿಗೆ ಸಲಹೆ ನೀಡಲು ಯಾರಿಗೂ ಸಾಧ್ಯವಿರಲಿಲ್ಲ.

ಒಂದು ದಿನ ಮಾತ್ರ ಅವನಿಗೆ ಸ್ವಾಮೀಜಿಯನ್ನೊಮ್ಮೆ ನೋಡಿಯೇಬಿಡಬೇಕೆಂಬ ಮನಸ್ಸಾ ಯಿತು. ಅವನ ಅಭಿಮಾನ ತಡೆಯುತ್ತಿದ್ದರೂ ಯಾವುದೋ ಅವ್ಯಕ್ತ ಶಕ್ತಿಯಿಂದ ಪ್ರೇರಿತನಾದವ ನಂತೆ, ನೇರವಾಗಿ ಸ್ವಾಮೀಜಿ ಇಳಿದುಕೊಂಡಿದ್ದ ತನ್ನ ಸ್ನೇಹಿತನ ಮನೆಗೆ ಬಂದ. ಮನೆಯ ಸೇವಕ ಬಾಗಿಲು ತೆರೆದ. ರಾಕ್​ಫೆಲ್ಲರ್ ಅವನನ್ನು ಪಕ್ಕಕ್ಕೆ ತಳ್ಳಿ, “ನಾನು ಆ ಸಂನ್ಯಾಸಿಯನ್ನು ನೋಡ ಬೇಕು” ಎಂದ. ಸೇವಕ ಅವನನ್ನು ಸ್ವಾಮೀಜಿ ವಾಸವಾಗಿದ್ದ ಕೋಣೆಗೆ ಕರೆದುಕೊಂಡು ಹೋದ. ತಾನು ಬಂದಿರುವ ವಿಷಯವನ್ನು ತಿಳಿಸಿ ಅವರ ಅನುಮತಿ ಪಡೆದು ಒಳಹೋಗುವ ಸಭ್ಯತೆಯನ್ನೂ ತೋರದೆ ಸೀದಾ ಕೋಣೆಯೊಳ್ಳಕ್ಕೆ ನುಗ್ಗಿದ.

ಸ್ವಾಮೀಜಿ ಮೇಜಿನ ಮುಂದೆ ಕುಳಿತು ಏನನ್ನೋ ಬರೆಯುತ್ತಿದ್ದರು. ಕೋಣೆಯೊಳಕ್ಕೆ ಬಂದ ವರು ಯಾರೆಂದು ತಲೆಯೆತ್ತಿಯೂ ನೋಡಲಿಲ್ಲ. ಇದರಿಂದ ಅವನಿಗೆ ಅವಮಾನವಾದಂತಾ ಯಿತು. ಆದರೂ ಹೇಳದೆ ಕೇಳದೆ ನುಗ್ಗಿದ್ದು ತನ್ನದೇ ತಪ್ಪಾದ್ದರಿಂದ ಸುಮ್ಮನೆ ನಿಂತಿದ್ದ. ಬಳಿಕ ಸ್ವಾಮೀಜಿ ಬರೆಯುವುದನ್ನು ನಿಲ್ಲಿಸಿ ತಲೆಯೆತ್ತಿ ನೋಡಿ ಅವನನ್ನು ಮಾತನಾಡಿಸಿದರು. ಮಾತ ನಾಡುವಾಗ ಅವರು ಆತನ ಪೂರ್ವವೃತಾಂತಗಳ ಬಗ್ಗೆ ಪ್ರಸ್ತಾಪಿಸಿದರು. ತನಗಲ್ಲದೆ ಯಾರಿಗೂ ತಿಳಿದಿರದಿದ್ದ ವಿಷಯಗಳು ಇವರಿಗೆ ಹೇಗೆ ಗೊತ್ತಾಯಿತು!–ಎಂದು ರಾಕ್​ಫೆಲ್ಲರ್ ದಂಗಾದ. ಆದರೆ ಅವನು ಅಮೆರಿಕದ ಅತ್ಯಂತ ಪ್ರಭಾವಶಾಲೀ ವ್ಯಕ್ತಿಯೆಂಬುದರ ಲಕ್ಷ್ಯವೇ ಇಲ್ಲದಂತೆ ಸ್ವಾಮೀಜಿ ತಾವು ಹೇಳಬೇಕಾದುದನ್ನೆಲ್ಲ ನೇರವಾಗಿ ಹೇಳಿಬಿಟ್ಟರು–“ನೋಡಿ, ನೀನು ಈಗ ಕೂಡಿ ಹಾಕಿರುವ ಹಣ ನಿಮ್ಮ ಸ್ವಂತದ್ದಲ್ಲ. ಅದು ಜಗತ್ತಿಗೆ ಸೇರಿದ್ದು, ನಿಮ್ಮ ಕರ್ತವ್ಯವೇನಿದ್ದರೂ ಅದರ ಮೂಲಕ ಜಗತ್ತಿಗೆ ಒಳಿತನ್ನು ಮಾಡುವುದಷ್ಟೆ. ಆ ಹಣಕ್ಕೆ ನೀವೊಬ್ಬ ಮೇಲ್ವಿಚಾರಕರಷ್ಟೆ. ಭಗವಂತ ನಿಮಗೆ ಈ ಭಾಗ್ಯವನ್ನೆಲ್ಲ ಕೊಟ್ಟಿರುವುದೇಕೆಂದರೆ ಜನರಿಗೆ ನೆರವಾಗಲು ನಿಮಗೊಂದು ಅವಕಾಶವಾಗಲೆಂದು ಮಾತ್ರ. ಈ ಹಣವೆಲ್ಲವೂ ಸದ್ವಿನಿಯೋಗವಾಗಲೇಬೇಕು.” ಈ ಅಪರಿಚಿತ ಯುವಸಂನ್ಯಾಸಿ ತನ್ನಂಥವನಿಗೆ ಉಪದೇಶ ಮಾಡುತ್ತಿದ್ದಾನಲ್ಲ! ತಾನು ಏನು ಮಾಡಬೇಕೆಂಬು ದನ್ನು ಇವನು ಹೇಳಿಕೊಡಬೇಕೆ?–ಎಂದು ರಾಕ್​ಫೆಲ್ಲರ್​ನಿಗೆ ತುಂಬ ಅಸಮಾಧಾನವಾಯಿತು. ಉದ್ವಿಗ್ನನಾಗಿ ಕೋಣೆಯಿಂದಾಚೆ ಹೊರಟುಹೋದ. ಹೋಗಿಬರುತ್ತೇನೆ ಎಂದೂ ಹೇಳಲಿಲ್ಲ.

ಇದಾಗಿ ಸುಮಾರು ಒಂದು ವಾರ ಕಳೆದಿರಬಹುದು. ಒಂದು ದಿನ ರಾಕ್​ಫೆಲ್ಲರ್ ಹಿಂದಿ ನಂತೆಯೇ ಮತ್ತೊಮ್ಮೆ ಆತುರಾತುರವಾಗಿ ಕೋಣೆಯೊಳಕ್ಕೆ ನುಗ್ಗಿದ. ಅಂದು ಕೂಡ ಸ್ವಾಮೀಜಿ ಏನನ್ನೋ ಬರೆಯುತ್ತ ಕುಳಿತಿದ್ದರು. ರಾಕ್​ಫೆಲ್ಲರ್ ಕೋಣೆಯೊಳಕ್ಕೆ ಬಂದವನೇ ಕಾಗದ ವೊಂದನ್ನು ಸ್ವಾಮೀಜಿಯ ಮೇಜಿನ ಮೇಲೆ ಬಿಸಾಡಿದ. ಸಾರ್ವಜನಿಕ ಸಂಸ್ಥೆಯೊಂದಕ್ಕೆ ಬಹು ದೊಡ್ಡ ಮೊತ್ತವೊಂದನ್ನು ದಾನವಾಗಿ ನೀಡುವ ಅವನ ಯೋಜನೆಯ ಕುರಿತಾಗಿ ಅದರಲ್ಲಿ ಬರೆದಿತ್ತು.

“ಇಗೋ, ಇಲ್ಲಿದೆ ನೋಡಿ. ಈಗ ನಿಮಗೆ ತೃಪ್ತಿಯಾಗಿರಬೇಕು. ಇದಕ್ಕಾಗಿ ನೀವು ನನಗೆ ಕೃತಜ್ಞತೆ ಸಲ್ಲಿಸಬೇಕು” ಎಂದ ರಾಕ್​ಫೆಲ್ಲರ್.

ಸ್ವಾಮೀಜಿ ರಾಕ್​ಫೆಲ್ಲರ್​ನ ಮಾತುಗಳನ್ನು ಕೇಳಿದರು. ಆದರೆ ಅವನನ್ನು ತಲೆಯೆತ್ತಿಯೂ ನೋಡಲಿಲ್ಲ. ಮೌನವಾಗಿ ಆ ಕಾಗದವನ್ನು ಓದಿದರು. ಬಳಿಕ ಅವನಿಗೆ ಹೇಳಿದರು:

“ಕೃತಜ್ಞತೆಯನ್ನು ಸಲ್ಲಿಸಬೇಕಾದವರು ನೀವು.”

ಸಾರ್ವಜನಿಕ ಕಲ್ಯಾಣಕ್ಕಾಗಿ ರಾಕ್​ಫೆಲ್ಲರ್ ನೀಡಿದ ಮೊದಲ ದೊಡ್ಡ ವಂತಿಗೆ ಅದಾಗಿತ್ತು. ಇದಾದನಂತರ ಅವನು ಸಾರ್ವಜನಿಕ ಉದ್ದೇಶಗಳಿಗಾಗಿ ಅಪಾರ ಹಣವನ್ನು ಖರ್ಚುಮಾಡಿದ. ಮತ್ತು ಈ ಕಾರ್ಯಗಳನ್ನು ಮುಂದುವರಿಸಿಕೊಂಡು ಹೋಗಲು ಟ್ರಸ್ಟ್​ಗಳನ್ನು ಸ್ಥಾಪಿಸಿದ. ಮುಂದೆ ಇವನ ಮಗನೂ ತನ್ನ ತಂದೆಯಂತೆಯೇ ಲೋಕಹಿತಕಾರ್ಯಗಳನ್ನು ಕೈಗೊಂಡ.

