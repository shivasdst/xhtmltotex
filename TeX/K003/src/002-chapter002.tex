
\chapter{ಲೋಕಾತೀತನ ಲೋಕಾನುಭವ}

\noindent

೧೮೯೧ರ ಅಕ್ಟೋಬರ್ ತಿಂಗಳ ಅಂತ್ಯಕ್ಕೆ ಸ್ವಾಮೀಜಿ ಖೇತ್ರಿಯಿಂದ ಹೊರಟರು. ಮಹಾ ರಾಜನೂ ಇತರ ಭಕ್ತ-ಶಿಷ್ಯವರ್ಗದವರೂ ಭಾರವಾದ ಹೃದಯದಿಂದ ಅವರನ್ನು ಬೀಳ್ಕೊಟ್ಟರು. ರಜಪುತಾನದ ಭೇಟಿಯನ್ನು ಮುಗಿಸಿಕೊಂಡು ಪಶ್ಚಿಮದತ್ತ ಹೊರಟಿದ್ದ ಸ್ವಾಮೀಜಿ ಮೊದಲು ವಿರಮಿಸಿದ್ದು ಅಜ್ಮೀರದಲ್ಲಿ. ಇಲ್ಲಿ ಅವರು ಸುಮಾರು ಮೂರು ವಾರಗಳ ಕಾಲ ಉಳಿದು ಕೊಂಡಿದ್ದರು. ಮೌಂಟ್ ಅಬುವಿನಲ್ಲಿ ತಾವು ಭೇಟಿ ಮಾಡಿದ್ದ ಹರವಿಲಾಸ್ ಸಾರದಾನ ಮನೆ ಯಲ್ಲಿ ಅವರು ಮೊದಲ ಮೂರು-ನಾಲ್ಕು ದಿನಗಳನ್ನು ಕಳೆದರು. ಅನಂತರ ಅವರು ಸಾರದಾನ ಸ್ನೇಹಿತನಾದ ಶ್ಯಾಮ್​ಜಿ ಕೃಷ್ಣವರ್ಮ ಎಂಬವನ ಮನೆಯಲ್ಲಿ ಇಳಿದುಕೊಂಡರು.

ಅಜ್ಮೀರದಲ್ಲಿ ಸುಮಾರು ಮೂರು ವಾರಗಳನ್ನು ಕಳೆದ ಸ್ವಾಮೀಜಿ, ಪಶ್ಚಿಮ ಭಾರತದತ್ತ ತಮ್ಮ ಪ್ರಯಾಣವನ್ನು ಮುಂದುವರಿಸಿ ಅಹಮದಾಬಾದಿಗೆ ಬಂದರು. ಇಲ್ಲಿ ಅವರು ಒಬ್ಬ ಸಾಮಾನ್ಯ ಸಾಧುವಿನಂತೆ ಮಧುಕರಿಯ ಮೇಲೆ ಜೀವಿಸುತ್ತ ಹಲವಾರು ಪ್ರೇಕ್ಷಣೀಯ ಸ್ಥಳಗಳನ್ನು ಸಂದರ್ಶಿಸುತ್ತ ಓಡಾಡಿಕೊಂಡಿದ್ದರು. ಕೆಲವು ದಿನಗಳ ಮೇಲೆ ಅಹಮದಾಬಾದಿನ ಉಪ ನ್ಯಾಯಾಧೀಶನಾದ ಲಾಲ್​ಶಂಕರ್ ಉಮಿಯಾ ಎಂಬವನ ಅತಿಥಿಯಾಗಿ ಇಳಿದುಕೊಂಡರು. ಇಲ್ಲಿನ ಜೈನ ದೇವಾಲಯಗಳು, ಮಸೀದಿ-ಗೋರಿಗಳು ಹಾಗೂ ಇತರ ಚಾರಿತ್ರಿಕ ಸ್ಥಳಗಳನ್ನು ಅವರು ಸಂದರ್ಶಿಸಿದರು; ಅವುಗಳ ಅದ್ಭುತವನ್ನು ಕಂಡು ಬೆರಗಾದರು. ಅಲ್ಲದೆ ಇಲ್ಲಿನ ಹಲವಾರು ಜೈನ ಪಂಡಿತರೊಂದಿಗೆ ಸಂಭಾಷಿಸಿ, ಜೈನ ಧರ್ಮ-ಸಂಸ್ಕೃತಿಗಳ ಕುರಿತಾದ ತಮ್ಮ ಜ್ಞಾನ ವನ್ನು ವೃದ್ಧಿಗೊಳಿಸಿಕೊಳ್ಳುವ ಸುಯೋಗ ಅವರಿಗೆ ದೊರಕಿತು.

ಅಹಮದಾಬಾದಿನಲ್ಲಿ ಕೆಲದಿನವಿದ್ದು, ಕಾಥೇವಾಡದ ವಾಧ್ವಾನ್ ಎಂಬಲ್ಲಿಗೆ ಬಂದರು ಸ್ವಾಮೀಜಿ. ಸುತ್ತುಮುತ್ತಲ ಹಲವಾರು ಸ್ಥಳಗಳನ್ನು ವೀಕ್ಷಿಸುತ್ತ ಒಂದೆರಡು ದಿನಗಳನ್ನು ಕಳೆದು ಲಿಂಬ್ಡಿಯ ಕಡೆಗೆ ಹೊರಟರು. ಕಾಲುನಡಿಗೆಯಲ್ಲೇ ಮುಂದುವರಿಯುತ್ತ, ಭಿಕ್ಷಾನ್ನವನ್ನವಲಂಬಿಸಿ, ಆಶ್ರಯ ಸಿಕ್ಕಲ್ಲಿ ಮಲಗುತ್ತ ಸಾಗಿದರು. ಲಿಂಬ್ಡಿಯನ್ನು ತಲುಪಿದಾಗ, ಅಲ್ಲೊಂದು ಕಡೆ ಕೆಲವು ಸಾಧುಗಳು ವಾಸವಾಗಿರುವುದಾಗಿ ತಿಳಿಯಿತು. ಅದೊಂದು ಜನದಟ್ಟಣೆಯಿಂದ ದೂರವಾದ ಜಾಗ. ಸ್ವಾಮೀಜಿ ಅಲ್ಲಿಗೆ ಹೋದಾಗ ಆ ಸಾಧುಗಳು ಅವರನ್ನು ಆದರದಿಂದ ಸ್ವಾಗತಿಸಿ, ತಮಗಿಷ್ಟ ಬಂದಷ್ಟು ದಿನ ಅವರು ಅಲ್ಲಿರಬಹುದು ಎಂದು ಹೇಳಿದರು. ದೂರ ಪ್ರಯಾಣದಿಂದ ದಣಿದು ಹಸಿದಿದ್ದ ಸ್ವಾಮೀಜಿ, ಸಾಧುಗಳ ಆಹ್ವಾನವನ್ನು ಸಂತೋಷದಿಂದ ಸ್ವೀಕರಿಸಿದರು. ಅವರಿಗೆ ಒಂದು ಒಳ್ಳೆಯ ಕೋಣೆಯಲ್ಲಿ ಇಳಿದುಕೊಳ್ಳಲು ಅವಕಾಶ ಮಾಡಿಕೊಡಲಾಯಿತು.

ಆದರೆ ತಾವು ಬಂದಿರುವುದು ಎಂಥಾ ಅಪಾಯದ ಜಾಗಕ್ಕೆ ಎಂಬುದರ ಕಲ್ಪನೆಯೇ ಅವರಿಗಿರ ಲಿಲ್ಲ. ಆ ‘ಸಾಧು’ಗಳೆಲ್ಲ ಅತ್ಯಂತ ಕೀಳುಮಟ್ಟದ ವಾಮಾಚಾರಿಗಳು, ವಿಕಟ ಲೈಂಗಿಕತೆಯಲ್ಲಿ ಮುಳುಗಿದ್ದವರು, ಆದರೆ ಸರಳ ಸ್ವಭಾವದ ಸ್ವಾಮೀಜಿ ಇದನ್ನರಿಯದೆ ಆ ಮೃತ್ಯುಕೂಪದಲ್ಲಿ ಸಿಲುಕಿಕೊಂಡುಬಿಟ್ಟಿದ್ದರು! ಅವರು ಅಲ್ಲಿಗೆ ಹೋದ ಒಂದೆರಡು ದಿನಗಳ ಬಳಿಕ, ಪಕ್ಕದ ಕೋಣೆ ಯಿಂದ ಹೆಂಗಸರೂ ಗಂಡಸರೂ ಕೂಡಿ ಉಚ್ಚರಿಸುತ್ತಿದ್ದ ಮಾಟ-ಮಂತ್ರದ ದನಿ ಕೇಳಿಬಂತು. ಅನುಮಾನಗೊಂಡ ಸ್ವಾಮೀಜಿ ತಕ್ಷಣ ಅಲ್ಲಿಂದ ಹೊರಟುಬಿಡಲು ನೋಡಿದಾಗ ಅನುಮಾನ ದೃಢವಾಯಿತು–ಅವರ ಕೋಣೆಯ ಬಾಗಿಲಿಗೆ ಹೊರಗಿನಿಂದ ಬೀಗ ಬಿದ್ದಿದೆ! ಅಲ್ಲದೆ ಅವರು ತಪ್ಪಿಸಿಕೊಂಡು ಹೋಗದಂತೆ ಒಬ್ಬ ಕಾವಲುಗಾರನನ್ನು ಬೇರೆ ನಿಲ್ಲಿಸಲಾಗಿದೆ! ಭಗವಂತ ತಮ್ಮನ್ನು ಇದೆಂತಹ ಜಾಗಕ್ಕೆ ತಂದುಬಿಟ್ಟನಪ್ಪ ಎಂದು ಸ್ವಾಮೀಜಿ ಭಯಾಶ್ಚರ್ಯದಿಂದ ತೊಳಲುತ್ತಿದ್ದಾಗ ಆ ಸಾಧುಗಳ ಮುಖಂಡ ಅವರನ್ನು ತನ್ನ ಬಳಿಗೆ ಕರೆಸಿಕೊಂಡು ಹೇಳಿದ– “ಏನಪ್ಪ, ನಿನ್ನ ಮುಖದ ಕಳೆಯನ್ನು ನೋಡಿದರೆ ಬಹಳ ಸಾಧನೆ ಮಾಡಿ ಒಳ್ಳೆಯ ಶಕ್ತಿಯನ್ನು ಪಡೆದುಕೊಂಡಿರುವಂತೆ ಕಾಣುತ್ತೀ? ಎಷ್ಟೋ ವರ್ಷಗಳಿಂದ ಕಠೋರ ಬ್ರಹ್ಮಚರ್ಯವನ್ನು ಪಾಲಿಸಿಕೊಂಡು ಬಂದಿರಲೇ ಬೇಕು! ಬಹಳ ಸಂತೋಷ ಬಹಳ ಸಂತೋಷ... ಆದರೆ ಈಗ ನೋಡು, ನಿನ್ನ ಆ ತಪಸ್ಸಿನ ಫಲವನ್ನೆಲ್ಲ ನಮಗೆ ಕೊಟ್ಟುಬಿಡಬೇಕು. ನಾವೊಂದು ವಿಶೇಷ ಸಾಧನೆ ಮಾಡಿ ಸಿದ್ಧಿಗಳನ್ನು ಪಡೆದುಕೊಳ್ಳಲು ನಿನ್ನ ಬ್ರಹ್ಮಚರ್ಯವನ್ನು ಭಂಗಮಾಡಲಿದ್ದೇವೆ.”

ಸ್ವಾಮೀಜಿ ನಡುಗಿದರು. ಹಿಂದೆ ಶ್ರೀಶಂಕರಾಚರ್ಯರಿಗೂ ಇಂಥದೇ ಒಂದು ಅಪಾಯ ಎದುರಾಗಿತ್ತು. ಆಗ ಕಾಪಾಲಿಕರು ಅವರನ್ನೇ ಬಲಿಕೊಟ್ಟು ಸಿದ್ಧಿ ಮಾಡಿಕೊಳ್ಳಲು ನೋಡಿದರೆ, ಇಲ್ಲಿ ಇವರು ಸ್ವಾಮೀಜಿಯ ಬ್ರಹ್ಮಚರ್ಯವನ್ನು ಬಲಿಕೊಡುವುದರ ಮೂಲಕ ತಮ್ಮ ಉದ್ದೇಶ ವನ್ನು ಸಾಧಿಸಿಕೊಳ್ಳಲು ಸಿದ್ಧರಾಗಿದ್ದಾರೆ! ಆದರೆ ಸ್ವಾಮೀಜಿ ಪ್ರತ್ಯುತ್ಪನ್ನಮತಿಯುಳ್ಳವರು. ಆದ್ದರಿಂದ ಅಧೀರರಾಗಲಿಲ್ಲ. ಸಾಧುಗಳ ಮಾತನ್ನು ಹಗುರವಾಗಿ ತೆಗೆದುಕೊಂಡವರಂತೆ ಶಾಂತವಾಗಿದ್ದುಬಿಟ್ಟರು. ಆದರೆ ಅಲ್ಲಿಂದ ತಪ್ಪಿಸಿಕೊಳ್ಳುವ ಉಪಾಯವೊಂದನ್ನು ಮನದಲ್ಲೇ ಸಿದ್ಧಪಡಿಸಿಕೊಂಡರು.

ಈ ವೇಳೆಗೆ ಇಲ್ಲಿ ಅವರಿಗೆ ಒಬ್ಬ ಹುಡುಗನ ಪರಿಚಯವಾಗಿತ್ತು. ಈತ ಪ್ರತಿದಿನ ಬಂದು ಅವರನ್ನು ಮಾತನಾಡಿಸಿಕೊಂಡು ಹೋಗುತ್ತಿದ್ದ. ಮರುದಿನ ಈ ಹುಡುಗ ಬಂದಾಗ ಸ್ವಾಮೀಜಿ, ಕೈಗೆ ಸಿಕ್ಕಿದ ಒಂದು ಮಡಕೆಯ ಚೂರಿನ ಮೇಲೆ ಇದ್ದಿಲಿನಿಂದ ತಮ್ಮ ಪರಿಸ್ಥಿತಿಯನ್ನು ವಿವರಿಸಿ ಎರಡು ಸಾಲು ಬರೆದುಕೊಟ್ಟು, ಅದನ್ನು ಕೂಡಲೇ ಅಲ್ಲಿನ ರಾಜನಿಗೆ ತಲುಪಿಸುವಂತೆ ಹೇಳಿದರು. ಹುಡುಗ ಬುದ್ಧಿವಂತ; ಯಾರಿಗೂ ತಿಳಿಯದಂತೆ ಅದನ್ನು ತೆಗೆದುಕೊಂಡುಹೋಗಿ ಲಿಂಬ್ಡಿಯ ರಾಜನಾದ ಠಾಕೂರ್ ಜಸವಂತಸಿಂಗನಿಗೆ ತಲುಪಿಸಿದ. ಸಂದೇಶವನ್ನು ಓದಿಕೊಂಡ ರಾಜ, ತಕ್ಷಣ ಸೈನಿಕರನ್ನು ಕಳಿಸಿ ಸ್ವಾಮೀಜಿಯನ್ನು ಬಿಡಿಸಿದ. ಅನಂತರ ಸ್ವಾಮೀಜಿ, ಜಸವಂತ ಸಿಂಗನ ಆಹ್ವಾನದಂತೆ ಅರಮನೆಯಲ್ಲೇ ಉಳಿದುಕೊಂಡರು. ಎಂದೂ ಮರೆಯಲಾಗದ ಪಾಠವೊಂದನ್ನು ಅವರು ಲಿಂಬ್ಡಿಯಲ್ಲಿ ಕಲಿತಿದ್ದರು.

ಈ ಘಟನೆಯನ್ನು ಓದಿದಾಗ ನಮ್ಮಲ್ಲೊಂದು ಅನುಮಾನವೇಳಬಹುದು. ‘ಸ್ವಾಮೀಜಿ ಯಂತಹ ಮಹಾತಪಸ್ವಿಗಳು, ಜ್ಞಾನಿಗಳು ಹೊಸದಾಗಿ ಪಾಠ ಕಲಿಯಬೇಕಾಯಿತೆ! ಅತ್ಯುನ್ನತ ಆಧ್ಯಾತ್ಮಿಕ ಅನುಭವಗಳನ್ನು ಹೊಂದಿದವರಾದ ಸ್ವಾಮೀಜಿ ಮನುಷ್ಯರನ್ನು ನೋಡಿದ ಮಾತ್ರ ದಿಂದಲೇ ಅವರ ಯೋಗ್ಯತೆಯನ್ನು, ಗುಣಾವಗುಣಗಳನ್ನು ತಿಳಿಯಬಲ್ಲವರಾಗಿರಬೇಕಿತ್ತಲ್ಲವೆ? ಹೀಗಿರುವಾಗ ಅವರು ಈ ದುರಾಚಾರಿಗಳ ಕೈಯಲ್ಲೇಕೆ ಸಿಕ್ಕಿಹಾಕಿಕೊಳ್ಳಬೇಕಾಗಿತ್ತು?’ ಆದರೆ ಇಲ್ಲಿ ನಾವೊಂದು ವಿಷಯವನ್ನು ತಿಳಿದಿರಬೇಕಾಗುತ್ತದೆ. ಏನೆಂದರೆ, ಜ್ಞಾನಿಗಳಾದವರು ಸುಮ್ಮ ಸುಮ್ಮನೆ ಕಂಡ ಕಂಡ ವ್ಯಕ್ತಿಗಳ ಯೋಗ್ಯತೆಯನ್ನಾಗಲಿ, ಅಂತರಂಗವನ್ನಾಗಲಿ ಅಳೆದುನೋಡುವ ಗೋಜಿಗೆ ಹೋಗುವುದಿಲ್ಲ. ಅಲ್ಲದೆ ಎಂತಹ ಯೋಗಿಗಳೇ ಆದರೂ, ಅವರು ತಮ್ಮ ಮನಸ್ಸನ್ನು ಯೋಗದೃಷ್ಟಿಯಲ್ಲಿ ಏಕಾಗ್ರಗೊಳಿಸಿದಾಗ ಮಾತ್ರ ಮನುಷ್ಯರ ಅಂತರಂಗ ಹಾಗೂ ಭೂತ- ಭವಿಷ್ಯಗಳು ತಿಳಿದುಬರುತ್ತವೆ. ಆದರೆ ಅವರು ತಮ್ಮ ಮನಸ್ಸನ್ನು ವ್ಯಾವಹಾರಿಕ ಸ್ತರದಲ್ಲಿರಿಸಿ ದಾಗ ಅವರೂ ಸಾಮಾನ್ಯರಂತೆಯೇ ಇರುತ್ತಾರೆ. ಸಾಮಾನ್ಯ ಜನರಲ್ಲಾದರೆ ಕುಯುಕ್ತಿಯ ವ್ಯವಹಾರಗಳು ಕಂಡುಬರಬಹುದು; ಆದರೆ ಈ ಮಹಾತ್ಮರು ಮುಗ್ಧ ಬಾಲಕರಂತಿರುತ್ತಾರೆ. ಆದ್ದರಿಂದ ಮಹಾತ್ಮರನ್ನು ಮೋಸಗೊಳಿಸುವುದು ಇನ್ನೂ ಸುಲಭ. ಆದರೆ ಒಮ್ಮೆ ಇಂತಹ ಭಯಾನಕ ಅನುಭವವಾದಮೇಲೆ ಸ್ವಾಮೀಜಿ, ಇನ್ನು ಮುಂದೆ ಸ್ವಲ್ಪ ಎಚ್ಚರದಿಂದಿರಬೇಕೆಂದು ನಿಶ್ಚಯಿಸಿದರು.

ರಾಜಾ ಜಸವಂತಸಿಂಗನ ಆಸ್ಥಾನದಲ್ಲಿ ಸ್ವಾಮೀಜಿ ಅನೇಕ ಪಂಡಿತರೊಂದಿಗೆ ಸಂಸ್ಕೃತದಲ್ಲಿ ಚರ್ಚೆಗಳನ್ನು ನಡೆಸಿದರು. ಈ ಸಂದರ್ಭದಲ್ಲಿ ಪುರೀಕ್ಷೇತ್ರದ ಗೋವರ್ಧನ ಮಠದ ಶ್ರೀಶಂಕರಾ ಚಾರ್ಯರೂ ಉಪಸ್ಥಿತರಿದ್ದರು. ಈ ಯುವಸಂನ್ಯಾಸಿಯ ಪ್ರಕಾಂಡ ಪಾಂಡಿತ್ಯ, ಆಗಾಧ ತಿಳಿವಳಿಕೆ ಹಾಗೂ ತೀವ್ರ ಅನುಕಂಪೆಯ ಹೃದಯ–ಇವುಗಳನ್ನು ಕಂಡು ಸ್ವಾಮಿಗಳು ವಿಸ್ಮಯಾನಂದಿತ ರಾದರು.

ಜಸವಂತಸಿಂಗ್ ಅತ್ಯಂತ ಪುರೋಗಾಮೀ ಮನೋವೃತ್ತಿಯುಳ್ಳವನಾಗಿದ್ದು, ಸನಾತನ ಧರ್ಮದ ನಿಷ್ಠಾವಂತ ಪ್ರತಿಪಾದಕನಾಗಿದ್ದ. ಜೊತೆಗೆ ಆಧುನಿಕ ವಿಚಾರಧಾರೆಯನ್ನೂ ಹೊಂದಿದ್ದ. ಈತ ಕೆಲ ವರ್ಷಗಳ ಹಿಂದೆ ಇಂಗ್ಲೆಂಡ್​-ಅಮೆರಿಕಗಳನ್ನು ಸಂದರ್ಶಿಸಿ ಅಲ್ಲಿನ ಪ್ರಗತಿಯನ್ನು ಕಂಡು ಪ್ರಭಾವಿತನಾಗಿದ್ದ. ಸ್ವಾಮೀಜಿಯ ಮಾತುಕತೆಗಳನ್ನೂ ಚರ್ಚೆಗಳನ್ನೂ ಕೇಳಿ, ಅವರ ಜ್ಞಾನದಾಳವನ್ನು ಮನಗಂಡು ಅವನು ರೋಮಾಂಚನಗೊಂಡ. ವೇದಾಂತ ಪ್ರಚಾರಕಾರ್ಯಕ್ಕಾಗಿ ಅವರು ಪಾಶ್ಚಾತ್ಯ ರಾಷ್ಟ್ರಗಳಿಗೆ ಹೋಗಬೇಕು ಎಂದು ಸಲಹೆ ನೀಡಿದ ಮೊದಲಿಗರಲ್ಲಿ ಇವನೂ ಒಬ್ಬ.

ಲಿಂಬ್ಡಿಯಲ್ಲಿ ಕೆಲಕಾಲದ ವಾಸ್ತವ್ಯದ ನಂತರ ಕೆಲವು ಪ್ರಮುಖ ವ್ಯಕ್ತಿಗಳಿಗೆ ಇಲ್ಲಿನ ಮಹಾರಾಜ ಬರೆದುಕೊಟ್ಟಿದ್ದ ಪರಿಚಯ ಪತ್ರಗಳನ್ನು ಪಡೆದುಕೊಂಡು, ಜುನಾಗಢದ ಕಡೆಗೆ ಹೊರಟರು ಸ್ವಾಮೀಜಿ. ದಾರಿಯಲ್ಲಿ ಅವರು ಭಾವನಗರ ಹಾಗೂ ಸಿಹೋರ್​ಗಳನ್ನೂ ಸಂದರ್ಶಿಸಿ ದರು. ಜುನಾಗಢದಲ್ಲಿ ಅವರು ಆ ರಾಜ್ಯದ ದಿವಾನರಾದ ಹರಿದಾಸ್ ವಿಹಾರಿದಾಸ್ ದೇಸಾಯಿ ಎಂಬವರ ಅತಿಥಿಯಾಗಿ ಇಳಿದುಕೊಂಡರು. ಇವರು ಅತ್ಯಂತ ಸಮರ್ಥ, ಪ್ರಾಮಾಣಿಕ ಆಡಳಿತ ಗಾರನಾಗಿ ಹಾಗೂ ತಮ್ಮ ಸಚ್ಚಾರಿತ್ರ್ಯಕ್ಕಾಗಿ ಹೆಸರುವಾಸಿಯಾಗಿದ್ದವರು. ಹರಿದಾಸ್ ದೇಸಾಯಿ ಸ್ವಾಮೀಜಿಯ ಪ್ರಬಲ ಆಕರ್ಷಣೆಯಲ್ಲಿ ಸಿಲುಕಿಕೊಂಡು ಮಂತ್ರಮುಗ್ಧರಾದರು. ಪ್ರತಿ ಸಂಜೆಯೂ ರಾಜ್ಯದ ಹಲವಾರು ಅಧಿಕಾರಿಗಳನ್ನು ಆಹ್ವಾನಿಸಿ ಸ್ವಾಮೀಜಿಯೊಂದಿಗೆ ಸುದೀರ್ಘ ಸಂಭಾಷಣೆಗಳಲ್ಲಿ ತೊಡಗುತ್ತಿದ್ದರು. ಇವರು ಸ್ವಾಮೀಜಿಯ ಆಪ್ತ ಸ್ನೇಹಿತರಲ್ಲೊಬ್ಬರಾಗಿ, ಮುಂದೆ ಅವರೊಂದಿಗೆ ಪತ್ರ ಸಂಬಂಧವನ್ನೂ ಇಟ್ಟುಕೊಂಡಿದ್ದರು. ಇವರನ್ನು ಸ್ವಾಮೀಜಿ, ‘ದಿವಾನ್​ಜಿ ಸಾಹೇಬ್​’ ಎಂದು ಸಂಬೋಧಿಸುತ್ತಿದ್ದರು.

ಇಲ್ಲಿ ಅವರು ಭೇಟಿಯಾದವರಲ್ಲಿ ದಿವಾನರ ಕಛೇರಿಯ ಮೇಲ್ವಿಚಾರಕನಾದ ಸಿ. ಹೆಚ್. ಪಾಂಡ್ಯ ಎಂಬುವನೊಬ್ಬ. ಅವನ ಆಮಂತ್ರಣವನ್ನು ಮನ್ನಿಸಿ ಸ್ವಾಮೀಜಿ ಅವನ ಮನೆಯಲ್ಲೂ ಕೆಲವು ದಿನಗಳನ್ನು ಕಳೆದರು. ಈತ, ಸ್ವಾಮೀಜಿ ತನ್ನ ಮನೆಯಲ್ಲಿ ಕಳೆದ ಆ ದಿನಗಳ ಒಂದು ಸೊಗಸಾದ ವರ್ಣನೆಯನ್ನು ನೀಡಿದ್ದಾನೆ. ಸ್ವಾಮೀಜಿಯ ಸರಳ ಜೀವನ, ಮುಕ್ತ ಮನಸ್ಸು, ಕಲೆ-ವಿಜ್ಞಾನಗಳಲ್ಲಿ ಅವರಿಗಿದ್ದ ಅಪಾರ ಪಾಂಡಿತ್ಯ, ಧರ್ಮನಿಷ್ಠೆ, ಅವರ ವಿಶಾಲ ದೃಷ್ಟಿಕೋನ, ಆಕರ್ಷಕ ಹಾಗೂ ಅಸಾಧಾರಣ ವ್ಯಕ್ತಿತ್ವ, ಪ್ರಚಂಡ ವಾಕ್ಪಟುತ್ವ–ಇವು ಅವರೆಲ್ಲರ ಮನಸೂರೆ ಗೊಂಡವು. ಅಲ್ಲದೆ ಸ್ವಾಮೀಜಿಯ ಅಲೌಕಿಕ ಕಂಠಸ್ವರದಿಂದ ಕೂಡಿದ ಭಾವಪೂರ್ಣ ಹಾಡು ಗಾರಿಕೆ ಅಲ್ಲಿನ ಜನರನ್ನು ಅಯಸ್ಕಾಂತದಂತೆ ಸೆಳೆದ ಮತ್ತೊಂದು ಅಂಶ. ಇದೆಲ್ಲ ಸಾಲ ದೆಂಬಂತೆ, ಅವರು ತಮ್ಮ ಆಪ್ತರ ಬುದ್ಧಿ-ಹೃದಯಗಳೊಂದಿಗೆ ನಾಲಿಗೆಗೂ ಸವಿಯನ್ನುಣಿಸಿದರು! ನಿಷ್ಣಾತ ಪಾಕಶಾಸ್ತ್ರಜ್ಞರಾಗಿದ್ದ ಸ್ವಾಮೀಜಿ, ತಮಗಾಗಿ ಬಾಯಲ್ಲಿ ನೀರೂರಿಸುವಂತಹ ರಸಗುಲ್ಲಾ ಗಳನ್ನು ತಯಾರಿಸಿ ಬಡಿಸಿ ಉಪಚರಿಸಿದರು ಎಂದೂ ಪಾಂಡ್ಯ ಬರೆಯುತ್ತಾನೆ.

ಜುನಾಗಢದಲ್ಲಿ ಸ್ವಾಮೀಜಿ, ಸಂಭಾಷಣೆಗಳ ಸಂದರ್ಭದಲ್ಲಿ ಏಸುಕ್ರಿಸ್ತನ ಬಗ್ಗೆ ಬಹಳವಾಗಿ ಹೇಳುತ್ತಿದ್ದರು. ಪಾಶ್ಚಾತ್ಯ ನಾಗರಿಕತೆಯ ನೈತಿಕ ಮೌಲ್ಯಗಳನ್ನು ಪುನಃ ಸಂಸ್ಥಾಪಿಸುವಲ್ಲಿ ಕ್ರಿಸ್ತನ ಪ್ರಭಾವವನ್ನು ತಾವು ಬಹಳ ಹಿಂದಿನಿಂದಲೂ ಮನಗಂಡಿರುವುದಾಗಿ ನುಡಿದರು. ಮಧ್ಯಕಾಲೀನ ಐರೋಪ್ಯ ಸಂಸ್ಕೃತಿಯ ಹಿರಿಮೆಯ ಅಂಶಗಳಾದ ಅದ್ಭುತ ಕಲಾಕೃತಿಗಳು, ಅನರ್ಘ್ಯ ವಾಸ್ತು ಶಿಲ್ಪಗಳು, ರಾಜಕೀಯ ಬೆಳವಣಿಗೆ ಮುಂತಾದವುಗಳೆಲ್ಲವೂ ‘ಸಂನ್ಯಾಸಿ ಕ್ರಿಸ್ತ’ನ ಬೋಧನೆ ಗಳೊಂದಿಗೆ ಹೇಗೆ ಹಾಸುಹೊಕ್ಕಾಗಿವೆ ಎಂಬುದನ್ನು ಸುದೀರ್ಘವಾಗಿ ವಿವರಿಸಿದರು ಸ್ವಾಮೀಜಿ. ಬಳಿಕ ಸನಾತನ ಧರ್ಮದ ಉತ್ಕೃಷ್ಟತೆಯನ್ನು ಕೇಳುಗರ ಮನಮುಟ್ಟುವಂತೆ ಬಣ್ಣಿಸಿದರು. ಭಾರತೀಯ ಸಂಸ್ಕೃತಿಯು ಪ್ರತಿನಿಧಿಸುವ ಮೌಲ್ಯಗಳು, ಮತ್ತು ಸಮಸ್ತ ವಿಶ್ವದ ಆಧ್ಯಾತ್ಮಿಕ ಭಾವನೆಗಳ ವಿಕಾಸದಲ್ಲಿ ಹಿಂದೂಗಳ ಅನುಭವಗಳ ಮಹತ್ವ–ಇವುಗಳನ್ನು ವಿಶ್ಲೇಷಿಸಿದರು. ಕಡೆಯಲ್ಲಿ ದಕ್ಷಿಣೇಶ್ವರದ ದೇವಮಾನವನ ಜೀವನ-ಬೋಧನೆಗಳ ಬಗ್ಗೆ ತಿಳಿಸಿ, ತಮ್ಮ ಗುರು ದೇವನನ್ನು ಭಾರತದ ಪಶ್ಚಿಮ ಮೂಲೆಯ ಈ ಪ್ರಾಂತ್ಯದಲ್ಲೂ ಪರಿಚಯಿಸಿಕೊಟ್ಟರು. ಅಲ್ಲದೆ, ಅನೇಕ ಸಂಪ್ರದಾಯಸ್ಥ ಪಂಡಿತರೊಂದಿಗೆ ಶಾಸ್ತ್ರಗ್ರಂಥಗಳ ಕುರಿತಾಗಿ ದೀರ್ಘವಾಗಿ ಚರ್ಚಿಸಿದರು.

ಜುನಾಗಢದಲ್ಲಿ ಭಾರತದ ಪುರಾತನ ಕಾಲದ ದಿನಗಳನ್ನು ನೆನಪಿಸಿಕೊಡುವ ಹಲವಾರು ಶಿಥಿಲ ಸ್ಮಾರಕಗಳಿವೆ. ಬುದ್ಧನ ಕಾಲದ ಕೆಲವು ಗುಹೆಗಳೂ ಅಶೋಕನ ಹೆಸರಿನ ಒಂದು ಬೃಹತ್ ಶಿಲೆಯೂ ಇಲ್ಲಿವೆ. ಭಾರತದ ಇತಿಹಾಸದ ಬಗ್ಗೆ ಅಸಾಮಾನ್ಯ ಪರಿಜ್ಞಾನವನ್ನೂ ಆಸಕ್ತಿಯನ್ನೂ ಹೊಂದಿದ್ದ ಸ್ವಾಮೀಜಿ, ಇವುಗಳನ್ನೆಲ್ಲ ಕೂಲಂಕಷವಾಗಿ ವೀಕ್ಷಿಸಿದರು. ಇಲ್ಲಿಂದ ಎರಡು ಮೈಲಿ ದೂರದಲ್ಲಿರುವ ಗಿರಿನಾರ್ ಬೆಟ್ಟಗಳಿಗೂ ಅವರು ಭೇಟಿನೀಡಿದರು. ಹಿಂದೂ, ಬೌದ್ಧ ಹಾಗೂ ಜೈನ ಮತಸ್ಥರಿಗೆ ಸಮಾನವಾಗಿ ಪವಿತ್ರವಾದ ಯಾತ್ರಾಸ್ಥಳ–ಈ ಗಿರಿಶಿಖರಗಳು. ಇಲ್ಲಿ ಬೇರೆ ಬೇರೆ ಬೆಟ್ಟಗಳ ಮೇಲಿರುವ ಹಲವಾರು ಹಿಂದೂ ಹಾಗೂ ಜೈನ ದೇವಾಲಯಗಳನ್ನು ಮುಟ್ಟಲು, ಅತ್ಯಂತ ಕಿರಿದಾದ ಹಾಗೂ ಕಡಿದಾದ ನೂರಾರು ಮೆಟ್ಟಲುಗಳನ್ನೇರಿ ಹೋಗಬೇಕಾಗುತ್ತದೆ. ಪರ್ವತಗಳ ಹಾದಿಯಲ್ಲಿ ನಡೆದು ಅಭ್ಯಾಸವಿದ್ದ ಸ್ವಾಮೀಜಿ ಇವುಗಳನ್ನೆಲ್ಲ ಸುಲಭವಾಗಿಯೇ ಕ್ರಮಿಸಿದರು. ಆದರೆ ಇಲ್ಲಿಗೆ ಯಾತ್ರಾರ್ಥಿಯಾಗಿ ಬಂದಿದ್ದ ಅವರಿಗೆ, ಇಲ್ಲಿನ ಪ್ರಶಾಂತ-ಏಕಾಂತ ವಾತಾವರಣವನ್ನು ಕಂಡಾಗ, ಧ್ಯಾನಾನಂದದಲ್ಲಿ ಮುಳುಗಿಬಿಡುವ ಉತ್ಕಟೇಚ್ಛೆಯಾಯಿತು. ಅಲ್ಲಿಯೇ ಒಂದು ನಿರ್ಜನ ಗುಹೆಯನ್ನು ಆಶ್ರಯಿಸಿಕೊಂಡು ಅನೇಕ ದಿನಗಳವರೆಗೆ ಧ್ಯಾನ ನಿರತರಾದರು. ಈ ದಿನಗಳಲ್ಲಿ ಜುನಾಗಢದ ದಿವಾನರಾದ ಹರಿದಾಸ್ ದೇಸಾಯಿ ಅವರ ಕಡೆಗೆ ವಿಶೇಷ ಎಚ್ಚರಿಕೆ ವಹಿಸಿ ಅವರ ಆವಶ್ಯಕತೆಗಳನ್ನೆಲ್ಲ ಪೂರೈಸುತ್ತಿದ್ದರು.

ಗಿರಿನಾರ್​ನಲ್ಲಿ ಕೆಲವು ಆನಂದದ ದಿನಗಳನ್ನು ಕಳೆದು, ಸ್ವಾಮೀಜಿ ನವೋತ್ಸಾಹ ಭರಿತರಾಗಿ ಜುನಾಗಢಕ್ಕೆ ಮರಳಿದರು. ಇದೇ ವೇಳೆಗೆ ಗುಜರಾತಿನಲ್ಲಿ ಪರಿವ್ರಾಜಕರಾಗಿ ಸಂಚರಿಸುತ್ತಿದ್ದ ಸ್ವಾಮಿ ಅಭೇದಾನಂದರೂ ಜುನಾಗಢಕ್ಕೆ ಬಂದರು. ಇಲ್ಲಿ ‘ಸಚ್ಛಿದಾನಂದ’ ಎಂಬ ಹೆಸರಿನ, ಉನ್ನತ ಇಂಗ್ಲಿಷ್ ವಿದ್ಯಾಭ್ಯಾಸ ಪಡೆದ ಬಂಗಾಳೀ ಸಾಧುಗಳೊಬ್ಬರು ಬಂದಿರುವುದು ತಿಳಿಯಿತು. ವಿಚಾರಿಸಿ ನೋಡಿದಾಗ, ಆ ಸಾಧುಗಳು ಅಲ್ಲಿನ ನವಾಬನ ಆಪ್ತಕಾರ್ಯದರ್ಶಿಯಾದ ಮನ್​ಸುಖ ರಾಮ್ ತ್ರಿಪಾಠಿ ಎಂಬವನ ಮನೆಯಲ್ಲಿರುವುದಾಗಿ ಗೊತ್ತಾಯಿತು. ಅಭೇದಾನಂದರು ಕಾತರ ವನ್ನು ತಾಳಲಾರದೆ ತಕ್ಷಣ ತ್ರಿಪಾಠಿಯ ಮನೆಗೆ ಧಾವಿಸಿದರು. ನೋಡಿದರೆ... ಅವರ ಊಹೆ ನಿಜವಾಗಿತ್ತು! ‘ಸಚ್ಚಿದಾನಂದ’ರು ಬೇರಾರೂ ಅಲ್ಲ. ಅವರ ಪ್ರಿಯ ನರೇಂದ್ರನೇ! ಅನಿರೀಕ್ಷಿತ ವಾಗಿ ನಮ್ಮ ಗುರುಭಾಯಿಯನ್ನು ಕಂಡು ಸ್ವಾಮೀಜಿಯ ಮುಖ ಬೆಳಗಿತು. ಅಭೇದಾನಂದರಿ ಗಂತೂ ಆನಂದದ ಕಣ್ಣೀರನ್ನು ತಡೆಹಿಡಿಯಲೇ ಆಗಲಿಲ್ಲ. ಅಭೇದಾನಂದರು ಅಲ್ಲಿಗೆ ಬಂದಾಗ ಸ್ವಾಮೀಜಿ, ಪಂಡಿತ ತ್ರಿಪಾಠಿಯೊಂದಿಗೆ ಅದ್ವೈತ ವೇದಾಂತದ ಬಗ್ಗೆ ಚರ್ಚಿಸುತ್ತಿದ್ದರು. ತಮ್ಮ ಗುರುಭಾಯಿಯನ್ನು ತ್ರಿಪಾಠಿಗೆ ಪರಿಚಯಿಸಿಕೊಡುತ್ತ ಸ್ವಾಮೀಜಿ ಹೇಳಿದರು, “ಇವರು ನನ್ನ ಗುರುಭಾಯಿ; ಅದ್ವೈತ ವೇದಾಂತದ ನಿಷ್ಠಾವಂತ ಪ್ರತಿಪಾದಕರು. ಈಗ ನಾವು ಚರ್ಚಿಸುತ್ತಿದ್ದ ವಿಷಯವನ್ನು ಇವರು ವಿವರಿಸುತ್ತಾರೆ.” ಅಭೇದಾನಂದರಿಗೆ ಒಂದು ಕ್ಷಣ ದಿಗ್ಭ್ರಮೆಯಾಯಿತು! ಪಾಪ, ಅವರು ಮೊದಲೇ ದಣಿದಿದ್ದಾರೆ; ಅಲ್ಲದೆ ದೀರ್ಘಕಾಲದ ಬಳಿಕ ತಮ್ಮ ಸೋದರನನ್ನು ಕಂಡು ಆನಂದಿತರಾಗಿದ್ದಾರೆ. ಇನ್ನೂ ಅವನೊಡನೆ ಒಂದು ಮಾತನ್ನೂ ಆಡಿಲ್ಲ. ಅಷ್ಟರಲ್ಲೇ ಅವರಿಗೆ ಅದ್ವೈತದ ಬಗ್ಗೆ ಚರ್ಚೆ ನಡೆಸುವಂತೆ ಹೇಳುತ್ತಿದ್ದಾರೆ ಸ್ವಾಮೀಜಿ! ಆದರೂ ತಮ್ಮ ನಾಯಕನ ಮಾತನ್ನು ಮನ್ನಿಸಿ ಚರ್ಚೆ ಪ್ರಾರಂಭಿಸಿದರು. ಪಂಡಿತ ತ್ರಿಪಾಠಿ ಪ್ರಶ್ನೆಗಳನ್ನು ಕೇಳ ಲಾರಂಭಿಸಿದ. ಅಭೇದಾನಂದರು ಎಲ್ಲಕ್ಕೂ ಸಮರ್ಪಕವಾಗಿ ಉತ್ತರಿಸಿದರು. ಕೊನೆಗೆ ಪಂಡಿತ ತ್ರಿಪಾಠಿ ಸಂತುಷ್ಟನಾಗಿ ಅವರಿಗೆ ಕೈಮುಗಿದು ನಮಸ್ಕರಿಸಿದ. ಸಂಭಾಷಣೆಯನ್ನು ಆಲಿಸುತ್ತ ಕುಳಿತಿದ್ದ ಸ್ವಾಮೀಜಿಗಂತೂ ಹಿಡಿಸಲಾರದಷ್ಟು ಸಂತೋಷ, ಹೆಮ್ಮೆ.

ಪಂಡಿತ ತ್ರಿಪಾಠಿಯ ಕೋರಿಕೆಯಂತೆ ಅಭೇದಾನಂದರು ಅವನ ಮನೆಯಲ್ಲೇ ಉಳಿದು ಕೊಂಡರು. ಗುರುಭಾಯಿಗಳಿಬ್ಬರೂ ಮೂರು-ನಾಲ್ಕು ದಿನ ಒಟ್ಟಾಗಿ ಆನಂದದಿಂದ ಕಳೆದರು. ಅಭೇದಾನಂದರು ಸ್ವಾಮೀಜಿಗೆ ಬಾರಾನಾಗೋರ್ ಮಠದ ಪರಿಸ್ಥಿತಿಯ ಬಗ್ಗೆ ತಿಳಿಸಿದರು. ಮಠದ ಆರ್ಥಿಕ ಸ್ಥಿತಿ ಹಿಂದಿನಂತೆಯೇ–ಶೋಚನೀಯವಾಗಿಯೇ–ಇತ್ತು. ಕಡೆಗೆ ಅಭೇದಾನಂದರು ದ್ವಾರಕೆಯ ಕಡೆಗೆ ಹೊರಟುನಿಂತರು. ಅವರನ್ನು ಬೀಳ್ಕೊಳ್ಳುವಾಗ ಸ್ವಾಮೀಜಿಯ ಕಣ್ಣು ಮಂಜಾ ಗಿತ್ತು. ಅಭೇದಾನಂದರಿಗೆ ಕಾಶೀಪುರದಲ್ಲಿ ತಮ್ಮ ಗುರುದೇವರು ಹಾಗೂ ಸೋದರರೊಂದಿಗೆ ಕಳೆದ ಆನಂದದ ದಿನಗಳ ನೆನಪಾಗಿ ಅವರ ಕಂಗಳಿಂದಲೂ ಅಶ್ರುಧಾರೆ ಹರಿಯಿತು. ಮನದಲ್ಲೇ ಶ್ರೀರಾಮಕೃಷ್ಣರಿಗೆ ನಮಿಸಿ, ಅಭೇದಾನಂದರು ದ್ವಾರಕೆಯ ಕಡೆಗೆ ಹೊರಟರು.

ಕಾಥೇವಾಡ ಪ್ರದೇಶದಲ್ಲಿ ಸ್ವಾಮೀಜಿ ಒಟ್ಟು ಸುಮಾರು ಒಂದು ವರ್ಷದಷ್ಟು ದೀರ್ಘಕಾಲ ವನ್ನು ಕಳೆದರು. ಜುನಾಗಢವನ್ನು ಕೇಂದ್ರವಾಗಿಟ್ಟುಕೊಂಡು, ಸುತ್ತಮುತ್ತಲಿನ ಹಲವಾರು ಸ್ಥಳಗಳನ್ನು ಅವರು ಸಂದರ್ಶಿಸಿದರು. ಪೋರ್​ಬಂದರ್ ಮೊದಲಾದ ಸ್ಥಳಗಳಲ್ಲಿ ಅವರು ಬಹಳಷ್ಟು ಕಾಲ ಉಳಿದುಕೊಂಡಿದ್ದು, ಮತ್ತೆ ಮತ್ತೆ ಜುನಾಗಢಕ್ಕೆ ಹಿಂದಿರುಗುತ್ತಿದ್ದರು. ಆದರೆ ಈ ದಿನಗಳ ಅವರ ಪರಿವ್ರಜನದ ವಿವರಗಳನ್ನು ನಿಖರವಾಗಿ ಗುರುತಿಸುವುದು ಸಾಧ್ಯವಾಗಿಲ್ಲ.

ಈಗ ಸ್ವಾಮಿ ವಿವೇಕಾನಂದರ ಪ್ರಯಾಣ ಕಥನವನ್ನು ಮುಂದುವರಿಸುವ ಮೊದಲು ಸ್ವಲ್ಪ ನಿಂತು, ಒಂದೆರಡು ಪ್ರಮುಖ ಅಂಶಗಳ ಬಗ್ಗೆ ಚಿಂತನೆ ನಡೆಸಿ ಮುನ್ನಡೆಯೋಣ.

ನಾವು ಈಗಾಗಲೇ ನೋಡಿದಂತೆ, ಮತ್ತು ಇನ್ನು ಮುಂದೆಯೂ ನೋಡಲಿರುವಂತೆ, ಸ್ವಾಮೀಜಿ ರಾಜರು-ದಿವಾನರೊಂದಿಗೆ ಆತ್ಮೀಯ ಸಂಪರ್ಕ ಸ್ನೇಹಗಳನ್ನು ಬೆಳೆಸಿಕೊಳ್ಳುತ್ತಿದ್ದರು. ಅವರೊಂದಿಗೆ ಮಾತುಕತೆಯಲ್ಲಿ ಬಹಳಷ್ಟು ಸಮಯವನ್ನು ಕಳೆಯುತ್ತಿದ್ದರು. ಅಲ್ಲದೆ, ಅವರ ಅತಿಥಿಯಾಗಿದ್ದುಕೊಂಡು ರಾಜೋಪಚಾರವನ್ನು ಸ್ವೀಕರಿಸುತ್ತಿದ್ದರು. ಇದನ್ನೆಲ್ಲ ಕಂಡು-ಅಥವಾ, ಇದನ್ನು ಮಾತ್ರ ಕಂಡು–ಅನೇಕರಿಗೆ ಸ್ವಾಮೀಜಿಯ ಈ ವರ್ತನೆ ಅರ್ಥವಾಗದಿರಬಹುದು; ಅವರ ಪ್ರಾಮಾಣಿಕತೆಯ ಬಗ್ಗೆ ಶಂಕೆಯುಂಟಾಗಬಹುದು. ಎಷ್ಟೋ ಜನ ಇದಕ್ಕಾಗಿ ಅವರನ್ನು ಕಟುವಾಗಿ ಟೀಕಿಸಿದ್ದೂ ಉಂಟು. ಇನ್ನು ಕೆಲವರು ಅವರನ್ನೇ ಈ ಬಗ್ಗೆ ಪ್ರಶ್ನಿಸಿದ್ದರು. ಸರ್ವಸಂಗ ಪರಿತ್ಯಾಗಿಗಳೆನಿಸಿಕೊಂಡ ಸಂನ್ಯಾಸಿಯಾಗಿ ರಾಜಮಹಾರಾಜರ ಸಹವಾಸದಲ್ಲಿರುವುದು, ಅವರ ಆತಿಥ್ಯವನ್ನು ಸ್ವೀಕರಿಸುವುದು ಸರಿಯೆ?

ಈ ಪ್ರಶ್ನೆಗೆ ಉತ್ತರ ಬಹುಮುಖವಾದುದು. ಸ್ವಾಮೀಜಿಯೇ ಈ ಪ್ರಶ್ನೆಗೆ ಹೀಗೆ ಉತ್ತರಿಸು ತ್ತಾರೆ–“ರಾಜರ ಮೇಲೆ ಪ್ರಭಾವ ಬೀರಿ, ಅವರ ಗಮನವನ್ನು ಧಾರ್ಮಿಕ ಜೀವನದತ್ತ ತಿರುಗಿಸಿ, ಅವರು ತಮ್ಮ ಸ್ವಧರ್ಮದಲ್ಲಿ ನಿಷ್ಠರಾಗಿರುವಂತೆ–ಎಂದರೆ ತಮ್ಮ ಪ್ರಜೆಗಳ ಒಳಿತಿಗಾಗಿ ಪ್ರಾಮಾಣಿಕವಾಗಿ ದುಡಿಯುವಂತೆ ಪ್ರೇರೇಪಿಸುವುದೇ ನನ್ನ ಉದ್ದೇಶ. ಪ್ರಜೆಗಳ ತತ್ಕಾಲದ ಒಳಿತಷ್ಟೇ ಅಲ್ಲದೆ ಅವರ ಭವಿಷ್ಯತ್ತಿನ ಕ್ಷೇಮಾಭ್ಯುದಯವೂ ಈ ರಾಜರನ್ನೇ ಅವಲಂಬಿಸಿದೆ. ಸುಧಾರಣಾ ಕ್ರಮಗಳನ್ನು ಕೈಗೊಳ್ಳಲು, ಶಿಕ್ಷಣ ಕ್ರಮವನ್ನು ಅಭಿವೃದ್ಧಿಪಡಿಸಲು ಹಾಗೂ ಜನರ ಮೂಲಭೂತ ಆವಶ್ಯಕತೆಗಳನ್ನು ಪೂರೈಸಲು ಅವರಿಂದ ಮಾತ್ರ ಸಾಧ್ಯ... ಯಾರ ಕೈಯಲ್ಲಿ ಅಧಿಕಾರ-ಹಣ ಇದೆಯೋ ಅವರ ಹೃದಯವನ್ನು ಗೆಲ್ಲಲು ಸಾಧ್ಯವಾದರೆ ನನ್ನ ಉದ್ದೇಶ ಶೀಘ್ರ ವಾಗಿ ಕೈಗೂಡುತ್ತದೆ. ಹೀಗೆ, ಒಬ್ಬ ಮಹಾರಾಜನ ಮೇಲೆ ಪ್ರಭಾವ ಬೀರುವುದರ ಮೂಲಕ ನಾನು ಪರೋಕ್ಷವಾಗಿ ಸಾವಿರಾರು ಜನರಿಗೆ ನೆರವಾಗಬಹುದು.”

ಮೊಟ್ಟಮೊದಲನೆಯದಾಗಿ, ಸ್ವಾಮೀಜಿ ಸಂನ್ಯಾಸವ್ರತವನ್ನು ಕೈಗೊಂಡು ಪರಿವ್ರಾಜಕರಾಗಿ ಭಾರತದಾದ್ಯಂತ ಸುತ್ತಾಡಿದುದರ ಹಿನ್ನೆಲೆಯನ್ನು ಗಮನಿಸಬೇಕು ನಾವು. ಜೀವನದ ಯಾವುದೇ ರಂಗವನ್ನು ಪ್ರವೇಶಿಸಿದ್ದರೂ ಅದ್ವಿತೀಯನಾಗಿ ರಾರಾಜಿಸಬಹುದಾಗಿದ್ದ, ಸಮಸ್ತ ಸದ್ಗುಣ ಸಂಪನ್ನನೂ ಅಪ್ರತಿಮ ಪ್ರತಿಭಾವಂತನೂ ಆಗಿದ್ದ ನರೇಂದ್ರ, ಸಕಲ ಲೌಕಿಕ ಭೋಗವನ್ನೂ ತಿರಸ್ಕರಿಸಿ ಸಂನ್ಯಾಸಿಯಾದದ್ದೇ ನಿರ್ವಿಕಲ್ಪಸಮಾಧಿಯ ಆನಂದವನ್ನು ಹೊಂದಬೇಕೆಂಬ ಅತ್ಯುನ್ನತ ಆಕಾಂಕ್ಷೆಯಿಂದ. ಆದರೆ ಸ್ವಂತ ಮುಕ್ತಿಯನ್ನರಸುವುದೂ ಸ್ವಾರ್ಥತೆಯೇ ಎಂಬುದ ನ್ನರಿತು, ಮಹಾ ವಟವೃಕ್ಷದಂತೆ ಸಹಸ್ರಾರು ದೀನಾರ್ತರಿಗೆ ಆಶ್ರಯವನ್ನಿತ್ತು ಸಂತೈಸುವುದೇ ತಮ್ಮ ಗುರಿಯೆಂಬುದನ್ನು ಕಂಡುಕೊಂಡ ಸ್ವಾಮೀಜಿ. ಅದನ್ನು ಸಾಧಿಸುವ ಮಾರ್ಗವನ್ನರಿಸಿಯೇ ಪರಿವ್ರಾಜಕ ಜೀವನವನ್ನು ಕೈಗೊಂಡದ್ದು, ಆದರೆ ಈ ಪರಿವ್ರಾಜಕ ಜೀವನದ ಉಪವಾಸ-ವನ ವಾಸಗಳ ನಡುವೆಯೂ, ತನ್ನ ಜೀವನೋದ್ದೇಶವನ್ನಿನ್ನೂ ಸಾಧಿಸಲಾಗಿಲ್ಲವೆಂಬ ಮಹಾಚಿಂತೆಯ ನಡುವೆಯೂ, ಅವರ ಮನಸ್ಸು ಯೋಗ್ಯ ಪರಿಸರವೊಂದನ್ನು ಕಂಡೊಡನೆಯೇ ಧ್ಯಾನಾನಂದದಲ್ಲಿ ಮಗ್ನವಾಗಿ ತನ್ನ ಸಹಜ ನೆಲೆಗೇರಲು ತವಕಿಸುತ್ತಿತ್ತು! ಇಂತಹ ಜೀವನ್ಮುಕ್ತರಾದ ವಿವೇಕಾನಂದ ರಲ್ಲಿ ಸ್ವಾರ್ಥದ ಲವಲೇಶವಾದರೂ ಇದ್ದಿರಲು ಸಾಧ್ಯವೆ?

ಸ್ವಾಮೀಜಿ ರಾಜಮಹಾರಾಜರ ಅತಿಥಿಯಾಗಿ ಉಳಿದುಕೊಳ್ಳುತ್ತಿದ್ದುದು ಸತ್ಯವೇ. ಆದರೆ ಅವರು ಪಾಳುಬಿದ್ದ ದೇವಾಲಯದಲ್ಲೋ ಊರಾಚೆಯ ಆಲದಮರದ ನೆರಳಿನಲ್ಲೋ ಉರುಳಿ ಕೊಳ್ಳುತ್ತಿದ್ದುದು ಇನ್ನೂ ಹೆಚ್ಚು ಸತ್ಯ. ಇಂದು ಅವರಿಗೆ ಷಡ್ರಸೋಪೇತ ಭೋಜನದ ಆತಿಥ್ಯ ವಾದರೆ, ನಾಳೆ ಅಂತ್ಯಜನೊಬ್ಬನ ಗುಡಿಸಲಿನಲ್ಲಿ ಒಣರೊಟ್ಟಿಯ ಔತಣ! ತಮ್ಮ ಅತಿಥಿಯಾಗಿ ಇದ್ದುಕೊಂಡು ಜೀವನಪರ್ಯಂತ ನೆಲಸಬೇಕೆಂದು ಅವರನ್ನು ಪ್ರಾರ್ಥಿಸಿಕೊಂಡವರು ಅದೆಷ್ಟು ಜನರೋ! ಅವರಿಗೆ ಪರಿಪರಿಯಾದ ಉಡುಗೊರೆಗಳನ್ನೂ ಹಣವನ್ನೂ ನೀಡಲು ಮುಂದಾದವ ರೆಷ್ಟು ಜನರೋ! ಆದರೆ ಯಾವುದರಿಂದಲೂ ವಿಚಲಿತರಾಗದೆ, ಒಬ್ಬ ಭಿಕಾರಿ ಸಂನ್ಯಾಸಿ ಯಂತೆಯೇ ಸಾಗಿದರು ಸ್ವಾಮೀಜಿ.

ಗಿರಿನಾರ್​ನಿಂದ ಜುನಾಗಢಕ್ಕೆ ಹಿಂದಿರುಗಿದ ಸ್ವಾಮೀಜಿ, ಮತ್ತೊಮ್ಮೆ ಏಕಾಂಗಿಯಾಗಿ ಸಂಚರಿಸುತ್ತ ಮುನ್ನಡೆಯಲು ನಿರ್ಧರಿಸಿದರು. ದಿವಾನ ಹರಿದಾಸ್ ದೇಸಾಯಿ ನೆರೆರಾಜ್ಯಗಳ ಅಧಿಕಾರಿಗಳಿಗೆ ಬರೆದುಕೊಟ್ಟ ಪರಿಚಯ ಪತ್ರಗಳನ್ನು ಪಡೆದುಕೊಂಡು, ತಮ್ಮ ಮಿತ್ರರಿಂದ ಬೀಳ್ಕೊಂಡು ಕಚ್ ಪ್ರಾಂತ್ಯದ ರಾಜಧಾನಿಯಾದ ಭೂಜ್​ಗೆ ಬಂದರು. ಇಲ್ಲಿಯೂ ಅವರು ರಾಜ್ಯದ ದಿವಾನನ ಅತಿಥಿಯಾಗಿ ಉಳಿದುಕೊಂಡರು. ಕೈಗಾರಿಕೆಗಳು, ಕೃಷಿ, ದೇಶದ ಆರ್ಥಿಕ ಸಮಸ್ಯೆಗಳು ಹಾಗೂ ಜನಸಾಮಾನ್ಯರ ಶಿಕ್ಷಣ ಮೊದಲಾದ ವಿಚಾರಗಳ ಬಗ್ಗೆ ದಿವಾನನೊಂದಿಗೆ ದೀರ್ಘ ಸಂಭಾಷಣೆ ನಡೆಸಿದರು. (ಸ್ವಾಮೀಜಿಯ ಶಿಷ್ಯನೊಬ್ಬ ಹಲವಾರು ವರ್ಷಗಳ ಬಳಿಕ ಈ ದಿವಾನನನ್ನು ಸಂಪರ್ಕಿಸಿದಾಗ ಈತ ಸ್ವಾಮೀಜಿಯ ಅತಿಮಾನುಷ ಬುದ್ಧಿಮತ್ತೆ, ಶೋಭನೀಯ ವ್ಯಕ್ತಿತ್ವ ಹಾಗೂ ಕ್ಲಿಷ್ಟ ವಿಚಾರಗಳನ್ನು ಸುಲಲಿತವಾಗಿ ಮುಂದಿಡುವ ಅವರ ಶಕ್ತಿಯನ್ನು ಸ್ಮರಿಸುತ್ತಾನೆ.) ಸ್ವಾಮೀಜಿಯನ್ನು ಅವನು ಕಚ್​ನ ಮಹಾರಾಜನಿಗೆ ಪರಿಚಯಿಸಿದ. ಮಹಾ ರಾಜನೂ ಸ್ವಾಮೀಜಿಯಿಂದ ಬಹಳವಾಗಿ ಪ್ರಭಾವಿತನಾದ.

ಭೂಜ್​ನಲ್ಲೂ ಸ್ವಾಮೀಜಿ, ಎಂದಿನಂತೆ, ಸುತ್ತಮುತ್ತಲಿನ ಪ್ರೇಕ್ಷಣೀಯ ಸ್ಥಳಗಳನ್ನೆಲ್ಲ ಸಂದರ್ಶಿಸಿದರು. ಯಾತ್ರಾರ್ಥಿಗಳು ಹಾಗೂ ಸಾಧುಸಂನ್ಯಾಸಿಗಳೊಂದಿಗೆ ಬೆರೆತು ತಮ್ಮ ಜ್ಞಾನ- ಅನುಭವಗಳನ್ನು ಮತ್ತಷ್ಟು ವಿಸ್ತರಿಸಿಕೊಳ್ಳುವ ಪ್ರಯತ್ನ ಮಾಡಿದರು. ಹೀಗೆ ಕೆಲವು ದಿನಗಳಾದ ಮೇಲೆ ಭೂಜ್​ನಿಂದ ಜುನಾಗಢಕ್ಕೆ ಹಿಂದಿರುಗಿ, ಕೆಲ ಕಾಲ ವಿಶ್ರಾಂತಿ ಪಡೆದು ಅರಬ್ಬೀ ಸಮುದ್ರ ತೀರದಲ್ಲಿರುವ ವೆರಾವಲ್ ಹಾಗೂ ಸೋಮನಾಥ ಕ್ಷೇತ್ರಗಳ ಸಂದರ್ಶನಕ್ಕೆ ಹೊರಟರು. ವೆರಾ ವಲ್ ಪುರಾತನ ಅವಶೇಷಗಳಿಗಾಗಿ ಪ್ರಸಿದ್ಧವಾಗಿದೆ. ಸೋಮನಾಥ ಕ್ಷೇತ್ರದ ಸುಪ್ರಸಿದ್ಧ ದೇವಾಲಯವನ್ನು ಮುಸಲ್ಮಾನ ಆಕ್ರಮಣಕಾರರು ಮೂರು ಸಲ ನೆಲಸಮ ಮಾಡಿದ್ದರು; ಮೂರು ಸಲವೂ ಅದು ಸ್ವಾಭಿಮಾನೀ ಭಕ್ತರಿಂದ ತಲೆಯೆತ್ತಿ ನಿಂತಿತು. ಈ ಮಹಾದೇವಾಲಯದ ಮುಂದೆ ನಿಂತು ಭಾರತದ ಗತವೈಭವವನ್ನು ಮೆಲುಕುಹಾಕುತ್ತ ಮೈಮರೆತರು ಸ್ವಾಮೀಜಿ.

ಸೋಮನಾಥ ಕ್ಷೇತ್ರದಲ್ಲಿ ಸ್ವಾಮೀಜಿ ಮತ್ತೆ ಕಚ್​ನ ಮಹಾರಾಜನನ್ನು ಭೇಟಿಯಾದರು. ಸ್ವಾಮೀಜಿಯ ಆಯಸ್ಕಾಂತೀಯ ವ್ಯಕ್ತಿತ್ವದ ಸೆಳೆತಕ್ಕೆ ಸಿಲುಕಿದ ಮಹಾರಾಜ, ಅವರ ಪ್ರಕಾಂಡ ಪಾಂಡಿತ್ಯವನ್ನು ಕಂಡು ಒಮ್ಮೆ ನಿಬ್ಬೆರಗಾಗಿ ಉದ್ಗರಿಸಿದ, “ಒಮ್ಮೆಗೇ ಹಲವಾರು ಗ್ರಂಥಗಳನ್ನು ಓದಿಬಿಟ್ಟರೆ ಕಣ್ಣಿಗೆ ಕತ್ತಲು ಕವಿಯುವಂತೆ, ನಿಮ್ಮ ವಾಕ್ಪ್ರವಾಹವನ್ನು ಕೇಳಿದಾಗಲೂ ತಲೆ ಸುತ್ತು ವಂತೆ ಭಾಸವಾಗುತ್ತದೆ! ಸ್ವಾಮೀಜಿ, ನೀವು ಇಷ್ಟೆಲ್ಲ ಪ್ರತಿಭೆಯನ್ನು ಹೇಗೆ ಉಪಯೋಗಿಸಿ ಕೊಳ್ಳುತ್ತೀರೊ! ಖಂಡಿತವಾಗಿ ನೀವು ಅದ್ಭುತ ಕಾರ್ಯಗಳನ್ನು ಸಾಧಿಸುವವರೆಗೂ ವಿರಮಿಸಲಾರಿರಿ.”

ವೆರಾವೆಲ್ ಹಾಗೂ ಸೋಮನಾಥವನ್ನು ಸಂದರ್ಶಿಸಿ ಸ್ವಾಮೀಜಿ ಜುನಾಗಢಕ್ಕೆ ಹಿಂದಿರುಗಿ ದರು. ಇಲ್ಲಿ ಕೆಲದಿನಗಳನ್ನು ಕಳೆದು ದಿವಾನನು ಕೊಟ್ಟ ಪರಿಚಯ ಪತ್ರದೊಂದಿಗೆ ಪೋರ್ ಬಂದರ್​ಗೆ ಹೊರಟರು. ಸುದಾಮಪುರಿ ಎಂದು ಭಾಗವತ ಪುರಾಣದಲ್ಲಿ ಉಲ್ಲೇಖಿಸಲ್ಪಟ್ಟಿರುವ ಈ ನಗರದಲ್ಲಿರುವ ಪುರಾತನ ಸುದಾಮ ದೇವಾಲಯಕ್ಕೆ ಭೇಟಿ ನೀಡಿದರು. ಅವರನ್ನು ಇಲ್ಲಿನ ದಿವಾನನಾದ ಪಂಡಿತ ಶಂಕರ ಪಾಂಡುರಂಗ ಹಾರ್ದಿಕವಾಗಿ ಸ್ವಾಗತಿಸಿದ. ಈತ ಸಮರ್ಥ ಆಡಳಿತಗಾರನೆಂದೂ ಮಹಾ ವೈದಿಕವಿದ್ವಾಂಸನೆಂದೂ ಸುಪ್ರಸಿದ್ಧನಾಗಿದ್ದು, ಆ ಸಮಯದಲ್ಲಿ ಅಥರ್ವಣ ವೇದವನ್ನು ಭಾಷಾಂತರಿಸಿ ಪ್ರಕಟಿಸುವ ಕಾರ್ಯದಲ್ಲಿ ತೊಡಗಿದ್ದ. ಇವನು ಐರೋಪ್ಯ ರಾಷ್ಟ್ರಗಳಿಗೂ ಭೇಟಿ ನೀಡಿ ಅಪಾರಜ್ಞಾನವನ್ನು ಸಂಪಾದಿಸಿದ್ದ. ಈತ ಸ್ವಾಮೀಜಿಯೊಂದಿಗೆ ಒಂದೆರಡು ಮಾತನಾಡುತ್ತಿದ್ದಂತೆ ಅವರ ವಿದ್ವತ್ತನ್ನು ಮನಗಂಡು ಬೆರಗಾದ. ವೇದಗಳ ಭಾಷಾಂತರ ಕಾರ್ಯದಲ್ಲಿ ತನಗೆ ಎದುರಾಗಿದ್ದ ಹಲವಾರು ಕ್ಲಿಷ್ಟ ಭಾಗಗಳ ಬಗ್ಗೆ ಅವರನ್ನು ಪ್ರಶ್ನಿಸಿದ. ಸ್ವಾಮೀಜಿ ಆ ಪ್ರಶ್ನೆಗಳಿಗೆ ತಮ್ಮ ಸಹಜ-ಸರಳ ಶೈಲಿಯಲ್ಲಿ ನಿರರ್ಗಳವಾಗಿ ಉತ್ತರಿಸಿ ದರು. ದಿವಾನನ ಕೋರಿಕೆಯಂತೆ ಅವನ ಈ ಕಾರ್ಯದಲ್ಲಿ ನೆರವಾಗಲು ಸ್ವಾಮೀಜಿ ಕೆಲವು ತಿಂಗಳು ಪೋರ್​ಬಂದರ್​ನಲ್ಲೇ ಉಳಿದುಕೊಂಡರು. ಇದರಿಂದ ಅವರಿಗೂ ವೈಯಕ್ತಿಕವಾಗಿ ಲಾಭವಾಯಿತು. ಇಬ್ಬರೂ ಕೂಡಿ ಕೆಲಸ ಮಾಡುವಾಗ ಸ್ವಾಮೀಜಿಯ ಮನಸ್ಸು ವೇದಗಳಲ್ಲಿನ ಉನ್ನತ ಆಧ್ಯಾತ್ಮಿಕ ಭಾವನೆಗಳಲ್ಲಿ ಹೆಚ್ಚು ಹೆಚ್ಚು ಆಳವಾಗಿ ಮುಳುಗತೊಡಗಿತು. ಇದರಿಂದಾಗಿ ವೈದಿಕ ತತ್ತ್ವಗಳ ಘನತೆ-ಮಹತ್ವಗಳನ್ನು ಅವರು ಇನ್ನೂ ಸ್ಪಷ್ಟವಾಗಿ ಮನಗಾಣುವಂತಾಯಿತು.

ಇದೇ ಅವಧಿಯಲ್ಲಿ ಅವರು ಪಂಡಿತ ಪಾಂಡುರಂಗನ ನೆರವಿನಿಂದ ಪತಂಜಲಿಯ ಮಹಾ ಭಾಷ್ಯದ ಅಧ್ಯಯನವನ್ನು ಪೂರೈಸಿದರು. ಆತನ ಬಳಿ ಹಲವಾರು ಉತ್ತಮ ಗ್ರಂಥಗಳ ಸಂಗ್ರಹವೇ ಇದ್ದು, ಅದನ್ನು ಯಥೇಚ್ಛವಾಗಿ ಬಳಸಿಕೊಳ್ಳುವಂತೆ ಅವನು ಪ್ರಾರ್ಥಿಸಿಕೊಂಡಿದ್ದ. ಅದರಂತೆ ಸ್ವಾಮೀಜಿ ಅನೇಕ ಗ್ರಂಥಗಳನ್ನು ಓದಿ ಮುಗಿಸಿದರು.

ಸ್ವಾಮೀಜಿಯ ವಿಚಾರಧಾರೆಯ ವೈಶಾಲ್ಯ ಹಾಗೂ ಸ್ವಂತಿಕೆಯನ್ನು ಅರಿತುಕೊಂಡ ಪಂಡಿತ ಪಾಂಡುರಂಗ ಹೇಳುತ್ತಾನೆ, “ಸ್ವಾಮೀಜಿ, ನನಗನ್ನಿಸುತ್ತದೆ, ನೀವು ಈ ದೇಶದಲ್ಲಿ ಹೆಚ್ಚಿನದೇ ನನ್ನೂ ಸಾಧಿಸಲಾರಿರಿ ಎಂದು. ಏಕೆಂದರೆ, ಇಲ್ಲಿ ನಿಮ್ಮನ್ನು ಅರ್ಥಮಾಡಿಕೊಳ್ಳಬಲ್ಲವರು ಬಹಳ ಕಡಿಮೆ. ನೀವು ಪಾಶ್ಯಾತ್ಯ ರಾಷ್ಟ್ರಗಳಿಗೆ ಹೋಗಬೇಕು. ಅಲ್ಲಿನ ಜನ ನಿಮ್ಮ ಘನತೆಯನ್ನು ಅರ್ಥಮಾಡಿಕೊಳ್ಳುತ್ತಾರೆ. ಅಲ್ಲಿ ನೀವು ಸನಾತನ ಧರ್ಮವನ್ನು ಪ್ರಚಾರ ಮಾಡುವುದರಿಂದ ಖಂಡಿತವಾಗಿ ಪಾಶ್ಚಾತ್ಯರಿಗೂ ಹೊಸ ಬೆಳಕನ್ನು ನೀಡಬಲ್ಲಿರಿ.” ಈ ಅಭಿಪ್ರಾಯ ಸ್ವಾಮೀಜಿಗೆ ಒಪ್ಪಿಗೆಯಾಗುವಂತಿತ್ತು. ಅವರು ಇದೇ ರೀತಿಯ ಭಾವನೆಯನ್ನು, ಆದರೆ ಅಸ್ಪಷ್ಟವಾಗಿ, ಈ ಹಿಂದೆ ಜುನಾಗಢದಲ್ಲಿ ಸಿ. ಹೆಚ್. ಪಾಂಡ್ಯನ ಮುಂದೆ ವ್ಯಕ್ತಪಡಿಸಿದ್ದರು. ಈಗ ಸ್ವಾಮೀಜಿ ಪಂಡಿತನಿಗೆ ಹೇಳಿದರು, “ಆಗಲಿ; ನಾನೊಬ್ಬ ಸಂನ್ಯಾಸಿ. ನಾನು ಪಾಶ್ಚಾತ್ಯ ರಾಷ್ಟ್ರಗಳಿಗೆ ಹೋಗ ಬೇಕಾಗಿ ಬಂದಲ್ಲಿ ಖಂಡಿತವಾಗಿಯೂ ಹೋಗುತ್ತೇನೆ.” ಆ ದಿನಗಳಲ್ಲಿ ಸಮುದ್ರಯಾನವು ಸಂಪ್ರದಾಯಸ್ಥ ಹಿಂದೂಗಳಿಗೆ, ಅದರಲ್ಲೂ ಸಂನ್ಯಾಸಿಗಳಿಗೆ, ವರ್ಜಿತವಾಗಿತ್ತೆಂಬುದನ್ನು ನಾವು ನೆನಪಿಸಿಕೊಳ್ಳಬೇಕು.

ಜುನಾಗಢ ಅಥವಾ ಪೋರ್​ಬಂದರಿನಲ್ಲೇ ಸ್ವಾಮೀಜಿ, ಅಮೆರಿಕದಲ್ಲಿ ಮರುವರ್ಷ ನಡೆಯ ಲ್ಲಿದ್ದ ‘ಧರ್ಮ ಸಮ್ಮೇಳನ’ವೊಂದರ ಬಗ್ಗೆ ಮೊತ್ತಮೊದಲು ಕೇಳಿದ್ದು. ಅದರಲ್ಲಿ ಅವರು ಭಾಗವಹಿಸಬೇಕೆಂದು ಕೆಲವರು ಸಲಹೆ ಮಾಡಿದರು. ಸ್ವಾಮೀಜಿ ಕೂಡ ಸಮ್ಮೇಳನದ ಬಗ್ಗೆ ವಿವರ ಗಳನ್ನು ತಿಳಿದುಕೊಳ್ಳಲು ಉತ್ಸುಕರಾಗಿದ್ದರು.

ಈ ಸಂದರ್ಭದಲ್ಲಿ ದಿವಾನನು ಸ್ವಾಮೀಜಿಗೆ ಫ್ರೆಂಚ್ ಭಾಷೆಯನ್ನು ಕಲಿಯುವಂತೆ ಸಲಹೆ ಮಾಡಿದ. ಐರೋಪ್ಯ ರಾಷ್ಟ್ರಗಳಲ್ಲಿ, ಮುಖ್ಯವಾಗಿ ಫ್ರಾನ್ಸ್​ನಲ್ಲಿ ಸನಾತನಧರ್ಮದ ಪ್ರಸಾರ ಕಾರ್ಯವನ್ನು ಕೈಗೊಳ್ಳುವ ಸಂದರ್ಭವೊದಗಿದಾಗ ಅದರಿಂದ ಬಹಳ ಸಹಾಯವಾಗುತ್ತದೆ ಎಂದು ಅವನು ಅಭಿಪ್ರಾಯಪಟ್ಟ. ಸ್ವತಃ ಆತನಿಗೂ ಫ್ರೆಂಚ್, ಜರ್ಮನ್ ಹಾಗೂ ಇನ್ನೂ ಕೆಲವು ಭಾಷೆಗಳ ಪರಿಚಯವಿತ್ತು. ಸ್ವಾಮೀಜಿ ಈ ಸಲಹೆಯನ್ನು ಒಪ್ಪಿ ಅವನಿಂದ ಫ್ರೆಂಚ್ ಭಾಷೆಯನ್ನು ಅಭ್ಯಾಸ ಮಾಡಿದರು.

ಆಲಂಬಜಾರ್ ಮಠ\footnote{* ೧೮೯೧ರ ನವೆಂಬರ್​ನಲ್ಲಿ ಮಠವನ್ನು ಬಾರಾನಾಗೋರ್​ನಿಂದ ಕಲ್ಕತ್ತದ ಬಳಿಯಿರುವ ಆಲಂಬಜಾರ್ ಎಂಬಲ್ಲಿನ ಹೆಚ್ಚು ವಿಶಾಲವಾದ ಕಟ್ಟಡವೊಂದಕ್ಕೆ ಸ್ಥಳಾಂತರಿಸಲಾಗಿತ್ತು.}ದಲ್ಲಿದ್ದ ರಾಮಕೃಷ್ಣ ಸಂಘದ ಸಂನ್ಯಾಸಿಗಳಿಗೆ ಒಂದು ದಿನ ಪೋರ್ ಬಂದರ್​ನಿಂದ ನಾಲ್ಕು ಪುಟಗಳ ಪತ್ರವೊಂದು ತಲುಪಿತು. ಪತ್ರವನ್ನು ತೆರೆದು ನೋಡಿದ ಗುರುಭಾಯಿಗಳಿಗೆ, ಅದರ ಭಾಷೆಯೂ ಅರ್ಥವಾಗಲಿಲ್ಲ, ಅದನ್ನು ಬರೆದವರಾರೆಂಬುದೂ ತಿಳಿಯಲಿಲ್ಲ! ಆದರೆ ಅದನ್ನು ಸೂಕ್ಷ್ಮವಾಗಿ ಗಮನಿಸಿದ ಮೇಲೆ, ಅಲ್ಪಸ್ವಲ್ಪ ಫ್ರೆಂಚ್ ತಿಳಿದಿದ್ದ ರಾಮಕೃಷ್ಣಾನಂದರೂ ಶಾರದಾನಂದರೂ ಪತ್ರದ ಭಾಷೆ ಫ್ರೆಂಚ್ ಎಂದೂ, ಅದನ್ನು ಬರೆದವರು ತಮ್ಮ ನರೇನ್​ಭಾಯಿ ಎಂದೂ ಘೋಷಿಸಿದರು. ಬಳಿಕ ಅದನ್ನು, ಫ್ರೆಂಚ್ ಭಾಷೆಯನ್ನು ಬಲ್ಲವ ರಾಗಿದ್ದ ಶ್ರೀ ಅಘೋರನಾಥ ಚಟರ್ಜಿ(ಶ್ರೀಮತಿ ಸರೋಜಿನಿ ನಾಯ್ಡುರವರ ತಂದೆ)ಯವರ ಬಳಿಗೆ ತಂದರು. ಅವರು ಪತ್ರವನ್ನು ಓದಿ ಅರ್ಥವನ್ನು ವಿವರಿಸಿದರು.

ಸ್ವಾಮೀಜಿ ಕಾಥೇವಾಡ ಪ್ರದೇಶದಲ್ಲಿ ಸುತ್ತಾಡುತ್ತಿದ್ದ ಈ ಸಮಯದಲ್ಲಿ ಅವರ ಮನಸ್ಸು ಅತ್ಯಂತ ಪ್ರಕ್ಷುಬ್ಧಗೊಂಡು, ಪಂಜರದ ಸಿಂಹದಂತೆ ಚಡಪಡಿಸುತ್ತಿತ್ತು. ‘ನರೇಂದ್ರ ಜಗತ್ತನ್ನೇ ಅಲುಗಾಡಿಸಬಲ್ಲ’ ಎಂಬ ಶ್ರೀರಾಮಕೃಷ್ಣನ ಮಾತಿನ ಸತ್ಯ ಅವರಿಗೆ ನಿಧಾನವಾಗಿ ಅರಿವಾಗತೊಡ ಗಿತ್ತು. ಅವರು ಎಲ್ಲೆಲ್ಲಿ ಹೋದರೂ ಅವರನ್ನು ಸಂಧಿಸಿದ ರಾಜರು-ಪಂಡಿರೆಲ್ಲ ಅವರಲ್ಲಿ ‘ದೇಶ ಕ್ಕಾಗಿ ನಾನೇನು ಮಾಡಲಿ?’ ಎಂಬ ತೀವ್ರ ಚಡಪಡಿಕೆಯನ್ನು ಕಾಣುತ್ತಿದ್ದರು. ಈಗ ಅವರ ಮನ ವನ್ನು ಆಕ್ರಮಿಸಿದ್ದ ಒಂದೇ ಒಂದು ಆಲೋಚನೆಯೆಂದರೆ “ಭಾರತದ ಪುನರುದ್ಧಾರ.” ಸಂಪ್ರ ದಾಯಗಳ ಹೆಸರಿನ ದುರಾಚಾರಗಳಿಂದಾದ ಹಾನಿಯನ್ನು ಅವರು ಮನಗಾಣುತ್ತಿದ್ದಾರೆ; ಜೊತೆಗೆ ಸಮಾಜ ಸುಧಾರಕರೆನ್ನಿಸಿಕೊಂಡವರ ಇತಿಮಿತಿಯನ್ನೂ ಕಾಣುತ್ತಿದ್ದಾರೆ. ಎಲ್ಲೆಲ್ಲಿಯೂ ಕೇವಲ ಕ್ಷುದ್ರ ದ್ವೇಷಾಸೂಯೆಗಳನ್ನು, ಸಾಮರಸ್ಯದ ಅಭಾವವನ್ನು ನೋಡುತ್ತಿದ್ದಾರೆ. ತಾವು ಬೋಧಿಸುವ ತತ್ತ್ವಗಳನ್ನು ತಮ್ಮ ಜೀವನದಲ್ಲೇ ಅನುಷ್ಠಾನಕ್ಕೆ ತರುವ ಯೋಗ್ಯತೆಯಿಲ್ಲದೆ, ಕೇವಲ ವೇದಿಕೆಯ ಮೇಲೆ ನಿಂತು ಭಾಷಣ ಮಾಡಿ ಗಲಭೆಯೆಬ್ಬಿಸುವ ನಾಯಕರೆನ್ನಿಸಿಕೊಂಡವರ ಮೂರ್ಖವರ್ತನೆಯನ್ನು, ಪಾಶ್ಚಾತ್ಯ ನಾಗರಿಕತೆಯ ಥಳುಕಿಗೆ ಮರುಳಾಗಿ ಸನಾತನ ಹಿಂದೂ ಸಂಸ್ಕೃತಿಯನ್ನೇ ಅನಾಗರಿಕವೆಂದು ತುಚ್ಛೀಕರಿಸುತ್ತಿರುವ ಇವರ ದುರ್ವರ್ತನೆಯನ್ನು, ಇವರ ಕೈಗೆ ಸಿಲುಕಿದ ಭವ್ಯ ಭಾರತವು ಸರ್ವನಾಶದೆಡೆಗೆ ವೇಗವಾಗಿ ಧಾವಿಸುತ್ತಿರುವುದನ್ನು ಕಂಡು ಮಮ್ಮಲ ಮರುಗಿದರು ಸ್ವಾಮೀಜಿ. ಇದೀಗ ನವನಿರ್ಮಾಣದ ಕಾಲವೊಂದು ಸನ್ನಿಹಿತವಾಗಿದೆ ಎಂಬುದನ್ನು ಹೋದಹೋದಲ್ಲೆಲ್ಲ ತಮ್ಮ ಭಕ್ತರಿಗೆ, ವಿಶ್ವಾಸಿಗಳಿಗೆ ಮನಗಾಣಿಸಿದರು. ಅದ ರಲ್ಲೂ ವಿಶೇಷವಾಗಿ ಭಾರತದ ರಾಜಮಹಾರಾಜರಿಗೆ, ಮಂತ್ರಿಗಳಿಗೆ, ಅಧಿಕಾರಿಗಳಿಗೆ ತಮ್ಮ ಈ ಸಂದೇಶವನ್ನು ಮನಮುಟ್ಟುವಂತೆ ಬೋಧಿಸಲು ಯತ್ನಿಸಿದರು. ಆತ್ಮಸಾಕ್ಷಾತ್ಕಾರದ ಕುಲುಮೆ ಯಿಂದ ಒಡಮೂಡಿ ನಿಂತ ಆ ದಿವ್ಯಮಾನುಷಮೂರ್ತಿಯ ಅಪ್ರತಿಹತ ಆಕರ್ಷಣೆಗೆ ಸಿಲುಕಿ, ಅವರ ವ್ಯಕ್ತಿತ್ವದ ಹಿರಿಮೆಯನ್ನು ಮನಗಂಡ ಈ ಜನರೆಲ್ಲರೂ ಅವರ ಮಾತುಗಳನ್ನು ನಿಬ್ಬೆರಗಾಗಿ ಆಲಿಸಿದರು. ಸಮಸ್ತ ನಾಗರಿಕ ಜಗತ್ತಿಗೆ ಭಾರತದ ನಿಜ ಅಂತಸ್ಸತ್ವ ಎಂತಹದೆನ್ನುವುದು ಗೊತ್ತಾಗಬೇಕಾದರೆ, ಸನಾತನ ಧರ್ಮದ ಘನತೆ-ವೈಭವಗಳನ್ನು ಮೊಟ್ಟಮೊದಲಿಗೆ ಪಾಶ್ಚಾತ್ಯ ರಾಷ್ಟ್ರಗಳಲ್ಲಿ ಮೊಳಗಬೇಕು ಎಂದು ಸ್ವಾಮೀಜಿ ತೀವ್ರವಾಗಿ ಭಾವಿಸತೊಡಗಿದ್ದರು. ವೇದಗಳನ್ನು ಹೆಚ್ಚೆಚ್ಚು ಅಧ್ಯಯನ ಮಾಡಿದಂತೆ, ಆರ್ಯ ಮಹರ್ಷಿಗಳಿಂದ ಸಾರಲ್ಪಟ್ಟ ಮಹಾತತ್ತ್ವಗಳ ಬಗ್ಗೆ ಹೆಚ್ಚೆಚ್ಚು ಆಲೋಚಿಸಿದಂತೆ ಅವರಿಗೆ ಸ್ಪಷ್ಟವಾಯಿತು–ಭಾರತವು ನಿಜಕ್ಕೂ ಸಕಲ ಧರ್ಮಗಳ ತವರೂರು, ಆಧ್ಯಾತ್ಮಿಕತೆಯ ಉಗಮಸ್ಥಾನ, ನಾಗರಿಕತೆಯ ತೊಟ್ಟಿಲು–ಎಂದು.

ಸ್ವಾಮೀಜಿಯ ಮನಸ್ಥಿತಿಯನ್ನು ನಿರೂಪಿಸುವ ಘಟನೆಯೊಂದು ಪೋರ್ ಬಂದರಿನಲ್ಲಿ ನಡೆ ಯಿತು. ಕಾಥೇವಾಡ ಪ್ರದೇಶದ ಯಾತ್ರಾಸ್ಥಳಗಳನ್ನು ಸಂದರ್ಶಿಸುತ್ತ ಸ್ವಾಮಿ ತ್ರಿಗುಣಾತೀತಾ ನಂದರು ಕೆಲವು ಪರಿವ್ರಾಜಕ ಸಂನ್ಯಾಸಿಗಳೊಂದಿಗೆ ಪೋರ್​ಬಂದರಿಗೆ ಬಂದಿದ್ದರು. ಈಗ ಪಾಕಿಸ್ತಾನದಲ್ಲಿರುವ ಹಿಂಗಲಾಜ್ ಎಂಬಲ್ಲಿಗೆ ಯಾತ್ರೆ ಹೋಗಬೇಕೆಂದು ಅವರು ನಿರ್ಧರಿಸಿ ದ್ದರು. ಆದರೆ ಅದು ಹಲವು ನೂರು ಮೈಲಿಗಳ ದುರ್ಭರ ಪ್ರಯಾಣ. ಈಗಾಗಲೇ ನಡೆದು ನಡೆದು ಅವರೆಲ್ಲ ಹಣ್ಣಾಗಿ ಬಿಟ್ಟಿದ್ದರು. ಆದ್ದರಿಂದ ಸ್ಟೀಮರಿನಲ್ಲಿ ಕರಾಚಿಯವರೆಗೆ ಪ್ರಯಾಣ ಮಾಡಿ, ಅಲ್ಲಿಂದ ಮುಂದೆ ಒಂಟೆಯ ಮೇಲೆ ಹಿಂಗಲಾಜಿಗೆ ಹೋಗವುದೆಂದು ಈ ಪರಿವ್ರಾಜಕ ಸಂನ್ಯಾಸಿಗಳೆಲ್ಲ ತೀರ್ಮಾನಿಸಿದರು. ಆದರೆ ದಾರಿಖರ್ಚಿಗೆ ಅವರ ಬಳಿ ದುಡ್ಡಿರಲಿಲ್ಲ. ಈಗೇನು ಮಾಡುವುದೆಂದು ತಿಳಿಯದೆ ಚಿಂತಿಸುತ್ತ ಕುಳಿತರು. ಆಗ ಅವರಲ್ಲೊಬ್ಬರು “ಈ ಪೋರ್ ಬಂದರಿನ ದಿವಾನರೊಂದಿಗೆ ಮಹಾಜ್ಞಾನಿಗಳಾದ ಪರಮಹಂಸರೊಬ್ಬರು ಉಳಿದುಕೊಂಡಿದ್ದಾ ರಂತೆ. ಅಲ್ಲದೆ ನಿರರ್ಗಳವಾಗಿ ಇಂಗ್ಲಿಷ್ ಮಾತನಾಡುತ್ತಾರಂತೆ! ಈಗ ನಾವೆಲ್ಲ ಹೋಗಿ ಅವರನ್ನು ಭೇಟಿ ಮಾಡೋಣ. ಆ ಮಹಾತ್ಮರು ದಿವಾನರಿಗೆ ಹೇಳಿ ನಮ್ಮ ದಾರಿ ಖರ್ಚನ್ನು ಹೊಂದಿಸಿಕೊಡಬಹುದು” ಎಂದರು. ಈ ಸೂಚನೆಗೆ ಎಲ್ಲರೂ ಒಪ್ಪಿ ಇಂಗ್ಲಿಷ್ ಬಲ್ಲ ತ್ರಿಗುಣಾತೀತಾನಂದರನ್ನು ತಮ್ಮ ಮುಂದಾಳಾಗಿರಿಸಿಕೊಂಡು ಅರಮನೆಯ ಕಡೆಗೆ ಹೊರಟರು.

ಆ ಸಮಯಕ್ಕೆ ಸರಿಯಾಗಿ ಅರಮನೆಯ ತಾರಸಿಯ ಮೇಲೆ ಅಡ್ಡಾಡುತ್ತಿದ್ದ ಸ್ವಾಮೀಜಿ ದೂರದಿಂದಲೇ ಈ ಗುಂಪನ್ನು ನೋಡಿದರು. ಆ ಸಾಧುಗಳ ಗುಂಪಿನಲ್ಲಿ ಸ್ವಾಮಿ ತ್ರಿಗುಣಾತೀತಾ ನಂದರು! ಸ್ವಾಮೀಜಿಗೆ ಆಶ್ಚರ್ಯವಾಯಿತು. ಆದರೆ ತಕ್ಷಣ ಒಂದು ರೀತಿಯ ತಟಸ್ಥ ಮುಖಭಾವವನ್ನು ಧರಿಸಿ, ಕೆಳಗೆ ಹೋಗಿ ಎಲ್ಲರನ್ನೂ ಬರಮಾಡಿಕೊಂಡರು. ತಾವು ಕಾಣ ಬಂದ ಪರಮಹಂಸರು ತಮ್ಮ ಪ್ರೀತಿಯ ನಾಯಕ ನರೇಂದ್ರನೇ ಆಗಿರುವುದನ್ನು ಕಂಡು ತ್ರಿಗುಣಾತೀತಾ ನಂದರಿಗೆ ಹಿಡಿಸಲಾರದಷ್ಟು ಸಂತೋಷ. ಆದರೆ ಸ್ವಾಮೀಜಿ ಗಂಭೀರಭಾವವನ್ನು ಬದಲಿಸದೆ, “ನಾನು ಅಷ್ಟು ಹೇಳಿದ್ದರೂ ನನ್ನನ್ನು ಹಿಂಬಾಲಿಸುತ್ತಿದ್ದೀಯಲ್ಲ, ಏಕೆ?” ಎಂದು ಗದರಿಸಿದರು. ತ್ರಿಗುಣಾತೀತಾನಂದರು ತಬ್ಬಿಬ್ಬಾಗಿ, “ಇಲ್ಲ ಇಲ್ಲ; ನಾನು ನಿನ್ನನ್ನು ಹಿಂಬಾಲಿಸುತ್ತಿಲ್ಲ! ನೀನು ಪೋರ್​ಬಂದರಿನಲ್ಲಿದ್ದೀಯೆ ಎಂಬುದರ ಕಲ್ಪನೆಯೇ ನನಗಿರಲಿಲ್ಲ. ಹಿಂಗಲಾಜಿಗೆ ಹೋಗಲು ದಾರಿಖರ್ಚಿಗೆ ಹಣ ಕೇಳುವುದಕ್ಕಾಗಿ ನಾವೆಲ್ಲ ಇಲ್ಲಿಗೆ ಬಂದೆವು ಅಷ್ಟೆ.” ಎಂದರು. ಮೊದಲು ಸ್ವಾಮೀಜಿ ಇದಕ್ಕೊಪ್ಪದೆ, “ಸಂನ್ಯಾಸಿಗಳು ಹಣವನ್ನು ಬೇಡಬಾರದು; ತಾನಾಗಿ ಬಂದುದನ್ನು ಸ್ವೀಕರಿಸಬೇಕು” ಎಂದು ಹೇಳಿಬಿಟ್ಟರು. ತ್ರಿಗುಣಾತೀತಾನಂದರು ಸಪ್ಪೆ ಮುಖ ಹಾಕಿಕೊಂಡು ಅಲ್ಲಿಂದ ಹೊರಟರು. ಆದರೆ ತಮ್ಮ ಪ್ರಿಯ ಗುರುಭಾಯಿಗೆ ನಿರಾಶೆಯುಂಟುಮಾಡಲಾರದೆ ಅವರನ್ನು ಮತ್ತೆ ಕರೆದರು ಸ್ವಾಮೀಜಿ. ಬಳಿಕ ಅವರೊಂದಿಗೆ ಕೆಲಕಾಲ ಆನಂದದಿಂದ ಸಂಭಾಷಿ ಸುತ್ತ ಕುಳಿತರು. ಮಾತುಕತೆಯ ಮಧ್ಯದಲ್ಲಿ ಸ್ವಾಮೀಜಿ ಹೇಳುತ್ತಾರೆ, “ನೋಡು ಪ್ರಸನ್ನ, ಈಗ ನನಗೆ ಶ್ರೀಗುರುಮಹಾರಾಜರು ನನ್ನ ಬಗ್ಗೆ ಹೇಳಿದ ಮಾತಿನ ಸತ್ಯ ಅರಿವಾಗುತ್ತಿದೆ. ನಿಜಕ್ಕೂ ಈಗ ನನ್ನಲ್ಲಿ ಎಷ್ಟು ಶಕ್ತಿ ತುಂಬಿದೆಯೆಂದರೆ, ನಾನೀಗ ಇಡೀ ಜಗತ್ತಿನಲ್ಲೇ ಕ್ರಾಂತಿಯುಂಟು ಮಾಡಿಬಿಡಬಲ್ಲೆ ಎನ್ನಿಸುತ್ತಿದೆ!”

ಹೀಗೆ ಸ್ವಲ್ಪ ಹೊತ್ತು ಸಂತೋಷದಿಂದ ಕಳೆದಮೇಲೆ, ಸ್ವಾಮೀಜಿ ದಿವಾನರೊಂದಿಗೆ ಮಾತ ನಾಡಿ ತಮ್ಮ ಗುರುಭಾಯಿ ಹಾಗೂ ಅವರ ಸಂಗಡಿಗರಿಗೆ ದಾರಿ ಖರ್ಚಿನ ಹಣವನ್ನು ಒದಗಿಸಿ ಕೊಟ್ಟು ಅವರನ್ನು ಬೀಳ್ಕೊಂಡರು.

ಪೋರ್​ಬಂದರಿನಲ್ಲಿ ಅಧ್ಯಯನಾದಿಗಳಲ್ಲಿ ಹಲವಾರು ತಿಂಗಳುಗಳನ್ನು ಕಳೆದ ಸ್ವಾಮೀಜಿ, ಅಲ್ಲಿಂದ ಹೊರಟು ದ್ವಾರಕಾ ಕ್ಷೇತ್ರದತ್ತ ಮುನ್ನಡೆದರು. ಭಾರತದ ‘ಚತುರ್ಧಾಮ’ಗಳಲ್ಲಿ ಒಂದಾದ ಬದರೀ ಕ್ಷೇತ್ರವನ್ನು ಮುಟ್ಟುವಲ್ಲಿ ವಿಫಲರಾಗಿದ್ದ ಅವರು, ಅಲ್ಲಿಂದ ಪಶ್ಚಿಮದ ಧಾಮವಾದ ದ್ವಾರಕೆಯತ್ತ ಹೊರಟಿದ್ದರು. ಶ್ರೀಕೃಷ್ಣನ ದಿವ್ಯ ಲೀಲೆಗಳ ಪವಿತ್ರ ಸ್ಮೃತಿಗಳನ್ನು ಹೊತ್ತು ನಿಂತಿರುವ ಪಟ್ಟಣ ದ್ವಾರಕೆ. ಆದರೆ ಶ್ರೀಕೃಷ್ಣನ ಭವ್ಯ ನಗರವಾಗಿದ್ದ ಈ ದ್ವಾರಕೆಯ ಬಹುಪಾಲನ್ನು ಈಗ ಸಮುದ್ರ ಕಬಳಿಸಿದೆ. ಅಂದಿನ ವೈಭವವನ್ನು ಪ್ರತಿಬಿಂಬಿಸುವಂಥದೇನೂ ಇಂದು ಉಳಿದಿಲ್ಲ. ಇಲ್ಲಿನ ಸಮುದ್ರತೀರದಲ್ಲಿ ಕುಳಿತು ಭೋರ್ಗರೆಯುವ ಸಮುದ್ರವನ್ನು ವೀಕ್ಷಿಸುತ್ತಿದ್ದಂತೆ ಸ್ವಾಮೀಜಿಯ ಮನದಲ್ಲೂ ಆಲೋಚನೆಗಳ ಅಲೆಗಳೆದ್ದುವು. ‘ದ್ವಾರಕೆಗೆ ಇಂದು ಬಂದೊದಗಿರುವ ದುಃಸ್ಥಿತಿಯೇ ಭಾರತದ ಸ್ಥಿತಿಯೂ ಆಗಿರುವುದಲ್ಲ! ಪಾಳುಬಿದ್ದ ಅವಶೇಷಗಳ ಹೊರತಾಗಿ ಪುರಾತನ ಭವ್ಯ ಭಾರತದ ಸಂಪತ್ತು ಇಂದೇನೂ ಉಳಿದಿಲ್ಲವಲ್ಲ!’ ಎಂದು ಮರುಗಿದರು. ಭಾರತದ ಭವಿಷ್ಯದಾಳವನ್ನು ಇಣಿಕಿ ನೋಡುವ ಪ್ರಯತ್ನದಲ್ಲಿ ತೊಡಗಿ ಕುಳಿತಲ್ಲೇ ಕುಳಿತುಬಿಟ್ಟರು. ಹೀಗೆ ಎಷ್ಟು ಹೊತ್ತು ಕುಳಿತಿದ್ದರೊ! ಬಳಿಕ ಕನಸಿನಿಂದೆಚ್ಚರ ಗೊಂಡವರಂತೆ ಮೇಲೆದ್ದು ನಡೆದರು.

ದ್ವಾರಕೆಯಲ್ಲಿ ಸ್ವಾಮೀಜಿ ಶ್ರೀಶಂಕರಾಚಾರ್ಯರಿಂದ ಸ್ಥಾಪಿಸಲ್ಪಟ್ಟ ಶಾರದಾಮಠವನ್ನು ಸಂದರ್ಶಿಸಿದರು. ಮಠದ ಮಹಂತರು ಅವರನ್ನು ಬರಮಾಡಿಕೊಂಡು ಉಳಿದುಕೊಳ್ಳಲು ಅವರಿ ಗೊಂದು ಕೋಣೆಯನ್ನು ಬಿಟ್ಟುಕೊಟ್ಟರು. ಕೋಣೆಯಲ್ಲಿ ವಿರಮಿಸಲು ಒರಗಿಕೊಳ್ಳುತ್ತಾರೆ ಸ್ವಾಮೀಜಿ, ಮತ್ತೆ ಮನದಲ್ಲಿ ಅದೇ ಚಿಂತೆ! ತಮ್ಮ ಕೋಣೆಯ ಮೌನದಾಳದಲ್ಲಿ ಅವರೊಂದು ಮಹಾಜ್ಯೋತಿಯನ್ನು ಕಾಣುತ್ತಾರೆ–ಅದೇ ಭವ್ಯಭಾರತ!

ಈ ಪ್ರದೇಶದ ಉಳಿದ ತೀರ್ಥಸ್ಥಳಗಳನ್ನೆಲ್ಲ ಬೇಗ ನೋಡಿ ಮುಂದೆ ಸಾಗುವ ಉದ್ದೇಶದಿಂದ ಸ್ವಾಮೀಜಿ, ಇಲ್ಲಿಂದ ಭೆಟ್​ದ್ವಾರಕೆಗೆ (ಭೆಟ್ ಎಂದರೆ ದ್ವೀಪ) ಹೋದರು. ಬಳಿಕ ಕಚ್​ನ ಮಹಾರಾಜನ ಆಹ್ವಾನದಂತೆ ಸುತ್ತುಮುತ್ತಲಿನ ನಾರಾಯಣ ಸರೋವರ, ಆಶಾಪುರಿ, ಮಾಂಡವಿಗಳಿಗೆ ಭೇಟಿನೀಡಿದರು.

