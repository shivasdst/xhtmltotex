
\chapter{ವಿಶ್ವವೇದಿಕೆ-ವಿಜಯಪತಾಕೆ!}

\noindent

೧೮೯೩ರ ಸೆಪ್ಟೆಂಬರ್ ೧೧ರಿಂದ ೨೭ರವರೆಗೆ ಶಿಕಾಗೋದ ‘ಕೊಲಂಬಿಯನ್ ಜಾಗತಿಕ ಮೇಳ’ದ ಅಂಗವಾಗಿ ನಡೆದ ವಿಶ್ವಧರ್ಮ ಸಮ್ಮೇಳನವು ಪ್ರಪಂಚದ ಇತಿಹಾಸದಲ್ಲೇ ಅತ್ಯಂತ ಮಹತ್ವ ಪೂರ್ಣ ಘಟನೆಗಳಲ್ಲೊಂದು ಎನ್ನಬಹುದು. ವಿಶ್ವದ ಧರ್ಮಗಳ ಚರಿತ್ರೆಯಲ್ಲಿ, ಅದರಲ್ಲೂ ಮುಖ್ಯವಾಗಿ ಹಿಂದೂಧರ್ಮದ ಚರಿತ್ರೆಯಲ್ಲಿ, ಇದೊಂದು ಹೊಸ ಅಧ್ಯಾಯಕ್ಕೇ ನಾಂದಿ ಯಾಯಿತು. ಪ್ರಪಂಚದ ಸಕಲ ಮತಧರ್ಮಗಳ ಪ್ರತಿನಿಧಿಗಳು ಜಗತ್ತಿನ ಎಲ್ಲೆಡೆಗಳಿಂದ ಬಂದು ಇದರಲ್ಲಿ ಭಾಗವಹಿಸಿದರು. ಇದು ಕೇವಲ ಒಂದು ಧರ್ಮಸಮ್ಮೇಳನ ಮಾತ್ರವೇ ಆಗಿರದೆ ಸಮಸ್ತ ಮಾನವತೆಯ ಸಮ್ಮೇಳನವಾಗಿತ್ತು. ಈ ಸಮ್ಮೇಳನದ ಉದ್ದೇಶವು, ಅದರ ಯೋಜಕರು ಹೇಳಿಕೊಂಡಂತೆ, ಸಮಗ್ರ ಮಾನವತೆಯ ಧಾರ್ಮಿಕ ದರ್ಶನಗಳನ್ನು ಒಂದುಗೂಡಿಸುವುದಾಗಿತ್ತು.

ಈ ಚರಿತ್ರಾರ್ಹ ಧರ್ಮಸಮ್ಮೇಳನದ ಹಿನ್ನೆಲೆಯೂ ಬಹಳ ಕುತೂಹಲಕರವಾಗಿದೆ. ಅಮೆರಿಕದ ಹಲವಾರು ಅತ್ಯುನ್ನತ ಕ್ರೈಸ್ತಧರ್ಮಾಧಿಕಾರಿಗಳು ಇಂಥದೊಂದು ಸಮ್ಮೇಳನದ ಆವಶ್ಯಕತೆ ಯನ್ನೂ ಅದರ ಲಾಭಗಳನ್ನೂ ಆಗ ಕೆಲಕಾಲದಿಂದಲೂ ಒತ್ತಿ ಹೇಳುತ್ತಿದ್ದರು. ೧೮೯೩ರಲ್ಲಿ ಶಿಕಾಗೋ ನಗರದಲ್ಲಿ ಜಾಗತಿಕ ಮೇಳವೊಂದನ್ನು ಭಾರೀ ಪ್ರಮಾಣದಲ್ಲಿ ನಡೆಸಬೇಕೆಂದು ನಿಶ್ಚಯವಾಗಿತ್ತು. ಮಾನವನ ಲೌಕಿಕ ಅಭಿವೃದ್ಧಿಯ ಸಾಧನೆಗಳೆಲ್ಲವನ್ನೂ ಪ್ರದರ್ಶಿಸುವುದೇ ಇದರ ಉದ್ದೇಶವಾಗಿತ್ತು. ಆದರೆ ಜಗತ್ತಿನ ಆಲೋಚನೆ-ಚಿಂತನೆಗಳನ್ನೂ ಪ್ರತಿನಿಧಿಸದೆ ಪ್ರದರ್ಶನವು ಪೂರ್ಣವಾಗುವಂತಿರಲಿಲ್ಲ. “ಮಾನವಕೋಟಿಗೆ ಸಂಬಂಧಿಸಿದ ಅತಿ ಮುಖ್ಯ ವಿಷಯಗಳೆಲ್ಲದರ ಮೇಲೂ ಸಮ್ಮೇಳನಗಳು ನಡೆಯಬೇಕು; ಹಾಗೂ ಅವು ವ್ಯಾಪಕವಾಗಿದ್ದು ಜಗತ್ತಿನ ಎಲ್ಲೆಡೆಗಳಿಂದಲೂ ಪ್ರತಿನಿಧಿಗಳು ಬರುವಂತಾಗಬೇಕು” ಎಂದು ಚಾರ್ಲ್ಸ್ ಬಾನಿ ಎಂಬ ಸುಪ್ರಸಿದ್ಧ ವಕೀಲ-ಲೇಖಕ, ೧೮೮೯ರಲ್ಲೇ ಸಲಹೆ ಮಾಡಿದ್ದ. ಈ ಸಲಹೆಗೆ ಭಾರೀ ಪ್ರಚಾರ ಸಿಕ್ಕಿ, ಬಹುತೇಕ ಜನರ ಪ್ರೋತ್ಸಾಹ ದೊರಕಿತು. ಕಡೆಗೆ ‘ಕೊಲಂಬಿಯನ್ ಮೇಳ’ಕ್ಕೆ ಪೂರಕವಾದ ಜಾಗತಿಕ ಸಮ್ಮೇಳನಗಳಿಗೆ ಬಾನಿಯನ್ನು ಅಧ್ಯಕ್ಷನನ್ನಾಗಿ ನೇಮಿಸಲಾಯಿತು. ಮುಂದಿನ ಎರಡೂ ವರೆ ವರ್ಷಗಳಲ್ಲಿ ಅತ್ಯಂತ ವಿವರಪೂರ್ಣವೂ ಜಟಿಲವೂ ಆದ ಯೋಜನೆಗಳನ್ನು ಸಿದ್ಧಪಡಿಸಲಾ ಯಿತು. ಆಂತಿಮವಾಗಿ, ಸ್ತ್ರೀಯರ ಮುನ್ನಡೆ, ವೈದ್ಯಕೀಯ ಶಾಸ್ತ್ರ, ವಿಜ್ಞಾನ, ಸಂಗೀತ, ಸರಕಾರ ಹಾಗೂ ಕಾನೂನು ಸುಧಾರಣೆ, ಧರ್ಮ–ಇವೇ ಮೊದಲಾದ ಇಪ್ಪತ್ತು ವೈವಿಧ್ಯಮಯ ವಿಚಾರ ಗಳ ಬಗ್ಗೆ ಸಮ್ಮೇಳನಗಳು ಏರ್ಪಾಡಾದುವು. ಇವೆಲ್ಲವನ್ನೂ ೧೮೯೩ರ ಮೇ ನಿಂದ ಅಕ್ಟೋಬರ್ ಅಂತ್ಯದವರೆಗಿನ ಅವಧಿಯಲ್ಲಿ ನಡೆಸಲಾಯಿತು.

ಆದರೆ ಈ ಎಲ್ಲ ಅದ್ಭುತ ‘ಲೌಕಿಕ’ ವಿಚಾರಗಳ ಮಧ್ಯೆ ಧರ್ಮವನ್ನೇಕೆ ತಂದರು? ಸಮ್ಮೇಳನ ಗಳ ಮುಖ್ಯ ಸಂಘಟಕರಲ್ಲೊಬ್ಬರೂ ‘ಸಾಮಾನ್ಯ ಸಮಿತಿ’ಯ ಚೇರ್​ಮನ್ನರೂ ಆಗಿದ್ದ ರೆವ ರೆಂಡ್ ಜಾನ್ ಹೆನ್ರಿ ಬರೋಸ್ ಹೇಳುತ್ತಾರೆ: “ಯಾವ ದೈವೀಶಕ್ತಿಗೆ ತಾವು ಪೂಜೆಯನ್ನು ಸೇವೆ ಯನ್ನೂ ಸಲ್ಲಿಸಬೇಕೆಂದು ಜನರು ನಂಬಿದ್ದಾರೋ, ಆ ಭಗವಂತನ ಮೇಲಿನ ಶ್ರದ್ಧೆಯು– ಸೂರ್ಯನು ಹೇಗೋ ಹಾಗೆ–ಮಾನವನ ಬೌದ್ಧಿಕ ಹಾಗೂ ನೈತಿಕ ಬೆಳವಣಿಗೆಗೆ ಆಧಾರಭೂತ ಶಕ್ತಿಯಾಗಿದೆ. ಆದ್ದರಿಂದ ಕೊಲಂಬಿಯನ್ ಮೇಳದಲ್ಲಿ ‘ವಿದ್ಯಾಭ್ಯಾಸ’, ‘ಕಲೆ’ ಅಥವಾ ‘ವಿದ್ಯು ಚ್ಛಕ್ತಿ’ಗಳಿಗೆ ಅವಕಾಶ ಮಾಡಿಕೊಟ್ಟಿರುವಾಗ ‘ಧರ್ಮ’ವನ್ನು ಮಾತ್ರ ಹೊರಗಿಡಬೇಕೆಂದು ನಮಗನ್ನಿಸಲಿಲ್ಲ.” ಎಂದರೆ, ಇತರ ‘ಲೌಕಿಕ’ ವಿಚಾರಗಳು ಮಾನವನ ಅಭಿವೃದ್ಧಿಗೆ ಹೇಗೆ ಕಾರಣವಾಗಿದೆಯೋ ಹಾಗೆಯೇ, ಭಗವಂತನ ಮೇಲಿನ ಶ್ರದ್ಧೆಗೆ ಮೂಲಭೂತವಾದ ಧರ್ಮವೂ ಅಷ್ಟೇ ಕಾರಣವಾಗಿದೆ ಎಂಬುದು ಅವರ ಅಭಿಪ್ರಾಯ.

ಈ ಎಲ್ಲ ಸಮ್ಮೇಳನಗಳ ಪೈಕಿ ‘ವಿಶ್ವಧರ್ಮಸಮ್ಮೇಳನ’ವೇ ಅತ್ಯಂತ ಪ್ರಸಿದ್ಧವೂ, ಜನ ಪ್ರಿಯವೂ ಆಯಿತು. ಆ ಹಿಂದೆ ನಡೆದ ಯಾವೊಂದು ಸಭೆಯೂ ಜಗತ್ತಿನಾದ್ಯಂತ ಇದರಷ್ಟು ಕುತೂಹಲ ಕೆರಳಿಸಿರಲಿಲ್ಲ. ಜಗತ್ತಿನ ಧರ್ಮಗಳ ಚರಿತ್ರೆಯಲ್ಲೇ ಅದೊಂದು ಅಭೂತಪೂರ್ವ ಸಂಗತಿಯಾಗಿತ್ತು. ಏಕೆಂದರೆ, ವಿವಿಧ ಧರ್ಮಗಳ ಪ್ರತಿನಿಧಿಗಳು ಒಟ್ಟಾಗಿ ಸೇರಿ ತಂತಮ್ಮ ಧರ್ಮದ ಬಗ್ಗೆ ಸಹಸ್ರಾರು ಜನರ ಮುಂದೆ ನಿರ್ಭೀತಿಯಿಂದ ಮಾತನಾಡುವ ಅವಕಾಶ ಹಿಂದೆಂದೂ ಒದಗಿಬಂದಿರಲಿಲ್ಲ. ಅಂದಿನ ತೀವ್ರ ಅಸಹಿಷ್ಣುತೆಯ ಹಾಗೂ ಭೋಗವಾದದ ಉಚ್ಛ್ರಾಯದ ದಿನಗಳಲ್ಲಿ ಇಂತಹ ಧರ್ಮಸಮ್ಮೇಳನವೊಂದರ ಪ್ರಸ್ತಾಪ ಬಂದಾಗ, ಅದು ಮನುಷ್ಯಮಾತ್ರರಿಂದ ಸಾಧ್ಯವಾಗುವಂಥದಲ್ಲ ಎಂದು ಎಷ್ಟೋಜನ ಭಾವಿಸಿದರು. ಆದರೆ ಅದು ಕಾರ್ಯಗತವಾಗುವ ದಿನ ಹತ್ತಿರವಾದಂತೆ, ಅದರ ಹಿಂದೆ ಯಾವುದೋ ಅಲೌಕಿಕ ಶಕ್ತಿ ಕೆಲಸ ಮಾಡುತ್ತಿದ್ದಂತೆ ಒಬ್ಬ ಸಾಧಾರಣ ವೀಕ್ಷಕನಿಗೂ ತೋರುತ್ತಿತ್ತು. ಆದ್ದರಿಂದ ಸ್ವಾಮೀಜಿ ಅಮೆರಿಕೆಗೆ ಹೊರಡುವ ಮುನ್ನ ತುರೀಯಾನಂದರನ್ನು ಭೇಟಿಯಾದಾಗ, ‘ನೋಡು, ಅಲ್ಲಿ ಏನೇನು ವ್ಯವಸ್ಥೆ ನಡೆಯುತ್ತಿದೆಯೋ ಅವೆಲ್ಲ ಇದಕ್ಕಾಗಿಯೇ’ ಎಂದು ತಮ್ಮನ್ನು ತೋರಿಸಿ ಕೊಂಡು ಹೇಳಿದ್ದುದರಲ್ಲಿ ಅಚ್ಚರಿಯೇನೂ ಇಲ್ಲ!

ಈ ವಿಶ್ವಧರ್ಮಸಮ್ಮೇಳನದ ಘೋಷಿತ ಉದ್ದೇಶಗಳು ಬಹಳ ಅದ್ಭುತ-ರಮ್ಯವಾಗಿದ್ದುವು. ಈ ಯೋಜನೆಯ ಜನಕವಾದ ಚಾರ್ಲ್ಸ್ ಬಾನಿ, ತನ್ನ ಉದ್ದೇಶದ ಬಗ್ಗೆ ಪ್ರಾಮಾಣಿಕನಾಗಿದ್ದಂತೆ ಕಾಣುತ್ತದೆ. ‘ವಿಶ್ವಭ್ರಾತೃತ್ವದ ನಿರ್ಮಾಣ’, ‘ಜಗತ್ತಿನ ಧರ್ಮಗಳ ನಡುವೆ ಪರಸ್ಪರ ಅರಿವು- ಹೊಂದಾಣಿಕೆಗಳನ್ನುಂಟುಮಾಡುವುದು,’ ‘ವಿಶ್ವಶಾಂತಿಗಾಗಿ ಶ್ರಮಿಸುವುದು’–ಇವೇ ಮೊದ ಲಾದ ಘನ ಉದ್ದೇಶಗಳನ್ನು ದೃಷ್ಟಿಯಲ್ಲಿಟ್ಟುಕೊಂಡು ಸಮ್ಮೇಳನವನ್ನು ಯೋಜಿಸಿರುವುದಾಗಿ ಪ್ರಚಾರ ಮಾಡಲಾಗಿದೆ. ಆದರೆ ಘೋಷಿತ ಉದ್ದೇಶಗಳು ಎಷ್ಟೇ ಘನವಾಗಿ ತೋರಿದ್ದರೂ, ಅದರ ಹಿಂದೆ ಕೆಲಮಟ್ಟಿನ ಸ್ವಾರ್ಥಸಾಧನೆಯ ‘ವ್ಯಾವಹಾರಿಕ’ ಉದ್ದೇಶಗಳೂ ಇದ್ದುವೆಂಬುದು ಬೇಗ ಬಯಲಿಗೆ ಬಂದಿತು. ಸರ್ವಧರ್ಮಸಮ್ಮೇಳನದ ಹೆಸರಿನಲ್ಲಿ ತಮ್ಮ ಬೇಳೆಯನ್ನು ಬೇಯಿಸಿಕೊಳ್ಳಲು ಅನೇಕ ಕ್ರೈಸ್ತ ಧರ್ಮ ಪ್ರಚಾರಕರು ಉದ್ಯುಕ್ತರಾಗಿದ್ದರು. ಸ್ವಾಮೀಜಿಗೆ ಸಮ್ಮೇಳನದ ಈ ಹಿನ್ನೆಲೆಗಳಾವುವೂ ಮೊದಲಿಗೆ ತಿಳಿದಿರಲಿಲ್ಲವಾದರೂ, ಕೆಲದಿನಗಳಲ್ಲೇ ಎಲ್ಲವೂ ಸ್ಪಷ್ಟವಾಯಿತು. ಅವರು ತಮ್ಮ ಮದರಾಸೀ ಶಿಷ್ಯರಿಗೆ ಬರೆದ ಪತ್ರದಲ್ಲಿ ಹೇಳಿದರು: “ಜಗತ್ತಿನ ಎಲ್ಲ ಧರ್ಮಗಳಿಗಿಂತಲೂ ಕ್ರೈಸ್ತ ಧರ್ಮವೇ ಶ್ರೇಷ್ಠವೆಂಬುದನ್ನು ಸಾಬೀತುಗೊಳಿಸು ವುದಕ್ಕಾಗಿಯೇ ಈ ಸಮ್ಮೇಳನವನ್ನು ಏರ್ಪಡಿಸಲಾಗಿತ್ತು.” ಇನ್ನೊಮ್ಮೆ ಸಂದರ್ಶನವೊಂದರಲ್ಲಿ ಸ್ವಾಮೀಜಿ ಹೇಳುತ್ತಾರೆ: “ನನಗೆ ತೋರುವಂತೆ, ಸರ್ವಧರ್ಮ ಸಮ್ಮೇಳನದ ಹಿನ್ನೆಲೆಯಲ್ಲಿ ಪ್ರಪಂಚದ ಮುಂದೆ ‘ಕ್ರೈಸ್ತೇತರ ಅನಾಗರಿಕ’ರನ್ನು ಪ್ರದರ್ಶಿಸುವ ಉದ್ದೇಶವಿತ್ತು.” ಸ್ವಾಮೀಜಿಯ ಈ ಮಾತುಗಳು ಬಹಳ ಕಟುವಾಗಿ ಕಾಣಬಹುದು. ಆದರೆ ಈ ಸಮ್ಮೇಳನದ ಸಿದ್ಧತೆಗಳ ಹಾಗೂ ನಡೆವಳಿಗಳ ವರದಿಗಳನ್ನು ಓದಿದಾಗ ಅವರ ಮಾತಿನ ಸತ್ಯ ಮನವರಿಕೆಯಾಗು ತ್ತದೆ. ಕ್ರೈಸ್ತಧರ್ಮವು ಇತರ ಎಲ್ಲ ಧರ್ಮಗಳ ಮೇಲಿನ ತನ್ನ ದೊಡ್ಡಸ್ತಿಕೆಯನ್ನು ಅನುಮಾನಕ್ಕೆಡೆ ಇಲ್ಲದಂತೆ ಸಾಧಿಸಿ ತೋರಿಸುತ್ತದೆ ಎಂದು ಸಮ್ಮೇಳನದ ಬಹುಪಾಲು ವ್ಯವಸ್ಥಾಪಕರು ತೀರ್ಮಾನಿಸಿಬಿಟ್ಟಿದ್ದರು.

ಆದರೆ ಇಂಥವರ ನಡುವೆ, ಈ ಸಮ್ಮೇಳನಕ್ಕೆ ಸಂಬಂಧಿಸಿದವರಲ್ಲೇ ಸದುದ್ದೇಶವನ್ನು ಹೊಂದಿದ್ದ ವ್ಯಕ್ತಿಗಳೂ ಇರಲಿಲ್ಲವೆಂದಲ್ಲ. ಸ್ವಾರ್ಥಪ್ರೇರಿತ ಭಾವನೆಗಳಿಂದಲೂ ಧಾರ್ಮಿಕ ಪೂರ್ವಗ್ರಹದಿಂದಲೂ ದೂರವಾಗಿದ್ದ ಕೆಲವರು, ಇದನ್ನು ತುಂಬ ವಿಶಾಲವಾದ ಅರ್ಥದಲ್ಲಿ ಸ್ವೀಕರಿಸಿದ್ದರು. ವಿವಿಧ ಪಥಾವಲಂಬಿಗಾಳಾದ ಸತ್ಯಾನ್ವೇಷಿಗಳ ನಡುವೆ ಪರಸ್ಪರ ಅರಿವು ಹಾಗೂ ಸಹೃದಯತೆಯನ್ನು ಬೆಸೆಯಲು ಇದೊಂದು ಅಪೂರ್ವ ಅವಕಾಶ ಎಂದು ಅವರು ನಂಬಿ, ಸಮ್ಮೇಳನವನ್ನು ಕಾತರದಿಂದ ಇದಿರು ನೋಡುತ್ತಿದ್ದರು. ಇಂಥವರಲ್ಲಿ ಅಧ್ಯಕ್ಷ ಚಾರ್ಲ್ಸ್ ಬಾನಿ ಒಬ್ಬನಾಗಿದ್ದ.

ಆದರೆ ಈ ಸಮ್ಮೇಳನದ ಯೋಜನೆಯ ಸುದ್ದಿ ಹರಡುತ್ತಿದ್ದಂತೆ ಕ್ರೈಸ್ತ ಸಂಸ್ಥೆಗಳಿಂದಲೂ ಧರ್ಮಪ್ರಚಾರಕರಿಂದಲೂ ಪ್ರತಿಭಟನೆಯ ಗೊಣಗಾಟ ಕೇಳಿಬಂತು. ಬರಬರುತ್ತ ಇದು ಪ್ರಬಲ ವಾಗಿ ಎಲ್ಲಡೆಯಿಂದಲೂ ತೀವ್ರ ವಿರೋಧದ ದನಿ ಪ್ರತಿಧ್ವನಿಸಿತು. ತಾವು ನಂಬಿರುವ ಹಾಗೂ ಇತರರಿಗೂ ನಂಬುವಂತೆ ಬೋಧಿಸುತ್ತಿರುವ ಏಕ ಮಾತ್ರ ಅವತಾರನಾದ ಕ್ರಿಸ್ತನಿಗೂ, ಏಕಮಾತ್ರ ಧರ್ಮವಾದ ಕ್ರೈಸ್ತಧರ್ಮಕ್ಕೂ ಇದರಿಂದ ಅಪಚಾರವಾಗಬಹುದೆಂದು ಇವರು ಹೆದರಿದ್ದರು. ತಾನು ಈ ಸಭೆಯಲ್ಲಿ ಭಾಗವಹಿಸಲು ಸಾಧ್ಯವಿಲ್ಲವೆಂದು ಘೋಷಿಸಿದ ಕ್ಯಾಂಟರ್​ಬರಿಯ ಆರ್ಚ್​ಬಿಷಪ್, ಸಮ್ಮೇಳನದ ಅಧಿಕಾರಿಗಳಿಗೆ ಬರೆದ ಪತ್ರದಲ್ಲಿ ಅದಕ್ಕೆ ಕಾರಣವನ್ನು ತಿಳಿಸಿದ: “... ನನ್ನೀ ನಿರ್ಧಾರಕ್ಕೆ, ದೂರ ಪ್ರಯಾಣ ಅಥವಾ ಇತರ ಅನನುಕೂಲತೆಗಳು ಕಾರಣವಲ್ಲ. ಆದರೆ ‘ಧರ್ಮವೆಂದರೆ ಕ್ರೈಸ್ತ ಧರ್ಮವೊಂದೇ’ ಎಂಬ ವಾಸ್ತವಾಂಶವೇ ನಾನು ಬಾರದಿರುವು ದಕ್ಕೆ ಕಾರಣ. ಸರ್ವಧರ್ಮ ಸಮ್ಮೇಳನದಲ್ಲಿ ಭಾಗವಹಿಸಲು ಇಚ್ಛಿಸುವ ಇತರ ಪ್ರತಿನಿಧಿಗಳಿಗೆ ಕ್ರೈಸ್ತಧರ್ಮದೊಂದಿಗೆ ಸಮಾನ ಮನ್ನಣೆ-ಸ್ಥಾನಮಾನಗಳನ್ನು ಕೊಡದೆ ಸಮ್ಮೇಳನವನ್ನು ಹೇಗೆ ನಡೆಸಲು ಸಾಧ್ಯವೋ ನನಗಂತೂ ಅರ್ಥವಾಗುತ್ತಿಲ್ಲ!” ಇಂತಹ ಒಂದು ಸಮ್ಮೇಳನವನ್ನು ನಡೆಸಿದರೆ, ಸರ್ವೋತ್ಕೃಷ್ಟವಾದ ತನ್ನ ಕ್ರೈಸ್ತಧರ್ಮವನ್ನು ಇತರ ‘ಅನಾಗರಿಕ ಧರ್ಮ’ಗಳ ಪಂಕ್ತಿಯಲ್ಲೇ ಕುಳ್ಳಿರಿಸಬೇಕಾಗುತ್ತದಲ್ಲ! ಇಂತಹ ಅನ್ಯಾಯ ಎಲ್ಲಾದರೂ ಉಂಟೆ?–ಇದು ಆತನ ಆತಂಕ. ಎಷ್ಟೋ ಜನ ಈ ಅಭಿಪ್ರಾಯವನ್ನು ಸಮರ್ಥಿಸಿದರು. ಹಾಂಗ್​ಕಾಂಗಿನ ಪಾದ್ರಿಯೊಬ್ಬ ಬರೆದ: “ನೀವು ನಿಮ್ಮನ್ನು ದಾರಿತಪ್ಪಿಸಿಕೊಂಡರೂ ಪರವಾಗಿಲ್ಲ; ಆದರೆ ಇತರ ರನ್ನು ತಪ್ಪು ದಾರಿಗೆಳೆಯಬೇಡಿ. ನಿಮ್ಮನ್ನು ಬೇಡಿಕೊಳ್ಳುತ್ತೇನೆ–ಪೊಳ್ಳುಧರ್ಮಗಳೊಂದಿಗೆ ಹುಡುಗಾಟಿಕೆಯಾಡಲು ಹೋಗಿ ನಿಮ್ಮ ಅಮೂಲ್ಯ ಆತ್ಮವನ್ನು ಗಂಡಾಂತರಕ್ಕೆ ಸಿಕ್ಕಿಸಿಕೊಳ್ಳ ಬೇಡಿ... ನೀವು ನಿಮಗೆ ಅರಿವಿಲ್ಲದಂತೆ ಕ್ರಿಸ್ತನ ವಿರುದ್ಧ ಪಿತೂರಿ ನಡೆಸುತ್ತಿದ್ದೀರಿ.”

ಆದರೆ, ಈ ಆರ್ಚ್​ಬಿಷಪ್​ನಂತಹವರ ನಿಲುವನ್ನು ಕೆಲವು ಕಟ್ಟಾ ಕ್ರೈಸ್ತರೇ ಟೀಕಿಸಿದರು– ಅವರು ಮತಭ್ರಾಂತರಂತೆ ವರ್ತಿಸುತ್ತಿದ್ದಾರೆ ಎಂದಲ್ಲ; ಬದಲಾಗಿ, ಕ್ರೈಸ್ತ ಧರ್ಮವು ಇತರ ಕ್ಷುದ್ರ ಧರ್ಮಗಳಿಗೆ ಹೆದರಬೇಕಾದ ಗತಿಯೇನೂ ಬಂದಿಲ್ಲ, ಎಂದು! ಅಮೆರಿಕದ ಒಬ್ಬ ಪಾದ್ರಿ ಬರೆದ, “ನನ್ನ ನಿಶ್ಚಿತ ಅಭಿಪ್ರಾಯದಂತೆ, ಕ್ರೈಸ್ತಧರ್ಮವು ಸರ್ವಶ್ರೇಷ್ಠವೆಂಬುದನ್ನು ಮನದಟ್ಟು ಮಾಡಿಸುವುದಕ್ಕಾಗಿ, ಅದನ್ನು ವೈಭವಯುತವಾಗಿ ಸ್ಫೂರ್ತಿಯುತವಾಗಿ ಮಂಡಿಸಲು ಯಾವ ಕ್ರೈಸ್ತನೂ ಹಿಂಜರಿಯಬೇಕಾಗಿಲ್ಲ. ಸಮ್ಮೇಳನದಲ್ಲಿ ಪ್ರಚಂಡ ಜನಸಮೂಹ ನೆರೆಯುವಂತಾ ದಾಗ ಆ ಸುವರ್ಣಾವಕಾಶವನ್ನು ಉಪಯೋಗಿಸಿಕೊಂಡು ಭಗವಂತ ಹೇಗೆ ತನ್ನ ಸತ್ಯವನ್ನು ಮೊಳಗುತ್ತಾನೆ, ಮತ್ತು ಹೇಗೆ ಕ್ರಿಸ್ತನ ಮುಂದೆ ಪ್ರತಿಯೊಬ್ಬನೂ ಮಂಡಿಯೂರಿ ತಲೆಬಾಗುವಂತೆ ಮಾಡುತ್ತಾನೆ ಎಂಬುದು ಅವನಲ್ಲದೆ ಬೇರಾರು ಬಲ್ಲರು!” ಹೀಗೆ ಕ್ರೈಸ್ತಧರ್ಮದ ಜಾಹೀರಾತಿನ ದೃಷ್ಟಿಯಿಂದ ಸಮ್ಮೇಳನವನ್ನು ನೂರಾರು ಜನ ಬೆಂಬಲಿಸಿದರು. ಈ ಅಭಿಪ್ರಾಯಗಳು ಸರ್ವ ಧರ್ಮಸಮ್ಮೇಳನದ ಉನ್ನತ ಆದರ್ಶಕ್ಕೆ ತದ್ವಿರುದ್ಧವಾಗಿದೆಯೆಂಬುದು, ಅದರ ಅಧ್ಯಕ್ಷರಾದ ಡಾ. ಬರೋಸ್​ರಿಗೂ ಹೊಳೆಯಲಿಲ್ಲ! ಅವರು ತಮ್ಮ ‘ಉದಾತ್ತ’ ಭಾವನೆಯನ್ನು ಅಪ್ಯಾಯ ಮಾನವಾದ ಮಾತುಗಳಲ್ಲಿ ವ್ಯಕ್ತಪಡಿಸಿದರು: “... ಕ್ರೈಸ್ತಧರ್ಮವು ಇತರ ಎಲ್ಲ ಧರ್ಮಗಳ ಸ್ಥಾನವನ್ನು ಆಕ್ರಮಿಸಬೇಕಾಗಿದೆಯೆಂದು ನಾವು ನಂಬುತ್ತೇವೆ. ಏಕೆಂದರೆ ಅದು ಇತರೆಲ್ಲ ಧರ್ಮಗಳಲ್ಲಿರುವ ಸತ್ಯವನ್ನೂ ತನ್ನಲ್ಲಿ ಹೊಂದಿರುವುದಲ್ಲದೆ, ಅವುಗಳಲ್ಲಿ ಇಲ್ಲದಿರುವ ಅತಿ ಮುಖ್ಯ ಅಂಶವಾದ ಪಾಪವಿಮೋಚನೆಯ ಭರವಸೆಯನ್ನು ನೀಡುತ್ತದೆ... ಬೆಳಕಿಗೆ ಕತ್ತಲೆ ಯೊಂದಿಗೇನೂ ಸಂಬಂಧವಿಲ್ಲದಿದ್ದರೂ, ನಸುಬೆಳಕಿನೊಂದಿಗೆ ಅದು ನಿಶ್ಚಿತವಾಗಿಯೂ ಒಡನಾಟವನ್ನಿಟ್ಟುಕೊಂಡಿದೆ. ಆದ್ದರಿಂದ ಶಿಲುಬೆಯ ಪೂರ್ಣ ಬೆಳಕನ್ನು ಹೊಂದಿರುವವರು ಮಸಕು ಬೆಳಕಿನಲ್ಲಿ ತಡಕಾಡುತ್ತಿರುವವರಿಗೆ ಸಹೃದಯ ಭ್ರಾತೃ ಭಾವದಿಂದ ಸಹಾನುಭೂತಿ ತೋರಬೇಕಾಗಿದೆ.”

ಆದರೆ, ಇಲ್ಲಿ ಇನ್ನೊಂದು ಮಾತನ್ನೂ ಹೇಳಲೇಬೇಕು. ಹೆಚ್ಚಿನ ಕ್ರೈಸ್ತಧರ್ಮಾಧಿಕಾರಿಗಳ ಅಪಪ್ರಚಾರ ಹಾಗೂ ಆಧುನಿಕ ನಾಗರಿಕತೆಯ ಉಗ್ರ ಭೌತಿಕವಾದದ ನಡುವೆಯೂ ಸಹಸ್ರಾರು ಸ್ತ್ರೀ-ಪುರುಷರು ದಿವ್ಯೋದ್ದೇಶವೊಂದರ ಈಡೇರಿಕೆಗಾಗಿ ಸರ್ವಧರ್ಮ ಸಮ್ಮೇಳನವನ್ನು ಕಾತರದ ಕಣ್ಣಿನಿಂದ ಇದಿರುನೋಡುತ್ತಿದ್ದರು. ಏಕೆಂದರೆ, ಆಧ್ಯಾತ್ಮಿಕ ಸತ್ಯಗಳು ಎಲ್ಲೇ ಕಾಣಸಿಗಲಿ, ಅವುಗಳನ್ನು ಅನ್ವೇಷಿಸುವ ಹಾಗೂ ಬಿಚ್ಚು ಮನಸ್ಸಿನಿಂದ ಸ್ವಾಗತಿಸುವ ಪ್ರಾಮಾಣಿಕ ಬುದ್ಧಿ ಅಮೆರಿಕದ ಜನರಲ್ಲಿತ್ತು. ಆದರೆ ಜನರು ಹೀಗೆ ಆಧ್ಯಾತ್ಮಿಕತೆಗಾಗಿ ಕಾತರರಾಗಿದರೂ ನಿಜವಾದ ಉದಾರತೆಗೆ ಒಟ್ಟಾರೆ ಸಮಾಜದ ಅಥವಾ ಕ್ರೈಸ್ತ ಧರ್ಮಾಧಿಕಾರಿಗಳ ಬೆಂಬಲವಿರಲಿಲ್ಲ. ಆದರೆ ವಿಡಂಬನೆಯ ಸಂಗತಿಯೇನೆಂದರೆ, ಸರ್ವಧರ್ಮ ಸಮ್ಮೇಳನವು ಪ್ರಚಾರದ ಉದ್ದೇಶದಿಂದಲೇ ಯೋಜಿತಗೊಂಡದ್ದಾದರೂ, ಅಂತಿಮವಾಗಿ ಅದು ಮತಾಂಧತೆಯ ವಿನಾಶಕ್ಕೆ ಎಡೆಮಾಡಿ ಕೊಟ್ಟಿತು.

ಈ ಐತಿಹಾಸಿಕವೂ ಅಭೂತಪೂರ್ವವೂ ಆದ ಸರ್ವಧರ್ಮ ಸಮ್ಮೇಳನವು ನಡೆಯಲಿದ್ದುದು ಶಿಕಾಗೋದ ಮಿಚಿಗನ್ ಅವೆನ್ಯೂನಲ್ಲಿ ನೂತನವಾಗಿ ನಿರ್ಮಿಸಲ್ಪಟ್ಟಿದ್ದ ‘ಆರ್ಟ್ ಇನ್ಸ್​ಟಿಟ್ಯೂಟ್​’ ಎಂಬ ಭವ್ಯ ಕಟ್ಟಡದಲ್ಲಿ. ಇಲ್ಲಿ ಹದಿನೇಳು ದಿನಗಳ ಕಾಲ ನಡೆದ ವೈಭವಪೂರ್ಣ ಸಮ್ಮೇಳನ ದಲ್ಲಿ ಮಾನವತೆಯ ಭಾರೀ ಸಮೂಹ ಪಾಲ್ಗೊಂಡಿತು. ಇದರಲ್ಲಿ ವಿಶ್ವದ ಅತ್ಯಂತ ವಿಖ್ಯಾತ ವ್ಯಕ್ತಿಗಳು ಸೇರಿದ್ದರು. ಸಭಿಕರ ಪೈಕಿ ಎಲ್ಲ ರಂಗಗಳಿಗೂ ಸೇರಿದ ಪ್ರಮುಖರಿದ್ದರು. ಸುಪ್ರಸಿದ್ಧ ರಾದ ಅನೇಕ ಪಾಶ್ಚಾತ್ಯ ತತ್ತ್ವಶಾಸ್ತ್ರಜ್ಞರು ಪ್ರತಿದಿನವೂ ಸಮ್ಮೇಳನದಲ್ಲಿ ಹಾಜರಿರುತ್ತಿದ್ದರು. ಪ್ರತಿನಿಧಿಗಳಲ್ಲಿ ಕೆಲವರು, ವಿಭಿನ್ನ ಕ್ರೈಸ್ತಪಂಗಡಗಳಿಗೆ ಸೇರಿದ ಉನ್ನತ ಧರ್ಮಾಧಿಕಾರಿಗಳು. ಸಮ್ಮೇಳನದ ಮುಖ್ಯ ಸಭೆ ನಡೆಯುತ್ತಿದ್ದುದು ಈ ಕಟ್ಟಡದ ‘ಹಾಲ್ ಆಫ್ ಕೊಲಂಬಸ್​’ ಎಂಬ ಸಭಾಭವನದಲ್ಲಿ. ಇದರಲ್ಲಿ ಸುಮಾರು ನಾಲ್ಕು ಸಾವಿರ ಜನರು ಕುಳಿತುಕೊಳ್ಳಲು ಅವಕಾಶವಿತ್ತು. ಆದರೆ ಬರಬರುತ್ತ ಜನಸಂದಣಿ ಹೆಚ್ಚಾಗುತ್ತ ಬಂದು, ನಾಲ್ಕನೆಯ ದಿನ ಭವನದ ಪಕ್ಕದ ‘ಹಾಲ್ ಆಫ್ ವಾಷಿಂಗ್​ಟನ್​’ ಎಂಬ ಮತ್ತೊಂದು ಭವನದಲ್ಲೂ ಜನ ತುಂಬಿದ್ದರು. ಅವರಿಗಾಗಿ ಅಂದು ಇಡೀ ಕಾರ್ಯಕ್ರಮವನ್ನು ಪುನರಾವರ್ತನೆ ಮಾಡಲಾಯಿತು! ಆದರೆ ಐದನೆಯ ದಿನದಿಂದ ಸರ್ವಧರ್ಮ ಸಮ್ಮೇಳನದ ‘ವೈಜ್ಞಾನಿಕ ವಿಭಾಗ’ದಲ್ಲಿ ಕಾರ್ಯಕ್ರಮಗಳು ಪ್ರಾರಂಭವಾದವು. ಈ ‘ವೈಜ್ಞಾನಿಕ ವಿಭಾಗ’ದಲ್ಲಿ ವಿವಿಧ ಧರ್ಮಗಳ ಉಗಮ, ಅವುಗಳ ಪ್ರಮಾಣಗ್ರಂಥಗಳು ಹಾಗೂ ಇತರ ವಿಷಯಗಳಿಗೆ ಸಂಬಂಧಿಸಿದಂತೆ ಶಾಸ್ತ್ರೀಯವಾದ ಅಧ್ಯಯನ-ಚರ್ಚೆಗಳು ನಡೆಯುತ್ತಿದ್ದುವು. ಈ ಕಾರ್ಯಕ್ರಮಗಳು ‘ಹಾಲ್ ಆಫ್ ವಾಷಿಂಗ್ ಟನ್​’ನಲ್ಲಿ ನಡೆಯುತ್ತಿದ್ದು, ಸಭಿಕರು ಈ ಎರಡು ಸಭಾಭವನಗಳಲ್ಲಿ ಹಂಚಿಹೋಗುತ್ತಿದ್ದರು. ವೈಜ್ಞಾನಿಕ ವಿಭಾಗದ ಸಭೆಗಳನ್ನೂ ಉದ್ದೇಶಿಸಿ ಸ್ವಾಮೀಜಿ ಅನೇಕ ಉಪನ್ಯಾಸಗಳನ್ನು ನೀಡುವು ದನ್ನು ಮುಂದೆ ನೋಡಲಿದ್ದೇವೆ.

೧೮೯೩ನೇ ಸೆಪ್ಟೆಂಬರ್ ೧೧ರಂದು ಸೋಮವಾರ ಬೆಳಿಗ್ಗೆ ೧ಂ ಗಂಟೆಗೆ ಸರಿಯಾಗಿ ಸರ್ವ ಧರ್ಮ ಸಮ್ಮೇಳನವು ಪ್ರಾರಂಭವಾಯಿತು. ಆ ಭವ್ಯವಾದ ಸಭಾಂಗಣವು ಅದಕ್ಕೆ ಮುಂಚಿತವಾಗಿ ಬಹಳ ಹೊತ್ತಿನಿಂದಲೇ ಕಿಕ್ಕಿರಿದು ತುಂಬಿತ್ತು. ಪ್ರಾರಂಭವನ್ನು ಸೂಚಿಸಲು ‘ನವಸ್ವಾತಂತ್ರ್ಯ ಘಂಟೆ’ ಹತ್ತು ಸಲ ಬಾರಿಸಿತು. ಸಮಸ್ತ ಮಾನವ ಜನಾಂಗದ ಮತ-ಧರ್ಮಗಳ ಪ್ರತಿನಿಧಿಗಳು ಮೆರವಣಿಗೆಯಲ್ಲಿ ಬಂದು ವೇದಿಕೆಯ ಮೇಲೆ ಆಸೀನರಾದರು. ವೇದಿಕೆಯ ನಡುಭಾಗದಲ್ಲಿ ಅಮೆರಿಕದ ರೋಮನ್ ಕ್ಯಾಥೋಲಿಕರ ಅತ್ಯುನ್ನತ ಧರ್ಮಗುರುಗಳಾದ ಕಾರ್ಡಿನಲ್ ಗಿಬ್ಬನ್ಸ್ ಕುಳಿತಿದ್ದರು. ಅವರ ಎರಡೂ ಪಕ್ಕಗಳಲ್ಲಿ ವಿವಿಧ ವೇಷಭೂಷಣಧಾರಿಗಳಾದ ಪ್ರತಿನಿಧಿಗಳು ಕುಳಿತಿದ್ದರು. ಇವರೆಲ್ಲರ ನಡುವೆ, ದಿವ್ಯವಾದ ಕಿತ್ತಲೆಬಣ್ಣದ ನಿಲುಂಗಿಯನ್ನೂ ಪೀತವರ್ಣದ ಭರ್ಜರಿ ಪೇಟವನ್ನೂ ಧರಿಸಿದ್ದ ಸ್ವಾಮೀಜಿ, ಭಾವಭರಿತರಾಗಿ ರಾಜಗಾಂಭೀರ್ಯದಿಂದ ಕುಳಿತಿ ದ್ದರು. ತಮ್ಮ ವಿಶಿಷ್ಟ ಉಡುಪಿನಿಂದಲೂ ಭವ್ಯ ವ್ಯಕ್ತಿತ್ವದಿಂದಲೂ ಸಮಸ್ತ ಸಭಿಕರ ಗಮನ ಸೆಳೆದಿದ್ದರು.

ಸಾಂಪ್ರದಾಯಿಕ ಪ್ರಾರ್ಥನೆಗಳೊಂದಿಗೆ ಸಮ್ಮೇಳನವು ಪ್ರಾರಂಭವಾಯಿತು. ಮೊದಲ ದಿನವು ಸಮ್ಮೇಳನದ ಸಂಘಟಕರಿಂದ ಸ್ವಾಗತಭಾಷಣಗಳಿಗೆ ಹಾಗೂ ಪ್ರತಿನಿಧಿಗಳ ಪ್ರತಿಕ್ರಿಯೆಗಳಿಗೆ ಮೀಸಲಾಗಿತ್ತು. ಮೊದಲಿಗೆ ಸ್ವಾಗತಸಮಿತಿಯ ಏಳು ಜನ ಅಧಿಕಾರಿಗಳು ತಮ್ಮ ಉಜ್ವಲ ವಾಗ್ವೈಖರಿಯನ್ನು ಮೆರೆಸುತ್ತ ಸುದೀರ್ಘ ಭಾಷಣಗಳನ್ನು ಮಾಡಿದರು. ಬಳಿಕ ಪ್ರತಿನಿಧಿಗಳು ಒಬ್ಬೊಬ್ಬರಾಗಿ ಅದಕ್ಕೆ ಉತ್ತರ ಕೊಡುತ್ತ ಮಾತನಾಡಿದರು. ಬೆಳಗಿನ ಅಧಿವೇಶನದಲ್ಲಿ ಎಂಟು ಜನ ಚುಟುಕಾದ ಭಾಷಣಗಳನ್ನು ಮಾಡಿದರು. ಇವರಲ್ಲಿ ಅನೇಕರಿಗೆ ಸಭಾಸದರಿಂದ ಅತ್ಯಂತ ಉತ್ತೇಜನಕಾರಿಯಾದ ಪ್ರತಿಕ್ರಿಯೆ ಸಿಕ್ಕಿತು. ಬ್ರಾಹ್ಮಸಮಾಜದ ಪ್ರತಿನಿಧಿ ಪ್ರತಾಪ್​ಚಂದ್ರ ಮಜುಮ್ದಾರ್ (ಬ್ರಾಹ್ಮಸಮಾಜವನ್ನು ಒಂದು ಸ್ವತಂತ್ರ ಮತವೆಂದು ಪರಿಗಣಿಸಿ, ವಿಶೇಷ ಪ್ರಾತಿನಿಧ್ಯ ನೀಡಲಾಗಿತ್ತು) ಹಾಗೂ ಶ್ರೀಲಂಕಾದ ಬೌದ್ಧರ ಪ್ರತಿನಿಧಿ ಧರ್ಮಪಾಲ-ಇವರಿಗೂ ಸಭಿಕರಿಂದ ಒಳ್ಳೆಯ ಪ್ರತಿಕ್ರಿಯೆ ಸಿಕ್ಕಿತು.

ವೇದಿಕೆಯ ಮೇಲೆ ಕುಳಿತಿದ್ದ ಸ್ವಾಮೀಜಿ, ಆಗುಹೋಗುಗಳನ್ನೆಲ್ಲ ಮೌನವವಾಗಿ ವೀಕ್ಷಿಸು ತ್ತಿದ್ದರು. ನಿಜಕ್ಕೂ ಆ ಭವ್ಯ ಸಭಾಂಗಣದ ಕುರ್ಚಿಗಳಲ್ಲೂ ಎತ್ತರದ ಗ್ಯಾಲರಿಗಳಲ್ಲೂ ಕಿಕ್ಕಿರಿದು ತುಂಬಿದ್ದ ಜನಸಾಗರವನ್ನು ನೋಡಿದರೆ, ಅದರ ಮುಂದೆ ನಿಂತು ಮಾತನಾಡಲು ಎಂತಹ ನುರಿತ ಭಾಷಣಕಾರನಿಗೂ ಒಂದು ಕ್ಷಣ ದಿಗಿಲಾಗದಿರದು! ಅಂತಹ ಮಹಾ ಬುದ್ಧಿಮತ್ತೆಯ, ತೀಕ್ಷ್ಣವಿಮರ್ಶಕರಾದ ಸಹಸ್ರಾರು ಸಭಿಕರ ಮುಂದೆ ನಿಂತು ಸುಸಂಬದ್ಧವಾಗಿ ವಿಚಾರಗಳನ್ನು ಮಂಡಿಸಲು ಬಲವಾದ ಆತ್ಮವಿಶ್ವಾಸ-ಆತ್ಮಶಕ್ತಿಗಳಿರಬೇಕಾಗುತ್ತದೆ. ಆ ಭವ್ಯ ವೇದಿಕೆಯ ಮೇಲೆ ಕುಳಿತ ಸ್ವಾಮೀಜಿ, ಅಸಂಖ್ಯಾತ ಸಭಿಕರ ಕುತೂಹಲಭರಿತ ಕಣ್ಣುಗಳು ವೇದಿಕೆಯ ಮೇಲೆ ನಿಂತು ಮಾತನಾಡುತ್ತಿರುವ ಪ್ರತಿನಿಧಿಗಳನ್ನೇ ದಿಟ್ಟಿಸಿ ನೋಡುತ್ತ, ಅವರಾಡುವ ಮಾತುಗಳಿಗೆ ಸೂಕ್ಷ್ಮ ವಾಗಿ ಪ್ರತಿಕ್ರಿಯಿಸುವ ಪರಿಯನ್ನು ಗಮನಿಸಿದರು. ಆ ವಿಶ್ವವೇದಿಕೆಯ ಮೇಲೆ ತಮ್ಮ ಅಕ್ಕಪಕ್ಕ ದಲ್ಲಿ ಅಧಿಕಾರಯುತವಾದ ಗಂಭೀರ ಮುಖಮುದ್ರೆ ಧರಿಸಿ ಕುಳಿತಿದ್ದ ಕ್ರೈಸ್ತ ಧರ್ಮಾಧಿಕಾರಿಗಳ ವೈಭವವನ್ನು ಕಂಡರು. ಈ ಹಿಂದೆ ಸ್ವಾಮೀಜಿ ಇಂತಹ ಬೃಹತ್ಸಭೆಯೊಂದನ್ನು ಉದ್ದೇಶಿಸಿ ಮಾತನಾಡುವುದಿರಲಿ, ಕಣ್ಣಿನಿಂದ ಕಂಡೂ ಇರಲಿಲ್ಲ. ಈ ವೈಭವ, ಈ ಆಡಂಬರಗಳೆಲ್ಲ ಅವರಿಗೆ ತೀರ ಹೊಸತು. ಇಂತಹ ದಿವ್ಯಾದ್ಭುತ ಕಾರ್ಯಕ್ರಮದಲ್ಲಿ ತಮ್ಮಂತಹ ಒಬ್ಬ ಬಡಸಂಸ್ಯಾಸಿಗೇನು ಕೆಲಸ! ಇಂತಹ ಅತ್ಯುನ್ನತ ಮಟ್ಟದ ಭಾಷಣಕಾರರಿಗೆ ಸರಿಸಮನಾದದ್ದು ತಮ್ಮಲ್ಲೇನಿದೆ ಎಂದು ಸ್ವಾಮೀಜಿ ಅಚ್ಚರಿಯಿಂದ ಆಲೋಚಿಸಿದರು. ಇಂತಹ ಸಭಿಕರನ್ನುದ್ದೇಶಿಸಿ ತಾವೇನು ಮಾತ ನಾಡಲು ಸಾಧ್ಯ? ಏನು ಮಾತನಾಡಿದರೆ ತಾನೆ ಏನಾದೀತು ಎಂದು ಅವರ ಎದೆ ಅಳುಕಿತು. ಆದರೆ ಅವರಿಂದಲೇ ಅದ್ಭುತವೊಂದು ನಡೆಯಲಿದೆ ಎಂಬುದು ಯಾರಿಗೂ ಗೊತ್ತಿಲ್ಲ. ಸ್ವತಃ ಸ್ವಾಮೀಜಿಗೂ ಅದರ ಪರಿವೆಯಿಲ್ಲ. ಆದರೆ ಸಮಸ್ತ ಸಭಿಕರೂ ಆ ಭವ್ಯ ವೇದಿಕೆಯ ಮೇಲೆ ಬಗೆಬಗೆಯ ಪೋಷಾಕುಗಳನ್ನು ಧರಿಸಿ ಕುಳಿತ ಪ್ರತಿನಿಧಿಗಳ ಪೈಕಿ ದಿವ್ಯತೇಜಸ್ಸಿನಿಂದ ಕೂಡಿದ ಸ್ವಾಮಿ ವಿವೇಕಾನಂದರನ್ನೇ ವಿಶೇಷ ಆಸಕ್ತಿಯಿಂದ ನೋಡುತ್ತಿದ್ದರು. ಈ ಅಪರಿಚಿತ ವ್ಯಕ್ತಿ ಯಾರಿರಬಹುದು? ಈತ ಏನು ಮಾತನಾಡಬಹುದು ಎಂದು ಕುತೂಹಲ ತಾಳಿದ್ದರು. ಸ್ವಾಮೀಜಿ ಮಾತ್ರ ಇತರ ಭಾಷಣಕಾರರ ವಾಗ್ವೈಖರಿಯನ್ನು ಕೇಳಿ ಬೆರಗಾಗಿ ಕುಳಿತಿದ್ದರು. ಎಲ್ಲರೂ ತಮ್ಮ ಭಾಷಣಗಳನ್ನು ಬರೆದುಕೊಂಡು, ಸಭೆಯನ್ನೆದುರಿಸಲು ಸಿದ್ಧರಾಗಿ ಬಂದಿದ್ದರು. ಆದರೆ ಸ್ವಾಮೀಜಿ ಭಾಷಣವನ್ನು ಸಿದ್ಧಪಡಿಸಿಕೊಂಡಿರಲೂ ಇಲ್ಲ. ಏನು ಮಾತನಾಡಬೇಕೆಂದು ಆಲೋಚಿಸಿರಲೂ ಇಲ್ಲ! ತಮ್ಮ ಮೂರ್ಖತನಕ್ಕಾಗಿ ಸ್ವಾಮೀಜಿ ತಮ್ಮನೇ ಹಳಿದುಕೊಂಡರು.

ಕಡೆಗೊಮ್ಮೆ ಅವರ ಸರದಿ ಬಂದಿತು. ಸಭಾಧ್ಯಕ್ಷರಾದ ಚಾರ್ಲ್ಸ್ ಬಾನಿಯವರು, ಭಾಷಣ ಮಾಡಲು ಸ್ವಾಮೀಜಿಯನ್ನು ಆಹ್ವಾನಿಸಿದರು. ಆದರೆ ಸ್ವಾಮೀಜಿ, “ಈಗಲೇ ಬೇಡ, ಆಮೇಲೆ ಮಾಡುತ್ತೇನೆ” ಎಂದರು. ಮತ್ತೊಬ್ಬರು ಮಾತನಾಡಿದ ಮೇಲೆ, “ಈಗ ಮಾತನಾಡಿ” ಎಂದು ಅಧ್ಯಕ್ಷರು ಹೇಳಿದಾಗ, ಮತ್ತೆ ಸ್ವಾಮೀಜಿ “ಆಮೇಲೆ ಮಾತನಾಡುತ್ತೇನೆ” ಎಂದರು. ಇದು ಹೀಗೆಯೇ ಮೂರ್ನಾಲ್ಕು ಸಲ ನಡೆದಾಗ ಅಧ್ಯಕ್ಷರು ‘ಇವರೇನು ಉಪನ್ಯಾಸ ಮಾಡುವುದೇ ಇಲ್ಲವೋ ಹೇಗೆ!’ ಎಂದು ಅಚ್ಚರಿಗೊಂಡರು. ಕಡೆಗೊಮ್ಮೆ ಅವರೀಗ ಮಾತನಾಡಲೇಬೇಕೆಂದು ಅಧ್ಯಕ್ಷರು ಹೇಳಿದರು. ಸ್ವಾಮೀಜಿಯ ಪಕ್ಕದಲ್ಲಿ ಕುಳಿತಿದ್ದ ಬಾನೆಟ್ ಮಾರಿ ಎಂಬ ಫ್ರೆಂಚ್ ಪಾದ್ರಿಗಳು ಅವರನ್ನು ಪ್ರೋತ್ಸಾಹಿಸಿ ಧೈರ್ಯ ತುಂಬಿದರು.

ಈಗ ಸ್ವಾಮೀಜಿ ಎದ್ದು ನಿಂತರು. ಅವರ ಮುಖಮಂಡಲ ಅಗ್ನಿಜ್ವಾಲೆಯಂತೆ ಬೆಳಗಿತು. ಅವರ ವಿಶಾಲವಾದ ಕಂಗಳಲ್ಲಿ ಮಿಂಚಿನ ಹೊಳಪು ಕಂಡಿತು. ಕ್ಷಣಾರ್ಧದಲ್ಲಿ ಸಮಸ್ತ ಜನಸ್ತೋಮವನ್ನು ಒಮ್ಮೆ ವೀಕ್ಷಿಸಿದರು. ವರ್ಷಾಂತರಗಳಿಂದ ಅವರಲ್ಲಿ ಒತ್ತಿ ಹಿಡಿಯಲ್ಪಟ್ಟಿದ್ದ ಪ್ರಚಂಡ ಶಕ್ತಿಯ ಜ್ವಾಲಾಮುಖಿ ಸ್ಫೋಟಿಸಿತು. ನಾಲ್ಕು ಸಹಸ್ರಕ್ಕೂ ಮಿಗಿಲಾದ ಪ್ರೇಕ್ಷಕರು ಉಸಿರಾಡುವುದನ್ನೂ ಮರೆತು ಅವರನ್ನೇ ದಿಟ್ಟಿಸಿದರು. ಆ ಬೃಹತ್ ಸಭಾಂಗಣದಲ್ಲಿ ಒಂದು ಸೂಜಿ ಬಿದ್ದರೂ ಕೇಳುವಂತಹ ನೀರವತೆ ನೆಲಸಿತು. ಈಗ ಸ್ವಾಮೀಜಿ ವಾಗ್ದೇವಿಯನ್ನು ಸ್ಮರಿಸಿ, ತಮ್ಮ ಮೇಘನಾದ ಸದೃಶ ವಾಣಿಯಲ್ಲಿ \textbf{“ಅಮೆರಿಕದ ಸೋದರ ಸೋದರಿಯರೇ!”} ಎಂದು ಸಭಿಕರನ್ನು ಸಂಬೋಧಿಸಿದರು.

ಏನದ್ಭುತ! ‘ಅಮೆರಿಕದ ಸೋದರ ಸೋದರಿಯರೆ!’ ಹೀಗೆಂದ ಸ್ವಾಮೀಜಿ ಇನ್ನೊಂದು ಶಬ್ದವನ್ನೂ ಉಚ್ಚರಿಸಿಲ್ಲ, ಆಗಲೇ ಸಮಸ್ತ ಸಭಿಕರಲ್ಲಿ ಒಂದು ನವಸ್ಫೂರ್ತಿ ಸಂಚಾರ ವಾದಂತಾಯಿತು! ಆವೇಶಭರಿತರಾಗಿ ಕರತಾಡನ ಮಾಡುತ್ತ ಸಹಸ್ರಾರು ಜನ ಎದ್ದುನಿಂತು ಬಿಟ್ಟರು! ಪ್ರತಿಯೊಬ್ಬರೂ ಉಚ್ಚಕಂಠದಿಂದ ಜಯಘೋಷ ಮಾಡುತ್ತ ಕರವಸ್ತ್ರಗಳನ್ನೂ ಹ್ಯಾಟುಗಳನ್ನೂ ಬೀಸುತ್ತಿದ್ದರು! ಇದನ್ನು ಕಂಡು ಸ್ವಾಮೀಜಿ ದಂಗಾಗಿ ನಿಂತರು. ತಮ್ಮ ಮಾತನ್ನು ಮುಂದುವರಿಸಲು ನೋಡಿದರು; ಆದರೆ ಅವರ ದನಿ ಅವರಿಗೇ ಕೇಳುವಂತಿರಲಿಲ್ಲ! ಜನಸ್ತೋಮದ ಹುಚ್ಚು ಉತ್ಸಾಹ ಅವರನ್ನು ಕೊಚ್ಚಿಕೊಂಡು ಹೋಗುವಂತಿತ್ತು. ಅಧ್ಯಕ್ಷರಾದ ಚಾರ್ಲ್ಸ್ ಬಾನಿ ಎದ್ದುನಿಂತು ತಮ್ಮ ಕೈಯನ್ನೆತ್ತಿ ಸಭಿಕರನ್ನು ಶಾಂತವಾಗಿಸಲು ಪ್ರಯತ್ನಿಸಿದರು. ಆದರೆ ಕಡೆಯಪಕ್ಷ ಎರಡು ನಿಮಿಷ ಈ ಕರತಾಡನ ಜಯಕಾರಗಳೆಲ್ಲ ಮುಂದುವರಿದುವು. ಅಂತೂ ಸಭೆ ಮತ್ತೆ ಶಾಂತವಾದ ಮೇಲೆ ಸ್ವಾಮೀಜಿ ತಮ್ಮ ಧೀರಗಂಭೀರ ಸಂಗೀತಮಯ ದನಿಯಲ್ಲಿ ಮಾತನ್ನು ಪ್ರಾರಂಭಿಸಿದರು:

“ನೀವು ನಮಗೆ ನೀಡಿದ ಉತ್ಸಾಹಯುತ ಆದರದ ಸ್ವಾಗತಕ್ಕೆ ಪ್ರತಿಕ್ರಿಯಿಸಲು ಹೊರಟಾಗ ಅನಿರ್ವಚನೀಯ ಆನಂದವೊಂದು ನನ್ನ ಹೃದಯವನ್ನು ತುಂಬುತ್ತದೆ. ಪ್ರಪಂಚದ ಅತ್ಯಂತ ಪ್ರಾಚೀನವಾದ ಸಂನ್ಯಾಸಿಗಳ ಸಂಘದ ಪರವಾಗಿ ನಾನು ನಿಮಗೆ ಕೃತಜ್ಞತೆಯನ್ನು ಸಲ್ಲಿಸುತ್ತೇನೆ. ಬೌದ್ಧಧರ್ಮ ಜೈನಧರ್ಮಗಳೂ ಯಾವುದರ ಶಾಖೆಗಳು ಮಾತ್ರವೋ ಅಂತಹ ಸಕಲ ಧರ್ಮಗಳ ಮಾತೆಯಾದ ಹಿಂದೂಧರ್ಮದ ಪರವಾಗಿ ನಾನು ನಿಮಗೆ ಕೃತಜ್ಞತೆಯನ್ನು ಸಲ್ಲಿಸುತ್ತೇನೆ. ಮತ್ತು ವಿವಿಧ ಜಾತಿಮತಗಳಿಗೆ ಸೇರಿದ ಕೋಟ್ಯನುಕೋಟಿ ಹಿಂದುಗಳ ಪರವಾಗಿ ನಾನು ನಿಮಗೆ ಕೃತಜ್ಞತೆ ಯನ್ನು ಸಲ್ಲಿಸುತ್ತೇನೆ. ಅಲ್ಲದೆ, ‘ಇಲ್ಲಿ ಕಂಡುಬರುತ್ತಿರುವ ಸಹಿಷ್ಣುತಾ ಭಾವವನ್ನು ದೂರ ದೂರದ ದೇಶಗಳಿಂದ ಬಂದಿರುವ ಈ ಪ್ರತಿನಿಧಿಗಳು ತಮ್ಮೊಂದಿಗೆ ಒಯ್ದು ಪ್ರಸಾರ ಮಾಡು ತ್ತಾರೆ’ ಎಂದು ಸಾರಿದ ಈ ವೇದಿಕೆಯ ಮೇಲಿನ ಕೆಲವು ಭಾಷಣಕಾರರಿಗೂ ನನ್ನ ಕೃತಜ್ಞತೆಗಳು ಸಲ್ಲುತ್ತವೆ.

“ಜಗತ್ತಿಗೆ ಸಹಿಷ್ಣುತೆಯನ್ನೂ ಸರ್ವಧರ್ಮ ಸ್ವೀಕಾರಭಾವವನ್ನೂ ಬೋಧಿಸಿದ ಧರ್ಮಕ್ಕೆ ಸೇರಿದವನು ನಾನೆಂಬ ಹೆಮ್ಮೆ ನನ್ನದು. ನಾವು ಸರ್ವಮತ ಸಹಿಷ್ಣುತೆಯನ್ನು ಒಪ್ಪುತ್ತೇವಷ್ಟೇ ಅಲ್ಲದೆ, ಸಕಲ ಧರ್ಮಗಳೂ ಸತ್ಯವೆಂದು ನಂಬುತ್ತೇವೆ. ಯಾವ ಧರ್ಮದ ಪವಿತ್ರ ಭಾಷೆಗೆ \eng{‘Exclusion’ (}ಹೊರಗಿಡುವುದು, ಬಹಿಷ್ಕಾರ ಹಾಕುವುದು) ಎಂಬ ಪದವನ್ನು ಅನುವಾದಿಸಲು ಸಾಧ್ಯವೇ ಇಲ್ಲವೋ ಅಂತಹ ಧರ್ಮಕ್ಕೆ ಸೇರಿದ ಹೆಮ್ಮೆ ನನ್ನದು. (ಸಭಿಕರ ಕರತಾಡನ.) ಪ್ರಪಂಚದ ಎಲ್ಲ ಧರ್ಮಗಳ ಎಲ್ಲ ರಾಷ್ಟ್ರಗಳ ಸಂಕಟ ಪೀಡಿತ ನಿರಾಶ್ರಿತರಿಗೆ ಆಶ್ರಯವನ್ನಿತ್ತ ರಾಷ್ಟ್ರಕ್ಕೆ ಸೇರಿದವನೆಂಬ ಹೆಮ್ಮೆ ನನ್ನದು. ರೋಮನ್ನರ ದಬ್ಬಾಳಿಕೆಗೆ ಗುರಿಯಾಗಿ ತಮ್ಮ ಪವಿತ್ರ ದೇವಾಲಯವು ನುಚ್ಚುನೂರಾದಾಗ ದಕ್ಷಿಣ ಭಾರತಕ್ಕೆ ವಲಸೆ ಬಂದ ಇಸ್ರೇಲಿಯರ ಒಂದು ಗುಂಪನ್ನು ನಮ್ಮ ಮಡಿಲಲ್ಲಿಟ್ಟುಕೊಂಡು ಆಶ್ರಯ ನೀಡಿದ್ದೇವೆ ಎಂದು ಹೇಳಲು ನನಗೆ ಹೆಮ್ಮೆ. ಘನ ಜರತುಷ್ಟ್ರ ಜನಾಂಗದ ಅವಶೇಷಕ್ಕೆ ಆಶ್ರಯವಿತ್ತ ಹಾಗೂ ಅವರನ್ನು ಈಗಲೂ ಪೋಷಿಸು ತ್ತಿರುವ ಧರ್ಮಕ್ಕೆ ನಾನು ಸೇರಿದವನೆಂಬುದು ನನ್ನ ಹೆಮ್ಮೆ.

“ಸೋದರರೇ, ನಾನು ಬಾಲ್ಯಾರಭ್ಯದಿಂದ ಪಠಿಸುತ್ತಿದ್ದ, ಮತ್ತು ಈಗಲೂ ಲಕ್ಷಾಂತರ ಹಿಂದೂಗಳು ಪಠಿಸುವ ಶ್ಲೋಕವೊಂದರಿಂದ ಕೆಲ ಸಾಲುಗಳನ್ನು ನಿಮಗೆ ಉದ್ಧರಿಸಿ ಹೇಳುತ್ತೇನೆ:

\begin{verse}
ತ್ರಯೀ ಸಾಂಖ್ಯಂ ಯೋಗಃ ಪಶುಪತಿಮತಂ ವೈಷ್ಣಮತಿ ।\\ಪ್ರಭಿನ್ನೇ ಪ್ರಸ್ಥಾನೇ ಪರಮಿದಮದಃ ಪಥ್ಯಮಿತಿ ಚ ॥\\ರುಚೀನಾಂ ವೈಚಿತ್ರ್ಯಾತ್ ಋಜು ಕುಟಿಲ ನಾನಾ ಪಥ ಜುಷಾಂ ।\\ನೃಣಾಮೇಕೋ ಗಮ್ಯಃ ತ್ವಮಸಿ ಪಯಸಾಂ ಅರ್ಣವ ಇವ ॥
\end{verse}

“ಎಂದರೆ, ‘ಹೇ ಭಗವಂತ! ಭಿನ್ನಭಿನ್ನ ಸ್ಥಾನಗಳಿಂದುದಿಸಿದ ನದಿಗಳೆಲ್ಲವೂ ಹರಿಯುತ್ತ ಹೋಗಿ ಕೊನೆಗೆ ಸಾಗರದೊಳಗೊಂದಾಗುವಂತೆ, ಮಾನವರು ತಮ್ಮತಮ್ಮ ವಿಭಿನ್ನ ಅಭಿರುಚಿ ಗಳಿಂದಾಗಿ ಅನುಸರಿಸುವ ಅಂಕುಡೊಂಕಿನ ಬೇರೆಬೇರೆ ದಾರಿಗಳೆಲ್ಲವೂ ಕೊನೆಗೆ ನಿನ್ನನ್ನೇ ಬಂದು ಸೇರುತ್ತವೆ.’

“ಜಗತ್ತಿನಲ್ಲಿ ಇಲ್ಲಿಯವರೆಗೆ ನಡೆಸಲ್ಪಟ್ಟಿರುವ ಮಹಾದ್ಭುತ ಸಮ್ಮೇಳನಗಳಲ್ಲೊಂದಾದ ಇಂದಿನ ಈ ಸಭೆಯು, ಭಗವದ್ಗೀತೆಯು ಬೋಧಿಸಿರುವ ಈ ಅದ್ಭುತ ತತ್ತ್ವಕ್ಕೆ ಸಾಕ್ಷಿಯಾಗಿದೆ, ಮತ್ತು ಅದನ್ನೇ ಸಾರುತ್ತದೆ:

\begin{verse}
ಯೇ ಯಥಾಂ ಮಾಂ ಪ್ರಪದ್ಯಂತೇ ತಾಂಸ್ತಥೈವ ಭಜಾಮ್ಯಹಂ\\ಮಮ ವರ್ತ್ಯಾನುವರ್ತಂತೇ ಮನುಷ್ಯಾಃ ಪಾರ್ಥ ಸರ್ವಶಃ ॥
\end{verse}

‘ಯಾರು ಯಾರು ನನ್ನಲ್ಲಿಗೆ ಯಾವಯಾವ ಮಾರ್ಗದಿಂದ ಬರುತ್ತಾರೋ ಅವರವರನ್ನು ನಾನು ಅದದೇ ಮಾರ್ಗದಿಂದ ತಲುಪುತ್ತೇನೆ. ಮಾನವರು ಅನುಸರಿಸುವ ಮಾರ್ಗಗಳೆಲ್ಲ ಕೊನೆ ಯಲ್ಲಿ ನನ್ನನ್ನೇ ಬಂದು ಸೇರುತ್ತವೆ.’

“ಗುಂಪುಗಾರಿಕೆ, ಅತಿಯಾದ ಸ್ವಮತಾಭಿಮಾನ ಹಾಗೂ ಅದರ ಘೋರ ಪರಿಣಾಮವಾದ ಧರ್ಮಾಂಧತೆಗಳು ಬಹುಕಾಲದಿಂದ ಈ ಸುಂದರ ಪೃಥ್ವಿಯನ್ನು ಆಕ್ರಮಿಸಿಕೊಂಡಿವೆ. ಅವು ಭೂಮಿಯನ್ನು ಹಿಂಸೆಯಿಂದ ತುಂಬಿ, ಮತ್ತೆಮತ್ತೆ ಮಾನವನ ರಕ್ತದಿಂದ ತೋಯಿಸಿವೆ; ನಾಗರಿಕತೆಗಳನ್ನು ನಾಶಮಾಡಿವೆ, ದೇಶದೇಶಗಳನ್ನೇ ನಿರಾಶೆಯ ಕೂಪಕ್ಕೆ ತಳ್ಳಿವೆ. ಆ ಘೋರ ರಾಕ್ಷಸತನವಿಲ್ಲದಿದ್ದಲ್ಲಿ ಮಾನವನ ಸಮಾಜವು ಈಗಿರುವುದಕ್ಕಿಂತಲೂ ಎಷ್ಟೋ ಹೆಚ್ಚು ಮುಂದುವರಿಯುತ್ತಿತ್ತು. ಆದರೆ ಈಗ ಆ ರಾಕ್ಷಸತನದ ಅಂತ್ಯಕಾಲ ಸಮೀಪಿಸಿದೆ. ಈ ಸಮ್ಮೇಳನದ ಪ್ರತಿನಿಧಿಗಳ ಗೌರವಾರ್ಥವಾಗಿ ಇಂದು ಬೆಳಿಗ್ಗೆ ಮೊಳಗಿದ ಘಂಟಾನಾದವು ಎಲ್ಲ ಮತಾಂಧತೆಗೆ ಮೃತ್ಯುಘಾತವನ್ನೀಯುವುದೆಂದು ಆಶಿಸುತ್ತೇನೆ. (ಕರತಾಡನ.) ಮತ್ತು ಅದು ಖಡ್ಗ-ಲೇಖನಿಗಳ ಮೂಲಕ ಸಂಭವಿಸುವ ಹಿಂಸಾದ್ವೇಷಗಳಿಗೆ, ಹಾಗೂ ಒಂದೇ ಗುರಿಯೆಡೆಗೆ ಸಾಗುತ್ತಿರುವ ಪಥಿಕರೊಳಗಿನ ಅಸಹನೆ-ಮನಸ್ತಾಪಗಳಿಗೆ ಮೃತ್ಯುಘಾತವನ್ನೀಯುವುದೆಂದು ಹೃತ್ಪೂರ್ವಕವಾಗಿ ಆಶಿಸುತ್ತೇನೆ.”

ಹೀಗೆ ಸ್ವಾಮೀಜಿ ತಮ್ಮ ಉಪನ್ಯಾಸವನ್ನು ಮುಗಿಸಿ ಕುಳಿತುಕೊಳ್ಳುತ್ತಿದ್ದಂತೆ, ಮತ್ತೊಮ್ಮೆ ಹರ್ಷೋದ್ಗಾರ ಕರತಾಡನಗಳು ಸಿಡಿಲಿನಂತೆ ಮೊಳಗಿದುವು. ಕೆಲವೇ ಮಾತುಗಳಲ್ಲಿ ಸ್ವಾಮೀಜಿ ತಮ್ಮ ಭಾವನೆಗಳನ್ನೆಲ್ಲ ಅತ್ಯಂತ ಸ್ಪಷ್ಟವಾಗಿ ವ್ಯಕ್ತಪಡಿಸಿದ್ದರು. ಅವರ ಈ ಪುಟ್ಟ ಭಾಷಣ ಅತ್ಯಂತ ಪರಿಣಾಮಕಾರಿಯಾಗಿತ್ತು. ಅವರ ಮಾತುಗಳು, ಮರುಭೂಮಿಯ ಪ್ರಯಾಣಿಕನಿಗೆ ಎದುರಾದ ಸರೋವರದ ತಿಳಿನೀರಿನಂತೆ, ಜನರಿಗೆ ತಂಪನ್ನೆರೆದು ಹೊಸ ಭರವಸೆಯನ್ನು ಮೂಡಿಸಿದ್ದುವು. ಸರ್ವಧರ್ಮಸಮ್ಮೇಳನದ ನಿಜವಾದ ಉದ್ದೇಶವನ್ನು ಸ್ವಾಮೀಜಿ ಪ್ರತಿಯೊಬ್ಬ ರಿಗೂ ಮನಮುಟ್ಟಿಸಿದ್ದರು. ಸರ್ವಧರ್ಮಸ್ಥಾಪಕಾಚಾರ್ಯರೂ ಸರ್ವಧರ್ಮಸ್ವರೂಪಿಗಳೂ ಆದ ಶ್ರೀರಾಮಕೃಷ್ಣ ಪರಮಹಂಸರ ಶಿಷ್ಯಾಗ್ರಣಿಯಾದ ಸ್ವಾಮಿ ವಿವೇಕಾನಂದರು, ತಮ್ಮ ಗುರು ದೇವನ ಸಂದೇಶವನ್ನು ಈ ವಿಶ್ವವೇದಿಕೆಯ ಮೇಲೆ ಅತಿ ಯಶಸ್ವಿಯಾಗಿ ಸಾರಿದ್ದರು! ‘ಎಷ್ಟು ಮತವೋ ಅಷ್ಟು ಪಥ’ ಎಂಬ ಸತ್ಯವನ್ನು ದೀರ್ಘ ಸಾಧನೆಯ ಮೂಲಕ ಸಾಕ್ಷಾತ್ಕರಿಸಿಕೊಂಡ ಆ ದಕ್ಷಿಣೇಶ್ವರದ ದೇವಮಾನವನ ಪದತಲದಲ್ಲಿ ಕುಳಿತು ಕಲಿತಪಾಠವನ್ನು ಸ್ವಾಮೀಜಿ ತಪ್ಪಿಲ್ಲದೆ ಒಪ್ಪಿಸಿದ್ದರು!

ಆದರೆ, ಸ್ವಾಮೀಜಿಯ ಭಾಷಣಕ್ಕೆ ವ್ಯಕ್ತವಾದ ಮೆಚ್ಚುಗೆಗಿಂತ ಹೆಚ್ಚಾಗಿ, ಅವರು ಸಭಿಕರನ್ನು ‘ಅಮೆರಿಕದ ಸೋದರ ಸೋದರಿಯರೇ ’ ಎಂದು ಸಂಬೋಧಿಸಿದಾಗ ಅದಕ್ಕೆ ತೋರಿದ ಪ್ರತಿಕ್ರಿಯೆ ಅತ್ಯದ್ಭುತವಾದದ್ದು. ಯಾವ ಆಡಂಬರವೂ ನಾಟಕೀಯತೆಯೂ ಇಲ್ಲದ ಈ ಮೂರು ಸರಳ ಶಬ್ದಗಳಿಗೆ ಪ್ರತಿಕ್ರಿಯೆಯಾಗಿ ಸಭೆಗೆ ಸಭೆಯೇ ಹುಚ್ಚೆದ್ದು ಹರ್ಷೋದ್ಗಾರ ಮಾಡಿತಲ್ಲ! ಅದೂ ಪೂರ್ಣ ಎರಡು ನಿಮಿಷಗಳ ಕಾಲ! ಇದಕ್ಕೆ ಕಾರಣವೇನಿದ್ದಿರಬಹುದು?

ಬಹುಶಃ ಹೀಗೆ ಪ್ರತಿಕ್ರಿಯಿಸಿದ್ದ ಈ ಜನಗಳಿಗೇ ಕಾರಣವೇನೆಂದು ಗೊತ್ತಿದ್ದಿರಲಾರದು! ಈ ಸಭೆಯಲ್ಲಿ ಹಾಜರಿದ್ದ, ಹಾಗೂ ಮುಂದೆ ಲಾಸ್ ಏಂಜಲಿಸ್​ನಲ್ಲಿ ಸ್ವಾಮೀಜಿಯ ಆತಿಥೇಯಳಾದ ಶ್ರೀಮತಿ ಎಸ್. ಕೆ. ಬ್ಲಾಜೆಟ್ ಎಂಬವಳು ಹೇಳುತ್ತಾಳೆ: “ಯಾವಾಗ ಆ ನವಯುವಕ ಎದ್ದು ನಿಂತು, ‘ಅಮೆರಿಕದ ಸೋದರ ಸೋದರಿಯರೇ’ ಎಂದು ಸಂಭೋಧಿಸಿದನೋ ಆಗ ಸಭೆ ಯಲ್ಲಿದ್ದ ಏಳುಸಾವಿರ ಜನ ತಮಗರಿವಿಲ್ಲದ ಯಾವುದಕ್ಕೋ ಗೌರವ ಸೂಚಿಸುವಂತೆ ಎದ್ದು ನಿಂತುಬಿಟ್ಟರು. ತಾವು ಯಾವುದಕ್ಕೆ ಗೌರವವನ್ನು ಅರ್ಪಿಸುತ್ತಿರುವುದೆಂದು ಅವರಿಗೇ ಗೊತ್ತಿಲ್ಲ! ಕಡೆಗೆ ಜನರು ಶಾಂತವಾಗಿ ಕುಳಿತುಕೊಳ್ಳುವ ವೇಳೆಗೆ ನೋಡುತ್ತೇನೆ–ಹಲವಾರು ಸ್ತ್ರೀಯರು ಆತನನ್ನು ಹತ್ತಿರದಿಂದ ನೋಡಲು ಬೆಂಚುಗಳನ್ನು ದಾಟಿಕೊಂಡು ಮುಂದೆ ಧಾವಿಸಿ ಬರುತ್ತಿ ದ್ದಾರೆ! ಇದನ್ನು ಕಂಡು ನಾನು ನನ್ನಷ್ಟಕ್ಕೆ ಅಂದುಕೊಂಡೆ, ‘ಮಗು, ಇಂತಹ ಆಕ್ರಮಣವನ್ನು ತಡೆದುಕೊಳ್ಳಲು ನಿನ್ನಿಂದ ಸಾಧ್ಯವಾದರೆ ನಿಜಕ್ಕೂ ನೀನು ದೇವರೇ ಸರಿ!’ ಎಂದು.”

ಸಭಿಕರ ಈ ಪ್ರಚಂಡ ಪ್ರತಿಕ್ರಿಯೆಗೆ ಕಾರಣವನ್ನು ಶ್ರೀಮತಿ ಮೇರಿ ಲೂಯಿಸ್ ಬರ್ಕ್ \eng{(Swami Vivekananda In The West-New Discoveries} ಎಂಬ ಗ್ರಂಥಮಾಲೆಯ ಕರ್ತೃ) ಹೀಗೆ ಊಹಿಸುತ್ತಾರೆ: “ನಮಗೆ ತಿಳಿದಿರುವಂತೆ, ಭಾಷಣಕ್ಕೆ ಅಬ್ಬರತದ ಕರತಾಡನವನ್ನು ಗಿಟ್ಟಿಸಿಕೊಂಡ ಭಾಷಣಕಾರರಲ್ಲಿ ಸ್ವಾಮೀಜಿಯೇ ಮೊದಲಿಗರೇನಾಗರಲಿಲ್ಲ. ಇನ್ನು ಕೆಲವರ ಭಾಷಣಕ್ಕೂ ಜನ ಸ್ವಲ್ಪ ಜೋರಾಗಿಯೇ ಚಪ್ಪಾಳೆ ತಟ್ಟಿದ್ದರು. ಇನ್ನು ಆಧ್ಯಾತ್ಮಿಕ ತತ್ತ್ವಗ್ರಹಣ ಸಾಮರ್ಥ್ಯದ ದೃಷ್ಟಿಯಿಂದ ನೋಡಿದರೆ, ಆ ಶ್ರೋತೃವರ್ಗ ಸಾಧಾರಣಮಟ್ಟದ್ದು. ಅದರ ಆಧ್ಯಾತ್ಮಿಕ ವ್ಯಾಕುಲತೆಯೆಂಬುದೇನಿದ್ದರೂ ಭೋಗ ಸಂಪ್ರದಾಯದ ಪದರಗಳಡಿಯಲ್ಲಿ ಮುಚ್ಚಿ ಹೋಗಿತ್ತು. ಅಥವಾ, ಹಿರಿಮೆಯೆಂದರೆ ಆಧ್ಯಾತ್ಮಿಕ ಹಿರಿಮೆಯೊಂದೇ ಎಂದು ಭಾವಿಸಲು ಮತ್ತು ನೋಡಿದ ಮಾತ್ರಕ್ಕೆ ಒಬ್ಬನ ಆಧ್ಯಾತ್ಮಿಕ ಸತ್ವವನ್ನು ಗುರುತುಹಿಡಿಯಲು ಅದೇನು ಭಾರತವೆ! ಸ್ವಾಮೀಜಿಯ ಬಾಯಿಯಿಂದ ಹೊರಬಿದ್ದ ಮೊದಲ ಶಬ್ದಗಳಿಗೇ ಸಮ್ಮೇಳನದ ಆ ಶ್ರೋತೃ ವರ್ಗ ಅಷ್ಟೊಂದು ಉತ್ಸಾಹದಿಂದ ಪ್ರತಿಕ್ರಿಯಿಸಿದ್ದೇಕೆ ಎಂದು, ಶ್ರೀಮತಿ ಬ್ಲಾಜೆಟ್ ಹೇಳು ವಂತೆ, ಒಟ್ಟಾರೆಯಾಗಿ ಆ ಶ್ರೋತೃಗಳಿಗೇ ತಿಳಿದಿರಲಾರದು. ಇತರರ ಭಾಷಣಗಳಿಗೆ ಪ್ರೋತ್ಸಾಹ ಕರ ಪ್ರತಿಕ್ರಿಯೆ ನೀಡಿದ್ದಕ್ಕೆ ಸ್ಪಷ್ಟ ಕಾರಣಗಳಿದ್ದುವು–ರಾಜಕೀಯ ಅಥವಾ ಧಾರ್ಮಿಕ ಸಹಾನು ಭೂತಿ, ಭಾಷಣಕಾರನ ಪೂರ್ವಪರಿಚಯ, ತಮ್ಮ ರಾಷ್ಟ್ರದ ಪಾಪಕೃತ್ಯಕ್ಕೆ ಪಶ್ಚಾತ್ತಾಪ, ಇತ್ಯಾದಿ. ಆದರೆ ಸ್ವಾಮೀಜಿಯ ವಿಷಯದಲ್ಲಿ ಇಂಥದೇನೂ ಇರಲಿಲ್ಲ. ಅಥವಾ, ಅವರು ‘ಸೋದರ ಸೋದರಿಯರೆ’ ಎಂದು ಸಂಬೋಧಿಸಿದ್ದರಷ್ಟರಿಂದಲೇ ಜನ ಸ್ಫೂರ್ತಿಗೊಂಡು ಹಾಗೆ ಕಿವಿಗಡ ಚಿಕ್ಕುವಂತೆ ಚಪ್ಪಾಳೆ ಹೊಡೆದರು ಎನ್ನುವಂತೆಯೂ ಇಲ್ಲ. ಏಕೆಂದರೆ, ವಿಶ್ವಭಾತೃತ್ವದ ಮಾತು ಗಳನ್ನು ಬೆಳಗಿನಿಂದ ಮಧ್ಯಾಹ್ನದವರೆಗೂ ಎಲ್ಲರೂ ಆಡಿದ್ದರು. ಹಾಗಾದರೆ ಜನರ ಈ ಪ್ರತಿ ಕ್ರಿಯೆಯು ಸ್ವಾಮೀಜಿಯ ಮಾತುಗಳೊಂದಿಗೆ ಹೊರಹೊಮ್ಮಿದ ಯಾವುದೋ ಅವ್ಯಕ್ತ ಶಕ್ತಿ ಯಿಂದಲೇ ಪ್ರಚೋದಿತಗೊಂಡಿರಬೇಕಲ್ಲವೆ? ಅಮೆರಿಕದ ಇಷ್ಟು ಭಾರೀ ಸಂಖ್ಯೆಯ ಸಭಿಕರನ್ನು ಉದ್ದೇಶಿಸಿ ಸ್ವಾಮೀಜಿ ಮಾತನಾಡಿದ್ದು ಇದೇ ಮೊದಲ ಸಲ, ಮತ್ತು ಸ್ವತಃ ಸ್ವಾಮೀಜಿಯೂ ಅತೀವ ಭಾವಪರವಶರಾಗಿದ್ದರು ಎಂಬುದನ್ನು ಗಮನಿಸಿದಾಗ, ನಿಸ್ಸಂದೇಹವಾಗಿ ಅನ್ನಿಸುತ್ತದೆ– ಅವರು ಸಮ್ಮೇಳನದ ವೇದಿಕೆಯ ಮೇಲೆ ಎದ್ದುನಿಂತ ಆ ಕ್ಷಣದಲ್ಲಿ ಅವರು ಸ್ವಸ್ವರೂಪದಾಳ ದಲ್ಲಿದ್ದ ಮಹಾಶಕ್ತಿಯು ಪರಿಪೂರ್ಣವಾಗಿ ಜಾಗೃತವಾಗಿತ್ತು; ಮತ್ತು ಅಲ್ಲಿ ನೆರೆದಿದ್ದ ಸಮಸ್ತ ಸ್ತ್ರೀಪುರುಷ ಸಮೂಹದೊಂದಿಗೆ ತಾವು ಆಧ್ಯಾತ್ಮಿಕವಾಗಿ ಅನನ್ಯರೆಂಬ ಭಾವ ಅವರ ಸಮಸ್ತ ವ್ಯಕ್ತಿತ್ವವನ್ನೇ ಆಕ್ರಮಿಸಿ ಅವರ ದನಿಯಲ್ಲಿ ಸ್ಪಂದನಗೊಳ್ಳುತ್ತಿತ್ತು. ಮತ್ತು ಶ್ರೋತೃಗಳೊಳಗೂ ಆ ಭಾವವು ರಭಸದಿಂದ ಪ್ರವೇಶ ಮಾಡಿಬಿಟ್ಟಿತು, ಎಂದು.”

ಒಟ್ಟಿನಲ್ಲಿ, ಸ್ವಾಮೀಜಿ ‘ಅಮೆರಿಕದ ಸೋದರ ಸೋದರಿಯರೇ’ ಎಂದು ಸಂಬೋಧಿಸಿದಾಗ ಅದರಲ್ಲಿ ಹೃತ್ಪೂರ್ವಕತೆಯಿತ್ತು, ಶ್ರದ್ಧೆಯಿತ್ತು. ಆದ್ದರಿಂದಲೇ ಆ ಕರೆ ಸಮಸ್ತ ಸಭಿಕರ ಹೃದಯ ವನ್ನು ಮುಟ್ಟಿತು. ಅಲ್ಲದೆ ಇಲ್ಲಿ ಮತ್ತೊಂದು ಪ್ರಮುಖ ಅಂಶವನ್ನು ಗಮನಿಸಬೇಕು– ಏನೆಂದರೆ, ಸ್ವಾಮೀಜಿಯ ವ್ಯಕ್ತಿತ್ವವು ಆತ್ಮಸಾಕ್ಷಾತ್ಕಾರದಿಂದ ಪರಿಪೂರ್ಣಗೊಂಡ ವ್ಯಕ್ತಿತ್ವ; ಜೊತೆಗೆ ಶ್ರೀರಾಮಕೃಷ್ಣರು ಧಾರೆಯೆರೆದುಕೊಟ್ಟ ಸಮಸ್ತ ಆಧ್ಯಾತ್ಮಿಕ ಶಕ್ತಿಯನ್ನೊಳಗೊಂಡ ವ್ಯಕ್ತಿತ್ವ. ಇಂತಹ ವ್ಯಕ್ತಿತ್ವದಿಂದ ಸ್ವಾಮೀಜಿ, ಸಭಿಕರನ್ನು ‘ಸೋದರ ಸೋದರಿಯರೆ’ ಎಂದು ಸಂಬೋಧಿಸಿದಾಗ ಆ ಶಬ್ದತರಂಗದ ಮೂಲಕ ಅವರ ಆಧ್ಯಾತ್ಮಿಕ ಶಕ್ತಿಯೂ ತರಂಗತರಂಗ ವಾಗಿ ಹರಿದು ಪ್ರತಿಯೊಬ್ಬನ ಹೃದಯವನ್ನೂ ಸ್ಪಂದಿಸುವಂತೆ ಮಾಡಿತು. ಆ ಬಗೆಯ ಶಕ್ತಿ ಸಂಚಾರದ ಪ್ರಭಾವಕ್ಕೊಳಗಾದ ಸಭಿಕರು ಭಾವಾವೇಶಭರಿತರಾಗಿ, ಆ ಭಾವಾವೇಶದ ಆವೇಗ ವನ್ನು ಪ್ರಚಂಡ ಕರತಾಡನ-ಹರ್ಷೋದ್ಗಾರಗಳ ಮೂಲಕ ವ್ಯಕ್ತಪಡಿಸಿದರು.

ಹಿಂದಿನ ದಿನದವರೆಗೂ ಹೆಚ್ಚುಕಡಿಮೆ ಅನಾಮಧೇಯರಾಗಿದ್ದ ಸ್ವಾಮೀಜಿ ಇದ್ದಕ್ಕಿದ್ದಂತೆ ತಾರಾ ಪದವಿಗೇರಿದ್ದರು. ಎಲ್ಲೆಲ್ಲಿ ನೋಡಿದರೂ ಎಲ್ಲೆಲ್ಲಿ ಕೇಳಿದರೂ ವಿವಾಕನಂದರ ಹೆಸರೇ, ವಿವೇಕಾನಂದರ ವಿಷಯವೇ! ಮರುದಿನದ ಪತ್ರಿಕೆಗಳು ಅವರ ಭಾವಚಿತ್ರವನ್ನು ಮುಖಪುಟದಲ್ಲಿ ಮುದ್ರಿಸಿ, ಅವರ ಭಾಷಣವನ್ನು ವಿವರವಾಗಿ ಪ್ರಕಟಿಸಿದುವು. ಅಲ್ಲದೆ ಅವರ ಭಾಷಣವನ್ನೂ, ವ್ಯಕ್ತಿತ್ವವನ್ನೂ ಮನಸಾರೆ ಪ್ರಶಂಸಿಸಿದುವು. ಸಾವಿರಾರು ಜನ ಅವರನ್ನು ಮಾತನಾಡಿಸಿ ಕೈಕುಲು ಕಲು, ಇಲ್ಲವೆ ಅವರ ಬಟ್ಟೆಯ ಅಂಚನ್ನು ಮುಟ್ಟಲು, ಇಲ್ಲವೆ ದೂರದಿಂದಲಾದರೂ ಅವರನ್ನು ನೋಡಿ ತೃಪ್ತರಾಗಲು ಹಾತೊರೆಯುತ್ತಿದ್ದರು. ಕೋಟ್ಯಧಿಪತಿಗಳು ಅವರನ್ನು ತಮ್ಮ ಮನೆಗೆ ಆಹ್ವಾನಿಸುವ ಹೆಮ್ಮೆಗಾಗಿ ಸ್ಪರ್ಧಿಸಿದರು.

ಸಮ್ಮೇಳನದ ಉದ್ಘಾಟನೆಯ ದಿನ ಸಭೆಯಲ್ಲಿ ಹಾಜರಿದ್ದ ಶ್ರೀಮತಿ ಆ್ಯನ್ನಿಬೆಸೆಂಟರು ತಾವು ಕಂಡ ಅದ್ಭುತವನ್ನು ಮುಂದೊಮ್ಮೆ ಬಣ್ಣಿಸುತ್ತಾರೆ:

“ಶಿಕಾಗೋ ನಗರದ ದಟ್ಟವಾತಾವರಣದ ಮಧ್ಯದಲ್ಲಿ ಪ್ರಖರ ಪೀತಾರುಣ ವಸ್ತ್ರಧಾರಿಯಾಗಿ ಭಾರತೀಯ ಸೂರ್ಯನಂತೆ ಕಂಗೊಳಿಸುತ್ತಿದ್ದ ತೇಜೋಮಯ ಮೂರ್ತಿ; ಸಿಂಹಸದೃಶ ಶಿರ; ಇತರರ ಅಂತರಂಗವನ್ನು ಹೊಕ್ಕು ನೋಡಬಲ್ಲ ದೃಷ್ಟಿ, ಚುರುಕಿನ ಚಲನವಲನ–ಇದು, ಸಮ್ಮೇಳನದ ಪ್ರತಿನಿಧಿಗಳಿಗಾಗಿ ಕಾದಿರಿಸಿದ್ದ ಕೋಣೆಯಲ್ಲಿ ಸ್ವಾಮಿ ವಿವೇಕಾನಂದರನ್ನು ಮೊಟ್ಟ ಮೊದಲು ಭೇಟಿಯಾದಾಗ ನಾನು ಕಂಡ ದೃಶ್ಯ. ಅವರನ್ನು ಜನ ‘ಸಂನ್ಯಾಸಿ’ ಎನ್ನುತ್ತಿದ್ದರು; ಹೌದು, ಅವರೊಬ್ಬ ಯೋಧ-ಸಂನ್ಯಾಸಿಯೇ. ಆದರೆ ಪ್ರಥಮ ನೋಟದಲ್ಲಿ, ನನಗೆ ಅವರು ಸಂನ್ಯಾಸಿ ಎನ್ನುವುದಕ್ಕಿಂತ ಹೆಚ್ಚಾಗಿ ಯೋಧನಂತೆಯೇ ಕಂಡರು! ಏಕೆಂದರೆ ಆಗ ಅವರ ವ್ಯಕ್ತಿತ್ವದಲ್ಲಿ ರಾಷ್ಟ್ರಾಭಿಮಾನ-ಸ್ವಜನಾಭಿಮಾನಗಳು ಕಿಡಿಗಳಂತೆ ಸಿಡಿಯುತ್ತಿದ್ದುವು. ಜಗತ್ತಿನ ಪ್ರಾಚೀನತಮ-ಜೀವಂತ ಧರ್ಮದ (ಹಿಂದೂ ಧರ್ಮದ) ಆ ಪ್ರತಿನಿಧಿಯು ಎಂದಿಗೂ ತಲೆ ಬಾಗಲು ಸಿದ್ಧನಿರಲಿಲ್ಲ. ತನ್ನ ಸನಾತನ ಹಿಂದೂ ಧರ್ಮವು ಇಲ್ಲಿನ (ಅಮೆರಿಕೆಯ) ‘ಶ್ರೇಷ್ಠತಮ’ ಧರ್ಮಕ್ಕಿಂತಲೂ ತೃಣಮಾತ್ರವಾದರೂ ಕನಿಷ್ಠವೆಂಬುದನ್ನು ಒಪ್ಪಿಕೊಳ್ಳಲು ಆತನು ಸಿದ್ಧನಿರ ಲಿಲ್ಲ. ಭಾರತಾಂಬೆಯ ಸುಪುತ್ರನೂ ದೂತನೂ ಆದ ಆತನ ಸಮ್ಮುಖದಲ್ಲಿ, ಸೊಕ್ಕಿದ ಪಾಶ್ಚಾತ್ಯ ರಾಷ್ಟ್ರಗಳ ಮುಂದೆ ಭಾರತವು ಅಪಮಾನಕ್ಕೀಡಾಗಲು ಸಾಧ್ಯವಿರಲಿಲ್ಲ. ಆತನು ಭಾರತಮಾತೆಯ ಸಂದೇಶವನ್ನು ತಂದಿದ್ದನು; ಆಕೆಯ ಹೆಸರಿನಲ್ಲಿ ಮಾತನಾಡಿದನು; ಜನರು ಆಕೆಯ ಹಿರಿಮೆ ಯನ್ನು ಸ್ಮರಿಸುವಂತೆ ಮಾಡಿದನು. ವೀರ್ಯವಂತನೂ, ಬಲಾಢ್ಯನೂ ಆದ ಅವನು ಪುರುಷಸಿಂಹ ನಂತೆ ನಿಂತು ಛಲದಿಂದ ತನ್ನ ಕಾರ್ಯವನ್ನು ಸಾಧಿಸಲು ಸಮರ್ಥನಾದನು.

“ಆದರೆ ವೇದಿಕೆಯ ಮೇಲೆ ಎದ್ದುನಿಂತಾಗ ಸ್ವಾಮೀಜಿಯ ವ್ಯಕ್ತಿತ್ವದ ಮತ್ತೊಂದು ಮುಖ ಪ್ರಕಟವಾಯಿತು. ಅದೇ ಗಾಂಭೀರ್ಯವೂ ಅವರ ಹುಟ್ಟುಗುಣವಾದ ಆತ್ಮವಿಶ್ವಾಸ-ಆತ್ಮ ಶಕ್ತಿಯೂ ಕಾಣುತ್ತಿದ್ದುವು; ಆದರೆ ಅವರು ತಂದಿದ್ದ ಆಧ್ಯಾತ್ಮಿಕ ಸಂದೇಶದ ಆತುಲ ಸೌಂದರ್ಯದ ಮುಂದೆ, ಭಾರತದ ಜೀವನಾಡಿಯಾದ ಅದ್ಭುತ ಆತ್ಮತತ್ತ್ವದ ಪರಮೋದಾತ್ತ ಸಂದೇಶದ ಮುಂದೆ, ಆ ಗುಣಗಳು ಮರೆಯಾಗಿದ್ದುವು. ಮಂತ್ರಮುಗ್ಧವಾದ ಅಗಾಧ ಜನ ಸ್ತೋಮವು, ಅವರ ಕಂಠದಿಂದ ಹೊಮ್ಮಿಬಂದ ಒಂದೇ ಒಂದು ಶಬ್ದವೂ ವ್ಯರ್ಥವಾಗದಂತೆ ಎಚ್ಚರಿಕೆಯಿಂದ ತೆರೆದ ಕಿವಿಗಳಿಂದ ಆಲಿಸಿತು. ಆ ಮಹಾ ಸಭಾಭವನದಿಂದಾಚೆ ಬರುತ್ತಿದ್ದಂತೆ ಒಬ್ಬ ಹೇಳಿದ, ‘ಅವರೊಬ್ಬ “ಅನಾಗರಿಕ ವಿಧರ್ಮಿ”! ನಾವು ಅವರ ದೇಶಕ್ಕೆ ಧರ್ಮಪ್ರಚಾರಕ ರನ್ನು ಕಳಿಸಿಕೊಡುತ್ತೇವೆ! ಆದರೆ ನಿಜಕ್ಕೂ, ಅವರು ನಮ್ಮಲ್ಲಿಗೆ ಧರ್ಮಪ್ರಚಾರಕರನ್ನು ಕಳಿಸಿ ಕೊಡುವುದೇ ಹೆಚ್ಚು ಸೂಕ್ತ’.”

ಹೀಗೆ ಒಂದೇ ದಿನದಲ್ಲಿ ಸ್ವಾಮೀಜಿ ಕೀರ್ತಿಶಿಖರವನ್ನೇರಿದ್ದರು. ಆದರೆ ಅದರಿಂದಾಗಿ ಅವರಿಗೆ ಒಂದಿನಿತೂ ಸಂತೋಷವಾಗಲಿಲ್ಲ. ಇನ್ನು ಅಹಂಕಾರದ ಮಾತೇ ಇಲ್ಲ. ಸಮಾಜದ ಶ್ರೀಮಂತರು ಪ್ರತಿಷ್ಠಿತರೆಲ್ಲ ಅವರ ಆಜ್ಞಾನುವರ್ತಿಗಳಾಗಿದ್ದರು. ಆದರೆ ಈ ಎಲ್ಲ ವೈಭವ-ಆಡಂಬರಗಳು ಅವರನ್ನು ಸೋಕಲೂ ಇಲ್ಲ. ಸದಾ ಅವರ ಮನದಲ್ಲೊಂದೇ ಚಿಂತೆ–ತನ್ನ ಮಾತೃಭೂಮಿಯ ಕೋಟ್ಯಂತರ ದೀನಾರ್ತ ದರಿದ್ರರನ್ನು ಕುರಿತಾದ ಚಿಂತೆ. ಸಮ್ಮೇಳನದ ಉದ್ಘಾಟನೆಯ ದಿನ, ತಮ್ಮ ಮೊದಲ ಭಾಷಣವನ್ನು ಮಾಡಿ ಜಗತ್ಪ್ರಸಿದ್ಧರಾದಂದಿನ ರಾತ್ರಿ ಅವರು ಶ್ರೀಮಂತನೊಬ್ಬನ ಅತಿಥಿಯಾಗಿದ್ದರು. ತಮಗಾಗಿ ಬಿಟ್ಟುಕೊಟ್ಟಿದ್ದ ವೈಭವೋಪೇತವಾದ ಕೊಠಡಿಯಲ್ಲಿ ಹಂಸ ತೂಲಿಕಾತಲ್ಪದಂತಹ ಶಯ್ಯೆಯ ಮೇಲೆ ಮಲಗಿದರು; ತಕ್ಷಣವೇ ಅವರ ಹೃದಯದಲ್ಲಿ ಅಡಕವಾಗಿದ್ದ ಭಾವನೆಗಳೆಲ್ಲ ಒತ್ತರಿಸಿ ಬಂದುವು. ‘ಓ ತಾಯಿ! ನನಗೇಕೆ ಈ ಕೀರ್ತಿ? ನನಗೇಕೆ ಈ ವೈಭೋಗ? ಅಲ್ಲಿ ನನ್ನ ಸೋದರರಾದ ಅಸಂಖ್ಯಾತ ಜನ, ಒಂದು ಹೊತ್ತಿನ ಊಟಕ್ಕೂ ಗತಿಯಿಲ್ಲದೆ ಪಶುಗಳಂತೆ ಜೀವಿಸುತ್ತಿರುವಾಗ ಈ ಹೆಸರು-ಕೀರ್ತಿಗಳಿಂದ ನನಗೆ ಆಗಬೇಕಾ ದ್ದೇನು?’ ಎಂದು ಹಲುಬಿದರು. ಉಕ್ಕಿ ಹರಿದ ಕಣ್ಣೀರು ದಿಂಬನ್ನು ತೋಯಿಸಿತು. ಅಸಹನೀಯ ಸಂಕಟದಿಂದಾಗಿ, ಆ ರಾಜಯೋಗ್ಯ ಹಾಸಿಗೆಯು ಶರಶಯ್ಯೆಯಂತೆ ಭಾಸವಾಯಿತು. ಅದರ ಮೇಲೆ ಮಲಗಲಾರದೆ ಸ್ವಾಮೀಜಿ, ನೆಲದ ಮೇಲೆ ಉರುಳಿಕೊಂಡರು. ‘ಅಮ್ಮಾ, ಈ ಹೆಸರು- ಕೀರ್ತಿ-ಗೌರವಗಳನ್ನೆಲ್ಲ ನೀನೇ ಇಟ್ಟುಕೊ. ಬದಲಾಗಿ ನಾನು ನನ್ನ ದೇಶಬಾಂಧವರ ಶೋಕವನ್ನು ಶಮನ ಮಾಡಲಾಗುವಂತೆ ನನಗೆ ನೆರವಾಗು’ ಎಂದು ಪ್ರಾರ್ಥಿಸಿದರು. ತಮ್ಮನ್ನರಸಿ ಬಂದ ಗೌರವ-ಮನ್ನಣೆಗಳಿಗೆ ಸ್ವಾಮೀಜಿಯ ಪ್ರತಿಕ್ರಿಯೆ ಇಂಥದು!

ಸೆಪ್ಟೆಂಬರ್ ೧೫ರಂದು ಸಮ್ಮೇಳನದಲ್ಲಿ ಸ್ವಾಮೀಜಿ “ನಮ್ಮಲ್ಲೇಕೆ ಒಮ್ಮತವಿಲ್ಲ” ಎಂಬ ವಿಷಯವಾಗಿ ಒಂದು ಪುಟ್ಟ ಭಾಷಣ ಮಾಡಿದರು. ಬಾವಿಯಲ್ಲೇ ಹುಟ್ಟಿ ಬೆಳೆದು, ಆಚೆಯ ಪ್ರಪಂಚವನ್ನೇ ಕಾಣದ ಕಪ್ಪೆಯ ಕತೆಯನ್ನು ಅವರು ಉದಾಹರಿಸಿದರು. ಸಾಗರದಿಂದ ಬಂದ ಮತ್ತೊಂದು ಕಪ್ಪೆ ಈ ಬಾವಿಯ ಕಪ್ಪೆಯನ್ನು ಸಂಧಿಸಿ ಸಾಗರದ ಬಗ್ಗೆ ತಿಳಿಸಿದಾಗ, ಈ ‘ಕೂಪ ಮಂಡೂಕ’ ತನ್ನ ಬಾವಿಗಿಂತಲೂ ದೊಡ್ಡದಾದ ಜಲಾಶಯ ಮತ್ತೊಂದಿರಲಾರದೆಂದು ವಾದಿಸಿ, ಸಾಗರದ ಕಪ್ಪೆಯನ್ನು ಆಚೆಗೆ ದೂಡುತ್ತದೆ. ಈ ಕಥೆಯನ್ನು ತಮ್ಮ ಅನನುಕರಣೀಯ ಶೈಲಿಯಲ್ಲಿ ಹೇಳಿ ಸ್ವಾಮೀಜಿ, ಇಂತಹ ಕೂಪಮಂಡೂಕ ಬುದ್ಧಿಯೇ ಧರ್ಮಾಂಧತೆಗೆ ಮೂಲಕಾರಣವೆಂದು ವಿವರಿಸಿದರು.

ಆದರೆ ಸ್ವಾಮೀಜಿ ಈ ಪುಟ್ಟ ಭಾಷಣವನ್ನು ಮಾಡಿದುದು ನೀರಸ ಅಧಿವೇಶನವೊಂದರ ಕೊನೆಯಲ್ಲಿ–ಅಧ್ಯಕ್ಷರ ವಿಶೇಷ ಕೋರಿಕೆಯ ಮೇರೆಗೆ. ಈಗಾಗಲೇ ಅವರ ಆಕರ್ಷಣ ಶಕ್ತಿಯನ್ನು ಅಧ್ಯಕ್ಷರು ಚೆನ್ನಾಗಿ ಕಂಡುಕೊಂಡಿದ್ದರು. ಆದ್ದರಿಂದ, ಸಭಿಕರು ಕಡೆಯವರೆಗೂ ಕುಳಿತುಕೊಳ್ಳು ವಂತೆ ಮಾಡಲು, ಸಭೆಯ ಕಡೆಯಲ್ಲಿ ವಿವೇಕಾನಂದರು ಪುಟ್ಟ ಭಾಷಣವೊಂದನ್ನು ಮಾಡುತ್ತಾ ರೆಂದು ತಿಳಿಸುತ್ತಿದ್ದರು. ಹೀಗೆ ಅವರು ಅನೇಕ ಸಣ್ಣಪುಟ್ಟ ಭಾಷಣಗಳನ್ನು ಮಾಡಿರಬಹು ದಾದರೂ ತಮ್ಮ ಮೊದಲ ಮುಖ್ಯ ಉಪನ್ಯಾಸವನ್ನು ನೀಡಿದುದು ಸೆಪ್ಟೆಂಬರ್ ೧೯ರಂದು. ಅಂದು ಅವರು, ‘ಹಿಂದೂ ಧರ್ಮ’ ಎಂಬ ತಮ್ಮ ಸುಪ್ರಸಿದ್ಧ ಲೇಖನವನ್ನು ಓದಿದರು. ಈ ಉಪನ್ಯಾಸಕ್ಕಾಗಿ ಜನ ಅತ್ಯಂತ ಉತ್ಸಾಹ ಕಾತರಗಳಿಂದ ಬೆಳಗಿನಿಂದಲೂ ಕಾದಿದ್ದರು. ಸಭಾಂಗಣ ಕಿಕ್ಕಿರಿದು ಹೋಗಿತ್ತು. ಸ್ವಾಮಿ ವಿವೇಕಾನಂದರ ಮಾತಿನ ಮೋಡಿಯಿಂದ ಆಕರ್ಷಿತರಾಗಿ ಬಂದ ವರು ಹಲವರಾದರೆ, ಈ ‘ಅರೆನಾಗರಿಕ’ ಧರ್ಮದ ತತ್ತ್ವವೇನಿರಬಹುದು–ಎಂದು ಕುತೂಹಲಿತ ರಾಗಿದ್ದವರು ಕೆಲವರು. ಸ್ಥಳಾವಕಾಶವಿಲ್ಲದೆ ಇನ್ನೂ ನೂರಾರು ಜನ ನಿರಾಶರಾಗಿ ಹಿಂದಿರುಗಿದ್ದರು.

ಇದಕ್ಕೆ ಹಿಂದಿನ ಕೆಲದಿನಗಳಲ್ಲಿ, ಕೆಲವರು ಉಗ್ರ ಕ್ರೈಸ್ತ ಧರ್ಮಾಧಿಕಾರಿಗಳು ಹಿಂದೂ ಧರ್ಮ ವನ್ನೂ ಇತರ ಪೌರ್ವಾತ್ಯ ಧರ್ಮಗಳನ್ನೂ ಕಟುವಾಗಿ ಟೀಕಿಸಿದ್ದರು. ಅದಕ್ಕೆ ಉತ್ತರವಾಗಿ ಅಂದು (ಸೆಪ್ಟೆಂಬರ್ ೧೯ರಂದು), ಬೌದ್ಧ ಪ್ರತಿನಿಧಿಗಳಲ್ಲೊಬ್ಬರಾದ ಧರ್ಮಪಾಲ ಮತ್ತಿತರರು ಕ್ರೈಸ್ತರ ಆ ವಾದವನ್ನು ಖಂಡಿಸಿದ್ದರು. ಹೀಗಾಗಿ ಒಂದು ತೀವ್ರ ಕುತೂಹಲಕಾರಿ ವಾತಾವರಣ ನಿರ್ಮಾಣ ಗೊಂಡಿತ್ತು. ಅಂದಿನ ತಮ್ಮ ಉಪನ್ಯಾಸದಲ್ಲಿ ಸ್ವಾಮೀಜಿ ಹಿಂದೂಧರ್ಮದ ತತ್ತ್ವ ಹಾಗೂ ಸಾಮಾನ್ಯ ರೂಪುರೇಷೆಗಳನ್ನು ವಿವರಿಸಿದರಲ್ಲದೆ, ಹಿಂದೂಧರ್ಮದ ಅತ್ಯುನ್ನತ ಬೋಧನೆಗಳೇ ಅದರ ಸಾರಸರ್ವಸ್ವವೆಂದೂ, ಅದು ಸಾರ್ವತ್ರೀಕರಿಸಬಹುದಾದ ನಿಯಮಗಳ ಆಧಾರದ ಮೇಲೆ ನಿಂತ ಧರ್ಮವೆಂದೂ ಸಾರಿದರು. ಒಟ್ಟಿನಲ್ಲಿ ಹಿಂದೂಧರ್ಮದ ಬಗ್ಗೆ ಅದೊಂದು ಅಭೂತ ಪೂರ್ವವಾದ ಭಾಷಣವಾಗಿತ್ತು. ಸಮ್ಮೇಳನದಲ್ಲಿ ಭಾಗವಹಿಸಿದ ಏಕಮಾತ್ರ ಭಾರತೀಯರೇನಲ್ಲ ಸ್ವಾಮೀಜಿ. ಅಥವಾ ಏಕಮಾತ್ರ ಬಂಗಾಳಿಯೂ ಅಲ್ಲ. ಆದರೂ ನಿಜವಾದ, ವಿಶಾಲ ಹಿಂದೂ ಧರ್ಮವನ್ನು ಪ್ರತಿನಿಧಿಸಿದ್ದವರು ಅವರೊಬ್ಬರೇ. ಇತರ ಹಿಂದೂ ಪ್ರತಿನಿಧಿಗಳು ಯಾವುದೋ ಸಮಾಜವನ್ನೋ ಪಂಥವನ್ನೋ ಪ್ರತಿನಿಧಿಸಿದ್ದರು. ಆದ್ದರಿಂದ ಸ್ವಾಮೀಜಿಯ ಉಪನ್ಯಾಸ ವೈಶಿಷ್ಟ್ಯಪೂರ್ಣವಾಗಿತ್ತು. ಅದನ್ನು ವಿವರವಾಗಿ ಪರಿಶೀಲಿಸುವುದು ಸೂಕ್ತ.

ಸಂಜೆ ಐದು ಗಂಟೆಯ ವೇಳೆಗೆ ಸ್ವಾಮೀಜಿಯ ಸರದಿ ಬಂದಿತು. ಇತರ ಮೂವರು ಉಪನ್ಯಾಸಕರ ಮಾತುಗಳನ್ನು ಕೇಳುತ್ತ ಸಭಿಕರು ಸಹನೆಯಿಂದ ಕುಳಿತಿದ್ದರು. ಸ್ವಾಮೀಜಿಯನ್ನು ಸಭೆಗೆ ಸಾಂಪ್ರದಾಯಿಕವಾಗಿ ಪರಿಚಯಿಸಿಕೊಡುತ್ತಿದ್ದಂತೆ ಕರತಾಡನ ಪ್ರಾರಂಭವಾಯಿತು. ಸ್ವಾಮೀಜಿ ವೇದಿಕೆಯ ಮಧ್ಯಕ್ಕೆ ಬಂದು ನಿಂತರು. ಪ್ರಚಂಡ ಕರತಾಡನ ಹರ್ಷೋದ್ಗಾರಗಳಿಂದ ಸಭೆ ಕಂಪಿಸಿತು. ಮನಮೋಹಕವಾದ ಉಡುಗೆಯಲ್ಲಿ ಕಂಗೊಳಿಸುತ್ತಿದ್ದ ಸ್ವಾಮೀಜಿ ಮಂದಹಾಸ ವದನರಾಗಿ ಅಭಿನಂದನೆಯನ್ನು ಸ್ವೀಕರಿಸಿ ತಮ್ಮ ಮಾತನ್ನು ಪ್ರಾರಂಭಿಸಿದರು.

ತಮ್ಮ ಲಿಖಿತ ಭಾಷಣವನ್ನು ಓದುವ ಮೊದಲು ಅವರು, ಹಿಂದಿನ ಕೆಲವು ದಿನಗಳಲ್ಲಿ ಪೌರ್ವಾತ್ಯ ಧರ್ಮಗಳ ಮೇಲೆ ಕೆಲವರು ಮಾಡಿದ್ದ ಟೀಕೆಗಳನ್ನು ಪ್ರಸ್ತಾಪಿಸಿ, ಕೆಲವು ತೀವ್ರ ಸಂಪ್ರದಾಯಸ್ಥ ಕ್ರೈಸ್ತರ ಧೋರಣೆಯನ್ನು ಖಂಡಿಸಿದರು. ಅವರು ಹೇಳಿದರು: “ಪೌರ್ವಾತ್ಯ ದೇಶಗಳಿಂದ ಬಂದು ಸಮ್ಮೇಳನದಲ್ಲಿ ಭಾಗವಹಿಸುತ್ತಿರುವ ನಾವು, ‘ಕ್ರೈಸ್ತ ರಾಷ್ಟ್ರಗಳು ಅತ್ಯಂತ ಸಂಪದ್ಭರಿತ ರಾಷ್ಟ್ರಗಳಾದ್ದರಿಂದ ನೀವೂ ಕ್ರೈಸ್ತಧರ್ಮವನ್ನು ಸ್ವೀಕರಿಸಬೇಕು’ ಎಂಬ ಅದೇ ಪಲ್ಲವಿಯನ್ನು ಮತ್ತೆ ಮತ್ತೆ ಕೇಳಿದ್ದೇವೆ. ಸುತ್ತಲೂ ಕಣ್ಣು ಹಾಯಿಸಿದರೆ ನಮಗೇನು ಕಾಣುತ್ತದೆ? ಜಗತ್ತಿನ ಅತ್ಯಂತ ಸಂಪದ್ಭರಿತ ಕ್ರೈಸ್ತ ರಾಷ್ಟ್ರವಾದ ಇಂಗ್ಲೆಂಡ್, ಇಪ್ಪತ್ತೈದು ಕೋಟಿ ಏಷ್ಯನ್ನರ ಕುತ್ತಿಗೆಯ ಮೇಲೆ ಕಾಲಿಟ್ಟು ನಿಂತಿದೆ. ಚರಿತ್ರೆಯನ್ನು ನೋಡಿದರೆ ಗೊತ್ತಾಗುತ್ತದೆ–ಕ್ರೈಸ್ತ ಯೂರೋಪಿನ ಭಾಗ್ಯ ಪ್ರಾರಂಭವಾದದ್ದು ಸ್ಪೆಯಿನಿನಿಂದ. ಸ್ಪೆಯಿನಿನ ಭಾಗ್ಯ ಪ್ರಾರಂಭವಾದದ್ದು ಮೆಕ್ಸಿಕೋ ಮೇಲಿನ ಆಕ್ರಮಣದೊಂದಿಗೆ. ಕ್ರೈಸ್ತ ಧರ್ಮವು, ತನ್ನ ಸಹಮಾನವರ ಕತ್ತನ್ನು ಕತ್ತರಿಸುವುದರ ಮೂಲಕ ತನ್ನ ಸೌಭಾಗ್ಯವನ್ನು ಸಂಪಾದಿಸಿಕೊಳ್ಳುತ್ತದೆ. ಆದರೆ ಹಿಂದುವಾದವನು ಆ ಬೆಲೆಯನ್ನು ತೆತ್ತು (ಹಾಗೆ ಕತ್ತು ಕುಯ್ಯುವುದರ ಮೂಲಕ) ಎಂದೂ ಸಂಪತ್ತನ್ನು ಗಳಿಸಲಾರ.... ಇಲ್ಲಿ ಕುಳಿತು (ಕೆಲವು ಭಾಷಣಗಳಲ್ಲಿ) ನಾನು ಅಹಸನೆಯ ಪರಮಾವಧಿಯನ್ನು ಕಂಡಿ ದ್ದೇನೆ. ಮುಸ್ಲಿಮರ ಖಡ್ಗವು ಭಾರತವನ್ನು ನಾಶಗೊಳಿಸಿರುವಾಗ, ಇಲ್ಲಿ ಮುಸ್ಲಿಮ್ ಮತದ ಧೋರಣೆಗಳನ್ನು ಜನರು ಕರತಾಡನದಿಂದ ಸ್ವಾಗತಿಸುವುದನ್ನು ಕಂಡಿದ್ದೇನೆ. ಖಡ್ಗ ಮತ್ತು ರಕ್ತ–ಇವು ಹಿಂದೂವಿಗೆ ಒಪ್ಪಿಗೆಯಾಗುವುಂಥದಲ್ಲ. ಹಿಂದೂಗಳ ಧರ್ಮವು ನಿಂತಿರುವುದು ಪ್ರೀತಿಯ ನಿಯಮದ ಮೇಲೆ.”

ಇಂತಹ ತೀಕ್ಷ್ಣವಾದ ಟೀಕೆಗಳಿಂದ ಸ್ವಾಮೀಜಿ, ತಾವು ಗಳಿಸಿದ್ದ ಜನಪ್ರಿಯತೆಯನ್ನು ಕಳೆದು ಕೊಳ್ಳುವ ಸಂಭವವಿತ್ತು. ಆದರೆ ಅವರು ಅದಕ್ಕೆ ಸ್ವಲ್ಪವೂ ಅಂಜದೆ, ತಮ್ಮ ಅಭಿಪ್ರಾಯಗಳನ್ನು ನಿಸ್ಸಂಕೋಚವಾಗಿ, ಸುಲಲಿತ ವಾಗ್ಝರಿಯಲ್ಲಿ ಮಂಡಿಸಿದರು. ಆದರೆ ಜನ ಅವರನ್ನು ಸರಿಯಾಗಿ ಅರ್ಥಮಾಡಿಕೊಂಡರು. ಅವರು ತಮ್ಮ ಹೇಳಿಕೆಯನ್ನು ಮುಗಿಸುತ್ತಿದ್ದಂತೆ ಸಭಿಕರು ಕರತಾಡನದ ಮೂಲಕ ತಮ್ಮ ಒಪ್ಪಿಗೆಯನ್ನು ಸೂಚಿಸಿದರು.

ಇದಾದನಂತರ ಸ್ವಾಮೀಜಿ ತಮ್ಮ ಲಿಖಿತ ಭಾಷಣವನ್ನು ಓದಲಾರಂಭಿಸಿದರು. ಮೊದಲಿಗೆ ಹಿಂದೂಧರ್ಮದ ಸರ್ವವ್ಯಾಪಕತೆಯ ಅಂಶವನ್ನು ವಿವರಿಸಿದರು. ಬೌದ್ಧ, ಜೈನ, ಕ್ರೈಸ್ತ ಧರ್ಮ ಗಳೆಲ್ಲದರ ಸಾರವನ್ನು ತನ್ನಲ್ಲಡಗಿಸಿಕೊಂಡಿರುವ ಹಿಂದೂ ಧರ್ಮವು ಅತ್ಯುನ್ನತವಾದ ಏಕಾತ್ಮ ವಾದದೊಂದಿಗೆ ಅತ್ಯಂತ ಕನಿಷ್ಠವಾದ ವಿಗ್ರಹಾರಾಧನೆಯನ್ನೂ ಅನುಮೋದಿಸುತ್ತದೆ ಎಂದರು. ಹಿಂದೂಧರ್ಮವು ಆಧ್ಯಾತ್ಮಿಕ ಅನುಭವಗಳ ಖನಿಯಾದ ವೇದಗಳ ಮೇಲೆ ಆಧಾರಿತವಾಗಿದೆ. ಇತರ ಧರ್ಮಗಳೆಲ್ಲ ಒಬ್ಬ ವ್ಯಕ್ತಿಯನ್ನು ಆಧರಿಸಿರುವಾಗ, ಹಿಂದೂ ಧರ್ಮ ಮಾತ್ರವೇ ಹೀಗೆ ತತ್ತ್ವಗಳನ್ನು ಆಧರಿಸಿರುವುದು. ವೇದವು ಅಪೌರುಷೇಯವೆಂದು ಹೇಳಿದ ಸ್ವಾಮೀಜಿ ಜನರ ಮನ ಮುಟ್ಟುವ ಉದಾಹರಣೆಯೊಂದನ್ನು ಕೊಟ್ಟರು. ಹೇಗೆ ನ್ಯೂಟನ್ನನು ಗುರುತ್ವಾಕರ್ಷಣೆಯನ್ನು ‘ಕಂಡುಹಿಡಿಯುವ’ ಮೊದಲೂ ಅದು ಇದ್ದೇ ಇತ್ತೋ, ಅದೇ ರೀತಿಯಲ್ಲಿ, ವೇದಗಳಲ್ಲಿ ಹೇಳಲ್ಪಟ್ಟಿರುವ ಆಧ್ಯಾತ್ಮಿಕ ತತ್ತ್ವಗಳನ್ನು ‘ಬರೆಯುವ’ ಮೊದಲೂ ಅವು ಇದ್ದುವು. ಎಂದರೆ ಇವು ಅನಾದಿಯಿಂದಲೂ ಇದ್ದುವು. ಆದ್ದರಿಂದ ಸೃಷ್ಟಿಯೇ ಇಲ್ಲದಿದ್ದಂತಹ ಒಂದು ಕಾಲ, ಎಂದರೆ ‘ಆದಿ’ ಎಂಬುದು ಇರಲೇ ಇಲ್ಲ ಎಂದು ಸ್ವಾಮೀಜಿ ವಾದಿಸಿದರು. ಇದು ಕ್ರೈಸ್ತ ತತ್ತ್ವಗಳಿಗೆ ತದ್ವಿರುದ್ಧವಾದ, ಕ್ರೈಸ್ತ ಧರ್ಮ ಪ್ರಚಾರಕರನ್ನು ಕೆರಳಿಸಿದ ಅಂಶ. ಮಾನವನೆಂದರೆ ಆತನ ದೇಹವಲ್ಲ, ಆತನೊಳಗಿರುವ ಆತ್ಮ ಎಂಬುದು ಹಿಂದೂಗಳ ನಂಬಿಕೆಯೆಂದು ಸ್ವಾಮೀಜಿ ಸಾರಿದರು. ‘ಸೃಷ್ಟಿ’ಸಲ್ಪಟ್ಟ ಪ್ರತಿಯೊಂದು ವಸ್ತುವಿಗೂ ‘ವಿನಾಶ’ವಿರುವುದರಿಂದ ಈ ದೇಹವು ನಾಶವಾಗುತ್ತದೆ; ಆದರೆ ಅನಾದಿಯಾದ, ಎಂದರೆ ಎಂದೂ ಸೃಷ್ಟಿಸಲ್ಪಡದ ಆತ್ಮವು ನಾಶ ವಾಗಲೂ ಸಾಧ್ಯವಿಲ್ಲ ಎಂದು ವಾದಿಸಿದರು. ಸಾವೆಂದರೆ ಶರೀರವನ್ನು ತ್ಯಜಿಸುವುದು ಅಷ್ಟೇ. ಈ ಆತ್ಮತತ್ತ್ವವನ್ನು ಸಾಕ್ಷಾತ್ಕರಿಸಿಕೊಳ್ಳುವುದೇ ಜೀವನದ ಪರಮೋದ್ದೇಶ, ಹಾಗೂ ಮುಕ್ತಿಗೆ ದಾರಿ ಎಂದು ಘೋಷಿಸಿದರು. ಆತ್ಮತತ್ತ್ವವನ್ನು ಸಾಕ್ಷಾತ್ಕರಿಸಿಕೊಳ್ಳಬೇಕಾದರೆ ‘ನಾನು’ ‘ನನ್ನದು’ ಎಂಬ ಅಹಂಕಾರವನ್ನು ತ್ಯಜಿಸಲೇಬೇಕು. ಆದರೆ ಹೀಗೆ ಅಹಂಕಾರವನ್ನು ತ್ಯಜಿಸಿದರೂ ಅದು ನಿಜ ವ್ಯಕ್ತಿತ್ವವನ್ನೇ ತ್ಯಜಿಸಿದಂತಲ್ಲ; ಬದಲಾಗಿ ಅದೇ ನಿಜವ್ಯಕ್ತಿತ್ವದ ಪರಿಪೂರ್ಣತೆ ಎಂದು ಸ್ವಾಮೀಜಿ ವಿವರಿಸಿದರು. ಸ್ವಾರ್ಥಬುದ್ಧಿಯ ಸುತ್ತ ನಾವು ಹೆಣೆದುಕೊಂಡಿರುವ ಸಂಕುಚಿತ ಅಹಂಕಾರವನ್ನು ಹೋಗಲಾಡಿಸಿಕೊಂಡಾಗ ಅನಂತತೆಯ ಅನುಭವವಾಗುತ್ತದೆ, ಇದೇ ವಿಶ್ವಾತ್ಮಭಾವ ಎಂಬ ಹಿಂದೂ ಕಲ್ಪನೆಯನ್ನು ಅವರು ವಿವರಿಸಿದರು.

ಸ್ವಾಮೀಜಿಯ ಬಾಯಿಂದ ಅತ್ಯದ್ಭುತವಾದ, ತಾವು ಕಂಡುಕೇಳರಿಯದಿದ್ದ ವಿಚಾರಧಾರೆ ಯನ್ನು ಕೇಳುತ್ತಿದ್ದ ಜನ ರೋಮಾಂಚಿತರಾದರು. ಒಂದೊಂದೂ ಬರಸಿಡಿಲಿನಂತೆ ಬಂದು ಬಡಿಯುತ್ತಿದ್ದ ಹೊಸ ಹೊಸ ಭಾವನೆಗಳನ್ನು ಕೇಳಿದ ಜನ ಮೂಕವಿಸ್ಮತರಾಗಿ ಕುಳಿತಿದ್ದಂತೆ ಸ್ವಾಮೀಜಿ ತಮ್ಮ ಮಾತನ್ನು ಮುಂದುವರಿಸಿದರು:

“ಮಾನವನ ಆತ್ಮ ಆದಿ-ಅಂತರಹಿತ. ಮರಣವೆಂದರೆ ಒಂದು ದೇಹದಿಂದ ಮತ್ತೊಂದು ದೇಹಕ್ಕೆ ಆತ್ಮನ ವರ್ಗಾವಣೆ ಅಷ್ಟೆ. ನಮ್ಮ ಇಂದಿನ ಸ್ಥಿತಿಯು ನಮ್ಮ ಹಿಂದಿನ ಕರ್ಮಗಳಿಂದ ನಿರ್ಧಾರಿತವಾಗಿದೆ. ಅಂತೆಯೇ ನಮ್ಮ ಮುಂದಿನ ಜೀವನವು ಇಂದಿನ ಕರ್ಮಗಳಿಂದ ನಿರ್ಧಾರಿತ ವಾಗುತ್ತದೆ. ಜನನ-ಮರಣಗಳು ಕಳೆದಂತೆಲ್ಲ ಆತ್ಮವು ಮುಂದುಮುಂದಕ್ಕೆ ವಿಕಾಸವಾಗುತ್ತ ಅಥವಾ ಹಿಂದುಹಿಂದಕ್ಕೆ ಕುಗ್ಗುತ್ತ ಹೋಗುತ್ತದೆ. ಆದರೆ ಈಗ ಇಲ್ಲೊಂದು ಪ್ರಶ್ನೆಯೇಳುತ್ತದೆ– ಈ ಮಾನವನು ಭೀಕರ ಚಂಡಮಾರುತದಲ್ಲಿ ಸಿಲುಕಿ ತುಯ್ದಾಡುತ್ತಿರುವ ಒಂದು ಕಿರುದೋಣಿಯೆ ಹಾಗಾದರೆ? ಸತ್ಕರ್ಮ-ದುಷ್ಕರ್ಮಗಳ ತಾಡನಕ್ಕೆ ತುತ್ತಾಗಿ ಅಲೆದಾಡುತ್ತಿರುವವನೇ ಈ ಮಾನವ? ವಿಧವೆಯರ ಕಂಬನಿಗೆ ಕರಗದೆ, ತಬ್ಬಲಿಗಳ ಗೋಳಿಗೆ ಮರುಗದೆ, ತನ್ನ ದಾರಿಯಲ್ಲಿ ಸಿಕ್ಕಿದುದನ್ನೆಲ್ಲ ಕುಟ್ಟಿ ಪುಡಿಗೈಯುತ್ತಿರುವ ಕಾರ್ಯ ಕಾರಣವೆಂಬ ಮಹಾಚಕ್ರದ ಅಡಿಯಲ್ಲಿ ಸಿಲುಕಿರುವ ಒಂದು ಕೀಟವೆ ಈ ಮಾನವ?....ಇದನ್ನು ಭಾವಿಸಿದರೆ ಹೃದಯ ತಲ್ಲಣಿಸುತ್ತದೆ. ಆದರೂ ಇದೇ ಪ್ರಕೃತಿಯ ನಿಯಮ. ಹಾಗಾದರೆ, ಇದರಿಂದ ಬಿಡುಗಡೆಯೇ ಇಲ್ಲವೆ? ನಮಗೆ ಭರವಸೆಯೇ ಇಲ್ಲವೆ?–ಮಾನವನ ಹತಾಶಹೃದಯದಾಳದಿಂದ ಹೊರಹೊಮ್ಮಿದ ಆರ್ತನಾದ ಇದು. ಈ ಕೂಗು ಪರಮಕರುಣಾಮೂರ್ತಿಯಾದ ಭಗವಂತನ ಪೀಠವನ್ನು ಮುಟ್ಟಿತು; ಮತ್ತು ಅಲ್ಲಿಂದ ಭರವಸೆ-ಸಮಾಧಾನಗಳ ನುಡಿ ಇಳೆಯೆಡೆಗೆ ಇಳಿಯಿತು. ಆ ನುಡಿಯಿಂದ ಸ್ಫೂರ್ತಿ ಗೊಂಡ ಉಪನಿಷತ್ತಿನ ಪುಷಿ ಜಗದೆದುರಿಗೆ ನಿಂತು ಜಯಘೋಷದ ದನಿಯಲ್ಲಿ ಈ ಆನಂದದ ವಾರ್ತೆಯನ್ನು ಸಾರಿದ:

\begin{verse}
ಶೃಣ್ವಂತು ವಿಶ್ವೇ ಅಮೃತಸ್ಯ ಪುತ್ರಾಃ\\ಆಯೇ ಧಾಮಾನಿ ದಿವ್ಯಾನಿ ತಸ್ತುಃ\\ವೇದಾಹಮೇತಂ ಪುರುಷಂ ಮಹಾಂತಂ\\ಆದಿತ್ಯವರ್ಣಂ ತಮಸಃ ಪರಸ್ತಾತ್​\\ತಮೇವ ವಿದಿತ್ವಾ ಅತಿಮೃತ್ಯುಮೇತಿ\\ನಾನ್ಯಃ ಪಂಥಾ ವಿದ್ಯತೇsಯನಾಯ ॥
\end{verse}

‘ಓ ಅಮೃತಪುತ್ರರೇ ಕೇಳಿ! ಸ್ವರ್ಗದಲ್ಲಿರುವವರು ಕೂಡ ಕೇಳಿ! ಎಲ್ಲ ಮೋಹದಾಚೆ, ಎಲ್ಲ ಅಜ್ಞಾನದಾಚೆ, ಎಲ್ಲ ತಮಸ್ಸು-ಭ್ರಾಂತಿಗಳಾಚೆ ಇರುವ ಸನಾತನ ಪುರುಷನನ್ನು ನಾನು ಕಂಡಿರುವೆ! ಅವನನ್ನು ಅರಿಯುವುದರಿಂದ ಮಾತ್ರವೇ ಮೃತ್ಯುವಿನಿಂದ ಪಾರಾಗಲು ಸಾಧ್ಯ. ಇದರ ಹೊರತು ಅಮರತ್ವಕ್ಕೆ ಬೇರೊಂದು ದಾರಿಯಿಲ್ಲ.’

“‘ಅಮೃತ ಪುತ್ರರು!’ ಎಂತಹ ಅದ್ಭುತವಾದ ಹೆಸರು! ಎಂತಹ ಆಶಾದಾಯಕ ಹೆಸರು! ಸೋದರರೇ, ಆ ಮಧುರನಾಮದಿಂದ–ಅಮೃತಪುತ್ರರು ಎಂಬ ಮಧುರನಾಮದಿಂದ– ನಿಮ್ಮನ್ನು ಸಂಬೋಧಿಸಲು ಅನುಮತಿ ನೀಡಿ! ಹೌದು, ಹಿಂದುವು ನಿಮ್ಮನ್ನು ಪಾಪಿಗಳೆಂದು ಕರೆಯಲು ನಿರಾಕರಿಸುತ್ತಾನೆ. ನೀವು ಭಗವಂತನ ಪುತ್ರರು, ಅಮೃತಾನಂದದಲ್ಲಿ ಭಾಗಿಗಳು; ನೀವು ಪವಿತ್ರರು, ನೀವು ಪರಿಪೂರ್ಣರು. ಇಂತಹ ನೀವು...ಪಾಪಿಗಳೆ!? ಮಾನವನನ್ನು ಪಾಪಿ ಎಂದು ಕರೆಯುವುದೇ ಒಂದು ಮಹಾಪಾಪ! ಮಾನವನ ಸ್ವಭಾವಕ್ಕೆ ಒಂದು ಮಹಾಕಳಂಕ! ಮೇಲೆದ್ದು ಬನ್ನಿ, ಓ ವೀರಕೇಸರಿಗಳೇ; ನೀವು ಕುರಿಗಳೆಂದೆಂಬ ಭ್ರಾಂತಿಯನ್ನು ಕೊಡಹಿ. ನೀವು ಪ್ರಕೃತಿಯಲ್ಲ. ನೀವು ದೇಹವಲ್ಲ. ಪ್ರಕೃತಿ ನಿಮ್ಮ ಅಡಿಯಾಳು, ನೀವು ಪ್ರಕೃತಿಯ ಅಡಿಯಾಳಲ್ಲ.”

ಸ್ವಾಮೀಜಿಯ ಈ ಮಾತುಗಳು ಉಂಟುಮಾಡಿದ ಪರಿಣಾಮ ವರ್ಣಿಸಲದಳವಾದದ್ದು. ‘ಮಾನವನು ಪಾಪಿ’ ಎಂಬ ಕ್ರೈಸ್ತ ಮತದ ಮೂಲಭೂತ ತತ್ತ್ವಕ್ಕೆ ಇದು ಸುತ್ತಿಗೆಯೇಟು. ಈ ಮಾತುಗಳನ್ನಾಲಿಸುತ್ತಿದ್ದ ಸಹಸ್ರಾರು ಜನ ದಂಗು ಬಡಿದಂತೆ ಕುಳಿತಿದ್ದರು. ಕ್ರೈಸ್ತಧರ್ಮ ಪ್ರಚಾರಕರಿಗೆ ಮಾತ್ರ ಇದೊಂದು ಮರ್ಮಾಘಾತ. ಏಕೆಂದರೆ ಮಾನವನ್ನು ಪಾಪಿಯೆನ್ನುವುದೇ ಮಹಾಪಾಪವೆಂದಮೇಲೆ, ಅವರು ಪಾಪ ವಿಮೋಚನೆಯ ಭರವಸೆಯನ್ನು ನೀಡುವುದಾದರೂ ಯಾರಿಗೆ! ಅಲ್ಲದೆ ಸ್ವಾಮೀಜಿ ಆಡುತ್ತಿದ್ದ ಪ್ರತಿಯೊಂದು ಮಾತೂ ಕೇವಲ ತರ್ಕದ ಆಧಾರದ ಮೇಲೆ ನಿಂತಂತೆ, ಅಥವಾ ಯಾವುದೋ ಮತವನ್ನು ಸಮರ್ಥಿಸಿಕೊಳ್ಳುವುದಕ್ಕಾಗಿ ಆಡಿದಂತೆ ತೋರುತ್ತಿರಲಿಲ್ಲ. ಬದಲಾಗಿ ಅವು ಅವರ ಅನುಭವದ ಆಳದಿಂದ ಮೂಡಿಬರುತ್ತಿರುವು ದೆಂಬುದು ಸ್ಪಷ್ಟವಾಗಿ ತೋರುತ್ತಿತ್ತು.

ವೇದಾಂತದ ಅದ್ವೈತವಾದವನ್ನು ಸ್ಫುಟವಾಗಿ ವಿವರಿಸಿದ ಬಳಿಕ ಸ್ವಾಮೀಜಿ, ಯಾವಾಗಲೂ ಹಿಂದೂಧರ್ಮದ ಬಗ್ಗೆ ಪಾಶ್ಚಾತ್ಯರ ತಲೆ ತಿನ್ನುವ ಒಂದು ಪ್ರಶ್ನೆಯನ್ನು ಕೈಗೆತ್ತಿಕೊಂಡರು. ‘ಇದೆಲ್ಲ ಸರಿಯೇ; ಆದರೆ ಈ ಹಿಂದೂಧರ್ಮವು ಬಹುದೇವತಾ ಉಪಾಸನೆಯನ್ನು ಅನು ಮೋದಿಸುತ್ತದಲ್ಲ? ಒಬ್ಬನೇ ದೇವರೆಂದ ಮೇಲೆ ಅಷ್ಟೊಂದು ಬಗೆಯ, ಅಷ್ಟೊಂದು ಚಿತ್ರ ವಿಚಿತ್ರವಾದ ದೇವತೆಗಳೇಕೆ?’ ಇದಕ್ಕುತ್ತರವಾಗಿ ಸ್ವಾಮೀಜಿ, ಕೆಳಮಟ್ಟದ ಧಾರ್ಮಿಕ ಭಾವನೆಗಳ ಮತ್ತು ಉಪಾಸನೆಗಳ ಆವಶ್ಯಕತೆಯನ್ನು ಮನೋವಿಜ್ಞಾನದ ಆಧಾರದ ಮೇಲೆ ವಿವರಿಸಿದರು. ವಿಗ್ರಹವು ದೈವತ್ವದ ಚಿಹ್ನೆಯಾದಾಗ, ಆ ಪೂಜೆಯು ‘ವಿಗ್ರಹಾರಾಧನೆ’ ಎನ್ನಿಸಿಕೊಳ್ಳಲಾರದು. ಅಲ್ಲದೆ, ಹಿಂದೂಗಳ ಪಾಲಿಗೆ ಧರ್ಮದ ಸಾರವಡಗಿರುವುದು ಯಾವುದೋ ಒಂದು ತತ್ತ್ವವನ್ನು ಕೇವಲ ಅಂಗೀಕರಿಸುವುದರಲ್ಲಿ ಅಥವಾ ನಿರಾಕರಿಸುವುದರಲ್ಲಿ ಅಲ್ಲ. ಧರ್ಮವೆಂದರೆ ಅದೊಂದು ಅನುಭವ, ಸಾಕ್ಷಾತ್ಕಾರ. ಆದ್ದರಿಂದ ರೂಪಗಳು, ಚಿಹ್ನೆಗಳು, ಆಚರಣೆಗಳೆಲ್ಲವೂ ಆಧ್ಯಾತ್ಮಿಕ ಶೈಶವದಲ್ಲಿ ವ್ಯಕ್ತಿಯ ಬೆಳವಣಿಗೆಗೆ ಸಾಧನಗಳಷ್ಟೆ. ಮತ್ತು ಮತಗಳ ವಿವಿಧತೆಯ ಹಿಂದೆ ಏಕತೆಯೊಂದಿದೆ ಎಂಬ ಸತ್ಯವನ್ನು ಸಾಕ್ಷಾತ್ಕರಿಸಿಕೊಂಡಿರುವ ಸ್ವಾಮೀಜಿ, ಭಗವದ್ಗೀತೆಯ ಮಾತೊಂದನ್ನು ಉದಾಹರಿಸಿ ಅದನ್ನು ವಿವರಿಸಿದರು. ಬಳಿಕ ತಮ್ಮ ಉಪನ್ಯಾಸದ ಉಪಸಂಹಾರ ದಲ್ಲಿ ವಿಶ್ವಧರ್ಮದ (ಜಗತ್ತಿನ ಎಲ್ಲ ಜನರೂ ಸ್ವೀಕರಿಸಲು ಯೋಗ್ಯವಾದಂತಹ ಧರ್ಮದ) ಆದರ್ಶವನ್ನು ಮುಂದಿಟ್ಟರು;

“ಈ ಜಗತ್ತಿನಲ್ಲಿ ವಿಶ್ವಧರ್ಮ ಎನ್ನಿಸಿಕೊಳ್ಳುವಂಥದೇನಾದರೂ ಇರಬೇಕಾದರೆ ಅದು ದೇಶ- ಕಾಲಗಳ ಮಿತಿಯನ್ನು ಮೀರಿರಬೇಕು. ಅದು ತಾನೂ ಅನಂತವಾಗಿರಬೇಕಲ್ಲದೆ, ಅನಂತನಾದ ಭಗವಂತನನ್ನು ಸಾರಬೇಕು. ಆ ಧರ್ಮಭಾಸ್ಕರನ ಬೆಳಕು ಕೃಷ್ಣನ ಅನುಯಾಯಿಗಳ ಮೇಲೆ, ಕ್ರಿಸ್ತನ ಅನುಯಾಯಿಗಳ ಮೇಲೆ, ಪಾಪಿಗಳ ಮೇಲೆ, ಪುಣ್ಯವಂತರ ಮೇಲೆ–ಎಲ್ಲರ ಮೇಲೂ ಒಂದೇ ಸಮನಾಗಿ ಬೀಳಬೇಕು. ಅದು ವೈದಿಕವಾಗಲಿ, ಮಹಮದೀಯವಾಗಲಿ, ಬೌದ್ಧವಾಗಲಿ ಅಥವಾ ಕ್ರೈಸ್ತವಾಗಲಿ ಆಗಿರದೆ, ಇವೆಲ್ಲದರ ಸಮಗ್ರೀಕರಣವಾಗಿರಬೇಕು. ಸುಧಾರಣೆ ಅಥವಾ ಅಭಿವೃದ್ಧಿಗೆ ಅಪಾರ ಅವಕಾಶವನ್ನು ಹೊಂದಿರಬೇಕು. ಅದು ಪಶುಸದೃಶ ಕಾಡುಜನರಿಂದ ಹಿಡಿದು, ಹೃದಯ-ಬುದ್ಧಿಗಳ ತುತ್ತತುದಿಗೇರಿ ಸಾಮಾನ್ಯ ಮಾನವತೆಯ ಮೇರೆಯನ್ನೇ ಮೀರಿ ರುವ ಮಹಾಮಾನವರವರೆಗೆ ಸರ್ವರಿಗೂ ಆಶ್ರಯವನ್ನೀಯುವಂಥದಾಗಿರಬೇಕು. ಅದು ತನ್ನ ನೈಜ ದಿವ್ಯಸ್ವರೂಪವನ್ನು ಕಂಡುಕೊಳ್ಳುವಂತೆ ಮಾಡುವಲ್ಲಿ ಆ ಧರ್ಮದ ವಿಶಾಲ ಪರಿಧಿಯು ಕೇಂದ್ರೀಕೃತವಾಗಿರಬೇಕು. ಅಂತಹ ಧರ್ಮವೊಂದನ್ನು ಕೊಡಿ; ಆಗ ಜಗತ್ತಿನ ರಾಷ್ಟ್ರಗಳೆಲ್ಲವೂ ಅದನ್ನು ಅನುಸರಿಸುವುವು.”

ಹಿಂದೂಧರ್ಮದ ಕುರಿತಾದ ತಮ್ಮ ಉಪನ್ಯಾಸದಲ್ಲಿ ಸ್ವಾಮೀಜಿ, ವಿಶ್ವಧರ್ಮವೊಂದರ ಲಕ್ಷಣಗಳನ್ನು ವಿವರಿಸುವಲ್ಲಿ ಒಂದು ಮುಖ್ಯ ಉದ್ದೇಶವಿತ್ತು. ಕ್ರೈಸ್ತ ಧರ್ಮವು ವಿಶ್ವಧರ್ಮ ವಾಗಬಲ್ಲ ಶ್ರೇಷ್ಠತಮ ಧರ್ಮವೆಂದೂ ಇತರರು ಅದನ್ನು ಸ್ವೀಕರಿಸಿ ಧನ್ಯರಾಗಬೇಕೆಂದೂ ಮನದಟ್ಟು ಮಾಡಿಸುವುದಕ್ಕಾಗಿ ಕ್ರೈಸ್ತಧರ್ಮಪ್ರಚಾರಕರು ಚತುರೋಪಾಯಗಳನ್ನೂ ಪ್ರಯೋಗಿ ಸುತ್ತ, ಶಕ್ತಿಮೀರಿ ಪ್ರಚಾರ ಮಾಡುತ್ತಿದ್ದರು. ಮತ್ತು, ನಾವು ಹಿಂದೆಯೇ ನೋಡಿದಂತೆ, ಸರ್ವ ಧರ್ಮ ಸಮ್ಮೇಳನದ ಹಿಂದೆಯೂ ಇದೇ ಉದ್ದೇಶ ಬಹುಮಟ್ಟಿಗೆ ಕೆಲಸ ಮಾಡಿತ್ತು. ಆದ್ದರಿಂದ ಸ್ವಾಮೀಜಿ ತಮ್ಮ ಉಪನ್ಯಾಸದಲ್ಲಿ ನಿಜವಾದ ವಿಶ್ವಧರ್ಮದ ಲಕ್ಷಣಗಳನ್ನು ವಿವರಿಸುವುದರ ಮೂಲಕ, ಆ ಘೋಷಣೆಯ ಪೊಳ್ಳುತನವನ್ನು ಬಯಲಿಗೆಳೆದರು. ಬಳಿಕ ತಮ್ಮ ಮಾತನ್ನು ಮುಕ್ತಾಯಗೊಳಿಸುತ್ತ, ಸಮ್ಮೇಳನಕ್ಕೆ ಕಾರಣರಾದವರನ್ನು ಅಭಿನಂದಿಸಿದರು:

“ಅಶೋಕನ ಆಸ್ಥಾನ ಮುಖ್ಯವಾಗಿ ಬೌದ್ಧ ಸಮಿತಿಯಾಯಿತು. ಅಕ್ಬರನದು ಕಾರ್ಯತಃ ಇದಕ್ಕಿಂತ ಹೆಚ್ಚು ವಿಸ್ತಾರವಾಗಿದ್ದರೂ ಅದೊಂದು ಚಾವಡಿಯ ಚರ್ಚಾಕೂಟವಾಯಿತಷ್ಟೆ. ಆದರೆ ಭಗವಂತ ಸರ್ವಧರ್ಮಗಳಲ್ಲೂ ಇದ್ದಾನೆ ಎಂಬುದನ್ನು ಜಗತ್ತಿನ ಮೂಲೆಮೂಲೆಗೆ ಸಾರಿದ ಕೀರ್ತಿ ಅಮೆರಿಕೆಯದಾಯಿತು.

“ಯಾರು ಹಿಂದೂಗಳ ಬ್ರಹ್ಮವೊ, ಜರತುಷ್ಟ್ರೀಯರ ಅಹುರ ಮಸ್ದನೊ, ಬೌದ್ಧರ ಬುದ್ಧನೊ, ಯಹೂದ್ಯರ ಜೆಹೋವನೊ, ಕ್ರೈಸ್ತರ ‘ಸ್ವರ್ಗದಲ್ಲಿರುವ ತಂದೆ’ಯೊ ಅವನು ನಿಮಗೆ ಈ ಘನ ಉದ್ದೇಶವನ್ನು ಅನುಷ್ಠಾನಗೊಳಿಸಲು ಶಕ್ತಿಯನ್ನೀಯಲಿ... ಪೂರ್ವ ದಿಗಂತದಲ್ಲಿ ಧರ್ಮ ನಕ್ಷತ್ರವುದಿಸಿತು; ಸ್ಥಿರಗತಿಯಿಂದ ಅದು ಪಶ್ಚಿಮದೆಡೆಗೆ ಪಯಣಿಸಿತು; ಕೆಲವೊಮ್ಮೆ ಮಸಕಾ ಗುತ್ತ ಕೆಲವೊಮ್ಮೆ ಪ್ರಖರವಾಗುತ್ತ ಜಗತ್ತನ್ನು ಪ್ರದಕ್ಷಿಣೆ ಮಾಡಿ, ಈಗ ಅದು ಮತ್ತೊಮ್ಮೆ ಅದೇ ಪೂರ್ವದ ಪೆಸಿಫಿಕ್ (ಶಾಂತ) ಸಾಗರದ ದಿಗಂತದಲ್ಲಿ, ಹಿಂದಿಗಿಂತಲೂ ಸಾಸಿರಮಡಿ ಪ್ರಕಾಶಯುಕ್ತವಾಗಿ ಉದಿಸುತ್ತಿದೆ.”

“ಭಲೆ ಕೊಲಂಬಿಯ! ಸ್ವಾತಂತ್ರ್ಯದ ತಾಯ್ನಾಡೇ! ನಾಗರಿಕತೆಯ ಮುಂಚೂಣಿಯಲ್ಲಿ ಸಮನ್ವಯ ಧ್ವಜವನ್ನು ಹಿಡಿದು ಮುನ್ನಡೆಯುವ ಮಹಾಕಾರ್ಯ ನಿನ್ನ ಪಾಲಿನದಾಯಿತು; ನೆರೆಯವರ ರಕ್ತದಲ್ಲಿ ಕೈಯದ್ದದ, ನೆರೆಯವರನ್ನು ದೋಚುವುದೇ ಶ್ರೀಮಂತನಾಗಲು ಅತಿ ಸುಲಭದ ಉಪಾಯ ಎಂಬ ಮಾರ್ಗವನ್ನು ಅನುಸರಿಸದ, ನಿನ್ನ ಪಾಲಿನದಾಯಿತು.”

ಹಿಂದೂಧರ್ಮವನ್ನು ಕುರಿತು ಸ್ವಾಮೀಜಿ ಮಾಡಿದ ಈ ಉಪನ್ಯಾಸವು ಸಮಸ್ತ ಧರ್ಮಗಳ ಇತಿಹಾಸದಲ್ಲೇ ಅತ್ಯಂತ ಪರಿಣಾಮಕಾರಿಯಾದ ಭವಿಷ್ಯವಾಣಿಯಾಗಿತ್ತು ಎಂದರೆ ಅತಿಶಯೋಕ್ತಿ ಯಾಗಲಾರದು. ಮಾನವತ್ವದ ಏಕತೆ ಹಾಗೂ ದೈವತ್ವಗಳನ್ನು ಅದು ಎತ್ತಿತೋರಿತು. ಈ ಉಪನ್ಯಾಸದಲ್ಲಿ ಟೀಕೆಯ ದನಿಯಾಗಲಿ ಎದುರುಹಾಕಿಕೊಳ್ಳುವ ಗುಣವಾಗಲಿ ಇರಲಿಲ್ಲ. ಅದು ಎಷ್ಟೋ ಮತಗಳ ನಂಬಿಕೆಗಳಿಗೆ ವಿರುದ್ಧವಾದ ದಾರಿಯನ್ನೂ ಹಿಡಿಯಿತಾದರೂ ಆ ಯಾವುದರ ಮೇಲೂ ಅದು ಆಕ್ರಮಣ ಮಾಡಲಿಲ್ಲ. ಸ್ವಾಮೀಜಿ ನೀಡಿದ ‘ವಿಶ್ವಧರ್ಮ’ವು ತನ್ನ ಹೊಸತನ ದಿಂದ ಎಲ್ಲರನ್ನೂ ಬೆರಗುಗೊಳಿಸುವಂತಿದ್ದು, ಮತಭ್ರಾಂತತೆಯ ಮೂಲಕ್ಕೇ ಕುಠಾರದೇಟನ್ನು ಹಾಕಿತ್ತು.

ಸ್ವಾಮೀಜಿ ಅಧಿಕಾರವಾಣಿಯಿಂದ ಮಾತನಾಡಿದರು; ಏಕೆಂದರೆ ಅವರೊಬ್ಬ ಸಿದ್ಧಪುರುಷರು. ಇತರ ಮತೀಯರು ಕೇವಲ ತಮ್ಮ ಮತವೇ ಸರಿಯೆಂಬ ಕುರುಡುನಂಬಿಕೆಯನ್ನೇ ಪಟ್ಟಾಗಿ ಹಿಡಿದು ಕುಳಿತಿದ್ದರೆ ಸ್ವಾಮೀಜಿ ತಮ್ಮ ನೇರ ಸಾಕ್ಷಾತ್ಕಾರದ ಬಲದಿಂದ ಉಚ್ಚ ಆಧ್ಯಾತ್ಮಿಕ ಸತ್ಯವನ್ನು ಬೋಧಿಸಿದರು. ಹೀಗೆ ಅವರ ಮೂಲಕ ಸಮಸ್ತ ಆಧ್ಯಾತ್ಮಿಕತೆಯ ಭಾವಪ್ರಕಾಶವು ಸಭೆಯ ಮೇಲೆ ಚಿರಮುದ್ರಿತವಾಯಿತು. ಸರ್ವಧರ್ಮಗಳ ಪ್ರತಿನಿಧಿಗಳಿಂದ ಕೂಡಿದ ಆ ಸಮಸ್ತ ಸಭೆಯೇ ಅವರನ್ನು ನವನೂತನ ಧಾರ್ಮಿಕ ಚಿಂತನೆಯ ಪ್ರವರ್ತನಾಚಾರ್ಯರೆಂದು ಉತ್ಸಾಹ ದಿಂದ ಉದ್ಘೋಷಿಸಿತು.

ಆದರೆ ಈ ಮೂಲಕ ಅವರಿಂದ ನಿಜಕ್ಕೂ ಅತಿ ದೊಡ್ಡ ಸೇವೆ ಸಂದದ್ದು ಭಾರತಕ್ಕೆ. ಹೇಗೆಂದರೆ, ತಮ್ಮ ಧರ್ಮದಲ್ಲಿರುವ ಯಾವ ಏಕತೆಯು ಹಿಂದೂಗಳಿಗೇ ಇನ್ನೂ ಗೋಚರಿಸಿರಲಿಲ್ಲವೋ ಅಂತಹ ಭಾರತೀಯ ಆದರ್ಶಗಳ ಏಕತೆಯನ್ನು ತಿಳಿಸಿಕೊಡುವುದರ ಮೂಲಕ ಹಿಂದೂ ಜೀವನಾದರ್ಶಕ್ಕೆ ಒಂದು ಗೌರವವನ್ನು ತಂದುಕೊಟ್ಟರು. ಇನ್ನೂ ನಿಶ್ಚಿತವಾಗಿ ಹೇಳಬೇಕೆಂದರೆ ಸ್ವಾಮೀಜಿಯ ಉಪನ್ಯಾಸದಲ್ಲಿ ಹಿಂದೂಧರ್ಮಕ್ಕೆ ಅವರ ಪ್ರಧಾನ ಕೊಡುಗೆಯಾಗಿ ಮೈದಾಳಿ ದ್ದೆಂದರೆ ಇವು–ಮೊದಲನೆಯದಾಗಿ, ಪುರಾತನ ಹಿಂದೂದರ್ಶನಗಳ ತಾತ್ವಿಕ ಹಾಗೂ ಧಾರ್ಮಿಕ ಸಮನ್ವಯ; ಎರಡನೆಯದಾಗಿ, ಅತ್ಯಂತ ಕೆಳಮಟ್ಟದಿಂದ ಹಿಡಿದು ಅತ್ಯುನ್ನತ ಮಟ್ಟದವರೆಗಿನ ಎಲ್ಲ ಬಗೆಯ ಧಾರ್ಮಿಕ ಆಚರಣೆಗಳನ್ನೂ ತನ್ನಲ್ಲಿ ಒಳಗೊಳ್ಳುವ ವಿಶ್ವಧರ್ಮದ ಕುರಿತಾದ ದರ್ಶನ; ಮೂರನೆಯದಾಗಿ, ತಮ್ಮ ಪಾಂಡಿತ್ಯಪೂರ್ಣ ಆಧ್ಯಾತ್ಮಿಕ ವ್ಯಾಖ್ಯಾನದಿಂದ ಅವರು ಪಾಶ್ಚಾತ್ಯ ಜಗತ್ತಿನ ಉಜ್ವಲ ಚಿಂತಕರ ಹಾಗೂ ಮತಧರ್ಮ ಶಾಸ್ತ್ರಜ್ಞರ ಮಧ್ಯದಲ್ಲಿ ಹಿಂದೂ ಧರ್ಮಕ್ಕೆ ಗಳಿಸಿಕೊಟ್ಟ ಘನ ಗೌರವಯುತ ಸುಸ್ಥಿರಸ್ಥಾನ. ಈ ಉಪನ್ಯಾಸದ ಕುರಿತಾಗಿ ಸೋದರಿ ನಿವೇದಿತೆ ಹೇಳುತ್ತಾಳೆ: “ಸ್ವಾಮೀಜಿ ಸರ್ವಧರ್ಮಸಮ್ಮೇಳನದ ಮುಂದೆ ಮಾಡಿದ ಈ ಉಪನ್ಯಾಸದ ಕುರಿತಾಗಿ ಹೀಗೆ ಹೇಳಬಹುದು: ಅವರು ಮಾತನಾಡಲು ಪ್ರಾರಂಭಿಸಿದಾಗ ಅದು ‘ಹಿಂದೂಗಳ ಧಾರ್ಮಿಕ ಭಾವನೆಗಳು’ ಎಂಬ ವಿಷಯವನ್ನು ಕುರಿತಾಗಿತ್ತು. ಆದರೆ ಅವರು ತಮ್ಮ ಮಾತನ್ನು ಮುಗಿಸಿದಾಗ ಹಿಂದೂಧರ್ಮವೇ ಸೃಷ್ಟಿಯಾಗಿತ್ತು.”

ಹಿಂದೂಧರ್ಮದ ಕುರಿತು ಉಪನ್ಯಾಸ ಮಾಡಿದ ಮರುದಿನ (ಸೆಪ್ಟೆಂಬರ್ ೨ಂರಂದು) ಸ್ವಾಮೀಜಿ ಒಂದು ಚಿಕ್ಕ ಭಾಷಣ ಮಾಡಿ ‘ಧರ್ಮಕ್ಕಲ್ಲ ಭರತಖಂಡದಲ್ಲಿ ಬರಗಾಲ’ ಎಂಬು ದನ್ನು ಸ್ಪಷ್ಟಪಡಿಸಿದರು. ಈ ಸಂದರ್ಭದಲ್ಲಿ ಮಾತನಾಡುತ್ತ ಅವರೆಂದರು:

“ಕ್ರೈಸ್ತರು ಸದ್ವಿಮರ್ಶೆಗೆ ಸಿದ್ಧರಾಗಿರಬೇಕು. ನಿಮ್ಮಲ್ಲಿರುವ ಒಂದೆರಡು ಲೋಪದೋಷ ಗಳನ್ನು ತೋರಿಸಿಕೊಟ್ಟರೆ ನನ್ನನ್ನು ಮನ್ನಿಸುವಿರೆಂದು ನಂಬುತ್ತೇನೆ. ನೀವು ಕ್ರೈಸ್ತರು, ಭಾರತೀಯ ‘ಅನಾಗರಿಕ’ರ ಆತ್ಮದ ರಕ್ಷಣೆಯ ಸಲುವಾಗಿ ಪಾದ್ರಿಗಳನ್ನು ಕಳಿಸುತ್ತೀರಿ. ಆದರೆ ಹೊಟ್ಟೆಗಿಲ್ಲದೆ ಸಾಯುವವರ ದೇಹದ ಸಂರಕ್ಷಣೆಯನ್ನೇಕೆ ಮಾಡಲು ಪ್ರಯತ್ನಿಸುವುದಿಲ್ಲ? (ಕರತಾಡನ.) ಭಾರತದಲ್ಲಿ ಭಯಂಕರ ಬರಗಾಲದಲ್ಲಿ ಲಕ್ಷಾಂತರ ಜನರು ಹಸಿವೆಯಿಂದ ಸತ್ತರು. ಕ್ರೈಸ್ತರಾದ ನೀವು ಅವರಿಗಾವ ಸಹಾಯವನ್ನೂ ಮಾಡಲಿಲ್ಲ. ಆದರೆ ಭಾರತದಲ್ಲಿ ಎಲ್ಲೆಲ್ಲಿಯೂ ಇಗರ್ಜಿ ಗಳನ್ನು ಮಾತ್ರ ಕಟ್ಟುತ್ತೀದ್ದೀರಿ. ಕ್ರೈಸ್ತ ಮಿಷನರಿಗಳು ಹಿಂದೂಗಳಿಗೆ ನವಜೀವನವನ್ನು ಕೊಡು ತ್ತೇವೆ ಎಂದು ಬರುತ್ತಾರೆ. ಆದರೆ ಯಾವ ಷರತ್ತಿನ ಮೇಲೆ? ಹಿಂದೂಗಳು ತಮ್ಮ ಸ್ವಧರ್ಮ ವನ್ನು ತ್ಯಜಿಸಿ ಕ್ರೈಸ್ತರಾಗಬೇಕು! ಇದು ಸರಿಯೆ? ಭ್ರಾತೃತ್ವದ ಅರ್ಥವನ್ನು ತೋರಿಸಿಕೊಡ ಬೇಕೆಂದು ನಿಮಗೆ ಇಚ್ಛೆಯಿದ್ದರೆ, ಒಬ್ಬ ಹಿಂದುವು ತನ್ನ ಧರ್ಮಕ್ಕೆ ನಿಷ್ಠಾವಂತನಾಗಿದ್ದರೂ ಅವನನ್ನು ಹೆಚ್ಚು ಕರುಣೆಯಿಂದ ನೋಡಿ. ಸರಿಯಾಗಿ ಒಂದು ತುತ್ತು ಅನ್ನವನ್ನು ಸಂಪಾದಿಸಿ ಕೊಳ್ಳುವುದು ಹೇಗೆ ಎಂಬುದನ್ನು ಕಲಿಸಿಕೊಡಲು ಅವರಲ್ಲಿಗೆ ನಿಮ್ಮ ಮಿಷನರಿಗಳನ್ನು ಕಳಿಸಿ ಕೊಡಿ. ಅಸಂಬದ್ಧ ತಾತ್ವಿಕತೆಯನ್ನು ಬೋಧಿಸುವುದಕ್ಕಲ್ಲ. (ಭಾರೀ ಕರತಾಡನ.)”

ಇಷ್ಟು ಹೇಳಿ ಸ್ವಾಮೀಜಿ ತಮ್ಮ ಆರೋಗ್ಯ ಅಷ್ಟು ಸರಿಯಿಲ್ಲದ್ದರಿಂದ ತಾವಿನ್ನು ವಿಶ್ರಮಿಸಲು ಸಭಿಕರ ಅನುಮತಿ ಬೇಡಿದರು. ಆದರೆ ಮತ್ತೆ ಕರತಾಡನ ಹಾಗೂ “ಮುಂದುವರಿಸಿರಿ!” ಎಂಬ ಕೂಗುಗಳು ಕೇಳಿಬಂದದ್ದರಿಂದ ತಮ್ಮ ಮಾತನ್ನು ಮುಂದುವರಿಸಿದರು:

“ನಾನು ಭಿಕ್ಷುಕರೆಂದು ಕರೆಯಲ್ಪಟ್ಟಿರುವ ಸಂನ್ಯಾಸಿಗಳ ವರ್ಗಕ್ಕೆ ಸೇರಿದವನು. ಆದರೆ ಅದು ನನ್ನ ಜೀವನದ ಒಂದು ಹೆಮ್ಮೆಯ ವಿಚಾರ. ಆ ಅರ್ಥದಲ್ಲಿ ‘ಕ್ರಿಸ್ತನಂತಹ’ ಜೀವನವನ್ನು ನಡೆಸಲು ನಾನು ಹೆಮ್ಮೆ ಪಡುತ್ತೇನೆ. ಇಂದು ನನಗೇನು ದೊರಕುತ್ತದೆಯೋ ಅದನ್ನು ತಿನ್ನುತ್ತೇನೆ. ನಾಳಿನ ಬಗ್ಗೆ ಚಿಂತಿಸುವುದೇ ಇಲ್ಲ. ಇದೇ ವೇದಿಕೆಯ ಮೇಲೆ ಕುಳಿತ ಕೆಲವು ಮಹನೀಯರು ನನ್ನ ಮಾತಿನ ಸತ್ಯತೆಯನ್ನು ಪ್ರಮಾಣೀಕರಿಸಬಲ್ಲರು. ನಾನು ಭಗವಂತನ ಸಲುವಾಗಿ ಒಬ್ಬ ಭಿಕ್ಷುಕನಾಗಿರಲು ಹೆಮ್ಮೆ ಪಡುತ್ತೇನೆ. ಭಾರತದಲ್ಲಿ, ಯಾವುದನ್ನೇ ಆಗಲಿ ಧನಾರ್ಜನೆಗಾಗಿ ಬೋಧಿಸುವುದು ತುಚ್ಛವೆಂದು ಪರಿಗಣಿಸಲ್ಪಡುತ್ತದೆ. ಆದರೆ ಹಣಕ್ಕಾಗಿ ಧರ್ಮವನ್ನು ಬೋಧಿಸು ವುದಂತೂ ಎಂತಹ ಅವಮಾನವೆಂದರೆ, ಯಾರಾದರೂ ಹಾಗೆ ಮಾಡಿದರೆ ಅವನ ಮುಖದ ಮೇಲೆ ಉಗುಳಿ, ಜಾತಿಯಿಂದ ಬಹಿಷ್ಕಾರ ಹಾಕುತ್ತಾರೆ... ಆದರೆ ಭಾರತದ ಈ ಸಂನ್ಯಾಸಿ ಗಳನ್ನೆಲ್ಲ ಒಗ್ಗೂಡಿಸಲು ಸಾಧ್ಯವಾದರೆ ಅದ್ಭುತವಾದ ಶಕ್ತಿಯ ಮೂಲವೊಂದನ್ನು ನಿರ್ಮಿಸಿ ದಂತಾಗುತ್ತದೆ. ಸಮಾಜವನ್ನು ಪುನಶ್ಚೇತನಗೊಳಿಸಲು ಈ ಶಕ್ತಿಯನ್ನು ಬಳಸಬಹುದು. ನಾನು ಅದನ್ನು ಒಗ್ಗೂಡಿಸಲು ಪ್ರಯತ್ನಿಸಿದೆ. ಆದರೆ ಹಣದ ಕೊರತೆಯಿಂದಾಗಿ ಅದರಲ್ಲಿ ವಿಫಲನಾದೆ. ಬಹುಶಃ ನನಗೆ ಬೇಕಾದ ಸಹಾಯವು ಅಮೆರಿಕೆಯಲ್ಲಿ ದೊರಕಬಹುದು.”

ಹೀಗೆಂದ ಸ್ವಾಮೀಜಿ ನಸುನಗುತ್ತ ನುಡಿದರು: “ಆದರೆ ಕ್ರೈಸ್ತಧರ್ಮೀಯರಿಂದ ‘ಅನಾಗರಿಕ’ ನೊಬ್ಬ ಯಾವುದೇ ಸಹಾಯ ಪಡೆಯುವುದು ಎಷ್ಟು ಕಷ್ಟದ್ದೆಂಬುದು ನಿಮಗೆಲ್ಲ ತಿಳಿದೇ ಇದೆ! (ಭಾರೀ ಕರತಾಡನ.) ಆದರೂ ಸ್ವಾತಂತ್ರ್ಯದ ತವರೂರೆನಿಸಿದ ಈ ದೇಶದ ಬಗ್ಗೆ ನಾನು ಬಹಳಷ್ಟು ಕೇಳಿರುವುದರಿಂದ ನಿರಾಶನಾಗಿಲ್ಲ... ”

ಹೀಗೆ ಹೇಳಿ ಸ್ವಾಮೀಜಿ, ತಮ್ಮ ಮಾತನ್ನು ಮುಕ್ತಾಯಗೊಳಿಸಲು ನೋಡಿದರು. ಆದರೆ ಮತ್ತೆ ಸಭಿಕರು ಹಷೋದ್ಗಾರ ಮಾಡಿ, ಮಾತನ್ನು ಮುಂದುವರಿಸುವಂತೆ ಕೇಳಿಕೊಂಡರು. ಆಗ ಸ್ವಾಮೀಜಿ ವಿಧಿಯಿಲ್ಲದೆ ಜನರ ಒತ್ತಾಯಕ್ಕೆ ಮಣಿದು, ಭಾಷಣವನ್ನು ಮುಂದುವರಿಸಿ ಪುನರ್ಜನ್ಮ ಸಿದ್ಧಾಂತವನ್ನು ವಿವರಿಸಿದರು.

ಹಿಂದೆಯೇ ಹೇಳಿದಂತೆ, ಸರ್ವಧರ್ಮ ಸಮ್ಮೇಳನದ ಮುಖ್ಯ ಸಭೆಯಲ್ಲಿ ಮಾತ್ರವಲ್ಲದೆ, ಅದರ ವೈಜ್ಞಾನಿಕ ವಿಭಾಗದಲ್ಲೂ ಸ್ವಾಮೀಜಿ ಹಲವಾರು ಉಪನ್ಯಾಸಗಳನ್ನು ಕೊಟ್ಟರು. ವೈಜ್ಞಾನಿಕ ವಿಭಾಗದ ಅಧ್ಯಕ್ಷರಾಗಿದ್ದ ಮರ್ವಿನ್ ಮಾರಿ ಸ್ನೆಲ್​ರವರು ಸ್ವಾಮೀಜಿಯ ಆತ್ಮೀಯ ಗೆಳೆಯರೂ ಹಿಂದೂಧರ್ಮದ ನಿಷ್ಠಾವಂತ ಪ್ರತಿಪಾದಕರೂ ಆದರು. ಈ ವಿಭಾಗದ ಸಭೆಯನ್ನುದ್ದೇಶಿಸಿ ಸ್ವಾಮೀಜಿ ಸೆ. ೨೨ರಂದು ಮೊದಲ ಬಾರಿ ಮಾತನಾಡಿದರು. ವಿಷಯ: “ಸಾಂಪ್ರದಾಯಿಕ ಹಿಂದೂಧರ್ಮ ಹಾಗೂ ವೇದಾಂತತತ್ತ್ವ.” ಎಂದಿನಂತೆ ಅವರ ಭಾಷಣಕ್ಕೆ ಜನ ಕಿಕ್ಕಿರಿದಿದ್ದರು. ಭಾಷಣ ಮುಗಿದ ಮೇಲೆ ಸಭಿಕರು ನೂರಾರು ಪ್ರಶ್ನೆಗಳನ್ನು ಹಾಕಿದರು. ಅವುಗಳನ್ನೆಲ್ಲ ಸ್ವಾಮೀಜಿ ಅದ್ಭುತ ಕೌಶಲದಿಂದ ಹಾಗೂ ಅತ್ಯಂತ ಸರಳವಾದ ರೀತಿಯಲ್ಲಿ ಉತ್ತರಿಸಿದರು. ಅದೇ ದಿನ ಮಧ್ಯಾಹ್ನ ಮತ್ತೆ ಅವರು “ಭಾರತದ ಆಧುನಿಕ ಧರ್ಮಗಳು” ಎಂಬ ವಿಷಯವಾಗಿ ಮಾತನಾಡಿ ದರು. ಸೆ. ೨೫ರ ಮಧ್ಯಾಹ್ನದ ಅಧಿವೇಶನದಲ್ಲಿ “ಹಿಂದೂಧರ್ಮದ ತಿರುಳು” ಎಂಬ ವಿಷಯದ ಬಗ್ಗೆ ಉಪನ್ಯಾಸ ನೀಡಿದರು. ಇವುಗಳಲ್ಲದೆ ಅವರು ಅನೇಕ ಆಶುಭಾಷಣಗಳನ್ನೂ ಮಾಡಿದರು.

ಸೆ. ೨೬ರ ಸಂಜೆ, “ಬೌದ್ಧ ಧರ್ಮಕ್ಕೆ ಬೆಂಬಲವಾಗಿ” ಎಂಬ ವಿಷಯದ ಬಗ್ಗೆ ಚರ್ಚೆ ನಡೆ ಯಿತು. ಈ ವಿಷಯದ ಬಗ್ಗೆ ಮಾತನಾಡಿದ ಸಿಂಹಳದ ಬೌದ್ಧರ ಪ್ರತಿನಿಧಿಯಾದ ಧರ್ಮಪಾಲರು ತಮ್ಮ ಮಾತನ್ನು ಮುಕ್ತಾಯಗೊಳಿಸುವಾಗ, ಬೌದ್ಧಧರ್ಮದ ಬಗ್ಗೆ ತಮ್ಮ ವಿಮರ್ಶೆಗಳನ್ನು ತಿಳಿಸುವಂತೆ ವಿವೇಕಾನಂದರನ್ನು ಪ್ರಾರ್ಥಿಸಿಕೊಂಡರು. ಏಕೆಂದರೆ ಅವರೆಷ್ಟೇ ಆದರೂ ಬೌದ್ಧ ರಲ್ಲದವರಾದ್ದರಿಂದ, ಅವರು ಬೌದ್ಧಧರ್ಮದ ಬಗ್ಗೆ ಪಕ್ಷಪಾತ ತೋರಲಾರರೆಂದು ಧರ್ಮ ಪಾಲರು ಹೇಳಿದರು. ಅದಕ್ಕೊಪ್ಪಿ ಸ್ವಾಮೀಜಿ, ಸಭಿಕರ ಹಷೋದ್ಗಾರದ ನಡುವೆ “ಬೌದ್ಧಧರ್ಮ –ಹಿಂದೂಧರ್ಮಕ್ಕೆ ಪೂರಕ” ಎಂಬ ತಮ್ಮ ಭಾಷಣವನ್ನು ಪ್ರಾರಂಭಿಸಿದರು:

“... ನಿಮಗೆ ತಿಳಿದಿರುವಂತೆ ನಾನು ಬೌದ್ಧನಲ್ಲ. ಆದರೂ ನಾನೊಬ್ಬ ಬೌದ್ಧ. ಚೀನಾ, ಜಪಾನ್ ಮತ್ತು ಸಿಂಹಳದ ಜನರು ಬುದ್ಧನನ್ನು ಒಬ್ಬ \textit{ಮಹಾಪುರುಷ}ನೆಂದು ಪರಿಗಣಿಸಿ ಅವನ ಘನಸಂದೇಶವನ್ನು ಅನುಸರಿಸಿದರೆ, ಭರತಖಂಡ ಅವನನ್ನು \textit{ಅವತಾರಪುರುಷ}ನೆಂದು ಆರಾಧಿಸು ತ್ತದೆ. ನಾನು ಬೌದ್ಧಧರ್ಮವನ್ನೀಗ ವಿಮರ್ಶಿಸುತ್ತೇನೆ ಎಂದು ನೀವು ಈಗ ತಾನೆ ಕೇಳಿದಿರಿ. ಆದರೆ ನೀವು ಅದರಿಂದ ಇಷ್ಟನ್ನು ಮಾತ್ರ ತಿಳಿದುಕೊಳ್ಳಬೇಕು: ನಾನು ಯಾರನ್ನು ಅವತಾರವೆಂದು ಪೂಜಿಸುತ್ತೇನೆಯೋ ಅಂತಹ ಬುದ್ಧನನ್ನು ಟೀಕಿಸುವ ಗೋಜಿಗೆ ಹೋಗುವುದಿಲ್ಲ. ಆದರೆ ಅವನ ಶಿಷ್ಯರು ಅವನನ್ನು ಚೆನ್ನಾಗಿ ಅರಿತುಕೊಳ್ಳಲಿಲ್ಲವೆಂಬುದು ನಮ್ಮ ಅಭಿಪ್ರಾಯ. ಹಿಂದೂ ಧರ್ಮಕ್ಕೂ ಬೌದ್ಧ ಧರ್ಮಕ್ಕೂ ಇರುವ ಸಂಬಂಧ ಎಂತಹದೆಂದರೆ ಕ್ರೈಸ್ತಧರ್ಮಕ್ಕೂ ಯಹೂದ್ಯಧರ್ಮಕ್ಕೂ ಇರುವ ಸಂಬಂಧದಂತೆ. ಏಸುಕ್ರಿಸ್ತನು ಯಹೂದ್ಯನಾಗಿದ್ದನು, ಬುದ್ಧನು ಹಿಂದುವಾಗಿದ್ದನು. ಯಹೂದ್ಯರು ಕ್ರಿಸ್ತನನ್ನು ತಿರಸ್ಕರಿಸಿದ್ದು ಮಾತ್ರವಲ್ಲದೆ, ಅವನನ್ನು ಶಿಲುಬೆಗೆ ಏರಿಸಿದರು. ಆದರೆ ಹಿಂದುಗಳು ಶಾಕ್ಯಮುನಿಯನ್ನು ದೇವರೆಂದು ಸ್ವೀಕರಿಸಿ ಆರಾಧಿಸಿದರು. ಯಾವುದನ್ನು ಬುದ್ಧನ ಉಪದೇಶವೆಂದು ಭಾವಿಸುತ್ತೇವೆಯೋ ಅದಕ್ಕೂ ಆಧುನಿಕ ಬೌದ್ಧ ಧರ್ಮಕ್ಕೂ ನಾವು ತೋರುವ ವ್ಯತ್ಯಾಸ ಇದು: ಶಾಕ್ಯಮುನಿ ಹೊಸದೇನನ್ನೂ ಬೋಧಿಸಲು ಬರಲಿಲ್ಲ. ಏಸುಕ್ರಿಸ್ತನಂತೆ ಅವನೂ ಪೂರ್ಣಗೊಳಿಸಲು ಬಂದನೆ ಹೊರತು ಧ್ವಂಸ ಮಾಡು ವುದಕ್ಕಲ್ಲ. ಏಸುಕ್ರಿಸ್ತನ ವಿಷಯದಲ್ಲಿ ಆಗ ಅವನನ್ನು ಅರ್ಥಮಾಡಿಕೊಳ್ಳದಿದ್ದವರೆಂದರೆ ಆಗ ಇದ್ದ ಯಹೂದ್ಯರು. ಆದರೆ ಬುದ್ಧನ ವಿಷಯದಲ್ಲಿ ಅವನನ್ನು ಅರಿತುಕೊಳ್ಳದಿದ್ದವರೆಂದರೆ ಅವನ ಅನುಯಾಯಿಗಳೇ. ಹೇಗೆ ತಮ್ಮ ಹಳೆಯ ಒಡಂಬಡಿಕೆಯು \eng{(Old Testament)} ಕ್ರಿಸ್ತ ನಿಂದ ಪರಿಪೂರ್ಣವಾಯಿತೆಂದು ಯಹೂದ್ಯರು ತಿಳಿಯದೆ ಹೋದರೋ, ಹಾಗೆಯೇ ಹಿಂದೂ ಧರ್ಮದ ಸತ್ಯವು ಬುದ್ಧನಿಂದ ಪರಿಪೂರ್ಣತೆಯನ್ನು ಪಡೆಯಿತು ಎಂಬುದನ್ನು ಬೌದ್ಧರು ತಿಳಿದುಕೊಳ್ಳಲಿಲ್ಲ....

“ಶಾಕ್ಯಮುನಿಯು ಜಾತಿಮತಗಳ ಕಟ್ಟಳೆಗಳನ್ನೆಲ್ಲ ಮೀರಿದ ಒಬ್ಬ ಸಂನ್ಯಾಸಿಯಾಗಿದ್ದ. ಗುಪ್ತವಾಗಿದ್ದ ವೇದಶಾಸ್ತ್ರಗಳಿಂದ ತತ್ತ್ವವನ್ನು ಹೊರತೆಗೆದು ಜಗತ್ತಿನಲ್ಲಿ ಘಂಟಾಘೋಷವಾಗಿ ಸಾರುವಷ್ಟು ಅವನ ಹೃದಯ ವಿಶಾಲವಾಗಿತ್ತು. ಇದೇ ಅವನ ಮಹಿಮೆ. ಒಬ್ಬ ಲೋಕಗುರುವಿನ ಮಹಾಮಹಿಮೆಯಿರುವುದು, ಅವನು ಪ್ರತಿಯೊಬ್ಬರ ಮೇಲೂ–ಅದರಲ್ಲೂ ಅಜ್ಞಾನಿಗಳ ಹಾಗೂ ದೀನರ ಮೇಲೆ–ತೋರುವ ಸಹಾನುಭೂತಿಯಲ್ಲಿ. ಬುದ್ದನ ಕಾಲದಲ್ಲಿ ಸಂಸ್ಕೃತವು ಜನಸಾಮಾನ್ಯರ ಭಾಷೆಯಾಗಿರಲಿಲ್ಲ; ಅದು ಕೇವಲ ಪಂಡಿತರ ಭಾಷೆಯಾಗಿತ್ತು. ಬುದ್ಧನ ಕೆಲವು ಬ್ರಾಹ್ಮಣಶಿಷ್ಯರು ಅವನ ಉಪದೇಶಗಳನ್ನು ಸಂಸ್ಕೃತಕ್ಕೆ ಅನುವಾದಿಸಲು ಇಚ್ಛಿಸಿದರು. ಆದರೆ ಬುದ್ಧ ಅದಕ್ಕೊಪ್ಪದೆ ಹೇಳಿದ, “ನಾನು ಬಡವರು ಹಾಗೂ ಜನಸಾಮಾನ್ಯರಿಗಾಗಿ ಅವರ ಭಾಷೆ ಯಲ್ಲೇ ಮಾತನಾಡುತ್ತೇನೆ” ಎಂದು. ಆದ್ದರಿಂದಲೇ ಇಂದಿಗೂ ಅವನ ಉಪದೇಶದ ಬಹು ಪಾಲು ಇರುವುದು ಅಂದಿನ ಪಾಳೀ ಭಾಷೆಯಲ್ಲೇ.

“ಸಿದ್ಧಾಂತಗಳು ಏನೇ ಹೇಳಲಿ, ಎಲ್ಲಿಯವರೆಗೂ ಪ್ರಪಂಚದಲ್ಲಿ ಸಾವು ಎಂಬುದು ಇರು ತ್ತದೆಯೋ, ಮಾನವನ ಹೃದಯದಲ್ಲಿ ದುರ್ಬಲತೆ ಎಂಬುದು ಇರುತ್ತದೆಯೋ, ಎಲ್ಲಿಯವರೆಗೂ ಬಳಲಿದಾಗ ಮಾನವನ ಹೃದಯಾಂತರಾಳದಿಂದ ಒಂದು ಪ್ರಾರ್ಥನೆ ಹೊಮ್ಮುವುದೋ ಅಲ್ಲಿಯ ವರೆಗೂ ದೇವರಲ್ಲಿ ನಂಬಿಕೆಯೆಂಬುದು ಇದ್ದೇ ಇರುತ್ತದೆ. ಬುದ್ಧನ ಅನುಯಾಯಿಗಳು ದೇವರ ಅಸ್ತಿತ್ವವನ್ನೇ ಅಲ್ಲಗಳೆಯುವುದರ ಮೂಲಕ ಅತಿದೊಡ್ಡ ತೊಂದರೆಯನ್ನು ತಂದುಕೊಂಡರು. ತತ್ತ್ವರಂಗದಲ್ಲಿ ಇವರು ವೇದಗಳೆಂಬ ದೊಡ್ಡ ಬಂಡೆಗೆ ಡಿಕ್ಕಿ ಹೊಡೆದರು; ಅದನ್ನು ಭೇದಿಸ ಲಾರದೆ ಸೋತರು. ಮತ್ತೊಂದು ಕಡೆ, ಎಂದರೆ ಅನುಷ್ಠಾನ ರಂಗದಲ್ಲಿ, ಪ್ರತಿಯೊಬ್ಬ ಸ್ತ್ರೀಪುರುಷನೂ ಪ್ರೀತಿಯಿಂದ ಅಪ್ಪಿಕೊಂಡಿದ್ದ ದೇವರನ್ನು ಕಸಿದುಕೊಂಡರು. ಅದರ ಪರಿಣಾಮ ವಾಗಿ ಬೌದ್ಧಧರ್ಮವು ಭಾರತದಿಂದ ಮಾಯವಾಗಬೇಕಾಯಿತು. ಕಡೆಗೆ, ಬೌದ್ಧಧರ್ಮದ ತವರೂರಾದ ಭರತಖಂಡದಲ್ಲಿ ತಾನು ಬೌದ್ಧನೆಂದು ಹೇಳಿಕೊಳ್ಳುವ ಒಬ್ಬನೂ ಇಲ್ಲದಂತಾಯಿತು.

“ಆದರೆ ಬೌದ್ಧಧರ್ಮವನ್ನು ತಿರಸ್ಕರಿಸಿದ್ದರಿಂದ ವೇದಧರ್ಮಕ್ಕೂ ಸ್ವಲ್ಪ ನಷ್ಟವಾಯಿತು. ಬೌದ್ಧಧರ್ಮವು ಜನಸಾಮಾನ್ಯರನ್ನು ಮೇಲೆತ್ತಬೇಕೆಂಬ ಉತ್ಕಟೇಚ್ಛೆಯಿಂದ ಸರ್ವರೆಡೆಗೂ ಸಹಾನುಭೂತಿ ಹಾಗೂ ಕರುಣೆಯನ್ನು ಬೀರಿತ್ತು. ಆಗಿನ ಭರತ ಖಂಡದ ವಿಷಯವಾಗಿ ಗ್ರೀಕ್ ಇತಿಹಾಸಜ್ಞನೊಬ್ಬ ಬರೆದಿದ್ದ–‘ಸುಳ್ಳಾಡುವ ಹಿಂದುವೇ ಗೊತ್ತಿಲ್ಲ. ವ್ಯಭಿಚಾರಿಣಿಯೆಂದು ಕರೆಸಿಕೊಳ್ಳುವ ಹೆಂಗಸೇ ಇಲ್ಲ’ ಎಂದು. ಆದರೆ ಹಿಂದೂ ಸಮಾಜವನ್ನು ಇಂತಹ ಉತ್ತಮ ಸ್ಥಿತಿಗೆ ತಂದ ಸುಧಾರಣೆಗಳೆಲ್ಲ ಕಾಲಕ್ರಮದಲ್ಲಿ ಕಣ್ಮರೆಯಾದುವು... 

“ಹಿಂದೂಧರ್ಮವು ಬೌದ್ಧಧರ್ಮವಿಲ್ಲದೆ ಬಾಳಲಾರದು. ಬೌದ್ಧಧರ್ಮವೂ ಹಿಂದೂಧರ್ಮ ವಿಲ್ಲದೆ ಬಾಳಲಾರದು. ಆದ್ದರಿಂದ ಇಬ್ಬರ ನಡುವಣ ಭೇದವು ಏನನ್ನು ಎತ್ತಿತೋರಿಸಿದೆ ಎಂಬುದನ್ನು ಗಮನಿಸಿ: ಬ್ರಾಹ್ಮಣರ ಪಾಂಡಿತ್ಯ ಹಾಗೂ ತತ್ತ್ವವಿಲ್ಲದೆ ಬೌದ್ಧರು ನಿಲ್ಲಲಾರರು; ಅಥವಾ ಬುದ್ದದೇವನ ಹೃದಯವಿಲ್ಲದೆ ಬ್ರಾಹ್ಮಣರು ನಿಲ್ಲಲಾರರು. ಬೌದ್ಧರಿಗೂ ಬ್ರಾಹ್ಮಣ ರಿಗೂ ಉಂಟಾದ ಭಿನ್ನಾಭಿಪ್ರಾಯವೆ ಭರತಖಂಡದ ಅವನತಿಗೆ ಕಾರಣ. ಆದ್ದರಿಂದಲೇ ಇಂದು ಭಾರತದಲ್ಲಿ ೩ಂ ಕೋಟಿ ಭಿಕ್ಷುಕರಿರುವುದು; ಮತ್ತು ಕಳೆದ ಸಾವಿರ ವರ್ಷದಿಂದಲೂ ಅದು ತನ್ನನ್ನು ಗೆದ್ದವರ ಗುಲಾಮರಾಗಿರುವುದು. ಆದ್ದರಿಂದ ಈಗಲಾದರೂ ಬ್ರಾಹ್ಮಣರ ಅದ್ಭುತ ಧೀಶಕ್ತಿಯೊಂದಿಗೆ, ಆ ಮಹಾಗುರುವಾದ ಬುದ್ಧಭಗವಂತನ ಪ್ರೇಮ, ಮತ್ತೊಬ್ಬರಿಗಾಗಿ ಮರುಗುವ ಆ ವಿಶ್ವಾನುಕಂಪ ಶಕ್ತಿ–ಇವುಗಳನ್ನು ಒಂದುಗೂಡಿಸೋಣ.”

ಹೀಗೆ ಸಮ್ಮೇಳನದ ಅಧಿಕೃತ ಸಭೆಗಳಲ್ಲೂ ವೈಜ್ಞಾನಿಕ ವಿಭಾಗದ ಸಭೆಗಳಲ್ಲೂ ಅಲ್ಲದೆ ಸ್ವಾಮೀಜಿ ಶಿಕಾಗೋದ ಇತರ ಹಲವಾರು ಸ್ಥಳಗಳಲ್ಲಿ ಉಪನ್ಯಾಸಗಳನ್ನು ನೀಡಿದರು. ಸೆಪ್ಟೆಂಬರ್ ೨೪ರಂದು ಅವರು ಅಲ್ಲಿನ ‘ಮೂರನೇ ಯುನಿಟೇರಿಯನ್ ಚರ್ಚ್​’ ಎಂಬಲ್ಲಿ ‘ಭಗವತ್ಪ್ರೇಮ’ ಎಂಬ ವಿಷಯವಾಗಿ ಮಾತನಾಡಿದರು. ಮತ್ತು ಸಮ್ಮೇಳನದ ಮತ್ತೊಂದು ಅಂಗವಾದ ‘ವಿಶ್ವ ಧಾರ್ಮಿಕ ಏಕತಾ ಸಭೆ\eng{’ (Universal Religious Unity Congress)}ಯಲ್ಲೂ ಅವರೊಮ್ಮೆ ಮಾತನಾಡಿದರು. ಈ ಉಪನ್ಯಾಸಗಳೊಂದಿಗೆ, ಅನೇಕ ಸತ್ಕಾರಕೂಟಗಳಲ್ಲೂ ಸಮಾರಂಭ ಗಳಲ್ಲೂ ಅನೌಪಚಾರಿಕ ಭಾಷಣಗಳನ್ನು ಮಾಡಿದರು. ತನ್ಮೂಲಕ ಅವರಿಗೆ ಅಮೆರಿಕದ ಹಾಗೂ ಇತರ ರಾಷ್ಟ್ರಗಳ ಅಸಂಖ್ಯ ಗಣ್ಯವ್ಯಕ್ತಿಗಳ ಹಾಗೂ ಪ್ರತಿಷ್ಠಿತರ ಪರಿಚಯವಾಯಿತು.

ಸಮ್ಮೇಳನದ ಪ್ರಾರಂಭದ ಸಂಜೆ, ವಿದೇಶಿಯ ಪ್ರತಿನಿಧಿಗಳಿಗಾಗಿ ರೆವರೆಂಡ್ ಜೆ. ಹೆಚ್. ಬರೋಸ್​ರವರು ಸತ್ಕಾರಕೂಟವನ್ನು ಏರ್ಪಡಿಸಿದ್ದರು. ವಿಶಾಲವಾದ ಬಂಗಲೆಯೊಂದರಲ್ಲಿ ಅದ್ದೂರಿಯ ಸಮಾರಂಭ ಏರ್ಪಾಡಾಗಿದ್ದು, ಅಲ್ಲಿನ ಕೋಣೆಗಳು ವಿವಿಧ ದೇಶಗಳ ನೂರಾರು ಧ್ವಜಗಳಿಂದ ಅಲಂಕೃತಗೊಂಡಿದ್ದುವು. ಮರುದಿನ ಸಮ್ಮೇಳನದ ಅಧ್ಯಕ್ಷರಾದ ಚಾರ್ಲ್ಸ್ ಬಾನಿಯವರು ತಮ್ಮ ಸ್ಥಾನಕ್ಕೆ ಅನುಸಾರವಾಗಿ ಆರ್ಟ್ ಇನ್ಸ್​ಟಿಟ್ಯೂಟ್​ನಲ್ಲಿ ಒಂದು ಭಾರೀ ಸರ್ವಾಜನಿಕ ಸಮಾರಂಭವನ್ನೇ ವ್ಯವಸ್ಥೆ ಮಾಡಿದ್ದರು. ಸೆಪ್ಟೆಂಬರ್ ೧೪ರಂದು, ಕೊಲಂಬಿಯನ್ ಮೇಳದ ಮಹಿಳಾ ನಿರ್ವಾಹಕರ ಒಕ್ಕೂಟದ ಅಧ್ಯಕ್ಷಿಣಿಯಾದ ಶ್ರೀಮತಿ ಪಾಟರ್ ಪಾಮರ್ ಎಂಬುವರು ಸಮ್ಮೇಳನದ ಪ್ರತಿನಿಧಿಗಳಿಗಾಗಿ ಸ್ವಾಗತ ಸಮಾರಂಭವೊಂದನ್ನು ನಡೆಸಿದರು. ಅಂದು ಆತಿಥೇಯರ ಕೋರಿಕೆಯಂತೆ ಪ್ರತಿನಿಧಿಗಳೆಲ್ಲ ತಮ್ಮತಮ್ಮ ರಾಷ್ಟ್ರದ ಮಹಿಳೆಯರ ಸ್ಥಾನಮಾನಗಳ ಬಗ್ಗೆ ಮಾತನಾಡಿದರು. ಸ್ತ್ರೀಸ್ವಾತಂತ್ರ್ಯ ಚಳವಳಿಯು ಅಮೆರಿಕದಲ್ಲಿ ಪೂರ್ಣ ರಭಸದಿಂದ ಸಾಗುತ್ತಿದ್ದ ದಿನಗಳು ಅವು. ಭಾರತವೊಂದು ಅನಾಗರಿಕ ರಾಷ್ಟ್ರವೆಂದು ಭಾವಿಸಿದ್ದ ಅಮೆರಿಕನ್ನರು, ಸಹಗಮನ ಮುಂತಾದ ದುಷ್ಟ ಪದ್ಧತಿಗಳ ಬಗ್ಗೆ ಅತಿಶಯೋಕ್ತಿಪೂರ್ಣವಾದ ವಿಕೃತ ವರದಿಗಳನ್ನು ಕೇಳಿ ಮತ್ತಷ್ಟು ಭಯಂಕರವಾದ ಚಿತ್ರವೊಂದನ್ನು ಕಲ್ಪಿಸಿಕೊಂಡಿದ್ದರು. ಸ್ತ್ರೀಸ್ವಾತಂತ್ರ್ಯ ಚಳವಳಿಯ ಬೆಂಬಲಿಗರಿಗಂತೂ ಭಾರತೀಯ ಸ್ತ್ರೀಯರ ಬಗ್ಗೆ ತೀವ್ರ ಕಳಕಳಿ ಇತ್ತು. ಮುಖ್ಯವಾಗಿ ಈ ಹಿಂದೆ ಹೇಳಿದ ರಮಾಬಾಯಿ ಮತ್ತು ಅವಳ ಹಿಂಬಾಲಕರು ಇಂತಹ ಪರಿಸ್ಥಿತಿಯನ್ನು ನಿರ್ಮಾಣ ಮಾಡಿ, ಅದನ್ನು ಸ್ವಾರ್ಥೋದ್ದೇಶಗಳಿಗೆ ಬಳಸಿಕೊಳ್ಳುತ್ತಿದ್ದರು. ಜನರ ಮನಸ್ಸಿನಿಂದ ಇಂತಹ ವಿಕೃತ ಕಲ್ಪನೆಗಳನ್ನೆಲ್ಲ ದೂರ ಮಾಡುವ ಕೆಲಸ ಈಗ ವಿವೇಕಾ ನಂದರದಾಗಿತ್ತು. ಆದ್ದರಿಂದಲೇ ಅವರು ಅಮೆರಿಕದಲ್ಲಿ ಎಲ್ಲೆಲ್ಲಿ ಹೋದರೂ, ಭಾರತೀಯ ಸ್ತ್ರೀಯರ ಬಗ್ಗೆ ಮಾತನಾಡುವಂತೆ ಬೇಡಿಕೆ ಬರುತ್ತಿತ್ತು. ಶ್ರೀಮತಿ ಪಾಮರ್ ಏರ್ಪಡಿಸಿದ್ದ ಸಮಾರಂಭದಲ್ಲಿ ಅವರು, ಶ್ರೋತೃಗಳ ಮುಂದೆ ಭಾರತೀಯ ಮಹಿಳೆಯರ ವಾಸ್ತವಿಕ ಚಿತ್ರ ವನ್ನಿಟ್ಟು, ಹಿಂದೂ ಸಂಸ್ಕೃತಿಯಲ್ಲಿ ಮಹಿಳೆಗೆ ಮೀಸಲಾಗಿರುವ ಉನ್ನತ ಸ್ಥಾನವನ್ನು ವಿವರಿಸಿ ದರು. ಶಿಕಾಗೋ ನಗರದ ಇನ್ನೂ ಅನೇಕ ಗಣ್ಯ ನಾಗರಿಕರು ಹಾಗೂ ಕ್ರೈಸ್ತ ಧರ್ಮಾಧಿಕಾರಿಗಳು ಏರ್ಪಡಿಸಿದ್ದ ಸಮಾರಂಭಗಳಲ್ಲಿ ಸಹಪ್ರತಿನಿಧಿಗಳೊಂದಿಗೆ ಸ್ವಾಮೀಜಿ ಭಾಗವಹಿಸಿದರು. ಈ ಎಲ್ಲ ಸತ್ಕಾರಕೂಟಗಳಲ್ಲೂ ಆಕರ್ಷಣೆಯ ಕೇಂದ್ರವಾಗಿದ್ದವರು ಅವರೇ ಎಂಬುದು ಪತ್ರಿಕಾ ವರದಿಗಳಿಂದ ತಿಳಿದುಬರುತ್ತದೆ.

ಸಮ್ಮೇಳನದ ಅಧಿಕೃತ ಸಭೆಗಳಲ್ಲಿ ಸ್ವಾಮೀಜಿ ಅನೇಕ ಬಾರಿ ಅನೌಪಚಾರಿಕ ಭಾಷಣಗಳನ್ನು ಮಾಡಿದರು. ಜನರ ಬೇಡಿಕೆಯ ಮೇರೆಗೆ, ಯಾವಾಗಲೂ ಅವರಿಗೆ ನಿಗದಿತವಾದ ಅರ್ಧಗಂಟೆ ಗಿಂತ ಹೆಚ್ಚಿನ ಕಾಲಾವಕಾಶವನ್ನು ನೀಡಲಾಗುತ್ತಿತ್ತು. ಅಲ್ಲದೆ ಹಿಂದೆಯೇ ಹೇಳಿದಂತೆ ಸಭೆಯ ಕಣ್ಮಣಿಯಾಗಿದ್ದ ಅವರ ಭಾಷಣವನ್ನು ಯಾವಾಗಲೂ ಅಧಿವೇಶನದ ಕಡೆಯಲ್ಲೇ ಇಡಲಾಗು ತ್ತಿತ್ತು! ‘ಬಾಸ್ಟನ್ ಈವನಿಂಗ್ ಟ್ರಾನ್ ಸ್ಕ್ರಿಪ್ಟ್​’ ಎಂಬ ಪತ್ರಿಕೆ ಈ ಬಗ್ಗೆ ವರದಿ ಮಾಡಿತು:

“ವಿವೇಕಾನಂದರು ತಮ್ಮ ಮಾತು-ಭಾವನೆಗಳ ವೈಭವದಿಂದಲೂ ಅದ್ಭುತ ಆಕಾರದಿಂದಲೂ ಸಮ್ಮೇಳನದ ಕಣ್ಮಣಿಯಾಗಿದ್ದಾರೆ. ಅವರು ವೇದಿಕೆಯ ಮೇಲೆ ಸುಮ್ಮನೆ ಅತ್ತಿಂದಿತ್ತ ಹೋದರೆ ಸಾಕು, ಕರತಾಡನ ಕೇಳಿಬರುತ್ತದೆ. ಆದರೆ ಅವರು ಸಾವಿರಾರು ಜನರ ಈ ಬಗೆಯ ಮೆಚ್ಚುಗೆ ಯನ್ನು, ಅಹಂಕಾರದ ಲೇಶವೂ ಇಲ್ಲದೆ ಶಿಶು ಸಹಜ ಸಂತೃಪ್ತಿಯಿಂದ ಸ್ವೀಕರಿಸುತ್ತಾರೆ. ಬಡತನ ಹಾಗೂ ಅನಾಮಧೇಯ ಸ್ಥಿತಿಯಿಂದ ಒಮ್ಮೆಗೇ ಸಮೃದ್ಧಿ ಹಾಗೂ ಪ್ರಖ್ಯಾತಿಗೆ ಏರುವಂತಾದದ್ದು ಈ ವಿನಮ್ರ ಬ್ರಾಹ್ಮಣಸಂನ್ಯಾಸಿಗೆ ಬಹಳ ವಿಚಿತ್ರವಾದ ಅನುಭವವೇ ಆಗಿರಬೇಕು.”

ಜೈನಧರ್ಮದ ಪ್ರತಿನಿಧಿಯಾಗಿದ್ದ ವೀರಚಂದ್ ಗಾಂಧಿಯವರು ಸ್ವಾಮೀಜಿಯ ಜನಪ್ರಿಯತೆ ಯನ್ನು ನೆನಪಿಸಿಕೊಂಡು ಹೇಳುತ್ತಾರೆ:

“ಸರ್ವಧರ್ಮಸಮ್ಮೇಳನದಲ್ಲಿ, ಭಾರತದ ಒಬ್ಬ ಪ್ರಖ್ಯಾತ ವಾಗ್ಮಿ (ಸ್ವಾಮೀಜಿ) ತನ್ನ ಭಾಷಣ ವನ್ನು ಮುಗಿಸಿದ ಕೂಡಲೆ, ಕೊಲಂಬಸ್ ಹಾಲ್​ನಲ್ಲಿ ಸೇರಿದ್ದ ಸಭಿಕರ ಪೈಕಿ, ಕೆಲವೊಮ್ಮೆ ಮೂರನೆಯ ಎರಡರಷ್ಟು ಜನ ಎದ್ದು ಮನೆಗೆ ಹೊರಟುಬಿಡುತ್ತಿದ್ದರು. ಅಲ್ಲದೆ ಇನ್ನು ಕೆಲವು ಭಾರತೀಯ ಪ್ರತಿನಿಧಿಗಳನ್ನು ಸಮ್ಮೇಳನದ ಅಧಿಕಾರಿಗಳು, ಜನರನ್ನು ಹಿಡಿದಿಡಲು ಉಪಯೋಗಿಸಿಕೊಳ್ಳುತ್ತಿದ್ದರು. ಆದ್ದರಿಂದ ಸಾವಿರಾರು ಜನರು ವಿಧಿಯಿಲ್ಲದೆ ಕುಳಿತು, ಕ್ರೈಸ್ತರ ದೀರ್ಘ-ಶುಷ್ಕ ಭಾಷಣಗಳನ್ನು ಕೇಳಬೇಕಾಗುತ್ತಿತ್ತು. ತಮಗೆ ಈ ಭಾಷಣಗಳು ಇಷ್ಟವಿಲ್ಲವೆಂದು ಜನರು ಸ್ಪಷ್ಟವಾಗಿ ತೋರ್ಪಡಿಸುತ್ತಿದ್ದರು. ಆದರೂ ಮುಂದಿನ ಭಾಷಣಕಾರರು ತಮ್ಮ ನೆಚ್ಚಿನ ನಾಯಕರಾದ ವಿವೇಕಾನಂದರೇ ಇರಬಹುದೆಂದು ಆಶಿಸಿ, ಗೊಣಗುಟ್ಟಿಕೊಳ್ಳುತ್ತ ಕುಳಿತಿರುತ್ತಿದ್ದರು.”

ಹೀಗೆಯೇ ಅಮೆರಿಕದ ಅನೇಕ ಪ್ರಮುಖ ಪತ್ರಿಕೆಗಳಾದ \eng{\textit{Rutherford American, Press of America, Interir Chicago, Chicago Tribune, Chicago Inter Ocean}} ಮೊದ ಲಾದುವು ಸ್ವಾಮೀಜಿಯ ಬಗ್ಗೆ ಶ್ಲಾಘನೆಯ ಮಾತುಗಳನ್ನು ಧಾರಾಳವಾಗಿ ಬರೆಯುತ್ತಿದ್ದುವು. ಸುಪ್ರಸಿದ್ಧ ಪತ್ರಿಕೆಗಳಲ್ಲಿ ಅವರ ಭಾಷಣಗಳ ವಿವರಪೂರ್ಣ ವರದಿಗಳು ಪ್ರಕಟವಾಗುತ್ತಿದ್ದುವು. ನ್ಯೂಯಾರ್ಕಿನ \eng{\textit{Critic}} ಎಂಬ ಪತ್ರಿಕೆ ಅವರನ್ನು “ದೈವದತ್ತ ವಾಗ್ಮಿ” ಎಂದು ಕೊಂಡಾಡಿತು. \eng{\textit{Review of Reviews}} ಪತ್ರಿಕೆಯು ಅವರ ಭಾಷಣಗಳನ್ನು ‘ಶ್ರೇಷ್ಠ ಮತ್ತು ಉದಾತ್ತ’ ಎಂದ ವರ್ಣಿಸಿತು.

ಹದಿನೇಳು ದಿನಗಳ ಸುದೀರ್ಘ ಅಧಿವೇಶನಗಳ ನಂತರ, ಸೆ. ೨೭ರಂದು ಸರ್ವಧರ್ಮ ಸಮ್ಮೇಳನವು ಮುಕ್ತಾಯಗೊಂಡಿತು. ಅಂದಿನ ಸಮಾರಂಭವು ಅತ್ಯಂತ ಉತ್ಸಾಹಪೂರ್ಣ ವಾತಾವರಣದಿಂದ ಕೂಡಿದ್ದು, ಎರಡು ಮುಖ್ಯ ಸಭಾಂಗಣಗಳಲ್ಲಿ ಏಳು ಸಹಸ್ರಕ್ಕೂ ಹೆಚ್ಚಿನ ಜನ ತುಂಬಿದ್ದರು. ಅಂದಿನ ಸಭೆಯಲ್ಲಿ ಪ್ರತಿನಿಧಿಗಳು ಸಮ್ಮೇಳನದ ಬಗ್ಗೆ ತಮ್ಮ ಅನಿಸಿಕೆಗಳನ್ನು, ಆಶೋತ್ತರಗಳನ್ನು ವ್ಯಕ್ತಪಡಿಸಿದರು. ಅಂದು ಸ್ವಾಮೀಜಿ ಮತ್ತೊಮ್ಮೆ ಅತ್ಯಂತ ಶ್ರೇಷ್ಠವೂ ಪರಿಣಾಮಕಾರಿಯೂ ಆದ ಭಾಷಣವೊಂದನ್ನು ಮಾಡಿದರು:

“ಕ್ರೈಸ್ತನಾದವನು ಹಿಂದುವೋ ಬೌದ್ಧನೋ ಆಗಿ ಮತಾಂತರಗೊಳ್ಳಬೇಕಾಗಿಲ್ಲ; ಅಂತೆಯೇ ಹಿಂದುವೋ ಬೌದ್ಧನೋ ಕ್ರೈಸ್ತನಾಗಬೇಕಿಲ್ಲ. ಆದರೆ ಪ್ರತಿಯೊಬ್ಬನೂ ಉಳಿದವರ ಉನ್ನತ ಭಾವನೆಗಳನ್ನು ಮೈಗೂಡಿಸಿಕೊಂಡು, ತನ್ನ ವೈಯಕ್ತಿಕತೆಯನ್ನು ಕಳೆದುಕೊಳ್ಳದೆ, ತನ್ನದೇ ಆದ ಬೆಳವಣಿಗೆಯ ನಿಯಮಾನುಸಾರವಾಗಿ ಬೆಳೆಯಬೇಕು....

“ಈ ಸಮ್ಮೇಳನವು ಜಗತ್ತಿಗೇನಾದರೂ ತೋರಿಸಿಕೊಟ್ಟಿದ್ದರೆ ಅದು ಇದನ್ನು: ಪಾವಿತ್ರ್ಯ, ಪರಿಶುದ್ಧತೆ ಮತ್ತು ಅನುಕಂಪೆ–ಇವುಗಳೆಲ್ಲ ಪ್ರಪಂಚದ ಯಾವುದೇ ಒಂದು ಮತಕ್ಕೆ ಮಾತ್ರವೇ ಮೀಸಲಾಗಿಲ್ಲ; ಮತ್ತು ಜಗತ್ತಿನ ಪ್ರತಿಯೊಂದು ಮತವೂ ಅತ್ಯುನ್ನತ ಶೀಲಸಂಪನ್ನರಾದ ಸ್ತ್ರೀಪುರುಷರನ್ನು ನಿರ್ಮಾಣ ಮಾಡಿದೆ. ಇಂತಹ ಸ್ಪಷ್ಟ ನಿದರ್ಶನಗಳಿದ್ದರೂ ಕೂಡ ಯಾರಾ ದರೂ, ತಮ್ಮ ಧರ್ಮ ಮಾತ್ರವೇ ಉಳಿಯುತ್ತದೆ, ಉಳಿದುದೆಲ್ಲ ಅಳಿಯುತ್ತವೆ ಎಂದು ತಿಳಿಯು ವುದಾದರೆ, ನನಗೆ ಅವರ ಬಗ್ಗೆ ಕನಿಕರವಾಗುತ್ತದೆ. ಮತ್ತು ಯಾರೆಷ್ಟೇ ವಿರೋಧಿಸಿದರೂ ಬಹು ಶೀಘ್ರದಲ್ಲೇ ಪ್ರತಿಯೊಂದು ಧರ್ಮದ ಧ್ವಜದ ಮೇಲೂ ‘ಯುದ್ಧವಲ್ಲ, ಸಹಕಾರ’ ‘ನಾಶವಲ್ಲ, ಸ್ವೀಕಾರ’ ‘ವೈಮನಸ್ಯವಲ್ಲ, ಶಾಂತಿ-ಸಮನ್ವಯ’ ಎಂದು ಬರೆಯಲಾಗುತ್ತದೆ.”

ಹೀಗೆ, ಕೆಲವೇ ತಿಂಗಳ ಹಿಂದೆ ಅಜ್ಞಾತ-ಅನಾಮಧೇಯನಾಗಿದ್ದ ಒಬ್ಬ ಪರಿವ್ರಾಜಕ ಸಂನ್ಯಾಸಿ, ನವಯುಗದ ಪ್ರವರ್ತನಾಚಾರ್ಯನಾಗಿ ಪ್ರಕಟಗೊಂಡಿದ್ದ! ಎಲ್ಲೆಲ್ಲೂ ಸ್ವಾಮಿ ವಿವೇಕಾನಂದರ ಹೆಸರು ಪ್ರತಿಧ್ವನಿಗೊಳ್ಳುವಂತಾಗಿತ್ತು. ಶಿಕಾಗೋ ನಗರದ ಬೀದಿಗಳಲ್ಲಿ ಆಳಡಿ ಎತ್ತರದ ಅವರ ಭಾವಚಿತ್ರವನ್ನು ನಿಲ್ಲಿಸಿ, ಅದರ ಕೆಳಗೆ \eng{“The Monk Vivekananda” (}ಸಂನ್ಯಾಸಿ ವಿವೇಕಾನಂದ) ಎಂದು ಬರೆಯಲಾಗಿದ್ದಿತು. ದಾರಿಯಲ್ಲಿ ಓಡಾಡುವವರೆಲ್ಲ ಅವರ ಮುಂದೆ ನಿಂತು ಗೌರವ ದಿಂದ ತಲೆಬಾಗಿ ಮುನ್ನಡೆಯುತ್ತಿದ್ದರು. (ಇಂತಹ ಭಾವಚಿತ್ರಗಳು ಅವರ ಕೆಲವು ಅಮೆರಿಕದ ಶಿಷ್ಯರ ಮನೆಗಳಲ್ಲೂ ಇದ್ದುವೆಂಬುದು ತಿಳಿದುಬಂದಿದೆ.) ಅಮೆರಿಕದ ಚರಿತ್ರೆಯಲ್ಲೇ ಯಾವೊಬ್ಬ ವಿದೇಶಿಯನಿಗೂ ಇಂತಹ ಗೌರವ ಸಂದ ವರದಿಯಿಲ್ಲ. ಪತ್ರಕರ್ತರಂತೂ ಯಾವಾಗಲೂ ಅವರ ಸುತ್ತ ದುಂಬಾಲು ಬೀಳುತ್ತಿದ್ದರು. ಅಲ್ಲಿನ ಅತ್ಯಂತ ಪ್ರಸಿದ್ಧವೂ ಸಂಪ್ರದಾಯನಿಷ್ಠವೂ ಆದ ಒಂದು ಪತ್ರಿಕೆ ಅವರನ್ನು ‘ಒಬ್ಬ ಪ್ರವಾದಿ ಹಾಗೂ ಪುಷಿ’ ಎಂದು ಸಾರಿತು. \eng{New York Herald} ಬರೆಯಿತು: “ನಿಸ್ಸಂಶಯವಾಗಿಯೂ ಅವರು ಸಮ್ಮೇಳನದ ಅತಿಶ್ರೇಷ್ಠ ವ್ಯಕ್ತಿ. ಅವರ ಮಾತುಗಳನ್ನು ಕೇಳಿದ ಮೇಲೆ ‘ಇಂತಹ ತಿಳುವಳಿಕಸ್ಥ ದೇಶಕ್ಕೆ ಧರ್ಮ ಪ್ರಚಾರಕರನ್ನು ಕಳಿಸುವುದು ಎಂತಹ ಮೂರ್ಖತನ!’ಎನ್ನಿಸುತ್ತದೆ.”

ವೈಯಕ್ತಿಕವಾಗಿ ಅವರ ಬಗ್ಗೆ ಎಷ್ಟೋ ಜನ ತಮ್ಮ ಅಪಾರ ಮೆಚ್ಚಿಗೆಯನ್ನು ವ್ಯಕ್ತಪಡಿಸಿದರು. ವೈಜ್ಞಾನಿಕ ವಿಭಾಗದ ಅಧ್ಯಕ್ಷ ಆನರಬಲ್ ಮರ್ವಿನ್ ಮಾರಿ ಸ್ನೆಲ್ ಬರೆದರು: “ಸರ್ವಧರ್ಮ ಸಮ್ಮೇಳನದ ಸಭೆಯ ಮೇಲೂ ಅಮೆರಿಕದ ಜನವರ್ಗದ ಮೇಲೂ ಹಿಂದೂಧರ್ಮವು ಉಂಟು ಮಾಡಿದಷ್ಟು ತೀವ್ರ ಪರಿಣಾಮವನ್ನು ಬೇರಾವ ಧರ್ಮವೂ ಮಾಡಲಿಲ್ಲ. ಅದರಲ್ಲೂ ಹಿಂದೂ ಧರ್ಮದ ಅತಿಮುಖ್ಯ ಪ್ರತಿನಿಧಿಯಾದ ಸ್ವಾಮಿ ವಿವೇಕಾನಂದರು ನಿಸ್ಸಂಶಯವಾಗಿ ಸಮ್ಮೇಳನ ದಲ್ಲೇ ಅತ್ಯಂತ ಜನಪ್ರಿಯ ಹಾಗೂ ಪರಿಣಾಮಕಾರಿ ವ್ಯಕ್ತಿಯಾಗಿದ್ದರು. ಅವರು ಪ್ರತಿಸಲ ಮಾತ ನಾಡಿದಾಗಲೂ ಸಭಿಕರು ಅವರನ್ನು ಇತರೆಲ್ಲ ಭಾಷಣಕಾರರಿಗಿಂತ ಹೆಚ್ಚಿನ ಉತ್ಸಾಹದಿಂದ ಸ್ವಾಗತಿಸಿದರು. ಅವರು ಎಲ್ಲಿ ಹೋದರಲ್ಲಿ ಅವರನ್ನು ಜನರು ಮುತ್ತಿಕೊಳ್ಳುತ್ತಿದ್ದರು; ಅವರು ಉಚ್ಚರಿಸುವ ಪ್ರತಿಯೊಂದು ಶಬ್ದವನ್ನೂ ಕಾತರದಿಂದ ನಿರೀಕ್ಷಿಸುತ್ತಿದ್ದರು. ಅತ್ಯಂತ ಸಂಪ್ರದಾ ಯಸ್ಥ ಕಟ್ಟಾ ಕ್ರೈಸ್ತರೂ ಸ್ವಾಮಿ ವಿವೇಕಾನಂದರನ್ನು ಕುರಿತು ಹೇಳುತ್ತಾರೆ, ‘ಮನುಷ್ಯರ ಮಧ್ಯೆ ಅವರೊಬ್ಬ ರಾಜನೇ ಸರಿ’ ಎಂದು.”

ಸಮ್ಮೇಳನದ ಅನೇಕ ಪ್ರೇಕ್ಷಕರಿಗೆ ಸ್ವಾಮೀಜಿಯವರು ಕ್ರೈಸ್ತಧರ್ಮದಲ್ಲಿ ಹಳೆಯ ಕೊಳೆ ಯಾಗಿ ಪರಿಣಮಿಸಿದ್ದ ಸಂಪ್ರದಾಯಸ್ಥ ಆಲೋಚನೆಯನ್ನು ಚುಚ್ಚಿ, ಕ್ರೈಸ್ತರು ಎಚ್ಚತ್ತುಕೊಳ್ಳು ವಂತೆ ಮಾಡಿದ ಮಹಾನುಭಾವನಂತೆ ಕಂಡರು. ಸರ್ ಹಿರಮ್ ಮ್ಯಾಕ್ಸಿಮ್ ಇಂಥವರಲ್ಲೊಬ್ಬ. ಇವನು ಆ ಕಾಲದ ಒಬ್ಬ ಪ್ರಚಂಡ ಎಂಜಿನಿಯರ್, ಅನ್ವೇಷಕ. ಇಪ್ಪತ್ತು ವರ್ಷಗಳ ಬಳಿಕ ಈತ ತಾನು ಬರೆದ \eng{\textit{Li Hung Chang’s Scrap-Book}} ಎಂಬ ಪುಸ್ತಕದ ಮುನ್ನುಡಿಯಲ್ಲಿ ಸಮ್ಮೇಳನದ ಹಾಗೂ ಅದರ ಪರಿಣಾಮದ ಬಗ್ಗೆ ಹೀಗೆ ಬರೆಯುತ್ತಾನೆ:

“ಕೆಲವರ್ಷಗಳ ಹಿಂದೆ ಶಿಕಾಗೋದಲ್ಲಿ ಒಂದು ಸರ್ವಧರ್ಮ ಸಮ್ಮೇಳನ ನಡೆಯಿತು. ಇಂಥದೊಂದು ನಡೆಯಲು ಸಾಧ್ಯವೇ ಇಲ್ಲವೆಂದು ಕೆಲವರು ಹೇಳಿದ್ದರು. ಪ್ರತಿಯೊಂದು ಮತವೂ ಸಂಪೂರ್ಣ ಸತ್ಯವಾಗಿದ್ದು ಇತರ ಎಲ್ಲವೂ ಸಂಪೂರ್ಣ ತಪ್ಪಾಗಿರುವಾಗ ಯಾವುದೇ ಬಗೆಯ ಒಪ್ಪಂದಕ್ಕೆ ಬರಲು ಹೇಗೆ ತಾನೆ ಸಾಧ್ಯ? ಆದರೂ ಸಮ್ಮೇಳನ ನಡೆಯಿತು. ಮತ್ತು ಅಮೆರಿಕದ ಜನತೆಗೆ ಪ್ರತಿ ವರ್ಷವೂ ಹತ್ತು ಲಕ್ಷ ಡಾಲರ್​ಗೂ ಹೆಚ್ಚು ಹಣವನ್ನು ಮಿಗಿಸಿ ಕೊಟ್ಟಿತು. ಇನ್ನು ವಿದೇಶಗಳಲ್ಲಿ ಅದು ಉಳಿಸಿದ ಎಷ್ಟೋ ಜೀವಗಳ ಬಗ್ಗೆಯಂತೂ ಹೇಳಬೇಕಾ ಗಿಯೇ ಇಲ್ಲ! ಇವೆಲ್ಲವೂ ಸಾಧ್ಯವಾದದ್ದು ಒಬ್ಬ ಧೀರ, ಪ್ರಾಮಾಣಿಕ ಮನುಷ್ಯನಿಂದ. ಅವನೇ ಕಲ್ಕತ್ತದಿಂದ ಬಂದಿದ್ದ ಸಂನ್ಯಾಸಿ, ವಿವೇಕಾನಂದ. ಈತನದು ಅತ್ಯಂತ ಪ್ರಭಾವಶಾಲಿ ವ್ಯಕ್ತಿತ್ವ. ಅಪರಿಮಿತ ಜ್ಞಾನ. ವೆಬ್​ಸ್ಟರ್​ನ ಪದವೀಧರನಂತೆ ಇಂಗ್ಲಿಷ್ ಮಾತನಾಡಬಲ್ಲವನು. ಸಮ್ಮೇಳನ ದಲ್ಲಿ ಇತರೆಲ್ಲ ಪ್ರತಿನಿಧಿಗಳಿಗಿಂತಲೂ ಅತಿಹೆಚ್ಚು ಸಂಖ್ಯೆಯಲ್ಲಿದ್ದ ಅಮೆರಿಕನ್ ಪ್ರಾಟೆಸ್ಟೆಂಟರು ತಮ್ಮ ಕೆಲಸ ಬಹಳ ಸುಲಭವಾಗಿರುತ್ತದೆಂದು ಮೊದಲು ಭಾವಿಸಿದ್ದರು. ‘ನಿನ್ನನ್ನು ಹೇಗೆ ನಿರ್ನಾಮ ಮಾಡುತ್ತೇನೆ ನೋಡುತ್ತಿರು!’ ಎಂಬ ಧೋರಣೆಯಿಂದ ಕಾರ್ಯವನ್ನು ಪ್ರಾರಂಭಿಸಿ ದರು. ಆದರೆ ಅವರ ಬಳಿಯಿದ್ದ ಸರಕೆಂದರೆ ಅದೇ ಚರ್ವಿತಚರ್ವಣವಾದ ಗೊಡ್ಡು ಕಥೆಗಳು....

“ಆದರೆ ಯಾವಾಗ ವಿವೇಕಾನಂದರು ಮಾತನಾಡಿದರೋ, ಆಗ ಅವರಿಗೆ ಗೊತ್ತಾಯಿತು– ತಾವೊಬ್ಬ ನೆಪೋಲಿಯನ್​ನನ್ನು ಎದುರಿಸಬೇಕಾಗಿದೆ, ಎಂದು. ವಿವೇಕಾನಂದರ ಪ್ರಥಮ ಭಾಷಣವೇ ಹೊಸ ದಿಗಂತವೊಂದನ್ನು ತೋರಿಸಿಕೊಟ್ಟಿತು. ಅವರಾಡಿದ ಪ್ರತಿಯೊಂದು ಪದ ವನ್ನೂ ವರದಿಗಾರರು ಎಚ್ಚರಿಕೆಯಿಂದ ಆಲಿಸಿ, ದೇಶದ ಮೂಲೆಮೂಲೆಯ ಸಾವಿರಾರು ಪತ್ರಿಕೆಗಳಿಗೆ ತಂತಿ ಮೂಲಕ ಕಳಿಸಿದರು. ವಿವೇಕಾನಂದರು ಅಂದಿನ ಸಿಂಹವಾದರು. ಶ್ರೀಘ್ರ ದಲ್ಲೇ ಅಸಂಖ್ಯಾತ ಜನ ಅವರ ಹಿಂಬಾಲಕರಾದರು. ಅವರ ಭಾಷಣವನ್ನು ಕೇಳಲು ಬಂದ ಜನರಿಗೆ ಯಾವ ಸಭಾಂಗಣವೂ ಸಾಲುತ್ತಿರಲಿಲ್ಲ. ಅಮೆರಿಕದ ಕ್ರೈಸ್ತರು, ಏಷಿಯಾದ ಬಡ ‘ಅನಾಗರಿಕ’ನ ಆತ್ಮವನ್ನು ರಕ್ಷಿಸಲು ಲಕ್ಷಾಂತರ ಡಾಲರುಗಳನ್ನೂ ಅರೆಶಿಕ್ಷಿತ ಮೂರ್ಖರನ್ನೂ (ಧರ್ಮಪ್ರಚಾರಕರು) ಕಳಿಸಿಕೊಡುತ್ತಿದ್ದರು. ಇಂತಹ ‘ರಕ್ಷಣೆ’ಯಿಂದ ತಪ್ಪಿಸಿಕೊಂಡು ಉಳಿದಿ ದ್ದವರಲ್ಲೊಬ್ಬನು (ಸ್ವಾಮಿ ವಿವೇಕಾನಂದರು) ಇಲ್ಲಿದ್ದ. ಅಮೆರಿಕದ ಎಲ್ಲ ಪಾದ್ರಿಗಳೂ ಪ್ರಚಾರಕರೂ ಒಟ್ಟಾರೆ ತಿಳಿದಿದ್ದುದಕ್ಕಿಂತ ಹೆಚ್ಚಿನ ಧರ್ಮವನ್ನೂ ತತ್ತ್ವಜ್ಞಾನವನ್ನೂ ಈತ ಬಲ್ಲವನಾಗಿದ್ದ. ಜನರಿಗೆ ನಿಜಕ್ಕೂ ಒಪ್ಪಿಗೆಯಾಗುವಂತಹ ರೀತಿಯಲ್ಲಿ ಪ್ರಥಮ ಬಾರಿಗೆ ಈತ ಧರ್ಮವನ್ನು ಬೋಧಿಸಿದ. ಜನರು ಎಂದೂ ಊಹಿಸಿರದಿದ್ದಷ್ಟು ಸತ್ತ್ವ ಅದರಲ್ಲಿತ್ತು. ಆ ಬಗ್ಗೆ ವಾದಕ್ಕೆ ಎಡೆಯೇ ಇರಲಿಲ್ಲ. ಬೆಕ್ಕು ಇಲಿಯೊಂದಿಗೆ ಆಟವಾಡುವಂತೆ ಈತ ಕ್ರೈಸ್ತ ಮಿಷನರಿ ಗಳೊಂದಿಗೆ ಆಡಿದ. ಅವರು ದಿಗ್ಭ್ರಾಂತರಾದರು. ಅವರಿಗೇನು ಮಾಡಲು ಸಾಧ್ಯವಿತ್ತು? ಅವರೇನು ಮಾಡಿದರು? ಅಂಥವರು ಯಾವಾಗಲೂ ಮಾಡುವುದನ್ನೇ ಮಾಡಿದರು–ಈತನನ್ನು ದೆವ್ವದ ಪ್ರತಿನಿಧಿಯೆಂದು ದೂಷಿಸಿದರು. ಆದರೆ ಆಗಬೇಕಾದದ್ದು ಆಗಿಹೋಗಿತ್ತು. ಬೀಜವನ್ನು ಬಿತ್ತಿಯಾಗಿತ್ತು. ಅಮೆರಿಕನ್ನರು ಚಿಂತಿಸಲಾರಂಭಿಸಿದ್ದರು. ಅವರು ತಮ್ಮಷ್ಟಕ್ಕೆ ಹೇಳಿಕೊಂಡರು: ‘ಇವನಂತಹ ಜನರಿಗೆ ಬೋಧಿಸುವುದಕ್ಕಾಗಿ, ಇವನಿಗೆ ಹೋಲಿಸಿದರೆ ಧರ್ಮವನ್ನು ಏನೇನೂ ಅರಿಯದ ನಮ್ಮ ಮಿಷನರಿಗಳನ್ನು ಕಳಿಸಿ, ಹಣವನ್ನು ಪೋಲು ಮಾಡೋಣವೆ? ಎಂದಿಗೂ ಇಲ್ಲ!’ ತತ್ಪರಿಣಾಮವಾಗಿ, ವಿಷನರಿಗಳ ಆದಾಯ ವರ್ಷಕ್ಕೆ ಹತ್ತು ಲಕ್ಷಕ್ಕೂ ಹೆಚ್ಚು ಡಾಲರ್ ಗಳಷ್ಟು ಕಡಿಮೆಯಾಯಿತು....”

ಆದರೆ ಸ್ವಾಮೀಜಿ ಬಯಸಿದ್ದುದು ಇದನ್ನಲ್ಲ. ಅವರು ಅಮೆರಿಕೆಗೆ ಬಂದದ್ದು ಕ್ರೈಸ್ತ ಮಿಷನರಿಗಳೊಂದಿಗೆ ಜಗಳವಾಡಲೆಂದಲ್ಲ. ಭಾರತದ ವಿರುದ್ಧವಾಗಿ ನಡೆಸುತ್ತಿದ್ದ ಅಪಪ್ರಚಾರ ವನ್ನು ನಿಲ್ಲಿಸುವಂತೆ ಅವರು ಮಿಷನರಿಗಳಿಗೆ ಹೇಳಿದರೇ ಹೊರತು ಭಾರತವನ್ನೇ ಬಿಟ್ಟು ಹೋಗುವಂತೆ ಹೇಳಲಿಲ್ಲ. ಬದಲಾಗಿ, ಜನರನ್ನು ಮತಾಂತರಿಸುವ ಪ್ರಯತ್ನವನ್ನು ಬಿಟ್ಟು, ಅವರಿಗೆ ಸ್ವಾವಲಂಬಿಗಳಾಗಿ ಜೀವಿಸಲು ನೆರವಾಗಿ ಎಂದು ಅಮೆರಿಕದ ಜನರನ್ನೂ ವಿಷನರಿ ಗಳನ್ನೂ ಕೇಳಿಕೊಂಡರು. ಆದರೂ ಅವರು ವಿವಾದದ ಸುಳಿಯಲ್ಲಿ ಸಿಕ್ಕಿಕೊಳ್ಳುವುದು ಅನಿವಾರ್ಯ ವಾಗಿತ್ತು. ಅದರೊಂದಿಗೆ, ಅಪಾರ ಹೆಸರು-ಕೀರ್ತಿಯ ಕಿರೀಟದ ಭಾರವನ್ನೂ ತಡೆದುಕೊಳ್ಳಬೇಕಾ ಯಿತು. ಒಮ್ಮೆ ಕಾರ್ಯರಂಗಕ್ಕಿಳಿದ ಮೇಲೆ ಅವರೆಡೂ ಹಿಂಬಾಲಿಸಿಕೊಂಡು ಬರುವಂಥದೇ!

ಇತ್ತ ಸ್ವಾಮೀಜಿ ಸರ್ವಧರ್ಮ ಸಮ್ಮೇಳನದಲ್ಲಿ ಅದ್ಭುತ ಯಶಸ್ಸನ್ನು ಗಳಿಸಿ, ಜನಪ್ರಿಯತೆಯ ಶಿಖರವನ್ನೇರಿದಾಗ, ಅತ್ತ ಭಾರತದಲ್ಲಿ ಈ ಸಂತೋಷ ಸುದ್ದಿಯಿನ್ನೂ ಪ್ರಸಾರವಾಗಿರಲಿಲ್ಲ. ಸೆಪ್ಟೆಂಬರ್ ತಿಂಗಳಲ್ಲೇ ಸಮ್ಮೇಳನದ ಬಗ್ಗೆಯೂ ಸ್ವಾಮೀಜಿಯ ಬಗ್ಗೆಯೂ ಭಾರತೀಯ ವೃತ್ತಪತ್ರಿಕೆಗಳಲ್ಲಿ ಸಣ್ಣಪುಟ್ಟ ವರದಿಗಳು ಬರುತ್ತಿದ್ದುವಾದರೂ ಇವು ಓದುಗರ ಗಮನವನ್ನು ಸೆಳೆದಿರಲಿಲ್ಲ. ಆದರೆ ನವೆಂಬರಿನಲ್ಲಿ ಮುಂಬಯಿ, ಕಲ್ಕತ್ತಾ ಹಾಗೂ ಮದ್ರಾಸಿನ ಪ್ರಮುಖ ದಿನಪತ್ರಿಕೆಗಳಲ್ಲಿ \eng{“Hindus at the Fair” (}ಸಮ್ಮೇಳನದಲ್ಲಿ ಹಿಂದುಗಳು) ಎಂಬ ದೀರ್ಘ ಲೇಖನವೊಂದು ಪ್ರಕಟವಾದಾಗ ಬಹಳಷ್ಟು ಜನರು ಇದರಿಂದ ಕುತೂಹಲಿತರಾದರು. \eng{\textit{Boston Evening Transcript}} ಪತ್ರಿಕೆಯಲ್ಲಿ ಸೆಪ್ಟೆಂಬರ್ ೨೩ರಂದು ಮೊದಲಬಾರಿಗೆ ಪ್ರಕಟವಾಗಿದ್ದ ಈ ಲೇಖನವು ನವೆಂಬರ್ ೧೧ರಂದು ಕಲ್ಕತ್ತದ \eng{\textit{Indian Mirror}} ಪತ್ರಿಕೆಯಲ್ಲಿ ಪುನರ್ಮುದ್ರಿತ ವಾಯಿತು. ಈ ಲೇಖನದಲ್ಲಿ ವರದಿಗಾರನೊಬ್ಬ ಸ್ವಾಮೀಜಿಯೊಂದಿಗೆ ನಡೆಸಿದ್ದ ಒಂದು ಅನೌಪಚಾರಿಕ ಸಂದರ್ಶನದ ವಿವರಗಳಿದ್ದುವು. ಅಲ್ಲದೆ, ಸ್ವಾಮೀಜಿಯ ಆದರ್ಶಗಳ ಘನತೆ ಯನ್ನು ಹಾಗೂ ಸಹಸ್ರಾರು ಜನ ತುಂಬಿದ್ದ ಸಭೆಯಲ್ಲಿ ವಿದ್ಯುತ್ ಸಂಚಾರವಾಗುವಂತೆ ಮಾಡಿದ ಅವರ ಅದ್ಭುತ ವಾಕ್​ಸಾಮರ್ಥ್ಯವನ್ನು ಹೃದಯಂಗಮವಾಗಿ ಬಣ್ಣಿಸಲಾಗಿದ್ದಿತು. ದೂರದ ಅಮೆರಿಕದಲ್ಲಿ ಏನೋ ಅದ್ಭುತವೊಂದು ನಡೆಯುತ್ತಿದೆಯೆಂಬ ಸುಳಿವು ಭಾರತೀಯರಿಗೆ ಸಿಕ್ಕಿದ್ದು ಆಗಲೇ. ಆ ಲೇಖನದ ಕೆಲವು ಅಂಶಗಳು ಹೀಗಿದ್ದುವು:

“ಸಮ್ಮೇಳನದ ಭವನದಲ್ಲಿ ಪ್ರತಿನಿಧಿಗಳು ವಿಶ್ರಮಿಸುವ ಕೋಣೆಯೊಂದಿದೆ. ಹಲವಾರು ದೇಶಗಳ ವೈವಿಧ್ಯಮಯ ವ್ಯಕ್ತಿಗಳನ್ನು ಇಲ್ಲಿ ಕಾಣಬಹುದು. ಈ ಕೋಣೆಯೊಳಕ್ಕೆ ಪ್ರವೇಶ ದೊರಕಿಸಿಕೊಂಡು ಹೋದರೆ, ಅಲ್ಲಿಎಲ್ಲರಿಗಿಂತ ಮಿಗಿಲಾಗಿ ಎದ್ದುಕಾಣುವ ವ್ಯಕ್ತಿಯೆಂದರೆ ಬ್ರಾಹ್ಮಣ ಸಂನ್ಯಾಸಿ\footnote{* ನೋಡಿ ಅನುಬಂಧ ೪.} ಯಾದ ಸ್ವಾಮಿ ವಿವೇಕಾನಂದರು....

“ಅವರದು ಕಟ್ಟುಮಸ್ತಾದ ಹಿಂದೂಸ್ತಾನಿ ಶರೀರರಚನೆ; ನುಣ್ಣಗೆ ಕ್ಷೌರಮಾಡಿಕೊಂಡ ದುಂಡನೆಯ ಮುಖ, ಅಚ್ಚುಕಟ್ಟಾದ ಅಂಗಾಂಗಗಳು, ಧವಳ ದಂತಪಂಕ್ತಿ; ಸಂಭಾಷಿಸುವಾಗ ಮನೋಹರವಾದ ಮುಗುಳ್ನಗೆಯೊಂದಿಗೆ ಅರೆಬಿರಿಯುವ, ಸುಂದರವಾದ ಕಟೆಯಲ್ಪಟ್ಟ ತುಟಿ ಗಳು; ಅವರ ಸ್ಥಿರವಾದ ಶಿರದ ಮೇಲೆ ಮುಕುಟವಿಟ್ಟಂತಿರುವ, ನಿಂಬೆ ಬಣ್ಣದ ಅಥವಾ ಕೆಂಪು ಬಣ್ಣದ ಪೇಟ; ಕಿತ್ತಳೆ ಹಾಗೂ ಕಡುಗೆಂಪಿನ ಮಿಶ್ರವರ್ಣದ, ಮಂಡಿಯುದ್ದದ ನೀಳ ನಿಲುವಂಗಿ; ಅದಕ್ಕೊಪ್ಪುವ ಸೊಂಟಪಟ್ಟಿ. ಅವರಾಡುವ ಇಂಗ್ಲಿಷ್ ಅತ್ಯುತ್ತಮವಾದದ್ದು. ಪ್ರಾಮಾಣಿಕವಾಗಿ ಕೇಳಿದ ಯಾವುದೇ ಪ್ರಶ್ನೆಗೆ ತಕ್ಷಣವೇ ಉತ್ತರಸಿದ್ಧ.

“ಸಮ್ಮೇಳನದಲ್ಲಿ ಅವರು ಮಾಡಿದ ಭಾಷಣವು ಗಗನದಷ್ಟು ವಿಶಾಲವಾಗಿದ್ದು, ವಿಶ್ವಧರ್ಮದ ಆದರ್ಶಕ್ಕೆ ತಕ್ಕಂತೆ ಜಗತ್ತಿನ ಸರ್ವಧರ್ಮಗಳ ಅತ್ಯುನ್ನತ ಅಂಶಗಳನ್ನು ಒಳಗೊಂಡಿತ್ತು. ಶಿಕ್ಷೆಯ ಭಯದಿಂದಾಗಲಿ ಪ್ರತಿಫಲದ ಆಕಾಂಕ್ಷೆಯಿಂದಾಗಲಿ ಒಡಮೂಡಿರದ ಕೇವಲ ಭಗವತ್ ಪ್ರೀತ್ಯರ್ಥವಾದ ನಿಷ್ಕಾಮಕರ್ಮವನ್ನೂ ಸಕಲ ಮಾನವರೆಡೆಗೆ ಸಹೃದಯತೆಯನ್ನೂ ಅದು ಪ್ರತಿಪಾದಿಸಿತು.

“ವಿವೇಕಾನಂದರು ತಮ್ಮ ಮಾತು-ಭಾವನೆಗಳ ವೈಭವದಿಂದಲೂ ಅದ್ಭುತ ಆಕಾರದಿಂದಲೂ ಸಮ್ಮೇಳನದ ಕಣ್ಮಣಿಯಾಗಿದ್ದಾರೆ. ಅವರು ವೇದಿಕೆಯ ಮೇಲೆ ಸುಮ್ಮನೆ ಅತ್ತಿಂದಿತ್ತ ಹೋದರೆ ಸಾಕು, ಕರತಾಡನ ಕೇಳಿಬರುತ್ತದೆ. ಆದರೆ ಅವರು ಸಾವಿರಾರು ಜನರ ಈ ಬಗೆಯ ಮೆಚ್ಚಿಗೆಯನ್ನು, ಅಹಂಕಾರದ ಲೇಶವೂ ಇಲ್ಲದೆ ಶಿಶು ಸಹಜ ಸಂತೃಪ್ತಿಯಿಂದ ಸ್ವೀಕರಿಸುತ್ತಾರೆ. ಬಡತನ ಹಾಗೂ ಅನಾಮಧೇಯ ಸ್ಥಿತಿಯಿಂದ ಒಮ್ಮೆಗೇ ಸಮೃದ್ಧಿ ಹಾಗೂ ಪ್ರಖ್ಯಾತಿಗೆ ಏರುವಂತಾದದ್ದು ಈ ವಿನಮ್ರ ಬ್ರಾಹ್ಮಣ ಸಂನ್ಯಾಸಿಗೆ ಬಹಳ ವಿಚಿತ್ರವಾದ ಅನುಭವವೇ ಆಗಿರಬೇಕು.”

ಇದಾದನಂತರ, ಶಿಕಾಗೋದಲ್ಲಿ ಸ್ವಾಮೀಜಿಯ ಅದ್ಭುತ ಯಶಸ್ಸಿನ ವರದಿಗಳು ಅಮೆರಿಕದ ಪತ್ರಿಕೆಗಳ ಮುಖಾಂತರ ಭಾರತಕ್ಕೆ ಹರಿದುಬಂದಂತೆ ಭಾರತದ ಪತ್ರಿಕೆಗಳು ಅವನ್ನು ಉತ್ಸಾಹ ದಿಂದ ಪುನರ್ಮುದ್ರಿಸಲಾರಂಭಿಸಿದುವು. ಅದರಲ್ಲೂ ಮುಖ್ಯವಾಗಿ ಕಲ್ಕತ್ತದ ‘ಇಂಡಿಯನ್ ಮಿರರ್​’ ವಿವರಪೂರ್ಣವಾದ ವರದಿಗಳೊಂದಿಗೆ, ಆ ಬಗ್ಗೆ ಸಂಪಾದಕೀಯಗಳನ್ನು ಬರೆದು ಪ್ರಶಂಸೆಯನ್ನು ವ್ಯಕ್ತಪಡಿಸುತ್ತಿತ್ತು. ಆದರೆ ಕೆಲವು ಪತ್ರಿಕೆಗಳು ಸ್ವಾಮೀಜಿ ಗಳಿಸಿದ್ದ ಜನಪ್ರಿಯತೆ ಯನ್ನೂ ಅವರ ಯಶಸ್ಸನ್ನೂ ಸಹಿಸಲಾರದೆ ಟೀಕಿಸಿ ಬರೆದದ್ದೂ ಉಂಟು. ಅಸೂಯೆ ಹಾಗೂ ಮತೀಯ ಪೂರ್ವಾಗ್ರಹವೇ ಈ ದ್ವೇಷಕ್ಕೆ ಕಾರಣ. ಆದರೆ ಈ ಪತ್ರಿಕೆಗಳು ಅವರ ಪರವಾಗಿಯೇ ಬರೆದಿರಲಿ ವಿರುದ್ಧವಾಗಿಯೇ ಬರೆದಿರಲಿ–‘ಯುವಸಂನ್ಯಾಸಿಯೊಬ್ಬ ಸಮುದ್ರವನ್ನು ದಾಟಿ ಅಮೆರಿಕೆಗೆ ಹೋದನಂತೆ, ಅಲ್ಲಿನ ಜನರೊಂದಿಗೆ ಬೆರೆತನಂತೆ, ಪ್ರಸಿದ್ಧವಾದ ಧರ್ಮ ಸಮ್ಮೇಳನ ದಲ್ಲಿ ಜಯಭೇರಿ ಬಾರಿಸಿದನಂತೆ’ ಎಂಬ ಕೌತುಕದ ವರ್ತಮಾನ ಭಾರತದಾದ್ಯಂತ ಹರಡಿತು. ಪ್ರತಿದಿನವೂ ಎಂಬಂತೆ, ಸಮ್ಮೇಳನದಲ್ಲಿ ಅವರು ಮಾಡಿದ ಭಾಷಣದ ವರದಿಗಳು ಭಾರತದ ಪತ್ರಿಕೆಗಳಲ್ಲಿ ಪ್ರಕಟವಾಗತೊಡಗಿದುವು. ಡಿಸೆಂಬರ್ ೬ರಂದು ‘ಇಂಡಿಯನ್ ಮಿರರ್​’ ಬರೆದ ಸಂಪಾದಕೀಯದ ಪ್ರಸ್ತುತ ಭಾಗ ಹೀಗಿದೆ:

“... ಆದರೆ ಎಲ್ಲ ಸಭಿಕರ ಕಣ್ಣುಗಳೂ ನೆಟ್ಟಿದ್ದ ಒಬ್ಬ ವ್ಯಕ್ತಿ–ಒಬ್ಬ ಭಾರತೀಯ ಪ್ರತಿನಿಧಿ–ಎಂದರೆ ಸ್ವಾಮಿ ವಿವೇಕಾನಂದರು. ಅವರದು ಸಂನ್ಯಾಸಿಯ ಉಡುಗೆ, ಸುಂದರಾಕಾರ, ಭವ್ಯ ನಿಲುವು, ವಜ್ರದಂತೆ ಮಿನುಗುವ ಕಣ್ಣುಗಳು. ಇದು ಅವರ ಆಕರ್ಷಕ ಮೇಲ್ನೋಟದ ದೃಶ್ಯ. ಆದರೆ ಅವರು ಮಾತನಾಡಿದಾಗ, ಅವರ ಅಂತರಾಳದ ವ್ಯಕ್ತಿ ಸಿಡಿದುನಿಂತಾಗ, ಆ ಶಕ್ತಿ ಇಮ್ಮಡಿಸಿತು. ವೇದಾಂತಧರ್ಮದ ಕುರಿತು ಅವರು ನೀಡಿದ ಉಜ್ವಲ ವ್ಯಾಖ್ಯಾನವನ್ನು ಆ ಬೃಹತ್ ಸಭೆ ಮಂತ್ರಮುಗ್ಧವಾಗಿ ಆಲಿಸಿತು. ಸ್ವಾಮಿ ವಿವೇಕಾನಂದರ ಮೇಲಿನ ಅಮೆರಿಕನ್ನರ ಉತ್ಸಾಹವನ್ನು ನಾವು ಸುಲಭವಾಗಿಯೇ ಅರ್ಥಮಾಡಿಕೊಳ್ಳಬಹುದು... ”

ಸಂನ್ಯಾಸಿಯೊಬ್ಬ ಸನಾತನ ಧರ್ಮದ ಕೀರ್ತಿಪತಾಕೆಯನ್ನು ದೂರದ ಅಮೆರಿಕದಲ್ಲಿ ಮೆರೆಸಿದ ಸುದ್ದಿ ಕೇಳಿ ಸಹಸ್ರಾರು ಭಾರತೀಯರು ಪುಳಕಿತರಾದರು. ಈ ಸಂನ್ಯಾಸಿ ಯಾರಿರಬಹುದು? ಎಂದು ಅಚ್ಚರಿಗೊಂಡರು. ಯಾರಿಗೇಕೆ–ಸ್ವತಃ ಅವರ ಗುರುಭಾಯಿಗಳಿಗೂ ಅವರು ಯಾರು ಎಂಬುದು ಗೊತ್ತಾಗಲಿಲ್ಲ! ‘ಸ್ವಾಮಿ ವಿವೇಕಾನಂದ’ ಎಂಬ ಹೆಸರು ಈ ಯುವಸಂನ್ಯಾಸಿಗಳಿಗೆ ಹೊಸದು. ಈ ಹಿಂದೆ ಅವರು ಈ ಹೆಸರನ್ನು ಕೇಳಿರಲಿಲ್ಲವೆಂದಲ್ಲ; ಆದರೆ ಸ್ವಾಮೀಜಿ ತಮ್ಮ ಹೆಸರನ್ನು ಎಷ್ಟೋ ಸಲ ಬದಲಾಯಿಸಿಕೊಂಡಿದ್ದರು. ಆದ್ದರಿಂದ ಆ ಸಂನ್ಯಾಸಿ ತಮ್ಮ ನರೇನ್ ಭಾಯಿಯೇ ಇರಬಹುದೇ... ಎಂಬ ಅನುಮಾನವುಂಟಾಯಿತಾದರೂ ಯಾವುದೂ ನಿರ್ದಿಷ್ಟ ವಾಗಲಿಲ್ಲ. ಅವನು ಯಾರಾದರೇನಂತೆ ಅಂತೂ ವೇದಧರ್ಮಕ್ಕೆ ಗೌರವ ಸಂದಿತಲ್ಲ! ಎಂದು ಸಂತೋಷ ತಾಳಿದರು. ಆದರೆ ನವೆಂಬರ್ ೧೫ರ ಹೊತ್ತಿಗೆ ಎಲ್ಲರಿಗೂ ನಿಜಸಂಗತಿ ತಿಳಿದು ಬರುವಂತಾಯಿತು. ‘ಇಂಡಿಯನ್ ಮಿರರ್​’ ಪತ್ರಿಕೆ ವಿವೇಕಾನಂದರ ಹಿನ್ನೆಲೆಯನ್ನು ಪತ್ತೆ ಮಾಡಿ, ‘ಅವರು ಕಲ್ಕತ್ತಾ ವಿಶ್ವವಿದ್ಯಾಲಯದ ಪದವೀಧರರು; ಹೆಸರು ನರೇಂದ್ರನಾಥದತ್ತ ಎಂದು; ಅವರು ದಕ್ಷಿಣೇಶ್ವರದ ಪೂಜ್ಯ ಪರಮಹಂಸ ರಾಮಕೃಷ್ಣರ ಶಿಷ್ಯರು’ ಎಂದು ಪ್ರಕಟಿಸಿತು. ‘ಸ್ವಾಮಿ ವಿವೇಕಾನಂದ’ರು ತಮ್ಮ ನೆಚ್ಚಿನ ಗುರುಭಾಯಿ ನರೇಂದ್ರನೇ ಎಂದು ಗೊತ್ತಾದಾಗ ಆಲಂಬಜಾರ್ ಮಠದ ಸಂನ್ಯಾಸಿಗಳು ಅತ್ಯಾನಂದದಿಂದ ಕುಣಿದಾಡಿದರು. ತಮ್ಮ ಗುರುದೇವರು ನುಡಿದಿದ್ದ ಭವಿಷ್ಯವಾಣಿ ಇಷ್ಟು ಬೇಗ ಸತ್ಯವಾದುದನ್ನು ಕಂಡು ವಿಸ್ಮಯಾನಂದ ಭರಿತರಾದರು.

ಆದರೆ ಸ್ವಾಮೀಜಿ ತಮ್ಮ ಯಶಸ್ಸಿನ ಸುದ್ದಿಯನ್ನು ತಾವಾಗಿಯೇ ತಮ್ಮ ಗುರುಭಾಯಿಗಳಿಗೆ ತಿಳಿಸಲಿಲ್ಲ. ಅವರ ಮನಸ್ಸಿನಲ್ಲೇನಿತ್ತೆಂದು ಊಹಿಸುವುದು ಕಷ್ಟ. ಬಹುಶಃ ತಮ್ಮ ಕಾರ್ಯ ಯೋಜನೆಯು ಇನ್ನಷ್ಟು ಸ್ಪಷ್ಟವಾಗಲಿ, ಇನ್ನಷ್ಟು ದೃಢವಾಗಿ ನೆಲೆಯೂರಿ ನಿಲ್ಲಲಿ ಎಂದು ಅವರು ಆಲೋಚಿಸಿರಬಹುದು. ಅಥವಾ ತಮ್ಮ ಯೋಜನೆಯನ್ನು ತಮ್ಮ ಗುರುಭಾಯಿಗಳಿಗೆ ತಿಳಿಸಿ, ಆ ಯೋಜನೆಯ ಪ್ರಕಾರವೇ ನಡೆಯುವಂತೆ ಅವರಿಗೆ ಹೇಳುವ ಮೊದಲು, ಮತ್ತೊಮ್ಮೆ ಶ್ರೀರಾಮ ಕೃಷ್ಣರ ಆದೇಶಕ್ಕಾಗಿ ಕಾಯುತ್ತಿದ್ದರೋ? ಕಾರಣವೇನೇ ಇರಲಿ, ಅವರು ಭಾರತದಲ್ಲಿನ ತಮ್ಮ ಯೋಜನೆಗಳನ್ನು ಕಾರ್ಯಗತಗೊಳಿಸಲು ತಮ್ಮ ಸೋದರ ಸಂನ್ಯಾಸಿಗಳಿಗೆ ಸೂಚನೆ ನೀಡಿದ್ದು ಬಹಳ ಮುಂದೆಯೇ.

ಅಮೆರಿಕದಲ್ಲಿ ಸ್ವಾಮೀಜಿ ಪಡೆದ ಗೌರವ-ಮನ್ನಣೆಗಳ ಬಗ್ಗೆ ತಿಳಿದು ಅವರ ಸ್ನೇಹಿತರು ಹಾಗೂ ಶಿಷ್ಯರು ತಮಗೇ ಆ ಗೌರವ ಸಂದಂತೆ ಆನಂದೋತ್ಸಾಹಭರಿತರಾದರು. ದೇಶದಾದ್ಯಂತ ಅನೇಕ ನಗರಗಳಲ್ಲಿ, ಮುಖ್ಯವಾಗಿ ಮದರಾಸು, ಬೆಂಗಳೂರುಗಳಲ್ಲಿ ಅವರ ಶಿಷ್ಯರು ವಿಜಯೋತ್ಸವಗಳನ್ನೂ ಆಚರಿಸಿದರು. ಅಲ್ಲದೆ ಖೇತ್ರಿಯ ಮಹಾರಾಜ ಅಜಿತ್​ಸಿಂಗ್, ಮುನ್ಷಿ ಜಗಮೋಹನಲಾಲ್, ಜುನಾಗಢದ ದಿವಾನ ಹರಿದಾಸ್ ದೇಸಾಯಿ ಮೊದಲಾದ ಪ್ರತಿಷ್ಠಿತರು ಸ್ವಾಮೀಜಿಗೆ ಅಭಿನಂದನಾ ಪತ್ರಗಳನ್ನು ಬರೆದರು. ಹರಿದಾಸ್ ದೇಸಾಯಿಗೆ ಸ್ವಾಮೀಜಿ ಒಂದು ಪತ್ರ ಬರೆದು, ಅದರೊಂದಿಗೆ ‘ನ್ಯೂಯಾರ್ಕ್ ಹೆರಾಲ್ಡ್​’ ಹಾಗೂ ‘ನ್ಯೂಯಾರ್ಕ್ ಕ್ರಿಟಿಕ್​’ ಪತ್ರಿಕೆಗಳು ತಮ್ಮ ಬಗ್ಗೆ ಬರೆದಿದ್ದ ಶ್ಲಾಘನೆಯ ಲೇಖನಗಳ ಒಂದೆರಡು ತುಣುಕುಗಳನ್ನು ಕಳಿಸಿಕೊಟ್ಟಿದ್ದರು. ಪತ್ರದಲ್ಲಿ ಅವರು ಬರೆದಿದ್ದರು, “ನಾನು ಇನ್ನೂ ಹೆಚ್ಚಾಗಿ ಈ ಬಗ್ಗೆ (ತಮ್ಮ ಯಶಸ್ಸಿನ ಬಗ್ಗೆ) ತಿಳಿಸಲು ಇಷ್ಟಪಡುವುದಿಲ್ಲ. ಏಕೆಂದರೆ ನೀವು ನನ್ನನ್ನು ಅಹಂಕಾರಿಯೆಂದು ತೀರ್ಮಾನಿಸುತ್ತೀರಿ. ಆದರೆ ನಾನು ಇದನ್ನಾದರೂ ಏಕೆ ತಿಳಿಸಿದೆನೆಂದರೆ, ನೀವು ಬಾವಿಯ ಕಪ್ಪೆ ಗಳಾಗಿದ್ದೀರಿ; ಸುತ್ತಲಿನ ಪ್ರಪಂಚದಲ್ಲಿ ಏನಾಗುತ್ತಿದೆಯೆಂಬ ಅರಿವೇ ನಿಮಗಿಲ್ಲ. ದಿವಾನ್​ಜಿ ಸಾಹೇಬ್, ನಾನು ಈ ಮಾತುಗಳನ್ನು ನಿಮ್ಮನ್ನು ಉದ್ದೇಶಿಸಿ ಬರೆಯುತ್ತಿಲ್ಲ. ಆದರೆ ನಮ್ಮ ಇಡೀ ರಾಷ್ಟ್ರವನ್ನು ಮನಸ್ಸಿನಲ್ಲಿಟ್ಟುಕೊಂಡು ಹೇಳುತ್ತಿದ್ದೇನೆ, ಅಷ್ಟೆ.” ಸ್ವಾಮೀಜಿ ಕಳಿಸಿಕೊಟ್ಟಿದ್ದ ಪತ್ರಿಕೆಗಳ ತುಣುಕುಗಳನ್ನು ಹರಿದಾಸ್ ದೇಸಾಯಿ, ಆಲಂಬಜಾರಿಗೆ ಕಳಿಸಿಕೊಟ್ಟರು. ಇದನ್ನೆಲ್ಲ ಓದಿ ಆ ಗುರುಭಾಯಿಗಳು ಮತ್ತಷ್ಟು ಹಿಗ್ಗಿದರು. ಬಳಿಕ ಸ್ವಾಮಿ ತ್ರಿಗುಣಾತೀತಾನಂದರು ‘ಇಂಡಿಯನ್ ಮಿರರ್​’ಗೆ ಸ್ವಾಮೀಜಿಯ ಬಗ್ಗೆ ಒಂದು ಪರಿಚಯ ಲೇಖನವನ್ನೂ ಅಮೆರಿಕದ ಪತ್ರಿಕೆಗಳ ತುಣುಕುಗಳನ್ನೂ ಕಳಿಸಿಕೊಟ್ಟರು. ಇವೆರಡನ್ನೂ ಆ ಪತ್ರಿಕೆ ಪ್ರಕಟಿಸಿತು.

ವಿವೇಕಾನಂದರು ಸರ್ವಧರ್ಮ ಸಮ್ಮೇಳನದಲ್ಲಿ ಬೀರಿದ ಪ್ರಭಾವವು ಕೇವಲ ಕರ್ಣಾಕರ್ಣಿತ ವಾಗಿ ಉಳಿಯದೆ, ಒಂದು ಐತಿಹಾಸಿಕ ದಾಖಲೆಯಾಗಿ ಮುದ್ರಿತವಾಯಿತು. ಸಮ್ಮೇಳನದ ಸಾಮಾನ್ಯ ಸಮಿತಿಯ ಚೇರ್​ಮನ್ನರಾದ ಡಾ. ಜಾನ್ ಹೆನ್ರಿ ಬರೋಸ್​ರವರು ಸಮ್ಮೇಳನದ ಬಗ್ಗೆ ವಿವರಪೂರ್ಣವಾದ ಅಧಿಕೃತ ವರದಿಯನ್ನು ಬರೆದರು. ಇದರಲ್ಲಿ ಅವರು ಸ್ವಾಮೀಜಿಯ ಭಾಷಣಗಳ ಕೆಲವು ಭಾಗಗಳನ್ನು ಉದಾಹರಿಸಿ, ಅವರ ವ್ಯಕ್ತಿತ್ವವನ್ನೂ ಭಾಷಣಗಳನ್ನೂ ಹೃತ್ಪೂರ್ ವಕವಾಗಿ ಶ್ಲಾಘಿಸಿದ್ದರು. ಡಾ. ಬರೋಸ್​ರಂತಹ ಗಣ್ಯ ವ್ಯಕ್ತಿ ಅಷ್ಟೊಂದು ಹೇಳಬೇಕೆಂದರೆ ಅದೊಂದು ಸಾಮಾನ್ಯ ಸಾಧನೆಯಲ್ಲ. ಮತ್ತು ಅದೊಂದು ಉತ್ಪ್ರೇಕ್ಷೆಯಾಗಿರಲೂ ಸಾಧ್ಯವಿಲ್ಲ. ಇದನ್ನು ಮನಗಂಡು, ಸ್ವಾಮೀಜಿಯ ಸಾಧನೆಯ ಬಗ್ಗೆ ಅಸಡ್ಡೆಯ ಮನೋಭಾವ ತಾಳಿದ್ದವರೂ ಈಗ ಎಚ್ಚರಗೊಂಡರು. ‘ಇಂಡಿಯನ್ ಮಿರರ್​’ನ ಸಂಪದಾಕೀಯ ಈ ಅಂಶಗಳನ್ನು ವಿಸ್ತಾರ ವಾಗಿ ಬಣ್ಣಿಸಿ, ಕಡೆಯಲ್ಲಿ ಬಹಳ ಸೊಗಸಾಗಿ ವಿಮರ್ಶಿಸಿತು:

“... ಅಮೆರಿಕೆಗೆ ಸ್ವಾಮಿ ವಿವೇಕಾನಂದರು ಕೈಗೊಂಡ ಪ್ರಚಾರಯಾತ್ರೆಯ ವ್ಯಾವಹಾರಿಕ ಪರಿಣಾಮಗಳೇನೇ ಇರಲಿ, ಅದು ನಿಜವಾದ ಹಿಂದೂಧರ್ಮದ ಘನತೆಯನ್ನು ಜಗತ್ತಿನ ಮುಂದೆ ಅತಿ ಎತ್ತರಕ್ಕೆತ್ತಿದೆ ಎಂಬುದರಲ್ಲಿ ಸಂಶಯವಿಲ್ಲ. ನಿಜಕ್ಕೂ ಈ ಒಂದು ಸಾಧನೆಗಾಗಿ ಸಮಸ್ತ ಹಿಂದೂ ಜನಾಂಗವೇ ಅವರಿಗೆ ಚಿರಪುಣಿಯಾಗಿರಬೇಕು.”

ಈ ಬಗೆಯ ಪತ್ರಿಕಾ ವರದಿಗಳು ಹಾಗೂ ಸಂಪಾದಕೀಯಗಳು ಒಟ್ಟಾರೆಯಾಗಿ ಹಿಂದೂ ಜನತೆಯಲ್ಲಿ ನವೋತ್ಸಾಹವನ್ನೂ ಕೃತಜ್ಞತಾಭಾವನ್ನೂ ಉಂಟುಮಾಡಿದುವು. ಆದರೆ ಇದರಿಂದಾಗಿ ಅಸೂಯಾಪರರ ದ್ವೇಷಾಗ್ನಿಗೆ ತುಪ್ಪ ಸುರಿದಂತಾಯಿತು. ಇವರಲ್ಲಿ ಮುಖ್ಯರಾದವರೆಂದರೆ ಕ್ರೈಸ್ತ ಪ್ರಚಾರ ಸಂಸ್ಥೆಗಳು, ಥಿಯಾಸೊಫಿಸ್ಟರು ಹಾಗೂ ಬ್ರಾಹ್ಮಸಮಾಜದ ಕೆಲವರು. ಆದರೆ ವಿವೇಕಾ ನಂದರನ್ನು ಬೆಂಬಲಿಸಿದವರೂ ಸಾಕಷ್ಟು ಜನರಿದ್ದರು. ಈಗಾಗಲೇ ಅವರ ಬಗ್ಗೆ ಗಾಢ ವಿಶ್ವಾಸ ತಾಳಿದ್ದ ವೈಜ್ಞಾನಿಕ ವಿಭಾಗದ ಅಧ್ಯಕ್ಷರಾದ ಮರ್ವಿನ್ ಮಾರಿ ಸ್ನೆಲ್​ರವರು ಈ ಅಪಪ್ರಚಾರವನ್ನು ಕಂಡು, ತಾವಾಗಿಯೇ ಅವರ ಸಮರ್ಥನೆಗೆ ನಿಂತರು. ಶ್ರೀ ಸ್ನೆಲ್​ರವರು ಅಲಹಾಬಾದಿನ ‘ಪಯೊನಿಯರ್​’ ಎಂಬ ಆಂಗ್ಲೊ ಇಂಡಿಯನ್ ಪತ್ರಿಕೆಗೆ ದೀರ್ಘಪತ್ರವೊಂದನ್ನು ಬರೆದು ಸ್ವಾಮೀಜಿಯ ಬಗ್ಗೆ ತಮ್ಮ ಹೃತ್ಪೂರ್ವಕ ಮೆಚ್ಚುಗೆಯನ್ನು ವ್ಯಕ್ತಪಡಿಸಿದರು. ಇದರ ಒಂದು ಭಾಗವನ್ನು ನಾವು ಈಗಾಗಲೇ ನೋಡಿದ್ದೇವೆ. ಅದರ ಇನ್ನು ಕೆಲವಂಶಗಳು ಹೀಗಿವೆ:

“ಅಮೆರಿಕದ ಪತ್ರಿಕೆಗಳಿಂದ ಹಾಗೂ ಜನರಿಂದ ಸರ್ವಧರ್ಮ ಸಮ್ಮೇಳನದಲ್ಲಿ ಸ್ವಾಮಿ ವಿವೇಕಾನಂದರು ಪಡೆದ ಒಕ್ಕೊರಲಿನ ಮೆಚ್ಚುಗೆಯಲ್ಲಿ ಒಂದೆರಡು ಒಡಕು ದನಿಗಳೂ ಕೇಳಿ ಬಂದಿವೆ. ಆದ್ದರಿಂದ ನಾನು ನಿಮ್ಮ ಜನರಿಗೆ ನಿಜಸ್ಥಿತಿಯನ್ನು ಅರಿವು ಮಾಡಿಕೊಡುವುದಕ್ಕಾಗಿ ನಮ್ಮ ಸುಸಂಸ್ಕೃತ ಹಾಗೂ ವಿಶಾಲ ಮನಸ್ಸಿನ ಜನರ ಒಮ್ಮತದ ಕೃತಜ್ಞತೆ, ಮೆಚ್ಚುಗೆಗಳನ್ನು ತಿಳಿಸಲು ಇಚ್ಛಿಸುತ್ತೇನೆ. ಅಲ್ಲದೆ ಸಮ್ಮೇಳನದ ವೈಜ್ಞಾನಿಕ ವಿಭಾಗದ ಅಧ್ಯಕ್ಷನಾಗಿ ಹಾಗೂ ಇಲ್ಲಿನ ಆಗುಹೋಗುಗಳಿಗೆ ಒಬ್ಬ ಸಾಕ್ಷಿಯಾಗಿ, ಸ್ವಾಮಿ ವಿವೇಕಾನಂದರಿಗೆ ಇಲ್ಲಿ ಯಾವ ಉನ್ನತ ಸ್ಥಾನ ವನ್ನು ನೀಡಲಾಗಿದೆ, ಮತ್ತು ಅವರು ಮಾಡುತ್ತಿರುವ ಸತ್ಕಾರ್ಯ ಎಂಬಹುದು ಎಂಬುದರ ಬಗ್ಗೆ ನನ್ನ ವೈಯಕ್ತಿಕ ಪ್ರಮಾಣವನ್ನು ನೀಡಲು ನಾನು ಉತ್ಸಾಹಿತನಾಗಿದ್ದೇನೆ... ಮಿಷನರಿಗಳು ಸಾರಿದ್ದ ಅರೆಬರೆ ಸತ್ಯಗಳ ಹಾಗೂ ಪೊಳ್ಳು ಕತೆಗಳ ಪರಿಣಾಮವಾಗಿ ಹಿಂದೂಗಳೆಂದರೆ ಮೂಢರು-ಅನಾಗರಿಕರೆಂದು ನಂಬಿದ್ದ ಸಾರ್ವಜನಿಕರ ಮೇಲೆ, ವಿವೇಕಾನಂದರು ತಮ್ಮ ವ್ಯಕ್ತಿತ್ವ ದಿಂದಲೂ ಅಸಾಧಾರಣ ಭಾಷಾ ಸಾಮರ್ಥ್ಯದಿಂದಲೂ ಉಂಟುಮಾಡಿದ ಪರಿಣಾಮವೂ ಗಳಿಸಿದ ಮೆಚ್ಚುಗೆಯೂ ಅಸದೃಶವಾದದ್ದು. ಇಲ್ಲಿನ ಜನಗಳಲ್ಲಿ ಉಂಟಾಗಿರುವ ಉತ್ಸಾಹಕ್ಕೆ, ಮುಖ್ಯ ವಾಗಿ, ವಿವೇಕಾನಂದರ ಮೂಲಕ ಭಾರತವು ಅಮೆರಿಕದ ಜನತೆಯಲ್ಲಿ ಉಂಟುಮಾಡಿದ ಆಧ್ಯಾತ್ಮಿಕ ಸತ್ಯಗಳ ಬಗೆಗಿನ ನಿಜವಾದ ಜಿಜ್ಞಾಸೆಯೇ ಕಾರಣವೆಂಬುದರಲ್ಲಿ ಸಂದೇಹವಿಲ್ಲ....

“ಈವರೆಗೆ ಅಮೆರಿಕದಲ್ಲಿ ಹಿಂದುತ್ವೀಕರಣದ ನಿಟ್ಟಿನಲ್ಲಿ ಕೆಲಸ ಮಾಡುತ್ತಿದ್ದ ಎಲ್ಲ ಶಕ್ತಿಗಳೂ ಸ್ವಾಮಿ ವಿವೇಕಾನಂದರ ಶ್ರಮದಿಂದಾಗಿ ಹೊಸ ಸ್ಫೂರ್ತಿಯನ್ನು ಪಡೆದಿವೆ. ಹಿಂದೂಧರ್ಮದ ಕುರಿತಾಗಿ ಈ ದೇಶದಲ್ಲಿ ಪ್ರಚಲಿತವಿರುವುದು ಕೇವಲ ಆಂಗ್ಲೀಕರಣಗೊಂಡ ಸತ್ವರಹಿತ ಅಭಿ ಪ್ರಾಯಗಳು ಮಾತ್ರವೇ. ನಿಜವಾದ ಹಿಂದೂ ಧರ್ಮದ ಇಂತಹ ಅಧಿಕೃತ ಪ್ರತಿನಿಧಿಯೊಬ್ಬ ಅಮೆರಿಕದ ಚಿಂತನಶೀಲ ಜನರಿಗೆ ಹಿಂದೆಂದೂ ಲಭ್ಯವಾಗಿರಲಿಲ್ಲ! ಅವರನ್ನು ಕಳಿಸಿಕೊಟ್ಟಿದ್ದ ಕ್ಕಾಗಿ ಅಮೆರಿಕವು ಭಾರತಕ್ಕೆ ಅತ್ಯಂತ ಕೃತಜ್ಞವಾಗಿದೆ.”

ಮರ್ವಿನ್ ಸ್ನೆಲ್​ರಂತಹ ಪ್ರತಿಷ್ಠಿತ ವ್ಯಕ್ತಿ ಬರೆದ ಈ ಪತ್ರವನ್ನು ಹಲವಾರು ಪತ್ರಿಕೆಗಳು ಪ್ರಕಟಿಸಿ, ವ್ಯಾಪಕ ಪ್ರಚಾರ ನೀಡಿದುವು. ಇದರಿಂದಾಗಿ ವಿವೇಕಾನಂದರ ಯಶಸ್ಸಿನ ಬಗ್ಗೆ ಭಾರತೀ ಯರಿಗೆ ಮತ್ತಷ್ಟು ಖಚಿತ ಅಭಿಪ್ರಾಯವುಂಟಾಗಲು ಸಾಧ್ಯವಾಯಿತು.

ಸ್ವಾಮೀಜಿಯ ಪ್ರಖ್ಯಾತಿಯು ಭಾರತದಲ್ಲಿ ಬೃಹತ್ತರವಾದ ಅಲೆಯೋಪಾದಿಯಲ್ಲಿ ಹರಡಿ ಕೊಳ್ಳಲು ಕಾರಣವಾದ ಇನ್ನೊಂದು ಅಂಶವೆಂದರೆ, ಅವರು ಸಮ್ಮೇಳನದಲ್ಲಿ ಮಾಡಿದ “ಹಿಂದೂ ಧರ್ಮ” ಎಂಬ ಭಾಷಣ. ಈ ಭಾಷಣದ ಪೂರ್ಣಪಾಠವನ್ನು ಅಚ್ಚುಹಾಕಿಸಿ ಮದ್ರಾಸಿನಲ್ಲಿ ಹಾಗೂ ಕಲ್ಕತ್ತದಲ್ಲಿ ಹಂಚಲಾಯಿತು. ಈ ಭಾಷಣವು ಜನರಿಗೆ ಅತಿ ಹೆಚ್ಚಿನ ವಿಸ್ಮಯವನ್ನುಂಟು ಮಾಡಿತು. ಏಕೆಂದರೆ ಅದನ್ನೋದಿದವರಿಗೆ, ಸ್ವಾಮೀಜಿ ಅಮೆರಿಕೆಯ ಜನರ ಮೇಲೆ ಯಾವ ರೀತಿಯ ಪ್ರಭಾವವನ್ನು ಬೀರಿದರು ಮತ್ತು ಅವರ ಅದ್ಭುತ ಯಶಸ್ಸಿಗೆ ಕಾರಣವೇನು ಎಂಬುದು ಸ್ಪಷ್ಟವಾಯಿತು. ಭಾರತೀಯ ಪತ್ರಿಕೆಗಳು ಈ ಲೇಖನದ ಕೆಲವು ಭಾಗಗಳನ್ನು ಪ್ರಕಟಿಸಿ, ಅದನ್ನು ಪ್ರಶಂಸಿಸಿ ವಿಮರ್ಶೆಗಳನ್ನೂ ಬರೆದುವು.

ಹಿಂದೂಧರ್ಮದ ಕುರಿತಾದ ಸ್ವಾಮೀಜಿಯ ಭಾಷಣವು ಭಾರತದಾದ್ಯಂತ ಪ್ರಸಾರವಾಗಿ, ಅದೊಂದು ಅಲ್ಲೋಲಕಲ್ಲೋಲವನ್ನೇ ಉಂಟುಮಾಡಿತು. ಅವರ ಕಾರ್ಯೋದ್ದೇಶದ ಅದ್ಭುತ ಐತಿಹಾಸಿಕ ಮಹತ್ವವು ಸ್ಪಷ್ಟವಾಗಿ ವ್ಯಕ್ತವಾಗಿತ್ತು. ಇದರಿಂದಾಗಿ ಸಂತೋಷಪಟ್ಟವರು ಹಲವ ರಾದರೆ ಅಸೂಯೆ ತಾಳಿದವರು ಕೆಲವರು. ಸ್ವಾಮೀಜಿ ಗಳಿಸಿದ ಯಶಸ್ಸು-ಕೀರ್ತಿಗಳ ಮುಂದೆ ನಿಸ್ತೇಜರಾದ ಥಿಯಾಸೊಫಿಸ್ಟರೂ ಬ್ರಾಹ್ಮಸಮಾಜೀಯರೂ ಕ್ರೈಸ್ತರೊಂದಿಗೆ ಕೈಜೋಡಿಸಿ ಅವ ರನ್ನು ಕಟುವಾಗಿ ನಿಂದಿಸಿದರು. ಅವರ ಭಾಷಣವು ಹಿಂದೂಧರ್ಮವನ್ನು ಯಾವ ರೀತಿಯಲ್ಲೂ ಪ್ರತಿನಿಧಿಸುವುದಿಲ್ಲವೆಂದು ಪ್ರಚಾರ ಮಾಡಿದರು. ಇದಕ್ಕೆ ಉತ್ತರವಾಗಿ ತನ್ನ ಪ್ರತಿನಿಧಿಯೊಬ್ಬರು ಬರೆದ ಪತ್ರವನ್ನು ‘ಇಂಡಿಯನ್ ಮಿರರ್​’ ಪತ್ರಿಕೆ ಪ್ರಕಟಿಸಿತು:

“ಸ್ವಾಮಿ ವಿವೇಕಾನಂದರ ಅಭೂತಪೂರ್ವ ಯಶಸ್ಸು, ಕ್ರೈಸ್ತರಲ್ಲಿ ಹಾಗೂ ಬ್ರಾಹ್ಮಸಮಾಜೀ ಯರಲ್ಲಿ ಪ್ರಬಲವಾದ ಅಸೂಯೆಯನ್ನು ಹುಟ್ಟಿಸಿಬಿಟ್ಟಿದೆ. ಆ ಹೊಟ್ಟೆಯುರಿಯನ್ನು ತಾಳಲಾರದೆ ಇವರೆಲ್ಲ ವಿವೇಕಾನಂದರ ಕೀರ್ತಿಗೆ ಮಸಿ ಬಳಿಯಲು ಶಕ್ತಿಮೀರಿ ಪ್ರಯತ್ನಿಸುತ್ತಿರುವುದನ್ನು ಕಂಡು ನನಗೆ ಅತ್ಯಂತ ದುಃಖವುಂಟಾಗಿದೆ. ಇವರು ಸ್ವಾಮೀಜಿಯ ಬೆನ್ನಹಿಂದೆ ಅವರ ವಿರುದ್ಧವಾಗಿ ಪ್ರಚಾರಮಾಡುತ್ತ ಅವರೊಂದಿಗೆ ಯುದ್ಧವನ್ನೇ ಪ್ರಾರಂಭಿಸಿಬಿಟ್ಟಿದ್ದಾರೆ. ಅದರೆ ಇವರಿಗೆ ಆ ಯುದ್ಧದಲ್ಲಿ ಸೋಲು ಕಟ್ಟಿಟ್ಟದ್ದು. ಸ್ವಾಮಿ ವಿವೇಕಾನಂದರು ಈಗ ಒಂದು ಮಹಾ ಶಕ್ತಿ. ಅವರ ಸುಸಂಸ್ಕೃತ ವರ್ತನೆ, ಸುಲಲಿತ ವಾಗ್ಝರಿ, ಆಶ್ಚರ್ಯಕರ ವ್ಯಕ್ತಿತ್ವ–ಇವು ಜಗತ್ತಿನ ಜನರಲ್ಲಿ ಹಿಂದೂಧರ್ಮದ ಕುರಿತಾಗಿ ತೀರ ವಿಭಿನ್ನವಾದ ಒಂದು ಹೊಸ ಅಭಿಪ್ರಾಯವನ್ನು ಮೂಡಿಸಿದೆ. ಹಿಂದೂಧರ್ಮದ ಬಗೆಗಿನ ಅವರ ಲೇಖನವು ನಿಜಕ್ಕೂ ಒಂದು ಅನರ್ಘ್ಯ ರತ್ನ. ಅದನ್ನು ಪ್ರತಿಯೊಬ್ಬನೂ ಓದಿ, ಮನನ ಮಾಡಬೇಕು; ಮತ್ತು ಅದರ ಮಹತ್ವವನ್ನು ಸರಿಯಾಗಿ ಗ್ರಹಿಸಬೇಕು. ಅವರಾಡಿದ ಒಂದೊಂದು ಮಾತೂ ಸಂಗ್ರಹಯೋಗ್ಯವಾದ ವಿಷಯ. ಕೇವಲ ಅರ್ಧ ಗಂಟೆಯ ಅಲ್ಪಾವಧಿಯಲ್ಲಿ ಸ್ವಾಮೀಜಿ, ಇಡೀ ಹಿಂದೂಧರ್ಮದ ಸ್ಪಷ್ಟ ಚಿತ್ರಣವನ್ನು ನೀಡಲು ಹೇಗೆ ಸಮರ್ಥರಾದರೆಂಬುದೇ ಒಂದು ಆಶ್ಚರ್ಯಕರ ಅಂಶ....”

ಅಂತೂ ಸರ್ವಧರ್ಮ ಸಮ್ಮೇಳನದ ಮೂಲಕ ವಿವೇಕಾನಂದರ ಹೆಸರು ವಿಶ್ವದಾದ್ಯಂತ ಪ್ರಖ್ಯಾತವಾಗಿತ್ತು. ಅದರೊಂದಿಗೆ, ಅವರಿಗೆ ಪ್ರಿಯವಾದ ಸ್ವತಂತ್ರ ಪರಿವ್ರಾಜಕ ಜೀವನವೂ ಕೊನೆಗೊಂಡಿತ್ತು. ಇದೀಗ ಅವರ ಜೀವನದಲ್ಲಿ ಹೊಸದೊಂದು ಅಧ್ಯಾಯ ಪ್ರಾರಂಭವಾಗಿತ್ತು. ಅವರ ನಿಜವಾದ ಕಾರ್ಯ ಇನ್ನು ಮುಂದೆ ಶುರುವಾಗಬೇಕಷ್ಟೆ. ಏಕೆಂದರೆ ಸರ್ವಧರ್ಮ ಸಮ್ಮೇಳನವು ಆ ದಿಸೆಯಲ್ಲಿ ಅವರ ಪ್ರಥಮ ಮೆಟ್ಟಿಲಷ್ಟೆ. ಹಿಂದೆಯೇ ನೋಡಿರುವಂತೆ, ಅವರು ಅಮೆರಿಕೆಗೆ ಬಂದದ್ದರ ಹಿಂದಿದ್ದ ಮುಖ್ಯ ಉದ್ದೇಶಗಳು ಇವು: ಮೊದಲನೆಯದಾಗಿ, ಭಾರತದ ಕೋಟ್ಯಂತರ ದೀನದರಿದ್ರರ ಸರ್ವತೋಮುಖ ಪ್ರಗತಿಗಾಗಿ ಸಂನ್ಯಾಸಿಗಳ ಸಂಘವೊಂದನ್ನು ನಿರ್ಮಿಸಲು ಆರ್ಥಿಕ ನೆರವನ್ನು ಪಡೆಯುವುದು; ಎರಡನೆಯದಾಗಿ, ಹಿಂದೂಧರ್ಮದ ಅನರ್ಘ್ಯ ನಿಧಿಯಾದ ಆಧ್ಯಾತ್ಮಿಕತೆಯನ್ನು ಹೊರತೆಗೆದು, ಅದು ಪಾಶ್ಚಾತ್ಯರೂ ಸೇರಿದಂತೆ ಪ್ರತಿಯೊಬ್ಬ ರಿಗೂ ದೊರಕುವಂತೆ ಮಾಡುವುದು. ಈ ಎರಡು ಮಹಾಯೋಜನೆಗಳನ್ನು ಅನುಷ್ಠಾನಕ್ಕೆ ತರುವತ್ತ ಸ್ವಾಮೀಜಿ ಈಗ ಕಾರ್ಯೋನ್ಮುಖರಾದರು.

