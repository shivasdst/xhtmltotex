
\chapter{“ಪಾತಾಳಕ್ಕೆ ಹೋದರೂ ಬಿಡೆ”}

\noindent

ದೆಹಲಿಯಲ್ಲಿ ಸ್ವಾಮೀಜಿ ತಮ್ಮ ಗುರುಭಾಯಿಗಳಿಂದ ಬೇರ್ಪಟ್ಟು ಮುನ್ನಡೆದಾಗ ಅಖಂಡಾ ನಂದರು ತಾವೂ ಬರುವುದಾಗಿ ಕಾಡಿದ್ದರು. ಅದರಂತೆ, ಅಲ್ಲಿಂದಲೂ ಅವರು ಸ್ವಾಮೀಜಿಯ ಬೆನ್ನಟ್ಟಿ ಬರುತ್ತಲೇ ಇದ್ದರು. ಅವರ ಪಾಲಿಗೆ ಇದೊಂದು ಪತ್ತೇದಾರಿಯೇ ಆಗಿ ಪರಿಣಮಿಸಿತ್ತು. ಏಕೆಂದರೆ ಸ್ವಾಮೀಜಿ ತಮ್ಮ ಹೆಸರನ್ನು ಆಗಾಗ ಬದಲಿಸಿಕೊಳ್ಳುತ್ತಿದ್ದರು. ಅಲ್ಲವೆ, ಸಾಧ್ಯ ವಾದಷ್ಟೂ ಅಜ್ಞಾತರಾಗಿರಲು, ಅನಾಮಧೇಯರಾಗಿರಲು ಪ್ರಯತ್ನಿಸುತ್ತಿದ್ದರು. ಜೊತೆಗೆ ಯಾರಿಗೂ ತಮ್ಮ ಪೂರ್ವಾಪರಗಳ ಬಗ್ಗೆ ಹೆಚ್ಚಾಗಿ ಹೇಳುತ್ತಿರಲಿಲ್ಲ. ಆದರೆ ‘ಇಂಗ್ಲಿಷ್ ಮಾತ ನಾಡುವ ಅದ್ಭುತ ಸಂನ್ಯಾಸಿ’ಯ ವರ್ಣನೆಯನ್ನು ಕೇಳುತ್ತಿದ್ದ ಅಖಂಡಾನಂದರಿಗೆ, ಅದು ತಮ್ಮ ಪ್ರಿಯ ಗುರುಭಾಯಿಯೇ ಎಂಬುದರಲ್ಲಿ ಸಂದೇಹವಿರಲಿಲ್ಲ. ಆದ್ದರಿಂದ ಅವರು ಸ್ವಾಮೀಜಿ ಯನ್ನು ಹಿಂಬಾಲಿಸುವ ಕೆಲಸವನ್ನು ಬಿಡಲಿಲ್ಲ. ಈ ಅಲೆದಾಟಗಳ ವಿವರಗಳನ್ನೆಲ್ಲ ಅವರು ತಮ್ಮ ಆತ್ಮಕಥೆಯಲ್ಲಿ ಬರೆದಿಟ್ಟಿದ್ದು, ವಿವೇಕಾನಂದರ ಪರಿವ್ರಾಜಕ ಜೀವನದ ಪರಿಚಯ ಮಾಡಿ ಕೊಳ್ಳಲು ಇವು ಸಹಾಯಕವಾಗಿವೆ.

ಹಿಂದೆ ಜೈಪುರದಲ್ಲಿ ಸ್ವಾಮೀಜಿಯನ್ನು ಕೂಡಿಕೊಳ್ಳಲು ಸಮರ್ಥರಾಗಿದ್ದ ಅಖಂಡಾನಂದರು, ಅವರ ಛೀಮಾರಿಯನ್ನೂ ಗಣಿಸದೆ ಪುನಃ ಅವರ ಬೆನ್ನುಹತ್ತಿದ್ದರು. ಆದರೆ ಸ್ವಾಮೀಜಿಗೆ ಇದು ತಿಳಿದಿರಲಿಲ್ಲ. ಜೈಪುರವನ್ನು ಬಿಟ್ಟಮೇಲೆ ಅವರು, ಕೆಲವು ತಿಂಗಳ ಕಾಲ ಖೇತ್ರಿಯಲ್ಲಿದ್ದು ಅಜ್ಮೀರಕ್ಕೆ ಹೋದರೆಂಬುದನ್ನು ತಿಳಿದು, ಅಖಂಡಾನಂದರು ಅಜ್ಮೀರಕ್ಕೆ ಧಾವಿಸಿದರು. ಆದರೆ ಸ್ವಾಮೀಜಿ ಅಹಮದಾಬಾದಿಗೆ ಹೊರಟರೆಂದು ಗೊತ್ತಾದಾಗ ತಾವೂ ಅಲ್ಲಿಗೆ ಹೊರಟರು. ದಾರಿಯಲ್ಲಿರುವ ಬಿಯಾವರ್​ನಲ್ಲಿ ಅವರಿಗೆ ತಿಳಿದುಬಂತು–ಕೆಲದಿನಗಳ ಹಿಂದೆತಾನೆ ಸ್ವಾಮೀಜಿ ಅಜ್ಮೀರಿಗೆ ಹಿಂದಿರುಗಿದರು ಎಂದು. ಆದರೆ ಈ ವೇಳೆಗೆ ಅವರು ಮತ್ತೆ ಅಹಮದಾಬಾದಿಗೆ ಹೋಗಿರಬಹುದು ಎಂದು ಆಲೋಚಿಸಿ, ಅಖಂಡಾನಂದರು ಅಹಮದಾಬಾದಿಗೇ ಹೋದರು. (ಅಜ್ಮೀರದಿಂದ ಅಹಮದಾಬಾದಿಗೆ ೩ಂ೫ ಮೈಲಿ!) ಅಹಮದಾಬಾದಿಗೆ ಹೋಗಿ ನೋಡುತ್ತಾರೆ –ಸ್ವಾಮೀಜಿ ಅಲ್ಲಿಂದ ವಾಧ್ವಾನ್ ಹೊರಟುಬಿಟ್ಟಿದ್ದಾರೆ! ಇದರಿಂದ ನಿರಾಶರಾಗದೆ ಅಖಂಡಾ ನಂದರು ತಾವೂ ವಾಧ್ವಾನ್​ಗೆ ಓಡಿದರು. ಇಲ್ಲಿಗೆ ಬಂದಾಗ, ಸ್ವಾಮೀಜಿ ಜುನಾಗಢದಲ್ಲಿ ದಿವಾನರ ಅತಿಥಿಯಾಗಿ ಉಳಿದುಕೊಂಡಿದ್ದಾರೆ ಎಂದು ಗೊತ್ತಾಯಿತು. ಸರಿ ತಕ್ಷಣ ಜುನಾಗಢಕ್ಕೆ ಹೋದರು. ಇಲ್ಲಾದರೂ ಸಿಕ್ಕರೆ ಸ್ವಾಮೀಜಿ? ಇಲ್ಲ! ನಾಲ್ಕು ದಿನಗಳ ಹಿಂದೆ ತಾನೆ ಅವರು ಪೋರ್​ಬಂದರ್ ಮಾರ್ಗವಾಗಿ ದ್ವಾರಕೆಗೆ ಹೊರಟರೆಂಬ ಸುದ್ದಿ ತಿಳಿಯಿತು. ಛಲಬಿಡದ ತ್ರಿವಿಕ್ರಮನಂತೆ ದ್ವಾರಕೆಗೆ ಹೋದ ಅಖಂಡಾನಂದರಿಗೆ ತಿಳಿದುಬಂದದ್ದೇನೆಂದರೆ, ಸ್ವಾಮಿ ಗಳೊಬ್ಬರು ಭೆಟ್​ದ್ವಾರಕೆಗೆ ಹೋದರು ಎಂದು. ಆದರೆ ಈ ವೇಳೆಗೆ ಸ್ವಾಮೀಜಿ ಬೇಗಬೇಗನೆ ಸುತ್ತುಮುತ್ತಲ ಕ್ಷೇತ್ರಗಳ ಸಂದರ್ಶನವನ್ನು ಮುಗಿಸಿ ಮುನ್ನಡೆವ ಆತುರದಲ್ಲಿದ್ದರು. ಅಖಂಡಾ ನಂದರು ಭೆಟ್​ದ್ವಾರಕೆಗೆ ಹೋದರೆ, ಸ್ವಾಮೀಜಿ ಕಚ್​ನ ಮಹಾರಾಜನ ಆಹ್ವಾನದ ಮೇರೆಗೆ ಮಾಂಡವಿಗೆ ತೆರಳಿದರೆಂದು ತಿಳಿದುಬಂತು.

ನೋಡಿದರೆ ಅಖಂಡಾನಂದರಿಂದ ತಪ್ಪಿಸಿಕೊಳ್ಳುವುದಕ್ಕಾಗಿಯೇ ಸ್ವಾಮೀಜಿ ಹೀಗೆ ಓಡು ತ್ತಿದ್ದಾರೆಯೋ ಎಂಬಂತಿದೆ. ಅಖಂಡಾನಂದರಿಗಂತೂ ಇದು ಮರುಮರೀಚಿಕೆಯ ಬೆನ್ನಟ್ಟಿದ ಅನುಭವ. ಆದರೇನು? ಅವರು ತಮ್ಮ ನಿರ್ಧಾರವನ್ನು ಬಿಡಲೊಲ್ಲರು. ಸರಿ, ಮಾಂಡವಿಗೆ ಹೊರಟರು. ಅವರು ಇಲ್ಲಿಗೆ ಬರುವಷ್ಟರಲ್ಲಿ ಮಹಾರಾಜನು ಒದಗಿಸಿದ್ದ ಕೆಲವು ಅಂಗರಕ್ಷಕ ರೊಂದಿಗೆ ಸ್ವಾಮೀಜಿ ನಾರಾಯಣ ಸರೋವರದ ದಾರಿಹಿಡಿದಾಗಿತ್ತು. ಅಖಂಡಾನಂದರು ಬೆನ್ನಟ್ಟಿ ಹೋಗಲು ಸಿದ್ಧರಾದರು. ಅಲ್ಲಿಂದ ನಾರಾಯಣ ಸರೋವರಕ್ಕೆ ಎಂಬತ್ತು ಮೈಲಿಯ ಅಂತರ. ಹಾದಿಯೂ ಚೆನ್ನಾಗಿರಲಿಲ್ಲ. ಸಾಲದಕ್ಕೆ ಉದ್ದಕ್ಕೂ ಡಕಾಯಿತರ ಹಾವಳಿ. ‘ಬಂದದ್ದೆಲ್ಲಾ ಬರಲಿ...’ ಎಂದು ಅಖಂಡಾನಂದರು ಹಳ್ಳಿಯ ಒಬ್ಬ ಹುಡುಗನನ್ನು ಮಾರ್ಗದರ್ಶಿಯಾಗಿ ಕರೆದು ಕೊಂಡು ಹೊರಟೇಬಿಟ್ಟರು! ಅವರು ಹುಡುಕಿಕೊಂಡು ಹೊರಟಿರುವುದು ಸ್ವಾಮೀಜಿಯನ್ನ ಲ್ಲವೆ? ಈ ಡಕಾಯಿತರಿಗೆಲ್ಲ ಹೆದರಿ ಕುಳಿತಿರಲು ಸಾಧ್ಯವೆ!

ಆದರೆ ಅಖಂಡಾನಂದರು ಒಂದು ಮುನ್ನೆಚ್ಚರಿಕೆಯನ್ನು ತೆಗೆದುಕೊಂಡಿದ್ದರು. ಏನದು? ಆ ಹುಡುಗನನ್ನು ಅವರು ಮೊದಲೇ ಕೇಳಿದ್ದರು–“ನನ್ನ ಹತ್ತಿರ ಇರುವುದನ್ನೆಲ್ಲ ತೆಗೆದುಕೊ. ಆದರೆ ನನ್ನನ್ನು ಕೊಲ್ಲಬೇಡ” ಎನ್ನಬೇಕಾದರೆ ಆ ಪ್ರದೇಶದ ಕಚ್ಚೀ ಭಾಷೆಯಲ್ಲಿ ಏನು ಹೇಳಬೇಕು, ಎಂದು. ಆ ಹುಡುಗ ಹೇಳಿಕೊಟ್ಟಿದ್ದ–“ಮೇರೆ ಗನೊ, ಮೇರೆ ಗನೊ, ಮುಕೇ ಮರ್ಯೋ ಮ್ಯು” ಎಂದು. ಅಖಂಡಾನಂದರು ಈ ತಾರಕಮಂತ್ರವನ್ನು ಜಪಿಸುತ್ತ ನಡೆದರು.

ಸ್ವಲ್ಪ ದೂರ ನಡೆದ ಮೇಲೆ, ಅಖಂಡಾನಂದರು ಆ ಹುಡುಗನನ್ನು ವಾಪಸ್ಸು ಕಳಿಸಿಕೊಟ್ಟು ತಾವೊಬ್ಬರೇ ಮುನ್ನಡೆದರು. ಹೀಗೆ ಒಂದೇ ಸಮನೆ ಸುಮಾರು ಐವತ್ತು ಮೈಲಿ ನಡೆದರು. ನಾರಾ ಯಣ ಸರೋವರಕ್ಕೆ ಇನ್ನೂ ಮೂವತ್ತು ಮೈಲಿ ಇತ್ತು. ಇಲ್ಲಿಯವರೆಗೂ ಯಾವ ಡಕಾಯಿತರೂ ಎದುರಾಗಿರಲಿಲ್ಲ. ಇನ್ನು ಮುಂದಿನ ದಾರಿ ಹಳ್ಳಿಗಳ ಮೂಲಕ ಹಾದುಹೋಗುವಂಥದಾದ್ದರಿಂದ ಅಷ್ಟಾಗಿ ಭಯವಿರಲಿಲ್ಲ. ಹೀಗೇ ನಡೆದು ರಾತ್ರಿಯ ವೇಳೆಗೆ ಒಂದು ಹಳ್ಳಿಯನ್ನು ಮುಟ್ಟಿದರು. ಇಲ್ಲಿ ಆ ರಾತ್ರಿಯನ್ನು ಕಳೆದು ಮರುದಿನ ಪ್ರಯಾಣವನ್ನು ಮುಂದುವರಿಸಿದರು.

ಇಲ್ಲಿಂದ ನಾರಾಯಣ ಸರೋವರಕ್ಕೆ ೧೪ ಮೈಲಿ. ಕಾಲುದಾರಿಯಲ್ಲಾದರೆ ಇನ್ನೂ ಎರಡು ಮೈಲಿ ಕಡಮೆ; ಅಲ್ಲದೆ ಅದು ಹೆಚ್ಚು ಸುರಕ್ಷಿತವೂ ಆಗಿತ್ತು. ಎತ್ತಿನ ಗಾಡಿಯ ದಾರಿ ನಿರ್ಜನ ವಾಗಿತ್ತು. ಆದರೆ, ಎತ್ತಿನಗಾಡಿಯಲ್ಲಿ ಪ್ರಯಾಣ ಮಾಡಿದ್ದ ಸ್ವಾಮೀಜಿ, ಹಿಂದಿರುಗುವಾಗಲೂ ಎತ್ತಿನಗಾಡಿಯಲ್ಲೇ ಪ್ರಯಾಣ ಮಾಡುವುದು ನಿಶ್ಚಯವಾಗಿತ್ತು. ಆದ್ದರಿಂದ ತಾವು ಅಲ್ಲಿಗೆ ಹೋಗುವಷ್ಟರಲ್ಲಿ ಸ್ವಾಮೀಜಿ ಹೊರಟು ಬಿಟ್ಟರೇನು ಮಾಡುವುದು ಎಂದು ಅಖಂಡಾನಂದರು ಗಾಡಿಯ ರಸ್ತೆಯಲ್ಲೇ ನಡೆದರು. ಅವರ ಪವಿತ್ರ ಸತ್ಸಂಗವನ್ನು ಬಯಸಿ, ಯಾತ್ರಿಕನೊಬ್ಬ ಅವರ ಜೊತೆಯಲ್ಲೇ ನಡೆದ.ಅವನ ಕೈಯಲ್ಲಿ ಪಾತ್ರೆ-ಪಡಗ ತುಂಬಿದ್ದ ಕೊಳಕಾದ ಚೀಲವೊಂದಿತ್ತು. ಅಖಂಡಾನಂದರು ತಮ್ಮ ಈ ಸಂಗಾತಿಯೊಂದಿಗೆ ಆನಂದದಿಂದ ಸಾಗಿದರು. ಆದರೆ ದಾರಿಯಲ್ಲಿ ಏಕೋ ಈತ ಸ್ವಲ್ಪ ಹಿಂದುಳಿದ. ಅಷ್ಟರಲ್ಲಿ ಡಕಾಯಿತರು ಎದುರಾಗಿಯೇ ಬಿಟ್ಟರು. ತಕ್ಷಣ ಸ್ವಾಮಿಗಳು “ಮೇರೆ ಗನೊ... ” ಮಂತ್ರ ಪಠಿಸಿದರು. ಅಷ್ಟರಲ್ಲೇ ಎರಡು ದೊಣ್ಣೆಯ ಪೆಟ್ಟು ಗಳು ಬೆನ್ನಮೇಲೆ ಬಿದ್ದುವು. ಅದೃಷ್ಟಕ್ಕೆ, ದಪ್ಪನೆಯ ಕೋಟು ಧರಿಸಿದ್ದರಿಂದ ಪೆಟ್ಟಾಗಲಿಲ್ಲ. ಆಗ ಡಕಾಯಿತರು ಹೊಳೆಯುವ ಚೂರಿಗಳನ್ನು ಝಳಪಿಸುತ್ತ ಆ ಕೋಟನ್ನು ಬಿಚ್ಚಿಕೊಡುವಂತೆ ಹೇಳಿ ದರು. ಆದರೆ ಎಷ್ಟು ಹುಡುಕಿದರೂ ಅವರಿಗೆ ಸಿಕ್ಕಿದ್ದೆಂದರೆ ಒಂದೆರಡು ಪುಸ್ತಕಗಳು, ಹುಲಿಯ ಚರ್ಮ ಇಂತಹ ‘ಕೆಲಸಕ್ಕೆ ಬಾರದ’ ವಸ್ತುಗಳಷ್ಟೇ. ಡಕಾಯಿತರು ಅಖಂಡಾನಂದರನ್ನು ತಪಾಸಣೆ ಮಾಡುತ್ತಿರುವಾಗ, ಹಿಂದಿನಿಂದ ಬರುತ್ತಿದ್ದ ಅವರ ಸಂಗಡಿಗ ಹತ್ತಿರ ಬಂದ. ಡಕಾಯಿತರನ್ನು ಕಂಡು ಅವನ ಪ್ರಾಣಪಕ್ಷಿ ಚಡಪಡಿಸಲಾರಂಭಿಸಿತು. ಕೈಯಲ್ಲಿದ್ದ ಪಾತ್ರೆಗಳ ಗಂಟು ಢಣಾರ್ ಎಂದು ಬಿದ್ದುಹೋಯಿತು. ಆತ ಅಲ್ಲಿಂದಲೇ “ಅಯ್ಯೋ, ನನ್ನ ಪಾತ್ರೆಗಳನ್ನು ತೆಗೆದುಕೊಳ್ಳ ಬೇಡ್ರಪ್ಪೋ! ನಿಮ್ಮ ದಮ್ಮಯ್ಯಾ! ಬಿಟ್​ಬಿಡ್ರಪ್ಪೋ. ಅಯ್ಯಯ್ಯೋ!” ಎಂದು ಕಿರಿಚಿಕೊಳ್ಳ ಲಾರಂಭಿಸಿದ. ಅಖಂಡಾನಂದರಿಗೆ ಒಂದೆಡೆ ಗಾಬರಿ, ಇನ್ನೊಂದೆಡೆ ನಗು. ಅವರು ತಮ್ಮ ಸ್ನೇಹಿತ ನಿಗೆ ಕೂಗಿ ಹೇಳಿದರು, “ಏಯ್, ಸುಮ್ಮನೆ ಅದನ್ನೆಲ್ಲ ಅವರಿಗೆ ಕೊಟ್ಟುಬಿಡು. ನಾನು ನಿನಗೆ ಎಲ್ಲಿಂದಲಾದರೂ ಹೊಸ ಪಾತ್ರೆಗಳನ್ನು ಕೊಡಿಸುತ್ತೇನೆ. ಯಾಕೆ ಅನ್ಯಾಯವಾಗಿ ಒದೆ ತಿನ್ನು ತ್ತೀಯೆ?” ಆದರೆ ಅವನು ಸುಮ್ಮನಾಗಲಿಲ್ಲ. ಇನ್ನಷ್ಟು ಬೊಬ್ಬೆ ಹೊಡೆದ. ಅಷ್ಟರಲ್ಲಿ ಡಕಾಯಿ ತರು ಬಂದು ಅವನಿಗೊಂದೆರಡು ಕೊಟ್ಟರು. ಆದರೆ ಅವನ ಚೀಲವನ್ನು ನೋಡಿ ಅವರಿಗೆ ಎಷ್ಟು ಅಸಹ್ಯವಾಯಿತೆಂದರೆ, ಅದರಲ್ಲಿ ಚಿನ್ನವೇ ಇದ್ದರೂ ತಮಗದು ಬೇಕಿಲ್ಲ ಎಂದು ತೀರ್ಮಾನಿಸಿ ದರು. ಆದರೆ ಇದರಿಂದಾಗಿ ತಮಗೆ ವೃಥಾ ಶ್ರಮ, ಸಮಯ ನಷ್ಟವಾಯಿತಲ್ಲ ಎಂಬ ಸಿಟ್ಟಿನಿಂದ ಇಬ್ಬರನ್ನೂ ಅವರ ಬಟ್ಟೆಗಳಿಂದಲೇ ಕಟ್ಟಿಹಾಕಿ ಹೊರಟರು. ಆಗ ಅಖಂಡಾನಂದರು ಅವರನ್ನು ಕೂಗಿ ಕರೆದರು–“ನೋಡಿ ಇಲ್ಲಿ, ಈ ಉಣ್ಣೆಯ ಬಟ್ಟೆಯನ್ನೆಲ್ಲ ತೆಗೆದುಕೊಂಡು ಹೋಗಿ. ಪಾಪ, ನಿಮಗೆ ಚಳಿಗಾಲಕ್ಕೆ ಆಗುತ್ತದೆ.” ಆಗ ಆ ಡಕಾಯಿತರ ನಾಯಕ ಹಿಂದಿರುಗಿ ಬಂದು ಅವರ ಕಟ್ಟುಗಳನ್ನೆಲ್ಲ ಬಿಚ್ಚಿ, “ಮಹಾರಾಜ್, ನೀವು ಹೋಗಬಹುದು” ಎಂದು ಹೇಳಿ, ನಮಸ್ಕರಿಸಿ ಓಡಿ ಹೋದ.

ಅಂತೂ ಅಖಂಡಾನಂದರು ‘ಕ್ಷೇಮವಾಗಿ’ ಬಂದು ನಾರಾಯಣ ಸರೋವರವನ್ನು ತಲುಪಿ ದರು. ಪಾಪ, ಇಷ್ಟೆಲ್ಲ ಕಷ್ಟಪಟ್ಟು ಬಂದರೆ, ಹಿಂದಿನ ರಾತ್ರಿ ತಾನೆ ಸ್ವಾಮೀಜಿ ಬೇರೊಂದು ಮಾರ್ಗವಾಗಿ ಆಶಾಪುರಿ ಎಂಬ ಕ್ಷೇತ್ರಕ್ಕೆ ಹೊರಟರೆಂದು ಗೊತ್ತಾಯಿತು. ವಿಪರೀತ ದಣಿವಿನಿಂ ದಾಗಿ ಅಖಂಡಾನಂದರಿಗೆ ಜ್ವರ ಬಂದಂತಾಗಿತ್ತು. ಇಲ್ಲಿನ ಒಂದು ಮಠದ ಮಹಂತರು ಅವ ರನ್ನು ಉಪಚರಿಸಿದರು. ಆದರೆ ಸ್ವಾಮಿಗಳು ಅಲ್ಲಿ ನಿಲ್ಲದೆ ಆಶಾಪುರಿಗೆ ಹೊರಟರು. ಆಗ ಮಹಂ ತರು ಅವರಿಗೊಂದು ಕುದುರೆಯನ್ನೂ ಒಬ್ಬ ಅಂಗರಕ್ಷಕನನ್ನೂ ಒದಗಿಸಿಕೊಟ್ಟು ಬೀಳ್ಕೊಂಡರು.

ಆದರೆ ಆಶಾಪುರಿಗೆ ಬಂದ ಅಖಂಡಾನಂದರಿಗೆ ಎದುರಾದದ್ದು ನಿರಾಶೆಯೇ! ಸ್ವಾಮೀಜಿ ಮಾಂಡವಿಗೆ ತೆರಳಿದ್ದರು. ಆದರೇನು? ಅಖಂಡಾನಂದರು ಬಿಡಬೇಕಲ್ಲ! ಹೊರಟೇಬಿಟ್ಟರು. ದಾರಿಯುದ್ದಕ್ಕೂ ಹಳ್ಳಿಗಳ ಜನ ಅವರಿಗೆ ಮುಂದಿನ ಹಳ್ಳಿಯವರೆಗೂ ಹೋಗಲು ಕುದುರೆ ಯನ್ನೂ ಒಬ್ಬ ಅಂಗರಕ್ಷಕನನ್ನೂ ಒದಗಿಸಿಕೊಡುತ್ತಿದ್ದರು. ಹೀಗೆ ಒಂದು ಹಳ್ಳಿಗೆ ಬಂದಾಗ, ಅಲ್ಲಿನ ಜಮೀನ್ದಾರ, “ಸ್ವಾಮೀಜಿ, ರಾತ್ರಿಯ ಹೊತ್ತು ಬಹಳ ತಂಪಾಗಿರುವುದರಿಂದ ಪ್ರಯಾಣ ಸುಖಕರವಾಗಿರುತ್ತದೆ” ಎಂದು ಹೇಳಿ ರಾತ್ರಿಯಲ್ಲೇ ಅವರನ್ನು ಕಳಿಸಿಕೊಟ್ಟ.

ಕುದುರೆಯನ್ನೇರಿ ಅಖಂಡಾನಂದರು ಸಿಪಾಯಿಯೊಂದಿಗೆ ಹೋಗುತ್ತಿದ್ದಾರೆ. ಮಧ್ಯರಾತ್ರಿಯ ವೇಳೆ. ಇದ್ದಕ್ಕಿದ್ದಂತೆ ಸಿಪಾಯಿ ನಿಂತುಬಿಟ್ಟ. ಏಕೆಂದು ಕೇಳಿದರು ಅಖಂಡಾನಂದರು. ಆಗ ಅವನು ಹೇಳಿದ, “ಏನಿಲ್ಲ... ಆ ಮರದ ಹಿಂದೆ ಯಾರೋ ಒಬ್ಬ ದುಷ್ಟ ನಿಂತಿದ್ದಾನೆ...” ಸ್ವಾಮಿಗಳಿಗೆ ಸ್ವಲ್ಪ ಭಯವಾಯಿತು. ಸಿಪಾಯಿ ಧೈರ್ಯ ಹೇಳಿದ, “ಹೆದರಬೇಡಿ ಮಹಾರಾಜ್... ನಾನೂ ಅವನ ಗುಂಪಿನವನೇ. ನಾನಿರುವವರೆಗೆ ನಿಮಗೇನೂ ಭಯವಿಲ್ಲ.”

ಅಖಂಡಾನಂದರಿಗೆ ಎದೆ ಧಸಕ್ಕೆಂದಿತು. ತಮ್ಮನ್ನು ರಕ್ಷಿಸಲು ಒಬ್ಬ ಡಕಾಯಿತ! ಭಗವನ್ನಾಮವನ್ನುಚ್ಚರಿಸಲಾರಂಭಿಸಿದರು. ಆಗ ಆ ಸಿಪಾಯಿ ಹೇಳಿದ:

“ಏನು ಮಾಡುವುದು ಮಹಾರಾಜ್? ನನಗೆ ಜಮೀನ್ದಾರ ಕೊಡುವ ಸಂಬಳ ಏನೇನೂ ಸಾಕಾಗುವುದಿಲ್ಲ. ಆದ್ದರಿಂದ ಬಿಡುವು ಸಿಕ್ಕಾಗ ಈ ಕೆಲಸ ಮಾಡುತ್ತೇನೆ.”

ಬೆಳಗಿನ ಹೊತ್ತಿಗೆ ಅಖಂಡಾನಂದರು ತಮ್ಮ ಸಂಗಡಿಗನೊಂದಿಗೆ ಒಂದು ಹಳ್ಳಿಗೆ ಬಂದು, ಧರ್ಮಶಾಲೆಯೊಂದರಲ್ಲಿ ಉಳಿದುಕೊಂಡರು. ಆಗ ಆ ಸಂಗಡಿಗ ಬೇಸರದಿಂದ ಹೇಳಿದ: “ನೋಡಿ, ಈ ಸಾಧುಗಳ ಸಹವಾಸ ಸ್ವಲ್ಪವೂ ಒಳ್ಳೆಯದಲ್ಲ. ಈಗ ನನ್ನ ಕುದುರೆಗೆ ಹಾಕಲು ಹುಲ್ಲಿಲ್ಲ; ಆ ಪಕ್ಕದ ಮನೆಯಲ್ಲಿ ಸಾಕಷ್ಟು ಮೇವು ತುಂಬಿದೆ, ನೀವು ಜೊತೆಯಲ್ಲಿಲ್ಲದಿದ್ದರೆ ಈಗ ಗೋಡೆ ಹಾರಿ ಹುಲ್ಲು ತರುತ್ತಿದ್ದೆ... ಏನು ಮಾಡುವುದು?”

ಅಂತೂ ಈ ಮಹಾರಾಯನ ರಕ್ಷಣೆಯಲ್ಲಿ ಅಖಂಡಾನಂದರು ಮಾಂಡವಿಯನ್ನು ಮುಟ್ಟಿ ದರು. ಇಲ್ಲಿಂದ ಸ್ವಾಮೀಜಿ ಇನ್ನೆಲ್ಲಿಗೆ ಹೋದರೋ ಎಂದು ವಿಚಾರಿಸುವಾಗ ತಿಳಿದುಬಂತು– ಅವರು ಇಲ್ಲೇ ಇದ್ದಾರೆ ಎಂದು! ಹೌದು, ಸ್ವಾಮೀಜಿ ಮಾಂಡವಿಯಲ್ಲೇ ಇದ್ದಾರೆ!! ಅಖಂಡಾ ನಂದರಿಗೆ ತಮ್ಮ ಬೆನ್ನು ತಾವೇ ತಟ್ಟಿಕೊಳ್ಳುವಷ್ಟು ಸಂತೋಷವಾಯಿತು. ಸ್ವಾಮೀಜಿ ಒಬ್ಬ ಭಟಿ ಯಾನ ಮನೆಯಲ್ಲಿದ್ದರು. ಅಖಂಡಾನಂದರು ಒಂದು ನಿಮಿಷವೂ ತಡಮಾಡದೆ ಅಲ್ಲಿಗೆ ಓಡಿ ದರು. ಅಲ್ಲಿ ಸ್ವಾಮೀಜಿಯ ದರ್ಶನ ಮಾಡಿದಾಗ ಅವರಿಗಾದ ಆನಂದಕ್ಕೆ ಪಾರವೇ ಇಲ್ಲ. ಅಲ್ಲದೆ ಅವರು ಸ್ವಾಮೀಜಿಯ ಶರೀರಲಕ್ಷಣದಲ್ಲಿ ಒಂದು ಅದ್ಭುತ ಪರಿವರ್ತನೆಯನ್ನು ಕಂಡರು. ಹಲವಾರು ಕಾಯಿಲೆಗಳಿಂದ ಜರ್ಜರಿತವಾಗಿದ್ದ ಸ್ವಾಮೀಜಿಯ ಶರೀರ, ಈಗ ಆರೋಗ್ಯಪೂರ್ಣ ವಾಗಿ ಕಂಡುಬರುತ್ತಿದೆ! ಅವರ ಮೊಗದಲ್ಲೊಂದು ಅಪೂರ್ವಕಾಂತಿ ಮಿನುಗುತ್ತಿದೆ! ಅವರ ಸೌಂದರ್ಯದಿಂದ ಇಡೀ ಕೋಣೆಯೇ ಬೆಳಗುತ್ತಿರುವಂತೆ ಕಾಣುತ್ತಿದೆ! ಅಖಂಡಾನಂದರನ್ನು ಬಹುಕಾಲದ ಅನಂತರ ಅನಿರೀಕ್ಷಿತವಾಗಿ ಕಂಡ ಸ್ವಾಮೀಜಿಗೂ ಆದ ಆಶ್ಚರ್ಯ-ಆನಂದ ಅಷ್ಟಿಷ್ಟಲ್ಲ. “ಗ್ಯಾಂಜಿಸ್! ಇಲ್ಲಿಗೆ ಹೇಗೆ ಬಂದೆ?” ಎಂದು ಉದ್ಗರಿಸಿದರು ಸ್ವಾಮೀಜಿ. ಆಗ ಅಖಂಡಾನಂದರು, ಅವರನ್ನು ಹಿಡಿಯಲು ತಾವು ಕೈಗೊಂಡ ಪತ್ತೇದಾರೀ ಕಾರ್ಯವನ್ನು ವಿವರಿಸಿದರು. ಡಕಾಯಿತರೊಂದಿಗಿನ ಅನುಭವಗಳನ್ನು ಕೇಳಿದಾಗ ಸ್ವಾಮೀಜಿಗೆ ನಗು, ಆಶ್ಚರ್ಯ, ಗಾಬರಿ ಎಲ್ಲ ಒಟ್ಟಿಗೆ ಉಂಟಾದುವು. ಬಳಿಕ ಸ್ವಾಮೀಜಿ ತಮ್ಮ ಮೆಚ್ಚಿನ ಗುರುಭಾಯಿಯೊಂದಿಗೆ ಹರಟುತ್ತ, ನಕ್ಕು ನಲಿಯುತ್ತ ಹಲವಾರು ಗಂಟೆಗಳ ಕಾಲ ಕಳೆದರು.

ಹೀಗೆ ಕೆಲವು ದಿನಗಳಾದ ಮೇಲೆ, ಸ್ವಾಮೀಜಿ ಏಕೋ ಚಡಪಡಿಸುತ್ತಿರುವಂತೆ ಅಖಂಡಾ ನಂದರಿಗೆ ಭಾಸವಾಯಿತು. ತಮ್ಮ ಗುರುಭಾಯಿ ಅಷ್ಟೆಲ್ಲ ಕಷ್ಟಪಟ್ಟು ತಮ್ಮನ್ನು ಹುಡುಕಿದನೆಂದ ಮೇಲೆ, ಇನ್ನು ಅವನು ತಮ್ಮನ್ನು ಒಂಟಿಯಾಗಿರಲು ಬಿಡುವುದೇ ಇಲ್ಲವೇನೋ ಎಂಬ ಶಂಕೆ ಯುಂಟಾಗಿತ್ತು ಸ್ವಾಮೀಜಿಗೆ. ಕಡೆಗೊಂದು ದಿನ ಬಾಯಿಬಿಟ್ಟು ಹೇಳಿಬಿಟ್ಟರು. “ನೋಡು ಗಂಗಾಧರ, ನಾನು ಸಾಧಿಸಬೇಕಾದ ಮಹಾಕಾರ್ಯವೊಂದಿದೆ. ನೀವು ಯಾರಾದರೂ ನನ್ನ ಜೊತೆ ಯಲ್ಲಿದ್ದರೆ ಅದನ್ನು ಸಾಧಿಸಲು ನನಗೆ ಎಂದಿಗೂ ಸಾಧ್ಯವಾಗುವುದೇ ಇಲ್ಲ.” ಆದರೆ ಈ ಮಾತನ್ನು ತಮ್ಮನ್ನು ಉದ್ದೇಶಿಸಿ ಹೇಳಿದ್ದಲ್ಲವೇನೋ ಎಂಬಂತಿದ್ದುಬಿಟ್ಟರು ಅಖಂಡಾನಂದರು.

ಅಖಂಡಾನಂದರು ಹೀಗೆ ಸ್ವಾಮೀಜಿಯ ಬೆನ್ನಟ್ಟಿ ಬಂದುದಕ್ಕೆ ಕಾರಣವನ್ನು ಊಹಿಸಬಹುದು. ಮೊದಲನೆಯದಾಗಿ, ಸ್ವಾಮೀಜಿಯ ಬಗ್ಗೆ ಅವರಿಗಿದ್ದ ಅಪರಿಮಿತ ಪ್ರೇಮ, ವಿಶ್ವಾಸ, ಹಾಗೂ ತಮ್ಮ ನಾಯಕನೆಂಬ ಗೌರವಾದರಗಳು ಅವರನ್ನು ಸ್ವಾಮೀಜಿಯತ್ತ ಸೆಳೆಯುತ್ತಿದ್ದವು. ಎರಡನೆ ಯದಾಗಿ, ಬಾರಾನಾಗೋರನ್ನು ಬಿಟ್ಟು ಹೊರಡುವಾಗ ಅವರ ಮೇಲೆ ಶ್ರೀಮಾತೆಯವರು ಹೊರಿಸಿದ್ದ ಜವಾಬ್ದಾರಿ. ಶ್ರೀಶಾರದಾದೇವಿಯವರು ಹೇಳಿದ್ದರು, “ಗಂಗಾಧರ, ನನ್ನ ಸರ್ವಸ್ವ ವನ್ನೂ ನಿನ್ನ ಕೈಲಿರಿಸುತ್ತಿದ್ದೇನೆ. ಅದನ್ನು ಎಚ್ಚರಿಕೆಯಿಂದ ನೋಡಿಕೋ” ಎಂದು. ಅದಕ್ಕೆ ಸರಿ ಯಾಗಿ ಸ್ವಾಮೀಜಿ ತೀವ್ರ ಕಾಯಿಲೆಗೆ ಗುರಿಯಾಗಿ ಸಂಪೂರ್ಣವಾಗಿ ಇಳಿದು ಹೋಗಿದ್ದರಲ್ಲದೆ, ಸ್ವಲ್ಪ ಸುಧಾರಿಸಿಕೊಳ್ಳುತ್ತಿದ್ದಂತೆಯೇ ಎಲ್ಲ ಗುರುಬಾಯಿಗಳನ್ನೂ ಬಿಟ್ಟು ತಾವೊಬ್ಬರೇ ಅಜ್ಞಾತ ರಾಗಿ ಹೊರಟುಬಿಟ್ಟಿದ್ದರು. ಈ ಅಜ್ಞಾತವಾಸ ಇನ್ನು ಎಷ್ಟು ವರ್ಷವೋ ಯಾರಿಗೆ ಗೊತ್ತು? ಅಲ್ಲದೆ, ತಮ್ಮ ಗುರಿಯನ್ನು ಸಾಧಿಸುವವರೆಗೂ ತಾವು ಮಠಕ್ಕೆ ಹಿಂದಿರುಗುವುದಿಲ್ಲ ಎಂದು ಬೇರೆ ಹೇಳಿಬಿಟ್ಟಿದ್ದಾರೆ ಸ್ವಾಮೀಜಿ. ಹೀಗಾದರೆ ತಮ್ಮ ಸಂಘದ ಗತಿಯೇನು ಎಂಬ ಚಿಂತೆಯೂ ಅಖಂಡಾನಂದರ ಮನಸ್ಸಿನಲ್ಲಿತ್ತೆಂದು ತೋರುತ್ತದೆ.

ಅಖಂಡಾನಂದರು ತಮ್ಮ ಸೂಚನೆಯನ್ನು ಗಮನಕ್ಕೇ ತೆಗೆದುಕೊಳ್ಳಲಿಲ್ಲವಲ್ಲ ಎಂದು ಸ್ವಾಮೀಜಿ ಚಿಂತೆಗೊಳಗಾದರು. ಬಳಿಕ ಒಂದು ಉಪಾಯ ಹೂಡಿ, ಅಖಂಡಾನಂದರಿಗೆ ಹೇಳಿದರು: “ನೋಡು ಗಂಗಾಧರ, ನಾನೀಗ ಮೊದಲಿನ ನರೇಂದ್ರನಲ್ಲ. ಪೂರ್ತಿ ಕೆಟ್ಟುಹೋಗಿ ಬಿಟ್ಟಿದ್ದೇನೆ. ಸುಮ್ಮನೆ ನನ್ನನ್ನೇಕೆ ಹಿಂಬಾಲಿಸಿಕೊಂಡು ಬರುತ್ತೀ?”

ಅಷ್ಟು ಸುಲಭವಾಗಿ ಮೋಸ ಹೋಗುವವರೇ ಅಖಂಡಾನಂದರು! ನಸುನಕ್ಕು ಹೇಳುತ್ತಾರೆ: “ನೀನು ಒಂದುವೇಳೆ ಶೀಲಭ್ರಷ್ಟನಾಗಿದ್ದೀ ಎಂದೇ ಇಟ್ಟುಕೊಂಡರೂ ಅದರಿಂದ ನನಗೇನು? ನಾನಂತೂ ನಿನ್ನನ್ನು ಪ್ರೀತಿಸುತ್ತೇನೆ. ನಿನ್ನ ಶೀಲ ಒಳ್ಳೆಯದೇ ಆಗಿರಲಿ ಕೆಟ್ಟದ್ದೇ ಆಗಿರಲಿ, ಅದ ರಿಂದ ನನ್ನ ಪ್ರೀತಿಯೇನೂ ಹೆಚ್ಚುಕಡಮೆಯಾಗುವುದಿಲ್ಲ! ಆದರೆ ನಿನ್ನ ದಾರಿಗೆ ಅಡ್ಡವಾಗಿ ಬರ ಬೇಕೆಂದು ನನಗೇನೂ ಇಷ್ಟವಿಲ್ಲ. ನಿನ್ನನ್ನು ನೋಡಬೇಕೆಂಬ ತೀವ್ರ ಹಂಬಲ ನನಗಿತ್ತು. ಈಗ ನಿನ್ನನ್ನು ನೋಡಿದ ಮೇಲೆ ನನ್ನ ಮನಸ್ಸಿಗೆ ತೃಪ್ತಿಯಾಗಿದೆ. ನೀನಿನ್ನು ಏಕಾಂಗಿಯಾಗಿ ಹೋಗಬಹುದು.”

ಅಖಂಡಾನಂದರ ಬಿಚ್ಚುಮನಸ್ಸಿನ ಮಾತುಗಳನ್ನು ಕೇಳಿ ಸ್ವಾಮೀಜಿಗೆ ಸಂತೋಷವಾಯಿತು. ಮರುದಿನವೇ ಅವರು ಮಾಂಡವಿಯಿಂದ ಕಚ್​ನ ರಾಜಧಾನಿ ಭೂಜ್​ಗೆ ತೆರಳಿದರು. ಅದರ ಮರುದಿನ ಅಖಂಡಾನಂದರು ಅವರನ್ನು ಕೂಡಿಕೊಂಡರು. ಆದರೆ ಅಖಂಡಾನಂದರು ತಮ್ಮನ್ನು ಬೆನ್ನಟ್ಟಿ ಬರುತ್ತಿಲ್ಲ ಎಂದು ಖಚಿತವಾಗಿದ್ದುದರಿಂದ ಸ್ವಾಮೀಜಿ ಚಿಂತಿಸಲಿಲ್ಲ. ಇಲ್ಲಿ ಅವರಿ ಬ್ಬರೂ ಮಹಾರಾಜನ ಅತಿಥಿಗಳಾಗಿ, ಎರಡು ದಿನವಿದ್ದರು. ಇನ್ನೂ ಕೆಲದಿನ ಇರುವಂತೆ ಮಹಾರಾಜ ವಿನಂತಿಸಿಕೊಂಡಿದ್ದ. ಆದರೆ ಇನ್ನು ತಾವು ಅಲ್ಲಿ ಉಳಿದುಕೊಂಡರೆ ಜನರ ಹಾಗೂ ಅಧಿಕಾರಿ ಗಳ ದ್ವೇಷಕ್ಕೆ ಗುರಿಯಾಗಬಹುದೆಂಬ ಶಂಕೆಯಿಂದ, ರಾಜನಿಂದ ಬೀಳ್ಕೊಂಡು ಮತ್ತೆ ಮಾಂಡವಿಗೆ ಬಂದರು ಸ್ವಾಮೀಜಿ. (ಕೆಲವರ್ಷಗಳ ಹಿಂದೆ ಇಲ್ಲಿನ ಅಧಿಕಾರಿಗಳು ಸಂನ್ಯಾಸಿ ಯೊಬ್ಬರನ್ನು ವಿಷವಿಕ್ಕಿ ಕೊಂದಿದ್ದ ಸುದ್ದಿ ಅವರಿಗೆ ತಿಳಿದುಬಂದಿತ್ತು.) ಅಖಂಡಾನಂದರು ಒಂದು ವಾರದನಂತರ ಮಾಂಡವಿಗೆ ಬಂದರು. ಇಲ್ಲಿ ಅವರು ಜೊತೆಯಾಗಿ ಭಾರತದ ಸ್ಥಿತಿ-ಗತಿ ಗಳ ಬಗ್ಗೆ ಆಳವಾಗಿ ಸಮಾಲೋಚಿಸುತ್ತ ಸುಮಾರು ಹದಿನೈದು ದಿನಗಳನ್ನು ಕಳೆದರು. ಬಳಿಕ ಅವರು ಪರಸ್ಪರರಿಂದ ಬೇರೆಯಾಗಿ, ಒಂದು ವಾರವಾದ ಮೇಲೆ ಪೋರ್​ಬಂದರ್​ನಲ್ಲಿ ಕೂಡಿ ಕೊಂಡರು. ಇಲ್ಲಿ ಅವರು ದಿವಾನ ಶಂಕರಪಾಂಡುರಂಗನ ಮನೆಯಲ್ಲಿ ಇಳಿದುಕೊಂಡರು. ಸ್ವಾಮೀಜಿ ತಮ್ಮ ಆತಿಥೇಯನೊಂದಿಗೆ ಸಂಸ್ಕೃತದಲ್ಲಿ ಸಂಭಾಷಣೆ ನಡೆಸುವುದರ ಮೂಲಕ, ಸಂಸ್ಕೃತದ ಮೇಲಿನ ತಮ್ಮ ಪ್ರಭುತ್ವವನ್ನು ಮತ್ತಷ್ಟು ಹೆಚ್ಚಿಸಿಕೊಂಡರು. ಕೆಲವು ದಿನಗಳಾದ ಮೇಲೆ ಸ್ವಾಮೀಜಿ ಮತ್ತೆ ಜುನಾಗಢಕ್ಕೆ ಹೊರಟರು. ಇಲ್ಲಿಂದ ಮುಂದೆ ತಾವು ಮುಂಬಯಿ ಪ್ರಾಂತದತ್ತ ತೆರಳುವುದಾಗಿ ಅವರು ಹೇಳಿದ್ದರು. ಅಖಂಡಾನಂದರು ಸ್ವಾಮೀಜಿಯನ್ನು ಅವರ ಷ್ಟಕ್ಕೆ ಬಿಟ್ಟು, ತಾವು ಕಾಥೇವಾಡದಲ್ಲಿ ಮತ್ತಷ್ಟು ದಿನ ಉಳಿದುಕೊಂಡರು. ಸ್ವಾಮೀಜಿ ಅಮೆರಿಕ ದಿಂದ ಹಿಂದಿರುಗುವ ಮುನ್ನ ಅಖಂಡಾನಂದರು ಅವರನ್ನು ಸಂಧಿಸಿದ್ದು ಇದೇ ಕಡೆಯ ಸಲ.

