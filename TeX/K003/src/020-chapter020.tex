
\chapter{ಅಮೆರಿಕದಲ್ಲಿ ಆಧ್ಯಾತ್ಮಿಕ ಶಿಕ್ಷಣ}

\noindent

೧೮೯೫ರ ಜನವರಿಯ ವೇಳೆಗೆ, ಪಶ್ಚಿಮ ರಾಷ್ಟ್ರಗಳಲ್ಲಿನ ಸ್ವಾಮೀಜಿಯ ಕಾರ್ಯದಲ್ಲಿ ಒಂದು ಹೊಸ ಅಧ್ಯಾಯ ಪ್ರಾರಂಭವಾಯಿತು. ಆ ತಿಂಗಳ ಕೊನೆಯ ದಿನಗಳಲ್ಲಿ ಅವರು ನ್ಯೂಯಾರ್ಕ್ ನಗರದ ಒಂದು ಸಾಧಾರಣ ಪರಿಸರದ ವಸತಿಗೃಹದಲ್ಲಿ ಕೋಣೆಯೊಂದನ್ನು ಬಾಡಿಗೆಗೆ ತೆಗೆದು ಕೊಂಡರು. ಇದನ್ನು ತಮ್ಮ ವಾಸಕ್ಕಾಗಿಯೂ ಉದ್ದೇಶಿತ ತರಗತಿಗಳನ್ನು ನಡೆಸಲೂ ಬಳಸಬೇಕು ಎಂಬುದು ಅವರ ಯೋಜನೆಯಾಗಿತ್ತು. ನಾವು ಹಿಂದೆಯೇ ನೋಡಿದಂತೆ, ಸಾವಿರಾರು ಜನರು ಸೇರುವ ಸಾರ್ವಜನಿಕ ಸಭೆಗಳು, ವೃತ್ತಪತ್ರಿಕೆಗಳ ಅಬ್ಬರ, ಅವುಗಳಿಂದ ಸಿಗುವ ಪ್ರಚಾರ, ಅಭಿನಂದನೆಯ ಸುರಿಮಳೆ ಇವೆಲ್ಲ ಅವರಿಗೆ ಸಾಕುಸಾಕೆನಿಸಿತ್ತು. ಅಲ್ಲದೆ ಇವೆಲ್ಲ ಕೇವಲ ತೋರಾಣಿಕೆಯದಾಗಿ ಕಂಡಿತ್ತು. ಇವುಗಳಿಂದ ಹೆಚ್ಚಿನ ಉಪಯೋಗವೇನೂ ಆಗಲಾರದೆಂದು ಅವರಿಗೆ ಅನ್ನಿಸಿತು. ಆದ್ದರಿಂದ ಯಾವುದಾದರೂ ಶಾಂತವಾದ ಸ್ಥಳದಲ್ಲಿ ಶ್ರದ್ಧಾವಂತರಾದ ಕೆಲವೇ ವ್ಯಕ್ತಿಗಳಿಗೆ ತೀವ್ರವಾದ ತರಬೇತಿ ನೀಡಬೇಕೆಂದು ಆಶಿಸಿ, ತಮ್ಮ ಸ್ವಂತ ಕೋಣೆಯಲ್ಲಿ ತಾವು ಉಚಿತ ತರಗತಿಗಳನ್ನು ತೆಗೆದುಕೊಳ್ಳುವುದಾಗಿ ಅವರು ಪ್ರಕಟಪಡಿಸಿದರು. ಈಗಾಗಲೇ ಅವರಿಂದ ಆಕರ್ಷಿತರಾಗಿದ್ದ ಹಲವಾರು ಜನ ತಕ್ಷಣ ಉತ್ಸಾಹದಿಂದ ಮುಂದೆ ಬಂದರು.

ಜನವರಿ ೨೯ರಿಂದ ಈ ತರಗತಿಗಳು ಪ್ರಾರಂಭವಾದುವು. ಸ್ವಾಮೀಜಿಯ ಕೋಣೆ ಇದ್ದುದು ಕಟ್ಟಡದ ಎರಡನೇ ಅಂತಸ್ತಿನಲ್ಲಿ. ಸುಮಾರು ಹತ್ತು-ಹದಿನೈದು ಜನ ಸುಲಭವಾಗಿ ಕುಳಿತುಕೊಳ್ಳ ಬಹುದಾಗಿತ್ತು. ಆದರೆ ತರಗತಿ ಪ್ರಾರಂಭವಾಗುತ್ತಿದ್ದಂತೆಯೇ ವಿದ್ಯಾರ್ಥಿಗಳ ಸಂಖ್ಯೆ ವೇಗವಾಗಿ ಬೆಳೆಯಿತು. ಬ್ರೂಕ್ಲಿನ್ನಿನಲ್ಲಿ ಅವರ ಉಪನ್ಯಾಸಗಳನ್ನು ಕೇಳಿ ಆಕರ್ಷಿತರಾಗಿದ್ದ ಅನೇಕರು, ತರಗತಿಗಳಲ್ಲಿ ಭಾಗವಹಿಸುವ ಉದ್ದೇಶದಿಂದ ನ್ಯೂಯಾರ್ಕಿಗೆ ಬಂದರು. ವಿದ್ಯಾರ್ಥಿಗಳ ಸಂಖ್ಯೆ ಹೆಚ್ಚಿ ಕೋಣೆ ತುಂಬಿ ತುಳುಕಾಡಿತು. ಈ ತರಗತಿಗಳಿಗೆ ಹಾಜರಿದ್ದ ಮಿಸ್ ಸಾರಾ ವಾಲ್ಡೊ ಹೇಳುವಂತೆ ಅದೊಂದು ತುಂಬ ರಮ್ಯವಾದ ದೃಶ್ಯ. ಭಾರತೀಯ ಪದ್ಧತಿಗನುಸಾರವಾಗಿ ಸ್ವಾಮೀಜೀ ನೆಲದ ಮೇಲೆ ಕುಳಿತುಕೊಳ್ಳುತ್ತಿದ್ದರು. ಬಹುತೇಕ ಶಿಷ್ಯರೂ ಅವರಂತೆಯೇ ನೆಲದ ಮೇಲೆ ಕುಳಿತುಕೊಳ್ಳುತ್ತಿದ್ದರು. ಆದರೆ ಎಲ್ಲರೂ ಕುಳಿತುಕೊಳ್ಳುವಷ್ಟು ಸ್ಥಳಾವಕಾಶವಿಲ್ಲದ್ದ ರಿಂದ ವಿದ್ಯಾರ್ಥಿಗಳು ಸೋಫಾದ ಕೈಯಾಸರೆಗಳ ಮೇಲೆ, ಮೂಲೆಯಲ್ಲಿದ್ದ ಕೈತೊಳೆಯುವ ಕಟ್ಟೆಯ ಮೇಲೆ, ಹೀಗೆ ಎಲ್ಲಿ ಜಾಗ ಸಿಕ್ಕಿದರಲ್ಲಿ ಕುಳಿತು ಶ್ರದ್ಧೆಯಿಂದ ಆಲಿಸುತ್ತಿದ್ದರು. ಇಷ್ಟಾ ದರೂ ಜಾಗ ಸಾಲದೆ ಬಂದದ್ದರಿಂದ ಕೋಣೆಯ ಆಚೆ ಮೆಟ್ಟಿಲುಗಳ ಮೇಲೂ ಕೆಲವರು ಕುಳಿತು ಆಲಿಸುತ್ತಿದ್ದರು. ಹೀಗೆ ಆ ಶಿಷ್ಯರು ತಮ್ಮ ಬಿಗುಮಾನವನ್ನೆಲ್ಲ ಬಿಟ್ಟು, ಎಲ್ಲ ಅನನುಕೂಲತೆ ಗಳನ್ನೂ ಮರೆತು, ಸ್ವಾಮೀಜಿಯ ಬಾಯಿಂದ ಹೊರಬಿದ್ದ ಪ್ರತಿಯೊಂದು ಶಬ್ದವನ್ನೂ ಕಾತರ ದಿಂದ ಆಲಿಸುತ್ತಿದ್ದರು. ಶಿಸ್ತಿನ ಸಿಪಾಯಿಗಳಾದ ಆ ಪಾಶ್ಚಾತ್ಯರು–ಅದರಲ್ಲೂ ಬುದ್ಧಿವಂತರು, ಮೇಲ್ವರ್ಗಕ್ಕೆ ಸೇರಿದವರು–ಇಷ್ಟರ ಮಟ್ಟಿಗೆ ವಿನಮ್ರತೆಯನ್ನೂ ವಿಶ್ವಾಸವನ್ನೂ ತೋರಿಸಿದ ರೆಂದರೆ ಅದೊಂದು ಅಸಾಮಾನ್ಯ ಸಂಗತಿಯೇ ಸರಿ.

ಹೀಗೆ ನ್ಯೂಯಾರ್ಕ್ ಮಹಾನಗರದಲ್ಲಿ ಸ್ವಾಮೀಜಿ ವೇದಾಂತಪ್ರಸಾರ ಕಾರ್ಯವನ್ನು ಅತ್ಯಂತ ಸರಳವಾದರೂ ತುಂಬ ಅದ್ಭುತವಾದ ರೀತಿಯಲ್ಲಿ ಪ್ರಾರಂಭಿಸಿದರು. ಈ ತರಗತಿಗಳನ್ನು ಅವರು ಹಿಂದೆಯೇ ಪ್ರಕಟಿಸಿದ್ದಂತೆ ಉಚಿತವಾಗಿ ನಡೆಸಿಕೊಂಡು ಬಂದರು. ಅವರು ವಾಸವಾಗಿದ್ದ ಕೋಣೆಯ ಬಾಡಿಗೆಗೆ ವಿದ್ಯಾರ್ಥಿಗಳು ತಾವಾಗಿಯೇ ವಂತಿಗೆ ನೀಡಿದರು. ಆದರೆ ಇದರಿಂದಲೂ ಅವರ ಖರ್ಚುವೆಚ್ಚಗಳು ತೂಗದೆ ಬಂದಾಗ ಸ್ವಾಮೀಜಿ, ಮತ್ತೊಂದು ಹಜಾರವನ್ನು ಬಾಡಿಗೆಗೆ ತೆಗೆದುಕೊಂಡು ಅಲ್ಲಿ ಸಾರ್ವಜನಿಕ ಉಪನ್ಯಾಸಗಳನ್ನು ನೀಡಿ ಹಣವನ್ನು ಸಂಗ್ರಹಿಸಿದರು. ಹಿಂದೆ ಭಾರತೀಯ ಗುರುಕುಲ ಸಂಪ್ರದಾಯದಲ್ಲಿ ಗುರುಗಳು ತರಗತಿಗಳ ವೆಚ್ಚವನ್ನು ಭರಿಸುವುದಲ್ಲದೆ ಬಡವಿದ್ಯಾರ್ಥಿಗಳನ್ನು ತಮ್ಮ ಸ್ವಂತ ಖರ್ಚಿನಲ್ಲೇ ಸಾಕುತ್ತಿದ್ದರು. ಅಂತೆಯೇ ಸ್ವಾಮೀಜಿ, ಧರ್ಮಬೋಧನೆಯನ್ನು ಉಚಿತವಾಗಿ ಕೊಡುವುದು ತಮ್ಮ ಕರ್ತವ್ಯವೆಂದೇ ಭಾವಿಸಿದ್ದರು. ಶ್ರದ್ಧಾವಂತರಾದ ಜಿಜ್ಞಾಸುಗಳಾದ ಕೆಲವು ಶಿಷ್ಯರಿಗೆ ತಮ್ಮ ಅತ್ಯುನ್ನತ ಬೋಧನೆಗಳನ್ನು ನೀಡುವುದೇ ಅವರಿಗೊಂದು ಆನಂದದ ಸಂಗತಿಯಾಗಿತ್ತು.

ಸ್ವಾಮೀಜಿಯ ತರಗತಿಗೆ ಬರುತ್ತಿದ್ದವರಲ್ಲಿ ಅವರ ಬೋಧನೆಗಳಿಂದ ಹಾಗೂ ವೇದಾಂತ ತತ್ತ್ವಗಳಿಂದ ಆಕರ್ಷಿತರಾಗಿದ್ದವರು ಕೆಲವರು. ಇನ್ನು ಕೆಲವರು ಕುತೂಹಲ ಪ್ರೇರಿತರಾಗಿ ಬಂದವರು. ಈ ತರಗತಿಗಳಲ್ಲದೆ ಇತರ ವೇಳೆಗಳಲ್ಲಿ ಅವರನ್ನು ಸಂದರ್ಶಿಸಲೆಂದು ಹಲವಾರು ಜನ ಧಾವಿಸಿಬರುತ್ತಿದ್ದರು. ಅವರ ವ್ಯಕ್ತಿತ್ವವನ್ನೂ ಸಿದ್ಧಾಂತಗಳನ್ನೂ ಪರೀಕ್ಷಿಸಿನೋಡಲೆಂದು ಬರುತ್ತಿದ್ದವರು ಕೆಲವರು. ಇವರಲ್ಲದೆ, ಅನೇಕ ವೃತ್ತಪತ್ರಿಕೆಗಳ ಹಾಗೂ ನಿಯತಕಾಲಿಕಗಳ ವರದಿಗಾರರು ಅವರನ್ನು ಯಾವಾಗಲೂ ಮುತ್ತಿಕೊಂಡಿರುತ್ತಿದ್ದರು.

ಹೀಗೆ ಈ ತರಗತಿಗಳ ಕಾರ್ಯವು ಕೂಡ ಸಾಕಷ್ಟು ಶ್ರಮದಾಯಕವಾಗಿದ್ದರೂ ಇದು ಸ್ವಾಮೀಜಿಗೆ ಹೆಚ್ಚು ತೃಪ್ತಿಕರವಾಗಿತ್ತು. ಹಾಗೆ ನೋಡಿದರೆ, ಇಷ್ಟುದಿನವೂ ಅವರು ಸಾರ್ವಜನಿಕ ಕ್ಷೇತ್ರದಲ್ಲಿ ಆಕರ್ಷಣೆಯ ಕೇಂದ್ರವಾಗಿದ್ದರು. ಒಬ್ಬ ಭಾಷಣಕಾರನಾಗಿ, ಧರ್ಮಪ್ರಚಾರಕನಾಗಿ ಅವರು ಪ್ರಚಂಡ ಯಶಸ್ಸು ಗಳಿಸಿದ್ದರು. ಹೆಸರು ಕೀರ್ತಿಯಂತೂ ಅವರ ಬೆನ್ನಟ್ಟಿ ಬರುತ್ತಿತ್ತು. ಧನಸಂಗ್ರಹಣೆಯೂ ನಿರಾಶಾದಾಯಕವಾಗೇನೂ ಇರಲಿಲ್ಲ. ಆದರೂ ಅವರು ಇದನ್ನೆಲ್ಲ ತಿರ ಸ್ಕರಿಸಿ ದೂರ ಸರಿಯಬೇಕಾದರೆ, ಸಾಧಾರಣ ಜನರ ದೃಷ್ಟಿಕೋನಕ್ಕಿಂತ ಅವರದು ಎಷ್ಟು ವಿಭಿನ್ನ ವಾಗಿತ್ತೆಂಬುದನ್ನು ಗಮನಿಸಬೇಕು. ಮೊದಲನೆಯದಾಗಿ, ಅವರಿದ್ದಂತಹ ಸ್ಥಿತಿಯಲ್ಲಿ ಬೇರೆ ಯಾರಾದರೂ ಇದ್ದಿದ್ದರೆ, ಆ ಉಪನ್ಯಾಸ ಸಂಸ್ಥೆಯನ್ನು ಬಿಟ್ಟುಬಿಡುವ ಸಾಹಸ ಮಾಡಿ, ತನ್ಮೂಲಕ ಆರ್ಥಿಕ ನಷ್ಟಕ್ಕೆ ಗುರಿಯಾಗಲು ಸಿದ್ಧರಿರುತ್ತಿದ್ದರೆ? ಅಥವಾ ಗಣ್ಯರ ಒಡನಾಟವನ್ನು, ಶ್ರೀಮಂತರ ಆಮಂತ್ರಣಗಳನ್ನು ತ್ಯಜಿಸಿ, ಸರ್ವೇಸಾಮಾನ್ಯವಾದ ಪರಿಸರವನ್ನು ಆರಿಸಿಕೊಂಡು ಅಲ್ಲಿ ಧರ್ಮಬೋಧನೆ ಮಾಡುತ್ತ ಕುಳಿತುಕೊಳ್ಳುತ್ತಿದ್ದರೆ? ಹಿಂದೆ ಅವರಿಗೆ ಔತಣಕೂಟಗಳಿಗೆ ಬರುತ್ತಿದ್ದ ಆಹ್ವಾನಗಳಿಗೆ ಲೆಕ್ಕವೇ ಇರಲಿಲ್ಲ. ಅವೆಲ್ಲ ನಡೆಯುತ್ತಿದ್ದುದು ಅತ್ಯಂತ ವೈಭವೋ ಪೇತವಾದ ಸ್ಥಳಗಳಲ್ಲಿ. ತಮ್ಮ ಸ್ನೇಹಿತರ ಒತ್ತಾಯದಿಂದಾಗಿ, ಗತ್ಯಂತರವಿಲ್ಲದೆ ಅವುಗಳಿಗೆ ಹೋಗಲೇಬೇಕಾಗಿತ್ತು. ಇದರಿಂದಾಗಿ ಅವರ ಅಮೂಲ್ಯ ಸಮಯ ಹಾಳಾಗುತ್ತಿತ್ತಲ್ಲದೆ ದೇಹಾ ರೋಗ್ಯಕ್ಕೂ ತೊಂದರೆಯಾಗುತ್ತಿತ್ತು. ಆದ್ದರಿಂದ ಇವುಗಳಿಂದೆಲ್ಲ ತಪ್ಪಿಸಿಕೊಂಡು ಬಂದು, ಇಲ್ಲಿ ಅತ್ಯಂತ ಸರಳವಾದ ಜೀವನವನ್ನು ಪ್ರಾರಂಭಿಸಿದ್ದರು. ಈ ಮನೆಗೆ ಬಂದ ಮೇಲೂ ಅವರಿಗೆ ಬರುತ್ತಿದ್ದ ಆಹ್ವಾನಗಳೇನೂ ಕಡಿಮೆಯಾಗಲಿಲ್ಲ. ಆದರೆ ಸ್ವಾಮೀಜಿ ಅವುಗಳನ್ನೆಲ್ಲ ನಿರಾಕರಿಸಿ ತಮ್ಮದೇ ಆದ ಊಟದ ವ್ಯವಸ್ಥೆಯನ್ನು ಮಾಡಿಕೊಂಡರು. ಅವರು ತಮ್ಮ ಸ್ನೇಹಿತ ಲ್ಯಾಂಡ್ಸ್ ಬರ್ಗ್​ನೊಂದಿಗೆ (ಮುಂದೆ ಸ್ವಾಮಿ ಕೃಪಾನಂದ) ಅನ್ನವನ್ನೋ ಬಾರ್ಲಿ ಗಂಜಿಯನ್ನೋ ತಯಾ ರಿಸಿಕೊಂಡು, ಬೇಯಿಸಿದ ಸೊಪ್ಪಿನೊಂದಿಗೆ ಊಟಮಾಡಿ ಮುಗಿಸುತ್ತಿದ್ದರು. ಈ ದಿನಗಳ ಬಗ್ಗೆ ಅವರು ಶ್ರೀಮತಿ ಸಾರಾಬುಲ್ಲಳಿಗೆ ಬರೆಯುತ್ತಾರೆ, “ಅಮೆರಿಕದಲ್ಲಿ ಹಿಂದೆಂದಿಗಿಂತ ಈಗ ನಾನೊಬ್ಬ ನಿಜವಾದ ಸಂನ್ಯಾಸಿಯಂತಿರುವಂತೆ ನನಗೆ ಅನ್ನಿಸುತ್ತಿದೆ.”

ಭಗವಂತನೇ ತಮ್ಮ ಕೈಹಿಡಿದು ನಡೆಸುತ್ತಿರುವನೆಂಬ ದೃಢ ವಿಶ್ವಾಸದಿಂದ ಸ್ವಾಮೀಜಿ ತಮ್ಮ ಕಾರ್ಯವನ್ನು ಮುಂದುವರಿಸಿದರು. ಸಂಭಾಷಣೆಯ ಮೂಲಕ ಹಾಗೂ ತರಗತಿಗಳ ಮೂಲಕ ಮುಮುಕ್ಷುಗಳಾದ, ಜಿಜ್ಞಾಸುಗಳಾದ ಶ್ರದ್ಧಾವಂತ ವ್ಯಕ್ತಿಗಳಿಗೆ ಶಿಕ್ಷಣ ನೀಡುತ್ತ ಬಂದರು. ಧ್ಯಾನಾಭ್ಯಾಸದ ಮೂಲಕ ಚಂಚಲ ಮನಸ್ಸನ್ನು ಶಾಂತಗೊಳಿಸುವ ವಿಧಾನವನ್ನು ಕ್ರಮಬದ್ಧವಾಗಿ ಬೋಧಿಸಿದರು. ಎಷ್ಟೋ ಸಲ ಹೀಗೆ ಧ್ಯಾನದ ಕ್ರಮವನ್ನು ಬೋಧಿಸುವಾಗ ಸ್ವಾಮೀಜಿ ತಾವೇ ಧ್ಯಾನಮಗ್ನರಾಗಿಬಿಡುತ್ತಿದ್ದರು. ಅವರು ಕೆಲವೊಮ್ಮೆ ಎಷ್ಟು ಗಾಢವಾಗಿ ಧ್ಯಾನಲೀನರಾಗುತ್ತಿದ್ದ ರೆಂದರೆ, ಬಹಳ ಹೊತ್ತಾದರೂ ಆ ಸ್ಥಿತಿಯಿಂದ ಹಿಂದಿರುಗುತ್ತಿರಲಿಲ್ಲ. ಆದರೆ ಇತರರಿಗೆ– ಅದರಲ್ಲೂ ಆಗತಾನೆ ಧ್ಯಾನಜೀವನವನ್ನು ಪ್ರವೇಶಿಸುತ್ತಿರುವವರಿಗೆ–ಅಷ್ಟು ಹೊತ್ತು ಧ್ಯಾನ ಮಗ್ನರಾಗಿ ಕುಳಿತಿರಲು ಸಾಧ್ಯವೆ! ಆದ್ದರಿಂದ ಆ ವಿದ್ಯಾರ್ಥಿಗಳು ಸ್ವಾಮೀಜಿಯನ್ನು ಬಹಿರ್ಮುಖ ರಾಗಿಸಲು ಪ್ರಯತ್ನಿಸುತ್ತಿದ್ದರು. ಆದರೆ ಏನು ಮಾಡಿದರೂ ಅವರಿಗೆ ಎಚ್ಚರವಾಗುತ್ತಿರಲಿಲ್ಲ. ಈ ಶಿಷ್ಯರಿಗೆ ಕಾದುಕಾದು ಸಾಕಾಗಿ, ಕಡೆಗೆ ಒಬ್ಬೊಬ್ಬರಾಗಿ ಅಲ್ಲಿಂದೆದ್ದು ಹೊರಡುತ್ತಿದ್ದರು. ಸ್ವಾಮೀಜಿ ಬಹಿರ್ಮುಖರಾಗಿ ಮೈತಿಳಿದೆದ್ದಾಗ ತಮ್ಮ ಬಗ್ಗೆ ತಾವೇ ಅಸಮಾಧಾನ ತಾಳುತ್ತಿದ್ದರು. ಏಕೆಂದರೆ ಅವರಿಗೀಗ ತಾವು ಗುರುವಾಗಿ, ಶಿಕ್ಷಕನಾಗಿ ಇರಬೇಕಾದದ್ದು ಮುಖ್ಯವಾಗಿತ್ತೇ ಹೊರತು ಯೋಗಿಯಾಗಿರುವುದಲ್ಲ. ನಿಜ, ಧ್ಯಾನವೂ ಸೇರಿದಂತೆ, ಯಾವುದೇ ವಿಷಯವನ್ನು ಮಾತಾಡಿ ತೋರಿಸುವುದಕ್ಕಿಂತ ಮಾಡಿ ತೋರಿಸುವುದೇ ಹೆಚ್ಚು ಪರಿಣಾಮಕಾರಿ. ಆದರೂ ಧ್ಯಾನದ ಪ್ರಕ್ರಿಯೆ ಗಳನ್ನು ಹಾಗೂ ಅದಕ್ಕೆ ಸಂಬಂಧಪಟ್ಟ ಎಲ್ಲ ವಿವರಗಳನ್ನು ಮಾತಿನಿಂದ ತಾನೆ ವಿವರಿಸಬೇಕಾ ದದ್ದು? ಆದ್ದರಿಂದ ಈ ರೀತಿ ತಾವು ಧ್ಯಾನಸ್ಥಿತಿಗೇರದಂತೆ ಸ್ವಾಮೀಜಿ ತುಂಬ ಎಚ್ಚರಿಕೆ ವಹಿಸಿದರು. ಕೆಲವೊಮ್ಮೆ ತಮ್ಮ ಒಬ್ಬಿಬ್ಬರು ಶಿಷ್ಯರೊಂದಿಗೆ ಧ್ಯಾನಕ್ಕೆಂದೇ ಕುಳಿತಾಗಲೂ, ತಮ್ಮನ್ನು ಅತ್ಯುನ್ನತ ಧ್ಯಾನಾವಸ್ಥೆಯಿಂದ–ಸಮಾಧಿಯ ಸ್ಥಿತಿಯಿಂದ–ಇಳಿಸಲು ತಮ್ಮ ಕಿವಿ ಯಲ್ಲಿ ಯಾವ ಮಂತ್ರವನ್ನು ಹೇಗೆ ಉಚ್ಚರಿಸಬೇಕೆಂಬುದನ್ನು ತಮ್ಮ ಶಿಷ್ಯರಿಗೆ ತಿಳಿಸಿರುತ್ತಿದ್ದರು. ಇತರರಿಗೆ ಧ್ಯಾನಸ್ಥಿತಿಯನ್ನು ಮುಟ್ಟುವುದೇ ಅತಿ ದೊಡ್ಡ ಸಮಸ್ಯೆಯಾದರೆ ಸ್ವಾಮೀಜಿಗೆ ಆ ಸ್ಥಿತಿಯಿಂದ ಇಳಿದುಬರುವುದೇ ಸಮಸ್ಯೆ! ಈ ಸಮಯದಲ್ಲಿ ಸ್ವಾಮೀಜಿಯ ವ್ಯಕ್ತಿತ್ವದಲ್ಲಿ ಆಧ್ಯಾ ತ್ಮಿಕತೆಯು ವಿಶೇಷವಾಗಿ ಪ್ರಕಾಶಮಾನವಾಗಿತ್ತು. ಅಂದು ದಕ್ಷಿಣೇಶ್ವರದಲ್ಲಿ ಶ್ರೀರಾಮಕೃಷ್ಣರ ಸುತ್ತ ಅವರ ಶಿಷ್ಯರು ಯಾವ ಆಧ್ಯಾತ್ಮಿಕ ವಾತಾವರಣವನ್ನು ಅನುಭವಿಸುತ್ತಿದ್ದರೋ ಅದೇ ಬಗೆಯ ಆನಂದಮಯ ವಾತಾವರಣವನ್ನು ಇಲ್ಲಿ ಸ್ವಾಮೀಜಿಯ ಸುತ್ತ ಕಾಣಬಹುದಾಗಿತ್ತು. ಅವರ ತರಗತಿಗಳಿಗೆ ಬರುತ್ತಿದ್ದವರೆಲ್ಲ ಅಲ್ಲೊಂದು ಮಂಗಳಕರ ವಾತಾವರಣವನ್ನು, ಅಪೂರ್ವ ಶಾಂತಿಯನ್ನು ಅನುಭವಿಸುತ್ತಿದ್ದರು.

ಈ ಸಮಯದಲ್ಲಿ ಸ್ವಾಮೀಜಿಯವರನ್ನು ಸಂದರ್ಶಿಸಿದ ಡಾ ॥ ಎಡ್ಗರ್ ಸಿ. ಬಿಯಾಲ್ ಎಂಬವರು ನ್ಯೂಯಾರ್ಕಿನ \eng{Phrenological Journal (}ಮುಖ ಸಾಮುದ್ರಿಕ ಶಾಸ್ತ್ರ ಪತ್ರಿಕೆ)ಯಲ್ಲಿ ಸ್ವಾಮೀಜಿಯ ಕುರಿತಾಗಿ ಕುತೂಹಲಕರ ಲೇಖನವೊಂದನ್ನು ಬರೆದರು. ಅದರ ಒಂದು ಭಾಗ ಹೀಗಿದೆ:

“ಸ್ವಾಮಿ ವಿವೇಕಾನಂದರು ಹಲವಾರು ಅಂಶಗಳಲ್ಲಿ ಅವರ ಜನಾಂಗದ ಅತ್ಯುತ್ತಮ ಮಾದರಿ ಎನ್ನಬಹುದು. ಅವರು ಐದು ಅಡಿ ಎಂಟೂವರೆ ಅಂಗುಲ ಎತ್ತರವಿದ್ದಾರೆ. ಅವರ ತೂಕ ನೂರ ಎಪ್ಪತ್ತು ಪೌಂಡುಗಳು. ತಲೆ ಸುತ್ತಳತೆ ಇಪ್ಪತ್ತೊಂದೂ ಮುಕ್ಕಾಲು ಅಂಗುಲ. ತಲೆಯ ಮೇಲಿಂದ ಕಿವಿಯಿಂದ ಕಿವಿಗೆ ಹದಿನಾಲ್ಕು ಅಂಗುಲ. ಶರೀರ ಹಾಗೂ ಮೆದುಳು ಇವೆರಡೂ ಶ್ರೇಷ್ಠ ಪರಿಮಾಣದಲ್ಲಿವೆ. ಅತ್ಯಂತ ಸುಕೋಮಲವಾದ ಅವರ ಪ್ರವೃತ್ತಿಯು ದಾಂಪತ್ಯಭಾವ ಗಳೊಂದಿಗೆ ಹೊಂದಿಕೊಳ್ಳಲು ಸಾಧ್ಯವೇ ಇಲ್ಲ. ಅಲ್ಲದೆ ಅವರೇ ಹೇಳುತ್ತಾರೆ. ಯಾವ ಸ್ತ್ರೀಯ ವಿಷಯದಲ್ಲೂ ತಾವು ಎಂದೂ ಕಿಂಚಿತ್ತಾದರೂ ಕಾಮಭಾವನೆಯನ್ನು ತಾಳಿಯೇ ಇಲ್ಲ ಎಂದು. ಅವರು ಯುದ್ಧವಿರೋಧಿಗಳಾಗಿದ್ದು ಅತ್ಯಂತ ಸಭ್ಯವಾದ ಧರ್ಮವನ್ನು ಬೋಧಿಸುತ್ತಾರಾದ್ದ ರಿಂದ, ಆಕ್ರಮಣಕಾರೀ ಹಾಗೂ ವಿನಾಶಕಾರೀ ಪ್ರವೃತ್ತಿಗಳ ನೆಲೆಯಾದ ಕಿವಿಗಳ ಪ್ರದೇಶದಲ್ಲಿ ಅವರ ತಲೆ ಭಾಗ ಸ್ವಲ್ಪ ಸಂಕುಚಿತಗೊಂಡಿರಬೇಕೆಂದು ನಾವು ಊಹಿಸಬಹುದು. ಹೌದು, ಅದು ವಾಸ್ತವವಾಗಿಯೂ ಹಾಗೆಯೇ ಇದೆ. ಇದೇ ರೀತಿ, ಮುಚ್ಚುಮರೆಯ ಹಾಗೂ ಕೂಡಿಹಾಕುವ (ಸಂಗ್ರಹಣೆಯ) ಪ್ರವೃತ್ತಿಗಳ ಭಾಗವಾದ ಕಿವಿಯ ಸ್ವಲ್ಪ ಮೇಲ್ಭಾಗವೂ ಸಂಕುಚಿತವಾಗಿದೆ. ಹಣಕಾಸು ಹಾಗೂ ಒಡೆತನದ ವಿಷಯವನ್ನು ಅವರ ಬಳಿ ಪ್ರಸ್ತಾಪಿಸಿದರೆ, ತಮಗಾವ ಆಸ್ತಿಯೂ ಇಲ್ಲ ಮತ್ತು ಆ ಬಗ್ಗೆ ತಲೆ ಕೆಡಿಸಿಕೊಳ್ಳಲೂ ತಮಗಿಷ್ಟವಿಲ್ಲವೆಂದು ಹೇಳಿ, ಅವರು ಆ ವಿಷಯವನ್ನು ಒಮ್ಮೆಗೇ ತಳ್ಳಿಹಾಕುತ್ತಾರೆ. ಈ ಭಾವನೆಗಳೆಲ್ಲ ಅಮೆರಿಕನ್ನರಿಗೆ ಸ್ವಲ್ಪ ವಿಲಕ್ಷಣ ವೆನ್ನಿಸಬಹುದು. ಆದರೆ ವಿವೇಕಾನಂದರ ಮುಖವು ಅಮೆರಿಕದ ಹಲವಾರು ಕೋಟ್ಯಧೀಶರ ಮುಖ ದಲ್ಲಿ ಕಾಣುವುದಕ್ಕಿಂತ ಹೆಚ್ಚಿನ ಸಂತೃಪ್ತಿಭಾವವನ್ನು ಪ್ರತಿಬಿಂಬಿಸುತ್ತದೆಯೆಂಬುದನ್ನಂತೂ ಒಪ್ಪಿಕೊಳ್ಳಬೇಕು. ದಾರ್ಢ್ಯ ಹಾಗೂ ಆತ್ಮನಿಷ್ಠೆ ಅವರಲ್ಲಿ ಸಂಪೂರ್ಣವಾಗಿ ಅಭಿವೃದ್ಧಿಗೊಂಡಿವೆ. ದಯಾಪರತೆ ಅವರಲ್ಲಿ ಅತ್ಯಂತ ಸ್ಪಷ್ಟವಾಗಿ ವ್ಯಕ್ತವಾಗುತ್ತದೆ. ಅವರ ಗಂಡಸ್ಥಳ (ಕಿವಿ ಹಾಗೂ ಹಣೆಯ ನಡುವಿನ ಚಪ್ಪಟೆಭಾಗ) ವಿಸ್ತಾರವಾಗಿದ್ದು, ಸಂಗೀತದ ಲಕ್ಷಣಗಳನ್ನು ಸ್ಪಷ್ಟವಾಗಿ ಪ್ರದರ್ಶಿಸುತ್ತದೆ. ಅವರ ಎದ್ದುತೋರುವ ನಯನಗಳು ಅತ್ಯುತ್ತಮ ಜ್ಞಾಪಕಶಕ್ತಿಯ ಪ್ರತೀಕ ವಾಗಿದ್ದು, ಅವರ ಉಪನ್ಯಾಸಗಳಲ್ಲಿ ಕಂಡುಬರುವ ವಾಗ್ವೈಖರಿಯನ್ನು ಬಹಳಮಟ್ಟಿಗೆ ವಿವರಿಸು ತ್ತವೆ. ಹಣೆಯ ಮೇಲ್ಭಾಗ ಚೆನ್ನಾಗಿ ಬೆಳವಣಿಗೆಗೊಂಡಿದ್ದು, ಅದರಲ್ಲಿ ಅವರ ವ್ಯಕ್ತಿತ್ವದ ಮಾಧುರ್ಯ ಹಾಗೂ ಪರಿಪೂರ್ಣ ಮನುಷ್ಯತ್ವವನ್ನು ಗುರುತಿಸಬಹುದು. ಒಟ್ಟಿನಲ್ಲಿ ಅವರ ಲಕ್ಷಣಗಳನ್ನೆಲ್ಲ ಸಂಗ್ರಹವಾಗಿ ಹೇಳುವುದಾದರೆ, ದಯೆ, ಸಹಾನುಭೂತಿ, ತಾತ್ವಿಕ ಬುದ್ಧಿಮತ್ತೆ ಹಾಗೂ ಉನ್ನತ ಶಿಕ್ಷಣ ಕ್ಷೇತ್ರದಲ್ಲಿ ಯಶಸ್ಸನ್ನು ಸಾಧಿಸಬೇಕೆಂಬ ಆಕಾಂಕ್ಷೆ–ಇವು ಅವರ ಪ್ರಮುಖ ಲಕ್ಷಣಗಳೆಂಬುದನ್ನು ಗಮನಿಸಬಹುದು. ವಿವೇಕಾನಂದರು ಕಲ್ಕತ್ತ ವಿಶ್ವವಿದ್ಯಾಲಯದ ಪದವೀಧರರಾಗಿದ್ದು, ಇಂಗ್ಲಿಷನ್ನು ಹೆಚ್ಚು ಕಡಿಮೆ ಇಂಗ್ಲೆಂಡಿನವರಷ್ಟೇ ಸರಿಯಾಗಿ ಮಾತನಾಡು ತ್ತಾರೆ. ಅವರು ಜಾಗತಿಕ ಮೇಳದಲ್ಲಿ ತೋರಿಸಿದಂತಹ ಅದ್ಭುತ ಹೃದಯವೈಶಾಲ್ಯವನ್ನು ಕೇವಲ ಮುಂದುವರಿಸಿಕೊಂಡು ಹೋದರೂ ಸಾಕು, ನಮ್ಮ ನಡುವೆ ಅವರ ಕಾರ್ಯೋದ್ದೇಶ ಖಂಡಿತ ವಾಗಿ ಅತ್ಯಂತ ಯಶಸ್ವಿಯಾಗುತ್ತದೆ.”

ನ್ಯೂಯಾರ್ಕಿನಲ್ಲಿ ಪ್ರಾರಂಭಿಸಿದ್ದ ತರಗತಿಗಳಲ್ಲಿ ಸ್ವಾಮೀಜಿ ರಾಜಯೋಗ-ಜ್ಞಾನಯೋಗ ಗಳನ್ನು ವಿಸ್ತಾರವಾಗಿ ಬೋಧಿಸಿದರು. ರಾಜಯೋಗದ ಮೂಲಕ ಅವರು ತಮ್ಮ ಶಿಷ್ಯರಿಗೆ ಅನುಷ್ಠಾನ ಅಧ್ಯಾತ್ಮವನ್ನು–ಎಂದರೆ, ಇಂದ್ರಿಯಗಳ ಪ್ರವೃತ್ತಿಯನ್ನು ಹಿಡಿತದಲ್ಲಿಟ್ಟುಕೊಳ್ಳು ವುದು ಹೇಗೆ, ಮನಸ್ಸನ್ನು ಸ್ಥಿರಗೊಳಿಸುವುದು ಹೇಗೆ, ಎಂಬೀ ವಿಷಯಗಳನ್ನು ಬೋಧಿಸಿದರು. ಮನಸ್ಸನ್ನು ಏಕಾಗ್ರಗೊಳಿಸುವುದು ಹೇಗೆಂಬುದರ ಬಗ್ಗೆ ಶಿಕ್ಷಣ ನೀಡುವುದಕ್ಕಾಗಿಯೇ ಅವರು ನಿಯತವಾಗಿ ತರಗತಿಗಳನ್ನು ತೆಗೆದುಕೊಂಡರು. ಏಕೆಂದರೆ, ಏಕಾಗ್ರ ಮನಸ್ಸಿನಿಂದ ಮಾಡುವ ಧ್ಯಾನವೇ ಆಧ್ಯಾತ್ಮಿಕ ಜೀವನದ ಕೀಲಿಕೈ ಎಂಬುದು ಸ್ವಾಮೀಜಿಯ ಸ್ಪಷ್ಟ ಅಭಿಪ್ರಾಯವಾಗಿತ್ತು. ಸ್ವತಃ ಸ್ವಾಮೀಜಿಯೇ ನೆಲದ ಮೇಲೆ ಯೋಗಮುದ್ರೆಯಲ್ಲಿ ಗಂಟೆಗಟ್ಟಲೆ ಕಾಲ ಧ್ಯಾನಮಗ್ನರಾಗಿ ಕುಳಿತುಬಿಡುತ್ತಿದ್ದರು. ಧ್ಯಾನದ ಬಗೆಯನ್ನು ಬೋಧಿಸಲು ಅವರಿಗಿಂತ ಉತ್ತಮರಾದ, ಸಮರ್ಥ ರಾದ ಗುರು ಯಾರಿದ್ದಾರು? ಅಲ್ಲದೆ ಅವರು ಶ್ರೀರಾಮಕೃಷ್ಣರ ಆಧ್ಯಾತ್ಮಿಕ ಗರಡಿಯಲ್ಲಿ ಪಳಗಿದ ಮಹಾ ಶಕ್ತಿವಂತರು; ಶ್ರೀರಾಮಕೃಷ್ಣರ ಮಾರ್ಗದರ್ಶನದಲ್ಲಿ ವಿವಿಧ ಆಧ್ಯಾತ್ಮಿಕ ಸಾಧನೆಗಳನ್ನು ಕೈಗೊಂಡು ಸಿದ್ಧಿಸಿಕೊಂಡವರು; ಬಗೆಬಗೆಯ ಯೋಗಾಭ್ಯಾಸಗಳನ್ನು ಮಾಡಿದವರು. ಹೀಗೆ ಅವರ ಜೀವನವೇ ಹೊತ್ತಿ ಉರಿಯುವ ಆಧ್ಯಾತ್ಮಿಕ ಜ್ವಾಲೆಯಾಗಿತ್ತು. ಪ್ರತಿಯೊಬ್ಬನ ಮನೋಭಾವ ವನ್ನು, ಸಾಮರ್ಥ್ಯವನ್ನು ನಿಖರವಾಗಿ ಅರಿಯಬಲ್ಲವರಾಗಿದ್ದ ಅವರು, ಪ್ರತಿಯೊಬ್ಬನಿಗೂ ಅವನವನದೇ ಆದ ವಿಶಿಷ್ಟ ಆದರ್ಶಗಳನ್ನು ಮನಗಾಣಿಸಿಕೊಟ್ಟು, ಅದಕ್ಕನುಗುಣವಾದ ಧ್ಯಾನದ ಕುರಿತಾಗಿ ಶಿಕ್ಷಣ ನೀಡುತ್ತಿದ್ದರು. ಹೀಗೆ ಸ್ವಾಮೀಜಿಯ ಕ್ರಮಬದ್ಧ, ಸ್ಫೂರ್ತಿಯುತ ಮಾರ್ಗ ದರ್ಶನದಲ್ಲಿ ಅವರ ಅಮೆರಿಕನ್ ಶಿಷ್ಯರು, ದೇಹದ ಹಾಗೂ ಮನಸ್ಸಿನ ಸಮತೋಲವನ್ನು ತಂದುಕೊಡಬಲ್ಲಂತಹ ಶಾರೀರಿಕ ಹಾಗೂ ಆಧ್ಯಾತ್ಮಿಕ ಸಾಧನೆಗಳಲ್ಲಿ ನಿರತರಾದರು. ಧರ್ಮವು ಕೇವಲ ‘ನಂಬುವದ’ರಲ್ಲಿಲ್ಲ. ‘ಆಗುವುದ’ರಲ್ಲಿದೆ; ಅರ್ಥಾತ್, ಧರ್ಮವನ್ನು ಕೇವಲ ನಂಬಿದರೆ ಸಾಲದು–ಧರ್ಮಸ್ವರೂಪಿಗಳೇ ಆಗಬೇಕು, ಎನ್ನುವುದನ್ನು ಆ ಶಿಷ್ಯರು ಅರಿತರು.

ರಾಜಯೋಗದ ಅಭ್ಯಾಸದಲ್ಲಿ ಸರಿಯಾದ ಮುನ್ನೆಚ್ಚರಿಕೆಗಳನ್ನು ವಹಿಸದಿದ್ದರೆ, ಅದು ಅಪಾಯ ಕರವಾಗಿ ಪರಿಣಮಿಸುವ ಸಾಧ್ಯತೆಯಿದೆ. ಬುದ್ಧಿಭ್ರಮಣೆಯೂ ಸೇರಿದಂತೆ, ಅನೇಕ ಶಾರೀರಿಕ ಹಾಗೂ ಮಾನಸಿಕ ಅಸ್ವಸ್ಥತೆಗಳಿಗೆ ಅದು ಕಾರಣವಾಗಬಹುದು. ಆದ್ದರಿಂದ ಸ್ವಾಮೀಜಿ ಈ ಬಗ್ಗೆ ತಮ್ಮ ಶಿಷ್ಯರಿಗೆ ತಿಳಿಯಹೇಳಿ, ಅವರಿಗೆ ಪರಿಶುದ್ಧ ಜೀವನದ ಹಾಗೂ ಸರಳ-ಸಾತ್ವಿಕ ಆಹಾರದ ನಿಯಮಗಳನ್ನು ವಿಧಿಸಿದ್ದರು. ಹೀಗಾಗಿ, ಅವರ ತರಗತಿ ಕ್ರಮೇಣ ಒಂದು ಸಂನ್ಯಾಸಿಗಳ ಗುಂಪಿ ನಂತೆ ಕಂಡು ಬರುತ್ತಿದ್ದುದು ಆಶ್ಚರ್ಯದ ಸಂಗತಿಯಲ್ಲ. ಸ್ವಾಮೀಜಿ ತಮ್ಮ ಶಿಷ್ಯರಿಗೆ ನೀಡಿದ ಮತ್ತೊಂದು ಎಚ್ಚರಿಕೆಯೆಂದರೆ, ಅತೀಂದ್ರಿಯ ಶಕ್ತಿಗಳನ್ನು ಕುರಿತಾದದ್ದು. ರಾಜಯೋಗದ ಸಾಧಕರಿಗೆ ಈ ಶಕ್ತಿಗಳ ಹಂಬಲವು ಅತಿ ದೊಡ್ಡ ಅಪಾಯ. ಈ ಸಿದ್ಧಿಗಳು ನಿಜವಾದ ಆಧ್ಯಾತ್ಮಿಕ ಪ್ರಗತಿಗೆ ಪ್ರತಿಬಂಧಕಗಳು ಎಂದು ಸ್ವಾಮೀಜಿ ಎಚ್ಚರಿಕೆ ನೀಡಿ, ಸಾಧಕನು ದಾರಿ ತಪ್ಪಿದ್ದಾನೆ ಎಂಬುದರ ಗುರುತು ಈ ಸಿದ್ಧಿಗಳು ಎಂದು ಮನದಟ್ಟು ಮಾಡಿಸಿದರು.

ಆಧ್ಯಾತ್ಮಿಕತೆಯನ್ನು ಬಲಿತೆತ್ತು ಕೇವಲ ಪವಾಡಶಕ್ತಿಗಳ ಸಂಪಾದನೆ ಮಾಡುವುದನ್ನೇ ಬೋಧಿ ಸುವ ವ್ಯಕ್ತಿಗಳನ್ನೂ ಪಂಥಗಳನ್ನೂ ಸ್ವಾಮೀಜಿ ನಿಷ್ಠುರವಾಗಿ ಖಂಡಿಸುತ್ತಿದ್ದರು. ತಮ್ಮ ಗುರು ಶ್ರೀರಾಮಕೃಷ್ಣರು ಮತ್ತೆ ಮತ್ತೆ ಒತ್ತಿಹೇಳುತ್ತಿದ್ದ ಮಾತನ್ನೇ ಅವರೂ ಹೇಳುತ್ತಿದ್ದರು,“ಕೇವಲ ಭಗವಂತನನ್ನು ಮಾತ್ರ ಅರಸಿರಿ”ಎಂದು.

ಸ್ವಾಮೀಜಿ ಸ್ವತಃ ತಾವೇ ಒಬ್ಬ ಪ್ರವಾದಿಯಾಗಿದ್ದರು; ಅವರ ಮಾತುಗಳೇ ಸ್ವಯಂಪೂರ್ಣ ಶಾಸ್ತ್ರಗಳಾಗಿದ್ದುವು. ಅವರ ಮಾತಿಗೆ ಯಾವ ಇತರ ಗ್ರಂಥದ ಪ್ರಮಾಣವೂ ಬೇಕಿರಲಿಲ್ಲ. ಆದರೂ ಅವರು ತಮ್ಮ ಸನಾತನ ಸಂಪ್ರದಾಯವನ್ನು ಬಿಟ್ಟುಕೊಡಲಿಲ್ಲ. ತಮ್ಮ ಪ್ರತಿಯೊಂದು ಬೋಧನೆ ಯನ್ನೂ ಅವರು ಅಧಿಕೃತ ಶಾಸ್ತ್ರವಾಕ್ಯಗಳ ಬೆಳಕಿನಲ್ಲೇ ಎತ್ತಿಹಿಡಿಯುತ್ತಿದ್ದರು. ಮತ್ತು ಉಪ ನಿಷತ್ತುಗಳೇ ಮೊದಲಾದ ಕೆಲವು ಹಿಂದೂ ಶಾಸ್ತ್ರಗ್ರಂಥಗಳನ್ನು ನೇರವಾಗಿ ಬೋಧಿಸಬೇಕೆಂದು ಅವರು ಬಯಸಿದರು. ಆದರೆ ಅಮೆರಿಕದಲ್ಲಿ ಅವರು ತರಗತಿಗಳನ್ನು ತೆಗೆದುಕೊಳ್ಳುತ್ತಿದ್ದ ಸಮಯದಲ್ಲಿ ಅವರಿಗೆ ಬೇಕಾದಂತಹ ಸಂಸ್ಕೃತ ಗ್ರಂಥಗಳು ಲಭ್ಯವಿರಲಿಲ್ಲ. ಈ ಕೆಲವು ಗ್ರಂಥ ಗಳು ಅವರಿಗೆ ತುರ್ತಾಗಿ ಬೇಕಾಗಿದ್ದರಿಂದ ಭಾರತದಿಂದ ಅವುಗಳನ್ನು ತರಿಸಿಕೊಂಡರು. ಇವು ಅವರ ಕೈಸೇರುವಲ್ಲಿ ಬಹಳ ತಡವಾಯಿತಾದರೂ ತುಂಬ ಉಪಯುಕ್ತವಾಗಿ ಪರಿಣಮಿಸಿದುವು.

ತಾವು ಬೋಧಿಸುತ್ತಿದ್ದ ರಾಜಯೋಗ-ಜ್ಞಾನಯೋಗಗಳನ್ನು ಸ್ವಾಮೀಜಿ ಆಮೂಲಾಗ್ರವಾಗಿ ಅರಿತುಕೊಂಡಿದ್ದರಲ್ಲದೆ ಅವುಗಳನ್ನು ಸ್ವತಃ ಅಭ್ಯಸಿಸಿ ಸಾಕ್ಷಾತ್ಕರಿಸಿಕೊಂಡಿದ್ದರು. ಮತ್ತು ಆ ವಿಷಯವನ್ನು ಅವರೇ ತಮ್ಮ ಶಿಷ್ಯರಿಗೆ ಹೇಳಿದ್ದರು. ಅಲ್ಲದೆ ರಾಜಯೋಗವನ್ನು ಅವರು ಪಾಶ್ಚಾತ್ಯ ಮನೋವಿಜ್ಞಾನ ಹಾಗೂ ನರಮಂಡಲ ಶಾಸ್ತ್ರಗಳ ಹಿನ್ನೆಲೆಯಲ್ಲಿ ಆಳವಾಗಿ ಅಧ್ಯಯನ ಮಾಡಿದ್ದರು. ಸಾಧಕರಾಗಿ ಸ್ವಾಮೀಜಿಯ ಬಳಿಗೆ ಬಂದಿದ್ದ ಅವರ ಶಿಷ್ಯರು ಯೋಗದ ಆಧ್ಯಾತ್ಮಿಕ ಅಂಶಗಳಲ್ಲಿ ಹಾಗೂ ಅನುಷ್ಠಾನ ಯೋಗ್ಯವಾದ ವಿವರಗಳಲ್ಲಿ ಆಸಕ್ತರಾಗಿದ್ದರು. ಆದರೆ ಯೋಗಾಭ್ಯಾಸದಲ್ಲಿ ನರಮಂಡಲ, ಮೆದುಳು, ಮನಸ್ಸು–ಇವುಗಳದು ಪ್ರಮುಖ ಪಾತ್ರ. ಆದ್ದರಿಂದ ಯೋಗದ ರಹಸ್ಯಗಳನ್ನು ವಿವರಿಸುವಾಗ ಸ್ವಾಮೀಜಿ, ನರಮಂಡಲದ ರಚನೆ, ಮೆದುಳು-ನರಮಂಡಲ-ಮನಸ್ಸುಗಳ ಪರಸ್ಪರ ಸಂಬಂಧವೇ ಮೊದಲಾದ ವಿಷಯಗಳನ್ನು ಪ್ರಸ್ತಾಪಿಸ ಬೇಕಾಗುತ್ತಿತ್ತು. ಈ ವಿವರಣೆಗಳೆಲ್ಲ ಎಷ್ಟು ಅದ್ಭುತವಾಗಿರುತ್ತಿದ್ದುವೆಂದರೆ ಈ ಬಗ್ಗೆ ತಿಳಿದ ಹಲವಾರು ಹೆಸರಾಂತ ವೈದ್ಯರು, ಮನಃಶಾಸ್ತ್ರಜ್ಞರು ಅವುಗಳಿಂದ ಆಕರ್ಷಿತರಾದರು. ಅಲ್ಲದೆ ಅನೇಕರು ಆ ವಿವರಣೆಗಳು ಸರಿಯೆಂದು ಸಮರ್ಥಿಸಿದರು. ಸ್ವಾಮೀಜಿಯ ಅನೇಕ ಹೇಳಿಕೆಗಳ ಬಗ್ಗೆ ನೂತನ ಶೋಧನೆಗಳು ನಡೆಯಬೇಕೆಂದು ಕೆಲವರು ಅಭಿಪ್ರಾಯಪಟ್ಟರು. ಕ್ರಮಬದ್ಧವಾದ ದೀರ್ಘಧ್ಯಾನದಿಂದ ಮನಸ್ಸಿನ ಹಲವಾರು ಸುಪ್ತಶಕ್ತಿಗಳು ಜಾಗೃತವಾಗುತ್ತವೆ ಹಾಗೂ ಅತೀಂದ್ರಿಯ ಅನುಭವಗಳಾಗುತ್ತವೆ ಎಂಬ ಸ್ವಾಮೀಜಿಯ ಮಾತು ಸುಪ್ರಸಿದ್ಧ ಮನೋವಿಜ್ಞಾನಿಗಳ–ಮುಖ್ಯವಾಗಿ ಹಾರ್ವರ್ಡ್ ವಿಶ್ವವಿದ್ಯಾನಿಲಯದ ಪ್ರೊ ॥ ವಿಲಿಯಂ ಜೇಮ್ಸ್​ರವರ ಆಸಕ್ತಿಯನ್ನು ಕೆರಳಿಸಿತು. ಅಲ್ಲಿಯವರೆಗೆ ಇಂದ್ರಿಯಾತೀತವಾದ ಎಲ್ಲ ಅನುಭವ ಗಳೂ ಪವಾಡಗಳೆಂದು ಪರಿಗಣಿಸಲ್ಪಡುತ್ತಿದ್ದುವು ಎಂಬುದನ್ನು ಇಲ್ಲಿ ಗಮನಿಸಬೇಕು.

ನ್ಯೂಯಾರ್ಕಿನಲ್ಲಿ ಸ್ವಾಮೀಜಿ ತಮ್ಮ ನಿಯೋಜಿತ ತರಗತಿಗಳನ್ನು ತೆಗೆದುಕೊಳ್ಳುವುದಲ್ಲದೆ ಆಗಾಗ ಇತರೆಡೆಗಳಲ್ಲಿ ಕೆಲವು ಗುಂಪುಗಳನ್ನು ಉದ್ದೇಶಿಸಿ ಮಾತನಾಡುತ್ತಿದ್ದರು. ಅಲ್ಲಿನ ಒಬ್ಬ ಭಾರೀ ಶ್ರೀಮಂತನ ಮಗಳಾದ ಮಿಸ್ ಕಾರ್ಬಿನ್ ಎಂಬಾಕೆ, ತನ್ನ ಮನೆಯಲ್ಲಿ ಆಹ್ವಾನಿತರಾದ ಕೆಲವರಿಗಾಗಿ ತರಗತಿಗಳನ್ನು ತೆಗೆದುಕೊಳ್ಳಬೇಕೆಂದು ಸ್ವಾಮೀಜಿಯನ್ನು ಕೇಳಿಕೊಂಡಳು. ಮಿಸ್ ಕಾರ್ಬಿನ್ ಬುದ್ಧಿವಂತೆ ಹಾಗೂ ಆಧ್ಯಾತ್ಮಿಕ ಮನೋವೃತ್ತಿಯವಳು ಎಂದು ಕಂಡುಕೊಂಡಿದ್ದ ಸ್ವಾಮೀಜಿ, ಅವಳ ಮೇಲಿನ ವಿಶ್ವಾಸದಿಂದ ತರಗತಿಗಳನ್ನು ತೆಗೆದುಕೊಳ್ಳಲು ಒಪ್ಪಿದರು. ಸುಮಾರು ಹನ್ನೆರಡು ಜನರ ತರಗತಿ ಅದು. ಭಾರೀ ಶ್ರೀಮಂತಳಾದ ಕಾರ್ಬಿನ್ನಳ ಈ ಆತಿ ಥೇಯರೂ ಅವಳ ವರ್ಗಕ್ಕೆ ಸೇರಿದವರೇ. ಆದರೆ ಸ್ವಾಮೀಜಿಗೆ, ತಮ್ಮ ವಿದ್ಯಾರ್ಥಿಗಳು ಬಡವರೋ ಶ್ರೀಮಂತರೋ ಎಂಬುದರ ಗಣನೆಯೇ ಇರಲಿಲ್ಲ. ಇವರ ಪೈಕಿ ಶ್ರದ್ಧಾವಂತರೂ, ಆಧ್ಯಾತ್ಮಿಕ ಪ್ರವೃತ್ತಿಯವರೂ ಇದ್ದಾರೆಯೆ, ಎಂಬುದಷ್ಟೇ ಅವರ ಆಲೋಚನೆ. ಭಾನುವಾರ ಗಳಂದು ನಡೆಯುತ್ತಿದ್ದ ಈ ತರಗತಿ ಒಂದು ತಿಂಗಳ ಕಾಲ ಹೇಗೋ ಸಾಗಿತು. ಆದರೆ ಈ ಶ್ರೀಮಂತ ವಿದ್ಯಾರ್ಥಿಗಳಿಗಿಂತ, ತಮ್ಮ ಬಡಕುಟೀರಕ್ಕೇ ಬರುವ ವಿದ್ಯಾರ್ಥಿಗಳು ಎಷ್ಟೋ ಮೇಲು ಎಂದು ಸ್ವಾಮೀಜಿಗೆ ಅನ್ನಿಸಿತು. ಆದ್ದರಿಂದ ಅವರು ಈ ತರಗತಿಗಳನ್ನು ಅಲ್ಲಿಗೆ ಅಂತ್ಯಗೊಳಿಸಿ ದರು. ಈ ಬಗ್ಗೆ ಅವರು ಶ್ರೀಮತಿ ಸಾರಾಬುಲ್​ಗೆ ಪತ್ರವೊಂದರಲ್ಲಿ ಬರೆದರು–“ನಾನು ಕಳೆದ ವಾರ ಮಿಸ್ ಕಾರ್ಬಿನ್ನಳ ಬಳಿಗೆ ಹೋಗಿ, ಇನ್ನು ಈ ತರಗತಿಗಳನ್ನು ಮುಂದುವರಿಸಲು ನನಗೆ ಸಾಧ್ಯವಿಲ್ಲ ಎಂದು ಹೇಳಿಬಿಟ್ಟೆ. ಈ ಪ್ರಪಂಚದ ಇತಿಹಾಸದಲ್ಲಿ ಯಾವುದಾದರೂ ಒಂದು ದೊಡ್ಡ ಕೆಲಸ ಶ್ರೀಮಂತರಿಂದ ಸಾಧ್ಯವಾಗಿದೆಯೆ? ಎಂದೆಂದಿಗೂ ಅದು ಸಾಧ್ಯವಾಗುವುದು ಬುದ್ಧಿ- ಹೃದಯಗಳಿಂದಲೇ ಹೊರತು ಹಣದ ಥೈಲಿಯಿಂದಲ್ಲ.” ಇಲ್ಲಿ ಸ್ವಾಮೀಜಿ ಹಣದ ಪ್ರಾಮುಖ್ಯತೆ ಯನ್ನು ಅಲ್ಲಗಳೆಯುತ್ತಿಲ್ಲ; ಬದಲಾಗಿ ಅದೇ ಎಲ್ಲಕ್ಕಿಂತ ಮಿಗಿಲಾದದ್ದಲ್ಲ ಎಂಬುದನ್ನು ತೋರಿಸಿಕೊಡುತ್ತಿದ್ದಾರೆ. ಅವರು ಅಮೆರಿಕೆಗೆ ಬಂದ ಒಂದು ಮುಖ್ಯ ಉದ್ದೇಶವೇ ಧನ ಸಂಗ್ರಹಣೆಯಲ್ಲವೆ? ಆದರೆ ಕೇವಲ ಹಣ ಮಾತ್ರ ಇದ್ದು, ಅದನ್ನು ನಿರ್ವಹಿಸಬಲ್ಲ ಬುದ್ಧಿ ಯನ್ನೂ ಯೋಗ್ಯತೆಯನ್ನೂ ಹೊಂದಿರುವ ಜನರಿಲ್ಲದಿದ್ದರೆ ಆ ಹಣದಿಂದೇನು ಪ್ರಯೋಜನ ವಾಗಬಲ್ಲುದು? ಅಲ್ಲದೆ ಅಪಾರವಾದ ಹಣವನ್ನು ಸರಿಯಾದ ರೀತಿಯಲ್ಲಿ ವಿನಿಯೋಗಿಸ ಬೇಕಾದರೂ ವಿವೇಕಪ್ರಜ್ಞೆ ಬೇಕೇಬೇಕು. ಆದ್ದರಿಂದ ಕೇವಲ ಹಣವಂತರಿಂದ ಯಾವ ಮಹಾ ಕಾರ್ಯವೂ ಆಗಿಯೂ ಇಲ್ಲ, ಆಗುವುದೂ ಇಲ್ಲ ಎನ್ನುತ್ತಾರೆ ಸ್ವಾಮೀಜಿ.

ಈ ಅವಧಿಯಲ್ಲಿ ಹಲವಾರು ಸಾರ್ವಜನಿಕ ಉಪನ್ಯಾಸಗಳನ್ನೂ ನೀಡಿದರು. ಈ ಉಪನ್ಯಾಸಗಳ ಮೂಲಕ ಅವರ ಸಂಪರ್ಕಕ್ಕೆ ಬಂದವರಲ್ಲಿ ಕುಮಾರಿ ಲಾರಾಗ್ಲೆನ್ ಒಬ್ಬಳು. ಈಕೆ ಅವರ ಶಿಷ್ಯೆ ಯಾದಳಲ್ಲದೆ ಮುಂದೆ ಸೋದರಿ ದೇವಮಾತಾ ಎಂಬ ಹೆಸರಿನಿಂದ ಬ್ರಹ್ಮಚಾರಿಣಿಯಾದಳು. ಅದಾಗಲೇ ಅವಳು ಹಿಂದೂಧರ್ಮದ ಬಗ್ಗೆ ಓದಿಕೊಂಡಿದ್ದಳು ಮತ್ತು ಸ್ವಾಮೀಜಿಯ ಬಗ್ಗೆಯೂ ಕೇಳಿ ತಿಳಿದಿದ್ದಳು. ಆದರೆ ಅವರನ್ನು ಕಣ್ಣಾರೆ ನೋಡಿರಲಿಲ್ಲ. ಮೊದಲ ಸಲ ಅವರ ಉಪನ್ಯಾಸ ವನ್ನು ಕೇಳಲೆಂದು ಬಂದಾಗ ತನಗಾದ ಅನುಭವವನ್ನು ಮಿಸ್ ಲಾರಾಗ್ಲೆನ್ ಹೀಗೆ ಬಣ್ಣಿಸುತ್ತಾಳೆ:

“ಒಂದು ದಿನ ನಾನು ಮ್ಯಾಡಿಸನ್ ಅವಿನ್ಯೂದಲ್ಲಿ ನಡೆದುಕೊಂಡು ಹೋಗುತ್ತಿದ್ದಾಗ ‘ವಿಶ್ವಭ್ರಾತೃತ್ವ ಭವನ’ದ ಕಿಟಕಿಯಲ್ಲಿ ಒಂದು ಸಣ್ಣ ಸೂಚನಾಫಲಕವನ್ನು ನೋಡಿದೆ. ಅದರಲ್ಲಿ ‘ಮುಂದಿನ ಭಾನುವಾರ ಅಪರಾಹ್ನ ಮೂರು ಗಂಟೆಗೆ ಇಲ್ಲಿ ಸ್ವಾಮಿ ವಿವೇಕಾನಂದರು– “ವೇದಾಂತ ಎಂದರೇನು?” ಎಂಬ ವಿಷಯವಾಗಿಯೂ ಅದರ ಮುಂದಿನ ಭಾನುವಾರ “ಯೋಗ ಎಂದರೇನು?” ಎಂಬ ವಿಷಯವಾಗಿಯೂ ಮಾತನಾಡುತ್ತಾರೆ’ ಎಂದು ಬರೆದಿತ್ತು. ನಾನು ಆ ದಿನ ಅಲ್ಲಿಗೆ ಮೂರು ಗಂಟೆಗೆ ಇಪ್ಪತ್ತು ನಿಮಿಷವಿರುವಾಗಲೇ ಹೋದೆ. ಸಭಾಭವನ ಆಗಲೇ ಅರ್ಧಕ್ಕಿಂತ ಹೆಚ್ಚು ಭರ್ತಿಯಾಗಿತ್ತು. ಅದು ಅಷ್ಟೇನೂ ದೊಡ್ಡದಲ್ಲ; ಅದನ್ನೊಂದು ಹಜಾರ ಎನ್ನಬಹುದು. ಮೂರು ಗಂಟೆಯಾಗುವ ಹೊತ್ತಿಗೆ ಆ ಹಜಾರ, ಮೆಟ್ಟಲು, ಕಿಟಕಿಗಳ ಗೂಡುಗಳು, ಕಟೆಕಟೆ ಎಲ್ಲವೂ ತಮ್ಮ ಸಾಮರ್ಥ್ಯ ಮೀರಿ ತುಂಬಿದ್ದುವು. ಆ ಹಜಾರವಿದ್ದುದು ಎರಡನೆಯ ಮಹಡಿಯಲ್ಲಿ. ಸ್ವಾಮೀಜಿಯ ಮಾತುಗಳ ಕೆಲವು ತುಣುಕುಗಳು ಅಸ್ಪಷ್ಟವಾಗಿಯಾದರೂ ಕೇಳಿಸಿಯಾವೆಂಬ ಆಸೆಯಿಂದ ಕೆಲವರು ಕೆಳಗಡೆಯೂ ನಿಂತಿದ್ದರು. ಇದ್ದಕ್ಕಿದ್ದಂತೆ ಸಭೆ ನಿಶ್ಶಬ್ದವಾಯಿತು; ಮೆಟ್ಟಿಲ ಮೇಲೆ ಹೆಜ್ಜೆಯ ದನಿ ಕೇಳಿಸಿತು. ಸ್ವಾಮಿ ವಿವೇಕಾನಂದರು ರಾಜ ಗಾಂಭೀರ್ಯದಿಂದ ನಡೆದುಬಂದು ವೇದಿಕೆಯನ್ನೇರಿ ನಿಂತು ಮಾತನಾಡಲಾರಂಭಿಸಿದರು. ತಕ್ಷಣವೇ ಕಾಲ, ದೇಶ, ಜನ, ಸ್ಮೃತಿ ಎಲ್ಲವೂ ಕರಗಿಹೋಯಿತು. ಶೂನ್ಯದಲ್ಲಿ ಮೊಳಗುತ್ತಿದ್ದ ಧ್ವನಿಯೊಂದನ್ನು ಬಿಟ್ಟರೆ ಇನ್ನೇನೂ ಉಳಿದಿರಲಿಲ್ಲ ಅಲ್ಲಿ. ದ್ವಾರವೊಂದು ಬೀಸಿ ತೆರೆದುಕೊಂಡು ನಾನು ಅನಂತಸಿದ್ಧಿಯ ಮಾರ್ಗದಲ್ಲಿ ನಡೆಯುತ್ತಿರುವಂತಿತ್ತು. ಅದರ ಅಂತ್ಯ ಮಾತ್ರ ಕಾಣುತ್ತಿರ ಲಿಲ್ಲ. ಆದರೆ ಅದು ಹೇಗಿರಬಹುದೆಂಬುದರ ಭರವಸೆಯು, ನನಗೆ ಆ ದಾರಿಯನ್ನು ತೋರಿದವರ ಮಾತುಗಳಲ್ಲಿ ಮಿನುಗುತ್ತಿತ್ತು. ಅವರ ವ್ಯಕ್ತಿತ್ವದಲ್ಲಿ ಮಿಂಚುತ್ತಿತ್ತು. ನನ್ನ ಪಾಲಿಗೆ ಅವರು ಅನಂತತೆಯ ಪ್ರವಾದಿಯಾಗಿ ನಿಂತಿದ್ದರು.

“ಖಾಲಿಯಾದ ಹಜಾರದ ನಿಶ್ಶಬ್ದತೆಯು ಇದ್ದಕ್ಕಿದ್ದಂತೆ ನನಗೆ ಮೈತಿಳಿಯುವಂತೆ ಮಾಡಿತು. ಸ್ವಾಮೀಜಿ ಹಾಗೂ ವೇದಿಕೆಯ ಬಳಿ ನಿಂತಿದ್ದ ಇನ್ನಿಬ್ಬರನ್ನು ಬಿಟ್ಟರೆ ಎಲ್ಲರೂ ಹೊರಟುಹೋಗಿ ದ್ದರು. ಅವರು ಸ್ವಾಮೀಜಿಯ ಅನುಯಾಯಿಗಳಾದ ಶ್ರೀಮತಿ ಮತ್ತು ಶ್ರೀ ಗುಡ್​ಇಯರ್ ಎಂದು ಮುಂದೆ ನನಗೆ ಗೊತ್ತಾಯಿತು.”

ಶ್ರದ್ಧೆಯೊಂದು ಇದ್ದುಬಿಟ್ಟರೆ ಎಂತಹ ತದೇಕಚಿತ್ತತೆಯುಂಟಾಗುತ್ತದೆ ಎಂಬುದಕ್ಕೆ ಲಾರಾ ಗ್ಲೆನ್ನಳೇ ಸಾಕ್ಷ್ಯ. ಅಲ್ಲದೆ ಇಲ್ಲಿ ಸ್ವಾಮೀಜಿಯ ಆಧ್ಯಾತ್ಮಿಕ ಶಕ್ತಿಯೂ ಗಮನಾರ್ಹ. ಇದೇ ಸಮಯದಲ್ಲಿ ಅವರ ಸ್ಫೂರ್ತಿದಾಯಕ ಮಾತುಗಳಿಂದ ಪ್ರಭಾವಿತಳಾದ ಇನ್ನೊಬ್ಬಳೆಂದರೆ ಶ್ರೀಮತಿ ಎಲ್ಲಾ ವೀಲರ್ ವಿಲ್​ಕಾಕ್ಸ್ ಎಂಬ ಪ್ರಸಿದ್ಧ ಕವಯಿತ್ರಿ. ಇವಳು ೧೯ಂ೭ರಲ್ಲಿ ‘ನ್ಯೂಯಾರ್ಕ್ ಅಮೆರಿಕನ್​’ ಪತ್ರಿಕೆಯಲ್ಲಿ ಬರೆದ ಲೇಖನದ ಒಂದು ಭಾಗ ಇಂತಿದೆ:

“ಹನ್ನೆರಡು ವರ್ಷಗಳ ಹಿಂದೆ ಒಂದು ಸಂಜೆ, ಭಾರತದ ತತ್ತ್ವಬೋಧಕರಾದ ವಿವೇಕಾ ನಂದರು ಎಂಬವರೊಬ್ಬರು ಉಪನ್ಯಾಸ ಮಾಡಲಿರುವರೆಂದು ಕೇಳಿದೆ. ಕುತೂಹಲಗೊಂಡು ನಾನು, ನನ್ನ ಪತಿ ಇಬ್ಬರೂ ಅಲ್ಲಿಗೆ ಹೋದೆವು. ಕಾರ್ಯಕ್ರಮ ಪ್ರಾರಂಭವಾಗಿ ಇನ್ನೂ ಹತ್ತು ನಿಮಿಷ ಕೂಡ ಆಗಿಲ್ಲ. ಆಗಲೇ ನಮಗೆ ಎಂತಹ ಒಂದು ಅಪೂರ್ವವಾದ ಚೈತನ್ಯಯುಕ್ತವಾದ ಅದ್ಭುತ ವಾತಾವರಣಕ್ಕೆ ಏರಿಸಲ್ಪಟ್ಟ ಅನುಭವವಾಯಿತೆಂದರೆ, ಉಪನ್ಯಾಸದ ಕೊನೆಯವರೆಗೂ ಮೂಕವಿಸ್ಮಿತರಾಗಿ, ಉಸಿರಾಡದೆ ಕುಳಿತುಬಿಟ್ಟೆವು.

“ಉಪನ್ಯಾಸ ಮುಗಿದ ಮೇಲೆ ನಾವು ಜೀವನದ ಏರುಪೇರುಗಳನ್ನೆಲ್ಲ ಎದುರಿಸಬಲ್ಲ ಹೊಸ ಧೈರ್ಯದಿಂದ ಹೊಸ ಭರವಸೆಯಿಂದ, ಹೊಸ ಶಕ್ತಿ-ಉತ್ಸಾಹದಿಂದ ಹಿಂದಿರುಗಿದೆವು. ನನ್ನ ಪತಿ ಹೇಳಿದರು, ‘ನಾನು ಅರಸುತ್ತಿದ್ದ ದೇವರು-ಧರ್ಮ-ತತ್ತ್ವ ಎಲ್ಲವೂ ಇದೆ!’ ಅಲ್ಲಿಂದ ಮುಂದೆ, ಸ್ವಾಮಿ ವಿವೇಕಾಂದರು ಸನಾತನ ಧರ್ಮವನ್ನು ವಿವರಿಸುವುದನ್ನು ಕೇಳಲು, ಮತ್ತು ಅದ್ಭುತ ರತ್ನಗಳಾದ ಶಕ್ತಿ-ಉತ್ಸಾಹದ ಆಲೋಚನೆಗಳನ್ನು ಪಡೆದುಕೊಳ್ಳಲು ತನ್ನ ಪತಿಯವರು ನನ್ನನ್ನು ಕರೆದುಕೊಂಡು ಹೋಗುತ್ತಿದ್ದರು. ಎಲ್ಲೆಲ್ಲೂ ಸಂಪೂರ್ಣ ಆರ್ಥಿಕ ಕುಸಿತವುಂಟಾಗಿದ್ದ ಭಯಂಕರ ಚಳಿಗಾಲ ಅದು. ಚಿಂತೆ ಕಳವಳಗಳಿಂದಾಗಿ ನಿದ್ರೆಯೇ ಇಲ್ಲದ ಹಲವಾರು ರಾತ್ರಿ ಗಳನ್ನು ಕಳೆದ ಮೇಲೆ ನನ್ನ ಪತಿ, ನನ್ನೊಂದಿಗೆ ಸ್ವಾಮೀಜಿಯ ಉಪನ್ಯಾಸವನ್ನು ಕೇಳಲು ಹೋಗುತ್ತಿದ್ದರು. ಅನಂತರ ಅವರು ಅಲ್ಲಿಂದ ಹೊರಟು ಚಳಿಗಾಲದ ಮಸುಕಿನಿಂದ ಕೂಡಿದ ಬೀದಿಯಲ್ಲಿ ನಡೆದುಬರುವಾಗ ಮುಗುಳ್ನಗುತ್ತ ಹೇಳುತ್ತಿದ್ದರು, ‘ಈಗ ಎಲ್ಲವೂ ಸರಿಯಾಗಿದೆ; ಇನ್ನು ಚಿಂತಿಸಬೇಕಾದ ಕಾರಣವಿಲ್ಲ’ ಎಂದು. ನಾನು ಕೂಡ ಅದೇ ನಿರಾಳವಾದ ಮನಸ್ಸಿನಿಂದ ಮತ್ತು ವಿಶಾಲಗೊಂಡ ದೃಷ್ಟಿಯಿಂದ ನನ್ನ ಕೆಲಸ ಕಾರ್ಯಗಳಿಗೆ ಹಿಂದಿರುಗುತ್ತಿದ್ದೆ.

“ಈ ಕಷ್ಟಕೋಟಲೆಗಳ ಯುಗದಲ್ಲಿ ಯಾವುದೇ ಧರ್ಮ ಯಾವುದೇ ತತ್ತ್ವಶಾಸ್ತ್ರ ಇದನ್ನು ಮಾಡಬಲ್ಲುದಾದರೆ, ಮತ್ತು ಇದರೊಂದಿಗೆ ಭಗವಂತನಲ್ಲಿನ ಶ್ರದ್ಧೆಯನ್ನು ತೀವ್ರಗೊಳಿಸಿ ತಮ್ಮ ಸಹಮಾನವರ ಮೇಲಿನ ಸಹಾನುಭೂತಿಯನ್ನು ಹೆಚ್ಚಿಸಿ, ಮುಂಬರುವ ಜನ್ಮಗಳ ಕುರಿತಾಗಿ ಆನಂದದ ಭರವಸೆಯನ್ನು ಕೊಡಬಲ್ಲುದಾದರೆ ಅದೊಂದು ಒಳ್ಳೆಯ ಧರ್ಮ–ಮಹಾ ಧರ್ಮವೇ ಸರಿ.”

ಸ್ವಾಮೀಜಿಯ ವ್ಯಕ್ತಿತ್ವ, ಅವರ ಮಾತುಕತೆ, ಅವರ ಉಪನ್ಯಾಸಗಳು–ಇವೆಲ್ಲ ಆಳವಾದ ಪರಿಣಾಮವನ್ನುಂಟುಮಾಡದ ವ್ಯಕ್ತಿಯೇ ಇಲ್ಲವೆಂಬಂತೆ ತೋರುತ್ತದೆ. ಅಲ್ಲದೆ ಪ್ರತಿಯೊಬ್ಬ ವ್ಯಕ್ತಿಯ ಮೇಲೂ ಅವು ಉಂಟುಮಾಡುತ್ತಿದ್ದ ಪರಿಣಾಮ ಒಂದೊಂದು ತೆರನಾದದ್ದು. ಅವರ ಉಪನ್ಯಾಸಗಳಿಗೆ ಮುತ್ತುತ್ತಿದ್ದ ಸಾವಿರಾರು ಜನರಲ್ಲಿ ಒಬ್ಬೊಬ್ಬರಿಗೂ ಆದ ಅನುಭವ ಎಂತಹ ದಿರಬಹುದೊ! ಅವರ ಒಂದೇ ಒಂದು ಮಾತು, ಒಂದೇ ಒಂದು ಸ್ಪರ್ಶ, ಕಡೆಗೆ ಒಂದೇ ಒಂದು ನೋಟ, ಅದೆಷ್ಟು ಜನರನ್ನು ಧನ್ಯವಾಗಿಸಿತೋ ಬಲ್ಲವರಾರು! ಇನ್ನು ಅವರನ್ನು ಹತ್ತಿರದಿಂದ ಕಂಡ, ಅವರ ಸಂಪರ್ಕಕ್ಕೆ ಬಂದ ಭಾಗ್ಯಶಾಲಿಗಳ ಅನುಭವಗಳು ಇನ್ನೆಂಥವೋ! ಆದರೆ ಅವುಗಳಲ್ಲಿ ಕೆಲವಾದರೂ ಸ್ಮೃತಿಚಿತ್ರಣಗಳ ಮೂಲಕ ನಮಗೆ ತಿಳಿದುಬಂದಿರುವುದು ನಮ್ಮ ಭಾಗ್ಯವೇ ಸರಿ.

ಸ್ವಾಮೀಜಿ ತಮ್ಮ ಧ್ಯಾನ-ತರಗತಿ-ಉಪನ್ಯಾಸ-ಮಾತುಕತೆ ಇವುಗಳನ್ನು ಬಿಟ್ಟು ಉಳಿದ ಸಮಯದಲ್ಲಿ, ಅಮೆರಿಕದ ಜನಜೀವನದ ಬಗೆಯನ್ನು ಅಧ್ಯಯನ ಮಾಡಲು ಇಚ್ಛಿಸುತ್ತಿದ್ದರು. ಏಕೆಂದರೆ ಅಮೆರಿಕದ ಲೌಕಿಕ ಜೀವನಕ್ರಮದಲ್ಲೂ ಒಂದು ವೈಶಿಷ್ಟ್ಯವಿದೆ. ಒಂದು ಸೊಬಗಿದೆ. ಸ್ವಾಮೀಜಿ ಇಹಜೀವನವನ್ನು ಗೌರವಿಸುವವರೇ ಹೊರತು ದ್ವೇಷಿಸುವವರಲ್ಲ. ಆದರೆ ತಾವು ಗಳಿಸಿದ ಲೌಕಿಕ ಜ್ಞಾನವನ್ನವರು ತಮ್ಮ ಆಧ್ಯಾತ್ಮಿಕ ಜ್ಞಾನಕ್ಕೆ ತಿರುಗಿಸಿಕೊಳ್ಳುತ್ತಿದ್ದರು–ಇದು ಅವರ ವೈಶಿಷ್ಟ್ಯ.

ಸ್ವಾಮೀಜಿ ನ್ಯೂಯಾರ್ಕಿನ ನಾಗರಿಕತೆಯ ಅಲ್ಲೋಲಕಲ್ಲೋಲಗಳ ನಡುವೆ ವಾಸವಾಗಿದ್ದರೂ ಮಾನಸಿಕವಾಗಿ ಒಂದು ಪ್ರಶಾಂತ ವಾತಾವರಣವನ್ನು ನಿರ್ಮಿಸಿಕೊಂಡು, ನಿರಂತರ ಭಗವತ್ ಸ್ಮರಣೆ ಹಾಗೂ ಆಳ ಚಿಂತನೆಯಲ್ಲಿ ನಿರತರಾಗಿರುತ್ತಿದ್ದರು. ಅವರು ತಮ್ಮ ಪಾಶ್ಚಾತ್ಯ ಜೀವನ ದುದ್ದಕ್ಕೂ ತಮ್ಮ ಧ್ಯಾನದ ಸ್ವಭಾವವನ್ನು ಉಳಿಸಿಕೊಂಡುಬಂದದ್ದು ಒಂದು ಗಮನಾರ್ಹ ಅಂಶ. ಏಕೆಂದರೆ ಅವರಿಗೆ ಬಂದೊದಗುತ್ತಿದ್ದ ಅಡಚಣೆಗಳು ಅಸಂಖ್ಯಾತ. ಇಷ್ಟೆಲ್ಲ ಅಡ್ಡಿ ಆತಂಕಗಳ ನಡುವೆಯೂ ಅವರು ಪ್ರಾತಃಕಾಲ ಅಥವಾ ಸಾಯಂಕಾಲದಲ್ಲಿ, ಇಲ್ಲವೆ ನೀರವ ಮಧ್ಯರಾತ್ರಿಯ ವೇಳೆಯಲ್ಲಿ ಧ್ಯಾನನಿರತರಾಗಿಬಿಡುತ್ತಿದ್ದರು. ಎಷ್ಟೋ ಸಲ ಅವರು ಧ್ಯಾನಮಗ್ನರಾಗಿ ಕುಳಿತಿರ ದಿದ್ದರೂ ಧ್ಯಾನಭಾವರಂಜಿತರಾಗಿರುತ್ತಿದ್ದರು. ಆಗ ಅವರು ಕಣ್ಣುಗಳನ್ನು ತೆರೆದಿದ್ದರೂ, ಬಾಹ್ಯ ಜಗತ್ತಿನ ಪರಿವೆಯೇ ಇಲ್ಲದೆ ಆತ್ಮಾನಂದದಲ್ಲಿ ಮುಳುಗಿರುತ್ತಿದ್ದರು. ಕೆಲವೊಮ್ಮೆ ದಿಗಂತ ದಾಚೆಗೆ ದೃಷ್ಟಿಯನ್ನು ನೆಟ್ಟಿರುತ್ತಿದ್ದ ಅವರ ಮುಖಮಂಡಲವು, ಅವರ ಮನಸ್ಸು ಎಲ್ಲ ವ್ಯಾವಹಾರಿಕ ಜಂಜಡಗಳಿಂದ ಮುಕ್ತವಾಗಿ ಪರಬ್ರಹ್ಮಭಾವದಲ್ಲಿ ಲೀನವಾಗಿರುವುದನ್ನು ಸೂಚಿ ಸುತ್ತಿತ್ತು. ಅವರ ಸುತ್ತಲಿರುವವರೆಲ್ಲ ಸಂತೋಷದಿಂದ ಹರಟುತ್ತಿದ್ದರೆ ಸ್ವಾಮೀಜಿಯ ಕಣ್ಣು ಗಳು ಮೆಲ್ಲನೆ ಸ್ಥಿರಗೊಳ್ಳುತ್ತಿದ್ದುವು. ಅವರ ದೇಹ ಚಲನರಹಿತವಾಗುತ್ತಿತ್ತು. ಶ್ವಾಸೋಚ್ಛ್ವಾಸ ಸಾವಧಾನಗೊಂಡು ಬಳಿಕ ಅದು ಒಂದು ಕ್ಷಣ ನಿಂತೇ ಹೋಗುತ್ತಿತ್ತು; ಕೆಲ ನಿಮಿಷಗಳ ಬಳಿಕ ಅವರು ಮತ್ತೆ ಮೆಲ್ಲನೆ ಎಚ್ಚರಗೊಂಡು ಬಾಹ್ಯಪರಿಸರಕ್ಕೆ ಮರಳುತ್ತಿದ್ದರು. ಅವರ ಪರಿಚಿತರಿ ಗೆಲ್ಲ ಈ ವಿಷಯ ಚೆನ್ನಾಗಿ ತಿಳಿದಿತ್ತು. ಆದ್ದರಿಂದ ಅವರು ಅದಕ್ಕನುಗುಣವಾಗಿ ನಡೆದುಕೊಳ್ಳು ತ್ತಿದ್ದರು. ಸ್ವಾಮೀಜಿ ತಮ್ಮ ಸ್ನೇಹಿತರ ಆಮಂತ್ರಣವನ್ನು ಅಂಗೀಕರಿಸಿ ಅವರ ಮನೆಗೆ ಹೋದರೂ ಅಲ್ಲಿ ಮಾತನಾಡದೆ ಮೌನವಾಗಿರಬಹುದು; ಆದರೆ ಅವರ ಸ್ವಭಾವವನ್ನು ಅರಿತ ಮನೆಯವರು ಅದನ್ನು ತಪ್ಪಾಗಿ ಭಾವಿಸುತ್ತಿರಲಿಲ್ಲ, ಬದಲಾಗಿ ಸಹಕರಿಸುತ್ತಿದ್ದರು. ಅವರು ತಮ್ಮ ಕೋಣೆಯಲ್ಲಿ ಮೌನವಾಗಿ ಕುಳಿತಿರುವುದನ್ನು ಕಂಡರೆ ಯಾರೂ ಅವರಿಗೆ ತೊಂದರೆ ಕೊಡುತ್ತಿರಲಿಲ್ಲ. ಕೆಲವೊಮ್ಮೆ ಯಾರಾದರೂ ಬಂದರೆ ಸ್ವಾಮೀಜಿ ತಾವಾಗಿಯೇ ಎದ್ದು ಬಂದು ಅವರನ್ನು ಮಾತನಾಡಿಸುತ್ತಿದ್ದರು. ಆದರೆ ಅವರ ಭಾವಲಹರಿ ಮಾತ್ರ ಹಾಗೆಯೇ ಮುಂದು ವರಿಯುತ್ತಿತ್ತು. ತಮ್ಮ ಆಲೋಚನಾಸರಣಿಯನ್ನು ಕೆಡಿಸಿಕೊಳ್ಳದೆ ಬಂದವರೊಡನೆ ಮಾತನಾಡು ವುದು ಸ್ವಾಮೀಜಿಯ ವೈಶಿಷ್ಟ್ಯ. ಹೀಗೆ ಅವರ ನಿಜವಾದ ಆಸಕ್ತಿ ಗಮನವೆಲ್ಲ ತಮ್ಮ ಅಂತರಂಗದ ಕಡೆಗೇ ಹೊರತು ಬಾಹ್ಯಜಗತ್ತಿನ ಕಡೆಗಲ್ಲ. ಜನರಿಗೆ ಅವರ ಭಾವನೆಗಳ ಔನ್ನತ್ಯದ ಸುಳಿವು ಸಿಗುತ್ತಿದ್ದುದು ಅವರು ಯಾವಾಗಲಾದರೂ ಆಡುತ್ತಿದ್ದ ಮಾತುಗಳ ಮೂಲಕವೇ.

ನ್ಯೂಯಾರ್ಕಿನಲ್ಲಿ ಸ್ವಾಮೀಜಿ ಅನೇಕ ಹೊಸ ಸ್ನೇಹಿತರನ್ನು ಸಂಪಾದಿಸಿಕೊಂಡರು. ಈ ದಿನ ಗಳಲ್ಲಿ ಅವರ ನಿಕಟಸಂಪರ್ಕಕ್ಕೆ ಬಂದವರು ಹಲವಾರು ಜನ. ಇವರಲ್ಲಿ ಅನೇಕರು ಈಗಾಗಲೇ ಅವರಿಗೆ ಪರಿಚಿತರಾಗಿದ್ದವರು. ಹೀಗೆ ಅವರ ವಿಶ್ವಾಸಿಗರಾದವರಲ್ಲಿ ಹಾಗೂ ಶಿಷ್ಯರಾದವರಲ್ಲಿ ಕೆಲವರೆಂದರೆ ಶ್ರೀಮತಿ ಸಾರಾ ಬುಲ್, ಡಾ ॥ ಆಲನ್ ಡೇ, ಮಿಸ್ ಎಲೆನ್ ವಾಲ್ಡೋ, ಮಿಸ್ ಮೇರಿ ಫಿಲಿಪ್ಸ್, ಪ್ರೊ ॥ ವೈಮನ್, ಪ್ರೊ ॥ ರೈಟ್, ಶ್ರೀ ಫ್ರಾನ್ಸಿಸ್ ಲೆಗೆಟ್ ಹಾಗೂ ಶ್ರೀಮತಿ ಬೆಸ್ಸಿ ಸ್ವರ್ಜಸ್ (ಮುಂದೆ ಶ್ರೀಮತಿ ಫ್ರಾನ್ಸಿಸ್ ಲೆಗೆಟ್​). ನ್ಯೂಯಾರ್ಕಿನ ಮಿಸ್ ಜೋಸೆಫಿನ್ ಮೆಕ್​ಲಾಡ್ ಸ್ವಾಮೀಜಿಯ ಅತ್ಯಂತ ಆಪ್ತ ಶಿಷ್ಯೆಯಾದಳಲ್ಲದೆ, ಕಡೆಯವರೆಗೂ ಅವರಿಗೆ ನಾನಾ ವಿಧದಲ್ಲಿ ನೆರವಾದಳು. ಡಾ ॥ ಎಗ್​ಬರ್ಟ್ ಗರ್ನ್​ಸೇ ದಂಪತಿಗಳಂತೂ ಸ್ವಾಮೀಜಿಯನ್ನು ತಮ್ಮ ಸ್ವಂತ ಪುತ್ರನಂತೆ ನೋಡಿಕೊಳ್ಳುತ್ತಿದ್ದರು. ಇನ್ನೊಬ್ಬ ಶ್ರೀಮಂತ ಗಣ್ಯವ್ಯಕ್ತಿಯಾದ ಆಸ್ಟಿನ್ ಕಾರ್ಬಿನ್ ಹಾಗೂ ಅವನ ಮಗಳು, ಮತ್ತು ಸುಪ್ರಸಿದ್ಧ ಗಾಯಕಿ ಎಮ್ಮಾ ಥರ್ಸ್​ಬಿ –ಇವರೂ ಸ್ವಾಮೀಜಿಯ ಅನುಯಾಯಿಗಳಾದರು.

ತಾವು ಅಮೆರಿಕದಿಂದ ಹೊರಟ ಮೇಲೂ ತಾವು ಪ್ರಾರಂಭಿಸಿದ ಕಾರ್ಯ ನಿರಂತರವಾಗಿ ನಡೆದುಕೊಂಡು ಹೋಗಬೇಕಾದರೆ ಅತ್ಯಂತ ಸಮರ್ಥರಾದ ವ್ಯಕ್ತಿಗಳ ಆವಶ್ಯಕತೆಯಿರುವುದರ ಅರಿವು ಸ್ವಾಮೀಜಿಗಿತ್ತು. ಆದ್ದರಿಂದ ಅವರು ಯೋಗ್ಯರಾದ ಕೆಲವರಿಗೆ ಸಂನ್ಯಾಸ ನೀಡಲು ಇಚ್ಛಿಸಿದರು. ಅವರ ಶಿಷ್ಯರ ಪೈಕಿ ಮೇಡಮ್ ಮೇರಿ ಲೂಯಿಸ್ ಹಾಗೂ ಲಿಯಾನ್ ಲ್ಯಾಂಡ್ಸ್​ಬರ್ಗ್ ಎಂಬಿಬ್ಬರು ಸಂನ್ಯಾಸ ಸ್ವೀಕರಿಸಲು ಸಿದ್ಧರಾಗಿದ್ದರು. ಲಿಯಾನ್ ಲ್ಯಾಂಡ್ಸ್ ಬರ್ಗ್ ನ್ಯೂಯಾರ್ಕಿನಲ್ಲಿ ಸ್ವಾಮೀಜಿಯೊಂದಿಗೇ ಎರಡೂವರೆ ತಿಂಗಳ ಕಾಲ ವಾಸವಾಗಿದ್ದು ಅವರ ಸಂಘಟನಾ ಕಾರ್ಯದಲ್ಲಿ ನೆರವಾದ.

೧೮೯೫ರ ಏಪ್ರಿಲ್ ತಿಂಗಳಲ್ಲಿ ಸ್ವಾಮೀಜಿ ಶ್ರೀ ಲೆಗೆಟ್ಟರ ಆಹ್ವಾನವನ್ನು ಒಪ್ಪಿ, ‘ರಿಡ್ಜ್​ಲಿ ಮ್ಯಾನರ್​\eng{’ (Ridgely Manor)} ಎಂಬ ಅವರ ಹಳ್ಳಿಯ ಮನೆಗೆ ಹೋದರು. ಇದು ಇರುವುದು ನ್ಯೂಯಾರ್ಕ್ ರಾಜ್ಯದಲ್ಲಿ, ಹಡ್ಸನ್ ನದಿಯ ತೀರದಲ್ಲಿ. ಅದೊಂದು ಸುಂದರವಾದ ವನಸಿರಿ ಯಿಂದ, ಭವ್ಯವಾದ ಪರ್ವತಗಳಿಂದ ಕೂಡಿದ ಸ್ಥಳ. ಇದು ‘ರಿಡ್ಜ್​ಲಿ ಮ್ಯಾನರ್​’ಗೆ ಸ್ವಾಮೀಜಿಯ ಪ್ರಥಮ ಭೇಟಿ. (ಅವರು ಇಲ್ಲಿಗೆ ಎರಡನೆಯ ಸಲ ಬಂದದ್ದು ಅದೇ ವರ್ಷದ ಕ್ರಿಸ್​ಮಸ್​ನಲ್ಲಿ. ಮತ್ತು ಕಡೆಯ ಹಾಗೂ ಅತಿ ಮುಖ್ಯವಾದ ಭೇಟಿ ನೀಡಿದ್ದು ೧೮೯೯ರಲ್ಲಿ–ಆಗ ಅವರು ಇಲ್ಲಿ ಹತ್ತು ಅಮೂಲ್ಯ ವಾರಗಳನ್ನು ಕಳೆದರು.)

ಲೆಗೆಟ್ಟರು ತಮ್ಮ ತೋಟದ ಮನೆಗೆ ಸ್ವಾಮೀಜಿಯೊಂದಿಗೆ ಮಿಸ್ ಮೆಕ್​ಲಾಡ್​ನ್ನೂ ಆಕೆಯ ಸೋದರಿ ಬೆಸ್ಸಿ ಸ್ಟರ್ಜಸ್​ಳನ್ನೂ ಆಹ್ವಾನಿಸಿದ್ದರು. ಬೆಸ್ಸಿ ಕೆಲಕಾಲದ ಹಿಂದೆ ತಾನೇ ವಿಧವೆಯಾಗಿ ದ್ದವಳು. ಅವಳನ್ನು ವಿವಾಹವಾಗಲು ಲೆಗೆಟ್ ನಿಶ್ಚಯಿಸಿದ್ದರು. ಈಕೆಯ ಹದಿನೆಂಟು ವರ್ಷದ ಮಗಳು ಆಲ್ಬರ್ಟಾ ಹಾಗೂ ಹದಿನಾರು ವರ್ಷದ ಮಗ ಹಾಲಿಸ್ಟರ್​–ಇವರಿಬ್ಬರೂ ರಿಡ್ಜ್​ಲಿ ಮ್ಯಾನರ್​ಗೆ ಬಂದಿದ್ದರು. ಇಲ್ಲಿ ಸ್ವಾಮೀಜಿ ಇವರೆಲ್ಲರೊಂದಿಗೆ ಬಹಳ ಸಂತೋಷಕರವಾದ ಎರಡು ವಾರಗಳನ್ನು ಕಳೆದರು. ಈ ಸಂದರ್ಭದಲ್ಲಿ ನಡೆದ ಕೆಲವು ಕುತೂಹಲಕರ ಘಟನೆಗಳು ಇತ್ತೀಚೆಗೆ ತಿಳಿದುಬಂದಿವೆ.

ಒಂದು ದಿನ ಮೆಕ್​ಲಾಡಳು ಲೆಗೆಟ್ಟರ ಕೋಣೆಗೆ ಬಂದಳು. ಕೋಣೆಯ ಬಾಗಿಲು ಹಾಕಿತ್ತು. ಒಳಗೆ ಯಾರೂ ಇರಲಾರರೆಂದು ತಿಳಿದು ಆಕೆ ಸೀದಾ ಬಾಗಿಲನ್ನು ತೆರೆದು ಒಳಗೆ ಅಡಿಯಿಟ್ಟಳು –ಅಲ್ಲಿ ನಡೆದ ಒಂದು ಅದ್ಭುತವನ್ನು ಕಂಡು ಗರಬಡಿದಂತೆ ನಿಂತುಬಿಟ್ಟಳು! ಅವಳು ಕಂಡದ್ದೇನು? ಫ್ರಾನ್ಸಿಸ್ ಲೆಗೆಟ್ ಗಾಳಿಯಲ್ಲಿ ಹಾರಾಡುತ್ತಿದ್ದಾರೆ! ಅಥವಾ ಅವಳಿಗೆ ಹಾಗನ್ನಿ ಸಿರಬಹುದೆ? ಅದರೆ ಮರುಕ್ಷಣವೇ ಲೆಗೆಟ್ಟರು ಧರೆಗುರುಳಿದ್ದರು! ಕಾಲುಗಳನ್ನು ಮೇಲೆತ್ತಿ ಬೆನ್ನ ಮೇಲೆ ಮಲಗಿದ್ದ ಸ್ವಾಮೀಜಿ ಛಂಗನೆದ್ದು ನಿಂತರು! ತಮ್ಮ ಬಟ್ಟೆಯನ್ನು ಸರಿಪಡಿಸಿ ಕೊಂಡು ಸಿಟ್ಟಿನ ದನಿಯಲ್ಲಿ ಕೂಗಿದರು “ಇದು ಪುರುಷನ ಕೋಣೆ. ಬಾಗಿಲು ತಟ್ಟದೆ ಸೀದಾ ಒಳಗೆ ಬಂದೆಯೇಕೆ?” ಮೆಕ್​ಲಾಡ್ ಏನು ಹೇಳಲೂ ತೋಚದೆ ಸುಮ್ಮನೆ ನಿಂತಿದ್ದಳು. ಒಂದು ಕ್ಷಣ ಮೌನವಾಗಿದ್ದು ಬಳಿಕ ಸ್ವಾಮೀಜಿ ನಕ್ಕು ಹೇಳಿದರು, “ಹೋಗಲಿ, ಬಾ; ಫ್ರಾನ್ಸಿಸ್ಸನದು ಏನೇನು ಉಳಿದಿದೆಯೋ (ಎಂದರೆ ಮೂಳೆಗಳು ಇತ್ಯಾದಿ!) ಆರಿಸಿಕೊಳ್ಳೋಣ.”

ವಿಷಯವೇನೆಂದರೆ ಲೆಗೆಟ್ಟರೊಂದಿಗೆ ತುಂಬ ಆತ್ಮೀಯರಾಗಿದ್ದ ಸ್ವಾಮೀಜಿ, ಅವರೊಂದಿಗೆ ಕೆಲವು ಕಸರತ್ತುಗಳನ್ನು ಮಾಡುತ್ತಿದ್ದರು. ಮೆಕ್​ಲಾಡ್ ಕೋಣೆಯೊಳಕ್ಕೆ ಪ್ರವೇಶಿಸುವ ವೇಳೆಗೆ ಸ್ವಾಮೀಜಿ ಬೆನ್ನ ಮೇಲೆ ಮಲಗಿಕೊಂಡು ಮೇಲೆ ಚಾಚಿದ ತಮ್ಮ ಕಾಲುಗಳ ಮೇಲೆ ಲೆಗಟ್ಟರನ್ನು ನಿಲ್ಲಿಸಿಕೊಂಡಿದ್ದರು! ಆಗ ಸ್ವಾಮೀಜಿಯ ಷರಾಯಿಯ ತುದಿ, ಮೊಳಕಾಲಿನವರೆಗೆ ಜೋತು ಬಿದ್ದಿತ್ತು. ಇದ್ದಕ್ಕಿದ್ದಂತೆ ಒಬ್ಬಳು ಹೆಂಗಸು ಅಲ್ಲಿ ಕಾಣಿಸಿಕೊಂಡಾಗ ಸ್ವಾಮೀಜಿ ಸಂಕೋಚ ಗೊಂಡು ತಕ್ಷಣ ಎದ್ದುನಿಂತರು. ಆಗ ಮೇಲಿನಿಂದ ಹಾರಿಬಿದ್ದ ಲೆಗೆಟ್ಟರನ್ನು ಕಂಡು ಮೆಕ್ ಲಾಡ್, ಅವರು ಹಾರಾಡುತ್ತಿದ್ದರೆಂದು ಊಹಿಸಿದ್ದಳು!

ರಿಡ್ಜ್​ಲಿ ಮ್ಯಾನರಿಗೆ ಸೇರಿದಂತೆ ಒಂದು ದೊಡ್ಡ ಗಾಲ್ಫ್ ಮೈದಾನ, ಕೆಲವು ಟೆನ್ನಿಸ್ ಕ್ರೀಡಾಂಗಣಗಳು, ಒಂದು ಕಸರತ್ತಿನ ಕೋಣೆ ಮುಂತಾದವೆಲ್ಲ ಇದ್ದುವು. ಒಂದು ದಿನ ಸ್ವಾಮೀಜಿ, ಬೆಸ್ಸಿಯ ಮಗ ಹಾಲಿಸ್ಟರ್​ನೊಂದಿಗೆ ಗಾಲ್ಫ್ ಮೈದಾನದಲ್ಲಿ ಅಡ್ಡಾಡಲು ಹೊರಟಿ ದ್ದರು. ಗಾಲ್ಪ್ ಮೈದಾನವೆಂದರೆ ಅದೊಂದು ತುಂಬ ವಿಶಾಲವಾದ ಬಯಲು. ಇದರಲ್ಲಿ ಪರಸ್ಪರ ತುಂಬ ದೂರದಲ್ಲಿರುವ ‘ಬದ್ದು’ಗಳು (ಸಣ್ಣ ಕುಳಿಗಳು)ಇರುತ್ತವೆ. ಆಟಗಾರರು ದಾಂಡಿನಿಂದ ಚೆಂಡನ್ನು ಬೀಸಿ ಹೊಡೆದು ಈ ಬದ್ದುಗಳೊಳಗೆ ಬೀಳುವಂತೆ ಮಾಡಬೇಕು. ಇದಕ್ಕೆ ಶಕ್ತಿಯೂ ನಿಖರತೆಯೂ ಬೇಕು. ಹೀಗೆ ಬದ್ದಿನೊಳಗೆ ಹಾಕಿದರೆ ಆಟಗಾರನಿಗೆ, ಅವನು ಮಾಡಿದ ಪ್ರಯತ್ನ ಗಳ ಸಂಖ್ಯೆಯ ಆಧಾರದ ಮೇಲೆ ಶ್ರೇಯಾಂಕಗಳು ಲಭಿಸುತ್ತವೆ. ದೂರದೂರದಲ್ಲಿರುವ ಬದ್ದು ಗಳ ಸ್ಥಾನವನ್ನು ಸೂಚಿಸಲು ಸಣ್ಣ ಬಾವುಟಗಳಿರುತ್ತವೆ. ಆದರೆ ಸ್ವಾಮೀಜಿಗೆ ಗಾಲ್ಫ್ ಆಟವಾಗಲಿ ಅದರ ನಿಯಮಾವಳಿಗಳಾಗಲಿ ತಿಳಿಯದು. ಮೈದಾನದಲ್ಲಿ ಅಡ್ಡಾಡುವಾಗ ದೂರದಲ್ಲೊಂದು ಬಾವುಟ ಹಾರಾಡುವುದನ್ನು ಕಂಡು, “ಅದೇಕೆ ಆ ಬಾವುಟವನ್ನು ಅಲ್ಲಿ ಹಾರಿಸಿದ್ದೀರಿ?” ಎಂದು ಹಾಲಿಸ್ಟರ್​ನನ್ನು ಕೇಳಿದರು. ಆ ಹುಡುಗ ಅವರಿಗೆ ಆಟವನ್ನು ವಿವರಿಸಿ, ಬಳಿಕ ಅದನ್ನು ತೋರಿಸಿ ಕೊಡಲು ಓಡಿಹೋಗಿ ದಾಂಡು-ಚೆಂಡುಗಳನ್ನು ತಂದ. ಎದುರಿನಲ್ಲಿ ಕಾಣುತ್ತಿರುವ ಬದ್ದಿಗೆ ನಿಗದಿತವಾಗಿರುವ ಶ್ರೇಯಾಂಕ ಇಂತಿಷ್ಟು ಎಂದು ವಿವರಿಸಿದ. ಸ್ವಾಮೀಜಿ ಮುಗುಳ್ನಕ್ಕರು. “ನಾನು ನಿನ್ನೊಂದಿಗೆ ಪಂಥ ಕಟ್ಟುತ್ತೇನೆ; ಈಗ ನಾನು ಒಂದೇ ಏಟಿಗೆ ಚೆಂಡನ್ನು ಬದ್ದಿನೊಳಕ್ಕೆ ಹಾಕುತ್ತೇನೆ. ನಿನ್ನ ಪಂಥವೇನು ಹೇಳು” ಎಂದರು. ಬಾಲಕ ಹಾಲಿಸ್ಟರ್ ೫ಂ ಸೆಂಟ್ (ಅರ್ಧ ಡಾಲರ್​) ಮುಂದಿಟ್ಟ. ಸ್ವಾಮೀಜಿ ಒಂದು ಡಾಲರ್ ತೆಗೆದರು. ಆ ವೇಳೆಗೆ ಅಲ್ಲಿಗೆ ಬಂದ ಲೆಗೆಟ್ಟರು “ಏನು ಸಮಾಚಾರ” ಎಂದು ವಿಚಾರಿಸಿದರು. ಪಂಥದ ವಿಷಯ ತಿಳಿದಾಗ ಅವರು, “ಸ್ವಾಮೀಜಿ, ಅದು ಬಹಳ ಕಷ್ಟ. ಎಂತೆಂಥಾ ಅನುಭವಿಗಳಿಗೂ ಅದು ಸಾಧ್ಯವಿಲ್ಲ” ಎಂದರು. ಆಗ ಸ್ವಾಮೀಜಿ, “ಇರಲಿ; ನೀವೆಷ್ಟು ಕಟ್ಟುತ್ತೀರಿ ಹೇಳಿ ಈಗ” ಎಂದರು. ಲೆಗೆಟ್ ಹತ್ತು ಡಾಲರ್ ತೆಗೆದಿಟ್ಟರು. “ಆಗಬಹುದು” ಎಂದರು ಸ್ವಾಮೀಜಿ. ಅದೊಂದು ಭಾರೀ ಮೊತ್ತವೇ ಸರಿ–ಈಗಿನ (೧೯೯೯) ಲೆಕ್ಕ ದಲ್ಲಿ ಸುಮಾರು ೪೩ಂ ರೂಪಾಯಿ!

ಹೋಗಿ ಬಾವುಟದ ಬಳಿ ನಿಲ್ಲುವಂತೆ ಸ್ವಾಮೀಜಿ ಹಾಲಿಸ್ಟರ್​ನಿಗೆ ಹೇಳಿದರು. ಬಳಿಕ ಅಂಗಿಯ ತೋಳನ್ನು ಮೇಲಕ್ಕೆ ಸರಿಸಿದರು. ಒಮ್ಮೆ ಗುರಿ ನೋಡಿ ಜೋರಾಗಿ ಬೀಸಿದರು. ಚೆಂಡು ನೇರವಾಗಿ ಹೋಗಿ ಬದ್ದನ್ನು ಸೇರಿತು.

ಸ್ವಲ್ಪ ಸುಧಾರಿಸಿಕೊಂಡ ಮೇಲೆ ಲೆಗೆಟ್ ಕೇಳಿದರು:

“ಸ್ವಾಮೀಜಿ, ನಿಮ್ಮ ಯೋಗಕ್ಕೂ ಇದಕ್ಕೂ ಏನಾದರೂ ಸಂಬಂಧ...”

“ಇಂಥಾ ಸಣ್ಣಪುಟ್ಟ ವಿಷಯಕ್ಕೆಲ್ಲಾ ನಾನು ಯೋಗವನ್ನು ಬಳಸುವುದಿಲ್ಲ. ನಾನು ಏನು ಮಾಡಿದೆ ಎಂಬುದನ್ನು ಎರಡೇ ವಾಕ್ಯದಲ್ಲಿ ಹೇಳುತ್ತೇನೆ: ಮೊದಲು ನಾನು ಕಣ್ಣಲ್ಲೇ ದೂರವನ್ನು ಅಳೆದೆ, ಮತ್ತು ನನ್ನ ತೋಳಿನ ಬಲ ನನಗೆ ಗೊತ್ತಿದೆ. ಎರಡನೆಯದಾಗಿ, ಚೆಂಡನ್ನು ಗುರಿ ಮುಟ್ಟಿಸಿದರೆ ನಾನು ಹತ್ತೂವರೆ ಡಾಲರ್​ನಷ್ಟು ಶ್ರೀಮಂತನಾಗುತ್ತೇನೆ ಎಂದು ನನ್ನ ಮನಸ್ಸಿಗೆ ಹೇಳಿದೆ. ಬಳಿಕ ದಾಂಡು ಬೀಸಿದೆ.”

ರಿಡ್ಜ್​ಲಿಯಲ್ಲಿ ನಡೆದ ಮತ್ತೊಂದು ಕೌತುಕವನ್ನು ಮುಂದೆ ಹಾಲಿಸ್ಟರ್ ತಿಳಿಸುತ್ತಾನೆ. ಒಮ್ಮೆ ಅವನು ಸ್ವಾಮೀಜಿಯ ಕೋಣೆಯ ಮುಂದೆ ಹಾದು ಹೋಗುತ್ತಿದ್ದ. ಕೋಣೆಯ ಬಾಗಿಲು ಮುಚ್ಚಿತ್ತು. ಆದರೆ ಒಳಗಿನಿಂದ ನಗೆ ಅಲೆಅಲೆಯಾಗಿ ಕೇಳಿಬರುತ್ತಿತ್ತು. ಸ್ವಾಮೀಜಿ ಕೋಣೆಯಿಂದ ಆಚೆಗೆ ಬಂದಾಗ ಹಾಲಿಸ್ಟರ್ ಕೇಳಿದ, “ಸ್ವಾಮೀಜಿ, ನೀವು ಯಾರೊಂದಿಗೆ ಮಾತನಾಡುತ್ತಿ ದ್ದುದು?” ಸ್ವಾಮೀಜಿ ಹೇಳಿದರು, “ನಾನೊಬ್ಬನೇ ಧ್ಯಾನನಿರತನಾಗಿದ್ದೆ.” ಹಾಲಿಸ್ಟರ್​ನಿಗೆ ಅದು ತಿಳಿದೇ ಇತ್ತು. ಆದ್ದರಿಂದ ಅವನು ಬಿಡದೆ ಮತ್ತೆ ಕೇಳಿದ, “ಆದರೆ ನಗುವಿನ ಶಬ್ದ ಕೇಳುತ್ತಿ ತ್ತಲ್ಲ?!” ಆಗ ಸ್ವಾಮೀಜಿ ಒಂದು ಕ್ಷಣ ಚಿಂತಿಸಿ ಅದನ್ನು ನೆನಪಿಸಿಕೊಂಡು ಹೇಳಿದರು, “ಓಹ್, ದೇವರು ಅದೆಷ್ಟು ತಮಾಷೆಯವನು \eng{!” (God is so funny!)}

ಮನಸ್ಸು ತುಂಬ ಮೃದುವಾಗಿ-ಸೂಕ್ಷ್ಮವಾಗಿರುವ ಎಳೆ ವಯಸ್ಸಿನಲ್ಲಿ ಇಂತಹ ಅನೇಕ ಘಟನೆಗಳು ಸಹಜವಾಗಿ ನಡೆದುಕೊಂಡು ಹೋಗುವುದನ್ನು ಕಂಡವರು ಮುಂದೆ ಎಂದೆಂದಿಗೂ ಸ್ವಾಮೀಜಿಯಲ್ಲಿ–ಮತ್ತು ದೇವರಲ್ಲಿ–ನಂಬಿಕೆಯಿಡಲೇ ಬೇಕು. ಹಾಲಿಸ್ಟರ್ ದೊಡ್ಡವನಾದ ಮೇಲೆ ತತ್ತ್ವವಾದದ ಬಗ್ಗೆ ಅಷ್ಟೇನೂ ತಲೆ ಕೆಡಿಸಿಕೊಳ್ಳಲಿಲ್ಲ. ಆದರೆ ಮುಂದೊಮ್ಮೆ ಅವನ ಪುಟ್ಟ ಮಗ, ಧರ್ಮ-ದೇವರು ಇವುಗಳ ವಿಷಯಾಗಿ ಅವನನ್ನು ಪ್ರಶ್ನಿಸಿದಾಗ, ಹಾಲಿಸ್ಟರ್ ಹೇಳುತ್ತಾನೆ, “ನಿಜ ಹೇಳಬೇಕೆಂದರೆ, ನಾನು ಆ ವಿಷಯವಾಗೆಲ್ಲ ಆಲೋಚಿಸಿಯೇ ಇಲ್ಲ. ಆದರೆ ಒಬ್ಬ ಭಗವಂತ ಇದ್ದಾನೆ ಎಂದು ಮಾತ್ರ ನನಗೆ ಗೊತ್ತು; ಏಕೆಂದರೆ ಸ್ವಾಮೀಜಿ ಹಾಗೆ ಹೇಳಿದರು.”

ಲೆಗೆಟ್ಟರ ಮೇಲೂ ಸ್ವಾಮೀಜಿ ಬೀರಿದ ಪ್ರಭಾವ ಅಷ್ಟೇ ಆಳವಾಗಿತ್ತು. ಆದರೆ ಭಾರೀ ಉದ್ಯಮಿಯಾಗಿದ್ದ ಲೆಗೆಟ್ಟರ ಮನೋಭಾವ ಸಂಪೂರ್ಣ ವಿಭಿನ್ನವಾದದ್ದು. ಸ್ವಾಮೀಜಿಯೊಂದಿಗೆ ಎರಡು ವಾರಗಳನ್ನು ಕಳೆದ ಮೇಲೆ ಒಂದು ಸಲ ಅವರು ಉದ್ಗರಿಸಿದರಂತೆ, “ವಿವೇಕಾನಂದರು ನಾನು ನೋಡಿದವರಲ್ಲೆಲ್ಲ ಅತ್ಯಂತ ಶ್ರೇಷ್ಠರಾದವರು” ಎಂದು. ಆಗ ಅಲ್ಲಿದ್ದವರೊಬ್ಬರು, “ನಿಮಗೆ ಹಾಗೆನ್ನಿಸಿದ್ದೇಕೆಂದು ಹೇಳಬಲ್ಲಿರಾ?” ಎಂದು ಕೇಳಿದಾಗ ಲೆಗೆಟ್ಟರು ಹೇಳಿದರು– “ಏಕೆಂದರೆ, ನಾನು ನೋಡಿದವರಲ್ಲೆಲ್ಲ ಅತಿ ಹೆಚ್ಚಿನ ವಿವೇಕಪ್ರಜ್ಞೆ \eng{(Common Sense)} ಅವರಲ್ಲಿದೆ.”

ಆಹ್ಲಾದಕರವಾದ ಎರಡು ವಾರಗಳನ್ನು ಕಳೆದ ಸ್ವಾಮೀಜಿ ರಿಡ್ಜ್​ಲಿ ಮ್ಯಾನರಿನಿಂದ ನ್ಯೂ ಯಾರ್ಕಿಗೆ ಹಿಂದಿರುಗಿದರು. ಇಲ್ಲಿ ಅವರಿಗೊಂದು ಆಶ್ಚರ್ಯ ಕಾದಿತ್ತು. ಅಲ್ಲಿ ಅವರೊಂದಿಗೆ ವಾಸವಾಗಿದ್ದ ಅವರ ಶಿಷ್ಯ ಲ್ಯಾಂಡ್ಸ್​ಬರ್ಗ್ ಹೇಳದೆ ಕೇಳದೆ ಹೊರಟುಹೋಗಿದ್ದ. ಶ್ರೀಮತಿ ಸಾರಾ ಬುಲ್​ಗೆ ಬರೆದ ಒಂದು ಪತ್ರದಲ್ಲಿ ಸ್ವಾಮೀಜಿ ಅವನ ಬಗ್ಗೆ ಹೀಗೆ ಪ್ರಸ್ತಾಪಿಸುತ್ತಾರೆ, “ಪಾಪ! ಲ್ಯಾಂಡ್ಸ್​ಬರ್ಗ್ ಇಲ್ಲಿಂದ ಹೊರಟು ಹೋಗಿದ್ದಾನೆ. ತನ್ನ ವಿಳಾಸವನ್ನೂ ಅವನು ಬಿಟ್ಟುಹೋಗಿಲ್ಲ. ಅವನು ಎಲ್ಲಿಗೇ ಹೋಗಿರಲಿ, ಭಗವಂತ ಅವನನ್ನು ಚೆನ್ನಾಗಿಟ್ಟಿರಲಿ. ನಾನು ಈ ಜನ್ಮದಲ್ಲಿ ನೋಡಿದ ಅತ್ಯಂತ ಪ್ರಾಮಾಣಿಕ-ವಿಶ್ವಾಸಪಾತ್ರ ವ್ಯಕ್ತಿಗಳಲ್ಲಿ ಅವನೊಬ್ಬ.” ಆದರೆ ಮುಂದೆ ಕೆಲ ಕಾಲದಲ್ಲಿ ಸ್ವಾಮೀಜಿ \eng{Thousand Island Park}ಗೆ ಹೋದಾಗ (ಇದನ್ನು ‘ಸಹಸ್ರದ್ವೀಪೋದ್ಯಾನ’ ಎಂದು ಈಗಾಗಲೇ ಕನ್ನಡಿಸಲಾಗಿರುವುದಿಂದ ಇನ್ನು ಮುಂದೆ ಇದನ್ನು ಹಾಗೆಯೇ ಕರೆಯೋಣ.) ಈತ ಮತ್ತೆ ಅಲ್ಲಿ ಕಾಣಿಸಿಕೊಂಡ. ಸ್ವಾಮೀಜಿಯೂ ಅವನ ಮೇಲೆ ಕೃಪೆ ಮಾಡಿ ಸಂನ್ಯಾಸ ನೀಡಿದರು–ಹೆಸರು, ಸ್ವಾಮಿ ಕೃಪಾನಂದ. ಅಲ್ಲದೆ ಈ ಹಿಂದೆ ಹೇಳಲಾದ ಮೇಡಂ ಮೇರಿ ಲೂಯಿಸ್​ಳಿಗೂ ಸಂನ್ಯಾಸ ನೀಡಿ, ಸ್ವಾಮಿ ಅಭಯಾನಂದಾ ಎಂಬ ಹೆಸರನ್ನು ಕೊಟ್ಟರು. ಇವರಿಬ್ಬರೂ ಕೆಲ ವರ್ಷಗಳ ಕಾಲ ಸ್ವಾಮೀಜಿಯ ಕಾರ್ಯವನ್ನು ಉತ್ಸಾಹದಿಂದ ಮಾಡಿದರಾದರೂ ಮುಂದೆ ಇಬ್ಬರೂ ಸ್ವಾರ್ಥಪ್ರೇರಿತರಾಗಿ ಅವರಿಂದ ದೂರವಾದರು.

ನ್ಯೂಯಾರ್ಕಿನಲ್ಲಿ ಸ್ವಾಮೀಜಿ ಸಾರ್ವಜನಿಕ ಉಪನ್ಯಾಸಗಳ ಮೂಲಕ ಧನಸಂಗ್ರಹಣೆ ಮಾಡದೆ, ತರಗತಿಗಳನ್ನಷ್ಟೇ ನಡೆಸಿಕೊಂಡು ಹೋಗುತ್ತಿದ್ದರೆಂಬುದನ್ನು ನೋಡಿದ್ದೇವೆ. ತರಗತಿ ಗಳಿಗೆ ಬರುತ್ತಿದ್ದ ವಿದ್ಯಾರ್ಥಿಗಳು ತಾವಾಗಿಯೇ ನೀಡುತ್ತಿದ್ದ ವಂತಿಗೆಯಿಂದ ಅದು ಹೇಗೋ ಸಾಗಿತಾದರೂ ಸ್ವಾಮೀಜಿಯ ಮೇಲೆ ಇದರಿಂದ ತೀವ್ರ ಆರ್ಥಿಕ ಒತ್ತಡ ಬಿದ್ದಿತು. ಆದರೆ ಅದನ್ನವರು ಲೆಕ್ಕಿಸಲಿಲ್ಲ. ಏಕೆಂದರೆ ಅವರ ಮುಖ್ಯವಾದ ಕಾಳಜಿಯೆಂದರೆ ತಮ್ಮ ಶಿಷ್ಯರ ಆಧ್ಯಾತ್ಮಿಕ ಉನ್ನತಿ. ಈ ಅವಧಿಯ ಅಂತ್ಯದ ವೇಳೆಗೆ ಬರೆದ ಪತ್ರವೊಂದರಲ್ಲಿ ಸ್ವಾಮೀಜಿ ಹೇಳುತ್ತಾರೆ, “ಆರ್ಥಿಕವಾಗಿ ಈ ಚಳಿಗಾಲದಲ್ಲಿ ದೊರೆತ ಯಶಸ್ಸು ಏನೇನೂ ಇಲ್ಲ. ನನ್ನನ್ನು ನಾನು ನೋಡಿಕೊಳ್ಳುವುದೇ ಕಷ್ಟವಾಗಿತ್ತು. ಆದರೆ ಆಧ್ಯಾತ್ಮಿಕವಾಗಿ ಅದು ಅತ್ಯಂತ ಯಶಸ್ವಿಯಾಯಿತು.”

ಆಧ್ಯಾತ್ಮಿಕತೆಯೇ ಅವರ ಪ್ರತಿಯೊಂದು ಕಾರ್ಯಯೋಜನೆಯ ತಳಹದಿಯಾಗಿತ್ತು. ಆದ್ದ ರಿಂದಲೇ ಅವರು ಆಧ್ಯಾತ್ಮಿಕ ಶಿಕ್ಷಣಕ್ಕೆ ಅಷ್ಟೊಂದು ಮಹತ್ವವನ್ನು ನೀಡಿದ್ದು, ಮತ್ತು ಅಷ್ಟೊಂದು ಶ್ರಮಿಸಿದ್ದು. ಇಂತಹ ಆಧ್ಯಾತ್ಮಿಕ ಶಿಕ್ಷಣವು ಅನಿರ್ಬಂಧಿತವಾಗಿ ಸಾಗಬೇಕಾದರೆ ಸಂಸ್ಥೆಯೊಂದನ್ನು ನಿರ್ಮಿಸಲೇಬೇಕೆಂದು ಅವರಿಗೆ ಮನದಟ್ಟಾಗಿತ್ತು. ತಮ್ಮ ಕಾರ್ಯಯೋಜನೆ ಗಳ ಭವಿಷ್ಯದ ಬಗ್ಗೆ ಅವರು ಎಷ್ಟು ಆಳವಾಗಿ ಚಿಂತಿಸಿದ್ದರೆಂಬುದನ್ನು ಅವರು ಫೆಬ್ರವರಿಯಲ್ಲಿ ಶ್ರೀಮತಿ ಬುಲ್​ಗೆ ಬರೆದ ಪತ್ರದಲ್ಲಿ ಕಾಣಬಹುದು:

“ನನ್ನ ಗುರುದೇವರು ಹೇಳುತ್ತಿದ್ದರು–ಈ ಹಿಂದೂ, ಕ್ರೈಸ್ತ ಎಂಬ ಹೆಸರುಗಳೆಲ್ಲ ಮನುಷ್ಯ ಮನುಷ್ಯರ ನಡುವಣ ಭ್ರಾತೃತ್ವಭಾವನೆಗೆ ದೊಡ್ಡ ಪ್ರತಿಬಂಧಕಗಳಾಗಿ ನಿಲ್ಲುತ್ತವೆ, ಎಂದು. ನಾವು ಮೊದಲು ಅವುಗಳನ್ನು ಕುಟ್ಟಿ ಕೆಡಹಬೇಕು. ಈಗ ಅವು ತಮ್ಮ ಒಳ್ಳೆಯ ಪ್ರಭಾವವನ್ನೆಲ್ಲ ಕಳೆದುಕೊಂಡು ಕೇವಲ ಕಂಟಕಗಳಾಗಿ ಉಳಿದುಕೊಂಡಿದೆ. ಅವುಗಳ ಪೈಶಾಚಿಕ ಪ್ರಭಾವಗಳಿಗೆ ಒಳಗಾಗಿ, ನಮ್ಮಲ್ಲಿ ಉತ್ತಮರೆನಿಸಿಕೊಂಡವರು ಕೂಡ ರಾಕ್ಷಸರಂತೆ ವರ್ತಿಸುತ್ತಾರೆ. ನಾವು ಕಷ್ಟಪಟ್ಟು ಕೆಲಸ ಮಾಡಿ ಜಯಶಾಲಿಗಳಾಗಬೇಕಾಗಿದೆ.

“ಆದ್ದರಿಂದಲೇ ಒಂದು ಸಂಸ್ಥೆಯನ್ನು ನಿರ್ಮಿಸಬೇಕೆಂದು ನಾನು ಅಷ್ಟೊಂದು ಇಷ್ಟ ಪಡುವುದು. ಸಂಸ್ಥೆ ಎಂದಮೇಲೆ ಅದಕ್ಕೆ ತನ್ನದೇ ಆದ ದೋಷಗಳಿರುತ್ತವೆ ನಿಜ. ಆದರೆ ಸಂಸ್ಥೆಯಿಲ್ಲದೆ ಏನೇನೂ ಸಾಧ್ಯವಿಲ್ಲ. ಏಕಕಾಲದಲ್ಲಿ ಸಮಾಜವನ್ನೂ ಸಂತೋಷಪಡಿಸಿಕೊಂಡು ಮಹಾಕಾರ್ಯಗಳನ್ನೂ ಮಾಡುವಲ್ಲಿ ಎಂದೂ ಯಾರೂ ಯಶಸ್ವಿಯಾಗಿಲ್ಲ. ತನ್ನೊಳಗಿನಿಂದ ಆದೇಶ ಬಂದಂತೆ ವ್ಯಕ್ತಿ ಕೆಲಸ ಮಾಡಬೇಕು. ಅವನ ಕೃತ್ಯವು ಸರಿಯಾದದ್ದೂ ಒಳ್ಳೆಯದೂ ಆಗಿ ದ್ದರೆ, ಅವನು ಸತ್ತುಹೋದ ಒಂದು ಶತಮಾನದ ಅನಂತರವಾದರೂ ಸಮಾಜ ಅದನ್ನು ಒಪ್ಪಿ ಕೊಳ್ಳಲೇಬೇಕು. ನಮ್ಮ ಕಾರ್ಯಕ್ಕೆ ನಾವು ನಮ್ಮ ತನುಮನಗಳನ್ನು ಅರ್ಪಿಸಿಬಿಡಬೇಕು. ನಾವು ನಮ್ಮ ಸರ್ವಸ್ವವನ್ನೂ ಆ ಏಕಮಾತ್ರ ಆದರ್ಶಕ್ಕೆ ಸಮರ್ಪಿಸಲು ಸಿದ್ಧರಾಗುವವರೆಗೂ ನಾವು ಬೆಳಕನ್ನು ಕಾಣಲಾರೆವು. ಮಾನವತೆಗೆ ಸಹಾಯ ಮಾಡಬಯಸುವವರು ತಮ್ಮ ಸ್ವಂತ ಸುಖ- ದುಃಖ, ಹೆಸರು-ಕೀರ್ತಿ ಮತ್ತು ಎಲ್ಲ ಬಗೆಯ ಸ್ವಾರ್ಥೋದ್ದೇಶಗಳನ್ನೂ ಗಂಟು ಕಟ್ಟಿ ಸಮುದ್ರ ಕ್ಕೆಸೆದು ಅನಂತರ ಭಗವಂತನ ಬಳಿಗೆ ಬರಬೇಕು. ಎಲ್ಲ ಮಹಾತ್ಮರೂ ಹೇಳಿದುದೂ ಮಾಡಿದುದೂ ಇದನ್ನೇ.”

ಸಂಸ್ಥೆಯನ್ನು ನಿರ್ಮಿಸುವುದರ ಮಹತ್ವವನ್ನು ಮನಗಂಡಿದ್ದ ಸ್ವಾಮೀಜಿ, ಅದಾಗಲೇ ‘ವೇದಾಂತ ಸೊಸೈಟಿ’ ಎಂಬ ಹೆಸರಿನಲ್ಲಿ ಒಂದು ಸಂಸ್ಥೆಯನ್ನು ಸ್ಥಾಪಿಸಿದರು. ಆದರೆ ಆಗ ಅದು, ಅವರಿಗೆ ನೆರವಾಗುತ್ತಿದ್ದ ಕೆಲವು ಸ್ನೇಹಿತರ ಒಕ್ಕೂಟವಷ್ಟೇ. ಕಾಲಕ್ರಮದಲ್ಲಿ ಈ ಸಂಸ್ಥೆ ಬೃಹತ್ಪ್ರಮಾಣದಲ್ಲಿ ಬೆಳೆಯುತ್ತದೆ ಎಂಬುದರಲ್ಲಿ ಸ್ವಾಮೀಜಿಗೆ ದೃಢವಿಶ್ವಾಸವಿತ್ತು. ತಮ್ಮ ಕಾರ್ಯ ಯಶಸ್ವಿಯಾಗುವುದರಲ್ಲಿ ಅವರಿಗೆ ಸಂದೇಹವೇ ಇರಲಿಲ್ಲ. ಆದರೆ ಈ ನಡುವೆ ಅವರು ಸಾರ್ವಜನಿಕ ಜೀವನವನ್ನು ತ್ಯಜಿಸಿ, ಸಂನ್ಯಾಸಿಯ ಸರಳ-ನಿರಾತಂಕ ಜೀವನವನ್ನು ಕೈಗೊಂಡ ದ್ದನ್ನು ಕಂಡೋ ಏನೋ, ಅವರ ಕೆಲವು ವಿಶ್ವಾಸಿಗಳು-ಹಿತೈಷಿಗಳು ಅವರ ಕಾರ್ಯವಿಧಾನದ ಕುರಿತಾಗಿ ಸಲಹೆಗಳನ್ನು ನೀಡುತ್ತಿದ್ದರು. ಇವರೆಲ್ಲ ಸದುದ್ದೇಶದಿಂದಲೇ ಹಾಗೆ ಮಾಡಿರಬಹುದು; ಆದರೆ ನಿಜಕ್ಕೂ ಸ್ವಾಮೀಜಿಗೆ ಆ ಸಲಹೆಗಳ ಆವಶ್ಯಕತೆಯಿತ್ತೆ ಎಂಬುದನ್ನು ಮಾತ್ರ ಇವರು ಆಲೋಚಿಸಿರಲಿಲ್ಲ. ಉದಾಹರಣೆಗೆ ಬಾಸ್ಟನ್ನಿನ ಒಬ್ಬಳು ಆಡಂಬರದ ಮಹಿಳೆ ಅವರಿಗೊಂದು ಪುಕ್ಕಟೆ ಸಲಹೆ ನೀಡಿದಳು–“ಸ್ವಾಮೀಜಿ, ನೀವು ಭಾಷಣಕಲೆಯ ವಿಷಯವಾಗಿ ನುರಿತವರಿಂದ ಹೇಳಿಸಿಕೊಳ್ಳಿ” ಎಂದು. ಅವರು ಭಾಷಣಕಲೆಯನ್ನು ಶಿಸ್ತುಬದ್ಧವಾಗಿ ಕಲಿತರೆ ಅವರು ಮತ್ತಷ್ಟು ಚೆನ್ನಾಗಿ ಜನರನ್ನು ಆಕರ್ಷಿಸಲು ಸಮರ್ಥರಾಗಬಲ್ಲರು ಎಂಬುದು ಅವಳ ನಂಬಿಕೆ. ಅಮೆರಿಕ ದಲ್ಲೆಲ್ಲ ‘ದೈವದತ್ತ ವಾಗ್ಮಿ’ ಎಂದು ಬಣ್ಣಿಸಲ್ಪಟ್ಟ, ಸರ್ವಧರ್ಮ ಸಮ್ಮೇಳನದ ಸಭೆಗೆ ಸಭೆಯನ್ನೇ ಅಲುಗಾಡಿಸಿಬಿಟ್ಟ ಸ್ವಾಮೀಜಿಗೆ ಈಕೆ, ಭಾಷಣಕಲೆಯನ್ನು ಇತರರಿಂದ ಕಲಿತುಕೊಳ್ಳು ವಂತೆ ಹೇಳುತ್ತಾಳೆ! ಇನ್ನೊಬ್ಬ ವ್ಯಕ್ತಿ ಅವರಿಗೆ, ಹೇಗೆ ಸಂಘಟನೆ ಮಾಡಬೇಕು ಎಂಬುದರ ಬಗ್ಗೆ ಒಂದೇ ಸಮನೆ ಕೊರೆದ. ಮತ್ತೊಬ್ಬ ಹೇಳಿದ, “ಸ್ವಾಮೀಜಿ, ನೀವು ಸಮಾಜದ ಗಣ್ಯ ವ್ಯಕ್ತಿಗಳ ಮೇಲೆ ಪ್ರಭಾವ ಬೀರಲು ಸಾಧ್ಯವಾಗಬೇಕಾದರೆ, ನೀವು ಇನ್ನೂ ಠಾಕುಠೀಕಾಗಿರಬೇಕು.” ಬೇರೊಬ್ಬ ಸೂಚಿಸಿದ, “ನೀವು ಪಾದ್ರಿಗಳನನ್ನು ಸಮಾಧಾನಗೊಳಿಸಬೇಕು.” ಇಂತಹ ಸಲಹೆಗಳು ಅತಿಯಾದಾಗ ಸ್ವಾಮೀಜಿ ಅಸಹನೆಗೊಂಡು,“ಅಂಥ ಅಸಂಬದ್ಧಗಳನ್ನೆಲ್ಲ ನಾನೇಕೆ ಮಾಡ ಬೇಕು!” ಎಂದು ಗುಡುಗುತ್ತಿದ್ದರು. ಇಂತಹ ಒಂದು ಸಂದರ್ಭದಲ್ಲಿ ಅವರ ಸತ್ವಪೂರ್ಣ ವ್ಯಕ್ತಿತ್ವದಿಂದ ಹೊರಹೊಮ್ಮಿದ ಕಿಡಿಗಳನ್ನು ಅವರು ಮೇರಿಗೆ ಬರೆಯುವ ಪತ್ರದಲ್ಲಿ ಕಾಣಬಹುದು:

“ನಾನು ಸತ್ಯವನ್ನು ಅನಂತ ಶಕ್ತಿಯುಳ್ಳ ಒಂದು ಸುಡುವ ಪದಾರ್ಥಕ್ಕೆ ಹೋಲಿಸುತ್ತೇನೆ. ಅದು ತಾನೆಲ್ಲೇ ಬೀಳಲಿ, ಸುಡುತ್ತದೆ–ಮೃದು ಪದಾರ್ಥಗಳನ್ನು ಒಮ್ಮೆಗೇ ಸುಡುತ್ತದೆ. ಗಟ್ಟಿ ಪದಾರ್ಥಗಳನ್ನು ಸ್ವಲ್ಪ ನಿಧಾನವಾಗಿ ಸುಡುತ್ತದೆ; ಅಂತೂ ಅದು ಸುಟ್ಟೇ ಸುಡುತ್ತದೆ. ನಿಯಮ ಯಾವಾಗಲೂ ನಿಯಮವೇ. ಸೋದರಿ, ನನ್ನನ್ನು ಕ್ಷಮಿಸು. ನಾನೆಂದಿಗೂ ಸುಳ್ಳಿನೊಂದಿಗೆ ಹೊಂದಾಣಿಕೆ ಮಾಡಿಕೊಳ್ಳಲಾರೆ; ಸುಳ್ಳಿನೊಂದಿಗೆ ಸಾಮರಸ್ಯದಿಂದಿರಲಾರೆ. ಅದು ನನಗೆ ಸಾಧ್ಯವೇ ಇಲ್ಲ. ಅದಕ್ಕಾಗಿ ನಾನು ನನ್ನ ಜೀವನವೆಲ್ಲ ಕಷ್ಟಪಟ್ಟಿದ್ದೇನೆ. ಪ್ರಯತ್ನದ ಮೇಲೆ ಪ್ರಯತ್ನಮಾಡಿದ್ದೇನೆ. ಆದರೂ ನನಗದು ಸಾಧ್ಯವಾಗಿಲ್ಲ. ದೇವರು ದೊಡ್ಡವನು. ನಾನೊಬ್ಬ ಆಷಾಢಭೂತಿಯಾಗಲು ಅವನು ಬಿಡುವುದಿಲ್ಲ. ಈಗ ಒಳಗೇನೇನಿದೆಯೋ ಎಲ್ಲ ಹೊರಕ್ಕೆ ಬಂದುಬಿಡಲಿ. ಎಲ್ಲರನ್ನೂ ಮೆಚ್ಚಿಸಬಲ್ಲ ಮಾರ್ಗವನ್ನು ನಾನಿನ್ನೂ ಕಂಡುಕೊಂಡಿಲ್ಲ. ಮತ್ತು ಈಗ ನಾನು ಏನಾಗಿರುವೆನೋ ಅದಕ್ಕಿಂತ ಬೇರೆಯಾಗಲು ಸಾಧ್ಯವಿಲ್ಲ... ಹೇ ಸತ್ಯಸ್ವರೂಪನಾದ ಭಗವಂತ, ನೀನೊಬ್ಬನೇ ನನಗೆ ಮಾರ್ಗದರ್ಶಕನಾಗಿರು. ಸೊಗಸುಗಾರಿಕೆಗೆ ಹಿಂದಿರುಗಲಾರ ದಷ್ಟು ವಯಸ್ಸಾಗಿದೆ ನನಗೆ. ಈಗ ನಾನು ಹೇಗಿರುವೆನೋ ಹಾಗೆಯೇ ಇರಲು ನನ್ನನ್ನು ಬಿಟ್ಟುಬಿಡಿ. ‘ಅಂಜಿಕೆಯಿಲ್ಲದೆ, ಲೆಕ್ಕಾಚಾರದ ಬುದ್ಧಿಯಿಲ್ಲದೆ, ಶತ್ರುಗಳನ್ನಾಗಲಿ ಮಿತ್ರರನ್ನಾಗಲಿ ಲೆಕ್ಕಿಸದೆ, ಓ ಸಂನ್ಯಾಸಿ, ನೀನು ಸತ್ಯವನ್ನು ಹಿಡಿದುಕೊ. ಈ ಕ್ಷಣದಿಂದಲೇ ಇಹಪರಗಳೆಲ್ಲವನ್ನು, ಮುಂಬರು ವುದೆಲ್ಲವನ್ನು, ಅವುಗಳ ಭೋಗವನ್ನು, ಪೊಳ್ಳುತನವನ್ನು ತ್ಯಜಿಸಿಬಿಡು. ಸತ್ಯ ಮಾತ್ರವೇ ನಿನಗೆ ದಾರಿ ತೋರಲಿ.’ ಸೋದರಿ, ಸಂಪತ್ತಿನಲ್ಲಾಗಲಿ, ಹೆಸರು ಕೀರ್ತಿಗಳಲ್ಲಾಗಲಿ, ಭೋಗದಲ್ಲಾಗಲಿ ನನಗೆ ಯಾವ ಆಸೆಯೂ ಇಲ್ಲ. ಅವೆಲ್ಲ ನನ್ನ ಪಾಲಿಗೆ ಧೂಳಿನ ಕಣ. ನಾನು ನನ್ನ ಸೋದರರಿಗೆ ಸ್ವಲ್ಪ ನೆರವಾಗಲು ಇಚ್ಛಿಸಿದೆ. ಹಣ ಸಂಪಾದಿಸುವ ಕೌಶಲ ನನ್ನಲ್ಲಿಲ್ಲ. ಭಗವಂತ ನನ್ನನ್ನು ಆಶೀರ್ವದಿಸಲಿ! ಸತ್ಯದ ಆದೇಶಕ್ಕೆ ತಲೆಬಾಗುವುದನ್ನು ಬಿಟ್ಟು ನಾನೇಕೆ ನನ್ನ ಸುತ್ತಲಿನ ಪ್ರಪಂಚದ ಹುಚ್ಚಾಟಗಳಿಗೆ ಹೊಂದಿಕೊಳ್ಳಬೇಕು? ಸೋದರಿ, ನನ್ನ ಮನಸ್ಸಿನ್ನೂ ದುರ್ಬಲ ವಾಗಿಯೇ ಇದೆ. ಅದು ಕೆಲವೊಮ್ಮೆ ಯಾಂತ್ರಿಕವಾಗಿ ಪ್ರಪಂಚದವರ ಸಹಾಯಕ್ಕೆ ಅಂಟಿ ಕೊಳ್ಳುತ್ತದೆ. ಆದರೆ ನನಗೆ ಹೆದರಿಕೆಯಿಲ್ಲ. ನನ್ನ ಧರ್ಮ ಸಾರುವ ಮಹಾಪಾಪ ಯಾವುದೆಂದರೆ ಹೆದರಿಕೆ-ಭಯ!

“ಮತಾಂಧ ಪಾದ್ರಿಗಳೊಂದಿಗಿನ ಕೊನೆಯ ಹೋರಾಟ ಮತ್ತು ಅನಂತರ ಶ್ರೀಮತಿ ಸಾರಾ ಳೊಂದಿಗಿನ ದೀರ್ಘತಿಕ್ಕಾಟ, ಇವೆಲ್ಲ ನನಗೆ ಮನು ಮಹರ್ಷಿಯು ಸಂನ್ಯಾಸಿಗೆ ಹೇಳುವುದನ್ನು ಮತ್ತಷ್ಟು ಸ್ಪಷ್ಟವಾಗಿಸಿದೆ. ಅವನು ಹೇಳುತ್ತಾನೆ, ‘ಏಕಾಂಗಿಯಾಗಿರು, ಏಕಾಂಗಿಯಾಗಿ ನಡೆ’ ಎಂದು. ಹೇ ನನ್ನಾತ್ಮ! ಅಚಲನಾಗಿರು, ಏಕಾಂಗಿಯಾಗಿರು! ಭಗವಂತ ನಿನ್ನೊಂದಿಗಿದ್ದಾನೆ, ಹೆದರಬೇಡ! ಸೋದರಿ, ದಾರಿ ಬಲು ದೂರ, ಸಮಯ ಅಲ್ಪ, ಸಂಧ್ಯೆ ಸಮೀಪಿಸುತ್ತಿದೆ. ನಾನು ಶೀಘ್ರದಲ್ಲೇ ಮನೆಗೆ ಮರಳಬೇಕಾಗಿದೆ. ನನ್ನ ನಡವಳಿಕೆಗಳನ್ನು ನಾಜೂಕು ಮಾಡಿಕೊಳ್ಳಲು ನನಗೆ ವ್ಯವಧಾನವಿಲ್ಲ. ನೀವು ಒಳ್ಳೆಯವರು, ವಿಶ್ವಾಸಿಗಳು. ನಾನು ನಿಮಗಾಗಿ ಏನನ್ನು ಬೇಕಾದರೂ ಮಾಡಬಲ್ಲೆ. ದಯವಿಟ್ಟು ಕೋಪಿಸಿಕೊಳ್ಳಬೇಡಿ. ನಿಮ್ಮನ್ನೆಲ್ಲ ನಾನು ಕೇವಲ ಮಕ್ಕಳೆಂಬಂತೆ ಕಾಣುತ್ತೇನೆ.

“ಕನಸು ಕಾಣದಿರು, ಹೇ ನನ್ನಾತ್ಮ! ಕನಸು ಕಾಣದಿರು. ಒಂದೇ ಒಂದು ಪದದಲ್ಲಿ ನನ್ನದೊಂದು ಸಂದೇಶವನ್ನು ಕೊಡುವುದಿದೆ. ಪ್ರಪಂಚದ ವಿಷಯದಲ್ಲಿ ‘ಮಧುರ’ವಾಗಿರಲು ನನಗೆ ಸಾಧ್ಯವಿಲ್ಲ. ಅಲ್ಲದೆ, ಹಾಗಿರುವಲ್ಲಿನ ಪ್ರತಿಯೊಂದು ಪ್ರಯತ್ನವೂ ನನ್ನನ್ನು ಆಷಾಢಭೂತಿ ಯನ್ನಾಗಿಸುತ್ತದೆ. ಪುಸುಕಲು ಜೀವನ ನಡೆಸುತ್ತ ಈ ಮೂರ್ಖ ಜಗತ್ತಿನ ಪ್ರತಿಯೊಂದು ಮಾತನ್ನೂ ಕೇಳಿಕೊಂಡಿರುವುದರ ಬದಲಾಗಿ ನಾನು ಸಾವಿರ ಸಲ ಸಾವನ್ನಪ್ಪಲು ಸಿದ್ಧನಿದ್ದೇನೆ. ಶ್ರೀಮತಿ ಬುಲ್ ಭಾವಿಸುವಂತೆ, ನಾನು ಮಾಡಬೇಕಾದ ಒಂದು ಕರ್ತವ್ಯವಿದೆ ಎಂದು ನೀವು ಭಾವಿಸಿದ್ದರೆ, ನೀವು ಬಹಳ ತಪ್ಪು ತಿಳಿದಿದ್ದೀರಿ. ಈ ಜಗತ್ತಿನಲ್ಲಾಗಲಿ ಇದರಾಚೆಯ ಜಗತ್ತಿ ನಲ್ಲಾಗಲಿ ನನಗಾವ ಕೆಲಸವೂ ಇಲ್ಲ. (ಆದರೆ) ನನ್ನದೊಂದು ಸಂದೇಶವಿದೆ. ಅದನ್ನು ನನ್ನದೇ ಆದ ರೀತಿಯಲ್ಲಿ ಕೊಡುತ್ತೇನೆ. ನಾನು ನನ್ನ ಸಂದೇಶವನ್ನು ಹಿಂದುತ್ವೀಕರಿಸುವುದೂ ಇಲ್ಲ, ಕ್ರೈಸ್ತೀಕರಿಸುವುದೂ ಇಲ್ಲ, ಅಥವಾ ಇನ್ನಾವ ಮತಕ್ಕೂ ಒಳಪಡಿಸುವುದಿಲ್ಲ. ನಾನದನ್ನು ಕೇವಲ ‘ನನ್ನ-ತನ’ಗೊಳಿಸುತ್ತೇನೆ ಅಷ್ಟೆ. ಬಿಡುಗಡೆ-ಮುಕ್ತಿ, ಇಷ್ಟೇ ನನ್ನ ಧರ್ಮ. ಅದನ್ನು ತಡೆಗಟ್ಟುವ ಪ್ರತಿಯೊಂದನ್ನೂ ಹೋರಾಟದಿಂದಲೋ ಪಲಾಯನದಿಂದಲೋ ನಿವಾರಿಸುತ್ತೇನೆ. ಥೂ! ನಾನು ಪಾದ್ರಿಗಳನ್ನು ಸಮಾಧಾನಪಡಿಸಲೆತ್ನಿಸುವುದೆ! ಸೋದರಿ, ದಯವಿಟ್ಟು ತಪ್ಪು ತಿಳಿಯಬೇಡ. ಆದರೆ ನೀವೆಲ್ಲ ಮಕ್ಕಳು; ಹೇಳಿದ ಮಾತು ಕೇಳಬೇಕು. ‘ತರ್ಕವನ್ನು ಅತರ್ಕವನ್ನಾಗಿಯೂ ಮರ್ತ್ಯವನ್ನು ಅಮರವನ್ನಾಗಿಯೂ ಪ್ರಪಂಚವನ್ನು ಶೂನ್ಯವನ್ನಾಗಿಯೂ ಮಾನವನನ್ನು ದೇವನ ನ್ನಾಗಿಯೂ ಮಾಡುವ, ಈ ಅಮೃತ ಸ್ರೋತವನ್ನು ನೀವಿನ್ನೂ ಪಾನಮಾಡಿಲ್ಲ. ‘ಜಗತ್ತು’ ಎಂದು ಜನರು ಕರೆಯುವ ಈ ಮೂರ್ಖತನದ ಬಲೆಯಿಂದ, ನಿಮಗೆ ಸಾಧ್ಯವಿದ್ದರೆ ಹೊರಗೆ ಬನ್ನಿ. ಆಗ ನಾನು ನಿಮ್ಮನ್ನು ನಿಜಕ್ಕೂ ಧೀರರು ಮತ್ತು ಮುಕ್ತರು ಎಂದು ಕರೆಯುತ್ತೇನೆ. ನಿಮಗದು ಸಾಧ್ಯವಾಗದಿದ್ದರೆ, ಈ ಸಮಾಜವೆಂಬ ಸುಳ್ಳು ದೇವರನ್ನು ನೆಲ್ಕಕೆ ಅಪ್ಪಳಿಸಿ ಅದರ ಕೊನೆಯಿಲ್ಲದ ಮಿಥ್ಯಾಚಾರವನ್ನು ಕಾಲಿನಡಿಯಲ್ಲಿ ಹೊಸಕಿಹಾಕಿದವರಿಗೆ ಜಯಘೋಷ ಮಾಡಿ. ನಿಮ್ಮಿಂದ ಅದೂ ಸಾಧ್ಯವಾಗದಿದ್ದರೆ ದಯವಿಟ್ಟು ಬಾಯಿಮುಚ್ಚಿಕೊಂಡಿರಿ; ಆದರೆ, ರಾಜಿಮಾಡಿಕೊಳ್ಳು ವುದು ನಾಜೂಕುಗಾರರಾಗುವುದು ಮೊದಲಾದ ಮೂರ್ಖತನಗಳ ಕೊಚ್ಚೆಗೆ ಅವರನ್ನು ಎಳೆಯ ಬೇಡಿ.

“ಈ ಜಗತ್ತನ್ನು, ಈ ಕನಸನ್ನು, ಈ ಮತಗಳನ್ನು-ಕುತರ್ಕಗಳನ್ನು, ನೀತಿಯನ್ನು-ನೀಚತನವನ್ನು, ಇದರ ಸುಂದರ ಮುಖಗಳನ್ನು-ಕೊಳೆತ ಹೃದಯಗಳನ್ನು, ಮೇಲ್ಮೇಲೆ ಬೊಬ್ಬಿಡುವ ನೈತಿಕತೆ ಯನ್ನು-ಅದರ ಒಳಗೊಳಗಿನ ಪಕ್ಕಾ ಪೊಳ್ಳುತನವನ್ನು, ಮತ್ತು ಎಲ್ಲಕ್ಕಿಂತ ಹೆಚ್ಚಾಗಿ, ಅದರ ‘ಪವಿತ್ರೀಕರಿಸಿದ’ ವ್ಯಾಪಾರೀ ಬುದ್ಧಿಯನ್ನು ನಾನು ದ್ವೇಷಿಸುತ್ತೇನೆ. ಏನು! ಈ ಪ್ರಪಂಚದ ಗುಲಾಮರು ಹೇಳುವ ಅಳತೆಗೋಲಿಗೆ ತಕ್ಕಂತೆ ನನ್ನನ್ನು ನಾನು ಸರಿಹೊಂದಿಸಿಕೊಳ್ಳಬೇಕೆ! ಥೂ! ಸೋದರಿ, ನೀನಿನ್ನೂ ಸಂನ್ಯಾಸಿಯನ್ನು ಅರಿಯೆ. ‘ಅವನು ವೇದಗಳ ಶಿರಮೆಟ್ಟಿ ನಿಲ್ಲುವನು’ ಎನ್ನುತ್ತವೆ ವೇದಗಳು. ಏಕೆಂದರೆ ಅವನು ಮತಪಂಥಗಳಿಂದ, ಧರ್ಮಗಳಿಂದ, ಧರ್ಮಗುರು ಗಳಿಂದ ಮತ್ತು ಅಂಥವುಗಳೆಲ್ಲದರಿಂದಲೂ ಮುಕ್ತ!

“ಮಿಷನರಿಗಳು-ಮಿಷನರಿಗಳಲ್ಲದವರು ಎಲ್ಲರೂ ಬೇಕಾದಷ್ಟು ಅರಚಿಕೊಳ್ಳಲಿ. ತಮ್ಮ ಕೈಲಾ ದಷ್ಟೂ ನನ್ನ ಮೇಲೆ ಆಕ್ರಮಣ ಮಾಡಲಿ. ನಾನು ಮಾತ್ರ ಭರ್ತೃಹರಿ ಹೇಳುವಂತೆ ನಡೆಯುತ್ತೇನೆ. ‘ಓ ಸಂನ್ಯಾಸಿ, ನಿನ್ನ ದಾರಿಯಲ್ಲಿ ನೀ ನಡೆ. ಕೆಲವರು ಹೇಳುತ್ತಾರೆ–ಯಾರೀ ಹುಚ್ಚ? ಎಂದು. ಕೆಲವರು ಹೇಳುತ್ತಾರೆ–ಯಾರೀ ಚಂಡಾಲ? ಎಂದು. ಉಳಿದ ಕೆಲವರು ನಿನ್ನನ್ನು ಪುಷಿ ಎಂದು ಅರಿಯುತ್ತಾರೆ. ಪ್ರಾಪಂಚಿಕರ ಅಪಲಾಪಗಳನ್ನು ಕೇಳಿ ನಕ್ಕುಬಿಡು. ಆದರೆ ಅವರು ಆಕ್ರಮಣ ಮಾಡಿದಾಗ ತಿಳಿದುಕೊ: ‘ಆನೆ ನಡೆದುಹೋಗುವ ರಾಜಮಾರ್ಗದಲ್ಲಿ ಸದಾ ಕುನ್ನಿಗಳಿರುತ್ತವೆ. ಆದರೆ ಆನೆ ಅದನ್ನು ಲೆಕ್ಕಿಸುವುದಿಲ್ಲ. ಅಂತೆಯೇ ಮಹಾತ್ಮನೊಬ್ಬ ಉದಿಸಿದಾಗ ಅವನನ್ನು ಕಂಡು ಬೊಗಳುವವರು ಬಹಳ’ ಎಂದು.

“....ಭಗವಂತ ಎಂದೆಂದಿಗೂ ನಿಮ್ಮನ್ನು ಹರಸಲಿ. ಈ ದೊಡ್ಡ ಸೋಗಿನ ಜಾಲದಿಂದ–ಈ ಪ್ರಪಂಚದಿಂದ–ಶೀಘ್ರವೇ ಅವನು ನಿಮ್ಮನ್ನು ಪಾರುಮಾಡಲಿ! ಪ್ರಪಂಚವೆಂಬ ಈ ಮುದಿ ಮಾಟಗಾತಿಯ ದೆಸೆಯಿಂದ ನೀವು ಎಂದೆಂದೂ ಮರುಳುಗೊಳ್ಳದಿರುವಂತಾಗಲಿ! ಶಂಕರನು ನಿಮಗೆ ನೆರವಾಗಲಿ. ಉಮೆಯು ನಿಮಗಾಗಿ ಸತ್ಯದ ದ್ವಾರವನ್ನು ತೆರೆದು ನಿಮ್ಮ ಭ್ರಾಂತಿಯನ್ನೆಲ್ಲ ನಾಶಮಾಡಲಿ!”

ಅತ್ಯಂತ ಅದ್ಭುತವೂ ಪ್ರಭಾವಪೂರ್ಣವೂ ಆದ ಈ ಪತ್ರದಲ್ಲಿ ಸ್ವಾಮೀಜಿ, ತಮ್ಮ ಅಂತ ರಾಳದ ಹಲವಾರು ಭಾವನೆಗಳನ್ನು ಯಾವುದೇ ಅಡೆತಡೆಯಿಲ್ಲದಂತೆ ನೇರವಾಗಿ ಹೊರಗೆಡವಿ ದ್ದಾರೆ. ತಮಗೆ ಈ ಜಗತ್ತಿನಲ್ಲಿ ಒಂದು ‘ಕರ್ತವ್ಯ’ ಎಂಬುದಿದೆ ಎನ್ನುವುದನ್ನೇ ಅವರು ತಳ್ಳಿಹಾಕು ತ್ತಾರೆ! ‘ತಮ್ಮ’ ಕಾರ್ಯೋದ್ದೇಶದ ಸಾಧನೆಗಾಗಿ ಸ್ವಾಮೀಜಿ ಪಡುವ ಶ್ರಮವನ್ನು ಕಂಡ ಯಾರಿಗೇ ಆದರೂ ಅವರು ಸಾಧಿಸಬೇಕಾದ ಕಾರ್ಯವಾವುದೂ ಇಲ್ಲ ಎಂದರೆ ಹೇಗೆ ಅರ್ಥ ಆಗಬೇಕು! ಶ್ರೀಮತಿ ಬುಲ್ ಕೂಡ, ಸ್ವಾಮೀಜಿ ಮಾಡಬೇಕಾದ ಒಂದು ಕಾರ್ಯವಿದೆ ಎಂದೇ ನಂಬಿದ್ದವಳು. ಆದರೆ ಅದನ್ನು ಅವರು ಖಡಾಖಂಡಿತವಾಗಿ ತಳ್ಳಿಹಾಕುವುದನ್ನು ಈ ಪತ್ರದಲ್ಲಿ ಕಾಣುತ್ತೇವೆ. ಅಲ್ಲದೆ, ಜಗತ್ತಿನ ಜನರು ಯಾವುದನ್ನು ಅತಿಶ್ರೇಷ್ಠವೆಂದು ಭಾವಿಸುತ್ತಾರೋ ಅದೆಲ್ಲ ತಮ್ಮ ಪಾಲಿಗೆ ಎಷ್ಟು ನಿಕೃಷ್ಟವೆಂಬುದನ್ನೂ ಸ್ವಾಮೀಜಿ ಇಲ್ಲಿ ಸ್ಪಷ್ಟವಾಗಿ ಹೇಳುತ್ತಿದ್ದಾರೆ. ತಮಗೆ ಸಲಹೆ ನೀಡಿದ ಮತ್ತೊಬ್ಬ ಮಹಿಳೆಯ ವಿಚಾರವನ್ನು ಪ್ರಸ್ತಾಪಿಸಿ ಅವರು, ಶ್ರೀಮತಿ ಸಾರಾಳಿಗೆ ಮತ್ತೊಂದು ಪತ್ರದಲ್ಲಿ ಬರೆಯುತ್ತಾರೆ:

“ಮಿಸ್ ಹ್ಯಾಮ್ಲಿನ್ ನನಗೆ ಬಹಳವಾಗಿ ಸಹಾಯ ಮಾಡುತ್ತಿದ್ದಾಳೆ. ನಾನವಳಿಗೆ ತುಂಬಾ ಕೃತಜ್ಞನಾಗಿದ್ದೇನೆ. ಅವಳು ತುಂಬಾ ವಿಶ್ವಾಸಯುತಳು, ಪ್ರಾಮಾಣಿಕಳು. ಅವಳು ನನ್ನನ್ನು ‘ಸರಿ ಯಾದ ಜನರಿಗೆ’ ಪರಿಚಯ ಮಾಡಿಕೊಡಬೇಕೆಂದಿದ್ದಾಳೆ. ಇದು, ‘ನಿನ್ನನ್ನು ನೀನು ಭದ್ರ ಪಡಿಸಿಕೊ’ ಎಂಬ ವ್ಯವಹಾರದ ಎರಡನೆಯ ಅಂಕವೇನೋ ಎಂಬುದು ನನ್ನ ಶಂಕೆ. ಆದರೆ ಯಾರನ್ನು ಭಗವಂತ ನನ್ನ ಬಳಿಗೆ ಕಳುಹಿಸಿಕೊಡುತ್ತಾನೆಯೋ ಅವರು ಮಾತ್ರವೇ ‘ಸರಿಯಾದ ಜನ’–ಇದು ನಾನು ನನ್ನ ಜೀವನದ ಅನುಭವದಿಂದ ಅರ್ಥಮಾಡಿಕೊಂಡಿರುವುದು. ಅವರು ಮಾತ್ರವೇ ನನಗೆ ಸಹಾಯ ಮಾಡಬಲ್ಲವರು, ಮತ್ತು ಮಾಡುತ್ತಾರೆ. ಇನ್ನುಳಿದವರನ್ನೆಲ್ಲ ಭಗವಂತನೇ ನೋಡಿಕೊಂಡು ಕಾಪಾಡಲಿ!

“ಈ ರೀತಿಯಾಗಿ ನಾನೇ ಒಂದು ಬಡಮನೆಯನ್ನು ಬಾಡಿಗೆಗೆ ತೆಗೆದುಕೊಂಡು ತರಗತಿಗಳನ್ನು ಮಾಡುತ್ತಿರುವುದೆಲ್ಲ ಕೊನೆಗೆ ಶೂನ್ಯದಲ್ಲಿ ಪರ್ಯವಸಾನವಾಗುತ್ತದೆ ಎಂದೂ ಯಾವೊಬ್ಬ ಮಹಿಳೆಯೂ ಇಲ್ಲಿಗೆ ಬರುವುದಿಲ್ಲವೆಂದೂ ನನ್ನ ಗೆಳೆಯರೆಲ್ಲರ ಎಣಿಕೆಯಾಗಿತ್ತು. ಅದರಲ್ಲೂ ಮಿಸ್​ಹ್ಯಾಮ್ಲಿನ್ನಳಂತೂ, ಈ ನಿರ್ಗತಿಕ ಜೋಪಡಿಯಲ್ಲಿ ತನ್ನಷ್ಟಕ್ಕೆ ತಾನು ವಾಸವಾಗಿರುವ ಮನುಷ್ಯನ ಹತ್ತಿರ ಬಂದು ಅವನ ಉಪನ್ಯಾಸಗಳನ್ನು ಕೇಳುವ ಸ್ಥಿತಿಗಿಂತ ತಾನು ಎಷ್ಟೋ ಉತ್ತಮ ಮಟ್ಟದಲ್ಲಿರುವುದಾಗಿ ನಂಬಿದ್ದಳು. ಆದರೆ ಹೀಗಿದ್ದರೂ ‘ಸರಿಯಾದ ಜನ’ ಬಂದರು, ಹಗಲಿರುಳೂ ಬಂದರು, ಜೊತೆಗೆ ಅವಳೂ ಕೂಡ ಬಂದಳು! ಭಗವಂತ! ನಿನ್ನನ್ನೂ ನಿನ್ನ ಕರುಣೆಯನ್ನೂ ನಂಬುವುದು ಮನುಷ್ಯರಿಗೆ ಎಷ್ಟು ಕಷ್ಟ! ಶಿವ ಶಿವ! ತಾಯಿ, ಸರಿಯಾದವರು ಯಾರು, ಸರಿಯಲ್ಲದವರು ಯಾರು? ಎಲ್ಲವೂ ಅವನೆ! ಹುಲಿಯಲ್ಲಿ, ಕುರಿಯಲ್ಲಿ, ಸಂತನಲ್ಲಿ, ಪಾಪಿಯಲ್ಲಿ–ಎಲ್ಲರಲ್ಲೂ ಅವನೆ! ಅವನಲ್ಲಿ ನಾನು ಶರಣಾಗತನಾಗಿದ್ದೇನೆ. ನನ್ನ ದೇಹ ಪ್ರಾಣ ಆತ್ಮಗಳನ್ನು ಅವನಿಗೆ ಸಮರ್ಪಿಸಿಬಿಟ್ಟಿದ್ದೇನೆ. ಇಷ್ಟು ದಿನಗಳವರೆಗೆ ನನ್ನನ್ನು ತನ್ನ ಕೈಗಳಲ್ಲೆತ್ತಿ ಕೊಂಡಿದ್ದವನು ಈಗ ನನ್ನ ಕೈಬಿಡುವನೇನು?... ಹೇ ಸರ್ವಮಂಗಳ ಪ್ರಭು, ನನ್ನ ಜೀವನವಿಡೀ ನಾ ನಿನ್ನ ದಾಸ, ನಾನು ಕೇವಲ ನಿನ್ನವನು ಎಂಬುದನ್ನರಿತ ನೀನು ನನ್ನ ಕೈಬಿಡುವೆಯಾ? ನಾನು ಇತರರ ಆಟದ ವಸ್ತುವಾಗಲು ಬಿಡುವೆಯಾ?....ತಾಯಿ, ನನಗೆ ಖಂಡಿತವಾಗಿಯೂ ಗೊತ್ತು, ಅವನು ನನ್ನನ್ನೆಂದಿಗೂ ತೊರೆಯುವುದಿಲ್ಲ.”

ತಾವೊಂದು ಸಂಸ್ಥೆಯನ್ನು ಕಟ್ಟಬೇಕಾಗಿದೆ ಎಂದು ಸ್ವಾಮೀಜಿ ಹೇಳಿದುದನ್ನು ಕೆಲವರು ಅಪಾರ್ಥ ಮಾಡಿಕೊಂಡಿದ್ದರು. ಸ್ವಾಮೀಜಿಯ ಉದ್ದೇಶವನ್ನು ಅರ್ಥಮಾಡಿಕೊಳ್ಳಲಾರದೆ, ಅವರು ಭಾವಿಸಿದ್ದರು–ಒಂದು ಸಂಸ್ಥೆಯನ್ನು ಕಟ್ಟುವುದರ ಮೂಲಕ ಸ್ವಾಮೀಜಿ ಭಾರೀ ಯಶಸ್ಸನ್ನು ಗಳಿಸಲು ಒಂದು ಹಂಚಿಕೆ ಹಾಕಿದ್ದಾರೆ ಎಂದು. ಈ ತಪ್ಪು ಅಭಿಪ್ರಾಯಗಳನ್ನು ಹೋಗಲಾಡಿಸಲು ಸ್ವಾಮೀಜಿ ತಮ್ಮ ಶಿಷ್ಯೆಯೊಬ್ಬಳಿಗೆ ಬರೆದರು:

“ನಮಗೆ ಯಾವ ಸಂಸ್ಥೆಯೂ ಇಲ್ಲ, ಮತ್ತು ಅಂಥದೊಂದನ್ನು ಕಟ್ಟಲು ಇಚ್ಛಿಸುವುದೂ ಇಲ್ಲ. ಪ್ರತಿಯೊಬ್ಬನೂ ತನಗೆ ಇಷ್ಟಬಂದದ್ದನ್ನು ಬೋಧಿಸಲು ಸಂಪೂರ್ಣ ಸ್ವತಂತ್ರ... ನಿನ್ನೊಳಗೆ ನಿಜವಾದ ಚೈತನ್ಯವಿದ್ದರೆ, ಇತರರನ್ನು ಆಕರ್ಷಿಸುವಲ್ಲಿ ನೀನೆಂದಿಗೂ ಸೋಲುವುದಿಲ್ಲ.

“ವೈಯಕ್ತಿಕತೆ ನನ್ನ ಮಂತ್ರ. ವ್ಯಕ್ತಿಗಳನ್ನು ತರಬೇತುಗೊಳಿಸುವುದಕ್ಕಿಂತ ಹೆಚ್ಚಾಗಿ ನನಗಾವ ಆಸೆಯೂ ಇಲ್ಲ. ನನಗೆ ಗೊತ್ತಿರುವುದು ಬಹಳ ಅಲ್ಪ. ಅಷ್ಟನ್ನೇ ನಾನು ಮುಚ್ಚುಮರೆಯಿಲ್ಲದೆ ಬೋಧಿಸುತ್ತೇನೆ. ನನಗೆ ಯಾವುದು ಗೊತ್ತಿಲ್ಲವೋ ಅದನ್ನು ಗೊತ್ತಿಲ್ಲ ಎಂದು ಒಪ್ಪಿಕೊಳ್ಳುತ್ತೇನೆ... ನಾನೊಬ್ಬ ಸಂನ್ಯಾಸಿ. ಆದ್ದರಿಂದ ನಾನು ನನ್ನನ್ನು ಒಬ್ಬಸೇವಕನೆಂದು ಭಾವಿಸುತ್ತೇನೆ, ಯಜಮಾನನೆಂದಲ್ಲ.

“ಜನ ನನ್ನನ್ನು ಪ್ರೀತಿಸಲಿ ಅಥವಾ ದ್ವೇಷಿಸಲಿ–ಎಲ್ಲಕ್ಕೂ ಒಂದೇ ಬಗೆಯ ಸ್ವಾಗತ... ನಾನು ಯಾವ ಸಹಾಯವನ್ನೂ ಅರಸುವುದೂ ಇಲ್ಲ, ಯಾವ ಸಹಾಯವನ್ನೂ ತಿರಸ್ಕರಿಸುವುದೂ ಇಲ್ಲ. ನಾನು ಸಂನ್ಯಾಸವನ್ನು ಸ್ವೀಕರಿಸಿದಾಗ ಎಲ್ಲದಕ್ಕೂ–ಹಸಿವು ಹಾಗೂ ಅತ್ಯಂತ ಘೋರ ಸಂಕಟಗಳಿಗೂ–ಸಿದ್ಧನಾಗಿಯೇ ಇದ್ದೆ... ”

ಹೀಗೆ ಸ್ವಾಮೀಜಿ ತಮ್ಮ ಅಭಿಪ್ರಾಯ-ಧೋರಣೆಗಳನ್ನು ಪತ್ರಗಳ ಮೂಲಕವೂ ಸಂಭಾಷಣೆ ಗಳ ಮೂಲಕವೂ ಸುಸ್ಪಷ್ಟವಾಗಿ–ಕೆಲವೊಮ್ಮೆ ಮನಸ್ಸಿಗೆ ನಾಟುವಂತೆ–ತಿಳಿಯಪಡಿಸಿದ್ದರಿಂದ ಅವರ ಶಿಷ್ಯರು ಕ್ರಮೇಣ ಅವರನ್ನು ಹೆಚ್ಚು ಚೆನ್ನಾಗಿ ಅರ್ಥಮಾಡಿಕೊಂಡರು. ಆದರೆ ಸಂಘಸಂಸ್ಥೆ ಗಳ ವಿಷಯದಲ್ಲಿ ಪಾಶ್ಚಾತ್ಯರ ದೃಷ್ಟಿಕೋನವೇ ಬೇರೆ. ಅವರ ಲೌಕಿಕ ದೃಷ್ಟಿಗೆ ಅನುಗುಣವಾಗಿ, ಒಂದು ಸಂಸ್ಥೆಯೆಂದರೆ ಅದಕ್ಕೊಂದು ಭವ್ಯವಾದ ಕಟ್ಟಡ ಇರಬೇಕು; ಸುಂದರವಾದ ಮೇಜು ಕುರ್ಚಿಗಳನ್ನೆಲ್ಲ ಅಚ್ಚುಕಟ್ಟಾಗಿ ಜೋಡಿಸಿರಬೇಕು; ನಾಗರಿಕ ಉಡಿಗೆತೊಡಿಗೆಗಳನ್ನು ನಾಜೂಕಾಗಿ ಧರಿಸಿರುವ ಸದಸ್ಯರು ಅಲ್ಲಿ ಸೇರಿ ದೊಡ್ಡದೊಡ್ಡ ಆದರ್ಶಗಳ ಕುರಿತಾಗಿ ಚರ್ಚಿಸಬೇಕು. ಈ ಬಗೆಯ ಸಂಸ್ಥೆಯ ಕಲ್ಪನೆಯಿರುವಂತಹ ಆ ಪಾಶ್ಚಾತ್ಯರಿಗೆ, ಸ್ವಾಮೀಜಿಯ ಆಧ್ಯಾತ್ಮಿಕ ದೃಷ್ಟಿಯ ಸಂಸ್ಥೆಯೊಂದರ ಕಲ್ಪನೆಯನ್ನು ಮಾಡಿಕೊಳ್ಳಲು ಸಾಕಷ್ಟು ಸಮಯವೇ ಹಿಡಿಯಿತು. ಸ್ವಾಮಿಜಿಯ ದೃಷ್ಟಿಯಲ್ಲಿ ಸಂಸ್ಥೆಯೆಂದರೆ ಕೇವಲ ಕಟ್ಟಡವಲ್ಲ, ಬದಲಾಗಿ, ಉದಾತ್ತಚರಿತರಾದ, ಶುದ್ಧ ಹೃದಯದ, ಉತ್ಸಾಹಶೀಲರಾದ ಹಾಗೂ ಕಾರ್ಯದಕ್ಷರಾದ ವ್ಯಕ್ತಿಗಳ ಸಮುದಾಯ. ಇಂತಹ ಸಂಸ್ಥೆಯೊಂದನ್ನು ನಿರ್ಮಿಸಲು ಅವರು ಎಲ್ಲ ಅಡೆತಡೆಗಳನ್ನೂ ಎದುರಿಸಿ ನಿರಂತರವಾಗಿ ಶ್ರಮಿಸಿದರು.

ಹೀಗೆ ಅವಿರತವಾಗಿ ದುಡಿದುದರ ಫಲವಾಗಿ ಅವರು ದೈಹಿಕವಾಗಿಯೂ ಮಾನಸಿಕವಾಗಿಯೂ ತೀವ್ರವಾಗಿ ಬಳಲಿದರು. ಪ್ರಚಂಡ ಶಕ್ತ್ಯುತ್ಸಾಹಗಳನ್ನು ತುಂಬಿಕೊಂಡು ಅಮೆರಿಕೆಗೆ ಬಂದ ಅವರು, ಎರಡು ವರ್ಷಗಳ ಕಾಲ ಏಕಾಂಗಿವೀರನಾಗಿ ನಡೆಸಿದ ಹೋರಾಟವೆಷ್ಟು! ಉಪನ್ಯಾಸಗಳ ಮೇಲೆ ಉಪನ್ಯಾಸಗಳು, ಸಂಭಾಷಣೆಗಳ ಮೇಲೆ ಸಂಭಾಷಣೆಗಳು, ನಿರಂತರ ಪ್ರಯಾಣ, ಪಾದ್ರಿ ಗಳೇ ಮೊದಲಾದವರೊಂದಿಗಿನ ದೀರ್ಘ ಕಾಳಗಗಳು, ಜೊತೆಗೆ ತಮ್ಮೊಳಗಿನ ಹೋರಾಟಗಳು– ಒಂದೇ ಎರಡೇ! ಆದರೆ ಅದರಿಂದಾದ ಸಾಧನೆ ಕಡಿಮೆಯೇನಲ್ಲ. ಅಮೆರಿಕದಾದ್ಯಂತ ಅವರ ಸಂದೇಶ ಪ್ರಸಾರವಾಗಿತ್ತು; ಅದರಿಂದ ಅನೇಕ ಸತ್ಪರಿಣಾಮಗಳುಂಟಾಗಿದ್ದುವು. ಅವರ ಸಂದೇಶ ವನ್ನು ಮತ್ತಷ್ಟು ಆಲಿಸಲು ಸಾವಿರಾರು ಜನ ಉತ್ಸುಕರಾಗಿದ್ದರು; ನೂರಾರು ಜನ ಅವರ ಅನುಯಾಯಿಗಳಾಗಿ ಬಂದಿದ್ದರು. ಅಲ್ಲದೆ ಅವರು ಹೊಮ್ಮಿಸಿದ ತರಂಗಗಳು ಪ್ರಬಲವಾಗಿ ಮುಂದುವರಿಯುತ್ತಿದ್ದುವು. ಅವರು ಸಾರಿದ ನೂತನ ಸಂದೇಶ ಹಾಗೂ ಭಾವನೆಗಳು ವೇದಿಕೆಗಳ ಮೇಲೆ ಹಲವಾರು ಉಪನ್ಯಾಸಗಳ ಮೂಲಕ ಪ್ರತಿಧ್ವನಿತವಾಗುತ್ತಿದ್ದುವು. ಆದ್ದರಿಂದ ತಮ್ಮ ಶ್ರಮ ವ್ಯರ್ಥವಾಗಿಲ್ಲವೆಂಬ ಸಮಾಧಾನ ಅವರಿಗಿತ್ತು. ಖೇತ್ರಿಯ ಮಹಾರಾಜ ಅಜಿತ್​ಸಿಂಗನಿಗೆ ಬರೆದ ಒಂದು ಪತ್ರದಲ್ಲಿ ಅವರು ಹೀಗೆ ಹೇಳುತ್ತಾರೆ:

“ನಾನು ಈ ದೇಶದಲ್ಲಿ ಒಂದು ಬೀಜವನ್ನು ಬಿತ್ತಿದ್ದೇನೆ. ಅದು ಈಗಾಗಲೇ ಒಂದು ಗಿಡವಾಗಿ ಬೆಳೆದಿದೆ. ಅತಿ ಶೀಘ್ರದಲ್ಲೇ ಅದು ಒಂದು ಮರವಾಗುವುದೆಂದು ನಿರೀಕ್ಷಿಸುತ್ತೇನೆ. ನನಗೆ ಕೆಲವು ನೂರು ಅನುಯಾಯಿಗಳಿದ್ದಾರೆ. ನಾನು ಕೆಲವರನ್ನು ಸಂನ್ಯಾಸಿಗಳನ್ನಾಗಿ ಮಾಡುತ್ತೇನೆ. ಬಳಿಕ ಕೆಲಸವನ್ನು ಅವರಿಗೆ ಬಿಟ್ಟು ನಾನು ಭಾರತಕ್ಕೆ ಬರುತ್ತೇನೆ. ಕ್ರೈಸ್ತ ಪಾದ್ರಿಗಳು ನನ್ನನ್ನು ಹೆಚ್ಚು ಹೆಚ್ಚಾಗಿ ವಿರೋಧಿಸಿದಷ್ಟೂ ಅವರ ದೇಶದ ಮೇಲೆ ಅಚ್ಚಳಿಯದ ಗುರುತನ್ನುಂಟುಮಾಡುವ ನನ್ನ ನಿರ್ಧಾರ ಹೆಚ್ಚುಹೆಚ್ಚು ದೃಢವಾಗುತ್ತದೆ.” ಹೀಗೆ ಅವಿರತ ಹೋರಾಟದ ಅನಂತರವೂ ಅವರ ಉತ್ಸಾಹವು ಅಪ್ರತಿಹತವಾಗಿಯೇ ಉಳಿದಿತ್ತು. ಆದರೆ ಜರ್ಜರಿತವಾಗಿದ್ದ ಅವರ ಮನಸ್ಸು- ದೇಹಗಳಿಗೆ ವಿಶ್ರಾಂತಿಯ ತೀವ್ರ ಅಗತ್ಯವಿತ್ತು. ಆದ್ದರಿಂದ ಕೆಲಕಾಲದ ಮಟ್ಟಿಗಾದರೂ ಪ್ರಶಾಂತ ಸ್ಥಳಕ್ಕೆ ಹೋಗಿ ವಿಶ್ರಮಿಸಿಬೇಕೆಂದು ಆಶಿಸಿದರು.

ಜೂನ್ ತಿಂಗಳಲ್ಲಿ ಸ್ವಾಮೀಜಿಯ ಶಿಷ್ಯರಾದ ಫ್ರಾನ್ಸಿಸ್ ಲೆಗೆಟ್ಟರು ಅವರನ್ನು ತಮ್ಮ ವಿಶ್ರಾಂತಿಧಾಮಕ್ಕೆ ಆಹ್ವಾನಿಸಿದರು. ಇದು ಇದ್ದದ್ದು ನ್ಯೂ ಹ್ಯಾಂಪ್​ಶೈರ್ ರಾಜ್ಯದ ಕ್ಯಾಂಪ್ ಪರ್ಸಿ ಎಂಬಲ್ಲಿ. ‘ವೈಟ್ ಮೌಂಟೆನ್ಸ್​’ನ ಬಳಿಯ ಕ್ರಿಸ್ಟೀನ್ ಸರೋವರದ ತೀರದಲ್ಲಿದ್ದ ಈ ವಿಶ್ರಾಂತಿಧಾಮವು, ಸುತ್ತಲೂ ದಟ್ಟವಾದ ವನರಾಶಿಯಿಂದ ಕೂಡಿದ್ದು ಬಹಳ ಆಹ್ಲಾದಕರ ವಾಗಿತ್ತು. ಇಲ್ಲಿನ ಪರಮ ಪ್ರಶಾಂತ ವಾತಾವರಣವು ಸ್ವಾಮೀಜಿಗೆ ನವಚೇತನವನ್ನು ನೀಡಿತು. ತುಂಬ ಬಳಲಿದ್ದ ಅವರ ಮನಸ್ಸು ಶಾಂತಗೊಂಡು ಉನ್ನತ ಸ್ಥಿತಿಗೇರತೊಡಗಿತು. ಇಲ್ಲಿಂದ ಶ್ರೀಮತಿ ಬುಲ್ಲಳಿಗೆ ಬರೆದ ಒಂದು ಪತ್ರದಲ್ಲಿ ಅವರು ತಮ್ಮ ಸಂತಸವನ್ನು ವ್ಯಕ್ತಪಡಿಸಿದರು– “ಇದು ನಾನು ನೋಡಿದ ಅತ್ಯಂತ ಸುಂದರ ತಾಣಗಳಲ್ಲೊಂದು. ಕಾಡಿನಿಂದ ಸಂಪೂರ್ಣವಾಗಿ ಆವೃತವಾದ ಬೆಟ್ಟಗಳ ನಡುವೆ ಒಂದು ಸರೋವರವನ್ನು ಊಹಿಸಿಕೊ. ಅಲ್ಲಿ ನಮ್ಮನ್ನು ಬಿಟ್ಟು ಬೇರೆ ಯಾರೂ ಇಲ್ಲ ಎಂದು ಭಾವಿಸು...! ಅಷ್ಟೊಂದು ಮನೋಹರ, ಅಷ್ಟೊಂದು ಪ್ರಶಾಂತ, ಅಷ್ಟೊಂದು ಆರಾಮದಾಯಕ! ಅದರಲ್ಲೂ, ನಗರಗಳ ಗದ್ದಲಗಳನ್ನೆಲ್ಲ ಕೇಳಿ ಸಾಕಾದ ಮೇಲೆ, ಇಲ್ಲಿರಲು ನನಗೆ ಎಷ್ಟು ಆನಂದವಾಗಿದೆ ಎಂಬುದನ್ನು ನೀನು ಊಹಿಸಬಹುದು. ಇಲ್ಲಿನ ವಾಸದಿಂದ ನನಗೆ ಹೊಸ ಜೀವ ಬಂದಂತಾಗಿದೆ. ನಾನಿಲ್ಲಿ ಗಂಟೆಗಟ್ಟಲೆ, ದಿನಗಟ್ಟಲೆ ಧ್ಯಾನ ಮಾಡುತ್ತೇನೆ. ಇಲ್ಲಿ ನನ್ನಷ್ಟಕ್ಕೆ ನಾನಿರಬಹುದಾಗಿದೆ. ಈ ಸ್ಥಿತಿಯ ಕಲ್ಪನೆಯೇ ನನಗೆ ಅಷ್ಟು ಸ್ಫೂರ್ತಿದಾಯಕವಾಗಿದೆ!”

ಇಲ್ಲಿ ಸ್ವಾಮೀಜಿ ದಟ್ಟವಾದ ವನದಲ್ಲಿ ಒಬ್ಬರೇ ಓಡಾಡುತ್ತಿದ್ದರು. ಇಲ್ಲವೆ ಸರೋವರದ ತೀರದಲ್ಲಿ ಇಷ್ಟಬಂದಷ್ಟು ಹೊತ್ತು ಕುಳಿತಿರುತ್ತಿದ್ದರು. ಕೆಲವೊಮ್ಮೆ ತಮ್ಮ ಸ್ನೇಹಿತರೊಂದಿಗೆ ತುಂಬ ಖುಷಿಯಾಗಿ ಮಾತುಕತೆಯಾಡುತ್ತಿದ್ದರು. ಹಾಲಿಸ್ಟರ್ ಹಾಗೂ ಆಲ್ಬರ್ಟಾ ಕೂಡ ಅಲ್ಲಿ ಇದ್ದರು. ಕ್ಯಾಂಪ್ ಪರ್ಸಿಯಲ್ಲಿ ತಾವು ಕಳೆದ ಆಹ್ಲಾದಕರ ದಿನಗಳನ್ನು ನೆನಪಿಸಿಕೊಂಡು, ಅಲ್ಲಿಂದ ತಾವು ತೆರಳಿದ ಕೆಲವು ವಾರಗಳ ಮೇಲೆ ಸ್ವಾಮೀಜಿ ಆಲ್ಬರ್ಟಾಳಿಗೆ ಬರೆದರು– “ಪರ್ಸಿಯಲ್ಲಿ ನಾವು ಶ್ರೀ ಲೆಗೆಟ್ಟರೊಂದಿಗೆ ಎಷ್ಟು ಖುಷಿಯಾಗಿ ಕಾಲ ಕಳೆದೆವು! ನಿಜಕ್ಕೂ ಅವರು (ಲೆಗೆಟ್​) ಒಬ್ಬ ಸಂತನಲ್ಲವೆ?.... ನಾವು ಸಾಕಷ್ಟು ದೋಣಿ ನಡೆಸಿದೆವು. ಹುಟ್ಟು ಹಾಕುವುದರಲ್ಲಿ ನಾನು ಒಂದೆರಡು ಹೊಸ ವಿಷಯಗಳನ್ನು ಕಲಿತೆ. ತಾನು ಅಷ್ಟೊಂದು ಮಧುರವಾಗಿರುವುದಕ್ಕಾಗಿ ಜೋಜೋ ಚಿಕ್ಕಮ್ಮ (ಜೋಸೆಫಿನ್ ಮೆಕ್​ಲಾಡ್​) ದಂಡ ತೆರಬೇಕಾ ಯಿತು; ನೊಣಗಳೂ, ಸೊಳ್ಳೆಗಳೂ ಒಂದು ಕ್ಷಣವೂ ಅವಳನ್ನು ಬಿಡಲೊಲ್ಲವು. ಆದರೆ ಅವು ಅತ್ಯಂತ ಸಂಪ್ರದಾಯಸ್ಥ ಕ್ರೈಸ್ತರೆಂದು ಕಾಣುತ್ತದೆ–ಆದ್ದರಿಂದ ‘ವಿಧರ್ಮೀಯ ಅನಾಗರಿಕ’ ನಾದ ನನ್ನಿಂದ ಅವು ಸ್ವಲ್ಪ ದೂರವೇ ಉಳಿದುಕೊಂಡುವು. ಅಲ್ಲದೆ, ನಾನು ಪರ್ಸಿಯಲ್ಲಿ ತುಂಬ ಹಾಡೂ ಹೇಳುತ್ತಿದ್ದೆ; ಆದ್ದರಿಂದ ಅವು ಬೆದರಿ ಓಡಿಹೋಗಿರಬೇಕು. ಅಲ್ಲಿ ಅಷ್ಟು ಒಳ್ಳೆಯ ಬರ್ಚ್ (ಭೂರ್ಜ) ಮರಗಳಿದ್ದುವು. ಅವುಗಳ ತೊಗಟೆಯನ್ನೆಲ್ಲ ಸೇರಿಸಿ, ನಮ್ಮ ದೇಶದಲ್ಲಿ ಹಿಂದೆ ಮಾಡುತ್ತಿದ್ದಂತೆ, ಒಂದು ಪುಸ್ತಕ ಮಾಡಿ, ಅವುಗಳ ಮೇಲೆ ನಿನ್ನ ಅಮ್ಮನಿಗಾಗಿ (ಬೆಸ್ಸಿ ಸ್ಟರ್ಜಸ್​) ಹಾಗೂ ಚಿಕ್ಕಮ್ಮನಿಗಾಗಿ ಸಂಸ್ಕೃತದಲ್ಲಿ ಶ್ಲೋಕಗಳನ್ನು ಬರೆದೆ.”

ಹುಟ್ಟು ಹಾಕುವುದರಲ್ಲಿ ಸ್ವಾಮೀಜಿ ‘ಹೊಸ ವಿಷಯಗಳನ್ನು ಕಲಿತ’ ಕತೆ ಇದು: ಒಂದು ಸಲ ಸರೋವರದಲ್ಲಿ ಇತರರೊಂದಿಗೆ ದೋಣಿ ನಡೆಸುವಾಗ, ಸ್ವಾಮೀಜಿ ಹುಟ್ಟುಹಾಕಿದ ಕ್ರಮ ತಪ್ಪಾಗಿಹೋಯಿತಂತೆ. ಆಗ ದೋಣಿ ಇನ್ನೇನು ಮಗುಚಿಕೊಳ್ಳುವಂತಾಗಿ ಅವರು ಹಿಂದಕ್ಕೆ ಬಿದ್ದುಬಿಟ್ಟರು. ಅಂತೂ ದೋಣಿ ಹೇಗೋ ಸಾವರಿಸಿಕೊಂಡು ನಿಂತಾಗ ಸ್ವಾಮೀಜಿ ಗಟ್ಟಿಯಾಗಿ ನಗುತ್ತಿದ್ದರು!

ಆದರೆ, ಇಂತಹ ಹುಡುಗಾಟಿಕೆಯ ಮನೋಭಾವದಿಂದ ಅತ್ಯುನ್ನತ ಸಮಾಧಿಸ್ಥಿತಿಗೆ ನಿರಾ ಯಾಸವಾಗಿ ಏರಬಲ್ಲವರಾಗಿದ್ದರು ಸ್ವಾಮೀಜಿ. ನ್ಯೂಯಾರ್ಕಿನ ತರಗತಿಗಳ ವೇಳೆಯಲ್ಲಿ ಬಲ ವಂತವಾಗಿ ನಿಯಂತ್ರಿಸಬೇಕಾಗುತ್ತಿದ್ದ ಅವರ ಮನಸ್ಸು, ಇಲ್ಲಿನ ನಿಸರ್ಗದ ಮಡಿಲಲ್ಲಿ ಹೆಚ್ಚು ಸ್ವತಂತ್ರವಾಯಿತು. ಈ ಸಮಯದಲ್ಲಿ ಅವರೊಮ್ಮೆ ನಿರ್ವಿಕಲ್ಪ ಸಮಾಧಿಯ ಸ್ಥಿತಿಗೇರಿದರೆಂಬ ಸಂಗತಿ, ಮೆಕ್​ಲಾಡಳ ಮೂಲಕ ತಿಳಿದು ಬರುತ್ತದೆ.

ಒಂದು ದಿನ ಬೆಳಗಿನ ಉಪಾಹಾರಕ್ಕೆ ಮುಂಚೆ ಸ್ವಾಮೀಜಿ, ಕೈಯಲ್ಲಿ ಭಗವದ್ ಗೀತೆಯ ಒಂದು ಪ್ರತಿಯನ್ನು ಹಿಡಿದು ಕೋಣೆಯಿಂದಾಚೆ ಬಂದರು. ಆಗ ಮೆಕ್​ಲಾಡಳನ್ನು ಕಂಡು ಸ್ವಾಮೀಜಿ, “ಜೋ, ನಾನು ಅಲ್ಲಿ ಪೈನ್ ಮರದ ಕೆಳಗೆ ಕುಳಿತಿರುತ್ತೇನೆ. ಇವತ್ತು ಉಪಾಹಾರ ಸ್ವಲ್ಪ ಭರ್ಜರಿಯಾಗಿರುವಂತೆ ನೋಡಿಕೊ” ಎಂದು ಹೇಳಿ ಹೊರಟರು. ಸುಮಾರು ಅರ್ಧ ಗಂಟೆಯಾದ ಮೇಲೆ ‘ಜೋ’ ಸ್ವಾಮೀಜಿಯನ್ನು ಉಪಾಹಾರಕ್ಕೆ ಕರೆಯಲು ಹೋದಳು. ಸ್ವಾಮೀಜಿ ಚಲನರಹಿತ ಸ್ಥಿತಿಯಲ್ಲಿ ಕುಳಿತಿದ್ದರು. ಅವರ ಕೈಯಿಂದ ಪುಸ್ತಕ ಬಿದ್ದುಹೋಗಿತ್ತು. ಅವರ ಅಂಗಿ ಕಣ್ಣೀರಿನಿಂದ ತೊಯ್ದಿತ್ತು.

ಮತ್ತಷ್ಟು ಹತ್ತಿರ ಹೋಗಿ ನೋಡಿದಾಗ ಮೆಕ್​ಲಾಡಳಿಗೆ ಗೊತ್ತಾಯಿತು–ಸ್ವಾಮೀಜಿಯ ಉಸಿರಾಟ ನಿಂತುಹೋಗಿದೆ ಎಂದು. ಆಕೆ ಭಯದಿಂದ ನಡುಗಿದಳು. ಸ್ವಾಮೀಜಿ ದೇಹತ್ಯಾಗ ಮಾಡಿರಬೇಕೆಂದು ಅವಳಿಗನ್ನಿಸಿತು. ತಕ್ಷಣ ಆಕೆ ಒಳಗೋಡಿ, “ಬೇಗ ಬನ್ನಿ, ಸ್ವಾಮೀಜಿ ನಮ್ಮಿಂದ ಹೊರಟುಹೋದರು!” ಎಂದು ಕೂಗಿದಳು. ಅದನ್ನು ಕೇಳಿ ಲೆಗೆಟ್ ಹಾಗೂ ಬೆಸ್ಸಿ ಗಾಬರಿ ಗೊಂಡು ಅಳುತ್ತ ಓಡಿ ಬಂದರು. ಸ್ವಾಮೀಜಿ ಅದೇ ಸ್ಥಿತಿಯಲ್ಲಿದ್ದರು. ಅದನ್ನು ಕಂಡ ಲೆಗೆಟ್ಟರು, “ಸ್ವಾಮೀಜಿ ಸಮಾಧಿಸ್ಥಿತಿಯಲ್ಲಿದ್ದಾರೆ, ಅಷ್ಟೆ. ನಾವು ಅವರನ್ನು ಅಲುಗಾಡಿಸಿ ಎಬ್ಬಿಸೋಣ” ಎಂದು ಮುಂದಾದರು. ಆಗ ಮೆಕ್​ಲಾಡ್ ಹಾಗೆ ಮಾಡಲೇಬಾರದು ಎಂದು ಕೂಗಿ ಅವರನ್ನು ತಡೆದಳು. ಏಕೆಂದರೆ, ತಾವು ಧ್ಯಾನದಲ್ಲಿರುವಾಗ ತಮ್ಮನ್ನು ಯಾರೂ ಮುಟ್ಟಬಾರದೆಂದು ಸ್ವಾಮೀಜಿ ಹೇಳಿದ್ದುದು ಅವಳಿಗೆ ನೆನಪಿತ್ತು. ಅದಾಗಲೇ ಅವರೆಲ್ಲ ಬಂದು ಸುಮಾರು ಹತ್ತು ನಿಮಿಷವಾಗಿರಬೇಕು. ಇನ್ನೂ ಉಸಿರಾಟ ನಡೆಯುತ್ತಿರಲಿಲ್ಲ. ಕಾತರರಾಗಿ ಮತ್ತೆ ಐದು ನಿಮಿಷ ಕಾದರು. ಆಗ ನಿಧಾನವಾಗಿ ಜೀವದ ಚಿಹ್ನೆಗಳು ಕಂಡುಬರಲಾರಂಭಿಸಿದುವು. ಅರ್ಧ ಮುಚ್ಚಿ ಕೊಂಡಿದ್ದ ಅವರ ಕಣ್ಣಿನ ರೆಪ್ಪೆಗಳು ತೆರೆದುಕೊಂಡುವು. “ಯಾರು ನಾನು ಎಲ್ಲಿದ್ದೇನೆ?” ಎಂದು ಸ್ವಾಮೀಜಿ ತಮ್ಮಷ್ಟಕ್ಕೇ ಹೇಳಿಕೊಂಡರು. ಹೀಗೆ ಎರಡು ಮೂರು ಸಲ ಹೇಳಿಕೊಳ್ಳುವ ಹೊತ್ತಿಗೆ ಅವರಿಗೆ ಬಾಹ್ಯ ಪ್ರಜ್ಞೆ ಪೂರ್ಣವಾಗಿ ಮರಳಿತ್ತು. ಎದುರಿಗಿದ್ದವರನ್ನು ಗುರುತಿಸಿದಾಗ ಅವರಿಗೆ ತುಂಬ ಕಸಿವಿಸಿ-ನಾಚಿಕೆಯಾಯಿತು. “ಕ್ಷಮಿಸಿ, ನಿಮ್ಮನ್ನೆಲ್ಲ ನಾನು ಹೆದರಿಸಿಬಿಟ್ಟೆ. ಆದರೇನು ಮಾಡುವುದು, ಇಂತಹ ಅವಸ್ಥೆ ನನಗೆ ಆಗಾಗ ಬರುತ್ತದೆ. ಆದರೆ ನೀವೇನೂ ಚಿಂತಿಸಬೇಕಿಲ್ಲ. ನಾನು ನಿಮ್ಮ ದೇಶದಲ್ಲಿ ದೇಹತ್ಯಾಗ ಮಾಡುವುದಿಲ್ಲ... ಬೆಸ್ಸಿ, ನನಗೆ ತುಂಬ ಹಸಿವಾಗುತ್ತಿದೆ. ಬೇಗ ಹೋಗೋಣ...” ಎನ್ನುತ್ತ ಎದ್ದು ನಿಂತರು.

ಹೀಗೆ ಅತ್ಯಂತ ಆನಂದದಾಯಕವೂ ಸ್ಫೂರ್ತಿದಾಯಕವೂ ಆದ ದಿನಗಳನ್ನು ಕಳೆದು ಸ್ವಾಮೀಜಿ ಕ್ಯಾಂಪ್ ಪರ್ಸಿಯಿಂದ ಹೊರಟುನಿಂತರು.

ಎರಡು ವಾರಗಳ ಹಿಂದೆ ಅವರು ನ್ಯೂಯಾರ್ಕಿನಿಂದ ಕೆಲದಿನಗಳ ವಿಶ್ರಾಂತಿಗಾಗಿ ಕ್ಯಾಂಪ್ ಪರ್ಸಿಗೆ ಹೊರಟಾಗ, ತರಗತಿಗಳಿಗೆ ಬೇಸಿಗೆ ರಜಾ ಕೊಡಬೇಕೆಂಬ ಮನಸ್ಸಿತ್ತು. ಸುಡುವ ಬೇಸಿಗೆ ದಿನಗಳಲ್ಲಿ ಅವರಿಗೆ ನ್ಯೂಯಾರ್ಕಿನಲ್ಲಿರಲು ಸಾಧ್ಯವಿರಲಿಲ್ಲ. ಅಲ್ಲದೆ ನಿರಂತರ ಚಟುವಟಿಕೆ ಯಿಂದಾಗಿ ಅವರು ಸಂಪೂರ್ಣ ಬಳಲಿದ್ದರು. ಅವರ ಕೆಲವು ವಿದ್ಯಾರ್ಥಿಗಳೂ ಬೇಸಿಗೆಯಲ್ಲಿ ವಿಹಾರಧಾಮಗಳಿಗೆ ಹೋಗುವ ಆಲೋಚನೆಯಲ್ಲಿದ್ದರು. ಆದ್ದರಿಂದ ಸುಮಾರು ಎರಡು ತಿಂಗಳು ತರಗತಿಗಳನ್ನು ನಿಲ್ಲಿಸಬೇಕೆಂದು ಅವರು ಹೆಚ್ಚುಕಡಿಮೆ ನಿರ್ಧರಿಸಿಯೇಬಿಟ್ಟರು. ಆದರೆ ವಿದ್ಯಾರ್ಥಿಗಳಲ್ಲೇ ಕೆಲವರು ಈ ಅಭಿಪ್ರಾಯಕ್ಕೆ ವಿರುದ್ಧವಾಗಿದ್ದರು. ತುಂಬ ಚೆನ್ನಾಗಿ ನಡೆದು ಕೊಂಡುಬರುತ್ತಿದ್ದ ತರಗತಿಗಳನ್ನು ನಿಲ್ಲಿಸಬಾರದೆಂದು ಅವರು ಸ್ವಾಮೀಜಿಯನ್ನು ವಿನಂತಿಸಿ ಕೊಂಡರು.

ಈಗೇನು ಮಾಡುವುದೆಂದು ಸ್ವಾಮೀಜಿ ಆಲೋಚಿಸುತ್ತಿದ್ದಾಗ, ಸಮಸ್ಯೆ ತಾನಾಗಿಯೇ ಬಗೆ ಹರಿಯಿತು. ಅಮೆರಿಕದ \eng{Thousand Island Park (}ಸಹಸ್ರದ್ವೀಪೋದ್ಯಾನ) ಎಂಬ ದ್ವೀಪ ಸಮುದಾಯದಲ್ಲಿ, ಸ್ವಾಮೀಜಿಯ ಶಿಷ್ಯೆಯರಲ್ಲೊಬ್ಬಳಾದ ಮಿಸ್ ಡಚರ್ ಎಂಬವಳಿಗೆ ಸೇರಿದ ಒಂದು ಮನೆಯಿತ್ತು. ಇದೊಂದು ಬೇಸಿಗೆಯ ತಾಣ ಕೂಡ. ತರಗತಿಗಳಿಗಾಗಿ ಈ ಸ್ಥಳವನ್ನು ಬಿಟ್ಟುಕೊಡಲು ಅವಳು ಮುಂದಾದಳು. ಸ್ವಾಮೀಜಿ ಹಾಗೂ ಕೆಲವು ಶಿಷ್ಯರು ಈ ಮನೆಯಲ್ಲಿ ಉಳಿದುಕೊಳ್ಳಬಹುದಾಗಿತ್ತು. ಇದಕ್ಕೆ ಸ್ವಾಮೀಜಿ ಸಂತೋಷದಿಂದ ಒಪ್ಪಿದರು.

ಈ ದ್ವೀಪಗಳಿರುವುದು ನ್ಯೂಯಾರ್ಕಿನಿಂದ ೩ಂಂ ಮೈಲಿ ದೂರದಲ್ಲಿ; ಸೈಂಟ್ ಲಾರೆನ್ಸ್ ನದಿಯು ಆಂಟೇರಿಯೋ ಸರೋವರವನ್ನು ಸೇರುವಲ್ಲಿ. ಮಿಸ್ ಡಚರಳ ಮನೆಯಿದ್ದುದು ಈ ದ್ವೀಪಸಮುದಾಯದ ಎರಡನೇ ಅತಿ ದೊಡ್ಡ ದ್ವೀಪವಾದ ವೆಲ್ಲೆಸ್ಲೀ ದ್ವೀಪದಲ್ಲಿ. ಇದೊಂದು ಸುಂದರ ಪ್ರಶಾಂತ ಸ್ಥಳ. ಆದರೆ ಅಷ್ಟು ದೂರದ ಸ್ಥಳಕ್ಕೆ ಹೋಗಬೇಕೆಂದರೆ ಎಲ್ಲರಿಗೂ ಸಾಧ್ಯ ವಾಗುತ್ತಿರಲಿಲ್ಲ. ಆದ್ದರಿಂದ ಸ್ವಾಮೀಜಿ, ನ್ಯೂಯಾರ್ಕಿನ ತಮ್ಮ ಶಿಷ್ಯರಿಗೆ ಹೇಳಿದರು, “ನಿಮ್ಮಲ್ಲಿ ಯಾರಿಗೆ ಇತರ ಎಲ್ಲ ಆಸಕ್ತಿಗಳನ್ನೂ ಚಿಂತೆಗಳನ್ನೂ ಬದಿಗೊತ್ತಿ, ಕೇವಲ ವೇದಾಂತದ ಅಧ್ಯಯನ ಹಾಗೂ ಅನುಷ್ಠಾನ ಮಾಡುವ ಶ್ರದ್ಧೆಯಿದೆಯೋ ಹಾಗೂ ಅದಕ್ಕಾಗಿ ಮುನ್ನೂರು ಮೈಲಿ ದೂರಕ್ಕೆ ಬರಬಲ್ಲಿರೋ ಅಂತಹವರು ಮಾತ್ರ ಅಲ್ಲಿಗೆ ಬನ್ನಿ. ಅವರನ್ನು ನಾನು ನನ್ನ ಶಿಷ್ಯರೆಂದು ಸ್ವೀಕರಿಸುತ್ತೇನೆ.” ಅಷ್ಟು ದೂರದ ಸ್ಥಳಕ್ಕೆ ಬಹಳ ಜನ ಬಂದಾರೆಂದು ಅವರು ನಿರೀಕ್ಷಿಸಿರಲಿಲ್ಲ. ಆದರೆ ಬಂದವರಿಗೆಲ್ಲ ಸಂಪೂರ್ಣ ಸಹಕಾರ ಸಹಾಯ ನೀಡಲು ಅವರು ಸಿದ್ಧರಾಗಿದ್ದರು. ಸಹಸ್ರದ್ವೀಪೋದ್ಯಾನಕ್ಕೆ ಬರಲು ನಾಲ್ಕೈದು ಜನ ಮುಂದಾದರು. ಆದರೆ ಉಳಿದವರು ಬರಲು ಉತ್ಸುಕರಾಗಿರಲಿಲ್ಲವೆಂದಲ್ಲ, ದುರದೃಷ್ಟವಶಾತ್ ಅವರಿಗೆ ಬರಲು ಸಾಧ್ಯವಿರಲಿಲ್ಲ, ಅಷ್ಟೆ. ಸ್ವಾಮೀಜಿ ಸಹಸ್ರದ್ವೀಪೋದ್ಯಾನಕ್ಕೆ ಹೋದರೂ ಉತ್ಸಾಹಿಗಳಾದ ವಿದ್ಯಾರ್ಥಿಗಳು ಇತ್ತ ನ್ಯೂಯಾರ್ಕಿನ ತರಗತಿಗಳನ್ನು ಮುಂದುವರಿಸಿಕೊಂಡು ಬಂದರು. ಮುಂದೆ ಇದನ್ನು ಕೇಳಿದಾಗ ಸ್ವಾಮೀಜಿ ಅತ್ಯಂತ ಸಂತೋಷಪಟ್ಟರು.

ಕ್ಯಾಂಪ್ ಪರ್ಸಿಯಿಂದ ಸ್ವಾಮೀಜಿ ನೇರವಾಗಿ ಸಹಸ್ರದ್ವೀಪೋದ್ಯಾನಕ್ಕೆ ಹೋಗುವುದೆಂದು ಮೊದಲೇ ನಿಶ್ಚಿತವಾಗಿತ್ತು. ಅವರ ವಿದ್ಯಾರ್ಥಿಗಳೆಲ್ಲ ಅದಾಗಲೇ ಅಲ್ಲಿಗೆ ಹೋಗಿ ಅವರನ್ನು ಸ್ವಾಗತಿಸಲು ಸಿದ್ಧರಾಗಿದ್ದರು. ಮಿಸ್ ಡಚರ್​ಳಿಗೆ ಸೇರಿದ್ದ ಮನೆಗೆ ಹೊಸ ಕೋಣೆಗಳನ್ನು ಸೇರಿಸಿ ಸರ್ವಸಿದ್ಧತೆಗಳನ್ನೂ ಮಾಡಲಾಗಿತ್ತು. ಅಲ್ಲದೆ ಧ್ಯಾನಾದಿಗಳಿಗಾಗಿ ಇನ್ನೊಂದು ಹಜಾರವನ್ನು ಕಟ್ಟಿಸಲಾಗಿತ್ತು. ಅತ್ಯಂತ ಸರಳವೂ ಆಹ್ಲಾದಕರವೂ ಆದ ಈ ಕಟ್ಟಡ ವಿಶಾಲವಾದ ಕಿಟಕಿಗಳಿಂದ ಕೂಡಿತ್ತು. ಇದು ಮೂರು ಮಹಡಿಗಳ ಎತ್ತರದ ಕಟ್ಟಡ. ಇದರ ತಾರಸಿಯು ಮೇಲೆ ನಿಂತರೆ ಹಲವಾರು ಮೈಲಿಗಳವರೆಗಿನ ವಿಹಂಗಮ ನೋಟ ಕಾಣುತ್ತಿತ್ತು. (ಸ್ವಾಮೀಜಿಯ ವಾಸದಿಂದ, ಅವರ ಅತಿ ಶ್ರೇಷ್ಠ ಆಧ್ಯಾತ್ಮಿಕ ಬೋಧನೆಗಳಿಂದ ಪುನೀತವಾದ ಈ ಕಟ್ಟಡವನ್ನು ನ್ಯೂಯಾರ್ಕಿನ ರಾಮಕೃಷ್ಣ ವಿವೇಕಾನಂದ ಕೇಂದ್ರವು ೧೯೪೭ರಲ್ಲಿ ತನ್ನ ವಶಕ್ಕೆ ತೆಗೆದುಕೊಂಡಿತು. ಮತ್ತು ಕಟ್ಟಡದ ಮೂಲ ಆಕೃತಿಗೆ ಸ್ವಲ್ಪವೂ ಧಕ್ಕೆಯುಂಟಾಗದಂತೆ ಸಂಪೂರ್ಣವಾಗಿ ದುರಸ್ತಿ ಮಾಡಲಾಯಿತು.)

ಕಟ್ಟಡದ ಮೇಲಿನ ಮಹಡಿಯ ಕೈಸಾಲೆಯೇ ಅತಿ ಮುಖ್ಯವಾದ ಸ್ಥಳವಾಗಿತ್ತು. ಸ್ವಾಮೀಜಿ ತರಗತಿಗಳನ್ನು ನಡೆಸುತ್ತಿದ್ದುದು ಇಲ್ಲೇ. ಪ್ರತಿ ರಾತ್ರಿ ಊಟವಾದ ಮೇಲೆ ಎಲ್ಲರೂ ಈ ಕೈಸಾಲೆ ಯಲ್ಲಿ ಸೇರುತ್ತಿದ್ದರು. (ಆಗ ಅಲ್ಲಿ ವಿದ್ಯುದ್ದೀಪದ ಸೌಕರ್ಯವಿನ್ನೂ ಇರಲಿಲ್ಲ.) ಕತ್ತಲಲ್ಲಿ ಶಿಷ್ಯ ರೆಲ್ಲ ಶಾಂತವಾಗಿ ಕುಳಿತು, ತಮ್ಮ ಪರಮಪೂಜ್ಯ ಪರಮಪ್ರಿಯ ಗುರುವಿನ ಮುಖಕಮಲದಿಂದ ಹರಿದುಬರುವ ದಿವ್ಯವಾಣಿಯ ಅಮೃತವನ್ನು ಸವಿಯುತ್ತಿದ್ದರು. ಸುತ್ತಲೂ ದಟ್ಟವಾದ ಮರಗಳು, ಜೊತೆಗೆ ಜನ ಸಂಚಾರವೂ ತೀರ ಕಡಿಮೆ; ಆದ್ದರಿಂದ ಆ ಮನೆಯು ನಿರ್ಜನವಾದ ಅರಣ್ಯ ದಲ್ಲಿದ್ದಂತಿತ್ತು. ಮರಗಳಾಚೆಗೆ ಸೈಂಟ್ ಲಾರೆನ್ಸ್ ನದಿಯೂ ದೂರದಲ್ಲಿ ಚಿಕ್ಕಪುಟ್ಟ ದ್ವೀಪಗಳೂ ಕಾಣುತ್ತಿದ್ದುವು. ಆ ದ್ವೀಪಗಳಲ್ಲಿ ಮಿನುಗುತ್ತಿದ್ದ ದೀಪಗಳ ಬೆಳಕು ನದಿಯಲ್ಲಿ ಪ್ರತಿಬಿಂಬಿತ ವಾಗುತ್ತಿತ್ತು. ಒಟ್ಟಿನಲ್ಲಿ, ಈ ಕೋಣೆಯಿಂದಾಚೆ ನೋಡಿದರೆ ಅಲ್ಲೊಂದು ಕಿನ್ನರಲೋಕವೇ ಕಾಣುತ್ತಿತ್ತು. ಹೀಗೆ ನೀರವ ರಾತ್ರಿಯಲ್ಲಿ ಸ್ವಾಮೀಜಿ ತಮ್ಮ ಜೇನು ಕಂಠದಿಂದ ಆಧ್ಯಾತ್ಮಿಕ ವಿಷಯಗಳನ್ನು ವಿವರಿಸುತ್ತಿದ್ದರೆ ಯಾರಿಗೂ ಸಮಯ ಉರುಳಿದ್ದರ ಪರಿವೆಯೇ ಇರುತ್ತಿರಲಿಲ್ಲ. ಒಮ್ಮೆಯಂತೂ, ಉದಿಸಿದ ಪೂರ್ಣಚಂದ್ರ ಪಶ್ಚಿಮ ದಿಗಂತವನ್ನು ತಲುಪುವವರೆಗೂ ಪ್ರವಚನ ಸಾಗಿತ್ತು. ಇಂತಹ ಮನೋಹರ ವಾತಾವರಣದಲ್ಲಿ ಆ ಭಾಗ್ಯಶಾಲೀ ಶಿಷ್ಯರು ಏಳು ಅತ್ಯಮೂಲ್ಯ ವಾರಗಳನ್ನು ಕಳೆದರು.

ಸಹಸ್ರದ್ವೀಪೋದ್ಯಾನಕ್ಕೆ ಬಂದ ಶಿಷ್ಯರೆಂದರೆ, ಮನೆಯ ಒಡತಿ ಮಿಸ್ ಡಚರ್, ಮಿಸ್ ಸಾರಾ ಎಲೆನ್ ವಾಲ್ಡೊ (ಮುಂದೆ ಹರಿದಾಸಿ), ಮಿಸ್ ರೂತ್, ಎಲ್ಲಿಸ್, ಕ್ರಿಸ್ಟೀನಾ ಗ್ರೀನ್ಸ್​ಟೈಡೆಲ್ (ಮುಂದೆ ಸೋದರಿ ಕ್ರೀಸ್ಟೀನ), ಡಾ ॥ ವೈಟ್, ಶ್ರೀಮತಿ ಸ್ಟೆಲ್ಲಾ ಕ್ಯಾಂಬೆಲ್, ವಾಲ್ಟರ್ ಗುಡ್ ಇಯರ್ ಹಾಗೂ ಆತನ ಪತ್ನಿ ಫ್ರಾನ್ಸೆಸ್ ಗುಡ್ ಇಯರ್, ಮೇಡಮ್ ಮೆರಿ ಲೂಯಿಸ್ (ಮುಂದೆ ಸ್ವಾಮಿ ಅಭಯಾನಂದಾ), ಲಿಯಾನ್ ಲ್ಯಾಂಡ್ಸ್​ಬರ್ಗ್ (ಮುಂದೆ ಸ್ವಾಮಿ ಕೃಪಾನಂದ) ಮತ್ತು ಶ್ರೀಮತಿ ಮೇರಿಫಂಕೆ. (ಮತ್ತೊಬ್ಬರು ಯಾರೆಂಬುದು ತಿಳಿದುಬಂದಿಲ್ಲ.) ಇವರಲ್ಲಿ ಡಾ॥ ವೈಟ್ ಎಪ್ಪತ್ತು ವರ್ಷದ ‘ಬಾಲಕ’. ಹಿಂದೆಯೇ ತಿಳಿಸಿದಂತೆ ನ್ಯೂಯಾರ್ಕಿನಲ್ಲಿ ಸ್ವಾಮೀಜಿ ಯಿಂದ ದೂರವಾಗಿದ್ದ ಲ್ಯಾಂಡ್ಸ್​ಬರ್ಗ್, ಸಹಸ್ರದ್ವೀಪೋದ್ಯಾನದಲ್ಲಿ ಅವರನ್ನು ಮತ್ತೆ ಕೂಡಿಕೊಂಡ.

ಸ್ವಾಮೀಜಿಯ ದಿವ್ಯ ಸಾನ್ನಿಧ್ಯದಲ್ಲಿ ಕಳೆದ ಅತ್ಯಂತ ಆನಂದದಾಯಕವಾದ ಈ ದಿನಗಳ ಬಗ್ಗೆ ಮಿಸ್ ವಾಲ್ಡೊ ತನ್ನ ಸ್ಮೃತಿಚಿತ್ರಣದಲ್ಲಿ ಹೀಗೆ ಬರೆಯುತ್ತಾಳೆ: 

“ಸ್ವಾಮೀಜಿಯೊಂದಿಗೆ ಅಲ್ಲಿ ವಾಸಿಸಿದ ಭಾಗ್ಯವಂತರ ಪಾಲಿಗೆ ಆ ದಿನಗಳು ಅತ್ಯಂತ ಚಿರಸ್ಮರಣೀಯವಾದವುಗಳು. ಆಧ್ಯಾತ್ಮಿಕ ಬೆಳವಣಿಗೆಗೆ ಅಲ್ಲಿ ಅಪೂರ್ವ ಅವಕಾಶವಿತ್ತು. ನ್ಯೂಯಾರ್ಕಿನಿಂದ ಅಲ್ಲಿಗೆ ಸ್ವಾಮೀಜಿಯನ್ನು ಹಿಂಬಾಲಿಸಿ ಹೋಗಿ, ಪ್ರತಿದಿನವೂ ಸಂತೋಷ ದಿಂದ ಅವರ ಸೇವೆ ಮಾಡುತ್ತ, ಹೃತ್ಪೂರ್ವಕ ಕೃತಜ್ಞತಾಭಾವದಿಂದ ಅವರ ಮಾತುಗಳನ್ನಾಲಿಸಿದ ಕೆಲವೇ ಮಂದಿಯ ಪಾಲಿಗೆ ಆ ದಿನಗಳು ಎಷ್ಟು ಮಹತ್ವ ಪೂರ್ಣವಾಗಿದ್ದುವು. ಮತ್ತು ಇಂದಿಗೂ ಈ ವ್ಯಕ್ತಿಗಳ ಜೀವನದಲ್ಲಿ ಅವುಗಳ ಪಾತ್ರವೆಂಥದು ಎಂಬುದನ್ನು ಶಬ್ದಗಳಿಂದ ವರ್ಣಿಸಲು ಸಾಧ್ಯವಿಲ್ಲ. (ಈ ಮಾತುಗಳನ್ನು ಮಿಸ್ ವಾಲ್ಡೊ ಬರೆದದ್ದು ೧೯೨೭ರಲ್ಲಿ, ಎಂದರೆ ಸುಮಾರು ಮೂವತ್ತೆರಡು ವರ್ಷಗಳ ಅನಂತರ.) ಸ್ವಾಮೀಜಿ ತಮ್ಮ ಹೃದಯಾಂತರಾಳದಿಂದ ಬೋಧನೆ ಗಳನ್ನು ನೀಡಿದರು; ಅವರು ದೈವೀ ಸ್ಫೂರ್ತಿಯಿಂದ ಮಾತನಾಡಿದರು.

“ಅವರ ಆ ಮಾತುಗಳೆಲ್ಲವನ್ನೂ ಬರೆದಿಟ್ಟುಕೊಳ್ಳಲು ಸಾಧ್ಯವಾಗಿಲ್ಲ. ಆದರೆ ಕೇಳಿದವರ ಹೃದಯಗಳಲ್ಲಿ ಮಾತ್ರ ಅವು ಸಂರಕ್ಷಿಸಲ್ಪಟ್ಟಿವೆ. ಆ ದಿವ್ಯ ಸಮಯದಲ್ಲಿ ನಮಗಾಗುತ್ತಿದ್ದ ಉತ್ಕರ್ಷದ ಅನುಭವವನ್ನು, ನಾವು ನಡೆಸಿದ ತೀವ್ರತರ ಆಧ್ಯಾತ್ಮಿಕ ಜೀವನವನ್ನು ನಾವೆಂದೂ ಮರೆಯಲಾರೆವು. ಸ್ವಾಮೀಜಿಯ ಹೃದಯದ ಭಾವನೆಗಳೆಲ್ಲ ಆಗ ಉಕ್ಕಿಹರಿಯುತ್ತಿದ್ದುವು. ಅವರು ಪಟ್ಟ ಕಷ್ಟಗಳು–ಎದುರಿಸಿದ ಹೋರಾಟಗಳೆಲ್ಲವೂ ನಮ್ಮ ಮುಂದೆ ಅಭಿನಯಿತವಾಗು ತ್ತಿತ್ತು. ನಮ್ಮ ಎಲ್ಲ ಸಂದೇಹಗಳಿಗೆ ಉತ್ತರ ನೀಡಲು, ಎಲ್ಲ ಭಯವನ್ನು ನಿವಾರಿಸಲು ಸ್ವಯಂ ಅವರ ಗುರುದೇವರ ಚೈತನ್ಯವೇ ಅವರ ಮೂಲಕ ಮಾತನಾಡುತ್ತಿರುವಂತಿತ್ತು. ಎಷ್ಟೋ ಸಲ ಸ್ವಾಮೀಜಿ ಮಾತನಾಡುತ್ತಿರುವಾಗ, ಅವರಿಗೆ ನಮ್ಮ ಕಡೆಗೆ ಗಮನವೇ ಇದ್ದಂತೆ ಕಾಣುತ್ತಿರಲಿಲ್ಲ. ಆಗ ನಾನು, ಎಲ್ಲಿ ಅವರ ಗಂಭೀರ ಆಲೋಚನಾಲಹರಿಗೆ ತಡೆಯುಂಟಾಗುತ್ತದೆಯೋ ಎಂಬ ಭಯದಿಂದ ಉಸಿರು ಬಿಗಿಹಿಡಿದು ಕುಳಿತಿರುತ್ತಿದ್ದೆವು. ಅವರು ಕೆಲವು ಸಲ ಸ್ಫೂರ್ತಿಯ ರಭಸ ದಲ್ಲಿ ತಮ್ಮ ಆಸನದಿಂದೆದ್ದು ಆ ಪಡಸಾಲೆಯಲ್ಲಿ ಅತ್ತಿಂದಿತ್ತ ನಡೆದಾಡುತ್ತ ತಮ್ಮ ವಾಗ್ಮಿತೆಯ ಪರಿಪೂರ್ಣಪ್ರವಾಹವನ್ನೇ ಹರಿಯಿಸಿಬಿಡುತ್ತಿದ್ದರು... ಅವರ ಅಗ್ನಿಜ್ವಾಲೆಯಂತಹ ಮಾತು ಗಳು ಎಷ್ಟು ತೀವ್ರತರವಾಗಿರುತ್ತಿದ್ದುವು ಮತ್ತು ಎಷ್ಟು ಪ್ರಶ್ನಾತೀತವಾಗಿರುತ್ತಿದ್ದುವೆಂದರೆ, ಅವು ಕೇಳುಗರ ಹೃದಯದ ಮೇಲಿನ ಸಂದೇಹದ ಪರೆಗಳನ್ನೆಲ್ಲ ದಹಿಸುತ್ತ ನೇರವಾಗಿ ಒಳಪ್ರವೇಶಿಸುತ್ತಿದ್ದುವು.

“ಹೀಗೆ ಮಾತನಾಡುವ ಸಮಯಗಳಲ್ಲಿ ಅವರು ಎಂದಿಗಿಂತಲೂ ಅತ್ಯಂತ ಮೃದುವಾಗಿರುತ್ತಿ ದ್ದರು, ಮತ್ತು ಪ್ರೀತಿಯುತರಾಗಿರುತ್ತಿದ್ದರು. ಅವರ ಬೋಧನೆಯ ರೀತಿ ಹೇಗಿತ್ತೆಂದರೆ– ಬಹುಶಃ ಅವರ ಗುರುದೇವನ (ಶ್ರೀರಾಮಕೃಷ್ಣರ) ರೀತಿಯಂತೆಯೇ–ಅವರ ಅಂತರಾತ್ಮವು ತನ್ನೊಂದಿಗೇ ನಡೆಸುವ ಸಂಭಾಷಣೆಯನ್ನು ಆಲಿಸಲು ಇತರರಿಗೆ ಅವಕಾಶ ಮಾಡಿಕೊಡುತ್ತಿರುವಂತಿತ್ತು!”

“ಸ್ವಾಮಿ ವಿವೇಕಾನಂದರಂತಹ ವ್ಯಕ್ತಿಯೊಂದಿಗೆ ಇರುವುದೆಂದರೆ ಅದೊಂದು ಚಿರಂತನ ಸ್ಫೂರ್ತಿಯ ಅನುಭವ. ಬೆಳಗಿನಿಂದ ರಾತ್ರಿಯವರೆಗೂ ನಾವು ಸತತವಾಗಿ ಒಂದೇ ರೀತಿಯ ತೀವ್ರ ಆಧ್ಯಾತ್ಮಿಕ ವಾತಾವರಣದಲ್ಲಿದ್ದೆವು. ಎಷ್ಟೋ ಸಲ ಸ್ವಾಮೀಜಿ ತಮಾಷೆ ಮಾಡುತ್ತ, ಅಣಕವಾಡುತ್ತ, ಖುಷಿಯಾಗಿರುತ್ತ, ಚುರುಕು ಪ್ರತ್ಯುತ್ತರಗಳನ್ನು ನೀಡುತ್ತ ಇದ್ದರೂ ತಮ್ಮ ಜೀವನತತ್ತ್ವದ ಪ್ರಧಾನ ಸ್ವರದಿಂದ ವಿಚಲಿತರಾಗುತ್ತಿರಲಿಲ್ಲ. ಪ್ರತಿಯೊಂದು ಘಟನೆಯೂ ಅವರಿಗೊಂದು ವಿಷಯವನ್ನೊದಗಿಸುತ್ತಿತ್ತು, ಉಪಮೆಯನ್ನೊದಗಿಸುತ್ತಿತ್ತು. ಅವರ ಬಳಿ ಎಂದಿಗೂ ಮುಗಿಯದ ಪೌರಾಣಿಕ ಕಥೆಗಳ ಗಣಿಯೇ ಇತ್ತು. ಅವುಗಳ ಅತ್ಯಂತ ಗಾಢ ತತ್ತ್ವಜ್ಞಾನ ದಲ್ಲಿ ನಾವು ಕೊಚ್ಚಿಕೊಂಡು ಹೋಗುತ್ತಿದ್ದೆವು... 

“ಸ್ವಾಮೀಜಿ ನಮ್ಮ ಮುಂದಿಡುತ್ತಿದ್ದ ಹಿಂದೂ ತತ್ತ್ವವಾದಗಳ ಭಾವನೆಗಳು ನಾವು ಎಂದೂ ಕೇಳರಿಯದಂಥವು. ಆದ್ದರಿಂದ ಅವುಗಳನ್ನು ಜೀರ್ಣಿಸಿಕೊಳ್ಳುವಲ್ಲಿ ನಾವು ಸ್ವಲ್ಪ ಹಿಂದುಳಿಯು ತ್ತಿದ್ದೆವು. ಆದರೆ ಸ್ವಾಮೀಜಿಯ ಉತ್ಸಾಹ-ತಾಳ್ಮೆಗಳೆಂದೂ ಕಡಿಮೆಯಾಗಲಿಲ್ಲ.

“ದಿನಗಳುರುಳಿದಂತೆ, ನಾವು ಕೇಳುತ್ತಿದ್ದ ಆ ಮಾತುಗಳ ನಿಜವಾದ ಅರ್ಥವನ್ನು ಮನಗಾಣಲು ತೊಡಗಿದೆವು. ಆ ಬೋಧನೆಗಳನ್ನು ನಾವು ಹೃತ್ಪೂರ್ವಕವಾಗಿ ಸ್ವೀಕರಿಸಿದೆವು. ನಮ್ಮಲ್ಲಿ ಪ್ರತಿ ಯೊಬ್ಬರೂ ಸ್ವಾಮೀಜಿಯಿಂದ ಮಂತ್ರದೀಕ್ಷೆ ಪಡೆದೆವು. ತನ್ಮೂಲಕ ನಾವು ನಿಜಾರ್ಥದಲ್ಲಿ ಅವರ ಶಿಷ್ಯರಾದೆವು, ಅವರು ನಮ್ಮ ಗುರುವಾದರು. ಪತಿ-ಪತ್ನಿಯರ ಅಥವಾ ತಾಯಿ-ಮಕ್ಕಳ ಪ್ರೇಮ ಬಂಧನಕ್ಕಿಂತಲೂ ಗುರು-ಶಿಷ್ಯರ ಸಂಬಂಧವು ಹೆಚ್ಚು ಬಲವಾದುದೆಂದು ಭಾರತದಲ್ಲಿ ಪರಿಗಣಿಸ ಲ್ಪಡುತ್ತದೆ. ನಾವು ಸರಿಯಾಗಿ ಹನ್ನೆರಡು ಜನರಿದ್ದುದು ಎಷ್ಟು ಕಾಕತಾಳೀಯ! (ಏಸುಕ್ರಿಸ್ತನಿಗೂ ಹನ್ನೆರಡು ಜನ ಶಿಷ್ಯರಿದ್ದರೆಂಬುದನ್ನು ಸ್ಮರಿಸಬಹುದು.)

“ದೀಕ್ಷಾಪ್ರದಾನ ಕಾರ್ಯಕ್ರಮವು ಅದರ ಅತೀವ ಸರಳತೆಯಿಂದಾಗಿ ತುಂಬು ಮನಮುಟ್ಟು ವಂತಿತ್ತು. ಒಂದು ಸಣ್ಣ ಅಗ್ನಿಕುಂಡ, ಸುಂದರವಾದ ಹೂಗಳು–ಇವುಗಳು ಮಾತ್ರವೇ ಆ ಕಾರ್ಯಕ್ರಮವು ನಮ್ಮ ದೈನಂದಿನ ಕಾರ್ಯಕ್ರಮಕ್ಕಿಂತ ವಿಭಿನ್ನವಾಗಿತ್ತೆಂಬುದರ ಗುರುತು ಗಳಾಗಿದ್ದುವು. ಅದು ಜರುಗಿದುದು ಸುಂದರ ಬೇಸಿಗೆಯ ದಿನವೊಂದರ ಸೂರ್ಯೋದಯ ಸಮಯದಲ್ಲಿ. ಆ ದೃಶ್ಯವು ನಮ್ಮ ನೆನಪಿನಲ್ಲಿ ಇನ್ನೂ ಹಸಿರಾಗಿದೆ.

“ನಮ್ಮ ಪೈಕಿ ಇಬ್ಬರಿಗೆ ಸ್ವಾಮೀಜಿ ಸಂನ್ಯಾಸ ದೀಕ್ಷೆಯನ್ನು ಅನುಗ್ರಹಿಸಿದರು ಮತ್ತು ಐವರು ಯುವತಿಯರಿಗೆ ಬ್ರಹ್ಮಚರ್ಯದೀಕ್ಷೆ ಕೊಟ್ಟು ಬ್ರಹ್ಮಚಾರಿಣಿಯರನ್ನಾಗಿ ಮಾಡಿದರು.”

ಸಹಸ್ರದ್ವೀಪೋದ್ಯಾನದಲ್ಲಿ ಶಿಷ್ಯರೂ ಗುರುವೂ ಸೇರಿ, ಪರಸ್ಪರ ಸಹಕಾರದ ಸಹಜೀವನ ನಡೆಸುತ್ತಿದ್ದರು. ಇಲ್ಲಿ ಅಡಿಗೆಗಾಗಲಿ ಕೆಲಸಕ್ಕಾಗಲಿ ಯಾರನ್ನೂ ನೇಮಿಸಿಕೊಳ್ಳಬಾರದು, ಎಲ್ಲ ವನ್ನೂ ತಾವೇ ಮಾಡಿಕೊಳ್ಳಬೇಕು ಎಂದು ಸ್ವಾಮೀಜಿ ಮೊದಲೇ ಹೇಳಿಬಿಟ್ಟಿದ್ದರು. ಅದಕ್ಕೆ ಎಲ್ಲರೂ ಒಪ್ಪಿಕೊಂಡಿದ್ದರು. ಮುಂಜಾನೆ ಎದ್ದೊಡನೆ ಎಲ್ಲರೂ ತಮ್ಮತಮ್ಮ ಪಾಲಿನ ಕರ್ತವ್ಯ ಗಳನ್ನು ಅಚ್ಚುಕಟ್ಟಾಗಿ, ಶ್ರದ್ಧೆಯಿಂದ ನಿರ್ವಹಿಸುತ್ತಿದ್ದರು. ಕಡಿಮೆ ಸ್ಥಳದಲ್ಲಿ ಹೆಚ್ಚು ಜನ ಇರ ಬೇಕಾದುದರಿಂದ ಸ್ವಲ್ಪ ಅನನುಕೂಲವಾಯಿತಾದರೂ ಯಾರೂ ಅದನ್ನು ಲೆಕ್ಕಿಸಲಿಲ್ಲ. ಸಾಧ್ಯ ವಾದಷ್ಟು ಬೇಗ ಕೆಲಸಗಳನ್ನೆಲ್ಲ ಮುಗಿಸಿ, ತರಗತಿಗೆ ಓಡಿಬರುತ್ತಿದ್ದರು. ಅಷ್ಟು ಹೊತ್ತಿಗಾಗಲೇ ಸ್ವಾಮೀಜಿ ಸಿದ್ಧರಾಗಿ ಇತರರನ್ನು ಅವಸರಪಡಿಸುತ್ತಿದ್ದರು. ಅಡಿಗೆ ಮಾಡುವುದರಲ್ಲಿ ಸಿದ್ಧಹಸ್ತ ರಾದ ಸ್ವಾಮೀಜಿ, ಎಷ್ಟೋ ಸಲ ಉಪಹಾರವನ್ನೂ ಅಡಿಗೆಯನ್ನೂ ತಾವೇ ತಯಾರಿಸುತ್ತಿದ್ದರು.

ಬೆಳಗಿನ ತರಗತಿಗಳಲ್ಲಿ ಭಗವದ್ಗೀತೆ, ಉಪನಿಷತ್ತುಗಳು, ವೇದಾಂತ ಸೂತ್ರಗಳು ಇವುಗಳಲ್ಲಿ ಯಾವುದಾದರೊಂದರ ಬಗ್ಗೆ ಪ್ರವಚನ ನಡೆಯುತ್ತಿತ್ತು. ಕೆಲವೊಮ್ಮೆ ಮಧ್ವಾಚಾರ್ಯರ ದ್ವೈತ ಸಿದ್ಧಾಂತವನ್ನು, ಇಲ್ಲವೆ ರಾಮಾನುಜಾಚಾರ್ಯರ ವಿಶಿಷ್ಟಾದ್ವೈತ ಸಿದ್ಧಾಂತವನ್ನು ವಿವರಿಸು ತ್ತಿದ್ದರು. ಆದರೆ ಅವರು ಹೆಚ್ಚಾಗಿ ವಿವರಿಸುತ್ತಿದ್ದುದು ಶಂಕರಾಚಾರ್ಯರ ಅದ್ವೈತ ಸಿದ್ಧಾಂತ ವನ್ನು. ಅದ್ವೈತ ಸಿದ್ಧಾಂತವು ಬಹಳ ಸೂಕ್ಷ್ಮವಾದದ್ದು. ಆ ಪಾಶ್ಚಾತ್ಯಶಿಷ್ಯರಿಗೆ ಅದು ಸುಲಭವಾಗಿ ಅರ್ಥವಾಗಲು ಸಾಧ್ಯವಿರಲಿಲ್ಲ. ಭಾರತೀಯ ತತ್ತ್ವಶಾಸ್ತ್ರವೇ ಅವರಿಗೆ ಅಷ್ಟೇನೂ ಸುಲಭವಲ್ಲ. ಅದರಲ್ಲೂ ಅದ್ವೈತ ಅಷ್ಟು ಬೇಗ ಜೀರ್ಣವಾದೀತೆ! ಇನ್ನು ಸ್ವಾಮೀಜಿಯ ಬೋಧನೆಗಳ ರಭಸವನ್ನು ತಡೆದುಕೊಳ್ಳಬೇಕಾದರೆ ಸಾಮಾನ್ಯರಿಂದ ಸಾಧ್ಯವೇ ಇರಲಿಲ್ಲ.

ಕೆಲವೊಮ್ಮೆ ಸ್ವಾಮೀಜಿ ನಾರದ ಭಕ್ತಿಸೂತ್ರಗಳ ಮೇಲೆ ತರಗತಿಗಳನ್ನು ಮಾಡುತ್ತಿದ್ದರು. ಪುರಾಣಗಳು ಹಾಗೂ ರಾಮಾಯಯಣ-ಮಹಾಭಾರತಗಳನ್ನೂ ಆಗಾಗ ಕೈಗೆತ್ತಿಕೊಳ್ಳುತ್ತಿದ್ದರು. ಪುರಾಣಗಳ ವಿಚಿತ್ರ ಕಥೆಗಳನ್ನು ತಮ್ಮ ಅನನುಕರಣೀಯ ಶೈಲಿಯಲ್ಲಿ ವಿವರಿಸಿ, ಅವುಗಳ ಹಿಂದಿನ ತತ್ತ್ವವನ್ನು ಎತ್ತಿಹಿಡಿಯುತ್ತಿದ್ದರು. ಈ ಕಥೆಗಳೆಂದರೆ ಶಿಷ್ಯರಿಗೆಲ್ಲ ವಿಶೇಷ ಆಸಕ್ತಿ. ಎಂತಹ ನೀರಸ ವಸ್ತುವನ್ನೂ ರಸಭರಿತವನ್ನಾಗಿಸಬಲ್ಲ ಸ್ವಾಮೀಜಿ, ಭಾರತೀಯರ ಮೆಚ್ಚಿನ ಕಥೆಗಳನ್ನು ವಿವರಿಸತೊಡಗಿದರೆ ವಿದ್ಯಾರ್ಥಿಗಳಿಗೆ ಅದೆಂಥ ರಸಾನುಭವ!

ಈ ದಿನಗಳಲ್ಲೇ ಸ್ವಾಮೀಜಿ ಮೊಟ್ಟಮೊದಲ ಬಾರಿಗೆ ತಮ್ಮ ಪಾಶ್ಚಾತ್ಯ ಶಿಷ್ಯರಿಗೆ ಶ್ರೀರಾಮ ಕೃಷ್ಣರ ಬಗ್ಗೆ ವಿವರವಾಗಿ ತಿಳಿಸಿದರು. ಅತ್ಯಂತ ರೋಮಾಂಚಕಾರಿಯಾದ ಶ್ರೀರಾಮಕೃಷ್ಣರ ಜೀವನದ ಬಗ್ಗೆ ಕೇಳಿದಾಗ ಶಿಷ್ಯರಿಗಾದ ಆನಂದಾಶ್ಚರ್ಯ ವರ್ಣನಾತೀತ. ತಮ್ಮ ಪ್ರಿಯ ಗುರು ದೇವನ ಬಗ್ಗೆ ಮಾತನಾಡುವಾಗ ಸ್ವಾಮೀಜಿ ವಿಶೇಷ ಸ್ಫೂರ್ತಿಯಿಂದ, ವಿಶೇಷ ಭಾವದಿಂದ ಕೂಡಿರುತ್ತಿದ್ದರು. ಶ್ರೀರಾಮಕೃಷ್ಣರೊಂದಿಗಿನ ತಮ್ಮ ದೈನಂದಿನ ಜೀವನ ಹೇಗಿತ್ತು, ತಮ್ಮ ಅಪ ನಂಬಿಕೆಗಳನ್ನು-ಸಂದೇಹಗಳನ್ನು ಹೋಗಲಾಡಿಸಲು ಹೆಜ್ಜೆಹೆಜ್ಜೆಗೂ ಅವರು ಹೇಗೆ ಹೋರಾಡ ಬೇಕಾಯಿತು, ಎಂಬುದನ್ನೆಲ್ಲ ಸ್ವಾಮೀಜಿ ಕಣ್ಣಿಗೆ ಕಟ್ಟುವಂತೆ ಬಣ್ಣಿಸುತ್ತಿದ್ದರು. ತಮ್ಮ ಗುರು ದೇವನ ವಿಷಯ ಬಂದೊಡನೆ ಅವರು ಆವೇಶಭರಿತರಾಗಿ ಮಾತನಾಡುವುದನ್ನು ಕಂಡಾಗ ಆ ಶಿಷ್ಯರಿಗೆ ಗುರುಭಕ್ತಿಯೆಂದರೇನೆಂಬುದರ ಕಲ್ಪನೆಯಾಯಿತು.

ಸಹಸ್ರದ್ವೀಪೋದ್ಯಾನಕ್ಕೆ ಸ್ವಾಮೀಜಿಯನ್ನು ಅನುಸರಿಸಿ ಬಂದ ಶ್ರೀಮತಿ ಫಂಕೆ ಹಾಗೂ ಮಿಸ್ ಕ್ರಿಸ್ಟೀನ ಇವರ ಕಥೆ ಸ್ವಾರಸ್ಯಕರವಾಗಿದೆ. ಇವರು ಡೆಟ್ರಾಯ್ಟಿನಲ್ಲಿ ಸ್ವಾಮೀಜಿಯ ಉಪನ್ಯಾಸ ಗಳನ್ನು ಕೇಳಿ ಆಕರ್ಷಿತರಾಗಿದ್ದರೂ ಅವರನ್ನು ವೈಯಕ್ತಿಕವಾಗಿ ಕಂಡು ಮಾತನಾಡಲು ಸಾಧ್ಯ ವಾಗಿರಲಿಲ್ಲ. ಆದರೆ ಇವರಾಗಲೇ ಸ್ವಾಮೀಜಿಯ ವ್ಯಕ್ತಿತ್ವದಿಂದ ಹಾಗೂ ಬೋಧನೆಗಳಿಂದ ಎಷ್ಟು ಗಾಢವಾಗಿ ಪ್ರಭಾವಿತರಾಗಿದ್ದರೆಂದರೆ, ಏನಾದರೂ ಮಾಡಿ ಅವರನ್ನು ನೋಡಲೇಬೇಕು– ಅದಕ್ಕಾಗಿ ಅರ್ಧ ಪ್ರಪಂಚವನ್ನೇ ಸುತ್ತಿಕೊಂಡು ಹೋಗಬೇಕಾಗಿ ಬಂದರೂ ಸರಿಯೆ, ಎಂದು ತೀರ್ಮಾನಿಸಿಬಿಟ್ಟರು. ಆದರೆ ಸುಮಾರು ಒಂದೂವರೆ ವರ್ಷಗಳವರೆಗೆ ಇವರಿಗೆ ಸ್ವಾಮೀಜಿ ಯವರ ಸುಳಿವೇ ಸಿಕ್ಕಿರಲಿಲ್ಲ. ಆದ್ದರಿಂದ ಅವರು ಭಾರತಕ್ಕೆ ಹಿಂದಿರುಗಿರಬೇಕು ಎಂದು ಭಾವಿಸಿದ್ದರು. ಒಂದು ದಿನ ಈ ಸ್ನೇಹಿತೆಯರಿಗೆ ಯಾರಿಂದಲೋ ತಿಳಿದುಬಂದಿತು–ಸ್ವಾಮೀಜಿ ಇನ್ನೂ ಅಮೆರಿಕದಲ್ಲೇ ಇದ್ದಾರೆ; ಬೇಸಿಗೆಯನ್ನು ಕಳೆಯಲು ಸಹಸ್ರದ್ವೀಪೋದ್ಯಾನಕ್ಕೆ ಹೋಗಿ ದ್ದಾರೆ, ಎಂದು. ಈ ವರ್ತಮಾನ ಕಿವಿಗೆ ಬಿದ್ದ ತಕ್ಷಣವೇ ಶ್ರೀಮತಿ ಫಂಕೆ ಹಾಗೂ ಮಿಸ್ ಕ್ರಿಸ್ಟೀನ ಇಬ್ಬರೂ ಮತ್ತೆ ಚುರುಕಾದರು. ಮರುದಿನವೇ ಬೆಳಿಗ್ಗೆ ಇಬ್ಬರೂ ಅಲ್ಲಿಗೆ ಹೊರಟೇಬಿಟ್ಟರು!

ಆದರೆ ಸಹಸ್ರದ್ವೀಪೋದ್ಯಾನಕ್ಕೆ ಬಂದು ತಲುಪಿದ ಮೇಲೆ ಅವರಿಗೆ ಭಯಹುಟ್ಟಿಕೊಂಡಿತು. ತಾವು ಭಂಡಧೈರ್ಯ ಮಾಡಿ, ಹೇಳದೆ ಕೇಳದೆ ಇಲ್ಲಿಗೆ ಬಂದುಬಿಟ್ಟೆವಲ್ಲ, ಸ್ವಾಮೀಜಿಯ ಶಾಂತ-ಏಕಾಂತ ವಾಸಕ್ಕೆ ಭಂಗ ತಂದೆವೋ ಹೇಗೋ ಎಂದು ಮಾತನಾಡಿಕೊಂಡರು. ಆದರೆ ಸ್ವಾಮೀಜಿ ಅದಾಗಲೇ ಅವರ ಹೃದಯದಲ್ಲಿ ವ್ಯಾಕುಲತೆಯ ಕಿಡಿಯನ್ನು ಹೊತ್ತಿಸಿಯಾಗಿತ್ತು. ಸ್ವಾಮೀಜಿಯನ್ನು ಕಂಡು, ಅವರ ಬೋಧನೆಗಳ ಅಮೃತವನ್ನು ಸವಿಯುವವರೆಗೂ ಅದು ಆರು ವಂತಿರಲಿಲ್ಲ. ಈ ತೀವ್ರ ವ್ಯಾಕುಲತೆಯೇ ಇವರಿಬ್ಬರನ್ನು ಇಲ್ಲಿಯವರೆಗೆ ಎಳೆತಂದದ್ದು. ಇವರು ಆ ದ್ವೀಪಕ್ಕೆ ತಲುಪುವ ವೇಳೆಗಾಗಲೇ ರಾತ್ರಿಯಾಗಿತ್ತು. ಜೊತೆಗೆ ಮಳೆಯೂ ಸುರಿಯುತ್ತಿತ್ತು. ಯಾರನ್ನೋ ವಿಚಾರಿಸಿ ಸ್ವಾಮೀಜಿಯ ವಿಳಾಸವನ್ನು ಪಡೆದುಕೊಂಡರು. ಮೊದಲು ಸ್ವಾಮೀಜಿ ಯನ್ನು ನೋಡಿದ ಮೇಲೆಯೇ ವಿಶ್ರಾಂತಿಯ ಪ್ರಶ್ನೆ ಎಂದು ತೀರ್ಮಾನಿಸಿದರು. ಬಳಿಕ ದಾರಿ ತೋರಿಸಲು ಒಬ್ಬ ಕೂಲಿಯವನನ್ನು ನೇಮಿಸಿಕೊಂಡರು. ಅವನು ಲಾಟೀನು ಹಿಡಿದು ನಡೆದಂತೆ ಇವರಿಬ್ಬರೂ ಅವನನ್ನು ಹಿಂಬಾಲಿಸಿ ಮಳೆಯಲ್ಲೇ ಬಂದು ತಲುಪಿದರು. ಸ್ವಾಮೀಜಿಯನ್ನು ಭೇಟಿಯಾದಾಗ ಅವರೊಂದಿಗೆ ಏನು ಹೇಳಬೇಕು ಎಂಬುದನ್ನೆಲ್ಲ ಮೊದಲೇ ತೀರ್ಮಾನಿಸಿ ಸುಂದರವಾದ ವಾಕ್ಯಗಳನ್ನೆಲ್ಲ ಜೋಡಿಸಿಕೊಂಡಿದ್ದರು. ಆದರೆ ಸ್ವಾಮೀಜಿ ತಮ್ಮೆದುರಿಗೆ ನಿಂತಿರು ವುದರ ಅರಿವಾದಾಗ, ಅವರು ಹೇಳಬೇಕೆಂದುಕೊಂಡಿದ್ದೆಲ್ಲ ಮರೆತೇ ಹೋಯಿತು. “ಸ್ವಾಮೀಜಿ, ನಾವು ಡೆಟ್ರಾಯ್ಟಿನಿಂದ ಬರುತ್ತಿದ್ದೇವೆ; ಶ್ರೀಮತಿ ಪ. ನಮ್ಮನ್ನು ಕಳಿಸಿಕೊಟ್ಟರು” ಎಂದು ಒಬ್ಬಳು ಒದರಿಬಿಟ್ಟಳು. ಮತ್ತೊಬ್ಬಳು ಸ್ವಲ್ಪ ಧೈರ್ಯತಾಳಿ, “ಈ ಭೂಮಿಯಲ್ಲಿ ಕ್ರಿಸ್ತನಿನ್ನೂ ಇದ್ದಿದ್ದರೆ, ಅವನ ಬಳಿಗೆ ಹೋಗಿ ನಮಗೆ ಬೋಧಿಸುವಂತೆ ಹೇಗೆ ಕೇಳುತ್ತಿದ್ದೆವೋ ಹಾಗೆಯೇ ನಾವೀಗ ನಿಮ್ಮ ಬಳಿಗೆ ಬಂದಿದ್ದೇವೆ” ಎಂದಳು. ಅವರಿಬ್ಬರನ್ನೂ ಸ್ವಾಮೀಜಿ ತುಂಬ ಕರುಣೆ ಯಿಂದ ನೋಡುತ್ತ ಮೃದುವಾಗಿ ಹೇಳಿದರು, “ಓಹ್! ನಿಮ್ಮನ್ನು ಈಗಲೇ ಮುಕ್ತರನ್ನಾಗಿಸಲು, ಕ್ರಿಸ್ತನಲ್ಲಿರುವ ಶಕ್ತಿ ನನ್ನಲ್ಲಿದ್ದರೆ!” ಕ್ಷಣಕಾಲ ಅವರು ಆಲೋಚನಾಪರರಾಗಿ ನಿಂತರು. ಬಳಿಕ ಅಲ್ಲಿಯೇ ನಿಂತಿದ್ದ, ಮನೆಯ ಆತಿಥೇಯಳಾದ ಮಿಸ್ ಡಚರಳ ಕಡೆಗೆ ತಿರುಗಿ, “ಇವರು ಡೆಟ್ರಾಯ್ಟಿನಿಂದ ಬರುತ್ತಿದ್ದಾರೆ; ದಯವಿಟ್ಟು ಇವರನ್ನು ಮಹಡಿಯ ಮೇಲಕ್ಕೆ ಕರೆದುಕೊಂಡು ಹೋಗಿ, ವಸತಿಗೆ ವ್ಯವಸ್ಥೆ ಮಾಡು” ಎಂದರು.

ಸ್ವಾಮೀಜಿ ತಮ್ಮನ್ನು ಸ್ವೀಕರಿಸಿದಾಗ ಶ್ರೀಮತಿ ಫಂಕೆ ಹಾಗೂ ಮಿಸ್ ಕ್ರಿಸ್ಟೀನರಿಗಾದ ನೆಮ್ಮದಿ-ಸಮಾಧಾನ ಅಷ್ಟಿಷ್ಟಲ್ಲ. ಮರುದಿನದಿಂದಲೇ ಇಬ್ಬರೂ ತರಗತಿಗಳಲ್ಲಿ ಭಾಗವಹಿಸ ಲಾರಂಭಿಸಿದರು. ಹೀಗೆ ಸಹಸ್ರದ್ವೀಪೋದ್ಯಾನಕ್ಕೆ ಬಂದು ಸ್ವಾಮೀಜಿಯ ಸತ್ಸಂಗದಲ್ಲಿ ದಿನಗಳನ್ನು ಕಳೆಯಲಾರಂಭಿಸಿದ ಮೇಲೆ, ಅಲ್ಲಿನ ಜೀವನಕ್ರಮವನ್ನು ಅತ್ಯಂತ ಉತ್ಸಾಹದಿಂದ ಬಣ್ಣಿಸುತ್ತ ಶ್ರೀಮತಿ ಫಂಕೆ ತನ್ನ ಸ್ನೇಹಿತೆಯೊಬ್ಬಳಿಗೆ ಒಂದು ಪತ್ರ ಬರೆದಳು:

“ಅಂತೂ ಈಗ ನಾವು ಇಲ್ಲಿದ್ದೇವೆ–ವಿವೇಕಾನಂದರಿರುವ ಮನೆಯಲ್ಲೇ ಇದ್ದೇವೆ! ಬೆಳಿಗ್ಗೆ ಎಂಟು ಗಂಟೆಯಿಂದ ಹಿಡಿದು ರಾತ್ರಿ ಬಹಳ ಹೊತ್ತಿನವರೆಗೂ ಅವರ ಮಾತುಗಳನ್ನು ಕೇಳು ತ್ತಿದ್ದೇವೆ. ನನ್ನ ಅತ್ಯಂತ ವಿಚಿತ್ರ ಕನಸುಗಳಲ್ಲಿಯೂ ನಾನು ಇಷ್ಟು ಅದ್ಭುತವಾದದ್ದನ್ನು ಕಲ್ಪಿಸಿಕೊಂಡಿರಲಿಲ್ಲ–ಅಷ್ಟೊಂದು ಪರಿಪೂರ್ಣವಾಗಿದೆ ಇದು! ವಿವೇಕಾನಂದರೊಂದಿಗೆ ಇರುವುದು! ಅವರಿಂದ ಸ್ವೀಕೃತರಾಗುವುದು!

“ಓಹ್, ವಿವೇಕಾನಂದರ ದಿವ್ಯ ಬೋಧನೆಗಳೆಂಥವು! ಅಲ್ಲಿ ಅರ್ಥಹೀನವಾದದ್ದೊಂದೂ ಇಲ್ಲ; ಭೂತಪ್ರೇತಗಳ ವಿಚಾರವಿಲ್ಲ. ಅಲ್ಲಿರುವುದೆಲ್ಲ ಕೇವಲ ಭಗವಂತ, ಏಸುಕ್ರಿಸ್ತ, ಬುದ್ಧ. ಇನ್ನು ನಾನು ಹಳೆಯ ‘ನಾನಾ’ಗಿ ಉಳಿಯುವುದಿಲ್ಲವೆಂದು ನನಗನ್ನಿಸುತ್ತಿದೆ–ಏಕೆಂದರೆ, ನಾನು ಪರಮಸತ್ಯದ ಇಣುಕುನೋಟವೊಂದನ್ನು ಪಡೆದುಕೊಂಡಿದ್ದೇನೆ!

“ನಾವೀಗ ಆಕಾಶದಲ್ಲಿ ವಿಹರಿಸುತ್ತಿರುವಂತಿದೆ. ಡೆಟ್ರಾಯ್ಟ್ ಎಂಬುದೊಂದಿದೆ ಎನ್ನುವು ದನ್ನೇ ಮರೆತುಬಿಡಿ ಎನ್ನುತ್ತಾರೆ ಸ್ವಾಮೀಜಿ. ತಮ್ಮ ಮಾತುಗಳನ್ನು ಆಲಿಸುವಾಗ ನಮ್ಮ ಮನ ಸ್ಸನ್ನು ಯಾವುದೇ ಇತರ ಆಲೋಚನೆಯೂ ಆವರಿಸಲು ಬಿಡಬಾರದೆಂಬುದು ಅವರ ಇಚ್ಛೆ.... ಹುಲ್ಲುಕಡ್ಡಿಯಿಂದ ಹಿಡಿದು ಮನುಷ್ಯನವರೆಗೂ–ಅತ್ಯಂತ ಪೈಶಾಚಿಕ ವ್ಯಕ್ತಿಯಲ್ಲೂ–ಭಗ ವಂತನನ್ನು ಕಾಣುವುದು ಹೇಗೆಂಬುದನ್ನು ಅವರು ಬೋಧಿಸುತ್ತಾರೆ.

“ನಿಜಕ್ಕೂ ಇಲ್ಲಿ ಪತ್ರವನ್ನು ಬರೆಯಲು ಸಮಯವನ್ನು ಕಂಡುಕೊಳ್ಳುವುದು ಬಹಳ ಕಷ್ಟ. ಆರಾಮವಾಗಿ ಇರಲು, ವಿಶ್ರಾಂತಿ ಪಡೆಯಲು ಇಲ್ಲಿ ಸಮಯವೇ ಇಲ್ಲ. ಏಕೆಂದರೆ ಸ್ವಾಮೀಜಿ ಇಷ್ಟರಲ್ಲೇ ಇಂಗ್ಲೆಂಡಿಗೆ ತೆರಳಲಿರುವುದರಿಂದ, ಉಳಿದಿರುವ ಸಮಯ ತೀರ ಅಲ್ಪವೆಂದು ನಮಗೆ ಅನ್ನಿಸುತ್ತದೆ. ನಾವು ನಮ್ಮ ಉಡಿಗೆತೊಡಿಗೆಯ ಕಡೆಗೂ ಸರಿಯಾಗಿ ಗಮನಕೊಡುವುದಿಲ್ಲ. ಎಲ್ಲಿ ನಾವು ಕೆಲವು ಅಮೂಲ್ಯ ರತ್ನಗಳನ್ನು ಕಳೆದುಕೊಂಡುಬಿಡುತ್ತೇವೆಯೋ ಎಂದು ನಮಗೆ ಅಷ್ಟು ಭಯ!... ಸ್ವಾಮೀಜಿಯ ಮಾತಿನ ಪಲ್ಲವಿ ಒಂದೇ–‘ಮೊದಲು ದೇವರನ್ನು ಕಂಡುಕೊಳ್ಳಿ, ಉಳಿದುದೆಲ್ಲ ಆಮೇಲೆ.’ ತಾವು ಮಾತನಾಡುವಾಗ ಅವರು ವಿಷಯವನ್ನು ಬಿಟ್ಟು ಬಹುದೂರ ಹೋದರೂ, ತಮ್ಮೆಲ್ಲ ಬೋಧನೆಯ ಅಡಿಪಾಯದಂತಿರುವ ಆ ಪ್ರಧಾನ ವಿಷಯಕ್ಕೆ ಸದಾ ಹಿಂದಿರುಗುತ್ತಾರೆ.”

ಶಿಷ್ಯರಲ್ಲೊಬ್ಬರಾದ ಡಾ ॥ ವೈಟ್ ಹಾಸ್ಯ ಸ್ವಭಾವದ ವ್ಯಕ್ತಿ; ಯಾವಾಗಲೂ ಚಟುವಟಿಕೆ ಯಿಂದ ಕೂಡಿ ನಗುನಗುತ್ತಿದ್ದರು. ಸ್ವಾಮೀಜಿಯ ಬೋಧನೆಗಳನ್ನು ಆಲಿಸುತ್ತ ಅವರು ತನ್ಮಯ ರಾಗಿಬಿಡುತ್ತಿದ್ದರು. ಆದರೆ ಎಪ್ಪತ್ತು ವರ್ಷ ದಾಟಿದ ಈ ಮುದುಕರಿಗೆ ಹಿಂದೂತತ್ತ್ವಗಳು– ಅದರಲ್ಲೂ ವೇದಾಂತ–ಕಡಿಯಲಾಗದ ಕಡಲೆ. ಪ್ರತಿದಿನವೂ ಇಂತಹ ಒಂದು ತರಗತಿ ಮುಗಿದ ಮೇಲೆ ಡಾ ॥ ವೈಟ್, ಅದರ ಆಘಾತದಿಂದ ಚೇತರಿಸಿಕೊಳ್ಳುತ್ತ, ತಮ್ಮ ಬೋಳು ಮಂಡೆಯನ್ನು ಸವರಿಸಿಕೊಳ್ಳುತ್ತ ಕೇಳುತ್ತಿದ್ದರು, “ಸರಿ, ಸ್ವಾಮೀಜಿ; ಹಾಗಾದರೆ ಇದೆಲ್ಲದರ ಸಾರಾಂಶ ವೇನೆಂದರೆ ‘ನಾನು ಬ್ರಹ್ಮವಾಗಿದ್ದೇನೆ, ನಾನು ಪರತತ್ತ್ವವಾಗಿದ್ದೇನೆ’ ಎಂದು. ಅಲ್ಲವೆ?” ಆಗ ಸ್ವಾಮೀಜಿ ತಮ್ಮ ಮಧುರವಾದ ಮಂದಹಾಸವನ್ನು ಬೀರುತ್ತ ಮೃದುವಾಗಿ ಹೇಳುತ್ತಿದ್ದರು. “ಹೌದು ಡಾಕೀ (ಡಾಕ್ಟರ್​), ನೀವು ನಿಜವಾದ ಅರ್ಥದಲ್ಲಿ ಬ್ರಹ್ಮವೇ ಆಗಿದ್ದೀರಿ, ಪರತತ್ತ್ವವೇ ಆಗಿದ್ದೀರಿ.” ಬಳಿಕ ಆತ ಊಟಕ್ಕೆ ಬರುವಾಗ ಸ್ವಾಮೀಜಿ ಅತ್ಯಂತ ಗಂಭೀರ ಮುಖಮುದ್ರೆ ಯಿಂದ, ಆದರೆ ತಮ್ಮ ಕಣ್ಣುಗಳಲ್ಲಿ ಚೇಷ್ಟೆಯನ್ನು ಮಿನುಗಿಸುತ್ತ ಹೇಳುತ್ತಿದ್ದರು, “ಬ್ರಹ್ಮ ಬರುತ್ತಿದ್ದಾರೆ!” “ಇಗೋ, ಇಲ್ಲಿ ದಯಮಾಡಿಸುತ್ತಿದ್ದಾರೆ ಪರತತ್ತ್ವ!” ಇಂತಹ ಸಂದರ್ಭ ಗಳಲ್ಲಿ ಅಲ್ಲಿದ್ದವರಿಗೆಲ್ಲ ತಡೆಯಲಾರದ ನಗು.

ಕೆಲವೊಮ್ಮೆ ಸ್ವಾಮೀಜಿ ಅಂದು ತಾವು ಅಡಿಗೆ ಮಾಡುವುದಾಗಿ ಘೋಷಿಸುತ್ತಿದ್ದರು. ಆಗ ಅಲ್ಲಿ ಒಂದು ಕೋಲಾಹಲವೇ ಏರ್ಪಡುತ್ತಿತ್ತು! ಸ್ವಾಮೀಜಿಯ ಕೈಯಡಿಗೆ ತುಂಬ ವಿಶಿಷ್ಟ ವಾದದ್ದು ಎಂಬುದು ಒಂದು ಕಾರಣ. ಆದರೆ ಲ್ಯಾಂಡ್ಸ್​ಬರ್ಗ್ ಮಾತ್ರ ಅದನ್ನು ಕೇಳಿದ ಕೂಡಲೇ “ಭಗವಂತ! ನೀನೇ ಕಾಪಾಡಬೇಕು” ಎಂದು ಉದ್ಗರಿಸುತ್ತಿದ್ದ. ಕಾರಣವಿಷ್ಟೆ–ಸ್ವಾಮೀಜಿ ಅಡಿಗೆ ಮಾಡಿದರೆ ಮನೆಯಲ್ಲಿದ್ದ ಪ್ರತಿಯೊಂದು ಪಾತ್ರೆಯೂ ಬೇಕಾಗುತ್ತಿತ್ತು. ಊಟವಾದ ಮೇಲೆ ಅವುಗಳನ್ನೆಲ್ಲ ತೊಳೆಯಬೇಕಾದವರು ಆ ಶಿಷ್ಯರೇ ತಾನೆ!ಅವರು ಮಾಡುತ್ತಿದ್ದ ಅಡಿಗೆ ಅತ್ಯಂತ ರುಚಿಕರ, ಆದರೆ ತುಂಬ ಖಾರ! ಆ ಪಾಶ್ಚಾತ್ಯ ಶಿಷ್ಯರು ಕಣ್ಣಲ್ಲಿ ನೀರು ಸುರಿಯುತ್ತಿದ್ದರೂ ಅದನ್ನೇ ಚಪ್ಪರಿಸಿಕೊಂಡು ತಿನ್ನುತ್ತಿದ್ದರು. ಶ್ರೀಮತಿ ಫಂಕೆ ತನ್ನ ಸ್ನೇಹಿತೆಗೆ ಬರೆದ ಪತ್ರದಲ್ಲಿ ಆ ಬಗ್ಗೆ ಹೀಗೆ ಹೇಳುತ್ತಾಳೆ–“ಅವರು ತಯಾರಿಸುವ ಅಡಿಗೆ ರುಚಿಕರವಾಗೇನೋ ಇರುತ್ತದೆ; ಆದರದು ಅನೇಕ ಮಸಾಲೆಗಳಿಂದ ಕೂಡಿದ್ದು, ವಿಪರೀತ ಖಾರ ಅಷ್ಟೆ! ಆದರೆ ಅದನ್ನು ತಿಂದು ನನ್ನ ಉಸಿರೇ ಕಟ್ಟಿಹೋದರೂ ಸರಿಯೇ, ನಾನದನ್ನು ತಿಂದೇ ತಿನ್ನುತ್ತೇನೆ ಎಂದು ನಿರ್ಧರಿಸಿಬಿಟ್ಟೆ. ಮತ್ತು ಅದು ಹಾಗೆಯೇ ಆಯಿತು ಕೂಡ ಅನ್ನು! ಒಬ್ಬ ವಿವೇಕಾನಂದರು ನನಗಾಗಿ ಅಡಿಗೆ ಮಾಡುವುದಾದರೆ ಕನಿಷ್ಠ ಪಕ್ಷ ನಾನು ಮಾಡಬಹುದಾದ ಕೆಲಸ–ಅದನ್ನು ತಿನ್ನುವುದು!”

ಶ್ರೀಮತಿ ಫಂಕೆ ಸ್ವಲ್ಪ ಎದೆಗಾರಿಕೆಯ ಸ್ವಭಾವದವಳು. ಎಷ್ಟೋ ಸಲ ಆಕೆ ಯಾವ ಸಂಕೋಚವೂ ಇಲ್ಲದೆ ಪ್ರಶ್ನೆಗಳನ್ನು ಕೇಳುತ್ತಿದ್ದಳು. ಆಕೆಯ ಧೈರ್ಯವನ್ನು ಕಂಡು ಇತರರಿಗೆ ನಾಲಿಗೆ ಕಚ್ಚಿಕೊಳ್ಳುವಂತಾಗುತ್ತಿತ್ತು. ಹೀಗೆ ಪ್ರಶ್ನೆ ಕೇಳುವಲ್ಲಿ ಆಕೆಗೆ ಮತ್ತೊಂದು ಉದ್ದೇಶವೂ ಇತ್ತು–ಇಂತಹ ಪರಿಸ್ಥಿತಿಯಲ್ಲಿ ಸ್ವಾಮೀಜಿ ಹೇಗೆ ವರ್ತಿಸುತ್ತಾರೆ ಎಂಬುದನ್ನು ಗಮನಿಸು ವುದು! ಒಮ್ಮೆ ಸ್ವಾಮೀಜಿ ತುಂಬ ಸ್ಫೂರ್ತಿಯಿಂದ ಪವಿತ್ರ ಸ್ತ್ರೀತ್ವದ ಆದರ್ಶದ ಕುರಿತಾಗಿ ಮಾತ ನಾಡುತ್ತ ಸೀತೆಯ ಕತೆಯನ್ನು ಹೇಳಿದರು. ಅದನ್ನವರು ಕಣ್ಣಿಗೆ ಕಟ್ಟುವಂತೆ ಹೇಳುತ್ತಿದ್ದಾಗ ಶ್ರೀಮತಿ ಫಂಕೆಯ ಮನಸ್ಸಿನಲ್ಲಿ ಒಂದು ಆಲೋಚನೆ ಸುಳಿಯಿತು–‘ನಮ್ಮ ಪಾಶ್ಚಾತ್ಯ ಸಮಾಜದ ಸೌಂದರ್ಯರಾಣಿಯರು, ಅದರಲ್ಲೂ ಮರುಳುಗೊಳಿಸುವ ಕಲೆಯಲ್ಲಿ ನಿಷ್ಣಾತರಾದ ವರು, ಸ್ವಾಮೀಜಿಯ ಕಣ್ಣಿಗೆ ಹೇಗೆ ಕಂಡಾರು...?’ ಅವಳ ಮನಸ್ಸಿನಲ್ಲಿ ಆ ಆಲೋಚನೆ ಸುಳಿಯುವಷ್ಟರಲ್ಲೇ ಅದು ಅವಳ ನಾಲಿಗೆಯಿಂದ ಜಾರಿತ್ತು! ತಕ್ಷಣ ಅವಳಿಗೆ ದಿಗ್ಭ್ರಾಂತಿ ಯಾಯಿತು. ಆದರೆ ಸ್ವಾಮೀಜಿ ಸ್ವಲ್ಪವೂ ಕಸಿವಿಸಿಗೊಳ್ಳಲಿಲ್ಲ. ತಮ್ಮ ವಿಶಾಲವಾದ ನೇತ್ರಗಳಿಂದ ಅವಳತ್ತ ನೋಡುತ್ತ ಶಾಂತ-ಗಂಭೀರ ಭಾವದಿಂದ ಉತ್ತರಿಸಿದರು–“ಈ ಪ್ರಪಂಚದ ಅತ್ಯಂತ ಸುಂದರಿಯಾದವಳೂ ನನ್ನತ್ತ ಅಶ್ಲೀಲ ದೃಷ್ಟಿಯಿಂದ ನೋಡಿದರೆ, ತಕ್ಷಣವೇ ಅವಳೊಂದು ಕೊಳಕು ಕಪ್ಪೆಯಾಗಿಬಿಡುತ್ತಾಳೆ. (ಎಂದರೆ, ತಮ್ಮ ಕಣ್ಣಿಗೆ ಆಕೆ ಒಂದು ಕಪ್ಪೆಯಂತೆ ತೋರು ತ್ತಾಳೆ ಎಂದರ್ಥ.) ಕಪ್ಪೆಯನ್ನು ಯಾರು ತಾನೆ ಮೆಚ್ಚಿಕೊಂಡಾರು?”

ಕ್ರಿಸ್ಟೀನ ಹಾಗೂ ಶ್ರೀಮತಿ ಫಂಕೆ ಸಹಸ್ರದ್ವೀಪೋದ್ಯಾನಕ್ಕೆ ಬಂದದ್ದು ತಡವಾಗಿ; ಮತ್ತು ಅದಕ್ಕೆ ಮೊದಲು ಅವರು ಸ್ವಾಮೀಜಿಯ ನಿಕಟ ಸಂಪರ್ಕಕ್ಕೆ ಬಂದಿರಲಿಲ್ಲ ಎಂಬುದನ್ನು ಈಗ ಆಗಲೇ ನೋಡಿದ್ದೇವೆ. ಅವರು ಇಲ್ಲಿಗೆ ಬಂದ ಎರಡು ದಿನಗಳಲ್ಲೇ ಸ್ವಾಮೀಜಿ ಎಲ್ಲರಿಗೂ ಮಂತ್ರದೀಕ್ಷೆಯನ್ನು ನೀಡಲು ಯೋಜಿಸಿದ್ದರು. ಇತರರೊಂದಿಗೆ ಇವರಿಬ್ಬರಿಗೂ ಮಂತ್ರದೀಕ್ಷೆ ನೀಡಲು ಅವರು ಸಿದ್ಧರಿದ್ದರು. ಆದರೆ ಇವರಿಬ್ಬರೂ ಅದಕ್ಕೆ ಮಾನಸಿಕವಾಗಿ ಸಿದ್ಧರಿದ್ದಾರೆಯೆ ಎಂಬುದನ್ನು ಅವರು ತಿಳಿದುಕೊಳ್ಳಬೇಕಾಗಿತ್ತು. ನಿಜಕ್ಕೂ, ಮಂತ್ರದೀಕ್ಷೆಯನ್ನು ಕೊಡಬೇಕಾದರೆ ಶಿಷ್ಯನ ಸ್ವಭಾವ, ಸಾಮರ್ಥ್ಯ, ಹಿನ್ನೆಲೆಗಳೆಲ್ಲ ಗುರುವಾದವನಿಗೆ ತಿಳಿದಿರಬೇಕಾಗುತ್ತದೆ. ಆದ್ದರಿಂದ ಸ್ವಾಮೀಜಿ, ಕ್ರಿಸ್ಟೀನ ಹಾಗೂ ಶ್ರೀಮತಿ ಫಂಕೆ–ಇವರನ್ನು ಪರೀಕ್ಷೆ ಮಾಡಿದರು. ಅವರು ಪರೀಕ್ಷೆ ಮಾಡಿದ ಕ್ರಮ ತುಂಬ ಸ್ವಾರಸ್ಯವಾಗಿದೆ.

ಇವರಿಗೆ ಸ್ವಾಮೀಜಿ ಹೇಳಿದರು, “ಮಂತ್ರದೀಕ್ಷೆಗೆ ಸಿದ್ಧರಿದ್ದೀರಿ ಎಂಬುದು ನನಗೆ ಖಚಿತ ವಾಗುವಷ್ಟರ ಮಟ್ಟಿಗೆ ನಿಮ್ಮನ್ನು ನಾನು ಅರಿತಿಲ್ಲ...” ಬಳಿಕ ಸ್ವಲ್ಪ ಸಂಕೋಚದಿಂದ ಹೇಳಿದರು, “ನನ್ನಲ್ಲೊಂದು ಶಕ್ತಿಯಿದೆ–ಆದರೆ ಅದನ್ನು ನಾನು ಉಪಯೋಗಿಸುವುದು ತೀರ ಅಪರೂಪ– ಅದು ಇನ್ನೊಬ್ಬರ ಮನಸ್ಸನ್ನು ಓದುವ ಶಕ್ತಿ. ಈಗ ನಾನು ನಿಮಗೆ ಮಂತ್ರದೀಕ್ಷೆ ನೀಡಬೇಕಾ ಗಿರುವುದರಿಂದ, ನೀವು ಒಪ್ಪಿಗೆ ಕೊಟ್ಟರೆ, ನಿಮ್ಮ ಮನಸ್ಸನ್ನು ಓದಲು ಇಚ್ಛಿಸುತ್ತೇನೆ.” ತಾವಾಗಿಯೇ ಇತರರ ಮನಸ್ಸನ್ನು ಓದುವ ಸ್ವಾತಂತ್ರ್ಯ ವಹಿಸದೆ ಸ್ವಾಮೀಜಿ ಅವರ ಅನುಮತಿ ಕೇಳುವ ವಿನಯವನ್ನು ಗಮನಿಸಬೇಕು. ಸ್ವಾಮೀಜಿ ಹಾಗೆ ಕೇಳಿದಾಗ ಇಬ್ಬರೂ ಸಂತೋಷದಿಂದ ಒಪ್ಪಿಕೊಂಡರು. ಸ್ವಾಮೀಜಿಗೆ ಇಬ್ಬರ ಬಗ್ಗೆಯೂ ಸಮಾಧಾನವಾಗಿರಲೇಬೇಕು. ಅಂತೆಯೇ ಅವರು ಇತರರೊಂದಿಗೆ ಇವರಿಬ್ಬರಿಗೂ ಮರುದಿನ ಮಂತ್ರದೀಕ್ಷೆಯನ್ನು ಅನುಗ್ರಹಿಸಿದರು.

ಅನಂತರ ಇವರು, “ಸ್ವಾಮೀಜಿ, ನಮ್ಮ ಮನಸ್ಸನ್ನು ಓದಿದಾಗ ನೀವು ಏನು ಕಂಡಿರಿ?” ಎಂದು ಕೇಳಿದರು. ಆಗ ಸ್ವಾಮೀಜಿ ತಾವು ಕಂಡದ್ದನ್ನು ಸ್ವಲ್ಪ ಮಾತ್ರ ಹೇಳಿದರು. ಇಬ್ಬರೂ ನಿಷ್ಠಾವಂತರಾಗಿ ಉಳಿದುಕೊಳ್ಳುತ್ತಾರೆ ಹಾಗೂ ಆಧ್ಯಾತ್ಮಿಕ ಜೀವನದಲ್ಲಿ ಪ್ರಗತಿ ಹೊಂದುತ್ತಾರೆ ಎಂಬುದು ತಮಗೆ ಗೊತ್ತಾಯಿತೆಂದು ಅವರು ತಿಳಿಸಿದರು. ತಮಗೆ ಗೋಚರವಾದ ಅನೇಕ ಚಿತ್ರ ಗಳನ್ನು ಅವರು ತಿಳಿಸಿದರಾದರೂ ಅವೆಲ್ಲದರ ಅರ್ಥವನ್ನು ವಿವರಿಸಲಿಲ್ಲ. ಇವರಲ್ಲಿ ಒಬ್ಬಳ (ಕ್ರಿಸ್ಟೀನಳ) ವಿಚಾರವಾಗಿ ಹೇಳುತ್ತ, ಆಕೆ ಪೌರ್ವಾತ್ಯ ರಾಷ್ಟ್ರಗಳಲ್ಲಿ ವಿಸ್ತೃತವಾಗಿ ಪ್ರಯಾಣ ಮಾಡಲಿದ್ದಾಳೆ ಎಂದರು. ಅಲ್ಲದೆ, ಆಕೆ ಎಂತಹ ಮನೆಗಳಲ್ಲಿ ವಾಸಿಸಲಿದ್ದಾಳೆ, ಆಕೆಯ ಸುತ್ತ ಮುತ್ತ ವಾಸಿಸಲಿರುವವರು ಎಂಥವರು, ಆಕೆಯ ಜೀವನದ ಮೇಲೆ ಯಾವಯಾವ ಅಂಶಗಳು ಪ್ರಭಾವ ಬೀರಲಿವೆ ಎಂಬುದನ್ನೆಲ್ಲ ಅವರು ವಿವರಿಸಿದರು. ಇದನ್ನೆಲ್ಲ ಕೇಳಿ ಆಶ್ಚರ್ಯಗೊಂಡ ಕ್ರಿಸ್ಟೀನ ಹಾಗೂ ಶ್ರೀಮತಿ ಫಂಕೆ, “ಇದೆಲ್ಲ ನಿಮಗೆ ಗೊತ್ತಾದದ್ದು ಹೇಗೆ?” ಎಂದು ಪ್ರಶ್ನಿಸಿದರು. ಅದೊಂದು ಸಾಮಾನ್ಯ ವಿಷಯವೆಂಬಂತೆ ಸ್ವಾಮೀಜಿ ಹೇಳಿದರು, “ಆ ಶಕ್ತಿಯನ್ನು ಯಾರು ಬೇಕಾ ದರೂ ಪಡೆದುಕೊಳ್ಳಬಹುದು. ಹೇಗೆಂದರೆ, ಮೊದಲು ಎಲ್ಲೆಲ್ಲೂ ಹರಡಿರುವ ಅನಂತ ನೀಲಾ ಕಾಶವನ್ನು ಮನಸ್ಸಿನಲ್ಲಿ ಭಾವಿಸಿಕೊಳ್ಳಬೇಕು. ಈ ಆಕಾಶದ ಕುರಿತಾಗಿ ತೀವ್ರವಾಗಿ ಧ್ಯಾನ ಮಾಡಿ ದಾಗ ಮನಸ್ಸಿನಲ್ಲಿ ದೃಶ್ಯಗಳು ಕಾಣಿಸಿಕೊಳ್ಳತೊಡಗುತ್ತವೆ. ಆದರೆ ಈ ದೃಶ್ಯಗಳು ಸಾಂಕೇತಿಕ ವಾಗಿದ್ದು, ಇವುಗಳನ್ನು ಸರಿಯಾದ ರೀತಿಯಲ್ಲಿ ವಿಶ್ಲೇಷಿಸಿ ಅರ್ಥಮಾಡಿಕೊಳ್ಳಬೇಕಾಗುತ್ತದೆ.” ಹೀಗೆ ಆ ವಿಷಯವನ್ನು ಅವರು ಬಹಳ ಸರಳವಾಗಿ ವಿವರಿಸಿದರು! ಕ್ರಿಸ್ಟೀನಳ ವಿಷಯದಲ್ಲಿ ಅವರ ಈ ಭವಿಷ್ಯವಾಣಿ ಸಂಪೂರ್ಣ ಸತ್ಯವಾಯಿತು.

ಸಹಸ್ರದ್ವೀಪೋದ್ಯಾನದಲ್ಲಿ ತಾವು ಕಳೆದ ಅದ್ಭುತ ದಿನಗಳ ಬಗ್ಗೆ ಕ್ರಿಸ್ಟೀನ ಮುಂದೊಮ್ಮೆ ಬರೆಯುತ್ತಾಳೆ:

“ನಾವಲ್ಲಿಗೆ ಹೋದಮೇಲೆ ಉರುಳಿದ ಅದ್ಭುತ ವಾರಗಳನ್ನು ಬಣ್ಣಿಸುವುದು ಬಹಳ ಕಷ್ಟ. ನಾವು ಅಂದು ಯಾವ ಉನ್ನತ ಸ್ಥಿತಿಯಲ್ಲಿ ಜೀವಿಸಿದ್ದೆವೋ ಆ ಹಂತಕ್ಕೆ ಏರಿದ ಹೊರತು, ಆ ಅನುಭವವನ್ನು ಮತ್ತೆ ಪಡೆದುಕೊಳ್ಳಲಾರೆವು. ನಾವು ಆನಂದಭರಿತರಾಗಿದ್ದೆವು. ನಾವು ಸ್ವಾಮೀಜಿಯ ಜ್ಯೋತಿರ್ವಲಯದಲ್ಲಿದ್ದೇವೆಯೆಂಬುದು ಆಗ ನಮಗೆ ತಿಳಿಯಲಿಲ್ಲ. ತಮ್ಮ ಸ್ಫೂರ್ತಿಯ ರೆಕ್ಕೆಗಳ ಮೇಲೆ ಕುಳ್ಳಿರಿಸಿಕೊಂಡು ಅವರು ನಮ್ಮನ್ನೆಲ್ಲ ತಮ್ಮ ಸಹಜಧಾಮಕ್ಕೆ ಕರೆದೊಯ್ದರು. ಮುಂದೆ ಆ ಬಗ್ಗೆ ಮಾತನಾಡುತ್ತ, ‘ಆಗ ನಾನು ನನ್ನ ಶಕ್ತ್ಯುತ್ಸಾಹದ ತುತ್ತತುದಿಯಲ್ಲಿದ್ದೆ’ ಎಂದು ಸ್ವಾಮೀಜಿಯೇ ಹೇಳಿದರು...

“ಅಲ್ಲಿನ ಪರಿಸರ-ಪರಿಸ್ಥಿತಿಗಳು ನಮ್ಮ ಉದ್ದೇಶಕ್ಕೆ ಹೇಳಿ ಮಾಡಿಸಿದಂತಿದ್ದುವು. ಅಂತಹ ದೊಂದು ಜಾಗ ಅಮೆರಿಕದಲ್ಲಿದೆ ಎಂದರೆ ನಂಬಲು ಸಾಧ್ಯವಿರಲಿಲ್ಲ. ಎಂಥ ಮಹದ್ಭಾವನೆಗಳು ಅಲ್ಲಿ ಧ್ವನಿಸಲ್ಪಟ್ಟುವು! ಎಂಥ ವಾತಾವರಣ ನಿರ್ಮಾಣವಾಯಿತು! ಎಂಥ ಶಕ್ತಿ ಉತ್ಪನ್ನ ವಾಯಿತು! ಹೊಸ ಆಲೋಚನೆಗಳು ಹುಟ್ಟಿ ಅರಳುವುದನ್ನು ನಾವು ಕಂಡೆವು. ಕಾಲಾಂತರದಲ್ಲಿ ಕಾರ್ಯರೂಪಕ್ಕೆ ಬಂದ ಅನೇಕ ಹೊಸ ಯೋಜನೆಗಳು ಅಲ್ಲಿ ಮೈದಾಳುವುದನ್ನು ನಾವು ಕಂಡೆವು. ಅದೊಂದು ಪವಿತ್ರ ಅನುಭವ. ಮಿಸ್ ವಾಲ್ಡೊ ಉದ್ಗರಿಸಿದಳು–‘ಇದನ್ನೆಲ್ಲ ಪಡೆದುಕೊಳ್ಳು ವಂತಹ ಅರ್ಹತೆ ನಮ್ಮಲ್ಲಿ ಏನಿದೆ?’ ಅದೇ ನಮ್ಮೆಲ್ಲರ ಅನಿಸಿಕೆಯೂ ಆಗಿತ್ತು.”

ಸ್ವಾಮೀಜಿಯ ಬೋಧನೆಗಳು ಮಾತ್ರವಲ್ಲದೆ, ಅಲ್ಲಿನ ಸಹಜೀವನವೇ ಶಿಷ್ಯರೆಲ್ಲರ ಪಾಲಿಗೆ ಒಂದು ದೊಡ್ಡ ಪಾಠವಾಗಿತ್ತು. ಪ್ರತಿಯೊಂದು ಕೆಲಸವನ್ನೂ ವಿದ್ಯಾರ್ಥಿಗಳೇ ಮಾಡಿಕೊಂಡು ಹೋಗಬೇಕೆಂದು ಒಪ್ಪಿಗೆಯೇನೋ ಆಗಿತ್ತು. ಆದರೆ ಅವರಾರಿಗೂ ಇಂತಹ ಕೆಲಸಗಳನ್ನು ಮಾಡಿದ ಅನುಭವವಿರಲಿಲ್ಲ. ಪರಿಣಾಮವಾಗಿ ಅಲ್ಲೊಂದು ಭಾರೀ ಗೊಂದಲವೇರ್ಪಟ್ಟಿತು. ಕಿಷ್ಕಿಂಧಾಕಾಂಡದ ಪುನರಾವರ್ತನೆಯಾಗುವಂತೆ ಕಂಡುಬಂದಿತು... ಕೆಲವರಿಗೆ ಪಾತ್ರೆಗಳನ್ನು ತೊಳೆಯುವ ಕೆಲಸ ಮಾತ್ರ ಬರುತ್ತಿತ್ತು. ಬ್ರೆಡ್ಡನ್ನು ಕತ್ತರಿಸಬೇಕಾಗಿ ಬಂದವಳು, ಆ ಕೆಲಸದಲ್ಲಿ ತೊಡಗಿದಾಗಲೆಲ್ಲ ಕೊಸಗುಟ್ಟಿ, ಗೊಣಗುಟ್ಟಿ, ಗೋಳಾಡುತ್ತಿದ್ದಳು. ಇಂತಹ ಸಣ್ಣ ವಿಷಯ ಗಳಲ್ಲಿ ಮನುಷ್ಯನ ಶೀಲ ಹೇಗೆ ಪರೀಕ್ಷೆಗೊಳಾಗುತ್ತದೆ ಎಂಬುದು ತುಂಬ ಸ್ವಾರಸ್ಯಕರ. ಬಹುಶಃ ಈ ಶಿಷ್ಯರಲ್ಲಿ ಅನೇಕರ ದೌರ್ಬಲ್ಯಗಳು ಅವರ ಜೀವಮಾನವಿಡೀ ಬಯಲಾಗುತ್ತಿರಲಿಲ್ಲವೇನೋ. ಅವೆಲ್ಲ ಇಲ್ಲಿನ ಸಂಘಜೀವನದಲ್ಲಿ ಒಂದೇ ದಿನಕ್ಕೆ ಬಯಲಾದುವು. ಆದರೆ ಇದು ಸ್ವಾಮೀಜಿಯ ಮೇಲೆ ಉಂಟು ಮಾಡಿದ ಪರಿಣಾಮ ತದ್ವಿರುದ್ಧವಾದುದು. ಆ ಹನ್ನೆರಡು ಜನರಲ್ಲಿ ಒಬ್ಬರನ್ನು ಬಿಟ್ಟರೆ ಉಳಿದವರೆಲ್ಲರಿಗಿಂತ ಸ್ವಾಮೀಜಿ ಕಿರಿಯರು. ಆದರೆ ಅವರೆಲ್ಲರಿಗೂ ಸ್ವಾಮೀಜಿ ತಂದೆಯಂತಿದ್ದರು; ಅಥವಾ ಸಹನೆ ಹಾಗೂ ಸೌಮ್ಯತೆಯಲ್ಲಿ ತಾಯಿಯಂತಿದ್ದರು ಎನ್ನಬಹುದು. ‘ಯಾರಪ್ಪ ಅಡಿಗೆ ಮಾಡುವವರು’ ಎಂಬ ಗೊಣಗಾಟ ಮಿತಿಮೀರುವಂತಾದಾಗ ಸ್ವಾಮೀಜಿ ಅತ್ಯಂತ ಮಧುರವಾದ ದನಿಯಲ್ಲಿ, “ಇಂದು ನಾನು ಅಡಿಗೆ ಮಾಡುತ್ತೇನೆ” ಎನ್ನುತ್ತಿದ್ದರು.

ಬಳಿಕ ತರಗತಿಯ ದೃಶ್ಯ ಹೇಗಿರುತ್ತಿತ್ತೆಂಬುದನ್ನು ಸೋದರಿ ಕ್ರಿಸ್ಟೀನ ವಿವರಿಸುತ್ತಾಳೆ– “ಆದರೆ ಮನೆಗೆಲಸವೆಲ್ಲ ಮುಗಿದು ಶಿಷ್ಯರು ತರಗತಿಗೆ ಸಿದ್ಧರಾಗಿ ಕುಳಿತರೆಂದರೆ, ವಾತಾವರಣ ಸಂಪೂರ್ಣ ಬದಲಾಗುತ್ತಿತ್ತು. ಅಲ್ಲಿನ್ನು ಯಾವುದೇ ಬಗೆಯ ಅಡಚಣೆಯಿರುತ್ತಿರಲಿಲ್ಲ. ನಾವು ನಮ್ಮ ಶರೀರಗಳನ್ನೂ ಶರೀರಪ್ರಜ್ಞೆಯನ್ನೂ ಹೊರಗಡೆಯೇ ಬಿಟ್ಟುಬಿಟ್ಟಿರುವಂತೆ ಅನ್ನಿಸುತ್ತಿತ್ತು. ನಾವು ಅರ್ಧವೃತ್ತಾಕಾರದಲ್ಲಿ ಕುಳಿತು ಕಾಯುತ್ತಿದ್ದೆವು. ಅಮೃತತ್ವದೆಡೆಗೆ ಇಂದಾವ ಹೊಸ ಬಾಗಿಲು ತೆರೆಯಬಹುದೊ! ನಮ್ಮ ಕಣ್ಣಿಗೆ ಇನ್ನಾವ ದಿವ್ಯ ದೃಶ್ಯ ಕಾಣಸಿಗಬಹುದೊ! ಅಲ್ಲಿ ಸದಾ ಹೊಸ ಸಾಹಸದ ರೋಮಾಂಚವಿರುತ್ತಿತ್ತು. ನಾವಿನ್ನೂ ಕಂಡರಿಯದ ದಿವ್ಯಧಾಮದ, ದುಃಖವಿಲ್ಲದ ನಾಡಿನ ಕಲ್ಪನೆಯು, ನಮ್ಮ ಮುಂದೆ ಭರವಸೆ-ಸೌಂದರ್ಯಗಳ ಹೊಸ ಪಥವನ್ನೇ ತೆರೆಯುತ್ತಿತ್ತು. ಆದರೂ ನಮಗೆ ದೊರಕುತ್ತಿದ್ದ ಅನುಭವವು, ನಮ್ಮೆಲ್ಲ ನಿರೀಕ್ಷೆಯನ್ನೂ ಮೀರಿದುದಾಗಿರುತ್ತಿತ್ತು. ತದನಂತರದ ನಮ್ಮ ಸಾಧನೆ ಎಷ್ಟೇ ಗಣ್ಯವಾಗಿರಲಿ ಅಥವಾ ಅಗಣ್ಯ ವಾಗಿರಲಿ, ಒಂದು ವಿಷಯವನ್ನು ಮಾತ್ರ ನಾವು ಮರೆಯುವಂತಿಲ್ಲ–ಯಾವ ದಿವ್ಯಧಾಮದ ಭರವಸೆಯನ್ನು ಸ್ವಾಮೀಜಿ ನಮ್ಮ ಮುಂದಿಟ್ಟರೋ ಆ ಧಾಮವನ್ನು ನಾವು ಕಣ್ಣಾರೆ ಕಂಡೆವು. ಮತ್ತು ಈ ಜಗತ್ತಿನ ಜಂಜಡಗಳೆಲ್ಲ ನಮ್ಮ ಪಾಲಿಗೆ ಸತ್ಯವೆಂದೇ ತೋರಲಿಲ್ಲ.

“ನಮ್ಮ ಮೇಲೆ ತಾವು ಬೀರಿದ್ದ ಪ್ರಭಾವ ಎಷ್ಟು ಆಳವಾದುದೆಂಬುದನ್ನು ಗಮನಿಸಿ ಸ್ವಾಮೀಜಿ ಮುಗುಳ್ನಗುತ್ತ ಹೇಳುತ್ತಿದ್ದರು, ‘ನಿಮ್ಮನ್ನು ನಾಗರಹಾವು ಹಿಡಿದುಕೊಂಡಿದೆ.\footnote{* ನೋಡಿ: ಅನುಬಂಧ ೫.} ನೀವದರಿಂದ ತಪ್ಪಿಸಿಕೊಳ್ಳಲಾರಿರಿ.’ ಇಲ್ಲವೆ ಹೀಗೆ ಹೇಳುತ್ತಿದ್ದರು: ‘ನಾನು ನಿಮ್ಮನ್ನು ನನ್ನ ಬಲೆಯಲ್ಲಿ ಹಿಡಿ ದಿಟ್ಟಿದ್ದೇನೆ. ಇನ್ನು ನೀವೆಂದಿಗೂ ಹೊರಹೋಗುವಂತಿಲ್ಲ’ ಎಂದು.

ಆ ಸಹಸ್ರದ್ವೀಪೋದ್ಯಾನದಲ್ಲಿ ಶಿಷ್ಯರೆಲ್ಲ ಮೊದಲ ಬಾರಿಗೆ ಒಂದು ಘನಸತ್ಯವನ್ನು ಅರಿತು ಕೊಂಡರು: ಧರ್ಮವೆನ್ನುವುದು ನೈತಿಕತೆಯಿಂದ ಪ್ರಾರಂಭವಾಗುತ್ತದೆ ಎಂದು. ಏಕೆಂದರೆ ಅಹಿಂಸೆ, ಸತ್ಯ, ಅಸ್ತೇಯ (ಕದಿಯದಿರುವುದು), ಬ್ರಹ್ಮಚರ್ಯ, ಅಪರಿಗ್ರಹ, ಶೌಚ, ತಪಸ್ಸು– ಇವುಗಳಿಲ್ಲದೆ, ಯಾವ ಆಧ್ಯಾತ್ಮಿಕತೆಯೂ ಇರದು. ಬ್ರಹ್ಮಚರ್ಯ-ಪರಿಶುದ್ಧತೆ! ಇದು ಸ್ವಾಮೀಜಿಯ ಮನಸ್ಸನ್ನು ತೀವ್ರವಾಗಿ ಕೊರೆಯುತ್ತಿದ್ದ ವಿಷಯ. ಒಮ್ಮೊಮ್ಮೆ ತರಗತಿಯ ಕೋಣೆಯಲ್ಲಿ ಆ ಕಡೆಯಿಂದ ಈ ಕಡೆಗೆ ತುಂಬ ಉದ್ವೇಗದಿಂದ ಓಡಾಡುತ್ತ, ಬಳಿಕ ಇದ್ದಕ್ಕಿ ದ್ದಂತೆ ಯಾರಾದರೊಬ್ಬರ ಮುಂದೆ ನಿಂತು, ಕೋಣೆಯಲ್ಲಿ ಇನ್ನಾರೂ ಇಲ್ಲದೆ ಅವನೊಬ್ಬನೇ ಇರುವನೇನೋ ಎಂಬಂತೆ ಅವನಿಗೆ ಹೇಳುತ್ತಿದ್ದರು, “ನೀನು ನೋಡಿಲ್ಲವೆ? ಸಂನ್ಯಾಸಿಗಳ ಸಂಘಗಳಲ್ಲಿ ಬ್ರಹ್ಮಚರ್ಯವನ್ನು ವಿಧಿಸಿರುವುದಕ್ಕೆ ಕಾರಣವಿದೆ. ಎಲ್ಲಿ ಕಟ್ಟುನಿಟ್ಟಾದ ಬ್ರಹ್ಮ ಚರ್ಯ ಪಾಲನೆಯಿರುತ್ತದೆಯೋ ಅಲ್ಲಿ ಮಾತ್ರ ಶ್ರೇಷ್ಠ ಆಧ್ಯಾತ್ಮಿಕ ವ್ಯಕ್ತಿಗಳ ನಿರ್ಮಾಣ ವಾಗುತ್ತದೆ. ಇದಕ್ಕೆ ಕಾರಣವಿದೆ ಎನ್ನುವುದು ನಿನಗೆ ಗೊತ್ತಿದೆಯೆ? ರೋಮನ್ ಕ್ಯಾಥೋಲಿಕ್ ಚರ್ಚು ದೊಡ್ಡದೊಡ್ಡ ಸಂತರನ್ನು ನಿರ್ಮಾಣ ಮಾಡಿದೆ–ಅಸ್ಸಿಸಿಯ ಸೈಂಟ್ ಫ್ರಾನ್ಸಿಸ್, ಇಗ್ನೇಷಿಯಸ್ ಲಯೊಲಾ, ಸೈಂಟ್ ಥೆರೆಸಾ–ಹೀಗೆ ಇನ್ನೂ ಹಲವಾರು ಸಂತರನ್ನು ನಿರ್ಮಾಣ ಮಾಡಿದೆ. ಆದರೆ ಪ್ರಾಟೆಸ್ಟೆಂಟ್ ಚರ್ಚು ಆಧ್ಯಾತ್ಮಿಕವಾಗಿ ಆ ಸಂತರಿಗೆ ಸರಿದೂಗುವ ಒಬ್ಬನೇ ಒಬ್ಬನನ್ನೂ ನಿರ್ಮಿಸಿಲ್ಲ. ಆಧ್ಯಾತ್ಮಿಕತೆಗೂ ಬ್ರಹ್ಮಚರ್ಯಕ್ಕೂ ನಿಕಟ ಸಂಬಂಧವಿದೆ. ಏಕೆಂದರೆ, ಬ್ರಹ್ಮಚರ್ಯಪಾಲನೆ ಮಾಡುವ ಈ ಸ್ತ್ರೀಪುರುಷರು ತೀವ್ರ ಪ್ರಾರ್ಥನೆ-ಧ್ಯಾನಗಳ ಮೂಲಕ ದೇಹದೊಳಗಿನ ಅತ್ಯಂತ ಪ್ರಬಲ ಶಕ್ತಿಯಾದ ಲೈಂಗಿಕ ಶಕ್ತಿಯನ್ನು ಆಧ್ಯಾತ್ಮಿಕ ಶಕ್ತಿಯಾಗಿ ಮಾರ್ಪಡಿಸಿಕೊಳ್ಳುತ್ತಾರೆ. ಭಾರತದಲ್ಲಿ ಇದನ್ನು ಚೆನ್ನಾಗಿ ಅರ್ಥಮಾಡಿಕೊಳ್ಳಲಾಗಿದೆ. ಯೋಗಿ ಗಳು ಇದನ್ನು ಪ್ರಜ್ಞಾಪೂರ್ವಕವಾಗಿ ಸಾಧಿಸಿ ಸಿದ್ಧಿಸಿಕೊಳ್ಳುತ್ತಾರೆ. ಹೀಗೆ ಪರಿವರ್ತಿತವಾದ ಶಕ್ತಿಗೆ ‘ಓಜಸ್​’ ಎಂದು ಹೆಸರು. ಇದು ಮಿದುಳಿನಲ್ಲಿ ಸಂಗ್ರಹವಾಗುತ್ತದೆ. ಇದು ಅತಿ ಕೆಳಗಿನ ಮೂಲಾ ಧಾರದಿಂದ ಅತಿ ಮೇಲಿನ ಸಹಸ್ರಾರದವರೆಗೂ ಏರುತ್ತದೆ.”

ಸ್ವಾಮೀಜಿಯವರು ಈ ವಿಷಯವನ್ನು ವಿವರಿಸುತ್ತಿದ್ದ ರೀತಿ ತುಂಬ ಹೃದಯಸ್ಪರ್ಶಿಯಾಗಿರು ತ್ತಿತ್ತು. ತಮ್ಮ ಈ ಬೋಧನೆಯನ್ನು ಅತ್ಯಮೂಲ್ಯ ರತ್ನವೆಂಬಂತೆ ಸ್ವೀಕರಿಸಿ ತಮ್ಮ ಶಿಷ್ಯರು ಅದರಂತೆ ನಡೆಯಬೇಕು ಎಂದು ಅವರು ತೀವ್ರವಾಗಿ ಬಯಸಿದರು. ಅದನ್ನು ಸ್ವಾಮೀಜಿ ಹೇಳು ತ್ತಿದ್ದರೆ ಅದಕ್ಕಾಗಿ ಅವರು ವಿನಂತಿಸಿಕೊಳ್ಳುತ್ತಿದ್ದಾರೆಯೋ ಎಂಬಂತೆ ತೋರುತ್ತಿತ್ತು. ಅವರ ಶಿಷ್ಯರು ಈ ದಿಸೆಯಲ್ಲಿ ಮುಂದುವರಿದು ದೃಢಗೊಂಡ ಹೊರತು ಅವರ ನಿರೀಕ್ಷೆಗೆ ತಕ್ಕ ಶಿಷ್ಯರಾಗಲು ಸಾಧ್ಯವೇ ಇರಲಿಲ್ಲ. ಸ್ವಾಮೀಜಿ ಹೇಳುತ್ತಿದ್ದರು, “ಯಾರಲ್ಲಿ ಏನೂ ಕಾವೇ ಇಲ್ಲವೋ ಅವರು ನಿಗ್ರಹಿಸುವುದಾದರೂ ಏನನ್ನು?... ನನಗೆ ಒಂದು ಐದಾರು ಜನ ಇನ್ನೂ ಯೌವನದ ಹೊಸ್ತಿಲಲ್ಲಿರುವವರುಬೇಕು.”

ಸ್ವಾಮೀಜಿಯ ಮಾತುಗಳಿಗಿಂತ, ಅವರ ಸಾನ್ನಿಧ್ಯವೇ ಅತಿ ಪ್ರಬಲವಾದ ಶಕ್ತಿಯಾಗಿ ಕೆಲಸ ಮಾಡುತ್ತಿತ್ತು. ಅವರ ಮಾತುಗಳನ್ನು ಓದಿದಾಗ ಅವು ಉಂಟುಮಾಡುವ ಪ್ರಭಾವಕ್ಕಿಂತ ಅವರ ತೇಜೋವಲಯದಲ್ಲಿದ್ದವರು ಅನುಭವಿಸುತ್ತಿದ್ದ ಪ್ರಭಾವ ಅತ್ಯಧಿಕವಾಗಿರುತ್ತಿತ್ತು. ಅವರ ಬಳಿ ಇದ್ದವರಿಗೆಲ್ಲ, ಜೀವನ್ಮರಣಗಳನ್ನು ಮೀರಿದ ಅವಸ್ಥೆಗೇರಬೇಕೆಂಬ ಉತ್ಕಟೇಚ್ಛೆಯುಂಟಾಗು ತ್ತಿತ್ತು. ಸ್ವಾಮೀಜಿಯಲ್ಲಿ ಅವರು ನೋಡುತ್ತಿದ್ದುದೇನು? ಲೋಕದ ಬಂಧನದಿಂದ ತಪ್ಪಿಸಿ ಕೊಂಡು ಹಾರಿಹೋಗಲು ಚಡಪಡಿಸುತ್ತಿದ್ದ ಒಂದು ಆತ್ಮವನ್ನು! ಸ್ವಾಮೀಜಿಯ ಪಾಲಿಗೆ ದೇಹವು ಒಂದು ಸಂಕೋಲೆಯಾಗಿತ್ತು; ಅದೊಂದು ಅಸಹನೀಯವಾದ, ಅವಹೇಳನಕರವಾದ ಬಂಧನವಾಗಿತ್ತು. ಇಂಥದೊಂದು ದೃಶ್ಯವನ್ನು ಕಣ್ಣೆದುರಿಗೇ ಕಾಣುತ್ತಿದ್ದ ಅವರ ಶಿಷ್ಯರೂ ಆ ಭಾವದಿಂದ ಪ್ರೇರಿತರಾಗುತ್ತಿದ್ದರು. ಆ ದಿನಗಳಲ್ಲಿ ತಮ್ಮ ಬುದ್ಧಿ-ಮನಸ್ಸುಗಳು ಯಾವ ಸ್ಥಿತಿ ಯಲ್ಲಿದ್ದುವೆಂಬುದನ್ನು ಸೋದರಿ ಕ್ರಿಸ್ಟೀನ ಮುಂದೆ ಬರೆಯುತ್ತಾಳೆ:

“ಆಗ ನಾವು ಬೇರೆಯೇ ಒಂದು ಲೋಕದಲ್ಲಿದ್ದೆವು. ನಾವು ಸಾಧಿಸಬೇಕಾಗಿದ್ದ ಗುರಿ–ಮುಕ್ತಿ! ನಮ್ಮನ್ನು ಕಟ್ಟಿಹಾಕಿರುವ, ಸಕಲ ಜೀವಕೋಟಿಯನ್ನೂ ಬಂಧಿಸಿರುವ ಮಾಯೆಯಿಂದ ಮುಕ್ತಿ! ಮಾಯೆಯ ಬಲೆಯಿಂದ ತಪ್ಪಿಸಿಕೊಳ್ಳುವ ಅವಕಾಶ ಪ್ರತಿಯೊಬ್ಬನಿಗೂ ಒಂದಲ್ಲ ಒಂದು ಸಲ ದೊರೆಯುತ್ತದೆ. ಅಂತೆಯೇ ನಮ್ಮ ಅವಕಾಶವೂ ಬಂದಿತ್ತು. ಸಹಸ್ರದ್ವೀಪೋದ್ಯಾನದ ಆ ದಿನ ಗಳಲ್ಲಿ ನಮ್ಮ ಪ್ರತಿಯೊಂದು ಬಯಕೆ, ಪ್ರತಿಯೊಂದು ಆಕಾಂಕ್ಷೆ, ಪ್ರತಿಯೊಂದು ಹೋರಾಟವೂ ಆ ಒಂದು ಗುರಿಯುತ್ತ ಕೇಂದ್ರೀಕೃತವಾಗಿತ್ತು. ಇದನ್ನು ನಮ್ಮ ಗುರುದೇವರು ಉದ್ದೇಶಪೂರ್ವಕ ವಾಗಿ ನಮ್ಮಿಂದ ಮಾಡಿಸುತ್ತಿದ್ದರು; ನಾವು ನಮಗರಿವಿಲ್ಲದಂತೆಯೇ ಕಣ್ಮುಚ್ಚಿ ಸಾಗುತ್ತಿದ್ದೆವು.”

ಹೀಗೆ ಆ ಶಿಷ್ಯರೆಲ್ಲ ಸ್ವಾಮೀಜಿಯ ಆಧ್ಯಾತ್ಮಿಕ ಶಕ್ತಿಪ್ರವಾಹದಲ್ಲಿ ಸಿಲುಕಿಕೊಂಡು ವೇಗವಾಗಿ ಸಾಗುತ್ತಿದ್ದರು. ಅವರಿಗೆ ಬೇರೆ ದಾರಿಯೇ ಇರಲಿಲ್ಲ. ಆದರೆ ಸ್ವಾಮೀಜಿಯ ವಿಚಾರಧಾರೆಯನ್ನು ಎಲ್ಲರೂ ಸುಲಭವಾಗಿ ಸ್ವೀಕರಿಸಲು ಸಾಧ್ಯವಿರಲಿಲ್ಲ. ಇದಕ್ಕೆ ಅತ್ಯುತ್ತಮ ಉದಾಹರಣೆಯೆಂದರೆ ಸ್ವಾಮೀಜಿಯನ್ನು ಹಾಗೂ ಇತರ ಶಿಷ್ಯರನ್ನು ತನ್ನ ಮನೆಗೆ ಆಹ್ವಾನಿಸಿ, ತರಗತಿಗಳಿಗಾಗಿಯೇ ತನ್ನ ಮನೆಯನ್ನು ಬಿಟ್ಟುಕೊಟ್ಟಿದ್ದ ಮಿಸ್ ಡಚರ್. ಈಕೆ ಅತ್ಯಂತ ಸಂಪ್ರದಾಯನಿಷ್ಠಳಾದ ಮಹಿಳೆ; ಸುಮಾರು ಐವತ್ತು ವರ್ಷ ವಯಸ್ಸಿನ ಅವಿವಾಹಿತೆ. ಈಕೆ ಅತ್ಯಂತ ಶ್ರದ್ಧಾವಂತೆ; ಧಾರ್ಮಿಕ ಜೀವನದಲ್ಲಿ ಪ್ರಾಮಾಣಿಕವಾದ ಆಸಕ್ತಿಯಿದ್ದವಳು. ಇವಳು ಸ್ವಾಮೀಜಿಯ ಉಪನ್ಯಾಸಗಳ ಮೂಲಕ ಅವರ ಸಂಪರ್ಕಕ್ಕೆ ಬಂದಿದ್ದಳು. ಅವರು ಬೋಧಿಸಿದ ತತ್ತ್ವಗಳೆಲ್ಲ ಡಚರಳ ನಂಬಿಕೆ ಗಳಿಗೆ ವಿರುದ್ಧವಾದವುಗಳೇ. ಆದರೆ ಸ್ವಾಮೀಜಿಯಿಂದ, ಅವರ ತತ್ತ್ವಗಳಿಂದ ಆಕರ್ಷಿತಳಾದ ಆಕೆ ಅವರನ್ನು ಅನುಸರಿಸಿ ಬರದಿರಲು ಹೇಗೆ ಸಾಧ್ಯ! ನ್ಯೂಯಾರ್ಕಿನಲ್ಲಿ ಅವರ ಬೋಧನೆಗಳು ಅವಳಿಗೊಂದು ಆಘಾತವನ್ನೇ ಉಂಟು ಮಾಡಿದ್ದರೂ ಅವರ ಅತಿಶಯ ಆಕರ್ಷಣೆಗೆ ಸಿಲುಕಿದ ಆಕೆ, ಸಹಸ್ರದ್ವೀಪೋದ್ಯಾನಕ್ಕೆ ಅವರನ್ನು ಆಹ್ವಾನಿಸಿದಳು. ಹಿಂದೆಂದಿಗಿಂತಲೂ ಪ್ರಬಲವಾದ ವಿಚಾರಧಾರೆಯನ್ನು ಅವಳು ಮೊದಲೇ ನಿರೀಕ್ಷಿಸಿದ್ದಿರಬೇಕು. ಆದರೇನಂತೆ? ಸಿಡಿಲನ್ನು ನಿರೀಕ್ಷಿಸಿದ್ದ ಮಾತ್ರಕ್ಕೆ ಅದು ತಲೆಯ ಮೇಲೆಯೇ ಬಡಿದಾಗ ಆಘಾತವೇನಾದರೂ ಕಡಿಮೆಯಿರು ತ್ತದೆಯೆ? ಮಿಸ್ ಡಚರ್ ಪಟ್ಟ ಕಷ್ಟ ಅಷ್ಟಿಷ್ಟಲ್ಲ. ಅವಳು ಆ ಮೊದಲು ಹೊಂದಿದ್ದ ಎಲ್ಲ ನಂಬಿಕೆಗಳು, ಎಲ್ಲ ಜೀವನಮೌಲ್ಯಗಳು, ಕಲ್ಪನೆಗಳು ಈಗ ನುಚ್ಚುನೂರಾಗಿಬಿಟ್ಟುವು–ಅಥವಾ ಅವಳಿಗೆ ಹಾಗೆನ್ನಿಸಿತು. ಆದರೆ ನಿಜಕ್ಕೂ ಅವೆಲ್ಲ ಮಾರ್ಪಾಡಾಗಿದ್ದುವು ಅಷ್ಟೆ. ಇದರ ಪರಿಣಾಮ ಅವಳ ಮೇಲೆ ಎಷ್ಟು ಸ್ಪಷ್ಟವಾಗಿತ್ತೆಂದರೆ, ಆಕೆಗೆ ಅನಾರೋಗ್ಯವುಂಟಾಗಿ ಕೆಲವೊಮ್ಮೆ ತರಗತಿ ಗಳಿಗೆ ಬರುತ್ತಲೇ ಇರಲಿಲ್ಲ. ಒಮ್ಮೊಮ್ಮೆ ಅವಳು ಒಟ್ಟಿಗೆ ಎರಡು ಮೂರು ದಿನ ಕಾಣಿಸಿಕೊಳ್ಳು ತ್ತಿರಲಿಲ್ಲ. ಇದನ್ನು ಗಮನಿಸಿದ ಸ್ವಾಮೀಜಿ ಇತರ ಶಿಷ್ಯರ ಹತ್ತಿರ ಹೇಳಿದರು, “ನೋಡಿದಿರಾ ಅವಳ ಅಸ್ವಸ್ಥತೆಯನ್ನು? ಇದು ಸಾಮಾನ್ಯವಾದ ಅಸ್ವಸ್ಥತೆಯಲ್ಲ. ಇದು ಅವಳ ಮನಸ್ಸಿನಲ್ಲಿ ನಡೆಯುತ್ತಿರುವ ತೀವ್ರ ಕೋಲಾಹಲಕ್ಕೆ ಅವಳ ಶರೀರ ತೋರುತ್ತಿರುವ ಪ್ರತಿಕ್ರಿಯೆ. ಅವಳಿಗದನ್ನು ಸಹಿಸಿಕೊಳ್ಳಲು ಸಾಧ್ಯವಾಗುತ್ತಿಲ್ಲ.” ಆದರೆ ಸ್ವಾಮೀಜಿಯ ಬೋಧನೆಗಳನ್ನೆಲ್ಲ ಆಕೆ ತಳ್ಳಿ ಹಾಕಿ, ತನ್ನಷ್ಟಕ್ಕೆ ತಾನಿರಲು ಸಾಧ್ಯವೆ? ಇಲ್ಲ, ಅವೆಲ್ಲ ಅತ್ಯಂತ ಸುಸಂಬದ್ಧವಾಗಿವೆ! ಅಲ್ಲದೆ ತನ್ನ ಮನಸ್ಸಿಗೆ ಒಪ್ಪಿಗೆಯಾಗುವ ರೀತಿಯಲ್ಲಿ ಸ್ವಾಮೀಜಿ ಅವುಗಳನ್ನು ವಿವರಿಸುತ್ತಿದ್ದಾರೆ! ಹೋಗಲಿ, ತನ್ನ ಸಾಂಪ್ರದಾಯಿಕ ಭಾವನೆಗಳನ್ನೇ ಸಡಿಲಿಸಿಕೊಳ್ಳೋಣವೇ? ಅದೂ ಸಾಧ್ಯವಿಲ್ಲ; ಅವು ಹುಟ್ಟಿನಿಂದಲೂ ತನ್ನೊಂದಿಗೆ ಬೆಳೆದುಬಂದವುಗಳಲ್ಲವೆ! ಹೀಗೆ ದ್ವಂದ್ವಕ್ಕೆ ಸಿಲುಕಿದ ಆಕೆಯ ಮನಸ್ಸು ಡೋಲಾಯಮಾನವಾಯಿತು. ಆದರೆ ನಿರಂತರವಾಗಿ ಶಿಕ್ಷಣವನ್ನು ಪಡೆದುಕೊಳ್ಳುತ್ತ, ಕ್ರಮೇಣ ಅವರ ಬೋಧನೆಗಳ ಸತ್ತ್ವವನ್ನು ಮನಗಂಡು ಮನಸ್ಸನ್ನು ಸ್ತಿಮಿತಗೊಳಿಸಿಕೊಳ್ಳಲು ಆಕೆ ಸಮರ್ಥಳಾದಳು.

ಸಹಸ್ರದ್ವೀಪೋದ್ಯಾನದ ದಿನಗಳ ಅವಧಿ ಸ್ವಾಮೀಜಿಯ ಜೀವನದಲ್ಲೇ ಅತ್ಯಂತ ಮಹತ್ವ ಪೂರ್ಣವಾದವುಗಳಲ್ಲೊಂದು. ಅಲ್ಲಿ ಅವರಾಡಿದ ಪ್ರತಿಯೊಂದು ಮಾತೂ ವಿಶೇಷ ಶಕ್ತಿ- ಸ್ಫೂರ್ತಿಗಳಿಂದ ಕೂಡಿತ್ತು. ಅವು ಅತ್ಯಂತ ಅನರ್ಘ್ಯವಾದ ರತ್ನಗಳೇ ಸರಿ. ಇದನ್ನು ಮನಗಂಡ ಕೆಲವು ವಿದ್ಯಾರ್ಥಿಗಳು ಅವುಗಳನ್ನೆಲ್ಲ ಬರೆದಿಟ್ಟುಕೊಳ್ಳಲು ಪ್ರಯತ್ನಿಸಿದರು. ಆದರೆ ಅವರ ಪ್ರತಿಯೊಂದು ನುಡಿಯೂ ಅಮೂಲ್ಯವಾಗಿರುವಾಗ ಯಾವುದನ್ನು ಬರೆದುಕೊಳ್ಳುವುದು, ಯಾವು ದನ್ನು ಬಿಡುವುದು! ಅಲ್ಲದೆ ಅವು ಸ್ವಾಮೀಜಿಯ ಸುಮಧುರ ಕಂಠದಿಂದ ಸಂಗೀತದಂತೆ ಹರಿದು ಬರುತ್ತಿರುವಾಗ ಅದನ್ನು ಆಲಿಸುವುದನ್ನು ಬಿಟ್ಟು ಬರೆದುಕೊಳ್ಳುತ್ತ ಕುಳಿತುಕೊಳ್ಳುವಷ್ಟು ತಾಳ್ಮೆ ಯಾರಿಗಿದ್ದೀತು! ಹೀಗೆ ಬರೆದುಕೊಳ್ಳಲು ಪ್ರಯತ್ನಿಸಿದ ಶ್ರೀಮತಿ ಫಂಕೆ, ಅವರ ಮಾತುಗಳಲ್ಲಿ ಎಲ್ಲೋ ಕೆಲವನ್ನು ಗುರುತುಮಾಡಿಟ್ಟುಕೊಂಡಳು. ಆದರೆ ಮಿಸ್ ವಾಲ್ಡೊ ಅತ್ಯಂತ ಪರಿಪೂರ್ಣ ವಾದ ಟಿಪ್ಪಣಿಗಳನ್ನು ಬರೆದುಕೊಂಡಳು.

ಇವುಗಳನ್ನು ಆಕೆ ಬರೆದಿಟ್ಟುಕೊಂಡದ್ದು ತನ್ನ ವೈಯಕ್ತಿಕ ಲಾಭಕ್ಕಾಗಿ. ಇದಾದ ಹನ್ನೆರಡು ವರ್ಷಗಳ ಬಳಿಕ ಅದನ್ನು ನೋಡಿದ ಮಿಸ್ ಲಾರಾ ಗ್ಲೆನ್ (ಸೋದರಿ ದೇವಮಾತಾ) ಉದ್ಗರಿಸಿ ದಳು, “ಇವುಗಳನ್ನು ನೀನೊಬ್ಬಳೇ ಇಟ್ಟುಕೊಳ್ಳುವುದು ಅಪರಾಧ! ಇವು ಇಡಿಯ ಜಗತ್ತಿಗೆ ಸೇರಿದವು!” ಆಗ ಮಿಸ್ ವಾಲ್ಡೊ (ಹರಿದಾಸಿ, ಮುಂದೆ ಸೋದರಿ ಯತಿಮಾತಾ) ಸಂಕೋಚ ದಿಂದ ಹೇಳಿದಳು, “ಆದರೆ ನನಗೆ ಅವು ತೀರ ಚೂರುಪಾರುಗಳಂತೆ ಕಾಣುತ್ತವೆ. ಪ್ರಕಟಿಸ ಬೇಕೆಂದರೆ ಅದು ತುಂಬ ಕಡಿಮೆ... ಅಲ್ಲದೆ ಆ ಆರು ವಾರಗಳ ಅವಧಿಯಲ್ಲಿ ಸ್ವಾಮೀಜಿ ನೀಡಿದ ಅದ್ಭುತ ಬೋಧನೆಯನ್ನು ಸರಿಯಾಗಿ ಪ್ರತಿನಿಧಿಸುವ ಬದಲಾಗಿ ಇವು ತಪ್ಪು ಕಲ್ಪನೆ ಯನ್ನು ನೀಡಬಹುದು.” ಆದರೆ ಅವುಗಳನ್ನು ತಿದ್ದಿ ಪ್ರಕಟಿಸಲು ಆಕೆ ಲಾರಾ ಗ್ಲೆನ್ನಳಿಗೆ ಅನುಮತಿ ಕೊಟ್ಟಳು. ಹೀಗೆ ಪ್ರಕಟವಾದ ಪುಸ್ತಕವೇ \eng{\textit{Inspired Talks}(}ಕನ್ನಡದಲ್ಲಿ ‘ಸ್ಫೂರ್ತಿವಾಣಿ’. ಇವು ‘ಸ್ವಾಮಿ ವಿವೇಕಾನಂದರ ಕೃತಿಶ್ರೇಣಿ’ಯಲ್ಲೂ ಪ್ರಕಟವಾಗಿವೆ.) ಆ ಮಾತುಗಳಲ್ಲಿ ಸ್ವಾಮೀಜಿ ಐತಿಹಾಸಿಕ, ತಾತ್ವಿಕ, ಆಧ್ಯಾತ್ಮಿಕ, ಲೌಕಿಕ ವಿಷಯಗಳೆಲ್ಲದರ ಮೇಲೂ ಬೆಳಕು ಚೆಲ್ಲಿದ್ದಾರೆ. ಅವುಗಳಲ್ಲಿ ಅವರ ಸಮಗ್ರ ವ್ಯಕ್ತಿತ್ವವೇ ಹರಿದುಬಂದಂತಿದೆ. ಈ ಸ್ಫೂರ್ತಿವಾಣಿಗಳಿಗಾಗಿ ಸ್ವಾಮೀಜಿಯ ಅನುಯಾಯಿಗಳೆಲ್ಲರೂ ಮಿಸ್ ವಾಲ್ಡೊಳಿಗೆ ಕೃತಜ್ಞರಾಗಿರಬೇಕು.

ಸಹಸ್ರದ್ವೀಪೋದ್ಯಾನದ ನೀರವ-ಪ್ರಶಾಂತ ವಾತಾವರಣವು, ಹಿಂದೆ ಸ್ವಾಮೀಜಿ ಶ್ರೀರಾಮ ಕೃಷ್ಣರ ದಿವ್ಯಸನ್ನಿಧಿಯಲ್ಲಿ ಕಳೆದ ಪವಿತ್ರ ದಕ್ಷಿಣೇಶ್ವರದಂತಿತ್ತು. ಬಿಡುವಿನ ಅವಧಿಯಲ್ಲಿ ಅವರು, ಅಲ್ಲಿನ ಮರಗಳ ನಡುವೆ ಅಥವಾ ನದಿಯ ತೀರದಲ್ಲಿ ಓಡಾಡುತ್ತಿದ್ದರು; ಇಲ್ಲವೆ ತಾವು ಭಾರತ ದಿಂದ ತರಿಸಿಕೊಂಡ ಬ್ರಹ್ಮಸೂತ್ರ ಭಾಷ್ಯವೇ ಮೊದಲಾದ ಬೃಹತ್ ಗ್ರಂಥಗಳ ಅವಲೋಕನ ಮಾಡುತ್ತಿದ್ದರು. ಒಂದು ದಿನ ನದಿಯ ದಡದ ಮೇಲೆ ಕುಳಿತು ಧ್ಯಾನಮಾಡುತ್ತಿದ್ದಾಗ ಅವರು ನಿರ್ವಿಕಲ್ಪ ಸಮಾಧಿಯ ಸ್ಥಿತಿಗೇ ಏರಿಬಿಟ್ಟರು. ಸ್ವತಃ ಸ್ವಾಮೀಜಿಯೇ ಆ ಅನುಭವವನ್ನು ತಮ್ಮ ಜೀವಮಾನದಲ್ಲಾದ ಅತ್ಯುನ್ನತ ಮಟ್ಟದ ಅನುಭವವೆಂದು ಪರಿಗಣಿಸಿದ್ದರು. ಇಂತಹ ದಿವ್ಯಾ ನಂದದ ಮನಃಸ್ಥಿತಿಯಲ್ಲಿ ಅವರು ಅಳಸಿಂಗರಿಗೆ ಬರೆದ ಒಂದು ಪತ್ರ ಹೀಗಿದೆ:

“ನಾನು ಮುಕ್ತ, ನನ್ನ ಬಂಧನಗಳು ಹರಿದಿವೆ. ನನ್ನೀ ದೇಹ ಬಾಳಲಿ ಅಥವಾ ಬೀಳಲಿ, ನನಗದರಿಂದೇನು? ಭಗವಂತನ ಶಿಶುವಾದ ನಾನು ಒಂದು ಸತ್ಯವನ್ನು ಬೋಧಿಸಬೇಕಾದದ್ದಿದೆ. ಯಾರು ನನಗೀ ಸತ್ಯವನ್ನು ನೀಡಿದನೋ ಅವನೇ ನನ್ನ ಪ್ರಸಾರಕಾರ್ಯದಲ್ಲಿ ನೆರವಾಗಲು ಈ ಪ್ರಪಂಚದ ಅತ್ಯಂತ ಶ್ರೇಷ್ಠರಾದ ಹಾಗೂ ಧೈರ್ಯಶಾಲಿಗಳಾದ ಕಾರ್ಯಕರ್ತರನ್ನು ಕಳಿಸಿಕೊಟ್ಟೇ ಕೊಡುತ್ತಾನೆ.”

ಸ್ವಾಮೀಜಿಗೆ ವಿರುದ್ಧ ಪರಿಸ್ಥಿತಿಗಳು ಎದುರಾದಾಗ ಅಥವಾ ಅಡೆತಡೆಗಳು ಬಂದು ನಿಂತಾಗ ಅವರೊಬ್ಬ ವೀರಸಂನ್ಯಾಸಿಯಾಗಿ ಗುಡುಗುವ ರಭಸವು ರೋಮಾಂಚಕಾರಿಯಾಗಿರುತ್ತಿತ್ತು. ಸ್ವಾಮೀಜಿ ಅಲ್ಲಿನ ಶ್ರೀಮಂತ ಗಣ್ಯವ್ಯಕ್ತಿಗಳ ಕಡೆಗೆ ಗಮನಕೊಡುವುದನ್ನು ಬಿಟ್ಟು ಸಾಮಾನ್ಯರನ್ನು ಮೇಲೆತ್ತುವ ಪ್ರಯತ್ನದಲ್ಲಿ ತೊಡಗಿರುವುದನ್ನು ಕಂಡು ಅವರಿಗೊಬ್ಬರು ಒಂದು ಪತ್ರವನ್ನು ಬರೆದು ಅವರ ವಿಧಾನವನ್ನು ಟೀಕಿಸಿದರು. ನಾವು ಹಿಂದೆಯೇ ನೋಡಿದಂತೆ ಸ್ವಾಮೀಜಿ ತಮ್ಮ ಕಾರ್ಯವಿಧಾನದಲ್ಲಿ ಮಧ್ಯೆ ಪ್ರವೇಶ ಮಾಡಬಂದವರನ್ನೇ ಆಗಲಿ, ಅಧಿಕಪ್ರಸಂಗ ಮಾಡು ವವರನ್ನೇ ಆಗಲಿ ತರಾಟೆಗೆ ತೆಗೆದುಕೊಳ್ಳುತ್ತಿದ್ದರು. ಒಂದು ಬಣ್ಣದ ಪತಂಗ ಹಾರುವುದು ಕಾಣಿಸಿದರೇ ಸಾಕು, ವಿಕಸಿತ ಪುಷ್ಪವೊಂದು ಕಣ್ಣಿಗೆ ಬಿದ್ದರೇ ಸಾಕು, ಅವರು ಭಾವಭರಿತರಾಗಿ ಅದರ ಮೇಲೊಂದು ಭಾಷಣವನ್ನೇ ಮಾಡಿಬಿಡುತ್ತಿದ್ದರು. ಹೀಗೆ ಅತ್ಯಂತ ಸೂಕ್ಷ್ಮವಾದ ಪ್ರಚೋದನೆಯೂ ಅವರೊಳಗಿನ ಪ್ರಚಂಡ ಶಕ್ತಿಯನ್ನು ಜಾಗೃತಗೊಳಿಸಿ, ಆ ಶಕ್ತಿಯು ಮಾತು ಗಳ ಮೂಲಕ ಹೊರಹೊಮ್ಮುವಂತೆ ಮಾಡಬಲ್ಲುದಾಗಿತ್ತು. ಅಂತೆಯೇ ಲೋಕದ ಜನ ತಮ್ಮ ಶೋಕಿಗನುಸಾರವಾಗಿ ಸ್ವಾಮೀಜಿಯೂ ನಡೆದುಕೊಳ್ಳಬೇಕೆಂದು ಅವರನ್ನು ನಿರ್ಬಂಧಪಡಿಸುವ ಪ್ರಯತ್ನ ಮಾಡಿದರೆ ಸ್ವಾಮೀಜಿ ಕೆಣಕಿದ ಕೇಸರಿಯಾಗುತ್ತಿದ್ದರು. ಹಿಂದೆ ಇಂತಹ ಒಂದು ಸಂದರ್ಭದಲ್ಲಿ ಅವರು ಮೇರಿಗೆ ಬರೆದ ಪತ್ರವನ್ನು ನೋಡಿದ್ದೇವೆ. ಈಗ ಮತ್ತೊಬ್ಬ ವ್ಯಕ್ತಿ ಅವರನ್ನು ಟೀಕಿಸಿ ಬರೆದ ಪತ್ರವನ್ನು ಕಂಡಾಗ ಅವರೊಳಗಿನ ತ್ಯಾಗ-ವೈರಾಗ್ಯದ ನರಸಿಂಹ ಎಚ್ಚತ್ತು ಗರ್ಜಿಸಿದ. ಸಹಸ್ರದ್ವೀಪೋದ್ಯಾನದಲ್ಲಿ ಅತ್ಯುನ್ನತ ಮಾನಸಿಕ ಸ್ಥಿತಿಯಲ್ಲಿದ್ದ ಅವರೊಳಗಿ ನಿಂದ, ಅತ್ಯಂತ ಸತ್ವಪೂರ್ಣವಾದ ಪ್ರತಿಕ್ರಿಯೆ ಹೊಮ್ಮಿತು.

ಒಂದು ದಿನ ಅವರು, ಶ್ರೀಮತಿ ಫಂಕೆ ಮೊದಲಾದವರೊಂದಿಗೆ ತ್ಯಾಗ-ವೈರಾಗ್ಯಗಳ ಗರಿಮೆ ಯನ್ನು, ಕಾಷಾಯವಸ್ತ್ರದ ಮೂಲಕ ದೊರೆಯಬಹುದಾದ ಸ್ವಾತಂತ್ರ್ಯದ ಸವಿಯನ್ನು ಬಣ್ಣಿಸು ತ್ತಿದ್ದರು. ಹಾಗೆ ಬಣ್ಣಿಸುತ್ತಿದ್ದವರು ಇದ್ದಕ್ಕಿದ್ದಂತೆ ಅವರನ್ನೆಲ್ಲ ಬಿಟ್ಟು ಸ್ವಲ್ಪ ಹೊತ್ತು ಕಣ್ಮರೆ ಯಾದರು. ಬಳಿಕ ಹಿಂದಿರುಗಿದಾಗ ಅವರ ಕೈಯಲ್ಲೊಂದು ಕವನವಿತ್ತು! ತ್ಯಾಗ-ವೈರಾಗ್ಯಗಳ ಪ್ರಖರತೆಯು ಪ್ರಕಟಗೊಂಡಿರುವ ಸುಂದರ ಕವನವೊಂದನ್ನು ಅತ್ಯಂತ ಅಲ್ಪಾವಧಿಯಲ್ಲಿ ರಚಿಸಿದ್ದರು. ಯಾವನ ಅಧಿಕಪ್ರಸಂಗದ ಪತ್ರವೊಂದು ಈ ದಿವ್ಯ ಕವನದ ಆವಿರ್ಭಾವಕ್ಕೆ ಕಾರಣ ವಾಯಿತೋ ಅವನಿಗೆ ಈ ಕವನವನ್ನು ಕಳಿಸಿಕೊಟ್ಟರು. ಆತ್ಮಸಾಕ್ಷಾತ್ಕಾರದ ವೈಭವದಿಂದ ವಿರಾಜಿಸುವ ವೀರಸಂನ್ಯಾಸಿಯೊಬ್ಬನ ಧೀರಗಂಭೀರ ವ್ಯಕ್ತಿತ್ವವನ್ನು ಇದರಲ್ಲಿ ಕಾಣಬಹುದಾಗಿದೆ. ಈ ಕವನವೇ ‘ಸಂನ್ಯಾಸಿಗೀತೆ\eng{’–Song of the Sannyasin.(}ಇದನ್ನು “ವೀರಸಂನ್ಯಾಸಿ ವಿವೇಕಾ ನಂದ” ಗ್ರಂಥದ ‘ಮೊದಲ ಮಾತಿ’ನಲ್ಲಿ ನೋಡಬಹುದು.)

ಹೀಗೆ ಸ್ವಾಮೀಜಿಯ ದಿವ್ಯ ಸಂಪರ್ಕದಿಂದ ಆ ಇಡೀ ಸಹಸ್ರದ್ವೀಪೋದ್ಯಾನವೇ ಸಮಸ್ತ ಜಗತ್ತಿನಲ್ಲಿ ಒಂದು ವಿಶಿಷ್ಟ ಸ್ಥಾನವನ್ನು ಪಡೆಯಿತು. ೧೮೯೫ರ ಆಗಸ್ಟ್ ೬ರಂದು ಅವರು ತಮ್ಮ ಸ್ಫೂರ್ತಿಯುತ ಬೋಧನೆಯ ಕಾರ್ಯವನ್ನು ಸಮಾಪ್ತಿಗೊಳಿಸಿ ನ್ಯೂಯಾರ್ಕಿನ ಕಡೆಗೆ ಪ್ರಯಾಣ ಹೊರಟರು. ಅಂದಿನ ಆ ಕೊನೆಯ ದಿನವನ್ನು ವರ್ಣಿಸುತ್ತ ಶ್ರೀಮತಿ ಫಂಕೆ ತನ್ನ ಸ್ನೇಹಿತೆಗೆ ಪತ್ರವೊಂದರಲ್ಲಿ ಬರೆದಳು:

“೭ನೇ ಆಗಸ್ಟ್ ಬುಧವಾರ. ಅಯ್ಯೊ! ಸ್ವಾಮೀಜಿ ಬೀಳ್ಗೊಂಡೇಬಿಟ್ಟರು. ಅಂದು ರಾತ್ರಿ ಅವರು ರೈಲಿನಲ್ಲಿ ನ್ಯೂಯಾರ್ಕಿಗೆ ಪ್ರಯಾಣ ಮಾಡಿದರು. ಅಲ್ಲಿಂದ ಅವರು ಇಂಗ್ಲೆಂಡಿಗೆ ಹೋಗಲಿದ್ದಾರೆ.

“ಇಲ್ಲಿನ ಕೊನೆಯ ದಿನ ಅತ್ಯಂತ ಅದ್ಭುತವಾಗಿತ್ತು. ಮತ್ತು ಅತ್ಯಮೂಲ್ಯವಾಗಿತ್ತು. ಅಂದು ಬೆಳಿಗ್ಗೆ ತರಗತಿ ಇರಲಿಲ್ಲ. ಅವರು ನಮ್ಮಿಬ್ಬರೊಂದಿಗೆ ಸ್ವಲ್ಪ ಹೊತ್ತು ಬೇರೆಯಾಗಿರಲು ಇಷ್ಟಪಟ್ಟ ಕಾರಣ, ನನ್ನನ್ನು ಮತ್ತು ಕ್ರಿಸ್ಟೀನಳನ್ನು ತಮ್ಮೊಂದಿಗೆ ತಿರುಗಾಡಿಕೊಂಡು ಬರಲು ಕರೆದರು. (ಇತರರು ಅವರೊಂದಿಗೆ ಇಡೀ ಬೇಸಿಗೆಯನ್ನು ಕಳೆದಿದ್ದರು. ಆದ್ದರಿಂದ ಅವರು ನಮ್ಮೊಂದಿಗೆ ಕೊನೆಯ ಬಾರಿಗೆ ಮಾತುಕತೆಯಾಡಲು ಇಚ್ಛಿಸಿದರು.) ನಾವು ಅರ್ಧ ಮೈಲಿ ದೂರದ ಗುಡ್ಡದ ಮೇಲಕ್ಕೆ ನಡೆದುಕೊಂಡು ಹೋದೆವು. ಅಲ್ಲೆಲ್ಲ ಬರಿಯ ಕಾಡು, ನಿರ್ಜನ ಪ್ರದೇಶ. ಕಡೆಗೆ ಸ್ವಾಮೀಜಿ ಒಂದು ಮರದ ಕೆಳಗೆ ಕುಳಿತರು. ನಾವೂ ಅವರೊಂದಿಗೆ ಕುಳಿತೆವು. ನಾವು ನಿರೀಕ್ಷಿಸಿದ್ದಂತೆ ಮಾತುಕತೆಯಾಡುವುದರ ಬದಲಾಗಿ ಅವರು ಇದ್ದಕ್ಕಿದ್ದಂತೆ ಹೇಳಿದರು, ‘ಬೋಧಿವೃಕ್ಷದ ಕೆಳೆಗೆ ಕುಳಿತ ಬುದ್ಧನಂತೆ ನಾವೀಗ ಧ್ಯಾನ ಮಾಡೋಣ’ ಎಂದು. ಸ್ವಾಮೀಜಿ ಕಂಚಿನ ವಿಗ್ರಹದಂತೆ ಸ್ಥಿರವಾಗಿಬಿಟ್ಟರು. ಸ್ವಲ್ಪ ಹೊತ್ತಿಗೆ ಗುಡುಗು ಸಿಡಿಲುಗಳೊಂದಿಗೆ ಮಳೆ ಧಾರಾಕಾರವಾಗಿ ಸುರಿಯತೊಡಗಿತು. ಆದರೆ ಅವರದನ್ನು ಗಮನಿಸಲೇ ಇಲ್ಲ. ನಾನು ನನ್ನ ಛತ್ರಿಯನ್ನು ಮೇಲೆತ್ತಿ ಹಿಡಿದು ಸಾಧ್ಯವಾದಷ್ಟು ಅವರಿಗೆ ರಕ್ಷಣೆ ನೀಡಿದೆ. ಅವರು ಸಂಪೂರ್ಣ ವಾಗಿ ಧ್ಯಾನದಲ್ಲಿ ಮುಳುಗಿಬಿಟ್ಟಿದ್ದರು. ಅವರಿಗೆ ಯಾವುದರ ಪರಿವೆಯೂ ಇರಲಿಲ್ಲ. ಸ್ವಲ್ಪ ಹೊತ್ತಿನಲ್ಲಿ ನಮಗೆ ದೂರದಲ್ಲಿ ಯಾರೋ ಕೂಗುತ್ತಿರುವುದು ಕೇಳಿತು. ಇತರರೆಲ್ಲ ನಮ್ಮನ್ನು ಹುಡುಕಿಕೊಂಡು ಮಳೆಯಂಗಿಗಳು ಹಾಗೂ ಛತ್ರಿಗಳೊಂದಿಗೆ ಬಂದುಬಿಟ್ಟಿದ್ದರು. ಸ್ವಾಮೀಜಿ ಧ್ಯಾನದಿಂದೆದ್ದು ಸುತ್ತಲೂ ವಿಷಾದದಿಂದ ನೋಡಿದರು... ಏಕೆಂದರೆ ನಾವೆಲ್ಲ ಅಗಲಲೇ ಬೇಕಿತ್ತು!”

ಹೀಗೆ ಸ್ವಾಮೀಜಿ, ಅಮೆರಿಕೆಗೆ ತಾವು ನೀಡಿದ ಮೊದಲ ಭೇಟಿಯ ಅವಧಿಯಲ್ಲಿ ಮಾನವ ಮಾತ್ರರಿಂದ ದುಸ್ಸಾಧ್ಯವಾದ ಕಾರ್ಯವನ್ನು ಸಾಧಿಸಿ ಅಲ್ಲಿ ಅಚ್ಚಳಿಯದ ಮುದ್ರೆಯೊಂದನ್ನು ಒತ್ತಿದರು. ಒಬ್ಬ ನಿರ್ಗತಿಕ ಅನಾಮಧೇಯ ಸಂನ್ಯಾಸಿಯಾಗಿದ್ದ ಅವರು ತಮ್ಮ ಅಂತಸ್ಸತ್ವದ ಬಲದಿಂದ ಮತ್ತು ಭಗವತ್ಕೃಪೆಯಿಂದ ಕೇವಲ ಎರಡೂವರೆ ವರ್ಷಗಳಲ್ಲಿ ಇದೆಲ್ಲವನ್ನೂ ಸಾಧಿಸಿ ದರು. ಅವರ ಈ ಅದ್ಭುತ ಸಾಧನೆಯು ಅಮೆರಿಕದಲ್ಲಿ ಒಂದು ನೂತನ ಆಧ್ಯಾತ್ಮಿಕ ಪರಂಪರೆ ಯನ್ನೇ ನಿರ್ಮಿಸಿತು.

