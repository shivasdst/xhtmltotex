
\chapter{ಇಂಗ್ಲೆಂಡಿನಲ್ಲಿ ಜಯಭೇರಿ}

\noindent

ಲಂಡನ್ನಿನಲ್ಲಿ ಸ್ವಾಮೀಜಿ ಕೆಲದಿನ ಸೇವಿಯರ್ ದಂಪತಿಗಳ ಮನೆಯಲ್ಲಿ ಉಳಿದುಕೊಂಡರು. ಬಳಿಕ ಅವರು ರಿಡ್ಜ್​ಲಿ ಗಾರ್ಡನ್ಸ್ ಎಂಬಲ್ಲಿದ್ದ ಮಿಸ್ ಹೆನ್ರಿಟಾ ಮುಲ್ಲರಳ ನಿವಾಸಕ್ಕೆ ಬಂದು ತಮ್ಮ ಪ್ರಚಾರ ಕಾರ್ಯವನ್ನು ಪುನರಾರಂಭಿಸಿದರು. ಇಲ್ಲಿ ಅವರು ಅತ್ಯಂತ ಯಶಸ್ವಿಯಾದ ಕೆಲವು ಬೈಠಕ್​ಖಾನೆಯ ಉಪನ್ಯಾಸಗಳನ್ನು ಮಾಡಿದರು. ಇದಾದ ಮೇಲೆ ಅವರು ಲಂಡನ್ನಿನ ಉಪನಗರವಾದ ವಿಂಬಲ್ಡನ್ನಿನಲ್ಲಿ ವಾರಕ್ಕೊಂದರಂತೆ ರಾಜಯೋಗದ ಹಾಗೂ ಧ್ಯಾನದ ತರಗತಿ ಗಳನ್ನು ನಡೆಸಿಕೊಂಡು ಬಂದರು.

ಅಕ್ಟೋಬರ್ ತಿಂಗಳ ವೇಳೆಗೆ ಸ್ವಾಮೀಜಿ, ಲಂಡನ್ನಿನಲ್ಲಿ ತಮ್ಮ ವಾರದ ತರಗತಿಗಳನ್ನು ಪೂರ್ಣಪ್ರಮಾಣದಲ್ಲಿ ಪ್ರಾರಂಭಿಸಿದರು. ಈ ಬಾರಿ ತರಗತಿಗಳಿಗಾಗಿ ನಗರದ ಅತ್ಯಂತ ಜನ ಭರಿತ ಸ್ಥಳಗಳಲ್ಲೊಂದಾದ ವಿಕ್ಟೋರಿಯಾ ಸ್ಟ್ರೀಟ್ ಎಂಬಲ್ಲಿ ಸ್ಟರ್ಡಿ ದೊಡ್ಡ ಹಜಾರವನ್ನು ಗೊತ್ತುಮಾಡಿದ್ದ. ಹೆಚ್ಚಿನ ಜನರಿಗೆ ಸ್ವಾಮೀಜಿಯ ಮಾತುಗಳನ್ನು ಕೇಳುವ ಅವಕಾಶವಾಗಲಿ ಎಂದು ಈ ವ್ಯವಸ್ಥೆ ಮಾಡಲಾಗಿತ್ತು. ಹಿಂದೆಯೇ ನೋಡಿದಂತೆ, ಶರತ್ಕಾಲದ ತರಗತಿಗಳಿಗಾಗಿ ಅವರ ಶಿಷ್ಯರು ಅದಾಗಲೇ ಹಣಸಂಗ್ರಹ ಮಾಡಿಬಿಟ್ಟಿದ್ದರು. ಈ ಹಜಾರದಲ್ಲಿ ಸುಮಾರು ಇನ್ನೂರು ಜನರಿಗೆ ಸ್ಥಳಾವಕಾಶವಿತ್ತು. ಜೊತೆಗೆ ಒಂದು ಚಿಕ್ಕ ಗ್ರಂಥಾಲಯದ ಸೌಲಭ್ಯವೂ ಇತ್ತು.

ಈ ವೇಳೆಗೆ ಸ್ವಾಮೀಜಿಯ ಕರೆಯನ್ನು ಮನ್ನಿಸಿ, ಸ್ವಾಮಿ ಅಭೇದಾನಂದರು ಲಂಡನ್ನಿಗೆ ಆಗಮಿಸಿದ್ದರು. ಅಲ್ಲದೆ, ನೆಚ್ಚಿನ ಸಹಾಯಕ ಗುಡ್​ವಿನ್ನನೂ ಅಮೆರಿಕದಿಂದ ಹಿಂದಿರುಗಿದ್ದ. ಇವರಿಬ್ಬರೊಂದಿಗೆ ಸ್ವಾಮೀಜಿ, ವೆಸ್ಟ್​ಮಿನಿಸ್ಟರ್​ನ ಮನೆಯೊಂದರಲ್ಲಿ ಇಳಿದುಕೊಂಡರು.

ಆ ವರ್ಷದ ಕೊನೆಯಲ್ಲಿ ಭಾರತಕ್ಕೆ ಹಿಂದಿರುಗಲು ಸ್ವಾಮೀಜಿ ಆಲೋಚಿಸಿದ್ದರು. ತಮ್ಮ ಕಾರ್ಯವನ್ನು ಮುಂದುರಿಸಿಕೊಂಡು ಹೋಗಲು ಅಭೇದಾನಂದರನ್ನು ಸಿದ್ಧಗೊಳಿಸುವುದು ಅವರ ಉದ್ದೇಶವಾಗಿತ್ತು. ಏಕೆಂದರೆ, ಯಾವ ಕಾರ್ಯವನ್ನು ಅವರು ಲೀಲಾಜಾಲವಾಗಿ ನಡೆಸಿಕೊಂಡು ಹೋಗುತ್ತಿದ್ದರೋ ಅದನ್ನು ಇತರರು ಅಷ್ಟೇ ಸುಲಭವಾಗಿ ನಡೆಸಲು ಸಾಧ್ಯವೇ ಇರಲಿಲ್ಲ. ಅದಕ್ಕೆ ಅಪಾರ ಜ್ಞಾನ, ಶ್ರದ್ಧೆ, ಉತ್ಸಾಹ, ಶಕ್ತಿ, ಧೈರ್ಯ ಬೇಕು. ಅಲ್ಲದೆ ಆ ಕಾರ್ಯವನ್ನು ನಡೆಸ ಬೇಕೆಂದರೆ ಆಧ್ಯಾತ್ಮಿಕವಾಗಿಯೂ ಬೌದ್ಧಿಕವಾಗಿಯೂ ಪ್ರಚಂಡರಾದವರಿಂದ ಮಾತ್ರವೇ ಸಾಧ್ಯ. ಅದೊಂದು ಗುರುತರ ಜವಾಬ್ದಾರಿ. ಅಭೇದಾನಂದರು ಈ ಕಾರ್ಯಕ್ಕೆ ತುಂಬ ಸೂಕ್ತರಾದ ವ್ಯಕ್ತಿ ಯಾಗಿದ್ದರೂ ಕಾರ್ಯರಂಗಕ್ಕಿಳಿಯಲು ಅವರಿನ್ನೂ ಹಿಂಜರಿಯುತ್ತಿದ್ದರು. ಆದ್ದರಿಂದ ಅವರನ್ನು ಪ್ರಚೋದಿಸಿ ಉತ್ಸಾಹಗೊಳಿಸಲು ಹಾಗೂ ಅವರಲ್ಲಿ ಹೊಣೆಗಾರಿಕೆಯ ಅರಿವುಂಟುಮಾಡಿಸಲು ಸ್ವಾಮೀಜಿ ಪ್ರಯತ್ನಿಸಿದರು.

ಲಂಡನ್ನಿನಲ್ಲಿ ಸ್ವಾಮೀಜಿಯವರ ಕಾರ್ಯ ತೀವ್ರಗತಿಯಲ್ಲಿ ಸಾಗುತ್ತಿತ್ತು. ತಮ್ಮ ಉದ್ದೇಶ ಕೈಗೂಡುವುದೆಂಬ ಭರವಸೆ ಅವರಿಗುಂಟಾಗಿತ್ತು. ಶ್ರದ್ಧಾಳುಗಳೂ ಸಮರ್ಥರೂ ಆದ ಇಪ್ಪತ್ತು ಜನ ಸಿಕ್ಕಿದರೆ, ಒಂದಿಪ್ಪತ್ತು ವರ್ಷಗಳಲ್ಲಿ ಪಾಶ್ಚಾತ್ಯ ದೇಶಗಳಲ್ಲೆಲ್ಲ ಶಾಶ್ವತವಾದ ಪ್ರಭಾವವನ್ನು ಬೀರಬಹುದೆಂದು ಅವರಿಗೆ ವಿಶ್ವಾಸವಿತ್ತು. ಅಲ್ಲದೆ ಪಾಶ್ಚಾತ್ಯ ದೇಶಗಳಲ್ಲಿನ ತಮ್ಮ ಯಶಸ್ಸು, ಭಾರತದಲ್ಲಿ ಉಂಟುಮಾಡಬಹುದಾದ ಸತ್ಪರಿಣಾಮವನ್ನು ಅವರು ಕಂಡುಕೊಂಡಿದ್ದರು. “ಭಾರತದ ಹೊರಗೆ ಕೊಟ್ಟ ಒಂದು ಏಟು, ಭಾರತದ ಒಳಗೆ ಕೊಟ್ಟ ನೂರು ಏಟುಗಳಿಗೆ ಸಮ” ಎಂದು ಅವರು ಪತ್ರವೊಂದರಲ್ಲಿ ಬರೆದರು. ಅಂದಿನ ಪರಿಸ್ಥಿತಿ ಇಂದಿಗಿಂತ ಹೆಚ್ಚೇನೂ ಭಿನ್ನವಾಗಿರಲಿಲ್ಲ!

ಲಂಡನ್ನಿನಲ್ಲಿ ಸ್ವಾಮೀಜಿ ತಮ್ಮ ಕಾರ್ಯದ ಕೊನೆಯ ಹಂತವನ್ನೀಗ ಪ್ರಾರಂಭಿಸಿದ್ದರು. ಇದು ಅತ್ಯಂತ ಚಟುವಟಿಕೆಯಿಂದ ಕೂಡಿದುದಾಗಿರುತ್ತದೆಂದು ಅವರು ಮೊದಲೇ ಊಹಿಸಿ ದ್ದರು. ಎರಡು ತಿಂಗಳ ಪ್ರವಾಸ ಹಾಗೂ ವಿಶ್ರಾಂತಿಯ ಬಳಿಕ ಅವರೀಗ ಬಹಳಷ್ಟು ಚೇತರಿಸಿ ಕೊಂಡಿದ್ದರು. ನವ ಸ್ಫೂರ್ತಿ, ಉತ್ಸಾಹಗಳಿಂದ ಕೂಡಿ ಅವರು ಪೂರ್ಣ ರಭಸದಿಂದ ಕಾರ್ಯ ಮಗ್ನರಾದರು. ಅಲ್ಲದೆ ಅವರ ಮನಸ್ಸು-ಬುದ್ಧಿ ಮತ್ತಷ್ಟು ತಿಳಿಯಾಗಿದ್ದು, ಅವರಲ್ಲೀಗ ಹೊಸ ಹೊಸ ಆಲೋಚನೆಗಳೂ ಭಾವನೆಗಳೂ ಮೂಡುತ್ತಿದ್ದುವು. ವೇದಾಂತದ ಕುರಿತಾಗಿ ಅವರೀಗ ಹೊಚ್ಚಹೊಸ ಅರ್ಥವನ್ನು ನೀಡಬಲ್ಲವರಾಗಿದ್ದರು. ಅಕ್ಟೋಬರ್​-ನವೆಂಬರ್ ತಿಂಗಳ ಅವಧಿ ಯಲ್ಲಿ ಅವರ ಕಾರ್ಯದ ಅತ್ಯಂತ ಗಮನಾರ್ಹ ಅಂಶ ಯಾವುದೆಂದರೆ, ಅವರು ವೇದಾಂತವನ್ನು ಅತ್ಯಂತ ಅನುಷ್ಠಾನಾತ್ಮಕವಾಗಿ ಮತ್ತು ಅತ್ಯುನ್ನತ ತಾತ್ವಿಕ ದೃಷ್ಟಿಕೋನದಿಂದ ವಿವರಿಸಿದುದು.

ಸ್ವಾಮೀಜಿಯೇ ಹೇಳಿದಂತೆ ಬೌದ್ಧಿಕವಾಗಿ ಅವರ ಶ್ರೇಷ್ಠತಮ ಕೊಡುಗೆಯೆಂದರೆ ಮಾಯೆಯ ಕುರಿತಾದ ಅವರ ವ್ಯಾಖ್ಯೆ. ಈ ಅವಧಿಯಲ್ಲಿ ವಾರಕ್ಕೆ ಮೂರು ದಿನ ನಡೆಯುತ್ತಿದ್ದ ತರಗತಿಗಳಲ್ಲಿ ಹೆಚ್ಚಿನ ಭಾಗ ಜ್ಞಾನಯೋಗಕ್ಕೆ ಮೀಸಲಾಗಿತ್ತು. ಗಹನವಾದ ಮಾಯಾತತ್ತ್ವದ ಅಪ್ರತಿಮ ವ್ಯಾಖ್ಯಾನದೊಂದಿಗೆ ಸ್ವಾಮೀಜಿ ತಮ್ಮ ಈ ತರಗತಿಗಳನ್ನು ಪ್ರಾರಂಭಿಸಿದರು. ಅತ್ಯಂತ ಸೂಕ್ಷ್ಮವೂ ಕಠಿಣವೂ ಆದ ಈ ಸಿದ್ಧಾಂತವನ್ನು ಸ್ಪಷ್ಟವಾಗಿ ವ್ಯಾಖ್ಯಾನಿಸುವುದರಲ್ಲಿ ಭಾರತದ ಪ್ರಾಚೀನ ವಿದ್ವಾಂಸರೂ ಅಸಮರ್ಥರಾಗಿದ್ದರು. ಅಲ್ಲದೆ ಆಧುನಿಕ ತರ್ಕಶಾಸ್ತ್ರವನ್ನು ಜೀರ್ಣಿಸಿ ಕೊಂಡ ಪಾಶ್ಚಾತ್ಯ ಪಂಡಿತರನ್ನೂ ಇದು ತಬ್ಬಿಬ್ಬಾಗಿಸಿದೆ. ಈ ಸರಣಿಯಲ್ಲಿ ಸ್ವಾಮೀಜಿಯ ಎಲ್ಲ ಉಪನ್ಯಾಸಗಳೂ ಮುಖ್ಯವಾಗಿ ಮಾಯೆಗೆ ಸಂಬಂಧಿಸಿದುದೇ ಆಗಿತ್ತು. ಅವರ ಈ ಉಪನ್ಯಾಸ ಗಳನ್ನು ಅಧ್ಯಯನ ಮಾಡಿದಾಗ, ಅತ್ಯಂತ ಕಷ್ಟಕರವಾದ ಈ ಕಾರ್ಯದಲ್ಲಿ ಅವರು ಎಷ್ಟು ಯಶಸ್ವಿ ಯಾದರೆಂಬುದು ತಿಳಿಯುತ್ತದೆ. ಇವುಗಳಲ್ಲಿ ಕೆಲವು ಉಪನ್ಯಾಸಗಳ ವಿಷಯಗಳು ಹೀಗಿದ್ದುವು: “ಮಾಯೆ ಮತ್ತು ಭ್ರಾಂತಿ”, “ಮಾಯೆ ಮತ್ತು ದೇವರ ಕುರಿತಾದ ಕಲ್ಪನೆಯ ವಿಕಾಸ”, “ಮಾಯೆ ಮತ್ತು ಮುಕ್ತಿ” ಹಾಗೂ “ಪರತತ್ತ್ವ ಮತ್ತು ಅದರ ಅಭಿವ್ಯಕ್ತಿ.” ಇವುಗಳ ಅನಂತರ ಸ್ವಾಮೀಜಿ ನೀಡಿದ ಉಪನ್ಯಾಸಗಳಾದ “ಎಲ್ಲದರಲ್ಲೂ ಭಗವಂತ”, “ಸಾಕ್ಷಾತ್ಕಾರ”, “ವಿವಿಧತೆಯಲ್ಲಿ ಏಕತೆ” ಮತ್ತು “ಆತ್ಮದ ಮುಕ್ತಿ” ಹಾಗೂ “ಅನುಷ್ಠಾನ ವೇದಾಂತ” (ಹಾಗೆಂದು ಹೆಸರಿಸಲಾದ ಅವರ ಕಡೆಯ ನಾಲ್ಕು ಉಪನ್ಯಾಸಗಳು)–ಇವುಗಳಲ್ಲಿ ಅದ್ವೈತದ ಭಾವ ಉಕ್ಕಿಹರಿಯುವುದನ್ನು ಕಾಣ ಬಹುದಾಗಿದೆ. ‘ಸತ್​-ಚಿತ್​-ಆನಂದ’ವೇ ಮಾನವನ ನಿಜಸ್ವರೂಪ; ಮಾನವನ ಆತ್ಮವು ಎಂದೆಂದಿಗೂ ಮುಕ್ತವಾದುದು ಮತ್ತು ಎಲ್ಲ ಇಂದ್ರಿಯಗ್ರಾಹ್ಯ ವಿಷಯಗಳೂ ಆ ಆತ್ಮದ ವಿಭಿನ್ನ ರೂಪಗಳಷ್ಟೆ ಎಂದು ಸ್ವಾಮೀಜಿ ಬೋಧಿಸಿದರು. ಅವರು ಪ್ರತಿಪಾದಿಸಿದುದು ವೈಚಾರಿಕ ಧರ್ಮವನ್ನು. ಪ್ರಾಯಶಃ ಅದಕ್ಕಿಂತ ಉತ್ತಮವೂ ವೈಚಾರಿಕವೂ ಆದ ರೀತಿಯಲ್ಲಿ ಧರ್ಮವನ್ನು ಪ್ರತಿಪಾದಿಸಿದವರು ಯಾರೂ ಇಲ್ಲ. ಯೂರೋಪಿನ ಮುಕ್ತಿ ಅವಲಂಬಿಸಿರುವುದು ಈ ವೈಚಾರಿಕ ಧರ್ಮವನ್ನೇ ಎಂಬುದು ಸ್ವಾಮೀಜಿಯ ಅಭಿಪ್ರಾಯವಾಗಿತ್ತು. ಅತ್ಯುನ್ನತ ಅತೀಂದ್ರಿಯ ಸತ್ಯ ಗಳಿಗೆ ಕಾವ್ಯಾತ್ಮಕ ಅಭಿವ್ಯಕ್ತಿಯನ್ನು ಕೊಡುವಲ್ಲಿ ಅವರದು ಅಸಾಧಾರಣ ಸಾಮರ್ಥ್ಯ. ಉನ್ನತ ಅದ್ವೈತ ತತ್ತ್ವವು ಅವರ ವ್ಯಾಖ್ಯಾನದಲ್ಲಿ ಹೊಸ ಕಳೆಯನ್ನು ತಳೆದು, ಮಾನವನ ಉದಾತ್ತ ಉತ್ಕಾಂಕ್ಷೆಯಾಗಿ ವ್ಯಕ್ತಗೊಂಡಿತು. ಅವರ ಉಪನ್ಯಾಸಗಳೆಲ್ಲ ಯಾವುದೇ ಕನಿಷ್ಠ ತಯಾರಿಯೂ ಇಲ್ಲದೆ, ಆ ಕ್ಷಣದ ಸ್ಫೂರ್ತಿಯಿಂದುದಿಸಿದಂಥವು. ಆದರೆ ಆ ಮಾತುಗಳ ಹಿಂದಿನ ಅವರ ಶಕ್ತಿ ಶಾಲೀ ವ್ಯಕ್ತಿತ್ವವು, ಆ ಮಾತುಗಳನ್ನು ಶ್ರೋತೃವರ್ಗದ ಮೇಲೆ ಮಿಂಚಿನಂತೆರಗುವಂತೆ ಮಾಡಿತು. ಒಮ್ಮೆಯಂತೂ ಮಾಯೆಯ ಕುರಿತಾಗಿ ಉಪನ್ಯಾಸ ಮಾಡುತ್ತಿದ್ದಾಗ ಸ್ವಾಮೀಜಿ ಎಂತಹ ಉನ್ನತ ಭಾವಾವಸ್ಥೆಗೇರಿದರೆಂದರೆ, ಅವರ ಮಾತುಗಳನ್ನು ಆಲಿಸುತ್ತಿದ್ದವರಿಗೆ ಬೇರೆಯೇ ಒಂದು ಲೋಕ ವನ್ನು ಪ್ರವೇಶಿಸುತ್ತಿರುವಂತೆ ಭಾಸವಾಯಿತು. ಆ ಒಂದು ಕ್ಷಣದಲ್ಲಿ ಅವರೆಲ್ಲರಿಗೂ ತಾವು ಪರ ಬ್ರಹ್ಮದಲ್ಲಿ ಒಂದೆಂಬ ಅನುಭವವಾಯಿತು. ಅತ್ಯುನ್ನತ ಆಧ್ಯಾತ್ಮಿಕ ಸಾಕ್ಷಾತ್ಕಾರವನ್ನು ಮಾಡಿ ಕೊಂಡ ವ್ಯಕ್ತಿಯೊಬ್ಬನು ಕೇವಲ ತನ್ನ ವಾಣಿಯಿಂದಲೇ ಇತರರಲ್ಲಿ ಆಧ್ಯಾತ್ಮಿಕತೆ ಪ್ರವಹಿಸು ವಂತೆ ಮಾಡಬಲ್ಲನೆಂಬುದು ಸ್ಪಷ್ಟವಾಗಿ ವ್ಯಕ್ತವಾಗುತ್ತಿತ್ತು.

ಸ್ವಾಮೀಜಿಯ ಸಹಾಯಕನಾಗಿದ್ದ ಗುಡ್​ವಿನ್ ಕೇವಲ ಒಬ್ಬ ಶೀಘ್ರಲಿಪಿಕಾರನಾಗಿರಲಿಲ್ಲ. ಅವನು ಸ್ವತಃ ಸ್ವಾಮೀಜಿಯ ಉಪನ್ಯಾಸಗಳನ್ನು ಶ್ರದ್ಧೆಯಿಂದ ಆಲಿಸುತ್ತ ಅವರ ಬೋಧನೆಗಳನ್ನು ಗ್ರಹಿಸುತ್ತಿದ್ದ. ಅದ್ವೈತ ವೇದಾಂತದ ಕುರಿತಾದ ಅವರ ಪ್ರವಚನಗಳನ್ನು ಆಲಿಸಿ, ಅದರ ಉದಾತ್ತತೆಗೆ ಮಾರುಹೋಗಿದ್ದ. ಅವನು ಶ್ರೀಮತಿ ಸಾರಾಬುಲ್ಲಳಿಗೆ ಬರೆದ ಒಂದು ಪತ್ರದಲ್ಲಿ ಈ ಪ್ರವಚನಗಳ ಬಗ್ಗೆ ತಿಳಿಸುತ್ತಾನೆ: “ಅವರ ಉಪನ್ಯಾಸಗಳು ದಿನದಿನಕ್ಕೂ ಹೆಚ್ಚು ಹೆಚ್ಚು ಉತ್ತಮವಾಗುತ್ತಿರುವಂತೆ ಅನ್ನಿಸುತ್ತದೆ. ಅವರು ಬುಧವಾರ ಸಂಜೆ ಮಾಡಿದ ಉಪನ್ಯಾಸವು ವೇದಾಂತದ ನವನೂತನ ಪ್ರತಿಪಾದನೆಯಾಗಿತ್ತು. ಅದರಲ್ಲಿ ಅವರು ವ್ಯಕ್ತಪಡಿಸಿದ ಭಾವನೆಗಳು, ಅವರು ಇತ್ತೀಚೆಗಷ್ಟೇ ಆಲೋಚಿಸಿದುದಾಗಿತ್ತು. ಅದು ನಿಜಕ್ಕೂ ಅತ್ಯುದ್ಭುತವಾಗಿತ್ತು.” ಹೀಗೆ ಹೇಳಿದ ಗುಡ್​ವಿನ್, ಆ ಭಾಷಣದ ಸಾರಾಂಶವನ್ನೂ ಬರೆದು ತಿಳಿಸಿದ್ದ. ಈ ಪತ್ರದಲ್ಲಿ ಅವನು ಪ್ರಸ್ತಾಪಿಸಿದ ಭಾಷಣವು “ಅನುಷ್ಠಾನ ವೇದಾಂತ” ಎಂಬ ಉಪನ್ಯಾಸ ಮಾಲಿಕೆಯಲ್ಲಿ ಕೊನೆಯದು.

ತಾವು ಲಂಡನ್ನಿನಲ್ಲಿದ್ದ ಅವಧಿಯಲ್ಲಿ ಸ್ವಾಮೀಜಿ ಡಾ ॥ ಪಾಲ್ ಡಾಯ್ಸನ್ನರೊಂದಿಗೆ ಸಂಪರ್ಕ ಇಟ್ಟುಕೊಂಡಿದ್ದರು. ಸ್ವಾಮೀಜಿಯೊಂದಿಗೆ ವೇದಾಂತದ ಬಗ್ಗೆ ಚರ್ಚಿಸುವ ಉದ್ದೇಶದಿಂದಲೇ ಲಂಡನ್ನಿಗೆ ಬಂದಿದ್ದ ಡಾಯ್ಸನ್ನರು, ಮೊದಲ ಎರಡು ವಾರಗಳಂತೂ ಹೆಚ್ಚುಕಡಿಮೆ ಅವ ರೊಂದಿಗೇ ಇದ್ದುಬಿಟ್ಟಿದ್ದರು. ಈ ಚರ್ಚೆಗಳ ಫಲವಾಗಿ ಡಾಯ್ಸನ್ನರು ವೇದಾಂತ ತತ್ತ್ವದ ಬಗ್ಗೆ ಹೊಸ ಪರಿಜ್ಞಾನವನ್ನು ಪಡೆದುಕೊಂಡರು. ಸ್ವತಃ ವೇದಾಂತದ ನಿಷ್ಠಾವಂತ ಪ್ರತಿಪಾದಕರಾಗಿದ್ದ ಅವರು, ಅದರ ಬಗ್ಗೆ ಹೊಸ ದೃಷ್ಟಿಕೋನವನ್ನು ಬೆಳೆಸಿಕೊಂಡರು. ‘ಭಾರತೀಯ ತತ್ತ್ವಜ್ಞಾನವನ್ನು ಪೂರ್ವಗ್ರಹಪೀಡಿತ ದೃಷ್ಟಿಯಿಂದಲೇ ನೋಡುವ ಅಭ್ಯಾಸವಾಗಿರುವುದರಿಂದಲೇ ಪಾಶ್ಚಾತ್ಯ ತತ್ತ್ವಶಾಸ್ತ್ರಜ್ಞರಿಗೆ ವೇದಾಂತ ಸಿದ್ಧಾಂತವನ್ನು ಯಥಾರ್ಥವಾಗಿ ಅರಿತುಕೊಳ್ಳಲು ಸಾಧ್ಯವಾಗು ತ್ತಿಲ್ಲ’ ಎಂದು ಸ್ವಾಮೀಜಿ ಹೇಳಿದರು. ಈ ಅಭಿಪ್ರಾಯನ್ನು ಪ್ರೊ ॥ ಡಾಯ್ಸನ್ ಒಪ್ಪಿಕೊಂಡರು. ಅಲ್ಲದೆ, ಸ್ವಾಮೀಜಿಯೊಂದಿಗಿನ ಸಂಪರ್ಕ ಹೆಚ್ಚು ನಿಕಟವಾದಂತೆ ಅವರಿಗೆ ಅರಿವಾಯಿತು: ಭಾರತೀಯ ದರ್ಶನಗಳ ನಿಜಾರ್ಥವನ್ನು ಅರಿಯಬೇಕಾದರೆ ಪಾಶ್ಚಾತ್ಯರು ತಮ್ಮ ‘ಪಾಶ್ಚಾತ್ಯ ದೃಷ್ಟಿಕೋನ’ದ ಮಬ್ಬಿನಿಂದ ಪಾರಾಗಬೇಕಾಗುತ್ತದೆ; ಏಕೆಂದರೆ ಅವು ಕೇವಲ ಬೌದ್ಧಿಕ ಕಸರತ್ತು ಗಳಲ್ಲ, ಬುದ್ಧಿಯಿಂದರಿಯಲಾಗದ ಮತ್ತು ಹೃದಯ ಮಾತ್ರವೇ ಕಾಣಬಲ್ಲ ಆಧ್ಯಾತ್ಮಿಕ ದರ್ಶನ ಗಳು, ಎಂದು. ಇದೇ ವೇಳೆಯಲ್ಲಿ ಪ್ರೊ ॥ ಮ್ಯಾಕ್ಸ್ ಮುಲ್ಲರರೂ ಸ್ವಾಮೀಜಿಯ ಸಂಪರ್ಕದಲ್ಲಿ ದ್ದರು. ಹೀಗೆ ಮೂರು ಮಹಾ ಬುದ್ಧಿಮತ್ತೆಗಳು ಪರಸ್ಪರ ಸಂಭಾಷಣೆಯಲ್ಲಿ ತೊಡಗಿದ್ದುವು.

ಲಂಡನ್ನಿನಲ್ಲಿ ಸ್ವಾಮೀಜಿ ಹಲವಾರು ಸುಪ್ರಸಿದ್ಧ ವಿಚಾರವಂತ ವ್ಯಕ್ತಿಗಳನ್ನು ಸಂಧಿಸಿದರು. ಇವರಲ್ಲಿ ಕೆಲವರೆಂದರೆ ಖ್ಯಾತ ಮನೋವಿಜ್ಞಾನಿಯೂ ಲೇಖಕನೂ ಆದ ಫೆಡ್ರಿಕ್ ಮೈಯರ್ಸ್, ಪ್ರಸಿದ್ಧ ಪಾದ್ರಿ ಮಾನ್​ಕ್ಯೂರ್ ಡಿ. ಕಾನ್​ವೇ, ತತ್ತ್ವ ಶಾಸ್ತ್ರಜ್ಞನೂ ಶಾಂತಿವಾದಿಯೂ ಆದ ಡಾ॥ ಸ್ಟ್ಯಾಂಟನ್ ಕಾಯ್ಟ್ ಹಾಗೂ ಮತ್ತೊಬ್ಬ ಹೆಸರಾಂತ ಧರ್ಮಾಧಿಕಾರಿಯಾದ ಕ್ಯಾನನ್ ಆರ್. ಹೇವೀಸ್. ಇವರಲ್ಲದೆ ಇನ್ನೂ ಹಲವಾರು ಗಣ್ಯ ವ್ಯಕ್ತಿಗಳ ಪರಿಚಯದಿಂದ ಸ್ವಾಮೀಜಿಗೆ ಎರಡು ಬಗೆಯ ಲಾಭವಾಯಿತು: ಈ ವ್ಯಕ್ತಿಗಳೊಂದಿಗೆ ಅಭಿಪ್ರಾಯವಿನಿಮಯ ಮಾಡಿಕೊಳ್ಳುವ ಅವ ಕಾಶ ಸಿಕ್ಕಿದುದು ಒಂದಾದರೆ, ಅವರ ಕಾರ್ಯಯೋಜನೆಗೆ ನೈತಿಕ ಬೆಂಬಲ ಹಾಗೂ ಮನ್ನಣೆ ದೊರೆತದುದು ಮತ್ತೊಂದು. ಶ್ರೀಮತಿ ಬುಲ್​ಗೆ ಪತ್ರವೊಂದರಲ್ಲಿ ಸ್ವಾಮೀಜಿ ಬರೆದರು, “ಇಲ್ಲಿ ಇಂಗ್ಲೆಂಡಿನಲ್ಲಿ ಎಲ್ಲವೂ ಅನುಕೂಲಕರವಾಗಿ ನಡೆದುಕೊಂಡು ಬರುತ್ತಿದೆ. ಇದೊಂದು ದೊಡ್ಡ ಮುನ್ನಡೆ. ಇಲ್ಲಿ ನನ್ನ ಕಾರ್ಯವು ಜನಪ್ರಿಯವಾಗಿರುವುದಲ್ಲದೆ ಗೌರವಯುತವೂ ಆಗಿದೆ.” ಆದರೆ ಸ್ವಾಮೀಜಿಯ ಸಂಪರ್ಕಕ್ಕೆ ಬಂದವರಿಗೆ ಪ್ರತಿಯಾಗಿ ಆದ ಲಾಭ ಇನ್ನೂ ಹೆಚ್ಚಿನದು. ಸ್ವಾಮೀಜಿಯ ಪರಿಶುದ್ಧ-ಪವಿತ್ರ-ತೇಜೋಮಯ ವ್ಯಕ್ತಿತ್ವ ಹಾಗೂ ಬೋಧನೆಗಳು ಅವರ ಮೇಲೆ ಆಳವಾದ ಪ್ರಭಾವ ಬೀರಿದುವು. ಇವರಲ್ಲಿ ಅನೇಕ ಸಂಪ್ರದಾಯಸ್ಥರೂ ವೇದಾಂತ ತತ್ತ್ವದ ನಿಷ್ಠಾವಂತ ಅನುಯಾಯಿಗಳಾದರು. ಕೆಲವು ಕ್ರೈಸ್ತ ಪ್ರಚಾರಕರು ಕೂಡ ಸ್ವಾಮೀಜಿಯ ಬೋಧನೆಗಳನ್ನು ತಮ್ಮ ಪ್ರವಚನಗಳಲ್ಲಿ ಅಳವಡಿಸಿಕೊಂಡಿದ್ದುದು ಸುಸ್ಪಷ್ಟವಾಗಿತ್ತು.

ಹೀಗೆ ಸ್ವಾಮೀಜಿ ಲಂಡನ್ನಿನಲ್ಲಿ ತೀವ್ರ ಚಟುವಟಿಕೆಯಲ್ಲಿ ಮಗ್ನರಾಗಿದ್ದರೂ ಭಾರತದಲ್ಲಿನ ತಮ್ಮ ಕೆಲಸಕಾರ್ಯಗಳನ್ನು ಅವರು ಮರೆತಿರಲಿಲ್ಲ. ತಮ್ಮ ಪ್ರತಿಯೊಬ್ಬ ಗುರುಭಾಯಿಯ ಬಗ್ಗೆಯೂ ಅವರು ವಿವರವಾದ ಮಾಹಿತಿಯನ್ನು ಪಡೆದುಕೊಳ್ಳುತ್ತಿದ್ದರು. ಇವರಲ್ಲಿ ಯಾರ್ಯಾರು ಏನೇನು ಮಾಡುತ್ತಿದ್ದಾರೆಂಬುದನ್ನು ಗಮನಿಸಿ ಅವರಿಗೆಲ್ಲ ವೈಯಕ್ತಿಕ ಸಲಹೆ -ಸೂಚನೆಗಳನ್ನು ನೀಡುತ್ತಿದ್ದರು. ಕೆಲಸ ಚೆನ್ನಾಗಿ ನಡೆಯುತ್ತಿದೆಯೆಂದು ಕಂಡುಬಂದರೆ ಪ್ರಶಂಸೆಯನ್ನೂ, ಪ್ರೋತ್ಸಾಹವನ್ನೂ ಧಾರಾಳವಾಗಿ ಕೊಡುತ್ತಿದ್ದರು. ಮತ್ತು ಜೊತೆಗೇ, ಅಗತ್ಯವಿದ್ದರೆ, ಆರ್ಥಿಕ ನೆರವನ್ನೂ, ನೀಡುತ್ತಿದ್ದರು. ಎಲ್ಲಕ್ಕಿತ ಹೆಚ್ಚಾಗಿ, ಮಠದ ಜೀವನ ಸುವ್ಯವಸ್ಥಿತವಾಗಿ ನಡೆಯು ತ್ತಿದೆಯೆ, ತಮ್ಮ ಅತ್ಯುನ್ನತ ಕಲ್ಪನೆಗೆ ತಕ್ಕಂತಿದೆಯೆ ಎಂಬುದು ಅವರ ಅತಿಮುಖ್ಯ ಕಾಳಜಿ. ಅವರು ಅಮೆರಿಕದಿಂದ ಲಂಡನ್ನಿಗೆ ಎರಡನೆಯ ಬಾರಿ ಬಂದಾಗ ಸ್ವಾಮಿ ಶಾರದಾನಂದರನ್ನು ಪ್ರಶ್ನಿಸಿದುದು ಮೊದಲನೆಯದಾಗಿ ಮಠದ ಬಗ್ಗೆ. ಆಲಂಬಜಾರಿನ ತಮ್ಮ ಮಠದ ಬಗ್ಗೆ ಶಾರದಾ ನಂದರು ತಿಳಿಸಿದ ವಿಷಯಗಳನ್ನು ಕೇಳಿದ ಕೂಡಲೇ ಸ್ವಾಮೀಜಿ ಅಲ್ಲಿನ ಪರಿಸ್ಥಿತಿಯನ್ನು ಗ್ರಹಿಸಿ ತಮ್ಮ ಮುಂದಿನ ಕೆಲಸವೇನೆಂದು ನಿರ್ಧರಿಸಿದರು. ತಕ್ಷಣ ಅವರು ತಮ್ಮ ಗುರುಭಾಯಿಗಳಿ ಗೊಂದು ದೀರ್ಘ ಪತ್ರ ಬರೆದು ಹಲವಾರು ವಿಚಾರಗಳ ಬಗ್ಗೆ ಕಟ್ಟುನಿಟ್ಟಾದ ಸೂಚನೆಗಳನ್ನು ನೀಡಿದರು. ಅವರ ಗ್ರಹಣ ಸಾಮರ್ಥ್ಯವೂ ದೂರದೃಷ್ಟಿಯೂ ಎಷ್ಟು ಅದ್ಭುತವಾಗಿತ್ತೆಂಬುದನ್ನು ಈ ಪತ್ರದಲ್ಲಿ ನೋಡಬಹುದು:

“....ನಿಮಗೆಲ್ಲರಿಗೂ ನಾನೀಗ ಕೆಲವು ವಿಚಾರಗಳನ್ನು ಬರೆಯುತ್ತೇನೆ. ನಿಮ್ಮೆಲ್ಲರ ಹಿತಕ್ಕಾಗಿ ಮತ್ತು ಭಗವಂತನು ಯಾವ ಉದ್ದೇಶಕ್ಕಾಗಿ ಜನ್ಮವೆತ್ತಿದನೋ ಅದನ್ನು ಪೂರೈಸುವುದಕ್ಕಾಗಿ ನಾನಿದನ್ನು ಮಾಡುತ್ತಿರುವೆನೇ ಹೊರತು ನನ್ನ ಹೆಗ್ಗಳಿಕೆಗೋಸ್ಕರವಲ್ಲ. ಶ್ರೀರಾಮಕೃಷ್ಣರು ನಿಮ್ಮನ್ನೆಲ್ಲ ನೋಡಿಕೊಳ್ಳುವ ಜವಾಬ್ದಾರಿಯನ್ನು ನನಗೊಪ್ಪಿಸಿ ಹೋಗಿದ್ದಾರೆ. ಮತ್ತು–ಇನ್ನೂ ನಿಮ್ಮಲ್ಲಿ ಹೆಚ್ಚಿನವರಿಗೆ ಈ ವಿಷಯ ತಿಳಿದಿಲ್ಲ–ನೀವೀಗ ಜಗತ್ತಿನ ಶ್ರೇಯಸ್ಸಿಗಾಗಿ ನಿಮ್ಮ ಕೊಡುಗೆಯನ್ನು ನೀಡಬೇಕಾಗಿದೆ. ನಾನೀಗ ಈ ಪತ್ರವನ್ನು ಬರೆಯುತ್ತಿರುವುದರ ಮುಖ್ಯ ಉದ್ದೇಶವೇ ಇದು. ನೀವು ನಿಮ್ಮ ಅಹಂಕಾರ ಬೆಳೆಯಲು ಅವಕಾಶ ಕೊಟ್ಟುದೇ ಆದರೆ, ನಿಮ್ಮ ನಿಮ್ಮಲ್ಲೇ ಅಸೂಯೆ ತಲೆಯೆತ್ತಲು ಆಸ್ಪದ ಕೊಟ್ಟುದೇ ಆದರೆ, ಅದೊಂದು ಅತ್ಯಂತ ಶೋಚ ನೀಯ ಅಂಶವೇ ಸರಿ. ತಮ್ಮತಮ್ಮಲ್ಲೇ ಅನ್ಯೋನ್ಯತೆಯಿಂದ ಜೀವಿಸಲು ಸಾಧ್ಯವಿಲ್ಲದವರಿಗೆ, ಇಡೀ ಜಗತ್ತಿನಲ್ಲಿ ಸ್ನೇಹ ಸಂಬಂಧವನ್ನು ಬೆಳೆಸಿ ನೆಲೆಗೊಳಿಸಲು ಎಂದಿಗಾದರೂ ಸಾಧ್ಯವೆ? ನಿಯಮಗಳಿಂದ ಬಂಧಿತರಾಗುವುದೇನೋ ಕೆಟ್ಟದ್ದೇ ನಿಜ; ಆದರೆ ಇನ್ನೂ ಅಪ್ರಬುದ್ಧ ಸ್ಥಿತಿಯಲ್ಲಿರುವವರಿಗೆ ನಿಯಮ ಕಟ್ಟಳೆಗಳ ಆವಶ್ಯಕತೆಯಿದೆ. ಗುರುಮಹಾರಾಜರು ಹೇಳುತ್ತಿದ್ದ ರಲ್ಲ–ಎಳೆಯ ಸಸಿಗೆ ಬೇಲಿ ಬೇಕು ಎಂದು–ಹಾಗೆ. ಎರಡನೆಯದಾಗಿ, ಬುದ್ಧಿಗೆ ಕೆಲಸವಿಲ್ಲದ ವರು ಗುಸುಗುಸು ಮಾತನಾಡುವುದು, ಗುಂಪುಗಾರಿಕೆ ಮಾಡುವುದು–ಇದೆಲ್ಲ ತೀರ ಸಹಜ. ಆದ್ದರಿಂದ ನಾನು ನಿಮಗೆಲ್ಲ ಕೆಲವು ಸಲಹೆಗಳನ್ನು ಕೊಡುತ್ತೇನೆ. ನೀವು ಅವುಗಳನ್ನು ಅನುಸರಿಸಿ ನಡೆದರೆ ನಿಶ್ಚಯವಾಗಿಯೂ ಲಾಭ ಹೊಂದುತ್ತೀರಿ. ಇಲ್ಲದೆ ಹೋದರೆ, ನಮ್ಮ ಕೆಲಸಗಳೆಲ್ಲವೂ ಬಿದ್ದುಹೋಗುವ ಅಪಾಯವಿದೆ.” ಬಳಿಕ ಸ್ವಾಮೀಜಿ, ಮಠದ ಆಡಳಿತಕ್ಕೆ ಸಂಬಂಧಪಟ್ಟಂತೆ ಹತ್ತು ನಿಯಮಗಳನ್ನು ಸೂಚಿಸಿದರು. ಅನಂತರ ಸರ್ವರೂ ಅನುಸರಿಸಬೇಕಾದ ಮೂರು ಮುಖ್ಯ ನಿಯಮಗಳನ್ನು ಸೂಚಿಸಿದರು–(೧) ಶಾಸ್ತ್ರಾಧ್ಯಯನ, (೨) ಪ್ರವಚನಾದಿಗಳ ಮೂಲಕ ಧರ್ಮ ಪ್ರಸಾರ ಕಾರ್ಯ, (೩) ಆಧ್ಯಾತ್ಮಿಕ ಸಾಧನೆ.

ಇದಲ್ಲದೆ ಅವರು ಮಠಕ್ಕೆ ಒಂದು ಸಂವಿಧಾನವನ್ನು ರೂಪಿಸಿದ್ದರು. ಸಂಘದಲ್ಲಿ ಒಬ್ಬೊ ಬ್ಬರು ಅಧ್ಯಕ್ಷರು, ಕಾರ್ಯದರ್ಶಿ, ಖಜಾಂಚಿಗಳಾಗಿ ಕಾರ್ಯ ನಿರ್ವಹಿಸಬೇಕು, ಹಾಗೂ ಇವರ ನ್ನೆಲ್ಲ ಪ್ರತಿವರ್ಷವೂ ಎಲ್ಲರೂ ಸೇರಿ ಆಯ್ಕೆ ಮಾಡಬೇಕು ಎಂದು ಸ್ವಾಮೀಜಿ ಸೂಚಿಸಿದರು. ಆದರೆ ಆ ವರ್ಷದ ಮಟ್ಟಿಗೆ ಸ್ವಾಮಿ ಬ್ರಹ್ಮಾನಂದರು ಅಧ್ಯಕ್ಷರಾಗಿ ಹೊಣೆ ವಹಿಸಿಕೊಳ್ಳ ಬೇಕೆಂದೂ ನಿರ್ದಿಷ್ಟವಾಗಿ ಸೂಚಿಸಿದರು. ಇವುಗಳೊಂದಿಗೆ, ಶ್ರೀಮಾತೆಯವರ ಹೆಸರಿನಲ್ಲಿ ಒಂದು ಸ್ತ್ರೀಮಠವನ್ನು ಸ್ಥಾಪಿಸುವುದರ ಬಗ್ಗೆಯೂ ಸೂಚನೆ ಕೊಟ್ಟು ಆ ಬಗ್ಗೆ ಸ್ವಲ್ಪ ಕಠಿಣ ವಾಗಿಯೇ ಕೆಲವು ಮಾತುಗಳನ್ನು ಬರೆದರು:

“ನನ್ನ ಭಾವನೆಗಳಿಗನುಸಾರವಾಗಿ ನಡೆದುಕೊಳ್ಳುವುದು ಒಳ್ಳೆಯದೆಂದು ನಿಮಗನ್ನಿಸುವುದಾ ದರೆ, ಮತ್ತು ಆ ನಿಯಮಗಳನ್ನು ನೀವು ಅನುಸರಿಸುವಿರಾದರೆ, ನಾನು ನಿಮಗೆ ಎಲ್ಲ ಹಣವನ್ನೂ ಒದಗಿಸಿಕೊಡುತ್ತೇನೆ... ಈ ಪತ್ರವನ್ನು ಗೌರಿ ಮಾ, ಯೋಗಿನ್ ಮಾ ಮತ್ತಿತರಿರಿಗೂ ತೋರಿಸಿ. ಒಂದು ವರ್ಷದ ಮಟ್ಟಿಗೆ ಗೌರೀ ಮಾ ಅದರ ಅಧಕ್ಷರಾಗಲಿ. ಆದರೆ ನಿಮ್ಮಲ್ಲಿ ಯಾರೂ ಅಲ್ಲಿಗೆ ಹೋಗುವಂತಿಲ್ಲ. ಅವರು ತಮ್ಮ ವ್ಯವಹಾರವನ್ನು ತಾವೇ ನೋಡಿಕೊಳ್ಳುತ್ತಾರೆ. ಅವರು ನಿಮ್ಮಗಳ ಅಣತಿಯಂತೇನೂ ಕೆಲಸಮಾಡಬೇಕಾಗಿಲ್ಲ. ನಾನು ಆ ಕೆಲಸದ ಖರ್ಚು ವೆಚ್ಚಕ್ಕೂ ಹಣವನ್ನು ಒದಗಿಸಿಕೊಡುತ್ತೇನೆ...”

ಕೆಲವೊಮ್ಮೆ ಸ್ವಾಮೀಜಿ ಕಟುವಾದ ಮಾತುಗಳಿಂದ ತಮ್ಮ ಸೋದರ ಸಂನ್ಯಾಸಿಗಳನ್ನು ಚುಚ್ಚಿ ಅವರನ್ನು ಕಾರ್ಯೋನ್ನುಖರಾಗಿಸುತ್ತಿದ್ದರಾದರೂ ಇನ್ನು ಕೆಲವೊಮ್ಮೆ ಅವರ ಸಾಧನೆಗಳನ್ನೂ ಅಂತಸ್ಸತ್ವವನ್ನೂ ಅವರಿಗೇ ನೆನಪಿಸಿಕೊಟ್ಟು ಹೃತ್ಪೂರ್ವಕವಾಗಿ ಪ್ರಶಂಸಿಸುತ್ತಿದ್ದರು. ಇವರ ಲ್ಲದೆ, ತಮ್ಮ ಮದ್ರಾಸೀ ಶಿಷ್ಯರಿಗೂ ಪತ್ರಗಳನ್ನು ಬರೆದು ಹುರಿದುಂಬಿಸುತ್ತಿದ್ದರು. ಸ್ವಾವಲಂಬಿ ಗಳಾಗುವಂತೆ, ತಮ್ಮಷ್ಟೇ ಉತ್ಸಾಹಭರಿತರಾಗುವಂತೆ, ಭಾರತದಾದ್ಯಂತ ಹೊಸ ಬೆಳಕನ್ನು ಹರಡುವಂತೆ ಅವರಿಗೆ ಕರೆ ನೀಡಿದರು. ಅಳಸಿಂಗರಿಗೆ ಒಂದು ಪತ್ರದಲ್ಲಿ ಅವರು, ಅಮೆರಿಕದ ತಮ್ಮ ಕಾರ್ಯದ ಬಗ್ಗೆ ಪ್ರಸ್ತಾಪಿಸುತ್ತ ಬರೆದರು, “ಶಕ್ತಿಶಾಲಿಗಳಾದ ಅಪ್ಪಟ ವ್ಯಕ್ತಿಗಳು ಸಾಕಷ್ಟು ಸಂಖ್ಯೆಯಲ್ಲಿ ದೊರೆತರೆ, ಹತ್ತು ವರ್ಷಗಳಲ್ಲಿ ಅರ್ಧ ಅಮೆರಿಕವನ್ನೇ ಜಯಿಸಬಹುದು. ನನ್ನ ಸುತನೆ, ನಾನು ನಿರೀಕ್ಷಿಸುವುದೇನೆಂದರೆ, ಕಬ್ಬಿಣದ ಮಾಂಸಖಂಡಗಳು, ಉಕ್ಕಿನ ನರಗಳು ಹಾಗೂ ಅದರೊಳಗೆ ಸಿಡಿಲಿನಂಥದೇ ವಸ್ತುವಿನಿಂದ ಮಾಡಲ್ಪಟ್ಟ ಮೆದುಳನ್ನು ಹೊಂದಿದ ವ್ಯಕ್ತಿಗಳನ್ನು. ಶಕ್ತಿ, ಪುರುಷತ್ವ, ಕ್ಷಾತ್ರವೀರ್ಯ+ಬ್ರಹ್ಮತೇಜ... ಭಾರತದಲ್ಲಿ ಉಜ್ವಲ ಸಾಧ್ಯತೆಗಳಡಗಿರುವ ಅತ್ಯುತ್ತಮ ವ್ಯಕ್ತಿಗಳಿದ್ದಾರೆ. ಆದರೆ ಈ ವಿವಾಹವೆಂದು ಕರೆಯಲ್ಪಡುವ ಕ್ರೂರ ಬಲಿವೇದಿಕೆಯ ಮೇಲೆ ಅವರೆಲ್ಲ ಬಲಿಯಾಗುತ್ತಿದ್ದಾರೆ... ಕಡೆಯ ಪಕ್ಷ ಒಂದು ನೂರು ಮದ್ರಾಸೀ ಯುವಕ ರಾದರೂ ಈ ಜಗತ್ತಿನಿಂದ ದೂರ ನಿಂತು, ಟೊಂಕ ಕಟ್ಟಿ, ಸತ್ಯದ ಪರವಾದ ಯುದ್ಧವನ್ನು ಹೂಡಿ ದೇಶದಿಂದ ದೇಶಕ್ಕೆ ಸಾಗಬೇಕೆಂದು ನಾನು ಬಯಸುತ್ತೇನೆ.” ಮತ್ತೊಂದು ಪತ್ರದಲ್ಲಿ ತಮ್ಮ ಮದ್ರಾಸೀ ಶಿಷ್ಯರಿಗೆ ಬರೆದರು, “ಹೆದರದಿರಿ. ನಿಮ್ಮಿಂದ ಮಹತ್ಕಾರ್ಯಗಳು ಸಾಧ್ಯವಾಗಲಿವೆ. ಧೈರ್ಯತಾಳಿ... ಇದೇ ಚಳಿಗಾಲದಲ್ಲಿ ನಾನು ಭಾರತಕ್ಕೆ ಹಿಂದಿರುಗಿ ಬಂದು ಅಲ್ಲಿನ ಕೆಲಸಗಳ ನ್ನೆಲ್ಲ ವ್ಯವಸ್ಥೆಗೊಳಿಸುತ್ತೇನೆ. ಧೀರಹೃದಯಿಗಳೇ, ಕಾರ್ಯನಿರತರಾಗಿ. ಇಲ್ಲ ಎಂದೆನ್ನದಿರಿ. ಕಾರ್ಯನಿರತರಾಗಿ. ಭಗವಂತನೇ ನಿಮ್ಮ ಕಾರ್ಯದ ಹಿಂದಿದ್ದಾನೆ. ಮಹಾಶಕ್ತಿಯೇ ನಿಮ್ಮೊಡನಿದ್ದಾಳೆ.”

ಈ ಅವಧಿಯಲ್ಲಿ ಸ್ವಾಮೀಜಿ ಭಾರತದಲ್ಲಿ ಕಾರ್ಯಗತಗೊಳಿಸಿದ ಒಂದು ಮುಖ್ಯ ಯೋಜನೆ ಯೆಂದರೆ ‘ಪ್ರಬುದ್ಧ ಭಾರತ’ ಎಂಬ ಆಂಗ್ಲ ಮಾಸಪತ್ರಿಕೆಯ ಪ್ರಾರಂಭ. ಅದಾಗಲೇ ‘ಬ್ರಹ್ಮವಾದಿನ್​’ ಪಾಕ್ಷಿಕ ಪತ್ರಿಕೆಯ ಮೂಲಕ ಅವರ ಮದ್ರಾಸೀ ಶಿಷ್ಯರು ವೇದಾಂತ ಪ್ರಸಾರ ವನ್ನು ಪ್ರಾರಂಭಿಸಿದರು. ಈ ‘ಬ್ರಹ್ಮವಾದಿನ್​’ ಪತ್ರಿಕೆಯು ಕೇವಲ ಧಾರ್ಮಿಕ ಹಾಗೂ ತಾತ್ವಿಕ ವಿಚಾರಗಳಿಗೆ ಮೀಸಲಾಗಿತ್ತು. ಆದರೆ ಇವುಗಳಲ್ಲಿ ಹೆಚ್ಚಿನ ವಿಚಾರಗಳು ಜನಸಾಮಾನ್ಯರಿಗೆ ಎಟುಕುವಂತಿರಲಿಲ್ಲ. ಏಕೆಂದರೆ ಈ ಪತ್ರಿಕೆಯು ಪ್ರೌಢವೂ ವಿದ್ವತ್ಪೂರ್ಣವೂ ಆಗಿರಬೇಕು ಎಂಬುದೇ ಸ್ವಾಮೀಜಿಯ ಆದೇಶವಾಗಿತ್ತು. ಆದ್ದರಿಂದ, ಉನ್ನತ ವಿಚಾರಗಳನ್ನು ಜನ ಸಾಮಾನ್ಯರೂ ಗ್ರಹಿಸುವ ರೀತಿಯಲ್ಲಿ ತಿಳಿಸಿಕೊಡಲು ‘ಪ್ರಬುದ್ಧ ಭಾರತ’ ಎಂಬ ಆಂಗ್ಲ ಮಾಸ ಪತ್ರಿಕೆಯನ್ನು ಪ್ರಾರಂಭಿಸಲಾಯಿತು. ಈ ಕಾರ್ಯದ ಹೊಣೆಯನ್ನು ವಹಿಸಿಕೊಂಡವರು ಬೆಂಗಳೂರಿನವರಾದ ಡಾ ॥ ನಂಜುಂಡರಾವ್. ಶ್ರೀ ರಾಜಂ ಅಯ್ಯರರು ಪತ್ರಿಕೆಯ ಸಂಪಾದಕ ರಾದರು. ‘ಪ್ರಬುದ್ಧ ಭಾರತ’ದ ಹಿಂದಿದ್ದ ಕಾರ್ಯಕರ್ತರಿಗೆ ಸ್ವಾಮೀಜಿ ತಮ್ಮ ಹೃತ್ಪೂರ್ವಕ ಆಶೀರ್ವಾದಗಳನ್ನು, ಪ್ರೋತ್ಸಾಹವನ್ನು ಕಳಿಸಿದರು. ಡಾ ॥ ನಂಜುಂಡರಾವ್​ಗೆ ಬರೆದ ಪತ್ರದಲ್ಲಿ ಅವರು ಪತ್ರಿಕೆಯ ಬಗ್ಗೆ ಸೂಚನೆಗಳನ್ನು ಕೊಟ್ಟು, ಕಡೆಯಲ್ಲಿ ಹೀಗೆ ಬರೆದರು–“ಧೈರ್ಯವಾಗಿ ಮುಂದೆ ಸಾಗು. ಒಂದು ದಿನದಲ್ಲಿ ಅಥವಾ ಒಂದು ವರ್ಷದಲ್ಲಿ ಯಶಸ್ಸು ಬರುವುದೆಂದು ನಿರೀಕ್ಷಿಸಬೇಡ. ಯಾವಾಗಲೂ ಅತ್ಯಂತ ಶ್ರೇಷ್ಠವಾದ್ದನ್ನೇ ಅವಲಂಬಿಸು. ನಿಶ್ಚಲನಾಗಿರು. ಅಸೂಯೆಯನ್ನೂ ಸ್ವಾರ್ಥವನ್ನೂ ದೂರವಿಟ್ಟಿರು. ಸತ್ಯದ, ಮಾವನತೆಯ ಹಾಗೂ ನಿನ್ನ ರಾಷ್ಟ್ರದ ಸಲುವಾದ ಕಾರ್ಯಕ್ಕೆ ಎಂದೆಂದಿಗೂ ವಿಧೇಯನಾಗಿರು, ನಿಷ್ಠಾವಂತನಾಗಿರು. ಆಗ ನೀನು ಇಡೀ ಜಗತ್ತನ್ನೇ ಅಲುಗಾಡಿಸಬಲ್ಲೆ. ನೆನಪಿಟ್ಟುಕೊ–ಶಕ್ತಿಯ ಹಿಂದಿರುವ ರಹಸ್ಯವೆಂದರೆ ವ್ಯಕ್ತಿಯ ಶೀಲ, ಅವನ ಜೀವನ–ಮತ್ತೇನೂ ಅಲ್ಲ. ಈ ಪತ್ರವನ್ನು ಉಳಿಸಿಕೊ, ಮತ್ತು ಯಾವಾಗ ಲಾದರೂ ನಿನ್ನನ್ನು ಚಿಂತೆ ಆವರಿಸಿದರೆ ಇಲ್ಲವೆ ಮತ್ಸರವುಂಟಾದರೆ ಈ ಪತ್ರದ ಕಡೆಯ ಸಾಲು ಗಳನ್ನು ಓದು. ಮಾತ್ಸರ್ಯವೇ ಗುಲಾಮರೆಲ್ಲರ ಬೇನೆ. ಅದೇ ನಮ್ಮ ರಾಷ್ಟ್ರಕ್ಕಂಟಿದ ಜಾಡ್ಯ. ಅದನ್ನು ಎಂದಿಗೂ ದೂರವಿಡು. ಸಕಲ ಯಶಸ್ಸೂ ನಿನ್ನದಾಗಲೆಂದು ಹಾರೈಸುತ್ತೇನೆ.”

‘ಪ್ರಬುದ್ಧ ಭಾರತ’ವು ಮದರಾಸಿನಲ್ಲಿ ಪ್ರಾರಂಭಗೊಂಡಿತು. ಇದರ ಪ್ರಥಮ ಸಂಚಿಕೆಗಳನ್ನು ನೋಡಿ ಸ್ವಾಮೀಜಿ ಸಂತಸವನ್ನು ವ್ಯಕ್ತಪಡಿಸಿದರಲ್ಲದೆ, ಅದರ ಸುಧಾರಣೆಗೆ ಅನೇಕ ಸಲಹೆಗಳನ್ನು ನೀಡಿದರು. ಅಲ್ಲದೆ, ಇದೇ ಪತ್ರಿಕೆಯ ರೀತಿಯಲ್ಲಿ, ಕನ್ನಡ, ತೆಲುಗು, ತಮಿಳು ಮೊದಲಾದ ಪ್ರಾದೇಶಿಕ ಭಾಷೆಗಳಲ್ಲೂ ಪತ್ರಿಕೆಗಳನ್ನು ಹೊರಡಿಸುವಂತೆ ಅವರು ತಮ್ಮ ಶಿಷ್ಯರಿಗೆ ಕರೆನೀಡಿ ದರು. ‘ಪ್ರಬುದ್ಧ ಭಾರತ’ವು ಇಂದಿಗೂ ಶ್ರೀರಾಮಕೃಷ್ಣ ಸಂಘದ ಅತಿ ಮುಖ್ಯ ಪತ್ರಿಕೆಯಾಗಿದೆ.

ಇತ್ತ ಅಮೆರಿಕ-ಇಂಗ್ಲೆಂಡುಗಳಲ್ಲಿನ ತಮ್ಮ ಕಾರ್ಯವು ತಮ್ಮ ನಿರ್ಗಮನದ ಅನಂತರವೂ ಅಬಾಧಿತವಾಗಿ ಮುಂದುವರಿಯಬೇಕೆಂದು ಕಾತರರಾಗಿದ್ದ ಸ್ವಾಮೀಜಿಗೆ, ಅಮೆರಿಕದಲ್ಲಿ ಶಾರದಾ ನಂದರ ಮತ್ತು ಮಿಸ್ ಎಲೆನ್ ವಾಲ್ಡೊಳ ಯಶಸ್ಸು ಹಾಗೂ ಲಂಡನ್ನಿನಲ್ಲಿ ಅಭೇದಾನಂದರ ಯಶಸ್ಸು ತುಂಬ ಸಂತೋಷ-ಸಮಾಧಾನಗಳನ್ನುಂಟುಮಾಡಿದುವು.

ಲಂಡನ್ನಿನಲ್ಲಿ ಒಂದು ತಿಂಗಳಿನಿಂದಲೂ ಸ್ವಾಮೀಜಿಯೊಂದಿಗಿದ್ದ ಅಭೇದಾನಂದರು, ತಮ್ಮ ಹಿರಿಯ ಗುರುಭಾಯಿಯ ಕಾರ್ಯವಿಧಾನವನ್ನು ಗಮನಿಸುತ್ತ ಅದನ್ನು ತಾವು ಮುಂದುವರಿಸಲು ಬೇಕಾದ ಸಿದ್ಧತೆಯನ್ನು ಮಾಡಿಕೊಳ್ಳುತ್ತಿದ್ದರು. ಆದರೆ ಅವರ ಹಿಂಜರಿಕೆಯಿನ್ನೂ ದೂರವಾಗಿರ ಲಿಲ್ಲ. ಇದನ್ನು ಸರಿಪಡಿಸಲು ಸ್ವಾಮೀಜಿ ಒಂದು ಉಪಾಯ ಹೂಡಿದರು. ಒಮ್ಮೆ ಅವರು ಲಂಡನ್ನಿನ ‘ಕ್ರೈಸ್ಟ್ ಥಿಯೊಸಾಫಿಕಲ್ ಸೊಸೈಟಿ’ಯಲ್ಲಿ ಭಾಷಣ ಮಾಡುವ ಕಾರ್ಯಕ್ರಮವಿತ್ತು. ಆದರೆ ಆ ಕಾರ್ಯಕ್ರಮಕ್ಕೆ ಕೆಲವೇ ದಿನಗಳಿರುವಾಗ ಅವರು, ಅನಾರೋಗ್ಯದ ಕಾರಣದಿಂದ ತಮಗೆ ಆ ಕಾರ್ಯಕ್ರಮವನ್ನು ನಡೆಸಲು ಸಾಧ್ಯವಿಲ್ಲವೆಂದು ಹೇಳಿ, ತಮ್ಮ ಬದಲಿಗೆ ತಮ್ಮ ಸೋದರ ಸಂನ್ಯಾಸಿಗಳಾದ ಸ್ವಾಮಿ ಅಭೇದಾನಂದರು ಮಾತನಾಡುತ್ತಾರೆಂದು ಸಂಚಾಲಕರಿಗೆ ತಿಳಿಸಿದರು. ಅದರಂತೆ ಕಾರ್ಯಕ್ರಮದ ಪಟ್ಟಿಯಲ್ಲಿ ಅಭೇದಾನಂದರ ಹೆಸರನ್ನು ಸೇರಿಸಿ ಕರಪತ್ರಗಳನ್ನೂ ಪ್ರಕಟಿಸಲಾಯಿತು. ಆದರೆ ಅಭೇದಾನಂದರಿಗೆ ಈ ವಿಷಯಗಳೊಂದೂ ತಿಳಿದಿರಲೇ ಇಲ್ಲ!

ಕಡೆಗೊಂದು ದಿನ ಸ್ವಾಮೀಜಿಯವರು ಅಭೇದಾನಂದರಿಗೆ ಒಂದು ಕರಪತ್ರವನ್ನು ಕೊಡುತ್ತ, “ನೋಡು, ಕ್ರೈಸ್ಟ್ ಥಿಯೊಸಾಫಿಕಲ್ ಸೊಸೈಟಿಯಲ್ಲಿ ನೀನೊಂದು ಭಾಷಣ ಮಾಡಬೇಕಾಗಿದೆ” ಎಂದರು. ಅದನ್ನು ಕೇಳಿ ಅಭೇದಾನಂದರು ಆಶ್ಚರ್ಯಗೊಂಡು, “ಭಾಷಣ ಮಾಡಬೇಕಾದವನು ನೀನಲ್ಲವೆ!” ಎಂದರು. ಆಗ ಸ್ವಾಮೀಜಿ, “ಇಲ್ಲ; ನನಗೆ ತುಂಬ ಆಯಾಸವಾಗಿದೆ. ಆದ್ದರಿಂದ ನಾನು ನನ್ನ ಬದಲಾಗಿ ಭಾಷಣ ಮಾಡಲು ನಿನ್ನ ಹೆಸರನ್ನು ಸೂಚಿಸಿದ್ದೇನೆ. ನೀನು ಮುಖ್ಯ ಉಪನ್ಯಾಸಕ ಎಂದು ಆಗಲೇ ಅಚ್ಚಾಗಿಬಿಟ್ಟಿದೆ” ಎಂದರು. ಅಭೇದಾನಂದರು ದಿಗ್ಭ್ರಾಂತರಾಗಿ “ಅದರೆ, ಇದೇನಿದು! ನರೇನ್, ನಾನೆಂದೂ ಭಾಷಣ ಮಾಡಿದವನೇ ಅಲ್ಲ. ಈಗ ಅಲ್ಲಿ ಮಾತ ನಾಡಲು ನನಗೆ ಸಾಧ್ಯವೇ ಇಲ್ಲ!” ಎಂದರು. ಇವರು ಹೀಗೆ ಹೇಳಿಯಾರು ಎಂದು ಸ್ವಾಮೀಜಿಗೆ ಮೊದಲೇ ಗೊತ್ತಿತ್ತು. ಆದ್ದರಿಂದ ಅವರು ಶಾಂತವಾಗಿ ಉತ್ತರಿಸಿದರು. “ನಾನು ಸರ್ವಧರ್ಮ ಸಮ್ಮೇಳನದ ವೇದಿಕೆಯ ಮೇಲೆ ನಿಂತಾಗ ಭಾಷಣ ಮಾಡುವುದರ ಬಗ್ಗೆ ನನಗಾದರೂ ಏನು ಗೊತ್ತಿತ್ತು? ನಾನು ಏನ್ನನ್ನು ಸಾಧಿಸಿದೆನೋ ಅದೆಲ್ಲ ಶ್ರೀರಾಮಕೃಷ್ಣರ ಕೃಪೆಯಿಂದಷ್ಟೆ. ನೀನವರಲ್ಲಿ ಶ್ರದ್ಧೆಯಿಡು; ನೀನೊಬ್ಬ ಅತ್ಯುತ್ತಮ ವಾಗ್ಮಿಯಾಗಿ ಬೆಳಗುತ್ತೀಯೆ.” ಸ್ವಾಮೀಜಿ ಹೀಗೆ ಹೇಳಿಬಿಟ್ಟಾಗ, ಅಭೇದಾನಂದರಿಗೆ ಇನ್ನೇನು ಹೇಳಲೂ ಅವಕಾಶವೇ ಇರಲಿಲ್ಲ. ವಿಧಿ ಇಲ್ಲದೆ, ಉಪನ್ಯಾಸ ಮಾಡಲು ಒಪ್ಪಿಕೊಂಡರು.

ಕೊನೆಗೂ ಆ ದಿನ ಬಂದಿತು. ಸುಶಿಕ್ಷಿತ ಶ್ರೋತೃಗಳಿಂದ ಸಭೆ ಕಿಕ್ಕಿರಿದಿತ್ತು. ಸ್ವಾಮೀಜಿಯೂ ವೇದಿಕೆಯ ಮೇಲೆ ಉಪಸ್ಥಿತರಿದ್ದರು. ಬಹಳಷ್ಟು ಜನರು ವಿವೇಕಾನಂದರು ಭಾಷಣ ಮಾಡುತ್ತಾ ರೆಂದೇ ಭಾವಿಸಿ ಬಂದಿದ್ದವರು. ಆದ್ದರಿಂದ ಅವರ ಬದಲಾಗಿ ಮತ್ತೊಬ್ಬರು ಹೊಸ ಸ್ವಾಮಿಗಳು ಮಾತನಾಡಲಿದ್ದಾರೆಂದು ಗೊತ್ತಾದಾಗ ಜನರಿಗೆ ಸ್ವಲ್ಪ ನಿರಾಶೆಯಾಯಿತು. ಆದರೆ ಅಭೇದಾ ನಂದರ ಉಪನ್ಯಾಸವು ಅತ್ಯಂತ ಯಶಸ್ವಿಯಾಯಿತು. ಅಂದು ಅವರು ‘ಪಂಚದಶೀ’ ಎಂಬ ಗ್ರಂಥದಲ್ಲಿ ಪ್ರತಿಪಾದಿತವಾದ ತತ್ತ್ವ ಹಾಗೂ ಅದರ ಅನುಷ್ಠಾನದ ಕುರಿತಾಗಿ ಅತ್ಯುತ್ತಮವಾಗಿ ಮಾತನಾಡಿದರು. ಶ್ರೋತೃಗಳು ಸಂತಸಗೊಂಡು ತಮ್ಮ ಹಾರ್ದಿಕ ಮೆಚ್ಚುಗೆಯನ್ನು ಸೂಚಿಸಿ ದರು. ಸ್ವಾಮೀಜಿಗಂತೂ ಹಿಡಿಸಲಾರದಷ್ಟು ಆನಂದವಾಗಿದ್ದುದು ಸ್ಪಷ್ಟವಾಗಿ ವ್ಯಕ್ತವಾಗುತ್ತಿತ್ತು. ಅದೊಂದು ಅಪೂರ್ವ ದೃಶ್ಯ. ಅದರ ಸ್ವಾರಸ್ಯವನ್ನು ಸಭಿಕರು ಸಂಪೂರ್ಣವಾಗಿ ಅಸ್ವಾದಿಸುತ್ತಿ ದ್ದರು. ಅತ್ಯಂತ ಬುದ್ಧಿವಂತನಾದ ತನ್ನ ಪ್ರಿಯ ಶಿಷ್ಯನೊಬ್ಬನ ಯಶಸ್ಸನ್ನು ಕಂಡು ಗುರುವು ಹಿಗ್ಗುವಂತೆ ತಮ್ಮ ಸೋದರನ ಸಾಧನೆಯನ್ನು ಕಂಡು ಸ್ವಾಮೀಜಿ ಹಿಗ್ಗಿದರು. ಕಡೆಯಲ್ಲಿ ಅವರು ಎದ್ದುನಿಂತು, “ಒಬ್ಬ ಆಂಗ್ಲ ಉಪನ್ಯಾಸಕನಾಗಿ–ಅದರಲ್ಲೂ ಆಂಗ್ಲ ಶ್ರೋತೃಗಳ ಮುಂದೆ– ಇದು ನನ್ನ ಪ್ರಿಯ ಸೋದರನ ಪ್ರಥಮ ಅನುಭವ... ” ಎಂದು ಹೇಳಿದರು. ತಕ್ಷಣ ಶ್ರೋತೃ ಗಳು ಕರತಾಡನದ ಮೂಲಕ ತಮ್ಮ ಅಭಿನಂದನೆಯನ್ನು ಸಲ್ಲಿಸಿದಾಗ ಅದು ತಮಗೇ ಸಂದದ್ದೋ ಎಂಬಂತೆ ಸ್ವಾಮೀಜಿ ಹರ್ಷಿಸಿದರು. ಈ ಇಡೀ ಸನ್ನಿವೇಶದಲ್ಲಿ ಸುವ್ಯಕ್ತವಾದ ಸ್ವಾಮೀಜಿಯ ಸ್ವಾರ್ಥರಹಿತತೆ ಅಲ್ಲಿದ್ದವರೆಲ್ಲರ ಮನಃಪಟಲದ ಮೇಲೆ ಚಿರಮುದ್ರಿತವಾಯಿತು.

ಇದೇ ಸಮಯದಲ್ಲಿ ಸ್ವಾಮೀಜಿಗೆ ಸಂತೋಷವುಂಟುಮಾಡುವ ಮತ್ತೊಂದು ಸಮಾಚಾರ ಅಮೆರಿಕದಿಂದ ಬರುತ್ತಿತ್ತು–ಅದು ಸ್ವಾಮಿ ಶಾರದಾನಂದರ ಉಪನ್ಯಾಸಗಳೂ ಅತ್ಯಂತ ಯಶಸ್ವಿ ಯಾಗಿ ನಡೆದುಕೊಂಡು ಬರುತ್ತಿವೆ, ಹಾಗೂ ಅಲ್ಲಿ ವೇದಾಂತದ ಪ್ರಭಾವವು ನಿರಂತರವಾಗಿ ವ್ಯಾಪಿಸುತ್ತಿದೆ ಎಂಬುದು. ಇದು ವೃತ್ತಪತ್ರಿಕೆಗಳ ತುಣುಕುಗಳ ಮೂಲಕ ಅವರಿಗೆ ತಿಳಿದುಬರು ತ್ತಿತ್ತು. ಅಲ್ಲದೆ ತಮ್ಮ ನಿರ್ಗಮನದಿಂದ ಅಲ್ಲಿ ತೆರವಾಗಿದ್ದ ಸ್ಥಾನವನ್ನು ಶಾರದಾನಂದರು ಸಮರ್ಥವಾಗಿ ತುಂಬಿದ್ದಾರೆಂದು ಅವರಿಗೆ ತಮ್ಮ ಶಿಷ್ಯರ ಪತ್ರಗಳ ಮೂಲಕ ಗೊತ್ತಾಗುತ್ತಿತ್ತು. ತಮ್ಮ ನಿರೀಕ್ಷೆಗೆ ತಕ್ಕಂತಹ ಮುಂದಾಳುವೊಬ್ಬರು ದೊರೆತದ್ದರಿಂದ ಅವರಿಗೆಲ್ಲ ತುಂಬ ಸಮಾಧಾನವಾಗಿತ್ತು. ಅಲ್ಲದೆ, ಶಾರದಾನಂದರ ಶಾಂತ-ಮಧುರ ವ್ಯಕ್ತಿತ್ವವನ್ನು ಅಲ್ಲಿನ ವಿದ್ಯಾರ್ಥಿ ಗಳೆಲ್ಲ ಬಹುವಾಗಿ ಮೆಚ್ಚಿದ್ದರು. ಹಿಂದೂಧರ್ಮದ ಬಗೆಗಿನ ಅವರ ಪ್ರೌಢ ವಿಶ್ಲೇಷಣೆ-ವಿವರಣೆ ಗಳಿಂದ ನೂರಾರು ಜನ ಆಕರ್ಷಿತರಾಗಿ ಬಂದಿದ್ದರು. ಶಾರದಾನಂದರು ಅಮೆರಿಕೆಗೆ ಬಂದ ಕೆಲ ದಿನಗಳಲ್ಲೇ, ಮಿಸ್ ಸಾರಾ ಫಾರ್ಮರ್ ನಡೆಸುತ್ತಿದ್ದ ‘ಗ್ರೀನ್ ಏಕರ್ ಸಮ್ಮರ್ ಸ್ಕೂಲ್​’ ಎಂಬ ಹೆಸರಾಂತ ಧಾರ್ಮಿಕ ಅಧ್ಯಯನ ಕೇಂದ್ರದಲ್ಲಿ ಉಪನ್ಯಾಸಕರಾಗುವಂತೆ ಅವರಿಗೆ ಆಹ್ವಾನ ಬಂದಿತು. (ಎರಡು ವರ್ಷದ ಹಿಂದೆ ವಿವೇಕಾನಂದರು ಹಲವಾರು ತರಗತಿಗಳನ್ನು ನಡೆಸಿದ್ದರು.) ೧೮೯೬ರ ನವೆಂಬರ್​ವರೆಗೂ ಶಾರದಾನಂದರು ಹಲವಾರು ತರಗತಿಗಳನ್ನೂ ಉಪನ್ಯಾಸಗಳನ್ನೂ ನಡೆಸಿ, ಬಳಿಕ ನ್ಯೂಯಾರ್ಕಿನ ವೇದಾಂತ ಸೊಸೈಟಿಯ ೧೮೯೬-೯೭ನೇ ಸಾಲಿನ ಚಟುವಟಿಕೆಗಳನ್ನು ಪ್ರಾರಂಭಿಸಿದರು.

ಆದರೆ ಅವರು ಇದೇ ವೇಳೆಗೆ, ಮಸಾಚುಸೆಟ್ಸ್ ರಾಜ್ಯದ ಕೇಂಬ್ರಿಡ್ಜಿನಲ್ಲಿ ನಡೆಯುತ್ತಿದ್ದ ‘ಕೇಂಬ್ರಿಡ್ಜ್ ಕಾನ್ಫರೆನ್ಸಸ್​’ ಎಂಬಲ್ಲಿ ಭಾಷಣಗಳನ್ನು ಮಾಡಲು ಒಪ್ಪಿ ಅಲ್ಲಿಗೆ ಹೋದದ್ದರಿಂದ, ಇತ್ತ ನ್ಯೂಯಾರ್ಕಿನ ವೇದಾಂತ ಕೇಂದ್ರದಲ್ಲಿ ತರಗತಿಗಳನ್ನು ನಡೆಸಲು ಯಾರೂ ಇಲ್ಲದಂತಾ ಯಿತು. ಆದ್ದರಿಂದ ಸ್ವಾಮೀಜಿ ತಮ್ಮ ನೆಚ್ಚಿನ ಶಿಷ್ಯೆ ಮಿಸ್ ಸಾರಾಎಲೆನ್​ಮಾಲ್ಡೊಳಿಗೆ ಬರೆದರು:

“ಪ್ರಿಯ ಮಿಸ್ ವಾಲ್ಡೊ, ನೀನೇಕೆ ಬೋಧಿಸಲು ಪ್ರಾರಂಭಿಸಬಾರದು? ಧೈರ್ಯದಿಂದ ಪ್ರಾರಂಭಿಸಿಬಿಡು; ಜಗನ್ಮಾತೆ ನಿನಗೆ ಶಕ್ತಿ ನೀಡುತ್ತಾಳೆ. ಸಾವಿರಾರು ಮುಂದಿ ನಿನ್ನ ಬಳಿಗೆ ಬರುತ್ತಾರೆ. ಮಹಾಕಾರ್ಯದಲ್ಲಿ ಮುಳುಗಿಬಿಡು. ಆ ವ್ಯಕ್ತಿಯನ್ನೋ ಈ ವ್ಯಕ್ತಿಯನ್ನೋ ಅವಲಂಬಿಸಿಕೊಂಡಿರುವುದರಿಂದ ಏನು ಪ್ರಯೋಜನ? ತಮ್ಮ ಮಕ್ಕಳು ಎಲ್ಲೆಲ್ಲಿ ಧೈರ್ಯದಿಂದ ಮುನ್ನುಗ್ಗುತ್ತಾರೆಯೋ ಅಲ್ಲೆಲ್ಲ ಅವರ ಹಿಂದೆ ಶ್ರೀರಾಮಕೃಷ್ಣರು ಇದ್ದೇ ಇರುತ್ತಾರೆ... ಈ ಬೋಧನಾ ಕಾರ್ಯವನ್ನು ಎಷ್ಟೋ ಜನ ಹಿಂದೂಗಳು ಕೈಗೆತ್ತಿಕೊಂಡು ಯಶಸ್ವಿಗಳಾಗುವುದ ಕ್ಕಿಂತ, ಅಥವಾ ನನ್ನ ಸೋದರ ಸಂನ್ಯಾಸಿಗಳಲ್ಲೇ ಒಬ್ಬರು ಮಾಡುವುದಕ್ಕಿಂತ, ಅದನ್ನು ನಿಮ್ಮ ಅಮೆರಿಕನ್ನರಲ್ಲಿ ಯಾರಾದರೊಬ್ಬರು ಮಾಡಿದರೆ, ನನಗೆ ಸಾವಿರ ಪಾಲು ಹೆಚ್ಚು ಸಂತೋಷವಾಗು ತ್ತದೆ. ‘ಮನುಷ್ಯ ಎಲ್ಲೆಲ್ಲೂ ವಿಜಯಿಯಾಗಲು ಬಯಸುತ್ತಾನೆ; ಆದರೆ ತನ್ನ ಮಕ್ಕಳ ಕೈಯಲ್ಲಿ ಸೋಲನ್ನೇ ಅಪೇಕ್ಷಿಸುತ್ತಾನೆ’ ಎಂಬುದೊಂದು ಗಾದೆ. ನಾನು ಇಂದಿನಿಂದಲೇ ನಿಮ್ಮೆಲ್ಲರೆಡೆಗೂ ಅತ್ಯಂತ ಶಕ್ತಿಶಾಲಿಯಾದ ಆಲೋಚನಾ ತರಂಗಗಳನ್ನು ಹರಿಯಿಸುತ್ತೇನೆ...”

ಸ್ವಾಮೀಜಿಯ ಆದೇಶದಂತೆ ಮಿಸ್ ವಾಲ್ಡೊ ವೇದಾಂತಕೇಂದ್ರದ ತರಗತಿಗಳನ್ನು ವ್ಯವಸ್ಥೆ ಗೊಳಿಸಿ, ನವೆಂಬರ್​-ಡಿಸೆಂಬರ್ ತಿಂಗಳುಗಳಲ್ಲಿ ಅದನ್ನು ಅತ್ಯಂತ ಯಶಸ್ವಿಯಾಗಿ ನಡೆಸಿ ಕೊಂಡು ಬಂದಳು. ಹೀಗೆ ಅವಳು ಇಂತಹ ಕಷ್ಟಕರವಾದ ಕೆಲಸವನ್ನು ಆತ್ಮವಿಶ್ವಾಸದಿಂದ ನಡೆಸಿ ವೇದಾಂತ ವಿದ್ಯಾರ್ಥಿಗಳೆಲ್ಲರ ಮೆಚ್ಚುಗೆಗೆ ಪಾತ್ರಳಾದಳು. ಈ ವಿಷಯ ತಿಳಿದಾಗ ಸ್ವಾಮೀಜಿಗಾದ ಆನಂದ ಅಪಾರ. ತಮ್ಮ ಅನುಪಸ್ಥಿತಿಯಲ್ಲಿ ಅಮೆರಿಕ-ಇಂಗ್ಲೆಂಡುಗಳೆರಡೂ ದೇಶಗಳಲ್ಲಿನ ತಮ್ಮ ಕಾರ್ಯವು ಅನಿರ್ಬಂಧಿತವಾಗಿ ಸಾಗುವುದೆಂಬ ವಿಶ್ವಾಸ ಅವರಲ್ಲಿ ಮೂಡಿತು.

ಸುಮಾರು ಇದೇ ಸಮಯಕ್ಕೆ ಅವರ ಮನಸ್ಸು ಹೆಚ್ಚುಹೆಚ್ಚಾಗಿ ಭಾರತದ ಕಡೆಗೆ ವಾಲತೊಡ ಗಿತ್ತು. ಈ ಬಗ್ಗೆ ಅವರು ಯಾವಾಗಲಾದರೊಮ್ಮೆ ಸೇವಿಯರ್ ದಂಪತಿಗಳ ಬಳಿ ಪ್ರಸ್ತಾಪಿಸು ತ್ತಿದ್ದರು. ಆದರೆ ತಮ್ಮ ಯೋಜನೆಯ ಬಗ್ಗೆ ಅವರಿನ್ನೂ ಸ್ಪಷ್ಟವಾಗಿ ಏನೂ ಹೇಳಿರಲಿಲ್ಲ. ಬಹುಶಃ ತಮ್ಮ ಮುಂದಿನ ಕಾರ್ಯಕ್ರಮದ ಬಗ್ಗೆ ಅವರು ಯಾವ ನಿರ್ಧಾರಕ್ಕೂ ಬಂದಿರಲಿಲ್ಲ. ಆದರೆ ಅವರು ಭಾರತಕ್ಕೆ ಹೊರಡಲಿದ್ದಾರೆಂಬ ವದಂತಿ ಕೇಳಿದೊಡನೆಯೇ ಅನೇಕ ಶಿಷ್ಯರು ಅವರನ್ನು ಆ ಬಗ್ಗೆ ಪ್ರಶ್ನಿಸತೊಡಗಿದರು. ಕೆಲವರು ಅವರ ಭಾರತದ ಕಾರ್ಯೋದ್ದೇಶಗಳಿಗೆ ಧನಸಹಾಯ ಮಾಡಲೂ ಮುಂದಾದರು. ತಮ್ಮ ಸಂನ್ಯಾಸೀ ಸಂಘದ ಕೇಂದ್ರಸ್ಥಾನಕ್ಕಾಗಿ ಗಂಗಾತೀರದಲ್ಲೊಂದು ದೇವಾಲಯವನ್ನು ನಿರ್ಮಾಣ ಮಾಡಬೇಕೆಂಬ ಸ್ವಾಮೀಜಿ ಬಹುಕಾಲದ ಯೋಜನೆಯನ್ನು ಕಾರ್ಯಗತಗೊಳಿಸುವುದಕ್ಕಾಗಿ ಧನಸಹಾಯ ಮಾಡಲು ಅನೇಕರು ಮುಂದಾ ದರು. ಇವರಲ್ಲಿ ಮುಖ್ಯರಾದವರೆಂದರೆ ಮಿಸ್ ಹೆನ್ರಿಟಾ ಮುಲ್ಲರ್, ಎಡ್ವರ್ಡ್ ಸ್ಟರ್ಡಿ ಹಾಗೂ ಅಮೆರಿಕದ ಶ್ರೀಮತಿ ಸಾರಾ ಒಲೆ ಬುಲ್. ಆದರೆ ತಾವು ಭಾರತಕ್ಕೆ ಹಿಂದಿರುಗಿದ ಮೇಲೆ ಸರಿ ಯಾದ ಯೋಜನೆಯೊಂದನ್ನು ರೂಪಿಸಿ ಕಾರ್ಯವನ್ನು ಪ್ರಾರಂಭಿಸಿದ ಬಳಿಕ ಆ ಬಗ್ಗೆ ಅವರಿಗೆಲ್ಲ ತಿಳಿಸುವುದಾಗಿ ಹೇಳಿದರು ಸ್ವಾಮೀಜಿ.

೧೮೯೬ರ ಮಳೆಗಾಲದಲ್ಲಿ ಮುಂಗಾರು ಕೈಕೊಟ್ಟದ್ದರಿಂದ ಬಂಗಾಳ ಪ್ರಾಂತ್ಯದಲ್ಲಿ ಭೀಕರ ಕ್ಷಾಮ ಬಡಿಯಿತು. ನವೆಂಬರ್​-ಡಿಸೆಂಬರ್ ಹೊತ್ತಿಗೆ ಕ್ಷಾಮ ಪರಿಸ್ಥಿತಿ ಮತ್ತಷ್ಟು ಹದಗೆಟ್ಟಿತ್ತು. ಜನರಲ್ಲಿ ಹಾಹಾಕಾರವೆದ್ದಿತು. ಲಂಡನ್ನಿನ ಪತ್ರಿಕೆಗಳಲ್ಲೆಲ್ಲ ಈ ವಿಷಯ ದಪ್ಪಕ್ಷರಗಳಲ್ಲಿ ಪ್ರಕಟ ವಾಯಿತು. ಕ್ಷಾಮದ ವಿಷಯವನ್ನು ತಿಳಿದು ಸ್ವಾಮೀಜಿ ವ್ಯಾಕುಲಿತರಾದರು. ತಮ್ಮ ಬಡ ದೇಶ ಬಾಂಧವರ ದುಃಸ್ಥಿತಿಯನ್ನು ಸ್ಮರಿಸುತ್ತ ಒಂದೇಸಮನೆ ಚಡಪಡಿಸಿದರು. ಮತ್ತು ‘ಇಂಡಿಯನ್ ಮಿರರ್​’ ಪತ್ರಿಕೆಯು ಸಂಗ್ರಹಿಸುತ್ತಿದ್ದ ಕ್ಷಾಮ ಪರಿಹಾರ ನಿಧಿಗೆ ತಮ್ಮ ಬಳಿಯಿದ್ದ ಹಣವನ್ನೆಲ್ಲ ಕಳಿಸಿಕೊಟ್ಟುಬಿಟ್ಟರು. ನೊಂದ ಜನರ ಸಂಕಟವನ್ನು ಕಿಂಚಿತ್ತಾದರೂ ದೂರಮಾಡುವುದು ಸ್ವಾಮೀಜಿಗೆ ಮಠದ ನಿರ್ಮಾಣಕ್ಕಿಂತಲೂ ಮುಖ್ಯವಾಗಿತ್ತು. ಈ ವಿಷಯವನ್ನು ತಿಳಿದು ಅವರ ಶಿಷ್ಯರೂ ತಮ್ಮ ಕಾಣಿಕೆಯನ್ನು ಸಲ್ಲಿಸಿದರು. ಮಿಸ್ ಎಮೆಲಿನ್ ಸೌಟರ್ ಎಂಬ ಮಹಿಳೆ ಕ್ಷಾಮ ಪರಿಹಾರ ನಿಧಿಗೆ ಹಲವು ನೂರು ಪೌಂಡುಗಳಷ್ಟು ಭಾರೀ ಕಾಣಿಕೆಯನ್ನೇ ಸಲ್ಲಿಸಿದಳು. ಹೀಗೆ ಸ್ವಾಮೀಜಿ, ಮುಂದೆ ಸ್ಥಾಪಿತವಾಗಲಿದ್ದ ರಾಮಕೃಷ್ಣ ಮಿಷನ್ನಿನ ಮೂಲಕ ಯಾವ ಸೇವಾಕಾರ್ಯ ವನ್ನು ನಡೆಸಲಿದ್ದರೋ ಅದಕ್ಕೆ ಅವರೀಗ ಮೇಲ್ಪಂಕ್ತಿಯನ್ನು ಹಾಕಿಕೊಟ್ಟರು.

ನವೆಂಬರ್ ೨೧ರಂದು ಕೇಂಬ್ರಿಡ್ಜಿನ ಭಾರತೀಯ ಸಂಘದವರು ಇಬ್ಬರು ಪ್ರತಿಷ್ಠಿತ ಭಾರತೀಯ ವಿದ್ಯಾರ್ಥಿಗಳಾದ ರಾಜಕುಮಾರ ರಣಜಿತ್ ಸಿಂಗ್ ಹಾಗೂ ಅತುಲಚಂದ್ರ ಚಟರ್ಜಿ–ಇವರ ಗೌರವಾರ್ಥವಾಗಿ ಏರ್ಪಡಿಸಿದ್ದ ಔತಣಕೂಟಕ್ಕೆ ಸ್ವಾಮೀಜಿಯನ್ನು ಆಹ್ವಾನಿಸ ಲಾಗಿತ್ತು. ಕೆಲದಿನಗಳ ಹಿಂದೆ ಇಂಗ್ಲೆಂಡ್​-ಆಸ್ಟ್ರೇಲಿಯಗಳ ನಡುವೆ ನಡೆದಿದ್ದ ಕ್ರಿಕೆಟ್ ಟೆಸ್ಟ್ ಪಂದ್ಯದಲ್ಲಿ ರಣಜಿತ್ ಸಿಂಗ್ ಅತ್ಯುತ್ತಮ ಆಟವಾಡಿ ಇಂಗ್ಲೆಂಡಿನ ಮಾನ ಉಳಿಸಿದ್ದರು. ಅತುಲ ಚಂದ್ರ ಚಟರ್ಜಿ ಆ ವರ್ಷದ ಐ. ಸಿ. ಎಸ್. ಪರೀಕ್ಷೆಯಲ್ಲಿ ಪ್ರಪ್ರಥಮ ಸ್ಥಾನವನ್ನು ಗಳಿಸಿದ್ದರು. ಈ ಸಮಾರಂಭದ ಅತಿಥಿಗಳನ್ನು ಉದ್ದೇಶಿಸಿ ಭಾರತಕ್ಕೆ ‘ಸ್ವಸ್ತಿ\eng{’ (Toast)}ಯನ್ನು ಕೋರುವ ಭಾಷಣ ಮಾಡುವಂತೆ ಸ್ವಾಮೀಜಿಯನ್ನು ವಿನಂತಿಸಿಕೊಳ್ಳಲಾಯಿತು.

ಗಡಚಿಕ್ಕುವ ಜೈಕಾರ-ಹರ್ಷೋದ್ಗಾರಗಳ ನಡುವೆ ಸ್ವಾಮೀಜಿ ಮಾತನಾಡಲು ಎದ್ದುನಿಂತರು. “ಭಾರತಕ್ಕೆ ಸ್ವಸ್ತಿ ಕೋರುವ ಈ ಉಪನ್ಯಾಸವನ್ನು ಮಾಡಲು ನನ್ನನ್ನೇ ಏಕೆ ಆರಿಸಿದರೋ ನನಗೆ ಗೊತ್ತಿಲ್ಲ” ಎನ್ನುತ್ತ ತಮಾಷೆ ಮಾಡಿದರು–“ಬಹುಶಃ ಗಾತ್ರದಲ್ಲಿ ನನ್ನ ಶರೀರವು ಭಾರತದ ರಾಷ್ಟ್ರೀಯ ಮೃಗವನ್ನು ಬಹುಮಟ್ಟಿಗೆ ಹೋಲುವುದೆಂಬ ಒಂದೇ ಕಾರಣಕ್ಕೆ ನನ್ನನ್ನು ಆರಿಸಿದ ರೆಂದು ಕಾಣುತ್ತದೆ!” (ನಗು) ಬಳಿಕ ಸ್ವಾಮೀಜಿ, ಸಮಾರಂಭದ ಮುಖ್ಯ ಅತಿಥಿಗಳಾದ ರಣಜಿತ್ ಸಿಂಗ್ ಹಾಗೂ ಅತುಲಚಂದ್ರ ಚಟರ್ಜಿಯವರನ್ನು ಹಾರ್ದಿಕವಾಗಿ ಅಭಿನಂದಿಸಿದರು. ಅನಂತರ ಅವರು ಭಾರತದ ಚರಿತ್ರೆಯ ಬಗ್ಗೆ ಪ್ರಸ್ತಾಪಿಸಿ, ಹಿಂದೆ ಭಾರತವು, ಘೋರ ದುಸ್ಥಿತಿಗಳಿಗೆ ಸಿಲುಕಿ ಕೊಂಡಿತ್ತಾದರೂ ಅವೆಲ್ಲದರಿಂದಲೂ ಅದು ಪಾರಾದಂತೆಯೇ, ಮತ್ತೊಮ್ಮೆ ಅದು ಉನ್ನತಿಯ ಶಿಖರಕ್ಕೇರುತ್ತದೆಯೆಂದು ಹೇಳಿದರು. ಭಾರತದ ಉಜ್ವಲ ಭವಿಷ್ಯದ ಬಗ್ಗೆ ತಮ್ಮ ಸುಲಲಿತ ಶೈಲಿಯಲ್ಲಿ ಅತ್ಯಂತ ಸ್ಫೂರ್ತಿಯುತವಾಗಿ ಮಾತನಾಡಿ ಸ್ವಾಮೀಜಿ, ಅಲ್ಲಿ ನೆರೆದಿದ್ದ ಎಲ್ಲ ಕ್ಷೇತ್ರ ಗಳಿಗೆ ಸೇರಿದ ಪ್ರಮುಖ ಭಾರತೀಯ ಶ್ರೋತೃ ವರ್ಗದ ಮೇಲೆ ಆಳವಾದ ಪ್ರಭಾವ ಬೀರಿದರು.

ಈ ವೇಳೆಗೆ, ಸೇವಿಯರ್ ದಂಪತಿಗಳು ಸ್ವಾಮೀಜಿಯೊಂದಿಗೆ ಭಾರತಕ್ಕೆ ಹೊರಡುವುದು ಖಚಿತವಾಗಿತ್ತು. ಸ್ವಾಮೀಜಿಗೆ ಅತ್ಯಂತ ಪ್ರಿಯವಾದ ಯೋಜನೆಗಳಲ್ಲೊಂದೆಂದರೆ ಹಿಮಾಲಯದ ಮಡಿಲಲ್ಲಿ, ಅದ್ವೈತ ಸಾಧನೆಗೆ ಮೀಸಲಾದ ಆಶ್ರಮವೊಂದನ್ನು ಸ್ಥಾಪಿಸುವುದು. ಈ ಕನಸನ್ನು ನನಸಾಗಿಸುವಲ್ಲಿ ಸೇವಿಯರ್ ದಂಪತಿಗಳು ಅತಿ ಮುಖ್ಯ ಪಾತ್ರವಹಿಸಲಿದ್ದರು. ಇವರೀಗ ಪ್ರಾಪಂಚಿಕ ಜೀವನವನ್ನು ತ್ಯಜಿಸಿ ವಾನಪ್ರಸ್ಥಾಶ್ರಮವನ್ನು ಸ್ವೀಕರಿಸಲು ನಿರ್ಧರಿಸಿ ದ್ದರು; ಸ್ವಾಮೀಜಿಯ ಕಾರ್ಯೋದ್ದೇಶಗಳಿಗಾಗಿ ತಮ್ಮ ತನು ಮನ ಧನಗಳನ್ನರ್ಪಿಸಲು ಸಿದ್ಧ ರಾಗುತ್ತಿದ್ದರು. ಆದ್ದರಿಂದ ಈ ದಂಪತಿಗಳು ತಮ್ಮ ಮನೆವಾರ್ತೆಯ ವ್ಯವಹಾರಗಳನ್ನೆಲ್ಲ ಒಂದು ಮುಕ್ತಾಯ ಘಟ್ಟಕ್ಕೆ ತರುವುದರಲ್ಲಿ ಉದ್ಯುಕ್ತರಾದರು. ಹೀಗೆ ಸ್ವಾಮೀಜಿಯ ಸಂಪರ್ಕಕ್ಕೆ ಬಂದ ಕೆಲಕಾಲದಲ್ಲೇ ಈ ದಂಪತಿಗಳು ತಮ್ಮ ಜೀವನದ ಅತಿ ಮುಖ್ಯ ನಿರ್ಧಾರವನ್ನು ಕೈಗೊಂಡರು. ತಮ್ಮ ಮನೆ, ಪೀಠೋಪಕರಣಗಳು, ಆಭರಣಗಳು, ಪುಸ್ತಕಗಳು–ಎಲ್ಲವನ್ನೂ ಮಾರಿ, ಬಂದ ಹಣವನ್ನೆಲ್ಲ ಸ್ವಾಮೀಜಿಯ ಕೈಗೊಪ್ಪಿಸಿಬಿಟ್ಟರು. ಬಳಿಕ ತತ್ಕಾಲಕ್ಕಾಗಿ ಒಂದು ಬಾಡಿಗೆಯ ಮನೆ ಯಲ್ಲಿ ಉಳಿದುಕೊಂಡರು. ಮತ್ತು ಸ್ವಾಮೀಜಿ ಕರೆದೊಡನೆಯೇ ಅವರೊಂದಿಗೆ ಹೊರಡಲು ಸಿದ್ಧರಾಗಿ ನಿಂತರು. ಇದು ಸೇವಿಯರ್ ದಂಪತಿಗಳ ಮಹಾತ್ಯಾಗವೂ ಹೌದು, ಸ್ವಾಮೀಜಿಯ ಮಹಾಶಕ್ತಿಯೂ ಹೌದು.

ಇದೇ ವೇಳೆಗೆ ಸ್ವಾಮೀಜಿಯ ತರಗತಿಗಳೂ ಉಪನ್ಯಾಸಗಳೂ ಪೂರ್ಣರಭಸದಿಂದ ನಡೆಯು ತ್ತಿದ್ದುವು. ಹೀಗಿರುವಾಗ ಇದ್ದಕ್ಕಿದ್ದಂತೆ ಒಂದು ದಿನ ಅವರು ಭಾರತಕ್ಕೆ ಹಿಂದಿರುಗುವ ತಮ್ಮ ನಿರ್ಧಾರವನ್ನು ಪ್ರಕಟಿಸಿದರು. ಅವರು ಲಂಡನ್ನಿನಿಂದ ಹೊರಡುವ ಬಗ್ಗೆ ಆಲೋಚಿಸುತ್ತಿದ್ದಾ ರೆಂಬ ಊಹಾಪೋಹವಿದ್ದರೂ, ಆಗ ಅದಿನ್ನೂ ಖಚಿತವಾಗಿರಲಿಲ್ಲ. ಅಲ್ಲದೆ ಕೆಲಸಕಾರ್ಯಗಳು ನಡೆದುಕೊಂಡು ಬರುತ್ತಿದ್ದ ರೀತಿಯನ್ನು ಕಂಡು, ಅವರು ಇನ್ನು ಕೆಲತಿಂಗಳಾದರೂ ಇಲ್ಲಿಯೇ ಉಳಿದುಕೊಳ್ಳುತ್ತಾರೆಂದು ಎಲ್ಲರೂ ಭಾವಿಸಿದ್ದರು. ಆದರೆ ಒಂದು ದಿನ ತರಗತಿಯ ಬಳಿಕ ಸ್ವಾಮೀಜಿ ಶ್ರೀಮತಿ ಸೇವಿಯರ್​ರನ್ನು ಕರೆದು ತಮ್ಮ ನಿರ್ಧಾರವನ್ನು ತಿಳಿಸಿದರು. ಮತ್ತು ನೇಪಲ್ಸ್ ನಗರದಿಂದ ಭಾರತಕ್ಕೆ ಹೋಗುವ ಒಂದು ಅನುಕೂಲಕರವಾದ ಹಡಗಿನಲ್ಲಿ ನಾಲ್ಕು ಟೆಕೆಟ್ಟುಗಳನ್ನು ಕಾದಿರಿಸಲು ಹೇಳಿದರು. ಸ್ವಾಮೀಜಿಯ ನಿರ್ಧಾರ ಶ್ರೀಮತಿ ಸೇವಿಯರ್​ರಿಗೂ ಅನಿರೀಕ್ಷಿತವಾದುದೇ ಆದರೂ ಆಕೆ ಮರುಮಾತನಾಡದೆ ಅಂದೇ ಆ ಕೆಲಸವನ್ನು ಮಾಡಿ ಮುಗಿಸಿದರು.

ಡಿಸೆಂಬರ್ ೧೬ ರಂದು ಸ್ವಾಮೀಜಿ ಸೇವಿಯರ್ ದಂಪತಿಗಳೊಂದಿಗೆ ಲಂಡನ್ನಿನಿಂದ ಹೊರಡ ಲಿದ್ದರು. ಅಲ್ಲಿಂದ ನೇರವಾಗಿ ಭಾರತಕ್ಕೆ ಹೋಗುವ ಹಡಗಿನಲ್ಲಿ ಪ್ರಯಾಣ ಮಾಡಬಹು ದಾಗಿತ್ತು; ಆದರೆ ಸಮುದ್ರ ಪ್ರಯಾಣದ ದೂರವನ್ನು ಸಾಧ್ಯವಾದಷ್ಟು ಕಡಿಮೆ ಮಾಡುವ ಉದ್ದೇಶದಿಂದ ಇಟಲಿಯ ನೇಪಲ್ಸ್ ನಗರದವರೆಗೂ ಹೋಗಿ ಅಲ್ಲಿ ಹಡಗು ಹತ್ತಲು ನಿರ್ಧರಿಸ ಲಾಗಿತ್ತು. ಅಲ್ಲದೆ, ದಾರಿಯಲ್ಲಿ ಯೂರೋಪಿನ ಹಲವಾರು ನಗರಗಳನ್ನು ಸಂದರ್ಶಿಸುವ ಅವಕಾಶವೂ ಇತ್ತು. ಹೀಗೆ ಆಲೋಚಿಸಿ ಸ್ವಾಮೀಜಿಯ ಸ್ನೇಹಿತರು ಈ ವ್ಯವಸ್ಥೆ ಮಾಡಿದರು. ನೇಪಲ್ಸ್ ನಗರದಲ್ಲಿ ಗುಡ್​ವಿನ್ ಅವರನ್ನು ಕೂಡಿಕೊಳ್ಳಲಿದ್ದ. ಡಿಸೆಂಬರ್ ೩ಂರಂದು ಹಡಗು ನೇಪಲ್ಸ್​ನಿಂದ ಭಾರತದತ್ತ ಪಯಣಿಸಲಿತ್ತು.

ಸ್ವಾಮೀಜಿ ಹೊರಡುವ ಸುದ್ದಿ ಕೇಳಿದಾಗ ಅವರ ಶಿಷ್ಯವರ್ಗಕ್ಕೆಲ್ಲ ಆಶ್ಚರ್ಯವೂ ಖೇದವೂ ಒಟ್ಟಿಗೆ ಉಂಟಾಯಿತು. ಏಕೆಂದರೆ ಇಂಗ್ಲೆಂಡಿನಲ್ಲಿ ಕೆಲಸಗಳೆಲ್ಲ ಹದಕ್ಕೆ ಬರುತ್ತಿರುವಾಗ ಸ್ವಾಮೀಜಿ ನಿರ್ಗಮಿಸಿಬಿಟ್ಟರೆ ಮತ್ತೆ ಅವು ಕುಂಠಿತವಾಗಬಹುದೆಂದು ಅವರು ಶಂಕಿಸಿದರು. ಸ್ವಾಮೀಜಿಗೂ ಅದರ ಅರಿವಿತ್ತು. ಮಿಸ್ ಮೆಕ್​ಲಾಡಳಿಗೆ ಬರೆದ ಒಂದು ಪತ್ರದಲ್ಲಿ ಅವರು ಆ ಬಗ್ಗೆ ಪ್ರಸ್ತಾಪಿಸಿದರು, “... ಈಗ ತಾನೆ ಲಂಡನ್ನಿನಲ್ಲಿ ನಮ್ಮ ಯೋಜನೆಗಳೆಲ್ಲ ಬಿಸಿಯೇರುತ್ತಿವೆ. ಆದರೆ ಆ ನನ್ನ ಪ್ರಿಯ ಭಾರತ ನನ್ನನ್ನು ಕರೆಯುತ್ತಿದೆ. ನಾನು ಹೋಗಲೇಬೇಕು...ಈ ಸ್ಥಿತಿಯಲ್ಲಿ ಇದನ್ನು ಬಿಟ್ಟುಹೋಗುವುದು ಮೂರ್ಖತನವೇ ಸರಿ ಎಂದು ಇಲ್ಲಿ ಎಲ್ಲರೂ ಭಾವಿಸುತ್ತಾರೆ. ಆದರೆ ಪ್ರಿಯ ಭಗವಂತ ಹೇಳುತ್ತಾನೆ, ‘ಭಾರತಕ್ಕೆ ಹೊರಡು!’ ಎಂದು. ನಾನದನ್ನು ವಿಧೇಯ ನಾಗಿ ಪಾಲಿಸುತ್ತೇನೆ.” ಯಾವುದೇ ಮುಖ್ಯ ನಿರ್ಧಾರವನ್ನು ತೆಗೆದುಕೊಳ್ಳುವ ಮೊದಲು ಸ್ವಾಮೀಜಿ “ಮೇಲಿನಿಂದ” ಹುಕುಂ ಬರುವವರೆಗೂ ತಡೆಯುತ್ತಿದ್ದರು. ಅಂತೆಯೇ, ಈಗಲೂ ಅದು ಬಂದೊ ಡನೆ ಅದಕ್ಕೆ ತಲೆಬಾಗಿದ್ದರು.

ಭಾರತವನ್ನು ತಲುಪಿದ ಕೂಡಲೇ ತಾವು ಕೈಗೊಳ್ಳಬೇಕಾದ ಕಾರ್ಯಗಳಾವುವು ಎಂಬುದರ ಬಗ್ಗೆ ಸ್ವಾಮೀಜಿ ಬಹಳಷ್ಟು ಆಲೋಚಿಸಿ, ತಮ್ಮ ಆದ್ಯತೆಗಳನ್ನು ನಿರ್ಧರಿಸಿದ್ದರು. ಸದ್ಯಕ್ಕೆ ಭಾರತ ದಲ್ಲಿ ಮೂರು ಕೇಂದ್ರಗಳನ್ನು–ಕಲ್ಕತ್ತದಲ್ಲೊಂದು, ಮದ್ರಾಸಿನಲ್ಲೊಂದು ಹಾಗೂ ಸೇವಿಯರ್ ದಂಪತಿಗಳ ಸಹಾಯದಿಂದ ಹಿಮಾಲಯದಲ್ಲೊಂದು–ಸ್ಥಾಪಿಸಬೇಕೆಂಬ ತಮ್ಮ ಇಂಗಿತವನ್ನು ಅವರು ತಮ್ಮ ಶಿಷ್ಯರಿಗೆ ಬರೆದ ಒಂದು ಪತ್ರದಲ್ಲಿ ವ್ಯಕ್ತಪಡಿಸಿದರು. ಬಳಿಕ ಅವರು ಬರೆದರು, “ಈಗ ಮೊದಲು ಈ ಮೂರು ಕೇಂದ್ರಗಳಿಂದ ನಮ್ಮ ಕೆಲಸವನ್ನು ಪ್ರಾರಂಭಿಸೋಣ. ಬಳಿಕ ಮುಂಬಯಿ, ಅಲಹಾಬಾದ್​ಗಳಲ್ಲೂ ಕೇಂದ್ರಗಳನ್ನು ತೆರೆಯೋಣ. ಆಮೇಲೆ, ಭಗವಂತ ಇಚ್ಛಿಸಿ ದಲ್ಲಿ ಇವುಗಳ ಮೂಲಕ ಇಡೀ ಭಾರತವನ್ನೇ ಆಕ್ರಮಣ ಮಾಡೋಣ; ಅಷ್ಟೇ ಅಲ್ಲ–ವಿಶ್ವದ ಪ್ರತಿಯೊಂದು ರಾಷ್ಟ್ರಕ್ಕೂ ವೇದಾಂತ ಪ್ರಚಾರಕರ ಪಡೆಯನ್ನೇ ಕಳಿಸೋಣ.” ಸ್ವಾಮೀಜಿಯ ಉದ್ದೇಶಿತ ಕಾರ್ಯದ ವ್ಯಾಪ್ತಿಯು ಭಾರತ-ಇಂಗ್ಲೆಂಡ್​-ಅಮೆರಿಕಗಳಿಗಷ್ಟೇ ಸೀಮಿತವಾಗಿರ ಲಿಲ್ಲ. ರಷ್ಯಾ, ಜಪಾನ್, ಚೀನಾ ಮೊದಲಾದ ಅನೇಕ ರಾಷ್ಟ್ರಗಳಲ್ಲಿ ಅವರು ವೇದಾಂತ ಪ್ರಸಾರ ಕಾರ್ಯವನ್ನು ಕೈಗೊಳ್ಳಲು ಉದ್ದೇಶಿಸಿದ್ದರು. ಇದು ಅವರೇ ಬರೆದ ಹಾಗೂ ಅವರ ಶಿಷ್ಯರು ಬರೆದ ಕೆಲವು ಪತ್ರಗಳಿಂದ ತಿಳಿದುಬರುತ್ತದೆ. ಅದರಲ್ಲೂ ಜಪಾನಿಗೆ ಭೇಟಿ ನೀಡುವ ಯೋಜನೆ ಯಂತೂ ಅವರ ಕಡೆಯ ದಿನಗಳವರೆಗೂ ಇದ್ದೇ ಇತ್ತು. ಈ ದೇಶಗಳಲ್ಲೆಲ್ಲ ವೇದಧರ್ಮವನ್ನು ಬೋಧಿಸುವುದು ಒಂದು ಉದ್ದೇಶವಾದರೆ, ಭಾರತದ ಸರ್ವತೋಮುಖ ಪ್ರಗತಿಗೆ ಈ ದೇಶಗಳ ನೆರವನ್ನು ಪಡೆದುಕೊಳ್ಳುವುದು ಅವರ ಪರೋಕ್ಷ ಉದ್ದೇಶವೆಂಬುದನ್ನು ಮರೆಯುವಂತಿಲ್ಲ. ಅಲ್ಲದೆ ಪರದೇಶವೊಂದರಲ್ಲಿ ದೊರಕಿದ ಸ್ವಲ್ಪ ಯಶಸ್ಸೂ ಕೂಡ ಭಾರತದಲ್ಲಿ ಉಂಟು ಮಾಡಬಲ್ಲ ಪರಿಣಾಮ ಅಗಾಧವಾದುದು ಎಂಬುದು ಅವರಿಗೆ ಚೆನ್ನಾಗಿ ತಿಳಿದಿತ್ತು. ಆದರೆ ಇಂತಹ ಮಹದಾಕಾಂಕ್ಷೆಯ ಅನೇಕ ಯೋಜನೆಗಳನ್ನು ಸಾಧಿಸಲು ಅವರ ಅಲ್ಪಾಯುಸ್ಸು ಸಾಲದೆ ಹೋಯಿತು.

ಭಾರತದ ವಿವಿಧ ಕಡೆಗಳಲ್ಲಿ ಕೇಂದ್ರಗಳನ್ನು ಸ್ಥಾಪಿಸುವುದರೊಂದಿಗೆ, ಇತರ ಅನೇಕ ಯೋಜನೆಗಳು ಸ್ವಾಮೀಜಿಯ ಮನಸ್ಸಿನಲ್ಲಿದ್ದುವು. ದೀನ-ಆರ್ತ ಜನರಿಗಾಗಿ ವಿಧವಿಧದ ಸೇವಾ ಕಾರ್ಯಗಳನ್ನು ಕೈಗೊಳ್ಳುವುದು, ಪ್ರಾಥಮಿಕ ಹಾಗೂ ಉನ್ನತ ಶಿಕ್ಷಣದ ಕೇಂದ್ರಗಳನ್ನು ಸ್ಥಾಪಿಸು ವುದು, ತಾಂತ್ರಿಕ ಶಿಕ್ಷಣ ಶಾಲೆಗಳನ್ನು ತೆರೆಯುವುದು, ಸ್ತ್ರೀಯರಿಗಾಗಿ ಒಂದು ಮಠವನ್ನು ಸ್ಥಾಪಿಸುವುದು, ವೃತ್ತಪತ್ರಿಕೆಗಳ ಮೂಲಕ ಜನ ಜಾಗೃತಿಯುಂಟುಮಾಡುವುದು ಹಾಗೂ ಧರ್ಮ ಪ್ರಸಾರ ಮಾಡುವುದು ಮೊದಲಾದ ಅನೇಕ ಯೋಜನೆಗಳ ಬಗ್ಗೆ ಸ್ವಾಮೀಜಿ ತಮ್ಮ ಗುರುಭಾಯಿ ಗಳಿಗೂ ಶಿಷ್ಯರಿಗೂ ಪತ್ರಗಳನ್ನು ಬರೆದು, ಆ ನಿಟ್ಟಿನಲ್ಲಿ ಅವರನ್ನು ಕಾರ್ಯೋನ್ಮುಖರನ್ನಾಗಿಸಿ ದ್ದರು. ಈ ಯೋಜನೆಗಳ ಪೈಕಿ, ಸ್ತ್ರೀಯರ ಏಳಿಗೆಗೆ ಸ್ವಾಮೀಜಿ ತುಂಬ ಪ್ರಾಶಸ್ತ್ಯ ನೀಡಿದರು. ಆದ್ದರಿಂದ ಸ್ತ್ರೀಯರ ವಿದ್ಯಾಭ್ಯಾಸಕ್ಕಾಗಿ ಭಾರತದಾದ್ಯಂತ ಸಂಸ್ಥೆಗಳನ್ನು ಕಟ್ಟಿ, ತನ್ಮೂಲಕ ವಿದ್ಯಾವಂತ ಮಹಿಳೆಯರ ತಲೆಮಾರನ್ನೇ ಸೃಷ್ಟಿಸಬೇಕು; ಉತ್ತಮ ಗೃಹಿಣಿಯರನ್ನು, ಮತ್ತು ರಾಷ್ಟ್ರದ ಸಮಗ್ರ ಸ್ತ್ರೀಕುಲದ ಹಿತಸಾಧನೆಗೆ ದುಡಿಯುವಂತಹ ಬ್ರಹ್ಮಚಾರಿಣಿಯರನ್ನು ನಿರ್ಮಿಸ ಬೇಕು ಎಂಬ ಅಭಿಲಾಷೆ ಸ್ವಾಮೀಜಿಯದಾಗಿತ್ತು. ಈ ದಿಸೆಯಲ್ಲಿ ಕುಮಾರಿ ಮಾರ್ಗರೆಟ್ಟಳು ಪ್ರಮುಖ ಪಾತ್ರ ವಹಿಸಬಹುದೆಂದು ಅವರು ನಿರೀಕ್ಷಿಸಿದ್ದರು. ಮುಂದೆ ಈ ಕಾರ್ಯಕ್ಕಾಗಿ ಅವಳನ್ನು ಭಾರತಕ್ಕೆ ಕರೆತರುವ ಉದ್ದೇಶ ಅವರಿಗಿತ್ತು. ಅಲ್ಲದೆ ಮಿಸ್ ಹೆನ್ರಿಟಾ ಮುಲ್ಲರಳೂ ಭಾರತೀಯ ಮಹಿಳೆಯರ ಏಳ್ಗೆಗಾಗಿ ಶ್ರಮಿಸಲು ಮುಂದಾಗಿದ್ದಳು. ಕೆಲಕಾಲದಲ್ಲೇ ಆಕೆ ಸ್ವಾಮೀಜಿಯನ್ನು ಹಿಂಬಾಲಿಸಿ ಭಾರತಕ್ಕೆ ಬರಲಿದ್ದಳು.

ಸ್ವಾಮೀಜಿ ಲಂಡನ್ನಿನಿಂದ ನಿರ್ಗಮಿಸುವ ಸುದ್ದಿ ತಿಳಿದಾಗ ಅವರ ಶಿಷ್ಯರಿಗೆಲ್ಲ ತುಂಬ ದುಃಖ ವಾಯಿತು. ಅವರನ್ನು ಇನ್ನೂ ಕೆಲವು ತಿಂಗಳ ಕಾಲ ತಮ್ಮ ನಡುವೆಯೇ ಉಳಿಸಿಕೊಳ್ಳಲು ಪ್ರಯತ್ನ ಮಾಡಿದರಾದರೂ ಸ್ವಾಮೀಜಿ ತಮ್ಮ ನಿರ್ಧಾರವನ್ನು ಬದಲಿಸಲಿಲ್ಲ; ತಾವು ಭಾರತಕ್ಕೆ ಕೂಡಲೇ ಹೊರಡಬೇಕಾಗಿದೆ, ಅಲ್ಲಿ ತಮ್ಮ ಕೆಲಸವನ್ನು ಮುಂದುವರಿಸಬೇಕಾಗಿದೆ ಎಂದರು. ಆಗ ಲಂಡನ್ನಿನ ಅವರ ಶಿಷ್ಯರು, ಅವರನ್ನು ಬೀಳ್ಗೊಡಲು ಸಿದ್ಧತೆಗಳನ್ನು ಮಾಡಿಕೊಳ್ಳತೊಡಗಿದರು. ಅವರಿಗೆ ತಮ್ಮ ಹೃತ್ಪೂರ್ವಕ ಕೃತಜ್ಞತೆಯನ್ನು ಸಲ್ಲಿಸಲು ಲಂಡನ್ನಿನ ಭಕ್ತರು-ವಿಶ್ವಾಸಿಗರೆಲ್ಲ ಸೇರಿ ವಿದಾಯ ಸಮಾರಂಭವನ್ನು ಏರ್ಪಡಿಸಿದರು.

ಪಿಕಾಡಿಲಿಯ ‘ಪ್ರಿನ್ಸಸ್ ಹಾಲ್​’ನಲ್ಲಿ ಡಿಸೆಂಬರ್ ೧೩ ರಂದು ಭಾನುವಾರ ಈ ಸಮಾರಂಭ ವನ್ನು ಏರ್ಪಡಿಸಲಾಗಿತ್ತು. ಲಂಡನ್ ನಗರದ ದೂರದೂರದ ಬಡಾವಣೆಗಳಿಂದಲೂ ಜನ ಪ್ರವಾಹ ಹರಿದು ಬಂದು ಸಭಾಂಗಣವನ್ನು ತುಂಬಿತು. ಇದೇ ಸ್ಥಳದಲ್ಲಿ ಸ್ವಾಮೀಜಿಯ ಉಪನ್ಯಾಸಗಳು ಎಷ್ಟೋ ಸಲ ನಡೆದಿದ್ದುವು. ಆಗಲೂ ಜನರು ಹೀಗೆಯೇ ಕಿಕ್ಕಿರಿದಿದ್ದರು. ಆದರೆ ಆ ಸಂದರ್ಭಗಳಲ್ಲಿ ಕುತೂಹಲ-ಜಿಜ್ಞಾಸೆಯ ವಾತಾವರಣವಿರುತ್ತಿತ್ತು. ಆದರೆ ಇಂದು ಒಂದು ಬಗೆಯ ದುಃಖ-ದುಮ್ಮಾನದ ನೀರವತೆ ಕವಿದಿತ್ತು. ಭಾರವಾದ ಹೃದಯದಿಂದ ಸ್ವಾಮೀಜಿ ಸಭಾಂಗಣವನ್ನು ಪ್ರವೇಶಿಸಿ, ಸುಂದರವಾಗಿ ಅಲಂಕರಿಸಲ್ಪಟ್ಟಿದ್ದ ವೇದಿಕೆಯನ್ನೇರಿದರು. ಅವರೊಂದಿಗೆ ಸ್ವಾಮಿ ಅಭೇದಾನಂದರೂ ಅಲ್ಲಿ ಉಪಸ್ಥಿತರಿದ್ದರು. ಈ ವೇಳೇಗಾಗಲೇ ಅಭೇದಾ ನಂದರು ತಮ್ಮ ಉಪನ್ಯಾಸಗಳಿಂದ ಲಂಡನ್ ಮಹಾನಗರದಲ್ಲಿ ಹೆಸರುಗಳಿಸಿದ್ದರು. ಸ್ವಾಮೀಜಿಯ ನಿರ್ಗಮನದ ಬಳಿಕ ಅವರ ಕಾರ್ಯವನ್ನು ಮುಂದುವರಿಸಿಕೊಂಡು ಬರಲು ಅಭೇದಾನಂದರು ಸಿದ್ಧರಾಗಿದ್ದರು. ಆದ್ದರಿಂದ ಸ್ವಾಮೀಜಿಯ ಭಕ್ತ-ಶಿಷ್ಯವೃಂದ ಹಾಗೂ ವೇದಾಂತದತ್ತ ಆಕರ್ಷಿತರಾಗಿದ್ದ ನೂರಾರು ಜನರು ಈಗ ಸಾಂತ್ವನಕ್ಕಾಗಿ ಸ್ವಾಮಿ ಅಭೇದಾ ನಂದರತ್ತ ತಿರುಗಿದ್ದರು.

ವಿದಾಯ ಸಮಾರಂಭದ ಅಂಗವಾಗಿ ನಗರದ ಪ್ರಸಿದ್ಧ ಗಾಯಕರಿಂದ ಸಂಗೀತದ ಕಾರ್ಯ ಕ್ರಮವಿತ್ತು. ಬಳಿಕ ಸ್ವಾಗತ ಭಾಷಣವಾಗುತ್ತಲೇ ಸಭಿಕರಲ್ಲಿ ಹಲವಾರು ಸ್ತ್ರೀಪುರುಷರು ಭಾವ ಪೂರ್ಣ ಭಾಷಣಗಳನ್ನು ಮಾಡಿದರು. ಲಂಡನ್ನಿನಲ್ಲಿ ಸ್ವಾಮೀಜಿ ಗಳಿಸಿದ್ದ ಗೌರವಾದರ ಭಾವವು ಎಷ್ಟು ಆಳವಾಗಿತ್ತೆಂಬುದಕ್ಕೆ ಈ ಭಾಷಣಗಳು ಸಾಕ್ಷ್ಯವಾಗಿದ್ದುವು. ಪ್ರತಿಯೊಬ್ಬರ ಮಾತನ್ನೂ ಅನುಮೋದಿಸುವಂತೆ ಸಭಿಕರು ಮತ್ತೆ ಮತ್ತೆ ಕರತಾಡನ ಮಾಡಿದರು. ಹೆಚ್ಚಿನವರು ಸ್ವಾಮೀಜಿ ಯನ್ನು ಬೀಳ್ಕೊಳ್ಳಬೇಕಾದ ದುಃಖದಿಂದಾಗಿ ಮೌನವಾಗಿ ಕುಳಿತಿದ್ದರು. ಕೆಲವರ ಕಣ್ಣಂಚಿನಲ್ಲಿ ನೀರು ತುಂಬಿದ್ದನ್ನು ಕಾಣಬಹುದಾಗಿತ್ತು. ಶೀಘ್ರದಲ್ಲೇ ಸ್ವಾಮೀಜಿ ಮರಳಿ ಬರುವಂತಾಗಲೆಂಬ ಆಶಯವನ್ನು ಎಲ್ಲ ಭಾಷಣಕರ್ತರೂ ವ್ಯಕ್ತಪಡಿಸಿದರು.

ಬಳಿಕ ಅಂದಿನ ಸಭೆಯ ಅಧ್ಯಕ್ಷನಾದ ಇ. ಟಿ. ಸ್ಟರ್ಡಿ, ಲಂಡನ್ನಿನ ಶಿಷ್ಯರ ಪರವಾಗಿ ಸ್ವಾಮೀಜಿಗೆ ಬಿನ್ನವತ್ತೆಳೆಯೊಂದನ್ನು ಅರ್ಪಿಸಿದ:

“ಪರಮ ಪೂಜ್ಯ ಸ್ವಾಮೀಜಿ,

ಲಂಡನ್ನಿನ ವೇದಾಂತದ ವಿದ್ಯಾರ್ಥಿಗಳಾದ ನಾವು, ಅದ್ಭುತ ಸಾಮರ್ಥ್ಯಶಾಲೀ ಗುರುಗಳಾದ ನಿಮ್ಮ ಆಶ್ರಯದಲ್ಲಿದ್ದೆವು. ಧರ್ಮದ ಅಧ್ಯಯನದಲ್ಲಿ ನೀವು ನಮಗೆ ನೀಡಿರುವ ಸಹಾಯಕ್ಕಾಗಿ ಮತ್ತು ನೀವು ಕೈಗೊಂಡಿರುವ ಉದಾತ್ತವೂ ಸ್ವಾರ್ಥರಹಿತವೂ ಆದ ಕಾರ್ಯಕ್ಕಾಗಿ ನಿಮಗೆ ನಮ್ಮ ಹೃತ್ಪೂರ್ವಕ ಕೃತಜ್ಞತೆಯನ್ನೂ ಮೆಚ್ಚುಗೆಯನ್ನೂ ವ್ಯಕ್ತಪಡಿಸದಿದ್ದಲ್ಲಿ ಕರ್ತವ್ಯಚ್ಯುತಿಯಾದೀ ತೆಂದು ಭಾವಿಸುತ್ತೇವೆ.

“ನೀವು ಇಂಗ್ಲೆಂಡಿನಿಂದ ಇಷ್ಟು ಶೀಘ್ರದಲ್ಲಿ ನಿರ್ಗಮಿಸುತ್ತಿರುವುದು ನಮಗೆ ಅತ್ಯಂತ ವಿಷಾದವನ್ನುಂಟುಮಾಡಿದೆ. ಆದರೆ ನೀವು ನಮ್ಮ ಭಾರತೀಯ ಬಂಧು-ಭಗಿನಿಯರಿಗಾಗಿ ಮಾಡ ಬೇಕಾದ ಕಾರ್ಯಗಳಿವೆ ಎಂಬುದನ್ನು ನಾವು ಅರಿತಿದ್ದೇವೆ. ನೀವು ಆ ಕಾರ್ಯದಲ್ಲಿ ಅತ್ಯುನ್ನತ ಯಶಸ್ಸನ್ನು ಗಳಿಸುವಂತಾಗಲಿ ಎಂಬುದು ನಮ್ಮೆಲ್ಲರ ಒಮ್ಮನಸ್ಸಿನ ಪ್ರಾರ್ಥನೆ.

“ನಿಮ್ಮ ಬೋಧನೆಗಳ ಪ್ರಭಾವಕ್ಕೆ ಒಳಗಾದ ನಾವೆಲ್ಲ ಉತ್ಕರ್ಷಪೂರ್ಣ ಅನುಭವವನ್ನು ಪಡೆದಿದ್ದೇವೆ. ವೇದಾಂತದ ಸಚೇತನ ರೂಪವಾದ ನಿಮ್ಮ ವ್ಯಕ್ತಿತ್ವವೂ ನಮಗೆ ಅತ್ಯಂತ ಉಪಯುಕ್ತವಾದ ಬೆಂಬಲವಾಗಿದೆ. ಅಲ್ಲದೆ ಭಗವಂತನ ಬಗ್ಗೆ ನಿಜವಾದ ಪ್ರೀತಿಯನ್ನು ಬೆಳೆಸಿ ಕೊಳ್ಳಲು ನಿಮ್ಮ ಬೋಧನೆಗಳೂ ವ್ಯಕ್ತಿತ್ವವೂ ನಮಗೆ ಸ್ಫೂರ್ತಿಯ ಸೆಲೆಯಾಗಿವೆ.

“ನೀವು ಈ ದೇಶಕ್ಕೆ ಶೀಘ್ರವಾಗಿ ಮರಳುವುದನ್ನೇ ನಾವು ಅತ್ಯಂತ ಆಸಕ್ತಿಯಿಂದ ಮತ್ತು ತೀವ್ರ ಕಾತರತೆಯಿಂದ ಇದಿರು ನೋಡುತ್ತಿರುತ್ತೇವೆ. ಆದರೆ ಅದರೊಂದಿಗೇ, ನೀವು ನಮಗೆ ಯಾವ ಭಾರತವನ್ನು ಹೊಸದೊಂದು ಬೆಳಕಿನಲ್ಲಿ ನೋಡಲು ಕಲಿಸಿಕೊಟ್ಟಿರುವರೋ ಮತ್ತು ಯಾವ ಭಾರತವನ್ನು ಪ್ರೀತಿಸಲು ಹೇಳಿಕೊಟ್ಟಿರುವಿರೋ ಆ ಭಾರತವು, ನೀವು ಇಡಿಯ ಜಗತ್ತಿಗೆ ಸಲ್ಲಿಸುತ್ತಿರುವ ಸೇವೆಯಲ್ಲಿ ನಮ್ಮೊಂದಿಗೆ ಸಹಭಾಗಿಯಾಗಲಿದೆ ಎಂಬುದನ್ನು ಭಾವಿಸಲು ನಮಗೆ ಬಹಳ ಸಂತೋಷವೂ ಆಗುತ್ತದೆ.

“ಕೊನೆಯದಾಗಿ, ಭಾರತದ ಜನರಿಗೆ ನಮ್ಮ ಪ್ರೀತಿಪೂರ್ವಕ ಸಹಾನುಭೂತಿಯನ್ನು ತಿಳಿಸ ಬೇಕೆಂದು, ಮತ್ತು ನಾವೆಲ್ಲರೂ ಅದೇ ಭಗವಂತನ ಅಂಶಗಳೆಂದು ನಿಮ್ಮಿಂದಾಗಿ ಅರಿತಿರುವ ನಾವು, ಅವರ ಧ್ಯೇಯಗಳನ್ನು ನಮ್ಮದೆಂದೇ ಭಾವಿಸುತ್ತೇವೆ ಎಂಬ ಭರವಸೆಯನ್ನು ಅವರಿಗೆ ತಿಳಿಸಬೇಕೆಂದು ನಿಮ್ಮನ್ನು ವಿಶೇಷವಾಗಿ ಪ್ರಾರ್ಥಿಸಿಕೊಳ್ಳುತ್ತೇವೆ.”

ವಿಶ್ವಾಸಪೂರ್ಣವಾದ ಈ ಬಿನ್ನವತ್ತಳೆಯನ್ನು ಹಾಗೂ ತಮ್ಮ ಸ್ನೇಹಿತರು-ಶಿಷ್ಯರು-ಭಕ್ತರು- ಅಭಿಮಾನಿಗಳೆಲ್ಲ ತಮ್ಮ ಮೇಲಿರಿಸಿರುವ ಪ್ರೀತ್ಯಾದರವನ್ನು ಕಂಡು ಸ್ವಾಮೀಜಿಯ ಎದೆ ತುಂಬಿ ಬಂದಿತು. ಲಂಡನ್ನಿನ ವೇದಾಂತ ವಿದ್ಯಾರ್ಥಿಗಳ ಅಭಿನಂದನೆಗೆ ಉತ್ತರವಾಗಿ ಅವರು ಎದ್ದು ನಿಂತು ಸಾವಕಾಶವಾಗಿ, ಗಂಭೀರವಾಗಿ, ಅಷ್ಟೇ ವಿಶ್ವಾಸದಿಂದ ಮಾತಿಗಾರಂಭಿಸಿದರು. ತಾವು ಕೆಲ ಕಾಲದಲ್ಲೇ ಮತ್ತೆ ಲಂಡನ್ನಿಗೆ ಭೇಟಿ ನೀಡುವ ಆಶಯವನ್ನು ಅವರು ವ್ಯಕ್ತಪಡಿಸಿದರು. ಅಲ್ಲದೆ, ಶೀಘ್ರದಲ್ಲೇ ಭಾರತದಿಂದ ವೇದಾಂತ ಪ್ರಚಾರಕರ ತಂಡವೇ ಬಂದು ಹಿಂದೂ ಧರ್ಮದ ಅತ್ಯುನ್ನತ ಸಂದೇಶಗಳನ್ನು ಎಲ್ಲೆಲ್ಲಿಯೂ ಪ್ರಸಾರ ಮಾಡುವುದೆಂದು ತಾವು ನಿರೀಕ್ಷಿಸುವುದಾಗಿ ಅವರು ಹೇಳಿದರು. ಸ್ವಾಮೀಜಿಯ ಈ ಮಾತುಗಳು ಅವರ ಶಿಷ್ಯವೃಂದಕ್ಕೆ ಬಹಳಷ್ಟು ಸಮಾಧಾನ ಉಂಟುಮಾಡಿದುವು. ಅಲ್ಲದೆ ಅವರು ಉನ್ನತ ಭಾವದಲ್ಲಿದ್ದಾಗ ಕೆಲವೊಮ್ಮೆ ತಮ್ಮ ಆಪ್ತಶಿಷ್ಯರ ಮುಂದೆ ಹೇಳುತ್ತಿದ್ದರು, “ಜೀರ್ಣವಾದ ವಸ್ತ್ರವನ್ನು ಎಸೆದುಬಿಡುವಂತೆ ನಾನು ಈ ದೇಹವನ್ನು ಯಾವಾಗಬೇಕಾದರೂ ತ್ಯಜಿಸಿಬಿಡಬಹುದು. ಆದರೆ ಸಮಸ್ತ ಮಾನವಕೋಟಿಯೇ ಆ ಮಹಾಸತ್ಯ ವನ್ನು ಅರಿಯುವವರೆಗೂ ನಾನು ಈ ಬೋಧನೆಯ ಕಾರ್ಯದಿಂದ, ಜನತೆಗೆ ನೆರವು ನೀಡುವ ಈ ಕಾರ್ಯದಿಂದ ವಿರಮಿಸುವುದಿಲ್ಲ.” ಸ್ವಾಮೀಜಿಯ ಇಂತಹ ಉದ್ಗಾರಗಳು ಶಿಷ್ಯರಲ್ಲಿ ಹೊಸ ಭರವಸೆಯನ್ನುಂಟುಮಾಡಿ, ನಿರಂತರಸ್ಫೂರ್ತಿಯ ಆಗರದಂತೆ ಕೆಲಸ ಮಾಡುತ್ತಿದ್ದುವು. ಸ್ವಾಮೀಜಿ ಕಣ್ಮರೆಯಾಗಿ ಹಲವಾರು ವರ್ಷಗಳೇ ಸಂದಿವೆ. ಅವರನ್ನು ಕಣ್ಣಾರೆ ಕಾಣದ ಲಕ್ಷೋಪಲಕ್ಷ ಜನ ಇಂದು, ಅವರು ಬಿಟ್ಟುಹೋಗಿರುವ ಸಂದೇಶಗಳಲ್ಲಿ ಆಧ್ಯಾತ್ಮಿಕ ಪ್ರಭೆಯನ್ನು ಕಾಣುತ್ತಿದ್ದಾರೆ, ಪ್ರಭಾವಿತರಾಗುತ್ತಿದ್ದಾರೆ. ಅವರ ವೀರವಾಣಿಯಲ್ಲಿ ಅವರೇ ನೆಲಸಿದ್ದಾರೆ. ಆದ್ದರಿಂದ ಅದನ್ನು ಓದಿದವರಲ್ಲಿ ವಿದ್ಯುತ್ ಸಂಚಾರವಾಗುತ್ತಿದೆ. ವಿವೇಕಾನಂದರನ್ನು ಪ್ರತ್ಯಕ್ಷ ವಾಗಿ ಕಾಣದಿರುವವರಿಗೇ ಅವರ ಮಾತುಗಳು ಇಷ್ಟೊಂದು ಸ್ಫೂರ್ತಿ ನೀಡುತ್ತಿರುವಾಗ, ಅವರನ್ನು ಕಣ್ಣಾರೆ ಕಂಡು, ಅವರ ಮಾತುಗಳನ್ನು ಕಿವಿಯಾರೆ ಕೇಳಿದವರ ಮೇಲೆ ಉಂಟಾದ ಪ್ರಭಾವ ಇನ್ನೆಷ್ಟಿದ್ದಿರಬಹುದು!

ಇಂಗ್ಲೆಂಡಿನಲ್ಲಿ ತಮ್ಮ ಕಾರ್ಯದ ಯಶಸ್ಸಿನ ಬಗ್ಗೆ ಸ್ವಾಮೀಜಿಗೆ ತುಂಬ ತೃಪ್ತಿಯಾಗಿತ್ತು. ವೇದಾಂತದ ಉನ್ನತ ಬೋಧನೆಗಳನ್ನು ಅರಗಿಸಿಕೊಳ್ಳಲು ಇಂಗ್ಲಿಷರಿಗೆ ಸ್ವಲ್ಪ ಹೆಚ್ಚು ಸಮಯ ಬೇಕಾಗಬಹುದಾದರೂ, ಒಮ್ಮೆ ಅವುಗಳನ್ನು ಅವರು ಸ್ವೀಕರಿಸಿದರೆಂದರೆ ಅವು ಸ್ಥಿರವಾಗಿ ಬೇರೂರಿ ನಿಲ್ಲುತ್ತವೆಯೆಂದು ಸ್ವಾಮೀಜಿಗೆ ನಂಬಿಕೆಯುಂಟಾಗಿತ್ತು. ಲಂಡನ್ನಿನಿಂದ ಹೊರಡುವ ಮುನ್ನ ಹೇಲ್ ಸೋದರಿಯರಿಗೆ ಬರೆದ ಒಂದು ಪತ್ರದಲ್ಲಿ ಹೇಳಿದರು, “ಲಂಡನ್ನಿನ ಕಾರ್ಯ ಪ್ರಚಂಡವಾಗಿ ಯಶಸ್ವಿಯಾಗಿದೆ. ಇಂಗ್ಲಿಷರು ಅಮೆರಿಕನ್ನರಷ್ಟು ಚುರುಕಲ್ಲ. ಆದರೆ ಅವರ ಹೃದಯವನ್ನು ನೀವು ಒಮ್ಮೆ ಸ್ಪರ್ಶಿಸಿದರೆ ಸಾಕು, ಅದು ಎಂದೆಂದಿಗೂ ನಿಮ್ಮದಾಗಿಬಿಡುತ್ತದೆ. ಇಲ್ಲಿ ನಾನು ನಿಧಾನವಾಗಿ ಯಶಸ್ಸನ್ನು ಗಳಿಸಿದ್ದೇನೆ. ಆರು ತಿಂಗಳಲ್ಲಿ ನಾನು (ಸಾರ್ವಜನಿಕ ಉಪನ್ಯಾಸಗಳನ್ನು ಬಿಟ್ಟು) ಸುಮಾರು ನೂರಿಪ್ಪತ್ತು ಜನರಿರುವ ಸ್ಥಿರವಾದ ತರಗತಿಯೊಂದನ್ನು ನಡೆಸಲು ಸಾಧ್ಯವಾಯಿತೆಂದರೆ ಅದೊಂದು ಗಮನಾರ್ಹ ವಿಷಯವಲ್ಲವೆ? ಇಲ್ಲಿ ಪ್ರತಿ ಯೊಬ್ಬರೂ ಕೆಲಸದ ಪ್ರಜ್ಞೆಯುಳ್ಳವರು–ಕೆಲಸಕ್ಕೆ ಪ್ರಾಮುಖ್ಯ ಕೊಡುವವರು. ಗುಡ್​ವಿನ್ ಹಾಗೂ ಸೇವಿಯರ್ ದಂಪತಿಗಳು ಸೇವೆ ಸಲ್ಲಿಸಲು ನನ್ನೊಂದಿಗೆ ಭಾರತಕ್ಕೆ ಬರುತ್ತಿದ್ದಾರೆ– ಮತ್ತು ಅದಕ್ಕಾಗಿ ತಮ್ಮ ಹಣವನ್ನೇ ಖರ್ಚುಮಾಡಲು ಸಿದ್ಧರಾಗಿದ್ದಾರೆ!... ಅಲ್ಲದೆ ಭಾರತ ದಲ್ಲಿ ನನ್ನ ಕಾರ್ಯವನ್ನು ಪ್ರಾರಂಭಿಸಲು ಅದಾಗಲೇ ಹಣದ ರೂಪದಲ್ಲಿ ನೆರವು ಬಂದಿದೆ ಮತ್ತು ಇನ್ನಷ್ಟು ಬರಲಿದೆ. ಇಂಗ್ಲಿಷರ ಬಗ್ಗೆ ನನ್ನ ಅಭಿಪ್ರಾಯವು ಸಂಪೂರ್ಣ ಬದಲಾಗಿದೆ. ನನಗೆ ಈಗ ಅರ್ಥವಾಗುತ್ತಿದೆ, ಇತರೆಲ್ಲ ಜನಾಂಗಗಳಿಗಿಂತಲೂ ಹೆಚ್ಚಾಗಿ ಅವರ ಮೇಲೆ ಭಗವಂತ ಅಷ್ಟೊಂದು ಕೃಪೆ ಮಾಡಿರುವುದೇಕೆ ಎಂದು. ಅವರು ಸ್ಥಿರಬುದ್ಧಿಯುಳ್ಳವರು. ಅಡಿ ಯಿಂದ ಮುಡಿಯವರೆಗೂ ಪ್ರಾಮಾಣಿಕರು, ಅಗಾಧ ಭಾವನಾ ತೀವ್ರತೆಯುಳ್ಳವರು. ಅವರು ಹೊರ ಪದರದಲ್ಲಿ ಮಾತ್ರ ಉದಾಸೀನ ಮನಸ್ಕರು ಅಷ್ಟೆ. ಆ ಹೊರ ಪದರವನ್ನು ಒಡೆದರೆ ಅವರು ನಮ್ಮವರಾದಂತೆಯೇ ಸರಿ...”

ಇಂಗ್ಲಿಷರೊಂದಿಗಿನ ದೀರ್ಘ, ನಿಕಟ ಸಂಪರ್ಕದ ಫಲವಾಗಿ ಸ್ವಾಮೀಜಿ ಆ ಜನರನ್ನು ಹೆಚ್ಚು ಚೆನ್ನಾಗಿ ಅರಿತುಕೊಳ್ಳಲು ಸಾಧ್ಯವಾಯಿತು. ಇಂಗ್ಲಿಷರ ಅನೇಕ ಗುಣಗಳನ್ನು ಅವರು ಹೃತ್ಪೂರ್ವಕ ವಾಗಿ ಮೆಚ್ಚಿಕೊಂಡರು. ಭಾರತಕ್ಕೆ ಹಿಂದಿರುಗಿದ ಮೇಲೆ ಕಲ್ಕತ್ತದಲ್ಲಿ ನಡೆದ ಸಾರ್ವಜನಿಕ ಸಮಾರಂಭದಲ್ಲಿ ಸ್ವಾಮೀಜಿ ಬಿಚ್ಚೆದೆಯ ಅಭಿಪ್ರಾಯಗಳನ್ನು ವ್ಯಕ್ತಪಡಿಸಿದರು–“ಇಂಗ್ಲಿಷರ ನಾಡಿಗೆ, ಅವರ ಮೇಲೆ ನನ್ನಷ್ಟು ದ್ವೇಷವಿಟ್ಟುಕೊಂಡು ಪ್ರವೇಶಿಸಿದವರು ಯಾರೂ ಇರಲಾರರು. ಇದೇ ವೇದಿಕೆಯ ಮೇಲೆ ಕುಳಿತಿರುವ ನನ್ನ ಆಂಗ್ಲ ಸ್ನೇಹಿತರು ಈ ಮಾತಿಗೆ ಸಾಕ್ಷಿಯಾಗಬಲ್ಲರು. ಆದರೆ ನಾನು ಅವರೊಂದಿಗೆ ಬೆರತು ಅವರ ಜೀವನಾಡಿಯ ಮಿಡಿತವನ್ನು ಅರಿಯುತ್ತ ಹೋದಂತೆ ಅವರನ್ನು ಹೆಚ್ಚುಹೆಚ್ಚಾಗಿ ಪ್ರೀತಿಸಲಾರಂಭಿಸಿದೆ. ಮತ್ತು ಸೋದರರೆ, ಈಗ ಇಲ್ಲಿರುವವರಲ್ಲಿ, ಆಂಗ್ಲರನ್ನು ನನ್ನಷ್ಟು ಪ್ರೀತಿಸುವವರು ಯಾರೂ ಇಲ್ಲ!” ಆದರೆ ಇಲ್ಲಿ ಸ್ವಾಮೀಜಿ ಇಂಗ್ಲಿಷರ ವೈಯಕ್ತಿಕ ಸ್ವಭಾವವನ್ನು ಹಾಗೂ ಅವರ ಇತರ ಶ್ರೇಷ್ಠ ಗುಣಗಳನ್ನು ಮೆಚ್ಚಿ ಈ ಮಾತುಗಳನ್ನಾಡಿ ದ್ದಾರಷ್ಟೇ ಎಂಬುದನ್ನು ಗಮನಿಸಬೇಕು. ಇಂಗ್ಲಿಷರ ಸಾಮ್ರಾಜ್ಯಶಾಹೀ ಮನೋಭಾವವನ್ನಾಗಲಿ ದೌರ್ಜನ್ಯವನ್ನಾಗಲಿ ಸ್ವಾಮೀಜಿ ಎಂದಿಗೂ ಅನುಮೋದಿಸಲಿಲ್ಲ. ಅಲ್ಲದೆ ಭಾರತದ ವಿಷಯವಾಗಿ ಹೇಳುವುದಾದರೆ, ಇಂಗ್ಲಿಷರ ಆಳ್ವಿಕೆಯಿಂದ ಭಾರತೀಯರಿಗಾದ ಉಪಕಾರಕ್ಕಿಂತ ಅಪಕಾರವೇ ಹೆಚ್ಚೆಂದು ಅವರು ಹೇಳುತ್ತಿದ್ದರು.

ಆದರೆ, ಆ ವಿಷಯ ಹಾಗಿರಲಿ, ಸ್ವಾಮೀಜಿಯ ಕಾರ್ಯೋದ್ದೇಶವಂತೂ ಅವರ ನಿರೀಕ್ಷೆಗೂ ಮೀರಿ ಯಶಸ್ವಿಯಾಗಿತ್ತು. ಅವರು ಸಾಧಿಸಿದುದು ಒಂದು ಅದ್ಭುತವಲ್ಲದೆ ಮತ್ತೇನು? ಭಾರತದ ವಿಷಯದಲ್ಲಿ, ಅವರ ಧರ್ಮ-ಸಂಸ್ಕೃತಿಗಳ ವಿಷಯದಲ್ಲಿ, ಭಾರತದ ಆಳರಸರಾಗಿದ್ದ ಇಂಗ್ಲಿಷರು ಗೌರವಾದರ ತಾಳುವಂತೆ ಮಾಡಿದ್ದೊಂದು ಸಾಮಾನ್ಯ ಸಂಗತಿಯಲ್ಲ. ಅಲ್ಲದೆ ಆ ದೇಶದಲ್ಲಿ ಅಷ್ಟೊಂದು ಭಕ್ತಿ-ಗೌರವ-ಪ್ರೀತಿಯನ್ನು ಗೆದ್ದುಕೊಂಡ ಪ್ರಥಮ ಭಾರತೀಯರು ಸ್ವಾಮಿ ವಿವೇಕಾ ನಂದರು. ಭಾರತೀಯರ ಆತ್ಮಾಭಿಮಾನ ಜಾಗೃತಗೊಳ್ಳಲು, ಭಾರತೀಯರ ಕೆಚ್ಚು ಕೆರಳಲು ಇದೊಂದು ಮಹತ್ವದ ಅಂಶವಾಯಿತು.

ಹೀಗೆ ತಮ್ಮ ಕಾರ್ಯೋದ್ದೇಶದ ಅತಿಮುಖ್ಯ ಹಂತವನ್ನು ಮುಕ್ತಾಯಗೊಳಿಸಿದ ಸ್ವಾಮೀಜಿ, ಇನ್ನುಳಿದುದನ್ನು ಕೈಗೆತ್ತಿಕೊಳ್ಳಲು ತಮ್ಮ ತಾಯ್ನಾಡಿಗೆ ಹೊರಡಲು ಅನುವಾಗಿ ನಿಂತರು. ತಮ್ಮ ಈವರೆಗಿನ ಸಾಧನೆಯ ಬಗ್ಗೆ ಅವರಲ್ಲೊಂದು ಕೃತಕೃತ್ಯತೆಯ ಭಾವ ತುಂಬಿತ್ತು. ಅದೊಂದು ಅದ್ಭುತ ಪವಾಡವೆಂದಲ್ಲ; ಬದಲಾಗಿ, ತಮ್ಮ ಗುರುದೇವರು ತಮಗೊಪ್ಪಿಸಿದ ಕರ್ತವ್ಯವನ್ನು ತಾವು ಯಥಾಶಕ್ತಿ ನೆರವೇರಿಸಿದುದಕ್ಕಾಗಿ. ಅಲ್ಲದೆ, ಶ್ರೀರಾಮಕೃಷ್ಣರ ಶಕ್ತಿಯ ಬೀಜವು ಎಲ್ಲೆಲ್ಲಿ ಬಿತ್ತಲ್ಪಡುತ್ತದೆಯೋ ಅಲ್ಲೆಲ್ಲ ಅದು ಮೊಳೆತು, ಚಿಗುರಿ, ಮರವಾಗಿ ಬೆಳೆದು ಫಲವನ್ನು ಕೊಡಲೇಬೇಕು–ಅದು ಇಂದೇ ಆಗಬಹುದು ಅಥವಾ ಇನ್ನೊಂದು ನೂರು ವರ್ಷಗಳ ಅನಂತರ ವಾಗಬಹುದು–ಎಂದು ಸ್ವಾಮೀಜಿ ದೃಢವಾಗಿ ನಂಬಿದ್ದರು. ಜೊತೆಗೆ, ತಮ್ಮ ಸೋದರ ಸಂನ್ಯಾಸಿಗಳಾದ ಅಭೇದಾನಂದರು ಮತ್ತು ಶಾರದಾನಂದರು ಇಂಗ್ಲೆಂಡ್​-ಅಮೆರಿಕಗಳಲ್ಲಿ ತಾವು ಪ್ರಾರಂಭಿಸಿದ ಕಾರ್ಯವನ್ನು ಯಶಸ್ವಿಯಾಗಿ ಮುಂದುವರಿಸಿಕೊಂಡು ಬರುತ್ತಿದ್ದುದು ಅವರಿಗೆ ಹೆಚ್ಚಿನ ಸಮಾಧಾನಕ್ಕೆ ಕಾರಣವಾಗಿತ್ತು.

ತಮ್ಮ ಪ್ರಿಯ ಭಾರತವನ್ನು ಮತ್ತೊಮ್ಮೆ ಕಾಣುವ ಆಲೋಚನೆಯಿಂದ ಅವರ ಹೃದಯ ಹಿಗ್ಗಿತು. ಈಗ ಅವರ ಕನಸು ಮನಸುಗಳಲ್ಲೆಲ್ಲ ಭಾರತದ ಕುರಿತಾದ ಭಾವನೆಗಳೇ. ಸೇವಿಯರ್ ದಂಪತಿಗಳ ಬಳಿ ಅವರು ಅದನ್ನು ಹೇಳಿಯೂ ಬಿಟ್ಟರು; “ಈಗ ನನ್ನ ಮನಸ್ಸಿನಲ್ಲಿರುವ ಒಂದೇ ಒಂದು ಆಲೋಚನೆಯೆಂದರೆ–ಭಾರತ! ನಾನೀಗ ಇದಿರು ನೋಡುತ್ತಿರುವುದು ಭಾರತವನ್ನು– ಭಾರತವನ್ನು ಮಾತ್ರ!” ಲಂಡನ್ನಿನಿಂದ ಅವರು ಹೊರಡುವ ಮುನ್ನ ಅವರ ಆಂಗ್ಲ ಸ್ನೇಹಿತ ನೊಬ್ಬ ಕೇಳಿದ, “ಸ್ವಾಮೀಜಿ, ಈ ಸಂಪದ್ಭರಿತ, ಶಕ್ತಿಯುತ, ವಿಲಾಸಯುತ ಪಾಶ್ಚಾತ್ಯ ರಾಷ್ಟ್ರ ಗಳಲ್ಲಿ ನಾಲ್ಕು ವರ್ಷಗಳನ್ನು ಕಳೆದ ಮೇಲೆ, ಈಗ ನಿಮಗೆ ನಿಮ್ಮ ಮಾತೃಭೂಮಿಯ ಬಗ್ಗೆ ಏನೆನ್ನಿಸುತ್ತದೆ?” ಈ ಪ್ರಶ್ನೆಗೆ ಸ್ವಾಮೀಜಿ ನೀಡಿದ ಉತ್ತರವು ಅದ್ಭುತವಾದದ್ದು, ಚರಿತ್ರಾರ್ಹ ವಾದದ್ದು. ಅವರು ಮಿಂಚಿನಂತೆ ಉತ್ತರಿಸುತ್ತಾರೆ: “ನಾನು ಭಾರತದಿಂದ ಹೊರಟಾಗ ಭಾರತವನ್ನು ಪ್ರೀತಿಸುತ್ತಿದ್ದೆ. ಆದರೆ ಈಗ ನನಗೆ ಭಾರತದ ಕಣಕಣವೂ ಪವಿತ್ರವಾಗಿದೆ, ಭಾರತದ ಗಾಳಿಯೂ ಪವಿತ್ರವಾಗಿದೆ; ಈಗ ನನಗದು ಪುಣ್ಯಭೂಮಿ, ತೀರ್ಥಕ್ಷೇತ್ರ!”

ಡಿಸೆಂಬರ್ ೧೬, ೧೮೯೬; ಅಂದು ಸ್ವಾಮೀಜಿ ಸೇವಿಯರ್ ದಂಪತಿಗಳೊಂದಿಗೆ ಲಂಡನ್ನಿ ನಿಂದ ತೆರಳಲಿದ್ದರು. ಶಿಷ್ಯರ ಪೈಕಿ ಬಹಳಷ್ಟು ಜನ ಅದಾಗಲೇ ಅವರಿಗೆ ಶುಭ ಪ್ರಯಾಣವನ್ನು ಹಾರೈಸಿ ಬೇಸಿಗೆಯ ರಜಾದಿನಗಳನ್ನು ಕಳೆಯಲು ಹೊರಟುಬಿಟ್ಟಿದ್ದರು. ಆದರೆ ಸ್ವಾಮೀಜಿಯ ಆಪ್ತರಾದ ಇ. ಟಿ. ಸ್ಟರ್ಡಿ, ಮಿಸ್ ನೋಬೆಲ್, ಮಿಸ್ ಹೆನ್ರಿಟಾ ಮುಲ್ಲರ್, ಜೆ. ಜೆ. ಗುಡ್​ವಿನ್, ಸೋದರಸಂನ್ಯಾಸಿಯಾದ ಸ್ವಾಮಿ ಅಭೇದಾನಂದರು ಮತ್ತಿತರರು ರೈಲು ನಿಲ್ದಾಣದಲ್ಲಿ ನೆರೆದಿ ದ್ದರು. ಡೋವರ್​ನಲ್ಲಿ ಸ್ವಾಮೀಜಿ ಸ್ಟೀಮರ್ ಹತ್ತಿ ಇಂಗ್ಲಿಷ್ ಕಾಲುವೆಯನ್ನು ದಾಟಿ ಯೂರೋಪು ಸೇರಲಿದ್ದರು. ಕೆಲದಿನಗಳಲ್ಲೇ ಗುಡ್​ವಿನ್ನನೂ ಅವರನ್ನು ನೇಪಲ್ಸಿನಲ್ಲಿ ಕೂಡಿ ಕೊಳ್ಳಲಿದ್ದ. ಸ್ವಾಮೀಜಿ ತಮ್ಮಿಂದ ದೂರವಾಗಲಿದ್ದಾರೆಂಬ ಆಲೋಚನೆಯಿಂದ ಎಲ್ಲರ ಹೃದಯಗಳೂ ಭಾರವಾಗಿದ್ದುವು. ಆದರೆ ಶೀಘ್ರದಲ್ಲೇ ಅವರು ಮರಳಲಿದ್ದಾರೆಂದು ಮತ್ತೆಮತ್ತೆ ತಮಗೆ ತಾವೇ ಸಮಾಧಾನ ಹೇಳಿಕೊಳ್ಳುತ್ತ ಅವರಿಗೆ ಶುಭ ಕೋರಿ ವಿದಾಯ ಹೇಳಿದರು. ಪ್ರತಿ ಯೊಬ್ಬರನ್ನೂ ಸ್ವಾಮೀಜಿ ತುಂಬ ಹೃದಯದಿಂದ ಹರಸಿದರು. ಅಭೇದಾನಂದರಿಗೆ ಹೃತ್ಪೂರ್ವಕ ಆಶೀರ್ವಾದಗಳನ್ನು ನೀಡಿ, ಅವರ ಬೆನ್ನು ತಟ್ಟಿ ಸ್ಫೂರ್ತಿ ತುಂಬಿದರು. ಸ್ವಾಮೀಜಿಯ ಮನಸ್ಸೀಗ ನಿರಾಳವಾಗಿತ್ತು. ತಮ್ಮ ಪ್ರಿಯ ಭಾರತಕ್ಕೆ ಮರಳುವ, ಅಲ್ಲಿನ ತಮ್ಮ ಕಾರ್ಯವನ್ನು ಪ್ರಾರಂಭಿ ಸುವ ಆಲೋಚನೆಯೊಂದೇ ಅವರಲ್ಲಿ ತುಂಬಿತ್ತು. ಹೀಗೆ ‘ಭಾರತ ಭಾವ’ರಂಜಿತರಾದ ಸ್ವಾಮೀಜಿ ಲಂಡನ್ ನಗರಕ್ಕೆ ವಿದಾಯ ಹೇಳಿ ಭಾರತದೆಡೆಗೆ ಸಾಗಿದರು.

