
\chapter{ತ್ಯಾಗಭೂಮಿಯಿಂದ ಭೋಗಭೂಮಿಯೆಡೆಗೆ}

\noindent

ಸ್ವಾಮೀಜಿ ಅಮೆರಿಕೆಗೆ ಹೊರಡಲು ಸಿದ್ಧತೆಗಳು ಭರದಿಂದ ನಡೆದುವು. ಮೇ ೩೧ರಂದು ಹೊರಟು ಮದರಾಸಿನಿಂದ ಸಮುದ್ರಮಾರ್ಗವಾಗಿ ಪ್ರಯಾಣ ಮಾಡುವುದೆಂದು ನಿಶ್ಚಯ ವಾಯಿತು. ನಿರ್ಗಮನದ ದಿನ ಹತ್ತಿರವಾದಂತೆ ಶಿಷ್ಯವೃಂದದಲ್ಲಿ ಉತ್ಸಾಹ-ಆನಂದ-ಉದ್ವೇಗ- ಕಾತರತೆಗಳಿಂದ ಕೂಡಿದ ವಾತಾವರಣವೇರ್ಪಟ್ಟಿತು. ಹೀಗಿರುವಾಗ ಇದ್ದಕ್ಕಿದ್ದಂತೆ ಈ ಪರಿಸ್ಥಿತಿ ಬೇರೊಂದು ರೀತಿಯಲ್ಲಿ ತಿರುಗಿಕೊಂಡಿತು. ಖೇತ್ರಿಯ ಮಹಾರಾಜ ಅಜಿತ್​ಸಿಂಗನ ಆಪ್ತ ಕಾರ್ಯದರ್ಶಿಯಾದ ಮುನ್ಷಿ ಜಗಮೋಹನ ಲಾಲ್ ಆಕಾಶದಿಂದಲೋ ಎಂಬಂತೆ ಅಲ್ಲಿಗೆ ಬಂದಿಳಿದ.

ಸುಮಾರು ಎರಡು ವರ್ಷಗಳ ಹಿಂದೆ ಸ್ವಾಮೀಜಿ ಖೇತ್ರಿಯಲ್ಲಿದ್ದಾಗ, ಪುತ್ರ ಸಂತಾನ ಭಾಗ್ಯದಿಂದ ವಂಚಿತನಾಗಿದ್ದ ಅಜಿತ್​ಸಿಂಗ್, ತನಗೆ ಗಂಡುಮಗುವಾಗುವಂತೆ ಹರಸಬೇಕೆಂದು ಅವರನ್ನು ಪ್ರಾರ್ಥಿಸಿಕೊಂಡಿದ್ದ. ಅವನ ಕಳಕಳಿಯ ಪ್ರಾರ್ಥನೆಗೆ ಓಗೊಟ್ಟು ಸ್ವಾಮೀಜಿ ಅದರಂತೆಯೇ ಹರಸಿದ್ದರು. ಅವರ ಆಶೀರ್ವಾದದ ಫಲವಾಗಿ ೧೮೯೨ರ ಜನವರಿಯಲ್ಲಿ ಅವರಿ ಗೊಂದು ಗಂಡುಮಗು ಜನಿಸಿತ್ತು. ಮಗುವಿಗೆ ಜೈಸಿಂಗ್ ಎಂದು ನಾಮಕರಣ ಮಾಡಿದ್ದರು. ತನಗೊಬ್ಬ ಉತ್ತರಾಧಿಕಾರಿ ಜನಿಸಿದ್ದರಿಂದ ಅತ್ಯಂತ ಆನಂದಗೊಂಡಿದ್ದ ಮಹಾರಾಜ ಸಮಾರಂಭವೊಂದನ್ನು ವಿಜೃಂಭಣೆಯಿಂದ ಆಚರಿಸಲು ನಿಶ್ಚಯಿಸಿದ್ದ. ‘ಈ ಸಂತಸದ ಘಳಿಗೆ ಯಲ್ಲಿ ಸ್ವಾಮೀಜಿ ನಮ್ಮೊಂದಿಗಿರುವಂತಾದರೆ!’ ಎಂದು ಹಂಬಲಿಸಿದ. ಸ್ವಾಮೀಜಿ ಮದರಾಸಿ ನಲ್ಲಿದ್ದಾರೆ, ಹಾಗೂ ಅವರ ಶಿಷ್ಯರು ಅವರನ್ನು ಅಮೆರಿಕೆಗೆ ಕಳಿಸಿಕೊಡಲು ನಿಧಿಸಂಗ್ರಹಣೆಯಲ್ಲಿ ತೊಡಗಿದ್ದಾರೆ ಎಂಬ ಸುದ್ದಿ ತಿಳಿದುಬಂದಿತು. ಆದ್ದರಿಂದ ಹೇಗಾದರೂ ಮಾಡಿ ಅವರನ್ನು ತಕ್ಷಣ ಕರೆತರುವಂತೆಯೂ ಅಮೆರಿಕ ಪ್ರಯಾಣದ ಖರ್ಚಿನ ಬಗ್ಗೆ ಚಿಂತಿಸಬಾರದೆಂದು ಅವರಿಗೆ ತಿಳಿಸುವಂತೆಯೂ ಹೇಳಿ, ಸ್ವತಃ ತನ್ನ ಆಪ್ತಕಾರ್ಯದರ್ಶಿಯನ್ನೇ ಕಳಿಸಿಕೊಟ್ಟಿದ್ದ.

ಮದರಾಸಿಗೆ ಬಂದಿಳಿದ ಜಗಮೋಹನನಿಗೆ ಸ್ವಾಮೀಜಿ ಮನ್ಮಥನಾಥರ ಮನೆಯಲ್ಲಿರುವುದು ತಿಳಿಯಿತು. ಬಹಳ ಕಷ್ಟಪಟ್ಟು ಅವರ ಮನೆಯನ್ನು ಹುಡುಕಿಕೊಂಡು ಬಂದ. ಸ್ವಾಮೀಜಿ ಸಿಗುತ್ತಾರೋ ಇಲ್ಲವೋ ಎಂಬ ಆತಂಕವಿದ್ದೇ ಇತ್ತೆಂದು ಕಾಣುತ್ತದೆ. ಬರುತ್ತಿದ್ದಂತೆಯೇ ಅಲ್ಲಿದ್ದ ಸೇವಕನನ್ನು “ಸ್ವಾಮೀಜಿಯವರೆಲ್ಲಿ?” ಎಂದು ಕೇಳಿದ ಅದಕ್ಕವನು “ಸಮುದ್ರಕ್ಕೆ ಹೋಗಿದ್ದಾರೆ” ಎಂದು ಉತ್ತರಿಸಿದ. ಜಗಮೋಹನನಿಗೆ ಸೇವಕ ಹೇಳಿದ್ದು ಅರ್ಥವಾಗಲಿಲ್ಲ. ಗಲಿಬಿಲಿಗೊಂಡು ಕೇಳಿದ, “ಏನು! ಹಾಗಾದರೆ ಅವರು ಅಮೆರಿಕೆಗೆ ಹೊರಟುಬಿಟ್ಟರೆ?” ಅಷ್ಟ ರಲ್ಲಿ ಕಾವಿ ವಸ್ತ್ರವೊಂದನ್ನು ಮೊಳಗೆ ತಗಲಿಹಾಕಿದ್ದು ಕಂಡು, ಅವನಿಗೆ ಹೋದ ಜೀವ ಬಂದಂತಾ ಯಿತು. ‘ಇಲ್ಲ ಅವರು ಹೋಗಿರಲು ಸಾಧ್ಯವಿಲ್ಲ’ ಎಂದು ತನ್ನಷ್ಟಕ್ಕೆ ಹೇಳಿಕೊಂಡ.

ಆ ಹೊತ್ತಿಗೆ ಸರಿಯಾಗಿ ಮನೆಯ ಮುಂದೆ ಸಾರೋಟು ಬಂದು ನಿಂತ ಶಬ್ದ ಕೇಳಿತು. ತಮ್ಮ ಆತಿಥೇಯರೊಂದಿಗೆ ಸಮುದ್ರತೀರಕ್ಕೆ ಹೋಗಿದ್ದ ಸ್ವಾಮೀಜಿ ಕೆಳಗಿಳಿದರು. ತಕ್ಷಣವೇ ಜಗ ಮೋಹನ ಅವರ ಪಾದಗಳಿಗೆ ಸಾಷ್ಟಾಂಗ ಪ್ರಣಾಮ ಮಾಡಿದ. ಅನಿರೀಕ್ಷಿತವಾಗಿ ಅವನನ್ನು ಕಂಡು ಸ್ವಾಮೀಜಿಗೆ ಆಶ್ಚರ್ಯವಾಯಿತು. ತಾನು ಬಂದ ಸಮಾಚಾರವನ್ನು ಜಗಮೋಹನ ವಿಸ್ತಾರವಾಗಿ ತಿಳಿಸಿ, ತನ್ನೊಂದಿಗೆ ಹೊರಡುವಂತೆ ಅವರನ್ನು ಕೇಳಿಕೊಂಡ. ಆಗ ಸ್ವಾಮೀಜಿ, “ಜಗಮೋಹನ್, ಅಲ್ಲಿಗೆ ಬರಲು ನಿಜಕ್ಕೂ ನನಗೆ ತುಂಬ ಇಷ್ಟವಿದೆ. ಆದರೆ ನಾನು ಅಮೆರಿಕೆಗೆ ಹೊರಡಲು ಇನ್ನು ಒಂದೂವರೆ ತಿಂಗಳು ಮಾತ್ರ ಉಳಿದಿದೆ. ಅಷ್ಟರಲ್ಲೇ ಇನ್ನೂ ಎಷ್ಟೋ ಸಿದ್ಧತೆಗಳಾಗಬೇಕಿದೆ. ಹೀಗಿರುವಾಗ ನಾನು ಅಷ್ಟು ದೂರ ಹೇಗೆ ಬರಲಿ! ನಾನು ಮತ್ತೆಂದಾದರೂ ಖಂಡಿತ ಬರು ತ್ತೇನೆ” ಎಂದರು. ಆದರೆ ಜಗಮೋಹನಲಾಲ ಅವರನ್ನು ಕರೆದುಕೊಂಡೇ ಹೋಗಬೇಕೆಂದು ಬಂದಿದ್ದವನು. ಆದ್ದರಿಂದ ಪಟ್ಟುಬಿಡದೆ, “ಸ್ವಾಮೀಜಿ, ನೀವು ಖೇತ್ರಿಗೆ ಬರಲೇಬೇಕು–ಕಡೆಗೆ ಒಂದೇ ಒಂದು ದಿನದ ಮಟ್ಟಿಗಾದರೂ ಸರಿಯೆ. ನೀವೇನಾದರೂ ಬರದೆಯಿದ್ದರೆ ಮಹಾರಾಜರಿಗೆ ತಡೆಯಲಾರದ ದುಃಖ-ನಿರಾಶೆಯಾಗುತ್ತದೆ... ಅಮೆರಿಕೆಗೆ ಹೊರಡುವುದರ ಬಗ್ಗೆ ನೀವು ಏನೇನೂ ಚಿಂತಿಸಬೇಕಿಲ್ಲ. ನಮ್ಮ ಮಹಾರಾಜರು ಅದೆಲ್ಲವನ್ನೂ ನೋಡಿಕೊಳ್ಳುತ್ತಾರೆ. ಈಗ ಸುಮ್ಮನೆ ನನ್ನೊಂದಿಗೆ ಬಂದುಬಿಡಿ” ಎಂದು ಬಲವಂತ ಮಾಡಿದ. ಕಟ್ಟಕಡೆಗೆ ಸ್ವಾಮೀಜಿ ಒಪ್ಪಿಕೊಳ್ಳಲೇಬೇಕಾಯಿತು. ತಮ್ಮ ಪ್ರಯಾಣದ ಖರ್ಚಿಗಾಗಿ ಶಿಷ್ಯರು ಮೊದಲ ಸಲ ಸಂಗ್ರಹಿ ಸಿದ ವಂತಿಗೆ ಹಣ ಬಹಳ ಕಡಿಮೆಯಿದ್ದುದನ್ನು ಕಂಡಾಗ ಸ್ವಾಮೀಜಿ, ಉತ್ತರ ಭಾರತಕ್ಕೆ ಹೋಗಿ ತಮ್ಮ ಅನುಯಾಯಿಗಳಾದ ರಾಜರನ್ನೆಲ್ಲ ಮತ್ತೆ ಕಾಣುವ ಬಗ್ಗೆ ಆಲೋಚಿಸಿದ್ದರು. ಆದರೆ ಅಲ್ಲಿಗೆ ಹೋದರೆ ಅವರಿಂದ ಬಿಡಿಸಿಕೊಂಡು ಬರುವುದೇ ಕಷ್ಟವಾಗಬಹುದೆಂದು ಊಹಿಸಿ, ಆ ಆಲೋಚನೆಯನ್ನು ಕೈಬಿಟ್ಟಿದ್ದರು. ಈಗ ಮತ್ತೆ ಅಲ್ಲಿ ಹೋಗಲೇಬೇಕಾದ ಪರಿಸ್ಥಿತಿಯೊದಗಿದೆ! ಅವರ ಅಮೆರಿಕ ಪ್ರಯಾಣದ ಬಗ್ಗೆ ಮತ್ತೆ ಮಾತುಕತೆ ನಡೆದು, ಮದರಾಸಿನ ಬದಲು ಮುಂಬಯಿ ಯಿಂದ ಹೊರಡುವುದೆಂದು ನಿರ್ಧಾರವಾಯಿತು.

ಈಗ ಸ್ವಾಮೀಜಿ ಮದರಾಸಿನಿಂದ ಖೇತ್ರಿಗೆ ಹೊರಟುನಿಂತರು. ಹೊರಡುವ ದಿನ ಅತ್ಯಂತ ಹೃದಯಸ್ಪರ್ಶಿಯಾದ ದೃಶ್ಯವೇರ್ಪಟ್ಟಿತು. ಶಿಷ್ಯರೆಲ್ಲ ಸ್ವಾಮೀಜಿಯ ಪಾದಗಳಿಗೆ ಸಾಷ್ಟಾಂಗ ಪ್ರಣಾಮ ಮಾಡಿ, ಆಶೀರ್ವದಿಸುವಂತೆ ಬೇಡಿಕೊಂಡರು. ಎಲ್ಲರನ್ನೂ ಸ್ವಾಮೀಜಿ ಹೃತ್ಪೂರ್ವಕ ವಾಗಿ ಹರಸಿದರು. ಅವರೊಂದಿಗೆ ಎಲ್ಲರೂ ರೈಲು ನಿಲ್ದಾಣಕ್ಕೆ ಬಂದರು. ಟ್ರೈನು ಹೊರಡುತ್ತಿ ದ್ದಂತೆ ಶಿಷ್ಯರೆಲ್ಲ ಆಕಾಶ ಮಾರ್ದನಿಸುವಂತೆ ಜಯಕಾರ ಮಾಡಿದರು. ಸ್ವಾಮೀಜಿ ಬಾಗಿಲಲ್ಲಿ ನಿಂತು ಆಶೀರ್ವಾದಪೂರ್ವಕವಾಗಿ ಬಲಗೈಯನ್ನು ಬೀಸುತ್ತಿದ್ದಂತೆ ಟ್ರೈನು ಸಾಗಿ ಕಣ್ಮರೆ ಯಾಯಿತು. ಶಿಷ್ಯರು ಕಣ್ಣೊರೆಸಿಕೊಂಡು, ಬರಿದಾದ ಹೃದಯದೊಂದಿಗೆ ಹಿಂದಿರುಗಿದರು.

ಖೇತ್ರಿಗೆ ಹೋಗುವ ದಾರಿಯಲ್ಲಿ ಸ್ವಾಮೀಜಿ, ಮುಂಬಯಿ ಹಾಗೂ ವಾಪಿಂಗನದಲ್ಲಿ ಕೆಲದಿನ ಗಳ ಮಟ್ಟಿಗೆ ಇಳಿದುಕೊಂಡರು. ಮುಂಬಯಿಯಲ್ಲಿ ಅವರು ಅಮೆರಿಕೆಗೆ ಹೋಗುವ ಹಡಗಿನಲ್ಲಿ ತಮ್ಮ ಸ್ಥಳವನ್ನು ಕಾದಿರಿಸಿಕೊಳ್ಳಬೇಕಾದ ಕೆಲಸವಿತ್ತು. ಇಲ್ಲಿ ಅವರು ಒಬ್ಬ ಪಂಡಿತನ ಮನೆ ಯಲ್ಲಿ ಇಳಿದುಕೊಂಡರು. ಅದೇ ಸಮಯಕ್ಕೆ ಕರಾಚಿಯಿಂದ ದೋಣಿಯಲ್ಲಿ ಮುಂಬಯಿಗೆ ಬಂದಿದ್ದ ಬ್ರಹ್ಮಾನಂದರೂ ತುರೀಯಾನಂದರೂ ಹೇಗೋ ಸ್ವಾಮೀಜಿಯ ಇರವನ್ನು ಕಂಡು ಕೊಂಡು ಪಂಡಿತನ ಮನೆಗೆ ಬಂದುಬಿಟ್ಟರು! ಅನಿರೀಕ್ಷಿತವಾಗಿ ತಮ್ಮ ಪ್ರಿಯ ಗುರುಭಾಯಿ ಗಳನ್ನು ಕಂಡು ಸ್ವಾಮೀಜಿ ಆನಂದಾಶ್ಚರ್ಯಗೊಂಡರು. ಬಳಿಕ, ಇವರಿಬ್ಬರನ್ನೂ ತಮ್ಮ ಆತಿಥೇಯನಾದ ಪಂಡಿತ ಅಷ್ಟೊಂದು ಆದರಿಂದ ನೋಡಿಕೊಳ್ಳಲಾರನೆಂದು ಊಹಿಸಿ ಅವರೊಂದಿಗೆ ಕಾಳೀಪದ ಘೋಷ್ ಎಂಬವನ ಮನೆಗೆ ಹೋದರು. ಶ್ರೀರಾಮಕೃಷ್ಣರ ಗೃಹೀ ಭಕ್ತನಾದ ಕಾಳೀಪದ ಮುಂಬಯಿಯಲ್ಲಿ ಉದ್ಯೋಗದಲ್ಲಿದ್ದ. ಈ ಮೂವರನ್ನೂ ಅವನು ಅತ್ಯಾದರದಿಂದ ಸ್ವಾಗತಿಸಿದ. ಅವರಿಗೆ ಮುಂಬಯಿಯ ಅನೇಕ ಪ್ರೇಕ್ಷಣೀಯ ಸ್ಥಳಗಳನ್ನು ತೋರಿಸಿದ. ತಾವು ಅಮೆರಿಕೆಗೆ ಹೋಗಲಿರುವ ವಿಷಯವನ್ನು ತಮ್ಮ ಗುರುಭಾಯಿಗಳಿಗೆ ತಿಳಿಸಿದ ಸ್ವಾಮೀಜಿ, ತುರೀಯಾನಂದರಿಗೆ ಹೇಳುತ್ತಾರೆ, “ಹರಿಭಾಯ್, ಅಲ್ಲಿ ಏನೇನು ನಡೆಯುತ್ತಿದೆ ಯೆಂದು ನೀನು ಕೇಳಿದ್ದೀಯೋ ಅವೆಲ್ಲ ಇದಕ್ಕಾಗಿ. ಇದಕ್ಕಾಗಿಯೇ ಅಲ್ಲಿ ಪ್ರತಿಯೊಂದು ವ್ಯವಸ್ಥೆಯೂ ಆಗುತ್ತಿದೆ!” ‘ಇದಕ್ಕಾಗಿ’ ಎನ್ನುವಾಗ ಸ್ವಾಮೀಜಿ ತಮ್ಮ ಎದೆಯನ್ನು ತಟ್ಟಿ ತೋರಿಸಿಕೊಳ್ಳುತ್ತಾರೆ. ಎಂದರೆ ಅಮೆರಿಕೆಯಲ್ಲಿ ವಿಶ್ವಧರ್ಮ ಸಮ್ಮೇಳನವೆಂಬುದು ಏರ್ಪಾಟಾಗುತ್ತಿರುವುದು ತಮಗಾಗಿಯೇ; ತಾವು ಆ ವಿಶ್ವವೇದಿಕೆಯ ಮೇಲೆ ನಿಂತು ಸರ್ವಧರ್ಮ ಗಳ ಸಮನ್ವಯದ ಸಂದೇಶವನ್ನೂ ಸನಾತನ ಹಿಂದೂಧರ್ಮದ ಘನತೆಯನ್ನೂ ಸಾರುವು ದಕ್ಕಾಗಿಯೇ ಎಂಬುದು ಅವರ ಮಾತಿನ ಅರ್ಥ. ಇದೀಗ ಅವರ ಆತ್ಮವಿಶ್ವಾಸ ಎಷ್ಟರ ಮಟ್ಟಿಗೆ ಜಾಗೃತವಾಗಿದೆಯೆಂಬುದನ್ನು ನಾವಿಲ್ಲಿ ನೋಡಬಹುದು.

ಸ್ವಾಮೀಜಿ ಕಾಳೀಪದ ಘೋಷನ ಮನೆಯಲ್ಲಿದ್ದಾಗ, ಪ್ರತಿದಿನವೂ ಹಲವಾರು ಜನ ಅವರ ದರ್ಶನಾರ್ಥಿಗಳಾಗಿ ಬರುತ್ತಿದ್ದರು. ಇವರೊಂದಿಗೆ ಸ್ವಾಮೀಜಿ ಧಾರ್ಮಿಕ-ಆಧ್ಯಾತ್ಮಿಕ ವಿಚಾರಗಳ ಕುರಿತಾಗಿ ಸಂಭಾಷಣೆ, ಪ್ರವಚನಗಳನ್ನು ನಡೆಸುತ್ತಿದ್ದರು. ಆದರೆ ಒಂದು ಸಂಜೆ ಅವರಿಗೆ ಸ್ವಲ್ಪ ಅನಾರೋಗ್ಯವಾಗಿತ್ತು. ಆದ್ದರಿಂದ ಅಂದಿನ ಮಟ್ಟಿಗೆ ತಮ್ಮ ಬದಲು ಮಾತನಾಡುವಂತೆ ತುರೀಯಾನಂದರನ್ನು ಕೇಳಿಕೊಂಡರು. ತುರೀಯಾನಂದರು ಒಲ್ಲದ ಮನಸ್ಸಿನಿಂದಲೇ ಒಪ್ಪಿ ಕೊಂಡು, ಅಂದು ಕೆಲವು ಮಾತುಗಳನ್ನಾಡಿದರು. ಅವರು ತ್ಯಾಗ-ವೈರಾಗ್ಯಕ್ಕೆ ವಿಶೇಷ ಒತ್ತುಕೊಟ್ಟು ಮಾತನಾಡಿರಬೇಕು; ಆದ್ದರಿಂದ ಆಮೇಲೆ ಸ್ವಾಮೀಜಿ ಅವರಿಗೆ ಹೇಳಿದರು, “ಹರಿಭಾಯ್, ಏನಿದು! ಅವರೆಲ್ಲ ಗೃಹಸ್ಥರು ಎಂಬುದನ್ನು ಮರೆತುಬಿಟ್ಟೆಯಾ? ನೀನು ಅವರಿಗೆ ಸಂಪೂರ್ಣ ತ್ಯಾಗ-ವೈರಾಗ್ಯಗಳ ಬಗ್ಗೆ ಹೇಳಿಬಿಟ್ಟೆಯಲ್ಲ! ನೀನೊಬ್ಬ ಸಂನ್ಯಾಸಿಯಾಗಿರಬಹುದು, ಆದರೆ ಅವರೆಲ್ಲ ಸಂಸಾರಸ್ಥರು. ಅವರಿಗೆ ಉಪಯೋಗವಾಗುವಂತಹ ಇನ್ನೇನಾದರೂ ವಿಷಯಗಳನ್ನು ಹೇಳಬೇಕಾಗಿತ್ತು. ನೀನು ಹೇಳಿದ್ದನ್ನು ಕೇಳಿ ಅವರೆಲ್ಲ ಹೆದರಿ ಕಂಗಾಲಾಗಿಬಿಡುತ್ತಾರೆ! ಆದ್ದರಿಂದ ಅವರಿಗೆ ಸುಲಭವಾಗಿ ಎಟುಕುವಂತಹ ವಿಚಾರಗಳನ್ನು ತಿಳಿಸಿಕೊಡಬೇಕಿತ್ತು.” ಸ್ವಾಮೀಜಿ ಹೀಗೆ ಆಕ್ಷೇಪಿಸಿದಾಗ ತುರೀಯಾನಂದರು ತಮ್ಮ ಕಷ್ಟವನ್ನು ವಿವರಿಸಿದರು, “ಹೌದು, ನೀನು ಹೇಳು ವುದು ಸರಿಯೇ. ಆದರೆ ಏನಾಯಿತೆಂದರೆ, ನಾನು ಹೇಳುತ್ತಿರುವುದನ್ನೆಲ್ಲ ನೀನು ಕೇಳಿಸಿಕೊಳ್ಳುತ್ತಿ ದ್ದೀಯೆ ಎಂಬ ಭಯವಿತ್ತು. ಆದ್ದರಿಂದ ಇವರಿಗೆ ಸಣ್ಣಪುಟ್ಟ ವಿಚಾರಗಳನ್ನೆಲ್ಲ ಹೇಳಬಾರದು, ಸ್ವಲ್ಪ ತೂಕವಾದ ವಿಷಯಗಳನ್ನೇ ಹೇಳಬೇಕು ಎಂದು ಆಲೋಚಿಸಿ ಹಾಗೆ ಮಾತನಾಡಿದೆ ಅಷ್ಟೆ!”

ಮುಂಬಯಿಯಲ್ಲಿ ಕೆಲದಿನಗಳಿದ್ದು, ಸ್ವಾಮೀಜಿ ಜಗಮೋಹನನೊಂದಿಗೆ ಖೇತ್ರಿಗೆ ಹೊರ ಟರು. ಬ್ರಹ್ಮಾನಂದರೂ ತುರೀಯಾನಂದರೂ ಆಬು ರಸ್ತೆಯ ನಿಲ್ದಾಣದವರೆಗೂ ಬಂದು, ಅಲ್ಲಿಂದ ತಮ್ಮ ದಾರಿ ತಾವು ಹಿಡಿದರು. ಜೈಪುರದವರೆಗೂ ಸ್ವಾಮೀಜಿ ಟ್ರೈನಿನಲ್ಲಿ ಪ್ರಯಾಣ ಮಾಡಿದರು. ಉಳಿದ ದೂರವನ್ನು ಗಾಡಿಯಲ್ಲಿ ಕ್ರಮಿಸಿ, ಏಪ್ರಿಲ್ ೨೧ರಂದು ಸಂಜೆಯ ಹೊತ್ತಿಗೆ ಖೇತ್ರಿ ನಗರವನ್ನು ತಲುಪಿದರು. ಆಗ ಖೇತ್ರಿಯಲ್ಲಿ ಅತ್ಯಂತ ವಿಜೃಂಭಣೆಯಿಂದ ಉತ್ಸವ ನಡೆಯುತ್ತಿತ್ತು. ಇಡೀ ಅರಮನೆಯೇ ದೀಪಗಳಿಂದ ಬೆಳಗುತ್ತಿತ್ತು. ನವಜಾತ ರಾಜಕುಮಾರನನ್ನು ಸ್ವಾಗತಿಸಲು ಸಮಸ್ತ ನಗರವೇ ಸಿಂಗರಿಸಿಕೊಂಡು ಸಿದ್ಧವಾದಂತಿತ್ತು. ಎಲ್ಲೆಲ್ಲೂ ಸಂಗೀತ-ನೃತ್ಯ- ರಂಜನೆಗಳು ನಡೆಯುತ್ತಿದ್ದುವು. ಸ್ವಾಮೀಜಿಯನ್ನು ಹೊತ್ತ ಸಾರೋಟು ಅರಮನೆಯ ಬಳಿಗೆ ಬರುತ್ತಿದ್ದಂತೆಯೇ ಅವರಿಗೆ ಭವ್ಯ ಸ್ವಾಗತ ನೀಡಲಾಯಿತು. ಖೇತ್ರಿ ರಾಜ್ಯದ ಹಾಗೂ ನೆರೆರಾಜ್ಯ ಗಳ ಹಲವಾರು ಪ್ರಮುಖ ವ್ಯಕ್ತಿಗಳಿಂದ ಆವೃತನಾಗಿ, ಸಾಲಂಕೃತ ಪೀಠದಲ್ಲಿ ಆಸೀನನಾಗಿದ್ದ ರಾಜಾ ಅಜಿತ್​ಸಿಂಗ್, ಸ್ವಾಮೀಜಿಯ ಬರವನ್ನೇ ನಿರೀಕ್ಷಿಸುತ್ತ ಕುಳಿತಿದ್ದ. ಮುನ್ಷಿ ಜಗಮೋಹನ ಲಾಲ್ ಅವರನ್ನು ಸಾಂಪ್ರದಾಯಿಕವಾಗಿ ಕರೆತಂದಾಗ, ರಾಜ ತನ್ನ ಪರಮಪೂಜ್ಯ ಗುರುವಿಗೆ ಉದ್ದಂಡ ಪ್ರಣಾಮ ಮಾಡಿದ. ಸ್ವಾಮೀಜಿ ಅವನನ್ನು ಆಶೀರ್ವದಿಸಿ ಮೇಲೆತ್ತಿದರು. ಬಳಿಕ ಅಲ್ಲಿ ಉಪಸ್ಥಿತರಿದ್ದವರೆಲ್ಲ ಅವರಿಗೆ ಬಾಗಿ ನಮಿಸಿದರು. ಅನಂತರ ಸ್ವಾಮೀಜಿಯನ್ನು ಅವರಿಗಾಗಿ ಕಾದಿರಿಸಿದ್ದ ಗೌರವಾಸನದ ಬಳಿಗೆ ಕರೆದೊಯ್ಯಲಾಗುತ್ತಿದ್ದಂತೆ ಆಸ್ಥಾನ ಗಾಯಕರು ಸ್ವಾಗತಗೀತೆ ಯೊಂದನ್ನು ಹಾಡಿದರು. ಬಳಿಕ ರಾಜ ಅಲ್ಲಿ ನೆರೆದಿದ್ದ ಮಹಾಸಭಿಕರಿಗೆ ಔಪಚಾರಿಕವಾಗಿ ಸ್ವಾಮೀಜಿಯ ಪರಿಚಯ ಮಾಡಿಕೊಟ್ಟ. ಅವರ ಅಪಾರ ಕೃಪೆಯ ಫಲವಾಗಿಯೇ ತನಗೆ ಈ ಗಂಡುಸಂತಾನವಾಯಿತು ಎಂದು ಹೇಳಿದ. ಅಲ್ಲದೆ, ಸನಾತನಧರ್ಮವನ್ನು ಪ್ರಸಾರ ಮಾಡುವುದ ಕ್ಕಾಗಿ ಅವರು ಅಮೆರಿಕೆಗೆ ಹೋಗಲಿದ್ದಾರೆ ಎಂಬುದನ್ನು ತಿಳಿಸಿದ. ಇದನ್ನು ಕೇಳಿ ಸಭಿಕರೆಲ್ಲ ಸಂತೋಷದಿಂದ ಕರತಾಡನ ಮಾಡಿದರು. ಬಳಿಕ ಸ್ವಾಮೀಜಿಯವರಿಂದ ಆಶೀರ್ವಾದ ಮಾಡಿ ಸಲು ಮಗುವನ್ನು ಅಂತಃಪುರದಿಂದ ಕರೆತರಲಾಯಿತು. ರಾಜ ಆನಂದದಲ್ಲಿ ಮೈಮರೆತಿದ್ದ.

ಸ್ವಾಮೀಜಿ ಖೇತ್ರಿಯಲ್ಲಿ ಈ ಸಲ ಸುಮಾರು ಮೂರು ವಾರಗಳವರೆಗೆ ಉಳಿದುಕೊಂಡರು. ರಾಜನೊಂದಿಗೆ ಹಲವಾರು ವಿಷಯಗಳ ಬಗ್ಗೆ ಮಾತನಾಡುತ್ತ ತಮ್ಮ ಹೆಚ್ಚಿನ ಸಮಯವನ್ನು ಕಳೆದರು. ಅವರನ್ನು ಬೀಳ್ಗೊಡಲು ಅಜಿತ್​ಸಿಂಗನಿಗೆ ಇಷ್ಟವೇ ಇಲ್ಲ. ಇನ್ನಷ್ಟು ದಿನ ಸ್ವಾಮೀಜಿ ಇಲ್ಲಿಯೇ ಇರಲಿ ಎಂದು ಅವನ ಬಯಕೆ. ಕಡೆಗೆ ಸ್ವಾಮೀಜಿ “ನಾನಿನ್ನು ಮುಂಬಯಿಗೆ ಹೊರಡಲೇ ಬೇಕು. ಅಮೆರಿಕೆಗೆ ಹೋಗಲು ನಾನಿನ್ನೂ ಸಿದ್ಧತೆಗಳನ್ನು ಮಾಡಿಕೊಳ್ಳಬೇಕಾಗಿದೆ” ಎಂದಾಗ ಅವ ರನ್ನು ಕಳಿಸಿಕೊಡಲು ಒಪ್ಪಲೇಬೇಕಾಯಿತು. ಅವರೊಂದಿಗೆ ತಾನೂ ಜೈಪುರದವರೆಗೆ ಹೋಗಲು ಅವನು ಇಚ್ಛಿಸಿದ. ಆದರೆ ಹಾಗೆ ಮಾಡದಿರುವಂತೆ ಸ್ವಾಮೀಜಿ ಅವನ ಮನವೊಲಿಸಿದರು.

ಮೇ ೧೦ರಂದು ಬೀಳ್ಕೊಡುಗೆಯ ಸಮಾರಂಭವೊಂದನ್ನು ಆಚರಿಸಲಾಯಿತು. ಬಳಿಕ ಸ್ವಾಮೀಜಿ, ಮುನ್ಷಿ ಜಗಮೋಹನಲಾಲ್​ನೊಂದಿಗೆ ಗಾಡಿಯಲ್ಲಿ ಹೊರಟರು. ಮುಂಬಯಿಯ ವರೆಗೂ ಹೋಗಿ, ಅಲ್ಲಿಂದ ಸ್ವಾಮೀಜಿ ಅಮೆರಿಕೆಗೆ ಹೊರಡಲು ಬೇಕಾದ ಸಕಲ ವ್ಯವಸ್ಥೆಗಳನ್ನೂ ಮಾಡಿಬರುವಂತೆ ಮಹಾರಾಜನು ಜಗಮೋಹನನಿಗೆ ತಿಳಿಸಿದ್ದ. ಇನ್ನು ಮುಂದೆ ತಮ್ಮನ್ನು ‘ಸಚ್ಚಿದಾನಂದ’, ‘ವಿವಿದಿಶಾನಂದ’ ಮೊದಲಾದ ಹೆಸರುಗಳಿಗೆ ಬದಲಾಗಿ, ‘ಸ್ವಾಮಿ ವಿವೇಕಾ ನಂದ’ ಎಂಬ ಹೆಸರಿನಿಂದಲೇ ಕರೆದುಕೊಳ್ಳುವಂತೆ ಅಜಿತ್​ಸಿಂಗ್ ಅವರನ್ನು ಬೇಡಿಕೊಂಡಿದ್ದ. ಅದಕ್ಕೊಪ್ಪಿ ಸ್ವಾಮೀಜಿ, ತಮ್ಮನ್ನು ವಿವೇಕಾನಂದರೆಂಬ ಹೆಸರಿನಿಂದಲೇ ಕರೆದುಕೊಳ್ಳಲಾರಂಭಿಸಿ ದರು. ಮತ್ತು ಆ ಹೆಸರಿನಿಂದಲೇ ವಿಶ್ವವಿಖ್ಯಾತರಾದರು.

ಮುಂಬಯಿಗೆ ಹೋಗುವ ದಾರಿಯಲ್ಲಿ ಸ್ವಾಮೀಜಿ, ಆಬು ರಸ್ತೆಯ ನಿಲ್ದಾಣದ ಬಳಿ, ಹಿಂದೆ ಪರಿವ್ರಾಜಕ ದಿನಗಳಲ್ಲಿ ತಮ್ಮ ಆತಿಥೇಯನಾಗಿದ್ದ ಒಬ್ಬ ರೈಲ್ವೆ ನೌಕರನ ಮನೆಯಲ್ಲಿ ಒಂದು ರಾತ್ರಿ ಉಳಿದುಕೊಂಡರು. ಸ್ವಾಮೀಜಿ ಈ ದಾರಿಯಾಗಿ ಹೋಗುತ್ತಾರೆಂದು ತಿಳಿದು ಬ್ರಹ್ಮಾ ನಂದರೂ ತುರೀಯಾನಂದರೂ ಮೌಂಟ್ ಅಬುವಿನಿಂದ ಇಷ್ಟು ದೂರ ಪ್ರಯಾಣ ಮಾಡಿ ಬಂದಿದ್ದರು. ಮರುದಿನ ಅಲ್ಲಿಂದ ಹೊರಡುವ ಮುನ್ನ ಸ್ವಾಮೀಜಿ ತುರೀಯಾನಂದರಿಗೆ “ರಾಖಾಲನನ್ನು ಅವನಷ್ಟಕ್ಕೆ ಬಿಟ್ಟು ನೀನು ಮಠಕ್ಕೆ ಹಿಂದಿರುಗು. ಅಲ್ಲಿ ಗುರುಮಹಾರಾಜರ ಕಾರ್ಯ ಮಾಡುತ್ತ ಮಠದ ಅಭಿವೃದ್ಧಿಯ ಕಡೆಗೆ ಗಮನ ಕೊಡು” ಎಂದು ಆಜ್ಞೆ ಮಾಡಿದರು. ತಾವು ಪರಿವ್ರಾಜಕರಾಗಿ ಸುತ್ತುತ್ತಿದ್ದರೂ ಸ್ವಾಮೀಜಿಗೆ ಮಠದ ಕುರಿತಾದ ಚಿಂತೆ ಇದ್ದೇಇತ್ತು ಎಂಬುದು ಇಲ್ಲಿ ವ್ಯಕ್ತವಾಗುತ್ತದೆ.

ತಮ್ಮ ಗುರುಭಾಯಿಗಳೊಂದಿಗೆ ಮಾತನಾಡುತ್ತ ಸ್ವಾಮೀಜಿ ತಮ್ಮ ಅಂತರಂಗದಲ್ಲಿ ದಾವಾ ನಲದಂತೆ ಕುದಿಯುತ್ತಿದ್ದ ಭಾವನೆಗಳನ್ನು ಹೊರಗೆಡವಿದರು. ಅವರು ಅಮೆರಿಕೆಗೆ ಹೋಗುವ ಮೊದಲು ಅವರನ್ನು ಕಡೆಯಬಾರಿ ಭೇಟಿ ಮಾಡಿದ ಈ ಸಂದರ್ಭವನ್ನು ನೆನಪಿಸಿಕೊಂಡು ಮುಂದೊಮ್ಮೆ ಸ್ವಾಮಿ ತುರೀಯಾನಂದರು ಹೇಳುತ್ತಾರೆ: “ಅಂದು ಸ್ವಾಮೀಜಿ ಆಡಿದ ಮಾತು ಗಳೆಲ್ಲ ನನ್ನ ಮನಸ್ಸಿನ ಮೇಲೆ ಅಚ್ಚಳಿಯದ ಮುದ್ರೆಯೊತ್ತಿವೆ. ಅವರು ಅತ್ಯಂತ ಕಳಕಳಿಯಿಂದ, ತಮ್ಮ ಭಾವದಾಳದಿಂದ ಆಡಿದ ಪ್ರತಿಯೊಂದು ಶಬ್ದವೂ ಪ್ರತಿಯೊಂದು ಉಚ್ಚಾರವೂ ಇಂದಿಗೂ ನನ್ನ ಕಿವಿಗಳಲ್ಲಿ ಪ್ರತಿಧ್ವನಿಸುತ್ತಿದೆ. ಅವರು ಹೇಳಿದರು, ‘ಹರಿಭಾಯ್, ನೀನು ಯಾವುದನ್ನು “ಧರ್ಮ” ಎಂದು ಕರೆಯುತ್ತೀಯೋ ಅದೆಲ್ಲ ನನಗಿನ್ನೂ ಅರ್ಥವಾಗುತ್ತಿಲ್ಲ...’ ಹೀಗೆ ಹೇಳುತ್ತಿರುವಾಗಲೇ ಅವರ ಮುಖವನ್ನು ಒಂದು ತೀವ್ರ ದುಃಖದ ಭಾವ ಆವರಿಸಿ ಕೊಂಡಿತು. ಉದ್ವೇಗದಿಂದ ಅವರ ಇಡೀ ಶರೀರ ಅದುರಲಾರಂಭಿಸಿತು. ತಮ್ಮ ಎದೆಯ ಮೇಲೆ ಕೈಯನ್ನಿಟ್ಟುಕೊಂಡು ಅವರು ಮತ್ತೆ ಉದ್ಗರಿಸಿದರು, ‘ಆದರೆ... ಆದರೆ ನನ್ನ ಹೃದಯ ಮಾತ್ರ ತುಂಬ ವಿಶಾಲಗೊಂಡಿದೆ. ನಾನೀಗ ಭಾವಿಸುವುದನ್ನು ಕಲಿತಿದ್ದೇನೆ. ನಂಬು ನನ್ನನ್ನು–ನಾನೀಗ ನಿಜಕ್ಕೂ ಅತ್ಯಂತ ಗಾಢವಾಗಿ ಭಾವಿಸಬಲ್ಲೆ.’ ಅಷ್ಟರಲ್ಲೇ ಭಾವಾತಿಶಯದಿಂದ ಅವರ ಸ್ವರ ಗದ್ಗದವಾಯಿತು. ಮತ್ತೆ ಅವರ ಬಾಯಿಂದ ಮಾತು ಹೊರಡದಾಯಿತು. ಅವರ ಕಂಗಳಿಂದ ಆಶ್ರುಧಾರೆ ಹರಿಯಲಾರಂಭಿಸಿತು. ಸ್ವಲ್ಪ ಹೊತ್ತಿನವರೆಗೆ ಗಾಢ ನೀರವತೆ ತಾನೇ ತಾನಾಯಿತು.”

ಮುಂದೆ ಈ ಘಟನೆಯನ್ನು ತಮ್ಮ ಶಿಷ್ಯರೆದುರು ಬಣ್ಣಿಸುತ್ತಿದ್ದಂತೆ, ತುರೀಯಾನಂದರಿಗೇ ಭಾವವುಕ್ಕಿ ಬಂದಿತು. ಅವರ ಕಣ್ಣುಗಳಲ್ಲೂ ನೀರು ತುಂಬಿತು. ತಮ್ಮ ಭಾವಾವೇಶವನ್ನು ಬಲವಂತವಾಗಿ ಒತ್ತಿಹಿಡಿಯುತ್ತ ಸ್ವಲ್ಪ ಹೊತ್ತು ಹಾಗೇ ಸುಮ್ಮನೆ ಕುಳಿತರು. ಬಳಿಕ ಧೀರ್ಘ ಶ್ವಾಸ ಬಿಡುತ್ತ ಮತ್ತೆ ಹೇಳಿದರು, “ಸ್ವಾಮೀಜಿ ಹಾಗೆ ಮಾತನಾಡಿದಾಗ ನನ್ನ ಮನಸ್ಸಿನಲ್ಲಿ ಸುಳಿದ ಭಾವನೆಯೇನೆಂದು ಊಹಿಸಬಲ್ಲೆಯಾ? ನನಗನ್ನಿಸಿತು–‘ಇವೆಲ್ಲ ಬುದ್ಧ ಭಗವಂತನ ಮಾತು ಗಳಲ್ಲವೆ! ಈ ಭಾವನೆಗಳೆಲ್ಲ ತದ್ವತ್ ಬುದ್ಧನದೇ ಅಲ್ಲವೆ!’ ಎಂದು. ಮಾನವತೆಯ ದುಃಖ ಸಂಕಟಗಳೇ ಅವರ ವ್ಯಕ್ತಿತ್ವದಲ್ಲಿ ಮಿಡಿಯುವುದನ್ನು ನಾನು ಸ್ಪಷ್ಟವಾಗಿ ಕಂಡೆ. ಅವರ ಹೃದಯವು ಮನುಕುಲದ ಕಷ್ಟಕಾರ್ಪಣ್ಯಗಳನ್ನೆಲ್ಲ ಗುಣಪಡಿಸುವ ಯಾವುದೋ ಒಂದು ಮುಲಾಮು ತಯಾರಾಗುತ್ತಿದ್ದ ಕೊಳಗದಂತಿತ್ತು.”

ಸಮಸ್ತ ಭಾರತದಲ್ಲಿ ತಾಂಡವವಾಡುತ್ತಿದ್ದ ದಾರಿದ್ರ್ಯ, ರೋಗ, ಅಜ್ಞಾನ, ಮೌಢ್ಯಗಳನ್ನೆಲ್ಲ ಕಂಡು ಸ್ವಾಮೀಜಿಯ ಮನಸ್ಸು ಜರ್ಜರಿತವಾಗಿತ್ತು. ಸಾಲದಕ್ಕೆ ಆಗ ಭಾರತವು ಪಾರತಂತ್ರ್ಯದ ಶೃಂಖಲೆಯಲ್ಲಿ ಸಿಕ್ಕಿಬಿದ್ದಿತ್ತು. ಆಂಗ್ಲರ ದಬ್ಬಾಳಿಕೆ ಪರಮಾವಧಿಯಲ್ಲಿದ್ದ ಕಾಲ ಅದು. ಒಬ್ಬ ಯಃಕಶ್ಚಿತ್ ಐರೋಪ್ಯನೂ ಭಾರತೀಯನನ್ನು ಹಿಯಾಳಿಸುತ್ತಿದ್ದ. ಆದರೆ ಹೀಗೆ ಆತ್ಮಗೌರವಕ್ಕೆ ಧಕ್ಕೆ ತರುವ ಸಂದರ್ಭವುಂಟಾದಾಗಲೆಲ್ಲ ಸ್ವಾಮೀಜಿ ಅದನ್ನು ತೀವ್ರವಾಗಿ ಪ್ರತಿಭಟಿಸುತ್ತಿದ್ದರು. ಈಗ ಅವರು ಆಬುವಿನ ರೈಲುನಿಲ್ದಾಣದಲ್ಲಿದ್ದಾಗಲೇ ಇಂತಹ ಒಂದು ಘಟನೆ ನಡೆಯಿತು.

ಮೊದಲನೇ ದರ್ಜೆಯ ಬೋಗಿಯಲ್ಲಿ ಜಗಮೋಹನಲಾಲನೊಂದಿಗೆ ಕುಳಿತಿದ್ದ ಸ್ವಾಮೀಜಿ, ತಮ್ಮ ವಿಶ್ವಾಸಿಯಾದ ಒಬ್ಬ ಬಂಗಾಳೀ ಸಭ್ಯನೊಂದಿಗೆ ಮಾತನಾಡುತ್ತಿದ್ದರು. ಈತನೂ ಒಬ್ಬ ರೈಲ್ವೇ ನೌಕರನೇ. ಟ್ರೈನು ಹೊರಡುವವರೆಗೂ ಅವರೊಂದಿಗೆ ಮಾತನಾಡಲು ಈತ ಅಲ್ಲಿ ಕುಳಿತಿದ್ದ. ಆಗ ಅಲ್ಲಿಗೆ ಬಂದ ಒಬ್ಬ ಐರೋಪ್ಯ ಟಿಕೆಟ್ ಕಲೆಕ್ಟರ್ (ಟಿ. ಸಿ.), ಯಾವುದೋ ಇಲ್ಲದ ಕಾನೂನನ್ನು ತೆಗೆದು, “ಇಳಿಯಯ್ಯಾ ಕೆಳಗೆ” ಎಂದು ಈತನಿಗೆ ಒರಟಾಗಿ ಹೇಳಿದ. ತಾನು ಐರೋಪ್ಯ, ಬಿಳಿ ಚರ್ಮದವನು ಎಂಬಷ್ಟೇ ಕಾರಣಕ್ಕೆ ಅವನ ಈ ಜಬರದಸ್ತು. ಆದರೆ ಆ ಬಂಗಾಳೀ ವ್ಯಕ್ತಿಯೂ ರೈಲ್ವೆಯವನೇ ಆದ್ದರಿಂದ, “ನಾನು ಕೆಳಗಿಳಿಯಬೇಕೆಂಬುದಕ್ಕೆ ಯಾವ ಕಾನೂನೂ ಇಲ್ಲ” ಎಂದು ಸೌಮ್ಯವಾಗಿಯೇ ಪ್ರತಿಭಟಿಸಿದ. ಆಗ ಆ ಟಿ. ಸಿ. ಗೆ, ಈ ಭಾರತೀಯ ತನಗೇ ಎದುರುತ್ತರ ಕೊಡುತ್ತಾನಲ್ಲ ಎಂದು ಇನ್ನಷ್ಟು ಸಿಟ್ಟೇರಿತು. ಇದನ್ನು ಕಂಡು ಸ್ವಾಮೀಜಿ ಮಧ್ಯೆ ಪ್ರವೇಶಿಸಿದರು. ತಕ್ಷಣ ಅವನು ಇವರ ಕಡೆಗೆ ತಿರುಗಿ “ತುಮ್ ಕಾಹೆ ಬಾತ್ ಕರ್​ತೇ ಹೋ?–ನೀನ್ಯಾರಯ್ಯ ಮಧ್ಯೆ ಮಾತನಾಡುವುದಕ್ಕೆ?” ಎಂದು ತೀಕ್ಷ್ಣವಾಗಿ ಹೇಳಿದ. ಹಿಂದಿ ಯಲ್ಲಿ ‘ತುಮ್​’ ಎಂದರೆ ನೀನು ಎಂದರ್ಥ. ಈ ಶಬ್ದವನ್ನು ನಮ್ಮ ಸ್ನೇಹಿತರನ್ನು ಇಲ್ಲವೆ ನಮ ಗಿಂತ ಕೆಳಗಿನವರನ್ನು ಸಂಬೋಧಿಸುವಾಗ ಉಪಯೋಗಿಸಬಹುದು. ಆದರೆ ಸಮಾನಸ್ಕಂಧರನ್ನು ಇಲ್ಲವೆ ನಮಗಿಂತ ಮೇಲಿನವರನ್ನು ಅಥವಾ ಅಪರಿಚಿತರನ್ನು ಸಂಬೋಧಿಸುವಾಗ ‘ಆಪ್​’ (ನೀವು, ತಾವು) ಎಂದೇ ಹೇಳಬೇಕು. ಆ ಟಿ. ಸಿ. ಅವಮರ್ಯಾದೆಯಾಗಿ ಮಾತನಾಡಿದಾಗ ಸ್ವಾಮೀಜಿಗೆ ಕೋಪ ಬಂದಿತು. ಈತ ತಮ್ಮನ್ನು ‘ನೀನು’ ಎಂದು ಕರೆದನೆಂಬ ಕಾರಣಕ್ಕಾಗಿ ಮಾತ್ರವಲ್ಲ; ‘ತಾನು ಐರೋಪ್ಯ, ಈ ಭಾರತೀಯರು ತಮ್ಮ ಗುಲಾಮರು’ ಎಂಬ ಅವನ ಧೋರಣೆಯನ್ನು ಕಂಡು. ತಕ್ಷಣ ಸ್ವಾಮೀಜಿ ಇಂಗ್ಲಿಷಿನಲ್ಲಿ ಆರ್ಭಟಿಸಿದರು, “ಏನು, ‘ತುಮ್​’ ಎನ್ನುತ್ತಿದ್ದೀರಲ್ಲ, ಏನು ಸಮಾಚಾರ? ಮರ್ಯಾದೆಯಾಗಿ ನಡೆದುಕೊಳ್ಳಲು ಬರುವುದಿಲ್ಲವೆ? ಮೊದಲನೇ ಮತ್ತು ಎರಡನೇ ದರ್ಜೆಯ ಪ್ರಯಾಣಿಕರೊಂದಿಗೆ ವ್ಯವಹರಿಸುವವರು ನೀವು. ನೋಡಿದರೆ ನಿಮಗೇ ಸಭ್ಯತೆ ಎಂದರೇನೆಂದು ತಿಳಿದಿಲ್ಲ! ‘ಆಪ್​’ ಎಂದು ಹೇಳಲು ಬರುವು ದಿಲ್ಲವೆ?” ಹೀಗೆ ಸ್ವಾಮೀಜಿ ದಬಾಯಿಸಿದಾಗ ಆ ಟಿ. ಸಿ. ತಣ್ಣಗಾಗಿ ಕ್ಷಮೆ ಕೇಳಲು ಹೊರಟ: “ಕ್ಷಮಿಸಬೇಕು. ನನಗೆ ಹಿಂದಿ ಚೆನ್ನಾಗಿ ಬರುವುದಿಲ್ಲ. \eng{I only wanted this \textit{man} to.. (}ನಾನು ಈ ಮನುಷ್ಯನನ್ನು... )” ಎನ್ನುವಷ್ಟರಲ್ಲಿ ಸ್ವಾಮೀಜಿ ಮಧ್ಯದಲ್ಲೇ ತಡೆದು ಮತ್ತೆ ಛೀಮಾರಿ ಹಾಕಿದರು. “ಈಗ ತಾನೇ ನೀವು ಹೇಳಿದಿರಿ, ‘ನನಗೆ ಹಿಂದಿ ಚೆನ್ನಾಗಿ ಬರುವುದಿಲ್ಲ’ ಎಂದು. ಈಗ ನೋಡಿದರೆ ನಿಮಗೆ ನಿಮ್ಮ ಭಾಷೆಯೇ ಸರಿಯಾಗಿ ಗೊತ್ತಿಲ್ಲ. ನೀವು ‘ಈ ಮನುಷ್ಯ’ ಎಂದು ಕರೆದಿರಲ್ಲ. ಅವರೊಬ್ಬರು ‘ಸಭ್ಯವ್ಯಕ್ತಿ\eng{’–This \textit{man of whom you speak is a gentleman.}”} ಪರಿಸ್ಥಿತಿ ತನಗೆದುರಾದುದನ್ನು ಕಂಡು ಆ ಟಿ. ಸಿ. ಅಲ್ಲಿಂದ ಕಾಲುಕಿತ್ತ.

ಅವನು ಹೋದಮೇಲೆ ಸ್ವಾಮೀಜಿ ಜಗಮೋಹನನಿಗೆ ಹೇಳುತ್ತಾರೆ, “ನೋಡು, ಐರೋಪ್ಯ ರೊಂದಿಗೆ ವ್ಯವಹರಿಸುವಾಗ ನಮ್ಮಲ್ಲಿ ಮುಖ್ಯವಾಗಿ ಇರಬೇಕಾದದ್ದು ಆತ್ಮಗೌರವ. ನಾವು ಯಾರ ಜೊತೆ ಹೇಗೆ ವರ್ತಿಸಬೇಕೋ ಹಾಗೆ ವರ್ತಿಸುವುದಿಲ್ಲ. ಆದ್ದರಿಂದ ಅವರು ನಮ್ಮನ್ನು ಶೋಷಿಸಲು ನೋಡುತ್ತಾರೆ. ಇತರರ ಮುಂದೆ ನಾವು ನಮ್ಮ ಘನತೆಯನ್ನು ಬಿಟ್ಟುಕೊಡಬಾರದು. ಹಾಗೆ ಮಾಡದಿದ್ದರೆ, ಇತರರು ನಮ್ಮನ್ನು ಹೀನಾಯವಾಗಿ ನೋಡಲು ನಾವೇ ಅನುವು ಮಾಡಿ ಕೊಟ್ಟಂತಾಗುತ್ತದೆ.”

ತಾವು ಮುಂಬಯಿಯಿಂದ ಅಮೆರಿಕೆಗೆ ಹೊರಡುವುದಾಗಿ ಸ್ವಾಮೀಜಿ, ಎರಡು ವಾರಗಳ ಹಿಂದೆಯೇ ತಮ್ಮ ಮದರಾಸೀ ಶಿಷ್ಯರಿಗೆ ಪತ್ರಮುಖೇನ ತಿಳಿಸಿದ್ದರು. ಈ ಸುದ್ದಿ ತಿಳಿದ ತಕ್ಷಣ ಅಳಸಿಂಗ ಪೆರುಮಾಳ್, ಅವರನ್ನು ಬೀಳ್ಗೊಡುವುದಕ್ಕಾಗಿ ಅಷ್ಟು ದೂರದ ಮದರಾಸಿನಿಂದ ಮುಂಬಯಿಗೆ ಬಂದುಬಿಟ್ಟಿದ್ದರು! ಜಗಮೋಹನಲಾಲನೊಂದಿಗೆ ಸ್ವಾಮೀಜಿ ಮುಂಬಯಿಗೆ ಬಂದು ನೋಡುತ್ತಾರೆ–ತಮ್ಮ ಪ್ರಿಯಶಿಷ್ಯ ತಮಗಾಗಿ ಕಾದುನಿಂತಿದ್ದಾನೆ! ಇವರಿಬ್ಬರೊಂದಿಗೆ ಸ್ವಾಮೀಜಿ ಒಂದು ಕಡೆ ಇಳಿದುಕೊಂಡರು. ಸ್ವಾಮೀಜಿಯ ಸೇವೆ ಮಾಡಲು ಜಗಮೋಹನ ಹಾಗೂ ಅಳಸಿಂಗ–ಇಬ್ಬರೂ ನಾಮುಂದು ತಾಮುಂದೆಂದು ಸ್ಪರ್ಧಿಸಿದರು. ಅದೊಂದು ಅಪೂರ್ವ ಭಾಗ್ಯವೇ ಸರಿ! ತನ್ನ ರಾಜನ ಅಪ್ಪಣೆಯಂತೆ ಜಗಮೋಹನ, ಸ್ವಾಮೀಜಿಯನ್ನು ಮುಂಬಯಿಯ ಅತ್ಯುತ್ತಮ ಅಂಗಡಿಗಳಿಗೆ ಕರೆದೊಯ್ದು ಅವರಿಗೆ ಬೇಕಾದ, ಬೇಕಾಗಬಹುದಾದ ಎಲ್ಲ ಬಗೆಯ ವಸ್ತುಗಳನ್ನು ಕೊಡಿಸಿದ. ತಮಗಾಗಿ ಅವನು ಅತ್ಯಂತ ಉತ್ಕೃಷ್ಟವಾದ ರೇಷ್ಮೆಯ ಸೂಟು, ನಿಲುವಂಗಿ, ಪೇಟಾಗಳಿಗೆ ಆದೇಶ ನೀಡುತ್ತಿರುವುದನ್ನು ಕಂಡು ಸ್ವಾಮೀಜಿ ಪ್ರತಿಭಟಿಸಿ ದರು. ತಮಗೆ ಸಾಧಾರಣ ಬಟ್ಟೆಬರೆಯೇ ಸಾಕು; ತಮ್ಮಂತಹ ಸಂನ್ಯಾಸಿಗೆ ಇಷ್ಟೆಲ್ಲ ಬೆಲೆಬಾಳುವ ಉಡುಗೆಯೇಕೆಂದು ಎಷ್ಟೋ ಹೇಳಿದರು. ಆದರೆ ಶಿಷ್ಯರು ಅವರ ಮಾತನ್ನು ಕಿವಿಯ ಮೇಲೇ ಹಾಕಿಕೊಳ್ಳಲಿಲ್ಲ. ಕಡೆಗೆ ಸ್ವಾಮೀಜಿ ತಮ್ಮ ಶಿಷ್ಯರ ಮಾತಿಗೇ ಒಪ್ಪಿಕೊಳ್ಳಬೇಕಾಯಿತು. ಪರಿಣಾಮ ವಾಗಿ ಈ ಹೊಸ ವೇಷಭೂಷಣಗಳಲ್ಲಿ ಅವರು ಮಹಾರಾಜನಂತೆ ಕಂಗೊಳಿಸಿದರು! ಈ ಸುಂದರ ಉಡುಗೆ ಅವರ ದೈವದತ್ತ ಸೌಂದರ್ಯವನ್ನು ಮತ್ತಷ್ಟು ಹೆಚ್ಚಿಸಿತು. ನಿಜಕ್ಕೂ ಈ ಉಡುಗೆ ಮುಂದೆ ಸಾಕಷ್ಟು ಕೆಲಸವನ್ನೇ ಮಾಡಿತು. ವಿಶ್ವಧರ್ಮ ಸಮ್ಮೇಳನದಲ್ಲಿ ಅವರಿನ್ನೂ ಒಂದು ಮಾತನ್ನೂ ಆಡುವ ಮೊದಲೇ ಅದು ಪ್ರೇಕ್ಷಕರೆಲ್ಲರ ಗಮನ ಸೆಳೆದಿತ್ತು. (ಆದರೆ ಇದು ತೀರ ಕಣ್ಣಿಗೆ ಹೊಡೆಯುವಂತಿದೆ ಎಂದು ಸ್ವಾಮೀಜಿಗೆ ಅನ್ನಿಸಿದಾಗ ಪಾಶ್ಚಾತ್ಯ ಕಣ್ಣುಗಳಿಗೆ ಹೆಚ್ಚು ಪರಿಚಿತವಾದ ಹಾಗೂ ಚಳಿಯಿಂದ ಹೆಚ್ಚಿನ ರಕ್ಷಣೆ ನೀಡುವಂತಹ ವಸ್ತ್ರಗಳನ್ನು ಧರಿಸಲಾರಂಭಿಸಿದರು.)

ಸ್ವಾಮೀಜಿಯ ಪ್ರಯಾಣಕ್ಕೆ ಎಲ್ಲ ಸಿದ್ಧತೆಗಳೂ ಆಗುತ್ತ ಬಂದಿದ್ದುವು. ಈಗಾಗಲೇ ಜಪಾನಿನ ವರೆಗೂ ಹೋಗಲು ಜಗಮೋಹನ ‘ಪೆನಿನ್ಸುಲಾರ್​’ ಎಂಬ ಹಡಗಿನಲ್ಲಿ ಮೊದಲ ದರ್ಜೆಯ ಟಿಕೆಟನ್ನು ವ್ಯವಸ್ಥೆ ಮಾಡಿದ್ದ. ಇದು ತಮಗೆ ತೀರ ಹೆಚ್ಚಾಯಿತೆಂದು ಸ್ವಾಮೀಜಿ ಪ್ರತಿಭಟಿಸಿದ್ದರು. ಆದರೆ ಜಗಮೋಹನ, “ಮಹಾರಾಜರ ಗುರುಗಳು ಮಹಾರಾಜರಂತೆಯೇ ಪ್ರಯಾಣಿಸಬೇಕು!” ಎಂದು ಅವರನ್ನು ಸುಮ್ಮನಾಗಿಸಿದ್ದ. ಹಡಗು ಮೇ ತಿಂಗಳ ೩೧ರಂದು ಮುಂಬಯಿಯಿಂದ ಹೊರಡಲಿತ್ತು. ಜಪಾನಿನಿಂದ ವ್ಯಾಂಕೋವರ್​ವರೆಗೆ ಮತ್ತೊಂದು ಹಡಗಿನಲ್ಲಿ ಸ್ಥಳವನ್ನು ಕಾದಿರಿಸಲಾಗಿತ್ತು. ಸ್ವಾಮೀಜಿಗೆ ಬೇಕಾದ ವಸ್ತುಗಳನ್ನೆಲ್ಲ ಕೊಡಿಸಿದ ಮೇಲೆ ಜಗಮೋಹನ, ಸಾಕಷ್ಟು ಹಣ ತುಂಬಿದ್ದ ಪರ್ಸ್ ಒಂದನ್ನು ಅವರ ಕೈಯಲ್ಲಿಟ್ಟ.

ಮುಂಬಯಿಯಲ್ಲಿ ಕಡೆಯ ಕೆಲವು ದಿನಗಳನ್ನು ಸ್ವಾಮೀಜಿ ಧ್ಯಾನ-ಪ್ರಾರ್ಥನೆಗಳಲ್ಲಿ ಕಳೆದರು. ತಮ್ಮ ಮುಂದಿರುವ ಮಹಾಕಾರ್ಯಕ್ಕೆ ಮೌನವಾಗಿ ಸಿದ್ಧರಾದರು. ಕೆಲವೊಮ್ಮೆ ತಮ್ಮ ಹಳೆಯ ಸ್ನೇಹಿತರನ್ನು ಭೇಟಿಯಾದರಲ್ಲದೆ ಅಲ್ಲಲ್ಲಿ ಪ್ರವಚನಗಳನ್ನು ನೀಡಿದರು. ಮತ್ತೆಮತ್ತೆ ಅವರ ಮನಸ್ಸು ಬಹುದೂರದ ಬಂಗಾಳದಲ್ಲಿರುವ ಆಲಂಬಜಾರಿನ ಮಠಕ್ಕೆ ಹಾರಿಹೋಗುತ್ತಿತ್ತು. ಮಠ ಹೇಗೆ ನಡೆಯುತ್ತಿರಬಹುದು? ತಮ್ಮ ಪ್ರೀತಿಯ ಗುರುಭಾಯಿಗಳು ಹೇಗಿರಬಹುದು?– ಎಂದು ಅದು ಚಿಂತಿಸುತ್ತಿತ್ತು. ಶ್ರೀರಾಮಕೃಷ್ಣರ ಕೃಪೆಯಿಂದ ಎಲ್ಲವೂ ಚೆನ್ನಾಗಿರಲೇಬೇಕು ಎಂದು ಹಾರೈಸಿದರು. ಆದರೆ ಇತ್ತ ಮಠದಲ್ಲಿ ಅವರ ಗುರುಭಾಯಿಗಳು, ತಮ್ಮ ಪ್ರಾಣಸಖ ನರೇಂದ್ರನ ಮಧುರ ಸ್ಮರಣೆ ಮಾಡುತ್ತಿದ್ದರು. “ನಮ್ಮ ನರೇನ್ ಈಗ ಎಲ್ಲಿರಬಹುದೋ? ಏನು ಮಾಡುತ್ತಿರಬಹುದೋ?” ಎಂದು ತಮ್ಮತಮ್ಮಲ್ಲೇ ಕೇಳಿಕೊಳ್ಳುತ್ತಿದ್ದರು. ತಮ್ಮ ನರೇಂದ್ರನಿಗೆ ಶುಭವಾಗಲಿ ಎಂದು ಶ್ರೀರಾಮಕೃಷ್ಣರಲ್ಲಿ ಮೊರೆಯಿಡುತ್ತಿದ್ದರು.

ಕೊನೆಗೂ ಹೊರಡುವ ದಿನ ಬಂದೇಬಿಟ್ಟಿತು. ಸ್ವಾಮೀಜಿ ಸಿದ್ಧರಾಗಿ ನಿಂತರು. ಆದರೆ ಹಡಗಿನ ಪ್ರಯಾಣ, ಪ್ರಯಾಣಕ್ಕೆ ಸಂಬಂಧಿಸಿದ ವಿಧಿನಿಯಮಗಳು, ಆ ಜನಸಂದಣಿ-ಗಲಭೆ, ಅಷ್ಟೊಂದು ವೈಯಕ್ತಿಕ ಸಾಮಾನು-ಸರಂಜಾಮುಗಳು ಇವೆಲ್ಲ ಅವರಿಗೆ ತೀರ ಹೊಸದು. ಇವೆಲ್ಲ ಅವರಿಗೆ ಕಸಿವಿಸಿಯುಂಟುಮಾಡುತ್ತಿದ್ದುವು. ಜೊತೆಗೆ ಅವರ ಸ್ನೇಹಿತರ-ಶಿಷ್ಯರೆಲ್ಲ ಸೇರಿ, ಅವರು ರೇಷ್ಮೆಯ ನಿಲುವಂಗಿ ಹಾಗೂ ರೇಷ್ಮೆಯ ರುಮಾಲನ್ನು ಧರಿಸುವಂತೆ ಮಾಡಿದ್ದರು. ಈ ಅದ್ದೂರಿಯ ಉಡುಗೆಯನ್ನು ಧರಿಸಲು ಅವರಿಗೆ ಸಂಕೋಚವೆನಿಸುತ್ತಿತ್ತು. ಅವರ ಹೃದಯದಲ್ಲಿ ಮಾತ್ರ ಭಾರತದ ಹಾಗೂ ತಮ್ಮ ಮುಂದಿನ ಕಾರ್ಯದ ಕುರಿತಾದ ನೂರಾರು ಭಾವನೆಗಳು ನಾಟ್ಯವಾಡುತ್ತಿದ್ದವು. ಈ ಎಲ್ಲ ಆಲೋಚನೆಗಳನ್ನು ಹೊತ್ತು ಸ್ವಾಮೀಜಿ ಹಡಗನ್ನು ಪ್ರವೇಶಿಸಿ ದರು. ಜಗಮೋಹನ ಹಾಗೂ ಅಳಸಿಂಗ ಇಬ್ಬರೂ ಅವರೊಂದಿಗೆ ಹೋಗಿ ಮಾತನಾಡುತ್ತ ನಿಂತರು. ಹಡಗು ಹೊರಡುವ ಸಮಯವಾಯಿತು. ಗಾಂಗ್ ಬಾರಿಸಿತು. ಇಬ್ಬರೂ ಸ್ವಾಮೀಜಿಯ ಪಾದಗಳಿಗೆ ಸಾಷ್ಟಾಂಗ ಪ್ರಣಾಮ ಮಾಡಿದರು. ಬಳಿಕ ಆಶ್ರುನಯನರಾಗಿ ಹಡಗನ್ನು ಬಿಟ್ಟು ಬಂದರು. ಹಡಗು ತೂಗಾಡುತ್ತ ನಿಧಾನವಾಗಿ ಜರುಗಿತು. ಹಡಗಿನ ಅಂಗಳದಲ್ಲಿ ರಾಜಯೋಗ್ಯ ದಿರಿಸನ್ನು ಧರಿಸಿ ಮಂದಸ್ಮಿತವದನರಾಗಿ ನಿಂತಿದ್ದ ಸ್ವಾಮೀಜಿಯನ್ನು ಅವರು ಕಣ್ಮರೆಯಾಗುವವ ರೆಗೂ ನೋಡಿದರು.

ಹಡಗಿನ ಅಂಚಿನಲ್ಲಿ ತಮ್ಮ ಪ್ರೀತಿಯ ಭರತಭೂಮಿಯನ್ನೇ ದಿಟ್ಟಿಸುತ್ತ ನಿಂತ ಸ್ವಾಮೀಜಿ ತಮ್ಮ ಪ್ರೀತಿಪಾತ್ರರಾದ ಎಲ್ಲರಿಗೂ ಅಲ್ಲಿಂದಲೇ ಆಶೀರ್ವಾದಗಳನ್ನು ಕೋರಿದರು. ಅವರ ವಿಶಾಲ ನೇತ್ರಗಳು ಕಣ್ಣೀರಿನಿಂದ ಮಂಜಾದುವು; ಅವರ ಉದಾತ್ತ ಮೃದುಹೃದಯ ಭಾವಭರಿತ ವಾಯಿತು. ಪರಮಗುರು ಶ್ರೀರಾಮಕೃಷ್ಣರನ್ನು, ಜಗನ್ಮಾತೆ ಶ್ರೀಶಾರದಾದೇವಿಯವರನ್ನು, ಅವರೀರ್ವರ ಪುತ್ರರಾದ ತಮ್ಮ ಸಂನ್ಯಾಸೀ ಸೋದರರನ್ನು ಸ್ಮರಿಸಿದರು. ಮತ್ತೆ ಅಪ್ರಯತ್ನವಾಗಿ ಸಮಸ್ತ ಭಾರತವೇ ಅವರ ಮನಸ್ಸು-ಬುದ್ಧಿಗಳನ್ನಾವರಿಸಿತು. ಭಾರತದ ದಿವ್ಯ ಸಂಸ್ಕೃತಿ ಪರಂಪರೆ ಯನ್ನು ಭಾವಿಸಿ ನೋಡಿದರು. ಮರುಕ್ಷಣ ಭಾರತದ ದುಃಖದಾರಿದ್ರ್ಯಗಳು ಅವರ ಕಣ್ಣುಂದೆ ಬಂದು ನಿಂತವು. ಬಳಿಕ ಸನಾತನ ಪುಷಿಪರಂಪರೆಯನ್ನೂ ವೇದಧರ್ಮದ ಉತ್ಕೃಷ್ಟತೆಯನ್ನೂ ಸ್ಮರಿಸಿ ಅವರ ಹೃದಯ ಹಿಗ್ಗಿತು. ಇವುಗಳನ್ನೆಲ್ಲ ಭಾವಿಸುತ್ತಿದ್ದಂತೆ ಅವರ ಹೃದಯದಲ್ಲಿ ಮಾತೃಭೂಮಿಯ ಮೇಲಿನ ಆಭಿಮಾನ-ಪ್ರೀತಿ-ಗೌರವಗಳು ಮಹಾಪೂರವಾಗಿ ಉಕ್ಕಿದುವು.

ಈಗ ತಾವು ಹೊರಟಿರುವ ಸಂಪದ್ಭರಿತ ಆಧುನಿಕ ಅಮರಾವತಿಯಾದ ಅಮೆರಿಕೆಯ ಚಿತ್ರ ಅವರ ಮನಃಪಟಲದ ಮೇಲೆ ಮೂಡಿತು. ಭೋಗವೇ ಬದುಕಿನ ಪರಮೋಚ್ಚ ಧ್ಯೇಯವೆಂದು ನಂಬಿ ನಡೆವ ಪಾಶ್ಚಾತ್ಯರ ಜೀವನ ವಿಧಾನದ ಬಗ್ಗೆ ಆಲೋಚಿಸಿದರು. ಸ್ವಾಮೀಜಿ ತಮ್ಮೊಳಗೆ ಹೇಳಿಕೊಂಡರು–‘ಹೌದು, ನಾನೀಗ ತ್ಯಾಗ ಭೂಮಿಯಿಂದ ಭೋಗಭೂಮಿಯೆಡೆಗೆ ಸಾಗುತ್ತಿ ದ್ದೇನೆ... ’ ಆದರೆ ಅವರು ಆ ಭೋಗ ಭೂಮಿಗೆ ಹೋಗುತ್ತಿರುವುದು ಭೋಗವನ್ನನುಭವಿಸುವು ದಕ್ಕಲ್ಲ. ತ್ಯಾಗ-ಯೋಗಗಳನ್ನು ಬೋಧಿಸಲು. ಅಲ್ಲಿ ಅವರಿಗೆ ಕಾದಿರುವುದು ಭೋಗವಲ್ಲ, ತಮ್ಮ ಜೀವನವನ್ನೇ ಮಾತೃಭೂಮಿಗಾಗಿ ತೇಯುವ ತ್ಯಾಗ! ಇಹಲೋಕದಲ್ಲಿ ಅವರ ಜೀವಿತಾವಧಿ ಇನ್ನು ಕೇವಲ ಒಂಬತ್ತೇ ವರ್ಷಗಳು, ಅಷ್ಟರೊಳಗೇ ಅವರು ವಿಶ್ವವಿಜೇತರಾಗಿ, ವಿಶ್ವಮಾನವ ರಾಗಿ ಬೆಳೆದು ನಿಲ್ಲುವುದನ್ನು ನಾವು ನೋಡಲಿದ್ದೇವೆ.

