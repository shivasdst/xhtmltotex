
\chapter{ಉದಯರವಿ}

\noindent

ಜುಲೈ ೨೫ರಂದು ‘ಎಂಪ್ರೆಸ್ ಆಫ್ ಇಂಡಿಯ’ ಹಡಗು ಬ್ರಿಟಿಷ್ ಕೊಲಂಬಿಯಕ್ಕೆ ಸೇರಿದ್ದ ವ್ಯಾಂಕೋವರನ್ನು ತಲುಪಿತು. ಇಲ್ಲಿಂದ ಶಿಕಾಗೋ ನಗರಕ್ಕೆ ೨ಂಂಂ ಮೈಲಿಗಳ ಪ್ರಯಾಣ. ಮರುದಿನ ಸ್ವಾಮೀಜಿ, ಸಹಪ್ರಯಾಣಿಕರೊಂದಿಗೆ ವ್ಯಾಂಕೋವರಿನಲ್ಲಿ ‘ಅಟ್ಲಾಂಟಿಕ್ ಎಕ್ಸ್ ಪ್ರೆಸ್​’ ಎಂಬ ಟ್ರೈನು ಹತ್ತಿ ಪೂರ್ವಾಭಿಮುಖವಾಗಿ ಹೊರಟರು. ನಯನಮನೋಹರ‘ರಾಕಿ’ ಪರ್ವತಗಳ ಮೂಲಕ ಸಾಗಿ, ಮೂರು ದಿನಗಳ ನಂತರ ಅದು ವಿನ್ನಿಪೆಗ್​ಗೆ ಬಂದಿತು. ಇಲ್ಲಿ ಸ್ವಾಮೀಜಿ ಅಮೆರಿಕೆಗೆ ಹೋಗುವ ಟ್ರೈನು ಹತ್ತಿ ಸೇಂಟ್​ಪಾಲ್​ಗೆ ಬಂದರು. ಇಲ್ಲಿ ಮತ್ತೆ ಟ್ರೈನು ಬದಲಾಯಿಸಿ, ೧೮೯೩ರ ಜುಲೈ ೩ಂರಂದು ೪ಂಂ ಮೈಲಿ ದೂರದ ಶಿಕಾಗೋ ನಗರವನ್ನು ಸೇರಿದರು.

ಆ ಮಹಾನಗರದ ಬೃಹತ್ ನಿಲ್ದಾಣದಲ್ಲಿ ಇಳಿದ ಸ್ವಾಮೀಜಿ ಅಲ್ಲಿನ ಗಡಿಬಿಡಿ ಗಲಾಟೆಗಳನ್ನು ಕಂಡು ಕ್ಷಣಕಾಲ ದಿಗ್ಭ್ರಾಂತರಾದರು. ಅವರು ಈ ಹಿಂದೆಯೇ ಎಷ್ಟೋ ದೊಡ್ಡದೊಡ್ಡ ನಗರ ಗಳನ್ನೂ ಅಲ್ಲಿನ ಜನಸಂದಣಿಯನ್ನೂ ಕಂಡಿದ್ದರು. ಅಲ್ಲದೆ ಅಮೆರಿಕೆಯ ಬಗ್ಗೆ ಸಾಕಷ್ಟು ಕೇಳಿದ್ದರು, ಓದಿದ್ದರು. ಆದರೂ ಅಲ್ಲಿಗೆ ತಾವೇ ಬಂದಾಗ ಅವರಿಗೆ ಚಿಂತೆಗಿಟ್ಟುಕೊಂಡಿತು. ಇಲ್ಲಿಯವರೆಗಂತೂ ಹೇಗೋ ಬಂದದ್ದಾಯಿತು; ಆದರೆ ಇಲ್ಲಿ ಯಾವ ಕಡೆ ಹೋಗುವುದೊ? ಎಲ್ಲಿ ಉಳಿದುಕೊಳ್ಳುವುದೊ?–ಎಂದು ಆಲೋಚಿಸುತ್ತ ನಿಂತರು. ಶಿಕಾಗೋ ನಗರದಲ್ಲಿ ಆಗ್ಗೆ ಮೂರು ತಿಂಗಳಿನಿಂದಲೂ ಜಾಗತಿಕ ಮೇಳವೊಂದು ನಡೆಯುತ್ತಿದ್ದು, ಅದನ್ನು ನೋಡಲು ಎಲ್ಲೆಡೆಯಿಂದಲೂ ಜನ ಧಾವಿಸಿ ಬರುತ್ತಿದ್ದರು. (ಈ ಜಾಗತಿಕ ಮೇಳದ ಅಂಗವಾಗಿಯೇ ವಿಶ್ವ ಧರ್ಮ ಸಮ್ಮೇಳನವೂ ಏರ್ಪಾಡಾಗಿದ್ದುದು.) ಆದ್ದರಿಂದ ರೈಲುನಿಲ್ದಾಣ ಜನರಿಂದ ತುಂಬಿ ಗಿಜಿಗುಟ್ಟುತ್ತಿತ್ತು. ಈಗ ತಮ್ಮ ಮಣಗಟ್ಟಲೆ ಭಾರದ ಪೆಟ್ಟಿಗೆಗಳನ್ನು ಬೇರೆ ಎತ್ತಿಕೊಂಡು ಹೋಗ ಬೇಕಲ್ಲ ಎಂದು ಸ್ವಾಮೀಜಿ ಆಲೋಚಿಸುವಷ್ಟರಲ್ಲಿ ಅವರಿಗೆ ಮೊದಲ ಅಘಾತ ಸಿಕ್ಕಿತು–ಅಲ್ಲಿದ್ದ ರೈಲ್ವೆ ಕೂಲಿಗಳು ಅವರ ಸಾಮಾನುಗಳನ್ನು ಹೊತ್ತು ತರಲು ಅತಿ ದುಬಾರಿ ಮಜೂರಿ ಕೇಳ ಲಾರಂಭಿಸಿದರು. ಆ ಬೆಲೆಯನ್ನು ಕೇಳಿಯೇ ಅವರ ಎದೆ ಧಸಕ್ಕೆಂದಿತು. ಕಡೆಗೆ ವಿಧಿಯಿಲ್ಲದೆ ಆ ಬೆಲೆಗೇ ಒಪ್ಪಿಕೊಳ್ಳಬೇಕಾಯಿತು.

ನಿಲ್ದಾಣದ ಆಚೆಗೆ ಬರುತ್ತಿದ್ದಂತೆಯೇ ಹಲವಾರು ಹೋಟೆಲುಗಳ ದಳ್ಳಾಳಿಗಳು ಅವರನ್ನು ಮುತ್ತಿಕೊಂಡು ತಮ್ಮ ಹೋಟೆಲಿಗೇ ಬರುವಂತೆ ಕಾಡಿದರು. ಅಪರಿಚಿತ ಸ್ಥಳದಲ್ಲಿರುವಾಗ ಒಳ್ಳೆಯ ಜಾಗದಲ್ಲಿರಬೇಕಾದದ್ದು ಬಹಳ ಮುಖ್ಯ ಎಂದು ಆಲೋಚಿಸಿ ಸ್ವಾಮೀಜಿ, ತಮ್ಮದು ಅತ್ಯುತ್ತಮ ಹೋಟೆಲ್ ಎಂದು ಹೇಳಿಕೊಂಡವನೊಂದಿಗೆ ಸಾರೋಟಿನಲ್ಲಿ ಹೊರಟರು. ನೋಡು ತ್ತಾರೆ–ಅದೊಂದು ಉತ್ತಮ ದರ್ಜೆಯ ಆಧುನಿಕ ಹೋಟೆಲ್ಲೇ ನಿಜ. ಬಳಿಕ ಎಲಿವೇಟರ್​ನ ಮೂಲಕ ಮೇಲೇರಿ, ತಮ್ಮ ಕೋಣೆಯನ್ನು ತಲುಪಿದರು. ಕೂಲಿಗಳು ಅವರ ಸಾಮಾನುಗಳನ್ನು ತಂದಿಟ್ಟು ಹೊರಟರು. ಇನ್ನು ಸದ್ಯಕ್ಕೆ ತಮ್ಮನ್ನು ಕಾಡುವವರು ಯಾರೂ ಇಲ್ಲ ಎಂದು ದೃಢ ವಾದಾಗ ಸ್ವಾಮೀಜಿ, ಆ ಪೆಟ್ಟಿಗೆಗಳ ಮಧ್ಯದಲ್ಲೇ ಕುಳಿತು, ಗಲಿಬಿಲಿಗೊಂಡ ಮನಸ್ಸನ್ನು ಶಾಂತ ವಾಗಿಸಿಕೊಳ್ಳುವ ಪ್ರಯತ್ನ ಮಾಡಿದರು.

ಮರುದಿನ ಬೆಳಿಗ್ಗೆ ಸ್ವಾಮೀಜಿ ಜಾಗತಿಕ ಮೇಳವನ್ನು ನೋಡಿಕೊಂಡು ಬರಲು ಹೊರಟರು. ಈ ಮೇಳದಲ್ಲಿ ಇಡೀ ಜಗತ್ತಿನ ಅತ್ಯಾಧುನಿಕ ಅನ್ವೇಷಣೆಗಳೆಲ್ಲವನ್ನೂ ಕಾಣಬಹುದಾಗಿತ್ತು. ವಿಶ್ವದ ಎಲ್ಲೆಡೆಗಳಿಂದ ನವನೂತನ ವಸ್ತುವಿಶೇಷಗಳು, ಶ್ರೇಷ್ಠಕಲಾಕೃತಿಗಳು, ಅದ್ಭುತ ಯಂತ್ರ ಸಾಮಗ್ರಿಗಳು, ವಿದ್ಯುತ್ ಉಪಕರಣಗಳು ಮೊದಲಾದ ಅಸಂಖ್ಯಾತ ವಸ್ತುಗಳನ್ನು ಇಲ್ಲಿ ಪ್ರದರ್ಶಿಸಲಾಗಿತ್ತು. ಮಾನವನ ಬುದ್ಧಿಶಕ್ತಿಗೂ ಕಲ್ಪನಾಶಕ್ತಿಗೂ ಸಾಕ್ಷಿಯಾದ ಈ ಅನ್ವೇಷಣೆ ಗಳನ್ನು ಕಂಡು ಸ್ವಾಮೀಜಿ ಬೆರಗಾದರು. ಈ ಇಡೀ ಪ್ರದರ್ಶನವನ್ನು ಸರಿಯಾಗಿ ನೋಡಲು, ಅವರೇ ಹೇಳುವಂತೆ, ಏನಿಲ್ಲವೆಂದರೂ ಹತ್ತು ದಿನಗಳಾದರೂ ಬೇಕಿತ್ತು. ಯಾವುದೇ ಉತ್ತಮ ವಸ್ತುವನ್ನು ಆಸ್ವಾದಿಸುವ ಸಾಮರ್ಥ್ಯವನ್ನೂ ಅಭಿರುಚಿಯನ್ನೂ ಹೊಂದಿದ್ದ ಸ್ವಾಮೀಜಿ, ಪ್ರತಿ ದಿನವೂ ಅಲ್ಲಿಗೆ ಹೋಗುತ್ತಿದ್ದರು. ಅಲ್ಲಿನ ಹಲವಾರು ಪ್ರದರ್ಶನ ಭವನಗಳಿಗೆ ಭೇಟಿಯಿತ್ತು ಪ್ರತಿಯೊಂದು ಪ್ರದರ್ಶಿತ ವಸ್ತುವನ್ನೂ ಪರಿಶೀಲಿಸಿ ಸಂತೋಷಪಡುತ್ತಿದ್ದರು.

ಆದರೆ ಈ ಜನಾರಣ್ಯದ ನಡುವೆ ತಾವು ಒಬ್ಬಂಟಿಗರೆಂಬ ಭಾವನೆ ಅವರ ಮನಸ್ಸಿಗೆ ಬರುತ್ತಿತ್ತು. ಅವರಿಗೆ ಪರಿಚಿತರಾದವರೊಬ್ಬರೂ ಆ ದೇಶದಲ್ಲೇ ಇರಲಿಲ್ಲ. ಅಲ್ಲದೆ ಇಲ್ಲಿನ ಸಾಧಾರಣ ಜನರು ಅವರ ವಿಚಿತ್ರ ಉಡುಗೆಯನ್ನು ಕಂಡು, ನಾನಾ ರೀತಿಯಲ್ಲಿ ತೊಂದರೆ ಕೊಡು ತ್ತಿದ್ದರು. ಒಮ್ಮೆಮೇಳದಲ್ಲಿ ಹಿಂದಿನಿಂದ ಯಾವನೋ ಒಬ್ಬ ಅವರ ಪೇಟದ ಅಂಚನ್ನು ಎಳೆದ. ತಿರುಗಿ ನೋಡಿದರೆ, ಹಾಗೆ ಮಾಡಿದವನು ಅತ್ಯಂತ ಸಭ್ಯನಂತೆ ಕಂಡ. ಆಶ್ಚರ್ಯದಿಂದ ಸ್ವಾಮೀಜಿ “ಹಾಗೇಕೆ ಮಾಡಿದಿರಿ?” ಎಂದಾಗ, ಆ ಮನುಷ್ಯ ತಬ್ಬಿಬ್ಬಾದ. ಇವರಿಗೆ ಇಂಗ್ಲಿಷ್ ಬರುತ್ತದೆಂದು ಅವನು ಊಹಿಸಿರಲಿಲ್ಲ. ಮುಂದೆ ಅವರು ಬಾಸ್ಟನ್ನಿನಲ್ಲಿದ್ದಾಗ ಬೀದಿಯ ಉಡಾಳರು ಕೆಟ್ಟದಾಗಿ ಕೂಗಿ ಅವರನ್ನು ಅಣಕಿಸುತ್ತಿದ್ದರು. ಒಮ್ಮೆಯಂತೂ ದೊಡ್ಡವರೂ ಚಿಕ್ಕವರೂ ಸೇರಿದ ಇಂತಹ ಒಂದು ಗುಂಪು ಅವರನ್ನು ಅಟ್ಟಿಸಿಕೊಂಡು ಬಂದಿತು. ಸ್ವಾಮೀಜಿ ವೇಗವಾಗಿ ಓಡಿ, ಕತ್ತಲೆಯ ಗಲ್ಲಿಯೊಂದರೊಳಗೆ ನುಸುಳಿ ಆ ದುಷ್ಟರಿಂದ ತಪ್ಪಿಸಿಕೊಂಡರು. ನಾಗರಿಕ-ಸುಸಂಸ್ಕೃತ ಜನಾಂಗ ವೆಂದು ಹೆಮ್ಮೆ ಪಟ್ಟುಕೊಳ್ಳುವ ಅಮೆರಿಕನ್ನರ ಬಗ್ಗೆ ಅವರಿಗೆ ತೀವ್ರ ಜುಗುಪ್ಸೆಯಾಯಿತು. ಮುಂದೆ ಅವರೊಂದಿಗೆ ಮಾತನಾಡುವಾಗ ಯಾರಾದರೂ ಭಾರತೀಯರನ್ನು ಅಸಂಸ್ಕೃತರೆಂದು ಹೀನಾಯ ಮಾಡಿದರೆ, ಸ್ವಾಮೀಜಿ ತಮ್ಮ ಈ ಅನುಭವಗಳನ್ನು ಹೇಳಿ ಬಳಿಕ ಅಮೆರಿಕದ ‘ಸಂಸ್ಕೃತಿ’ಯನ್ನು ಅತ್ಯಂತ ಕಟುವಾಗಿ ಟೀಕಿಸುತ್ತಿದ್ದರು.

ಸ್ವಾಮೀಜಿ ಶಿಕಾಗೋಗೆ ಬಂದ ಸಮಯದಲ್ಲೇ ಕಪುರ್ತಲ ಸಂಸ್ಥಾನದ ಮಹಾರಾಜನೂ ಅಲ್ಲಿಗೆ ಬಂದಿದ್ದ. ಈತ ತನ್ನ ಪೋಷಾಕಿನಿಂದಲೂ ದುಂದುವೆಚ್ಚದಿಂದಲೂ ಎಲ್ಲರ ಗಮನ ಸೆಳೆದಿದ್ದ. ಆದ್ದರಿಂದ ಎಷ್ಟೋ ಜನ ಸ್ವಾಮೀಜಿಯ ಉಡುಗೆಯನ್ನು ಕಂಡು ಅವರನ್ನೂ ಒಬ್ಬ ಮಹಾರಾಜನೆಂದೇ ಭಾವಿಸಿದರು. ಆದ್ದರಿಂದ ಅವರು ಹೋದಹೋದಲ್ಲೆಲ್ಲ ಜನ ಅವರಿಂದ ದುಡ್ಡು ಕೀಳಲು ನೋಡುತ್ತಿದ್ದರು. ಮತ್ತೆ ಕೆಲವರು, ಅವರ ಭವ್ಯವ್ಯಕ್ತಿತ್ವದಿಂದ ಆಕರ್ಷಿತರಾಗಿ ಅವರ ಬಳಿಗೆ ಬಂದು ಮಾತನಾಡಿಸುತ್ತಿದ್ದರು. ಜೊತೆಗೆ, ಯಾವಾಗಲೂ ಹೊಸ ಸುದ್ದಿಗಾಗಿ ಕಾದು ನಿಂತಿರುತ್ತಿದ್ದ ಪತ್ರಿಕಾ ವರದಿಗಾರರು ಸ್ವಾಮೀಜಿಯನ್ನು ಕಂಡೊಡನೆ ಬಂದು ಮುತ್ತಿದರು. ಬಗೆಬಗೆಯ ಪ್ರಶ್ನೆಗಳನ್ನು ಕೇಳಿದರು. ಇನ್ನು ಕೆಲವರು ಅವರು ಇಳಿದುಕೊಂಡಿದ್ದ ಹೋಟೆಲಿಗೇ ಬಂದು ಸಂದರ್ಶಿಸಿದರಲ್ಲದೆ, ಹೋಟೆಲಿನ ಮ್ಯಾನೇಜರಿನಿಂದಲೂ ಸಾಧ್ಯವಾದಷ್ಟು ಸುದ್ದಿ ಸಂಗ್ರಹಿಸಿದರು. ಸ್ವಾಮೀಜಿಯ ಇಷ್ಟಾನಿಷ್ಟಗಳನ್ನು ಗಣಿಸದೆ ತಮ್ಮತಮ್ಮ ಪತ್ರಿಕೆಗಳಲ್ಲಿ ಅವರ ಬಗ್ಗೆ ವರದಿಗಳನ್ನು ಬರೆದರು. ಒಂದೆರಡು ಪತ್ರಿಕೆಗಳಲ್ಲಿ ಅವರನ್ನು ‘ರಾಜಾ ವಿವೇಕಾನಂದ’ ಎಂದು ಸಂಬೋಧಿಸಲಾಗಿತ್ತು! ಮತ್ತೆ ಕೆಲವು, ಅವರನ್ನು ಒಬ್ಬ ಜ್ಞಾನಿಯೆಂದೂ ಬುದ್ಧಿವಂತ ನೆಂದೂ ಬಣ್ಣಸಿದ್ದುವು.

ಆದರೆ ಇದಕ್ಕಿಂತ ತಮಾಷೆಯಾದ ಇನ್ನೊಂದು ಘಟನೆ ನಡೆಯಿತು. ಈ ಪ್ರದರ್ಶನದ ಸ್ಥಳಕ್ಕೆ ಒಬ್ಬ ವಿಲಕ್ಷಣ ಮರಾಠೀ ಬ್ರಾಹ್ಮಣ ಬಂದಿದ್ದ. ಇಷ್ಟು ದೂರ ಅದು ಹೇಗೆ ಬಂದಿದ್ದನೋ ಮಹಾರಾಯ! ಇವನು ಉಗುರಿನಿಂದ ಚಿತ್ರಗಳನ್ನು ಬರೆದು ಮಾರುತ್ತಿದ್ದ. ಒಮ್ಮೆ ಇವನನ್ನು ಪತ್ರಿಕಾ ವರದಿಗಾರರು ಮಾತನಾಡಿಸಿದಾಗ, ಆಗ ಅಲ್ಲಿ ಪ್ರಸಿದ್ಧನಾಗಿದ್ದ ಕಪುರ್ತಲದ ರಾಜನ ಸಂಬಂಧವಾಗಿ, “ಅವನು ಕೀಳುಜಾತಿಯವನು; ಈ ರಾಜರು ಎಷ್ಟೇ ಆದರೂ ಬ್ರಿಟಿಷ್ ಸರ್ಕಾರದ ಗುಲಾಮರು ಮಾತ್ರವೇ; ಇವರೆಲ್ಲ ನೀತಿಗೆಟ್ಟವರು... ” ಎಂಬಿತ್ಯಾದಿಯಾಗಿ ಸಿಕ್ಕಿದಂತೆ ಟೀಕೆ ಮಾಡಿದ. ಇಂತಹ ರಸಭರಿತವಾದ ಸುದ್ದಿ ಸಿಕ್ಕಿದ್ದರಿಂದ ಪತ್ರಿಕಾ ಸಂಪದಾಕರಿಗೆ ಬಹಳ ಖುಷಿಯಾಯಿತು. ಆದರೆ ಆ ಮಾತುಗಳನ್ನಾಡಿದವನು ಒಬ್ಬ ಯಃಕಿಶ್ಚಿತ್ ಬ್ರಾಹ್ಮಣ ನೆಂದರೆ ಅಷ್ಟು ಪರಿಣಾಮಕಾರಿಯಾಗಿರಲಾರದೆಂದು ಊಹಿಸಿ, ಅವುಗಳನ್ನೆಲ್ಲ ಸ್ವಾಮೀಜಿಯ ತಲೆಗೆ ಕಟ್ಟಿದರು! ಭಾರತದಿಂದ ಬಂದಿರುವ ಈ ಮಹಾಜ್ಞಾನಿಯ-ಎಂದರೆ ವಿವೇಕಾನಂದರ– ಬಗ್ಗೆ ತಮ್ಮ ಪತ್ರಿಕೆಗಳಲ್ಲಿ ದೊಡ್ಡದೊಡ್ಡ ಕಾಲಂಗಳ ತುಂಬ ಬರೆದು, ಅವರನ್ನು ಹೊಗಳಿ ಆಕಾಶ ಕ್ಕೇರಿಸಿದರು; ಬಳಿಕ ಆ ವಿಲಕ್ಷಣ ಬ್ರಾಹ್ಮಣನ ಮಾತುಗಳೆಲ್ಲ ಸ್ವಾಮೀಜಿಯ ಮಾತುಗಳೆಂಬಂತೆ ಬರೆದುಬಿಟ್ಟರು! ಮರುದಿನ ಸ್ವಾಮೀಜಿ ದಿನಪತ್ರಿಕೆಯನ್ನು ಓದುತ್ತಿರುವಾಗ, ಆ ಮರಾಠೀ ಬ್ರಾಹ್ಮಣನ ಅಸಂಬಂದ್ಧ ಮಾತುಗಳನ್ನೆಲ್ಲ ತಮ್ಮ ಹೆಸರಿನಲ್ಲಿ ಹಾಕಿ ತಮ್ಮನ್ನು ಸುದ್ದಿಗೆಳೆದ ದ್ದನ್ನು ಕಂಡು ಆಶ್ಚರ್ಯಚಕಿತರಾದರು. ಅಂತಹ ಮಾತುಗಳನ್ನು ಅವರು ಊಹಿಸಲೂ ಸಾಧ್ಯವಿರ ಲಿಲ್ಲ. ಅಂತೂ ಆ ಸಂಪಾದಕರ ಊಹೆ ನಿಜವಾಯಿತು. ಪತ್ರಿಕೆಗಳಲ್ಲಿ ಬರೆದದ್ದನ್ನೆಲ್ಲ ನಂಬಿದ ಶಿಕಾಗೋದ ಸಮಾಜ ಆ ರಾಜನನ್ನು ಎಷ್ಟು ತಿರಸ್ಕಾರದಿಂದ ಕಂಡಿತೆಂದರೆ, “ಇನ್ನು ಸಾಕಪ್ಪ ಈ ಶಿಕಾಗೋ” ಎಂದು ಅವನು ಅಲ್ಲಿಂದ ಓಡಿಹೋದ. ಅಮೆರಿಕದ ಪತ್ರಿಕಾ ಸಂಪಾದಕರ ಈ ವಿಪರೀತ ಬುದ್ಧಿಯನ್ನು ಕಂಡು ಸ್ವಾಮೀಜಿ ದಂಗಾದರು. ತಮ್ಮನ್ನು ಉಪಯೋಗಿಸಿಕೊಂಡು ತಮ್ಮ ದೇಶವನಿಗೇ ತಿರಸ್ಕಾರ ದೊರೆಯುವಂತೆ ಮಾಡಿದ ರೀತಿಯನ್ನು ಕಂಡು ಸ್ವತಃ ಅವರೂ ಅಲ್ಲಿನ ರೀತಿನೀತಿಗಳ ಬಗ್ಗೆ ಒಂದು ಪಾಠ ಕಲಿತರು. ಅವರು ತಮ್ಮ ಮದರಾಸೀ ಶಿಷ್ಯರಿಗೆ ಬರೆದ ಪತ್ರದಲ್ಲಿ ಈ ಬಗ್ಗೆ ತಿಳಿಸುತ್ತ “ಇದು ಈ ದೇಶದಲ್ಲಿ ಹಣ ಹಾಗೂ ಬಿರುದುಗಳಿಗಿಂತ ಬುದ್ಧಿಗೇ ಹೆಚ್ಚು ಬೆಲೆ ಎಂಬುದನ್ನು ತೋರಿಸುತ್ತದೆ” ಎನ್ನುತ್ತಾರೆ.

ಕ್ರಮೇಣ ಸ್ವಾಮೀಜಿಗೆ ಈ ವಿಚಿತ್ರ ಪರಿಸರ ಅಭ್ಯಾಸವಾಗುತ್ತ ಬಂದಿತು. ಆದರೆ ಅವರಿಗೆ ಕೆಲವೊಮ್ಮೆ ಬೇಸರವೂ ಆಗುತ್ತಿತ್ತು. ಏಕೆಂದರೆ ಅಲ್ಲಿ ಈಗಾಗಲೇ ಕೆಲವರು ಪರಿಚಿತರಾಗಿದ್ದರೂ ನಿಜವಾದ ಸ್ನೇಹಿತರು ಯಾರೂ ಇರಲಿಲ್ಲ. ಆದ್ದರಿಂದ ತಮ್ಮ ಮನಸ್ಸಿನ ಭಾವನೆಗಳನ್ನು ತೋಡಿ ಕೊಳ್ಳಲು ಅವರಿಗೆ ಒಬ್ಬನೇ ಒಬ್ಬನೂ ಇರಲಿಲ್ಲ. ಜಾಗತಿಕ ಮೇಳದ ನಿರ್ದೇಶಕರಲ್ಲೊಬ್ಬನಿಗೆ ಮದರಾಸಿನ ಭಕ್ತರಾದ ವರದರಾವ್ ಎಂಬವರು ಪರಿಚಯಪತ್ರ ಕೊಟ್ಟಿದ್ದರು. ಸ್ವಾಮೀಜಿ ಆ ನಿರ್ದೇಶಕನನ್ನು ಕಂಡಾಗ, ಅವನೂ ಅವನ ಪತ್ನಿಯೂ ಅವರನ್ನು ಸ್ವಾಗತಿಸಿ, ಚೆನ್ನಾಗಿ ಮಾತ ನಾಡಿಸಿದರು. ಆದರೆ ಸ್ವಾಮೀಜಿಗೆ ಅವರಿಂದ ಏನೇನೂ ಉಪಕಾರವಾಗಲಿಲ್ಲ. ಅವರು ತಮ್ಮ ಕುತೂಹಲ ತೀರುವವರೆಗೆ ಮಾತನಾಡಿಸಿ ಬಳಿಕ ಸುಮ್ಮನಾದರು. (ಈ ದಂಪತಿಗಳ ಹೆಸರು ಏನೆಂಬುದು ತಿಳಿದುಬಂದಿಲ್ಲ.) ಎಷ್ಟೋ ಜನರಿಗೆ ಅವರೊಂದು ಅಚ್ಚರಿಯ ವಸ್ತುವಿನಂತೆ ಕಾಣುತ್ತಿದ್ದುದರಿಂದ ಸುಮ್ಮನೆ ಮಾತನಾಡಿಸಿ ಮುನ್ನಡೆಯುತ್ತಿದ್ದರು.

ವಿಶ್ವಧರ್ಮ ಸಮ್ಮೇಳನದಲ್ಲಿ ಭಾಗವಹಿಸುವುದಕ್ಕಾಗಿ ಇಷ್ಟು ದೂರ ಬಂದಿದ್ದರೂ, ಅದು ಪ್ರಾರಂಭವಾಗುವುದು ಯಾವಾಗ ಎಂದು ಅವರಿಗಿನ್ನೂ ತಿಳಿದಿರಲಿಲ್ಲ. ಇನ್ನೇನು ಕೆಲದಿನಗಳಲ್ಲಿ ಅದು ಪ್ರಾರಂಭವಾದೀತೆಂದು ಅವರು ನಿರೀಕ್ಷಿಸಿದ್ದರು. ಆದರೆ ಅಲ್ಲಿನ ವಿಚಾರಣಾ ವಿಭಾಗದಲ್ಲಿ ವಿಚಾರಿಸಿದಾಗ ಅವರಿಗೆ ಅತಿ ದೊಡ್ಡ ಆಘಾತ ಕಾದಿತ್ತು. ಮೊದಲನೆಯಾದಗಿ ಧರ್ಮಸಮ್ಮೇಳ ನವು ಪ್ರಾರಂಭವಾಗುವುದು ಸೆಪ್ಟೆಂಬರ್​ನ ಎರಡನೇ ವಾರದಲ್ಲಿ, ಅದಕ್ಕೆ ಮುಂಚೆಯಲ್ಲ ಎಂದು ತಿಳಿಯಿತು. ಎರಡನೆಯದಾಗಿ, ಸಮ್ಮೇಳನದಲ್ಲಿ ಭಾಷಣ ಮಾಡುವವರು ಯಾವುದಾದರೂ ಸಂಘ-ಸಂಸ್ಥೆಯ ಪ್ರತಿನಿಧಿಯಾಗಿರಬೇಕಿತ್ತು. ಇಲ್ಲವೆ ಸುಪ್ರಸಿದ್ಧ ವ್ಯಕ್ತಿಗಳಿಂದ ಪರಿಚಯಪತ್ರ ವನ್ನಾದರೂ ತಂದಿರಬೇಕಾಗಿತ್ತು. ಕಡೆಯದಾಗಿ, ಮತ್ತು ಅತಿಮುಖ್ಯವಾಗಿ ಪ್ರತಿನಿಧಿಗಳನ್ನು ದಾಖಲು ಮಾಡಿಕೊಳ್ಳುವ ವಾಯಿದೆ ಮುಗಿದುಹೋಗಿತ್ತು! ಇದನ್ನು ಕೇಳಿ ಸ್ವಾಮೀಜಿ ಕುಸಿದು ಹೋದರು. ಅವರಿಗಾದ ನಿರಾಶೆ ಅಷ್ಟಿಷ್ಟಲ್ಲ. ಹೇಗಾದರೂ ಮಾಡಿ ಯಾರಿಂದಲಾದರೂ ಪರಿಚಯ ಪತ್ರವನ್ನಾದರೂ ಪಡೆದುಕೊಳ್ಳಬಹುದೇನೊ ಎಂದರೆ ಪ್ರವೇಶಕ್ಕೆ ಗಡುವು ತೀರಿ ಹೋಗಿದೆ!

ತಾವು ಅಷ್ಟೆಲ್ಲ ಕಷ್ಟಪಟ್ಟುಕೊಂಡು ಭಾರತದಿಂದ ಇಲ್ಲಿಯವರೆಗೆ ಬಂದು ಕೊನೆಗೆ ಏನೂ ಪ್ರಯೋಜನವಿಲ್ಲದಂತಾಯಿತಲ್ಲ ಎಂದು ಸ್ವಾಮೀಜಿ ಮರುಗಿದರು. ‘ನಾನೆಂಥ ಮೂರ್ಖ ಕೆಲಸ ಮಾಡಿದೆ! ಏನೋ ಸಾಧಿಸಿಬಿಡುತ್ತೇನೆಂದು ಬಂದುಬಿಟ್ಟೆನಲ್ಲ! ಇಷ್ಟೆಲ್ಲ ರೀತಿನೀತಿಗಳಿರುತ್ತವೆ ಯೆಂದು ನನಗೆ ತೋಚಬೇಡವೆ? ಹೋಗಲಿ, ನನ್ನನ್ನು ಇಲ್ಲಿಗೆ ಕಳಿಸಿಕೊಟ್ಟ ಆ ಮದರಾಸೀ ಶಿಷ್ಯರಿಗಾದರೂ ಅದು ತಿಳಿದಿರಬೇಡವೆ? ಅಂತೂ ಆ ಭೋಳೇಶಂಕರರ ಮಾತು ಕೇಳಿಕೊಂಡು ಇಲ್ಲಿಗೆ ಬಂದದ್ದು ನನ್ನದೇ ತಪ್ಪು’ ಎಂದು ಪಶ್ಚಾತ್ತಾಪ ಪಟ್ಟರು. ಮುಂದೆ ಸೋದರಿ ನಿವೇದಿತಾ ಈ ಶಿಷ್ಯರ ಅಂದಿನ ಮನಸ್ಥಿತಿಯ ಬಗ್ಗೆ ಬಹಳ ಚೆನ್ನಾಗಿ ಬರೆಯುತ್ತಾಳೆ:

“ಈ ಶಿಷ್ಯರಿಗೆ ಸ್ವಾಮೀಜಿಯ ಸಾಮರ್ಥ್ಯದ ಮೇಲಿದ್ದ ಅವಿಚಲ ವಿಶ್ವಾಸದಿಂದಾಗಿ, ತಾವು ಅವರಿಂದ ಮನುಷ್ಯಮಾತ್ರರಿಗೆ ಸಾಧ್ಯವಿಲ್ಲದಂತಹ ಸಾಧನೆಯನ್ನು ನಿರೀಕ್ಷಿಸುತ್ತಿದ್ದೇವೆ ಎಂಬುದು ಹೊಳೆಯಲೇ ಇಲ್ಲ. ವಿವೇಕಾನಂದರು ಅಲ್ಲಿಗೆ ಹೋಗಿ ಮುಖ ತೋರಿಸಿದರೆ ಸಾಕು, ಸಮ್ಮೇಳನದಲ್ಲಿ ಮಾತನಾಡಲು ಅವರಿಗೆ ಅವಕಾಶ ಸಿಕ್ಕಿಬಿಡುತ್ತದೆ ಎಂದು ಈ ಶಿಷ್ಯರು ನಂಬಿ ದ್ದರು. ಪ್ರಪಂಚದ ರೀತಿನೀತಿಗಳ ವಿಷಯದಲ್ಲಿ ಸ್ವಾಮೀಜಿಯೂ ತಮ್ಮ ಶಿಷ್ಯರಷ್ಟೇ ಮುಗ್ಧರಾಗಿ ದ್ದರು. ಅಲ್ಲದೆ, ತಾವು ಈ ಕಾರ್ಯವನ್ನು ಕೈಗೊಂಡಿರುವುದು ಭಗವದಿಚ್ಛೆಯ ಪ್ರಕಾರವೇ ಎಂದು ಒಮ್ಮೆ ಅವರಿಗೆ ದೃಢವಾದ ಮೇಲೆ, ಇನ್ನಾವುದೇ ಬಗೆಯ ಅಡಚಣೆಯನ್ನೂ ಅವರು ನಿರೀಕ್ಷಿಸಿ ರಲೇ ಇಲ್ಲ. ಹೀಗೆ, ಜಗತ್ತಿನ ಶಕ್ತಿ -ಐಶ್ವರ್ಯಗಳ ನಾಡಿನ ಅಭೇದ್ಯ ದ್ವಾರವನ್ನು ಪ್ರವೇಶಿಸಲು, ಯಾವುದೇ ಬಗೆಯ ರುಜುವಾತುಪತ್ರವಿಲ್ಲದೆ ತನ್ನ ಪ್ರತಿನಿಧಿಯನ್ನು ಕಳಿಸಿಕೊಟ್ಟ ಹಿಂದೂ ಧರ್ಮದ ಅವ್ಯವಸ್ಥೆಗೆ ಇದಕ್ಕಿಂತ ಒಳ್ಳೆಯ ಉದಾಹರಣೆ ಬೇಕಿರಲಿಲ್ಲ.”

ಹೌದು. ನಿಜಕ್ಕೂ ಅಲ್ಲಿ ಯೋಜನೆಯೆಂಬುದೇ ಇರಲಿಲ್ಲ. ಮದರಾಸೀ ಶಿಷ್ಯರಾದ ಅಳಸಿಂಗ ಮೊದಲಾದವರು, ಮನ್ಮಥಬಾಬು, ಮುನ್ಷಿ ಜನಮೋಹನಲಾಲ್, ಮಹಾರಾಜ ಅಜಿತ್​ಸಿಂಗ್​– ಎಲ್ಲರೂ ಒಬ್ಬರಿಗಿಂತ ಒಬ್ಬರು ಬುದ್ಧಿವಂತರೇ. ಎಲ್ಲರೂ ಲೋಕದ ವಿಧಾನಗಳನ್ನು ತಿಳಿದವರೇ. ಹೀಗಿದ್ದೂ ಅವರು, ಸ್ವಾಮೀಜಿ ತಮ್ಮ ದಾರಿಯನ್ನು ತಾವೇ ರೂಪಿಸಿಕೊಳ್ಳುತ್ತಾರೆ ಎಂದು ಊಹಿಸಿ ಯಾವುದೇ ಯೋಜನೆಯಿಲ್ಲದೆ ಕಳಿಸಿಕೊಟ್ಟದ್ದನ್ನು ನೋಡಿದರೆ ಅತ್ಯಾಶ್ಚರ್ಯವಾಗ ದಿರದು. ಆದರೆ ಕಡೆಗೂ ಅವರ ಊಹೆ ನಿಜವಾದದ್ದು ಮಾತ್ರ ಅತಿ ದೊಡ್ಡ ಆಶ್ಚರ್ಯ!

ಈಗ ಸ್ವಾಮೀಜಿ, ಮುಂದೇನು ಮಾಡುವುದೆಂದು ಚಿಂತಿಸತೊಡಗಿದರು. ವಿಶ್ವಧರ್ಮ ಸಮ್ಮೇಳನವನ್ನು ಕೇವಲ ಒಬ್ಬ ಪ್ರೇಕ್ಷಕರಾಗಿ ಕುಳಿತು ನೋಡೋಣವೆಂದರೆ ಅದಕ್ಕೂ ಇನ್ನೊಂದು ತಿಂಗಳ ಕಾಲಾವಕಾಶವಿತ್ತು. ಅಷ್ಟು ದಿನ ಅವರು ಶಿಕಾಗೋದಲ್ಲಿ ಉಳಿದುಕೊಳ್ಳಲು ಸಾಧ್ಯವೇ ಇರಲಿಲ್ಲ. ಅವರ ಹಣದ ಚೀಲ ವೇಗವಾಗಿ ಬರಿದಾಗುತ್ತಿತ್ತು. ಉತ್ತಮ ದರ್ಜೆಯ ಹೋಟೆಲಿ ನಲ್ಲಿ ಇಳಿದುಕೊಳ್ಳುವ ಅವರ ಆಲೋಚನೆಯೇನೋ ಸರಿಯಾದದ್ದೇ ಆಗಿತ್ತು. ಆದರೆ ಖರ್ಚು ಮಾತ್ರ ತೀರ ಹೆಚ್ಚಾಗಿತ್ತು. ಊಟ-ವಸತಿಯ ಖರ್ಚು ಮಾತ್ರವಲ್ಲದೆ, ಪ್ರತಿಯೊಂದು ವಸ್ತುವೂ ಅಲ್ಲಿ ಅತಿ ದುಬಾರಿಯಾಗಿತ್ತು. ಆದ್ದರಿಂದ ಬೇಗ ಏನಾದರೊಂದನ್ನು ಮಾಡಲೇಬೇಕಿತ್ತು. ತಾವೇ ಖಾಸಗಿಯಾಗಿ ಉಪನ್ಯಾಸಗಳನ್ನು ನೀಡುವ ಬಗ್ಗೆಯೂ ಅವರು ಆಲೋಚಿಸಿದರು. ಆದರೆ ಅಮೆರಿಕದಲ್ಲಿ ಇಂತಹ ಉಪನ್ಯಾಸಗಳು ಪ್ರಾರಂಭವಾಗುವುದು ಮಾಗಿಯ ಕಾಲದಲ್ಲಿ. ಎಂದರೆ ಅದಕ್ಕೂ ಅವರು ಮೂರು ತಿಂಗಳು ಕಾಯಬೇಕಿತ್ತು. ಇಷ್ಟೆಲ್ಲ ಆದರೂ ಅವರು ನಿರಾಶರಾಗಲಿಲ್ಲ. ಭಾರತಕ್ಕೆ ಹಿಂದಿರುಗುವ ಆಲೋಚನೆಯನ್ನು ಅವರು ತಳ್ಳಿಹಾಕಿದರು. ಭಗವಂತನ ಕೃಪೆಯನ್ನೇ ಅವರು ಇಲ್ಲಿಯವರೆಗೂ ನೆಚ್ಚಿಕೊಂಡದ್ದು, ಈಗಲೂ ಭಗವಂತ ತಮ್ಮ ಕೈಬಿಡಲಾರನೆಂದು ಅವರಿಗೆ ಸಂಪೂರ್ಣ ವಿಶ್ವಾಸವಿತ್ತು. ಆದ್ದರಿಂದ ಅವನ ಇಚ್ಛೆಯಂತಾಗಲಿ ಎಂದು ತೀರ್ಮಾನಿಸಿದರು.

ಈ ವೇಳೆಗೆ ಅವರಿಗೆ ಯಾರಿಂದಲೋ ತಿಳಿದುಬಂತು–ಮಸ್ಸಾಚುಸೆಟ್ಸ್ ರಾಜ್ಯದ ಬಾಸ್ಟನ್ ನಗರದಲ್ಲಿ ಬೆಲೆಗಳು ಶಿಕಾಗೋಗಿಂತ ಬಹಳಷ್ಟು ಕಡಿಮೆ ಎಂದು. ಆದ್ದರಿಂದ ಸ್ವಾಮೀಜಿ, ಇನ್ನೂ ಒಂದು ತಿಂಗಳ ಕಾಲವನ್ನು ಅಲ್ಲಿ ಕಳೆಯುತ್ತ ಮತ್ತೊಮ್ಮೆ ತಮ್ಮ ಅದೃಷ್ಟವನ್ನು ಪರೀಕ್ಷಿಸಿ ನೋಡಿಕೊಳ್ಳುವುದು ಎಂದು ತೀರ್ಮಾನಿಸಿದರು. ಶಿಕಾಗೋದಲ್ಲಿನ ಹನ್ನೆರಡು ದಿನಗಳ ವಾಸದ ಬಳಿಕ, ಅಲ್ಲಿಂದ ಸುಮಾರು ಒಂದು ಸಾವಿರ ಮೈಲಿ ದೂರದ ಬಾಸ್ಟನ್ನಿಗೆ ಹೋಗಲು ಟ್ರೈನು ಹತ್ತಿದರು.

ನಿಜಕ್ಕೂ ಭಗವಂತನ ವಿಧಾನ ಅತಿ ವಿಚಿತ್ರ. ಸ್ವಾಮೀಜಿ ಪರಿವ್ರಾಜಕರಾಗಿ ಓಡಾಡುತ್ತಿದ್ದಾಗ, ಅವರ ಅತ್ಯಂತ ಕಷ್ಟದ ಸಂದರ್ಭಗಳಲ್ಲಿ ಭಗವಂತನ ಸಹಾಯಹಸ್ತ ಒದಗಿಬಂದದ್ದನ್ನು ಕಂಡಿದ್ದೇವೆ. ಈಗಲೂ ಮತ್ತೊಮ್ಮೆ ಆ ಭಗವಂತನೇ ನೆರವಿಗೆ ಬಂದ–ಅದೇ ಬೋಗಿಯಲ್ಲಿ ಪ್ರಯಾಣ ಮಾಡುತ್ತಿದ್ದ ಮಿಸ್ ಕ್ಯಾಥರಿನ್ ಆಬಟ್ ಸ್ಯಾನ್​ಬಾರ್ನ್ \eng{(Miss Catherine Abbot Sanborn)} ಎಂಬ ಮಹಿಳೆಯ ಮೂಲಕ. ಸ್ವಾಮೀಜಿಯ ತೇಜಸ್ಸು, ಅವರ ನಿರ್ಲಿಪ್ತ-ಗಂಭೀರ ಮುಖಭಾವಗಳನ್ನು ಕಂಡು ಆಕರ್ಷಿತಳಾದ ಆ ಮಹಿಳೆ, ತಾನಾಗಿಯೇ ಅವರ ಬಳಿಗೆ ಬಂದು ಮಾತಿಗೆ ತೊಡಗಿದಳು. ಅವರೊಬ್ಬ ಹಿಂದೂ ಸಂನ್ಯಾಸಿಯೆಂದೂ ವೇದಾಂತದ ಸತ್ಯಗಳನ್ನು ಪ್ರಸಾರ ಮಾಡಲು ಅಮೆರಿಕೆಗೆ ಬಂದಿದ್ದಾರೆಂದೂ ತಿಳಿದಾಗ ಆಕೆಯ ಕುತೂಹಲ ಮತ್ತಷ್ಟು ಹೆಚ್ಚಾಯಿತು. ಬಳಿಕ ಸ್ವಾಮೀಜಿ ತಮ್ಮ ಅವಸ್ಥೆಯನ್ನು ಹೇಳಿಕೊಂಡರು. ಅಮೆರಿಕದಲ್ಲಿ ಅವರಿಗೆ ಯಾರೂ ಪರಿಚತರಿಲ್ಲವೆಂಬುದು ತಿಳಿದಾಗ, ತಕ್ಷಣ ಆ ಮಹಿಳೆ ಅವರಿಗೆ ನೆರವಾಗಲು ಮುಂದಾ ದಳು. “ಸ್ವಾಮೀಜಿ, ನಿಮಗೆ ಒಪ್ಪಿಗೆಯಾಗುವುದಾದರೆ, ದಯವಿಟ್ಟು ನಮ್ಮ ಮನೆಗೆ ಬಂದಿರಿ. ಬಹುಶಃ ನಿಮ್ಮ ಉದ್ದೇಶ ನೆರವೇರುವಂತೆ ಏನಾದರೂ ಅನುಕೂಲವಾದರೂ ಆಗಬಹುದು” ಎಂದು ಸ್ವಾಗತಿಸಿದಳು. ಜಗನ್ಮಾತೆಯ ಇಚ್ಛೆಯಂತಾಗಲಿ ಎಂದು ಸ್ವಾಮೀಜಿ ಕೂಡಲೇ ಆಹ್ವಾನಕ್ಕೆ ಒಪ್ಪಿದರು.

ಮಿಸ್ ಸ್ಯಾನ್​ಬಾರ್ನಳು ಸ್ವಾಮೀಜಿಯನ್ನು ಮೆಟ್​ಕಾಫ್ ಎಂಬ ಪಟ್ಟಣದ ಬಳಿಯಿದ್ದ ತನ್ನ ‘ಬ್ರೀಸಿ ಮೆಡೋಸ್​\eng{’ (Breezy Meadows)} ಎಂಬ ಸುಂದರವಾದ ಹೊಲದ ಮನೆಗೆ ಕರೆ ದೊಯ್ದಳು. ಮೆಟ್​ಕಾಫ್ ಇರುವುದು ಬಾಸ್ಟನ್ ನಗರಕ್ಕೆ ಸಮೀಪದಲ್ಲೇ. ಮಧ್ಯ ವಯಸ್ಕಳಾದ ಮಿಸ್ ಸ್ಯಾನ್​ಬಾರ್ನ್ ಅಧ್ಯಾಪಕಿಯಾಗಿಯೂ ಒಳ್ಳೆಯ ಗ್ರಂಥಕರ್ತೆಯಾಗಿಯೂ ಹೆಸರು ಗಳಿಸಿ ದ್ದಳು. ಅಲ್ಲದೆ ಈಕೆಯ ಪರಿಚಿತರಲ್ಲಿ ಆ ಕಾಲದ ಪ್ರಸಿದ್ಧ ಚಿಂತಕರೂ ಬುದ್ಧಿಜೀವಿಗಳೂ ಇದ್ದರು.

ಮಿಸ್ ಸ್ಯಾನ್​ಬಾರ್ನ್ ತನ್ನ ಅತಿಥಿಯಾದ ಸ್ವಾಮೀಜಿಯನ್ನು ಬಹಳ ಚೆನ್ನಾಗಿಯೇ ನೋಡಿ ಕೊಂಡಳು. ಸುತ್ತುಮುತ್ತಲಿನ ಹಲವಾರು ಜನರನ್ನೂ ಇತರ ಸ್ನೇಹಿತರನ್ನೂ ಮನೆಗೆ ಆಹ್ವಾನಿಸಿ, ಅವರಿಗೆ ಸ್ವಾಮೀಜಿಯನ್ನು “ಭಾರತದಿಂದ ಬಂದಿರುವ ಕುತೂಹಲಕರ ವ್ಯಕ್ತಿ” ಎಂದು ಪರಿಚಯಿ ಸಿದಳು. ಹೀಗೆ ಬರುತ್ತಿದ್ದವರಲ್ಲೆಷ್ಟೋ ಜನ, ಇವರನ್ನು ಒಬ್ಬ ‘ಅಸಂಸ್ಕೃತ ವಿಧರ್ಮಿ’ಯೆಂದು ಭಾವಿಸಿ, ಅವರನ್ನು ನಾನಾ ಬಗೆಯ ವಿಚಿತ್ರ ಪ್ರಶ್ನೆಗಳಿಂದ ಕಾಡುತ್ತಿದ್ದರು. ಜೊತೆಗೇ ಇನ್ನು ಕೆಲವು ವಿಚಾರವಂತ ಜನರೂ ಬಂದು, ತಮ್ಮ ತಮ್ಮ ಸಾಮರ್ಥ್ಯಕ್ಕನುಗುಣವಾಗಿ ಅವರೊಂದಿಗೆ ಮಾತು ಕತೆ ನಡೆಸಿ ಹೋಗುತ್ತಿದ್ದರು.

ಹಲವಾರು ದಿನಗಳವರೆಗೆ ಸ್ವಾಮೀಜಿಯ ಪಾಲಿಗೆ ಪರಿಸ್ಥಿತಿ ನಿರಾಶಾದಾಯಕವಾಗಿಯೇ ಮುಂದುವರಿಯಿತು. ಇದರ ಜೊತೆಗೆ ಅವರು ತೀವ್ರ ಆರ್ಥಿಕ ಸಂಕಷ್ಟಕ್ಕೆ ಸಿಲುಕಿಕೊಂಡರು. ಆಗಸ್ಟ್ ತಿಂಗಳೆಂದರೆ ಅಮೆರಿಕದಲ್ಲಿ ಬೇಸಿಗೆ. ಆದರೆ ಆ ಕಡುಬೇಸಿಗೆಯೂ ಅವರಿಗೆ ಚಳಿಗಾಲ ದಂತೆ ತೋರಿತು. ಅವರ ಬಳಿ ಬೆಚ್ಚನೆಯ ಉಣ್ಣೆಬಟ್ಟೆಗಳಿರಲಿಲ್ಲ. ಅಲ್ಲದೆ ಅಮೆರಿಕನ್ನರಿಗೆ ಹೆಚ್ಚು ಪರಿಚಿತವಾದ ಉಡುಗೆಯನ್ನು ಧರಿಸುವಂತೆ ಅವರ ಆತಿಥೇಯಳು ಅವರನ್ನು ಒತ್ತಾಯಿಸು ತ್ತಿದ್ದಳು. ಅವರಿಗೂ ಅದರ ಆವಶ್ಯಕತೆ ಮನದಟ್ಟಾಗಿತ್ತು. ಈಗಾಗಲೇ ರಸ್ತೆಯಲ್ಲಿ ಓಡಾಡುವಾಗ ಅನೇಕ ಕಹಿ ಅನುಭವಗಳಾಗಿದ್ದುವು. ಅವರ ವಿಚಿತ್ರ ವೇಷವನ್ನು ಕಂಡು ಎಲ್ಲರೂ ಹಂಗಿಸು ತ್ತಿದ್ದರು. ಆದ್ದರಿಂದ ಅವರು ಹೊಸ ಬಟ್ಟೆಯನ್ನು ಕೊಳ್ಳಲೇಬೇಕಾಗಿತ್ತು. ತೀರ ಸಾಧಾರಣದ ಉಣ್ಣೆಬಟ್ಟೆಯನ್ನು ಕೊಳ್ಳಬೇಕೆಂದರೂ ೧ಂಂ ಡಾಲರ್ (ಆಗ ಸುಮಾರು ೩೩ಂ ರೂ.) ಖರ್ಚಾಗು ತ್ತಿತ್ತು. ಅವರೀಗ ಅಷ್ಟೆಲ್ಲ ಹಣವನ್ನು ಖರ್ಚುಮಾಡುವ ಸ್ಥಿತಿಯಲ್ಲಿರಲಿಲ್ಲ. ಅಲ್ಲದೆ, ಮಿಸ್ ಸ್ಯಾನ್​ಬಾರ್ನ್​ಳಿಗಾಗಲಿ ಇತರರಿಗಾಗಲಿ ಅವರ ನಿಜಸ್ಥಿತಿ ತಿಳಿದಿರಲಿಲ್ಲವೆಂದು ಕಾಣುತ್ತದೆ. ಸ್ವಾಮೀಜಿ ತಾವಾಗಿಯೇ ತಮ್ಮ ಕಷ್ಟವನ್ನು ಹೇಳಿಕೊಳ್ಳಲೂ ಇಲ್ಲ. ಆದರೆ ವಿಧಿಯಿಲ್ಲವೆನಿಸಿದಾಗ ತಮ್ಮ ಮದರಾಸೀ ಶಿಷ್ಯರಿಗೆ ಒಂದು ಪತ್ರ ಬರೆದು ತಮ್ಮ ಪರಿಸ್ಥಿತಿಯನ್ನು ಹೇಳಿಕೊಂಡರು. ತಕ್ಷಣ ಈ ಶಿಷ್ಯರು ಕಾರ್ಯಮಗ್ನರಾಗಿ ಒಟ್ಟು ಸುಮಾರು ೮ಂಂ ರೂಪಾಯಿಗಳನ್ನು ಸಂಗ್ರಹಿಸಿ ಕಳಿಸಿಕೊಟ್ಟರಾದರೂ, ಅದು ಅವರನ್ನು ತಲುಪಿದ್ದು ಸೆಪ್ಟೆಂಬರ್ ಏಳರ ಅನಂತರವೇ. ಅಷ್ಟು ಹೊತ್ತಿಗಾಗಲೇ ಅವರು ಆ ದುಃಸ್ಥಿತಿಯಿಂದ ಪಾರಾಗಿದ್ದರು. ಈ ನಡುವಿನ ಅವಧಿಯಲ್ಲಿ ಎದುರಾದ ಎಲ್ಲ ಕಷ್ಟಗಳನ್ನೂ ಅವರು ಅವಡುಗಚ್ಚಿ ಸಹಿಸಿಕೊಂಡರು.

ಈ ಮಧ್ಯೆ ‘ಬ್ರೀಸಿ ಮೆಡೋಸ್​’ನಲ್ಲಿ ಸ್ವಾಮೀಜಿ ಹೊಸ ಹೊಸ ವ್ಯಕ್ತಿಗಳನ್ನು ಭೇಟಿಯಾಗು ತ್ತಿದ್ದರು. ಮಿಸ್ ಸ್ಯಾನ್​ಬಾರ್ನ್​ಳ ಆಹ್ವಾನದ ಮೇರೆಗೆ ಅಲ್ಲಿಗೆ ಬಂದವರಲ್ಲಿ ಸಮೀಪದ ಮಹಿಳಾ ಕಾರಾಗೃಹದ ಮೇಲ್ವಿಚಾರಿಕಿಯೂ ಒಬ್ಬಳು. ಸ್ವಾಮೀಜಿ ಈ ಕಾರಾಗೃಹವನ್ನು ಸಂದರ್ಶಿಸಲು ಬಹಳ ಕುತೂಹಲ ತಾಳಿದ್ದರಿಂದ, ಆಕೆ ಅವರನ್ನು ಅಲ್ಲಿಗೆ ಕರೆದೊಯ್ದಳು. ಅಲ್ಲಿನ ವ್ಯವಸ್ಥೆ ಯನ್ನೂ ಕ್ರಮ ನಿಯಮಗಳನ್ನೂ ಕಂಡು ಸ್ವಾಮೀಜಿ ಅತ್ಯಂತ ಆನಂದಿತಾರದರು. ಆದರೆ ಮರುಕ್ಷಣವೇ ಅವರಿಗೆ ಭಾರತದ ನೆನಪು! ಅಲ್ಲಿನ ದೃಶ್ಯ ಅವರ ಮೇಲೆ ಎಷ್ಟು ಪ್ರಭಾವ ಬೀರಿತೆಂದರೆ, ಅವರು ತಮ್ಮ ಶಿಷ್ಯರಿಗೆ ಬರೆಯುತ್ತಾರೆ:

“ಇಲ್ಲಿ ಇದನ್ನು ಕಾರಾಗೃಹ ಎಂದು ಕರೆಯುವುದಿಲ್ಲ. ಬದಲಾಗಿ ‘ಸುಧಾರಣಾ ಗೃಹ\eng{’ (Reformatory)} ಎಂದು ಕರೆಯುತ್ತಾರೆ. ಇದು ನಾನು ಅಮೆರಿಕೆಯಲ್ಲಿ ಕಂಡ ಅತ್ಯಂತ ಭವ್ಯ ವಾದ ವಿಷಯ. ಇಲ್ಲಿನ ಬಂಧಿತರನ್ನು ಎಷ್ಟು ದಯಾಪೂರಿತ ದೃಷ್ಟಿಯಿಂದ ನೋಡಿಕೊಳ್ಳುತ್ತಾರೆ! ಅವರನ್ನೆಲ್ಲ ಹೇಗೆ ಪರಿವರ್ತಿಸಿ, ಸಮಾಜಕ್ಕೆ ಉಪಯುಕ್ತ ವ್ಯಕ್ತಿಗಳನ್ನಾಗಿಸಿ ಕಳಿಸಿಕೊಡುತ್ತಾರೆ! ನಿಜಕ್ಕೂ ಇದೆಷ್ಟು ಭವ್ಯ, ಇದೆಷ್ಟು ಸುಂದರ! ಇದನ್ನು ನಂಬಬೇಕಾದರೆ ನೀವು ಇಲ್ಲಿಗೆ ಬಂದು ನೋಡಬೇಕು. ಅಯ್ಯೋ! ನಮ್ಮ ದೇಶದ ಬಡವರನ್ನು, ನಿಮ್ನವರ್ಗದವರನ್ನು ನೆನೆಸಿಕೊಂಡಾಗ ನನಗೆಷ್ಟು ಸಂಕಟವಾಯಿತು ಬಲ್ಲೆಯ? ಉದ್ಧಾರವಾಗಲು ಅವರಿಗೆ ಅವಕಾಶವೇ ಇಲ್ಲ. ಮೇಲೇ ರಲು ಅವರಿಗೆ ಮಾರ್ಗವೇ ಇಲ್ಲ...”

ಈ ಸುಧಾರಣಾ ಗೃಹದ ನಿವಾಸಿಗಳನ್ನು (ಬಂಧಿತರನ್ನು) ಉದ್ದೇಶಿಸಿ ಸ್ವಾಮೀಜಿ ಮಾತನಾಡಿದ ರೆಂದು ತಿಳಿದುಬರುತ್ತದೆ. ಬಹುಶಃ ಸ್ವಾಮೀಜಿ ಮಾನವನ ನಿಜ ಸ್ವರೂಪದ ಬಗ್ಗೆ ಅಥವಾ ಆತ್ಮ ಅಕಳಂಕವೆಂಬುದರ ಬಗ್ಗೆ ಮಾತನಾಡಿರಬಹುದು. ಅವರು ಏನೇ ಮಾತನಾಡಿರಲಿ, ಕೆಲವರ ಮೇಲಂತೂ ಅದು ಅಚ್ಚಳಿಯದ ಪರಿಣಾಮವನ್ನು ಬೀರಿ, ಅವರನ್ನು ನಿಜವಾದ ಪರಿವರ್ತನೆ ಯುಂಟುಮಾಡಿದ್ದರೆ ಆಶ್ಚರ್ಯವಿಲ್ಲ.

ಇದಾದ ಕೆಲದಿನಗಳಲ್ಲೇ, ಬಾಸ್ಟನ್ನಿನ ದೊಡ್ಡ ಮಹಿಳಾಸಂಘವೊಂದರಲ್ಲಿ ಮಾತನಾಡುವಂತೆ ಅವರನ್ನು ಆಹ್ವಾನಿಸಲಾಯಿತು. ಈ ಸಂಘದ ಸದಸ್ಯೆಯರು ಭಾರತದ ‘ಸಮಾಜ ಸುಧಾರಕಿ’ ರಮಾಬಾಯಿಯ ಬೆಂಬಲಿಗರು. ಕ್ರೈಸ್ತಳಾಗಿ ಪರಿವರ್ತಿತಳಾಗಿದ್ದ ರಮಾಬಾಯಿ, ಭಾರತೀಯ ಬಾಲವಿಧವೆಯರ ಸಹಾಯಾರ್ಥವಾಗಿ ಅಮೆರಿಕದಲ್ಲಿ ಹಲವಾರು ಸಂಘಗಳನ್ನು ಸ್ಥಾಪಿಸಿ ತನ್ಮೂಲಕ ನಿಧಿ ಸಂಗ್ರಹಿಸುತ್ತಿದ್ದಳು. ಆದರೆ ಈಕೆ ಭಾರತೀಯ ವಿಧವೆಯರ ದುಃಸ್ಥಿತಿಯನ್ನು ಅತಿಶಯವಾಗಿ ಬಣ್ಣಿಸಿ, ಹೆಚ್ಚು ಸಹಾನುಭೂತಿಯನ್ನೂ ಸಹಾಯವನ್ನೂ ಗಳಿಸುವ ಪ್ರಯತ್ನದಲ್ಲಿ ತೊಡಗಿದ್ದಳು. ಮುಂದೆ ಸ್ವಾಮೀಜಿ ಪ್ರಸಿದ್ಧರಾದ ಮೇಲೆ ಈ ಸಂಘಗಳ ಸದಸ್ಯರು ಅವರನ್ನು ವಿರೋಧಿಸಿ ಸುಳ್ಳು ದೋಷಾರೋಪಣೆಗಳನ್ನು ಮಾಡಿದರಾದರೂ, ಅವರು ಮೊತ್ತಮೊದಲು ತಮ್ಮ ಸಂಘದಲ್ಲಿ ನೀಡಿದ ಉಪನ್ಯಾಸವನ್ನು ಮೆಚ್ಚಿಕೊಂಡರು.

ಮಿಸ್ ಸ್ಯಾನ್​ಬಾರ್ನ್​ಳ ಹತ್ತಿರದ ಸಂಬಂಧಿಯಾದ ಫ್ರಾಂಕ್ಲಿನ್ ಬೆಂಜಮಿನ್​ಸ್ಯಾನ್​ಬಾರ್ನ್ ಎಂಬಾತ ಪತ್ರಿಕೋದ್ಯಮಿಯಾಗಿ ಹಾಗೂ ಸಮಾಜ ಸೇವಕನಾಗಿ ಸುಪ್ರಸಿದ್ಧನಾಗಿದ್ದ. ಅಲ್ಲದೆ ಈತ ಅತೀಂದ್ರಿಯ ತತ್ತ್ವವಾದದಲ್ಲಿ ಪರಿಣತನಾಗಿದ್ದ. ತನ್ನ ಈ ವಿಶೇಷ ಅತಿಥಿಯನ್ನು ಭೇಟಿ ಯಾಗುವಂತೆ ಮಿಸ್ ಸ್ಯಾನ್​ಬಾರ್ನ್ ಈತನಿಗೆ ಪತ್ರ ಬರೆದು ಆಹ್ವಾನಿಸಿದಳು. ಒಬ್ಬ ನಿಜವಾದ ಹಿಂದೂ ಸಂನ್ಯಾಸಿಯನ್ನು ಭೇಟಿಯಾಗಲೂ ಫ್ರಾಂಕ್ಲಿನ್ ಸ್ಯಾನ್​ಬಾರ್ನ್ ಉತ್ಸುಕನಾಗಿದ್ದರೂ ಮೋಸ ಹೋಗಲು ಅವನು ಸಿದ್ಧನಿರಲಿಲ್ಲ. ಆದ್ದರಿಂದ ಸ್ವಾಮೀಜಿಯನ್ನು ಭೇಟಿಯಾದಾಗ ಎಚ್ಚರಿಕೆಯಿಂದಲೇ ಮಾತುಕತೆ ಪ್ರಾರಂಭಿಸಿದ. ಆದರೆ ಅವರ ಜಾಜ್ವಲ್ಯಮಾನ ವ್ಯಕ್ತಿತ್ವವನ್ನೂ ಪ್ರಕಾಂಡ ಪಾಂಡಿತ್ಯವನ್ನೂ ಕಂಡು ಮಾರುಹೋದ. ಬಳಿಕ ಈ ಅಸಾಧಾರಣ ವ್ಯಕ್ತಿಯನ್ನು ಹೆಚ್ಚು ಜನರಿಗೆ ಪರಿಚಯಿಸುವ ಉದ್ದೇಶದಿಂದ ನ್ಯೂಯಾರ್ಕ್ ನಗರದ ಸರಟೋಗ ಸ್ಪ್ರಿಂಗ್ಸ್ ಎಂಬಲ್ಲಿನ \eng{American Social Science Association (}ಅಮೆರಿಕ ಸಮಾಜ-ವಿಜ್ಞಾನ ಸಂಘ) ಎಂಬ ವಿದ್ವತ್ಸಭೆಯನ್ನುದ್ದೇಶಿ ಮಾತನಾಡುವಂತೆ ಆಹ್ವಾನಿಸಿದ.

ಇದೇ ವೇಳೆಗೆ, ಸ್ವಾಮೀಜಿಯ ಆತಿಥೇಯಳ ಮತ್ತೊಬ್ಬ ಪರಿಚಯಸ್ಥರಾದ ಡಾ. ಜಾನ್ ಹೆನ್ರಿ ರೈಟ್​ರವರು, ಅವರನ್ನು ಕಾಣಲು ಉತ್ಸುಕರಾಗಿದ್ದರು. ಜಗತ್ಪ್ರಸಿದ್ಧ ಹಾರ್ವರ್ಡ್ ವಿಶ್ವವಿದ್ಯಾ ನಿಲಯದಲ್ಲಿ ಗ್ರೀಕ್ ಸಾಹಿತ್ಯದ ಪ್ರಾಧ್ಯಾಪಕರಾಗಿದ್ದ ಇವರು ಈಗಾಗಲೇ ಸ್ವಾಮೀಜಿಯ ಬಗ್ಗೆ ಬಹಳಷ್ಟು ಕೇಳಿದ್ದರು. ಆಗ ಇವರು ಬಾಸ್ಟನ್ನಿನಿಂದ ೪ಂ ಮೈಲಿ ದೂರದ ಆನ್ನಿಸ್ಕ್ವಾಮ್ \eng{(Annisquam)} ಎಂಬ ಸಮುದ್ರತೀರದ ಸುಂದರ ಹಳ್ಳಿಯಲ್ಲಿ ಕುಟುಂಬ ಸಮೇತರಾಗಿ ಬೇಸಿಗೆ ಯನ್ನು ಕಳೆಯುತ್ತಿದ್ದರು. ವಾರಾಂತ್ಯಕ್ಕೆ ತಮ್ಮಲ್ಲಿಗೆ ಬರಬೇಕೆಂದು ಪ್ರೊ. ರೈಟರು ಆಹ್ವಾನಿಸಿದಾಗ ಸ್ವಾಮೀಜಿ ಅದಕ್ಕೊಪ್ಪಿ ಆಗಸ್ಟ್ ೨೫ರಂದು ಅಲ್ಲಿಗೆ ಹೋದರು.

ಪ್ರೊ. ರೈಟರು ಸರ್ವವ್ಯಾಪಕ ಪಾಂಡಿತ್ಯವನ್ನು ಗಳಿಸಿದವರೆಂದು ಹೆಸರಾಗಿದ್ದವರು. ಸ್ವಾಮೀಜಿಯ ಪಾಂಡಿತ್ಯವನ್ನು ಅಳೆಯುವ ಸಾಮರ್ಥ್ಯ ಅವರಲ್ಲಿತ್ತು. ಮಾತುಕತೆ ಪ್ರಾರಂಭಿಸಿದ ಕೆಲನಿಮಿಷಗಳಲ್ಲೇ ಅವರು ಸ್ವಾಮೀಜಿಯ ಮಾತಿನ ಮೋಡಿಗೆ ಒಳಗಾದರು. ಇಬ್ಬರು ಮಹಾ ದಿಗ್ಗಜಗಳು ಗಂಟೆಗಟ್ಟಲೆ ನಿರಂತರವಾಗಿ ಸಂಭಾಷಿಸಿದರು. ಪ್ರೊ. ರೈಟರಿಗೆ ಮಾತ್ರವಲ್ಲದೆ ಅವರ ಕುಟುಂಬವರ್ಗದವರಿಗೂ, ಅಲ್ಲಿ ನೆರೆಯುತ್ತಿದ್ದ ಇತರರೆಲ್ಲರಿಗೂ ಸ್ವಾಮೀಜಿಯ ಮಾತು ರಸದೌತಣವನ್ನೊದಗಿಸಿತು. ಅವರು ಬೆಳಗಿನಿಂದ ಮಧ್ಯರಾತ್ರಿಯವರೆಗೂ ಸ್ವಾಮೀಜಿಯೊಂದಿಗೆ ಮಾತನಾಡಿ, ಮರುದಿನ ಮುಂಜಾನೆ ಮತ್ತೆ ನವೋತ್ಸಾಹದಿಂದ ಪುನರಾರಂಭಿಸುತ್ತಿದ್ದರು! ಅವರೆಲ್ಲರ ಮೇಲೂ ಸ್ವಾಮೀಜಿಯ ಪ್ರಭಾವ ಅಚ್ಚಳಿಯದೆ ಉಳಿಯಿತು.

ಇಷ್ಟು ಹೊತ್ತಿಗಾಗಲೇ ಸ್ವಾಮೀಜಿ ವಿಶ್ವಧರ್ಮ ಸಮ್ಮೇಳನದಲ್ಲಿ ಭಾಗವಹಿಸುವ ಆಸೆಯನ್ನೂ ಉದ್ದೇಶವನ್ನೂ ತೊರೆದಿದ್ದರು. ಆದರೆ ಅವರ ಯೋಗ್ಯತೆಯೇನೆಂಬುದನ್ನು ಕಂಡುಕೊಂಡಿದ್ದ ಪ್ರೊ. ರೈಟರು, ಸಮ್ಮೇಳನದಲ್ಲಿ ಭಾಗವಹಿಸುವಂತೆ ಅವರನ್ನು ಒತ್ತಾಯ ಪಡಿಸುತ್ತ, “ನೀವು ಇಡೀ ಅಮೆರಿಕ ರಾಷ್ಟ್ರಕ್ಕೆ ಪರಿಚಿತರಾಗುವಂತಾಗಲೂ ಇದೊಂದೇ ದಾರಿ” ಎಂದು ಹೇಳಿದರು. ಆಗ ಸ್ವಾಮೀಜಿ, ತಮ್ಮ ಬಳಿ ಒಂದು ಪರಿಚಯ ಪತ್ರವೂ ಇಲ್ಲವೆಂಬ ವಿಷಯವನ್ನು ಹೇಳಿ ತಮ್ಮ ಕಷ್ಟವನ್ನು ವಿವರಿಸಿದರು. ಅದನ್ನು ಕೇಳಿದ ಪ್ರೊಫೆಸರರು ಅಚ್ಚರಿಯಿಂದ ಉದ್ಗರಿಸಿದರು, “ಸ್ವಾಮೀಜಿ, ನಿಮ್ಮನ್ನು ಅರ್ಹತಾಪತ್ರ ತೋರಿಸುವಂತೆ ಕೇಳುವುದೆಂದರೆ, ಸೂರ್ಯನನ್ನು ‘ಬೆಳಗಲು ನಿನಗೇನು ಅಧಿಕಾರ’ ಎಂದು ಕೇಳಿದಂತೆ!” ಬಳಿಕ ಅವರು, “ನೀವು ವಿಶ್ವಧರ್ಮ ಸಮ್ಮೇಳನದಲ್ಲಿ ಪ್ರತಿನಿಧಿಯಾಗಿ ಭಾಗವಹಿಸುವಂತೆ ಮಾಡುವ ಹೊಣೆ ನನ್ನದು” ಎಂದು ಭರವಸೆ ನೀಡಿದರು. ಪ್ರೊ. ರೈಟರಿಗೆ ಈ ಸಮ್ಮೇಳನಕ್ಕೆ ಸಂಬಂಧಿಸಿದ ಹಲವಾರು ಪ್ರಮುಖ ವ್ಯಕ್ತಿಗಳ ನಿಕಟ ಪರಿಚಯವಿತ್ತು. ತಕ್ಷಣ ಅವರು, ಪ್ರತಿನಿಧಿಗಳ ಆಯ್ಕೆಯ ಸಮಿತಿಯ ಅಧ್ಯಕ್ಷರಿಗೆ ಪತ್ರವೊಂದನ್ನು ಬರೆದು, ಸ್ವಾಮೀಜಿಯ ಬಗ್ಗೆ ಹೇಳುತ್ತಾರೆ: “ನಮ್ಮ ಮೇಧಾವಿಗಳಾದ ಪ್ರೊಫೆಸರುಗಳನ್ನೆಲ್ಲ ಒಟ್ಟುಗೂಡಿಸಿದರೆ ಅವರೆಲ್ಲರನ್ನೂ ಮೀರಿಸುವಂಥವರು ಇವರು.” ಅಲ್ಲದೆ ಸ್ವಾಮೀಜಿಯ ಬಳಿ ತೀರ ಕಡಿಮೆ ಹಣವಿರುವ ವಿಷಯವನ್ನು ತಿಳಿದು ಅವರಿಗೆ ಶಿಕಾಗೋವರೆಗಿನ ಪ್ರಯಾಣದ ಖರ್ಚನ್ನೂ ಕೊಟ್ಟರು. ಜೊತೆಗೆ, ಸಮ್ಮೇಳನದಲ್ಲಿ ಪೌರ್ವಾತ್ಯ ಪ್ರತಿನಿಧಿಗಳ ಊಟ ವಸತಿಯನ್ನು ನೋಡಿಕೊಳ್ಳುವ ಮುಖ್ಯಾಧಿಕಾರಿಗೂ ಪರಿಚಯಪತ್ರವನ್ನು ಬರೆದುಕೊಟ್ಟರು. ನಿಜಕ್ಕೂ ಇದು ಭಗವಂತನ ಕೈವಾಡವಲ್ಲದೆ ಮತ್ತೇನು! ಸಕಾಲದಲ್ಲಿ ತಾನೇತಾನಾಗಿ ಒದಗಿಬಂದ ದೈವಸಹಾಯವನ್ನು ಕಂಡು ಸ್ವಾಮೀಜಿಗೆ ಅತ್ಯಂತ ಆನಂದವಾಯಿತು. ಅಂತೂ ಯಾರೂ ಊಹಿಸಿರದಿದ್ದ ರೀತಿಯಲ್ಲಿ ಅವರ ಕಾರ್ಯೋದ್ದೇಶ ನೆರವೇರುವಂತಾಗಿತ್ತು.

ಸ್ವಾಮೀಜಿಯ ವಿಷಯ ಆ್ಯನ್ನಿಸ್ಕ್ವಾಮ್ ಹಳ್ಳಿಯಲ್ಲಿ ಕಾಳ್ಗಿಚ್ಚಿನಂತೆ ಹಬ್ಬಿ, ಹಳ್ಳಿಗೆ ಹಳ್ಳಿಯೇ ಅವರನ್ನು ಮುತ್ತಿಕೊಂಡಿತು. ಆ ಜನರೊಂದಿಗೆ ಸ್ವಾಮೀಜಿ ಹಲವಾರು ವಿಷಯಗಳ ಬಗ್ಗೆ ಮಾತನಾಡಿದರು; ಪ್ರಶ್ನೆಗಳನ್ನು ಉತ್ತರಿಸಿದರು. ಭಾನುವಾರದಂದು ಆ ಹಳ್ಳಿಯ ಇಗರ್ಜಿಯಲ್ಲಿ ಸ್ವಾಮೀಜಿಯ ಭಾಷಣವೊಂದನ್ನು ಏರ್ಪಡಿಸಲಾಯಿತು. ಪಾಶ್ಚಾತ್ಯ ಜಗತ್ತಿನಲ್ಲಿ ಇದೇ ಅವರ ಪ್ರಪ್ರಥಮ ಸಾರ್ವಜನಿಕ ಭಾಷಣ. ಈ ಭಾಷಣದಲ್ಲಿ ಅವರು, ಭಾರತೀಯರಿಗೆ ಅತ್ಯಾವಶ್ಯಕ ವಾಗಿರುವುದು ಧರ್ಮಬೋಧೆಯಲ್ಲ, ಬದಲಾಗಿ ಅವರನ್ನು ಅಭಿವೃದ್ಧಿಯ ಪಥಕ್ಕೆ ತರುವಂತಹ ವಿದ್ಯಾಭ್ಯಾಸ, ಮುಖ್ಯವಾಗಿ ತಾಂತ್ರಿಕ ವಿದ್ಯಾಭ್ಯಾಸ ಎಂದು ಸಾರಿದರು. ಭಾರತದ ಪುನರುತ್ಥಾನ ಕ್ಕಾಗಿ ಶ್ರಮಿಸಲು ತಾವು ಕಂಕಣಬದ್ಧರಾಗಿರುವುದಾಗಿಯೂ ಅದಕ್ಕೆ ಬೇಕಾದ ಆರ್ಥಿಕ ನೆರವನ್ನು ಪಡೆಯಲು ತಾವಿಲ್ಲಿಗೆ ಬಂದಿರುವುದಾಗಿಯೂ ಅವರು ಹೇಳಿದರು. ಅವರ ಈ ಭಾಷಣ ನಿಜಕ್ಕೂ ತುಂಬ ಪರಿಣಾಮಕಾರಿಯಾಗಿತ್ತು–ಎಷ್ಟರ ಮಟ್ಟಿಗೆಂದರೆ, ಭಾರತದಲ್ಲಿ ‘ವಿಧರ್ಮೀಯ’ ಕಾಲೇಜೊಂದನ್ನು ಪ್ರಾರಂಭಿಸಿ, ಅದನ್ನು ಕಟ್ಟುನಿಟ್ಟಾಗಿ ‘ವಿಧರ್ಮೀಯ’ ತತ್ತ್ವಗಳ ಮೇಲೆಯೇ ನಡೆಸಲು ನೆರವಾಗುವುದಕ್ಕಾಗಿ ಕೆಲವರು ವಂತಿಗೆ ಎತ್ತಿದರು! ಸಾಕಷ್ಟು ಹಣ ಸಂಗ್ರಹವಾಯಿತು ಕೂಡ! (ವಿಧರ್ಮೀಯ-\eng{-Heathen–}ಎಂಬುದು ಕ್ರೈಸ್ತರಲ್ಲದವರು, ಅಸಂಸ್ಕೃತ-ಅನಾಗರಿಕರು ಎಂಬರ್ಥದ, ತಿರಸ್ಕಾರ ಸೂಚಕವಾದ ಪದ.) ಈ ವಿಚಿತ್ರ ದೃಶ್ಯವನ್ನು ಬಣ್ಣಿಸಿ ತಮ್ಮ ತಾಯಿಗೆ ಬರೆದ ಪತ್ರದಲ್ಲಿ ಪ್ರೊ. ರೈಟರ ಪತ್ನಿ ಹೇಳುತ್ತಾರೆ: “ಇದನ್ನು ಕಂಡು ನಾನೊಂದು ಮೂಲೆಗೆ ಹೋಗಿ ಅಳು ಬರುವವರೆಗೂ ನಕ್ಕೆ!”

ರೈಟ್ ಕುಟುಂಬದವರೆಲ್ಲರ ಮೇಲೂ ಸ್ವಾಮೀಜಿ ಉಂಟುಮಾಡಿದ್ದ ಪ್ರಭಾವ ಎಷ್ಟು ಗಾಢವಾಗಿತ್ತೆಂದರೆ ಒಂದು ತಲೆಮಾರು ಕಳೆದ ಮೇಲೂ ಆ ಮನೆಮಂದಿ ‘ನಮ್ಮಸ್ವಾಮೀಜಿ’ ಎಂದು ಮಾತನಾಡುತ್ತಿದ್ದರು. ಸ್ವಾಮೀಜಿಯನ್ನು ಬಣ್ಣಿಸುತ್ತ ಶ್ರೀಮತಿ ರೈಟ್ ತಮ್ಮ ಪತ್ರದಲ್ಲಿ ಬರೆಯುತ್ತಾರೆ: “ಕಾಲಗತಿಯ ಪ್ರಕಾರ ಅವರ ವಯಸ್ಸು ಸುಮಾರು ಮೂವತ್ತು ವರ್ಷ, ಆದರೆ ಸಂಸ್ಕೃತಿಯ ದೃಷ್ಟಿಯಲ್ಲಿ ಅವರು ಯುಗಯುಗಗಳಷ್ಟು ಹಿರಿಯರು\eng{” (He is about thirty years old in time, ages in civilization’).} ಸ್ವಾಮೀಜಿ ಮನೆಯ ಮಕ್ಕಳನ್ನು ರಂಜಿಸುತ್ತ, ಅವರಲ್ಲೇ ಒಬ್ಬರಂತೆ ಆಟವಾಡುತ್ತಿದ್ದರು. ಒಂದು ಕಡ್ಡಿಯನ್ನು ಕೈಬೆರಳುಗಳಿಂದ ಚಮತ್ಕಾರ ಪೂರ್ಣವಾಗಿ ತಿರುಗಿಸಿ ತೋರಿಸಿ, ಮಕ್ಕಳು ತಮ್ಮನ್ನು ಸರಿಗಟ್ಟಲು ಅಸಮರ್ಥರಾದಾಗ ಬಾಲಕ ನಂತೆ ನಗುತ್ತಿದ್ದರು. ಇದನ್ನು ಕಂಡ ಹಿರಿಯರು, ಈ ಮಹಾಪುರುಷನ ಶಿಶುಸಹಜ ಸರಳತನವನ್ನು ಕಂಡು ಪರಮಾಶ್ಚರ್ಯಗೊಳ್ಳುತ್ತಿದ್ದರು.

ಇಷ್ಟು ಹೊತ್ತಿಗಾಗಲೇ ಬಾಸ್ಟನ್​ನ ಸುತ್ತಮುತ್ತಲ ಪ್ರದೇಶದಲ್ಲಿ ಸ್ವಾಮೀಜಿ ಹಲವಾರು ಸ್ನೇಹಿತರನ್ನು ಸಂಪಾದಿಸಿಕೊಂಡಿದ್ದರು. ತಮ್ಮಲ್ಲಿಗೆ ಭೇಟಿಕೊಡುವಂತೆ ಹಾಗೂ ಉಪನ್ಯಾಸ ನೀಡುವಂತೆ ಅನೇಕ ಆಹ್ವಾನಗಳು ಬರುತ್ತಿದ್ದುವು. ಇಂತಹ ಒಂದು ಆಹ್ವಾನವನ್ನು ಮನ್ನಿಸಿ ಅವರು ಆ್ಯನ್ನಿಸ್ಕ್ವಾಮ್​ನಿಂದ ೧೫ ಮೈಲಿ ದೂರದ ಸೇಲಮ್ ಎಂಬಲ್ಲಿಗೆ ಹೋದರು. ಇಲ್ಲಿನ ಶ್ರೀಮತಿ ಕೇಟ್ ಟನ್ನಾಟ್ ವುಡ್ಸ್ ಎಂಬಾಕೆ ಅವರನ್ನು ಬರಮಾಡಿಕೊಂಡಿದ್ದಳು. ಮಿಸ್ ಸ್ಯಾನ್ ಬಾರ್ನ್​ಳಂತೆ ಶ್ರೀಮತಿ ವುಡ್ಸ್ ಕೂಡ ಒಬ್ಬಳು ಪ್ರಭಾವಶಾಲೀ ಅಧ್ಯಾಪಕಳು ಹಾಗೂ ಲೇಖಕಿ. ಅಲ್ಲದೆ ಅಲ್ಲಿನ ಅತ್ಯುನ್ನತ ಮಟ್ಟದ ಸಂಘವಾದ \eng{Thought and Work Culb}ನ ಸಂಸ್ಥಾಪಕಿ. ಈಕೆಯ ಮನೆಯಲ್ಲಿ ಸ್ವಾಮೀಜಿ ಒಂದು ವಾರ ಇದ್ದರು. ಈ ಅವಧಿಯಲ್ಲಿ ಅವರು ಎರಡು ಭಾಷಣಗಳನ್ನು ಮಾಡಿದರು–ಒಮ್ಮೆ\eng{Thought and Work Club}ನ ಸದಸ್ಯರನ್ನು ಉದ್ದೇಶಿಸಿ ಹಾಗೂ ಮತ್ತೊಮ್ಮೆ ಅಲ್ಲಿನ ಚರ್ಚಿನಲ್ಲಿ ಸಾರ್ವಜನಿಕರನ್ನುದ್ದೇಶಿಸಿ.

ಸ್ವಾಮೀಜಿ ಇಲ್ಲಿ ಮಾಡಿದ ಮೊದಲ ಭಾಷಣದಲ್ಲಿ ಭಾರತದ ಬಗ್ಗೆ ಹಾಗೂ ಹಿಂದೂಧರ್ಮದ ಬಗ್ಗೆ ಜನರಲ್ಲಿ ವ್ಯಾಪಕವಾಗಿ ಪ್ರಚಲಿತವಿದ್ದ ತಪ್ಪುಕಲ್ಪನೆಗಳನ್ನು ಹೋಗಲಾಡಿಸುವ ಪ್ರಯತ್ನ ಮಾಡಿದರು. ಕ್ರೈಸ್ತ ಮಿಷನರಿಗಳು ಭಾರತೀಯರಿಗೆ ಧರ್ಮಬೋಧನೆ ಮಾಡುವ ಬದಲಿಗೆ ಜನಸಾಮಾನ್ಯರ ಸ್ಥಿತಿಗತಿಗಳನ್ನು ಸುಧಾರಿಸುವ ಕಾರ್ಯಕ್ರಮಗಳನ್ನು ಕೈಗೊಳ್ಳಬೇಕೆಂದು ಅವರು ಹೇಳಿದರು. ಸ್ವಾಮೀಜಿಯ ಮಾತುಗಳು ಬಹಳಷ್ಟು ಜನರ ಕಣ್ದೆರೆಸಿತಾದರೂ, ಅಲ್ಲಿದ್ದ ಕೆಲವು ಪಾದ್ರಿಗಳಿಗೆ ಅದು ಸಹ್ಯವಾಗಲಿಲ್ಲ. ಇವರು ಮತ್ತೆ ಮತ್ತೆ ಪ್ರಶ್ನೆಗಳನ್ನು ಹಾಕುತ್ತ ಭಾಷಣಕ್ಕೆ ತಡೆಯುಂಟುಮಾಡುತ್ತಿದ್ದರು. ಈ ಪಾದ್ರಿಗಳು ಅಷ್ಟು ಉಗ್ರವಾಗಿ ಹಾಗೂ ಅಸಭ್ಯವಾಗಿ ವರ್ತಿಸಿ ದರೂ ಸ್ವಾಮೀಜಿ ಅತ್ಯಂತ ಸೌಮ್ಯವಾಗಿ ವಿನಯದಿಂದಲೇ ಪ್ರಶ್ನೆಗಳನ್ನೆಲ್ಲ ಉತ್ತರಿಸಿದರು. ಇದು ಕ್ರೈಸ್ತ ಮಿಷನರಿಗಳೊಂದಿಗೆ ಅವರ ಪ್ರಥಮ ಸೆಣಸಾಟ. ಇದೊಂದು ಸಣ್ಣ ಘಟನೆಯಾದರೂ ಮುಂದೆ ತಮಗೆ ಒದಗಬಹುದಾದ ಅತ್ಯಂತ ಭಯಂಕರ ಪ್ರತಿಭಟನೆಯ ಮುನ್ಸೂಚಿ ಇದೆ ಎಂಬುದು ಅವರಿಗೆ ಅರಿವಾಯಿತು.

ಈ ಎರಡು ಭಾಷಣಗಳಲ್ಲದೆ, ಒಮ್ಮೆ ಅವರು ಶ್ರೀಮತಿ ವುಡ್ಸ್​ಳ ಮನೆಯಲ್ಲಿ ಆ ಊರಿನ ಬಾಲಕ ಬಾಲಕಿಯರನ್ನುದ್ದೇಶಿಸಿ ಮಾತನಾಡಿದರು. ಭಾರತೀಯ ಮಕ್ಕಳ ಶಾಲೆಗಳು, ಆಟಗಳು ಹಾಗೂ ರೀತಿನೀತಿಗಳ ಬಗ್ಗೆ ತುಂಬ ರಸವತ್ತಾಗಿ ಮಾತನಾಡಿ ಮಕ್ಕಳನ್ನು ರಂಜಿಸಿದರು.

ಸೆಪ್ಟೆಂಬರ್ ೪ರಂದು ಸ್ವಾಮೀಜಿ ಸೇಲಮನ್ನು ಬಿಟ್ಟು ನ್ಯೂಯಾರ್ಕಿನ ಸರಟೋಗಕ್ಕೆ ಬಂದರು. ಇಲ್ಲಿ ಅವರು ಅಮೆರಿಕನ್ ಸೋಶಿಯಲ್ ಸೈನ್ಸ್ ಅಸೋಸಿಯೇಶನ್ನಿನ ಸಭೆಯನ್ನು ಉದ್ದೇಶಿಸಿ ಭಾಷಣ ಮಾಡುವ ಕಾರ್ಯಕ್ರಮವಿತ್ತು. ಫ್ರಾಂಕ್​ಲಿನ್ ಸ್ಯಾನ್​ಬಾರ್ನ್ ಈ ಸಂಘದ ಕಾರ್ಯ ದರ್ಶಿ. ಇಲ್ಲಿನ ಗಣ್ಯವ್ಯಕ್ತಿಗಳಿಂದ ತುಂಬಿದ ಸಭೆಯಲ್ಲಿ ಸ್ವಾಮೀಜಿ ಮೂರು ಭಾಷಣ ಮಾಡಿ ದರು; ಹಾಗೂ ಒಬ್ಬರ ಮನೆಯಲ್ಲಿ ಎರಡು ಭಾಷಣ ಮಾಡಿದರು. ಇಂತಹ ಉನ್ನತ ಮಟ್ಟದ ಸಂಘವೊಂದರಲ್ಲಿ ಒಬ್ಬ ಅನಾಮಧೇಯ ಸಂನ್ಯಾಸಿಯಾದ ವಿವೇಕಾನಂದರಿಗೆ ಭಾಷಣ ಮಾಡುವ ಅವಕಾಶವನ್ನು ಕಲ್ಪಿಸಿಕೊಟ್ಟದ್ದೇ ಅವರಿಗೆ ಸಂದ ಒಂದು ವಿಶೇಷ ಗೌರವ. ಆದರೆ ಆ ಜನ ಅವರನ್ನು ಅಷ್ಟರಮಟ್ಟಿಗೆ ಆದರಿಸಲು ಅವರ ಅಪಾರ ಬುದ್ಧಿಮತ್ತೆ, ಭವ್ಯ ವಿಚಾರಧಾರೆ ಹಾಗೂ ನಿಷ್ಕಳಂಕ ಚಾರಿತ್ರ್ಯ–ಇವುಗಳೇ ಕಾರಣ. ನಿಜಕ್ಕೂ ಇವುಗಳೇ ಅವರ ಪರಿಚಯ ಪತ್ರ. ಅಥವಾ ಪ್ರೊ. ರೈಟ್ ಹೇಳಿದಂತೆ ಅವರ ಈ ಪರಿಚಯ ಪತ್ರವು ಬೇಕಾದ್ದಕ್ಕಿಂತ ಹೆಚ್ಚೇ ಆಯಿತು.

ಅಲ್ಲಿ ಅವರಿಗೆ ಕೊಟ್ಟಿದ್ದ ಎರಡು ಭಾಷಣಗಳ ವಿಷಯಗಳು ಸಂನ್ಯಾಸಿಗಳ ಪಾಲಿಗೆ ತುಂಬ ಹೊಸತು–“ಭಾರತದಲ್ಲಿ ಮುಸಲ್ಮಾನರ ಆಳ್ವಿಕೆ” ಹಾಗೂ “ಭಾರತದಲ್ಲಿ ಬೆಳ್ಳಿಯ ಉಪ ಯೋಗ.” (ಆ ದಿನಗಳಲ್ಲಿ ಅಮೆರಿಕದ ರಾಜಕಾರಣದಲ್ಲಿ ಬೆಳ್ಳಿನಾಣ್ಯದ ಚಲಾವಣೆಯು ಅತ್ಯಂತ ವಿವಾದಾತ್ಮಕ ವಿಷಯವಾಗಿತ್ತು.) ನಿಜಕ್ಕೂ ಅಂತಹ ಅತ್ಯುನ್ನತ ವಿದ್ಯಾವಂತರು ಸೇರಿದ್ದ ಸಭೆಯಲ್ಲಿ ಯಾವ ವಿಷಯವೆಂದರೆ ಆ ವಿಷಯದ ಮೇಲೆ ಮಾತನಾಡಬೇಕಾದರೆ ಸಾಕಷ್ಟು ಸಮರ್ಥರೇ ಆಗಿರಬೇಕಾಗುತ್ತದೆ. ಸ್ವಾಮೀಜಿ ಈ ಸವಾಲನ್ನು ಲೀಲಾಜಾಲವಾಗಿ ಎದುರಿಸಿ, ಸಭೆಯ ಮೆಚ್ಚುಗೆಗೆ ಪಾತ್ರರಾದರು.

ಇಷ್ಟು ಹೊತ್ತಿಗೆ, ಪ್ರೊ. ರೈಟರು ವಿಶ್ವಧರ್ಮ ಸಮ್ಮೇಳನದ ಅಧಿಕಾರಿಗಳಿಗೆ ಬರೆದ ಪತ್ರಕ್ಕೆ ಉತ್ತರ ಬಂದಿತು. ಸ್ವಾಮೀಜಿಯನ್ನು ಒಬ್ಬ ಪ್ರತಿನಿಧಿಯಾಗಿ ಸ್ವೀಕರಿಸಲಾಗಿತ್ತು. ತಮ್ಮ ದಾರಿಗಡ್ಡ ವಾಗಿ ಬಂದ ಅಡೆತಡೆಗಳನ್ನೆಲ್ಲ ಭಗವಂತ ನಿವಾರಿಸಿದ ಪರಿಯನ್ನು ಕಂಡು ಸ್ವಾಮೀಜಿ ಆನಂದ ಭರಿತರಾದರು. ಇದುವರೆಗಿನ ಸುಮಾರು ಮೂರು ವಾರಗಳ ಅವಧಿಯಲ್ಲಿ ಗಳಿಸಿದ್ದ ಅತ್ಯಂತ ಉಪಯುಕ್ತ ಅನುಭವವನ್ನು ಹೊತ್ತು, ನವೋತ್ಸಾಹದಿಂದ ಕೂಡಿ ಶಿಕಾಗೋ ನಗರದೆಡೆಗೆ ಹೊರಟರು.

ಆದರೆ ಒಂದೇ ಒಂದು ತೊಂದರೆಯಿತ್ತು–ಶಿಕಾಗೋದಲ್ಲಿ ತಾವು ಹೋಗಬೇಕಾದ ಸ್ಥಳ ಅವರಿಗೆ ತಿಳಿದಿರಲಿಲ್ಲ. ಯಾರೋ ಅವರಿಗೆ ವಿಳಾಸದ ವಿವರಗಳನ್ನು ಒಂದು ಚೀಟಿಯಲ್ಲಿ ಬರೆದು ಕೊಟ್ಟಿದ್ದರು. ಜೊತೆಗೆ ಟ್ರೈನಿನಲ್ಲಿ ಅವರ ಸಹಪ್ರಯಾಣಿಕನಾಗಿದ್ದ ಒಬ್ಬ ವರ್ತಕ, ಶಿಕಾಗೋದಲ್ಲಿ ಅವರನ್ನು ಸರಿಯಾದ ಸ್ಥಳಕ್ಕೆ ತಲುಪಿಸುವುದಾಗಿ ಮಾತುಕೊಟ್ಟಿದ್ದ. ಆದರೆ ಶಿಕಾಗೋ ರೈಲು ನಿಲ್ದಾಣವನ್ನು ತಲುಪುತ್ತಿದ್ದಂತೆಯೇ ಅವನು ತನ್ನ ಗಡಿಬಿಡಿಯಲ್ಲಿ ಸ್ವಾಮೀಜಿಗೆ ಅವರ ದಾರಿ ಯನ್ನು ತೋರಿಸಿಕೊಡುವುದನ್ನು ಮರೆತು ಹೊರಟೇ ಹೋದ. ಈಗ ಸ್ವಾಮೀಜಿ, ಸಮ್ಮೇಳನದ ‘ಸಾಧಾರಣ ಸಮಿತಿ’ಯ ಅಧ್ಯಕ್ಷರಾದ ಡಾ. ಜಾನ್ ಹೆನ್ರಿಬರೋಸ್​ರವರ ಕಛೇರಿಗೆ ಹೋಗ ಬೇಕಿತ್ತು. ಹೋಗಲಿ, ತಮ್ಮಲ್ಲಿರುವ ವಿಳಾಸದ ಸಹಾಯದಿಂದ ತಾವೇ ಹುಡುಕಿಕೊಂಡು ಹೋಗೋಣವೆಂದು ಜೇಬಿಗೆ ಕೈಹಾಕಿ ನೋಡುತ್ತಾರೆ–ಆ ಚೀಟಿ ಎಲ್ಲಿಯೋ ಕಳೆದು ಹೋಗಿ ಬಿಟ್ಟಿದೆ! ಆದರೇನಂತೆ, ಯಾರನ್ನಾದರೂ ಕೇಳಿಕೊಂಡು ಹೋಗೋಣ ಎಂದು ದಾರಿಹೋಕರನ್ನು ವಿಚಾರಿಸಿದರೆ ಒಬ್ಬರಿಗಾದರೂ ಅವರ ಮಾತು ಅರ್ಥವಾಗಬೇಕಲ್ಲ! ಏಕೆಂದರೆ ಶಿಕಾಗೋದ ಆ ಭಾಗದಲ್ಲಿ ವಾಸಿಸುತ್ತಿದ್ದವರೆಲ್ಲ ಜರ್ಮನರೇ. ಅವರಿಗೆ ಇಂಗ್ಲಿಷ್ ಬಾರದು, ಇವರಿಗೆ ಜರ್ಮನ್ ಬಾರದು. ಜೊತೆಗೆ ಇನ್ನು ಕೆಲವರು ಸ್ವಾಮೀಜಿಯನ್ನು ಒಬ್ಬ ನೀಗ್ರೋ ಎಂದು ಭಾವಿಸಿ ತಿರಸ್ಕಾರ ದಿಂದ ಕಂಡರು. ಆಗಲೇ ಕತ್ತಲಾಗುತ್ತ ಬಂದಿತ್ತು. ಹೋಟೆಲಿಗಾದರೂ ಹೋಗೋಣವೆಂದು ದಾರಿ ತೋರಿಸುವಂತೆ ಕೇಳಿದಾಗ ಆ ಜನರಿಗೆ ಅದೂ ಅರ್ಥವಾಗಲಿಲ್ಲ! ಸ್ವಾಮೀಜಿಗೆ ಇನ್ನೇನು ಮಾಡಲೂ ತೋಚಲಿಲ್ಲ. ಅವರನ್ನು ಭಗವಂತ ತಮಾಷೆ ಮಾಡಿ ನೋಡುತ್ತಿದ್ದಾನೋ ಎಂಬಂತಿತ್ತು.

ಕಡೆಗೆ ಸ್ವಾಮೀಜಿ ಅಲ್ಲೇ ಎಲ್ಲಾದರೂ ರಾತ್ರಿಯನ್ನು ಕಳೆಯಲು ನಿಶ್ಚಯಿಸಿದರು. ಅಲ್ಲೊಂದು ಖಾಲಿ ಗುಡ್ಸ್​ವ್ಯಾಗನ್ ನಿಂತಿದ್ದುದು ಕಂಡಿತು. ‘ಗಗನವೇ ಮನೆ ಹಸುರೆ ಹಾಸುಗೆ’ ಎಂಬ ತಮ್ಮ ಪರಿವ್ರಾಜಕ ದಿನಗಳನ್ನು ನೆನಪಿಸಿಕೊಂಡರು. ಸೀದಾ ಹೋಗಿ ಗೂಡ್ಸ್ ವ್ಯಾಗನ್ನಿನಲ್ಲಿ ಉರುಳಿ ಕೊಂಡರು.\footnote{* ಸ್ವಾಮೀಜಿ ಒಂದು ಪೆಟ್ಟಿಗೆಯಲ್ಲಿ ಮಲಗಿ ಆ ರಾತ್ರಿಯನ್ನು ಕಳೆದರು ಎಂಬ ನಂಬಿಕೆ ಪ್ರಚಲಿತವಾಗಿದೆ. ಆದರೆ ಸ್ವಾಮಿ ವಿವೇಕಾನಂದರ ಜೀವನದ ಬಗ್ಗೆ ನಿರಂತರವಾಗಿ ಹೊಸ ಶೋಧನೆಗಳನ್ನು ಮಾಡುತ್ತಿರುವ ಶ್ರೀಮತಿ ಮೇರಿ ಲೂಯಿಸ್ ಬರ್ಕ್​ರವರ ಪ್ರಕಾರ ಅದೊಂದು \eng{Box} ಅಲ್ಲ, \eng{Box-car (}ಗೂಡ್ಸ್ ವ್ಯಾಗನ್​). ಅಲ್ಲದೆ, ಅಮೆರಿಕದಲ್ಲಿ ನಿರಾಶ್ರಿತರಿಗೂ ನಿರ್ಗತಿಕರಿಗೂ ಬಹಳ ಹಿಂದಿನಿಂದಲೂ ಈ ಗೂಡ್ಸ್ ವ್ಯಾಗನ್ ರಾತ್ರಿಯ ತಂಗು ದಾಣವಾಗಿ ಸೇವೆ ಸಲ್ಲಿಸಿದೆ.} ‘ಭಗವಂತ! ಇಲ್ಲಿಯವರೆಗೂ ದಾರಿ ತೋರಿಸಿದವನು ನೀನೇ; ಇನ್ನು ಮುಂದೆ ತೋರಬೇಕಾದವನೂ ನೀನೇ’ ಎಂದು ಪ್ರಾರ್ಥಿಸಿಕೊಂಡು, ಬಳಿಕ ನಿಶ್ಚಿಂತರಾಗಿ ನಿದ್ರಿಸಿದರು. ಇನ್ನು ಎರಡೇ ದಿನಗಳಲ್ಲಿ ವಿಶ್ವವೇದಿಕೆಯ ಮೇಲೆ ನಿಂತು, ಚರಿತ್ರಪ್ರಸಿದ್ಧ ಭಾಷಣವನ್ನು ಮಾಡಿ ಇಡೀ ಅಮೆರಿಕವನ್ನೇ ನಡುಗಿಸಲಿದ್ದಾರೆ; ಆದರೆ ಈಗ ಅವರನ್ನು ವಿಧಿ ಯಾವ ಅವಸ್ಥೆಯಲ್ಲಿಟ್ಟಿದೆ! – ಊರ ಹೊರಗಿನ ಅಸ್ಪೃಶ್ಯನಂತೆ ಈ ಸಾಮಾನುಗಳ ಬೋಗಿಯಲ್ಲಿ ಮಲಗಿದ್ದಾರೆ! ಆದರೆ ಅವರಿಗೆ ಮಾತ್ರ ಇದರ ಪರಿವೆಯೇ ಇಲ್ಲ. ಅವರಿಗೆ ಇಂತಹ ಪರಿಸ್ಥಿತಿಯೇನು ಹೊಸದೆ? ಬಟ್ಟ ಬಯಲಿನಲ್ಲಿ ಮಲಗಿ ಅಭ್ಯಾಸವಿರುವವರಿಗೆ ಗೂಡ್ಸ್ ವ್ಯಾಗನ್​ನಲ್ಲಿ ಮಲಗಲು ಕಷ್ಟವೆ? ಆದ್ದರಿಂದ ಅಲ್ಲೇ ಸುಖವಾಗಿ ನಿದ್ರಿಸಿದರು!

ಬೆಳಗಾಗುತ್ತಿದ್ದಂತೆ ಸ್ವಾಮೀಜಿ ಅಲ್ಲಿಂದ ಎದ್ದು ಹೊರಬಂದರು. ಭಗವಂತನನ್ನು ಸ್ಮರಿಸುತ್ತ ಒಂದು ರಸ್ತೆ ಹಿಡಿದು ಹೊರಟರು. ಯಾವ ರಸ್ತೆ ಎಲ್ಲಿಗೆ ಹೋಗುತ್ತದೆ ಎಂಬುದೂ ಗೊತ್ತಿಲ್ಲ. ಕೇಳಿದರೆ ಹೇಳುವವರು ಇಲ್ಲ. ಅವರೇ ಹೇಳುವಂತೆ, ‘ಬಾಯಾರಿದ ಪ್ರಾಣಿ ನೀರಿನ ವಾಸನೆ ಹಿಡಿದು ಹೋಗುವ ಹಾಗೆ’ ನಡೆದು ಲೇಕ್ ಶೋರ್ ಡ್ರೈವ್ ಎಂಬ ರಸ್ತೆಗೆ ಬಂದರು. ಇದು ನಗರದ ಅತ್ಯಂತ ಪ್ರತಿಷ್ಠಿತರು, ಕೋಟ್ಯಧೀಶರು ವಾಸಿಸುತ್ತಿದ್ದ ಸುಂದರ ರಸ್ತೆ, ಇಕ್ಕೆಲಗಳಲ್ಲೂ ಅರಮನೆಯಂತಹ ಭವ್ಯ ಮಹಲುಗಳು. ಹಿಂದಿನ ಮಧ್ಯಾಹ್ನದಿಂದಲೂ ಉಪವಾಸವಿದ್ದುದರಿಂದ ಹೊಟ್ಟೆ ಭಯಂಕರವಾಗಿ ಹಸಿದಿತ್ತು. ಸರಿ, ಅಭ್ಯಾಸ ಬಲದಿಂದ ಹಾಗೂ ಸಂನ್ಯಾಸಧರ್ಮಕ್ಕನು ಸಾರವಾಗಿ ಮನೆಯಿಂದ ಮನೆಗೆ ಭಿಕ್ಷೆ ಬೇಡುತ್ತ ನಡೆದರು. ಜೊತೆಗೆ ಸಮ್ಮೇಳನದ ಸಮಿತಿಯ ಕಛೇರಿಗೆ ದಾರಿ ತೋರಿಸುವಂತೆಯೂ ಕೇಳಿದರು. ಆದರೆ ಅಮೆರಿಕ ಭಾರತವಲ್ಲವೆಂಬುದು ಮತ್ತೆ ಅನುಭವಕ್ಕೆ ಬಂತು. (ಉತ್ತರ ಭಾರತದ ಕೆಲವು ‘ಧಾರ್ಮಿಕ’ ಮಹಿಳೆಯರು ಒಂದು ಚಪಾತಿ ಯನ್ನು ನಾಲ್ಕು ಚೂರು ಮಾಡಿ ನಾಲ್ವರು ಸಾಧುಗಳಿಗೆ ಹಾಕಿ ಪುಣ್ಯ ಕಟ್ಟಿಕೊಳ್ಳುತ್ತಿದ್ದರಂತೆ!) ಯಾರೂ ಅವರಿಗೆ ಸಹಾನುಭೂತಿ ತೋರಲಿಲ್ಲ. ಸ್ವಾಮೀಜಿಯ ವಿಚಿತ್ರ ಉಡುಗೆಯನ್ನೂ ದಣಿದ ಮುಖವನ್ನೂ ಕಂಡು ಜನ ಸಂಶಯಪಟ್ಟರು. ಕೆಲವು ಮನೆಗಳಲ್ಲಿ ಸೇವಕರೇ ಅವರನ್ನು ಅವಮಾನಿಸಿ ಓಡಿಸಿದರು. ಇನ್ನು ಕೆಲವರು ಅವರು ಬರುತ್ತಿದ್ದಂತೆ ಮುಖದೆದುರಿಗೇ ದಢಾರನೆ ಬಾಗಿಲು ಹಾಕಿಕೊಂಡರು. ಸ್ವಾಮೀಜಿಗೆ ತುಂಬ ನಿರಾಶೆಯಾಯಿತು. ಯಾರಿಗಾದರೂ ಪೋನ್ ಮಾಡೋಣವೆಂದರೆ ಯಾರ ಫೋನ್ ನಂಬರೂ ಗೊತ್ತಿಲ್ಲ. ಹೀಗೇ ಕಾಲೆಳೆದುಕೊಂಡು ಸುಮಾರು ಎರಡೂವರೆ ಮೈಲಿಯವರೆಗೆ ನಡೆದರು. ಒಂದುಕಡೆ ಹಸಿವು, ಇನ್ನೊಂದು ಕಡೆ ದಣಿವು, ಮತ್ತೊಂದುಕಡೆ ನಿರಾಶೆ! ಕಡೆಗೆ ಆಯಾಸ ತಾಳಲಾರದೆ ರಸ್ತೆಯ ಒಂದು ಪಕ್ಕದಲ್ಲಿ ಕುಳಿತು ಬಿಟ್ಟರು. ಅವರಿಂದ ಇನ್ನೇನು ಮಾಡಲೂ ಸಾಧ್ಯವಿಲ್ಲ. ಇನ್ನೇನು ಆಗಬೇಕಾದರೂ ಭಗವಂತ ನಿಂದಲೇ ಆಗಬೇಕಾಗಿದೆ. ಅವನಿಚ್ಛೆ ಹೇಗಿದೆಯೋ ಹಾಗಾಗಲಿ ಎಂದು ಶಾಂತವಾಗಿ ಕುಳಿತರು.

ಈಗ ಭಗವಂತನ ಇಚ್ಛೆ ಕೆಲಸ ಮಾಡಲು ಪ್ರಾರಂಭಿಸಿತು. ಅವರು ಕುಳಿತ ಸ್ಥಳದ ಎದುರಿನಲ್ಲಿದ್ದ ಒಂದು ಭವ್ಯವಾದ ಬಂಗಲೆಯ ಬಾಗಿಲು ತೆರೆದುಕೊಂಡಿತು. ರಾಣಿಯಂತೆ ಕಾಣುತ್ತಿದ್ದ ಒಬ್ಬಳು ಮಹಿಳೆ ಹೊರಬಂದಳು. ನೇರವಾಗಿ ಸ್ವಾಮೀಜಿಯ ಬಳಿಗೆ ಬಂದು, ಮೃದು ಮಧುರ ದನಿಯಲ್ಲಿ ಕೇಳಿದಳು, “ಮಹಾಶಯರೇ, ನೀವು ಧರ್ಮಸಮ್ಮೇಳನಕ್ಕೆ ಪ್ರತಿನಿಧಿಯಾಗಿ ಬಂದಿರುವವರೇ?” ಸ್ವಾಮೀಜಿ ಹೌದೆನ್ನುತ್ತ ತಮ್ಮ ಪರಿಸ್ಥಿತಿಯನ್ನು ಹೇಳಿಕೊಂಡರು. ತಕ್ಷಣವೇ ಆ ದಯಾವಂತ ಮಹಿಳೆ ಅವರನ್ನು ತನ್ನ ಮನೆಯೊಳಗೆ ಬರಮಾಡಿಕೊಂಡಳು. ಬಳಿಕ ಅವರನ್ನು ಒಂದು ಕೋಣೆಗೆ ಕರೆದೊಯ್ದು ಅವರಿಗೆ ಬೇಕಾದ ಅನುಕೂಲತೆಗಳನ್ನೆಲ್ಲ ಮಾಡಿಕೊಡುವಂತೆ ಸೇವಕರಿಗೆ ಸೂಚಿಸಿದಳು. ಮತ್ತು ಅವರ ಉಪಾಹಾರವಾದ ಬಳಿಕ ತಾನೇ ಅವರನ್ನು ಸಮ್ಮೇಳನದ ಕಛೇರಿಗೆ ಕರೆದುಕೊಂಡು ಹೋಗುವುದಾಗಿ ಭರವಸೆ ಕೊಟ್ಟಳು. ಕೃತಜ್ಞತೆಯಿಂದ ಸ್ವಾಮೀಜಿಯ ಹೃದಯ ತುಂಬಿಬಂದಿತು.

ಹೀಗೆ ಅವರನ್ನು ಅಸಹಾಯಕ ಪರಿಸ್ಥಿತಿಯಿಂದ ಪಾರುಮಾಡಿದ ಆ ಮಹಿಳೆ ಶ್ರೀಮತಿ ಬೆಲ್ ಹೇಲ್. ಈಕೆ ಮುಂದೆ ಎಂದೆಂದಿಗೂ ಅವರನ್ನು ಅತ್ಯಂತ ವಿಶ್ವಾಸದಿಂದ ನೋಡಿಕೊಂಡಳು. ಈಕೆಯ ಪತಿ ಶ್ರೀ ಜಾರ್ಜ್ ಹೇಲ್, ಅವರ ಅತ್ಯಂತ ಗೌರವಾನ್ವಿತ ಸ್ನೇಹಿತರಾದರು. ಈ ದಂಪತಿ ಗಳ ಪುತ್ರಿಯರಾದ ಮೇರಿ ಹಾಗೂ ಹ್ಯಾರಿಯೆಟ್, ಮತ್ತು ಜಾರ್ಜ್​ರವರ ಸೋದರಸೊಸೆಯರಾದ ಇಸಾಬೆಲ್ ಮೆಕ್​ಕಿಂಡ್ಲಿ ಹಾಗೂ ಹ್ಯಾರಿಯೆಟ್ ಮೆಕ್​ಕಿಂಡ್ಲಿ–ಇವರು ನಾಲ್ವರನ್ನು (“ಹೇಲ್ ಸೋದರಿಯರು”) ಸ್ವಾಮೀಜಿ ಸ್ವಂತ ಸೋದರಿಯರಂತೆ ಕಂಡರು. ಶಿಕಾಗೋದ ಡಿಯರ್​ಬಾರ್ನ್ ಅವೆನ್ಯೂನಲ್ಲಿದ್ದ ಇವರ ಮನೆ ಮುಂದೆ ಸ್ವಾಮೀಜಿಯ ‘ಕೇಂದ್ರಸ್ಥಾನ’ವಾಗಿ ಪರಿಣಮಿಸಿತು.

ಈಗ ಸ್ವಾಮೀಜಿಯನ್ನು ಒಂದು ನವಸ್ಫೂರ್ತಿ ಆವರಿಸಿತು. ಭಗವಂತ ಸದಾ ತಮ್ಮೊಂದಿಗೆ ಇದ್ದು ತಮ್ಮನ್ನು ಸರ್ವವಿಧದಿಂದಲೂ ನೋಡಿಕೊಳ್ಳುತ್ತಿದ್ದಾನೆ ಎಂದು ಅವರಿಗೆ ಮತ್ತೆ ಖಚಿತ ವಾಯಿತು. ಮುಂದಾಗಬಹುದಾದ್ದನ್ನು ಅವರು ಉತ್ಸಾಹ-ಕಾತರಗಳಿಂದ ನಿರೀಕ್ಷಿಸತೊಡಗಿದರು. ಶ್ರೀಮತಿ ಬೆಲ್ ಹೇಲ್​ಳೊಂದಿಗೆ ಸಮ್ಮೇಳನದ ಸಮಿತಿಯವರನ್ನು ಕಂಡು ತಮ್ಮ ಪರಿಚಯ ಪತ್ರವನ್ನು ಕೊಟ್ಟರು. ಸಮಿತಿಯವರು ಸ್ವಾಮೀಜಿಯನ್ನು ಆದರಿಂದ ಸ್ವಾಗತಿಸಿ, ಪ್ರತಿನಿಧಿಯಾಗಿ ಸ್ವೀಕರಿಸಿದರು. ಬಳಿಕ ಅವರಿಗೆ ಪೌರ್ವಾತ್ಯ ಪ್ರತಿನಿಧಿಗಳಿಗಾಗಿ ಮೀಸಲಾದ ವಸತಿಗೃಹದಲ್ಲಿ ಅವಕಾಶ ಮಾಡಿಕೊಡಲಾಯಿತು. (ಇದಾದ ಕೆಲದಿನಗಳಲ್ಲೇ ಅವರಿಗೆ ಶ್ರೀಮತಿ ಮತ್ತು ಶ್ರೀ ಜಾನ್ ಬಿ. ಲಿಯಾನ್ ಎಂಬವರ ಮನೆಯಲ್ಲಿ ವ್ಯವಸ್ಥೆ ಮಾಡಲಾಯಿತು.)

ಈಗ ಒಂದೊಂದು ನಿಮಿಷವೂ ಉರುಳಿದಂತೆ, ಸ್ವಾಮೀಜಿಗೆ ತಮ್ಮ ಪರೀಕ್ಷಾ ಕಾಲ ಸನ್ನಿಹಿತ ವಾಗುತ್ತಿರುವಂತೆ ಭಾಸವಾಯಿತು. ಸರ್ವಧರ್ಮ ಸಮ್ಮೇಳನವು ತಮ್ಮ ಪಾಲಿನ ಸತ್ವಪರೀಕ್ಷೆ ಯಾಗಲಿದೆ, ಹಾಗೂ ಇದರಲ್ಲಿ ಯಶಸ್ವಿಯಾದರೆ ತಮ್ಮ ಮುಂದಿನ ಕಾರ್ಯ ಸುಗಮವಾಗುತ್ತದೆ ಎಂದು ಅವರಿಗನ್ನಿಸಿತು. ಅವರು ಮತ್ತೆಮತ್ತೆ ಧ್ಯಾನಮಗ್ನರಾದರು. ತಾವು ಕೇವಲ ಭಗವಂತನ ಒಂದು ಉಪಕರಣವಾಗುವಂತಾಗಲಿ, ಶ್ರೀರಾಮಕೃಷ್ಣರ ಸಂದೇಶವನ್ನು ಅತ್ಯಂತ ಸಮರ್ಪಕವಾಗಿ ಪ್ರಸಾರ ಮಾಡುವಂತಾಗಲಿ, ತಾವು ಹಿಂದೂ ಧರ್ಮದ ಅತ್ಯಂತ ಸಮರ್ಥ ಪ್ರತಿನಿಧಿಯಾಗು ವಂತಾಗಲಿ ಎಂದು ಮೌನವಾಗಿ ಪ್ರಾರ್ಥಿಸಿಕೊಂಡರು. ಈಗಾಗಲೇ ಅವರಿಗೆ ಅಲ್ಲಿನ ಹಲವಾರು ಗಣ್ಯವ್ಯಕ್ತಿಗಳ ಪರಿಚಯವಾಗಿ ಅವರುಗಳ ನಡುವೆಯೇ ಓಡಾಡಿಕೊಂಡಿದ್ದರೂ ಅವರ ಮನಸ್ಸು ಮಾತ್ರ ಅನ್ಯಚಿಂತೆಯಲ್ಲಿ ಮುಳುಗಿತ್ತು. ಶ್ರೀರಾಮಕೃಷ್ಣರು ತಮಗೆ ವಹಿಸಿದ ಕಾರ್ಯವನ್ನು ತಾವು ಕೈಗೊಂಡು ಯಶಸ್ವಿಗೊಳಿಸಬೇಕಾಗಿದೆ ಎಂಬ ಒಂದೇ ಆಲೋಚನೆ ಅವರ ಮನದಲ್ಲಿ ತುಂಬಿತ್ತು.

