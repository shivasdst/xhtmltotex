
\chapter{ಪ್ರಭುರಾಷ್ಟ್ರದ ಮೇಲೆ ಪ್ರಭುತ್ವ}

\noindent

೧೮೯೬ರ ಬೇಸಿಗೆಯಲ್ಲೇ ಸ್ವಾಮೀಜಿಗೆ ಇಂಗ್ಲೆಂಡಿನಿಂದ ಪತ್ರಗಳ ಸುರಿಮಳೆಯಾಗಲು ಪ್ರಾರಂಭವಾಗಿತ್ತು. ಆ ಪತ್ರಗಳಲ್ಲಿ ಅವರ ಶಿಷ್ಯರು ಸಲ್ಲಿಸಿದ್ದ ನಿವೇದನೆ ಏನೆಂದರೆ, ಸಾಧ್ಯ ವಾದಷ್ಟು ಬೇಗನೆ ಅವರು ಇಂಗ್ಲೆಂಡಿಗೆ ಹಿಂದಿರುಗಿ ಬಂದು ಅವರು ಅದಾಗಲೇ ಪ್ರಾರಂಭಿಸಿದ್ದ ಕಾರ್ಯಗಳನ್ನು ಒಂದು ವ್ಯವಸ್ಥೆಗೆ ತರಬೇಕು, ಎಂದು. ಅದು ಸಾಧ್ಯವಾದಷ್ಟು ಬೇಗ ಆಗಬೇಕಾದ ಕೆಲಸವೇ ಎಂದು ಸ್ವಾಮೀಜಿಗೂ ಅನ್ನಿಸಿದ್ದರಿಂದ ಏಪ್ರಿಲ್ ೧೫ರಂದೂ ನ್ಯೂಯಾರ್ಕಿನಿಂದ ಲಿವರ್​ಪೂಲಿಗೆ ಪ್ರಯಾಣ ಬೆಳಸಿದರು. ಈ ಸಲ ಅವರೊಂದಿಗೆ ಅವರ ನೆಚ್ಚಿನ ಶೀಘ್ರ ಲಿಪಿಕಾರ ಗುಡ್​ವಿನ್ ಕೂಡ ಇದ್ದ.

ಹಡಗಿನ ಪ್ರಯಾಣ ಈ ಬಾರಿ ಹೆಚ್ಚು ಸುಖಕರವಾಗಿತ್ತು. ಹಿಂದಿನ ಸಲದಂತೆ ಈ ಸಲ ಅಸ್ವಸ್ಥತೆಯುಂಟಾಗದಿರಲು ಸ್ವಾಮೀಜಿ ಎಚ್ಚರ ವಹಿಸಿದ್ದರಿಂದ ಯಾವ ತೊಂದರೆಯೂ ಉಂಟಾಗಲಿಲ್ಲ. ಆದರೆ ಪ್ರಯಾಣ ಬಹಳ ನೀರಸವಾಗಿತ್ತೆಂದು ಅವರೇ ಪತ್ರವೊಂದರಲ್ಲಿ ತಿಳಿಸಿ ದರು. ಅವರ ಪಾಲಿಗೇನೋ ಪ್ರಯಾಣ ನೀರಸವಾಗಿದ್ದಿರಬಹುದು; ಆದರೆ ಹಡಗಿನಲ್ಲಿ ಇತರ ಪ್ರಯಾಣಿಕರ ಕಣ್ಣಿಗೆ ಅವರೊಂದು ಅದ್ಭುತ-ಕೌತುಕದ ವ್ಯಕ್ತಿಯಾಗಿ ತೋರಿರಲೇಬೇಕು. ಏಕೆಂ ದರೆ ಸ್ವಾಮೀಜಿ ಎಲ್ಲೇ ಇರಲಿ, ಏನೇ ಮಾಡುತ್ತಿರಲಿ ಅವರನ್ನು ಕಂಡು ಭಯ-ಭಕ್ತಿ-ಗೌರವದ ಭಾವವನ್ನು ತಾಳದವರು ಯಾರು ಇರಲಿಲ್ಲ! ಆದ್ದರಿಂದ ಅವರ ಸಾನ್ನಿಧ್ಯ ಮಾತ್ರದಿಂದಾಗಿಯೇ ಇತರರಿಗೆ ಆ ಪ್ರಯಾಣ ರೋಮಾಂಚಕಾರಿಯಾಗಿ ಪರಿಣಿಮಿಸಿದ್ದರೆ ಅಚ್ಚರಿಯಿಲ್ಲ.

ಲಿವರ್​ಪೂಲ್ ನಗರವನ್ನು ತಲುಪಿದ ಸ್ವಾಮೀಜಿ ಅಲ್ಲಿಂದ ಸೀದಾ ರೀಡಿಂಗ್​ಗೆ ಪ್ರಯಾಣ ಮಾಡಿದರು. ಈ ಸಲವೂ ಅವರು ಇ. ಟಿ. ಸ್ಟರ್ಡಿಯ ಮನೆಯಲ್ಲಿ ಇಳಿದುಕೊಳ್ಳುವ ಏರ್ಪಾಡಾ ಗಿತ್ತು. ಇಲ್ಲಿಗೆ ಬಂದಕೂಡಲೇ ಸ್ವಾಮೀಜಿ ತಮ್ಮ ಪ್ರಿಯ ಶಿಷ್ಯೆ ಮಿಸ್ ಮೇರಿ ಹೇಲ್​ಗೆ ಬರೆದರು, “ಪ್ರಯಾಣ ಸುಖಕರವಾಗಿತ್ತು. ಈ ಸಲ ಯಾವ ಕಾಯಿಲೆಯೂ ಬರಲಿಲ್ಲ... ಈಗ ನಾನು ಮತ್ತೊಮ್ಮೆ ಈ ರೀಡಿಂಗ್ ಪಟ್ಟಣದಲ್ಲಿ ಆತ್ಮ-ಮಾಯೆ-ಜೀವ–ಇವುಗಳ ನಡುವೆ ಇದ್ದೇನೆ. (ಅರ್ಥಾತ್, ಸ್ಟರ್ಡಿಯೊಂದಿಗೆ ಈ ವಿಚಾರಗಳ ಚರ್ಚೆಯಲ್ಲಿ ತೊಡಗಿದ್ದೇನೆ.) ಈ ಇನ್ನೊಬ್ಬರು ಸಂನ್ಯಾಸಿಗಳು (ಸ್ವಾಮಿ ಶಾರದಾನಂದರು) ಇಲ್ಲಿದ್ದಾರೆ. ನಾನು ಕಂಡ ಅತ್ಯುತ್ತಮ ವ್ಯಕ್ತಿಗಳ ಲ್ಲೊಬ್ಬರು ಅವರು; ಅಲ್ಲದೆ ಸಾಕಷ್ಟು ಚೆನ್ನಾಗಿ ತಿಳಿದಿರುವವರು. ನಾವೀಗ ಪುಸ್ತಕಗಳನ್ನು ಬರೆಯುವುದರಲ್ಲಿ ನಿರತರಾಗಿದ್ದೇವೆ... ”

ರೀಡಿಂಗಿನಲ್ಲಿ ಶಾರದಾನಂದರನ್ನು ಕಂಡಾಗ ಸ್ವಾಮೀಜಿಗಾದ ಆನಂದಕ್ಕೆ ಪಾರವೇ ಇಲ್ಲ. ಅವರು ತಮ್ಮ ಗುರುಭಾಯಿಗಳೊಬ್ಬರನ್ನು ಭೇಟಿಯಾಗಿ ಸುಮಾರು ಮೂರು ವರ್ಷವಾಗಿತ್ತು. ಇಷ್ಟು ದೀರ್ಘ ಕಾಲದ ಬಳಿಕ ತಮ್ಮ ಪ್ರಿಯ ನರೇಂದ್ರನನ್ನು ಕಂಡು ಶಾರದಾನಂದರಿಗಾದ ಸಂತೋಷವೂ ಅಷ್ಟಿಷ್ಟಲ್ಲ. ಉಭಯ ಕುಶಲೋಪರಿಗಳಾದ ಮೇಲೆ ಸ್ವಾಮಿ ಶಾರದಾನಂದರು ಆಲಂಬಜಾರ್ ಮಠದ ಸಮಸ್ತ ವಿವರಗಳನ್ನು ಮತ್ತು ಇತರ ಎಲ್ಲ ಗುರುಭಾಯಿಗಳ ಹಾಗೂ ಆಪ್ತರಾದ ಗೃಹೀಭಕ್ತರ ಯೋಗಕ್ಷೇಮದ ವಿಚಾರವನ್ನು ಸವಿಸ್ತಾರವಾಗಿ ತಿಳಿಸಿದರು. ಮಠದ ಚಟುವಟಿಕೆಗಳ ವಿವರಗಳನ್ನು ಸ್ವಾಮೀಜಿ ವಿಶೇಷ ಆಸ್ಥೆಯಿಂದ ವಿಚಾರಿಸಿಕೊಂಡರು. ಬಳಿಕ ತಮ್ಮ ಹಲವಾರು ಕಾರ್ಯಯೋಜನೆಗಳ ಬಗ್ಗೆ ಹೇಳಿದರು. ಅವರ ಆದಮ್ಯ ಶಕ್ತಿ-ಉತ್ಸಾಹಗಳನ್ನು ಹಾಗೂ ಧರ್ಮಪ್ರಸಾರಕನ ಸ್ಫೂರ್ತಿಯನ್ನು ಕಂಡು ಶಾರದಾನಂದರು ಚಕಿತರಾದರು. ತಾವು ಅಂದು ಕಂಡ ನರೇಂದ್ರನಿಗೂ ಇಂದು ಕಾಣುತ್ತಿರುವ ವಿವೇಕಾನಂದರಿಗೂ ಎಷ್ಟು ವ್ಯತ್ಯಾಸ! ಆಲಂಬಜಾರ್ ಮಠದ ಬಗ್ಗೆ ಶಾರದಾನಂದರು ತಿಳಿಸಿದ ವಿಷಯಗಳನ್ನು ಕೇಳಿದ ಕೂಡಲೇ ಸ್ವಾಮೀಜಿಯ ಮನಸ್ಸು ನೇರವಾಗಿ ಅಲ್ಲಿಗೆ ಧಾವಿಸಿತು. ಅದಾಗಲೇ ಅವರು ಮಠದ ಕಾರ್ಯ ಚಟುವಟಿಕೆಗಳ ಬಗ್ಗೆ ಹಾಗೂ ತಮ್ಮ ಗುರುಭಾಯಿಗಳಲ್ಲಿ ಪ್ರತಿಯೊಬ್ಬರೂ ಏನೇನು ಮಾಡ ಬೇಕೆಂಬುದರ ಬಗ್ಗೆ ವಿವರವಾದ ಸೂಚನೆಗಳನ್ನು ನೀಡಿ ಅವರನ್ನೆಲ್ಲ ಕಾರ್ಯೋನ್ಮುಖರನ್ನಾಗಿಸಿ ದ್ದರು. ಈಗ ಮುಖತಃ ಇನ್ನು ಕೆಲವು ವಿಚಾರಗಳು ತಿಳಿದುಬಂದಾಗ, ತಾವು ಮಠದ ಕಡೆಗೆ ಇನ್ನೂ ಹೆಚ್ಚಿನ ಗಮನಕೊಡಬೇಕೆಂದು ಅವರು ನಿಶ್ಚಯಿಸಿದರು.

ಇಂಗ್ಲೆಂಡಿನಲ್ಲಿ ಅವರು ತಮ್ಮ ಸೋದರ ಮಹೇಂದ್ರನಾಥನನ್ನು ಭೇಟಿಯಾದರು. ಇವನು ಕಾನೂನು ಶಿಕ್ಷಣಕ್ಕಾಗಿ ಇಂಗ್ಲೆಂಡಿಗೆ ಬಂದಿದ್ದ. (ಸ್ವಾಮೀಜಿಯ ಕುಟುಂಬದವರು ಬಡತನದಲ್ಲಿ ಜೀವಿಸುತ್ತಿರುವುದನ್ನು ತಿಳಿದ ಖೇತ್ರಿಯ ಮಹಾರಾಜ ಅಜಿತ್​ಸಿಂಗ್, ಸ್ವಾಮೀಜಿಯ ತಮ್ಮಂದಿ ರಾದ ಮಹೇಂದ್ರನಾಥ ಹಾಗೂ ಭೂಪೇಂದ್ರನಾಥರ ವಿದ್ಯಾಭ್ಯಾಸದ ಸಂಪೂರ್ಣ ಜವಾಬ್ದಾರಿ ಯನ್ನು ಹೊತ್ತುಕೊಂಡಿದ್ದನಲ್ಲದೆ, ಸ್ವಂತ ಅಣ್ಣನಂತೆ ಅವರ ಯೋಗಕ್ಷೇಮವನ್ನು ನೋಡಿಕೊಳ್ಳು ತ್ತಿದ್ದ. ಮಹೇಂದ್ರನಾಥನ ಉನ್ನತ ವಿದ್ಯಾಭ್ಯಾಸಕ್ಕೂ ಅವನೇ ನೆರವಾಗಿದ್ದ.) ಈ ಹಿಂದೆ ಅಮೆರಿಕ ದಲ್ಲಿ ತಮಗೆ ಪರಿಚಿತನಾಗಿದ್ದ ಜಾನ್. ಪಿ. ಫಾಕ್ಸ್​ನನ್ನೂ ಇಂಗ್ಲೆಂಡಿನಲ್ಲಿ ಸ್ವಾಮೀಜಿ ಸಂಧಿಸಿ ದರು. ಇವನು ಅವರ ಕಾರ್ಯದಲ್ಲಿ ನೆರವಾಗಲು ಮುಂದೆ ಬಂದ. ರೀಡಿಂಗಿನಲ್ಲಿ ಸುಮಾರು ಒಂದು ವಾರವಿದ್ದು ಬಳಿಕ ಸ್ವಾಮೀಜಿ ಲಂಡನ್ನಿಗೆ ಹೊರಟರು. ನೈಪುತ್ಯ ಲಂಡನ್ನಿನಲ್ಲಿ ಅವರಿಗಾಗಿ ಇ. ಟಿ. ಸ್ಟರ್ಡಿ ಒಂದು ಮನೆಯನ್ನು ಬಾಡಿಗೆಗೆ ಗೊತ್ತುಮಾಡಿದ್ದ. ಈ ಮನೆಯಲ್ಲಿ ಅವರು ಸ್ವಾಮಿ ಶಾರದಾನಂದರು, ಮಹೇಂದ್ರನಾಥ, ಗುಡ್​ವಿನ್ ಹಾಗೂ ಜಾನ್ ಫಾಕ್ಸ್​– ಇವರೊಂದಿಗೆ ವಾಸಿಸಲಾರಂಭಿಸಿದರು. ಅಲ್ಲದೆ ಸ್ಟರ್ಡಿಯೂ ಆಗಾಗ ಬಂದು ಇಲ್ಲಿ ಉಳಿದು ಕೊಳ್ಳುತ್ತಿದ್ದ. ಆ ಸಯಮದಲ್ಲಿ ಯೂರೋಪಿನಲ್ಲಿದ್ದ ಮಿಸ್ ಮೆಕಲ್​ಲಾಡಳೂ ಕೆಲದಿನಗಳಲ್ಲಿ ಲಂಡನ್ನಿಗೆ ಬಂದಳು. ಸ್ವಾಮೀಜಿಯ ಶಿಷ್ಯೆಯಾಗಿದ್ದ ಮಿಸ್ ಹೆನ್ರಿಟಾ ಮುಲ್ಲರ್ ಅವರಿಗಾಗಿ ಆ ಮನೆಯ ಕೆಳಗೇ ಇದ್ದ ಕೆಲವು ಕೋಣೆಗಳನ್ನು ಬಾಡಿಗೆಗೆ ಕೊಡಿಸಿದಳು, ಮತ್ತು ಅವರ ಖರ್ಚುಗಳಲ್ಲಿ ಕೆಲಭಾಗವನ್ನು ತಾನೇ ವಹಿಸಿಕೊಂಡಳು. ಈ ಮನೆಯಿಂದ ಸ್ವಾಮೀಜಿ ಮೇರಿ ಹೇಲ್​ಗೆ ಒಂದು ಪತ್ರದಲ್ಲಿ ಹೀಗೆ ಬರೆದರು, “ಇಲ್ಲಿ ನಮ್ಮ ಕೆಲವು ಹಳೆಯ ಸ್ನೇಹಿತರಿದ್ದಾರೆ; ಮಿಸ್ ಮೆಕ್​ಲಾಡ್ ಕೂಡ ಬಂದಿದ್ದಾಳೆ–ಚಿನ್ನದಂಥ ಹೆಂಗಸು ಅವಳು. ಈ ನಮ್ಮ ಮನೆಯಲ್ಲಿ ಒಂದು ಚಿಕ್ಕ ಸಂಸಾರವೇ ಇದೆ. ನನ್ನೊಂದಿಗೆ ಭಾರತದ ಇನ್ನೊಬ್ಬರು ಸಂನ್ಯಾಸಿಯೂ ಇದ್ದಾರೆ. ಪಾಪ ಅವರೊಬ್ಬ ಪಕ್ಕಾ ಹಿಂದೂ–ನನ್ನಂತೆ ಎಲ್ಲರನ್ನೂ ಹಿಂದೆ ತಳ್ಳಿ ಓಡಲಾರರು. ತುಂಬ ಭಾವನಾತ್ಮಕ ವ್ಯಕ್ತಿ. ವಿನಯಶಾಲಿ, ಮಧುರ ಸ್ವಭಾವದವರು. ಆದರೆ ಅವರು ಹೀಗಿದ್ದರೆ ಆಗುವುದಿಲ್ಲ. ನಾನು ಅವರಲ್ಲಿ ಸ್ವಲ್ಪ ಚಟುವಟಿಕೆಯನ್ನು ತುಂಬಲು ಪ್ರಯತ್ನಿಸುತ್ತೇನೆ... ”

ಲಂಡನ್ನಿನಲ್ಲಿ ತಮ್ಮ ತರಗತಿಗಳ ಕಾರ್ಯವನ್ನು ಪುನರಾರಂಭಿಸುವ ಮೊದಲು ಸ್ವಾಮೀಜಿ ಕೆಲದಿನಗಳ ಕಾಲ ಸ್ವಾಮಿ ಶಾರದಾನಂದರೊಡನೆ ಮಿಸ್ ಮುಲ್ಲರಳ ಮನೆಯಲ್ಲಿ ವಾಸಿಸಿದರು. ಈಕೆಯ ಮನೆಯಿದ್ದದ್ದು ರೀಡಿಂಗಿನಿಂದ ಸುಮಾರು ಹನ್ನೆರಡು ಮೈಲಿ ದೂರದಲ್ಲಿ; ಥೇಮ್ಸ್ ನದಿಯ ತೀರದಲ್ಲಿದ್ದ ಮೇಡನ್​ಹೆಡ್ ಎಂಬ ಊರಿನಲ್ಲಿ. ಬಹುಶಃ ಸ್ವಾಮೀಜಿ ಲಂಡನ್ನಿನಿಂದಾಚೆ ಗ್ರಾಮಾಂತರ ಪ್ರದೇಶದಲ್ಲಿ ಓಡಾಡಬೇಕಾದ ಈ ಸಂದರ್ಭದಲ್ಲೇ ಒಂದು ರೋಮಾಂಚಕಾರಿ ಘಟನೆ ನಡೆಯಿತು. ಅಪಾಯ ಎದುರಾದಾಗ ಅವರು ಅದನ್ನು ಎಂತಹ ಧೈರ್ಯದಿಂದ ಎದುರಿಸು ತ್ತಿದ್ದರು ಎಂಬುದಕ್ಕೆ ಈ ಘಟನೆ ಒಳ್ಳೆಯ ಉದಾಹರಣೆ:

ಒಮ್ಮೆ ಮಿಸ್ ಮುಲ್ಲರ್ ಹಾಗೂ ಒಬ್ಬ ಆಂಗ್ಲ ಸ್ನೇಹಿತನೊಂದಿಗೆ ಸ್ವಾಮೀಜಿ ಒಂದು ಹೊಲದ ದಾರಿಯಾಗಿ ನಡೆದುಕೊಂಡು ಹೋಗುತ್ತಿದ್ದರು. ಆಗ ಇದ್ದಕ್ಕಿದ್ದಂತೆ ಒಂದು ಹುಚ್ಚು ಗೂಳಿ ಅವರಿಗೆ ಎದುರಾಗಿ, ಅವರೆಡೆಗೇ ನುಗ್ಗಿ ಬಂತು. ಆಂಗ್ಲ ಸ್ನೇಹಿತ ಧೈರ್ಯವಾಗಿ ಓಡಿ ಹೋಗಿ ಒಂದು ಗುಡ್ಡದ ಹಿಂದೆ ಮರೆಯಾದ. ಮಿಸ್ ಮುಲ್ಲರ್ ಕೂಡ ಸ್ವಲ್ಪ ದೂರ ಓಡಿದಳು. ಆದರೆ ಆಕೆ ಮಧ್ಯವಯಸ್ಸು ಮೀರಿದ ಮಹಿಳೆ. ಹೆಚ್ಚು ಓಡಲಾರದೆ ಕುಸಿದು ಕುಳಿತಳು. ಸ್ವಾಮೀಜಿ ಇದನ್ನು ನೋಡಿದರು. ಆದರೆ ಆ ಗಳಿಗೆಯಲ್ಲಿ ಅವರು ಅವಳಿಗೆ ಯಾವ ರೀತಿಯಲ್ಲೂ ನೆರ ವಾಗಲು ಸಾಧ್ಯವಿರಲಿಲ್ಲ. ಆದ್ದರಿಂದ ಅವರು ಆ ಹುಚ್ಚು ಗೂಳಿಗೆ ಇದಿರಾಗಿ ಅವಳ ಬಳಿಯಲ್ಲಿ ಕೈಕಟ್ಟಿಕೊಂಡು ನಿಂತು ಅದನ್ನೇ ತೀಕ್ಷ್ಣವಾಗಿ ದಿಟ್ಟಿಸಿದರು. ಆ ಗೂಳಿ ಇವರ ಹತ್ತಿರಕ್ಕೇ ಓಡಿ ಬಂದಿತು. ಆದರೆ ಸ್ವಾಮೀಜಿಯನ್ನು ಕಂಡು ಅದು ಸ್ವಲ್ಪ ದೂರದಲ್ಲಿ ನಿಂತುಬಿಟ್ಟಿತು; ಬಳಿಕ ತಲೆಯಾಡಿಸುತ್ತ ಮಂಕುಬಡಿದಂತೆ ಹಿಂದಿರುಗಿ ಹೊರಟುಹೋಯಿತು! ಬಳಿಕ ದೂರಕ್ಕೆ ಓಡಿ ಹೋಗಿದ್ದ ಅವರ ಸ್ನೇಹಿತ ವಾಪಸು ಬಂದ. ಗೂಳಿ ಶಾಂತವಾದುದನ್ನು ಕಂಡು ಸ್ವಾಮೀಜಿಯ ಸ್ನೇಹಿತರಿಬ್ಬರಿಗೂ ಪರಮಾಶ್ಚರ್ಯ. ಆ ಸ್ನೇಹಿತ, “ಸ್ವಾಮೀಜಿ, ಆ ಗೂಳಿ ನುಗ್ಗಿ ಬರುತ್ತಿದ್ದಾಗ ಹಾಗೆ ಕೈಕಟ್ಟಿಕೊಂಡು ನಿಂತುಬಿಟ್ಟಿರಲ್ಲ, ಆಗ ನೀವು ಏನು ಯೋಚಿಸುತ್ತಿದ್ದಿರಿ?” ಎಂದು ಕೇಳಿದ. ಸ್ವಾಮೀಜಿ ಹಾಸ್ಯವಾಗಿ ಉತ್ತರಿಸಿದರು, “ಏನಿಲ್ಲ, ಒಂದು ವೇಳೆ ಆ ಗೂಳಿ ನನ್ನನ್ನು ಗುಮ್ಮಿ ಎಸೆ ದರೆ ಎಷ್ಟು ದೂರ ಹೋಗಿ ಬೀಳುತ್ತೇನೆ ಎಂದು ಲೆಕ್ಕಾಚಾರ ಹಾಕುತ್ತಿದ್ದೆ ಅಷ್ಟೆ!”

ಮೇ ತಿಂಗಳ ಕೊನೆಯ ವೇಳೆಗೆ ಸ್ವಾಮೀಜಿ ಲಂಡನ್ನಿನಲ್ಲಿ ತಮ್ಮ ಕಾರ್ಯವನ್ನು ಪೂರ್ಣ ರಭಸದಿಂದ ಪ್ರಾರಂಭಿಸಿದರು. ತಮ್ಮ ಮೊದಲ ಭೇಟಿಯ ಅವಧಿಯಲ್ಲೇ ಅವರು ತಮ್ಮ ಕಾರ್ಯಕ್ಷೇತ್ರವನ್ನು ಸಿದ್ಧಪಡಿಸಿಟ್ಟುಕೊಂಡಿದ್ದರು. ಆದರೆ ಈಗ ಅವರು ಮತ್ತೆ ಕಷ್ಟಪಟ್ಟು ಶ್ರಮಿಸಬೇಕಾದ ಆವಶ್ಯಕತೆಯಿತ್ತು. ಏಕೆಂದರೆ ಅವರು ಅದಾಗಲೇ ಯಾರ ಮೇಲೆ ಪ್ರಭಾವ ಬೀರಿದ್ದರೋ ಅದಕ್ಕಿಂತ ಹೆಚ್ಚಿನ ಸಂಖ್ಯೆಯ ಹೊಸ ಹೊಸ ಜನರನ್ನು ತಮ್ಮ ಬಳಿಗೆ ಬರಮಾಡಿ ಕೊಳ್ಳಬೇಕಾಗಿತ್ತು. ಅವರೇ ಹಿಂದೆ ಹೇಳಿದ್ದಂತೆ, ಸುಶಿಕ್ಷಿತರೂ ಬುದ್ಧಿಜೀವಿಗಳೂ ಆದ ಆ ಆಂಗ್ಲ ರಲ್ಲಿ ಒಬ್ಬೊಬ್ಬರ ಮೇಲೆ ಪ್ರಭಾವ ಬೀರುವುದೂ ಕಠಿಣವಾದ ಕಾರ್ಯವೇ ಸರಿ. ಆ ಜನರ ಎಲ್ಲ ಬಗೆಯ ವಿರೋಧವನ್ನೂ ಎದುರಿಸಿ ಅವರನ್ನು ಗೆದ್ದುಕೊಳ್ಳಬೇಕಾದರೆ ಅಪಾರ ಶಕ್ತಿ-ಉತ್ಸಾಹ- ಛಲವಂತಿಕೆ ಬೇಕು. ಆದರೆ ಈ ಕಾರ್ಯವನ್ನು ಪ್ರಾರಂಭಿಸಲು ಸ್ವಾಮೀಜಿ ಮಾನಸಿಕವಾಗಿ ಸಿದ್ಧರಾಗಿಯೇ ಇದ್ದರು.

ಮೇ ೭ರಿಂದ ಜುಲೈ ೧೬ರವರೆಗೆ ಅವರು ತಮ್ಮ ವಾಸಗೃಹದಲ್ಲೇ ನಿಯತವಾಗಿ ವಾರಕ್ಕೆ ಐದು ತರಗತಿಗಳನ್ನು ನಡೆಸಿದರು. ಇವುಗಳಲ್ಲಿ ಒಂದು ಪ್ರಶ್ನೋತ್ತರಗಳ ತರಗತಿ. ಈ ತರಗತಿ ಗಳಿಗೆ ಬರುತ್ತಿದ್ದವರ ಸಂಖ್ಯೆ ಮೊದಲು ಕಡಿಮೆಯಿದ್ದರೂ ಕ್ರಮೇಣ ಅದು ಹೆಚ್ಚಾಯಿತು. ತಮ್ಮ ಮೊದಲ ತರಗತಿಗಳಲ್ಲಿ ಆರ್ಯ ಜನಾಂಗದ ಇತಿಹಾಸ, ಅದರ ಬೆಳವಣಿಗೆ, ಅದರ ಧಾರ್ಮಿಕ ಬೆಳವಣಿಗೆ ಮತ್ತು ಅದು ತನ್ನ ಧಾರ್ಮಿಕ ಪ್ರಭಾವವನ್ನು ಹರಡಿದ ಬಗೆ–ಇವುಗಳ ಕುರಿತಾಗಿ ವಿವರಿಸಿದರು. ಇದಾದ ಮೇಲೆ ಅವರು ಕ್ರಮವಾಗಿ ಜ್ಞಾನಯೋಗ, ರಾಜಯೋಗ ಹಾಗೂ ಭಕ್ತಿಯೋಗದ ವಿಷಯವಾಗಿ ತರಗತಿಗಳ ಸರಣಿಗಳನ್ನು ನಡೆಸಿದರು. ಇವುಗಳಲ್ಲಿ ಹೆಚ್ಚಿನವನ್ನು ಗುಡ್​ವಿನ್ ಬರೆದಿಟ್ಟ.

ಸ್ವಾಮೀಜಿಯ ಈ ತರಗತಿಗಳ ಮಹತ್ವವೆಂಥದು ಎಂಬುದನನು ಲಂಡನ್ನಿನ ‘ಕ್ವೀನ್​’ ಪತ್ರಿಕೆಯ ಈ ವರದಿಯಿಂದ ನೋಡಬಹುದು–“ಸ್ವಾಮಿ ವಿವೇಕಾನಂದರು ತಮ್ಮ ತರಗತಿಗಳ ಪ್ರಾರಂಭದಿಂದಲೂ ಅಷ್ಟೊಂದು ಜನರನ್ನು ಆಕರ್ಷಿಸಲು ಸಮರ್ಥರಾಗಿರುವುದು ನಿಜಕ್ಕೂ ದೊಡ್ಡ ಆಶ್ಚರ್ಯವೇ ಸರಿ. ಲಂಡನ್ನಿನಂಥ ಈ ಭೋಗವಾದಿಗಳ ನಗರದಲ್ಲಿ ರಾಜಕಾರಣವೇ ಜನರ ತಲೆಗಳಲ್ಲಿ ಯಾವಾಗಲೂ ತುಂಬಿರುವುದು. ಧರ್ಮವೆಂದರೆ ಎಲ್ಲರೂ ಮೂಗು ಮುರಿ ಯುವವರೇ. ಹಾಗಿರುವಾಗ, ನರ್ತನಕೂಟಗಳ, ಔತಣಗಳ ಹಾಗೂ ಮನರಂಜನೆಯ ಈ ಪುತುವಿನಲ್ಲಿ ಅಷ್ಟೊಂದು ಜನ ಅವರ ತರಗತಿಗಳಿಗೆ ಹೋಗುತ್ತಾರೆಂದರೆ!”

ಈ ತರಗತಿಗಳ ಯಶಸ್ಸಿನಿಂದ ಸ್ಫೂರ್ತಿಗೊಂಡ ಸ್ವಾಮೀಜಿಯ ಸ್ನೇಹಿತರು, ಅವರ ಸಾರ್ವ ಜನಿಕ ಉಪನ್ಯಾಸಗಳ ಸರಣಿಯೊಂದನ್ನು ಏರ್ಪಡಿಸಿದರು. ಪಿಕಾಡಿಲಿಯ ‘ಪ್ರಿನ್ಸಸ್ ಹಾಲ್​’ನಲ್ಲಿ ಜೂನ್ ತಿಂಗಳ ಮೂರು ಭಾನುವಾರಗಳಂದು “ಧರ್ಮದ ಅವಶ್ಯಕತೆ”, “ಒಂದು ವಿಶ್ವಾತ್ಮಕ ಧರ್ಮ” ಮತ್ತು “ವಾಸ್ತವಿಕ ಹಾಗೂ ತೋರಿಕೆಯ ಮಾನವ” ಎಂಬ ವಿಷಯಗಳ ಬಗ್ಗೆ ಅವರು ಮಾತನಾಡಿದರು. ಈ ಉಪನ್ಯಾಸಗಳು ಪ್ರಚಂಡ ಯಶಸ್ಸು ಗಳಿಸಿದ್ದರಿಂದ, ಕೂಡಲೇ ಮೂರು ಉಪನ್ಯಾಸಗಳ ಮತ್ತೊಂದು ಸರಣಿಯನ್ನು ವ್ಯವಸ್ಥೆಗೊಳಿಸಲಾಯಿತು. ಇವುಗಳ ವಿಷಯಗಳು –“ಭಕ್ತಿಯೋಗ”, “ತ್ಯಾಗ” ಹಾಗೂ “ಸಾಕ್ಷಾತ್ಕಾರ”. ಇವೆಲ್ಲವನ್ನೂ ಗುಡ್​ವಿನ್ ಶ್ರೀಘ್ರಲಿಪಿ ಯಲ್ಲಿ ಬರೆದುಕೊಂಡ.

ಸ್ವಾಮೀಜಿ ನಡೆಸಿಕೊಂಡು ಬರುತ್ತಿದ್ದ ಈ ಎಲ್ಲ ಉಪನ್ಯಾಸಗಳು ಹಾಗೂ ತರಗತಿಗಳು, ಲಂಡನ್ನಿನಲ್ಲಿನ ಅವರ ಚಟುವಟಿಕೆಗಳ ಒಂದು ಸಣ್ಣ ಅಂಶ ಮಾತ್ರವೇ. ಇವುಗಳಲ್ಲದೆ, ಅವರು ಇನ್ನೆಷ್ಟೋ ಸಭೆಗಳಲ್ಲಿ ಮಾತನಾಡಿದರು, ಸಂದರ್ಶನಗಳನ್ನು ನೀಡಿದರು ಮತ್ತು ನಗರದ ಗಣ್ಯ ವ್ಯಕ್ತಿಗಳ ಒಕ್ಕೂಟಗಳಲ್ಲಿ ಪಾಲ್ಗೊಂಡರು. ಹೀಗೆ ಅವರು ಮಾಡಿದ ಅನೌಪಚಾರಿಕ ಭಾಷಣಗಳಲ್ಲಿ ಮುಖ್ಯವಾದುದೆಂದರೆ, ಲಂಡನ್ನಿನ ಪ್ರತಿಷ್ಠಿತ ಮಹಿಳೆಯರ “ಸೆಸೇಮ್ ಕ್ಲಬ್​”ನಲ್ಲಿ ‘ವಿದ್ಯಾ ಭ್ಯಾಸ’ಎಂಬ ವಿಷಯವಾಗಿ ಮಾತನಾಡಿದುದು. (ಕುಮಾರಿ ಮಾರ್ಗರೇಟ್ ನೋಬೆಲ್ ಈ ಕ್ಲಬ್ಬಿನ ಕಾರ್ಯದರ್ಶಿಯಾಗಿದ್ದಳು.) ವಿದ್ಯಾಭ್ಯಾಸದ ತತ್ತ್ವದ ಬಗ್ಗೆಯೂ ಇತರ ಅಂಶಗಳ ಬಗ್ಗೆಯೂ ಸ್ವಾಮೀಜಿ ವ್ಯಕ್ತಪಡಿಸಿದ್ದ ಭಾವನೆಗಳು ಅತ್ಯಂತ ಸ್ವಂತಿಕೆಯಿಂದ ಕೂಡಿದ್ದುವು. ವಿದ್ಯಾಭ್ಯಾಸ ಪದ್ಧತಿಗಳ ಸುಧಾರಣೆಯ ಬಗ್ಗೆ ಮತ್ತು ಅವುಗಳ ಅನುಷ್ಠಾನದ ಬಗ್ಗೆ ಅವರು ನೀಡಿರುವ ಸಲಹೆಗಳು ಕ್ರಾಂತಿಕಾರಕವಾದವು. ಇಂದಿಗೂ ಅವು ಹೊಸತಾಗಿಯೇ ಇವೆ. ಅವುಗಳ ಮಹತ್ವ ಸ್ವಲ್ಪವೂ ಕಡಿಮೆಯಾಗಿಲ್ಲ. ತಮ್ಮ ಈ ನೂತನ ಭಾವನೆಗಳಲ್ಲಿ ಕೆಲವನ್ನು ಸ್ವಾಮೀಜಿ ಅಂದಿನ ಉಪನ್ಯಾಸದ ಸಂದರ್ಭದಲ್ಲಿ ವ್ಯಕ್ತಪಡಿಸಿದರು. ಹೆಚ್ಚಾಗಿ ಶಿಕ್ಷಣ ಕ್ಷೇತ್ರಕ್ಕೆ ಸಂಬಂಧಿಸಿದವ ರಿಂದಲೇ ಅಂದಿನ ಸಭೆ ತುಂಬಿತ್ತು. ಅಂದು ಸ್ವಾಮೀಜಿ ಪುರಾತನ ಭಾರತದ ವಿದ್ಯಾಭ್ಯಾಸ ಪದ್ಧತಿಯನ್ನು ವಿವರಿಸಿ, ಅದರ ಹಿರಿಮೆಯನ್ನು ಬಣ್ಣಿಸಿದರು. ಆಧುನಿಕ ಜಗತ್ತಿನಲ್ಲಿ ಶಿಕ್ಷಣವು ತನ್ನ ಪವಿತ್ರತೆಯನ್ನು ಕಳೆದುಕೊಂಡು ಒಂದು ವ್ಯಾಪಾರವಾಗಿ ಪರಿಣಮಿಸಿರುವುದನ್ನು ಕಟುವಾಗಿ ಖಂಡಿಸಿದರು. ಅವರ ಕೆಲವು ಮಾತುಗಳಂತೂ ಸಭಿಕರ ಮೇಲೆ ಚಾಟಿಯೇಟಿನಂತೆ ಕೆಲಸ ಮಾಡಿದುವು. ವಿದ್ಯಾಭ್ಯಾಸದ ಅತ್ಯಂತ ಪ್ರಧಾನವಾದ ಉದ್ದೇಶವೆಂದರೆ ವ್ಯಕ್ತಿನಿರ್ಮಾಣ ಮಾಡು ವುದೇ ಹೊರತು ಅಸಂಖ್ಯಾತ ವಿಷಯಗಳನ್ನು ಮೆದುಳಿನೊಳಕ್ಕೆ ತುರುಕುವುದಲ್ಲ. ಇದು ಸ್ವಾಮೀಜಿಯ ಅತಿ ಮುಖ್ಯ ತತ್ತ್ವ.

ನಿಜಕ್ಕೂ ಈ ಉಪನ್ಯಾಸವು ಸಭಿಕರ ಮೇಲೆ ಬೀರಿದ ಪ್ರಭಾವ ಪ್ರಚಂಡವಾದದ್ದು. ಆ ಸಭೆಯಲ್ಲಿ ಹಾಜರಿದ್ದ ಎರಿಕ್ ಹ್ಯಾಮಂಡ್ ಎಂಬವನು ತನ್ನ ಸ್ಮೃತಿ ಲೇಖನದಲ್ಲಿ ಆ ಉಪ ನ್ಯಾಸದ ಬಗ್ಗೆ ಹೀಗೆ ಬರೆದಿದ್ದಾನೆ:

“ಆ ಸಂದರ್ಭದಲ್ಲಿ ನೆರೆದಿದ್ದವರ ಪೈಕಿ ಹೆಚ್ಚಿನವರು ಶಾಲಾ ಉಪಾಧ್ಯಾಯರು-ಉಪಾ ಧ್ಯಾಯಿನಿಯರು. ಉಪನ್ಯಾಸದ ವಿಷಯ ‘ವಿದ್ಯಾಭ್ಯಾಸ’ ಎಂದು ಅದಾಗಲೇ ಪ್ರಕಟವಾಗಿತ್ತು. ಸ್ವಾಮೀಜಿ ವೇದಿಕೆಯ ಮೇಲೆ ಕಾಣಿಸಿಕೊಂಡರು. ಆದರೆ ಅವರಿಗೆ ತಾವು ಮಾತನಾಡಬೇಕಾದ ವಿಷಯ ಯಾವುದು ಎಂಬುದರ ಮುನ್ಸೂಚನೆ ಸಿಕ್ಕಿರಲಾರದು. ಆದರೂ ಅವರು, ತಾವು ಯಾವುದೇ ಸಂದರ್ಭದ ನಿರೀಕ್ಷೆಗಳಿಗೂ ಮೀರಿ ನಿಲ್ಲಬಲ್ಲೆವೆಂಬುದನ್ನು ತೋರಿಸಿಕೊಟ್ಟರು. ಶಾಂತಭಾವದಿಂದ, ಸಂಪೂರ್ಣ ಆತ್ಮವಿಶ್ವಾಸದಿಂದ ತಮ್ಮ ಭಾಷಣವನ್ನು ಪ್ರಾರಂಭಿಸಿದರು. ಅವರೊಬ್ಬ ಹಿಂದೂ–ತನ್ನ ಪ್ರಾಚೀನ ನಾಗರಿಕತೆ, ಸಂಸ್ಕೃತಿಗಳ ಹೆಮ್ಮೆಯಿಂದ ಸ್ಫೂರ್ತಿಗೊಂಡ, ಹಿಂದೂ ಸಂಪ್ರದಾಯ ಹಿಂದೂ ತತ್ತ್ವಗಳಿಂದ ಭರಿತವಾದ ಹೃದಯ-ನಾಲಗೆಗಳಿಂದ ಕೂಡಿದ ಹಿಂದೂ! ಅವರು ಮಾತನಾಡುತ್ತಿದ್ದುದೊಂದು ಅದ್ಭುತ, ಅಪೂರ್ವ ದೃಶ್ಯವಾಗಿತ್ತು; ಅವಿ ಸ್ಮರಣೀಯ ಅನುಭವವಾಗಿತ್ತು. ಅವರ ಕೃಷ್ಣವರ್ಣ, ತೀಕ್ಷ್ಣದೃಷ್ಟಿಯನ್ನು ಬೀರುವ ಪ್ರಖರ ಕಂಗಳು, ಮತ್ತು ಅವರ ಉಡುಪು ಕೂಡ–ಆಕರ್ಷಣೀಯವಾಗಿದ್ದುವು, ಮನಮೋಹಕವಾಗಿ ದ್ದುವು. ಎಲ್ಲಕ್ಕಿಂತ ಮಿಗಿಲಾಗಿ, ಅವರ ಸ್ಫೂರ್ತಿಯುತ ವಾಗ್​ಝರಿಯು ಅವರಿಗೆ ಜಯಘೋಷ ವನ್ನು ಗಳಿಸಿಕೊಟ್ಟಿತು. ಇಂಗ್ಲಿಷ್ ಭಾಷೆಯ ಮೇಲೆ ಅವರಿಗಿದ್ದ ಪ್ರಭುತ್ವವು ಸಭಿಕರಿಗೆ ತುಂಬ ಸಂತಸವನ್ನು ನೀಡಿತಲ್ಲದೆ, ಎಲ್ಲರ ಗಮನವನ್ನೂ ಅವರತ್ತ ಸೆಳೆಯುವಂತಿತ್ತು. ಆ ಸಭೆಯಲ್ಲಿ ಬಹಳಷ್ಟು ಸ್ತ್ರೀಪುರುಷರ ಉದ್ಯೋಗವೇ ಆಂಗ್ಲ ವಿದ್ಯಾರ್ಥಿಗಳಿಗೆ ಅವರ ಮಾತೃಭಾಷೆಯನ್ನು, ಮತ್ತು ಅದೇ ಭಾಷಾಮಾಧ್ಯಮದಲ್ಲಿ ಇತರ ಜ್ಞಾನಶಾಖೆಗಳನ್ನು ಬೋಧಿಸುವುದಾಗಿತ್ತು ಎಂಬು ದನ್ನು ನೆನಪಿಡಬೇಕು. ಅಲ್ಲದೆ ತಾವು ಇತಿಹಾಸ-ಅರ್ಥಶಾಸ್ತ್ರಗಳಲ್ಲೂ ಅಷ್ಟೇ ಪರಿಣತರೆಂಬು ದನ್ನೂ ಸ್ವಾಮೀಜಿ ತೋರಿಸಿಕೊಟ್ಟರು. ಹೀಗೆ, ಆ ಸಭಿಕರು ಯಾವ ಕ್ಷೇತ್ರದಲ್ಲಿ ಅಪ್ರತಿಮರೆನ್ನಿಸಿ ಕೊಂಡಿದ್ದರೋ ಅದೇ ಕ್ಷೇತ್ರದಲ್ಲಿ ಸ್ವಾಮೀಜಿ ಅವರನ್ನು ಇದಿರಿಸಿದರು. ಸ್ವಾಮೀಜಿ ನಿರ್ಭೀತರಾಗಿ ಮಾತನಾಡಿದರು. ಅವರ ದನಿಯಲ್ಲಿ ಕೃಪಾಯಾಚನೆಯ ಸುಳಿವಿರಲಿಲ್ಲ. ಅವರು ‘ಹಣದಾಸೆ ಯಿಂದ ಶಿಕ್ಷಣ ನೀಡುವವನು ಅತ್ಯುನ್ನತ ಸತ್ಯಕ್ಕೆ ದ್ರೋಹವೆಸಗುತ್ತಾನೆ’ ಎಂಬ ಹಿಂದೂ ತತ್ತ್ವ ವನ್ನು ಒತ್ತಿಹೇಳಿ, ಸಭಿಕರಿಗೆ ಪೆಟ್ಟಿನ ಮೇಲೆ ಪೆಟ್ಟನ್ನು ಕೊಟ್ಟರು. ಶಿಕ್ಷಣವು ಧರ್ಮದ ಅವಿಭಾಜ್ಯ ಅಂಗವೆಂದು ಸಾರಿ, ಅದನ್ನು ವ್ಯಾಪಾರಕ್ಕಾಗಿ ಬಳಸಬಾರೆಂಬುದನ್ನು ಒತ್ತಿ ಹೇಳಿದರು. ಈಟಿಯ ಮೊನೆಯಂತಹ ಅವರ ಮಾತುಗಳು, ಸಂಪ್ರದಾಯಬದ್ಧ ಆಲೋಚನೆಯ ತರ್ಕದ ಕವಚವನ್ನು ಭೇದಿಸಿದುವು. ಆದರೂ ಅವರ ಮಾತಿನಲ್ಲಿ ಯಾವ ದುರುದ್ದೇಶವೂ ಇಲ್ಲದಿದ್ದುದರಿಂದ ಅದು ಅವರ ಉಪನ್ಯಾಸದ ಭಾವಕ್ಕೆ ಧಕ್ಕೆಯುಂಟುಮಾಡಲಿಲ್ಲ. ಅವರು ತಮ್ಮ ಸುಸಂಸ್ಕೃತ-ವಿನಯ ಶೀಲ ವರ್ತನೆಯಿಂದ ಮತ್ತು ನಿರ್ದಯಿ ಟೀಕೆಗಾರರನ್ನೂ ಗೆದ್ದುಗೊಳ್ಳಬಲ್ಲ ಅಪೂರ್ವ ಮಂದ ಹಾಸದಿಂದ ಒಂದು ವಿಶಿಷ್ಟ ಸ್ಥಾನವನ್ನು ಪಡೆದುಕೊಂಡರಲ್ಲದೆ, ಅಲ್ಲೊಂದು ಅಚ್ಚಳಿಯದ ಮುದ್ರೆಯನ್ನೊತ್ತಿದ್ದರು. ಆ ಮುದ್ರೆಯನ್ನೊತ್ತಲು ಅವರನ್ನು ಇಲ್ಲಿಗೆ ಕಳಿಸಿದ್ದುದು ಸ್ವತಃ ಶ್ರೀರಾಮಕೃಷ್ಣರ ಚೇತನವೇ. ಆ ಕಾರ್ಯದಲ್ಲಿ ಅವರು ಪ್ರಥಮ ಪ್ರಯತ್ನದಲ್ಲೇ ಯಶಸ್ವಿಗಳಾಗಿದ್ದರು.

“ಈ ಉಪನ್ಯಾಸದ ಬಳಿಕ ಚರ್ಚೆ ಪ್ರಾರಂಭವಾಯಿತು. ಶಿಕ್ಷಣಕ್ಕೆ ಶುಲ್ಕ ವಿಧಿಸುವುದಕ್ಕೆ ಆಧುನಿಕ ಯುಗದ ಬದಲಾದ ಪರಿಸರವೇ ಮೊದಲಾದ ಕಾರಣಗಳನ್ನು ಮುಂದಿಡಲಾಯಿತು. ಆದರೆ ಸ್ವಾಮೀಜಿ ತಮ್ಮ ವಾದವನ್ನು ಬದಲಿಸಲಿಲ್ಲ.

“ಹೀಗಿತ್ತು ಅವರೊಂದಿಗೆ ನನ್ನ ಮೊದಲ ಭೇಟಿ. ಈ ಭೇಟಿಯ ಪರಿಣಾಮವಾಗಿ ನಾನು ಅವರ ಗಾಢ ಸ್ನೇಹ ಬೆಳೆಸುವಂತೆ, ಅವರನ್ನು ಹೃತ್ಪೂರ್ವಕವಾಗಿ ಮೆಚ್ಚಿಕೊಳ್ಳುವಂತೆ ಅವರನ್ನು ಕೃತಜ್ಞತಾಪೂರ್ವಕವಾಗಿ ಸ್ಮರಿಸಿಕೊಳ್ಳುವಂತೆ ಆಯಿತು.”

ಮೊದಮೊದಲು ಲಂಡನ್ನಿನಲ್ಲಿ ಸ್ವಾಮೀಜಿಗೆ ಸಿಕ್ಕಿದ ಸ್ವಾಗತವೇನೂ ಜನರ ಸ್ವಯಂಸ್ಫೂರ್ತಿ ಯಿಂದ ಉಕ್ಕಿಬಂದದ್ದಲ್ಲ. ಅವರು ಆ ಸ್ವಾಗತವನ್ನು ತಮ್ಮ ಸ್ವಂತ ಅರ್ಹತೆಯಿಂದಲೇ ಗಳಿಸಿಕೊಂಡದ್ದು. ಇದನ್ನು ಗಮನಿಸಿದ್ದ ಎರಿಕ್ ಹ್ಯಾಮಂಡ್ ತನ್ನ ಸ್ಮೃತಿ ಚಿತ್ರಣದಲ್ಲಿ ಅದನ್ನು ವಿವರಿಸುತ್ತಾನೆ:

“ಸ್ವಾಮಿ ವಿವೇಕಾನಂದರು ಲಂಡನ್ನಿಗೆ ಆಗಮಿಸಿದಾಗ ಅಲ್ಲಿನ ಜನ ತಮಗೆ ಸ್ವಭಾವಸಹಜ ವಾದ ಮೌನ-ಬಿಗುಮಾನ-ಅರೆಲೆಕ್ಕಾಚಾರದ ರೀತಿಯಲ್ಲೇ ಅವರಿಗೆ ಸ್ವಾಗತವನ್ನು ಕೋರಿದರು. ಬಹುಶಃ ಧರ್ಮಪ್ರಚಾರಕನೊಬ್ಬನಿಗೆ ಎಲ್ಲೆಡೆಯೂ ಎದುರಾಗುವ ವಾತಾವರಣವು ಸಂಪೂರ್ಣ ಪ್ರತಿಕೂಲವಾಗಿರದಿದ್ದರೂ, ಹೆಚ್ಚೆಂದರೆ ಅದು ಸಂಶಯಪೂರ್ಣದ್ದಾಗಿರುತ್ತದೆ. ಈ ಬಗೆಯ ಸಂಶಯ ಹಾಗೂ ಕುತೂಹಲದ ವಾತಾವರಣವನ್ನು ಸ್ವಾಮೀಜಿ ಚೆನ್ನಾಗಿ ಗುರುತಿಸಿದರು. ಆದರೆ ಅವರ ಅಜೇಯ ವ್ಯಕ್ತಿತ್ವವು ಆ ಪ್ರತಿಬಂಧಕಗಳನ್ನು ಪಕ್ಕಕ್ಕೆ ಸರಿಸುತ್ತ ಮುನ್ನುಗ್ಗಿತು; ಹಲವಾರು ಹೃದಯಗಳಲ್ಲಿ ಸಂತಸದ ಸ್ವಾಗತವನ್ನು ಕಂಡುಕೊಂಡಿತು.

“ಕ್ಲಬ್ಬುಗಳು, ಸೊಸೈಟಿಗಳು, ದಿವಾನ್​ಖಾನೆಗಳು ಅವರಿಗೆ ತಮ್ಮ ಬಾಗಿಲುಗಳನ್ನು ತೆರೆದುವು. ವಿದ್ಯಾರ್ಥಿಗಳ ತಂಡಗಳು ಅವರ ಮಾತುಗಳನ್ನು ಗುಂಪುಗಟ್ಟಿ ಆಲಿಸಿದುವು. ಅವರ ಮಾತುಗಳನ್ನು ಕೇಳಿದವರು ಇನ್ನಷ್ಟು ಮತ್ತಷ್ಟು ಕೇಳಲು ಬಯಸಿದರು.”

ಸ್ವಾಮೀಜಿ ಇತರೆಡೆಗಳಲ್ಲಿ ಮಾಡಿದ ಭಾಷಣಗಳಲ್ಲಿ “ಹಿಂದೂ ಕಲ್ಪನೆಯಲ್ಲಿಆತ್ಮ” ಎಂಬು ದೊಂದು. ಲಂಡನ್ನಿನ ಶ್ರೀಮತಿ ಜಾನ್ ಮಾರ್ಟಿನ್ ಎಂಬ ಗಣ್ಯ ಮಹಿಳೆಯ ಮನೆಯಲ್ಲಿ ನಡೆದ ಈ ಭಾಷಣದ ಕಾರ್ಯಕ್ರಮಕ್ಕೆ ಹಲವಾರು ಪ್ರತಿಷ್ಠಿತ ನಾಗರಿಕರು ಆಗಮಿಸಿದ್ದರು. ಸಭಿಕರ ಪೈಕಿ ಮಾರುವೇಷದಲ್ಲಿ ಕುಳಿತಿದ್ದ ಲಂಡನ್ನಿನ ರಾಜಮನೆತನದ ಕೆಲವರೂ ಇದ್ದರು. ಆದರೆ ಇವರು ಹೀಗೆ ಗುಪ್ತವಾಗಿ ಕುಳಿತುಕೊಳ್ಳಬೇಕಾದ ಅವಶ್ಯಕತೆಯೇನಿತ್ತು? ಇವರೇನೂ ಸ್ವಾಮೀಜಿಯ ಸತ್ವ ವನ್ನು ಪರೀಕ್ಷಿಸಲು ಬಂದವರಲ್ಲ. ಆದರೆ ರಾಜಮನೆತನದ ಘನವ್ಯಕ್ತಿಗಳಾದ ಇವರಂಥವರು ತಮ್ಮ ಗುಲಾಮರಾಷ್ಟ್ರದವನಾದ ಒಬ್ಬ ಬಡ ಫಕೀರನ ಭಾಷಣವನ್ನು ಕೇಳಲು ಹೋದರೆಂದು ಜನ ಆಡಿಕೊಳ್ಳಬಹುದಲ್ಲವೆ? ಅದರಲ್ಲೂ ವಿಧರ್ಮೀಯನೊಬ್ಬನನ್ನು ಪ್ರೋತ್ಸಾಹಿಸುವುದರ ಮೂಲಕ ಸ್ವಧರ್ಮಕ್ಕೆ ಅಪಚಾರವೆಸಗಿದ್ದಾರೆಂದು ಸಂಪ್ರದಾಯಸ್ಥರು ಟೀಕಿಸಬಹುಲ್ಲವೆ? ಆದ್ದ ರಿಂದ ಇವರು ಸ್ವಾಮೀಜಿಯ ಉಪನ್ಯಾಸವನ್ನು ಕೇಳಬೇಕೆಂಬ ತಮ್ಮ ಕುತೂಹಲವನ್ನು ಈ ರೀತಿಯಾಗಿ ತಣಿಸಿಕೊಂಡಿರಬಹುದು. ಲಂಡನ್ನಿನಲ್ಲಿ ಸ್ವಾಮೀಜಿ ಎಂತಹ ಪ್ರಭಾವ ಬೀರಿದ್ದ ರೆಂಬುದಕ್ಕೆ ಇದೊಂದು ಸಾಕ್ಷ್ಯ.

ಜುಲೈ ೯ರಂದು ಸ್ವಾಮೀಜಿ ಥಿಯಸಾಫಿಕಲ್ ಸೊಸೈಟಿಯ ಪ್ರಮುಖ ವ್ಯಕ್ತಿಯಾದ ಶ್ರೀಮತಿ ಆ್ಯನ್ನಿ ಬೆಸೆಂಟರ ಆಹ್ವಾನದ ಮೇರೆಗೆ ಲಂಡನ್ನಿನ ಅವರ ಮನೆಯಲ್ಲಿ ಭಾಷಣ ಮಾಡಿದರು. ಭಾಷಣದ ವಿಷಯ ‘ಭಕ್ತಿಯೋಗ’. ಆ ದಿನಗಳಲ್ಲಿ ಥಿಯಸಾಫಿಕಲ್ ಸೊಸೈಟಿಯ ಪ್ರಭಾವ ಹಾಗೂ ಹಿಡಿತ, ಅದರಲ್ಲೂ ಇಂಗ್ಲೆಂಡಿನಲ್ಲಿ, ಅತ್ಯಂತ ಪ್ರಬಲವಾಗಿತ್ತು. ಸ್ವಾಮೀಜಿ ಈ ಸೊಸೈಟಿಯ ತತ್ತ್ವಗಳನ್ನೂ ಕಾರ್ಯವಿಧಾನಗಳನ್ನೂ ಅನುಮೋದಿಸುತ್ತಿರಲಿಲ್ಲವೆಂಬುದು ಎಲ್ಲ ರಿಗೂ ತಿಳಿದಿದ್ದ ವಿಷಯವೇ. ಆದರೆ, ಯಾವುದನ್ನೂ ಟೀಕಿಸದಿರುವುದು ಮತ್ತು ಯಾರನ್ನೂ ಎದುರುಹಾಕಿಕೊಳ್ಳದಿರುವುದು ಸ್ವಾಮೀಜಿಯ ತತ್ತ್ವವಾಗಿತ್ತು. ಆದ್ದರಿಂದಲೇ ಅವರು ಶ್ರೀಮತಿ ಬೆಸೆಂಟರ ಮನೆಯಲ್ಲಿ ಮಾತನಾಡಲು ಒಪ್ಪಿಕೊಂಡದ್ದು. ಹೀಗೆಯೇ ಅವರು ಇತರ ಹಲವಾರು ಗಣ್ಯವ್ಯಕ್ತಿಗಳ ಮನೆಗಳಲ್ಲಿ, ಕ್ಲಬ್ಬುಗಳಲ್ಲಿ, ತುಂಬಿದ ಸಭೆಗಳನ್ನುದ್ದೇಶಿಸಿ ಮಾತನಾಡಿದರು. ಅಲ್ಲದೆ ಅನೇಕ ಔತಣಕೂಟಗಳಲ್ಲೂ ಭಾಗವಹಿಸಿ, ತನ್ಮೂಲಕ ಸಮಾಜದ ಎಲ್ಲ ರಂಗಗಳಿಗೆ ಸೇರಿದ ಹಲವಾರು ಪ್ರಮುಖ ವ್ಯಕ್ತಿಗಳನ್ನು ಭೇಟಿಯಾದರು.

ಲಂಡನ್ನಿನಲ್ಲಿ ಸ್ವಾಮೀಜಿ ನೀಡಿದ ಮತ್ತೊಂದು ಉಪನ್ಯಾಸದ ದೃಶ್ಯವು ಎರಿಕ್ ಹ್ಯಾಮಂಡನ ಮೂಲಕ ಕಾಣಸಿಗುತ್ತದೆ:

ಆ ಉಪನ್ಯಾಸದ ಕೊನೆಯಲ್ಲಿ ಬಿಳಿಗೂದಲಿನ ಪ್ರಸಿದ್ಧ ತತ್ತ್ವಶಾಸ್ತ್ರಜ್ಞನೊಬ್ಬ ಎದ್ದುನಿಂತು ಹೇಳಿದ:

“ನೀವು ನಿಜಕ್ಕೂ ಅತ್ಯಂತ ಅಮೋಘವಾಗಿ ಮಾತನಾಡಿದ್ದೀರಿ, ಮಹಾಶಯರೆ; ಅದಕ್ಕಾಗಿ ನಾನು ನಿಮ್ಮನ್ನು ಹಾರ್ದಿಕವಾಗಿ ಅಭಿನಂದಿಸುತ್ತೇನೆ. ಆದರೆ ನೀವು ನಮಗೆ ಹೊಸತಾದುದೇನನ್ನೂ ಹೇಳಿಲ್ಲ.”

ತುಂಬಿದ ಸಭೆಯಲ್ಲಿ ವಯೋವೃದ್ಧ-ಜ್ಞಾನವೃದ್ಧನೊಬ್ಬ ಹೀಗೇ ನೇರವಾಗಿ ಟೀಕಿಸಿದರೆ ಎಂತಹ ವಾಗ್ಮಿಯೂ ಅಪ್ರತಿಭನಾಗಿ ತೊದಲಲೇಬೇಕು! ಎಂತಹ ಆತ್ಮವಿಶ್ವಾಸವೂ ಕೊಚ್ಚಿ ಹೋಗಲೇಬೇಕು! ಆದರೆ ಸ್ವಾಮೀಜಿ ಕಿಂಚಿತ್ತಾದರೂ ವಿಚಲಿತರಾಗದೆ ತಮ್ಮ ಸಹಜ ಮುಗುಳ್ನಗೆ ಯನ್ನು ಬೀರುತ್ತ, ಮಧುರ ಗಂಭೀರ ದನಿಯಲ್ಲಿ ಉತ್ತರಿಸಿದರು:

“ಮಹಾಶಯರೆ, ನಾನು ನಿಮಗೆ ಏಕಮಾತ್ರವಾದ ಪರಮ ಸತ್ಯವನ್ನು ತಿಳಿಸಿದ್ದೇನೆ. ಆ ಸತ್ಯವು ಅನಾದಿ ಕಾಲದ ಪರ್ವತಗಳಷ್ಟು ಹಳೆಯದು, ಮಾನವತೆಯಷ್ಟು ಹಳೆಯದು, ಸೃಷ್ಟಿಯಷ್ಟು ಹಳೆಯದು, ಆ ಮಹೇಶ್ವರನಷ್ಟೇ ಹಳೆಯದು! ಆದರೆ ನಾನದನ್ನು, ನಿಮ್ಮನ್ನು ಆಲೋಚಿಸುವಂತೆ ಮಾಡುವ ಹಾಗೂ ಆಲೋಚಿಸಿ ಅದಕ್ಕೆ ತಕ್ಕಂತೆ ನಡೆದುಕೊಳ್ಳುವಂತೆ ಮಾಡುವ ರೀತಿಯಲ್ಲಿ ಹೇಳಿರುವೆನಾದರೆ, ನಿಜಕ್ಕೂ ನಾನು ಮಾಡಿದುದು ಸಾರ್ಥಕವಲ್ಲವೆ?”

ಪ್ರಶ್ನೆಗೆ ತಕ್ಕಂತಹ ಉತ್ತರ! ಸ್ವಾಮೀಜಿ ನೀಡಿದ ಈ ಉತ್ತರದ ಮರ್ಮವನ್ನು ಗಮನಿಸಬೇಕು. ‘ನಿಮ್ಮ ಮಾತಿನಲ್ಲಿ ಹೊಸತೇನೂ ಇಲ್ಲ’ ಎಂಬ ತತ್ತ್ವಶಾಸ್ತ್ರಜ್ಞನ ಮಾತಿಗೆ ಅವರು ಉತ್ತರದ ಭಾವವಿಷ್ಟೆ: ‘ಹೌದು; ನಾನು ನಿಮಗೆ ತಿಳಿಸಿದುದು ಖಂಡಿತವಾಗಿಯೂ ಹೊಸ ವಿಚಾರವಲ್ಲ. ಅದು ಸಾಕಷ್ಟು ಹಳೆಯದು; ಮಾತ್ರವಲ್ಲ, ಸನಾತನವಾದುದು, ಅನಾದಿಯಾದುದು!! ಆದರೆ ಇಂತಹ ಪರಮ ಸತ್ಯವನ್ನು ತಿಳಿದೂ ನಿಮಗೆ (ತತ್ತ್ವವಾದಿಗಳಿಗೆ) ಅದರಿಂದೇನಾಯಿತು? ಈಗ ನಾನು ಅದನ್ನೇ ನೀವು ಕಾರ್ಯಗತಗೊಳಿಸಲಾಗುವ ರೀತಿಯಲ್ಲಿ ಹೇಳಿದರೆ ಸಾಲದೆ?’ ಎಂದು. ಹೌದು; ಉಪಯೋಗಕ್ಕೆ ಬಾರದಿದ್ದ ಮೇಲೆ ಯಾವ ತತ್ತ್ವಜ್ಞಾನದಿಂದ ತಾನೆ ಏನು ಪ್ರಯೋಜನ? ಸ್ವಾಮೀಜಿಯ ಉತ್ತರವನ್ನು ಕೇಳಿ ಕರತಾಡನದಿಂದ ಸಭೆ ಗಡಚಿಕ್ಕಿತು. ‘ಭಲೇ! ಭಲೇ!’ ಎಂಬ ಹರ್ಷೋದ್ಗಾರವೂ ಕೇಳಿಬಂದಿತು. ಸಭಿಕರ ಮಧ್ಯದಲ್ಲಿದ್ದ ಮಹಿಳೆಯೊಬ್ಬಳು ಬಳಿಕ ಹೇಳಿದಳು:

“ನಾನು ಜೀವನದುದ್ದಕ್ಕೂ ಚರ್ಚುಗಳಲ್ಲಿ ಎಷ್ಟೋ ಆರಾಧನೆಗಳಲ್ಲಿ ಭಾಗವಹಿಸಿದ್ದೇನೆ. ಆದರೆ ಅವು ತೀರ ಸತ್ವಹೀನವಾಗಿ, ಸಪ್ಪೆಯಾಗಿ ತೋರಿದ್ದುವು. ಇತರರು ಹೋಗುತ್ತಾರಲ್ಲ ಎಂದು ನಾನೂ ಹೋಗುತ್ತಿದ್ದೆ. ಅಲ್ಲದೆ ಇತರರೆಲ್ಲರಿಗಿಂತ ತಾನೊಬ್ಬನೇ ಬೇರೆಯಾಗಿರಲು ಯಾರು ತಾನೆ ಇಷ್ಟಪಡುತ್ತಾರೆ? ನಾನು ಸ್ವಾಮೀಜಿಯ ಮಾತುಗಳನ್ನು ಕೇಳಿದಾಗಿನಿಂದ, ಧರ್ಮದೊಳಕ್ಕೆ ಹೊಸ ಬೆಳಕೊಂದು ಹರಿದು ಬಂದಂತೆ ತೋರುತ್ತಿದೆ. ಅದು ಜೀವಂತ! ಅದು ಸತ್ಯ! ಈಗ ಅದಕ್ಕೊಂದು ಹೊಸ, ಆಹ್ಲಾದಕರ ಅರ್ಥವಿದೆ; ಮತ್ತು ಧರ್ಮದ ಸ್ವರೂಪವು ನನ್ನ ಪಾಲಿಗೆ ಸಂಪೂರ್ಣವಾಗಿ ಪರಿವರ್ತಿತವಾದಂತೆ ಕಾಣುತ್ತಿದೆ.”

ಸ್ವಾಮೀಜಿ ಅಂದು ನೀಡಿದ ಉಪನ್ಯಾಸ ಹಾಗೂ ಪ್ರಶ್ನೆಗಳಿಗೆ ಕೊಟ್ಟ ಉತ್ತರಗಳ ಬಗ್ಗೆ ಎರಿಕ್ ಹ್ಯಾಮಂಡ್ ಬರೆಯುತ್ತಾನೆ:

“.... ಬಳಿಕ ಸ್ವಾಮೀಜಿ, ‘ನಾನು ಹೇಗೆ ಸತ್ಯವನ್ನು ಕಂಡುಕೊಂಡೆ ಎಂಬುದನ್ನು ಈಗ ತಿಳಿಸುತ್ತೇನೆ’ ಎನ್ನುತ್ತ ತಮ್ಮ ಮಾತನ್ನು ಮುಂದುವರಿಸಿದರು. ಅವರ ವ್ಯಕ್ತಿತ್ವದ ಮುಖ್ಯ ಅಂಶವಾದ ಸರಳತೆಯ ಮತ್ತು ಪ್ರತಿಯೊಂದು ಧರ್ಮದಲ್ಲೂ ಸತ್ಯವನ್ನರಸಿ ಅವರು ನಡೆಸಿದ ಅವಿರತ ಹೋರಾಟದ ಅರಿವು ಜನರಿಗೆ ಉಂಟಾಯಿತು.

“ಸ್ವಾಮೀಜಿ ಹೇಳಿದರು: ‘ನಾನು ಸತ್ಯವನ್ನು ಕಂಡುಕೊಂಡೆ–(ಹೇಗೆಂದರೆ,) ಅದು ನನ್ನೊಳಗೆ ಇತ್ತು! ಆತ್ಮವಂಚನೆ ಮಾಡಿಕೊಳ್ಳಬೇಡಿ. ನೀವು ಆ ಸತ್ಯವನ್ನು ಆ ಮತದಲ್ಲೋ ಈ ಮತದಲ್ಲೋ ಪಡೆದುಕೊಳ್ಳುತ್ತೀರೆಂದು ನಿರೀಕ್ಷಿಸಬೇಡಿ. ಅದು ನಿಮ್ಮೊಳಗೇ ಇದೆ. ನಿಮ್ಮ ಮತ ಅದನ್ನು ನಿಮಗೆ ಕೊಡುವುದಿಲ್ಲ. ನೀವು ಅದನ್ನು ನಿಮ್ಮ ಮತಕ್ಕೆ ಕೊಡಬೇಕು. ನೀವು ಅದನ್ನು–ಆ ಅಮೂಲ್ಯ ರತ್ನವನ್ನು–ನಿಮ್ಮೊಳಗೇ ಹೊಂದಿದ್ದೀರಿ. ಇರುವುದೆಲ್ಲ ಒಂದೇ. ತತ್ತ್ವಮಸಿ! –ನೀನು ಅದೇ ಆಗಿದ್ದೀಯೆ!’

“ಈ ಉಪನ್ಯಾಸದ ಮೊದಲಿನಿಂದ ಕಡೆಯವರೆಗೂ ಸ್ವಾಮೀಜಿ ತಮ್ಮ ಗುರುದೇವ ಶ್ರೀರಾಮ ಕೃಷ್ಣರ ಸಂದೇಶದ ಕುರಿತಾಗಿಯೇ ಮಾತನಾಡಿದರು. ಅವರು ಹೇಳಿದರು–‘ನಿಮಗೆ ಹೇಳಲು ನನಲ್ಲಿ ನನ್ನದೇ ಆದ ಒಂದೇ ಒಂದು ಪದವೂ ಇಲ್ಲ; ನಿಮಗೆ ವಿವರಿಸಲು ನನ್ನಲ್ಲಿ ನನ್ನದೇ ಆದ ವಿಚಾರ ಒಂದು ಸಾಸಿವೆ ಕಾಳಿನಷ್ಟೂಇಲ್ಲ. ಪ್ರತಿಯೊಂದೂ ಕೂಡ–ನನಗೆ ನಾನು ಏನಾಗಿ ದ್ದೇನೆಯೋ, ನಿಮ್ಮ ಪಾಲಿಗೆ ನಾನು ಏನಾಗಿದ್ದೇನೆಯೋ, ಅಥವಾ ಇಡೀ ಜಗತ್ತಿನ ಪಾಲಿಗೆ ನಾನು ಏನಾಗಿದ್ದೇನೆಯೋ ಅದೆಲ್ಲವೂ–ಆ ಏಕೈಕ ಮೂಲದಿಂದ, ಎಂದರೆ ಪರಿಶುದ್ಧಾತ್ಮರೂ ಅಸೀಮ ಸ್ಫೂರ್ತಿಯ ಸೆಲೆಯೂ ಆದ ಶ್ರೀರಾಮಕೃಷ್ಣರಿಂದ ಬಂದದ್ದು....’

“ಅದ್ಭುತ ವಾಕ್​ಲಹರಿಯಿಂದ ಸ್ವಾಮೀಜಿ, ಶ್ರೀರಾಮಕೃಷ್ಣರ ವಿಷಯವನ್ನು ವಿವರಿಸಿದರು. ಆಗ ಅವರಿಗೆ ತಮ್ಮತನವೇ ಸಂಪೂರ್ಣವಾಗಿ ಮರೆತುಹೋಗಿತ್ತು. ಅವರ ಅಹಮಿಕೆಯು ಸಂಪೂರ್ಣ ಅಳಿಸಿಹೋಗಿತ್ತು. ಅವರು ಹೇಳಿದರು, ‘ನಾನು ಈಗ ಏನಾಗಿದ್ದೇನೆಯೋ ಅದೆಲ್ಲವೂ ಶ್ರೀರಾಮಕೃಷ್ಣರ ಕೃಪೆಯಿಂದಲೇ. ನನ್ನಲ್ಲಿ, ನನ್ನ ಮಾತುಗಳಲ್ಲಿ ಏನಾದರೂ ಒಳ್ಳೆಯದಿದ್ದರೆ ಅದು ನನಗೆ ಬಂದದ್ದು ಅವರ ಬಾಯಿಂದ, ಅವರ ಹೃದಯದಿಂದ, ಅವರ ಚೇತನದಿಂದ. ಜಗತ್ತಿನ ಆ ಕಾಲದ ಧಾರ್ಮಿಕ ಜೀವನ ಹಾಗೂ ಅದರ ಚಟುವಟಿಕೆಗಳೆಲ್ಲಕ್ಕೂ ಶ್ರೀರಾಮಕೃಷ್ಣರೇ ಚಿಲುಮೆಯಿದ್ದಂತೆ. ನಾನು ಜಗತ್ತಿಗೆ ನನ್ನ ಗುರುದೇವನ ಒಂದೇ ಒಂದು ಇಣುಕು ನೋಟವ ನ್ನಾದರೂ ತೋರಿಸಬಲ್ಲೆನಾದರೆ, ನನ್ನ ಜೀವಿತವು ನಿರರ್ಥಕವಾಗಿಲ್ಲವೆಂದು ತಿಳಿಯುತ್ತೇನೆ.’”

ಲಂಡನ್ನಿನಲ್ಲಿ ಸ್ವಾಮೀಜಿಯ ಉಪನ್ಯಾಸಗಳು ಬಹಳ ಜನಪ್ರಿಯವಾಗಿ, ಚೆನ್ನಾಗಿ ನಡೆದು ಕೊಂಡು ಬರುತ್ತಿದ್ದುವು. ಆದರೆ ಅವರ ಯಶಸ್ಸು ಅನೇಕರನ್ನು ರೊಚ್ಚಿಗೆಬ್ಬಿಸಿತ್ತು. ಅದರಲ್ಲೂ ಜನಾಂಗಭೇದ-ವರ್ಣಭೇದಗಳ ಪೂರ್ವಾಗ್ರಹ ಪೀಡಿತರಾಗಿದ್ದ ಕೆಲವು ಅಧಿಕಾರಿಗಳು–ಮುಖ್ಯ ವಾಗಿ ಭಾರತದಿಂದ ಹಿಂದಿರುಗಿದವರು–ಪೌರ್ವಾತ್ಯವಾದದ್ದನ್ನೆಲ್ಲ ದ್ವೇಷಿಸುತ್ತಿದ್ದರು. ಇಂಥವ ರಲ್ಲೊಬ್ಬನು ಸ್ವಾಮೀಜಿಯ ಉಪನ್ಯಾಸಕ್ಕೆ ಅಡ್ಡಿ ಮಾಡಿದ ಕಹಿ ಪ್ರಸಂಗವೊಂದನ್ನು ಮಹೇಂದ್ರನಾಥ ತಿಳಿಸುತ್ತಾನೆ:

ಒಂದು ಸಂಜೆ ಸ್ವಾಮೀಜಿ ವೇದಿಕೆಯ ಮೇಲೆ ನಿಂತು, ರಾಜಯೋಗದ ಮೇಲೆ ಸ್ಫೂರ್ತಿ ಯುತವೂ ವಿಚಾರಪೂರ್ಣವೂ ಆದ ಪ್ರವಚನವೊಂದನ್ನು ಆಗತಾನೆ ಪ್ರಾರಂಭಿಸಿದ್ದರು. ಟಿಪ್ಪಣಿ ಗಳನ್ನು ಬರೆದುಕೊಳ್ಳಲು ಗುಡ್​ವಿನ್ ಸಿದ್ಧನಾಗುತ್ತಿದ್ದ. ಸ್ವಾಮಿ ಶಾರದಾನಂದರು, ಇ. ಟಿ. ಸ್ಟರ್ಡಿ, ಜಾನ್ ಫಾಕ್ಸ್ ಹಾಗೂ ಮಹೇಂದ್ರನಾಥ–ಇವರೆಲ್ಲ ಅಲ್ಲಿ ಉಪಸ್ಥಿತರಿದ್ದರು. ಸಭಿಕರು ಅತ್ಯಂತ ಆಸಕ್ತಿಯಿಂದ ನಿಶ್ಶಬ್ದವಾಗಿ ಕುಳಿತು ಸ್ವಾಮೀಜಿಯ ಮಾತುಗಳನ್ನು ಆಲಿಸುತ್ತಿದ್ದರು. ಇದಕ್ಕಿದ್ದಂತೆ ಕೋಣೆಯ ಒಂದು ಮೂಲೆಯಿಂದ ಕೆಟ್ಟ ಸ್ವರದಲ್ಲಿ \eng{‘Oh, Thank you!’} ಎಂದು ಯಾರೋ ಅರಚಿದರು. ಸಭಿಕರೆಲ್ಲ ಆಘಾತಗೊಂಡು ಆ ಕಡೆಗೆ ತಿರುಗಿ ನೋಡಿದರು. ಅವನೊಬ್ಬ ಭಾರತದಿಂದ ಸ್ವದೇಶಕ್ಕೆ ಹಿಂದಿರುಗಿದ್ದ ನಿವೃತ್ತ ಅಧಿಕಾರಿ. ಅತ್ಯಂತ ಅಲಕ್ಷ್ಯಭಾವದಿಂದ ಆತ ಕುಳಿತಿದ್ದ. ಆದರೆ ಸ್ವಾಮೀಜಿ ಅದಕ್ಕೆ ಸ್ವಲ್ಪವೂ ಗಮನಕೊಡದೆ ತಮ್ಮ ಮಾತನ್ನು ಮುಂದುವರಿಸಿ ದ್ದರಿಂದ ಜನ ಸುಮ್ಮನಾಗಬೇಕಾಯಿತು. ಆದರೆ ಆ ಮನುಷ್ಯ ಮತ್ತೆಮತ್ತೆ ಅದೇ ರೀತಿಯಾಗಿ ಆಗಾಗ ಕೂಗಲಾರಂಭಿಸಿದ. ಸಭಿಕರಿಗೆಲ್ಲ ಭಯಂಕರ ಕೋಪ ಬಂದಿತಾದರೂ ಸ್ವಾಮೀಜಿ ಮಾತ್ರ ತಮ್ಮ ಮಾತನ್ನು ನಿಲ್ಲಿಸದಿದ್ದುದರಿಂದ ಏನು ಮಾಡಲೂ ತೋಚದೆ ಸುಮ್ಮನೆ ಅವನತ್ತ ತಿರು ತಿರುಗಿ ದುರುಗುಟ್ಟಿದರು. ಸ್ವಾಮೀಜಿ ಬುದ್ಧನ ತ್ಯಾಗ ಹಾಗೂ ಅವನ ಮಹೋನ್ನತ ಶಾಂತಿ ಸಂದೇಶವನ್ನು ವಿವರಿಸುತ್ತಿದ್ದಾಗ ಈ ಮನುಷ್ಯ ಕೂಗಿದ–“ಬುದ್ಧ ಒಬ್ಬ ಮಹಾ ಸ್ವಾರ್ಥಿ, ಕಟುಕ, ಹೆಂಡತಿ-ಮಕ್ಕಳನ್ನು ಬಿಟ್ಟು ಓಡಿಹೋದವನು. ಅವನೊಬ್ಬ ನಾಸ್ತಿಕ!” ಆದರೂ ಸ್ವಾಮೀಜಿ ಶಾಂತವಾಗಿಯೇ ಇದ್ದರು. ಬಳಿಕ ಅವರು ಬುದ್ಧನ ವಿಶಾಲ ಮನೋಭಾವವನ್ನು ವರ್ಣಿಸಿ ಇಂದಿಗೂ ಭಾರತದಲ್ಲಿ ಅತ್ಯಂತ ಶ್ರೇಷ್ಠವರ್ಗದ ತಪಸ್ವಿಗಳಿದ್ದಾರೆ ಎಂದು ಹೇಳಿದರು. ಆಗ ಆ ವಕ್ರ ಮತ್ತೆ ಕೆಟ್ಟದಾಗಿ ಕೂಗಿದ–“ಇಲ್ಲ, ಅದು ಸುಳ್ಳು! ಆ ಸಾಧುಗಳೆನ್ನಿಸಿಕೊಂಡವರೆಲ್ಲ ಕಳ್ಳಕೊರಮರು. ನಾನು ಭಾರತದಲ್ಲಿದ್ದಾಗ ಇಂಥವರ ಮೇಲೆಲ್ಲ ಕಣ್ಣಿಟ್ಟಿರುವಂತೆ ಪೋಲೀಸರಿಗೆ ಹೇಳಿರುತ್ತಿದ್ದೆ. ಎಷ್ಟೋ ಸಲ ಅವರನ್ನು ನಾನು ಊರಿನಿಂದ ಆಚೆಗೆ ಓಡಿಸಿದ್ದೂ ಉಂಟು.”

ಇಷ್ಟೆಲ್ಲ ಆದರೂ ಸ್ವಾಮೀಜಿ ಶಾಂತವಾಗಿಯೇ ಇದ್ದರು! ಅವರು ಆ ಮನುಷ್ಯನಿಗೆ ಹೆದರಿ ಕೊಂಡು ಸುಮ್ಮನಿದ್ದುದಲ್ಲ; ಶಿಶುಪಾಲನ ನಿಂದೆಯನ್ನೆಲ್ಲ ಶ್ರೀಕೃಷ್ಣ ತಾಳಿಕೊಂಡಂತೆ, ಸ್ವಾಮೀಜಿಯೂ ಇವನ ಬೈಗುಳವನ್ನೆಲ್ಲ ಸಹಿಸಿಕೊಂಡಿದ್ದರು ಅಷ್ಟೆ. ಆದರೆ ಇತರರಿಗೆ ಸಿಟ್ಟು ತಡೆದುಕೊಳ್ಳುವುದು ಕಷ್ಟವಾಗಿತ್ತು. ಸ್ಟರ್ಡಿ ಕುಳಿತಲ್ಲಿಂದ ದಾಪುಗಾಲು ಹಾಕುತ್ತ ಎದ್ದು ಬಂದು ಅವನನ್ನು ಗದರಿಸಿದ. ಆದರೆ ಅವನು ಹೆದರಲೇ ಇಲ್ಲ. ಸ್ಟರ್ಡಿಯ ಮಾತು ಕೇಳಿದಾಗ, ಸ್ವಾಮೀಜಿ ಬಂಗಾಳಿಗಳು ಎಂಬುದು ಆ ಮನುಷ್ಯನಿಗೆ ಇದ್ದಕ್ಕಿದ್ದಂತೆ ಹೊಳೆಯಿತು. ಈಗ ಅವನು ತನ್ನ ಧ್ವನಿಯನ್ನು ಬದಲಿಸಿಕೊಂಡು, “ಓಹೋ! ನೀವು ಮದ್ರಾಸಿಗಳೆಂದು ತಿಳಿದಿದ್ದೆ ನಾನು! ಬಂಗಾಳೀ ಬಾಬುವೇ ನೀವು? ಸಿಪಾಯಿದಂಗೆಯ ಕಾಲದಲ್ಲಿ ನಿಮ್ಮನ್ನು ಉಳಿಸಿದವರು ನಾವೇ ಎಂಬುದು ನೆನಪಿದೆ ತಾನೆ?” ಎಂದು ಸ್ವಾಮೀಜಿಯನ್ನೇ ನೇರವಾಗಿ ಕೇಳಿದ. ಸ್ಟರ್ಡಿ ವಾಪಸ್ಸು ತನ್ನ ಜಾಗಕ್ಕೆ ಹೋಗಿದ್ದವನು ಮತ್ತೆ ಓಡಿಬಂದ. ಸಿಟ್ಟಿನಿಂದ ಅವನ ಮೈಯೆಲ್ಲ ಅದರುತ್ತಿತ್ತು. “ಆದರೆ ನಿಮಗೆ (ಇಂಗ್ಲಿಷರಿಗೆ) ಸರಿಯಾಗಿ ಸಿಕ್ಕಿತಲ್ಲ ವಾಪಸ್ಸು” ಎಂದು ಸ್ಟರ್ಡಿ ಗಟ್ಟಿಯಾಗಿ ಕಿರುಚಿದ. ಆ ವ್ಯಕ್ತಿಯನ್ನು ಬಲಾತ್ಕಾರವಾಗಿ ಹೊರಗೆ ದಬ್ಬಿಬಿಡೋಣವೆ ಎಂದು ಆಲೋಚಿಸುತ್ತ ತೋಳನ್ನು ಸರಿಪಡಿಸಿಕೊಂಡ. ಅಷ್ಟು ಹೊತ್ತೂ ಸುಮ್ಮನೆ ಕುಳಿತಿದ್ದ ಗುಡ್​ವಿನ್ನನೂ ಎದ್ದುಬಂದ. ಸ್ವಾಮೀಜಿ ಏನು ಹೇಳುತ್ತಾರೋ ಎಂಬ ಒಂದೇ ಶಂಕೆಯಿಂದ ಇಬ್ಬರೂ ತಮ್ಮನ್ನು ತಡೆದು ಕೊಂಡು ಅವರತ್ತ ನೋಡಿದರು. ಮಹೇಂದ್ರನಾಥನೂ ಶಾರದಾನಂದರೂ ದಿಗ್ಭ್ರಾಂತರಾಗಿ ಕುಳಿತಿದ್ದರು. ಈ ಪರದೇಶದಲ್ಲಿ ಈಗ ಇನ್ನೇನು ಸಂಭವಿಸುವುದೋ ಎಂದು ಅವರು ಹೆದರಿದರು.

ಇದೀಗ ಸ್ವಾಮೀಜಿಯ ಮುಖಭಾವ ಬದಲಾಯಿತು. ಇಷ್ಟು ಹೊತ್ತೂ ಧ್ಯಾನಮಗ್ನ ಶಿವ ನಂತಿದ್ದ ಅವರೀಗ ತ್ರಿಪುರಾಂತಕರಾದರು! ಆ ಮನುಷ್ಯ ಕುಳಿತಿದ್ದ ದಿಕ್ಕಿಗೆ ತಿರುಗಿ ನಿಂತು ಮಾತನಾಡಲಾರಂಭಿಸಿದರು. ಆತನ ಯಾವ ನಿಂದೆಯನ್ನೂ ಲೆಕ್ಕಿಸದಿದ್ದವರು, ತಮ್ಮ ತಾಯ್ನಾಡಿನ ಬಗ್ಗೆ ಆತ ತುಚ್ಛವಾಗಿ ಮಾತನಾಡಿದಾಗ ಮಾತ್ರ ಕೆಣಕಿದ ಫಣಿಯಾದರು! ಆದರೂ ಅವರು ಅವನಿಗೆ ನೇರವಾಗಿ ಬೈಯಲು ಹೋಗಲಿಲ್ಲ; ಅಥವಾ ಅವನನ್ನು ಹೊರಗಟ್ಟುವಂತೆ ಇತರರಿಗೆ ಹೇಳಲಿಲ್ಲ. ಬದಲಾಗಿ, ಇಂಗ್ಲಿಷರ ಚರಿತ್ರೆಯನ್ನು ಪದರಪದರವಾಗಿ ಬಿಚ್ಚಿ ವಿವರಿಸಲಾರಂಭಿಸಿ ದರು. ಐದನೆಯ ಶತಮಾನದಿಂದ ಹಿಡಿದು ಇಪ್ಪತ್ತನೆಯ ಶತಮಾನದವರೆಗೂ ಜಗತ್ತಿನ ಎಲ್ಲೆಡೆಗಳಲ್ಲಿ ಇಂಗ್ಲಿಷರು ಎಸಗಿದ ಅತ್ಯಾಚಾರ-ಅನಾಚಾರಗಳನ್ನು ಕಣ್ಣಿಗೆ ಕಟ್ಟುವಂತೆ ಬಣ್ಣಿಸ ತೊಡಗಿದರು. ತಮ್ಮ ವಾದವನ್ನು ಸಮರ್ಥಿಸಲು ಅಂಕಿ ಅಂಶಗಳನ್ನು, ಇಸವಿ-ತಿಂಗಳು-ದಿನ ಗಳನ್ನು ನಿರರ್ಗಳವಾಗಿ ಉದ್ಧರಿಸುತ್ತ ಜ್ವಾಲಾಮುಖಿಯಂತೆ ಮಾತಿನ ಬೆಂಕಿಯನ್ನೇ ಚಿಮ್ಮಿದರು! ಈಗ ಆ ಟೀಕೆಗಾರನ ಬಾಯಿ ಮುಚ್ಚಿಹೋಯಿತು. ಅವನ ಬಳಿಯೀಗ ಒಂದೇ ಒಂದು ಮಾತೂ ಇರಲಿಲ್ಲ. ಆಳರಸರ ನಾಡಿನವನು ತಾನೆಂಬ ಅವನ ಅಹಂಕಾರ ಛಿದ್ರಛಿದ್ರವಾಗಿ, ಅವನು ತಲೆ ತಗ್ಗಿಸಿ ಕುಳಿತುಬಿಟ್ಟ. ಜನರೆಲ್ಲ ಸ್ವಾಮೀಜಿಯ ವಾಕ್ ಪ್ರವಾಹವನ್ನು ಕಂಡು ಸ್ತಂಭೀಭೂತ ರಾದರು. ಸ್ವಾಮೀಜಿಯ ಮಾತಿನ ಸೊಬಗೇನು! ಅವರ ಜ್ಞಾನದ ವೈಶಾಲ್ಯವೆಷ್ಟು! ಅವರ ಇತಿಹಾಸಪ್ರಜ್ಞೆಯೆಂಥದು! ತಮ್ಮ ವಾದವನ್ನು ಸಮರ್ಥಿಸಲು ಅಂಕಿ ಅಂಶಗಳನ್ನು ಬಳಸಿ ಕೊಳ್ಳುವ ಅವರ ಚಾತುರ್ಯವೆಂಥದು! ಜ್ಞಾನಯೋಗದ ಬಗ್ಗೆ ತಿಳಿಯಲು ಬಂದಿದ್ದವರಿಗೆ ಈಗ ಸ್ವಾಮೀಜಿಯ ಇನ್ನೊಂದು ಮುಖದ ದರ್ಶನವಾಗಿತ್ತು. ಇತ್ತ ಈ ಟೀಕೆಗಾರನ ಅವಸ್ಥೆ ನೋಡ ಬೇಕು! ಅಪಮಾನವನ್ನು ಸಹಿಸಲಾರದೆ ಆತ ಬಿಕ್ಕಲಾರಂಭಿಸಿದ. ಅಲ್ಲಿಂದ ಎದ್ದು ಹೋದರೆ ಇನ್ನೂ ಹೆಚ್ಚಿನ ಅಪಮಾನವಲ್ಲವೆ! ಸ್ವಾಮೀಜಿ ಮೂವತ್ತೈದು ನಿಮಿಷಗಳ ಕಾಲ ತಮ್ಮ ಮಾತಿನ ಚಾಟಿಯನ್ನು ಎಡೆಬಿಡದೆ ಬೀಸಿದರು. ಈ ನಿವೃತ್ತ ಅಧಿಕಾರಿ ಕಣ್ಣು ಮೂಗುಗಳನ್ನು ಕರವಸ್ತ್ರದಿಂದ ಒರೆಸಿಕೊಳ್ಳುತ್ತ ಕುಳಿತೇ ಇದ್ದ; ಅವನ ಮೂರು ಕರವಸ್ತ್ರಗಳು ಸಂಪೂರ್ಣ ತೊಯ್ದುಹೋದುವು!

ತಾವು ಹೇಳಬೇಕಾದುದನ್ನೆಲ್ಲ ಹೇಳಿ ಮುಗಿಸಿದ ಸ್ವಾಮೀಜಿ, ಮತ್ತೆ ಅತ್ಯಂತ ಶಾಂತ ಮುಖಮುದ್ರೆಯನ್ನು ಧರಿಸಿ ಸಭಿಕರ ಕಡೆಗೆ ತಿರುಗಿದರು. ಏನೂ ಆಗಲೇ ಇಲ್ಲವೋ ಎಂಬಂತೆ, “ಈಗ ಪ್ರತ್ಯಾಹಾರ-ಧಾರಣಗಳ ಬಗ್ಗೆ ನೋಡೋಣ” ಎನ್ನುತ್ತ ತಮ್ಮ ಮಾತನ್ನು ಮುಂದುವರಿಸಿದರು!

ಅಂದಿನ ಉಪನ್ಯಾಸ ಮುಗಿದ ಕೂಡಲೇ ಸಭಿಕರೆಲ್ಲ ಸ್ವಾಮೀಜಿಯನ್ನು ಮುತ್ತಿಕೊಂಡು ಬಿಟ್ಟರು. ಅತ್ಯಂತ ಉದ್ರೇಕಕಾರಿ ಪರಿಸ್ಥಿತಿಯನ್ನು ಅವರು ನಿಭಾಯಿಸಿದ ರೀತಿಯನ್ನು ಕಂಡು ಜನ ಆನಂದಿತರಾಗಿದ್ದರು. “ಸ್ವಾಮೀಜಿ, ಆ ರೀತಿಯಲ್ಲಿ ನಮಗೆ ಯಾರಾದರೂ ಅವಮಾನ ಮಾಡಿದ್ದಲ್ಲಿ ಖಂಡಿತವಾಗಿಯೂ ನಮಗದನ್ನು ಸಹಿಸಿಕೊಳ್ಳಲು ಸಾಧ್ಯವಾಗುತ್ತಿರಲಿಲ್ಲ. ಆದರೆ ನೀವು ನಮಗೆ ಸಹನೆಯ ದೊಡ್ಡ ಪಾಠವನ್ನು ಕಲಿಸಿದ್ದೀರಿ” “ಸ್ವಾಮೀಜಿ ನೀವೊಬ್ಬ ಸಂತರೇ ಸರಿ! ನೀವು ನಿಜಕ್ಕೂ ಮಹಾಪುರುಷರು” ಎಂದು ಜನ ಅವರನ್ನು ಕೊಂಡಾಡಿ, ತಮ್ಮ ಕೃತಜ್ಞತೆಯನ್ನು ಸಲ್ಲಿಸಿದರು. ಅಷ್ಟೇ ಅಲ್ಲ, ಆ ಟೀಕೆಗಾರನೂ ಸ್ಟರ್ಡಿಯ ಬಳಿಗೆ ಬಂದು ಕ್ಷಮೆ ಯಾಚಿಸಿ, ಅಲ್ಲಿಂದ ಬೇಗನೆ ಹೊರಟುಹೋದ.

ಸ್ವಾಮೀಜಿಯ ವಿಚಾರಧಾರೆಯಿಂದ ಆಕರ್ಷಿತರಾದ ಲಂಡನ್ನಿನಲ್ಲಿ ನೆಲಸಿದ ಭಾರತೀಯ ವಿದ್ಯಾರ್ಥಿಗಳು ಅವರ ಮಾರ್ಗದರ್ಶನವನ್ನು ಅರಸಿ ಬಂದರು. ಸ್ವಾಮೀಜಿ ಅವರನ್ನು ಅತ್ಯಂತ ಆದರದಿಂದ ಬರಮಾಡಿಕೊಂಡು ಅವರೊಂದಿಗೆ ಚಿರಪರಿಚಿತರಂತೆ ಮಾತನಾಡಿದರು; ಅವರಿಗೆ ಅನೇಕ ರೀತಿಯಲ್ಲಿ ನೆರವಾದರು. ಬ್ರಿಟನ್ ಮತ್ತು ಐರ್​ಲೆಂಡಿನಲ್ಲಿ ನೆಲಸಿದ್ದ ಭಾರತೀಯರು ಲಂಡನ್ನಿನ ಹಿಂದೂಸಂಸ್ಥೆಯ ಆಶ್ರಯದಲ್ಲಿ ಏರ್ಪಡಿಸಿದ ಸಾಮಾಜಿಕರ ಸಮ್ಮೇಳನದಲ್ಲಿ ಸ್ವಾಮೀಜಿ ಅಧ್ಯಕ್ಷಸ್ಥಾನ ವಹಿಸಿದರು. ಅಂದಿನ ಉಪನ್ಯಾಸದ ವಿಷಯ “ಹಿಂದೂಗಳು ಮತ್ತು ಅವರ ಅವಶ್ಯಕತೆಗಳು.” ಆ ಸಭೆಯಲ್ಲಿ ಹಲವಾರು ಆಂಗ್ಲ ಸ್ತ್ರೀಪುರುಷರೂ ಭಾಗವಹಿಸಿದ್ದರು. ಅಲ್ಲಿ ನೆರದಿದ್ದ ಯುವ ಹಿಂದೂಗಳಿಗೆ ಸ್ವಾಮೀಜಿ, “ನೀವು ಭಾರತಕ್ಕೆ ಹಿಂದಿರುಗಿದ ಕೂಡಲೇ ಈ ಐರೋಪ್ಯ ಉಡಿಗೆತೊಡಿಗೆಯನ್ನೂ ಐರೋಪ್ಯ ರೀತಿನೀತಿಗಳನ್ನೂ ಬದಿಗೊತ್ತಬೇಕು; ಭಾರತೀಯರೊಂದಿಗೆ ಒಂದಾಗಿ ಬೆರೆತು ಅವರನ್ನು ಮೇಲೆತ್ತುವ ಪ್ರಯತ್ನಮಾಡಬೇಕು” ಎಂದು ಕರೆನೀಡಿದರು. ಜಾತಿಭೇದದ ಪಿಡುಗನ್ನು ಸ್ವಾಮೀಜಿ ಕಟುವಾಗಿ ಖಂಡಿಸಿದರಲ್ಲದೆ ಭಾರತೀಯ ಸಮಾಜದಲ್ಲಿ ಹಿಂದೂ ಸ್ತ್ರೀಯರು ಹೊಂದಿದ್ದ ಉನ್ನತ ಸ್ಥಿತಿಯ ಕುರಿತು ವಿಚಾರಪೂರ್ಣವಾದ ಭಾಷಣ ಮಾಡಿದರು.

ಸ್ವಾಮೀಜಿ ಲಂಡನ್ ನಗರದಲ್ಲಿದ್ದಾಗ ನಡೆದ ಅವಿಸ್ಮರಣೀಯ ಘಟನೆಗಳಲ್ಲಿ ಒಂದೆಂದರೆ ಆಕ್ಸ್​ಫರ್ಡ್ ವಿಶ್ವವಿದ್ಯಾನಿಲಯದ ಪ್ರಖ್ಯಾತ ಪೌರ್ವಾತ್ಯಶಾಸ್ತ್ರವಿಶಾರದರಾಗಿದ್ದು ಮುಂದೆ ಇಂಗ್ಲೆಂಡಿನಲ್ಲಿ ನೆಲಸಿದ್ದ ಪ್ರೊ ॥ ಮಾಕ್ಸ್ ಮುಲ್ಲರರು (ಇವರ ನಿಜವಾದ ಹೆಸರು ಫ್ರೆಡ್​ರಿಕ್ ಮ್ಯಾಕ್ಸ್​ಮಿಲನ್ ಎಂದು) ಮಹಾ ವಿದ್ವಾಂಸರು, ಭಾಷಾಪಂಡಿತರು, ಜ್ಞಾನವೃದ್ಧರು. ಭಾರತೀಯ ಶಾಸ್ತ್ರಗಳ ಬಗ್ಗೆ ಅವರು ನಡೆಸಿದ ಸಂಶೋಧನೆಗಳು ಅವರನ್ನು ಜಗತ್ತಿನ ಚರಿತ್ರೆಯಲ್ಲೇ ಚಿರಸ್ಥಾಯಿಯಾಗಿಸಿವೆ. ಸ್ವಾಮೀಜಿ ಪ್ರೊ ॥ ಮುಲ್ಲರರನ್ನು ಭೇಟಿಯಾದಾಗ ಅವರಿಗಾಗಲೇ ೭೩ ವರ್ಷ ವಯಸ್ಸಾಗಿತ್ತು. ಆಗ \eng{The Sacred books of the East} ಎಂಬ ಅವರ ಮಹಾಗ್ರಂಥವು ೫೧ ಸಂಪುಟಗಳಲ್ಲಿ ಪ್ರಕಟಗೊಳ್ಳುತ್ತಿದ್ದು, ಪ್ರಕಟಣೆಯ ಕಾರ್ಯ ಮುಗಿಯುತ್ತ ಬಂದಿತ್ತು. ಅಲ್ಲದೆ ೧೮೪೬ರಲ್ಲೇ ಪುಗ್ವೇದದ ಪ್ರಕಟಣೆಗಾಗಿ ಈಸ್ಟ್ ಇಂಡಿಯಾ ಕಂಪೆನಿ ಮುಲ್ಲರರಿಗೆ ೯ ಲಕ್ಷ ರೂಪಾಯಿಗಳ ಧನಸಹಾಯವನ್ನು ನೀಡಿತ್ತು! ಈ ಕೆಲಸಕ್ಕಾಗಿ ನೂರಾರು ಭಾರತೀಯ ಪಂಡಿತರನ್ನು ನೇಮಿಸಿಕೊಳ್ಳಲಾಗಿತ್ತು. ಸ್ವತಃ ಮ್ಯಾಕ್ಸ್ ಮುಲ್ಲರರೇ ಈ ಗ್ರಂಥದ (ಮೂಲ, ಅನುವಾದ, ಭಾಷ್ಯಗಳ ಸಹಿತ) ಹಸ್ತಪ್ರತಿಗಳನ್ನು ತಯಾರು ಮಾಡಲು ೨೫ ವರ್ಷಗಳನ್ನು ಕಳೆದಿದ್ದರು. ಗ್ರಂಥದ ಪ್ರಕಟಣೆಗೆ ೨ಂ ವರ್ಷ ಬೇಕಾಯಿತು! ಇದಲ್ಲದೆ ಮುಂದೆ ಅವರು ಇತರ ಅನೇಕ ಶ್ರೇಷ್ಠ ಗ್ರಂಥಗಳನ್ನು ರಚಿಸಿದರು. ೧೮೭೫ರಲ್ಲಿ, ಮೇಲೆ ಹೇಳಿದ \eng{The Sacred Books of the East} ಗ್ರಂಥದ ಪ್ರಕಟಣೆ ಪ್ರಾರಂಭವಾಗಿತ್ತು. ಹೀಗೆ ಮುಲ್ಲರರು ಹಿಂದೂ ಧರ್ಮಕ್ಕೆ ಮಾತ್ರವಲ್ಲ, ಇಡೀ ಮಾನವತೆಗೆ ಮಹದುಪಕಾರವನ್ನು ಸಲ್ಲಿಸಿದರು.

ಆದರೆ ಪ್ರೊ ॥ ಮುಲ್ಲರರು ಕೇವಲ ಒಣ ತತ್ತ್ವಜ್ಞಾನಿಯಾಗಿರಲಿಲ್ಲ. ಜ್ಞಾನ-ಭಕ್ತಿಗಳಲ್ಲಿ ಅವರೊಬ್ಬ ಆಧುನಿಕ ಪುಷಿಯೇ ಆಗಿದ್ದರು. ಭಾರತದ ಬಗ್ಗೆಯೂ ಹಿಂದೂಧರ್ಮದ ಬಗ್ಗೆಯೂ ಅವರಿಗಿದ್ದ ಆದರ ಅಪಾರ. ಮ್ಯಾಕ್ಸ್ ಮುಲ್ಲರರ ಕುರಿತಾಗಿ ಸ್ವಾಮೀಜಿಗಿದ್ದ ಪೂಜ್ಯ-ಗೌರವ ಭಾವ ಅತ್ಯಂತ ಆಳವಾದುದು. ಆದರೆ ಎಲ್ಲಕ್ಕಿಂತ ಹೆಚ್ಚಾಗಿ, ಸ್ವಾಮೀಜಿ ಅವರಲ್ಲಿ ಕಂಡದ್ದೇ ನೆಂದರೆ ಶ್ರೀರಾಮಕೃಷ್ಣರ ಬಗೆಗಿನ ಅವರ ತಿಳಿವಳಿಕೆ, ಭಕ್ತಿ. ಶ್ರೀರಾಮಕೃಷ್ಣರ ದಾಸಾನುದಾಸ ತಾನೆಂದು ಹೇಳಿಕೊಳ್ಳಲು ಅಭಿಮಾನ ಪಡುತ್ತಿದ್ದ ಸ್ವಾಮೀಜಿ, ಇಂತಹ ಒಬ್ಬ ಮಹಾಜ್ಞಾನಿಯು ಶ್ರೀರಾಮಕೃಷ್ಣರ ಮಹಿಮೆಯನ್ನು ಗುರುತಿಸಿದ್ದುದನ್ನು ಕಂಡು ಅತ್ಯಾನಂದಗೊಂಡಿದ್ದರಲ್ಲಿ ಆಶ್ಚರ್ಯವೇನಿದೆ?

ಪ್ರೊ ॥ ಮುಲ್ಲರರ ಆಹ್ವಾನದ ಮೇರೆಗೆ ಮೇ ೨೮ರಂದು ಸ್ವಾಮೀಜಿ ಆಕ್ಸ್​ಫರ್ಡಿನ ಅವರ ಮನೆಯಲ್ಲಿ ಅವರನ್ನು ಭೇಟಿಯಾದರು. ಇದೊಂದು ತಮ್ಮ ಪಾಲಿನ ಭಾಗ್ಯವಿಶೇಷವೆಂದು ಸ್ವಾಮೀಜಿ ಬಗೆದರು. ಈ ಭೇಟಿಯ ಬಗ್ಗೆ ಅವರೇ ಬ್ರಹ್ಮವಾದಿನ್ ಪತ್ರಿಕೆಗೆ ಬರೆಯುತ್ತಾರೆ: “ಎಂತಹ ಅದ್ಭುತ-ಅಸಾಧಾರಣ ವ್ಯಕ್ತಿ ಪ್ರೊ ॥ ಮ್ಯಾಕ್ಸ್ ಮುಲ್ಲರರು! ಕೆಲದಿನಗಳ ಹಿಂದೆ ನಾನವರನ್ನು ಭೇಟಿ ಮಾಡಿದೆ. ನಾನು ಹೋದುದು, ಅವರಿಗೆ ನನ್ನ ಗೌರವವನ್ನು ಸಲ್ಲಿಸಲು ಎನ್ನ ಬೇಕು. ಏಕೆಂದರೆ, ಯಾವನು ಶ್ರೀರಾಮಕೃಷ್ಣರನ್ನು ಪ್ರೀತಿಸುವನೋ–ಅವನ ಜಾತಿ ಯಾವುದೇ ಇರಲಿ, ರಾಷ್ಟ್ರ ಯಾವುದೇ ಇರಲಿ–ಆ ವ್ಯಕ್ತಿಯಿರುವಲ್ಲಿಗೆ ಹೋಗುವುದನ್ನು ನಾನು ತೀರ್ಥ ಯಾತ್ರೆಯೆಂದೇ ಪರಿಗಣಿಸುತ್ತೇನೆ.”

ಇಲ್ಲಿ ಸ್ವಾಮೀಜಿಯ ದಿವ್ಯ ವಿನಯವನ್ನು ನಾವು ಗಮನಿಸದಿರುವಂತಿಲ್ಲ. ಶ್ರೀರಾಮಕೃಷ್ಣರ ಶ್ರೇಷ್ಠತಮ ಶಿಷ್ಯರು ಅವರು, ಜಗತ್ತಿಗೆ ಶ್ರೀರಾಮಕೃಷ್ಣರ ರಾಯಭಾರಿ ಅವರು ಎಂಬುದನ್ನು ನೆನೆಸಿಕೊಂಡಾಗ ಅವರ ಹಿರಿಮೆಯೇನೆಂಬುದು ಇಲ್ಲಿ ಸ್ಪಷ್ಟವಾಗುತ್ತದೆ. ಪ್ರೊ॥ ಮುಲ್ಲರರನ್ನು ಕೊಂಡಾಡಿ ಸ್ವಾಮೀಜಿ ಬರೆದ ಲೇಖನವು ಇಂಗ್ಲಿಷ್ ಭಾಷಾ ಸಾಹಿತ್ಯದ ದೃಷ್ಟಿಯಿಂದಲೂ ಒಂದು ಶ್ರೇಷ್ಠ ಕೃತಿಯೇ ಸರಿ. ಅದರಲ್ಲಿ ಅವರು ಬರೆಯುತ್ತಾರೆ:

“ನಿಜಕ್ಕೂ ಆ ಭೇಟಿಯು ನನ್ನ ಪಾಲಿಗೊಂದು ಹೊಸ ಅನುಭವವೆನ್ನಬೇಕು. ಆ ಸುಂದರವಾದ ಪುಟ್ಟ ಮನೆ, ಅದರ ಸುತ್ತಲಿನ ರಮ್ಯವಾದ ಉದ್ಯಾನ, ಅಲ್ಲಿ ಆ ರಜತ ಕೇಶದ ಪುಷಿ, ಅವರ ಪ್ರಶಾಂತ-ಕರುಣಾಪೂರ್ಣ ವದನ, ಎಪ್ಪತ್ತು ಶರತ್ಕಾಲಗಳು ಕಳೆದಿದ್ದರೂ ಮಗುವಿನ ಹಣೆ ಯಂತೆ ಸುಕೋಮಲವಾಗಿರುವ ಅವರ ಹಣೆ!... ಅವರ ವದನದಲ್ಲಿ ಮೂಡಿರುವ ಪ್ರತಿ ಯೊಂದು ರೇಖೆಯೂ ಅದರ ಹಿಂದೆ ಅಡಗಿರುವ ಅಗಾಧ ಆಧ್ಯಾತ್ಮಿಕ ಗಣಿಯನ್ನು ಸೂಚಿಸು ತ್ತಿತ್ತು. ಅವರ ಪತ್ನಿಯು ಉದಾತ್ತ ಚರಿತೆಯಾದ ಸಾಧ್ವೀಮಣಿ. ಪ್ರೊಫೆಸರ್ ಮುಲ್ಲರರ ಜೀವನವು ಸ್ವಾರಸ್ಯಮಯ ಹಾಗೂ ಕ್ರಮಯುತವಾದ ವಿದ್ವಜ್ಜೀವನ; ಈ ಜೀವನದಲ್ಲಿ ಅವರಿಗೆ ಎದುರಾದ ವಿರೋಧ ಹಾಗೂ ತಿರಸ್ಕಾರ ಅಪಾರ; ಇದೆಲ್ಲವನ್ನೂ ಎದುರಿಸಿ ಯಶಸ್ವಿಗಳಾಗಿ ಪುರಾತನ ಭಾರತದ ಮಹರ್ಷಿಗಳ ಬಗ್ಗೆಯೂ ಅವರ ಆಲೋಚನೆಗಳ ಬಗ್ಗೆಯೂ ಗೌರವವನ್ನು ಉಂಟುಮಾಡಿದವರು ಮುಲ್ಲರರು. ಅವರ ಈ ಸುದೀರ್ಘ ಯಶಸ್ವೀ ಜೀವನದಲ್ಲಿ ಸಹಭಾಗಿನಿ ಯಾದವರು ಅವರ ಅರ್ಧಾಂಗಿ–ಇಂತಹ ಈ ದಂಪತಿಗಳು ಹಾಗೂ ಅವರ ಮನೆಯ ಸುತ್ತಲಿನ ವೃಕ್ಷಗಳು, ಪುಷ್ಪಗಳು, ಸ್ವಚ್ಛವಾದ ನೀಲಾಕಾಶ–ಇವೆಲ್ಲ ನನ್ನನ್ನು ಸನಾತನ ಭಾರತದ ವೈಭವ ಯುತ ಕಾಲಕ್ಕೆ, ನಮ್ಮ ಬ್ರಹ್ಮರ್ಷಿ-ರಾಜರ್ಷಿಗಳ ಕಾಲಕ್ಕೆ, ಮಹಾನ್ ವಾನಪ್ರಸ್ಥಾಶ್ರಮಿಗಳ ಕಾಲಕ್ಕೆ, ಆರುಂಧತಿ-ವಸಿಷ್ಠರುಗಳ ಕಾಲಕ್ಕೆ ಕರೆದೊಯ್ದಿತು.

“ಅಲ್ಲಿ ನಾನು ಕಂಡದ್ದು ಓರ್ವ ಭಾಷಾಶಾಸ್ತ್ರಜ್ಞನನ್ನಲ್ಲ ಅಥವಾ ವಿದ್ವಾಂಸನನ್ನೂ ಅಲ್ಲ; ನಾನು ನೋಡಿದ್ದು ನಿರಂತರವೂ ಬ್ರಹ್ಮದಲ್ಲಿ ತಾನು ಒಂದಾಗಿರುವವನು ಎಂದರಿತಿರುವ ಚೇತನವನ್ನು; ಸಮಸ್ತ ವಿಶ್ವದೊಂದಿಗೆ ಒಂದಾಗಲೆಣಿಸಿ ಅನುಕ್ಷಣವೂ ವಿಕಾಸಗೊಳ್ಳುವ ಹೃದಯ ವನ್ನು. ಎಲ್ಲಿ ಇತರರು ನೀರಸವಾದ ವಿವರಗಳ ಮರಳುಗಾಡಿನಲ್ಲಿ ದಾರಿ ತಪ್ಪುತ್ತಿರುವರೋ ಅಲ್ಲಿ ಇವರು ಚೈತನ್ಯದ ಚಿಲುಮೆಯನ್ನು ಮುಟ್ಟಿದ್ದಾರೆ. ‘ಆತ್ಮವನ್ನು ಮಾತ್ರ ಅರಿ, ಉಳಿದ ಮಾತು ಗಳನ್ನೆಲ್ಲ ತ್ಯಜಿಸು’ ಎಂಬ ಉಪನಿಷತ್ತಿನ ಸಂದೇಶದೊಂದಿಗೆ ಅವರ ಹೃದಯ ಸ್ಪಂದಿಸುತ್ತಿದೆ.

“ಪ್ರೊ ॥ ಮುಲ್ಲರರು ಪ್ರಪಂಚದಲ್ಲೆಲ್ಲ ಪ್ರಖ್ಯಾತರಾದ ವಿದ್ವಾಂಸರಾಗಿದ್ದರೂ ದಾರ್ಶನಿಕ ರಾಗಿದ್ದರೂ ಅವರ ವಿದ್ವತ್ತು ಮತ್ತು ದರ್ಶನ ಅವರನ್ನು ಆತ್ಮಸಾಕ್ಷಾತ್ಕಾರದ ಶಿಖರದೆಡೆಗೆ ಒಯ್ದಿವೆ.

“ಭಾರತದ ಮೇಲೆ ಅವರಿಗಿರುವ ಪ್ರೀತಿಯಾದರೂ ಎಂಥದು! ಆ ಪ್ರೀತಿಯ ನೂರನೇ ಒಂದು ಭಾಗವಾದರೂ ನನ್ನ ಸ್ವಂತ ತಾಯ್ನಾಡಿನ ಮೇಲೆ ನನಗಿದ್ದರೆ ನನ್ನನ್ನು ನಾನು ಧನ್ಯನೆಂದುಕೊಳ್ಳುತ್ತೇನೆ. ಅಸಾಧಾರಣವೂ ತೀವ್ರವಾಗಿ ಕ್ರಿಯಾಶೀಲವೂ ಆದ ಮನಸ್ಸಿನಿಂದ ಅವರು ಐವತ್ತು ವರ್ಷಗಳಿಗೂ ಹೆಚ್ಚುಕಾಲ ಭಾರತೀಯ ವಿಚಾರಧಾರೆಯ ಜಗತ್ತಿನಲ್ಲಿ ಜೀವಿಸಿ ದ್ದಾರೆ, ವಿಹರಿಸಿದ್ದಾರೆ. ಮತ್ತು ಸಂಸ್ಕೃತ ಸಾಹಿತ್ಯದ ಮಹಾರಣ್ಯದಲ್ಲಿ ಬೆಳಕು-ಕತ್ತಲೆಗಳ ಸೂಕ್ಷ್ಮ ಬದಲಾವಣೆಗಳನ್ನು ಅವರು ಅಗಾಧ ಆಸಕ್ತಿಯಿಂದ ಮತ್ತು ಹೃತ್ಪೂರ್ವಕ ಪ್ರೀತಿಯಿಂದ, ಅವು ತಮ್ಮ ಜೀವದಾಳಕ್ಕಿಳಿದು ತಮ್ಮ ವ್ಯಕ್ತಿತ್ವವನ್ನೇ ರಂಜಿತಗೊಳಿಸುವವರೆಗೂ ಗಮನಿಸಿದ್ದಾರೆ.

“ಮ್ಯಾಕ್ಸ್ ಮುಲ್ಲರರು ವೇದಾಂತಿಗಳಲ್ಲಿ ವೇದಾಂತಿಗಳು. ಅವರು ನಿಜಕ್ಕೂ ವೇದಾಂತದ ಜೀವನಾಡಿಯನ್ನು, ಜಗತ್ತಿನ ಎಲ್ಲ ಮತ-ಪಂಥಗಳಿಗೂ ಬೆಳಕು ಬೀರುವ ಏಕೈಕ ಬೆಳಕಾದ ವೇದಾಂತವನ್ನು, ಎಲ್ಲ ಧರ್ಮಗಳೂ ಯಾವ ಏಕೈಕ ತತ್ತ್ವದ ಅನ್ವಯಗಳು ಮಾತ್ರವೋ ಆ ವೇದಾಂತದ ಜೀವನಾಡಿಯನ್ನು ಅದರ ಎಲ್ಲ ಒಡಕುಗಳ ಹಾಗೂ ಸಾಮರಸ್ಯಗಳ ಹಿನ್ನೆಲೆಯಲ್ಲಿ ಕಂಡುಹಿಡಿದ್ದಾರೆ. ಅಲ್ಲದೆ ಶ್ರೀರಾಮಕೃಷ್ಣರಾದರೂ ಯಾರು? ಈ ಸನಾತನ ತತ್ತ್ವದ ಸಾಕ್ಷಾತ್ ನಿದರ್ಶನವೇ ಅವರು; ಸನಾತನ ಭಾರತದ ಮೂರ್ತರೂಪವೇ ಅವರು. ಅಲ್ಲದೆ ಮುಂದೆ ಭಾರತವು ಏನಾಗಲಿರುವುದೋ ಅದರ ಮುನ್ಸೂಚನೆಯೇ ಅವರು; ಮತ್ತು ಜಗತ್ತಿನ ಎಲ್ಲ ರಾಷ್ಟ್ರಗಳಿಗೆ ಆಧ್ಯಾತ್ಮಿಕತೆಯ ದೀಪಧಾರಿಯೇ ಅವರು. ರತ್ನಗಳ ಮೌಲ್ಯವನ್ನು ರತ್ನವ್ಯಾಪಾರಿ ಮಾತ್ರವೇ ಅರಿತುಕೊಳ್ಳಬಲ್ಲ–ಇದೊಂದು ಹಳೆಯ ಗಾದೆ. ಈ ಪಾಶ್ಚಾತ್ಯ ಪುಷಿಯು ಭಾರತೀಯ ವಿಚಾರಧಾರೆಯ ಕ್ಷಿತಿಜದಲ್ಲುದಿಸುವ ಪ್ರತಿಯೊಂದು ಹೊಸ ತಾರೆಯನ್ನು–ಸ್ವಯಂ ಭಾರತೀಯರು ಅದರ ಘನತೆಯನ್ನರಿಯುವ ಮೊದಲೇ–ತಾವು ಅಧ್ಯಯನ ಮಾಡಿ ಆಸ್ವಾದಿಸು ವರೆಂದರೆ ಅದರಲ್ಲಿ ಅಚ್ಚರಿಯೇನಿದೆ?

“ಪ್ರೊ ॥ ಮಾಕ್ಸ್ ಮುಲ್ಲರರು ನನ್ನಲ್ಲಿ ವಿಚಾರಿಸಿದ ಮೊದಲ ಸಂಗತಿಯೆಂದರೆ ಶ್ರೀರಾಮ ಕೃಷ್ಣರ ಬಗ್ಗೆ. ಬ್ರಾಹ್ಮಸಮಾಜದ ಮುಖಂಡನಾದ ದಿ. ಕೇಶವಚಂದ್ರಸೇನನಲ್ಲಿ ಇದ್ದಕ್ಕಿದ್ದಂತೆ ಆದ ಗಮನಾರ್ಹ ಬದಲಾವಣೆಯನ್ನು ಉಂಟುಮಾಡಿದವರು ಶ್ರೀರಾಮಕೃಷ್ಣರು ಎನ್ನುವುದು ತಿಳಿದುಬಂದಾಗಿನಿಂದಲೂ ಅವರು ಶ್ರೀರಾಮಕೃಷ್ಣರ ಜೀವನ ಮತ್ತು ಸಂದೇಶಗಳ ಶ್ರದ್ಧಾವಂತ ವಿದ್ಯಾರ್ಥಿಯಾಗಿದ್ದಾರೆ. ಅವರೊಡನೆ ನಾನು, ‘ಶ್ರೀರಾಮಕೃಷ್ಣರನ್ನು ಇಂದು ಸಾವಿರಾರು ಮಂದಿ ಪೂಜಿಸುತ್ತಿದ್ದಾರೆ, ಪ್ರೊಫೆಸರ್​’ ಎಂದಾಗ ಅವರೆಂದರು, ‘ಇಂಥವರಿಗಲ್ಲದೆ ಮತ್ತಾರಿಗೆ ಪೂಜೆ ಸಲ್ಲಬೇಕು?’ ಪ್ರೊಫೆಸರರು ನನ್ನನ್ನು ಮತ್ತು ಸ್ಟರ್ಡಿಯನ್ನು ಊಟಕ್ಕೆ ಕರೆದರು. ಅವರು ನಮಗೆ ಆಕ್ಸ್​ಫರ್ಡಿನ ಹಲವು ಕಾಲೇಜುಗಳನ್ನು ಹಾಗೂ ಗ್ರಂಥಾಲಯವನ್ನು ತೋರಿಸಿದರು. ಬಳಿಕ ರೈಲು ನಿಲ್ದಾಣದವರೆಗೂ ನಮ್ಮೊಡನೆ ಬಂದರು. ಅವರು ದಯೆಯೇ ಮೂರ್ತಿವೆತ್ತಂತಿದ್ದರು. ಅವರು ನಮಗಾಗಿ ಇಷ್ಟೆಲ್ಲ ಮಾಡಿದ್ದಕ್ಕೆ ಅವರು ಕೊಟ್ಟ ಕಾರಣ ಇದು: ‘ಶ್ರೀರಾಮಕೃಷ್ಣರ ಶಿಷ್ಯರೊಬ್ಬ ರನ್ನು ಪ್ರತಿ ದಿನವೂ ನೋಡಲು ಸಾಧ್ಯವಿಲ್ಲವಲ್ಲ!’

“ನಾನವರಿಗೆ ಹೇಳಿದೆ: ‘ನೀವು ಭಾರತಕ್ಕೆ ಯಾವಾಗ ಬರುವಿರಿ? ಸನಾತನ ಭಾರತೀಯ ವಿಚಾರಧಾರೆಯ ನಿಜ ಸ್ವರೂಪವನ್ನು ಎತ್ತಿ ಹಿಡಿಯುವುದಕ್ಕಾಗಿ ಅಷ್ಟೊಂದು ಶ್ರಮಿಸಿದವರಾದ ನಿಮ್ಮನ್ನು ಅಲ್ಲಿನ ಪ್ರತಿಯೊಂದು ಹೃದಯವೂ ಸ್ವಾಗತಿಸುತ್ತದೆ.’ ಆ ವೃದ್ಧ ಮಹರ್ಷಿಯ ಮುಖ ಬೆಳಗಿತು; ಅವರ ಕಣ್ಣಂಚಿನಲ್ಲಿ ಒಂದು ಹನಿ ಮಿನುಗಿತು; ಮೆಲ್ಲನೆ ತಲೆದೂಗುತ್ತ ಅವರು ನಿಧಾನವಾಗಿ ಉತ್ತರಿಸಿದರು, ‘ಒಂದು ವೇಳೆ ನಾನಲ್ಲಿಗೆ ಬಂದರೆ ನಾನು ಹಿಂದಿರುಗಲಾರೆ. ನೀವು ನನ್ನನ್ನು ಅಲ್ಲಿಯೇ ಸಮಾಧಿ ಮಾಡಬೇಕಾಗುತ್ತದೆ.’ ಅವರನ್ನು ಮತ್ತೆ ಪ್ರಶ್ನಿಸುವುದೆಂದರೆ, ಅವರ ಪವಿತ್ರತಮ ಭಾವನೆಗಳು ಹುದುಗಿರುವ ಸ್ಥಳಕ್ಕೆ ಅಕ್ರಮ ಪ್ರವೇಶ ಮಾಡಿದಂತಾಗುತ್ತದೆ ಎಂದು ನನಗೆ ತೋರಿತು.”

ಹೀಗೆ ಸ್ವಾಮೀಜಿ, “ಬ್ರಹ್ಮವಾದಿನ್​”ಗೆ ಕಳಿಸಿಕೊಟ್ಟ ಲೇಖನದಲ್ಲಿ ಮ್ಯಾಕ್ಸ್ ಮುಲ್ಲರರನ್ನು ಕೊಂಡಾಡಿದರು. ಅವರನ್ನು ಸ್ವಾಮೀಜಿ ಭೇಟಿಯಾಗುವ ವೇಳೆಗೆ ಅವರು ಶ್ರೀರಾಮಕೃಷ್ಣರ ಕುರಿತಾಗಿ ‘ನಿಜ ಮಹಾತ್ಮ’ ಎಂಬ ಲೇಖನವನ್ನು ‘ನೈನ್​ಟೀನ್ತ್ ಸೆಂಚುರಿ’ ಎಂಬ ಮಾಸಪತ್ರಿಕೆಗೆ ಕಳಿಸಿಕೊಟ್ಟಿದ್ದರು. ಇದು ಆ ವರ್ಷದ ಆಗಸ್ಟ್ ಸಂಚಿಕೆಯಲ್ಲಿ ಪ್ರಕಟವಾಯಿತು. ಸ್ವಾಮೀಜಿಯ ಸ್ಫೂರ್ತಿಯುತ ಮಾತುಗಳನ್ನು ಕೇಳಿ ಉತ್ಸಾಹಿತರಾದ ಮ್ಯಾಕ್ಸ್ ಮುಲ್ಲರರು ಶ್ರೀರಾಮಕೃಷ್ಣರ ಬಗ್ಗೆ ಇನ್ನೂ ಹೆಚ್ಚು ತಿಳಿಯುವ ಆಸಕ್ತಿ ತೋರಿದರು. ಅಲ್ಲದೆ ಅವರು “ಆ ಮಹಾತ್ಮನನ್ನು ಜಗತ್ತಿಗೆ ಪರಿಚಯಿಸುವ ಸಲುವಾಗಿ ನೀವು ಏನು ಮಾಡಲಿದ್ದೀರಿ?” ಎಂದು ಸ್ವಾಮೀಜಿಯನ್ನು ಕೇಳಿದರು. ಅಷ್ಟೇ ಅಲ್ಲ, ಶ್ರೀರಾಮಕೃಷ್ಣರ ಸಂಬಂಧವಾಗಿ ಇನ್ನೂ ಹೆಚ್ಚಿನ ವಿವರಗಳೇನಾದರೂ ದೊರೆಯುವಂತಿದ್ದಲ್ಲಿ ಅವರ ಜೀವನ-ಸಂದೇಶಗಳ ಕುರಿತಾಗಿ ಹೆಚ್ಚು ವಿವರಪೂರ್ಣವಾದ ಪ್ರಬಂಧವೊಂದನ್ನು ರಚಿಸಲು ಅವರು ಆಸಕ್ತಿ ತೋರಿದರು. ತಕ್ಷಣವೇ ಸ್ವಾಮೀಜಿ, ಶ್ರೀರಾಮ ಕೃಷ್ಣರ ಜೀವನದ ಸಮಸ್ತ ಘಟನಾವಳಿಗಳನ್ನು ಮತ್ತು ಅವರ ಎಲ್ಲ ಸಂದೇಶಗಳನ್ನು ಸಂಗ್ರಹಿ ಸುವುದಕ್ಕಾಗಿ ಭಾರತದೊಡನೆ ಸಂಪರ್ಕ ಬೆಳೆಸಲು ಸ್ವಾಮಿ ಶಾರದಾನಂದರನ್ನು ನೇಮಿಸಿದರು. ಶಾರದಾನಂದರು ಈ ಕಾರ್ಯವನ್ನು ಯಶಸ್ವಿಯಾಗಿ ಮಾಡಿದರು. ಬಳಿಕ ಮ್ಯಾಕ್ಸ್ ಮುಲ್ಲರರು, ಅವರು ಒದಗಿಸಿದ ಮಾಹಿತಿಗಳ ಆಧಾರದ ಮೇಲೆ ಗ್ರಂಥವೊಂದನ್ನು ರಚಿಸಲು ಕುಳಿತರು. ಆ ಗ್ರಂಥವೇ \eng{\textit{Sri Ramakrishna: His Life and Teachings.”} (}ಶ್ರೀರಾಮಕೃಷ್ಣ:ಅವರ ಜೀವನ ಮತ್ತು ಬೋಧನೆಗಳು.)

ಈ ಕೃತಿಯಲ್ಲಿ ಶ್ರೀರಾಮಕೃಷ್ಣರ ಜೀವನಕ್ಕಿಂತ ಅವರ ಬೋಧನೆಗಳ ಬಗ್ಗೆ ಹೆಚ್ಚು ವಿವರವಾಗಿ ಪ್ರಸ್ತಾಪಿಸಲಾಗಿದೆ. ಇದು ಭಾವನಾತ್ಮಕವಾಗಿಯೂ ಭಕ್ತಿಪ್ರದವಾಗಿಯೂ ಇದೆಯಾದರೂ ಗಂಭೀರ ವಿಮರ್ಶಾತ್ಮಕ ವಿವರಗಳಿಂದಲೂ ಕೂಡಿದೆ. ಸ್ವಾಮೀಜಿಯ ಕಾರ್ಯಕ್ಕೆ ಇದು ಸಾಕಷ್ಟು ಸಹಾಯಕಾರಿಯಾಯಿತು. ಪರಸ್ಪರರ ಭೇಟಿಯ ಅನಂತರ ಸ್ವಾಮೀಜಿ ಹಾಗೂ ಮ್ಯಾಕ್ಸ್​ಮುಲ್ಲರ್ ಅತ್ಯಂತ ಆತ್ಮೀಯ ಸ್ನೇಹಿತರಾದರು. ಆಗಾಗ ಅವರಿಬ್ಬರ ನಡುವೆ ಪತ್ರವ್ಯವಹಾರವೂ ನಡೆಯು ತ್ತಿತ್ತು. ಸ್ವಾಮೀಜಿಗೆ ಮುಲ್ಲರರ ಮೇಲೆ ಅತಿ ಹೆಚ್ಚಿನ ಆದರಾಭಿಮಾನವಿತ್ತು.

ಲಂಡನ್ನಿನಲ್ಲಿ ವೇದಾಂತ ಪ್ರಚಾರಕಾರ್ಯವು ಯಶಸ್ವಿಯಾಗಿ ನಿಶ್ಚಿತ ಗತಿಯಲ್ಲಿ ಮುನ್ನಡೆ ದಂತೆಲ್ಲ ಅಲ್ಲೇ ಒಬ್ಬರು ಸಂನ್ಯಾಸಿಗಳು ಖಾಯಂ ಆಗಿ ನೆಲಸಬೇಕಾದ ಆವಶ್ಯಕತೆಯನ್ನು ಸ್ವಾಮೀಜಿ ಕಂಡುಕೊಂಡರು. ಆದರೆ ಅಮೆರಿಕದ ಕಾರ್ಯವನ್ನೂ ಕಡೆಗಣಿಸುವಂತಿರಲಿಲ್ಲ. ಅಲ್ಲೂ ಒಬ್ಬರು ಸಂನ್ಯಾಸಿಗಳು ನೆಲೆ ನಿಂತು ಮಾರ್ಗದರ್ಶನ-ಪ್ರೋತ್ಸಾಹಗಳನ್ನು ನೀಡಬೇಕಾದ ಆವಶ್ಯಕತೆಯಿತ್ತು. ಆದ್ದರಿಂದ ಸ್ವಾಮಿ ಶಾರದಾನಂದರನ್ನು ಅಮೆರಿಕೆಗೆ ಕಳಿಸಲು ಮತ್ತು ಸ್ವಾಮಿ ಅಭೇದಾನಂದರನ್ನು ಇಂಗ್ಲೆಂಡಿಗೆ ಕರೆಸಿಕೊಳ್ಳಲು ಸ್ವಾಮೀಜಿ ನಿಶ್ಚಯಿಸಿದರು. ಜೂನ್ ೨೪ರಂದು ಸ್ವಾಮಿ ರಾಮಕೃಷ್ಣಾನಂದರಿಗೆ ಅವರೊಂದು ಪತ್ರವನ್ನು ಬರೆದರು:

“ನಾಳೆ ಶರತ್ (ಸ್ವಾಮಿ ಶಾರದಾನಂದರು) ಅಮೆರಿಕೆಗೆ ಹೊರಡುತ್ತಾನೆ. ಇಲ್ಲಿನ ಕಾರ್ಯ ಒಂದು ಮುಖ್ಯ ಘಟ್ಟವನ್ನು ತಲಪುತ್ತಿದೆ. ಲಂಡನ್ನಿನಲ್ಲಿ ಒಂದು ಕೇಂದ್ರವನ್ನು ತೆರೆಯಲು ಹಣಕಾಸಿನ ಸೌಲಭ್ಯ ದೊರೆತಿದೆ. ಮುಂದಿನ ತಿಂಗಳು ನಾನು ಸ್ವಿಟ್ಸರ್​ಲ್ಯಾಂಡಿಗೆ ತೆರಳಿ ಅಲ್ಲಿ ಒಂದೆರಡು ತಿಂಗಳು ಕಳೆದು ಅನಂತರ ಲಂಡನ್ನಿಗೆ ಹಿಂದಿರುಗುತ್ತೇನೆ. (ಏಕೆಂದರೆ) ಈಗ ನಾನು ಭಾರತಕ್ಕೆ ಮರಳುವುದರಿಂದೇನು ಪ್ರಯೋಜನ? ಈ ಲಂಡನ್ ನಗರವು ಇಡೀ ಪ್ರಪಂಚದ ಕೇಂದ್ರಬಿಂದು. ಭಾರತದ ಹೃದಯವೇ ಇಲ್ಲಿದೆ. ಇಲ್ಲಿ ಒಂದು ಭದ್ರ ಬುನಾದಿಯನ್ನು ಹಾಕದೆ ಇಲ್ಲಿಂದ ತೆರಳುವುದಾದರೂ ಹೇಗೆ? ಸದ್ಯಕ್ಕೆ ಕಾಳಿ (ಸ್ವಾಮಿ ಅಭೇದಾನಂದರು) ಇಲ್ಲಿಗೆ ಬರಲಿ. ಆತನಿಗೆ ಸಿದ್ಧವಾಗಿರಲು ಹೇಳು. ಅವನಿಗೊಂದು ಪತ್ರ ಬರೆಯುತ್ತೇನೆ. ಅದು ತಲುಪಿದ ಕೂಡಲೇ ಅವನು ಹೊರಡಲಿ. ಶಕ್ತಿಯಿರುವುದು ಸಂಘಟನೆಯಲ್ಲಿ ಮತ್ತು ವಿಧೇಯತೆಯೇ ಅದರ ರಹಸ್ಯ.”

ಲಂಡನ್ನಿನಲ್ಲಿದ್ದಾಗ ಕೆಲವೊಮ್ಮೆ ಸ್ವಾಮೀಜಿ ಅತ್ಯುನ್ನತ ಆಧ್ಯಾತ್ಮಿಕ ಭಾವಾವಸ್ಥೆಯಲ್ಲಿರು ತ್ತಿದ್ದರು. ಅವರ ಇಡೀ ವ್ಯಕ್ತಿತ್ವವೇ ತೇಜಃಪುಂಜವಾಗಿರುತ್ತಿತ್ತು, ಭಾವಪೂರ್ಣವಾಗಿರುತ್ತಿತ್ತು; ಪ್ರತಿಯೊಬ್ಬರ ಮೇಲೂ ಅವರು ಅನಂತ ಪ್ರೇಮ ಹಾಗೂ ಅನಂತ ಕರುಣೆಯ ಸ್ರೋತವನ್ನೇ ಹರಿಸುತ್ತಿದ್ದರು. ಅವರ ಈ ದಿವ್ಯ ಭಾವಗಳು, ಅವರು ತಮ್ಮ ಮಿತ್ರರಾದ ಫ್ರಾನ್ಸಿಸ್ ಲೆಗೆಟ್ಟರಿಗೆ ಬರೆದ ಪತ್ರದಲ್ಲಿ ವ್ಯಕ್ತವಾಗಿವೆ: “ಈಗೀಗ ನಾನು ಈ ಗರ್ವಿಷ್ಠರಾದ ‘ಆಂಗ್ಲೋ ಇಂಡಿಯನ್ನ’ ರಲ್ಲಿಯೂ ಅದೇ ಭಗವಂತನನ್ನು ಕಾಣಲು ತೊಡಗಿದ್ದೇನೆ ಎನಿಸುತ್ತದೆ. ನಾನು ನಿಧಾನವಾಗಿ ಯಾವ ಸ್ಥಿತಿಗೆ ತಲುಪುತ್ತಿದ್ದೇನೆಂದರೆ ಒಂದು ದೆವ್ವವನ್ನು ಕೂಡ–ಅದಿರುವುದೇ ಹೌದಾದಲ್ಲಿ ಅದನ್ನೂ–ಪ್ರೀತಿಸುವ ಸ್ಥಿತಿಗೆ.

“ಇಪ್ಪತ್ತನೆಯ ವಯಸ್ಸಿನಲ್ಲಿ ನಾನು ಅತ್ಯಂತ ನಿಷ್ಠುರನಾಗಿದ್ದೆ; ಯಾವುದರೊಂದಿಗೂ ರಾಜಿ ಮಾಡಿಕೊಳ್ಳದಷ್ಟು ಕಟುವಾಗಿದ್ದೆ. ನಾನು ಕಲ್ಕತ್ತದ ಬೀದಿಗಳಲ್ಲಿ ನಾಟಕ ಮಂದಿರಗಳ ಪಕ್ಕದ ಕಾಲುದಾರಿಯಲ್ಲೂ ನಡೆಯುತ್ತಿರಲಿಲ್ಲ. ಈಗ ನನ್ನ ಮೂವತ್ತಮೂರನೆಯ ವಯಸ್ಸಿನಲ್ಲಿ ನಾನು ವೇಶ್ಯೆಯರೊಂದಿಗೆ ಒಂದೇ ಮನೆಯಲ್ಲಿ ವಾಸವಾಗಿರಬಲ್ಲೆ–ಅವರ ಕುರಿತಾಗಿ ಒಂದಿನಿತೂ ಜುಗುಪ್ಸೆ ಅಥವಾ ಅಸಹನೆ ತಾಳದೆ. ಇದೇನು ಅಧಃಪತನವೆ? ಅಥವಾ ಸಾಕ್ಷಾತ್ ಭಗವಂತನೇ ತಾನಾಗಿರುವ ವಿಶ್ವಪ್ರೇಮವು ನನ್ನೆದೆಯನ್ನು ತುಂಬುತ್ತಿದೆಯೆ? ಅಲ್ಲದೆ ನಾನು ಕೇಳಿದ್ದೇನೆ –ವ್ಯಕ್ತಿಯೊಬ್ಬ ತನ್ನ ಸುತ್ತಲೂ ಕೆಡುಕನ್ನು ಕಾಣದಿದ್ದರೆ ಆತ ಒಳ್ಳೆಯ ಕೆಲಸವನ್ನು ಮಾಡಲಾರ, ಆತ ಒಂದು ಬಗೆಯ ನಿರಾಶಾಭಾವಕ್ಕೆ ಗುರಿಯಾಗಿಬಿಡುತ್ತಾನೆ ಎಂದು. ಆದರೆ ನನಗೇನೂ ಹಾಗೆನ್ನಿಸುವುದಿಲ್ಲ. ತದ್ವಿರುದ್ಧವಾಗಿ, ನನ್ನ ಕಾರ್ಯಶಕ್ತಿ ಅಗಾಧವಾಗಿ ವೃದ್ಧಿಗೊಳ್ಳುತ್ತಿದೆ. ಮತ್ತು ಅಗಾಧವಾಗಿ ಪರಿಣಾಮಕಾರಿಯಾಗುತ್ತಿದೆ. ಕೆಲವೊಮ್ಮೆ ನಾನೊಂದು ಬಗೆಯ ಭಾವಾವಸ್ಥೆಗೆ ತಲುಪಿಬಿಡುತ್ತೇನೆ. ಆಗ ನಾನು ಪ್ರತಿಯೊಬ್ಬರಿಗೂ–ಎಲ್ಲರಿಗೂ, ಎಲ್ಲಕ್ಕೂ–ಶುಭಕೋರ ಬೇಕು, ಎಲ್ಲವನ್ನೂ ಒಪ್ಪಬೇಕು, ಅಪ್ಪಬೇಕು, ಎಂದು ಭಾವಿಸತೊಡಗುತ್ತೇನೆ. ಈ ‘ಕೆಟ್ಟದ್ದೆ’ನ್ನು ವುದೆಲ್ಲ ಕೇವಲ ಭ್ರಾಂತಿಯೆಂಬುದಾಗಿ ನಾನು ಕಾಣುತ್ತಿದ್ದೇನೆ. ಪ್ರಿಯ ಫ್ರಾನ್ಸಿಸ್, ನಾನೀಗ ಇಂತಹ ಮನಸ್ಥಿತಿಯೊಂದರಲ್ಲಿದ್ದೇನೆ. ನೀನು ಹಾಗೂ ನಿನ್ನ ಪತ್ನಿ ನನ್ನ ವಿಷಯದಲ್ಲಿ ಹೊಂದಿ ರುವ ಪ್ರೀತಿ ವಿಶ್ವಾಸಗಳನ್ನು ಸ್ಮರಿಸುತ್ತ ನಾನು ಅಕ್ಷರಶಃ ಆನಂದದ ಕಣ್ಣೀರನ್ನು ಸುರಿಸುತ್ತಿದ್ದೇನೆ. ಧನ್ಯ ನನ್ನ ಜನ್ಮದಿನ! ನನಗಿಲ್ಲಿ ಬಹಳಷ್ಟು ಪ್ರೀತಿವಿಶ್ವಾಸಗಳು ದೊರಕಿವೆ. ಮತ್ತು ನನ್ನನ್ನು ಅಸ್ತಿತ್ವಕ್ಕೆ ತಂದ ಆ ಪ್ರೇಮಸ್ವರೂಪನಾದ ಭಗವಂತ ನನ್ನ ಪ್ರತಿಯೊಂದು ಕಾರ್ಯವನ್ನೂ– ಅದು ಒಳ್ಳೆಯದಿರಲಿ ಕೆಟ್ಟದ್ದಿರಲಿ–(ಹೆದರಬೇಡ!) ಕಾಪಾಡಿದ್ದಾನೆ. ಏಕೆಂದರೆ ನಾನು ಈಗ, ಮತ್ತು ಎಂದೆಂದಿಗೂ, ಕೇವಲ ಅವನ ಕೈಯಲ್ಲಿನ ಉಪಕರಣವೇ ಅಲ್ಲವೆ? ಅವನ ಸೇವೆಗಾಗಿ ನನ್ನ ಸರ್ವಸ್ವವನ್ನು, ನನ್ನ ಬಂಧುಬಳಗವನ್ನು, ನನ್ನ ಸುಖ ಸಂತೋಷಗಳನ್ನು, ನನ್ನ ಜೀವನವನ್ನು ತ್ಯಾಗ ಮಾಡಲಿಲ್ಲವೆ! ಅವನು ನನ್ನ ನೆಚ್ಚಿನ ಆಟದ ಸಂಗಾತಿ, ನಾನವನ ಒಡನಾಡಿ. ಈ ವಿಶ್ವದಲ್ಲಿ ಕಾರಣ-ಹೂರಣಗಳೊಂದೂ ಇಲ್ಲ! ಏಕೆಂದರೆ ಯಾವ ನಿಯಮ ತಾನೆ ಅವನನ್ನು ಕಟ್ಟಬಲ್ಲುದು? ಲೀಲಾಮಯನಾದ ಆತ ಈ ಅಳು-ನಗುವಿನ ಆಟವಾಡುತ್ತಿದ್ದಾನೆ. ‘ಜೋ’ (ಮಿಸ್ ಮೆಕ್​ಲಾಡ್​) ಹೇಳುವಂತೆ ಇದು ಭಾರೀ ತಮಾಷೆ!

“ಇದೊಂದು ತಮಾಷೆಯ ಪ್ರಪಂಚ. ಈ ಪ್ರಪಂಚದಲ್ಲಿ ನಮಗೆ ಕಾಣಸಿಗುವ ಅತ್ಯಂತ ತಮಾಷೆಯ ಆಸಾಮಿಯೆಂದರೆ ಆತ್ಮ–ಅನಂತನಾದ ಭಗವಂತ! ನೆಚ್ಚಿನ ಭಗವಂತ! ಇದು ತಮಾಷೆಯಲ್ಲವೆ? ಆಟದ ಮೈದಾನದಲ್ಲಿ ಆಟವಾಡಲು ಹೊರಬಿಟ್ಟ ಗಲಾಟೆಯ ಮಕ್ಕಳ ಶಾಲೆ ಈ ಜಗತ್ತು! ಹೌದು ತಾನೆ! ಯಾರನ್ನು ಹೊಗಳುವುದು, ಯಾರನ್ನು ತೆಗಳುವುದು? ಎಲ್ಲ ಅವನ ಆಟ! ಜನರಿಗೆ ಇದೆಲ್ಲ ಹೀಗೇಕೆ ಎಂಬ ವಿವರಣೆ ಬೇಕಂತೆ. ಆದರೆ ಆ ಭಗವಂತನನ್ನು ಹೇಗೆ ವಿವರಿಸಲಾದೀತು? ಅವನು ಮಿದುಳಿಲ್ಲದವನು. ಅವನಿಗೆ ಕಾರಣ ಗೀರಣ ಏನೂ ಇಲ್ಲ. ನಮಗೆ ಅಲ್ಪಸ್ವಲ್ಪ ಮಿದುಳನ್ನೂ ಬುದ್ಧಿಯನ್ನೂ ಕೊಟ್ಟು ನಮ್ಮನ್ನು ಮರುಳುಗೊಳಿಸುತ್ತಿದ್ದಾನೆ. ಆದರೆ ಈ ಸಲ ನಾನು ಮೋಸಹೋಗಲಾರೆ.

“ನಾನು ಈ ಒಂದೆರಡು ವಿಷಯವನ್ನು ಮಾತ್ರ ಕಲಿತಿದ್ದೇನೆ: ‘ಪ್ರೇಮ’, ‘ಪ್ರಿಯತಮ’–ಈ ಭಾವನೆಗಳು ವಿಚಾರ, ಪಾಂಡಿತ್ಯ ಹಾಗೂ ತರ್ಕಕ್ಕೆ ಅತೀತವಾದದ್ದು. ಆ ಭಗವದಾನಂದಸುಧೆ ಯನ್ನು ಸವಿದು ಹುಚ್ಚರಾಗೋಣ.

\begin{flushright}
ಹುಚ್ಚುತನದಲ್ಲಿ ಸದಾ ನಿನ್ನವ,\\ವಿವೇಕಾನಂದ”
\end{flushright}

ಇದು ಸ್ವಾಮೀಜಿಯ ಹಲವಾರು ಮುಖಗಳಲ್ಲಿ ಒಂದು ಅಷ್ಟೆ. ಅವರು ಯಾವ ಕ್ಷಣದಲ್ಲಿ ಯಾರಿಗೆ ಹೇಗೆ ತೋರುತ್ತಾರೆಂದು ಹೇಳಲು ಸಾಧ್ಯವಿರಲಿಲ್ಲ. ಅವರ ಮಾತುಗಳೂ ಅಷ್ಟೆ; ಯಾರಿಗೆ ಯಾವ ರೀತಿಯ ಬೋಧನೆ ನೀಡುತ್ತಾರೆಂದು ಊಹಿಸುವುದು ಅಸಾಧ್ಯವಾಗಿತ್ತು. ಆದರೆ ಅವರ ಪ್ರತಿಯೊಂದು ಮಾತಿನ ಹಿಂದೆಯೂ ಪ್ರತಿಯೊಂದು ಕೃತಿಯ ಹಿಂದೆಯೂ ಒಂದು ಉದ್ದೇಶವಿರುತ್ತಿತ್ತು, ಒಂದು ಹಿನ್ನೆಲೆಯಿರುತ್ತಿತ್ತು. ಹಲವಾರು ವಿಭಿನ್ನ ಕಾರ್ಯಗಳ ಕಡೆಗೆ ಅವರು ಏಕಕಾಲದಲ್ಲಿ ಗಮನಕೊಡಬಲ್ಲವರಾಗಿದ್ದರು. ಹೀಗೆ, ಒಂದೆಡೆ ಅವರ ಮನಸ್ಸಿನಲ್ಲಿ ಅತ್ಯಂತ ಮೃದುವಾದ-ಸೂಕ್ಷ್ಮವಾದ ಭಾವಗಳು ಚಿಮ್ಮುತ್ತಿರುವಾಗಲೂ ತಮ್ಮ ತಕ್ಷಣದ ಕೆಲಸ ಕಾರ್ಯಗಳ ಕಡೆಗೆ ಅವರು ಎಚ್ಚರ ತಪ್ಪುತ್ತಿರಲಿಲ್ಲ.

ಸ್ವಾಮೀಜಿ ಇಂಗ್ಲೆಂಡಿಗೆ ಮೊದಲ ಬಾರಿ ಬಂದಿದ್ದಾಗ, ನಾರದೀಯ ಭಕ್ತಿಸೂತ್ರಗಳನ್ನು ಅನುವಾದಿಸುವಲ್ಲಿ ಸ್ಟರ್ಡಿಗೆ ಬಹಳಷ್ಟು ನೆರವಾಗಿದ್ದರು. ಅವರ ಭಾಷ್ಯಗಳನ್ನೊಳಗೊಂಡಿದ್ದ ಈ ಗ್ರಂಥವು ೧೮೯೬ರ ಏಪ್ರಿಲ್ಲಿನಲ್ಲಿ ಪ್ರಕಟಗೊಂಡು, ಬಹುಬೇಗ ಜನಪ್ರಿಯವಾಗಿತ್ತು. ಅದೇ ವರ್ಷ ಅವರ ಇತರ ಮೂರು ಮಹತ್ ಕೃತಿಗಳು ಪ್ರಕಟಗೊಂಡುವು–ನ್ಯೂಯಾರ್ಕ್ ತರಗತಿಗಳ ಆಧಾರದ ಮೇಲೆ ರಚಿಸಲ್ಪಟ್ಟ “ಕರ್ಮಯೋಗ” ಅಮೆರಿಕದಲ್ಲಿ ಪ್ರಕಟವಾಯಿತು; ರಾಜ ಯೋಗದ ಮೇಲಿನ ಪ್ರವಚನಗಳನ್ನು ಹಾಗೂ ಯೋಗಸೂತ್ರಗಳ ಮೇಲಿನ ಅವರ ಭಾಷ್ಯಗಳ ನ್ನೊಳಗೊಂಡ ಗ್ರಂಥವು “ರಾಜಯೋಗ” ಎಂಬ ಹೆಸರಿನಲ್ಲಿ ಲಂಡನ್ನಿನ ಸುಪ್ರಸಿದ್ಧ ಲಾಂಗ್ ಮನ್ಸ್ ಕಂಪೆನಿಯಿಂದ ಪ್ರಕಟಿಸಲ್ಪಟ್ಟಿತು; ಮತ್ತು “ಭಕ್ತಿಯೋಗ” ಎಂಬ ಕೃತಿಯು ಮದ್ರಾಸಿ ನಲ್ಲಿ ಪ್ರಕಟವಾಯಿತು.

ಲಂಡನ್ನಿನಲ್ಲಿ ಸ್ವಾಮೀಜಿಯ ಕಾರ್ಯದ ಅತಿಮುಖ್ಯ ಫಲಗಳಲ್ಲೊಂದೆಂದರೆ–‘ನಿವೇದಿತಾ’. ಆ ದಿನಗಳಲ್ಲಿ ಮಿಸ್ ಮಾರ್ಗರೆಟ್ ನೋಬೆಲ್ ಒಬ್ಬ ಬುದ್ಧಿವಂತ ಯುವತಿಯೆಂದು ಗಣ್ಯರ ವಲಯದಲ್ಲಿ ಹೆಸರು ಗಳಿಸಿದ್ದಳು. ವಿದ್ಯಾಭ್ಯಾಸದ ಬಗ್ಗೆ ಆಕೆಗೆ ತನ್ನದೇ ಆದ ಹಲವಾರು ಕಲ್ಪನೆ ಗಳಿದ್ದುವು. ಸ್ವಾಮೀಜಿಯೊಂದಿಗಿನ ಆಕೆಯ ಮೊದಲ ಭೇಟಿಯನ್ನು ಈಗಾಗಲೇ ನೋಡಿದ್ದೇವೆ. ಸ್ವಾಮೀಜಿ ಅಮೆರಿಕೆಗೆ ಹೊಡುವ ಮೊದಲೇ ಅವರನ್ನು ಆಕೆ ‘ಗುರುದೇವ’ ಎಂದು ಕರೆದಿದ್ದಳು. ಆಗ ಅವಳು ಆಕರ್ಷಿತಳಾಗಿದ್ದುದು ಅವರ ಬೋಧನೆಗಳಿಗಿಂತ ಹೆಚ್ಚಾಗಿ ಅವರ ವ್ಯಕ್ತಿತ್ವದಿಂದ. ಅವರು ಲಂಡನ್ನಿಗೆ ಹಿಂದಿರುಗಿದ ಮೇಲೆ ಮಾರ್ಗರೆಟ್ಟಳು ಅವರ ಪ್ರತಿಯೊಂದು ತರಗತಿಯಲ್ಲೂ ಪ್ರತಿಯೊಂದು ಉಪನ್ಯಾಸದಲ್ಲೂ ಹಾಜರಿರುತ್ತಿದ್ದಳು. ವಿಷಯವನ್ನು ಸಂಪೂರ್ಣವಾಗಿ ಗ್ರಹಿಸ ಬೇಕೆಂಬ ಅವಳ ತೀವ್ರ ಆಸಕ್ತಿಯು, ಅವಳು ಮತ್ತೆ ಮತ್ತೆ ಕೇಳುತ್ತಿದ್ದ “ಆದರೆ ಸ್ವಾಮೀಜಿ....” “ಏಕೆ ಸ್ವಾಮೀಜಿ...”ಎಂಬ ಪ್ರಶ್ನೆಗಳ ಮೂಲಕ ವ್ಯಕ್ತವಾಗುತ್ತಿತ್ತು. ಇದರಿಂದಾಗಿ ಸ್ವಾಮೀಜಿಯ ಗಮನ ಅವಳೆಡೆಗೆ ಹರಿಯುವಂತಾಯಿತು. ಮೊದ ಮೊದಲು ಆಕೆಯ ವರ್ತನೆಯಿಂದ ಸ್ವಾಮೀಜಿ ಸ್ವಲ್ಪ ಅಚ್ಚರಿಗೊಂಡರೂ, ಈಕೆ ಇತರ ಶ್ರೋತೃಗಳಿಗಿಂತ ಸಂಪೂರ್ಣ ವಿಭಿನ್ನಳೆಂಬುದನ್ನು ಅವರು ಕಂಡುಕೊಂಡರು. ಉನ್ನತ ಜೀವನದ ಬಗೆಗಿನ ತೀವ್ರ ಹಂಬಲವೂ ಯಾವುದಾದರೊಂದು ಮಹಾ ಆದರ್ಶಕ್ಕೆ ತನ್ನ ಜೀವನವನ್ನು ಸಮರ್ಪಿಸಿಕೊಳ್ಳಬೇಕೆಂಬ ಉತ್ಕಟೇಚ್ಛೆಯೂ ಇವಳ ಎಳೆಯ ಹೃದಯವನ್ನು ತುಂಬಿದುದನ್ನು ಅವರು ಗುರುತಿಸಿದರು. ಜಗತ್ತಿನ ಜನಸಮೂಹವು ಸವೆಸಿದ ದಾರಿಯಲ್ಲಿ ಈಕೆ ಸಾಗುವವಳಲ್ಲವೆಂಬುದು ಅವರಿಗೆ ಗೋಚರವಾಯಿತು. ತಾವು ಶ್ರೀ ರಾಮಕೃಷ್ಣರ ಪದದಡಿಯಲ್ಲಿ ಕಳೆದ ದಿನಗಳ ನೆನಪು ಅವರ ಮನಸ್ಸಿನಲ್ಲಿ ಮೂಡಿದ್ದರೆ ಅಚ್ಚರಿಯೇನಲ್ಲ.

ಸ್ವಾಮೀಜಿಯ ಬೋಧನೆಗಳು ಈ ತರುಣಿಯ ಭಾವಪ್ರಪಂಚದಲ್ಲೊಂದು ಚಂಡಮಾರುತ ವನ್ನೇ ಎಬ್ಬಿಸಿಬಿಟ್ಟಿದ್ದುವು. ಮೊದಲಿನಿಂದಲೂ ಮಾರ್ಗರೆಟ್ಟಳು ಇತರರಿಗಿಂತ ನಾಲ್ಕು ಹೆಜ್ಜೆ ಮುಂದಿದ್ದವಳೇ. ತನ್ನ ಧರ್ಮದ ಎಷ್ಟೋ ಸಂಪ್ರದಾಯಗಳಲ್ಲಿ ಆಚರಣೆಗಳಲ್ಲಿ ಆಕೆಗೆ ವಿಶ್ವಾಸ ವಿರಲಿಲ್ಲ. ಆದರೆ ಇತರ ಕೆಲವು ನಂಬಿಕೆಗಳಿಗೆ ಅವಳು ಬಲವಾಗಿ ಅಂಟಿಕೊಂಡಿದ್ದಳು. ಒಳ್ಳೆಯ ಜೀವನ ನಡೆಸಬೇಕೆನ್ನುವವರೆಲ್ಲ ಈ ತತ್ತ್ವಗಳಿಗೆ ಬದ್ಧರಾಗಿರಲೇಬೇಕೆಂದು ಆಕೆ ನಂಬಿದ್ದಳು. ಆದರೆ ಸ್ವಾಮೀಜಿಯ ಸಂಪರ್ಕಕ್ಕೆ ಬರುತ್ತಿದ್ದಂತೆಯೇ ಇವುಗಳೆಲ್ಲ ಕಳಚಿ ಬೀಳತೊಡಗಿದುವು. ಉದಾಹರಣೆಗೆ, ಪರೋಪಕಾರವೇ ಶ್ರೇಷ್ಠತಮ ಧರ್ಮ ಹಾಗೂ ಮುಕ್ತಿಗೆ ದಾರಿ ಎಂಬುದು ಪಾಶ್ಚಾತ್ಯರ ವಾದ. ಆದರೆ ‘ಉಪಕಾರ’ ಎಂಬ ಮಾತನ್ನೇ ಸ್ವಾಮೀಜಿ ಖಂಡಿಸಿ, ಅದೊಂದು ಪೊಳ್ಳು ವಾದವೆಂದು ತೋರಿಸಿಕೊಟ್ಟಾಗ ಆಕೆಗಾದ ಆಘಾತ ಅಷ್ಟಿಷ್ಟಲ್ಲ. ಆದರೆ ಕಡೆಗೂ ಅವಳು ಸ್ವಾಮೀಜಿಯ ತರ್ಕಕ್ಕೆ ಸೋಲಲೇಬೇಕಾಯಿತು. ತನ್ನ ಒಂದೊಂದು ನಂಬಿಕೆಯನ್ನು ಬಿಡಬೇಕಾ ದಾಗಲೂ ಆಕೆ ಪಟ್ಟ ಕಷ್ಟ ಅಪಾರ. ತಾನು ಕಲಿತಿದ್ದ ಪಾಠಗಳನ್ನೆಲ್ಲ ಮರೆತು ಅವಳು ಹೊಸ ಪಾಠಗಳನ್ನು ಕಲಿಯಬೇಕಾಗಿತ್ತು. ತಾನು ಹಿಡಿದಿರುವ ದಾರಿಯು ಕಲ್ಲುಮುಳ್ಳುಗಳಿಂದ ಕೂಡಿದ ಸುದೀರ್ಘ-ದುರ್ಗಮ ಮಾರ್ಗವೆಂಬುದನ್ನು ಆಕೆ ಬೇಗನೆ ಕಂಡುಕೊಂಡಳು. ಆದರೆ, ಬಂದ ಕಷ್ಟಗಳನ್ನೆಲ್ಲ ಎದುರಿಸಿ ಮುನ್ನಡೆಯಲು ಸಿದ್ಧಳಾದಳು.

ಹೀಗೆ ಅವಳು ತ್ಯಾಗಕ್ಕೆ ನಿಧಾನವಾಗಿ ಸಿದ್ಧಳಾಗತೊಡಗಿದ್ದರೂ ಸ್ವಾಮೀಜಿ ಅವಳನ್ನಿನ್ನೂ ನೇರವಾಗಿ ಆಹ್ವಾನಿಸಿರಲಿಲ್ಲ. ಒಂದು ದಿನ ತರಗತಿಯಲ್ಲಿ ಅವರ ಕರೆ ಇದ್ದಕ್ಕಿದ್ದಂತೆ ಮೊಳಗಿತು. ಆ ಸಂದರ್ಭವನ್ನು ಮುಂದೆ ನಿವೇದಿತೆಯೇ ಬಣ್ಣಿಸುತ್ತಾಳೆ: ಪ್ರಶ್ನೋತ್ತರದ ಅವಧಿಯಲ್ಲಿ, ಯಾವುದೋ ವಿಷಯವಾಗಿ ಚರ್ಚೆ ಆರಂಭವಾಗಿತ್ತು. ತಮ್ಮ ನಿಲುವನ್ನು ಸಮರ್ಥಿಸಿಕೊಳ್ಳುವಾಗ ಸ್ವಾಮೀಜಿ, “ಇಂದು ಜಗತ್ತಿಗೆ ಬೇಕಾಗಿರುವುದೇನೆಂದರೆ...” ಎನ್ನುತ್ತಿದ್ದಂತೆ ಅವರಲ್ಲೊಂದು ಸ್ಫೂರ್ತಿಯುಕ್ಕಿತು. “ಇಂದು ಜಗತ್ತಿಗೆ ಬೇಕಾಗಿರುವುದೇನೆಂದರೆ, ಬೀದಿಯಲ್ಲಿ ನಿಂತು, ‘ನನ್ನ ಕೈಯಲ್ಲಿ ಭಗವಂತನಲ್ಲದೆ ಬೇರಾವ ಆಸ್ತಿಯೂ ಇಲ್ಲ’ ಎಂದು ಸಾರಬಲ್ಲ ಕೆಚ್ಚೆದೆಯ ಇಪ್ಪತ್ತು ಸ್ತ್ರೀಪುರುಷರು. ಯಾರಿದ್ದಾರೆ ಅಂಥವರು?” ಎಂದು ಗುಡುಗಿದರು. ಬಳಿಕ ಮೇಲೆದ್ದು ನಿಂತು, ಸಭಿಕರನ್ನು ಆಹ್ವಾನಿಸುವಂತೆ ಒಂದು ಬಗೆಯ ಕಾತರದ ದೃಷ್ಟಿಯಿಂದ ಸುತ್ತಲೂ ನೋಡಿದರು ಬಳಿಕ ಅವರು ಪ್ರಚಂಡ ಶ್ರದ್ಧೆ-ವಿಶ್ವಾಸ ತುಂಬಿದ್ದ ದನಿಯಲ್ಲಿ ಮೊಳಗಿದರು, “ಏಕೆ ಹೆದರಬೇಕು? ಭಗವಂತನೊಬ್ಬನೇ ನಿತ್ಯವೆಂಬ ಮಾತು ಸತ್ಯವೇ ಆಗಿದ್ದರೆ, ಬೇರಾವುದರಿಂದ ತಾನೆ ಏನು ಪ್ರಯೋಜನ? ಅಥವಾ, ಆ ಮಾತು ಸತ್ಯವಲ್ಲದಿದ್ದರೆ, ನಮ್ಮ ಈ ಜೀವನದಿಂದ ತಾನೆ ಏನು ಪ್ರಯೋಜನ?” ಈ ಕರೆಗೆ ತಕ್ಷಣವೇ ಓಗೊಡುವ ಸಾಮರ್ಥ್ಯಶಾಲಿಗಳು ಯಾರೂ ಅಲ್ಲಿ ಕಾಣಲಿಲ್ಲ. ಆದರೆ ಅವರ ಮಾತುಗಳು ಮಾರ್ಗರೆಟ್ಟಳ ಹೃದಯದಲ್ಲಿ ಹಗಲಿರುಳೂ ಪ್ರತಿಧ್ವನಿಸಿ ದುವು. ಈ ದಿನಗಳಲ್ಲೇ ಆಕೆಗೆ ಅವರಿಂದ ಒಂದು ಪತ್ರ ಬಂದಿತು. ಜಗತ್ತಿನ ಹಿತಕ್ಕಾಗಿ ತನ್ನದೆನ್ನುವುದನ್ನೆಲ್ಲ ತ್ಯಾಗ ಮಾಡುವಂತೆ ಪ್ರೇರೇಪಿಸುವ ಪತ್ರ ಆದಾಗಿತ್ತು–“ಅನೇಕರ ಒಳಿತಿ ಗಾಗಿ, ಲೋಕದ ಸಲುವಾಗಿ, ಈ ಭೂಮಿಯ ಮೇಲಿನ ಅತ್ಯಂತ ಶಕ್ತರೂ ಸಮರ್ಥರೂ ಆದ ವ್ಯಕ್ತಿಗಳು ತ್ಯಾಗ ಮಾಡಿ ಮುಂದೆ ಬರಬೇಕಾಗುತ್ತದೆ. ಅನಂತ ಪ್ರೇಮವನ್ನೂ ಅನುಕಂಪೆಯನ್ನೂ ಹೊತ್ತ ನೂರಾರು ಬುದ್ಧರು ಬೇಕಾಗಿದ್ದಾರೆ... ಇಂದು ಜಗತ್ತಿಗೆ, ಇಡಿಯ ಜೀವನವೇ ಜ್ವಲಂತ ನಿಃಸ್ವಾರ್ಥ ಪ್ರೇಮಮಯವಾಗಿರುವಂತಹ ವ್ಯಕ್ತಿಗಳ ಆವಶ್ಯಕತೆಯಿದೆ. ಆ ಪ್ರೇಮವು, ನಾವು ಆಡುವ ಪ್ರತಿಯೊಂದು ಮಾತನ್ನೂ ಸಿಡಿಲನ್ನಾಗಿಸುತ್ತದೆ... ನನಗೆ ಖಚಿತವಾಗಿದೆ–ಜಗತ್ತನ್ನೇ ಅಲುಗಿಸಬಲ್ಲ ಸಾಮರ್ಥ್ಯ ನಿನಗಿದೆ ಎಂದು. ನಮಗಿಂದು ಬೇಕಾದುದು ಧೀರ ನುಡಿಗಳು, ಧೀರತರ ಕಾರ್ಯಗಳು. ಎದ್ದೇಳಿ, ಎದ್ದೇಳಿ, ಓ ಮಹಿಮಾನ್ವಿತರೇ! ಸಮಸ್ತ ಜಗತ್ತೇ ದುಃಖದಿಂದ ದಹಿಸುತ್ತಿರುವಾಗ ನೀವು ಹೇಗೆ ತಾನೆ ನಿದ್ರಿಸಬಲ್ಲಿರಿ? ಮಲಗಿರುವ ದೇವರುಗಳೆಲ್ಲ ಎಚ್ಚರ ಗೊಳ್ಳುವವರೆಗೆ, ಅವರೊಳಗಿನ ಭಗವಂತನೇ ಓಗೊಡುವವರೆಗೆ, ನಾವವರನ್ನು ಕೂಗಿ ಎಬ್ಬಿಸಲು ಪ್ರಯತ್ನಿಸೋಣ. ಜೀವನದಲ್ಲಿ ಇದಕ್ಕಿಂತ ಮಿಗಿಲಾದುದೇನಿದೆ? ಇದಕ್ಕಿಂತ ಮಹತ್ತಾದ ಕಾರ್ಯ ವಾವುದಿದೆ? ನಾವು ಕೆಲಸದಲ್ಲಿ ಮುಂದುವರಿದಂತೆ, ಏನು ಮಾಡಬೇಕೆಂಬುದು ತಾನಾಗಿಯೇ ಗೊತ್ತಾಗುತ್ತದೆ. ನಾನೆಂದೂ ಯೋಜನೆಗಳನ್ನು ಹಾಕುತ್ತ ಕುಳಿತುಕೊಳ್ಳುವುದಿಲ್ಲ. ಯೋಜನೆ ಗಳೆಲ್ಲ ತಾವಾಗಿಯೇ ಮೂಡಿ ಕಾರ್ಯೋನ್ಮುಖವಾಗುತ್ತದೆ. ನಾನು ‘ಎಚ್ಚರಗೊಳ್ಳಿ, ಎಚ್ಚರ ಗೊಳ್ಳಿ’ ಎಂದಷ್ಟೇ ಹೇಳುತ್ತೇನೆ. ಎಂದೆಂದಿಗೂ ನಿನ್ನ ಮೇಲೆ ಭಗವಂತನ ಕೃಪೆಯಿರಲಿ!”

ಸ್ವಾಮೀಜಿಯ ಈ ಕರೆ ಮಾರ್ಗರೆಟ್ಟಳ ಹೃದಯಾಂತರಾಳವನ್ನು ಹೊಕ್ಕು ಕಾರ್ಯಪ್ರವೃತ್ತ ವಾಯಿತು. ಮತ್ತು ತನ್ನ ಉದ್ದೇಶಿತ ಗುರಿಯನ್ನು ಸಾಧಿಸುವವರೆಗೂ ಅದು ನಿಲ್ಲಲಿಲ್ಲ!

ಲಂಡನ್ನಿನಲ್ಲಿ ಸ್ವಾಮೀಜಿ ಅವಿಶ್ರಾಂತವಾಗಿ ದುಡಿಯುತ್ತಿದ್ದರು; ಹಿಂದಿನ ಸಲಕ್ಕಿಂತಲೂ ಹೆಚ್ಚಾಗಿ ಶ್ರಮಿಸುತ್ತಿದ್ದರು. ಆದರೆ ಅಮೆರಿಕದಲ್ಲಿ ನಿರ್ಮಿಸಿದಂತೆ ಇಲ್ಲಿಯೂ ಒಂದು ಸಂಸ್ಥೆ ಯನ್ನು ನಿರ್ಮಿಸಲು ಅವರು ಆತುರಪಟ್ಟಂತೆ ಕಾಣಲಿಲ್ಲ. ಬಹುಶಃ ಇಂಗ್ಲೆಂಡಿನ ಕಾರ್ಯಕಲಾಪ ಗಳು ಸಾವಕಾಶವಾಗಿ, ಶಾಂತವಾಗಿ, ಆದರೆ ಸ್ಥಿರವಾಗಿ ನಡೆದುಕೊಂಡು ಬರಲಿ ಎಂಬುದು ಅವರ ಇಚ್ಛೆಯಾಗಿತ್ತು. ಭದ್ರವಾದ ಹಾಗೂ ಶಾಶ್ವತವಾದ ಅಡಿಪಾಯದ ಮೇಲೆ ಆ ಸಂಸ್ಥೆಯನ್ನು ನಿರ್ಮಾಣ ಮಾಡಲು ಅವರು ಆಶಿಸಿದರು. ತಮ್ಮ ಆ ಒಂದು ಕಾರ್ಯೋದ್ದೇಶದೆಡೆಗೆ ಶಕ್ತಿಶಾಲಿ ಗಳಾದ, ಉದ್ದೇಶನಿಷ್ಠರಾದ ಹಾಗೂ ಬುದ್ಧಿವಂತರಾದ ಸ್ತ್ರೀಪುರುಷರನ್ನು ಆಕರ್ಷಿಸಿ ಅವರ ಆಧಾರದ ಮೇಲೆ ಆ ಸಂಸ್ಥೆಯನ್ನು ನಿರ್ಮಿಸುವುದು ಅವರ ಉದ್ದೇಶವಾಗಿತ್ತು. ತಾವು ಯಾವ ಒಂದು ಧಾಟಿಯಲ್ಲಿ ತಮ್ಮ ಸಂದೇಶ ಪ್ರಸಾರದ ಕಾರ್ಯಕ್ರಮವನ್ನು ನಡೆಸುತ್ತಿದ್ದರೋ ಅದೇ ಧಾಟಿಯಲ್ಲಿ ಅದನ್ನು ಮುಂದುವರಿಸಿಕೊಂಡು ಹೋಗಬಲ್ಲ ಸ್ತ್ರೀಪುರುಷರನ್ನು ಅವರು ನಿರೀಕ್ಷಿಸು ತ್ತಿದ್ದರು. ಶ್ರೀಮತಿ ಸಾರಾಳಿಗೆ ಬರೆದ ಒಂದು ಪತ್ರದಲ್ಲಿ ಅವರು ಹೀಗೆ ತಿಳಿಸಿದ್ದರು: “ನಾನು ಮತ್ತು ಸ್ಟರ್ಡಿ ಕೆಲವು ಅತ್ಯುತ್ತಮ ವ್ಯಕ್ತಿಗಳನ್ನು, ಎಂದರೆ ಶಕ್ತಿಶಾಲಿಗಳೂ ಬುದ್ಧಿಶಾಲಿಗಳೂ ಆದವರನ್ನು ನಿರೀಕ್ಷಿಸುತ್ತಿದ್ದೇವೆ. ಅಂಥವರಿಂದ ಒಂದು ಸಂಸ್ಥೆಯನ್ನು ನಿರ್ಮಿಸಬೇಕೆಂಬುದು ನಮ್ಮ ಇಚ್ಛೆ. ಆದ್ದರಿಂದ ನಾವು ನಿಧಾನವಾಗಿ ಮುಂದುವರಿಯಬೇಕಾಗಿದೆ... ಇದೊಂದು ಕೇವಲ ಖಯಾಲಿಯಾಗದಂತೆ ಮೊದಲಿನಿಂದಲೂ ಎಚ್ಚರಿಕೆ ವಹಿಸುವುದು ಒಳ್ಳೆಯದು.” ಕ್ರಿಸ್ಟೀನ ಳಿಗೆ ಬರೆದ ಪತ್ರದಲ್ಲಿ, “ಎಲ್ಲ ಮಹಾಕಾರ್ಯಗಳೂ ನಿಧಾನವಾಗಿ ಮುನ್ನಡೆಯಬೇಕಾದದ್ದು ಅತ್ಯಗತ್ಯ” ಎಂದು ಹೇಳಿದ್ದರು. ಆದರೆ ಸ್ವಾಮೀಜಿ ಕೈಗೊಂಡಿದ್ದ ಕಾರ್ಯಗಳು ಸಂಘಟಿತವಾಗಿಲ್ಲ ದಿದ್ದರೂ ವ್ಯವಸ್ಥಿತವಾಗಿ ನಡೆಯುತ್ತಿದ್ದುವು. ಅಮೆರಿಕದ ತಮ್ಮ ಕಾರ್ಯಕ್ಕಿಂತಲೂ ಇಂಗ್ಲೆಂಡಿನ ಕಾರ್ಯ ಹೆಚ್ಚು ಚೆನ್ನಾಗಿ ನಡೆದುಕೊಂಡು ಬರುತ್ತಿದೆಯೆಂದೂ ಅವರು ಅಭಿಪ್ರಾಯಪಟ್ಟಿದ್ದರು.

ಲಂಡನ್ನಿನಲ್ಲಿ ನಡೆಸಿದ ಅವಿಶ್ರಾಂತ ದುಡಿಮೆಯ ಪರಿಣಾಮವಾಗಿ ಸ್ವಾಮೀಜಿಯವರ ದೇಹಾರೋಗ್ಯ ಬಹಳಷ್ಟು ಹದಗೆಟ್ಟಿತು. ಒಂದು ದಿನ ಅವರು ಭೋಜನಾನಂತರ ಒರಗುಕುರ್ಚಿ ಯಲ್ಲಿ ಕುಳಿತು ದೀರ್ಘಾಲೋಚನೆಯಲ್ಲಿ ಮಗ್ನರಾಗಿದ್ದರು. ಆಗ ಇದ್ದಕ್ಕಿದ್ದಂತೆ ಅವರ ಮುಖ ದಲ್ಲಿ ಒಂದು ತೀವ್ರ ನೋವಿನ ರೇಖೆ ಕಾಣಿಸಿಕೊಂಡಿತು. ಕ್ಷಣಕಾಲವಾದ ಮೇಲೆ ಅವರು ದೀರ್ಘವಾಗಿ ಉಸಿರುಬಿಡುತ್ತ ಬಳಿಯಲ್ಲಿದ್ದ ಜಾನ್ ಫಾಕ್ಸನಿಗೆ ಹೇಳಿದರು, “ನೋಡು ಫಾಕ್ಸ್, ಈಗ ತಾನೆ ನಾನು ನನ್ನ ಹೃದಯದಲ್ಲಿ ಅತೀವ ಯಾತನೆ ಅನುಭವಿಸಿದೆ. ಹೃದಯದ ಬಡಿತವು ಇನ್ನೇನು ನಿಂತೇಹೋಗುವಂತಿತ್ತು. ನನ್ನ ತಂದೆ ತೀರಿಕೊಂಡದ್ದೂ ಹೃದಯಾಘಾತದಿಂದ. ಇದು ನಮ್ಮ ವಂಶದ ಕಾಯಿಲೆಯೆಂದು ಕಾಣುತ್ತದೆ.” ಇಂತಹ ತೀವ್ರ ಯಾತನೆಗಳನ್ನನುಭವಿಸು ತ್ತಿದ್ದರೂ ಸ್ವಾಮೀಜಿ ಅವನ್ನು ಲೆಕ್ಕಿಸದೆ ಮತ್ತು ತೋರ್ಪಡಿಸಿಕೊಳ್ಳದೆ ನಿರಂತರವಾಗಿ ಕಾರ್ಯ ಮಗ್ನರಾಗಿರುತ್ತಿದ್ದರು.

ಮತ್ತೊಂದು ದಿನ ಸ್ವಾಮೀಜಿ ಅದೇ ರೀತಿಯಾಗಿ ಕುರ್ಚಿಯಲ್ಲಿ ಕುಳಿತು ಯಾವುದೋ ವಿಷಯದ ಬಗ್ಗೆ ಗಾಢವಾಗಿ ಆಲೋಚಿಸುತ್ತಿದ್ದರು. ಅವರು ತಾವಾಗಿಯೇ ಹೇಳಿದ ಹೊರತು ಅವರ ಮನಸ್ಸಿನಲ್ಲೇನಿದೆಯೆಂಬುದು ಯಾರಿಗೂ ತಿಳಿಯುವಂತಿರಲಿಲ್ಲ. ಮತ್ತು ಅವರು ಅದನ್ನು ಬಹಿರಂಗಪಡಿಸಿದಾಗ ಅದು ಇತರರಿಗೆ ಅಷ್ಟೇ ಅನಿರೀಕ್ಷಿತವಾಗಿರುತ್ತಿತ್ತು. ಅಂದು ಅವರು ಇದ್ದಕ್ಕಿದ್ದಂತೆ ಮೇಲೆದ್ದು ಬಳಿಯಲ್ಲಿದ್ದ ಫಾಕ್ಸ್​ನಿಗೆ ಹೇಳುತ್ತಾರೆ, “ನೋಡು, ಸಂತ ಪಾಲನು ಒಬ್ಬ ‘ಸುಶಿಕ್ಷಿತ’ ಧಾರ್ಮಿಕ ಮತಾಂಧನಾಗಿದ್ದ. ನಾನೂ ಅವನಂತೆಯೇ ಒಬ್ಬ ಸುಶಿಕ್ಷಿತ ಧಾರ್ಮಿಕ ಮತಾಂಧ. ಅಲ್ಲದೆ ನಾನು ‘ಪಂಡಿತ ಮತಾಂಧ’ರ ಒಂದು ಗುಂಪನ್ನೇ ನಿರ್ಮಿಸಲು ಇಚ್ಛಿಸುತ್ತೇನೆ. ಕೇವಲ ಧಾರ್ಮಿಕ ಮತಾಂಧರಿಂದ ಏನೂ ಪ್ರಯೋಜನವಿಲ್ಲ. ಅಂತಹ ಮತಾಂ ಧತೆಯು ಕೇವಲ ಒಂದು ಬಗೆಯ ಮಿದುಳಿನ ರೋಗ ಅಷ್ಟೆ. ಆದರೆ ಪಂಡಿತನಾದವನೇನಾದರೂ ಧಾರ್ಮಿಕ ಮತಾಂಧನಾದಲ್ಲಿ ಅವನು ಬಹಳಷ್ಟು ಉಪಯುಕ್ತವಾದದ್ದನ್ನು ಸಾಧಿಸಬಲ್ಲ. ಸಂತ ಪಾಲ್ ಕೂಡ ಒಬ್ಬ ಪಂಡಿತ ಮತಾಂಧನಾಗಿದ್ದ. ಆದ್ದರಿಂದಲೇ ಅವನು ಗ್ರೀಕ್ ತತ್ತ್ವಶಾಸ್ತ್ರದ ಹಾಗೂ ರೋಮನ್ ನಾಗರಿಕತೆಯ ಪಥವನ್ನೇ ಬದಲಿಸಿಬಿಟ್ಟ.”

ಲಂಡನ್ನಿನಲ್ಲಿದ್ದಾಗ ಸ್ವಾಮೀಜಿಯ ವ್ಯಕ್ತಿತ್ವದಲ್ಲಿ ಎದ್ದು ಕಾಣುತ್ತಿದ್ದ ಮತ್ತೊಂದು ಅಂಶ ವೆಂದರೆ ಅವರ ಸರ್ವಸಮದರ್ಶಿತ್ವ. ತಮ್ಮನ್ನು ಅಲ್ಲಿನ ರಾಜ ಮನೆತನಸ್ಥನೊಬ್ಬ ಔತಣಕೂಟಕ್ಕೆ ಆಹ್ವಾನಿಸಿದಾಗ ಅವರು ಅಲ್ಲಿನ ಉಚ್ಚಕುಲದವರೊಂದಿಗೆ ಹೇಗೆ ಬೆರೆಯುತ್ತಿದ್ದರೋ ಅಷ್ಟೇ ಸಹಜವಾಗಿ ತಮ್ಮನ್ನು ಕಾಣಬಂದ ಶ್ರೀಸಾಮಾನ್ಯರೊಂದಿಗೂ ಬೆರೆಯುತ್ತಿದ್ದರು. ಅವರೆಲ್ಲರನ್ನೂ ತಮ್ಮ ಸ್ವಂತದವರೆಂಬಂತೆ ಕಾಣುತ್ತಿದ್ದರು. ಆ ಜನರಾದರೂ ಸ್ವಾಮೀಜಿಯನ್ನು ಪರಕೀಯ ರೆಂದು ಭಾವಿಸದೆ, ಯಾವ ಬಿಗುಮಾನವೂ ಇಲ್ಲದೆ ಅವರೊಂದಿಗೆ ತಮ್ಮ ನೋವು-ನಲಿವುಗಳನ್ನು ತೋಡಿಕೊಳ್ಳುತ್ತಿದ್ದರು. ಸಾಮಾನ್ಯವಾಗಿ ಅಪರಿಚಿತರೊಂದಿಗೆ ಮಾತನಾಡುವಾಗ ಕಂಡುಬರು ವಂತಹ ಸಂಕೋಚ ಆಲ್ಲಿ ಕಂಡುಬರುತ್ತಿರಲಿಲ್ಲ. ಅಲ್ಲದೆ, ಇಂಗ್ಲೆಂಡ್ ಹೆಚ್ಚು ಸಂಪ್ರದಾಯ ನಿಷ್ಠ ದೇಶವಾಗಿತ್ತು. ಮತ್ತು ತಾವು ಆಳಲೆಂದೇ ಹುಟ್ಟಿದವರು ಎಂಬ ಭಾವನೆ ಇಂಗ್ಲಿಷರಲ್ಲಿ ರಕ್ತಗತವಾಗಿತ್ತು ಎಂಬುದನ್ನೂ ಗಮನಿಸಿದಾಗ, ಸ್ವಾಮೀಜಿ ಬೀರಿದ್ದ ಪ್ರಭಾವದ ಅರಿವಾಗುತ್ತದೆ.

ಲಂಡನ್ನಿನಲ್ಲಿ ಸ್ವಾಮೀಜಿಯ ವ್ಯಕ್ತಿತ್ವದ ಮತ್ತು ಅವರ ಸಂದೇಶದ ಮೋಡಿಗೊಳಗಾಗಿ, ಅವರ ಕಾರ್ಯೋದ್ದೇಶಕ್ಕಾಗಿ ಸರ್ವಾರ್ಪಣ ಮಾಡಿಕೊಳ್ಳಲು ಸಿದ್ಧರಾಗಿ ಮುಂದೆ ಬಂದ ಇನ್ನಿಬ್ಬರೆಂದರೆ ಸೇವಿಯರ್ ದಂಪತಿಗಳು–ಕ್ಯಾಪ್ಟನ್ ಜೇಮ್ಸ್ ಹೆನ್ರಿ ಸೇವಿಯರ್ ಮತ್ತು ಶ್ರೀಮತಿ ಚಾರ್ಲಟ್ ಸೇವಿಯರ್. ಕ್ಯಾಪ್ಟನ್ ಸೇವಿಯರ್ ಒಬ್ಬ ನಿವೃತ್ತ ಸೈನ್ಯಾಧಿಕಾರಿ. ಸ್ವಾಮೀಜಿ ಲಂಡನ್ನಿನಲ್ಲಿ ನಡೆಸುತ್ತಿದ್ದ ತರಗತಿಗಳಿಂದ ಈ ದಂಪತಿಗಳು ಅವರ ಸಂಪರ್ಕಕ್ಕೆ ಬಂದಿದ್ದರು. ಇವರಿಬ್ಬರೂ ಧರ್ಮದಲ್ಲಿ, ಧಾರ್ಮಿಕ ಜೀವನದಲ್ಲಿ ಪ್ರಾಮಾಣಿಕ ಶ್ರದ್ಧೆಯುಳ್ಳವರಾಗಿದ್ದರು. ಅದಾಗಲೇ ಅವರು ತಮ್ಮ ಆಧ್ಯಾತ್ಮಿಕ ಹಸಿವನ್ನು ಹಿಂಗಿಸಬಲ್ಲ ತತ್ತ್ವವನ್ನು ನಾನಾ ಮತಪಂಥಗಳಲ್ಲಿ ನಾನಾ ಗ್ರಂಥಗಳಲ್ಲಿ ಅರಸಿದ್ದರು, ತೊಳಲಿ ಬಳಲಿದ್ದರು. ಆದರೆ ತಮ್ಮ ಮನಸ್ಸಿಗೆ ಶಾಂತಿಯನ್ನೀಯ ಬಲ್ಲ, ಹೃದಯದ ಕತ್ತಲೆಯನ್ನಳಿಸಬಲ್ಲ ಧರ್ಮವನ್ನು ಅವರೆಲ್ಲೂ ಕಂಡಿರಲಿಲ್ಲ. ಇದರಿಂದಾಗಿ ಈ ಸೇವಿಯರ್ ದಂಪತಿಗಳು ಅತೃಪ್ತಮನಸ್ಕರಾಗಿದ್ದರು. ಧರ್ಮದ ಹೆಸರಿನಲ್ಲಿ ಕೇವಲ ಅರ್ಥ ಹೀನ ಆಚರಣೆಗಳು ಹಾಗೂ ಪೊಳ್ಳು ಸಿದ್ಧಾಂತಗಳು ಮಾತ್ರ ಎಲ್ಲೆಲ್ಲೂ ಪ್ರಚಲಿತವಾಗಿದ್ದುದನ್ನು ಕಂಡು ನಿರಾಶರಾಗಿದ್ದರು. ಈ ಸಂದರ್ಭದಲ್ಲಿ ಇವರಿಗೆ “ಭಾರತೀಯ ಯೋಗಿ”ಯೊಬ್ಬರು ಪೌರ್ವಾತ್ಯ ತತ್ತ್ವಶಾಸ್ತ್ರದ ಮೇಲೆ ತರಗತಿಗಳನ್ನು ನಡೆಸಲಿದ್ದಾರೆಂಬ ವಿಷಯವು ಸ್ನೇಹಿತರೊಬ್ಬ ರಿಂದ ತಿಳಿಯಿತು. ತಕ್ಷಣ ಈ ದಂಪತಿಗಳು ಆ ಯೋಗಿಯನ್ನು ಭೇಟಿ ಮಾಡಲು ಕುತೂಹಲ- ಕಾತರಗಳಿಂದ ಧಾವಿಸಿದರು. ಅವರೆದೆಯಲ್ಲಿ ನೂತನ ಧರ್ಮದ ಪ್ರತೀಕ್ಷೆ ತುಂಬಿಕೊಂಡಿತ್ತು. ಆ ಭಾರತೀಯ ಯೋಗಿ ಮತ್ತಾರೂ ಅಲ್ಲ, ಸ್ವಾಮಿ ವಿವೇಕಾನಂದರೇ. ಅತ್ಯಾಶ್ಚರ್ಯದ ಸಂಗತಿ ಯೆಂದರೆ, ಸ್ವಾಮೀಜಿಯ ಮಾತುಗಳನ್ನು ಕೇಳಿದಾಗ ದಂಪತಿಗಳಿಬ್ಬರಿಗೂ ಏಕಕಾಲದಲ್ಲಿ ಅನ್ನಿ ಸಿತು–ಯಾವ ತತ್ತ್ವಕ್ಕಾಗಿ, ಯಾವ ವ್ಯಕ್ತಿಗಾಗಿ ತಾವು ಇಷ್ಟು ದಿನವೂ ವ್ಯರ್ಥವಾಗಿ ತಡಕಾಡಿ ದೆವೋ ಆ ತತ್ತ್ವ ಇಲ್ಲಿದೆ, ಆ ವ್ಯಕ್ತಿ ಇಲ್ಲಿದ್ದಾನೆ, ಎಂದು. ಇಬ್ಬರೂ ಪರಸ್ಪರರ ಅಭಿಪ್ರಾಯ ಗಳನ್ನು ವಿನಿಮಯ ಮಾಡಿಕೊಂಡಾಗ ಈ ವಿಷಯ ವ್ಯಕ್ತವಾಯಿತು. ಇದನ್ನು ಕಂಡು ಅವರಿಗೇ ಆಶ್ಚರ್ಯವಾಯಿತು. ಸೇವಿಯರ್ ದಂಪತಿಗಳ ಮನಸ್ಸಿನ ಮೇಲೆ ತೀವ್ರ ಪ್ರಭಾವ ಬೀರಿದ್ದೆಂದರೆ ಅದ್ವೈತ ವೇದಾಂತ. ಅದರಲ್ಲೂ, ಅತ್ಯುನ್ನತವಾದ ಅದ್ವೈತಸಿದ್ಧಾಂತವನ್ನು ಸ್ವಾಮೀಜಿ ವಿವರಿಸಿದ ಪರಿಯನ್ನು ಕಂಡು ಇವರು ಅದಕ್ಕೆ ಮಾರುಹೋದರು. ಅತ್ಯಂತ ಕ್ಲಿಷ್ಟವೆಂದು ಪರಿಗಣಿಸಲ್ಪಡುವ, ಹಾಗೂ ಕ್ರೈಸ್ತ ತತ್ತ್ವಗಳಿಗೆ ತದ್ವಿರುದ್ಧವಾದ ಅದ್ವೈತ ತತ್ತ್ವಕ್ಕೆ ಈ ದಂಪತಿಗಳು ಮರುಳಾದುದು ಒಂದು ಅದ್ಭುತವೇ ಸರಿ.

ಸ್ವಾಮೀಜಿಯ ಬೋಧನೆಗಳಿಗಿಂತ ಹೆಚ್ಚಾಗಿ ಈ ದಂಪತಿಗಳನ್ನು ಬಲವಾಗಿ ಸೆಳೆದ ಅಂಶ ವೆಂದರೆ ಅವರ ವ್ಯಕ್ತಿತ್ವ. ಇವರಿಬ್ಬರೂ ಸ್ವಾಮೀಜಿಯನ್ನು ವೈಯಕ್ತಿಕವಾಗಿ ಭೇಟಿ ಮಾಡಿದ ಪ್ರಥಮ ಸಂದರ್ಭದಲ್ಲೇ ಸ್ವಾಮೀಜಿ ಶ್ರೀಮತಿ ಸೇವಿಯರ್​ರನ್ನು “ತಾಯಿ” ಎಂದು ಸಂಬೋ ಧಿಸಿದರು. (ಭಾರತದಲ್ಲಿ ಈ ಸಂಬೋಧನೆ ಸಾಮಾನ್ಯವಾದದ್ದಾದರೂ, ಔಪಚಾರಿಕತೆ-ಶಿಷ್ಟಾ ಚಾರಗಳ ನಾಡಾದ ಪಶ್ಚಿಮ ದೇಶವೊಂದರಲ್ಲಿ ‘ತಾಯಿ’ ಎಂಬ ಸಂಬೋಧನೆ ಉಂಟು ಮಾಡಬಲ್ಲ ಪರಿಣಾಮವೇ ಬೇರೆ.) ಬಹುಶಃ ಈ ದಂಪತಿಗಳನ್ನು ಕಂಡ ತಕ್ಷಣವೇ ಸ್ವಾಮೀಜಿಗೆ ಇವರ ಶ್ರದ್ಧೆಯೆಂತಹದು, ಸಾಮರ್ಥ್ಯವೇನು ಎಂಬುದರ ಅರಿವಾಗಿರಬೇಕು. ಶ್ರೀಮತಿ ಸೇವಿ ಯರ್​ಳನ್ನು ಸ್ವಾಮೀಜಿ ನೇರವಾಗಿ ಕೇಳಿದರು, “ತಾಯಿ, ನೀವು ನನ್ನೊಡನೆ ಭಾರತಕ್ಕೆ ಬರುವು ದಿಲ್ಲವೆ? ನಾನು ನನ್ನ ಅತ್ಯುನ್ನತ ಜ್ಞಾನವನ್ನು ನಿಮಗೆ ನೀಡುತ್ತೇನೆ.” ಆ ಕರೆ ವ್ಯರ್ಥವಾಗಲಿಲ್ಲ. ಅಂದಿನಿಂದಲೇ ಈ ದಂಪತಿಗಳ ಹಾಗೂ ಸ್ವಾಮೀಜಿಯ ನಡುವೆ ಆತ್ಮೀಯ ಸಂಬಂಧ ಬೆಳೆದು ಬಂದಿತು. ಇವರು ಸ್ವಾಮೀಜಿಯನ್ನು ಗುರುವೆಂದು ಸ್ವೀಕರಿಸಿದರು. ಅಲ್ಲದೆ ಅವರನ್ನು ತಮ್ಮ ಸ್ವಂತ ಪುತ್ರನೆಂಬಂತೆ ಕಾಣತೊಡಗಿದರು. ಹೀಗೆ ಪ್ರಾರಂಭವಾದ ಬಾಂಧವ್ಯವು, ಸ್ವಾಮೀಜಿಯ ಅತಿ ಮುಖ್ಯ ಕಾರ್ಯೋದ್ದೇಶವೊಂದರ ಈಡೇರಿಕೆಗೆ ಕಾರಣವಾಯಿತು.

ಜೂನ್ ೨೫ರಂದು ಸ್ವಾಮಿ ಶಾರದಾನಂದರು ಅಮೆರಿಕದಲ್ಲಿ ವೇದಾಂತ ಪ್ರಸಾರ ಕಾರ್ಯ ವನ್ನು ಮುಂದುವರಿಸಲು ಲಂಡನ್ನಿನಿಂದ ಹೊರಟರು. ಆ ಗುರುತರ ಜವಾಬ್ದಾರಿಯನ್ನು ಹೊತ್ತು ಕೊಳ್ಳಲು ಬೇಕಾದ ಆತ್ಮವಿಶ್ವಾಸವನ್ನು ಸ್ವಾಮೀಜಿ ಅವರಲ್ಲಿ ತುಂಬಿದ್ದರು. ಸ್ವಾಮೀಜಿಯ ಕಾರ್ಯವಿಧಾನವನ್ನು ಗಮನಿಸುವುದರ ಮೂಲಕ ಅವರು ಅನೇಕ ಹೊಸ ವಿಚಾರಗಳನ್ನು ಕಲಿತಿದ್ದರು. ಅಲ್ಲದೆ ಈ ಅಪರಿಚಿತ ನಾಡಿನಲ್ಲಿ ಅವರಿಗೆ ನೆರವಾಗಲು ಸ್ವಾಮೀಜಿಯ ನೆಚ್ಚಿನ ಬಂಟ ಗುಡ್​ವಿನ್ನನೂ ಹೊರಟಿದ್ದ. ಸ್ವಾಮೀಜಿಯ ಸಂಪೂರ್ಣ ಬೆಂಬಲವನ್ನೂ ಶುಭ ಹಾರೈಕೆ ಯನ್ನೂ ಹೊತ್ತು ಶಾರದಾನಂದರು ಅಮೆರಿಕವನ್ನು ತಲುಪಿದರು. ನ್ಯೂಯಾರ್ಕ್ ಹಾಗೂ ಬಾಸ್ಟನ್ ನಗರಗಳ ವೇದಾಂತ ವಿದ್ಯಾರ್ಥಿಗಳನ್ನು ಗುಡ್​ವಿನ್ ಅವರಿಗೆ ಪರಿಚಯಿಸಿಕೊಟ್ಟ. ಹೀಗೆ ಅಮೆರಿಕದಲ್ಲಿ ಸ್ವಾಮೀಜಿಯ ಕಾರ್ಯಯೋಜನೆಯಲ್ಲೊಂದು ಹೊಸ ಅಧ್ಯಾಯ ಪ್ರಾರಂಭವಾಯಿತು.

