
\chapter{ಮಹಾವೃಕ್ಷದ ಬೀಜಾಂಕುರ}

\noindent

ಸ್ವಾಮೀಜಿ ಅಮೆರಿಕದಿಂದ ದೂರವಾಗಿದ್ದ ಸುಮಾರು ನಾಲ್ಕು ತಿಂಗಳ ಅವಧಿಯಲ್ಲಿ, ಅಲ್ಲಿನ ಅವರ ಕಾರ್ಯ ಸಾಕಷ್ಟು ಯಶಸ್ವಿಯಾಗಿಯೇ ನಡೆದುಕೊಂಡು ಬಂದಿತ್ತೆನ್ನಬಹುದು. ಸ್ವಾಮೀಜಿಯೇ ಪ್ರಾರಂಭಿಸಿದ್ದ ಕಾರ್ಯವು ಮುಂದುವರಿಯಿತಲ್ಲದೆ, ಅವರ ಶಿಷ್ಯರು ಸ್ವಯಂ ಉತ್ಸಾಹದಿಂದ ಆ ಕಾರ್ಯವನ್ನು ವಿಸ್ತರಿಸಿದರು. ಇದೊಂದು ಗಮನಾರ್ಹ ವಿಷಯವೇ ಸರಿ. ಸ್ವಾಮೀಜಿ ಇಂಗ್ಲೆಂಡಿಗೆ ಹೊರಟಾಗ ಇತ್ತ ಅವರ ಕಾರ್ಯ ಸ್ಥಗಿತಗೊಳ್ಳುತ್ತದೆಂದೇ ಅಮೆರಿಕ ದಲ್ಲಿ ಎಷ್ಟೋ ಜನ ಭಾವಿಸಿದ್ದರು. ಆದರೆ ಹಾಗಾಲಿಲ್ಲವಷ್ಟೇ ಅಲ್ಲ, ಅದು ಇನ್ನಷ್ಟು ಚೆನ್ನಾಗಿ ಬೆಳೆದು ಗಳಿತವಾಗಿತ್ತು. ಅವರ ಇಬ್ಬರು ಸಂನ್ಯಾಸೀಶಿಷ್ಯರಾದ ಸ್ವಾಮಿ ಕೃಪಾನಂದ (ಲಿಯಾನ್ ಲ್ಯಾಂಡ್ಸ್​ಬರ್ಗ್​) ಹಾಗೂ ಸ್ವಾಮಿ ಅಭಯಾನಂದ (ಮೇಡಮ್ ಮೇರಿ ಲೂಯಿಸ್​)– ಇವರಿಬ್ಬರೂ ಅವರು ಹಿಂದಿರುಗಿ ಬರುವವರೆಗೂ ವೇದಾಂತ ಸಂದೇಶವನ್ನು ಜೀವಂತವಾಗಿ ಇಟ್ಟಿರಲು ಉತ್ಸಾಹಪೂರ್ಣ ಪ್ರಯತ್ನ ಮಾಡಿದ್ದರು. ಸ್ವಾಮೀಜಿಯ ಇನ್ನೊಬ್ಬ ಶಿಷ್ಯೆಯಾದ ಮಿಸ್ ಎಲೆನ್ ವಾಲ್ಡೊ ಕೂಡ ಅವರ ಚಟುವಟಿಕೆಗಳು ಸ್ಫೂರ್ತಿಯಿಂದ ನಡೆದುಕೊಂಡು ಬರಲು ನೆರವಾಗಿದ್ದಳು. ಬಫೆಲೊ ಹಾಗೂ ಡೆಟ್ರಾಯ್ಟ್ ನಗರಗಳಲ್ಲಿ ವೇದಾಂತ ಅಧ್ಯಯನ ಕೇಂದ್ರ ಗಳನ್ನು ಹೊಸದಾಗಿ ಪ್ರಾರಂಭಿಸಲಾಗಿತ್ತು.

೧೮೯೫ರ ಡಿಸೆಂಬರ್ ೬ರಂದು ಸ್ವಾಮೀಜಿ ನ್ಯೂಯಾರ್ಕ್ ತಲುಪಿದರು. ದಾರಿಯಲ್ಲಿ ಅವರು ಸಮುದ್ರಜಾಡ್ಯದಿಂದ ಸ್ವಲ್ಪ ಕಷ್ಟಪಟ್ಟರೂ ತುಂಬ ಉತ್ಸಾಹದಿಂದಿದ್ದರು. ಲಂಡನ್ನಿನ ತೀವ್ರ ಚಟುವಟಿಕೆ ಅವರಿಗೆ ಸಾಕಷ್ಟು ಪರಿಶ್ರಮ ಉಂಟುಮಾಡಿದ್ದರೂ ಅವರ ನಿರೀಕ್ಷೆ ಮೀರಿ ಯಶಸ್ವಿಯಾಗಿದ್ದು, ಅವರಿಗೆ ಆನಂದದಾಯಕವಾಗಿದ್ದಿತು. ಆದ್ದರಿಂದ ಅವರು ನವಸ್ಫೂರ್ತಿ ಯಿಂದ ಅಮೆರಿಕದಲ್ಲಿ ಮತ್ತೊಮ್ಮೆ ಕಾರ್ಯನಿರತರಾಗಲು ಸಿದ್ಧರಾದರು. ನ್ಯೂಯಾರ್ಕಿನಲ್ಲಿ ಅವರು ಸ್ವಾಮಿ ಕೃಪಾನಂದರೊಂದಿಗೆ ವಸತಿಗೃಹವೊಂದರಲ್ಲಿ ಎರಡು ದೊಡ್ಡ ದೊಡ್ಡ ಕೋಣೆ ಗಳನ್ನು ಬಾಡಿಗೆಗೆ ತೆಗೆದುಕೊಂಡು ಅಲ್ಲಿ ಇಳಿದುಕೊಂಡರು. ಈ ಕೋಣೆಗಳಲ್ಲಿ ಒಟ್ಟು ಸುಮಾರು ೧೫ಂ ವಿದ್ಯಾರ್ಥಿಗಳಿಗೆ ಸ್ಥಳಾವಕಾಶವಿತ್ತು. ಇಲ್ಲಿಗೆ ಬಂದ ಮೂರು ದಿನಗಳಲ್ಲೇ ಸ್ವಾಮೀಜಿ ತರಗತಿಗಳನ್ನು ಪುನರಾರಂಭಿಸಿದರು.

ಈ ನಡುವೆ, ಅವರ ಚಳಿಗಾಲದ ತರಗತಿಗಳ ಖರ್ಚನ್ನೆಲ್ಲ ವಹಿಸಿಕೊಳ್ಳುವುದಾಗಿ ಮಾತು ಕೊಟ್ಟಿದ್ದ ಶ್ರೀಮಂತ ಮಹಿಳೆ ಏಕೋ ಏನೋ ಹಿಂಜರಿದುಬಿಟ್ಟಳು. ಆದರೆ ಸ್ವಾಮೀಜಿ ಮನುಷ್ಯರ ನೆರವನ್ನು ನೆಚ್ಚಿಕೊಂಡು ಕುಳಿತವರಲ್ಲವಲ್ಲ! ಅವರು ತಮ್ಮ ಗುರುದೇವನ್ನೊಬ್ಬನನ್ನೇ ಆಶ್ರಯಿಸಿ ದ್ದವರು. ಈಗಲೂ ಅದೇ ದಿವ್ಯ ಕೃಪೆಯನ್ನೇ ಅವಲಂಬಿಸಿ, ತಮ್ಮ ಕಾರ್ಯದಲ್ಲಿ ಧುಮುಕಿಯೇ ಬಿಟ್ಟರು. ಹಿಂದಿನ ತರಗತಿಗಳ ಅವಧಿಯಲ್ಲಿ ಮಾಡಿದ್ದಂತೆಯೇ ಕರ್ಮಯೋಗ, ಭಕ್ತಿಯೋಗ, ರಾಜಯೋಗ ಹಾಗೂ ಜ್ಞಾನಯೋಗಗಳ ಕುರಿತಾಗಿ ತರಗತಿಗಳನ್ನು ಪ್ರಾರಂಭಿಸಿದರು. ದಿನಕ್ಕೆ ಎರಡು ತರಗತಿಗಳಂತೆ ಸುಮಾರು ಹದಿನೈದು ದಿನ ತರಗತಿಗಳು ರಭಸದಿಂದ ನಡೆದುವು. ಇವುಗಳೊಂದಿಗೆ ಬೃಹತ್ತಾದ ಪತ್ರವ್ಯವಹಾರದ ಕಾರ್ಯ ಬೇರೆ ಇತ್ತು–ತಮ್ಮ ಗುರುಭಾಯಿ ಗಳಿಗೆ, ಮದ್ರಾಸಿನ ಶಿಷ್ಯರಿಗೆ, ಭಾರತದ ಇತರ ಹಲವಾರು ಪರಿಚಿತರಿಗೆ, ಇಂಗ್ಲೆಂಡಿನ ಶಿಷ್ಯರು- ಭಕ್ತರಿಗೆ, ಅಮೆರಿಕದ ಅನೇಕ ಆತ್ಮೀಯರಿಗೆ ಅವರು ಪತ್ರಗಳನ್ನು ಬರೆಯುತ್ತಲೇ ಇರಬೇಕಾಗಿತ್ತು. ಈ ಮಧ್ಯೆ ಅವರು ಪತಂಜಲಿ ಮಹರ್ಷಿಗಳ ಯೋಗಸೂತ್ರಗಳ ಮೇಲೆ ವ್ಯಾಖ್ಯಾನ ಬರೆದರು. ಮುಂದೆ ಪ್ರಸಿದ್ಧವಾದ ರಾಜಯೋಗ ಗ್ರಂಥದಲ್ಲಿ ಈ ವ್ಯಾಖ್ಯಾನವನ್ನು ಸೇರಿಸಲಾಯಿತು. ಇದೇ ಅವಧಿಯಲ್ಲಿ ಅವರು ಅನೇಕ ಪ್ರತಿಕೆಗಳಿಗೆ ಸಂದರ್ಶನಗಳನ್ನು ಕೊಟ್ಟರು. ಮತ್ತು ಅಳಸಿಂಗ ಪೆರುಮಾಳ್ ಪ್ರಾರಂಭಿಸಿದ್ದ ‘ಬ್ರಹ್ಮವಾದಿನ್​’ಪತ್ರಿಕೆಗೆ, ಭಕ್ತಿಯೋಗದ ಮೇಲೆ ಲೇಖನಗಳನ್ನು ಬರೆದು ಕಳಿಸಿದರು.

ಎಂದಿನಿಂದಲೂ ಸ್ವಾಮೀಜಿಯ ಭಾಷಣಗಳು ಹಾಗೂ ತರಗತಿಗಳ ಉಪನ್ಯಾಸಗಳೆಲ್ಲವೂ ಪೂರ್ವತಯಾರಿಯಿಲ್ಲದೆ ಮಾಡಲ್ಪಟ್ಟವುಗಳು. ಭಾಷಣಗಳನ್ನು ತಯಾರು ಮಾಡುವ ಅಥವಾ ಬರೆದಿಟ್ಟುಕೊಳ್ಳುವ ಆವಶ್ಯಕತೆ ಅವರಿಗಂತೂ ಇರಲಿಲ್ಲ! ಆದ್ದರಿಂದ ಅತ್ಯಮೂಲ್ಯವಾದ ಅವರ ಅನೇಕ ಉಪನ್ಯಾಸಗಳು ಹಾಗೂ ಸಂಭಾಷಣೆಗಳು ದಾಖಲಾಗದೆ ನಷ್ಟವಾಗಿ ಹೋಗಿದ್ದುವು. ಆದ್ದರಿಂದ ಅವರ ಶಿಷ್ಯರು, ಇನ್ನು ಮುಂದಾದರೂ ಹಾಗಾಗಬಾರದೆಂದು ಬಯಸಿ, ಅವುಗಳನ್ನೆಲ್ಲ ಬರೆದಿಡಲು ಶೀಘ್ರಲಿಪಿಕಾರನೊಬ್ಬನನ್ನು ನೇಮಿಸಿಕೊಳ್ಳಲು ನಿಶ್ಚಯಿಸಿದರು.

ಸ್ವಾಮೀಜಿಯೇ ಹಿಂದಿನ ವರ್ಷ ಪ್ರಾರಂಭಿಸಿದ್ದ ‘ವೇದಾಂತ ಸೊಸೈಟಿ’ಯ ಅಧಿಕಾರಿಗಳು ೧೮೯೫ರ ಡಿಸೆಂಬರ್​ನಲ್ಲಿ ಒಬ್ಬನನ್ನು ಆ ಕೆಲಸಕ್ಕೆ ತೆಗೆದುಕೊಂಡರು. ಆದರೆ ಸ್ವಾಮೀಜಿ ಮಾತ ನಾಡುತ್ತಿದ್ದ ವಿಷಯಗಳೂ ಅವರ ಪದಪ್ರಯೋಗಗಳೂ ಅವನಿಗೆ ಸಂಪೂರ್ಣ ಅಪರಿಚಿತವಾಗಿ ದ್ದುದರಿಂದ, ಅವರ ರಭಸಪೂರ್ಣ ಭಾಷಾಪ್ರವಾಹದೊಂದಿಗೆ ಹೊಂದಿಕೊಳ್ಳಲು ಅಸಮರ್ಥ ನಾದ. ಆಗ ಅವನನ್ನು ಬಿಟ್ಟು ಇನ್ನೊಬ್ಬನನ್ನು ನೇಮಕ ಮಾಡಿಕೊಂಡರು. ಆದರೆ ಅದರಿಂದಲೂ ಏನೂ ಪ್ರಯೋಜನವಾಗಲಿಲ್ಲ. ಕೊನೆಗೆ ೨೫ ವರ್ಷದ ಯುವಕನೊಬ್ಬ ತಾನಾಗಿಯೇ ಆ ಕೆಲಸಕ್ಕೆ ಮುಂದಾದ. ಅವನ ಹೆಸರು ಜೊಸೈಯಾ ಜೆ. ಗುಡ್​ವಿನ್. ಇಂಗ್ಲೆಂಡಿನವನಾದ ಈತ ಕೆಲಕಾಲದ ಹಿಂದೆತಾನೆ ನ್ಯೂಯಾರ್ಕಿಗೆ ಬಂದಿದ್ದ. ಶೀಘ್ರಲಿಪಿ ಹಾಗೂ ಬೆರಳಚ್ಚಿನಲ್ಲಿ ಅವನಿಗೆ ಅದಾಗಲೇ ಹನ್ನೊಂದು ವರ್ಷದ ಅನುಭವವಿತ್ತು. ಇವನನ್ನು ಭಗವಂತನೇ ಸ್ವಾಮೀಜಿಯ ಬಳಿಗೆ ಕಳಿಸಿರ ಬೇಕು. ಈತ ಒಬ್ಬ ಅತ್ಯಂತ ಸಮರ್ಥ ಶೀಘ್ರಲಿಪಿಕಾರನಾಗಿದ್ದನಷ್ಟೇ ಅಲ್ಲದೆ ಸ್ವಾಮೀಜಿಯ ‘ಬಲಗೈ’ಯೇ ಆದ. ತರಗತಿಗಳಲ್ಲಿ ಹಾಗೂ ಭಾಷಣಗಳಲ್ಲಿ ಅವರಾಡಿದ ಪ್ರತಿಯೊಂದೂ ಮಾತನ್ನೂ ಯಥಾವತ್ತಾಗಿ ಬರೆದುಕೊಳ್ಳಲು ಈತ ಸಮರ್ಥನಾಗಿದ್ದ. ಅಲ್ಲದೆ ಅವರ ದೀರ್ಘ ಭಾಷಣಗಳನ್ನೆಲ್ಲ ಆಯಾದಿನವೇ ಬೆರಳಚ್ಚು (ಟೈಪ್​) ಮಾಡಿ ವೃತ್ತಪತ್ರಿಕೆಗಳಿಗೆ ಕಳಿಸಲು ಸಿದ್ಧ ಪಡಿಸುತ್ತಿದ್ದ. ಗುಡ್​ವಿನ್ ತನ್ನ ಶಾಂತ, ವಿನೀತ, ಸಂಭಾವಿತ ವರ್ತನೆಯಿಂದ ಎಲ್ಲರಿಗೂ ಪ್ರಿಯ ನಾದ.

ಆದರೆ ಎಲ್ಲರಿಗಿಂತ ಹೆಚ್ಚು ಉಪಕಾರವಾದುದು ಬಹುಶಃ ಗುಡ್​ವಿನ್ನನಿಗೇ. ಸ್ವಾಮೀಜಿಯ ಸಂಪರ್ಕಕ್ಕೆ ಬರುತ್ತಿದ್ದಂತೆಯೇ ಅವನ ವ್ಯಕ್ತಿತ್ವದಲ್ಲೊಂದು ಮಹತ್ತರ ಬದಲಾವಣೆಯಾಯಿತು. ಅಲ್ಲಿಯವರೆಗೂ ಒಬ್ಬ ಸಾಧಾರಣ ಪ್ರಾಪಂಚಿಕ ವ್ಯಕ್ತಿಯಾಗಿದ್ದ ಗುಡ್​ವಿನ್ ಪ್ರಾಪಂಚಿಕ ಜೀವನವನ್ನು, ಪ್ರಾಪಂಚಿಕ ಆಸೆ ಆಕಾಂಕ್ಷೆಗಳನ್ನು ಸಂಪೂರ್ಣ ಬಿಟ್ಟೇಬಿಟ್ಟ. ಅನೇಕ ವರ್ಷಗಳ ಕಾಲ ಈತ ವೇದಾಂತ ಸೊಸೈಟಿಯು ನೀಡುತ್ತಿದ್ದ ಅತಿ ಕಡಿಮೆ ಸಂಬಳವನ್ನೇ ಸ್ವೀಕರಿಸಿ, ಹೃತ್ಪೂರ್ವಕವಾಗಿ ಕೆಲಸ ಮಾಡಿದ. ಆ ಸಂಬಳ ಅವನ ಜೀವನಾಧಾರಕ್ಕೆ ಸಾಲುತ್ತಿತ್ತಷ್ಟೆ. ಆದರೆ ಪ್ರತಿಫಲವನ್ನು ಲೆಕ್ಕಿಸದೆ, ವೇದಾಂತ ಪ್ರಸಾರ ಕಾರ್ಯಕ್ಕೆ ತನ್ನ ತನುಮನಗಳನ್ನರ್ಪಿಸಿದ. ಸ್ವಾಮೀಜಿ ಅವನಿಗೆ ಅವನ ಹಿಂದಿನ ಜನ್ಮವೃತ್ತಾಂಗಳನ್ನು ಹೇಳಿದ್ದರಿಂದ ಅವನಲ್ಲೊಂದು ವಿಶೇಷ ನೈತಿಕ ಪ್ರಜ್ಞೆ ಜಾಗೃತವಾಯಿತು. ಅವನು ಸ್ವಾಮೀಜಿಯ ಅತ್ಯಂತ ನಿಷ್ಠಾವಂತ ಶಿಷ್ಯನಾದ. ಅವರು ಭಾಷಣ ಮಾಡಲು ಎಲ್ಲಿಗೇ ಹೋಗಲಿ, ಗುಡ್​ವಿನ್ ಅವರನ್ನು ಹಿಂಬಾಲಿಸುತ್ತಿದ್ದ. ಅಮೆರಿಕದ ದೂರದೂರದ ನಗರಗಳಾದ ಡೆಟ್ರಾಯ್ಟಿಗೆ, ಬಾಸ್ಟನ್ನಿಗೆ, ಅಲ್ಲದೆ ಇಂಗ್ಲೆಂಡಿಗೆ– ಅಷ್ಟೇಕೆ ಮುಂದೆ ಭಾರತಕ್ಕೂ–ಆತ ಸ್ವಾಮೀಜಿಯೊಂದಿಗೆ ಬಂದ. ಅವನನ್ನು ಸ್ವಾಮಿ “ನನ್ನ ನಿಷ್ಠಾವಂತ ಗುಡ್​ವಿನ್​” ಎಂದು ಕರೆಯುತ್ತಿದ್ದರು.

ಗುಡ್​ವಿನ್​ನನ್ನು ಸ್ವಾಮೀಜಿಯತ್ತ ಬಲವಾಗಿ ಸೆಳೆದದ್ದು ಅವರ ಅಹೈತುಕ ಪ್ರೇಮ. ತೀರ ಸಣ್ಣ ಉಪಕಾರಕ್ಕೂ ಅತ್ಯಂತ ಕೃತಜ್ಞನಾಗಿರುವುದು ಗುಡ್​ವಿನ್ನನ ಸ್ವಭಾವ. ಇಂಥವನು ಸ್ವಾಮೀಜಿಯ ಅನುಕಂಪೆ-ಪ್ರೇಮಗಳಿಗೆ ಮಾರುಹೋಗಿದ್ದರಲ್ಲೇನು ಆಶ್ಚರ್ಯ? ಮುಂದೆ ಸ್ವಾಮಿ ಶಾರದಾನಂದರ ಬಳಿ ಗುಡ್​ವಿನ್ ಹೇಳಿದ, “ಬಡತನದಲ್ಲೇ ಬೆಳೆದ ನಾನು ಜೀವನೋ ಪಾಯಕ್ಕಾಗಿ ಹಲವಾರು ಕಡೆ ಸುತ್ತಾಡಿದ್ದೇನೆ. ಎಲ್ಲ ಬಗೆಯ ಜನರೊಂದಿಗೂ ಬೆರೆತಿದ್ದೇನೆ. ಅವ ರೆಲ್ಲ ನನಗೆ ಕೆಲಸವನ್ನೂ ಸಂಬಳವನ್ನೂ ಕೊಟ್ಟರು; ಆದರೆ ಯಾರೂ ತಮ್ಮ ಹೃದಯದ ಪ್ರೇಮ ವನ್ನು ಕೊಡಲಿಲ್ಲ. ಆಗ ನಾನು ಅಮೆರಿಕದಲ್ಲಿ ವಿವೇಕಾನಂದರನ್ನು ಸಂಧಿಸಿದೆ. ಆಗಲೇ ನನಗೆ ಅರ್ಥವಾದದ್ದು–ಪ್ರೀತಿಯೆಂದರೇನು ಎಂದು. ಆದ್ದರಿಂದ ಆದಾಯವಿರಲಿ ಇಲ್ಲದಿರಲಿ, ನಾನು ಸಿಕ್ಕಿಕೊಂಡು ಬಿಟ್ಟಿದ್ದೇನೆ! ವಿವೇಕಾನಂದರಂತಹ ಉದಾತ್ತ ಚರಿತರನ್ನು ನಾನೆಲ್ಲೂ ಕಂಡಿಲ್ಲ. ವಿವೇಕಾನಂದರ ಸಂಪರ್ಕಕ್ಕೆ ಬಂದವರು, ಅವರು ತಮ್ಮ ಸ್ವಂತದವರೇನೋ ಎಂಬಂತೆ ಸೆಳೆಯಲ್ಪಡುತ್ತಾರೆ.”

ಡಿಸೆಂಬರ್ ತಿಂಗಳ ಅಂತ್ಯಕ್ಕೆ ಸ್ವಾಮೀಜಿ, ತಮ್ಮ ಆಪ್ತ ಸ್ನೇಹಿತರಾದ ಲೆಗೆಟ್ ದಂಪತಿಗಳ ಆಹ್ವಾನದ ಮೇರೆಗೆ, ಹಡ್ಸನ್ ನದೀತೀರದ ಅವರ ಮನೆ ‘ರಿಡ್ಜ್​ಲಿ ಮ್ಯಾನರ್​’ಗೆ ಹೋದರು. ಕ್ರಿಸ್​ಮಸ್ ರಜದ ಸಂದರ್ಭದಲ್ಲಿ ಸ್ವಾಮೀಜಿ ಇಲ್ಲಿ ಹತ್ತು ದಿನಗಳನ್ನು ಕಳೆದರು. ಲೆಗೆಟ್ಟರ ಅತ್ಯಂತ ಆದರಪೂರ್ವಕ ಅತಿಥ್ಯದಲ್ಲಿ ಸ್ವಾಮೀಜಿ ಆನಂದದಿಂದಿದ್ದರೆಂದು ಊಹಿಸಬಹುದು. ಆ ದಿನಗಳ ವಿವರಗಳು ತಿಳಿದುಬಂದಿಲ್ಲವಾದರೂ ಲೆಗೆಟ್ಟರು ಮಿಸ್ ಮೆಕ್​ಲಾಡಳಿಗೆ ಕೆಲದಿನಗಳ ಬಳಿಕ ಬರೆದ ಒಂದು ಪತ್ರದಿಂದ ಅಲ್ಲಿನ ಒಂದು ದೃಶ್ಯ ಕಾಣಸಿಗುತ್ತದೆ. ಆಗ ಮೆಕ್​ಲಾಡ್ ಯೂರೋಪಿನಲ್ಲೇ ಉಳಿದುಕೊಂಡಿದ್ದು, ಈ ಹತ್ತು ದಿನಗಳ ಸ್ವಾಮೀಜಿಯ ಅದ್ಭುತ ಸಾನ್ನಿಧ್ಯ ದಿಂದ ವಂಚಿತಳಾದಳು. ಲೆಗೆಟ್ಟರು ಪತ್ರದಲ್ಲಿ ಹೀಗೆ ಬರೆದರು, “ರಿಡ್ಜ್​ಲಿಯಲ್ಲಿ ಒಂದು ರಾತ್ರಿ ನಾವೆಲ್ಲ ಅವರ ವಾಗ್ಝರಿಯನ್ನು ಕೇಳಿ ಸ್ತಂಭೀಭೂತರಾದೆವು. ಅವರು ಈ ಎರಡೂವರೆ ಗಂಟೆಗಳ ಕಾಲದಲ್ಲಿ ವ್ಯಕ್ತಪಡಿಸಿದಂತಹ ಆಲೋಚನೆಗಳು ಒಬ್ಬ ಮನುಷ್ಯ ಮಾತ್ರನಿಂದ ಬಂದುದನ್ನು ನಾನೆಲ್ಲೂ ಕಂಡಿಲ್ಲ. ನಾವೆಲ್ಲ ಆಳವಾಗಿ ಪ್ರಭಾವಿತರಾದೆವು. ಆ ಮಾತುಗಳನ್ನೆಲ್ಲ ಯಾರಾದರೂ ಬರೆದುಕೊಟ್ಟರೆ ನಾನದಕ್ಕೆ ನೂರು ಡಾಲರ್ ಕೊಡಲು ಸಿದ್ಧನಾಗಿದ್ದೇನೆ. ನಾವು ಎಂದೂ ಅನುಭವಿಸಿರದಿದ್ದಂತಹ ಸ್ಫೂರ್ತಿಯನ್ನು ಸ್ವಾಮೀಜಿ ನಮ್ಮಲ್ಲಿ ಉಂಟುಮಾಡಿದರು. ಅವರು ನಮ್ಮಿಂದ ಶೀಘ್ರದಲ್ಲೇ ಬೀಳ್ಕೊಳ್ಳಲಿದ್ದಾರೆ ಮತ್ತು ನಾವು ಅವರನ್ನು ಮತ್ತೆಂದೂ ನೋಡದಿರ ಬಹುದು. ಆದರೆ ಅವರು ನಮ್ಮ ಹೃದಯಗಳ ಮೇಲೆ ಎಂತಹ ಅಚ್ಚಳಿಯದ ಮುದ್ರೆಯನ್ನೊತ್ತು ತ್ತಾರೆಂದರೆ, ನಾವು ಜೀವಿಸಿರುವವರೆಗೂ ಅದು ನಮಗೆ ಶಾಂತಿ-ಸಮಾಧಾನ ನೀಡುತ್ತದೆ.”

ರಿಡ್ಜ್​ಲಿ ಮ್ಯಾನರಿನಲ್ಲಿ ರಜಾ ದಿನಗಳನ್ನು ಕಳೆದು ಹಿಂದಿರುಗಿದ ಸ್ವಾಮೀಜಿ ನ್ಯೂಯಾರ್ಕಿನಲ್ಲಿ ಹೊಸ ಉತ್ಸಾಹದಿಂದ ಭಾಷಣ ಕಾರ್ಯವನ್ನು ಪ್ರಾರಂಭಿಸಿದರು. ಭಾನುವಾರಗಳಂದು ನೀಡ ಲಾಗುತ್ತಿದ್ದ ಈ ಸಾರ್ವಜನಿಕ ಭಾಷಣಗಳಿಗೆ ಪ್ರವೇಶ ಶುಲ್ಕವಿರಲಿಲ್ಲ. ಅವರ ಕೆಲಸಕಾರ್ಯಗಳಿಗೆ ಹಣದ ಆವಶ್ಯಕತೆ ಬಹಳವಾಗಿಯೇ ಇದ್ದಿತಾದರೂ ವೇದಾಂತ ಸಂದೇಶಗಳು ಹೆಚ್ಚು ಜನರನ್ನು ಮುಟ್ಟಲಿ ಎಂಬ ಉದ್ದೇಶದಿಂದ ಅವರು ಈ ಉಪನ್ಯಾಸಗಳನ್ನು ಉಚಿತವಾಗಿ ಮಾಡಿದರು. ಅಲ್ಲದೆ ತಾವು ಭಾರತಕ್ಕೆ ಹಿಂದಿರುಗಿದ ಮೇಲೆಯೂ ವೇದಾಂತ ಸೊಸೈಟಿಯ ಅವಶ್ಯಕತೆಗಳಿಗಾಗಿ ಅದರ ಕಾರ್ಯಕರ್ತರು ಧನಸಂಗ್ರಹಣೆಯ ಉದ್ದೇಶದಿಂದ ಭಾಷಣಗಳನ್ನು ಮಾಡುವಂತಾಗ ಬಾರದೆಂಬುದು ಅವರ ಉದ್ದೇಶವಾಗಿತ್ತು. ಆದ್ದರಿಂದಲೇ ಸ್ವಾಮೀಜಿ ತಾವೇ ಇತರರಿಗೊಂದು ಮೇಲ್ಪಂಕ್ತಿ ಹಾಕಿಕೊಟ್ಟರು.

೧೮೯೬ ನೆಯ ಇಸವಿಯ ತಮ್ಮ ಪಾಲಿಗೆ ತೀವ್ರ ಚಟುವಟಿಕೆಯ ವರ್ಷವಾಗಲಿದೆ ಎಂದು ಸ್ವಾಮೀಜಿ, ಬ್ರಹ್ಮಾನಂದರಿಗೆ ಬರೆದ ಒಂದು ಪತ್ರದಲ್ಲಿ ಮುಂದಾಗಿಯೇ ತಿಳಿಸಿದ್ದರು. ಅದಕ್ಕೆ ಸರಿಯಾಗಿ, ಆ ವರ್ಷದ ಆರಂಭವು ತುಂಬ ಚಟುವಟಿಕೆಯದೇ ಆಗಿತ್ತು. ಪ್ರತಿದಿನವೂ ಅವರು ನಾಲ್ಕು ಯೋಗಗಳ ವಿಷಯವಾಗಿ ಎರಡು ತರಗತಿಗಳನ್ನು ತೆಗೆದುಕೊಳ್ಳುತ್ತಿದ್ದರು. ಈ ತರಗತಿ ಗಳೊಂದಿಗೆ ‘ಸಾಂಖ್ಯ ಮತ್ತು ವೇದಾಂತ’ ಎಂಬ ವಿಷಯವಾಗಿ ಮತ್ತೊಂದು ತರಗತಿಗಳ ಸರಣಿಯನ್ನು ಪ್ರಾರಂಭಿಸಿದರು. ಅಲ್ಲದೆ ತಾವು ಅದಾಗಲೇ ಬರೆಯುತ್ತಿದ್ದಂತಹ ಪಾತಂಜಲ ಯೋಗಸೂತ್ರಗಳ ಅನುವಾದ ಮತ್ತು ವ್ಯಾಖ್ಯಾನಗಳನ್ನು ಮುಂದುವರಿಸಿದರು.

ತಮ್ಮ ಸಾರ್ವಜನಿಕ ಉಪನ್ಯಾಸಗಳ ಮೊದಲ ಸರಣಿಯಲ್ಲಿ ಸ್ವಾಮೀಜಿ ಈ ನಾಲ್ಕು ವಿಷಯ ಗಳ ಕುರಿತಾಗಿ ಮಾತನಾಡಿದರು–‘ಧರ್ಮ ನೀಡುವ ಆಶ್ವಾಸನೆ: ಅದರ ಸತ್ಯ ಮತ್ತು ಪ್ರಯೋ ಜನ’, ‘ವಿಶ್ವ: ಬ್ರಹ್ಮಾಂಡವಾಗಿ’, ‘ವಿಶ್ವ: ಪಿಂಡಾಂಡವಾಗಿ’ ಮತ್ತು ‘ವಿಶ್ವಧರ್ಮದ ಆದರ್ಶ’. ಇವಲ್ಲದೆ ಬ್ರೂಕ್ಲಿನ್ ನಗರದ ಮೆಟಾಫಿಸಿಕಲ್ ಸೊಸೈಟಿಯಲ್ಲಿ ‘ಅಮರತ್ವ’ ಎಂಬ ವಿಷಯವಾಗಿ ಮತ್ತು ಹಾರ್ಟ್ ಪೋರ್ಡಿನಲ್ಲಿ ‘ವಿಶ್ವಧರ್ಮದ ಆದರ್ಶ’ ಎಂಬ ವಿಷಯವಾಗಿ ಮಾತನಾಡಿದರು. ಈ ಎಲ್ಲ ಭಾಷಣಗಳಿಗೂ ಜನ ಕಿಕ್ಕಿರಿದು ವಿಶೇಷ ಆಸಕ್ತಿಯಿಂದ ಆಲಿಸಿದರು.

ಯೋಗಸೂತ್ರಗಳಿಗೆ ಸ್ವಾಮೀಜಿ ವ್ಯಾಖ್ಯಾನ ಬರೆಯುತ್ತಿದ್ದ ರೀತಿ ಸ್ವಾರಸ್ಯಕರವಾಗಿತ್ತು. ಈ ವ್ಯಾಖ್ಯಾನಗಳನ್ನು ಅವರು, ತಮ್ಮ ಆಪ್ತ ಶಿಷ್ಯೆಯಾದ ಮಿಸ್ ಸಾರಾ ಎಲೆನ್ ವಾಲ್ಡೊಳಿಗೆ ಹೇಳಿ ಬರೆಸುತ್ತಿದ್ದರು. ಈಕೆ ಪ್ರತಿದಿನವೂ ಸ್ವಾಮೀಜಿಯ ವಸತಿಗೃಹಕ್ಕೆ ಬಂದು ಅವರಿಗಾಗಿ ಅಡಿಗೆ ಮಾಡುತ್ತಿದ್ದಳು. ಮನೆಯನ್ನು ಚೊಕ್ಕಟವಾಗಿಡುತ್ತಿದ್ದಳು; ಅವರ ಎಲ್ಲ ವೈಯಕ್ತಿಕ ಕೆಲಸ ಕಾರ್ಯಗಳನ್ನು ಮಾಡುತ್ತಿದ್ದಳು. ಇವುಗಳೊಂದಿಗೆ ಸ್ವಾಮೀಜಿ ಹೇಳುವ ವ್ಯಾಖ್ಯಾನಗಳನ್ನು ಬರೆದುಕೊಳ್ಳುತ್ತಿದ್ದಳು. ಈ ಸಂಬಂಧವಾಗಿ ಮಿಸ್ ವಾಲ್ಡೊ ಹೇಳುತ್ತಾಳೆ:

“ಸ್ವಾಮೀಜಿ ನನಗೆ ಯೋಗಸೂತ್ರಗಳ ವ್ಯಾಖ್ಯಾನಗಳನ್ನು ಬರೆದುಕೊಳ್ಳಲು ಹೇಳುತ್ತಿದ್ದ ಸಮಯದಲ್ಲಿ ಅವರ ನೋಟ ಅತ್ಯಂತ ಸ್ಫೂರ್ತಿಯುತವಾಗಿರುತ್ತಿತ್ತು. ವ್ಯಾಖ್ಯಾನಗಳನ್ನು ಹೇಳು ವಾಗ ಅವರು ಆಗಾಗ ಯಾವುದಾದರೊಂದು ವಿಷಯದ ಬಗ್ಗೆ ಚಿಂತಿಸುತ್ತ ಆಳವಾದ ಧ್ಯಾನದ ಸ್ಥಿತಿಗೇರುತ್ತಿದ್ದರು. ಅವರು ಆ ಸ್ಥಿತಿಯಿಂದ ಹಿಂದಿರುಗಿ ಬರುವುದನ್ನೇ ನಿರೀಕ್ಷಿಸುತ್ತ ನಾನು ಲೇಖನಿಯನ್ನು ಶಾಯಿಯಲ್ಲಿ ಅದ್ದಿಟ್ಟುಕೊಂಡಿರುತ್ತಿದ್ದೆ. ಕೆಲವೊಮ್ಮೆ ಅವರು ಹೀಗೆ ದೀರ್ಘ ಕಾಲದವರೆಗೆ ಧ್ಯಾನಾವಸ್ಥೆಯಲ್ಲಿ ಮುಳುಗಿರುತ್ತಿದ್ದರು. ಬಳಿಕ ಇದ್ದಕ್ಕಿದ್ದಂತೆ ಎಚ್ಚರಗೊಂಡು ಯಾವುದಾದರೊಂದು ಉತ್ಸಾಹಪೂರ್ಣ ಉದ್ಗಾರದಿಂದ, ಇಲ್ಲವೇ ದೀರ್ಘ ವಿವರಣೆಯಿಂದ ತಮ್ಮ ಮೌನವನ್ನು ಮುರಿಯುತ್ತಿದ್ದರು.”

ನಾಲ್ಕು ಯೋಗಗಳ ಕುರಿತಾಗಿ ಸ್ವಾಮೀಜಿ ತೆಗೆದುಕೊಳ್ಳುತ್ತಿದ್ದ ತರಗತಿಗಳಿಗೆ ಬರುತ್ತಿದ್ದವರ ಸಂಖ್ಯೆ ದಿನದಿನಕ್ಕೂ ಹೆಚ್ಚಾಗುತ್ತಿತ್ತು. ಈ ತರಗತಿಗೆ ಬರಲು ಯಾವುದೇ ಅರ್ಹತೆ ಬೇಕಿರಲಿಲ್ಲ– ಶ್ರದ್ಧೆಯೊಂದಿದ್ದರೆ ಸಾಕಾಗಿತ್ತು. ಆದರೆ ಬಂದವರಿಗೆಲ್ಲ ಇದರಿಂದ ವಿಶೇಷ ಲಾಭವಾಗುತ್ತ ದೆಂದು ಸ್ವಾಮೀಜಿಯೇನೂ ಭಾವಿಸಿರಲಿಲ್ಲ. ತಮ್ಮ ಬೋಧನೆಗಳನ್ನು ಗಂಭೀರವಾಗಿ ಸ್ವೀಕರಿಸಿ ಅನುಷ್ಠಾನ ಮಾಡಲು ಪ್ರಯತ್ನಿಸುವವರು ಕೇವಲ ಬೆರಳೆಣಿಕೆಯ ಮಂದಿ ಮಾತ್ರವೇ ಎಂಬುದು ಅವರಿಗೆ ಚೆನ್ನಾಗಿ ತಿಳಿದಿತ್ತು. ಬರುವಾಗಲೇ ಆಸಕ್ತಿಯಿಂದ ಬರುತ್ತಿದ್ದವರು ಕೆಲವರಾದರೆ ಬಂದ ಮೇಲೆ ಆಸಕ್ತಿ ತಾಳುತ್ತಿದ್ದವರು ಕೆಲವರು. ಇನ್ನುಳಿದವರು ತಾವಾಗಿಯೇ ಅದರಿಂದ ದೂರವಾಗು ತ್ತಿದ್ದರು. ಈ ತರಗತಿಗಳ ಬಗ್ಗೆ ಮಿಸ್ ಲಾರಾ ಗ್ಲೆನ್ (ಸೋದರಿ ದೇವಮಾತಾ) ಬರೆಯುತ್ತಾಳೆ:

“ಸಾಧಾರಣ ಗುಣಟ್ಟದ ಆ ವಸತಿಗೃಹದಲ್ಲಿ ನಡೆಯುತ್ತಿದ್ದ ತರಗತಿಗಳಿಗೆ ಬಗೆಬಗೆಯ ಜನ ಬರುತ್ತಿದ್ದರು. ಅವರಲ್ಲಿ ಮುದುಕರು, ಯುವಕರು, ಶ್ರೀಮಂತರು, ಬಡವರು, ತಿಳಿದವರು, ಮೂರ್ಖರು, ಕಾಣಿಕೆಯ ಡಬ್ಬಕ್ಕೆ ಹತ್ತು ಸೆಂಟ್ ಹಾಕುತ್ತಿದ್ದ ಜಿಪುಣರು, ಒಂದು ಡಾಲರ್ ಇಲ್ಲವೆ ಎರಡು ಡಾಲರ್​ವರೆಗೂ ಹಾಕುತ್ತಿದ್ದ ಉದಾರಿಗಳು–ಹೀಗೆ ಎಲ್ಲ ಬಗೆಯ ಜನರೂ ಇದ್ದರು. ದಿನದಿನವೂ ಅಲ್ಲಿ ಸೇರುತ್ತಿದ್ದ ನಾವು ಮಾತಿಲ್ಲದೆಯೇ ಸ್ನೇಹಿತರಾದೆವು. ನಮ್ಮಲ್ಲಿ ಕೆಲವರು ಒಂದು ತರಗತಿಯನ್ನೂ ತಪ್ಪಿಸುತ್ತಿರಲಿಲ್ಲ. ಭಕ್ತಿಯೋಗ ಹಾಗೂ ಜ್ಞಾನಯೋಗದ ತರಗತಿಗಳನ್ನು ಮುಗಿಸಿದೆವು. ಅದರೊಂದಿಗೇ ರಾಜಯೋಗ ಕರ್ಮಯೋಗಗಳ ಪಥವನ್ನೂ ಅನುಸರಿಸಿದೆವು. ಆದರೆ ಕೇವಲ ನಾಲ್ಕೇ ಯೋಗಗಳಿವೆಯಲ್ಲ ಎಂದು ನಮಗೆ ಸ್ವಲ್ಪ ದುಃಖವೇ ಆಯಿತು. ಆರೋ ಎಂಟೋ ಯೋಗಗಳಿದ್ದಿದ್ದರೆ ತರಗತಿಗಳೂ ಹೆಚ್ಚಾಗುತ್ತಿದ್ದುವಲ್ಲಾ ಎಂದು ಆಲೋಚಿಸಿದೆವು...

“ನಮ್ಮಲ್ಲಿ ಅನೇಕರು ಸ್ವಾಮೀಜಿಯನ್ನು ಅವರು ಹೋದಲ್ಲೆಲ್ಲ ಹಿಂಬಾಲಿಸುತ್ತಿದ್ದರು. ಇವರು ಶ್ರದ್ಧಾವಂತರಾಗಿದ್ದಷ್ಟೇ ಅವಿಶ್ರಾಂತರೂ ಆಗಿದ್ದರು. ರಜಾ ದಿನವೆಂದೋ ಬೇರಾವುದಾದರೂ ಕಾರಣಕ್ಕೋ ಸ್ವಾಮೀಜಿಒಂದು ದಿನದ ಪಾಠವನ್ನು ರದ್ದುಪಡಿಸುವ ಸೂಚನೆ ಕೊಟ್ಟರೆ, ತಕ್ಷಣವೇ ಅದಕ್ಕೆ ಗಟ್ಟಿಯಾದ ಪ್ರತಿಭಟನೆ ಕೇಳಿಬರುತ್ತಿತ್ತು. ಅವರ ಒಂದು ಮಾತನ್ನೂ ಕೇಳಿಸಿಕೊಳ್ಳದೆ ಬಿಡಲು ಯಾರಿಗೂ ಇಷ್ಟವಿರಲಿಲ್ಲ. ಹೀಗಾಗಿ ಸ್ವಾಮೀಜಿಗೆ ವಿರಾಮವೇ ಇರಲಿಲ್ಲ.”

ಇಷ್ಟರಮಟ್ಟಿಗೆ ಯಶಸ್ಸನ್ನು ಸಾಧಿಸಲು ಸ್ವಾಮೀಜಿ ಎಷ್ಟು ಹೋರಾಡಬೇಕಾಯಿತು ಎಂಬು ದನ್ನು ನಾವು ಅದಾಗಲೇ ನೋಡಿದ್ದೇವೆ. ಆದರೆ ಆಗಿನ ಕಾಲದಲ್ಲೇ ಅವರು ಅನುಯಾಯಿಗಳು ಹಾಗೂ ಬೆಂಬಲಿಗರಾಗಿದ್ದ ಎಷ್ಟೋ ಜನರಿಗೆ–ಅದರಲ್ಲೂ ಭಾರತೀಯರಿಗೆ–ಅದರ ಅರಿವಿರ ಲಿಲ್ಲ. ತಾವು ಸಂನ್ಯಾಸಿಗಳು ಅಮೆರಿಕದಲ್ಲಿ ಬದುಕಿ ಉಳಿಯಬೇಕಾದರೆ ಪ್ರತಿಯೊಂದು ಚೂರು ರೊಟ್ಟಿಗೂ ಹೋರಾಡಬೇಕು ಎಂದು ಸ್ವಾಮೀಜಿ ತಮ್ಮ ಗುರಭಾಯಿಗಳಿಗೆ ಬರೆದಿದ್ದರು. ವಸ್ತು ಸ್ಥಿತಿ ನಿಜಕ್ಕೂ ಭಯಂಕರವಾಗಿಯೇ ಇತ್ತು. ಸ್ವಾಮೀಜಿಯ ಕಾರ್ಯದ ನಿಜವಾದ ಮಹತ್ವ ವೆಂಥದು ಎಂಬುದನ್ನು ಅವರ ಶಿಷ್ಯರಾದ ಸ್ವಾಮಿ ಕೃಪಾನಂದರು, ಮದರಾಸಿನ ‘ಬ್ರಹ್ಮವಾದಿನ್​’ ಪತ್ರಿಕೆಗೆ ಕಳಿಸಿಕೊಟ್ಟ ಒಂದು ದೀರ್ಘ ಪತ್ರದಲ್ಲಿ ವಿವರವಾಗಿ ವಿಶ್ಲೇಷಿಸಿದರು. (ಆದರೆ ಸ್ವಾಮೀಜಿಗೆ ಈ ಪತ್ರದ ಬಗ್ಗೆ ಮುಂದಾಗಿಯೇ ತಿಳಿದಿರಲಿಲ್ಲ.) ಸತ್ಯಾಂಶವಾವುದನ್ನೂ ಮರೆ ಮಾಚದೆ ಬರೆದ ಈ ಪತ್ರ, ಆಗಿನ ಕಾಲಕ್ಕೆ ತುಂಬ ಹರಿತವಾಗಿತ್ತು. ಅದನ್ನು ‘ಬ್ರಹ್ಮವಾದಿನ್​’ ಪತ್ರಿಕೆ ಹಾಗೆಯೇ ಪ್ರಕಟಿಸಿತು. ಆ ಲೇಖನದ ಕೆಲಭಾಗಗಳನ್ನು ಇಲ್ಲಿ ನಾವು ಅವಲೋಕಿಸಬಹುದಾಗಿದೆ:

“ಹಿಂದೂಧರ್ಮದ ಸಂದೇಶಗಳನ್ನು ಅಮೆರಿಕದಲ್ಲಿ ಪ್ರಸಾರ ಮಾಡುವಲ್ಲಿ ಸ್ವಾಮಿ ವಿವೇಕಾ ನಂದರು ಸಾಧಿಸಿದ ಅದ್ಭುತ-ಅಪೂರ್ವ ಯಶಸ್ಸು ಸಾಧ್ಯವಾದುದು ಅನೇಕ ಅನೂಕೂಲಕರ ಪರಿಸ್ಥಿತಿಗಳು ಕಾಕತಾಳೀಯವಾಗಿ ಕೂಡಿಬಂದುದರಿಂದ ಮಾತ್ರವೇಹೊರತು, ವಿವೇಕಾನಂದರ ಸ್ವಸ್ವಾಮರ್ಥ್ಯದಿಂದೇನೂ ಅಲ್ಲ ಎಂದು ಯಾರಾದರೂ ತಪ್ಪು ತೀರ್ಮಾನಕ್ಕೆ ಬರುವ ಸಂಭವ ವಿದೆ. ಆದರೆ ನಾವು ಹತ್ತೊಂಬತ್ತನೆಯ ಶತಮಾನದ ಅಂತ್ಯಭಾಗದ ಇಂದಿನ ಕಾಲದಲ್ಲಿ ಅಮೆರಿಕ ದಲ್ಲಿ ಪ್ರಚಲಿತವಿರುವ ವಿಲಕ್ಷಣ ಪರಿಸ್ಥಿತಿಯನ್ನು ಅಧ್ಯಯನ ಮಾಡಬೇಕು; ವೀವೇಕಾನಂದರು ಎದುರಿಸಬೇಕಾದ ವಿರೋಧೀ ಶಕ್ತಿಗಳು ಎಂಥವೆಂಬುದನ್ನು ಗಮನಿಸಬೇಕು; ಇಂತಹ ಭಯಂಕರ ಪರಿಸ್ಥಿತಿಯೊಂದಿಗೆ ಹೋರಾಡುವಲ್ಲಿ ಅವರು ಹೇಗೆ ಅಸಂಖ್ಯಾತ ಕಷ್ಟ-ಸಂಕಟಗಳನ್ನು ಎದುರಿಸ ಬೇಕಾಯಿತೆಂಬುದನ್ನು ಪರಿಗಣಿಸಬೇಕು–ಆಗ ಮಾತ್ರವೇ ನಾವು ಅವರ ಕಾರ್ಯಸಿದ್ಧಿಯ ವೈಭವವನ್ನು ಗುರುತಿಸಿ ಪ್ರಶಂಸಿಸಲು ಸಮರ್ಥರಾಗುತ್ತೇವೆ. ಆಗ ಮಾತ್ರವೇ, ಅವರ ಈ ಯಶಸ್ಸು-ಕಾರ್ಯಸಿದ್ಧಿಗಳೆಲ್ಲ ಲೋಕಗುರುವಾದ ಅವರ ಅಸಾಧಾರಣ ನೈತಿಕ, ಬೌದ್ಧಿಕ ಹಾಗೂ ಆಧ್ಯಾತ್ಮಿಕ ಶಕ್ತಿಯಿಂದಲೇ ಸಾಧ್ಯವಾಯಿತೆಂಬುದನ್ನು ತಿಳಿಯಲು ಸಮರ್ಥರಾಗುತ್ತೇವೆ.

“ಸರ್ವಧರ್ಮ ಸಮ್ಮೇಳನದ ಸಂದರ್ಭದಲ್ಲಿ ಅನೇಕ ಭಾರತೀಯರು ಜಗತ್ತಿನ ಗಮನವನ್ನು ಭಾರತದ ಬೆಳಕಿನೆಡೆಗೆ ಸೆಳೆಯುವಲ್ಲಿ ಯಶಸ್ವಿಯಾದರು ಮತ್ತು ನಮ್ಮ (ಅಮೆರಿಕ) ರಾಷ್ಟ್ರದ ಮೇಲೆಲ್ಲ ಒಂದು ಹೊಸ ಅಲೆ ಹರಿಯುವಂತೆ ಮಾಡಿದರು ಎನ್ನುವುದೇನೋ ನಿಜವೇ. ಆದರೆ ಸ್ವಾಮಿ ವಿವೇಕಾನಂದರು ಮಾತ್ರ, ತಾವು ಹಿಡಿದ ಪಟ್ಟನ್ನು ಬಿಡದೆ ಪಾಶ್ಚಾತ್ಯ ಜಡವಾದದ ಮೇಲೆ ಹಿಂದೂ ಭಾವನೆಗಳನ್ನು ಕಸಿಮಾಡುವ ನಿಟ್ಟಿನಲ್ಲಿ ಶ್ರಮಿಸಿ, ತಮ್ಮ ಕಾರ್ಯವು ಫಲಪ್ರದವಾಗುವ ವರೆಗೂ ದುಡಿದರು. ಈ ಒಬ್ಬ ವ್ಯಕ್ತಿಯ ಪ್ರಯತ್ನಗಳು ಇಲ್ಲದೆ ಹೋಗಿದ್ದ ಪಕ್ಷದಲ್ಲಿ, ಈ ನೂತನ ಅಲೆಯು ಯಾವುದೇ ಶಾಶ್ವತ ಪರಿಣಾಮವನ್ನುಂಟುಮಾಡದೆ, ತಾನು ಉದಿಸಿದಷ್ಟೇ ಶೀಘ್ರವಾಗಿ ಅಂತ್ಯಗೊಳ್ಳುತ್ತಿತ್ತು.

“ಈ ಸಮಯದಲ್ಲಿ ಅಮೆರಿಕನ್ನರ ಮನಸ್ಸು, ಹಳೆಯ ಕಾಲದಿಂದಲೂ ಬೆಳೆದುಬಂದಿದ್ದ ಮೂಢನಂಬಿಕೆ-ಮತಾಂಧತೆಗಳ ದಪ್ಪನೆಯ ಪದರಗಳಿಂದ ಆವೃತವಾಗಿತ್ತು. ಎಲ್ಲ ವಿಧದ ಠಕ್ಕು-ಕಪಟ-ವಂಚನೆಗಳು ಅದರಲ್ಲಿ ಬೇರೂರಿದ್ದು, ಇವುಗಳನ್ನು ಕಿತ್ತುಹಾಕುವುದೆಂದರೆ ಅತೀವ ದುಸ್ಸಾಧ್ಯದ ಕೆಲಸವಾಗಿತ್ತು. ಅಮೆರಿಕನ್ನರು ಸ್ವಭಾವತಃ ಎಲ್ಲವನ್ನೂ ಸ್ವೀಕರಿಸುವ ಜನ. ಆದ್ದ ರಿಂದಲೇ ಈ ದೇಶವು ಎಲ್ಲ ಬಗೆಯ ಧಾರ್ಮಿಕ ಹಾಗೂ ಅಧಾರ್ಮಿಕ ಜಂಜಡಗಳ ನೆಲೆವೀಡಾಗಿ ಇರುವುದು. ಇಲ್ಲಿನ ಪೇಟೆಯಲ್ಲಿ ತಕ್ಷಣವೇ ಬಿಕರಿಯಾಗದ ಯಾವ ಅಸಂಬದ್ಧ ತತ್ತ್ವವಾಗಲಿ, ಅರ್ಥಹೀನ ಬೋಧನೆಯಾಗಲಿ, ಆಗಸದೆತ್ತರದ ಆಶ್ವಾಸನೆಯಾಗಲಿ, ಮುಚ್ಚುಮರೆಯಿಲ್ಲದ ಮೋಸವಾಗಲಿ ಇಲ್ಲವೇ ಇಲ್ಲ ಎನ್ನಬೇಕು. ಇವುಗಳಲ್ಲಿ ಪ್ರತಿಯೊಂದಕ್ಕೂ ಅಸಂಖ್ಯಾತ ಅನುಯಾಯಿಗಳು ಸಿಕ್ಕೇ ಸಿಗುತ್ತಾರೆ. ಮಾಯಾಮಂತ್ರಗಳ ಬಗ್ಗೆ, ಪ್ರೇತವಿದ್ಯೆಯ ಬಗ್ಗೆ, ಅತಿಮಾನುಷ ಶಕ್ತಿಗಳ ಬಗ್ಗೆ ಜನರಲ್ಲಿ ಬೆಳೆದಿರುವ ಈ ಭಯಂಕರ ಆಸಕ್ತಿಯು, ಮಧ್ಯ ಯುಗವನ್ನು (ಸುಮಾರು ಕ್ರಿ. ಶ. ೬ಂಂ ರಿಂದ ೧೪ಂಂರವರೆಗಿನ ಕಾಲ) ಪುನಾರವರ್ತನೆಗೊಳಿಸು ತ್ತದೆ. ಈ ಗೀಳನ್ನು ತೃಪ್ತಿಪಡಿಸಲು, ಜನರ ಮೌಢ್ಯಕ್ಕೆ ತುಪ್ಪ ಸರಿಯಲು ನೂರಾರು ಸಂಘಗಳು, ಪಂಥಗಳು ಹುಟ್ಟಿಕೊಂಡುವು; ಜಗತ್ತಿಗೆ ಮೋಕ್ಷದ ಭರವಸೆಕೊಡುತ್ತ, ‘ಪ್ರವಾದಿ’ಗಳಿಗೆ ಒಬ್ಬೊಬ್ಬ ದೀಕ್ಷಾರ್ಥಿಯಿಂದಲೂ ೨೫ರಿಂದ ೧ಂಂ ಡಾಲರ್​ವರೆಗೂ ದಕ್ಕಿಸಿಕೊಡುತ್ತಿದ್ದುವು. ದೆವ್ವಗಳು, ಪಿಶಾಚಿಗಳು, ‘ಮಹಾತ್ಮ’ರು ಮತ್ತು ಹೊಸ ಹೊಸ ಅವತಾರಪುರುಷರು ದಿನದಿನವೂ ಹುಟ್ಟಿಕೊಳ್ಳುತ್ತಿದ್ದರು. ಚಿತ್ರವಿಚಿತ್ರ ಧಾರ್ಮಿಕ ಭಾವನೆಗಳ ಈ ಹುಚ್ಚರ ಸಂತೆಯಲ್ಲಿ, ಮೋಸ-ಸೋಗು-ವಂಚನೆಗಳ ಈ ದೆವ್ವದ ಪಾಕಶಾಲೆಯಲ್ಲಿ ಸ್ವಾಮಿ ವಿವೇಕಾನಂದರು ಉದಿಸಿ ದರು–ಭವ್ಯವಾದ ವೇದಧರ್ಮವನ್ನು, ವೇದಾಂತದ ಅದ್ಭುತ ತತ್ತ್ವವನ್ನು, ಸನಾತನ ಪುಷಿಗಳ ಪರಮೋದಾತ್ತ ಜ್ಞಾನವನ್ನು ಬೋಧಿಸಲು, ಆ ಕಾರ್ಯಕ್ಕೆ ಅತ್ಯಂತ ಪ್ರತಿಕೂಲವಾದ ಪರಿಸರ! ತಮ್ಮ ಮಹಾಕಾರ್ಯವನ್ನು ಸಾಧಿಸುವ ಮೊದಲು ಅವರು ಠಕ್ಕು-ಮೂಢನಂಬಿಕೆ-ಮತಾಂಧತೆಗಳ ಈ ಕೊಳಕು ಲಾಯವನ್ನು ಶುಚಿಗೊಳಿಸುವ ಭಗೀರಥ ಪ್ರಯತ್ನವನ್ನು ಕೈಗೊಳ್ಳಬೇಕಾಗಿತ್ತು. ಆ ಪ್ರಯತ್ನವೋ ಎಂತಹ ಸಿಡಿಲೆದೆಗೂ ಅಧೈರ್ಯವನ್ನು ತರುವಂಥದು; ಎಂತಹ ಪ್ರಬಲ ಇಚ್ಛೆ ಯನ್ನೂ ನಿಸ್ಸತ್ವಗೊಳಿಸುವಂಥದು. ಆದರೆ ಕಷ್ಟಗಳಿಗೆ ಅಂಜುವ ವ್ಯಕ್ತಿಯಾಗಿರಲಿಲ್ಲ ಸ್ವಾಮೀಜಿ. ಬಡವರೂ ಏಕಾಂಗಿಯೂ ಆಗಿದ್ದು, ಭಗವಂತನ ನೆರವು ಹಾಗೂ ಮಾನವತೆಯ ಮೇಲಿನ ಪ್ರೇಮ–ಇವೆರಡನ್ನು ಬಿಟ್ಟರೆ ಬೇರಾವ ಆಧಾರವೂ ಇಲ್ಲದೆ ಅವರು ಸಮಾಧಾನಚಿತ್ತದಿಂದ ಕಾರ್ಯನಿಮಗ್ನರಾದರು. ತಾವು ಸಾರಲಿದ್ದ ಸಂದೇಶವು ಸತ್ಯಾನ್ವೇಷಿಗಳಾದ ಸ್ತ್ರೀಪುರುಷರೆದೆಯನ್ನು ತಲುಪುವವರೆಗೂ ಹಿಡಿದ ಕಾರ್ಯವನ್ನು ಬಿಡದಿರಲು ಅವರು ನಿಶ್ಚಯಿಸಿದ್ದರು.

“ಪ್ರಾರಂಭದಲ್ಲಿ ಅವರ ಉಪನ್ಯಾಸಗಳಿಗೆ ಜನ ಇನ್ನಿಲ್ಲದಂತೆ ಮುತ್ತಿದರು. ಇವರಲ್ಲಿ ಕೆಲವರು ಕುತೂಹಲಾಕಾಂಕ್ಷಿಗಳು; ಇನ್ನು ಕೆಲವರು ಈ ಮೇಲೆ ಹೇಳಿದಂತಹ ಮೋಸ-ವಂಚನೆ ಗಳ ಕುಟಿಲ ಜನ. ತಮ್ಮ ಹಿತಾಸಕ್ತಿಗಳನ್ನು ಪೂರೈಸಿಕೊಳ್ಳಲು ವಿವೇಕಾನಂದರು ಬಹಳ ಒಳ್ಳೆಯ ಉಪಕರಣವಾಗಬಲ್ಲರೆಂದು ಇವರು ಭಾವಿಸಿದ್ದರು. ಇಂಥವರಲ್ಲಿ ಬಹಳಷ್ಟು ಜನ ಮೊದಲು ಸ್ವಾಮೀಜಿಗೆ ತಮ್ಮ ಬೆಂಬಲದ ಭರವಸೆಯನ್ನಿತ್ತು, ಅವರನ್ನು ತಮ್ಮ ಪಂಥಕ್ಕೆ ಸೇರಿಸಿಕೊಳ್ಳಲು ಪ್ರಯತ್ನಿಸಿದರು. ಅದಕ್ಕೆ ಸ್ವಾಮೀಜಿ ಒಪ್ಪದಿದ್ದಾಗ ಅಪಾಯದ ಬೆದರಿಕೆಯನ್ನೂ ಹಾಕಿದರು. ಆದರೆ ಕೊನೆಯಲ್ಲಿ ಅವರೆಲ್ಲ ಅತ್ಯಂತ ನಿರಾಶರಾಗಬೇಕಾಯಿತು. ಅವರು ಮೊಟ್ಟ ಮೊದಲ ಬಾರಿಗೆ, ತಾವು ಖರೀದಿಸಲಾಗದ ಇಲ್ಲವೆ ಬೆದರಿಸಲಾಗದ ವ್ಯಕ್ತಿಯೊಬ್ಬನನ್ನು ಸಂಧಿಸಿದ್ದರು. ಪೋಲಿಷ್ ಭಾಷೆಯ ಗಾದೆಯಂತೆ ‘ಮಚ್ಚುಗತ್ತಿ ಬಂಡೆಯನ್ನು ಬಡಿದಿತ್ತು’. ಅವರ ಎಲ್ಲ ಬಗೆಯ ಆಹ್ವಾನಗಳಿಗೆ ಸ್ವಾಮೀಜಿಯದು ಒಂದೇ ಉತ್ತರ: ‘ನಾನು ಸತ್ಯವನ್ನು ಪ್ರತಿನಿಧಿಸುತ್ತೇನೆ. ಸತ್ಯ ವೆಂದಿಗೂ ಸುಳ್ಳಿನೊಂದಿಗೆ ಹೊಂದಾಣಿಕೆ ಮಾಡಿಕೊಳ್ಳುವುದಿಲ್ಲ. ಇಡೀ ಜಗತ್ತೇ ನನಗೆ ವಿರುದ್ಧ ವಾದರೂ ಕೊನೆಯಲ್ಲಿ ಸತ್ಯ ಗೆಲ್ಲಲೇಬೇಕು.’ ಮೋಸ-ಮೂಢನಂಬಿಕೆಗಳು ಯಾವುದೇ ರೂಪ ಧರಿಸಿ ಬಂದರೂ ಅವುಗಳನ್ನು ಸ್ವಾಮೀಜಿ ಉಚ್ಛಾಟಿಸಿದರು. ಆ ಎಲ್ಲ ಅಸಂಬದ್ಧತೆ-ಅವಿವೇಕಗಳು ಈ ಸತ್ಯದೂತನ ಮುಂದೆ ಸೂರ್ಯನ ಬೆಳಕನ್ನು ಕಂಡು ಅವಿತುಕೊಳ್ಳುವ ಬಾವಲಿಗಳಂತೆ ಎತ್ತಲೋ ಹುದುಗಿಕೊಂಡುವು.

“ಕ್ರೈಸ್ತ ಮಿಷನರಿಗಳ ವಿವಿಧ ಕಟಿಲೋಪಾಯಗಳು ಎಲ್ಲರಿಗೂ ಚೆನ್ನಾಗಿ ತಿಳಿದಿರುವಂಥವೇ. ಕ್ರೈಸ್ತಧರ್ಮವನ್ನು ಸ್ವಾಮೀಜಿ, ಮಿಷನರಿಗಳು ಅರ್ಥಮಾಡಿಕೊಂಡಿದ್ದ ರೀತಿಯಲ್ಲೇ ಬೋಧಿಸಿ ದ್ದರೆ ಅದು ಅವರಿಗೆ ಪ್ರಿಯವಾಗಿರುತ್ತಿತ್ತು. ಆದರೆ ಅದೆಂದಿಗೂ ಸಾಧ್ಯವಿರಲಿಲ್ಲ. ಮಿಷನರಿಗಳು ತಮ್ಮ ಬಗ್ಗೆ ಹರಡಿದ ಹೊಲಸು ಕತೆಗಳ ವಿಷಯದಲ್ಲಿ ನಿರ್ಲಿಪ್ತರಾಗಿದ್ದು, ಸ್ವಾಮೀಜಿ ಶಾಂತವಾಗಿ ಭಗವಂತ-ಪ್ರೇಮ-ಸತ್ಯಗಳನ್ನು ಬೋಧಿಸಿದರು. ಆ ಎಲ್ಲ ಗಾಳಿಮಾತುಗಳು ಸ್ವಾಮೀಜಿಯ ಉಪನ್ಯಾಸಗಳಿಗೆ ಕೇವಲ ಜಾಹೀರಾತಾಗಿ ಕೆಲಸ ಮಾಡಿದುವು. ಮತ್ತು ಅವರಿಗೆ ವಿವೇಕಿಗಳಾದ ಜನರ ಸಹಾನುಭೂತಿಯನ್ನು ಗಳಿಸಿಕೊಟ್ಟುವು.”

ಸ್ವಾಮಿ ಕೃಪಾನಂದರು ‘ಬ್ರಹ್ಮವಾದಿನ್​’ ಪತ್ರಿಕೆಗೆ ಕಳಿಸಿಕೊಟ್ಟ ಈ ಪತ್ರವು, ೧೮೯೬ರ ಫೆಬ್ರುವರಿ ೧೫ರ ಸಂಚಿಕೆಯಲ್ಲಿ ಪ್ರಕಟವಾಯಿತು. ಸ್ವಾಮೀಜಿಯ ಸಾಧನೆ ನಿಜಕ್ಕೂ ಎಷ್ಟು ಅದ್ಭುತವಾದದ್ದು ಎಂಬುದನ್ನೂ ಅಮೆರಿಕದ ಪ್ರಸ್ತುತ ಪರಿಸ್ಥಿತಿ ಎಷ್ಟು ಭೀಕರವಾಗಿತ್ತೆಂಬು ದನ್ನೂ ಈ ಪತ್ರದಲ್ಲಿ ಅವರು ಸ್ಪಷ್ಟವಾಗಿ ಚಿತ್ರಿಸಿದ್ದರು. ಆದರೆ ಆ ಲೇಖನದ ಶೈಲಿ ಸ್ವಾಮೀಜಿಗೆ ಸ್ವಲ್ಪವೂ ಸರಿಬರಲಿಲ್ಲ. ಏಕೆಂದರೆ ಕಟು ಟೀಕೆಯ ವಿಧಾನವನ್ನು ಅವರೆಂದೂ ಪ್ರೋತ್ಸಾಹಿಸುತ್ತಿರ ಲಿಲ್ಲ. ಆದ್ದರಿಂದ ಅದರ ಬಗ್ಗೆ ಜುಗುಪ್ಸೆಗೊಂಡು ಅವರು, ಪತ್ರಿಕೆಯ ಪ್ರಕಾಶಕರಾದ ಅಳಸಿಂಗ ಪೆರಮಾಳರಿಗೆ ಆ ಬಗ್ಗೆ ಒಂದು ಖಾರವಾದ ಪತ್ರ ಬರೆದರು:

“ನೀನು ‘ಬ್ರಹ್ಮವಾದಿನ್​’ ಪತ್ರಿಕೆಯಲ್ಲಿ ಕೃಪಾನಂದರ ಪತ್ರವನ್ನು ಪ್ರಕಟಿಸಿದುದು ದುರದೃಷ್ಟ ಕರ. ಇಲ್ಲಿನ ಕ್ರೈಸ್ತರು ಅವರನ್ನು ತರಾಟೆಗೆ ತೆಗೆದುಕೊಂಡಿದ್ದರಿಂದ ಅವರು ಸಿಟ್ಟಿಗೆದ್ದು ಆ ಬಗೆಯ ಪತ್ರ ಬರೆದಿದ್ದಾರೆ. ಆದರೆ ಎಲ್ಲವನ್ನೂ ಹಳ್ಳಕ್ಕೆ ತಳ್ಳುವಂತಹ ಆ ಪತ್ರ ಅಶ್ಲೀಲವೇ ಸರಿ. ಅದು ‘ಬ್ರಹ್ಮವಾದಿನ್​’ ಪತ್ರಿಕೆಗೆ ಹೊಂದಿಕೆಯಾಗುವಂಥದಲ್ಲ. ಆದ್ದರಿಂದ ಇನ್ನು ಮುಂದೆ ಸ್ವಾಮಿ ಕೃಪಾನಂದರು ಪತ್ರ ಬರೆದರೆ, ಅದರಲ್ಲಿ ಯಾವುದೇ ಮತ-ಪಂಥವನ್ನು ಖಂಡಿಸಿ ಬರೆದ ಅಂಶಗಳನ್ನು–ಆ ಪಂಥಗಳು ಎಷ್ಟೇ ವಿಚಿತ್ರ-ವಿಕೃತವಾದದ್ದಾಗಿರಲಿ–ಸಂಪೂರ್ಣವಾಗಿ ತಿದ್ದ ಬೇಕು, ಇಲ್ಲವೆ ತೆಗೆದುಹಾಕಬೇಕು. ಯಾವುದೇ ಪಂಥಕ್ಕೆ ವಿರುದ್ಧವಾದುದಾವುದೂ ‘ಬ್ರಹ್ಮ ವಾದಿನ್​’ ಪತ್ರಿಕೆಯಲ್ಲಿ ಪ್ರಕಟವಾಗಬಾರದು. ಆದರೆ, ಧೂರ್ತರಿಗೆ ಸಹಾನುಭೂತಿ ತೋರ ಬೇಕೆಂದು ಇದರ ಅರ್ಥವಲ್ಲ.”

ಕೃಪಾನಂದರು ಬರೆದ ಈ ಇಡೀ ಪತ್ರ ತಮ್ಮ ಘನತೆಯನ್ನು ಎತ್ತಿಹಿಡುವಂಥದಾಗಿದ್ದರೂ ಸ್ವಾಮೀಜಿ ಅದನ್ನು ‘ಅಶ್ಲೀಲ’ವೆಂದು ಕರೆಯುತ್ತಾರೆ! ಇಲ್ಲಿ ಅವರ ಸಭ್ಯತೆ-ಸೌಜನ್ಯಗಳು ಮತ್ತೊಮ್ಮೆ ಸ್ಪಷ್ಟವಾಗಿ ವ್ಯಕ್ತವಾಗುತ್ತವೆ. ಅಲ್ಲದೆ, ಕೇವಲ ಧರ್ಮಪ್ರಸಾರವನ್ನು ಉದ್ದೇಶವಾಗಿ ಇಟ್ಟುಕೊಂಡು, ತಮ್ಮ ಪ್ರೋತ್ಸಾಹದಿಂದಲೇ ಪ್ರಾರಂಭವಾದ ‘ಬ್ರಹ್ಮವಾದಿನ್​’ ಪತ್ರಿಕೆಯು ಇಂತಹ ‘ಅಶ್ಲೀಲ’ ಲೇಖನಗಳನ್ನು ಪ್ರಕಟಿಸಬಾರದು; ಮತ್ತು ಯಾವುದೇ ಕಾರಣಕ್ಕಾಗಿ ಪರಮತ ಗಳನ್ನು ಟೀಕಿಸುವ ವಾಕ್ಯಗಳಂತೂ ಅದರಲ್ಲಿ ಇರಲೇಬಾರದು ಎಂದು ಅವರು ಸೂಚನೆ ನೀಡು ದುದು ಅವರಲ್ಲಿ ಆಚಾರ್ಯಪುರುಷನ ಲಕ್ಷಣಗಳನ್ನು ಎತ್ತಿಹಿಡಿಯುವಂತಿದೆ.

ಅಮೆರಿಕದಲ್ಲಿದ್ದುಕೊಂಡೇ ಸ್ವಾಮೀಜಿ ಭಾರತದಲ್ಲಿ ಸಾಧಿಸಿದ್ದ ಅತಿಮುಖ್ಯ ಕಾರ್ಯಗಳ ಲ್ಲೊಂದೆಂದರೆ ‘ಬ್ರಹ್ಮವಾದಿನ್​’ ಪತ್ರಿಕೆಯ ಪ್ರಾರಂಭ. ಭಾರತದ ಪುನರ್​ಜಾಗೃತಿಯ ಹಾಗೂ ಧರ್ಮಪ್ರಸಾರದ ಕಾರ್ಯದಲ್ಲಿ ನಿಯತಕಾಲಿಕ ಪತ್ರಿಕೆಯೊಂದು ಎಷ್ಟು ಮಹತ್ವದ ಪಾತ್ರ ವಹಿಸಬಲ್ಲುದೆಂಬುದು ಅವರಿಗೆ ಚೆನ್ನಾಗಿ ತಿಳಿದಿತ್ತು. ಆದ್ದರಿಂದ ಅವರು ತಮ್ಮ ಮದ್ರಾಸೀ ಶಿಷ್ಯರಿಗೆ ಪತ್ರಗಳನ್ನು ಬರೆದು, ವೇದಾಂತ ಪ್ರಸಾರಕ್ಕಾಗಿ ಪತ್ರಿಕೆಯೊಂದನ್ನು ಪ್ರಾರಂಭಿಸುವಂತೆ ಬಹಳ ಕಾಲದಿಂದಲೂ ಒತ್ತಾಯಿಸುತ್ತಿದ್ದರು. ಅದರಂತೆಯೇ ತಮ್ಮ ಸೋದರಸಂನ್ಯಾಸಿಗಳಿಗೂ ಪತ್ರಗಳನ್ನು ಬರೆದು, ಕಲ್ಕತ್ತದಲ್ಲೊಂದು ಪತ್ರಿಕೆಯನ್ನು ಹೊರಡಿಸುವಂತೆ ಹೇಳುತ್ತಿದ್ದರು. ಮದ್ರಾಸಿನ ಪತ್ರಿಕೆಗಾಗಿ ಅವರು ಹಣವನ್ನೂ ಕಳಿಸಿದರು. ಅಂತೂ ಅವರು ಅಸಹನೆಯಿಂದ ಸಾಕಷ್ಟು ಕಾಲ ಕಾದಮೇಲೆ ೧೮೯೫ರ ಸೆಪ್ಟೆಂಬರ್ ತಿಂಗಳಿನಲ್ಲಿ ‘ಬ್ರಹ್ಮವಾದಿನ್​’ ಎಂಬ ಆಂಗ್ಲ ಮಾಸ ಪತ್ರಿಕೆ ಅಸ್ತಿತ್ವಕ್ಕೆ ಬಂದಿತು. ಪತ್ರಿಕೆ ಬಿಡುಗಡೆಯಾದ ಮೇಲೂ ಸ್ವಾಮೀಜಿ, ಅದರ ಪ್ರಕಾಶಕರಾದ ಅಳಸಿಂಗರಿಗೆ ಮತ್ತೆಮತ್ತೆ ಅನೇಕ ಸಲಹೆ-ಸೂಚನೆಗಳನ್ನು ಕೊಡುತ್ತ, ಕೆಲವೊಮ್ಮೆ ಹೊಗಳುತ್ತ, ಕೆಲವೊಮ್ಮೆ ಕಟುವಾಗಿ ಟೀಕಿಸುತ್ತ ಪತ್ರಿಕೆಯ ಶ್ರೇಯಸ್ಸಿಗಾಗಿ ಶ್ರಮಿಸಿದರು.

ಈ ವೇಳೆಗೆ ಅವರು ನ್ಯೂಯಾರ್ಕಿನಲ್ಲಿ ಸಾರ್ವಜನಿಕ ಉಪನ್ಯಾಸಕರಾಗಿ ಅತ್ಯಂತ ಯಶಸ್ವಿ ಯಾಗಿದ್ದರು. ಅವರನ್ನು ಜನ ‘ಮಿಂಚಿನ ಮಾತುಗಾರ’ನೆಂದು ಕರೆಯುತ್ತಿದ್ದರು. ಭಾನುವಾರದ ಉಪನ್ಯಾಸಗಳ ಮೊದಲ ಸರಣಿಗೆ ಜನ ಕಿಕ್ಕಿರಿಯುತ್ತಿದ್ದರು. ಆದ್ದರಿಂದ ಈಗ ಮತ್ತೊಂದು ಸರಣಿಯನ್ನು ಪ್ರಾರಂಭಿಸಲು ಸ್ವಾಮೀಜಿ ನಿರ್ಧರಿಸಿದರು. ಸಾಧ್ಯವಾದಷ್ಟು ಹೆಚ್ಚು ಜನರಿಗೆ ವೇದಾಂತದ ಬಗ್ಗೆ ತಿಳಿಯುವಂತೆ ಮಾಡುವುದೇ ಈ ಉಪನ್ಯಾಸಗಳ ಉದ್ದೇಶವಾಗಿತ್ತು. ಆದ್ದ ರಿಂದ ‘ಮ್ಯಾಡಿಸನ್ ಸ್ಕ್ವೇರ್ ಗಾರ್ಡನ್​’ ಎಂಬ ಭವ್ಯವಾದ ಸಭಾಂಗಣವನ್ನು ಬಾಡಿಗೆಗೆ ತೆಗೆದು ಕೊಳ್ಳಲಾಯಿತು. ಇದರಲ್ಲಿ ಸಾವಿರದ ಐನೂರಕ್ಕೂ ಹೆಚ್ಚು ಜನರಿಗೆ ಸ್ಥಳಾವಕಾಶವಿತ್ತು. ಈ ಸಭಾಂಗಣದಲ್ಲಿ ಸ್ವಾಮೀಜಿ ‘ಭಕ್ತಿಯೋಗ’ ‘ನೈಜ ಹಾಗೂ ತೋರಿಕೆಯ ಮಾನವ’ ಹಾಗೂ ‘ನನ್ನ ಗುರುದೇವ, ಶ್ರೀರಾಮಕೃಷ್ಣ ಪರಮಹಂಸ’ ಎಂಬ ವಿಷಯಗಳ ಕುರಿತಾಗಿ ಮಾತನಾಡಿದರು.

ಈ ಸರಣಿಯ ಕಡೆಯ ಭಾಷಣವು ಆ ಸಂದರ್ಭಕ್ಕೆ ಬಹಳ ಚೆನ್ನಾಗಿ ಹೊಂದಿಕೆಯಾಗುತ್ತಿತ್ತು. ಅಲ್ಲದೆ ಅದು ಶ್ರೀರಾಮಕೃಷ್ಣರ ಜನ್ಮದಿನದ ಸಂದರ್ಭವೂ ಆಗಿತ್ತು. ಸ್ವಾಮೀಜಿ ತಮ್ಮ ಗುರು ದೇವನ ಬಗ್ಗೆ ಬಹಿರಂಗ ಸಭೆಯಲ್ಲಿ ಮಾತನಾಡಿದುದು ಬಹುಶಃ ಅದೇ ಮೊದಲ ಸಲ. ಅಂದು ಅವರು ಭಾವೋನ್ಮತ್ತರಾದಂತೆ ಕಂಡುಬರುತ್ತಿತ್ತು. ಅಂದಿನ ಭಾಷಣವು ಅವರ ಭಾಷಣಗಳಲ್ಲೆಲ್ಲ ಅತ್ಯಂತ ಸ್ಮರಣೀಯವಾದವುಗಳಲ್ಲೊಂದು. ಆ ಸಂದರ್ಭವನ್ನು ನೆನೆಸಿಕೊಂಡು ಸೋದರಿ ದೇವಮಾತಾ ಬರೆದಳು:

“ಕಡೆಯ ಉಪನ್ಯಾಸದ ದಿನ ಸಭಾಂಗಣವು ಕಾಲಿಡಲೂ ಸ್ಥಳವಿಲ್ಲದಂತೆ ತುಂಬಿಹೋಗಿತ್ತು– ಪ್ರತಿಯೊಂದು ಕುರ್ಚಿ, ನಿಂತುಕೊಳ್ಳುವ ಸ್ಥಳ ಎಲ್ಲವೂ ಭರ್ತಿಯಾಗಿತ್ತು... ಸಭಾಂಗಣದ ಪಕ್ಕದ ಬಾಗಿಲೊಂದರಿಂದ ಸ್ವಾಮೀಜಿ ಒಳಪ್ರವೇಶಿಸುತ್ತಿದ್ದಂತೆ ಅವರು ಬೇರೊಂದು ಭಾವದಲ್ಲಿ ಇದ್ದಂತೆ ಕಂಡಿತು. ಅವರು ವೇದಿಕೆಯ ಮೇಲೆ ನಿಂತಾಗ ಏಕೋ ಹಿಂಜರಿಯುತ್ತಿರುವಂತೆ ಅಧೈರ್ಯಗೊಂಡಂತೆ, ಭಾಸವಾಯಿತು... ಅವರು ತಮ್ಮ ಉಪನ್ಯಾಸವನ್ನು ದೀರ್ಘವಾದ (ಚಾರಿತ್ರಿಕ ಹಿನ್ನೆಲೆಯ) ಪೀಠಿಕೆಯೊಂದಿಗೆ ಪ್ರಾರಂಭಿಸಿದರು. ಆದರೆ ಒಮ್ಮೆ ವಿಷಯಕ್ಕೆ ಬಂದ ಕೂಡಲೇ, ಅವರು ಆದರಲ್ಲಿ ಕೊಚ್ಚಿಹೋದರು. ಅದರ ರಭಸವು ಅವರನ್ನು ವೇದಿಕೆಯ ಒಂದು ಮೂಲೆಯಿಂದ ಇನ್ನೊಂದು ಮೂಲೆಗೆ ಎಳೆಯುತ್ತಿತ್ತು. ಅವರ ಭಾವವು ವೇಗವಾಗಿ ಹರಿಯುವ ವಾಗ್ಝರಿಯ ಮೂಲಕ ಪ್ರವಹಿಸಿತು. ಜನಸ್ತೋಮವು ವಿಸ್ಮಯಮೂಕವಾಗಿ ಆಲಿಸಿತು. ಭಾಷಣ ಮುಗಿದಾಗ ಎಷ್ಟೋ ಜನ ಒಂದು ಮಾತನ್ನೂ ಆಡದೆ ಅಲ್ಲಿಂದ ನೇರವಾಗಿ ನಡೆದುಬಿಟ್ಟರು. ನಾನಂತೂ ದಿಗ್ಭ್ರಾಂತಳಾಗಿ ಕುಳಿತಿದ್ದೆ. ಅವರು ಚಿತ್ರಿಸಿದ ಅಲೌಕಿಕ ಚಿತ್ರ ನನ್ನನ್ನು ಸ್ತಂಭಿತಳನ್ನಾಗಿ ಸಿತು. ನನಗೆ (ತ್ಯಾಗ ಮಾಡಲು) ಕರೆ ಬಂದಿತ್ತು, ಮತ್ತು ನಾನದಕ್ಕೆ ಓಗೊಟ್ಟೆ.”

ಆದರೆ ಅತ್ಯಂತ ಅದ್ಭುತವೂ ಮತ್ತು ಚರಿತ್ರಾರ್ಹವೂ ಆದ ಈ ಉಪನ್ಯಾಸದ ಬಗ್ಗೆ ಸ್ವಾಮೀಜಿಗೆ ಒಂದು ಬಗೆಯ ಜುಗುಪ್ಸೆಯಿತ್ತು. ಆ ಉಪನ್ಯಾಸವನ್ನು ಪ್ರಕಟಿಸಬಾರದೆಂದು ಅವರು ನಿಷೇಧಿಸಿದ್ದರು. ಈ ಉಪನ್ಯಾಸದಲ್ಲಿ ಅವರು ಪಾಶ್ಚಾತ್ಯ ಭೋಗವಾದವನ್ನು ಉಗ್ರವಾಗಿ ಖಂಡಿಸಿದ್ದುದೇ ಅದಕ್ಕೆ ಕಾರಣವೆಂದು ತೋರುತ್ತದೆ. ಆ ಬಗ್ಗೆ ಅವರೇ ಮುಂದೊಮ್ಮೆ ಹೇಳಿದರು, “ನಾನದನ್ನು ಪ್ರಕಟಿಸಗೊಡಲಿಲ್ಲ, ಏಕೆಂದರೆ ನಾನು ನನ್ನ ಗುರುದೇವರಿಗೆ ಅನ್ಯಾಯ ಮಾಡಿದ್ದೆ. ನನ್ನ ಗುರುದೇವರು ಎಂದೂ ಯಾರನ್ನೂ ದೂಷಿಸಿದವರಲ್ಲ. ಆದರೆ ನಾನು ಅವರ ಬಗ್ಗೆ ಮಾತನಾಡುವಾಗ ಅಮೆರಿಕದ ಜನರ‘ಡಾಲರ್​ಆರಾಧನೆಯ’ ಬುದ್ಧಿಯನ್ನು ಟೀಕಿಸಿಬಿಟ್ಟೆ. ನನ್ನ ಗುರುವಿನ ಬಗ್ಗೆ ಮಾತನಾಡುವಷ್ಟು ಯೋಗ್ಯತೆ ನನಗಿನ್ನೂ ಬಂದಿಲ್ಲವೆಂಬ ಪಾಠವನ್ನು ಅಂದು ನಾನು ಕಲಿತುಕೊಂಡೆ.”

ಆದರೆ ಆ ಉಪನ್ಯಾಸದಲ್ಲಿ ಸ್ವಾಮೀಜಿ, ಪಾಶ್ಚಾತ್ಯರ ಭೋಗವಾದವನ್ನು ಟೀಕಿಸಿದುದಕ್ಕೂ ಒಂದು ಕಾರಣವಿತ್ತು. ತಮ್ಮ ಗುರುದೇವನ ಕುರಿತಾದ ಆಲೋಚನೆಯಿಂದ ಉನ್ನತ ಭಾವಾವಸ್ಥೆ ಗೇರಿದ ಸ್ವಾಮೀಜಿ, ಸಭಾಂಗಣದಲ್ಲಿ ನೆರೆದಿದ್ದ ಅತ್ಯಂತ ಪ್ರಾಪಂಚಿಕ ಬುದ್ಧಿಯ ಸಹಸ್ರಾರು ಸ್ತ್ರೀಪುರುಷರನ್ನು ಕಂಡಾಗ ಅವರಲ್ಲಿ ಒಂದು ಬಗೆಯ ಜುಗುಪ್ಸೆ ಉಂಟಾಯಿತು ಎಂದು ಹೇಳ ಲಾಗಿದೆ. ಆದರೆ ಆ ವಿಷಯ ಹೇಗೇ ಇರಲಿ, ಶ್ರೀರಾಮಕೃಷ್ಣರ ಬಗ್ಗೆ ಯಾವಾಗ ಮಾತನಾಡಬೇಕಾ ದರೂ ಸ್ವಾಮೀಜಿ ಹಿಂದೇಟು ಹಾಕುತ್ತಿದ್ದರು. ತಾವು ಶ್ರೀರಾಮಕೃಷ್ಣರನ್ನು ತಪ್ಪಾಗಿ ಪ್ರತಿನಿಧಿಸ ಬಹುದು, ಅವರ ಬಗ್ಗೆ ಮಾತನಾಡಲು ತಮಗೆ ಸಾಮರ್ಥ್ಯವೂ ಯೋಗ್ಯತೆಯೂ ಸಾಲದು ಎಂದೆಲ್ಲ ಸ್ವಾಮೀಜಿ ಹೆದರುತ್ತಿದ್ದರು. ಇಷ್ಟೆಲ್ಲ ಆದರೂ, ಅದೃಷ್ಟವಶಾತ್ ಅವರು ಅಂದು ನೀಡಿದ ಉಪನ್ಯಾಸವನ್ನು ಅವರ ಕೆಲವರು ಶಿಷ್ಯರು ಬರೆದಿಟ್ಟುಕೊಂಡದ್ದರಿಂದ ಇಂದು ನಮಗೆ ಲಭ್ಯವಾಗಿದೆ. (ಈ ಉಪನ್ಯಾಸವನ್ನೂ ಇದೇ ವಿಷಯದ ಮೇಲೆ ಅವರು ಮುಂದೆ ಲಂಡನ್ನಿನಲ್ಲಿ ನೀಡಿದ ಮತ್ತೊಂದು ಉಪನ್ಯಾಸವನ್ನೂ ಸೇರಿಸಿ ‘ನನ್ನ ಗುರುದೇವ’ ಎಂಬ ಶೀರ್ಷಿಕೆಯಲ್ಲಿ ಅವರ ‘ಕೃತಿಶ್ರೇಣಿ’ಯಲ್ಲಿ ಪ್ರಕಟಿಸಲಾಗಿದೆ.)

ಅವರ ಉಪನ್ಯಾಸಗಳು ಆ ದಿನಗಳಲ್ಲಿ ಸಾರ್ವಜನಿಕವಾಗಿ ಬೀರುತ್ತಿದ್ದ ಅದ್ಭುತ ಪ್ರಭಾವದ ಕುರಿತಾಗಿ ಸ್ವಾಮಿ ಕೃಪಾನಂದರು ‘ಬೃಹ್ಮವಾದಿನ್​’ಗೆ ಬರೆದ ಒಂದು ಪತ್ರದಲ್ಲಿ ತಿಳಿಸಿದರು:

“ಸ್ವಾಮೀಜಿಯ ಉಪನ್ಯಾಸಗಳ ಹಾಗೂ ಬರವಣಿಗೆಗಳ ಮೂಲಕ ಹೊರಹೊಮ್ಮುವ ಶಕ್ತಿ ಯುತ ಧಾರ್ಮಿಕ ಚಿಂತನಪ್ರವಾಹವು, ಸದ್ದಿಲ್ಲದಂತೆ ಅಗೋಚರವಾಗಿ ಕೆಲಸ ಮಾಡುತ್ತಿದ್ದರೂ ಜನಸಾಮಾನ್ಯರ ಮನಸ್ಸಿನ ಮೇಲೆ ಅಚ್ಚಳಿಯದ ಪರಿಣಾಮವನ್ನುಂಟುಮಾಡುತ್ತಿದೆ; ಮತ್ತು ಸಮಾಜದ ಆಧ್ಯಾತ್ಮಿಕ ಉನ್ನತಿಗೆ ಅದೊಂದು ಮಹತ್ವದ ಅಂಶವಾಗಿದೆ. ವೇದಾಂತ ಸಾಹಿತ್ಯಕ್ಕಾಗಿ ಬೆಳೆಯುತ್ತಿರುವ ಬೇಡಿಕೆಯನ್ನು ಗಮನಿಸಿದಾಗ, ಮತ್ತು ಯಾರ ಬಾಯಲ್ಲಿ ಸಂಸ್ಕೃತ ಪದಗಳು ಬರಬಹುದೆಂದು ಊಹಿಸಲೂ ಸಾಧ್ಯವಿಲ್ಲವೋ ಅಂತಹ ಜನ ಅವುಗಳನ್ನು ಆಗಾಗ ಬಳಸುವು ದನ್ನು ಕಂಡಾಗ ಈ ಪರಿಣಾಮ ಸುವ್ಯಕ್ತವಾಗುತ್ತದೆ. ‘ಆತ್ಮ’ ‘ಕರ್ಮ’ ‘ಯೋಗ’ ‘ಮೋಕ್ಷ’ ಇತ್ಯಾದಿ ಪದಗಳು ಮನೆಮಾತಾಗಿವೆ; ಶಂಕರಾಚಾರ್ಯ-ರಾಮಾನುಜಾಚಾರ್ಯರ ಹೆಸರುಗಳು ಹಕ್ಸ್​ಲೀ ಹಾಗೂ ಸ್ಪೆನ್ಸರರ ಹೆಸರುಗಳಷ್ಟೇ ಸುಪರಿಚಿತವಾಗುತ್ತಿವೆ. ಸಾರ್ವಜನಿಕ ಗ್ರಂಥಾಲಯಗಳು ಭಾರತದ ಉಲ್ಲೇಖವುಳ್ಳ ಪುಸ್ತಕಗಳನ್ನೆಲ್ಲ ಪಡೆದುಕೊಳ್ಳಲು ಆತುರ ತೋರುತ್ತಿವೆ. ಮ್ಯಾಕ್ಸ್ ಮುಲ್ಲರ್, ಕೋಲ್​ಬ್ರೂಕ್, ಡಾಯ್ಸನ್, ಬರ್ನಾಫ್ ಮೊದಲಾದ ಪ್ರಸಿದ್ಧ ಲೇಖಕರ, ಹಾಗೂ ಹಿಂದೂ ತತ್ತ್ವಶಾಸ್ತ್ರದ ಬಗ್ಗೆ ಬರೆದಿರುವ ಎಲ್ಲ ಲೇಖಕರ ಪುಸ್ತಕಗಳೂ ವೇಗವಾಗಿ ಮಾರಾಟ ವಾಗುತ್ತಿವೆ. ತಲೆ ಚಿಟ್ಟು ಹಿಡಿಸುವ ಶೋಪೆನ್​ಹಾರ್​ನ ಪುಸ್ತಕಗಳನ್ನೂ ಜನರು ಅವನ ವೇದಾಂತದ ಹಿನ್ನೆಲೆಯಿಂದಾಗಿ ಅತ್ಯಂತ ಕುತೂಹಲದಿಂದ ಅಧ್ಯಯನ ಮಾಡುತ್ತಿದ್ದಾರೆ.

“ಒಂದು ಧರ್ಮವಾಗಿ ಹೃದಯಕ್ಕೂ ಒಂದು ತತ್ತ್ವಶಾಸ್ತ್ರವಾಗಿ ಬುದ್ಧಿಗೂ ಒಪ್ಪಿಗೆಯಾಗುವ ಮತ್ತು ಮಾನವನ ಧಾರ್ಮಿಕ ತೃಷೆಯನ್ನು ತೃಪ್ತಿಪಡಿಸಬಲ್ಲ ಹಿಂದೂ ಸಿದ್ಧಾಂತವನ್ನು ಜನಗಳು ಬಹುಬೇಗನೆ ಮೆಚ್ಚಿಕೊಳ್ಳುತ್ತಿದ್ದಾರೆ. ಅದರಲ್ಲೂ, ತಮ್ಮ ಅದ್ಭುತ ವಾಕ್​ಶಕ್ತಿಯಿಂದ ಮಾನವ ನಲ್ಲಿ ಸುಪ್ತವಾಗಿರುವ ದಿವ್ಯಪ್ರೇಮವನ್ನು ಜಾಗೃತಗೊಳಿಸಬಲ್ಲ ಮತ್ತು ತಮ್ಮ ಹರಿತವಾದ ಹಾಗೂ ಅಲ್ಲಗಳೆಯಲಾಗದ ತರ್ಕದಿಂದ ಅತ್ಯಂತ ವೈಜ್ಞಾನಿಕ ಬುದ್ಧಿಯುಳ್ಳವನ ಅತಿ ಮೊಂಡು ಬುದ್ಧಿಯನ್ನೂ ಸುಲಭವಾಗಿ ಗೆದ್ದುಕೊಳ್ಳಬಲ್ಲ ನಮ್ಮ ಗುರುಗಳಂಥವರೊಬ್ಬರು ಈ ಸಿದ್ಧಾಂತ ವನ್ನು ವಿವರಿಸುವಾಗಲಂತೂ ಇದು ಸರಿಯೇ ಸರಿ. ಆದ್ದರಿಂದ, ಸಮಾಜದ ಎಲ್ಲ ವರ್ಗಗಳಲ್ಲೂ ಹಿಂದೂ ಚಿಂತನೆಗಳ ಕುರಿತಾಗಿ ಆಸಕ್ತಿಯುಂಟಾಗಿರುವುದರಲ್ಲಿ ಅಚ್ಚರಿಯೇನೂ ಇಲ್ಲ.”

ಸ್ವಾಮೀಜಿ ನ್ಯೂಯಾರ್ಕಿನಲ್ಲಿದ್ದಾಗ ವಿಜ್ಞಾನಿಗಳು-ತತ್ವಶಾಸ್ತ್ರಜ್ಞರು ಸೇರಿದಂತೆ ಅನೇಕ ಸುಪ್ರಸಿದ್ಧ ವ್ಯಕ್ತಿಗಳು ಅವರ ಉಪನ್ಯಾಸಗಳಿಗೆ ಮತ್ತು ಅವರನ್ನು ವೈಯಕ್ತಿಕವಾಗಿ ಭೇಟಿಯಾಗಲು ಬರುತ್ತಿದ್ದರು. ಕೆಲವರು ಅವರ ಆಪ್ತ ಸ್ನೇಹಿತರಾದರು. ಇನ್ನು ಕೆಲವರು ಅವರ ಶಿಷ್ಯರಾದರು. ಹೀಗೆ ಬರುತ್ತಿದ್ದವರಲ್ಲಿ, ಆ ಕಾಲದಲ್ಲಿ ಯಶಸ್ಸು ಕೀರ್ತಿಗಳ ಶಿಖರವನ್ನು ಮುಟ್ಟಿದ ತರುಣ ವಿಜ್ಞಾನಿ ನಿಕೋಲಾ ಟೆಸ್ಲಾ ಒಬ್ಬನು. ಆದಾಗಲೇ ಅವನು ರೇಡಿಯೋ ಅಲೆಗಳ ಹಾಗೂ ದೂರ ನಿಯಂತ್ರಣದ ಮೂಲತತ್ತ್ವಗಳನ್ನು ಕಂಡುಹಿಡಿದಿದ್ದ; ಟ್ಯೂಬ್​ಲೈಟನ್ನು ಕಂಡುಹಿಡಿದಿದ್ದ. ಮತ್ತು ವಿದ್ಯುಚ್ಛಕ್ತಿಯನ್ನು ದೂರದೂರಕ್ಕೆ ಸಾಗಿಸುವ ವಿಧಾನವನ್ನು ಆವಿಷ್ಕರಿಸಿದ್ದ. ಅಲ್ಲದೆ ವಿಶ್ವದ ಮೂಲಭೂತ ನಿಯಮಗಳ ಹಿಂದಿರುವ ಗಣಿತದ ಪ್ರಮೇಯಗಳ ಬಗ್ಗೆಯೂ ಈತ ಸಂಶೋಧನೆ ನಡೆಸುತ್ತಿದ್ದ. ಸಾಂಖ್ಯದರ್ಶನದ ಸೃಷ್ಟಿಕ್ರಮ ವಿಚಾರದ ಬಗೆಗಿನ ಸ್ವಾಮೀಜಿಯ ವಿವರಣೆಯನ್ನು ಕೇಳಿ ನಿಕೋಲಾ ಟೆಸ್ಲಾ ತೀವ್ರವಾಗಿ ಆಕರ್ಷಿತನಾದ. ಸೃಷ್ಟಿಯ ಬಗೆಗಿನ ಸಮಸ್ಯೆ ಗಳ ಪರಿಹಾರಕ್ಕೆ ಆಧುನಿಕ ವಿಜ್ಞಾನವು ವೇದಾಂತದತ್ತ ತಿರುಗಬಹುದೆಂದು ಇವನಿಗೆ ಅನ್ನಿಸಿತು. ಸ್ವಾಮೀಜಿಯ ತರಗತಿಗಳಿಗೂ ಉಪನ್ಯಾಸಗಳಿಗೂ ಹಾಜರಿರುತ್ತಿದ್ದ ಈ ವಿಜ್ಞಾನಿ ಕ್ರಮೇಣ ಅವರ ಆತ್ಮೀಯ ಸಂರ್ಪಕ್ಕೆ ಬಂದ. ವೇದಾಂತದಲ್ಲಿ ಪ್ರತಿಪಾದಿತವಾಗಿರುವ ಈ ವಿಚಾರಗಳನ್ನು ತಾನು ಗಣಿತದ ದೃಷ್ಟಿಯಿಂದ ವೈಜ್ಞಾನಿಕವಾಗಿ ಸಾಬೀತುಪಡಿಸಬಲ್ಲೆ ಎಂದು ಈತ ಅವರಿಗೆ ಹೇಳಿದ. ಆದರೆ ಅದನ್ನು ವೈಜ್ಞಾನಿಕವಾಗಿ ಸಾಬೀತುಪಡಿಸಲು ಇಂದಿನವರೆಗೂ ಸಾಧ್ಯವಾಗಿಲ್ಲ.

ಈ ಸಂದರ್ಭದಲ್ಲೇ ಸುಪ್ರಸಿದ್ಧಿ ಫ್ರೆಂಚ್ ನಟಿಯಾದ ಸಾರಾ ಬರ್ನ್​ಹಾರ್ಟ್ ಎಂಬವಳು ಸ್ವಾಮೀಜಿಯ ದರ್ಶನವನ್ನು ಬಯಸಿಬಂದಳು. ಆಗ ಅವಳು ಜನಪ್ರಿಯತೆಯ ತುತ್ತತುದಿಗೇರಿ ದ್ದಳು. ತನ್ನ ಅಸಾಮಾನ್ಯ ಅಭಿನಯದಿಂದ \eng{‘Divine Sarah’ (}ದೈವಿಕ ಸಾರಾ) ಎಂದು ಹೆಸರಾಗಿ ದ್ದಳು. ವಿವೇಕಾನಂದರ ಅತ್ಯಂತ ಉದಾತ್ತವಾದ ಬೋಧನೆಗಳಿಂದ ಆಕರ್ಷಿತಳಾದ ಈಕೆ, ಅವರನ್ನು ಕಂಡು ಅವರ ಬೋಧನೆಗಳ ಬಗ್ಗೆ ತನ್ನ ಹೃತ್ಪೂರ್ವಕ ಮೆಚ್ಚುಗೆಯನ್ನು ವ್ಯಕ್ತಪಡಿಸಿ ದಳು. ಈ ದಿನಗಳಲ್ಲಿ ಸ್ವಾಮೀಜಿ ಭೇಟಿ ಮಾಡಿದ ಗಣ್ಯ ವ್ಯಕ್ತಿಗಳಲ್ಲಿ, ಉದಾರವಾದಿಗಳಾದ ಅನೇಕ ಪ್ರಮುಖ ಕ್ರೈಸ್ತ ಧರ್ಮಾಧಿಕಾರಿಗಳಿದ್ದರು. ಪ್ರಸಿದ್ಧನೂ ಪ್ರಭಾವಶಾಲಿಯೂ ಆದ ಡಾ॥ ಲೈಮನ್ ಆಬಟ್ ಎಂಬ ಪ್ರಧಾನ ಧರ್ಮಾಧಿಕಾರಿ ಇವರಲ್ಲೊಬ್ಬ. ಈತ \eng{‘Outlook’}ಎಂಬ ಪ್ರಮುಖ ಮಾಸಪತ್ರಿಕೆಯೊಂದರ ಸಂಪಾದಕನೂ ಆಗಿದ್ದ. ಡಾ ॥ ಆಬಟ್ ಸ್ವಾಮೀಜಿಯನ್ನು ಅನೇಕ ಸಲ ತನ್ನ ಮನೆಗೆ ಭೋಜನಕ್ಕೆ ಆಹ್ವಾನಿಸಿದನಲ್ಲದೆ ಅವರೊಂದಿಗೆ ದೀರ್ಘವಾಗಿ ಚರ್ಚಿಸಿದ. ಇಂತಹ ಅನೇಕ ವ್ಯಕ್ತಿಗಳ ಸಂಪರ್ಕದಿಂದ, ಸ್ವಾಮೀಜಿಯ ಸಂದೇಶ ಎಲ್ಲೆಡೆಗೂ ವೇಗವಾಗಿ ಹರಡಲು ಸಾಧ್ಯವಾಯಿತು.

ಸ್ವಾಮೀಜಿಯ ಆಪ್ತ ಶಿಷ್ಯರಲ್ಲಿ ಡಾ ॥ ಸ್ಟ್ರೀಟ್ ಎಂಬವನೊಬ್ಬ. ಈತ ವೇದಾಂತದ ತತ್ತ್ವಗಳ ಹಾಗೂ ಸ್ವಾಮೀಜಿ ನೀಡಿದ ತ್ಯಾಗ-ವೈರಾಗ್ಯದ ಬೋಧನೆಯ ಪ್ರಭಾವಕ್ಕೊಳಗಾಗಿ, ಸಂನ್ಯಾಸ ಸ್ವೀಕರಿಸಲು ಬಯಸಿದ. ಸಂನ್ಯಾಸಕ್ಕಾಗಿ ಯಾರ್ಯಾರು ತೀವ್ರವಾಗಿ ಹಂಬಲಿಸುತ್ತಾರೆಯೋ ಅವರಿ ಗೆಲ್ಲ ಸಂನ್ಯಾಸದೀಕ್ಷೆ ನೀಡಬೇಕೆಂಬುದು ಸ್ವಾಮೀಜಿಯ ತೀರ್ಮಾನವಾಗಿತ್ತು. ಆದ್ದರಿಂದ ಅವರು, ಡಾ ॥ ಸ್ಟ್ರೀಟ್​ನಿಗೆ ಸಂನ್ಯಾಸದೀಕ್ಷೆ ನೀಡಲು ಒಪ್ಪಿದರು. ಫೆಬ್ರವರಿ ೨ಂರಂದು ತಮ್ಮ ಇತರ ಸಂನ್ಯಾಸೀ ಹಾಗೂ ಗೃಹೀ ಶಿಷ್ಯರ ಸಮ್ಮುಖದಲ್ಲಿ ಈತನಿಗೆ ಸಂನ್ಯಾಸವನ್ನು ಅನುಗ್ರಹಿಸಿ ಸ್ವಾಮಿ ಯೋಗಾನಂದ ಎಂಬ ಹೆಸರನ್ನು ಕೊಟ್ಟರು. ಈತನಿಗೆ ಯೋಗದಲ್ಲಿ ತೀವ್ರವಾದ ಆಸಕ್ತಿಯಿದ್ದುದ ರಿಂದ ಅವರು ಆ ಹೆಸರನ್ನು ನೀಡಿದರು. ಡಾ ॥ಸ್ಟ್ರೀಟ್ ಸ್ವಾಮೀಜಿಯನ್ನು ಭೇಟಿಯಾಗುವ ಮೊದಲೇ, ತನ್ನದೇ ಆದ ವಿಶಿಷ್ಟ ಸಾಧನೆಗಳ ಮೂಲಕ ಯೋಗಶಕ್ತಿಯನ್ನು ಸಂಪಾದಿಸಿದ್ದ. ಅಲ್ಲದೆ ಸರಳ-ಪ್ರಾಮಾಣಿಕ ಸ್ವಭಾವದವನಾಗಿದ್ದ ಈತನನ್ನು ಅವರು ತುಂಬ ಮೆಚ್ಚಿಕೊಂಡಿದ್ದರು.

ಇದಾದ ಕೆಲದಿನಗಳ ಅನಂತರ ಸ್ವಾಮೀಜಿ ಇತರ ಅನೇಕ ಯುವಕ-ಯುವತಿಯರಿಗೆ ಮಂತ್ರ ದೀಕ್ಷೆಯನ್ನು ಅನುಗ್ರಹಿಸಿದರು. ಈ ದಿನಗಳಲ್ಲಿ ಅವರು ತಮ್ಮ ಅನುಯಾಯಿಗಳಾದ ಜೆ. ಜೆ. ಗುಡ್​ವಿನ್, ಮಿಸ್ ವಾಲ್ಡೊ ಹಾಗೂ ವಾನ್​ಹಾಗೆನ್ ಎಂಬೊಬ್ಬ ತರುಣನಿಗೆ ಬ್ರಹ್ಮಚರ್ಯ ದೀಕ್ಷೆಯನ್ನು ನೀಡಿದರು. ಹೀಗೆ, ವಿದ್ಯೆ-ಸ್ಥಾನಮಾನಗಳ ದೃಷ್ಟಿಯಿಂದ ಅತ್ಯಂತ ಉನ್ನತ ಮಟ್ಟ ದಲ್ಲಿದ್ದ ಮೂವರು ಪಾಶ್ಚಾತ್ಯರು ಸ್ವಾಮೀಜಿಯ ಕರೆಗೆ ಓಗೊಟ್ಟು ಸರ್ವಸಂಗ ಪರಿತ್ಯಾಗ ಮಾಡಿದರಲ್ಲದೆ ಇತರ ಹಲವಾರು ಜನರು ಪವಿತ್ರವಾದ ಮಂತ್ರದೀಕ್ಷೆಯನ್ನೂ ಬ್ರಹ್ಮಚರ್ಯ ದೀಕ್ಷೆಯನ್ನೂ ಸ್ವೀಕರಿಸಿದರು. ಇದನ್ನು ಗಮನಿಸಿದಾಗ ಆ ಭೋಗದ ನಾಡಿನಲ್ಲಿ ಸ್ವಾಮೀಜಿ ಕೆಲವರಿಗಾದರೂ ತ್ಯಾಗದ ಮಹಿಮೆಯನ್ನು ಮನಗಾಣಿಸಿದರೆಂಬುದನ್ನು ಮರೆಯುವಂತಿಲ್ಲ. ಅವರ ಸಂಪರ್ಕಕ್ಕೆ ಬಂದಮೇಲೆ ಮೊದಲಿನ ವ್ಯಕ್ತಿಯಾಗಿಯೇ ಉಳಿದವರೊಬ್ಬರೂ ಇರಲಿಲ್ಲ ವೆಂದು ಹೇಳಬಹುದು. ಆದ್ದರಿಂದ ಅವರ ಪ್ರಭಾವಕ್ಕೆ ಒಳಗಾಗಿ ಅವರನ್ನು ‘ಗುರುದೇವ’ ಎಂದು ಕರೆಯತೊಡಗಿದವರ ಸಂಖ್ಯೆ ಎಷ್ಟು ಹೆಚ್ಚಾಯಿತೆಂದರೆ, ಸ್ವಾಮಿ ಕೃಪಾನಂದರು ಆ ಬಗ್ಗೆ ಸ್ವಲ್ಪ ತಮಾಷೆಯಾಗಿ ಬರೆದರು:

“ಈಗ ಭಾರತವು ಸ್ವಾಮೀಜಿಯ ಮೇಲೆ ತನಗಿರುವ ಒಡೆತನವನ್ನು ಈ ಕೂಡಲೇ ಸ್ಪಷ್ಟಪಡಿಸು ವುದು ಒಳ್ಳೆಯದು. ಏಕೆಂದರೆ ಇಲ್ಲಿನವರು ಅಮೆರಿಕ ಸಂಯುಕ್ತ ಸಂಸ್ಥಾನದ ‘ರಾಷ್ಟ್ರೀಯ ವಿಶ್ವ ಕೋಶ’ಕ್ಕಾಗಿ ಸ್ವಾಮೀಜಿಯ ಜೀವನ ಚರಿತ್ರೆಯನ್ನು ಬರೆಯುವವರಿದ್ದು ಅದರಲ್ಲಿ ಅವರನ್ನು ಅಮೆರಿಕದ ಪ್ರಜೆಯನ್ನಾಗಿಸಲಿದ್ದಾರೆ. ಹೋಮರನಿಗೆ ಜನ್ಮ ನೀಡಿದ ಹೆಮ್ಮೆಯನ್ನು ತಮ್ಮದಾಗಿಸಿ ಕೊಳ್ಳಲು ಏಳು ನಗರಗಳು ಹೊಡೆದಾಡಿದುವು. ಹಾಗೆಯೇ, ನಮ್ಮ ಗುರುದೇವನನ್ನು ತಮ್ಮವ ನೆಂದು ಹೇಳಿಕೊಂಡು, ಲೋಕೋತ್ತರ ಪುತ್ರನೋರ್ವನನ್ನು ನಿರ್ಮಿಸಿದ ಕೀರ್ತಿಯನ್ನು ಭಾರತದ ಕೈಯಿಂದ ಕಸಿದುಕೊಳ್ಳಲು ಏಳು ರಾಷ್ಟ್ರಗಳು ಹೊಡೆದಾಡುವ ಕಾಲವೊಂದು ಬರಬಹುದು.”

ಈ ವೇಳೆಗೆ ಸ್ವಾಮೀಜಿ ಕರ್ಮಯೋಗ, ರಾಜಯೋಗ, ಭಕ್ತಿಯೋಗ ಮತ್ತು ಜ್ಞಾನಯೋಗಗಳ ಮೇಲಿನ ತಮ್ಮ ಉಪನ್ಯಾಸ ಮಾಲಿಕೆಗಳನ್ನು ಮುಕ್ತಾಯಗೊಳಿಸಿದ್ದರು. ಗುಡ್​ವಿನ್ನನ ಪರಿಶ್ರಮ ದಿಂದಾಗಿ ಕರ್ಮಯೋಗ ಮತ್ತು ರಾಜಯೋಗಗಳ ಟಿಪ್ಪಣಿಗಳು ಪ್ರಕಟಣೆಗೆ ಸಿದ್ಧಗೊಂಡಿದ್ದು ಪುಸ್ತಕರೂಪದಲ್ಲಿ ಪ್ರಕಟವಾದುವು. ಅಮೆರಿಕದ ವಿವಿಧ ಪತ್ರಿಕೆಗಳಲ್ಲಿ ಈ ಪುಸ್ತಕಗಳ ಕುರಿತಾಗಿ ಉನ್ನತ ಪ್ರಶಂಸೆಯ ವಿಮರ್ಶೆಗಳು ಪ್ರಕಟವಾಗಿ ಪುಸ್ತಕಗಳು ಹೆಚ್ಚುಹೆಚ್ಚಾಗಿ ಮಾರಾಟವಾಗ ತೊಡಗಿದುವು. ಅದರಲ್ಲೂ ‘ರಾಜಯೋಗ’ವು ಅಮೆರಿಕದ ವಿಶ್ವವಿದ್ಯಾಲಯಗಳ ಕೆಲವು ಖ್ಯಾತ ಮನಶ್ಶಾಸ್ತ್ರಜ್ಞರನ್ನು ಮತ್ತು ಶರೀರಶಾಸ್ತ್ರಜ್ಞರನ್ನು ವಿಶೇಷವಾಗಿ ಆಕರ್ಷಿಸಿತು. ‘ಭಕ್ತಿಯೋಗ’ವು ಭಾರತದಲ್ಲಿ ಇಷ್ಟರಲ್ಲೇ ಪ್ರಕಟವಾಗುವುದರಲ್ಲಿತ್ತು. ಅಲ್ಲದೆ ಮುಂದೆ ‘ಜ್ಞಾನಯೋಗ’ ಎಂಬ ಪುಸ್ತಕದಲ್ಲಿ ಪ್ರಕಟವಾದ ಸ್ವಾಮೀಜಿಯ ಹಲವಾರು ಉಪನ್ಯಾಸಗಳು ಅದಾಗಲೇ ಚಿಕ್ಕಚಿಕ್ಕ ಪುಸ್ತಕಗಳ ರೂಪದಲ್ಲಿ ಪ್ರಕಟಗೊಂಡಿದ್ದುವು.

ಈ ದಿನಗಳಲ್ಲಿ ಸ್ವಾಮೀಜಿ ಗಣನೀಯ ಪ್ರಮಾಣದಲ್ಲಿ ವೇದಾಂತ ಸಾಹಿತ್ಯವನ್ನು ಪ್ರಕಟಿಸಿ ದ್ದಲ್ಲದೆ, ಅಭಿವೃದ್ಧಿ ಹೊಂದುತ್ತಿದ್ದ ವೇದಾಂತ ಸೊಸೈಟಿಗೆ ಇನ್ನಷ್ಟು ಖಚಿತವಾದ ರೂಪಕೊಟ್ಟು ಅದನ್ನು ಪುನರ್ ವ್ಯವಸ್ಥೆಗೊಳಿಸುವ ಆಲೋಚನೆಯನ್ನು ಕೂಡ ಮಾಡಿದರು. ಉಪನ್ಯಾಸಗಳ ಹಾಗೂ ತರಗತಿಗಳ ಏರ್ಪಾಡು, ತಮ್ಮ ಪುಸ್ತಕಗಳ ಪ್ರಕಟಣೆ ಹಾಗೂ ವಿತರಣೆ, ಲೆಕ್ಕಪತ್ರಗಳ ನಿರ್ವಹಣೆ ಹಾಗೂ ಇನ್ನಿತರ ವ್ಯವಹಾರಗಳನ್ನೆಲ್ಲ ನಿರ್ವಹಿಸಲು ಕಾರ್ಯಕಾರೀ ಸಮಿತಿ ಯೊಂದನ್ನು ಸ್ಥಾಪಿಸಲು ಅವರು ಉದ್ದೇಶಿಸಿದರು. ವೇದಾಂತ ಸೊಸೈಟಿಯು ಮುಂದೆ ಬಹಳ ಕಾಲದವರೆಗೂ ಸಾರ್ವಜನಿಕರಿಗೆ ಸದಸ್ಯತ್ವವನ್ನು ತೆರೆಯದಿದ್ದರೂ, ಸಂಘದ ಪದಾಧಿಕಾರಿಗಳು ಮಾತ್ರ ಅದಾಗಲೇ ವೇದಾಂತ ಪ್ರಸಾರಕಾರ್ಯಕ್ಕಾಗಿ ತಮ್ಮನ್ನು ತಾವು ಸಮರ್ಪಿಸಿಕೊಂಡಾಗಿತ್ತು. ಮುಂದೆ ವೇದಾಂತ ಸೊಸೈಟಿ ತನ್ನ ಸದಸ್ಯರಾಗಲು ಮತ್ತು ವೇದಾಂತದ ಅಧ್ಯಯನ ಮಾಡಲು ಎಲ್ಲ ಮತಪಂಥಗಳಿಗೆ ಹಾಗೂ ಸಂಸ್ಥೆಗಳಿಗೆ ಆಹ್ವಾನ ನೀಡಿತು. ವೇದಾಂತ ಸೊಸೈಟಿಯ ಸದಸ್ಯರಾಗಲು ಯಾರೂ ತಾವು ಅದಾಗಲೇ ನಂಬಿದ್ದ ಮತವನ್ನು ಬದಲಾಯಿಸಬೇಕಿರಲಿಲ್ಲ. ಸರ್ವಧರ್ಮಗಳ ಸ್ವೀಕಾರ ಹಾಗೂ ಸಹಾನುಭೂತಿಗಳೇ ಈ ಸೊಸೈಟಿಯ ಧ್ಯೇಯಮಂತ್ರ ವಾಗಿತ್ತು. ನಗರದ ಶ್ರೀಮಂತ ಹಾಗೂ ಪ್ರಭಾವಶಾಲೀ ನಾಗರಿಕರಲ್ಲಿ ಒಬ್ಬರಾದ ಫ್ರಾನ್ಸಿಸ್ ಲೆಗೆಟ್​ರನ್ನು ಸ್ವಾಮೀಜಿಯವರು ವೇದಾಂತ ಸೊಸೈಟಿಯ ಅಧ್ಯಕ್ಷರನ್ನಾಗಿ ಮಾಡಿದರು. ಕಾರ್ಯ ದರ್ಶಿಯ ಸ್ಥಾನಕ್ಕೆ ಮಿಸ್ ಮೇರಿ ಫಿಲಿಪ್ಸ್​ಳನ್ನು ಆರಿಸಲಾಯಿತು. ಈಕೆ ಧರ್ಮಸಂಸ್ಥೆಗಳ ಹಾಗೂ ಬೌದ್ಧಿಕ ಚಟುವಟಿಕೆಗಳ ವಲಯದಲ್ಲಿ ಒಬ್ಬ ಪ್ರಮುಖ ಮಹಿಳೆಯಾಗಿದ್ದಳು. ಈ ದಿನಗಳಲ್ಲಿ ವೇದಾಂತ ಪ್ರಸಾರದಲ್ಲಿ ಆಸಕ್ತರಾಗಿ ಸ್ವಾಮೀಜಿಯೊಂದಿಗೆ ಅತ್ಯಂತ ಉತ್ಸಾಹದಿಂದ ಕೆಲಸ ಮಾಡಿದವರೆಂದರೆ ಮಿಸ್ ವಾಲ್ಡೊ, ಶ್ರೀಮತಿ ಆರ್ಥರ್ ಸ್ಮಿತ್, ಶ್ರೀ ಮತ್ತು ಶ್ರೀಮತಿ ವಾಲ್ಟರ್ ಗುಡ್​ಇಯರ್ ಮತ್ತು ಖ್ಯಾತ ಗಾಯಕಿ ಎಮ್ಮಾ ಥರ್ಸ್​ಬಿ.

ಸ್ವಾಮೀಜಿ ತಮ್ಮ ತರಗತಿಗಳನ್ನು ಒಂದು ಸಂಘವಾಗಿ ರೂಪುಗೊಳಿಸಿದ್ದರಲ್ಲಿ, ಪ್ರಾಚ್ಯ- ಪಾಶ್ಚಾತ್ಯ ಆದರ್ಶಗಳ ಹಾಗೂ ಭಾವನೆಗಳ ವಿನಿಮಯವಾಗಬೇಕೆಂಬ ಉದ್ದೇಶ ಕೂಡ ಸೇರಿತ್ತು. ಈ ಮೂಲಕ ಪ್ರಾಚ್ಯ-ಪಾಶ್ಚಾತ್ಯ ಜಗತ್ತುಗಳ ನಡುವೆ ಆತ್ಮೀಯತೆ-ಸೌಹಾರ್ದತೆಗಳು ನೆಲೆಗೊಳ್ಳು ವಂತಾಗಬೇಕು, ಒಂದು ಮತ್ತೊಂದರಿಂದ ಕಲಿಯುವಂತಾಗಬೇಕು ಎನ್ನುವುದು ಅವರ ಉದ್ದೇಶ ವಾಗಿತ್ತು. ವೇದಾಂತ ಸೊಸೈಟಿಯಲ್ಲಿ ತರಗತಿಗಳನ್ನು ನಡೆಸಿಕೊಂಡು ಹೋಗುವುದಕ್ಕಾಗಿ ಭಾರತ ದಿಂದ ಇನ್ನೊಬ್ಬರು ಸೋದರ ಸಂನ್ಯಾಸಿಯನ್ನು ಕರೆಸಿಕೊಳ್ಳುವಂತೆ ನಿಕಟವರ್ತಿಗಳು ಅವರನ್ನು ಕೇಳುತ್ತಲೇ ಇದ್ದರು. ಈ ನಡುವೆ, ಸ್ವಾಮೀಜಿಯ ಆದೇಶದಂತೆ ಸ್ವಾಮಿ ಶಾರದಾನಂದರು ಇಂಗ್ಲೆಂಡಿಗೆ ಬಂದು ಕಾರ್ಯಾರಂಭ ಮಾಡಿದ್ದರು. ಮುಂದೆ ೧೮೯೬ರ ಜುಲೈ ತಿಂಗಳಲ್ಲಿ ಅವರು ಅಮೆರಿಕೆಗೆ ಬರುವುದೆಂದು ನಿರ್ಧಾರವಾಯಿತು. ಈ ಹಿಂದೆ ಸ್ವಾಮೀಜಿ ಅಳಸಿಂಗ ಪೆರಮಾಳರಿಗೆ ಬರೆದಿದ್ದರು–“ಇಂಗ್ಲೆಂಡಿನ ಕಾರ್ಯವನ್ನು ನೋಡಿಕೊಳ್ಳುವುದಕ್ಕಾಗಿ ಒಂದು ಸಂನ್ಯಾಸಿಯನ್ನು ಕಳಿಸಿಕೊಡುವಂತೆ ನಾನು ತಿಳಿಸಿದ್ದೆ. ಈಗ ಅಮೆರಿಕೆಗೆ ಇನ್ನೂ ಒಬ್ಬರು ಬೇಕಾಗಿದ್ದಾರೆ. ನೀನು ನಿಮ್ಮ ಮದ್ರಾಸಿನಿಂದ ಒಬ್ಬ ಗಟ್ಟಿವ್ಯಕ್ತಿಯನ್ನು ಕಳಿಸಲಾರೆಯಾ? ಖರ್ಚಿನ ಬಾಬತ್ತನು ನಾನು ನೋಡಿಕೊಳ್ಳುತ್ತೇನೆ. ನನಗೆ ನನ್ನವನಾದ ಒಬ್ಬ ಮನುಷ್ಯ ಬೇಕು. ಆಧ್ಯಾತ್ಮಿಕ ಪ್ರಗತಿಯಲ್ಲಿ ಗುರುಭಕ್ತಿ ಎಂಬುದು ಅಡಿಗಲ್ಲು.” ಆದರೆ ಅಮೆರಿಕದಂತಹ ದೇಶದಲ್ಲಿ ಸ್ವಾಮೀಜಿಯಂಥವರ ಕಾರ್ಯವನ್ನು ವಹಿಸಿಕೊಂಡು ಮುನ್ನಡೆಸಲು ಯಾರಿಂದ ಸಾಧ್ಯ? ಸ್ವಾಮೀಜಿಯ ಈ ಪ್ರಯತ್ನ ಈಡೇರಲಿಲ್ಲವೆಂದು ಹೇಳಬೇಕಾಗಿಯೇ ಇಲ್ಲ.

ಅಮೆರಿಕದ ಹಾಗೂ ಇಂಗ್ಲೆಂಡಿನ ತಮ್ಮ ಕೆಲವು ಶಿಷ್ಯರು ಭಾರತಕ್ಕೆ ಹೋಗಿ ಅಲ್ಲಿ ಶಿಕ್ಷಣ ಹಾಗೂ ಬೋಧನಾಕಾರ್ಯಗಳನ್ನು ಕೈಗೊಳ್ಳಬೇಕೆನ್ನುವುದೂ ಕೂಡ ಸ್ವಾಮೀಜಿಯ ಅಪೇಕ್ಷೆ ಯಾಗಿತ್ತು. ಅಮೆರಿಕೆಯಲ್ಲಿ ಧಾರ್ಮಿಕ ಬೋಧನೆ ಮುಖ್ಯವಾದರೆ ಭಾರತದಲ್ಲಿ ವ್ಯಾವಹಾರಿಕ ಶಿಕ್ಷಣ, ತಾಂತ್ರಿಕ ಶಿಕ್ಷಣ, ಕೈಗಾರಿಕೆ, ಅರ್ಥಶಾಸ್ತ್ರ, ಸಂಘಟನೆ ಹಾಗೂ ಸಹಕಾರ ಇವುಗಳ ಬೋಧನೆಯೇ ಮುಖ್ಯವಾಗಬೇಕು ಎನ್ನುವುದು ಅವರ ಅಭಿಪ್ರಾಯವಾಗಿತ್ತು. ಪ್ರಾಚ್ಯ ಹಾಗೂ ಪಾಶ್ಚಾತ್ಯ–ಈ ಎರಡು ವಿಭಿನ್ನ ಜಗತ್ತುಗಳನ್ನು ಸಮನ್ವಯಗೊಳಿಸುವುದರ ಕುರಿತಾಗಿ ಹಗಲು ಇರುಳೂ ಆಲೋಚಿಸಿದ್ದವರು ಅವರು. ಅಲ್ಲದೆ ಅವರು ಭವಿಷ್ಯದರ್ಶನ ಮಾಡಿದವರಂತೆ ತಮ್ಮ ಅಮೆರಿಕೆಯ ಶಿಷ್ಯರಿಗೆ ಆಗಾಗ ಹೇಳುತ್ತಿದ್ದರು–“ಆಲೋಚನೆ ಹಾಗೂ ಆಚರಣೆಗಳೆರಡರಲ್ಲೂ ಪ್ರಾಚ್ಯ-ಪಾಶ್ಚಾತ್ಯಗಳ ನಡುವಣ ಅಡ್ಡಗೋಡೆಯು ಸಂಪೂರ್ಣ ಬಿದ್ದುಹೋಗುವ ಕಾಲವೊಂದು ಬರಲಿದೆ” ಎಂದು.

ಪ್ರಾಚ್ಯ-ಪಾಶ್ಚಾತ್ಯಗಳ ನಡುವಣ ಈ ಸಮನ್ವಯವು ಸಾಧ್ಯವಾಗಬೇಕಾದರೆ ಯಾವುದಾದ ರೊಂದು ಆಧಾರಭೂತ ತತ್ತ್ವವಿರಬೇಕು. ವೇದಾಂತವೇ ಈ ಆಧಾರವೆಂಬುದು ಸ್ವಾಮೀಜಿಯ ಅಭಿಪ್ರಾಯವಾಗಿತ್ತು. ಆ ವೇದಾಂತದಲ್ಲಿ ಅತ್ಯಂತ ವಿಶಾಲವಾದ ಹಾಗೂ ಸಮಸ್ತ ಮಾನವತೆಗೇ ಅನ್ವಯವಾಗಬಲ್ಲ ಬೋಧನೆಗಳು ಈ ಆಧಾರಭೂತ ತತ್ತ್ವವಾಗಲು ಸಾಧ್ಯವೆಂದು ಅವರು ಹೇಳುತ್ತಿದ್ದರು. ಇಂತಹ ಆದರ್ಶವು ತಮ್ಮ ಗುರುದೇವ ಶ್ರೀರಾಮಕೃಷ್ಣರ ಜೀವನದಲ್ಲಿ ರೂಪ ತಾಳಿದುದನ್ನು ಅವರು ಕಂಡುಕೊಂಡಿದ್ದರು. ಎಂದಿಗೂ ಯಾರನ್ನೂ ದೂಷಿಸದ ಶ್ರೀರಾಮಕೃಷ್ಣರ ಅದ್ಭುತ ಜೀವನ ಅವರೆದುರಿನಲ್ಲಿ ವೇದಾಂತ ಒಂದು ಹೊಸ ಅರ್ಥವನ್ನೇ ತೆರೆದಿತ್ತು. ಶ್ರೀರಾಮ ಕೃಷ್ಣರು ಎಲ್ಲ ವರ್ಗದ ಜನರ ಮುಂದೆಯೂ ಆಧ್ಯಾತ್ಮಿಕ ಜೀವನದ ದಾರಿಯನ್ನು ತೆರೆದವರು. ಅಲ್ಲಿ ಎಲ್ಲರಿಗೂ ಸ್ವಾಗತವಿತ್ತು. ಆ ಕರೆಗೆ ಓಗೊಟ್ಟು ಬಂದವರೆಲ್ಲ ಬೆಳಕನ್ನು ಪಡೆಯಬಹುದಾ ಗಿತ್ತು. ಈ ಕುರಿತಾಗಿ ಸ್ವಾಮೀಜಿ ಸಹಸ್ರದ್ವೀಪೋದ್ಯಾನದಲ್ಲಿ ಸ್ಫೂರ್ತಿಯುತ ಭಾವದಲ್ಲಿ ನುಡಿದಿದ್ದರು: “ಶ್ರೀರಾಮಕೃಷ್ಣರು ಬಂದದ್ದು ಇಂದಿನ ಧರ್ಮವನ್ನು ಬೋಧಿಸುವುದಕ್ಕಾಗಿ–ಆ ಧರ್ಮವು ನಿರ್ಮಾಣಕಾರಿಯೇ ಹೊರತು ವಿನಾಶಕಾರಿಯಲ್ಲ. ಸತ್ಯವನ್ನರಸಲು ಅವರು ಹೊಸ ದಾಗಿ ಪ್ರಕೃತಿಯ ಒಳಗೆ ಪ್ರವೇಶಿಸಿದರು; ತತ್ಫಲವಾಗಿ ‘ವೈಜ್ಞಾನಿಕ ಧರ್ಮ’ವನ್ನು ಪಡೆದರು. ಈ ವೈಜ್ಞಾನಿಕ ಧರ್ಮವು ಎಂದಿಗೂ ‘ನಂಬಿ’ ಎನ್ನುವುದಿಲ್ಲ. ‘ನೋಡಿ!’ ಎನ್ನುತ್ತದೆ. ‘ನಾನು ನೋಡುತ್ತಿದ್ದೇನೆ, ನೀನು ಕೂಡ ನೋಡಬಹುದು’ ಎನ್ನುತ್ತದೆ. ಅದೇ ವಿಧಾನವನ್ನು ಅನುಸರಿಸಿರಿ; ನೀವು ಅದೇ ಪರಮದರ್ಶನವನ್ನು ಪಡೆಯುವಿರಿ.”

ಈ ನೂತನ ಸಂದೇಶವು ಎಷ್ಟೇ ಉದಾತ್ತವಾಗಿದ್ದರೂ, ಎಷ್ಟೇ ಶ್ರೇಷ್ಠವಾದದ್ದಾಗಿದ್ದರೂ, ಅದು ಸಾರ್ವತ್ರಿಕವಾಗಿ ಸ್ವೀಕರಿಸಲ್ಪಡಬೇಕಾದರೆ ಸುಲಭಸಾಧ್ಯವಲ್ಲವೆಂಬುದು ಸ್ವಾಮೀಜಿಗೆ ತಿಳಿದೇ ಇತ್ತು. ಅತ್ಯಂತ ಶ್ರದ್ಧಾವಂತರೂ ಬುದ್ಧಿವಂತರೂ ಆದ ವ್ಯಕ್ತಿಗಳು ಒಟ್ಟಾಗಿ ಸೇರಿ ಅತ್ಯಂತ ತಾಳ್ಮೆಯಿಂದ ನಿರಂತರವಾಗಿ ಶ್ರಮಿಸಬೇಕಾದ ಆವಶ್ಯಕತೆಯಿತ್ತು. ಆದ್ದರಿಂದಲೇ ಅಮೆರಿಕ-ಇಂಗ್ಲೆಂಡುಗಳ ಕೇಂದ್ರಸ್ಥಾನಗಳಾದ ನ್ಯೂಯಾರ್ಕ್ ಹಾಗೂ ಲಂಡನ್ ನಗರಗಳಲ್ಲಿ ವ್ಯವಸ್ಥಿತ ಕೇಂದ್ರಗಳನ್ನು ತೆರೆಯಲು ಸಾಧ್ಯವಾದರೆ, ವೇದಾಂತದ ಸಂದೇಶ ಇಡೀ ಪಾಶ್ಚಾತ್ಯ ಜಗತ್ತಿನಲ್ಲಿ ಪ್ರಸಾರಗೊಳ್ಳಲು ಸಾಧ್ಯವಾಗುತ್ತದೆಂದು ಭಾವಿಸಿ ಸ್ವಾಮೀಜಿ, ಆ ನಿಟ್ಟಿನಲ್ಲಿ ಶ್ರಮಿಸಿ ದರು. ಈ ಉದ್ದೇಶದಿಂದಲೇ ಅವರು ತಮ್ಮ ಕಾರ್ಯವನ್ನು ಮುಂದುವರಿಸಿಕೊಂಡು ಹೋಗ ಬಲ್ಲ ಶಿಷ್ಯರನ್ನು ತರಬೇತುಗೊಳಿಸಿದರು. ಈ ಶಿಷ್ಯರ ಪೈಕಿ ಮಿಸ್ ವಾಲ್ಡೊಳ ಮೇಲೆ ಅವರಿಗೆ ಹೆಚ್ಚಿನ ಭರವಸೆಯಿತ್ತು. ಆಕೆಗೆ ಅವರು ವಿಶೇಷ ಆಧ್ಯಾತ್ಮಿಕ ಶಕ್ತಿಯನ್ನು ಹಾಗೂ ಅಧಿಕಾರವನ್ನೂ ನೀಡಿ, ವೇದಾಂತ ಪ್ರಸಾರಕ್ಕೆ ಅವಳನ್ನು ನಿಯೋಜಿತಗೊಳಿಸಿದರು. ಮುಂದೆ ಮಿಸ್ ವಾಲ್ಡೊ ಸೋದರಿ ಹರಿದಾಸಿ ಎಂಬ ಹೆಸರಿನಿಂದ ಪ್ರಸಿದ್ಧಳಾದಳು. ಅಲ್ಲದೆ ಸಂನ್ಯಾಸಿಗಳಾದ ಕೃಪಾನಂದ, ಅಭಯಾನಂದ, ಯೋಗಾನಂದ ಹಾಗೂ ಇನ್ನಿತರ ಬ್ರಹ್ಮಚಾರಿಗಳಿಗೆ ವಿಶೇಷ ತರಬೇತಿ ನೀಡಿ ದ್ವೈತ-ವಿಶಿಷ್ಟಾದ್ವೈತ-ಅದ್ವೈತಗಳಲ್ಲಿ ಅವರು ಪರಿಣತರಾಗುವಂತೆ ಮಾಡಿದರು.

ಫೆಬ್ರವರಿ ೨೩ರಂದು ನೀಡಿದ ‘ನನ್ನ ಗುರುದೇವ’ ಎಂಬ ಉಪನ್ಯಾಸದೊಂದಿಗೆ ನ್ಯೂಯಾರ್ಕಿ ನಲ್ಲಿ ತಮ್ಮ ಕಾರ್ಯವನ್ನು ಮುಗಿಸಿದ ಸ್ವಾಮೀಜಿ, ಮಾರ್ಚ್ ೩ರಂದು ಡೆಟ್ರಾಯ್ಟಿಗೆ ಪ್ರಯಾಣ ಮಾಡಿದರು. ತಮ್ಮ ನಗರದಲ್ಲಿ ತರಗತಿಯನ್ನೂ ತೆಗೆದುಕೊಳ್ಳುವಂತೆ ಡೆಟ್ರಾಯ್ಟಿನಿಂದ ಅವರನ್ನು ಆಹ್ವಾನಿಸಲಾಗಿತ್ತು. ಸರಿಯಾಗಿ ಎರಡು ವರ್ಷಗಳ ಕೆಳಗೆ ಅವರು ಡೆಟ್ರಾಯಿಟ್ಟಿನಲ್ಲಿದ್ದಾಗ ಅವರಿಗೆ ದೊರಕಿದ ಸ್ವಾಗತ ಎಷ್ಟು ವೈಭವಪೂರ್ಣವಾಗಿತ್ತು, ಅವರು ಸಮಾಜದ ಮೇಲೆ ಬೀರಿದ ಪ್ರಭಾವವೆಂಥದು ಮತ್ತು ಅಲ್ಲಿನ ಪತ್ರಿಕೆಗಳಲ್ಲಿ ನಡೆದ ವಾಗ್ವಾದವೆಂಥದು ಎಂಬುದನ್ನು ಈ ಹಿಂದೆ ನೋಡಿದ್ದೇವೆ. ಈ ಸಲ ಅವರು ಇಲ್ಲಿಗೆ ಮತ್ತೊಮ್ಮೆ ಬಂದಾಗಲೂ ಅದೇ ವಾದವಿವಾದದ ಅಲೆಗಳು ನಗರದ ಪತ್ರಿಕಾವಲಯಗಳಲ್ಲಿ ಭೋರ್ಗರೆದುವು.

ಡೆಟ್ರಾಯ್ಟಿನಲ್ಲಿ ಸ್ವಾಮೀಜಿಯ ಬೆಂಬಲಿಗರಲ್ಲಿ ಅವರಿಗೆ ಅತ್ಯಂತ ನಿಷ್ಠಾವಂತರೂ ಪ್ರಭಾವ ಶಾಲಿಯೂ ಆದವರೆಂದರೆ ಶ್ರೀಮತಿ ಜಾನ್ ಬ್ಯಾಗ್​ಲೀ. ಆದರೆ ದುರದೃಷ್ಟಶಾತ್, ಸ್ವಾಮೀಜಿ ಈ ಸಲ ಇಲ್ಲಿಗೆ ಬಂದಾಗ ಆಕೆ ಊರಿನಲ್ಲಿರಲಿಲ್ಲವಾದ್ದರಿಂದ ಅವರಿಬ್ಬರೂ ಭೇಟಿಯಾಗಲು ಸಾಧ್ಯವಾಗಲಿಲ್ಲ. (ಅಲ್ಲದೆ ಮತ್ತೆ ಮುಂದೆಂದೂ ಆಕೆಗೆ ಸ್ವಾಮೀಜಿಯ ದರ್ಶನಭಾಗ್ಯ ದೊರೆಯ ಲಿಲ್ಲ. ಏಕೆಂದರೆ ಎರಡು ವರ್ಷಗಳ ಬಳಿಕ ಆಕೆ ತೀರಿಕೊಂಡರು.) ಈ ಸಲ ಸ್ವಾಮೀಜಿ ಡೆಟ್ರಾಯ್ಟಿ ನಲ್ಲಿ ಎರಡು ವಾರಗಳ ಕಾಲ ಉಳಿದುಕೊಂಡರು. ಈ ಅಲ್ಪಾವಧಿಯಲ್ಲೇ ಅವರು ಅಲ್ಲಿ ೨೨ ತರಗತಿಗಳನ್ನು ತೆಗೆದುಕೊಂಡರಲ್ಲದೆ ಮೂರು ಸಾರ್ವಜನಿಕ ಉಪನ್ಯಾಸಗಳನ್ನು ಮಾಡಿದರು! ಈ ಭೇಟಿಯ ಕುರಿತಾಗಿ ಡೆಟ್ರಾಯ್ಟಿನಲ್ಲಿ ಸ್ವಾಮೀಜಿಯ ಆಪ್ತ ಶಿಷ್ಯೆಯರಲ್ಲೊಬ್ಬಳಾದ ಶ್ರೀಮತಿ ಮೇರಿ ಸಿ. ಫಂಕೆ ಬರೆಯುತ್ತಾಳೆ:

“ತಮ್ಮ ನಿಷ್ಠಾವಂತ ಶೀಘ್ರಲಿಪಿಕಾರನಾದ ಗುಡ್​ವಿನ್ನನೊಂದಿಗೆ ಬಂದಿದ್ದ ಸ್ವಾಮೀಜಿ, ಹೋಟೆಲೊಂದರಲ್ಲಿ ಕೋಣೆಗಳನ್ನು ಬಾಡಿಗೆಗೆ ತೆಗೆದುಕೊಂಡಿದ್ದರು. ತರಗತಿಗಳಿಗೆ ಅಲ್ಲಿನ ವಿಶಾಲವಾದ ಕೋಣೆಯೊಂದನ್ನು ಉಪಯೋಗಿಸಿಕೊಂಡರು. ಆದರೆ ಅಲ್ಲಿಗೆ ಮುತ್ತುತ್ತಿದ್ದ ಜನಸಂದಣಿಗೆ ಆ ಕೋಣೆ ಸಾಲುತ್ತಿರಲಿಲ್ಲ. ಆದ್ದರಿಂದ ಬಹಳಷ್ಟು ಜನ ಸ್ಥಳಾವಕಾಶವಿಲ್ಲದೆ ಹಿಂದಿರುಗಬೇಕಾಯಿತು. ಹೋಟೆಲಿನ ಕೋಣೆ, ಹಜಾರ, ಮಹಡಿ ಮೆಟ್ಟಿಲುಗಳು, ವಾಚನಾಲಯ –ಎಲ್ಲವೂ ಜನರಿಂದ ಕಿಕ್ಕಿರಿದಿತ್ತು. ಸ್ವಾಮೀಜಿ ಭಕ್ತಿಯೋಗದ ಕುರಿತಾಗಿ ಮಾತನಾಡುತ್ತಿದ್ದರು. ಆ ಸಂದರ್ಭದಲ್ಲಿ ಅವರೊಳಗೆಲ್ಲ ಕೇವಲ ಭಕ್ತಿಯೇ ತುಂಬಿಕೊಂಡಂತಿತ್ತು. ಒಂದು ಬಗೆಯ ದೈವೀಉನ್ಮಾದ ಅವರನ್ನು ಆವರಿಸಿಕೊಂಡಂತೆ ಕಂಡುಬರುತ್ತಿತ್ತು. ಅವರ ಪ್ರೀತಿಯ ಜಗನ್ಮಾತೆಯ ವ್ಯಾಕುಲದಿಂದ ಅವರ ಹೃದಯ ಬಿರಿಯುವಂತೆ ತೋರುತ್ತಿತ್ತು.

“ಅವರು ಡೆಟ್ರಾಯ್ಟಿನಲ್ಲಿ ಕಡೆಯಭಾರಿ ಸಾರ್ವಜನಿಕ ಉಪನ್ಯಾಸ ಮಾಡಿದುದು ಒಂದು ಭಾನುವಾರದ ಸಂಜೆ. ಜನಸಂದಣಿ ಎಷ್ಟು ದೊಡ್ಡದಾಗಿತ್ತೆಂದರೆ ಅದನ್ನು ಕಂಡು ನಮಗೆ ಭಯವೇ ಆಯಿತು. ರಸ್ತೆಯಲ್ಲೆಲ್ಲ ಜನ ತುಂಬಿಹೋಗಿದ್ದರು. ಸ್ಥಳಾವಕಾಶವಿಲ್ಲದೆ ನೂರಾರು ಜನ ನಿರಾಶರಾಗಿ ಹಿಂದಿರುಗಬೇಕಾಯಿತು. ಅಂದಿನ ಆ ಭಾರೀ ಸಭೆಯನ್ನು ಸ್ವಾಮೀಜಿ ಮಂತ್ರ ಮುಗ್ಧವಾಗಿಸಿದ್ದರು. ಭಾಷಣದ ವಿಷಯ–‘ಪಶ್ಚಿಮಕ್ಕೆ ಭಾರತದ ಸಂದೇಶ.’ ನಮಗವರು ಅತ್ಯಂತ ವಿದ್ವತ್ಪೂರ್ಣವೂ ಉಜ್ವಲವೂ ಆದ ಉಪನ್ಯಾಸವನ್ನು ನೀಡಿದರು. ಅವರು ಅಂದು ಕಾಣಿಸಿಕೊಂಡಷ್ಟು ಪ್ರಭಾವಶಾಲಿಗಳಾಗಿ ನನಗೆ ಮತ್ತೆಂದೂ ಕಾಣಿಸಿರಲಿಲ್ಲ. ಅಂದು ಅವರಲ್ಲಿ ಕಂಡುಬಂದ ಸೌಂದರ್ಯವು ಈ ಲೋಕದ್ದಾಗಿರಲಿಲ್ಲ. ಅವರ ಆತ್ಮವು ಅವರ ಶರೀರದ ಬಂಧನವನ್ನೇ ಬಿರಿದುಕೊಂಡು ಹೊರಬಂದಂತೆ ತೋರುತ್ತಿತ್ತು. ಆಗಲೇ ನಾನು ಅವರ ಕೊನೆ ಗಾಲ ಸಮೀಪಿಸುತ್ತಿರುವುದರ ಮೊದಲ ಸೂಚನೆಯನ್ನು ಕಂಡುಕೊಂಡೆ. ಹಲವಾರು ವರ್ಷಗಳ ಅತಿಶ್ರಮದಿಂದ ವಿಪರೀತವಾಗಿ ಬಳಲಿದ್ದರು. ಅವರಿನ್ನು ಈ ಲೋಕದಲ್ಲಿ ಹೆಚ್ಚು ಕಾಲ ಇರುವವ ರಲ್ಲ ಎಂಬುದು ಆಗಲೇ ಕಂಡುಬರುತ್ತಿತ್ತು. ಈ ಆಲೋಚನೆಯನ್ನು ನನ್ನ ಮನಸ್ಸಿನಿಂದ ತಳ್ಳಿ ಹಾಕಲು ಪ್ರಯತ್ನಿಸಿದೆ. ಆದರೆ ನನ್ನ ಹೃದಯದಲ್ಲಿ ನನಗೆ ಸತ್ಯದ ಅರಿವಾಗಿತ್ತು. ಅವರಿಗೆ ವಿಶ್ರಾಂತಿಯ ಆವಶ್ಯಕತೆಯಿತ್ತು. ಆದರೆ ಅವರು ತಮ್ಮ ಕಾರ್ಯವನ್ನು ಮುಂದುವರಿಸಲೇಬೇಕು ಎಂದು ಭಾವಿಸಿದ್ದರು.”

ಡೆಟ್ರಾಯ್ಟಿನಿಂದ ಸ್ವಾಮೀಜಿ ಬಾಸ್ಟನ್ ನಗರಕ್ಕೆ ಬಂದರು. ಇಲ್ಲಿ ಅವರು ಎರಡುವಾರ ಉಳಿದುಕೊಂಡರು. ಅವರ ಇಡೀ ಕಾರ್ಯೋದ್ದೇಶಕ್ಕೆ ಸಂಬಂಧಿಸಿದಂತೆ ಅತ್ಯಂತ ಗಮನಾರ್ಹ ವಾದ ಒಂದು ಘಟನೆ ಇಲ್ಲಿ ನಡೆಯಿತು. ಇಲ್ಲಿನ ಪ್ರಸಿದ್ಧ ಹಾರ್ವರ್ಡ್ ವಿಶ್ವವಿದ್ಯಾನಿಲಯಕ್ಕೆ ಸೇರಿದ ‘ಗ್ರ್ಯಾಜುಯೇಟ್ ಫಿಲಾಸಫಿಕಲ್ ಕ್ಲಬ್​’ ಎಂಬ ಸಂಸ್ಥೆಯ ಆಶ್ರಯದಲ್ಲಿ ಅವರು ಪ್ರಾಧ್ಯಾಪಕರು ಹಾಗೂ ವಿದ್ಯಾರ್ಥಿಗಳನ್ನುದ್ದೇಶಿಸಿ ವೇದಾಂತದ ಕುರಿತಾಗಿ ಮಾತನಾಡಿದರು. ಇದನ್ನು ವ್ಯವಸ್ಥೆ ಮಾಡಿದವಳು ಅವರ ಅತ್ಯಂತ ಆಪ್ತ ಶಿಷ್ಯೆಯೂ ಬಾಸ್ಟನ್ ನಗರದ ಅತ್ಯಂತ ಗಣ್ಯ ನಾಗರಿಕಳೂ ಆದ ಶ್ರೀಮತಿ ಸಾರಾ ಬುಲ್. ಅಲ್ಲದೆ ಹಿಂದೆಯೇ ಈ ಕ್ಲಬ್ಬಿನ ಅಧಿಕಾರಿಗಳು ಸ್ವಾಮೀಜಿಯನ್ನು ತಮ್ಮಲ್ಲಿಗೆ ಆಹ್ವಾನಿಸಿದ್ದರು. ಈ ಸಂಸ್ಥೆಯಲ್ಲಿ ಸೇರಲಿದ್ದ ಸಭಿಕರು ಅತ್ಯಂತ ಬುದ್ಧಿಶಾಲಿಗಳು ಮತ್ತು ತತ್ತ್ವಶಾಸ್ತ್ರದಲ್ಲಿ ನುರಿತವರು; ಇಡೀ ಅಮೆರಿಕದಲ್ಲೇ ಅತ್ಯುತ್ಕೃಷ್ಟರಾದ ಹಾಗೂ ಅಗ್ರಗಣ್ಯರಾದ ವ್ಯಕ್ತಿಗಳು. ಅಂದಿನ ಕಾಲದಲ್ಲಿ ಈ ಸಂಸ್ಥೆಯು ಇಡೀ ಜಗತ್ತಿನಲ್ಲೇ ಪ್ರಮುಖ ಚಿಂತಕರಿಂದ ಕೂಡಿದ್ದಾಗಿದ್ದು, ಇವರ ಪೈಕಿ ಜಾರ್ಜ್ ಹೆಚ್. ಪಾಮರ್, ವಿಲಿಯಂ ಜೇಮ್ಸ್, ಜಾರ್ಜ್ ಸಂತಯಾನ ಮೊದಲಾದ ಶ್ರೇಷ್ಠ ಚಿಂತಕರಿದ್ದರು. ಇಂತಹ ಬುದ್ಧಿಮಾನ್ ವ್ಯಕ್ತಿಗಳಿಂದ, ವಿಮರ್ಶಕರಿಂದ ತುಂಬಿದ ಸಭೆಯಲ್ಲಿ ಸ್ವಾಮೀಜಿ ಮಾತನಾಡಬೇಕಾಗಿತ್ತು. ಎಂಥವರಿಗಾದರೂ ಇದು ಅತ್ಯುಗ್ರವಾದ ಅಗ್ನಿಪರೀಕ್ಷೆಯೇ ಸರಿ. ಆದರೆ ಸ್ವಾಮೀಜಿ ಸಭೆಯ ಗುಣಮಟ್ಟಕ್ಕೆ ತಕ್ಕಂತೆ ಅತ್ಯಂತ ಯಶಸ್ವಿಯಾಗಿ ಮಾತನಾಡಿದರು. ವೇದಾಂತದ ತತ್ವಗಳಿಗೆ ಸಂಬಂಧಿಸಿದಂತೆ ಅವರು ನೀಡಿದ ವ್ಯಾಖ್ಯಾನಗಳ ವೈಖರಿಯು ಎಲ್ಲ ಪ್ರೊಫೆಸರುಗಳ ಮನಸ್ಸಿನ ಮೇಲೆ ಅಚ್ಚಳಿಯದ ಮುದ್ರೆಯನ್ನೊತ್ತಿತ್ತು. ಮತ್ತು ಅವರೆಲ್ಲರ ಹೃತ್ಪೂರ್ವಕ ಮೆಚ್ಚುಗೆಯನ್ನು ಗಳಿಸಿತು. ಸ್ವಾಮೀಜಿಯ ವಿದ್ವತ್ತಿನಿಂದ ತೀವ್ರವಾಗಿ ಪ್ರಭಾವಿತರಾದ ಅಲ್ಲಿನ ಪ್ರೊಫೆಸರುಗಳು ಅವರಿಗೆ ಹಾರ್ವರ್ಡ್ ವಿಶ್ವವಿದ್ಯಾನಿಲಯಲ್ಲಿ ಪೌರ್ವಾತ್ಯ ತತ್ತ್ವಶಾಸ್ತ್ರ ವಿಭಾಗದ ಮುಖ್ಯಸ್ಥರ ಸ್ಥಾನವನ್ನು ನೀಡಲು ಮುಂದಾದರು. ಆದರೆ ಸ್ವಾಮೀಜಿ ತಾವು ಸಂನ್ಯಾಸಿಗಳಾದ್ದರಿಂದ ಅದನ್ನು ತಾವು ಸ್ವೀಕರಿಸುವಂತಿಲ್ಲ ಎಂದರು.

ಸ್ವಾಮೀಜಿಯ ಈ ಭಾಷಣವನ್ನೂ ಪ್ರಶ್ನೆಗಳಿಗೆ ಅವರು ನೀಡಿದ ಉತ್ತರವನ್ನೂ “ವೇದಾಂತ ತತ್ತ್ವಶಾಸ್ತ್ರ” ಎಂಬ ಒಂದು ಪುಸ್ತಿಕೆಯ ರೂಪದಲ್ಲಿ ಹಾರ್ವರ್ಡ್ ವಿಶ್ವವಿದ್ಯಾನಿಲಯವು ಪ್ರಕಟಿ ಸಿತು. “ಹಾರ್ವರ್ಡ್ ಡಿವಿನಿಟಿ ಸ್ಕೂಲ್​” ಎಂಬ ಸಂಸ್ಥೆಯ ಮುಖ್ಯಸ್ಥರಾದ ರೆವರೆಂಡ್ ಡಾ ॥ ಎವರೆಟ್ ಎಂಬವರು, ಈ ಪುಸ್ತಿಕೆಯ ಪ್ರಸ್ತಾವನೆಯಲ್ಲಿ ಹೀಗೆ ಬರೆದರು:

“... ಇಲ್ಲಿ ವಿವೇಕಾನಂದರು ತಮ್ಮ ಬಗ್ಗೆಯೂ ತಮ್ಮ ಕಾರ್ಯದ ಬಗ್ಗೆಯೂ ಹೆಚ್ಚಿನ ಆಸಕ್ತಿ ಮೂಡಿಸುವುದರಲ್ಲಿ ಸಮರ್ಥರಾಗಿದ್ದಾರೆ. ನಿಜಕ್ಕೂ, ಹಿಂದೂ ತತ್ತ್ವಗಳಿಗಿಂತ ಹೆಚ್ಚು ಆಸಕ್ತಿಕರವಾದ ಅಧ್ಯಯನಯೋಗ್ಯ ವಿಷಯಗಳು ಅಪರೂಪವೇ ಸರಿ. ವೇದಾಂತವು ತಮಗೆ ನಿಲುಕಲಾರದಷ್ಟು ಎತ್ತರದಲ್ಲಿದೆಯೆಂದೂ ಅಪ್ರಾಯೋಗಿಕವೆಂದೂ ಬಹಳಷ್ಟು ಜನರು ನಂಬು ತ್ತಾರೆ. ಹಾಗಿರುವಾಗ, ಅತ್ಯುನ್ನತಮಟ್ಟದ ಬುದ್ಧಿವಂತರಾದ ವ್ಯಕ್ತಿಯೊಬ್ಬರು ಈ ವೇದಾಂತ ವನ್ನು ಪ್ರತಿನಿಧಿಸುತ್ತಿರುವುದನ್ನು ಕಣ್ಣಾರೆ ಕಾಣುವುದೊಂದು ನಮ್ಮ ವಿಶೇಷ ಭಾಗ್ಯ. ಈ ಪದ್ಧತಿಯನ್ನು ಕೇವಲ ಒಂದು ಕುತೂಹಲದ ವಿಷಯವೆಂದಾಗಲಿ, ಊಹಾಪೋಹದ ಕಟ್ಟುಕತೆ ಯೆಂದಾಗಲಿ ಪರಿಗಣಿಸಲಾಗದು. ವೇದಾಂತವನ್ನು ಎಲ್ಲ ತತ್ತ್ವವಾದದ ಮೂಲವೆಂದು ಹೆಚ್ಚು ನಿರ್ಧಾರಪೂರ್ವಕವಾಗಿ ಹೇಳಬಹುದು. ನಾವು ಪಾಶ್ಚಾತ್ಯರು ಯಾವಾಗಲೂ ಜಗತ್ತಿನ ವೈವಿಧ್ಯದ ಬಗ್ಗೆಯೇ ಚಿಂತಿಸುತ್ತೇವೆ. ಆದರೆ ನಮಗೆ ಈ ವೈವಿಧ್ಯದ ಹಿಂದಿರುವ ಒಂದು ಪರಮತತ್ತ್ವದ ಪರಿವೆಯಿಲ್ಲದೆ ಹೋದರೆ, ಈ ವೈವಿಧ್ಯವನ್ನು ನಾವು ಅರಿತುಕೊಳ್ಳಲಾರೆವು. ಭಾರತವು ಆ ಏಕತ್ವದ ಸತ್ಯವನ್ನು ನಮಗೆ ತಿಳಿಸಿಕೊಡಬಲ್ಲುದು; ಮತ್ತು ಈ ಪಾಠವನ್ನು ಅಷ್ಟೊಂದು ಪರಿಣಾಮಕಾರಿ ಯಾಗಿ ನಮಗೆ ತಿಳಿಸಿಕೊಟ್ಟ ವಿವೇಕಾನಂದರಿಗೆ ನಾವು ಚಿರಪುಣಿಗಳಾಗಿದ್ದೇವೆ.”

ಹಾರ್ವರ್ಡ್​ನ ಗ್ರ್ಯಾಜುಯೇಟ್ ಫಿಲಾಸಫಿಕಲ್ ಕ್ಲಬ್ಬಿನಲ್ಲಿ ಸ್ವಾಮೀಜಿ ಕೊಟ್ಟ ಉತ್ತರಗಳು ತುಂಬ ಚುರುಕಾಗಿದ್ದುವು, ಪಾಕ್ ಚಾತುರ್ಯದಿಂದ ಕೂಡಿದ್ದುವು, ತಾತ್ವಿಕವಾಗಿ ಒಂದು ಹೊಸತನದಿಂದೊಡಗೂಡಿ ಜೀವಂತವಾಗಿದ್ದುವು. ತಮ್ಮ ಉಪನ್ಯಾಸದಲ್ಲಿ ಸ್ವಾಮೀಜಿ ವೇದಾಂತದ ಮೂಲಭೂತ ತತ್ತ್ವಗಳ ಮತ್ತು ವಿಶ್ವನಿಯಮಗಳ ಕುರಿತಾಗಿ ಅತ್ಯಂತ ಸ್ಪಷ್ಟವಾದ ವ್ಯಾಖ್ಯಾನವನ್ನು ನೀಡಿದರು. ಜಡ ಹಾಗೂ ಚೇತನಗಳ ಕುರಿತಾಗಿ ಆಧುನಿಕ ವೈಜ್ಞಾನಿಕ ಸಿದ್ಧಾಂತ ಗಳು ಹಾಗೂ ವೇದಾಂತದ ನಡುವಣ ಸಮನ್ವಯವನ್ನು ತೋರಿಸಿಕೊಟ್ಟರು. ಬಳಿಕ ವಿಮರ್ಶಾ ಸ್ಫೂರ್ತಿಯಿಂದ ಎಲ್ಲ ಪ್ರಶ್ನೆಗಳಿಗೂ ಉತ್ತರ ನೀಡಿದರು. ಗ್ರೀಕರ \eng{Stoic} ತತ್ತ್ವಜ್ಞಾನದ ಮೇಲೆ ಹಿಂದೂ ತತ್ತ್ವಶಾಸ್ತ್ರದ ಪ್ರಭಾವ, ಭಾರತದಲ್ಲಿನ ಜಾತಿಪದ್ಧತಿ, ಅದ್ವೈತ ಹಾಗೂ ದ್ವೈತಗಳ ನಡುವಣ ಸಂಬಂಧ, ಪರಮಾರ್ಥ ತತ್ತ್ವ, ಸ್ವ-ಸಂಮೋಹಿನಿ \eng{(Self hypnotism)} ಹಾಗೂ ರಾಜಯೋಗಗಳಿಗಿರುವ ವ್ಯತ್ಯಾಸ–ಈ ಎಲ್ಲದರ ಕುರಿತಾಗಿ ಸೂಕ್ತ ಉತ್ತರಗಳನ್ನು ನೀಡಿ ತಮ್ಮ ಅಭಿಪ್ರಾಯವನ್ನು ಸ್ಪಷ್ಟಪಡಿಸಿದರು. ರಾಜಯೋಗದ ಕುರಿತಾಗಿ ಮಾತನಾಡುತ್ತ ಅವರೆಂದರು, “ಪೌರ್ವಾತ್ಯ ಮನಶ್ಶಾಸ್ತ್ರವು ಪಾಶ್ಚಾತ್ಯ ಮನಶ್ಶಾಸ್ತ್ರಕ್ಕಿಂತ ಹೆಚ್ಚು ವಿವರಪೂರ್ಣವಾದದ್ದು. ಅದು ಮನುಷ್ಯ ಅದಾಗಲೇ ಸಂಮೋಹಿನಿಗೆ ಒಳಗಾಗಿದ್ದಾನೆಂಬುದನ್ನು ಒತ್ತಿಹೇಳಿ, ಯೋಗವು ತನ್ನನ್ನು ತಾನು ಜಾಗೃತಗೊಳಿಸಿಕೊಳ್ಳುವ \eng{(De-hypnotise)} ಪ್ರಯತ್ನ ಎಂದು ಹೇಳುತ್ತದೆ. ಅದ್ವೈತಿಯು ಮಾತ್ರವೇ ತನ್ನನ್ನು ತಾನು ಸಂಮೋಹಿನಿಗೆ ಒಳಪಡಿಸಿಕೊಳ್ಳುವುದನ್ನು ತಿರಸ್ಕರಿಸುತ್ತಾನೆ. ಎಲ್ಲ ಬಗೆಯ ದ್ವೈತದಿಂದಲೂ ಸಂಮೋಹಕತೆ ಉಂಟಾಗುತ್ತದೆ ಎಂದು ತಿಳಿಯುವ ಏಕೈಕ ಸಿದ್ಧಾಂತ ಅವನದು. ಅಲ್ಲದೆ ಅವನು ಹೇಳುತ್ತಾನೆ: ‘ಜಗತ್ತನ್ನು ಎಸೆದುಬಿಡಿ, ಈ ಸಂಮೋಹಿನಿಯನ್ನು ಸಂಪೂರ್ಣವಾಗಿ ಕಿತ್ತೆಸೆಯುವಲ್ಲಿ ಯಾವುದೂ ಅಡ್ಡವಾಗಿ ನಿಲ್ಲದಿರಲಿ’ಎಂದು.” ಯೋಗಶಕ್ತಿಗಳ ಕುರಿತಾಗಿ ಒಬ್ಬರು ಪ್ರಶ್ನಿಸಿದಾಗ ಅವರು ಹೇಳಿದರು–“ಯೋಗಿ ಪವಾಹಾರಿ ಬಾಬಾರ ಜೀವನ ದಲ್ಲಿ ಅತ್ಯುನ್ನತ ಬಗೆಯ ಯೋಗಶಕ್ತಿಯು ವೇದಾಂತದ ಸ್ವರೂಪದಲ್ಲಿ ಹಾಗೂ ನಿರಂತರ ದಿವ್ಯಾ ನುಭವದಲ್ಲಿ ಅಭಿವ್ಯಕ್ತಗೊಂಡಿತು.”ಇದಕ್ಕೆ ಸಂಬಂಧಿಸಿದಂತೆ ಅವರೊಂದು ಘಟನೆಯನ್ನು ಹೇಳಿದರು–“ಒಮ್ಮೆ ಪವಾಹಾರಿ ಬಾಬಾರಿಗೆ ನಾಗರಹಾವು ಕಚ್ಚಿದ್ದರಿಂದ ಅವರು ಮೂರ್ಛಿತ ರಾಗಿ ಬಿದ್ದುಬಿಟ್ಟರು. ಸಂಜೆಯ ಹೊತ್ತಿಗೆ ಅವರಿಗೆ ಗುಣವಾಯಿತು. ಏನಾಯಿತೆಂದು ಅವರಿನ್ನು ಯಾರೋ ಕೇಳಿದಾಗ ಅವರು, ‘ನನ್ನ ಪ್ರಿಯತಮನಾದ ಭಗವಂತನಿಂದ ಒಬ್ಬ ದೂತ ಬಂದಿದ್ದ’ ಎಂದರು. ದ್ವೇಷ, ಕೋಪ, ಅಸೂಯೆಗಳೆಲ್ಲ ಅವರಲ್ಲಿ ಸುಟ್ಟುಹೋಗಿದ್ದುವು. ಎಂತಹ ಪ್ರಚೋದನೆಯೂ ಅವರಲ್ಲಿ ಉದ್ವೇಗವುಂಟಾಗುವಂತೆ ಮಾಡಲು ಸಾಧ್ಯವಿರಲಿಲ್ಲ. ಅವರು ಅನಂತ ಪ್ರೇಮವೇ ಆಗಿದ್ದರು. ಹೀಗಿರುವವನೇ ನಿಜವಾದ ಯೋಗಿ.” ಪ್ರೊಫೆಸರರೊಬ್ಬರು “ವೇದಾಂತದ ದೃಷ್ಟಿಕೋನದಲ್ಲಿ, ನಾಗರಿಕತೆಯೆಂಬುದು ಹೇಗಿರುತ್ತದೆ?” ಎಂದು ಕೇಳಿದಾಗ ಸ್ವಾಮೀಜಿ ಹೇಳಿದರು–“ಮಾನವನೊಳಗಿರುವ ದೈವಿಕತೆ ಅಭಿವ್ಯಕ್ತಿಗೊಳ್ಳುವ ನಾಗರಿಕತೆಯೇ ನಿಜವಾದ ನಾಗರಿಕತೆ, ಮತ್ತು ಎಲ್ಲಿ ಅತ್ಯುನ್ನತ ಆದರ್ಶಗಳು ಅನುಷ್ಠಾನಸಾಧ್ಯವಾಗುತ್ತವೆಯೋ ಆ ನಾಡೇ ನಿಜಕ್ಕೂ ಅತಿ ಹೆಚ್ಚು ನಾಗರಿಕವಾದದ್ದು.”

ಸ್ವಾಮೀಜಿಯವರು ಹಾರ್ವರ್ಡ್ ವಿಶ್ವವಿದ್ಯಾಲಯದಲ್ಲಿ ಈ ಪ್ರಸಿದ್ಧ ಉಪನ್ಯಾಸವನ್ನು ನೀಡು ವುದಕ್ಕೆ ಮೊದಲು ಮಾರ್ಚ್ ೨೨ ಹಾಗೂ ೨೪ರಂದು ಕೇಂಬ್ರಿಡ್ಜಿನ ಶ್ರೀಮತಿ ಸಾರಾ ಬುಲ್ಲಳ ಮನೆಯಲ್ಲಿ ಹಾರ್ವರ್ಡ್​ನ ವಿದ್ಯಾರ್ಥಿಗಳ ಗುಂಪೊಂದನ್ನು ಉದ್ದೇಶಿಸಿ ಮಾತನಾಡಿದರು. ಅಲ್ಲದೆ ಮಾರ್ಚ್ ೨೮ರ ಮಧ್ಯಾಹ್ನ ‘ಟ್ವೆಂಟಿಯತ್ ಸೆಂಚುರಿ ಕ್ಲಬ್ಬಿ’ನಲ್ಲಿ “ವೇದಾಂತ–ಅದರ ಅನು ಷ್ಠಾನ; ಇತರ ತತ್ತ್ವಜ್ಞಾನಗಳಿಗಿಂತ ಹೇಗೆ ಅದು ವಿಭಿನ್ನವಾದುದು” ಎಂಬ ವಿಷಯವಾಗಿ ಮಾತನಾಡಿದರು.

ಸ್ವಾಮೀಜಿ ಸಾರಿದ ವೇದಾಂತದ ಸಂದೇಶವು ಬೌದ್ಧಿಕವಾಗಿ ಅಷ್ಟೊಂದು ತೃಪ್ತಿದಾಯಕವಾಗಿ ತೋರಿದರೂ ಅದು ಕೇವಲ ಒಂದು ಬೌದ್ಧಿಕ ಕಸರತ್ತಲ್ಲ; ಅದು ಅತ್ಯುನ್ನತ ಮಟ್ಟದ ನೈತಿಕತೆ ಯನ್ನೂ ಮಾನವಸಹಜವಾದ ಉದಾತ್ತ ಭಾವಗಳನ್ನೂ ಒಳಗೊಂಡಿದೆ ಎಂಬುದನ್ನು ಅವರ ಸಂಪರ್ಕಕ್ಕೆ ಬಂದವರೆಲ್ಲರೂ ಕಂಡುಕೊಂಡರು. ಅವರ ಸಿದ್ಧಾಂತದಲ್ಲಿ ತರ್ಕಕ್ಕೆ ಎಡೆಯಿದ್ದರೂ ತನ್ನ ಆಂತರಿಕ ಸ್ವರೂಪದಲ್ಲಿ ಅದು ಧಾರ್ಮಿಕವೇ ಆಗಿದ್ದು ಸಾಕ್ಷಾತ್ಕಾರವನ್ನೇ ಮಾನವನ ಗುರಿಯೆಂದು ಎತ್ತಿಹಿಡಿದಿತ್ತು. ಸ್ವಾಮೀಜಿಯ ಪಾಲಿಗೆ ಮಾತ್ರ ಹಾರ್ವರ್ಡಿನಲ್ಲಾದ ಈ ಅನು ಭವ ಕೇವಲ ಅವರ ಇನ್ನಿತರ ನೂರಾರು ಅನುಭವಗಳಲ್ಲಿ ಒಂದಾಗಿತ್ತು, ಅಷ್ಟೆ. ಅವರು ಹೋದ ಕಡೆಗಳಲ್ಲೆಲ್ಲ ಜನರ ಗುಂಪು ಅವರನ್ನು ಮುತ್ತಿಕೊಂಡು ಪ್ರಶ್ನೆಗಳ ಮಳೆಗರೆಯುತ್ತಿತ್ತು. ಅವರೆಲ್ಲ ರಿಗೂ ಸ್ವಾಮೀಜಿಯ ಸಿದ್ಧ ಉತ್ತರ ಇದ್ದೇ ಇರುತ್ತಿತ್ತು. ಈ ಪ್ರಾಶ್ನಿಕರ ಗುಂಪಿನಲ್ಲಿ ಅವರ ಬೌದ್ಧಿಕತೆಯ ಔನ್ನತ್ಯ ಎದ್ದು ಕಾಣುತ್ತಿತ್ತು.

ಬಾಸ್ಟನ್ನಿನಲ್ಲಿದ್ದ ಈ ಸಂದರ್ಭದಲ್ಲಿ ಸ್ವಾಮೀಜಿ ತಮ್ಮ ಅನೇಕ ಹಳೆಯ ಮಿತ್ರರನ್ನು ಭೇಟಿ ಮಾಡಿದರು. ಅವರು ತಮ್ಮ ನೆಚ್ಚಿನ ಗೆಳೆಯರಾದ ರೈಟರೊಂದಿಗೆ ದೀರ್ಘ ಸಂಭಾಷಣೆ ನಡೆಸಿ ದರು. ಅವರ ಮುಂದೆ ಸ್ವಾಮೀಜಿ ತಮ್ಮ ಸ್ವಂತ ತಂದೆಯ ಮುಂದೆ ಹೇಳಿಕೊಳ್ಳುವಂತೆ ಅಮೆರಿಕ ದಲ್ಲಿ ತಮಗಾದ ನಾನಾ ಬಗೆಯ ಅನುಭವಗಳನ್ನೆಲ್ಲ ಹೇಳಿಕೊಂಡರು. ಅವರು ಪಡೆದ ಯಶಸ್ಸನ್ನು ಕಂಡು ಪ್ರೊ ॥ ರೈಟ್ ಮನಸಾರೆ ಹಿಗ್ಗಿದರು. ದೀರ್ಘಕಾಲದ ಬಳಿಕ ಮತ್ತೊಮ್ಮೆ ಭೇಟಿಯಾಗುವ ಅವಕಾಶ ಒದಗಿದ್ದಕ್ಕಾಗಿ ರೈಟರಿಗೆ ಅತೀವ ಆನಂದವಾಯಿತು. ಅವರು ತಮ್ಮ ಪತ್ನಿಗೆ ಪತ್ರ ಬರೆಯುತ್ತ ತಿಳಿಸಿದರು–“ಸ್ವಾಮಿ ವಿವೇಕಾನಂದರು ಬಹಳಷ್ಟು ಮೃದುವಾಗಿದ್ದಾರೆ, ಹೆಚ್ಚಿನ ಲೋಕಾನುಭವ ಗಳಿಸಿದ್ದಾರೆ, ಮತ್ತು ಮಧುರರಾಗಿದ್ದಾರೆ. ನಿಜಕ್ಕೂ ಅವರು ಅತ್ಯಂತ ಆಕರ್ಷಕ ರಾಗಿದ್ದಾರೆ... ಅವರು ಪ್ರೊ ॥ ಜೇಮ್ಸ್​ರನ್ನು ಭಾವಪರವಶರನ್ನಾಗಿ ಮಾಡಿಬಿಟ್ಟರು. ಅವರ ಉಪನ್ಯಾಸವನ್ನು ಕೇಳುವ ಪ್ರತಿಯೊಂದು ಅವಕಾಶವನ್ನೂ ಉಪಯೋಗಿಸಿಕೊಳ್ಳಲು, ಪ್ರೊ ॥ ವಿಲಿಯಂ ಜೇಮ್ಸ್ ಇಂದು ಸಂಜೆ ಬಾಸ್ಟನ್ನಿಗೆ ಹೋಗುತ್ತಿದ್ದಾರೆ...”

ಬಾಸ್ಟನ್ನಿನಿಂದ ಸ್ವಾಮೀಜಿ ಮಾರ್ಚ್ ೩ಂರಂದು ಶಿಕಾಗೋಗೆ ಪಯಣಿಸಿದರು. ಇಲ್ಲಿ ಅವರು ತಮ್ಮ ಮಿತ್ರರು ಏರ್ಪಡಿಸಿದ್ದ ಅನೇಕ ತರಗತಿಗಳನ್ನು ನಡೆಸಿದರು. ಅವರು ಸುಮಾರು ಎರಡು ವಾರಗಳ ಕಾಲ ಶಿಕಾಗೋದಲ್ಲಿದ್ದು ತಮ್ಮ ಪ್ರೀತಿಪಾತ್ರರಾದ ಹೇಲ್ ಕುಟುಂಬದವರೆನ್ನೆಲ್ಲ ಭೇಟಿಯಾಗಿ ನ್ಯೂಯಾರ್ಕಿಗೆ ಮರಳಿದರು. ಹಾರ್ವರ್ಡ್ ವಿಶ್ವವಿದ್ಯಾನಿಲಯವು ತಮ್ಮ ಭಾಷಣ ಹಾಗೂ ಪ್ರಶ್ನೋತ್ತರ ಕಾರ್ಯಕ್ರಮವನ್ನು ಆಧರಿಸಿ ಪ್ರಕಟಿಸಲಿದ್ದ ಪುಸ್ತಿಕೆಗೆ ಅವರು ವಿವರಣಾ ತ್ಮಕವಾದ ಟಿಪ್ಪಣಿಗಳನ್ನು ಬರೆದು ಕಳಿಸಿಕೊಟ್ಟರು.

ಅಮೆರಿಕದಲ್ಲಿ ಹೆಚ್ಚೆಚ್ಚು ದಿನಗಳನ್ನು ಕಳೆದಂತೆಲ್ಲ ತಮ್ಮ ಧಾರ್ಮಿಕ ಬೋಧನೆಗಳಿಗೆ ವ್ಯವಸ್ಥಿತ ರೂಪವೊಂದನ್ನು ಕೊಡುವುದು ಆವಶ್ಯಕವೆಂದು ಅವರಿಗೆ ತೋರಿತು. ಆದ್ದರಿಂದ ಸಮಗ್ರ ಭಾರತೀಯ ತತ್ತ್ವಜ್ಞಾನವನ್ನೇ ಪುನರ್ವ್ಯವಸ್ಥೆಗೊಳಿಸಿ ಅದರ ಪ್ರಮುಖ ಲಕ್ಷಣಗಳನ್ನು ವಿಂಗಡಿಸಿ ಪಾಶ್ಚಾತ್ಯರಿಗೆ ಅವು ಚೆನ್ನಾಗಿ ಮನಮುಟ್ಟುವಂತೆ ಮಾಡಬೇಕೆಂದು ಆಲೋಚಿಸಿದರು. ಬೇರೆ ಬೇರೆ ವೇದಾಂತ ದರ್ಶನಗಳ ಪ್ರಕಾರ ಜೀವ-ಈಶ್ವರ ಸಂಬಂಧ, ಜೀವನದ ಪರಮಗತಿ, ಜಡ-ಚೇತನ ಗಳಿಗಿರುವ ಪರಸ್ಪರ ಸಂಬಂಧ, ವಿಶ್ವದ ರಚನೆ ಹಾಗೂ ವಿಕಾಸದ ಕುರಿತಾಗಿ ವೇದಾಂತದ ವಿಭಿನ್ನ ಶಾಖೆಗಳ ಅಭಿಪ್ರಾಯಗಳು–ಇವುಗಳನ್ನೆಲ್ಲ ಸೂಕ್ತವಾಗಿ ವಿವರಿಸಬೇಕೆಂದು ಆಶಿಸಿದರು. ಆಧುನಿಕ ವಿಜ್ಞಾನದ ಕೆಲವು ಅದ್ಭುತ ಆವಿಷ್ಕಾರಗಳೂ ಕೂಡ ತಲೆತಲಾಂತರಗಳ ಹಿಂದೆ ಭಾರತದಲ್ಲಿ ಕಂಡುಹಿಡಿದಿದ್ದರ ಪುನರಾವಿಷ್ಕಾರಗಳಷ್ಟೆ ಎಂಬುದನ್ನು ತೋರಿಸಿಕೊಡಲು ಇಚ್ಛಿಸಿ ದರು. ಅಲ್ಲದೆ ಉಪನಿಷತ್ತುಗಳನ್ನು ಅದ್ವೈತ, ವಿಶಿಷ್ಟಾದ್ವೈತ ಹಾಗೂ ದ್ವೈತ ಭಾವನೆಗಳಿಗನುಗುಣ ವಾಗಿ ವಿಂಗಡಿಸಿ ಅವುಗಳೆಲ್ಲದರ ಸಮನ್ವಯವನ್ನು ಎತ್ತಿತೋರಿಸುವುದೂ ಅವರ ಉದ್ದೇಶಗಳ ಲ್ಲೊಂದಾಗಿತ್ತು. ತಮ್ಮ ಈ ಎಲ್ಲ ಆಲೋಚನೆಗಳಿಗೆ ಸ್ಪಷ್ಟ ರೂಪವನ್ನು ನೀಡಿ ಪುಸ್ತಕವೊಂದನ್ನು ಬರೆಯಲು ಅವರು ನಿಶ್ಚಯಿಸಿದರು. ತಮ್ಮ ಈ ನಿಶ್ಚಯದ ಕುರಿತಾಗಿ ಸುಮಾರು ಒಂದು ವರ್ಷದ ಹಿಂದೆಯೇ ಅಳಸಿಂಗ ಪೆರಮಾಳರಿಗೆ ಒಂದು ಪತ್ರವನ್ನು ಬರೆದಿದ್ದರು. ಅದೊಂದು ವಿಚಾರ ಪೂರ್ಣವಾದ ಪತ್ರ. ಅದು ಹೀಗಿದೆ:

“...ಈಗ ನಾನು ನಿನಗೆ ನನ್ನ ಸಂಶೋಧನೆಯನ್ನು ತಿಳಿಸುತ್ತೇನೆ. ಧರ್ಮದ ಸಮಸ್ತ ವಿಚಾರ ಗಳೂ ವೇದಾಂತದಲ್ಲಿ, ಎಂದರೆ ವೇದಾಂತ ತತ್ತ್ವದ ಮೂರು ಸ್ತರಗಳಾದ ದ್ವೈತ-ವಿಶಿಷ್ಟಾದ್ವೈತ- ಅದ್ವೈತಗಳಲ್ಲಿ ಅಡಕವಾಗಿವೆ. ಇವು ಒಂದಾದ ಮೇಲೊಂದರಂತೆ ಸರದಿಯಲ್ಲಿ ಬರುತ್ತವೆ. ಇವು ಮಾನವನ ಆಧ್ಯಾತ್ಮಿಕ ಪ್ರಗತಿಯ ಮೂರು ಹಂತಗಳು. ಇವುಗಳಲ್ಲಿ ಪ್ರತಿಯೊಂದೂ ಕೂಡ ಆವಶ್ಯಕ. ಧರ್ಮದ ಸಾರವೆಂದರೆ ಇದೇ: ವೇದಾಂತವನ್ನು ಭಾರತದ ಬೇರೆಬೇರೆ ಜನಾಂಗಗಳ ಸಂಪ್ರದಾಯಗಳಿಗೆ ಹಾಗೂ ಮತಗಳಿಗೆ ಅನ್ವಯಿಸಿದಾಗ ಅದು ಹಿಂದೂಧರ್ಮ. ಮೊದಲನೆಯ ಸ್ತರವನ್ನು, ಎಂದರೆ ದ್ವೈತವನ್ನು, ಯೂರೋಪಿನ ಜನಾಂಗಗಳಿಗೆ ಅನ್ವಯಿಸಿದಾಗ ಅದು ಕ್ರೈಸ್ತ ಧರ್ಮ. ಅದನ್ನೇ ಸಿಮಿಟಿಕ್ ಪಂಗಡಗಳಿಗೆ ಅನ್ವಯಿಸಿದರೆ ಅದು ಇಸ್ಲಾಂ ಧರ್ಮ. ಅದ್ವೈತವನ್ನು ಯೋಗದ ದೃಷ್ಟಿಯಿಂದ ನೋಡಿದಾಗ ಅದು ಬೌದ್ಧಧರ್ಮ. ಈಗ, ಧರ್ಮವೆಂದರೆ ವೇದಾಂತ ವೆಂದೇ ಅರ್ಥ. ಬೇರೆ ಬೇರೆ ದೇಶಗಳ ಬೇರೆಬೇರೆ ಪರಿಸರ, ಬೇರೆಬೇರೆ ಅವಶ್ಯಕತೆಗಳು ಮತ್ತು ಇತರ ಪರಿಸ್ಥಿತಿಗಳಿಗೆ ಅನುಗುಣವಾಗಿ ವೇದಾಂತವನ್ನು ಅನ್ವಯಿಸಿಕೊಳ್ಳುವ ರೀತಿ ಬದಲಾಗ ಬೇಕು, ಅಷ್ಟೆ. ತತ್ತ್ವ ಒಂದೇ ಆಗಿದ್ದರೂ ಅದನ್ನು ಶಾಕ್ತರು, ಶೈವರು ಮೊದಲಾದವರು ತಮ್ಮ ತಮ್ಮ ಪಂಗಡಗಳಿಗೆ ಅಥವಾ ಆಚರಣೆಗಳಿಗೆ ಬೇಕಾದಂತೆ ಅನ್ವಯಿಸಿಕೊಳ್ಳುತ್ತಾರೆ. ಈ ಮೂರು ಸಿದ್ಧಾಂತಗಳ ಸಮನ್ವಯವನ್ನು ಎತ್ತಿತೋರಿಸಿ, ಮತ್ತು ಕೇವಲ ಬಾಹ್ಯಾಚರಣೆಗಳಿಗೆ ಸಂಬಂಧಿಸಿ ದವುಗಳನ್ನು ಸಂಪೂರ್ಣ ಬದಿಗೊತ್ತಿ ನಿನ್ನ ಪತ್ರಿಕೆಯಲ್ಲಿ ಒಂದಾದ ಮೇಲೊಂದರಂತೆ ಲೇಖನ ಗಳನ್ನು ಬರೆ. ಎಂದರೆ, ತತ್ತ್ವವನ್ನು–ತತ್ತ್ವದ ಆಧ್ಯಾತ್ಮಿಕ ಅಂಶವನ್ನು ಮಾತ್ರ ಪ್ರಚಾರ ಮಾಡು. ಜನ ಅದನ್ನು ತಮ್ಮತಮ್ಮ ಆಚರಣೆಗಳಿಗೆ ತಾವೇ ಹೊಂದಿಸಿಕೊಳ್ಳಲಿ. ನಾನು ಈ ವಿಷಯದ ಮೇಲೊಂದು ಪುಸ್ತಕವನ್ನು ಬರೆಯಬೇಕೆಂದಿದ್ದೇನೆ...”

ಪಾಶ್ಚಾತ್ಯರ ಆಧುನಿಕ ವಿಚಾರಧಾರೆಗೆ ಒಪ್ಪಿಗೆಯಾಗುವಂತೆ ಭಾರತೀಯ ವಿಚಾರಧಾರೆಯನ್ನು ವಿವರಿಸಿ ಹೇಳುವುದು ತಮ್ಮ ತಕ್ಷಣದ ಕೆಲಸವೆಂದು ಸ್ವಾಮೀಜಿ ಆಲೋಚಿಸಿದರು. ಈ ಸಂಬಂಧವಾಗಿ ಅಳಸಿಂಗ ಪೆರುಮಾಳರಿಗೆ ಬರೆದ ಮತ್ತೊಂದು ಪತ್ರದಲ್ಲಿ ತಮ್ಮ ಆಲೋಚನೆ ಯನ್ನು ವ್ಯಕ್ತಪಡಿಸಿದರು:

“ಹಿಂದೂ ಧಾರ್ಮಿಕ ತತ್ತ್ವಗಳನ್ನು ಪ್ರತಿಯೊಬ್ಬರಿಗೂ ಅರ್ಥವಾಗುವಂತೆ ಇಂಗ್ಲಿಷಿನಲ್ಲಿ ಹೇಳಬೇಕಾಗಿದೆ; ಮತ್ತು ಒಣ ತತ್ತ್ವಶಾಸ್ತ್ರ, ವಿಚಿತ್ರವಾದ ಪುರಾಣ ಕಥೆಗಳು ಹಾಗೂ ದಿಗ್ಭ್ರಮೆ ಗೊಳಿಸುವಂತಹ ಮನಶ್ಶಾಸ್ತ್ರ–ಇವುಗಳಿಂದ ಕೂಡಿದ ಹಿಂದೂ ಗ್ರಂಥರಾಶಿಯಿಂದ ಸುಲಭವೂ ಸರಳವೂ ಸರ್ವಜನಗ್ರಾಹ್ಯವೂ ಆದಂತಹ ಒಂದು ಧರ್ಮವನ್ನು ಆವಿಷ್ಕರಿಸಬೇಕಾಗಿದೆ. ಅಲ್ಲದೆ ಈ ಧರ್ಮವು, ಅತ್ಯುನ್ನತ ಬುದ್ಧಿಮತ್ತೆಯನ್ನೂ ತೃಪ್ತಿಪಡಿಸುವಂತಿರಬೇಕಾಗುತ್ತದೆ. ಈ ಕಾರ್ಯ ವನ್ನು ಕೈಗೆತ್ತಿಕೊಂಡರೆ ಎದುರಿಸಬೇಕಾದ ಕಷ್ಟಗಳೆಂಥದು ಎಂಬುದನ್ನು, ಯಾರು ಆ ಪ್ರಯತ್ನ ದಲ್ಲಿ ತೊಡಗುತ್ತಾರೆಯೋ ಅವರು ಮಾತ್ರವೇ ಬಲ್ಲರು. ಸುಲಭಗ್ರಾಹ್ಯವಲ್ಲದ ಅದ್ವೈತ ತತ್ತ್ವವು ದಿನನಿತ್ಯದ ಜೀವನದಲ್ಲಿ ಜೀವಂತವಾಗಬೇಕು, ಕಾವ್ಯಮಯವಾಗಬೇಕು. ಜುಗುಪ್ಸೆಗೊಳಿಸುವಷ್ಟು ಜಟಿಲವಾಗಿರುವ ಪುರಾಣಗಳಿಂದ ಅತ್ಯಂತ ಸ್ಫುಟವಾದ ನೈತಿಕ ಆದರ್ಶಗಳು ಒಡಮೂಡಬೇಕು. ದಿಗ್ಭ್ರಮೆಗೊಳಿಸುವಂತಿರುವ ಯೋಗಸಿದ್ಧಾಂತದಿಂದ ಅತ್ಯಂತ ವೈಜ್ಞಾನಿಕವೂ ಅನುಷ್ಠಾನ ಯೋಗ್ಯವೂ ಆದ ಮನಶ್ಶಾಸ್ತ್ರ ಹೊರಹೊಮ್ಮಬೇಕು. ಮತ್ತು ಇವೆಲ್ಲವನ್ನೂ ಒಂದು ಮಗುವೂ ಅರ್ಥಮಾಡಿಕೊಳ್ಳಲು ಸಾಧ್ಯವಾಗುವಂತೆ ಅತ್ಯಂತ ಸರಳ ರೂಪದಲ್ಲಿ ತಿಳಿಯಹೇಳಬೇಕು. ಇದೇ ನನ್ನ ಜೀವನದ ಕಾರ್ಯ. ಇದರಲ್ಲಿ ನಾನು ಎಷ್ಟರಮಟ್ಟಿಗೆ ಯಶಸ್ವಿಯಾಗುತ್ತೇನೆಯೋ ಭಗವಂತ ನಿಗೇ ಗೊತ್ತು. ಕರ್ಮಮಾಡಲು ಮಾತ್ರವೇ ನಮಗೆ ಅಧಿಕಾರವುಂಟು, ಫಲಕ್ಕಲ್ಲ. ಇದು ಕಷ್ಟದ ಕೆಲಸ, ಮಗು, ಕಷ್ಟದ ಕೆಲಸ!”

ಸ್ವಾಮೀಜಿ ಕೈಗೊಂಡಿದ್ದ ಕಾರ್ಯವು–ಅವರೇ ಹೇಳುವಂತೆ–ಒಂದು ಭಗೀರಥ ಪ್ರಯತ್ನವೇ ಆಗಿತ್ತು. ಆದರೆ ಆ ಪ್ರಯತ್ನದಲ್ಲಿ ಅವರು ಬಹುಮಟ್ಟಿಗೆ ಯಶಸ್ವಿಯಾದರೆಂಬುದು ಅತಿ ಮುಖ್ಯವಾದ ಅಂಶ. ಅವರ ಬೋಧನೆಗಳೆಲ್ಲವೂ ವೇದಾಂತದ ತತ್ತ್ವಗಳನ್ನು ಬಲವಾಗಿ ಆಧರಿಸಿದ್ದರೂ ಅವು ಆಧುನಿಕ ಯುಗಕ್ಕೆ ತಕ್ಕಂತಿದ್ದುವು, ಸುಸ್ಪಷ್ಟವಾಗಿದ್ದುವು. ಹಿಂದೂ ತತ್ತ್ವಶಾಸ್ತ್ರದ ಕುರಿತಾದ ಅವರ ವ್ಯಾಖ್ಯಾನಗಳು ಅಚ್ಚರಿಯುಂಟುಮಾಡುವಷ್ಟು ವಿಶಿಷ್ಟವೂ ಸ್ವಂತಿಕೆಯುಳ್ಳವೂ ಆಗಿವೆ. ಭಾರತೀಯ ಆಧ್ಯಾತ್ಮಿಕ ಚಿಂತನೆಗಳು ವೈಜ್ಞಾನಿಕ ತಳಹದಿಯ ಮೇಲೆ ನಿಂತಿವೆಯೆಂಬುದನ್ನು ತೋರಿಸಿಕೊಟ್ಟ ಮೊದಲ ಭಾರತೀಯರೆಂದರೆ ಸ್ವಾಮಿ ವಿವೇಕಾನಂದ ರೆಂದು ನಿಸ್ಸಂದೇಹವಾಗಿ ಹೇಳಬಹುದು. ಅಲ್ಲದೆ, ಹಿಂದೂ ಪೌರಾಣಿಕ ಕಥೆಗಳ ಸಾಂಕೇತಿಕ ಮಹತ್ವವನ್ನೂ ಅವು ಎತ್ತಿಹಿಡಿಯುವ ನೈತಿಕ ಮೌಲ್ಯಗಳನ್ನೂ ತೋರಿಸಿಕೊಟ್ಟು ಜನರ ಕಣ್ದೆರೆಸಿ ದವರಲ್ಲಿ ಅವರು ಅಗ್ರಗಣ್ಯರು. ಈ ದೃಷ್ಟಿಯಿಂದಲೂ, ಹಿಂದೂಧರ್ಮದ ಪುನರುತ್ಥಾನಕ್ಕೆ ಸ್ವಾಮೀಜಿ ಸಲ್ಲಿಸಿದ ಸೇವೆ ತುಂಬ ಮಹತ್ವಪೂರ್ಣವಾದುದು.

