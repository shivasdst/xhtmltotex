
\chapter{ವೈವಿಧ್ಯಮಯ ವಿಹಾರಗಳು}

\noindent

ಈ ವೇಳೆಗೆ ಲಂಡನ್ನಿನಲ್ಲಿ ಸ್ವಾಮೀಜಿಯ ತರಗತಿಗಳೂ ಮುಗಿಯುತ್ತ ಬಂದಿದ್ದುವು. ಜುಲೈ ತಿಂಗಳೆಂದರೆ ಇಂಗ್ಲೆಂಡಿನಲ್ಲಿ ಉಪನ್ಯಾಸಗಳ ಪುತುವಿನ ಮುಕ್ತಾಯದ ಕಾಲ, ಮತ್ತು ಬೇಸಿಗೆಯ ಪ್ರಾರಂಭ. ಆಗ ನಗರಗಳ ಜನರು ಗುಂಪುಗುಂಪಾಗಿ ಗ್ರಾಮೀಣ ಪ್ರದೇಶಗಳಿಗೆ ಇಲ್ಲವೆ ಸಮುದ್ರ ತೀರಕ್ಕೆ ಹೊರಟುಬಿಡುತ್ತಾರೆ. ಈ ಸಮಯಕ್ಕೆ ತಮ್ಮ ಉಪನ್ಯಾಸಗಳನ್ನೂ ತರಗತಿಗಳನ್ನೂ ನಿಲ್ಲಿಸಲು ಸ್ವಾಮೀಜಿ ಕೂಡ ನಿರ್ಧರಿಸಿದ್ದರು. ಅಲ್ಲದೆ ಅತಿಯಾದ ಪರಿಶ್ರಮದಿಂದ ಅವರ ದೇಹಾರೋಗ್ಯವು ದಿನದಿನಕ್ಕೂ ಹದಗೆಡುತ್ತಿದ್ದುದನ್ನು ಗಮನಿಸಿ ಅವರ ಆಪ್ತರು ಆತಂಕ ಗೊಂಡಿದ್ದರು. ಈ ರಜಾದಿನಗಳಲ್ಲಿ ವಿಶ್ರಾಂತಿಗಾಗಿ ಅವರನ್ನು ತಮ್ಮೊಂದಿಗೆ ವಿಹಾರಧಾಮಗಳಿಗೆ ಕರೆದೊಯ್ಯುವ ಯೋಜನೆ ಹಾಕಿದರು. ಆದರೆ ಅವರ ಮುಂದಿನ ಪುತುವಿನಲ್ಲಿ ತರಗತಿಗಳು ಮುಂದುವರಿಯುವುದನ್ನು ಖಾತ್ರಿ ಮಾಡಿಕೊಳ್ಳಲೋ ಎಂಬಂತೆ, ಅವರ ವಿದ್ಯಾರ್ಥಿಗಳೆಲ್ಲ ಅದಕ್ಕಾಗಿ ಆಗಲೇ ವಂತಿಗೆ ಸಂಗ್ರಹಿಸಿ, ಸಿದ್ಧತೆಗಳನ್ನು ಮಾಡಿದ್ದರು.

ಸ್ವಾಮೀಜಿಯ ಆತ್ಮೀಯ ಶಿಷ್ಯರಾದ ಸೇವಿಯರ್ ದಂಪತಿಗಳು ಮತ್ತು ಮಿಸ್ ಹೆನ್ರಿಟಾ ಮುಲ್ಲರ್, ತಮ್ಮೊಂದಿಗೆ ಯೂರೋಪಿನ ಪ್ರವಾಸಕ್ಕೆ ಬರುವಂತೆ ಅವರನ್ನು ಆಹ್ವಾನಿಸಿದರು. ಯೂರೋಪ್ ಪ್ರವಾಸದ ಈ ಯೋಜನೆಯನ್ನು ಕೇಳಿ ಸ್ವಾಮೀಜಿ ಮಗುವಿನಂತೆ ಉತ್ಸಾಹಭರಿತ ರಾದರು. ಹಿಮಾಮೃತವಾದ ಆಲ್ಪ್ಸ್ ಪರ್ವತಶ್ರೇಣಿ ಹಾಗೂ ಅದರ ಮಡಿಲಲ್ಲಿರುವ ಸ್ವಿಟ್ಸರ್ ಲ್ಯಾಂಡನ್ನು ಸಂದರ್ಶಿಸುವ ಆಲೋಚನೆಯೇ ಅವರಲ್ಲಿ ನವಸ್ಫೂರ್ತಿಯನ್ನುಕ್ಕಿಸಿತು. ಅದರಲ್ಲೂ ಹಿಮಬಂಡೆಗಳಿಂದ ಕೂಡಿದ ಗ್ಲೇಷಿಯರ್​(ನೀರ್ಗಲ್ಲಿನ ನದಿ)ಗಳನ್ನು ಯಾವಾಗ ದಾಟಿಯೇ ನೆಂದು ಅವರು ಉತ್ಕಂಠಿತರಾದರು.

ಸ್ವಿಟ್ಸರ್​ಲ್ಯಾಂಡ್ ಜಗತ್ತಿನ ಮುಖ್ಯ ಪ್ರವಾಸೀ ಆಕರ್ಷಣೆಗಳಲ್ಲೊಂದು. ಇದಕ್ಕೆ ಯೂರೋ ಪಿನ ಕ್ರೀಡಾಂಗಣವೆಂದೇ ಹೆಸರು. ಈ ದೇಶದಲ್ಲಿ ನೂರಾರು ಅತ್ಯುನ್ನತ ಪರ್ವತ ಶಿಖರಗಳಿವೆ. ಚಳಿಗಾಲದಲ್ಲಿ ಇವು ವಿಧವಿಧದ ಜಾರುವ ಆಟಗಳಿಗೆ ಅತ್ಯಂತ ಸೂಕ್ತವಾದ ಕ್ರೀಡಾಂಗಣಗಳು; ಬೇಸಿಗೆಯು ಪರ್ವತಾರೋಹಣಕ್ಕೆ ಅನುಕೂಲವಾದದ್ದು. ಆದ್ದರಿಂದ ವರ್ಷವಿಡೀ ಪ್ರವಾಸಿಗರು ಇಲ್ಲಿಗೆ ಮುತ್ತುತ್ತಾರೆ. ಅಲ್ಲದೆ ಅಸಂಖ್ಯಾತ ಜಲಪಾತಗಳೂ ನೀರ್ಗಲ್ಲ ನದಿಗಗಳೂ ತುಂಬಿರುವ ನಾಡು ಸ್ವಿಟ್ಸರ್​ಲ್ಯಾಂಡ್. ಇಂತಹ ದೇಶವನ್ನು ಸಂದರ್ಶಿಸಲು ಸ್ವಾಮೀಜಿ ಉತ್ಸುಕರಾದದ್ದು ಸಹಜವೇ ಆಗಿದೆ.

ಆದರೆ, ಸೌಂದರ್ಯದ ಬೀಡಾದ ಸ್ಟಿಟ್ಸರ್​ಲ್ಯಾಂಡಿನ ಪ್ರವಾಸವು ಆ ದಿನಗಳಲ್ಲಿ ಸಾಹಸ ಮಯವೂ ಆಗಿತ್ತು. ಈಗಿನಂತೆ ಆಗ ರಸ್ತೆಗಳ ಅಥವಾ ರೈಲುದಾರಿಗಳ ಅನುಕೂಲವಿರಲಿಲ್ಲ. ಇಲ್ಲಿನ ಕಡಿದಾದ ಪರ್ವತ ಪ್ರದೇಶದಲ್ಲಿ ಪ್ರಯಾಣ ಮಾಡಲು ಹೇಸರಗತ್ತೆಗಳನ್ನು ಇಲ್ಲವೆ ದಂಡಿ(ಮೇನೆ)ಗಳನ್ನು ಅವಲಂಬಿಸಬೇಕಾಗಿತ್ತು. ಅಲ್ಲದೆ ದುಬಾರಿ ವಿಶ್ರಾಂತಿಧಾಮಗಳು ಮತ್ತು ವಾರಗಟ್ಟಲೆಯ ಪ್ರಯಾಣದಿಂದಾಗಿ ಹಣವನ್ನು ಧಾರಾಳವಾಗಿ ಖರ್ಚುಮಾಡಲು ಸಿದ್ಧರಿರ ಬೇಕಾಗಿತ್ತು. ಆದರೆ ಸ್ವಾಮೀಜಿಯ ಸಂಗಡಿಗರು ಸಾಕಷ್ಟು ಶ್ರೀಮಂತರೂ ಉದಾರಿಗಳೂ ಆಗಿದ್ದುದರಿಂದ ಹಣದ ವಿಷಯಲ್ಲಿ ಅವರೇನೂ ಚಿಂತಿಸಬೇಕಾಗಿರಲಿಲ್ಲ. ಅಲ್ಲದೆ ಆ ಸಾಹಸದ ಯಾತ್ರೆಗೆ ಅವರು ಮಾನಸಿಕವಾಗಿ ಸಿದ್ಧರಿದ್ದರು. ಈ ಪ್ರವಾಸ ಒಟ್ಟು ಒಂಬತ್ತು ವಾರಗಳ ದ್ದಾಗಿತ್ತು. ಸ್ವಿಟ್ಸರ್​ಲ್ಯಾಂಡ್ ಸಂದರ್ಶನದ ಬಳಿಕ ಜರ್ಮನಿಯ ಕೆಲವು ನಗರಗಳನ್ನು ನೋಡಿ ಕೊಂಡು, ಆಮ್ಸ್​ಟರ್​ಡ್ಯಾಮಿನ ಮೂಲಕ ಲಂಡನ್ನಿಗೆ ಹಿಂದಿರುಗುವುದೆಂದು ತೀರ್ಮಾನ ವಾಯಿತು.

ಪ್ರವಾಸ ಹೊರಡುವಂದು ಸ್ವಾಮೀಜಿ ಬಾಲಕನಂತೆ ಆನಂದೋತ್ಸಾಹಭರಿತರಾಗಿದ್ದರು. ತಮ್ಮ ಎಲ್ಲ ಚಿಂತೆಗಳನ್ನೂ ಅವರು ಸದ್ಯಕ್ಕೆ ಕೊಡವಿಕೊಂಡುಬಿಟ್ಟಿದ್ದರು. ಮಿತ್ರರೆಲ್ಲರೂ ಸುಖ ಪ್ರಯಾಣ ಕೋರಿ, ಶುಭ ಹಾರೈಕೆಗಳೊಂದಿಗೆ ಅವರನ್ನು ಬೀಳ್ಗೊಂಡರು. ಇನ್ನೆರಡೇ ತಿಂಗಳಲ್ಲಿ ಅವರು ತಮ್ಮಲ್ಲಿಗೆ ಹಿಂದಿರುಗಲಿದ್ದಾರೆಂಬ ಭರವಸೆಯಿದ್ದುದರಿಂದ ಅಲ್ಲಿ ದುಃಖದ ಸುಳಿವಿರಲಿಲ್ಲ.

ಜುಲೈ ೧೯ರಂದು ಭಾನುವಾರ ಬೆಳಿಗ್ಗೆ ಸ್ವಾಮೀಜಿ ಹಾಗೂ ಅವರ ಮೂವರು ಆಪ್ತಶಿಷ್ಯರು ಲಂಡನ್ನಿನಿಂದ ಹೊರಟು, ಡೋವರ್ ಎಂಬಲ್ಲಿ ಇಂಗ್ಲಿಷ್ ಕಡಲ್ಗಾಲುವೆಯನ್ನು ದಾಟಿದರು. ಸಾಮಾನ್ಯವಾಗಿ ಈ ಕಾಲುವೆಯು ಪ್ರಕ್ಷುಬ್ಧವಾಗಿರುತ್ತದಾದರೂ ಅಂದು ಸ್ವಲ್ಪ ಶಾಂತವಾಗಿದ್ದುದ ರಿಂದ ಪ್ರಯಾಣ ಸುಖಕರವಾಗಿತ್ತು. ಡೋವರ್ ಜಲಸಂಧಿಯನ್ನು ದಾಟಿದೊಡನೆಯೇ ಸಿಗು ವುದು ಫ್ರಾನ್ಸ್ ದೇಶದ ಕಲೇಸ್. ಇಲ್ಲಿಂದ ಸ್ವಿಟ್ಸರ್​ಲ್ಯಾಂಡಿನ ರಾಜಧಾನಿ ಜಿನೀವಾವರೆಗೆ ನೇರವಾದ ರೈಲುದಾರಿಯಿದೆ. ಆದರೆ ಒಂದೇ ಸಲಕ್ಕೆ ಅಷ್ಟು ದೂರದ ಪ್ರಯಾಣ ತುಂಬ ಆಯಾಸಕರ. ಆದ್ದರಿಂದ ಸ್ವಾಮೀಜಿಯ ತಂಡದವರು ಅಂದು ಸುಮಾರು ಅರ್ಧ ದಾರಿ ಪ್ರಯಾಣ ಮಾಡಿ ಫ್ರಾನ್ಸಿನ ರಾಜಧಾನಿಯಾದ ಪ್ಯಾರಿಸ್ಸಿನಲ್ಲಿ ಇಳಿದರು. ಅಲ್ಲಿ ಅಂದಿನ ರಾತ್ರಿಯನ್ನು ಸುಖವಾಗಿ ಕಳೆದು ಮರುದಿನ ಪ್ರಯಾಣವನ್ನು ಮುಂದುವರಿಸಿ ಎರಡನೆಯ ದಿನ ಮಧ್ಯಾಹ್ನ ಜಿನೀವಾ ತಲುಪಿದರು. ಇಲ್ಲಿ ಅವರೆಲ್ಲ ಒಂದು ಹೋಟೆಲಿನಲ್ಲಿ ಇಳಿದುಕೊಂಡರು. ಈ ಹೋಟೆಲು ರಮ್ಯವೂ ಪ್ರಶಾಂತವೂ ಆದ ಲೀಮನ್ ಸರೋವರಕ್ಕೆ ಅಭಿಮುಖವಾಗಿದ್ದು, ಪ್ರವಾಸಿಗರ ಸಂತಸವನ್ನು ಮತ್ತಷ್ಟು ಹೆಚ್ಚಿಸಿತು. ಇಲ್ಲಿನ ನವಚೇತನವನ್ನೀಯುವ ತಂಪುಗಾಳಿ, ಸರೋವರದ ನೀರಿನ ಗಾಢ ನೀಲಿ, ನಿರ್ಮಲ ಆಕಾಶ, ಸುತ್ತಲಿನ ಹೊಲಗದ್ದೆಗಳು, ಸುಂದರ ದೃಶ್ಯವನ್ನು ನಿರ್ಮಿಸಿದ ಸಾಲುಮನೆಗಳು ಇವೆಲ್ಲ ಸ್ವಾಮೀಜಿಯ ಮನಸ್ಸಿಗೆ ವಿಶೇಷ ಸಂತೋಷ ನೀಡಿದುವು.

ದೀರ್ಘ ಪ್ರಯಾಣದಿಂದಾಗಿ ಸ್ವಾಮೀಜಿ ಆಯಾಸಗೊಂಡಿದ್ದರೂ ಅವರು ಅಲ್ಲಿ ವಿಶ್ರಾಂತಿ ಪಡೆದದ್ದು ಸ್ವಲ್ಪ ಹೊತ್ತು ಮಾತ್ರವೇ. ಪುನಃ ಹೊರಗೆ ಹೋಗಿ ಪ್ರಕೃತಿಯ ಮಡಿಲನ್ನು ಸೇರಲು ಅವರು ಉತ್ಸುಕರಾಗಿದ್ದರು. ಆಗ ಜಿನೀವಾದಲ್ಲಿ ರಾಷ್ಟ್ರೀಯ ವಸ್ತುಪ್ರದರ್ಶನವೊಂದು ನಡೆಯು ತ್ತಿದ್ದು, ಸ್ವಿಟ್ಸರ್​ಲ್ಯಾಂಡಿನ ಕೈಗಾರಿಕೋತ್ಪನ್ನಗಳನ್ನು ಅದರಲ್ಲಿ ಪ್ರದರ್ಶಿಸಲಾಗಿತ್ತು. ದಿನದ ಹೆಚ್ಚಿನ ಭಾಗವನ್ನೆಲ್ಲ ಸ್ವಾಮೀಜಿ ತಮ್ಮ ಒಡನಾಡಿಗಳೊಂದಿಗೆ ವಸ್ತುಪ್ರದರ್ಶನದಲ್ಲೇ ಕಳೆದರು. ಸ್ಥಳೀಯ ಕಲೆ ಹಾಗೂ ಕರಕುಶವಸ್ತುಗಳಲ್ಲಿ–ಅದರಲ್ಲೂ ಸುಪ್ರಸಿದ್ಧವಾದ ಮರದ ಕೆತ್ತನೆಯ ಕೆಲಸದಲ್ಲಿ–ಅವರು ತುಂಬ ಆಸಕ್ತಿ ತೋರಿಸಿದರು. ಆದರೆ ಈ ಇಡೀ ವಸ್ತುಪ್ರದರ್ಶನದಲ್ಲಿ ಎಲ್ಲಕ್ಕಿಂತ ಹೆಚ್ಚಾಗಿ ಸ್ವಾಮೀಜಿಯನ್ನು ಆಕರ್ಷಿಸಿದ್ದು ಹಗ್ಗಗಳಿಂದ ನೆಲಕ್ಕೆ ಬಿಗಿಯಲ್ಪಟ್ಟಿದ್ದ ಭಾರೀ ಗಾತ್ರದ ಆಕಾಶಬುಟ್ಟಿ. ಅದನ್ನು ನೋಡಿದ ತಕ್ಷಣ ಅವರು, “ಓ, ನಾವೂ ಅದರಲ್ಲಿ ಕುಳಿತು ಮೇಲಕ್ಕೆ ಹೋಗಲೇಬೇಕು!” ಎಂದು ಉದ್ಗರಿಸಿದರು. ಆಕಾಶಬುಟ್ಟಿಯಲ್ಲಿ ಕುಳಿತು ಆಗಸದಲ್ಲಿ ತೇಲಾಡುವ ಬಯಕೆ ಅವರನ್ನಾವರಿಸಿಬಿಟ್ಟಿತು. ಆದರೆ ಆಕಾಶಬುಟ್ಟಿಯನ್ನು ಮೇಲಕ್ಕೆ ಬಿಡು ತ್ತಿದ್ದುದು ಸೂರ್ಯಾಸ್ತದ ಬಳಿಕವೇ. ಅಲ್ಲಿಯವರೆಗೂ ಸ್ವಾಮೀಜಿ ಬಾಲಕನಂತೆ ಚಡಪಡಿಸು ತ್ತಿದ್ದರು. ಕ್ಯಾಪ್ಟನ್ ಸೇವಿಯರ್ ಕೂಡ ಆಕಾಶಬುಟ್ಟಿಯಲ್ಲಿ ವಿಹರಿಸಲು ಕಾತರರಾಗಿದ್ದರು. ಆದರೆ ಶ್ರೀಮತಿ ಸೇವಿಯರ್ ಮಾತ್ರ ಗಟ್ಟಿ ನೆಲದ ಮೇಲೆ ಭದ್ರವಾಗಿರಲೇ ಇಷ್ಟಪಟ್ಟರು. ಏಕೆ ಸುಮ್ಮನೆ ಇಲ್ಲದ ತೊಂದರೆಗಳಿಗೆ ಸಿಕ್ಕಿಕೊಳ್ಳುವುದು ಎಂಬುದು ಆಕೆಯ ವಾದ. ಆದರೆ ಸ್ವಾಮೀಜಿ ಆಕೆಯ ಮಾತುಗಳಿಗೆ ಕಿವಿಗೊಡಲೇ ಇಲ್ಲ. ಕೊನೆಗೆ ಶ್ರೀಮತಿ ಸೇವಿಯರ್ ಅವರ ಒತ್ತಾಯಕ್ಕೆ ಮಣಿಯಲೇ ಬೇಕಾಯಿತು. ಎಲ್ಲರೂ ಆಕಾಶಬುಟ್ಟಿಯ ಒಳಹೊಕ್ಕರು. ಆಕಾಶಬುಟ್ಟಿ ನೆಲದಿಂದ ಮೇಲೆದ್ದಿತು! ಇನ್ನೂ ಮೇಲಕ್ಕೆ, ಮತ್ತೂ ಮೇಲಕ್ಕೆ ಏರುತ್ತಲೇ ಹೋಯಿತು. ಭೂಮಿಯ ಮೇಲೆ ಉಳಿದುಕೊಂಡ ಜನರೆಲ್ಲ ಬೊಂಬೆಗಳಂತೆ ಕಂಡರು. ಮನೆಗಳು-ದೋಣಿಗಳೆಲ್ಲ ಸುಂದರವಾದ ಆಟಿಕೆಗಳಾದುವು. ದೂರದ ಬೆಟ್ಟಗಳು ಗುಡ್ಡಗಳಾಗತೊಡಗಿದುವು. ಜೊತೆಗೆ, ಅಂದಿನ ವಾತಾ ವರಣ ಬಹಳ ಆಹ್ಲಾದಕರವಾಗಿತ್ತು. ಸೂರ್ಯಾಸ್ತಮಾನ ದಿಗಂತಕ್ಕೆ ಹೊಸ ಮೆರುಗನ್ನು ನೀಡಿತ್ತು. ಆಕಾಶಬುಟ್ಟಿ ಸಂಜೆಯ ತಂಗಾಳಿಯಲ್ಲಿ ತೇಲುತ್ತ ಸಾಗಿತು. ಈ ಆಕಾಶವಿಹಾರದಿಂದ ಸ್ವಾಮೀಜಿಗಾದ ಆನಂದ ಅಷ್ಟಿಷ್ಟಲ್ಲ. ಅಂತೆಯೇ ಸೇವಿಯರ್ ದಂಪತಿಗಳೂ ತುಂಬ ಸಂತೋಷ ಪಟ್ಟರು. ಎಂದಿಗೂ ಕೆಳಗಿಳಿಯದೇ ಹೀಗೆಯೇ ತೇಲುತ್ತಿರೋಣವೆಂದು ಎಲ್ಲರಿಗೂ ಅನ್ನಿಸಿ ಬಿಟ್ಟಿತ್ತು. ಕೊನೆಗೆ ಆಕಾಶಬುಟ್ಟಿ ಕೆಳಗಿಳಿಯತೊಡಗಿದಾಗ ಎಲ್ಲರೂ ಹಾರಾಡುತ್ತಿದ್ದ ತಮ್ಮ ಮನಸ್ಸನ್ನು ಬಲವಂತದಿಂದ ಕೆಳಗಿಳಿಸಬೇಕಾಯಿತು. ಆಕಾಶ ಬುಟ್ಟಿ ನೆಲವನ್ನು ಮುಟ್ಟಿತು. ಈ ಅನುಭವ ಸ್ವಾಮೀಜಿಯ ಮನಸ್ಸನ್ನು ಎಷ್ಟರಮಟ್ಟಿಗೆ ಸೆರೆಹಿಡಿದುಬಿಟ್ಟಿತ್ತೆಂದರೆ ಅವರು ಮತ್ತೊಮ್ಮೆ ಮೇಲೇರಿ ಹೋಗುವ ಉತ್ಸಾಹ ತೋರಿದರು. ಆದರೆ ಇನ್ನಿತರ ಕಾರ್ಯಕ್ರಮ ಗಳಿಂದಾಗಿ ಅದು ಸಾಧ್ಯವಾಗಲಿಲ್ಲ. ಎಲ್ಲರೂ ಹತ್ತಿರದ ಭೋಜನಗೃಹದಲ್ಲಿ ಊಟ ಮುಗಿಸಿ ಕೊಂಡು ತಾವು ತಂಗಿದ್ದ ವಿಶ್ರಾಂತಿಧಾಮಕ್ಕೆ ಹಿಂದಿರುಗಿದರು. ಹಿಂದಿರುಗುವಾಗ ಎಲ್ಲರ ಮನಸ್ಸಿನಲ್ಲೂ ಆಕಾಶವಿಹಾರದ ಸವಿನೆನಪೊಂದೇ ತುಂಬಿತ್ತು. ಆಕಾಶಬುಟ್ಟಿಯಿಂದ ಕೆಳಗಿಳಿದ ಕೂಡಲೇ ಅವರು ತಮ್ಮೆಲ್ಲರ ಛಾಯಾಚಿತ್ರಗಳನ್ನು ತೆಗೆಸಿಕೊಂಡಿದ್ದರು. ಅದರಲ್ಲಿ ಸ್ವಾಮೀಜಿ ಹಸನ್ಮುಖರಾಗಿ ನಿಂತಿದ್ದರು. ಆದರೆ ದುರದೃಷ್ಟವಶಾತ್ ಅತ್ಯಂತ ಅಪೂರ್ವವಾದ ಈ ಛಾಯಾ ಚಿತ್ರಗಳು ಇಂದು ನಮ್ಮ ಪಾಲಿಗೆ ಸಿಕ್ಕಿಲ್ಲ.

ಜಿನೀವಾದ ಸುತ್ತಮುತ್ತಲಿನ ಇತರ ಕೆಲವು ಸ್ಥಳಗಳನ್ನು ನೋಡಿಕೊಂಡು ಸ್ವಾಮೀಜಿ ಮತ್ತು ಅವರ ಜೊತೆಗಾರರು, ಫ್ರಾನ್ಸಿನ ಗಡಿಯಲ್ಲಿರುವ ಷಮೋನಿಕ್ಸ್ ಎಂಬಲ್ಲಿಗೆ ಹೊರಟರು. ಜಿನೀವಾದಿಂದ ಇಲ್ಲಿಗೆ ೫೬ ಮೈಲಿ ದೂರ. ಈ ಸ್ಥಳ ಸಮೀಪವಾದಂತೆ ದೂರದ ‘ಬ್ಲಾಂಕ್​’ ಪರ್ವತ ಶಿಖರದ ಭವ್ಯ ನೋಟ ಎದುರಾಯಿತು. ಇದು ಆ ಪ್ರದೇಶದಲ್ಲೇ ಅತ್ಯುನ್ನತವಾದ ಶಿಖರ; ಸಮುದ್ರಮಟ್ಟಕ್ಕಿಂತ ಹದಿನೇಳೂಮುಕ್ಕಾಲು ಸಾವಿರ ಅಡಿ ಎತ್ತರ. ಅದನ್ನು ಕಂಡು ಸ್ವಾಮೀಜಿ ಹರ್ಷೋದ್ವೇಗಗೊಂಡು, “ಇದು ನಿಜಕ್ಕೂ ಅದ್ಭುತ! ನಾವೀಗ ವಾಸ್ತವವಾಗಿಯೂ ಹಿಮದ ನಡುವೆಯೇ ಇದ್ದೇವೆ!” ಎಂದುದ್ಗರಿಸಿದರು. ಆ ಶಿಖರವನ್ನೇರಲು ಅವರು ತುಂಬ ಉತ್ಸುಕರಾಗಿದ್ದರು. ಬಹುಶಃ ದೂರದಿಂದ ಆ ಶಿಖರ ಕಣ್ಣಿಗೆ ತುಂಬ ನುಣುಪಾಗಿ, ಸುಗಮವಾಗಿ ಕಂಡಿರಬೇಕು. ಆದರೆ ಅಲ್ಲಿನ ಮಾರ್ಗದರ್ಶಕರು, ಆ ಪರ್ವತನ್ನೇರುವ ಸಾಹಸವನ್ನು ಪರಿಣತ ಪರ್ವತಾರೋಹಿಗಳು ಮಾತ್ರವೇ ಮಾಡಲು ಸಾಧ್ಯ ಎಂದು ಹೇಳಿ, ಆ ಕಾರ್ಯವನ್ನು ಕೈಗೊಳ್ಳ ದಂತೆ ಎಚ್ಚರಿಸಿದರು. ಇದನ್ನು ಕೇಳಿ ಸ್ವಾಮೀಜಿಗೆ ತುಂಬ ನಿರಾಶೆಯಾಯಿತು. ಆದರೆ ತಾವೇ ದೂರದರ್ಶಕದ ಮೂಲಕ ಅದನ್ನು ನೋಡಿದಾಗ, ಪರ್ವತಾರೋಹಣದ ದಾರಿ ಅತ್ಯಂತ ಕಡಿದಾ ಗಿದ್ದು, ಎದೆಯನ್ನು ತಲ್ಲಣಗೊಳಿಸುವಂತಿರುವುದನ್ನು ಕಂಡರು. ಇಂತಹ ಶಿಖರವನ್ನೇರುವುದು ದುಸ್ಸಾಧ್ಯವಾದುದೆಂಬುದನ್ನು ಮನಗಂಡು ಆ ಆಲೋಚನೆಯನ್ನು ಕೈಬಿಟ್ಟರು. ಆದರೆ ಎಷ್ಟೇ ಕಷ್ಟವಾದರೂ ಒಂದು ನೀರ್ಗಲ್ಲ ನದಿಯನ್ನು ಮಾತ್ರ ದಾಟಲೇಬೇಕೆಂದುಕೊಂಡರು ಸ್ವಾಮೀಜಿ. ಅದನ್ನು ದಾಟದಿದ್ದರೆ ತಾವು ಆಲ್ಪ್ಸ್ ಪರ್ವತಕ್ಕೆ ನೀಡಿದ ಭೇಟಿ ಅಪೂರ್ಣವಾದಂತೆ ಎಂದು ಅವರಿಗೆ ಅನ್ನಿಸಿತು. ಅದೃಷ್ಟವಶಾತ್ ಮರ್​-ಡಿ-ಗ್ಲೇಸ್ ಎಂಬ ಭಾರೀ ನೀರ್ಗಲ್ಲ ನದಿ ಅಲ್ಲಿಗೆ ಹತ್ತಿರದಲ್ಲೇ ಇತ್ತು. ಸ್ವಾಮೀಜಿ ಮತ್ತು ಅವರ ಪರಿವಾರದವರು ಹೇಸರಗತ್ತೆಗಳ ಮೇಲೆ ಪ್ರಯಾಣ ಮಾಡಿ ಒಂದು ಹಳ್ಳಿಗೆ ಬಂದು ತಲುಪಿದರು. ಈ ಹಳ್ಳಿಯಿಂದ ಮುಂದಕ್ಕೆ ನೀರ್ಗಲ್ಲ ನದಿಯ ಮೇಲೆ ಕಾಲ್ನಡಿಗೆಯ ಪ್ರಯಾಣ. ನೀರ್ಗಲ್ಲ ನದಿಯೆಂದರೆ ಅದೊಂದು ಅತ್ಯಂತ ವಿಶಾಲವಾದ, ನಿಧಾನವಾಗಿ ಚಲಿಸುತ್ತಿರುವ ಮಂಜಿನ ಹಾಸು ಎನ್ನಬಹುದು. ಇದು ನೆಲದಷ್ಟೇ ಗಟ್ಟಿಯಾಗಿರುತ್ತದೆ, ಆದರೆ ಮೇಲ್ಮೈ ಮಾತ್ರ ತುಂಬ ನುಣುಪು. ಇದರ ಮೇಲೆ ನಡೆಯುವು ದೊಂದು ರೋಮಾಂಚಕಾರೀ ಅನುಭವವಾದರೂ ಕಾಲು ಜಾರುವ ಇಲ್ಲವೆ ಮಂಜು ಕುಸಿಯುವ ಅಪಾಯವೂ ಇದ್ದೇ ಇರುತ್ತದೆ. ಅಂತೂ, ಸ್ವಾಮೀಜಿ ನಿರೀಕ್ಷಿಸಿದ್ದಷ್ಟೇನೂ ಸುಖಕರವಾಗಿರಲಿಲ್ಲ ಈ ಪ್ರಯಾಣ. ಕಾಲು ಜಾರದಂತೆ ನೋಡಿಕೊಳ್ಳುವುದು ಹಾಗೂ ತಮ್ಮ ಸಮತೋಲವನ್ನು ಕಾಯ್ದುಕೊಳ್ಳುವುದು ಅವರಿಗೆ ತುಂಬ ಕಷ್ಟವಾಯಿತು. ಆದರೂ ಅವರು ಮಧ್ಯೆ ಮಧ್ಯೆ ನಿಂತು ಅಲ್ಲಲ್ಲಿ ಕಾಣುತ್ತಿದ್ದ ಆಳವಾದ ಬಿರುಕುಗಳ ಮೂಲಕ ಇಣಕಿ ನೋಡುತ್ತಿದ್ದರು. ಮತ್ತು ಸುತ್ತಮುತ್ತ ಎಲ್ಲೆಲ್ಲೂ ಬಿಳಿಯ ಮಂಜಿನ ಮಧ್ಯದಲ್ಲಿ ಎದ್ದು ಕಾಣುತ್ತಿದ್ದ ದಟ್ಟ ಹಸಿರು ಪಾಚಿಯ ತೇಪೆಗಳ ಸೌಂದರ್ಯವನ್ನು ಆಸ್ವಾದಿಸುತ್ತಿದ್ದರು. ನೀರ್ಗಲ್ಲ ನದಿಯನ್ನು ದಾಟಿದ ಮೇಲೆ, ಸಮೀಪದ ಹಳ್ಳಿಯನ್ನು ತಲುಪಲು ಒಂದು ಅತ್ಯಂತ ಕಡಿದಾದ ಜಾಗದಲ್ಲಿ ಮೇಲೇರಿ ಹೋಗಬೇಕಾಗಿತ್ತು. ಈ ಜಾಗದಲ್ಲಿ ಸ್ವಾಮೀಜಿಗೆ ಬಹಳವೇ ಕಷ್ಟವಾಯಿತು. ಅವರಿಗೆ ಜೀವನ ದಲ್ಲೇ ಮೊದಲ ಬಾರಿಗೆ, ತಲೆ ಸುತ್ತಿ ಬವಳಿ ಬರುವಂತಾಯಿತು. ಒಂದೆರಡು ಸಲ ಕಾಲು ಜಾರಿದರಾದರೂ ಅಪಾಯವೇನೂ ಸಂಭವಿಸಲಿಲ್ಲ. ಅಂತೂ ಕ್ಷೇಮವಾಗಿ ಹಳ್ಳಿಯನ್ನು ತಲುಪಿ ದಾಗ ಎಲ್ಲರೂ ಸಮಾಧಾನದ ನಿಟ್ಟುಸಿರೆಳೆದರು. ಸ್ವಲ್ಪ ಬಿಸಿಬಿಸಿ ಕಾಫಿ ಕುಡಿದ ಮೇಲೆ ಸ್ವಾಮೀಜಿ ಚೇತರಿಸಿಕೊಂಡರು.

ಹಿಮಾಮೃತವಾದ ಆಲ್ಪ್ಸ್ ಪರ್ವತಗಳ ಸಾನ್ನಿಧ್ಯದಲ್ಲಿ ಸ್ವಾಮೀಜಿಗೆ ಹಿಮಾಲಯದ ನೆನಪು ಒತ್ತರಿಸಿ ಬರುತ್ತಿತ್ತು. ಅಲ್ಲಿನ ಪುಷ್ಯಾಶ್ರಮಗಳ, ಪರಿವ್ರಾಜಕ ಸಂನ್ಯಾಸಿಗಳ ಹಾಗೂ ಸಾಧಕರ ಚಿತ್ರ ಅವರ ಕಣ್ಣೆದುರಿಗೆ ನಿಲ್ಲುತ್ತಿತ್ತು. ಹಿಮಾಲಯದ ಶಿಖರಗಳ ನಡುವಿನಲ್ಲಿ ಮಠವೊಂದನ್ನು ಸ್ಥಾಪಿಸುವುದು ಅವರ ಮನಸ್ಸಿನಲ್ಲಿ ಎಂದಿನಿಂದಲೂ ಇದ್ದ ಒಂದು ಮಹದಾಸೆ. ಈ ದಿನಗಳಲ್ಲೇ ಅವರು ತಮ್ಮ ಈ ಕನಸನ್ನು ಸೇವಿಯರ್ ದಂಪತಿಗಳಿಗೆ ಉತ್ಸಾಹದಿಂದ ಬಣ್ಣಿಸಿದರು–“ಓಹ್​! ನನ್ನ ಜೀವನದ ಕೆಲಸಕಾರ್ಯಗಳಿಂದ ನಿವೃತ್ತನಾದ ಮೇಲೆ ಇನ್ನುಳಿದ ದಿನಗಳನ್ನು ಧ್ಯಾನದಲ್ಲಿ ಕಳೆಯಲಾಗುವಂತಹ ಒಂದು ಮಠವನ್ನು ಸ್ಥಾಪಿಸಬೇಕೆಂಬ ತೀವ್ರ ಬಯಕೆ ನನ್ನಲ್ಲಿದೆ. ಅದು ಖಂಡಿತವಾಗಿಯೂ ನನಸಾಗುತ್ತದೆ. ಆ ಮಠವು ನನ್ನ ಭಾರತೀಯ ಹಾಗೂ ಪಾಶ್ಚಾತ್ಯ ಶಿಷ್ಯರು ಒಟ್ಟಾಗಿದ್ದು ಧ್ಯಾನ ಹಾಗೂ ಕರ್ಮಗಳನ್ನು ನಡೆಸಿಕೊಂಡು ಹೋಗುವ ಕೇಂದ್ರವಾಗಿರುತ್ತದೆ. ಅವರೆಲ್ಲರಿಗೂ ತರಬೇತಿ ನೀಡಿ ನಾನವರನ್ನು ದಕ್ಷ ಕಾರ್ಮಿಕರನ್ನಾಗಿಸುತ್ತೇನೆ. ನನ್ನ ಭಾರತೀಯ ಶಿಷ್ಯರು ವೇದಾಂತ ಪ್ರಸಾರಕ್ಕಾಗಿ ಪಶ್ಚಿಮ ದೇಶಗಳಿಗೆ ಹೋಗುತ್ತಾರೆ; ಪಾಶ್ಚಾತ್ಯ ಶಿಷ್ಯರು, ಭಾರತದ ಒಳಿತಿಗಾಗಿ ತಮ್ಮ ಜೀವನವನ್ನು ಮುಡಿಪಾಗಿಡುತ್ತಾರೆ.” ಇದನ್ನು ಕೇಳುತ್ತಿದ್ದಂತೆಯೇ ಸೇವಿಯರ್ ದಂಪತಿಗಳ ಮನಸ್ಸಿನಲ್ಲಿಯೂ ಅದ್ಭುತ ಆಲೋಚನೆಯೊಂದು ದೃಶ್ಯದಂತೆ ತೇಲಿತು. “ಹೌದು! ನಮಗೆ ಅಂತಹ ಮಠವೊಂದು ಬೇಕು!” ಎಂದು ಕ್ಯಾಪ್ಟನ್ ಸೇವಿಯರ್ ಉದ್ಗರಿಸಿ ದರು. ಆಗ ಅದೊಂದು ಕೇವಲ ಉದ್ಗಾರದಂತೆ ತೋರಿತಾದರೂ, ಕೆಲವೇ ವರ್ಷಗಳಲ್ಲಿ ಅದು ನನಸಾಯಿತು. ಸ್ವಾಮೀಜಿಯವರ ಮಾತು ಸೇವಿಯರ್ ದಂಪತಿಗಳ ಮನದಲ್ಲಿ ಎಷ್ಟು ಆಳವಾಗಿ ಇಳಿಯಿತೆಂದರೆ, ಅದಕ್ಕಾಗಿ ಅವರು ತಮ್ಮಿಂದಾದುದನ್ನೆಲ್ಲ ಮಾಡಲು ಸಿದ್ಧರಾದರು. ನಿಜಕ್ಕೂ ಸ್ವಾಮೀಜಿಯ ಆ ಮಹತ್ವಾಕಾಂಕ್ಷೆ ಪೂರೈಸುವಂತಾದದ್ದು ಸೇವಿಯರ್ ದಂಪತಿಗಳ ಪರಿಶ್ರಮ ಮತ್ತು ಸಹಕಾರದಿಂದಲೇ ಎನ್ನಬಹುದು.

ಷಮೋನಿಕ್ಸ್​ನಿಂದ ಸ್ವಾಮೀಜಿ ಮತ್ತು ಅವರ ಅನುಯಾಯಿಗಳು ಬೆಟ್ಟಗುಡ್ಡಗಳ ದಾರಿಯಾಗಿ ನಾಲ್ಕು ದಿನ ಕಾಲ್ನಡಿಗೆಯಲ್ಲಿ ಪ್ರಯಾಣ ಮಾಡಿ ‘ಲಿಟ್ಲ್ ಸೈಂಟ್ ಬರ್ನಾರ್ಡ್​’ ಎಂಬ ಹಳ್ಳಿ ಯನ್ನು ತಲುಪಿದರು. ಪರ್ವತ ಶಿಖರಗಳಲ್ಲಿ ದಾರಿ ತಪ್ಪಿ ಕಳೆದುಹೋದವರ ನೆರವಿಗಾಗಿಯೇ ಇಲ್ಲಿ ಬೆಳೆಸಲಾಗುವ ಸೈಂಟ್ ಬರ್ನಾರ್ಡ್ ಎಂಬ ತಳಿಯ ನಾಯಿಯನ್ನು ನೋಡಿ ಸ್ವಾಮೀಜಿ ಬಹುವಾಗಿ ಮೆಚ್ಚಿಕೊಂಡರು. ಇಲ್ಲಿನ ವಿಶ್ರಾಂತಿಗೃಹದಲ್ಲಿ ಕೆಲಕಾಲ ವಿರಮಿಸಿದನಂತರ ಎಲ್ಲರೂ ಸೆರ್​ಮ್ಯಾಟ್ ಎಂಬ ಅತ್ಯಂತ ರಮ್ಯವಾದ ಒಂದು ಪ್ರೇಕ್ಷಣೀಯ ಸ್ಥಳಕ್ಕೆ ಹೋದರು. ಇದು ಅನೇಕ ನೀರ್ಗಲ್ಲ ನದಿಗಳಿಂದ ಕೂಡಿದ್ದು, ಉನ್ನತ ಪರ್ವತಶಿಖರಗಳಿಂದ ಆವೃತವಾಗಿದೆ. ಇದರ ಸಮೀಪದ ಗಾರ್ನೆರ್​ಗ್ರಾಟ್ ಎಂಬ ಶಿಖರವನ್ನು ಹತ್ತಿ ಸುತ್ತಲಿನ ವಿಹಂಗಮ ನೋಟವನ್ನು ಆಸ್ವಾದಿಸಬೇಕೆಂದು ಎಲ್ಲರಿಗೂ ಇಚ್ಛೆಯಾಯಿತು. ಆದರೆ ಮೇಲೇರಿದಂತೆ ವಾಯು ವಿರಳ ವಾಗಿರುತ್ತಿದ್ದುದರಿಂದ ಕ್ಯಾಪ್ಟನ್ ಸೇವಿಯರ್ ಮಾತ್ರ ಮೇಲೇರಲು ಸಮರ್ಥರಾದರು.

ಮಿಸ್ ಮುಲ್ಲರಳ ಕೋರಿಕೆಯಂತೆ ಸ್ವಾಮೀಜಿ ಮತ್ತವರ ಸಂಗಡಿಗರು ಅಲ್ಲಿಂದ ರ್ಹೋನ್ ನದಿಯ ಕಣಿವೆಯ ಒಂದು ಪುಟ್ಟ ಗ್ರಾಮವಾದ ಸಾಸ್​-ಫೀ ಎಂಬಲ್ಲಿಗೆ ಪಯಣಿಸಿದರು. ಇಲ್ಲಿ ಅವರು ಸುಮಾರು ಎರಡು ವಾರಗಳಷ್ಟು ದೀರ್ಘಕಾಲ ತಂಗಿದ್ದರು. ಆಲ್ಪ್ಸ್ ಪರ್ವತದ ತಪ್ಪಲಿನಲ್ಲಿರುವ ಈ ಹಳ್ಳಿಯಲ್ಲಿ ಸ್ವಾಮೀಜಿ ಅತ್ಯಂತ ಆನಂದದ ಸ್ಥಿತಿಗೇರಿದರು. ಎತ್ತ ನೋಡಿದರತ್ತ ಹಿಮಾಚ್ಛಾದಿತ ಬೆಟ್ಟಗಳು; ಎಲ್ಲೆಡೆಯೂ ಮೌನ, ಗ್ರಾಮಜೀವನದ ಪ್ರಶಾಂತ ವಾತಾವರಣ. ಇಲ್ಲಿ ಸ್ವಾಮೀಜಿ, ತಮ್ಮ ಜೀವನದ ಕೆಲವು ಅತ್ಯುಜ್ವಲ ಆಧ್ಯಾತ್ಮಿಕ ಕ್ಷಣಗಳನ್ನು ಕಳೆದರು. ಆಗ ಅವರಿಗೆ ಇಹ ಜಗತ್ತಿನ ಪರಿವೆಯೇ ಇರಲಿಲ್ಲ. ತಮ್ಮ ಶರೀರದ ಇರವನ್ನು ಮರೆತು ಅವರು ಬಹು ದೂರಕ್ಕೆ ಹೋದಂತಿತ್ತು. ತಮ್ಮ ಅನುಯಾಯಿಗಳ ಪಾಲಿಗೆ ತಾವು ಗುರುವೆಂಬ ಭಾವವೂ ಅವರಲ್ಲೀಗ ಕಂಡುಬರುತ್ತಿರಲಿಲ್ಲ. ಅವರು ಕೇವಲ ಒಬ್ಬ ಪರಿವ್ರಾಜಕ- ಧ್ಯಾನನಿರತ ಸಂನ್ಯಾಸಿಯಾಗಿದ್ದರು. ಕೆಲವೊಮ್ಮೆ ಅವರು ಪರ್ವತದ ದಾರಿಯಲ್ಲಿ ಏಕಾಂಗಿಯಾಗಿ ನಡೆದಾಡುತ್ತಿದ್ದರು. ಇಲ್ಲವೆ, ಯಾವುದಾದರೂ ಎತ್ತರದ ಬಂಡೆಯನ್ನೇರಿ ದಿಗಂತದಾಚೆಗೆ ನೋಟವನ್ನು ನೆಟ್ಟು, ಎಷ್ಟೋ ಹೊತ್ತಿನವರೆಗೆ ನಿಂತುಬಿಡುತ್ತಿದ್ದರು. ಸಂನ್ಯಾಸ ಜೀವನದ ಸ್ವತಂತ್ರ ಮನೋವೃತ್ತಿ ಅವರ ಮುಖದಲ್ಲೇ ಎದ್ದು ಕಾಣುತ್ತಿತ್ತು. ಸ್ವಾಮೀಜಿಯೊಂದಿಗೆ ತಾವು ಧ್ಯಾನ-ಶಾಂತಿಗಳ ಹೊಸ ಲೋಕವೊಂದನ್ನು ಪ್ರವೇಶಿಸಿದಂತೆ ಆ ಮೂವರು ಸಂಗಡಿಗರಿಗೂ ಅನುಭವವಾಗುತ್ತಿತ್ತು.

ಸ್ವಾಮೀಜಿಯ ಈ ಅಪೂರ್ವ ಭಾವಾವಸ್ಥೆಯು ಅವರು ಈ ದಿನಗಳಲ್ಲಿ ಬರೆದ ಪತ್ರಗಳಲ್ಲಿ ಸ್ಪಷ್ಟವಾಗಿ ವ್ಯಕ್ತವಾಗಿದೆ. ಶ್ರೀಮತಿ ಸಾರಾಳಿಗೆ ಬರೆದ ಪತ್ರದಲ್ಲಿ ಹೇಳುತ್ತಾರೆ, “ನಾನೀಗ ಸ್ವಿಟ್ಸರ್​ಲ್ಯಾಂಡಿನ ಅತ್ಯಂತ ನಿರ್ಜನವಾದ ಮೂಲೆಗಳಲ್ಲಿ ಸಂಚರಿಸುತ್ತಿದ್ದೇನೆ. ಕನಿಷ್ಠಪಕ್ಷ ಮುಂದಿನ ಎರಡು ತಿಂಗಳಾದರೂ ಜಗತ್ತನ್ನು ಸಂಪೂರ್ಣವಾಗಿ ಮರೆತು ಧ್ಯಾನಮಗ್ನನಾಗಿರ ಬೇಕೆಂದುಕೊಂಡಿದ್ದೇನೆ. ಅದೇ ನನ್ನ ಪಾಲಿಗೆ ವಿಶ್ರಾಂತಿ.” ಸ್ವಾಮಿ ಕೃಪಾನಂದರಿಗೆ ಅವರೊಂದು ಪತ್ರದಲ್ಲಿ ಬರೆಯುತ್ತಾರೆ, “ನಿನ್ನೆ ನಾನು ಮಾಂಟೆರೋಸಾದ ಹೆಪ್ಪುಗಟ್ಟಿದ ಸರೋವರಕ್ಕೆ ಹೋಗಿ ದ್ದಾಗ, ಅಲ್ಲಿ ನಿರಂತರ ಹಿಮದ ನಡುವೆಯೇ ಬೆಳೆಯುತ್ತಿದ್ದ ಕೆಲವು ‘ಗಟ್ಟಿ’ ಪುಷ್ಪಗಳನ್ನು ಸಂಗ್ರಹಿಸಿಕೊಂಡು ಬಂದೆ. ಅವುಗಳಲ್ಲೊಂದನ್ನು ಈ ಪತ್ರದೊಂದಿಗೆ ನಿನಗೆ ಕಳಿಸಿಕೊಡುತ್ತಿ ದ್ದೇನೆ. ಜೀವನದ ಮಂಜು-ಹಿಮಗಳೆಲ್ಲದರ ನಡುವೆ, ನೀನು ಇಂತಹ ಆಧ್ಯಾತ್ಮಿಕ ಗಟ್ಟಿತನಕ್ಕೆ ಏರುವಂತಾಗಲೆಂದು ನನ್ನ ಹಾರೈಕೆ.” ಬಳಿಕ ಆಗಸ್ಟ್ ೮ರಂದು ಗುಡ್​ವಿನ್ನನಿಗೆ ಬರೆಯುತ್ತಾರೆ, “ಈ ಇಡೀ ಜಗತ್ತೇ ಮಕ್ಕಳಾಟ–ಧರ್ಮಪ್ರಚಾರ, ಬೋಧನೆಯ ಕೆಲಸಗಳೂ ಕೂಡ. ‘ಯಾವನು ಬಯಸುವುದಿಲ್ಲವೋ ಯಾವನು ದ್ವೇಷಿಸುವುದಿಲ್ಲವೋ ಅವನನ್ನು ಸಂನ್ಯಾಸಿಯೆಂದರಿ.’ ಮತ್ತೆ ಮತ್ತೆ ಮರುಕಳಿಸುವ ಜರಾಮರಣರೋಗಗಳಿಂದ ಕೂಡಿದ ಈ ಕ್ಷುದ್ರಪ್ರಪಂಚದ ಕೊಚ್ಚೆಗುಂಡಿ ಯಲ್ಲಿ ಬಯಸುವಂಥದಾದರೂ ಏನಿದೆ? ಜಗತ್ತಿನ ಜನ ಯಾವುದನ್ನು ‘ಕರ್ಮ’ ಎಂದು ಕರೆಯು ತ್ತಾರೆಯೋ ಅದರಲ್ಲಿ ನನ್ನ ಪಾಲಿನ ಅನುಭವವನ್ನು ಪಡೆದದ್ದಾಯಿತು ಎಂದು ನನಗನ್ನಿಸುತ್ತದೆ. ಈಗ ನನ್ನ ಕೆಲಸವೆಲ್ಲವೂ ಮುಗಿದಿದೆ. ನಾನೀಗ ಹೊರಟುಬಿಡಲು ಚಡಪಡಿಸುತ್ತಿದ್ದೇನೆ.” ಮತ್ತೆ ಆಗಸ್ಟ್ ೨೩ರಂದು ಶ್ರೀಮತಿ ಬುಲ್​ಗೆ ಬರೆಯುತ್ತಾರೆ, “ನಾನೀಗ ಸಾಕಷ್ಟು ಕೆಲಸ ಮಾಡಿದ್ದಾಗಿದೆ ಯೆಂದು ನನಗನ್ನಿಸುತ್ತದೆ. ನಾನಿನ್ನು ನಿವೃತ್ತನಾಗುವವನಿದ್ದೇನೆ. ನಾನೀಗ ಭಾರತಕ್ಕೆ ಇನ್ನೊಬ್ಬ ವ್ಯಕ್ತಿಗಾಗಿ (ಅಭೇದಾನಂದರಿಗಾಗಿ) ಹೇಳಿಕಳಿಸಿದ್ದೇನೆ. ನಾನು ಕೆಲಸವನ್ನು ಪ್ರಾರಂಭಮಾಡಿದ್ದಾ ಗಿದೆ; ಈಗ ಇತರರು ಅದನ್ನು ಪೂರ್ಣಗೊಳಿಸಲಿ. ನಾನೀಗ ಈ ನರಕಕ್ಕೆ–ಈ ಜಗತ್ತಿಗೆ– ಎಂದೆಂದಿಗೂ ಮರಳದಿರುವಂತೆ ಹೊರಟುಹೋಗಲು ಸಿದ್ಧನಾಗತೊಡಗಿದ್ದೇನೆ. ಈ ಜಗತ್ತಿನ ಧರ್ಮಗಳು ಮತ್ತು ವೈವಿಧ್ಯಗಳು ಕೂಡ ನನಗೆ ಸಪ್ಪೆಯಾಗತೊಡಗಿವೆ. ನಾನಿನ್ನೆಂದಿಗೂ ಹಿಂದಿರುಗಿ ಬರದಿರುವಂತೆ ಜಗನ್ಮಾತೆ ನನ್ನನ್ನು ತನ್ನ ಬಳಿಗೆ ಸೆಳೆದುಕೊಳ್ಳಲಿ. ‘ಒಳ್ಳೆಯದನ್ನು ಮಾಡುವುದು’ ಎಂಬೀ ಕೆಲಸಗಳೆಲ್ಲ ಮನಸ್ಸನ್ನು ಶುದ್ಧಿಗೊಳಿಸಲು ಮಾಡಬೇಕಾದ ಕಸರತ್ತುಗಳು ಅಷ್ಟೆ. ನಾನದನ್ನೆಲ್ಲ ಬೇಕಾದಷ್ಟು ಮಾಡಿದ್ದಾಗಿದೆ. ಈ ಪ್ರಪಂಚ ಎಂದೆಂದಿಗೂ ಪ್ರಪಂಚ ವಾಗಿಯೇ ಉಳಿದಿರುತ್ತದೆ. ನಾವು ಏನಾಗಿರುತ್ತೇವೆಯೋ ಅದನ್ನೇ ನಾವು ಪ್ರಪಂಚದಲ್ಲಿಯೂ ಕಾಣುತ್ತೇವೆ. ಕೆಲಸ ಮಾಡುವವರು ಯಾರು? ಯಾರ ಕೆಲಸ? ಈ ಜಗತ್ತೆನ್ನುವುದೇ ಇಲ್ಲ. ಎಲ್ಲವೂ ಭಗವಂತನೇ. ಭ್ರಮೆಗೊಳಗಾಗಿದ್ದಾಗ ನಾವದನ್ನು ಜಗತ್ತೆಂದು ಕರೆಯುತ್ತೇವೆ. ನಾನೂ ಇಲ್ಲ, ನೀನೂ ಇಲ್ಲ, ಯಾರೂ ಇಲ್ಲ. ಎಲ್ಲವೂ ಭಗವಂತನೇ, ಎಲ್ಲವೂ ಒಂದೇ.”

ಸ್ವಾಮೀಜಿ ಏನೇ ಭಾವಿಸಿದರೂ, ಏನೇ ಹೇಳಿದರೂ, ಅವರ ‘ಕರ್ಮ’ ಇನ್ನೂ ಸವೆದಿರಲಿಲ್ಲ! ಮೇಲೆಮೇಲೇರಿ, ಗುರುತ್ವಾಕರ್ಷಣೆಯಿಂದ ತಪ್ಪಿಸಿಕೊಳ್ಳಲೆನ್ನಿಸುತ್ತಿದ್ದ ಅವರ ಮನಸ್ಸನ್ನು ಕೆಳೆಕ್ಕೆಳೆ ಯುತ್ತಿದ್ದ ಸೂತ್ರಗಳಿನ್ನೂ ಸಂಪೂರ್ಣವಾಗಿ ಕಡಿದಿರಲಿಲ್ಲ. ಭಾರತದಿಂದಲೂ ಅಮೆರಿಕ ದಿಂದಲೂ ಅವರ ಹೆಸರಿಗೆ ಬರುತ್ತಿದ್ದ ಪತ್ರಗಳನ್ನೆಲ್ಲ ಲಂಡನ್ನಿನಿಂದ ಅವರಿರುವಲ್ಲಿಗೆ ಕಳಿಸಿಕೊಡ ಲಾಗುತ್ತಿತ್ತು. ಇವುಗಳಲ್ಲಿ ಅವರು ಗಮನಹರಿಸಬೇಕಾದ ಅನೇಕ ಪ್ರಮುಖ ವಿಷಯಗಳಿರುತ್ತಿ ದ್ದುವು. ಹೀಗಾಗಿ ಅವರು ನಿರಾತಂಕವಾಗಿ ಸಮಾಧಿಸ್ಥಿತಿಗೇರಲು ಸಾಧ್ಯವಿರಲಿಲ್ಲ. ಇವುಗಳಲ್ಲಿ ಒಂದೆಂದರೆ ‘ಬ್ರಹ್ಮವಾದಿನ್​’ ಪತ್ರಿಕೆಯ ಆರ್ಥಿಕ ಸಂಕಷ್ಟಗಳ ಬಗ್ಗೆ ಅಳಸಿಂಗ ಪೆರಮಾಳರು ಬರೆದ ಪತ್ರ. ಇದನ್ನು ನೋಡಿದ ತಕ್ಷಣ ಸ್ವಾಮೀಜಿ ಅವರಿಗೊಂದು ಪತ್ರವನ್ನು ಬರೆದು, ಪರಿಸ್ಥಿತಿ ಯನ್ನು ಸರಿಪಡಿಸಿಕೊಳ್ಳಲು ಏನೇನು ಮಾಡಬೇಕೆಂಬುದನ್ನು ವಿವರಿಸಿದರು. “ಬ್ರಹ್ಮವಾದಿನ್ ಒಂದು ರತ್ನ; ಅದೆಂದೂ ಕೊನೆಗಾಣಬಾರದು” ಎಂದು ಬರೆದರು. ಆ ಸಮಯದಲ್ಲಿ ಪೆರು ಮಾಳರು ಜೀವನಯಾಪನೆಗಾಗಿ ಉಪಾಧ್ಯಾಯ ವೃತ್ತಿಯನ್ನೇ ಅವಲಂಬಿಸಿದ್ದರು. ಅವರು ತಮ್ಮೆಲ್ಲ ಸಮಯವನ್ನೂ ಶಕ್ತಿಯನ್ನೂ ಪತ್ರಿಕೆಗಾಗಿ ವಿನಿಯೋಗಿಸಲು ಸಾಧ್ಯವಾಗಿಸುವುದಕ್ಕಾಗಿ ಉಪಾಧ್ಯಾಯ ವೃತ್ತಿಯನ್ನು ಬಿಟ್ಟುಬಿಡುವಂತೆ ಸ್ವಾಮೀಜಿ ಆದೇಶ ನೀಡಿದರು. ಬದಲಾಗಿ ತಾವು ಪ್ರತಿತಿಂಗಳೂ ಒಂದು ನೂರು ರೂಪಾಯಿಗಳನ್ನು ಕಳಿಸಿಕೊಡುವ ಭರವಸೆ ನೀಡಿದರು. “ಬಲ ವಾಗಿ ಹಿಡಿದುಕೊ! ಕೆಲಸ ಮಾಡುತ್ತ ಹೋಗು! ಧೀರನಾಗು! ಏನೇ ಬರಲಿ, ಧೈರ್ಯವಾಗಿ ಎದುರಿಸು” ಎಂದು ಪ್ರೋತ್ಸಾಹಿಸಿದರು. ಅಲ್ಲದೆ ಗುರುಸ್ಥಾನದಿಂದ ಒಂದೆರಡು ವೈಯಕ್ತಿಕ ಸೂಚನೆಗಳನ್ನೂ ನೀಡಿದರು–“ಬ್ರಹ್ಮಚರ್ಯದ ಕಡೆಗೆ ವಿಶೇಷ ಎಚ್ಚರಿಕೆಯಿರಲಿ. ಈಗಾಗಲೇ ನಿನಗೆ ಸಾಕಷ್ಟು ಮಕ್ಕಳಿದ್ದಾರೆ; ಇನ್ನು ಸಾಕು!”

ಸ್ವಾಮೀಜಿಯ ಮನಸ್ಸನ್ನು ವಿಚಲಿತಗೊಳಿಸುವಂತಹ ಮತ್ತೊಂದು ಪತ್ರ ಕಲ್ಕತ್ತದಿಂದ ಬಂದಿತ್ತು. ಶ್ರೀರಾಮಕೃಷ್ಣರ ಜನ್ಮದಿನೋತ್ಸವದಂದು ಕೆಲವು ವೇಶ್ಯೆಯರು ಸಮಾರಂಭದಲ್ಲಿ ಪಾಲ್ಗೊಂಡರೆಂದೂ ಅದರಿಂದ ಅಸಮಾಧಾನಗೊಂಡ ಅನೇಕ ಸಭ್ಯ ವ್ಯಕ್ತಿಗಳು ಮಠಕ್ಕೆ ಬರುವ ಬಗ್ಗೆ ಹಿಂದೆಮುಂದೆ ನೋಡುತ್ತಿದ್ದಾರೆಂದೂ ಅದರಲ್ಲಿ ಬರೆಯಲಾಗಿತ್ತು. ಈಗ ಆ ಬಗ್ಗೆ ಏನು ಮಾಡುವುದೆಂದು ಸ್ವಾಮೀಜಿಯನ್ನು ಪ್ರಶ್ನಿಸಲಾಗಿತ್ತು. ಅದಕ್ಕುತ್ತರವಾಗಿ ಆ ದಿನವೇ ಅವರು ಸ್ವಾಮಿ ರಾಮಕೃಷ್ಣಾನಂದರಿಗೊಂದು ಪತ್ರ ಬರೆದರು. ವಿವೇಕಾನಂದರ ಹೃದಯವೈಶಾಲ್ಯವನ್ನು, ಬುದ್ಧಿಸೂಕ್ಷ್ಮತೆಯನ್ನು ಮತ್ತು ಅವರ ಮಾಹಾತ್ಮ್ಯವನ್ನು ಸ್ಪಷ್ಟವಾಗಿ ತೋರುವ ಆ ಪತ್ರ ಹೀಗಿತ್ತು–

“ವಾರಾಂಗನೆಯರನ್ನು ದಕ್ಷಿಣೇಶ್ವರದಂತಹ ತೀರ್ಥಕ್ಷೇತ್ರಕ್ಕೆ ಬಿಡದಿದ್ದರೆ, ಅವರು ಇನ್ನೆಲ್ಲಿಗೆ ತಾನೇ ಹೋಗಬೇಕು?... ನರಕದ ಬಾಗಿಲುಗಳಾದ ಜಾತಿ-ಕುಲ-ಐಶ್ವರ್ಯ-ಲಿಂಗದ ಭೇದ ಭಾವಗಳೆಲ್ಲ ಪ್ರಪಂಚಕ್ಕೆ ಮೀಸಲಾಗಿರಲಿ. ಪುಣ್ಯಕ್ಷೇತ್ರಗಳಲ್ಲೂ ಅಂತಹ ಭೇದಭಾವಗಳೆಲ್ಲ ಉಳಿದುಕೊಂಡರೆ, ಅವುಗಳಿಗೂ ನರಕಕ್ಕೂ ಏನು ವ್ಯತ್ಯಾಸ?... ನಮ್ಮ ಸಂಸ್ಥೆಯು ಭಗವಂತ ನಿಗೆ ಸೇರಿದ ಒಂದು ಮಹಾನಗರದಂತೆ–ಅದರಲ್ಲಿ ಪಾಪಿಗಳಿಗೆ, ಪುಣ್ಯವಂತರಿಗೆ, ಸಂತರಿಗೆ, ದುಷ್ಟರಿಗೆ, ಹೆಂಗಸರಿಗೆ-ಗಂಡಸರಿಗೆ-ಮಕ್ಕಳಿಗೆ–ಎಲ್ಲರಿಗೂ ಸಮಾನ ಹಕ್ಕುಗಳಿವೆ. ವರ್ಷದಲ್ಲಿ ಒಂದು ದಿನದ ಮಟ್ಟಿಗಾದರೂ ಜನರು ಭೇದಭಾವಗಳನ್ನೂ ತಾವು ಪಾಪಿಗಳೆಂಬ ನಂಬಿಕೆಯನ್ನೂ ತೊರೆದು, ಭಗವಂತನ ನಾಮವನ್ನು ಹಾಡಿದರೆ ಅದೇ ಒಂದು ಅತಿ ದೊಡ್ಡ ಶ್ರೇಯಸ್ಸು.” ಬಳಿಕ ಮತ್ತಷ್ಟು ನಿಷ್ಠುರವಾಗಿ ಬರೆದರು–“ಭಗವಂತನ ಸಾನ್ನಿಧ್ಯದಲ್ಲಿಯೂ ಯಾರು ‘ಈಕೆ ವೇಶ್ಯಾ ಸ್ತ್ರೀ, ಆತ ಅಂತ್ಯಜ, ಇವನು ಬಡವ, ಅವನು ಸಾಧಾರಣ ಮನುಷ್ಯ’ ಎಂದು ಆಲೋಚಿಸುತ್ತಾ ರೆಯೋ, ಅಂತಹ ಯಾರನ್ನು ನೀನು ‘ಸಭ್ಯಸ್ಥ’ರು ಎನ್ನುತ್ತೀಯೋ, ಅಂಥವರ ಸಂಖ್ಯೆ ಕಡಿಮೆ ಯಾದಷ್ಟೂ ಒಳ್ಳೆಯದು. ಭಕ್ತಾದಿಗಳ ಕಸುಬು ಯಾವುದು, ಜಾತಿ ಯಾವುದು ಎಂದೇ ನೋಡು ತ್ತಿರುವವರು, ನಮ್ಮ ಗುರುಮಹಾರಾಜರನ್ನು ಮೆಚ್ಚಬಲ್ಲರೆ? ಇಲ್ಲ; ನೂರಾರು ವೇಶ್ಯಾ ಸ್ತ್ರೀಯರು ಬಂದು ಆತನ ಪಾದಗಳಿಗೆ ಶಿರಬಾಗಲಿ ಎಂದು ನಾನು ಭಗವಂತನನ್ನು ಪ್ರಾರ್ಥಿಸಿ ಕೊಳ್ಳುತ್ತೇನೆ! ಸಭ್ಯಸ್ಥನೆನ್ನಿಸಿಕೊಂಡ ಒಬ್ಬನೇ ಒಬ್ಬನೂ ಬಾರದಿದ್ದರೂ ಕೂಡ ನನಗೆ ಚಿಂತೆ ಯಿಲ್ಲ. ವೇಶ್ಯೆಯರು ಬರಲಿ, ಕುಡುಕರು ಬರಲಿ, ಕಳ್ಳರು ಬರಲಿ, ಎಲ್ಲರೂ ಬರಲಿ–ಆತನ ಬಾಗಿಲು ಎಲ್ಲರಿಗೂ ತೆರೆದಿದೆ. ‘ಶ್ರೀಮಂತನು ಭಗವಂತನ ರಾಜ್ಯವನ್ನು ಪ್ರವೇಶಿಸುವುದ ಕ್ಕಿಂತಲೂ ಸೂಜಿಯ ಕಣ್ಣೊಳಗೆ ಒಂಟೆ ತೂರುವುದು ಸುಲಭ’ (ಏಸುಕ್ರಿಸ್ತ)... ” ಆದರೆ ಸ್ವಾಮೀಜಿ ಆ ಪತ್ರದ ಕಡೆಯಲ್ಲಿ ಒಂದು ಎಚ್ಚರಿಕೆಯ ಮಾತನ್ನೂ ಸೇರಿಸಿದರು–“ಆದರೆ ಸ್ವಲ್ಪ ಜಾಗರೂಕತೆಯನ್ನೂ ವಹಿಸಬೇಕಾಗುತ್ತದೆ... ವಯಸ್ಕರಾದ ಕೆಲವರು ಸ್ವಯಂಸೇವಕರು ಉತ್ಸ ವದ ಸ್ಥಳದಲ್ಲಿ ಓಡಾಡಿಕೊಂಡಿದ್ದು, ಯಾರಾದರೂ ಅಸಭ್ಯವಾಗಿ ಮಾತನಾಡಿದರೆ ಅಥವಾ ವರ್ತಿಸಿದರೆ, ಅವರನ್ನು ತಕ್ಷಣ ಆಚೆಗೆ ಕಳಿಸಬೇಕು. ಆದರೆ, ಮರ್ಯಾದಸ್ಥರಾಗಿರಲಿ ಆಗಿಲ್ಲದಿರಲಿ –ಅವರು ಸರಿಯಾದ ರೀತಿಯಲ್ಲಿ ವರ್ತಿಸುತ್ತಿರುವವರೆಗೆ ಅವರು ಭಕ್ತರೇ ಸರಿ; ಮತ್ತು ಅವರನ್ನು ಎಲ್ಲರೂ ಗೌರವದಿಂದ ಕಾಣಬೇಕು.”

ಆಲ್ಪ್ಸ್ ಪರ್ವತದ ಪರಮ ಪ್ರಶಾಂತ ಪರಿಸರದಲ್ಲಿ ಸ್ವಾಮೀಜಿ ಅತ್ಯುನ್ನತ ಭಾವಗಳಿಗೇರುತ್ತಿ ದ್ದಾಗ, ಇಂತಹ ಪ್ರಸಂಗಗಳಿಂದಾಗಿ ಅವರ ಮನಸ್ಸು ವ್ಯಾವಹಾರಿಕ ಸ್ತರಕ್ಕಿಳಿಯಬೇಕಾಗಿ ಬರುತ್ತಿ ದ್ದುದು ಕೇವಲ ಕಾಕತಾಳೀಯವಲ್ಲ. ನಾವು ಹಿಂದೆಯೂ ನೋಡಿದಂತೆ, ಇಂತಹ ಅವಕಾಶ ಸಿಕ್ಕಾಗ ಲೆಲ್ಲ ಅವರು ‘ಕರ್ತವ್ಯ ವಿಮುಖ’ರಾಗುತ್ತಿದ್ದರು. ಆದರೆ ಜಗನ್ಮಾತೆಯ ಕಾರ್ಯವನ್ನು ಮಾಡಿ ಮುಗಿಸುವವರೆಗೆ ಅವರಿಗೆ ವಿಶ್ರಾಂತಿಯೆಲ್ಲಿ? ಅವರ ಆಧ್ಯಾತ್ಮಿಕ ಅನುಭವಗಳ ಕೀಲಿಕೈ ಶ್ರೀರಾಮಕೃಷ್ಣರ ಬಳಿ ಭದ್ರವಾಗಿದೆಯಲ್ಲವೆ? ಆದ್ದರಿಂದ, ಅವರಿಗೆ ತಮ್ಮ ಕರ್ತವ್ಯವನ್ನು ನೆನಪಿಸಿಕೊಡಲು ಇಂತಹ ತೊಂದರೆಗಳು ಬೇಕಾಗಿದ್ದುವು!

ಆದರೆ ಒಟ್ಟಿನಲ್ಲಿ ಸಾಸ್ ಫೀನಲ್ಲಿನ ಶಾಂತ ಜೀವನವು ಬಳಲಿದ್ದ ಸ್ವಾಮೀಜಿಗೆ ನವಚೈತನ್ಯ ವನ್ನೇ ನೀಡಿತೆನ್ನಬಹುದು. ಒಮ್ಮೆ ಮಾತ್ರ ಒಂದು ಅಹಿತಕರವಾದ ಘಟನೆ ನಡೆಯಿತು. ಒಂದು ದಿನ ಬೆಳಿಗ್ಗೆ ಸ್ವಾಮೀಜಿ ತಮ್ಮ ಸಂಗಾತಿಗಳೊಂದಿಗೆ ಪರ್ವತದ ದಾರಿಯಾಗಿ ಹಿಮರಾಶಿಯಲ್ಲಿ ನಡೆದುಬರುತ್ತಿದ್ದರು. ದಾರಿ ತುಂಬ ಜಾರುವಂತಿದ್ದುದರಿಂದ, ಈಟಿಯಂತಹ ತುದಿಯುಳ್ಳ ಊರುಗೋಲಿನ ಸಹಾಯದಿಂದ ಎಚ್ಚರಿಕೆಯಿಂದ ನಡೆಯಬೇಕಾಗಿತ್ತು. ಸ್ವಾಮೀಜಿ ಉಪನಿಷತ್ತಿನ ಶ್ಲೋಕಗಳನ್ನು ಪಠಿಸುತ್ತ ಅದರ ಅರ್ಥವನ್ನು ವಿವರಿಸಿಕೊಂಡು ಬರುತ್ತಿದ್ದರು. ಹಾಗೆಯೇ ಅವರು ಭಾವಸ್ಥರಾಗುತ್ತ ಬಂದು ಮೌನರಾದರು. ಹೆಜ್ಜೆಗಳು ನಿಧಾನವಾದುವು. ಕ್ರಮೇಣ ಸ್ವಾಮೀಜಿ ಹಿಂದುಳಿದರು. ಸೇವಿಯರ್ ದಂಪತಿಗಳು ಮತ್ತು ಮಿಸ್ ಮುಲ್ಲರ್ ಮಾತಿನಲ್ಲಿ ಮಗ್ನರಾಗಿ ಮುಂದೆ ಸಾಗಿದರು. ಸುಮಾರು ಮೂರ್ನಾಲ್ಕು ನಿಮಿಷಗಳಾಗಿರಬಹುದು; ಹಿಂದಿನಿಂದ ಸ್ವಾಮೀಜಿ ಗಟ್ಟಿಯಾಗಿ ಏನೋ ಹೇಳುತ್ತ ವೇಗವಾಗಿ ನಡೆದುಬರುವುದನ್ನು ಅವರು ಕಂಡರು.“ಭಗವಂತನ ಕೃಪೆಯಿಂದ ನಾನು ಪಾರಾದೆ!” ಎಂದು ಸ್ವಾಮೀಜಿ ಉದ್ವೇಗದಿಂದ ಕೂಗುತ್ತ ಬರುತ್ತಿದ್ದರು. ಹತ್ತಿರ ಬಂದು ಏನಾಯಿತೆಂಬುದನ್ನು ವಿವರಿಸಿದರು–“ನಾನು ನನ್ನ ಕಬ್ಬಿಣದ ಊರುಗೋಲನ್ನು ಊರುತ್ತ ನಡೆದುಕೊಂಡು ಬರುತ್ತಿದ್ದೆ. ಆಗ ಅದು ಇದ್ದಕ್ಕಿದ್ದಂತೆ ಒಂದು ಪೊಳ್ಳಾದ ಜಾಗದಲ್ಲಿ ಮಂಜನ್ನು ಸೀಳಿಕೊಂಡು ಒಳಕ್ಕೆ ಹೋಗಿಬಿಟ್ಟಿತು. ಅದೊಂದು ಆಳವಾದ ಹಳ್ಳ. ಅದರೊಳಕ್ಕೆ ನಾನಿನ್ನೇನು ಬಿದ್ದೇಬಿಟ್ಟಿದ್ದೆ. ನಾನು ಪಾರಾದದ್ದೊಂದು ಪವಾಡವೇ ಸರಿ!” ಇದನ್ನು ಕೇಳಿ ಆ ಸ್ನೇಹಿತರ ಮನಸ್ಸು ಕದಡಿಹೋಯಿತು. ಸದ್ಯ, ಅಪಾಯದಿಂದ ಸ್ವಾಮೀಜಿ ಪಾರಾದರಲ್ಲ ಎಂದು ಎಲ್ಲರೂ ಸಮಾಧಾನದ ನಿಟ್ಟುಸಿರು ಬಿಟ್ಟರು. ಈ ಅವಗಢ ಸಂಭವಿಸಿದಂದಿನಿಂದ ಎಲ್ಲರೂ ಸ್ವಾಮೀಜಿಯನ್ನು ತುಂಬ ಎಚ್ಚರಿಕೆಯಿಂದ ನೋಡಿಕೊಳ್ಳುತ್ತಿದ್ದರು. ಮತ್ತು ಅವರನ್ನು ಎಲ್ಲಿಗೂ ಒಂಟಿಯಾಗಿ ಹೋಗಲು ಬಿಡುತ್ತಿರಲಿಲ್ಲ.

ಇವರೆಲ್ಲ ತಾವು ತಂಗಿದ್ದಲ್ಲಿಗೆ ಹಿಂದಿರುಗುವ ದಾರಿಯಲ್ಲಿ, ಗುಡ್ಡದ ಬದಿಯಲ್ಲೊಂದು ಪುಟ್ಟ ಕ್ರೈಸ್ತ ದೇವಾಲಯವಿತ್ತು. ಅದನ್ನು ನೋಡಿ ಸ್ವಾಮೀಜಿ, “ಕನ್ಯೆ ಮೇರಿಯ ಪಾದಗಳಿಗೆ ಪುಷ್ಪಾಂಜಲಿಯನ್ನು ಅರ್ಪಿಸೋಣ!” ಎನ್ನುತ್ತ ಅಲ್ಲಿ ಬಿಡುವ ಒಂದು ಬಗೆಯ ಹೂಗಳನ್ನು ಆರಿಸಿಕೊಂಡು ಬಂದರು. ಬಳಿಕ ಅದನ್ನು ಶ್ರೀಮತಿ ಸೇವಿಯರ್​ರ ಕೈಯಲ್ಲಿಡುತ್ತ, ಒಂದು ವಿಶಿಷ್ಟ ಭಾವದಲ್ಲಿ ಹೇಳಿದರು, “ನನ್ನ ಭಕ್ತಿ-ಕೃತಜ್ಞತೆಗಳ ಕುರುಹಾಗಿ ಇದನ್ನು ಮೇರಿಯ ಪಾದಗಳಿಗೆ ಅರ್ಪಿಸಿ. ಏಕೆಂದರೆ.. ಅವಳು ಜಗನ್ಮಾತೆಯೆ!”

ಸಾಸ್​ಫೀಯಲ್ಲಿ ಸ್ವಾಮೀಜಿ, ತಾವಿದ್ದಂತಹ ಉನ್ನತಭಾವದಲ್ಲೇ ಇನ್ನಷ್ಟು ದಿನ ಉಳಿದು ಕೊಳ್ಳಲು ನಿರ್ಧರಿಸುತ್ತಿದ್ದರೇನೋ. ಆದರೆ ಅವರನ್ನು ಅಲ್ಲಿಂದ ಬಲವಂತವಾಗಿ ಹೊರಡಿಸಲೋ ಎಂಬಂತೆ ಅವರಿಗೊಂದು ಪತ್ರ ಬಂದಿತು. ಇದರಿಂದಾಗಿ ಅವರ ಪ್ರವಾಸದ ಗತಿಯೇ ಬದಲಾ ಯಿತು. ಅದು ಜರ್ಮನಿಯ ಸುಪ್ರಸಿದ್ಧ ಪೌರ್ವಾತ್ಯ ಶಾಸ್ತ್ರಜ್ಞರಾದ ಪಾಲ್ ಡಾಯ್ಸನ್ \eng{(Paul Deussen)}ರಿಂದ ಬಂದ ಆಹ್ವಾನ. ಇವರು ಜರ್ಮನಿಯ ಕೀಲ್ ವಿಶ್ವವಿದ್ಯಾನಿಲಯದಲ್ಲಿ ತತ್ತ್ವ ಶಾಸ್ತ್ರದ ಪ್ರೊಫೆಸರರಾಗಿದ್ದರು. ವೇದಾಂತದಲ್ಲಿ ಆಳವಾದ ಆಸಕ್ತಿ ಹೊಂದಿದ್ದ ಪ್ರೊ ॥ ಡಾಯ್ಸ ನ್ನರು, ಪಕ್ಕಾ ವೇದಾಂತಿಯೇ ಆಗಿಬಿಟ್ಟಿದ್ದರು. ಕೆಲದಿನಗಳ ಹಿಂದೆಯಷ್ಟೇ ಅವರು ಭಾರತದ ಪ್ರವಾಸವನ್ನು ಮುಗಿಸಿಕೊಂಡು ಹಿಂದಿರುಗಿದ್ದರು. ಸ್ವಾಮೀಜಿಯ ಉಪನ್ಯಾಸಗಳನ್ನೂ ಬೋಧನೆ ಗಳನ್ನೂ ಅಧ್ಯಯನ ಮಾಡಿ, ಅವರೊಬ್ಬ ಸ್ವತಂತ್ರ ಚಿಂತಕ ಹಾಗೂ ಪ್ರಚಂಡ ಆಧ್ಯಾತ್ಮಿಕ ಪ್ರಭಾವಶಾಲಿ ಎಂದು ಕೊಂಡುಕೊಂಡಿದ್ದರು. ವಿವೇಕಾನಂದರು ಇಂಗ್ಲೆಂಡಿಗೆ ಬಂದಿರುವ ಈ ಸದವಕಾಶವನ್ನು ಉಪಯೋಗಿಸಿಕೊಂಡು ಅವರೊಂದಿಗೆ ಹಲವಾರು ವಿಷಯಗಳ ಬಗ್ಗೆ ಚರ್ಚಿ ಸಲು ಡಾಯ್ಸನ್ ಉತ್ಸುಕರಾದರು. ಆದ್ದರಿಂದ ಅವರು ಸ್ವಾಮೀಜಿಯ ಲಂಡನ್ನಿನ ವಿಳಾಸ ಕ್ಕೊಂದು ಪತ್ರ ಬರೆದು, ಕೀಲ್ ನಗರಕ್ಕೆ ಬಂದು ಕೆಲದಿನಗಳ ಮಟ್ಟಿಗಾದರೂ ತಮ್ಮ ಅತಿಥಿಯಾಗಿ ಉಳಿದುಕೊಳ್ಳಬೇಕೆಂದು ಅವರನ್ನೂ ಅವರ ಆತಿಥೇಯನಾದ ಸ್ಪರ್ಡಿಯನ್ನೂ ಕೇಳಿಕೊಂಡಿದ್ದರು. ಈ ಪತ್ರವನ್ನು ಸ್ಟರ್ಡಿ ಸಾಸ್​ಫೀಗೆ ಕಳಿಸಿಕೊಟ್ಟಿದ್ದ. ಆದರೆ ಕಾರ್ಯಾಂತರಗಳಿಂದಾಗಿ ತನಗೆ ಬರಲು ಸಾಧ್ಯವಾಗುವುದಿಲ್ಲವೆಂದು ಆತ ತಿಳಿಸಿದ. ಸ್ವಾಮೀಜಿಗೂ ಈ ಆಮಂತ್ರಣವನ್ನು ಅಂಗೀಕರಿಸಲು ಮನಸ್ಸಿರಲಿಲ್ಲ. ಏಕೆಂದರೆ, ಮೊದಲನೆಯದಾಗಿ, ಆಗ ತಾವಿದ್ದ ಪ್ರಶಾಂತ- ಏಕಾಂತ ಪರಿಸರವನ್ನು ಬಿಟ್ಟುಹೋಗಲು ಇಷ್ಟವಿಲ್ಲದುದರಿಂದ; ಎರಡನೆಯದಾಗಿ, ಈ ದೀರ್ಘ ಪ್ರವಾಸದಿಂದಾಗಿ ತಮ್ಮ ಸಂಗಡಿಗರ ಮೇಲೆ ಮತ್ತಷ್ಟು ಆರ್ಥಿಕ ಒತ್ತಡ ಬೀಳುತ್ತದೆ ಎಂಬ ಕಾರಣಕ್ಕಾಗಿ. ಆದರೆ ಮಿಸ್ ಮುಲ್ಲರ್ ಸ್ವಾಮೀಜಿಯ ಈ ಆಕ್ಷೇಪಗಳನ್ನೆಲ್ಲ ತಳ್ಳಿಹಾಕಿ, ಸ್ವಾಮೀಜಿ ಆಹ್ವಾನವನ್ನು ಅಂಗೀಕರಿಸಿದ್ದಾರೆಂದು ಡಾಯ್ಸನ್ನರಿಗೆ ತಂತಿ ಕಳಿಸಿದಳು. ಮರುದಿನವೇ (ಆಗಸ್ಟ್ ಒಂಬತ್ತರಂದು) ಡಾಯ್ಸನ್ನರಿಂದ ಮರುತಂತಿ ಬಂದಿತು. ಸ್ವಾಮೀಜಿಯನ್ನು ಹಾರ್ದಿಕಾಗಿ ಸ್ವಾಗತಿಸಿ ಅವರು ಮತ್ತೆ ಪತ್ರ ಬರೆದರು. ಸೆಪ್ಟೆಂಬರ್​೧೯ ರಂದು ಭೇಟಿಯೆಂದು ಕಡೆಗೆ ನಿರ್ಧಾರ ವಾಯಿತು. ಕೀಲ್​ಗೆ ಹೊರಡುವ ಮುನ್ನ ಸ್ವಿಟ್ಸರ್​ಲ್ಯಾಂಡಿನ ಪ್ರವಾಸವನ್ನು ಪೂರ್ಣಗೊಳಿಸ ಬೇಕೆಂದು ಸ್ವಾಮೀಜಿಯ ಸಹಚರರು ಅವರನ್ನು ಒಪ್ಪಿಸಿದರು. ಅಲ್ಲದೆ ಜರ್ಮನಿಯ ಇತರ ಕೆಲವು ನಗರಗಳನ್ನು ನೋಡುವ ವ್ಯವಸ್ಥೆಯನ್ನೂ ಮಾಡಿದರು.

ಈ ವೇಳೆಗೆ, ಮಿಸ್ ಮುಲ್ಲರ್ ಯಾವುದೋ ಒಂದು ತುರ್ತಿನ ಕಾರ್ಯಕ್ಕಾಗಿ ಲಂಡನ್ನಿಗೆ ಹಿಂದಿರುಗಬೇಕಾಗಿಬಂತು. ಆದ್ದರಿಂದ ಕಾರ್ಯಕ್ರಮವನ್ನು ಸ್ವಲ್ಪ ಬದಲಿಸಿಕೊಂಡು, ಸಾಸ್​ಫೀ ನಿಂದ ಎಲ್ಲರೂ ಲ್ಯೂಸರ್ನ್ ನಗರಕ್ಕೆ ಹೊರಟರು. ಇಲ್ಲಿ ಮಿಸ್ ಮುಲ್ಲರ್ ಕೆಲವು ದಿನಗಳನ್ನು ಇತರರೊಂದಿಗೆ ಕಳೆದು, ಬಳಿಕ ಅವರಿಂದ ಬೀಳ್ಕೊಂಡು ಹೊರಟಳು. ಇದು ಸ್ವಿಟ್ಸರ್​ಲ್ಯಾಂಡಿನ ಅತ್ಯಂತ ರಮಣೀಯ ಸ್ಥಳಗಳಲ್ಲೊಂದು. ಸ್ವಾಮೀಜಿ ತಮ್ಮ ಸಂಗಡಿಗರೊಂದಿಗೆ, ಇದರ ಸುತ್ತ ಮುತ್ತಲಿನ ಎಲ್ಲ ಪ್ರೇಕ್ಷಣೀಯ ಸ್ಥಳಗಳನ್ನೂ ಸಂದರ್ಶಿಸಿದರು. ಬಳಿಕ ಕ್ಯಾಪ್ಟನ್ ಸೇವಿಯರ ರನ್ನುಳಿದು ಮಿಕ್ಕವರೆಲ್ಲ ರೈಲಿನಲ್ಲಿ ಮೌಂಟ್ ರಿಗಿ ಎಂಬ ಪರ್ವತ ಶಿಖರವನ್ನೇರಿದರು. ಈ ಪ್ರಯಾಣವೇ ಒಂದು ರೋಮಾಂಚಕಾರಿ ಅನುಭವ. ಈ ಶಿಖರದ ಮೇಲೆ ನಿಂತು ನೋಡಿದರೆ ಜಗತ್ತಿನಲ್ಲೇ ಅತಿ ಸುಂದರವಾದ ಹಿಮಮಯ ದೃಶ್ಯ ಕಾಣಸಿಗುತ್ತದೆ. ಲ್ಯೂಸರ್ನ್​ನಲ್ಲಿ ಸ್ವಾಮೀಜಿಯ ಮನಸೆಳೆದದ್ದು ಅಲ್ಲಿನ ಸಿಂಹಸ್ಮಾರಕ. ಇದೊಂದು ಸಾಯುತ್ತಿರುವ ಸಿಂಹದ ಆಕೃತಿ. ಇದನ್ನು ಮರಳುಗಲ್ಲನ್ನು ಜೋಡಿಸಿ ರಚಿಸಲಾಗಿದೆ. ಸರೋವರವೊಂದರ ಮೇಲೆ ರಚಿತ ವಾದ ಈ ಸಿಂಹದ ರೂಪವು ನೀರಿನಲ್ಲಿ ಪ್ರತಿಫಲಿಸಿ ಅದು ಅತ್ಯಂತ ವಿಶಿಷ್ಟವಾಗಿ ಹಾಗೂ ಭವ್ಯ ವಾಗಿ ಕಾಣುವಂತಾಗಿದೆ. ಸ್ವಾಮೀಜಿ ಅಲ್ಲಿ ನೋಡಿದ ಇನ್ನೆರಡು ವಿಶೇಷಗಳೆಂದರೆ ಅಲ್ಲಿನ ರಿಯಸ್ ಎಂಬ ನದಿಯ ಮೇಲಿನ ಎರಡು ಮರದ ಸೇತುವೆಗಳು. ಇವುಗಳಲ್ಲಿ ಒಂದರ ಮೇಲೆ ಸ್ವಿಟ್ಸರ್​ಲ್ಯಾಂಡ್ ದೇಶದ ಇತಿಹಾಸವನ್ನು ನಿರೂಪಿಸಿರುವ ಅನೇಕ ಚಿತ್ರಫಲಕಗಳಿವೆ. ಇನ್ನೊಂದ ರಲ್ಲಿ ಮೃತ್ಯುನರ್ತನವನ್ನು ಬಣ್ಣಿಸುವ ಚಿತ್ರಗಳಿವೆ. ಬಹುಶಃ ಆ ಸೇತುವೆಯ ಮೇಲೆ ಸ್ವಲ್ಪ ಎಚ್ಚರ ದಿಂದ ನಡೆದರೆ ವಾಸಿ ಎಂಬುದನ್ನು ಸೂಚಿಸಲು ಈ ಬಗೆಯ ಚಿತ್ರವನ್ನು ಬರೆಯಲಾಗಿದೆಯೋ ಏನೋ! ಇವುಗಳಲ್ಲದೆ ಸ್ವಾಮೀಜಿ ಲ್ಯೂಸರ್ನಿನಲ್ಲಿ ಅಲ್ಲಿನ ವಸ್ತುಪ್ರದರ್ಶನಾಲಯವನ್ನೂ ಐತಿ ಹಾಸಿಕ ಚರ್ಚನ್ನೂ ಸಂದರ್ಶಿಸಿದರು. ಆ ಚರ್ಚಿನಲ್ಲಿ ೧೭ನೇ ಶತಮಾನದ ವಿಶಿಷ್ಟ ಆರ್ಗನ್ ವಾದ್ಯವನ್ನು ಇಡಲಾಗಿದೆ. “ವಾಕ್ಸ್ ಹ್ಯುಮೇನ” (‘ಮನುಷ್ಯರ ಧ್ವನಿ’) ಎಂದು ಕರೆಯಲ್ಪಡುವ, ಪೀಪಿಯಂತಹ ಈ ವಾದ್ಯದ ನಾದವು ಮನುಷ್ಯರ ಕಂಠಸ್ವರವನ್ನು ಬಹುವಾಗಿ ಹೋಲುತ್ತದೆ. ಇದನ್ನು ಕೇಳಿದ ತಕ್ಷಣ ಸ್ವಾಮೀಜಿ, ನಿಜಕ್ಕೂ ಅದನ್ನು ಮನುಷ್ಯನ ಧ್ವನಿಯೆಂದೇ ಭಾವಿಸಿದರಂತೆ!

ಒಂದು ದಿನ ಸ್ವಾಮೀಜಿಯ ತಂಡದವರು ಸುಂದರವಾದ ಲ್ಯೂಸರ್ನ್ ಸರೋವರದಲ್ಲಿ ನೌಕಾ ಯಾನ ಮಾಡಿದರು. ಸರೋವರದ ಸುತ್ತಲಿನ ಮನೋಹರ ದೃಶ್ಯವನ್ನು ಕಂಡು ಎಲ್ಲರೂ ಆನಂದ ಭರಿತರಾದರು. ಸ್ವಿಟ್ಸರ್​ಲ್ಯಾಂಡಿನ ಪ್ರಸಿದ್ಧ ದೇಶಭಕ್ತ ಸಾಹಸಿ ವಿಲಿಯಂ ಟೆಲ್ಲನಿಗೆ ಅರ್ಪಿತವಾದ ಒಂದು ದೇವಾಲಯವನ್ನು ಕಂಡು ಸ್ವಾಮೀಜಿ, ಅವನ ಅದ್ಭುತ ಜೀವನವನ್ನು ಸ್ಮರಿಸಿದರು. ಇವುಗಳಲ್ಲದೆ, ಇಲ್ಲಿ ಅವರು ಕಂಡ ಮತ್ತೊಂದು ವಿಶೇಷವೆಂದರೆ, ಪಾಶ್ಚಾತ್ಯ ದೇಶಗಳಲ್ಲಿ ಅವರಿಗೆ ದೊರಕಿದ್ದ ಅತಿಖಾರದ ಮೆಣಸಿನಕಾಯಿಗಳು. ಇಂತಹ ಮೂರ್ನಾಲ್ಕು ಮೆಣಸಿನಕಾಯಿ ಗಳನ್ನು ಒಟ್ಟಿಗೆ ಬಾಯಿಗೆ ಹಾಕಿಕೊಂಡು ಸಂತೋಷದಿಂದ ಅಗಿಯುತ್ತ ಸ್ವಾಮೀಜಿ ಅಂಗಡಿ ಯವನನ್ನು ಕೇಳಿದರು, “ಏನಪ್ಪ, ಇದಕ್ಕಿಂತ ಖಾರದ ನಮೂನೆ ನಿಮ್ಮಲ್ಲಿಲ್ಲವೆ?”

ಲ್ಯೂಸರ್ನಿನಿಂದ ಮಿಸ್ ಮುಲ್ಲರ್ ಇಂಗ್ಲೆಂಡಿಗೆ ಹೊರಟಳು. ಅವಳಿಗೆ ವಿದಾಯ ಕೋರಿ, ಸ್ವಾಮೀಜಿ ಹಾಗೂ ಸೇವಿಯರ್ ದಂಪತಿಗಳು, ಶಾಫ್ ಹೌಸನ್ನಿಗೆ ಬಂದರು. ಇಲ್ಲಿ ರೈನ್ ನದಿಯ ಸುಂದರ ಜಲಪಾತವನ್ನು ವೀಕ್ಷಿಸಿ, ಅಲ್ಲಿಂದ ಜರ್ಮನಿಯ ಹೈಡೆಲ್​ಬರ್ಗ್​ಗೆ ಟ್ರೈನಿನಲ್ಲಿ ಪ್ರಯಾಣ ಬೆಳೆಸಿದರು. ಜರ್ಮನಿಯ ಪುರಾತನವೂ ಪ್ರಸಿದ್ಧವೂ ಆದ ಹಲವು ವಿಶ್ವವಿದ್ಯಾನಿಲಯ ಗಳಲ್ಲಿ ಒಂದು ಈ ನಗರದಲ್ಲಿದೆ. ಸ್ವಾಮೀಜಿ ಈ ವಿಶ್ವವಿದ್ಯಾನಿಲಯವನ್ನು ಸಂದರ್ಶಿಸಿ ಅಲ್ಲಿ ವ್ಯಾಸಂಗಮಾಡಲು ಅವಕಾಶವಿದ್ದ ವಿಷಯಗಳನ್ನೂ ಅಲ್ಲಿನ ಇತರ ಸೌಲಭ್ಯಗಳನ್ನೂ ಪರಿಶೀಲಿಸಿ ದರು. ಉನ್ನತ ವ್ಯಾಸಂಗಕ್ಕೆ ಅಲ್ಲಿ ಲಭ್ಯವಿದ್ದ ಅನುಕೂಲತೆಗಳನ್ನು ಕಂಡು ಅವರಿಗೆ ಆನಂದ ಆಶ್ಚರ್ಯಗಳುಂಟಾದುವು. ಭಾರತದಲ್ಲಿ ಇಂತಹ ಪ್ರಮಾಣದಲ್ಲಿ ವಿಶ್ವವಿದ್ಯಾನಿಲಯಗಳು ಪ್ರಾರಂಭವಾಗುವುದೆಂದು? ಎಂದು ಅವರ ಮನಸ್ಸು ಚಿಂತಿಸಿತು. ಇಲ್ಲಿಂದ ಅವರು ರೈನ್ ನದಿಯ ಮಾರ್ಗವಾಗಿ ಕೊಲೋನ್ ನಗರಕ್ಕೆ ಬಂದು, ಇಲ್ಲಿನ ಹಲವಾರು ಸ್ಥಳಗಳನ್ನು ವೀಕ್ಷಿಸುತ್ತ ಕೆಲವು ದಿನಗಳನ್ನು ಕಳೆದರು. ಕೊಲೋನ್ ನಗರದಲ್ಲಿ ಪ್ರತಿಬಿಂಬಿತವಾಗುತ್ತಿದ್ದ ಭವ್ಯ ಜರ್ಮನ್ ಸಂಸ್ಕೃತಿಯನ್ನು ಬಹಳವಾಗಿ ಮೆಚ್ಚಿಕೊಂಡರು. ಇಲ್ಲಿನ ಬೃಹತ್ ಚರ್ಚೊಂದರ ಪ್ರಾರ್ಥನಾ ಸಭೆ ಯಲ್ಲೂ ಅವರು ಭಾಗವಹಿಸಿದರು. ಆ ಚರ್ಚಿನ ಭವ್ಯತೆಯನ್ನು, ಗಾಥಿಕ್ ಶೈಲಿಯ ಶಿಲ್ಪ ಚಾತುರ್ಯವನ್ನು, ಅದರ ವೈಭವಪೂರ್ಣ ಕಲಾಕೌಶಲವನ್ನು ಕಂಡು ಅವರು ವಿಸ್ಮಿತರಾದರು.

ಸ್ವಾಮೀಜಿಯ ಅಪೇಕ್ಷೆಯಂತೆ, ಸೇವಿಯರ್ ದಂಪತಿಗಳು ಅವರನ್ನು ಜರ್ಮನಿಯ ರಾಜಧಾನಿ ಯಾದ ಬರ್ಲಿನ್ನಿಗೆ ಕರೆದೊಯ್ದರು. ಆಗ ಅತ್ಯಂತ ಸಂಪದ್ಭರಿತವೂ ಪ್ರಬಲವೂ ಆಗಿದ್ದ ಅವಿಭಜಿತ ಜರ್ಮನಿಯ ವೈಭವವನ್ನು ಕಂಡು ಸ್ವಾಮೀಜಿ ಬೆರಗಾದರು. ಅವರು ಅಲ್ಲಿನ ಎಲ್ಲ ಐತಿಹಾಸಿಕ ಹಾಗೂ ಬೌದ್ಧಿಕ ಪ್ರಾಮುಖ್ಯತೆಯ ಸ್ಥಳಗಳನ್ನು ಸಂದರ್ಶಿಸಿ ಆನಂದಿಸಿದರು. ಅವರು ಪ್ರತಿಯೊಂದು ವಿಷಯದಲ್ಲೂ ಆಸಕ್ತರಾಗಿದ್ದರು ಮತ್ತು ಮಾನವನ ಬುದ್ಧಿಶಕ್ತಿಯಿಂದ ಒಡ ಮೂಡಿದ ಯಾವುದೇ ವಸ್ತುವನ್ನು ಕಂಡರೂ ಅದನ್ನು ಆಸ್ವಾದಿಸಿ ಅದರ ಬಗ್ಗೆ ಮೆಚ್ಚುಗೆ ವ್ಯಕ್ತ ಪಡಿಸುತ್ತಿದ್ದರು. ಶ್ರೀಮತಿ ಸೇವಿಯರ್ ಆ ಬಗ್ಗೆ ಹೇಳುತ್ತಾರೆ, “ಮನುಷ್ಯನ ಚಟುವಟಿಕೆಯ ಪ್ರತಿಯೊಂದು ಕ್ಷೇತ್ರವೂ, ಜ್ಞಾನದ ಪ್ರತಿಯೊಂದು ಅಂಗವೂ ಸ್ವಾಮೀಜಿಯ ಪಾಲಿಗೆ ಆಸಕ್ತಿಕರ ವಾಗಿದ್ದುವು. ಸದಾ ಪ್ರಪುಲ್ಲವೂ ಕರುಣಾಪೂರಿತವೂ ಆದ ಅವರ ಮನೋಭಾವ, ಅವರ ಬುದ್ಧಿಮತ್ತೆ ಮತ್ತು ವ್ಯಕ್ತಿತ್ವದ ಸೊಬಗು–ಇವುಗಳಿಂದಾಗಿ ಅವರೊಂದಿಗೆ ಪ್ರಯಾಣ ಮಾಡುವು ದೆಂದರೆ ಅದೊಂದು ಅತ್ಯಂತ ಆಹ್ಲಾದಕರ ಅನುಭವವಾಗಿತ್ತು!”

ಬರ್ಲಿನ್ನಿನಿಂದ ಸ್ವಾಮೀಜಿ ಮತ್ತು ಅವರ ಸಹಚರರು, ಪ್ರೊ ॥ ಪಾಲ್ ಡಾಯ್ಸನ್ನರನ್ನು ಭೇಟಿ ಯಾಗಲು ಕೀಲ್ ಪಟ್ಟಣದೆಡೆಗೆ ಹೊರಟು, ಸೆಪ್ಟೆಂಬರ್ ೮ರಂದು ಅಲ್ಲಿಗೆ ತಲುಪಿದರು. ಅದು ಬಾಲ್ಟಿಕ್ ಸಮುದ್ರತೀರದ ಒಂದು ಸುಂದರ ಪಟ್ಟಣ. ಸ್ವಾಮೀಜಿ ಮತ್ತು ಅವರ ಸಂಗಡಿಗರ ಆಗಮನದ ಬಗ್ಗೆ ತಿಳಿದ ಪ್ರೊ ॥ ಡಾಯ್ಸನ್ನರು ಅವರನ್ನು ಅತ್ಯಾದರದಿಂದ ಸ್ವಾಗತಿಸಿದರು. ಉಭಯಕುಶಲೋಪರಿಯಾದ ಮೇಲೆ, ಮಾತುಕತೆ ತಾನಾಗಿಯೇ ವೇದಾಂತದತ್ತ ತಿರುಗಿತು. “ವೇದಾಂತವೆಂಬುದು ಸತ್ಯಾನ್ವೇಷಣೆಯಲ್ಲಿ ತೊಡಗಿದ ಮಾನವನ ಬುದ್ಧಿಶಕ್ತಿಯಿಂದುದಿಸಿದ ಅತ್ಯಂತ ಭವ್ಯವಾದ ವಸ್ತು, ಅನರ್ಘ್ಯರತ್ನ” ಎಂದು ಡಾಯ್ಸನ್ನರು ಅಭಿಪ್ರಾಯಪಟ್ಟರು. ಅಲ್ಲದೆ, ಈ ವೇದಾಂತವು ಕೇವಲ ಬುದ್ದಿಜೀವಿಗಳ ಸರಕಲ್ಲ; ಬದಲಾಗಿ ಪರಿಶುದ್ಧ ನೈತಿಕತೆಯನ್ನು ಎತ್ತಿಹಿಡಿಯುವ ಒಂದು ಪ್ರಬಲ ಶಕ್ತಿ, ಮತ್ತು ಜೀವನ್ಮರಣಗಳ ಜಂಜಡದಲ್ಲಿ ಸಿಲುಕಿ ನರಳುವ ಮಾನವನಿಗೆ ಸಾಂತ್ವನ ನೀಡಬಲ್ಲ ಚಿಲುಮೆ ಇದು ಎಂದು ಪ್ರೊ ॥ ಡಾಯ್ಸನ್ ಉದ್ಗರಿಸಿದರು. ಶಾಂಕರಭಾಷ್ಯದಿಂದೊಡಗೂಡಿದ ಉಪನಿಷತ್ತುಗಳು ಹಾಗೂ ವೇದಾಂತ ಸೂತ್ರಗಳಿಂದ ಅವರು ತೀವ್ರವಾಗಿ ಪ್ರಭಾವಿತರಾಗಿದ್ದರು. ಅವರು ಉಪನಿಷತ್ತುಗಳಲ್ಲಿ ಪ್ರತಿಪಾದಿತವಾಗಿರುವ ತಾತ್ವಿಕತೆಯ ಕುರಿತಾಗಿ ಜರ್ಮನ್ ಭಾಷೆಯಲ್ಲಿ ಒಂದು ಉದ್ಗ್ರಂಥವನ್ನು ಬರೆದಿದ್ದು, ಅದು ಇಂಗ್ಲಿಷಿನಲ್ಲಿ \eng{‘The Philosphy of the Upanishads’} ಎಂಬ ಹೆಸರಿನಲ್ಲಿ ಜಗತ್ಪ್ರಸಿದ್ಧವಾಗಿದೆ. ಆದರೆ ತಮ್ಮ ಸಂಸ್ಕೃತಿಯ ಘನತೆ-ಮಹತ್ವಗಳ ಬಗ್ಗೆಯೇ ಅರಿವಿಲ್ಲದೆ ಭಾರತೀಯರು ತಮ್ಮ ಸನಾತನ ಶಾಸ್ತ್ರಗ್ರಂಥಗಳನ್ನು ಕಡೆಗಣಿಸುವುದರ ಬಗ್ಗೆ ಅವರು ಎಚ್ಚರಿಕೆ ನೀಡಿದ್ದರು. ಕೆಲದಿನಗಳ ಹಿಂದೆಯಷ್ಟೇ ಪ್ರೊ ॥ ಡಾಯ್ಸನ್ನರು ಭಾರತಕ್ಕೆ ಭೇಟಿ ನೀಡಿದ್ದಾಗ ‘ರಾಯಲ್ ಏಷಿಯಾಟಿಕ್ ಸೊಸೈಟಿ’ಯ ಮುಂಬಯಿ ಶಾಖೆಯಲ್ಲಿ ಭಾಷಣ ಮಾಡಿ, ವೇದಾಂತದ ಮಹತ್ವವನ್ನು ಕೊಂಡಾಡಿ ದ್ದರು. ಡಾ ॥ ಡಾಯ್ಸನ್ನರು ಮಾಡುತ್ತಿದ್ದ ಕೆಲವು ಭಾಷಾಂತರಗಳಲ್ಲಿ ಸ್ವಾಮೀಜಿ ಆಸಕ್ತಿ ತೋರಿಸಿ ದರು. ಸಂಸ್ಕೃತ ಗ್ರಂಥಗಳ ಕೆಲವು ಶ್ಲೋಕಗಳ ನಿಜವಾದ ಭಾವ ಹಾಗೂ ಪ್ರಾಮುಖ್ಯತೆಯ ಬಗ್ಗೆ ಡಾಯ್ಸನ್ನರು ಸ್ವಾಮೀಜಿಯನ್ನು ಪ್ರಶ್ನಿಸಿದರು. ವೇದಾಂತದ ಧೀರ ಪ್ರತಿಪಾದಕರಾದ ಸ್ವಾಮೀಜಿ ಅವುಗಳ ಬಗ್ಗೆ ನೀಡಿದ ವಿವರಣೆಗಳಿಗೆ ಡಾಯ್ಸನ್ನರು ಸಂಪೂರ್ಣ ಮನಸೋತರು.

ಸ್ವಲ್ಪ ಹೊತ್ತಾದ ಮೇಲೆ ಸ್ವಾಮೀಜಿ ಅಲ್ಲಿಂದ ಹೊರಟುನಿಂತರು. ಆದರೆ ಡಾಯ್ಸನ್ನರು ಅವರನ್ನು ಬಲವಂತವಾಗಿ ಮತ್ತಷ್ಟು ಹೊತ್ತು ಉಳಿಸಿಕೊಂಡರು. ಅಂದು ಡಾಯ್ಸನ್ನರ ನಾಲ್ಕು ವರ್ಷದ ಮಗಳು ಎರಿಕಾಳ ಹುಟ್ಟಿದಹಬ್ಬ. ಆದ್ದರಿಂದ ಅಂದು ತಮ್ಮ ಮನೆಯಲ್ಲೇ ಭೋಜನ ವನ್ನು ಸ್ವೀಕರಿಸಿ ಸಂಜೆಯ ಸಂತೋಷಕೂಟದಲ್ಲೂ ಭಾಗವಹಿಸಿ ಹೋಗಬೇಕೆಂದು ಒತ್ತಾಯಿಸಿ ಒಪ್ಪಿಸಿದರು. ಅಂದು ಅಲ್ಲೊಂದು ವಿಶೇಷ ಉಲ್ಲಾಸದ ವಾತವರಣ ನಿರ್ಮಾಣವಾಗಿತ್ತು. ಭೋಜನದ ಬಳಿಕ ಡಾ ॥ ಡಾಯ್ಸನ್ನರು ತಮ್ಮ ಭಾರತ ಪ್ರವಾಸದ ಅನುಭವಗಳನ್ನೆಲ್ಲ ಸುವಿಸ್ತಾರ ವಾಗಿ ಬಣ್ಣಿಸಿದರು. ಭಾರತದ ತೀರ್ಥಕ್ಷೇತ್ರಗಳು, ಪವಿತ್ರ ಗಂಗಾನಂದಿ, ಅಲ್ಲಿ ನೆರೆಯುವ ಅಸಂಖ್ಯಾತ ಭಕ್ತರು–ಇವೆಲ್ಲ ಅವರ ಪಾಲಿಗೆ ಅತ್ಯಂತ ಕೌತುಕದ ವಿಷಯಗಳಾಗಿದ್ದುವು. ಭಾರತೀಯರ ದೃಷ್ಟಿಕೋನ, ಚಿಂತನ ವಿಧಾನ, ಪರಧರ್ಮ ಸಹಿಷ್ಣುತೆ–ಒಟ್ಟಿನಲ್ಲಿ ಭಾರತೀಯ ವಾದ ಪ್ರತಿಯೊಂದೂ ಅವರಿಗೆ ಅಚ್ಚುಮೆಚ್ಚಾಗಿತ್ತು.

ಅಂದು ಮಧ್ಯಾಹ್ನ ಸ್ವಾಮೀಜಿಯವರು ಪ್ರೊ ॥ ಡಾಯ್ಸನ್ನರ ಗ್ರಂಥಾಲಯದಲ್ಲಿ ಯಾವುದೋ ಒಂದು ಕಾವ್ಯಸಂಪುಟವನ್ನು ತೆರೆದು ನೋಡುತ್ತಿದ್ದರು. ಆಗ ಅಲ್ಲಿಗೆ ಬಂದು ಪ್ರೊಫೆಸರರು, “ಸ್ವಾಮೀಜಿ...” ಎಂದು ಅವರನ್ನು ಕರೆದರು. ಆದರೆ ಸ್ವಾಮೀಜಿ ಉತ್ತರಿಸಲಿಲ್ಲ. ಅವರು ಮತ್ತೊಮ್ಮೆ ಕರೆದಾಗಲೂ ಉತ್ತರ ಬರಲಿಲ್ಲ. ಸ್ವಾಮೀಜಿ ಆ ಪುಸ್ತಕದಲ್ಲೇ ತನ್ಮಯರಾಗಿಬಿಟ್ಟಿ ದ್ದರು. ಎಷ್ಟೋ ಹೊತ್ತಾದ ಮೇಲೆ ಸ್ವಾಮೀಜಿಗೆ ಈ ವಿಷಯ ತಿಳಿಯಿತು. ತಕ್ಷಣ ಪ್ರೊಫೆಸರರ ಕ್ಷಮೆ ಯಾಚಿಸಿದರು. ತಾವು ಓದಿನಲ್ಲಿ ಮುಳುಗಿಹೋಗಿದ್ದುದರಿಂದ ತಮಗೆ ಅವರು ಕೂಗಿದ್ದು ಕೇಳಿಸಲಿಲ್ಲವೆದು ಹೇಳಿದರು. ಈ ಉತ್ತರದಿಂದ ಡಾಯ್ಸನ್ನರಿಗೆ ತೃಪ್ತಿಯಾಗಲಿಲ್ಲ. ಆದರೆ ಬಳಿಕ ಸಂಭಾಷಣೆಯ ವೇಳೆಯಲ್ಲಿ ಸ್ವಾಮೀಜಿ ಆ ಕಾವ್ಯಸಂಪುಟದಿಂದ ಕೆಲವು ಕವನಗಳನ್ನು ಹಾಗೆ ಹಾಗೆಯೇ ಉದ್ಧರಿಸಿ ಅವುಗಳಿಗೆ ವಿಶೇಷವಾದ ವಿವರಣೆ ಕೊಟ್ಟಾಗ ಡಾಯ್ಸನ್ನರು ದಂಗುಬಿಡಿದು ಬಿಟ್ಟರು. ಅವರ ಬಾಯಿಯಿಂದ ಒಂದು ಕ್ಷಣ ಮಾತೇ ಹೊರಡಲಿಲ್ಲ. ಬಳಿಕ, “ಸ್ವಾಮೀಜಿ, ಇಂತಹ ಅದ್ಭುತವಾದ ಜ್ಞಾಪಕಶಕ್ತಿಯನ್ನು ನೀವು ಹೇಗೆ ಸಂಪಾದಿಸಿಕೊಂಡಿರಿ?” ಎಂದು ಅವರನ್ನೇ ಕೇಳಿದರು. ಆಗ ಸ್ವಾಮೀಜಿ, ಭಾರತೀಯ ಯೋಗಿಗಳು ಏಕಾಗ್ರತೆಯನ್ನು ಸಾಧಿಸುವ ವಿಧಾನವನ್ನು ಪ್ರಸ್ತಾಪಿಸಿ, ತಮ್ಮ ಸ್ವಂತ ಅನುಭವದ ಆಧಾರದ ಮೇಲೆ ಅದನ್ನು ವಿವರಿಸಿದರು. ತೀವ್ರ ಏಕಾಗ್ರತೆಯ ಸ್ಥಿತಿಯಲ್ಲಿದ್ದಾಗ ಯೋಗಿಗಳು ಶರೀರಪ್ರಜ್ಞೆಯನ್ನು ಸಂಪೂರ್ಣವಾಗಿ ಮೀರಿ ಹೋಗುತ್ತಾರೆಂದು ಅವರು ತಿಳಿಸಿದರು.

ಅದೇ ಸಮಯದಲ್ಲಿ ಕೀಲ್ ಪಟ್ಟಣದಲ್ಲಿ ಪ್ರದರ್ಶನವೊಂದು ನಡೆಯುತ್ತಿತ್ತು. ಪ್ರದರ್ಶನ ವನ್ನು ಸ್ವಾಮೀಜಿ ನೋಡಲೇಬೇಕೆಂದು ಡಾಯ್ಸನ್ನರು ಒತ್ತಾಯಿಸಿ, ಸಂಜೆಯ ವೇಳೆಗೆ ತಾವೇ ಅವರನ್ನು ಅಲ್ಲಿಗೆ ಕರೆದುಕೊಂಡು ಹೋದರು. ಜರ್ಮನಿಯ ಕಲೆಗಳು, ವಿವಿಧ ಕೈಗಾರಿಕೋತ್ಪನ್ನ ಗಳು ಮೊದಲಾದವನ್ನು ಅಲ್ಲಿ ಪ್ರದರ್ಶಿಸಲಾಗಿತ್ತು. ಎಲ್ಲವನ್ನೂ ಸ್ವಾಮೀಜಿ ತುಂಬ ಆಸಕ್ತಿ ಯಿಂದ ವೀಕ್ಷಿಸಿದರು. ಮರುದಿನ ಪ್ರೊಫೆಸರರು ಸ್ವಾಮೀಜಿ ಹಾಗೂ ಸೇವಿಯರ್ ದಂಪತಿಗಳನ್ನು ಕರೆದೊಯ್ದು, ಆ ಊರಿನ ಆಸುಪಾಸಿನ ಅನೇಕ ಪ್ರೇಕ್ಷಣೀಯ ಸ್ಥಳಗಳನ್ನು ತೋರಿಸಿದರು. ಅಲ್ಲದೆ ಆಗ ಕೆಲದಿನಗಳ ಹಿಂದೆಯಷ್ಟೇ ಉದ್ಘಾಟಿಸಲ್ಪಟ್ಟಿದ್ದ ರೇವಿಗೂ ಕರೆದುಕೊಂಡು ಹೋದರು.

ಸ್ವಾಮೀಜಿ ಲಂಡನ್ನಿನಿಂದ ಹೊರಟು ಅದಾಗಲೇ ಆರು ವಾರಗಳಾಗಿತ್ತು. ಆದ್ದರಿಂದ ಇನ್ನು ತಡಮಾಡದೆ ಕೀಲ್​ನಿಂದ ಹೊರಡಲು ಅವರು ನಿರ್ಧರಿಸಿದರು. ಆದರೆ ಪ್ರೊ ॥ ಡಾಯ್ಸನ್ನರು, ಅವರನ್ನು ಅಷ್ಟು ಬೇಗ ಬಿಟ್ಟುಕೊಡಲು ಸಿದ್ಧರಿರಲಿಲ್ಲ. ಇನ್ನು ಕೆಲವು ದಿನಗಳಾದರೂ ತಮ್ಮ ಅತಿಥಿಗಳಾಗಿ ಉಳಿದುಕೊಳ್ಳುವಂತೆ ಅವರನ್ನು ಡಾಯ್ಸನ್ನರು ಕೇಳಿಕೊಂಡರು. ಅವರೊಂದಿಗೆ ಹಲವಾರು ವಿಷಯಗಳ ಬಗ್ಗೆ ವಿವರವಾಗಿ ಚರ್ಚಿಸುವ ಇಚ್ಛೆ ಪ್ರೊಫೆಸರರಿಗಿತ್ತು. ಆದರೆ ತಾವು ತುರ್ತಾಗಿ ಲಂಡನ್ನಿಗೆ ಹಿಂದಿರುಗಿ, ತಮ್ಮ ಕಾರ್ಯವನ್ನು ಮುಂದುವರಿಸಬೇಕಾಗಿದೆ; ಮತ್ತು ಭಾರತಕ್ಕೆ ಹಿಂದಿರುಗುವ ಮುನ್ನ ಅಲ್ಲೊಂದು ಭದ್ರ ಬುನಾದಿಯನ್ನು ಹಾಕಬೇಕಾದ ಆವಶ್ಯಕತೆ ಇದೆ ಎಂದು ಸ್ವಾಮೀಜಿ ವಿನಂತಿಸಿಕೊಂಡರು. ಆಗ ಡಾಯ್ಸನ್ನರು, “ಹಾಗಾದರೆ ಸರಿ ಸ್ವಾಮೀಜಿ; ನಾನು ನಿಮ್ಮನ್ನು ಹ್ಯಾಂಬರ್ಗಿನಲ್ಲಿ ಮತ್ತೆ ಭೇಟಿಯಾಗುತ್ತೇನೆ. ಅಲ್ಲಿಂದ ನಾವು ಒಟ್ಟಿಗೆ ಲಂಡನ್ನಿಗೆ ಹೋಗೋಣ. ಅಲ್ಲಿ ನಾನು ನಿಮ್ಮೊಂದಿಗೆ ಬೇಕೆಂಬಷ್ಟು ಮಾತನಾಡಲು ಸಾಧ್ಯವಾಗು ತ್ತದೆ” ಎಂದು ಹೇಳಿ ಅವರನ್ನು ಬೀಳ್ಗೊಂಡರು.

ಸ್ವಾಮೀಜಿ ಹಾಗೂ ಸೇವಿಯರ್ ದಂಪತಿಗಳು ಕೀಲ್​ನಿಂದ ಹೊರಟು ಹ್ಯಾಂಬರ್ಗ್ ನಗರವನ್ನು ಸೇರಿದರು. ಅಲ್ಲೂ ಅವರು ಅನೇಕ ಪ್ರೇಕ್ಷಣೀಯ ಸ್ಥಳಗಳಿಗೆ ಭೇಟಿಯಿತ್ತರು. ಅವುಗಳಲ್ಲಿ ಮುಖ್ಯವಾದುದೆಂದರೆ ಅಲ್ಲಿನ ಪ್ರಸಿದ್ಧ ಪ್ರಾಣಿವನ. ಮೂರು ದಿನಗಳ ಬಳಿಕ ಹ್ಯಾಂಬರ್ಗಿನಲ್ಲಿ ಪ್ರೊ ॥ ಡಾಯ್ಸನ್ ಈ ಮೂವರನ್ನು ಕೂಡಿಕೊಂಡರು. ಶ್ರೀಮತಿ ಡಾಯ್ಸನ್ನ ರಿಗೂ ಮತ್ತೆ ಸ್ವಾಮೀಜಿಯನ್ನು ಭೇಟಿಯಾಗುವ ತೀವ್ರ ಹಂಬಲವಿತ್ತಾದರೂ ಅವರಿಗೆ ಬರಲು ಸಾಧ್ಯವಾಗಿರಲಿಲ್ಲ. ಹ್ಯಾಂಬರ್ಗಿನಿಂದ ಎಲ್ಲರೂ ಹಾಲೆಂಡಿನ ಪ್ರಮುಖ ನಗರವಾದ ಆಮ್ಸ್​್ಯಟರ್ ಡ್ಯಾಮಿಗೆ ಪ್ರಯಾಣ ಮಾಡಿದರು. ಇಲ್ಲಿ ಅವರು ಮತ್ತೆ ಮೂರು ದಿನಗಳ ಕಾಲ ಉಳಿದುಕೊಂಡು ಇಲ್ಲಿನ ಕಲಾಶಾಲೆಗಳನ್ನೂ ವಸ್ತುಸಂಗ್ರಹಾಲಯಗಳನ್ನೂ ವೀಕ್ಷಿಸಿದರು. ಹಾಲೆಂಡಿನ ಕೆಲವೆಡೆ ಗಳಲ್ಲಿ ಕಾಲುವೆಗಳನ್ನೇ ರಸ್ತೆಗಳಂತೆ ನಿರ್ಮಿಸಿಕೊಂಡು ಉಪಯೋಗಿಸುವ ರೀತಿ ಸ್ವಾಮೀಜಿ ಯನ್ನು ಬೆರಗುಗೊಳಿಸಿತು.

ಸೆಪ್ಟೆಂಬರ್ ೧೫ರಂದು ಸ್ವಾಮೀಜಿ ಮತ್ತು ಅವರ ಸಂಗಡಿಗರು ಹಾಲೆಂಡಿನಿಂದ ಹೊರಟರು. ಅಲ್ಲಿಂದ ಸಮುದ್ರ ಮಾರ್ಗವಾಗಿ ಪಯಣಿಸಿ, ಇಂಗ್ಲೆಂಡಿನ ಹಾರ್​ವಿಚ್ ಎಂಬ ಸ್ಥಳವನ್ನು ತಲುಪಿದರು. ಬಳಿಕ ಇಲ್ಲಿಂದ ಅವರು ನೇರವಾಗಿ ಲಂಡನ್ನಿಗೆ ಬಂದು ಸೇರಿದರು.

