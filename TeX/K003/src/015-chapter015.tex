
\chapter{ಸುಂಟರಗಾಳಿ ಸಂನ್ಯಾಸಿ}

\noindent

ಉಪನ್ಯಾಸ ಸಂಸ್ಥೆಯೊಂದಿಗೆ ಮಾಡಿಕೊಂಡಿದ್ದ ಒಪ್ಪಂದದ ಪ್ರಕಾರ ಸ್ವಾಮೀಜಿ ೧೮೯೩ರ ನವೆಂಬರ್ ೩ಂರಂದು ಶಿಕಾಗೋವನ್ನು ಬಿಟ್ಟು ಅಮೆರಿಕದ ಮಧ್ಯಪಶ್ಚಿಮ ಹಾಗೂ ದಕ್ಷಿಣ ಪ್ರಾಂತ್ಯಗಳ ನಗರಗಳಿಗೆ ಹೊರಟರು. ಈ ಭಾಷಣ ಪ್ರವಾಸದ ಕಾರ್ಯ ಸಾಕಷ್ಟು ಪ್ರಯಾಸಕರ ವಾದದ್ದಾಗಿತ್ತು. ಅಲ್ಲದೆ ಆಗತಾನೆ ಆ ಸ್ಥಳಗಳಲ್ಲಿ ಚಳಿಗಾಲ ಪ್ರಾರಂಭವಾಗುತ್ತಿತ್ತು. ಅವರು ತಲುಪಿದ ಮೊದಲ ನಗರ, ವಿಸ್​ಕಾನ್​ಸಿನ್ ರಾಜ್ಯದ ರಾಜಧಾನಿಯಾದ ಮ್ಯಾಡಿಸನ್. ಇಲ್ಲಿ ಅವರು ಯೂನಿಟೇರಿಯನ್ ಚರ್ಚ್​ನಲ್ಲಿ ಹಿಂದೂಧರ್ಮದ ಕುರಿತಾಗಿ ಭಾಷಣ ಮಾಡಿದರು. ಮರುದಿನವೇ ಅವರು ಮಿನೆಸೋಟ ರಾಜ್ಯದ ಮಿನಿಯಾಪೊಲಿಸ್ ನಗರವನ್ನು ತಲುಪಿದರು. ಸ್ವಾಮೀಜಿ ಅಲ್ಲಿಗೆ ಹೋದ ದಿನವೇ ಆ ವರ್ಷದ ಮೊದಲ ಹಿಮಪಾತವಾಯಿತು. ಕೊರೆಯುವ ಚಳಿಯಿತ್ತಾದರೂ, ಅದನ್ನು ಸಹಿಸಿಕೊಳ್ಳಲು ಸಾಧ್ಯವಾದರೆ ಚಳಿಗಾಲವನ್ನು ಚೆನ್ನಾಗಿಯೇ ಆನಂದಿಸಬಹುದಾಗಿತ್ತು. ಸ್ವಾಮೀಜಿಗಂತೂ ಚಳಿಗಾಲದ ಅದ್ಭುತ ದೃಶ್ಯಗಳನ್ನು ಕಂಡು ಬಹಳ ಖುಷಿಯಾಯಿತು. ಈ ಕುರಿತಾಗಿ ಅವರು ಶ್ರೀಮತಿ ಬೆಲ್ ಹೇಲ್​ಳಿಗೆ ಬರೆದರು:

“ನಾನಿಲ್ಲಿಗೆ ಕಾಲಿಟ್ಟ ದಿನವೇ ಈ ವರ್ಷದ ಮೊದಲ ಹಿಮಪಾತವಾಯಿತು. ಅಂದು ಹಗಲು- ರಾತ್ರಿ ಹಿಮ ಬೀಳುತ್ತಿತ್ತು. ನನ್ನ ಚಳಿಗಾಲದ ಬೂಟ್ಸ್ ತುಂಬ ಉಪಯೋಗಕ್ಕೆ ಬಂತು. ನಾನು ಗೆಡ್ಡೆಕಟ್ಟಿದ ಮನಿಹಹ ಜಲಪಾತವನ್ನು ನೋಡಲು ಹೋಗಿದ್ದೆ. ಅದು ತುಂಬ ಸುಂದರವಾಗಿದೆ. ಇಲ್ಲಿ ಇಂದಿನ ಉಷ್ಣತೆ ಸೊನ್ನೆಗಿಂತಲೂ ೨೧ ಡಿಗ್ರಿ ಕಡಿಮೆ. ಆದರೂ ನಾನು ಒಳಗೆ ಕುಳಿತು ಕೊಳ್ಳದೆ, ಹಿಮದ ಮೇಲೆ ಗಾಡಿಯಲ್ಲಿ ಕುಳಿತು ಜಾರುವ ಆಟ (ಸ್ಲೇ ಆಟ) ಆಡಿ ಬಹಳ ಸಂತೋಷ ಪಟ್ಟೆ. ನನ್ನ ಕಿವಿಗಳ ಇಲ್ಲವೆ ಮೂಗಿನ ತುದಿಯನ್ನು ಕಳೆದುಕೊಳ್ಳಬಹುದೆಂದು ನನಗೆ ಸ್ವಲ್ಪವೂ ಭಯವಿಲ್ಲ.

“ಈ ದೇಶದ ಇತರ ಎಲ್ಲ ದೃಶ್ಯಗಳಿಗಿಂತಲೂ ಈ ಹಿಮದ ದೃಶ್ಯ ನನ್ನನ್ನು ವಿಶೇಷವಾಗಿ ಸಂತೋಷಗೊಳಿಸಿದೆ. ನಿನ್ನೆ ಇಲ್ಲಿನ ಹೆಪ್ಪುಗಟ್ಟಿದ ಸರೋವರದ ಮೇಲೆ ಕೆಲವರು ಸ್ಕೇಟಿಂಗ್ ಆಡುವುದನ್ನು ಕಂಡೆ... ”

ಮಿನಿಯಾಪೊಲಿಸ್ ನಗರದಲ್ಲಿ ಸ್ವಾಮೀಜಿ ಎರಡು ಉಪನ್ಯಾಸಗಳನ್ನು ಮಾಡಿದರು. ಒಮ್ಮೆ ಹಿಂದೂಧರ್ಮದ ಬಗೆಗೂ ಮತ್ತೊಮ್ಮೆ ಪಾಶ್ಚಾತ್ಯರ ಹಾಗೂ ಪೌರ್ವಾತ್ಯರ ಧಾರ್ಮಿಕ ದೃಷ್ಟಿ ಕೋನಗಳ ಬಗೆಗೂ ಮಾತನಾಡಿದರು. ಈ ಎರಡನೆಯ ಉಪನ್ಯಾಸದಲ್ಲಿ ಸ್ವಾಮೀಜಿ ಹೇಳಿದರು:

“ಕ್ರೈಸ್ತರು ಅನುಸರಿಸುತ್ತಿರುವುದು ‘ವ್ಯವಹಾರದ ಧರ್ಮ’ವನ್ನು. ಅವರು ಯಾವಾಗಲೂ ‘ಓ ಭಗವಂತ! ಅದು ಕೊಡು, ಇದು ಕೊಡು’ ಎಂದು ಬೇಡುತ್ತಾರೆ. ಆದರೆ ಹಿಂದೂವಿಗೆ ಇದು ಅರ್ಥವಾಗುವುದಿಲ್ಲ ಅವನ ದೃಷ್ಟಿಯಲ್ಲಿ ಭಗವಂತನನ್ನು ಬೇಡುವುದು ತಪ್ಪು. ಧಾರ್ಮಿಕ ವ್ಯಕ್ತಿಯಾದವನು ಬೇಡಬಾರದು, ಕೊಡಬೇಕು. ಆದ್ದರಿಂದ ಹಿಂದುವು ಭಗವಂತನಿಗೂ ಅವನ ಭಕ್ತರಿಗೂ ಏನನ್ನಾದರೂ ಕೊಡುತ್ತಾನೆ. ಪಾಶ್ಚಾತ್ಯನು ಹಣಕ್ಕಾಗಿ ವಾರವಿಡೀ ದುಡಿಯುತ್ತಾನೆ. ಅದು ದೊರಕಿದ ಮೇಲೆ ‘ಭಗವಂತ, ಇದಕ್ಕಾಗಿ ನಿನಗೆ ಕೃತಜ್ಞತೆಗಳು’ ಎಂದು ಹೇಳುತ್ತಾನೆ; ಬಳಿಕ ಹಣವನ್ನು ಜೇಬಿಗಿಳಿಸುತ್ತಾನೆ. ಆದರೆ ಹಿಂದುವು ಹಣವನ್ನು ಗಳಿಸಿದಮೇಲೆ ಬಡವರಿಗೆ ನೆರವಾಗು ವುದರ ಮೂಲಕ ಭಗವಂತನಿಗೆ ಹಿಂದಿರುಗಿಸುತ್ತಾನೆ... ನಿಮ್ಮಲ್ಲಿ ಭಗವಂತನ ಪೂಜೆಗಾಗಿ ಅರಮನೆಯಂತಹ ಕಟ್ಟಡಗಳಿವೆ. ವಾರಕ್ಕೊಮ್ಮೆ ನೀವು ಅಲ್ಲಿಗೆ ಹೋಗುತ್ತೀರಿ. ಆದರೆ ನಿಜಕ್ಕೂ ಭಗವಂತನನ್ನೇ ಪೂಜಿಸಲು ಹೋಗುವವರೆಷ್ಟು ಜನ? ಪಾಶ್ಚಾತ್ಯರಲ್ಲಿ ಚರ್ಚಿಗೆ ಹೋಗುವು ದೊಂದು ‘ಫ್ಯಾಷನ್​’. ಇಂತಹ ನಿಮಗೆ ‘ಭಗವಂತನು ನಮಗೆ ಮಾತ್ರ ಸೇರಿದವನು’ ಎಂದು ಹೇಳಲು ಯಾವ ಹಕ್ಕಿದೆ?”

ಹೀಗೆ ಸ್ವಾಮೀಜಿ ನೇರವಾಗಿ ಕನ್ನಡಿ ಹಿಡಿದಾಗ, ಸಭೆ ಕರತಾಡನದಿಂದ ತುಂಬಿ ಹೋಯಿತು. ಬಳಿಕ ಅವರು ಮತ್ತೆ ಮುಂದುವರಿಸಿದರು:

“ನೀವು ಪಾಶ್ಚಾತ್ಯರು ಅನ್ವೇಷಣೆಗಳಲ್ಲಿ, ವ್ಯಾಪಾರ ವ್ಯವಹಾರದಲ್ಲಿ ಕಾರ್ಯಶೀಲರು. ನಾವು ಧರ್ಮದಲ್ಲಿ ಕಾರ್ಯಶೀಲರು. ನೀವು ಭಾರತಕ್ಕೆ ಬಂದು ಹೊಲದಲ್ಲಿ ಕೆಲಸ ಮಾಡುವ ರೈತನನ್ನು ಮಾತನಾಡಿಸಿ ನೋಡಿ. ಅವನಿಗೆ ರಾಜಕಾರಣದ ಬಗ್ಗೆ ಏನೇನೂ ಗೊತ್ತಿರುವುದಿಲ್ಲ. ‘ಏನಪ್ಪ, ನಿಮ್ಮದ್ದು ಯಾವ ಸರ್ಕಾರ?’ ಎಂದರೆ ಅವನೆನ್ನುತ್ತಾನೆ, ‘ಅದೇನೋ ನನಗೆ ಗೊತ್ತಿಲ್ಲ. ನಾನು ತೆರಿಗೆ ಕಟ್ಟುತ್ತೇನೆ, ಅಷ್ಟೇ ನನಗೆ ಗೊತ್ತಿರುವುದು’ ಎಂದು. ಆದರೆ ಧರ್ಮದ ಬಗ್ಗೆ ಕೇಳಿದರೆ ಎಂಥವನೂ ಉತ್ತರಿಸಬಲ್ಲ. ನಾನು ನಿಮ್ಮ ರೈತಾಪಿಗಳೊಂದಿಗೆ, ಕೂಲಿಗಳೊಂದಿಗೆ ಮಾತನಾಡಿ ದ್ದೇನೆ. ಅವರೆಲ್ಲ ಒಂದೊಂದು ರಾಜಕೀಯ ಪಕ್ಷವನ್ನು ಬೆಂಬಿಸುವವರು. ಆದರೆ ಧರ್ಮದ ಬಗ್ಗೆ ಕೇಳಿ, ಅದರಲ್ಲಿ ಭಾರತದ ರೈತನಂತೆ ಮುಗ್ಧರು. ಅವರು ತಮ್ಮದೇ ಆದ ಚರ್ಚಿಗೆ ಹೋಗುತ್ತಾರೆ. ಆದರೆ ಅದು ಏನನ್ನು ಹೇಳುತ್ತದೆಂಬುದು ಮಾತ್ರ ಅವರಿಗೆ ಗೊತ್ತಿಲ್ಲ.”

ನವೆಂಬರ್ ೨೬ರಂದು ಸ್ವಾಮೀಜಿ ಮಿನಿಯಾಪೊಲಿಸ್​ನಿಂದ ಹೊರಟು, ೨೬ಂ ಮೈಲಿ ದಕ್ಷಿಣ ಕ್ಕಿರುವ ಅಯೋವಾದ ರಾಜಧಾನಿ ಡೆಸ್ ಮೊಯಿನ್ಸ್​ಗೆ ಬಂದರು. ಇಲ್ಲಿನ ಜನತೆಯಿಂದಲೂ ಪ್ರತಿಕೆಯಿಂದಲೂ ಅವರಿಗೆ ಆದರದ ಸ್ವಾಗತ ದೊರಕಿತು. ಅವರ ಹೆಸರು ಆಗಲೇ ಪ್ರಸಿದ್ಧವಾಗಿದ್ದ ರಿಂದ ಭಾಷಣಕ್ಕಾಗಿ ಟಿಕೆಟ್ಟುಗಳೆಲ್ಲ ಮುಂದಾಗಿಯೇ ಖರ್ಚಾಗಿದ್ದುವು. ಸ್ವಾಮೀಜಿ ತಮ್ಮ ಸ್ನೇಹ ಪೂರ್ಣ ಮಾತುಕತೆಯಿಂದ ಮತ್ತಷ್ಟು ಜನಪ್ರಿಯರಾದರು. ಆದರೆ ಅಗತ್ಯವೆನಿಸಿದಾಗ ತೀಕ್ಷ್ಣವಾದ ಮಾತುಗಳಿಂದ ಕುಹಕಿಗಳಿಗೆ ಇಬ್ಬಾಯ ಖಡ್ಗದಂತೆ ತೋರಿದರು. ಡೆಸ್ ಮೊಯಿನ್ಸ್​ನಲ್ಲಿ ಸ್ವಾಮೀಜಿ ನೀಡಿದ ಉಪನ್ಯಾಸಗಳು ಹಾಗೂ ಅವರು ಗಳಿಸಿದ ಜನಾನುರಾಗದ ಬಗ್ಗೆ \eng{\textit{Iowa State Register}} ಪತ್ರಿಕೆ ಹೀಗೆ ವರದಿ ಮಾಡಿತು:

“ಹಿಂದೂ ಸಂನ್ಯಾಸಿಗಳಾದ ಸ್ವಾಮಿ ವಿವೇಕಾನಂದರು ಡೆಸ್ ಮೊಯಿನ್ಸ್​ನಲ್ಲಿ ಮೂರು ಉಪನ್ಯಾಸಗಳನ್ನು ಮಾಡಿದರು. ಅವರು ಇಲ್ಲಿಂದ ಪಶ್ಚಿಮ ಅಮೆರಿಕೆಗೆ ಹೋಗುವ ಕಾರ್ಯಕ್ರಮ ಗಳಿದ್ದರೂ ತತ್ಕಾಲಕ್ಕೆ ಅವು ರದ್ದಾದದ್ದರಿಂದ ಅವರು ಇಲ್ಲಿ ಇನ್ನಷ್ಟು ಕಾಲ ಉಳಿದುಕೊಳ್ಳು ವಂತಾಯಿತು. ಇಲ್ಲಿನ ವಿದ್ಯಾವಂತ ಜನ ಅವರೊಂದಿಗೆ ಧಾರ್ಮಿಕ ಹಾಗೂ ಆಧ್ಯಾತ್ಮಿಕ ವಿಷಯ ಗಳ ಬಗ್ಗೆ ಚರ್ಚಿಸುತ್ತ, ಈ ಸಮಯವನ್ನು ಸದುಪಯೋಗಪಡಿಸಿಕೊಂಡರು. ಆದರೆ ಯಾವನಾ ದರೂ ಅವರ ವಾದಸರಣಿಯನ್ನೇ ಉಪಯೋಗಿಸಿಕೊಂಡು ಅವರನ್ನು ಸೋಲಿಸಲು ಪ್ರಯತ್ನ ಮಾಡಿದರೆ, ಅವನು ಬಹಳ ಕಷ್ಟಕ್ಕೆ ಗುರಿಯಾಗಬೇಕಾಗಿತ್ತು. ಅವರ ಉತ್ತರಗಳು ಮಿಂಚಿನ ಝಳಪಿನಂತೆ ಹೊಮ್ಮುತ್ತಿದ್ದು, ಆ ಭಾರತೀಯ ಬುದ್ಧಿಮತ್ತೆಯ ಪ್ರಖರ ಕಿರಣಗಳ ಮುಂದೆ ಎಂತಹ ಧೈರ್ಯಶಾಲಿಯಾದ ಪ್ರಾಶ್ನಿಕನೂ ನಿಸ್ತೇಜನಾಗಲೇಬೇಕಾಗಿತ್ತು.

“ಅವರ ಮನಸ್ಸು ಕೆಲಸ ಮಾಡುವ ರೀತಿ ತುಂಬ ಸೂಕ್ಷ್ಮ, ತೇಜಸ್ವಿ; ಅವರ ಆಲೋಚನೆಗಳು ಸುವ್ಯವಸ್ಥಿತ, ಸುಶಿಕ್ಷಿತ. ಇವುಗಳನ್ನು ಕಂಡವರು ಕೆಲವೊಮ್ಮೆ ಅವಾಕ್ಕಾಗುತ್ತಿದ್ದರು. ಆದರೆ ಅವರ ಆಲೋಚನೆಗಳನ್ನು ಗಮನಿಸುವುದೇ ಒಂದು ಸ್ವಾರಸ್ಯಕರ ಅಧ್ಯಯನ. ಅವರೆಂದೂ ಅಹಿತಕರ ಮಾತನ್ನಾಡುವವರಲ್ಲ. ಏಕೆಂದರೆ ಅದು ಅವರ ಸ್ವಭಾವಕ್ಕೇ ವಿರುದ್ಧವಾದದ್ದು. ಅವರ ನಿಕಟ ಪರಿಚಯ ಮಾಡಿಕೊಂಡವರಿಗೆ ವಿವೇಕಾನಂದರು ಮನುಷ್ಯರಲ್ಲೆಲ್ಲ ಅತ್ಯಂತ ಸಭ್ಯ ಹಾಗೂ ಪ್ರಿಯ ವ್ಯಕ್ತಿಯಾಗಿ, ಪ್ರಾಮಾಣಿಕ ವ್ಯಕ್ತಿಯಾಗಿ ಕಂಡುಬಂದರು. ಅವರು ಬಿಚ್ಚುಮನಸ್ಸಿನಿಂದ ಕೂಡಿದ್ದು ಯಾವುದೇ ಬಗೆಯ ಸೋಗು ಅವರಲ್ಲಿಇಲ್ಲದಿರುವುದು ಎದ್ದುಕಾಣುತ್ತಿತ್ತು. ತಮಗೆ ಸೌಹಾರ್ದವನ್ನು ತೋರಿದವರಿಗೆಲ್ಲ ಅವರು ಕೃತಜ್ಞರಾಗಿದ್ದರು. ವಿವೇಕಾನಂದರಿಗೂ ಅವರ ಕಾರ್ಯಯೋಜನೆಗೂ ಇಲ್ಲಿನ ನಿಜವಾದ ಕ್ರೈಸ್ತರೆಲ್ಲರ ಹೃದಯದಲ್ಲಿ ಒಂದು ಸ್ಥಾನ ಸಿಕ್ಕಿತು.”

ಸ್ವಾಮೀಜಿಯ ಭಾಷಣಪ್ರವಾಸ ನಿರಂತರವಾಗಿ ನಡೆದುಬಂತು. ಉಪನ್ಯಾಸಗಳನ್ನು ನೀಡುತ್ತ ಅವರು ಅವಿಶ್ರಾಂತವಾಗಿ ಸಂಚರಿಸಿದರು. ಈ ಬಗ್ಗೆ ಅವರು ತಮ್ಮ ಸೋದರಸಂನ್ಯಾಸಿಗಳಿಗೆ ಬರೆದ ಪತ್ರದಲ್ಲಿ ಹೇಳಿದರು, “ಆವಶ್ಯಕತೆ ಎಂಬುದು ಒಂದು ದಿನ ನನ್ನನ್ನು ಕೆನಡಾದ ಗಡಿಯ ವರೆಗೆ ಹೋಗುವಂತೆ ಮಾಡಿದರೆ, ಮರುದಿನವೇ ಅಮೆರಿಕದ ದಕ್ಷಿಣತುದಿಯಲ್ಲಿ ಭಾಷಣಕೊಡು ವಂತೆ ಮಾಡುತ್ತದೆ.” ಸರ್ವಧರ್ಮ ಸಮ್ಮೇಳನದಲ್ಲಿ ಅವರ ಭಾಷಣವನ್ನು ಕೇಳಿದ್ದವರೆಲ್ಲ ಈಗ ಅವರನ್ನು ತಮ್ಮತಮ್ಮ ಊರುಗಳಿಗೆ ಆಹ್ವಾನಿಸಿ ಭಾಷಣ ಮಾಡುವಂತೆ ಕೇಳಿಕೊಳ್ಳುತ್ತಿದ್ದರು. ಕೆಲವು ನಗರಗಳಲ್ಲಿ ಅವರು ಒಂದು ವಾರಕ್ಕಿಂತಲೂ ಹೆಚ್ಚು ಕಾಲ ಉಳಿದುಕೊಂಡು ಹಲವಾರು ಉಪನ್ಯಾಸಗಳನ್ನು ನೀಡುತ್ತಿದ್ದರು. ಅಲ್ಲದೆ ಹಲವಾರು ಪತ್ರಕರ್ತರು ಅವರೊಂದಿಗೆ ದೀರ್ಘ ಸಂದರ್ಶನಗಳನ್ನು ನಡೆಸುತ್ತಿದ್ದರು.

ಉಪನ್ಯಾಸಕ್ಕಾಗಿ ಸ್ವಾಮೀಜಿ ಆರಿಸಿಕೊಳ್ಳುತ್ತಿದ್ದ ವಿಷಯಗಳಲ್ಲಿ ಮುಖ್ಯವಾದುವೆಂದರೆ ಹಿಂದೂಧರ್ಮ, ಭಾರತದ ರೀತಿನೀತಿಗಳು, ಭಾರತದ ಸ್ಥಿತಿಗತಿ ಹಾಗೂ ಅದರ ಆವಶ್ಯಕತೆಗಳು– ಇವುಗಳನ್ನು ಕುರಿತಾದವು. ಅಮೆರಿಕನ್ನರಿಗೆ ಈ ವಿಷಯಗಳ ಬಗ್ಗೆ ತೀವ್ರ ಕುತೂಹಲವಿತ್ತು. ಅಲ್ಲದೆ ಕ್ರೈಸ್ತ ಮಿಷನರಿಗಳ ಅಪಪ್ರಚಾರದಿಂದಾಗಿ ಎಲ್ಲೆಲ್ಲೂ ಭಾರತದ ಹಾಗೂ ಹಿಂದೂಧರ್ಮದ ಬಗ್ಗೆ ವಿಕೃತ ನಂಬಿಕೆಗಳು ನೆಲೆಯೂರಿದ್ದುವು. ಈ ಬಗೆಯ ಪೂರ್ವಾಗ್ರಹಗಳಿಂದ ಕೂಡಿದ್ದ ಜನರನ್ನುದ್ದೇಶಿಸಿ ಭಾಷಣ ಮಾಡುವುದೆಂದರೆ ಅದೊಂದು ಸಾಹಸವೇ ಆಗಿತ್ತು. ಯಾವಾಗಲೂ ಸ್ವಾಮೀಜಿ ಸಕಲ ಧರ್ಮಗಳ ಸತ್ಯತೆಯನ್ನು ಸಾರುತ್ತಿದ್ದರು. ಮಾನವನಲ್ಲಿ ಮೂಲಭೂತವಾಗಿ ಅಡಗಿರುವ ದೈವತ್ವದ ಅಂಶವನ್ನು ಅವರು ಎತ್ತಿ ಹಿಡಿಯುತ್ತಿದ್ದರು. ಹಿಂದೂಧರ್ಮದ ಶ್ರೇಷ್ಠತೆ ಯನ್ನು ತೋರಿಸಿಕೊಡುವುದಲ್ಲದೆ, ತಮ್ಮ ಗುರುದೇವನಾದ ಶ್ರೀರಾಮಕೃಷ್ಣರ ಸಂದೇಶಗಳನ್ನು ಬಿತ್ತಿದರು. ತಮ್ಮ ಕಾರ್ಯ ಇಂದಲ್ಲ ನಾಳೆ ಖಂಡಿತವಾಗಿಯೂ ಫಲಿಸುವುದೆಂಬ ವಿಶ್ವಾಸ ಅವರಲ್ಲಿತ್ತು. ತಮ್ಮ ಸೋದರಸಂನ್ಯಾಸಿಗಳಿಗೆ ಬರೆದ ಪತ್ರದಲ್ಲಿ ಅವರು ಹೇಳುತ್ತಾರೆ, “ಎಲ್ಲೆಲ್ಲಿ ಶ್ರೀರಾಮಕೃಷ್ಣರ ಶಕ್ತಿ ಬೀಜ ಬೀಳುತ್ತದೆಯೋ ಅಲ್ಲೆಲ್ಲ ಅದು ಮೊಳೆತು ಫಲ ಕೊಡಲೇ ಬೇಕು–ಅದು ಇಂದೇ ಆಗಬಹುದು ಅಥವಾ ನೂರು ವರ್ಷಗಳ ಬಳಿಕ ಆಗಬಹುದು.”

ಶ್ರೀರಾಮಕೃಷ್ಣರ ಬಗ್ಗೆ ಸ್ವಾಮೀಜಿ ಪ್ರಸ್ತಾಪಿಸುತ್ತಿದ್ದರಾದರೂ ಅವರ ಬಗ್ಗೆ ಉಪನ್ಯಾಸಗಳನ್ನು ಮಾಡಿದ್ದು ತೀರ ಅಪರೂಪ. ಶ್ರೀರಾಮಕೃಷ್ಣರ ಬೋಧನೆಗಳನ್ನಲ್ಲದೆ ಅವರು ಬೇರೇನನ್ನೂ ಬೋಧಿಸಲಿಲ್ಲ. ಆದರೂ ಶ್ರೀರಾಮಕೃಷ್ಣರ ವ್ಯಕ್ತಿತ್ವ-ಜೀವನಗಳನ್ನು ಅವರೇಕೆ ಸಾರಲಿಲ್ಲ ಎಂಬ ಪ್ರಶ್ನೆಯೇಳಬಹುದು. ಸ್ವಾಮೀಜಿ ಯಾವಾಗಲೂ ವ್ಯಕ್ತಿಗಿಂತ ತತ್ತ್ವಕ್ಕೆ ಪ್ರಾಧಾನ್ಯವನ್ನು ನೀಡು ತ್ತಿದ್ದರು. ತತ್ತ್ವದೊಂದಿಗೆ ವ್ಯಕ್ತಿಯ ಬಗೆಗೂ ಹೇಳಿದ್ದರೆ ತಪ್ಪೇನೂ ಆಗುತ್ತಿರಲಿಲ್ಲ. ಆದರೆ ಭಾರತೀಯ ಚಿಂತನೆ ಹಾಗೂ ಹಿಂದೂಧರ್ಮದ ಮಹಾತತ್ತ್ವ-ಆದರ್ಶಗಳು ತೀರ ಅಪರಿಚಿತ ವಾಗಿದ್ದ ಅಮೆರಿಕನರ ಮುಂದೆ ಶ್ರೀರಾಮಕೃಷ್ಣರ ವಿಚಾರವನ್ನು ತಿಳಿಸಲು ಕಾಲ ಒದಗಿಬಂದಿರ ಲಿಲ್ಲ. ಮೊದಲು ಈ ತತ್ತ್ವಗಳು ನೆಲೆಗೊಂಡಮೇಲೆ ಅವುಗಳನ್ನು ತನ್ನ ಜೀವನದಲ್ಲಿ ಆಚರಿಸಿ ತೋರಿಸಿದ ವ್ಯಕ್ತಿಯ ವಿಷಯವನ್ನು ಹೇಳುವುದೇ ಸೂಕ್ತ. ಇದಲ್ಲದೆ ಅವರು ಶ್ರೀರಾಮಕೃಷ್ಣರ ಬಗ್ಗೆ ಹೆಚ್ಚಾಗಿ ಹೇಳದಿದ್ದುದಕ್ಕೆ ಇನ್ನೊಂದು ಕಾರಣವಿತ್ತು–ಶ್ರೀರಾಮಕೃಷ್ಣರ ಬಗ್ಗೆ ಮಾತನಾಡು ವಷ್ಟು ಯೋಗ್ಯತೆ ತಮಗಿಲ್ಲವೆಂದು ಅವರು ಪ್ರಾಮಾಣಿಕವಾಗಿ ನಂಬಿದ್ದರು! ಇನ್ನು ಕೆಲವೊಮ್ಮೆ ತಮ್ಮ ಗುರುದೇವನ ಬಗ್ಗೆ ಮಾತನಾಡಲು ಹೊರಟರೆ ಅವರ ಸ್ವರ ಗದ್ಗದವಾಗಿ ಶಬ್ದಗಳೇ ಹೊರಡುತ್ತಿರಲಿಲ್ಲ!

ಸ್ವಾಮೀಜಿಗೆ ತಮ್ಮ ಇಡೀ ವ್ಯಕ್ತಿತ್ವದ ಒಳಗೂ ಹೊರಗೂ ಯಾವುದೋ ಒಂದು ಅದ್ಭುತ ಶಕ್ತಿ ವ್ಯಾಪಿಸಿಕೊಂಡು ಕೆಲಸ ಮಾಡುತ್ತಿರುವುದರ ಅರಿವಾಗುತ್ತಿತ್ತು. ಈ ವಿಷಯವಾಗಿ ಅವರು ತಮ್ಮ ಸೋದರಸಂನ್ಯಾಸಿಗಳಿಗೆ ಬರೆಯುತ್ತಾರೆ, “ಭಗವಂತನ ಕೃಪೆಯನ್ನು ಕಂಡು ನಾನು ಆಶ್ಚರ್ಯಭರಿತನಾಗಿದ್ದೇನೆ. ನಾನು ಯಾವುದೇ ನಗರಕ್ಕೆ ಹೋಗಲಿ, ಅಲ್ಲೊಂದು ಉತ್ಸಾಹದ ತರಂಗವೇ ಎದ್ದುಬಿಡುತ್ತದೆ. ಅವರೆಲ್ಲ ನನಗೆ ‘ಸುಂಟರಗಾಳೀ ಸಂನ್ಯಾಸಿ’ ಎಂದು ಹೆಸರಿಟ್ಟಿ ದ್ದಾರೆ. ನೆನಪಿಟ್ಟುಕೊಳ್ಳಿ–ಇದೆಲ್ಲ ಅವನ ಇಚ್ಛೆ. ನಾನೊಂದು ಆಕಾರವಿಲ್ಲದ ಧ್ವನಿ ಮಾತ್ರ.”

ಸ್ವಾಮೀಜಿ ಭಾಷಣ ಪ್ರವಾಸ ಮಾಡುತ್ತ ಹೋದೆಡೆಯಲ್ಲೆಲ್ಲ ಯಾರಾದರೊಬ್ಬರ ಅತಿಥಿ ಯಾಗಿ ಉಳಿದುಕೊಳ್ಳುತ್ತಿದ್ದರು. ಎಷ್ಟೋ ಸ್ಥಳಗಳಲ್ಲಿ ಆದರದ ಆತಿಥ್ಯ ದೊರಕುತ್ತಿತ್ತಾದರೂ ಈ ಪ್ರವಾಸದ ಕಾರ್ಯಕ್ರಮ ಸುಲಭದ್ದೇನೂ ಆಗಿರಲಿಲ್ಲ. ಮೂಳೆಗಳನ್ನು ನಡುಗಿಸುವಂತಹ ಕೊರೆಯುವ ಥಂಡಿ ಅವರಿಗೆ ಸಾಕಷ್ಟು ಬಾಧೆಕೊಟ್ಟಿತು. ದೀರ್ಘ ರೈಲುಪ್ರಯಾಣಗಳೂ ಬಹಳ ದಣಿವು ಉಂಟುಮಾಡುತ್ತಿದ್ದುವು. ಕೆಲವು ಸಲ ಸಣ್ಣಸಣ್ಣ ಊರುಗಳ ಕೆಳದರ್ಜೆಯ ಹೋಟೆಲು ಗಳಲ್ಲಿ ತಂಗಬೇಕಾಗುತ್ತಿತ್ತು. ಸಾಲದಕ್ಕೆ ನಿರಂತರವಾಗಿ ಒಂದು ಕಡೆ ಮುಗಿದತಕ್ಷಣ ಇನ್ನೊಂದು ಕಡೆ ಭಾಷಣ ಮಾಡಬೇಕಾಗಿತ್ತು. ಹೋದಹೋದಲ್ಲೆಲ್ಲ ಜನರ ಗಂಪು ಅವರನ್ನು ಮುತ್ತಿಕೊಳ್ಳು ತ್ತಿತ್ತು. ಪಾದ್ರಿಗಳು, ಅದರಲ್ಲೂ ಮುಖ್ಯವಾಗಿ ‘ಯೂನಿಟೇರಿಯನ್​’ ಪಂಗಡದವರು, ತಮ್ಮ ಚರ್ಚುಗಳ ವೇದಿಕೆಯ ಮೇಲಿಂದ ಮಾತನಾಡುವಂತೆ ಅವರನ್ನು ಕಾಡುತ್ತಿದ್ದರು. ಕೆಲವೆಡೆಗಳಲ್ಲಿ ಈ ಭಾಷಣಗಳಿಗೆ ಪ್ರತಿಕ್ರಿಯೆ ಬಹಳ ಆದರಪೂರ್ವಕವಾಗಿರುತ್ತಿತ್ತು. ಆದರೆ ಕೆಲವೊಮ್ಮೆ ಭಾರತದ ಬಗ್ಗೆ ತಪ್ಪು ತಿಳಿವಳಿಕೆಯನ್ನು ಹೊಂದಿದ್ದ ಸಭಿಕರಿಂದ ಹಲವಾರು ಬಗೆಯ ಕಿರಿಕಿರಿ ಪ್ರಶ್ನೆಗಳನ್ನು ಎದುರಿಸಬೇಕಾದ ಸಂದರ್ಭವೊದಗುತ್ತಿತ್ತು. ಮತ್ತೆ ಕೆಲವು ಸಲ ತಮಗೇ ಏನೂ ತಿಳಿದಿರದಿದ್ದ ವಿಷಯಗಳ ಬಗ್ಗೆ ಕೆಲವರು ವಾದ ಹೂಡುತ್ತಿದ್ದರು. ಇಂತಹ ಉದ್ಧಟತನವನ್ನು ಸ್ವಾಮೀಜಿ ಹರಿತವಾದ ವ್ಯಂಗ್ಯದಿಂದ ಖಂಡಿಸುತ್ತಿದ್ದರು. ಮತ್ತೆ ಕೆಲವರು ಭಾರತೀಯರ ಸಂಪ್ರದಾಯದ ಬಗ್ಗೆ ಅತ್ಯಂತ ವಿಚಿತ್ರವಾದ ಪ್ರಶ್ನೆಗಳನ್ನು ಕೇಳುತ್ತಿದ್ದರು. ಇಂತಹ ಪ್ರಶ್ನೆಗಳಲ್ಲಿ ಒಂದೆಂದರೆ, “ಸ್ವಾಮೀಜಿ, ಹಿಂದೂಗಳು ಮಕ್ಕಳನ್ನು ಮೊಸಳೆ ಬಾಯಿಗೆ ಕೊಡುತ್ತಾರಂತೆ ನಿಜವೆ?” ಎಂಬುದು. ಈ ಪ್ರಶ್ನೆಯನ್ನು ಅಮೆರಿಕದ ಎಷ್ಟು ಕಡೆಗಳಲ್ಲಿ ಎಷ್ಟು ಜನ ಕೇಳಿದರೋ ಲೆಕ್ಕವೇ ಇಲ್ಲ. ಆದರೆ ಪ್ರತಿಸಲವೂ ಸ್ವಾಮೀಜಿ ಅದಕ್ಕೆ ಬೇರೆ ಬೇರೆ ಉತ್ತರಗಳನ್ನು ಕೊಡುತ್ತಿ ದ್ದರು. ಮಿನಿಯಾಪೊಲಿಸ್​ನಲ್ಲಿ ಒಬ್ಬಳು ಮಹಿಳೆ ಅದೇ ಪ್ರಶ್ನೆಯನ್ನು ಕೇಳಿದಳು. ಆಗ ಸ್ವಾಮೀಜಿ ವ್ಯಂಗ್ಯ ಪರಿಹಾಸ್ಯದಿಂದ ಉತ್ತರಿಸಿದರು. “ಹೌದು ತಾಯಿ. ನನ್ನನ್ನೂ ನನ್ನ ಅಮ್ಮ ಮೊಸಳೆಯ ಬಾಯಿಗೆ ಎಸೆದುಬಿಟ್ಟಿದ್ದಳು. ಆದರೆ ನಾನು ಹೇಗೋ ಮಾಡಿ ತಪ್ಪಿಸಿಕೊಂಡು ಬಂದುಬಿಟ್ಟೆ, ಸದ್ಯ!”

ಬಹುಶಃ ಅವರು ಅತ್ಯಂತ ಕಠಿಣ ಪರಿಸ್ಥಿತಿಯನ್ನು ಎದುರಿಸಿದ್ದು ಮಧ್ಯಪಶ್ಚಿಮದ ಒಂದು ಸಣ್ಣ ಊರಿನಲ್ಲಿ. ಇಲ್ಲಿನ ಕೆಲ ಯುವಕರು ಸ್ವಾಮೀಜಿ ಭಾರತೀಯ ತತ್ತ್ವಶಾಸ್ತ್ರದ ಬಗ್ಗೆ ಮಾತನಾಡುವುದನ್ನು ಕೇಳಿದರು. ಈ ಯುವಕರು ವಿಶ್ವವಿದ್ಯಾನಿಲಯದ ಪದವೀಧರರಾದರೂ ಕುದುರೆಯ ಮೇಲೆ ಕುಳಿತು ಬೇಟೆಯಾಡುವ ವೃತ್ತಿಯನ್ನು ಕೈಗೊಂಡಿದ್ದವರು. (ಇವರನ್ನು\eng{ Cowboys}ಎನ್ನುತ್ತಾರೆ. ಇವರಿಗೆ ಇವರದೇ ಆದ ವಿಶಿಷ್ಟ ಒರಟು ಸಂಸ್ಕೃತಿಯಿದೆ.) ಸ್ವಾಮೀಜಿ ಭಾಷಣದ ಸಂದರ್ಭದಲ್ಲಿ ಹೇಳಿದರು–ಅತ್ಯುನ್ನತ ಸಾಕ್ಷಾತ್ಕಾರ ಮಾಡಿಕೊಂಡವನು ಎಲ್ಲ ಪರಿಸ್ಥಿತಿಗಳಲ್ಲೂ ಸ್ಥಿರಬುದ್ಧಿಯುಳ್ಳವನಾಗಿರುತ್ತಾನೆ ಎಂದು. ಇದನ್ನು ಕೇಳಿದ ಆ ಯುವಕರು, ಈ ಮಾತು ಇವರ ವಿಷಯದಲ್ಲಿ ಹೇಗೆ ಅನ್ವಯವಾಗುತ್ತದೆಯೋ ನೋಡೋಣ ಎಂದು ತಮ್ಮದೇ ಆದ ರೀತಿಯಲ್ಲಿ ಪರೀಕ್ಷೆ ಮಾಡಿ ನೋಡಲು ಸಿದ್ಧರಾದರು.

ಈ ಯುವಕರು ಬಂದು, ತಮಗಾಗಿ ಒಂದು ಭಾಷಣ ಮಾಡುವಂತೆ ಸ್ವಾಮೀಜಿಯನ್ನು ಕೇಳಿ ಕೊಂಡರು. ಅವರ ಉದ್ದೇಶವನ್ನರಿಯದೆ ಸ್ವಾಮೀಜಿ ಒಪ್ಪಿಕೊಂಡರು. ಆ ಊರಿನ ಸಾರ್ವಜನಿಕ ಚೌಕದಲ್ಲಿ ಒಂದು ವೇದಿಕೆಯನ್ನು ನಿರ್ಮಿಸಲಾಯಿತು. ವೇದಿಕೆ ಎಂದರೆ ಅದೊಂದು ಬೋರಾಲು ಹಾಕಿದ ಮರದ ದೊಡ್ಡ ಪೀಪಾಯಿ, ಅಷ್ಟೆ. ಅದರ ಮೇಲೆಯೇ ನಿಂತು ಸ್ವಾಮೀಜಿ ಭಾಷಣವನ್ನು ಆರಂಭಿಸಿದರು. ಎರಡೇ ನಿಮಿಷದಲ್ಲಿ ಅವರು ಭಾಷಣದಲ್ಲಿ ಸಂಪೂರ್ಣ ತಲ್ಲೀನರಾಗಿಬಿಟ್ಟರು. ಆಗ ಇದ್ದಕ್ಕಿದ್ದಂತೆ ಕಿವಿ ಕಿವುಡಾಗುವಂತೆ ಶಬ್ದವಾಯಿತು. ಅದು ತುಪಾಕಿಯಿಂದ ಗುಂಡುಗಳು ಸಿಡಿದ ಶಬ್ದ. ಈ ಯುವಕರೇ ದೂರದಲ್ಲಿ ನಿಂತು ಆ ಗುಂಡುಗಳನ್ನು ಹಾರಿಸಿದ್ದರು. ಇವು ಅವರ ಕಿವಿಗಳ ಪಕ್ಕದಲ್ಲೇ ಸುಂಯ್​ಗುಡತ್ತ ಹಾದುಹೋದುವು. ಆದರೆ ಸ್ವಾಮೀಜಿ ಮಾತ್ರ ಏನೂ ಆಗಿಲ್ಲವೋ ಎಂಬಂತೆ ತಮ್ಮ ಪಾಡಿಗೆ ತಾವು ಭಾಷಣ ಮಾಡುವುದರಲ್ಲಿ ಮಗ್ನರಾಗಿದ್ದರು! ಅವರು ಸಂಪೂರ್ಣ ಅಂತರ್ಮುಖಿಗಳಾಗಿದ್ದುಕೊಂಡು, ಅತ್ಯಂತ ಅಂತರಾಳದೊಳಗಿಂದ ಮಾತ ನಾಡುತ್ತಿದ್ದರು. ಆದ್ದರಿಂದ ಅವರಿಗೆ ತಾವಾಡುವ ವಿಷಯದ ಮೇಲೆ ಗಮನವಿತ್ತೇ ಹೊರತು ಸುತ್ತಲಿನ ಪರಿಸರದ ಪರಿಜ್ಞಾನವೇ ಇರಲಿಲ್ಲ. ಸ್ವಾಮೀಜಿ ಭಾಷಣ ಮುಗಿಸಿದಾಗ ಈ ಕೌಬಾಯ್ ಗಳು ಆಶ್ಚರ್ಯ-ಸಂತೋಷಗಳಿಂದ ಹುಚ್ಚೆದ್ದು ಬಂದು ಅವರನ್ನು ಮುತ್ತಿಕೊಂಡರು. ತಾವಿಟ್ಟ ಕಠಿಣ ಪರೀಕ್ಷೆಯಲ್ಲಿ ಅವರು ಉತ್ತೀರ್ಣರಾದದ್ದನ್ನು ಕಂಡು, ಇವರು ಗಟ್ಟಿಗರೇ ಸರಿ ಎಂದು ತೀರ್ಮಾನಿಸಿ ಅಭಿನಂದಿಸಿದರು. ಆದರೆ ಸ್ವಾಮೀಜಿ ಸ್ವಲ್ಪವೇ ವಿಚಲಿತರಾಗಿದ್ದರೂ ಅವರನ್ನು ಅಪಹಾಸ್ಯ ಮಾಡಿಬಿಡುತ್ತಿದ್ದರು ಎನ್ನುವುದು ಖಂಡಿತ.

ಈ ಭಾಷಣ ಪ್ರವಾಸದ ಸಮಯದಲ್ಲಿ ಅವರಿಗೆ ಹೊಸ ಹೊಸ ಅನುಭವಗಳಾಗುತ್ತಿದ್ದವು. ದೀರ್ಘ ರೈಲು ಪ್ರಯಾಣದ ಅವಧಿಯಲ್ಲಿ ಕೆಲವೊಮ್ಮೆ ವಿಚಿತ್ರ ರೀತಿಯ ಜನಗಳ ಭೇಟಿಯಾಗು ತ್ತಿತ್ತು. ಉಪನ್ಯಾಸದ ಸಮಯದಲ್ಲಿ ಇಂಥವರನ್ನೆಲ್ಲ ಕಾಣಲು ಸಾಧ್ಯವಿರಲಿಲ್ಲ. ಅವರೊಂದಿಗಿನ ಮಾತುಕತೆ ಒಂದೊಂದು ಸಲ ತುಂಬ ಕುತೂಹಲಕರವಾಗಿರುತ್ತಿತ್ತು. ಕೆಲವು ಸಲ ಒಂದೇ ದಿನಕ್ಕೆ ಮೂರು ಬೇರೆ ಬೇರೆ ಸ್ಥಳಗಳಲ್ಲಿ ಭಾಷಣ ಮಾಡಿದ್ದೂ ಇತ್ತು. ಇದರಿಂದಾಗಿ ಅವರಿಗೆ ದೈಹಿಕ ವಾಗಿಯೂ ಮಾನಸಿಕವಾಗಿಯೂ ಅತ್ಯಂತ ಆಯಾಸವಾಗುತ್ತಿತ್ತು. ಒಮ್ಮೆ ಅವರು ತಾವು ಭಾಷಣ ಮಾಡಬೇಕಾದ ಸ್ಥಳಕ್ಕೆ ಬಂದರು. ಅದಾಗಲೇ ಅವರು ಬಳಲಿದ್ದರು. ಅಲ್ಲಿನ ಸ್ವಾಗತಸಮಿತಿಯ ಕಾರ್ಯದರ್ಶಿ ಅವರನ್ನು ಗೌರವದಿಂದ ಬರಮಾಡಿಕೊಂಡು ಒಂದು ಪುಟ್ಟ ಕೋಣೆಯೊಳಕ್ಕೆ ಕರೆದುಕೊಂಡು ಬಂದ. ಅಲ್ಲಿರುವ ಒರಗು ಕುರ್ಚಿಯ ಮೇಲೆ ಕುಳಿತುಕೊಳ್ಳುವಂತೆ ಹೇಳಿ ಹೊರಟುಹೋದ. ಆದರೆ ಕುರ್ಚಿ ಎಲ್ಲಿದೆಯೆಂದೂ ಕಾಣದಷ್ಟು ಕತ್ತಲು ಆ ಕೋಣೆಯಲ್ಲಿ! ಆದರೂ ಹೇಗೋ ತಡಕಾಡಿ ನೋಡಿ ಕುರ್ಚಿಯನ್ನು ಕಂಡುಕೊಂಡು ಮೆಲ್ಲನೆ ಅದರ ಮೇಲೆ ಕುಳಿತರು. ಆದರೆ ಅದೆಂಥ ಮಾಯಾಕುರ್ಚಿಯೊ, ಕುಳಿತುಕೊಳ್ಳುತ್ತಿದ್ದಂತೆಯೇ ಅದೂ ಕುಸಿದು ಕುಳಿತುಕೊಂಡಿತು! ಅಲ್ಲದೆ, ತುಂಬ ವಿಚಿತ್ರವಾದ ರೀತಿಯಲ್ಲಿ ಕುಸಿದುಬಿತ್ತು! ಸ್ವಾಮೀಜಿಯೂ ಅದರ ಸಂದಿಯಲ್ಲಿ ಸಿಕ್ಕಿಕೊಂಡು ಬಿಟ್ಟರು. ಇದೇನು ಗ್ರಹಚಾರವಪ್ಪ, ಎಂದು ಅವರು ಏಳಲು ನೋಡಿದರು. ಆದರೆ ಅದು ಸಾಧ್ಯವೇ ಆಗಲಿಲ್ಲ. ಇನ್ನೂ ಹೆಚ್ಚಾಗಿ ಬಲಪ್ರಯೋಗ ಮಾಡಿದ್ದರೆ ಮೈಕೈ ತರಚಿಕೊಂಡು ಬಟ್ಟೆಯನ್ನೂ ಹರಿದುಕೊಳ್ಳುವ ಅಪಾಯವಿತ್ತು. ಆದ್ದರಿಂದ ಗತ್ಯಂತರ ವಿಲ್ಲದೆ ಅದೇ ಸ್ಥಿತಿಯಲ್ಲೇ ಸ್ವಲ್ಪಹೊತ್ತು ಕುಳಿತಿದ್ದರು. ಸ್ವಾಮೀಜಿ ತಾವಾಗಿಯೇ ಬರುತ್ತಾರೆಂದು ಆ ಕಾರ್ಯದರ್ಶಿ ಕಾಯುತ್ತಿದ್ದ. ಅವರು ಎಷ್ಟು ಹೊತ್ತಾದರೂ ಬಾರದಿದ್ದಾಗ ತಾನೇ ಅವರನ್ನು ಕೂಗುತ್ತ ಬಂದ, “ಬನ್ನಿ ಸ್ವಾಮೀಜಿ, ಸಭಿಕರು ನಿಮಗಾಗಿ ಕಾಯುತ್ತಿದ್ದಾರೆ!” ಎಂದು. ಆಗ ಸ್ವಾಮೀಜಿ ಈ ಸ್ಥಿತಿಯಲ್ಲಿ ಕೂಗಿಹೇಳಿದರು, “ನೀನು ಹೇಳುವುದು ಸರಿಯೇ, ಆದರೆ ನೀನೀಗ ಈ ಕುರ್ಚಿಯನ್ನು ಮುರಿದು ಈ ಅವಸ್ಥೆಯಿಂದ ನನ್ನನ್ನು ಬಿಡಿಸಿಕೊಂಡು ಹೋಗುವವರೆಗೂ ಅವರು ಸ್ವಲ್ಪ ಕಾಯುತ್ತಿರಬೇಕಾಗುತ್ತದೆ.” ಅದನ್ನು ಕೇಳಿ ಅವನು ಈ ಕತ್ತಲ ಕೋಣೆಗೆ ಬಂದು ಕಣ್ಣಗಲಿಸಿ ಕೊಂಡು ನೋಡಿದ. ಆಗಲೇ ಅವನಿಗೆ ಪರಿಸ್ಥಿತಿ ಅರಿವಾದದ್ದು. ತಕ್ಷಣ ಅವನು ಅವರನ್ನು ಎಚ್ಚರಿಕೆ ಯಿಂದ ಮೇಲೆಬ್ಬಿಸಿದ. ಬಳಿಕ ಇಬ್ಬರೂ ಹೊಟ್ಟೆತುಂಬ ನಕ್ಕರು. ಮುಂದೆ ಸ್ವಾಮೀಜಿ ಈ ಘಟನೆ ಯನ್ನು ತಮ್ಮ ಶಿಷ್ಯರ ಮುಂದೆ ಒಳ್ಳೇ ರಸವತ್ತಾಗಿ ಹೇಳಿದಾಗ ಅಲ್ಲಿದ್ದವರೆಲ್ಲ ಬಿದ್ದುಬಿದ್ದು ನಕ್ಕರು.

ಇದೊಂದು ತಮಾಷೆಯ ಘಟನೆಯಾದರೂ ಗಂಭೀರ ಸ್ವರೂಪದವು ಹಲವಾರು. ಸ್ವಾಮೀಜಿ ಗೌರವರ್ಣದವರಾದರೂ ಪಾಶ್ಚಾತ್ಯರಿಗೆ ಅವರು ಕರಿಯರಂತೆ ಕಾಣುತ್ತಿದ್ದರು. ಎಷ್ಟೋ ಜನ, ಅದರಲ್ಲೂ ಮುಖ್ಯವಾಗಿ ಅಲ್ಲಿನ ದಕ್ಷಿಣ ಭಾಗದವರು, ಅವರನ್ನು ನೀಗ್ರೋ ಎಂದೇ ಭಾವಿಸಿ ದರು. ನೀಗ್ರೋಗಳೆಂದರೆ ಅಲ್ಲಿನ ಬಿಳಿಯರಿಗೆ ವಿಶೇಷ ತಿರಸ್ಕಾರ, ದ್ವೇಷ. ಪೌರ್ವಾತ್ಯ ದೇಶದವರ ಬಗೆಗೂ ತಿರಸ್ಕಾರ ಮನೋಭಾವವಿತ್ತಾದರೂ ನೀಗ್ರೋಗಳ ಪರಿಸ್ಥಿತಿಗಿಂತ ಅದು ಉತ್ತಮ ವಾಗಿತ್ತು. ಸ್ವಾಮೀಜಿಯನ್ನು ನೀಗ್ರೋ ಎಂದು ಭಾವಿಸಿ ಎಷ್ಟೋ ಜನ ಅವರನ್ನು ಅಪಹಾಸ್ಯ ಮಾಡಿದ್ದು, ತಿರಸ್ಕಾರದ ದೃಷ್ಟಿಯನ್ನು ಬೀರಿದ್ದು ಇದ್ದಿತಾದರೂ ಸ್ವಾಮೀಜಿ ಅವುಗಳನ್ನೆಲ್ಲ ನಿರ್ಲಕ್ಷ್ಯದಿಂದ ಕಾಣುತ್ತಿದ್ದರು. ಅಥವಾ, ಒಂದು ಬಗೆಯ ಆಧ್ಯಾತ್ಮಿಕತವಾದ ‘ಬಿಗುಮಾನ’ ದಿಂದ ಕಾಣುತ್ತಿದ್ದರು. ತಾವು ನೀಗ್ರೋ ಅಲ್ಲವೆಂಬ ಮಾತನ್ನು ಅವರೆಂದೂ ಆಡಲಿಲ್ಲ. ಒಮ್ಮೆ ಒಂದು ರೈಲು ನಿಲ್ದಾಣದಲ್ಲಿ ಸ್ವಾಗತ ಸಮಿತಿಯವರು ಅವರನ್ನು ಸಂಭ್ರಮದಿಂದ ಬರಮಾಡಿ ಕೊಳ್ಳುವುದನ್ನು ಒಬ್ಬ ನೀಗ್ರೋ ಕೂಲಿ ನೋಡಿದ. ತಮ್ಮ ಜಾತಿಯವನೊಬ್ಬ ಎಂತಹ ಉನ್ನತ ಸ್ಥಿತಿಗೇರಿದ್ದಾನೆ!–ಎಂದು ಅವನಿಗೆ ಬಹಳ ಸಂತೋಷವಾಯಿತು. ತಕ್ಷಣ ಅವನು ಸ್ವಾಮೀಜಿಯ ಬಳಿಗೆ ಬಂದು ಅವರನ್ನೇ ಅಡಿಯಿಂದ ಮುಡಿಯವರೆಗೂ ನೋಡಿದ. ಇಂಥವರ ಕೈಕುಲುಕುವ ಸೌಭಾಗ್ಯ ದೊರಕಿದರೆ ತಾನೇ ಧನ್ಯ ಎಂದು ಭಾವಿಸಿ, “ಅಣ್ಣ, ನೀನು ನಮ್ಮ ಜಾತಿಯವನೇ ಆಗಿದ್ದು ಇಷ್ಟೊಂದು ಉನ್ನತ ಮಟ್ಟಕ್ಕೆ ಬಂದಿರುವುದನ್ನು ಕಂಡು ನನಗೆ ಬಹಳ ಆನಂದವಾಗು ತ್ತಿದೆ. ನಿನ್ನ ಕೈಯನ್ನೊಮ್ಮೆ ಕುಲುಕಬೇಕೆಂಬ ಆಸೆ ನನಗೆ” ಎಂದ. ತಕ್ಷಣ ಸ್ವಾಮೀಜಿ ತುಂಬ ವಿಶ್ವಾಸದಿಂದ ಅವನ ಹಸ್ತಗಳನ್ನು ಬಿಗಿಯಾಗಿ ಹಿಡಿದುಕೊಂಡು ಕುಲುಕುತ್ತ “ಥ್ಯಾಂಕ್ ಯು ಬ್ರದರ್ ಥ್ಯಾಂಕ್ ಯು!” ಎಂದರು. ಆ ಕೂಲಿಯವನಿಗಾದ ಆನಂದಕ್ಕೆ ಪಾರವೇ ಇಲ್ಲ.

ಈ ರೀತಿ ಹಲವಾರು ನಿಗ್ರೋಗಳು ಸ್ವಾಮೀಜಿ ತಮ್ಮಲ್ಲೊಬ್ಬರೆಂದು ಭಾವಿಸಿ ವಿಶ್ವಾಸವಿಟ್ಟಿ ದ್ದರು. ತಮ್ಮನ್ನು ಜನರು ನೀಗ್ರೋಗಳಲ್ಲೊಬ್ಬ ಎಂದು ಭಾವಿಸಿದರೆ ಸ್ವಾಮೀಜಿ ಸ್ವಲ್ಪವೂ ಬೇಸರಿಸಿಕೊಳ್ಳುತ್ತಿರಲಿಲ್ಲ. ಅಲ್ಲಿನ ಎಷ್ಟೋ ಕ್ಷೌರಿಕರ ಅಂಗಡಿಗಳವರು ಅವರನ್ನು ನೀಗ್ರೋ ಎಂದು ತಿಳಿದು ಪ್ರವೇಶವನ್ನು ನಿರಾಕರಿಸಿದರು. ಎಷ್ಟೋ ದೊಡ್ಡ ನಗರಗಳಲ್ಲಿ ಅವರು ಹೋಟಲಿಗೆ ಹೋದರೆ ಕೆಲವರು ಅವರನ್ನು ಹೊರಗಟ್ಟಿದರು. ತಾವು ಪೌರ್ವಾತ್ಯರು, ನೀಗ್ರೋ ಅಲ್ಲ ಎಂದು ಹೇಳಿಕೊಂಡಿದ್ದರೆ ಅವರಿಗೆ ಪ್ರವೇಶ ದೊರಕುತ್ತಿತ್ತು. ಆದರೆ ಇಷ್ಟು ಕಷ್ಟವಾದರೂ, ಇಷ್ಟು ಅವಮಾನವಾದರೂ ‘ನಾನು ನೀಗ್ರೋ ಅಲ್ಲ’ ಎಂಬ ಮಾತು ಅವರ ಬಾಯಿಯಿಂದ ಬರಲೇ ಇಲ್ಲ. ಇದರಿಂದಾಗಿ ಉಪನ್ಯಾಸ ಸಂಸ್ಥೆಯ ಮೇಲ್ವಿಚಾರಕರು ಅವರಿಗೋಸ್ಕರ ಬೇರೆ ಏರ್ಪಾಡು ಮಾಡಬೇಕಾಗುತ್ತಿತ್ತು. ಹೋಟೆಲು ಸಾಹುಕಾರರು ಸ್ವಾಮೀಜಿಯನ್ನು ಹೋಟೆಲೊಳಗೆ ಬಿಡದಿದ್ದರೂ, ಮರುದಿನವೇ ವರ್ತಮಾನ ಪತ್ರಿಕೆಗಳಲ್ಲಿ ಅವರ ಬಗೆಗಿನ ಲೇಖನವನ್ನು ಓದಿದಾಗ, ಇಲ್ಲವೆ ಅವರ ಕೀರ್ತಿಯನ್ನು ಕೇಳಿದಾಗ, ಓಡಿಬಂದು ಅವರ ಕ್ಷಮೆ ಯಾಚಿಸುತ್ತಿದ್ದರು. ಕೆಲವು ವರ್ಷಗಳ ಮೇಲೆ ಒಮ್ಮೆ ಸ್ವಾಮೀಜಿ ಈ ವಿಷಯಗಳನ್ನೆಲ್ಲ ತಮ್ಮ ಒಬ್ಬ ಅಮೆರಿಕನ್ ಶಿಷ್ಯನ ಮುಂದೆ ಹೇಳಿದಾಗ ಅವನು ಅಚ್ಚರಿಯಿಂದ “ಸ್ವಾಮೀಜಿ, ಅಂತಹ ಸಂದರ್ಭಗಳಲ್ಲಿ ನಿಮ್ಮನ್ನು ನೀವು ಪೌರ್ವಾತ್ಯ ಎಂದು ಹೇಳಿಕೊಳ್ಳಬಹುದಾಗಿತ್ತಲ್ಲ!” ಎಂದು ಪ್ರಶ್ನಿಸಿದ. ಇದನ್ನು ಕೇಳಿದಾಗ ಸ್ವಾಮೀಜಿ ನಸುಗೋಪದಿಂದ ಉದ್ಗರಿಸಿದರು. “ಏನು! ಇತರರನ್ನು ಕೆಳಕ್ಕೆ ತಳ್ಳಿ ನಾನು ಮೇಲೇ ರಲೆ? ನಾನು ಆ ಉದ್ದೇಶಕ್ಕಾಗಿ ಹುಟ್ಟಿದವನಲ್ಲ!” ತಾವು ಭಾರತೀಯ, ನೀಗ್ರೋ ಅಲ್ಲ ಎಂದು ಹೇಳಿದಾಗ, ತಾವು ನೀಗ್ರೋಗಳಿಗಿಂತ ಉತ್ತಮರು–ನೀಗ್ರೋಗಳು ತಮಗಿಂತ ಕೀಳು ಎಂದು ಪರೋಕ್ಷವಾಗಿ ಹೇಳಿದಂತಾಯಿತಲ್ಲವೆ? ಎಂದ ಮೇಲೆ ವರ್ಣಭೇದ ಬುದ್ಧಿಯಲ್ಲಿ ತಾವು ಆ ಬಿಳಿಯರಿಗಿಂತ ಯಾವ ರೀತಿಯಲ್ಲಿ ಭಿನ್ನರಾದೆವು!–ಎಂಬುದು ಸ್ವಾಮೀಜಿಯ ಅಭಿಪ್ರಾಯ.

ಇದೇ ಸಮಯದಲ್ಲಿ ವೃತ್ತಪತ್ರಿಕೆಗಳ ಮೂಲಕ ಸ್ವಾಮೀಜಿಯ ಹೆಸರು ಅವರು ಹೋದ ಹೋದಲ್ಲೆಲ್ಲ ಪ್ರಚಾರವಾಗುತ್ತಿತ್ತು. ಬಹುತೇಕ ಪತ್ರಿಕೆಗಳು ಅವರಿಗೆ ಬೆಂಬಲವಾಗಿ ಬರೆಯುತ್ತಿ ದ್ದುವು. ಅಲ್ಲದೆ ಅವರು ಉಪನ್ಯಾಸಸಂಸ್ಥೆಯ ಮೂಲಕ ಭಾಷಣಗಳನ್ನು ಮಾಡುತ್ತಿದ್ದುದರಿಂದ ಅವರಿಗೆ ಸಾಕುಸಾಕೆನಿಸುವಷ್ಟು ಪ್ರಚಾರ ಸಿಗುತ್ತಿತ್ತು. ವರದಿಗಾರರು, ಸಂಪಾದಕರು ಅವರನ್ನು ಮುತ್ತಿಕೊಂಡೇ ಇರುತ್ತಿದ್ದರು. ಇವರೆಲ್ಲ ಕೇಳದ ಪ್ರಶ್ನೆಗಳೇ ಇಲ್ಲ. ಸ್ವಾಮೀಜಿಯ ವೈಯಕ್ತಿಕ ಅಭ್ಯಾಸಗಳ ಬಗ್ಗೆ, ಅವರ ಧರ್ಮದ ಬಗ್ಗೆ, ಸಿದ್ಧಾಂತದ ಬಗ್ಗೆ, ಕಾರ್ಯಯೋಜನೆಗಳ ಬಗ್ಗೆ, ಆಹಾರಪದ್ಧತಿಯ ಬಗ್ಗೆ, ಅವರಿಗಾದ ಲೋಕಾನುಭವಗಳ ಬಗ್ಗೆ, ಭಾರತದ ಜನರ ರೀತಿ-ರಿವಾಜು ಗಳ ಬಗ್ಗೆ–ಹೀಗೆ ಹಲವಾರು ವಿಷಯಗಳ ಬಗ್ಗೆ ನಾನಾ ಬಗೆಯ ಪ್ರಶ್ನೆಗಳನ್ನು ಕೇಳುತ್ತಿದ್ದರು. ತನ್ಮೂಲಕ ಸ್ವಾಮೀಜಿಯ ಹಾಗೂ ಅವರ ರಾಷ್ಟ್ರದ ಕುರಿತಾಗಿ ಪತ್ರಿಕೆಗಳಲ್ಲಿ ವಿವರವಾಗಿ ಪ್ರಕಟ ವಾಗುತ್ತಿತ್ತು. ಅದರಲ್ಲೂ ಭಾರತದ ಕುರಿತು ಸ್ವಾಮೀಜಿ ವಿಶೇಷ ಆಸಕ್ತಿಯಿಂದ ಮಾತನಾಡು ತ್ತಿದ್ದರು. ಏಕೆಂದರೆ ಅವರ ಪ್ರೀತಿಯ ತಾಯ್ನಾಡಿನ ಒಳಿತು ಹಾಗೂ ಪ್ರಗತಿ–ಇದೇ ಅವರ ಚಿಂತನೆಯ ಪಲ್ಲವಿಯಾಗಿತ್ತಲ್ಲವೆ? ಅದರ ಕುರಿತಾಗಿ ಎಷ್ಟು ಮಾತನಾಡಿದರೂ ಅವರಿಗೆ ದಣಿವಿಲ್ಲ.

ಅಮೆರಿಕದಲ್ಲಿ ಅವರು ಹಿಂದೂಧರ್ಮದ ದಿವ್ಯ ಸಂದೇಶಗಳನ್ನು ಪ್ರಸಾರ ಮಾಡುವುದಕ್ಕಾ ಗಿಯೇ ಪರ್ಯಟನೆ ಮಾಡುತ್ತಿದ್ದಂತೆ ಕಂಡುಬರುತ್ತಿದ್ದರೂ, ತಮ್ಮ ಬಡ ಭಾರತೀಯ ಸೋದರ ರಿಗಾಗಿ ಏನಾದರೂ ಸಹಾಯವನ್ನು ಪಡೆಯಲು ಸಾಧ್ಯವಾದೀತೆ, ಎಂಬುದೇ ಅವರ ನಿಜವಾದ ಆಲೋಚನೆಯಾಗಿತ್ತು. ಇದನ್ನೇ ಅವರು ಹರಿಪದ ಮಿತ್ರನಿಗೆ ಬರೆದ ಪತ್ರದಲ್ಲಿ ತಿಳಿಸುತ್ತಾರೆ:

“ನಾನು ಈ ರಾಷ್ಟ್ರಕ್ಕೆ ಬಂದದ್ದು ನನ್ನ ಕುತೂಹಲವನ್ನು ತಣಿಸಲೆಂದಲ್ಲ; ಅಥವಾ ಹೆಸರು- ಕೀರ್ತಿಗಾಗಿಯೂ ಅಲ್ಲ. ಆದರೆ ಭಾರತದ ಬಡಜನರ ಜೀವನಾಧಾರಕ್ಕೆ ಏನಾದರೊಂದು ಮಾರ್ಗ ವನ್ನು ಕಂಡುಕೊಳ್ಳಲು ಸಾಧ್ಯವೆ ಎಂಬುದನ್ನು ನೋಡಲು ಬಂದೆ. ಭಗವಂತ ನನಗೆ ನೆರವಾದರೆ ಆ ಮಾರ್ಗ ಯಾವುದು ಎಂಬುದನ್ನು ಕಾಲಾಂತರದಲ್ಲಿ ನೀನೇ ನೋಡುವೆಯಂತೆ.”

ಈ ಸಮಯದಲ್ಲಿ ಅಮೆರಿಕೆಯ ವರ್ತಮಾನ ಪತ್ರಿಕೆಗಳು ಸ್ವಾಮೀಜಿಯ ಭಾಷಣಗಳಲ್ಲಿ ಕಂಡುಕೊಂಡ ಒಂದು ಗಮನಾರ್ಹ ವಿಷಯವೆಂದರೆ ಅವರ ರಾಷ್ಟ್ರಪ್ರೇಮ. ಆ ಬಗ್ಗೆ ಒಂದು ಪತ್ರಿಕೆ ಹೀಗೆ ಬರೆಯಿತು–“ಅವರ ರಾಷ್ಟ್ರಪ್ರೇಮವು ಅತ್ಯಂತ ಉಜ್ವಲವಾದದ್ದು. ಅವರು ‘ನನ್ನ ರಾಷ್ಟ್ರ’ದ ಬಗ್ಗೆ ಮಾತನಾಡುವ ರೀತಿ ಅತ್ಯಂತ ಹೃದಯಸ್ಪರ್ಶಿಯಾಗಿದೆ. ಅವರ ಆ ಒಂದು ಮಾತು, ಅವರು ಕೇವಲ ಸಂನ್ಯಾಸಿ ಮಾತ್ರವೇ ಅಲ್ಲ, ಅವರು ತಮ್ಮ ಜನತೆಯ ಪ್ರತಿನಿಧಿ ಎಂಬು ದನ್ನು ತೆರೆದುತೋರುತ್ತದೆ.”

ತಮ್ಮಲ್ಲಿಗೆ ಬಂದು ಭಾಷಣ ಮಾಡುವಂತೆ ಸ್ವಾಮೀಜಿಗೆ ಆಹ್ವಾನಗಳ ಮೇಲೆ ಆಹ್ವಾನಗಳು ಬರುತ್ತಿದ್ದುವು. ಚರ್ಚುಗಳು, ಕ್ಲಬ್ಬುಗಳು, ಖಾಸಗಿ ಸಂಸ್ಥೆಗಳು, ವಿದ್ಯಾಸಂಸ್ಥೆಗಳು–ಹೀಗೆ ಸಮಾಜದ ವಿವಿಧ ವರ್ಗಗಳ ಜನರು ಅವರನ್ನು ಆಹ್ವಾನಿಸುತ್ತಿದ್ದರು. ಇವುಗಳಲ್ಲಿ ಹೆಚ್ಚಿನವನ್ನು ಸ್ವಾಮೀಜಿ ಅಂಗೀಕರಿಸುತ್ತಿದ್ದರು. ಸನಾತನ ವೇದಾಂತತ್ತ್ವಗಳನ್ನು ಪ್ರಸಾರ ಮಾಡಲೂ, ಭಾರತದ ಆವಶ್ಯಕತೆಗಳನ್ನು ಅಮೆರಿಕದ ಜನತೆಯ ಮುಂದಿಡಲೂ ಅವುಗಳು ಸದವಕಾಶವೆಂದು ಅವರು ತಿಳಿದಿದ್ದರು. ಆದ್ದರಿಂದ ತಮ್ಮ ತನುಮನಗಳೆರಡೂ ದಣಿದು ಬಸವಳಿದು, ‘ಇನ್ನು ನಮ್ಮಿಂದ ಸಾಧ್ಯವೇ ಇಲ್ಲ’ ಎಂದು ಅವು ಮುಷ್ಕರ ಹೂಡುವವರೆಗೂ ದುಡಿಯುತ್ತಿದ್ದರು. ಕೆಲ ವೊಮ್ಮೆ ವಾರಕ್ಕೆ ಹದಿನೈದು ಇಲ್ಲವೆ ಇನ್ನೂ ಹೆಚ್ಚು ಉಪನ್ಯಾಸಗಳನ್ನು ನೀಡುತ್ತಿದ್ದರು. ಇದಕ್ಕಾಗಿ ಅವರು ಯಾವುದೇ ಬಗೆಯ ತಯಾರಿಯನ್ನೂ ಮಾಡಿಕೊಳ್ಳುತ್ತಿರಲಿಲ್ಲ. ಕೆಲವೊಮ್ಮೆ ತಮ್ಮ ಬುದ್ಧಿಶಕ್ತಿ ಎಲ್ಲವೂ ಖರ್ಚಾಗಿಹೋಯಿತೋ ಎಂಬಂತೆ ಅವರಿಗನ್ನಿಸುತ್ತಿತ್ತು. ಹೇಳಲು ಇನ್ನಾವ ವಿಷಯವೂ ಉಳಿದಿಲ್ಲವೆಂಬಂತೆ ತೋರುತ್ತಿತ್ತು. ಆಗ ಅವರು ಚಿಂತಾಕ್ರಾಂತರಾಗಿ, ‘ನಾಳೆಯ ದಿನ ಭಾಷಣದಲ್ಲಿ ಏನು ಹೇಳುವುದು?’ ಎಂದು ಆಲೋಚಿಸುತ್ತ ಕುಳಿತುಕೊಳ್ಳುತ್ತಿದ್ದರು. ಅಂತಹ ಕಷ್ಟದ ಸಂದರ್ಭದಲ್ಲಿ ಅವರಿಗೆ ಎಲ್ಲಿಂದಲೋ ಅದ್ಭುತವಾದ ನೆರವು ಒದಗಿಬರುತ್ತಿತ್ತು. ಒಂದೊಂದು ಸಲ ನೀರವ ನಡುರಾತ್ರಿಯ ವೇಳೆಗೆ, ತಾವು ಮಾರನೆಯ ದಿನ ಮಾಡಬೇಕಾದ ಭಾಷಣದ ವಿಷಯದ ಬಗ್ಗೆ ಯಾರೋ ಗಟ್ಟಿಯಾಗಿ ಹೇಳುತ್ತಿದ್ದಂತೆ ಅವರಿಗೆ ಕೇಳುತ್ತಿತ್ತು. ಮತ್ತೆ ಕೆಲವೊಮ್ಮೆ ಎರಡು ದನಿಗಳು ಯಾವುದೋ ವಿಷಯದ ಬಗ್ಗೆ ದೀರ್ಘವಾಗಿ ಚರ್ಚಿಸುತ್ತಿರುವಂತೆ ಕೇಳಿಬರುತ್ತಿತ್ತು. ಇದೇ ವಿಷಯಗಳ ಬಗ್ಗೆ ಸ್ವಾಮೀಜಿ ಮರುದಿನ ಮಾತನಾಡುತ್ತಿದ್ದರು. ಎಷ್ಟೋ ಸಲ ಈ ಭಾವನೆಗಳೆಲ್ಲ ಅವರಿಗೆ ಹೊಚ್ಚಹೊಸದಾಗಿರುತ್ತಿದ್ದುವು.

ಆದರೆ ಈ ಅದ್ಭುತ ಘಟನೆಗಳಲ್ಲಿ ಸ್ವಾಮೀಜಿ ಯಾವ ವೈಚಿತ್ರ್ಯವನ್ನೂ ಕಾಣುತ್ತಿರಲಿಲ್ಲ. ಇವೆಲ್ಲವೂ ಅವರಿಗೆ ಸಹಜವಾಗಿಯೇ ತೋರುತ್ತಿತ್ತು. ಈ ಅನುಭವಗಳನ್ನು ಅವರು, ಮನಸ್ಸಿನ ಸುಪ್ತಶಕ್ತಿಗಳ ಅಭಿವ್ಯಕ್ತಿಗಳೆಂದು ಭಾವಿಸುತ್ತಿದ್ದರು. ಮನಶ್ಶಕ್ತಿಯನ್ನು ಬೆಳೆಸಿಕೊಂಡವರಿಗೆ ತಾನೇ ತಾನಾಗಿ ಆಗುವ ಅನುಭವಗಳು ಇವು ಎಂದು ಅವರು ಹೇಳುತ್ತಿದ್ದರು. ಮನಸ್ಸಿಗೆ ಚಿಂತಿಸಲು ಕೆಲವು ಬಗೆಯ ಭಾವನೆಗಳನ್ನು ಕೊಟ್ಟಾಗ, ಆ ಮನಸ್ಸು ತನ್ನೆಲ್ಲ ಶಕ್ತಿಯನ್ನು ಉಪಯೋಗಿಸಿ ಕೊಂಡು ಆ ಆಲೋಚನೆಗಳನ್ನು ಚೆನ್ನಾಗಿ ಸಂಸ್ಕರಿಸಿ, ವಿಸ್ತರಿಸಿ ಅವುಗಳನ್ನು ಸ್ಪಷ್ಟವಾದ ಮಾತುಗಳ ರೂಪದಿಂದ ಹೊರಹೊಮ್ಮಿಸುತ್ತದೆ. ಶ್ರೀರಾಮಕೃಷ್ಣರು ಹೇಳುತ್ತಿದ್ದಂತೆ, ಶುದ್ಧ ವಾದ ಮನಸ್ಸೇ ಗುರುವಾಗಿ ಕೆಲಸ ಮಾಡುವುದು ಇಂತಹ ಸಂದರ್ಭದಲ್ಲೇ. ಮನಸ್ಸಿನ ಈ ಅದ್ಭುತ ಶಕ್ತಿಯನ್ನು ಗಮನಿಸಿಯೇ ಸ್ವಾಮೀಜಿ ಉಹಿಸಿದರು–ಪುರಾತನ ಮಹರ್ಷಿಗಳು ಬರೆದ ಉಪನಿಷದ್ವಾಕ್ಯಗಳು ಅವರ ಶುದ್ಧ ಮನಸ್ಸಿನ ಆಳದಲ್ಲಿ ತಾನೇ ತಾನಾಗಿ ಆವಿರ್ಭವಿಸಿರಬೇಕು, ಎಂದು. ತಮಗಾದ ಈ ಅನುಭವವನ್ನು ಬಣ್ಣಿಸುತ್ತ ಸ್ವಾಮೀಜಿ, “ಸ್ಫೂರ್ತಿಯೆಂದು ಯಾವುದನ್ನು ಕರೆಯುತ್ತಾರೋ ಅದೇ ಇದು” ಎಂದು ಮುಂದೆ ಆಪ್ತಶಿಷ್ಯರ ಬಳಿ ಹೇಳಿದರು. ತಮ್ಮ ಈ ಅನು ಭವಗಳೆಲ್ಲ ಆಂತರಿಕವಾದವೆಂದು ಸ್ವಾಮೀಜಿ ಹೇಳಿದರಾದರೂ ಈ ಸಂಭಾಷಣೆಯ ಮಾತುಗಳು ಗಟ್ಟಿಯಾಗಿಯೇ ಕೇಳಿ ಬರುತ್ತಿದ್ದುವು! ಒಮ್ಮೆ ಆ ಮನೆಯಲ್ಲಿದ್ದವರು ಇದನ್ನು ಕೇಳಿಸಿಕೊಂಡು ದಿಗ್ಭ್ರಾಂತರಾಗಿ ಮರುದಿನ ಅವರನ್ನು “ಸ್ವಾಮೀಜಿ, ನಿನ್ನೆ ರಾತ್ರಿ ನೀವು ಯಾರೊಂದಿಗೋ ಗಟ್ಟಿಯಾಗಿ, ಉತ್ಸಾಹದಿಂದ ಮಾತನಾಡುವ ಶಬ್ದ ಕೇಳಿಬರುತ್ತಿತ್ತಲ್ಲ? ಅದನ್ನು ಕೇಳಿ ನಮಗೆಲ್ಲ ಆಶ್ಚರ್ಯವಾಯಿತು. ಯಾರವರು ಸ್ವಾಮೀಜಿ!” ಎಂದು ಕೇಳಿದಾಗ ಸ್ವಾಮೀಜಿ, ಮುಗುಳ್ನಗುತ್ತ ಆ ವಿಷಯವನ್ನು ತಟ್ಟಿಹಾರಿಸುವಂತೆ ಉತ್ತರಿಸಿದರು. ಇದರಿಂದಾಗಿ ಆ ಜನರ ಕುತೂಹಲ ಮತ್ತಷ್ಟು ಕೆರಳಿತು. ಅದೇನೇ ಆದರೂ ಇವುಗಳು ಪವಾಡಗಳಂತೂ ಅಲ್ಲವೇ ಅಲ್ಲ; ಪ್ರಕೃತಿಯ ನಿಯಮಗಳಿಗೆ ಒಳಪಟ್ಟಂತಹ ವಿಷಯಗಳು ಮಾತ್ರವಷ್ಟೇ ಎಂದು ಸ್ವಾಮೀಜಿ ಸ್ಪಷ್ಟವಾಗಿ ಹೇಳುತ್ತಿದ್ದರು. ಆದರೆ ಅವರ ಶಿಷ್ಯರಿಗೆ ಕಡೆಗೂ ಅದೊಂದು ಒಗಟಾಗಿಯೇ ಉಳಿಯಿತು.

ಇದಾದನಂತರ ತಮ್ಮಲ್ಲಿ ಅಸಾಮಾನ್ಯ ಯೋಗಶಕ್ತಿಗಳು ತಾನೇತಾನಾಗಿ ವ್ಯಕ್ತಗೊಳ್ಳುವುದನ್ನು ಅವರು ಗಮನಿಸಿದರು. ಮನಸ್ಸು ಮಾಡಿದರೆ, ಕೇವಲ ಸ್ಪರ್ಶಮಾತ್ರದಿಂದಲೇ ಅವರು ವ್ಯಕ್ತಿ ಯೊಬ್ಬನ ಜೀವನದ ದಿಕ್ಕನ್ನೇ ಬದಲಾಯಿಸಬಲ್ಲವರಾಗಿದ್ದರು. ಹಿಂದೆ ಜೈಪುರದಲ್ಲಿದ್ದಾಗ ಸರ್ದಾರ್ ಹರಿಸಿಂಗನಿಗೆ ತಮ್ಮ ಸ್ಪರ್ಶದಿಂದ ಅಲೌಕಿಕ ಅನುಭವವನ್ನು ಮಾಡಿಸಿಕೊಟ್ಟದ್ದನ್ನು ನೋಡಿದ್ದೇವೆ. ಎಷ್ಟು ಜನರಿಗೆ ಸ್ವಾಮೀಜಿಯಿಂದ ಇಂತಹ ಅನುಭವಗಳಾದುವೋ ಯಾರಿಗೆ ಗೊತ್ತು? ಅಲ್ಲದೆ ಅವು ಅತ್ಯಂತ ವೈಯಕ್ತಿಕ ವಿಚಾರಗಳಾದ್ದರಿಂದ ಎಷ್ಟೋ ಜನ ಅವುಗಳನ್ನು ಬಹಿರಂಗಪಡಿಸಲು ಇಚ್ಛಿಸಲಾರರು. ಆದರೂ ಸ್ವಾಮೀಜಿಯ ಯೋಗಶಕ್ತಿಗಳು ಸ್ಪಷ್ಟವಾಗಿ ವ್ಯಕ್ತವಾದಂತಹ ಹಲವಾರು ನಿದರ್ಶನಗಳಿವೆ. ಎಮ್ಮಾ ಕಾಲ್ವೆ, ರಾಕ್​ಫೆಲ್ಲರ್ ಮೊದಲಾದವರು ಇವರಲ್ಲಿ ಎಲ್ಲೋ ಕೆಲವರಷ್ಟೆ. ಅಲ್ಲದೆ ಅವರಿಂದ ಮಂತ್ರದೀಕ್ಷೆಯನ್ನು ಪಡೆದವರಿಗೆಲ್ಲ ಅವರ ಅಸಾಧಾರಣ ಶಕ್ತಿಗಳ ಪರಿಚಯವಾಗುತ್ತಿತ್ತು. ಮತ್ತೆ ಕೆಲವೊಮ್ಮೆ ಯಾವಾಗಲಾದರೂ ಸ್ವಾಮೀಜಿ ಈ ಬಗ್ಗೆ ಪ್ರಸ್ತಾಪಿಸಿದಾಗ ಅವರ ಕೆಲವು ಶಿಷ್ಯರು ಆ ಶಕ್ತಿಗಳನ್ನು ಪ್ರದರ್ಶಿಸುವಂತೆ ಅವರನ್ನು ಕಾಡಿದ್ದರು. ಇಂತಹ ಶಕ್ತಿಪ್ರದರ್ಶನಗಳನ್ನು ಸ್ವಾಮೀಜಿ ಎಂದೂ ಇಷ್ಟಪಡುತ್ತಿರಲಿಲ್ಲ. ಆದರೂ ಕೆಲವು ಸಲ ಆ ಶಿಷ್ಯರ ಒತ್ತಾಯಕ್ಕೆ ಮಣಿದು, ತಮ್ಮ ಮಾತುಗಳನ್ನು ಪರೀಕ್ಷಿಸಿ ನೋಡಲು ಅವರಿಗೆ ಅವಕಾಶ ಮಾಡಿಕೊಡುತ್ತಿದ್ದುದೂ ಉಂಟು. ಇಂತಹ ಪ್ರತಿಯೊಂದು ಸಂದರ್ಭದಲ್ಲಿಯೂ ಸ್ವಾಮೀಜಿಯ ಮಾತು ನಿಜವಾಗಿರುತ್ತಿತ್ತು! ಮತ್ತಿನ್ನೆಷ್ಟೋ ಸಲ ಅವರ ಶಿಷ್ಯರು ಮನಸ್ಸಿನಲ್ಲಿ ಹಲವಾರು ಸಂಶಯಗಳನ್ನು ಹೊತ್ತು ಅವರ ಬಳಿ ಬರುತ್ತಿದ್ದರು. ತಮ್ಮ ಮನಸ್ಸಿನಲ್ಲಿರುವುದನ್ನು ಹೇಳಿಕೊಳ್ಳಲಾರದೆ ಚಡಪಡಿಸುತ್ತಿದ್ದರು. ಆಗ ಸ್ವಾಮೀಜಿ ತಾವಾಗಿಯೇ ಆ ವಿಷಯವನ್ನು ಪ್ರಸ್ತಾಪಿಸಿ, ಶಿಷ್ಯರ ಸಮಸ್ಯೆಯನ್ನು ಪರಿಹರಿಸುತ್ತಿದ್ದರು.

ಒಮ್ಮೆ ಸ್ವಾಮೀಜಿ ಉಪನ್ಯಾಸವೊಂದರಲ್ಲಿ ಯೋಗಶಕ್ತಿಗಳ ಬಗ್ಗೆ ಹೇಳಿದಾಗ, ಅದನ್ನು ಕೇಳಿದ ಶಿಕಾಗೋದ ಶ್ರೀಮಂತನೊಬ್ಬ ಅವರ ಬಳಿಗೆ ಬಂದ. ಸ್ವಾಮೀಜಿ ಹೇಳಿದುದೆಲ್ಲ ಅರ್ಥವಿಲ್ಲದ ಕಂತೆ ಎಂದು ಲೇವಡಿಮಾಡಿದ. “ಮಹಾಶಯರೆ, ನೀವು ಹೇಳುವುದರಲ್ಲಿ ಏನಾದರೂ ಸತ್ಯವಿದ್ದರೆ, ನನ್ನ ಮನಸ್ಸಿನ ಆಲೋಚನೆಗಳ ಬಗೆಗೋ ಅಥವಾ ನನ್ನ ಹಿಂದಿನ ಜೀವನದ ಬಗೆಗೋ ಏನಾದರೂ ಹೇಳಿ ನೋಡೋಣ” ಎಂದು ಸವಾಲೊಡ್ಡಿದ. ಸ್ವಾಮೀಜಿ ಒಂದು ಕ್ಷಣ ಹಿಂಜರಿದರು. ಬಳಿಕ ಇದ್ದಕ್ಕಿದ್ದಂತೆ ತಮ್ಮ ದೃಷ್ಟಿಯನ್ನು ಆ ವ್ಯಕ್ತಿಯ ಕಣ್ಣುಗಳಲ್ಲಿ ನೆಟ್ಟು ತೀಕ್ಷ್ಣವಾಗಿ ದಿಟ್ಟಿಸಿದರು. ತಕ್ಷಣ ಆ ಶ್ರೀಮಂತ ಅಪ್ರತಿಭನಾಗಿ ನಡುಗಿದ. ಸ್ವಾಮೀಜಿಯ ದೃಷ್ಟಿ ತನ್ನ ಅಂತರಂಗವನ್ನು ಭೇದಿಸಿ ಒಳಹೊಕ್ಕು ನೋಡುತ್ತಿರುವಂತೆ ಅವನಿಗೆ ಭಾಸವಾಯಿತು. “ಅಯ್ಯಯ್ಯೋ!ಸ್ವಾಮೀಜಿ ಏನು ಮಾಡುತ್ತಿದ್ದೀರಿ ನೀವು? ನನ್ನ ಜೀವವನ್ನೇ ಕಡೆಯುತ್ತಿರುವಂತಿದೆ ನಿಮ್ಮ ನೋಟ! ನನ್ನ ಜೀವನದ ರಹಸ್ಯಗಳೆಲ್ಲ ಆಚೆ ಬರುತ್ತಿವೆ!” ಎಂದು ಕೂಗಿಕೊಂಡ. ಸ್ವಾಮೀಜಿ ತಮ್ಮ ದೃಷ್ಟಿ ಯನ್ನು ಹಿಂದೆಗೆದುಕೊಂಡರು. ಇಂತಹ ಶಕ್ತಿಗಳನ್ನು ಅವರು ಹೊಂದಿದ್ದರಾದರೂ ಅವುಗಳನ್ನು ಅವರು ಆಧ್ಯಾತ್ಮಿಕತೆಯ ಚಿಹ್ನೆ ಎಂದು ಪರಿಗಣಿಸುತ್ತಿರಲಿಲ್ಲ. ಅಲ್ಲದೆ ಅವುಗಳನ್ನು ಅವರೆಂದಾ ದರೂ ಬಳಸಿಕೊಂಡರೂ ಅದೂ ಇತರರ ಒಳಿತಿಗಾಗಿ ಮಾತ್ರವೇ ಆಗಿತ್ತು.

೧೩ನೇ ಜನವರಿ ೧೮೯೪ರಂದು ಸ್ವಾಮೀಜಿ ದಕ್ಷಿಣದ ಟೆನ್ನಿಸ್ಸೀ ರಾಜ್ಯದ ಮೆಂಫಿಸ್ ನಗರಕ್ಕೆ ಬಂದರು. ಇಲ್ಲಿನ ‘ನೈನ್​ಟೀನ್ತ್ ಸೆಂಚುರಿ ಕ್ಲಬ್​’ ಅವರನ್ನು ಆಹ್ವಾನಿಸಿತ್ತು. ಅವರನ್ನು ಸಂದರ್ಶಿಸಿದ ‘ಅಪೀಲ್ ಅವಲಾನ್ಷ್​’ ಪತ್ರಿಕೆ ಬರೆಯಿತು:

“‘ವೇದಿಕೆಯ ಮೇಲಿನ ಪ್ರಚಂಡರಲ್ಲೊಬ್ಬರು’,‘ ಅವರ ಜನಾಂಗದ ಸಮರ್ಥ ಪ್ರತಿನಿಧಿ’, ‘ಸರ್ವಧರ್ಮ ಸಮ್ಮೇಳನದ ಅದ್ಭುತಗಳಲ್ಲೊಬ್ಬರು’, ‘ದೈವದತ್ತವಾಗ್ಮಿ’–ಈ ಗುಣ ವಾಚಕ ಗಳೆಲ್ಲವೂ ಸ್ವಾಮಿವಿವೇಕಾನಂದರ ವಿಚಾರದಲ್ಲಿ ಸತ್ಯ; ಅಷ್ಟೇ ಅಲ್ಲ, ಇವೂ ಕೂಡ ಕಡಿಮೆಯೇ!... ” (ಈ ಗುಣವಾಚಕಗಳನ್ನು ಹಿಂದೆ ಶಿಕಾಗೋದ ‘ಕ್ರಿಟಿಕ್​’ ಪತ್ರಿಕೆ ಬಳಸಿತ್ತು.)

ಮತ್ತೆ ಮರುದಿನ ಅದೇ ಪತ್ರಿಕೆಯಲ್ಲಿ ಹೀಗೆ ಬರೆಯಲಾಗಿತ್ತು–“ಸಂಭಾಷಣೆಯಲ್ಲಿ ಅವರು ಅತ್ಯಂತ ಆಹ್ಲಾದಕರ ಸಂಭಾವಿತ ವ್ಯಕ್ತಿ. ಅವರು ಬಳಸುವ ಪದಗಳು ಇಂಗ್ಲಿಷ್ ಭಾಷೆಯ ರತ್ನ ಗಳು. ಅವರ ಸಾಮಾನ್ಯ ವರ್ತನೆಯು ಪಾಶ್ಚಾತ್ಯರಲ್ಲಿ ಅತ್ಯಂತ ಸುಂಸ್ಕೃತರಾದವರ ರೀತಿನೀತಿಗಳಿಗೆ ಸಮನಾದದ್ದು. ಒಬ್ಬ ಸಂಗಾತಿಯಾಗಿ ಅವರು ಅತ್ಯಂತ ಮನಮೋಹಕ ವ್ಯಕ್ತಿ. ಒಬ್ಬ ಸಂಭಾಷಣ ಚತುರನಾಗಿ ಅವರು ಇಡೀ ಪಾಶ್ಚಾತ್ಯ ಜಗತ್ತಿನಲ್ಲೇ ಅದ್ವಿತೀಯರಾದವರು. ಅವರು ಇಂಗ್ಲಿಷ್​ನ್ನು ಸ್ಪಷ್ಟವಾಗಿ ಮಾತ್ರವಲ್ಲದೆ ನಿರರ್ಗಳವಾಗಿ ಮಾತನಾಡುತ್ತಾರೆ. ಹೊಚ್ಚಹೊಸದೂ ಪ್ರಖರವೂ ಆದ ಅವರ ಭಾವನೆಗಳು, ದಿಗ್ಭ್ರಮೆಗೊಳಿಸುವಷ್ಟು ಚೆನ್ನಾಗಿ ಅಲಂಕಾರಮಯ ಭಾಷೆಯ ಮೂಲಕ ಅವರ ನಾಲಿಗೆಯಿಂದ ಹೊರಬೀಳುತ್ತವೆ...”

ಮೆಂಫಿಸ್​ನಲ್ಲಿ ಸ್ವಾಮೀಜಿ ಒಂಬತ್ತು ದಿನ ಇದ್ದರು. ಈ ಅವಧಿಯಲ್ಲಿ ಅವರು ಹಲವಾರು ಭಾಷಣಗಳನ್ನು ಮಾಡಿದರು. ಅವುಗಳನ್ನು ಅಲ್ಲಿನ ಪತ್ರಿಕೆಗಳು ಉತ್ಸಾಹದಿಂದ ವಿಮರ್ಶಿಸಿ ವರದಿ ಮಾಡಿದುವು. ಹಿಂದೂಧರ್ಮದ ಕುರಿತಾದ ಅವರ ಒಂದು ಭಾಷಣದಲ್ಲಿ ವ್ಯಕ್ತವಾದ ಸರ್ವಧರ್ಮ ಸಹಿಷ್ಣುತೆಯ ಅಂಶವನ್ನು ಮರುದಿನದ ಪತ್ರಿಕೆಗಳಲ್ಲಿ ಪ್ರಶಂಸಿಸಲಾಯಿತು.ಒಂದು ಪತ್ರಿಕೆ ಬರೆಯಿತು:

“–ಹೌದು, ಅವರು ಬೋಧಿಸುತ್ತಾರೆ; ಅವರ ಪ್ರತಿಯೊಂದು ಮಾತೂ, ಅವರ ಪ್ರತಿ ಯೊಂದು ಪ್ರಖರ-ಜೀವಂತ ಭಾವನೆಯೂ ಸಹಿಷ್ಣುತೆ, ಸತ್ಯ ಹಾಗೂ ಪವಿತ್ರತೆಯ ಸಂದೇಶ. ಸುಲಲಿತ ವಾಗ್ಝರಿಯ ತರ್ಕಬದ್ಧ ವಾಕ್​ಸರಣಿಯ ಪಂಡಿತರಾದ ಅವರ ಬೆರಳತುದಿಗಳಲ್ಲಿ ಬೈಬಲ್, ಕೊರಾನ್, ವೇದಗಳು ಅಡಕವಾಗಿವೆ. ವಿದ್ಯುತ್ತಿನಂಥ ಚುರುಕುತನದ ಸ್ಫುಟವಾದ ಮನಸ್ಸು. ಅವರೊಬ್ಬ ಕಲಾಕಾರ, ಪಂಡಿತ, ಭಗವಂತನ ಶ್ರೇಷ್ಠತಮ ಸೇವಕ.

“ಅವರು ತಾವು ಆಲೋಚಿಸುತ್ತಾರೆ, ನಿಮ್ಮನ್ನೂ ಆಲೋಚಿಸುವಂತೆ ಮಾಡುತ್ತಾರೆ. ಅವರ ಮಾತನ್ನು ಕೇಳಿದರೆ ಖಂಡಿತ ನಿಮಗೆ ಒಳಿತಾಗಲೇಬೇಕು. ಅವರ ಭಾಷಣದ ಬಗ್ಗೆ ವಿವರವಾಗಿ ಬರೆಯಬೇಕೆಂದರೆ, ಪುಟಗಟ್ಟಲೆ ವಿವರಿಸಬೇಕಾಗುತ್ತದೆ. ಒಂದು ಸಾಧಾರಣ ವರದಿಯಲ್ಲಿ ಅವರ ಭಾಷಣದ ಮೇಲ್ಮೈನೋಟವನ್ನೂ ಕೊಡಲು ಸಾಧ್ಯವಿಲ್ಲ. ಅವರ ಭಾಷಣವನ್ನು ಸಭಿಕರು ತುಂಬ ಆಸಕ್ತಿಯಿಂದ ಆಲಿಸಿದರು ಎಂದಷ್ಟೇ ಹೇಳಿದರೆ, ಅದು ತೀರ ಕಡಿಮೆಯಾಗುತ್ತದೆ.

“ಅವರ ಉಪನ್ಯಾಸವು ಉನ್ನತ ಮಟ್ಟದ್ದಾಗಿದ್ದರೂ ಅದು ಸಭಿಕರ ಬುದ್ಧಿಮಟ್ಟಕ್ಕೆ ಎಟುಕಲಾರ ದಷ್ಟು ಕಠಿಣವಾಗಿರಲಿಲ್ಲ. ನಮ್ಮಲ್ಲಿನ ಅತ್ಯಂತ ಬುದ್ಧಿವಂತ ಜನರು ಸಭಿಕರಾಗಿದ್ದರು... ಅವರ ಅಭಿಪ್ರಾಯಗಳನ್ನು ಕೆಲವರು ಅನುಮೋದಿಸಿರಬಹುದು; ಇನ್ನು ಕೆಲವರು ಒಪ್ಪದಿದ್ದಿರ ಬಹುದು. ಆದರೆ ‘ಸಹಿಷ್ಣುತೆಯೇ ನಿಜವಾದ ಹಿರಿಮೆಯ ಲಕ್ಷಣ’ ಎಂಬುದನ್ನು ಎಲ್ಲರೂ ಒಪ್ಪಿ ಕೊಂಡರು. ವಿವೇಕಾನಂದರಂತಹ ಇನ್ನೂ ಹಲವಾರು ರಾಯಭಾರಿಗಳ ಭೇಟಿಯು, ಇಂದಿನ ಈ ಸ್ವಾರ್ಥತೆಯ ಯುಗದಲ್ಲಿ ಒಂದು ವರವಾಗಿ ಪರಿಣಮಿಸುತ್ತದೆ.”

ಅದೇ ದಿನ ಸಂಜೆ ಸ್ವಾಮೀಜಿ ಮತ್ತೊಂದೆಡೆ “ಮಾನವನ ವಿಧಿ” ಎಂಬ ವಿಷಯವಾಗಿ ಮಾತ ನಾಡಿದರು. ಅದನ್ನು ‘ಅಪೀಲ್ ಅವಲಾನ್ಷ್​’ ಪತ್ರಿಕೆಯಲ್ಲಿ ಹೀಗೆ ವಿಮರ್ಶಿಸಲಾಯಿತು:

“ಅಮೆರಿಕದ ಭಾಷಣಕಾರರಿಗಿಂತ ಈ ವಾಗ್ಮಿ (ಸ್ವಾಮೀಜಿ) ಒಂದು ವಿಷಯದಲ್ಲಿ ಸಂಪೂರ್ಣ ವಿಭಿನ್ನ. ಗಣಿತದ ಪ್ರಾಧ್ಯಾಪಕರು ಬೀಜಗಣಿತದ ಪ್ರಮೇಯವನ್ನು ಎಷ್ಟು ಎಚ್ಚರಿಕೆಯಿಂದ, ಶಾಸ್ತ್ರೀಯವಾಗಿ ನಿರೂಪಿಸುತ್ತಾರೆಯೋ, ವಿವೇಕಾನಂದರು ತಮ್ಮ ವಾದವನ್ನೂ ಅಷ್ಟೇ ಶಿಸ್ತುಬದ್ಧ ವಾಗಿ ಮಂದಿಡುತ್ತಾರೆ. ವಿವೇಕಾನಂದರು ತಮ್ಮ ಅಂತಶ್ಶಕ್ತಿಯಲ್ಲಿ ಸಂಪೂರ್ಣ ವಿಶ್ವಾಸದಿಂದ, ಎಲ್ಲ ಬಗೆಯ ವಾದವನ್ನೂ ತಳ್ಳಿಹಾಕುವ ರೀತಿಯಲ್ಲಿ ಮಾತನಾಡುತ್ತಾರೆ. ತಾವು ಮುಂದಿಟ್ಟ ಯಾವುದೇ ಅಭಿಪ್ರಾಯವನ್ನೂ, ತಾವು ಒತ್ತಿಹೇಳುವ ಯಾವುದೇ ಅಂಶವನ್ನೂ ಅವರು ತರ್ಕಬದ್ಧ ವಾಗಿ ನಿರೂಪಿಸದಿರುವುದಿಲ್ಲ... ಅವರು ಹೇಳುತ್ತಾರೆ–‘ದೇವರು ವಿಶ್ವದ ಯಾವುದೋ ಮೂಲೆಯಲ್ಲಿ ಕುಳಿತುಕೊಂಡು, ಮನುಷ್ಯರಿಗೆ ಅವರವರ ಕೃತ್ಯಗಳಿಗೆ ಅನುಸಾರವಾಗಿ ಶಿಕ್ಷೆ ಯನ್ನೋ ಬಹುಮಾನವನ್ನೋ ಕೊಡುತ್ತಿರುವ ರಾಜನಲ್ಲ. ಮನುಷ್ಯನ ಸತ್ಯವನ್ನರಿತು ಎದ್ದು ನಿಂತು ‘ನಾನೇ ದೇವರು’ ಎಂದು ಘೋಷಿಸುವ ಕಾಲ ಬರುತ್ತದೆ. ನಮ್ಮ ನಿಜಸ್ವರೂಪವೇ, ನಮ್ಮ ಅಮರತ್ವವೇ ದೇವರಾಗಿರುವಾಗ, ದೇವರು ಎಲ್ಲೋ ದೂರದಲ್ಲಿದ್ದಾನೆ ಎಂದು ಬೋಧಿ ಸುವುದೇಕೆ?’ ಎಂದು. ಧರ್ಮದ ಬಗ್ಗೆ ಅವರು ಹೇಳುತ್ತಾರೆ–‘ಧರ್ಮವು ಮನುಷ್ಯನ ದೌರ್ಬಲ್ಯದಿಂದ ಉದಿಸಿದುದಲ್ಲ; ಅಥವಾ, ಯಾವುದೋ ಘೋರ ಉಪದ್ರವಕ್ಕೆ ಹೆದರಿ ಅವನು ಮಾಡಿಕೊಂಡದ್ದಲ್ಲ. ಧರ್ಮವು ಪ್ರೇಮದ ವಿಕಸನ, ಬೆಳವಣಿಗೆ’ ಎಂದು.”

ಜನವರಿ ೨೨ರಂದು ಸ್ವಾಮೀಜಿ ಮೆಂಫಿಸ್​ನಿಂದ ಹೊರಟು ಶಿಕಾಗೋ ಸೇರಿದರು. ೨೫ರಂದು ಇಲ್ಲಿ ಅವರದೊಂದು ಭಾಷಣದ ಕಾರ್ಯಕ್ರಮವಿತ್ತು. ಮತ್ತೆ ಫೆಬ್ರುವರಿ ೧೩ರಂದು ಇಲ್ಲಿಂದ ಹೊರಟು ಟ್ರೈನಿನಲ್ಲಿ ೨೭ಂಮೈಲಿ ಪ್ರಯಾಣ ಮಾಡಿ ಮಿಷಿಗನ್ ರಾಜ್ಯದ ಡೆಟ್ರಾಯ್ಟ್​ಗೆ ಬಂದರು. ಡೆಟ್ರಾಯ್ಟ್​ನಲ್ಲಿನ ಅವರ ಕಾರ್ಯವು ಅಮೆರಿಕದಲ್ಲಿನ ಅವರ ಕಾರ್ಯಕ್ರಮಗಳಲ್ಲೇ ಅತ್ಯಂತ ಪ್ರಾಮುಖ್ಯದ್ದಾಗಿ ಪರಿಣಮಿಸಿತು. ಅಮೆರಿಕದ ಅಗ್ರಗಣ್ಯ ಕೈಗಾರಿಕಾ ನಗರವಾದ ಡೆಟ್ರಾಯ್ಟ್, ಅತ್ಯಂತ ಉತ್ಸಾಹಯುತರಾದ ನಾಗರಿಕರಿಂದ ಕೂಡಿದ್ದು, ಸ್ವಾಮೀಜಿಯ ಸಂದೇಶ ಗಳಿಗೆ ಇಲ್ಲಿ ಅತ್ಯುತ್ತಮ ಪ್ರತಿಕ್ರಿಯೆ ವ್ಯಕ್ತವಾಯಿತು.

ಡೆಟ್ರಾಯ್ಟ್​ನಲ್ಲಿ ಸ್ವಾಮೀಜಿ ಸರ್ವಧರ್ಮ ಸಮ್ಮೇಳನದ ಸಮಯದಲ್ಲಿ ತಮ್ಮನ್ನು ಭೇಟಿ ಯಾಗಿದ್ದ ಶ್ರೀಮತಿ ಜಾನ್ ಜೆ. ಬ್ಯಾಗ್​ಲೀ ಎಂಬ ಅತ್ಯಂತ ಪ್ರಭಾವಶಾಲೀ ಮಹಿಳೆಯ ಅತಿಥಿಗಳಾಗಿದ್ದರು. ಈಕೆ ಮಿಷಿಗನ್ನಿನ ರಾಜ್ಯಪಾಲರಾಗಿದ್ದ ದಿವಂಗತ ಶ್ರೀ ಬ್ಯಾಗ್​ಲೀಯವರ ಪತ್ನಿ; ಅತ್ಯಂತ ಸುಸಂಸ್ಕೃತೆ ಹಾಗೂ ಅಸಾಮಾನ್ಯ ಆಧ್ಯಾತ್ಮಿಕ ಪ್ರವೃತ್ತಿಯ ಮಹಿಳೆ. ಬ್ಯಾಗ್​ಲೀ ಕುಟುಂಬದವರು ಸ್ವಾಮೀಜಿಯನ್ನು ಅತ್ಯಂತ ಆದರಿಂದ ಸ್ವಾಗತಿಸಿದರು. ಡೆಟ್ರಾಯ್ಟ್​ನಲ್ಲಿ ಸ್ವಾಮೀಜಿ ಹೆಚ್ಚುಕಡಿಮೆ ೧೮೯೪ ಮಾರ್ಚ್ ಅಂತ್ಯದವರೆಗೂ ಇದ್ದರು.

ಸ್ವಾಮೀಜಿ ಡೆಟ್ರಾಯ್ಟ್​ಗೆ ಆಗಮಿಸಿದ ಸುದ್ದಿಯನ್ನು ಪತ್ರಿಕೆಗಳು ಮುಖ್ಯಸುದ್ದಿಯಾಗಿ ಪ್ರಕಟಿ ಸಿದುವು. ಆ ದಿನ ಸಂಜೆ ಶ್ರೀಮತಿ ಬ್ಯಾಗ್​ಲೀ ಸ್ವಾಮೀಜಿಗಾಗಿ ಅತ್ಯಂತ ವೈಭವೋಪೇತವಾದ ಸಂತೋಷಕೂಟವೊಂದನ್ನು ಏರ್ಪಡಿಸಿದರು. ಡೆಟ್ರಾಯ್ಟ್​ನ ಪ್ರತಿಯೊಬ್ಬ ಗಣ್ಯವ್ಯಕ್ತಿಯನ್ನೂ ಇದಕ್ಕೆ ಆಹ್ವಾನಿಸಲಾಗಿತ್ತು. ಬ್ಯಾಗ್​ಲೀಯವರ ಭವ್ಯ ಬಂಗಲೆಯಲ್ಲಿ ಕಾರ್ಯಕ್ರಮ ನಡೆಯಿತು. ಪಾದ್ರಿಗಳು, ಧರ್ಮಬೋಧಕರು, ಪ್ರಾಧ್ಯಾಪಕರು, ನಗರದ ಮೇಯರ್ ಮೊದಲಾದವರು ಸೇರಿದಂತೆ ಕಡೆಯ ಪಕ್ಷ ಮುನ್ನೂರು ಜನರನ್ನು ಆಹ್ವಾನಿಸಲಾಗಿತ್ತು.

ಆದರೆ ಸ್ವಾಮೀಜಿಯ ಬೋಧನೆಗಳಿಗೆ ಸಮ್ಮೇಳನದ ದಿನಗಳಿಂದಲೇ ವ್ಯಕ್ತವಾಗಿದ್ದ ವಿಷನರಿ ಗಳ ವಿರೋಧ, ಅವರ ಜನಪ್ರಿಯತೆಯೊಂದಿಗೆ ಹೆಚ್ಚುತ್ತ ಬಂದಿತ್ತು. ಅವರು ಡೆಟ್ರಾಯ್ಟ್​ಗೆ ಬರುವ ವೇಳೆಗೆ ಅಪಪ್ರಚಾರ ಅತ್ಯಂತ ತೀವ್ರವಾಗಿತ್ತು. ಆದ್ದರಿಂದ ವಿವೇಕಾನಂದರಂತಹ ‘ವಿಧರ್ಮಿ’ಯನ್ನು ಅತಿಥಿಯಾಗಿರಿಸಿಕೊಂಡಿದ್ದರಿಂದ ನಗರದ ಸಂಪ್ರದಾಯಸ್ಥ ಕ್ರೈಸ್ತರೆಲ್ಲ ಶ್ರೀಮತಿ ಬ್ಯಾಗ್​ಲೀಯ ಬಗ್ಗೆ ಗುಸುಗುಸು ಮಾತನಾಡಿಕೊಂಡರು. ಆದರೆ ಯಾರಿಗೂ ಆಕೆಯನ್ನು ವಿರೋಧಿಸುವ ಧೈರ್ಯವಿರಲಿಲ್ಲ. ಅಲ್ಲದೆ ಆಕೆಯ ಆಮಂತ್ರಣವನ್ನೂ ನಿರಾಕರಿಸಲಾಗದೆ ಇವರಲ್ಲಿ ಎಷ್ಟೋ ಜನ ಸಂತೋಷಕೂಟದಲ್ಲಿ ಹಾಜರಿದ್ದರು. ಮತ್ತು ತನ್ಮೂಲಕ ಸ್ವಾಮೀಜಿಯ ಮಾತುಗಳನ್ನು ಕೇಳುವ ಸದವಕಾಶವನ್ನು ಹೊಂದಿದರು. ಈ ಸಭೆಗೆ ಶ್ರೀಮತಿ ಬ್ಯಾಗ್​ಲೀ ಎಲ್ಲ ಕ್ರೈಸ್ತಪಂಥಗಳಿಗೂ ಸೇರಿದ ಬುದ್ಧಿಜೀವಿಗಳನ್ನು, ಚಿಂತಕರನ್ನು ಆಹ್ವಾನಿಸಿದ್ದರು. ಸ್ವಾಮೀಜಿಗೆ ಈ ಎಲ್ಲ ಸಡಗರದಲ್ಲಿ ಪಾಲ್ಗೊಳ್ಳುವ ಇಚ್ಛೆಯಿತ್ತೋ ಇಲ್ಲವೋ ಎನ್ನುವುದು ಬೇರೆ ವಿಚಾರ. ಆದರೆ ಡೆಟ್ರಾಯ್ಟ್ ನಗರದ ಚರಿತ್ರೆಯಲ್ಲೇ ಅದೊಂದು ಅಪೂರ್ವ ಸಮಾರಂಭವೆಂಬುದರಲ್ಲಿ ಸಂದೇಹವಿರಲಿಲ್ಲ. ಮರುದಿನದ ಪತ್ರಿಕೆಗಳಲ್ಲಿ ಈ ಸಂಭ್ರಮದ ಸಮಾರಂಭದ ಬಗ್ಗೆ ವಿವರ ಪೂರ್ಣವಾದ ವರದಿ ಪ್ರಕಟವಾಯಿತು. ‘ಡೆಟ್ರಾಯ್ಟ್ ಜರ್ನಲ್​’ ಪತ್ರಿಕೆ ಬರೆಯಿತು\eng{–“The social lion of the day is Swami (brother) Vivekananda.”}

ಡೆಟ್ರಾಯ್ಟ್​ನಲ್ಲಿ ಸ್ವಾಮೀಜಿ ತಮ್ಮ ಭಾಷಣಗಳಿಂದಾಗಿ ಹಾಗೂ ತಮ್ಮ ವ್ಯಕ್ತಿತ್ವದಿಂದಾಗಿ ಎಂತಹ ಅಲ್ಲೋಲಕಲ್ಲೋಲವನ್ನೇ ಉಂಟುಮಾಡಿದರೆಂದರೆ, ಅವರ ಬಗ್ಗೆ ಅನೇಕ ಊಹಾ ಪೋಹಗಳು ಹುಟ್ಟಿಕೊಂಡಿದ್ದುವು. ಪ್ರತಿದಿನವೆಂಬಂತೆ ಅವರ ಹೆಸರು ಪತ್ರಿಕೆಗಳ ಮುಖಪುಟ ದಲ್ಲಿ ಪ್ರಕಟವಾಗುತ್ತಿತ್ತು. ಅವರೊಂದಿಗಿನ ಸಂದರ್ಶನಗಳೂ ಲೇಖನಗಳೂ ತುಂಬ ಸ್ವಾರಸ್ಯ ಕರವಾಗಿದ್ದು ಓದುಗರೆಲ್ಲರಲ್ಲಿ ಆಸಕ್ತಿ ಮೂಡಿಸಿದ್ದುವು.

ಶ್ರೀಮತಿ ಬ್ಯಾಗ್​ಲೀಯವರ ಅತಿಥಿಗಳಾಗಿದ್ದ ಸ್ವಾಮೀಜಿ, ಕೆಲದಿನಗಳ ಮಟ್ಟಿಗೆ, ಥಾಮಸ್ ಪಾಮರ್ ಎಂಬವರ ಆಹ್ವಾನವನ್ನು ಮನ್ನಿಸಿ ಅವರ ಮನೆಗೆ ಹೋದರು. ಇವರು ಜಾಗತಿಕ ಮೇಳದ ಉನ್ನತ ಅಧಿಕಾರಿಗಳಾಗಿದ್ದರಲ್ಲದೆ, ಹಿಂದೆ ಅಮೆರಿಕದ ಪಾರ್ಲಿಮೆಂಟಿನ ಸದಸ್ಯರೂ ಸ್ಪೆಯಿನ್ ದೇಶಕ್ಕೆ ರಾಯಭಾರಿಯೂ ಆಗಿದ್ದರು. ಶ್ರೀ ಪಾಮರ್ ಅವರು ಹಾಸ್ಯಶೀಲ ಉತ್ಸಾಹೀ ವ್ಯಕ್ತಿ. ಇವರ ಸ್ನೇಹಮಯ ವ್ಯಕ್ತಿತ್ವವನ್ನು ಸ್ವಾಮೀಜಿ ಬಹಳ ಇಷ್ಟಪಟ್ಟರು. ಆದರೆ ಇವರಿಬ್ಬರ ಸ್ನೇಹಸಂಬಂಧವನ್ನು ಕಂಡು ವೃತ್ತಪತ್ರಿಕೆಯೊಂದು ಅದ್ಭುತವಾದ ಸುದ್ದಿಯೊಂದನ್ನು ‘ನಿರ್ಮಿಸಿತು’–

“ಸುಂಟರಗಾಳೀ ಸಂನ್ಯಾಸಿಯು ಈಗ ಶ್ರೀ ಪಾಮರ್​ರವರ ಅತಿಥಿಗಳಾಗಿದ್ದಾರೆ. ಶ್ರೀ ಪಾಮರ್​ರವರು ಹಿಂದುವಾಗಿ ಮತಾಂತರ ಹೊಂದಿದ್ದಾರೆ. ಅವರು ಭಾರತಕ್ಕೆ ಹೋಗಲಿದ್ದಾರೆ. ಆದರೆ ಈ ಎರಡು ಸುಧಾರಣೆಗಳನ್ನು ಕಾರ್ಯಗತಗೊಳಿಸಬೇಕೆಂದು ಮಾತ್ರ ಅವರು ಒತ್ತಾಯಿಸು ತ್ತಿದ್ದಾರೆ–ಮೊದಲನೆಯದಾಗಿ, ಜಗನ್ನಾಥನ ರಥವನ್ನು ಎಳೆಯಲು ಪಾಮರ್​ರವರಿಗೆ ಸೇರಿದ ಕುದುರೆಗಳನ್ನೇ ಉಪಯೋಗಿಸಬೇಕು; ಎರಡನೆಯದಾಗಿ ‘ಜರ್ಸಿ’ ಹಸುಗಳನ್ನು ಹಿಂದೂಗಳ ಪವಿತ್ರ ಗೋವುಗಳ ಸಾಲಿಗೆ ಸೇರಿಸಬೇಕು.”

ಇದನ್ನು ಓದಿ ಸ್ವಾಮೀಜಿ ಹಾಗೂ ಪಾಮರ್ ಇಬ್ಬರೂ ತಲೆ ಚಚ್ಚಿಕೊಂಡರೋ ಅಥವಾ ಬಿದ್ದುಬಿದ್ದು ನಕ್ಕರೋ, ತಿಳಿಯದು. ಆದರೆ ಪಾಮರ್​ರವರ ಮನೆಯಲ್ಲಿ ಸ್ವಾಮೀಜಿ ಹೆಚ್ಚು ದಿನ ಉಳಿದುಕೊಳ್ಳಲು ಸಾಧ್ಯವಾಗಲಿಲ್ಲ. ಏಕೆಂದರೆ ಅವರು ಅಷ್ಟೊಂದು ದಿನ ಪಾಮರ್​ರವರ ಮನೆಯಲ್ಲಿ ಇದ್ದದ್ದರಿಂದ ತಮಗೆ ಅವರ ಆತಿಥೇಯರಾಗಿರುವ ಭಾಗ್ಯ ತಪ್ಪಿಹೋಯಿತಲ್ಲ ಎಂದು ಶ್ರೀಮತಿ ಬ್ಯಾಗ್​ಲೀ ಅವರನ್ನು ತಮ್ಮ ಮನೆಗೇ ಒತ್ತಾಯದಿಂದ ಮತ್ತೆ ಕರೆತಂದರು. ಪಾಮರ್​ರವರಿಗೆ ಇದರಿಂದ ದುಃಖವಾಯಿತಾದರೂ, ಇವರಿಬ್ಬರ ಮಧ್ಯೆ ಸ್ವಾಮೀಜಿ ಏನೂ ಮಾಡುವಂತಿರಲಿಲ್ಲ!

ಡೆಟ್ರಾಯ್ಟ್​ನಲ್ಲಿ ಸ್ವಾಮೀಜಿ ಫೆಬ್ರುವರಿ ೧೨ರಿಂದ ೨೩ರವರೆಗೂ ಮತ್ತೆ ಮಾರ್ಚ್ ೯ರಿಂದ ೩ಂರವರೆಗೂ ಇದ್ದರು. ಈ ಅವಧಿಯಲ್ಲಿ ಅವರು ಇಲ್ಲಿ ಒಟ್ಟು ಎಂಟು ಸಾರ್ವಜನಿಕ ಉಪನ್ಯಾಸಗಳನ್ನು ಮಾಡಿದರು. ಇದಲ್ಲದೆ ಹಲವಾರು ಅನೌಪಚಾರಿಕ ಸಭೆಗಳಲ್ಲಿ ಮಾತನಾಡಿ ದರು. ಈ ಭಾಷಣಗಳಿಂದಾಗಿ ಸ್ವಾಮೀಜಿಯ ಬೆಂಬಲಿಗರಾದ ಪುರೋಗಾಮಿಗಳಿಗೂ, ಬದ್ಧ ವಿರೋಧಿಗಳಾದ ಸಂಪ್ರದಾಯಸ್ಥರಿಗೂ ವೃತ್ತಪತ್ರಿಕೆಗಳ ಮುಖಾಂತರ ತೀವ್ರ ಸೆಣಸಾಟಗಳು ಪ್ರಾರಂಭವಾದುವು. ಹಲವು ಪತ್ರಿಕೆಗಳು ಸ್ವಾಮೀಜಿಗೆ ಬೆಂಬಲವಾಗಿ ಲೇಖನಗಳನ್ನೂ ಸಂಪಾದ ಕೀಯಗಳನ್ನೂ ಬರೆದರೆ ಇನ್ನು ಕೆಲವು ಅವರ ಬಾಯಿ ಮುಚ್ಚಿಸಲು ಸೆಣಸಾಡಿದುವು. ಸರ್ವ ಧರ್ಮಸಮ್ಮೇಳನದಲ್ಲಿ ಸ್ವಾಮೀಜಿ ಭಾರತದ ಹಿರಿಮೆಯನ್ನು ಎತ್ತಿಹಿಡಿದು, ಅಮೆರಿಕನ್ನರಲ್ಲಿ ಮನೆ ಮಾಡಿಕೊಂಡಿದ್ದ ತಪ್ಪು ಭಾವನೆಗಳನ್ನು ಹೋಗಲಾಡಿಸಲು ಆರಂಭಿಸಿದಂದಿನಿಂದಲೂ ಅವರ ಬಗ್ಗೆ ಕ್ರೈಸ್ತ ಪ್ರಚಾರಕರಲ್ಲಿ ಅಸಮಾಧಾನ ಒಳಗೊಳಗೇ ಕುದಿಯುತ್ತಿತ್ತು. ಈಗ ಡೆಟ್ರಾಯ್ಟ್​ನಲ್ಲಿ ಅದಿನ್ನು ಹತ್ತಿಕ್ಕಲಾರದಂತೆ ಸ್ಫೋಟವಾಯಿತು. ಇದಕ್ಕೆ ಕಾರಣವೆಂದರೆ ಸ್ವಾಮೀಜಿ ಮತ್ತಷ್ಟು ಬಿಚ್ಚುಮಾತುಗಳನ್ನು ಆಡಲು ಆರಂಭಿಸಿದುದು. ಅವರ ಮಾತುಗಳ ಮೂಲಕ ಪ್ರಚಂಡ ಶಕ್ತಿ ಹೊರಹೊಮ್ಮುವಂತಿತ್ತು. ಇಲ್ಲಿ ಅವರ ಪ್ರತಿಯೊಂದು ಭಾಷಣವನ್ನೂ ಆಲಿಸಿದ, ಮುಂದೆ ಅವರ ಅತ್ಯಂತ ಆಪ್ತಶಿಷ್ಯೆಯಾದ ಕುಮಾರಿ ಕ್ರಿಸ್ಟೀನ ಬರೆಯುತ್ತಾಳೆ, “ಈ ನಿಗೂಢವ್ಯಕ್ತಿಯಿಂದ ಹೊರಹೊಮ್ಮುತ್ತಿದ್ದ ಶಕ್ತಿ ಎಷ್ಟು ಪ್ರಬಲವಾಗಿತ್ತೆಂದರೆ ಅದನ್ನು ಕಂಡ ವರು ಅದನ್ನೆದುರಿಸಲಾರದೆ ಕುಸಿದುಹೋಗುತ್ತಿದ್ದರು. ಅದು ಅಪ್ರತಿಮವಾಗಿತ್ತು.” ಈ ಊರಿನ ಮಿಸ್ ಮಾರ್ಗರೆಟ್ ಕುಕ್ ಎಂಬ ಜರ್ಮನ್ ಮಹಿಳೆ ಸ್ವಾಮೀಜಿಯ ಉಪನ್ಯಾಸವೊಂದರಲ್ಲಿ ಹಾಜರಿದ್ದಳು. ಜರ್ಮನರು ಸ್ವಭಾವತಃ ಗಡಸು. ಈಕೆಯೂ ಅದಕ್ಕೆ ಹೊರತೇನಲ್ಲ. ಇವರಲ್ಲಿ ಸೂಕ್ಷ್ಮ ಭಾವನೆಗಳನ್ನು ಪ್ರಚೋದಿಸುವುದು ಬಹಳ ಕಷ್ಟ. ಆದರೆ ಸ್ವಾಮೀಜಿಯ ಭಾಷಣವನ್ನು ಆಲಿಸಿದ ಮಿಸ್ ಕುಕ್​ಳಿಗೆ, ಅವಳ ಜೀವನದಲ್ಲೇ ಮೊದಲಬಾರಿಗೆ, ಭಾಷಣಕಾರನನ್ನು ಅಭಿ ನಂದಿಸಬೇಕೆಂಬ ಉತ್ಕಟೇಚ್ಛೆಯಾಯಿತು. ಆದ್ದರಿಂದ ಅನಂತರ ಅವಳು ಹೋಗಿ ಅವರ ಕೈ ಕುಲುಕಿದಳು. ಆದರೆ ಏನಾದರೂ ಹೇಳಬೇಕೆಂದು ನೋಡಿದಾಗ ಇದ್ದಕ್ಕಿದ್ದಂತೆ ಆಕೆ ಭಾವೋದ್ರೇಕ ಗೊಂಡಳು. ಆಕೆಯ ಬಾಯಿಯಿಂದ ಮಾತುಗಳೇ ಹೊರಡಲಿಲ್ಲ! ಹಲವಾರು ಕ್ಷಣಗಳವರೆಗೆ ಸ್ವಾಮೀಜಿಯ ಕೈಹಿಡಿದು ಆಕೆ ಗೊಂಬೆಯಂತೆ ನಿಂತುಬಿಟ್ಟಳು. ಅನೇಕ ವರ್ಷಗಳಾದಮೇಲೆ ಅವಳು ಹೇಳುತ್ತಾಳೆ–“ಅವರ ಅನ್ವೇಷಕ ನೋಟವನ್ನು ನಾನೆಂದೂ ಮರೆಯಲಾರೆ. ಅವರ ಹಿರಿಮೆಯೂ ಪವಿತ್ರತೆಯೂ ನನಗೆ ಎಷ್ಟು ಸ್ಪಷ್ಟವಾಗಿ ಅರಿವಾಗಿತ್ತೆಂದರೆ, ಮೂರು ದಿನ ಗಳವರೆಗೆ ಕೈತೊಳೆದುಕೊಳ್ಳಲು ಮನಸ್ಸು ಒಪ್ಪುತ್ತಲೇ ಇರಲಿಲ್ಲ!”\footnote{* ಸ್ವಾಮೀಜಿಯವರ ವ್ಯಕ್ತಿತ್ವ ಅದೆಷ್ಟು ಪ್ರಭಾವಶಾಲಿಯಾಗಿತ್ತು ಎಂಬುದಕ್ಕೆ ಅವರನ್ನು ಕಣ್ಣಾರೆ ಕಂಡು ಮಾತನಾಡಿಸಿದ್ದ ಒಬ್ಬ ವ್ಯಕ್ತಿಯ ಅನುಭವವನ್ನು ಉದಾಹರಿಸಬಹುದು. ಇವರು ತಮ್ಮ ತಾರುಣ್ಯದಲ್ಲಿ ವಿವೇಕಾ ನಂದರನ್ನು ಕಂಡಿದ್ದವರು. ಈಚೆಗೆ, ಎಂದರೆ ೧೯೫ಂ-೬ಂರ ನಡುವೆ, ಇವರನ್ನು ಒಮ್ಮೆ ಭೇಟಿಯಾಗುವ ಅವಕಾಶ ದೊರೆತಾಗ, ರಾಮಕೃಷ್ಣ ಮಠದ ಬ್ರಹ್ಮಚಾರಿಗಳೊಬ್ಬರು ಕೇಳಿದರಂತೆ, “ಮಹಾಶಯರೆ, ನೀವು ವಿವೇಕಾನಂದರನ್ನು ಕಂಡಿರಲ್ಲ, ಆಗ ನಿಮಗಾದ ಅನುಭವವೇಂಥದು?” ಎಂದು.

ಅದಕ್ಕೆ ಆ ವ್ಯಕ್ತಿ ಗಂಭೀರ ಸ್ವರದಲ್ಲಿ ಮರುಪ್ರಶ್ನೆ ಹಾಕಿದರಂತೆ, “ನೀವು ಸಿಂಹವನ್ನು ನೋಡಿದ್ದೀರಾ?” “ಹೌದು, ನೋಡಿದ್ದೇನೆ,” “ಎಲ್ಲಿ?” “ಮೃಗಾಲಯದಲ್ಲಿ.” “ಸರಿ. ಅದರೊಂದಿಗೆ ಕೈಕುಲುಕಿದ್ದೀರಾ?”

ಸಿಂಹದೊಂದಿಗೆ ಕೈಕುಲುಕುವುದು!!

ಸ್ವಾಮೀಜಿಯವರ ವ್ಯಕ್ತಿತ್ವದಿಂದ ಹೊಮ್ಮುತ್ತಿದ್ದ ಶಕ್ತಿ ಇಂಥದು.}

ಸ್ವಾಮೀಜಿ ಮಾಡಿದ ಮೊದಲ ಭಾಷಣದಿಂದಲೇ ವಿವಾದ ಪ್ರಾರಂಭವಾಯಿತು. ಫೆಬ್ರುವರಿ ೧೪ರಂದು ಅವರು ಅಲ್ಲಿನ ‘ಯೂನಿಟೇರಿಯನ್ ಚರ್ಚ್​’ನಲ್ಲಿ ಭಾರತ ಹಾಗೂ ಹಿಂದೂ ಧರ್ಮದ ಬಗ್ಗೆ ಮಾತನಾಡಿದರು. ರೆ ॥ ನಿಂಡೆ ಎಂಬ ಪಾದ್ರಿಯೊಬ್ಬರು ಅವರನ್ನು ಪರಿಚಯಿಸಿ ದರು. ಬಳಿಕ ಸ್ವಾಮೀಜಿ ಭಾರತೀಯರ ಬಗ್ಗೆ ಹಾಗೂ ಹಿಂದೂಧರ್ಮದ ಬಗ್ಗೆ ಸಂಕ್ಷೇಪವಾಗಿ ಮಾತನಾಡಿ, ಹಿಂದೂಗಳು ಸ್ವಭಾವತಃ ಅತ್ಯಂತ ನೀತಿವಂತ ಜನಾಂಗವೆಂದೂ ಅವರಲ್ಲಿಗೆ ಯಾವುದೇ ಬಗೆಯ ಧರ್ಮಪ್ರಚಾರಕರನ್ನು ಕಳಿಸಿಕೊಡುವ ಅವಶ್ಯಕತೆಯಿಲ್ಲವೆಂದೂ ಘೋಷಿಸಿದರು.

ಸ್ವಾಮೀಜಿಯ ಮಾತಿನಿಂದ ಧರ್ಮಾಂಧ ಕ್ರೈಸ್ತರು ಕುಪಿತರಾದರು. ತಕ್ಷಣವೇ ಅವರ ಮಾತು ಗಳನ್ನು ಖಂಡಿಸಿದರಲ್ಲದೆ, ಅವರನ್ನು ಪರಿಚಯಿಸಿದ ಬಿಷಪ್ ನಿಂಡೆಯವರನ್ನೂ ತರಾಟೆಗೆ ತೆಗೆದುಕೊಂಡರು. ಬಿಷಪ್ ನಿಂಡೆ, ಕೆಲವೇ ದಿನಗಳಲ್ಲಿ ತಮ್ಮ ಧರ್ಮಪ್ರಚಾರಕಾರ್ಯಕ್ಕಾಗಿ ಪೌರ್ವಾತ್ಯ ರಾಷ್ಟ್ರಗಳಿಗೆ ಹೊರಟಿದ್ದವರು. ಈ ಸಂದರ್ಭದಲ್ಲಿ ತಾವು ಸಂಪ್ರದಾಯಸ್ಥರ ಸಿಟ್ಟಿಗೆ ಗುರಿಯಾದದ್ದನ್ನು ಕಂಡು ತಬ್ಬಿಬ್ಬಾಗಿಹೋದರು. ತಕ್ಷಣವೇ ಅವರು ಪತ್ರಿಕೆಯೊಂದಕ್ಕೆ ಪತ್ರ ಬರೆದು, ವಿವೇಕಾನಂದರು ತಮಗೆ ವಿಶ್ವಾಸದ್ರೋಹ ಮಾಡಿದರೆಂದು ಆಪಾದಿಸಿ, ತಾವು ಸಂಪ್ರ ದಾಯಸ್ಥರ ಕೋಪದಿಂದ ತಪ್ಪಿಸಿಕೊಂಡರು! ಆದರೆ ಅವರ ಈ ನಿಲುವನ್ನು ಅಲ್ಲಿನ ವಿಚಾರವಂತ ವ್ಯಕ್ತಿಗಳು ಉಗ್ರವಾಗಿ ಖಂಡಿಸಿ, ಸ್ವಾಮೀಜಿಯನ್ನು ಬೆಂಬಲಿಸಿದರು. ಹೀಗೆ ಪ್ರಾರಂಭವಾದ ಸೆಣಸಾಟ, ಅವರ ಪ್ರತಿಯೊಂದು ಉಪನ್ಯಾಸದೊಂದಿಗೂ ಹೆಚ್ಚುತ್ತಹೋಯಿತು. ಸ್ವಾಮೀಜಿ ತಾವು ಹೇಳಬೇಕೆಂದಿದ್ದುದನ್ನು ಸ್ಪಷ್ಟವಾಗಿ ಹೇಳಿದರು. ಆದರೆ ಅಪಪ್ರಚಾರದ ವಿರುದ್ಧವಾಗಿ ತಮ್ಮನ್ನೆಂದೂ ಸಮರ್ಥಿಸಿಕೊಳ್ಳುವ ಪ್ರಯತ್ನ ಮಾಡಲಿಲ್ಲ. ಅಥವಾ ತಮ್ಮ ನಿಂದಕರನ್ನು ಸಮಾಧಾನಗೊಳಿಸಿ ರಾಜಿಯನ್ನೂ ಮಾಡಿಕೊಳ್ಳಲಿಲ್ಲ. ಸಂನ್ಯಾಸಿಯಾದವನು ಆತ್ಮರಕ್ಷಣೆ ಮಾಡಿ ಕೊಳ್ಳಬಾರದೆಂದು ಅವರು ನಂಬಿದ್ದರು. ಅವರೊಮ್ಮೆ ಹೇಳಿದರು, “ನಾನು ಸ್ವಲ್ಪ ಬಿಚ್ಚುಮಾತಿ ನವನು. ಆದರೆ ನನ್ನ ಉದ್ದೇಶ ಒಳ್ಳೆಯದು. ನಾನು ನಿಮ್ಮನ್ನು ಹೊಗಳಿ ಉಬ್ಬಿಸಲು ಇಲ್ಲಿಗೆ ಬಂದಿಲ್ಲ. ಅದಲ್ಲ ನನ್ನ ಕೆಲಸ. ಹಾಗೆ ಮಾಡುವುದೇ ನನ್ನ ಉದ್ದೇಶವಾಗಿದ್ದರೆ, ನ್ಯೂಯಾರ್ಕಿನ ಒಂದು ಪ್ರಮುಖ ರಸ್ತೆಯಲ್ಲಿ ಅತ್ಯಾಧುನಿಕವಾದ, ಥಳುಕಿನ ಭವನವೊಂದನ್ನು ತೆರೆಯುತ್ತಿದ್ದೆ. ನೀವು ನನ್ನ ಮಕ್ಕಳು. ನಿಮಗೆ ನಿಮ್ಮ ಲೋಪದೋಷಗಳನ್ನು, ಪೊಳ್ಳುತನವನ್ನು ತೋರಿಸಿಕೊಡು ವುದರ ಮೂಲಕ, ನಾನು ನಿಮಗೆ ನಿಮ್ಮ ಆ ಅಲ್ಪವ್ಯಕ್ತಿತ್ವವನ್ನು ಮೀರಿ ದೈವತ್ವದೆಡೆಗೆ ನಡೆಯುವ ದಾರಿಯನ್ನು ತೋರಿಸುತ್ತಿದ್ದೇನೆ. ಆದ್ದರಿಂದ ನಾನು ಈಗಿನ ಕ್ರೈಸ್ತಧರ್ಮವನ್ನು ಅಥವಾ ನಿಮ್ಮ ನಾಗರಿಕತೆಯ ಆದರ್ಶಗಳನ್ನು, ಇಲ್ಲವೆ ಪಾಶ್ಚಾತ್ಯ ನೈತಿಕ ಮಾನದಂಡದಿಂದ ರೂಪಿಸಲ್ಪಟ್ಟ ನಿಮ್ಮ ವಿಚಿತ್ರ ರೀತಿನೀತಿಗಳನ್ನು ಹೊಗಳಲಾರೆ.” ಡೆಟ್ರಾಯ್ಟ್​ನಲ್ಲಿ ಸ್ವಾಮೀಜಿ ಮಾಡಿದ ಈ ಉಪನ್ಯಾಸದಲ್ಲಿ ಅವರ ಮುಚ್ಚುಮರೆಯಿಲ್ಲದ ನೇರಸ್ವಭಾವದ ಹಾಗೂ ಕಟುತರ ವಿಮರ್ಶಾ ಪ್ರಜ್ಞೆಯ ಅರಿವಾಗುತ್ತದೆ:

“ನಾನು ನಿಮಗೊಂದು ಮಾತು ಹೇಳುತ್ತೇನೆ; ಆದರೆ ಅದನ್ನೊಂದು ನಿಷ್ಠುರದ ಟೀಕೆಯೆಂದು ತಿಳಿಯಬೇಡಿ. ಈಗ ನೀವು ಮಿಷನರಿಗಳನ್ನು ಚೆನ್ನಾಗಿ ಸಲಹಿ, ತರಬೇತಿ ನೀಡಿ, ಹಣಕೊಟ್ಟು ಕಳಿಸುತ್ತಿದ್ದೀರಿ. ಆದರೆ ಏತಕ್ಕೆ?–ನಮ್ಮ ದೇಶಕ್ಕೆ ಬಂದು ನಮ್ಮ ಪೂರ್ವಜರನ್ನು, ನಮ್ಮ ಧರ್ಮವನ್ನು, ನಮ್ಮ ಎಲ್ಲವನ್ನೂ ಶಪಿಸುವುದಕ್ಕಾಗಿ, ದೂರವುದಕ್ಕಾಗಿ. ಅವರು ದೇವಾಲಯ ವೊಂದರ ಬಳಿಗೆ ಬರುತ್ತಾರೆ; ಅಲ್ಲಿ ನಿಂತು ಕೂಗಿ ಹೇಳುತ್ತಾರೆ–‘ಓ, ವಿಗ್ರಹಾರಾಧಕರೆ! ನೀವು ನರಕಕ್ಕೆ ಹೋಗುತ್ತೀರಿ!’ ಎಂದು. ಆದರೆ ಹಿಂದು ಸಾಧುಸ್ವಭಾವದವನು. ಅವನು ಸುಮ್ಮನೆ ಮುಗುಳ್ನಕ್ಕು ‘ಮೂರ್ಖರು ಹರಟಿಕೊಳ್ಳಲಿ’ ಎಂದು ಮುಂದಕ್ಕೆ ಸಾಗುತ್ತಾನೆ. ಅದವನ ಧೋರಣೆ. ಹೋಗಲಿ. ಆದರೆ ನಮ್ಮನ್ನು ಬೈಯಲು, ಟೀಕಿಸಲು ಜನಗಳನ್ನು ತಯಾರು ಮಾಡು ವಂತಹ ನೀವು, ನಾನೇನಾದರೂ ನಿಮ್ಮನ್ನು ಸ್ವಲ್ಪವೇ ಟೀಕಿಸಿದರೂ ಸಾಕು–ಅದೂ ಪ್ರಾಮಾಣಿಕ ವಾದ ಸದುದ್ದೇಶದಿಂದ–ನೀವು ‘ಮುಟ್ಟಿದರೆ ಮುನಿ’ಯಂತೆ ಸುರುಟಿಕೊಂಡು ‘ಓ ನಮ್ಮನ್ನು ಮುಟ್ಟಬೇಡಿ; ನಾವು ಅಮೆರಿಕನ್ನರು; ನಾವು ಜಗತ್ತಿನ ‘ಕ್ರೈಸ್ತೇತರ ಅನಾಗರಿಕ’ರನ್ನೆಲ್ಲ ಟೀಕಿಸಿ, ಬೈದು, ಶಾಪ ಹಾಕುತ್ತೇವೆ. ಆದರೆ ನಮ್ಮನ್ನು ಮಾತ್ರ ಮುಟ್ಟಬೇಡಿ; ನಾವು ತುಂಬ ನಾಜೂಕು!’ ಎಂದು ಕೂಗಿಕೊಳ್ಳುತ್ತೀರಿ. ನೀವು ನಿಮ್ಮ ಮನಸ್ಸಿಗೆ ಬಂದಂತೆ ಏನಾದರೂ ಮಾಡಿಕೊಳ್ಳ ಬಹುದು, ಅಲ್ಲವೆ? ನಮಗಂತೂ ನಾವು ಈಗಿರುವಂತೆ ಜೀವಿಸಿರುವುದರಲ್ಲೇ ತೃಪ್ತಿಯಿದೆ. ಆದರೆ ಒಂದು ವಿಷಯದಲ್ಲಿ ನಾವೆಷ್ಟೋ ಮೇಲು–ನಾವು ನಮ್ಮ ಮಕ್ಕಳ ತಲೆಯಲ್ಲಿ ‘ಈ ಪ್ರಪಂಚ ವೆಲ್ಲ ಸುಂದರ, ಆದರೆ ಮನುಷ್ಯ ಮಾತ್ರ ಪಾಪಿ’ ಎಂಬಂತಹ ದುರ್ಭಾವನೆಗಳನ್ನು ತುಂಬು ವುದಿಲ್ಲ. ನಿಮ್ಮ ಧರ್ಮಪ್ರಚಾರಕನೊಬ್ಬ ನಮ್ಮನ್ನು ಬೈಯುವಾಗಲೆಲ್ಲ ಇದನ್ನು ನೆನಪಿಟ್ಟಿರಲಿ: ಇಡೀ ಭಾರತವೇ ಎದ್ದು ನಿಂತು ಹಿಂದೂಮಹಾಸಾಗರದ ತಳದಲ್ಲಿರುವ ಕೆಸರನ್ನೆಲ್ಲ ತೆಗೆದು ಕೊಂಡು ಪಾಶ್ಚಾತ್ಯ ರಾಷ್ಟ್ರಗಳ ಮೇಲೆ ಎಸೆದರೂ ಕೂಡ, ಅದು ನೀವು ನಮಗೆ ಮಾಡುತ್ತಿರುವ ಅನ್ಯಾಯದಲ್ಲಿ ಸಾವಿರದಲ್ಲೊಂದುಪಾಲು ಕೂಡ ಅಲ್ಲ. ಏಕೆ, ನಾವೆಂದಾದರೂ ಪಾಶ್ಚಾತ್ಯರನ್ನು ಮತಾಂತರಗೊಳಿಸಲು ಒಬ್ಬನೇ ಒಬ್ಬ ಮಿಷನರಿಯನ್ನಾದರೂ ಕಳಿಸಿದ್ದೇವೆಯೆ? ನಾವು ನಿಮಗೆ ಹೇಳುವುದಿಷ್ಟೆ: ನಿಮ್ಮ ಧರ್ಮವನ್ನು ನೀವು ಇಟ್ಟುಕೊಳ್ಳಿ, ಸಂತೋಷ; ಆದರೆ ನಾವು ನಮ್ಮ ಧರ್ಮವನ್ನು ಇಟ್ಟುಕೊಳ್ಳಲು ಬಿಟ್ಟುಬಿಡಿ, ಎಂದು. ನೀವು ನಿಮ್ಮ ಧರ್ಮವನ್ನು ಬಹಳ ಪ್ರಬಲವಾದದ್ದೆಂದು ಹೇಳಿಕೊಳ್ಳುತ್ತೀರಿ. ಅದು ನಿಜವೆಂದೇ ಇಟ್ಟುಕೊಳ್ಳೋಣ. ಆದರೆ ನಿಜಕ್ಕೂ ನೀವು ಎಷ್ಟು ಜನರನ್ನು ಪರಿವರ್ತಿಸಲು ಸಮರ್ಥರಾಗಿದ್ದೀರಿ? ಈ ಪ್ರಪಂಚದ ಜನಸಂಖ್ಯೆಯಲ್ಲಿ ಪ್ರತಿ ಆರನೆಯ ವ್ಯಕ್ತಿ ಒಬ್ಬ ಚೀನಿ ಪ್ರಜೆ–ಒಬ್ಬ ಬೌದ್ಧ; ಸರಿ ತಾನೆ? ಅಲ್ಲದೆ ಜಪಾನ್, ಟಿಬೆಟ್, ರಷ್ಯಾ, ಸೈಬೀರಿಯ, ಬರ್ಮಾ, ಸಯಾಮ್​–ಇವೆಲ್ಲ ಬೇರೆ ಇವೆ....! ನನ್ನ ಈ ಮಾತು ನಿಮಗೆ ರುಚಿಸದೆ ಹೋಗಬಹುದು–ಆದರೆ ನೀವು ಕ್ರೈಸ್ತರದೆಂದು ಹೇಳಿಕೊಳ್ಳುವ ಈ ನೈತಿಕತೆ, ಕ್ಯಾಥೋಲಿಕರ ಚರ್ಚ್ ಮತ್ತು ಇನ್ನೂ ಎಷ್ಟೆಷ್ಟೋ ಅವುಗಳಿಂದಲೇ ಬಂದದ್ದು. ಇದೆಲ್ಲ ಹೇಗೆ ಸಾಧ್ಯವಾಯಿತು? ಅದೂ ಒಂದು ಹನಿ ರಕ್ತವೂ ಬೀಳದೆ! ನೀವು ಇಷ್ಟೆಲ್ಲ ಜಂಬ ಕೊಚ್ಚಿಕೊಳ್ಳುತ್ತೀರಿ; ಆದರೆ ಖಡ್ಗದ ನೆರವಿಲ್ಲದೆ ನಿಮ್ಮ ಕ್ರೈಸ್ತಧರ್ಮ ಎಲ್ಲಿ ಯಶಸ್ವಿಯಾಗಿದೆ ಹೇಳಿ?–ಈ ಇಡೀ ಜಗತ್ತಿನಲ್ಲಿ ಒಂದೇ ಒಂದು ಸ್ಥಳವನ್ನಾದರೂ ತೋರಿಸಿ ನೋಡೋಣ! ಒಂದೇ ಒಂದು ಸಾಕು, ನನಗೆ ಎರಡು ಬೇಕಿಲ್ಲ. ನಿಮ್ಮ ಪೂರ್ವಜರನ್ನೆಲ್ಲ ಹೇಗೆ ಮತಾಂತರ ಗೊಳಿಸಲಾಯಿತು ಎಂಬುದು ನನಗೆ ಗೊತ್ತಿದೆ. ಒಂದೋ ಅವರು ಮತಾಂತರಗೊಳ್ಳಬೇಕಾಗಿತ್ತು, ಇಲ್ಲವೆ ಸಾಯಬೇಕಾಗಿತ್ತು. ನಿಮ್ಮ ಧೋರಣೆ ಏನೆಂದರೆ ‘ನಾವೇ ಜಗತ್ತಿನ ಏಕೈಕ ಹಕ್ಕುದಾರರು’ ಎಂದು. ಏಕೆಂದರೆ ನೀವು ಇತರರನ್ನು ಕೊಲ್ಲಬಲ್ಲವರು! ಅರಬ್ಬರೂ ಹಾಗೆಯೇ ಹೇಳಿ ಜಂಬ ಕೊಚ್ಚಿಕೊಂಡರು. ಈಗೆಲ್ಲಿದ್ದಾರೆ ಅವರು? ಮರಳುಗಾಡಿನಲ್ಲಿ ದರೋಡೆಕೋರರಾಗಿ ಜೀವಿಸು ತ್ತಿದ್ದಾರೆ. ರೋಮನ್ನರೂ ಅದನ್ನೇ ಹೇಳುತ್ತಿದ್ದರು. ಈಗ ಅವರೆಲ್ಲಿ? ಆದರೆ ನಾವಿನ್ನೂ ಸ್ಥಿರವಾಗಿ ನಿಂತಿದ್ದೇವೆ. ‘ಶಾಂತಿದೂತರೇ ಧನ್ಯರು; ಏಕೆಂದರೆ ಅವರನ್ನು ಭಗವಂತನ ಪುತ್ರರೆಂದು ಕರೆಯಲಾಗುತ್ತದೆ’ (ಬೈಬಲ್​). ಉಳಿದುವೆಲ್ಲ ನೆಲಕ್ಕೆ ಕುಸಿಯುತ್ತವೆ. ಏಕೆಂದರೆ ಅವನ್ನೆಲ್ಲ ಕಟ್ಟಿರುವುದು ಮರಳ ಮೇಲೆ. ಅವು ಹೆಚ್ಚು ಕಾಲ ಬಾಳಲಾರವು.

“ಯಾವುದಕ್ಕೆ ಸ್ವಾರ್ಥವೇ ಆಧಾರವಾಗಿದೆಯೋ, ಯಾವುದಕ್ಕೆ ಸ್ಪರ್ಧೆಯೇ ಮೂಲಮಂತ್ರ ವಾಗಿದೆಯೋ, ಯಾವುದಕ್ಕೆ ಭೋಗವೇ ಪರಮಗುರಿಯಾಗಿದೆಯೋ ಅದೆಲ್ಲ ಇಂದಲ್ಲ ನಾಳೆ ಸಾಯಲೇಬೇಕು. ನೀವು ಬದುಕಿಕೊಳ್ಳಬೇಕಾದರೆ ಕ್ರಿಸ್ತನೆಡೆಗೆ ಹಿಂದಿರುಗಿ; ಕ್ರಿಸ್ತನನ್ನೇ ಅನುಸರಿಸಿ... ನಿಮ್ಮದು ಭೋಗದ ಹೆಸರಿನಲ್ಲಿ ಪ್ರಚಾರ ಮಾಡಿದ ಧರ್ಮ–ಎಂತಹ ವಿಧಿಯ ಅಣಕವಿದು! ನೀವು ಬದುಕಬೇಕಾದರೆ ನಿಮ್ಮ ಧೋರಣೆಯನ್ನು ಸಂಪೂರ್ಣ ಬದಲಾಯಿಸಿಕೊಳ್ಳಿ; ಭೋಗದ ಹೆಸರಿನಲ್ಲಿ ಧರ್ಮಪ್ರಚಾರ ಮಾಡುವುದನ್ನು ಬಿಟ್ಟು ತ್ಯಾಗದ ಹೆಸರಿನಲ್ಲಿ ಮಾಡಿ. ನಾನು ಈ ದೇಶದಲ್ಲಿ ಕಂಡಿರುವುದೆಲ್ಲ ಕೇವಲ ಆಷಾಢಭೂತಿತನ. ಈ ದೇಶ ಉಳಿಯಬೇಕೆಂದಿದ್ದರೆ ಅದು ಕ್ರಿಸ್ತನತ್ತ ತಿರುಗಲಿ. ನೀವು ಭೋಗವನ್ನೂ ಭಗವಂತನನ್ನೂ ಏಕಕಾಲಕ್ಕೆ ಆರಾಧಿಸಲಾರಿರಿ. ಈ ಭೋಗವಿಲಾಸಗಳೆಲ್ಲ ಕ್ರಿಸ್ತನಿಂದಲೇ ಬಂದದ್ದೆಂದು ತಿಳಿದುಕೊಂಡಿದ್ದೀರೋ? ಕ್ರಿಸ್ತನಿದ್ದಿದ್ದರೆ ಇಂತಹ ಪಾಷಂಡವಾದವನ್ನು ತಿರಸ್ಕರಿಸಿಬಿಡುತ್ತಿದ್ದ. ಭೋಗಾರಾಧನೆಯ ಮೂಲಕ ಬರುವಂತಹ ಗೆಲುವೆಲ್ಲ ಕ್ಷಣಿಕ. ನಿತ್ಯದ ನೆಲೆ ಭಗವಂತ ಮಾತ್ರವೇ. ನಿಮಗೆ ನಿಮ್ಮ ಈ ಅತ್ಯದ್ಭುತ ಪ್ರಾಪಂಚಿಕ ಭೋಗವೈಭವಗಳನ್ನು ಕ್ರಿಸ್ತನ ಆದರ್ಶದೊಂದಿಗೆ ಅನುಭವಿಸಲು ಸಾಧ್ಯವಾದರೆ ಒಳ್ಳೆಯದೇ. ಆದರದು ಸಾಧ್ಯವಾಗದಿದ್ದರೆ, ಈ ಎಲ್ಲ ಪೊಳ್ಳುವ್ಯವಹಾರಗಳನ್ನು ಬಿಟ್ಟು ಕ್ರಿಸ್ತನೆಡೆಗೆ ಹಿಂದಿರುಗು ವುದು ಕ್ಷೇಮ. ಕ್ರಿಸ್ತನಿಲ್ಲದ ಅರಮನೆಗಳಲ್ಲಿ ವಾಸಿಸುವುದಕ್ಕಿಂತ ಕ್ರಿಸ್ತನಿರುವ ಗುಡಿಸಲುಗಳಲ್ಲಿ ಇರುವುದು ಒಳ್ಳೆಯದು.”

ಮತ್ತೊಂದು ಉಪನ್ಯಾಸದಲ್ಲಿ ಸ್ವಾಮೀಜಿ ಇನ್ನಷ್ಟು ಕಠಿಣವಾಗಿ ಹೇಳಿದರು, “ಎಲ್ಲಿದೆ ನಿಮ್ಮ ಕ್ರೈಸ್ತಧರ್ಮ? ನಿಮ್ಮ ಈ ಸ್ವಾರ್ಥಪರ ಹೋರಾಟದಲ್ಲಿ, ನಿಮ್ಮ ಈ ನಿರಂತರ ವಿಧ್ವಂಸಕ ಪ್ರವೃತ್ತಿಯಲ್ಲಿ ಏಸುಕ್ರಿಸ್ತನಿಗೆ ಜಾಗ ಎಲ್ಲಿದೆ? ನಿಜಕ್ಕೂ ಅವನೇನಾದರೂ ಇಲ್ಲಿದ್ದಿದ್ದರೆ ಒಂದು ಕ್ಷಣ ಕುಳಿತುಕೊಳ್ಳಲು ಒಂದೇ ಒಂದು ಕಲ್ಲು ಕೂಡ ಸಿಗುತ್ತಿರಲಿಲ್ಲ ಅವನಿಗೆ.”

ಇಂತಹ ಮಾತುಗಳಿಂದ ಕ್ರೈಸ್ತ ಸಮಾಜದ ಮೇಲೆ ಉಂಟಾದ ಪರಿಣಾಮವನ್ನು ಚೆನ್ನಾಗಿಯೇ ಊಹಿಸಿಕೊಳ್ಳಬಹುದು. ವೃತ್ತಪತ್ರಿಕೆಗಳು, ನಿಯತಕಾಲಿಕಗಳು, ಚರ್ಚುಗಳ ಸಭೆಗಳು– ಎಲ್ಲೆಲ್ಲೂ ವಿವೇಕಾನಂದರದೇ ವಿಚಾರ! ಎಲ್ಲೆಲ್ಲೂ ಅವರ ಬಗ್ಗೆಯೇ ಚರ್ಚೆ! ಅವರನ್ನು ಕೊಂಡಾಡುವವರು ಹಲವರು, ಹಿಡಿಶಾಪ ಹಾಕುವವರು ಕೆಲವರು. ಕ್ರೈಸ್ತ ಪ್ರಚಾರಕರಿಗಂತೂ ಸ್ವಾಮೀಜಿಯ ಮಾತುಗಳನ್ನು ಕೇಳಿ ಹೊಟ್ಟೆಯಲ್ಲಿ ಕಿಚ್ಚಿಟ್ಟಂತಾಯಿತು. ಜೊತೆಗೆ ಅವರ ಹೆಚ್ಚುತ್ತಿರುವ ಜನಪ್ರಿಯತೆಯನ್ನು ಕಂಡು ಮತ್ತಷ್ಟು ಸಂಕಟವಾಯಿತು. ಆದ್ದರಿಂದ ಈ ಧರ್ಮಾಧಿಕಾರಿಗಳೆಲ್ಲ ಜೊತೆಗೂಡಿ, ಸ್ವಾಮೀಜಿಯ ಕೀರ್ತಿಗೆ ಮಸಿಬಳಿಯುವ ಪ್ರಯತ್ನದಲ್ಲಿ ತೊಡಗಿದರು. ಏನಾದರೂ ಮಾಡಿ ಅವರ ಹುಟ್ಟಡಗಿಸಬೇಕೆಂದು ನಿರ್ಧರಿಸಿದರು. ತಮ್ಮದೇ ಆದ ಹಾಗೂ ತಮಗೆ ಬೆಂಬಲವಾದ ವೃತ್ತಪತ್ರಿಕೆಗಳಲ್ಲಿ ಅವರನ್ನೂ ಅವರ ಭಾಷಣಗಳನ್ನೂ ನಾನಾ ವಿಧವಾಗಿ ಜರೆದರು. ಅವರಿಗೆ ಅಪಮಾನವುಂಟುಮಾಡಲು ತಮ್ಮಿಂದಾದದ್ದನ್ನೆಲ್ಲ ಮಾಡಿದರು. ಈ ದಿಸೆಯಲ್ಲಿ ಆ ಧರ್ಮಪ್ರಚಾರಕರು ಎಲ್ಲಿಯವರೆಗೆ ಹೋದರೆಂದರೆ, ನವಯುವತಿಯರನ್ನು ಸ್ವಾಮೀಜಿಯ ಮುಂದೆ ಬಿಟ್ಟು ಅವರನ್ನು ಪ್ರಲೋಭನೆಗೊಳಿಸಲೂ ಪ್ರಯತ್ನಿಸಿದರು! ಈ ಪ್ರಯತ್ನದಲ್ಲಿ ಯಶಸ್ವಿಗಳಾದರೆ, ಈ ಯುವತಿಯರಿಗೆ ಧಾರಾಳವಾದ ಸಂಭಾವನೆಯ ಭರವಸೆ ಯನ್ನು ನೀಡಲಾಗಿತ್ತು. ಯಾವೊಬ್ಬ ಸಭ್ಯವ್ಯಕ್ತಿಗೇ ಆಗಲಿ, ಸ್ತ್ರೀಯೊಂದಿಗೆ ಅನೈತಿಕ ಸಂಪರ್ಕ ವಾಗಿಬಿಟ್ಟರೆ, ಅವನ ಕೀರ್ತಿಶರೀರ ನಾಶವಾಗಲು ಅದಕ್ಕಿಂತ ಹೆಚ್ಚೇನೂ ಬೇಕಾಗಿಲ್ಲ. ಅದರಲ್ಲೂ ಆ ವ್ಯಕ್ತಿ ಸಂನ್ಯಾಸಿಯಾಗಿದ್ದರಂತೂ ಅವನ ಕತೆ ಮುಗಿದಂತೆಯೇ. ಆದ್ದರಿಂದ ಕ್ರೈಸ್ತಧರ್ಮ ಪ್ರಚಾರಕರು ಅಂತಹ ಹೀನ ಉಪಾಯವನ್ನು ಹೂಡಿದ್ದು. ಆ ಯುವತಿಯರು ಸ್ವಾಮೀಜಿಯನ್ನು ಪ್ರಲೋಭನೆಗೊಳಿಸಿ ಕೆಳಗೆಳೆಯಲು ಸಾಕಷ್ಟು ಪ್ರಯತ್ನ ಮಾಡಿದರು. ಆದರೆ ಅದಲ್ಲಿ ಅವರು ವಿಫಲರಾದರೆಂದು ಹೇಳಬೇಕಾಗಿಯೇ ಇಲ್ಲ. ಸ್ವಾಮೀಜಿಯ ಶಿಶುಸಹಜ ಸರಳತೆ-ಪಾವಿತ್ರ್ಯ ಗಳನ್ನು ಮನಗಂಡ ಆ ಯುವತಿಯರೇ ನಾಚಿ ಹಿಂಜರಿದರು.

ಕೆಲವೊಮ್ಮೆ ಸ್ವಾಮೀಜಿಯ ಆತಿಥೇಯರಾಗಿದ್ದವರಿಗೆ ಹಾಗೂ ಅವರನ್ನು ತಮ್ಮ ಮನೆಗೆ ಆಹ್ವಾನಿಸುತ್ತಿದ್ದವರಿಗೆ ಅನಾಮಧೇಯ ಪತ್ರಗಳು ಬರುತ್ತಿದ್ದುವು. ವಿವೇಕಾನಂದರನ್ನು ಸ್ತ್ರೀ ಲೋಲನೆಂದೂ ಗೋಮುಖವ್ಯಾಘ್ರನೆಂದೂ ಬಣ್ಣಿಸಿ ಈ ‘ಹಿತೈಷಿ’ಗಳು, ಅವರ ಬಗ್ಗೆ ‘ಸ್ವಲ್ಪ ಎಚ್ಚರಿಕೆ’ ವಹಿಸುವಂತೆ ಸಲಹೆ ನೀಡುತ್ತಿದ್ದರು. ಸ್ವಾಮೀಜಯನ್ನು ಚೆನ್ನಾಗಿ ಅರ್ಥಮಾಡಿ ಕೊಂಡಂತಹ ಅವರ ಸ್ನೇಹಿತರಿಗೂ ಶಿಷ್ಯರಿಗೂ, ಈ ಪ್ರಯತ್ನಗಳನ್ನು ಕಂಡು ಅದೆಷ್ಟು ಯಾತನೆಯಾಗುತ್ತಿತ್ತೋ! ಆದರೆ ಇನ್ನು ಕೆಲವರ ಮೇಲೆ ಈ ಪತ್ರಗಳು ಉದ್ದೇಶಿತ ಪರಿಣಾಮ ವನ್ನು ಬೀರಿದ್ದೂ ಉಂಟು. ಅಂತಹವರು ಸ್ವಾಮೀಜಿಯನ್ನು ಈಗಾಗಲೇ ಆಹ್ವಾನಿಸಿದ್ದರೆ, ಅವರು ಬರುವ ವೇಳೆಗೆ ಮನೆಗೆ ಬೀಗ ಹಾಕಿಕೊಂಡು ಮಾಯವಾಗುತ್ತಿದ್ದರು! ಆದರೆ ಕೆಲಕಾಲದಲ್ಲೇ ಇವರಿಗೂ ನಿಜಸಂಗತಿಯ ಅರಿವಾಗುತ್ತಿತ್ತು. ತಕ್ಷಣ ಬಂದು ಸ್ವಾಮೀಜಿಯ ಕ್ಷಮೆ ಯಾಚಿಸು ತ್ತಿದ್ದರು; ಮೊದಲಿಗಿಂತ ಹೆಚ್ಚು ಆತ್ಮೀಯರಾಗುತ್ತಿದ್ದರು.

ಏನೇ ಆದರೂ ಸ್ವಾಮೀಜಿಯ ವಿರುದ್ಧ ಅಪಪ್ರಚಾರ ಮಾಡುತ್ತಿದ್ದ ಕ್ರೈಸ್ತ ಮಿಷನರಿಗಳಿಗೆ ಅನೇಕ ಅನುಕೂಲತೆಗಳಿದ್ದುವು. ಈಗಾಗಲೇ ಈ ಮಿಷನರಿಗಳು ಜನರಲ್ಲಿ ಭಾರತದ ಕುರಿತಾಗಿ ಮೂಡಿಸಿದ್ದ ಪೂರ್ವಾಗ್ರಹವನ್ನು ಹೋಗಲಾಡಿಸುವುದು ಸ್ವಾಮೀಜಿಗೆ ಅಷ್ಟೇನೂ ಸುಲಭವಾಗಿರ ಲಿಲ್ಲ. ಅಲ್ಲದೆ ಸ್ವಾಮೀಜಿ ಒಬ್ಬಂಟಿಗರು. ಅವರನ್ನು ಬೆಂಬಲಿಸಿ ನಿಂತವರ ಸಂಖ್ಯೆ ಬಹಳಷ್ಟು ಕಡಿಮೆಯೇ. ಅದರಲ್ಲೂ ಶ್ರೀಮತಿ ಬ್ಯಾಗ್​ಲೀಯವರಂತಹ ಪ್ರಭಾವಶಾಲಿ ವ್ಯಕ್ತಿಗಳು ಇನ್ನೂ ಕಡಿಮೆ. ಅವರು ಯಾವ ಭಾರತವನ್ನು ಎತ್ತಿ ಹಿಡಿಯಲು ತಮ್ಮ ರಕ್ತ ಬಸಿಯುತ್ತಿದ್ದರೋ ಆ ದೇಶದ ಜನಗಳೇ ಇನ್ನೂ ಒಟ್ಟಾಗಿ ನಿಂತು ಅವರಿಗೆ ತಮ್ಮ ಅಭಿನಂದನೆಯನ್ನೂ ಕೃತಜ್ಞತೆಯನ್ನೂ ಸಲ್ಲಿಸಿರಲಿಲ್ಲ. ಆದ್ದರಿಂದ ಅಂದಿನ ಅವರ ಅಸಹಾಯಕ ಪರಿಸ್ಥಿತಿಯು, ಅವರಿಗೆ ಹಾನಿಯುಂಟು ಮಾಡಬೇಕೆನ್ನುವವರಿಗೆ ಅನುಕೂಲವಾಗಿತ್ತು. ಆದರೆ ಇಷ್ಟೆಲ್ಲ ಕಷ್ಟತೊಂದರೆಗಳ ನಡುವೆಯೂ ಸ್ವಾಮೀಜಿ, ಶ್ರೀರಾಮಕೃಷ್ಣರ ಮೇಲೆ ಭಾರ ಹಾಕಿ ಶಾಂತರಾಗಿದ್ದು ಮನಸ್ಸಿನ ಸಮತೋಲವನ್ನು ಕಾಪಾಡಿಕೊಂಡರು. ಅಲ್ಲದೆ ಅಮೆರಿಕದಲ್ಲಿ ತಮ್ಮನ್ನು ಆದರಿಸುವ ಉದಾರ ಬುದ್ಧಿಯ ಪಾದ್ರಿ ಗಳೂ ಇತರ ಗಣ್ಯಕ್ರೈಸ್ತರೂ ಇದ್ದಾರೆಂಬುದನ್ನು ನೆನೆಸಿಕೊಂಡು ತಮಗೆ ತಾವೇ ಸಮಾಧಾನ ತಂದುಕೊಂಡರು. ಅಲ್ಲದೆ ಎಷ್ಟೋ ಜನ ಅವರ ಕಾರ್ಯೋದ್ದೇಶವನ್ನು ಸ್ವೀಕರಿಸಿ, ಸಮರ್ಥಿಸಿ, ಅವರ ಅನುಯಾಯಿಗಳಾದರು. ಇವೆಲ್ಲಕ್ಕಿಂತ ಹೆಚ್ಚಾಗಿ, ತಮ್ಮ ಸಂದೇಶವು ಪ್ರಸಾರವಾಗಬೇಕೆಂ ಬುದು ಭಗವಂತನ ಇಚ್ಛೆಯೇ ಹೌದಾದಲ್ಲಿ, ಈ ಜಗತ್ತಿನ ಯಾವುದೇ ಶಕ್ತಿಯೂ ಅದನ್ನೆದುರಿಸಿ ನಿಲ್ಲಲಾರದು ಎನ್ನುವುದು ಅವರಿಗೆ ದೃಢವಾಗಿತ್ತು.

ಮನಸ್ಸು ಮಾಡಿದ್ದರೆ ಸ್ವಾಮೀಜಿ, ತಮ್ಮ ವಿರುದ್ಧ ನಡೆಯುತ್ತಿದ್ದ ಅಪಪ್ರಚಾರವನ್ನು ಎದುರಿಸಿ ನಿಲ್ಲಬಹುದಾಗಿತ್ತು. ಅದರಲ್ಲವರು ಬಹುಮಟ್ಟಿಗೆ ಯಶಸ್ವಿಗಳಾಗುತ್ತಿದ್ದರೆಂದೂ ಹೇಳಬಹುದು. ಆದರೆ ಅವರು ಹಾಗೆ ಮಾಡಲಿಲ್ಲ. ಅವರ ಅನೇಕ ವಿಶ್ವಾಸಿಗಳು ಈ ಆಪಾದನೆಗಳಿಗೆಲ್ಲ ತಕ್ಕ ಉತ್ತರ ನೀಡಿ, ಆ ದುಷ್ಟರ ಬಾಯಿ ಮುಚ್ಚಿಸುವಂತೆ ಅವರನ್ನು ಕೇಳಿಕೊಂಡರು. ಆದರೆ ಸ್ವಾಮೀಜಿ, “ನಾನೊಬ್ಬ ಸಂನ್ಯಾಸಿ; ನಾನೆಂದಿಗೂ ಪ್ರತಿ ಆಕ್ರಮಣ ಮಾಡಲಾರೆ. ಹಾಗೆ ಮಾಡುವುದು ಸಂನ್ಯಾಸಧರ್ಮಕ್ಕೆ ವಿರುದ್ಧ. ಸತ್ಯವು ತನ್ನ ದಾರಿಯನ್ನು ಕಂಡುಕೊಳ್ಳುತ್ತದೆ. ನನ್ನನ್ನು ನಂಬಿ; ಸತ್ಯವೇ ಗೆಲ್ಲುತ್ತದೆ” ಎಂದುತ್ತರಿಸಿದರು. ಕೆಲವೊಮ್ಮೆ ಯಾರಾದರೂ ಬಂದು ಸ್ವಾಮೀಜಿಯ ವಿರುದ್ಧ ಹೊಸದಾಗಿ ಹುಟ್ಟಿಕೊಂಡ ಅಪವಾದಗಳ ಬಗ್ಗೆ ತಿಳಿಸಿದರೆ, ಅದಕ್ಕೆ ಅವರ ಉತ್ತರ ಒಂದೇ–ಮೌನ ಪ್ರಾರ್ಥನೆ.

ತಮ್ಮ ಮೇಲೆ ವೈಯಕ್ತಿಕವಾಗಿ ನಡೆಯುತ್ತಿದ್ದ ಅಪಪ್ರಚಾರಗಳಿಗೆ ಸ್ವಾಮೀಜಿ ಉತ್ತರಿಸಲು ನಿರಾಕರಿಸುತ್ತಿದ್ದರೂ ತಮ್ಮ ತಾಯ್ನಾಡಿನ ಪರವಾಗಿ ಮಾತ್ರ ತಮ್ಮದೇ ಆದ ರೀತಿಯಲ್ಲಿ ಕೆಚ್ಚೆದೆಯ ಹೋರಾಟವನ್ನು ಮುಂದುವರಿಸಿದರು. ತಮ್ಮ ಮಾತೃಭೂಮಿಯ ಮೇಲಿನ, ತಮ್ಮ ಸ್ವಜನ-ಸ್ವಧರ್ಮಗಳ ಮೇಲಿನ ಅಪವಾದಗಳನ್ನು ಕಳೆದು, ಕಳಂಕರಹಿತ ಗರಿಮೆಯಲ್ಲಿ ಅದರ ಮಹಿಮೆಯನ್ನು ಎತ್ತಿಹಿಡಿಯಲು ಅವರು ಸದಾ ಕಂಕಣಬದ್ಧರಾಗಿದ್ದರು. ಭಾರತದ ಧಾರ್ಮಿಕ ಹಾಗೂ ಸಾಮಾಜಿಕ ಸಂಪ್ರಾಯಗಳಿಗೆ ಸಂಬಂಧಿಸಿದಂತೆ ವಸ್ತುಸ್ಥಿತಿಯನ್ನು ಜನರಿಗೆ ಮನಗಾಣಿ ಸಲು ಡೆಟ್ರಾಯ್ಟ್​ನಲ್ಲೂ ಇನ್ನಿತರ ಸ್ಥಳಗಳಲ್ಲೂ ತಮ್ಮ ಉಪನ್ಯಾಸಗಳನ್ನು ಮುಂದುವರಿಸಿದರು. ಈ ಉಪನ್ಯಾಸಗಳಿಂದ ಬಹುಪಾಲು ಅಮೆರಿಕನ್ನರಿಗೆ ಮೊಟ್ಟಮೊದಲ ಬಾರಿಗೆ ಹಿಂದೂಧರ್ಮದ ಶ್ರೇಷ್ಠ ಸಂಸ್ಕೃತಿಯ ಹಾಗೂ ಅದರ ಉನ್ನತ ಸತ್ಯಗಳ ಪರಿಚಯವಾಯಿತು. ಅಳಸಿಂಗ ಪೆರುಮಾಳ್​ಗೆ ಬರೆದ ಒಂದು ಪತ್ರದಲ್ಲಿ ಸ್ವಾಮೀಜಿ ಹೇಳುತ್ತಾರೆ, “ತನ್ನ ತಾಯ್ನಾಡನ್ನು ರಕ್ಷಿಸುವ ಸಾಹಸ ಮಾಡಿದ ಒಬ್ಬ ವ್ಯಕ್ತಿಯೆಂದರೆ ನಾನೇ. ನಾನವರಿಗೆ ಎಂತಹ ವಿಚಾರಗಳನ್ನು ನೀಡಿದ್ದೇ ನೆಂದರೆ, ಒಬ್ಬ ಹಿಂದುವಿನಿಂದ ಅವರೆಂದೂ ಅಂತಹದನ್ನು ನಿರೀಕ್ಷಿಸಿರಲಿಲ್ಲ.” ನಿಜಕ್ಕೂ ಸ್ವಯಂ ಸ್ವಾಮೀಜಿಯ ವ್ಯಕ್ತಿತ್ವವೇ ಅಲ್ಲಿಯವರಿಗೊಂದು ಹೊಸ ಅನುಭವ. ಅವರು ವೇದಿಕೆಯಿ ಮೇಲೆ ಬಂದು ನಿಂತರೆ ಸಾಕು, ‘ನಾವು ಶತಮಾನಗಳಿಂದಲೂ ನಮ್ಮ ಪಾದ್ರಿಗಳ ಬಾಯಲ್ಲಿ ಯಾವ ಅನಾಗರಿಕ ಕಾಡುಜನರ ಬಗ್ಗೆ ಕೇಳಿದ್ದೆವೋ, ನಮ್ಮ ಶಾಲಾ ಪುಸ್ತಕಗಳಲ್ಲಿ ಓದಿದ್ದೆವೋ, ಆ ಭಾರತದಿಂದ ಬಂದವನೆ ಈತ!’ ಎಂದು ಸಭಿಕರು ದಂಗಾಗುತ್ತಿದ್ದರು. ಅಲೌಕಿಕವಾದ ಶಕ್ತಿ ಯೊಂದು ಅವರ ಮಾತು, ನೋಟ, ನಡೆ ನುಡಿಗಳ ಮೂಲಕ ಉಕ್ಕಿ ಹರಿದು, ತಮ್ಮ ಮುಂದಿರುವವನು ಕೇವಲ ವ್ಯಕ್ತಿಯಲ್ಲ, ಶಕ್ತಿ ಎಂಬುದನ್ನು ಸಭಿಕರಿಗೆ ಅನುಭವ ಮಾಡಿ ಕೊಡುತ್ತಿತ್ತು. ಮತ್ತೊಂದು ಪತ್ರದಲ್ಲಿ ಸ್ವಾಮೀಜಿ ಬರೆಯುತ್ತಾರೆ, “ನನ್ನ ಹಿಂದಿರುವ ಶಕ್ತಿ ವಿವೇಕಾನಂದನದಲ್ಲ, ಭಗವಂತನದು.”

ಸ್ವಾಮೀಜಿ ತಮ್ಮ ತಾಯ್ನಾಡಿನ ಅತ್ಯುಜ್ವಲ ಪ್ರತಿನಿಧಿಯಾಗಿದ್ದರೆಂಬುದರಲ್ಲಿ ಸಂದೇಹವಿಲ್ಲ. ಆದರೆ ಸಹಸ್ರಾರು ಅಮೆರಿಕನ್ನರ ಪಾಲಿಗೆ ಅವರು ಅಷ್ಟು ಮಾತ್ರವೇ ಆಗಿರದೆ, ದಿವ್ಯಸತ್ಯವನ್ನು ಹೊತ್ತು ತಂದ ಪ್ರವಾದಿಯಾಗಿದ್ದರು. ಜನರ ದೈನಂದಿನ ಜೀವನದ ನೀರಸತೆಯನ್ನು ಮುರಿದು, ಅವರನ್ನು ಉನ್ನತ ಜೀವನದೆಡೆಗೆ, ಸಾರ್ಥಕ ಜೀವನದೆಡೆಗೆ ತಿರುಗಿಸಿದ ಉದ್ಧಾರಕರಾಗಿದ್ದರು ಸ್ವಾಮೀಜಿ. ಅನೇಕ ವರ್ಷಗಳಿಂದಲೂ ಲೆಕ್ಕವಿಲ್ಲದಷ್ಟು ಭಾಷಣಕಾರರ ಭಾಷಣಗಳನ್ನು ಅಮೆರಿಕದ ಜನ ಕೇಳಿದ್ದರು. ಅನೇಕಾನೇಕ ನೂತನ ಮತಪಂಥಗಳ ನಾಯಕರ ಉಪನ್ಯಾಸಗಳನ್ನು ಕೇಳಿಕೇಳಿ ದಣಿದಿದ್ದರು. ಇದರಿಂದ ಅವರ ಆತ್ಮದ ಹಸಿವು-ಅತೃಪ್ತಿಗಳು ಇನ್ನಷ್ಟು ಹೆಚ್ಚೇ ಆಗಿದ್ದುವು. ಮತ್ತು ಇದರಿಂದಾಗಿ ಸ್ವಾಮೀಜಿಯ ಸಂದೇಶದ ಬೀಜಗಳು ಬಿದ್ದು ಮೊಳೆಯಲು ನೆಲ ಹದವಾದಂತಿತ್ತು. ಇಂತಹ ಜನ ಅವರ ಭಾಷಣಗಳನ್ನು ಕೇಳಿ ತಮ್ಮ ಹಸಿವನ್ನು ಹಿಂಗಿಸಿ ಕೊಂಡರು. ಬಹುಕಾಲದಿಂದ ಸತ್ಯದ ಮಾರ್ಗವನ್ನು ಅರಸಿ ಅರಸಿ ಬಳಲಿ ಬೆಂಡಾಗಿದ್ದ ಜನ ಈಗ ಸ್ವಾಮೀಜಿಯ ಸಾನ್ನಿಧ್ಯದಲ್ಲಿ ಸತ್ಯದ ಬಾಗಿಲು ತಾನಾಗಿಯೇ ತೆರೆದುಕೊಳ್ಳುತ್ತಿರುವುದನ್ನು ಕಂಡರು. ಸ್ವಾಮೀಜಿಯೊಂದಿಗೆ ಆತ್ಮೀಯ ಸಂಪರ್ಕಕ್ಕೆ ಬಂದವರು ಅವರಲ್ಲಿ ಇನ್ನೂ ಹೆಚ್ಚಿನ ಶಕ್ತಿಯನ್ನು ಸ್ಪರ್ಶಿಸಿದರು.

ಸ್ವಾಮೀಜಿ ಡೆಟ್ರಾಯ್ಟ್​ಗೆ ನೀಡಿದ ಈ ಪ್ರಥಮ ಭೇಟಿಯ ಕುರಿತಾಗಿ ಮುಂದೆ ಅವರ ಅತ್ಯಂತ ಆಪ್ತಶಿಷ್ಯೆಯರಲ್ಲೊಬ್ಬಳಾದ ಶ್ರೀಮತಿ ಮೇರಿ ಸಿ. ಫಂಕೆ ಎಂಬವಳು ಬರೆಯುತ್ತಾಳೆ:

“೧೮೯೪ನೇ ಇಸವಿಯ ಫೆಬ್ರುವರಿ ೧೪ನೇ ದಿನ ನನ್ನ ಜೀವನದಲ್ಲಿ ಅವಿಸ್ಮರಣೀಯವಾಗಿ ಉಳಿದಿರುವ ಪವಿತ್ರದಿನ, ಪುಣ್ಯದಿನ. ಏಕೆಂದರೆ ಮಹಾತ್ಮರೂ ಆಧ್ಯಾತ್ಮಿಕತೆಯ ಮಹಾಶಿಖರ ದಂಥವರೂ ಆದ ವಿವೇಕಾನಂದರನ್ನು ನಾನು ಮೊದಲ ಬಾರಿ ನೋಡಿದ್ದು, ಅವರ ಮಾತುಗಳನ್ನು ಮೊದಲಬಾರಿ ಕೇಳಿದ್ದು ಅಂದೇ. ಅದಾದ ಎರಡು ವರ್ಷಗಳ ಅನಂತರ ಅವರು ನನ್ನನ್ನು ಶಿಷ್ಯಳನ್ನಾಗಿ ಸ್ವೀಕರಿಸಿದರು. ಇದು ನನ್ನ ಪಾಲಿನ ಅತ್ಯಂತ ಆನಂದದ ಹಾಗೂ ಅಚ್ಚರಿಯ ವಿಷಯ.

“ಆ ಸಮಯದಲ್ಲಿ ಅವರು ಅಮೆರಿಕದ ದೊಡ್ಡದೊಡ್ಡ ನಗರಗಳಲ್ಲಿ ಉಪನ್ಯಾಸ ಮಾಡು ತ್ತಿದ್ದರು. ಮೇಲೆ ಹೇಳಿದ ದಿನ ಅವರು ಡೆಟ್ರಾಯ್ಟ್​ನ ಯೂನಿಟೇರಿಯನ್ ಚರ್ಚಿನಲ್ಲಿ ತಮ್ಮ ಉಪನ್ಯಾಸಮಾಲೆಯ ಮೊದಲ ಉಪನ್ಯಾಸವನ್ನು ನೀಡಿದರು. ವಿಶಾಲವಾದ ಆ ಕಟ್ಟಡವು ಅಕ್ಷರಶಃ ಜನರಿಂದ ತುಂಬಿಹೋಗಿತ್ತು. ಸ್ವಾಮೀಜಿ ಬರುತ್ತಿದ್ದಂತೆಯೇ ಅವರನ್ನು ಹರ್ಷೋದ್ಗಾರಗ ಳೊಂದಿಗೆ ಸ್ವಾಗತಿಸಲಾಯಿತು. ಅವರು ವೇದಿಕೆಯ ಬಳಿಗೆ ಬಂದ ದೃಶ್ಯ ಇನ್ನೂ ನನ್ನ ಕಣ್ಣಿಗೆ ಕಟ್ಟಿದಂತಿದೆ. ಅವರದು ರಾಜಠೀವಿ, ಆಜಾನುಬಾಹು ವ್ಯಕ್ತಿತ್ವ, ಚೈತನ್ಯಪೂರ್ಣ ನಿಲವು, ಕಣ್ಸೆಳೆ ಯುವ ಭಂಗಿ; ಅವರ ಕಂಠವೋ ಅದ್ಭುತ-ಸಂಗೀತಮಯ–ಹಾರ್ಪ್ ವಾದ್ಯದ ತಂತಿಯನ್ನು ಮೀಟಿದಂತೆ ಗಂಭೀರ ಸ್ಪಂದನವನ್ನು ಹೊರಸೂಸುತ್ತ ಎಲ್ಲೆಲ್ಲೂ ಅನುರಣಿತವಾಗುತ್ತಿತ್ತು. ಎಲ್ಲೆಲ್ಲೂ ನೀರವತೆ ಮುಟ್ಟಿ ಅನುಭವಿಸಬಹುದೇನೋ ಎಂಬಂತೆ ತುಂಬಿತ್ತು. ಇಡೀ ಸಭೆ ಒಬ್ಬನೇ ವ್ಯಕ್ತಿಯಂತೆ ಉಸಿರಾಡುತ್ತಿತ್ತು.

“ಇಲ್ಲಿ ಸ್ವಾಮೀಜಿ ಅನೇಕ ಸಾರ್ವಜನಿಕ ಉಪನ್ಯಾಸಗಳನ್ನು ನೀಡಿದರು. ಅವರು ತಮ್ಮ ಸಭಿಕರನ್ನು ಬಲವಾಗಿ ಆಕರ್ಷಿಸಿಬಿಟ್ಟಿದ್ದರು. ಏಕೆಂದರೆ ಅವರದು ಒಮ್ಮೆ ಹಿಡಿದರೆ ಬಿಡುವ ಕೈಯಲ್ಲ. ಅವರು ಅಧಿಕಾರವಾಣಿಯಿಂದ ಮಾತನಾಡಿದರು. ಅವರ ವಾದವು ಅತ್ಯಂತ ತರ್ಕ ಬದ್ಧವಾಗಿದ್ದು, ಕೇಳಿದವರಿಗೆ ಅಹುದಹುದೆನ್ನಿಸುತ್ತಿತ್ತು. ಅಲ್ಲದೆ ಅವರು ತಮ್ಮ ಅತ್ಯುತ್ತಮ ವಾಗ್ಮಿತೆಯಿಂದ ಮಾತನಾಡುತ್ತಿದ್ದರೂ, ತಾವು ಹೇಳಬೇಕೆಂದಿದ್ದ ಮುಖ್ಯ ವಿಷಯವನ್ನು ಮರೆಯುತ್ತಿರಲಿಲ್ಲ.’

ಇಲ್ಲಿಯವರೆಗೂ ಸ್ವಾಮೀಜಿ ಉಪನ್ಯಾಸಗಳನ್ನು ನೀಡುತ್ತಿದ್ದುದು ಸ್ಲೇಟನ್ ಎಂಬಾತನ ಉಪನ್ಯಾಸಸಂಸ್ಥೆಯ ಆಶ್ರಯದಲ್ಲಿ. ಈ ಸಂಸ್ಥೆಯೊಂದಿಗೆ ಅವರು ಮೂರು ವರ್ಷಗಳಿಗೆ ಒಪ್ಪಂದವೊಂದಕ್ಕೆ ಸಹಿ ಹಾಕಿದ್ದರು. ಅಮೆರಿಕದ ಜನರಿಗೆ ಹಿಂದೂ ಧರ್ಮದ ಉನ್ನತ ವಿಚಾರ ಗಳನ್ನು ಬೋಧಿಸುವುದರೊಂದಿಗೆ, ಭಾರತದ ಬಡಜನತೆಗಾಗಿ ಸಾಧ್ಯವಾದಷ್ಟು ಹಣಸಂಪಾದನೆ ಮಾಡುವುದೂ ಅವರ ಉದ್ದೇಶವಾಗಿದ್ದುದರಿಂದ, ಅವರು ತಮ್ಮ ದೇಹಶ್ರಮವನ್ನೂ ಲೆಕ್ಕಿಸದೆ ದುಡಿಯುತ್ತಿದ್ದರು. ಆದರೆ ಈ ಸಂಸ್ಥೆಯೊಂದಿಗೆ ಅವರು ಕರಾರು ಮಾಡಿಕೊಂಡು ಒಂದು ದೊಡ್ಡತಪ್ಪು ಮಾಡಿದ್ದರು. ಏಕೆಂದರೆ ಆ ಕರಾರನ್ನು ಉಪನ್ಯಾಸ ಸಂಸ್ಥೆಯವರು ದುರುಪಯೋಗ ಪಡಿಸಿಕೊಂಡು, ಸ್ವಾಮೀಜಿಯಿಂದ ಹಣವನ್ನು ಬೇಕೆಂಬಂತೆ ಸೆಳೆಯಲು ಅವಕಾಶವಿತ್ತು. ಪ್ರಾಪಂಚಿಕ ವಿಚಾರಗಳಲ್ಲಿ ತುಂಬ ಮುಗ್ಧರಾದ ಸ್ವಾಮೀಜಿಗೆ ಇವೆಲ್ಲ ಗೊತ್ತಿರಲಿಲ್ಲ. ಆದರೆ ಅವರ ಸ್ನೇಹಿತರು-ಶಿಷ್ಯರು ಯಾರೂ ಅವರಿಗೆ ಈ ಬಗ್ಗೆ ತಿಳಿಸಿರದಿದ್ದುದು ಆಶ್ಚರ್ಯವೇ ಸರಿ. ಅಥವಾ ಇತರರಿಗೂ ಈ ಕರಾರಿನ ಹಿಂದಿನ ಮೋಸ ತಿಳಿಯಲಿಲ್ಲವೆಂದರೆ ಅದೂ ಒಂದು ಆಶ್ಚರ್ಯ! ಕರಾರು ಪತ್ರಕ್ಕೇನೋ ಸ್ವಾಮೀಜಿ ಸಹಿ ಹಾಕಿಬಿಟ್ಟರು. ಅದರಂತಯೇ ಆ ಸಂಸ್ಥೆ ಸ್ವಾಮೀಜಿಗೆ ಮೋಸಮಾಡಲು ಪ್ರಾರಂಭಿಸಿತು. ಮೊದಮೊದಲಿಗೆ ಸಂಸ್ಥೆಯವರು ಒಂದು ಉಪನ್ಯಾಸಕ್ಕೆ ೯ಂಂ ಡಾಲರ್​ವರೆಗೂ ಕೊಟ್ಟಿದ್ದರು. ಇದು ಅವರಲ್ಲಿ ನಂಬಿಕೆ ಹುಟ್ಟಿಸಲು ಮಾತ್ರ. ಆದರೆ ಬರಬರುತ್ತ ಈ ಸಂಭಾವನೆಯನ್ನು ಕಡಿಮೆ ಮಾಡುತ್ತ ಬಂದರು. ಕಡೆಗೊಮ್ಮೆ ಒಂದು ದಿನ ಸುಮಾರು ೨೫ಂಂ ಡಾಲರ್ ಸಂಗ್ರಹವಾದರೆ, ಸ್ವಾಮೀಜಿಗೆ ಆ ಸಂಸ್ಥೆ ಕೊಟ್ಟಿದ್ದು ಕೇವಲ ೨ಂಂ ಡಾಲರ್! ಕ್ರಮೇಣ ಇವರು ಮಾಡುತ್ತಿರುವ ಅನ್ಯಾಯ ಎಷ್ಟು ಸ್ಪಷ್ಟವಾಯಿತೆಂದರೆ, ಪ್ರಾಪಂಚಿಕ ವ್ಯವಹಾರದ ಅರಿವೇ ಇಲ್ಲದ ಸ್ವಾಮೀಜಿಗೂ ಅದು ಅರ್ಥವಾಯಿತು. ಅವರು ಡೆಟ್ರಾಯ್ಟ್​ನಿಂದ ಶ್ರೀಮತಿ ಬೆಲ್ ಹೇಲ್​ರಿಗೆ ಬರೆದರು:

“ನನಗೆ ಈ ಸ್ಲೇಟನ್ ಸಂಸ್ಥೆಯ ವ್ಯವಹಾರವನ್ನು ಕಂಡು ಜುಗುಪ್ಸೆಯಾಗಿ ಬಿಟ್ಟಿದೆ. ನಾನು ಈ ಸಂಸ್ಥೆಯನ್ನು ಸೇರಿದಾಗಿನಿಂದ ಕಳೆದುಕೊಂಡದ್ದು ಕನಿಷ್ಠ ಪಕ್ಷ ೫ಂಂಂ ಡಾಲರ್ ಆದರೂ ಆಗಿರಬಹುದು. ಇದರಿಂದ ಬಿಡಿಸಿಕೊಳ್ಳಲು ನಾನು ತೀವ್ರವಾಗಿ ಪ್ರಯತ್ನಿಸುತ್ತಿದ್ದೇನೆ. ನಾನೀಗ ಖಾಸಗಿಯಾಗಿ ಕೆಲವು ಭಾಷಣ ಮಾಡಬೇಕೆಂದಿದ್ದೇನೆ. ಬಳಿಕ ಓಹಿಯೋಗೆ ಹೋಗಿ ಅಲ್ಲಿಂದ ಶಿಕಾಗೋಗೆ ಹಿಂದಿರುಗುತ್ತೇನೆ. ಪ್ರೆಸಿಡೆಂಟ್ ಪಾಮರ್​ರವರು ನನ್ನನ್ನು ಈ ಸ್ಲೇಟನ್ ಸುಳ್ಳು ಗಾರನ ಕೈಯಿಂದ ಬಿಡಿಸಲು ಶಿಕಾಗೋಗೆ ಹೋಗಿದ್ದಾರೆ. ಇಲ್ಲಿನ ಹಲವಾರು ನ್ಯಾಯಾಧೀಶರು ನಮ್ಮ ಕರಾರು ಪತ್ರವನ್ನು ನೋಡಿದ್ದಾರೆ. ಈ ಕರಾರು ಒಂದು ನಾಚಿಕೆಗೇಡಿನ ಮೋಸ, ಅದನ್ನು ಯಾವಾಗ ಬೇಕಾದರೂ ಮುರಿದುಹಾಕಬಹುದು ಎಂದು ಅವರು ಹೇಳಿದ್ದಾರೆ. ಆದರೆ ನಾನೊಬ್ಬ ಸಾಧು–ನಾನು ಆತ್ಮರಕ್ಷಣೆ ಮಾಡಿಕೊಳ್ಳಲಾರೆ. ಆದ್ದರಿಂದ ಇವೆಲ್ಲವನ್ನೂ ಕಿತ್ತೆಸೆದು ಭಾರತಕ್ಕೆ ಹಿಂದಿರುಗುವ ಮನಸ್ಸಾಗುತ್ತದೆ.”

ಆದರೆ ಕೆಲದಿನಗಳಲ್ಲೇ ಶ್ರೀ ಪಾಮರ್​ರವರ ಸಹಾಯದಿಂದಾಗಿ ಕರಾರು ಮುರಿದುಬಿತ್ತು. ಹೀಗೆ ಈ ಸಂಸ್ಥೆಯೊಂದಿಗೆ ಸ್ವಾಮೀಜಿ ಮಾಡುತ್ತಿದ್ದ ಭಾಷಣ ಪ್ರವಾಸವೂ ನಿಂತುಹೋಯಿತು. ಅವರೀಗ ತಮ್ಮ ಅನುಕೂಲಕ್ಕೆ ತಕ್ಕಂತೆ ನಡೆಯಲು ಸ್ವತಂತ್ರರಾದರು. ಧನಸಂಪಾದನೆಯ ವಿಷಯದಲ್ಲಿ ಅವರು ಎಷ್ಟರಮಟ್ಟಿಗೆ ಯಶಸ್ವಿಯಾದರೆಂದು ತಿಳಿಯುವುದು ಕಷ್ಟ. ಆದರೆ ಭಾಷಣ ಗಳ ಮೂಲಕ ಮಾತ್ರವಲ್ಲದೆ, ಸ್ವಯಂಸ್ಫೂರ್ತಿಯಿಂದ ಅನೇಕರು ನೀಡುತ್ತಿದ್ದ ಕಾಣಿಕೆ ಗಳಿಂದಲೂ ಧನ ಸಂಗ್ರಹಣೆ ನಡೆಯುತ್ತಿತ್ತು. ಉದಾಹರಣೆಗೆ ಚಾರ್ಲ್ಸ್ ಫ್ರೀಯರ್ ಎಂಬೊಬ್ಬ ಶ್ರೀಮಂತ ವ್ಯಕ್ತಿ ಅವರಿಗೆ ೨ಂಂ ಡಾಲರ್​ಗಳನ್ನು ಕಳಿಸಿಕೊಟ್ಟಿದ್ದ. ಇದು ಶ್ರೀಮತಿ ಬ್ಯಾಗ್​ಲೀ ಯವರ ಮಗಳಾದ ಫ್ಲಾರೆನ್ಸ್ ಬರೆಯುವ ಈ ಕೆಳಗಿನ ಪತ್ರದಿಂದ ತಿಳಿದುಬರುತ್ತದೆ. ಅಲ್ಲದೆ, ಬಹುಶಃ ತಮ್ಮ ಕಾರ್ಯ ತೃಪ್ತಿಕರವಾಗಿ ಸಾಗುತ್ತಿಲ್ಲವೆಂಬ ಕಾರಣದಿಂದ ಅವರು ಭಾರತಕ್ಕೆ ಹಿಂದಿರುಗುವ ಬಗ್ಗೆಯೂ ಆಲೋಚಿಸಿದ್ದರೆಂಬುದೂ ತಿಳಿದುಬರುತ್ತದೆ.

“ಪ್ರಿಯ ಶ್ರೀ ವಿವೇಕಾನಂದ,

ನನ್ನ ತಾಯಿಯ ಪರವಾಗಿ ನಾನು ಈ ಪತ್ರವನ್ನು ಬರೆಯುತ್ತಿದ್ದೇನೆ. ಇಲ್ಲಿ ನಾವು ಪಡೆದು ಕೊಂಡ ಹಣವನ್ನು ಇಂದೇ ಕಳಿಸಲಾಗದಿದ್ದುದರ ಬಗ್ಗೆ ಆಕೆ ವಿಷಾದಿಸುತ್ತಾಳೆ. ಅದನ್ನು ಸೋಮ ವಾರ ಬೆಳಗ್ಗೆ ಕಳಿಸಲಾಗುವುದು. ನಿಮಗೆ ಇದರಿಂದಾಗಿ ಏನೂ ತೊಂದರೆಯಾಗುವುದಿಲ್ಲವೆಂದು ಆಶಿಸುತ್ತೇನೆ.

“ನಾವು ಊಟಮಾಡಿದ ಮನೆಯವರಾದ ಶ್ರೀ ಫ್ರೀಯರ್ ಮಹಾಶಯರು ನಿಮಗೆ ಕಳಿಸಲೆಂದು ನಮ್ಮ ತಾಯಿಯ ಕೈಗೆ ೨ಂಂ ಡಾಲರ್ ಕೊಟ್ಟರು. ನಿಮಗೆ ಕಳಿಸುವ ಬ್ಯಾಂಕ್ ಡ್ರಾಫ್ಟ್​ನೊಂದಿಗೆ ಅದನ್ನು ಕಳಿಸಿಕೊಡುತ್ತೇನೆ.

“ನನ್ನನ್ನು ನಂಬಿ, ಸ್ವಾಮೀಜಿ–ನೀವು ಶ್ರೀ ಫ್ರೀಯರ್​ರಂಥವರ ನೆರವನ್ನೂ ಪ್ರೋತ್ಸಾಹ ವನ್ನೂ ಗಳಿಸಲು ಸಮರ್ಥರಾಗಿರುವಾಗ, ನಿಮ್ಮ ಕಾರ್ಯೋದ್ದೇಶ ಸಾಧನೆಗಾಗಿ ನೀವು ಇನ್ನೂ ಹೆಚ್ಚಿನ ಮತ್ತೊಂದು ಪ್ರಯತ್ನವನ್ನು ಮಾಡದೆ ಭಾರತಕ್ಕೆ ಹಿಂದಿರುಗಲು ನಿಮಗೆ ಹಕ್ಕಿಲ್ಲ! ನಿಮ್ಮ ಪ್ರಬಲ, ಅಯಸ್ಕಾಂತೀಯ ವ್ಯಕ್ತಿತ್ವದಿಂದೊಡಗೂಡಿದಾಗ ಮಾತ್ರವೇ ನಿಮ್ಮ ಮನವಿಯು ಹೆಚ್ಚು ಪರಿಣಾಮಕಾರಿಯಾಗುತ್ತದೆ. ನಾವು ನಿಮಗೆ ಯಾವ ರೀತಿಯಲ್ಲಾದರೂ ನೆರವಾಗಬಲ್ಲೆ ವಾದರೆ–ನೀವು ನಮ್ಮ ವಿಷಯದಲ್ಲಿ ವಿಶ್ವಾಸವಿಟ್ಟು, ನಮ್ಮ ಸೇವೆಯನ್ನು ಪಡೆದುಕೊಳ್ಳುತ್ತೀ ರೆಂದು ನಂಬಿದ್ದೇನೆ.”

ತಮಗೆ ಈ ಭಾಷಣಗಳು ಅಪಾರ ಕೀರ್ತಿಯನ್ನೂ ಜನಪ್ರಿಯತೆಯನ್ನೂ ತಂದುಕೊಡುತ್ತಿದ್ದ ಈ ಸಮಯದಲ್ಲಿ ಸ್ವಾಮೀಜಿ ಯಾವ ಮನಸ್ಥಿತಿಯಲ್ಲಿದ್ದರು ಎಂಬುದನ್ನು ಗಮನಿಸಿದಾಗ ಅತ್ಯಾಶ್ಚರ್ಯವಾಗುತ್ತದೆ. ತಮ್ಮ ನಿರೀಕ್ಷಿಗೆ ತಕ್ಕಂತೆ ಧನಸಂಗ್ರಹಣೆ ಮಾಡಲು ಸಾಧ್ಯವಾಗುತ್ತಿಲ್ಲ ವಲ್ಲ ಎಂಬ ಕಾರಣಕ್ಕಾಗಿ ಅವರು ನಿರಾಶರಾಗಿದ್ದರೋ ಎಂಬಂತೆ ತೋರುತ್ತದೆ. ಆದರೆ ನಿಜ ಸಂಗತಿ ಅದಲ್ಲ. ನಿಜಕ್ಕೂ ಅವರ ಮನಸ್ಸಿನ ಕಾರ್ಯವಿಧಾನವನ್ನು ಅರಿಯುವುದೇ ಬಹಳ ಕಷ್ಟ. ಅವರು ತಮ್ಮ ಆತ್ಮೀಯ ಶಿಷ್ಯೆಯರಾದ ಹೇಲ್ ಸೋದರಿಯರಿಗೆ ಪತ್ರವೊಂದರಲ್ಲಿ ಬರೆಯುತ್ತಾರೆ:

“ನಿಜ ಹೇಳಬೇಕೆಂದರೆ ನನ್ನ ಉಪನ್ಯಾಸಗಳು ಹೆಚ್ಚುಹೆಚ್ಚು ಜನಪ್ರಿಯವಾದಂತೆಲ್ಲ ಮತ್ತು ಅವಕಾಶಗಳು ಹೆಚ್ಚಾದಂತೆಲ್ಲ ನನಗೆ ಅದರ ಮೇಲೆ ಹೆಚ್ಚು ಹೆಚ್ಚು ಜುಗುಪ್ಸೆಯಾಗುತ್ತಿದೆ. ನನ್ನ ಇತ್ತೀಚಿನ ಉಪನ್ಯಾಸವೇ ನಾನು ಇದುವರೆಗೆ ಮಾಡಿದವುಗಳಲ್ಲೆಲ್ಲ ಅತ್ಯುತ್ತಮವಾದದ್ದು. ಶ್ರೀ ಪಾಮರ್​ರವರು ಆನಂದೋದ್ರೇಕಗೊಂಡಿದ್ದರು, ಮತ್ತು ಸಭಿಕರೆಲ್ಲ ಮಂತ್ರಮುಗ್ಧರಾದಂತೆ ಕುಳಿತಿದ್ದರು. ಅದು ಎಷ್ಟರ ಮಟ್ಟಿಗೆಂದರೆ ಉಪನ್ಯಾಸವಾದ ಮೇಲೆಯೇ ನನಗೆ ಗೊತ್ತಾದದ್ದು, ನಾನು ಬಹಳ ಹೊತ್ತು ಮಾತನಾಡಿಬಿಟ್ಟೆ ಎಂದು. ತನ್ನ ಶ್ರೋತೃಗಳಿಗೆ ಕಷ್ಟವಾಗುತ್ತಿದ್ದರೆ ಭಾಷಣ ಕಾರನಿಗೆ ಸದಾ ಅದರ ಅರಿವಿರುತ್ತದೆ. ಆದ್ದರಿಂದ ಈ ಅರಿವನ್ನು ಕಳೆದುಕೊಳ್ಳುವಂತಹ ಅವಿವೇಕ ದಿಂದ ಭಗವಂತ ನನ್ನನ್ನು ಪಾರು ಮಾಡಲಿ! ನನಗೆ ಸಾಕಾಗಿ ಹೋಗಿದೆ.”

ಶ್ರೋತೃಗಳು ಮಂತ್ರಮುಗ್ಧರಾಗಿ ಕುಳಿತುಕೊಳ್ಳುವಂತಹ ಸ್ಥಿತಿಯಲ್ಲಿರುವಾಗ, ಅವರಿಗೆ ಕಷ್ಟ ವೆಲ್ಲಿಯದು! ಆದರೂ ಇಂತಹ ಸ್ಥಿತಿಯಲ್ಲಿ ತಾವು ಮೈಮರೆಯುವುದನ್ನು ಸ್ವಾಮೀಜಿ ಅವಿವೇಕ ವೆಂದು ಕರೆಯುತ್ತಿದ್ದಾರೆ! ಅತ್ಯುತ್ತಮ ವಾಗ್ಮಿಯಾಗಿ ಯಶಸ್ಸಿನ ಶಿಖರವನ್ನು ಮುಟ್ಟಿದ ಈ ಸಂತಸದ ಸಂದರ್ಭದಲ್ಲಿ ಅವರಿಗೆ ಜುಗುಪ್ಸೆಯಾಗಿದೆಯಂತೆ! ಈ ವಿರೋಧಾಭಾಸವನ್ನು ಅವರೇ ಮತ್ತೊಂದು ಪತ್ರದಲ್ಲಿ ಹೇಲ್ ಸೋದರಿಯರಿಗೆ ವಿವರಿಸುತ್ತಾರೆ:

“ಇಲ್ಲಿಯವರೆಗೆ ಎಲ್ಲವೂ ಚೆನ್ನಾಗಿಯೇ ಇದೆ. ಆದರೆ ಅದೇಕೋ ಇಲ್ಲಿಗೆ ಬಂದಾಗಿನಿಂದ ನನ್ನೆದೆಯಲ್ಲಿ ಏನೋ ಒಂದು ಬಗೆಯ ವಿಷಾದ ಆವರಿಸಿದೆ–ಏಕೆಂದು ಗೊತ್ತಿಲ್ಲ.

“ಈ ಉಪನ್ಯಾಸವೇ ಮೊದಲಾದ ಹುಚ್ಚುತನಗಳಿಂದ ನನಗೆ ಸಾಕಾಗಿಹೋಗಿದೆ. ಈ ನೂರಾರು ಬಗೆಯ ಮನುಷ್ಯ ಪ್ರಾಣಿಗಳ ಒಡನಾಟ ನನ್ನ ಶಾಂತಿಯನ್ನು ಕದಡಿದೆ. ನನ್ನ ನಿಜವಾದ ಅಭಿರುಚಿ ಏನೆಂದು ಹೇಳುತ್ತೇನೆ: ನಾನು ಬರೆಯಲಾರೆ, ನಾನು ಮಾತನಾಡಲಾರೆ; ಆದರೆ ನಾನು ಆಳವಾಗಿ ಆಲೋಚಿಸಬಲ್ಲೆ, ಮತ್ತು ನನ್ನಲ್ಲಿ ಕಾವೇರಿದಾಗ ಕಿಡಿಗಳನ್ನು ಕಾರುತ್ತ ಮಾತನಾಡಬಲ್ಲೆ. ಆದರೆ ನಾನು ಮಾತನಾಡುವುದೇನಿದ್ದರೂ ಆಯ್ದ ಕೆಲಮಂದಿಗೆ, ಕೆಲವೇ ಮಂದಿಗೆ ಮಾತ್ರ. ಅವರು ಬೇಕಾದರೆ ನನ್ನ ಸಂದೇಶವನ್ನು ತೆಗೆದುಕೊಂಡು ಹೋಗಿ ಪ್ರಸಾರ ಮಾಡಲಿ–ನಾನದನ್ನು ಮಾಡಲಾರೆ. ಇದು ಕೆಲಸವನ್ನು ನಾಲ್ಕಾರು ಜನ ಹಂಚಿಕೊಂಡಂತೆ ಅಷ್ಟೆ. ಆಲೋಚಿಸುವುದು ಮತ್ತು ತನ್ನ ಭಾವನೆಗಳನ್ನು ಪ್ರಸಾರ ಮಾಡುವುದು, ಈ ಎರಡೂ ಕಾರ್ಯಗಳಲ್ಲಿ ಒಬ್ಬನೇ ವ್ಯಕ್ತಿ ಎಂದೂ ಯಶಸ್ವಿಯಾಗಿಲ್ಲ. ಆಲೋಚನೆ ಮಾಡಬೇಕಾದರೆ, ಅದಲ್ಲೂ ಆಧ್ಯಾತ್ಮಿಕ ವಿಚಾರಗಳ ಬಗ್ಗೆ ಆಲೋಚಿಸಬೇಕಾದರೆ, ಅವನು ಸ್ವತಂತ್ರನಾಗಿರಬೇಕು....

“ನಿಜಕ್ಕೂ ನಾನು ‘ಸುಂಟರಗಾಳಿ’ ಅಲ್ಲವೇ ಅಲ್ಲ. ಅದಕ್ಕಿಂತ ನಾನು ಸಂಪೂರ್ಣ ವಿಭಿನ್ನ. ನನಗೆ ಬೇಕಾದದ್ದು ಇಲ್ಲಿಲ್ಲ. ಅಲ್ಲದೆ ಇನ್ನು ನನ್ನಿಂದ ಈ ಸುಂಟರಗಾಳಿಯ ವಾತಾವರಣವನ್ನು ಸಹಿಸಿಕೊಳ್ಳಲು ಸಾಧ್ಯವೂ ಇಲ್ಲ. ಪರಿಪೂರ್ಣತೆಗೆ ದಾರಿ ಇದೇ–ಏನೆಂದರೆ ಪರಿಪೂರ್ಣರಾಗಲು ಶ್ರಮಿಸುವುದು, ಹಾಗೂ ಕೆಲವು ಸ್ತ್ರೀ-ಪುರುಷರನ್ನು ಪರಿಪೂರ್ಣರಾಗಿಸಲು ಶ್ರಮಿಸುವುದು. ಒಳಿತನ್ನು ಸಾಧಿಸುವ ವಿಷಯದಲ್ಲಿ ನನ್ನ ಆಲೋಚನೆ ಇದು: ಹಂದಿಗಳ ಮುಂದೆ ಮುತ್ತು ಚೆಲ್ಲಿ ಸಮಯ, ಆರೋಗ್ಯ, ಶಕ್ತಿಗಳನ್ನು ಹಾಳುಮಾಡಿಕೊಳ್ಳುವುದಲ್ಲ; ಬದಲಿಗೆ ಪ್ರಚಂಡರಾದ ಕೆಲವೇ ವ್ಯಕ್ತಿಗಳನ್ನು ನಿರ್ಮಿಸುವುದು...”

ಆದರೆ ಸ್ವಾಮೀಜಿಗೆ ಎಷ್ಟೇ ಆಯಾಸವಾಗಿರಲಿ, ಜುಗುಪ್ಸೆಯಾಗಿರಲಿ–ದೈವೇಚ್ಛೆಯೆಂಬುದು ಅವರಿಂದ ಬಲವಂತವಾಗಿ ಕೆಲಸ ಮಾಡಿಸಿಕೊಳ್ಳುತ್ತಿತ್ತು. ಶ್ರಮದ ವಿಭಜನೆಯಾಗಬೇಕು, ತಮ್ಮ ಕೆಲಸವನ್ನು ನಾಲ್ಕಾರು ಜನ ಹಂಚಿಕೊಳ್ಳುವಂತಾಗಬೇಕು ಎಂದು ಅವರು ಎಷ್ಟೇ ಹೇಳಿದರೂ ಅದೆಲ್ಲ ಅಷ್ಟು ಬೇಗ ಸಾಧ್ಯವಾಗುವಂತಿರಲಿಲ್ಲ. ಆದ್ದರಿಂದ ಸ್ವಾಮೀಜಿ ತಾವೇ ಭಾಷಣಗಳನ್ನು ಮಾಡಿ, ತಮ್ಮ ಭಾವನೆಗಳನ್ನು ತಾವೇ ಪ್ರಸಾರ ಮಾಡಬೇಕಾಯಿತು. ಅಷ್ಟೇ ಅಲ್ಲ, ಅವರು ತಮ್ಮ ಕಾರ್ಯಕ್ಷೇತ್ರವನ್ನು ಮತ್ತಷ್ಟು ವಿಸ್ತರಿಸಬೇಕಾಯಿತು. ತಾವು ಇನ್ನು ಮುಂದೆ ಕೆಲವೇ ಕೆಲವು ಆಯ್ದ ವ್ಯಕ್ತಿಗಳಿಗೆ ಮಾತ್ರ ಬೋಧನೆ ಮಾಡಬೇಕೆಂಬ ಬಯಕೆ ಅವರಲ್ಲಿ ಮೂಡಿತ್ತಾದರೂ, ಅದು ಕೈಗೂಡಿದ್ದು ಕೆಲ ಕಾಲದ ಅನಂತರವೇ. ಎಂದರೆ ಅವರು ತಮ್ಮ ಸಂದೇಶಗಳನ್ನು ವ್ಯಾಪಕ ವಾಗಿ ಪ್ರಸಾರ ಮಾಡಿ, ತಮ್ಮ ಕಾರ್ಯೋದ್ದೇಶದ ಮೊದಲ ಹಂತವನ್ನು ಮುಗಿಸಿದ ಬಳಿಕವೇ. ಆಗಿಂದಾಗ್ಗೆ ಸ್ವಾಮೀಜಿ ತಮ್ಮ ಮನಸ್ಸಿನ ಒಲವನ್ನು ವ್ಯಕ್ತಪಡಿಸಬಹುದು, ಇಲ್ಲವೆ ಯಾವುದಾ ದರೂ ತಮ್ಮ ನಿರೀಕ್ಷೆಗೆ ವಿರುದ್ಧವಾಗಿ ನಡೆದದ್ದರ ಬಗ್ಗೆ ಅಸಹನೆ ತಾಳಬಹುದು. ಆದರೆ ಅವರು ಯಾವಾಗಲೂ ಭಗವದಿಚ್ಛೆಗೆ ಮಣಿಯುತ್ತಿದ್ದರು. ತಮ್ಮ ಉದ್ದೇಶಿತ ಯೋಜನೆಯು ಭಗವ ದಿಚ್ಛೆಗೆ ಅನುಗುಣವಾಗಿಲ್ಲವೆಂದು ಅವರಿಗೆ ಗೊತ್ತಾದರೆ, ತಕ್ಷಣ ಅದನ್ನು ಕೈಬಿಡಲು ಸಿದ್ಧರಾಗಿರು ತ್ತಿದ್ದರು. ಈ ಸಂಗತಿಯನ್ನು ಅವರು ತಮ್ಮ ಪತ್ರಗಳಲ್ಲಿ ವ್ಯಕ್ತಪಡಿಸಿದ್ದುಂಟು.

ಸರ್ವಧರ್ಮಸಮ್ಮೇಳನಾ ನಂತರದ ದಿನಗಳ ಸ್ವಾಮೀಜಿಯ ಜೀವನವನ್ನು ಅಧ್ಯಯನ ಮಾಡಿ ದಾಗ, ಅದರಲ್ಲೂ ಡೆಟ್ರಾಯ್ಟ್​ನ ‘ಕಾಳಗ’ದ ಬಳಿಕ, ಅವರ ವ್ಯಕ್ತಿತ್ವದಲ್ಲಿ ಪೌರುಷ ಹಾಗೂ ಶರಣಾಗತಿಭಾವಗಳು ಸಮರಸವಾಗಿ ವ್ಯಕ್ತವಾದದ್ದನ್ನು ಕಾಣಬಹುದಾಗಿದೆ. ಸ್ವಾಮೀಜಿಯ ಧೀರತೆ ಯನ್ನು ಕಂಡು ಅಲ್ಲಿನ ಜನ ಅವರನ್ನು ‘ಯೋಧಯೋಗಿ\eng{’ ‘Militant Mystic’}ಎಂದು ಕರೆದರು. ಅವರು ಯೋಧನಾಗಿ ಕಾಣುವುದೇನೋ ನಿಜವೇ. ಆದರೆ ಅವರು ಹಾಗೆ ಕಾಣುವುದು, ಅವರ ಬಾಹ್ಯವ್ಯಕ್ತಿತ್ವವನ್ನು ಮಾತ್ರ ನೋಡುವವರಿಗೆ, ಆದ್ದರಿಂದ ಅವರ ಪೌರುಷಪೂರ್ಣ ಯೋಧ ವ್ಯಕ್ತಿತ್ವಕ್ಕಿಂತಲೂ ಒಳಹೊಕ್ಕು ನೋಡಲು ಸಾಧ್ಯವಾಗದಿದ್ದವರಿಗೆ, ಅವರ ಶರಣಾಗತಿಭಾವವು ಕಾಣಲೇ ಇಲ್ಲ; ಅವರು ತಮ್ಮೆಲ್ಲ ಕಾರ್ಯಕಲಾಪಗಳಲ್ಲೂ ಭಗವಂತನ ಇಚ್ಛೆಗೆ ಮಣಿದು ಅವನ ನೇರ ಮಾರ್ಗದರ್ಶನವನ್ನು ಪಡೆಯುತ್ತಿದ್ದರೆಂಬುದು ತಿಳಿಯಲೇ ಇಲ್ಲ. ಅವರು ತಮ್ಮ ಸೋದರ ಸಂನ್ಯಾಸಿಗಳಾದ ಸ್ವಾಮಿ ರಾಮಕೃಷ್ಣಾನಂದರಿಗೆ ಬರೆದ ಒಂದು ಪತ್ರದಲ್ಲಿ ಅವರ ಈ ಭಾವವು ಎದ್ದುಕಾಣುತ್ತದೆ:

“ಭಗವಂತನ ಇಚ್ಛೆಯಿಂದ, ಹೆಸರು ಕೀರ್ತಿಗಳ ಮೋಹವು ನನ್ನ ಹೃದಯವನ್ನಿನ್ನೂ ಹೊಕ್ಕಿಲ್ಲ; ಮತ್ತು ಹೋಗುವುದೂ ಇಲ್ಲವೆಂದು ಹೇಳುವ ಧೈರ್ಯಮಾಡುತ್ತೇನೆ. ನಾನು ಯಂತ್ರ, ಅವನು ಚಾಲಕ. ಈ ಯಂತ್ರದ ಮೂಲಕ ಭಗವಂತ ಈ ದೂರದ ನಾಡಿನಲ್ಲಿ ಸಾವಿರಾರು ಜನರ ಹೃದಯ ದಲ್ಲಿ ಧಾರ್ಮಿಕ ಪ್ರಜ್ಞೆಯನ್ನು ಜಾಗೃತಗೊಳಿಸುತ್ತಿದ್ದಾನೆ. ಸಾವಿರಾರು ಸ್ತ್ರೀಪುರುಷರು ನನ್ನನ್ನು ಪ್ರೀತಿಸುತ್ತಾರೆ, ಗೌರವಿಸುತ್ತಾರೆ... ‘ಮೂಕಂ ಕರೋತಿ ವಾಚಾಲಂ ಪಂಗುಂ ಲಂಘಯತೇ ಗಿರಿಂ’. ಭಗವಂತನ ಕೃಪೆಯನ್ನು ಕಂಡು ನಾನು ವಿಸ್ಮಯಮೂಕನಾಗಿದ್ದೇನೆ. ನಾನು ಹೋಗುವ ಊರುಗಳಲ್ಲೆಲ್ಲ ಒಂದು ಕೋಲಾಹಲವೇ ಏರ್ಪಡುತ್ತದೆ. ಅವರು ನನಗೆ ‘ಸುಂಟರಗಾಳೀ ಹಿಂದು’ ಎಂದು ಹೆಸರು ಕೊಟ್ಟಿದ್ದಾರೆ. ನೆನಪಿಡು, ಇವೆಲ್ಲ ಭಗವಂತನ ಇಚ್ಛೆ; ನಾನು ಆಕಾರ ವಿಲ್ಲದ ಒಂದು ದನಿ ಮಾತ್ರ...

“ನಾನು ಇಲ್ಲಿಂದ ಇಂಗ್ಲೆಂಡಿಗೆ ಹೋಗುವೆನೋ ಅಥವಾ ಮತ್ತಾವ ಸ್ಥಳಕ್ಕೆ ಹೋಗುವೆನೋ, ಭಗವಂತನಿಗೆ ಗೊತ್ತು. ಅವನೇ ಎಲ್ಲಕ್ಕೂ ವ್ಯವಸ್ಥೆ ಮಾಡುತ್ತಾನೆ.”

ಅವರಿಗೆ ಭಗವಂತನ ಆದೇಶ ದೊರುಕುತ್ತಿದ್ದ ಬಗೆ ಹೇಗೆಂಬುದು ಮಾತ್ರ ಯಾರಿಗೂ ತಿಳಿಯುವಂತಿರಲಿಲ್ಲ. ಆದರೆ ಒಂದು ನಿರ್ದಿಷ್ಟ ಸಂದರ್ಭದಲ್ಲಿ ಅದು ಅವರಿಗೆ ದೊರಕಿತೇ ಇಲ್ಲವೇ ಎಂಬುದನ್ನು ತಿಳಿಸುವ ಅಪೂರ್ವ ಪತ್ರವೊಂದು ಇಲ್ಲಿದೆ. ಶ್ರೀಮತಿ ಬೆಲ್​ಹೇಲ್​ಳಿಗೆ ೧೮೯೪ರ ಜುಲೈನಲ್ಲಿ ಅವರು ಹೀಗೆ ಬರೆದರು–“ಹೆಚ್ಚಿನ ಭಾಗ, ಸದ್ಯದಲ್ಲೇ ನಾನು ಇಂಗ್ಲೆಂಡಿಗೆ ಹೋಗುವುದು ಖಂಡಿತ. ಆದರೆ, ಒಂದು ಗುಟ್ಟಿನ ವಿಷಯ, ನಾನು ಒಂದು ಬಗೆಯ ಅತೀಂದ್ರಿಯ ವಾದಿ \eng{(Mystic);} ಆದ್ದರಿಂದ ನಾನು ಆದೇಶವಿಲ್ಲದೆ ಮುಂದುವರಿಯಲಾರೆ. ಆ ಆದೇಶವಿನ್ನೂ ದೊರೆತಿಲ್ಲ.”

ಹೀಗೆ ಸ್ವಾಮೀಜಿ, ಭಗವದಿಚ್ಛೆಯ ಅಲೆಯಲ್ಲಿ ತೇಲಿ ಸಾಗುತ್ತ, ಅವಿರತವಾಗಿ ತಮ್ಮ ಕಾರ್ಯ ವನ್ನು ಮುಂದುವರಿಸಿದರು. ಮುಖ್ಯವಾಗಿ ಅಮೆರಿಕದ ಪೂರ್ವದ ರಾಜ್ಯಗಳಲ್ಲಿ ಸ್ವತಂತ್ರವಾಗಿ ಸಂಚರಿಸುತ್ತ, ಉಪನ್ಯಾಸಗಳನ್ನು ಮುಂದುವರಿಸಿದರು. ಆಗಾಗ ಅವರು ಕೆಲದಿನಗಳ ಮಟ್ಟಿಗೆ ಶಿಕಾಗೋಗೆ ಹಿಂದಿರುಗುತ್ತಿದ್ದರು. ಸಾರ್ವಜನಿಕ ಉಪನ್ಯಾಸಗಳನ್ನು ನೀಡುವಂತೆ ತಮಗೆ ಬರು ತ್ತಿದ್ದ ಆಹ್ವಾನಗಳೊಂದಿಗೆ, ಇತರ ಸಂಘ ಸಂಸ್ಥೆಗಳ ಆಶ್ರಯದಲ್ಲಿ ಅನೌಪಚಾರಿಕ ಉಪನ್ಯಾಸ ಗಳ ಆಹ್ವಾನಗಳನ್ನೂ ಅವರು ಅಷ್ಟೇ ವಿಶ್ವಾಸದಿಂದ ಸ್ವೀಕರಿಸುತ್ತಿದ್ದರು.

ಸ್ವಾಮೀಜಿ ಡೆಟ್ರಾಯ್ಟ್​ನಲ್ಲಿದ್ದಾಗಲೇ, ಪೂರ್ವ ಕರಾವಳಿಯ ನಗರಗಳಲ್ಲಿನ ಅವರ ಭಾಷಣ ಗಳ ಕಾರ್ಯಕ್ರಮ ಸ್ವಲ್ಪಮಟ್ಟಿಗೆ ನಿಶ್ಚಿತವಾಗಿತ್ತು. ಅವರೀಗ ಸ್ವತಂತ್ರವಾಗಿ ಭಾಷಣಪ್ರವಾಸ ವನ್ನು ಕೈಗೊಂಡದ್ದರಿಂದ, ವಿಶ್ವಾಸಿಗಳ ಸಹಕಾರದ ಅಗತ್ಯ ಮೊದಲಿಗಿಂತ ಹೆಚ್ಚಾಗಿತ್ತು. ಆದರೆ ಅವರಿಗೆ ಬೇಕಾದ ಸಹಾಯ ತಾನಾಗಿಯೇ ಬಂದೊದಗುತ್ತಿತ್ತು. ಒಬ್ಬರ ಜೊತೆಗಾದ ಪರಿಚಯ ದಿಂದಾಗಿ ಮತ್ತೊಬ್ಬರ ಪರಿಚಯವಾಗುತ್ತಿತ್ತು. ಒಂದು ಆಹ್ವಾನದ ಮೂಲಕ ಇನ್ನೊಂದು ಆಹ್ವಾನಕ್ಕೆ ದಾರಿಯಾಗುತ್ತಿತ್ತು. ಹೀಗೆ ಅವರು ಮುನ್ನಡೆದಂತೆಲ್ಲ ಮುಂದಿನ ದಾರಿ ತೆರೆಯುತ್ತ ಬಂದು, ಅವರ ಕಾರ್ಯ ನಿರಾತಂಕವಾಗಿ, ಅವಿರತವಾಗಿ ಸಾಗಿತು. ಒಟ್ಟಿನಲ್ಲಿ ಅದೊಂದು ಪೂರ್ವ ನಿಯೋಜಿತ ನಾಟಕದೋಪಾದಿಯಲ್ಲಿ ನಡೆದುಕೊಂಡು ಹೋಗುತ್ತಿತ್ತು.

ಏಪ್ರಿಲ್ ಆರಂಭದ ಸುಮಾರಿಗೆ ಸ್ವಾಮೀಜಿ ನ್ಯೂಯಾರ್ಕ್ ನಗರಕ್ಕೆ ಬಂದರು. ಇಲ್ಲಿಗೆ ಅವರು ತಮಗೆ ಮೇರಿ ಹೇಲ್​ಳ ಮೂಲಕ ಪರಿಚಯವಾಗಿದ್ದ ಡಾ ॥ ಎಗ್​ಬರ್ಟ್ ಗರ್ನ್​ಸೇ ದಂಪತಿಗಳ ಆಹ್ವಾನದ ಮೇರೆಗೆ ಬಂದಿದ್ದರು. ಶೀಘ್ರದಲ್ಲೇ ಇವರು ಸ್ವಾಮೀಜಿಯ ಭಕ್ತರಾದರು. ಡಾ ॥ ಗರ್ನ್​ಸೇ ನ್ಯೂಯಾರ್ಕಿನ ವೈದ್ಯಕೀಯ ಪತ್ರಿಕೆಯೊಂದರ ಮುಖ್ಯ ಸಂಪಾದಕರೂ ‘ಯೂನಿಯನ್ ಲೀಗ್ ಕ್ಲಬ್​’ ಎಂಬ ಪ್ರತಿಷ್ಠಿತ ಕ್ಲಬ್ಬಿನ ಸಂಸ್ಥಾಪಕರೂ ಆಗಿದ್ದರು. ಈ ದಂಪತಿಗಳು ಸ್ವಾಮೀಜಿ ಯನ್ನು ಬಹಳ ಚೆನ್ನಾಗಿ ನೋಡಿಕೊಂಡರು. ಇವರ ಮೂಲಕ ಸ್ವಾಮೀಜಿಗೆ ನಗರದ ಇತರ ಕೆಲವು ಗಣ್ಯರ ಸ್ನೇಹವಾಯಿತು. ಆ ಕಾಲದ ಅತಿ ಶ್ರೀಮಂತರಲ್ಲೊಬ್ಬನಾದ ಜೇ ಗೌಲ್ಡ್ ಎಂಬವನ ಮಗಳಾದ ಮಿಸ್ ಹೆಲೆನ್ ಗೌಲ್ಡ್ ಕೂಡ ಇವರಲ್ಲೊಬ್ಬಳು. ನ್ಯೂಯಾರ್ಕಿನಲ್ಲಿ ಸ್ವಾಮೀಜಿ ಸುಮಾರು ಎರಡು ವಾರ ಇದ್ದರು. ಈ ಅವಧಿಯಲ್ಲಿ ಅವರೊಮ್ಮೆ ‘ಯೂನಿಯನ್ ಲೀಗ್ ಕ್ಲಬ್​’ನಲ್ಲಿ ನಗರದ ಅತ್ಯಂತ ಬುದ್ಧಿವಂತ ವರ್ಗದ ಕೆಲವರನ್ನುದ್ದೇಶಿಸಿ ಮಾತನಾಡಿದರು. ಇಂತಹ ಅನೌಪಚಾರಿಕ ಸಭೆಗಳಲ್ಲಿನ ಭಾಷಣಗಳು ಅತ್ಯಂತ ಮಹತ್ವದ್ದಾಗಿದ್ದು, ಎಷ್ಟೋ ಸಲ ಸಾರ್ವಜನಿಕ ಭಾಷಣಗಳಿಗಿಂತಲೂ ಹೆಚ್ಚು ಉಪಯುಕ್ತವೂ ಪ್ರಭಾವಶಾಲಿಯೂ ಆಗಿ ಪರಿಣಮಿಸುತ್ತಿದ್ದುವು.

ಏಪ್ರಿಲ್ ೧೭ರಂದು ಮಸಾಚುಸೆಟ್ಸ್​ನ ಲಿನ್ ಎಂಬಲ್ಲಿ ಸ್ವಾಮೀಜಿಯ ಭಾಷಣದ ಕಾರ್ಯ ಕ್ರಮ ನಿಶ್ಚಯವಾಗಿತ್ತು. ಆದರೆ ಈ ಮಧ್ಯೆ ಏಪ್ರಿಲ್ ೧೪ರಂದು ಅದೇ ರಾಜ್ಯದ ನಾರ್ಥಾಂಟನ್ ನಲ್ಲೂ ಭಾಷಣ ನೀಡಲು ಆಹ್ವಾನ ಬಂದದ್ದರಿಂದ ಅವರು ನ್ಯೂಯಾರ್ಕಿನಿಂದ ಮೊದಲು ಅಲ್ಲಿಗೆ ಹೊರಟರು. ಇಲ್ಲಿ ಅವರು ಎರಡು ಉಪನ್ಯಾಸಗಳನ್ನು ನೀಡಬೇಕಾಗಿದ್ದು, ಇಲ್ಲಿನ ಪ್ರಸಿದ್ಧ ಸ್ಮಿತ್ ಕಾಲೇಜಿನಲ್ಲಿ ಅವರ ಎರಡನೆಯ ಉಪನ್ಯಾಸ ಏರ್ಪಾಡಾಗಿತ್ತು. ಈ ಸಾರ್ವಜನಿಕ ಉಪನ್ಯಾಸದ ಬಳಿಕ, ಸ್ವಾಮೀಜಿ ಇಳಿದುಕೊಂಡಿದ್ದ ವಸತಿಗೃಹಕ್ಕೆ ನಗರದ ಅನೇಕ ಗಣ್ಯವ್ಯಕ್ತಿಗಳು ಬಂದು ಅವರೊಂದಿಗೆ ಸಂಭಾಷಿಸಿದರು. ಈ ಸಂದರ್ಭದಲ್ಲಿ ಹಾಜರಿದ್ದ ಮಾರ್ಥಾ ಬ್ರೌನ್ ಫಿಂಕೆ ಎಂಬ ವಿದ್ಯಾರ್ಥಿನಿ, ತನ್ನ ಸ್ಮೃತಿ ಲೇಖನದಲ್ಲಿ ಈ ದೃಶ್ಯವನ್ನು ಹೀಗೆ ಬಣ್ಣಿಸುತ್ತಾಳೆ:

“ಸ್ವಾಮೀಜಿ ಅಂದು ಮಾಡಿದ ಭಾಷಣದ ವಿಷಯ ನನಗೆ ಅಷ್ಟಾಗಿ ನೆನಪಿಲ್ಲ. ಆದರೆ ಆ ಬಳಿಕ ನಡೆದ ಚರ್ಚೆ ಬಹಳ ಚೆನ್ನಾಗಿ ನೆನಪಿದೆ. ಈ ಸಭೆಯಲ್ಲಿ ನಮ್ಮ ಕಾಲೇಜಿನ ಅಧ್ಯಕ್ಷರು, ತತ್ತ್ವಶಾಸ್ತ್ರವಿಭಾಗದ ಮುಖ್ಯಸ್ಥರು ಹಾಗೂ ಇತರ ಹಲವಾರು ಪ್ರಾಧ್ಯಾಪಕರು, ಊರಿನ ಅನೇಕ ಚರ್ಚುಗಳ ಧರ್ಮಾಧಿಕಾರಿಗಳು, ಮುಂತಾದವರಿದ್ದರು. ಕೋಣೆಯ ಒಂದು ಮೂಲೆಯಲ್ಲಿ ನಾವು ಕೆಲವು ಹುಡುಗಿಯರು ಇಲಿಮರಿಗಳಂತೆ ಕುಳಿತುಕೊಂಡು ಮಾತುಕತೆಯನ್ನು ಕಾತರತೆಯಿಂದ ಆಲಿಸುತ್ತಿದ್ದೆವು. ಚರ್ಚೆಯ ವಿಷಯ ಕ್ರೈಸ್ತಧರ್ಮ. ಕ್ರೈಸ್ತಧರ್ಮಾಧಿಕಾರಿಗಳೆಲ್ಲ ಸೇರಿಕೊಂಡು, ಕ್ರೈಸ್ತಧರ್ಮವೊಂದೇ ನಿಜವಾದ ಧರ್ಮ ಎಂಬ ವಾದವನ್ನು ಹೂಡಿದ್ದರು. ಕಪ್ಪು ಕೋಟುಗಳನ್ನು ಧರಿಸಿದ್ದ, ಗಂಭೀರ ಮುಖಮುದ್ರೆಯ ಹಲವಾರು ಪಂಡಿತರು, ಸ್ವಾಮೀಜಿಗೆ ಸವಾಲು ಹಾಕು ವಂತೆ ಕುಳಿತಿದ್ದರು. ಈ ನಮ್ಮ ಧರ್ಮಾಧಿಕಾರಿಗಳಿಗೆ ಖಂಡಿತವಾಗಿಯೂ ಹೆಚ್ಚಿನ ಸೌಲಭ್ಯವಿತ್ತು. ಇವರಿಗಾದರೆ ನಮ್ಮ ಬೈಬಲಿನ ವಿಷಯ, ಇತರ ಪಾಶ್ಚಾತ್ಯ ತತ್ತ್ವಶಾಸ್ತ್ರಗಳ ವಿಷಯ ಹಾಗೂ ಹಲವಾರು ಕವಿಗಳ ಮತ್ತು ಭಾಷ್ಯಕಾರರ ಕೃತಿಗಳ ವಿಷಯವೆಲ್ಲ ಚೆನ್ನಾಗಿ ಗೊತ್ತಿತ್ತು. ಆದರೆ ಅಷ್ಟು ದೂರದ ನಾಡಿನಿಂದ ಬಂದ ಈ ಯುವಸಂನ್ಯಾಸಿ–ಅವರು ತಮ್ಮ ಶಾಸ್ತ್ರಗಳಲ್ಲಿ ಎಷ್ಟೇ ಪರಿಣತರಿರಬಹುದು–ಇಲ್ಲಿನ ಈ ಪಂಡಿತರೊಂದಿಗೆ ವಾದ ಮಾಡಿ ಗೆಲ್ಲಲು ಹೇಗೆ ಸಾಧ್ಯ? ಆದರೆ ಫಲಿತಾಂಶ ಮಾತ್ರ ಅತ್ಯಾಶ್ಚರ್ಯಕರವಾಗಿತ್ತು.

“ಪಾದ್ರಿಗಳು ತಮ್ಮ ವಾದದ ಸಮರ್ಥನೆಗಾಗಿ ಬಳಸಿಕೊಂಡ ಬೈಬಲಿನ ಮಾತುಗಳಿಗೆ ಪ್ರತಿ ಯಾಗಿ ಸ್ವಾಮೀಜಿ, ಅದೇ ಬೈಬಲಿನ ತದ್ವಿರುದ್ಧಾರ್ಥದ ಸಾಲುಗಳನ್ನು ಉದ್ಧರಿಸಿದರು! ಅಲ್ಲದೆ ತಮ್ಮ ವಾದಕ್ಕೆ ಬೆಂಬಲವಾಗಿ ಅವರು ಧಾರ್ಮಿಕ ವಿಷಯಗಳ ಬಗ್ಗೆ ಬರೆದಿರುವ ಆಂಗ್ಲ ತತ್ತ್ವ ಶಾಸ್ತ್ರಜ್ಞರನ್ನೂ ಕವಿಗಳನ್ನೂ ಉದ್ಧರಿಸಿದರು. ಆಂಗ್ಲ ಕವಿಗಳ ಕೃತಿಗಳೆಲ್ಲವನ್ನೂ ಅವರು ಆಮೂಲಾಗ್ರವಾಗಿ ತಿಳಿದಂತಿತ್ತು. ಆದರೆ ನನ್ನ ಬೆಂಬಲವು ನಮ್ಮವರ ಪರವಾಗಿ ಇರಲಿಲ್ಲವೇಕೆ? ಸ್ವಾಮೀಜಿಯ ಅತ್ಯಂತ ಉದಾರ ಧಾರ್ಮಿಕ ವಿಚಾರಗಳನ್ನು ಕೇಳಿ, ನನಗೆ ಒಂದು ಬಗೆಯ ಸ್ವಚ್ಛಂದ ವಾತಾವರಣದಲ್ಲಿ ವಿಹರಿಸಿದಂತಾಯಿತಲ್ಲ! ನನ್ನ ಸುಪ್ತ ಭಾವನೆಗಳೇ ಅವರ ಮಾತು ಗಳಲ್ಲಿ ಪ್ರತಿಧ್ವನಿತವಾದದ್ದರಿಂದ ಇರಬಹುದೆ? ಅಥವಾ ಅವರ ಐಂದ್ರಜಾಲಿಕ ವ್ಯಕ್ತಿತ್ವದ ಪ್ರಭಾವವಿರಬಹುದೆ? ನನಗೆ ಗೊತ್ತಿಲ್ಲ. ಆದರೆ ಅವರಿಗೆ ಸಂದ ವಿಜಯವು ನನ್ನದೂ ಹೌದೆಂದು ನನಗನ್ನಿಸಿತು. ಅಲ್ಲದೆ ಅಂದಿನ ಆ ವಿಜಯದ ಆನಂದವು ಇಂದಿಗೂ ನನ್ನೆದೆಯಲ್ಲಿ ಹಸಿರಾಗಿದೆ.”

ಏಪ್ರಿಲ್ ೧೪ರ ಸಂಜೆ ಸ್ವಾಮೀಜಿ ನಾರ್ಥಾಂಟನ್​ನ ಪುರಭವನದಲ್ಲಿ ಮಾತನಾಡಿದರು. ಅಂದಿನ ಭಾಷಣದಲ್ಲಿ ಅವರು, ಜಗತ್ತಿನ ವಿವಿಧ ಜನಾಂಗಗಳು ಮೈಬಣ್ಣದಲ್ಲಿ, ಧರ್ಮ-ಸಂಸ್ಕೃತಿ- ಭಾಷೆಗಳಲ್ಲಿ ಸ್ವಲ್ಪಸ್ವಲ್ಪ ವಿಭಿನ್ನವಾಗಿರಬಹುದಾದರೂ ವಾಸ್ತವಿಕವಾಗಿ ಅವೆಲ್ಲವೂ ಪರಸ್ಪರ ಹತ್ತಿರದ ಬಂಧುಗಳೇ ಎಂಬುದನ್ನು ಸಾಬೀತು ಮಾಡಿದರು. ಬಳಿಕ ಅವರು ಹಿಂದೂಗಳ ಹಾಗೂ ಐರೋಪ್ಯರ ಸಂಪ್ರದಾಯಗಳನ್ನು ಹೋಲಿಸಿ ಮಾತನಾಡಿದರು. ಅನಂತರ ಅವರು ‘ಅತಿ ವಿಲಾಸದ ಬೆನ್ನಟ್ಟಿರುವ ಪಾಶ್ಚಾತ್ಯರ ರಾಷ್ಟ್ರೀಯ ದುರ್ಗಣ’ವನ್ನು ಪ್ರಸ್ತಾಪಿಸಿ ಆ ಬಗ್ಗೆ ಪಾಶ್ಚಾತ್ಯ ರಿಗೆ ಛೀಮಾರಿ ಹಾಕಿದರು. ಮರುದಿನ ಆ ಊರಿನ ‘ನಾರ್ಥಾಂಟನ್ ಡೈಲಿ ಹೆರಾಲ್ಡ್​’ ಎಂಬ ಪತ್ರಿಕೆ, ಸ್ವಾಮೀಜಿಯ ಭಾಷಣದ ಹಲವಾರು ಅಂಶಗಳನ್ನು ಪ್ರಶಂಸಿಸಿತು. ಮುಖ್ಯವಾಗಿ, ಯೂರೋಪ್ ಹಾಗೂ ಅಮೆರಿಕದ ಶ್ರೀಮಂತ ವರ್ಗಗಗಳ ‘ಡಾಲರ್ ಜಾತಿ’ಯ ವಿನಾಶಕಾರೀ ಬುದ್ಧಿಯ ಬಗ್ಗೆ ಸ್ವಾಮೀಜಿಯ ಟೀಕೆ ತುಂಬ ಸಮರ್ಪಕವಾಗಿದೆಯೆಂದು ಆ ಪತ್ರಿಕೆ ಹೇಳಿತು. ಅದೇ ಪತ್ರಿಕೆಯ ಸಂಪಾದಕೀಯ ಒಂದು ಭಾಗ ಹೀಗಿತ್ತು:

“ಸಾವಿರಾರು ವರ್ಷಗಳ ಇತಿಹಾಸವಿರುವ ಹಿಂದೂ ಜನಾಂಗದ ಬೌದ್ಧಿಕ, ನೈತಿಕ ಹಾಗೂ ಆಧ್ಯಾತ್ಮಿಕ ಸಂಸ್ಕೃತಿಯ ಫಲವಾದ ಪ್ರಖರ ಜ್ಯೋತಿಯನ್ನು ಕಾಣಲು ಇಚ್ಛಿಸುವ, ಬುದ್ಧಿವಂತನೂ ಸಮದೃಷ್ಟಿಯವನೂ ಆದ ಯಾವ ಅಮೆರಿಕನ್ನನೂ, ವಿವೇಕಾನಂದರನ್ನು ನೋಡುವ ಹಾಗೂ ಅವರ ಮಾತುಗಳನ್ನು ಕೇಳುವ ಅವಕಾಶವನ್ನು ಕಳೆದುಕೊಳ್ಳಲಾಗದು. ಅವರದ್ದಕ್ಕೆ ಹೋಲಿಸಿದರೆ ನಮ್ಮ ಇತಿಹಾಸವು ತೀರ ಇತ್ತೀಚಿನದು ಎಂಬುದನ್ನು ನಾವು ನೆನಪಿಟ್ಟುಕೊಳ್ಳಬೇಕು...”

ನಾರ್ಥಾಂಟನ್​ನಿಂದ ಸ್ವಾಮೀಜಿ ಲಿನ್ ಎಂಬ ಕೈಗಾರಿಕಾ ಪಟ್ಟಣಕ್ಕೆ ಬಂದರು. ಇಲ್ಲಿ ಅವರು ಶ್ರೀಮತಿ ಫ್ರಾನ್ಸಿಸ್ ಬ್ರೀಡ್ ಎಂಬವಳ ಅತಿಥಿಯಾಗಿದ್ದರು. ಈಕೆ ಈ ಊರಿನ ‘ನಾರ್ತ್ ಶೋರ್ ಕ್ಲಬ್​’ ಎಂಬುದರ ಅಧ್ಯಕ್ಷಿಣಿಯಾಗಿದ್ದಳು. ಈ ಕ್ಲಬ್​ನ ಸದಸ್ಯರಿಗಾಗಿ ಸ್ವಾಮೀಜಿ ‘ಭಾರತೀಯ ಸಂಪ್ರದಾಯಗಳು ಹಾಗೂ ರೀತಿನೀತಿಗಳು’ ಎಂಬ ವಿಷಯವಾಗಿ ಮಾತನಾಡಿದರು. ಎಂದಿನಂತೆ ಅವರು ಭಾರತದ ಬಗ್ಗೆ ವಿಚಿತ್ರವಾದ ಪ್ರಶ್ನೆಗಳ ಸುರಿಮಳೆಯನ್ನೇ ಎದುರಿಸಬೇಕಾಯಿತು– “ಭಾರತದಲ್ಲಿ ತಾಯಂದಿರೇ ತಮ್ಮ ನವಜಾತ ಶಿಶುಗಳನ್ನು ಮೊಸಳೆಯ ಬಾಯಿಗೆ ಕೊಡುತ್ತಾ ರಂತೆ, ನಿಜವೆ?” “ಜಗನ್ನಾಥನ ರಥದ ಅಡಿಗೆ ಬಿದ್ದು ನೂರಾರು ಜನ ಸಾಯುತ್ತಾರಂತಲ್ಲ ಸ್ವಾಮೀಜಿ!” ಇಂತಹ ಪ್ರಶ್ನೆಗಳಿಗೆ ಸ್ವಾಮೀಜಿ ಕೆಲವೊಮ್ಮೆ ಹಾಸ್ಯವಾಗಿ, ಮತ್ತೆ ಕೆಲವೊಮ್ಮೆ ವ್ಯಂಗ್ಯವಾಗಿ ಉತ್ತರಿಸುತ್ತಿದ್ದರು. “ಭಾರತದಲ್ಲಿ ವಿಧವೆಯನ್ನು ಜೀವಸಹಿತ ಸುಡುತ್ತಾರಂತೆ, ನಿಜವೆ?” ಎಂಬ ಪ್ರಶ್ನೆಗೆ ಉತ್ತರವಾಗಿ ಅವರು ‘ಸತಿ’ ಪದ್ಧತಿಯ ಆರಂಭ ಹಾಗೂ ಬೆಳವಣಿಗೆಯ ಬಗ್ಗೆ ತಿಳಿಸುತ್ತಿದ್ದರು. ಆದರೆ ಕೆಲವೊಮ್ಮೆ “ನಾವು ಎಂದೂ ಮಾಟಗಾತಿಯರನ್ನು ಸುಡಲಿಲ್ಲ” ಎಂದು ತೀಕ್ಷ್ಣವಾಗಿ ಪ್ರತಿಕ್ರಿಯಿಸುತ್ತಿದ್ದರು. (ಮಾಟಗಾತಿಯರನ್ನು–ಅಥವಾ ಹಾಗೆಂದು ಆಪಾದಿಸಲ್ಪಟ್ಟವರನ್ನು–ಜೀವಸಹಿತವಾಗಿ ಸುಡುವುದು ಕ್ರೈಸ್ತರ ಹಳೆಯ ‘ಸಂಪ್ರ ದಾಯ’.)ಹೀಗೆ ಸ್ವಾಮೀಜಿ ಸಭಿಕರಿಗೆ ಅವರ ಪ್ರಶ್ನೆಯ ಹಿಂದೆ ಅಡಕವಾಗಿದ್ದ ಮೌಢ್ಯವನ್ನು ಎತ್ತಿತೋರಿಸುತ್ತಿದ್ದರು.

ಲಿನ್​ನಿಂದ ಕೇವಲ ಹತ್ತು ಮೈಲಿ ದೂರದ ಬಾಸ್ಟನ್ ನಗರಕ್ಕೆ ಸ್ವಾಮೀಜಿ ಮತ್ತೆ ಭೇಟಿ ನೀಡಿದರು. ತಮ್ಮ ಆತ್ಮೀಯ-ಗೌರವಾನ್ವಿತ ಸ್ನೇಹಿತರಾದ ಪ್ರೊ ॥ ಜಾನ್ ಹೆನ್ರಿ ರೈಟರನ್ನು ಕಂಡು ಮನಸಾರೆ ಮಾತನಾಡಿದರು. ಅಲ್ಲದೆ ಇನ್ನೂ ಅನೇಕ ಹೊಸ ಸ್ನೇಹಿತರನ್ನು ಸಂಪಾದಿಸಿಕೊಂಡರು. ಬಾಸ್ಟನ್ನಿನ ಸಮಾಜ ಹಾಗೂ ಪತ್ರಿಕೆಗಳು ಅವರನ್ನು ಆದರದಿಂದ ಸ್ವಾಗತಿಸಿದುವು. ಇಲ್ಲಿಯೂ ಅನೇಕ ಭಾಷಣಗಳನ್ನು ನೀಡುವಂತೆ ಅವರಿಗೆ ಆಹ್ವಾನಗಳು ಬರುತ್ತಿದ್ದುವು. ಆದರೆ ಈ ಭಾಷಣ ಕೊಡುವ ಕೆಲಸದ ಬಗ್ಗೆ ಮತ್ತೆ ಅವರಿಗೆ ಜುಗುಪ್ಸೆಯುಂಟಾಯಿತು. ಜೊತೆಜೊತೆಗೇ, ಅಮೆರಿಕದ ಜನರಲ್ಲಿ ಆಧ್ಯಾತ್ಮಿಕ ಜಾಗೃತಿಯನ್ನುಂಟುಮಾಡಬೇಕೆಂಬ ಬಲವಾದ ಇಚ್ಛೆಯುಂಟಾಗಿ, ಅವರ ಜುಗುಪ್ಸೆಯ ಭಾವ ಕೊಚ್ಚಿಹೋಯಿತು. ಹೇಲ್ ಸಹೋದರಿಯರಲ್ಲೊಬ್ಬಳಾದ ಇಸಾಬೆಲ್ ಮೆಕ್ ಕಿಂಡ್ಲಿಗೆ ಅವರು ಸಾಂದರ್ಭಿಕವಾಗಿ ಬರೆದಿದ್ದರು: “ನನಗೆ ಈ ಬಾಸ್ಟನ್ನಿನಲ್ಲಿ ಹಣ ಸಂಪಾದನೆಯಾದೀತೆಂಬ ನಿರೀಕ್ಷೆಯೇನಿಲ್ಲ. ಆದರೂ, ನನ್ನಿಂದ ಸಾಧ್ಯವಾಗುವುದಾದರೆ, ನಾನು ಈ ಅಮೆರಿಕನ್ನರ ಮೆದುಳನ್ನು ಮುಟ್ಟಿ, ಅಲ್ಲೊಂದು ಉತ್ಕ್ರಾಂತಿಯನ್ನು ಮಾಡಬೇಕೆಂದಿದ್ದೇನೆ.”

ಆದರೂ ಬಾಸ್ಟನ್ನಿಗೆ ಹಿಂದಿರುಗುವ ಮುನ್ನ ಅವರು ಅಮೆರಿಕದ ಅತಿ ಮುಖ್ಯ ನಗರವಾದ ನ್ಯೂಯಾರ್ಕಿನಲ್ಲಿ ಅನೇಕ ಸಾರ್ವಜನಿಕ ಭಾಷಣಗಳನ್ನು ಮಾಡಿದರು. ಈ ದಿನಗಳಲ್ಲೇ ಅವರಿಗೆ ಅನೇಕ ಹೊಸ ವ್ಯಕ್ತಿಗಳ ಪರಿಚಯವಾಯಿತು. ಮುಂದೆ ನ್ಯೂಯಾರ್ಕ್ ನಗರದಲ್ಲಿ ‘ವೇದಾಂತ ಸೊಸೈಟಿ’ಯನ್ನು ಸ್ಥಾಪಿಸುವಲ್ಲಿ ನೆರವಾದ ಮಿಸ್ ಮೇರಿ ಫಿಲಿಪ್ಸ್, ಶ್ರೀಮತಿ ಅರ್ಥರ್ ಸ್ಮಿತ್, ಪ್ರಸಿದ್ಧ ಸಂಗೀತಗಾರ್ತಿ ಮಿಸ್ ಎಮ್ಮಾಥರ್ಸ್​ಬಿ ಮೊದಲಾದವರು ಈ ಸಂದರ್ಭದಲ್ಲೇ ಅವರ ಸಂಪರ್ಕಕ್ಕೆ ಬಂದರು.

ಮೇ ೭ ರಂದು ಬಾಸ್ಟನ್ನಿಗೆ ಮರಳಿದ ಸ್ವಾಮೀಜಿ, ಜಗತ್ಪ್ರಸಿದ್ಧ ಹಾರ್ವರ್ಡ್ ವಿಶ್ವ ವಿದ್ಯಾಲಯದ ಆಶ್ರಯದಲ್ಲಿ ಮಾತನಾಡಿದರು. ಅವರು ಭಾರತದಲ್ಲಿನ ತಮ್ಮ ಉದ್ದೇಶಿತ ಕಾರ್ಯಕ್ಕಾಗಿ ಬೆವರಿಳಿಸಿ ದುಡಿಯುತ್ತಿದ್ದರಾದರೂ, ತಮ್ಮಲ್ಲಿಗೇ ‘ದೇಹಿ’ ಎಂದು ಬಂದವರಿಗೆ ನೆರವಾಗದಿರುತ್ತಿರಲಿಲ್ಲ. ಹೀಗೆ ನೆರವನ್ನು ಬೇಡಿ, ಅಲ್ಲಿನ ನರ್ಸರಿ ಶಾಲೆಯ ಮೇಲ್ವಿಚಾರಕರು ಅವರಲ್ಲಿಗೆ ಬಂದಾಗ ಸ್ವಾಮೀಜಿ ಅದಕ್ಕೊಪ್ಪಿ, ಎರಡು ಸಹಾಯಾರ್ಥ ಉಪನ್ಯಾಸಗಳನ್ನು ನೀಡಿದರು.

ಹೀಗೆ ಅವರು ತಮ್ಮ ಕಾರ್ಯೋದ್ದೇಶದ ಸಾಧನೆಗಾಗಿ ೧೮೯೩ರ ಸೆಪ್ಟೆಂಬರ್​ನಿಂದ ಮರು ವರ್ಷದ ಮೇ ತಿಂಗಳವರೆಗೆ ಅವಿರತವಾಗಿ ದುಡಿದರು. ಮೇ ತಿಂಗಳೊಂದಿಗೆ ಅಮೆರಿಕದಲ್ಲಿ ಸೆಕೆಗಾಲ ಪ್ರಾರಂಭ. ಉರಿಯುವ ಧಗೆಯನ್ನು ತಾಳಲಾರದೆ ನಗರವಾಸಿಗಳಲ್ಲಿ ಬಹುತೇಕ ಜನ ಹಳ್ಳಿಗಳ ತೋಟದ ಮನೆಗಳಿಗೆ ಇಲ್ಲವೆ ಸಮುದ್ರ ತೀರಕ್ಕೆ ಹೊರಟುಬಿಡುತ್ತಾರೆ. ಉಪನ್ಯಾಸವೇ ಮೊದಲಾದ ಕಾರ್ಯಕ್ರಮಗಳಿಗೂ ಆಗ ಬಿಡುವು. ಆದ್ದರಿಂದ ಸ್ವಾಮೀಜಿ ತಮ್ಮ ಭಾಷಣ ಪ್ರವಾಸವನ್ನು ತತ್ಕಾಲಕ್ಕೆ ನಿಲ್ಲಿಸಬೇಕಾಯಿತು. ಅವರು ಶಿಕಾಗೋಗೆ ಹೋಗಿ, ತಮ್ಮ ನೆಚ್ಚಿನ ಹೇಲ್ ಕುಟುಂಬದ ಮನೆಯಲ್ಲಿ ಜೂನ್ ೨೮ರವರೆಗೆ ಉಳಿದುಕೊಂಡರು.

