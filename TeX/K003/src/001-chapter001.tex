
\chapter{ರಜಪುತಾನದ ರಾಜರೊಂದಿಗೆ}

\noindent

ದೆಹಲಿಯಲ್ಲಿ ತಮ್ಮ ಗುರಭಾಯಿಗಳ ಸ್ನೇಹದ ಸುವರ್ಣಶೃಂಖಲೆಯಿಂದ ಬಿಡಿಸಿಕೊಂಡು ವನರಾಜನಂತೆ ಏಕಾಕಿಯಾಗಿ ಹೊರಟ ಸ್ವಾಮೀಜಿ, ಐತಿಹಾಸಿಕ ಭೂಮಿಯಾದ ಸುಂದರ ರಜಪುತಾನದ ಅಲ್ವರಿನತ್ತ ನಡೆದರು.

ರಜಪುತಾನ! ಅದೆಂಥ ಹೆಸರು!ಆ ಹೆಸರನ್ನು ಕೇಳಿದರೇ ಸಾಕು, ವೀರ ಸೇನಾನಿಗಳ ಹಾಗೂ ಅಸಾಮಾನ್ಯ ಧೀರರ ಅಸಂಖ್ಯಾತ ನೆನಪುಗಳು ಕಣ್ಮುಂದೆ ಸುಳಿಯತೊಡಗಿ ಭಾರತೀಯನ ಹೃದಯವು ಹೆಮ್ಮೆಯಿಂದ ತುಡಿಯುವಂತೆ ಮಾಡುತ್ತವೆ. ರಜಪುತಾನದಲ್ಲಿ ಭಾರತದ ಚರಿತ್ರೆ ಯೆಲ್ಲ ಘನೀಭೂತವಾದಂತಿದೆ. ಇಲ್ಲಿ ಸಾಮ್ರಾಜ್ಯಗಳ ಮೇಲೆ ಸಾಮ್ರಾಜ್ಯಗಳನ್ನು ಸ್ಥಾಪಿಸಿ ಆಳಿದ ವೀರ ರಜಪೂತರು ಎಂಥವರೆಂದರೆ, ಅವರು ಅಕ್ಬರನನ್ನೇ ಪರಾಭವಗೊಳಿಸಿದವರು! ಇಲ್ಲಿನ ಪ್ರತಿಯೊಬ್ಬ ಮಹಿಳೆಯೂ ರಾಣಿಯಂತೆ ಮೆರೆದವಳು. ಇಲ್ಲಿ ಅಲ್ವರ್ ಎಂಬುದು ರಜಪುತಾನಕ್ಕೆ ರತ್ನಪ್ರಾಯವಾದ ಸ್ಥಳ. ಸುದೂರ ಪಶ್ಚಿಮದಲ್ಲಿ ಉನ್ನತ ಶಿಖರಗಳ ಹಿನ್ನೆಲೆಯಿರುವ ಹಾಗೂ ನಾಲ್ದೆಸೆಗಳಲ್ಲೂ ಬೆಟ್ಟಗಳಿಂದ ಸುತ್ತುವರಿದ ಈ ಸ್ಥಳ ನಯನಮನೋಹರವಾಗಿದೆ. ಈ ಪ್ರದೇಶ ಅಮೃತಶಿಲೆಯಿಂದ ಕೂಡಿದುದು. ಇಲ್ಲಿನ ಭೂಮಿ ಫಲವತ್ತಾದದ್ದು. ಇಲ್ಲಿನ ಪ್ರಕೃತಿ ಸೌಂದರ್ಯ ಹೃನ್ಮನಗಳಿಗೆ ತಂಪನ್ನುಂಟುಮಾಡುವಂಥದು.

೧೮೯೧ನೇ ಫೆಬ್ರುವರಿಯ ಒಂದು ದಿನ ಬೆಳಿಗ್ಗೆ ಅಲ್ವರ್ ರೈಲು ನಿಲ್ದಾಣದಲ್ಲಿ ಬಂದಿಳಿದರು ಸ್ವಾಮೀಜಿ. ಇಕ್ಕೆಲಗಳಲ್ಲೂ ತೋಟಗಳು, ಉದ್ಯಾನಗಳು, ಹೊಲಗದ್ದೆಗಳು ಹಾಗೂ ಸುಂದರ ವಾದ ಸಾಲುಮನೆಗಳಿಂದ ಕೂಡಿದ ರಸ್ತೆಯಲ್ಲಿ ನಡೆದು ಕೊನೆಗೆ ಸರ್ಕಾರಿ ದವಾಖಾನೆಯ ಬಳಿಗೆ ಬಂದು ತಲುಪಿದರು. ಗುರುಚರಣ ಲಸ್ಕರ್ ಎಂಬ ಬಂಗಾಳೀ ಮಹಾಶಯನೊಬ್ಬ ಅಲ್ಲಿ ನಿಂತಿದ್ದ. ಅವನು ಆ ದವಾಖಾನೆಯ ಕಾರ್ಯನಿರ್ವಾಹಕ ಡಾಕ್ಟರು. ಸ್ವಾಮೀಜಿ ಅವನನ್ನು ಬಂಗಾಳಿಯಲ್ಲೇ ಕೇಳಿದರು, “ಇಲ್ಲಿ ಸಂನ್ಯಾಸಿಗಳು ಇಳಿದುಕೊಳ್ಳುವಂತಹ ಸ್ಥಳವೇನಾದರೂ ಇದೆಯೆ?” ಅವರ ಭವ್ಯ ನಿಲುವಿನಿಂದ ಪ್ರಭಾವಿತನಾದ ಡಾಕ್ಟರು ಮೊದಲು ಅವರಿಗೆ ಪ್ರಣಾಮ ಮಾಡಿದ. ಬಳಿಕ ಸಂತೋಷದಿಂದ ಅವರನ್ನು ಬಜಾರಿಗೆ ಕರೆದುಕೊಂಡು ಹೋಗಿ ಅಲ್ಲಿ ಅಂಗಡಿ ಯೊಂದರ ಮಹಡಿಯ ಮೇಲಿದ್ದ ಕೋಣೆಯೊಂದನ್ನು ತೋರಿಸಿ ಹೇಳಿದ, “ಸ್ವಾಮೀಜಿ, ಇದು ಸಂನ್ಯಾಸಿಗಳಿಗಾಗಿಯೇ ಇರುವಂಥದು. ಸದ್ಯಕ್ಕೆ ನೀವು ಇಲ್ಲಿರಬಹುದಲ್ಲ?” “ಸಂತೋಷವಾಗಿ!” ಎಂದರು ಸ್ವಾಮೀಜಿ. ಆ ಡಾಕ್ಟರು ಅವರ ತತ್ಕಾಲದ ಆವಶ್ಯಕತೆಗಳನ್ನೆಲ್ಲ ಪೂರೈಸಿಕೊಟ್ಟು ತನ್ನೊಬ್ಬ ಮುಸಲ್ಮಾನ ಸ್ನೇಹಿತನ ಮನೆಯ ಕಡೆಗೆ ಓಡಿದ. ಅವನ ಈ ಸ್ನೇಹಿತ ಪ್ರೌಢಶಾಲೆ ಯೊಂದರಲ್ಲಿ ಉರ್ದು ಹಾಗೂ ಪರ್ಶಿಯನ್ ಭಾಷೆಗಳ ಅಧ್ಯಾಪಕ. ಡಾಕ್ಟರು ಅವನ ಬಳಿಗೆ ಹೋಗಿ ಹೇಳಿದ: “ಓ ಮೌಲ್ವೀ ಸಾಹೇಬ್, ಈಗತಾನೆ ಒಬ್ಬರು ಬಂಗಾಳಿ ದರ್ವಿಶ್ (ಪರಿವ್ರಾಜಕ ಸಂನ್ಯಾಸಿ) ಬಂದಿದ್ದಾರೆ! ನೀವು ಈಗಲೇ ಬಂದು ಅವರನ್ನು ನೋಡಿ! ನಾನು ಅಂಥ ಮಹಾತ್ಮ ರನ್ನು ಹಿಂದೆಂದೂ ನೋಡಿರಲಿಲ್ಲ. ನೀವು ಅವರೊಂದಿಗೆ ಸ್ವಲ್ಪ ಮಾತನಾಡುತ್ತಿರಿ. ಅಷ್ಟರಲ್ಲಿ ನಾನು ನನ್ನ ಕೆಲಸಕಾರ್ಯ ಮುಗಿಸಿ ಬಂದು ನಿಮ್ಮನ್ನು ಸೇರಿಕೊಳ್ಳುತ್ತೇನೆ.” ಬಳಿಕ ಡಾಕ್ಟರು ಮೌಲ್ವೀಸಾಹೇಬನನ್ನು ಸ್ವಾಮೀಜಿಯ ಬಳಿಗೆ ಕರೆದೊಯ್ದ. ಇಬ್ಬರೂ ತಮ್ಮ ಪಾದರಕ್ಷೆಗಳನ್ನು ಹೊರಗೇ ಬಿಟ್ಟು ಕೋಣೆಯೊಳಗೆ ಹೋಗಿ ಸ್ವಾಮೀಜಿಗೆ ಭಕ್ತಿಯಿಂದ ನಮಸ್ಕರಿಸಿದರು. ಡಾಕ್ಟರು ಮೌಲ್ವಿಯನ್ನು ಸ್ವಾಮೀಜಿಗೆ ಪರಿಚಯಿಸಿಕೊಟ್ಟು ತನ್ನ ಕೆಲಸದ ಕಡೆಗೆ ಹೊರಟ. ಆಗ ಸ್ವಾಮೀಜಿಯ ಬಳಿಯಿದ್ದ ಸಾಮಾನುಗಳೆಂದರೆ ಬಟ್ಟೆಯೊಂದರಲ್ಲಿ ಗಂಟುಕಟ್ಟಿದ್ದ ಕೆಲವು ಪುಸ್ತಕ ಗಳು, ಕಾವಿಬಟ್ಟೆಗಳು, ದಂಡ-ಕಮಂಡಲು.

ಸ್ವಾಮೀಜಿ ಮೌಲ್ವೀಸಾಹೇಬನನ್ನು ಸನಿಹಕ್ಕೆ ಕರೆದು ಕುಳ್ಳಿರಿಸಿಕೊಂಡು ತುಂಬ ವಿಶ್ವಾಸದಿಂದ ಧಾರ್ಮಿಕ ವಿಷಯಗಳ ಬಗ್ಗೆ ಮಾತನಾಡಿದರು. ಮಾತಿನ ಸಂದರ್ಭದಲ್ಲಿ ಮುಸಲ್ಮಾನರ ಪವಿತ್ರ ಗ್ರಂಥವಾದ ಕುರಾನಿನ ಬಗ್ಗೆ ಹೇಳುತ್ತಾರೆ, “ಕುರಾನಿನ ವಿಷಯದಲ್ಲಿ ಒಂದು ಗಮನಾರ್ಹ ಅಂಶವೇನೆಂದರೆ ಅದು ಸಾವಿರದ ನೂರು ವರ್ಷಗಳ ಹಿಂದೆ ಅಸ್ತಿತ್ವಕ್ಕೆ ಬಂದಾಗ ಹೇಗಿತ್ತೋ ಇಂದಿಗೂ ಹಾಗೆಯೇ ಇದೆ. ಅದು ತನ್ನ ಪಾಠ ಪರಿಶುದ್ಧತೆಯನ್ನು ಉಳಿಸಿಕೊಂಡಿದೆ. ಅದರಲ್ಲಿ ಯಾರೂ ಹಸ್ತಕ್ಷೇಪ ಮಾಡಿಲ್ಲ.” ಹಿಂದೂ ಸಂನ್ಯಾಸಿಯೊಬ್ಬರು ತಮ್ಮ ಮತಗ್ರಂಥವಾದ ಕುರಾನಿನ ಮಹತ್ವದ ಬಗ್ಗೆ ಹೇಳುವುದನ್ನು ಕೇಳಿದಾಗ ಆ ಮೌಲ್ವಿಗೆ ಎಷ್ಟು ಆನಂದವಾಗಿರಲಿಕ್ಕಿಲ್ಲ!

ಡಾಕ್ಟರ್ ಗುರುಚರಣ ದವಾಖಾನೆಯಿಂದ ಹಿಂದಿರುಗಿ ಬರುವಾಗ ಎಲ್ಲರಿಗೂ ತಾನು ಭೇಟಿ ಯಾದ ಮಹಾಸಂನ್ಯಾಸಿಯ ಬಗ್ಗೆ ತಿಳಿಸಿದ. ಹಾಗೆಯೇ ಮೌಲ್ವಿ ಸಾಹೇಬನೂ ತನ್ನ ಮುಸಲ್ಮಾನ ಸ್ನೇಹಿತರಿಗೆಲ್ಲ ಸುದ್ದಿ ಮುಟ್ಟಿಸಿದ. ಇದರಿಂದಾಗಿ ಶೀಘ್ರದಲ್ಲೇ ಅಲ್ಲೊಂದು ಜನಜಾತ್ರೆಯೇ ನೆರೆಯುವಂತಾಯಿತು. ಸ್ವಾಮೀಜಿಯ ಮಾತುಗಳನ್ನು ಕೇಳಲು ಅವರ ಕೋಣೆಯಲ್ಲಿ ಮಾತ್ರ ವಲ್ಲದೆ ಸುತ್ತಲಿನ ಜಗುಲಿಯಲ್ಲೂ ಜನ ಕಿಕ್ಕಿರಿಯಲಾರಂಭಿಸಿದರು. ಸ್ವಾಮೀಜಿ ತಮ್ಮ ಮಾತಿನ ಮಧ್ಯೆ ಉರ್ದು ಗೀತೆಗಳನ್ನೂ ಹಿಂದೀ ಭಜನೆಗಳನ್ನೂ ಬಂಗಾಳಿ ಕೀರ್ತನೆಗಳನ್ನೂ, ವಿದ್ಯಾಪತಿ, ಚಂಡಿದಾಸ, ರಾಮಪ್ರಸಾದನೇ ಮುಂತಾದವರ ರಚನೆಗಳನ್ನೂ ಹಾಡುತ್ತಿದ್ದರು. ಅಲ್ಲದೆ ಕೆಲವೊಮ್ಮೆ ವೇದೋಪನಿಷತ್ತುಗಳಿಂದ, ಪುರಾಣಗಳಿಂದ ಹಾಗೂ ಬೈಬಲಿನಿಂದ ಕೆಲವೊಂದು ಭಾಗಗಳನ್ನು ಪಠಿಸುತ್ತಿದ್ದರು. ತಾವು ವಿವರಿಸುತ್ತಿದ್ದ ಶಾಸ್ತ್ರವಿಚಾರಗಳನ್ನು ಬುದ್ಧ, ಶಂಕರಾ ಚಾರ್ಯ, ರಾಮಾನುಜಾಚಾರ್ಯ, ಗುರುನಾನಕ್, ಚೈತನ್ಯ, ತುಲಸೀದಾಸ ಹಾಗೂ ಶ್ರೀರಾಮ ಕೃಷ್ಣರೇ ಮೊದಲಾದ ಮಹಾಪುರುಷರ ಜೀವನದ ಘಟನೆಗಳ ನಿದರ್ಶನಗಳ ಮೂಲಕ ಸ್ಪಷ್ಟಪಡಿಸುತ್ತಿದ್ದರು.

ಕೆಲವೇ ದಿನಗಳಲ್ಲಿ ಸ್ವಾಮೀಜಿಯ ಭಕ್ತರು ಹಾಗೂ ವಿಶ್ವಾಸಿಗಳ ಸಂಖ್ಯೆ ಎಷ್ಟು ಬೆಳೆಯಿ ತೆಂದರೆ ಅವರಿಗೆ ಅಲ್ವರ್ ರಾಜ್ಯದ ನಿವೃತ್ತ ಇಂಜಿನಿಯರಾದ ಪಂಡಿತ ಶಂಭುನಾಥ ಎಂಬವರ ಮನೆಯಲ್ಲಿ ವಸತಿಯ ವ್ಯವಸ್ಥೆ ಮಾಡಲಾಯಿತು. ಇಲ್ಲಿ ಅವರಿಗೆ ತಮ್ಮ ದೈನಂದಿನ ಜೀವನಕ್ರಮ ವನ್ನು ವ್ಯವಸ್ಥಿತಗೊಳಿಸಿಕೊಳ್ಳಲು ಸಾಧ್ಯವಾಯಿತು. ಇಲ್ಲಿ ಅವರು ಬ್ರಾಹ್ಮೀ ಮುಹೂರ್ತದಿಂದ ಆರಂಭಿಸಿ ಬೆಳಿಗ್ಗೆ ಒಂಬತ್ತು ಗಂಟೆಯವರೆಗೆ ಧ್ಯಾನ ಮಾಡುತ್ತಿದ್ದರು. ಧ್ಯಾನ ಮುಗಿಸಿ ಅವರು ತಮ್ಮ ಕೋಣೆಯಿಂದ ಹೊರಗೆ ಬರುವ ವೇಳೆಗಾಗಲೇ ಸುಮಾರು ಇಪ್ಪತ್ತು ಮೂವತ್ತು ಜನ ಅವರಿಗಾಗಿ ಕಾಯುತ್ತ ನಿಂತಿರುತ್ತಿದ್ದರು. ಹೀಗೆ ಸೇರಿದ್ದವರಲ್ಲಿ ಎಲ್ಲ ವರ್ಗ, ವರ್ಣ ಹಾಗೂ ಮತಗಳ ಜನರೂ ಇರುತ್ತಿದ್ದರು. ಇವರಲ್ಲಿ ಶೈವರು, ವೈಷ್ಣವರು, ಸುನ್ನಿಗಳು, ಷಿಯಾಗಳು, ವಿದ್ಯೆ- ಅಧಿಕಾರ-ಸಂಪತ್ತು ಇರುವವರು, ಬಡ ಅವಿದ್ಯಾವಂತರು–ಹೀಗೆ ಎಲ್ಲ ಬಗೆಯವರೂ ಇರುತ್ತಿ ದ್ದರು. ಸ್ವಾಮೀಜಿ ಅವರೆಲ್ಲರೊಂದಿಗೂ ಒಂದೇ ರೀತಿಯಾಗಿ ಬೆರೆತು ಸಮನಾಗಿವ್ಯವಹರಿಸುತ್ತಿ ದ್ದರು.

ಇಲ್ಲಿ ನಾವು, ಸ್ವಾಮೀಜಿಯ ಬಳಿಗೆ, ಎಂದರೆ ಒಬ್ಬ ಹಿಂದೂ ಸಂನ್ಯಾಸಿಯ ಬಳಿಗೆ, ಎಲ್ಲ ವರ್ಗ, ವರ್ಣ ಹಾಗೂ ಮತಗಳ ಜನರು ಬರುತ್ತಿದ್ದುದನ್ನು ಗಮನಿಸಿ ನೋಡಬೇಕಾಗಿದೆ. ಆ ಕಾಲದಲ್ಲಿ ಮತಾಂಧತೆ ಸಾಕಷ್ಟು ಉತ್ಕಟಾವಸ್ಥೆಯಲ್ಲೇ ಇತ್ತು. ಹೀಗಿರುವಾಗ ಶೈವರು, ವೈಷ್ಣ ವರು, ಮುಸಲ್ಮಾನರು, ಬಡವರು-ಶ್ರೀಮಂತರು, ನಿರಕ್ಷರಸ್ಥರು-ವಿದ್ಯಾವಂತರು–ಹೀಗೆ ನಾನಾ ಬಗೆಯ ಜನರು ಜಾತಿಭೇದ, ವರ್ಣಭೇದ, ವರ್ಗಭೇದಗಳನ್ನು ಮರೆತು ಈ ಹಿಂದೂ ಸಂನ್ಯಾಸಿಯ ಬಳಿಗೆ ಬರುವಂತಾದುದು ಹೇಗೆ? ಇದನ್ನು ಅರ್ಥ ಮಾಡಿಕೊಳ್ಳಬೇಕಾದರೆ ಸ್ವಾಮೀಜಿಯ ಹಿನ್ನೆಲೆಯಲ್ಲಿ ಸಾಕ್ಷಾತ್ ಶ್ರೀರಾಮಕೃಷ್ಣರೇ ನಿಂತಿದ್ದಾರೆ ಎಂದು ತಿಳಿಯಬೇಕಾಗುತ್ತದೆ; ಅಥವಾ ಸ್ವಾಮೀಜಿಯ ವ್ಯಕ್ತಿತ್ವದಲ್ಲಿ ಶ್ರೀರಾಮಕೃಷ್ಣರೇ ಆವಿರ್ಭಾವಗೊಂಡಿದ್ದಾರೆ ಎಂದು ತಿಳಿಯ ಬೇಕಾಗುತ್ತದೆ. ಶ್ರೀರಾಮಕೃಷ್ಣರೆಂದರೆ ಯಾರು? ಸಕಲ ಮತಗಳಿಗೆ ಅನುಸಾರವಾಗಿ ಸಾಧನೆ ಮಾಡಿ ಆಯಾದೇವತೆಗಳನ್ನು ತಮ್ಮಲ್ಲಿ ಲೀನವಾಗಿಸಿಕೊಂಡವರು. ಅಂತಹ ಶ್ರೀರಾಮಕೃಷ್ಣರು, ಸ್ವಾಮಿ ವಿವೇಕಾನಂದರ ವ್ಯಕ್ತಿತ್ವದಲ್ಲಿ ಆವಿರ್ಭಾವಗೊಂಡಿರುವುದರಿಂದಲೇ ಅವರ ಬಳಿಗೆ ಜನ ಯಾವ ಮತಭೇದಗಳನ್ನೂ ಗಣಿಸದೆ ಬರುತ್ತಿದ್ದರು. ಅದಕ್ಕೆ ತಕ್ಕಂತೆ ಸ್ವಾಮೀಜಿಯೂ ಕೂಡ ಅವರೆಲ್ಲರನ್ನೂ ಸಮಾನಭಾವದಿಂದ ಆದರಿಸುತ್ತಿದ್ದರು.

ಹೀಗೆ ಸ್ವಾಮೀಜಿಯ ಭೇಟಿಗಾಗಿ ಬರುತ್ತಿದ್ದ ಜನರೆಲ್ಲ ಬಗೆಬಗೆಯ ಪ್ರಶ್ನೆಗಳನ್ನು ಕೇಳು ತ್ತಿದ್ದರು. ಸ್ವಾಮೀಜಿ ಮಧ್ಯಾಹ್ನದವರೆಗೂ ಅವರ ಪ್ರಶ್ನೆಗಳಿಗೆ ಉತ್ತರಿಸುತ್ತಿದ್ದರು. ಕೇಳುಗರು ತಮಗಿಷ್ಟಬಂದ ಯಾವ ಪ್ರಶ್ನೆಯನ್ನಾದರೂ ಕೇಳಬಹುದಾಗಿತ್ತು. ಈ ಅವಕಾಶವನ್ನು ಉಪ ಯೋಗಿಸಿಕೊಂಡು ಕೆಲವರು ಒಮ್ಮೊಮ್ಮೆ ವಿಚಿತ್ರ ಪ್ರಶ್ನೆಗಳನ್ನು ಕೇಳುತ್ತಿದ್ದುದುಂಟು. ಉದಾ ಹರಣೆಗೆ, ಅವರು ಗಹನವಾದ ತಾತ್ವಿಕ ಪ್ರಶ್ನೆಯೊಂದಕ್ಕೆ ಉತ್ತರ ನೀಡುತ್ತಿರುವಾಗ ಒಬ್ಬ ಕೇಳುತ್ತಾನೆ: “ಸ್ವಾಮೀಜಿ, ನಿಮ್ಮ ಶರೀರ ಯಾವ ಜಾತಿಗೆ ಸೇರಿದ್ದು?” ಆಗ ಅವರು ಈ ಪ್ರಾಶ್ನಿಕನ ಮನಸ್ಸಿಗೂ ನೋವನ್ನುಂಟುಮಾಡಲು ಇಚ್ಛಿಸದೆ ಕೂಡಲೇ ಉತ್ತರಿಸುತ್ತಾರೆ: “ಕಾಯಸ್ಥ”. (ಕಾಯಸ್ಥ ಎಂದರೆ ಕ್ಷತ್ರಿಯ). ಈ ಸಂದರ್ಭದಲ್ಲಿ ಬೇರೆ ಸಂನ್ಯಾಸಿಗಳಾಗಿದ್ದರೆ, ಜನರು ತಮ್ಮನ್ನು ಬ್ರಾಹ್ಮಣ ಸಂನ್ಯಾಸಿ ಎಂದು ತಿಳಿದುಕೊಳ್ಳಲೆಂದು ನೇರವಾಗಿ ಉತ್ತರಿಸದೆ ಸುತ್ತುಬಳಸು ಉತ್ತರ ಕೊಡುತ್ತಿದ್ದರೇನೋ. ಆದರೆ ಸ್ವಾಮೀಜಿ ಜಾತಿ ಕುಲ ಮಾನಗಳ ಎಲ್ಲ ಮೇರೆಗಳನ್ನೂ ಮೀರಿ ನಿಂತವರು. ಇನ್ನೊಬ್ಬ ಕೇಳುತ್ತಾನೆ: “ಸ್ವಾಮೀಜಿ, ನೀವು ಕಾವಿ ಧರಿಸಿರುವುದು ಏಕೆ?” ಇದಕ್ಕೆ ಅವರೆನ್ನುತ್ತಾರೆ: “ಏಕೆಂದರೆ ಇದು ಭಿಕ್ಷುಕರ ವೇಷ. ನಾನು ಬಿಳಿಬಟ್ಟೆ ತೊಟ್ಟುಕೊಂಡದ್ದೇ ಆದರೆ ಇತರ ಭಿಕ್ಷುಕರು ನನ್ನನ್ನು ಭಿಕ್ಷೆ ಕೇಳುತ್ತಾರೆ. ಆದರೆ ನಾನೇ ಸ್ವತಃ ಭಿಕ್ಷುಕನಾಗಿರುವುದರಿಂದ ಎಷ್ಟೋ ಸಲ ಅವರಿಗೆ ಕೊಡಲು ಒಂದು ಬಿಡಿಗಾಸೂ ನನ್ನ ಹತ್ತಿರ ಇರುವುದಿಲ್ಲ. ಭಿಕ್ಷೆ ಬೇಡಿದವರಿಗೆ ‘ಇಲ್ಲ’ ಎನ್ನಲು ನನ್ನ ಮನಸ್ಸಿಗೆ ನೋವಾಗುತ್ತದೆ. ಆದರೆ ನನ್ನ ಕಾವಿಬಟ್ಟೆಯನ್ನು ನೋಡಿ ಭಿಕ್ಷುಕರು ನಾನೂ ಒಬ್ಬ ಭಿಕ್ಷುಕನೆಂದೇ ತಿಳಿದುಕೊಳ್ಳು ತ್ತಾರೆ. ಆದ್ದರಿಂದ ಅವರು ನನ್ನ ಬಳಿ ಭಿಕ್ಷೆ ಬೇಡುವುದಿಲ್ಲ. ಭಿಕ್ಷುಕನ ಬಳಿಯೇ ಯಾರಾದರೂ ಭಿಕ್ಷೆ ಕೇಳುತ್ತಾರೆಯೇ?” ಸ್ವಾಮೀಜಿ ಕೊಟ್ಟ ಈ ಉತ್ತರ ಮನಮುಟ್ಟುವಂತಿದೆ.

ಕೆಲವೊಮ್ಮೆ ಈ ಸಂಭಾಷಣೆ ಮಾತೃಪೂಜೆಯ ಕಡೆಗೆ, ಎಂದರೆ ಜಗನ್ಮಾತೆಯ ಪೂಜೆಯ ವಿಷಯದ ಕಡೆಗೆ ತಿರುಗಿಕೊಳ್ಳುತ್ತಿತ್ತು. ಆಗ ಸ್ವಾಮೀಜಿ ಇದ್ದಕ್ಕಿದ್ದಂತೆ ಭಾವೋನ್ಮತ್ತರಾಗಿ, ಬೇರೇನೂ ಹೇಳಲಾರದವರಾಗಿ ಕೇವಲ ‘ತಾಯಿ, ತಾಯಿ!’ ಎಂದು ಉದ್ಗರಿಸುತ್ತ ಕುಳಿತು ಬಿಡುತ್ತಿದ್ದರು. ಅವರು ಹೀಗೆ ‘ತಾಯಿ ತಾಯಿ’ ಎಂದು ಹೇಳಲಾರಂಭಿಸಿದಾಗ ಮೊದಲು ಗಟ್ಟಿಯಾಗಿ ತುಂಬುಕಂಠದಿಂದ ಕೂಡಿರುತ್ತಿದ್ದ ಅವರ ದನಿ ಕ್ರಮೇಣ ಮೃದುವಾಗುತ್ತ, ಕ್ಷೀಣ ವಾಗುತ್ತ ಬಂದು ಅವರ ಆತ್ಮದೊಂದಿಗೆ ಎಲ್ಲೋ ದೂರ-ಸುದೂರ ವಲಯಕ್ಕೆ ಪಯಣಿಸುತ್ತಿರು ವಂತೆ ಭಾಸವಾಗುತ್ತಿತ್ತು. ಕ್ರಮೇಣ ಆ ದನಿಯೂ ನಿಂತುಹೋಗಿ ಅವರು ಕಣ್ಣುಚ್ಚಿ ಭಾವಪರವಶ ರಾಗಿ ಕುಳಿತುಬಿಡುತ್ತಿದ್ದರು. ಅವರ ಕಣ್ಣುಗಳಿಂದ ಧಾರಾಕಾರವಾಗಿ ಆನಂದಾಶ್ರು ಹರಿಯ ತೊಡಗುತ್ತಿತ್ತು. ಜಗನ್ಮಾತೆಯೊಂದಿಗೆ ಈ ದಿವ್ಯ ಪುತ್ರನಿಗಿದ್ದ ನಿಕಟ ಸಂಪರ್ಕವನ್ನು ಆ ದೃಶ್ಯ ಮನಗಾಣಿಸುತ್ತಿರುವಂತಿತ್ತು. ಅಲ್ಲಿ ನೆರೆದಿದ್ದ ಭಕ್ತಜನ ಸ್ವಾಮೀಜಿಯ ಭಾವಪರವಶತೆಯಲ್ಲಿ ತಾವೂ ಭಾಗಿಗಳಾಗಿ ಆನಂದಾಶ್ರು ಸುರಿಸುತ್ತ ‘ಜೈ ಮಾ ಕಾಳಿ! ಜೈ ಮಾ ಕಾಳಿ!’ ಎಂದು ಉದ್ಘೋಷಿಸುತ್ತಿದ್ದರು. ಸ್ವಲ್ಪ ಹೊತ್ತಿನ ಮೇಲೆ ಸ್ವಾಮೀಜಿ ಪ್ರಕೃತಿಸ್ಥರಾಗಿ ಹಾಡಲಾರಂಭಿಸು ತ್ತಿದ್ದರು. ಆಗ ದಿವ್ಯಪ್ರೇಮದ ಅಂತರಗಂಗೆಯ ಪ್ರವಾಹವೊಂದು ಅವರ ಹೃದಯಾಂತರಾಳ ದಿಂದ ಉಕ್ಕಿಹರಿದು ತಮ್ಮೆಲ್ಲರನ್ನೂ ಸೆಳೆದೊಯ್ಯುತ್ತಿರುವಂತೆ ಅಲ್ಲಿದ್ದವರಿಗೆ ಅನುಭವವಾಗು ತ್ತಿತ್ತು. ಮಧ್ಯಾಹ್ನ ಹಾಗೂ ಸಾಯಂಕಾಲದ ವೇಳೆಗಳಲ್ಲೂ ಭಜನೆ, ಪ್ರಾರ್ಥನೆ ಮತ್ತು ಸಂಭಾಷಣೆ ಗಳ ಕಾರ್ಯಕ್ರಮವಿರುತ್ತಿತ್ತು. ಆಧ್ಯಾತ್ಮಿಕ ವಿಚಾರಗಳೆಲ್ಲ ಯಾವಾಗಲೂ ಗಂಭೀರವಾದವು ಗಳಲ್ಲವೆ? ಇಂತಹ ವಿಚಾರಗಳನ್ನು ಬಹಳ ಹೊತ್ತು ಕೇಳುತ್ತಿದ್ದರೆ ಅಭ್ಯಾಸವಿಲ್ಲದವರಿಗೆ ಆಯಾಸವಾಗಿಬಿಡುತ್ತದೆ. ಆದ್ದರಿಂದ ಸ್ವಾಮೀಜಿ ಅಲ್ಲಿ ನೆರೆದಿದ್ದ ಜನರೊಂದಿಗೆ ಈ ಬಗೆಯ ಸಂಭಾಷಣೆ ನಡೆಸುತ್ತಿದ್ದಾಗ ಕೆಲವೊಮ್ಮೆ ಜನರ ಮನಸ್ಸನ್ನು ಸ್ವಲ್ಪ ಹಗುರಗೊಳಿಸಲು, ವಿವಿಧ ದೇಶಗಳ ಜನರ ವಿಚಿತ್ರ ರೀತಿನೀತಿಗಳನ್ನೂ ನಡೆವಳಿಕೆಗಳನ್ನೂ ಸಂಪ್ರದಾಯಗಳನ್ನೂ ತಮಾಷೆ ಯಾಗಿ ಬಣ್ಣಿಸುತ್ತಿದ್ದರು. ಅವರು ಅವುಗಳನ್ನೆಲ್ಲ ಹಾವಭಾವಗಳ ಮೂಲಕ ಬಣ್ಣಿಸಿ ಹೇಳುವು ದನ್ನು ಕೇಳಿದ ಜನ ಹೊಟ್ಟೆ ಹುಣ್ಣಾಗುವಷ್ಟು ನಗುತ್ತಿದ್ದರು.

ಇನ್ನು ಕೆಲವೊಮ್ಮೆ ಸ್ವಾಮೀಜಿ ಶ್ರೀಕೃಷ್ಣನ ಬೃಂದಾವನ ಲೀಲೆಯ ಕುರಿತಾಗಿ ಹಾಡುತ್ತಿದ್ದರು. ಹಾಡುತ್ತಿದ್ದಂತೆ ಅವರ ಕಪೋಲಗಳ ಮೇಲೆ ಅಶ್ರುಧಾರೆ ಹರಿಯುತ್ತಿತ್ತು. ಅವರ ಭಾವದಾಳವನ್ನು ಕಂಡ ಭಕ್ತಜನರು ಆಶ್ರುಧಾರೆಯ ಮೂಲಕವೇ ಪ್ರತಿಸ್ಪಂದಿಸುತ್ತಿದ್ದರು. ಅಲ್ಲಿ ನೆರೆದಿದ್ದವರಿಗೆ ಅನ್ನಿಸುತ್ತಿತ್ತು, ‘ಈಗ ಸ್ವಾಮೀಜಿಗೆ ಶ್ರೀಕೃಷ್ಣನ ದರ್ಶನವಾಗುತ್ತಿದೆ! ಅವರ ಗಾಯನ ಹೇಗೆ ನಮ್ಮೆಲ್ಲರ ಹೃದಯಗಳನ್ನು ಅಪಹರಿಸಿಬಿಡುತ್ತಿದೆ!’ ಎಂದು. ಸ್ವಾಮೀಜಿಯ ಆಧ್ಯಾತ್ಮಿಕಶಕ್ತಿ ಅವರ ಈ ಪರಿವ್ರಾಜಕ ದಿನಗಳಲ್ಲಿ ಸ್ಪಷ್ಟವಾಗಿ ವ್ಯಕ್ತವಾಗುತ್ತಿದೆ. ಜಗನ್ಮಾತೆಯ ಹೆಸರೆತ್ತಿದರೆ ಅವರು ಜಗನ್ಮಾತೆಯ ಭಾವದಲ್ಲೇ ತನ್ಮಯರಾಗಿಬಿಡುತ್ತಾರೆ. ಆಗ ಅಲ್ಲಿ ಜಗನ್ಮಾತೆ ಓಡಿ ಬರಲೇ ಬೇಕು! ಶ್ರೀಕೃಷ್ಣನ ಸಂಕೀರ್ತನೆ ಮಾಡುತ್ತಿದ್ದರೆ ಕೃಷ್ಣಭಾವದಿಂದ ರಂಜಿತರಾಗಿ ಬಿಡುತ್ತಾರೆ. ಆಗ ಅವರ ಬಳಿಗೆ ಶ್ರೀಕೃಷ್ಣ ಓಡಿಬರಲೇಬೇಕು! ಬರಬರುತ್ತ ಅವರ ಕಂಠ ಗದ್ಗದಿತವಾಗಿ, ನಿಶ್ಚಲವಾಗಿ ಕುಳಿತುಬಿಡುತ್ತಿದ್ದರು. ಒಮ್ಮೆ ಇಂಥದೊಂದು ಸಂದರ್ಭದಲ್ಲಿ ಒಬ್ಬ ಭಕ್ತ ಉದ್ಗರಿಸಿದ, “ಆಗೋ ನೋಡಿ, ಸ್ವಾಮೀಜಿಯ ಮುಖ ಕೃಷ್ಣಪ್ರೇಮದಿಂದ ಉನ್ನತ್ತಳಾದ ಗೋಪಿಯ ಮುಖದಂತೆ ಕಾಣುತ್ತಿದೆ!” ಸ್ವಾಮೀಜಿ ಬಂಗಾಳಿ ಕೀರ್ತನೆಯೊಂದನ್ನು ಹಾಡುವ ಮೊದಲು ಅದರ ಅರ್ಥವನ್ನು ವಿವರಿಸಿ ಹಾಡನ್ನು ಬರೆದುಕೊಳ್ಳುವಂತೆ ಹೇಳುತ್ತಿದ್ದರು. ಬಹು ಪಾಲು ರಾಜಸ್ತಾನಿಗಳೆಲ್ಲ ಶ್ರೀಕೃಷ್ಣಭಕ್ತರು. ಅದಕ್ಕಾಗಿ ಅವರ ಮುಂದೆ ಶ್ರೀಕೃಷ್ಣ ಸಂಕೀರ್ತನೆ.

ಹೀಗೆಯೆ ದಿನಗಳುರುಳಿದವು. ಕಾಲ ಸರಿಯುತ್ತಿರುವುದರ ಪರಿವೆಯೇ ಯಾರಿಗೂ ಇದ್ದಂತಿರ ಲಿಲ್ಲ. ಕೆಲವೊಮ್ಮೆ ಭಜನೆ-ಸಂಭಾಷಣೆಗಳು ಮಧ್ಯರಾತ್ರಿಯವರೆಗೂ ಮುಂದುವರಿಯುತ್ತಿದ್ದುವು. ಭಕ್ತರೆಲ್ಲ ಸ್ವಾಮೀಜಿಯನ್ನು ತುಂಬ ಹಚ್ಚಿಕೊಂಡುಬಿಟ್ಟಿದ್ದರು. ಅಲ್ಲದೆ, ತಾನೇ ಅವರಿಗೆ ಅತ್ಯಂತ ಮೆಚ್ಚಿನವನು ಎಂದು ಪ್ರತಿಯೊಬ್ಬನೂ ಭಾವಿಸಲಾರಂಭಿಸಿದ್ದ! ಕೆಲವು ಭಕ್ತರಿಗೆ ಸ್ವಾಮೀಜಿ ಮಂತ್ರದೀಕ್ಷೆ ನೀಡಿ ಅನುಗ್ರಹಿಸಿದರು.

ಸ್ವಾಮೀಜಿಯ ಅತ್ಯಂತ ನಿಷ್ಠಾವಂತ ಭಕ್ತರಲ್ಲಿ ಮೌಲ್ವೀ ಸಾಹೇಬನೂ ಒಬ್ಬನಾಗಿದ್ದ. ಅವರನ್ನು ತನ್ನ ಮನೆಗೆ ಆಮಂತ್ರಿಸಿ ಉಣಬಡಿಸಬೇಕು ಎಂಬುದು ಆತನ ಬಲವಾದ ಆಸೆ ಯಾಗಿತ್ತು. ಅವನು ಚಿಂತಿಸುತ್ತಿದ್ದ–‘ಸ್ವಾಮೀಜಿ ಜಾತಿಭಾವನೆಯನ್ನು ಮೀರಿದ ಮಹಾಸಾಧು ಗಳು; ಅವರೇನೋ ನನ್ನ ಆಮಂತ್ರಣವನ್ನು ಮನ್ನಿಸಿ ಬರಬಹುದು. ಆದರೆ ಅವರ ಆತಿಥೇಯ ರಾದ ಪಂಡಿತ್​ಜಿಯ ಮನೆಯವರು ಇದಕ್ಕೆ ಆಕ್ಷೇಪಿಸಬಹುದು, ಎಂದು. ಆದರೂ, ಹೇಗಾದರಾ ಗಲಿ ಎಂದು ಅವನೊಂದು ದಿನ ಸಂಜೆ ಪಂಡಿತ ಶಂಭುನಾಥಜಿಯ ಮನೆಗೆ ಹೋದ. ಅಲ್ಲಿದ್ದವ ರೆಲ್ಲರ ಮುಂದೆ ಕೈ ಜೋಡಿಸಿ ನಿಂತು ಹೇಳಿದ: “ನನ್ನದೊಂದು ವಿನಂತಿ. ದಯವಿಟ್ಟು ನಾಳೆದಿನ ನನ್ನ ಮನೆಗೆ ಸ್ವಾಮೀಜಿಯವರನ್ನು ಊಟಕ್ಕೆ ಕಳಿಸಿಕೊಡಬೇಕು... ನನ್ನ ಮನೆಯ ಬೈಠಕ್ ಖಾನೆಯ ಆಸನಗಳನ್ನೆಲ್ಲ ಬ್ರಾಹ್ಮಣರಿಂದಲೇ ತೊಳೆಸಿ ಶುದ್ಧಪಡಿಸುತ್ತೇನೆ. ಅಲ್ಲದೆ ಬ್ರಾಹ್ಮಣ ರಿಂದಲೇ, ಅವರು ತಮ್ಮ ಮನೆಗಳಿಂದಲೇ ತಂದ ಪಾತ್ರೆಗಳಲ್ಲಿ, ಅವರೇ ತಂದ ಪದಾರ್ಥಗಳಿಂದ ಸ್ವಾಮೀಜಿಗಾಗಿ ಅಡಿಗೆ ಮಾಡಿಸುತ್ತೇನೆ. ಆ ಆಹಾರವನ್ನು ಸ್ವಾಮೀಜಿ ಸ್ವೀಕರಿಸುವುದನ್ನೂ ದೂರ ದಿಂದ ಕಂಡರೂ ಸಾಕು, ಈ ಮ್ಲೇಚ್ಛನು ಧನ್ಯನಾದನೆಂದು ಭಾವಿಸುತ್ತೇನೆ.” ಮೌಲ್ವೀಸಾಹೇಬ ಈ ಮಾತುಗಳನ್ನು ಎಷ್ಟು ಪ್ರಾಮಾಣಿಕತೆಯಿಂದ ಹಾಗೂ ವಿನಯಪೂರ್ವಕವಾಗಿ ಹೇಳಿದನೆಂದರೆ ಅದನ್ನು ಕೇಳಿದವರ ಮನಸ್ಸಿನ ಮೇಲೆ ಅವು ತುಂಬ ಪ್ರಭಾವ ಬೀರಿದುವು. ಪಂಡಿತ್​ಜಿ ಸ್ನೇಹ ಭಾವದಿಂದ ಮೌಲ್ವಿಯ ಕೈ ಹಿಡಿದುಕೊಂಡು ಹೇಳಿದರು, “ಮೌಲ್ವೀಸಾಹೇಬರೇ, ಸ್ವಾಮೀಜಿ ಒಬ್ಬ ಸಂನ್ಯಾಸಿ. ಅವರಿಗೆಂಥ ಜಾತಿ! ಅವರು ಜಾತಿ-ಮತಗಳ ಕಟ್ಟುಕಟ್ಟಳೆಗಳನ್ನು ಮೀರಿನಿಂತ ವರು. ನೀವು ಅಷ್ಟೆಲ್ಲ ಆತಂಕಪಟ್ಟುಕೊಳ್ಳಬೇಕಾಗಿಲ್ಲ. ಅವರು ನಿಮ್ಮ ಮನೆಯಲ್ಲಿ ಊಟ ಮಾಡುವ ಬಗ್ಗೆ ನನ್ನ ಕಡೆಯಿಂದೇನೂ ಅಭ್ಯಂತರವಿಲ್ಲ. ನೀವು ಯಾವ ವ್ಯವಸ್ಥೆ ಮಾಡಿದರೂ ಅದು ನಮಗೆ ಒಪ್ಪಿಗೆಯೇ. ನಿಜಕ್ಕೂ, ನೀವು ಮಾಡಿಸುತ್ತಿರುವ ವ್ಯವಸ್ಥೆಯನ್ನು ನೋಡಿದರೆ ನಾನೇ ನಿಮ್ಮ ಮನೆಯಲ್ಲಿ ಊಟಮಾಡಿಬಿಡಬಲ್ಲೆ! ಇನ್ನು ಮುಕ್ತಪುರುಷರಾದ ಸ್ವಾಮೀಜಿಯ ವಿಚಾರ ವಾಗಿ ಹೇಳುವುದೇನಿದೆ?” ಅಂತೂ ಮೌಲ್ವೀಸಾಹೇಬ ಸ್ವಾಮೀಜಿಯನ್ನು ತನ್ನ ಮನೆಗೆ ಊಟಕ್ಕೆ ಆಹ್ವಾನಿಸಿ ಧನ್ಯನಾದ. ಇದರಿಂದ ಉತ್ಸಾಹಿತರಾದ ಇನ್ನೂ ಅನೇಕ ಮುಸ್ಲಿಂ ಭಕ್ತರು ಸ್ವಾಮೀಜಿ ಯನ್ನು ತಮ್ಮ ಮನೆಗಳಿಗೆ ಆಮಂತ್ರಿಸಿ ಉಣಬಡಿಸಿದರು. ನಿಜಕ್ಕೂ, ಹಿಂದೂ ಸಂನ್ಯಾಸಿ ಯೊಬ್ಬನಿಗೆ ಅಂದಿನ ಕಾಲದಲ್ಲೂ ಮುಸಲ್ಮಾನರು ತೋರಿದ ಪ್ರೀತಿ-ಗೌರವ ಅಪೂರ್ವವಾದುದೇ ಸರಿ. ಅಂತೆಯೇ ಆ ಹಿಂದೂ ಸಂನ್ಯಾಸಿಯು ಮ್ಲೇಚ್ಛರೆಂದು ಕರೆಯಲ್ಪಡುವವರ ಪ್ರೀತ್ಯಾದರ ಗಳನ್ನು ಸ್ವೀಕರಿಸಿದ್ದೂ ಅಷ್ಟೇ ಅಪೂರ್ವವಾದುದು.

ಸ್ವಾಮೀಜಿಯ ವಿಚಾರವಾಗಿ ಕೇಳಿ ತಿಳಿದ ಅಲ್ವರ್ ರಾಜ್ಯದ ದಿವಾನನಾದ ಮೇಜರ್ ರಾಮ ಚಂದ್ರಜಿ ಎಂಬವನು ಅವರನ್ನು ತನ್ನ ಮನೆಗೆ ಆಹ್ವಾನಿಸಿದ. ಸ್ವಲ್ಪ ಚೆನ್ನಾಗಿ ಸ್ವಾಮೀಜಿಯ ಪರಿಚಯವಾದ ಮೇಲೆ ಅವನಿಗನ್ನಿಸಿತು, ‘ಇಂಗ್ಲಿಷರ ರೀತಿನೀತಿಗಳಿಂದ ತುಂಬ ಪ್ರಭಾವಿತನಾಗಿ ಬಿಟ್ಟಿರುವ ನಮ್ಮ ಮಹಾರಾಜ ಮಂಗಳಸಿಂಗನ ಮನಸ್ಸನ್ನು ಸೆಳೆದು ಭಾರತೀಯತೆಯಲ್ಲಿ ನೆಲಸು ವಂತೆ ಮಾಡಲು ಈ ಸ್ವಾಮೀಜಿ ಸಮರ್ಥರಾಗಿರುವಂತೆ ಕಾಣುತ್ತಿದೆ. ಯಾಕೊಂದು ಪ್ರಯತ್ನ ಮಾಡಿ ನೋಡಬಾರದು?’ ಎಂದು. ಆದ್ದರಿಂದ ಅವನು, ಆ ಸಮಯದಲ್ಲಿ ಅಲ್ಲಿಂದ ಎರಡು- ಮೂರು ಮೈಲಿಗಳ ದೂರದಲ್ಲಿದ್ದ ಅರಮನೆಯಲ್ಲಿ ವಾಸಿಸುತ್ತಿದ್ದ ಮಹಾರಾಜನಿಗೆ ಬರೆದು ಕಳಿಸಿದ–“ಪ್ರಕಾಂಡ ಇಂಗ್ಲಿಷ್ ಪಾಂಡಿತ್ಯವಿರುವ ಮಹಾಸಾಧುಗಳೊಬ್ಬರು ಇಲ್ಲಿಗೆ ಬಂದಿ ದ್ದಾರೆ” ಎಂದು. ಆ ರಾಜ ಇಂಗ್ಲಿಷರಿಂದ ಪ್ರಭಾವಿತನಾದವನಲ್ಲವೆ? ಅಂಥವನು ಸಾಮಾನ್ಯ ಸಂನ್ಯಾಸಿಗಳ ಕಡೆಗೆಲ್ಲ ಗಮನ ಕೊಟ್ಟಾನೆಯೆ! ಆದ್ದರಿಂದ ದಿವಾನ ‘ಪ್ರಕಾಂಡ ಇಂಗ್ಲಿಷ್ ಪಾಂಡಿತ್ಯವಿರುವ ಮಹಾಸಾಧುಗಳು’ ಎಂದು ಬರೆಯುತ್ತಾನೆ. ಹೀಗೆ ಬರೆದರೆ ಮಹಾರಾಜ ಓಡಿ ಬರುತ್ತಾನೆ ಎಂಬುದು ಆತನ ಉಪಾಯ. ಅವನ ಉಪಾಯ ಫಲಿಸಿತು. ಮರುದಿನವೇ ಮಹಾರಾಜ ಈ ‘ಇಂಗ್ಲಿಷ್ ಬಲ್ಲ ಸಂನ್ಯಾಸಿ’ಯನ್ನು ನೋಡಲು ದಿವಾನನ ಮನೆಗೆ ಬಂದ.

ಸ್ವಾಮೀಜಿಯನ್ನು ಕಂಡೊಡನೆ ಮಹಾರಾಜ ಮೊದಲು ಅವರಿಗೆ ನಮಸ್ಕರಿಸಿ ಅವರನ್ನು ಆಸೀನರಾಗುವಂತೆ ಕೇಳಿಕೊಂಡ. ಹಿಂದೂ ಸಂಸ್ಕೃತಿ-ಸಂಪ್ರದಾಯದ ಪ್ರಕಾರ ರಾಜಮಹಾ ರಾಜರೂ ಸಂನ್ಯಾಸಿಗಳ ಮುಂದೆ ತಲೆ ಬಾಗಲೇಬೇಕು; ಮತ್ತು ಅವರು ಕುಳಿತ ಮೇಲೆಯೇ ತಾವು ಕುಳಿತುಕೊಳ್ಳಬೆಕು. ಈ ಮಹಾರಾಜ ಮಂಗಳಸಿಂಗ್ ಇಂಗ್ಲಿಷ್ ನಾಗರಿಕತೆಗೆ ಒಲಿದವ ನಾದರೂ ಈಗ ಸ್ವಲ್ಪ ಭಾರತೀಯವಾಗಿ ನಡೆದುಕೊಂಡ. ಅಲ್ಲದೆ ಎದುರಿಗಿರುವ ಸ್ವಾಮಿಗಳು ಇಂಗ್ಲಿಷ್ ಜ್ಞಾನವಿರುವವರಲ್ಲವೆ! ಮಹಾರಾಜ ಮಂಗಳಸಿಂಗ್ ಸಂಭಾಷಣೆ ಪ್ರಾರಂಭಿಸಿದ: “ಒಳ್ಳೆಯದು ಸ್ವಾಮೀಜಿ ಮಹಾರಾಜ್, ತಾವೊಬ್ಬರು ದೊಡ್ಡ ವಿದ್ವಾಂಸರು ಎಂದು ಕೇಳಿದ್ದೇನೆ. ತಾವು ಮನಸ್ಸು ಮಾಡಿದರೆ ಪ್ರತಿತಿಂಗಳೂ ಕೈ ತುಂಬ ಹಣ ಸಂಪಾದಿಸಬಹುದು. ಆದರೂ ತಾವು ಭಿಕ್ಷೆ ಬೇಡುತ್ತ ಅಲೆದಾಡುವುದೇಕೆ?” ಇದಕ್ಕೆ ಸ್ವಾಮೀಜಿಯ ಉತ್ತರ ಸಿದ್ಧವಾಗಿಯೇ ಇತ್ತು. ಇನ್ನೊಂದು ಪ್ರಶ್ನೆಯ ರೂಪದಲ್ಲಿ–“ಮಹಾರಾಜ, ನೀನು ಯಾವಾಗಲೂ ನಿನ್ನ ಪಾಶ್ಚಾತ್ಯ ಸ್ನೇಹಿತರ ಜೊತೆಯಲ್ಲೇ ಕಾಲ ಕಳೆಯುತ್ತ ಅವರ ಜೊತೆಯಲ್ಲಿ ಯಾವಾಗಲೂ ಬೇಟೆಯಾಡುತ್ತ ನಿನ್ನ ಪ್ರಜೆಗಳ ಕಡೆಗಿನ ಕರ್ತವ್ಯಗಳನ್ನೆಲ್ಲ ಕಡೆಗಣಿಸಿಬಿಟ್ಟಿರುವೆಯಲ್ಲ, ಏಕೆಂದು ಹೇಳ ಬಲ್ಲೆಯಾ?”

ಈ ಉತ್ತರವನ್ನು ಕೇಳಿ ಅಲ್ಲಿದ್ದವರೆಲ್ಲ ದಂಗಾದರು. ‘ಎಂಥ ಎದೆಗೆಚ್ಚು ಈ ಸಂನ್ಯಾಸಿಗೆ! ಈತ ತನ್ನ ಈ ಮಾತಿನ ಫಲವನ್ನುಣ್ಣಬೇಕಾಗುತ್ತದೆ’ ಎಂದು ಮನಸ್ಸಿನಲ್ಲೇ ಅಂದುಕೊಂಡರು. ಆದರೆ ಮಹಾರಾಜ ಮಾತ್ರ ಸ್ವಾಮೀಜಿಯ ಮಾತುಗಳನ್ನು ಶಾಂತವಾಗಿಯೇ ಸ್ವೀಕರಿಸಿದ.

“ಏಕೆಂದು ನನಗೆ ಗೊತ್ತಿಲ್ಲ. ಆದರೆ ಅದು ನನಗಿಷ್ಟ ಎಂಬುದರಲ್ಲಿ ಮಾತ್ರ ಸಂಶಯವಿಲ್ಲ” –ಸ್ವಲ್ಪ ಆಲೋಚಿಸಿ ಹೇಳಿದ.

“ನನಗೂ ಅಷ್ಟೇ. ಆ ಕಾರಣಕ್ಕಾಗಿಯೇ ನಾನು ಭಿಕ್ಷೆ ಬೇಡುತ್ತ ತಿರುಗುವುದು!” ಎಂದರು ಸ್ವಾಮೀಜಿ.

ಮಹಾರಾಜ ಇದಕ್ಕೇನು ತಾನೆ ಹೇಳಿಯಾನು!

ಈಗ ಅವನು ತನ್ನ ಮತ್ತೊಂದು ಪ್ರಶ್ನೆಯನ್ನು ಮುಂದಿಟ್ಟ–“ಸ್ವಾಮೀಜಿ ಮಹಾರಾಜ್, ನನಗೆ ಮೂರ್ತಿಪೂಜೆಯಲ್ಲಿ ನಂಬಿಕೆಯಿಲ್ಲ. ಮುಂದೆ ನನ್ನ ಗತಿ ಏನಾಗಬಹುದು?”

ಮಹಾರಾಜ ಪ್ರಶ್ನೆಯನ್ನೇನೋ ಕೇಳಿದ. ಆದರೆ ಆ ಪ್ರಶ್ನೆಯನ್ನು ಕೇಳುವಾಗ ಅವನ ಮುಖದಲ್ಲಿ ಕುಚೇಷ್ಟೆಯ ಮುಗುಳ್ನಗು ಕಾಣುತ್ತಿತ್ತು. ‘ಈ ಮೂರ್ತಿ ಪೂಜೆ-ಗೀರ್ತಿಪೂಜೆಯೆಲ್ಲ ಮೂಢನಂಬಿಕೆ; ಆದ್ದರಿಂದ ಇದನ್ನು ಮಾಡದಿದ್ದ ಮಾತ್ರಕ್ಕೆ ತನಗೇನಾದರೂ ಕೆಡುಕುಂಟಾಗುವ ಪ್ರಶ್ನೆಯೇ ಇಲ್ಲ’ ಎಂಬುದು ಅವನ ನಿಶ್ಚಿತ ಅಭಿಪ್ರಾಯ. ಆದರೆ ತಾನು ಆ ಬಗ್ಗೆ ಪ್ರಶ್ನಿಸಿದಾಗ ಸ್ವಾಮೀಜಿ ತನಗೆ ನರಕದ ಭಯ ಹುಟ್ಟಿಸಬಹುದು; ಆಗ ಅದು ಒಳ್ಳೇ ತಮಾಷೆಯಾಗಿರುತ್ತದೆ ಎಂದು ಅವನು ನಿರೀಕ್ಷಿಸುತ್ತಿದ್ದ. ಇದನ್ನು ಗಮನಿಸಿ ಸ್ವಾಮೀಜಿ ಸ್ವಲ್ಪ ಅಸಮಾಧಾನಗೊಂಡು ಹೇಳಿದರು–“ನೀನು ತಮಾಷೆ ಮಾಡುತ್ತಿದ್ದೀಯೆ.”

“ಇಲ್ಲ ಸ್ವಾಮೀಜಿ, ಖಂಡಿತ ಇಲ್ಲ. ನಿಜಕ್ಕೂ ನನಗೆ ಎಲ್ಲರೂ ಮಾಡುವಂತೆ ಮರ, ಮಣ್ಣು, ಕಲ್ಲು, ಲೋಹಗಳನ್ನು ಪೂಜಿಸಲು ಸಾಧ್ಯವಿಲ್ಲ. ಇದರಿಂದಾಗಿ ನಾನು ಮುಂದೆ ದುರ್ಗತಿಗೀಡಾಗು ತ್ತೇನೆಯೇ ಹೇಗೆ ಎಂದು ಕೇಳಿದೆ ಅಷ್ಟೆ.”

“ನೋಡು, ಪ್ರತಿಯೊಬ್ಬನೂ ತನ್ನ ತನ್ನ ನಂಬಿಕೆಗೆ ಅನುಗುಣವಾಗಿ ಧಾರ್ಮಿಕ ಆದರ್ಶಗಳನ್ನು ಅನುಸರಿಸಬೇಕು ಎಂಬುದು ನನ್ನ ಅನಿಸಿಕೆ.”

ಇದನ್ನು ಕೇಳಿ, ಅಲ್ಲಿ ನೆರೆದಿದ್ದ ಸ್ವಾಮೀಜಿಯ ಭಕ್ತರೆಲ್ಲ ತಬ್ಬಿಬ್ಬಾದರು. ಏಕೆಂದರೆ ಸ್ವಾಮೀಜಿ ಮೂರ್ತಿಪೂಜೆಯನ್ನು ಅನುಮೋದಿಸುತ್ತಾರೆಂಬ ಸಂಗತಿ ಅವರಿಗೆಲ್ಲ ತಿಳಿದಿತ್ತು. ಈಗ ನೋಡಿ ದರೆ, ಮಹಾರಾಜನ ಅಭಿಪ್ರಾಯವನ್ನೇ ಸ್ವಾಮೀಜಿ ಅನುಮೋದಿಸುವಂತೆ ಕಾಣುತ್ತಿದೆ! ಅವನು ತನ್ನ ನಂಬಿಕೆಗೆ ಅನುಗುಣವಾಗಿ ತನಗೆ ಯಾವುದು ಸರಿಕಾಣುತ್ತದೆಯೋ ಅದನ್ನು ಅನುಸರಿಸ ಬಹುದು ಎಂದಂತಾಯಿತಲ್ಲ!... ಆದರೆ ಸ್ವಾಮೀಜಿ ತಮ್ಮ ಉತ್ತರವನಿನ್ನೂ ಮುಗಿಸಿರಲಿಲ್ಲ. ಅವರ ದೃಷ್ಟಿ ಗೋಡೆಯ ಮೇಲಿದ್ದ ಮಹಾರಾಜನ ಭಾವಚಿತ್ರದತ್ತ ಹರಿಯಿತು. ಅದನ್ನು ತಮ್ಮ ಕೈಗೆ ತಂದುಕೊಡುವಂತೆ ಹೇಳಿದರು. ಬಳಿಕ ಅದನ್ನು ಹಿಡಿದುಕೊಂಡು ಕೇಳಿದರು, “ಇದು ಯಾರ ಭಾವಚಿತ್ರ?”

“ನಮ್ಮ ಮಹಾರಾಜರದು!” ದಿವಾನ ಉತ್ತರಿಸಿದ.

ಆದರೆ ಮರುಕ್ಷಣದಲ್ಲೇ ಸ್ವಾಮೀಜಿ ದಿವಾನನಿಗೆ ನೀಡಿದ ಆದೇಶದಿಂದ ಅಲ್ಲಿದ್ದವರೆಲ್ಲ ಭಯದಿಂದ ನಡುಗಿದರು: “ಉಗುಳು ಇದರ ಮೇಲೆ!”

ದಿವಾನ ಬೆಪ್ಪುಗಟ್ಟಿ ಸುಮ್ಮನೆ ನಿಂತುಬಿಟ್ಟ.

ಸ್ವಾಮೀಜಿ ಮತ್ತೆ ಗುಡುಗಿದರು–“ನಿಮ್ಮಲ್ಲಿ ಯಾರಾದರೂ ಸರಿಯೆ, ಇದರ ಮೇಲೆ ಉಗುಳ ಬಹುದು. ಇದೊಂದು ಕೇವಲ ಕಾಗದದ ಚೂರಲ್ಲವೆ? ಇದರ ಮೇಲೆ ಉಗುಳಲು ಅಭ್ಯಂತರವೇನಿದೆ?”

ದಿವಾನ ದಂಗುಬಡಿದುಹೋದ. ಸಭಿಕರೆಲ್ಲರ ಕಣ್ಣುಗಳು ಭಯಾಶ್ಚರ್ಯಗಳಿಂದ ಮಹಾರಾಜ ನಿಂದ ಸ್ವಾಮೀಜಿಯ ಕಡೆಗೆ ಸ್ವಾಮೀಜಿಯಿಂದ ಮಹಾರಾಜನ ಕಡೆಗೆ ಹರಿದಾಡಿದುವು. ಈಗ ಏನೋ ಒಂದು ಅನಾಹುತ ಸಂಭವಿಸಲಿದೆಯೆಂದು ಎಲ್ಲರ ಎದೆಯಲ್ಲೂ ಡವಡವ.

“ಉಗುಳಿ ಇದರ ಮೇಲೆ! ನಾನು ಹೇಳುತ್ತಿದ್ದೇನೆ, ಉಗುಳಿ ಇದರ ಮೇಲೆ!”

“ಇದೇನು ಸ್ವಾಮೀಜಿ ನೀವು ಹೇಳುತ್ತಿರುವುದು! ಇದರ ಮೇಲೆ ಉಗುಳುವುದೆ? ಇದು ನಮ್ಮ ಮಹಾರಾಜರ ಭಾವಚಿತ್ರ!!” ದಿವಾನ ಕೂಗಿಕೊಂಡ.

“ಇರಬಹುದು. ಆದರೆ ನಿಮ್ಮ ಮಹಾರಾಜರೇನೂ ಇದರಲ್ಲಿ ಸಶರೀರಿಯಾಗಿ ಇಲ್ಲವಲ್ಲ! ಇದೊಂದು ಕೇವಲ ಕಾಗದದ ಚೂರು, ಇದರಲ್ಲಿ ಅವರ ಮೂಳೆ ರಕ್ತಮಾಂಸ ಏನೂ ಇಲ್ಲ. ಇದು ನಿಮ್ಮ ಮಹಾರಾಜರಂತೆ ಓಡಾಡುವುದಿಲ್ಲ, ಮಾತನಾಡುವುದಿಲ್ಲ. ಆದರೂ ಇದರ ಮೇಲೆ ಉಗುಳಲು ಹಿಂಜರಿಯುತ್ತಿದ್ದೀರಲ್ಲ. ಏಕೆ?”

ಎಲ್ಲರೂ ಮೌನ! ಎಲ್ಲೆಲ್ಲೂ ಮೌನ! ಸ್ವಾಮೀಜಿಯೇ ಗಂಭೀರವಾಗಿ ಮಾತನ್ನು ಮುಂದುವರಿಸಿದರು–“ಏಕೆಂದರೆ ಈ ಭಾವಚಿತ್ರದಲ್ಲಿ ನೀವು ಮಹಾರಾಜರ ಛಾಯೆಯನ್ನು ಕಾಣುತ್ತಿದ್ದೀರಿ. ಈ ಭಾವಚಿತ್ರದ ಮೇಲೆ ಉಗುಳಿದರೆ ನಿಮ್ಮ ಮಹಾರಾಜರ ಮೇಲೆಯೇ ಉಗುಳಿ ದಂತೆ ಎಂದು ನಿಮ್ಮ ಭಾವನೆ.” ಬಳಿಕ ಮಂಗಳಸಿಂಗನತ್ತ ತಿರುಗಿ ಹೇಳಿದರು, “ನೋಡು ಮಹಾರಾಜ, ಈ ಭಾವಚಿತ್ರ ಒಂದು ದೃಷ್ಟಿಯಿಂದ ನೀನಲ್ಲದಿದ್ದರೂ ಇನ್ನೊಂದು ದೃಷ್ಟಿಯಿಂದ ನೀನೇ. ಆದ್ದರಿಂದ ನಿನ್ನ ನಿಷ್ಠಾವಂತ ಸೇವಕರು, ನಾನು ‘ಇದರ ಮೇಲೆ ಉಗುಳಿ’ ಎಂದಾಗ ದಿಗ್ಭ್ರಾಂತರಾದದ್ದು. ಇದು ನಿನ್ನ ಛಾಯೆ; ಇದನ್ನು ನೋಡಿದಾಗಲೆಲ್ಲ ನಿನ್ನ ವ್ಯಕ್ತಿತ್ವವೇ ಅವರ ಕಣ್ಮುಂದೆ ಬರುತ್ತದೆ. ಇದರತ್ತ ಕಣ್ಣು ಹಾಯಿಸಿದಾಗಲೆಲ್ಲ ಅವರು ಇದರಲ್ಲಿ ನಿನ್ನನ್ನೇ ಕಾಣು ತ್ತಾರೆ. ಆದ್ದರಿಂದ ಅವರು ನಿನ್ನನ್ನು ಎಷ್ಟು ಗೌರವದಿಂದ ಕಾಣುತ್ತಾರೋ ಈ ಭಾವಚಿತ್ರವನ್ನೂ ಅಷ್ಟೇ ಗೌರವದಿಂದ ಕಾಣುತ್ತಾರೆ. ದೇವದೇವಿಯರ ಕಲ್ಲಿನ ಅಥವಾ ಲೋಹದ ವಿಗ್ರಹಗಳನ್ನು ಪೂಜಿಸುವ ಭಕ್ತರ ವಿಷಯದಲ್ಲೂ ಹೀಗೆಯೇ. ವಿಗ್ರಹ ಎನ್ನುವುದು ಭಕ್ತರ ಮನಸ್ಸಿಗೆ ಅವರ ಇಷ್ಟದೇವತೆಯನ್ನೋ ಅಥವಾ ಭಗವಂತನ ಕೆಲವೊಂದು ವಿಶೇಷ ಗುಣರೂಪಗಳನ್ನೋ ತಂದು ಕೊಟ್ಟು ಮನಸ್ಸನ್ನು ಏಕಾಗ್ರಗೊಳಿಸಲು ಸಹಾಯಮಾಡುತ್ತದೆ. ಭಕ್ತರು ಹೀಗೆ ಪೂಜೆ ಮಾಡು ವುದು ಮನಸ್ಸಿನಲ್ಲಿ ಮೂಡಿದ ಭಗವಂತನ ದಿವ್ಯ ಗುಣ-ರೂಪಗಳನ್ನು. ಕಲ್ಲು ಅಥವಾ ಲೋಹ ವನ್ನೇ ಯಾರೂ ಪೂಜಿಸುವುದಿಲ್ಲ. ನಾನು ಬಹಳಷ್ಟು ಕಡೆ ತಿರುಗಿಬಂದಿದ್ದೇನೆ. ಆದರೆ, ‘ಓ ಕಲ್ಲೇ, ನಾನು ನಿನ್ನನ್ನು ಪೂಜಿಸುತ್ತೇನೆ; ಹೇ ಲೋಹವೇ, ನನ್ನ ಮೇಲೆ ಕೃಪೆ ಮಾಡು’ ಎನ್ನುತ್ತ ಪೂಜೆ ಮಾಡುವ ಒಬ್ಬನೇ ಒಬ್ಬ ಹಿಂದುವನ್ನೂ ನಾನೆಲ್ಲೂ ಕಂಡಿಲ್ಲ! ಎಲ್ಲರೂ ಪೂಜಿಸುವುದು ಆ ಸರ್ವಶಕ್ತನಾದ ಪರಮಾತ್ಮನನ್ನೇ. ಅವನೇ ಶುದ್ಧಜ್ಞಾನಸ್ವರೂಪಿ. ಭಗವಂತ ಅವರವರ ತಿಳಿವಳಿಕೆ ಹಾಗೂ ಭಾವನೆಗಳಿಗೆ ಅನುಗುಣವಾಗಿ ಕಾಣಿಸಿಕೊಳ್ಳುತ್ತಾನೆ... ಮಹಾರಾಜ, ಇದು ನನ್ನ ಅಭಿಪ್ರಾಯ. ಆದರೆ ನಿನ್ನ ವಿಚಾರ ಹೇಗೋ ನನಗೆ ಗೊತ್ತಿಲ್ಲ.”

ಇದೊಂದು ಅದ್ಭುತ ಘಟನೆಯೇ ಸರಿ. ಮೂರ್ತಿಪೂಜೆಯ ಮಹತ್ವವನ್ನು ಮನಗಾಣಿಸಲು ಸ್ವಾಮೀಜಿ ನೀಡಿದ ಇಂತಹ ದೃಷ್ಟಾಂತಪೂರ್ವಕವಾದ ಉತ್ತರವನ್ನು ಈ ಹಿಂದೆ ಯಾರೂ ಕೊಟ್ಟಿರಲಾರರು; ಇನ್ನು ಮುಂದೆ ಯಾರೂ ಕೊಡಬೇಕಾಗಿಯೂ ಇಲ್ಲ. ಈ ಉತ್ತರ ಸೂರ್ಯ ಚಂದ್ರರಿರುವವರೆಗೂ ಸಾಕು. ಈ ಉತ್ತರವನ್ನು ಕೇಳಿದಾಗ ಎಂತಹ ಮೂರ್ತಿಭಂಜಕರೂ ಕೂಡ ಮೂರ್ತಿಪೂಜಕರಾಗಲು ಸಾಧ್ಯವಿದೆ. ಆದರೆ ಭಂಡರ ವಿಷಯ ಬೇರೆ. ಇಲ್ಲಿ ಇನ್ನೊಂದು ಸ್ವಾರಸ್ಯ ವನ್ನು ಗಮನಿಸಬೇಕು. ಹಿಂದೆ ಯಾವ ನರೇಂದ್ರ ಮೂರ್ತಿಪೂಜೆಯನ್ನು ಉಗ್ರವಾಗಿ ಖಂಡಿಸು ತ್ತಿದ್ದನೋ ಇಂದು ಅದೇ ನರೇಂದ್ರ ಸ್ವಾಮಿ ವಿವೇಕಾನಂದರಾಗಿ ಅದೇ ಮೂರ್ತಿಪೂಜೆಯನ್ನು ಅಷ್ಟೇ ಭವ್ಯವಾಗಿ ಮಂಡಿಸುತ್ತಿರುವುದು ಎಷ್ಟು ಅದ್ಭುತವಾಗಿದೆ! ಅಲ್ಲದೆ ಇಲ್ಲಿ ಇನ್ನೊಂದು ವಿಷಯವನ್ನೂ ಗಮನಿಸಬಹುದು. ಮಂಗಳಸಿಂಗನೋ ನಿರಂಕುಶಮತಿಯಾದ ಮಹಾರಾಜ; ಮನಸ್ಸಿಗೆ ಬಾರದವರನ್ನು ವಿಚಾರಣೆಯಿಲ್ಲದೆ ಶಿಕ್ಷೆಗೆ ಗುರಿಪಡಿಸಬಲ್ಲ ಅಂದಿನ ಕಾಲದ ರಾಜ. ಸ್ವಾಮೀಜಿಯಾದರೋ ಕೇವಲ ಒಬ್ಬ ಸಂನ್ಯಾಸಿ; ಯಾವುದೇ ಬಗೆಯ ಸ್ಥಾನಮಾನ-ಮಠಾಧಿಪತ್ಯ ಗಳೂ ಇಲ್ಲದ ನಿರ್ಗತಿಕ ಸಂನ್ಯಾಸಿ. ಇಂತಹ ಸ್ವಾಮೀಜಿ ಅಪರಿಚಿತ ದೇಶದಲ್ಲಿ ಆ ಮಹಾರಾಜನ ಮುಂದೆ ನಿಂತು ಈ ರೀತಿ ನಡೆದುಕೊಳ್ಳಬೇಕಾದರೆ ಅವರು ವೀರಸಂನ್ಯಾಸಿಯಲ್ಲದೆ ಮತ್ತೇನು?

ಸ್ವಾಮೀಜಿಯ ಮಾತುಗಳನ್ನೆಲ್ಲ ಕಿವಿಗೊಟ್ಟು ಕೇಳುತ್ತಿದ್ದ ಮಹಾರಾಜ ಮಂಗಳಸಿಂಗ್ ಈಗ ಕೈಜೋಡಿಸಿಕೊಂಡು ಹೇಳಿದ, “ಮೂರ್ತಿಪೂಜೆಯ ತತ್ವವನ್ನು ತಾವು ವಿವರಿಸಿದ ಪ್ರಕಾರ ನೋಡಿದಾಗ ಕಲ್ಲು, ಮರ, ಲೋಹಗಳನ್ನು ಪೂಜೆ ಮಾಡುವವರು ಯಾರೂ ಇಲ್ಲವೆಂದು ಒಪ್ಪಿ ಕೊಳ್ಳಬೇಕು. ಈ ಮೊದಲು ನನಗೆ ಮೂರ್ತಿಪೂಜೆಯ ಅಂತರಾರ್ಥ ತಿಳಿದಿರಲಿಲ್ಲ. ತಾವು ನಿಜಕ್ಕೂ ನನ್ನ ಕಣ್ಣು ತೆರೆಸಿದಿರಿ... ಆದರೆ ಮುಂದೆ ನನ್ನ ಗತಿಯೇನು ಸ್ವಾಮೀಜಿ? ದಯವಿಟ್ಟು ನನ್ನ ಮೇಲೆ ಕೃಪೆ ಮಾಡಬೇಕು.”

“ಮಹಾರಾಜ, ಭಗವಂತನೊಬ್ಬನೇ ಕೃಪೆ ತೋರಬಲ್ಲವನು. ಅವನು ಕೃಪಾಸಾಗರ. ಅವನನ್ನು ಪ್ರಾರ್ಥಿಸಿಕೊ. ಅವನು ಕೃಪೆ ಮಾಡುತ್ತಾನೆ.”

ಅಂತೂ ಮಹಾರಾಜ ದಾರಿಗೆ ಬಂದ. ಇಲ್ಲಿ ಸ್ವಾಮೀಜಿ ‘ಕೃಪೆ ದೋರುವವನು ನಾನಲ್ಲ, ಭಗವಂತ’ ಎಂದು ಹೇಳುವುದರ ಮೂಲಕ ತಮ್ಮ ವಿನಯವನ್ನು ವ್ಯಕ್ತಪಡಿಸಿದ್ದು ಒಂದು ವಿಚಾರ ವಾದರೆ, ಮಹಾರಾಜನ ಮನಸ್ಸನ್ನು ಭಗವಂತನ ಕಡೆಗೆ ಹರಿಯಿಸಿದ್ದು ಇನ್ನೊಂದು ವಿಚಾರ. ಹೀಗೆ ಅಂದಿನ ಮಾತುಕತೆ ಅಲ್ಲಿಗೆ ಮುಗಿಯಿತು. ಸ್ವಾಮೀಜಿ ರಾಜನಿಂದ ಬೀಳ್ಗೊಂಡು ತಮ್ಮ ಬಿಡಾರಕ್ಕೆ ಹೊರಟರು. ಆದರೆ ಇತ್ತ ಮಹಾರಾಜ ಮಾತ್ರ ಆಲೋಚನಾಪರನಾಗಿ ಸ್ವಲ್ಪ ಹೊತ್ತು ಅಲ್ಲೇ ಕುಳಿತಿದ್ದ. ಅವನ ಮನಸ್ಸಿನ ತುಂಬೆಲ್ಲ ಸ್ವಾಮೀಜಿಯೇ ತುಂಬಿಕೊಂಡಿದ್ದರು. ಬಳಿಕ ದಿವಾನನನ್ನು ಕರೆದು ಹೇಳಿದ: “ದಿವಾನ್​ಜಿ, ಇಂತಹ ಮಹಾತ್ಮರನ್ನು ನಾನು ಹಿಂದೆಂದೂ ಕಂಡಿರಲಿಲ್ಲ. ಅವರನ್ನು ಇನ್ನೂ ಕೆಲಕಾಲ ನಿಮ್ಮ ಬಳಿ ಉಳಿದುಕೊಳ್ಳುವಂತೆ ಮಾಡಿ.”

“ಮಹಾರಾಜ, ನನ್ನ ಕೈಯಲ್ಲಿ ಸಾಧ್ಯವಾದಷ್ಟು ಪ್ರಯತ್ನ ಮಾಡುತ್ತೇನೆ. ಆದರೆ ಇದರಲ್ಲಿ ನಾನು ಎಷ್ಟರ ಮಟ್ಟಿಗೆ ಯಶಸ್ವಿಯಾದೇನೋ ಹೇಳಲಾರೆ. ಏಕೆಂದರೆ ಅವರದು ಬೆಂಕಿಯಂತಹ ಸ್ವಭಾವ. ಯಾರೂ ಅವರನ್ನು ಹಿಡಿದಿಡಲಾರರು.”

ದಿವಾನ ರಾಮಚಂದ್ರಜಿ ಸ್ವಾಮೀಜಿಯ ಬಳಿಗೆ ಬಂದು ಮಹಾರಾಜನ ಕೋರಿಕೆಯನ್ನು ಮುಂದಿಟ್ಟಾಗ ಅವರು ಮೊದಮೊದಲು ಒಪ್ಪಿಕೊಳ್ಳಲಿಲ್ಲ. ಆದರೆ ದಿವಾನನೂ ಬಿಡಲಿಲ್ಲ. ಮತ್ತೆ ಮತ್ತೆ ಬಿನ್ನವಿಸಿಕೊಂಡ. ಕೊನೆಗೆ ಸ್ವಾಮೀಜಿ ಒಂದು ಷರತ್ತು ಹಾಕಿದರು, “ನೋಡು, ಶ್ರೀಮಂತರು-ವಿದ್ಯಾವಂತರಿಗೆ ನನ್ನನ್ನು ಕಾಣಲು ಸಾಧ್ಯವಿರುವಂತೆ, ಬಡವರು-ಅವಿದ್ಯಾವಂತ ರಿಗೂ ಯಾವಾಗೆಂದರೆ ಆವಾಗ ಬಂದು ನನ್ನನ್ನು ಕಾಣಲು ಅವಕಾಶವಿರಬೇಕು. ಹಾಗಿದ್ದಲ್ಲಿ ಮಾತ್ರ ನಾನು ಇಲ್ಲಿರುತ್ತೇನೆ.” ದಿವಾನ ಇದಕ್ಕೆ ಕೂಡಲೇ ಒಪ್ಪಿಕೊಂಡ. ಸ್ವಾಮೀಜಿಯೂ ದಿವಾನನ ಮನೆಗೆ ಬಂದಿರಲು ಒಪ್ಪಿದರು.

ಸ್ವಾಮೀಜಿ ಹಾಕಿದ ಆ ಷರತ್ತು ತುಂಬ ಸೂಕ್ತವಾಗಿದೆ. ಏಕೆಂದರೆ ಸ್ವಾಮೀಜಿ ವಾಸವಾಗಿ ಇರುವುದು ದಿವಾನರ ಬಂಗಲೆಯಲ್ಲಿ. ಬಂಗಲೆಯೊಳಗೆ ಬಡವರಿಗೆಲ್ಲ ಪ್ರವೇಶವಿರಲು ಸಾಧ್ಯವೇ? ಬಡವರೆಲ್ಲ ಸಾಧಾರಣವಾಗಿ ಅವಿದ್ಯಾವಂತರೂ ಆಗಿರುತ್ತಿದ್ದರು. ಸಹಜವಾಗಿಯೇ ದೊಡ್ಡವರ ಬಳಿಗೆ ಬರಲು, ದೊಡ್ಡ ದೊಡ್ಡ ಬಂಗಲೆಗಳಿಗೆ ಪ್ರವೇಶಿಸಲು ಅವರಿಗೆ ದಿಗಿಲು. ಆದರೆ ಸ್ವಾಮೀಜಿ ಈ ಧರೆಗೆ ಅವತರಿಸಿರುವುದೇ ಮುಖ್ಯವಾಗಿ ದೀನದಲಿತರಿಗಾಗಿ. ಆದ್ದರಿಂದಲೇ ಬಡವರು- ಅವಿದ್ಯಾವಂತರಿಗೆ ತಮ್ಮನ್ನು ನೋಡಲು ಅವಕಾಶವಿರಬೇಕು ಎಂದು ಅವರು ಹೇಳಿದ್ದು; ಅದೂ ಕೂಡ, ಯಾವಾಗೆಂದರೆ ಆವಾಗ ನೋಡಲು ಅವಕಾಶವಿರಬೇಕು ಎಂದು. ಏಕೆ? ಅವಿದ್ಯಾವಂತ ರಿಗೆ ಹೊತ್ತು ಗೊತ್ತು ನೋಡಿಕೊಂಡು ಬರುವ ಕ್ರಮ ಗೊತ್ತೂ ಇಲ್ಲ; ಅಲ್ಲದೆ ದೂರದೂರ ದಿಂದ ಬರಬಹುದಾದ್ದರಿಂದ ಸಕಾಲದಲ್ಲಿ ಬರಲು ಸಾಧ್ಯವೂ ಇಲ್ಲ. ಸ್ವಾಮೀಜಿ ಇವನ್ನೆಲ್ಲ ಗಮನಿಸಿಯೇ ಇದ್ದರು.

ಸ್ವಾಮೀಜಿಯನ್ನು ಕಾಣಲು ದಿವಾನನ ಮನೆಗೆ ಹಲವಾರು ಜನ ಬರಲಾರಂಭಿಸಿದರು. ಸ್ವಾಮೀಜಿಯ ಸಂಪರ್ಕಕ್ಕೆ ಬಂದವರಲ್ಲಿ ಅನೇಕರು ತಮ್ಮ ಜೀವನಕ್ರಮವನ್ನು ತಿದ್ದಿಕೊಂಡರು. ಪ್ರತಿದಿನ ಬರುತ್ತಿದ್ದವರಲ್ಲಿ ಒಬ್ಬ ವೃದ್ಧನೂ ಇದ್ದ. ಆತ ದಿನವೂ ಬಂದು, “ಸ್ವಾಮೀಜಿ, ತಾವು ನನ್ನ ಮೇಲೆ ಕೃಪೆ ಮಾಡಬೇಕು; ನನ್ನ ಮೇಲೆ ಕರುಣೆ ತೋರಬೇಕು” ಎಂದು ಕೇಳಿಕೊಳ್ಳುತ್ತಿದ್ದ. ಸ್ವಾಮೀಜಿ ಅವನಿಗೆ ಕೆಲವು ಸಾಧನಾವಿಧಾನಗಳನ್ನು ಅನುಸರಿಸುವಂತೆ ಸಲಹೆ ನೀಡುತ್ತಿದ್ದರು. ಆದರೆ ಆ ಸಲಹೆಗಳನ್ನು ಸುಮ್ಮನೆ ಕೇಳಿಸಿಕೊಂಡು ಹೋಗುವುದಷ್ಟೇ ಆ ಮುದುಕನ ಪರಿಪಾಠ ವಾಗಿತ್ತು. ಅವುಗಳನ್ನು ಅನುಷ್ಠಾನಕ್ಕೆ ತರುವ ಗೊಡವೆಗೇ ಹೋಗಲಿಲ್ಲ ಆತ. ಇದನ್ನು ಗಮನಿಸಿದ ಸ್ವಾಮೀಜಿಗೆ ಆ ಮುದುಕನ ವಿಷಯದಲ್ಲಿ ಅಸಮಾಧಾನವಾಯಿತು. ಆದರೆ ಮುದುಕ ಮಾತ್ರ ಪ್ರತಿದಿನ ಬಂದು ಯಥಾಪ್ರಕಾರ ಕೃಪಾಯಾಚನೆ ಮಾಡುವುದನ್ನು ಬಿಡಲಿಲ್ಲ. ಕಡೆಗೊಂದು ದಿನ ಅವರಿಗೆ ಇವನಿಂದ ಹೇಗಾದರೂ ಮಾಡಿ ಬಿಡಿಸಿಕೊಳ್ಳಬೇಕು ಎಂದು ಅನ್ನಿಸಿಬಿಟ್ಟಿತು. ಆ ದಿನ ಆತ ‘ಎಂದಿನಂತೆ’ ತಮ್ಮ ಬಳಿಗೆ ಬರುತ್ತಿರುವುದನ್ನು ದೂರದಿಂದಲೇ ಕಂಡು, ಮುಖ ಗಂಟು ಹಾಕಿಕೊಂಡು ಕುಳಿತು ಬಿಟ್ಟರು. ವೃದ್ಧ ಮಹಾಶಯ ಬಂದ. ಯಥಾಪ್ರಕಾರ ಘನ ಆಧ್ಯಾತ್ಮಿಕ ವಿಷಯಗಳನ್ನು ಕುರಿತು ಗಹನವಾದ ಪ್ರಶ್ನೆಗಳನ್ನು ಕೇಳಿದ. “ಕೃಪೆ ಮಾಡಬೇಕು” ಎಂದು ಕೇಳು ವುದನ್ನು ಮರೆಯಲಿಲ್ಲ. ಆದರೆ ಸ್ವಾಮೀಜಿ ಮಾತ್ರ ಅವನ ಯಾವ ಪ್ರಶ್ನೆಗೂ ಉತ್ತರ ಕೊಡದೆ ಗಂಭೀರ ಮುಖಮುದ್ರೆಯಿಂದ ಕುಳಿತಿದ್ದರು. ಈ ವೇಳೆಗೆ ಅಲ್ಲಿ ಇನ್ನೂ ಹಲವಾರು ಭಕ್ತರು ಸೇರಿದ್ದರು. ಯಾರೊಂದಿಗೂ ಸ್ವಾಮೀಜಿಯವರದು ಮಾತಿಲ್ಲ ಕತೆಯಿಲ್ಲ. ಎಲ್ಲರೂ ಸುಮ್ಮನೆ ಮುಖಮುಖ ನೋಡುತ್ತ ಕುಳಿತುಕೊಳ್ಳುವಂತಾಯಿತು. ವಿಷಯವೇನೆಂದು ಯಾರಿಗೂ ಅರ್ಥ ವಾಗಲಿಲ್ಲ. ಹೀಗೆ ಸುಮಾರು ಒಂದೂವರೆ ಗಂಟೆ ಕಳೆಯಿತು. ಸ್ವಾಮೀಜಿ ವಿಗ್ರಹದಂತೆ ಕುಳಿತೇ ಇದ್ದರು. ಮುದುಕನಿಗೆ ಇನ್ನು ತಡೆಯಲಾಗಲಿಲ್ಲ; ಕೋಪ ಉಕ್ಕೇರಿತು. ಗಟ್ಟಿಯಾಗಿ ಶಪಿಸುತ್ತ ಅಲ್ಲಿಂದೆದ್ದು ಹೊರಟುಹೋದ. ಅವನು ಹೋದನೋ ಇಲ್ಲವೋ ಸ್ವಾಮೀಜಿ ಗಟ್ಟಿಯಾಗಿ ನಗ ತೊಡಗಿದರು. ಇತರರೂ ಅವರ ನಗುವಿನಲ್ಲಿ ಭಾಗಿಗಳಾದರು. ಆಗ ಯುವಕನೊಬ್ಬ ಕೇಳಿದ:

“ಸ್ವಾಮೀಜಿ, ನೀವು ಆ ಮುದುಕನ ವಿಷಯದಲ್ಲಿ ಅಷ್ಟೇಕೆ ಕಟುವಾಗಿ ನಡೆದುಕೊಂಡಿರಿ?”

ಸ್ವಾಮೀಜಿ ಪ್ರೀತಿಯ ದನಿಯಲ್ಲಿ ಹೇಳಿದರು:

“ನೋಡಿ, ನಾನು ನಿಮ್ಮಂತಹ ಯುವಕರಿಗಾಗಿ ನನ್ನ ಇಡೀ ಜೀವನವನ್ನೇ ತ್ಯಾಗ ಮಾಡಲು ಸಿದ್ಧ. ಏಕೆಂದರೆ ನಿಮಗೆ ನನ್ನ ಆದೇಶಗಳನ್ನು ಕಾರ್ಯಗತಗೊಳಿಸುವ ಮನಸ್ಸೂ ಇದೆ, ಸಾಮರ್ಥ್ಯವೂ ಇದೆ. ಆದರೆ ಈ ಮನುಷ್ಯನಾದರೋ ತನ್ನ ಜೀವನದ ತೊಂಬತ್ತು ಭಾಗವನ್ನೆಲ್ಲ ಪ್ರಾಪಂಚಿಕ ಭೋಗವನ್ನು ಬೆನ್ನಟ್ಟುವುದರಲ್ಲೇ ಕಳೆದಿದ್ದಾನೆ. ಈಗ ಇವನಿಗೆ ಪ್ರಾಪಂಚಿಕ ಜೀವನಕ್ಕೂ ಯೋಗ್ಯತೆಯಿಲ್ಲ, ಆಧ್ಯಾತ್ಮಿಕ ಜೀವನಕ್ಕೂ ಯೋಗ್ಯತೆಯಿಲ್ಲ. ಸುಮ್ಮನೆ ಬಾಯಿ ಮಾತಿನಲ್ಲಿ ಕೇಳುವುದರಿಂದಲೆ ಭಗವಂತನ ಕೃಪೆಯನ್ನು ಪಡೆದುಕೊಂಡುಬಿಡಬಹುದೆಂದು ಭಾವಿಸಿದ್ದಾನೆ. ಸತ್ಯಸಾಕ್ಷಾತ್ಕಾರ ಮಾಡಿಕೊಳ್ಳುವುದಕ್ಕೆ ಮುಖ್ಯವಾಗಿ ಬೇಕಾದುದು ಸ್ವಪ್ರಯತ್ನ. ಯಾರು ಯಾವ ಪ್ರಯತ್ನವನ್ನೂ ಮಾಡದೆ ಸುಮ್ಮನೆ ಕುಳಿತಿರುತ್ತಾನೋ ಅವನ ಮೇಲೆ ಭಗವಂತ ಹೇಗೆ ಕೃಪೆ ಮಾಡಿಯಾನು? ಯಾರಲ್ಲಿ ಪೌರುಷವಿರುವುದಿಲ್ಲವೋ ಅವರಲ್ಲಿರುವುದು ಬರಿಯ ತಮಸ್ಸು ಮಾತ್ರ. ವೀರಾಧಿವೀರನಾದ ಅರ್ಜುನ ಈ ಪೌರುಷವನ್ನು ಕಳೆದುಕೊಂಡವನಂತೆ ಕಂಡುಬಂದದ್ದರಿಂದಲೇ ಶ್ರೀಕೃಷ್ಣ ಅವನ ಸ್ವಭಾವಕ್ಕೆ ಅನುಗುಣವಾದ ಜೀವನಕರ್ತವ್ಯಗಳನ್ನು ಅನುಸರಿಸುವಂತೆ ಅವನಿಗೆ ಆಜ್ಞಾಪಿಸಿದ್ದು. ಹೀಗೆ ಪ್ರತಿಫಲಾಪಯೇಕ್ಷೆಯಿಲ್ಲದೆ ಕರ್ಮ ಮಾಡುವು ದರ ಮೂಲಕ ಅವನು ಹೃದಯಶುದ್ಧಿ, ಕರ್ಮತ್ಯಾಗ, ಹಾಗೂ ಶರಣಾಗತಿ ಎಂಬ ಸಾತ್ವಿಕ ಗುಣಗಳನ್ನು ಹೊಂದುವಂತಾಗಲಿ ಎಂಬುದೇ ಶ್ರೀಕೃಷ್ಣನ ಉದ್ದೇಶವಾಗಿತ್ತು. ಆದ್ದರಿಂದ ನೀವೆಲ್ಲ ಶಕ್ತಿಶಾಲಿಗಳಾಗಿ! ಪೌರುಷವಂತರಾಗಿ! ಶಕ್ತಿಶಾಲಿಯೂ ಪೌರುಷವಂತನೂ ಆಗಿರುವ ಒಬ್ಬ ನೀಚನನ್ನೂ ಕೂಡ ನಾನು ಗೌರವಿಸುತ್ತೇನೆ. ಏಕೆಂದರೆ ಅವನ ಪೌರುಷವೇ ಒಂದು ದಿನ ಅವನು ತನ್ನೆಲ್ಲ ನೀಚ ಕೃತ್ಯಗಳನ್ನು ಬಿಡುವಂತೆ ಮಾಡುತ್ತದೆ; ಅಷ್ಟೇ ಅಲ್ಲ, ಅವನು ಎಲ್ಲ ಸ್ವಾರ್ಥಪ್ರೇರಿತ ಕರ್ಮಗಳನ್ನೂ ತ್ಯಜಿಸಿಬಿಡುವಂತೆಯೂ ಮಾಡುತ್ತದೆ. ಹೀಗೆ ಅವನಿಗೆ ಕಾಲಕ್ರಮದಲ್ಲಿ ಸತ್ಯಸಾಕ್ಷಾತ್ಕಾರವಾಗುತ್ತದೆ.”

ಆ ಮುದುಕನ ವಿಷಯವಾಗಿ ಸ್ವಾಮೀಜಿ ನಡೆದುಕೊಂಡ ಬಗೆ ತುಂಬ ಗಮನೀಯವಾದದ್ದೇ ಸರಿ. ನವೋತ್ಸಾಹದಿಂದ ಕೂಡಿದ ಯುವಸಂನ್ಯಾಸಿಯಾದ ಸ್ವಾಮೀಜಿಗೆ ತಮ್ಮ ಪರಿವ್ರಾಜಕ ದಿನ ಗಳಲ್ಲಿ ಬಗೆಬಗೆಯ ಅನುಭವಗಳಾಗುತ್ತಿವೆ. ಆ ಮುದುಕನಿಗೆ ಆಧ್ಯಾತ್ಮಿಕ ಜೀವನದ ಸೂಕ್ಷ್ಮಗಳ ನ್ನೆಲ್ಲ ತಿಳಿಸಿಕೊಟ್ಟು, ಅವನು ಆ ಮಾರ್ಗದಲ್ಲಿ ಮುನ್ನಡೆಯುವಂತೆ ಸಹಾಯ ಮಾಡಲು ಅವರು ಪ್ರಾಮಾಣಿಕವಾಗಿಯೇ ಪ್ರಯತ್ನಿಸಿದರು. ಮುಮುಕ್ಷಗಳಿಗೆ ನೆರವಾಗಲೆಂದೇ ಬಂದವರಲ್ಲವೆ ಅವರು! ಆದರೆ ಆ ಬೋಧನೆಯ ಪರಿಣಾಮವೇನಾಯಿತು? ಏನೂ ಆಗಲಿಲ್ಲ. ಅಥವಾ ಪರಿಣಾಮವಾಯಿತೆಂದೂ ಹೇಳಬಹುದೇನೋ, ಏನೆಂದರೆ, ಆ ಮುದುಕ ಸ್ವಾಮೀಜಿಯಿಂದ ಆಕರ್ಷಿತನಾಗಿ ಪ್ರತಿದಿನ ಅವರ ಬೋಧನೆಯನ್ನು ಕೇಳಲು ಬಂದು ಕುಳಿತುಬಿಡುತ್ತಿದ್ದನಲ್ಲ! ಆಧ್ಯಾತ್ಮಿಕ ಬೋಧನೆಯನ್ನು ಕೇಳಲು ಪ್ರತಿದಿನ ತಪ್ಪದೆ ಬಂದು ಕುಳಿತುಕೊಳ್ಳಬೇಕಾದರೆ, ಅವರ ವಾಣಿ ಅವನ ಮೇಲೆ ಪರಿಣಾಮ ಬೀರಿದೆಯೆಂದೇ ಅರ್ಥವಲ್ಲವೆ? ಆದ್ದರಿಂದಲೇ ತಾನೆ ಅವನು ಉತ್ಸಾಹದಿಂದ ಬರುತ್ತಿದ್ದುದು? ಹೀಗಿರುವಾಗ ಸ್ವಾಮೀಜಿಯೇ ಅವನ ವಿಷಯದಲ್ಲಿ ನಿರುತ್ಸಾಹ ತಾಳಿ, ಅವನಿಂದ ಹೇಗಾದರೂ ಮಾಡಿ ಬಿಡಿಸಿಕೊಳ್ಳಬೇಕೆಂದು ತೀರ್ಮಾನಿಸಿದರಲ್ಲ, ಅದೇಕೆ? ಏಕೆಂದರೆ, ಅವರಿಗೇ ಈಗ ‘ಜ್ಞಾನೋದಯ’ವಾಗಿ ಬಿಟ್ಟಿದೆ–ಮುದುಕರ ಕೈಯಲ್ಲಿ ಆಧ್ಯಾತ್ಮಿಕ ಸಾಧನೆಯೆಲ್ಲ ಸಾಧ್ಯವೇ ಇಲ್ಲ; ಇವರಿಗೆ ಬೋಧನೆ ಮಾಡುವುದೆಂದರೆ ಕೇವಲ ಕಂಠಶೋಷಣೆ, ನೀರಿನಲ್ಲಿ ಮಾಡಿದ ಹೋಮ–ಎಂದು! ಆಧ್ಯಾತ್ಮಿಕ ಬೋಧನೆಗಳನ್ನು ಸಾಧನೆಗಿಳಿಸಬೇಕಾದರೆ ಶರೀರದಲ್ಲೂ ಬುದ್ಧಿಯಲ್ಲೂ ತಾಕತ್ತು ಇರಬೇಕಾಗುತ್ತದೆ ಎಂಬುದು ಅವರಿಗೀಗ ಮನದಟ್ಟಾಗುತ್ತಿದೆ.

ಸ್ವಾಮೀಜಿಯ ಪ್ರಬಲ ವ್ಯಕ್ತಿತ್ವದಿಂದ ಸೆಳೆಯಲ್ಪಟ್ಟು ಅನೇಕ ಯುವಕರು ಅವರ ಬಳಿಗೆ ಬರಲಾರಂಭಿಸಿದ್ದರು. ಈ ಯುವಕರಿಗೆ ಸ್ವಾಮೀಜಿ ಭಾರತೀಯ ಸಂಸ್ಕೃತಿಯ ಭವ್ಯ ಪರಂಪರೆ ಯನ್ನು ಅರಿಯುವ ಪ್ರಯತ್ನ ಮಾಡುವಂತೆ ಬೋಧಿಸಿದರು. ಈ ಯುವಭಕ್ತರಿಗೆ ಅವರು ಹೇಳುತ್ತಾರೆ, “ಸಂಸ್ಕೃತವನ್ನು ಕಲಿಯಿರಿ. ಆದರೆ ಅದರೊಂದಿಗೆ ಪಾಶ್ಚಾತ್ಯ ವಿಜ್ಞಾನವನ್ನೂ ಕಲಿಯಿರಿ. ಮತ್ತು ಕಲಿಯುವುದನ್ನು ನಿಖರವಾಗಿ ಕಲಿಯಿರಿ.” ಈ ಸಲಹೆಯನ್ನು ಅನುಸರಿಸಿ ಅಲ್ವರಿನ ಅನೇಕ ಯುವಕರು ಸಂಸ್ಕೃತಾಭ್ಯಾಸವನ್ನು ಪ್ರಾರಂಭಿಸಿದರು. ಕೆಲವೊಮ್ಮೆ ಸಂಸ್ಕೃತವನ್ನು ಕಲಿಯುವುದರಲ್ಲಿ ಸ್ವಾಮೀಜಿಯೂ ಅವರಿಗೆ ನೆರವಾದರು.

ಇಲ್ಲಿ ಸ್ವಾಮೀಜಿ ಭವ್ಯ ಭಾರತದ ನಿರ್ಮಾಣಕ್ಕೆ ಹೇಗೆ ಮಾರ್ಗದರ್ಶನ ನೀಡುತ್ತಿದ್ದಾರೆ ಎಂಬುದು ಕಂಡು ಬರುತ್ತದೆ. ಭಾರತದ ಭವ್ಯ ಸಂಸ್ಕೃತಿಯನ್ನು ಹಾಗೂ ಧರ್ಮವನ್ನು ಮೈಗೂಡಿಸಿ ಕೊಳ್ಳುವುದಕ್ಕೋಸ್ಕರ ಸಂಸ್ಕೃತವನ್ನು ಕಲಿಯಿರಿ ಎನ್ನುತ್ತಿದ್ದಾರೆ. ಆದರೆ ಬದುಕು ನಡೆಯಬೇಕಾ ದರೆ ಕೇವಲ ಸಂಸ್ಕೃತಿ-ಧರ್ಮ ಸಾಲದು; ಅನ್ನ-ಬಟ್ಟೆ ಹಾಗೂ ಇತರ ಜೀವನ ಸೌಕರ್ಯಗಳೂ ಅಷ್ಟೇ ಆವಶ್ಯಕ. ಆದ್ದರಿಂದ ಪಾಶ್ಚಾತ್ಯ ತಂತ್ರಜ್ಞಾನವನ್ನೂ ಪಡೆದುಕೊಳ್ಳುವಂತೆ ಹೇಳುತ್ತಿದ್ದಾರೆ. ಮತ್ತು ಹೀಗೆ ಹೇಳುವುದರ ಜೊತೆಗೆ, ‘ಕಲಿಯವುದನ್ನು ನಿಖರವಾಗಿ ಕಲಿಯಿರಿ’ ಎನ್ನುತ್ತಿದ್ದಾರೆ. ಅವರ ಈ ಮಾತು ತುಂಬ ಮಹತ್ತ್ವಪೂರ್ಣವಾದದ್ದು. ಕೇವಲ ಪರೀಕ್ಷೆ ಪಾಸು ಮಾಡುವುದಕ್ಕಾಗಿ ಉರು ಹೊಡೆದು ವಿದ್ಯಾವಂತರೆನ್ನಿಸಿಕೊಂಡವರಿಂದ ಏನಾದೀತು? ಸಂಸ್ಕೃತವನ್ನು ನಿಖರವಾಗಿ ಅಧ್ಯಯನ ಮಾಡುವುದರಿಂದ ಧರ್ಮಗ್ರಂಥಗಳಲ್ಲಿರುವ ಅಮೂಲ್ಯ ವಿಚಾರಗಳನ್ನು ಅರಿತು ಕೊಳ್ಳಲು ಸಾಧ್ಯವಾಗುತ್ತದೆ; ಧರ್ಮದ ಮರ್ಮವನ್ನು ಅರಿತುಕೊಂಡಾಗ ಆತ್ಮವಿಶ್ವಾಸ ಜಾಗೃತವಾಗುತ್ತದೆ. ತನ್ಮೂಲಕ ಸಮಾಜನಿಷ್ಠೆ, ರಾಷ್ಟ್ರಪ್ರಜ್ಞೆ ಕೂಡ ಜಾಗೃತಗೊಳ್ಳುತ್ತವೆ. ಅಂತೆಯೇ ಪಾಶ್ಚಾತ್ಯ ವಿಜ್ಞಾನ-ತಂತ್ರಜ್ಞಾನಗಳನ್ನು ನಿಖರವಾಗಿ ಅರಿತು ಅಳವಡಿಸಿಕೊಳ್ಳುವುದರ ಮೂಲಕ ಐಹಿಕ ಸಂಪತ್ತನ್ನು ಹೆಚ್ಚಿಸಿಕೊಳ್ಳಬಹುದು; ತನ್ಮೂಲಕ ಭಾರತದ ಪ್ರತಿಯೊಬ್ಬ ಪ್ರಜೆಯೂ ಸಂತುಷ್ಟಿಯ ಜೀವನವನ್ನು ನಡಸುವಂತಾಗುತ್ತದೆ. ಆದ್ದರಿಂದಲೇ ಸ್ವಾಮೀಜಿ ಆ ಯುವಕರಿಗೆ ಭಾರತೀಯ ಧರ್ಮ-ಸಂಸ್ಕೃತಿಗಳೊಂದಿಗೆ ಪಾಶ್ಚಾತ್ಯ ವಿಜ್ಞಾನ-ತಂತ್ರಜ್ಞಾನಗಳನ್ನು ಅಳವಡಿಸಿ ಕೊಳ್ಳುವಂತೆ ಮತ್ತು ಕಲಿಯುವುದನ್ನು ನಿಖರವಾಗಿ ಕಲಿಯುವಂತೆ ಹೇಳಿದರು. ಅವರು ಈ ಕರೆಯನ್ನು ನೀಡಿ ಇಂದಿಗೆ ನೂರು ವರ್ಷಗಳೂ ಆಗಿಲ್ಲ. ಈ ಅಲ್ಪಾವಧಿಯಲ್ಲೇ ನಮ್ಮ ದೇಶದಲ್ಲಿ ಸಂಸ್ಕೃತ ಜ್ಞಾನ ಹಾಗೂ ಪಾಶ್ಚಾತ್ಯ ವಿಜ್ಞಾನಗಳೆರಡೂ ಸಾಕಷ್ಟು ವ್ಯಾಪಕವಾಗಿ ಬೆಳೆದಿರುವುದನ್ನು ಕಾಣಬಹುದಾಗಿದೆ. ಆದರೆ ಸ್ವಾಮೀಜಿ ಹೇಳಿದ ‘ನಿಖರತೆ’ ಮಾತ್ರ ಇನ್ನೂ ಬರಬೇಕಾಗಿದೆ.

ಭಾರತೀಯ ಸಂಸ್ಕೃತಿಯನ್ನು ಅಧ್ಯಯನ ಮಾಡುವಂತೆ ಅಲ್ವರಿನ ಯುವಕರನ್ನು ಪ್ರೋತ್ಸಾಹಿ ಸಿದ ಸ್ವಾಮೀಜಿ, ರಾಷ್ಟ್ರನಿರ್ಮಾಣದ ಮಹೋದ್ದೇಶವು ಈಡೇರಬೇಕಾದರೆ ಭಾರತದ ಚರಿತ್ರೆ ಯನ್ನು ಪುನರ್ನಿರೂಪಿಸಬೇಕಾದ ಆವಶ್ಯಕತೆಯನ್ನು ಒತ್ತಿ ಹೇಳುತ್ತಾರೆ:

“ಸಂಸ್ಕೃತ ಭಾಷೆಯನ್ನೂ ಪಾಶ್ಚಾತ್ಯರ ವೈಜ್ಞಾನಿಕ ವಿಧಾನಗಳನ್ನೂ ಶ್ರಮವಹಿಸಿ ಅಧ್ಯಯನ ಮಾಡಿ. ಆಗ ಭಾರತದ ಇತಿಹಾಸವನ್ನು ವೈಜ್ಞಾನಿಕ ತಳಹದಿಯ ಮೇಲೆ ನಿಲ್ಲಿಸಲು ಸಾಧ್ಯವಾಗು ವಂತಹ ಕಾಲವೊಂದು ಬರುತ್ತದೆ. ಈಗಿರುವ ಭಾರತದ ಇತಿಹಾಸವೆಲ್ಲ ಅವ್ಯವಸ್ಥಿತ. ಅದರ ಕಾಲಾನುಕ್ರಮಣಿಕೆಯೂ ನಿಖರವಾಗಿಲ್ಲ. ಪಾಶ್ಚಾತ್ಯರು ಬರೆದ ಈ ಚರಿತ್ರೆಯು ನಮ್ಮ ಮನಸ್ಸನ್ನು ದುರ್ಬಲಗೊಳಿಸುತ್ತದೆಯಷ್ಟೆ. ಏಕೆಂದರೆ ಅದು ಮೊದಲಿನಿಂದ ಕಡೆಯವರೆಗೂ ನಮ್ಮ ಅವನತಿ ಯನ್ನಷ್ಟೇ ಹೇಳುತ್ತದೆ. ನಮ್ಮ ಧರ್ಮ-ಸಂಸ್ಕೃತಿ-ತತ್ವಗಳನ್ನು, ಆಚಾರ-ವಿಚಾರ-ಸಂಪ್ರದಾಯ ಗಳನ್ನು ಅರ್ಥಮಾಡಿಕೊಳ್ಳಲಾರದ ಆ ವಿದೇಶೀಯರು, ಪೂರ್ವಗ್ರಹ ಪೀಡಿತವಲ್ಲದ ಪ್ರಾಮಾಣಿಕ –ನಿಷ್ಪಕ್ಷಪಾತ ಚರಿತ್ರೆಯನ್ನು ಬರೆಯಲು ಹೇಗೆ ಸಾಧ್ಯ? ಸ್ವಾಭಾವಿಕವಾಗಿಯೇ ಎಷ್ಟೋ ತಪ್ಪು ತಿಳಿವಳಿಕೆಗಳು, ತಪ್ಪು ತೀರ್ಮಾನಗಳು ಅವುಗಳಲ್ಲಿ ನುಸುಳಿವೆ. ಆದರೆ ಈ ನಮ್ಮ ಪ್ರಾಚೀನ ಇತಿಹಾಸ ಸಂಶೋಧನಾಕಾರ್ಯದಲ್ಲಿ ಮುಂದುವರಿಯುವುದು ಹೇಗೆಂಬುದನ್ನು ಐರೋಪ್ಯರೇ ನಮಗೆ ತೋರಿಸಿಕೊಟ್ಟಿದ್ದಾರೆ. ಈಗ ಇತಿಹಾಸ ಸಂಶೋಧನೆಯ ನೂತನ ಮಾರ್ಗವೊಂದನ್ನು ಸ್ವತಂತ್ರವಾಗಿ ಅನ್ವೇಷಿಸುವ ಕೆಲಸ ನಮಗೆ ಸೇರಿದ್ದು. ವೇದಗಳನ್ನು, ಪುರಾಣಗಳನ್ನು ಹಾಗೂ ಪ್ರಾಚೀನ ಐತಿಹಾಸಿಕ ದಾಖಲೆಗಳನ್ನು ಅಧ್ಯಯನ ಮಾಡಿ ಅದರ ಸಹಾಯದಿಂದ ನಿಷ್ಕೃಷ್ಟವಾದ, ಸಹಾನುಭೂತಿಪೂರ್ವಕವಾದ ಹಾಗೂ ಸ್ಫೂರ್ತಿದಾಯಕವಾದ ನಮ್ಮ ನಾಡಿನ ಇತಿಹಾಸವನ್ನು ಬರೆಯಬೇಕು. ಇದನ್ನು ನಮ್ಮ ಜೀವನಾವಧಿಯ ಕಾರ್ಯವನ್ನಾಗಿ ಮಾಡಿಕೊಳ್ಳಬೇಕು; ನಮ್ಮ ಜೀವನ-ಧ್ಯೇಯವನ್ನಾಗಿ ಮಾಡಿಕೊಳ್ಳಬೇಕು. ಭಾರತದ ಇತಿಹಾಸವನ್ನು ಬರೆಯಬೇಕಾದವರು ಭಾರತೀಯರು. ಆದ್ದರಿಂದ ಕಾಲಗರ್ಭದಲ್ಲಿ ಕಣ್ಮರೆಯಾಗಿ ಹೋಗಿರುವ ನಮ್ಮ ಅಮೂಲ್ಯ ಸಂಪತ್ತನ್ನು ಮರಳಿ ಪಡೆಯುವ ಕಾರ್ಯಕ್ಕೆ ಕೈಹಾಕಿ. ತನ್ನ ಮಗುವು ಕಾಣೆಯಾಗಿದ್ದರೆ ಒಬ್ಬ ಮನುಷ್ಯನು ಅದನ್ನು ಮರಳಿ ಪಡೆಯುವವರೆಗೂ ಸುಮ್ಮನಾಗುವುದಿಲ್ಲ. ಅದರಂತೆಯೇ, ಪ್ರಾಚೀನ ಭಾರತದ ವೈಭವವನ್ನು ಜನಮನದಲ್ಲಿ ಪುನಃಸ್ಥಾಪಿಸುವವರೆಗೂ ನಿಮ್ಮ ಕಾರ್ಯವನ್ನು ನಿಲ್ಲಿಸಬೇಡಿ. ನಿಜವಾದ ರಾಷ್ಟ್ರೀಯ ಶಿಕ್ಷಣವೆಂದರೆ ಅದು. ಇಂತಹ ರಾಷ್ಟ್ರೀಯ ಶಿಕ್ಷಣದ ಪ್ರಗತಿಯೊಂದಿಗೆ ನಿಜವಾದ ರಾಷ್ಟ್ರಪ್ರಜ್ಞೆಯೂ ಜಾಗೃತವಾಗುತ್ತದೆ.”

ಸರ್ವಸಂಗ ಪರಿತ್ಯಾಗಿಗಳಾದ, ‘ಬ್ರಹ್ಮಂ ಸತ್ಯಂ ಜಗನ್ಮಿಥ್ಯಾ’ ಎಂದರಿತ ಮತ್ತು ಪರಮಾರ್ಥ ವನ್ನೇ ಪರಮೋದ್ದೇಶವಾಗುಳ್ಳ ಸಂನ್ಯಾಸಿಗಳಾದ ಸ್ವಾಮಿ ವಿವೇಕಾನಂದರು ರಾಷ್ಟ್ರೀಯ ಭಾವನೆ ಯನ್ನು ಜಾಗೃತಗೊಳಿಸುವ ಮಾರ್ಗವನ್ನು ಬೋಧಿಸುತ್ತಿದ್ದಾರೆ. ಪ್ರತಿಯೊಬ್ಬ ರಾಷ್ಟ್ರಕನಲ್ಲೂ ರಾಷ್ಟ್ರಭಾವ ಜಾಗೃತವಾಗಿ ಅವನೊಬ್ಬ ಕೆಚ್ಚೆದೆಯ ಪ್ರಜೆಯಾಗಿ ನಿರ್ಮಾಣಗೊಳ್ಳಬೇಕಾದರೆ ಅವನಲ್ಲಿ ರಾಷ್ಟ್ರದ ಇತಿಹಾಸಪ್ರಜ್ಞೆ ಎಚ್ಚತ್ತಿರಬೇಕಾಗುತ್ತದೆ. ಅಂತಹ ಪ್ರಜ್ಞೆಯನ್ನು ಬಡಿದೆಬ್ಬಿಸಲು ಸತ್ವಪೂರ್ಣವಾದ ಇತಿಹಾಸವನ್ನು ಬರೆದು ಜನತೆಯ ಮುಂದಿಡಬೇಕಾಗಿದೆ. ಹಿಂದೂಧರ್ಮ- ಸಂಸ್ಕೃತಿಯ ಪರಿಚಯವಿರುವ ಭಾರತೀಯರೇ ಅಂತಹ ಸತ್ವಪೂರ್ಣ ಇತಿಹಾಸವನ್ನು ಬರೆಯ ಬೇಕು ಎನ್ನುತ್ತಿದ್ದಾರೆ ಸ್ವಾಮೀಜಿ.

ಸ್ವಾಮೀಜಿಯ ವ್ಯಕ್ತಿತ್ವ ಅಲ್ವರಿನಲ್ಲಿ ಅವರನ್ನು ಎಲ್ಲರಿಗೂ ಪ್ರೀತಿಪಾತ್ರರನ್ನಾಗಿಸಿತ್ತು. ಅಲ್ಲಿ ಅವರ ಬಳಿಗೆ ಒಬ್ಬ ಬ್ರಾಹ್ಮಣರ ಹುಡುಗ ಆಗಾಗ ಬರುತ್ತಿದ್ದ. ಅವನು ಸ್ವಾಮೀಜಿಯನ್ನು ತನ್ನ ಸ್ವಂತ ಗುರುವೆಂದೇ ಭಾವಿಸಿ ವಿಶ್ವಾಸ ತಾಳಿದ್ದ. ಅವನಿಗೆ ಉಪನಯನದ ವಯಸ್ಸಾಗಿತ್ತು. ಆದರೆ ಬಡವ. ಉಪನಯನ ಮಾಡಿಕೊಳ್ಳುವ ಅನುಕೂಲತೆಯಿರಲಿಲ್ಲ. ಸ್ವಾಮೀಜಿಗೆ ಇದು ತಿಳಿದು ಬಂದಾಗ ಅವರಿಂದ ಸುಮ್ಮನಿರಲಾಗಲಿಲ್ಲ. ಆದ್ದರಿಂದ ಅವರು ತಮ್ಮ ಕೆಲವು ಶ್ರೀಮಂತ ಭಕ್ತರ ಮುಂದೆ ಹೇಳಿದರು, “ನಿಮ್ಮಲ್ಲಿ ನನ್ನೊಂದು ಕೋರಿಕೆಯಿದೆ. ಈ ಬ್ರಾಹ್ಮಣ ಬಾಲಕ ತನ್ನ ಉಪನಯನದ ಖರ್ಚನ್ನು ಹೊಂದಿಸಿಕೊಳ್ಳಲಾರದವನಾಗಿದ್ದಾನೆ. ಅವನಿಗೆ ಸಹಾಯ ಮಾಡ ಬೇಕಾದದ್ದು ಗೃಹಸ್ಥರಾದ ನಿಮ್ಮೆಲ್ಲರ ಕರ್ತವ್ಯ. ಅವನ ಪರವಾಗಿ ನೀವು ಸ್ವಲ್ಪ ವಂತಿಗೆ ಎತ್ತಬೇಕು. ಈ ವಯಸ್ಸಿನ ಬ್ರಾಹ್ಮಣ ಬಾಲಕನೊಬ್ಬನಿಗೆ ತಾನು ಆಚರಿಸಬೇಕಾದ ಧಾರ್ಮಿಕ ವಿಧಿಗಳ ಬಗ್ಗೆ ತಿಳಿದಿಲ್ಲದಿರುವುದು ಸರಿಯಲ್ಲ. ಅಲ್ಲದೆ ಅವನ ವಿದ್ಯಾಭ್ಯಾಸಕ್ಕೂ ಒಂದು ವ್ಯವಸ್ಥೆ ಮಾಡಲು ಸಾಧ್ಯವಾಗುವುದಾದರೆ ಬಹಳ ಒಳ್ಳೆಯದು.”

ಈ ಸಂದರ್ಭದಲ್ಲೊಂದು ಸ್ವಾರಸ್ಯವನ್ನು ಗಮನಿಸಬಹುದು ನಾವು. ಸ್ವತಃ ಸ್ವಾಮೀಜಿ ಹುಟ್ಟಿನಿಂದ ಬ್ರಾಹ್ಮಣರಲ್ಲ. ಇನ್ನು ಸಂನ್ಯಾಸ ಸ್ವೀಕರಿಸಿದ ಮೇಲಂತೂ ಅವರು ಚತುರ್ವರ್ಣ ಗಳನ್ನೂ ಮೀರಿದವರಾದರು. ಆದರೆ ಅವರು ‘ಬ್ರಾಹ್ಮಣ ಸಂನ್ಯಾಸಿ’ಗಳಲ್ಲವೆಂಬ ಕಾರಣದಿಂದ ಸಂಪ್ರದಾಯಸ್ಥ ಬ್ರಾಹ್ಮಣರು ಅವರನ್ನು ಅನಾದರಿಸಿ, ಅವರೊಂದಿಗೆ ಅಹಿತಕರವಾಗಿ ನಡೆದು ಕೊಂಡ ಘಟನೆಗಳು ಅವರ ಜೀವನದುದ್ದಕ್ಕೂ ಹಲವಾರು. ವಸ್ತುಸ್ಥಿತಿ ಹೀಗಿದ್ದರೂ ಈಗ ಅವರು ಬ್ರಾಹ್ಮಣ ಬಾಲಕನೊಬ್ಬನ ಉಪನಯನದ ವಿಷಯದಲ್ಲಿ ಕಾಳಜಿ ವಹಿಸುತ್ತಿದ್ದಾರೆ! ಬ್ರಾಹ್ಮಣ ವರ್ಗಕ್ಕೆ ನಿಜಕ್ಕೂ ತಮ್ಮ ಬ್ರಾಹ್ಮಣ್ಯದಲ್ಲಿ ಅಭಿಮಾನವಿದ್ದದ್ದೇ ಆದಲ್ಲಿ, ತಮ್ಮ ಬ್ರಾಹ್ಮಣ ಕುಲದಲ್ಲಿ ಉಪನಯನವಾಗದೆ ಉಳಿದ ಒಬ್ಬನೇ ಒಬ್ಬ ಬಾಲಕನೂ ಇಲ್ಲದಿರುವಂತೆ ನೋಡಿ ಕೊಳ್ಳಬೇಕಾದುದು ಅದರ ಆದ್ಯಕರ್ತವ್ಯ. ಆರ್ಥಿಕ ಅಥವಾ ಇತರ ಅಡಚಣೆಗಳಿಂದಾಗಿ, ಪ್ರಾಪ್ತ ವಯಸ್ಸಿನ ಬ್ರಾಹ್ಮಣ ಬಾಲಕರು ಉಪನಯನವಾಗದೆ ಉಳಿದುಕೊಳ್ಳುವಂತಹ ದುಸ್ಥಿತಿ ನೂರು ವರ್ಷಗಳ ಹಿಂದೆಯೂ ಇದ್ದಿತೆಂಬುದು ತಿಳಿದುಬರುತ್ತದೆ. ಆದರೆ ಗಮನಾರ್ಹ ಅಂಶವೇನೆಂದರೆ ಇದನ್ನು ಕಂಡ ಸ್ವಾಮೀಜಿ ‘ಏನು ಮಾಡಲಾದೀತು, ಹುಡುಗನ ಹಣೇಬರಹ’ ಎಂದೋ, ಅಥವಾ ‘ಕಾಲ ಕೆಟ್ಟುಹೋಯಿತು’ ಎಂದೋ ದೂಷಿಸಿ ಸುಮ್ಮನಾಗಲಿಲ್ಲ. ಬದಲಾಗಿ ತಕ್ಷಣ ಅದರ ಬಗ್ಗೆ ಕ್ರಮ ಕೈಗೊಂಡು ಬಾಲಕನ ಉಪನಯನಕ್ಕೆ ವ್ಯವಸ್ಥೆ ಮಾಡಿಸಿದರು. ಹೀಗೆ ಮಾಡುವುದರ ಮೂಲಕ ಸಮಾಜಕ್ಕೆ ಒಂದು ಮೇಲ್ಪಂಕ್ತಿಯನ್ನು ಹಾಕಿಕೊಟ್ಟರು. ಸಂನ್ಯಾಸಿಗಳು ಪರಿವ್ರಾಜಕರಾಗಿ ದೇಶ ಸಂಚಾರ ಮಾಡಬೇಕಾದ ಉದ್ದೇಶಗಳಲ್ಲಿ ಇದೂ ಒಂದು–ಅವರು ತಮ್ಮದೇ ಆದ ರೀತಿಯಿಂದ ಸಮಾಜದಲ್ಲಿ ಧರ್ಮವನ್ನು ನೆಲೆಗೊಳಿಸುವಂತಾಗುತ್ತದೆ; ಸಂಸ್ಕೃತಿಯನ್ನು ಊರ್ಜಿತಗೊಳಿಸುವಂತಾಗುತ್ತದೆ.

ಈ ಬ್ರಾಹ್ಮಣಬಾಲಕನ ಉಪನಯನಕ್ಕಾಗಿ ವಂತಿಗೆ ಸಂಗ್ರಹಿಸುವಂತೆ ತಮ್ಮ ಕೆಲವು ಭಕ್ತರಿಗೆ ಸ್ವಾಮೀಜಿ ತಿಳಿಸಿದ್ದರೂ, ಬಳಿಕ ಕೆಲವೇ ದಿನಗಳಲ್ಲಿ ಅವರು ಅಲ್ವರಿನಿಂದ ಹೊರಟು ಮೌಂಟ್ ಅಬು ಎಂಬಲ್ಲಿಗೆ ಹೋದರು. ಆದರೆ ಅವರು ಆ ಹುಡುಗನ ಉಪನಯನದ ವಿಷಯವನ್ನು ಮರೆತಿರಲಿಲ್ಲವೆಂಬುದು ಅವರು ಮೌಂಟ್ ಅಬುವಿನಿಂದ ಅಲ್ವರಿನ ತಮ್ಮ ಶಿಷ್ಯನಾದ ಗೋವಿಂದ ಸಹಾಯ್ ಎಂಬವನಿಗೆ ಬರೆದ ಪತ್ರದಿಂದ ತಿಳಿದುಬರುತ್ತದೆ. ಆ ಪತ್ರ ಹೀಗಿದೆ –

“.... ನೀವು ಆ ಬ್ರಾಹ್ಮಣ ಬಾಲಕನ ಉಪನಯನವನ್ನು ನೆರವೇರಿಸಿದ್ದೀರಿ ತಾನೆ? ಅಲ್ಲದೆ ನೀನು ಸಂಸ್ಕೃತಾಭ್ಯಾಸವನ್ನು ಮಾಡುತ್ತಿದ್ದೀಯಾ? ಅದು ಎಲ್ಲಿಯವರೆಗೆ ಬಂದಿದೆ? ಈ ವೇಳೆಗೆ ಪ್ರಥಮ ಭಾಗವನ್ನು ಮುಗಿಸಿರಬೇಕೆಂದು ಭಾವಿಸುತ್ತೇನೆ... ಶಿವಪೂಜೆಯನ್ನು ಪಟ್ಟು ಹಿಡಿದು ಮಾಡುತ್ತಿರುವೆಯಷ್ಟೆ? ಇಲ್ಲದಿದ್ದರೆ ಇನ್ನು ಮುಂದಾದರೂ ಬಿಡದೆ ಮಾಡಲು ಪ್ರಯತ್ನಿಸು. ‘ಮೊದಲು ಭಗವಂತನ ಸಾಮ್ರಾಜ್ಯವನ್ನು ಪಡೆದುಕೊ; ಉಳಿದುದೆಲ್ಲವೂ ತಾನಾಗಿಯೇ ನಿನ್ನಲ್ಲಿಗೆ ಬಂದು ಕೂಡಿಕೊಳ್ಳುತ್ತವೆ.’ ನೀವು ಭಗವಂತನನ್ನು ಅನುಸರಿಸಿದರೆ ನಿಮಗೆ ಬೇಕಾದುದನ್ನೆಲ್ಲ ಪಡೆಯುತ್ತೀರಿ... ನನ್ನ ಪುತ್ರರಿರಾ, ಧರ್ಮದ ತಿರುಳಿರುವುದು ಸಿದ್ಧಾಂತಗಳಲ್ಲಲ್ಲ, ಆಚರಣೆ ಯಲ್ಲಿ. ಒಳ್ಳೆಯವರಾಗುವುದು ಮತ್ತು ಒಳ್ಳೆಯದನ್ನು ಮಾಡುವುದು–ಇದೇ ಧರ್ಮದ ಸರ್ವಸ್ವ. ‘ಭಗವಂತನಿಗೆ ಪ್ರಿಯನಾದವನು ಯಾರೆಂದರೆ, ಕೇವಲ ‘ಪ್ರಭು ಪ್ರಭು!’ ಎಂದು ಕೂಗಿಕೊಳ್ಳು ವವನಲ್ಲ, ಯಾರು ಅವನ ಅನುಜ್ಞೆಯನ್ನು ನೆರವೇರಿಸುತ್ತಾನೆಯೋ ಅವನು.’ ನೀವು ಅಲ್ವರೀ ಯುವಕರು ಒಳ್ಳೇ ಉತ್ಸಾಹಿಗಳು. ಶೀಘ್ರದಲ್ಲೇ ನೀವು ನಿಮ್ಮ ಸಮಾಜಕ್ಕೆ ರತ್ನಪ್ರಾಯರಾಗಿ, ನಿಮಗೆ ಜನ್ಮವಿತ್ತ ಈ ದೇಶಕ್ಕೆ ಹೆಮ್ಮೆ ತರುವವರಾಗುತ್ತೀರೆಂದು ಆಶಿಸುತ್ತೇನೆ.”

ಒಂದು ದಿನ ಸ್ವಾಮೀಜಿ, “ಇಲ್ಲಿ ಸಮೀಪದಲ್ಲಿ ಯಾರಾದರೂ ಸಾಧುಸಂತರು ವಾಸವಾಗಿ ಇದ್ದಾರೆಯೇ?” ಎಂದು ಅಲ್ವರಿನ ಕೆಲವು ಭಕ್ತರನ್ನು ಕೇಳಿದಾಗ, ಅಲ್ಲೊಬ್ಬ ವೃದ್ಧ ಬ್ರಹ್ಮಚಾರಿ ವಾಸವಾಗಿರುವ ವಿಷಯ ತಿಳಿದುಬಂದಿತು. ಕೆಲವು ಪರಿಚಯಸ್ಥರನ್ನು ಜೊತೆಗೂಡಿಕೊಂಡು ಅವನನ್ನು ನೋಡಲು ಹೊರಟರು. ಈತ ವೈಷ್ಣವ ಮತಾವಲಂಬಿ; ಇವನಿಗೆ ವೇದಾಂತಿಗಳಾದ ಸಂನ್ಯಾಸಿಗಳ ಮೇಲೆ ಬದ್ಧದ್ವೇಷ. ಸ್ವಾಮೀಜಿಯನ್ನು ನೋಡಿದ ಕೂಡಲೇ ಈ ಬ್ರಹ್ಮಚಾರಿ, ಸಂನ್ಯಾಸಧರ್ಮವನ್ನೂ ಕಷಾಯ ವಸ್ತ್ರದ ಧರಿಸುವಿಕೆಯನ್ನೂ ವಾಚಾಮಗೋಚರವಾಗಿ ಬೈಯ ಲಾರಂಭಿಸಿದ. ಆದರೆ ಸ್ವಾಮೀಜಿ ಯಾವ ಪ್ರತಿಕ್ರಿಯೆಯನ್ನೂ ತೋರದೆ, “ದೇವರು-ಧರ್ಮದ ಬಗ್ಗೆ ಏನಾದರೂ ಸ್ವಲ್ಪ ಹೇಳಿ” ಎಂದು ಬ್ರಹ್ಮಚಾರಿಯನ್ನು ವಿನಂತಿಸಿಕೊಂಡರು. ತನ್ನ ಬೈಗಳ ಅವರ ಮೇಲೆ ಯಾವುದೇ ಪರಿಣಾಮವನ್ನುಂಟುಮಾಡದಿದ್ದುದನ್ನು ಕಂಡ ಬ್ರಹ್ಮಚಾರಿ ತಣ್ಣಗಾಗಿ, “ಹೋಗಲಿ, ನನಗೆ ನಿಮ್ಮ ಮೇಲೇನೂ ಕೋಪವಿಲ್ಲ. ಈಗ ಏನಾದರೂ ಸ್ವಲ್ಪ ಉಪಾಹಾರವನ್ನು ತೆಗೆದುಕೊಳ್ಳಿ” ಎಂದ. ಆಗ ತಾನೇ ತಾವು ಊಟ ಮಾಡಿ ಬಂದಿರುವುದಾಗಿ ತಿಳಿಸಿದ ಸ್ವಾಮೀಜಿ ಅವನ ಆತಿಥ್ಯವನ್ನು ವಿನಯದಿಂದಲೇ ನಿರಾಕರಿಸಿದರು. ಆದರೆ ಬ್ರಹ್ಮಚಾರಿಗೆ ತಕ್ಷಣ ಕೋಪ ಬಂದುಬಿಟ್ಟಿತು. “ತೊಲಗಿಹೋಗಿ ಇಲ್ಲಿಂದ” ಎಂದು ಅಬ್ಬರಿಸಿದ. ಸ್ವಾಮೀಜಿ ಮರುಮಾತ ನಾಡದೆ ಅವನಿಗೆ ನಮಸ್ಕಾರ ಮಾಡಿ ಅಲ್ಲಿಂದ ಹೊರಟುಬಂದರು. ಆ ಬ್ರಹ್ಮಚಾರಿಯ ಆಶ್ರಮ ದಿಂದ ಹೊರಗೆ ಬಂದಕೂಡಲೇ ಸ್ವಾಮೀಜಿ ಗಟ್ಟಿಯಾಗಿ ನಗುತ್ತ ಭಕ್ತರ ಹತ್ತಿರ ಹೇಳುತ್ತಾರೆ, “ಆಹಾ, ಎಂಥ ವಿಚಿತ್ರ ಸಾಧುವಿನ ಬಳಿಗೆ ಕರೆದುಕೊಂಡು ಬಂದಿರಿ ನನ್ನನ್ನು! ಅಬ್ಬ! ಎಂಥಾ ಮುಂಗೋಪಿ!” ಅಲ್ಲದೆ ಅವರು ಆ ವೃದ್ಧನ ಮಾತಿನ ಧಾಟಿಯನ್ನು ಅಣಕವಾಗಿ ತೋರಿಸಿದಾಗ ಅಲ್ಲೊಂದು ನಗೆಯ ಹೊನಲೇ ಎದ್ದಿತು.

ಒಂದು ದಿನ ಅಲ್ವರಿನ ಒಬ್ಬ ಶಿಷ್ಯ ಸ್ವಾಮೀಜಿಯನ್ನು ತನ್ನ ಮನೆಗೆ ಊಟಕ್ಕೆ ಆಹ್ವಾನಿಸಿದ. ಅವರು ಅವನ ಮನೆಗೆ ಹೋದಾಗ ಅವನು ಮೈಗೆ ಎಣ್ಣೆ ತಿಕ್ಕಿಕೊಳ್ಳುತ್ತ ಸ್ನಾನಕ್ಕೆ ಸಿದ್ಧನಾಗುತ್ತಿದ್ದ. ಎಣ್ಣೆಯನ್ನು ತಿಕ್ಕಿಕೊಳ್ಳುತ್ತಲೇ ಆತ ಕೇಳಿದ, “ಸ್ವಾಮೀಜಿ, ಸ್ನಾನಕ್ಕೆ ಮುಂಚೆ ಮೈಗೆ ಎಣ್ಣೆ ತಿಕ್ಕಿಕೊಳ್ಳುವುದರಿಂದ ಏನಾದರೂ ಪ್ರಯೋಜನವಿದೆಯೆ?” ಪ್ರತಿದಿನ ಮೈಗೆ ಎಣ್ಣೆ ತಿಕ್ಕಿಕೊಳ್ಳು ವವನಿಗೆ ಅದರ ಪ್ರಯೋಜನ ತಿಳಿಯದಿರುತ್ತದೆಯೆ? ಆದರೂ ಸ್ವಾಮೀಜಿ ಏನು ಹೇಳುತ್ತಾರೋ ನೋಡೋಣ ಎಂಬ ಕುತೂಹಲ ಅವನಿಗೆ. ಸ್ವಾಮೀಜಿ ಬಹಳ ಸ್ವಾರಸ್ಯಕರವಾದ ಉತ್ತರವನ್ನೇ ಕೊಟ್ಟರು–“ಪ್ರಯೋಜನವಿದೆ; ಕಾಲು ಸೇರು ತುಪ್ಪ ತಿನ್ನುವುದೂ ಒಂದೇ, ಮೈಗೆ ಒಂದು ಚಟಾಕು ಎಣ್ಣೆ ತಿಕ್ಕಿಕೊಳ್ಳುವುದೂ ಒಂದೇ.”

ಊಟವಾದ ಮೇಲೆ ಆ ಶಿಷ್ಯ ಕೇಳಿದ–

“ಸ್ವಾಮೀಜಿ, ನೀವೇನೋ ನಮಗೆ ಸತ್ಯಸಂಧತೆ-ಪ್ರಾಮಾಣಿಕತೆ-ಧೈರ್ಯ-ಪವಿತ್ರತೆ-ನಿಷ್ಕಾಮ ಕರ್ಮ ಇತ್ಯಾದಿ ವಿಚಾರಗಳ ಬಗೆಗೆಲ್ಲ ಹೇಳುತ್ತಿರುತ್ತೀರಿ. ಆದರೆ ಉದ್ಯೋಗದಲ್ಲಿರುವವರಿಗೆ ಇದನ್ನೆಲ್ಲ ನಿಷ್ಠೆಯಿಂದ ಪಾಲಿಸಿಕೊಂಡು ಬರಲು ಸಾಧ್ಯವಾಗುವುದಿಲ್ಲ. ಏಕೆಂದರೆ ನಾವು ಧನಸಂಪಾದನೆಗಾಗಿಯೇ ಕೆಲಸ ಮಾಡುವವರು. ನಾವು ನಮ್ಮ ಕೆಲಸವನ್ನು ನಿಷ್ಕಾಮಕರ್ಮ ಎಂದು ಹೇಳಿಕೊಳ್ಳುವುದು ಹೇಗೆ? ಮತ್ತೊಬ್ಬರ ಕೈ ಕೆಳಗೆ ದುಡಿಯುವುದೆಂದರೆ ಅದೊಂದು ಬಗೆಯ ಗುಲಾಮಗಿರಿಯೇ ಸರಿ. ಅಲ್ಲದೆ ಇಂದಿನ ದಿನಗಳಲ್ಲಿ ವ್ಯಾಪಾರ-ವ್ಯವಹಾರ ಎಂಬುದು ಹೇಗಾಗಿಬಿಟ್ಟಿದೆಯೆಂದರೆ ಸತ್ಯಸಂಧತೆ–ಪ್ರಾಮಾಣಿಕತೆಗಳನ್ನು ಉಳಿಸಿಕೊಳ್ಳಲು ಸಾಧ್ಯವೇ ಇಲ್ಲ. ನಾವು ಈ ಪ್ರಪಂಚದಲ್ಲಿದ್ದುಕೊಂಡು ದುಡಿಯಬೇಕೆಂದರೆ ನೈತಿಕತೆಯನ್ನು ಉಳಿಸಿಕೊಳ್ಳಲು ಸಾಧ್ಯವಿಲ್ಲ. ಸ್ವಾಮೀಜಿ, ಈ ಬಗ್ಗೆ ನೀವೇನು ಹೇಳುತ್ತೀರಿ?”

ಆ ಶಿಷ್ಯ ಈ ಪ್ರಶ್ನೆಯನ್ನು ಕೇಳಿ ಒಂದು ಶತಮಾನವೇ ಸಲ್ಲುತ್ತ ಬಂದಿತು. ಆದರೆ ಇಂದಿಗೂ ಈ ಪ್ರಶ್ನೆ ಚಲಾವಣೆಯಲ್ಲಿದೆ. ‘ಹಿಂದಿನಕಾಲ ಎಂದರೆ ಸುವರ್ಣ ಯುಗ! ಈಗಿನ ಕಾಲ ನಿಜಕ್ಕೂ ಕಲಿಗಾಲ. ಎಲ್ಲ ಕೆಟ್ಟುಹೋಗಿದೆ’ ಎಂಬ ಮಾತನ್ನು ಆಗಾಗ ಕೇಳುತ್ತಲೇ ಇರುತ್ತೇವೆ. ಆದರೆ ಈ ಪ್ರಶ್ನೆಗೆ ಸ್ವಾಮೀಜಿ ನೀಡಿದ ಉತ್ತರ ಮಾತ್ರ ಅತ್ಯಂತ ಮನನೀಯವಾಗಿದೆ, ನಿತ್ಯನೂತನವಾಗಿದೆ–

“ನಾನೂ ಈ ವಿಷಯವಾಗಿ ಬಹಳಷ್ಟು ಆಲೋಚಿಸಿ ಕೊನೆಗೊಂದು ತೀರ್ಮಾನಕ್ಕೆ ಬಂದಿ ದ್ದೇನೆ. ಅದೇನೆಂದರೆ ತಮ್ಮ ಶೀಲವನ್ನು ಸರಿಯಾಗಿಟ್ಟುಕೊಂಡು ಹಣ ಸಂಪಾದನೆ ಮಾಡಲು ನಿಜಕ್ಕೂ ಯಾರಿಗೂ ಇಷ್ಟವಿಲ್ಲ. ಅಲ್ಲದೆ ನಿನ್ನಂತೆ ಯಾರಿಗೂ ಇದೊಂದು ಆಲೋಚಿಸಬೇಕಾದ ವಿಚಾರವೆಂದೂ ಅನ್ನಿಸುವುದಿಲ್ಲ. ಅಥವಾ ಇದೊಂದು ಸಮಸ್ಯೆಯಂತೆಯೂ ಕಂಡು ಬರುವು ದಿಲ್ಲ. ಇದು ನಮ್ಮ ಇಂದಿನ ಶಿಕ್ಷಣ ಕ್ರಮದಲ್ಲಿನ ದೋಷ. ವೈಯಕ್ತಿಕವಾಗಿ ನನಗನ್ನಿಸುತ್ತದೆ– ವ್ಯವಸಾಯವನ್ನು ಉದ್ಯೋಗವನ್ನಾಗಿ ಮಾಡಿಕೊಂಡರೆ ಅದರಲ್ಲಿ ತಪ್ಪೇನೂ ಇಲ್ಲ ಎಂದು. ಆದರೆ ನೀನು ಯಾರಿಗಾದರೂ ಹಾಗೆ ಹೇಳಿದರೆ ಅವನು ಕೇಳುತ್ತಾನೆ, ‘ಹಾಗಾದರೆ ನಾನಿಷ್ಟೆಲ್ಲ ಓದಿದ್ದೇಕೆ ಮತ್ತೆ? ದೇಶದಲ್ಲಿರುವವರೆಲ್ಲ ರೈತರಾಗಬೇಕೆ? ಈಗಾಗಲೇ ದೇಶದ ತುಂಬ ರೈತರಿದ್ದಾರೆ. ಅದಕ್ಕೇ ನಮ್ಮ ದೇಶ ಈ ದುರ್ಗತಿಗೆ ಬಂದಿರುವುದು’ ಎಂದು. ಆದರೆ ಈ ಮಾತು ಸರಿಯಲ್ಲ, ರಾಮಾಯಣವನ್ನು ಓದಿ ನೋಡು. ಅದಲ್ಲಿ ಜನಕ ಮಹಾರಾಜ ಒಂದು ಕೈಯಲ್ಲಿ ನೇಗಿಲು ಹಿಡಿದು ಇನ್ನೊಂದು ಕೈಯಲ್ಲಿ ವೇದಾಭ್ಯಾಸ ಮಾಡಿದ ಎಂದು ಹೇಳಿದೆ. ನಮ್ಮ ಸನಾತನ ಪುಷಿಗಳೆಲ್ಲ ಕೃಷಿಕರಾಗಿದ್ದರು. ಅಲ್ಲದೆ, ವ್ಯವಸಾಯದಲ್ಲಿ ಪ್ರಗತಿ ಸಾಧಿಸುವುದರ ಮೂಲಕ ಅಮೆರಿಕ ಹೇಗೆ ಮುಂದುವರಿದಿದೆ ನೋಡು! ಆದರೆ ನಾವು ನಮ್ಮ ದೇಶದ ಅನಕ್ಷರಸ್ಥ ರೈತರಂತೆ ವ್ಯವಸಾಯವನ್ನು ಕೈಗೊಳ್ಳಬೇಕೆಂದು ನಾನು ಹೇಳುತ್ತಿಲ್ಲ. ನಾವು ಕೃಷಿ ವಿಜ್ಞಾನವನ್ನು ನಿಖರವಾಗಿ ಕಲಿತು, ಆ ಜ್ಞಾನವನ್ನು ನಮ್ಮ ವ್ಯವಸಾಯಕ್ಕೆ ಅಳವಡಿಸಿಕೊಳ್ಳಬೇಕು. ನಮಗೆ ಅವಶ್ಯಕವಾದ ಜ್ಞಾನವನ್ನು ಗಳಿಸಿ, ನಾವು ಬುದ್ಧಿವಂತರ ರೀತಿಯಲ್ಲಿ ಕೆಲಸ ಮಾಡಬೇಕು. ಆದರೆ ಇಂದಿನ ದಿನಗಳಲ್ಲಿ ಹಳ್ಳಿಗಳ ಹುಡುಗರು ಒಂದೆರಡು ಇಂಗ್ಲಿಷ್ ಪುಸ್ತಕಗಳನ್ನು ಓದುವುದೇ ತಡ, ನಗರಗಳಿಗೆ ಓಡಿಬಿಡುತ್ತಾರೆ. ಹಳ್ಳಿಯಲ್ಲಿ ಅವರಿಗೆ ಬೇಕಾದಷ್ಟು ಜಮೀನಿದ್ದರೂ ಅದರಿಂದ ಅವರಿಗೆ ತೃಪ್ತಿಯಿರುವುದಿಲ್ಲ. ಅವರಿಗೆ ಯಾವುದಾದರೂ ನೌಕರಿಗೆ ಸೇರಿ ನಗರ ಜೀವನದ ಭೋಗವನ್ನು ಅನುಭವಿಸುವ ಆಸೆ. ಆದ್ದರಿಂದಲೇ ಇತರ ಜನಾಂಗಗಳಂತೆ ಹಿಂದೂಗಳು ಯಾವ ಪ್ರಗತಿಯನ್ನೂ ಸಾಧಿಸಿಲ್ಲ. ನಮ್ಮ ದೇಶದ ಸಾವಿನ ಪ್ರಮಾಣ ತುಂಬ ಹೆಚ್ಚು. ಇದು ಹೀಗೆಯೇ ಮುಂದುವರಿದರೆ, ನಮ್ಮ ದೇಶ ಶೀಘ್ರದಲ್ಲೇ ನಾಮಾವಶೇಷವಾಗಬಹುದು. ಇದಕ್ಕೆ ಮೂಲ ಕಾರಣವೆಂದರೆ ನಮ್ಮ ಕೃಷಿ ಉತ್ಪಾದನೆ ಕಡಿಮೆ. ಹಳ್ಳಿಗರಲ್ಲಿ ಪಟ್ಟಣಕ್ಕೆ ವಲಸೆ ಹೋಗುವ ಪ್ರವೃತ್ತಿ ಹೆಚ್ಚಾಗುತ್ತಿದೆ. ಒಬ್ಬ ರೈತನ ಮಗ, ಅಲ್ಪ ಸ್ವಲ್ಪ ಶಿಕ್ಷಣ ಪಡೆದನೆಂದರೆ, ತನ್ನ ಕಸುಬನ್ನು ಬಿಟ್ಟು ಪಟ್ಟಣ ಸೇರಿ ಅಲ್ಲಿ ಬಿಳಿಯರ ಕೈ ಕೆಳಗೆ (ಸ್ವಾತಂತ್ರ್ಯಪೂರ್ವದಲ್ಲಿ) ಕೆಲಸಕ್ಕೆ ಸೇರಿಕೊಳ್ಳು ತ್ತಾನೆ. ನಿಸರ್ಗದ ಮಡಿಲಾದ ಹಳ್ಳಿಗಳಲ್ಲಿ ರೋಗರುಜಿನಗಳು ಬಹಳ ಕಡಿಮೆ; ಅಲ್ಲದೆ ಅಲ್ಲಿ ವಾಸಿಸುವುದರಿಂದ ಆಯುಸ್ಸೂ ವೃದ್ಧಿಯಾಗುತ್ತದೆ. ವಿದ್ಯಾವಂತರು ಹಳ್ಳಿಗಳಿಗೆ ಹೋಗಿ ನೆಲಸಿ ದರೆ ಚಿಕ್ಕ ಚಿಕ್ಕ ಹಳ್ಳಿಗಳೂ ಅಭಿವೃದ್ಧಿ ಹೊಂದುತ್ತವೆ. ಅಲ್ಲದೆ ವ್ಯವಸಾಯವನ್ನು ವೈಜ್ಞಾನಿಕ ವಿಧಾನದಲ್ಲಿ ಮಾಡಿದರೆ ಉತ್ಪಾದನೆಯೂ ಹೆಚ್ಚಾಗುತ್ತದೆ. ಹೀಗಾಗಿ ರೈತರು ತಮ್ಮ ಕರ್ತವ್ಯದ ಕಡೆಗೆ ಎಚ್ಚತ್ತುಕೊಳ್ಳುವಂತಾಗುತ್ತದೆ, ಮತ್ತು ಅವರ ಬೌದ್ಧಿಕತೆಯೂ ಹೆಚ್ಚಾಗುತ್ತದೆ. ಇದರಿಂದಾಗಿ ಅವರಿಗೆ ಹೆಚ್ಚು ಹೆಚ್ಚು ಉಪಯುಕ್ತ ವಿಷಯಗಳನ್ನು ಕಲಿಯಲು ಸಾಧ್ಯವಾಗುತ್ತದೆ. ಇವೆಲ್ಲದರ ಜೊತೆಗೆ, ಇಂದಿಗೆ ಯಾವುದು ಅತ್ಯಾವಶ್ಯಕವಾಗಿದೆಯೋ ಅದು ಸಿದ್ಧಿಸುತ್ತದೆ.”

“ಅದು ಎಂದರೆ...?”

“ಮತ್ತಿನ್ನಾವುದು? ಮೇಲ್ವರ್ಗ ಹಾಗೂ ಕೆಳವರ್ಗದ ಜನರ ನಡುವೆ ಭ್ರಾತೃಭಾವದ ಬೆಳವಣಿಗೆ. ನಿನ್ನಂತಹ ವಿದ್ಯಾವಂತರು ಹಳ್ಳಿಗಳಿಗೆ ಹೋಗಿ ಬೇಸಾಯದಲ್ಲಿ ತೊಡಗಿ ಹಳ್ಳಿಗ ರೊಂದಿಗೆ ಕಲೆತು, ಅವರನ್ನು ಹೀನಾಯವಾಗಿ ನೋಡದೆ ನಿನ್ನ ಸ್ವಂತದವರೆಂದು ನೋಡಿಕೊಂಡರೆ ಅವರು ಎಷ್ಟು ಸಂತೋಷಪಡುತ್ತಾರೆ ಗೊತ್ತೇನು? ಅವರು ನಿನಗಾಗಿ ತಮ್ಮ ಪ್ರಾಣವನ್ನೇ ಕೊಡಲು ಸಿದ್ಧರಾಗುತ್ತಾರೆ. ಈಗ ಅತ್ಯಾವಶ್ಯಕವಾಗಿ ಆಗಬೇಕಾಗಿರುವಂತಹ ಜನಸಾಮಾನ್ಯರಿಗೆ ಶಿಕ್ಷಣ, ಉನ್ನತ ಸತ್ಯಗಳನ್ನು ಹರಿಜನರಿಗೂ ಬೋಧಿಸುವುದು, ಹಾಗೂ ಪರಸ್ಪರ ಪ್ರೀತಿ ಸಹಾನುಭೂತಿ–ಇವು ಕೂಡ ಸಿದ್ಧಿಸುತ್ತವೆ.”

“ಇವೆಲ್ಲ, ಹೇಗೆ ಸಾಧ್ಯವಾಗುತ್ತದೆ, ಸ್ವಾಮೀಜಿ?”

“ಏಕೆ! ನೀನೇ ನೋಡುತ್ತಿಲ್ಲವೇನು, ಯಾರಾದರೂ ವಿದ್ಯಾವಂತರು ಹಳ್ಳಿಗಳಿಗೆ ಹೋದರೆ ಹಳ್ಳಿಯ ಜನ ಅವರ ಸಂಪರ್ಕಕ್ಕೆ ಬರಲು ಹೇಗೆ ಹಾತೊರೆಯುತ್ತಾರೆ! ಜ್ಞಾನದಾಹ ಎನ್ನುವುದು ಎಲ್ಲರಲ್ಲೂ ಇದ್ದೇ ಇರುತ್ತದೆ. ಆದ್ದರಿಂದಲೇ ಅವರು ವಿದ್ಯಾವಂತನೊಬ್ಬ ಸಿಕ್ಕಾಗ ಅವನ ಸುತ್ತ ಕುಳಿತುಕೊಂಡು ಅವನು ಹೇಳುವುದನ್ನೆಲ್ಲ ಕಿವಿಗೊಟ್ಟು ಕೇಳುವುದು. ವಿದ್ಯಾವಂತ ಜನರು ಹಳ್ಳಿಗರ ಈ ಪ್ರವೃತ್ತಿಯನ್ನು ಸದುಪಯೋಗಪಡಿಸಿಕೊಂಡು, ಪ್ರತಿದಿನ ಸಂಜೆ ಕೆಲವು ಹಳ್ಳಿಗರನ್ನು ತಮ್ಮ ಮನೆಗೆ ಕರೆದು ಕಥೆ-ದೃಷ್ಟಾಂತಗಳ ಮೂಲಕ ಅವರಿಗೆ ಬೋಧಿಸಬೇಕು. ಇದರಿಂದ ಎಂತಹ ರಾಷ್ಟ್ರೀಯ ಜಾಗೃತಿಯುಂಟಾಗುತ್ತದೆಯೆಂದರೆ, ಒಂದು ಸಾವಿರ ವರ್ಷಗಳಲ್ಲಿ ಎಷ್ಟನ್ನು ಸಾಧಿಸಬಹುದೋ ಅದಕ್ಕಿಂತ ನೂರು ಪಟ್ಟು ಹೆಚ್ಚಿನದನ್ನು ಹತ್ತೇ ವರ್ಷಗಳಲ್ಲಿ ನಾವು ಸಾಧಿಸಬಹುದು.”

ಈ ಸಂಭಾಷಣೆಯಲ್ಲಿ ಶೀಲನಿರ್ಮಾಣದ ಹಾಗೂ ಗ್ರಾಮಾಭಿವೃದ್ಧಿಯ ಮೂಲಕ ರಾಷ್ಟ್ರ ನಿರ್ಮಾಣದ ಕುರಿತಾಗಿ ಸ್ವಾಮೀಜಿ ವ್ಯಕ್ತಪಡಿಸಿದ ಅಭಿಪ್ರಾಯಗಳು ಹಾಗೂ ನೀಡಿದ ಸಲಹೆಗಳು ಎಷ್ಟು ಅಮೂಲ್ಯವಾಗಿವೆಯೆಂಬುದು ಅವುಗಳನ್ನು ಮತ್ತೆ ಮತ್ತೆ ಓದಿ, ಮೆಲುಕು ಹಾಕಿದಾಗ ತಿಳಿದುಬರುತ್ತದೆ. ಅಲ್ಲದೆ ಅವರು ನೀಡಿದ ಸಲಹೆಗಳು ಅಂದಿಗೆ ಮಾತ್ರವಲ್ಲದೆ ಇಂದಿಗೂ ಮುಂದಿಗೂ ಪ್ರಕೃತವಾಗಿವೆ ಎಂಬುದನ್ನೂ ಗಮನಿಸಬಹುದಾಗಿದೆ.

ಹೀಗೆ ದಿನಗಳುರುಳಿ ವಾರಗಳಾದವು. ಸ್ವಾಮೀಜಿಯ ಪರಿವ್ರಾಜಕ ಪ್ರವೃತ್ತಿ ಮತ್ತೊಮ್ಮೆ ಎಚ್ಚರಗೊಂಡಿತು. ಶಿಷ್ಯರು ಹಾಗೂ ಭಕ್ತರ ಒತ್ತಾಯಕ್ಕೆ ಮಣಿದು ಅವರು ಸುಮಾರು ಏಳು ವಾರಗಳಷ್ಟು ದೀರ್ಘಾವಧಿಯನ್ನು ಅಲ್ವರಿನಲ್ಲಿ ಕಳೆದಿದ್ದರು. ಇನ್ನು ಇಲ್ಲಿ ನಿಲ್ಲುವುದು ತರವಲ್ಲ ವೆಂದು ನಿರ್ಧರಿಸಿ ತಮ್ಮ ಶಿಷ್ಯರಿಗೆ ಹೇಳಿದರು, “ನಾನಿನ್ನು ಹೊರಟೆ; ಸಂನ್ಯಾಸಿಯಾದವನು ಯಾವಾಗಲೂ ಸಂಚರಿಸುತ್ತಿರಬೇಕು” ಎಂದು. ೧೮೯೧ರ ಮಾರ್ಚ್ ೨೮ರಂದು ಸ್ವಾಮೀಜಿ ಅಲ್ವರಿನಿಂದ ಹೊರಟರು. ಆ ಶಿಷ್ಯರಿಗೆ ಸ್ವಾಮೀಜಿ ತಮ್ಮನ್ನು ಬೀಳ್ಕೊಂಡು ಹೊರಡುತ್ತಾರೆ ಎಂಬುದರ ಕಲ್ಪನೆಯೇ ಅಸಹನೀಯವಾಗಿತ್ತು. ಅವರೆಲ್ಲ ಅಶ್ರುಭರಿತ ಕಂಗಳಿಂದ ಪ್ರಣಾಮ ಸಲ್ಲಿಸಿದಾಗ ಸ್ವಾಮೀಜಿಗೂ ಹೃದಯದುಂಬಿಬಂದಿತು. ಆದರೆ ಲೋಕಗುರುವಾದ ಅವರು ತಮ್ಮ ದಿವ್ಯ ಬೋಧನೆಗಳಿಂದ ಜಗತ್ತಿನ ಜನರಿಗೆ ಬೆಳಕು ನೀಡುತ್ತ ಮಾನವಕೋಟಿಗೆ ನೆರವು ನೀಡುತ್ತ ಭಗವದಿಚ್ಛೆಯಂತೆ ಮತ್ತು ಭಗವಂತನ ಶಕ್ತಿಯಿಂದಲೇ ಮುನ್ನಡೆಯುತ್ತಲೇ ಇರಬೇಕಾಗಿದೆ. ಆದ್ದರಿಂದ ಸ್ವಾಮೀಜಿ ತಮ್ಮ ಭಕ್ತರು-ವಿಶ್ವಾಸಿಗಳೆಲ್ಲರನ್ನೂ ಬಿಟ್ಟು ಏಕಾಕಿಯಾಗಿಯೇ ಹೊರ ಟರು. ಅವರು ತಮ್ಮ ಶಿಷ್ಯನೊಬ್ಬನ ಆಹ್ವಾನವನ್ನು ಮನ್ನಿಸಿ ಜೈಪುರದತ್ತ ಹೊರಟಿದ್ದರು.

ಅಲ್ವರಿನಿಂದ ಜೈಪುರಕ್ಕೆ ನೂರು ಮೈಲಿಗೂ ಹೆಚ್ಚಿನ ಅಂತರ. ಅದರಲ್ಲಿ ಮೊದಲು ಸುಮಾರು ನಲವತ್ತು ಮೈಲಿಗಳನ್ನು ಕಡಿದಾದ ಬೆಟ್ಟಗುಡ್ಡಗಳ ದಾರಿಯಾಗಿ, ಅದರಲ್ಲೂ ಹೆಚ್ಚಿನ ಭಾಗವನ್ನು ಕಾಲ್ನಡಿಗೆಯಲ್ಲೇ ಕ್ರಮಿಸಬೇಕಾಗಿತ್ತು. ಆದರೆ ರಾಜಸ್ಥಾನದ ಆ ಉರಿಯುವ ಬಿಸಿಲಿನಲ್ಲಿ ಬೆಟ್ಟಗುಡ್ಡಗಳ ನಿರ್ಜನ ಹಾದಿಯಲ್ಲಿ ಒಬ್ಬೊಬ್ಬರೇ ಹೋಗುವುದು ತುಂಬ ಅಪಾಯಕರ. ಆದ್ದರಿಂದ ಸ್ವಾಮೀಜಿಯ ಭಕ್ತರು, ಅವರು ಹೀಗೆ ಒಬ್ಬರೇ ಹೋಗಲು ಬಿಡದೆ, ಅಲ್ಲಿಂದ ೧೮ ಮೈಲಿ ದೂರದಲ್ಲಿರುವ ಪಾಂಡುಪೋಲಿನವರೆಗಾದರೂ ಕಮಾನು ಗಾಡಿಯಲ್ಲಿ ಪ್ರಯಾಣ ಮಾಡ ಬೇಕೆಂದು ಒತ್ತಾಯಿಸಿದರು. ಅಲ್ಲದೆ, ಕಡೆಯಪಕ್ಷ ೫ಂ-೬ಂ ಮೈಲಿಯವರೆಗಾದರೂ ಅವರನ್ನು ತಾವೂ ಹಿಂಬಾಲಿಸಿ ಬರುವುದಾಗಿ ಹೇಳಿದರು. ಇದಕ್ಕೆಲ್ಲ ಸ್ವಾಮೀಜಿ ಮೊದಮೊದಲು ಸಮ್ಮತಿಸ ದಿದ್ದರೂ, ಕಡೆಗೆ ಅವರ ವಿಶ್ವಾಸದ ಒತ್ತಾಯಕ್ಕೆ ಮಣಿದು, ಕೆಲವು ಭಕ್ತರೊಂದಿಗೆ ಎತ್ತಿನ ಗಾಡಿಯಲ್ಲಿ ಹೊರಟರು.

ಪಾಂಡುಪೋಲಿನಲ್ಲಿರುವ ಸುಪ್ರಸಿದ್ಧವಾದ ಹನುಮಂತನ ದೇವಸ್ಥಾನದಲ್ಲಿ ಸ್ವಾಮೀಜಿ ಮತ್ತು ಅವರ ಅನುಚರರು ಆ ರಾತ್ರಿಯನ್ನು ಕಳೆದರು. ಮರುದಿನ ಬೆಳಿಗ್ಗೆ ಅವರು ತಮ್ಮ ಪ್ರಯಾಣವನ್ನು ಮುಂದುವರಿಸಿ, ಕಾಲ್ನಡಿಗೆಯಲ್ಲಿ ಸಾಗಿದರು. ಕ್ರೂರ ಪ್ರಾಣಿಗಳಿಂದ ಕೂಡಿದ ಆ ಪರ್ವತಪ್ರದೇಶದ ದಾರಿಯಾಗಿ ಹದಿನಾರು ಮೈಲಿ ದೂರ ನಡೆದು, ಸಂಜೆಯ ವೇಳೆಗ ತಾಹ್ಲಾ ಎಂಬ ಹಳ್ಳಿಯನ್ನು ತಲುಪಿದರು. ಪ್ರಯಾಸಕರವೂ ಆತಂಕಕರವೂ ಆಗಬಹುದಾಗಿದ್ದ ಈ ಪ್ರಯಾಣವು, ಸ್ವಾಮೀಜಿಯ ದಿವ್ಯ ಸಾನ್ನಿಧ್ಯದಿಂದಾಗಿ ಆ ಶಿಷ್ಯರ ಪಾಲಿಗೆ ಅತ್ಯಂತ ಆನಂದದ ಅನುಭವವಾಗಿ ಪರಿಣಮಿಸಿತು. ದಾರಿಯುದ್ದಕ್ಕೂ ಸ್ವಾಮೀಜಿ, ಮನಸೂರೆಗೊಳ್ಳುವ, ಗಂಭೀರವೂ ಕೆಲವೊಮ್ಮೆ ಹಾಸ್ಯಭರಿತವೂ ಆದ ಕಥೆಗಳಿಂದ ತಮ್ಮ ಸಂಗಡಿಗರ ಹೃನ್ಮನಗಳನ್ನು ರಂಜಿಸಿದರು.

ತಾಹ್ಲಾದಲ್ಲಿ ಆ ರಾತ್ರಿ ಸ್ವಾಮೀಜಿ ಮತ್ತು ಅವರ ಸಂಗಡಿಗರು ಅಲ್ಲಿನ ನೀಲಕಂಠ ಮಹಾ ದೇವನ ದೇವಸ್ಥಾನದಲ್ಲಿ ಉಳಿದುಕೊಂಡರು. ಇಲ್ಲಿ ಸ್ವಾಮೀಜಿ, ಮಹೇಶ್ವರನಿಗೆ‘ನೀಲಕಂಠ’ ಎಂಬ ಹೆಸರು ಬರಲು ಕಾರಣವಾದ ಪುರಾಣಪ್ರಸಿದ್ಧ ‘ಸಮುದ್ರ ಮಂಥನ’ದ ಕಥೆಯನ್ನು ವಿವರಿಸಿ, ಅದಕ್ಕೆ ತಮ್ಮದೇ ಆದ ಅತ್ಯಂತ ಸ್ವಾರಸ್ಯಕರವೂ ವಿಶಿಷ್ಟವೂ ಆದ ವಿವರಣೆಯನ್ನು ಕೊಡುತ್ತಾರೆ.

ವಿಷ್ಣುಪುರಾಣದಲ್ಲಿ ಸಮುದ್ರ ಮಂಥನದ ಕಥೆ ಬರುತ್ತದೆ. ಹಿಂದೆ ದೇವಾಸುರರು ಅಮೃತ ವನ್ನು ಬಯಸಿ, ಕ್ಷೀರಸಾಗರವನ್ನು ಮಂಥನ ಮಾಡಿದರು. ಆಗ ಅಮೃತವು ಉದ್ಭವಿಸುವ ಮೊದಲು, ಕಾಮಧೇನು, ಪಾರಿಜಾತ, ಚಂದ್ರ, ಐರಾವತ, ಅಪ್ಸರಸ್ತ್ರೀಯರು ಮೊದಲಾದ ಹಲ ವಾರು ಅದ್ಭುತ-ಸುಂದರ ವಸ್ತುಗಳು ಹುಟ್ಟಿ ಬಂದುವು. ಸುರರೂ ಅಸುರರೂ ಇವುಗಳಿಗಾಗಿ ಬಡಿದಾಡಿ, ಕೈಗೆ ಸಿಕ್ಕಿದ್ದನ್ನು ಬಾಚಿಕೊಂಡರು. ಆದರೆ ಯೋಗೀಶ್ವರನಾದ ಶಿವನು ಮಾತ್ರ ಈ ಯಾವ ವಸ್ತುಗಳನ್ನೂ ಬಯಸದೆ, ಗಾಢಸಮಾಧಿಯಲ್ಲಿ ಮುಳುಗಿ ತನ್ನ ಆನಂದದಲ್ಲಿ ತಾನಿದ್ದು ಬಿಟ್ಟ. ಇತ್ತ ಸಮುದ್ರ ಮಂಥನವು ಮುಂದುವರಿದಂತೆ ಹಾಲಾಹಲವೆಂಬ ಘೋರ ವಿಷವು ಉದ್ಧವಿಸಿತು. ಇದರ ಗಂಧವೇ ಮೃತ್ಯುಸ್ವರೂಪವಾಗಿದ್ದು, ಅದು ತ್ರಿಲೋಕಗಳಲ್ಲಿಯೂ ವ್ಯಾಪಿಸಿ ಕೊಳ್ಳಲಾರಂಭಿಸಿತು. ಕಂಗೆಟ್ಟ ದೇವಾಸುರರು ವಿಷದಿಂದ ಕಾಪಾಡುವಂತೆ ವಿಷ್ಣುವಿನ ಮೊರೆ ಹೊಕ್ಕಾಗ ಅವನು ಶಿವನನ್ನು ಪ್ರಾರ್ಥಿಸಿಕೊಳ್ಳುವಂತೆ ಸೂಚಿಸಿದ. ಆಗ ಅವರೆಲ್ಲ ಹೋಗಿ ಧ್ಯಾನಾನಂದಲ್ಲಿ ಮುಳುಗಿದ್ದ ಶಿವನನ್ನು ಪ್ರಾರ್ಥಿಸಿಕೊಂಡರು. ಧ್ಯಾನದಿಂದೆದ್ದ ಮಹೇಶ್ವರ ಆ ಘೋರ ವಿಷವನ್ನು ಅಂಗೈಯಲ್ಲಿ ಸಂಗ್ರಹಿಸಿ ಕುಡಿದುಬಿಟ್ಟ. ಆ ವಿಷವು ಅವನ ಕಂಠದಲ್ಲೇ ನಿಂತುಬಿಟ್ಟು, ಕಂಠವು ನೀಲವರ್ಣಕ್ಕೆ ತಿರುಗಿದ್ದರಿಂದ ಅವನಿಗೆ ನೀಲಕಂಠನೆಂಬ ಹೆಸರಾಯಿತು. ಹೀಗೆ ಯೋಗೀಶ್ವರನು ವಿಷದ ಭಯದಿಂದ ಜಗತ್ತನ್ನು ಕಾಪಾಡಿದ ಮೇಲೆ ಅಮೃತವೂ ಉತ್ಪನ್ನವಾಯಿತು.

ಈ ಕಥೆಗೆ ಸ್ವಾಮೀಜಿ ಹೀಗೆ ವಿವರಣೆ ನೀಡುತ್ತಾರೆ–ಲೌಕಿಕ ಪ್ರಜ್ಞಾವಸ್ಥೆಯಲ್ಲಿರುವ ಸಕಲ ಜೀವಿಗಳೂ ಮಾಯೆಯ ಸಾಗರವಾದ ಇಹ ಜೀವನವನ್ನು ಮಂಥನ ಮಾಡಿ, ಎಂದರೆ ಸುಖವನ್ನು ಅರಸಿ, ಇಂದ್ರಿಯ ಭೋಗವನ್ನು ನೀಡುವ ಹಲವಾರು ವಸ್ತುಗಳನ್ನು ಪಡೆಯುತ್ತಾರೆ. ಆದರೆ ಕಡೆ ಯಲ್ಲಿ ಸರ್ವವನ್ನೂ ನುಂಗಿ ಹಾಕುವ ವಿಷ-ಮೃತ್ಯು–ಉದ್ಭವಿಸಿ ನಿಂತಾಗ ದಿಕ್ಕುಗಾಣದೆ ತತ್ತರಿಸಿ ಹೋಗುತ್ತಾರೆ. ಆದರೆ ಸಂನ್ಯಾಸಿಯಾದವನು ಮಾತ್ರ, ಮಹೇಶ್ವರನಂತೆ, ಮಾಯೆಯು ಮುಂದೆ ಇಡುವ ವಸ್ತುಗಳಾವುದರಿಂದಲೂ ವಿಚಲಿತನಾಗದೆ ಆತ್ಮಾನಂದದಲ್ಲಿ ಮಗ್ನನಾಗಿರುತ್ತಾನೆ. ಮೃತ್ಯುಭೀತಿಪೀಡಿತರಾದ ಜನರು ದಿಕ್ಕುಗಾಣದೆ ಸಂನ್ಯಾಸಿಯ ಬಳಿಗೆ ಓಡಿಹೋಗಿ ಶರಣಾದಾಗ, ಅವನು ಮಾಯೆಯನ್ನು ಪರಿಹರಿಸಿ ಮೃತ್ಯುಭೀತಿಯಿಂದ ಅವರನ್ನು ವಿಮುಕ್ತರನ್ನಾಗಿಸುತ್ತಾನೆ. ಅಲ್ಲದೆ, ಭಗವತ್ಸಾಕ್ಷಾತ್ಕಾರವನ್ನು ಸಾಧಿಸಿಕೊಂಡವನಿಗೆ ಮರಣಭಯವಿಲ್ಲವೆಂಬುದನ್ನು ತೋರಿಸಿ ಕೊಡುತ್ತಾನೆ. ಸ್ವಾಮೀಜಿ ನೀಡುವ ಈ ವಿವರಣೆಯಲ್ಲಿ, ತನ್ನ ಸ್ವಂತ ಮುಕ್ತಿಯೊಂದಿಗೆ ಜಗತ್ತಿನ ಹಿತವೂ ಸಂನ್ಯಾಸಿಯ ಆದರ್ಶವಾಗಿರುತ್ತದೆ, ಆಗಿರಬೇಕು–ಎಂಬುದರ ಸೂಚನೆಯನ್ನೂ ಗಮನಿಸಬಹುದು.

ಮರುದಿನ ಬೆಳಿಗ್ಗೆ ಸ್ವಾಮೀಜಿ ಮತ್ತವರ ಸಂಗಡಿಗರು ತಾಹ್ಲಾವನ್ನು ಬಿಟ್ಟು ಹೊರಟು, ಕಾಲ್ನಡಿಗೆಯಲ್ಲೇ ಸುಮಾರು ೧೮ ಮೈಲಿ ದೂರವನ್ನು ಕ್ರಮಿಸಿ ನಾರಾಯಣಿ ಎಂಬ ಹಳ್ಳಿಯನ್ನು ಮುಟ್ಟಿದರು. ಇಲ್ಲಿ ಸ್ವಾಮೀಜಿ, ತಮ್ಮನ್ನು ಹಿಂಬಾಲಿಸಿ ಬಂದಿದ್ದ ಶಿಷ್ಯರು-ಭಕ್ತರನ್ನು ಬೀಳ್ಕೊಂಡು ಏಕಾಂಗಿಯಾಗಿ ಮುನ್ನಡೆದರು. ಇಲ್ಲಿಂದ ಮತ್ತೆ ಹದಿನಾರು ಮೈಲಿಗಳಷ್ಟು ದೂರವನ್ನು ಕಾಲ್ನಡಿಗೆಯಲ್ಲೇ ಸವೆಸಿ ಬಾಸ್ವಾ ಎಂಬ ಹಳ್ಳಿಗೆ ಬಂದರು. ಇಲ್ಲಿ ಅವರು ಜೈಪುರಕ್ಕೆ ಹೋಗುವ ಟ್ರೈನು ಹತ್ತಿದರು. ಸ್ವಾಮೀಜಿಯನ್ನು ಆಹ್ವಾನಿಸಿದ್ದ ಶಿಷ್ಯ, ಬಂದಿಕುಯಿ ಎಂಬಲ್ಲಿ ಅವರನ್ನು ಕೂಡಿಕೊಂಡು ಜೈಪುರಕ್ಕೆ ಕರೆದೊಯ್ದ.

ಜೈಪುರದಲ್ಲಿ ಈ ಶಿಷ್ಯ ಸ್ವಾಮೀಜಿಯನ್ನು, ಅವರ ಭಾವಚಿತ್ರ ತೆಗೆದುಕೊಳ್ಳಲು ಅನುಮತಿ ಬೇಡಿದ. ಸ್ವಾಮೀಜಿಗೆ ಇಷ್ಟವಿಲ್ಲದಿದ್ದರೂ ಅವನ ಒತ್ತಾಯಕ್ಕೆ ಮಣಿದು ಭಾವಚಿತ್ರವನ್ನು ತೆಗೆಸಿಕೊಂಡರು. ಇದು ಪರಿವ್ರಾಜಕರಾಗಿ ಅವರು ತೆಗೆಸಿಕೊಂಡ ಮೊದಲ ಭಾವಚಿತ್ರ. ಅವರ ಕಿರಿಯ ಸೋದರನಾದ ಮಹೇಂದ್ರನಾಥದತ್ತನಿಗೆ ಅದರ ಒಂದು ಪ್ರತಿ ಅಂಚೆಯ ಮೂಲಕ ತಲುಪಿತು. ಆ ಲಕೋಟೆಯನ್ನು ಕಳಿಸಿದ್ದವರ ಹೆಸರಾಗಲಿ ವಿಳಾಸವಾಗಲಿ ಅದರ ಮೇಲೆ ಬರೆದಿರ ಲಿಲ್ಲ. ಆದರೆ ಅದನ್ನು ಕಳಿಸಿದ್ದವರು ಸ್ವಾಮೀಜಿಯ ಇಚ್ಛೆಯ ಮೇರೆಗೇ ಕಳಿಸಿದ್ದರೆಂಬುದು ಸ್ಪಷ್ಟವಾಗಿತ್ತು. ಈ ಚಿತ್ರದಲ್ಲಿ ಅವರು ಉದ್ದನೆಯ ನಿಲುವಂಗಿಯನ್ನು ಧರಿಸಿದ್ದು, ಮೊದಲಿಗಿಂತ ಹೆಚ್ಚು ಆರೋಗ್ಯದಿಂದಿರುವಂತೆ ಕಾಣುತ್ತಿದ್ದರು. ಇದನ್ನು ಕಂಡು ಬಾರಾನಾಗೋರ್ ಮಠದ ಗರುಭಾಯಿಗಳಿಗೆ ತುಂಬ ಸಂತೋಷ-ಸಮಾಧಾನಗಳುಂಟಾದವು. ಇದಕ್ಕೆ ಕೆಲಕಾಲದ ಹಿಂದೆ ಸ್ವಾಮಿ ಶಾರದಾನಂದರ ಹೆಸರಿಗೆ ಬರೆದಿದ್ದ ಪತ್ರವೊಂದು ಮಠಕ್ಕೆ ಬಂದಿತ್ತು. ಆ ಪತ್ರಕ್ಕೂ ಸಹಿ ಇರಲಿಲ್ಲ. ಆದರೆ ಬರವಣಿಗೆಯಿಂದ ಅದು ತಮ್ಮ ನರೇಂದ್ರನದೇ ಎಂಬುದು ಸ್ಪಷ್ಟವಾಗಿತ್ತು. ಇದನ್ನು ಜೈಪುರದಿಂದ ಬರೆಯಲಾಗಿದ್ದು, ಯಾವುದೋ ಒಂದು ಔಷಧಿಯನ್ನು ಒಂದು ವಿಳಾಸಕ್ಕೆ ಕಳಿಸಿಕೊಡುವಂತೆ ಕೋರಲಾಗಿತ್ತು. ಹೀಗಾಗಿ ಸ್ವಾಮೀಜಿ ರಾಜಸ್ಥಾನದಲ್ಲಿರುವರೆಂಬುದು ಅವರ ಸೋದರ ಸಂನ್ಯಾಸಿಗಳಿಗೆ ತಿಳಿದುಬಂದಿತ್ತು.

ಜೈಪುರದಲ್ಲಿ ಒಮ್ಮೆ ಸ್ವಾಮೀಜಿ ಕೆಲವು ಭಕ್ತರೊಂದಿಗೆ ಮಾತನಾಡುತ್ತ ಕುಳಿತಿದ್ದಾಗ ಅಲ್ಲಿ ಇದ್ದಕ್ಕಿದ್ದಂತೆ ಸ್ವಾಮಿ ಅಖಂಡಾನಂದರು ಕಾಣಿಸಿಕೊಂಡರು. ತಮ್ಮ ಪ್ರಿಯ ಗುರುಭಾಯಿಯನ್ನು ಕಂಡು ಸ್ವಾಮೀಜಿಗೆ ಅಚ್ಚರಿಯೊಂದಿಗೆ ಆನಂದವೂ ಆಯಿತು. ಆದರೆ ಅದನ್ನು ತೋರ್ಪಡಿಸಿ ಕೊಳ್ಳದೆ ಕೋಪದ ನಟನೆ ಮಾಡುತ್ತ ತಮ್ಮನ್ನು ಹಿಂಬಾಲಿಸಿದ್ದೇಕೆಂದು ಗದರಿಸಿದರು. ಆದರೆ ಅಖಂಡಾನಂದರು ಅದನ್ನು ಗಂಭೀರವಾಗಿ ಪರಿಗಣಿಸಲಿಲ್ಲ. ಅಷ್ಟೇ ಅಲ್ಲ ತಾವೇ ಹಿಂದೊಮ್ಮೆ ಸ್ವಾಮೀಜಿಗೆ “ನೀನು ಪಾತಾಳ ಲೋಕಕ್ಕೇ ಹೋದರೂ ನಾನು ನಿನ್ನನ್ನು ಹುಡುಕದೆ ಬಿಡುವವನಲ್ಲ” ಎಂದು ಹೇಳಿದ್ದಂತೆ, ಅವರನ್ನು ಹಿಂಬಾಲಿಸುವುದನ್ನು ಬಿಡಲೂ ಇಲ್ಲ.

ಸ್ವಾಮೀಜಿ ಜೈಪುರದಲ್ಲಿ ಸುಮಾರು ಎರಡು ವಾರ ಇದ್ದರು. ಇಲ್ಲಿ ಅವರಿಗೆ ಖ್ಯಾತ ಸಂಸ್ಕೃತ ವ್ಯಾಕರಣ ಪಂಡಿತನೊಬ್ಬನ ಪರಿಚಯವಾಯಿತು. ತಮ್ಮ ಗುರುಭಾಯಿಗಳು ಸಂಸ್ಕೃತವನ್ನು ಆಳವಾಗಿ ಅಧ್ಯಯನ ಮಾಡಬೇಕೆಂದು ಅವರು ಇಚ್ಛಿಸಿದುದನ್ನು ನಾವು ಈ ಹಿಂದೆಯೇ ನೋಡಿ ದ್ದೇವೆ. ಈಗ ಈ ಪಂಡಿತನ ಸಹಾಯದಿಂದ ತಾವೂ ಪಾಣನಿಯ ‘ಅಷ್ಟಾಧ್ಯಾಯಿ’ ಎಂಬ ವ್ಯಾಕ ರಣ ಗ್ರಂಥವನ್ನು ಅಭ್ಯಾಸ ಮಾಡಬೇಕೆಂದು ಸ್ವಾಮೀಜಿ ನಿಶ್ಚಯಿಸಿದರು. ಪಂಡಿತನೂ ವ್ಯಾಕರಣ ವನ್ನು ಬೋಧಿಸಲು ಒಪ್ಪಿದ. ಆದರೆ ಆತ ದೊಡ್ಡ ವಿದ್ವಾಂಸನಾದರೂ ಅವನಿಗೆ ಬೋಧನೆಯ ಕಲೆ ಸಿದ್ಧಿಸಿರಲಿಲ್ಲ. ಅವನು ಈ ಗ್ರಂಥದ ಮೊದಲ ಸೂತ್ರದ ಭಾಷ್ಯವನ್ನು ಸ್ವಾಮೀಜಿಗೆ ಅರ್ಥ ಪಡಿಸಲು ಮೂರುದಿನಗಳ ಕಾಲ ಪ್ರಯತ್ನಿಸಿದ. ಕಡೆಗೂ ಆತ ಅದರಲ್ಲಿ ಯಶಸ್ವಿಯಾಗಲಿಲ್ಲ. ಇದನ್ನು ಕಂಡು ಅವನಿಗೇ ಬೇಸರ ಹುಟ್ಟಿ ಹೇಳಿದ, “ಸ್ವಾಮೀಜಿ, ನನ್ನ ಬಳಿಯಲ್ಲಿ ಅಧ್ಯಯನ ಮಾಡುವುದರಿಂದ ನಿಮಗೇನೂ ಪ್ರಯೋಜನವಾಗುತ್ತಿಲ್ಲ ಎನ್ನಿಸುತ್ತಿದೆ ನನಗೆ. ಮೂರು ದಿನಗಳಾ ದರೂ ಒಂದು ಸೂತ್ರವನ್ನೂ ನಿಮಗೆ ಅರ್ಥಪಡಿಸಲು ನನ್ನಿಂದ ಸಾಧ್ಯವಾಗಿಲ್ಲ...” ಈಗ ಅವರು, ಆ ಸೂತ್ರದ ಭಾಷ್ಯವನ್ನು ತಾವೇ ಅಧ್ಯಯನ ಮಾಡಿ ಕರಗತಗೊಳಿಸಿಕೊಳ್ಳುವುದೆಂದು ನಿಶ್ಚಯಿಸಿ ದರು. ಬಳಿಕ ಆ ಗ್ರಂಥವನ್ನು ಹಿಡಿದು ಅದರಲ್ಲೇ ಏಕಾಗ್ರಚಿತ್ತರಾಗಿ ಕುಳಿತುಬಿಟ್ಟರು. ಆ ಪಂಡಿತನಿಗೆ ಮೂರು ದಿನಗಳಲ್ಲಿ ಕಲಿಸಲಾಗದಿದ್ದುದನ್ನು ಮೂರೇ ಗಂಟೆಗಳಲ್ಲಿ ಸ್ವಪ್ರಯತ್ನ ದಿಂದ ಹೃದ್ಗತಗೊಳಿಸಿಕೊಂಡರು! ಬಳಿಕ ಆ ಪಂಡಿತನನ್ನು ಕಂಡು ಭಾಷ್ಯಾರ್ಥವನ್ನು ಒಪ್ಪಿಸಿ ದರು. ಆ ಪಂಡಿತನಿಗೆ ಪರಮಾಶ್ಚರ್ಯ. ಅನಂತರ ಅವರು ಸೂತ್ರದಿಂದ ಸೂತ್ರಕ್ಕೆ, ಅಧ್ಯಾಯ ದಿಂದ ಅಧ್ಯಾಯಕ್ಕೆ ಮುಂದುವರಿದು ಇಡೀ ಗ್ರಂಥವನ್ನೇ ಸ್ವಾಧೀನಪಡಿಸಿಕೊಂಡುಬಿಟ್ಟರು. ಮುಂದಿನ ದಿನಗಳಲ್ಲಿ ಈ ಘಟನೆಯನ್ನು ಪ್ರಸ್ತಾಪಿಸಿ ಅವರೆನ್ನುತ್ತಿದ್ದರು, “ಮನುಷ್ಯನಿಗೆ ತೀವ್ರ ಹಂಬಲವೊಂದಿದ್ದರೆ ಏನನ್ನಾದರೂ ಸಾಧಿಸಬಹುದು–ಬೇಕಾದರೆ ಪರ್ವತಗಳನ್ನೇ ಕುಟ್ಟಿ ಪುಡಿ ಪುಡಿ ಮಾಡಿಬಿಡಬಹುದು.”

ಜೈಪುರದಲ್ಲಿ ಸ್ವಾಮೀಜಿಗೆ ರಾಜ್ಯದ ಪ್ರಧಾನ ದಂಡನಾಯಕನಾದ ಸರ್ದಾರ್ ಹರಿಸಿಂಗ್ ಎಂಬವನೊಂದಿಗೆ ಆತ್ಮೀಯ ಪರಿಚಯ ಬೆಳೆಯಿತು. ಅವರು ಕೆಲವು ದಿನ ಅವನ ಮನೆಯಲ್ಲಿ ಉಳಿದುಕೊಂಡರು. ಇಬ್ಬರೂ ಆಧ್ಯಾತ್ಮಿಕ ವಿಚಾರಗಳ ಹಾಗೂ ಶಾಸ್ತ್ರವಿಚಾರಗಳ ಕುರಿತಾಗಿ ಸಂಭಾಷಣೆ ನಡೆಸುತ್ತಿದ್ದರು. ಒಂದು ದಿನ ಮೂರ್ತಿಪೂಜೆಯ ಪ್ರಸ್ತಾಪ ಬಂದಿತು. ಹರಿಸಿಂಗನಿಗೆ ಅದ್ವೈತ ವೇದಾಂತತತ್ತ್ವಗಳಲ್ಲಿ ತೀವ್ರ ಶ್ರದ್ಧೆಯಿತ್ತು. ಆದರೆ ಮೂರ್ತಿಪೂಜೆಯಲ್ಲಿ ಮಾತ್ರ ಕಿಂಚಿತ್ತೂ ನಂಬಿಕೆಯಿರಲಿಲ್ಲ. ಗಂಟೆಗಟ್ಟಲೆ ಚರ್ಚೆ ನಡೆಸಿಯೂ ಅವನಿಗೆ ಮೂರ್ತಿಪೂಜೆಯ ಮಹತ್ವವನ್ನು ಮನಗಾಣಿಸಲು ಸ್ವಾಮೀಜಿಗೆ ಸಾಧ್ಯವಾಗಲಿಲ್ಲ. ಆ ದಿನ ಸಂಜೆ ಅವರು ಹರಿಸಿಂಗ ನೊಂದಿಗೆ ಹೊರಗೆ ತಿರುಗಾಡಿಕೊಂಡುಬರಲು ಹೊರಟರು. ದಾರಿಯಲ್ಲಿ ಶ್ರೀಕೃಷ್ಣನ ಮೆರವಣಿಗೆ ಯೊಂದು ಬರುತ್ತಿತ್ತು. ಕೆಲವರು ಶ್ರೀಕೃಷ್ಣನ ವಿಗ್ರಹದ ಮುಂದೆ ಆನಂದದಿಂದ ಭಜನೆ ಮಾಡುತ್ತಿದ್ದರು. ಸ್ವಾಮೀಜಿ ಮತ್ತು ಹರಿಸಿಂಗ್​–ಇಬ್ಬರೂ ಆ ದೃಶ್ಯವನ್ನು ನೋಡುತ್ತ ಅಲ್ಲೇ ನಿಂತರು. ಆಗ ಸ್ವಾಮೀಜಿ ಇದ್ದಕ್ಕಿದ್ದಂತೆ ಹರಿಸಿಂಗನನ್ನು ಸ್ಪರ್ಶಮಾಡಿ, “ನೋಡು, ನೋಡಲ್ಲಿ! ಸಾಕ್ಷಾತ್ ಭಗವಂತ!” ಎಂದು ಉದ್ಗರಿಸಿದರು. ಹರಿಸಿಂಗನ ಕಣ್ಣು ಶ್ರೀಕೃಷ್ಣನ ವಿಗ್ರಹದ ಮೇಲೆ ಬಿದ್ದದ್ದೇ ತಡ, ಅವನು ಭಾವಾತಿಶಯದಿಂದ ಮೈಮರೆತು ನಿಂತುಬಿಟ್ಟ. ಅವನ ಕಂಗಳಿಂದ ಆನಂದದ ಆಶ್ರುಧಾರೆ ಸುರಿಯತೊಡಗಿತು! ಸ್ವಲ್ಪ ಹೊತ್ತಿತನಲ್ಲಿ ಸಾಧಾರಣ ಸ್ಥಿತಿಗೆ ಮರಳಿದಾಗ ಅವನು ಹೇಳುತ್ತಾನೆ, “ಸ್ವಾಮೀಜಿ, ನೀವು ನನ್ನ ಒಳಗಣ್ಣನ್ನು ತೆರೆಸಿದಿರಿ. ಗಂಟೆಗಟ್ಟಲೆ ಚರ್ಚಿಸಿ ದರೂ ಅರ್ಥಮಾಡಿಕೊಳ್ಳಲು ಸಾಧ್ಯವಾಗದಿದ್ದುದು ಕೇವಲ ನಿಮ್ಮ ಸ್ಪರ್ಶಮಾತ್ರದಿಂದ ಗ್ರಾಹ್ಯ ವಾಯಿತು. ನಿಜಕ್ಕೂ ನಾನು ಆ ವಿಗ್ರಹದಲ್ಲಿ ಸಾಕ್ಷಾತ್ ಭಗವಂತನನ್ನೇ ಕಂಡೆ!”

ಹಿಂದೆ ಶ್ರೀರಾಮಕೃಷ್ಣರು ನರೇಂದ್ರನಿಗೆ ತಮ್ಮ ಸ್ಪರ್ಶಮಾತ್ರದಿಂದ ಹಲವಾರು ಅದ್ಭುತ ಅಲೌಕಿಕ ಅನುಭವಗಳನ್ನು ಮಾಡಿಸಿಕೊಟ್ಟಿದ್ದರು. ಇಂದು ಅದೇ ನರೇಂದ್ರ ಸ್ವಾಮಿ ವಿವೇಕಾ ನಂದರಾಗಿ ಮತ್ತೊಬ್ಬನಿಗೆ ತಮ್ಮ ಸ್ಪರ್ಶದಿಂದ ಭಗವದ್ದರ್ಶನ ಮಾಡಿಸುತ್ತಿದ್ದಾನೆ. ಈಶ್ವರ ಕೋಟಿಗಳಿಂದ ಮಾತ್ರ ಸಾಧ್ಯವಾಗುವಂತಹ ಕೆಲಸ ಇದು. ಸ್ವಾಮೀಜಿ ಪರಿವ್ರಾಜಕರಾಗಿ ಹೊರಟ ಸಂದರ್ಭದಲ್ಲಿ ‘ಸ್ಪರ್ಶಮಾತ್ರದಿಂದ ಇತರರನ್ನು ಪರಿವರ್ತಿಸಬಲ್ಲ ಶಕ್ತಿಯನ್ನು ಪಡೆದುಕೊಳ್ಳು ವವರೆಗೆ ಹಿಂದಿರುಗುವುದಿಲ್ಲ’ ಎಂಬ ನಿಶ್ಚಯ ಮಾಡಿದ್ದನ್ನು ನೋಡಿದ್ದೇವೆ. ಇಂದು ಅವರಲ್ಲಿ ಆ ಶಕ್ತಿ ವ್ಯಕ್ತವಾಗುವುದನ್ನೂ ಕಾಣುತ್ತಿದ್ದೇವೆ. ಆದರೆ ಅವರು ತಮ್ಮ ಪರಿವ್ರಾಜಕ ಜೀವನದಲ್ಲಿ ಸಾಧಿಸಬೇಕಾದ ಮಹಾ ಆದರ್ಶ ಇದಲ್ಲ; ಇದು ಕೇವಲ ಒಂದು ಅಂಶ ಮಾತ್ರ. ಅಲ್ಲದೆ ಹೀಗೆ ಸ್ಪರ್ಶಮಾತ್ರದಿಂದ-ಅಷ್ಟೇಕೆ, ಕೇವಲ ತಮ್ಮ ಮಾತುಗಳಿಂದ–ಇತರರ ಭಾವಲಹರಿಯನ್ನು ತಮ್ಮ ಇಚ್ಛಾನುಸಾರ ನಿಯಂತ್ರಿಸಬಲ್ಲ ಶಕ್ತಿ ಅವರಲ್ಲಿದ್ದರೂ ಅವರು ಅದನ್ನು ಪ್ರಯೋಗಿಸಿದ್ದು ತೀರ ಅಪರೂಪ. ಈ ಶಕ್ತಿಯು ಕೆಲವೊಮ್ಮೆ ತಾನೇತಾನಾಗಿ ಹೊರಹರಿಯಲಾರಂಭಿಸುತ್ತಿತ್ತು. ಆಗ ಅದನ್ನು ಅವರು ಪ್ರಯತ್ನಪೂರ್ವಕವಾಗಿ ತಡೆಗಟ್ಟುತ್ತಿದ್ದರು. ಏಕೆಂದರೆ, ಪ್ರತಿಯೊಬ್ಬನೂ ಶಿಸ್ತುಬದ್ಧ ಸಾಧನೆಯಿಂದ ಕೂಡಿದ ಸ್ವಪ್ರಯತ್ನದಿಂದಲೇ ಮುಂದುವರಿಯಬೇಕು ಎಂಬುದು ಅವರ ಅಭಿಮತ.

ಒಂದು ದಿನ ಸ್ವಾಮೀಜಿ ತಮ್ಮ ಅನೇಕ ಭಕ್ತರೊಂದಿಗೆ ಕುಳಿತು ಆಧ್ಯಾತ್ಮಿಕ ವಿಷಯವಾಗಿ ಮಾತನಾಡುತ್ತಿದ್ದರು. ಆಗ ಅಲ್ಲಿಗೆ ಪಂಡಿತ ಸೂರಜ್​ನಾರಾಯಣ್ ಎಂಬೊಬ್ಬ ವಿದ್ವಾಂಸರು ಬಂದರು. ಇವರು ತಮ್ಮ ವಿದ್ವತ್ತಿಗಾಗಿ ಆ ಪ್ರಾಂತದಲ್ಲೇ ಹೆಸರುವಾಸಿಯಾದವರು. ಈ ಪಂಡಿ ತರು ಮಾತಿನ ಸಂದರ್ಭದಲ್ಲಿ ಸ್ವಾಮೀಜಿಯ ವಾದಸರಣಿಯನ್ನು ಹಿಡಿದು ಹೇಳಿದರು: “ಸ್ವಾಮೀಜಿ, ನಾನೊಬ್ಬ ವೇದಾಂತಿ. ಹಿಂದೂ ಪುರಾಣಗಳು ಹೇಳುವ ಅವತಾರಗಳಲ್ಲಿ ದೈವತ್ವವು ವಿಶೇಷವಾಗಿ ಪ್ರಕಟಗೊಂಡಿರುತ್ತದೆ ಎಂಬುದನ್ನು ನಾನು ಒಪ್ಪುವುದಿಲ್ಲ. ನಾವೆಲ್ಲರೂ ಬ್ರಹ್ಮವೇ. ಹೀಗಿರುವಾಗ, ಪುರಾಣಗಳು ಹೇಳುವ ಅವತಾರಗಳಿಗೂ ನನಗೂ ಏನು ವ್ಯತ್ಯಾಸ?” ವೇದಾಂತ ಸಾರುತ್ತದೆ–‘ಸರ್ವಂ ಖಲ್ವಿದಂ ಬ್ರಹ್ಮ’ ‘ಏನೇನಿದೆಯೋ ಅದೆಲ್ಲವೂ ಬ್ರಹ್ಮವೇ ಅಲ್ಲವೆ?’ಎಂದು. ಇದು ವೇದಾಂತ ಸಾರುವ ಅತ್ಯುನ್ನತ ಸತ್ಯ. ಇದನ್ನು ಸಾಕ್ಷಾತ್ಕರಿಸಿಕೊಂಡವರಿಗೆ ಮಾತ್ರ ಈ ಮಾತು ಅರ್ಥವಾಗುತ್ತದೆ. ಅಲ್ಲಿಯವರೆಗೂ ‘ಅಹಂ ಬ್ರಹ್ಮಾಸ್ಮಿ’ ಎಂದು ಬಾಯಲ್ಲಿ ಹೇಳಬಹುದಷ್ಟೆ. ಆದರೆ ಶಾಸ್ತ್ರಗಳನ್ನು ಓದಿ ತಿಳಿದ ಪಂಡಿತರಿಗೆ ಈ ಮಾತುಗಳೆಲ್ಲ ಲೀಲಾಜಾಲ. ಇಂಥ ಮಾತುಗಳ ಆಧಾರದ ಮೇಲೆಯೇ ಆ ಪಂಡಿತರು ತಮ್ಮ ಕುತರ್ಕವನ್ನು ಮುಂದಿಟ್ಟದ್ದು. ಅವರ ಪ್ರಶ್ನೆಗೆ ಸ್ವಾಮೀಜಿ ಶಾಂತವಾಗಿಯೇ ಉತ್ತರಿಸುತ್ತಾರೆ, “ಹೌದು ಹೌದು; ನೀವು ಹೇಳು ವುದು ನಿಜ. ಹಿಂದೂಗಳು ಮೀನು, ಆಮೆ, ಹಂದಿಗಳನ್ನು ಅವತಾರವೆಂದು ಪರಿಗಣಿಸುತ್ತಾರೆ. ನೀವು ಹೇಳುತ್ತಿದ್ದೀರಿ–ನೀವೂ ಒಂದು ಅವತಾರ ಎಂದು. ಹಾಗಾದರೆ ಈ ಅವತಾರಗಳ ಪೈಕಿ ನೀವು ಯಾವುದೆಂದು ಭಾವಿಸುತ್ತೀರಿ?” ಈ ಉತ್ತರದಿಂದ ದೊಡ್ಡದೊಂದು ನಗೆಯ ಅಲೆ ಯೆದ್ದಿತು. ಪಂಡಿತರು ಪುನಃ ಬಾಯಿಬಿಚ್ಚಲಿಲ್ಲ.

ಹೀಗೆ ಜೈಪುರದಲ್ಲಿ ಕೆಲದಿನಗಳನ್ನು ಕಳೆದ ಸ್ವಾಮೀಜಿ ಅಜ್ಮೀರದ ಕಡೆಗೆ ಹೊರಟರು. ದೆಹಲಿ-ಆಗ್ರಾಗಳಂತೆ ಅಜ್ಮೀರ್ ಕೂಡ ಮೊಘಲ್ ರಾಜವಂಶಗಳ ಆಳ್ವಿಕೆಯ ನೆನಹುಗಳನ್ನು ತುಂಬಿಕೊಂಡಿರುವ ನಗರ. ಇಲ್ಲಿ ಸ್ವಾಮೀಜಿ ಅಕ್ಬರನ ಅರಮನೆಯನ್ನೂ ಹಿಂದೂ-ಮುಸಲ್ಮಾನ ಇಬ್ಬರಿಗೂ ಪವಿತ್ರ ಸ್ಥಳವಾದ ಮೈನುದ್ದೀನ್ ಚಿಸ್ತೀ ಎಂಬ ಮುಸಲ್ಮಾನ ಸಂತ ಸಮಾಧಿಯನ್ನೂ ವೀಕ್ಷಿಸಿದರು. ಅಲ್ಲದೆ, ಅಜ್ಮೀರದ ಸಮೀಪದಲ್ಲಿ, ಪುಷ್ಕರ-ತೀರ್ಥದಲ್ಲಿರುವ ಸೃಷ್ಟಿಕರ್ತ ಬ್ರಹ್ಮನ ದೇವಾಲಯವನ್ನೂ ಅವರು ಸಂದರ್ಶಿಸಿದರು. ಇಡೀ ಭಾರತದಲ್ಲೆಲ್ಲ ಬ್ರಹ್ಮನ ದೇವಾಲಯ ಇದೊಂದೇ ಎನ್ನಲಾಗಿದೆ.

ಅಜ್ಮೀರದಲ್ಲಿ ಒಂದೆರಡು ವಾರಗಳನ್ನು ಕಳೆದ ಬಳಿಕ ಪಶ್ಚಿಮ ಭಾರತದ ಕಡೆಗೆ ಹೊರಟ ಸ್ವಾಮೀಜಿ, ಏಪ್ರಿಲ್ ೧೪ರಂದು ಮೌಂಟ್ ಆಬುವಿಗೆ ಬಂದು ತಲುಪಿದರು. ಇದು ಪಶ್ಚಿಮ ಭಾರತದಲ್ಲೇ ಅತ್ಯಂತ ಪ್ರಸಿದ್ಧವಾದ ಗಿರಿಧಾಮ. ಇಲ್ಲಿನ ಹವಾಗುಣ ತುಂಬ ಆಹ್ಲಾದಕರ ವಾದದ್ದು. ಅದರಲ್ಲೂ ರಾಜಾಸ್ಥಾನದ ಆ ಉರಿಬಿಸಿಲಿನ ಬೇಸಿಗೆಯಲ್ಲಂತೂ ಮೌಂಟ್ ಆಬುವಿನ ವಾಸ್ತವ್ಯ ಅಪ್ಯಾಯಮಾನವಾದದ್ದೇ ಸರಿ. ಇಲ್ಲಿ ಬೆಟ್ಟದ ತುದಿಯಲ್ಲಿ ಬಂಡೆಯೊಂದರ ಮೇಲೆ, ವಿಷ್ಣುವಿನ ಅವತಾರಗಳಲ್ಲೊಂದಾದ ಭೃಗು ಮುನಿಯ ಹೆಜ್ಜೆಯ ಗುರುತು ಎಂದು ಹೇಳಲಾಗುವ ಗುರುತನ್ನು ಕಾಣಬಹುದು. ಅಲ್ಲದೆ, ಮೌಂಟ್ ಆಬುವಿನಲ್ಲಿ ಅದ್ಭುತವಾದ ಜೈನ ದೇವಾಲಯ ವೊಂದಿದೆ. ಇದು ಅಮೃತಶಿಲೆಯಲ್ಲಿ ಕಟ್ಟಲ್ಪಟ್ಟಿದ್ದು, ಜೈನ ವಾಸ್ತುಶಿಲ್ಪದ ಅತ್ಯಂತ ಸುಂದರ ಉದಾಹರಣೆಯಾಗಿದೆ. ಇಲ್ಲಿನ ಕಲಾಕೃತಿಗಳ ಸೂಕ್ಷ್ಮತೆ ಹಾಗೂ ಸೌಂದರ್ಯಗಳು ಅದ್ವಿತೀಯ ವಾದವುಗಳು. ಸ್ವಾಮೀಜಿ ಈ ದೇವಾಲಯದ ಸೌಂದರ್ಯವನ್ನು ಆಸ್ವಾದಿಸುತ್ತ ಹಲವಾರು ದಿನ ಗಳನ್ನು ಕಳೆದರು. ಕೆಲವೊಮ್ಮೆ ಅವರು ಇಲ್ಲಿನ ಸುಂದರ ಸರೋವರದ ತೀರದಲ್ಲಿ ಅಡ್ಡಾಡುತ್ತಿದ್ದರು.

ಆ ದಿನಗಳಲ್ಲಿ ಆಬು ಬೆಟ್ಟವು ಆ ಪ್ರದೇಶದ ರಾಜಮನೆತನಗಳವರ ಬೇಸಿಗೆಯ ವಿಹಾರ ಸ್ಥಳವಾಗಿತ್ತು. ಆದ್ದರಿಂದ ಅಲ್ಲಿ ರಾಜರೂ ಉನ್ನತ ಅಧಿಕಾರಿಗಳೂ ಓಡಿಯಾಡುತ್ತಿದ್ದರು. ಸ್ವಾಮೀಜಿ ಸಾಧ್ಯವಾದಷ್ಟೂ ಯಾರ ಕಣ್ಣಿಗೂ ಬೀಳದಂತೆ ಏಕಾಂತದಲ್ಲಿದ್ದುಕೊಂಡು ಧ್ಯಾನ ಜಪ ತಪಗಳಲ್ಲಿ ನಿರತರಾಗಿರುತ್ತಿದ್ದರು. ಆದರೆ ಅವರು ತಮ್ಮಷ್ಟಕ್ಕೆ ತಾವಿರಬೇಕೆಂದು ಎಷ್ಟೇ ಪ್ರಯತ್ನಿಸಿದರೂ ಅದು ಸಾಧ್ಯವಾಗುವಂತಿರಲಿಲ್ಲ. ಅವರು ದ್ವಾರಕೆಗೆ ಹೋಗುವ ದಾರಿಯಾಗಿ ಇಲ್ಲಿಗೆ ಬಂದಿದ್ದರೂ ಇಲ್ಲಿನ ಪ್ರಶಾಂತ ವಾತಾವರಣದಲ್ಲಿ ತೀವ್ರ ಆಧ್ಯಾತ್ಮಿಕ ಸಾಧನೆಯಲ್ಲಿ ಕೆಲವು ದಿನಗಳನ್ನು ಕಳೆಯುವ ತೀರ್ಮಾನ ಮಾಡಿ ಇಲ್ಲಿ ಉಳಿದುಕೊಂಡಿದ್ದರು. ಆದರೆ ಇಲ್ಲಿಗೆ ಅವರನ್ನು ಕರೆತಂದದ್ದು ದೈವೇಚ್ಛೆಯಲ್ಲದೆ ಬೇರಲ್ಲ ಎಂಬುದನ್ನು ಕಾಣಲಿದ್ದೇವೆ.

ಕೆಲದಿನಗಳಲ್ಲೇ ಅವರ ಆಕರ್ಷಕ ನಿಲುವು-ತೇಜಸ್ಸುಗಳಿಂದ ಸೆಳೆಯಲ್ಪಟ್ಟು ಅನೇಕರು ಅವರ ಅನುಯಾಯಿಗಳಾದರು. ಸ್ವಾಮೀಜಿ ಕೆಲವೊಮ್ಮೆ ಅವರೊಂದಿಗೆ ಸಂಜೆಯ ವೇಳೆಗೆ ತಿರುಗಾಡಲು ಹೋಗುತ್ತಿದ್ದರು. ಒಂದು ದಿನ ಸಂಜೆ ಬೆಟ್ಟದ ಮೇಲೆ ಬಂಡೆಗಳ ಮಧ್ಯದ ಒಂದು ಸುಂದರ, ಪ್ರಶಾಂತ ಸ್ಥಳದಲ್ಲಿ ಸ್ವಾಮೀಜಿ ಇತರರೊಂದಿಗೆ ಕುಳಿತಿದ್ದರು. ಇಲ್ಲಿಂದ ಇಡೀ ಗಿರಿಧಾಮದ ದೃಶ್ಯ ತುಂಬ ಮನೋಹರವಾಗಿ ಕಾಣಿಸುತ್ತಿತ್ತು. ಆಗ ಸ್ವಾಮೀಜಿ ಭಾವಭರಿತರಾಗಿ ಹಾಡಲಾರಂಭಿ ಸಿದರು. ಸುತ್ತಲಿದ್ದವರು ಮಂತ್ರಮುಗ್ಧರಾಗಿ ಕುಳಿತು ಆಲಿಸುತ್ತಿದ್ದಂತೆ, ಅವರ ಗಾಯನ ಅನೇಕ ಗಂಟೆಗಳ ಕಾಲ ಮುಂದುವರಿಯಿತು. ಈ ವೇಳೆಗೆ ಅಲ್ಲಿ ಅಡ್ಡಾಡಲು ಬಂದಿದ್ದ ಕೆಲವು ಐರೋಪ್ಯರು ಆ ಸಮಧುರ ಗಾನವನ್ನು ಕೇಳಿ ಆಕರ್ಷಿತರಾಗಿ, ಆ ಅದ್ಭುತ ಗಾಯಕನ ಮುಖ ದರ್ಶನ ಮಾಡಬೇಕೆಂಬ ಕಾತರತೆಯಿಂದ ಕಾದುಕುಳಿತಿದ್ದರು. ಸ್ವಾಮೀಜಿ ಇತ್ತ ಎದ್ದುಬಂದಾಗ ಆ ಐರೋಪ್ಯರು ಅವರ ಮಧುರಕಂಠ ಹಾಗೂ ಭಾವಪೂರ್ಣ ಗಾಯನಗಳನ್ನು ಮನಸಾರೆ ಪ್ರಶಂಸಿಸಿದರು.

ಇಲ್ಲಿ ಸ್ವಾಮೀಜಿ ನಿರ್ಜನ ಸ್ಥಳದಲ್ಲಿದ್ದ ಗುಹೆಯೊಂದರಲ್ಲಿ ಉಳಿದುಕೊಂಡು ಆಧ್ಯಾತ್ಮಿಕ ಸಾಧನೆಯಲ್ಲಿ ನಿರತರಾಗಿದ್ದರು. ಒಂದು ದಿನ ಕಿಶನ್​ಘರ್ ಪ್ರಾಂತದ ರಾಜನ ಮುಸ್ಲಿಂ ವಕೀಲನಾದ ಫೈಯಾಸ್​ಅಲಿಖಾನ್ ಎಂಬವನು ಈ ಗುಹೆಯ ಮಾರ್ಗವಾಗಿ ಹೋಗುತ್ತಿದ್ದಾಗ ಅವರನ್ನು ನೋಡಿದ. ಅವರ ಭವ್ಯ ನಿಲುವಿನಿಂದ ಆಕರ್ಷಿತನಾಗಿ, ಅವರೊಂದಿಗೆ ಮಾತನಾಡಲು ಇಚ್ಛಿಸಿದ. ಒಂದೆರಡು ಮಾತನಾಡುತ್ತಿದ್ದಂತೆ, ವಕೀಲನಿಗೆ ಅವರ ಅಗಾಧ ಪಾಂಡಿತ್ಯದ ಅರಿ ವಾಯಿತು. ಅಂದಿನಿಂದ ಆತ ಅವರ ಬಳಿಗೆ ಆಗಾಗ ಬರಲಾರಂಭಿಸಿದ. ಬರಬರುತ್ತ ಅವನಿಗೆ ಸ್ವಾಮೀಜಿಯ ಮೇಲಿನ ಗೌರವಾದರ ಮತ್ತಷ್ಟು ಹೆಚ್ಚಿತು. ಒಂದು ದಿನ ಆತ, “ಸ್ವಾಮೀಜಿ, ನನ್ನಿಂದೇನಾದರೂ ಸೇವೆಯಾಗಬೇಕಿದ್ದರೆ ದಯವಿಟ್ಟು ಹೇಳಿ” ಎಂದು ಕೇಳಿಕೊಂಡ. ಅವನ ಪ್ರಾಮಾಣಿಕತೆಯನ್ನು ಮನಗಂಡಿದ್ದ ಸ್ವಾಮೀಜಿ ಅದಕ್ಕೊಪ್ಪಿ ಹೇಳಿದರು, “ನೋಡಿ, ಇನ್ನೇನು ಮಳೆಗಾಲ ಪ್ರಾರಂಭವಾಗುತ್ತದೆ. ಈ ಗುಹೆಗೆ ಬಾಗಿಲುಗಳಿಲ್ಲದಿರುವುದರಿಂದ ನೀರು ಒಳಕ್ಕೆ ಬರುತ್ತದೆ. ಆದ್ದರಿಂದ, ನಿಮಗೆ ಸಾಧ್ಯವಾಗುವುದಾದರೆ, ಈ ಗುಹೆಗೆ ಬಾಗಿಲು ಮಾಡಿಸಿಕೊಡಿ.” ಆಗ ಆ ವಕೀಲ ಹೇಳಿದ, “ಆದರೆ ಸ್ವಾಮೀಜಿ, ಈ ಗುಹೆ ಏನೇನೂ ಚೆನ್ನಾಗಿಲ್ಲ. ನೀವು ಅಪ್ಪಣೆ ಕೊಡುವುದಾದರೆ ನಾನೊಂದು ಸಲಹೆ ಮಾಡುತ್ತೇನೆ. ನಾನು ಇಲ್ಲೇ ಒಂದು ಒಳ್ಳೆಯ ಬಂಗಲೆ ಯಲ್ಲಿ ಒಬ್ಬನೇ ವಾಸವಾಗಿದ್ದೇನೆ. ಅಲ್ಲಿಗೆ ನೀವು ಬಂದಿರಲು ದಯಮಾಡಿ ಒಪ್ಪಿಕೊಳ್ಳುವು ದಾದರೆ ನನ್ನನ್ನು ಧನ್ಯನೆಂದುಕೊಳ್ಳುತ್ತೇನೆ.” ಸ್ವಾಮೀಜಿ ಇದಕ್ಕೆ ಸಮ್ಮತಿಸಿದರು. ಆಗ ವಕೀಲ, “ಸ್ವಾಮೀಜಿ, ನಾನೊಬ್ಬ ಮುಸಲ್ಮಾನ. ಆದರೆ ನೀವೇನೂ ಚಿಂತಿಸಬೇಕಿಲ್ಲ; ನಾನು ನಿಮ್ಮ ಆಹಾರಾದಿಗಳಿಗೆ ಪ್ರತ್ಯೇಕ ವ್ಯವಸ್ಥೆ ಮಾಡಿಸುತ್ತೇನೆ” ಎಂದು ಭರವಸೆ ನೀಡಿದ. ಸ್ವಾಮೀಜಿ ಅವನ ಮನೆಗೆ ಹೋಗಿರಲು ಒಪ್ಪಿಕೊಂಡರು. ಈ ವಕೀಲ, ತನಗೆ ಪರಿಚಯಸ್ಥರಾದ ಹಲವಾರು ಉನ್ನತ ಅಧಿಕಾರಿಗಳನ್ನು ಅವರಿಗೆ ಪರಿಚಯಿಸಿಕೊಟ್ಟ. ಇವರಲ್ಲಿ ಕೋಟಾಪ್ರಾಂತದ ರಾಜನ ವಕೀಲರೂ, ಅಲ್ಲಿನ ಮಂತ್ರಿಯಾದ ಫತೇಹ್​ಸಿಂಗರೂ ಇದ್ದರು. ಕೆಲದಿನಗಳಾದ ಮೇಲೆ, ಸ್ವಾಮೀಜಿಯ ಆತಿಥೇಯನಾದ ಫೈಯಾಸ್ ಅಲಿಖಾನ್, ಖೇತ್ರಿಯ ಮಹಾರಾಜನ ಆಪ್ತಕಾರ್ಯದರ್ಶಿಯಾದ ಮುನ್ಷಿ ಜಗಮೋಹನಲಾಲ್ ಎಂಬುವನನ್ನು ಭೇಟಿ ಮಾಡಿದ. ತನ್ನ ಅತಿಥಿಯಾಗಿ ಉಳಿದುಕೊಂಡಿ ರುವ ಅದ್ಭುತ ಸಂನ್ಯಾಸಿಗಳ ಬಗ್ಗೆ ಹೇಳಿ ಅವರನ್ನು ಕೊಂಡಾಡಿದನಲ್ಲದೆ, ತನ್ನ ಮನೆಗೆ ಬಂದು ಅವರನ್ನು ಭೇಟಿ ಮಾಡುವಂತೆ ಜಗಮೋಹನಲಾಲನನ್ನು ಆಹ್ವಾನಿಸಿದ.

ಜಗಮೋಹನಲಾಲ್ ಅತ್ಯಂತ ಸಂಪ್ರದಾಯಸ್ಥ ವೈಷ್ಣವ; ಹಲವಾರು ಭಾಷೆಗಳಲ್ಲೂ ಶಾಸ್ತ್ರ ಗಳಲ್ಲೂ ಪಾಂಡಿತ್ಯವಿದ್ದವನು. ಆದರೆ ಈತನಿಗೆ ಸಾಧು-ಸಂನ್ಯಾಸಿಗಳೆಂದರೆ ಅಷ್ಟಕಷ್ಟೆ. ಆದರೂ ತನ್ನ ಸ್ನೇಹಿತನ ಆಹ್ವಾನವನ್ನು ಮನ್ನಿಸಿ ಅವನ ಮನೆಗೆ ಬಂದ. ಆಗ ಸ್ವಾಮೀಜಿ ಒಂದು ಕೌಪೀನ ಹಾಗೂ ಒಂದು ತುಂಡು ಕಾವಿ ವಸ್ತ್ರವನ್ನು ಮಾತ್ರ ಧರಿಸಿ ಮಲಗಿದ್ದರು. ಅವರನ್ನು ನೋಡಿದ ತಕ್ಷಣ ಮುನ್ಷಿಗೆ ಅನ್ನಿಸಿತು–“ಓ, ಸರಿ ಸರಿ; ನೋಡಿದರೇ ಗೊತ್ತಾಗುತ್ತದೆ... ಕಾವಿಬಟ್ಟೆ ಹಾಕಿಕೊಂಡು ಅಲೆದಾಡುತ್ತ ರಾತ್ರಿ ಹೊತ್ತು ಬಾಗಿಲು ಮುರಿಯುವ ಪುಂಡರ ಪೈಕಿ ಇವನೂ ಒಬ್ಬ, ಅಷ್ಟೆ.”

ಅಷ್ಟುಹೊತ್ತಿಗೆ ಸ್ವಾಮೀಜಿ ಎಚ್ಚರಗೊಂಡು ಎದ್ದು ಕುಳಿತರು. ಜಗಮೋಹನ ಅವರಿಗೆ ನಮಸ್ಕರಿಸಿ ಮಾತುಕತೆ ಪ್ರಾರಂಭಿಸಿದ.

“ಸ್ವಾಮೀಜಿ, ನೀವೊಬ್ಬರು ಹಿಂದೂ ಸಾಧುಗಳಾಗಿದ್ದು, ಮುಸಲ್ಮಾನರ ಮನೆಯಲ್ಲಿ ಇಳಿದು ಕೊಂಡಿದ್ದೀರಲ್ಲ! ನೀವು ತೆಗೆದುಕೊಳ್ಳುವ ಆಹಾರ ಮೈಲಿಗೆಯಾಗಲು ಸಾಧ್ಯವಿಲ್ಲವೆ...?”

ಈ ಪ್ರಶ್ನೆ ಅವರನ್ನು ಸಿಟ್ಟಿಗೆಬ್ಬಿಸಿತು.

“ಏನು ನಿಮ್ಮ ಮಾತಿನ ಅರ್ಥ? ನಾನೊಬ್ಬ ಸಂನ್ಯಾಸಿ! ನಾನು ನಿಮ್ಮ ಸಾಮಾಜಿಕ ಕಟ್ಟುಕಟ್ಟಲೆ ಗಳನ್ನೂ ಕಂದಾಚಾರಗಳನ್ನೂ ಮೀರಿ ನಿಂತವನು. ಒಬ್ಬ ಜಾಡಮಾಲಿಯ ಜೊತೆಗೂ ಊಟಮಾಡ ಬಲ್ಲೆ ನಾನು. ದೇವರಿಗೆ ನಾನು ಹೆದರುವುದಿಲ್ಲ; ಏಕೆಂದರೆ ಅವನು ಇದನ್ನು ಒಪ್ಪುತ್ತಾನೆ. ಶಾಸ್ತ್ರಗಳಿಗೂ ಹೆದರಬೇಕಿಲ್ಲ ನಾನು–ಅವೂ ನನ್ನ ಈ ಕೃತ್ಯವನ್ನು ಅನುಮೋದಿಸುತ್ತವೆ. ಆದರೆ ನಾನು ನಿಮ್ಮಂತಹ ಜನಕ್ಕೆ ನಿಮ್ಮ ಈ ಸಮಾಜಕ್ಕೆ ಹೆದರಿಕೊಳ್ಳಬೇಕಾಗಿ ಬಂದಿದೆ. ದೇವರ ಬಗ್ಗೆ ನಿಮಗೇನು ಗೊತ್ತಿದೆ? ಶಾಸ್ತ್ರಗಳ ವಿಚಾರ ಏನು ತಿಳಿದಿದೆ ನಿಮಗೆ? ಎಲ್ಲೆಲ್ಲೂ ನಾನು ಬ್ರಹ್ಮವನ್ನೇ ಕಾಣುತ್ತೇನೆ. ಒಂದು ಕ್ಷುದ್ರ ಜಂತುವಿನಲ್ಲೂ, ಬ್ರಹ್ಮದ ಅವಿರ್ಭಾವವನ್ನು ಕಾಣುತ್ತೇನೆ. ನನಗೆ ಯಾವುದೂ ಮೇಲಲ್ಲ; ಯಾವುದೂ ಕೀಳಲ್ಲ... ಶಿವ ಶಿವ!”

ಹೀಗೆ ಗುಡುಗುವಾಗ ಸ್ವಾಮೀಜಿಯ ಮುಖದಲ್ಲಿ ಒಂದು ದಿವ್ಯ ಜ್ಯೋತಿ ಬೆಳಗುತ್ತಿತ್ತು. ಕಮ್ಮಾರನ ಸುತ್ತಿಗೆಯ ಪೆಟ್ಟು ಬಿದ್ದಂತಾಗಿ ಮುನ್ಷಿ ಒಂದು ಕ್ಷಣ ಸ್ತಬ್ಧನಾಗಿ ಕುಳಿತುಬಿಟ್ಟ. ಇವರೊಬ್ಬ ಅಸಾಧಾರಣ ವ್ಯಕ್ತಿಯೆಂಬುದು ಅವನಿಗೆ ಗೋಚರವಾಯಿತು. ಇಂಥವರನ್ನು ತಮ್ಮ ಮಹಾರಾಜ ಭೇಟಿ ಮಾಡಲೇಬೇಕು ಎಂದು ಅವನಿಗನ್ನಿಸಿತು. ಈಗ ಆತ ಕೇಳಿಕೊಂಡ–

“ಸ್ವಾಮೀಜಿ, ನಮ್ಮ ಮಹಾರಾಜರು ತಮ್ಮನ್ನು ಕಾಣಲು ಇಷ್ಟಪಡುತ್ತಾರೆ. ತಾವು ದಯವಿಟ್ಟು ಅರಮನೆಗೊಮ್ಮೆ ಬಂದುಹೋಗಬೇಕು.”

ಸ್ವಾಮೀಜಿ ಒಂದು ನಿಮಿಷ ಆಲೋಚಿಸಿ, ಬಳಿಕ ಹೇಳಿದರು–

“ಆಗಲಿ, ನಾನು ನಾಡಿದ್ದು ಅಲ್ಲಿಗೆ ಬರುತ್ತೇನೆ.”

ಖೇತ್ರಿಯ ಮಹಾರಾಜನಾದ ಅಜಿತ್​ಸಿಂಗ್ ತನ್ನ ಪ್ರಜೆಗಳ ಹಿತಚಿಂತನೆಯಲ್ಲೇ ಸದಾ ಮಗ್ನ ನಾಗಿದ್ದವನು. ತೀರಾ ಬಡವರೂ ಅವಿದ್ಯಾವಂತರೂ ಆದ ತನ್ನ ಪ್ರಜೆಗಳ ಜೀವನಮಟ್ಟವನ್ನು ಸುಧಾರಿಸುವ ಸಲುವಾಗಿ ಹಲವಾರು ಯೋಜನೆಗಳನ್ನು ಕೈಗೊಂಡಿದ್ದ. ಇವನು ಅತ್ಯಂತ ಪುರೋಗಾಮಿಯೂ ಪ್ರಾಮಾಣಿಕನೂ ಆಗಿದ್ದು ಆ ಕಾಲದ ಕೆಲವೇ ಮಂದಿ ಯೋಗ್ಯ ರಾಜ್ಯಾಡಳಿತ ಗಾರರಲ್ಲೊಬ್ಬನಾಗಿದ್ದ. ಆ ಸಮಯದಲ್ಲಿ ಮಹಾರಾಜ ಮೌಂಟ್ ಅಬುವಿನ ತನ್ನ ಬೇಸಿಗೆಯ ಅರಮನೆಯಲ್ಲಿ ಬೀಡುಬಿಟ್ಟಿದ್ದ. ಮುನ್ಷಿ ಜಗಮೋಹನಲಾಲ್ ಬಂದು ತಾನು ಭೇಟಿ ಮಾಡಿದ ಅಸಾಧಾರಣ ಸಾಧುಗಳೊಬ್ಬರ ಬಗ್ಗೆ ಹೇಳಿದಾಗ ರಾಜನಿಗೆ ಅವರನ್ನು ಕೂಡಲೇ ಕಾಣುವ ಇಚ್ಛೆಯುಂಟಾಯಿತು. ತನ್ನ ಕುತೂಹಲವನ್ನು ಹತ್ತಿಕ್ಕಲಾರದೆ ತಾನೇ ಅವರ ಬಳಿಗೆ ಹೋಗುವು ದೆಂದು ನಿರ್ಧರಿಸಿದ. ಆದರೆ ಸ್ವಾಮೀಜಿಗೆ ಈ ವಿಷಯ ತಿಳಿದಾಗ ತಕ್ಷಣ ಅವರೇ ಅರಮನೆಗೆ ಹೊರಟು ಬಂದರು.

ಅಂದು ೧೮೯೧ರ ಜೂನ್ ೪ನೇ ತಾರೀಖು, ಸಂಜೆ. ಮಹಾರಾಜನೊಂದಿಗೆ ಹಲವಾರು ಗಣ್ಯ ವ್ಯಕ್ತಿಗಳಿದ್ದರು. ರಾಜಾ ಅಜಿತ್​ಸಿಂಗ್ ಸ್ವಾಮೀಜಿಯನ್ನು ಆದರದಿಂದ ಸ್ವಾಗತಿಸಿದ. ಸ್ವಾಗತಹ ಔಪಚಾರಿಕ ಮಾತುಕತೆಗಳು ಮುಗಿದುವು. ಈಗ ಮಹಾರಾಜ ಗಂಭೀರ ಸಂಭಾಷಣೆಯಲ್ಲಿ ತೊಡಗಿ ಒಂದು ಪ್ರಶ್ನೆಯನ್ನು ಮುಂದಿಟ್ಟ–

“ಸ್ವಾಮೀಜಿ, ಜೀವನ ಎಂದರೇನು?”

ಇದೆಂಥ ಪ್ರಶ್ನೆ? ಜೀವನವೆಂದರೇನು ಎಂಬುದನ್ನೂ ತಿಳಿಯದವನೆ ಈ ಮಹಾರಾಜ! ಹುಟ್ಟು-ಸಾವುಗಳ ನಡುವಿನ ಅವಧಿಯೇ ಜೀವನ ಎನ್ನುವುದು ಪ್ರತಿಯೊಬ್ಬರೂ ತಿಳಿದಿರುವ ‘ಸತ್ಯ’. ಆದರೆ ನಿಜಕ್ಕೂ ಇದೊಂದು ಸ್ವಾರಸ್ಯಕರ ಪ್ರಶ್ನೆಯೇ ಸರಿ. ಸ್ವಲ್ಪ ತೊಡಕಿನ ಪ್ರಶ್ನೆ ಯೆಂದೂ ಹೇಳಬಹುದು. ಪ್ರಶ್ನೆಗೆ ಸ್ವಾಮೀಜಿಯ ಉತ್ತರ ಬಾಣದಂತೆ ಚಿಮ್ಮಿ ಬಂತು–

“ತನ್ನನ್ನು ಕೆಳಕ್ಕೊತ್ತುತ್ತಿರುವ ಪರಿಸರ-ಪರಿಸ್ಥಿತಿಗಳಲ್ಲಿ ಒಂದು ಜೀವಿಯ ವಿಕಸನ-ಮುನ್ನಡೆ ಗಳೇ ಜೀವನ.”\eng{\textit{(“Life is the unfoldment and development of a being under circumstances tending to press it down.”)}}

ಎಷ್ಟು ಅರ್ಥಪೂರ್ಣ ಈ ಉತ್ತರ! ಜೀವನವೆಂದರೆ ಅದೊಂದು ಅವಿರತ ಹೋರಾಟ ಎನ್ನುತ್ತಿದ್ದಾರೆ ಸ್ವಾಮೀಜಿ. ಪರಿಸರ-ಪರಿಸ್ಥಿತಿಗಳು ಮನುಷ್ಯನನ್ನು ಕೆಳಕ್ಕೆ ಒತ್ತುತ್ತಲೇ ಇರುತ್ತವೆ –ಗುರುತ್ವಾಕರ್ಷಣೆಯು ವಸ್ತುಗಳನ್ನು ಕೆಳಕ್ಕೆಳೆಯುವಂತೆ. ಇವುಗಳನ್ನು ಎದುರಿಸಿ ನಿಂತು ಹೋರಾಡುವುದೇ ಜೀವನ. ಇವುಗಳಿಗೆ ಶರಣಾಗಿ ಸೋಲನ್ನೊಪ್ಪಿಕೊಳ್ಳುವುದು ಮರಣಕ್ಕೆ ಸಮಾನ. ಪ್ರತಿಯೊಬ್ಬ ‘ಜೀವಿ’ಯೂ ಅರಿತುಕೊಳ್ಳಬೇಕಾದ ಪ್ರಥಮ, ಪ್ರಮುಖ ಪಾಠ ಇದು. ಮುಂದೆ ಸ್ವಾಮೀಜಿ ಈ ಅಭಿಪ್ರಾಯವನ್ನೇ ಘಂಟಾಘೋಷವಾಗಿ ಸಾರುತ್ತಾರೆ. ಅಲ್ಲದೆ, ಈ ಉತ್ತರದ ಹಿಂದಿದ್ದ ಸ್ವತಃ ಅವರ ಜೀವನ ಹಾಗೂ ಧೀರ ಮನೋವೃತ್ತಿಗಳು ಅವರ ಮಾತಿಗೆ ಅಸಾಮಾನ್ಯ ಭಾವದೀಪ್ತಿಯನ್ನು ತಂದುಕೊಟ್ಟಿದ್ದುವು.

ಈ ಉತ್ತರದ ಮರ್ಮವನ್ನು ಗ್ರಹಿಸಿದ ಅಜಿತ್​ಸಿಂಗ್ ತುಂಬ ಪ್ರಭಾವಿತನಾದ. ಒಂದು ಕ್ಷಣ ಆಲೋಚಿಸಿ ಮತ್ತೆ ಕೇಳಿದ–

“ಹಾಗಾದರೆ, ಸ್ವಾಮೀಜಿ, ಶಿಕ್ಷಣ ಎಂದರೇನು?”

ಈ ಪ್ರಶ್ನೆಗೂ ಅವರು ಕೊಟ್ಟ ಉತ್ತರ ವೈಶಿಷ್ಟ್ಯಪೂರ್ಣ, ವಿನೂತನ: “ಹೊಸ ಭಾವನೆಗಳನ್ನು ಸ್ವಭಾವ ಸಹಜವಾಗಿಸಿಕೊಳ್ಳುವುದನ್ನೇ ಶಿಕ್ಷಣ ಎನ್ನಬಹುದು.”

ಹೀಗೆ ಹೇಳಿ, ತಮ್ಮ ಈ ಮಾತಿನ ಅರ್ಥವನ್ನು ಸ್ವಾಮೀಜಿ ಮತ್ತೆ ವಿಶದಪಡಿಸಿದರು. ಶಿಕ್ಷಣ ಅಥವಾ ವಿದ್ಯಾಭ್ಯಾಸವೆಂದರೆ ಕೇವಲ ವಿಷಯ ಸಂಗ್ರಹಣೆಯಲ್ಲ; ನಿಜವಾದ ಜ್ಞಾನಕ್ಕೂ ಪುಸ್ತಕ ಪಾಂಡಿತ್ಯಕ್ಕೂ ಅಜಗಜಾಂತರ ಎಂದು ವಿವರಿಸಿ ಹೇಳಿ, ಎಲ್ಲಿಯವರೆಗೆ ಕಲಿತ ವಿಷಯಗಳು ವ್ಯಕ್ತಿಗೆ ಹೃದ್ಗತವಾಗಿ ಅವನ ಪ್ರತಿಯೊಂದು ನಡೆನುಡಿಯಲ್ಲೂ ಕಂಡುಬರುವುದಿಲ್ಲವೋ ಅಲ್ಲಿಯವರೆಗೆ ಅವುಗಳನ್ನು ಆ ವ್ಯಕ್ತಿ ಸ್ವಾಧೀನಪಡಿಸಿಕೊಂಡಿದ್ದಾನೆ ಎನ್ನಲಾಗುವುದಿಲ್ಲ ಎಂದರು ಸ್ವಾಮೀಜಿ.

ಅನಂತರ ಅವರು ಮಹಾರಾಜನಿಗೆ ತಮ್ಮ ಗುರುದೇವನಾದ ಶ್ರೀರಾಮಕೃಷ್ಣ ಪರಮಹಂಸರ ಜೀವನವನ್ನು ವರ್ಣಿಸಿದರು. ಆ ಅದ್ಭುತ ಚರಿತ್ರೆಯನ್ನು ಕಣ್ಣಿಗೆ ಕಟ್ಟುವ ರೀತಿಯಲ್ಲಿ ಹೇಳುತ್ತಿ ದ್ದಂತೆ ಮಹಾರಾಜನ ಹೃದಯದಲ್ಲಿ ಸತ್ಯಸಾಕ್ಷಾತ್ಕಾರದ ಹಂಬಲ ಭುಗಿಲೆದ್ದಿತು.

ಹೀಗೆ ಸ್ವಾಮೀಜಿ ಹಾಗೂ ರಾಜಾ ಅಜಿತ್​ಸಿಂಗರು ಪ್ರಥಮ ಭೇಟಿಯಲ್ಲೇ ಪರಸ್ಪರರಿಂದ ತೀವ್ರವಾಗಿ ಆಕರ್ಷಿತರಾದರು. ಅಜಿತ್​ಸಿಂಗ್ ಸ್ವಾಮೀಜಿಯ ಪ್ರಬಲ ಅಲೌಕಿಕ ವ್ಯಕ್ತಿತ್ವದಿಂದ, ಅದ್ಭುತ ವಾಗ್ಧಾರೆಯಿಂದ, ಆಧ್ಯಾತ್ಮಿಕ ಶಕ್ತಿಯಿಂದ ಸೆಳೆಯಲ್ಪಟ್ಟರೆ, ಸ್ವಾಮೀಜಿ ಅಜಿತ್​ಸಿಂಗನ ಪ್ರಾಮಾಣಿಕತೆ, ಸರಳತೆ, ಜಿಜ್ಞಾಸೆಗಳನ್ನು ಮೆಚ್ಚಿಕೊಂಡರು. ಇಂತಹ ವ್ಯಕ್ತಿಗಳಿಂದ ರಾಷ್ಟ್ರೋನ್ನತಿಯ ಕಾರ್ಯ ಸಾಧ್ಯವಾಗುವುದೆಂದು ಅವರು ಆಶಿಸಿದ್ದರು. ಇಬ್ಬರ ನಡುವಣ ಸ್ನೇಹ ದಿನದಿನಕ್ಕೆ ವೃದ್ಧಿಯಾಯಿತು. ಕಡೆಯವರೆಗೂ ಮಹಾರಾಜ, ಸ್ವಾಮೀಜಿಯ ಅತ್ಯಂತ ಆತ್ಮೀಯ ಸ್ನೇಹಿತನೂ ನಿಷ್ಠಾವಂತ ಭಕ್ತನೂ ಆಗಿ ಉಳಿದುಕೊಂಡ.

ಸ್ವಾಮೀಜಿ ಖೇತ್ರಿಯ ಮಹಾರಾಜನೊಂದಿಗೆ ನಡೆಸಿದ ಸಂಭಾಷಣೆ, ಒಡನಾಟಗಳು ಹಾಗೂ ಇತರ ಚಟುವಟಿಕೆಗಳ ಕುರಿತ ಕೆಲವು ವಿವರಗಳು, ಆ ರಾಜ್ಯದ ಸರಕಾರೀ ದಿನಚರಿ ಪುಸ್ತಕದಿಂದ ತಿಳಿದುಬರುತ್ತದೆ. ಇದರಲ್ಲಿ ಮಹಾರಾಜನನ್ನು ಸ್ವಾಮೀಜಿ ಭೇಟಿಮಾಡಿದ ದಿನಗಳು, ಸಮಯ ಹಾಗೂ ಅವರು ಜೊತೆಯಾಗಿ ಕೈಗೊಂಡ ಪ್ರವಾಸಗಳು ಮೊದಲಾದ ಹಲವಾರು ವಿವರಗಳು ನಿಷ್ಕೃಷ್ಟವಾಗಿ ಬರೆದಿಡಲ್ಪಟ್ಟಿವೆ.

ಮೊದಲ ಭೇಟಿಯಲ್ಲೇ ಸ್ವಾಮೀಜಿಯಿಂದ ತೀವ್ರವಾಗಿ ಆಕರ್ಷಿತನಾದ ಮಹಾರಾಜ ಅವರನ್ನು ಮತ್ತೆ ಮತ್ತೆ ತನ್ನ ಅರಮನೆಗೆ ಆಹ್ವಾನಿಸಿದ. ಆಹ್ವಾನವನ್ನು ಮನ್ನಿಸಿ ಅವರು ಆ ತಿಂಗಳಲ್ಲೇ ಹತ್ತಾರು ಬಾರಿ ಅವನನ್ನು ಭೇಟಿಯಾದರು. ಮಹಾರಾಜ ಅವರೊಂದಿಗೆ ವಿದ್ಯಾಭ್ಯಾಸ, ಧರ್ಮ, ತತ್ತ್ವಶಾಸ್ತ್ರ ಮೊದಲಾದ ಹಲವಾರು ವಿಷಯಗಳ ಕುರಿತಾಗಿ ಸಂಭಾಷಿಸುತ್ತಿದ್ದ. ಸಂಭಾಷಣೆ ನಡೆಯುತ್ತಿದ್ದುದು ಸಾಮಾನ್ಯವಾಗಿ ಇಂಗ್ಲಿಷಿನಲ್ಲಿ, ಕೆಲವೊಮ್ಮೆ ಸಂಸ್ಕೃತದಲ್ಲಿ. ರಾಜನ ಕೋರಿಕೆಯ ಮೇರೆಗೆ ಒಮ್ಮೊಮ್ಮೆ ಸ್ವಾಮೀಜಿ ಹಾಡುತ್ತಿದ್ದರು. ಸ್ವತಃ ಸಂಗೀತ ಪರಿಣತನಾದ ರಾಜ, ಅವರೊಂದಿಗೆ ಹಾರ್ಮೋನಿಯಂ ನುಡಿಸುತ್ತಿದ್ದ. ಮತ್ತೆ ಕೆಲವು ಸಲ ಇವರೊಂದಿಗೆ ಸಂಭಾಷಣೆ ನಡೆಸುವ ವೇಳೆಗೆ ಹಲವಾರು ಗಣ್ಯವ್ಯಕ್ತಿಗಳನ್ನೂ ನೆರೆ ರಾಜ್ಯಗಳ ಅಧಿಕಾರಿಗಳನ್ನೂ ಆಹ್ವಾನಿಸುತ್ತಿದ್ದ.

ಹೀಗೆ ಸ್ವಾಮೀಜಯನ್ನು ಭೇಟಿ ಮಾಡಿದ ಗಣ್ಯರಲ್ಲಿ ಅಜ್ಮೀರದಲ್ಲಿನ ಆರ್ಯಸಮಾಜದ ಅಧ್ಯಕ್ಷನಾಗಿದ್ದ ಹರವಿಲಾಸ್ ಸಾರಾದಾ ಎಂಬವನೂ ಒಬ್ಬ. ಈತ ಮೌಂಟ್ ಅಬುವಿನಲ್ಲಿ ಹಾಗೂ ಮುಂದೆ ಅಜ್ಮೀರದಲ್ಲಿ ಅವರನ್ನು ಮೂರು-ನಾಲ್ಕು ಬಾರಿ ಭೇಟಿಯಾದ. ಇವನು ಮುಂದೆ ರಜಪುತಾನದ ಇತಿಹಾಸಕಾರನಾಗಿ ಪ್ರಸಿದ್ಧನಾದನಲ್ಲದೆ ಭಾರತದ ಕೇಂದ್ರ ಶಾಸನ ಸಭೆಯ ಸದಸ್ಯನಾಗಿ ‘ಸಾರದಾ ಕಾಯಿದೆ’ ಎಂದು ಹೆಸರಾದ ಬಾಲ್ಯವಿವಾಹ ನಿಷೇಧ ಕಾನೂನನ್ನು ಜಾರಿಗೆ ತರುವಲ್ಲಿ ಪ್ರಮುಖ ಪಾತ್ರವಹಿಸಿದ. ಈತ ಸ್ವಾಮೀಜಿಯನ್ನು ಭೇಟಿ ಮಾಡಿದ್ದು ಕೆಲವೇ ಸಲವಾದರೂ ಅವರಿಂದ ಎಷ್ಟು ತೀವ್ರವಾಗಿ ಪ್ರಭಾವಿತನಾಗಿದ್ದನೆಂಬುದು ಮುಂದೆ ಇವನು ಅವರ ಕುರಿತಾಗಿ ಬರೆದ ಲೇಖನದಿಂದ ತಿಳಿದುಬರುತ್ತದೆ–

“ನಾನು ಸ್ವಾಮಿ ವಿವೇಕಾನಂದರನ್ನು ನಾಲ್ಕು ಬಾರಿ ಭೇಟಿ ಮಾಡಿದೆ. ಅಬುಪರ್ವತದಲ್ಲಿ ನಾನವರನ್ನು ಮೊದಲಸಲ ಕಂಡೆ. ಆಗ ನನಗಿನ್ನೂ ಇಪ್ಪತ್ತೊಂದು ವರ್ಷ ವಯಸ್ಸಷ್ಟೆ. ಅವರು ನನ್ನ ಸ್ನೇಹಿತ ಠಾಕೂರ್ ಮುಕುಂದಸಿಂಗರ ಮನೆಯಲ್ಲಿದ್ದರು. ಆಗ ನಾನು ಸ್ವಾಮೀಜಿಯೊಂದಿಗೆ ಠಾಕೂರರ ಮನೆಯಲ್ಲಿ ಹತ್ತು ದಿನ ವಾಸವಾಗಿದ್ದೆ. ನಾವಿಬ್ಬರೂ ಹಲವಾರು ವಿಷಯಗಳ ಬಗ್ಗೆ ಚರ್ಚಿಸುತ್ತಿದ್ದೆವು. ನಾನು ಅವರ ವ್ಯಕ್ತಿತ್ವದಿಂದ ತೀವ್ರವಾಗಿ ಪ್ರಭಾವಿತನಾದೆ. ಅವರೊಬ್ಬ ದೊಡ್ಡ ಪಂಡಿತರಾಗಿದ್ದರಲ್ಲದೆ ಅತ್ಯಂತ ಆಕರ್ಷಕ ಮಾತುಗಾರರಾಗಿದ್ದರು. ಮೊದಲ ದಿನ ಊಟವಾದ ಮೇಲೆ ಠಾಕೂರ್ ಸಾಹೇಬರ ಪ್ರಾರ್ಥನೆಯನ್ನು ಮನ್ನಿಸಿ ಸ್ವಾಮೀಜಿ ಹಾಡಿದರು. ಅವರ ಕಂಠಸ್ವರ ಅತ್ಯಂತ ಮಧುರವಾಗಿದ್ದು, ನನಗೆ ಅತೀವ ಆನಂದವನ್ನುಂಟು ಮಾಡಿತು. ಅಂದಿನಿಂದ, ಒಂದೆರಡು ಹಾಡುಗಳನ್ನು ಹಾಡಬೇಕೆಂದು ಪ್ರತಿದಿನವೂ ನಾನವರನ್ನು ಕೇಳಿಕೊಳ್ಳುತ್ತಿದ್ದೆ. ಅವರ ಮಧುರ ದನಿ ಹಾಗೂ ನಡವಳಿಕೆಗಳು ನನ್ನ ಮೇಲೆ ಅಚ್ಚಳಿಯದ ಮುದ್ರೆಯನ್ನೊತ್ತಿವೆ. ನಾವು ಕೆಲವೊಮ್ಮೆ ವೇದಾಂತದ ಬಗೆಗೆ ಮಾತನಾಡುತ್ತಿದ್ದೆವು. ಸ್ವಾಮಿ ವಿವೇಕಾನಂದರು ವಿವರಿಸುತ್ತಿದ್ದ ವೇದಾಂತ ವಿಚಾರಗಳು ನನ್ನಲ್ಲಿ ತುಂಬ ಆಸಕ್ತಿಯನ್ನು ಹುಟ್ಟಿಸಿದುವು. ಅನೇಕ ವಿಚಾರಗಳ ಮೇಲಿನ ಅವರ ಅಭಿಪ್ರಾಯಗಳು ದೇಶಪ್ರೇಮದಿಂದ ಒಡಮೂಡಿದವುಗಳಾಗಿದ್ದು ನಾನು ಅವುಗಳನ್ನು ವಿಶೇಷವಾಗಿ ಮೆಚ್ಚಿಕೊಂಡೆ. ಹಿಂದೂ ಸಂಸ್ಕೃತಿ ಹಾಗೂ ಮಾತೃಭೂಮಿಯ ಬಗ್ಗೆ ಅವರಿಗಿದ್ದ ಆದರಾಭಿಮಾನಗಳು ಅವರ ಮಾತುಗಳಲ್ಲಿ ತುಂಬಿತುಳುಕುತ್ತಿದ್ದುವು. ಅವರೊಂದಿಗೆ ನಾನು ಕಳೆದ ದಿನಗಳು ನನ್ನ ಜೀವನದಲ್ಲೇ ಅತ್ಯಂತ ಸುಮಧುರವಾದವುಗಳು. ಅದಲ್ಲೂ ಅವರ ಸ್ವತಂತ್ರ ವ್ಯಕ್ತಿತ್ವದಿಂದ ನಾನು ಬಹಳವಾಗಿ ಪ್ರಭಾವಿತನಾಗಿದ್ದೇನೆ.”

ಅಜಿತ್​ಸಿಂಗನಿಗೆ ಸ್ವಾಮೀಜಿಯ ಮೇಲಿನ ಭಕ್ತಿಗೌರವಗಳು ದಿನದಿನಕ್ಕೆ ಹೆಚ್ಚುತ್ತ ಬಂದವು. ಜುಲೈ ತಿಂಗಳಲ್ಲಿ ಅವನು ಮೌಂಟ್ ಅಬುವಿನಿಂದ ತನ್ನ ರಾಜಧಾನಿಯಾದ ಖೇತ್ರಿಗೆ ಹೊರಟಾಗ ಸ್ವಾಮೀಜಿಯನ್ನು ತನ್ನೊಂದಿಗೆ ಬರುವಂತೆ ಬೇಡಿಕೊಂಡ. ಸ್ವಾಮೀಜಿ ಅದಕ್ಕೆ ಒಪ್ಪಿದರು. ಅದರಂತೆ ಜುಲೈ ೨೪ರಂದು ಮಹಾರಾಜನೊಂದಿಗೆ ಅವರು ಖೇತ್ರಿಗೆ ಹೊರಟರು.

ಮಹಾರಾಜನ ತಂಡ ಮರುದಿನ ಬೆಳಿಗ್ಗೆ ಜೈಪುರವನ್ನು ತಲುಪಿತು. ಇಲ್ಲಿ ಅವರೆಲ್ಲ ಹತ್ತು ದಿನಗಳ ಕಾಲ ‘ಖೇತ್ರಿ ಭವನ’ದಲ್ಲಿ ಉಳಿದುಕೊಂಡರು. ಆಗಸ್ಟ್ ಮೂರರಂದು ಎಲ್ಲರೂ ಜೈಪುರದಿಂದ ಹೊರಟು ಏಳನೇ ತಾರೀಕಿನಂದು ಖೇತ್ರಿಯನ್ನು ತಲುಪಿದರು. ಪ್ರಯಾಣ ಕಾಲ ದಲ್ಲಿ ಮಹಾರಾಜ ಸ್ವಾಮೀಜಿಯೊಂದಿಗೆ ನಿರಂತರವಾಗಿ ಮಾತುಕತೆಯಲ್ಲಿ ತೊಡಗಿರುತ್ತಿದ್ದ. ತನ್ನ ಹಲವಾರು ಸಂಶಯಗಳನ್ನು ಮುಂದಿಟ್ಟು ಅವುಗಳಿಗೆ ಸೂಕ್ತ ಉತ್ತರವನ್ನು ಪಡೆದುಕೊಂಡು ಸಂತೃಪ್ತನಾದ.

ಒಂದು ದಿನ ಆತ ಕೇಳಿದ–

“ಸ್ವಾಮೀಜಿ, ಸತ್ಯ ಎಂದರೇನು?”

ಉತ್ತರ ಸಿದ್ಧವಾಗಿಯೇ ಇತ್ತು –

“ಮಹಾರಾಜ, ಸತ್ಯ ಎಂಬುದು ಏಕಮೇವಾದ್ವಿತೀಯವಾದುದು; ಅದು ಅತ್ಯುಚ್ಚವಾದುದು. ಮನುಷ್ಯ ಸದಾ ಸಾಗುತ್ತಿರುವುದು ಅದರ ಕಡೆಗೇ–ಅವನು ಸತ್ಯದಿಂದ ಸತ್ಯದೆಡೆಗೆ ಸಾಗುತ್ತಾ ನಲ್ಲದೆ ಅಸತ್ಯದಿಂದ ಸತ್ಯದೆಡೆಗಲ್ಲ.”

‘ಸತ್ಯ’ವನ್ನು ಹೀಗೆ ಅರ್ಥೈಸಿದ ಸ್ವಾಮೀಜಿ, ಬಳಿಕ ತಮ್ಮ ಈ ಉತ್ತರವನ್ನು ವಿಶದಪಡಿಸಲು ಆ ಬಗ್ಗೆ ಸುದೀರ್ಘವಾದ ವಿವರಣೆಯನ್ನು ಕೊಟ್ಟರು. ಎಲ್ಲ ಬಗೆಯ ಜ್ಞಾನ ಮತ್ತು ಅನುಭವಗಳ ಪ್ರಕಾರಗಳು, ವಿವಿಧ ಪೂಜಾ ವಿಧಾನಗಳು, ಬಗೆಬಗೆಯ ಆಧ್ಯಾತ್ಮಿಕ ಚಿಂತನೆಗಳು–ಇವುಗಳೆಲ್ಲ ಒಂದೇ ಪರಮ ಸತ್ಯದೆಡೆಗೆ ಕರೆದೊಯ್ಯುವ ಬೇರೆ ಬೇರೆ ಪಥಗಳು ಎಂಬುದನ್ನು ವಿವರಿಸಿ ಹೇಳಿ ದರು. ಒಂದೇ ಗುರಿಯನ್ನು ಸಂನ್ಯಾಸಿಗಳೂ ಗೃಹಸ್ಥರೂ ಬೇರೆಬೇರೆ ವಿಧಾನಗಳಿಂದ ಹೊಂದ ಬಹುದು ಎಂಬುದನ್ನು ತೋರಿಸಿಕೊಟ್ಟರು.

ಖೇತ್ರಿಯಲ್ಲಿ ಸ್ವಾಮೀಜಿ ಅರಮನೆಯಲ್ಲಿ ವಾಸವಾಗಿರದಿದ್ದರೂ ರಾಜನೊಂದಿಗೆ ನಿಕಟ ಸಂಪರ್ಕವನ್ನಿಟ್ಟುಕೊಂಡಿದ್ದರು. ಕೆಲವು ದಿನಗಳ ಅನಂತರ ಅವರು ಅಜಿತ್​ಸಿಂಗನಿಗೆ ಮಂತ್ರ ದೀಕ್ಷೆ ನೀಡಿ ಅನುಗ್ರಹಿಸಿದರು. ಅಲ್ಲದೆ ತಮಗೆ ಆಪ್ತರಾಗಿದ್ದ ಮುನ್ಷಿ ಜಗಮೋಹನಲಾಲ್ ಹಾಗೂ ಇತರ ಆಸ್ಥಾನಿಕರಿಗೂ ಮಂತ್ರದೀಕ್ಷೆ ನೀಡಿದರು. ಸ್ವಾಮೀಜಿಯ ಸತ್ಸಂಗದಲ್ಲಿ ಮಹಾ ರಾಜ, ದಿನಗಳನ್ನು ಅತ್ಯಂತ ಆನಂದದಿಂದ ಕಳೆಯಲಾರಂಭಿಸಿದ. ಪ್ರತಿದಿನ ಇಬ್ಬರೂ ಕುಳಿತು ಆಧ್ಯಾತ್ಮಿಕ ವಿಷಯಗಳ ಬಗ್ಗೆ ಗಂಟೆಗಟ್ಟಲೆ ವಿಚಾರ ಮಾಡುತ್ತಿದ್ದರು; ಹಾಡೂ ಹೇಳುತ್ತಿದ್ದರು. ಕೆಲವೊಮ್ಮೆ ಕುದುರೆಗಳನ್ನೇರಿ ಸಮೀಪದ ಪುಣ್ಯಕ್ಷೇತ್ರಗಳಿಗೆ ಹೋಗಿ ಬರುತ್ತಿದ್ದರು. ಮಹಾರಾಜ ತನ್ನ ಗುರುವಿನ ಬಗ್ಗೆ ಅತ್ಯಂತ ಪೂಜ್ಯಭಾವವನ್ನಿಟ್ಟುಕೊಂಡಿದ್ದ. ಸ್ವಾಮೀಜಿಯ ವೈಯಕ್ತಿಕ ಸೇವೆಯನ್ನು ಮಾಡುವುದೆಂದರೆ ಅವನಿಗೆ ಇನ್ನಿಲ್ಲದ ಆನಂದ. ಎಷ್ಟೋ ಸಲ ಸಭಿಕರೆದುರಿಗೇ ಅವರಿಗೆ ಗಾಳಿ ಬೀಸಲು, ಅಥವಾ ಇನ್ನಿತರ ಉಪಚಾರವನ್ನು ಮಾಡಲು ತೊಡಗುತ್ತಿದ್ದ. ಇಲ್ಲವೆ ಅವರು ಅರಮನೆಯಲ್ಲಿ ಮಲಗಿದ್ದಾಗ ಅವರ ಅರಿವಿಗೇ ಬಾರದಂತೆ ಅವರ ಪಾದಗಳನ್ನು ಒತ್ತುತ್ತಿದ್ದ. ಹಾಗೆಲ್ಲ ಮಾಡಬಾರದೆಂದು ಸ್ವಾಮೀಜಿ ಎಷ್ಟೇ ಹೇಳಿದರೂ, ತನ್ನ ಪಾಲಿನ ಈ ವಿಶೇಷ ಭಾಗ್ಯವನ್ನು ತನ್ನಿಂದ ಕಿತ್ತುಕೊಳ್ಳಬಾರದೆಂದು ಮಹಾರಾಜ ಬೇಡಿಕೊಂಡು ಅವರನ್ನು ಸುಮ್ಮನಾಗಿಸುತ್ತಿದ್ದ.

ಅಜಿತ್​ಸಿಂಗನ ಹೃದಯವಂತಿಕೆಯನ್ನು, ಪ್ರಾಮಾಣಿಕ ಪ್ರೇಮವನ್ನು ಮನಗಂಡ ಸ್ವಾಮೀಜಿ ಅವನತ್ತ ತಮ್ಮ ಪ್ರೀತಿಯ ಹೊನಲನ್ನೇ ಹರಿಸಿದರು. ಅವನನ್ನು ತಮ್ಮ ಅಂತರಂಗ ಶಿಷ್ಯರ ಲ್ಲೊಬ್ಬನನ್ನಾಗಿ ಸ್ವೀಕರಿಸಿದರು. ಅವನು ತಮ್ಮಲ್ಲಿ ಅಗಾಧ ಭಕ್ತಿ-ಪ್ರೇಮಗಳನ್ನಿರಿಸಿದ್ದಂತೆಯೇ ಸ್ವಾಮೀಜಿಯೂ ಅವನಲ್ಲಿ ಗಾಢ ವಿಶ್ವಾಸವನ್ನು ಹೊಂದಿದ್ದರು. ಮುಂದೆ ಅವರು ಅಮೆರಿಕೆಗೆ ಹೋದಾಗ ಅವರಿಂದ ಪತ್ರಗಳನ್ನು ಪಡೆಯುತ್ತಿದ್ದ ಕೆಲವೆ ಮಂದಿ ಭಾಗ್ಯಶಾಲಿ ಭಾರತೀಯರಲ್ಲಿ ರಾಜಾ ಅಜಿತ್​ಸಿಂಗನೂ ಒಬ್ಬನಾಗಿದ್ದ. ಅಲ್ಲದೆ ತಮ್ಮ ಅತಿ ಮುಖ್ಯವಾದ ‘ವೈಯಕ್ತಿಕ’ ಆವಶ್ಯಕತೆಗಳಿಗೆ ಸ್ವಾಮೀಜಿ ಈತನ ನೆರವನ್ನು ಪಡೆದುಕೊಂಡರು. ಸ್ವಾಮೀಜಿಯನ್ನು ಮಾತ್ರ ವಲ್ಲದೆ, ಅವರಿಗೆ ಸಂಬಂಧ ಪಟ್ಟ ಪ್ರತಿಯೊಬ್ಬ ವ್ಯಕ್ತಿಯನ್ನೂ ಮಹಾರಾಜ ಅತ್ಯಂತ ಗೌರವಾದರ ಗಳಿಂದ ನೋಡಿಕೊಂಡು ಧನ್ಯನಾದ.

ಅಜಿತ್​ಸಿಂಗ್ ಕ್ರಮಬದ್ಧ ಶಿಕ್ಷಣವನ್ನು ಪಡೆದವನಾಗಿರಲಿಲ್ಲ. ಸಂಸ್ಕೃತ-ಇಂಗ್ಲಿಷ್​ಗಳಲ್ಲಿ ಪರಿಣತಿಯನ್ನು ಹೊಂದಿದ್ದರೂ ಇತರ ಹಲವಾರು ವಿಷಯಗಳ ಪರಿಚಯವೇ ಅವನಿಗಿರಲಿಲ್ಲ. ಇದನ್ನು ಕಂಡು ಸ್ವಾಮೀಜಿ ಅವನ ಶಿಕ್ಷಣದ ಹೊಣೆಯನ್ನು ತಾವೇ ಹೊತ್ತರು. ಅವನಿಗೆ ಭೌತಶಾಸ್ತ್ರ, ರಸಾಯನಶಾಸ್ತ್ರ, ಜೀವಶಾಸ್ತ್ರ ಹಾಗೂ ಖಗೋಳವಿಜ್ಞಾನಗಳ ಮೂಲಭೂತ ತತ್ತ್ವ ಗಳನ್ನು ತಾವೇ ವಿವರಿಸಿದರು. ಸ್ವಾಮೀಜಿಯ ಸಲಹೆಯಂತೆ ಅರಮನೆಯ ಮಾಳಿಗೆಯ ಮೇಲೆ ಒಂದು ದೂರದರ್ಶಕವನ್ನು ಸ್ಥಾಪಿಸಲಾಯಿತು. ಅಲ್ಲದೆ ಒಂದು ಸೂಕ್ಷ್ಮದರ್ಶಕವನ್ನೂ ಖರೀದಿಸಿ, ಪ್ರಯೋಗಶಾಲೆಯೊಂದನ್ನು ನಿರ್ಮಿಸಲಾಯಿತು. ಗುರುಶಿಷ್ಯರಿಬ್ಬರೂ ಈ ಉಪಕರಣಗಳೊಂದಿಗೆ ಹಲವಾರು ಗಂಟೆಗಳ ಕಾಲ ಅಧ್ಯಯನದಲ್ಲಿ ತೊಡಗಿರುತ್ತಿದ್ದರು.

ಮಹಾರಾಜ ಅಜಿತ್​ಸಿಂಗನ ವಿಷಯದಲ್ಲಿ ಸ್ವಾಮೀಜಿ ಇಂತಹ ವಿಶೇಷ ಆಸಕ್ತಿ ವಹಿಸಿದುದಕ್ಕೆ ವಿಶೇಷ ಕಾರಣವೇ ಇತ್ತು. ಈತನ ಪ್ರಾಮಾಣಿಕತೆ, ಪವಿತ್ರತೆ, ಶ್ರದ್ಧೆಗಳನ್ನು ಅವರು ಮನ ಗಂಡಿದ್ದರು. ಇವನನ್ನು ಸರಿಯಾದ ರೀತಿಯಲ್ಲಿ ದಾರಿ ತೋರಿ ಮುನ್ನಡೆಸಿದರೆ ಈತ ತನ್ನ ರಾಜ್ಯ ವನ್ನೇ ಪ್ರಗತಿಯ ಹಾದಿಯಲ್ಲಿ ಕೊಂಡೊಯ್ಯಲು ಸಮರ್ಥನಾಗುತ್ತಾನೆ; ಈತನಿಗೆ ಶಿಕ್ಷಣದ ಮೂಲಕ ಜ್ಞಾನದ ಹಲವಾರು ಮುಖಗಳ ಪರಿಚಯ ಮಾಡಿಸಿದರೆ ತನ್ನ ಪ್ರಜೆಗಳೆಲ್ಲರಿಗೂ ಯೋಗ್ಯ ಶಿಕ್ಷಣವನ್ನು ನೀಡುವವನಾಗುತ್ತಾನೆ ಎಂಬುದು ಅವರ ಮುಂದಾಲೋಚನೆಯಾಗಿತ್ತು. ಹೀಗೆ ಉನ್ನತ ಅಧಿಕಾರದಲ್ಲಿರುವ ಕೆಲವು ವ್ಯಕ್ತಿಗಳನ್ನು ಪರಿವರ್ತಿಸಿ ಸರಿದಾರಿಗೆ ತರುವುದರಿಂದ ಸಮಸ್ತ ರಾಷ್ಟ್ರದ ಏಳಿಗೆಯನ್ನೇ ಸಾಧಿಸುವ ಮಹೋದ್ದೇಶ ಅವರದಾಗಿತ್ತು.

ಅಜಿತ್​ಸಿಂಗನ ಬಗೆಗೆ ಅವರು ಎಂತಹ ವಿಶ್ವಾಸವನ್ನಿರಿಸಿದ್ದರೆಂಬುದನ್ನು ಅವರು ತಮ್ಮ ನೆಚ್ಚಿನ ಶಿಷ್ಯೆಯಾದ ಓಲೇಬುಲ್ ಎಂಬವಳಿಗೆ ೧೮೯೫ರಲ್ಲಿ ಬರೆದ ಪತ್ರವೊಂದರಲ್ಲಿ ಕಾಣ ಬಹುದು. ಸ್ವಾಮೀಜಿ ಬರೆಯುತ್ತಾರೆ–“ಭಾರತದಲ್ಲಿ ಖೇತ್ರಿಯ ಮಹಾರಾಜ ಅಜಿತ್​ಸಿಂಗ್ ಮತ್ತು ಅಮೆರಿಕದಲ್ಲಿ ಶ್ರೀಮತಿ ಓಲೇಬುಲ್​–ಇವರಿಬ್ಬರು, ನಾನು ಸದಾ ನೆಚ್ಚಿಕೊಳ್ಳಬಹು ದಾದ ವ್ಯಕ್ತಿಗಳು. ಇಡೀ ಪ್ರಪಂಚದಲ್ಲೇ ನನಗಿರುವ ಸ್ನೇಹಿತರಲ್ಲೆಲ್ಲ ನೀವಿಬ್ಬರು ನಿಮ್ಮ ಕಾರ್ಯೋದ್ದೇಶದಲ್ಲಿ ತೋರುವ ಅವಿಚಲ ಶ್ರದ್ಧೆ ಹಾಗೂ ಶಾಂತ ಸ್ವಭಾವಗಳು ಅತ್ಯದ್ಭುತವಾದುವು... ”

ಹಲವಾರು ವರ್ಷಗಳ ನಂತರ, ರಾಜಾ ಅಜಿತ್​ಸಿಂಗ್ ನೀಡಿದ ನಿರಂತರ ಸಹಾಯವನ್ನು ಸ್ಮರಿಸುತ್ತ ಅವರೆನ್ನುತ್ತಾರೆ–

“ಕೆಲವೊಂದು ನಿರ್ದಿಷ್ಟ ಸನ್ನಿವೇಶಗಳಲ್ಲಿ ಕೆಲವು ನಿರ್ದಿಷ್ಟ ವ್ಯಕ್ತಿಗಳು, ನಿರ್ದಿಷ್ಟ ಕಾರ್ಯ ಗಳನ್ನು ಮಾಡಲು ಜನ್ಮವೆತ್ತುತ್ತಾರೆ. ಮಾನವತೆಯ ಒಳತಿಗಾಗಿ ಕೈಗೊಳ್ಳುವ ಮಹಾಕಾರ್ಯದಲ್ಲಿ ಪರಸ್ಪರ ಸಹಕರಿಸಲು ಹುಟ್ಟಿದವರು ನಾನು ಮತ್ತು ಅಜಿತ್​ಸಿಂಗ್. ನಾವಿಬ್ಬರು ಪರಸ್ಪರ ಪೂರಕ-ಪ್ರೇರಕ.”

ಮತ್ತೊಮ್ಮೆ ಹೇಳುತ್ತಾರೆ:

“ನಾನು ಅಜಿತ್​ಸಿಂಗನ್ನು ಸಂಧಿಸದೆ ಹೋಗಿದ್ದರೆ, ಭಾರತದ ಏಳಿಗೆಗಾಗಿ ನಾನು ಯಾವ ಕಿಂಚಿತ್ ಕಾರ್ಯವನ್ನು ಮಾಡಿದೆನೋ ಅದು ಸಾಧ್ಯವಾಗುತ್ತಿರಲಿಲ್ಲ.”

ಸ್ವಾಮೀಜಿಯ ಈ ಮಾತುಗಳು ಉತ್ಪ್ರೇಕ್ಷೆಯಿಂದ ಕೂಡಿದವುಗಳಲ್ಲ. ಇದನ್ನು ಅವರ ಜೀವನದಲ್ಲಿ ಮುಂದೆ ನಾವೇ ಕಾಣಬಹುದು.

ಖೇತ್ರಿಯಲ್ಲಿ ಅವರು, ಮಹಾರಾಜ ಹಾಗೂ ಅವನ ವಿಶೇಷ ಅತಿಥಿಗಳೊಂದಿಗೆ ಮತ್ತು ರಾಜ್ಯದ ಅಧಿಕಾರಿಗಳೊಂದಿಗೆ ವಿವಿಧ ವಿಷಯಗಳ ಬಗ್ಗೆ ಚರ್ಚಿಸುತ್ತಿದ್ದರು. ಕೆಲವೊಮ್ಮೆ ತಮ್ಮ ಸುಮಧುರ ಕಂಠದಿಂದ ಹಾಡಿ ಎಲ್ಲರನ್ನೂ ಆನಂದಿತರನ್ನಾಗಿಸುತ್ತಿದ್ದರು. ಇಲ್ಲವೆ ಅಧ್ಯಯನ- ಪ್ರವಚನಗಳಲ್ಲಿ ತೊಡಗಿರುತ್ತಿದ್ದರು.

ಅಜಿತ್​ಸಿಂಗನೊಂದಿಗೆ ಸ್ವಾಮೀಜಿ, ಸಿಖರ್ ಪ್ರಾಂತದಲ್ಲಿರುವ ರಜಪುತಾನದ ಪ್ರಸಿದ್ಧ ದೇವತೆ ಜಿನಮಾತಾ ದೇವಾಲಯವನ್ನು ಸಂದರ್ಶಿಸಿದರು. ವಿಜಯದಶಮಿಯ ದಿನದಂದು ಖೇತ್ರಿ ಯಲ್ಲಿ ವಿಶೇಷ ಪೂಜೆ, ಮೆರವಣಿಗೆಗಳನ್ನೂ ಔತಣಕೂಟವನ್ನೂ ಏರ್ಪಡಿಸಲಾಗಿತ್ತು. ಸ್ವಾಮೀಜಿಯ ಸಾನ್ನಿಧ್ಯದಿಂದಾಗಿ ಈ ಕಾರ್ಯಕ್ರಮಗಳಿಗೆಲ್ಲ ಒಂದು ದಿವ್ಯಕಳೆ ಬಂದಿತ್ತು.

ಖೇತ್ರಿಯಲ್ಲಿ ಸ್ವಾಮೀಜಿ ಯಾವಾಗಲೂ ಮಹಾರಾಜನ ಮತ್ತು ಅಧಿಕಾರಿಗಳ ಜೊತೆಯಲ್ಲೇ ಇರುತ್ತಿದ್ದರೆಂದಲ್ಲ. ಅವರ ಹೆಸರು ಆ ಸ್ಥಳದಲ್ಲೆಲ್ಲ ಮನೆಮಾತಾಗಿದ್ದು, ಅವರ್ನು ನೋಡಲು ಜನ ಧಾವಿಸಿ ಬರುತ್ತಿದ್ದರು. ಇವರಲ್ಲಿ ಅನೇಕರು ಅವರ ಅನುಯಾಯಿಗಳಾದರು. ಆಗಾಗ ಸ್ವಾಮೀಜಿ ತಮ್ಮ ಕೆಲವು ಸಾಧಾರಣ ವರ್ಗದ ಭಕ್ತರ ಮನೆಗಳಿಗೂ ಹೋಗುತ್ತಿದ್ದರು. ತಮ್ಮ ದರ್ಶನಾರ್ಥಿಗಳಾಗಿ ಬರುವ ಅತ್ಯಂತ ದೀನ-ದರಿದ್ರರನ್ನೂ ತಾವು ಅಜಿತ್​ಸಿಂಗನನ್ನು ಕಾಣು ತ್ತಿದ್ದಷ್ಟೇ ಪ್ರೀತಿ ವಿಶ್ವಾಸಗಳಿಂದ ಕಾಣುತ್ತಿದ್ದರು. ಪಂಡಿತ ಶಂಕರಲಾಲ್ ಎಂಬ ಬಡ ಬ್ರಾಹ್ಮಣ ನೊಬ್ಬನ ಮನೆಗೆ ಕೆಲವೊಮ್ಮೆ ಹೋಗಿ ಅತಿಥ್ಯ ಸ್ವೀಕರಿಸುತ್ತಿದ್ದರು.

ಖೇತ್ರಿಯ ಅರಮನೆಯಲ್ಲಿ ಸ್ವಾಮೀಜಿಗೆ ಪಂಡಿತ ನಾರಾಯಣದಾಸ್ ಎಂಬ ಪ್ರಸಿದ್ಧ ಸಂಸ್ಕೃತ ವ್ಯಾಕರಣ ಪಂಡಿತನ ಪರಿಚಯವಾಯಿತು. ಇದೊಂದು ಸುಯೋಗವೆಂದು ಭಾವಿಸಿದ ಸ್ವಾಮೀಜಿ, ಇವನಿಂದ ‘ಮಹಾಭಾಷ್ಯ’ವನ್ನು ಕಲಿಯುವುದರ ಮೂಲಕ, ತಾವು ಜೈಪುರದಲ್ಲಿ ಪ್ರಾರಂಭಿಸಿದ್ದ ವ್ಯಾಕರಣಭ್ಯಾಸವನ್ನು ಮುಂದುವರಿಸಲು ನಿಶ್ಚಯಿಸಿದರು. (‘ಮಹಾಭಾಷ್ಯ’ವು ಪಾಣಿನಿಯ ವ್ಯಾಕರಣ ಸೂತ್ರಗಳ ಮೇಲೆ ಪತಂಜಲಿ ಮಹರ್ಷಿಗಳು ರಚಿಸಿದ ಭಾಷ್ಯಗ್ರಂಥ.) ಸ್ವಾಮೀಜಿ ಯಂತಹ ಅಸಾಮಾನ್ಯ ವ್ಯಕ್ತಿ ತನ್ನಿಂದ ಕಲಿಯುವ ಇಚ್ಛೆ ವ್ಯಕ್ತಪಡಿಸಿದುದನ್ನು ಕಂಡು ಪಂಡಿತನಿಗೆ ಬಹಳ ಸಂತೋಷವಾಯಿತು ಅವನೂ ಉತ್ಸಾಹದಿಂದ ಪಾಠ ಹೇಳಿಕೊಡಲು ಪ್ರಾರಂಭಿಸಿದ. ಮೊದಲ ದಿನವೇ ಅವರ ಅದ್ಭುತ ಗ್ರಹಣಸಾಮರ್ಥ್ಯವನ್ನು ಕಂಡು ಬೆರಗಾಗಿ, ಆತ ಉದ್ಗರಿಸಿದ–“ಸ್ವಾಮೀಜಿ, ನಿಮ್ಮಂತಹ ವಿದ್ಯಾರ್ಥಿ ದೊರಕುವುದು ತುಂಬ ಅಪರೂಪವೇ ಸರಿ!” ಒಂದು ದಿನ ಪಂಡಿತ ನಾರಾಯಣದಾಸ್, ತಾನು ಹಿಂದಿನ ದಿನ ಮಾಡಿದ್ದ ಸುದೀರ್ಘ ಪಾಠ ವೊಂದರ ಕುರಿತಾಗಿ ಪ್ರಶ್ನಿಸಲಾಗಿ, ಸ್ವಾಮೀಜಿ ಅದನ್ನು ಒಂದಕ್ಷರವೂ ಬಿಡದಂತೆ ಒಪ್ಪಿಸಿ, ಜೊತೆಗೆ ವ್ಯಾಖ್ಯಾನವನ್ನೂ ಸೇರಿಸಿ ವಿವರಿಸಿದರು! ಪಂಡಿತ ವಿಸ್ಮಯಮೂಕನಾದ. ಕೆಲವು ದಿನಗಳ ಮೇಲಂತೂ, ತಮ್ಮ ಕೆಲವು ಪ್ರಶ್ನೆಗಳಿಗೆ ಪಂಡಿತನಿಂದ ಸೂಕ್ತ ಉತ್ತರ ದೊರೆಯದಿದ್ದಾಗ ಅವುಗಳಿಗೆ ಸ್ವಾಮೀಜಿ ತಾವೇ ಉತ್ತರವನ್ನು ಕಂಡುಕೊಳ್ಳಬೇಕಾಗಿ ಬಂದಿತು. ಈ ಪಂಡಿತನೋ ಸಮಸ್ತ ರಜಪುತಾನದಲ್ಲೇ ಶ್ರೇಷ್ಠ ವೈಯಾಕರಣಿಯೆಂದು ಪ್ರಸಿದ್ಧನಾದವನು. ಅವನನ್ನೇ ಸ್ವಾಮೀಜಿ ಮೀರಿಸಿದರೆಂದ ಮೇಲೆ...! ಕೊನೆಗೊಂದು ದಿನ ಅವನು, “ಸ್ವಾಮೀಜಿ, ಇನ್ನು ನಾನು ನಿಮಗೆ ಕಲಿಸಬೇಕಾದದ್ದೇನೂ ಇಲ್ಲ. ನನಗೆ ತಿಳಿದಿರುವುದನ್ನೆಲ್ಲ ನಿಮಗೆ ಕಲಿಸಿದ್ದೇನೆ, ಮತ್ತು ನೀವೂ ಅವುಗಳನ್ನೆಲ್ಲ ಅರಗಿಸಿಕೊಂಡಿದ್ದೀರಿ” ಎಂದುಬಿಟ್ಟ. ಪಂಡಿತನ ಉಪಕಾರಕ್ಕಾಗಿ ಕೃತಜ್ಞತೆಯನ್ನು ವ್ಯಕ್ತಪಡಿಸಿದ ಸ್ವಾಮೀಜಿ, ಗೌರವಪೂರ್ವಕ ನಮಸ್ಕಾರ ಸಲ್ಲಿಸಿದರು;ಹಲವಾರು ವಿಷಯಗಳಲ್ಲಿ ತಾವೇ ಆತನ ಗುರುವಾದರು.

ಒಂದು ದಿನ ಸಂಭಾಷಣೆಯ ಸಂದರ್ಭದಲ್ಲಿ ಮಹಾರಾಜ ಕೇಳುತ್ತಾನೆ–

“ಸ್ವಾಮೀಜಿ, ನಿಯಮ \eng{(Law)} ಎಂದರೇನು?”

ಇಲ್ಲಿ ಮಹಾರಾಜ ಪ್ರಸ್ತಾಪಿಸಿದ ‘ನಿಯಮ’ ಎಂಬುದು ರಾಜ್ಯಾಂಗದ ನೀತಿ-ನಿಯಮಾವಳಿಗಳ ಕುರಿತಾದದ್ದಲ್ಲ; ಅಥವಾ ನಮ್ಮ ಆಚಾರ-ವ್ಯವಹಾರಗಳ ಸಂಬಂಧವಾಗಿ ನಾವೇ ಹಾಕಿಕೊಂಡ ಕಟ್ಟುಪಾಡುಗಳಿಗೆ ಸಂಬಂಧಿಸಿದುದಲ್ಲ. ‘ದೈವನಿಯಮ’ ಎಂದು ಕರೆಯಬಹುದಾದ ವಿಷಯಕ್ಕೆ ಸಂಬಂಧಿಸಿದಂತೆ ಅವನ ಆ ಪ್ರಶ್ನೆ. ಸಮಸ್ತ ವಿಶ್ವವನ್ನೂ ನಿಯಂತ್ರಿಸುತ್ತಿರುವ ಆ ‘ನಿಯಮ’ ಎಂಬುದು ಏನು ಎಂದು ಮಹಾರಾಜ ಪ್ರಶ್ನಿಸುತ್ತಿದ್ದಾನೆ. ಸ್ವಲ್ಪವೂ ಅನುಮಾನಿಸದೆ ಸ್ವಾಮೀಜಿ ಉತ್ತರಿಸಿದರು–

“ನಿಯಮ ಎನ್ನುವುದು ಸಂಪೂರ್ಣ ಆಂತರಿಕ. ಅದು ನಮ್ಮಿಂದ ಹೊರಗೆ, ನಮ್ಮಿಂದ ಹೊರತಾಗಿ ಇರುವಂತಹದಲ್ಲ. ಅದೊಂದು ಬುದ್ಧಿ-ಅನುಭವಗಳ ಪ್ರಕ್ರಿಯೆ. ಇಂದ್ರಿಯಾನುಭವ ಗಳನ್ನು ವರ್ಗೀಕರಿಸಿ ನಿಯಮಗಳನ್ನು ರೂಪಿಸುವುದು ಈ ಮನಸ್ಸೇ. ಅನುಭವದ ಕ್ರಮ ಯಾವಾಗಲೂ ಆಂತರಿಕ, ಎಂದರೆ ನಮ್ಮ ಅಂತರಂಗದಲ್ಲಿ ನಡೆಯುವ ವಿಚಾರ. ಇಂದ್ರಿಯಗಳ ಮೂಲಕ ನಿರಂತರವಾಗಿ ಪಡೆಯಲ್ಪಡುವ ಅನುಭವಗಳು ಹಾಗೂ ಅವುಗಳಿಗೆ ಕ್ರಮಾಗತವಾಗಿ ಬುದ್ಧಿಯು ತೋರುವ ಪ್ರತಿಕ್ರಿಯೆ–ಇದನ್ನೇ ‘ನಿಯಮ’ ಎನ್ನುವುದು. ಇದನ್ನುಳಿದಂತೆ ಇನ್ನಾ ವುದೇ ನಿಯಮ ಎನ್ನುವಂತಹದು ಇಲ್ಲ. ವಿಜ್ಞಾನಿಗಳು ಹೇಳುತ್ತಾರೆ–ಇವೆಲ್ಲ ಏಕರೂಪದ ಸಂಬಂಧ–ಸ್ಪಂದನಗಳನ್ನುಳ್ಳವು ಎಂದು. ಅನುಭವಗಳು ಮತ್ತು ಅವುಗಳ ವರ್ಗೀಕರಣವೆಲ್ಲ ಸಂಪೂರ್ಣವಾಗಿ ಆಂತರಿಕ ಪ್ರಕ್ರಿಯೆ. ಹೀಗೆ ನಿಯಮ ಎಂಬುದು ಬುದ್ಧಿಪೂರ್ವಕವಾಗಿದ್ದು, ಬುದ್ಧಿಯಿಂದ ಸಂಶೋಧಿತವಾದದ್ದಾಗಿರುತ್ತದೆ” ಬಳಿಕ ಸ್ವಾಮೀಜಿ ಸಾಂಖ್ಯತತ್ತ್ವದ ಕುರಿತಾಗಿ ಹೇಳಿ, ಅದರ ನಿರ್ಣಯಗಳು ಹೇಗೆ ಆಧುನಿಕ ವಿಜ್ಞಾನದೊಂದಿಗೆ ಹೊಂದಿಕೆಯಾಗುತ್ತವೆ ಎಂಬುದನ್ನು ವಿವರಿಸಿದರು.

ಖೇತ್ರಿಯಲ್ಲಿ ಸ್ವಾಮೀಜಿಯ ಕಣ್ಣು ತೆರೆಸಿದ ಹೃದಯಸ್ಪರ್ಶಿ ಘಟನೆಯೊಂದು ನಡೆಯಿತು. ಒಂದು ಸಂಜೆ ಮಹಾರಾಜನ ಆಸ್ಥಾನದಲ್ಲಿ ಗಾಯಕಿಯೊಬ್ಬಳ ಸಂಗೀತದ ಕಾರ್ಯಕ್ರಮ ನಡೆಯುತ್ತಿತ್ತು. ಆಕೆ ವೇಶ್ಯೆಯರ ವರ್ಗಕ್ಕೆ ಸೇರಿದವಳು. ಅವಳ ಕಂಠಸ್ವರವೂ ಸಂಗೀತವೂ ಅತ್ಯಂತ ಮಧುರವಾಗಿದ್ದು, ಮಹಾರಾಜನಿಗೆ ಆಕೆಯ ಸಂಗೀತದ ಮೇಲೆ ವಿಶೇಷ ಅಭಿಮಾನ ವಿತ್ತು. ಸಂಗೀತ ನಡೆಯುತ್ತಿದ್ದಂತೆ, ಈ ಕಾರ್ಯಕ್ರಮದಲ್ಲಿ ತಮ್ಮೊಂದಿಗೆ ಭಾಗವಹಿಸುವಂತೆ ಸ್ವಾಮೀಜಿಗೆ ಹೇಳಿ ಕಳಿಸಿದ. ಆಗ ಅವರು ಅಲ್ಲೇ ಸಮೀಪದಲ್ಲಿದ್ದ ತಮ್ಮ ಕೋಣೆಯಲ್ಲಿದ್ದರು. ದೂತ ಬಂದು ಕರೆದಾಗ ಸ್ವಾಮೀಜಿ, “ನೋಡು, ನಾನೊಬ್ಬ ಸಂನ್ಯಾಸಿಯಾಗಿ ಇಂತಹ ಕಾರ್ಯ ಕ್ರಮಗಳಿಗೆಲ್ಲ ಬರುವಂತಿಲ್ಲ” ಎಂದು ಹೇಳಿದರು. ಸ್ವಾಮೀಜಿ ಬರುವುದಿಲ್ಲವೆಂದು ತಿಳಿದು ಬಂದಾಗ ಗಾಯಕಿಗೆ ತೀವ್ರ ನಿರಾಶೆಯೂ ದುಃಖವೂ ಉಮ್ಮಳಿಸಿ ಬಂತು. ಅವರ ಉತ್ತರಕ್ಕೆ ಪ್ರತಿಕ್ರಿಯೆಯೋ ಎಂಬಂತೆ ತನ್ನ ಅಂತರಂಗದ ಭಾವಗಳನ್ನು ಹೊರಹೊಮ್ಮಿಸುವ, ಸೂರದಾಸನ ಹಾಡೊಂದನ್ನು ಹಾಡಲಾರಂಭಿಸಿದಳು:

\begin{verse}
ಪ್ರಭು ಮೇರೇ ಅವಗುಣ ಚಿತ ನ ಧರೋ ॥
\end{verse}

\begin{verse}
ಸಮದರಶೀ ಹೈ ನಾಮ ತಿಹಾರೋ\\ಚಾಹೇ ತೋ ಪಾರ ಕರೋ ॥
\end{verse}

\begin{verse}
ಇಕ ಲೋಹ ಪೂಜಾ ಮೇ ರಾಖತ\\ಇಕ ರಹತ ವ್ಯಾಧ ಘರ ಪರೋ ।\\ಪಾರಸ ಕೇ ಮನ ದ್ವಿಧಾ ನಹೀ ಹೈ\\ದುಹು ಏಕ ಕಾಂಚನ ಕರೋ ॥
\end{verse}

\begin{verse}
ಇಕ ನದಿಯಾ ಇಕ ನಾರ ಕಹಾವತ\\ಮೈಲೋ ನೀರ ಭರೋ ।\\ಜಬ ಮಿಲಿ ದೋನೋ ಏಕ ವರಣ ಭಯೇ\\ಸುರಸರೀ ನಾಮ ಪರೋ ॥
\end{verse}

\begin{verse}
ಇಕ ಜೀವ ಇಕ ಬ್ರಹ್ಮ ಕಹಾವತ\\ಸೂರದಾಸ ಝಗರೋ ।\\ಅಜ್ಞಾನಸೇ ಭೇದ ಹೋವೇ\\ಜ್ಞಾನಿ ಕಾಹೇ ಭೇದ ಕರೋ ॥\footnote{* ಭಾವಾನುವಾದಕ್ಕೆ ನೋಡಿ: ಅನುಬಂಧ ೧.}
\end{verse}

ಮರ್ಮಭೇದಕವಾದ ಈ ಹಾಡಿನ ಸಾಲುಗಳು ನರ್ತಕಿಯ ಮಧುರ ಕಂಠಧ್ವನಿಯ ಮೂಲಕ ಸಂಜೆಯ ಗಾಳಿಯಲ್ಲಿ ತೇಲಿಬಂದು ಸ್ವಾಮೀಜಿಯ ಕಿವಿಗಳನ್ನು ಸೋಂಕಿದುವು. ಸಿಡಿಲು ಬಡಿ ದಂತಾಗಿ ಸ್ವಾಮೀಜಿ ಧಿಗ್ಗನೆದ್ದು ನಿಂತರು. ‘ಸಮದರ್ಶೀ ಹೈ ನಾಮ ತಿಹಾರೋ’ ‘ಸಮದರ್ಶಿ ಯಂತೆ ನಿನ್ನ ಹೆಸರು!’ ‘ಇಕ ಜೀವ ಇಕ ಬ್ರಹ್ಮ ಕಹಾವತ್​’ ‘ಒಂದನ್ನು ಜೀವವೆಂದು, ಮತ್ತೊಂದನ್ನು ಬ್ರಹ್ಮವೆಂದು ಕರೆಯುತ್ತಾರೆ’ ‘ಅಜ್ಞಾನ ಸೇ ಭೇದ ಹೋವೇ’ ‘ಅಜ್ಞಾನದಿಂದ ಈ ಭೇದವುಂಟಾಗುತ್ತದೆ ’...! ಹೌದು; ಈ ಸಾಲುಗಳು ಅವರಿಗೆ ತತ್ಕಾಲಕ್ಕೆ ಮರೆತುಹೋಗಿದ್ದ ಸಂಗತಿಯೊಂದನ್ನು ನೆನಪಿಸಿಕೊಟ್ಟವು. ಏನದು? ಸರ್ವಂ ಖಲ್ವಿದಂ ಬ್ರಹ್ಮ–ಸಕಲವೂ ಬ್ರಹ್ಮವೇ ಅಲ್ಲವೆ? ಎಲ್ಲ ವ್ಯಕ್ತಿಗಳ ಹಿಂದೆಯೂ ಬೆಳಗುತ್ತಿರುವುದು ಅದೇ ಶಕ್ತಿಯೇ ಅಲ್ಲವೇ? ತಾನು ಸಂನ್ಯಾಸಿ, ಪವಿತ್ರ; ಈಕೆ ವೇಶ್ಯೆ, ಅಪವಿತ್ರಳು–ಎಂಬ ಭಾವನೆಯಿಂದಲ್ಲವೆ ತಾನು ಅವಳ ಹಾಡನ್ನು ಕೇಳಲು ಹೋಗದಿದ್ದುದು? ಈ ವೇಶ್ಯೆಯ ಹಿಂದೆಯೂ ಬೆಳಗುತ್ತಿರುವುದು ತನ್ನಲ್ಲಿ ರುವ ಆ ಪರಬ್ರಹ್ಮವೇ ಅಲ್ಲವೆ? ಪಶ್ಚಾತ್ತಾಪದಿಂದ ಬೆಂದ ಸ್ವಾಮೀಜಿ ತಕ್ಷಣ ಸಭೆಗೆ ಬಂದು ಕುಳಿತರು. ಇದರಿಂದ ನರ್ತಕಿಗೆ ಬಹಳ ಆನಂದವಾಯಿತು. ಅಂದಿನಿಂದ ಅವರು ಆಕೆಯನ್ನು ‘ತಾಯಿ’ ಎಂದು ಕರೆಯುತ್ತಿದ್ದರು.

ಈ ಘಟನೆಯ ಕುರಿತಾಗಿ ಸ್ವತಃ ಸ್ವಾಮೀಜಿಯೇ ಮುಂದೊಮ್ಮೆ ಹೇಳುತ್ತಾರೆ– “ಆ ಹಾಡನ್ನು ಕೇಳಿದ ತಕ್ಷಣ ನನಗೆ ನಾನೇ ಹೇಳಿಕೊಂಡೆ, ‘ಇದೇಯೋ ನನ್ನ ಸಂನ್ಯಾಸ? ಸಂನ್ಯಾಸಿಯಾಗಿದ್ದು ಕೊಂಡು, ನನ್ನ ಹಾಗೂ ಈ ಹೆಂಗಸಿನ ನಡುವೆ ನಾನು ಭೇದಭಾವ ಮಾಡುತ್ತಿದ್ದೇನಲ್ಲ...!’ ಎಂದು. ಈ ಘಟನೆ ನನ್ನ ಕಣ್ಣಿನ ಪೊರೆಯನ್ನು ಕಳಚಿತು. ಪ್ರತಿಯೊಬ್ಬರೂ ಪರಬ್ರಹ್ಮನ ಆವಿರ್ಭಾವವೆಂದು ಮನಗಂಡ ನಾನು ಯಾರನ್ನೂ ತಿರಸ್ಕರಿಸಲಾರದವನಾದೆ.”

ಮುಂದೆ ಸ್ವಾಮಿ ವಿವೇಕಾನಂದರು ಜಗತ್ತಿನ ವೇದಿಕೆಯ ಮೇಲೆ ನಿಂತು, ಕೂಗಿ ಕೂಗಿ ಹೇಳುತ್ತಾರೆ–“ಯಾರನ್ನೂ ಪಾಪಿಯೆನ್ನಬೇಡ. ಪ್ರತಿಯೊಬ್ಬನೂ ಆ ನಿತ್ಯ ಶುದ್ಧ-ಬುದ್ಧ-ಮುಕ್ತ ನಾದ ಆತ್ಮನೇ. ಓ ಅಮೃತಪುತ್ರನೆ! ಈ ದಿವ್ಯ ಸತ್ಯವನ್ನು ಸಕಲರಿಗೂ ಸಾರಿ ಹೇಳು!” ಅಲ್ಲದೆ ತಾವು ಈ ಮಾತಿನಂತೆಯೇ ಅಕ್ಷರಶಃ ನಡೆದುಕೊಂಡು ಆಧ್ಯಾತ್ಮಿಕ ಇತಿಹಾಸದಲ್ಲೇ ಕ್ರಾಂತಿ ಯೊಂದನ್ನು ಉಂಟುಮಾಡಿದರು.

ಸ್ವಾಮೀಜಿ ಖೇತ್ರಿಗೆ ಬಂದು ಎರಡೂವರೆ ತಿಂಗಳಾಗಿತ್ತು. ಮತ್ತು ಈಗ ಸುಮಾರು ಐದು ತಿಂಗಳಿನಿಂದಲೂ ಮಹಾರಾಜ ಅಜಿತ್​ಸಿಂಗನ ಸಂಪರ್ಕ-ಸಹವಾಸದಲ್ಲೇ ಇದ್ದರು. ಅವರಿಗೆ ಈ ಅವಧಿಯಲ್ಲಿ ಹಲವಾರು ಅನುಭವಗಳಾಗಿದ್ದುವು;ಹಲವಾರು ಪಾಠಗಳನ್ನು ಕಲಿತಿದ್ದರು; ಮತ್ತು ಹಲವಾರು ಜನರ ಸ್ನೇಹ-ಪ್ರೇಮ-ಭಕ್ತಿ-ಆದರಗಳನ್ನು ಗಳಿಸಿದ್ದರು. ಇಡೀ ಖೇತ್ರಿಯೇ ಅವರಿಂದ ಆಕರ್ಷಿತವಾಗಿತ್ತು. ಎಲ್ಲಕ್ಕಿಂತ ಮುಖ್ಯವಾಗಿ ಅತ್ಯಂತ ಯೋಗ್ಯ ವ್ಯಕ್ತಿಯೊಬ್ಬನನ್ನು ಅವರು ತಮ್ಮವನನ್ನಾಗಿಸಿಕೊಂಡಿದ್ದರು. ಅವರ ಪರಿವ್ರಾಜಕ ಜೀವನದಲ್ಲಿ ಇದೊಂದು ಮುಖ್ಯವಾದ ಘಟ್ಟ. ಆದರೆ ತಮ್ಮ ಜೀವನೋದ್ದೇಶವನ್ನಿನ್ನೂ ಅವರು ಸ್ಫುಟವಾಗಿ ಕಂಡುಕೊಳ್ಳಬೇಕಾಗಿದೆ. ಸಾಧಿಸಬೇಕಾಗಿದೆ. ಆದ್ದರಿಂದ ಅವರಿನ್ನು ಅಲ್ಲಿ ಉಳಿದುಕೊಳ್ಳಲಾರರು. ಪರಿವ್ರಾಜಕ ಜೀವನ ವನ್ನು ಮುಂದುವರಿಸಿ ಸಾಗಬೇಕಾಗಿದೆ...\\ಸ್ವಾಮೀಜಿ ಹೊರಡುವ ಮೊದಲು, ತನಗೆ ಉತ್ತರಾಧಿಕಾರಿಯಾಗಿ ಒಂದು ಗಂಡುಮಗುವಾಗು ವಂತೆ ಅನುಗ್ರಹಿಸಬೇಕೆಂದು ಮಹಾರಾಜ ಅವರನ್ನು ಪ್ರಾರ್ಥಿಸಿಕೊಂಡ. ಆತನಿಗೆ ಇಬ್ಬರು ಹೆಣ್ಣುಮಕ್ಕಳಿದ್ದರಾದರೂ ಪುತ್ರಸಂತಾನವಿರಲಿಲ್ಲ. ಸ್ವಾಮೀಜಿ ಕೃಪೆದೋರಿದರೆ ತನ್ನ ಅಭೀಷ್ಟ ಖಂಡಿತ ಈಡೇರುವುದೆಂದು ಅವನಿಗೆ ದೃಢವಿಶ್ವಾಸ. ಆದರೆ ಸ್ವಾಮೀಜಿ ಏನೂ ಹೇಳದೆ ಮೌನವಾಗಿದ್ದುಬಿಟ್ಟರು. ಆದರೆ ಅಜಿತ್ ಸಿಂಗ್ ಬಿಡದೆ ಮತ್ತೆ ಮತ್ತೆ ಅವರನ್ನು ದೈನ್ಯದಿಂದ ಬೇಡಿಕೊಂಡ. ಕಡೆಗೆ ಅವರು ಮಹಾರಾಜನ ಪ್ರಾರ್ಥನೆಯನ್ನು ಮನ್ನಿಸಿ ಶ್ರೀರಾಮಕೃಷ್ಣರ ಹೆಸರಿನಲ್ಲಿ ಹರಿಸಿದರು. ಸ್ವಾಮೀಜಿಯ ಆಶೀರ್ವಾದಫಲದಿಂದ ಕೆಲಕಾಲದಲ್ಲೇ ಅವನು ಗಂಡು ಮಗುವೊಂದರ ತಂದೆಯಾದ.

