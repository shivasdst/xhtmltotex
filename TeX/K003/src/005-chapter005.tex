
\chapter{ಕನ್ನಡದ ನೆಲದಲ್ಲಿ}

\noindent

ಮರ್ಮಗೋವಾದಿಂದ ಹೊರಟ ಸ್ವಾಮೀಜಿ ಧಾರವಾಡದ ಮೂಲಕ ನೇರವಾಗಿ ಬೆಂಗಳೂರಿಗೆ ಬಂದು ತಲುಪಿದರು. ಬೆಂಗಳೂರಿನಲ್ಲಿ ಅವರು ಸರ್ಕಾರೀ ವೈದ್ಯಾಧಿಕಾರಿಯಾಗಿದ್ದ ಡಾ ॥ ಪಲ್ಪು ಎಂಬವರ ಅತಿಥಿಯಾಗಿ ಉಳಿದುಕೊಂಡರು. ಇವರು ಕೇರಳದ ಈಡಿಗ ಕೋಮಿಗೆ ಸೇರಿದವರು. ಇವರಿಗೆ ತಕ್ಕ ಅರ್ಹತೆಗಳಿದ್ದರೂ ಜಾತೀಯತೆಯ ಪೂರ್ವಗ್ರಹದ ದೆಸೆಯಿಂದಾಗಿ ಸ್ವಂತ ಸ್ಥಳವಾದ ಕೇರಳದಲ್ಲಿ ಸರ್ಕಾರಿ ಉದ್ಯೋಗ ಸಿಕ್ಕಿರಲಿಲ್ಲ. ಅಲ್ಲಿ ಅವರು ಸಮಾಜದ ಮೇಲ್ಜಾತಿ ಯವರ ಕಾಲಿನಡಿಯಲ್ಲಿ ಬದುಕಬೇಕಾಯಿತು. ಆದ್ದರಿಂದ ಅವರು ಮೈಸೂರು ರಾಜ್ಯಕ್ಕೆ ಬಂದು, ಯೋಗ್ಯ ಉದ್ಯೋಗವೊಂದನ್ನು ಪಡೆದುಕೊಂಡಿದ್ದರು. ಇವರ ಮೂಲಕ ಸ್ವಾಮೀಜಿಗೆ ಕೇರಳ ದಲ್ಲಿ ಪ್ರಚಲಿತವಿದ್ದ ಭಯಂಕರ ಸಾಮಾಜಿಕ ಪರಿಸ್ಥಿತಿಯ ಚಿತ್ರ ಸಿಕ್ಕಿತು. ಬೆಸ್ತರೇ ಮೊದಲಾದ ಕೆಳಜಾತಿಗಳ ಜನರು ಮೇಲ್ಜಾತಿಯವರ ತುಳಿತಕ್ಕೆ ಸಿಲುಕಿ ನರಳುತ್ತಿದ್ದರು. ಈ ಪರಿಸ್ಥಿತಿಯ ಪೂರ್ಣ ಪ್ರಯೋಜನವನ್ನು ಪಡೆದುಕೊಂಡ ಕ್ರೈಸ್ತಮತ ಪ್ರಚಾರಕರು, ಕೆಳಜಾತಿಯ ಜನರನ್ನು ಲಕ್ಷಗಟ್ಟಲೆ ಸಂಖ್ಯೆಯಲ್ಲಿ ಕ್ರೈಸ್ತರನ್ನಾಗಿ ಮತಾಂತರಿಸುತ್ತಿದ್ದರು. ಒಬ್ಬ ಶೂದ್ರನು ಹತ್ತಿರ ಬಂದರೆ ಅವನನ್ನು ಒಂದು ಸಾಂಕ್ರಾಮಿಕ ರೋಗದಂತೆ ದೂರವಿಡುತ್ತಿದ್ದ ಮೇಲ್ಜಾತಿಯ ಹಿಂದೂ ಗಳು–ಅವನು ಎಷ್ಟೇ ದರಿದ್ರನಾಗಿರಬಹುದು–ಇಗರ್ಜಿಗೆ ಹೋಗಿ ತಲೆಯ ಮೇಲೆ ಸ್ವಲ್ಪ ನೀರನ್ನೆರಚಿಸಿಕೊಂಡು ಕ್ರೈಸ್ತನೆಂಬ ಹೆಸರಿನಿಂದ ಬಂದರೆ, ಅವನ ಕೈ ಕುಲುಕಿ ಸ್ವಾಗತಿಸಿ ಉಪಚರಿಸುತ್ತಿದ್ದರು! ಇದಕ್ಕಿಂತ ಹಾಸ್ಯಾಸ್ಪದವಾದದ್ದುಂಟೆ! ಇದನ್ನೆಲ್ಲ ಪಲ್ಪುರವರೊಂದಿಗೆ ಚರ್ಚಿಸಿದ ಸ್ವಾಮೀಜಿ, ಈ ಸಮಸ್ಯೆಗೆ ತಮ್ಮದೇ ಆದ ಒಂದು ಪರಿಹಾರವನ್ನು ಸೂಚಿಸಿದರು: “ನೀವೆಲ್ಲ ಬ್ರಾಹ್ಮಣರ ಹಿಂದೆ ಅಂಗಲಾಚಿಕೊಂಡು ಹೋಗುವುದೇಕೆ? ನಿಮ್ಮಲ್ಲೇ ಉನ್ನತ ಮಟ್ಟದ ಯೋಗ್ಯ ವ್ಯಕ್ತಿಯೊಬ್ಬನನ್ನು ನಾಯಕನನ್ನಾಗಿ ಮಾಡಿಕೊಂಡು ಎಲ್ಲರೂ ಅವನನ್ನು ಅನು ಸರಿಸಿ. ಆಗ ನಿಮ್ಮ ಎಷ್ಟೋ ಸಮಸ್ಯೆಗಳು ಪರಿಹಾರವಾಗುತ್ತವೆ.” ಡಾ ॥ ಪಲ್ಪು ಈ ಸಲಹೆಯನ್ನು ತುಂಬ ಗಂಭೀರವಾಗಿಯೇ ಸ್ವೀಕರಿಸಿದರು. ಅಲ್ಲದೆ ಅಂತಹ ವ್ಯಕ್ತಿಗಾಗಿ ಅವರು ಹುಡುಕಾಟ ನಡೆಸಿ ಶ್ರೀ ನಾರಾಯಣ ಗುರು ಎಂಬವರನ್ನು ಕಂಡುಕೊಂಡರು. ಮುಂದೆ ಈ ನಾರಾಯಣ ಗುರುಗಳು ಹಿಂದುಳಿದವರ ಮುಖಂಡರಾಗಿ ಹಾಗೂ ಮಾರ್ಗದರ್ಶಕರಾಗಿ ಪ್ರಸಿದ್ಧರಾದರು. ಡಾ॥ ಪಲ್ಪು ತಮ್ಮ ಬಹಳಷ್ಟು ವೇಳೆಯನ್ನು ಹಾಗೂ ತಮ್ಮ ಆದಾಯದ ಹೆಚ್ಚಿನ ಭಾಗವನ್ನು ತಮ್ಮ ಕೋಮಿನ ಬಡವರ ಒಳಿತಿಗಾಗಿ ವಿನಿಯೋಗಿಸಿದರು.

ಬೆಂಗಳೂರಿಗೆ ಬಂದ ಕೆಲದಿನಗಳಲ್ಲೇ ಸ್ವಾಮೀಜಿಗೆ ಮೈಸೂರು ರಾಜ್ಯದ ದಿವಾನರಾದ ಸರ್ ಕೆ. ಶೇಷಾದ್ರಿ ಅಯ್ಯರ್​ರವರ ಪರಿಚಯವಾಯಿತು. ಪ್ರತಿಭೆಯನ್ನು ಗುರುತಿಸುವುದರಲ್ಲಿ ಅಯ್ಯರ್ ರವರದು ಎತ್ತಿದ ಕೈ ಎಂಬ ಪ್ರತೀತಿಯಿತ್ತು. ಕೆಲವೇ ನಿಮಿಷಗಳ ಸಂಭಾಷಣೆಯಿಂದಲೇ ದಿವಾನ ರಿಗೆ ಮನವರಿಕೆಯಾಯಿತು–ತಮ್ಮ ಮುಂದಿರುವ ಈ ಯುವಸಂನ್ಯಾಸಿ ತನ್ನ ಅಯಸ್ಕಾಂತೀಯ ವ್ಯಕ್ತಿತ್ವ ಹಾಗೂ ಅಪಾರ ಆಧ್ಯಾತ್ಮಿಕ ಶಕ್ತಿಗಳಿಂದ ಸಮಸ್ತ ಭಾರತದ ಇತಿಹಾಸದ ಮೇಲೆ ಸುಶೋಭಿತವಾದ ಸುವರ್ಣಮುದ್ರೆಯೊಂದನ್ನು ಅಂಕಿತಗೊಳಿಸದಿರಲಾರ ಎಂದು. ದಿವಾನರು ಸ್ವಾಮೀಜಿಯನ್ನು ಮೈಸೂರಿನ ತಮ್ಮ ಮನೆಗೆ ಆಹ್ವಾನಿಸಿದರು. ಸ್ವಾಮೀಜಿ ಸುಮಾರು ಮೂರು- ನಾಲ್ಕು ವಾರಗಳ ಕಾಲ ದಿವಾನರ ಅತಿಥಿಯಾಗಿದ್ದರು. ಮೈಸೂರಿನಲ್ಲಿ ಅವರು ಅರಮನೆಯ ಅಧಿಕಾರಿಗಳನ್ನೂ ಇತರ ಪ್ರಮುಖರನ್ನೂ ಭೇಟಿಯಾದರು. ಶೇಷಾದ್ರಿ ಅಯ್ಯರಿಗಂತೂ ಈ ಅದ್ಭುತ ಸಾಧುವಿನ ಭೇಟಿಯಾದದ್ದು ತುಂಬ ಸಂತೋಷವುಂಟುಮಾಡಿತ್ತು. ಒಂದು ಸಂದರ್ಭ ದಲ್ಲಿ ಅವರು ಸ್ವಾಮೀಜಿಯ ಸಂಬಂಧವಾಗಿ ಹೇಳುತ್ತಾರೆ, “ನಮ್ಮಲ್ಲೂ ಎಷ್ಟೋ ಜನ ಶಾಸ್ತ್ರ ಗಳನ್ನು ಓದಿದವರಿದ್ದಾರೆ. ಆದರೆ ಅದರಿಂದೇನಾಯಿತು? ಈ ಯುವ ಸಂನ್ಯಾಸಿಯ ಅಸಾಧಾರಣ ಜ್ಞಾನ ಹಾಗೂ ಅಂತರ್ದೃಷ್ಟಿ–ಇವು ನಾನು ಈವರೆಗೆ ಭೇಟಿಮಾಡಿದ ಪಂಡಿತರೆಲ್ಲರ ತಿಳಿವಳಿಕೆ ಯನ್ನು ಮೀರಿಸುತ್ತವೆ. ನಿಜಕ್ಕೂ ಇದು ಪರಮಾದ್ಭುತ. ಇವರು ತಮ್ಮ ಈ ಅಪಾರ ಜ್ಞಾನವನ್ನು ಹುಟ್ಟುವಾಗಲೇ ಪಡೆದುಕೊಂಡು ಬಂದಿರಬೇಕು. ಇಲ್ಲದಿದ್ದರೆ ಇಷ್ಟು ಚಿಕ್ಕ ವಯಸ್ಸಿನಲ್ಲಿಯೇ ಇಷ್ಟೆಲ್ಲ ಜ್ಞಾನವನ್ನು ಪಡೆದುಕೊಳ್ಳಲು ಹೇಗೆ ಸಾಧ್ಯ?”

ಶೇಷಾದ್ರಿ ಅಯ್ಯರ್​ರಿಗೆ ಈ ತರುಣಸಂನ್ಯಾಸಿಯನ್ನು ಮೈಸೂರು ಸಂಸ್ಥಾನದ ಮಹಾರಾಜ ರಾದ ಶ್ರೀ ಚಾಮರಾಜ ಒಡೆಯರಿಗೆ ಪರಿಚಯಿಸಲೇ ಬೇಕು ಎನ್ನಿಸಿತು. ಅತ್ಯಂತ ಯೋಗ್ಯ– ಸಮರ್ಥ ಆಡಳಿತಗಾರರೆಂದು ಜನಾದರಣೆ ಗಳಿಸಿದ್ದವರು ಚಾಮರಾಜ ಒಡೆಯರು. ಇನ್ನೂ ಮೂವತ್ತು ವರ್ಷದ ಯುವಕರು. ಅವರು ಸ್ವಾಮೀಜಿಯವರನ್ನು ಭೇಟಿಯಾಗುವುದು ಇಬ್ಬರಿಗೂ ಲಾಭಪ್ರದವಾದೀತೆಂದು ಮುಂಗಂಡ ದಿವಾನರು ಅವರನ್ನು ಅರಮನೆಗೆ ಕರೆದೊಯ್ದರು. ಕಾಷಾಯ ವಸ್ತ್ರಧಾರಿಯಾಗಿ ದಿವ್ಯ ತೇಜಸ್ಸಿನಿಂದ ಕಂಗೊಳಿಸುತ್ತಿದ್ದ ಸ್ವಾಮೀಜಿ ರಾಜ ಠೀವಿಯಿಂದ ಮಹಾರಾಜ ಚಾಮರಾಜ ಒಡೆಯರನ್ನು ಭೇಟಿಮಾಡಲು ನಡೆದುಬಂದರು. ದಿವಾನರು ಮಹಾ ರಾಜರಿಗೆ ಸ್ವಾಮೀಜಿಯನ್ನು ಪರಿಚಯಿಸಿದರು. ಸ್ವಾಮೀಜಿಯ ಭೇಟಿಯಿಂದ ಮಹಾರಾಜರಿಗೆ ತುಂಬ ಸಂತೋಷವಾಯಿತು. ಅವರ ಮಿಂಚಿನಂತಹ ಚಿಂತನಧಾರೆ, ಮಂತ್ರಮುಗ್ಧಗೊಳಿಸು ವಂತಹ ವ್ಯಕ್ತಿತ್ವ, ಬೆರಗಾಗಿಸುವಂತಹ ಬುದ್ಧಿಮತ್ತೆ, ಶಾಸ್ತ್ರಗಳ ಮರ್ಮವನ್ನು ಭೇದಿಸಿ ಅರಿಯ ಬಲ್ಲ ಅವರ ಅಂತರ್ದೃಷ್ಟಿ–ಇವುಗಳು ಮಹಾರಾಜರನ್ನು ಗೆದ್ದುವು. ಮಹಾರಾಜರು ಸ್ವಾಮೀಜಿ ಯನ್ನು ತಮ್ಮ ಅತಿಥಿಯಾಗಿರುವಂತೆ ಅತ್ಯಂತ ವಿನಯದಿಂದ ಕೇಳಿಕೊಂಡು ಒಪ್ಪಿಸಿದರು. ಅವರು ರಾಜ್ಯದ ಗೌರವ ಅತಿಥಿಯಾಗಿ ಅರಮನೆಯಲ್ಲಿ ಉಳಿದುಕೊಂಡರು. ಮಹಾರಾಜರು ಹಲವಾರು ಸಲ ಅವರನ್ನು ವೈಯಕ್ತಿಕವಾಗಿ ಭೇಟಿಮಾಡಿ ವಿವಿಧ ವಿಷಯಗಳ ಬಗ್ಗೆ ಚರ್ಚಿಸಿದರಲ್ಲದೆ ಹಲ ವಾರು ಪ್ರಮುಖ ವಿಚಾರಗಳಲ್ಲಿ ಅವರ ಸಲಹೆಗಳನ್ನು ಪಡೆದುಕೊಂಡರು.

ಹೀಗೆ ಕ್ರಮೇಣ ಇಬ್ಬರ ನಡುವಣ ಸ್ನೇಹವು ದೃಢವಾಗಿ, ಮಹಾರಾಜರು ಸ್ವಾಮೀಜಿಯ ಆಪ್ತ ವರ್ಗಕ್ಕೆ ಸೇರಿದವರಾದರು. ಆದರೆ ಖೇತ್ರಿಯ ಕಥೆ ಇಲ್ಲಿ ಪುನರಾವರ್ತನೆಗೊಳ್ಳಲಿಲ್ಲ. ಎಂದರೆ, ಸ್ವಾಮೀಜಿಯ ಸ್ನೇಹ-ಸಂಬಂಧಗಳು ಖೇತ್ರಿಯ ಮಹಾರಾಜ ಅಜಿತ್​ಸಿಂಗ್​ನೊಂದಿಗೆ ಇದ್ದಂತಹ ಪ್ರಮಾಣದಲ್ಲಿ ಮೈಸೂರು ಮಹಾರಾಜರೊಂದಿಗೂ ವೃದ್ಧಿಯಾಗಲು ಸಾಧ್ಯವಾಗ ಲಿಲ್ಲ. ಇದಕ್ಕೆ ಕಾರಣಗಳನ್ನು ತಿಳಿಯಬೇಕಾದರೆ ನಾವು ಅಂದಿನ ಮೈಸೂರು ರಾಜ್ಯದ ಪರಿಸ್ಥಿತಿ ಯನ್ನೂ ಅದರ ಚರಿತ್ರೆಯನ್ನೂ ಸ್ವಲ್ಪ ತಿಳಿದಿರಬೇಕಾಗುತ್ತದೆ.

‘ಮೈಸೂರು ಹುಲಿ’ ಎಂದು ಕರೆಯಲ್ಪಟ್ಟಿದ್ದ ಟಿಪ್ಪುಸುಲ್ತಾನನು ೧೭೯೯ರಲ್ಲಿ ಮೃತನಾದಾಗ ಬ್ರಿಟಿಷರು ಮೈಸೂರು ರಾಜ್ಯವನ್ನು ಪುನಃ ಸೃಷ್ಟಿಸಿ ಕೃಷ್ಣರಾಜನೆಂಬ ಐದು ವರ್ಷದ ಬಾಲಕನನ್ನು ರಾಜನೆಂದು ನಾಮಕರಣ ಮಾಡಿದರು. ಆದರೆ ಈತ ಪ್ರಾಪ್ತವಯಸ್ಕನಾಗಿ ರಾಜ್ಯಾಡಳಿತವನ್ನು ಕೈಗೊಂಡಾಗ ಅವನ ಪ್ರಜಾಕಂಟಕತನವನ್ನು ಸಹಿಸದೆ ಜನ ದಂಗೆಯೆದ್ದರು. ಆಗ ಬ್ರಿಟಿಷ್ ಸರಕಾರ ದಂಗೆಯನ್ನು ಮಟ್ಟಹಾಕಿ ಆಡಳಿತವನ್ನು ತಾನೇ ಕೈಗೆತ್ತಿಕೊಂಡಿತು. ಇದಾದ ಐವತ್ತು ವರ್ಷಗಳ ಅನಂತರ ಎಂದರೆ ೧೮೮೧ರಲ್ಲಿ, ಕೃಷ್ಣರಾಜನ ದತ್ತುಪುತ್ರನಾದ ಹದಿನೆಂಟು ವರ್ಷದ ತರುಣ ಚಾಮರಾಜೇಂದ್ರ ಒಡೆಯರನ್ನು ಸಿಂಹಾಸನದ ಮೇಲೆ ಕುಳ್ಳಿರಿಸಿತು. ಆದರೆ ಮಹಾ ರಾಜನು ಪ್ರತಿಯೊಂದು ವಿಚಾರದಲ್ಲೂ ಗವರ್ನರ್ ಜನರಲ್​ನ ‘ಸಲಹೆ’ಯಂತೆಯೇ ನಡೆಯ ಬೇಕೆಂಬ ಒಪ್ಪಂದವನ್ನು ಬ್ರಿಟಿಷ್ ಸರಕಾರ ಹೇರಿತ್ತು. ಅಲ್ಲದೆ ಶ್ರೀಮಂತ ವರ್ಗದವರೂ ಅಧಿಕಾರ ವರ್ಗದವರೂ ಈ ಹೊಸ ರಾಜನನ್ನು ಇಷ್ಟಪಡಲಿಲ್ಲ. ಹೀಗೆ ಮಹಾರಾಜರ ಮೇಲೆ ಅತ್ತ ಸರಕಾರದಿಂದಲೂ ಇತ್ತ ಅಧಿಕಾರವರ್ಗದಿಂದಲೂ ತೀವ್ರ ಒತ್ತಡವಿದ್ದು, ಸ್ವತಂತ್ರ ನಿರ್ಧಾರಗಳನ್ನು ಕೈಗೊಳ್ಳುವುದು ಬಹಳ ಕಷ್ಟವಾಗಿತ್ತು.

ಆದರೆ ಚಾಮರಾಜೇಂದ್ರ ಒಡೆಯರು ಅಧಿಕಾರ ವಹಿಸಿಕೊಂಡ ಈ ಹತ್ತು ವರ್ಷಗಳಲ್ಲಿ ರಾಜ್ಯವು ಸರ್ವಾಂಗೀಣ ಪ್ರಗತಿಯತ್ತ ಮುನ್ನಡೆಯತೊಡಗಿತ್ತು. ಕ್ಷಾಮಡಾಮರಗಳ ಹಾವಳಿ ಯಿಲ್ಲದೆ ಸುಭಿಕ್ಷೆ ನೆಲಸಿತ್ತು. ಅಲ್ಲದೆ ಮೈಸೂರು ರಾಜ್ಯವು ಕೋಮುವಾರು ಗಲಭೆಗಳಿಂದ ದೂರ ವಾಗಿತ್ತು. ಕೈಗಾರಿಕೆಗಳೂ ಅಭಿವೃದ್ಧಿ ಹೊಂದುತ್ತಿದ್ದು. ಅದು ಆದರ್ಶ ರಾಜ್ಯವೆಂಬ ಹೆಸರನ್ನು ಗಳಿಸುತ್ತಿತ್ತು. ಮಹಾರಾಜರನ್ನು ಸ್ವಾಮೀಜಿ ಭೇಟಿಯಾದಾಗ ಅವರಿನ್ನೂ ಸುಮಾರು ಮೂವತ್ತು ವರ್ಷದ ಯುವಕರು. ಪ್ರಥಮ ಭೇಟಿಯಲ್ಲೇ ಸ್ವಾಮೀಜಿಯಿಂದ ಬಹಳವಾಗಿ ಆಕರ್ಷಿತರಾಗಿದ್ದ ಮಹಾರಾಜರು ದಿನಕಳೆದಂತೆ ಅವರ ಆರಾಧಕರೇ ಆಗಿಬಿಟ್ಟರು. ಸ್ವಾಮೀಜಿ ಅಮೆರಿಕೆಗೆ ಹೋಗುವ ವಿಷಯದ ಪ್ರಸ್ತಾಪವೂ ಬಂದಿತು. ಆದರೆ ಬ್ರಿಟಿಷರನ್ನು ಎದುರುಹಾಕಿಕೊಳ್ಳುವ ಭಯವಿದ್ದುದ ರಿಂದ ಅವರನ್ನು ವಿಶ್ವಧರ್ಮ ಸಮ್ಮೇಳನಕ್ಕೆ ಸರ್ಕಾರದ ವತಿಯಿಂದ ಕಳಿಸಿಕೊಡುವುದು ಮಹಾ ರಾಜರಿಗೆ ಸಾಧ್ಯವಿರಲಿಲ್ಲ. ಅಲ್ಲದೆ ಸ್ವಾಮೀಜಿ, ಮಹಾರಾಜರ ಆಪ್ತರಾದುದು ಅನೇಕ ಅಧಿಕಾರಿಗಳ ಕಣ್ಣುರಿಗೆ ಕಾರಣವಾಗಿತ್ತೆಂದೂ ಕಾಣುತ್ತದೆ. ಆದರೆ ಅವರ ಪ್ರಯಾಣಕ್ಕಾಗಿ ಧನಸಹಾಯ ಮಾಡಲು ಮಹಾರಾಜರು ತಕ್ಷಣ ಮುಂದೆ ಬಂದರಲ್ಲದೆ, ಮುಂದೆ ಅವರ ಶಿಷ್ಯರು ಅದಕ್ಕಾಗಿ ಚಂದಾ ಎತ್ತಿದಾಗ ಅದರ ಬಹುಭಾಗವನ್ನು ತಾವೇ ತುಂಬಿಕೊಡುತ್ತಾರೆ.

ಒಂದು ದಿನ ಮಹಾರಾಜರು ತಮ್ಮ ಸಭಾಸದರ ಸಮ್ಮುಖದಲ್ಲಿ ಸ್ವಾಮೀಜಿಯನ್ನು ಕೇಳಿದರು, “ನನ್ನ ಆಸ್ಥಾನಿಕರ ವಿಷಯವಾಗಿ ನಿಮ್ಮ ಅಭಿಪ್ರಾಯವೇನು?”

“ಮಹಾರಾಜ, ನನಗನ್ನಿಸುತ್ತದೆ, ನಿನ್ನ ಹೃದಯವೇನೋ ಒಳ್ಳೆಯದು. ಆದರೆ ದುರದೃಷ್ಟವ ಶಾತ್ ನಿನ್ನ ಸುತ್ತಲೂ ಹೊಗಳುಭಟರು ಸುತ್ತುವರಿದಿದ್ದಾರೆ. ಹೊಗಳುಭಟರು ಎಷ್ಟಾದರೂ ಹೊಗಳುಭಟರೇ.”

ಈ ನೇರ ಉತ್ತರವನ್ನು ಮಹಾರಾಜರು ನಿರೀಕ್ಷಿಸಿರಲಿಲ್ಲ. ಆದರೂ ಸವಾರಿಸಿಕೊಂಡು ಮತ್ತೆ ಹೇಳಿದರು, “ಇಲ್ಲ, ಸ್ವಾಮೀಜಿ, ಕಡೇಪಕ್ಷ ನಮ್ಮ ದಿವಾನರಂತೂ ಅಂಥವರಲ್ಲ. ಅವರು ತುಂಬ ನಂಬಿಕಸ್ಥರು, ಬುದ್ಧಿವಂತರು.”

“ಆದರೆ ಮಹಾರಾಜ, ದಿವಾನ ಯಾರು ಎಂದರೆ ರಾಜನನ್ನು ಲೂಟಿ ಮಾಡಿ ಬ್ರಿಟಿಷ್ ಏಜೆಂಟನ ಕೈಗೆ ದಾಟಿಸುವವನು.” ಸ್ವಾಮೀಜಿ ಈಗಾಗಲೇ ಹಲವಾರು ರಾಜ್ಯಗಳ ದಿವಾನರನ್ನು ಕಂಡವರಲ್ಲವೆ!

ಮಹಾರಾಜರಿಗೀಗ ಕಷ್ಟಕ್ಕಿಟ್ಟುಕೊಂಡಿತು. ಈ ಮಾತನ್ನೇ ಬೆಳೆಸಿದರೆ ವಿಷಯ ಎಲ್ಲೆಲ್ಲಿಗೋ ಹೋಗುತ್ತದೆ ಎಂದು ತಿಳಿದು ತಕ್ಷಣ ಮಾತು ಬದಲಾಯಿಸಿದರು. ಆಮೇಲೆ ಅವರನ್ನು ಏಕಾಂತ ದಲ್ಲಿ ತಮ್ಮ ಹತ್ತಿರಕ್ಕೆ ಕರೆಯಿಸಿಕೊಂಡು ಹೇಳಿದರು, “ಸ್ವಾಮೀಜಿ, ಹೀಗೆ ಮುಚ್ಚುಮರೆಯಿಲ್ಲದೆ ಮಾತನಾಡುವುದು ಯಾವಾಗಲೂ ಕ್ಷೇಮವಲ್ಲ. ಇಂದು ನೀವು ನನ್ನ ಆಸ್ಥಾನಿಕರೆದುರು ಮಾತಾಡಿ ದಂತೆಯೇ ಮಾತಾಡುತ್ತಿದ್ದರೆ, ನನಗನ್ನಿಸುತ್ತದೆ, ನಿಮಗೆ ಯಾರಾದರೂ ವಿಷ ಹಾಕಿದರೂ ಆಶ್ಚರ್ಯವಿಲ್ಲ. ಆದ್ದರಿಂದ ದಯವಿಟ್ಟು ನೀವು ಈ ಬಗ್ಗೆ ಎಚ್ಚರ ವಹಿಸಬೇಕು.”

ಮಹಾರಾಜರ ಮಾತನ್ನು ಕೇಳಿ ಸ್ವಾಮೀಜಿ ಸಿಡಿದೆದ್ದರು:

“ಏನು! ಒಬ್ಬ ಪ್ರಾಮಾಣಿಕ ಸಂನ್ಯಾಸಿ ತನ್ನ ಪ್ರಾಣವೇ ಹೋಗುವಂಥ ಸಂದರ್ಭವೊದಗಿ ದರೂ ಸತ್ಯವನ್ನು ಹೇಳಲು ಹೆದರುತ್ತಾನೆಂದು ಭಾವಿಸಿದೆಯಾ? ನಾಳೆ ನಿನ್ನ ಮಗನೇ ಬಂದು, ‘ಸ್ವಾಮೀಜಿ, ನನ್ನ ತಂದೆಯವರ ಬಗ್ಗೆ ನಿಮ್ಮ ಅಭಿಪ್ರಾಯವೇನು?’ ಎಂದು ಕೇಳಿದರೆ ಆಗ ನಾನು ನಿನ್ನಲ್ಲಿ ಇಲ್ಲದಿರುವ ಸದ್ಗುಣಗಳೆಲ್ಲ ಇದೆ ಎಂದು ಹೇಳಬೇಕೆನು? ನಾನು ಸುಳ್ಳು ಹೇಳಲೆ! ಎಂದಿಗೂ ಇಲ್ಲ.”

ಆದರೆ ಸ್ವಾಮೀಜಿ ಹೀಗೆ ಹೇಳಿದರೂ ಮಹಾರಾಜರು ಎದುರಿಗಿಲ್ಲದಿರುವಾಗ ಇತರರ ಮುಂದೆ ಅವರ ಕುರಿತಾಗಿ ತುಂಬ ವಿಶ್ವಾಸಗೌರವಗಳಿಂದಲೇ ಮಾತನಾಡುತ್ತಿದ್ದರು; ಮಹಾರಾಜರ ಸದ್ಗುಣಗಳನ್ನು ಹೃತ್ಪೂರ್ವಕವಾಗಿ ಕೊಂಡಾಡುತ್ತಿದ್ದರು. ವ್ಯಕ್ತಿಯ ಎದುರಿನಲ್ಲಿ ಮುಚ್ಚುಮರೆ ಯಿಲ್ಲದೆ ಖಡಾಖಂಡಿತವಾಗಿ ಮಾತನಾಡಿ ಅವನ ಲೋಪದೋಷಗಳ ಬಗ್ಗೆ ಅವನನ್ನು ತರಾಟೆಗೆ ತೆಗೆದುಕೊಳ್ಳುವುದು ಸ್ವಾಮೀಜಿಯ ಸ್ವಭಾವ. ಆದರೆ ಆ ವ್ಯಕ್ತಿಯ ಬೆನ್ನಹಿಂದೆ ಅವನ ಬಗ್ಗೆ ಪ್ರಶಂಸೆಯ ಮಾತುಗಳಲ್ಲದೆ ಅವರ ಬಾಯಿಂದ ಬೇರೇನೂ ಬರುತ್ತಿರಲಿಲ್ಲ.

ಸ್ವಾಮೀಜಿ ಹೋದಹೋದಲ್ಲೆಲ್ಲ ಅವರ ವ್ಯಕ್ತಿತ್ವದಿಂದ ಆಕರ್ಷಿತರಾಗಿ ಅವರ ಬಳಿಗೆ ಬರುತ್ತಿದ್ದವರಲ್ಲಿ ಹಿಂದೂಗಳು ಮಾತ್ರವಲ್ಲದೆ ಇತರ ಧರ್ಮೀಯರೂ ಇರುತ್ತಿದ್ದರು. ಒಂದು ದಿನ ಮೈಸೂರಿನ ಒಬ್ಬ ಶಾಸಕನಾದ ಅಬ್ದುಲ್ ರಹಮಾನ್ ಸಾಹೇಬ್ ಎಂಬಾತ ಅವರ ಬಳಿಗೆ ಬಂದು ಮಾತನಾಡಿದಾಗ ಅವರಿಗೆ ಕುರಾನಿನ ಬಗ್ಗೆ ಆಳವಾದ ಜ್ಞಾನವಿರುವುದನ್ನು ಕಂಡ. ಈ ಹಿಂದೂ ಸಂನ್ಯಾಸಿ ಮುಸ್ಲಿಂ ಧರ್ಮಗ್ರಂಥದಲ್ಲಿ ಪರಿಣತಿ ಪಡೆದಿರುವುದನ್ನು ಕಂಡು ಅಲ್ಲಿದ್ದವರಿ ಗೆಲ್ಲ ಆಶ್ಚರ್ಯ. ಆ ಮುಸಲ್ಮಾನ, ಸ್ವಾಮೀಜಿಯೊಂದಿಗೆ ಸಂಭಾಷಣೆ ನಡೆಸಿ ಕುರಾನಿನ ಕೆಲವು ವಿಷಯಗಳ ಬಗ್ಗೆ ತನಗಿದ್ದ ಸಂಶಯಗಳನ್ನು ಪರಿಹರಿಸಿಕೊಂಡ. ಮೈಸೂರು ಅರಮನೆಯಲ್ಲಿ ದ್ದಾಗ ಸ್ವಾಮೀಜಿಗೆ ಆಸ್ಟ್ರಿಯಾ ದೇಶದ ಖ್ಯಾತ ಸಂಗೀತಗಾರನೊಬ್ಬನನ್ನು ಭೇಟಿ ಮಾಡುವ ಅವಕಾಶ ಒದಗಿತು. ಅವನೊಂದಿಗೆ ಅವರು ಪಾಶ್ಚಾತ್ಯ ಸಂಗೀತದ ಬಗ್ಗೆ ಸಂಭಾಷಿಸಿದರು. ಅವ ರಿಗೆ ಪಾಶ್ಚಾತ್ಯ ಸಂಗೀತದ ವಿಷಯದಲ್ಲೂ ಇದ್ದ ಪರಿಣತಿಯನ್ನು ಕಂಡು ಎಲ್ಲರೂ ಬೆರಗಾದರು. ಮತ್ತೊಂದು ದಿನ ಅವರು ಅರಮನೆಯ ವಿದ್ಯುದೀಕರಣದಲ್ಲಿ ತೊಡಗಿದ್ದ ತಂತ್ರಜ್ಞನೊಬ್ಬನನ್ನು ಸಂಧಿಸಿದರು. ಸಂಭಾಷಣೆಯ ಸಂದರ್ಭದಲ್ಲಿ ಮಾತು ವಿದ್ಯುಚ್ಛಕ್ತಿಯ ವಿಷಯದತ್ತ ಹೊರಳಿತು. ಆ ವಿಷಯದಲ್ಲೂ ಸ್ವಾಮೀಜಿ ತಂತ್ರಜ್ಞನನ್ನೇ ಮೀರಿಸಿದ್ದು ಸ್ಪಷ್ಟವಾಗಿತ್ತು.

ಸ್ವಾಮೀಜಿ ಮೈಸೂರಿನಲ್ಲಿದ್ದಾಗಲೇ ಒಮ್ಮೆ ಅರಮನೆಯಲ್ಲಿ ವಿದ್ವದ್ಗೋಷ್ಠಿಯೊಂದು ನಡೆ ಯಿತು. ಇದರಲ್ಲಿ ಅನೇಕ ಪಂಡಿತರು ಭಾಗವಹಿಸಿದ್ದರು. ಸಭೆಯ ಅಧ್ಯಕ್ಷರಾದ ದಿವಾನರು ಸ್ವಾಮೀಜಿಯವರನ್ನೂ ಸಭೆಗೆ ಆಹ್ವಾನಿಸಿದ್ದರು. ಚರ್ಚೆಯ ವಿಷಯ ವೇದಾಂತ. ನೆರೆದಿದ್ದ ಪಂಡಿತರು ವೇದಾಂತದ ಬೇರೆಬೇರೆ ಮುಖಗಳನ್ನು ಪ್ರತಿಪಾದಿಸಿ ಸಂಸ್ಕೃತದಲ್ಲಿ ಚರ್ಚೆ ನಡೆಸಿ ದರು. ಆದರೆ ಎಷ್ಟು ಹೊತ್ತು ಚರ್ಚಿಸಿದರೂ ಯಾವ ಒಂದು ತೀರ್ಮಾನಕ್ಕೂ ಬರುವ ಸೂಚನೆಯೇ ಕಾಣಲಿಲ್ಲ. ಕಡೆಗೆ ದಿವಾನರು ಈ ವಿಷಯವಾಗಿ ತಮ್ಮ ನಿಲುವನ್ನು ತಿಳಿಸುವಂತೆ ಸ್ವಾಮೀಜಿಯನ್ನು ಕೇಳಿಕೊಂಡರು. ಆಗ ಸ್ವಾಮೀಜಿ ಎದ್ದು ನಿಂತು ತಮ್ಮ ಮಾತನ್ನು ಪ್ರಾರಂಭಿಸಿ ದರು. ಮುಂದೆ ತಾವು ಪಾಶ್ಚಾತ್ಯ ದೇಶಗಳಲ್ಲಿ ಮೊಳಗಲಿರುವ ಅದ್ಭುತ ವೇದಾಂತ ಸಂದೇಶವನ್ನು ಸುಲಲಿತ ಸಂಸ್ಕೃತದಲ್ಲಿ ವಿವರಿಸತೊಡಗಿದಂತೆ ಪಂಡಿತರೆಲ್ಲ ದಂಗುಬಡಿದು ಕುಳಿತರು. ವೇದಾಂತದ ವಿಷಯವಾಗಿ ಸ್ವಾಮಿಗಳ ವೈಶಿಷ್ಟ್ಯಪೂರ್ಣ ಗ್ರಹಿಕೆ ಮತ್ತು ವಿವರಣಾ ವಿಧಾನ ಗಳನ್ನು ಕಂಡು ವಿಸ್ಮಯಾನಂದಗೊಂಡ ಪಂಡಿತವರ್ಗದವರೆಲ್ಲ ಏಕಕಂಠದಿಂದ ತಮ್ಮ ಮೆಚ್ಚುಗೆಯನ್ನು ವ್ಯಕ್ತಪಡಿಸಿದರು.

ಒಂದು ದಿನ ದಿವಾನರ ಸಮ್ಮುಖದಲ್ಲಿ ಮಹಾರಾಜರು ಸ್ವಾಮೀಜಿಯವರೊಂದಿಗೆ ಮಾತ ನಾಡುತ್ತ, “ಸ್ವಾಮೀಜಿ, ನಾನು ನಿಮಗಾಗಿ ಏನು ಮಾಡಲಿ?” ಎಂದು ಕೇಳಿದರು. ಈ ಪ್ರಶ್ನೆಗೆ ಸ್ವಾಮೀಜಿ ನೇರವಾಗಿ ಉತ್ತರಿಸದೆ, ತಾವು ಮುಂದೆ ಸಾಧಿಸಬೇಕಾದ ಕಾರ್ಯೋದ್ದೇಶದ ಬಗ್ಗೆ ಉಜ್ವಲವಾಗಿ ಮಾತನಾಡತೊಡಗಿದರು. ಅವರ ಮಾತು ಸುಮಾರು ಒಂದು-ಒಂದೂವರೆ ಗಂಟೆಗಳ ಕಾಲ ಮುಂದುವರಿಯಿತು. ಭಾರತದ ಪರಿಸ್ಥಿತಿಯನ್ನು ಮನಮುಟ್ಟುವಂತೆ ವರ್ಣಿಸುತ್ತ ಸ್ವಾಮೀಜಿ ಹೇಳಿದರು, “ಮಹಾರಾಜ, ನಮ್ಮ ಭಾರತ ಭೂಮಿಯು ಅನಾದಿಯಿಂದ ಪಡೆದು ಕೊಂಡು ಬಂದ ಆಸ್ತಿ ಯಾವುದೆಂದರೆ ಉನ್ನತ ತಾತ್ತ್ವಿಕ ಹಾಗೂ ಆಧ್ಯಾತ್ಮಿಕ ಮೌಲ್ಯಗಳು. ಈಗ ಅದಕ್ಕೆ ಬೇಕಾಗಿರುವುದು ಆಧುನಿಕ ವಿಜ್ಞಾನದ ಜೋಡಣೆ. ಭಾರತದಲ್ಲಿ ಅಡಿಯಿಂದ ಮುಡಿಯ ವರೆಗೆ ರಚನಾತ್ಮಕ ಸುಧಾರಣೆಗಳನ್ನು ತರಬೇಕಾಗಿದೆ... ತನ್ನ ಮಡಿಲಲ್ಲಿಟ್ಟುಕೊಂಡಿರುವ ಅಮೂಲ್ಯ ಆಧ್ಯಾತ್ಮಿಕ ಸಂಪತ್ತನ್ನು ಪಾಶ್ಚಾತ್ಯ ಜಗತ್ತಿಗೆ ಕೊಡುವ ಹೊಣೆ ಭಾರತಕ್ಕೆ ಸೇರಿದ್ದು....

“ನಾನು ವೇದಾಂತ ಪ್ರಚಾರಕ್ಕಾಗಿ ಅಮೆರಿಕೆಗೆ ಹೋಗಬೇಕೆಂದುಕೊಂಡಿದ್ದೇನೆ. ಪಾಶ್ಚಾತ್ಯ ರಾಷ್ಟ್ರಗಳು ಭಾರತೀಯರಿಗೆ ಆಧುನಿಕ ಕೃಷಿವಿಜ್ಞಾನ, ಕೈಗಾರಿಕೆ ಹಾಗೂ ಇತರ ತಾಂತ್ರಿಕ ವಿಷಯಗಳಲ್ಲಿ ಶಿಕ್ಷಣ ನೀಡಿ ನಮ್ಮ ಆರ್ಥಿಕ ಪರಿಸ್ಥಿತಿಯನ್ನು ಅಭಿವೃದ್ಧಿ ಪಡಿಸುವಲ್ಲಿ ನೆರವಾಗ ಬೇಕೆಂದು ನಾನು ನಿರೀಕ್ಷಿಸುತ್ತೇನೆ.”

ಸ್ವಾಮೀಜಿಯ ಮಾತುಗಳನ್ನು ಕೇಳಿ ಮಹಾರಾಜರು ಮಂತ್ರಮುಗ್ಧರಾದರು. ಒಬ್ಬ ಸರ್ವ ಸಂಗ ಪರಿತ್ಯಾಗಿಯಾದ ಸಂನ್ಯಾಸಿ ತನ್ನ ಮಾತೃಭೂಮಿಗಾಗಿ ಪರಿತಪಿಸುವುದನ್ನು ಕಂಡು ಅವರ ಹೃದಯ ತುಂಬಿಬಂತು. ಸ್ವಾಮೀಜಿ ಅಮೆರಿಕೆಗೆ ಹೋಗುವುದಾದರೆ ಅವರ ಎಲ್ಲ ಖರ್ಚನ್ನು ತಾನೇ ವಹಿಸಿಕೊಳ್ಳುವುದಾಗಿ ಮುಂದಾದರು. ಆದರೆ ಸ್ವಾಮೀಜಿ, “ಈಗಲೇ ಬೇಡ” ಎನ್ನುತ್ತ ಹಣವನ್ನು ನಿರಾಕರಿಸಿದರು. ಬಹುಶಃ ಅವರು ಆಗ ರಾಮೇಶ್ವರಕ್ಕೆ ಹೋಗಬೇಕಾಗಿದ್ದುದು ಒಂದು ಕಾರಣವಿರಬಹುದು. ಅವರ ಮಾತುಗಳನ್ನು ಕೇಳಿದಂದಿನಿಂದ, ಅವರು ದೇಶದ ಪುನರುತ್ಥಾನಕ್ಕಾಗಿ ಜನ್ಮವೆತ್ತಿ ಬಂದವರು ಎಂದು ಮಹಾರಾಜರಿಗೂ ದಿವಾನರಿಗೂ ಖಚಿತವಾಯಿತು. ದಿನಕಳೆ ದಂತೆಲ್ಲ ಮಹಾರಾಜರು ಅವರನ್ನು ಹೆಚ್ಚುಹೆಚ್ಚಾಗಿ ಹಚ್ಚಿಕೊಂಡರು; ಅವರ ಮೇಲಣ ಪ್ರೀತಿ ವಿಶ್ವಾಸ ಅಧಿಕವಾಗುತ್ತಬಂತು. ಆದ್ದರಿಂದ ಒಂದು ದಿನ ಸ್ವಾಮೀಜಿ ತಾವು ಮೈಸೂರಿನಿಂದ ಹೊರಡುವ ಬಗ್ಗೆ ಪ್ರಸ್ತಾಪಿಸಿದಾಗ ಮಹಾರಾಜರಿಗೆ ತುಂಬ ದುಃಖವಾಯಿತು. ಇನ್ನೂ ಕೆಲದಿನ ಅರಮನೆಯಲ್ಲೇ ಉಳಿದುಕೊಳ್ಳುವಂತೆ ಅವರನ್ನು ಬೇಡಿಕೊಂಡರು. ಬಳಿಕ ಅವರ ನೆನಪಿಗಾಗಿ ಏನಾದರೊಂದನ್ನು ಪಡೆದುಕೊಳ್ಳಲು ಇಚ್ಛಿಸಿದ ಮಹಾರಾಜರು ಸ್ವಾಮೀಜಿಯ ಕಂಠಧ್ವನಿಯನ್ನು ಗ್ರಾಮಾಫೋನ್ ಪ್ಲೇಟಿನಲ್ಲಿ ಧ್ವನಿಮುದ್ರಣ ಮಾಡಿಕೊಂಡರು. ಅದು ಈಗೆಲ್ಲೂ ಲಭ್ಯವಿಲ್ಲ. ಅಲ್ಲದೆ ಅದು ಹೆಚ್ಚು ಕಾಲ ಬಾಳಿಕೆ ಬರುವಂಥದೂ ಅಲ್ಲ.

ಸ್ವಾಮೀಜಿಯ ಬಗ್ಗೆ ಮಹಾರಾಜರ ಭಕ್ತಿ-ಗೌರವ ಎಷ್ಟರಮಟ್ಟಿಗೆ ಬೆಳೆದಿತ್ತೆಂದರೆ ಅವರನ್ನು ತಮ್ಮ ಕುಲಗುರುವಿಗೆ ಸಮವಾಗಿ ಕಂಡು, ಅವರ ಪಾದಪೂಜೆ ಮಾಡಲು ಇಷ್ಟಪಟ್ಟರು. ಆದರೆ ಸ್ವಾಮೀಜಿ ಇದಕ್ಕೆ ಅನುಮತಿ ಕೊಡಲಿಲ್ಲ.

ಕೆಲದಿನಗಳ ಬಳಿಕ ಸ್ವಾಮೀಜಿ ಕೇರಳ ರಾಜ್ಯದೆಡೆಗೆ ಹೊರಡಲು ಸಿದ್ಧರಾದರು. ಮಹಾರಾಜರು ಅವರಿಗೆ ಬೆಲೆಬಾಳುವ ಉಡುಗೊರೆಯೊಂದನ್ನು ಅರ್ಪಿಸಲು ಇಚ್ಛಿಸಿದರು. ಆದರೆ ಸ್ವಾಮೀಜಿ ಯಾವುದನ್ನೂ ಸ್ವೀಕರಿಸಲು ಒಪ್ಪಲೇ ಇಲ್ಲ.

ಅವರನ್ನು ಬೀಳ್ಕೊಡುವ ಮುನ್ನ ಮಹಾರಾಜರು ಶಿರಬಾಗಿ ನಮಸ್ಕರಿಸಿದರು. ದಿವಾನರು ಸ್ವಾಮೀಜಿಯ ಜೇಬಿಗೆ ನೋಟುಗಳ ಕಟ್ಟೊಂದನ್ನು ತುರುಕಲು ಪ್ರಯತ್ನಿಸಿದರು; ಆದರೆ ಯಶಸ್ವಿಯಾಗಲಿಲ್ಲ. ಆಗ ಸ್ವಾಮೀಜಿ ಹೇಳಿದರು. “ನೀವು ನನಗೋಸ್ಕರ ಏನಾದರೂ ಮಾಡಬೇಕು ಎಂದಿದ್ದರೆ ನನಗೊಂದು ರೈಲು ಟಿಕೆಟ್ಟು ತೆಗೆಸಿಕೊಡಿ.” ದಿವಾನರು ಸ್ವಾಮೀಜಿಗೆ ಎರಡನೆಯ ದರ್ಜೆಯ ಟಿಕೆಟನ್ನು ತರಿಸಿಕೊಟ್ಟು, ಕೊಚ್ಚಿನ್ನಿನ ದಿವಾನರಾದ ಶಂಕರಯ್ಯ ಎಂಬವರಿಗೆ ಪರಿಚಯ ಪತ್ರವೊಂದನ್ನು ಬರೆದು ಕೊಟ್ಟರು. ಸ್ವಾಮೀಜಿಯನ್ನು ಕುಳ್ಳಿರಿಸಿಕೊಂಡ ರೈಲು ಹೆಮ್ಮೆಯಿಂದ ಆರ್ಭಟಿಸುತ್ತ ಮೈಸೂರಿನಿಂದ ಹೊರಟುಹೋಯಿತು.

ಮೈಸೂರು ಮಹಾರಾಜರಲ್ಲಿ ಸ್ವಾಮೀಜಿ ತುಂಬ ಭರವಸೆಯಿಟ್ಟುಕೊಂಡಿದ್ದರು. ೧೮೯೪ರ ಜೂನ್ ತಿಂಗಳಲ್ಲಿ ಶಿಕಾಗೋದಿಂದ ಅವರಿಗೆ ಬರೆದ ಪತ್ರದಲ್ಲಿ ಸ್ವಾಮೀಜಿ, ಅವರ ಸಹಾಯಕ್ಕಾಗಿ ಹೃತ್ಪೂರ್ವಕ ಕೃತಜ್ಞತೆಯನ್ನು ವ್ಯಕ್ತಪಡಿಸುತ್ತಾರೆ. ಬಳಿಕ ಭಾರತದ ಬಗ್ಗೆ ಪ್ರಸ್ತಾಪಿಸಿ ಹೇಳುತ್ತಾರೆ, “ಭಾರತದ ಇಂದಿನ ದುರ್ಗತಿಗೆ ದಾರಿದ್ರ್ಯವೇ ಕಾರಣ... ಜನಸಾಮಾನ್ಯರಿಗೆ ಶಿಕ್ಷಣ ನೀಡಿ, ನಷ್ಟವಾದ ಅವರ ವ್ಯಕ್ತಿತ್ವವನ್ನು ಅವರಿಗೆ ಮರಳಿ ದೊರಕಿಸಿಕೊಡುವುದು ನಮ್ಮ ಪ್ರಥಮ ಕರ್ತವ್ಯ.” ಬಳಿಕ ತಮ್ಮ ಪತ್ರವನ್ನು ಮುಕ್ತಾಯಗೊಳಿಸುತ್ತ ಹೇಳುತ್ತಾರೆ:

“ನನ್ನ ನೆಚ್ಚಿನ ಮಹಾರಾಜ, ಜೀವನವು ಕ್ಷಣಿಕವಾದುದು; ಜಗತ್ತಿನ ಭೋಗಗಳು ತಾತ್ಕಾಲಿಕ ವಾದವುಗಳು. ಆದರೆ ಯಾರು ಇತರರಿಗಾಗಿ ಬದುಕುತ್ತಾರೆಯೋ, ಅವರದು ಮಾತ್ರವೇ ಜೀವನ; ಉಳಿದವರು ಬದುಕಿದ್ದರೂ ಸತ್ತಂತೆಯೇ. \eng{(They alone live, who live for others. The rest are more dead than alive.)} ಉದಾತ್ತಮನಸ್ಕರಾದ, ಭಾರತಮಾತೆಯ ಶ್ರೇಷ್ಠ ಪುತ್ರರಾದ ನಿಮ್ಮಂಥವರು, ಆಕೆಯು ಮೇಲೆದ್ದು ನಿಲ್ಲಲು ಬಹಳಷ್ಟು ನೆರವಾಗಬಲ್ಲಿರಿ; ತನ್ಮೂಲಕ ಶಾಶ್ವತ ಕೀರ್ತಿವಂತರಾಗಬಲ್ಲಿರಿ. ಆಜ್ಞಾನದಲ್ಲಿ ಮುಳುಗಿರುವ ಕೋಟ್ಯವಧಿ ದೀನ-ಆರ್ತ ಭಾರತೀಯರಿ ಗಾಗಿ ನಿನ್ನ ಆ ಉದಾತ್ತ ಹೃದಯವು ತೀವ್ರವಾಗಿ ಮಿಡಿಯಲಿ ಎಂದು ಪ್ರಾರ್ಥಿಸುವ, ವಿವೇಕಾ ನಂದ.”

ಆದರೆ ಇವರಿಬ್ಬರ ನಡುವಣ ಬಾಂಧವ್ಯವು ಹೆಮ್ಮರವಾಗಿ ಬೆಳೆದು ಫಲವನ್ನು ಕೊಡುವ ಮೊದಲೇ ಬಾಡಿಹೋಯಿತು. ಅದೇ ವರ್ಷದ ಡಿಸೆಂಬರಿನಲ್ಲಿ ಮಹಾರಾಜರು ಅಕಾಲಿಕವಾಗಿ ಮೃತ್ಯುವಶರಾದರು. ಅವರು ಮತ್ತಷ್ಟು ವರ್ಷ ಬದುಕಿದ್ದಿದ್ದರೆ ಸ್ವಾಮಿ ವಿವೇಕಾನಂದರ ಮಹೋದ್ದೇಶವನ್ನು ಕಾರ್ಯಗತಗೊಳಿಸುವಲ್ಲಿ ಅವರೊಂದು ಆಧಾರಸ್ತಂಭವಾಗುತ್ತಿದ್ದರೆಂಬು ದರಲ್ಲಿ ಸಂಶಯವಿಲ್ಲ. ಮಹಾರಾಜರ ನಿಧನದ ಸುದ್ಧಿ ತಿಳಿದುಬಂದಾಗ ಸ್ವಾಮೀಜಿ ಮದ್ರಾಸಿನ ತಮ್ಮ ಶಿಷ್ಯ ಅಳಸಿಂಗ ಪೆರುಮಾಳ್​ಗೆ ಬರೆಯುತ್ತಾರೆ. “ನಮ್ಮ ಅತಿದೊಡ್ಡ ಭರವಸೆಗಳಲ್ಲೊಬ್ಬ ರಾದ ಮೈಸೂರು ಮಹಾರಾಜರು ಇನ್ನಿಲ್ಲ...”

