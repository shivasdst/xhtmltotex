
\chapter{ಚಿತ್ರಗಳ ವಿವರಣೆ}

೧. ಕನ್ಯಾಕುಮಾರಿಯ ಸಮುದ್ರ ತೀರ ಹಾಗೂ “ವಿವೇಕಾನಂದ ಬಂಡೆ” (ಎಡಗಡೆ ಇರುವುದು). ಈಗ ಅದರ ಮೇಲೊಂದು ಭವ್ಯಸ್ಮಾರಕ ಕಂಗೊಳಿಸುತ್ತಿದೆ.\\
೨. ಶಿಕಾಗೋದಲ್ಲಿ ನಡೆದ ಜಾಗತಿಕ ಮೇಳದ ವಿಹಂಗಮ ನೋಟ.\\
೩. ಸರ್ವಧರ್ಮ ಸಮ್ಮೇಳನ ನಡೆದ ಆರ್ಟ್ ಇನ್ಸ್​ಟಿಟ್ಯೂಟ್ ಕಟ್ಟಡ.\\
೪. ಸಮ್ಮೇಳನದ ವೇದಿಕೆಯ ಮೇಲೆ ವಿರಾಜಮಾನರಾದ ವಿವೇಕಾನಂದರು.\\
೫. ಸಹಸ್ರದ್ವೀಪೋದ್ಯಾನದಲ್ಲಿ ಮಿಸ್ ಡಚರಳ ಮನೆ.\\
೬. ವೃತ್ತಪತ್ರಿಕೆಗಳಲ್ಲಿ ವಿವೇ‘ಕಾನಂದ’ರ ಭಾಷಣಗಳ ಜಾಹೀರಾತು.\\
೭. ಅಮೆರಿಕದಲ್ಲಿ ಸ್ವಾಮೀಜಿಯವರ ನೆರವಿಗೊದಗಿದ ಪ್ರಥಮ ವ್ಯಕ್ತಿ–ಮಿಸ್ ಕ್ಯಾಥರೀನ್ ಸ್ಯಾನ್​ಬಾರ್ನ್.\\
೮. ಪ್ರೊ ॥ ಜೆ. ಹೆಚ್. ರೈಟ್.\\
೯, ೧ಂ. ಶ್ರೀ ಜಾರ್ಜ್ ಹೇಲ್ ಹಾಗೂ ಶ್ರೀಮತಿ ಬೆಲ್ ಹೇಲ್.\\
೧೧. ‘ಹೇಲ್ ಸಹೋದರಿಯರಾ’ದ (ಎಡದಿಂದ) ಹ್ಯಾರಿಯೆಟ್ ಮೆಕ್​ಕಿಂಡ್ಲಿ, ಮೇರಿ ಹೇಲ್, ಇಸಾಬೆಲ್ ಮೆಕ್​ಕಿಂಡ್ಲಿ ಮತ್ತು ಹ್ಯಾರಿಯೆಟ್ ಹೇಲ್.\\
೧೨, ೧೩. ಶ್ರೀ ಜಾನ್ ಬಿ. ಇಲಿಯಾನ್ ಹಾಗೂ ಶ್ರೀಮತಿ ಎಮಿಲಿ ಲಿಯಾನ್.\\
೧೪. ನೆಚ್ಚಿನ ಸಹಾಯಕಿ, ಶಿಷ್ಯೆ, ಶ್ರೀಮತಿ ಸಾರಾ ಓಲೆ ಬುಲ್.\\
೧೫. ಆಪ್ತ ಶಿಷ್ಯೆ ಮಿಸ್ ಜೊಸೆಫಿನ್ ಮೆಕಾಲಾಡ್.\\
೧೬, ೧೭. ಫ್ರಾನ್ಸಿಸ್ ಲೆಗೆಟ್ ದಂಪತಿಗಳು.\\
೧೮. ಪ್ರಮುಖ ಬೆಂಬಲಿಗಳೂ ಅನುಯಾಯಿಯೂ ಆದ ಶ್ರೀಮತಿ ಬ್ಯಾಗ್​ಲಿ.\\
೧೯. ಪ್ರಿಯ ಶಿಷ್ಯೆ ಕ್ರಿಸ್ಟೀನ.\\
೨ಂ. ವಿವೇಕಾನಂದರ “ಆಧ್ಯಾತ್ಮಿಕ ಪುತ್ರಿ” ಸೋದರಿ ನಿವೇದಿತಾ.\\
೨೧. ನಿಷ್ಠಾವಂತ ಶಿಷ್ಯ, “ಸ್ವಾಮೀಜಿಯವರ ಬಲಗೈ”, ಜೆ. ಜೆ. ಗುಡ್​ವಿನ್.\\
೨೨, ೨೩. ಅದ್ವೈತಾಶ್ರಮವನ್ನು ಸ್ಥಾಪಿಸುವಲ್ಲಿ ನೆರವಾದ ಸೇವಿಯರ್ ದಂಪತಿಗಳು.\\
೨೪. “ಆಧುನಿಕ ಸಾಯಣಾಚಾರ್ಯ” ಪ್ರೊ ॥ ಮ್ಯಾಕ್ಸ್​ಮ್ಯುಲ್ಲರ್.\\
೨೫. ಡಾ ॥ ಪಾಲ್ ಡಾಯನ್ಸ್.\\
೨೬. ಅತ್ಯಂತ ಆಪ್ತ ಶಿಷ್ಯ, ಬೆಂಬಲಿಗ, ಖೇತ್ರಿಯ ಮಹಾರಾಜ ಅಜಿತ್​ಸಿಂಗ್.\\
೨೭. ಅಂದಿನ ಮೈಸೂರು ಮಹಾರಾಜರಾದ ಚಾಮರಾಜ ಒಡೆಯರ್.\\
೨೮. ರಾಮನಾಡಿನ ಅರಸ ಭಾಸ್ಕರ ಸೇತುಪತಿ.\\
೨೯. ನೆಚ್ಚಿನ ಮದ್ರಾಸೀ ಶಿಷ್ಯ ಅಳಸಿಂಗ ಪೆರುಮಾಳ್.\\
೩ಂ. ವಿವೇಕಾನಂದರು ಬರೆದ ಪತ್ರವೊಂದರ ಛಾಯಾಚಿತ್ರ.

