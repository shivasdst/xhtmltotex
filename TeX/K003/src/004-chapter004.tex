
\chapter{ಮಾರ್ಗಪ್ರವರ್ತನ–ಮಾರ್ಗದರ್ಶನ}

\noindent

ಈಗ ನಾವು ಸ್ವಾಮೀಜಿಯನ್ನು ನೋಡುವುದು ಗುಜರಾತಿನ ಪಾಲಿತಾನ ಎಂಬಲ್ಲಿ. ಇಲ್ಲಿ ಜೈನರ ಯಾತ್ರಾಸ್ಥಳವಾದ ಶತ್ರುಂಜಯವೆಂಬ ಬೆಟ್ಟವಿದೆ. ಸ್ವಾಮೀಜಿ ಈ ಬೆಟ್ಟವನ್ನೇರಿ, ಸುತ್ತಲಿನ ವಿಹಂಗಮ ದೃಶ್ಯವನ್ನು ಕಂಡು ಆನಂದಿಸಿದರು. ಪಾಲಿತಾನದಲ್ಲಿ ಅವರು ಇತರ ಹಲವಾರು ದೇವ ಸ್ಥಾನಗಳನ್ನೂ ಮುಸಲ್ಮಾನ ಸಂತನೊಬ್ಬನ ದರ್ಗಾವನ್ನೂ ವೀಕ್ಷಿಸಿದರು. ಕಾಥೇವಾಡದಲ್ಲಿ ಸ್ವಾಮೀಜಿ ಸಂದರ್ಶಿಸಿದ ಸ್ಥಳಗಳಲ್ಲಿ ಇದೇ ಕೊನೆಯದೆಂಬಂತೆ ತೋರುತ್ತದೆ.

ಸ್ವಾಮೀಜಿಯ ಸ್ನೇಹಿತರಾದ ಹರಿದಾಸ್ ದೇಸಾಯಿ (ದಿವಾನ್​ಜಿ ಸಾಹೇಬ್​), ಅವರನ್ನು ತನ್ನ ಸ್ವಂತ ಊರಾದ ನದಿಯಾದ್ ಎಂಬಲ್ಲಿಗೆ ಭೇಟಿಮಾಡಿ ಹೋಗಬೇಕೆಂದು ಪ್ರಾರ್ಥಿಸಿಕೊಂಡಿ ದ್ದರು. ಅದರಂತೆ ಸ್ವಾಮೀಜಿ ನದಿಯಾದ್​ಗೆ ಹೋದಾಗ ದಿವಾನ್​ಜಿಯ ಸೋದರರು ಅವರನ್ನು ಹಾರ್ದಿಕವಾಗಿ ಸ್ವಾಗತಿಸಿ ಸತ್ಕರಿಸಿದರು. ಅವರ ಸದ್ವರ್ತನೆಯಿಂದ ಸಂಪ್ರೀತರಾದ ಸ್ವಾಮೀಜಿ ದಿವಾನ್​ಜಿಯ ಕುಟುಂಬವರ್ಗದವರ ಸೇವೆಯನ್ನು ಕೊಂಡಾಡಿ, ಅವರ ಮೇಲೆ ಭಗವಂತ ಕೃಪಾವೃಷ್ಟಿಯನ್ನು ಕರೆಯಲಿ ಎಂದು ಹಾರೈಸುತ್ತ ದೇಸಾಯಿಗೆ ಪತ್ರವೊಂದನ್ನು ಬರೆಯುತ್ತಾರೆ.

ದಿವಾನ್​ಜಿ ಬರೋಡ ಹಾಗೂ ಇತರ ರಾಜ್ಯಗಳ ಹಲವಾರು ಪ್ರಮುಖರಿಗೆ ಸ್ವಾಮೀಜಿಯ ಬಗ್ಗೆ ಪರಿಚಯಪತ್ರಗಳನ್ನು ಕೊಟ್ಟಿದ್ದರು. ನದಿಯಾದ್​ನಿಂದ ಹೊರಟ ಸ್ವಾಮೀಜಿ ಬರೋಡಕ್ಕೆ ಬಂದರು. ಇಲ್ಲಿ ಅವರು ರಾಜ್ಯದ ಮಂತ್ರಿಗಳಲ್ಲೊಬ್ಬನಾದ ಬಹಾದೂರ್ ಮಣಿಭಾಯ್ ಎಂಬ ವನ ಅತಿಥಿಯಾಗಿ ಉಳಿದುಕೊಂಡರು. ಈತನ ಮುಖಾಂತರ ಅವರು ದಿವಾನನನ್ನೂ ಬರೋಡದ ಮಹಾರಾಜ ಸಯ್ಯಾಜಿರಾವ್​ಗಾಯಕವಾಡನನ್ನೂ ಭೇಟಿ ಮಾಡಿದರು. ಮಹಾರಾಜ ಗಾಯಕವಾಡ ಅತ್ಯಂತ ಯೋಗ್ಯ, ಸಮರ್ಥ ಆಡಳಿತಗಾರರಲ್ಲೊಬ್ಬನೆಂಬ ವಿಖ್ಯಾತಿ ಹೊಂದಿದ್ದವನು. ಅಲ್ಲದೆ ಭಾರತದ ಸ್ವಾತಂತ್ರ್ಯ ಚಳುವಳಿಯ ಬಗ್ಗೆ ಸಹಾನುಭೂತಿಯಿದ್ದವನು. ಸ್ವಾಮೀಜಿ ಈತನೊಂದಿಗೆ ನಡೆಸಿದ ಮಾತುಕತೆಯ ಬಗ್ಗೆ ಏನೂ ತಿಳಿದಿಲ್ಲ. ಆದರೆ ಸ್ವಾಮೀಜಿ ಈತನಿಂದ ತುಂಬ ಪ್ರಭಾವಿತ ರಾದರೆನ್ನುವುದು ಖಂಡಿತ. ಮುಂದೆ ಅವರು ತಿರುವನಂತಪುರದ ರಾಜಕುಮಾರ ಮಾರ್ತಾಂಡ ವರ್ಮನೊಂದಿಗೆ ಮಾತನಾಡುವಾಗ ಬರೋಡದ ಮಹಾರಾಜ ಸಯ್ಯಾಜಿರಾವ್ ಗಾಯಕವಾಡ ನನ್ನು ಆತನ ಸಾಮರ್ಥ್ಯ, ದೇಶಪ್ರೇಮ, ದೂರಾಲೋಚನೆಗಳಿಗಾಗಿ ಕೊಂಡಾಡುತ್ತಾರೆ.

ಬರೋಡದಲ್ಲಿ ಸ್ವಾಮೀಜಿ ಮಹಾಕಲಾವಿದ ರಾಜಾ ರವಿವರ್ಮನ ವರ್ಣಚಿತ್ರಗಳನ್ನು ನೋಡಿ ಆನಂದಿಸಿದರು. ಅಲ್ಲದೆ ರಾಜ್ಯದ ಗ್ರಂಥಾಲಯವನ್ನೂ ಸಂದರ್ಶಿಸಿ ಅದನ್ನು ಮೆಚ್ಚಿಕೊಂಡರು. ಈ ವೇಳೆಗೆ ಅವರ ಸ್ನೇಹಿತನಾದ ಲಿಂಬ್ಡಿಯ ರಾಜಠಾಕೂರ್ ಸಾಹೇಬ ಮಹಾಬಲೇಶ್ವರಕ್ಕೆ ಬರ ಲಿದ್ದ. ಅವನನ್ನು ಭೇಟಿ ಮಾಡುವ ಉದ್ದೇಶದಿಂದ, ಪ್ರಾಯಶಃ ಮೊದಲೇ ನಿಶ್ಚಯಿಸಿದ ಕಾರ್ಯ ಕ್ರಮದಂತೆ, ೧೮೯೨ರ ಏಪ್ರಿಲ್​ನಲ್ಲಿ ಸ್ವಾಮೀಜಿ ಮಹಾಬಲೇಶ್ವರಕ್ಕೆ ತೆರಳಿದರು. ಇದೊಂದು ಸುಂದರ ಗಿರಿಧಾಮ. ಇಲ್ಲಿ ಅವರು ಮೊರಾರ್ಜಿ ಗೋಕುಲದಾಸ್ ಎಂಬವನ ಅತಿಥಿಯಾಗಿ ಇಳಿದುಕೊಂಡರು. ಸ್ವಾಮೀಜಿಯ ಆಗಮನ ಮಹಾಬಲೇಶ್ವರದಲ್ಲಿ ಒಂದು ಅಲ್ಲೋಲಕಲ್ಲೋಲ ವನ್ನೇ ಉಂಟುಮಾಡಿತು. ಅವರನ್ನು ದರ್ಶಿಸಲು ಜನರು ಗುಂಪಾಗಿ ಬರಲಾರಂಭಿಸಿರು. ಪೂನಾ ದಿಂದ ಮಹಾಬಲೇಶ್ವರಕ್ಕೆ ಬಂದಿದ್ದ ಕೆಲವು ವಕೀಲರು ‘ಯಾರೋ ಒಬ್ಬ ಅಸಾಮಾನ್ಯ ಬಂಗಳೀ ಸಾಧು’ ಬಂದಿರುವ ವಿಷಯ ಕೇಳಿ ಅವರನ್ನು ನೋಡಲು ಹೋದರು. ಹಲವಾರು ಧಾರ್ಮಿಕ, ಆಧ್ಯಾತ್ಮಿಕ ವಿಚಾರಗಳ ಬಗ್ಗೆ ಹೇಳುವಾಗ ಅವರು ನೀಡುತ್ತಿದ್ದ ವಿವರಣೆಯನ್ನು ಕೇಳಿದಮೇಲೆ, ಅವರೊಬ್ಬ ಮಹಾಪುರುಷನೇ ಸರಿ ಎಂದು ಆ ವಕೀಲರಿಗೆ ಅನ್ನಿಸಿತು. ಅಲ್ಲದೆ ಅವರ ಇಂಗ್ಲಿಷ್ ಭಾಷಾ ಪ್ರಭುತ್ವವನ್ನು ಕಂಡು ಆ ವಕೀಲರು ವಿಸ್ಮಯಮೂಕರಾದರು.

ಗೋಕುಲ್​ದಾಸನ ಮನೆಯಲ್ಲಿ ಇಳಿದುಕೊಂಡಿದ್ದ ಸಂದರ್ಭದಲ್ಲಿ ಸ್ವಾಮೀಜಿಗೆ ತಮ್ಮ ಗುರುಭಾಯಿ ಸ್ವಾಮಿ ಅಭೇದಾನಂದರನ್ನು ಮತ್ತೊಮ್ಮೆ ಸಂಧಿಸುವ ಅವಕಾಶ ಒದಗಿಬಂದಿತು. ಕೆಲತಿಂಗಳ ಹಿಂದೆಯಷ್ಟೇ ಅವರನ್ನು ಸ್ವಾಮೀಜಿ ಜುನಾಗಢದಲ್ಲಿ ಭೇಟಿಯಾಗಿದ್ದರು. ಪಶ್ಚಿಮ ಭಾರತದಲ್ಲಿ ಪರಿವ್ರಾಜಕರಾಗಿ ಸುತ್ತಾಡುತ್ತಿದ್ದ ಅಭೇದಾನಂದರು ಮುಂಬಯಿಯಿಂದ ಮಹಾ ಬಲೇಶ್ವರಕ್ಕೆ ಬಂದರು. ಇಲ್ಲಿ ಗೋಕುಲ್​ದಾಸ್ ಎಂಬ ಸಹೃದಯರ ಮನೆಯಲ್ಲಿ ಸಾಧುಗಳಿಗೆ ಸ್ವಾಗತವಿದೆಯೆಂದು ಕೇಳಿತಿಳಿದ ಅಭೇದಾನಂದರು ಅಲ್ಲಿಗೆ ಹೋಗಿ ನೋಡುತ್ತಾರೆ–ತಮ್ಮ ನರೇಂದ್ರ ಹಿಂದಿನ ದಿನವಷ್ಟೇ ಅಲ್ಲಿಗೆ ಬಂದಿದ್ದಾನೆ! ಅಭೇದಾನಂದರನ್ನು ನೋಡಿ ಸ್ವಾಮೀಜಿ ನಕ್ಕು ಹೇಳಿದರು, “ಸೋದರ, ನನ್ನನ್ನೇಕೆ ವ್ಯರ್ಥವಾಗಿ ಹಿಂಬಾಲಿಸುತ್ತಿದ್ದೀಯೆ? ನಾವಿಬ್ಬರೂ ಠಾಕೂರರ ಹೆಸರಿನಲ್ಲೇ ಹೊರಟಿದ್ದೇವೆ. ಆದ್ದರಿಂದ, ನಾವು ಬೇರೆಬೇರೆಯಾಗಿ ಸಂಚರಿಸುವುದೇ ಒಳ್ಳೆಯದು.” ಆಗ ಅಭೇದಾನಂದರು, “ಅದು ಸರಿಯೇ, ಆದರೆ ನಾನೇಕೆ ನಿನ್ನ ಬೆನ್ನಟ್ಟಲಿ! ಇಲ್ಲಿಗೆ ನೀನು ಬಂದಂತೆಯೇ ನಾನೂ ಬಂದಿದ್ದೇನೆ, ಅಷ್ಟೆ” ಎಂದರು. ಇದನ್ನು ಕೇಳಿ ಸ್ವಾಮೀಜಿ ಗಟ್ಟಿಯಾಗಿ ನಕ್ಕರು. ಆಗ ಅಭೇದಾನಂದರು ಸ್ವಲ್ಪ ಹಾಸ್ಯವಾಗಿಯೇ ಹೇಳಿದರು “ಹೋಗಲಿ ಬಿಡು; ಈ ಸಲ ನಾನು ಪೂನಾ, ಬರೋಡ, ನಾಸಿಕ್ ಮಾರ್ಗವಾಗಿ ದಕ್ಷಿಣ ಭಾರತಕ್ಕೆ ಹೋಗು ತ್ತೇನೆ. ನೀನು ಉತ್ತರಭಾರತಕ್ಕೆ ಹೋಗು. ಆಗ ನಾವಿಬ್ಬರೂ ಮತ್ತೆ ಭೇಟಿಯಾಗುವ ಸಂಭವವಿರು ವುದಿಲ್ಲ.” ಈಗ ಸ್ವಾಮೀಜಿ ಮತ್ತೊಮ್ಮೆ ಹಾರ್ದಿಕವಾಗಿ ನಕ್ಕರು. (ಆದರೆ ನಿಜಕ್ಕೂ ದಕ್ಷಿಣ ಭಾರತದ ಕಡೆಗೇ ಅವರೂ ಹೊರಟಿದ್ದು.) ಇದನ್ನೆಲ್ಲ ನೋಡುತ್ತಿದ್ದ ಗೋಕುಲ್​ದಾಸನಿಗೆ ವಿಷಯ ವೇನೆಂಬುದು ತಿಳಿಯಲಿಲ್ಲ. ಆದರೆ ಅವರು ನಗುತ್ತ ಮಾತನಾಡುವುದನ್ನು ಕಂಡು ಅವನಿಗೂ ಸಂತೋಷವಾಗಿರಬೇಕು. “ನಿಮ್ಮಂತಹ ಇಬ್ಬರು ಮಹಾತ್ಮರನ್ನು ಒಟ್ಟಿಗೆ ಭೇಟಿ ಮಾಡುವಂತಾ ದದ್ದೇ ನನ್ನ ಪುಣ್ಯವಿಶೇಷ!” ಎಂದು ತನ್ನ ಸಂತೋಷವನ್ನು ವ್ಯಕ್ತಪಡಿಸಿದ.

ಗೋಕುಲ್​ದಾಸನ ಒತ್ತಾಯದ ಪ್ರಾರ್ಥನೆಗೆ ಮಣಿದು ಅಭೇದಾನಂದರು ಅವನ ಮನೆಯಲ್ಲಿ ಮೂರು ದಿನ ಉಳಿದುಕೊಂಡರು. ನಾಲ್ಕನೆಯ ದಿನ ಅವರು ಪೂನಾಕ್ಕೆ ಹೊರಟುನಿಂತಾಗ ಸ್ವಾಮೀಜಿ ಹೇಳಿದರು, “ಧೈರ್ಯವಾಗಿರು. ನಾವು ಶ್ರೀ ಗುರುಮಹಾರಾಜರ ಹೆಸರಿನಲ್ಲಿ ಹೊರಟಿ ರುವುದರಿಂದ, ಖಂಡಿತವಾಗಿ ನಮಗೆ ಎಲ್ಲದಕ್ಕೂ ಅವರೇ ವ್ಯವಸ್ಥೆ ಮಾಡಿಕೊಡುತ್ತಾರೆ.”

ಈ ದಿನಗಳಲ್ಲಿ ಸ್ವಾಮೀಜಿಯ ಭಾವಸಮುದ್ರ ಅಲ್ಲೋಲಕಲ್ಲೋಲಗೊಂಡಿದ್ದು, ಭಾರತದ ಪುನರ್ನಿರ್ಮಾಣದ ಕುರಿತಾದ ಭಾವತರಂಗಗಳು ಅವರಲ್ಲಿ ಮತ್ತೆಮತ್ತೆ ಪುಟಿದೇಳುವುದನ್ನು ಅಭೇದಾನಂದರು ಗಮನಿಸಿದರು. ಸ್ವಾಮೀಜಿಯ ಪ್ರಕ್ಷುಬ್ಧ ಮನಸ್ಥಿತಿಯನ್ನು ಕಂಡು ಅವರಿಗೆ ಆಶ್ಚರ್ಯದೊಂದಿಗೆ ಸ್ವಲ್ಪ ಭಯವೂ ಆಯಿತು. ‘ಭುಗಿಲೆದ್ದು ಮುಗಿಲೆತ್ತರಕ್ಕೆ ಉರಿಯುತ್ತಿರುವ ಜ್ವಲಂತ ಚೇತನದಂತಿದ್ದರು ಅವರು’ ಎಂದು ಅಭೇದಾನಂದರು ತಮ್ಮ ಆತ್ಮಕಥೆಯಲ್ಲಿ ಬರೆಯುತ್ತಾರೆ. ಮನಬಂದೆಡೆಗೆ ನುಗ್ಗಿ, ಅಡೆತಡೆಗಳನ್ನು ಲೆಕ್ಕಿಸದೆ ತಗನೆದುರಾದದ್ದನ್ನೆಲ್ಲ ಬೀಸಿ ದೂರಕ್ಕೆಸೆಯುವ ಚಂಡಮಾರುತದಂತಿದ್ದರು ಸ್ವಾಮೀಜಿ! ಅವರು ತಮ್ಮ ಸಂನ್ಯಾಸೀ ಸೋದರ ನಿಗೆ ಹೇಳುತ್ತಾರೆ, “ನನ್ನೊಳಗೆ ಎಂತಹ ಅದಮ್ಯ ಶಕ್ತಿಚೈತನ್ಯ ಪುಟಿದೇಳುತ್ತಿದೆಯೆಂದರೆ, ನಾನಿನ್ನೇನು ಸಿಡಿದುಬಿಡುತ್ತೇನೋ ಎಂಬಂತೆ ಭಾಸವಾಗುತ್ತಿದೆ.”

ಸ್ವಾಮೀಜಿ ಮಹಾಬಲೇಶ್ವರದಲ್ಲಿ ಕಳೆದ ದಿನಗಳ ಬಗ್ಗೆ ಇತ್ತೀಚೆಗೆ, ಎಂದರೆ ೧೯೭೬ರಲ್ಲಿ ಹೊಸ ಮಾಹಿತಿ ದೊರಕಿದೆ. ಇಲ್ಲಿ ಅವರು ಡಾ ॥ ವಿಶ್ವನಾಥ ಜೋಶಿ ಎಂಬ ವಕೀಲರ ಅತಿಥಿ ಯಾಗಿ ಸುಮಾರು ಒಂದು ತಿಂಗಳ ಕಾಲ ಇಳಿದುಕೊಂಡಿದ್ದರೆಂದು, ಅವರ ಮಗಳಾದ ಶ್ರೀಮತಿ ಲಕ್ಷ್ಮೀಬಾಯಿ ರಾಜವಾಡೆಯವರಿಂದ ತಿಳಿದು ಬಂದಿದೆ. ಇಲ್ಲಿಗೆ ಜೋಶಿಯವರು ಬೇಸಿಗೆಯನ್ನು ಕಳೆಯಲು ಬಂದಿದ್ದು, ಒಂದು ಪುಟ್ಟ ಮನೆಯನ್ನು ಬಾಡಿಗೆಗೆ ಪಡೆದಿದ್ದರು. ಒಂದು ದಿನ ಇವರು ಆಕಸ್ಮಿಕವಾಗಿ ಸ್ವಾಮೀಜಿಯನ್ನು ಸಂಧಿಸಿದರು. ಬಹುಶಃ ಆಗಿನ್ನೂ ಸ್ವಾಮೀಜಿ ಗೋಕುಲದಾಸನ ಮನೆಗೆ ಹೋಗಿರಲಿಲ್ಲವೆಂದು ಕಾಣುತ್ತದೆ. ತಾವು ಧ್ಯಾನ ಮಾಡಲು ಸೂಕ್ತವಾದ ಒಂದು ಕೋಣೆ ಯನ್ನು ಹುಡುಕುತ್ತಿರುವುದಾಗಿ ಅವರು ಹೇಳಿದಾಗ, ಜೋಶಿಯವರು ಅವರನ್ನು ತಮ್ಮ ಮನೆಗೇ ಆಹ್ವಾನಿಸಿದರು.

ಅವರ ಮನೆಯಲ್ಲಿ ಒಂದು ಅತ್ಯಂತ ಕುತೂಹಲಕರ ಘಟನೆ ನಡೆಯಿತು. ಆಗ ಆರು ವರ್ಷದವಳಾಗಿದ್ದ ಲಕ್ಷ್ಮೀಬಾಯಿಗೆ ಕಮಲ ಎಂಬ ಐದಾರು ತಿಂಗಳ ಪುಟ್ಟ ತಂಗಿಯಿದ್ದಳು. ಪ್ರತಿದಿನವೂ ಈ ಮಗು ಸಂಜೆಯಾಗುತ್ತಿದ್ದಂತೆ ತಂಟೆ ಮಾಡಲು ಪ್ರಾರಂಭಿಸುತ್ತಿತ್ತು. ಅಲ್ಲದೆ ರಾತ್ರಿಯೆಲ್ಲ ಒಂದೇ ಸಮನೆ ಅಳುತ್ತಿತ್ತು. ಇದರಿಂದಾಗಿ ಆಕೆಯ ತಾಯಿ ಯಶೋದಬಾಯಿಗೆ ನಿದ್ರೆಯೇ ಇಲ್ಲದಂತಾಗುತ್ತಿತ್ತು.

ಇದನ್ನು ಸ್ವಾಮೀಜಿ ಎರಡು ದಿನ ನೋಡಿದರು. ಬಳಿಕ ಆ ತಾಯಿಯ ಬಗ್ಗೆ ಮರುಕಗೊಂಡು ಆಕೆಗೆ ಹೇಳಿದರು, “ಅಮ್ಮ, ನೀವು ದಿನವಿಡೀ ಕೆಲಸ ಮಾಡುತ್ತಲೇ ಇರುತ್ತೀರಿ. ನಿಮಗೆ ವಿಶ್ರಾಂತಿ ಸಿಗುವುದೇ ಇಲ್ಲ. ರಾತ್ರಿಯೂ ನಿದ್ರೆಯಿಲ್ಲವೆಂದರೆ ಹೇಗಾದೀತು? ನೀವು ರಾತ್ರಿ ವೇಳೆ ಮಗು ವನ್ನು ನನ್ನ ಕೈಗೆ ಕೊಡಿ; ನಾನು ನೋಡಿಕೊಳ್ಳುತ್ತೇನೆ.” ಆಕೆ ಅವರ ಸದ್ಭಾವಕ್ಕಾಗಿ ಕೃತಜ್ಞತೆ ಸಲ್ಲಿಸಿದಳು. ಆದರೆ ಅಂತಹ ತಂಟೆಕೋರ ಮಗುವನ್ನು ಸಾಧುಗಳ ಕೈಗೆ ಕೊಟ್ಟು ಅವರಿಗೂ ತೊಂದರೆ ಕೊಡಲು ಆಕೆಯ ಮನಸ್ಸು ಒಪ್ಪಲಿಲ್ಲ. ಅಲ್ಲದೆ, ತನ್ನ ತಾಯಿ ಮಡಿಲಲ್ಲೇ ಸುಮ್ಮನೆ ಮಲಗಿಕೊಳ್ಳದ ಮಗು ಮತ್ತೊಬ್ಬರ ಬಳಿ–ಅದೂ ಒಬ್ಬ ಪುರುಷನ ಬಳಿ–ಸುಮ್ಮನಿದ್ದೀತೆ! ಆದರೆ ಸ್ವಾಮೀಜಿ ಬಿಡಲಿಲ್ಲ. ಜೋಶಿಯವರ ಬಳಿಗೆ ಹೋಗಿ ಅದೇ ಕೋರಿಕೆಯನ್ನು ಮುಂದಿಟ್ಟು, “ಸುಮ್ಮನೆ ಒಂದು ಸಲ ಪ್ರಯತ್ನ ಮಾಡಿ ನೋಡಬಾರದೇಕೆ?” ಎಂದು ಕೇಳಿದರು. ವಿಧಿಯಿಲ್ಲದೆ ಮಗುವನ್ನು ಅವರ ಕೈಗೆ ಕೊಡಲಾಯಿತು

ಸ್ವಾಮೀಜಿ ಮಗುವನ್ನು ತಮ್ಮ ತೊಡೆಯ ಮೇಲೆ ಮಲಗಿಸಿಕೊಂಡು ಕುಳಿತರು. ಮಗು ನಿದ್ರಿಸುವಂತೆ ಮಾಡಲು ಅವರು ಲಾಲಿ ಹಾಡಲಿಲ್ಲ. ಅಥವಾ ಅದರ ಬಾಯಲ್ಲಿ ಬೆಟ್ಟನ್ನೂ ಇಡ ಲಿಲ್ಲ. ತಮ್ಮಷ್ಟಕ್ಕೆ ತಾವು ಧ್ಯಾನಮಗ್ನರಾಗಿ ಕುಳಿತರು. ಏನಾಶ್ಚರ್ಯ! ಸ್ವಲ್ಪವೂ ತಂಟೆ ಮಾಡದೆ ಮಗು ಶಾಂತವಾಗಿ ನಿದ್ರಿಸಿತು! ಅಥವಾ ಮಗುವೂ ಧ್ಯಾನ ಮಗ್ನವಾಯಿತೋ?... ಅಂತೂ ಇದು ಹೀಗೆಯೇ ಎಷ್ಟೋ ದಿನ ನಡೆಯಿತು.

ಪುಟ್ಟ ಕಮಲೆ ಬೆಳೆದು ದೊಡ್ಡವಳಾದ ಮೇಲೆ ಸಾಂಗ್ಲಿಯ ಮಹಾರಾಜನ ಪಟ್ಟದರಸಿ ಯಾದಳು. ಅವಳ ಜೀವನದುದ್ದಕ್ಕೂ, ಎಂತಹ ಉದ್ವೇಗಕರ ಸನ್ನಿವೇಶವೇ ಒದಗಿದರೂ ಅವಳ ಮನಶ್ಶಾಂತಿ ಕದಡುತ್ತಿರಲಿಲ್ಲ. ಆಕೆಯ ನಿರ್ವಿಕಾರ ಮುಖಭಾವ ಅದನ್ನು ಸ್ಪಷ್ಟವಾಗಿ ಪ್ರತಿಬಿಂಬಿ ಸುತ್ತಿತ್ತು. ಇದನ್ನು ಕಂಡವರು, ಅಂದು ಸ್ವಾಮೀಜಿಯಿಂದ ಆಕೆ ಪಡೆದುಕೊಂಡ ಕೃಪೆಯ ಫಲ ಇದು ಎಂದು ಮಾತನಾಡಿಕೊಳ್ಳುತ್ತಿದ್ದರು.

ಮಹಾಬಲೇಶ್ವರದಲ್ಲಿ ಸುಮಾರು ಎರಡೂವರೆ ತಿಂಗಳಷ್ಟು ಕಾಲ ಇದ್ದ ಸ್ವಾಮೀಜಿ, ಮಳೆ ಗಾಲದ ಪ್ರಾರಂಭದಲ್ಲಿ ಮಹಾರಾಜ ಠಾಕೂರ್ ಸಾಹೇಬನೊಂದಿಗೆ ಪೂನಾಗೆ ಹೊರಟರು. ಇಲ್ಲಿಂದ ಮುಂದೆ ಹೈದರಾಬಾದಿನ ಮೂಲಕ, ಚತುರ್ಧಾಮಗಳಲ್ಲೊಂದಾದ ರಾಮೇಶ್ವರಕ್ಕೆ ಹೋಗಬೇಕೆನ್ನುವುದು ಅವರ ಉದ್ದೇಶವಾಗಿತ್ತು. ಆದರೆ ಸ್ವಾಮೀಜಿ ತಾವಿನ್ನು ಹೊರಡುವುದಾಗಿ ಹೇಳಿದಾಗ ಠಾಕೂರ್ ಸಾಹೇಬನಿಗೆ ಅವರನ್ನು ಬೀಳ್ಕೊಡಲು ಸ್ವಲ್ಪವೂ ಇಷ್ಟವಿರಲಿಲ್ಲ. ತನ್ನೊಡನೆ ಲಿಂಬ್ಡಿಗೆ ಬಂದು ಅಲ್ಲೇ ನೆಲಸಿಬಿಡಬೇಕೆಂದು ಅವರನ್ನು ಬಹಳವಾಗಿ ಒತ್ತಾಯಸಿದ. ಆಗ ಸ್ವಾಮೀಜಿ, “ಇಲ್ಲ ಮಹಾರಾಜ, ನಾನಿನ್ನೂ ಮಾಡಬೇಕಾದದ್ದು ಬಹಳವಿದೆ. ಈಗ ನಾನು ವಿಶ್ರಮಿಸುವಂತೆಯೇ ಇಲ್ಲ. ಆದರೆ, ಮುಂದೆಂದಾದರೂ ನಾನು ನಿವೃತ್ತ ಜೀವನವನ್ನು ನಡೆಸುವು ದಾದರೆ ಅದು ನಿನ್ನ ಜೊತೆಯಲ್ಲೇ” ಎಂದು ಸಮಾಧಾನ ಹೇಳಿದರು. ಕರ್ಮಕ್ಷೇತ್ರದಲ್ಲೇ ಕೊನೆಯುಸಿರಿನವರೆಗೂ ಹೋರಾಡಿದ ಸ್ವಾಮೀಜಿಗೆ ನಿವೃತ್ತಿಯೆ!

ಪೂನಾದಿಂದ ತಮ್ಮ ಸ್ನೇಹಿತ ಹರಿದಾಸ್ ದೇಸಾಯಿಗೆ ಬರೆದ ಪತ್ರವೊಂದರಲ್ಲಿ ತಾವು ಹೈದರಾಬಾದಿಗೆ ಹೋಗುವುದಾಗಿ ಬರೆದಿದ್ದರು ಸ್ವಾಮೀಜಿ. ಆದರೆ ಪೂನಾದಿಂದ ಅವರು ಬಂದದ್ದು ಆಗಿನ ‘ಮಧ್ಯಪ್ರಾಂತ’ಗಳಿಗೆ ಸೇರಿದ ಖಾಂಡ್ವಾ ಎಂಬಲ್ಲಿಗೆ. ಖಾಂಡ್ವಾದ ರಸ್ತೆಗಳಲ್ಲಿ ನಡೆಯುತ್ತ ಅವರು ಅಲ್ಲಿನ ವಕೀಲನಾದ ಹರಿದಾಸ ಚಟರ್ಜಿ ಎಂಬವನ ಮನೆಯ ಬಳಿಗೆ ಬಂದರು. ಆ ವೇಳೆಗೆ ನ್ಯಾಯಾಲಯದಿಂದ ಮನೆಗೆ ಹಿಂದಿರುಗಿದ್ದ ಹರಿದಾಸ, ಸ್ವಾಮೀಜಿಯನ್ನು ಕಂಡು ಮಾತನಾಡಿಸಿದ. ಮೊದಲಿಗೆ ಅವರನ್ನು ಯಾರೋ ಒಬ್ಬ ಸಾಧಾರಣ ಸಾಧುವಿರಬೇಕು ಎಂದು ಭಾವಿಸಿದ. ಸಾಮಾನ್ಯವಾಗಿ ಸ್ವಸ್ಥಳದಿಂದ ದೂರವಿರುವ ಬಂಗಾಳಿಗಳು ತಮ್ಮವರನ್ನು ಕಂಡರೆ ಸ್ವಾಗತಿಸುತ್ತಾರೆ. ಪ್ರಾಯಶಃ ಆ ಭಾವನೆಯಿಂದಲೇ, ಹರಿದಾಸ ಅವರನ್ನು ಒಳಗೆ ಬರ ಮಾಡಿಕೊಂಡ. ಆದರೆ ಸ್ವಲ್ಪ ಹೊತ್ತಿನಲ್ಲೇ ಅವನಿಗೆ ಅರಿವಾಯಿತು, ಇಂತಹವರನ್ನು ತನ್ನ ಅತಿಥಿ ಯಾಗಿ ಪಡೆಯಬೇಕಾದರೆ ತಾನು ಅಪೂರ್ವ ಭಾಗ್ಯಶಾಲಿಯಾಗಿರಬೇಕು ಎಂದು. ತನ್ನ ಮನೆ ಯಲ್ಲೇ ಉಳಿದುಕೊಳ್ಳಬೇಕೆಂದು ಸ್ವಾಮೀಜಿಯನ್ನು ಪ್ರಾರ್ಥಿಸಿಕೊಂಡು ಒಪ್ಪಿಸಿದ. ಸ್ವಾಮೀಜಿ ಆತನ ಮನೆಯಲ್ಲಿ ಸುಮಾರು ಮೂರು ವಾರಗಳ ಕಾಲ ಉಳಿದುಕೊಂಡಿದ್ದರು. ಚಟರ್ಜಿಯ ಮನೆಯವರೆಲ್ಲರೂ ಅವರನ್ನು ತಮ್ಮ ಕುಟುಂಬಕ್ಕೇ ಸೇರಿದವರೋ ಎಂಬಷ್ಟು ಪ್ರೀತಿವಿಶ್ವಾಸ ದಿಂದ ನೋಡಿಕೊಂಡರು.

ಘಾಜೀಪುರ ಹಾಗೂ ಇತರ ಕೆಲವು ಸ್ಥಳಗಳಂತೆ, ಖಾಂಡ್ವಾದಲ್ಲೂ ಅಲ್ಲಿನ ನಿವಾಸಿಗಳಾದ ಬಂಗಾಳಿಗಳು ಹೆಚ್ಚಿನ ಸಂಖ್ಯೆಯಲ್ಲಿ ಬಂದು ಸ್ವಾಮೀಜಿಯನ್ನು ಸಂದರ್ಶಿಸಲಾರಂಭಿಸಿದರು. ಇವರಲ್ಲಿ ಆ ಊರಿನ ಹಲವಾರು ಪ್ರಮುಖರೂ ಇದ್ದರು. ಸ್ವಾಮೀಜಿಯ ಪ್ರಕಾಂಡ ಶಾಸ್ತ್ರಜ್ಞಾನ ಅಲ್ಲಿನ ಜನರನ್ನು ಬೆರಗುಗೊಳಿಸಿಬಿಟ್ಟಿತು. ಅವರ ನೀತಿ-ನಡತೆಗಳ ಬಗ್ಗೆ ಮುಂದೊಮ್ಮೆ ಹರಿದಾಸ ಬಾಬು ಹೇಳುತ್ತಾನೆ:

“ಸ್ವಾಮೀಜಿಯ ಮಾತುಕತೆಗಳಲ್ಲಿ ಅಹಂಭಾವದ ಸೋಂಕೇ ಇರಲಿಲ್ಲ. ಅವರ ಅತ್ಯುನ್ನತ ಭಾವನೆಗಳು ಹಾಗೂ ಉದಾತ್ತ ಚಿಂತನೆಗಳು ಸಲಭ ಸಹಜವಾಗಿ ಮಾತಿನ ರೂಪ ತಾಳಿ ತಾನೇ ತಾನಾಗಿ ಹರಿಯುತ್ತಿದ್ದುವು. ಅವರಿಗೆ ತಮ್ಮ ಬಗ್ಗೆ ಎಂತಹ ಪ್ರಾಮಾಣಿಕ ವಿಶ್ವಾಸವಿತ್ತೆಂದರೆ, ಅವರನ್ನು ನೋಡಿದವರಿಗೆಲ್ಲ ಅವರಲ್ಲೊಂದು ದಿವ್ಯ ಸ್ಫೂರ್ತಿ ಉಕ್ಕಿಹರಿಯುತ್ತಿದ್ದುದು ಗೋಚರವಾಗದೆ ಇರುತ್ತಿರಲಿಲ್ಲ... ”

ಖಾಂಡ್ವಾದಲ್ಲಿ ಅಕ್ಷಯಕುಮಾರ್ ಘೋಷ್ ಎಂಬೊಬ್ಬ ತರುಣ ಸ್ವಾಮೀಜಿಯನ್ನು ನೋಡಲು ಬರುತ್ತಿದ್ದ. ಆತ ನಿರುದ್ಯೋಗಿ; ಉದ್ಯೋಗವನ್ನರಸಿ ಕಲ್ಕತ್ತದಿಂದ ಇಲ್ಲಿಗೆ ಬಂದಿದ್ದ. ಸ್ವಾಮೀಜಿಗೆ ಇವನ ಕುಟುಂಬದವರ ಪರಿಚಯ ಮೊದಲಿನಿಂದಲೂ ಇತ್ತು. ಮುಂದೆ ಈತ ಸ್ವಾಮೀಜಿಯನ್ನು ಮತ್ತೆ ಮುಂಬಯಿಯಲ್ಲಿ ಸಂಧಿಸಿದ. ಆಗ ಅವರು ದಿವಾನ್​ಜಿ ಸಾಹೇಬರಿಗೆ ಇವನ ಬಗ್ಗೆ ಒಂದು ಪರಿಚಯ ಪತ್ರವನ್ನು ಕೊಟ್ಟರು. ಕೆಲವು ವರ್ಷಗಳಲ್ಲಿ ಈತ ಇಂಗ್ಲೆಂಡಿಗೆ ಹೋಗಿ, ಅಲ್ಲಿನ ಮಿಸ್ ಹೆನ್ರಿಟಾ ಮುಲ್ಲರ್ ಎಂಬವಳ ಸಂಪರ್ಕಕ್ಕೆ ಬಂದು, ಆಕೆಯ “ದತ್ತು ಪುತ್ರ”ನಾಗಿ ಸ್ವೀಕರಿಸಲ್ಪಟ್ಟ. ಅಲ್ಲದೆ, ಈತನಿಂದ ಸ್ವಾಮೀಜಿಯ ಬಗ್ಗೆ ಕೇಳಿ ತಿಳಿದ ಆ ಮಹಿಳೆ, ಆ ಸಮಯದಲ್ಲಿ ಅಮೆರಿಕದಲ್ಲಿದ್ದ ಸ್ವಾಮೀಜಿಯನ್ನು ಇಂಗ್ಲೆಂಡಿಗೆ ಆಹ್ವಾನಿಸಿ ತಾನು ಅವರ ಆತಿಥೇಯಳಾದಳು. ಹೀಗೆ ಸ್ವಾಮೀಜಿ ಇಂಗ್ಲೆಂಡಿಗೆ ಹೋಗುವಂತಾಗಲು ಈ ಅಕ್ಷಯಕುಮಾರ ನನ್ನು ಭಗವಂತನು ತನ್ನ ಕಾರ್ಯದಲ್ಲಿ ಒಂದು ಉಪಕರಣವಾಗಿ ಬಳಸಿಕೊಳ್ಳುವಂತೆ ಕಾಣುತ್ತದೆ.

ಒಂದು ದಿನ ಈ ಊರಿನ ಸಿವಿಲ್ ನ್ಯಾಯಾಧೀಶನಾದ ಮಾಧವಚಂದ್ರ ಬ್ಯಾನರ್ಜಿ ಎಂಬವನು ಸ್ವಾಮೀಜಿಯ ಗೌರವಾರ್ಥವಾಗಿ ಅಲ್ಲಿನ ಬಂಗಾಳಿಗಳಿಗೆಲ್ಲ ಒಂದು ಔತಣಕೂಟವನ್ನು ಏರ್ಪಡಿ ಸಿದ. ಈ ಸಂದರ್ಭದಲ್ಲಿ ಬಹುಶಃ ಅಲ್ಲೊಂದು ಉನ್ನತ ಆಧ್ಯಾತ್ಮಿಕ ವಾತಾವರಣವನ್ನು ನಿರ್ಮಾಣ ಮಾಡುವ ಉದ್ದೇಶದಿಂದ, ಭೋಜನಕ್ಕೆ ಮೊದಲು ಹಾಗೂ ಅನಂತರ, ಸ್ವಾಮೀಜಿ ಕೆಲವು ಉಪನಿಷತ್ತುಗಳಿಂದ ಆಯ್ದ ಭಾಗಗಳನ್ನು ಸುಶ್ರಾವ್ಯವಾಗಿ ಪಠಿಸಿ ಅರ್ಥವನ್ನು ವಿವರಿಸಿ ದರು. ಕೆಲವು ಅತ್ಯಂತ ಕ್ಲಿಷ್ಟ ಭಾಗಗಳನ್ನೂ ಕೂಡ ಒಂದು ಮಗುವಿಗೂ ಅರ್ಥವಾಗುವಂತೆ ಬಣ್ಣಿಸಿದರು. ಆಹ್ವಾನಿತರಲ್ಲಿ ಪ್ಯಾರಿಲಾಲ್ ಗಂಗೂಲಿ ಎಂಬ ಒಬ್ಬ ಸಂಸ್ಕೃತ ಪಂಡಿತನೂ ಇದ್ದ. ವೃತ್ತಿಯಿಂದ ಈತ ವಕೀಲ. ವಿಮರ್ಶಕನ ಪಾತ್ರವನ್ನು ವಹಿಸುವಲ್ಲಿ ಈತನಿಗೆ ತುಂಬ ಆಸಕ್ತಿ. ಅಲ್ಲದೆ ಅದರಲ್ಲಿ ಪಳಗಿದವನೂ ಕೂಡ. ಈಗ ಸ್ವಾಮೀಜಿ ಉಪನಿಷತ್ತುಗಳ ಕ್ಲಿಷ್ಟ-ವಿವಾದಾಸ್ಪದ ವಿಚಾರಗಳನ್ನು ವಿವರಿಸಹೊರಟಾಗ, ಅವರನ್ನು ಪ್ರಶ್ನೆಗಳಿಂದ ತೊಡರಿಸಲು ಸಾಧ್ಯವಾಗುತ್ತ ದೆಂದು ಈತ ನಿರೀಕ್ಷಿಸಿರಬೇಕು. ಆದರೆ ಅವುಗಳನ್ನು ಅವರು ವಿವರಿಸಿದ ಪರಿಯನ್ನು ಕಂಡು ಪಂಡಿತ ವಿಸ್ಮಯಮೂಕನಾಗಿ ಕುಳಿತುಬಿಟ್ಟ. “ಸ್ವಾಮೀಜಿಯ ಬಾಹ್ಯನೋಟವೇ ಅವರ ಮಹಾತ್ಮ್ಯ ವನ್ನು ಸಾರುತ್ತದೆ”ಎಂದು ಆಮೇಲೆ ಈತ ಉದ್ಗರಿಸಿದ.

ಬರುವ ವರ್ಷ, ಎಂದರೆ ೧೮೯೩ರಲ್ಲಿ ಅಮೆರಿಕದ ಶಿಕಾಗೋದಲ್ಲಿ ವಿಶ್ವಧರ್ಮ ಸಮ್ಮೇಳನ ವೊಂದು ನಡೆಯಲಿದೆಯೆಂಬ ಸುದ್ದಿ ಈ ಸಮಯದಲ್ಲಿ ಎಲ್ಲೆಲ್ಲೂ ಪ್ರಚಾರದಲ್ಲಿತ್ತು. ಕಾಥೇವಾಡ ದಲ್ಲಿದ್ದಾಗಲೇ ಆ ಬಗ್ಗೆ ಕೇಳಿದ್ದ ಸ್ವಾಮೀಜಿ, ಅದರಲ್ಲಿ ಭಾಗವಹಿಸುವ ಬಗ್ಗೆ ಆಲೋಚಿಸಿದ್ದರು. ಈಗ ಹರಿದಾಸ ಚಟರ್ಜಿ ಮತ್ತು ಅವನ ಸ್ನೇಹಿತರು ಹಿಂದೂಧರ್ಮದ ಪ್ರತಿನಿಧಿಯಾಗಿ ಈ ಸಮ್ಮೇಳನದಲ್ಲಿ ಭಾಗವಹಿಸಬೇಕೆಂದು ಅವರಿಗೆ ಸಲಹೆ ಮಾಡಿದರು. ಆಗ ಅವರು ತಮ್ಮ ಇಚ್ಛೆಯನ್ನು ಮತ್ತಷ್ಟು ಸ್ಪಷ್ಟವಾಗಿ ವ್ಯಕ್ತಪಡಿಸುತ್ತ, “ನನಗೆ ಅಲ್ಲಿಗೆ ಹೋಗಲು ದಾರಿಖರ್ಚನ್ನು ಯಾರಾದರೂ ಒದಗಿಸಿ ಕೊಡುವುದಾದರೆ, ಎಲ್ಲವೂ ಸುಸೂತ್ರವಾಗಿ ಸಾಗಿದರೆ, ನಾನು ಹೋಗುತ್ತೇನೆ” ಎಂದರು.

ಖಾಂಡ್ವಾದಲ್ಲಿದ್ದ ಮೂರು ವಾರಗಳ ಅವಧಿಯಲ್ಲೇ ಒಮ್ಮೆ ಸ್ವಾಮೀಜಿ, ಅಲ್ಲಿಂದ ೪೦ ಮೈಲಿ ದೂರದಲ್ಲಿರುವ ಇಂದೋರ್ ಹಾಗೂ ಅದರ ಸಮೀಪದ ಕೆಲವು ಗ್ರಾಮಾಂತರ ಪ್ರದೇಶ ಗಳನ್ನು ಸಂದರ್ಶಿಸಿದರು. ಈ ಹಳ್ಳಿಗಳಲ್ಲಿ ಅವರು ಶೈಕ್ಷಣಿಕವಾಗಿ ಸಾಮಾಜಿಕವಾಗಿ ಹಾಗೂ ಆರ್ಥಿಕವಾಗಿ ಹಿಂದುಳಿದ ಜನಸಮುದಾಯವನ್ನು ಹತ್ತಿರದಿಂದ ಕಂಡರು. ಅಲ್ಲದೆ ಇಲ್ಲಿ ಅವರು ಸಮಾಜದ ಅತಿ ಕೆಳಸ್ತರಕ್ಕೆ ಸೇರಿದ ಭಂಗಿಗಳ ಕುಟುಂಬವೊಂದರ ಜೊತೆಯಲ್ಲೂ ವಾಸಿಸಿದರು. ಹಿಂದೆಯೂ ತಮ್ಮ ಪರಿವ್ರಾಜಕ ಜೀವನದಲ್ಲಿ ಕೆಲವೊಮ್ಮೆ ಇಂತಹವರ ನಿಕಟಸಂಪರ್ಕಕ್ಕೆ ಬಂದಿದ್ದ ಸ್ವಾಮೀಜಿ, ಇವರಲ್ಲಡಗಿರುವ ಅಪಾರ ಸತ್ತ್ವವನ್ನು ಗಮನಿಸಿ, ಇವರ ದುಃಸ್ಥಿತಿಗಾಗಿ ದುಃಖದ ಕಣ್ಣೀರ್ಗರೆದರು. ಅಲ್ಲದೆ ಈ ಜನರು ಈ ಸ್ಥಿತಿಗೆ ಬರಲು ಕಾರಣರಾದವರ ಬಗ್ಗೆ ಅವ ರಲ್ಲಿ ರೋಷ ಉಕ್ಕಿತು. ಕೆಲದಿನಗಳ ಬಳಿಕ ಅವರು ತಮ್ಮ ಸ್ನೇಹಿತ ದಿವಾನ್​ಜಿಗೆ ಬರೆದ ಪತ್ರದಲ್ಲಿ ಅವರ ಅಂತರಾಳದ ಈ ಭಾವನೆಗಳನ್ನು ಗುರುತಿಸಬಹುದು. ಅದರಲ್ಲಿ ಅವರು ಬರೆಯುತ್ತಾರೆ, “... ಈ ಭಾಗದ ಜನರ (ಬ್ರಾಹ್ಮಣರ)ಪಾಲಿಗೆ ತಿನ್ನುವುದು ಕುಡಿಯುವುದು ಸ್ನಾನ ಮಾಡು ವುದು–ಇವುಗಳಿಗೆ ಸಂಬಂಧಿಸಿದ ಕೆಲವು ಮೂಡನಂಬಿಕೆಗಳ ಕಂತೆಯೇ ಧರ್ಮದ ಸಾರ ಸರ್ವಸ್ವ... ಇಲ್ಲಿನ ಹಿಂದುಳಿದವರ ಇಂದಿನ ಈ ದುರ್ಗತಿಗೆ ತಮಗಾಗಲಿ ತಮ್ಮ ಪೂರ್ವಜರಿಗಾಗಲಿ ಏನೇನೂ ತಿಳಿಯದಿದ್ದ ವೇದಗಳ ಹೆಸರಿನಲ್ಲಿ ಅರ್ಚಕಕುಲದ ಧೂರ್ತರು ಅವರಿಗೆ ಮಾಡಿದ ಬೋಧನೆಗಳೇ ಕಾರಣ.”

ದ್ವಾರಕೆಯಿಂದ ರಾಮೇಶ್ವರದ ಸಂದರ್ಶನಕ್ಕೆಂದು ಸ್ವಾಮೀಜಿ ಹೊರಟು ಕೆಲವು ತಿಂಗಳುಗಳೇ ಸಂದಿದ್ದುವು. ಹಲವಾರು ‘ಅಡಚಣೆ’ಗಳಿಂದಾಗಿ–ಅಥವಾ ದೈವನಿಯಾಮಕದಂತೆ ಎನ್ನಬಹು ದೇನೋ–ಅವರ ಹಾದಿ ನೇರವಾಗಿರದೆ ನೂರಾರು ತಿರುವುಗಳಿಂದ ಕೂಡಿತ್ತು. ಯಾವುದೋ ಕಾಣದ ಕೈ ಅವರನ್ನು ಊರಿಂದೂರಿಗೆ ಕರೆದೊಯ್ಯುತ್ತಿತ್ತು. ಪೂನಾದಿಂದ ಹೈದರಾಬಾದಿನ ಮೂಲಕ ರಾಮೇಶ್ವರಕ್ಕೆ ಹೋಗಬೇಕೆಂದು ನಿರ್ಧರಿಸಿದ್ದ ಸ್ವಾಮೀಜಿ ಯಾವುದೋ ಕಾರಣಗಳಿ ಗಾಗಿ ಖಾಂಡ್ವಾಗೆ, ಅಲ್ಲಿಂದ ಮುಂದೆ ಮುಂಬಯಿಗೆ ಹಾಗೂ ಇತರ ಹಲವಾರು ನಗರಗಳಿಗೆ ಭೇಟಿ ನೀಡಿದರು. ಆದರೆ ಸಾಧ್ಯವಾದಷ್ಟು ಬೇಗ ರಾಮೇಶ್ವರ ಸಂದರ್ಶನವನ್ನು ಮುಗಿಸಬೇಕೆಂಬ ಕಾತರ ಮನಸ್ಸಿನಲ್ಲಿದ್ದೇ ಇತ್ತು. ಆದ್ದರಿಂದ ಅವರು ಖಾಂಡ್ವಾದಿಂದ ಹೊರಟುನಿಂತರು. ಹರಿ ದಾಸ ಹಾಗೂ ಇತರರು ಅವರನ್ನು ಮತ್ತಷ್ಟು ದಿನ ಅಲ್ಲಿಯೇ ಉಳಿದುಕೊಳ್ಳುವಂತೆ ಒತ್ತಾಯಿಸಿ ದರು. ಅವರಪ್ರೀತಿ-ವಿಶ್ವಾಸಗಳಿಗಾಗಿ ಕೃತಜ್ಞತೆ ವ್ಯಕ್ತಪಡಿಸಿದ ಸ್ವಾಮೀಜಿ, ತಾವು ಕೈಗೊಂಡಿರುವ ಯಾತ್ರೆಯನ್ನು ಬೇಗ ಪೂರೈಸಬೇಕಾಗಿದೆಯೆಂದು ಹೇಳಿ ಹೊರಟರು.

ಮುಂಬಯಿಗೆ ಹೋಗಲು ಸ್ವಾಮೀಜಿಗೆ ರೈಲು ಟಿಕೆಟ್ಟನ್ನು ತೆಗೆಸಿಕೊಟ್ಟ ಹರಿದಾಸ್​ಬಾಬು, ಮುಂಬಯಿಯಲ್ಲಿರುವ ತನ್ನ ಸೋದರನಿಗೊಂದು ಪರಿಚಯ ಪತ್ರವನ್ನು ಬರೆದುಕೊಡುತ್ತ ಹೇಳಿದ, “ಸ್ವಾಮೀಜಿ, ಅವನು ನಿಮ್ಮನ್ನು ಮುಂಬಯಿಯ ಖ್ಯಾತ ಬ್ಯಾರಿಸ್ಟರ್ ಸೇಠ್ ರಾಮದಾಸ್ ಛಬಿಲ್​ದಾಸ್​ರಿಗೆ ಪರಿಚಯಿಸುತ್ತಾನೆ. ನೀವು ಅಮೆರಿಕೆಗೆ ಹೋಗುವ ವಿಚಾರದಲ್ಲಿ ಅವರಿಂದೇ ನಾದರೂ ಸಹಾಯವಾಗಬಹುದು. ಸ್ವಾಮೀಜಿ, ಖಂಡಿತವಾಗಿಯೂ ನಿಮಗೆ ಭವ್ಯ ಭವಿಷ್ಯವಿದೆ.” ಖಾಂಡ್ವಾದ ತಮ್ಮ ಅಭಿಮಾನಿಗಳು-ಅನುಯಾಯಿಗಳಿಂದ ಬೀಳ್ಕೊಂಡು ಸ್ವಾಮೀಜಿ ೧೮೯೨ರ ಜುಲೈ ಕೊನೆಯ ವಾರದಲ್ಲಿ ಮುಂಬಯಿಗೆ ಬಂದರು.

ಹರಿದಾಸ್ ಬಾಬುವಿನ ಸೋದರ ಅವರನ್ನು ಸೇಠ್ ರಾಮದಾಸ್ ಛಬಿಲ್ ದಾಸ್​ಗೆ ಪರಿಚಯ ಮಾಡಿಸಿದ. ಸೇಠ್​ಜಿ ಸ್ವಾಮೀಜಿಯನ್ನು ಆದರದಿಂದ ಸ್ವಾಗತಿಸಿ, ತನ್ನ ಅತಿಥಿಯಾಗಿರುವಂತೆ ಕೇಳಿಕೊಂಡ. ಮುಂಬಯಿಯಲ್ಲಿ ಅವರು ಎರಡು ತಿಂಗಳ ಕಾಲ ಇದ್ದರು. ಈ ಸೇಠ್ ಆರ್ಯ ಸಮಾಜದ ಕಟ್ಟಾ ಅನುಯಾಯಿಯಾಗಿದ್ದು, ಅನೇಕ ಹಿಂದೂ ಆಚಾರ-ವಿಚಾರಗಳನ್ನು ತೀವ್ರವಾಗಿ ವಿರೋಧಿಸುತ್ತಿದ್ದ. ಸ್ವಾಮೀಜಿಯೊಂದಿಗೆ ಈತ ಪ್ರತಿದಿನವೂ ಗಂಟೆಗಟ್ಟಲೆ ಚರ್ಚೆ ನಡೆಸುತ್ತಿದ್ದ. ಒಂದು ದಿನ ಈತ, ವೇದಗಳು ಭಗವಂತನ ಸಾಕಾರ ರೂಪವನ್ನು ಅನುಮೋದಿಸುತ್ತವೆ ಎಂಬು ದನ್ನು ರುಜುವಾತು ಪಡಿಸುವುದಾದರೆ ತಾನು ಆರ್ಯಸಮಾಜವನ್ನೇ ಬಿಟ್ಟುಬಿಡುವುದಾಗಿ ಘೋಷಿ ಸಿದ. ವಾದ-ವಿವಾದಗಳು ಹಲವಾರು ದಿನಗಳವರೆಗೆ ನಡೆದು, ಕಡೆಗೆ ಸ್ವಾಮೀಜಿ ತಮ್ಮ ವಾದದ ಸತ್ಯತೆಯನ್ನು ಅವನಿಗೆ ಮನಗಾಣಿಸುವುದರಲ್ಲಿ ಯಶಸ್ವಿಯಾದರು. ತನ್ನ ಮಾತಿನಂತೆ ಆತ ಆರ್ಯ ಸಮಾಜವನ್ನು ಬಿಟ್ಟುಬಿಟ್ಟ.

ಮುಂಬಯಿಯಲ್ಲಿ ಸ್ವಾಮೀಜಿ ಸಾರ್ವಜನಿಕ ಭಾಷಣಗಳನ್ನೇನೂ ಮಾಡಲಿಲ್ಲ. ಆದರೆ ಅವರು ಅಲ್ಲಿ ಹಲವಾರು ಸಂಪ್ರದಾಯಸ್ಥ ಹಾಗೂ ‘ಪ್ರಗತಿಪರ’ ಪಂಡಿತರ ಸಂಪರ್ಕಕ್ಕೆ ಬಂದರು. ಸಂಪ್ರ ದಾಯಸ್ಥರು ಕೂಪಮಂಡೂಕಗಳಂತೆ ವಾದ ಹೂಡಿ, ಅಜ್ಜ ಹಾಕಿದ ಆಲದಮರಕ್ಕೆ ಸಮಾಜವನ್ನು ತೂಗುಹಾಕಲು ಪ್ರಯತ್ನಿಸುತ್ತಿದ್ದರೆ, ಪ್ರಗತಿಪರರೆನ್ನಿಸಿಕೊಂಡವರು ಪಾಶ್ಚಾತ್ಯ ಸಂಸ್ಕೃತಿಯ ಕಣ್ಣು ಕೋರೈಸುವ ಬೆಳಕಿನತ್ತ ಪತಂಗಗಳಂತೆ ನುಗ್ಗುತ್ತಿರುವುದನ್ನು ಕಂಡರು ಸ್ವಾಮೀಜಿ. ಆ ದಿನಗಳಲ್ಲಿ ಬಾಲ್ಯ ವಿವಾಹವನ್ನು ನಿಷೇಧಿಸುವ ಮಸೂದೆ \eng{(Age of Consent Bill–}ಒಪ್ಪಿಗೆಯ ವಯಸ್ಸಿನ ಮಸೂದೆ)ಯೊಂದು ಸದನದಲ್ಲಿ ಚರ್ಚಿಸಲ್ಪಡುತ್ತಿತ್ತು. ಹೆಣ್ಣುಮಕ್ಕಳ ಮದುವೆಯ ಕನಿಷ್ಟ ವಯೋಮಿತಿಯನ್ನು ೧೦ ವರ್ಷದಿಂದ ೧೨ ವರ್ಷಕ್ಕೆ ಏರಿಸಬೇಕೆಂಬ ಸಲಹೆಯ ಮಸೂದೆ ಅದು. ದೇಶದಾದ್ಯಂತ ಇದು ತೀವ್ರ ವಿವಾದವನ್ನೆಬ್ಬಿಸಿತ್ತು. ಸ್ವಾಮೀಜಿ ಇದಕ್ಕೆ ಸಂಪೂರ್ಣ ಪರವಾಗಿ ದ್ದರು. ಮುಂಬಯಿಯ ಖ್ಯಾತ ರಾಜಕಾರಣಿಯೊಬ್ಬರ ಮನೆಯಲ್ಲಿ ಅವರಿಗೆ ಒಬ್ಬರು ಕಲ್ಕತ್ತದ ಪತ್ರಿಕೆಯೊಂದನ್ನು ತೋರಿಸಿದರು. ಕಲ್ಕತ್ತದ ವಿದ್ಯಾವಂತ ವರ್ಗದಲ್ಲೂ ಬಹುತೇಕ ಜನ ಈ ಮಸೂದೆಯನ್ನು ವಿರೋಧಿಸುತ್ತಿರುವುದಾಗಿ ಅದರಲ್ಲಿ ವರದಿಯಾಗಿತ್ತು. ಇದನ್ನು ಕಂಡು ಸ್ವಾಮೀಜಿ ನಾಚಿಕೆಯಿಂದ ತಲೆತಗ್ಗಿಸುವಂತಾಯಿತು.

ಈ ದಿನಗಳಲ್ಲೇ ಒಮ್ಮೆ ಅವರು ಕನ್ಹೇರಿ ಗುಹೆಗಳನ್ನು ಸಂದರ್ಶಿಸಿದರು. ಇವು ಮುಂಬಯಿ ಇಂದ ಮೂವತ್ತು ಮೈಲಿ ದೂರದಲ್ಲಿರುವ ಸಾಲ್ಸೆತ್ತೆ ಎಂಬ ದ್ವೀಪದಲ್ಲಿವೆ. ಪುರಾತನ ಬೌದ್ಧಕಾಲ ದಲ್ಲಿ ಬೌದ್ಧಸಂನ್ಯಾಸಿಗಳು ಇಲ್ಲಿ ವಾಸವಾಗಿದ್ದರು. ಬುದ್ಧಘೋಷನೆಂಬ ಪ್ರಸಿದ್ಧ ಬೌದ್ಧ ಸಂತನೂ ತನ್ನ ಶಿಷ್ಯರೊಂದಿಗೆ ಇಲ್ಲಿ ವಾಸವಾಗಿದ್ದನೆಂದು ಹೇಳಲಾಗುತ್ತದೆ. ಸ್ವಾಮೀಜಿ ಈ ಸ್ಥಳ ದಿಂದ ತೀವ್ರವಾಗಿ ಆಕರ್ಷಿತರಾದರು. ತಾವು ಹಿಂದಿನ ಯಾವುದೋ ಒಂದು ಜನ್ಮದಲ್ಲಿ ಈ ಗುಹೆಯಲ್ಲಿ ವಾಸವಾಗಿದ್ದುದಾಗಿ ಸ್ವಾಮಿ ವಿವೇಕಾನಂದರಿಗೆ ಭಾಸವಾಯಿತೆಂದು ಅವರ ಶಿಷ್ಯರಾದ ಸ್ವಾಮಿ ಸದಾನಂದರು ಹೇಳುತ್ತಾರೆ. ಅಲ್ಲದೆ, ಮುಂದೆ ಈ ಸ್ಥಳವನ್ನು ಪಡೆದುಕೊಂಡು ಅದನ್ನು ತಾವು ಯೋಜಿಸಿದ್ದ ಕಾರ್ಯಸ್ಥಾನಗಳಲ್ಲೊಂದಾಗಿ ಮಾಡಬೇಕೆಂದು ಅವರಿಗೆ ಆಸೆಯಿತ್ತು ಎಂದೂ ಅವರು ಹೇಳುತ್ತಾರೆ.

ಮುಂಬಯಿಯಲ್ಲಿ ಸ್ವಾಮೀಜಿಗೆ ಕೆಲವು ಅಪೂರ್ವ ಗ್ರಂಥಗಳು ಲಭ್ಯವಾದುವೆಂದು ಅವರು ದಿವಾನ್​ಜಿಗೆ ಬರೆದ ಪತ್ರದಿಂದ ತಿಳಿದುಬರುತ್ತದೆ. ಅವುಗಳನ್ನೆಲ್ಲ ಓದಿ ಮುಗಿಸುವ ಉದ್ದೇಶ ದಿಂದ ಅಲ್ಲಿ ಬಹಳ ದಿನ ಉಳಿದುಕೊಂಡರು. ಆದರೆ ಅಮೆರಿಕೆಗೆ ಹೋಗುವ ವಿಚಾರದಲ್ಲಿ ಸೇಠ್​ಜಿಯಿಂದ ಯಾವ ನೆರವೂ ಭರವಸೆಯೂ ಸಿಗಲಿಲ್ಲ. (ಮರುವರ್ಷ ಸ್ವಾಮೀಜಿ ಯೋಕೋ ಹಾಮದಿಂದ ಶಿಕಾಗೋಗೆ ಪ್ರಯಾಣ ಮಾಡುವಾಗ ಈತ ಅವರ ಸಹಪ್ರಯಾಣಿಕನಾಗಿದ್ದ) ಆದರೆ ಇಲ್ಲಿಗೆ ಬಂದದ್ದರಿಂದ ಅವರಿಗೆ ಇತರ ಪ್ರಯೋಜನಗಳಾಗಿದ್ದುವು. ಹೀಗೆ ಮುಂಬಯಿಯಲ್ಲಿ ಸುಮಾರು ಎರಡು ತಿಂಗಳ ಕಾಲ ಇದ್ದು, ಪೂನಾದ ಮಾರ್ಗವಾಗಿ ದಕ್ಷಿಣ ಭಾರತದ ಕಡೆಗೆ ಹೊರಟರು ಸ್ವಾಮೀಜಿ.

ಅವರು ಪೂನಾಗೆ ಹೊರಟಿದ್ದ ಟ್ರೈನಿನಲ್ಲೇ, ಮುಂದೆ ಲೋಕಮಾನ್ಯರೆಂದು ಹೆಸರು ಗಳಿಸಿದ ಬಾಲಗಂಗಾಧರ ತಿಲಕರೂ ಹೊರಟಿದ್ದರು. ತಿಲಕರು ಕುಳಿತಿದ್ದ ಬೋಗಿಯೊಳಗೇ ಸ್ವಾಮೀಜಿ ಪ್ರವೇಶಿಸಿದರು. ತಿಲಕರಿಗೆ ಸ್ವಾಮೀಜಿಯ ಪರಿಚಯವಿರಲಿಲ್ಲ. ಆದರೆ ಸ್ವಾಮೀಜಿಯನ್ನು ಗುರುತಿ ಸಿದ ತಿಲಕರ ಸ್ನೇಹಿತರೊಬ್ಬರು ತಿಲಕರಿಗೆ ಅವರ ಪರಿಚಯ ಮಾಡಿಕೊಟ್ಟರು. ಬಳಿಕ ಆ ಸ್ನೇಹಿತರು ಸ್ವಾಮೀಜಿಯನ್ನು ತಿಲಕರ ಮನೆಯಲ್ಲೇ ಉಳಿದುಕೊಳ್ಳುವಂತೆ ಕೇಳಿಕೊಂಡರು. ಇದಕ್ಕೆ ಸ್ವಾಮೀಜಿ ಒಪ್ಪಿದರು.

ಬಾಲಗಂಗಾಧರ ತಿಲಕರು ‘ಮರಾಠಾ’ಹಾಗೂ ‘ಕೇಸರಿ’ ಎಂಬ ಸುಪ್ರಸಿದ್ಧ ಪತ್ರಿಕೆಗಳ ಸಂಪಾದಕರಾಗಿದ್ದರು. ನಿರ್ಭೀತ ಪತ್ರಿಕೋದ್ಯಮಿಯೆಂದು, ಮಹಾವಿದ್ವಾಂಸರೆಂದು ಹಾಗೂ ರಾಷ್ಟ್ರೀಯ ಧುರೀಣರೆಂದು ಖ್ಯಾತರಾಗಿದ್ದರು. ಆದರೆ \eng{Age of Consent} ಮಸೂದೆಯನ್ನು ಅವರು ತೀವ್ರವಾಗಿ ವಿರೋಧಿಸಿದ್ದರು. ತಮ್ಮ ಆತಿಥೇಯರು ಯಾರೆಂದು ಸ್ವಾಮೀಜಿಗೆ ತಿಳಿದಾಗ ಅವರಲ್ಲುಂಟಾದ ಭಾವನೆಗಳನ್ನು ನಾವು ಊಹಿಸಬಹುದು. ತಿಲಕರು ಸ್ವಾಮೀಜಿಯ ಹೆಸರನ್ನು ಕೇಳಿದಾಗ ಅವರು “ನಾನೊಬ್ಬ ಸಂನ್ಯಾಸಿ” ಎಂದಷ್ಟೇ ಹೇಳಿದರು. ತಿಲಕರು ಅವರನ್ನು ಮತ್ತೆ ಪ್ರಶ್ನಿಸಲಿಲ್ಲ.

ಪೂನಾದಲ್ಲಿ ಸ್ವಾಮೀಜಿ ತಿಲಕರ ಅತಿಥಿಗಳಾಗಿ ೮-೧೦ ದಿನ ಇದ್ದರು. ಕೆಲವೊಮ್ಮೆ ಅವರಿಬ್ಬರು ಅದ್ವೈತ, ಭಗವದ್ಗೀತೆ ಹಾಗೂ ವೇದಾಂತದ ಬಗ್ಗೆ ಚರ್ಚಿಸುತ್ತಿದ್ದರು. ಜನರೊಂದಿಗೆ ಬೆರೆಯು ವುದರಿಂದ ಸ್ವಾಮೀಜಿ ಸಾಧ್ಯವಾದಷ್ಟೂ ತಪ್ಪಿಸಿಕೊಳ್ಳುತ್ತಿದ್ದರು. ಆದರೆ ಒಮ್ಮೆ ತಿಲಕರೊಂದಿಗೆ ಅವರು ಪೂನಾದ ಡೆಕ್ಕನ್ ಕ್ಲಬ್ ಹೋಗಿದ್ದರು. ಅಲ್ಲಿ ಶ್ರೀ ಕಾಶಿನಾಥ ಎಂಬವರು ಯಾವುದೋ ತಾತ್ವಿಕ ವಿಷಯದ ಬಗ್ಗೆ ತುಂಬ ಚೆನ್ನಾಗಿ ಭಾಷಣ ಮಾಡಿದರು. ಅದರ ಬಗ್ಗೆ ಯಾರೂ ಚಕಾರವೆತ್ತುವಂತಿರಲಿಲ್ಲ. ಆದರೆ ಸ್ವಾಮೀಜಿ ಎದ್ದುನಿಂತು ಆ ಉಪನ್ಯಾಸಕರು ಪ್ರಸ್ತಾಪಿಸದಿದ್ದ ವಿಷಯದ ಮತ್ತೊಂದು ಮುಖವನ್ನು ಸೊಗಸಾದ ಇಂಗ್ಲಿಷಿನಲ್ಲಿ ಸುಸ್ಪಷ್ಟವಾದ ಸರಣಿಯಲ್ಲಿ ಮಂಡಿಸುತ್ತ ಮಾತನಾಡಿದರು. ಸಭಿಕರೆಲ್ಲ ಈ ಅಪರಿಚಿತ ಸಂನ್ಯಾಸಿ ಮಾತನಾಡಿದ ಕ್ರಮವನ್ನು ಕಂಡು ಬೆರಗಾದರು. ಅಂದಿನಿಂದ ಅವರನ್ನು ನೋಡಲು ಜನರು ಬರಲಾರಂಭಿಸಿದರು. ಅವರೊಂದಿಗೆ ಸ್ವಾಮೀಜಿ ಗೀತೆ, ಉಪನಿಷತ್ತುಗಳ ಕುರಿತಾಗಿ ಮಾತನಾಡುತ್ತಿದ್ದರು. ಆದರೆ ಸ್ವಾಮೀಜಿ ಅನಾಮಿಕರಾಗಿಯೇ ಉಳಿಯಲು ನಿಶ್ಚಯಿಸಿ, ಯಾರಿಗೂ ತಮ್ಮ ಹೆಸರನ್ನೇ ಹೇಳಲಿಲ್ಲ. ತಿಲಕರಿಗೂ ಸ್ವಾಮೀಜಿಯ ಹೆಸರಾಗಲಿ, ಅವರ ಪೂರ್ವಾಪರಗಳಾಗಲಿ ತಿಳಿಯಲೇ ಇಲ್ಲ. ಆದರೆ ಅವರಿಬ್ಬರೂ ‘ಒಪ್ಪಿಗೆಯ ವಯಸ್ಸಿನ ಕಾನೂನಿ’ನ ಬಗ್ಗೆ ಸಾಕಷ್ಟು ಚರ್ಚಿಸಿರಬೇಕೆಂದು ನಾವು ಊಹಿಸಬಹುದಾಗಿದೆ. ಇದಲ್ಲದೆ ಗೀತಾ ಸಂದೇಶದ ಬಗ್ಗೆ–ಮುಖ್ಯವಾಗಿ ಕರ್ಮಫಲತ್ಯಾಗದ ವಿಷಯವಾಗಿ–ತಾವಿಬ್ಬರೂ ಮಾತನಾಡಿದುದಾಗಿ ತಿಲಕರೇ ಹೇಳುತ್ತಾರೆ. ಇಷ್ಟೆಲ್ಲ ಆದರೂ ತಿಲಕರಿಗೆ ಸ್ವಾಮೀಜಿಯ ಆತ್ಮೀಯ ಪರಿಚಯವಾಗಲಿಲ್ಲ. ತಮ್ಮ ದರ್ಶನಾರ್ಥಿಗಳಾಗಿ ಬರುತ್ತಿದ್ದ ವರ ಸಂಖ್ಯೆ ತುಂಬ ಹೆಚ್ಚಾದಾಗ, ತಾವು ಮರುದಿನ ಹೊರಡುವುದಾಗಿ ಹೇಳಿದ ಸ್ವಾಮೀಜಿ, ಮನೆಯವರು ಏಳುವ ಮೊದಲೇ ಹೊರಟುಬಿಟ್ಟರು.

ಮುಂದೆ ಸ್ವಾಮೀಜಿ ಶಿಕಾಗೋದ ವಿಶ್ವಧರ್ಮ ಸಮ್ಮೇಳನದ ಮೂಲಕ ಸ್ವಾಮಿ ವಿವೇಕಾನಂದ ರೆಂಬ ಹೆಸರಿನಿಂದ ಜಗತ್ಪ್ರಸಿದ್ಧರಾದಾಗ, ಪತ್ರಿಕಾ ವರದಿಗಳನ್ನು ಕಂಡ ತಿಲಕರು, ಹಿಂದೊಮ್ಮೆ ತಮ್ಮ ಅತಿಥಿಯಾಗಿದ್ದ ಆ ಸಂನ್ಯಾಸಿಯೇ ಇವರಾಗಿರಬಹುದೇ ಎಂದು ಯೋಚಿಸಿದರು. ಸ್ವಾಮೀಜಿ ಭಾರತಕ್ಕೆ ಹಿಂದಿರುಗಿದ ಮೇಲೆ ಅವರ ಭಾವಚಿತ್ರಗಳನ್ನು ಕಂಡಾಗ ತಿಲಕರ ಅನುಮಾನ ಮತ್ತಷ್ಟು ಬಲವಾಯಿತು. ಆಗ ಅವರು ಸ್ವಾಮೀಜಿಗೆ ಪತ್ರವೊಂದನ್ನು ಬರೆದು, ತಮ್ಮ ಊಹೆ ಸರಿಯೋ ತಪ್ಪೋ ಎಂಬುದನ್ನು ತಿಳಿಸಬೇಕೆಂದು ವಿನಂತಿಸಿಕೊಂಡರು. ಅಲ್ಲದೆ ಮತ್ತೊಮ್ಮೆ ಪೂನಾಗೆ ಭೇಟಿ ನೀಡಬೇಕೆಂದೂ ಬೇಡಿಕೊಂಡರು. ತಕ್ಷಣ ಸ್ವಾಮೀಜಿ ತಿಲಕರ ಪತ್ರಕ್ಕೆ ಉತ್ತರಿಸಿ, ಹಿಂದೆ ಅವರ ಅತಿಥಿಯಾಗಿದ್ದವರು ತಾವೇ ಎಂಬುದನ್ನು ದೃಢಪಡಿಸಿದರು. ಆದರೆ ಈ ಸಂದರ್ಭದಲ್ಲಿ ತಾವು ಬರಲು ಸಾಧ್ಯವಿಲ್ಲವೆಂದು ತಿಳಿಸಿ ಅವರ ಕ್ಷಮೆಕೋರಿದರು.

೧೯೦೧ರಲ್ಲಿ ಕಲ್ಕತ್ತದಲ್ಲಿ ನಡೆದ ಭಾರತ ರಾಷ್ಟ್ರೀಯ ಕಾಂಗ್ರೆಸ್​ನ ಅಧಿವೇಶನದಲ್ಲಿ ಭಾಗವಹಿ ಸಲು ಬಂದಿದ್ದ ಬಾಲಗಂಗಾಧರ ತಿಲಕರು ಬೇಲೂರಿನ ಶ್ರೀರಾಮಕೃಷ್ಣ ಮಠವನ್ನು ಸಂದರ್ಶಿಸಿ ದರು. ಆಗ ಅವರನ್ನು ಸ್ವಾಮೀಜಿ ಹಾರ್ದಿಕವಾಗಿ ಸ್ವಾಗತಿಸಿದರು. ತಿಲಕರು ತಮ್ಮ ಸ್ನೇಹಿತರ ಜೊತೆಯಲ್ಲಿ ಚಹಾ ಸ್ವೀಕರಿಸಿದರು. ಮಾತಿನ ಸಂದರ್ಭದಲ್ಲಿ ಸ್ವಾಮೀಜಿ ತಿಲಕರಿಗೆ ಹಾಸ್ಯವಾಗಿ ಹೇಳುತ್ತಾರೆ, “ನೀವು ಸಂನ್ಯಾಸ ಸ್ವೀಕರಿಸಿ ಬಂಗಾಳಕ್ಕೆ ಬಂದು ಇಲ್ಲಿ ನನ್ನ ಕೆಲಸವನ್ನು ನಿರ್ವಹಿಸಿ; ನಾನು ಮಹಾರಾಷ್ಟ್ರಕ್ಕೆ ಹೋಗಿ ನಿಮ್ಮ ಕಾರ್ಯವನ್ನು ಮುಂದುವರಿಸುತ್ತೇನೆ... ಒಬ್ಬ ವ್ಯಕ್ತಿಯ ಪ್ರಭಾವ ಅವನ ಸ್ವಸ್ಥಳಕ್ಕಿಂತ ದೂರದ ಪ್ರಾಂತ್ಯದಲ್ಲೇ ಹೆಚ್ಚು!” ಬಳಿಕ ಅವರಿಬ್ಬರೂ ವೈಯಕ್ತಿಕ ವಾಗಿ ಬಹಳ ಹೊತ್ತು ಮಾತನಾಡಿದರೆಂದೂ ತಿಳಿದುಬರುತ್ತದೆ.

ಪೂನಾದಿಂದ ಸ್ವಾಮೀಜಿ ಕೊಲ್ಲಾಪುರಕ್ಕೆ ಬಂದರು. ಹಿಂದೆ ಅವರು ಲಿಂಬ್ಡಿಯಿಂದ ಹೊರಟು ಭಾವನಗರಕ್ಕೆ ಬಂದಿದ್ದಾಗ ಅಲ್ಲಿನ ಮಹಾರಾಜ ಅವರಿಗೆ ಕೊಲ್ಲಾಪುರದ ರಾಜನಿಗೆ ಉದ್ದೇಶಿಸಿದ ಪರಿಚಯ ಪತ್ರವನ್ನು ಕೊಟ್ಟಿದ್ದ. ಅರಮನೆಯಲ್ಲಿ ಸ್ವಾಮೀಜಿಗೆ ಆದರದ ಸ್ವಾಗತ ದೊರಕಿತು. ಅವರಿಂದ ತುಂಬ ಪ್ರಭಾವಿತಳಾದ ಮಹಾರಾಣಿ ಅವರಿಗೆ ಏನಾದರೊಂದು ಕಾಣಿಕೆಯನ್ನು ಅರ್ಪಿಸಲು ಮುಂದಾದಳು. ಆದರೆ ಸ್ವಾಮೀಜಿ ಏನನ್ನೂ ಸ್ವೀಕರಿಸಲು ಒಪ್ಪದಿದ್ದಾಗ ಬಲವಂತ ದಿಂದ ಅವರಿಗೊಂದು ಹೊಸ ಕಾವಿ ವಸ್ತ್ರವನ್ನು ಅರ್ಪಿಸಿದಳು.

ಖಾಸ್​ಬಾಗ್ ಎಂಬಲ್ಲಿ ಅವರಿಗೆ ಇಳಿದುಕೊಳ್ಳಲು ವ್ಯವಸ್ಥೆ ಮಾಡಲಾಗಿತ್ತು. ಅದ್ಭುತ ಸಂನ್ಯಾಸಿಗಳೊಬ್ಬರು ಆಗಮಿಸಿರುವ ಸುದ್ದಿ ಊರಿನಲ್ಲೆಲ್ಲ ಹರಡಿ, ಜನ ಬರಲಾರಂಭಿಸಿದರು. ಸ್ವಾಮೀಜಿಯನ್ನು ಭೇಟಿ ಮಾಡಿದ ಶ್ರೀ ವಿಜಾಪುರಕರ್ ಎಂಬವರು ಅವರ ಕುರಿತಾದ ಸ್ಮೃತಿ ಲೇಖನದಲ್ಲಿ ಹೀಗೆ ಬರೆಯುತ್ತಾರೆ:

“ಒಂದು ಸಂಜೆ ನಾವು ಅವರನ್ನು ನೋಡಲು ಹೋದೆವು. ಅಲ್ಲಿದ್ದವರೊಂದಿಗೆ ಮಾತನಾಡು ತ್ತಿದ್ದ ಸ್ವಾಮೀಜಿಯ ಧ್ವನಿ ದೂರದಿಂದ ಕೇಳಿಸಿತು. ಆ ಧ್ವನಿ ಎಷ್ಟು ಮಧುರ-ಗಂಭೀರ ಹಾಗೂ ಸುಸ್ಪಷ್ಟವಾಗಿತ್ತೆಂದರೆ, ಅದನ್ನು ಕೇಳಿದೊಡನೆಯೇ ಅವರೆಂತಹ ಅದ್ಭುತ ವ್ಯಕ್ತಿ ಎಂಬುದು ಮನವರಿಕೆಯಾಗುವಂತಿತ್ತು. ಕೇಳುಗರ ಪ್ರಶ್ನೆಗಳಿಗೆ ಸ್ವಾಮೀಜಿ ಅಡೆತಡೆಯಿಲ್ಲದೆ ಉತ್ತರಿಸುತ್ತಿ ದ್ದರು. ಮರುದಿನ ನಮ್ಮ ಸಂಘಕ್ಕೆ ಭೇಟಿಕೊಡುವಂತೆ ಅವರನ್ನು ಆಹ್ವಾನಿಸಿದೆವು. ಅಲ್ಲಿಯೂ ಅವರು ಅತ್ಯಂತ ನಿರರ್ಗಳವಾಗಿ, ಸೊಗಸಾಗಿ ಮಾತನಾಡಿದರು. ಮಾತಿನ ಸಂದರ್ಭದಲ್ಲಿ ಅವರು ಉದ್ಗರಿಸಿದರು. ‘ಬೌದ್ಧಧರ್ಮ ಎನ್ನುವುದು ಯಾವುದರ ಬಂಡುಗಾರ ಶಿಶುವೋ, ಕ್ರೈಸ್ತ ಧರ್ಮವು ಯಾವುದರ ಅಸ್ಪಷ್ಟ ಅನುಕರಣೆ ಮಾತ್ರವೋ, ಅಂತಹ ಹಿಂದೂಧರ್ಮ ನನ್ನದು! ಭೋಗಗಳ ಬೆನ್ನಟ್ಟಿ ಓಡುತ್ತಿರುವ ಪಾಶ್ಚಾತ್ಯರು ಧರ್ಮದ ಮರ್ಮವನ್ನು ಹೇಗೆ ಗ್ರಹಿಸಿಯಾರು?’...

“ಹೀಗೆ ಅವರು ಸುಮಾರು ಒಂದೂವರೆ ಗಂಟೆಗಳ ಕಾಲ ಸುಲಲಿತ ಶೈಲಿಯಲ್ಲಿ ಮಾತನಾಡಿ ದರು. ಮುಂದೆ ಅವರೊಬ್ಬ ಮಹಾವಾಗ್ಮಿಯಾಗಿ ಪ್ರಸಿದ್ಧರಾಗಬಲ್ಲರೆಂದು ನಾವೆಲ್ಲ ಊಹಿಸಿದೆವು...”

ಕೊಲ್ಲಾಪುರದಲ್ಲಿ ಕೆಲದಿನಗಳನ್ನು ಕಳೆದು, ದಕ್ಷಿಣ ಭಾರತದತ್ತ ಹೊರಟ ಸ್ವಾಮೀಜಿ, ೧೮೯೨ರ ಅಕ್ಟೋಬರ್ ೧೫ರಂದು ಬೆಳಗಾವಿಯನ್ನು ತಲುಪಿದರು. ಕೊಲ್ಲಾಪುರದ ರಾಜನ ಆಪ್ತಕಾರ್ಯ ದರ್ಶಿಯಿಂದ ಅವರು ಇಲ್ಲಿನ ಶ್ರೀಯುತ ಭಾಟೆ ಎಂಬವರಿಗೆ ಪರಿಚಯ ಪತ್ರವನ್ನು ತಂದಿದ್ದರು. ಇವರು ಸಂಪ್ರದಾಯಸ್ಥ ಮರಾಠಿಗರು. ಮೊದಲ ನೋಟಕ್ಕೇ ಇತರ ಸಂನ್ಯಾಸಿಗಳಿಗಿಂತ ತುಂಬ ಭಿನ್ನವಾಗಿ ಕಂಡುಬಂದ ಸ್ವಾಮೀಜಿಯನ್ನು ನೋಡಿ ಮನೆಯವರು ಅಚ್ಚರಿಗೊಂಡರು. ಅವರೊಂದಿಗೆ ಸ್ವಲ್ಪ ಹೊತ್ತು ಮಾತನಾಡುತ್ತಿದ್ದಂತೆಯೇ, ಅವರ ವಿಶಾಲ ಜ್ಞಾನ, ಅಪಾರ ಬುದ್ಧಿಮತ್ತೆಯನ್ನು ಮನಗಂಡ ಮನೆಯವರು ಮತ್ತಷ್ಟು ಆಶ್ಚರ್ಯಗೊಂಡರು.

ಸ್ವಾಮೀಜಿ ಇವರ ಮನೆಗೆ ಬಂದಾಗ ಭಾಟೆಯವರ ಮಗ ಜಿ. ಎಸ್. ಭಾಟೆ, ಸಂಸ್ಕೃತ ವ್ಯಾಕರಣ ಗ್ರಂಥ ‘ಅಷ್ಟಾಧ್ಯಾಯಿ’ಯನ್ನು ಅಧ್ಯಯನ ಮಾಡುತ್ತ ಕುಳಿತಿದ್ದ. (ಮುಂದೆ ಇವನೇ ಸ್ವಾಮೀಜಿಯ ಕುರಿತಾದ ಸ್ಮೃತಿಲೇಖನವನ್ನು ಬರೆಯುತ್ತಾನೆ.) ಅವನ ಪಾಲಿಗೆ ಅದೊಂದು ಕಬ್ಬಿಣದ ಕಡಲೆಯಾಗಿತ್ತು. ಅದರ ಕೆಲವು ಭಾಗಗಳನ್ನು ನೆನಪಿನಿಂದ ಹೇಳಲು ಹೊರಟಾಗ ಮತ್ತೆ ಮತ್ತೆ ತಪ್ಪುತ್ತಿದ್ದ. ಆಗ ಅಲ್ಲಿದ್ದ ಸ್ವಾಮೀಜಿ ನಸುನಗುತ್ತ ಅವನನ್ನು ತಿದ್ದಿದರು. ಆ ಕಷ್ಟದ ಭಾಗಗಳನ್ನೂ ಅವರು ಬಾಯಿಪಾಠ ಮಾಡಿದಂತೆ ಒಪ್ಪಿಸಿದಾಗ ಹುಡುಗನ ಆಶ್ಚರ್ಯಕ್ಕೆ ಪಾರವೇ ಇಲ್ಲ. ಆದರೆ ಅವನಿಗೆ ತುಂಬ ನಾಚಿಕೆಯೂ ಆಯಿತು. ಬಳಿಕ ಅವನ ತಂದೆ ಅಮರಕೋಶವನ್ನು ಹೇಳುವಂತೆ ಆಜ್ಞಾಪಿಸಿದರು. ಅವನಿಗೆ ಅದು ಬರುತ್ತಲೂ ಇತ್ತು. ಆದರೆ ತಾನು ಅದನ್ನು ಹೇಳು ವಾಗ ಏನಾದರೂ ತಪ್ಪಾಗಿಬಿಟ್ಟರೆ? ಅದನ್ನು ಈ ಸ್ವಾಮಿಗಳು ತಿದ್ದುವಂತಾದರೆ?! ಎಂದು ಆಲೋಚಿಸಿ, ಹುಡುಗ ತನಗೆ ಅದು ಬರುವುದೇ ಇಲ್ಲ ಎಂದುಬಿಟ್ಟ! ಆಗ ಅವನ ತಂದೆ, ಅವನು ನಿರೀಕ್ಷಿಸಿದ್ದಂತೆಯೇ, ಚೆನ್ನಾಗಿ ಬೈದು ಛೀಮಾರಿ ಹಾಕಿದರು. ಅಪ್ಪನ ಕೈಯಲ್ಲಿ ಬೈಸಿಕೊಂಡರೂ ಪರವಾಗಿಲ್ಲ ಎಂದು ಆತ ತನಗದು ಬರುವುದಿಲ್ಲವೆಂದೇ ಸಾಧಿಸಿಬಿಟ್ಟ.

ಸ್ವಾಮೀಜಿಯ ಸ್ವಭಾವವನ್ನು ಅರ್ಥಮಾಡಿಕೊಳ್ಳಲು ಮನೆಯವರಿಗೆ ಮೊದಮೊದಲು ಬಹಳವೇ ಕಷ್ಟವಾಯಿತು. ಆದರೆ ಒಂದೆರಡು ದಿನ ಕಳೆಯುವಷ್ಟರಲ್ಲಿ ಅವರಿಗೆ ಸ್ವಾಮೀಜಿಯ ವ್ಯಕ್ತಿತ್ವದ ಪರಿಚಯವಾಯಿತು. ಕೆಲವೇ ದಿನಗಳಲ್ಲಿ ಅವರೊಡನೆ ಸಂಭಾಷಿಸಲು ಅಲ್ಲಿನ ಪಂಡಿತರೂ ಜನನಾಯಕರೂ ಸೇರಿದಂತೆ ಹಲವಾರು ಜನ ಬರಲಾರಂಭಿಸಿದರು. ಎಲ್ಲ ರೊಂದಿಗೂ ನಿರ್ಭಿಡೆಯಿಂದ ಮಾತನಾಡುತ್ತಿದ್ದ ಸ್ವಾಮೀಜಿಯಲ್ಲಿ ಎದ್ದುಕಾಣುತ್ತಿದ್ದ ಒಂದು ಪ್ರಮುಖ ಅಂಶವೆಂದರೆ ಅವರ ಹಾಸ್ಯಪ್ರಜ್ಞೆ. ವಾದ-ವಿವಾದಗಳು ತಾರಕಕ್ಕೇರಿದಾಗಲೂ ಅವರು ಮಾತ್ರ ತಮ್ಮ ಹಾಸ್ಯ ಪ್ರಿಯತೆಯ ಸ್ವಭಾವವನ್ನು ಕಳೆದುಕೊಳ್ಳುತ್ತಿರಲಿಲ್ಲ. ಕೊಂಕಿನ ಪ್ರಶ್ನೆಗಳಿಗೆ ಅವರ ಉತ್ತರ ಸದಾ ಸಿದ್ಧವಾಗಿರುತ್ತಿತ್ತು. ಆದರೆ ಅದರಲ್ಲಿ ವಿಷವಿರುತ್ತಿರಲಿಲ್ಲ.

ಒಂದು ದಿನ ಬೆಳಗಾವಿಯ ಎಗ್ಸಿಕ್ಯುಟಿವ್ ಎಂಜಿನಿಯರ್ ಅವರನ್ನು ನೋಡಲು ಬಂದ. ಆ ಊರಿನಲ್ಲೇ ಅತಿ ಹೆಚ್ಚು ತಿಳಿದವನೆಂಬ ಹೆಸರು ಗಳಿಸಿದ್ದವನು ಈತ. ಹೊರನೋಟಕ್ಕೆ ನಿಷ್ಠಾವಂತ ~—ಸಂಪ್ರದಾಯಸ್ಥ ಹಿಂದುವಾಗಿದ್ದ ಈತ, ಅಂತರಂಗದಲ್ಲಿ ಸಂದೇಹವಾದಿ. ಆಧುನಿಕ ವಿಜ್ಞಾನದ ಕಡೆಗೇ ಇವನ ಒಲವು. ಇವನು ಸ್ವಾಮೀಜಿಯ ಮುಂದೆ, ಧರ್ಮದ ಕುರಿತಾದ ತನ್ನ ಅನಿಸಿಕೆಯನ್ನು ಮಂಡಿಸಿದ: “ಸ್ವಾಮೀಜಿ, ನನಗನ್ನಿಸುತ್ತದೆ, ಈ ಧರ್ಮ, ಧಾರ್ಮಿಕ ಆಚರಣೆಗಳು–ಇವುಗಳ ಲ್ಲೆಲ್ಲ ಏನೂ ತಿರುಳಿಲ್ಲ ಎಂದು. ಇವು ಹೀಗೇ ಎಷ್ಟೋ ಕಾಲದಿಂದ ನಡೆದುಕೊಂಡು ಬಂದು ಸಂಪ್ರದಾಯ ಅಂತ ಆಗಿಬಿಟ್ಟಿವೆ ಅಷ್ಟೆ.” ಧರ್ಮದ ಕುರಿತಾದ ಇಂತಹ ಪೊಳ್ಳು ವ್ಯಾಖ್ಯೆಯನ್ನು ಸ್ವಾಮೀಜಿ ಪುರಸ್ಕರಿಸಿಯಾರೆ? ವಯಸ್ಸೊಂದನ್ನು ಬಿಟ್ಟು ಉಳಿದ ಎಲ್ಲದರಲ್ಲೂ–ಧರ್ಮ, ವಿಜ್ಞಾನ, ಶಾಸ್ತ್ರಗಳು, ಆಧ್ಯಾತ್ಮಿಕ ಹಾಗೂ ಲೌಕಿಕಾನುಭವಗಳೆಲ್ಲದರಲ್ಲೂ–ಅವನಿಗಿಂತ ಹಿರಿಯ ರಾದ ಸ್ವಾಮೀಜಿ ಅವನ ವಾದವನ್ನು ಸುಲಭದಲ್ಲಿ ಅಡಗಿಸಿಬಿಟ್ಟರು. ಅಲ್ಲದೆ ಅವರೊಂದು ಮಾತನ್ನು ಹೇಳಿದರು, “ಒಂದು ಕಿಂಚಿತ್ತಾದರೂ ಆಧ್ಯಾತ್ಮಿಕ ಅನುಭವವಿಲ್ಲದಂತಹ ಧಾರ್ಮಿಕ ಜೀವನವೆಂದರೆ ಅದೆಂಥ ಧಾರ್ಮಿಕ ಜೀವನ! ಅಂತಹ ಧಾರ್ಮಿಕನಿಗಿಂತ ಕಡೆಯಪಕ್ಷ ಒಂದು ಪ್ರೇತವನ್ನು ಕಂಡಿರುವವನೇ ಎಷ್ಟೋ ಉತ್ತಮ.” ಪ್ರೇತವನ್ನು ಕಾಣುವುದೆಂದರೆ ಅದೊಂದು ಅಸಾಮಾನ್ಯ, ಅಲೌಕಿಕ ಅನುಭವವೇ ಸರಿ. ಅದು ಎಲ್ಲರಿಗೂ ಸುಲಭವಾಗಿ ಆಗುವಂತಹ ಅನುಭವವಲ್ಲ. ಇಂತಹ ಕ್ಷುದ್ರ ಅನುಭವವನ್ನು ಪಡೆದವನಾದರೂ ಸುಮ್ಮನೆ ಮಣಗಟ್ಟಲೆ ಶಾಸ್ತ್ರಗಳನ್ನು ಓದಿಕೊಂಡವನಿಗಿಂತ ಮೇಲು ಎಂಬುದು ಸ್ವಾಮೀಜಿಯ ಅಭಿಪ್ರಾಯ. ಏಕೆಂದರೆ ಧಾರ್ಮಿಕ ವ್ಯಕ್ತಿಯೆಂದು ಕರೆಸಿಕೊಳ್ಳಬೇಕಾದರೆ ಶಾಸ್ತ್ರಗ್ರಂಥಗಳನ್ನು ಓದಿಕೊಂಡು ಭಾಷಣ ಬಿಗಿದು ಜನರನ್ನು ಬೆರಗುಗೊಳಿಸಿದರೆ ಸಾಲದು; ಓದಿ ಗಳಿಸಿದ ಪಾಂಡಿತ್ಯವನ್ನು ಬಳಿಸಿಕೊಂಡು ಸಾಧನೆ ಮಾಡಬೇಕು. ಸಾಧನೆ ಮಾಡಿ ಇಂದ್ರಿಯಾತೀತವಾದ ಅನುಭವಗಳನ್ನು ಮಾಡಿಕೊಳ್ಳ ಬೇಕು. ಇಂತಹ ಅನುಭವಗಳನ್ನು ಪಡೆದುಕೊಂಡವರು ಆ ಶಾಸ್ತ್ರಗಳೆಲ್ಲ ಗೊಡ್ಡು ಕಂತೆಗಳು ಎಂದು ಕರೆಯುವ ಹೆಡ್ಡತನದ ಮಾತನಾಡಲಾರರು.

ಹೀಗೆ ಅತ್ಯಂತ ವೈಜ್ಞಾನಿಕವಾಗಿ, ತರ್ಕಬದ್ಧವಾಗಿ ಮಾತನಾಡಿ ಸ್ವಾಮೀಜಿ ತನ್ನನ್ನು ಸುಮ್ಮನಾಗಿ ಸಿದಾಗ ಆ ಎಂಜಿನಿಯರಿಗೆ ಅಸಾಧ್ಯ ಕೋಪ ಬಂದಿತು. ತಡೆಯಲಾರದೆ ಅವಿನಯದಿಂದ ವರ್ತಿಸಲಾರಂಭಿಸಿದ. ಆಗ ಸ್ವಾಮೀಜಿಯ ಆತಿಥೇಯರಾದ ಭಾಟೆ ಅದನ್ನು ಸಹಿಸದೆ ಪ್ರತಿಭಟಿಸಿ ದರು. ತಕ್ಷಣ ಸ್ವಾಮೀಜಿ ಮುಗುಳ್ನಗುತ್ತ ಮಧ್ಯೆ ಪ್ರವೇಶಿಸಿ, “ಇರಲಿ ಬಿಡಿ, ಪರವಾಗಿಲ್ಲ” ಎನ್ನುತ್ತ ಸಮಾಧಾನಗೊಳಿಸಿದರು.

ಸ್ವಾಮೀಜಿ ಈ ಬಗೆಯ ಚರ್ಚೆಗಳಲ್ಲಿ ಭಾಗವಹಿಸುತ್ತಿದ್ದರಾದರೂ, ಅವರ ಉದ್ದೇಶವು ಚರ್ಚೆಯಲ್ಲಿ ಗೆಲ್ಲಬೇಕೆಂಬುದಕ್ಕಿಂತ ಹೆಚ್ಚಾಗಿ, ಸತ್ಯಾಂಶವನ್ನು ತೋರಿಸಿಕೊಡುವುದಾಗಿತ್ತು. ಹಿಂದೂಧರ್ಮ ಇನ್ನೂ ಜೀವಂತವಾಗಿದೆಯೆಂಬ ಸತ್ಯವನ್ನು ಭಾರತಕ್ಕೆ ಮಾತ್ರವಲ್ಲದೆ ಇಡೀ ಜಗತ್ತಿಗೇ ತೋರಿಸಿಕೊಡಲು ಹಾಗೂ ವೇದಾಂತದಲ್ಲಡಗಿರುವ ಅನರ್ಘ್ಯ ಸತ್ಯವನ್ನು ಸಮಸ್ತ ವಿಶ್ವದ ಮೇಲೆ ವರ್ಷಿಸಲು ಕಾಲ ಸನ್ನಿಹಿತವಾಗಿದೆ ಎಂಬುದನ್ನು ತಿಳಿಸಿಕೊಡುವುದು ಅವರ ಇಚ್ಛೆಯಾಗಿತ್ತು. ಇದನ್ನು ಸ್ವತಃ ಅವರೇ ಸ್ಪಷ್ಟಪಡಿಸುತ್ತಾರೆ. ತಮ್ಮ ಜೀವನೋದ್ದೇಶದ ಒಂದು ಕಲ್ಪನೆ ಅವರಲ್ಲಿ ಮೂಡುತ್ತಿರುವುದನ್ನು ನಾವಿಲ್ಲಿ ಕಾಣಬಹುದು.

ಒಂದು ದಿನ ಭಾಟೆಯವರು ಸ್ವಾಮೀಜಿಯನ್ನು ಹರಿಪದ ಮಿತ್ರ ಎಂಬವನ ಮನೆಗೆ ಕರೆದು ಕೊಂಡು ಹೋಗಿ ಅವನಿಗೆ ಪರಿಚಯ ಮಾಡಿಸಿದರು. ಈತ ಅಲ್ಲಿನ ಅರಣ್ಯಾಧಿಕಾರಿ; ಒಳ್ಳೇ ವಿದ್ಯಾವಂತ, ಬುದ್ಧಿವಂತ. ಆದರೆ ದೇವರು, ಧರ್ಮ–ಇವುಗಳಲ್ಲಿ ಅವನ ನಂಬಿಕೆ ಸಂಪೂರ್ಣ ಅಳಿಸಿಹೋಗಿದ್ದಿತು. ಎಲ್ಲಾ ಸಾಧುಸಂನ್ಯಾಸಿಗಳನ್ನೂ ಮೋಸಗಾರರು ಎಂಬ ದೃಷ್ಟಿಯಿಂದಲೇ ನೋಡುತ್ತಿದ್ದ. ಈಗ ಸ್ವಾಮೀಜಿಯನ್ನು ನೋಡಿದಾಗ ತನಗರಿವಿಲ್ಲದಂತೆಯೇ ಅವರಿಂದ ಆಕರ್ಷಿತನಾದ. “ಶಾಂತ ಮುಖಮುದ್ರೆ, ಮಿಂಚಿನಂತೆ ಮಿನುಗುವ ಕಣ್ಣುಗಳು, ಮೈಮೇಲೆ ಕಾಷಾಯ ವಸ್ತ್ರ, ತಲೆಗೆ ಸುತ್ತಿದ್ದ ಪೇಟ, ಕಾಲಿಗೆ ಸಾಮಾನ್ಯರು ಬಳಸುವಂತಹ ಕೊಲ್ಲಾಪುರಿ ಚಪ್ಪಲಿ”–ಇದು ಹರಿಪದ ಮಿತ್ರನೇ ವರ್ಣಿಸುವಂತೆ ಸ್ವಾಮೀಜಿಯ ಚಿತ್ರ. ಆದರೂ ಅವರು ಬಂಗಾಳಿಗಳೆಂದು ತಿಳಿದುಬಂದಾಗ ಮೊದಲು ಅವನ ಮನಸ್ಸಿಗೆ ಬಂದ ಭಾವನೆಯೆಂದರೆ, “ಓಹೋ! ಈ ಬಂಗಾಳೀ ಸಂನ್ಯಾಸಿಗೆ ಮರಾಠಿಗರ ಮನೆ ಒಗ್ಗಲಿಲ್ಲವೆಂದು ಕಾಣುತ್ತದೆ. ಆದ್ದರಿಂದ ಈಗ ಬಂಗಾಳಿಯ ಮನೆಯನ್ನೇ ಹುಡುಕಿಕೊಂಡು ಬಂದಿದ್ದಾನೆ!” ಎಂದು. ಆದರೂ ಅವರನ್ನು ಸ್ವಾಗತಿಸಿ ಬರಮಾಡಿಕೊಂಡ. ಅವರೊಂದಿಗೆ ಹತ್ತು ನಿಮಿಷ ಮಾತನಾಡುವಷ್ಟರಲ್ಲಿ ಅವನಿಗೆ ಗೊತ್ತಾಯಿತು–ಇವರು ತನಗಿಂತ ಎಲ್ಲ ವಿಷಯದಲ್ಲೂ ಸಹಸ್ರಪಾಲು ಉತ್ತಮರು ಎಂದು. ಅವನು ನಿರೀಕ್ಷಿಸಿದ್ದಕ್ಕೆ ವಿರುದ್ಧವಾಗಿ ಸ್ವಾಮೀಜಿ ಅವನಿಂದ ಏನನ್ನೂ ಬೇಡಲೇ ಇಲ್ಲ. ಕಡೆಗೆ ಸೌಜನ್ಯಕ್ಕಾಗಿ, ಅವರನ್ನು ತನ್ನ ಮನೆಯಲ್ಲೇ ಉಳಿದುಕೊಳ್ಳುವಂತೆ ವಿನಂತಿಸಿಕೊಂಡ ಹರಿಪದಬಾಬು. ಆಗ ಅವರು, “ನನಗೆ ಆ ವಕೀಲರ ಮನೆ ಚೆನ್ನಾಗಿ ಒಗ್ಗಿದೆ. ಮನೆಯವರೆಲ್ಲ ನನ್ನ ವಿಷಯದಲ್ಲಿ ತುಂಬಾ ವಿಶ್ವಾಸ ತೋರಿಸುತ್ತಿದ್ದಾರೆ. ಈಗ ನಾನು ಬಂಗಾಳಿಗಳೊಬ್ಬರು ಸಿಕ್ಕಿದರೆಂದು ಅವರ ಮನೆಯಿಂದ ಹೊರಟುಬಿಟ್ಟರೆ, ಅವರೇನೆಂದುಕೊಂಡಾರು? ಆದ್ದರಿಂದ ಈ ಬಗ್ಗೆ ನಾನು ವಿಚಾರಮಾಡಿ ತಿಳಿಸುತ್ತೇನೆ” ಎಂದರು. ಬಳಿಕ ಹೆಚ್ಚಿನ ಮಾತುಕತೆಯೇನೂ ನಡೆಯಲಿಲ್ಲ. ಆದರೆ ಅವರಾಡಿದ ನಾಲ್ಕಾರು ಮಾತುಗಳಿಂದಲೇ ಹರಿಪದ ಬಾಬು ತೀವ್ರವಾಗಿ ಪ್ರಭಾವಿತನಾಗಿದ್ದ. ಮರುದಿನ ಬೆಳಗ್ಗೆ ಉಪಾಹಾರಕ್ಕೆ ತನ್ನ ಮನೆಗೆ ಆಹ್ವಾನಿಸಿದ.

ನಿಜಕ್ಕೂ ಎಲ್ಲ ಬಗೆಯ ‘ಬೇಕು’ಗಳಿಂದಲೂ ಮುಕ್ತರಾದ ಸಂನ್ಯಾಸಿಗಳಿರಲು ಸಾಧ್ಯವೆಂಬು ದನ್ನು ಅವನು ಈವರೆಗೆ ನಂಬಲು ಸಿದ್ಧನಿರಲಿಲ್ಲ. ಆದರೆ ಇಂದು ಅಂತಹವರೊಬ್ಬರನ್ನು ಕಾಣುವ ಸುಯೋಗ ಅವನಿಗೆ ಲಭಿಸಿತ್ತು. ‘ಮನಸ್ಸು ಮಾಡಿದ್ದರೆ ಸ್ವಾಮೀಜಿ ಸಾಕಷ್ಟು ಹಣವನ್ನು ಸಂಪಾದಿಸಬಹುದಾಗಿತ್ತು; ಆದರೆ ಅವರು ಸ್ವಾರ್ಥವನ್ನು ಸಂಪೂರ್ಣ ತ್ಯಜಿಸಿದ್ದರಿಂದಲೇ ಹೀಗೆ ಸಂನ್ಯಾಸಿಯಾಗಿದ್ದಾರೆ; ಮತ್ತು ಹೀಗೆ ಸ್ವಾರ್ಥದ ಲವಲೇಶವೂ ಇಲ್ಲದಿರುವುದರಿಂದಲೇ ಅವರು ಅಷ್ಟೊಂದು ಆನಂದಭರಿತರಾಗಿ, ಹಸನ್ಮುಖಿಗಳಾಗಿದ್ದಾರೆ’ಎಂದು ಆತ ಆಲೋಚಿಸಿದ.

ಮರುದಿನ ಬೆಳಗ್ಗೆ ಆರು ಗಂಟೆಯಿಂದಲೇ ಅವರಿಗಾಗಿ ಕಾಯುತ್ತ ಕುಳಿತ. ಆದರೆ ಗಂಟೆ ಎಂಟಾದರೂ ಸ್ವಾಮೀಜಿ ಬಾರದಿದ್ದಾಗ ತಾನೇ ವಕೀಲರ ಮನೆಗೆ ಹೊರಟ. ಅಲ್ಲಿ ಅವರು ಅನೇಕ ಜನ ಪಂಡಿತರು, ವಕೀಲರು ಹಾಗೂ ಇತರ ಗಣ್ಯ ವ್ಯಕ್ತಿಗಳಿಂದ ಸುತ್ತುವರಿದು ಕುಳಿತಿದ್ದಾರೆ; ಎಲ್ಲರೂ ಬೇರೆಬೇರೆ ವಿಷಯಗಳಿಗೆ ಸಂಬಂಧಪಟ್ಟಂತೆ ಪ್ರಶ್ನೆಗಳನ್ನು ಕೇಳುತ್ತಿದ್ದಾರೆ; ಅವುಗಳಿ ಗೆಲ್ಲ ಸ್ವಾಮೀಜಿ ಸ್ವಲ್ಪವೂ ಅಳುಕಿಲ್ಲದೆ, ಕೆಲವೊಮ್ಮೆ ಇಂಗ್ಲಿಷಿನಲ್ಲಿ, ಕೆಲವೊಮ್ಮೆಹಿಂದಿ ಅಥವಾ ಸಂಸ್ಕೃತದಲ್ಲಿ ಉತ್ತರಿಸುತ್ತಿದ್ದಾರೆ. ಕುಳಿತವರೆಲ್ಲ ಮಂತ್ರಮುಗ್ಧರಾಗಿ ಅವರ ಉತ್ತರಗಳನ್ನು ಆಲಿಸುತ್ತಿದ್ದಾರೆ. ಅವರು ಉತ್ತರಿಸುವ ವೈಖರಿಯನ್ನು ಕಂಡು ಹರಿಪದ ಬಾಬುವಿಗೆ ಅನ್ನಿಸಿತು, ‘ಇವರೇನು ಮನುಷ್ಯರೋ ಅಥವಾ ದೇವತೆಯೋ!’

ಈ ಪ್ರಶ್ನೋತ್ತರದ ಸಂದರ್ಭದಲ್ಲಿ ವಕೀಲನೊಬ್ಬ ಕೇಳಿದ:

“ಸ್ವಾಮೀಜಿ, ನಾವು ಬೆಳಗ್ಗೆ-ಸಂಜೆ ಸಂಧ್ಯಾವಂದನೆಯ ಸಮಯದಲ್ಲಿ ಹೇಳುವ ಮಂತ್ರಗಳೆಲ್ಲ ಸಂಸ್ಕೃತದಲ್ಲಿವೆ. ನಮಗೆ ಅವೆಲ್ಲ ಒಂದಿಷ್ಟೂ ಅರ್ಥವಾಗುವುದಿಲ್ಲ. ಅವುಗಳನ್ನು ಸುಮ್ಮನೆ ಹಾಗೆಯೇ ಹೇಳುತ್ತ ಹೋಗುವುದರಿಂದ ನಮಗೇನಾದರೂ ಪ್ರಯೋಜನವಿದೆಯೆ?”

ಸ್ವಾಮೀಜಿ ದೃಢವಾಗಿ ಉತ್ತರಿಸಿದರು:

“ಖಂಡಿತ ಅವುಗಳಿಂದ ಸತ್ಪರಿಣಾಮವುಂಟಾಗುತ್ತದೆ. ಆದರೆ, ಮೊದಲನೆಯದಾಗಿ, ನೀನು ಬ್ರಾಹ್ಮಣಕುಟುಂಬದಲ್ಲಿ ಹುಟ್ಟಿರುವುದರಿಂದ ಆ ಮಂತ್ರಗಳ ಅರ್ಥವನ್ನು ಸುಲಭವಾಗಿ ಅರಿತು ಕೊಳ್ಳಬಹುದಾಗಿತ್ತು. ನೀನು ಹಾಗೆ ಮಾಡದಿದ್ದರೆ ಅದು ಯಾರ ತಪ್ಪು? ಹೋಗಲಿ, ಅರ್ಥ ಗೊತ್ತಿಲ್ಲದಿದ್ದರೂ ಪರವಾಗಿಲ್ಲ. ಸಂಧ್ಯಾವಂದನೆಗೆ ಕುಳಿತಾಗ, ‘ನಾನು ಮಾಡುತ್ತಿರುವುದು ಪುಣ್ಯದ ಕೆಲಸ’, ‘ಪಾಪಕಾರ್ಯವಲ್ಲ’ ಎಂದು ನಿನಗೆ ಅನ್ನಿಸುವುದಿಲ್ಲವೆ? ‘ನಾನೊಂದು ಒಳ್ಳೆಯ ಕೆಲಸ ಮಾಡುತ್ತಿದ್ದೇನೆ’ ಎಂದು ಭಾವಿಸಿದರೂ ಸಾಕು, ಅದರಿಂದಲೇ ಎಷ್ಟೋ ಸತ್ಪರಿಣಾಮವುಂಟಾಗುತ್ತದೆ.”

ಈ ಸಂಭಾಷಣೆ ಇಂಗ್ಲಿಷಿನಲ್ಲಿ ನಡೆದಿರಬೇಕೆಂದು ಕಾಣುತ್ತದೆ. ಆದ್ದರಿಂದಲೋ ಏನೋ ಒಬ್ಬರು “ಧಾರ್ಮಿಕ ವಿಷಯಗಳ ಮೇಲಿನ ಸಂಭಾಷಣೆಯನ್ನು ಪರಭಾಷೆಯಲ್ಲಿ ಮಾಡಬಾರದು. ಹೀಗೆಂದು ಒಂದು ಪುರಾಣದಲ್ಲಿ ಹೇಳಿದೆ”ಎಂದು ಆಕ್ಷೇಪಿಸಿದರು. ಅದಕ್ಕೆ ಸ್ವಾಮೀಜಿ ತಕ್ಷಣ ಉತ್ತರಿಸಿದರು:

“ಧಾರ್ಮಿಕ ವಿಷಯವಾಗಿ ಮಾತನಾಡುವುದು ಒಳ್ಳೆಯದೇ–ಭಾಷೆ ಯಾವುದಾದರೂ ಪರವಾ ಗಿಲ್ಲ” ಹೀಗೆ ಹೇಳಿ ತಮ್ಮ ಮಾತನ್ನು ಸಮರ್ಥಿಸುವ ಶ್ಲೋಕವೊಂದನ್ನು ವೇದಗಳಿಂದ ಉದಾ ಹರಿಸಿ ಬಳಿಕ ಮತ್ತೆ ಹೇಳಿದರು, “ಉಚ್ಚ ನ್ಯಾಯಾಲಯದ ತೀರ್ಪನ್ನು ಜಿಲ್ಲಾ ನ್ಯಾಯಾಲಯ ಅನೂರ್ಜಿತಗೊಳಿಸಲಾಗದು!” ಎಂದರೆ, ವೇದಗಳು ಹಿಂದೂಧರ್ಮದ ಆಧಾರಗ್ರಂಥಗಳಾದ್ದ ರಿಂದ, ವೇದೋಕ್ತಿಯನ್ನು ಪುರಾಣದ ಮಾತೊಂದರಿಂದ ತಳ್ಳಿಹಾಕಲಾಗದು ಎಂದರ್ಥ.

ಸುಮಾರು ಒಂಬತ್ತು ಗಂಟೆಯ ವೇಳೆಗೆ ಬಹಳಷ್ಟು ಜನ ಸ್ವಾಮೀಜಿಗೆ ನಮಸ್ಕರಿಸಿ ಹೊರಟರು. ಆಗ ಹರಿಪದ ಮಿತ್ರನತ್ತ ತಿರುಗಿ ಸ್ವಾಮೀಜಿ ಹೇಳಿದರು, “ನೋಡು ಇಷ್ಟು ಜನರಿಗೆ ನಿರಾಶೆಯುಂಟುಮಾಡಿ ಬರಲು ನನಗೆ ಮನಸ್ಸಾಗಲಿಲ್ಲ. ಸಮಯಕ್ಕೆ ತಪ್ಪಿದ್ದಕ್ಕಾಗಿ ಕ್ಷಮಿಸುತ್ತೀ ಎಂದು ಭಾವಿಸುತ್ತೇನೆ.” ಹರಿಪದ ಬಾಬು ಸ್ವಾಮೀಜಿಯನ್ನು ತನ್ನ ಮನೆಗೇ ಬಂದಿರುವಂತೆ ಹೃತ್ಪೂರ್ವಕವಾಗಿ ಒತ್ತಾಯದಿಂದ ಬೇಡಿಕೊಂಡ. ಮೊದಲು ಅವರು ಒಪ್ಪಲಿಲ್ಲ. ಕಡೆಗೆ ಹೇಳಿ ದರು, “ನೀನು ಈ ಮನೆಯವರನ್ನು ಒಪ್ಪಿಸುವುದಾದರೆ ನಾನು ಬರಲು ಸಿದ್ಧ.” ಬಹಳ ಕಷ್ಟಪಟ್ಟು ಭಾಟೆಯವರನ್ನು ಒಪ್ಪಿಸಿ, ಸ್ವಾಮೀಜಿಯನ್ನು ಮತ್ತೂ ಒತ್ತಾಯಿಸಿ, ಹರಿಪದ ಅವರನ್ನು ತನ್ನ ಮನೆಗೆ ಕರೆದುಕೊಂಡು ಹೋದ. ಆಗ ಸ್ವಾಮೀಜಿಯ ಬಳಿಯಿದ್ದ ವಸ್ತುಗಳಲ್ಲಿ ಅವರು ಆ ಸಮಯದಲ್ಲಿ ಅಧ್ಯಯನ ಮಾಡುತ್ತಿದ್ದ ಫ್ರೆಂಚ್ ಸಂಗೀತದ ಕುರಿತಾದ ಪುಸ್ತಕವೂ ಒಂದು.

ಹರಿಪದ ಬಾಬುವಿನ ಮನೆಯಲ್ಲಿ ದಿನಗಳು ನಿರಂತರ ಚರ್ಚೆ ಸಂಭಾಷಣೆಗಳಲ್ಲಿ ಕಳೆದುವು. ಎಷ್ಟೋ ವರ್ಷಗಳಿಂದ ಅವನ ತಲೆಯಲ್ಲಿ ಗೂಡುಕಟ್ಟಿಕೊಂಡಿದ್ದ ಸಂಶಯ ಪಕ್ಷಿಗಳೆಲ್ಲ ಒಂದೊಂದಾಗಿ ಹಾರಿಹೋದವು. ಆ ಪ್ರಶ್ನೋತ್ತರಗಳು ತುಂಬ ಕುತೂಹಲಕರವಾಗಿದೆ. ಒಮ್ಮೆ ಅವನು ಕೇಳಿದ:

“ನನ್ನ ನಂಬಿಕೆಯೇನೆಂದರೆ ಸತ್ಯವು (ದೇವರು) ಒಂದೇ, ಅದು ನಿರಪೇಕ್ಷವಾದದ್ದು \eng{(Absolute)}ಎಂದು. ಆದರೆ ಆ ಸತ್ಯಕ್ಕೆ ದಾರಿಯೆಂದು ಹೇಳಲಾಗುವ ಹಲವಾರು ಧರ್ಮ ಗಳೆಲ್ಲವೂ ಏಕಕಾಲಕ್ಕೆ ನಿಜವಾಗಿರಲು ಹೇಗೆ ಸಾಧ್ಯ?”

“ನೋಡು, ಯಾವುದೇ ವಿಷಯದ ಬಗ್ಗೆ ನಮಗೆ ಈಗಿರುವ ಜ್ಞಾನವಾಗಲಿ, ಮುಂದೆ ನಾವು ತಿಳಿಯಬಹುದಾದದ್ದೇ ಆಗಲಿ–ಇವೆಲ್ಲ ಸಾಪೇಕ್ಷ \eng{(relative)}ಸತ್ಯಗಳು. ನಮ್ಮ ಈ ಸಾಂತ ಬುದ್ಧಿಶಕ್ತಿಗೆ ಅನಂತ ಸತ್ಯವನ್ನು ಗ್ರಹಿಸಲು ಸಾಧ್ಯವಿಲ್ಲ. ಆದ್ದರಿಂದ ಸತ್ಯವು ನಿರಪೇಕ್ಷವಾದದ್ದೇ ಆದರೂ, ಬೇರೆ ಬೇರೆ ಮನಸ್ಸುಗಳಿಗೆ ಹಾಗೂ ಬುದ್ಧಿಶಕ್ತಿಗಳಿಗೆ ಅದು ವಿಭಿನ್ನವಾಗಿ ತೋರುತ್ತದೆ. ಸತ್ಯದ ಈ ಎಲ್ಲ ಮುಖಗಳೂ ನಿರಪೇಕ್ಷ ಸತ್ಯದ, ಪರಮ ಸತ್ಯದ ವರ್ಗಕ್ಕೇ ಸೇರಿದವುಗಳು. ಅದು ಹೇಗೆಂದರೆ, ಒಬ್ಬನೇ ಸೂರ್ಯನ ಛಾಯಾ ಚಿತ್ರಗಳನ್ನು ಬೇರೆ ಬೇರೆ ಸಮಯಗಳಲ್ಲಿ ಬೇರೆ ಬೇರೆ ದಿಕ್ಕುಗಳಲ್ಲಿ ತೆಗೆದರೆ, ಪ್ರತಿಯೊಂದು ಚಿತ್ರದಲ್ಲೂ ಆತ ವಿಭಿನ್ನನಾಗಿ ಕಾಣುವು ದಿಲ್ಲವೆ? ಹೀಗೆ, ಈ ಬೇರೆಬೇರೆ ತುಲನಾತ್ಮಕ ಸತ್ಯಗಳು ಆ ಪರಮಸತ್ಯದೊಂದಿಗೆ ಏಕರೀತಿಯ ಸಂಬಂಧವುಳ್ಳವುಗಳಾಗಿವೆ. ಆದ್ದರಿಂದ ಪ್ರತಿಯೊಂದು ಧರ್ಮವೂ ಸತ್ಯ; ಏಕೆಂದರೆ, ಅದು ಒಂದೇ ಪರಮಧರ್ಮದ ಒಂದೊಂದು ರೂಪ.”

“ಆದರೆ, ಸ್ವಾಮೀಜಿ, ಶ್ರದ್ಧೆಯೇ ಪ್ರತಿಯೊಂದು ಧರ್ಮಕ್ಕೂ ಮೂಲ...”

ಈ ಮಾತಿನ ಅರ್ಥವೇನೆಂದರೆ, ‘ಪ್ರತಿಯೊಂದು ಧರ್ಮವೂ ಮೊದಲು ದೇವರಲ್ಲಿ ನಂಬಿಕೆ ಯಿರಿಸಿ ಮುಂದುವರಿಯುವಂತೆ ಹೇಳುತ್ತದೆ. ಮೊದಲೇ ನಂಬಿಕೆಯಿರಬೇಕೆಂದರೆ ಹೇಗೆ ಸಾಧ್ಯ?’ ಎಂದು. ಸ್ವಾಮೀಜಿ ನಸುನಕ್ಕು ಹೇಳಿದರು:

“ಒಬ್ಬ ಮನುಷ್ಯ ರಾಜನಾಗಿಬಿಟ್ಟನೆಂದರೆ ಅವನಿಗಿನ್ನು ಬಡತನವೇ ಇರುವುದಿಲ್ಲ. ಆದರೆ ಈಗ ಪ್ರಶ್ನೆಯೇನೆಂದರೆ, ರಾಜನಾಗುವುದು ಹೇಗೆ ಎಂದು. ಶ್ರದ್ಧೆಯನ್ನು ಹೊರಗಿನಿಂದ ತುರುಕಲು ಸಾಧ್ಯವೆ? ಯಾರಿಗೇ ಆಗಲಿ, ಸ್ವಂತ ಅನುಭವವಾದ ಹೊರತು ನಿಜವಾದ ಶ್ರದ್ಧೆಯುಂಟಾಗಲು ಸಾಧ್ಯವಿಲ್ಲ.”

ಈಗ ಹರಿಪದ ಬಾಬು ಸಂಸ್ಯಾಸಿಗಳ ಬಗ್ಗೆ ಕೇಳಿದ:

“ಸ್ವಾಮೀಜಿ ಸಂನ್ಯಾಸಿಗಳು ಭಿಕ್ಷೆಯ ಮೇಲೆಯೇ ಅವಲಂಬಿಸಿಕೊಂಡಿದ್ದು ಸುಮ್ಮನೆ ವ್ಯರ್ಥ ವಾಗಿ ಕಾಲಹರಣ ಮಾಡುತ್ತಾರಲ್ಲ ಏಕೆ? ಅವರೇಕೆ ಸಮಾಜಕ್ಕೆ ಉಪಯೋಗಕರವಾದ ಯಾವುದಾದರೂ ಕಾರ್ಯದಲ್ಲಿ ತೊಡಗಬಾರದು?”

“ಈಗ ನೋಡಿಲ್ಲಿ; ನೀನು ಅಷ್ಟೊಂದು ಕಷ್ಟಪಟ್ಟು ಸಂಪಾದಿಸುತ್ತಿರುವ ಹಣದಲ್ಲಿ, ಎಷ್ಟು ಹಣವನ್ನು ನಿನಗೋಸ್ಕರ ಖರ್ಚುಮಾಡುತ್ತಿದ್ದೀಯೆ? ಎಲ್ಲೋ ಸ್ವಲ್ಪ ಭಾಗ ಮಾತ್ರ, ಹೌದು ತಾನೆ? ಇನ್ನುಳಿದದ್ದರಲ್ಲಿ, ನೀನು ಯಾರನ್ನು ನಿನ್ನ ಸ್ವಂತದವರು ಎಂದು ಭಾವಿಸುತ್ತೀಯೋ, ಅವರಿಗೋಸ್ಕರ ಬಹುಪಾಲನ್ನು ಖರ್ಚುಮಾಡುತ್ತೀಯೆ.ಆದರೆ ನೀನು ಅವರಿಗೋಸ್ಕರ ಎಷ್ಟು ಖರ್ಚು ಮಾಡಿದರೂ, ಅದಕ್ಕೆ ಅವರು ನಿನಗೆ ಕೃತಜ್ಞತೆಯನ್ನೂ ತೋರಿಸುವುದಿಲ್ಲ, ತೃಪ್ತಿಯನ್ನೂ ಪಟ್ಟುಕೊಳ್ಳುವುದಿಲ್ಲ. ನೀನು ಕೂಡಿಡುವ ಹಣವೆಲ್ಲ ಯಕ್ಷನ ನಿಧಿಯಂತೆ\footnote{* ನೋಡಿ: ಅನುಬಂಧ ೨.}; ನಿನಗೆಂದಿಗೂ ಅದರಿಂದ ಸುಖವಿಲ್ಲ. ನೀನು ಸತ್ತ ಮೇಲೆ ಬೇರೆ ಇನ್ಯಾರೋ ಅದನ್ನು ಅನುಭೋಗಿಸುತ್ತಾರೆ. ಅಲ್ಲದೆ ಅವರು ನಿನ್ನನ್ನು ಶಪಿಸಲೂಬಹುದು–ಇವನ್ಯಾಕೆ ಇನ್ನೂ ಹೆಚ್ಚು ಕೂಡಿಡಲಿಲ್ಲ ಎಂದು. ಇದು ನಿನ್ನ ಅವಸ್ಥೆ, ಆದರೆ ನಾನಾದರೋ, ಏನನ್ನೂ ಮಾಡುವುದಿಲ್ಲ. ನನಗೆ ಹಸಿವಾದಾಗ ಭಿಕ್ಷೆ ಬೇಡಿ, ಏನು ಸಿಕ್ಕರೆ ಅದನ್ನು ತಿನ್ನುತ್ತೇನೆ. ನಾನು ಕಷ್ಟಪಡುವುದೂ ಇಲ್ಲ, ಕೂಡಿಡುವುದೂ ಇಲ್ಲ! ಈಗ ಹೇಳು, ನಮ್ಮಿಬ್ಬರಲ್ಲಿ ಯಾರು ಬುದ್ಧಿವಂತರು–ನಾನೋ ನೀನೋ?”

ಹರಿಪದ ಮಿತ್ರನಿಗೆ ಆಶ್ಚರ್ಯ. ಈ ಹಿಂದೆ ಯಾರೂ ಇಷ್ಟು ಧೈರ್ಯವಾಗಿ, ಮನಬಿಚ್ಚಿ ಮಾತನಾಡಿದ್ದನ್ನು ಅವನು ಕಂಡಿರಲಿಲ್ಲ. ಬಳಿಕ ಊಟವಾದ ಮೇಲೆ ಇಬ್ಬರೂ ವಕೀಲರ ಮನೆಗೆ ಹೋದರು. ಅಲ್ಲಿಯೂ ಯಥಾಪ್ರಕಾರ ಸ್ವಾಮೀಜಿ ಅನೇಕ ಜನರೊಂದಿಗೆ ಚರ್ಚೆ-ಸಂಭಾಷಣೆ ನಡೆಸಿದರು. ರಾತ್ರಿ ಹರಿಪದ ಬಾಬುವಿನ ಮನೆಗೆ ಹಿಂದಿರುಗುವಾಗ ಒಂಬತ್ತು ಗಂಟೆ. ದಾರಿಯಲ್ಲಿ ಅವನು ಕೇಳಿದ್ದ. “ಸ್ವಾಮೀಜಿ, ಈ ಎಲ್ಲ ಚರ್ಚೆಗಳಿಂದ ನಿಮಗೆ ತುಂಬ ಬೇಸರವಾಗಿರಬೇಕಲ್ಲವೆ?”

ಸ್ವಲ್ಪ ತಮಾಷೆಯಾಗಿಯೇ ಉತ್ತರಿಸಿದರು ಸ್ವಾಮೀಜಿ:

“ಅಲ್ಲ, ನಾನು ಒಂದು ಮಾತನ್ನೂ ಆಡದೆ ತೆಪ್ಪಗೆ ಒಂದು ಕಡೆ ಕುಳಿತಿದ್ದರೆ, ಪಕ್ಕಾ ಲೆಕ್ಕಾ ಚಾರದ ಬುದ್ಧಿಯವರಾದ ನೀವು ಪ್ರಾಪಂಚಿಕರು, ನನಗೆ ಒಂದು ತುತ್ತು ಅನ್ನವನ್ನಾದರೂ ಕೊಟ್ಟೀರಾ? ಅದಕ್ಕೇ ನಾನು ಈ ರೀತಿ ಹರಟುವುದು! ಇದರಿಂದ ಜನಗಳಿಗೆ ಖುಷಿಯಾಗಿ ನನ್ನ ಸುತ್ತ ಸೇರುತ್ತಾರೆ. ಆದರೆ ಒಂದು ಮಾತು ತಿಳಿದುಕೋ–ಯಾರು ಈ ರೀತಿ ವಾದ ಮಾಡು ತ್ತಾರೋ, ತುಂಬ ಜನ ಸೇರಿರುವಾಗ ಅವರ ಮುಂದೆ ಪ್ರಶ್ನೆ ಹಾಕುತ್ತಾರೋ, ಅವರಿಗೆ ಸತ್ಯವನ್ನು ತಿಳಿದುಕೊಳ್ಳಬೇಕು ಎಂಬ ಹಂಬಲ ಖಂಡಿತ ಇರುವುದಿಲ್ಲ. ನಾನೂ ಅವರ ಮನೋಭಾವವನ್ನು ಅರ್ಥಮಾಡಿಕೊಂಡು ಅದರಂತೆಯೇ ಉತ್ತರಿಸುತ್ತೇನೆ.”

“ಆದರೆ, ಸ್ವಾಮೀಜಿ, ಎಲ್ಲ ಪ್ರಶ್ನೆಗಳಿಗೂ ಚುರುಕಾದ, ನೇರವಾದ ಉತ್ತರಗಳು ನಿಮ್ಮ ಹತ್ತಿರ ಯಾವಾಗಲೂ ಸಿದ್ಧವಾಗಿರುತ್ತವಲ್ಲ, ಹೇಗದು?”

“ಈ ಪ್ರಶ್ನೆಗಳೆಲ್ಲ ನಿನಗೆ ಹೊಸದಿರಬಹುದು. ಆದರೆ ನಾನು ಇಂತಹ ಪ್ರಶ್ನೆಗಳಿಗೆಲ್ಲ ಈಗಾ ಗಲೇ ಎಷ್ಟು ಸಲ ಉತ್ತರಿಸಿದ್ದೇನೆಯೋ ಲೆಕ್ಕವೇ ಇಲ್ಲ!”

ಮತ್ತೆ ರಾತ್ರಿಯ ಊಟದ ವೇಳೆಯಲ್ಲೂ ಈ ಮಾತುಕತೆ ಮುಂದುವರಿಯಿತು. ಆಗ ಸ್ವಾಮೀಜಿ, ತಮ್ಮ ಪರಿವ್ರಾಜಕ ಜೀವನದ ರೋಮಾಂಚಕಾರಿ ಅನುಭವಗಳನ್ನೆಲ್ಲ ಬಣ್ಣಿಸಿದರು. ಆಹಾರ ಸಿಗದೆ ಅವರ ಉಪವಾಸ ಬಿದ್ದ ದಿನಗಳಿಗಂತೂ ಲೆಕ್ಕವೇ ಇಲ್ಲ. ಒಂದು ಸಲವಂತೂ ನಿರಂತರ ಉಪವಾಸದ ಬಳಿಕ ಖಾರದ ಚಟ್ಣಿಯನ್ನು ತಿಂದು, ಹೊಟ್ಟೆಯಲ್ಲಿ ಉಂಟಾದ ಉರಿ ಯಿಂದ ಪ್ರಾಣವೇ ಹೋಗುವಂತಾಗಿತ್ತು! (ಈ ಘಟನೆಯ ವಿವರವನ್ನು ಎಂಟನೆಯ ಅಧ್ಯಾಯ ದಲ್ಲಿ ವಿವರವಾಗಿ ತಿಳಿಸಲಾಗಿದೆ.) ಎಷ್ಟೋ ಸಲ ಭಿಕ್ಷೆಗಾಗಿಯೋ, ಉಳಿದುಕೊಳ್ಳಲು ಜಾಗವನ್ನು ಅರಸುತ್ತಲೋ ಹೋದರೆ, ಅವರ ಮುಖಕ್ಕೆ ರಾಚುವಂತೆ ಬಾಗಿಲು ಬಡಿದು ಜನ ಹೇಳಿದ್ದುಂಟು: “ಕಳ್ಳರು-ಸಂನ್ಯಾಸಿಗಳಿಗೆಲ್ಲ ಇಲ್ಲಿ ಜಾಗವಿಲ್ಲ.” ಕೆಲವೊಮ್ಮೆ ಅವರನ್ನು ಕ್ರಾಂತಿಕಾರಿಗಳೆಂದು ಅನುಮಾನಿಸಿ ಗುಪ್ತ ಪೋಲಿಸರು ನೆರಳಿನಂತೆ ಹಿಂಬಾಲಿಸಿದ್ದರು. ಹೀಗೆ ಅವರ ಪರಿವ್ರಾಜಕ ಜೀವನದಲ್ಲಿ ಇಂತಹ ಅನುಭವಗಳು ಅದೆಷ್ಟೋ! ಇದನ್ನೆಲ್ಲ ಅವರು ರಸವತ್ತಾಗಿ ಬಣ್ಣಿಸುತ್ತಿ ದ್ದಂತೆ ಹರಿಪದ ಬಾಬುವಿಗೆ ರಕ್ತಹೆಪ್ಪುಗಟ್ಟಿದಂತಹ ಅನುಭವ! ಆದರೆ ಸ್ವಾಮೀಜಿ ಇವೆಲ್ಲ ದೊಡ್ಡ ತಮಾಷೆಯೋ ಎಂಬಂತೆ ನಗುನಗುತ್ತ ಹೇಳುತ್ತಿದ್ದರು. ಅವರಿಗೆ ಇವೆಲ್ಲ ‘ಜಗನ್ಮಾತೆಯ ಲೀಲೆ’.

ಹೀಗೆ ಹರಿಪದ ಮಿತ್ರನ ಮನೆಯಲ್ಲಿ ಎರಡು ದಿನಗಳನ್ನು ಕಳೆದ ಮೇಲೆ, ಅಲ್ಲಿಂದ ಹೊರಡಲು ಸ್ವಾಮೀಜಿ ಅಣಿಯಾದರು. ಆಗ ಹರಿಪದ ಮಿತ್ರ ಅವರನ್ನು ತುಂಬ ಒತ್ತಾಯಿಸಿ, ಬಹಳವಾಗಿ ಕೇಳಿಕೊಂಡು, ಇನ್ನೂ ಕೆಲವು ದಿನ ಉಳಿದುಕೊಳ್ಳುವಂತೆ ಒಪ್ಪಿಸಿದ. ಮತ್ತೆ ಕೆಲವು ದಿನಗಳ ಕಾಲ ಅವರ ನಡುವಣ ಮಾತುಕತೆ ಮುಂದುವರಿಯಿತು. ಅತ್ಯಂತ ರಸವತ್ತಾದ ಈ ದಿನಗಳ ತನ್ನ ಅನುಭವವನ್ನು ಆತ ಸ್ಮೃತಿಲೇಖನವೊಂದರಲ್ಲಿ ಬರೆದಿದ್ದಾನೆ.

ಒಂದು ದಿನ ಸ್ವಾಮೀಜಿ, ಯಾವುದೋ ಸಂದರ್ಭದಲ್ಲಿ ‘ಪಿಕ್​ವಿಕ್ ಪೇಪರ್ಸ್​’ ಎಂಬ ಗ್ರಂಥದ ಕೆಲವು ಪುಟಗಳನ್ನು ಹಾಗೆಹಾಗೆಯೇ ಉದ್ಧರಿಸಿದರು. ಇದನ್ನು ಕಂಡು ಹರಿಪದನ ಆಶ್ಚರ್ಯಕ್ಕೆ ಪಾರವೇ ಇಲ್ಲ. ಅದೇನೂ ಉಪನಿಷತ್ತೂ ಅಲ್ಲ, ಗೀತೆಯೂ ಅಲ್ಲ; ಮನರಂಜಕ ವಾದ ಒಂದು ಇಂಗ್ಲಿಷ್ ಕಾದಂಬರಿ! ಸ್ವಾಮೀಜಿ ಅದನ್ನು ಮತ್ತೆ ಮತ್ತೆ ಓದಿರಬೇಕೆಂದು ಊಹಿಸಿ, “ಎಷ್ಟುಸಲ ಓದಿದ್ದೀರಿ ಅದನ್ನು?” ಎಂದು ಪ್ರಶ್ನಿಸಿದ. “ಒಟ್ಟು ಎರಡು ಸಲ; ಒಮ್ಮೆ ನಾನು ಶಾಲೆಯಲ್ಲಿ ಓದುತ್ತಿದ್ದಾಗ, ಇನ್ನೊಮ್ಮೆ ಈಗ ಆರು ತಿಂಗಳ ಕೆಳಗೆ” ಎಂದರು ಸ್ವಾಮೀಜಿ. ಅದನ್ನು ಕೇಳಿ ಅವನ ಆಶ್ಚರ್ಯ ಇನ್ನೂ ಹೆಚ್ಚಾಯಿತು. “ಹಾಗಾದರೆ, ಅದೆಲ್ಲ ನಿಮಗೆ ಮಾತ್ರ ನೆನಪಿರುತ್ತದೆ, ನಮಗೇಕೆ ನೆನಪಿರುವುದಿಲ್ಲ?” ಆಗ ಸ್ವಾಮೀಜಿ ವಿವರಿಸುತ್ತಾರೆ, “ದೈಹಿಕ ಶಕ್ತಿಗಳ ಮೇಲೆ ಹಿಡಿತ ತಂದುಕೊಳ್ಳುವುದರಿಂದ ಮನಶ್ಶಕ್ತಿ ಸಿದ್ಧಿಸುತ್ತದೆ. ದೈಹಿಕ ಶಕ್ತಿಗಳನ್ನು ವ್ಯರ್ಥವಾಗಿ ಸದೆ, ಮಾನಸಿಕ ಶಕ್ತಿಯಾಗಿ ಪರಿವರ್ತಿಸಬೇಕು. ಗೊತ್ತುಗುರಿಯಿಲ್ಲದೆ ನಿರ್ಲಕ್ಷ್ಯದಿಂದ ದೈಹಿಕ ಶಕ್ತಿಯನ್ನು ಭೋಗಲಾಲಸೆಗಾಗಿ ಬಳಸುವುದು ಅತ್ಯಂತ ಅಪಾಯಕಾರಿ. ಇದರಿಂದ ಮನಸ್ಸಿನ ಧಾರಣಶಕ್ತಿಯೇ ನಾಶವಾಗಿ ಹೋಗುತ್ತದೆ. ಅಲ್ಲದೆ, ನಾವು ಯಾವ ಕೆಲಸವನ್ನೇ ಮಾಡುವಾಗಲೂ ಅದಕ್ಕೆ ನಮ್ಮ ಹೃನ್ಮನಗಳನ್ನು ಸಂಪೂರ್ಣವಾಗಿ ಅರ್ಪಿಸಬೇಕು. ನಾನು ಹಿಂದೆ ಪವಾಹಾರಿ ಬಾಬಾ ಎಂಬ ಮಹಾ ಸಂತರನ್ನು ಭೇಟಿಯಾಗಿದ್ದೆ. ಅವರು ಜಪಧ್ಯಾನಗಳನ್ನು ಎಷ್ಟು ನಿಷ್ಠೆಯಿಂದ ಮಾಡುತ್ತಿದ್ದರೋ ಅಷ್ಟೇ ನಿಷ್ಠೆಯಿಂದ ತಮ್ಮ ಹಿತ್ತಾಳೆಯ ಪಾತ್ರೆಗಳನ್ನೂ ತೊಳೆದು, ಅದನ್ನು ಚಿನ್ನದಂತೆ ಹೊಳೆಯುವ ಹಾಗೆ ಮಾಡುತ್ತಿದ್ದರು.”

ಕೆಲವೊಮ್ಮೆ ಸ್ವಾಮೀಜಿ ತಮ್ಮ ಅಪೂರ್ವ ಹಾಸ್ಯದ ಮೂಲಕವೂ ಉನ್ನತ ವಿಷಯಗಳನ್ನು ತಿಳಿಸಿಕೊಡುತ್ತಿದ್ದರು. ಅತ್ಯಂತ ಗಂಭೀರವಾದ ಆಧ್ಯಾತ್ಮಿಕ ವಿಷಯಗಳ ಬಗ್ಗೆ ಮಾತನಾಡು ವಾಗಲೂ ಬಾಲಕನಂತೆ ತಮಾಷೆ ಮಾಡುತ್ತ ನಗುನಗುತ್ತಿದ್ದರು. ತಾವೂ ನಕ್ಕು ಇತರರನ್ನೂ ನಗಿಸುತ್ತಿದ್ದರು. ಅನಂತರ, ಇದ್ದಕ್ಕಿದ್ದಂತೆ ಗಂಭೀರಭಾವ ತಾಳಿ, ಸೂಕ್ಷ್ಮ ವಿಚಾರವೊಂದನ್ನು ವಿವರಿಸತೊಡಗುತ್ತಿದ್ದರು. ವಿಷಯದ ಮೇಲೂ, ಮತ್ತು ಸ್ವತಃ ತಮ್ಮ ಮೇಲೂ ಅವರಿಗಿರುವ ಪ್ರಭುತ್ವವನ್ನು ಕಂಡು ಜನ ನಿಬ್ಬೆರಗಾಗುತ್ತಿದ್ದರು. ‘ಈಗ ತಾನೇ ಇವರನ್ನು ನಮ್ಮಲ್ಲೇ ಒಬ್ಬರೆಂಬಂತೆ ಕಾಣಲಿಲ್ಲವೆ!’ ಎಂದು ಜನರು ಆಲೋಚಿಸುವಂತಾಗುತ್ತಿತ್ತು.

ಅವರನ್ನು ನೋಡಲು ಜನರು ಎಲ್ಲ ವೇಳೆಗಳಲ್ಲೂ ಬರುತ್ತಿದ್ದರು. ಒಬ್ಬೊಬ್ಬರ ಉದ್ದೇಶ ಒಂದೊಂದು. ಕೆಲವರು ಅವರ ಪುಜುತ್ವವನ್ನು, ಬುದ್ಧಿಶಕ್ತಿಯನ್ನು ಪರೀಕ್ಷೆ ಮಾಡಲು ಬರುತ್ತಿ ದ್ದರು; ಕೆಲವರು ನಕ್ಕು ನಲಿಸುವ ಅವರ ಮಾತುಗಳನ್ನು ಕೇಳಲು ಬರುತ್ತಿದ್ದರು; ಇನ್ನು ಕೆಲವರು ಸ್ವಾಮೀಜಿಯ ದರ್ಶನಾರ್ಥಿಗಳಾಗಿ ಬರುತ್ತಿದ್ದ ಊರಿನ ಶ್ರೀಮಂತರನ್ನು ಭೇಟಿ ಮಾಡಲು ಬರು ತ್ತಿದ್ದರು; ಮತ್ತೆ ಕೆಲವರು ಆಧ್ಯಾತ್ಮಿಕ ಸಂಭಾಷಣೆಗಳನ್ನು ಕೇಳಿ ಆನಂದಿಸಿ, ಧನ್ಯರಾಗಲು ಬರು ತ್ತಿದ್ದರು. ಅವರನ್ನು ನೋಡಿದೊಡನೆಯೇ ಸ್ವಾಮೀಜಿ, ಅವರ ಉದ್ದೇಶವನ್ನು ಗ್ರಹಿಸುತ್ತಿದ್ದ ರೀತಿ ತುಂಬ ಅದ್ಭುತವಾಗಿತ್ತು. ಸ್ವತಃ ತನಗೂ ಈ ಅನುಭವವಾಯಿತೆಂದು ಹರಿಪದ ಬರೆಯುತ್ತಾನೆ.

ಸ್ವಾಮೀಜಿಯ ಬಳಿಗೆ ಶ್ರೀಮಂತ ಮನೆತನದ ಯುವಕನೊಬ್ಬ ಬರುತ್ತಿದ್ದ. ಇವನಿಗೆ ಸಂನ್ಯಾಸ ಸ್ವೀಕಾರ ಮಾಡಬೇಕೆಂಬ ಮನಸ್ಸಾಗಿತ್ತು. ಇವನ ತಂದೆ ಹರಿಪದನ ಸ್ನೇಹಿತ. ಆದ್ದರಿಂದ ಹರಿಪದ ಕೇಳಿದ, “ಸ್ವಾಮೀಜಿ, ಅವನಿಗೆ ನೀವು ಸಂನ್ಯಾಸಿಯಾಗುವಂತೆ ಹೇಳಿದ್ದೀರಾ?” ಸ್ವಾಮೀಜಿ ಉತ್ತರಿಸಿದರು, “ಏನಿಲ್ಲ, ಅವನ ಕಾಲೇಜು ಪರೀಕ್ಷೆ ಹತ್ತಿರ ಬರುತ್ತಿದೆ. ಆದ್ದರಿಂದ ಅವನು ಪರೀಕ್ಷೆಯಿಂದ ತಪ್ಪಿಸಿಕೊಳ್ಳುವುದಕ್ಕೋಸ್ಕರ ‘ಸಂನ್ಯಾಸಿಯಾಗುತ್ತೇನೆ’ ಎನ್ನುತ್ತಿದ್ದಾನೆ, ಅಷ್ಟೆ. ನಾನವನಿಗೆ ಹೇಳಿದೆ–‘ನೀನು ಎಂ. ಎ. ಪಾಸು ಮಾಡಿಕೊಂಡು ಬಂದು ಸಂನ್ಯಾಸಿಯಾಗು; ಸಂನ್ಯಾಸ ಜೀವನ ನಡೆಸುವುಕ್ಕಿಂತ ಎಂ. ಎ. ಪದವಿ ಪಡೆಯುವುದು ಸುಲಭ’ ಅಂತ.”

ಹರಿಪದನಿಗೆ ಕೈತುಂಬ ಸಂಬಳ ಬರುವ ಉದ್ಯೋಗವಿತ್ತು. ಆದರೆ ಕಛೇರಿಯಲ್ಲಿ ಮೇಲಧಿಕಾರಿ ಗಳೊಂದಿಗೆ ಅವನ ಸಂಬಂಧ ಅಷ್ಟು ಚೆನ್ನಾಗಿರಲಿಲ್ಲ. ಮೇಲಧಿಕಾರಿಗಳು ತನ್ನ ವಿಷಯವಾಗಿ ಒಂದು ಮಾತು ಅಂದರೂ, ಅವನ ಮನಸ್ಸಿನ ಸ್ತಿಮಿತ ತಪ್ಪಿಹೋಗುತ್ತಿತ್ತು. ಇದರಿಂದಾಗಿ ಕಛೇರಿ ಯಲ್ಲಿ ಒಂದು ದಿನವೂ ನೆಮ್ಮದಿಯಿಂದಿರಲು ಅವನಿಗೆ ಸಾಧ್ಯವಾಗುತ್ತಿರಲಿಲ್ಲ. ತನ್ನ ಈ ಪರಿಸ್ಥಿತಿ ಯನ್ನು ಅವನು ಸ್ವಾಮೀಜಿಯ ಮುಂದೆ ಹೇಳಿಕೊಂಡಾಗ ಅವರಾಡಿದ ಮಾತುಗಳು ಮನನ ಯೋಗ್ಯವಾಗಿವೆ:

“ನೀನು ಆ ಕೆಲಸದಲ್ಲಿರುವುದೇಕೆ? ನೀನು ಪಡೆಯುವ ಸಂಬಳಕ್ಕಾಗಿ ತಾನೆ? ನಿನಗೆ ತಿಂಗಳು ತಿಂಗಳೂ ಸರಿಯಾಗಿ ಸಂಬಳ ಬರುತ್ತಿದೆ ಎಂದಮೇಲೆ ನೀನೇಕೆ ತಲೆ ಕೆಡಿಸಿಕೊಳ್ಳಬೇಕು? ಸಣ್ಣ ಪುಟ್ಟ ವಿಷಯಗಳ ಬಗೆಗೆಲ್ಲ ಮನಸ್ಸಿನ ಸ್ತಿಮಿತವನ್ನು ಕಳೆದುಕೊಂಡು, ‘ಅಯ್ಯೋ! ನಾನೆಂಥ ಬಂಧನದಲ್ಲಿ ಸಿಕ್ಕಿಹಾಕಿಕೊಂಡಿದ್ದೇನೆ!’ ಎಂದು ಕೊರಗುತ್ತ, ನಿನ್ನ ದುಃಖವನ್ನು ಇನ್ನಷ್ಟು ಹೆಚ್ಚಿಸಿ ಕೊಳ್ಳುತ್ತಿದ್ದೀಯೆ, ಅಷ್ಟೆ. ನಿನ್ನನ್ನು ಯಾರೂ ಹಿಡಿದಿಟ್ಟಿಲ್ಲ. ನಿನಗೆ ಇಷ್ಟ ಬಂದಾಗ ರಾಜೀನಾಮೆ ಕೊಡುವ ಸ್ವಾತಂತ್ರ್ಯ ನಿನಗಿದೆಯಲ್ಲ! ಹೀಗಿರುವಾಗ ನೀನು ಇನ್ನೊಬ್ಬರನ್ನು ದೂರವುದೇತಕ್ಕೆ? ಅಲ್ಲದೆ, ಇನ್ನೊಂದು ವಿಷಯ–ಮೇಲಧಿಕಾರಿಗಳು ನಿನ್ನ ವಿಷಯವಾಗಿ ಅಸಮಾಧಾನ ತಾಳಿರುವು ದಾಗಿ ಹೇಳುತ್ತೀಯಲ್ಲ, ನೀನು ಸಂಬಳಕ್ಕೋಸ್ಕರ ಕೆಲಸ ಮಾಡುವುದಲ್ಲದೆ, ನಿನ್ನ ಮೇಲಧಿಕಾರಿ ಗಳಿಗೆ ತೃಪ್ತಿಯಾಗುವಂಥದ್ದೇನು ಮಾಡಿದ್ದೀಯೆ? ಏನೂ ಇಲ್ಲ! ಹೀಗಿರುವಾಗ ನೀನು ಅವರ ಮೇಲೆ ಕೋಪಿಸಿಕೊಂಡಿರುವುದು ಸರಿಯೇ? ನೀನೀಗ ಅಸಹಾಯಕ ಪರಿಸ್ಥಿತಿಯಲ್ಲಿದ್ದೀಯೆಂದು ಭಾವಿಸುವುದಾದರೆ ಅದಕ್ಕೆ ಕಾರಣ ನೀನೇ, ಬೇರೆಯವರಲ್ಲ. ನೀನು ರಾಜೀನಾಮೆ ಕೊಟ್ಟರೆ ನಿನ್ನ ಮೇಲಧಿಕಾರಿಗಳಿಗೆ ಅದರಿಂದೇನು? ನಿನ್ನ ಜಾಗಕ್ಕೆ ಬರಲು ನೂರಾರು ಜನ ಸಿದ್ಧವಾಗಿದ್ದಾರೆ. ನಿನ್ನ ಜವಾಬ್ದಾರಿಗಳ ಕಡೆಗೆ ಸಂಪೂರ್ಣವಾಗಿ ಗಮನ ಕೊಡಬೇಕಾದದ್ದಷ್ಟೇ ನಿನ್ನ ಕರ್ತವ್ಯ. ನೀನು ಒಳ್ಳೆಯವನಾಗಿದ್ದರೆ ಇಡೀ ಜಗತ್ತೇ ನಿನಗೆ ಒಳ್ಳೆಯದಾಗಿ ಕಾಣಿಸುತ್ತದೆ. ಈ ಮಾತನ್ನು ಅರ್ಥ ಮಾಡಿಕೊಂಡವರು ಬಹಳ ಕಡಿಮೆ. ಇತರರ ಬಗ್ಗೆ ನಮಗಿರುವ ಅಭಿಪ್ರಾಯವೇ ಅವರೊಂದಿಗಿನ ನಮ್ಮ ವರ್ತನೆಯಲ್ಲಿ ಕಂಡುಬರುವುದು. ನಾವದನ್ನು ಬಾಯಿಬಿಟ್ಟು ಹೇಳದಿದ್ದರೂ ಇತರರು ಅದನ್ನು ಅರ್ಥಮಾಡಿಕೊಂಡು ಅದಕ್ಕೆ ತಕ್ಕಂತೆಯೇ ಪ್ರತಿಕ್ರಿಯಿಸುತ್ತಾರೆ. ಆದ್ದರಿಂದ, ಇತರರಲ್ಲಿ ತಪ್ಪು ಕಂಡುಹಿಡಿಯುವ ಅಭ್ಯಾಸವನ್ನು ಇಂದಿನಿಂದಲೇ ಬಿಟ್ಟುಬಿಡು. ನೀನಿದರಲ್ಲಿ ಎಷ್ಟರ ಮಟ್ಟಿಗೆ ಯಶಸ್ವಿಯಾಗುತ್ತೀಯೋ, ಇತರರ ದೃಷ್ಟಿಕೋನ ಹಾಗೂ ಪ್ರತಿಕ್ರಿಯೆಯೂ ಅಷ್ಟರ ಮಟ್ಟಿಗೆ ಬದಲಾಗುವುದನ್ನು ನೀನೇ ಕಂಡುಕೊಳ್ಳುತ್ತೀಯೆ.” ಈ ಮಾತುಗಳು ಹರಿಪದ ಮಿತ್ರನ ಮೇಲೆ ಎಂತಹ ಅಚ್ಚಳಿಯದ ಪರಿಣಾಮವನ್ನುಂಟುಮಾಡಿದುವೆಂದರೆ, ಅಂದಿನಿಂದ ಅವನ ಜೀವನಕ್ರಮವೇ ಬದಲಾಗಿಹೋಯಿತು.

ತಮ್ಮ ಪರಿವ್ರಾಜಕ ಜೀವನದ ದಿನಗಳಲ್ಲಿ ಸ್ವಾಮೀಜಿ ಹಣವನ್ನು ಮುಟ್ಟದೆ, ಸಂನ್ಯಾಸ ಜೀವನದ ನಿಯಮಗಳನ್ನು ಕಟ್ಟುನಿಟ್ಟಾಗಿ ಪರಿಪಾಲಿಸುತ್ತಿದ್ದರು. ಇದನ್ನು ಗಮನಿಸಿದವರೊಬ್ಬರು ಅವರನ್ನು ಕೇಳಿದರು, “ಸ್ವಾಮೀಜಿ, ತಮ್ಮಂತಹ ಪ್ರಚಂಡ ಇಚ್ಛಾಶಕ್ತಿಯುಳ್ಳವರಿಗೆ ಇಷ್ಟೊಂದು ನಿಯಮಪರಿಪಾಲನೆಯ ಆವಶ್ಯಕತೆಯೇನಿದೆ?” ಇದಕ್ಕೆ ಸ್ವಾಮೀಜಿ ಬಹಳ ಸೊಗಸಾಗಿ ಉತ್ತರಿಸಿದರು:

“ನೋಡಿಲ್ಲಿ, ಈ ಮನಸ್ಸು ಎನ್ನುವುದು ಎಂಥಾ ಹುಚ್ಚು ಸ್ವಭಾವದ್ದು, ಎಂಥಾ ಅಮಲಿನದ್ದು ಎಂದರೆ, ಸುಮ್ಮನೆ ಒಂದು ಕಡೆ ಕುಳಿತಿರಲು ಅದರಿಂದ ಸಾಧ್ಯವೇ ಇಲ್ಲ. ಸ್ವಲ್ಪ ಅವಕಾಶ ಸಿಕ್ಕರೆ ಸಾಕು, ಅದು ನಮ್ಮನ್ನು ಕೆಳಗೆಳೆಯುತ್ತದೆ. ಇಂತಹ ಮನಸ್ಸಿನ ಮೇಲೆ ಹಿಡಿತವನ್ನಿಟ್ಟುಕೊಳ್ಳಲು ಸಂನ್ಯಾಸಿಯಾದವನೂ ನಿಯಮ ಪರಿಪಾಲನೆ ಮಾಡಲೇಬೇಕು. ಎಲ್ಲರೂ ಏನು ತಿಳಿದುಕೊಂಡಿರು ತ್ತಾರೆಂದರೆ, ತಮಗೆ ತಮ್ಮ ಮನಸ್ಸಿನ ಮೇಲೆ ಸಂಪೂರ್ಣ ಹಿಡಿತ ಇದೆ; ಆದರೆ ತಾವಾಗಿಯೇ ಅದಕ್ಕೆ ಸ್ವಲ್ಪ ಸ್ವಾತಂತ್ರ್ಯ ಕೊಟ್ಟಿದ್ದೇವೆ, ಅಂತ. ಇದೊಂದು ಭ್ರಮೆ, ಆತ್ಮವಂಚನೆ. ಧ್ಯಾನ ಮಾಡಲು ಕುಳಿತುಕೊಂಡಾಗ ಗೊತ್ತಾಗುತ್ತದೆ, ನಿಜವಾಗಿ ನಮ್ಮ ಮನಸ್ಸಿನ ಮೇಲೆ ನಮಗೆ ಎಷ್ಟು ಹಿಡಿತ ಇದೆ, ಅಂತ. ಯಾವುದಾದರೊಂದು ವಿಷಯದ ಮೇಲೆ ಕೆಲಕಾಲ ಆಲೋಚನೆ ಮಾಡ ಬೇಕಾಗಿ ಬಂದರೂ, ಒಂದು ಹತ್ತು ನಿಮಿಷಕ್ಕಿಂತ ಹೆಚ್ಚಾಗಿ ಮನಸ್ಸನ್ನು ಬಿಡದೆ ಹಿಡಿದಿರಲು ಸಾಧ್ಯವಾಗುವುದಿಲ್ಲ. ಗಂಡಂದಿರೆಲ್ಲ ಏನು ತಿಳಿದುಕೊಂಡಿರುತ್ತಾರೆ ಎಂದರೆ, ತಮ್ಮ ಹೆಂಡಿರ ಮೇಲೆ ತಮಗಿರುವ ಪ್ರೀತಿಯಿಂದಾಗಿ ಮಾತ್ರವೇ ಅವರು ತಮ್ಮ ಮೇಲೆ ಅಧಿಕಾರ ಚಲಾಯಿಸಲು ಬಿಟ್ಟುಕೊಟ್ಟಿದ್ದೇವೆ ಎಂದು! ಮನಸ್ಸನ್ನು ಹಿಡಿತದಲ್ಲಿಟ್ಟುಕೊಂಡಿದ್ದೇನೆ ಎಂದು ಭಾವಿಸುವುದೂ ಹೀಗೆಯೇ. ಆದ್ದರಿಂದ, ನಾವು ನಮ್ಮ ಮನಸ್ಸನ್ನು ಗೆದ್ದಿದ್ದೇವೆ ಎಂಬ ಭ್ರಮೆಯಿಂದ ಅದರ ಮೇಲಿನ ಹಿಡಿತವನ್ನು ಎಂದಿಗೂ ಸಡಿಲ ಮಾಡಬಾರದು.”

ಒಂದು ದಿನ ಮಾತಿನ ಸಂದರ್ಭದಲ್ಲಿ ಹರಿಪದ ಬಾಬು ಹೇಳಿದ, “ನನಗನ್ನಿಸುತ್ತದೆ, ಧರ್ಮವನ್ನು ಅರ್ಥಮಾಡಿಕೊಳ್ಳಬೇಕಾದರೆ ಬಹಳಷ್ಟು ವಿದ್ಯಾವಂತರಾಗಿರಬೇಕು, ಎಂದು.” ಆಗ ಸ್ವಾಮೀಜಿ ಹೇಳಿದರು, “ಧರ್ಮವನ್ನು ಆಚರಣೆಗೆ ತರುವಷ್ಟರಮಟ್ಟಿಗೆ ಅರ್ಥ ಮಾಡಿಕೊಳ್ಳಲು ಉನ್ನತ ವಿದ್ಯಾಭ್ಯಾಸವೇನೂ ಬೇಕಿಲ್ಲ. ಆದರೆ ಇತರರಿಗೆ ವಿವರಿಸಿ ಹೇಳಬೇಕಾಗಿ ಬಂದಾಗ ಮಾತ್ರ ಬೇಕಾಗುತ್ತದೆ, ಅಷ್ಟೆ. ಶ್ರೀರಾಮಕೃಷ್ಣರು ತಮ್ಮ ಹೆಸರನ್ನು ‘ರಾಮ್​ಕೆಸ್ಟೊ’ ಎಂದು ಬರೆಯು ತ್ತಿದ್ದರು. ಆದರೆ ಧರ್ಮದ ಸಾರವನ್ನು ಅವರಿಗಿಂತ ಚೆನ್ನಾಗಿ ಅರಿತವರು ಯಾರಿದ್ದಾರೆ? ಧರ್ಮ ಸಿದ್ಧಿಸುವುದು ಪ್ರತ್ಯಕ್ಷಾನುಭವದಿಂದ. ‘ಕಡುಬಿನ ರುಚಿಗೆ ತಿಂದದ್ದೇ ಪ್ರಮಾಣ!’ ಧರ್ಮವನ್ನು ಸಾಕ್ಷಾತ್ಕರಿಸಿಕೊಳ್ಳಲು ಪ್ರಯತ್ನಿಸು. ಇಲ್ಲದಿದ್ದರೆ ನಿನಗೆ ಅದು ಅರ್ಥವಾಗುವುದಿಲ್ಲ.”

ಸ್ವಾಮೀಜಿಯನ್ನು ನೋಡುವ ಮೊದಲು ಹರಿಪದ ಬಾಬು ಭಾವಿಸಿದ್ದ, ಸಂನ್ಯಾಸಿಗಳು ಎಂದರೆ ಬಡಕಲು ಶರೀರದವರಾಗಿರಬೇಕು, ಸದಾ ಜೋಲು ಮೋರೆ ಹಾಕಿಕೊಂಡಿರಬೇಕು ಎಂದು. ಆದರೆ ಸ್ವಾಮೀಜಿ ಇದಕ್ಕೆ ತದ್ವಿರುದ್ಧ! ಅವರದು ಒಳ್ಳೇ ಕಟ್ಟುಮಸ್ತಾದ ಶರೀರ. ತುಂಬಿಕೊಂಡ ಅಂಗಾಂಗಳು; ಸುಂದರವಾದ ಹಸನ್ಮುಖ. ಇದು ನಿಜಕ್ಕೂ ವಿಚಿತ್ರ ಎನ್ನಿಸಿತು ಅವನಿಗೆ. ಆದ್ದ ರಿಂದ ಒಂದು ದಿನ ತಮಾಷೆ ಮಾಡುತ್ತ ತನ್ನ ಅನಿಸಿಕೆಯನ್ನು ಹೇಳಿಯೇಬಿಟ್ಟ. ಆಗ ಸ್ವಾಮೀಜಿ ತಮ್ಮ ಸ್ಥೂಲ ಶರೀರದ ಕಡೆಗೆ ಬೆರಳುಮಾಡಿ ತೋರಿಸುತ್ತ ಹೇಳುತ್ತಾರೆ, “ನೋಡು ಇದು ನನ್ನ ‘ಬರಗಾಲ ವಿಮಾ ನಿಧಿ\eng{’ (Famine Insurance Fund).} ನನಗೆ ದಿನಗಟ್ಟಲೆ ಆಹಾರ ಸಿಗದೆ ಹೋದರೂ, ನನ್ನ ಶರೀರದಲ್ಲಿರುವ ಕೊಬ್ಬಿನ ಅಂಶ ನನ್ನನ್ನು ಜೀವಂತವಾಗಿರಿಸುತ್ತದೆ. ನಿನಗಾ ದರೋ ಒಂದು ದಿನ ಹೊತ್ತಿಗೆ ಸರಿಯಾಗಿ ಊಟ ಬೀಳದಿದ್ದರೆ ಕಣ್ಣು ಮಬ್ಬಾಗುತ್ತದೆ. ಅಲ್ಲದೆ, ಯಾವ ಧರ್ಮ ಮನುಷ್ಯನಿಗೆ ಶಾಂತಿಯನ್ನು ತಂದುಕೊಡುವುದಿಲ್ಲವೋ, ಅದನ್ನು ಒಂದು ರೋಗದಂತೆ ದೂರವಿಡಬೇಕು.”

ಇಲ್ಲಿ ಸ್ವಾಮೀಜಿ, ಶಾಂತಿ ತಂದುಕೊಡದ ಧರ್ಮವನ್ನು ರೋಗಕ್ಕೆ ಹೋಲಿಸಿರುವುದರಲ್ಲಿ ಒಂದು ಅರ್ಥವಿದೆ. ಧರ್ಮವನ್ನು ಆಚರಿಸಬೇಕಾದರೆ ಶರೀರ ಸ್ಥಿರವಾಗಿರಬೇಕಾಗುತ್ತದೆ– ‘ಶರೀರಮಾದ್ಯಂ ಖಲು ಧರ್ಮಸಾಧನಂ.’ ಶರೀರ ಸ್ಥಿರವಾಗಿರಬೇಕಾದರೆ ಅದಕ್ಕೆ ಯೋಗ್ಯ ಆಹಾರ ಬೇಕಾಗುತ್ತದೆ. ಆದ್ದರಿಂದ ಯುಕ್ತ ಆಹಾರದಿಂದ ಶರೀರವನ್ನು ದೃಢವಾಗಿಟ್ಟುಕೊಂಡಾಗ ಮಾತ್ರ ಧರ್ಮವನ್ನು ಆಚರಿಸಬಹುದು. ಮತ್ತು ಧರ್ಮವನ್ನು ಆಚರಿಸುವುದರಿಂದ ಮಾತ್ರ ಮನುಷ್ಯನಿಗೆ ಶಾಂತಿ. ಆದ್ದರಿಂದ ಶಾಂತಿಯ ಮೂಲಕಾರಣವಾದ ಧರ್ಮವನ್ನು ಆಚರಿಸುವು ದರಲ್ಲಿ ದೃಢ ಶರೀರದ ಪಾತ್ರ, ಹಾಗೂ ಶರೀರವನ್ನು ದೃಢವಾಗಿರಿಸುವಲ್ಲಿ ಆಹಾರದ ಮಹತ್ವ– ಇವುಗಳನ್ನು ನಾವು ಕಡೆಗಣಿಸುವಂತಿಲ್ಲ. ಆದ್ದರಿಂದಲೇ ಮುಂದೆ ಸ್ವಾಮೀಜಿ ಸಾರುವವರಿದ್ದಾರೆ –‘ಹಸಿದ ಹೊಟ್ಟೆಗೆ ಧರ್ಮಬೋಧನೆ ಮಾಡುವುದೆಂದರೆ ಅದು ಪಾಷಂಡತನವೇ ಸರಿ’ ಎಂದು.

ಒಂದು ದಿನ ಧರ್ಮ ಹಾಗೂ ಮುಕ್ತಿಯ ಬಗ್ಗೆ ಮಾತನಾಡುವಾಗ ಸ್ವಾಮೀಜಿ ಹೇಳುತ್ತಾರೆ, “ಎಲ್ಲ ಜೀವಿಗಳೂ ಸುಖವನ್ನು ಹೊಂದುವುದಕ್ಕಾಗಿ ಸದಾ ಹಾತೊರೆಯುತ್ತಿರುತ್ತವೆ. ಅವು ಅದಕ್ಕಾಗಿ ಕೊನೆಯಿಲ್ಲದ ಪ್ರಯತ್ನ ಮಾಡುತ್ತಲೇ ಇರುತ್ತವೆ. ಆದರೆ ಅವು ತಮ್ಮ ಗುರಿಯನ್ನು ಎಂದಿಗೂ ಮುಟ್ಟುವಂತೆಯೇ ಕಾಣುವುದಿಲ್ಲ. ಆದರೂ ಬಹಳಷ್ಟು ಜನಕ್ಕೆ ತಾವು ಇಷ್ಟೊಂದಾಗಿ ಬಯಸುವ ಸುಖವನ್ನು ಪಡೆಯುವಲ್ಲಿ ತಾವೇಕೆ ವಿಫಲರಾಗಿದ್ದೇವೆ ಎಂದು ಕ್ಷಣಕಾಲ ನಿಂತು ಆಲೋಚಿಸುವಷ್ಟು ತಾಳ್ಮೆಯೂ ಇರುವುದಿಲ್ಲ. ಆದ್ದರಿಂದಲೇ ಅವರು ಸಂಕಟವನ್ನು ಅನುಭವಿಸು ವುದು. ಧರ್ಮದ ಬಗ್ಗೆ ಒಬ್ಬ ಮನುಷ್ಯನ ಕಲ್ಪನೆ ಎಂತಹದೇ ಆಗಿರಲಿ, ಅದರಿಂದ ತನಗೆ ನಿಜವಾದ ಸುಖ ಸಿಗುತ್ತಿದೆ ಎಂದು ಅವನು ನಂಬಿಕೊಂಡಿರುವವರೆಗೆ ಅದರಲ್ಲಿ ಅವನಿಗಿರುವ ಶ್ರದ್ಧೆಯನ್ನು ಯಾರೂ ಅಲುಗಾಡಿಸಲು ಹೋಗಬಾರದು. ಒಂದು ವೇಳೆ ಯಾರಾದರೂ ಅವನನ್ನು ತಿದ್ದುವ ಪ್ರಯತ್ನ ಮಾಡಿದರೂ, ಅವನೇ ಅದನ್ನು ಹೃತ್ಪೂರ್ವಕವಾಗಿ ಒಪ್ಪಿ ಸಹಕರಿಸದಿದ್ದರೆ, ಅದರಿಂದ ಯಾವ ಸತ್ಪರಿಣಾಮವೂ ಆಗುವುದಿಲ್ಲ... ಒಬ್ಬ ವ್ಯಕ್ತಿ ಏನೇ ಮಾತಾಡಲಿ, ಅವನಿಗೆ ಧಾರ್ಮಿಕ ವಿಚಾರಗಳನ್ನು ಕೇಳುವುದರಲ್ಲಿ ಮಾತ್ರ ಆಸಕ್ತಿಯಿದೆಯೇ ಹೊರತು ಅದನ್ನು ಅನು ಷ್ಠಾನ ಮಾಡಲು ಮನಸ್ಸಿಲ್ಲ ಎಂದು ಕಂಡುಬಂದರೆ, ಆಗ ಒಂದೇ ಸಲಕ್ಕೆ ತೀರ್ಮಾನಿಸಿಬಿಡ ಬಹುದು–ಅವನಿಗೆ ಯಾವುದರಲ್ಲೂ ನಿಶ್ಚಲ ಶ್ರದ್ಧೆಯಿಲ್ಲ ಎಂದು.”

ಬಳಿಕ ಮತ್ತೆ ಹೇಳುತ್ತಾರೆ, “ಮನುಷ್ಯನಿಗೆ ಶಾಂತಿಯನ್ನು ತಂದುಕೊಡುವುದೇ ಧರ್ಮದ ಮೂಲೋದ್ದೇಶ. ಮುಂದಿನ ಜನ್ಮದಲ್ಲಿ ಸುಖ ಸಿಕ್ಕೀತು ಎಂದು ಈ ಜನ್ಮದಲ್ಲಿ ನರಳುವುದು ಬುದ್ಧಿವಂತಿಕೆಯಲ್ಲ. ಇಲ್ಲೇ—ಈಗಲೇ ಶಾಂತಿ ದೊರಕುವಂತಾಗಬೇಕು. ಯಾವ ಧರ್ಮ ಇದನ್ನು ತಂದುಕೊಡಬಲ್ಲದೋ ಅದೇ ನಿಜವಾದ ಧರ್ಮ.

“ಎಷ್ಟಾದರೂ ಇಂದ್ರಿಯ ಸುಖ ಎನ್ನುವುದು ಕ್ಷಣಿಕ; ಅದು ದುಃಖದೊಂದಿಗೆ ಬೆರೆತು ಕೊಂಡೇ ಇರುತ್ತದೆ. ಕೇವಲ ಬಾಲಬುದ್ಧಿಯುಳ್ಳವರು, ಮೂರ್ಖರು ಮತ್ತು ಪ್ರಾಣಿಗಳು ಮಾತ್ರವೇ ಇದನ್ನೇ ನಿಜವಾದ ಆನಂದ ಎಂದು ಭ್ರಮಿಸಲು ಸಾಧ್ಯ. ಆದರೂ ಇಂತಹ ಸುಖವನ್ನೇ ಜೀವನದ ಸಾರಸರ್ವಸ್ವ ಎಂದು ಭಾವಿಸಿ, ಅದಕ್ಕೇ ಅಂಟಿಕೊಂಡು, ಅದರಿಂದಲೇ ತಾನು ನಿರಂತರ ಸುಖವನ್ನೂ ಪಡೆಯುತ್ತೇನೆ, ದುಃಖದಿಂದ ಬಿಡುಗಡೆಯನ್ನೂ ಪಡೆಯುತ್ತೇನೆ ಎನ್ನುವುದಾದರೆ, ಅದಕ್ಕೆ ನನ್ನ ಅಭ್ಯಂತರವೇನೂ ಇಲ್ಲ. ಆದರೆ ಅಂತಹ ವ್ಯಕ್ತಿಯೊಬ್ಬನನ್ನು ನಾನಿನ್ನೂ ಕಾಣಬೇಕಾ ಗಿದೆ!... ಅಲ್ಲದೆ ಇಂದ್ರಿಯಸುಖವೇ ಅತ್ಯಂತಿಕ ಆನಂದವೆಂದು ಭಾವಿಸುವವರು ತಮಗಿಂತ ಹೆಚ್ಚು ಭೋಗಿಗಳಾದವರನ್ನು, ಶ್ರೀಮಂತರನ್ನು ಕಂಡು ಅಸೂಯೆ ಪಡುತ್ತಾರೆ. ಇಡೀ ಜಗತ್ತನ್ನೇ ಗೆದ್ದ ಅಲೆಗ್ಸಾಂಡರ್, ತನಗೆ ಗೆಲ್ಲಲು ಇನ್ನಾವ ದೇಶವೂ ಇಲ್ಲವಲ್ಲ ಎಂದು ದುಃಖಿಸಿದನಂತೆ! ಆದ್ದರಿಂದಲೇ ಕೆಲವು ಚಿಂತನಶೀಲರು ಸುದೀರ್ಘ ಅನುಭವದಿಂದ ಎಲ್ಲವನ್ನೂ ವಿಶ್ಲೇಷಿಸಿ, ಕೊನೆಗೊಂದು ತೀರ್ಮಾನಕ್ಕೆ ಬಂದಿದ್ದಾರೆ. ಏನೆಂದರೆ, ಯಾವುದಾದರೊಂದು ಧರ್ಮದಲ್ಲಿ ಸಂಪೂರ್ಣ ಶ್ರದ್ಧೆಯುಂಟಾದಾಗ ಮಾತ್ರವೇ ಮನುಷ್ಯನಿಗೆ ನಿಜವಾದ ಸುಖ ಹಾಗೂ ದುಃಖ ವಿಮುಕ್ತಿ ಸಾಧ್ಯ ಎಂದು.”

ಒಂದು ದಿನ ಸ್ವಾಮೀಜಿ ಹರಿಪದ ಮಿತ್ರನ ಮುಂದೆ, ತಾವು ಶಿಕಾಗೋದಲ್ಲಿ ನಡೆಯಲಿರುವ ವಿಶ್ವಧರ್ಮ ಸಮ್ಮೇಳನದಲ್ಲಿ ಭಾಗವಹಿಸಲು ಅಮೆರಿಕೆಗೆ ಹೋಗಬೇಕೆಂದಿರುವುದಾಗಿ ಹೇಳಿದರು. ಅವರು ಈ ವಿಷಯವಾಗಿ ಎಷ್ಟು ಉಜ್ವಲವಾಗಿ ಮಾತನಾಡಿದರೆಂದರೆ ಪುರಾತನ ಪುಷಿ-ಮುನಿಗಳ ಹೃದಯ ಅದರಲ್ಲಿ ಒಂದಾಗಿ ಪ್ರಜ್ವಲಿಸುವಂತೆ ಕಾಣುತ್ತಿತ್ತು. ಅವರ ಮಾತಿನಲ್ಲಿ ದಾರ್ಶನಿಕನ ಮುನ್ನೋಟ ಕಾಣುತ್ತಿತ್ತು. ಅದನ್ನು ಕೇಳುತ್ತಿದ್ದಂತೆ ಹರಿಪದನಲ್ಲಿ ಉತ್ಸಾಹವುಕ್ಕಿತು. “ಹಾಗಾದರೆ ಸ್ವಾಮೀಜಿ, ಈಗಲೇ ಹಣ ಸಂಗ್ರಹ ಕಾರ್ಯ ಪ್ರಾರಂಭಿಸುತ್ತೇನೆ” ಎಂದು ಮುಂದೆ ಬಂದ. ಆದರೆ ಸ್ವಾಮೀಜಿ ಒಪ್ಪಿಕೊಳ್ಳಲಿಲ್ಲ. ಅವರು ಹೇಳಿದರು. “ಈಗಲೇ ಬೇಡ. ನಾನೀಗ ಮೊದಲು ರಾಮೇಶ್ವರಕ್ಕೆ ಹೊರಡಬೇಕು. ನಾನಲ್ಲಿಗೆ ಹೋಗುವ ವ್ರತತೊಟ್ಟಿದ್ದೇನೆ.”

ಸ್ವಾಮೀಜಿ ಬೆಳಗಾವಿಗೆ ಬರುವುದಕ್ಕೆ ಮೊದಲು ಹರಿಪದನ ಪತ್ನಿ ಇಂದುಮತಿ, ಯಾರಾದ ರೊಬ್ಬರು ಗುರುಗಳಿಂದ ಮಂತ್ರದೀಕ್ಷೆ ಪಡೆಯಬೇಕೆಂಬ ಇಚ್ಛೆಯನ್ನು ವ್ಯಕ್ತಪಡಿಸಿದ್ದಳು. ಆಗ ಹರಿಪದ ಅವಳಿಗೆ ಹೇಳಿದ್ದ, “ನೀನು ಮಂತ್ರದೀಕ್ಷೆಯನ್ನು ತೆಗೆದುಕೊಳ್ಳುವುದಾದರೆ ನನ್ನ ಗೌರವಕ್ಕೂ ಪಾತ್ರರಾಗಬಲ್ಲ ಗುರುಗಳೊಬ್ಬರನ್ನು ಆರಿಸಿಕೊ. ಇಲ್ಲವೆ, ನಿನ್ನ ಆ ಗುರುಗಳು ಮನೆ ಯೊಳಗೆ ಕಾಲಿಡುತ್ತಿದ್ದಂತೆಯೇ ನನ್ನ ಮನಸ್ಸಿಗೆ ಕಿರಿಕಿರಿಯಾಗುವಂತಿದ್ದರೆ ಆಗ ನಿನಗೆ ಸಂತೋಷವೂ ಇರುವುದಿಲ್ಲ, ಪ್ರಯೋಜನವೂ ಆಗುವುದಿಲ್ಲ. ನಿಜಕ್ಕೂ ಉತ್ತಮರಾದ ಗುರು ಗಳೊಬ್ಬರು ಸಿಕ್ಕಿದರೆ ಮಾತ್ರ ನಾನೂ ನೀನೂ ಇಬ್ಬರೂ ಖಂಡಿತವಾಗಿ ಅವರಿಂದ ಮಂತ್ರದೀಕ್ಷೆ ಪಡೆದುಕೊಳ್ಳೋಣ. ಇಲ್ಲದೆ ಹೋದರೆ ಬೇಡವೇ ಬೇಡ.” ಈಗ ಅವರ ಗುರುಗಳಾಗಬಹುದಾದ ವರು ಪ್ರತ್ಯಕ್ಷವಾಗಿ ಇಲ್ಲಿಯೇ ಇದ್ದಾರೆ. ಅವರ ವ್ಯಕ್ತಿತ್ವ ಒಪ್ಪಿಗೆಯಾಗಿದೆ. ಅವರ ಮಹಿಮೆ ಮನಸ್ಸಿಗೆ ಬಂದಿದೆ. ಇದೆಲ್ಲ ಸರಿ. ಮಂತ್ರದೀಕ್ಷೆ ಕೊಡಲು ಸ್ವಾಮೀಜಿ ಒಪ್ಪಿಯಾರೆ? ಎಂಬುದೇ ಪ್ರಶ್ನೆ. ಹರಿಪದ ಬಾಬು ಪತ್ನಿಯನ್ನು ಕೇಳಿದ, “ಸ್ವಾಮೀಜಿಯಿಂದ ಮಂತ್ರಿದೀಕ್ಷೆ ಪಡೆಯಲು ಇಷ್ಟವಿದೆಯೆ?”

“ಇಷ್ಟವೇನೋ ಖಂಡಿತ ಇದೆ. ಆದರೆ ಅವರು ನಮ್ಮ ಗುರುವಾಗಲು ಒಪ್ಪುತ್ತಾರೆಯೆ? ಒಪ್ಪಿದರೆ ನಮ್ಮ ಭಾಗ್ಯವೇ ಸರಿ.”

“ಪ್ರಯತ್ನಮಾಡಿ ನೋಡೋಣ; ಇದನ್ನು ಬಿಟ್ಟರೆ ಇಂತಹ ಅವಕಾಶ ಮತ್ತೆಂದೂ ಸಿಗಲಾರದು.”

ಬಳಿಕ ಹರಿಪದ ಬಾಬು ತುಂಬ ಹಿಂಜರಿಯುತ್ತಲೇ ಸ್ವಾಮೀಜಿಯ ಬಳಿ ವಿಷಯವನ್ನು ಪ್ರಸ್ತಾಪಿಸಿದ. ಆದರೆ ಸ್ವಾಮೀಜಿ ಮೊದಲು ಇದಕ್ಕೆ ಒಪ್ಪದೆ ಹಲವಾರು ನೆಪಗಳನ್ನು ಮುಂದೊಡ್ಡಿ ಆ ಹೊಣೆಗಾರಿಕೆಯಿಂದ ತಪ್ಪಿಸಿಕೊಳ್ಳಲು ಪ್ರಯತ್ನಿಸಿದರು. ಆದರೆ ಅವನು ಹಿಡಿದ ಪಟ್ಟನ್ನು ಬಿಟ್ಟರೆ ತಾನೆ? ಅವನ ಅವಿಚಲ ನಿರ್ಧಾರವನ್ನು ಕಂಡ ಸ್ವಾಮೀಜಿ ಅವನಿಗೂ ಅವನ ಪತ್ನಿಗೂ ಮಂತ್ರದೀಕ್ಷೆಯನ್ನು ಅನುಗ್ರಹಿಸಿದರು. ದಂಪತಿಗಳು ಕೃತಾರ್ಥ ಭಾವದಿಂದ ಆನಂದಭರಿತರಾದರು.

ಈಗ ಸ್ವಾಮೀಜಿ ತಾವು ಸಂಕಲ್ಪ ಮಾಡಿದ್ದಕ್ಕಿಂತ ಹೆಚ್ಚು ದಿನ ಬೆಳಗಾವಿಯಲ್ಲಿ ಉಳಿದುಕೊಂಡಿ ದ್ದಾರೆ. ಇನ್ನು ಅಲ್ಲಿ ಉಳಿದುಕೊಳ್ಳಲು ಅವರ ಮನಸ್ಸು ಒಪ್ಪುತ್ತಿಲ್ಲ. ಆದ್ದರಿಂದ ಅವರು ಹರಿಪದನಿಗೆ ಹೇಳಿದರು, “ನೋಡು, ನಾನಿನ್ನು ಹೊರಡಲೇಬೇಕು. ಈಗ ನಾನು ರಾಮೇಶ್ವರಕ್ಕೆ ಹೋಗಬೇಕು. ನಾನು ಇದೇ ಗತಿಯಲ್ಲಿ ಮುಂದುವರಿಯುತ್ತಿದ್ದರೆ ಅಲ್ಲಿಗೆ ಎಂದಿಗೂ ತಲುಪು ವುದೇ ಇಲ್ಲ, ಅಷ್ಟೆ.” ಅವರನ್ನು ಮತ್ತಷ್ಟು ದಿನ ತನ್ನ ಮನೆಯಲ್ಲಿ ಉಳಿದುಕೊಳ್ಳುವಂತೆ ಮಾಡುವ ಹರಿಪದನ ಪ್ರಯತ್ನ ಫಲಕಾರಿಯಾಗಲಿಲ್ಲ. ಕೊನೆಗೆ ಅವನು ಸ್ವಾಮೀಜಿಗೆ ಮರ್ಮ ಗೋವಾಗೆ ಹೋಗಲು ರೈಲು ಟಿಕೇಟನ್ನು ತಂದುಕೊಟ್ಟ. ಅವರನ್ನು ಟ್ರೈನಿನಲ್ಲಿ ಕುಳ್ಳಿರಿಸಿ ಅವರಿಗೆ ಸಾಷ್ಟಾಂಗ ಪ್ರಣಾಮ ಮಾಡಿ ಹೇಳುತ್ತಾನೆ, “ಸ್ವಾಮೀಜಿ, ನಾನು ನನ್ನ ಜೀವನದಲ್ಲೇ ಯಾರಿಗೂ ಹೃತ್ಪೂರ್ವಕವಾಗಿ ನಮಸ್ಕರಿಸಿದವನಲ್ಲ. ಈಗ ನಿಮಗೆ ಪ್ರಣಾಮ ಸಲ್ಲಿಸಿ ನಾನು ನಿಜಕ್ಕೂ ಧನ್ಯ ನಾದೆ.” ಸ್ವಾಮೀಜಿ ತಮ್ಮ ಶಿಷ್ಯನನ್ನು ಹೃದಯತುಂಬಿ ಹರಸಿ ಅವನಿಂದ ಬೀಳ್ಕೊಂಡರು.

ಸ್ವಾಮೀಜಿ ಬೆಳಗಾವಿಯಿಂದ ಹೊರಟ ಮೇಲೆ ನಾವು ಅವರನ್ನು ಕಾಣುವುದು ಮರ್ಮಗೋವಾ ದಲ್ಲಿ. ಅವರು ಬೆಳಗಾವಿಯಲ್ಲಿದ್ದಾಗಲೇ ಅಲ್ಲಿನ ಗಣ್ಯ ನಾಗರಿಕರಾದ ಡಾ ॥ ವಿ. ವಿ. ಶಿರಗಾಂವ್ ಕರ್ ಎಂಬವರ ಮುಂದೆ, ಗೋವಾಕ್ಕೆ ಭೇಟಿ ನೀಡಬೇಕೆಂಬ ತಮ್ಮ ಇಚ್ಛೆಯನ್ನು ವ್ಯಕ್ತಪಡಿಸಿ ದ್ದರು. ಇದೊಂದು ಸಾಮಾನ್ಯ ಭೇಟಿಯಾಗಿರದೆ ಇದರ ಹಿಂದೆ ಒಂದು ವಿಶೇಷ ಉದ್ದೇಶವಿತ್ತು. ಡಾ ॥ ಶಿರಗಾಂವ್​ಕರ್ ಅವರು ಗೋವಾದಲ್ಲಿದ್ದ ತಮ್ಮ ಸ್ನೇಹಿತರಾದ ಸುಬ್ರಾಯ ನಾಯಕ್ ಎಂಬವರಿಗೆ ಪತ್ರ ಬರೆದು, ಸ್ವಾಮೀಜಿ ಅಲ್ಲಿಗೆ ಬಂದಾಗ ಅವರಿಗೆ ಎಲ್ಲ ಬಗೆಯ ಅನುಕೂಲತೆ ಗಳನ್ನು ಒದಗಿಸಿಕೊಡುವಂತೆ ಕೋರಿದ್ದರು.

ಸ್ವಾಮೀಜಿ ಮರ್ಮಗೋವಾ ರೈಲುನಿಲ್ದಾಣದಲ್ಲಿ ಇಳಿದಾಗ ಅವರನ್ನು ಎದುರುಗೊಳ್ಳಲು ಬಂದಿದ್ದ ನೂರಾರು ಜನ ಅವರಿಗೆ ಹಾರ್ದಿಕ ಸ್ವಾಗತ ನೀಡಿದರು. ಬಳಿಕ ಅವರನ್ನು ಕುದುರೆ ಸಾರೋಟಿನಲ್ಲಿ ಕುಳ್ಳಿರಿಸಿ, ಅವರ ಅತಿಥೇಯರಾದ ಸುಬ್ರಾಯನಾಯ್ಕರ ಮನೆಗೆ ಮೆರವಣಿಗೆಯಲ್ಲಿ ಕರೆದೊಯ್ದರು.

ಸಂಸ್ಕೃತ ವಿದ್ವಾಂಸರೂ ಹಿಂದೂ ಶಾಸ್ತ್ರಗಳಲ್ಲಿ ಪರಿಣತರೂ ಆಗಿದ್ದ ಸುಬ್ರಾಯ ನಾಯ್ಕರು ಸ್ವಾಮೀಜಿಯ ಅಸಾಮಾನ್ಯ ಮೇಧಾಶಕ್ತಿ ಹಾಗೂ ಧಾರ್ಮಿಕ ವಿಷಯಗಳಲ್ಲಿ ಅವರಿಗಿದ್ದ ಅಗಾಧ ಪ್ರೌಢಿಮೆಯನ್ನು ಕಂಡು ಅವರೆಡೆಗೆ ತೀವ್ರವಾಗಿ ಸೆಳೆಯಲ್ಪಟ್ಟರು. ಸ್ವಾಮೀಜಿಯ ಗೋವಾ ಭೇಟಿಯ ಉದ್ದೇಶವು ಅಲ್ಲಿ ಮಾತ್ರ ಲಭ್ಯವಿದ್ದ ಹಳೆಯ ಲ್ಯಾಟಿನ್ ಗ್ರಂಥಗಳು ಹಾಗೂ ಅಮೂಲ್ಯ ಹಸ್ತಪ್ರತಿಗಳಲ್ಲಿರುವ ಕ್ರೈಸ್ತರ ಧರ್ಮಶಾಸ್ತ್ರಗಳನ್ನು ಅಧ್ಯಯನ ಮಾಡುವುದಾಗಿತ್ತು. ಇದನ್ನು ತಿಳಿದ ಸುಬ್ರಾಯ ನಾಯ್ಕರು ತಮ್ಮ ಕ್ರೈಸ್ತ ಸ್ನೇಹಿತರಾದ ಜೆ. ಪಿ. ಆಲ್ವಾರೆಸ್ ಎಂಬ ವರನ್ನು ತಮ್ಮ ಮನೆಗೆ ಕರೆಸಿ ಸ್ವಾಮೀಜಿಯ ಪರಿಚಯ ಮಾಡಿಸಿದರು. ಆಲ್ವಾರೆಸರೊಂದಿಗೆ ಸ್ವಾಮೀಜಿ ಸ್ವಲ್ಪ ಹೊತ್ತು ಸಂಭಾಷಿಸಿದರು. ಅವರ ಅಪಾರ ವಿದ್ವತ್ತಿನಿಂದ ಪ್ರಭಾವಿತರಾದ ಆಲ್ವಾರೆಸರು, ಸ್ವಾಮೀಜಿ ರಾಚೋಲ್ ಸೆಮಿನರಿನಲ್ಲಿ ಉಳಿದುಕೊಂಡು ಅಧ್ಯಯನ ಮಾಡಲು ಸಾಧ್ಯವಾಗುವಂತೆ ವಿಶೇಷ ವ್ಯವಸ್ಥೆ ಮಾಡಿಸಿದರು. ಮರ್ಮಗೋವಾದಿಂದ ನಾಲ್ಕು ಮೈಲಿ ದೂರದಲ್ಲಿರುವ ಈ ರಾಚೋಲ್ ಸೆಮಿನರಿಯು ಗೋವಾದ ಅತ್ಯಂತ ಪ್ರಾಚೀನ ಕಾನ್ವೆಂಟ್ ಕಾಲೇಜು. ಇಲ್ಲಿ ಲ್ಯಾಟಿನ್ ಭಾಷೆಯಲ್ಲಿರುವ ಅಪರೂಪದ ಧರ್ಮಗ್ರಂಥಗಳನ್ನು ಹಾಗೂ ಹಸ್ತಪ್ರತಿಗಳನ್ನು ಸಂರಕ್ಷಿಸಿಡಲಾಗಿದೆ.

ಇಲ್ಲಿ ಸ್ವಾಮೀಜಿ ಮೂರು ದಿನಗಳ ಕಾಲ ಅತ್ಯಂತ ತತ್ಪರತೆಯಿಂದ, ಲಭ್ಯವಿರುವ ಮುಖ್ಯ ಗ್ರಂಥಗಳನ್ನೆಲ್ಲ ಅಧ್ಯಯನ ಮಾಡಿದರು. ಆ ಗ್ರಂಥಗಳ ಮೇಲಿನ ಹಾಗೂ ಕ್ರೈಸ್ತ ಧರ್ಮದ ಬಗೆಗಿನ ಅವರದೇ ಆದ ವಿಶಿಷ್ಟ ವ್ಯಾಖ್ಯಾನಗಳನ್ನು ಆಲಿಸಿದ ಅಲ್ಲಿನ ಪಾದ್ರಿಗಳು ಮತ್ತು ಸೆಮಿನರಿಯ ಎಲ್ಲ ವಿದ್ಯಾರ್ಥಿಗಳು ಅವರ ಜ್ಞಾನದ ಆಳವನ್ನೂ ವಿಶ್ಲೇಷಣಾ ಸಾಮರ್ಥ್ಯವನ್ನೂ ಮನಗಂಡು ಪರಮಾಶ್ಚರ್ಯಗೊಂಡರು.

ಸ್ವಾಮೀಜಿ ಈ ಸೆಮಿನರಿಯಿಂದ ಮರ್ಮಗೋವಾಗೆ ಹಿಂದಿರುಗಿದ ಮೇಲೂ ಅನೇಕ ಪಾದ್ರಿ ಗಳು–ಕೆಲವೊಮ್ಮೆ ಎಷ್ಟೋ ದೂರದ ಸ್ಥಳಗಳಿಂದ–ಅವರನ್ನು ಭೇಟಿ ಮಾಡಲು ಬರುತ್ತಿದ್ದರು.

ಸ್ವಾಮೀಜಿ ಹೀಗೆ ಗೋವಾಕ್ಕೆ ಬಂದು, ವಿಶೇಷ ಪರಿಶ್ರಮ ವಹಿಸಿ, ಕ್ರೈಸ್ತ ಧರ್ಮಗ್ರಂಥಗಳನ್ನು ಅಷ್ಟೊಂದು ಆಳವಾಗಿ ಅಧ್ಯಯನ ಮಾಡಬೇಕಾದುದರ ಆವಶ್ಯಕತೆಯೇನಿದ್ದಿರಬಹುದು? ಬಹುಶಃ ಸ್ವತಃ ಅವರಿಗೂ ಆಗ ಅದರ ಅರಿವಿದ್ದಿರಲಾರದು! ಅವರು ಅಷ್ಟು ಆಸಕ್ತಿಯಿಂದ ಅವುಗಳನ್ನು ಅಧ್ಯಯನ ಮಾಡಿದ್ದು ಕೇವಲ ತಮ್ಮ ಜಿಜ್ಞಾಸೆಯಿಂದಾಗಿ. ಆದರೆ ಅವರಿಂದ ಅದನ್ನು ಮಾಡಿಸುತ್ತಿರುವವರು ಶ್ರೀರಾಮಕೃಷ್ಣರಲ್ಲದೆ ಇನ್ನಾರು? ತಮ್ಮ ಅಲ್ಪಾಯುಸ್ಸಿನಲ್ಲಿ ಯಾವ ಬೃಹತ್ ಕಾರ್ಯವೊಂದನ್ನು ಮಾಡಬೇಕಾಗಿದೆಯೋ ಅದನ್ನು ಸಾಧಿಸುವತ್ತ ಸ್ವಾಮೀಜಿ ವೇಗವಾಗಿ ಮುನ್ನಡೆಯುತ್ತಿದ್ದಾರೆ. ಅವರು ಅಮೆರಿಕೆಗೆ ಹೋಗಬೇಕಾಗಿದೆ; ಅಲ್ಲಿನ ವಿಶ್ವವೇದಿಕೆಯ ಮೇಲೆ ನಿಂತು ಭಾರತದ ಭವ್ಯತೆಯನ್ನು, ಸನಾತನ ಧರ್ಮದ ಸಾರವನ್ನು ಘೋಷಿಸಬೇಕಾಗಿದೆ. ಅಲ್ಲದೆ ಆ ಪರಕೀಯರಿಗೇ ಅವರ ಧರ್ಮದ ಮರ್ಮವನ್ನು ತೋರಿಸಿಕೊಟ್ಟು ಮತಾಂಧತೆಯ ಭಾವನೆ ಯನ್ನು ಹೋಗಲಾಡಿಸಬೇಕಾಗಿದೆ. ಶ್ರೀರಾಮಕೃಷ್ಣರು ಕೇವಲ ಹಿಂದೂಧರ್ಮವನ್ನು ಮಾತ್ರ ಸಂಸ್ಥಾಪಿಸುವುದಕ್ಕಾಗಿ ಅವತರಿಸಿದವರಲ್ಲ. ಅವರು ಜಗತ್ತಿನ ಸಕಲ ಧರ್ಮಗಳನ್ನೂ ಸಂಸ್ಥಾಪಿಸಿ ಸಕಲರೂ ತಮ್ಮತಮ್ಮ ಧರ್ಮದಲ್ಲಿ ಶಾಂತಿಯಿಂದಿರುವಂತೆ ಮಾಡುವುದಕ್ಕಾಗಿ ಬಂದರು. ಈಗ ಅದೇ ಶ್ರೀರಾಮಕೃಷ್ಣರ ಶಿಷ್ಯರಾದ ವಿವೇಕಾನಂದರು ಜಗತ್ತಿನ ಜನರಿಗೆ ಅವರವರ ಧರ್ಮಗಳ ತಿರುಳನ್ನು ತಿಳಿಸಿ ಶಾಂತಿಯಿಂದ ನೆಲೆಗೊಳ್ಳುವಂತೆ ಮಾಡಬೇಕಾಗಿದೆ. ಅಲ್ಲದೆ ಪಾಶ್ಚಾತ್ಯ ಜಗತ್ತಿನ ನೂರಾರು ವಿದ್ವಾಂಸರನ್ನೂ ಚಿಂತಕರನ್ನೂ ತೀವ್ರವಾಗಿ ಆಕರ್ಷಿಸುವಲ್ಲಿ, ಸಕಲ ಧರ್ಮಗಳ ಕುರಿತಾದ ಅವರ ಈ ಅಪರಿಮಿತ ಜ್ಞಾನವು ಮುಖ್ಯ ಪಾತ್ರವನ್ನು ವಹಿಸಲಿದೆ. ಹೆಚ್ಚೇಕೆ? ಮುಂದೆ ಅವರು ಕ್ರಿಸ್ತನ ಬಗ್ಗೆ ಮಾತನಾಡುತ್ತಿದ್ದರೆ ಅವರಲ್ಲಿ ಸಾಕ್ಷಾತ್ ಕ್ರಿಸ್ತನನ್ನೇ ಕಂಡವರೆಷ್ಟೋ ಜನ! ಅಲ್ಲದೆ, ಇಲ್ಲಿ ಗಮನಿಸಬೇಕಾದ ಮತ್ತೊಂದು ಅಂಶವಿದೆ. ಆ ಕ್ರೈಸ್ತ ಗ್ರಂಥಗಳನ್ನು, ಅದರಲ್ಲೂ ಲ್ಯಾಟೀನ್ ಭಾಷೆಯಲ್ಲಿರುವ ಗ್ರಂಥಗಳನ್ನು ಅಧ್ಯಯನ ಮಾಡಲು ಎಂತಹ ಮೇಧಾವಿಗಳಿ ಗಾದರೂ ಹಲವಾರು ವರ್ಷಗಳೇ ಬೇಕು. ಹಾಗಿರುವಲ್ಲಿ ಅವುಗಳನ್ನು ಸ್ವಾಮೀಜಿ ಕೇವಲ ಮೂರು ದಿನಗಳಲ್ಲಿ ಅಧ್ಯಯನ ಮಾಡಿ ವಿಷಯವನ್ನು ಗ್ರಹಿಸಿದ್ದೂ ಅಲ್ಲದೆ ಅವುಗಳ ಕುರಿತಾಗಿ ತಮ್ಮದೇ ಆದ ವ್ಯಾಖ್ಯಾನಗಳನ್ನು ನೀಡಿ ಅಲ್ಲಿನ ಪ್ರಮುಖ ಪಾದ್ರಿಗಳನ್ನೇ ಬೆರಗುಗೊಳಿಸುತ್ತಾರೆ. ಆಧ್ಯಾತ್ಮಿಕ ಅನುಭವಗಳಾಗಿದ್ದರೆ ಮಾತ್ರ ಆಧ್ಯಾತ್ಮಿಕ ಗ್ರಂಥಗಳಲ್ಲಿನ ರಹಸ್ಯಗಳನ್ನು ಮನಗಾಣಲು ಸಾಧ್ಯ. ತಮ್ಮ ಅಗಾಧ ಅನುಭವದ ಆಧಾರದಿಂದ ಸ್ವಾಮೀಜಿ ಕ್ರೈಸ್ತ ಗ್ರಂಥಗಳ ಮರ್ಮವನ್ನು ಕ್ರೈಸ್ತ ಧರ್ಮಗುರುಗಳಿಗೇ ತಿಳಿಸಿಕೊಟ್ಟದ್ದರಲ್ಲಿ ಅಚ್ಚರಿಯೇನೂ ಇಲ್ಲ.

ಸುಬ್ರಾಯನಾಯ್ಕರ ಮನೆಯಲ್ಲಿ ಪ್ರತಿ ಸಂಜೆಯೂ ಭಜನೆಯ ಕಾರ್ಯಕ್ರಮವಿರುತ್ತಿತ್ತು. ಒಂದು ದಿನ ಸ್ವಾಮೀಜಿ ಇದರಲ್ಲಿ ಪಾಲ್ಗೊಂಡಾಗ ಅವರ ಸಂಗೀತ ಪ್ರತಿಭೆಯನ್ನು ಕಂಡು ಎಲ್ಲರೂ ಅಚ್ಚರಿಗೊಂಡರು. ಮರುದಿನ ಪ್ರಖ್ಯಾತ ಸಂಗೀತಗಾರರನ್ನೂ ತಬಲಾ ವಾದಕರನ್ನೂ ಆಹ್ವಾನಿಸಲಾಯಿತು. ಬಹುತೇಕ ಶಾಸ್ತ್ರೀಯ ಸಂಗೀತಗಾರರು ಹಾಡುವಾಗ ನೂರಾರು ಅಂಗಭಂಗಿ ಗಳನ್ನು ಮಾಡುತ್ತಾರೆ. ಅದು ಎಷ್ಟರ ಮಟ್ಟಿಗೆಂದರೆ, ಶ್ರೋತೃಗಳ ಕಿವಿಗಳೊಂದಿಗೆ ಕಣ್ಣುಗಳಿಗೂ ರಂಜನೆ ಇರುತ್ತದೆ. ಆದರೆ ಹೀಗೆ ಅಂಗಭಂಗಿಗಳನ್ನು ಮಾಡದೆ ಹಾಡಲು ಸಾಧ್ಯವೇ ಇಲ್ಲ ಎಂಬುದು ಹಲವರ ನಿಶ್ಚಿತ ಅಭಿಮತ. ಅನೇಕ ಸಂಗೀತಗಾರರು ಸೇರಿದ್ದ ಆ ಸಂದರ್ಭದಲ್ಲಿ ಸ್ವಾಮೀಜಿ, ಈ ಅಭಿಪ್ರಾಯವು ತಪ್ಪು ಎಂದು ಹೇಳಿದರು. ಅಷ್ಟೇ ಅಲ್ಲ; ಶರೀರದ ಯಾವುದೇ ಅಂಗಾಂಗವನ್ನೂ ಚಲಿಸದೆ, ಬಾಯಿಯ ಚಲನೆಯನ್ನೂ ತೀರ ಮಿತವಾಗಿ ಬಳಸಿ, ಅನೇಕ ರಾಗ ಗಳನ್ನು ಸುಶ್ರಾವ್ಯಯವಾಗಿ ಹಾಡಿ ತೋರಿಸಿದರು. ಕಪೂರ್​ಜಿ ಎಂಬ ಪ್ರಸಿದ್ಧ ತಬಲಾ ವಾದಕ ಸ್ವಾಮೀಜಿಗೆ ಪಕ್ಕವಾದ್ಯ ನುಡಿಸುತ್ತಿದ್ದ. ಇವನೂ ಕೂಡ ತಬಲಾ ನುಡಿಸುವಾಗ ವಿಪರೀತ ತಲೆ-ಮೈ ಕೈ ಅಲುಗಾಡಿಸುತ್ತಿದ್ದ. ಸ್ವಾಮೀಜಿ ತಾವು ಹಾಡಿಯಾದ ಮೇಲೆ, ಸ್ವತಃ ತಾವೇ ತಬಲ ನುಡಿಸಿ, ಶರೀರವನ್ನು ಅನಾವಶ್ಯಕವಾಗಿ ಅಲುಗಾಡಿಸದೆ ತಬಲ ನುಡಿಸುವುದು ಸಾಧ್ಯವೆಂಬುದನ್ನು ತೋರಿಸಿಕೊಟ್ಟರು.

ಸ್ವಾಮೀಜಿಯ ಆತಥೇಯರಾದ ಸುಬ್ರಾಯನಾಯ್ಕರು ವೃತ್ತಿಯಿಂದ ಆಯುರ್ವೇದ ವೈದ್ಯ ರಾಗಿದ್ದರು. ಸ್ವಾಮೀಜಿ ಇವರ ಮನೆಯಲ್ಲಿದ್ದಾಗ ಅವರು ಉಳಿದುಕೊಂಡಿದ್ದ ಕೋಣೆಯನ್ನು, ಅವರು ಉಪಯೋಗಿಸುತ್ತಿದ್ದ ವಸ್ತುಗಳೊಂದಿಗೆ ಇಂದಿಗೂ ಹಾಗೆಯೇ ಸಂರಕ್ಷಿಸಿಡಲಾಗಿದೆ. ಅಲ್ಲದೆ ಆ ಸಂದರ್ಭದಲ್ಲಿ ತೆಗೆಯಲ್ಪಟ್ಟ ಅವರ ಒಂದು ಭಾವಚಿತ್ರವನ್ನೂ ಕಾಣಬಹುದಾಗಿದೆ. ಸ್ವಾಮೀಜಿಯಿಂದ ತುಂಬ ಪ್ರಭಾವಿತರಾಗಿದ್ದ ಸುಬ್ರಾಯನಾಯ್ಕರು ಮುಂದೆ ಸಂನ್ಯಾಸ ಸ್ವೀಕರಿಸಿ ‘ಸುಬ್ರಹ್ಮಣ್ಯಾನಂದ ತೀರ್ಥ’ರಾದರು. ಸ್ವಾಮೀಜಿಯ ವಾಸದಿಂದಾಗಿ ಪುನೀತವಾದ ಆ ಕೋಣೆಯಲ್ಲೇ ಅವರು ತಮ್ಮ ಇಹಲೋಕಯಾತ್ರೆಯನ್ನು ಮುಗಿಸಿದರು.

