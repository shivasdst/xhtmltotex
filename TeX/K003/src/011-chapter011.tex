
\chapter{ವಿಶಾಲ ವಾರಿಧಿಯಲ್ಲಿ}

\noindent

‘ಪೆನಿನ್ಸುಲಾರ್​’ ಹಡಗನ್ನು ಹತ್ತಿ ಮುಂಬಯಿಯಿಂದ ಹೊರಟ ಸ್ವಾಮೀಜಿ, ನಿಧಾನವಾಗಿ ಸಮುದ್ರ ಪ್ರಯಾಣಕ್ಕೆ ಒಗ್ಗಿಕೊಳ್ಳಲಾರಂಭಿಸಿದರು. ಮೊದಮೊದಲಿಗೆ ಅವರಿಗೆ ಎಲ್ಲಕ್ಕಿಂತ ಹೆಚ್ಚು ಕಿರಿಕಿರಿಯುಂಟುಮಾಡಿದ್ದೆಂದರೆ ಅವರದೇ ಆದ ಸಾಮಾನು ಸರಂಜಾಮುಗಳು! ಏಕೆಂದರೆ ಯಾವಾಗಲೂ ಅವುಗಳ ಕಡೆಗೆ ಗಮನವಿಟ್ಟಿರಬೇಕಲ್ಲ? ಅವುಗಳನ್ನೆಲ್ಲ ಎಚ್ಚರದಿಂದ ಕಾಪಾಡಿ ಕೊಳ್ಳುವ ಹಾಗೂ ಓರಣವಾಗಿಟ್ಟುಕೊಳ್ಳುವ ಹೊಸ ಹೊಣೆಗಾರಿಕೆಯಿದೆಯಲ್ಲ? ಈಗ ಅವರ ಹತ್ತಿರ ಒಂದು ದೊಡ್ಡ ಪೆಟ್ಟಿಗೆ, ಬಟ್ಟೆಯ ಕಪಾಟು, ಪುಟ್ಟ ಕೈಪೆಟ್ಟಿಗೆ–ಇವೆಲ್ಲ ಇದ್ದುವು. ಕೇವಲ ಕೈಯಲ್ಲಿ ದಂಡಕಮಂಡಲುಗಳನ್ನೂ ಪುಟ್ಟದೊಂದು ಗಂಟನ್ನೂ ಹಿಡಿದು ನಿರಾತಂಕವಾಗಿ ಭಾರತದ ಉದ್ದಗಲವನ್ನೆಲ್ಲ ಅಳೆದ ಅವರಿಗೆ ಈ ಹೊಸ ಸಾಮಾನುಗಳ ಯೋಗಕ್ಷೇಮದ ಕಡೆಗೆ ಗಮನ ಕೊಡುವುದು ಸಾಕಷ್ಟು ಕಷ್ಟವಾಯಿತು. ಆದರೆ ವಿಧಿಯಿಲ್ಲದ್ದರಿಂದ ಕಡೆಗೆ ಅದೇ ಸರಿ ಹೋಯಿತು! ಇದಲ್ಲದೆ ಹಡಗಿನ ಹೊಸ ಬಗೆಯ ಆಹಾರ, ವಿವಿಧ ದೇಶಗಳ ವಿವಿಧ ಭಾಷೆಗಳ ನ್ನಾಡುವ ಅಪರಿಚಿತ ಸ್ತ್ರೀಪುರುಷರ ಸಹವಾಸ ಮುಂತಾದ ಸಮಸ್ಯೆಗಳೂ ಕೆಲದಿನಗಳಲ್ಲೇ ಪರಿಹಾರವಾದುವು.

ಇದನ್ನುಳಿದಂತೆ ಹಡುಗುಪ್ರಯಾಣದ ತಮ್ಮ ಈ ನೂತನ ಅನುಭವವನ್ನು ಸ್ವಾಮೀಜಿ ಆನಂದಿಸಲಾರಂಭಿಸಿದರು. ದಿಗಂತದವರೆಗೂ ಹರಡಿಕೊಂಡಿರುವ ಅಗಾಧ ಜಲರಾಶಿ, ನಿರಂತರ ವಾಗಿ ಎದ್ದೇಳುವ ಭೀಮಾಕಾರದ ಅಲೆಗಳು, ಸಮುದ್ರದ ಮೇಲಿಂದ ಬೀಸುವ ತಂಗಾಳಿಯ ಸುಂಯ್​ಗುಡುವ ಶಬ್ದ, ಸೂರ್ಯೋದಯ–ಸೂರ್ಯಾಸ್ತಮಾನಗಳ ಸೊಬಗು, ಕ್ಷಣಕ್ಷಣಕ್ಕೂ ವಿವಿಧ ವಿಚಿತ್ರ ವರ್ಣವಿನ್ಯಾಸಗಳನ್ನು ತಾಳುವ ಮೇಘಗಳ ಸೌಂದರ್ಯ–ಇವುಗಳನ್ನೆಲ್ಲ ಅವರ ಭಾವನಾತ್ಮಕ ಮನಸ್ಸು ಸಂಪೂರ್ಣವಾಗಿ ಆಸ್ವಾದಿಸುತ್ತಿತ್ತು. ಅಲ್ಲದೆ ಸಹಪ್ರಯಾಣಿಕರ ಸೌಜನ್ಯ ಪೂರ್ಣ ನಡವಳಿಕೆಯೂ ಅವರಿಗೆ ಈ ಸಮುದ್ರಯಾನವನ್ನು ಪ್ರಿಯವಾಗಿಸಿತು. ಐರೋಪ್ಯರ ನಡೆನುಡಿಗಳನ್ನು ಸೂಕ್ಷ್ಮವಾಗಿ ಗಮನಿಸುತ್ತ ಅವರ ಶಿಷ್ಟಾಚಾರಗಳನ್ನೂ ಸಂಪ್ರದಾಯಗಳನ್ನೂ ಅರಿತರು. ಸ್ವಾಮೀಜಿಯ ಅತ್ಯಂತ ವಿಶಿಷ್ಟವಾದ ಕೇಸರಿ ಬಣ್ಣದ ರೇಷ್ಮೆ ಉಡುಗೆ, ಅವರ ರಾಜಠೀವಿ, ಪ್ರಖರ ನೇತ್ರಗಳು ಪ್ರತಿಯೊಬ್ಬರ ಗಮನವನ್ನೂ ಸೆಳೆದುವು. ಅವರೊಂದಿಗೆ ಒಂದೆರಡು ಮಾತನಾಡುವಷ್ಟರಲ್ಲಿ ಅವರ ವಿನಯಪೂರ್ಣವರ್ತನೆ ಎಂಥವರನ್ನೂ ಸಂತಸ ಗೊಳಿಸುತ್ತಿತ್ತು. ಅವರ ಮಂದಹಾಸಪೂರಿತ ಮುಖ, ಸಭ್ಯ ನಡವಳಿಕೆಗಳು ಅವರನ್ನು ಹಡಗಿನಲ್ಲಿ ಜನಪ್ರಿಯವಾಗಿಸಿದುವು.

ಕೆಲವೊಮ್ಮೆ ಸ್ವಾಮೀಜಿ ಹಡಗಿನ ಅಂಗಳದಲ್ಲಿ ಒಬ್ಬರೇ ನಿಧಾನವಾಗಿ ಅಡ್ಡಾಡುತ್ತಿದ್ದರು. ಇಲ್ಲವೆ ಅನ್ಯಮನಸ್ಕರಾಗಿ ನಿಂತು, ಹಡಗಿಗೆ ಬಡಿಯುವ ಅಲೆಗಳನ್ನು ದಿಟ್ಟಿಸುತ್ತಿದ್ದರು. ಹಡಗಿನ ಕ್ಯಾಪ್ಟನ್ ಅವರನ್ನು ಬಹಳವಾಗಿ ಇಷ್ಟಪಟ್ಟಿದ್ದ. ಒಮ್ಮೊಮ್ಮೆ ಅವನು, ಸ್ವಾಮೀಜಿ ಅಡ್ಡಾಡುವಾಗ ಅವರಿಗೆ ಜೊತೆಗೊಡುತ್ತಿದ್ದ. ಅವರನ್ನು ಹಡಗಿನ ಎಲ್ಲ ಭಾಗಗಳಿಗೂ ಕರೆದುಕೊಂಡು ಹೋದ ನಲ್ಲದೆ, ವಿವಿಧ ಯಂತ್ರಗಳು ಹೇಗೆ ಕೆಲಸಮಾಡುತ್ತವೆಯೆಂಬುದನ್ನು ವಿವರಿಸಿದ. ಕೆಲವೊಮ್ಮೆ ದೂರದ ದ್ವೀಪಗಳನ್ನೋ ನಡುಗಡ್ಡೆಗಳನ್ನೋ ತೋರಿಸಿ ಅವುಗಳ ಬಗ್ಗೆ ಹೇಳುತ್ತಿದ್ದ.

ಅವರ ಈ ಪ್ರಯಾಣದ ಒಟ್ಟು ದೂರ ಸುಮಾರು ೮ಂಂಂ ಮೈಲಿಗಳು; ಪ್ರಯಾಣದ ಅವಧಿ ಸುಮಾರು ಎರಡು ತಿಂಗಳು! ಈ ಸುದೀರ್ಘ ಪ್ರಯಾಣದುದ್ದಕ್ಕೂ ಸ್ವಾಮೀಜಿ ಹೊಸಹೊಸ ದೃಶ್ಯಗಳನ್ನು ಕಾಣುತ್ತಿದ್ದರು; ಹೊಸ ಹೊಸ ವಿಷಯಗಳನ್ನು ಅರಿಯುತ್ತಿದ್ದರು. ಇದೇ ಹಡಗಿ ನಲ್ಲಿ, ಹಿಂದೊಮ್ಮೆ ಮುಂಬಯಿಯಲ್ಲಿ ಅವರ ಆತಿಥೇಯನಾಗಿದ್ದ ಸೇಠ್ ಛಬಿಲ್​ದಾಸನೂ ಪ್ರಯಾಣ ಮಾಡುತ್ತಿದ್ದ.

ಮುಂಬಯಿಯಿಂದ ಹೊರಟ ಒಂದು ವಾರದ ಅನಂತರ ‘ಪೆನಿನ್ಸುಲಾರ್​’ ಸುಮಾರು ೮೮ಂ ಮೈಲಿಗಳನ್ನು ಕ್ರಮಿಸಿ ಈಗಿನ ಶ್ರೀಲಂಕಾದ ರಾಜಧಾನಿ ಕೊಲೊಂಬೊ ಬಂದರಿನಲ್ಲಿ ಲಂಗರು ಹಾಕಿತು. ಇಲ್ಲಿ ಅದು ಸುಮಾರು ಒಂದು ಇಡೀ ದಿನ ವಿಶ್ರಮಿಸಲಿತ್ತಾದ್ದರಿಂದ, ಈ ಅವಕಾಶವನ್ನು ಉಪಯೋಗಿಸಿಕೊಂಡು ಸ್ವಾಮೀಜಿ ನಗರದರ್ಶನಕ್ಕೆ ಹೊರಟರು. ಗಾಡಿಯಲ್ಲಿ ಕುಳಿತು ಕೊಲೊಂಬೊ ನಗರದ ರಸ್ತೆಗಳಲ್ಲಿ ಸಂಚರಿಸಿದರು. ಇಲ್ಲಿ ಅವರು ಬೌದ್ಧ ದೇವಾಲಯ ವೊಂದನ್ನು ಸಂದರ್ಶಿಸಿದರು. ಅಲ್ಲಿ ನಿರ್ವಾಣಾಸನದಲ್ಲಿ ಕುಳಿತ ಬುದ್ಧನ ಸುಂದರವಾದ ಬೃಹತ್ ವಿಗ್ರಹವನ್ನು ಕಂಡು ಆಶ್ಚರ್ಯಚಕಿತರಾದರು. ಕೊಲೊಂಬೊದ ಬಳಿಕ ಹಡಗಿನ ಮುಂದಿನ ನಿಲುಗಡೆ ಪೆನಾಂಗ್​ನಲ್ಲಿ. ಮುಸಲ್ಮಾನರೇ ಹೆಚ್ಚಾಗಿರುವ ಈ ಪ್ರದೇಶದ ಜನಗಳು, ಹಿಂದಿನ ಶತಮಾನಗಳಲ್ಲಿ ಜೀವನೋಪಾಯಕ್ಕಾಗಿ ವ್ಯಾಪಾರೀ ಹಡಗುಗಳ ದರೋಡೆಯನ್ನು ಅವಲಂಬಿಸಿ ದ್ದರು. ಆದರೆ ಸ್ವಾಮೀಜಿ ಮುಂದೆ ಪತ್ರವೊಂದರಲ್ಲಿ ಬರೆಯುವಂತೆ, “ಕೋವಿಗಳಿಂದ ಸುಸಜ್ಜಿತ ವಾದ ಆಧುನಿಕ ಯುದ್ಧನೌಕೆಗಳು ಈ ಜನರನ್ನು ಹತೋಟಿಗೆ ತಂದು ಶಾಂತಿಯುತ ಜೀವನವನ್ನ ವಲಂಬಿಸುವಂತೆ ಮಾಡಿವೆ.”

ಪೆನಾಂಗ್​ನಿಂದ ಹಡಗು ಸುಮಾತ್ರಾ ದ್ವೀಪದ ಮಾರ್ಗವಾಗಿ ಸಾಗಿ ಸಿಂಗಪುರವನ್ನು ತಲು ಪಿತು. ಇಲ್ಲಿ ಸ್ವಾಮೀಜಿ ವಸ್ತುಸಂಗ್ರಹಾಲಯವನ್ನೂ ಬೊಟಾನಿಕಲ್ ಗಾರ್ಡನ್ ಅನ್ನೂ (ಸಸ್ಯ- ವೈಜ್ಞಾನಿಕ ತೋಟ) ವೀಕ್ಷಿಸಿದರು. ಈ ಹೊಸ ದೇಶಗಳ ಹೊಸ ದೃಶ್ಯಗಳನ್ನು ಕಂಡು ಬಾಲಕನಂತೆ ಸಂತೋಷಪಟ್ಟರು.

ಇಲ್ಲಿಂದ ಹಡಗು ಮುಂದೆ ಸಾಗಿ ಹಾಂಗ್​ಕಾಂಗ್ ಬಂದರನ್ನು ಸೇರಿತು. ಇಲ್ಲಿ ಹಡಗಿಗೆ ಮತ್ತೆ ಮೂರು ದಿನಗಳ ಬಿಡುವು. ಹಾಂಗ್​ಕಾಂಗ್​ನಲ್ಲಿ ಅವರಿಗೆ ಚೀನಾದ ಮೊದಲ ಇಣುಕು ನೋಟ ಸಿಕ್ಕಿತು. ಚೀನಾ ದೇಶವೆಂದರೆ ಅದ್ಭುತ ಸಾಹಸಮಯ ಕಥೆಗಳ ಕಿನ್ನರಲೋಕವೆಂಬ ನಂಬಿಕೆ ಆಗ ಪ್ರಚಲಿತವಿತ್ತು. ಆದರೆ ಅಲ್ಲಿನ ಜನಜೀವನವನ್ನು ಕಣ್ಣಾರೆ ಕಂಡ ಮೇಲೆ ‘ಈ ಚೀನಾದ ಜನರಷ್ಟು ವ್ಯಾಪಾರೀ ಬುದ್ಧಿಯವರು ಬೇರೆಲ್ಲೂ ಇರಲಾರರು’ ಎಂದು ಅವರಿಗನ್ನಿಸಿತು. ಜಗತ್ತಿನಲ್ಲೇ ಅತ್ಯಂತ ಚಟುವಟಿಕೆಯ ರೇವುಗಳಲ್ಲೊಂದಾದ ಹಾಂಗ್​ಕಾಂಗ್​ನ ದೃಶ್ಯ ಅವರನ್ನು ಬೆರಗುಗೊಳಿಸಿತು. ಜುಲೈ ೧ಂರಂದು ಯೊಕೊಹಾಮದಿಂದ ತಮ್ಮ ಮದರಾಸೀ ಶಿಷ್ಯರಿಗೆ ಬರೆದ ಪತ್ರದಲ್ಲಿ ತಾವು ಕಂಡ ದೃಶ್ಯವನ್ನು ಬಣ್ಣಿಸುತ್ತಾರೆ:

“ಇಲ್ಲಿನ ದೋಣಿಗಳಿಗೆಲ್ಲ ಎರಡೆರಡು ಚುಕ್ಕಾಣಿಗಳು. ದೋಣಿಯವನು ತನ್ನ ಸಂಸಾರ ಸಮೇತನಾಗಿ ದೋಣಿಯಲ್ಲೇ ವಾಸ ಮಾಡುತ್ತಾನೆ. ಹೆಚ್ಚುಕಡಿಮೆ ಯಾವಾಗಲೂ ಚುಕ್ಕಾಣಿ ಹಿಡಿಯುವವಳು ಅವನ ಹೆಂಡತಿಯೇ. ಒಂದು ಚುಕ್ಕಾಣಿಯನ್ನು ಕೈಗಳಿಂದ, ಮತ್ತೊಂದನ್ನು ಕಾಲುಗಳಿಂದ ನಿಯಂತ್ರಿಸುತ್ತಾಳೆ. ಈ ಹೆಂಗಸರ ಪೈಕಿ ನೂರಕ್ಕೆ ತೊಂಬತ್ತು ಜನ ತಮ್ಮ ಹಸುಗೂಸನ್ನು ಬೆನ್ನಿಗೆ ಕಟ್ಟಿಕೊಂಡಿರುತ್ತಾರೆ. ಆ ಪುಟ್ಟ ಚೀನೀ ಮಗುವಿನ ಕೈಕಾಲುಗಳು ಮಾತ್ರ ಜೋತಾಡುತ್ತಿರುತ್ತವೆ. ತಾಯಿ ತುಂಬ ಚಟುವಟಿಕೆಯಿಂದ ಕಷ್ಟದ ಕೆಲಸಗಳನ್ನು ಮಾಡುತ್ತ ಓಡಾಡುತ್ತಿರುತ್ತಾಳೆ; ಭಾರವಾದ ವಸ್ತುಗಳನ್ನು ತಳ್ಳುತ್ತಿರುತ್ತಾಳೆ; ಇಲ್ಲವೆ ತುಂಬ ಚಾಲಾಕಿ ನಿಂದ ದೋಣಿಯಿಂದ ದೋಣಿಗೆ ಜಿಗಿಯುತ್ತಿರುತ್ತಾಳೆ. ಆದರೆ ಬೆನ್ನಿಗೆ ಕಟ್ಟಿದ ಆ ಪುಟ್ಟ ಕಿಟ್ಟಣ್ಣ ಮಾತ್ರ ತನ್ನ ಪಾಡಿಗೆ ಶಾಂತವಾಗಿರುತ್ತಾನೆ! ಇದೊಂದು ವಿಚಿತ್ರ ನೋಟವೇ ಸರಿ. ಅಲ್ಲದೆ ಅತ್ತಿಂದಿತ್ತ ಓಡಾಡುತ್ತಲೇ ಇರುವ ದೋಣಿಗಳ ಹಾಗೂ ದೊಡ್ಡ ಉಗಿದೋಣಿಗಳ ನೂಕು ನುಗ್ಗಲು ಅಷ್ಟಿಷ್ಟಲ್ಲ. ಈ ಬಾಲಕ ಎಂತಹ ಅಪಾಯದ ಮಡುವಿನಲ್ಲಿರುತ್ತಾನೆಂದರೆ ಯಾವ ಕ್ಷಣದಲ್ಲಾದರೂ ಅವನ ಪುಟ್ಟ ತಲೆಯು ಜಡೆ ಸಮೇತವಾಗಿ ಜಜ್ಜಿಹೋಗಬಹುದು. ಆದರೆ ಅವನು ಅದಕ್ಕೆಲ್ಲ ಹೆದರುವವನಲ್ಲ. ಈ ಗಡಿಬಿಡಿ ಜೀವನದಲ್ಲಿ ಅವನಿಗೆ ಕಿಂಚಿತ್ತೂ ಆಸಕ್ತಿಯಿರು ವಂತೆ ತೋರುವುದಿಲ್ಲ! ಕೆಲಸಕಾರ್ಯದ ಭರಾಟೆಯಲ್ಲಿ ಮುಳುಗಿಹೋಗಿರುವ ಅವನ ತಾಯಿ ಆಗಾಗ ಕೊಡುವ ಅಕ್ಕಿರೊಟ್ಟಿಯ ಚೂರನ್ನು ಕಡಿಯುವುದರಲ್ಲೇ ಅವನಿಗೆ ಸಂತೃಪ್ತಿ.

“ಚೀನೀ ಬಾಲಕ ಒಬ್ಬ ತತ್ತ್ವಜ್ಞಾನಿಯೇ ಸರಿ. ನಿಮ್ಮ ಭಾರತೀಯ ಹುಡುಗ ಇನ್ನೂ ಅಂಬೆಗಾಲಿಡುವುದರಲ್ಲೇ ಇರುವಾಗ ಈತ ಕೆಲಸಕ್ಕೆ ಹೊರಟುಬಿಡುತ್ತಾನೆ. ಜೀವನಾವಶ್ಯಕತೆಗಳ ತತ್ತ್ವ ಇವನಿಗೆ ಬಹಳ ಚೆನ್ನಾಗಿ ಗೊತ್ತಾಗಿಹೋಗಿದೆ. ಭಾರತೀಯರ ಹಾಗೂ ಚೀನೀಯರ ನಾಗರಿಕತೆಗಳು ಇನ್ನೂ ಮುದುಡಿಕೊಂಡೇ ಇರುವುದಕ್ಕೆ ಅವರ ಅತಿಯಾದ ಬಡತನವೇ ಮುಖ್ಯ ಕಾರಣ. ಒಬ್ಬ ಸಾಮಾನ್ಯ ಭಾರತೀಯನ ಇಲ್ಲವೆ ಚೀನೀಯನ ಮುಂದೆ ಅವನ ದೈನಂದಿನ ಜೀವನದ ಸಮಸ್ಯೆಗಳೇ ಭೀಕರವಾಗಿ ನಿಂತಿರುವುದರಿಂದ ಅವನಿಗೆ ಬೇರಾವುದರ ಬಗ್ಗೆಯೂ ಚಿಂತಿಸಲು ಅಸಾಧ್ಯವಾಗಿದೆ....”

ಹಾಂಗ್​ಕಾಂಗ್​ನಲ್ಲಿ ಹಡಗು ಮೂರು ದಿನ ಉಳಿದುಕೊಂಡದ್ದರಿಂದ ಪ್ರಯಾಣಿಕರಿಗೆ ಕ್ಯಾಂಟನ್ ನಗರವನ್ನು ಸಂದರ್ಶಿಸುವ ಅವಕಾಶವಾಯಿತು. ‘ಸಿ ಕಿಯಾಂಗ್​’ ನದಿಯ ಮಾರ್ಗ ವಾಗಿ ೮ಂ ಮೈಲಿ ಪ್ರಯಾಣ ಮಾಡಿ ಕ್ಯಾಂಟನ್ ತಲುಪಿದರು. ಅದೇ ಪತ್ರದಲ್ಲಿ ಸ್ವಾಮೀಜಿ ಈ ನಗರದ ಬಗ್ಗೆ ಬರೆಯುತ್ತಾರೆ:

“ಅಬ್ಬ! ಅದೆಂತಹ ಜೀವನ! ಅದೆಂತಹ ಗಡಿಬಿಡಿ! ನೀರಿನ ಮೇಲ್ಮೈಯನ್ನು ಸಂಪೂರ್ಣ ವಾಗಿ ಮುಚ್ಚಿಯೇ ಬಿಟ್ಟಿರುವಂತೆ ಕಾಣುವ ಅದೆಷ್ಟು ದೋಣಿಗಳು! ಕೇವಲ ವ್ಯಾಪಾರದ ದೋಣಿಗಳಷ್ಟೇ ಅಲ್ಲ. ಮನೆಗಳೂ ಆಗಿರುವ ನೂರಾರು ದೋಣಿಗಳು. ಕೆಲವಂತೂ ಅದೆಷ್ಟು ಸುಂದರ, ಅದೆಷ್ಟು ದೊಡ್ಡವು! ನಿಜ ಹೇಳಬೇಕೆಂದರೆ, ಕೆಲವು ದೋಣಿಗಳಲ್ಲಂತೂ ಎರಡು- ಮೂರು ಮಹಡಿಗಳು, ಮನೆಗಳ ಸುತ್ತ ಜಗಲಿಗಳು, ಅಲ್ಲದೆ ಓಡಾಡಲು ರಸ್ತೆಗಳು–ಎಲ್ಲ ಇವೆ!

ಇಲ್ಲಿ ಸ್ವಾಮೀಜಿ ಇದುವರೆಗೆ ತಾವು ಕೇಳರಿಯದಿದ್ದ ಎಷ್ಟೋ ಹೊಸ ವಿಷಯಗಳನ್ನು ಅರಿತರು. ಚೀನಾದಲ್ಲಿ ಆಗ ಜನಪ್ರಿಯವಾಗಿದ್ದ, ಪಾದಗಳನ್ನು ಕಟ್ಟುವ ಪದ್ಧತಿಯನ್ನು ಕಂಡರು. ಸ್ತ್ರೀಯ ಪಾದಗಳು ಚಿಕ್ಕದಾಗಿದ್ದಷ್ಟೂ ಅವಳ ಸೌಂದರ್ಯ ಅಧಿಕ ಎಂಬುದು ಚೀನೀಯರ ಭಾವನೆ. ಆದ್ದರಿಂದ ಬಾಲ್ಯದಿಂದಲೂ ಹೆಣ್ಣುಮಕ್ಕಳಿಗೆ ಬಲವಾದ ಬೂಟುಗಳನ್ನು ಹಾಕಿ ಭದ್ರ ಪಡಿಸಿ, ಪಾದಗಳು ಬೆಳೆಯದಂತೆ ಮಾಡುತ್ತಿದ್ದರು. ಆದ್ದರಿಂದ ಸ್ತ್ರೀಯರ ಪಾದಗಳು ಮಕ್ಕಳ ಪಾದದಂತೆಯೇ ಇರುತ್ತಿದ್ದವು. ಇವರು ನಡೆಯುವ ವಿಧಾನವನ್ನು ಕಂಡು ಸ್ವಾಮೀಜಿ ಬರೆಯು ತ್ತಾರೆ, “ಈ ಸ್ತ್ರೀಯರು ‘ನಡೆಯುತ್ತಾರೆ’ ಎನ್ನುವುದಕ್ಕಿಂತ ‘ಕುಪ್ಪಳಿಸುತ್ತಾರೆ’ ಎನ್ನಬಹುದು.” ಆದರೆ ೨ಂನೆ ಶತಮಾನದ ಪ್ರಾರಂಭದಿಂದೀಚೆಗೆ ಈ ಭಯಂಕರ ಪದ್ಧತಿ ನಿಂತುಹೋಗಿದೆ.

ಕ್ಯಾಂಟನ್ನಿನಲ್ಲಿ ಸ್ವಾಮೀಜಿ ಹಲವಾರು ಪ್ರಮುಖ ದೇವಾಲಯಗಳನ್ನು ವೀಕ್ಷಿಸಿದರು. ಇವುಗಳ ಲ್ಲೆಲ್ಲ ಅತ್ಯಂತ ದೊಡ್ಡದೆಂದರೆ, ಪ್ರಥಮ ಬೌದ್ಧ ಚಕ್ರವರ್ತಿಯ ಹಾಗೂ ಬುದ್ಧನ ಮೊದಲ ಐನೂರು ಶಿಷ್ಯರ ಹೆಸರಿನಲ್ಲಿ ಕಟ್ಟಿರುವ ದೇವಾಲಯ. ಇದರ ಮಧ್ಯ ಭಾಗದಲ್ಲಿ ಬುದ್ಧಭಗವಂತನ ಭವ್ಯ ಮೂರ್ತಿಯಿದೆ. ಅದರ ಬುಡದಲ್ಲಿ ಭಕ್ತಿಭಾವದಲ್ಲಿ ಧ್ಯಾನಮಗ್ನನಾಗಿ ಕುಳಿತ ಚಕ್ರವರ್ತಿಯ ವಿಗ್ರಹವಿದೆ. ಸುತ್ತಲೂ ಐನೂರು ಮಂದಿ ಶಿಷ್ಯರ ವಿಗ್ರಹಗಳಿವೆ. ಪ್ರಾಚೀನ ಬೌದ್ಧಕಾಲದ ಕೆತ್ತನೆ ಕೆಲಸವನ್ನು ಸ್ವಾಮೀಜಿ ಸೂಕ್ಷ್ಮವಾಗಿ ಅಧ್ಯಯಿಸಿದರು. ಬೌದ್ಧ ಹಾಗೂ ಹಿಂದೂ ದೇವಾಲಯಗಳ ನಡುವಿನ ಪರಸ್ಪರ ಸಾಮ್ಯವನ್ನೂ ವ್ಯತ್ಯಾಸವನ್ನೂ ಗಮನಿಸಿದ ಸ್ವಾಮೀಜಿ, ಇಲ್ಲಿನ ದೇವಾಲಯ ಗಳ ಸ್ವಂತಿಕೆ, ವೈಶಿಷ್ಟ್ಯಗಳನ್ನು ಮೆಚ್ಚಿಕೊಂಡರು.

ಇವೆಲ್ಲವುಗಳಿಗಿಂತ ಮುಖ್ಯವಾಗಿ, ಚೀನೀಯರ ಬೌದ್ಧ ಮಠವೊಂದನ್ನು ಸಂದರ್ಶಿಸ ಬೇಕೆಂಬುದು ಅವರ ಉತ್ಕಟೇಚ್ಛೆಯಾಗಿತ್ತು. ತಮ್ಮ ಜೊತೆಗಿದ್ದ ಮಾರ್ಗದರ್ಶಿಯನ್ನು ಅವರು ತಮ್ಮನ್ನು ಒಂದು ಬೌದ್ಧ ಮಠಕ್ಕೆ ಕರೆದುಕೊಂಡು ಹೋಗುವಂತೆ ಕೇಳಿದಾಗ, ಅವನು ಅದು ಸಾಧ್ಯವೇ ಇಲ್ಲ ಎಂದು ಹೇಳಿಬಿಟ್ಟ. ಏಕೆಂದರೆ ವಿದೇಶೀಯರಿಗೆ ಈ ಮಠಗಳೊಳಗೆ ಪ್ರವೇಶ ನಿಷಿದ್ಧವಾಗಿತ್ತು. ಆದರೆ ಇದನ್ನು ಕೇಳಿ, ಅಲ್ಲಿಗೆ ಹೋಗಬೇಕೆಂಬ ಅವರ ಇಚ್ಛೆ ಮತ್ತಷ್ಟು ಹೆಚ್ಚಾಯಿತು. ಅವರು ತಮ್ಮ ದುಭಾಷಿಯನ್ನು ಕೇಳಿದರು, “ಒಂದು ವೇಳೆ ಒಬ್ಬ ವಿದೇಶಿ ಅಲ್ಲಿಗೆ ಹೋಗಿಬಿಟ್ಟ ಎನ್ನು; ಆಗ ಏನಾಗುತ್ತದೆ?” ಆ ದುಭಾಷಿ ಆಶ್ಚರ್ಯಗೊಂಡು ಹೇಳಿದ, “ಆ! ಹಾಗೆ ಹೋದವರಿಗೆ ಒದೆ ಬೀಳುವುದು ಖಂಡಿತ.” ಆದರೆ ಸ್ವಾಮೀಜಿ ನಿರಾಶರಾಗಲಿಲ್ಲ. ತಾವೊಬ್ಬ ಹಿಂದೂ ಸಂನ್ಯಾಸಿಯೆಂದು ತಿಳಿಸಿಕೊಟ್ಟರೆ ತಮಗೇನೂ ತೊಂದರೆಯಾಗದು ಎಂದು ಅವರಿಗನ್ನಿಸಿತು. ಆದ್ದರಿಂದ ಅವರು ಒಂದು ಪ್ರಯತ್ನಮಾಡಿ ನೋಡಿಯೇಬಿಡುವುದೆಂದು ನಿರ್ಧರಿಸಿದರು. ಬಳಿಕ ತಮ್ಮ ಆಲೋಚನೆಯನ್ನು ಸಹಪ್ರಯಾಣಿಕರಿಗೂ ಮಾರ್ಗದರ್ಶಿಗೂ ಹೇಳಿ ದರು. ಅವರಿನ್ನೂ ಅನುಮಾನ ಮಾಡುತ್ತ ನಿಂತಾಗ ಸ್ವಾಮೀಜಿ ನಗುತ್ತ, “ಬನ್ನಿ, ಅವರೇನು ನಮ್ಮನ್ನು ಕೊಲ್ಲುತ್ತಾರೋ ಹೇಗೆಂದು ನೋಡಿಬಿಡೋಣ” ಎಂದು ಹೊರಡಿಸಿದರು. ಅಂತೂ ಒಂದು ಬೌದ್ಧ ಮಠದ ಬಳಿಗೆ ಬಂದರು. ಪ್ರವೇಶದ್ವಾರದ ಬಳಿ ಅವರನ್ನು ಯಾರೂ ತಡೆಯ ಲಿಲ್ಲ. ಆದರೆ ಸ್ವಲ್ಪ ಒಳಕ್ಕೆ ಹೋಗುವಷ್ಟರಲ್ಲಿ ಅವರ ಮಾರ್ಗದರ್ಶಿ ಕೂಗಿಕೊಂಡ, “ಓಡಿ, ಓಡಿ ಇಲ್ಲಿಂದ! ಅವರು ಬರುತ್ತಿದ್ದಾರೆ. ಅವರಿಗೆ ಸಿಟ್ಟು ಬಂದಿದೆ!” ನೋಡಿದರೆ, ಎಲ್ಲಿಂದಲೋ ಮೂವರು ಬಲಿಷ್ಠ ಭಿಕ್ಷುಗಳು, ಯಾವ ಮುನ್ನೆಚ್ಚರಿಕೆಯನ್ನೂ ಕೊಡದೆ ದೊಣ್ಣೆ ಹಿಡಿದು ಇವರತ್ತಲೇ ನುಗ್ಗುತ್ತಿದ್ದಾರೆ! ಇದನ್ನು ಕಂಡು ಎಲ್ಲರೂ ಹೌಹಾರಿ ಪಲಾಯನ ಸೂತ್ರ ಪಠಿಸಿ ದರು. ಸ್ವಾಮೀಜಿ ಮತ್ತು ಆ ದುಭಾಷಿ ಮಾತ್ರ ಅಲ್ಲಿಯೇ ನಿಂತರು. ಭಿಕ್ಷುಗಳು ಹತ್ತಿರ ಬಂದಂತೆ ಆ ದುಭಾಷಿಯೂ ಓಡಲು ಸಿದ್ಧನಾದ. ಆದರೆ ಸ್ವಾಮೀಜಿ ಅವನ ರಟ್ಟೆಯನ್ನು ಹಿಡಿದು ಮುಗುಳ್ನಗುತ್ತ, “ನೋಡು, ನೀನು ಇಲ್ಲಿಂದ ಓಡುವುದಕ್ಕೆ ಮೊದಲು ‘ಹಿಂದೂ ಯೋಗಿ’ ಎನ್ನುವುದಕ್ಕೆ ಚೀನೀ ಭಾಷೆಯಲ್ಲಿ ಏನೆನ್ನುತ್ತಾರೆ ಎಂಬುದನ್ನು ಹೇಳಿಬಿಡು” ಎಂದರು. ಅವನು ಅದನ್ನು ಹೇಳಿ ಅಲ್ಲಿಂದ ಪರಾರಿಯಾಗಿ ದೂರದಲ್ಲಿ ನಿಂತು ಕಾತರದಿಂದ ನೋಡತೊಡಗಿದ. ತಕ್ಷಣ ಸ್ವಾಮೀಜಿ ತಮ್ಮತ್ತ ಬರುತ್ತಿದ್ದ ಬೌದ್ಧ ಭಿಕ್ಷುಗಳಿಗೆ ಚೀನೀ ಭಾಷೆಯಲ್ಲಿ ‘ಹಿಂದೂ ಯೋಗಿ, ಹಿಂದೂ ಯೋಗಿ’ ಎಂದು ಕೂಗಿಹೇಳಿದರು. ಇದನ್ನು ಕೇಳಿ ಅವರು ಆಶ್ಚರ್ಯದಿಂದ ಮುಖಮುಖ ನೋಡಿಕೊಂಡು ಏನೋ ವಟಗುಟ್ಟಿಕೊಂಡರು. ಓಹೊ, ಏನದ್ಭುತ! ಅವರ ಮುಖಭಾವ ಸಂಪೂರ್ಣ ಬದಲಾಗಿ ಹೋಯಿತು. ಈಗ ಅವರ ಮುಖದಲ್ಲಿ ಸಿಟ್ಟಿಗೆ ಬದಲಾಗಿ ಭಯಭಕ್ತಿಮಿಶ್ರಿತವಾದ ಗೌರವ ಕಾಣಿಸಿತು. ಭಿಕ್ಷುಗಳು ತಮ್ಮ ದೊಣ್ಣೆಗಳನ್ನು ಬಿಸಾಕಿ ಸ್ವಾಮೀಜಿಯ ಕಾಲಿಗೆ ಬಿದ್ದರು. ಬಳಿಕ ಎದ್ದು ಅತ್ಯಂತ ಪೂಜ್ಯಭಾವದಿಂದ ಕೈಜೋಡಿಸಿ, ತಮ್ಮ ಭಾಷೆಯಲ್ಲಿ ಗಟ್ಟಿಯಾಗಿ ಅದೇನೋ ಹೇಳಿದರು. ಸ್ವಾಮೀಜಿಗೆ ಅವರು ಹೇಳಿದ್ದೊಂದೂ ಅರ್ಥ ವಾಗಲಿಲ್ಲ. ಆದರೆ ‘ಕಬಚ್​’ ಎಂಬ ಒಂದು ಶಬ್ದ ಮಾತ್ರ ಸ್ಪಷ್ಟವಾಗಿ ಕೇಳಿಸಿತು. ಬಂಗಾಳಿಗಳು ‘ವ’ಕಾರಕ್ಕೆ ಬದಲಾಗಿ ‘ಬ’ಕಾರ ಬಳಸುತ್ತಾರಾದ್ದರಿಂದ ‘ಕಬಚ್​’ ಎಂದರೆ ‘ಕವಚ’ವೇ ಇರ ಬೇಕೆಂದು ಅವರಿಗೆ ತಕ್ಷಣ ಹೊಳೆಯಿತು. (ಕವಚವೆಂದರೆ ‘ತಾಯಿತ’ ಅಥವಾ ‘ಯಂತ್ರ’.) ಆದರೂ ಒಮ್ಮೆ ತಮ್ಮ ದುಭಾಷಿಯನ್ನು ಕೇಳಿ ದೃಢಪಡಿಸಿಕೊಳ್ಳಲು ಅವನತ್ತ ತಿರುಗಿ “ಇವರೇನು ಹೇಳುತ್ತಿದ್ದಾರೆ?” ಎಂದು ಕೂಗಿ ಕೇಳಿದರು. ಇಲ್ಲಿ ನಡೆಯುತ್ತಿರುವ ವಿಚಿತ್ರವನ್ನೆಲ್ಲ ಕಂಡು ಆತ ಬೆಪ್ಪಾಗಿ ನಿಂತಿದ್ದ. ಪಾಪ, ತನ್ನ ‘ಸರ್ವಿಸಿ’ನಲ್ಲೇ ಆತ ಇಂಥದೇನನ್ನೂ ನೋಡಿರಲಿಲ್ಲ. ಆ ಭಿಕ್ಷುಗಳ ಮಾತನ್ನು ಕೇಳಿದ್ದ ಆತ ಅಲ್ಲಿಂದಲೇ ಹೇಳಿದ, “ಅವರಿಗೆ ತಾಯಿತ ಬೇಕಂತೆ! ಭೂತ ಪಿಶಾಚಿಗಳನ್ನು ದೂರವಿಡುವುದಕ್ಕೆ ಒಂದು ತಾಯಿತ ಬೇಕಂತೆ!” ಒಂದು ಕ್ಷಣ ಸ್ವಾಮೀಜಿ ಅವಾಕ್ಕಾದರು. ಅವರು ಇಂತಹದನ್ನು ಎಂದೂ ಯಾರಿಗೂ ಕೊಟ್ಟಿರಲೂ ಇಲ್ಲ. ತಾವೂ ಹಾಕಿ ಕೊಂಡಿರಲಿಲ್ಲ. ಬಳಿಕ ತಮ್ಮ ಜೇಬಿನಿಂದ ಒಂದು ಕಾಗದ ತೆಗೆದು ಅದನ್ನು ಅನೇಕ ಚೂರುಗಳಾಗಿ ಹರಿದು ಎಲ್ಲದರ ಮೇಲೂ ದೇವನಾಗರಿಯಲ್ಲಿ ‘ಓಂ’ ಎಂದು ಬರೆದು, ಮಡಿಸಿ ಕೊಟ್ಟರು. ಬೌದ್ಧ ಭಿಕ್ಷುಗಳು ಆ ಚೀಟಿಗಳನ್ನು ಅತ್ಯಂತ ಭಕ್ತಿಭಾವದಿಂದ ತಲೆಗೆ ಮುಟ್ಟಿಸಿಕೊಂಡು ಜೋಪಾನವಾಗಿ ಎತ್ತಿಟ್ಟುಕೊಂಡರು. ಬಳಿಕ ಅವರನ್ನು ಮಠದೊಳಗೆ ಕರೆದೊಯ್ದರು.

ಸ್ವಾಮೀಜಿ, ಮಠದ ಒಳಭಾಗವನ್ನು ಸೂಕ್ಷ್ಮವಾಗಿ ಗಮನಿಸುತ್ತ ನಡೆದರು. ಕಟ್ಟಡದ ಅತ್ಯಂತ ಒಳಭಾಗದಲ್ಲಿ ಜೋಪಾನವಾಗಿ ಇರಿಸಿದ್ದ ಹಲವಾರು ಸಂಸ್ಕೃತ ಗ್ರಂಥಗಳ ಹಸ್ತಪ್ರತಿಗಳನ್ನು ಅವರಿಗೆ ತೋರಿಸಲಾಯಿತು. ಆಶ್ಚರ್ಯ! ಆ ಹಸ್ತಪ್ರತಿಗಳನ್ನು ಹಳೆಯ ಕಾಲದ ಬಂಗಾಳೀ ಲಿಪಿ ಯಲ್ಲಿ ಬರೆಯಲಾಗಿತ್ತು! ಆಗ ಅವರಿಗೆ ತಾವು ಮೊದಲು ನೋಡಿದ್ದ ಆ ಐನೂರು ಬೌದ್ಧ ಅನುಯಾಯಿಗಳ ಮುಖಚರ್ಯೆಯು ನಿಸ್ಸಂದೇಹವಾಗಿ ಬಂಗಾಳಿಗಳ ಮುಖವನ್ನು ಹೋಲು ತ್ತಿತ್ತು ಎಂಬುದು ತಕ್ಷಣ ಹೊಳೆಯಿತು. ಹಿಂದೆಯೇ ಸ್ವಾಮೀಜಿ ಚೀನಾದ ಬೌದ್ಧ ಸಂಪ್ರದಾಯ ವನ್ನು ಅಧ್ಯಯನ ಮಾಡಿದ್ದರು. ಈಗ ತಾವು ಕಣ್ಣಾರೆ ಕಂಡದ್ದರ ಬಗ್ಗೆ ಆಲೋಚಿಸುತ್ತಿದ್ದಂತೆ ಅವರೊಂದು ತೀರ್ಮಾನಕ್ಕೆ ಬಂದರು–ಏನೆಂದರೆ, ಹಿಂದೆ ಒಂದಾನೊಂದು ಕಾಲದಲ್ಲಿ ಚೀನಾಕ್ಕೂ ಬಂಗಾಳಕ್ಕೂ ನಿಕಟ ಸಂಪರ್ಕವಿತ್ತು. ಸಹಸ್ರಾರು ಬಂಗಾಳೀ ಭಿಕ್ಷುಗಳು ಚೀನಾಕ್ಕೆ ಬಂದರಲ್ಲದೆ ಬುದ್ಧನ ಸಂದೇಶವನ್ನು ತಂದರು; ಮತ್ತು ಈ ಮೂಲಕ ಭಾರತೀಯ ಚಿಂತನೆಯು ಚೀನೀ ನಾಗರಿಕತೆಯ ಮೇಲೆ ಮಹತ್ವಪೂರ್ಣ ಪ್ರಭಾವವನ್ನು ಬೀರಿತು, ಎಂದು.

ಹಾಂಗ್​ಕಾಂಗ್​ನಿಂದ ಹೊರಟ ‘ಪೆನಿನ್ಸುಲಾರ್​’ ಜಪಾನಿನ ನಾಗಸಾಕಿಯನ್ನು ತಲುಪಿ ಲಂಗರು ಹಾಕಿತು. ನಗರವನ್ನು ನೋಡಿಕೊಂಡು ಬರಲು ಹೊರಟ ಸ್ವಾಮೀಜಿ, ಅಲ್ಲಿನ ನೋಟವನ್ನು ಕಂಡು ನಿಬ್ಬೆರಗಾದರು. ಅಲ್ಲಿನ ಪ್ರತಿಯೊಂದು ಅಂಶವೂ ಅವರಿಗೆ ಇಷ್ಟವಾಯಿತು. ಜಪಾನೀಯರನ್ನು ಕೊಂಡಾಡುತ್ತ ಅವರು ತಮ್ಮ ಮದರಾಸೀ ಶಿಷ್ಯರಿಗೆ ಬರೆಯುತ್ತಾರೆ:

“ಈ ಜಪಾನೀಯರು ಜಗತ್ತಿನಲ್ಲೇ ಅತ್ಯಂತ ಶುಚಿಯಾದವರು. ಇಲ್ಲಿ ಪ್ರತಿಯೊಂದು ತುಂಬ ಚೊಕ್ಕಟ, ಅಚ್ಚುಕಟ್ಟು. ಹೆಚ್ಚುಕಡಿಮೆ ಪ್ರತಿಯೊಂದು ರಸ್ತೆಯೂ ಅಗಲವಾಗಿದೆ, ನೇರವಾಗಿದೆ, ಮತ್ತು ಸಮತಟ್ಟಾಗಿದೆ. ಇವರ ಪುಟ್ಟ ಮನೆಗಳು ಹಕ್ಕಿಯ ಪಂಜರದಂತೆ. ಪ್ರತಿಯೊಂದು ಹಳ್ಳಿ-ಪಟ್ಟಣದ ಹಿನ್ನೆಲೆಯಲ್ಲೂ ಪೀತದಾರು ವಕ್ಷಗಳಿಂದಾವೃತವಾದ ನಿತ್ಯಹರಿದ್ವರ್ಣದ ಸುಂದರ ಬೆಟ್ಟವಿರುತ್ತದೆ. ಈ ಜಪಾನೀಯರ ವೇಷಭೂಷಣಗಳು, ನಡಿಗೆ, ಮಾತುಕತೆ, ಅಭಿರುಚಿಗಳು, ಹಾವಭಾವಗಳು–ಪ್ರತಿಯೊಂದೂ ಕಣ್ಣಗೆ ಹಬ್ಬವುಂಟುಮಾಡುವಂಥವು!.... ”

ಇಲ್ಲಿನ ಜನಜೀವನವನ್ನು ಕಂಡು ತುಂಬ ಕುತೂಹಲಗೊಂಡ ಸ್ವಾಮೀಜಿ, ಜಪಾನಿನ ಒಳಭಾಗಗಗಳಲ್ಲಿ ಸಂದರ್ಶಿಸಲು ಇಚ್ಛಿಸಿದರು. ಆದ್ದರಿಂದ ಅವರು ಕೋಬೆ ಎಂಬಲ್ಲಿ ಹಡಗಿ ನಿಂದ ಇಳಿದು, ರಸ್ತೆ ಮಾರ್ಗವಾಗಿ ಯೊಕೊಹಾಮಕ್ಕೆ ಹೊರಟರು. ತಮ್ಮ ಮುಂದಿನ ಪ್ರಯಾಣಕ್ಕೆ ಅವರು ಹಡಗು ಹತ್ತಬೇಕಾಗಿದ್ದದ್ದು ಯೊಕೊಹಾಮದಲ್ಲೇ. ದಾರಿಯಲ್ಲಿ ಅವರು ಮೂರು ಪ್ರಮುಖ ನಗರಗಳಿಗೆ ಭೇಟಿಯಿತ್ತರು–ಮುಖ್ಯ ಕೈಗಾರಿಕಾ ಕೇಂದ್ರವಾದ ಒಸಾಕ, ಹಿಂದಿನ ರಾಜಧಾನಿ ಕ್ಯೋಟೊ ಹಾಗೂ ಇಂದಿನ ರಾಜಧಾನಿ ಟೋಕಿಯೊ. ಜಪಾನಿನ ಭೇಟಿಯ ಈ ಸಂದರ್ಭದಲ್ಲಿ ಸ್ವಾಮೀಜಿ, ಇಲ್ಲಿನ ಜನಜೀವನದ ಮೂಲಭೂತ ಅಂಶಗಳ ಕಡೆಗೆ ವಿಶೇಷ ಗಮನ ಕೊಟ್ಟರು. ಹಾಗೂ ಅವರ ಸಂಸ್ಕೃತಿ ಮತ್ತು ಸಂಪ್ರದಾಯಗಳನ್ನು ಅರಿಯುವ ಪ್ರಯತ್ನ ಮಾಡಿದರು. ಆದರೆ ಎಲ್ಲಕ್ಕಿಂತ ಹೆಚ್ಚಾಗಿ ಅವರನ್ನು ಆಕರ್ಷಿಸಿದ್ದೆಂದರೆ, ಜೀವನದ ಪ್ರತಿಯೊಂದು ರಂಗದಲ್ಲೂ ಎದ್ದುಕಾಣುತ್ತಿರುವ ಪ್ರಗತಿಯ ಹಂಬಲ; ಅಭಿವೃದ್ಧಿಯ ಕುರಿತಾದ ತೀವ್ರ ಉತ್ಸಾಹ. ಕೈಗಾರಿಕೆ, ವಿದ್ಯಾಭ್ಯಾಸ, ವ್ಯವಸಾಯ ಮೊದಲಾದ ಪ್ರತಿಯೊಂದು ಕ್ಷೇತ್ರದಲ್ಲೂ ಮುನ್ನಡೆ ಸಾಧಿಸಿ ಸ್ವಾವಲಂಬಿಗಳಾಗಬೇಕೆಂಬ ಅವರ ಛಲವನ್ನು ಸ್ವಾಮೀಜಿ ಹೃತ್ಪೂರ್ವಕವಾಗಿ ಮೆಚ್ಚಿದರು. ಆದರೆ ಅವರಿಗೆ ಪ್ರತಿಕ್ಷಣವೂ ಭಾರತದ ಚಿಂತೆಯೇ ಎಂದು ಬೇರೆ ಹೇಳಬೇಕಾಗಿಲ್ಲ! ‘ಇವರಿಂದ ತಮ್ಮವರು ಕಲಿತುಕೊಳ್ಳಬೇಕಾದದ್ದು ಅದೆಷ್ಟಿದೆ!’ ಎಂದು ಅವರ ಮನಸ್ಸು ಮತ್ತೆ ಮತ್ತೆ ಆಲೋಚಿಸುತ್ತಿತ್ತು. ತಮ್ಮ ಶಿಷ್ಯರಿಗೆ ಬರೆದ ಪತ್ರದಲ್ಲಿ ಸ್ವಾಮೀಜಿ, ಜಪಾನೀಯರ ಮುನ್ನಡೆಯ ಬಗ್ಗೆ ತಿಳಿಸುತ್ತಾರೆ:

“ಈ ಜಪಾನೀಯರು ಆಧುನಿಕ ಯುಗದ ಬದಲಾದ ಜೀವನಾವಶ್ಯಕತೆಗಳ ಬಗ್ಗೆ ಸಂಪೂರ್ಣ ಎಚ್ಚತ್ತಿರುವಂತೆ ತೋರುತ್ತದೆ. ಈಗ ಅವರು ತಾವೇ ಅನ್ವೇಷಿಸಿದ ಹೊಸ ಬಗೆಯ ಕೋವಿಗಳಿಂದ ಸಜ್ಜಾದ ಅತ್ಯಾಧುನಿಕ ಸೈನಿಕಪಡೆಯನ್ನು ನಿರ್ಮಿಸಿಕೊಂಡಿದ್ದಾರೆ. ಅಲ್ಲದೆ ತಮ್ಮ ನೌಕಾಪಡೆಯನ್ನು ಬೆಳೆಸಿಕೊಂಡು ಹೋಗುತ್ತಲೇ ಇದ್ದಾರೆ... ಇಲ್ಲಿನ ಬೆಂಕಿಕಡ್ಡಿಯ ಕಾರ್ಖಾನೆಗಳು ನೋಡುವು ದಕ್ಕೇ ಅಷ್ಟು ಚಂದ! ಅಂತೂ ಅವರು ತಮ್ಮ ರಾಷ್ಟ್ರಕ್ಕೆ ಬೇಕಾದುದೆಲ್ಲವನ್ನೂ ತಾವೇ ಉತ್ಪಾದಿ ಸುವ ಹಟತೊಟ್ಟಂತಿದೆ....

ಸರ್ವಸಂಗ ಪರಿತ್ಯಾಗಿಯಾದ ಸಂನ್ಯಾಸಿಯೊಬ್ಬ ಲೌಕಿಕ ಜೀವನದ ಸಮೃದ್ಧಿ-ಸುವ್ಯವಸ್ಥೆಗಳಿಗೆ ಬೇಕಾದ ಆವಶ್ಯಕತೆಗಳನ್ನು ಎತ್ತಿ ತೋರಿಸುವ ಸ್ವಾರಸ್ಯವನ್ನು ನಾವಿಲ್ಲಿ ಕಾಣಬಹುದಾಗಿದೆ. ಅಲ್ಲದೆ ಭಾರತದ ಸದ್ಯದ ದುಃಸ್ಥಿತಿಗೆ ಬಹುಮಟ್ಟಿಗೆ ಕಾರಣವಾದ ಭಾರತೀಯರ ಆಲಸ್ಯ ಹಾಗೂ ಕೂಪ ಮಂಡೂಕ ಬುದ್ಧಿಯನ್ನು ತೀಕ್ಷ್ಣವಾಗಿ ಖಂಡಿಸಿ ಅವರು ಬರೆಯುತ್ತಾರೆ:

“ನಮ್ಮ ದೇಶದ ಯುವಕರು ಪ್ರತಿ ವರ್ಷವೂ ಹೆಚ್ಚೆಚ್ಚು ಸಂಖ್ಯೆಯಲ್ಲಿ ಚೀನಾ ಹಾಗೂ ಜಪಾನುಗಳಿಗೆ ಭೇಟಿ ನೀಡುವಂತಾಗಬೇಕು. ಜಪಾನೀಯರ ಪಾಲಿಗಂತೂ ಭಾರತವೆಂದರೆ ಇನ್ನೂ ಎಲ್ಲ ಶ್ರೇಷ್ಠ-ಉನ್ನತ ವಿಚಾರಗಳ ಸ್ವಪ್ನಲೋಕವೇ ಆಗಿದೆ. ಆದರೆ ನೀವು? ನೀವೆಂಥವರು?... ಜೀವನವಿಡೀ ಕಾಡುಹರಟೆಯಲ್ಲೇ ಮುಳುಗಿರುವವರು. ಕೇವಲ ಹರಟೆಮಲ್ಲರು. ಇನ್ನೆಂಥ ವರು ನೀವು? ಬನ್ನಿ ಇಲ್ಲಿಗೆ. ಈ ಜನರನ್ನು ನೋಡಿ. ಬಳಿಕ ನಾಚಿಕೆಯಿಂದ ಮುಖ ಮುಚ್ಚಿ ಕೊಂಡು ಕುಳಿತುಕೊಳ್ಳಿ. ಸಮುದ್ರ ದಾಟಿದರೆ ನಿಮ್ಮ ಜಾತಿ ಕೆಟ್ಟುಹೋಗುತ್ತದಲ್ಲವೆ....? ಬುದ್ಧಿಕೆಟ್ಟ ಅರಳುಮರಳು ಜನಾಂಗ! ಶತಶತಮಾನಗಳಿಂದಲೂ ಮೂಢನಂಬಿಕೆ-ಕಂದಾಚಾರ ಗಳ ಕಂತೆಗಳನ್ನೇ ತಲೆಯ ಮೇಲೆ ಹೊತ್ತು ಕುಳಿತಿರುವವರು!ಈ ಆಹಾರವನ್ನು ಮುಟ್ಟಬಹುದೆ, ಆ ಆಹಾರವನ್ನು ಮುಟ್ಟಬಾರದೆ ಎಂಬ ಬಹಳ ದೊಡ್ಡ ವಿಮರ್ಶೆಯಲ್ಲಿ ನೂರಾರು ವರ್ಷಗಳಿಂದ ಮುಳುಗಿದ್ದೀರಲ್ಲ–ಎಂಥ ಮನುಷ್ಯರು ನೀವೆಲ್ಲ! ಯುಗಯುಗಗಳಿಂದಲೂ ಅಮಾನುಷ ಸಾಮಾಜಿಕ ದಬ್ಬಾಳಿಕೆ ನಡೆಸಿ ಮಾನವೀಯತೆಯನ್ನೇ ಕಳೆದುಕೊಂಡಿದ್ದೀರಲ್ಲ–ನೀವು ಇನ್ನೆಂಥ ವರು? ಮತ್ತೆ ಈಗ ನೀವು ಮಾಡುತ್ತಿರುವುದಾದರೂ ಏನು? ಪರಕೀಯರು ಬರೆದ ಒಂದು ಪುಸ್ತಕವನ್ನು ಕೈಯಲ್ಲಿ ಹಿಡಿದು, ಆ ಐರೋಪ್ಯ ಮೆದುಳಿನಲ್ಲಿ ಉದ್ಭವಿಸಿದ ಆಲೋಚನೆಗಳ ಕೆಲವು ಚೂರುಪಾರುಗಳನ್ನೇ ಮೆಲುಕು ಹಾಕುತ್ತ ಸಮುದ್ರತೀರದಲ್ಲಿ ವಿಹರಿಸುವುದು! ನಿಮ್ಮ ತಲೆಯಲ್ಲಿ ರುವ ಅತ್ಯುನ್ನತ ಧ್ಯೇಯವೆಂದರೆ ತಿಂಗಳಿಗೆ ಮೂವತ್ತು ರೂಪಾಯಿ ಸಂಬಳದ ಗುಮಾಸ್ತೆಗಿರಿ! ಅಥವಾ ಇನ್ನೂ ಹೆಚ್ಚೆಂದರೆ ವಕೀಲನಾಗುವುದು–ಇದೇ ಭಾರತೀಯ ಯುವಕನ ಪರಮಾದರ್ಶ. ಇಷ್ಟೇ ನಿಮ್ಮ ಬದುಕು. ನಿಮ್ಮನ್ನೂ ನಿಮ್ಮ ಪುಸ್ತಕ-ಗೌನು ಪದವಿಗಳನ್ನೂ ಮುಳುಗಿಸುವಷ್ಟು ನೀರಿಲ್ಲವೆ ಆ ನಿಮ್ಮ ಸಮುದ್ರದಲ್ಲಿ?”

ಸ್ವಾಮೀಜಿಯ ಈ ಪತ್ರ ಯೊಕೊಹಾಮದಿಂದ ಮದರಾಸನ್ನು ತಲುಪಿದಾಗ ಅದು ಉಂಟು ಮಾಡಿದ ಪರಿಣಾಮವನ್ನು ನಾವು ಊಹಿಸಬಹುದು. ಅವರ ಉತ್ಸಾಹ-ಸ್ಫೂರ್ತಿಯ ಮಾತುಗಳನ್ನು ಓದಿ ಈ ಯುವಕರು ತಾವೂ ಉತ್ಸಾಹಿತರಾದರು. ಅವರ ಚಾಟಿಯೇಟಿನಂತಹ ಟೀಕೆಯನ್ನು ಕಂಡು ಅವಮಾನದಿಂದ ತಲೆ ತಗ್ಗಿಸಿದರು. ಅಲ್ಲದೆ ಸ್ವಾಮೀಜಿಯ ವ್ಯಂಗ್ಯದ ಹಿಂದೆ ಅಡಕವಾಗಿ ರುವ ದುಃಖವನ್ನೂ ಗಮನಿಸಿದರು. ಅವರ ಪ್ರತಿಯೊಂದು ಮಾತೂ ಮಾತೃಭೂಮಿಯ ಬಗ್ಗೆ ಅವರಿಗಿದ್ದ ತೀವ್ರ ಕಳಕಳಿಯನ್ನು ಪ್ರತಿಬಿಂಬಿಸುತ್ತಿತ್ತು. ಈ ಸಂನ್ಯಾಸಿಯ ದೇಶಪ್ರೇಮದಲ್ಲಿ ಒಂದು ತುಣುಕಾದರೂ ತಮ್ಮಲ್ಲಿಲ್ಲವೆಂಬುದನ್ನು ಕಂಡುಕೊಂಡ ಮದರಾಸೀ ಯುವಕರು, ಹೊಸ ಉತ್ಸಾಹದಿಂದ ತ್ಯಾಗಕ್ಕೆ ಸನ್ನದ್ಧರಾದರು.

ಜಪಾನಿನ ಜನಜೀವನವನ್ನೂ ಸರ್ವತೋಮುಖ ಪ್ರಗತಿಯನ್ನೂ ವಿವರವಾಗಿ ಅಧ್ಯಯನ ಮಾಡಿದ ಸ್ವಾಮೀಜಿ, ಅಲ್ಲಿನ ದೇವಸ್ಥಾನಗಳನ್ನೂ ಪೂಜಾ ವಿಧಿಗಳನ್ನೂ ಗಮನಿಸಲು ಉತ್ಸುಕ ರಾಗಿದ್ದರು. ಆದ್ದರಿಂದ ಅವರು, ತಾವು ಭೇಟಿ ನೀಡಿದ ನಗರಗಳಲ್ಲೆಲ್ಲ ಮುಖ್ಯ ದೇವಸ್ಥಾನಗಳನ್ನು ಸಂದರ್ಶಿಸಲು ಮರೆಯಲಿಲ್ಲ. ಈ ದೇವಸ್ಥಾನಗಳ ವಾಸ್ತುಶಿಲ್ಪವನ್ನೂ ಪೂಜಾಕ್ರಮಗಳನ್ನೂ ಸ್ವಾಮೀಜಿ ಸೂಕ್ಷ್ಮವಾಗಿ ಅವಲೋಕಿಸಿದರು. ಇಲ್ಲಿಯೂ ದೇವಸ್ಥಾನಗಳ ಗೋಡೆಗಳ ಮೇಲೆ ಹಳೆಯ ಬಂಗಾಳೀ ಲಿಪಿಯಲ್ಲಿ ಸಂಸ್ಕೃತ ಮಂತ್ರಗಳನ್ನು ಬರೆದಿರುವುದನ್ನು ಕಂಡು ಅಚ್ಚರಿ ಗೊಂಡರು. ಆದರೆ ಕೆಲವೇ ಕೆಲವು ಅರ್ಚಕರಿಗೆ ಮಾತ್ರ ಸಂಸ್ಕೃತ ಗೊತ್ತಿತ್ತು. ಆಧುನೀಕರಣದ ಉತ್ಸಾಹವು ಅರ್ಚಕರಲ್ಲೂ ಕಂಡುಬರುತ್ತಿದ್ದುದನ್ನು ಸ್ವಾಮೀಜಿ ಗಮನಿಸಿದರು. ಭಾರತದ ಬಗ್ಗೆ ಜಪಾನೀಯರು ಇನ್ನೂ ಅತ್ಯಂತ ಪೂಜ್ಯ ಭಾವವನ್ನಿಟ್ಟುಕೊಂಡಿರುವುದನ್ನು ಕಂಡು ಅವರಿಗೆ ತುಂಬ ಸಂತೋಷವಾಯಿತು.

ಕೋಬೆಯಿಂದ ಯೊಕೊಹಾಮ ನಗರಕ್ಕೆ ನೆಲಮಾರ್ಗವಾಗಿ ಪ್ರಯಾಣ ಮಾಡಿದ ಸ್ವಾಮೀಜಿ, ಯೊಕೊಹಾಮದಲ್ಲಿ ೬ಂಂಂ ಟನ್ ತೂಕದ ‘ಎಂಪ್ರೆಸ್ ಆಫ್ ಇಂಡಿಯ’ ಎಂಬ ಹಡಗನ್ನೇರಿ ದರು. ಈ ಹಡಗು ಜುಲೈ ೧೪ರಂದು ಹೊರಟು ವ್ಯಾಂಕೋವರ್ ಕಡೆಗೆ ಸಾಗಿತು. ಆದರೆ ಸ್ವಾಮೀಜಿಯ ಪಾಲಿಗೆ ಈ ಪ್ರಯಾಣ ಸುಖಕರವಾಗಿರಲಿಲ್ಲ. ಏಕೆಂದರೆ ಹಡಗು ಉತ್ತರಕ್ಕೆ ಸಾಗಿ ದಂತೆ ಕೊರೆಯುವ ಚಳಿಗಾಳಿ ಬೀಸಲಾರಂಭಿಸಿತು. ಅವರ ಶಿಷ್ಯರು ಬೆಲೆಬಾಳುವ ರೇಷ್ಮೆಯ ವಸ್ತ್ರಗಳನ್ನು ಕೊಡಿಸಿದ್ದರಾದರೂ ಬೆಚ್ಚನೆಯ ಉಣ್ಣೆಬಟ್ಟೆಗಳನ್ನು ಕೊಡಿಸಿರಲಿಲ್ಲ. ಉತ್ತರ ಪೆಸಿಫಿಕ್ ಸಾಗರದಲ್ಲಿನ ಪ್ರಯಾಣ ಕಾಲದಲ್ಲಿ ಹಾಗೂ ಅಮೆರಿಕೆಯಲ್ಲಿ ಅವರು ಕೊರೆಯುವ ಚಳಿಯನ್ನು ಎದುರಿಸಬೇಕಾಗುತ್ತದೆ ಎಂಬುದು ಯಾರ ಬುದ್ಧಿಗೂ ಹೊಳೆದಿರಲಿಲ್ಲ. ಆದರೆ ಸ್ವಾಮೀಜಿ ಕಷ್ಟಪಡುತ್ತಿರುವುದನ್ನು ಕಂಡ ಹಡಗಿನ ಕ್ಯಾಪ್ಟನ್, ಅವರಿಗೆ ಬೆಚ್ಚನೆಯ ಬಟ್ಟೆಗಳನ್ನು ತಾನೇ ಒದಗಿಸಿದನಂತೆ.

ಇದೇ ಹಡಗಿನಲ್ಲಿ ಭಾರತದ ಕೈಗಾರಿಕಾ ಸಾಮ್ರಾಟ್ ಜೆಮ್​ಶೆಟ್​ಜಿ ಟಾಟಾರವರೂ ಪ್ರಯಾಣ ಮಾಡುತ್ತಿದ್ದರು. ಟಾಟಾರವರಿಗೆ ಸ್ವಾಮೀಜಿಯ ಪರಿಚಯವಿರಲಿಲ್ಲವಾದರೂ, ಅವರ ಅಸಾಮಾನ್ಯ ವ್ಯಕ್ತಿತ್ವದಿಂದ ಕೂಡಲೇ ಆಕರ್ಷಿತರಾದರು. ಜಪಾನಿನ ಪ್ರಗತಿಯನ್ನು ಕಣ್ಣಾರೆ ಕಂಡು ತೀವ್ರವಾಗಿ ಪ್ರಭಾವಿತರಾಗಿದ್ದ ಸ್ವಾಮೀಜಿ, ಅಂತಹ ಕೈಗಾರಿಕಾ ಕ್ರಾಂತಿಯು ಭಾರತದಲ್ಲೂ ನಡೆಯಬೇಕೆಂಬ ತಮ್ಮ ಆಶಯವನ್ನು ವ್ಯಕ್ತಪಡಿಸಿದರು. ಅಲ್ಲದೆ ಭಾರತದ ಪುನರುತ್ಥಾನ ಕಾರ್ಯವನ್ನು ಕೈಗೊಳ್ಳುವಾಗ ಅದರ ಧಾರ್ಮಿಕ ಹಿನ್ನೆಲೆಯನ್ನು ಕಡೆಗಣಿಸಿಬಾರದೆಂದು ಒತ್ತಿ ಹೇಳಿದರು. ಅವರ ಮಾತುಗಳನ್ನೆಲ್ಲ ಆಲಿಸಿದ ಟಾಟಾರವರು, ಆ ಅಭಿಪ್ರಾಯಗಳನ್ನು ಗಂಭೀರ ವಾಗಿ ಪರಿಗಣಿಸಿದರೆಂಬುದು ಮುಂದೊಮ್ಮೆ ಟಾಟಾರವರು ಬರೆಯುವ ಒಂದು ಪತ್ರದಿಂದ ತಿಳಿದುಬರುತ್ತದೆ.

ಹತ್ತು ದಿನಗಳ ಕಾಲ ಶಾಂತಸಾಗರದಲ್ಲಿ ನಿರಂತರ ಪ್ರಯಾಣ ಮಾಡಿದ ಹಡಗು, ಶಾಂತವಾಗಿ ಉತ್ತರ ಅಮೆರಿಕವನ್ನು ಸಮೀಪಿಸಿತು.

