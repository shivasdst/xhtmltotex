
\chapter{ಧರ್ಮಪ್ರಸಾರ}

\noindent

ತಮ್ಮ ‘ಕೇಂದ್ರಸ್ಥಾನ’ವಾದ ಹೇಲ್ ಕುಟುಂಬದವರ ಮನೆಯಲ್ಲಿ ಸುಮಾರು ಒಂದೂವರೆ ತಿಂಗಳ ಬೇಸಿಗೆ ‘ರಜೆ’ಯನ್ನು ಕಳೆದ ಸ್ವಾಮೀಜಿ, ಜೂನ್ ೨೮ರಂದು ಶಿಕಾಗೋದಿಂದ ನ್ಯೂಯಾರ್ಕಿಗೆ ಬಂದರು. ಇಲ್ಲಿ ತಮಗೆ ಪರಿಚಿತರಾಗಿದ್ದ ಕೆಲವರೊಂದಿಗೆ ಮೂರ್ನಾಲ್ಕು ದಿನ ಗಳನ್ನು ಕಳೆದರು. ಆದರೆ ಅದಿನ್ನೂ ನಡು ಬೇಸಿಗೆಯ ಕಾಲವಾದ್ದರಿಂದ ಉಪನ್ಯಾಸದ ಪುತು ಪ್ರಾರಂಭವಾಗಿರಲಿಲ್ಲ. ಆದ್ದರಿಂದ ಅವರು ಬೇಗನೆ ನ್ಯೂಯಾರ್ಕಿನಿಂದ ಹೊರಟುಬಿಟ್ಟರು. ಬಹುಶಃ ಅವರು ಸುತ್ತಲಿನ ಕೆಲವು ಊರುಗಳಿಗೆ ಹೋಗಿದ್ದಿರಬೇಕು. ಜುಲೈ ತಿಂಗಳ ಮಧ್ಯದ ಹೊತ್ತಿಗೆ ಅವರು ನ್ಯೂಯಾರ್ಕಿಗೆ ಹಿಂದಿರುಗಿದರು. ಈಗ ಅವರು ಎಗ್​ಬರ್ಟ್ ಗರ್ನ್​ಸೇ ದಂಪತಿಗಳ ಅತಿಥಿಯಾಗಿ ಹಡ್ಸನ್ ನದಿಯ ತೀರದಲ್ಲಿನ ಅವರ ಬೇಸಿಗೆಮನೆಯಲ್ಲಿ ಇಳಿದು ಕೊಂಡರು. ನಾವು ಈ ಹಿಂದೆಯೇ ನೋಡಿದಂತೆ, ಆ ವರ್ಷದ ಏಪ್ರಿಲ್​ನಲ್ಲಿ ಸ್ವಾಮೀಜಿ ಈ ದಂಪತಿಗಳನ್ನು ಭೇಟಿಯಾಗಿದ್ದರು. ಆಗಿನಿಂದಲೂ ಇವರು ಸ್ವಾಮೀಜಿಯನ್ನು ದೈವಾಂಶ ಸಂಭೂತ ವ್ಯಕ್ತಿಯೆಂದು ಭಾವಿಸಿದ್ದರು. ಅಲ್ಲದೆ ಅವರನ್ನು ತಮ್ಮ ಸ್ವಂತ ಮಗನಂತೆ ಪ್ರೀತಿಸು ತ್ತಿದ್ದರು. ನದೀತೀರದ ಸುಂದರ ಬಂಗಲೆಯಲ್ಲಿ ಕೆಲದಿನಗಳನ್ನು ಆನಂದದಿಂದ ಕಳೆದ ಸ್ವಾಮೀಜಿ, ಈ ದಂಪತಿಗಳ ಪುತ್ರಿಯ ಆಹ್ವಾನವನ್ನು ಮನ್ನಿಸಿ, ಸ್ಟಾಂಪ್​ಸ್ಕಾಟ್ ಎಂಬಲ್ಲಿಗೆ ಹೋದರು. ಇದು ಕಡಲತೀರದ ಒಂದು ವಿಶ್ರಾಂತಿಧಾಮ. ಇಲ್ಲಿಯೂ ಕೆಲದಿನಗಳನ್ನು ಕಳೆದು ಅವರು, ತಮಗೆ ಈಗಾಗಲೇ ಪರಿಚಿತಳಾಗಿದ್ದ ಕುಮಾರಿ ಸಾರಾ ಫಾರ್ಮರ್ ಎಂಬವಳ ಕೋರಿಕೆಯ ಮೇರೆಗೆ, ಎಲಿಯಟ್ ನಗರದ ಬಳಿಯ ‘ಗ್ರೀನ್ ಏಕರ್​’ ಎಂಬ ಸ್ಥಳಕ್ಕೆ ಹೋದರು.

ಪಶ್ಚಿಮದಲ್ಲಿ ಅವರ ಸಂದೇಶಪ್ರಸಾರ ಕಾರ್ಯಕ್ಕೆ ಸಂಬಂಧಿಸಿದಂತೆ ಒಂದು ಪ್ರಮುಖ ಘಟನೆ ಇಲ್ಲಿ ನಡೆಯಿತು. ವಿಶಾಲವಾದ ಪಿಸ್ಕಾಟಾಕ್ವಾ ನದಿಯ ದಡದ ಮೇಲಿರುವ, ದಟ್ಟವಾದ ಗಿಡಮರಗಳಿಂದ ಕೂಡಿದ ಈ ಗ್ರೀನ್ ಏಕರ್ ಎಂಬಲ್ಲಿ ಕುಮಾರಿ ಸಾರಾ ಫಾರ್ಮರಳ ನೇತೃತ್ವ ದಲ್ಲಿ, ‘ಗ್ರೀನೇಕರ್ ರಿಲಿಜಸ್ ಕಾನ್​ಫರೆನ್ಸಸ್​’ ಎಂಬ ಅಧ್ಯಯನ ಕೂಟವೊಂದು ಆಗ ಕೆಲ ಕಾಲದ ಹಿಂದೆಯಷ್ಟೇ ಪ್ರಾರಂಭವಾಗಿತ್ತು. ಸ್ವಲ್ಪಮಟ್ಟಿಗೆ ಸರ್ವಧರ್ಮ ಸಮ್ಮೇಳನದಿಂದ ಸ್ಫೂರ್ತಿಪಡೆದಿದ್ದ ಈ ಒಕ್ಕೂಟವು, ಸಂಪ್ರದಾಯಬದ್ಧರಲ್ಲದ ಹಾಗೂ ವಿಶಾಲ ಮನೋಭಾವದ ವ್ಯಕ್ತಿಗಳಿಂದ ಕೂಡಿತ್ತು. ಇಲ್ಲಿ ಎಲ್ಲ ಬಗೆಯ ಧಾರ್ಮಿಕ ಉಪನ್ಯಾಸಕರಿಗೂ ಮುಕ್ತ ಪ್ರವೇಶ ವಿತ್ತು. ಹಾಗೂ ಕಲಿಯಬೇಕೆಂಬ ಇಚ್ಛೆಯುಳ್ಳವರೆಲ್ಲರೂ ಈ ಉಪನ್ಯಾಸಗಳಲ್ಲಿ ಭಾಗವಹಿಸಬಹು ದಾಗಿತ್ತು. ಈ ಅಧ್ಯಯನ ಕೂಟಕ್ಕೆ ಶ್ರದ್ಧಾವಂತ ಉಪನ್ಯಾಸಕರೂ ನಿಷ್ಠಾವಂತ ವಿದ್ಯಾರ್ಥಿಗಳೂ ಬರುತ್ತಿದ್ದರು. ಹೀಗೆ ಬರುತ್ತಿದ್ದವರಲ್ಲಿ ಮುಂದೆ ಸ್ವಾಮೀಜಿಯ ಅತ್ಯಂತ ಆಪ್ತ ಸ್ನೇಹಿತರಾದ ಶ್ರೀಮತಿ ಸಾರಾ ಓಲೇ ಬುಲ್ ಮತ್ತು ಡಾ ॥ ಲೂಯಿಸ್ ಜಿ. ಜೇನ್ಸ್ ಕೂಡ ಇದ್ದರು.

ಗ್ರೀನೇಕರ್​ನಲ್ಲಿ ಸ್ವಾಮೀಜಿ ಜುಲೈ ೨೬ರಿಂದ ಆಗಸ್ಟ್ ೧೩ ರವರೆಗೆ ಉಳಿದುಕೊಂಡರು. ಇಲ್ಲಿ ಅವರು ತರಗತಿಗಳ ಸರಣಿಯೊಂದನ್ನು ಪ್ರಾರಂಭಿಸಿದರು. ಪೌರ್ವಾತ್ಯ ಸಂಪ್ರದಾಯದಂತೆ ಮರದ ಬುಡದಲ್ಲಿ ತಮ್ಮ ಸುತ್ತಲೂ ಕುಳಿತು ಅತ್ಯಂತ ಆಸ್ಥೆಯಿಂದ ಆಲಿಸುತ್ತಿದ್ದ ಉತ್ಸಾಹೀ ವಿದ್ಯಾರ್ಥಿಗಳಿಗೆ ಸ್ವಾಮೀಜಿ ವೇದಾಂತ ತತ್ತ್ವದ ಕುರಿತಾಗಿ ಹೇಳುತ್ತಿದ್ದರು. ತರಗತಿಗಳು ನಡೆಯುತ್ತಿದ್ದುದು ಪೈನ್ (ಪೀತದಾರು) ವೃಕ್ಷವೊಂದರ ನೆರಳಲ್ಲಿ. ಅದಕ್ಕೆ \eng{Swami’s Pine (}ಸ್ವಾಮೀಜಿಯ ಪೈನ್​) ಎಂದೇ ಹೆಸರಾಯಿತು. ಪಾಶ್ಚಾತ್ಯ ರಾಷ್ಟ್ರಗಳಲ್ಲಿ ಸ್ವಾಮೀಜಿ ತೆಗೆದು ಕೊಂಡ ತರಗತಿಗಳ ಶ್ರೇಣಿಯಲ್ಲಿ ಇದೇ ಮೊದಲನೆಯದು. ಇದು ಅವರ ಕಾರ್ಯವು ಹೊಸ ತಿರುವನ್ನು ತೆಗೆದುಕೊಳ್ಳಲಿದ್ದುದರ ಮುನ್ಸೂಚನೆಯೂ ಆಗಿತ್ತು. ಈ ತರಗತಿಗಳ ಕುರಿತಾಗಿ ಸ್ವಾಮೀಜಿ ಹೇಲ್ ಸೋದರಿಯರಿಗೆ ಬರೆದರು, “ನಾನು ಅವರಿಗೆಲ್ಲ ‘ಶಿವೋ\eng{s}ಹಂ ಶಿವೋ\eng{s}ಹಂ’ ಹೇಳಿ ಕೊಡುತ್ತೇನೆ. ಅವರದನ್ನು ಅನುಸರಿಸುತ್ತಾರೆ. ಅವರು ಮುಗ್ಧರೂ ಪರಿಶುದ್ಧರೂ ಆದ್ದ ರಿಂದ, ಅವರ ಧೈರ್ಯಕ್ಕೆ ಎಲ್ಲೆಯೇ ಇಲ್ಲ.” ಈ ತರಗತಿಗಳಲ್ಲಿ ಸ್ವಾಮೀಜಿ ಸಾಮಾನ್ಯವಾಗಿ ಅವಧೂತ ಗೀತೆಯನ್ನು ವ್ಯಾಖ್ಯಾನಿಸಿ ಹೇಳುತ್ತಿದ್ದರು. ಇನ್ನು ಕೆಲವೊಮ್ಮೆ ರಾಜಯೋಗವನ್ನು ಬೋಧಿಸುತ್ತಿದ್ದರು.

ಗ್ರೀನೇಕರ್​ನಲ್ಲಿ ಉಪನ್ಯಾಸ-ಬೋಧನೆ-ವಿಹಾರಗಳಲ್ಲಿ ಸಮಯ ವೇಗವಾಗಿ ಸರಿಯಿತು. ಆದರೆ ವಿಹಾರ ಎನ್ನುವುದು ಅಪರೂಪವೇ ಆಗಿತ್ತು. ವಿದ್ಯಾರ್ಥಿಗಳ ಪಾಲಿಗಂತೂ ಸ್ವಾಮೀಜಿಯ ಸಾನ್ನಿಧ್ಯವೇ ಒಂದು ಅಪೂರ್ವ ಅನುಭವ. ಇನ್ನು ಅವರ ಮಾತುಗಳೋ ಅಮೃತಸಮಾನ. ಇಂತಹ ತರಗತಿಯು ಸ್ವಾಮೀಜಿಯ ಹೃದಯಕ್ಕೂ ಪ್ರಿಯವಾದದ್ದೇ. ಅಲ್ಲದೆ ಸಾರ್ವಜನಿಕ ಉಪನ್ಯಾಸಗಳಿ ಗಿಂತ ಇವು ಕಡಿಮೆ ಶ್ರಮದಾಯಕ ಹಾಗೂ ಹೆಚ್ಚು ತೃಪ್ತಿಕರ. ಆದರೆ ಈ ತರಗತಿಗಳೂ ದೈಹಿಕ ವಾಗಿ ಅವರಿಗೆ ಸಾಕಷ್ಟು ಆಯಾಸವುಂಟುಮಾಡುತ್ತಿದ್ದುವು. ಪತ್ರವೊಂದರಲ್ಲಿ ಅವರು ಶ್ರೀಮತಿ ಹೇಲ್​ಗೆ ಬರೆಯುತ್ತಾರೆ: “ನನಗೇನೋ ಸ್ವಲ್ಪ ವಿಶ್ರಾಂತಿ ಬೇಕಾಗಿದೆ. ಆದರೆ ಅದು ಭಗವಂತನ ಇಚ್ಛೆಗೆ ಬಂದಿಲ್ಲವೆಂದು ಕಾಣುತ್ತದೆ. ಗ್ರೀನೇಕರ್​ನಲ್ಲಿ ನಾನು ಸರಾಸರಿ ಏಳರಿಂದ ಎಂಟು ಗಂಟೆ ಕಾಲ ಮಾತನಾಡಬೇಕಾಗಿತ್ತು; ಬಹುಶಃ ವಿಶ್ರಾಂತಿ ಎಂಬುದೇನಾದರೂ ಇದ್ದರೆ ಅದೇ ಇರ ಬೇಕು! ಆದರೆ ನಾನು ಮಾತನಾಡುತ್ತಿದ್ದುದೆಲ್ಲ ಭಗವಂತನ ಬಗ್ಗೆ; ಆದ್ದರಿಂದ ಅದೇ ನವ ಚೈತನ್ಯವನ್ನು ತಂದುಕೊಡುತ್ತದೆ.”

ಹೀಗೆ ಸುಮಾರು ಹದಿನೈದು ದಿನಗಳನ್ನು ಕಳೆದು ಸ್ವಾಮೀಜಿ, ಅಲ್ಲಿಂದ ಮಸ್ಸಾಚುಸೆಟ್ಸ್ ರಾಜ್ಯದ ಪ್ಲೈಮತ್ ಎಂಬಲ್ಲಿಗೆ ಹೋದರು. ಇಲ್ಲಿನ ‘ಫ್ರೀ ರಿಲಿಜಸ್ ಅಸೋಸಿಯೇಶನ್​’ ಎಂಬ ಸಂಸ್ಥೆಯಲ್ಲಿ ಮಾತನಾಡುವಂತೆ ಅವರನ್ನು ಆಹ್ವಾನಿಸಲಾಗಿತ್ತು. ಪ್ರಖ್ಯಾತ ತತ್ತ್ವಶಾಸ್ತ್ರಜ್ಞನಾದ ರಾಲ್ಫ್ ವಾಲ್ಡೊ ಎಮರ್​ಸನ್ನನಿಂದ ಸ್ಥಾಪಿಸಲ್ಪಟ್ಟ ಈ ಸಂಸ್ಥೆಯು ಧಾರ್ಮಿಕ ಭಾವನೆಗಳನ್ನೂ ಆಚರಣೆಗಳನ್ನೂ ಸಾಂಪ್ರದಾಯಿಕ ಚಿಂತನೆಯ ಹಿಡಿತದಿಂದ ಮುಕ್ತವಾಗಿಸುವ ಉದ್ದೇಶವನ್ನು ಹೊಂದಿತ್ತು. ಈ ಸಂಸ್ಥೆಯ ಹಲವಾರು ಸದಸ್ಯರು ಇಡೀ ದೇಶದಲ್ಲೇ ಪ್ರಸಿದ್ಧ ಹಾಗೂ ಗಣ್ಯ ಸ್ತ್ರೀಪುರುಷರು. ಕೆಲವರು ಸ್ವಾಮೀಜಿಯ ಮಿತ್ರರು, ವಿಶ್ವಾಸಿಗಳು. ಇವರಲ್ಲಿ ಮುಖ್ಯರಾದವ ರೆಂದರೆ ಸಂಸ್ಥೆಯ ಅಧ್ಯಕ್ಷರಾದ ಥಾಮಸ್ ಹಿಗ್ಗಿನ್​ಸನ್, ಜೂಲಿಯಾ ವಾರ್ಡ್ ಹೋವ್ ಹಾಗೂ ಡಾ ॥ ಲೂಯಿಸ್ ಜೇನ್ಸ್. ಮುಂದೆ ಸ್ವಾಮೀಜಿಯ ಅತ್ಯಂತ ನಿಷ್ಠಾವಂತ ಬೆಂಬಲಿಗರಾದ ಡಾ॥ ಜೇನ್ಸ್​ರವರು ಪ್ರಖ್ಯಾತ ‘ಬ್ರೂಕ್ಲಿನ್ ಎಥಿಕಲ್ ಅಸೋಸಿಯೇಶನ್​’ನ ಅಧ್ಯಕ್ಷರು.

ಸ್ವಾಮೀಜಿ ಆಗಸ್ಟ್ ೧೯ರ ಸುಮಾರಿಗೆ ಫ್ಲೈಮತ್​ನಿಂದ ಪ್ರಯಾಣ ಮಾಡಿ ಕಡಲ ತೀರದ ಹಳ್ಳಿಯಾದ ಆ್ಯನಿಸ್ಕ್ವಾಮ್​ಗೆ ಬಂದರು. ಇದೇ ಆ್ಯನಿಸ್ಕ್ವಾಮ್​ನಲ್ಲೇ ಅವರು ಸುಮಾರು ಒಂದು ವರ್ಷದ ಹಿಂದೆ ಸರ್ವಧರ್ಮ ಸಮ್ಮೇಳನದ ಮುಂಚಿನ ದಿನಗಳಲ್ಲಿ ಜೆ. ಹೆಚ್. ರೈಟ್​ರವರ ಅತಿಥಿಯಾಗಿದ್ದುದು. ಈ ಸಲ ಅವರು, ತಮ್ಮ ನಿಷ್ಠಾವಂತ ಬೆಂಬಲಿಗರಾದ ಶ್ರೀಮತಿ ಜಾನ್ ಬ್ಯಾಗ್​ಲೀಯ ಬೇಸಿಗೆಮನೆಗೆ ಭೇಟಿ ನೀಡಿದರು. ಇಲ್ಲಿ ಸುಮಾರು ಎರಡು ವಾರಗಳಿಗೂ ಹೆಚ್ಚುಕಾಲ ಇದ್ದರು. ಈ ದಿನಗಳಲ್ಲಿ ಅವರು ಶ್ರೀಮತಿ ಪರ್ಸಿ ಸ್ಮಿತ್ ಎಂಬವಳ ಆಹ್ವಾನದ ಮೇರೆಗೆ ಕಡಲ ತೀರದ ಇನ್ನೊಂದು ವಿಶ್ರಾಂತಿಧಾಮವಾದ ಮ್ಯಾಗ್ನೋಲಿಯಾ ಎಂಬಲ್ಲಿಗೆ ಮೂರು ದಿನಗಳ ಮಟ್ಟಿಗೆ ಹೋಗಿಬಂದರು. ಈ ಭೇಟಿಯ ಸಂದರ್ಭದಲ್ಲಿ ಅವರು ಮ್ಯಾಗ್ನೋ ಲಿಯಾದಲ್ಲಿ ಒಂದು ಉಪನ್ಯಾಸ ನೀಡಿದರು. ಹಾಗೆಯೇ ಆ್ಯನಿಸ್ಕ್ವಾಮ್​ನಲ್ಲೂ ‘ಭಾರತದ ಧರ್ಮ’ ಎನ್ನುವ ವಿಷಯವಾಗಿ ಉಪನ್ಯಾಸ ಮಾಡಿದರು. ಈ ಉಪನ್ಯಾಸದಲ್ಲಿ ಹಾಜರಿದ್ದ ಪ್ರೊ॥ ಜೆ. ಹೆಚ್. ರೈಟರು ಅಂದಿನ ಸಭೆಗೆ ಸ್ವಾಮೀಜಿಯನ್ನು ಪರಿಚಯಿಸಿಕೊಟ್ಟರು. ಈ ದಿನಗಳಲ್ಲಿ ಸ್ವಾಮೀಜಿ ಚೆನ್ನಾಗಿ ವಿಶ್ರಾಂತಿ ಪಡೆಯುತ್ತ, ಮುಂದೆ ಮತ್ತೊಮ್ಮೆ ನಗರಗಳಲ್ಲಿ ಉಪನ್ಯಾಸ ಗಳನ್ನು ನೀಡಲು ಮಳೆಗಾಲವನ್ನು ನಿರೀಕ್ಷಿಸಿಕೊಂಡು ಕುಳಿತರು. ಇದೇ ಸಂದರ್ಭದಲ್ಲೇ ಸ್ವಾಮೀಜಿಯ ಕಲಾವಿದ ಗೆಳೆಯರೊಬ್ಬರು ಅವರ ಸುಂದರ ಶರೀರದ ವರ್ಣಚಿತ್ರವನ್ನು ಬರೆಯ ಬೇಕೆಂದು ಇಚ್ಛಿಸಿದ್ದರಿಂದ ಅವರು ಅದಕ್ಕಾಗಿ ಅವನ ಮುಂದೆ ಗಂಟೆಗಟ್ಟಲೆ ಕುಳಿತುಕೊಳ್ಳ ಬೇಕಾಯಿತು. ಆದರೆ ಆ ಚಿತ್ರ ಮಾತ್ರ ನಮಗೆ ಸಿಕ್ಕಿಲ್ಲ.

೧೮೯೪ರ ಸೆಪ್ಟೆಂಬರ್ ೬ರ ಸುಮಾರಿಗೆ ಸ್ವಾಮೀಜಿ ಆ್ಯನಿಸ್ಕ್ವಾಮ್​ನಿಂದ ಹೊರಟರು. ಬೇಸಿಗೆ ಕಳೆದಿತ್ತು. ಈಗ ಹೊಸದೊಂದು ಉಪನ್ಯಾಸಮಾಲೆ ಪ್ರಾರಂಭವಾಗಬೇಕಾಗಿತ್ತು. ಸ್ವಾಮೀಜಿ ತಡಮಾಡದೆ ಕಾರ್ಯರಂಗಕ್ಕೆ ಧುಮುಕಿದರು. ಅವರು ಈಗ ಬಂದದ್ದು ಬಾಸ್ಟನ್​ಗೆ. ಅವರಿಗೆ ಬಾಸ್ಟನ್ ನಗರದಲ್ಲಿ ಹಾಗೂ ಅದರ ಅಕ್ಕಪಕ್ಕದ ಗ್ರಾಮಗಳಲ್ಲಿ ಮೂರು ವಾರಗಳ ಕಾಲ ಬಿಡುವಿಲ್ಲದ ಉಪನ್ಯಾಸಗಳ ಕಾರ್ಯಕ್ರಮ ಇದ್ದಿತು. ಈ ದಿನಗಳಲ್ಲಿ ಅವರಿಗೆ ಅನ್ನಿಸಿತು– ಒಂದೆಡೆ ಕುಳಿತು ತಮ್ಮ ಆಲೋಚನೆಗಳನ್ನೆಲ್ಲ ಬರೆದಿಡಬೇಕು ಎಂದು. ಆದ್ದರಿಂದ ಅವರು, ಕೆಲ ದಿನಗಳನ್ನು ತನ್ನ ಮನೆಯಲ್ಲಿ ಕಳೆಯುವಂತೆ ಶ್ರೀಮತಿ ಸಾರಾ ಬುಲ್​ಳಿಂದ ಬಂದ ಆಹ್ವಾನವನ್ನು ಮನ್ನಿಸಿ ಅವಳ “ಕೇಂಬ್ರಿಡ್ಜ್ ಹೌಸ್​”ಗೆ ಹೋದರು. ಅಲ್ಲಿ ಶ್ರೀಮತಿ ಬುಲ್ ತನ್ನ ಮಗಳು, ಒಬ್ಬಳು ಜೊತೆಗಾತಿ ಹಾಗೂ ಯಾರಾದರೂ ಕೆಲವು ಅತಿಥಿಗಳೊಂದಿಗೆ ವಾಸವಾಗಿರುತ್ತಿದ್ದಳು. ಇವಳು ನಾರ್ವೇ ದೇಶದ ಪ್ರಸಿದ್ಧ ಪಿಟೀಲು ವಾದಕನಾಗಿದ್ದ ಮಿ ॥ ಬುಲ್ ಎಂಬವನ ವಿಧವೆಪತ್ನಿ. ಈಕೆ ಕಲೆ ಸಾಹಿತ್ಯಗಳ ಪೋಷಕಳೆಂದು ಪ್ರಖ್ಯಾತಳು. ಮುಂದೆ ಈಕೆ ಅಮೆರಿಕೆಯಲ್ಲಿ ಸ್ವಾಮೀಜಿಯ ನಿಷ್ಠಾವಂತ ಬೆಂಬಲಿಗಳೂ ಸದಾ ಸಹಾಯಕ್ಕೊದಗುವ ಸ್ನೇಹಿತೆಯೂ ಆದಳು. ಅಮೆರಿಕದ ಪಶ್ಚಿಮದಲ್ಲಿ ಹೇಲ್ ಕುಟುಂಬದವರ ಮನೆ ಸ್ವಾಮೀಜಿಯ ಕೇಂದ್ರಸ್ಥಾನವಾಗಿದ್ದಂತೆ ಪೂರ್ವ ತೀರದಲ್ಲಿ ಶ್ರೀಮತಿ ಬುಲ್​ಳ ಮನೆ ಅವರ ಕೇಂದ್ರಸ್ಥಾನವಾಯಿತೆನ್ನಬಹುದು.

ಕೇಂಬ್ರಿಡ್ಜ್​ನಲ್ಲಿ ಅಕ್ಟೋಬರ್ ೨ರಿಂದ ೧೨ರವರೆಗೆ ಇದ್ದು ಅಲ್ಲಿಂದ ಮೇರಿ ಲ್ಯಾಂಡ್​ನಲ್ಲಿ ರುವ ಬಾಲ್ಟಿಮೋರ್​ಗೆ ಹೋದರು. ಇಲ್ಲಿ ಸ್ವಾಮೀಜಿ ರೂಮನ್ ಸೋದರರೆಂಬ ಮೂವರು ಉತ್ಸಾಹಶಾಲೀ ಯುವಧರ್ಮಾಧಿಕಾರಿಗಳ ಅತಿಥಿಯಾಗಿದ್ದರು. ಈ ಧರ್ಮಾಧಿಕಾರಿಗಳು ತಮ್ಮ ಎರಡು ಸಭೆಗಳಲ್ಲಿ “ಕ್ರಿಯಾಶೀಲ ಧರ್ಮ” ಎಂಬ ವಿಷಯದ ಮೇಲೆ ಮಾತನಾಡುವಂತೆ ಸ್ವಾಮೀಜಿಯವರನ್ನು ಆಹ್ವಾನಿಸಿದರು. ಇಲ್ಲಿ ಅವರ ಉಪನ್ಯಾಸಗಳು ಬಹುದೊಡ್ಡ ಸಂಖ್ಯೆಯಲ್ಲಿ ಸಭಿಕರನ್ನು ಆಕರ್ಷಿಸಿದುವು. ಎರಡನೆಯ ಉಪನ್ಯಾಸಕ್ಕಂತೂ ಮೂರು ಸಾವಿರಕ್ಕಿಂತಲೂ ಹೆಚ್ಚು ಜನ ಕಿಕ್ಕಿರಿದಿದ್ದರು.

ಬಾಲ್ಟಿಮೋರ್​ನಿಂದ ಸ್ವಾಮೀಜಿ ಅಲ್ಲೇ ಹತ್ತಿರದ ವಾಷಿಂಗ್​ಟನ್​ಗೆ ಬಂದರು. ಇಲ್ಲಿ ಅವರು ಕರ್ನಲ್ ಇನೋಕ್ ಟೋಟನ್ ದಂಪತಿಗಳ ಅತಿಥಿಗಳಾಗಿದ್ದರು. ಶ್ರೀಮತಿ ಟೋಟನ್ ಒಬ್ಬ ಪ್ರಭಾವಶಾಲೀ ಮಹಿಳೆ. ಇವಳೊಬ್ಬ ತತ್ತ್ವಶಾಸ್ತ್ರಜ್ಞೆ ಕೂಡ. ಸ್ವಾಮೀಜಿ ವಾಷಿಂಗ್​ಟನ್​ನಲ್ಲಿ ಮಾಡಿದ ಭಾಷಣಗಳು ಮೂರು–‘ಸರ್ವಧರ್ಮಗಳ ಸಮಾನ ಆಧ್ಯಾತ್ಮಿಕ ಮೂಲಗಳು,’ ‘ಆರ್ಯ ಜನಾಂಗ’ ಮತ್ತು ‘ಪುನರ್ಜನ್ಮ’. ಮರುದಿನ ಸ್ವಾಮೀಜಿ ಬಾಲ್ಟಿಮೋರ್​ಗೆ ಹಿಂದಿರುಗಿ ಅಲ್ಲಿನ ನಾಗರಿಕರ ಮುಂದೆ ‘ಭಾರತ ಮತ್ತು ಅದರ ಧರ್ಮ’ ಎಂಬ ವಿಷಯದ ಮೇಲೆ ಉಪನ್ಯಾಸ ಮಾಡಿ, ಈ ಉಪನ್ಯಾಸಗಳಿಂದ ಬಂದ ಹಣವೆಲ್ಲವನ್ನೂ ರೂಮನ್ ಸೋದರರು ಹಾಕಿಕೊಂಡಿದ್ದ ‘ಅಂತರರಾಷ್ಟ್ರೀಯ ವಿದ್ಯಾಲಯ’ದ ಯೋಜನೆಗೆ ದಾನ ಮಾಡಿದರು.

ಈಗ ಸ್ವಾಮೀಜಿ, ತಮ್ಮ ಸಾರ್ವಜನಿಕ ಉಪನ್ಯಾಸಗಳನ್ನು ಕೆಲಕಾಲ ನಿಲ್ಲಿಸಿ, ತಾವಾಗಿಯೇ ಆಸಕ್ತರಾಗಿ ಬಂದವರಿಗೆ ವೇದಾಂತದ ಸಂದೇಶವನ್ನು ಹರಡುತ್ತ ತಮ್ಮ ಕಾರ್ಯವನ್ನು ವೇದಾಂತದ ಬಲವಾದ ಅಡಿಪಾಯದ ಮೇಲೆ ನೆಲೆನಿಲ್ಲಿಸಲು ನಿರ್ಧರಿಸಿದರು. ಶ್ರದ್ಧಾವಂತರಾದ ಮತ್ತು ಪ್ರಾಮಾಣಿಕರಾದ ಕೆಲವು ಶಿಷ್ಯರನ್ನು ನಿರ್ಮಾಣಮಾಡಿಕೊಂಡು, ಅವರಿಗೆ ಸಮರ್ಥಶಿಕ್ಷಣ ನೀಡಿ, ಅವರ ಮೂಲಕ ತಮ್ಮ ಸಂದೇಶ ಪ್ರಸಾರವಾಗುವಂತೆ ನೋಡಿಕೊಳ್ಳಬೇಕೆಂಬ ತೀವ್ರ ಅಭಿಲಾಷೆ ಅವರಲ್ಲಿತ್ತು. ಆದರೆ ಅದಕ್ಕಿನ್ನೂ ಕಾಲ ಒದಗಿಬಂದಿರಲಿಲ್ಲವೆಂದು ಕಾಣುತ್ತದೆ.

ಆಗಸ್ಟ್ ತಿಂಗಳ ಕೊನೆಯಲ್ಲಿ ಸ್ವಾಮೀಜಿ ಅಳಸಿಂಗ ಪೆರುಮಾಳರಿಗೆ ಬರೆದರು–“ಇಡೀ ಪ್ರಪಂಚಕ್ಕೆ ಬೆಳಕು ಬೇಕಾಗಿದೆ. ಬೆಳಕನ್ನು ಅದು ಇದಿರುನೋಡುತ್ತಿದೆ. ಆ ಬೆಳಕು ಇರುವುದು ಭಾರತದ ಬಳಿ ಮಾತ್ರವೇ. ಅದು ಮಂತ್ರಮಾಟಗಳಲ್ಲಿಲ್ಲ. ಅದು ಬೂಟಾಟಿಕೆಯಲ್ಲಿಲ್ಲ. ಅದಿರುವುದು ನಿಜವಾದ ಸಾರರೂಪದ ಬೋಧನೆಗಳಲ್ಲಿ, ಅತ್ಯುನ್ನತ ಆಧ್ಯಾತ್ಮಿಕ ಸತ್ಯದಲ್ಲಿ. ಆದ್ದರಿಂದಲೇ ಭಗವಂತ ನಮ್ಮ ಜನಾಂಗವನ್ನು ಇಷ್ಟೆಲ್ಲ ಏರಿಳಿತಗಳಿಂದ ಈ ಆಧುನಿಕ ಕಾಲದ ವರೆಗೂ ಸಂರಕ್ಷಿಸಿರುವುದು. ಇದೀಗ ಸಮಯ ಸನ್ನಿಹಿತವಾಗಿದೆ.” ಆಧ್ಯಾತ್ಮಿಕತೆ ಸ್ವಾಮೀಜಿಯ ಸಂದೇಶದ ಪ್ರಮುಖ ಅಂಶವಾಗಿತ್ತು. ಭಾರತವು ಅತ್ಯಂತ ತೀವ್ರವಾಗಿ ಆರ್ಥಿಕ ಸಹಾಯದ ಆವಶ್ಯಕತೆಯಲ್ಲಿದ್ದರೆ ಪಾಶ್ಚಾತ್ಯರಾಷ್ಟ್ರಗಳು ಅದಕ್ಕಿಂತ ಮಿಗಿಲಾಗಿ ಆಧ್ಯಾತ್ಮಿಕ ಸಹಾಯದ ಆವಶ್ಯಕತೆಯಲ್ಲಿವೆ ಎಂಬುದನ್ನು ಸ್ವಾಮೀಜಿ ಕಂಡುಕೊಂಡರು. ಭಾರತವು ಹೊಟ್ಟೆಯ ಹಸಿವಿ ನಿಂದ ಕಂಗಾಲಾಗಿದ್ದರೆ ಅಮೆರಿಕಾದಿ ರಾಷ್ಟ್ರಗಳು ಆತ್ಮದ ಹಸಿವಿನಿಂದ ತೊಳಲುತ್ತಿವೆ ಎಂಬು ದನ್ನು ಕಂಡರು. ಆದ್ದರಿಂದ ತಮ್ಮನ್ನು ತಾವು ಭಾರತಕ್ಕೆ ಹೇಗೋ ಹಾಗೆಯೇ ಪಾಶ್ಚಾತ್ಯ ರಾಷ್ಟ್ರಗಳಿಗೆ, ಅಷ್ಟೇಕೆ, ಸಮಸ್ತ ಜಗತ್ತಿಗೆ ಸಮರ್ಪಿಸಿಕೊಳ್ಳಲು ನಿರ್ಧರಿಸಿದರು.

ನವೆಂಬರ್ ನಾಲ್ಕರಂದು ಸ್ವಾಮೀಜಿ ಬಾಲ್ಟಿಮೋರ್​ನಿಂದ ನ್ಯೂಯಾರ್ಕಿಗೆ ಹಿಂದಿರುಗಿದರು. ಇಲ್ಲಿ ಅವರು ವೇದಾಂತದ ಒಂದು ಸಣ್ಣ ಕೇಂದ್ರವನ್ನು ಸಂಘಟಿಸಿದರು. ತಮ್ಮ ಕಾರ್ಯದ ಆರ್ಥಿಕ ಹಾಗೂ ಇತರ ವಿಚಾರಗಳ ಮೇಲ್ವಿಚಾರಣೆಯೊಂದಿಗೆ ಅವರ ಇನ್ನೊಂದು ಉದ್ದೇಶ ವೇನಾಗಿತ್ತೆಂದರೆ ವೇದಾಂತದ ಜ್ಞಾನವನ್ನು ಹೊರಜಗತ್ತಿಗೆ ಹರಿಯಿಸಿ ತನ್ಮೂಲಕ ತಾವು ಭಾರತದಲ್ಲಿ ಸ್ಥಾಪಿಸಲು ಯೋಚಿಸಿದ್ದ ವಿದ್ಯಾಸಂಸ್ಥೆಗಳಿಗೆ ನಿಧಿಯನ್ನು ಸಂಗ್ರಹಿಸುವುದು.

ಈ ಹಿಂದೆ ಸ್ವಾಮೀಜಿ ತಮ್ಮ ಸಂದೇಶಪ್ರಸಾರದ ಬೀಜವನ್ನು ಬಿತ್ತಿ ಅದಕ್ಕಾಗಿ ತಮ್ಮ ಅಗಾಧ ಶಕ್ತಿ-ಜ್ಞಾನಗಳನ್ನು ಧಾರೆಯೆರೆದಿದ್ದರು. ಅದರ ಉದ್ದೇಶ ಮಾತ್ರ ಒಂದೇ ಆಗಿತ್ತು–ಮಾನವನ ಆತ್ಮೋದ್ಧಾರ. ಈಗ ಅವರು ಅದೇ ಉದ್ದೇಶದಿಂದ ಒಂದು ಜಾಗದಲ್ಲಿ ನೆಲೆ ನಿಂತು ಶಾಶ್ವತ ಕೇಂದ್ರವೊಂದನ್ನು ಸ್ಥಾಪಿಸಿ, ಕೆಲವು ಶ್ರದ್ಧಾವಂತ ವ್ಯಕ್ತಿಗಳಿಗೆ ತೀವ್ರತರ ಆಧ್ಯಾತ್ಮಿಕ ಶಿಕ್ಷಣ ನೀಡಿ ತಾವು ಅಮೆರಿಕದಿಂದ ನಿರ್ಗಮಿಸಿದ ಮೇಲೂ ಅಲ್ಲಿ ತಮ್ಮ ಕಾರ್ಯ ಅಭಿವೃದ್ಧಿ ಹೊಂದುವಂತಹ ವ್ಯವಸ್ಥೆ ಮಾಡುವ ಯೋಜನೆಯನ್ನು ಹಾಕಿಕೊಂಡರು.

ನ್ಯೂಯಾರ್ಕಿನಲ್ಲಿ ನೆಲಸುವ ಮೊದಲು ಸ್ವಾಮೀಜಿ ಮತ್ತೊಮ್ಮೆ ಕೇಂಬ್ರಿಡ್ಜ್​ಗೆ ಭೇಟಿ ನೀಡಿ ದರು. ಇಲ್ಲಿ ಶ್ರೀಮತಿ ಬುಲ್ ಡಿಸೆಂಬರ್ ೫ರಿಂದ ೨೭ರವರೆಗೆ ಅವರ ತರಗತಿಗಳನ್ನು ಏರ್ಪಡಿಸಿ ದ್ದಳು. ಈ ತರಗತಿಗಳಲ್ಲಿ ಅವರು ಉಪನಿಷತ್ತುಗಳು, ಗೀತೆ ಸಾಂಖ್ಯವೇ ಮೊದಲಾದವುಗಳನ್ನು ಬೋಧಿಸಿದರು. ಇವುಗಳೊಂದಿಗೆ ಅವರು ಶ್ರೀಮತಿ ಬುಲ್​ಳ ಮನೆಯಲ್ಲಿ ಮೂರು ಉಪನ್ಯಾಸ ಗಳನ್ನೂ ನೀಡಿದರು. ಇವುಗಳಲ್ಲೊಂದರ ವಿಷಯ ‘ಭಾರತೀಯ ಮಹಿಳೆಯರ ಆದರ್ಶಗಳು.’ ಅವರ ಆತಿಥೇಯಳ ವಿಶೇಷ ಕೋರಿಕೆಯ ಮೇರೆಗೆ ನೀಡಲಾದ ಈ ಉಪನ್ಯಾಸವು ಹೃದಯಸ್ಪರ್ಶಿ ಯಾಗಿದ್ದು ಸ್ವಾಮೀಜಿಯ ದೇಶಪ್ರೇಮವನ್ನು ಎತ್ತಿತೋರಿಸುವಂತಿತ್ತು. ಈ ಉಪನ್ಯಾಸದಲ್ಲಿ ಅವರು ಭಾರತೀಯ ನಾರೀತ್ವದ ಸ್ವರೂಪ ಹಾಗೂ ಆದರ್ಶಗಳ ಸೊಬಗನ್ನು–ಅದರಲ್ಲೂ ವಿಶೇಷವಾಗಿ ಭಾರತೀಯ ಮಹಿಳೆಯರ ತಾಯ್ತನದ ಆದರ್ಶವನ್ನು–ಉಜ್ವಲವಾಗಿ ವಿವರಿಸಿ ದರು. ಭಾರತೀಯ ಮಹಿಳೆಯರ ಸ್ಥಿತಿಯ ಕುರಿತಾಗಿ ಕೆಲವು ನೀಚ ವ್ಯಕ್ತಿಗಳು ನಡೆಸುತ್ತಿದ್ದ ಅಪಪ್ರಚಾರಗಳಿಗೆ ಸ್ವಾಮೀಜಿಯ ಉಪನ್ಯಾಸವು ಉತ್ತರವಾಗಿತ್ತು. ಈ ಉಪನ್ಯಾಸಗಳನ್ನು ಕೇಳಿ ಕೇಂಬ್ರಿಡ್ಜ್​ನ ಮಹಿಳೆಯರು ಎಷ್ಟು ಆಳವಾಗಿ ಪ್ರಭಾವಿತರಾದರೆಂದರೆ ಅವರೆಲ್ಲ ಸೇರಿ ಕ್ರಿಸ್ ಮಸ್ ಸಮಯದಲ್ಲಿ ಸ್ವಾಮೀಜಿಗೇ ತಿಳಿಯದಂತೆ, ದೂರದ ಭಾರತದಲ್ಲಿದ್ದ ಅವರ ತಾಯಿ ಭುವನೇಶ್ವರಿದೇವಿಗೆ ಒಂದು ಪತ್ರವನ್ನೂ, ಜೊತೆಗೆ ಮೇರಿಯ ಮಡಿಲಲ್ಲಿ ಬಾಲ ಕ್ರಿಸ್ತ ಮಲಗಿ ರುವ ಒಂದು ಸುಂದರ ಚಿತ್ರವನ್ನೂ ಕಳಸಿಕೊಟ್ಟರು. ಅವರು ಬರೆದ ಪತ್ರ ಹೀಗಿದೆ:

\textbf{ವಿವೇಕಾನಂದರ ತಾಯಿಯವರಿಗೆ}

\noindent

“ಪ್ರಿಯ ತಾಯಿ,

ಭಗವಂತನು ಮೇರಿಯ ಪುತ್ರನನ್ನು ಧರೆಗೆ ಕಾಣಿಕೆಯಾಗಿತ್ತ ದಿನವನ್ನು ಆಚರಿಸಿ ಆನಂದಿಸುವ ಈ ಶುಭ ಕ್ರಿಸ್​ಮಸ್ ಸಂದರ್ಭವು ಮಧುರ ಸ್ಮರಣೆಗಳ ಸಮಯ. ನಿಮ್ಮ ಸುಪುತ್ರನನ್ನು ನಮ್ಮ ನಡುವೆ ಹೊಂದಿರುವ ನಾವು ನಿಮಗೆ ಶುಭಾಶಯಗಳನ್ನು ಕಳಿಸುತ್ತಿದ್ದೇವೆ. ಅವರು ಕೆಲದಿನಗಳ ಹಿಂದೆ ನಮ್ಮನ್ನುದ್ದೇಶಿಸಿ ಮಾತನಾಡುವಾಗ, ನಮ್ಮ ಸ್ತ್ರೀಯರಿಗೆ-ಪುರುಷರಿಗೆ-ಮಕ್ಕಳಿಗೆ ಅವರು ಸಲ್ಲಿಸಿರುವ ಉದಾರ ಸೇವೆಯ ಫಲವನ್ನೆಲ್ಲ ನಿಮ್ಮ ಚರಣಗಳಿಗೆ ಅರ್ಪಿಸಿದರು. ತಮ್ಮ ತಾಯಿ ಯನ್ನು ಅವರು ಆರಾಧಿಸಿದ ಬಗೆ, ಅವರ ಮಾತುಗಳನ್ನು ಕೇಳುತ್ತಿದ್ದವರಿಗೆಲ್ಲ ನಿರಂತರ ಸ್ಫೂರ್ತಿ, ಉತ್ತೇಜನ.

“ಪ್ರಿಯ ತಾಯಿ, ನಿಮ್ಮ ಮಗನ ರೂಪದಲ್ಲಿ ನಿಮ್ಮ ಜೀವನ ಹಾಗೂ ಕೃತಿಗಳನ್ನು ಕೃತಜ್ಞತೆ ಯಿಂದ ಗುರುತಿಸಿದ್ದೇವೆ. ಭಗವಂತನಿಂದಲೇ ಬಂದಂತಹ ಭ್ರಾತೃತ್ವ ಮತ್ತು ಏಕತೆಗಳ ನೆನಪಿ ಗಾಗಿ ಈ ನಮ್ಮ ಕಿರುಗಾಣಿಕೆಯನ್ನು ತಾವು ಸ್ವೀಕರಿಸುವುದಾಗಬೇಕು.”

ಅಂದು ‘ಭಾರತೀಯ ಮಹಿಳೆಯರ ಆದರ್ಶಗಳು’ ಎಂಬ ವಿಷಯವಾಗಿ ಮಾತನಾಡುತ್ತ ಸ್ವಾಮೀಜಿ ವೇದಗಳಿಂದ ಹಾಗೂ ಸಂಸ್ಕೃತ ಸಾಹಿತ್ಯದಿಂದ ಹಲವಾರು ಉದಾಹರಣೆಗಳನ್ನು ಕೊಟ್ಟು, ಭಾರತೀಯ ನಾರಿಯರಿಗೆ ಇಂದಿನ ಕಾಲಕ್ಕೆ ಅನುಸರಣೀಯವಾದ ನೀತಿನಿಯಮಗಳನ್ನು ವಿವರಿಸಿದ್ದರು. ಮತ್ತು ತನ್ನ ಪರಿಶುದ್ಧ ಚಾರಿತ್ರ್ಯ ಹಾಗೂ ನಿಷ್ಕಾಮ ಪ್ರೇಮದ ಮೂಲಕ ತಮ್ಮನ್ನು ಬೆಳೆಸಿ ಅಭಿವೃದ್ಧಿಗೆ ತಂದು ಕೊನೆಗೆ ಅತ್ಯಂತ ಶ್ರೇಷ್ಠವಾದ ಸಂನ್ಯಾಸ ಜೀವನವನ್ನು ಸ್ವೀಕರಿಸಿ ಜಗದ್ಧಿತವನ್ನು ಸಾಧಿಸಲು ಸಹಾಯಕಳಾದ ತಮ್ಮ ಜನನಿಗೆ ಹೃತ್ಪೂರ್ವಕ ಸನ್ಮಾನ ಗಳನ್ನು ಸಲ್ಲಿಸಿದ್ದರು.

ಸಾಧಾರಣವಾಗಿ ಸ್ವಾಮೀಜಿ ಸಂದರ್ಭವೊದಗಿಬಂದಾಗಲೆಲ್ಲ, ತಮ್ಮ ತಾಯಿಯ ಹಲವಾರು ಉದಾತ್ತ-ಶ್ರೇಷ್ಠ ಗುಣಗಳನ್ನು ಬಣ್ಣಿಸಿ ಆಕೆಯನ್ನು ಕೊಂಡಾಡುತ್ತಿದ್ದುದನ್ನು ಗಮನಿಸಬಹು ದಾಗಿತ್ತು. ಒಮ್ಮೆ ಅವರು ತಮ್ಮ ತಾಯಿಯ ಅದ್ಭುತ ಸಂಯಮದ ಬಗ್ಗೆ ಹೇಳುತ್ತ, ಆಕೆ ಒಂದು ಸಲ ಹದಿನಾಲ್ಕು ದಿನಗಳವರೆಗೆ ನಿಂತರ ಉಪವಾಸ ವ್ರತವನ್ನು ಆಚರಿಸಿದ ಬಗ್ಗೆ ಹೇಳಿದರು. “ನನ್ನ ಸಂನ್ಯಾಸ ಜೀವನಕ್ಕೆ ಸ್ಫೂರ್ತಿ ಸಿಕ್ಕಿದ್ದೇ ನನ್ನ ತಾಯಿಯಿಂದ. ನನ್ನ ಈ ಜೀವನಕ್ಕೆ ಹಾಗೂ ನಾನು ನಡೆಸಿಕೊಂಡು ಬರುತ್ತಿರುವ ಈ ಕಾರ್ಯಕ್ಕೆ ನನ್ನ ತಾಯಿಯ ಪರಮಪವಿತ್ರ ಚಾರಿತ್ರ್ಯವು ಒಂದು ನಿರಂತರ ಸ್ಫೂರ್ತಿ” ಎಂಬ ಮಾತು ಅವರ ಬಾಯಿಂದ ಆಗಾಗ ಕೇಳಿಬರುತ್ತಿತ್ತು.

ಕ್ರಿಸ್​ಮಸ್ ಹಬ್ಬವನ್ನು ಸಾರಾ ಬುಲ್ಲಳ ಮನೆಯಲ್ಲಿ ಕಳೆದು ಸ್ವಾಮೀಜಿ ಡಿಸೆಂಬರ್ ೨೮ರಂದು ಬ್ರೂಕ್ಲಿನ್ನಿಗೆ ಹೊರಟರು. ಇಲ್ಲಿನ ‘ಎಥಿಕಲ÷ ಅಸೋಸಿಯೇಶನ್​’ನಲ್ಲಿ ಉಪನ್ಯಾಸ ಮಾಲೆಯೊಂದನ್ನು ನೀಡುವಂತೆ, ಇದರ ಅಧ್ಯಕ್ಷರಾದ ಡಾ ॥ ಲೂಯಿಸ್ ಜೇನ್ಸ್​ರವರು ಆಹ್ವಾನಿಸಿದ್ದರು. ಡಾ ॥ ಜೇನ್ಸ್​ರವರು ಈ ಹಿಂದೆಯೇ ಸ್ವಾಮೀಜಿಯ ಅಸಾಧಾರಣ ಪ್ರತಿಭೆ ಯನ್ನು ಹಾಗೂ ಅವರ ಸಾಧನೆಗಳನ್ನು ಕಂಡು ಪ್ರಭಾವಿತರಾಗಿದ್ದರು. ಡಾ ॥ ಜೇನ್ಸ್​ರ ಪ್ರಾಮಾಣಿಕ ಮನೋಭಾವವನ್ನೂ ಉದಾತ್ತ ಚಾರಿತ್ರ್ಯವನ್ನೂ ಸ್ವಾಮೀಜಿ ಹೃತ್ಪೂರ್ವಕವಾಗಿ ಮೆಚ್ಚಿಕೊಂಡಿದ್ದರು. ಶೀಘ್ರದಲ್ಲೇ ಇಬ್ಬರೂ ಆತ್ಮೀಯ ಸ್ನೇಹಿತರಾದರು.

ಸ್ವಾಮೀಜಿಯ ವ್ಯಕ್ತಿತ್ವದಿಂದ ಆಕರ್ಷಿತರಾಗಿ ಬಂದ ಮತ್ತೊಬ್ಬರೆಂದರೆ ಅದೇ ಸಂಸ್ಥೆಯ ಒಬ್ಬ ಅಧಿಕಾರಿಯಾದ ಚಾರ್ಲ್ಸ್ ಹಿಗ್ಗಿನ್ಸ್. ಇವರು, ಸ್ವಾಮೀಜಿ ಬ್ರೂಕ್ಲಿನ್​ಗೆ ಬರುವ ವೇಳೆಗೆ, ಅಮೆರಿಕದ ಹಾಗೂ ಭಾರತದ ವೃತ್ತಪತ್ರಿಕೆಗಳು ಅವರ ಕುರಿತಾಗಿ ಪ್ರಕಟಿಸಿದ್ದ ಲೇಖನಗಳನ್ನೆಲ್ಲ ಸಂಗ್ರಹಿಸಿ ಒಂದು ಪುಟ್ಟ ಪುಸ್ತಕದ ರೂಪದಲ್ಲಿ ಅಚ್ಚು ಹಾಕಿಸಿ, ಸಭಿಕರಿಗೆ ಮುಂದಾಗಿಯೇ ಹಂಚಿದ್ದರು. ಬ್ರೂಕ್ಲಿನ್ ಎಥಿಕಲ್ ಅಸೋಸಿಯೇಷನ್ನಿನಲ್ಲಿ ಸ್ವಾಮೀಜಿ ನೀಡಿದ ‘ಭಾರತದ ಧರ್ಮಗಳು’ ಎಂಬ ಉಪನ್ಯಾಸವು ಅತ್ಯಂತ ಯಶಸ್ವಿಯಾಯಿತು. ಸನಾತನ ಧರ್ಮದ ಬಗ್ಗೆ ಸ್ವಾಮೀಜಿ ನೀಡಿದ ವರ್ಣನೆಯಿಂದ ಆ ವಿಷಯದ ಬಗ್ಗೆ ಸಭಿಕರಲ್ಲಿ ಎಂತಹ ತೀವ್ರ ಆಸಕ್ತಿ ಉಂಟಾಯಿತೆಂದರೆ, ಅವರು ಬ್ರೂಕ್ಲಿನ್ನಿನಲ್ಲಿ ಪ್ರತಿದಿನವೂ ತರಗತಿಗಳನ್ನು ನಡೆಸಬೇಕೆಂದು ಒತ್ತಾಯಿಸಿದರು.

ಈ ಊರಿನಲ್ಲಿ ಸ್ವಾಮೀಜಿಯವರು ಮಾಡಿದ ಭಾಷಣಗಳ ಬಗ್ಗೆ ‘ಬ್ರೂಕ್ಲಿನ್ ಸ್ಟ್ಯಾಂಡರ್ಡ್ ಯೂನಿಯನ್​’ ಎಂಬ ಪತ್ರಿಕೆ ಸ್ವಾರಸ್ಯಕರವಾಗಿ ವರದಿಮಾಡಿತು:

“ಎಥಿಕಲ್ ಸೊಸೈಟಿಯ ಆಹ್ವಾನವನ್ನು ಸ್ವೀಕರಿಸಿ ಬಂದು ಉಪನ್ಯಾಸ ಭವನದಲ್ಲೂ ಸುತ್ತಲಿನ ಕೋಣೆಗಳಲ್ಲೂ ಕಿಕ್ಕಿರಿದು ತುಂಬಿದ್ದ ಅನೇಕ ನೂರು ಜನರನ್ನು ಮಂತ್ರಮುಗ್ಧವಾಗಿಸಿ ಹಿಡಿದಿಟ್ಟ ಆ ದನಿಯು ವೇದ ಪುಷಿಗಳ ದನಿಯಾಗಿತ್ತು.

“ವಿಶಿಷ್ಟ-ಸುಂದರವಾದ ಹಾಗೂ ನಿರರ್ಗಳ ವಾಗ್ಝರಿಯಿಂದ ಕೂಡಿದ ಅವರ ಭಾರತೀಯ ಧರ್ಮದ ಸಮರ್ಥನೆಯನ್ನು ಆಲಿಸಲು ನಗರದ ಎಲ್ಲೆಡೆಗಳಿಂದಲೂ ಎಲ್ಲ ಬಗೆಯ ವೃತ್ತಿ- ವ್ಯಾಪಾರಗಳ ಜನರು–ವೈದ್ಯರು, ವಕೀಲರು, ನ್ಯಾಯಾಧೀಶರು, ಅಧ್ಯಾಪಕರು ಹಾಗೂ ಅನೇಕ ಮಹಿಳೆಯರು ಆಗಮಿಸಿದ್ದರು.

“ಅವರ್ಯಾರೂ ನಿರಾಶರಾಗಲಿಲ್ಲ. ಸ್ವಾಮಿ ವಿವೇಕಾನಂದರು ತಮ್ಮ ಕೀರ್ತಿಗಿಂತಲೂ ದೊಡ್ಡ ವರು. ಹಿಮಾಲಯದ ಪ್ರಖ್ಯಾತ ಪುಷಿಗಳ ಭವ್ಯಪಂಕ್ತಿಗೆ ಸೇರಿದವರು. ಕ್ರೈಸ್ತರ ನೈತಿಕತೆ ಯೊಂದಿಗೆ ಬೌದ್ಧರ ತತ್ತ್ವವನ್ನು ಒಂದುಗೂಡಿಸುವ ನೂತನ ಧರ್ಮವೊಂದರ ಪ್ರವಾದಿ ಅವರು...”

ಬ್ರೂಕ್ಲಿನ್​ನಲ್ಲಿ ನಿಯಮಿತವಾಗಿ ತರಗತಿಗಳನ್ನು ನಡೆಸಬೇಕೆಂಬ ಜನರ ಬೇಡಿಕೆಗೆ ಸ್ವಾಮೀಜಿ ಸಮ್ಮತಿಸಿದರು. ಅಲ್ಲದೆ ಅಲ್ಲಿನ ಅಸೋಸಿಯೇಷನ್ನಿನಲ್ಲಿ ಮೂರು ಭಾಷಣಗಳ ಸರಣಿಯನ್ನು ಮಾಡಲೂ ಒಪ್ಪಿಕೊಂಡರು. ಇವುಗಳ ವಿಷಯಗಳು–‘ಹಿಂದೂ, ಮುಸಲ್ಮಾನ, ಕ್ರೈಸ್ತ ಧರ್ಮ ಗಳ ಪ್ರಕಾರ ಸ್ತ್ರೀತ್ವದ ಆದರ್ಶಗಳು’, ‘ಬೌದ್ಧಧರ್ಮ–ಭಾರತೀಯರ ದೃಷ್ಟಿಯಲ್ಲಿ’ ಮತ್ತು ‘ವೇದ ಹಾಗೂ ಹಿಂದೂಧರ್ಮ: ವಿಗ್ರಹಾರಾಧನೆಯೆಂದರೇನು?’ ಈ ಉಪನ್ಯಾಸಗಳಿಗೆ ಪ್ರವೇಶ ಶುಲ್ಕವನ್ನಿಟ್ಟಿದ್ದು, ಸಂಗ್ರಹವಾದ ಹಣದಲ್ಲಿ ಒಂದು ಭಾಗ ಅಸೋಸಿಯೇಷನ್ನಿನ ಪುಸ್ತಕ ಪ್ರಕಟಣೆಯ ನಿಧಿಗಾಗಿ ವಿನಿಯೋಗವಾಗಬೇಕೆಂಬ ವ್ಯವಸ್ಥೆಯಾಗಿತ್ತು. ಈ ಉಪನ್ಯಾಸಗಳೆಲ್ಲವೂ ಅತ್ಯಂತ ಯಶಸ್ವಿಯಾಗಿ ನಡೆದುದು, ಕಾರ್ಯಕ್ರಮದ ಸಂಚಾಲಕರಾದ ಹಿಗ್ಗಿನ್ಸರಿಗಂತೂ ಹಿಡಿಸ ಲಾರದಷ್ಟು ಸಂತಸವುಂಟುಮಾಡಿತು.

ಸ್ವಾಮೀಜಿ ಎಲ್ಲೇ ಭಾಷಣ ಮಾಡಿದರೂ ಅಲ್ಲೊಂದು ವಿವಾದದ ಭಾರೀ ಕೋಲಾಹಲ ವೇರ್ಪಡುತ್ತಿದ್ದುದು ಅನಿವಾರ್ಯವಾಗಿತ್ತು. ಏಕೆಂದರೆ ಅವರಾಡುತ್ತಿದ್ದ ಮಾತುಗಳು, ಜನರ ಕಿವಿಗೂ ಬುದ್ಧಿಗೂ ಅಪರಿಚಿತವಾದವು, ಅಪಥ್ಯವಾದವು. ಅವರ ಭಾವನೆಗಳು ಜನರು ಊಹಿಸಿಯೂ ಇರದಿದ್ದಂಥವು. ಹಳೆಯ ನಂಬಿಕೆಗಳಿಗೆ ಜೋತು ಬಿದ್ದವರಿಗಂತೂ ಸ್ವಾಮೀಜಿಯ ನೂತನ ಸಂದೇಶಗಳು ದಿಗ್ಭ್ರಾಂತಿಯನ್ನೂ ಅಸಹನೆಯನ್ನೂ ಉಂಟುಮಾಡುತ್ತಿದ್ದುವು. ಆದ್ದ ರಿಂದ, ಹೆಚ್ಚುಕಡಿಮೆ ಎಲ್ಲೆಲ್ಲೂ ಅವರ ವಿಚಾರಗಳಿಗೆ ಪ್ರತಿಭಟನೆಯೆ ಒಂದೊಂದು ದನಿಗಳು ಕೇಳಿಬರುತ್ತಿದ್ದುವು. ಇತರೆಡೆಗಳಲ್ಲಿ ಅವರನ್ನು ವಿರೋಧಿಸಿದವರು ಮುಖ್ಯವಾಗಿ ಕ್ರೈಸ್ತ ಮಿಷನರಿ ಗಳು ಹಾಗೂ ಇತರ ಧರ್ಮಾಂಧ ಕ್ರೈಸ್ತರು. ಆದರೆ ಬ್ಲೂಕ್ಲಿನ್ನಿನಲ್ಲಿ ಅವರಿಗೆ ವಿರೋಧ ಎದುರಾದದ್ದು ರಮಾಬಾಯಿಯ ಅನುಯಾಯಿಗಳಿಂದ. ಈ ಹಿಂದೆಯೇ ಹೇಳಲಾಗಿರುವಂತೆ, ಈ ರಮಾಬಾಯಿ ಎಂಬವಳು ಕ್ರೈಸ್ತಳಾಗಿ ಮತಾಂತರಗೊಂಡವಳು; ಚಿಕ್ಕವಯಸ್ಸಿನಲ್ಲೇ ಮದುವೆಯಾಗಿ ಗಂಡನನ್ನು ಕಳೆದುಕೊಂಡು ಬಳಿಕ ಮತ್ತೊಮ್ಮೆ ಮದುವೆಯಾಗಿದ್ದಳು. ಆದರೆ ಆಕೆ ಮತ್ತೆ ವಿಧವೆಯಾದಳು. ಅನಂತರ ಅವಳು ಭಾರತೀಯ ವಿಧವೆಯರ ನೆರವಿಗೆಂದು ಅಮೆರಿಕ ದಲ್ಲಿ ಸಂಘಗಳನ್ನು ಸ್ಥಾಪಿಸಿ ಧನ ಸಂಗ್ರಹಣೆ ಮಾಡುತ್ತಿದ್ದಳು. ಆದರೆ ರಮಾಬಾಯಿಯೂ ಅವಳ ಅನುಯಾಯಿಗಳೂ ಭಾರತೀಯ ಸ್ತ್ರೀಯರ ದುಃಸ್ಥಿತಿಯನ್ನು ಉತ್ಪ್ರೇಕ್ಷೆಯಾಗಿ ಬಣ್ಣಿಸಿ, ಅಮೆರಿಕದ ಜನರ–ಮುಖ್ಯವಾಗಿ ಸ್ತ್ರೀಯರ-ಸಹಾನುಭೂತಿಯನ್ನು, ತನ್ಮೂಲಕ ಹೆಚ್ಚಿನ ಧನ ಸಹಾಯವನ್ನು ಪಡೆಯುವ ಹವಣಿಕೆಯಲ್ಲಿದ್ದವರು. ಆದರೆ ಸ್ವಾಮೀಜಿ, ಹಿಂದೂ ಸಂಸ್ಕೃತಿಯಲ್ಲಿ ನಾರಿಯರಿಗೆ ನೀಡಲಾಗಿರುವ ಅತ್ಯುನ್ನತ ಸ್ಥಾನವನ್ನೂ ಸಮಾಜದಲ್ಲಿ ವಾಸ್ತವಿಕವಾಗಿಯೂ ಅವರಿಗಿದ್ದ ಗೌರವವನ್ನೂ ಜನಮನಮುಟ್ಟುವಂತೆ ವಿವರಿಸಿದಾಗ, ಈ ರಮಾಬಾಯಿಯ ಬೆಂಬಲಿ ಗರು ಸಿಟ್ಟಿಗೆದ್ದರು. ಆದರೆ ಸ್ವಾಮೀಜಿಯ ಮಾತುಗಳನ್ನು ಎದುರಿಸಲು ಅವರಲ್ಲಿ ಸತ್ತ್ವವಿರದಿದ್ದ ರಿಂದ, ಸುಳ್ಳು ಸುದ್ದಿಗಳನ್ನು ಹಬ್ಬಿಸಿ ಸ್ವಾಮೀಜಿಯ ತೇಜೋವಧೆ ಮಾಡುವ ಕುಯುಕ್ತಿಯಲ್ಲಿ ತೊಡಗಿದರು. ಇವುಗಳಲ್ಲೊಂದೆಂದರೆ ಶ್ರೀಮತಿ ಬ್ಯಾಗ್​ಲೀಯ ಮನೆಯಲ್ಲಿ ಒಬ್ಬಳು ಕೆಲಸದ ಹುಡುಗಿಯ ಮೇಲೆ ಸ್ವಾಮೀಜಿ ಯಾವುದೋ ಕಾರಣಕ್ಕೆ ಸಿಟ್ಟಾದ್ದರಿಂದ ಅವಳನ್ನು ಕೆಲಸದಿಂದಲೇ ತೆಗೆಯಬೇಕಾಯಿತು ಎಂಬುದು. ತಮಗೆ ಕೇಡೆಸಗಿದರವನ್ನೇ ಸ್ನೇಹಭಾವದಿಂದ ಕಾಣುವಷ್ಟು ಉದಾರಚರಿತರಾದ ಸ್ವಾಮೀಜಿ, ಯಾವ ಕಾರಣಕ್ಕೇ ಆದರೂ ಒಬ್ಬಳು ಬಡಸೇವಕಿಯ ಮೇಲೆ ಸಿಟ್ಟಾಗಲು ಸಾಧ್ಯವೆ! ಇತರರಲ್ಲಿ ಕೇವಲ ಒಳ್ಳೆಯದನ್ನೇ ಕಾಣುವ ಸ್ವಭಾವ ಅವರದ್ದು. ಬಡವರು-ದುರ್ಬಲರು ಎಂದರಂತೂ ಅವರ ಹೃದಯ ಕರಗಿ ನೀರಾಗುತ್ತಿತ್ತು. ಇಂತಹ ಸ್ವಾಮೀಜಿ ಕೆಲಸದ ಹುಡುಗಿಯ ಮೇಲೆ ಸಿಟ್ಟಿಗೆದ್ದು ಅವಳನ್ನು ಕೆಲಸದಿಂದ ತೆಗೆದುಹಾಕಿಸಿದ ರೆಂದರೆ, ಅದೆಂತಹ ಹೀನ ಆರೋಪ! ಪಾಪ, ಇಂತಹ ಸುದ್ದಿ ಕಿವಿಗೆ ಬಿದ್ದಾಗ ಸ್ವಾಮೀಜಿಗೆ ಎಂತಹ ಆಘಾತವಾಗಿರಬಹುದು!ಆದರೆ ಅವರಿಗೆ ಈಗಾಗಲೇ ಲೋಕದ ರೀತಿನೀತಿಗಳು ಗೊತ್ತಾಗಿತ್ತಲ್ಲವೆ? ಶ್ರೀಮತಿ ಸಾರಾಬುಲ್​ಗೆ ಒಂದು ಪತ್ರದಲ್ಲಿ ಬರೆಯುತ್ತಾರೆ, “ಒಬ್ಬ ಮನುಷ್ಯ ಎಷ್ಟೇ ಸಂಭಾವಿತನಾಗಿ ನಡೆದುಕೊಂಡರೂ ಅಂಥವನ ಮೇಲೂ ಅತ್ಯಂತ ಕೆಟ್ಟ ಅಪವಾದವನ್ನು ಹುಟ್ಟಿಸುವವರು ಇದ್ದೇ ಇರುತ್ತಾರೆ ಎಂಬುದನ್ನು ನೋಡಿದಿರಾ? ಶಿಕಾಗೋ ದಲ್ಲಂತೂ ದಿನ ಬೆಳಗಾದರೆ ನಾನು ಇಂತಹ ಪರಿಸ್ಥಿತಿಗಳನ್ನು ಎದುರಿಸಬೇಕಾಗುತ್ತಿತ್ತು ಮತ್ತು ಈ ರೀತಿಯ ಕೆಲಸಗಳನ್ನು ಮಾಡುತ್ತಿರುವವರೆಲ್ಲ ಪಕ್ಕಾ ಕ್ರೈಸ್ತರೆನ್ನಿಸಿಕೊಂಡವರೇ.”

ಶ್ರೀಮತಿ ಬ್ಯಾಗ್​ಲೀ ಸ್ವಾಮೀಜಿಯ ನಿಷ್ಕಳಂಕ ವ್ಯಕ್ತಿತ್ವವನ್ನು ಪ್ರತಿಪಾದಿಸಿ ಬರೆದ ಸುಂದರ ಪತ್ರವೊಂದನ್ನು ಈ ಹಿಂದೆಯೇ ನೋಡಿದ್ದೇವೆ. ಅದಾದ ಸುಮಾರು ಒಂದು ವರ್ಷದ ಅನಂತರ ಈಗ ರಮಾಬಾಯಿಯ ಅನುಯಾಯಿಗಳ ಅಪವಾದದ ಮಾತುಗಳನ್ನು ಕೇಳಿ ಆ ಬಗ್ಗೆ ತನ್ನನ್ನು ಪ್ರಶ್ನಿಸಿದ ತನ್ನ ಸ್ನೇಹಿತೆಯೊಬ್ಬಳಿಗೆ ಬರೆಯುತ್ತಾರೆ:

“ಸ್ವಾಮಿ ವಿವೇಕಾನಂದರ ಕುರಿತಾದ ಈ ಎಲ್ಲದರ ಬಗ್ಗೆ ನಾನು ಹೇಳುವುದಿಷ್ಟೆ–ಅವೆಲ್ಲ ನೂರಕ್ಕೆ ನೂರು ಸುಳ್ಳು ಎಂದು. ಇದಕ್ಕಿಂತ ಹೆಚ್ಚು ಸುಳ್ಳಾದುದು ಮತ್ತೊಂದಿರಲು ಸಾಧ್ಯವಿಲ್ಲ. ಅವರು ನಮ್ಮೊಂದಿಗಿದ್ದ ಕಳೆದ ಆರು ವಾರಗಳಲ್ಲಿ ಪ್ರತಿಯೊಂದು ದಿನವನ್ನೂ ನಾವು ಆನಂದಿಸಿ ದ್ದೇವೆ. ಉನ್ನತ ವರ್ಗದವರ ಅನೇಕ ಕ್ಲಬ್​ಗಳಿಗೆ ಅವರನ್ನು ಆಹ್ವಾನಿಸಲಾಗಿತ್ತು. ಅಲ್ಲದೆ ಹೆಚ್ಚೆಚ್ಚು ಜನ ಅವರ ಮಾತುಗಳನ್ನು ಕೇಳಲು ಅನುಕೂಲವಾಗುವಂತೆ ಅನೇಕ ಶ್ರೀಮಂತರು ಔತಣಕೂಟಗಳನ್ನು ಏರ್ಪಡಿಸಿದ್ದರು. ಎಲ್ಲೆಡೆಯೂ ಯಾವಾಗಲೂ ಅವರನ್ನು ಆದರಿಸಿ ಗೌರವಿಸ ಲಾಗುತ್ತಿತ್ತು. ಈ ಎಲ್ಲ ಗೌರವಕ್ಕೂ ಅವರು ಅರ್ಹರಾಗಿದ್ದರು. ಅವರನ್ನು ಕಂಡವರೆಲ್ಲ ಅವರ ಅತ್ಯಂತ ಶ್ರೇಷ್ಠ ಚಾರಿತ್ರ್ಯವನ್ನೂ ಶಕ್ತಿಯುತ ಆಧ್ಯಾತ್ಮಿಕ ಪ್ರಕೃತಿಯನ್ನೂ ಗೌರವಿಸದಿರಲು ಸಾಧ್ಯವೇ ಇಲ್ಲ. ಕಳೆದ ಬೇಸಿಗೆಯಲ್ಲಿ ಆ್ಯನಿಸ್ಕ್ವಾಮ್​ನ ನಮ್ಮ ಮನೆಗೆ ಬರುವಂತೆ ಅವರನ್ನು ಆಮಂತ್ರಿಸಿದೆವು. ಅವರು ಅಲ್ಲಿಗೆ ಭೇಟಿ ನೀಡಿ, ಮೂರು ವಾರ ಉಳಿದುಕೊಂಡು ನಮ್ಮ ಮೇಲೆ ಕೃಪೆ ಮಾಡಿದರಷ್ಟೇ ಅಲ್ಲದೆ, ನಮ್ಮ ಸುತ್ತಮುತ್ತಲಿದ್ದವರಿಗೆಲ್ಲ ಆನಂದವುಂಟುಮಾಡಿದರು. ನಮ್ಮ ಮನೆಯಲ್ಲಿ ಹಲವಾರು ವರ್ಷಗಳಿಂದಲೂ ಅನೇಕ ಸೇವಕರಿದ್ದರು ಮತ್ತು ಅವರಲ್ಲಿ ಪ್ರತಿಯೊಬ್ಬರೂ ಈಗಲೂ ಇದ್ದಾರೆ. ಇದನ್ನು ಯಾರು ಬೇಕಾದರೂ ಕಣ್ಣಾರೆ ಕಂಡು ದೃಢಪಡಿಸಿ ಕೊಳ್ಳಬಹುದು. ಆದ್ದರಿಂದ ಅವರ ಬಗ್ಗೆ ಹುಟ್ಟಿಸಲಾದ ಕತೆಗಳಿಗೆ ಕಾಲೇ ಇಲ್ಲ ಎಂಬುದನ್ನು ನೀನೇ ಕಂಡುಕೊಳ್ಳಬಹುದು. ನೀನು ಹೇಳುವ ಆ ಡೆಟ್ರಾಯ್ಟ್​ನ ಹೆಂಗಸು ಯಾರೋ ನನಗಂತೂ ಗೊತ್ತಿಲ್ಲ. ನನಗೆ ಗೊತ್ತಿರುವುದೇನೆಂದರೆ ಅವಳ ಕತೆಯ ಪ್ರತಿಯೊಂದು ಪದವೂ ಸುಳ್ಳು–ಹಸೀ ಸುಳ್ಳು ಎಂಬುದಷ್ಟೇ. ವಿವೇಕಾನಂದರು ನಮಗೆಲ್ಲ ಚೆನ್ನಾಗಿಯೇ ತಿಳಿದಿರುವವರು. ಅವರ ಬಗ್ಗೆ ಹಾಗೆಲ್ಲ ಅಬದ್ಧವಾಗಿ ಮಾತನಾಡಲು ಇವರ್ಯಾರು?”

ಹೀಗೆ, ಸಮಾಜದ ಅತ್ಯಂತ ಪ್ರತಿಷ್ಠಿತರಲ್ಲೊಬ್ಬರಾದ ಶ್ರೀಮತಿ ಬ್ಯಾಗ್​ಲೀ, ಸ್ವಾಮೀಜಿಯ ವಿರುದ್ಧ ನಡೆದ ಮಿಥ್ಯಾಪವಾದಗಳ ಪ್ರಚಾರವನ್ನು ಗಂಭೀರವಾಗಿಯೇ ಪ್ರತಿಭಟಿಸಿದರು. ಈಕೆಯ ಮಗಳಾದ ಹೆಲೆನ್ ಕೂಡ ಇಂತಹ ಅಪಪ್ರಚಾರದ ಬಗ್ಗೆ ಸಿಟ್ಟಿಗೆದ್ದು ತನ್ನೊಬ್ಬ ಸ್ನೇಹಿತೆಗೆ ಹೀಗೆ ಬರೆಯುತ್ತಾಳೆ:

“ಆ ಕತೆ ಶ್ರೀಮತಿ ‘ರ’ಳಿಂದ ಪ್ರಚಾರವಾದದ್ದಲ್ಲ ಎಂಬುದನ್ನು ತಿಳಿದು ಸಂತೋಷವಾಯಿತು. ಸಾಧ್ಯವಾದರೆ ನಾನು ಶ್ರೀಮತಿ ‘ಸ’ಳನ್ನು ಮುಖತಃ ನೋಡಿ ಆ ಬಗೆಯ ಮಾತುಗಳನ್ನಾಡಲು ಅವಳ ಬಳಿ ಏನು ಆಧಾರವಿತ್ತು ಎಂದು ಕೇಳುತ್ತೇನೆ. ನಾನಿದನ್ನು ಶಾಂತವಾಗಿಯೇ ಮಾಡುತ್ತೇನೆ; ಆದರೆ ವಿವೇಕಾನಂದರ ಬಗೆಗಿನ ಈ ಸುಳ್ಳು ಸುದ್ದಿಗಳನ್ನೆಲ್ಲ ಯಾರು ಹರಡುತ್ತಾರೆಂಬುದನ್ನು ಸಾಧ್ಯವಾದರೆ ಪತ್ತೆಹಚ್ಚಿಯೇ ಬಿಡಲಿದ್ದೇನೆ. ಇಂತಹ ಸುದ್ದಿಗಳು ಬಹಳ ವೇಗವಾಗಿ ಸಂಚರಿಸಿ ಬಿಡುತ್ತವೆ. ಇವುಗಳಲ್ಲಿ ಒಂದನ್ನು ಒಂದು ಸಲ ಬುಡಮೇಲು ಮಾಡಿದರೆ, ಬಹುಶಃ ಈ ಹೆಂಗಸರು, ಅಷ್ಟು ಸುಲಭವಾಗಿ ಕತೆ ಹುಟ್ಟುಹಾಕುವ ಮೊದಲು ಒಮ್ಮೆ ಆಲೋಚಿಸುತ್ತಾರೆ.”

ಸ್ವಾಮೀಜಿಯ ಬಗ್ಗೆ ಮತ್ತು ಭಾರತದ ಬಗ್ಗೆ ರಮಾಬಾಯಿ ತಂಡದವರು ನಡೆಸುತ್ತಿದ್ದ ಅಪಪ್ರಚಾರ ಎಷ್ಟು ಜೋರಾಗಿತ್ತೆಂದರೆ, ಅದರ ದುಷ್ಪರಿಣಾಮವನ್ನು ಎದುರಿಸಲು ಸ್ವಾಮೀಜಿ ಬ್ರೂಕ್ಲಿನ್ನಿನಲ್ಲಿ ಮತ್ತೊಂದು ವಿಶೇಷ ಉಪನ್ಯಾಸ ಮಾಡಬೇಕಾಯಿತು. ಈ ಉಪನ್ಯಾಸದ ವಿಷಯ–‘ಹಿಂದೂಗಳ ಕೆಲವು ಸಂಪ್ರದಾಯಗಳು: ಅವುಗಳ ನಿಜಾರ್ಥವೇನು ಮತ್ತು ಅವು ಗಳನ್ನು ಹೇಗೆ ಅಪಾರ್ಥಗೊಳಿಸಲಾಗಿದೆ’ ಎನ್ನುವುದು. ಭಾರತದ ಬಾಲವಿಧವೆಯರಿಗೆ ನೆರವನ್ನೂ ವಿದ್ಯಾಭ್ಯಾಸವನ್ನೂ ನೀಡುವುದಕ್ಕೆ ತಾವು ವಿರುದ್ಧವಾಗಿಲ್ಲವೆಂಬುದನ್ನು ಸ್ಪಷ್ಟಪಡಿಸುವುದಕ್ಕಾಗಿ, ಅವರು ಈ ಉಪನ್ಯಾಸಕ್ಕೆ ಪ್ರವೇಶ ಶುಲ್ಕವನ್ನು ಸಂಗ್ರಹಿಸಲಿಲ್ಲ; ಬದಲಾಗಿ ಶಶಿಪದ ಬ್ಯಾನರ್ಜಿ ಎಂಬಾತ ನಡೆಸುತ್ತಿದ್ದ ವಿಧವೆಯರ ಶಾಲೆಯ ಸಹಾಯಾರ್ಥವಾಗಿ ವಂತಿಗೆಯನ್ನು ಸಂಗ್ರಹಿಸಿ ಕೊಟ್ಟರು. ಈ ಉಪನ್ಯಾಸದಲ್ಲಿ, ಭಾರತದ ಧರ್ಮಸಂಸ್ಕೃತಿಗಳ ಬಗೆಗಿನ ಪಾಶ್ಚಾತ್ಯರ ಅವಿವೇಕದ ಹಾಗೂ ಹೀನಾಯವಾದ ಟೀಕೆಯ ಮಾತುಗಳನ್ನು ಖಂಡಿಸುತ್ತ ಸ್ವಾಮೀಜಿ ಗುಡುಗಿದರು– “ಭಾರತವು ತನ್ನ ಧರ್ಮದ ವಿಷಯದಲ್ಲಿ ಪ್ರಾಮಾಣಿಕ ನಿಷ್ಠೆಯಿಂದಿರುವವರೆಗೆ ಯಾವ ತೊಂದರೆಯೂ ಇಲ್ಲ. ಆದರೆ ನಾಸ್ತಿಕರಾದ ಈ ಪಾಶ್ಚಾತ್ಯರು ಅದರ ಮೇಲೆ ಲಗ್ಗೆ ಹಾಕಿ ತಮ್ಮ ಆಷಾಢಭೂತಿತನವನ್ನೂ ನಾಸ್ತಿಕತೆಯನ್ನೂ ತುಂಬುವುದರ ಮೂಲಕ ಭಾರತದ ಹೃದಯದ ಮೇಲೆ ಬಲವಾದ ಪೆಟ್ಟು ಹಾಕಿದ್ದಾರೆ. ಪಾಶ್ಚಾತ್ಯರಾಷ್ಟ್ರಗಳು ಭಾರತಕ್ಕೆ ಗಾಡಿಗಟ್ಟಲೆ ದೂಷಣೆ ಗಳನ್ನು ಮಣಗಟ್ಟಲೆ ಬೈಗುಳಗಳನ್ನು ಹಡಗುಗಟ್ಟಲೆ ತಿರಸ್ಕಾರವನ್ನು ಕಳಿಸಿಕೊಡುವ ಬದಲು ಪ್ರೀತಿಯ ಸ್ರೋತವನ್ನು ಹರಿಯಿಸಲಿ!” ಸ್ವಾಮೀಜಿ ರಮಾಬಾಯಿ ತಂಡದವರಿಗೆ ನೇರವಾಗಿ ಬದಲು ಹೇಳದಿದ್ದರೂ ಈ ಉಪನ್ಯಾಸವೇ ಸಾಕಷ್ಟು ಪರಿಣಾಮಕಾರಿಯಾಗಿ ಅವರೆಲ್ಲರ ಬಾಯಿ ಮುಚ್ಚಿಸಿಬಿಟ್ಟಿತು.

ನ್ಯೂಯಾರ್ಕಿನಲ್ಲಿ ಶಾಶ್ವತ ಕೇಂದ್ರವೊಂದನ್ನು ಸ್ಥಾಪಿಸುವ ಪ್ರಯತ್ನದಲ್ಲಿದ್ದ ಸ್ವಾಮೀಜಿ ಕೆಲ ಕಾಲದ ಮಟ್ಟಿಗೆ ಬ್ರೂಕ್ಲಿನ್ನಿಗೆ ಬಂದಿದ್ದರು. ಈ ಅವಧಿಯಲ್ಲೇ ಅವರೊಮ್ಮೆ ನ್ಯೂಯಾರ್ಕಿಗೆ ಹೋಗಿ ತಮ್ಮ ಕೆಲಸವನ್ನು ಮುಂದುವರಿಸಿದರು. ಬ್ರೂಕ್ಲಿನ್ನಿನಲ್ಲಿ ಅವರು ಅತ್ಯಂತ ಪರಿಣಾಮ ಕಾರಿಯಾದ ಉಪನ್ಯಾಸಗಳನ್ನು ನೀಡುತ್ತಿದ್ದರಾದರೂ ಆ ಕೆಲಸದಲ್ಲಿ ಅವರಿಗೇ ಸ್ವಲ್ಪವೂ ಆಸಕ್ತಿ ಇರಲಿಲ್ಲ. ೧೮೯೫ರ ಏಪ್ರಿಲ್​ನಲ್ಲಿ ಅವರು ತಮ್ಮ ಒಬ್ಬ ಶಿಷ್ಯನಿಗೆ ಹೀಗೆ ಬರೆದರು– “ಕಳೆದ ವರ್ಷವಿಡೀ ನಾನು ವೇದಿಕೆಗಳ ಮೇಲೆ ನಿಂತು ಸಾರ್ವಜನಿಕ ಭಾಷಣಗಳನ್ನು ಮಾಡಿ, ಚಪ್ಪಾಳೆ ಗಿಟ್ಟಿಸಿಕೊಂಡೆ. ಆದರೆ ಈಗ ನೋಡಿದರೆ, ಅವೆಲ್ಲ ನನ್ನ ಕೀರ್ತಿಯನ್ನು ಹೆಚ್ಚಿಸಿಕೊಳ್ಳುವುದಕ್ಕಾ ಯಿತು, ಅಷ್ಟೆ. ಮುಂದಿನ ಸುಭದ್ರ ಜನಾಂಗವು ನಿರ್ಮಾಣವಾಗಬೇಕಾದರೆ ಅತ್ಯಂತ ಸಹನೆ ಯಿಂದ ಶೀಲವಂತ ವ್ಯಕ್ತಿಗಳನ್ನು ತಯಾರಿಸಬೇಕು, ಸತ್ಯಸಾಕ್ಷಾತ್ಕಾರಕ್ಕಾಗಿ ತೀವ್ರತರ ಹೋರಾಟ ನಡೆಯಬೇಕು. ಈ ನಿಟ್ಟಿನಲ್ಲಿ ಕೆಲಸ ಮಾಡಿ ಕೆಲವು ಸ್ತ್ರೀಪುರುಷರನ್ನಾದರೂ ತರಬೇತುಗೊಳಿಸ ಬೇಕೆಂಬುದು ನನ್ನ ಆಕಾಂಕ್ಷೆ. ಆದರೆ ಎಷ್ಟರಮಟ್ಟಿಗೆ ಯಶಸ್ವಿಯಾದೇನೋ ಗೊತ್ತಿಲ್ಲ.”

ಈ ರೀತಿ ಆಲೋಚಿಸಿದ ಸ್ವಾಮೀಜಿ ನ್ಯೂಯಾರ್ಕಿನಲ್ಲಿ ತಮ್ಮ ಕಾರ್ಯ ಯೋಜನೆಯ ಹೊಸ ಹಂತವನ್ನು ಪ್ರಾರಂಭಿಸಿದರು.

