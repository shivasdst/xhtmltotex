
\chapter{ಕೇರಳದಲ್ಲಿ ಕೆಲದಿನಗಳು}

\noindent

ಮೈಸೂರಿನಿಂದ ಹೊರಟ ಸ್ವಾಮೀಜಿ ಶೋರನೂರು ನಿಲ್ದಾಣದಲ್ಲಿ ಇಳಿದು ಎತ್ತಿನ ಗಾಡಿಯಲ್ಲಿ ತ್ರಿಚೂರಿನ ಕಡೆಗೆ ಮುಂದುವರಿದರು. ಹೀಗೆ ಪಯಣಿಸುತ್ತಿರುವಾಗ ಆಗಿನ ಕೊಚ್ಚಿನ್ ರಾಜ್ಯದ ಶಿಕ್ಷಣ ಇಲಾಖೆಯ ಅಧಿಕಾರಿಯಾಗಿದ್ದ ಸುಬ್ರಮಣ್ಯ ಅಯ್ಯರ್ ಎಂಬವರ ಮನೆಯ ಮುಂದಿ ನಿಂದ ಹಾದುಹೋಗಬೇಕಾಗಿ ಬಂದಿತು. ಮನೆಯ ಮುಂದೆ ನಿಂತಿದ್ದ ಅಯ್ಯರರನ್ನು ನೋಡಿ ಸ್ವಾಮೀಜಿ ಗಾಡಿಯಿಂದ ಕೆಳಗಿಳಿದು, “ಇಲ್ಲೆಲ್ಲಾದರೂ ಸ್ನಾನ ಮಾಡಲು ಸರಿಯಾದ ಜಾಗ ವಿದೆಯೆ?” ಎಂದು ಕೇಳಿದರು. ಸ್ವಾಮಿಗಳ ನಿಲುವಿನಿಂದ ಪ್ರಭಾವಿತರಾದ ಅಯ್ಯರ್​ರವರು ತಮ್ಮ ಮನೆಯಲ್ಲೇ ಸ್ನಾನಕ್ಕೆ ಅನುಕೂಲ ಮಾಡಿಕೊಟ್ಟು ಅವರನ್ನು ತಮ್ಮ ಅತಿಥಿಯಾಗಿ ಉಳಿಸಿ ಕೊಂಡರು. ಈ ಸಂದರ್ಭದಲ್ಲಿ ಸ್ವಾಮೀಜಿ, ಉಷ್ಣದಿಂದಾಗಿಯೋ ಏನೋ ಗಂಟಲಲ್ಲಿ ಹುಣ್ಣಾಗಿ ಬಾಧೆಪಡುತ್ತಿದ್ದರು. ಸುಬ್ರಮಣ್ಯ ಅಯ್ಯರರು ಅವರನ್ನು ಸಮೀಪದ ಸರಕಾರೀ ಆಸ್ಪತ್ರೆಗೆ ಕರೆದೊಯ್ದು ಔಷಧಿ ಕೊಡಿಸಿದರು.

ಸ್ವಾಮೀಜಿ ತ್ರಿಚೂರಿನಲ್ಲಿ ಕೆಲದಿನವಿದ್ದು ಕ್ರಂಗನೂರಿಗೆ (ಕೋಡಂಗಲ್ಲೂರ್​) ಹೋದರು. ಇಲ್ಲಿ ಒಂದು ಪ್ರಸಿದ್ಧ ಕಾಳೀ ದೇವಾಲಯವಿದೆ. ಒಂದು ದಿನ ಬೆಳಗಿನ ಜಾವ ಸ್ವಾಮೀಜಿ ಈ ದೇವಾಲಯದ ಬಳಿಯಿದ್ದ ಒಂದು ಆಲದ ಮರದ ಬುಡದಲ್ಲಿ ಕುಳಿತಿರುವುದನ್ನು ಅಲ್ಲಿನ ಜನ ಕಂಡರು. ದೇವಸ್ಥಾನದ ಬಾಗಿಲಿನ್ನೂ ತೆರೆದಿರಲಿಲ್ಲ. ಸ್ವಲ್ಪಹೊತ್ತಿನ ಅನಂತರ ಅರ್ಚಕರು ಬಂದ ಮೇಲೆ ದೇವಸ್ಥಾನದ ಬಾಗಿಲು ತೆರೆಯಿತು. ಸ್ವಾಮೀಜಿ ಒಳಗೆ ಪ್ರವೇಶಿಸಲು ಹೋದರು. ಆದರೆ ಕಾವಲುಗಾರರು ಅವರನ್ನು ಒಳಗೆ ಬಿಡಲಿಲ್ಲ. ಸ್ವಾಮೀಜಿ ಸ್ವಲ್ಪವೂ ಬೇಸರಿಸಿಕೊಳ್ಳದೆ, ಹೊರಗಿ ನಿಂದಲೇ ಜಗನ್ಮಾತೆಗೆ ನಮಸ್ಕರಿಸಿ ತಾವು ಮೊದಲು ಕುಳಿತಿದ್ದ ಸ್ಥಳಕ್ಕೇ ಬಂದು ಕುಳಿತರು. ಸಾಕ್ಷಾತ್ ಭಗವಂತನೇ ಬ್ರಾಹ್ಮಣೇತರನಾಗಿ ಜನ್ಮತಳೆದು ಬಂದರೂ ಈ ಹಿಂದೂ ದೇವಸ್ಥಾನದೊಳಕ್ಕೆ ಪ್ರವೇಶವಿಲ್ಲ!

ದೇವಸ್ಥಾನಕ್ಕೆ ಬಂದ ಭಕ್ತರಲ್ಲಿ ಒಬ್ಬ ಯುವಕ ಸ್ವಾಮೀಜಿಯನ್ನು ನೋಡಿದ. ಅವರ ಕಾಷಾಯ ವಸ್ತ್ರ, ಅವರ ತೇಜಃಪುಂಜ ಕಣ್ಣುಗಳು ಅವನನ್ನು ಆಕರ್ಷಿಸಿದುವು. ಆದರೆ ಭಿಕಾರಿಯಂತೆ ಕುಳಿತಿದ್ದ ಈ ಸಂನ್ಯಾಸಿಯನ್ನು ಸ್ವಲ್ಪ ಲೇವಡಿ ಮಾಡುವ ಉದ್ದೇಶದಿಂದ ಹತ್ತಿರಕ್ಕೆ ಬಂದ. ಆದರೆ ಆಗ ಅವನಿಗೆ ಅರಿವಾಯಿತು, ಇವರು ತಾನು ಅಂದುಕೊಂಡಂತಹ ವ್ಯಕ್ತಿಯಲ್ಲ ಎಂದು. ಆ ಸಮಯಕ್ಕೆ ಕ್ರಂಗನೂರಿನ ಇಬ್ಬರು ರಾಜಕುಮಾರರು ಅಲ್ಲಿಗೆ ಬಂದರು. ಅವರ ಹೆಸರು ಕೊಚ್ಚುಣ್ಣಿ ತಂಬುರಾನ್ ಮತ್ತು ಭಟ್ಟನ್ ತಂಬುರಾನ್. ಆ ಯುವಕನು ರಾಜಕುಮಾರರಿಗೆ ಸ್ವಾಮೀಜಿಯ ಬಗ್ಗೆ ತಿಳಿಸಿ ಅವರನ್ನು ಆಲದ ಮರದ ಬುಡಕ್ಕೆ ಕರೆತಂದ. ರಾಜಕುಮಾರರಿಬ್ಬರೂ ಶಾಸ್ತ್ರಗಳಲ್ಲಿ ಪಾರಂಗತರು. ಸ್ವಾಮೀಜಿಯನ್ನು ನೋಡಿದ ತಕ್ಷಣ ಇವರೊಬ್ಬ ಸಾಮಾನ್ಯ ವ್ಯಕ್ತಿಯಲ್ಲ ಎಂಬುದು ಅವರಿಗೆ ಅರಿವಾಯಿತು. ಅವರಿಬ್ಬರು ಬಂದು ನಮಸ್ಕರಿಸಿದಾಗ, ತಮ್ಮನ್ನು ದೇವ ಸ್ಥಾನದೊಳಕ್ಕೆ ಬಿಡದಿರಲು ಕಾರಣವೇನೆಂದು ಕೇಳಿದರು ಸ್ವಾಮೀಜಿ. ಅದಕ್ಕೆ ರಾಜಕುಮಾರರು ಹೇಳಿದರು, “ಸ್ವಾಮೀಜಿ, ಅಪರಿಚಿತರ, ಅದರಲ್ಲೂ ಕೇರಳದ ಹೊರಗಿನಿಂದ ಬಂದವರ ಜಾತಿ ಯನ್ನು ತಿಳಿದುಕೊಳ್ಳುವುದು ಕಷ್ಟ. ಆದ್ದರಿಂದಲೇ ಈ ಕಟ್ಟುಪಾಡು.” ಆಗ ಈ ವಿಷಯವಾಗಿ ಸ್ವಾಮೀಜಿ ಹಾಗೂ ರಾಜಕುಮಾರರ ನಡುವೆ ಸಂಸ್ಕೃತದಲ್ಲಿ ವಾಗ್ವಾದ ನಡೆಯಿತು. ಈ ಸಂದರ್ಭ ದಲ್ಲಿ ರಾಜಕುಮಾರರಿಗೆ ಸ್ವಾಮೀಜಿಯ ಯೋಗ್ಯತೆಯೇನೆಂಬುದು ಅರಿವಾಯಿತು. ದೇವಸ್ಥಾನ ವನ್ನು ಪ್ರವೇಶಿಸಲು ಅವರಿಗೆ ವಿಶೇಷ ಅನುಮತಿ ನೀಡಿದರು. ಆದರೆ ಸ್ಥಳೀಯ ಸಂಪ್ರದಾಯದಲ್ಲಿ ಮಧ್ಯೆ ಬರಲು ಇಷ್ಟಪಡದೆ ಅವರು ದೇವಸ್ಥಾನದೊಳಗೆ ಹೋಗಲಿಲ್ಲ.

ಈ ರಾಜಕುಮಾರರು ಮತ್ತೂ ಎರಡು ದಿನ ಸ್ವಾಮೀಜಿಯೊಂದಿಗೆ ಶಾಸ್ತ್ರಗಳ ಸಂಬಂಧವಾಗಿ ವಾದ ಮಾಡಿ ಕೊನೆಗೆ ತಮ್ಮ ಸೋಲನ್ನೊಪ್ಪಿಕೊಂಡರು. ಅವರ ಸತ್ಸಂಗವನ್ನು ಬಯಸಿ ಮೂರನೆಯ ದಿನವೂ ಬಂದಾಗ ಸ್ವಾಮೀಜಿ ನೀರವಧ್ಯಾನದಲ್ಲಿ ಲೀನರಾಗಿದ್ದರು. ಅವರ ತೇಜೋಮಯ ಪ್ರಶಾಂತಮೂರ್ತಿ, ರಾಜಕುಮಾರರಿಗೆ ಶಾಸ್ತ್ರಗಳಲ್ಲಿ ವರ್ಣಿಸಿರುವ ಧ್ಯಾನಸಿದ್ಧನ ನೆನಪು ತಂದುಕೊಟ್ಟಿತು. ಅವರ ಧ್ಯಾನ ಮುಗಿಯುವವರೆಗೂ ಅಲ್ಲೇ ಕಾದು ನಿಂತು, ಅವರು ಬಹಿರ್ಮುಖರಾದ ಮೇಲೆ ರಾಜಕುಮಾರರು ಅವರಿಗೆ ಪ್ರಣಾಮ ಸಲ್ಲಿಸಿ, ಶಾಸ್ತ್ರಗಳ ಸಂಬಂಧ ವಾಗಿ ಸಂಸ್ಕೃತದಲ್ಲಿ ಸಂಭಾಷಿಸಿದರು.

ಸ್ವಾಮೀಜಿಯ ಬಗ್ಗೆ ಕೇಳಿ ತಿಳಿದ ರಾಜಮನೆತನದ ಅನೇಕ ಸ್ತ್ರೀಯರು ಅವರನ್ನು ಭೇಟಿಮಾಡಿ ಅವರೊಂದಿಗೆ ಶುದ್ಧ ಸಂಸ್ಕೃತದಲ್ಲಿ ಸಂಭಾಷಣೆ ನಡೆಸಿದರು. ಸ್ತ್ರೀಯರು ಇಷ್ಟು ನಿರರ್ಗಳವಾಗಿ ಸಂಸ್ಕೃತದಲ್ಲಿ ಮಾತನಾಡುವುದನ್ನು ಕಂಡು ಸ್ವಾಮೀಜಿಗೆ ಆಶ್ಚರ್ಯವಾಯಿತು. ಇದನ್ನವರು ಭಾರತದಲ್ಲಿ ಬೇರೆಲ್ಲೂ ಕಂಡಿರಲಿಲ್ಲ. ವಿದ್ಯೆಯಲ್ಲಿ ಸ್ತ್ರೀಯರು ಇಷ್ಟೊಂದು ಮುಂದುವರಿದ ದ್ದನ್ನು ಕಂಡು ಅವರಿಗೆ ಬಹಳ ಸಂತೋಷವಾಯಿತು.

ಆ ರಾಜಕುಮಾರರೂ ರಾಜಮನೆತನದ ವಿದ್ಯಾವಂತ ಮಹಿಳೆಯರೂ ಸ್ವಾಮೀಜಿಯೊಡನೆ ಸಂಸ್ಕೃತದಲ್ಲಿ ಸಂಭಾಷಿಸಿದ್ದೇನೋ ಸರಿಯೆ. ಆದರೆ ಅವರು ಸ್ವಾಮೀಜಿಯನ್ನು ಮನೆಗೆ ಕರೆದು ಉಪಚರಿಸುವಷ್ಟು ಸಂಸ್ಕೃತಿಯನ್ನು ತೋರಲಿಲ್ಲವಲ್ಲ! ಕಾರಣವೇನಿದ್ದಿರಬಹುದು? ಕಾರಣ ಸ್ಪಷ್ಟವಾಗಿಯೇ ಇದೆ–ಜಾತಿ ಭಾವನೆ! ಸ್ವಾಮೀಜಿ ಬ್ರಾಹ್ಮಣಸಂನ್ಯಾಸಿಯಲ್ಲವೆಂಬ ಕಾರಣ ದಿಂದ ಅವರಿಗೆ ಇತ್ತ ದೇವಸ್ಥಾನಕ್ಕೂ ಪ್ರವೇಶವಿಲ್ಲ, ಅತ್ತ ಅರಮನೆಗೂ ಪ್ರವೇಶವಿಲ್ಲ! ಚಿಂತೆ ಯಿಲ್ಲ; ಸಂನ್ಯಾಸಿಗೆ ‘ಕರತಲ ಭಿಕ್ಷಾ, ತರುತಲ ವಾಸ’ ಇದ್ದೇ ಇದೆ.

ರಾಜಕುಮಾರರಿಬ್ಬರೂ ನಾಲ್ಕನೆಯ ದಿನವೂ ಸ್ವಾಮೀಜಿಯನ್ನು ಕಾಣಲು ಬಂದರು. ಆದರೆ ಅವರು ಆ ಆಲದಮರದ ಬುಡದಲ್ಲಿ ಇರಲಿಲ್ಲ. ಅವರು ಕ್ರಂಗನೂರನ್ನು ಬಿಟ್ಟು ಕೊಚ್ಚಿನ್ನಿನ ಕಡೆಗೆ ಹೊರಟು ಹೋಗಿದ್ದರು. ರಾಜಕುಮಾರರಿಗೆ ನಿರಾಶೆಯಾಯಿತು. ಅವರಿಗೆ ಸ್ವಾಮೀಜಿಯ ಪೂರ್ವಾಪರಗಳಾವುವೂ ಗೊತ್ತಾಗಿರಲಿಲ್ಲ. ಹಲವಾರು ತಿಂಗಳ ಮೇಲೆ ಶಿಕಾಗೋದ ವಿಶ್ವಧರ್ಮ ಸಮ್ಮೇಳನದಲ್ಲಿ ಅದ್ಭುತ ಯಶಸ್ಸನ್ನು ಗಳಿಸಿದ ಸ್ವಾಮೀಜಿಯೊಬ್ಬರ ಭಾವಚಿತ್ರವನ್ನು ನೋಡಿ ದಾಗ ಅವರಿಗೆ ಗೊತ್ತಾಯಿತು–ಅಂದು ತಾವು ಭೇಟಿಯಾದ ಸಂನ್ಯಾಸಿ ಇವರೇ ಎಂದು. ಆಗಲೇ ಅವರಿಗೆ ಸ್ವಾಮೀಜಿಯ ಹೆಸರು ಗೊತ್ತಾದದ್ದು.

ಈಗ ನಾವು ಸ್ವಾಮೀಜಿಯನ್ನು ಕಾಣುವುದು ಎರ್ನಾಕುಲಂನಲ್ಲಿ. ಆ ಊರಿನ ಚಂದೂಲಾಲನ್ ಮತ್ತು ರಾಮಯ್ಯ ಎಂಬವರು ಪ್ರತಿದಿನ ಬೆಳಿಗ್ಗೆ-ಸಂಜೆ ವಾಯುವಿಹಾರಕ್ಕೆ ಹೋಗುತ್ತಿದ್ದರು. ಒಂದು ದಿನ ಅವರು ಎಂದಿನಂತೆ ಸಮುದ್ರತೀರದಲ್ಲಿ ತಿರುಗಾಡುತ್ತಿದ್ದಾಗ ದೋಣಿಯೊಂದು ಬರುತ್ತಿರುವುದನ್ನು ಕಂಡರು. ಕಾಷಾಯವಸ್ತ್ರಧಾರಿಯಾಗಿ ಕೈಯಲ್ಲಿ ದಂಡ ಹಿಡಿದಿದ್ದ ಸಂನ್ಯಾಸಿ ಗಳೊಬ್ಬರು ದೋಣಿಯಿಂದ ಕೆಳಗಿಳಿದರು. ಅವರನ್ನು ಕಂಡಕೂಡಲೇ ಇಬ್ಬರು ಸ್ನೇಹಿತರೂ ಅವರೊಬ್ಬ ಮಹಾಪುರುಷರಿರಬೇಕು ಎಂದು ಊಹಿಸಿ ಅವರ ಭೇಟಿಯ ಅವಕಾಶವನ್ನು ಕಳೆದು ಕೊಳ್ಳಬಾರದೆಂದು ನಿಶ್ಚಯಿಸಿದರು. ಹೋಗಿ ಮಾತನಾಡಿಸಿದಾಗ ಅವರು ಹಿಂದಿಯಲ್ಲಿ ಮಾತ ನಾಡಿದರು. ಈ ಸ್ನೇಹಿತರಿಗೆ ಹಿಂದಿ ಬರುತ್ತಿರಲಿಲ್ಲವಾಗಿ, “ನಿಮಗೆ ಇಂಗ್ಲಿಷ್ ಬರುತ್ತದೆಯೆ?” ಎಂದು ಕೇಳಿದಾಗ ಸ್ವಾಮೀಜಿ, “ಅಲ್ಪಸ್ವಲ್ಪ ಗೊತ್ತಿದೆ” ಎಂದರು. ಬಳಿಕ ಸ್ವಾಮೀಜಿ ಅವರ ಲ್ಲೊಬ್ಬರ ಮನೆಯಲ್ಲಿ ಇಳಿದುಕೊಂಡರು. ಈ ವೇಳೆಗೆ ಅವರಿಗೆ ಗೊತ್ತಾಯಿತು–ಸ್ವಾಮೀಜಿ ಇಂಗ್ಲಿಷ್ ಭಾಷೆಯಲ್ಲಿ ನಿಷ್ಣಾತರು ಎಂದು. ಸುದ್ದಿ ಬೇಗನೆ ಹರಡಿತು. ಎರ್ನಾಕುಲಂನ ಜನ ಸ್ವಾಮೀಜಿಯನ್ನು ಮುತ್ತಿಕೊಂಡರು.

ಈ ಸಮಯದಲ್ಲಿ ಚಟ್ಟಂಬಿ ಸ್ವಾಮಿಗಳೆಂಬ ಒಬ್ಬ ಮಹಾತ್ಮರು ಎರ್ನಾಕುಲಂನಲ್ಲಿದ್ದರು. ಇವರ ಶಿಷ್ಯರೇ ಮುಂದೆ ಪ್ರಸಿದ್ಧರಾದ ನಾರಾಯಣ ಗುರುಗಳು. ಒಂದು ದಿನ ಚಟ್ಟಂಬಿ ಸ್ವಾಮಿ ಗಳು ಸ್ವಾಮೀಜಿಯನ್ನು ನೋಡಲು ಬಂದರು. ಆದರೆ ಸ್ವಾಮೀಜಿಯ ಸುತ್ತ ಜನ ಸೇರಿದ್ದರಿಂದ ಅವರನ್ನು ದೂರದಿಂದಲೇ ನೋಡಿಕೊಂಡು ಹೊರಟು ಹೋದರು. ಸ್ಥಳೀಯ ಜನ ಸ್ವಾಮೀಜಿಗೆ ಚಟ್ಟಾಂಬಿ ಸ್ವಾಮಿಗಳ ಕುರಿತಾಗಿ ತಿಳಿಸಿದರು. ಕೆಲವರು, ಅವರನ್ನು ಕರೆತಂದು ಸ್ವಾಮೀಜಿಗೆ ಪರಿಚಯ ಮಾಡಿಸುವುದಾಗಿ ಹೇಳಿದರು. ಅದಕ್ಕೆ ಸ್ವಾಮೀಜಿ, “ನೀವು ಹೇಳುವಂತೆ ಅವರೊಬ್ಬ ಮಹಾತ್ಮರಾಗಿದ್ದಲ್ಲಿ ಅವರು ನನ್ನನ್ನೇಕೆ ಹುಡುಕಿಕೊಂಡು ಬರಬೇಕು? ನಾನೇ ಹೋಗಿ ಅವರನ್ನು ನೋಡುತ್ತೇನೆ” ಎಂದು ಹೇಳಿ ತಾವೇ ಹೊರಟು, ಅವರನ್ನು ಭೇಟಿಯಾದರು.

ಚಟ್ಟಂಬಿ ಸ್ವಾಮಿಗಳಿಗೆ ಹಿಂದಿ ಬರುತ್ತಿರಲಿಲ್ಲವಾದ್ದರಿಂದ ಇಬ್ಬರೂ ಸಂಸ್ಕೃತದಲ್ಲಿ ಸಂಭಾಷಿಸಿದರು. ಏಕಾಂತದಲ್ಲಿ ಮಾತನಾಡಲು ಸಾಧ್ಯವಾಗುವಂತೆ ಅವರು ಸ್ವಾಮೀಜಿಯೊಂದಿಗೆ ಒಂದು ಮರದ ನೆರಳಿಗೆ ಹೋದರು. ಆ ಮರದ ಮೇಲೆ ಸ್ವಾಮಿಗಳು ಸಾಕಿದ್ದ ಕೋತಿಯೊಂದು ಕುಳಿತಿತ್ತು. ಈಗ ಇವರಿಬ್ಬರು ಮಾತನಾಡುತ್ತಿದ್ದಾಗ ಸುಮ್ಮನಿರಲಾರದ ಕೋತಿ ಮರದ ಕೊಂಬೆ ಯನ್ನು ಅಲ್ಲಾಡಿಸತೊಡಗಿತು. ಸ್ವಾಮೀಜಿ ಮೇಲೆ ನೋಡಿ ಕೋತಿಯ ಚಪಲಚಿತ್ತವನ್ನು ಕಂಡು ಹೇಳುತ್ತಾರೆ, “ನನ್ನ ಮನಸ್ಸಿನಂತೆಯೇ!” ತಕ್ಷಣ ಚಟ್ಟಂಬಿ ಸ್ವಾಮಿಗಳು ನುಡಿದರು, “ನಿಮ್ಮಂತಹ ಮಹಾತ್ಮರು ಮಾತ್ರವೇ ಹಾಗೆ ಹೇಳಲು ಸಾಧ್ಯ.”

ಸ್ವಾಮೀಜಿಯ ಕಂಠಮಾಧುರ್ಯವನ್ನು ಕಂಡು ಚಟ್ಟಂಬಿ ಸ್ವಾಮಿಗಳು ಮಂತ್ರ ಮುಗ್ಧರಾಗಿ ಬಿಟ್ಟಿದ್ದರು. ಆ ಕುರಿತಾಗಿ ಮಾತನಾಡುತ್ತ ಅವರೊಮ್ಮೆ ಉದ್ಗರಿಸುತ್ತಾರೆ. “ಓಹ್​! ಸ್ವಾಮೀಜಿ ಯವರ ಕಂಠ ಎಷ್ಟು ಮಧುರವಾಗಿತ್ತು! ಅವರು ಹಾಡುತ್ತಿದ್ದರೆ, ಚಿನ್ನದ ಕೊಡವನ್ನು ಬಡಿದಂತೆ ಸದ್ದಾಗುತ್ತಿತ್ತು.” ಒಮ್ಮೆ ಸ್ವಾಮಿಗಳಿಬ್ಬರೂ ಸಂಸ್ಕೃತದಲ್ಲಿ ಸಂಭಾಷಿಸುತ್ತಿದ್ದರು. ಆಗ ಸಮೀಪ ದಲ್ಲಿ ಪಂಡಿತನೊಬ್ಬ ಕುಳಿತು ಅವರ ಮಾತನ್ನು ಕೇಳುತ್ತಿದ್ದ. ಒಂದು ಸಲ ಆತ ಮಾತಿನ ಮಧ್ಯ ದಲ್ಲಿ ಪ್ರವೇಶಿಸಿ ಸ್ವಾಮೀಜಿಯ ಮಾತಿನಲ್ಲಿ ವ್ಯಾಕರಣದೋಷವೊಂದನ್ನು ಎತ್ತಿ ತೋರಿಸಿದ. ಆಗ ಸ್ವಾಮೀಜಿ ಹೇಳುತ್ತಾರೆ, “ನಾನು ವ್ಯಾಕರಣವನ್ನು ಅನುಸರಿಸಿ ಹೋಗಬೇಕಾಗಿಲ್ಲ; ವ್ಯಾಕರಣವೇ ನನ್ನನ್ನು ಅನುಸರಿಸುತ್ತದೆ.”

ಸ್ವಾಮೀಜಿಯ ವ್ಯಕ್ತಿತ್ವವನ್ನು ಬಹಳವಾಗಿ ಮೆಚ್ಚಿಕೊಂಡಿದ್ದ ಚಟ್ಟಂಬಿ ಸ್ವಾಮಿಗಳು ಅವರ ಬಗ್ಗೆ ಒಮ್ಮೆ ಹೇಳುತ್ತಾರೆ, “ವಿವೇಕಾನಂದ ಸ್ವಾಮಿಗಳಿಗೂ ನನಗೂ ಇರುವ ಅಂತರವೆಂದರೆ ಗರುಡ ಹಾಗೂ ಸೊಳ್ಳೆಯ ನಡುವಿನ ಅಂತರದಂತೆ!”

ಎರ್ನಾಕುಲಂನಲ್ಲಿ ಸ್ವಾಮೀಜಿ ಕೊಚ್ಚಿನ್ ರಾಜ್ಯದ ದಿವಾನರಾದ ಶಂಕರಯ್ಯನವರನ್ನು ಭೇಟಿ ಯಾದರು. ತನ್ಮೂಲಕ ಮಹಾರಾಜನ ಭೇಟಿ ಆಯಿತು. ಎರ್ನಾಕುಲಂನಿಂದ ಹೊರಟ ಸ್ವಾಮೀಜಿ ತಿರುವಾಂಕೂರಿನ ಮೂಲಕ ತಿರುವನಂತಪುರಕ್ಕೆ ಬಂದು ತಲುಪಿದರು. ಎರ್ನಾಕುಲಂನ ದಿವಾನರು ಒಬ್ಬ ಮುಸಲ್ಮಾನ ಸೇವಕನನ್ನು ಸ್ವಾಮೀಜಿಯ ಜೊತೆಗೆ ಕಳಿಸಿಕೊಟ್ಟಿದ್ದರು. ತಿರುವನಂತಪುರ ದಲ್ಲಿ ಸ್ವಾಮೀಜಿ ಪ್ರೊ. ॥ ಕೆ. ಸುಂದರರಾಮ ಅಯ್ಯರ್ ಎಂಬವರ ಮನೆಗೆ ಹೋದರು. ಅಯ್ಯರ್ ರವರ ಹನ್ನೆರಡು ವರ್ಷದ ಮಗ, ಮುಸಲ್ಮಾನನೊಬ್ಬನ ಜೊತೆಯಲ್ಲಿ ಬಂದ ಸಂನ್ಯಾಸಿಯನ್ನು ಕಂಡು ಒಳಗೆ ಓಡಿಹೋಗಿ ತನ್ನ ತಂದೆಗೆ ಕೂಗಿಹೇಳಿದ, “ಅಪ್ಪಾ, ಯಾರೋ ಒಬ್ಬ ಮುಸಲ್ಮಾನ ಫಕೀರರು ಬಂದಿದ್ದಾರೆ!” ಅಯ್ಯರರು ಹೊರಗೆ ಬಂದು ನೋಡಿದರೆ ಫಕೀರನಲ್ಲ, ಹಿಂದೂ ಸಂನ್ಯಾಸಿಯೇ! ದಕ್ಷಿಣಭಾರತದ ಸಂನ್ಯಾಸಿಗಳಂತಲ್ಲದ ಸ್ವಾಮೀಜಿಯವರ ಉಡಿಗೆತೊಡಿಗೆ ಯನ್ನೂ ಅವರೊಂದಿಗಿದ್ದ ಮುಸಲ್ಮಾನ ಸೇವಕನನ್ನೂ ನೋಡಿದ ಆ ಬಾಲಕನಿಗೆ ಅವರೊಬ್ಬ ಫಕೀರನಂತೆ ಕಂಡಿರಬೇಕು. ಸುಂದರರಾಮ ಅಯ್ಯರರು ಸ್ವಾಮೀಜಿಯನ್ನು ಸ್ವಾಗತಿಸಿದರು. ಸ್ವಾಮೀಜಿ ಅವರನ್ನು ಮೊದಲು ಕೇಳಿಕೊಂಡದ್ದೇ, “ಈ ನನ್ನ ಸೇವಕನ ಊಟಕ್ಕೆ ಏನಾದರೂ ವ್ಯವಸ್ಥೆ ಮಾಡುತ್ತೀರಾ?” ಎಂದು. ಸ್ವತಃ ಸ್ವಾಮೀಜಿ ಈ ಎರಡು ದಿನಗಳ ಪ್ರಯಾಣದಿಂದ ಬಳಲಿದ್ದರು. ಅಲ್ಲದೆ ಸ್ವಲ್ಪ ಹಾಲನ್ನು ಬಿಟ್ಟರೆ ಬೇರೇನನ್ನೂ ತೆಗೆದುಕೊಂಡಿರಲಿಲ್ಲ. ಆದರೆ ಈಗ ಅವರು ತಮ್ಮ ಸೇವಕನ ಊಟೋಪಚಾರಗಳ ವ್ಯವಸ್ಥೆ ಮಾಡಿ, ಬಳಿಕ ಅವನನ್ನು ಕಳಿಸಿಕೊಟ್ಟ ಮೇಲೆಯೇ ತಮ್ಮ ಆವಶ್ಯಕತೆಗಳ ಕಡೆಗೆ ಗಮನಹರಿಸಿದ್ದು.

ಸ್ವಾಮೀಜಿಯೊಡನೆ ಒಂದೆರಡು ಮಾತನಾಡುವಷ್ಟರಲ್ಲಿ ಅವರದು ಎಂತಹ ಭವ್ಯ ವ್ಯಕ್ತಿತ್ವ ಎಂಬುದು ಅಯ್ಯರರಿಗೆ ಮನವರಿಕೆಯಾಯಿತು. ಅವರು ಎರ್ನಾಕುಲಂನಿಂದ ಹೊರಟಾಗಿನಿಂದ ಊಟವನ್ನೇ ಮಾಡಿಲ್ಲ ಎಂದು ತಿಳಿದಾಗ ಅಯ್ಯರರಿಗೆ ದುಃಖವಾಯಿತು. ಆದರೆ ಉತ್ತರ ಭಾರತೀಯರಾದ ಅವರ ಆಹಾರ ಕ್ರಮವೆಂಥದೋ! ಆದ್ದರಿಂದ ಅವರನ್ನೇ ಕೇಳಿದರು, “ಸ್ವಾಮೀಜಿ, ನೀವೀಗ ಏನು ತೆಗೆದುಕೊಳ್ಳುತ್ತೀರಿ?”

“ಏನಾದರೂ ಸರಿಯೆ; ನಾವು ಸಂನ್ಯಾಸಿಗಳಿಗೆ ಇಂಥದೇ ಆಗಬೇಕು, ಅಂಥದೇ ಆಗಬೇಕು ಎಂದೇನಿಲ್ಲ.”

ಅಡಿಗೆಯಾಗಲು ಇನ್ನೂ ಸ್ವಲ್ಪ ಸಮಯವಿತ್ತು. ಆದ್ದರಿಂದ ಇಬ್ಬರೂ ಮಾತನಾಡುತ್ತ ಕುಳಿ ತರು. ಸ್ವಾಮೀಜಿ ಬಂಗಾಳಿಗಳು ಎಂದು ತಿಳಿದಾಗ ಅಯ್ಯರರು, ‘ಅನೇಕ ಮಹಾಪುರುಷರಿಗೆ ಜನ್ಮ ನೀಡಿದ ರಾಜ್ಯ ಬಂಗಾಳ’ ಎಂದು ಉದ್ಗರಿಸಿದರು. ಅಲ್ಲದೆ ಕೇಶವಚಂದ್ರಸೇನನನ್ನಂತೂ ಅವರು ಬಾಯ್ತುಂಬ ಹೊಗಳಿದರು.

ಆದರೆ, ಯಾರ ಅವತರಣದಿಂದಾಗಿ ಬಂಗಾಳ ಮಾತ್ರವಲ್ಲದೆ ಸಮಸ್ತ ಭಾರತವೇ ಪುನೀತ ವಾಗಿದೆಯೋ ಅಂತಹ ಶ್ರೀರಾಮಕೃಷ್ಣರ ಹೆಸರೂ ತಿಳಿದಿಲ್ಲ ಅಯ್ಯರರಿಗೆ! ಸ್ವತಃ ಕೇಶವಚಂದ್ರ ಸೇನನೂ ಯಾರಿಂದ ತೀವ್ರವಾಗಿ ಪ್ರಭಾವಿತನಾಗಿದ್ದನೋ, ಯಾರನ್ನು ಕೊಂಡಾಡಿದ್ದನೋ ಅಂತಹ ಶ್ರೀರಾಮಕೃಷ್ಣರ ಪರಮಹಂಸರ ಪರಿಚಯವೇ ಇಲ್ಲ! ಬಹುಶಃ ತಿಳಿದಿರಲು ಸಾಧ್ಯವೂ ಇರಲಿಲ್ಲ. ಆದ್ದರಿಂದ ಸ್ವಾಮೀಜಿ, ತಮ್ಮ ಗುರುದೇವನ ಹೆಸರನ್ನು ಪ್ರಸ್ತಾಪಿಸಿ ಅವರ ಜೀವನ- ಸಾಧನೆಗಳ ಒಂದು ಸೂಕ್ಷ್ಮಪರಿಚಯ ಮಾಡಿಕೊಟ್ಟರು. ಶ್ರೀರಾಮಕೃಷ್ಣರ ಮುಂದೆ ಕೇಶವಚಂದ್ರ ಸೇನ ಕೇವಲ ಒಂದು ಶಿಶು ಎಂದು ಸ್ವಾಮೀಜಿ ಹೇಳಿದಾಗ ಸುಂದರರಾಮ ಅಯ್ಯರರು ಆಶ್ಚರ್ಯ ಚಕಿತರಾದರು. ಬಳಿಕ ಸ್ವಾಮೀಜಿ ತಮ್ಮ ಮಾತನ್ನು ಮುಂದುವರಿಸುತ್ತ ಹೇಳುತ್ತಾರೆ:

“ಕೇಶವಚಂದ್ರಸೇನರು ಮಾತ್ರವೇ ಅಲ್ಲದೆ, ಹಿಂದಿನ ಹಾಗೂ ಇಂದಿನ ತಲೆಮಾರುಗಳ ಎಷ್ಟೋ ಜನ ಬಂಗಾಳೀ ಪ್ರಮುಖರು ಶ್ರೀರಾಮಕೃಷ್ಣರಿಂದ ಬಹಳವಾಗಿ ಪ್ರಭಾವಿತರಾಗಿದ್ದಾರೆ. ಕೇಶವಸೇನರು ತಮ್ಮ ಕೊನೆಯ ವರ್ಷಗಳಲ್ಲಂತೂ ಪರಮಹಂಸರಿಂದ ಅಪಾರ ಸ್ಫೂರ್ತಿ ಪಡೆದುಕೊಂಡರಲ್ಲದೆ, ಅವರ ಧಾರ್ಮಿಕ ದೃಷ್ಟಿಕೋನದಲ್ಲೇ ಒಂದು ಅಪೂರ್ವ ಬದಲಾವಣೆ ಯುಂಟಾಯಿತು. ಅಷ್ಟೇ ಅಲ್ಲ; ಹಲವಾರು ಐರೋಪ್ಯ ಮಹನೀಯರೂ ಅವರ ಬಗ್ಗೆ ಕೇಳಿ, ಅವರನ್ನು ದರ್ಶಿಸಲು ಕಾತರರಾಗಿ ಬಂದರು. ಶ್ರೀರಾಮಕೃಷ್ಣರನ್ನು ಕಣ್ಣಾರೆ ಕಂಡ ಮೇಲೆ, ಅವರೊಬ್ಬ ದೈವಾಂಶಸಂಭೂತ ವ್ಯಕ್ತಿ ಎಂದು ತೀರ್ಮಾನಿಸಿದರು. ಬಂಗಾಳದ ಸಾರ್ವಜನಿಕ ಶಿಕ್ಷಣ ಇಲಾಖೆಯ ನಿರ್ದೇಶಕರ ಉನ್ನತ ಸ್ಥಾನವನ್ನು ಹೊಂದಿದ್ದ ಶ್ರೀ ಸಿ. ಹೆಚ್. ಟಾನಿಯವರು, ಆ ಮಹರ್ಷಿಯ ವ್ಯಕ್ತಿತ್ವ, ಅಸಾಧಾರಣ ಬುದ್ಧಿಮತ್ತೆ ಹಾಗೂ ಸಾರ್ವತ್ರಿಕ ಸ್ವೀಕಾರಾರ್ಹವಾದ ಅವರ ಸಂದೇಶಗಳನ್ನು ಎತ್ತಿಹಿಡಿಯುವ ಲೇಖನವನ್ನು ಬರೆದಿದ್ದಾರೆ.”

ಸುಂದರರಾಮ ಅಯ್ಯರರಿಗೆ ಈ ವಿಷಯಗಳು ಕೇಳರಿಯದಂಥವು. ಅದರಲ್ಲೂ ಈ ಅದ್ಭುತ ಸಂನ್ಯಾಸಿಯ ಬಾಯಿಂದ ಅವುಗಳನ್ನೆಲ್ಲ ಕೇಳುತ್ತಿದ್ದಂತೆ ಅವರು ಆಶ್ಚರ್ಯ ಚಕಿತರಾಗಿ ಕುಳಿತರು. ಸ್ವಾಮೀಜಿಯ ಕಂಠಮಾಧುರ್ಯ, ಕಣ್ಣುಗಳ ಹೊಳಪು, ಸುಲಲಿತ ವಾಕ್ಪ್ರವಾಹದಲ್ಲಿ ಹರಿದು ಬರುತ್ತಿದ್ದ ಹೊಸ ವಿಚಾರಧಾರೆ–ಇವೆಲ್ಲ ಅಯ್ಯರರನ್ನು ಅಕ್ಷರಶಃ ಬಂಧಿಸಿಬಿಟ್ಟುವು. ಅಯ್ಯರರು ತಿರುವಾಂಕೂರಿನ ರಾಜಕುಮಾರ, ಮಾರ್ತಾಂಡವರ್ಮನಿಗೆ ಎಂ. ಎ. ತರಗತಿಯ ಪಾಠಗಳನ್ನು ಬೋಧಿಸಲು ಅರಮನೆಗೆ ಹೋಗಬೇಕಿತ್ತು. ಆದರೆ ಸ್ವಾಮೀಜಿಯೊಂದಿಗೆ ಮಾತನಾಡುತ್ತ ಅವರಿಗೆ ಕುಳಿತಲ್ಲಿಂದ ಎದ್ದೇಳಲೂ ಮನಸ್ಸಾಗಲಿಲ್ಲ. ಆದ್ದರಿಂದ ತಮ್ಮ ಅಂದಿನ ಪಾಠದ ಕಾರ್ಯಕ್ರಮ ವನ್ನು ರದ್ದುಗೊಳಿಸಿದರು. ಸಂಜೆಯ ವೇಳೆಗೆ, ಈ ಅದ್ಭುತ ಯುವಸಂನ್ಯಾಸಿಯನ್ನು ಇತರರಿಗೂ ಪರಿಚಯಿಸಲು, ತಿರುವನಂತಪುರದ ಕ್ಲಬ್​ಗೆ ಕರೆದೊಯ್ದರು. ಇಲ್ಲಿ ಸ್ವಾಮೀಜಿ ಭೇಟಿಯಾದವ ರಲ್ಲಿ ಪ್ರೊ॥ ರಂಗಾಚಾರ್ಯ ಹಾಗೂ ಪ್ರೊ॥ ಸುಂದರಂ ಪಿಳ್ಳೆ ಎಂಬ ಸುಪ್ರಸಿದ್ಧ ಪ್ರಾಚಾರ್ಯರು, ನಾರಾಯಣ ಮೆನನ್ ಎಂಬ ಅಧಿಕಾರಿ, ಒಬ್ಬರು ನಿವೃತ್ತ ದಿವಾನರು ಮೊದಲಾದವರಿದ್ದರು.

ಆ ವೇಳೆಗೆ ಅಲ್ಲೊಂದು ಕುತೂಹಲಕರ ಘಟನೆ ನಡೆಯಿತು. ಸ್ವಲ್ಪ ಹೊತ್ತು ಮಾತುಕತೆ ಯಾಡಿದ ಬಳಿಕ ನಾರಾಯಣ ಮೆನನ್ನರು ಮನೆಗೆ ಹೊರಟರು; ಹೊರಡುವ ಮುನ್ನ ಅಲ್ಲಿದ್ದ ದಿವಾನರಿಗೆ ಕೈಮುಗಿದು ನಮಸ್ಕರಿಸಿದರು. ದಿವಾನರು ತಮ್ಮ ಎಡಗೈಯನ್ನು ಸ್ವಲ್ಪ ಮಾತ್ರ ಮೇಲೆತ್ತಿ ಪ್ರತಿನಮಸ್ಕರಿಸಿದರು. ಸ್ವಾಮೀಜಿಯ ಕಣ್ಣುಗಳು ಇದನ್ನು ಸೂಕ್ಷ್ಮವಾಗಿ ಗಮನಿಸಿದುವು. ವಿಷಯವೇನೆಂದು ಅವರಿಗೆ ಕ್ಷಣಮಾತ್ರದಲ್ಲಿ ಅರಿವಾಯಿತು. ದಿವಾನರು ಬ್ರಾಹ್ಮಣರು; ನಾರಾಯಣ ಮೆನನ್ನರು ಶೂದ್ರರು. ಕೇರಳದಲ್ಲಿ ಬ್ರಾಹ್ಮಣರು ಶೂದ್ರರಿಗೆ ಪ್ರತಿವಂದಿಸುವ ಕ್ರಮ ಹೀಗೆಯೇ–ಎಡಗೈಯನ್ನು ಬಲಗೈಗಿಂತ ಸ್ವಲ್ಪ ಮೇಲೆತ್ತುವುದು. ಸ್ವಾಮೀಜಿಗೆ ಸಿಟ್ಟು ಉಕ್ಕೇರಿತು. ಆದರೂ ಒಂದು ಮಾತನ್ನೂ ಆಡದೆ ಸುಮ್ಮನಿದ್ದರು.

ಸ್ವಲ್ಪ ಸಮಯದ ಮೇಲೆ ಸುಂದರರಾಮ ಅಯ್ಯರರೊಂದಿಗೆ ಸ್ವಾಮೀಜಿ ಅಲ್ಲಿಂದ ಹೊರ ಟಾಗ ದಿವಾನರು ಸ್ವಾಮೀಜಿಗೆ ಕೈಮುಗಿದು ನಮಸ್ಕರಿಸಿದರು. ಆಗ ಸ್ವಾಮೀಜಿ, “ನಾರಾಯಣ!” ಎಂದು ಹೇಳಿ ಸುಮ್ಮನಾದರು. ದಿವಾನರಿಗೆ ಅಸಮಾಧಾನವಾಯಿತು. ಸ್ವಾಮೀಜಿ ಪ್ರತಿನಮಸ್ಕರಿಸು ತ್ತಾರೆಂದು ಅವರು ನಿರೀಕ್ಷಿಸಿದ್ದರು. ಅದೂ ತಮ್ಮಂತಹ ಹಿರಿಯ, ಗೌರವಾನ್ವಿತ ವ್ಯಕ್ತಿಗೆ ಈ ಅಪರಿಚಿತ ಯುವಸಂನ್ಯಾಸಿ ಹೀಗೆ ಪ್ರತಿವಂದಿಸಿದುದು ಅವಮಾನಕರ ಎಂದು ಭಾವಿಸಿದರು. ತಮ್ಮ ಅಸಮಾಧಾನವನ್ನು ತಕ್ಷಣ ವ್ಯಕ್ತಪಡಿಸಿಯೂ ಬಿಟ್ಟರು. ಆಗ ಸ್ವಾಮೀಜಿ ಯಾವ ಮುಲಾಜೂ ಇಲ್ಲದೆ ಹೇಳಿದರು, “ಅಲ್ಲ, ನೀವು ಮಾತ್ರ ನಾರಾಯಣ ಮೆನನ್ನರಿಗೆ ನಿಮ್ಮ ಸಂಪ್ರದಾಯದಂತೆ ಪ್ರತಿವಂದಿಸಬಹುದಂತೆ! ನಾನು ಹಿಂದೂ ಸಂನ್ಯಾಸಿಗಳ ಸಂಪ್ರದಾಯದಂತೆ ‘ನಾರಾಯಣ’ ಎಂದರೆ ಅದಕ್ಕೇಕೆ ಮುನಿಸಿಕೊಳ್ಳುತ್ತೀರಿ...” ಬಾಣ ಗುರಿಮುಟ್ಟಿತು. ತುಟಿಪಿಟಕ್ಕೆನ್ನದೆ ದಿವಾನರು ಹೊರಟುಬಿಟ್ಟರು. ಮರುದಿನ ತಮ್ಮ ಸೋದರನನ್ನು ಸ್ವಾಮೀಜಿಯ ಬಳಿಗೆ ಕಳಿಸಿ ತಮ್ಮ ವರ್ತನೆಗಾಗಿ ಕ್ಷಮೆಯಾಚಿಸಿದರು. ಅಂತೂ ಸ್ವಾಮೀಜಿ ಕ್ಲಬ್ಬಿನಲ್ಲಿದ್ದದ್ದು ಸ್ವಲ್ಪಹೊತ್ತು ಮಾತ್ರವಾದರೂ, ಅಲ್ಲೊಂದು ಅಚ್ಚಳಿಯದ ಮುದ್ರೆಯನ್ನೊತ್ತಿ ಹೋಗಿದ್ದರು.

ಮರುದಿನ ತನ್ನ ಗುರುವಿನಿಂದ ಅಸಾಮಾನ್ಯ ಸಂನ್ಯಾಸಿಗಳೊಬ್ಬರ ಬಗ್ಗೆ ಕೇಳಿದ ರಾಜಕುಮಾರ ಮಾರ್ತಾಂಡವರ್ಮ, ಅವರನ್ನು ಭೇಟಿಮಾಡಲು ಇಚ್ಛಿಸಿದ. ಅದರಂತೆ, ಸುಂದರರಾಮ ಅಯ್ಯರರು ರಾಜಕುಮಾರನಿಗೆ ಸ್ವಾಮೀಜಿಯ ಭೇಟಿ ಮಾಡಿಸಿದರು. ತಮ್ಮ ಪರಿವ್ರಾಜಕ ಜೀವನ ದಲ್ಲಿ ತಾವು ಹಲವಾರು ಮಹಾರಾಜರನ್ನೂ, ದಿವಾನರೇ ಮೊದಲಾದ ಉನ್ನತ ಅಧಿಕಾರಿಗಳನ್ನೂ ಭೇಟಿಯಾಗಿರುವುದಾಗಿ ಸ್ವಾಮೀಜಿ ಸಾಂದರ್ಭಿಕವಾಗಿ ತಿಳಿಸಿದರು. ತುಂಬ ಕುತೂಹಲಗೊಂಡ ರಾಜಕುಮಾರ, ಭಾರತದ ವಿವಿಧ ರಾಜರ ಬಗ್ಗೆ ಅವರ ಅಭಿಪ್ರಾಯವನ್ನು ಕೇಳಿ ತಿಳಿದುಕೊಂಡ. ಬಳಿಕ ಅವರು, ಮಾರ್ತಾಂಡವರ್ಮನ ಅಧ್ಯಯನಾದಿಗಳ ಬಗೆಗೂ ಆತನ ಜೀವನೋದ್ದೇಶಗಳ ಬಗೆಗೂ ಅತ್ಯಂತ ವಿಶ್ವಾಸದಿಂದ ವಿಚಾರಿಸಿಕೊಂಡರು. ತಿರುವಾಂಕೂರಿನ ಪ್ರಜೆಗಳ ಏಳ್ಗೆಯ ಬಗ್ಗೆ ತನಗೆ ಪ್ರಮಾಣಿಕವಾದ ಆಸಕ್ತಿಯಿರುವುದಾಗಿಯೂ ರಾಜಮನೆತನದ ಒಬ್ಬ ಸದಸ್ಯನಾಗಿ ಆ ಬಗ್ಗೆ ತನ್ನಿಂದಾದುದನ್ನೆಲ್ಲ ಮಾಡಲು ನಿಶ್ಚಯಿಸಿರುವುದಾಗಿಯೂ ರಾಜಕುಮಾರ ಹೇಳಿದ. ತಿರುವಾಂಕೂರಿನ ಮಹಾರಾಜರಾದ ತನ್ನ ದೊಡ್ಡಪ್ಪನನ್ನು ಭೇಟಿ ಮಾಡುವಂತೆಯೂ ಆತ ಸೂಚಿಸಿದ. ಅದರಂತೆ ಮಹಾರಾಜನನ್ನೂ ಸ್ವಾಮೀಜಿ ಭೇಟಿಯಾದರು. ಸ್ವಾಮೀಜಿಯ ಪ್ರಬಲ ವ್ಯಕ್ತಿತ್ವದ ಮೋಡಿಗೊಳಗಾದ ರಾಜಕುಮಾರ, ಅವರ ನೆನಪಿಗಾಗಿ ಸುಂದರ ಛಾಯಾಚಿತ್ರ ವೊಂದನ್ನು ತೆಗೆದುಕೊಂಡ. ಮಾರ್ತಾಂಡವರ್ಮನನ್ನು ತುಂಬ ಮೆಚ್ಚಿಕೊಂಡಿದ್ದ ಸ್ವಾಮೀಜಿ, ಅವನ ಗುರುಗಳಾದ ಸುಂದರರಾಮ ಅಯ್ಯರರ ಬಳಿ ಹೇಳುತ್ತಾರೆ, “ಈತನಲ್ಲಿ ತುಂಬ ಸಾಧ್ಯತೆ ಗಳಡಗಿವೆ. ಆದರೆ ವಿಶ್ವವಿದ್ಯಾಲಯದ ಶಿಕ್ಷಣದಿಂದಾಗಿ ಅವನ ನಿಜಪ್ರತಿಭೆ ಮಸುಕಾಗದಿದ್ದರೆ ಅದೃಷ್ಟ. ಆದರೂ, ಅವನನ್ನು ತನ್ನಷ್ಟಕ್ಕೆ ಬೆಳೆಯಲು ಬಿಟ್ಟುಬಿಡುವುದೇ ಒಳ್ಳೆಯದು.”

ಸುಂದರರಾಮ ಅಯ್ಯರರು ಸ್ವಾಮೀಜಿಯೊಂದಿಗೆ ಹಲವಾರು ವಿಷಯಗಳ ಬಗ್ಗೆ ಚರ್ಚಿಸುತ್ತ ದಿನದ ಬಹುಭಾಗವನ್ನು ಕಳೆಯುತ್ತಿದ್ದರು. ತೀರಾ ಸಂಪ್ರದಾಯಸ್ಥರಾದ ಅಯ್ಯರರೊಂದಿಗೆ ಮಾತನಾಡುವಾಗ ಅವರ ಮನೋಧರ್ಮಕ್ಕನುಗುಣವಾದ ವಿಷಯಗಳ ಬಗ್ಗೆಯೇ ಹೆಚ್ಚಾಗಿ ಮಾತನಾಡುತ್ತಿದ್ದರು ಸ್ವಾಮೀಜಿ. ಆದರೆ ಕೇವಲ ದೇಶಾಚಾರಗಳನ್ನು ಪಾಲಿಸುವುದರಲ್ಲೇ ಮಗ್ನ ರಾಗಿರುವುದನ್ನು ಅವರು ತೀಕ್ಷ ್ಣವಾಗಿ ಖಂಡಿಸುತ್ತಿದ್ದರು. ಒಂದು ದಿನ, ಮಾನವಜೀವನದ ಮೇಲೆ ವಿಜ್ಞಾನದ ಪ್ರಭುತ್ವವನ್ನು ಉತ್ಪ್ರೇಕ್ಷಣೀಯವಾಗಿ ಬಣ್ಣೀಸುವ ಮನೋಭಾವವನ್ನು ತುಚ್ಛೀಕರಿ ಸುತ್ತ ಸ್ವಾಮೀಜಿ ಹೇಳುತ್ತಾರೆ, “ಧರ್ಮದಲ್ಲಿ ಮೂಢನಂಬಿಕೆಗಳಿರುವುದಾದರೆ, ವಿಜ್ಞಾನದಲ್ಲೂ ಅದರದ್ದೇ ಆದ ಮೂಢನಂಬಿಕೆಗಳಿವೆ. ‘ಯಾಂತ್ರೀಯ ವಾದ’ (ಜಗತ್ತಿನ ಆಗುಹೋಗುಗಳೆಲ್ಲವೂ ಯಾಂತ್ರಿಕವಾಗಿ ನಡೆಯುತ್ತವೆ ಎಂಬ ವಾದ) ಹಾಗೂ ‘ವಿಕಾಸವಾದ’ಗಳು ಅಸಮರ್ಪಕವಾದು ವೆಂದು ಸಾಬೀತಾಗಿದ್ದರೂ, ಕೆಲವರು ಈ ಜಗತ್ತಿನಲ್ಲಿ ಇನ್ನು ಭೇದಿಸಬೇಕಾದ ರಹಸ್ಯವೇ ಇಲ್ಲ ಎಂಬಂತೆ ಮಾತನಾಡುತ್ತಾರೆ. ‘ಅಜ್ಞೇಯತಾವಾದ\eng{’ (Agnosticism)}ವನ್ನು ಈಚೆಗೆ ಕೆಲವರು ದೊಡ್ಡದಾಗಿ ಮಾಡಿದ್ದರೂ, ಭಾರತೀಯ ರಾಜಯೋಗದ ನಿಯಮಗಳನ್ನು ಕಡೆಗಣಿಸಿ ಅದು ತನ್ನ ಮೊಂಡುತನವನ್ನು ಪ್ರದರ್ಶಿಸಿಕೊಂಡಿದೆ. ಮಾನವನ ಅತೀಂದ್ರಿಯ ಪ್ರಜ್ಞೆಯ ಅಂಶವನ್ನು ಹಾಗೂ ಅದರ ನಿಯಮಗಳನ್ನು ವಿವರಿಸುವಲ್ಲಿ ಪಾಶ್ಚಾತ್ಯ ಮನೋವಿಜ್ಞಾನವು ಶೋಚನೀಯವಾಗಿ ವಿಫಲಗೊಂಡಿದೆ. ಎಲ್ಲಿ ಪಾಶ್ಚಾತ್ಯ ವಿಜ್ಞಾನವು ತನ್ನಿಂದ ಸಾಧ್ಯವಿಲ್ಲವೆಂದು ಕೈಚೆಲ್ಲಿ ಕುಳಿತುಕೊಳ್ಳು ತ್ತದೆಯೋ ಅಲ್ಲಿ ಭಾರತೀಯ ಮನಶ್ಶಾಸ್ತ್ರ ನೆರವಿಗೆ ಬರುತ್ತದೆ. ಧರ್ಮ ಮಾತ್ರವೇ–ಅದರಲ್ಲೂ ಭಾರತದ ಮಹರ್ಷಿಗಳು ಬೋಧಿಸಿ ತೋರಿಸಿಕೊಟ್ಟ ಸನಾತನ ಧರ್ಮ ಮಾತ್ರವೇ–ಮಾನವ ಮನಸ್ಸಿನ ಸೂಕ್ಷ್ಮಗಳನ್ನು, ರಹಸ್ಯಗಳನ್ನು ಅರಿತುಕೊಂಡು, ಇಂದ್ರಿಯ ತೃಷ್ಣೆಗಳನ್ನು ಜಯಿಸು ವಲ್ಲಿ ನೆರವಾಗಬಲ್ಲುದು. ತನ್ಮೂಲಕ ಪರಮ ಸತ್ಯವನ್ನು ಸಾಕ್ಷಾತ್ಕರಿಸಿಕೊಳ್ಳುವಲ್ಲಿ, ಹಾಗೂ ಈ ವ್ಯಕ್ತಜಗತ್ತಿನಲ್ಲಿ ಕಂಡುಬರುವುದೆಲ್ಲವೂ ಆ ಸತ್ಯದ ಸಾಂತರೂಪ ಮಾತ್ರವೇ ಎಂದು ಅರಿಯು ವಲ್ಲಿ ಮಾನವನಿಗೆ ಸಹಾಯಕಾರಿಯಾಗಬಲ್ಲದು.”

ಲೌಕಿಕ ಹಾಗೂ ಅಲೌಕಿಕ ವಸ್ತುಗಳ ನಡುವಣ ಭೇದವನ್ನು ವಿವರಿಸುತ್ತ, ಹೇಗೆ ಅವರೆಡೂ ಮನುಷ್ಯನನ್ನು ಇಂದ್ರಿಯಗಳ ಬಂಧನದಲ್ಲಿಟ್ಟಿರುತ್ತವೆ ಎಂಬುದನ್ನು ತೋರಿಸಿಕೊಟ್ಟರು ಸ್ವಾಮೀಜಿ. ಬಳಿಕ ಅವರು ಹೇಳುತ್ತಾರೆ, “ಯಾವನು ಈ ಎರಡನ್ನೂ ಅತಿಕ್ರಮಿಸಿ ಪ್ರಪಂಚದ ಕ್ಷುದ್ರ ಥಳುಕು-ಭಿನ್ನಾಣಗಳು ಎಲ್ಲೆಯನ್ನು ಮೀರಿ ಹೋಗುತ್ತಾನೆಯೋ ಅವನು ಮಾತ್ರ ಜೀವನದ ಪರಮಗುರಿಯಾದ ಮುಕ್ತಿಯನ್ನು ಪಡೆಯಬಲ್ಲ.” ವರ್ಣಾಶ್ರಮಧರ್ಮಗಳ ಕುರಿತಾಗಿ ಮಾತ ನಾಡುತ್ತ ಹೇಳುತ್ತಾರೆ, “ಎಲ್ಲಿಯವರೆಗೆ ಬ್ರಾಹ್ಮಣನಾದವನು ನಿಸ್ವಾರ್ಥಸೇವೆ ಮಾಡಿಕೊಂಡಿ ರುತ್ತ, ತಾನು ಪಡೆದುಕೊಂಡಿರುವ ಜ್ಞಾನವನ್ನು ಮುಕ್ತ ಮನಸ್ಸಿನಿಂದ ಎಲ್ಲರಿಗೂ ವಿತರಣೆಗೈಯು ತ್ತಿರುತ್ತಾನೆಯೋ ಅಲ್ಲಿಯವರೆಗೆ ಮಾತ್ರ ಸಾರ್ಥಕ ಬದುಕನ್ನು ನಡೆಸುತ್ತಿದ್ದಾನೆ ಎಂದರ್ಥ. ಬ್ರಾಹ್ಮಣರು ಈ ಹಿಂದೆ ಭಾರತಕ್ಕಾಗಿ ಬಹಳಷ್ಟನ್ನು ಮಾಡಿದ್ದಾರೆ. ಮತ್ತು ಇನ್ನು ಮುಂದೆ ಅದಕ್ಕಿಂತಲೂ ಹೆಚ್ಚಿನದನ್ನು ಮಾಡುವವರಿದ್ದಾರೆ.”

ಭಾರತೀಯ ಮಹಿಳೆಯ ಉತ್ಥಾನದ ಬಗ್ಗೆ ಮಾತನಾಡುತ್ತ, ಅತ್ಯಂತ ವಿಶಿಷ್ಟವಾದ ತಮ್ಮ ಅಭಿಪ್ರಾಯವನ್ನು ಮುಂದಿಡುತ್ತಾರೆ: “ಮಹಿಳೆಯರ ಸಾಮಾಜಿಕ ಸ್ಥಾನಮಾನದ ವಿಷಯದಲ್ಲೇ ಆಗಲಿ, ಅವರ ವೈವಾಹಿಕ ನೀತಿ-ನಿಯಮಗಳ ವಿಷಯದಲ್ಲೇ ಆಗಲಿ, ಯಾವುದೇ ಬಗೆಯ ಶಾಸ್ತ್ರವಾಕ್ಯಗಳ ನೆಪದಲ್ಲಿ ಮಧ್ಯೆ ಪ್ರವೇಶಿಸಿ ಶೋಷಿಸುವುದನ್ನು ನಾನು ವಿರೋಧಿಸುತ್ತೇನೆ. ಹಿಂದುಳಿದ ವರ್ಗಗಳ ಹಾಗೂ ಜಾತಿಗಳ ಜನರಿಗೆ ಹೇಗೋ ಹಾಗೆಯೇ ಮಹಿಳೆಯರಿಗೂ ಸಂಸ್ಕೃತ ವಿದ್ಯಾಭ್ಯಾಸ ದೊರಕಬೇಕು. ತನ್ಮೂಲಕ ಅವರು ಸನಾತನ ಆಧ್ಯಾತ್ಮಿಕ ಸಂಸ್ಕೃತಿಯನ್ನು ಮೈಗೂಡಿಸಿಕೊಂಡು, ಆ ಆಧ್ಯಾತ್ಮಿಕ ಆದರ್ಶಗಳನ್ನು ಅನುಷ್ಠಾನಕ್ಕೆ ತರಲು ಪ್ರಯತ್ನಿಸಬೇಕು. ಹೀಗೆ ತಾವು ಪಡೆದುಕೊಂಡ ಜ್ಞಾನದ ಬೆಳಕಿನಿಂದ ತಮ್ಮ ಕೊರತೆಗಳನ್ನೂ ಆವಶ್ಯಕತೆಗಳನ್ನೂ ಅರ್ಥ ಮಾಡಿಕೊಂಡು, ತಮ್ಮ ಸ್ಥಾನಮಾನದ ಸಮಸ್ಯೆಯನ್ನು ತಾವೇ ಪರಿಹರಿಸಿಕೊಳ್ಳಲು ಭಾರತೀಯ ಮಹಿಳೆಯರು ಸಮರ್ಥರಾಗುತ್ತಾರೆ.”

ಒಮ್ಮೆ ಸ್ವಾಮೀಜಿ ಇಂದ್ರಿಯನಿಗ್ರಹದ ವಿಷಯದ ಕುರಿತು ಸ್ಫೂರ್ತಿಯುತವಾಗಿ ಮಾತ ನಾಡುತ್ತ ಸಂತ ಬಿಲ್ವಮಂಗಲನ ಕಥೆಯನ್ನು ಉದಾಹರಿಸಿ ಹೇಳುತ್ತಾರೆ: “ಬಿಲ್ವಮಂಗಲ ಒಬ್ಬ ಮಹಾಭಕ್ತ. ಒಮ್ಮೆ ಈತ ಶ್ರೀಮಂತನ ಮಗಳೊಬ್ಬಳನ್ನು ಪಾಪದೃಷ್ಟಿಯಿಂದ ನೋಡಿದ. ಅನಂತರ ತನ್ನ ಆ ಪಾಪಕೃತ್ಯದಿಂದಾಗಿ ಅವನು ಎಷ್ಟರಮಟ್ಟಿಗೆ ಪಶ್ಚಾತ್ತಾಪದಿಂದ ಬೆಂದನೆಂದರೆ ಪ್ರಾಯಶ್ಚಿತ್ತಾರ್ಥವಾಗಿ ತನ್ನ ಕಣ್ಣುಗಳನ್ನೇ ಕಿತ್ತುಕೊಂಡುಬಿಟ್ಟ. ಬಳಿಕ ತನ್ನ ಉಳಿದ ಜೀವಮಾನ ವನ್ನು, ಬೃಂದಾವನದಲ್ಲಿ ವಾಸಿಸುತ್ತ ಭಗವಂತನ ಸ್ಮರಣೆಯಲ್ಲಿ ಕಳೆದ.” ಹೀಗೆ ಈ ಕಥೆಯನ್ನು ಹೇಳಿ ಸ್ವಾಮೀಜಿ ಮತ್ತೆ ಹೇಳುತ್ತಾರೆ, “ಹತೋಟಿಯಿಲ್ಲದೆ ಹರಿದಾಡುವ ಇಂದ್ರಿಯಗಳನ್ನು ಹಿಡಿತದಲ್ಲಿಟ್ಟುಕೊಂಡು ಮನಸ್ಸನ್ನು ಭಗವಂತನ ಕಡೆಗೆ ತಿರುಗಿಸಲು, ಅಗತ್ಯ ಬಂದರೆ, ಇಂತಹ ಅತಿರೇಕದ ಉಪಾಯವೂ ಮೊದಲ ಹೆಜ್ಜೆಯಷ್ಟೇ.”

ಸ್ವಾಮೀಜಿ ಸುಂದರರಾಮ ಅಯ್ಯರರ ಮನೆಗೆ ಬಂದು ಮೂರ್ನಾಲ್ಕು ದಿನಗಳಾದ ಮೇಲೆ ಅವರಿಗೆ ಮನ್ಮಥನಾಥ ಭಟ್ಟಾಚಾರ್ಯ ಎಂಬೊಬ್ಬ ಬಂಗಾಳೀ ವ್ಯಕ್ತಿ ತಿರುವನಂತಪುರದಲ್ಲಿ ಇರುವುದು ತಿಳಿದುಬಂತು. ಇವರು ಮದ್ರಾಸಿನಲ್ಲಿ ಉನ್ನತ ಸರ್ಕಾರೀ ಹುದ್ದೆಯಲ್ಲಿದ್ದು, ಈಗ ಕಾರ್ಯನಿಮಿತ್ತ ತಿರುವನಂತಪುರಕ್ಕೆ ಬಂದಿದ್ದರು. ಸ್ವಾಮೀಜಿಗೆ ಹಿಂದಿನಿಂದಲೇ ಇವರ ಪರಿಚಯ ವಿತ್ತು. ಸುಂದರರಾಮ ಅಯ್ಯರರ ನೆರವಿನಿಂದ ಸ್ವಾಮೀಜಿ ಅವರನ್ನು ಸಂಧಿಸಿದರು. ಮತ್ತು ಅಂದಿನಿಂದ ಬೆಳಗಿನ ಸಮಯವನ್ನು ಅವರೊಂದಿಗೆ ಕಳೆಯುತ್ತ ಅವರ ಮನೆಯಲ್ಲೇ ಭೋಜನ ವನ್ನು ಸ್ವೀಕರಿಸಲಾರಂಭಿಸಿದರು. ಇದನ್ನು ಕಂಡ ಸುಂದರರಾಮ ಅಯ್ಯರರು ದೂರುವ ದನಿ ಯಲ್ಲಿ ಕೇಳಿದರು: “ಏನು ಸ್ವಾಮೀಜಿ, ನೀವು ಈಗೀಗ ಯಾವಾಗಲೂ ಭಟ್ಟಾಚಾರ್ಯರ ಮನೆಯಲ್ಲೇ ಇರುತ್ತೀರಲ್ಲ!”

ಸ್ವಾಮೀಜಿ ತಮಾಷೆಯ ದನಿಯಲ್ಲಿ ಉತ್ತರಿಸಿದರು:

“ನಾವು ಬಂಗಾಳಿಗಳು ಎಷ್ಟಾದರೂ ಒಂದು, ಅಲ್ಲವೆ?... ಅಲ್ಲದೆ ಅವರು ನನಗೆ ಮೊದಲಿನಿಂದಲೂ ಪರಿಚಿತರು. ಜೊತೆಗೆ, ಅವರು ಕಲ್ಕತ್ತದ ಸಂಸ್ಕೃತ ಕಾಲೇಜಿನ ಪ್ರಿನ್ಸಿಪಾಲರಾ ಗಿದ್ದ ಪಂಡಿತ ಮಹೇಶಚಂದ್ರ ನ್ಯಾಯರತ್ನರ ಮಗ. ಇದರ ಜೊತೆಗೆ ಮತ್ತೂ ಒಂದು ಕಾರಣವಿದೆ –ಬಹಳ ದಿನಗಳಾದ ಮೇಲೆ ಈಗ ನನಗೆ ನಮ್ಮೂರಿನ ಅಡಿಗೆಯೂಟ ಸಿಗುವಂತಾಗಿದೆ....!”

ಒಂದು ದಿನ ಅಯ್ಯರರು ಕೇಳಿಕೊಂಡರು:

“ಸ್ವಾಮೀಜಿ, ನೀವೊಂದು ಸಾರ್ವಜನಿಕ ಉಪನ್ಯಾಸ ಮಾಡಬೇಕು.”

“ನಾನು ಈ ಹಿಂದೆಂದೂ ಸಾರ್ವಜನಿಕ ಉಪನ್ಯಾಸ ಮಾಡಿದವನಲ್ಲ. ಈಗ ನಾನು ಇದ್ದಕ್ಕಿ ದ್ದಂತೆ ಉಪನ್ಯಾಸ ಮಾಡಲು ಹೊರಟರೆ ಸುಮ್ಮನೆ ನಗೆಪಾಟಲಿಗೀಡಾಗಿ ಪಶ್ಚಾತ್ತಾಪ ಪಡಬೇಕಾಗಿ ಬರುತ್ತದೆ, ಅಷ್ಟೆ.”

“ಆದರೆ ಸ್ವಾಮೀಜಿ, ಮೈಸೂರು ಮಹಾರಾಜರು ನಿಮ್ಮನ್ನು ಹಿಂದೂಧರ್ಮದ ಪ್ರತಿನಿಧಿಯಾಗಿ ಶಿಕಾಗೋದ ವಿಶ್ವಧರ್ಮ ಸಮ್ಮೇಳನದಲ್ಲಿ ಭಾಗವಹಿಸುವಂತೆ ಕೇಳಿಕೊಂಡಿರುವರು ಎಂದಿರಲ್ಲ! ಈಗ ನೀವು ಹೇಳುತ್ತಿರುವ ಮಾತು ಸತ್ಯವಾದರೆ, ಮುಂದೆ ಅಂತಹ ದೊಡ್ಡ ಸಭೆಯನ್ನು ಎದುರಿಸಿ ಹೇಗೆ ಮಾತನಾಡಬಲ್ಲಿರಿ?”

“ಭಗವಂತ ಇಚ್ಛೆಪಟ್ಟರೆ ಅವನು ನನ್ನ ಮೂಲಕ ಮಾತನಾಡುತ್ತಾನೆ. ನನ್ನ ಮೂಲಕ ತನ್ನ ಕಾರ್ಯವನ್ನು ಮಾಡಿಸಿಕೊಳ್ಳಬೇಕು ಎಂಬುದು ಅವನ ಇಚ್ಛೆಯಾದರೆ ಅದಕ್ಕೆ ಬೇಕಾದ ಶಕ್ತಿ ಯನ್ನು ಅವನು ಖಂಡಿತ ನನಗೆ ಕೊಡುತ್ತಾನೆ.”

ಈ ಮಾತನ್ನು ಕೇಳಿದಾಗ ಅಯ್ಯರರಿಗೆ ಇದು ನುಣುಚಿಕೊಳ್ಳುವ ಉಪಾಯದಂತೆ ಕಂಡಿತು. “ಆದರೆ ನನಗೇಕೋ ನೀವು ಹೇಳುವ ಮಾತಿನಲ್ಲಿ ಸ್ವಲ್ಪ ಅನುಮಾನ. ಇಂತಹ ವಿಷಯಗಳಲ್ಲೆಲ್ಲ ಭಗವಂತ ತಲೆಹಾಕುವ ಸಂಭವವಿದೆ ಅಂತ ನನಗಂತೂ ಅನ್ನಿಸುವುದಿಲ್ಲ.”

ಈಗ ಸ್ವಾಮೀಜಿ ಬರಸಿಡಿಲಿನಂತೆರಗಿದರು:

“ನೀವು ನಿಮ್ಮ ದಿನನಿತ್ಯದ ಆಚರಣೆಯಲ್ಲಿ ಮಾತುಕತೆಗಳಲ್ಲಿ ಒಬ್ಬ ಸಂಪ್ರದಾಯಸ್ಥ ಹಿಂದೂ ಎಂದು ತೋರಿಸಿಕೊಳ್ಳುತ್ತಿದ್ದರೂ ಒಳಗೆ ನೀವೊಬ್ಬ ನಾಸ್ತಿಕರೇ ಸರಿ. ನೀವು ನಿಮ್ಮ ಅಲ್ಪ ಬುದ್ಧಿಯ ಆಧಾರದ ಮೇಲೆ ಭಗವಂತನ ಅಮಿತ ಶಕ್ತಿಯನ್ನು ಸೀಮಿತಗೊಳಿಸುತ್ತಿರುವಿರಲ್ಲ! ಜಗತ್ತಿನ ಹಿತಸಾಧನೆಗಾಗಿ ಆತ ಏನು ಮಾಡಬಲ್ಲ ಎಂಬುದರ ಬಗ್ಗೆ ನಿಮಗೇನು ತಿಳಿದಿದೆ?”

ಅಯ್ಯರರಿಗೆ ಪುನಃ ಮಾತೆತ್ತಲು ಶಬ್ದಗಳು ಸಿಗಲಿಲ್ಲ.

ಇನ್ನೊಮ್ಮೆ ಸ್ವಾಮೀಜಿ ಸಂಭಾಷಣೆಯ ಸಂದರ್ಭದಲ್ಲಿ ಹೇಳುತ್ತಾರೆ:

“ಸಕ್ರಿಯ ದೇಶಭಕ್ತಿಯೆಂದರೆ ಕೇವಲ ಭಾವುಕತೆಯಲ್ಲ. ಅಥವಾ ತಾಯ್ನಾಡಿನ ಮೇಲಿನ ಬರಿಯ ಪ್ರೀತಿಭಾವವೂ ಅಲ್ಲ. ಅದು ತನ್ನ ದೇಶಬಾಂಧವರ ಸೇವೆ ಮಾಡಬೇಕೆಂಬ ಉಜ್ವಲ ಉತ್ಸಾಹ. ನಾನು ದೇಶದಾದ್ಯಂತ ಕಾಲ್ನಡಿಗೆಯಲ್ಲಿ ಸಂಚರಿಸಿದ್ದೇನೆ; ನಮ್ಮ ಜನರ ಅಜ್ಞಾನ- ದಾರಿದ್ರ್ಯ-ಸಂಕಟಗಳನ್ನು ಕಣ್ಣಾರೆ ಕಂಡಿದ್ದೇನೆ. ಈಗ ನನ್ನ ಸಮಸ್ತ ಚೇತನವೇ ಹೊತ್ತಿ ಉರಿಯ ತೊಡಗಿದೆ. ಈ ಹೀನ ಪರಿಸ್ಥಿತಿಯಿಂದ ಭಾರತವನ್ನು ಮೇಲೆತ್ತುವ ಪ್ರಚಂಡ ಬಯಕೆ ನನ್ನನ್ನು ದಹಿಸುತ್ತಿದೆ... ಕರ್ಮ-ಗಿರ್ಮ ಎಂದು ಯಾರೂ ಮಾತನಾಡದಿರಲಿ. ಸಂಕಟಪಡುವುದು ಅವರ ವರ ಕರ್ಮವಾದರೆ ಅವರನ್ನು ಆ ಸಂಕಟದಿಂದ ಬಿಡಿಸುವುದು ನಮ್ಮ ಕರ್ಮ. ನೀನು ದೇವರನ್ನು ಕಾಣಬೇಕಾದರೆ ಮಾನವನ ಸೇವೆ ಮಾಡು. ನೀನು ನಾರಾಯಣನನ್ನು ಪಡೆಯಬೇಕಾದರೆ ಹಸಿದ ಹೊಟ್ಟೆಯಿಂದ ನರಳುತ್ತಿರುವ ಭಾರತದ ಲಕ್ಷಾಂತರ ದರಿದ್ರನಾರಾಯಣರ ಸೇವೆ ಮಾಡಬೇಕು.”

ಅಯ್ಯರರ ಮನೆಯಲ್ಲಿದ್ದಷ್ಟು ದಿನವೂ ಸ್ವಾಮೀಜಿ ತಮ್ಮ ಸೌಜನ್ಯ, ಗಾಂಭೀರ್ಯ ಹಾಗೂ ಸ್ನೇಹಪೂರ್ಣ ನಡವಳಿಕೆಯಿಂದ ಮನೆಯವರೆಲ್ಲರ ಮನಸೂರೆಗೊಂಡಿದ್ದರು. ಮನೆಯ ಮಕ್ಕಳಂತೂ ಸದಾ ಅವರೊಂದಿಗೇ ಇದ್ದುಬಿಟ್ಟಿದ್ದರು. ಹಿರಿಯ ಮಗ ರಾಮಸ್ವಾಮಿಗೆ ಆಗಿನ್ನೂ ೧೪ ವರ್ಷ ವಯಸ್ಸು. ಆದರೆ ಆತನ ಎಳೆಯ ಮನಸ್ಸಿನ ಮೇಲೆ ಸ್ವಾಮೀಜಿಯ ವ್ಯಕ್ತಿತ್ವ ಎಂತಹ ಪ್ರಭಾವವನ್ನು ಬೀರಿತ್ತೆಂದರೆ, ಅವನ ಜೀವನ ಪರ್ಯಂತ ಅವರ ನೆನಪುಗಳು ಹಸಿರಾಗಿದ್ದುವು. ಹಲವಾರು ದಶಕಗಳೇ ಕಳೆದ ಮೇಲೆ ರಾಮಸ್ವಾಮಿ ಅಯ್ಯರ್, ಈ ಮಧುರ ಸ್ಮೃತಿಚಿತ್ರಣವನ್ನು ಬರೆದಿಟ್ಟರು. ಅಯ್ಯರರ ಮನೆಯಲ್ಲಿದ್ದಾಗ ಸ್ವಾಮೀಜಿ ಕೆಲವು ತಮಿಳುಪದಗಳನ್ನು ಕಲಿತು ಕೊಂಡು, ಮನೆಯ ಅಡಿಗೆಯವನೊಂದಿಗೆ ಸರಸ ಸಂಭಾಷಣೆ ನಡೆಸುತ್ತ, ತಾವೂ ನಕ್ಕು ಇತರರನ್ನೂ ನಗಿಸುತ್ತಿದ್ದರು. ಅವರ ಮನೆಯಲ್ಲಿದ್ದ ಒಂಬತ್ತು ದಿನಗಳಲ್ಲಿ ಸ್ವಾಮೀಜಿ ಮನೆ ಮಂದಿಯಂತೆಯೇ ಆಗಿಬಿಟ್ಟಿದ್ದರು.

ಡಿಸೆಂಬರ್ ೨೨ರಂದು ಸ್ವಾಮೀಜಿ ತಿರುವನಂತಪುರದಿಂದ ಕನ್ಯಾಕುಮಾರಿಯೆಡೆಗೆ ಹೊರಟು ನಿಂತರು. ಅಯ್ಯರರ ಮನೆಯ ಮೆಟ್ಟಿಲಿಳಿದು ಹೋಗುತ್ತಿರುವಾಗ ಅಲ್ಲಿಗೆ ಪಂಡಿತ ವಾಂಛೀಶ್ವರ ಶಾಸ್ತ್ರಿಗಳು ಎಂಬವರೊಬ್ಬರು ಅವಸರವಸರವಾಗಿ ಬಂದರು. ಇವರು ಸಂಸ್ಕೃತ ವ್ಯಾಕರಣದ ಅಸಾಮಾನ್ಯ ಜ್ಞಾನದಿಂದಲೂ ಸದಾಚಾರದಿಂದಲೂ ಆ ಪ್ರಾಂತದಲ್ಲೆಲ್ಲ ಹೆಸರುವಾಸಿಯಾಗಿದ್ದ ವರು. ಇವರು ಸ್ವಾಮೀಜಿಯ ಬಗ್ಗೆ ಕೇಳಿ ತಿಳಿದಿದ್ದರೂ ಈವರೆಗೆ ಅವರನ್ನು ಭೇಟಿಯಾಗಲು ಸಾಧ್ಯವಾಗಿರಲಿಲ್ಲ. ಈಗ ಕಡೇ ಘಳಿಗೆಯಲ್ಲಿ ಓಡಿಬಂದು, ಒಂದೈದು ನಿಮಿಷವಾದರೂ ಸ್ವಾಮಿ ಗಳನ್ನು ಭೇಟಿಮಾಡುವ ಅವಕಾಶ ಕಲ್ಪಿಸಿಕೊಡುವಂತೆ ಅಯ್ಯರರನ್ನು ದೈನ್ಯದಿಂದ ಪ್ರಾರ್ಥಿಸಿ ಕೊಂಡರು. ಶಾಸ್ತ್ರಿಗಳ ಕೋರಿಕೆಗೆ ಸ್ವಾಮೀಜಿ ಸಂತೋಷದಿಂದ ಒಪ್ಪಿ, ಅವರೊಂದಿಗೆ ಸಂಸ್ಕೃತ ದಲ್ಲಿ ಸಂಭಾಷಣೆಯನ್ನಾರಂಭಿಸಿದರು. ಸಂಸ್ಕೃತ ವ್ಯಾಕರಣಕ್ಕೆ ಸಂಬಂಧಿಸಿದ, ತುಂಬ ಕ್ಲಿಷ್ಟವೂ ವಿವಾದಾಸ್ಪದವೂ ಆದ ಅಂಶವೊಂದರ ಬಗ್ಗೆ ಏಳೆಂಟು ನಿಮಿಷಗಳ ಕಾಲ ಮಾತುಕತೆ ನಡೆಯಿತು. ಶಾಸ್ತ್ರಿಗಳು ತೃಪ್ತರಾಗಿ ತುಂಬ ಸಂತೋಷದಿಂದ ಸ್ವಾಮಿಗಳಿಗೆ ನಮಸ್ಕರಿಸಿ ಬೀಳ್ಗೊಂಡರು. ಬಳಿಕ ಶಾಸ್ತ್ರಿಗಳು, ಸಂಸ್ಕೃತ ವ್ಯಾಕರಣದ ಮೇಲೂ ಭಾಷೆಯ ಮೇಲೂ ಸ್ವಾಮಿಗಳಿಗಿರುವ ಸಂಪೂರ್ಣ ಪ್ರಭುತ್ವವನ್ನು ಮೆಚ್ಚಿ ಅಯ್ಯರರ ಮುಂದೆ ಮನಸಾರೆ ಕೊಂಡಾಡಿದರು.

ಸ್ವಾಮೀಜಿಯೊಂದಿಗೆ ಅಪೂರ್ವ ‘ನವರಾತ್ರಿ’ಗಳನ್ನು ಕಳೆದ ಅಯ್ಯರರ ಕುಟುಂಬ ವರ್ಗದ ವರಿಗೆ, ಅವರನ್ನು ಬೀಳ್ಕೊಂಡಾಗ ಮನೆಯ ದೀಪವೇ ಅವರೊಂದಿಗೆ ಹೊರಟುಹೋದಂತೆ ಭಾಸವಾಯಿತು.

