\sethyphenation{kannada}{
ಅ
ಅಂಕ-ಗಳನ್ನು
ಅಂಕ-ಗಳೇ
ಅಂಗ
ಅಂಗಕ್ಕೂ
ಅಂಗ-ಪ್ರ-ತ್ಯಂ-ಗ-ವನ್ನೂ
ಅಂಚಿ-ನಲ್ಲೇ
ಅಂಟಿ-ಕೊಂ-ಡಿ-ರು-ತ್ತದೆ
ಅಂತ
ಅಂತಃ
ಅಂತ-ರಂ-ಗ-ದ-ಲ್ಲಿ-ರುವ
ಅಂತ-ರಂ-ಗ-ದಲ್ಲೇ
ಅಂತ-ರಂ-ಗ-ದೊ-ಳ-ಗಿ-ರುವ
ಅಂತ-ರ-ವಿ-ರ-ದಿ-ದ್ದರೆ
ಅಂತ-ರ-ವಿ-ರು-ವಂತೆ
ಅಂತ-ರ್ಯಾ-ಮಿ-ಯನ್ನು
ಅಂತಹ
ಅಂತೂ
ಅಂಥ
ಅಂಥದು
ಅಂಥವರು
ಅಂಥವು-ಗಳನ್ನು
ಅಂದಂದು
ಅಂದ-ಗೆ-ಟ್ಟರೆ
ಅಂದ-ಮೇಲೆ
ಅಂದ-ರೇ-ನು-ದಿ-ನದ
ಅಂದ-ವಾ-ಗು-ತ್ತದೆ
ಅಂದು
ಅಂಶ
ಅಂಶ-ಗಳ
ಅಂಶ-ಗಳನ್ನು
ಅಂಶ-ಗಳನ್ನೂ
ಅಂಶ-ಗಳು
ಅಂಶ-ವನ್ನು
ಅಕ್ಕ-ಸಾ-ಲಿ-ಗ-ರಲ್ಲಿ
ಅಕ್ಕ-ಸಾ-ಲಿ-ಗಳ
ಅಕ್ಷರ
ಅಕ್ಷ-ರ-ಗಳು
ಅಕ್ಷ-ರದ
ಅಕ್ಷ-ರ-ದೋ-ಷ-ವ-ನ್ನಂತೂ
ಅಗತ್ಯ
ಅಗತ್ಯ-ವಾದ
ಅಗತ್ಯ-ವಿ-ರುವ
ಅಗ್ನಿಯೇ
ಅಗ್ರ
ಅಚ್ಚು-ಕ-ಟ್ಟಾ-ಗಿ-ರ-ಬೇಕು
ಅಚ್ಚು-ಕ-ಟ್ಟು-ತ-ನವು
ಅಡಗಿದೆ
ಅಡಗಿ-ರುವ
ಅಡುಗೆ
ಅಡ್ಡಾಡಿ
ಅಡ್ಡಿ
ಅಣಿ
ಅಣಿ-ಗೊ-ಳಿ-ಸಿ-ಬಿ-ಡು-ತ್ತಾನೆ
ಅಣೆ-ಕಟ್ಟು
ಅತಿ
ಅತಿ-ಥಿ-ಗಳೋ
ಅತಿ-ಯಾಗಿ
ಅತಿ-ಯಾದ
ಅತಿ-ವಿ-ಶ್ವಾ-ಸ-ದಿಂದ
ಅತ್ತ
ಅತ್ತಿತ್ತ
ಅತ್ಯ
ಅತ್ಯಂತ
ಅತ್ಯಲ್ಪ
ಅತ್ಯು-ತ್ತಮ
ಅಥವಾ
ಅದ
ಅದ-ಕ್ಕಾಗಿ
ಅದಕ್ಕೆ
ಅದಕ್ಕೇ
ಅದ-ಕ್ಕೊಂದು
ಅದನ್ನು
ಅದನ್ನೇ
ಅದಮ್ಯ
ಅದರ
ಅದ-ರದೇ
ಅದ-ರಲ್ಲಿ
ಅದ-ರ-ಲ್ಲಿ-ರುವ
ಅದ-ರಲ್ಲೂ
ಅದ-ರಲ್ಲೇ
ಅದ-ರಿಂದ
ಅದ-ರಿಂ-ದೇನೂ
ಅದಾ-ಗ-ಬೇಕು
ಅದಾ-ಗಲೇ
ಅದು
ಅದೃ-ಷ್ಟ-ವ-ಶಾತ್
ಅದೆಲ್ಲ
ಅದೆಷ್ಟು
ಅದೆ-ಷ್ಟೆಂ-ದರೆ
ಅದೇ
ಅದೇ-ನೆಂದು
ಅದ್ಭುತ
ಅಧಿಕ
ಅಧಿ-ಕ-ವಾ-ಗಿ-ರು-ವಲ್ಲಿ
ಅಧಿ-ಕ-ವೆಂದೇ
ಅಧಿ-ಕಾ-ರ-ಬ-ಲ-ವಿ-ರು-ವ-ವ-ರಿಗೆ
ಅಧಿ-ಕಾ-ರ-ಬ-ಲ-ವಿ-ಲ್ಲ-ವೆಂದು
ಅಧೀ-ನ-ದ-ಲ್ಲಿ-ರ-ಬೇ-ಕಾ-ಗು-ತ್ತದೆ
ಅಧ್ಯ
ಅಧ್ಯ-ಕ್ಷರು
ಅಧ್ಯ-ಯನ
ಅಧ್ಯ-ಯ-ನಕ್ಕೆ
ಅಧ್ಯ-ಯ-ನದ
ಅಧ್ಯ-ಯ-ನ-ದಲ್ಲಿ
ಅಧ್ಯ-ಯ-ನ-ದೊಂ-ದಿಗೆ
ಅಧ್ಯ-ಯ-ನ-ವನ್ನು
ಅಧ್ಯ-ಯ-ನ-ವ-ನ್ನೇನೋ
ಅಧ್ಯ-ಯ-ನ-ವಾ-ಗದೆ
ಅಧ್ಯ-ಯ-ನ-ವೆಲ್ಲ
ಅಧ್ಯ-ಯ-ನವೇ
ಅಧ್ಯಾ-ಪ-ಕರ
ಅಧ್ಯಾ-ಪ-ಕ-ರದೋ
ಅಧ್ಯಾ-ಪ-ಕ-ರಲ್ಲೇ
ಅಧ್ಯಾ-ಪ-ಕ-ರಿಗೂ
ಅಧ್ಯಾ-ಪ-ಕ-ರಿಗೆ
ಅಧ್ಯಾ-ಪ-ಕರು
ಅಧ್ಯಾ-ಪ-ಕರೂ
ಅಧ್ವಾ-ನ-ವಾ-ಗಿ-ಬಿ-ಡು-ತ್ತದೆ
ಅನಂ-ತ-ತೆ-ಯನ್ನು
ಅನ-ನು-ಕೂ-ಲತೆ
ಅನಾ-ವ-ಶ್ಯಕ
ಅನಾ-ಹು-ತಕ್ಕೆ
ಅನಿ-ರೀ-ಕ್ಷಿ-ತ-ವಾಗಿ
ಅನಿ-ವಾರ್ಯ
ಅನಿ-ವಾ-ರ್ಯ-ವಾ-ದ್ದ-ರಿಂದ
ಅನೀ-ತಿ-ಅ-ಶಿ-ಸ್ತಿನ
ಅನು
ಅನು-ಕೂಲ
ಅನು-ಕೂ-ಲಕ್ಕೆ
ಅನು-ಕೂ-ಲ-ಗ-ಳಿಲ್ಲ
ಅನು-ಕೂ-ಲತೆ
ಅನು-ಕೂ-ಲ-ತೆ-ಗ-ಳಿವೆ
ಅನು-ಕೂ-ಲ-ತೆ-ಗಳೂ
ಅನು-ಕೂ-ಲ-ತೆ-ಯಿ-ಲ್ಲ-ದ-ವರು
ಅನು-ಕೂ-ಲ-ವಾ-ದಂ-ತಹ
ಅನು-ಕೂ-ಲವೂ
ಅನು-ಕೂ-ಲಿ-ಸು-ವಂತೆ
ಅನು-ಗ್ರಹ
ಅನು-ಬಂಧ
ಅನು-ಭವ
ಅನು-ಭ-ವಿ-ಸಿದ
ಅನು-ಭ-ವಿ-ಸಿಯೇ
ಅನು-ಭ-ವಿ-ಸುತ್ತ
ಅನು-ರಾಗ
ಅನು-ಸ-ರಿ-ಸ-ಬ-ಹು-ದುಈ
ಅನು-ಸ-ರಿ-ಸಿ-ದ-ರಾ-ಯಿತು
ಅನು-ಸ-ರಿ-ಸಿ-ದಿ-ಯಾ-ದರೆ
ಅನು-ಸಾ-ರ-ವಾಗಿ
ಅನೂ-ಕೂ-ಲವೂ
ಅನೇಕ
ಅನ್ವ-ಯಿ-ಸು-ತ್ತದೆ
ಅಪ-ಘಾತ
ಅಪ-ರೂಪ
ಅಪ-ಹ-ರಿ-ಸಿ-ಬಿ-ಡು-ತ್ತವೆ
ಅಪಾ-ಯ-ಗಳ
ಅಪಾರ
ಅಪೂರ್ವ
ಅಪೇ-ಕ್ಷಿ-ಸು-ತ್ತಾರೆ
ಅಭಿ-ಪ್ರಾಯ
ಅಭಿ-ರು-ಚಿಯ
ಅಭಿ-ಲಾಷೆ
ಅಭಿ-ವ್ಯ-ಕ್ತ-ವಾಗಿ
ಅಭ್ಯ-ಸಿಸ
ಅಭ್ಯ-ಸಿ-ಸ-ಬ-ಹು-ದಲ್ಲ
ಅಭ್ಯಾಸ
ಅಭ್ಯಾ-ಸ
ಅಭ್ಯಾ-ಸದ
ಅಭ್ಯಾ-ಸ-ದಿಂದ
ಅಭ್ಯಾ-ಸ-ಬ-ಲವೇ
ಅಭ್ಯಾ-ಸ-ಮಾ-ಡಿ-ದರೆ
ಅಭ್ಯಾ-ಸ-ಮಾ-ಡು-ವ-ವ-ರಂತೂ
ಅಭ್ಯಾ-ಸ-ವನ್ನು
ಅಭ್ಯಾ-ಸ-ವಿಟ್ಟು
ಅಭ್ಯಾ-ಸ-ವಿ-ದ್ದರೆ
ಅಭ್ಯಾ-ಸ-ವೆಂದರೆ
ಅಭ್ಯಾ-ಸ-ವೆಂ-ದ-ರೇನು
ಅಭ್ಯಾ-ಸವೇ
ಅಮೂಲ್ಯ
ಅಯ್ಯೋ
ಅರ-ಚಾ-ಟ-ದಿಂದ
ಅರ-ಚಾ-ಡಲು
ಅರಿತ
ಅರಿ-ತಿ-ರ-ಬೇಕು
ಅರಿ-ತು-ಕೊ-ಳ್ಳು-ವುದೇ
ಅರಿ-ಯಲು
ಅರಿ-ಯು-ತ್ತದೆ
ಅರಿ-ವನ್ನು
ಅರಿ-ವಿ-ರಲಿ
ಅರಿವು
ಅರಿವೂ
ಅರ್ಜುನ
ಅರ್ಜು-ನನ
ಅರ್ಜು-ನ-ನಿಗೆ
ಅರ್ಜು-ನ-ನೆನ್ನು
ಅರ್ಥ
ಅರ್ಥ-ಮಾಡಿ
ಅರ್ಥ-ಮಾ-ಡಿ-ಕೊಂಡು
ಅರ್ಥ-ಮಾ-ಡಿ-ಕೊಂ-ಡೆ-ಯಾ-ದರೆ
ಅರ್ಥ-ಮಾ-ಡಿ-ಕೊಳ್ಳ
ಅರ್ಥ-ಮಾ-ಡಿ-ಕೊ-ಳ್ಳದೆ
ಅರ್ಥ-ಮಾ-ಡಿ-ಕೊ-ಳ್ಳಲು
ಅರ್ಥ-ಮಾ-ಡಿ-ಕೊ-ಳ್ಳು-ತ್ತೇನೆ
ಅರ್ಥ-ವನ್ನು
ಅರ್ಥ-ವಾಗ
ಅರ್ಥ-ವಾ-ಗ-ತೊ-ಡ-ಗು-ವ-ವ-ರೆಗೆ
ಅರ್ಥ-ವಾ-ಗದ
ಅರ್ಥ-ವಾ-ಗ-ದಿದ್ದ
ಅರ್ಥ-ವಾ-ಗ-ಬೇ-ಕಾ-ದರೆ
ಅರ್ಥ-ವಾ-ಗಿ-ರ-ಲಿ-ಲ್ಲವೋ
ಅರ್ಥ-ವಾ-ಗು-ತ್ತವೆ
ಅರ್ಥ-ವಾ-ಗು-ವು-ದ-ರಿಂದ
ಅರ್ಥ-ವಾ-ಗು-ವುದು
ಅರ್ಥ-ವಾ-ಗು-ವು-ದುಂಟೆ
ಅರ್ಥ-ವಾ-ದ-ದ್ದನ್ನು
ಅರ್ಥವೇ
ಅರ್ಥ-ವೇನು
ಅರ್ಧ
ಅರ್ಧಕ್ಕೆ
ಅರ್ಧ-ಗಂ-ಟೆಯೋ
ಅಲು-ಗದೆ
ಅಲು-ಗಾ-ಡದೆ
ಅಲು-ಗಾ-ಡು-ತ್ತಿ-ದ್ದರೆ
ಅಲು-ಗಾ-ಡು-ವಂತೆ
ಅಲೆ-ದಾ-ಟ-ಗ-ಳಲ್ಲೇ
ಅಲೆ-ದಾ-ಡು-ವಂ-ತಿಲ್ಲ
ಅಲ್ಪ
ಅಲ್ಪ-ವಿ-ಕ್ರ-ಮ-ನಾ-ಗು-ವಂತೆ
ಅಲ್ಪಾ-ವ-ಧಿ-ಯಲ್ಲೇ
ಅಲ್ಲ
ಅಲ್ಲ-ಗ-ಳೆ-ಯು-ವು-ದಿಲ್ಲ
ಅಲ್ಲದೆ
ಅಲ್ಲಲ್ಲಿ
ಅಲ್ಲವೆ
ಅಲ್ಲ-ವೇ-ನಯ್ಯ
ಅಲ್ಲಿ
ಅಲ್ಲಿಂದ
ಅಲ್ಲಿಯೇ
ಅಲ್ಲೇ
ಅಳ-ವ-ಡಿ-ಸಿ-ಕೊಂ-ಡರೆ
ಅಳ-ವ-ಡಿ-ಸಿ-ರುವ
ಅಳು-ಮೋರೆ
ಅವ
ಅವ-ಕಾಶ
ಅವ-ಕಾ-ಶ-ವನ್ನು
ಅವ-ಧಾನ
ಅವ-ಧಿಗೆ
ಅವ-ಧಿ-ಯಲ್ಲಿ
ಅವ-ಧಿ-ಯೊ-ಳಗೆ
ಅವನ
ಅವ-ನನ್ನು
ಅವ-ನಲ್ಲಿ
ಅವನಿ
ಅವ-ನಿಗೆ
ಅವ-ನಿಗೇ
ಅವನು
ಅವನ್ನು
ಅವ-ನ್ನೆಲ್ಲ
ಅವರ
ಅವ-ರನ್ನು
ಅವ-ರ-ವರ
ಅವ-ರ-ವರು
ಅವ-ರ-ವರೇ
ಅವರಿ
ಅವ-ರಿಂದ
ಅವ-ರಿಗೆ
ಅವರು
ಅವರೆಲ್ಲ
ಅವರೇ
ಅವ-ರೇನೂ
ಅವ-ರೊಂ-ದಿಗೆ
ಅವ-ಲ-ಕ್ಕಿ-ಯಂ-ತಿ-ರ-ಬಾ-ರದು
ಅವ-ಶ್ಯ-ವಾಗಿ
ಅವ-ಶ್ಯ-ವಿ-ರುವ
ಅವಿ-ರತ
ಅವಿ-ವೇ-ಕಕ್ಕೆ
ಅವಿ-ವೇ-ಕ-ದಿಂದ
ಅವು
ಅವು-ಗಳ
ಅವು-ಗಳನ್ನು
ಅವು-ಗಳನ್ನೆಲ್ಲ
ಅವು-ಗಳಲ್ಲಿ
ಅವು-ಗಳು
ಅವೆಲ್ಲ
ಅವೆ-ಲ್ಲ-ವನ್ನೂ
ಅಶಿ-ಕ್ಷಿತ
ಅಶ್ಲೀಲ
ಅಶ್ಲೀ-ಲತೆ
ಅಶ್ಲೀ-ಲ-ತೆ-ಗಳನ್ನು
ಅಷ್ಟಕ್ಕೇ
ಅಷ್ಟನ್ನು
ಅಷ್ಟಾಗಿ
ಅಷ್ಟಾ-ಗೇನೂ
ಅಷ್ಟು
ಅಷ್ಟೆ
ಅಷ್ಟೇ
ಅಸಂ-ಖ್ಯಾತ
ಅಸಭ್ಯ
ಅಸ-ಭ್ಯತೆ
ಅಸ-ಮ-ರ್ಥ-ರಾ-ಗಿ-ರು-ತ್ತಾರೆ
ಅಸ-ಮ-ರ್ಥ-ವಾ-ಗು-ತ್ತದೆ
ಅಸ-ಹಾ-ಯಕ
ಅಸ-ಹಾ-ಯ-ಕ-ರಾಗಿ
ಅಸಾಧ್ಯ
ಅಸ್ತ-ವ್ಯ-ಸ್ತ-ವಾ-ಗಿ-ರು-ವುದು
ಆ
ಆಕ-ರ-ಗ್ರಂಥ
ಆಕ-ರ್ಷ-ಣೆ-ಗಳನ್ನು
ಆಕ-ರ್ಷ-ಣೆ-ಗಳು
ಆಕಾ-ಶ-ಗೋ-ಪುರ
ಆಕ್ರ-ಮಣ
ಆಗ
ಆಗ-ದಂತೆ
ಆಗ-ಬ-ಹು-ದಾದ
ಆಗ-ಬೇಕು
ಆಗ-ಲಾ-ರರು
ಆಗಲಿ
ಆಗಲೇ
ಆಗಾಗ
ಆಗಾ-ಗಲೇ
ಆಗಿ-ಬಿ-ಡ-ಬ-ಹುದು
ಆಗಿ-ಬಿ-ಡು-ತ್ತದೆ
ಆಗಿ-ರ-ಬೇ-ಕಾ-ದರೆ
ಆಗಿ-ರ-ಬೇ-ಕೆಂದು
ಆಗು-ತ್ತದೆ
ಆಗು-ತ್ತಿಲ್ಲ
ಆಗು-ತ್ತಿ-ಲ್ಲ-ವಲ್ಲ
ಆಗು-ತ್ತೇವೆ
ಆಗು-ವು-ದಿಲ್ಲ
ಆಗು-ವು-ದಿಷ್ಟೆ
ಆಗೇನು
ಆಗೋ-ದಿಲ್ಲ
ಆಟ
ಆಟ-ಗ-ಳ-ನ್ನಾ-ಡು-ವ-ವರು
ಆಡಿ-ಕೊ-ಳ್ಳಲಿ
ಆಡ್ತಿದ್ದೀ
ಆತಂ-ಕ-ಗಳು
ಆತನ
ಆತ್ಮ-ಜ್ಯೋ-ತಿಯ
ಆತ್ಮ-ವಿ-ಶ್ವಾಸ
ಆತ್ಮ-ವಿ-ಶ್ವಾ-ಸ-ವನ್ನು
ಆತ್ಮ-ಸಾ-ಕ್ಷಾ-ತ್ಕಾ-ರ-ವನ್ನು
ಆದ
ಆದ-ರದು
ಆದರೂ
ಆದರೆ
ಆದ-ರ್ಶ-ದಿಂದ
ಆದ್ದ
ಆದ್ದ-ರಿಂದ
ಆದ್ದ-ರಿಂ-ದಲೇ
ಆಧಾ-ರದ
ಆಧಾ-ರ-ವಾಗಿ
ಆಧ್ಯಾ-ತ್ಮಿಕ
ಆನಂ-ದ-ಮ-ಯ-ವಾ-ಗು-ತ್ತದೆ
ಆನಂ-ದ-ವಿದೆ
ಆನಂ-ದ-ವುಂ-ಟಾ-ಗು-ತ್ತದೆ
ಆನಂ-ದ-ವುಂ-ಟಾ-ಗು-ವುದ
ಆಫ್
ಆಫ್ಗಳು
ಆಮೂ-ಲಾ-ಗ್ರ-ವಾಗಿ
ಆಮೇ-ಲಾ-ಮೇಲೆ
ಆಮೇಲೆ
ಆಯಾ-ಸ-ವುಂ-ಟಾಗಿ
ಆಯಿತು
ಆರಂ-ಭ-ವಾ-ಗು-ತ್ತದೆ
ಆರಂ-ಭಿ-ಸಿ-ದು-ವೆಂದರೆ
ಆರಂ-ಭಿ-ಸು-ವು-ದ-ಕ್ಕಿಂತ
ಆರಾ-ಮ-ವಾಗಿ
ಆರಿ
ಆರಿದ
ಆರಿ-ಸಿ-ಕೊಂಡು
ಆರಿ-ಸಿ-ಕೊಳ್ಳ
ಆರಿ-ಸಿ-ಬಿ-ಡು-ತ್ತದೆ
ಆರೋ-ಗ್ಯಕ್ಕೆ
ಆಲಿಸು
ಆಲೋ-ಚ-ನೆ-ಗಳು
ಆಲೋಚಿ
ಆಲೋ-ಚಿ-ಸ-ಬೇಕು
ಆಲೋ-ಚಿಸಿ
ಆಲೋ-ಚಿ-ಸು-ವ-ವರ
ಆಲೋ-ಚಿ-ಸು-ವ-ವ-ರಂತೆ
ಆಳಕ್ಕೆ
ಆವ-ಶ್ಯಕ
ಆವ-ಶ್ಯ-ಕ-ತೆ-ಯಾ-ದರೂ
ಆಶಾ-ಗೋ-ಪುರ
ಆಶೀ-ರ್ವಾದ
ಆಶೀ-ರ್ವಾ-ದದ
ಆಶೀ-ರ್ವಾ-ದ-ವನ್ನು
ಆಶೀ-ರ್ವಾ-ದವೇ
ಆಶ್ಚ-ರ್ಯ-ವಿಲ್ಲ
ಆಶ್ರಮ
ಆಶ್ರಯ
ಆಸಕ್ತಿ
ಆಸ-ಕ್ತಿ-ಯಿಲ್ಲ
ಆಸ-ನ-ವಿರ
ಆಸೆ-ಗಳ
ಆಸೆ-ಗಳನ್ನು
ಆಸೆ-ಗ-ಳಿಗೆ
ಆಸೆ-ಗ-ಳಿ-ರು-ತ್ತವೆ
ಆಸ್ವಾ-ದಿಸ
ಆಹಾ-ರ-ದೊಂ-ದಿಗೆ
ಆಹಾ-ರ-ಪ-ದಾ-ರ್ಥ-ಗಳನ್ನು
ಆಹ್ವಾ-ನಕ್ಕೆ
ಇ
ಇಂಗ್ಲೀ-ಷಿ-ನಲ್ಲಿ
ಇಂಗ್ಲೀಷು
ಇಂತಹ
ಇಂತಿಂಥ
ಇಂತಿಂ-ಥ-ದನ್ನು
ಇಂತಿಷ್ಟು
ಇಂಥ
ಇಂಥ-ವ-ರಿಂದ
ಇಂಥ-ವ-ರಿ-ಗೆಲ್ಲ
ಇಂದಿಗೂ
ಇಂದಿನ
ಇಂದು
ಇಂದ್ರಿಯ
ಇಂದ್ರಿ-ಯ-ಗಳ
ಇಂದ್ರಿ-ಯ-ಗಳನ್ನು
ಇಂದ್ರಿ-ಯ-ಗ-ಳ-ಲ್ಲವೆ
ಇಂದ್ರಿ-ಯ-ಗಳು
ಇಂದ್ರಿ-ಯ-ಗಳೂ
ಇಚ್ಛಾ-ಶ-ಕ್ತಿ-ಯ-ನ್ನು-ಪ-ಯೋ-ಗಿಸಿ
ಇಟ್ಟಿ-ರುವ
ಇಟ್ಟುಕೊ
ಇಟ್ಟು-ಕೊ-ಳ್ಳ-ಬೇ-ಕಾ-ದಂ-ತೆಯೇ
ಇಟ್ಟು-ಕೊ-ಳ್ಳ-ಬೇಕು
ಇಡ
ಇಡೀ
ಇತರ
ಇತ-ರ-ರನ್ನು
ಇತ-ರ-ರಿಂದ
ಇತ-ರ-ರಿ-ಗಿಂತ
ಇತ-ರ-ರಿಗೆ
ಇತಿ-ಹಾ-ಸ-ವ-ನ್ನೊಮ್ಮೆ
ಇತ್ತು
ಇತ್ಯಾದಿ
ಇತ್ಯಾ-ದಿ-ಗಳ
ಇದಕ್ಕೆ
ಇದಕ್ಕೇ
ಇದನ್ನು
ಇದನ್ನೇ
ಇದರ
ಇದ-ರಲ್ಲಿ
ಇದ-ರ-ಲ್ಲಿ-ರುವ
ಇದ-ರಿಂದ
ಇದ-ರಿಂ-ದಾಗಿ
ಇದೀಗ
ಇದು
ಇದು-ವ-ರೆಗೆ
ಇದೆ
ಇದೆ-ಇದು
ಇದೆ-ಯಲ್ಲ
ಇದೇ
ಇದೊಂದು
ಇದ್ದ-ಕ್ಕಿ-ದ್ದಂತೆ
ಇದ್ದದ್ದೇ
ಇದ್ದರೂ
ಇದ್ದಾನೆ
ಇದ್ದಾರೆ
ಇದ್ದೇ
ಇನ್ನಷ್ಟು
ಇನ್ನಾವ
ಇನ್ನು
ಇನ್ನೂ
ಇನ್ನೇನು
ಇನ್ನೇನೂ
ಇನ್ನೊಂದು
ಇನ್ನೊಬ್ಬ
ಇಪ್ಪ-ತ್ತ-ನಾಲ್ಕು
ಇಪ್ಪ-ತ್ತ-ನಾಲ್ಕೇ
ಇಪ್ಪತ್ತು
ಇಬ್ಬ-ರಿಗೆ
ಇರ
ಇರ-ಬ-ಹುದು
ಇರ-ಬಾ-ರದು
ಇರ-ಬೇ-ಕಾ-ದದ್ದು
ಇರ-ಬೇ-ಕಾದ್ದು
ಇರ-ಬೇ-ಕಾದ್ದೇ
ಇರ-ಬೇಕು
ಇರ-ಲಾರ
ಇರಲಿ
ಇರು
ಇರು-ತ್ತದೆ
ಇರು-ತ್ತ-ದೆಯೋ
ಇರು-ತ್ತವೆ
ಇರು-ತ್ತಾರೆ
ಇರುವ
ಇರು-ವಂತೆ
ಇರು-ವ-ವ-ರಿಗೆ
ಇರು-ವಾ-ಗಲೇ
ಇರು-ವು-ದಿಲ್ಲ
ಇಲ್ಲ
ಇಲ್ಲ-ದಿ-ದ್ದರೆ
ಇಲ್ಲ-ದಿಲ್ಲ
ಇಲ್ಲ-ದ್ದ-ರಿಂದ
ಇಲ್ಲ-ವಲ್ಲ
ಇಲ್ಲ-ವಾ-ದರೆ
ಇಲ್ಲ-ವೆಂ-ದಂ-ತಾ-ಯಿತು
ಇಲ್ಲವೋ
ಇಲ್ಲಿ
ಇಲ್ಲಿದೆ
ಇಲ್ಲಿ-ಯ-ವ-ರೆಗೂ
ಇಲ್ಲಿ-ಯ-ವ-ರೆಗೆ
ಇಳಿ-ಜಾರಿ
ಇಳಿದ
ಇಳಿದು
ಇಳಿ-ಯಿರಿ
ಇಳಿ-ಯು-ತ್ತವೆ
ಇವ
ಇವ-ನಿಗೆ
ಇವನು
ಇವ-ರಿಗೆ
ಇವರು
ಇವರೆಲ್ಲ
ಇವರೆ-ಲ್ಲರ
ಇವರೆ-ಲ್ಲರೂ
ಇವಿಷ್ಟು
ಇವು
ಇವು-ಗಳನ್ನು
ಇವು-ಗ-ಳಿಗೆ
ಇವು-ಗ-ಳಿಗೇ
ಇವೆ
ಇವೆ-ರ-ಡಕ್ಕೂ
ಇವೆಲ್ಲ
ಇವೆ-ಲ್ಲಕ್ಕೂ
ಇವೆ-ಲ್ಲ-ದರ
ಇವೆ-ಲ್ಲ-ದ-ರೊಂ-ದಿಗೆ
ಇವೆ-ಲ್ಲವೂ
ಇವೇ
ಇಷ್ಟ
ಇಷ್ಟನ್ನು
ಇಷ್ಟ-ವಾ-ಗುವ
ಇಷ್ಟವೂ
ಇಷ್ಟು
ಇಸ್ಪೀ-ಟು-ಲಾ-ಟರಿ
ಈ
ಈಗ
ಈಗ-ತಾನೆ
ಈಗ-ಲಾ-ದರೂ
ಈಗಾ-ಗಲೇ
ಈಗಿ
ಈಗಿನ
ಈಗಿ-ರುವ
ಈಗೇನು
ಈಡಾ-ದ-ವರ
ಈಡೇರಿ
ಈಡೇ-ರಿಸಿ
ಈಡೇ-ರಿ-ಸಿ-ಕೊ-ಡಲು
ಈಡೇ-ರಿ-ಸಿ-ಕೊಡು
ಈತ
ಈಶ-ಶ-ಕ್ತಿ-ಯಿಂದ
ಉ
ಉಂಟಾ-ಗು-ತ್ತದೆ
ಉಂಟಾ-ಗುವ
ಉಂಟು
ಉಂಡರೆ
ಉಗುರು
ಉಡುಗಿ
ಉಡು-ಗಿ-ಸು-ವುದು
ಉತ್ತಮ
ಉತ್ತರ
ಉತ್ತ-ರ-ಪ-ತ್ರಿ-ಕೆ-ಗಳು
ಉತ್ತ-ರ-ವನ್ನು
ಉತ್ತ-ರ-ವನ್ನೇ
ಉತ್ತ-ರ-ವಿದೆ
ಉತ್ತ-ರ-ವಿಷ್ಟೆ
ಉತ್ತ-ರವೂ
ಉತ್ತರಿ
ಉತ್ತ-ರಿ-ಸಿ-ರು-ವುದು
ಉತ್ತೀ-ರ್ಣ-ನಾಗು
ಉತ್ತೀ-ರ್ಣ-ರಾ-ಗುವ
ಉತ್ತೀ-ರ್ಣ-ರಾ-ದಾರು
ಉತ್ಸಾಹ
ಉತ್ಸಾ-ಹಕ್ಕೆ
ಉತ್ಸಾ-ಹದ
ಉತ್ಸಾ-ಹ-ದಿಂದ
ಉತ್ಸಾ-ಹ-ದಿಂ-ದಲೂ
ಉತ್ಸಾ-ಹ-ಪೂರ್ಣ
ಉತ್ಸಾ-ಹ-ಪೂ-ರ್ಣ-ವಾ-ಗಿ-ರು-ತ್ತದೆ
ಉತ್ಸಾ-ಹ-ವನ್ನು
ಉದಾತ್ತ
ಉದಾ-ಹ-ರ-ಣಗೆ
ಉದಾ-ಹ-ರ-ಣೆ-ಗಳೂ
ಉದಾ-ಹ-ರ-ಣೆಗೆ
ಉದಾ-ಹ-ರ-ಣೆ-ಲೆಟ್
ಉದಿ-ಸಿ-ದರೆ
ಉದ್ದು-ದ್ದನೆ
ಉದ್ದೇ-ಶ-ವನ್ನು
ಉದ್ದೇ-ಶ-ವಿ-ಟ್ಟು-ಕೊಂಡು
ಉದ್ಭ-ವಿ-ಸಿತು
ಉನ್ನತ
ಉಪ
ಉಪ-ದೇ-ಶ-ಗಳನ್ನು
ಉಪ-ಯುಕ್ತ
ಉಪ-ಯೋ-ಗಕ್ಕೂ
ಉಪ-ಯೋ-ಗಿಸಿ
ಉಪ-ಯೋ-ಗಿ-ಸಿ-ಕೊಂಡು
ಉಪ-ಯೋ-ಗಿ-ಸಿ-ಕೊ-ಳ್ಳು-ತ್ತಿ-ದ್ದರೂ
ಉಪ-ಯೋ-ಗಿ-ಸುವ
ಉಪಾಯ
ಉಪಾ-ಯ-ಗಳನ್ನು
ಉಪಾ-ಯದ
ಉಪಾ-ಯ-ವನ್ನು
ಉಪಾ-ಯ-ವೇನು
ಉಪಾ-ಯ-ವೇನೆಂದರೆ
ಉಪ್ಪು
ಉರು
ಉಲ್ಲಾ-ಸ-ಭ-ರಿ-ತ-ವಾಗಿ
ಉಲ್ಲಾ-ಸ-ಭ-ರಿ-ತ-ವಾ-ಗಿ-ರು-ತ್ತದೆ
ಉಳಿ-ಗಾ-ಲ-ವಿಲ್ಲ
ಉಳಿ-ದಿ-ರು-ತ್ತವೆ
ಉಳಿ-ದು-ಬಿ-ಟ್ಟಿದೆ
ಉಳಿ-ದು-ಬಿ-ಡು-ತ್ತವೆ
ಉಳಿ-ದೆಲ್ಲ
ಉಳಿಯ
ಉಳಿ-ಯು-ತ್ತದೆ
ಊಟ
ಊರಿಗೆ
ಊರು
ಎಂ
ಎಂಟು
ಎಂಟು-ಹತ್ತು
ಎಂತಹ
ಎಂತ-ಹದು
ಎಂತೆಂಥ
ಎಂಥ
ಎಂಥ-ವ-ರಿಂ-ದಲೂ
ಎಂದ
ಎಂದರೆ
ಎಂದ-ರೇನು
ಎಂದರ್ಥ
ಎಂದಾ-ಕ್ಷಣ
ಎಂದು
ಎಂದು-ಬಿ-ಟ್ಟರೆ
ಎಂದೆ-ನಲ್ಲ
ಎಂದೆಲ್ಲ
ಎಂದೇ
ಎಂಬ
ಎಂಬಂತೆ
ಎಂಬ-ರ್ಥವೇ
ಎಂಬು
ಎಂಬು-ದನ್ನು
ಎಂಬು-ದನ್ನೂ
ಎಂಬು-ದ-ನ್ನೆಲ್ಲ
ಎಂಬು-ದರ
ಎಂಬುದು
ಎಂಬುದೂ
ಎಚ್ಚ-ತ್ತು-ಕೊಂ-ಡಿ-ರು-ವಾಗ
ಎಚ್ಚರ
ಎಚ್ಚ-ರ-ಗೇ-ಡಿ-ತ-ನ-ವನ್ನು
ಎಚ್ಚ-ರದ
ಎಚ್ಚ-ರ-ದಿಂ-ದಿ-ರ-ಬೇಕು
ಎಚ್ಚ-ರ-ದಿಂ-ದಿ-ರು-ತ್ತ-ದೆಯೋ
ಎಚ್ಚ-ರ-ವ-ಹಿ-ಸು-ವುದನ್ನು
ಎಚ್ಚ-ರ-ವಾ-ಗಿ-ಡಲು
ಎಚ್ಚ-ರ-ವಾ-ಗಿ-ಡು-ವು-ದ-ಕ್ಕಾ-ಗಿಯೇ
ಎಚ್ಚ-ರ-ವಾ-ಗಿ-ರಲು
ಎಚ್ಚ-ರ-ವಾ-ಗಿ-ರು-ತ್ತದೆ
ಎಚ್ಚ-ರ-ಸ್ಥಿತಿ
ಎಚ್ಚ-ರಿಕೆ
ಎಚ್ಚ-ರಿ-ಕೆ-ಯಿಂ-ದಲೂ
ಎಚ್ಚ-ರಿ-ಕೆ-ಯಿಂ-ದಲೆ
ಎಚ್ಚ-ರಿ-ಕೆ-ಯಿಂ-ದಿ-ದ್ದರೂ
ಎಚ್ಚ-ರಿ-ಕೆ-ಯಿಂ-ದಿ-ದ್ದರೆ
ಎಚ್ಚ-ರಿ-ಕೆ-ಯಿಂ-ದಿ-ರ-ಬೇಕು
ಎಣಿಸಿ
ಎಣ್ಣೆ
ಎತ್ತ-ರದ
ಎತ್ತಿ-ತೋ-ರಿ-ಸು-ತ್ತದೆ
ಎದು-ರು-ಮಾ-ತಿಲ್ಲ
ಎದ್ದ
ಎದ್ದರೂ
ಎದ್ದರೆ
ಎದ್ದಾ-ಗಿ-ನಿಂದ
ಎದ್ದಿ-ರ-ಲಿಲ್ಲ
ಎದ್ದು
ಎನ್ನ-ಬ-ಹುದು
ಎನ್ನು-ತ್ತದೆ
ಎನ್ನು-ತ್ತಾರೆ
ಎನ್ನುವ
ಎನ್ನು-ವ-ವ-ರುಂಟು
ಎನ್ನು-ವುದು
ಎನ್ನು-ವುದೇ
ಎರಗಿ
ಎರ-ಡಕ್ಕೂ
ಎರ-ಡ-ನೆಯ
ಎರ-ಡ-ನೆ-ಯ-ದಾಗಿ
ಎರ-ಡನೇ
ಎರಡು
ಎರಡೂ
ಎಲ್ಲ
ಎಲ್ಲರ
ಎಲ್ಲ-ರಿ-ಗಿಂ-ತಲೂ
ಎಲ್ಲ-ರಿಗೂ
ಎಲ್ಲರೂ
ಎಲ್ಲ-ವನ್ನೂ
ಎಲ್ಲ-ವ-ಸ್ತು-ಗಳೂ
ಎಲ್ಲವೂ
ಎಲ್ಲಾ
ಎಲ್ಲಿ
ಎಲ್ಲಿಂದ
ಎಲ್ಲಿ-ದ್ದಾನೆ
ಎಲ್ಲಿಯೂ
ಎಲ್ಲೆಂ-ದ-ರಲ್ಲಿ
ಎಲ್ಲೆಂ-ದ-ರ-ಲ್ಲಿಗೆ
ಎಳೆ-ದಾ-ಡು-ವಂ-ಥ-ವು-ಗ-ಳಾ-ದ್ದ-ರಿಂದ
ಎಳೆದು
ಎಷ್ಟರ
ಎಷ್ಟಾ-ಗಿವೆ
ಎಷ್ಟಾ-ದರೂ
ಎಷ್ಟಿ-ರು-ತ್ತದೆ
ಎಷ್ಟು
ಎಷ್ಟೇ
ಎಷ್ಟೊಂದು
ಎಷ್ಟೋ
ಎಸ್ಸೆ-ಸ್ಸೆಲ್ಸಿ
ಎಸ್ಸೆ-ಸ್ಸೆ-ಲ್ಸಿಯ
ಏಕ
ಏಕ-ಕಾಲ
ಏಕತ್ರ
ಏಕ-ತ್ರ-ಗೊ-ಳಿ-ಸಿ-ದ್ದ-ರಿಂದ
ಏಕಾ
ಏಕಾ-ಏಕಿ
ಏಕಾಗ್ರ
ಏಕಾ-ಗ್ರ-ಗೊಂಡ
ಏಕಾ-ಗ್ರ-ಗೊಂ-ಡಿ-ರು-ತ್ತದೆ
ಏಕಾ-ಗ್ರ-ಗೊ-ಳಿ-ಸ-ಬೇ-ಕಾದ
ಏಕಾ-ಗ್ರ-ಗೊ-ಳಿ-ಸ-ಬೇಕು
ಏಕಾ-ಗ್ರ-ಗೊ-ಳಿ-ಸ-ಬೇ-ಕೆಂ-ಬು-ದೇನೋ
ಏಕಾ-ಗ್ರ-ಗೊ-ಳಿ-ಸಲು
ಏಕಾ-ಗ್ರ-ಗೊ-ಳಿ-ಸಿ-ದಾಗ
ಏಕಾ-ಗ್ರ-ತಾ-ಸೂ-ತ್ರ-ಗಳನ್ನು
ಏಕಾ-ಗ್ರತೆ
ಏಕಾ-ಗ್ರ-ತೆ-ಇ-ವರ
ಏಕಾ-ಗ್ರ-ತೆಗೆ
ಏಕಾ-ಗ್ರ-ತೆಯ
ಏಕಾ-ಗ್ರ-ತೆ-ಯನ್ನು
ಏಕಾ-ಗ್ರ-ತೆ-ಯಲ್ಲಿ
ಏಕಾ-ಗ್ರ-ತೆ-ಯಿಂದ
ಏಕಾ-ಗ್ರ-ತೆ-ಯೆಂ-ದರೆ
ಏಕಾ-ಗ್ರ-ತೆ-ಯೆಂ-ಬುದು
ಏಕಾ-ಗ್ರ-ತೆಯೇ
ಏಕಾ-ಗ್ರ-ತೆ-ಸಿ-ದ್ಧಿ-ಸಿ-ರು-ತ್ತದೆ
ಏಕಾ-ಗ್ರ-ವಾ-ಗಲು
ಏಕಾ-ಗ್ರ-ವಾ-ಗು-ತ್ತದೆ
ಏಕಾ-ಗ್ರ-ವಾ-ಗುವ
ಏಕಿಲ್ಲ
ಏಕೆ
ಏಕೆಂ-ದರೆ
ಏಟಿನ
ಏಟು
ಏಟೋ
ಏನನ್ನು
ಏನಯ್ಯಾ
ಏನಾ-ಗ-ಬ-ಹು-ದೆಂದು
ಏನಾ-ದರೂ
ಏನಾ-ದ-ರೊಂದು
ಏನಿದೆ
ಏನು
ಏನೂ
ಏನೆಂ-ದರೆ
ಏನೆಂದು
ಏನೇ
ಏನೇ-ನನ್ನು
ಏನೇ-ನನ್ನೋ
ಏನೇನೋ
ಏನೋ
ಏನ್ರಿ
ಏರಿ-ಳಿ-ತ-ಗ-ಳಿಗೆ
ಏಳ
ಏಳು
ಏಳು-ತ್ತದೆ
ಏಳುವ
ಏಳು-ವಾಗ
ಏಳು-ವು-ದರ
ಏಳು-ವುದು
ಏಳೆಂಟು
ಏಳೇ
ಏಸು-ಕ್ರಿಸ್ತ
ಐದಕ್ಕೆ
ಐದ-ರಿಂದ
ಐದು
ಒಂದಲ್ಲ
ಒಂದಿಷ್ಟು
ಒಂದು
ಒಂದೂ-ವರೆ
ಒಂದೆ-ರಡು
ಒಂದೇ
ಒಂದೊಂದು
ಒಂದೋ
ಒಕ್ಕೊ-ರ-ಳಿ-ನಿಂದ
ಒಗ್ಗಿ-ಹೋಗಿ
ಒಗ್ಗೂ-ಡಿ-ಸ-ಬೇ-ಕಾ-ಗು-ತ್ತದೆ
ಒಗ್ಗೂ-ಡಿ-ಸ-ಬೇ-ಕಾ-ದರೆ
ಒಟ್ಟಿಗೇ
ಒಟ್ಟಿ-ನಲ್ಲಿ
ಒಣ-ಗಿ-ಸಿ-ಕೊ-ಳ್ಳು-ವು-ದೊಂದು
ಒತ್ತ-ಟ್ಟಿ-ಗಿಟ್ಟ
ಒತ್ತ-ಡ-ದಿಂದ
ಒದ-ಗಿಸಿ
ಒದ-ಗಿ-ಸಿ-ದಾ-ಗ-ಲೆಲ್ಲ
ಒದ-ಗಿ-ಸು-ತ್ತಿದೆ
ಒಪ್ಪು-ವಂ-ತಹ
ಒಬ್ಬ
ಒಬ್ಬ-ನಿಗೆ
ಒಬ್ಬೊ-ಬ್ಬ-ರದು
ಒಬ್ಬೊ-ಬ್ಬರು
ಒಮ್ಮ-ನ-ಸ್ಸಿ-ನಿಂದ
ಒಮ್ಮು-ಖ-ವಾ-ಗಿ-ರು-ವುದು
ಒಮ್ಮು-ಖ-ವಾ-ಗಿ-ಸು-ವುದು
ಒಮ್ಮೆ
ಒಳ-ಗಾ-ಗು-ತ್ತದೆ
ಒಳಗಿ
ಒಳ್ಳೆಯ
ಒಳ್ಳೆ-ಯದು
ಓ
ಓಗೊಟ್ಟು
ಓಗೊಟ್ಟೇ
ಓಗೊ-ಡದೆ
ಓಡುವ
ಓದ
ಓದದೆ
ಓದ-ಬ-ಲ್ಲ-ರ-ಲ್ಲದೆ
ಓದ-ಬೇಕು
ಓದಲು
ಓದಿ
ಓದಿ-ಕೊಂಡಿ
ಓದಿ-ಕೊಂಡು
ಓದಿ-ಕೊಂ-ಡು-ಹೋ-ದರೆ
ಓದಿ-ಕೊ-ಳ್ಳ-ಬೇ-ಕೆಂ-ಬುದೇ
ಓದಿ-ಕೊ-ಳ್ಳು-ವಾಗ
ಓದಿ-ದರೆ
ಓದಿ-ದ-ವರೆ-ಲ್ಲರೂ
ಓದಿ-ದ್ದನ್ನು
ಓದಿ-ದ್ದ-ನ್ನೆಲ್ಲ
ಓದಿ-ದ್ದನ್ನೇ
ಓದಿ-ದ್ದೀರಿ
ಓದಿದ್ದೂ
ಓದಿ-ದ್ದೆ-ಲ್ಲವೂ
ಓದಿಯೇ
ಓದೀ
ಓದು
ಓದುತ್ತ
ಓದು-ತ್ತಿ-ದ್ದರೆ
ಓದು-ತ್ತೇನೆ
ಓದುವ
ಓದು-ವ-ವರೂ
ಓದು-ವಾಗ
ಓದು-ವಿ-ಕೆ-ಯೊಂದೇ
ಓದು-ವು-ದ-ರಲ್ಲೇ
ಓದು-ವು-ದ-ರಿಂದ
ಓದು-ವುದು
ಓದು-ವು-ದು-ಬ-ರೆ-ಯು-ವುದನ್ನು
ಓದು-ವು-ದು-ಬ-ರೆ-ಯು-ವುದು
ಓದು-ವುದೇ
ಓಹ್
ಕಂಗಾ
ಕಂಡಂ-ತೆಯೇ
ಕಂಡ-ಕಂ-ಡ-ಕ-ಡೆ-ಗೆಲ್ಲ
ಕಂಡ-ಕಂ-ಡಲ್ಲಿ
ಕಂಡ-ದ್ದನ್ನು
ಕಂಡರೆ
ಕಂಡು
ಕಂಡು-ಕೊಂಡ
ಕಂಡು-ಕೊ-ಳ್ಳ-ಬೇಕು
ಕಂಡು-ಕೊ-ಳ್ಳಲು
ಕಂಡು-ಬಂ-ದರೆ
ಕಂಡು-ಬರು
ಕಂಡು-ಬ-ರು-ತ್ತದೆ
ಕಂಡು-ಬ-ರು-ವುದು
ಕಚ್ಚುತ್ತ
ಕಛೇರಿ
ಕಟು
ಕಟು-ವಾ-ಗಿಯೇ
ಕಟು-ಸತ್ಯ
ಕಟ್ಟ-ಡ-ಗಳ
ಕಟ್ಟ-ಡ-ಗ-ಳನ್ನೇ
ಕಟ್ಟಿ
ಕಟ್ಟಿ-ಕೊಂಡು
ಕಟ್ಟಿ-ಟ್ಟದ್ದು
ಕಟ್ಟಿಡ
ಕಟ್ಟಿ-ಡ-ಬೇ-ಕಾ-ಗು-ತ್ತದೆ
ಕಟ್ಟಿ-ಡುವ
ಕಟ್ಟಿ-ಡು-ವುದೂ
ಕಟ್ಟುವ
ಕಟ್ಟು-ವ-ವರ
ಕಡಿಮೆ
ಕಡಿ-ಮೆ-ಯ-ದೇ-ನಲ್ಲ
ಕಡೆ-ಗ-ಣಿ-ಸ-ಬೇಡ
ಕಡೆಗೂ
ಕಡೆಗೆ
ಕಡೆಯ
ಕಣ್ಣಲ್ಲಿ
ಕಣ್ಣಿಗೂ
ಕಣ್ಣಿ-ಟ್ಟಿ-ದ್ದರೆ
ಕಣ್ಣಿ-ಟ್ಟಿರ
ಕಣ್ಣಿ-ಟ್ಟಿ-ರ-ಬೇ-ಕಾ-ಗು-ತ್ತದೆ
ಕಣ್ಣಿಟ್ಟು
ಕಣ್ಣು
ಕಣ್ಣು-ಕಿವಿ
ಕಣ್ಣು-ಗಳನ್ನು
ಕಣ್ಣು-ಗ-ಳಿಗೂ
ಕಣ್ಣು-ಗ-ಳಿಗೆ
ಕಣ್ಣು-ಗಳು
ಕಣ್ಣೆ
ಕಣ್ಮ-ರೆ-ಯಾ-ಗು-ವಂ-ಥವು
ಕತ್ತ-ರಿಸಿ
ಕತ್ತಿ
ಕಥೆ-ಕಾ-ದಂ-ಬರಿ
ಕನಿಷ್ಠ
ಕನಿ-ಷ್ಠ-ಪಕ್ಷ
ಕನ್ನಡ
ಕನ್ನ-ಡ-ದಂ-ತೆಯೇ
ಕನ್ನ-ಡ-ದಲ್ಲಿ
ಕಬ್ಬಿ-ಣದ
ಕಮ್ಮಾ-ರನ
ಕಮ್ಮಾ-ರ-ರಲ್ಲಿ
ಕರ-ಗತ
ಕರ-ಣ-ಪೂ-ರ್ವ-ಕ-ವಾಗಿ
ಕರು-ಣಾ-ಮಯ
ಕರೆ-ದಾರು
ಕರೆ-ದೊ-ಯ್ಯು-ತ್ತಿ-ರು-ತ್ತವೆ
ಕರೆ-ಯ-ಬ-ಹುದು
ಕರೆ-ಯ-ಬೇಡಿ
ಕರೆಯು
ಕರೆ-ಯು-ವುದು
ಕರ್ಣ-ಕ-ರ್ಕಶ
ಕರ್ತವ್ಯ
ಕರ್ಮ-ಕೌ-ಶ-ಲ-ಗ-ಳೆಂಬ
ಕಲಿ-ಗಾ-ಲದ
ಕಲಿ-ತದ್ದು
ಕಲಿ-ತ-ದ್ದೆಲ್ಲ
ಕಲಿ-ತು-ಕೊ-ಳ್ಳ-ಬೇಕು
ಕಲಿ-ತು-ಕೊ-ಳ್ಳಲು
ಕಲಿ-ತೆ-ಯಾ-ದರೆ
ಕಲೆ
ಕಲೆ-ಕೈ-ಗಾ-ರಿ-ಕೆ-ಗಳನ್ನು
ಕಲೆ-ಕ್ಟ-ರನ
ಕಲ್ಪ-ನೆ-ಯನ್ನು
ಕಲ್ಪ-ನೆ-ಯಿ-ರಲಿ
ಕಳೆದು
ಕಳೆ-ದು-ಕೊಂ-ಡಾಗ
ಕಳೆ-ದು-ಕೊಂಡು
ಕಳೆ-ದು-ಹೋದ
ಕಳೆ-ದು-ಹೋ-ದರೆ
ಕಳೆ-ಯಲೇ
ಕಷ್ಟ
ಕಷ್ಟ-ನ-ಷ್ಟ-ಭ್ರಷ್ಟ
ಕಷ್ಟದ
ಕಷ್ಟ-ದಿಂದ
ಕಷ್ಟ-ಪ-ಡ-ಬೇಕು
ಕಷ್ಟ-ವಾ-ಗ-ಬ-ಹುದು
ಕಷ್ಟ-ವಿಲ್ಲ
ಕಷ್ಟ-ವೆಂ-ಬುದೂ
ಕಷ್ಟ-ವೆ-ನಿ-ಸಿ-ದರೂ
ಕಾಗ-ದದ
ಕಾಗಿ-ರು-ತ್ತದೆ
ಕಾಗು-ಣಿತ
ಕಾಡ-ಬ-ಹುದು
ಕಾಡಿಗೆ
ಕಾಣ
ಕಾಣ-ಬ-ಹುದು
ಕಾಣ-ಬೇ-ಕೆಂ-ದೆ-ನಿ-ಸು-ವುದೇ
ಕಾಣಿ-ಸಿ-ಕೊಂಡು
ಕಾಣಿಸು
ಕಾಣು
ಕಾಣುತ್ತ
ಕಾಣು-ತ್ತದೆ
ಕಾಣು-ತ್ತಿ-ರು-ವೆ-ಯಷ್ಟೆ
ಕಾತ-ರ-ಗೊಂಡೆ
ಕಾತ-ರ-ಗೊ-ಳ್ಳದೆ
ಕಾತ-ರತೆ
ಕಾತ-ರದ
ಕಾತ-ರ-ಪ-ಟ್ಟ-ದ್ದೆಲ್ಲ
ಕಾತ-ರೆ-ಯಿಂ-ದೇನೂ
ಕಾಪಾಡಿ
ಕಾಪಾ-ಡಿ-ಕೊಂ-ಡರೆ
ಕಾಪಾ-ಡಿ-ಕೊ-ಳ್ಳು-ತ್ತೀಯೋ
ಕಾಪಿ
ಕಾಫಿ-ಟೀ
ಕಾಫಿ-ತಿಂಡಿ
ಕಾಯ-ಬೇ-ಕಾ-ಗು-ತ್ತದೆ
ಕಾಯಿಲೆ
ಕಾಯಿ-ಲೆ-ಗಳು
ಕಾಯುತ್ತ
ಕಾಯು-ವ-ವರ
ಕಾಯು-ವಾಗ
ಕಾಯು-ವು-ದೇನೂ
ಕಾಯೋಣ
ಕಾರಣ
ಕಾರ-ಣ-ಅ-ವನು
ಕಾರ-ಣ-ಗಳು
ಕಾರ-ಣ-ವಾ-ಗಿದೆ
ಕಾರ-ಣ-ವಾ-ಗು-ತ್ತದೆ
ಕಾರ-ಣ-ವಿ-ದ್ದರೂ
ಕಾರ-ಣ-ವೇ-ನಿ-ರ-ಬ-ಹುದು
ಕಾರ-ಣ-ವೇನು
ಕಾರ-ಣಾಂ-ತ-ರ-ಗಳಿಂದ
ಕಾರ್ಯ
ಕಾರ್ಯ-ಕ-ಲಾ-ಪ-ಗಳನ್ನೂ
ಕಾರ್ಯಕ್ಕೆ
ಕಾರ್ಯ-ಗಳನ್ನು
ಕಾರ್ಯ-ಗಳಲ್ಲಿ
ಕಾರ್ಯ-ಗ-ಳೆಲ್ಲ
ಕಾರ್ಯ-ತ-ತ್ಪ-ರ-ನಾ-ಗ-ಬೇಕು
ಕಾರ್ಯ-ತ-ತ್ಪ-ರ-ನಾ-ಗು-ತ್ತಾನೆ
ಕಾರ್ಯದ
ಕಾರ್ಯ-ದಲ್ಲಿ
ಕಾರ್ಯ-ದಿಂದ
ಕಾರ್ಯ-ನಿ-ರ್ವ-ಹಿ-ಸುವ
ಕಾರ್ಯ-ವನ್ನು
ಕಾರ್ಯ-ವನ್ನೂ
ಕಾರ್ಯ-ವ-ನ್ನೆ-ಸ-ಗು-ತ್ತಿದೆ
ಕಾರ್ಯ-ವೊಂದು
ಕಾರ್ಯೋ-ನ್ಮು-ಖ-ನಾ-ಗು-ತ್ತಾನೆ
ಕಾಲ
ಕಾಲ-ಕ-ಳೆಯು
ಕಾಲಕ್ಕೂ
ಕಾಲಕ್ಕೆ
ಕಾಲ-ದಲ್ಲಿ
ಕಾಲ-ನಷ್ಟ
ಕಾಲ-ವನ್ನು
ಕಾಲ-ವ-ಶ-ರಾ-ಗಿ-ಬಿಟ್ಟ
ಕಾಲ-ವಾ-ದರೂ
ಕಾಲ-ವೆಂ-ಬುದು
ಕಾಲಾ-ವ-ಕಾಶ
ಕಾಲಾ-ವಧಿ
ಕಾಲಿ-ನ-ವ-ರೆಗೂ
ಕಾಲುವೆ
ಕಾಲು-ವೆ-ಯಲ್ಲಿ
ಕಾಲೇ-ಜಿ-ನಲ್ಲಿ
ಕಾಲೇಜು
ಕಾಲ್ಬೆ-ರ-ಳಿಗೆ
ಕಾವ-ಲಿನ
ಕಿಂಚಿತ್ತು
ಕಿಟ-ಕಿಯ
ಕಿರ-ಗ-ಣ-ಗಳನ್ನು
ಕಿರ-ಣ-ಗ-ಳಿಗೆ
ಕಿವಿ
ಕಿವಿ-ಗಳಿಂದ
ಕಿವಿ-ಯಿಂದ
ಕೀ
ಕೀಲಿಕೈ
ಕೀಲಿ-ಕೈಯೇ
ಕುಂಠಿ-ತ-ಗೊ-ಳ್ಳು-ತ್ತದೆ
ಕುಂಠಿ-ತ-ವಾ-ಗು-ತ್ತದೆ
ಕುಡಿದು
ಕುಡಿದೇ
ಕುಡಿ-ಯುವ
ಕುಡಿ-ಯು-ವುದನ್ನು
ಕುಡಿ-ಯು-ವು-ದ-ರಿಂದ
ಕುಣಿ-ದಾಡಿ
ಕುಣಿ-ದಾ-ಡು-ತ್ತಿ-ರು-ತ್ತದೆ
ಕುಣಿ-ದಾ-ಡುವ
ಕುಣಿ-ದಾ-ಡು-ವು-ದೊಂದೇ
ಕುಣಿ-ಯುವ
ಕುದಿಸಿ
ಕುರಿತು
ಕುರ್ಚಿ-ಗೆ-ಲ್ಲಿಗೆ
ಕುಳಿತ
ಕುಳಿ-ತರೆ
ಕುಳಿ-ತ-ರೆ-ನಿಂ-ತರೆ
ಕುಳಿ-ತಾಗ
ಕುಳಿ-ತಾ-ಯಿತು
ಕುಳಿ-ತಿದ್ದ
ಕುಳಿ-ತಿ-ದ್ದಾನೆ
ಕುಳಿ-ತಿ-ರು-ವಂ-ತಿಲ್ಲ
ಕುಳಿ-ತಿ-ರು-ವುದು
ಕುಳಿ-ತಿ-ರು-ವು-ದೊಂದೇ
ಕುಳಿತು
ಕುಳಿ-ತುಕೊ
ಕುಳಿ-ತು-ಕೊ-ಳ್ಳಲು
ಕುಳಿ-ತು-ಕೊ-ಳ್ಳುವ
ಕುಳಿ-ತು-ಕೊ-ಳ್ಳು-ವು-ದರ
ಕುಳಿ-ತೆ-ಯೆಂ-ದರೆ
ಕುಳಿ-ತೆ-ವೆಂದರೆ
ಕೂಡ
ಕೂಡಲೇ
ಕೂಡಿ
ಕೂಡಿದ
ಕೂಡಿ-ರ-ಬೇಕು
ಕೂದ-ಲನ್ನು
ಕೃತ-ಕೃತ್ಯ
ಕೃತ-ಜ್ಞ-ತೆ-ಯ-ನ್ನರ್ಪಿ
ಕೃತ-ವಾ-ಗಿ-ಲ್ಲವೆ
ಕೃತಾ-ರ್ಥ-ತೆಯ
ಕೃಷ್ಣ
ಕೃಷ್ಣನ
ಕೆಟ್ಟ
ಕೆಟ್ಟು
ಕೆಟ್ಟು-ಹೋ-ದಾಗ
ಕೆಡಿ-ಸಲು
ಕೆಡಿ-ಸಿ-ಕೊಂಡು
ಕೆಡಿ-ಸಿ-ಕೊ-ಳ್ಳ-ಬೇಡ
ಕೆಡಿ-ಸುವ
ಕೆಡು-ತ್ತದೆ
ಕೆಲ-ಕೆ-ಲವು
ಕೆಲ-ದಿನ
ಕೆಲ-ವ-ರಿಗೆ
ಕೆಲ-ವ-ರಿ-ದ್ದಾರೆ
ಕೆಲ-ವರು
ಕೆಲ-ವಾರು
ಕೆಲವು
ಕೆಲ-ವೊಮ್ಮೆ
ಕೆಲಸ
ಕೆಲ-ಸ-ಕಾ-ರ್ಯ-ಗಳ
ಕೆಲ-ಸ-ಕಾ-ರ್ಯ-ಗಳನ್ನು
ಕೆಲ-ಸಕ್ಕೆ
ಕೆಲ-ಸ-ಗ-ಳಿಗೆ
ಕೆಲ-ಸ-ವಂತೂ
ಕೆಲ-ಸ-ವನ್ನು
ಕೆಲ-ಸ-ವನ್ನೂ
ಕೆಲ-ಸವೇ
ಕೆಲ-ಸ-ವೊಂ-ದನ್ನು
ಕೆಲ-ಸ-ವೊಂದು
ಕೆಳ-ಗಿಳಿ-ದಿದೆ
ಕೇಂದ್ರೀ
ಕೇಂದ್ರೀ-ಕೃ-ತ-ಗೊ-ಳಿ-ಸು-ವುದು
ಕೇಂದ್ರೀ-ಕೃ-ತ-ವಾ-ಗಿ-ರು-ವುದು
ಕೇಂದ್ರೀ-ಕೃ-ತ-ವಾ-ದಾಗ
ಕೇಳ
ಕೇಳ-ದಿ-ರ-ಬೇಕು
ಕೇಳದೆ
ಕೇಳ-ಬಾ-ರ-ದ್ದನ್ನು
ಕೇಳ-ಬೇ-ಕಾ-ಗಿಲ್ಲ
ಕೇಳಲೂ
ಕೇಳಲೇ
ಕೇಳಿದ
ಕೇಳಿ-ದರೆ
ಕೇಳಿ-ದ್ದನ್ನು
ಕೇಳಿ-ದ್ದ-ನ್ನೆಲ್ಲ
ಕೇಳಿಯೇ
ಕೇಳಿಯೋ
ಕೇಳಿ-ಸಿತು
ಕೇಳಿ-ಸು-ವುದೇ
ಕೇಳು
ಕೇಳುತ್ತ
ಕೇಳು-ತ್ತಲೇ
ಕೇಳು-ತ್ತಾರೆ
ಕೇಳು-ತ್ತಿರು
ಕೇಳು-ತ್ತಿ-ರು-ತ್ತಾರೆ
ಕೇಳು-ನೀನು
ಕೇಳುವ
ಕೇಳು-ವುಕ್ಕೆ
ಕೇಳು-ವು-ದರ
ಕೇಳು-ವು-ದುಂಟು
ಕೇಳು-ವು-ದೇ-ನಿದೆ
ಕೇವಲ
ಕೈ
ಕೈಕೆ-ಲ-ಸ-ದಲ್ಲಿ
ಕೈಕೊಟ್ಟು
ಕೈಗಾ-ರಿಕೆ
ಕೈಗೂ-ಡದು
ಕೈಗೊಂಡ
ಕೈಗೊ-ಳ್ಳುವ
ಕೈಚೀಲ
ಕೈಚೆಲ್ಲಿ
ಕೈಬಿ-ಟ್ಟರೆ
ಕೈಮು-ರಿದು
ಕೈಹಾ-ಕಿದ
ಕೊಂಡದ್ದು
ಕೊಂಡರೆ
ಕೊಂಡಿರು
ಕೊಂಡು
ಕೊಂಡುದೇ
ಕೊಚ್ಚಿ
ಕೊಟ್ಟ
ಕೊಟ್ಟರೆ
ಕೊಟ್ಟಾಗ
ಕೊಟ್ಟಿತು
ಕೊಟ್ಟಿದೆ
ಕೊಟ್ಟಿ-ರುವ
ಕೊಟ್ಟು
ಕೊಟ್ಟೆಯೋ
ಕೊಡ
ಕೊಡ-ಬ-ಹುದು
ಕೊಡ-ಲಿ-ಚ್ಛಿ-ಸುವ
ಕೊಡಲು
ಕೊಡಲೇ
ಕೊಡು
ಕೊಡುತ್ತ
ಕೊಡು-ತ್ತಿತ್ತು
ಕೊಡು-ತ್ತಿಲ್ಲ
ಕೊಡು-ವುದು
ಕೊನೆ
ಕೊನೆಗೂ
ಕೊನೆಗೆ
ಕೊರ-ಗ-ಬಾ-ರದು
ಕೊರ-ಗುವ
ಕೊಲೆ-ಗೈ-ದ-ರೆಂದೇ
ಕೊಳ್ಳ
ಕೊಳ್ಳ-ಬ-ಲ್ಲರು
ಕೊಳ್ಳ-ಬೇಕು
ಕೊಳ್ಳಲು
ಕೊಳ್ಳಲೇ
ಕೊಳ್ಳು-ತ್ತದೆ
ಕೊಳ್ಳುವ
ಕೊಳ್ಳು-ವು-ದರ
ಕೊಳ್ಳು-ವುದು
ಕೋಟ್ಯಂ-ತರ
ಕೋಣೆಗೆ
ಕೋಣೆ-ಯ-ಲ್ಲಿ-ರುವ
ಕೋತಿ-ಯಂತೆ
ಕೋತಿ-ಯನ್ನು
ಕೋಪ
ಕೋರ್ಟು
ಕೋಶದ
ಕ್ತಿಕ
ಕ್ಯೂಗಳಲ್ಲಿ
ಕ್ರಮದ
ಕ್ರಮ-ದಲ್ಲಿ
ಕ್ರಮ-ಪ್ರ-ಕಾರ
ಕ್ರಮ-ವನ್ನು
ಕ್ರಮ-ವ-ರಿತ
ಕ್ರಮೇಣ
ಕ್ಲಿಷ್ಟ
ಕ್ಲ್ಪ್ತ-ವಾಗಿ
ಕ್ಷಣ-ಕ್ಷ-ಣಕ್ಕೂ
ಕ್ಷಣ-ವನ್ನೂ
ಕ್ಷೇತ್ರ-ಗ-ಳಲ್ಲೂ
ಕ್ಷೇತ್ರ-ದಲ್ಲಿ
ಕ್ಷೇಮ
ಕ್ಷೌರಿ-ಕನ
ಕ್ಷೌರಿ-ಕ-ರಲ್ಲಿ
ಖಂಡಿತ
ಖಂಡಿ-ತ-ವಾ-ಗಿಯೂ
ಖಚಿ-ತ-ಪ-ಡಿ-ಸಿ-ಕೊಂ-ಡು-ಬಿ-ಡ-ಬೇಕು
ಖರ್ಚಾ-ಗಿವೆ
ಗಂಟು
ಗಂಟೆ
ಗಂಟೆ-ಗಂ-ಟೆಗೂ
ಗಂಟೆ-ಗ-ಟ್ಟಲೆ
ಗಂಟೆ-ಗಳನ್ನು
ಗಂಟೆ-ಗಳು
ಗಂಟೆ-ಗ-ಳೆಂಬ
ಗಂಟೆಯ
ಗಂಟೆ-ಯಂತೆ
ಗಂಟೆ-ಯಲ್ಲಿ
ಗಂಭೀ-ರ-ವಾಗಿ
ಗಡಿ-ಬಿ-ಡಿ-ಯಲ್ಲಿ
ಗಣಿತ
ಗಣೇಶ
ಗತಿ-ಗೆ-ಡು-ವುದು
ಗತಿ-ಯಲ್ಲಿ
ಗತ್ಯ
ಗದ್ದ-ಲ-ಗ-ಳೆಲ್ಲ
ಗದ್ದ-ಲ-ಗ-ಳೊಂ-ದಿಗೆ
ಗದ್ದ-ಲವೂ
ಗಮನ
ಗಮ-ನ-ದ-ಲ್ಲಿ-ರಿ-ಸಿ-ಕೊಂಡು
ಗಮ-ನ-ವಿಟ್ಟು
ಗಮ-ನ-ವಿ-ರು-ವು-ದಕ್ಕೆ
ಗಮ-ನ-ವಿ-ರು-ವು-ದಿಲ್ಲ
ಗಮ-ನ-ಹ-ರಿ-ಸು-ವುದು
ಗಮ-ನಿಸ
ಗಮ-ನಿ-ಸ-ಬೇಕು
ಗಮ-ನಿಸಿ
ಗಮ-ನಿ-ಸಿ-ದಾಗ
ಗಲ-ಭೆ-ಯೆ-ಬ್ಬಿ-ಸು-ತ್ತಲೇ
ಗಲಿ-ಬಿಲಿ
ಗಲಿ-ಬಿ-ಲಿ-ಯಿಂದ
ಗಲಿ-ಬಿ-ಲಿ-ಯಿಲ್ಲ
ಗಳ
ಗಳನ್ನು
ಗಳಲ್ಲಿ
ಗಳಲ್ಲೇ
ಗಳಾದ
ಗಳಿಗೆ
ಗಳಿ-ಗೆಲ್ಲ
ಗಳಿವೆ
ಗಳಿ-ಸ-ಬ-ಹುದು
ಗಳಿ-ಸಿ-ಕೊ-ಳ್ಳ-ಬೇ-ಕೆಂಬ
ಗಳು
ಗಳೂ
ಗಳೆ-ರ-ಡನ್ನೂ
ಗಾಂಭೀರ್ಯ
ಗಾಡಿ
ಗಾಢ
ಗಾಢ-ನಿ-ದ್ರೆ-ಯೊಂದು
ಗಾಢ-ವಾಗಿ
ಗಾತ್ರ-ದ್ದಾ-ಗಿ-ದ್ದರೆ
ಗಾಳಿಗೆ
ಗಾಳಿ-ಯನ್ನು
ಗಾಳಿ-ಯಲ್ಲಿ
ಗಿಟ್ಟಿ-ಸಿ-ಕೊಂ-ಡ-ದ್ದು-ಉರು
ಗಿರುವ
ಗೀರೀನೂ
ಗುಂಗಿ-ನಲ್ಲೇ
ಗುಂಡು
ಗುಟ್ಟನ್ನು
ಗುಟ್ಟು
ಗುಣ
ಗುತ್ತದೆ
ಗುದ್ದಾ-ಡು-ತ್ತಿ-ರ-ಬೇಕಾ
ಗುರಿ
ಗುರಿ-ತ-ಪ್ಪಿ-ದರೆ
ಗುರಿ-ಯಾ-ಗುವು
ಗುರಿ-ಯಿಟ್ಟು
ಗುರು
ಗುರುವೇ
ಗುರು-ಹಿ-ರಿ-ಯರ
ಗುರು-ಹಿ-ರಿ-ಯ-ರಲ್ಲಿ
ಗುರು-ಹಿ-ರಿ-ಯರು
ಗುರು-ಹಿ-ರಿ-ಯರೂ
ಗುಲ್ಲು
ಗುಲ್ಲೋ
ಗೃಹಿಣಿ
ಗೆದ್ದೆ
ಗೆಲ್ಲಿತ್ತು
ಗೇಡಿ-ತ-ನವೇ
ಗೇನೇನೂ
ಗೊಂಡಾಗ
ಗೊಂದು
ಗೊಗ್ಗರು
ಗೊಡ-ವೆಗೆ
ಗೊಡು-ತ್ತಾನೆ
ಗೊತ್ತಿ-ರುವ
ಗೊತ್ತು
ಗೊತ್ತೆ
ಗೊತ್ತೇನು
ಗೊಳಿ-ಸಿ-ಕೊಂಡು
ಗೊಳಿ-ಸಿ-ಕೊಂಡೇ
ಗೋಡೆ-ಗ-ಳನ್ನೇ
ಗೌರ-ವ-ಗಳಿಂದ
ಗೌರ-ವ-ದಿಂದ
ಗ್ರತೆ
ಗ್ರತೆಯ
ಗ್ರಹಿ-ಸಲು
ಗ್ರಾಮಾಂ-ತರ
ಘಳಿ-ಗೆ-ಯಲ್ಲಿ
ಚಂಚಲ
ಚಂಚ-ಲ-ಗೊ-ಳಿ-ಸುವ
ಚಂಚ-ಲ-ವಾ-ಗಿ-ತ್ತೆ-ನ್ನು-ವಾಗ
ಚಂಚ-ಲ-ವಾಗು
ಚಂಚ-ಲ-ವಾ-ದದ್ದು
ಚಂಚ-ಲ-ವೆಂ-ಬುದು
ಚಟು-ವ-ಟಿ-ಕೆ-ಯಿಂದ
ಚಟು-ವ-ಟಿ-ಕೆ-ಯಿಂ-ದಿ-ರು-ತ್ತದೆ
ಚರಿಯ
ಚರ್ಚಿ-ಸಿದೆ
ಚರ್ಚಿ-ಸುವ
ಚರ್ಚಿ-ಸು-ವಾಗ
ಚರ್ಚಿ-ಸು-ವು-ದ-ರಿಂದ
ಚರ್ಚೆ
ಚರ್ಮ
ಚರ್ಮ-ಗ-ಳೆಂಬ
ಚಲ-ನ-ವ-ಲ-ನ-ಗಳನ್ನು
ಚಲಿ-ಸುತ್ತ
ಚಾಣಾ-ಕ್ಷ-ತೆಯೇ
ಚಿಂತಿ-ಸುತ್ತ
ಚಿಂತಿ-ಸು-ತ್ತಿ-ದ್ದೆವೋ
ಚಿಂತೆ-ಗ-ಳ-ನ್ನಿ-ಟ್ಟು-ಕೊಂಡು
ಚಿಂತೆ-ಯೊಂದು
ಚಿತ್ತು
ಚಿತ್ತು-ಗಳು
ಚಿತ್ರ-ಕಲೆ
ಚಿತ್ರ-ವಿ-ಚಿತ್ರ
ಚಿತ್ರಿ-ಸಿದೆ
ಚಿನ್ನದ
ಚಿಮ್ಮು-ವಂತೆ
ಚಿಲುಮೆ
ಚುರು
ಚುರು-ಕಾ-ಗಿ-ರು-ತ್ತದೆ
ಚುರು-ಕಾ-ಗಿ-ಸು-ವು-ದ-ರಿಂದ
ಚುರು-ಕಿ-ಲ್ಲ-ದ-ವರ
ಚೆಂಡಿ-ನಂ-ತಿ-ರ-ಬೇಕೇ
ಚೆನ್ನಾಗಿ
ಚೆನ್ನಾ-ಗಿ-ದೆಯೇ
ಚೆನ್ನಾ-ಗಿಯೇ
ಚೆನ್ನಾ-ಗಿ-ಲ್ಲ-ದಿ-ದ್ದರೆ
ಚೆನ್ನಾ-ಗಿ-ವೆಯೇ
ಚೈತ-ನ್ಯದ
ಚೊಕ್ಕಟ
ಚೊಕ್ಕ-ಟ-ವಾ-ಗಿ-ಟ್ಟಿ-ರ-ಬೇಕು
ಚೊಕ್ಕ-ಟ-ವಾ-ಗಿಯೂ
ಚೊಕ್ಕ-ಟ-ವಾ-ಗಿ-ರ-ಬೇಕು
ಚೊಕ್ಕ-ಟ-ವಾ-ಗಿ-ಲ್ಲ-ದಿ-ದ್ದರೆ
ಚೊಕ್ಕ-ಟವೂ
ಚೌತಿ
ಚ್ಯವ-ನ-ಪ್ರಾ-ಶಾ-ದಿ-ಗಳನ್ನು
ಛಲ-ವನ್ನು
ಜಗತ್ತಿ
ಜಗ-ತ್ತಿನ
ಜಗ-ತ್ತಿ-ನಲ್ಲಿ
ಜಗ-ತ್ತಿ-ನಲ್ಲೇ
ಜನ
ಜನರ
ಜನ-ರಿಗೆ
ಜನೆ-ಯಲ್ಲಿ
ಜಯ
ಜಯ-ಶಾ-ಲಿ-ಯಾ-ಗಿ-ಸುವ
ಜಯ-ಶಾ-ಲಿ-ಯಾ-ಗು-ತ್ತಾನೆ
ಜಲ-ಪಾ-ತ-ದಂತೆ
ಜವಾ-ಬು-ದಾರಿ-ಯುತ
ಜಾಂಬ-ವಂತ
ಜಾಗ
ಜಾಗೃ-ತ-ವಾ-ಗಿ-ರು-ತ್ತದೆ
ಜಾಡ್ಯ-ವನ್ನು
ಜಾಣ-ತ-ನದ
ಜಾಣರು
ಜಾರಲು
ಜಾರಿ-ದರೂ
ಜಾರು-ದಾರಿ-ಗ-ಳತ್ತ
ಜಾರು-ದಾರಿ-ಯಿಂದ
ಜೀವನ
ಜೀವ-ನದ
ಜೀವ-ನ-ದಲ್ಲಿ
ಜೀವ-ನ-ವನ್ನು
ಜೀವ-ನ-ವಿಡೀ
ಜೀವ-ನ-ವೆಂ-ದ-ರೇನು
ಜೂಜು-ಜು-ಗಾರಿ
ಜೂನ್
ಜೇಬಿಗೆ
ಜೇಬಿ-ನಲ್ಲಿ
ಜೊತೆಗೆ
ಜೊತೆ-ಯಲ್ಲಿ
ಜೊತೆ-ಯಲ್ಲೇ
ಜೋಗದ
ಜ್ಞಾನ-ವನ್ನು
ಜ್ಞಾನ-ವನ್ನೇ
ಜ್ಞಾನ-ವಿದ್ದೂ
ಜ್ಞಾನ-ವಿ-ರ-ಬೇಕು
ಜ್ಞಾನ-ಸಂ-ಪಾ-ದ-ನೆ-ಗಾಗಿ
ಜ್ಞಾನಾರ್
ಜ್ಞಾಪಕ
ಜ್ಞಾಪ-ಕ-ಶ-ಕ್ತಿ-ಯನ್ನು
ಜ್ವರ
ಜ್ವರಕ್ಕೆ
ಜ್ವರವೇ
ಟಾನಿಕ್ಕು
ಟಿಕೆಟು
ಟಿಕೆಟ್
ಟಿವಿ
ಟೀ-ಕಾಫೀ
ಟೀಕೆ
ಟೀವಿಯ
ಡುಮ್ಕಿ
ಡೆಸ್ಕ-ನ್ನಾ-ದರೂ
ಡೆಸ್ಕಾ-ದರೂ
ಡೆಸ್ಕಿಗೆ
ಡೆಸ್ಕಿನ
ತಂತ್ರ-ವನ್ನು
ತಂತ್ರೋ-ಪಾ-ಯ-ವಿ-ದ್ದರೆ
ತಂತ್ರೋ-ಪಾ-ಯವೂ
ತಂದು
ತಂದು-ಕೊಂ-ಡಾಗ
ತಂದು-ಕೊಂಡು
ತಂದು-ಕೊಳ್ಳ
ತಂದು-ಕೊ-ಳ್ಳ-ಬೇ-ಕಾ-ಗು-ತ್ತದೆ
ತಂದು-ಕೊ-ಳ್ಳ-ಬೇಕು
ತಂದು-ಕೊ-ಳ್ಳಲು
ತಂದು-ಕೊ-ಳ್ಳ-ಲೇ-ಬೇ-ಕೆಂ-ಬ-ವರು
ತಂದು-ಕೊ-ಳ್ಳು-ವಲ್ಲಿ
ತಂದು-ಕೊ-ಳ್ಳು-ವುದು
ತಂದೆಗೆ
ತಕ್ಕ
ತಕ್ಕಂತೆ
ತಕ್ಕಷ್ಟು
ತಡ-ವಾಗಿ
ತಡ-ವಾ-ಗು-ವು-ದಕ್ಕೆ
ತಡೆ-ದಿ-ಡು-ವುದೂ
ತಡೆ-ಹಿ-ಡಿದು
ತಣ್ಣೀರ
ತತ್ಕಾ-ಲಕ್ಕೆ
ತದ್ಭ-ವತಿ
ತದ್ವಿ-ಪ-ರೀ-ತ-ವಾಗಿ
ತನಗೆ
ತನ್ನ
ತನ್ನದೇ
ತನ್ನ-ಷ್ಟಕ್ಕೆ
ತನ್ಮ-ಯ-ವಾ-ಗಿ-ರು-ವಂತೆ
ತನ್ಮೂ-ಲಕ
ತಪ-ಸ್ಸ-ಲ್ಲವೆ
ತಪ್ಪದೆ
ತಪ್ಪಾಗಿ
ತಪ್ಪಾ-ಗಿ-ರು-ತ್ತದೆ
ತಪ್ಪಿದ
ತಪ್ಪಿ-ಸಿದ
ತಪ್ಪು
ತಪ್ಪು-ಗ-ಳಿ-ವೆಯೆ
ತಪ್ಪು-ಗಳು
ತಪ್ಪು-ತ್ತದೆ
ತಪ್ಪು-ವು-ದಿ-ಲ್ಲ-ವಾ-ದ್ದ-ರಿಂದ
ತಮ-ಗಿ-ಷ್ಟ-ವಾ-ದು-ದನ್ನು
ತಮಗೆ
ತಮ-ತ-ಮಗೆ
ತಮ್ಮ
ತಮ್ಮಟೆ
ತಮ್ಮ-ತಮ್ಮ
ತಮ್ಮದೇ
ತಮ್ಮೆ-ಡೆಗೆ
ತಯಾ-ರಾ-ಗಿ-ರ-ಬೇಕು
ತಯಾ-ರಿ-ಪು-ನ-ರಾ-ವ-ರ್ತ-ನೆಯ
ತಯಾ-ರಿ-ಸಿ-ಕೊಳ್ಳ
ತರ-ಗ-ತಿಗೆ
ತರ-ಗ-ತಿ-ಯಲ್ಲಿ
ತರ-ಬಾ-ರದು
ತರ-ಬೇಕು
ತರು-ಣ-ತ-ರು-ಣಿ-ಯ-ರಲ್ಲಿ
ತರ್ಕ-ಬದ್ಧ
ತಲು-ಪಿತು
ತಲು-ಪಿದ
ತಲು-ಪು-ವ-ಷ್ಟ-ರಲ್ಲಿ
ತಲೆ
ತಲೆ-ಮೈ
ತಲೆ-ಕೆ-ರೆ-ದು-ಕೊ-ಳ್ಳು-ವುದು
ತಲೆಗೂ
ತಲೆಗೆ
ತಲೆ-ಯಿಂದ
ತಲೆ-ಯಿಟ್ಟ
ತಲೆ-ಯೋ-ಡಿಸಿ
ತಲೆ-ಸ್ನಾನ
ತಲೆ-ಸ್ನಾ-ನದ
ತಲ್ಲೀ-ನ-ನಾ-ಗಿ-ರ-ಬೇಕು
ತವಕ
ತವ-ಕ-ದಿಂದ
ತಾನಾ-ಗಿಯೇ
ತಾನೆ
ತಾನೇ
ತಾನೇ-ತಾ-ನಾಗಿ
ತಾಪ-ತ್ರ-ಯ-ಗ-ಳಲ್ಲೇ
ತಾಯಿಗೆ
ತಾಯ್ತಂದೆ
ತಾಯ್ತಂ-ದೆ-ಯ-ರದೋ
ತಾಯ್ತಂ-ದೆ-ಯರು
ತಾಳ-ಬಾ-ರದು
ತಾಳ-ಬೇಕಾ
ತಾಳ-ಲಾ-ರದೆ
ತಾಳಿದ
ತಾಳುತ್ತ
ತಾಳುವ
ತಾಳ್ಮೆ-ಯಿಂದ
ತಾವು
ತಾವೇ
ತಿಂಗಳ
ತಿಂಗ-ಳ-ಲ್ಲಂತೂ
ತಿಂಗಳಲ್ಲಿ
ತಿಂಗಳು
ತಿಂಡಿಯ
ತಿಂದು
ತಿಂದು-ಹಾ-ಕ-ದಂತೆ
ತಿಂದೇ
ತಿಕ್ಕಾ-ಟ-ತಿ-ಣು-ಕಾ-ಟ-ಗ-ಳಿ-ರು-ವು-ದಿಲ್ಲ
ತಿದ್ದು-ತ್ತಿ-ದ್ದ-ವನು
ತಿನ್ನ-ದಿ-ರ-ಬೇಕು
ತಿನ್ನ-ಬಾ-ರ-ದ್ದನ್ನು
ತಿರು-ಗು-ವ-ವರ
ತಿಳಿದ
ತಿಳಿ-ದಂ-ತಾ-ಯಿತು
ತಿಳಿ-ದರೆ
ತಿಳಿ-ದಿ-ದ್ದರೆ
ತಿಳಿ-ದಿ-ರ-ಬೇಕು
ತಿಳಿ-ದಿ-ರಲಿ
ತಿಳಿ-ದಿಲ್ಲ
ತಿಳಿದು
ತಿಳಿ-ದುಕೊ
ತಿಳಿ-ದು-ಕೊಂಡು
ತಿಳಿ-ದು-ಕೊಳ್ಳ
ತಿಳಿ-ದು-ಕೊ-ಳ್ಳದೆ
ತಿಳಿ-ದು-ಕೊ-ಳ್ಳ-ಬೇಕು
ತಿಳಿಯ
ತಿಳಿ-ಯ-ಬ-ಹುದು
ತಿಳಿ-ಯಲು
ತಿಳಿ-ಯು-ತ್ತದೆ
ತಿಳಿ-ಯು-ವಂ-ತಾಗು
ತಿಳಿ-ಯು-ವು-ದಿ-ಲ್ಲವೆ
ತಿಳಿ-ಯು-ವುದು
ತಿಳಿ-ಸ-ಲೆ-ತ್ನಿ-ಸು-ತ್ತೇನೆ
ತಿಳಿಸಿ
ತಿಳಿ-ಸಿತು
ತಿಳಿ-ಸಿ-ರು-ವಿರಿ
ತಿಳಿ-ಸು-ತ್ತೇನೆ
ತೀರು-ತ್ತೇನೆ
ತೀವ್ರ-ಗೊ-ಳಿ-ಸು-ವುದು
ತೀವ್ರ-ವಾ-ಗು-ವು-ದುಂಟು
ತೀವ್ರ-ಹಂ-ಬ-ಲ-ವಿ-ದ್ದರೆ
ತುಂಬ
ತುಂಬಲು
ತುಂಬಾ
ತುಂಬಿ
ತುಂಬಿ-ಕೊಂ-ಡಿ-ದ್ದರೆ
ತುಂಬಿ-ಕೊಂ-ಡಿ-ರ-ಲಿ-ಅ-ಚ್ಚ-ಳಿ-ಯದ
ತುಂಬಿ-ಕೊ-ಳ್ಳಲು
ತುಂಬಿ-ಸಿ-ಕೊಳ್ಳು
ತುಂಬು-ತ್ತಾನೆ
ತುದಿ
ತುಪ್ಪ
ತೆಗೆ-ದು-ಕೊಂಡು
ತೆಗೆ-ದು-ಕೊ-ಳ್ಳ-ಬೇಕು
ತೆಗೆ-ದು-ಕೊಳ್ಳಿ
ತೆಗೆಯ
ತೆಗೆ-ಯು-ವು-ದಕ್ಕೆ
ತೆರ
ತೆರ-ನಾ-ಗಿ-ರು-ತ್ತವೆ
ತೆರ-ಬೇಕು
ತೆರ-ಲೇ-ಬೇ-ಕಾ-ಯಿತು
ತೆರು-ವಷ್ಟು
ತೆರೆ-ದಿ-ರು-ವಾಗ
ತೆರೆದು
ತೇಜ-ಸ್ಸನ್ನೂ
ತೇರ್ಗಡೆ
ತೇರ್ಗ-ಡೆ-ಯಾ-ಗ-ಬೇ-ಕಾ-ದರೆ
ತೇರ್ಗ-ಡೆ-ಹೊಂ-ದುವ
ತೊಟ್ಟರೆ
ತೊಡ-ಗ-ಬ-ಹುದು
ತೊಡ-ಗ-ಬೇ-ಕಾ-ದದ್ದು
ತೊಡ-ಗ-ಬೇಕು
ತೊಡ-ಗಿ-ದಾಗ
ತೊಡ-ಗಿ-ರ-ಬೇಕು
ತೊಡ-ಗಿ-ರು-ವಂ-ತಾ-ದರೆ
ತೊಡ-ಗಿ-ರು-ವಂತೆ
ತೊಡ-ಗಿ-ರು-ವ-ವರೆ-ಲ್ಲ-ರಲ್ಲೂ
ತೊಡ-ಗಿ-ರು-ವಾಗ
ತೊಡ-ಗುವ
ತೊಡ-ಗು-ವ-ವರ
ತೊಡ-ಗು-ವು-ದಕ್ಕೆ
ತೊಡ-ಗು-ವು-ದ-ರಿಂದ
ತೊಳೆ-ಯ-ದಿ-ದ್ದರೆ
ತೋಡಿ
ತೋರ
ತೋರಿಸಿ
ತೋರಿ-ಸು-ವುದು
ತೋರಿಸ್ರೀ
ತೋರು-ತ್ತದೆ
ತೋಳ್ಬ-ಲ-ದಲ್ಲಿ
ತೋಳ್ಬ-ಲ-ವಿ-ರ-ಬ-ಹುದು
ತ್ತದೆ
ತ್ತವೆ
ತ್ತಾರಲ್ಲ
ತ್ತಾರೆ
ತ್ತಿದ್ದರೂ
ತ್ತಿದ್ದೆವೋ
ತ್ತಿಯೂ
ತ್ತಿರ-ಬೇ-ಕುಓ
ತ್ತಿರುವೆ
ತ್ತಿಲ್ಲ
ತ್ತೇನೆ
ತ್ತೇವೆಂದು
ತ್ಯಜಿಸಿ
ತ್ರಿವಿ-ಕ್ರಮ
ದಂಡ
ದಂಡ-ವ-ನ್ನಂತೂ
ದಂತೆ
ದಕ್ಕೂ
ದಡ-ದಡ
ದದ್ದು
ದನ್ನು
ದಮ
ದಯ-ಮಾಡಿ
ದಯೆ-ಯಿಂದ
ದರೂ
ದರ್ಶ-ನದ
ದಲ್ಲಿ
ದಲ್ಲಿ-ಟ್ಟು-ಕೊ-ಳ್ಳು-ವುದು
ದಲ್ಲೇ
ದಾಗ
ದಾಡಿ
ದಾಡು-ತ್ತಿ-ರು-ತ್ತದೆ
ದಾರಿ
ದಾರಿ-ಗ-ಳಿವೆ
ದಾರಿ-ಯಿಲ್ಲ
ದಾರಿ-ಯು-ದ್ದಕ್ಕೂ
ದಾರಿ-ಯೆಂ-ದರೆ
ದಿಂದ
ದಿಂಬಿಗೆ
ದಿಕ್ಕಿ-ನತ್ತ
ದಿಕ್ಕು-ಗಳಲ್ಲಿ
ದಿಗಿ-ಲಿ-ನಿಂದ
ದಿಗಿಲು
ದಿಟ-ತಾನೆ
ದಿಢೀ-ರನೆ
ದಿದ್ದರೆ
ದಿನ
ದಿನಂ-ಪ್ರ-ತಿಯೂ
ದಿನಕ್ಕೆ
ದಿನ-ಗಳಲ್ಲಿ
ದಿನ-ಗ-ಳೆಲ್ಲ
ದಿನ-ಚರಿ
ದಿನದ
ದಿನ-ವಂತೂ
ದಿನ-ವನ್ನು
ದಿನ-ವಿಡೀ
ದಿನಾಲು
ದಿನಾಲೂ
ದಿನೇ
ದಿವ್ಯ
ದುಂಟು
ದುಃಖಿ-ಸು-ವ-ವನು
ದುಡಿ-ಯ-ದೆಯೇ
ದುಡಿ-ಸಿದ್ದೇ
ದುರ-ದೃಷ್ಟ
ದುರಾ-ಶೆ-ಯಿಂದ
ದುರ್ಬಲ
ದುರ್ಬ-ಲ-ಗೊಂ-ಡಾಗ
ದುರ್ಬ-ಲ-ರಂತೆ
ದುಷ್ಪ-ರಿ-ಣಾ-ಮ-ಗಳು
ದೂರ
ದೂರದ
ದೂರ-ವಾ-ಗ-ಬೇಕು
ದೂರ-ವಿ-ರ-ಬೇಕು
ದೂರ-ವಿ-ರು-ವಂತೆ
ದೃಗ್ಗೋ-ಚ-ರ-ವಾ-ಗು-ವಂತೆ
ದೃಢ
ದೃಢ-ತೆ-ಯಿ-ರಲಿ
ದೃಢ-ವಾಗಿ
ದೃಢ-ವಿ-ಶ್ವಾಸ
ದೃಢ-ಸಂ-ಕ-ಲ್ಪ-ಮಾಡಿ
ದೃಶ್ಯ
ದೃಶ್ಯ-ಗಳನ್ನು
ದೃಷ್ಟಿ-ಯಿಂದ
ದೆಸೆ-ಯಿಂ-ದಲೇ
ದೇನೋ
ದೇವ-ತೆ-ಯನ್ನು
ದೇವರ
ದೇವ-ರಲ್ಲಿ
ದೇವ-ರಿಗೆ
ದೇವರು
ದೇವಾ
ದೇಶ-ವನ್ನು
ದೈತ್ಯ
ದೈನಂ-ದಿನ
ದೊಂದು
ದೊಡ್ಡ
ದೊಡ್ಡದು
ದೊರ-ಕ-ಬೇ-ಕು-ಅ-ಲ್ಲವೆ
ದೊರ-ಕುವ
ದೊರೆ-ಯುವ
ದೋಷ-ಗಳನ್ನು
ದ್ದನ್ನು
ದ್ದರೆ
ದ್ದರೋ
ಧಗ-ಧ-ಗಿ-ಸುವ
ಧರಿ-ಸುವ
ಧರ್ಮಿ-ಷ್ಟ-ನಾದ
ಧೀರ
ಧೀರರು
ಧುಮುಕಿ
ಧುಮು-ಧು-ಮಿಸಿ
ಧೂಳು
ಧೈರ್ಯ-ದಿಂದ
ಧೈರ್ಯ-ದಿಂ-ದಿ-ದ್ದರೆ
ಧ್ಯಾನ
ಧ್ಯಾನಕ್ಕೆ
ಧ್ಯೇಯ-ವನ್ನು
ಧ್ವಂಸ-ಮಾ-ಡಿ-ಬಿ-ಡ-ಬ-ಲ್ಲವು
ಧ್ವನಿ
ಧ್ವನಿ-ಮಾ-ಲಿ-ನ್ಯ-ದಿಂದ
ಧ್ವನಿ-ವ-ರ್ಧ-ಕ-ಗಳು
ನಂತ-ಲ್ಲ-ದಿ-ದ್ದರೂ
ನಂಬಿಕೆ
ನಂಬಿ-ಕೆ-ಯಿಟ್ಟು
ನಂಬಿ-ಕೆ-ಯಿ-ರ-ಲಿಲ್ಲ
ನಂಬಿ-ಕೆ-ಯಿಲ್ಲ
ನಂಬಿ-ಕೆ-ಯಿ-ಲ್ಲದೆ
ನಂಬು-ವು-ದಕ್ಕೆ
ನಗ-ರ-ಗಳಲ್ಲಿ
ನಗ-ರ-ಗ-ಳ-ಲ್ಲಿ-ರುವ
ನಗ-ರದ
ನಗು-ಮೊ-ಗ-ದಿಂ-ದಿ-ದ್ದರೆ
ನಡ-ವ-ಳಿ-ಕೆ-ಗಳು
ನಡ-ವ-ಳಿ-ಕೆ-ಯನ್ನು
ನಡ-ಸ-ಬೇಕು
ನಡಸಿ
ನಡು-ನ-ಡುವೆ
ನಡೆದ
ನಡೆ-ದಿವೆ
ನಡೆದು
ನಡೆ-ದುಕೊ
ನಡೆ-ದು-ಕೊ-ಳ್ಳ-ಲಾಗು
ನಡೆ-ದು-ಕೊ-ಳ್ಳ-ಲಾ-ರಂ-ಭಿ-ಸು-ತ್ತವೆ
ನಡೆ-ದು-ಕೊ-ಳ್ಳು-ತ್ತಿ-ರು-ವೆಯೋ
ನಡೆ-ಯ-ದಂತೆ
ನಡೆ-ಯ-ಬೇಕು
ನಡೆ-ಯಲು
ನಡೆ-ಯು-ತ್ತಿವೆ
ನಡೆ-ಯುವ
ನಡೆ-ಯುವು
ನಡೆ-ಸ-ಬೇ-ಕಾ-ದದ್ದು
ನಡೆ-ಸಿ-ಕೊಂಡು
ನಡೆ-ಸಿ-ಕೊ-ಡು-ತ್ತಾನೆ
ನಡೆ-ಸುತ್ತ
ನಡೆ-ಸುವ
ನನ-ಗಿನ್ನೂ
ನನಗೆ
ನನ್ನ
ನನ್ನನ್ನು
ನನ್ನೀ
ನನ್ನು
ನನ್ನೊ-ಳ-ಗಿ-ರುವ
ನನ್ಹತ್ರ
ನಪಾ-ಸಾ-ಗು-ವ-ವರು
ನಪಾ-ಸಾ-ದರೆ
ನಪಾಸೇ
ನಮಗೆ
ನಮ್ಮ
ನಮ್ಮಂ-ತೆಯೇ
ನಮ್ಮ-ದಲ್ಲ
ನಮ್ಮ-ದ-ಲ್ಲದ
ನಮ್ಮನ್ನು
ನಮ್ಮಿಂದ
ನಮ್ಮಿಂ-ದಾಗಿ
ನರ-ಕ-ಸ-ದೃಶ
ನರ-ಮಂ-ಡ-ಲ-ವನ್ನು
ನಲ-ವ-ತ್ತೆಂಟು
ನಲ್ಲಿ
ನಳ-ನ-ಳಿ-ಸು-ತ್ತದೆ
ನವ-ರಾ-ತ್ರಿಯ
ನಷ್ಟ-ಗೊ-ಳಿ-ಸು-ವಲ್ಲಿ
ನಷ್ಟ-ವಾ-ಗು-ತ್ತದೆ
ನಷ್ಟ-ವಾ-ಗು-ವಂ-ತಾ-ದರೆ
ನಷ್ಟ-ವಾ-ಯಿತು
ನಷ್ಟ-ವಾ-ಯಿತೆ
ನಷ್ಟ-ವಾ-ಯಿ-ತೆಂದು
ನಷ್ಟ-ವಿಲ್ಲ
ನಸೆ
ನಾ
ನಾಗ-ರಿ-ಕ-ತೆಗೆ
ನಾಗು-ತ್ತಾನೆ
ನಾಗು-ತ್ತಾ-ನೆ-ಎಂ-ದರೆ
ನಾಜೂ-ಕಿ-ನದು
ನಾಟಕ
ನಾಟಿ-ಕೊಂ-ಡಿ-ರು-ವುದನ್ನು
ನಾಟಿ-ರು-ತ್ತದೆ
ನಾಟು-ತ್ತದೆ
ನಾಡಿಯ
ನಾನಾ-ರೀ-ತಿ-ಗಳಿಂದ
ನಾನೀಗ
ನಾನು
ನಾನೆಷ್ಟೇ
ನಾನೇಕೆ
ನಾನ್ನೂರು
ನಾಮ-ರೂ-ಪ-ಗಳಿಂದ
ನಾಲಗೆ
ನಾಳಿನ
ನಾಳೆಯ
ನಾವು
ನಾವೇ
ನಿಂತಿ-ದ್ದಾಗ
ನಿಂತಿ-ರುವ
ನಿಂತಿ-ರು-ವ-ವರು
ನಿಂತಿ-ರು-ವಾಗ
ನಿಂದಲೇ
ನಿಂದಿ-ಸಲೂ
ನಿಗ-ದಿತ
ನಿಗೇ
ನಿಗ್ರ-ಹಿಸಿ
ನಿಗ್ರ-ಹಿ-ಸಿದ
ನಿಗ್ರ-ಹಿ-ಸಿ-ಬಿ-ಡ-ಬ-ಹುದು
ನಿಗ್ರ-ಹಿ-ಸುವ
ನಿಗ್ರ-ಹಿ-ಸು-ವುದು
ನಿಜ
ನಿಜಕ್ಕೂ
ನಿಜ-ವಾ-ದರೂ
ನಿಜವೇ
ನಿತ್ಯೋ-ತ್ಸಾಹ
ನಿತ್ರಾಣ
ನಿದ್ರಾ-ವ-ಸ್ಥೆ-ಯ-ಲ್ಲಿ-ದ್ದರೂ
ನಿದ್ರೆ
ನಿದ್ರೆಯ
ನಿನ-ಗದು
ನಿನ-ಗಾಗಿ
ನಿನ-ಗಾರು
ನಿನಗೆ
ನಿನ್ನ
ನಿನ್ನದು
ನಿನ್ನದೇ
ನಿನ್ನನ್ನು
ನಿನ್ನಲ್ಲಿ
ನಿನ್ನಿ-ಚ್ಛೆ-ಯಂತೆ
ನಿನ್ನೆ-ಯ-ದಲ್ಲ
ನಿನ್ನೆಲ್ಲ
ನಿನ್ನೊ-ಳ-ಗಿ-ನಿಂ-ದಲೇ
ನಿನ್ನೊ-ಳ-ಗಿ-ರುವ
ನಿಬಂ-ಧನೆ
ನಿಮಗೆ
ನಿಮ-ನಿ-ಮಗೆ
ನಿಮಿಷ
ನಿಮಿ-ಷಕ್ಕೆ
ನಿಮಿ-ಷ-ಗ-ಳಾ-ದರೂ
ನಿಮಿ-ಷ-ದಲ್ಲಿ
ನಿಮಿ-ಷವೋ
ನಿಮ್ಮ
ನಿಮ್ಮೆ-ಲ್ಲರ
ನಿಮ್ಮೊ-ಳಗೆ
ನಿಯಂ-ತ್ರಿ-ಸು-ವು-ದೆಂ-ದರೆ
ನಿಯ-ತ್ತಾಗಿ
ನಿಯಮ
ನಿಯ-ಮ-ಗಳ
ನಿಯ-ಮ-ವ-ನ್ನಿ-ಟ್ಟು-ಕೊಂ-ಡರೆ
ನಿರ-ತ-ರಾ-ಗಿ-ರೋಣ
ನಿರ-ತ-ರಾ-ದ-ವರ
ನಿರ-ರ್ಥಕ
ನಿರೀ-ಕ್ಷಿ-ಸ-ಲುಂಟೆ
ನಿರೀ-ಕ್ಷಿ-ಸು-ತ್ತಿ-ರು-ವಾಗ
ನಿರೀಕ್ಷೆ
ನಿರ್ದಿಷ್ಟ
ನಿರ್ಮಾ-ಣ-ಮಾ-ಡಿ-ದ-ವನು
ನಿರ್ಮಿ-ಸಿ-ಕೊ-ಡು-ವುದು
ನಿರ್ಮಿ-ಸು-ತ್ತಿ-ದ್ದೇವೆ
ನಿಲು-ಕ-ಬಲ್ಲ
ನಿಲ್ದಾಣ
ನಿಲ್ದಾ-ಣ-ದಲ್ಲಿ
ನಿಲ್ಲು-ವಂತೆ
ನಿವಾ-ರಿ-ಸಿ-ಕೊ-ಳ್ಳ-ಬೇ-ಕಾ-ದು-ದುದು
ನಿಶ್ಚಯ
ನಿಶ್ಚ-ಯ-ವಾ-ಗಿಯೂ
ನಿಶ್ಚಿತ
ನಿಷ್ಟ್ರ-ಯೋ-ಜಕ
ನಿಸ-ರ್ಗ-ಸೌಂ-ದ-ರ್ಯ-ವನ್ನು
ನೀ
ನೀಡ-ಲಿ-ಚ್ಛಿ-ಸು-ತ್ತೇನೆ
ನೀಡಿದ
ನೀಡಿ-ರುವ
ನೀಡು
ನೀಡು-ತ್ತಿ-ದ್ದೇವೆ
ನೀನಾ-ಗುವೆ
ನೀನಿ-ದ್ದರೆ
ನೀನೀಗ
ನೀನು
ನೀನೂ
ನೀನೆ-ನ್ನು-ವುದು
ನೀನೇ
ನೀನೊಂದು
ನೀನೊಬ್ಬ
ನೀರನ್ನು
ನೀರನ್ನೇ
ನೀರಿಗೆ
ನೀರಿ-ನಂತೆ
ನೀರು
ನೀರೂ
ನೀವು
ನೀವೇ
ನೂರಾರು
ನೂರು
ನೂರು-ಮಡಿ
ನೃತ್ಯ
ನೆಗೆ-ತಕ್ಕೆ
ನೆಟ್ಟ-ಗಿ-ವೆಯೇ
ನೆಟ್ಟು
ನೆನ-ಪಾಗಿ
ನೆನ-ಪಿಗೆ
ನೆನ-ಪಿಟ್ಟು
ನೆನ-ಪಿ-ಟ್ಟು-ಕೊ-ಳ್ಳಲು
ನೆನ-ಪಿ-ಟ್ಟು-ಕೊ-ಳ್ಳು-ತ್ತೇನೆ
ನೆನ-ಪಿ-ಟ್ಟು-ಕೊ-ಳ್ಳುವ
ನೆನ-ಪಿದೆ
ನೆನ-ಪಿ-ನ-ಲ್ಲಿ-ಟ್ಚು-ಕೊ-ಳ್ಳು-ವುದು
ನೆನ-ಪಿ-ರುವ
ನೆನಪು
ನೆನೆ-ದಾ-ಗ-ಲೆಲ್ಲಾ
ನೆನೆ-ಸಿದ
ನೆಯ-ದಾಗಿ
ನೆರ-ವಾಗು
ನೆರ-ವಾ-ಗು-ತ್ತದೆ
ನೆಲೆ-ಗೊಂಡು
ನೇತಾ-ರರು
ನೇಯ್ಗೆ
ನೇಯ್ಗೆ-ಯ-ವರು
ನೇರ-ವಾಗಿ
ನೇರ-ವಾ-ಗಿ-ರ-ಬೇಕು
ನೈಪುಣ್ಯ
ನೋಟ್ಸು-ಗ-ಳ-ನ್ನು-ಅದೂ
ನೋಡ
ನೋಡ-ದಿ-ರ-ಬೇಕು
ನೋಡ-ಬಾ-ರ-ದ್ದನ್ನು
ನೋಡ-ಬೇಕು
ನೋಡ-ಲೇ-ಬಾ-ರದು
ನೋಡಿ
ನೋಡಿಕೊ
ನೋಡಿ-ಕೊಂ-ಡ-ರಾ-ಯಿತು
ನೋಡಿ-ಕೊಳ್ಳ
ನೋಡಿ-ಕೊ-ಳ್ಳ-ಬೇಕು
ನೋಡಿ-ಕೊ-ಳ್ಳು-ತ್ತಿರ
ನೋಡಿ-ಕೊ-ಳ್ಳು-ವುದು
ನೋಡಿದ
ನೋಡಿ-ದಂತೆ
ನೋಡಿ-ದ-ಪೆ-ಟ್ಟಿಗೆ
ನೋಡಿ-ದರೆ
ನೋಡಿ-ದಾಗ
ನೋಡಿ-ದೆಯಾ
ನೋಡಿದ್ದ
ನೋಡಿ-ರ-ಬ-ಹುದು
ನೋಡು
ನೋಡುವ
ನೋಡು-ವಂ-ತಿಲ್ಲ
ನೋಡುವೆ
ನೋಡೋಣ
ನೋವು
ನ್ನರಿ-ಯದೆ
ನ್ನೆಲ್ಲ
ನ್ಯಾಯ-ವಲ್ಲ
ಪಂಕ್ತಿ-ಗಳು
ಪಂಚೇಂ-ದ್ರಿ-ಯ-ಗಳು
ಪಕ್ಕ-ದಲ್ಲಿ
ಪಕ್ಷ
ಪಟ್ಟಣ
ಪಟ್ಟ-ಣ-ಗಳ
ಪಟ್ಟ-ಣ-ಗಳಲ್ಲಿ
ಪಟ್ಟಿಯ
ಪಟ್ಟಿಯೇ
ಪಠ್ಯ
ಪಠ್ಯ-ಪು-ಸ್ತ-ಕ-ಗಳ
ಪಠ್ಯ-ಪು-ಸ್ತ-ಕ-ಗಳನ್ನು
ಪಠ್ಯ-ಪು-ಸ್ತ-ಕದ
ಪಠ್ಯ-ವಿ-ಷ-ಯ-ಗಳನ್ನು
ಪಠ್ಯ-ವಿ-ಷ-ಯ-ಗಳು
ಪಠ್ಯ-ವಿ-ಷ-ಯದ
ಪಠ್ಯ-ವಿ-ಷ-ಯ-ದಲ್ಲೇ
ಪಠ್ಯ-ವಿ-ಷ-ಯ-ವನ್ನೂ
ಪಡ-ಬೇ-ಕಾ-ಗು-ತ್ತ-ದೆಂ-ಬು-ದನ್ನು
ಪಡ-ಬೇ-ಕಾ-ದುದು
ಪಡ-ಬೇಕು
ಪಡಿಸಿ
ಪಡಿ-ಸಿ-ಕೊಂ-ಡಿ-ದ್ದೇನೆ
ಪಡಿ-ಸಿ-ಕೊ-ಳ್ಳ-ಬ-ಲ್ಲೆ-ಯಾ-ದರೆ
ಪಡಿ-ಸಿ-ಕೊ-ಳ್ಳುವ
ಪಡುತ್ತ
ಪಡು-ತ್ತಿ-ರು-ತ್ತಾರೆ
ಪಡೆ-ದಿ-ರು-ತ್ತಾನೆ
ಪಡೆ-ದಿ-ರುವೆ
ಪಡೆದು
ಪಡೆ-ದು-ಕೊ-ಳ್ಳ-ಬ-ಹುದು
ಪಡೆಯ
ಪಡೆ-ಯ-ಬಲ್ಲ
ಪಡೆ-ಯ-ಬ-ಹು-ದಾದ
ಪಡೆ-ಯ-ಬ-ಹುದು
ಪಡೆ-ಯ-ಬ-ಹು-ದೇನೋ
ಪಡೆ-ಯುವ
ಪಡೆ-ಯು-ವಂ-ತಾ-ಗ-ಬೇಕು
ಪತ್ರ
ಪತ್ರ-ದಲ್ಲಿ
ಪತ್ರ-ವನ್ನು
ಪತ್ರ-ವ-ನ್ನೋ-ದಿ-ದಾಗ
ಪದ-ಗಳನ್ನು
ಪದ-ಪ್ರ-ಯೋಗ
ಪದ-ವಾದ
ಪದ-ವಿದೆ
ಪರಮ
ಪರ-ಮ-ಕ-ರು-ಣಾ-ಮ-ಯ-ನಾ-ದ್ದ-ರಿಂದ
ಪರ-ಮಾತ್ಮ
ಪರ-ವೂ-ರು-ಗ-ಳಿಗೆ
ಪರ-ಸ್ಪ-ರ-ರಲ್ಲಿ
ಪರ-ಸ್ಪ-ರ-ರಿಗೆ
ಪರಿ
ಪರಿ-ಚ-ಯ-ಸ್ಥರು
ಪರಿ-ಣಾ-ಮ-ಕಾರಿ
ಪರಿ-ಣಾ-ಮ-ಕಾ-ರಿ-ಯಾ-ದುದು
ಪರಿ-ಣಾ-ಮ-ವಾಗಿ
ಪರಿ-ಪಾ-ಠ-ವನ್ನೂ
ಪರಿ-ಶುದ್ಧ
ಪರಿ-ಶ್ರಮ
ಪರಿ-ಸ-ರ-ಗಳು
ಪರಿ-ಸ-ರ-ದಿಂದ
ಪರಿ-ಸ್ಥಿತಿ
ಪರಿ-ಸ್ಥಿ-ತಿ-ಗಳು
ಪರಿ-ಸ್ಥಿ-ತಿಯ
ಪರಿ-ಹಾ-ರ-ಗಳನ್ನು
ಪರಿ-ಹಾ-ರ-ಗಳು
ಪರಿ-ಹಾ-ರ-ವೊಂದು
ಪರಿ-ಹಾ-ರೋ-ಪಾ-ಯ-ಗಳನ್ನು
ಪರೀ-ಕ್ಷ-ಕರು
ಪರೀಕ್ಷಾ
ಪರೀಕ್ಷೆ
ಪರೀ-ಕ್ಷೆ-ಗಳೂ
ಪರೀ-ಕ್ಷೆ-ಗಾಗಿ
ಪರೀ-ಕ್ಷೆಗೆ
ಪರೀ-ಕ್ಷೆ-ಯಲ್ಲಿ
ಪಳ-ಗಿ-ಸುವ
ಪವಿ-ತ್ರ-ವಾದ
ಪಶ್ಚಾ-ತ್ತಾಪ
ಪಾಟು
ಪಾಠ
ಪಾಠಕ್ಕೆ
ಪಾಠ-ಗಳ
ಪಾಠ-ಗಳನ್ನು
ಪಾಠ-ಗಳು
ಪಾಠ-ಗಳೂ
ಪಾಠದ
ಪಾಠ-ದಲ್ಲೇ
ಪಾಠ-ಪ-ಟ್ಟಿ-ಗಳ
ಪಾಠ-ಪ-ಟ್ಟಿಯ
ಪಾಠ-ಪ್ರ-ವ-ಚ-ನ-ಗಳು
ಪಾಠ-ವನ್ನು
ಪಾಠ-ವನ್ನೇ
ಪಾಡು
ಪಾತ್ರ
ಪಾತ್ರ-ನಾದ
ಪಾತ್ರೆ
ಪಾರಾ-ಗ-ಬೇಕು
ಪಾರಾ-ಗಲು
ಪಾರಾ-ಯ-ಣ-ವನ್ನೋ
ಪಾಲಿಗೆ
ಪಾಶ್ಚಾತ್ಯ
ಪಾಸಾ-ಗ-ದಿ-ದ್ದರೂ
ಪಾಸಾ-ಗ-ಬೇಕು
ಪಾಸಾ-ಗ-ಲೆ-ತ್ನಿ-ಸು-ತ್ತಾರೆ
ಪಾಸಾ-ಗು-ವ-ವರು
ಪಾಸಾ-ದರೆ
ಪಾಸಾ-ದಾರು
ಪುಟ-ದಲ್ಲೂ
ಪುಟಿ-ಯುವ
ಪುಟ್ಟ
ಪುಣ್ಯ
ಪುನಃ
ಪುನ-ರಾ-ವ-ರ್ತನೆ
ಪುರಾಣ
ಪುರು-ಷ-ರೆ-ನಿ-ಸಿ-ದರು
ಪುರು-ಷೋ-ತ್ತ-ಮಾ-ನಂದ
ಪುರು-ಷೋ-ತ್ತ-ಮಾ-ನಂ-ದರು
ಪುಷಿ-ಗಳ
ಪುಷಿ-ಮು-ನಿ-ಗಳು
ಪುಷ್ಟಿ-ಗೊಂಡು
ಪುಸ್ತಕ
ಪುಸ್ತ-ಕಕ್ಕೂ
ಪುಸ್ತ-ಕ-ಗಳ
ಪುಸ್ತ-ಕ-ಗಳನ್ನೂ
ಪುಸ್ತ-ಕ-ಗ-ಳ-ನ್ನೋ-ದಿಯೋ
ಪುಸ್ತ-ಕ-ದ-ಲ್ಲಿ-ರುವ
ಪುಸ್ತ-ಕ-ದಿಂದ
ಪುಸ್ತ-ಕ-ವ-ನ್ನಿ-ಟ್ಟು-ಕೊಂಡು
ಪುಸ್ತ-ಕ-ವನ್ನು
ಪುಸ್ತಾ-ಕಾದಿ
ಪುಸ್ತಿ-ಕೆ-ಗಳ
ಪೂರ್ವ
ಪೂರ್ವ-ಕ-ವಾಗಿ
ಪೂರ್ವಜ
ಪೂರ್ವ-ತ-ಯಾರಿ
ಪೆನ್ನನ್ನು
ಪೆನ್ನನ್ನೋ
ಪೆನ್ನಿ-ನಿಂ-ದಲೇ
ಪೆನ್ನು
ಪೆನ್ನು-ಪೆ-ನ್ಸಿಲು
ಪೆನ್ನು-ಗ-ಳಿ-ರು-ವುದು
ಪೆನ್ಸಿ-ಲನ್ನೋ
ಪೇಟೆ
ಪೋಲೀ-ತ-ನಕ್ಕೆ
ಪೌಷ್ಟಿಕ
ಪ್ಯಾರಾ
ಪ್ಯಾರಾ-ಗಳನ್ನು
ಪ್ರಕಾ-ಶ-ಕ-ರಿಗೆ
ಪ್ರಕಾ-ಶ-ಕರು
ಪ್ರಕೃತಿ
ಪ್ರಕೃ-ತಿಯ
ಪ್ರಗ-ತಿಗೆ
ಪ್ರಚಂಡ
ಪ್ರಜೆ-ಗಳು
ಪ್ರಣಾಮ
ಪ್ರಣಾ-ಮ-ಗಳು
ಪ್ರತಿ
ಪ್ರತಿ-ಗಳು
ಪ್ರತಿ-ದಿನ
ಪ್ರತಿ-ದಿ-ನವೂ
ಪ್ರತಿ-ಬಂ-ಧ-ಕ-ವೆಂದರೆ
ಪ್ರತಿ-ಯೊಂದು
ಪ್ರತಿ-ಯೊಂದೂ
ಪ್ರತಿ-ಯೊಬ್ಬ
ಪ್ರತಿ-ಯೊ-ಬ್ಬನೂ
ಪ್ರತಿ-ಯೋ-ರ್ವ-ರಿಗೂ
ಪ್ರತಿ-ರೂ-ಪವೇ
ಪ್ರತ್ಯೇ-ಕ-ವಾಗಿ
ಪ್ರಥಮ
ಪ್ರದೇ-ಶದ
ಪ್ರಧಾನ
ಪ್ರಫು-ಲ್ಲವೂ
ಪ್ರಬಂಧ
ಪ್ರಬಲ
ಪ್ರಬ-ಲ-ವಾ-ದ-ವು-ಗ-ಳಾ-ದ್ದ-ರಿಂದ
ಪ್ರಭಾ-ವ-ದಿಂದ
ಪ್ರಭಾ-ವ-ದಿಂ-ದಲೋ
ಪ್ರಭಾ-ವಿ-ತ-ನಾಗಿ
ಪ್ರಮಾಣ
ಪ್ರಮಾ-ಣದ
ಪ್ರಮುಖ
ಪ್ರಯತ್ನ
ಪ್ರಯ-ತ್ನದ
ಪ್ರಯ-ತ್ನ-ದಲ್ಲಿ
ಪ್ರಯ-ತ್ನ-ದಿಂದ
ಪ್ರಯ-ತ್ನ-ಶೀ-ಲ-ನನ್ನು
ಪ್ರಯ-ತ್ನಿಸಿ
ಪ್ರಯಾಣ
ಪ್ರಯಾ-ಣ-ಕಾಲ
ಪ್ರಯಾ-ಣಿ-ಕ-ರಿಗೆ
ಪ್ರಯೋ-ಜನ
ಪ್ರಯೋ-ಜ-ನ-ಗ-ಳಿವೆ
ಪ್ರಯೋ-ಜ-ನ-ವನ್ನು
ಪ್ರಯೋ-ಜ-ನ-ವಿಲ್ಲ
ಪ್ರಲೋ-ಭ-ನೆ-ಆ-ಕ-ರ್ಷಣೆ
ಪ್ರವ-ಚ-ನ-ಗಳನ್ನು
ಪ್ರವ-ಚ-ನ-ಗ-ಳ-ನ್ನೊಮ್ಮೆ
ಪ್ರವ-ಚ-ನ-ಗಳು
ಪ್ರವ-ಚ-ನಾ-ದಿ-ಗಳನ್ನು
ಪ್ರವೇ-ಶ-ವಾ-ಗಲೂ
ಪ್ರವೇ-ಶಿಸಿ
ಪ್ರಶಾಂತ
ಪ್ರಶ್ನ-ಪ-ತ್ರಿ-ಕೆ-ಯ-ಲ್ಲಿ-ರುವ
ಪ್ರಶ್ನೆ
ಪ್ರಶ್ನೆ-ಗಳನ್ನೂ
ಪ್ರಶ್ನೆ-ಗ-ಳಿಗೆ
ಪ್ರಶ್ನೆ-ಗಳು
ಪ್ರಶ್ನೆಗೂ
ಪ್ರಶ್ನೆಗೆ
ಪ್ರಶ್ನೆಯ
ಪ್ರಶ್ನೆ-ಯನ್ನು
ಪ್ರಶ್ನೆ-ಯೇನು
ಪ್ರಶ್ನೆ-ಯೊಂದು
ಪ್ರಸ-ನ್ನ-ನಾ-ಗಿರು
ಪ್ರಾಣಿಗೂ
ಪ್ರಾಥ-ಮಿಕ
ಪ್ರಾಪ್ತಿ-ಮಾಡಿ
ಪ್ರಾಮಾ-ಣಿ-ಕ-ತೆ-ಯಿಂದ
ಪ್ರಾಮುಖ್ಯ
ಪ್ರಾಮು-ಖ್ಯದ್ದು
ಪ್ರಾಮು-ಖ್ಯ-ವಾದ
ಪ್ರಾಮು-ಖ್ಯವೋ
ಪ್ರಾಯ-ಪ್ರ-ಬು-ದ್ಧ-ನಾದ
ಪ್ರಾರಂಭ
ಪ್ರಾರಂ-ಭ-ದಿಂ-ದಲೇ
ಪ್ರಾರಂ-ಭ-ವಾ-ಗ-ಬೇಕು
ಪ್ರಾರಂ-ಭ-ವಾ-ಗಿವೆ
ಪ್ರಾರಂ-ಭಿಸು
ಪ್ರಾರಂ-ಭಿ-ಸು-ತ್ತಾರೆ
ಪ್ರಾರ್ಥನೆ
ಪ್ರಾರ್ಥ-ನೆಗೆ
ಪ್ರಾರ್ಥ-ನೆಯ
ಪ್ರಾರ್ಥ-ನೆ-ಯನ್ನು
ಪ್ರಾರ್ಥ-ನೆ-ಯಲ್ಲ
ಪ್ರಾರ್ಥ-ನೆಯೇ
ಪ್ರಾರ್ಥಿ-ಸ-ಬೇಕು
ಪ್ರಾರ್ಥಿಸಿ
ಪ್ರಾರ್ಥಿ-ಸಿ-ಕೊ-ಳ್ಳ-ಬೇಕು
ಪ್ರಾರ್ಥಿ-ಸಿರಿ
ಪ್ರಾರ್ಥಿ-ಸು-ವುದು
ಪ್ರಾವೀಣ್ಯ
ಪ್ರಿಯ
ಪ್ರೀತಿ
ಪ್ರೀತಿ-ವಿ-ಶ್ವಾಸ
ಪ್ರೀತಿ-ವಿ-ಶ್ವಾ-ಸಕ್ಕೆ
ಪ್ರೀತಿಯ
ಪ್ರೀತಿ-ಯನ್ನು
ಪ್ರೀತಿ-ಯಿಂದ
ಪ್ರೀತಿ-ಯಿ-ಡ-ಬೇಕು
ಪ್ರೀತಿ-ಯಿ-ರು-ವುದೋ
ಪ್ರೀತಿ-ಯಿ-ಲ್ಲ-ವಲ್ಲ
ಪ್ರೀತಿ-ಸು-ತ್ತೀಯೋ
ಪ್ರೋತ್ಸಾ-ಹ-ವನ್ನು
ಫಟಿಂಗ
ಫಲ
ಫಲ-ಕಾ-ರಿ-ಯಾ-ಗು-ವುದು
ಫಲ-ಪ್ರ-ದ-ವಾ-ಗು-ತ್ತದೆ
ಫಲ-ಪ್ರ-ದ-ವಾ-ಗು-ವುದು
ಫಲ-ವನ್ನು
ಫಸಲು
ಫೆಬ್ರು-ವ-ರಿ-ಮಾ-ರ್ಚ್
ಫೋನ್
ಬಂತು
ಬಂತೆಂ-ದರೆ
ಬಂದದ್ದೂ
ಬಂದ-ರಾ-ಯಿತು
ಬಂದರು
ಬಂದರೆ
ಬಂದ-ರೆಂ-ದರೆ
ಬಂದಾಗ
ಬಂದಿ-ದ್ದಾನೆ
ಬಂದಿ-ದ್ದೇನೆ
ಬಂದಿ-ರುವ
ಬಂದು
ಬಂದುವು
ಬಂದೇ
ಬಂಧು-ಗಳು
ಬಗೆ-ಬ-ಗೆಯ
ಬಗೆಯ
ಬಗೆ-ಯನ್ನು
ಬಗೆ-ಯಿಂದ
ಬಗ್ಗೆ
ಬಟ್ಟೆ
ಬಟ್ಟೆ-ಬ-ರೆ-ಯ-ವ-ರೆಗೆ
ಬಡ-ಗಿ-ಗಳಲ್ಲಿ
ಬಡ-ಗಿಗೆ
ಬಡ-ಪಾಯಿ
ಬಡ-ವಿ-ದ್ಯಾ-ರ್ಥಿ-ಗಳು
ಬಡಿದು
ಬಡಿ-ದು-ಕೊ-ಳ್ಳು-ವ-ವರೇ
ಬಡಿ-ದೆ-ಬ್ಬಿ-ಸು-ತ್ತದೆ
ಬದ-ಲಾಗಿ
ಬದಲು
ಬದಿ-ಗೊತ್ತ
ಬಯ-ಲಿ-ಗೆ-ಳೆ-ಯಿತು
ಬಯ-ಸು-ವ-ವರು
ಬರ-ದೆಯೇ
ಬರ-ಬೇ-ಕಾ-ಯಿತು
ಬರ-ಬೇಕು
ಬರ-ಲಾ-ಗ-ಲಿಲ್ಲ
ಬರ-ಲಿಲ್ಲ
ಬರ-ವ-ಣಿಗೆ
ಬರ-ವ-ಣಿ-ಗೆಯ
ಬರ-ವ-ಣಿ-ಗೆ-ಯಲ್ಲಿ
ಬರಿಸಿ
ಬರು-ತ್ತದೆ
ಬರು-ತ್ತವೆ
ಬರು-ತ್ತಿದೆ
ಬರು-ವಂ-ತಿಲ್ಲ
ಬರು-ವು-ದಿಲ್ಲ
ಬರು-ವುದು
ಬರು-ವು-ದುಂಟು
ಬರು-ವುದೂ
ಬರೆ
ಬರೆದ
ಬರೆ-ದರೂ
ಬರೆ-ದಿ-ದ್ದರೆ
ಬರೆ-ದಿ-ದ್ದಾನೆ
ಬರೆ-ದಿ-ರುವ
ಬರೆ-ದಿ-ರುವೆ
ಬರೆದು
ಬರೆ-ದು-ದ-ರಲ್ಲಿ
ಬರೆ-ದು-ದ-ರಿಂದ
ಬರೆ-ದು-ಬಂ-ದಿರು
ಬರೆದೂ
ಬರೆ-ಯ-ಬೇ-ಕಾ-ದರೂ
ಬರೆ-ಯಲು
ಬರೆ-ಯಿಸಿ
ಬರೆ-ಯಿ-ಸಿದ್ದು
ಬರೆಯು
ಬರೆ-ಯುತ್ತ
ಬರೆ-ಯುತ್ತಾ
ಬರೆ-ಯು-ತ್ತೇನೆ
ಬರೆ-ಯುವ
ಬರೆ-ಯು-ವಾಗ
ಬರೆ-ಯು-ವಿ-ಕೆಯೂ
ಬರೆ-ಯು-ವು-ದಕ್ಕೂ
ಬರೆ-ಯು-ವು-ದ-ರಿಂದ
ಬರೆ-ಯು-ವು-ದಲ್ಲ
ಬರೆ-ಯು-ವುದೂ
ಬರೆ-ಯು-ವು-ದೆಂ-ದರೆ
ಬರೆ-ಸು-ತ್ತಿ-ದ್ದರು
ಬಲ
ಬಲ-ಗೊ-ಳಿ-ಸು-ವು-ದ-ಲ್ಲದೆ
ಬಲ-ಗೊ-ಳ್ಳು-ವಂತೆ
ಬಲವೂ
ಬಲವೇ
ಬಲಿ-ಬಿ-ದ್ದ-ವರ
ಬಲಿ-ಬೀ-ಳುವ
ಬಲಿ-ಯಾ-ಗು-ವು-ದುಂಟು
ಬಲಿಷ್ಠ
ಬಲ್ಲ
ಬಳ-ಕೆಯ
ಬಳ-ಸಲು
ಬಳ-ಸಲೇ
ಬಳಸಿ
ಬಳ-ಸಿ-ದಾಗ
ಬಳ-ಸುವ
ಬಳಿ
ಬಳಿಕ
ಬಳಿ-ಯಲ್ಲೇ
ಬಳಿ-ಯಿದ್ದ
ಬಳಿಯೇ
ಬಸ್ಸು-ರೈ-ಲು-ಗಳಲ್ಲಿ
ಬಹಳ
ಬಹ-ಳ-ಮ-ಟ್ಟಿಗೆ
ಬಹಿ-ರಂ-ಗದ
ಬಹು
ಬಹುದು
ಬಹು-ದೂರ
ಬಹು-ದೇನೋ
ಬಹು-ಭಾಗ
ಬಹು-ಮುಖ್ಯ
ಬಹು-ಸಂ-ಖ್ಯೆ-ಯಲ್ಲಿ
ಬಾ
ಬಾಕಿ
ಬಾಕಿ-ಯಿದೆ
ಬಾಚಿ-ಕೊ-ಳ್ಳು-ವುದು
ಬಾಧ-ಕ-ವಿಲ್ಲ
ಬಾಯಿ-ಗಿ-ಟ್ಟು-ಕೊಂಡು
ಬಾಯಿ-ಪಾಠ
ಬಾಯ್ಬಾಯಿ
ಬಾಯ್ಬಿಚ್ಚಿ
ಬಾರ-ದಿ-ರಲಿ
ಬಾರದು
ಬಾರದೆ
ಬಾರಿಸು
ಬಾಲ-ಕ-ನಾ-ಗಿ-ದ್ದಾಗ
ಬಿಗಿ-ಹಿ-ಡಿ-ದು-ಕೊಂಡು
ಬಿಟ್ಟರೆ
ಬಿಟ್ಟಾಗ
ಬಿಟ್ಟಿತು
ಬಿಟ್ಟು
ಬಿಡ-ಬಾ-ರ-ದೆಂಬ
ಬಿಡ-ಬೇಡ
ಬಿಡು-ವಿನ
ಬಿಡು-ವುದೇ
ಬಿದ್ದಾಗ
ಬಿದ್ದು-ಕೊಂ-ಡಂ-ತಿ-ದ್ದರೆ
ಬಿದ್ದೆ
ಬಿರು-ಸಿನ
ಬಿಸಿ
ಬೀಗ
ಬೀದಿಯ
ಬೀರುತ್ತ
ಬೀರು-ವಿ-ನ-ಲ್ಲಿ-ಟ್ಟಿದ್ದೆ
ಬೀಳುವ
ಬೀಳು-ವು-ದುಂಟು
ಬೀಸುವ
ಬುಗ್ಗೆಯೇ
ಬುಟ್ಟಿ
ಬುದ್ಧಿ
ಬುದ್ಧಿಗೆ
ಬುದ್ಧಿ-ಬೇಕು
ಬುದ್ಧಿಯ
ಬುದ್ಧಿ-ಯು-ಪ-ಯೋ-ಗಿಸಿ
ಬುದ್ಧಿ-ವಂತ
ಬುದ್ಧಿ-ವಂ-ತ-ನಾ-ಗಿ-ದ್ದರೂ
ಬುದ್ಧಿ-ವಂ-ತ-ರಾ-ಗಿ-ರು-ವುದನ್ನು
ಬುದ್ಧಿ-ವಂ-ತಿ-ಕೆ-ಯನ್ನು
ಬುದ್ಧಿ-ಶ-ಕ್ತಿಯೇ
ಬೃಹತ್
ಬೆಂಕಿ
ಬೆಂಕಿಗೆ
ಬೆಂಗ-ಳೂ-ರಿ-ನಿಂದ
ಬೆಂಚಿ-ನಲ್ಲಿ
ಬೆಚ್ಚ-ಗಿನ
ಬೆಟ್ಟ-ದಿಂದ
ಬೆಣ್ಣೆ
ಬೆನ್ನ
ಬೆರ-ಗಾ-ಗ-ದಿರ
ಬೆಲೆ
ಬೆಳ-ಕಿ-ನಲ್ಲಿ
ಬೆಳ-ಕು-ವ-ಸ್ತು-ಗಳು
ಬೆಳಗ್ಗೆ
ಬೆಳ-ಗ್ಗೆ-ಸಂಜೆ
ಬೆಳ-ಸಿಕೊ
ಬೆಳ-ಸಿ-ಕೊ-ಳ್ಳ-ಬೇಕು
ಬೆಳ-ಸಿ-ಕೊ-ಳ್ಳಲು
ಬೆಳೆ
ಬೆಳೆದು
ಬೆಳೆ-ಯ-ಬೇ-ಕಾದ
ಬೆಳೆಸಿ
ಬೆಳೆ-ಸಿ-ಕೊಂಡು
ಬೆಳೆ-ಸಿ-ಕೊಂ-ಡೆ-ಯಾ-ದರೆ
ಬೆಳೆ-ಸಿ-ಕೊ-ಳ್ಳ-ಬ-ಹುದು
ಬೆಳೆ-ಸಿ-ಕೊ-ಳ್ಳಲು
ಬೆವರಿ
ಬೇಕಾ
ಬೇಕಾ-ಗ-ಬ-ಹುದು
ಬೇಕಾ-ಗಿದೆ
ಬೇಕಾ-ಗು-ತ್ತದೆ
ಬೇಕಾದ
ಬೇಕಾ-ದ-ಬೇ-ಡ-ವಾದ
ಬೇಕಾ-ದಂತೆ
ಬೇಕಾ-ದರೂ
ಬೇಕಾ-ದರೆ
ಬೇಕಾ-ಯಿತು
ಬೇಕಿಲ್ಲ
ಬೇಕು
ಬೇಕೆಂಬ
ಬೇಕೆಂ-ಬ-ಲ್ಲಿಗೆ
ಬೇಕೆಂ-ಬ-ವರು
ಬೇಕೆ-ನ್ನುವ
ಬೇಕೋ
ಬೇಗ
ಬೇಡ
ಬೇಡವೆ
ಬೇರಾ-ವುದೂ
ಬೇರೆ
ಬೈದದ್ದೂ
ಬೈದು
ಬೊಂಬಾ-ಯಿಗೆ
ಬೋರ್
ಬ್ರಹ್ಮ-ಚ-ರ್ಯ-ಪಾ-ಲ-ನೆಗೆ
ಭಕ್ತ-ರ-ಲ್ಲ-ವಲ್ಲ
ಭಕ್ತಿ
ಭಕ್ತಿ-ಪೂ-ರ್ವಕ
ಭಕ್ತಿ-ಯಿಂದ
ಭಗ-ವಂತ
ಭಗ-ವಂ-ತನ
ಭಗ-ವಂ-ತ-ನಲ್ಲಿ
ಭಗ-ವಂ-ತ-ನಿಗೇ
ಭಗ-ವಂ-ತನು
ಭಗ-ವ-ದ್ಗೀತೆ
ಭಗ-ವನ್
ಭಗ-ವ-ನ್ನಾ-ಮ-ಜಪ
ಭದ್ರ-ವಾ-ಗಿ-ರಲಿ
ಭಯ
ಭಯ-ಕ್ಕೇ-ನಾ-ದರೂ
ಭಯ-ಗ್ರ-ಸ್ತ-ವಾ-ಗುವ
ಭಯದ
ಭಯ-ವೆಂ-ಬು-ದಿಲ್ಲ
ಭಯ-ವೊಂದು
ಭವಿ-ಸು-ವುದು
ಭಾಗ-ವ-ಹಿ-ಸಲು
ಭಾಗ್ಯ
ಭಾಗ್ಯ-ವಿಲ್ಲ
ಭಾನು-ವಾರ
ಭಾನು-ವಾ-ರವೇ
ಭಾರ-ವಾ-ಗು-ವು-ದಿಲ್ಲ
ಭಾವ
ಭಾವ-ದಿಂದ
ಭಾವ-ನಾ-ಶ-ಕ್ತಿ-ಯಿಂದ
ಭಾವನೆ
ಭಾವ-ನೆ-ಗಳು
ಭಾವ-ನೆ-ಗಳೇ
ಭಾವ-ನೆಯ
ಭಾವ-ನೆ-ಯನ್ನು
ಭಾವ-ನೆ-ಯಿಂದ
ಭಾವ-ನೆಯೇ
ಭಾವ-ಭಂ-ಗಿ-ಗ-ಗಳಲ್ಲಿ
ಭಾವ-ಲೋ-ಕ-ದಲ್ಲಿ
ಭಾವ-ವೇನೆಂದರೆ
ಭಾವಿ-ಸ-ಬ-ಹುದು
ಭಾವಿ-ಸ-ಬೇಕು
ಭಾವಿಸಿ
ಭಾವಿ-ಸಿರಿ
ಭಾವಿ-ಸುತ್ತ
ಭಾವಿ-ಸು-ತ್ತಲೇ
ಭಾವಿ-ಸು-ತ್ತಿ-ರಲಿ
ಭಾವಿ-ಸು-ತ್ತೇ-ವೆಯೋ
ಭಾವಿ-ಸು-ವುದನ್ನು
ಭಾಷಣ
ಭಾಷಾ-ಜ್ಞಾನ
ಭಾಷೆ-ಗಳ
ಭಾಷೆ-ಯಲ್ಲಿ
ಭಾಷೆ-ಯಲ್ಲೂ
ಭೀಕರ
ಭೀತಿ
ಭೀತಿ-ಯಿಂದ
ಭೀತಿ-ಯೆಂಬು
ಭೂತ-ಕ-ನ್ನ-ಡಿಯ
ಭೂಪರೂ
ಭ್ರಷ್ಟ-ವಾ-ಗು-ತ್ತದೆ
ಭ್ರಾಂತಿ
ಮಂಗ-ನಂ-ತಾ-ಗು-ವುದು
ಮಂಡೆ-ಗೆಟ್ಟು
ಮಂತ್ರಿ-ಗಳೋ
ಮಂದ-ವಾಗಿ
ಮಂದ-ಹಾಸ
ಮಂದಿ
ಮಂದಿಗೆ
ಮಕ್ಕ-ಳಂತೆ
ಮಕ್ಕ-ಳಲ್ಲಿ
ಮಕ್ಕ-ಳಿಗೆ
ಮಕ್ಕಳು
ಮಗ್ನ-ರಾ-ಗಿ-ರು-ವುದು
ಮಜ-ವನ್ನು
ಮಟ್ಟದ
ಮಟ್ಟ-ದ್ದಾ-ಗಿ-ರು-ವಂತೆ
ಮಟ್ಟಿಗೆ
ಮಣಿ-ದಾಗ
ಮತ್ತಷ್ಟು
ಮತ್ತು
ಮತ್ತೆ
ಮತ್ತೊಮ್ಮೆ
ಮದ-ಭ-ರಿತ
ಮಧುರ
ಮನ
ಮನದ
ಮನ-ದ-ಟ್ಟಾ-ಗು-ತ್ತವೆ
ಮನ-ದಲ್ಲಿ
ಮನನ
ಮನ-ಬಂ-ದಂತೆ
ಮನ-ವ-ರಿಕೆ
ಮನ-ಶ್ಶಕ್ತಿ
ಮನ-ಶ್ಶಕ್ತಿಗೆ
ಮನ-ಶ್ಶಕ್ತಿ-ಯನ್ನು
ಮನ-ಶ್ಶಕ್ತಿಯು
ಮನ-ಸೆ-ಳೆ-ಯ-ದಂತೆ
ಮನಸ್ಕ
ಮನ-ಸ್ಕ-ನಾಗಿ
ಮನ-ಸ್ಥಿತಿ
ಮನ-ಸ್ಥಿ-ತಿ-ಯನ್ನು
ಮನ-ಸ್ಸನ್ನು
ಮನ-ಸ್ಸನ್ನೂ
ಮನ-ಸ್ಸಿಗೆ
ಮನ-ಸ್ಸಿ-ದ್ದರೆ
ಮನ-ಸ್ಸಿನ
ಮನ-ಸ್ಸಿ-ನಲ್ಲಿ
ಮನ-ಸ್ಸಿ-ನಿಂದ
ಮನ-ಸ್ಸಿ-ನೊಂ-ದಿಗೆ
ಮನ-ಸ್ಸಿ-ನೊ-ಳಗೆ
ಮನ-ಸ್ಸಿ-ರು-ತ್ತದೆ
ಮನ-ಸ್ಸಿಲ್ಲ
ಮನಸ್ಸು
ಮನ-ಸ್ಸು-ಗಳ
ಮನ-ಸ್ಸು-ಗ-ಳೆ-ರಡು
ಮನ-ಸ್ಸು-ಳ್ಳ-ವ-ರಿಗೆ
ಮನಸ್ಸೂ
ಮನಸ್ಸೇ
ಮನುಷ್ಯ
ಮನು-ಷ್ಯನ
ಮನು-ಷ್ಯನೇ
ಮನೆ
ಮನೆಗೆ
ಮನೆ-ಮಂದಿ
ಮನೆ-ಮಂ-ದಿಗೆ
ಮನೆ-ಯಲ್ಲೇ
ಮನೇ-ಲಿದೆ
ಮನೋ-ಭಾ-ವ-ದ-ವ-ರಾ-ದರೆ
ಮನೋ-ವೃ-ತ್ತಿ-ಯನ್ನು
ಮರಳಿ
ಮರ-ವಿಗೆ
ಮರವು
ಮರವೆ
ಮರ-ವೆ-ಯಿಂ-ದಾಗಿ
ಮರು-ಳಾ-ಗಿಯೋ
ಮರೆ-ತಿ-ದ್ದರೆ
ಮರೆತು
ಮರೆ-ತು-ಬಿಟ್ಟೆ
ಮರೆ-ತು-ಹೋ-ಗಿ-ಬಿ-ಡು-ತ್ತದೆ
ಮರೆ-ತೇ-ಬಿಟ್ಟೆ
ಮರೆ-ಯ-ಬಾ-ರದು
ಮರೆ-ಯುತ್ತ
ಮರೆ-ಯು-ವ-ವರ
ಮರೆ-ಯು-ವ-ವ-ರಿ-ದ್ದಾರೆ
ಮರೆ-ಯು-ವ-ವರೂ
ಮರ್ಕ-ಟ-ನಂತೆ
ಮರ್ಮ
ಮಲಗಿ
ಮಲ-ಗಿ-ಕೊ-ಳ್ಳುವ
ಮಲ-ಗುವ
ಮಲ-ಗು-ವ-ವರೆ-ಗಿನ
ಮಹ-ತ್ಕಾ-ರ್ಯ-ಗಳನ್ನು
ಮಹ-ತ್ಕಾ-ರ್ಯ-ಗ-ಳನ್ನೇ
ಮಹ-ತ್ಕಾ-ರ್ಯ-ವ-ನ್ನೇ-ನಾ-ದರೂ
ಮಹತ್ವ
ಮಹ-ತ್ವದ
ಮಹ-ತ್ವ-ಪೂರ್ಣ
ಮಹ-ತ್ವ-ಪೂ-ರ್ಣ-ವಾ-ದದ್ದು
ಮಹ-ತ್ವ-ವನ್ನು
ಮಹ-ದಾ-ಕಾಂಕ್ಷೆ
ಮಹ-ದಾ-ಕಾಂ-ಕ್ಷೆ-ಯ-ನ್ನಿ-ಟ್ಟು-ಕೊ-ಳ್ಳು-ವುದೇ
ಮಹಾ
ಮಹಾ-ಕಾ-ರ್ಯವೂ
ಮಹಾನ್
ಮಹಾ-ಪು-ರು-ಷ-ರನ್ನು
ಮಹಾ-ಪು-ರು-ಷ-ರಾ-ದದ್ದು
ಮಹಾ-ಮಾ-ರಿ-ಗಳು
ಮಹಿಮೆ
ಮಹಿ-ಮೆ-ಯನ್ನು
ಮಹೋ-ನ್ನತ
ಮಾಡ-ದಿ-ದ್ದರೆ
ಮಾಡ-ದಿರ
ಮಾಡದೆ
ಮಾಡ-ಬ-ಹುದು
ಮಾಡ-ಬಾ-ರ-ದ್ದನ್ನು
ಮಾಡ-ಬೇ-ಕಾ-ಗಿದೆ
ಮಾಡ-ಬೇ-ಕಾದ
ಮಾಡ-ಬೇ-ಕಾ-ದರೂ
ಮಾಡ-ಬೇಕು
ಮಾಡ-ಬೇ-ಕೆಂದು
ಮಾಡ-ಬೇ-ಕೆಂ-ದು-ಕೊಂ-ಡಿ-ರುವ
ಮಾಡ-ಬೇ-ಕೆಂಬು
ಮಾಡ-ಬೇ-ಕೆಂ-ಬು-ದ-ರಲ್ಲಿ
ಮಾಡ-ಬೇಡ
ಮಾಡ-ಲಾಗಿದೆ
ಮಾಡಲಿ
ಮಾಡಲು
ಮಾಡಲೇ
ಮಾಡ-ಲೇ-ಬೇಕು
ಮಾಡಿ
ಮಾಡಿಕೊ
ಮಾಡಿ-ಕೊಂ-ಡರೆ
ಮಾಡಿ-ಕೊಂ-ಡ-ವ-ರೆಷ್ಟೋ
ಮಾಡಿ-ಕೊಂ-ಡಿ-ರುವ
ಮಾಡಿ-ಕೊಂಡು
ಮಾಡಿ-ಕೊ-ಟ್ಟಾಗ
ಮಾಡಿ-ಕೊ-ಳ್ಳಲು
ಮಾಡಿ-ಕೊಳ್ಳು
ಮಾಡಿ-ಕೊ-ಳ್ಳು-ವುದು
ಮಾಡಿ-ಟ್ಟು-ಕೊಂ-ಡಿ-ರ-ಬೇ-ಕಾ-ಗು-ತ್ತದೆ
ಮಾಡಿದ
ಮಾಡಿ-ದಂ-ತಾ-ಗು-ವು-ದಿಲ್ಲ
ಮಾಡಿ-ದರೂ
ಮಾಡಿ-ದರೆ
ಮಾಡಿ-ದ-ವ-ರಲ್ಲ
ಮಾಡಿ-ದಾಗ
ಮಾಡಿ-ದ್ದನ್ನು
ಮಾಡಿ-ನೋ-ಡಲಿ
ಮಾಡಿ-ಬಿ-ಡು-ತ್ತಾರೆ
ಮಾಡಿಯೇ
ಮಾಡಿ-ಯೇ-ತೀ-ರಿ-ದರು
ಮಾಡಿ-ಯೇ-ತೀ-ರು-ತ್ತೇನೆ
ಮಾಡಿ-ರ-ಬೇ-ಕಾ-ಗು-ತ್ತದೆ
ಮಾಡಿ-ರು-ತ್ತಾರೆ
ಮಾಡಿ-ರು-ವು-ದಿಲ್ಲ
ಮಾಡಿ-ಸಲು
ಮಾಡಿ-ಸು-ತ್ತಾರೆ
ಮಾಡು
ಮಾಡುತ್ತ
ಮಾಡು-ತ್ತದೆ
ಮಾಡುತ್ತಾ
ಮಾಡು-ತ್ತಿ-ದ್ದರೆ
ಮಾಡು-ತ್ತಿ-ರ-ಬೇಕು
ಮಾಡು-ತ್ತಿ-ರು-ತ್ತದೆ
ಮಾಡು-ತ್ತೇನೆ
ಮಾಡುವ
ಮಾಡು-ವ-ವ-ನಿಗೇ
ಮಾಡು-ವ-ವನು
ಮಾಡು-ವ-ವರು
ಮಾಡು-ವಾಗ
ಮಾಡು-ವು-ದಕ್ಕೂ
ಮಾಡು-ವು-ದಕ್ಕೆ
ಮಾಡು-ವುದನ್ನು
ಮಾಡು-ವು-ದರ
ಮಾಡು-ವು-ದ-ರಿಂದ
ಮಾಡು-ವು-ದಿಲ್ಲ
ಮಾಡು-ವುದು
ಮಾಡು-ವು-ದು-ಇ-ದನ್ನು
ಮಾಡು-ವುದೂ
ಮಾತ
ಮಾತ-ನ್ನಾ-ಡು-ತ್ತಿ-ದ್ದೆವೋ
ಮಾತನ್ನು
ಮಾತಿನ
ಮಾತಿ-ನಲ್ಲಿ
ಮಾತು
ಮಾತು-ಕತೆ
ಮಾತು-ಕ-ತೆ-ಗಳಲ್ಲಿ
ಮಾತು-ಗಳು
ಮಾತ್ರ
ಮಾತ್ರಕ್ಕೆ
ಮಾತ್ರವೇ
ಮಾದರಿ
ಮಾನವ
ಮಾನ-ವ-ರಾದ
ಮಾನ್ಯ
ಮಾರಕ
ಮಾರ್ಗ-ದ-ರ್ಶ-ನ-ವಿ-ರು-ತ್ತದೆ
ಮಾರ್ಗ-ದಲ್ಲಿ
ಮಾರ್ಗ-ವಿದೆ
ಮಾಸ್ಟರ್
ಮಿಂಚಿ
ಮಿಂಚಿ-ನಂತೆ
ಮಿಗಿ-ಸ-ಬ-ಹುದು
ಮಿಗಿ-ಸ-ಬೇಕು
ಮಿಗಿಸಿ
ಮಿಗಿ-ಸಿದ
ಮಿಗುವ
ಮಿತ-ವಾಗಿ
ಮಿತ್ರ-ರಿಗೂ
ಮಿದು-ಳಿಗೂ
ಮಿದು-ಳಿಗೆ
ಮಿದುಳು
ಮುಂಚಿತ
ಮುಂದಾ-ಗು-ತ್ತದೆ
ಮುಂದಾ-ಗು-ತ್ತಿಲ್ಲ
ಮುಂದಿ-ಟ್ಟಿ-ದ್ದರು
ಮುಂದಿಟ್ಟು
ಮುಂದಿ-ಟ್ಟು-ಕೊಂಡು
ಮುಂದಿ-ಡು-ತ್ತೇನೆ
ಮುಂದಿನ
ಮುಂದು-ವ-ರಿದ
ಮುಂದು-ವ-ರಿ-ದರೂ
ಮುಂದು-ವ-ರಿದು
ಮುಂದು-ವ-ರಿ-ಯು-ತ್ತಿ-ರು-ತ್ತಾರೆ
ಮುಂದೆ
ಮುಕ್ಕಾಲು
ಮುಖ
ಮುಖ-ವಿಟ್ಟು
ಮುಖ್ಯ
ಮುಖ್ಯ-ವಾಗಿ
ಮುಖ್ಯ-ವಾಗು
ಮುಖ್ಯ-ವಾದ
ಮುಖ್ಯವೋ
ಮುಖ್ಯಾಂಶ
ಮುಖ್ಯಾಂ-ಶ-ಗಳು
ಮುಖ್ಯಾಂ-ಶ-ವನ್ನು
ಮುಗಿ-ದಿಲ್ಲ
ಮುಗಿದು
ಮುಗಿ-ಯು-ವ-ವ-ರೆಗೂ
ಮುಗಿ-ಸ-ಬೇ-ಕಾಗು
ಮುಗಿ-ಸಲು
ಮುಗಿ-ಸ-ಲೆ-ತ್ನಿಸಿ
ಮುಗಿಸಿ
ಮುಗಿ-ಸಿ-ಕೊಂಡು
ಮುಗಿ-ಸಿದ
ಮುಗಿ-ಸಿ-ದಾಗ
ಮುಗಿ-ಸು-ತ್ತೇನೆ
ಮುಗಿ-ಸು-ವು-ದರ
ಮುಚ್ಚಿ-ಕೊಂಡು
ಮುಚ್ಚಿ-ಕೊ-ಳ್ಳು-ವುದು
ಮುಚ್ಚಿಟ್ಟು
ಮುಟ್ಟಿ-ಸದೆ
ಮುಟ್ಟು-ತ್ತದೆ
ಮುದು-ಕ-ನಾದ
ಮುದ್ದು
ಮುನ್ನ
ಮುನ್ನ-ಡೆಸು
ಮುನ್ನುಡಿ
ಮುಳು-ಗ-ಬೇಕು
ಮುಳುಗಿ
ಮುಳು-ಗಿ-ರುವ
ಮೂಗು
ಮೂಡಿ-ದಂ-ತಿದೆ
ಮೂಡಿ-ಬ-ರು-ತ್ತದೆ
ಮೂಡಿಸ
ಮೂಡಿ-ಸುವ
ಮೂರನೇ
ಮೂರು
ಮೂರ್ನಾಲ್ಕು
ಮೂಲ
ಮೂಲಕ
ಮೂಲ-ಕವೇ
ಮೂಲೆ-ಯಲ್ಲಿ
ಮೂಲೆ-ಯ-ಲ್ಲಿ-ರುವ
ಮೃದು-ವಾಗಿ
ಮೃದು-ವಾ-ಗು-ತ್ತಾರೆ
ಮೆತ್ತ-ನೆಯ
ಮೆಲಕು
ಮೇಜಿನ
ಮೇಜು
ಮೇಜು-ಕುರ್ಚಿ
ಮೇಜು-ಕು-ರ್ಚಿಯ
ಮೇಲಿ-ರುವ
ಮೇಲೂ
ಮೇಲೆ
ಮೇಲೆಯೇ
ಮೈ
ಮೈ-ಮ-ನ-ಸ್ಸು-ಗಳನ್ನು
ಮೈಗೂ-ಡಿ-ಸಿ-ಕೊಂ-ಡಿ-ತೆಂ-ದರೆ
ಮೈಗೆ
ಮೈತಿ-ಳಿದು
ಮೈಮ-ರೆ-ತರೆ
ಮೈಮ-ರೆತು
ಮೈಸೂರು
ಮೊಟ್ಟ
ಮೊಟ್ಟ-ಮೊ-ದ-ಲ-ನೆ-ಯ-ದಾಗಿ
ಮೊದ-ಮೊ-ದಲು
ಮೊದಲ
ಮೊದ-ಲ-ನೆ-ಯ-ದಾಗಿ
ಮೊದ-ಲಾದ
ಮೊದಲು
ಮೊದಲೇ
ಮೊನೆ
ಮೋಹ-ದಿಂದ
ಮೌಲ್ಯ-ಮ-ಹತ್ತ್ವ
ಮೌಲ್ಯ-ವನ್ನು
ಯತ್ನಿ-ಸಿ-ದರೆ
ಯತ್ನಿಸು
ಯತ್ನಿ-ಸುವ
ಯದ್ಭಾವಂ
ಯದ್ವಾ-ತದ್ವಾ
ಯನ
ಯನಕ್ಕೆ
ಯನ್ನು
ಯನ್ನೂ
ಯರು
ಯಲು
ಯಲ್ಲಿ
ಯಲ್ಲಿ-ಡು-ವುದು
ಯಲ್ಲೇ
ಯವ-ರಲ್ಲಿ
ಯಶಸ್ವಿ
ಯಶ-ಸ್ವಿ-ಯಾ-ಗಲು
ಯಶ-ಸ್ವಿ-ಯಾ-ಗು-ತ್ತದೆ
ಯಶ-ಸ್ವಿ-ಯಾ-ಗು-ವು-ದ-ಕ್ಕಾಗಿ
ಯಶ-ಸ್ಸಿಗೆ
ಯಶ-ಸ್ಸಿನ
ಯಶಸ್ಸು
ಯಾಕಿನ್ನೂ
ಯಾಗಿ
ಯಾಗಿ-ಟ್ಟು-ಕೊಂ-ಡಿ-ರು-ವು-ದ-ರಿಂದ
ಯಾಗು-ತ್ತದೆ
ಯಾಗು-ವು-ದಕ್ಕೆ
ಯಾದ
ಯಾದಂ-ತೆ-ನಿ-ಸು-ವುದು
ಯಾದ-ವ-ಗಿರಿ
ಯಾರ
ಯಾರಾ-ದರೂ
ಯಾರಿಂ-ದಲೋ
ಯಾರಿ-ಗಾ-ದರೂ
ಯಾರಿಗೆ
ಯಾರು
ಯಾರೂ
ಯಾವ
ಯಾವದೋ
ಯಾವ-ಯಾವ
ಯಾವಾಗ
ಯಾವಾ-ಗಲೂ
ಯಾವುದನ್ನು
ಯಾವು-ದನ್ನೇ
ಯಾವು-ದರ
ಯಾವು-ದಾದ
ಯಾವು-ದಾ-ದರೂ
ಯಾವು-ದಿದೆ
ಯಾವುದು
ಯಾವುದೇ
ಯಾವುದೋ
ಯಿಂದ
ಯಿದೆ-ಯೆಂಬ
ಯಿಲ್ಲ-ದಿ-ರು-ವುದು
ಯುಂಟಾ-ದಾಗ
ಯುವ
ಯುವ-ಕ-ನಾ-ಗಿ-ರು-ವಾ-ಗ-ಲ-ಲ್ಲದೆ
ಯುವ-ಕರ
ಯುವ-ಕ-ರಿಗೆ
ಯುವ-ಕರು
ಯುವ-ಜ-ನರ
ಯುವ-ಜ-ನ-ರಿಗೆ
ಯುವ-ಜ-ನರೇ
ಯುವ-ವರೂ
ಯುವುದನ್ನು
ಯೆಂಬ
ಯೆಂಬುದು
ಯೊಂದು
ಯೊಬ್ಬನ
ಯೋಗ-ಪ-ಡಿ-ಸಿ-ಕೊಂಡು
ಯೋಗ-ರ-ಹ-ಸ್ಯ-ಗ-ಳನ್ನೇ
ಯೋಗಾ-ಸ-ನ-ಗಳನ್ನು
ಯೋಗಾ-ಸ-ನ-ಗಳು
ಯೋಗಿ-ಗ-ಳಾ-ಗಲು
ಯೋಗಿ-ಗ-ಳಿಗೆ
ಯೋಗಿ-ಗಳು
ಯೋಗಿಗೆ
ಯೋಗಿ-ಯಂ-ತೆಯೇ
ಯೋಗಿಸಿ
ಯೋಗ್ಯ-ತಾ-ಪ-ತ್ರ-ವನ್ನು
ಯೋಗ್ಯ-ತೆ-ಯನ್ನು
ಯೋಚನೆ
ಯೋಚಿ-ಸದೆ
ಯೋಚಿ-ಸ-ಬ-ಲ್ಲ-ವನು
ಯೋಚಿಸಿ
ಯೋಜ-ನೆ-ಗಳನ್ನು
ರಕ್ತ
ರಕ್ಷಿ-ಸಲು
ರಕ್ಷಿ-ಸು-ವ-ವರು
ರಜಾ
ರಜಾ-ದಿನ
ರಜಾ-ದಿ-ನ-ಗಳ
ರಜಾ-ದಿ-ನ-ಗಳಲ್ಲಿ
ರಜೆಯ
ರಲ್ಲೇ
ರಹಸ್ಯ
ರಹ-ಸ್ಯ-ವನ್ನು
ರಹ-ಸ್ಯ-ವಿದೆ
ರಹ-ಸ್ಯ-ವಿ-ರು-ವುದು
ರಾಜ-ಧಾ-ನಿ-ಗಳಲ್ಲಿ
ರಾಜ್ಯೋ-ತ್ಸವ
ರಾತ್ರಿ
ರಾತ್ರಿ-ಯಲ್ಲಿ
ರಾದರೆ
ರಾಮ
ರಾರ್ಥ-ವನ್ನು
ರಿಂದ
ರಿಗೇ
ರಿವಿ-ಷನ್
ರೀತಿ
ರೀತಿ-ಯಲ್ಲಿ
ರೀತಿ-ಯಾಗಿ
ರುಚಿ
ರುತ್ತವೆ
ರೂಢ-ಮೂ-ಲ-ವಾ-ಗಿರು
ರೂಢಿ-ಸಿ-ಕೊ-ಳ್ಳ-ಬೇ-ಕೆಂ-ದಿ-ದ್ದರೆ
ರೂಪ-ದಲ್ಲಿ
ರೆನ್ನಿ
ರೆಲ್ಲರೂ
ರೇನು
ರೇನ್ರಿ
ರೈಲಿ-ನಲ್ಲಿ
ರೈಲು
ರೈಲು-ತಂ-ಬಿಗೆ
ರೊಂದು
ಲಂಕೆ-ಯನ್ನು
ಲಂಚ
ಲಕ್ಷ-ಣ-ವನ್ನು
ಲಕ್ಷ-ಣವೇ
ಲಕ್ಷಾಂ-ತರ
ಲಕ್ಷ್ಮಿ
ಲಘು-ಪ್ರ-ಬಂ-ಧ-ಗಳಲ್ಲಿ
ಲಭಿ-ಸಿದೆ
ಲವ-ಲ-ವಿ-ಕೆ-ಗೊ-ಳ್ಳು-ತ್ತದೆ
ಲವ-ಲ-ವಿ-ಕೆ-ಯನ್ನು
ಲಾಗದು
ಲಾಗಿದೆ
ಲಾಗುವ
ಲಾಭ-ದಾ-ಯಕ
ಲಾಭ-ವಿದೆ
ಲಾರೆ
ಲಾಳಿ-ಗಳ
ಲೀನ-ವಾ-ಯಿತು
ಲೀಲಾ-ಜಾ-ಲ-ವಾಗಿ
ಲೆಕ್ಕ
ಲೆಟ್
ಲೇಖ-ನ-ಗಳನ್ನು
ಲೇಖ-ನ-ಗಳಲ್ಲಿ
ಲೇಖ-ನ-ದಲ್ಲಿ
ಲೇಖ-ನ-ವಾದ
ಲೇಖ-ನವು
ಲೇಖ-ನ-ಸಾ-ಮರ್ಥ್ಯ
ಲೋಕ-ರೂಢಿ
ಲೋಟ
ಲೋಪ
ಲೋಪ-ದೋ-ಷ-ಗಳು
ಲೋಪ-ದೋ-ಷ-ಗಳೂ
ಲೋಲು-ಪ-ತೆ-ಯಿಂದ
ಲೌಡ್ಸ್ಪೀ-ಕರ್
ಲ್ಲಿಟ್ಟಿ-ರು-ವುದು
ವಂತರು
ವಂತಾ-ಗು-ವುದೂ
ವಂತಿ-ರಲಿ
ವಂತಿಲ್ಲ
ವಂತೆ
ವಂಶ-ಪಾ-ರಂ-ಪ-ರ್ಯ-ವಾಗಿ
ವನ್ನ-ರಿ-ತ-ವರು
ವನ್ನ-ರಿತು
ವನ್ನು
ವನ್ನೋ
ವರ
ವರೆಗೆ
ವರೇ
ವರ್ಜಿ-ಸ-ಬೇಕು
ವರ್ಣಿ-ಸಿ-ದರೂ
ವರ್ಧಿ-ಸುತ್ತ
ವರ್ಧಿ-ಸು-ವು-ದರ
ವರ್ಷ-ಗಳಿಂದ
ವರ್ಷದ
ವರ್ಷ-ವಿಡೀ
ವವರು
ವವರೂ
ವಶ-ದ-ಲ್ಲಿ-ಟ್ಟಿರ
ವಶ-ದ-ಲ್ಲಿ-ಟ್ಟು-ಕೊ-ಳ್ಳುವ
ವಸ್ತು
ವಸ್ತು-ಗಳನ್ನು
ವಸ್ತು-ಗಳೂ
ವಸ್ತು-ವನ್ನು
ವಸ್ತು-ವನ್ನೂ
ವಸ್ತು-ವಲ್ಲ
ವಸ್ತು-ವಿಗೂ
ವಸ್ತು-ವಿ-ನಲ್ಲಿ
ವಸ್ತ್ರ-ಗಳು
ವಹಿ-ಸ-ಬೇಕು
ವಹಿಸಿ
ವಹಿ-ಸುವ
ವಾಂತಿ-ಭೇ-ದಿಗೆ
ವಾಕ್ಯ
ವಾಕ್ಯ-ಗಳಲ್ಲಿ
ವಾಕ್ಯ-ಗಳು
ವಾಕ್ಯ-ದೋ-ಷ-ಗಳು
ವಾಕ್ಯ-ವನ್ನು
ವಾಕ್ಯ-ವನ್ನೂ
ವಾಕ್ಯ-ವೊಂ-ದನ್ನು
ವಾಗ-ದಂತೆ
ವಾಗದು
ವಾಗ-ಬೇ-ಕಾ-ಗು-ತ್ತದೆ
ವಾಗಿ
ವಾಗಿಯೇ
ವಾಗಿ-ರುವ
ವಾಗಿ-ರು-ವಂತೆ
ವಾಗು-ತ್ತವೆ
ವಾಗುವ
ವಾಗು-ವು-ದಲ್ಲ
ವಾಗು-ವು-ದುಂಟು
ವಾತಾ-ವ-ರ-ಣ-ದಲ್ಲಿ
ವಾತಾ-ವ-ರ-ಣ-ದ-ಲ್ಲಿ-ರುವ
ವಾತಾ-ವ-ರ-ಣ-ವನ್ನು
ವಾತಾ-ವ-ರ-ಣ-ವೊಂದೇ
ವಾದ
ವಾದದ್ದು
ವಾದ-ವನ್ನು
ವಾದ-ಸ-ರ-ಣಿ-ಯನ್ನು
ವಾಮ-ನಾ-ಕಾ-ರದ
ವಾಯಿ-ತ-ಲ್ಲವೆ
ವಾರ-ಕ್ಕೊಮ್ಮೆ
ವಾರ್ಷಿಕ
ವಾಸದ
ವಾಸವಾ
ವಾಸ್ತ-ವಿಕ
ವಾಹ-ನ-ಗಳು
ವಿಕಾ-ರ-ಗ-ಳಿಗೆ
ವಿಖ್ಯಾ-ತ-ರಾ-ಗ-ಲಿ-ಲ್ಲವೆ
ವಿಚಾರ
ವಿಚಾ-ರ-ಗಳನ್ನು
ವಿಚಾ-ರ-ಗ-ಳಲ್ಲೂ
ವಿಚಾ-ರ-ಗ-ಳಾ-ಯಿತು
ವಿಚಾ-ರ-ಗಳು
ವಿಚಾ-ರ-ವನ್ನು
ವಿಚಾ-ರ-ವಾ-ಯಿತು
ವಿಚಾ-ರ-ವೆಂದೇ
ವಿಚಿತ್ರ
ವಿಜ್ಞಾ-ನ-ಇ-ವು-ಗಳಲ್ಲಿ
ವಿಟ್ಟು
ವಿದೆ
ವಿದ್ಯಾ-ದಾ-ನದ
ವಿದ್ಯಾ-ಭ್ಯಾಸ
ವಿದ್ಯಾ-ಭ್ಯಾ-ಸದ
ವಿದ್ಯಾ-ಭ್ಯಾ-ಸ-ವೆಂದರೆ
ವಿದ್ಯಾ-ಮಟ್ಟ
ವಿದ್ಯಾರ್ಥಿ
ವಿದ್ಯಾ-ರ್ಥಿ-ಗಳ
ವಿದ್ಯಾ-ರ್ಥಿ-ಗಳಲ್ಲಿ
ವಿದ್ಯಾ-ರ್ಥಿ-ಗ-ಳಲ್ಲೂ
ವಿದ್ಯಾ-ರ್ಥಿ-ಗ-ಳಿ-ಗಾಗಿ
ವಿದ್ಯಾ-ರ್ಥಿ-ಗ-ಳಿಗೆ
ವಿದ್ಯಾ-ರ್ಥಿ-ಗ-ಳಿ-ದ್ದಾರೆ
ವಿದ್ಯಾ-ರ್ಥಿ-ಗ-ಳಿ-ರು-ತ್ತಾರೆ
ವಿದ್ಯಾ-ರ್ಥಿ-ಗಳು
ವಿದ್ಯಾ-ರ್ಥಿ-ಗಳೂ
ವಿದ್ಯಾ-ರ್ಥಿ-ಗಾಗಿ
ವಿದ್ಯಾ-ರ್ಥಿ-ಗೊಂದು
ವಿದ್ಯಾ-ರ್ಥಿ-ನಿ-ಯರ
ವಿದ್ಯಾ-ರ್ಥಿ-ನಿ-ಯ-ರಿಗೆ
ವಿದ್ಯಾ-ರ್ಥಿ-ನಿ-ಯರು
ವಿದ್ಯಾ-ರ್ಥಿ-ಯ-ನ್ನು-ದ್ದೇ-ಶಿಸಿ
ವಿದ್ಯಾ-ರ್ಥಿ-ಯಲ್ಲೂ
ವಿದ್ಯಾ-ರ್ಥಿ-ಯಾ-ಗಿ-ದ್ದಾಗ
ವಿದ್ಯಾ-ರ್ಥಿ-ಯಾದ
ವಿದ್ಯಾ-ರ್ಥಿಯು
ವಿದ್ಯಾ-ರ್ಥಿಯೂ
ವಿದ್ಯಾ-ರ್ಥಿ-ಯೊಬ್ಬ
ವಿದ್ಯಾ-ರ್ಥಿ-ಸ-ಹ-ಪಾ-ಠಿ-ಗಳನ್ನು
ವಿದ್ಯಾರ್ಥೀ
ವಿದ್ಯಾ-ವಂ-ತ-ನ-ನ್ನಾಗಿ
ವಿದ್ಯಾ-ವಂ-ತ-ನಾ-ಗಿಯೇ
ವಿದ್ಯಾ-ವಂ-ತ-ನಾ-ಗು-ವುದು
ವಿದ್ಯಾ-ವಂ-ತ-ನೆಂದು
ವಿದ್ಯಾ-ವಂ-ತ-ರಾಗ
ವಿದ್ಯಾ-ವಂ-ತರು
ವಿದ್ಯೆ
ವಿದ್ಯೆಯ
ವಿದ್ಯೆ-ಯ-ನ್ನಾ-ಗಲೀ
ವಿದ್ಯೆ-ಯನ್ನು
ವಿಧಾ-ನ-ದಿಂದ
ವಿಧಾ-ನ-ವನ್ನು
ವಿಧಾ-ನ-ವನ್ನೂ
ವಿಧೇಯ
ವಿನ-ಯ-ವಿ-ಶ್ವಾ-ಸ-ದಿಂದ
ವಿನಾ-ಕಾ-ರಣ
ವಿನಾ-ಯಕ
ವಿನಿ-ಮಯ
ವಿನ್ಯಾ-ಸ-ಗ-ಳೆಲ್ಲ
ವಿಭಿನ್ನ
ವಿಮಾನ
ವಿರ-ಬ-ಹು-ದೆಂಬ
ವಿರಾಗ
ವಿಲಾ-ಸ-ದಲ್ಲಿ
ವಿಲಾಸೀ
ವಿವ-ರಿ-ಸ-ಬೇ-ಕಾ-ಗಿಲ್ಲ
ವಿವ-ರಿ-ಸ-ಲಾದ
ವಿವ-ರಿಸಿ
ವಿವ-ರಿ-ಸು-ತ್ತದೆ
ವಿವ-ರಿ-ಸು-ವಾಗ
ವಿವಿಧ
ವಿವೇಕ
ವಿವೇ-ಕ-ಪ್ರ-ಜ್ಞೆ-ಯಿ-ರು-ವಂತೆ
ವಿವೇಕಾನಂದರು
ವಿವೇ-ಕಿ-ಗಳ
ವಿವೇ-ಚ-ನೆ-ಯಿಂದ
ವಿಶೇಷ
ವಿಶೇ-ಷ-ವಾಗಿ
ವಿಶ್ವ
ವಿಶ್ವಾ-ಸ-ಶ್ರದ್ಧೆ
ವಿಶ್ವಾ-ಸ-ವಿಟ್ಟು
ವಿಶ್ವೇ-ಶ್ವ-ರಯ್ಯ
ವಿಶ್ವೇ-ಶ್ವ-ರ-ಯ್ಯ-ನ-ವರು
ವಿಷ-ದಂತೆ
ವಿಷಯ
ವಿಷ-ಯಕ್ಕೂ
ವಿಷ-ಯಕ್ಕೆ
ವಿಷ-ಯ-ಗಳ
ವಿಷ-ಯ-ಗಳನ್ನು
ವಿಷ-ಯ-ಗಳಲ್ಲಿ
ವಿಷ-ಯ-ಗ-ಳಲ್ಲೂ
ವಿಷ-ಯ-ಗ-ಳಿಗೂ
ವಿಷ-ಯ-ಗಳು
ವಿಷ-ಯ-ಗ-ಳೆ-ಡೆಗೆ
ವಿಷ-ಯದ
ವಿಷ-ಯ-ದಲ್ಲಿ
ವಿಷ-ಯ-ದಲ್ಲೂ
ವಿಷ-ಯ-ದಲ್ಲೇ
ವಿಷ-ಯ-ವನ್ನು
ವಿಷ-ಯ-ವ-ಸ್ತು-ಗಳಲ್ಲಿ
ವಿಷ-ಯ-ವಾಗಿ
ವಿಷ-ಯವು
ವಿಷ-ಯವೇ
ವಿಷ-ಯ-ವೊಂ-ದನ್ನು
ವಿಸ್ತಾ-ರ-ವಾ-ಗಿಯೇ
ವಿಹ-ರಿ-ಸು-ತ್ತಿ-ರು-ತ್ತದೆ
ವೀಕ್ಷಿ-ಸುತ್ತ
ವೀಕ್ಷಿ-ಸುವ
ವೀರ-ರನ್ನೇ
ವೀರ್ಯ-ಶ-ಕ್ತಿ-ಗಿಂತ
ವೀರ್ಯ-ಶ-ಕ್ತಿ-ಯನ್ನು
ವುದನ್ನು
ವುದ-ರಿಂದ
ವುದ-ಲ್ಲದೆ
ವುದಿಲ್ಲ
ವುದು
ವುದು-ಇದೇ
ವೃತ್ತಿಗೆ
ವೃತ್ತಿ-ಶಿ-ಕ್ಷ-ಣ-ವನ್ನು
ವೃಥಾ
ವೃದ್ಧಿ
ವೆಂದರೆ
ವೆಂದು
ವೆಂಬುದೂ
ವೆನೋ
ವೆನ್ನಿ
ವೇಗ-ವಾಗಿ
ವೇಗ-ವಾ-ಗಿಯೂ
ವೇನೆಂದು
ವೇಳಾ
ವೇಳಾ-ಪ-ಟ್ಟಿಗೆ
ವೇಳಾ-ಪ-ಟ್ಟಿಯ
ವೇಳಾ-ಪ-ಟ್ಟಿ-ಯಂತೆ
ವೇಳಾ-ಪ-ಟ್ಟಿ-ಯನ್ನು
ವೇಳಾ-ಪ-ಟ್ಟಿ-ಯ-ಲ್ಲದೆ
ವೇಳಾ-ಪ-ಟ್ಟಿ-ಯಲ್ಲಿ
ವೇಳಾ-ಪ-ಟ್ಟಿ-ಯೊಂ-ದನ್ನು
ವೇಳೆ
ವೇಳೆಗೆ
ವೇಳೆ-ಯನ್ನು
ವೇಳೆ-ಯಲ್ಲಿ
ವೇಳೆಯೂ
ವೇಳೆಯೇ
ವೈಖ-ರಿ-ಯನ್ನು
ವೈಭ-ವ-ದಲ್ಲಿ
ವೈಯ
ವೈಯ-ಕ್ತಿಕ
ವೈರಾಗ್ಯ
ವೈರಾ-ಗ್ಯದ
ವೈರಾ-ಗ್ಯ-ದಿಂದ
ವೈರಾ-ಗ್ಯ-ವೆಂದರೆ
ವೈರಾ-ಗ್ಯ-ವೆಂದು
ವೈರಿ
ವೈರಿಯ
ವೈರಿಯು
ವೈರಿಯೇ
ವೊಂದು
ವ್ಯಕ್ತ
ವ್ಯಕ್ತ-ಗೊಳಿ
ವ್ಯಕ್ತ-ವಾ-ಗು-ತ್ತದೆ
ವ್ಯಕ್ತಿ
ವ್ಯಕ್ತಿ-ಗ-ಳಾ-ಗ-ಬೇಕು
ವ್ಯಕ್ತಿ-ಯಾ-ಗು-ವುದು
ವ್ಯತಿ-ರಿ-ಕ್ತ-ವಾದ
ವ್ಯತ್ಯಾ-ಸ-ವನ್ನು
ವ್ಯತ್ಯಾ-ಸ-ವಾ-ಯಿ-ತೆಂ-ದರೆ
ವ್ಯಯ-ವಾ-ಗುವ
ವ್ಯರ್ಥ
ವ್ಯರ್ಥ-ಕಾ-ರ್ಯ-ಗಳು
ವ್ಯರ್ಥ-ವಾಗಿ
ವ್ಯರ್ಥ-ವಾ-ಯಿತೆ
ವ್ಯವ-ಸ್ಥಿ-ತ-ವಾ-ಗಿಯೂ
ವ್ಯಾಕ-ರ-ಣ-ಬ-ದ್ಧ-ವಾ-ಗಿರ
ವ್ಯಾಪಕ
ವ್ಯಾಯಾಮ
ವ್ಯಾಯಾ-ಮದ
ವ್ರತ
ವ್ರತ-ಧಾ-ರ-ಕ-ನಂತೆ
ವ್ರತ-ಧಾ-ರ-ಕ-ನೆಂದೇ
ವ್ರತ-ಧಾ-ರ-ಕರು
ವ್ರತ-ಧಾ-ರಣೆ
ವ್ರತವೇ
ಶಕ್ತಿ
ಶಕ್ತಿ-ವಿ-ವೇಕ
ಶಕ್ತಿ-ನಷ್ಟ
ಶಕ್ತಿಯ
ಶಕ್ತಿ-ಯನ್ನು
ಶಕ್ತಿ-ಯಲ್ಲಿ
ಶಕ್ತಿ-ಯಿದೆ
ಶಕ್ತಿ-ಯಿ-ದ್ದರೂ
ಶಕ್ತಿ-ಯು-ತ-ವಾ-ಗು-ವುದು
ಶಕ್ತಿ-ಯೆಲ್ಲ
ಶಕ್ತಿಯೇ
ಶಕ್ತಿ-ಶಾ-ಲಿ-ಯಾಗಿ
ಶಕ್ತಿ-ಶಾ-ಲಿ-ಯಾದ
ಶಕ್ತ್ಯ-ನು-ಸಾರ
ಶನಿ-ವಾ-ರದ
ಶಪಥ
ಶಬ್ದ
ಶಬ್ದ-ಕೋಶ
ಶಬ್ದ-ಕೋ-ಶದ
ಶಬ್ದಕ್ಕೂ
ಶಬ್ದ-ಗಳ
ಶಬ್ದ-ಗಳನ್ನು
ಶಬ್ದ-ಗಳು
ಶಬ್ದ-ಮಾ-ಲಿನ್ಯ
ಶಬ್ದ-ಮಾ-ಲಿ-ನ್ಯದ
ಶಬ್ದ-ವನ್ನು
ಶಬ್ದ-ವನ್ನೂ
ಶಬ್ದ-ವ-ನ್ನೇನೋ
ಶಮ
ಶರ-ಣಾ-ಗಿ-ದ್ದೇನೆ
ಶರೀರ
ಶರೀ-ರ
ಶರೀ-ರ-ಮ-ನ-ಸ್ಸಿ-ನ-ವ-ರಿಗೆ
ಶರೀ-ರ-ಮ-ನ-ಸ್ಸು-ಗಳ
ಶರೀ-ರ-ಮ-ನ-ಸ್ಸು-ಗಳನ್ನು
ಶರೀ-ರ-ಮ-ನೋ-ವಿ-ಕಾ-ರ-ಗ-ಳಿಗೆ
ಶರೀ-ರ-ವ-ಸ್ತ್ರ-ವ-ಸ್ತು-ಗಳನ್ನು
ಶರೀ-ರದ
ಶರೀ-ರವೂ
ಶಾಂತ
ಶಾಂತವೂ
ಶಾಂತಿ-ಸೌ-ಖ್ಯ-ಸ-ಮಾ-ಧಾ-ನ-ಗ-ಳೆಲ್ಲ
ಶಾಖ
ಶಾಲಾ
ಶಾಲಾ-ವಿ-ದ್ಯಾ-ರ್ಥಿ-ಗ-ಳೇ-ನ-ಲ್ಲ-ವಲ್ಲ
ಶಾಲೆ
ಶಾಲೆ-ಗಳು
ಶಾಲೆಗೆ
ಶಾಲೆಯ
ಶಾಲೆ-ಯಲ್ಲಿ
ಶಾಸ್ತ್ರದ
ಶಿಕ್ಷಣ
ಶಿಕ್ಷ-ಣ-ತ-ಜ್ಞ-ರಿಗೆ
ಶಿಕ್ಷ-ಣ-ತ-ಜ್ಞರು
ಶಿಕ್ಷ-ಣ-ವ-ನ್ನಾ-ಗಲೀ
ಶಿಸ್ತನ್ನು
ಶಿಸ್ತಿನ
ಶಿಸ್ತಿ-ನಿಂದ
ಶಿಸ್ತು-ವಿ-ಧೇ-ಯತೆ
ಶಿಸ್ತು-ಸು-ಸಂ-ಬ-ದ್ಧತೆ
ಶಿಸ್ತು-ಬದ್ಧ
ಶಿಸ್ತು-ಬ-ದ್ಧ-ವಾ-ಗಿ-ಸು-ವುದೇ
ಶಿಸ್ತು-ಬ-ದ್ಧ-ವಾದ
ಶುಚಿ-ಯಾ-ಗಿ-ಟ್ಟಿ-ರ-ಬೇಕು
ಶುಚಿ-ಯಾ-ಗಿ-ಟ್ಟು-ಕೊಳ್ಳು
ಶುಚಿ-ಯಾದ
ಶುದ್ಧ
ಶುಭ-ವಾ-ಗಲಿ
ಶುಭಾ-ಶ-ಯ-ಗ-ಳೊಂ-ದಿಗೆ
ಶುರು-ವಾ-ಗ-ದಂತೆ
ಶೌಚಾ-ದಿ-ಗಳನ್ನು
ಶ್ರದ್ಧಾ-ವಂತ
ಶ್ರದ್ಧೆ
ಶ್ರದ್ಧೆಯ
ಶ್ರದ್ಧೆ-ಯನ್ನು
ಶ್ರದ್ಧೆ-ಯಿಂದ
ಶ್ರದ್ಧೆ-ಯಿಂ-ದಲೇ
ಶ್ರದ್ಧೆ-ಯೆಂ-ದರೆ
ಶ್ರದ್ಧೆ-ಯೆಂಬ
ಶ್ರದ್ಧೆಯೇ
ಶ್ರಮ-ವಿ-ರು-ವು-ದಿಲ್ಲ
ಶ್ರಮಿ-ಸ-ಬೇ-ಕು-ವಿ-ದ್ಯಾ-ಭ್ಯಾಸ
ಶ್ರವಣ
ಶ್ರೀ
ಶ್ರೀಕೃಷ್ಣ
ಶ್ರೀಕೃ-ಷ್ಣನ
ಶ್ರೀರಾ-ಮ-ಕೃಷ್ಣ
ಶ್ರೀರಾ-ಮ-ನ-ವ-ಮಿಯ
ಶ್ರೇಣಿ
ಶ್ರೇಣಿ-ಯ-ಲ್ಲಾ-ದರೂ
ಶ್ರೇಣಿ-ಯಲ್ಲಿ
ಶ್ರೇಷ್ಠ
ಶ್ರೇಷ್ಠ-ವಾದ
ಶ್ಲೋಕ-ಗಳ
ಷೋಕಿ-ಲಾ-ಲರ
ಸಂಕಟ
ಸಂಕ-ಲ್ಪ-ಶ-ಕ್ತಿ-ಯಿಂದ
ಸಂಗ-ತಿ-ಯನ್ನು
ಸಂಗ-ತಿ-ಯೆಂ-ದರೆ
ಸಂಗ-ತಿ-ಯೊಂದು
ಸಂಗೀತ
ಸಂಚಾ-ರ-ವಾಗಿ
ಸಂದರ್ಭ
ಸಂದ-ರ್ಭ-ಗಳನ್ನು
ಸಂದ-ರ್ಭ-ಗಳು
ಸಂದ-ರ್ಭ-ದಲ್ಲಿ
ಸಂನ್ಯಾ-ಸ-ವೆಂದೆ
ಸಂನ್ಯಾ-ಸಿ-ಗಳು
ಸಂನ್ಯಾ-ಸಿ-ಯಾ-ಗು-ವುದು
ಸಂಪ-ತ್ತನ್ನು
ಸಂಪತ್ತು
ಸಂಪ-ರ್ಕ-ವನ್ನು
ಸಂಪಾ-ದಿ-ಸಲು
ಸಂಪಾ-ದಿಸಿ
ಸಂಪಾ-ದಿ-ಸು-ತ್ತಾರೆ
ಸಂಪು-ಟವು
ಸಂಪೂರ್ಣ
ಸಂಬಂ-ಧ-ಪಟ್ಟ
ಸಂಬಂ-ಧ-ವಾಗಿ
ಸಂಬಂ-ಧಿ-ಸಿದ
ಸಂಬಂ-ಧಿ-ಸಿ-ದಂತೆ
ಸಂಬ-ಧ-ಪಟ್ಟ
ಸಂಭವ
ಸಂಭ-ವಿ-ಸ-ಬ-ಹುದು
ಸಂಭ-ವಿ-ಸ-ಬ-ಹು-ದೆಂಬ
ಸಂಭಾ-ಷ-ಣೆ-ಯಲ್ಲಿ
ಸಂಯ-ಮಕ್ಕೆ
ಸಂಯ-ಮ-ದ-ಲ್ಲಿ-ಡು-ವು-ದಕ್ಕೆ
ಸಂಯೋ-ಗ-ಗೊಂ-ಡಿ-ರು-ತ್ತದೆ
ಸಂಶಯ
ಸಂಶ-ಯವೇ
ಸಂಸ್ಕೃತ
ಸಂಸ್ಕೃ-ತಿ-ಯೆಂಬ
ಸಕಲ
ಸಕಾ-ಲ-ದಲ್ಲಿ
ಸಚ್ಚಿ-ದಾ-ನಂದ
ಸಣ್ಣ
ಸಣ್ಣ-ಪುಟ್ಟ
ಸತತ
ಸತ್ಪ-ರಿ-ಣಾ-ಮದ
ಸತ್ಯ
ಸತ್ಯ-ವನ್ನು
ಸತ್ಯ-ವಾಗಿ
ಸತ್ಯಾಂಶ
ಸದ-ವ-ಕಾ-ಶ-ಗಳನ್ನು
ಸದ-ವ-ಕಾ-ಶ-ವನ್ನು
ಸದಾ
ಸದುಪ
ಸದು-ಪ-ಯೋಗ
ಸದು-ಪ-ಯೋ-ಗ-ಪ-ಡಿಸಿ
ಸದು-ಪ-ಯೋ-ಗ-ಪ-ಡಿ-ಸಿ-ಕೊಂ-ಡರೆ
ಸದು-ಪ-ಯೋ-ಗ-ಪ-ಡಿ-ಸಿ-ಕೊಂಡು
ಸದು-ಪ-ಯೋ-ಗ-ಪ-ಡಿ-ಸಿ-ಕೊ-ಳ್ಳ-ಬೇ-ಕೆಂ-ಬ-ವ-ರಿಗೆ
ಸದು-ಪ-ಯೋ-ಗ-ಪ-ಡಿ-ಸಿ-ಕೊ-ಳ್ಳ-ಬೇ-ಕೆನ್ನು
ಸದು-ಪ-ಯೋ-ಗ-ಪ-ಡಿ-ಸಿ-ಕೊ-ಳ್ಳಲು
ಸದು-ಪ-ಯೋ-ಗ-ಪ-ಡಿ-ಸಿ-ಕೊ-ಳ್ಳುವ
ಸದು-ಪ-ಯೋ-ಗ-ಪ-ಡಿ-ಸಿ-ಕೊ-ಳ್ಳು-ವುದು
ಸದ್ಭಾ-ವ-ನೆ-ಗಳನ್ನು
ಸನ್ಮಾ-ನಿತ
ಸಫ-ಲ-ತೆಗೆ
ಸಭೆ-ಯೊಂ-ದಕ್ಕೆ
ಸಮನೆ
ಸಮಯ
ಸಮ-ಯಕ್ಕೆ
ಸಮ-ಯದ
ಸಮ-ಯ-ದಲ್ಲಿ
ಸಮ-ಯ-ದ-ಲ್ಲಿಯೂ
ಸಮ-ಯ-ಪ್ರಜ್ಞೆ
ಸಮ-ಯ-ಪ್ರ-ಜ್ಞೆ-ಯನ್ನು
ಸಮ-ಯ-ವನ್ನು
ಸಮ-ಯವು
ಸಮ-ಯವೂ
ಸಮ-ಯ-ವೆಲ್ಲ
ಸಮ-ಯವೇ
ಸಮ-ಯೋ-ಚಿ-ತ-ವಾಗಿ
ಸಮರ್ಥ
ಸಮ-ರ್ಥ-ನಾ-ಗುವೆ
ಸಮ-ರ್ಥ-ರಾ-ಗು-ತ್ತೇವೆ
ಸಮ-ರ್ಥ-ವಾ-ಗು-ತ್ತದೆ
ಸಮ-ರ್ಥಿ-ಸಲು
ಸಮ-ರ್ಪಕ
ಸಮ-ರ್ಪ-ಕ-ವಾಗಿ
ಸಮ-ಸ್ಥಿ-ತಿಗೆ
ಸಮ-ಸ್ಥಿ-ತಿ-ಯನ್ನು
ಸಮ-ಸ್ಥಿ-ತಿ-ಯ-ಲ್ಲಿ-ಡು-ವುದು
ಸಮಸ್ಯೆ
ಸಮ-ಸ್ಯೆಗೆ
ಸಮ-ಸ್ಯೆ-ಯೆಂ-ದರೆ
ಸಮ-ಸ್ಯೆಯೇ
ಸಮ-ಸ್ಯೆ-ಯೊಂ-ದಿದೆ
ಸಮಾ-ಚಾರ
ಸಮಾ-ಜದ
ಸಮಾನ
ಸಮಾ-ನ-ವಾಗಿ
ಸಮೀ-ಪಿ-ಸಿ-ದಾಗ
ಸಮು-ದ್ರಕ್ಕೆ
ಸಮೃ-ದ್ಧ-ವಾ-ಗು-ತ್ತದೆ
ಸಮ್ಮೇ-ಳ-ನಕ್ಕೂ
ಸಮ್ಮೇ-ಳ-ನದ
ಸಮ್ಮೇ-ಳ-ನ-ವೊಂ-ದ-ರಲ್ಲಿ
ಸರಂ-ಜಾ-ಮು-ಗಳನ್ನೆಲ್ಲ
ಸರ-ಕ್ಕನೆ
ಸರಳ
ಸರ-ಳ-ವಾ-ಗಿದೆ
ಸರ-ಸ್ವತಿ
ಸರಾ-ಗ-ವಾಗಿ
ಸರಿ
ಸರಿ-ಪ-ಡಿ-ಸಿ-ಕೊ-ಳ್ಳ-ಬೇಕು
ಸರಿ-ಪ-ಡಿ-ಸು-ವು-ದರ
ಸರಿ-ಯಲು
ಸರಿ-ಯಾಗಿ
ಸರಿ-ಯಾ-ಗಿದೆ
ಸರಿ-ಯಾ-ಗಿ-ದ್ದರೂ
ಸರಿ-ಯಾ-ಗಿ-ವೆಯೇ
ಸರಿ-ಯಾದ
ಸರಿಯೆ
ಸರ್
ಸರ್ಚ್ಲೈಟು
ಸರ್ಚ್ಲೈಟೇ
ಸರ್ವಜ್ಞ
ಸರ್ವರೂ
ಸರ್ವ-ವ್ಯಾಪಿ
ಸರ್ವ-ಶಕ್ತ
ಸರ್ವ-ಶ-ಕ್ತ-ನಾ-ದು-ದ-ರಿಂದ
ಸರ್ವೇ
ಸಲ
ಸಲಕ್ಕೆ
ಸಲ-ಗ-ನಂತೆ
ಸಲ-ಗ-ವನ್ನು
ಸಲದ
ಸಲಹೆ
ಸಲ-ಹೆ-ಗ-ಳಂತೆ
ಸಲ-ಹೆ-ಗಳನ್ನು
ಸಲ-ಹೆ-ಗಳು
ಸಲ-ಹೆ-ಯೆಂ-ದರೆ
ಸಲು
ಸಲೇ
ಸಲ್ಲಿ-ಸ-ಬೇಕು
ಸಹ-ಕಾ-ರವೂ
ಸಹ-ಕಾರಿ
ಸಹ-ಜ-ವಾಗಿ
ಸಹ-ಜ-ವಾ-ಗಿ-ದೆಯೋ
ಸಹ-ಜ-ವಾ-ಗಿಯೇ
ಸಹ-ಜ-ವಾ-ದುದು
ಸಹ-ವಾ-ಸಕ್ಕೆ
ಸಹ-ವಿ-ದ್ಯಾ-ರ್ಥಿ-ಗ-ಳೊಂ-ದಿಗೆ
ಸಹ-ಸ್ರಾರು
ಸಹಾ-ಯಕ
ಸಹಾ-ಯ-ಕಾರಿ
ಸಹಾ-ಯ-ದಿಂದ
ಸಹಾ-ಯ-ವನ್ನು
ಸಹಾ-ಯ-ವಾ-ಗು-ತ್ತದೆ
ಸಹಾ-ಯ-ವಿ-ಲ್ಲ-ದೆಯೇ
ಸಾಕಷ್ಟು
ಸಾಕು
ಸಾಗ-ತೊ-ಡ-ಗಿ-ದಾಗ
ಸಾಗ-ರ-ಲಂ-ಘನ
ಸಾಗು-ವುದು
ಸಾಧ-ಕ-ನನ್ನು
ಸಾಧ-ಕರೂ
ಸಾಧ-ಕವೇ
ಸಾಧನ
ಸಾಧನೆ
ಸಾಧಿ-ಸ-ಬೇ-ಕಾ-ದರೆ
ಸಾಧಿ-ಸ-ಬೇಕು
ಸಾಧಿ-ಸ-ಲಾ-ರದ್ದು
ಸಾಧಿ-ಸಲು
ಸಾಧಿ-ಸಿದ್ದೇ
ಸಾಧಿ-ಸಿ-ಬಿ-ಡ-ಬ-ಹುದು
ಸಾಧ್ಯ
ಸಾಧ್ಯ-ವ-ಲ್ಲದ
ಸಾಧ್ಯ-ವಾಗ
ಸಾಧ್ಯ-ವಾ-ಗ-ದುದೇ
ಸಾಧ್ಯ-ವಾ-ಗದೆ
ಸಾಧ್ಯ-ವಾ-ಗು-ತ್ತದೆ
ಸಾಧ್ಯ-ವಾ-ಗು-ವು-ದಿಲ್ಲ
ಸಾಧ್ಯ-ವಾ-ಗು-ವುದು
ಸಾಧ್ಯ-ವಾ-ದರೆ
ಸಾಧ್ಯ-ವಾ-ದಷ್ಟು
ಸಾಧ್ಯ-ವಿದೆ
ಸಾಧ್ಯ-ವಿಲ್ಲ
ಸಾಮರ್ಥ್ಯ
ಸಾಮ-ರ್ಥ್ಯಕ್ಕೆ
ಸಾಮ-ರ್ಥ್ಯ-ವನ್ನು
ಸಾಮಾನು
ಸಾಮಾನ್ಯ
ಸಾಮಾ-ನ್ಯ-ವಾಗಿ
ಸಾಮೂ-ಹಿಕ
ಸಾಮ್ರಾ-ಟನೇ
ಸಾಯಲಿ
ಸಾಯಿ-ಸ-ದಿ-ರ-ಬ-ಹುದು
ಸಾರಿ
ಸಾರಿ-ಹೇ-ಳು-ತ್ತದೆ
ಸಾರೀನೂ
ಸಾರ್ಥಕ
ಸಾರ್ವ-ಜ-ನಿ-ಕರ
ಸಾಲ
ಸಾಲು-ಗಳು
ಸಾವಿರ
ಸಾವಿ-ರ-ಕ್ಕೊ-ಬ್ಬನೂ
ಸಾವಿ-ರದ
ಸಾಹಿತ್ಯ
ಸಾಹಿ-ತ್ಯವೇ
ಸಿಕ್ಕಿ
ಸಿಕ್ಕಿ-ಬೀ-ಳ-ದಂತೆ
ಸಿಗುವ
ಸಿಗು-ವಂ-ತಿ-ದ್ದ-ರಾ-ಯಿತು
ಸಿಗು-ವು-ದಾ-ದರೂ
ಸಿದ್ಧ-ರಾ-ಗಿ-ರ-ಬೇ-ಕಾ-ಗು-ತ್ತದೆ
ಸಿದ್ಧ-ವಿ-ರು-ವು-ದಿಲ್ಲ
ಸಿದ್ಧಿ
ಸಿದ್ಧಿಯ
ಸಿದ್ಧಿ-ಯಿದೆ
ಸಿದ್ಧಿ-ಸ-ಲಾ-ರದು
ಸಿದ್ಧಿ-ಸಿ-ಕೊಂ-ಡಿ-ದ್ದಾರೆ
ಸಿದ್ಧಿ-ಸಿ-ಕೊ-ಳ್ಳ-ಬ-ಹುದು
ಸಿದ್ಧಿ-ಸಿ-ಕೊ-ಳ್ಳಲು
ಸಿದ್ಧಿ-ಸಿ-ಕೊ-ಳ್ಳುವ
ಸಿದ್ಧಿ-ಸಿತು
ಸಿದ್ಧಿ-ಸಿ-ತೆಂದೇ
ಸಿದ್ಧಿ-ಸಿದೆ
ಸಿದ್ಧಿ-ಸಿಲ್ಲ
ಸಿದ್ಧಿ-ಸು-ತ್ತದೆ
ಸಿದ್ಧಿ-ಸು-ತ್ತ-ದೆ-ಯಾ-ದ್ದ-ರಿಂದ
ಸಿನೆಮಾ
ಸುಂದ-ರ-ವಾ-ಗಿ-ದ್ದರೆ
ಸುಂದ-ರ-ವಾ-ಗಿಯೂ
ಸುಂದ-ರ-ವಾ-ಗಿರ
ಸುಂದ-ರ-ವಾ-ಗಿ-ರ-ಬೇಕು
ಸುಂದ-ರ-ವಾದ
ಸುಂದ-ರವೂ
ಸುಖ
ಸುಗ-ಮ-ವಾ-ಗಲು
ಸುಡ-ಲಿ-ಲ್ಲ-ವಲ್ಲ
ಸುಡುವ
ಸುತ್ತ-ಮುತ್ತ
ಸುತ್ತದೆ
ಸುತ್ತಾನೆ
ಸುತ್ತಿಗೆ
ಸುದ್ದಿ
ಸುಮಾ-ರಾಗಿ
ಸುಮಾರು
ಸುಮ್ಮ-ನಿ-ದ್ದರೂ
ಸುಮ್ಮ-ನಿ-ರ-ಬಾ-ರದು
ಸುಮ್ಮ-ನಿ-ರು-ವು-ದಿಲ್ಲ
ಸುಮ್ಮನೆ
ಸುರಿ-ಯುವ
ಸುಲಭ
ಸುಲ-ಭ-ವಾಗಿ
ಸುಳಿ
ಸುಳಿ-ಯ-ಗೊ-ಡ-ದಿ-ರ-ಬೇಕು
ಸುಳಿ-ಯ-ದಂತೆ
ಸುವುದು
ಸುವ್ಯ-ವ-ಸ್ಥಿತ
ಸುಶಿ-ಕ್ಷಿ-ತ-ಗೊ-ಳಿಸಿ
ಸುಸಂ-ಬ-ದ್ಧ-ವಾಗಿ
ಸುಸೂ-ತ್ರ-ತೆಯ
ಸುಸೂ-ತ್ರ-ವಾಗಿ
ಸೂಕ್ತ
ಸೂಕ್ಷ್ಮ
ಸೂಚ-ನೆ-ಗಳನ್ನು
ಸೂಚ-ನೆ-ಗಳನ್ನೂ
ಸೂಚ-ನೆ-ಯನ್ನು
ಸೂಚಿ-ಸಿ-ದಂತೆ
ಸೂತ್ರ-ಗಳನ್ನು
ಸೂರ್ಯ-ಕಿ-ರ-ಣ-ಗ-ಳಿಗೆ
ಸೆಕೆ-ಗಾ-ಲದ
ಸೆಕೆ-ಗಾ-ಲ-ದಲ್ಲಿ
ಸೆಳೆ-ತ-ಗ-ಳಿಗೆ
ಸೆಳೆ-ಯು-ತ್ತಲೇ
ಸೆಳೆ-ಯುವ
ಸೆಳೆ-ಯು-ವಂ-ತಿ-ದ್ದರೆ
ಸೇರಿ
ಸೇರಿ-ಕೊಂ-ಡಿದೆ
ಸೇರಿ-ಕೊಂಡೇ
ಸೇರಿ-ರಲಿ
ಸೇರಿ-ರು-ತ್ತವೆ
ಸೇರಿ-ರುವ
ಸೇರಿವೆ
ಸೇರಿಸಿ
ಸೇರಿ-ಸಿ-ಕೊಂಡು
ಸೇರಿ-ಸಿ-ಕೊ-ಳ್ಳ-ಬ-ಹುದು
ಸೇವಕ
ಸೇವ-ಕ-ನಂತೆ
ಸೇವಿ-ಸ-ಬೇಕು
ಸೇವಿ-ಸಿ-ದರೆ
ಸೈನಿಕ
ಸೊಟ್ಟ-ಗಿ-ದ್ದರೆ
ಸೊತ್ತು
ಸೊನ್ನೆ
ಸೋಜಿ-ಗದ
ಸೋಮ-ಶೇ-ಖರ
ಸೋಮಾರಿ
ಸೋರಿ-ಹೋ-ಗುವ
ಸೌಂದ-ರ್ಯ-ವನ್ನು
ಸೌಭಾ-ಗ್ಯಕ್ಕೆ
ಸ್ತೋತ್ರ-ಪ-ಠನ
ಸ್ತ್ರೀಪು-ರು-ಷರು
ಸ್ಥಾನ-ಗಳನ್ನು
ಸ್ಥಾನ-ದಲ್ಲಿ
ಸ್ಥಿತಿ
ಸ್ಥಿತಿಗೆ
ಸ್ಥಿತಿ-ಯ-ಲ್ಲಿಟ್ಟು
ಸ್ಥಿರ-ವಾ-ಗಿ-ರು-ವಂತೆ
ಸ್ಥಿರ-ವಾ-ಗಿ-ರು-ವು-ದುಂಟೆ
ಸ್ಥಿರವೂ
ಸ್ನಾನ
ಸ್ನಾನಕ್ಕೆ
ಸ್ನಾನ-ದಿಂದ
ಸ್ನಾನ-ವನ್ನು
ಸ್ನಾನ-ವೆಂ-ದಿರಿ
ಸ್ನಾನಾ-ದಿ-ಗಳನ್ನು
ಸ್ನೇಹಿ-ತನ
ಸ್ನೇಹಿ-ತ-ರನ್ನು
ಸ್ನೇಹಿ-ತ-ರಿ-ಗಿಂ-ತಲೂ
ಸ್ನೇಹಿ-ತರು
ಸ್ನೇಹಿ-ತ-ರೊ-ಡನೆ
ಸ್ಪಷ್ಟ
ಸ್ಪಷ್ಟ-ವಾಗಿ
ಸ್ಫೂರ್ತಿ-ಯನ್ನು
ಸ್ವಂತ
ಸ್ವಂತದ
ಸ್ವಂತ-ದ್ದಲ್ಲ
ಸ್ವತಂ-ತ್ರ-ವಾಗಿ
ಸ್ವತಃ
ಸ್ವಪ್ನಾ-ವ-ಸ್ಥೆ-ಯ-ಲ್ಲಿ-ದ್ದರೂ
ಸ್ವಪ್ನಾ-ವ-ಸ್ಥೆ-ಯ-ಲ್ಲಿಯೇ
ಸ್ವಭಾವ
ಸ್ವಭಾ-ವ-ಮ-ರ್ಮ-ಗಳ
ಸ್ವಭಾ-ವತಃ
ಸ್ವಭಾ-ವ-ದಲ್ಲಿ
ಸ್ವಭಾ-ವ-ವನ್ನು
ಸ್ವಭಾ-ವ-ವಿ-ರು-ತ್ತ-ದೆ-ಗಾ-ಳಿಗೆ
ಸ್ವಭಾ-ವ-ಸ-ಹ-ಜ-ವಾಗಿ
ಸ್ವಲ್ಪ
ಸ್ವಲ್ಪ-ಸ್ವ-ಲ್ಪವೇ
ಸ್ವಷ್ಟ
ಸ್ವಷ್ಟ-ವಾಗಿ
ಸ್ವಷ್ಟ-ವಾ-ಗು-ತ್ತದೆ
ಸ್ವಾಧೀ-ನಕ್ಕೆ
ಸ್ವಾಧೀ-ನ-ದ-ಲ್ಲಿ-ಟ್ಟು-ಕೊ-ಳ್ಳ-ಬೇ-ಕೆಂ-ಬ-ವರು
ಸ್ವಾಮಿ
ಸ್ವಾರ-ಸ್ಯ-ವೆಂದರೆ
ಸ್ವೇಚ್ಛೆ-ಯಾಗಿ
ಸ್ಸಂಯ-ಮ-ವನ್ನು
ಸ್ಸನ್ನು
ಸ್ಸಿಗೆ
ಹಂಚಿ
ಹಂಚಿ-ಹೋ-ಗಿ-ರು-ವುದೇ
ಹಂಬಲ
ಹಂಬ-ಲ-ವನ್ನು
ಹಂಬ-ಲ-ವಿ-ದ್ದರೆ
ಹಂಬ-ಲ-ವೊಂ-ದಿ-ದ್ದು-ಬಿ-ಟ್ಟರೆ
ಹಕ್ಕು-ಗಳೂ
ಹಗ-ಲು-ಗ-ನಸು
ಹಗು-ರ-ವಾಗಿ
ಹಚ್ಚಿ-ಕೂ-ದ-ಲನ್ನು
ಹಟ
ಹಟ-ತೊಟ್ಟು
ಹಣ
ಹಣ-ಇ-ವೆಲ್ಲ
ಹಣ-ಕ್ಕಿಂ-ತಲೂ
ಹಣದ
ಹಣ-ವನ್ನು
ಹಣೆ-ಬ-ರ-ಹ-ವಾ-ದೀತು
ಹಣ್ಣಿನ
ಹತೋ-ಟಿಯ
ಹತೋ-ಟಿ-ಯಲ್ಲಿ
ಹತೋ-ಟಿ-ಯ-ಲ್ಲಿ-ಟ್ಟಿ-ರು-ವುದು
ಹತೋ-ಟಿ-ಯ-ಲ್ಲಿ-ಡದ
ಹತೋ-ಟಿ-ಯ-ಲ್ಲಿ-ಡ-ಬೇ-ಕಾದ
ಹತೋ-ಟಿ-ಯ-ಲ್ಲಿ-ಡಲು
ಹತೋ-ಟಿ-ಯ-ಲ್ಲಿ-ರ-ದಿ-ದ್ದರೆ
ಹತ್ತಕ್ಕೆ
ಹತ್ತ-ಬೇಕು
ಹತ್ತರ
ಹತ್ತಾರು
ಹತ್ತಿ
ಹತ್ತಿರ
ಹತ್ತಿ-ರ-ದಲ್ಲಿ
ಹತ್ತಿ-ರವೇ
ಹತ್ತು
ಹತ್ತು-ವಂ-ತಹ
ಹತ್ತು-ವುದು
ಹತ್ತು-ಹ-ಲವು
ಹತ್ತೇ
ಹದಕ್ಕೆ
ಹದ-ವಾ-ಗಿ-ರು-ತ್ತದೆ
ಹದಿ-ನೆಂಟು
ಹದಿ-ನೇಳು
ಹದಿ-ನೈದೇ
ಹದಿ-ಹ-ರೆ-ಯದ
ಹನು-ಮಂ-ತ-ನಲ್ಲಿ
ಹನ್ನೊಂದು
ಹಬ್ಬ
ಹರ-ಟುತ್ತ
ಹರಟೆ
ಹರ-ಟೆಯ
ಹರ-ಟೆ-ಯಿಂದ
ಹರ-ಡಿ-ಕೊಂ-ಡಿ-ದ್ದುವು
ಹರ-ಡಿ-ಕೊಂ-ಡಿರು
ಹರ-ಡಿ-ಕೊಂ-ಡಿ-ರುವ
ಹರಿ
ಹರಿ-ಕ-ಥೆ
ಹರಿ-ದಾ-ಡು-ತ್ತಲೇ
ಹರಿ-ದಾ-ಡುವ
ಹರಿದು
ಹರಿ-ದು-ಬಂದ
ಹರಿ-ದು-ಹೋ-ಗುವ
ಹರಿ-ಯ-ಗೊ-ಟ್ಟಂತೆ
ಹರಿ-ಯ-ಗೊ-ಡ-ದಿ-ರ-ಬೇಕು
ಹರಿ-ಯಿಸಿ
ಹರಿ-ಯಿ-ಸು-ತ್ತಾರೆ
ಹರಿ-ಯಿ-ಸು-ವುದು
ಹರಿ-ಯು-ತ್ತಿದೆ
ಹರಿ-ಯುವ
ಹರಿ-ಸ-ಬೇ-ಕಾ-ಗು-ತ್ತದೆ
ಹರಿ-ಸ-ಬೇ-ಕಾ-ದ-ದ್ದಿದೆ
ಹರ್ಷಾ-ನಂದ
ಹಲ-ವ-ರನ್ನು
ಹಲ-ವರು
ಹಲ-ವಾರು
ಹಲ-ವಿ-ಧದ
ಹಲವು
ಹಳ್ಳಿ-ಗಳೇ
ಹವಾ-ಗು-ಣ-ದಿಂದ
ಹಸ-ನಾದ
ಹಸಿ-ಯಾಗಿ
ಹಾ
ಹಾಕದೆ
ಹಾಕ-ಬ-ಹು-ದು-ಇ-ಲ್ಲವೆ
ಹಾಕಿಕೊ
ಹಾಕಿ-ಕೊಂ-ಡಿ-ದ್ದರೆ
ಹಾಕಿ-ಕೊಂ-ಡಿ-ರುವ
ಹಾಕಿ-ಕೊಂಡು
ಹಾಕಿ-ಕೊಳ್ಳ
ಹಾಕಿ-ಕೊ-ಳ್ಳದೆ
ಹಾಕಿ-ಕೊ-ಳ್ಳ-ಬೇಕು
ಹಾಕಿ-ಕೊ-ಳ್ಳು-ತ್ತಿ-ರ-ಬೇಕು
ಹಾಕುವ
ಹಾಗಲ್ಲ
ಹಾಗಾ-ದರೆ
ಹಾಗಿ-ರಲಿ
ಹಾಗೂ
ಹಾಗೆ
ಹಾಗೆಯೇ
ಹಾಗೇ
ಹಾಯಿಸಿ
ಹಾಯಿ-ಸಿದ
ಹಾರಿ-ಹೋ-ಗ-ಬ-ಹುದು
ಹಾರಿ-ಹೋಗಿ
ಹಾಲು
ಹಾಳಾದ
ಹಾಳು
ಹಾಸಿಗೆ
ಹಾಹಾ-ಕಾ-ರ-ಗೈ-ಯುವ
ಹಿಂದಿ
ಹಿಂದಿನ
ಹಿಂದಿ-ನಿಂ-ದಲೂ
ಹಿಂದಿ-ರು-ಗಿದ
ಹಿಂದಿ-ರು-ಗಿಸಿ
ಹಿಂದಿ-ರು-ಗಿ-ಸುವ
ಹಿಂದಿ-ರುವ
ಹಿಂದೆ
ಹಿಡಿ
ಹಿಡಿ-ತಕ್ಕೆ
ಹಿಡಿ-ತ-ವಿ-ಲ್ಲ-ದಿ-ದ್ದರೆ
ಹಿಡಿದ
ಹಿಡಿ-ದಿ-ಡು-ವುದು
ಹಿಡಿದು
ಹಿಡಿ-ದು-ಕೊಂ-ಡಿ-ರು-ತ್ತದೆ
ಹಿಡಿ-ದು-ಕೊಂಡು
ಹಿಡಿ-ಯಿತು
ಹಿಡಿ-ಯು-ತ್ತಿ-ದ್ದರೆ
ಹಿತ-ವಾಗಿ
ಹಿತ-ವಾದ
ಹಿನ್ನೆ-ಲೆ-ಯಲ್ಲಿ
ಹಿರಿ-ಯರ
ಹಿರಿ-ಯ-ರಾದ
ಹಿರಿ-ಯರು
ಹೀಗಿ-ರು-ವಾಗ
ಹೀಗೆ
ಹೀಗೆಂ-ದಾಗ
ಹೀಗೆಯೇ
ಹುಚ್ಚೆದ್ದು
ಹುಟ್ಟ-ದಂತೆ
ಹುಟ್ಟಿ-ಕೊ-ಳ್ಳಲು
ಹುಟ್ಟಿ-ಕೊ-ಳ್ಳು-ತ್ತದೆ
ಹುಟ್ಟಿ-ದಾಗ
ಹುಟ್ಟು-ತ್ತದೆ
ಹುಡುಕಿ
ಹುಡುಗ
ಹುಡು-ಗರ
ಹುಡು-ಗ-ರನ್ನು
ಹುಡು-ಗರು
ಹುಮ್ಮ-ಸ್ಸನ್ನೇ
ಹುಮ್ಮಸ್ಸು
ಹುರಿ-ದುಂ-ಬಿ-ಸು-ವು-ದರ
ಹುರುಪು
ಹುಲಿ-ಕ-ರ-ಡಿ-ಗಳಿಂದ
ಹೃತ್ಪೂ-ರ್ವಕ
ಹೃತ್ಪೂ-ರ್ವ-ಕ-ವಾಗಿ
ಹೃತ್ಪೂ-ರ್ವ-ಕ-ವಾ-ಗಿ-ರ-ಬೇಕು
ಹೃತ್ಪೂ-ರ್ವ-ಕ-ವಾ-ಗಿ-ರಲಿ
ಹೃದಯ
ಹೃದ-ಯ-ದಿಂದ
ಹೃದ-ಯ-ದೊ-ಳಕ್ಕೆ
ಹೃದ-ಯ-ದೊ-ಳಗೇ
ಹೃದ-ಯಲ್ಲೂ
ಹೃದ್ಗ-ತ-ಮಾ-ಡಿ-ಕೊ-ಳ್ಳ-ಬೇಕು
ಹೃದ್ಗ-ತ-ಮಾ-ಡಿ-ಕೊ-ಳ್ಳಲು
ಹೆಚ್ಚಾ-ಗಿ-ರ-ಬ-ಹುದು
ಹೆಚ್ಚಿ
ಹೆಚ್ಚಿಗೆ
ಹೆಚ್ಚಿನ
ಹೆಚ್ಚಿ-ಸಿ-ಕೊ-ಳ್ಳಲು
ಹೆಚ್ಚು
ಹೆಚ್ಚು-ಕ-ಡಿ-ಮೆ-ಯಾ-ದರೂ
ಹೆಚ್ಚು-ವುದನ್ನು
ಹೆಚ್ಚೆಚ್ಚು
ಹೆದ-ರ-ಬೇ-ಕಾ-ಗಿಯೇ
ಹೆದ-ರಿಕೆ
ಹೆದ-ರಿ-ಕೆಯ
ಹೆದ-ರಿ-ಕೆಯೇ
ಹೆದ-ರಿ-ಕೊಂ-ಡರೆ
ಹೆದ-ರಿ-ಕೊಂಡು
ಹೆದ-ರಿ-ಕೊಳ್ಳು
ಹೆಪ್ಪು-ಗ-ಟ್ಟಿ-ರು-ತ್ತದೆ
ಹೆಸ-ರಾ-ಗಿದೆ
ಹೆಸ-ರಿ-ಲ್ಲ-ದಂ-ತಾ-ಗು-ತ್ತದೆ
ಹೆಸರು
ಹೇ
ಹೇಗೆ
ಹೇಗೆಂ-ದರೆ
ಹೇರ-ಳ-ವಾಗಿ
ಹೇರ-ಳ-ವಾ-ಗಿವೆ
ಹೇಳ
ಹೇಳ-ಬೇ-ಕಾ-ಗಿಲ್ಲ
ಹೇಳ-ಬೇ-ಕಾ-ಗು-ತ್ತದೆ
ಹೇಳ-ಬೇಕು
ಹೇಳ-ಲಿಲ್ಲ
ಹೇಳಲು
ಹೇಳಲೆ
ಹೇಳಿ
ಹೇಳಿ-ಕೊ-ಳ್ಳಲಿ
ಹೇಳಿ-ಕೊಳ್ಳು
ಹೇಳಿ-ಕೊ-ಳ್ಳು-ವು-ದರ
ಹೇಳಿ-ಟ್ಟಿ-ರ-ಬೇಕು
ಹೇಳಿದ
ಹೇಳಿ-ದಂತೆ
ಹೇಳಿ-ದಾಗ
ಹೇಳಿ-ಬಿ-ಟ್ಟರೆ
ಹೇಳಿ-ಬಿ-ಡ-ಬೇ-ಕಾ-ಗು-ತ್ತದೆ
ಹೇಳಿ-ಬಿ-ಡು-ತ್ತೇನೆ
ಹೇಳಿ-ರು-ವಂತೆ
ಹೇಳು-ತ್ತಾನೆ
ಹೇಳು-ತ್ತಿದೆ
ಹೇಳು-ತ್ತೇನೆ
ಹೇಳುವ
ಹೇಳು-ವಂ-ತಿಲ್ಲ
ಹೇಳು-ವಂತೆ
ಹೇಳು-ವುದನ್ನು
ಹೇಳು-ವು-ದಾ-ದರೆ
ಹೇಳು-ವುದೂ
ಹೊಂದಿ-ಕೆ-ಯಿ-ದೆಯೇ
ಹೊಂದಿ-ಕೊ-ಳ್ಳದೆ
ಹೊಂದಿ-ಕೊ-ಳ್ಳ-ಬೇ-ಕೆಂ
ಹೊಂದಿ-ರ-ಬೇ-ಕಾ-ಗು-ತ್ತದೆ
ಹೊಂದು-ತ್ತೇನೆ
ಹೊಂದುವ
ಹೊಡೆ
ಹೊಡೆ-ತ-ವನ್ನು
ಹೊಡೆ-ದ-ದ್ದ-ಕ್ಕಿಂ-ತಲೂ
ಹೊಡೆದು
ಹೊಡೆ-ಯ-ಬೇ-ಕಾ-ದರೂ
ಹೊಡೆ-ಯುವ
ಹೊಡೆ-ಯು-ವ-ವರ
ಹೊಡೆ-ಯು-ವುದು
ಹೊತ್ತಿ-ಕೊ-ಳ್ಳು-ವುದು
ಹೊತ್ತಿಗೆ
ಹೊತ್ತಿಸಿ
ಹೊತ್ತು
ಹೊತ್ತು-ಗೊ-ತ್ತು-ಗ-ಳಿ-ಲ್ಲದೆ
ಹೊತ್ತೂ
ಹೊಮ್ಮಿ-ಸಿ-ಕೊ-ಳ್ಳ-ಬೇಕು
ಹೊರ
ಹೊರ-ಗಣ
ಹೊರ-ಗಿ-ನಿಂದ
ಹೊರಗೆ
ಹೊರಟ
ಹೊರ-ಟ-ಬ-ಹು-ಮುಖ್ಯ
ಹೊರ-ಟಿತು
ಹೊರ-ಟಿ-ಲ್ಲ-ವಲ್ಲ
ಹೊರಡ
ಹೊರ-ಡುವ
ಹೊರ-ತ-ರು-ತ್ತಿ-ದ್ದೇವೆ
ಹೊರತು
ಹೊರ-ಬ-ರಲು
ಹೊಲ-ಗಳಲ್ಲಿ
ಹೊಲ-ಗ-ಳಿಗೆ
ಹೊಳೆದೇ
ಹೊಳೆ-ಯು-ತ್ತದೆ
ಹೊಸ
ಹೊಸ-ದಾಗಿ
ಹೋಗ-ದಂತೆ
ಹೋಗ-ಬ-ಹುದು
ಹೋಗ-ಬಾ-ರದು
ಹೋಗ-ಬೇ-ಕಾದ
ಹೋಗ-ಬೇಕು
ಹೋಗ-ಬೇ-ಕು-ಎಂಬ
ಹೋಗ-ಬೇ-ಕೆಂ-ದಲ್ಲ
ಹೋಗ-ಬೇಡ
ಹೋಗ-ಲಾ-ಡಿ-ಸಿಕೊ
ಹೋಗ-ಲಾ-ಡಿ-ಸಿ-ಕೊ-ಳ್ಳು-ವುದು
ಹೋಗಲೇ
ಹೋಗ-ಲೇ-ಬಾ-ರದು
ಹೋಗಿ
ಹೋಗಿತ್ತು
ಹೋಗಿ-ರು-ತ್ತದೆ
ಹೋಗು
ಹೋಗು-ತ್ತದೆ
ಹೋಗು-ವ-ವರೆ-ಗಿನ
ಹೋಗುವು
ಹೋಟ-ಲು-ಗಳು
ಹೋದರೆ
ಹೋದ-ವ-ನೊಬ್ಬ
ಹೋದಾಗ
ಹೋರಾ-ಡ-ಬೇ-ಕಾ-ಗಿದೆ
ಹೋರಾಡಿ
}
