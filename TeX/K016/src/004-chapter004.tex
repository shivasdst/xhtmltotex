
\chapter{ಸಮಯಪ್ರಜ್ಞೆ}

ಇಂದು ಜಗತ್ತಿನಲ್ಲಿ ಹಣ ಮತ್ತು ಅದರಿಂದ ಪಡೆಯಬಹುದಾದ ವಸ್ತು –ಇವುಗಳಿಗೇ ಬೆಲೆ. ಇವುಗಳನ್ನು ನಿರ್ಮಾಣಮಾಡಿದವನು ಮನುಷ್ಯ. ಆದರೆ ಈ ಮನುಷ್ಯನೇ ಇಂದು ತನ್ನ ಮೌಲ್ಯವನ್ನು ಮರೆತು ಕುಳಿತಿದ್ದಾನೆ. ಇನ್ನೂ ಸೋಜಿಗದ ಸಂಗತಿಯೆಂದರೆ, ಈ ಮನುಷ್ಯ, ಆ ಹಣ–ಇವೆಲ್ಲ ಒಂದು ಅವಧಿಯಲ್ಲಿ ಕಾಣಿಸಿಕೊಂಡು ಕಣ್ಮರೆಯಾಗುವಂಥವು. ಈ ಅವಧಿಗೆ ಸಮಯ ವೆಂದು ಹೆಸರು. ಸಮಯ ಅಥವಾ ಕಾಲವೆಂಬುದು ಎಂಥ ಮಹತ್ವಪೂರ್ಣ ವಾದದ್ದು! ಅದನ್ನು ಅರಿಯಲು ಸಾಕಷ್ಟು ಆಲೋಚಿಸಿ ನೋಡಬೇಕು. ಆಗ ತಿಳಿಯುತ್ತದೆ \textbf{ಸಮಯವೇ ಸಂಪತ್ತು} ಎಂದು.

ಪ್ರತಿಯೊಬ್ಬನೂ ರಾತ್ರಿ ಮಲಗಿ ಬೆಳಗ್ಗೆ ಏಳುವ ವೇಳೆಗೆ ಇಪ್ಪತ್ತನಾಲ್ಕು ಗಂಟೆಗಳೆಂಬ ಸಂಪತ್ತನ್ನು, ದುಡಿಯದೆಯೇ ಪಡೆದಿರುತ್ತಾನೆ. ಇದನ್ನು ಸದುಪ ಯೋಗಪಡಿಸಿಕೊಂಡು ಮನುಷ್ಯ ಏನನ್ನು ಬೇಕಾದರೂ ಪಡೆಯಬಹುದು. ಸಮಯದ ಮೌಲ್ಯ-ಮಹತ್ತ್ವ ಅದೆಷ್ಟೆಂದರೆ ಅದನ್ನು ಸದುಪಯೋಗಪಡಿಸಿ ಕೊಂಡರೆ ಸಾಮ್ರಾಟನೇ ಆಗಿಬಿಡಬಹುದು!

ಸಮಯವನ್ನು ಸದುಪಯೋಗಪಡಿಸಿಕೊಳ್ಳಬೇಕೆಂಬವರಿಗೆ ಕೆಲವು ಸಲಹೆಗಳು:

೧. ಸಮಯವು ಅತ್ಯಂತ ಮಹತ್ವಪೂರ್ಣವಾದದ್ದು ಎಂಬ ಅರಿವಿರಲಿ.

೨. ಕಳೆದುಹೋದ ಕಾಲ ಮರಳಿ ಬಾರದು ಎಂಬ ಕಟುಸತ್ಯ ತಿಳಿದಿರಲಿ.

೩. ‘ಯಾವಯಾವ ಸಮಯದಲ್ಲಿ ಏನೇನನ್ನು ಮಾಡಬೇಕೆಂದು ಕೊಂಡಿರು ವೆನೋ ಅದನ್ನು ಮಾಡಿಯೇ ತೀರುತ್ತೇನೆ’ ಎಂಬ ದೃಢತೆಯಿರಲಿ.

೪. ನಿರ್ದಿಷ್ಟ ಕಾಲದಲ್ಲಿ ನಿರ್ದಿಷ್ಟ ಕಾರ್ಯವನ್ನು ಮಾಡುವ ಅಭ್ಯಾಸವಿದ್ದರೆ ‘ಈಗೇನು ಮಾಡಲಿ, ಇನ್ನೇನು ಮಾಡಲಿ’ ಎಂದು ತಲೆಕೆರೆದುಕೊಳ್ಳುವುದು ತಪ್ಪುತ್ತದೆ, ಸಮಯ ಉಳಿಯುತ್ತದೆ.

೫. ಕೈಗೊಳ್ಳುವ ಪುಟ್ಟ ಕಾರ್ಯದ ವಿಷಯದಲ್ಲೂ ಸ್ಪಷ್ಟ ಕಲ್ಪನೆಯಿರಲಿ.

೬. ಯಾವುದೇ ಕಾರ್ಯವನ್ನು ಉತ್ಸಾಹದಿಂದಲೂ ಎಚ್ಚರಿಕೆಯಿಂದಲೂ ಮಾಡದಿದ್ದರೆ ಸಮಯ ನಷ್ಟವಾಗುತ್ತದೆ ಎಂಬ ಸೂಕ್ಷ್ಮ ತಿಳಿದಿರಲಿ.

೭. ಮಾಡಿದ ಕೆಲಸವೊಂದು ಕೆಟ್ಟುಹೋದಾಗ ತಲೆಗೆ ಕೈಕೊಟ್ಟು ಚಿಂತಿಸುತ್ತ ಕುಳಿತುಕೊಳ್ಳುವುದರ ಬದಲು ಅದನ್ನು ಸರಿಪಡಿಸುವುದರ ಬಗ್ಗೆ ಆಲೋಚಿ ಸುವುದು ಅಥವಾ ಮುಂದಿನ ಕಾರ್ಯಕ್ಕೆ ಗಮನಹರಿಸುವುದು ಲಾಭದಾಯಕ.

೮. ಸಮಯೋಚಿತವಾಗಿ ಯೋಚಿಸಬಲ್ಲವನು ಎಲ್ಲ ವಿಷಯಗಳಲ್ಲೂ ಜಯಶಾಲಿಯಾಗುತ್ತಾನೆ.

\section{ಸಮಯದ ಕುರಿತು ಇನ್ನೂ ಕೆಲವು ಅಂಶಗಳು}

ಸಮಯವು ಹಣಕ್ಕಿಂತಲೂ ಶ್ರೇಷ್ಠ ಎನ್ನುವವರುಂಟು. ಆಲೋಚಿಸಿ ನೋಡಿದರೆ ಎರಡೂ ಶ್ರೇಷ್ಠ. ಸಮಯವನ್ನು ಸದುಪಯೋಗಪಡಿಸಿಕೊಂಡರೆ ಹಣ ಗಳಿಸಬಹುದು, ಹಣವನ್ನು ಸದುಪಯೋಗಪಡಿಸಿಕೊಂಡರೆ ಸಮಯ ಮಿಗಿಸಬಹುದು. ಆದರೆ ಸದುಪಯೋಗಪಡಿಸಿಕೊಳ್ಳುವ ಬುದ್ಧಿಬೇಕು ಅಷ್ಟೆ.

* ಹಗಲುಗನಸು ಕಾಣುತ್ತ ಆಕಾಶಗೋಪುರ ಕಟ್ಟುವವರ ಸಮಯ ನಿರರ್ಥಕ. ಸದ್ಭಾವನೆಗಳನ್ನು ತಾಳುತ್ತ ಸುಸಂಬದ್ಧವಾಗಿ ಆಲೋಚಿಸುವವರ ಸಮಯ ಸಾರ್ಥಕ.

ಒಂದೇ ಸಮನೆ ಹರಟೆ ಹೊಡೆಯುವವರ ಸಮಯ ನಿರರ್ಥಕ. ಉಪಯುಕ್ತ ಮಾತುಕತೆಗಳಲ್ಲಿ ತೊಡಗುವವರ ಸಮಯ ಸಾರ್ಥಕ.

ಜೂಜು-ಜುಗಾರಿ, ಇಸ್ಪೀಟು-ಲಾಟರಿ ಇವುಗಳಿಗೆ ಬಲಿಬಿದ್ದವರ ಸಮಯ ನಿರರ್ಥಕ. ಉಪಯುಕ್ತ ಕಾರ್ಯಗಳಲ್ಲಿ ನಿರತರಾದವರ ಸಮಯ ಸಾರ್ಥಕ.

* ಬುದ್ಧಿ ಚುರುಕಿಲ್ಲದವರ, ವಿಲಾಸದಲ್ಲಿ ಮೈ ಮರೆಯುವವರ, ಕಾಯಿಲೆ ಗಳಿಗೆ ಈಡಾದವರ, ಕೋರ್ಟು ಕಛೇರಿ ತಿರುಗುವವರ ಸಮಯವೆಲ್ಲ ಇಳಿಜಾರಿ ನಲ್ಲಿ ಹರಿದುಹೋಗುವ ನೀರಿನಂತೆ ಕಳೆದು ಹೋಗುತ್ತದೆ.

* ಹಣ ಕಳೆದುಹೋದರೆ ಬಾಯ್ಬಾಯಿ ಬಡಿದುಕೊಳ್ಳುವವರೇ ಎಲ್ಲ. ಸಮಯ ನಷ್ಟವಾಯಿತೆಂದು ದುಃಖಿಸುವವನು ಸಾವಿರಕ್ಕೊಬ್ಬನೂ ಇಲ್ಲ!

* ಸಮಯ ಬೇಕೆಂಬವರು ತಾವು ಮಾಡುವ ಪ್ರತಿಯೊಂದು ಕೆಲಸವನ್ನೂ ಸ್ವಲ್ಪ ಬೇಗ ಮಾಡಿ ಮುಗಿಸಲೆತ್ನಿಸಿ ಸಮಯ ಮಿಗಿಸಬೇಕು. ಎಂಟು ಗಂಟೆ ನಿದ್ರೆ ಮಾಡುವ ಬದಲು ಏಳೇ ಗಂಟೆ ನಿದ್ರೆ ಮಾಡುವುದರ ಮೂಲಕ, ಇಪ್ಪತ್ತು ನಿಮಿಷ ಕುಳಿತು ಊಟ ಮಾಡುವವರು ಹದಿನೈದೇ ನಿಮಿಷಕ್ಕೆ ಏಳುವುದರ ಮೂಲಕ, ಸ್ನಾನಕ್ಕೆ ಇಪ್ಪತ್ತು ನಿಮಿಷ ಹಿಡಿಯುತ್ತಿದ್ದರೆ ಹತ್ತೇ ನಿಮಿಷದಲ್ಲಿ ಮಾಡಿ ಮುಗಿಸುವುದರ ಮೂಲಕ ಸಮಯ ಮಿಗಿಸಬಹುದು. ಆದರೆ ಮಿಗಿಸಿದ ಸಮಯವನ್ನು ಸದುಪಯೋಗಪಡಿಸಿಕೊಳ್ಳುವುದು ಹೇಗೆ ಎನ್ನುವುದು ಮಾತ್ರ ತಿಳಿದಿರಬೇಕು.

* ಸಮಯ ಮಿಗಿಸಿ ಏನು ಮಾಡಬೇಕಾಗಿದೆ ಎನ್ನುವ ಭೂಪರೂ ಇದ್ದಾರೆ, ಇರಲಿ. ಆದರೆ ಮಿಗಿಸಿದ ಸಮಯವನ್ನು ಸದುಪಯೋಗಪಡಿಸಿಕೊಳ್ಳಬೇಕೆನ್ನು ವವರು ಮಾತ್ರ ಇಂಥವರಿಂದ ದೂರವಿರಬೇಕು!

* ಸಮಯವನ್ನು ನಷ್ಟಗೊಳಿಸುವಲ್ಲಿ ಮರವೆ ವಹಿಸುವ ಪಾತ್ರ ಬಹಳ ದೊಡ್ಡದು. ತನ್ನ ಕೋಣೆಗೆ ಬೀಗ ಹಾಕಿಕೊಂಡು ದೂರದ ಸ್ನೇಹಿತನ ಮನೆಗೆ ಹೋದವನೊಬ್ಬ ಅಲ್ಲಿ ಕಾಫಿ-ತಿಂಡಿ ತಿಂದು ಮಾತುಕತೆ ಮುಗಿಸಿ ಕೋಣೆಗೆ ಹಿಂದಿರುಗಿದ; ಬೀಗ ತೆಗೆಯ ಹೊರಟ. ಆದರೆ ಕೀಲಿಕೈಯೇ ಇಲ್ಲ! ಎಲ್ಲಿ?.....ಕೀಲಿಕೈ ಎಲ್ಲಿ? ಹಾ! ಸ್ನೇಹಿತನ ಮನೆಯಲ್ಲೇ ಬಿಟ್ಟು ಬಂದಿದ್ದಾನೆ! ಈ ಹಾಳಾದ ಮರವು! ಈಗ ಮತ್ತೆ ಅಷ್ಟು ದೂರ ಹೋಗಿ ಬರಬೇಕಾಯಿತು; ಸಮಯ ನಷ್ಟವಾಯಿತು.

* ಎಷ್ಟೋ ಜನರಿಗೆ ಕುಳಿತರೆ-ನಿಂತರೆ ಮರವು. ಈಗತಾನೆ ಊಟ ಮಾಡಿ ದ್ದನ್ನು ಇನ್ನೊಂದು ಘಳಿಗೆಯಲ್ಲಿ ಮರೆಯುವವರೂ ಉಂಟು. ಹಣದ ಅಗತ್ಯ ಬಿದ್ದಾಗ ಸ್ನೇಹಿತನ ನೆನಪಾಗಿ ಸಾಲ ತೆಗೆದುಕೊಂಡು ಹೋಗಿ, ಹಿಂದಿರುಗಿಸುವ ಸಮಯದಲ್ಲಿ ಮಾತ್ರ ಸುಲಭವಾಗಿ ಮರೆಯುವವರಿದ್ದಾರೆ. ಇದು ಮಾತ್ರ ಜಾಣತನದ ಮರವು! ಲಾಭದಾಯಕ ಮರವು!

ಮರವಿಗೆ ಕಾರಣವೇನಿರಬಹುದು? ಕಾರಣಗಳು ಬಹಳ. ಅವುಗಳಲ್ಲಿ ಎಚ್ಚರ ಗೇಡಿತನವೇ ಮುಖ್ಯ ಕಾರಣ.

ಎಚ್ಚರಗೇಡಿತನವನ್ನು ಹೋಗಲಾಡಿಸಿಕೊಳ್ಳುವುದು ಹೇಗೆ? ನಮಗೆ ನಾವೇ ಬುದ್ಧಿ ಹೇಳಿಕೊಳ್ಳುವುದರ ಮೂಲಕ. ಏನೆಂದು ಬುದ್ಧಿ ಹೇಳಬೇಕು? “ನೋಡು ನೀನು ಹೀಗೆ ಮರೆಯುತ್ತ ಹೋದರೆ ನಿನಗೆ ಕಷ್ಟ-ನಷ್ಟ-ಭ್ರಷ್ಟ ಕಟ್ಟಿಟ್ಟದ್ದು. ಆದ್ದರಿಂದ ಪ್ರತಿಯೊಂದು ವಿಷಯದಲ್ಲೂ ಎಚ್ಚರವಹಿಸುವುದನ್ನು ಅಭ್ಯಾಸ ಮಾಡಿಕೊ. ಸ್ವಲ್ಪ ಎಚ್ಚರಿಕೆಯಿಂದಿದ್ದರೆ ಎಷ್ಟು ಲಾಭವಿದೆ ಗೊತ್ತೆ?” ಎಂದು.

ಒಮ್ಮೆ ಒಬ್ಬ ಬೆಂಗಳೂರಿನಿಂದ ಬೊಂಬಾಯಿಗೆ ಹೊರಟ–ಬಹುಮುಖ್ಯ ಸಮ್ಮೇಳನವೊಂದರಲ್ಲಿ ಭಾಗವಹಿಸಲು. ಜೊತೆಯಲ್ಲಿ ತೆಗೆದುಕೊಂಡು ಹೋಗಬೇಕಾದ ಸಾಮಾನು ಸರಂಜಾಮುಗಳನ್ನೆಲ್ಲ ಎಣಿಸಿ ನೋಡಿದ–ಪೆಟ್ಟಿಗೆ, ಹಣ್ಣಿನ ಬುಟ್ಟಿ, ಕೈಚೀಲ ರೈಲುತಂಬಿಗೆ ಎಲ್ಲ ಸರಿಯಾಗಿದೆ. ಎಲ್ಲವನ್ನೂ ತೆಗೆದುಕೊಂಡು ದಡದಡ ನಡೆದ. ರೈಲು ಹತ್ತಿ ಕಿಟಕಿಯ ಬಳಿಯಿದ್ದ ತನ್ನ ಸ್ಥಾನದಲ್ಲಿ ಆರಾಮವಾಗಿ ಕುಳಿತ. ರೈಲು ಹೊರಟಿತು. ಇವನು ಕಿಟಕಿಯ ಹೊರಗೆ ಕಣ್ಣು ಹಾಯಿಸಿದ. ಓಡುವ ಗಾಡಿ ಕ್ಷಣಕ್ಷಣಕ್ಕೂ ಹೊಸ ಹೊಸ ದೃಶ್ಯಗಳನ್ನು ಒದಗಿಸಿ ಕೊಡುತ್ತಿತ್ತು. ನಗರದ ಕಟ್ಟಡಗಳನ್ನೇ, ಕಟ್ಟಡಗಳ ಗೋಡೆಗಳನ್ನೇ ನೋಡಿದ್ದ ಅವನ ಕಣ್ಣುಗಳಿಗೆ ಹಬ್ಬ. ಅವನ ಮನಸ್ಸು ಪ್ರಕೃತಿಯ ಅನಂತತೆಯನ್ನು ಅನುಭವಿಸುತ್ತ, ಅದರಲ್ಲೇ ಲೀನವಾಯಿತು. ಆಗ ಇದ್ದಕ್ಕಿದ್ದಂತೆ ಒಂದು ಧ್ವನಿ ಕೇಳಿಸಿತು:

“ಟಿಕೆಟ್ ತೋರಿಸ್ರೀ.”

ಟಿಕೆಟ್ ಕಲೆಕ್ಟರನ ಗೊಗ್ಗರು ಧ್ವನಿ ಅದು. ಈತ ವಾಸ್ತವಿಕ ಸ್ಥಿತಿಗೆ ಇಳಿದು ಜೇಬಿಗೆ ಕೈಹಾಕಿದ ಟಿಕೆಟ್ ತೆಗೆಯುವುದಕ್ಕೆ. ಆದರೆ! ಆದರೆ! ಜೇಬಿನಲ್ಲಿ ಟಿಕೆಟ್ ಇಲ್ಲ! ಎಲ್ಲಿ? ಟಿಕೆಟ್ ಎಲ್ಲಿ? 

“ಓಹ್! ಮರೆತೇಬಿಟ್ಟೆ. ಟಿಕೆಟು ಮನೆಯಲ್ಲೇ ಉಳಿದುಬಿಟ್ಟಿದೆ ಸರ್! ಭದ್ರವಾಗಿರಲಿ ಅಂತ ಬೀರುವಿನಲ್ಲಿಟ್ಟಿದ್ದೆ. ಹೊರಡುವ ಗಡಿಬಿಡಿಯಲ್ಲಿ ಮರೆತುಬಿಟ್ಟೆ. ಸಾರಿ ಸರ್!”

“ಸಾರೀನೂ ಇಲ್ಲ ಗೀರೀನೂ ಇಲ್ಲ, ಏನ್ರಿ, ನನ್ಹತ್ರ ನಾಟಕ ಆಡ್ತಿದ್ದೀ ರೇನ್ರಿ?”

“ಇಲ್ಲ ಸರ್! ಖಂಡಿತವಾಗಿಯೂ ಟಿಕೆಟ್ ಮನೇಲಿದೆ ಸರ್!”

“ನೋಡಿ, ಅದೆಲ್ಲ ಆಗೋದಿಲ್ಲ. ಒಂದೋ ನೀವು ದಂಡ ತೆರಬೇಕು. ಇಲ್ಲವಾದರೆ ಮುಂದಿನ ನಿಲ್ದಾಣದಲ್ಲಿ ಇಳಿಯಿರಿ.”

ಆತನ ಬಳಿ ದಂಡ ತೆರುವಷ್ಟು ಹಣ ಇಲ್ಲದ್ದರಿಂದ ಮುಂದಿನ ನಿಲ್ದಾಣ ದಲ್ಲಿ ಇಳಿದ. ಆದರೂ ಸ್ವಲ್ಪ ದಂಡವನ್ನಂತೂ ತೆರಲೇಬೇಕಾಯಿತು. ಬಳಿಕ ಹೊಸದಾಗಿ ಟಿಕೆಟ್ ಕೊಂಡು ಮುಂದಿನ ರೈಲಿನಲ್ಲಿ ಮುಂದುವರಿದ. ಆದರೆ ಬೊಂಬಾಯಿಗೆ ತಲುಪುವಷ್ಟರಲ್ಲಿ ಸಮ್ಮೇಳನದ ಬಹುಭಾಗ ಮುಗಿದು ಹೋಗಿತ್ತು. ಒಂದು ಸಣ್ಣ ಮರವೆಯಿಂದಾಗಿ ಈ ಕಷ್ಟ-ನಷ್ಟ-ಭ್ರಷ್ಟ! ಸಮ್ಮೇಳನಕ್ಕೂ ಸಮಯಕ್ಕೆ ಸರಿಯಾಗಿ ಬರಲಾಗಲಿಲ್ಲ.

ಎಷ್ಟೋ ಸಲ ಸಮಯವನ್ನು ವ್ಯರ್ಥವಾಗಿ ಕಳೆಯಲೇ ಬೇಕಾದ ಸಂದರ್ಭ ಗಳು ಬರುವುದೂ ಉಂಟು. ಉದಾಹರಣಗೆ ಪ್ರಯಾಣಕಾಲ, ಕ್ಯೂಗಳಲ್ಲಿ ನಿಂತಿದ್ದಾಗ, ಯಾರಿಗಾದರೂ ಕಾಯುವಾಗ, ಶಾಲಾ ಕಾಲೇಜಿನಲ್ಲಿ ‘ಲೆಟ್ ಆಫ್​’ ಕೊಟ್ಟಾಗ, ಇತ್ಯಾದಿ.

ಬಸ್ಸು-ರೈಲುಗಳಲ್ಲಿ ದೂರದ ಊರಿಗೆ ಪ್ರಯಾಣ ಮಾಡುವಾಗ ಉಪಾಯ ವಾಗಿ ಕಿಟಕಿಯ ಪಕ್ಕದಲ್ಲಿ ಜಾಗ ಮಾಡಿಕೊಂಡರೆ ದಾರಿಯುದ್ದಕ್ಕೂ ಕಾಣ ಸಿಗುವ ವಿಭಿನ್ನ ವಿಚಿತ್ರ ನಿಸರ್ಗಸೌಂದರ್ಯವನ್ನು ವೀಕ್ಷಿಸುತ್ತ ಆಸ್ವಾದಿಸ ಬಹುದು. ವ್ಯಕ್ತಿ ಆಧ್ಯಾತ್ಮಿಕ ಮನೋಭಾವದವರಾದರೆ ಭಗವನ್ನಾಮಜಪ ವನ್ನೋ ಪಾರಾಯಣವನ್ನೋ ಮಾಡಬಹುದು. ಇನ್ನು ಕೆಲವರು ನಿದ್ರೆ ಹೊಡೆ ಯುವವರೂ ಇದ್ದಾರೆ. ಅವರ ಪಾಲಿಗೆ ಅದೇ ಸಮಯದ ಸದುಪಯೋಗ! ವಿಮಾನ ಪ್ರಯಾಣಿಕರಿಗೆ ಪ್ರಕೃತಿ ಸೌಂದರ್ಯವನ್ನು ವೀಕ್ಷಿಸುವ ಭಾಗ್ಯವಿಲ್ಲ. ಅವರಿಗೆ ಪುಸ್ತಕ ಓದುವ ಅಭ್ಯಾಸವಿದ್ದರೆ ಒಳ್ಳೆಯದು.

ಕ್ಯೂಗಳಲ್ಲಿ ನಿಂತಿರುವಾಗ ಜೊತೆಯಲ್ಲಿ ನಿಂತಿರುವವರು ಸಮಾನ ಮನಸ್ಕ ರಾದರೆ ಅವರೊಂದಿಗೆ ಉಪಯುಕ್ತ ಸಂಭಾಷಣೆಯಲ್ಲಿ ತೊಡಗಬಹುದು. ಅಥವಾ ಓದಿ ತಿಳಿದ ಒಂದೆರಡು ಉದಾತ್ತ ವಿಚಾರಗಳನ್ನು ನೆನಪಿಗೆ ತಂದು ಕೊಂಡು ಮೆಲಕು ಹಾಕಬಹುದು.ಇಲ್ಲವೆ ನೆನಪಿರುವ ಕೆಲವು ಶ್ಲೋಕಗಳ ಮೇಲೆ ಮನನ ಮಾಡಬಹುದು. ಕೆಲವೊಮ್ಮೆ ಯಾರಿಗಾದರೂ ಕಾಯುತ್ತ ಕುಳಿತು ಕೊಳ್ಳುವ ಸಂದರ್ಭ ಬರುವುದುಂಟು. ಆಗ ಕಾಯುವವರ ಮನಸ್ಸಿನಲ್ಲಿ ‘ಅವರು ಯಾಕಿನ್ನೂ ಬರಲಿಲ್ಲ?–ಎಂಬ ಕಾತರತೆ ಆರಂಭವಾಗುತ್ತದೆ. ಆದರೆ ಈ ಕಾತರೆಯಿಂದೇನೂ ಪ್ರಯೋಜನವಿಲ್ಲ. ಏಕೆಂದರೆ, ನಾವು ಕಾತರಗೊಂಡೆ ವೆಂದು ಅವರೇನೂ ಬೇಗ ಬರುವುದಿಲ್ಲ. ಅಲ್ಲದೆ, ಅವರು ತಡವಾಗುವುದಕ್ಕೆ ಏನಾದರೂ ಕಾರಣವಿದ್ದರೂ ಇರಬಹುದು; ಅಥವಾ ಕಾರಣಾಂತರಗಳಿಂದ ಅವರು ಅಂದು ಬರದೆಯೇ ಹೋಗಬಹುದು. ಆಗ ಸಮಯ ನಷ್ಟವಾಯಿತೆ! ನಾವು ಸುಮ್ಮನೆ ಕಾತರಪಟ್ಟದ್ದೆಲ್ಲ ವ್ಯರ್ಥವಾಯಿತೆ! ಕಾಯುವುದೇನೂ ತಪ್ಪುವುದಿಲ್ಲವಾದ್ದರಿಂದ ಕಾಯೋಣ; ಕಾತರಗೊಳ್ಳದೆ ಮತ್ತು ಕಾಲ ವ್ಯರ್ಥ ವಾಗದಂತೆ ಏನಾದರೊಂದು ಕಾರ್ಯದಲ್ಲಿ ನಿರತರಾಗಿರೋಣ!

ವಿದ್ಯಾರ್ಥಿಗಳ ವಿಷಯದಲ್ಲಿ ಹೇಳುವುದಾದರೆ, ಅವರ ಸಮಯ ವ್ಯರ್ಥ ವಾಗುವ ಸಂದರ್ಭಗಳು ಬಹಳ. ಒಂದೇ ಒಂದು ಉದಾಹರಣೆ–‘ಲೆಟ್ ಆಫ್​’ಗಳು. ಅನಿರೀಕ್ಷಿತವಾಗಿ ಹೀಗೆ ಮತ್ತೆ ಮತ್ತೆ ದೊರಕುವ ಬಿಡುವಿನ ವೇಳೆಯನ್ನು ಮಿಂಚಿನಂತೆ ತಲೆಯೋಡಿಸಿ ಸದುಪಯೋಗಪಡಿಸಿಕೊಳ್ಳಲು ವಿದ್ಯಾರ್ಥಿಗಳು ಮೊದಲೇ ಸಿದ್ಧರಾಗಿರಬೇಕಾಗುತ್ತದೆ. 

ಸಮಯವನ್ನು ‘ವ್ಯರ್ಥ’ವಾಗಿ ಕಳೆಯಲೇ ಬೇಕಾದ ಸಂದರ್ಭಗಳನ್ನು ಸದುಪಯೋಗಪಡಿಸಿಕೊಳ್ಳಲು ಕೆಲವು ಸಲಹೆಗಳನ್ನು ಮಾತ್ರ ಇಲ್ಲಿ ಇಡ ಲಾಗಿದೆ. ಆದರೆ ಅವರವರೇ ಬುದ್ಧಿ ಉಪಯೋಗಿಸಿ ಇನ್ನೂ ಹಲವು ಉಪಾಯ ಗಳನ್ನು ಕಂಡುಕೊಳ್ಳಬೇಕು. ಸುಮ್ಮನೆ ಸೋರಿಹೋಗುವ ಸಮಯದ ಬಗ್ಗೆ ಒಂದು ಅರಿವನ್ನು ಮೂಡಿಸುವ ಪ್ರಯತ್ನ ಮಾಡಲಾಗಿದೆ ಇಲ್ಲಿ.

ಇನ್ನು ನಮ್ಮ ಸ್ನೇಹಿತರು, ಬಂಧುಗಳು, ಪರಿಚಯಸ್ಥರು ಇದ್ದಾರೆ; ಇರಬೇಕಾದ್ದೇ. ಆದರೆ ಇವರೆಲ್ಲ ತಮ್ಮ ಅತಿವಿಶ್ವಾಸದಿಂದ ನಮ್ಮ ಅಮೂಲ್ಯ ಸಮಯವನ್ನು ತಿಂದುಹಾಕದಂತೆ ಒಂದು ಕಣ್ಣಿಟ್ಟಿರಬೇಕಾಗುತ್ತದೆ. ಹಾಗೆಯೇ ಅವರ ಸಮಯವೂ ನಮ್ಮಿಂದಾಗಿ ನಷ್ಟವಾಗುವಂತಾದರೆ ನ್ಯಾಯವಲ್ಲ.

ಇಂಗ್ಲೀಷಿನಲ್ಲಿ \eng{‘Man-hour’}ಎಂಬ ಪದವಿದೆ. ಯೋಚಿಸಿ ನೋಡಿದರೆ ಇದೊಂದು ಅತ್ಯಂತ ಪ್ರಾಮುಖ್ಯವಾದ ಅಂಶ. \eng{Man-power} ಇರುವಂತೆ \eng{Man-hour.} ಅಂದರೇನು?–ದಿನದ ಇಪ್ಪತ್ತನಾಲ್ಕು ಗಂಟೆಗಳು ಪ್ರತಿ ಯೊಬ್ಬನ ಸೊತ್ತು ಎನ್ನುವ ವಿಷಯಕ್ಕೆ ಎದುರುಮಾತಿಲ್ಲ ತಾನೆ? ಹಾಗೆಯೇ ಕೋಟ್ಯಂತರ ಜನ ಈ ಸಮಯವನ್ನು ಉಪಯೋಗಿಸಿಕೊಳ್ಳುತ್ತಿದ್ದರೂ ದಿನದ ಗಂಟೆಗಳು ಮಾತ್ರ ಇಪ್ಪತ್ತನಾಲ್ಕೇ ಎಂಬುದೂ ದಿಟತಾನೆ? ಆದರೆ, ವಿಚಾರ ಮಾಡಿ ನೋಡಿದಾಗ ತಿಳಿಯುತ್ತದೆ, ಒಬ್ಬನಿಗೆ ಇಪ್ಪತ್ತನಾಲ್ಕು ಗಂಟೆಯಂತೆ ಇಬ್ಬರಿಗೆ ನಲವತ್ತೆಂಟು ಗಂಟೆ, ನೂರು ಮಂದಿಗೆ ಎರಡು ಸಾವಿರದ ನಾನ್ನೂರು ಗಂಟೆ! ಆದ್ದರಿಂದ, ಸಾವಿರ ಜನ ಸೇರಿರುವ ಸಾರ್ವಜನಿಕರ ಸಭೆಯೊಂದಕ್ಕೆ ಮುಖ್ಯ ಅತಿಥಿಗಳೋ, ಮಾನ್ಯ ಮಂತ್ರಿಗಳೋ ಒಂದು ಗಂಟೆ ತಡವಾಗಿ ಬಂದರೆಂದರೆ ಒಂದು ಸಾವಿರ ಗಂಟೆಗಳನ್ನು ಅವರು ‘ಕೊಲೆ’ಗೈದರೆಂದೇ ಅರ್ಥ. ಮಾನವ ಇತಿಹಾಸವನ್ನೊಮ್ಮೆ ಭಾವಿಸಿ ನೋಡಿದರೆ ಈ ರೀತಿಯಾಗಿ ಕಾಲನಷ್ಟ ಮಾಡಿಕೊಂಡವರೆಷ್ಟೋ! ಆದರೆ ಅವರೆಲ್ಲ ‘ಕಾಲವಶ’ರಾಗಿಬಿಟ್ಟ ರೆನ್ನಿ! ಈಗಲಾದರೂ ನಾವು ಸ್ವಲ್ಪ ಎಚ್ಚರ ವಹಿಸಿ ಬುದ್ಧಿ ಉಪಯೋಗಿಸಿ ಕಾಲವನ್ನು ವಶದಲ್ಲಿಟ್ಟುಕೊಳ್ಳುವ ತಂತ್ರವನ್ನು ಕಂಡುಕೊಳ್ಳಬೇಕು. ತನ್ಮೂಲಕ ಮಹತ್ಕಾರ್ಯಗಳನ್ನು ಸಾಧಿಸಬೇಕು. ಮಹಾನ್ ವ್ಯಕ್ತಿಗಳಾಗಬೇಕು.

