
\chapter*{ಮುನ್ನುಡಿ}

‘ಇಂದಿನ ಮಕ್ಕಳು ಮುಂದಿನ ಪ್ರಜೆಗಳು!’ ‘ಇಂದಿನ ಯುವಜನರೇ ನಾಳಿನ ನೇತಾರರು!’ ‘ನೀವೇ ಈ ದೇಶವನ್ನು ಮುಂದೆ ನಡಸಿ ನಮ್ಮನ್ನು ರಕ್ಷಿಸುವವರು!’ ಎಂದೆಲ್ಲ ನಾವು ನಮ್ಮ ಮಕ್ಕಳಿಗೆ, ಯುವಜನರಿಗೆ, ಉದ್ದುದ್ದನೆ ಉಪದೇಶಗಳನ್ನು ನೀಡುತ್ತಿದ್ದೇವೆ. ಆದರೆ ಅವರು ಆ ರೀತಿ ಬೆಳೆದು ಜವಾಬುದಾರಿಯುತ ಸ್ಥಾನಗಳನ್ನು ತುಂಬಲು ಬೇಕಾದ ವಿದ್ಯೆಯನ್ನಾಗಲೀ ಶಿಕ್ಷಣವನ್ನಾಗಲೀ ಕೊಡುತ್ತಿಲ್ಲ. ತದ್ವಿಪರೀತವಾಗಿ ಅವರು ಸಾಕಷ್ಟು ಕೆಟ್ಟು ಇತರರನ್ನು ಕೆಡಿಸಲು ಅನುಕೂಲವಾದಂತಹ ವಾತಾವರಣವನ್ನು ನಮ್ಮ ಅವಿವೇಕದಿಂದ ದುರಾಶೆಯಿಂದ ಇಂದ್ರಿಯ ಲೋಲುಪತೆಯಿಂದ ನಿರ್ಮಿಸುತ್ತಿದ್ದೇವೆ.

ಆದರೆ ಅದೃಷ್ಟವಶಾತ್ ಇಂದಿಗೂ ನಮ್ಮ ಮಕ್ಕಳಲ್ಲಿ, ತರುಣ-ತರುಣಿಯರಲ್ಲಿ, ಪೂರ್ವಜ ಪುಷಿಗಳ ರಕ್ತ ಇನ್ನೂ ಹಸಿಯಾಗಿ ಹರಿಯುತ್ತಿದೆ. ಸದವಕಾಶಗಳನ್ನು ಒದಗಿಸಿದಾಗಲೆಲ್ಲ ಅದು ಅಭಿವ್ಯಕ್ತವಾಗಿ ಅದ್ಭುತ ಕಾರ್ಯವನ್ನೆಸಗುತ್ತಿದೆ. ಇಂತಹ ಸದವಕಾಶವನ್ನು ನಿರ್ಮಿಸಿಕೊಡುವುದು ಹಿರಿಯರಾದ ನಮ್ಮ ಕರ್ತವ್ಯ. ಈ ಸಂಪುಟವು ಅಂತಹ ಸದವಕಾಶವನ್ನು ಸರಳ ಸುಲಭ ಭಾಷೆಯಲ್ಲಿ, ಎಲ್ಲರ ತಲೆಗೂ ಹತ್ತುವಂತಹ ಭಾಷೆಯಲ್ಲಿ, ಒದಗಿಸುತ್ತಿದೆ.

ಇಂದಿನ ವಿದ್ಯಾರ್ಥೀ ಯುವಜನರ ಅತ್ಯಂತ ಕಾತರದ ಸಮಸ್ಯೆಯೆಂದರೆ ತಮ್ಮ ಅಧ್ಯಯನ ದಲ್ಲಿ ಹೇಗೆ ಏಕಾಗ್ರತೆಯನ್ನು ಸಂಪಾದಿಸಿ ಓದಿದ್ದನ್ನೆಲ್ಲ ನೆನಪಿನಲ್ಲಿಟ್ಚುಕೊಳ್ಳುವುದು ಎಂಬುದು. ಇಂತಹ ಏಕಾಗ್ರತಾಸೂತ್ರಗಳನ್ನು ‘ಅಧ್ಯಯನದಲ್ಲಿ ಏಕಾಗ್ರತೆ’ ಎಂಬ ಪ್ರಥಮ ಲೇಖನವು ವಿವರಿಸುತ್ತದೆ. ಆದರೆ ಇದರ ಹಿಂದೆ ಜೀವನವನ್ನು ಸುವ್ಯವಸ್ಥಿತ ರೀತಿಯಲ್ಲಿ ನಡೆಸುವ ಕ್ರಮದ ಜ್ಞಾನವಿರಬೇಕು. ಅದನ್ನು ಎರಡನೆಯ ಲೇಖನವಾದ ‘ವಿದ್ಯಾರ್ಥಿಗೊಂದು ಪತ್ರ’ದಲ್ಲಿ ಹೇಳಿ ಕೊಟ್ಟಿದೆ. ಈ ಎರಡಕ್ಕೂ ಆಧಾರವಾಗಿ ‘ವಿದ್ಯೆ’ ಎಂಬ ಶಾಸ್ತ್ರದ ಅಥವಾ ‘ದರ್ಶನ’ದ ಮೂಲ ಸೂತ್ರಗಳನ್ನು ವಿಸ್ತಾರವಾಗಿಯೇ ಚರ್ಚಿಸಿದೆ ‘ವಿದ್ಯೆಯ ವೈಭವ’ದಲ್ಲಿ. ಇವೆಲ್ಲಕ್ಕೂ ಸಹಕಾರಿ ಗಳಾದ ಸಮಯಪ್ರಜ್ಞೆಯನ್ನು ಮತ್ತು ಪ್ರಾರ್ಥನೆ ಮಾಡಿ ಈಶಶಕ್ತಿಯಿಂದ ಸ್ಫೂರ್ತಿಯನ್ನು ಪಡೆಯುವ ವಿಧಾನವನ್ನು ಕಡೆಯ ಎರಡು ಲಘುಪ್ರಬಂಧಗಳಲ್ಲಿ ಚಿತ್ರಿಸಿದೆ.

ಯುವಜನರ ಸಂಪರ್ಕವನ್ನು ಹತ್ತುಹಲವು ವರ್ಷಗಳಿಂದ ಬೆಳೆಸಿಕೊಂಡು ಬಂದಿರುವ, ಅವರ ನಾಡಿಯ ಲಕ್ಷಣವನ್ನು ಚೆನ್ನಾಗಿ ಕರಗತ ಮಾಡಿಕೊಂಡಿರುವ, ಸ್ವಾಮಿ ಪುರುಷೋತ್ತಮಾನಂದರು ಈಗಾಗಲೇ ಈ ಲೇಖನಗಳನ್ನು ವಾಮನಾಕಾರದ ಪುಸ್ತಿಕೆಗಳ ರೂಪದಲ್ಲಿ ನಿಮ್ಮ ಮುಂದಿಟ್ಟಿದ್ದರು. ಅವುಗಳ ಲಕ್ಷಾಂತರ ಪ್ರತಿಗಳು ಇದುವರೆಗೆ ಖರ್ಚಾಗಿವೆ. ಈಗ ಅವೆಲ್ಲವನ್ನೂ ಸೇರಿಸಿ ತ್ರಿವಿಕ್ರಮ ನಂತಲ್ಲದಿದ್ದರೂ, ನಿಮ್ಮ ಬುದ್ಧಿಗೆ ಮತ್ತು ಜೇಬಿಗೆ ನಿಲುಕಬಲ್ಲ ಅಲ್ಪವಿಕ್ರಮನಾಗುವಂತೆ ಹೊರತರುತ್ತಿದ್ದೇವೆ. ಇದೀಗ ಒಂದು ಆಕರಗ್ರಂಥ; ನಿಮ್ಮೆಲ್ಲರ ಮೇಜಿನ ಮೇಲೂ ಸದಾ ಇರಬೇಕಾದದ್ದು. ಇದನ್ನು ಬಹುಸಂಖ್ಯೆಯಲ್ಲಿ ಕೊಂಡು ನಿಮ್ಮ ಮಿತ್ರರಿಗೂ ಹಂಚಿ!

\titleauthor{ಸ್ವಾಮಿ ಹರ್ಷಾನಂದ}

