
\chapter{ಅನುಬಂಧ}

\textbf{ಪ್ರಶ್ನೆ:} ಈ ಪತ್ರದಲ್ಲಿ ವೇಳಾ ಪಟ್ಟಿಯ ತಂತ್ರವನ್ನು ವಿವರಿಸಿ ಸಮಯ ವನ್ನು ಸದುಪಯೋಗ ಪಡಿಸಿಕೊಳ್ಳುವ ವಿಧಾನವನ್ನೂ ತಿಳಿಸಿರುವಿರಿ. ಆದರೆ ಸಾಕಷ್ಟು ಅಧ್ಯಯನ ಮಾಡಿದ ಮೇಲೂ ಮಿಗುವ ಸಮಯವನ್ನು ಬಳಸಿ ಕೊಳ್ಳುವುದು ಹೇಗೆ?

\textbf{ಉತ್ತರ:} ಇದು ಬುದ್ಧಿವಂತ ವಿದ್ಯಾರ್ಥಿಗಳ ಸಮಸ್ಯೆ. ಏಕೆಂದರೆ ಅವರು ಮಾತ್ರ ಕಡಿಮೆ ಸಮಯದಲ್ಲಿ ಹೆಚ್ಚು ಓದಬಲ್ಲರಲ್ಲದೆ ಓದಿದ್ದನ್ನು ನೆನಪಿಟ್ಟು ಕೊಳ್ಳಬಲ್ಲರು. ಅಂಥವರು ಚಿತ್ರಕಲೆ, ಸಂಗೀತ, ನೃತ್ಯ, ಕೈಗಾರಿಕೆ, ಸಾಹಿತ್ಯವೇ ಮೊದಲಾದ ವಿಷಯಗಳಲ್ಲಿ ತಮಗಿಷ್ಟವಾದುದನ್ನು ಆರಿಸಿಕೊಂಡು ಅಭ್ಯಸಿಸ ಬಹುದು. ಆದರೆ, ಮೂರನೇ ಶ್ರೇಣಿಯಲ್ಲಿ ಉತ್ತೀರ್ಣರಾಗುವ ಅಥವಾ ನಪಾಸೇ ಆಗಬಹುದಾದ ವಿದ್ಯಾರ್ಥಿಗಳು ತಮ್ಮ ಪಾಠಪಟ್ಟಿಗಳ ವಿಷಯ ಗಳಲ್ಲೇ ಹೆಚ್ಚು ಮಗ್ನರಾಗಿರುವುದು ಒಳ್ಳೆಯದು. ಹೀಗೆ ಮಾಡಿದರೆ, ನಪಾಸಾಗುವವರು ಮೂರನೇ ಶ್ರೇಣಿಯಲ್ಲಾದರೂ ಉತ್ತೀರ್ಣರಾದಾರು; ಮೂರನೇ ಶ್ರೇಣಿಯಲ್ಲಿ ಪಾಸಾಗುವವರು ಎರಡನೇ ಶ್ರೇಣಿಯಲ್ಲಿ ಪಾಸಾದಾರು. ಎರಡನೇ ಶ್ರೇಣಿಯಲ್ಲಿ ಪಾಸಾಗುವವರು ಪ್ರಥಮ ಶ್ರೇಣಿಯಲ್ಲಿ ಪಾಸಾದಾರು. ಕನಿಷ್ಠ ಪಕ್ಷ ಪ್ರಥಮ ಶ್ರೇಣಿಯಲ್ಲಾದರೂ ಪಾಸಾದರೆ ಮುಂದೆ ಕಾಲೇಜು ವಿದ್ಯಾಭ್ಯಾಸ ಸುಲಭ ಸುಗಮವಾಗಲು ಸಾಧ್ಯವಾಗುತ್ತದೆ. ಎಸ್ಸೆಸ್ಸೆಲ್ಸಿ ಯಲ್ಲೇ ಮೂರನೇ ಶ್ರೇಣಿಯಲ್ಲಿ ಪಾಸಾಗುವವರು ಕಾಲೇಜಿನಲ್ಲಿ ಡುಮ್ಕಿ ಹೊಡೆಯುವುದು ಖಂಡಿತ.

\textbf{ಪ್ರಶ್ನೆ:} ಸಾಮಾನ್ಯ ಮಟ್ಟದ ವಿದ್ಯಾರ್ಥಿಗಳು ಯಾವಾಗಲೂ ಅಧ್ಯಯನ ದಲ್ಲೇ ತೊಡಗಿರುವಂತಾದರೆ ಕಲೆ-ಕೈಗಾರಿಕೆಗಳನ್ನು ಕಲಿತುಕೊಳ್ಳಲು ಅವರಿಗೆ ಅವಕಾಶ ಇಲ್ಲವೆಂದಂತಾಯಿತು?

\textbf{ಉತ್ತರ:} ಏಕಿಲ್ಲ! ನವರಾತ್ರಿಯ ರಜಾ ದಿನಗಳಲ್ಲಿ, ಸೆಕೆಗಾಲದ ರಜಾದಿನ ಗಳಲ್ಲಿ ಅವರು ತಮ್ಮ ಅಭಿರುಚಿಯ ವಿಷಯವೊಂದನ್ನು ಅಭ್ಯಸಿಸಬಹುದಲ್ಲ.

\textbf{ಪ್ರಶ್ನೆ:} ಎರಡು ಹೊತ್ತೂ ಸ್ನಾನವೆಂದಿರಿ. ವಿದ್ಯಾರ್ಥಿನಿಯರಿಗೆ ದಿನಾಲೂ ತಲೆಸ್ನಾನ ಅಸಾಧ್ಯ. ಕೂದಲನ್ನು ಒಣಗಿಸಿಕೊಳ್ಳುವುದೊಂದು ದೊಡ್ಡ ಕೆಲಸವೇ ಆಗಿಬಿಡುತ್ತದೆ.

\textbf{ಉತ್ತರ:} ಈ ಪತ್ರವನ್ನು ವಿದ್ಯಾರ್ಥಿಯನ್ನುದ್ದೇಶಿಸಿ ಬರೆದುದರಿಂದ ವಿದ್ಯಾರ್ಥಿನಿಯರ ತಲೆಸ್ನಾನದ ಸಮಸ್ಯೆ ಎದ್ದಿರಲಿಲ್ಲ. ವಿದ್ಯಾರ್ಥಿನಿಯರು ವಾರಕ್ಕೊಮ್ಮೆ ತಲೆಸ್ನಾನ ಮಾಡಿದರೆ ಸಾಕು. ಚೆನ್ನಾಗಿ ಎಣ್ಣೆ ಹಚ್ಚಿಕೂದಲನ್ನು ಪ್ರತಿದಿನ ಬಾಚಿಕೊಳ್ಳುವುದು ಇದ್ದೇ ಇದೆಯಲ್ಲ!

\delimiter

ಈ ಪತ್ರವನ್ನೋದಿದಾಗ ಇನ್ನೂ ಕೆಲವು ಸಣ್ಣಪುಟ್ಟ ಪ್ರಶ್ನೆಗಳು ಏಳ ಬಹುದೇನೋ. ಉದಾಹರಣೆಗೆ: ಜ್ವರವೇ ಮೊದಲಾದ ಕಾಯಿಲೆಗಳು ಬಂದಾಗ ಅಥವಾ ಪರವೂರುಗಳಿಗೆ ಹೋದಾಗ ವೇಳಾಪಟ್ಟಿಯಂತೆ ನಡೆದುಕೊಳ್ಳಲಾಗು ವುದಿಲ್ಲ; ಆಗ ಪಾಠಪಟ್ಟಿಯ ಎಷ್ಟೋ ಅಂಶಗಳು ಅಧ್ಯಯನವಾಗದೆ ಬಾಕಿ ಉಳಿದುಬಿಡುತ್ತವೆ; ಆಗೇನು ಮಾಡುವುದು? ಎಂದು. ಇಂತಿಂಥ ಪ್ರಶ್ನೆ ಗಳಿಗೆಲ್ಲ ಅವರವರೇ ಉತ್ತರ ಕಂಡುಕೊಳ್ಳಬೇಕು. ಮನಸ್ಸಿದ್ದರೆ ಮಾರ್ಗವಿದೆ. ಅಧ್ಯಯನ ಮಾಡುವ ಪ್ರಬಲ ಹಂಬಲವೊಂದಿದ್ದುಬಿಟ್ಟರೆ ಸಮಸ್ಯೆಯೇ ಏಳು ವುದಿಲ್ಲ, ಎದ್ದರೂ ಅದಕ್ಕೆ ತಕ್ಕ ಪರಿಹಾರವೊಂದು ಹೊಳೆದೇ ಹೊಳೆಯುತ್ತದೆ.

\delimiter

ವಿದ್ಯಾರ್ಥಿಗಳ ಪರಿಸ್ಥಿತಿಗಳು ಹಲವು ತೆರನಾಗಿರುತ್ತವೆ:

೧. ಓದಲು ಮನಸ್ಸುಳ್ಳವರಿಗೆ ಅನುಕೂಲಗಳಿಲ್ಲ; ಅನುಕೂಲ ಇರುವವರಿಗೆ ಓದಲು ಮನಸ್ಸಿಲ್ಲ.

೨. ಕೆಲವರಿಗೆ, ಓದಲು ಮನಸ್ಸೂ ಇದೆ, ಅನೂಕೂಲವೂ ಇದೆ.

೩. ಇನ್ನು ಕೆಲವರಿಗೆ ಓದಲು ಅನುಕೂಲವೂ ಇಲ್ಲ, ಮನಸ್ಸೂ ಇಲ್ಲ.

೪. ಇನ್ನೂ ಕೆಲವರಿದ್ದಾರೆ, ಅವರಿಗೆ ಓದಲು ಅನುಕೂಲ, ಮನಸ್ಸು ಎರಡೂ ಇದ್ದರೂ ಬುದ್ಧಿಶಕ್ತಿಯೇ ಇರುವುದಿಲ್ಲ.

‘ಹಾಗಾದರೆ ಇಂಥವರಿಗೆಲ್ಲ ಏನು ಉಪಾಯ?’ ಎಂದು ಕೇಳಿದರೆ, ಇವ ರೆಲ್ಲರೂ ತಮ್ಮತಮ್ಮ ಸಾಮರ್ಥ್ಯಕ್ಕೆ ತಕ್ಕಷ್ಟು ವಿದ್ಯೆಯನ್ನು ಸಂಪಾದಿಸುತ್ತಾರೆ.

\delimiter

ಪರೀಕ್ಷೆಯಲ್ಲಿ ಪಾಸಾಗದಿದ್ದರೂ ಲಂಚ ಕೊಟ್ಟು ಯೋಗ್ಯತಾಪತ್ರವನ್ನು \eng{(Certificate)} ಪಡೆಯಬಹುದೇನೋ; ಆದರೆ ಯೋಗ್ಯತೆಯನ್ನು ಪಡೆಯ ಲಾಗದು.

\delimiter

ಕೆಲವು ವಿದ್ಯಾರ್ಥಿಗಳು ಪರೀಕ್ಷೆಗಾಗಿ ಅಧ್ಯಯನವನ್ನೇನೋ ಚೆನ್ನಾಗಿಯೇ ಮಾಡಿರುತ್ತಾರೆ. ಆದರೆ ಎಷ್ಟೋ ಪ್ರಶ್ನೆಗಳಿಗೆ ಅವರು ಉತ್ತರಿಸಿರುವುದು ಮಾತ್ರ ತಪ್ಪಾಗಿರುತ್ತದೆ. ಕಾರಣವೇನು? ಭಯ, ದಿಗಿಲು! ದಿಗಿಲಿನಿಂದ ಗಲಿಬಿಲಿ! ಗಲಿಬಿಲಿಯಿಂದ ತಪ್ಪು ಉತ್ತರ!

ಇನ್ನು ಕೆಲವು ವಿದ್ಯಾರ್ಥಿಗಳು ಅಷ್ಟಾಗೇನೂ ಅಧ್ಯಯನ ಮಾಡಿರುವುದಿಲ್ಲ. ಆದರೆ ಹೆಚ್ಚಿನ ಪ್ರಶ್ನೆಗಳಿಗೆ ಅವರು ಸರಿಯಾದ ಉತ್ತರವನ್ನೇ ಬರೆದುಬಂದಿರು ತ್ತಾರೆ. ಇದಕ್ಕೆ ಕಾರಣವೇನು? ಇವರಿಗೆ \textbf{ಭಯವೆಂಬುದಿಲ್ಲ.} ಪರಿಣಾಮವಾಗಿ \textbf{ಮನದಲ್ಲಿ ಗಲಿಬಿಲಿಯಿಲ್ಲ.} ಆದ್ದರಿಂದ, ಅವರು ಏನು ಸ್ವಲ್ಪ ಓದಿಕೊಂಡಿ ದ್ದರೋ ಅವೆಲ್ಲ ಉತ್ತರ ಬರೆಯುವ ಹೊತ್ತಿಗೆ ನೆನಪಿಗೆ ಬಂದುವು!

ಆದ್ದರಿಂದ ಹೆದರಿಕೊಂಡರೆ ಚೆನ್ನಾಗಿ ಓದಿದ್ದೂ ಮರೆತುಹೋಗಿಬಿಡುತ್ತದೆ; \textbf{ಧೈರ್ಯದಿಂದಿದ್ದರೆ, ಓದಿದ್ದೆಲ್ಲವೂ ಸಕಾಲದಲ್ಲಿ ನೆನಪಿಗೆ ಬರುತ್ತವೆ.}

\delimiter

\textbf{ಪ್ರಶ್ನೆ:} ಏಕಾಗ್ರತೆಯನ್ನು ತಂದುಕೊಳ್ಳುವುದು ಹೇಗೆ?

\textbf{ಉತ್ತರ:} ಈ ಪ್ರಶ್ನೆಯನ್ನು ವಿದ್ಯಾರ್ಥಿಗಳೂ ಕೇಳುತ್ತಿರುತ್ತಾರೆ, ಆಧ್ಯಾತ್ಮಿಕ ಸಾಧಕರೂ ಕೇಳುತ್ತಿರುತ್ತಾರೆ.

ಏಕ ಎಂದರೆ ಒಂದು; ಅಗ್ರ ಎಂದರೆ ತುದಿ ಅಥವಾ ಮೊನೆ. ಆದ್ದರಿಂದ \textbf{ಏಕಾಗ್ರತೆಯೆಂದರೆ ಒಮ್ಮುಖವಾಗಿರುವುದು ಅಥವಾ ಕೇಂದ್ರೀಕೃತವಾಗಿರುವುದು.} ಮನಸ್ಸಿನ ಏಕಾಗ್ರತೆಯೆಂದರೆ ಮನಸ್ಸನ್ನು ಕೇಂದ್ರೀಕೃತಗೊಳಿಸುವುದು ಅಥವಾ ಒಮ್ಮುಖವಾಗಿಸುವುದು. ಹಾಗಾದರೆ ನಮ್ಮ ಮನಸ್ಸು ಈಗ ಕೇಂದ್ರೀ ಕೃತವಾಗಿಲ್ಲವೆ? ಎಂದು ಕೇಳಿದರೆ ‘ಇಲ್ಲ’ ಎಂದೇ ಹೇಳಬೇಕಾಗುತ್ತದೆ. ಏಕೆಂದರೆ, ಸ್ವಭಾವತಃ ಈ ಮನಸ್ಸು ಚಂಚಲ. ಜೊತೆಗೆ ನಮ್ಮ ಕಣ್ಣು, ಕಿವಿ, ಮೂಗು, ನಾಲಗೆ, ಚರ್ಮಗಳೆಂಬ ಪಂಚೇಂದ್ರಿಯಗಳು ಈ ಮನಸ್ಸನ್ನು ತಮ್ಮೆಡೆಗೆ ಸದಾ ಸೆಳೆಯುತ್ತಲೇ ಇರುತ್ತವೆ. ನಮ್ಮ ಐದು ಇಂದ್ರಿಯಗಳೂ ಬಹಳ ಪ್ರಬಲವಾದವುಗಳಾದ್ದರಿಂದ ನಮ್ಮ ಮನಸ್ಸು ಆ ಐದು ಕಡೆಗೂ ಹರಿದಾಡುತ್ತಲೇ ಇರುತ್ತದೆ. ಅಲ್ಲದೆ, ಈ ಮನಸ್ಸಿಗೆ ತನ್ನದೇ ಆದ ಬಗೆಬಗೆಯ ಆಸೆಗಳಿರುತ್ತವೆ. ಈ ಆಸೆಗಳಿಗೆ ಸೆಳೆಯುವ ಶಕ್ತಿಯಿದೆ. ಹೀಗೆ ಆಸೆಗಳ ಸೆಳೆತಗಳಿಗೆ ಇಂದ್ರಿಯಗಳ ಸಹಕಾರವೂ ಸಿಕ್ಕಿ ಈ ಬಡಪಾಯಿ ಮನಸ್ಸು ಕೋತಿಯಂತೆ ಕುಣಿದಾಡುತ್ತಿರುತ್ತದೆ. ಹೀಗೆ ಕುಣಿದಾಡುವ ಮನಸ್ಸಿನಲ್ಲಿ ಏಕಾಗ್ರತೆಯನ್ನು ನಿರೀಕ್ಷಿಸಲುಂಟೆ?

ಹಾಗಾದರೆ ಈ ಮನಸ್ಸನ್ನು ಸಮಸ್ಥಿತಿಯಲ್ಲಿಡುವುದು ಹೇಗೆ? ಹತ್ತಾರು ದಿಕ್ಕುಗಳಲ್ಲಿ ಹರಿದಾಡುವ ಈ ಮನಸ್ಸನ್ನು ಒಂದೇ ದಿಕ್ಕಿನತ್ತ ಹರಿಯಿಸುವುದು ಹೇಗೆ? ಮನಸ್ಸನ್ನು ಸುಶಿಕ್ಷಿತಗೊಳಿಸಿ ಶಿಸ್ತುಬದ್ಧವಾಗಿಸುವುದೇ ಅದಕ್ಕೆ ತಕ್ಕ ಉಪಾಯ. ಮನಸ್ಸಿಗೆ ಶಿಕ್ಷಣ ಕೊಡುವುದು ಹೇಗೆ?–ಇದು ಎರಡನೆಯ ಪ್ರಶ್ನೆ. ಒಂದು ಸಲಕ್ಕೆ ಒಂದೇ ವಿಷಯದಲ್ಲಿ ತೊಡಗಿರುವಂತೆ ಮನಸ್ಸಿಗೆ ತಿಳಿಯ ಹೇಳಬೇಕು: ‘ಓ ಮನಸ್ಸೇ, ಈಗ ನಾನು ಈ ಪಾಠವನ್ನು ಓದಿ ತಿಳಿದುಕೊಳ್ಳ ಬೇಕಾಗಿದೆ. ಆದ್ದರಿಂದ ನೀನು ಇತರ ವಿಷಯಗಳನ್ನು ಯೋಚಿಸದೆ ಈ ಪಾಠದಲ್ಲೇ ನೆಲೆಗೊಂಡು ಅದನ್ನು ಹೃದ್ಗತಮಾಡಿಕೊಳ್ಳಲು ನನಗೆ ನೆರವಾಗು’ ಎಂದು. ಹೀಗೆ ಮೃದುವಾಗಿ ಹೇಳಿದ ಮಾತನ್ನು ಮನಸ್ಸು ಕೇಳದೆ ಹೋದರೆ ಸ್ವಲ್ಪ ಕಟುವಾಗಿಯೇ ಹೇಳಿಬಿಡಬೇಕಾಗುತ್ತದೆ: ‘ಓ ಮನಸ್ಸೇ! ನೀನು ನನ್ನ ಸೇವಕ. ನಾನು ಹೇಳಿದಂತೆ ನೀನು ಕೇಳಲೇ ಬೇಕು’ ಎಂದು.

ಆದರೆ ಇವೆಲ್ಲದರೊಂದಿಗೆ, \textbf{ಶಿಸ್ತುಬದ್ಧವಾದ ಜೀವನ ನಡೆಸಬೇಕಾದದ್ದು ಬಹಳ ಮುಖ್ಯ.} ಶಿಸ್ತಿನ ಜೀವನವೆಂದರೇನು? ಬೆಳಗ್ಗೆ ಎದ್ದಾಗಿನಿಂದ ರಾತ್ರಿ ಮಲಗುವವರೆಗಿನ ಸಮಯದಲ್ಲಿ, ಮಾಡಬೇಕೆಂದುಕೊಂಡಿರುವ ಸಕಲ ಕಾರ್ಯಕಲಾಪಗಳನ್ನೂ ಕ್ಲ್​ಪ್ತವಾಗಿ, ನಿಯತ್ತಾಗಿ ಮಾಡುತ್ತ ಬರುವುದು. ಜೊತೆಗೆ ಕೆಟ್ಟ ಆಲೋಚನೆಗಳು ಹುಟ್ಟದಂತೆ, ಹರಟೆಯ ಮಾತುಗಳು ಹೊರಡ ದಂತೆ, ವ್ಯರ್ಥಕಾರ್ಯಗಳು ನಡೆಯದಂತೆ ಎಚ್ಚರದಿಂದಿರಬೇಕು. ಆದರೆ ಎಷ್ಟೇ ಎಚ್ಚರಿಕೆಯಿಂದಿದ್ದರೂ ನಡುನಡುವೆ ತಪ್ಪು ಸಂಭವಿಸಬಹುದು. ಆಗ ಮನ ಸ್ಸಿಗೆ ಬೈದು ಬುದ್ಧಿ ಹೇಳಿ ಮುಂದೆ ಅಂಥ ತಪ್ಪು ಆಗದಂತೆ ನೋಡಿಕೊಳ್ಳ ಬೇಕು. ಆದರೂ ಮತ್ತೊಮ್ಮೆ ತಪ್ಪು ಸಂಭವಿಸಬಹುದು. ಆಗ ಪುನಃ ಮತ್ತೊಮ್ಮೆ ಪಾಠ ಹೇಳಬೇಕು. ಹೀಗೆ ಸ್ವಲ್ಪ ಕಾಲ ಹೋರಾಡಿ ಹೋರಾಡಿ ನಮ್ಮ ನಡವಳಿಕೆಯನ್ನು ಒಂದು ಹದಕ್ಕೆ ತಂದುಕೊಳ್ಳಬೇಕಾಗುತ್ತದೆ. ಈ ರೀತಿಯಾಗಿ ನಮ್ಮ ದೈನಂದಿನ ನಡವಳಿಕೆಗಳು ಒಂದು ನಿಶ್ಚಿತ ಗತಿಯಲ್ಲಿ ಸಾಗತೊಡಗಿದಾಗ ಮನಸ್ಸು ಬಹಳಮಟ್ಟಿಗೆ ಸಮಸ್ಥಿತಿಯನ್ನು ಮುಟ್ಟುತ್ತದೆ. ಇಂತಹ ಮನಸ್ಸು ಅತ್ಯಂತ ವಿಧೇಯ ಸೇವಕನಂತೆ ಸದಾ ಸಹಾಯಕಾರಿ ಯಾಗುತ್ತದೆ.

ನಾವು ಆಗಲೇ ನೋಡಿದಂತೆ ನಮ್ಮ ಐದು ಇಂದ್ರಿಯಗಳೂ ಮನಸ್ಸನ್ನು ಎಳೆದಾಡುವಂಥವುಗಳಾದ್ದರಿಂದ ಅವುಗಳ ಮೇಲೂ ಕಾವಲಿನ ಕಣ್ಣಿಟ್ಟಿರ ಬೇಕಾಗುತ್ತದೆ. ಅವುಗಳು ಹೊತ್ತುಗೊತ್ತುಗಳಿಲ್ಲದೆ ಏನೇನನ್ನೋ ಕೇಳುತ್ತಿರು ತ್ತವೆ. ಅವು ಎಷ್ಟಾದರೂ ನಮ್ಮ ಇಂದ್ರಿಯಗಳಲ್ಲವೆ ಎಂಬ ಮೋಹದಿಂದ ಅವು ಕೇಳಿದ್ದನ್ನು ಕೊಡುತ್ತ ಬಂದರೆ ಅವುಗಳ ಜೊತೆಯಲ್ಲೇ ಇರುವ ಈ ಮನಸ್ಸು ಮತ್ತೆ ಮಂಗನಂತಾಗುವುದು ಖಂಡಿತ. ಆದ್ದರಿಂದ ಮನಸ್ಸಿನ ಏಕಾಗ್ರತೆಯನ್ನು ಬಯಸುವವರು, ಇಂದ್ರಿಯಗಳು ಕೇಳಿದ್ದನ್ನೆಲ್ಲ ಕೊಟ್ಟು ಅತಿ ಮುದ್ದು ಮಾಡುವ ಅವಿವೇಕಕ್ಕೆ ಹೋಗಬಾರದು. 

ಇಂದಿನ ವಿದ್ಯಾರ್ಥಿಗಳು ಪಾಶ್ಚಾತ್ಯ ನಾಗರಿಕತೆಗೆ ಮರುಳಾಗಿಯೋ ಅಥವಾ ಕಲಿಗಾಲದ ಪ್ರಭಾವದಿಂದಲೋ ತಮ್ಮ ಇಂದ್ರಿಯಗಳನ್ನು ಸ್ವೇಚ್ಛೆಯಾಗಿ ಹರಿಯಗೊಟ್ಟಂತೆ ಕಂಡುಬರುತ್ತದೆ. ‘ಜೀವನದಲ್ಲಿ ಸುಖ ಪಡಬೇಕಾದುದು ಯುವಕನಾಗಿರುವಾಗಲಲ್ಲದೆ ಮುದುಕನಾದ ಮೇಲೆಯೇ?’ ಎಂಬ ತರ್ಕಬದ್ಧ ವಾದಸರಣಿಯನ್ನು ಮುಂದಿಟ್ಟು ಮಂದಹಾಸ ಬೀರುತ್ತ ನಿಂತಿರುವ ಆ ಯುವಕರಿಗೆ ಯಾವ ಗುರುಹಿರಿಯರೂ ವಿವೇಕ ಹೇಳಲು ಮುಂದಾಗುತ್ತಿಲ್ಲ. ಅಸಂಖ್ಯಾತ ಯುವಕರು ಒಕ್ಕೊರಳಿನಿಂದ ಹೇಳುವ ಮಾತಿನಲ್ಲಿ ಸತ್ಯಾಂಶ ವಿರಬಹುದೆಂಬ ಭ್ರಾಂತಿ ಮೂಡಿದಂತಿದೆ ಈ ಗುರುಹಿರಿಯರಲ್ಲಿ.

ಆದರೆ ಯಾವ ಗುರುಹಿರಿಯರು ಸುಮ್ಮನಿದ್ದರೂ ದುಷ್ಪರಿಣಾಮಗಳು ಸುಮ್ಮನಿರುವುದಿಲ್ಲ. ಶರೀರ ಮನಸ್ಸು ಇಂದ್ರಿಯಗಳ ಸ್ವಭಾವ-ಮರ್ಮಗಳ ನ್ನರಿಯದೆ ಅವುಗಳನ್ನು ಯದ್ವಾತದ್ವಾ ಬಳಸಿದಾಗ ಭೀಕರ ದುಷ್ಪರಿಣಾಮಗಳು ಎರಗಿ ಶಾಂತಿ-ಸೌಖ್ಯ-ಸಮಾಧಾನಗಳೆಲ್ಲ ಕೊಚ್ಚಿ ಹೋಗಿ ಹಾಹಾಕಾರಗೈಯುವ ಪರಿಸ್ಥಿತಿ ಬರುತ್ತದೆ; ಆಗ ಕಾಲ ಮಿಂಚಿ ಹೋಗಿರುತ್ತದೆ!

ಒಟ್ಟಿನಲ್ಲಿ ಹತೋಟಿಯಲ್ಲಿಡದ ಇಂದ್ರಿಯಗಳು ಮನಸ್ಸಿನ ಏಕಾಗ್ರತೆ ಯನ್ನು ಧ್ವಂಸಮಾಡಿಬಿಡಬಲ್ಲವು ಎಂಬ ಕಟು ಸತ್ಯ ಎಷ್ಟು ಬೇಗ ತಿಳಿದರೆ ಅಷ್ಟು ಕ್ಷೇಮ.

ಅಧ್ಯಯನದಲ್ಲಿ ಏಕಾಗ್ರತೆಯನ್ನು ತಂದುಕೊಳ್ಳಲು ಇನ್ನೊಂದು ಬಹು ಮುಖ್ಯ ಉಪಾಯವೇನೆಂದರೆ ಪಠ್ಯ ವಿಷಯವನ್ನು ಚೆನ್ನಾಗಿ, ಸ್ವಷ್ಟವಾಗಿ ಅರ್ಥಮಾಡಿಕೊಂಡು ಮುಂದೆ ಸಾಗುವುದು. ಒಂದು ವಾಕ್ಯವನ್ನು ಓದುವಾಗ ಅದರಲ್ಲಿ ಒಂದೆರಡು ಕಷ್ಟದ ಶಬ್ದಗಳು ಕಂಡುಬಂದರೆ ಕೂಡಲೇ ಶಬ್ದ ಕೋಶದ ಸಹಾಯದಿಂದ ಅದರ ಅರ್ಥವನ್ನು ಹುಡುಕಿ, ಆ ವಾಕ್ಯವನ್ನು ಓದಿ ಮುಗಿಸಿ ಆ ಇಡೀ ವಾಕ್ಯ ಏನು ಹೇಳುತ್ತಿದೆ ಎಂಬುದನ್ನು ಆಗಲೇ ಆಲೋಚಿಸಿ ತಿಳಿದುಕೊಳ್ಳಬೇಕು. ಹೀಗೆ ಒಂದು ಪ್ಯಾರಾ ಓದಿ ಮುಗಿಸಿದಾಗ ಆ ಇಡೀ ಪ್ಯಾರಾ ಯಾವ ಅಂಶವನ್ನು ತಿಳಿಸಿತು ಎಂಬುದನ್ನು ಆಗಾಗಲೇ ಮನಸ್ಸಿಗೆ ತಂದು ಕೊಳ್ಳಬೇಕು. ಈ ವಿಧಾನದಿಂದ ಓದುತ್ತಿದ್ದರೆ ವಿಷಯಗಳು ಅರ್ಥವಾಗುತ್ತವೆ; ಅರ್ಥವಾಗುವುದರಿಂದ ಆನಂದವುಂಟಾಗುತ್ತದೆ; ಆನಂದವುಂಟಾಗುವುದ ರಿಂದ ಅಧ್ಯಯನದ ಮೇಲೆ ಪ್ರೀತಿ ಹುಟ್ಟಿಕೊಳ್ಳುತ್ತದೆ; ಪ್ರೀತಿ ಹುಟ್ಟಿದಾಗ ತಾನೇತಾನಾಗಿ ಏಕಾಗ್ರತೆ ಬರುತ್ತದೆ. 

ಎಲ್ಲಿ ನಮ್ಮ ಪ್ರೀತಿ ಇರುತ್ತದೆಯೋ ಅಲ್ಲಿಯೇ ನಮ್ಮ ಮನಸ್ಸಿರುತ್ತದೆ; ಎಲ್ಲಿ ನಮ್ಮ ಮನಸ್ಸು ಇರುತ್ತದೆಯೋ ಅಲ್ಲಿಯೇ ಏಕಾಗ್ರತೆ ಮೂಡಿಬರುತ್ತದೆ –ಇದೊಂದು ಅನಿವಾರ್ಯ ನಿಯಮ. ‘ನನಗೆ ನನ್ನ ಪಾಠಗಳ ವಿಷಯದಲ್ಲೇ ಪ್ರೀತಿಯಿಲ್ಲವಲ್ಲ? ಏನು ಮಾಡಲಿ?’ ಎನ್ನುವ ವಿದ್ಯಾರ್ಥಿಗಳೂ ಇದ್ದಾರೆ. ಅಂಥವರು ಒಂದೋ ತಮಗೆ ಇಷ್ಟವಾಗುವ ಬೇರೆ ವಿಷಯವನ್ನು ಆರಿಸಿಕೊಳ್ಳ ಬೇಕು; ಅಥವಾ ಈಗಿರುವ ಪಠ್ಯವಿಷಯದಲ್ಲೇ ಪ್ರೀತಿಯನ್ನು ತಂದುಕೊಳ್ಳ ಬೇಕು.

ಮನಸ್ಸಿನ ಏಕಾಗ್ರತೆಗೆ ಅಭ್ಯಾಸಬಲವೇ ಕಾರಣ ಎಂಬುದು ಇನ್ನೊಂದು ಬಹುಮುಖ್ಯ ಅಂಶ. \textbf{ಅಭ್ಯಾಸ} ಎಂದರೇನು? \textbf{ಪುನಃ ಪುನಃ ಮಾಡುವ ಪ್ರಯತ್ನ.} ಬರೆದೂ ಬರೆದೂ ಅಭ್ಯಾಸ ಮಾಡಿದರೆ ಬರವಣಿಗೆ ಅಂದವಾಗುತ್ತದೆ. ಓದೀ ಓದೀ ಅಭ್ಯಾಸ ಮಾಡಿದರೆ ಅಧ್ಯಯನ ಯಶಸ್ವಿಯಾಗುತ್ತದೆ. ಗುರಿಯಿಟ್ಟು ಗುಂಡು ಹೊಡೆಯಬೇಕಾದರೂ ಸೈನಿಕ ಅಭ್ಯಾಸ ಮಾಡಿರಬೇಕಾಗುತ್ತದೆ. ರುಚಿ ಯಾದ ಅಡುಗೆ ಮಾಡಬೇಕಾದರೂ ಗೃಹಿಣಿ ಅಭ್ಯಾಸ ಮಾಡಿರಬೇಕಾಗುತ್ತದೆ. ಅಭ್ಯಾಸದಿಂದ ಸಿದ್ಧಿ ಎಂಬ ಮಾತು ಎಲ್ಲ ಕಾಲಕ್ಕೂ ಸತ್ಯ. ಆದರೆ ಶ್ರದ್ಧೆಯಿಂದ ಅಭ್ಯಾಸ ಮಾಡಬೇಕು. ಉತ್ಸಾಹದಿಂದ ಅಭ್ಯಾಸ ಮಾಡಬೇಕು. ಬುದ್ಧಿ ಉಪ ಯೋಗಿಸಿ ಅಭ್ಯಾಸ ಮಾಡಬೇಕು. ಈ ಬಗೆಯಿಂದ ಅಭ್ಯಾಸ ಮಾಡುತ್ತ ಹೋದರೆ ಏಕಾಗ್ರತೆ ಏಕೆ ಬರುವುದಿಲ್ಲ?

