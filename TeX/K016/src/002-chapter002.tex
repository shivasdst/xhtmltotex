
\chapter{ವಿದ್ಯಾರ್ಥಿಗೊಂದು ಪತ್ರ}

ಪ್ರಿಯ ಸೋಮಶೇಖರ,

ನಿನ್ನ ಪತ್ರ ತಲುಪಿತು. ನಿನ್ನ ಅಸಹಾಯಕ ಪರಿಸ್ಥಿತಿಯ ಅರಿವೂ ಆಯಿತು.

ಗ್ರಾಮಾಂತರ ಶಾಲೆಯ ವಿದ್ಯಾರ್ಥಿಯಾದ ನಿನಗೆ ಕೆಲವು ಅನನುಕೂಲತೆ ಗಳಿವೆ ಎಂಬುದು ನಿಜವಾದರೂ ಹಲವಾರು ಅನುಕೂಲತೆಗಳೂ ಇವೆ ಎಂಬು ದನ್ನು ನೀನು ಮರೆಯಬಾರದು. ಪೇಟೆ ಪಟ್ಟಣ ನಗರಗಳಲ್ಲಿರುವ ವಿದ್ಯಾರ್ಥಿ ಗಳ ಅಧ್ಯಯನಕ್ಕೆ ಅಡ್ಡಿ ಆತಂಕಗಳು ಅಧಿಕವೆಂದೇ ಹೇಳಬೇಕು. ಟಿ.ವಿ., ಸಿನೆಮಾ, ಹೋಟಲುಗಳು, ಲೌಡ್​ಸ್ಪೀಕರ್ ಅರಚಾಟದಿಂದ ಉಂಟಾಗುವ ಶಬ್ದಮಾಲಿನ್ಯ, ಇನ್ನೂ ಏನೇನೋ. ಗ್ರಾಮಾಂತರ ಪ್ರದೇಶದ ಪರಿಶುದ್ಧ ಹವಾಗುಣದಿಂದ ಕೂಡಿದ ಪ್ರಶಾಂತ ವಾತಾವರಣವೊಂದೇ ಸಾಕು, ಏಕಾಗ್ರತೆ ಯಿಂದ ಕುಳಿತು ಅಧ್ಯಯನ ಮಾಡಲು. ಪಟ್ಟಣಗಳ ವಿದ್ಯಾರ್ಥಿಗಳಿಗೆ ಅನೇಕ ಅನುಕೂಲತೆಗಳಿವೆ ಎಂಬ ಮಾತನ್ನು ಅಲ್ಲಗಳೆಯುವುದಿಲ್ಲ ನಾನು. ಆದರೆ ಈ ವಿದ್ಯಾರ್ಥಿಗಳು ಪೋಲೀತನಕ್ಕೆ ಬಲಿಬೀಳುವ ಸಂಭವ ಹೆಚ್ಚು. ಆದ್ದರಿಂದ ಅವರ ತಾಯ್ತಂದೆಯರು, ಅಧ್ಯಾಪಕರು ಅವರನ್ನು ರಕ್ಷಿಸಲು ಅದೆಷ್ಟು ಪರಿ ಪಾಟು ಪಡಬೇಕಾಗುತ್ತದೆಂಬುದನ್ನು ತಿಳಿದಿರಬೇಕು ನೀನು. ನಗರಗಳಲ್ಲಿ ಆದರ್ಶದಿಂದ ಜಾರಲು ನೂರಾರು ದಾರಿಗಳಿವೆ. ತಮ್ಮ ಹುಡುಗರನ್ನು ಯಾವದೋ ಒಂದು ಜಾರುದಾರಿಯಿಂದ ತಪ್ಪಿಸಿದ ಮಾತ್ರಕ್ಕೆ ತಾಯ್ತಂದೆ ಯರು ಸುಮ್ಮನೆ ಕುಳಿತಿರುವಂತಿಲ್ಲ. ಇತರ ಜಾರುದಾರಿಗಳತ್ತ ಹೋಗದಂತೆ ಕಣ್ಣಲ್ಲಿ ಕಣ್ಣಿಟ್ಟು ಕಾಯಬೇಕಾಗುತ್ತದೆ.

ಆ ವಿಷಯ ಹಾಗಿರಲಿ. ‘ನಾನೀಗ ಎಸ್ಸೆಸ್ಸೆಲ್ಸಿ ತರಗತಿಗೆ ಬಂದಿದ್ದೇನೆ, ಪಾಠ ಪ್ರವಚನಗಳು ಪ್ರಾರಂಭವಾಗಿವೆ, ಆದರೆ ಎಷ್ಟೋ ವಿಷಯಗಳು ಅರ್ಥವೇ ಆಗುತ್ತಿಲ್ಲ’ ಎಂದು ಬರೆದಿರುವೆ. ಇನ್ನೊಬ್ಬ ವಿದ್ಯಾರ್ಥಿಯೂ ಒಂದು ಪತ್ರ ಬರೆದಿದ್ದಾನೆ. ಅವನ ಅಧ್ಯಾಪಕರು ಸರಿಯಾಗಿ ಪಾಠವನ್ನೇ ಮಾಡುವುದಿಲ್ಲ ವಂತೆ; ಅರ್ಥವಾಗದ ವಿಷಯವನ್ನು ಕೇಳಿದರೆ ಉತ್ತರ ಹೇಳುವುದೂ ಇಲ್ಲ ವಂತೆ. ಅದು ಅವನ ದುರದೃಷ್ಟ. ನಿನ್ನ ಅಧ್ಯಾಪಕರು ಚೆನ್ನಾಗಿ ಪಾಠ ಮಾಡು ತ್ತಾರಲ್ಲ, ಅದು ನಿನ್ನ ಪುಣ್ಯ!

ನೋಡು, ಪಾಠಗಳು ನಿನಗೆ ಅರ್ಥವಾಗಬೇಕಾದರೆ, ಮುಂದೆ ನೀನು ಯಶಸ್ವಿ ಯಾಗಿ ತೇರ್ಗಡೆಯಾಗಬೇಕಾದರೆ ಅದಕ್ಕೆ ಅಗತ್ಯವಾದ ಕೆಲವು ಸಲಹೆಗಳನ್ನು ನಿನ್ನ ಮುಂದಿಡುತ್ತೇನೆ. ಅವುಗಳನ್ನು ಸುಮ್ಮನೆ ಅನುಸರಿಸಿದರಾಯಿತು; ಜಯ ನಿನಗೆ ಕಟ್ಟಿಟ್ಟದ್ದು.

೧. ಮೊಟ್ಟಮೊದಲನೆಯದಾಗಿ ನಿನಗೆ ಕೊಡಲಿಚ್ಛಿಸುವ ಸಲಹೆಯೆಂದರೆ, ನೀನು \textbf{ಬೆಳಗ್ಗೆ ಎದ್ದ ಕೂಡಲೇ ಶೌಚಾದಿಗಳನ್ನು ಮುಗಿಸಿಕೊಂಡು ದೇವರಿಗೆ, ತಾಯಿಗೆ, ತಂದೆಗೆ ಪ್ರಣಾಮ ಸಲ್ಲಿಸಬೇಕು.} ಈ ಕಾರ್ಯದಿಂದ ನಿನ್ನ ದಿನಚರಿ ಪ್ರಾರಂಭವಾಗಬೇಕು. ನಿನ್ನೆಲ್ಲ ಯಶಸ್ಸಿಗೆ, ಪ್ರಯತ್ನದ ಸಫಲತೆಗೆ ದೇವರ ಹಾಗೂ ಗುರುಹಿರಿಯರ ಆಶೀರ್ವಾದವೇ ಮುಖ್ಯ ಎಂಬ ವಿಷಯದಲ್ಲಿ ನಿನಗೆ ಸಂಶಯ ಬಾರದಿರಲಿ. ಸರಿಯಾದ ಪ್ರಯತ್ನ ಮಾಡಿದರೆ ಅದಕ್ಕೆ ತಕ್ಕ ಫಲ ಬಂದೇ ಬರುತ್ತದೆ ಎನ್ನುವುದು ನಿಜವೇ. ಆದರೆ ಸರಿಯಾದ ಪ್ರಯತ್ನದಲ್ಲಿ ತೊಡಗುವುದಕ್ಕೆ ಬೇಕಾದ ಮನಸ್ಥಿತಿಯನ್ನು ಕೊಡುವುದು ದೇವರ, ಗುರು ಹಿರಿಯರ ಆಶೀರ್ವಾದ. ಈ ಆಶೀರ್ವಾದದ ಮಹಿಮೆಯನ್ನು ಈಗ ನಾನೆಷ್ಟೇ ವರ್ಣಿಸಿದರೂ ನಿನಗದು ಸರಿಯಾಗಿ ಅರ್ಥವಾಗುವುದು ನೀನು ಸಾಕಷ್ಟು ಪ್ರಾಯಪ್ರಬುದ್ಧನಾದ ಮೇಲೆಯೇ. ಆದರೂ ತತ್ಕಾಲಕ್ಕೆ ಈ ಮಾತಿನಲ್ಲಿ ನಂಬಿಕೆಯಿಟ್ಟು ನಡೆಯಲು ಯತ್ನಿಸು. ಇನ್ನೊಂದು ಸ್ವಾರಸ್ಯವೆಂದರೆ, ಆಶೀರ್ವಾದವನ್ನು ನೀನು ಬಾಯ್ಬಿಚ್ಚಿ ಕೇಳಬೇಕಾಗಿಲ್ಲ. ಭಕ್ತಿಯಿಂದ ಮಣಿದಾಗ ದೇವರ ಹಾಗೂ \textbf{ಗುರುಹಿರಿಯರ ಹೃದಯದಿಂದ ‘ಇವನಿಗೆ ಶುಭವಾಗಲಿ’ ಎಂಬ ಆಶೀರ್ವಾದದ ಭಾವ ತಾನಾಗಿಯೇ ಹರಿದು ಬರುತ್ತದೆ.}

೨. ಈಗ ನಿನಗೆ ವೇಳಾಪಟ್ಟಿಯ ಮಹತ್ವವನ್ನು ತಿಳಿಸಲೆತ್ನಿಸುತ್ತೇನೆ. ನಿನ್ನ ಶಾಲೆಯಲ್ಲಿ ಪ್ರಾರ್ಥನೆ ಪಾಠ ಆಟ ಈ ಎಲ್ಲವೂ ಸುಸೂತ್ರವಾಗಿ ನಡೆಯುವು ದನ್ನು ಕಾಣುತ್ತಿರುವೆಯಷ್ಟೆ? ಶಾಲಾ ಅಧ್ಯಾಪಕರು ಹಾಕಿಕೊಂಡಿರುವ ವೇಳಾ ಪಟ್ಟಿಯೇ ಆ ಸುಸೂತ್ರತೆಯ ಗುಟ್ಟು. ದಿನಕ್ಕೆ ಸುಮಾರು ಏಳೆಂಟು ಕಾಲಾವಧಿ \eng{(period)}ಗಳಲ್ಲಿ ಬಗೆಬಗೆಯ ವಿಷಯಗಳ ಮೇಲೆ ಸ್ವಲ್ಪಸ್ವಲ್ಪವೇ ಪಾಠ ಪ್ರವಚನಗಳನ್ನು ನಡೆಸುತ್ತ ನಿಮಗೆ ಎಷ್ಟೊಂದು ಪಠ್ಯ ಪುಸ್ತಕಗಳ ಅಭ್ಯಾಸ ಮಾಡಿಸುತ್ತಾರೆ ನೋಡು! ಆದರೂ ಇಂದಿನ ವಿದ್ಯಾರ್ಥಿಗಳು ಈ ಗುಟ್ಟನ್ನು ಅರಿತುಕೊಳ್ಳುವುದೇ ಇಲ್ಲ. ಬದಲಾಗಿ ಅವರು ಮನೆಗೆ ಬಂದು, ಮಾಡಿದ ಪಾಠಗಳ ಪುನರಾವರ್ತನೆ ಮಾಡದೆ, ಅತಿಯಾದ ಆಟ, ಟಿ.ವಿ. ಕಥೆ-ಕಾದಂಬರಿ ಹಾಗೂ ಅಲೆದಾಟಗಳಲ್ಲೇ ಮೈಮರೆತು, ಕೊನೆಗೆ ಪರೀಕ್ಷೆ ಸಮೀಪಿಸಿದಾಗ ಮೈತಿಳಿದು, ಎಲ್ಲ ಪಠ್ಯ ಪುಸ್ತಕಗಳನ್ನೂ ಒಟ್ಟಿಗೇ ಓದಿ ತಲೆಗೆ ತುಂಬಿಸಿಕೊಳ್ಳು ತ್ತೇವೆಂದು ಪ್ರಯತ್ನಿಸಿ, ಅದು ಸಾಧ್ಯವಾಗದೆ ತಲೆ ಕೆಡಿಸಿಕೊಂಡು ಕಂಗಾ ಲಾಗುವ ದೃಶ್ಯ ಸರ್ವೇ ಸಾಮಾನ್ಯ. ಇದು ವಿದ್ಯಾರ್ಥಿಗಳ, ವಿದ್ಯಾಭ್ಯಾಸದ ಲಕ್ಷಣವೇ ಅಲ್ಲ. ಆದರೆ ಅಲ್ಲಲ್ಲಿ ಕೆಲವು ವಿದ್ಯಾರ್ಥಿಗಳಿರುತ್ತಾರೆ; ಅವರಿಗೆ ತಮ್ಮ ತಾಯ್ತಂದೆಯರದೋ ಅಧ್ಯಾಪಕರದೋ ಮಾರ್ಗದರ್ಶನವಿರುತ್ತದೆ, ಹಾಗೂ ಆ ವಿದ್ಯಾರ್ಥಿಗಳಲ್ಲೂ ಸ್ವಭಾವಸಹಜವಾಗಿ ಶಿಸ್ತು-ವಿಧೇಯತೆ ಇರು ತ್ತದೆ. ಅಂಥವರು ತಮ್ಮದೇ ಆದ ವೇಳಾಪಟ್ಟಿಯೊಂದನ್ನು ಹಾಕಿಕೊಂಡು ಪ್ರತಿದಿನ ತಪ್ಪದೆ ಓದುವುದು-ಬರೆಯುವುದನ್ನು ಮಾಡುತ್ತ ಲೀಲಾಜಾಲವಾಗಿ ಮುಂದುವರಿಯುತ್ತಿರುತ್ತಾರೆ. ಈ ಬುದ್ಧಿವಂತಿಕೆಯನ್ನು ನೀನೂ ಕಲಿತುಕೊಳ್ಳಬೇಕು.

ಶಾಲಾ ವೇಳೆಯನ್ನು ಬಿಟ್ಟರೆ ನಿನ್ನ ಸ್ವಂತದ ವೇಳೆ ಎಷ್ಟಿರುತ್ತದೆ ಎಂಬುದನ್ನು ನೋಡಿಕೊ. ಶನಿವಾರದ ಅರ್ಧ ದಿನ, ಭಾನುವಾರ ಹಾಗೂ ಇತರ ರಜಾ ದಿನಗಳೆಲ್ಲ ನಿನ್ನ ಸ್ವಂತ ವೇಳೆಯೇ ಅಲ್ಲವೆ? ಇನ್ನು ಆಗಾಗ ಸಿಗುವ ‘ಲೆಟ್ ಆಫ್​’ ವೇಳೆಯೂ ನಿನ್ನದೇ. ಈ ವೇಳೆಯನ್ನು ಹಿಡಿದು ಅದರ ಪ್ರತಿ ಯೊಂದು ಕ್ಷಣವನ್ನೂ ಸದುಪಯೋಗ ಪಡಿಸಿಕೊಳ್ಳಬಲ್ಲೆಯಾದರೆ ಈ ಜಗತ್ತಿ ನಲ್ಲಿ ನೀನೊಬ್ಬ ಮಹೋನ್ನತ ವ್ಯಕ್ತಿಯಾಗುವುದು ಖಂಡಿತ.

\textbf{ವೇಳಾಪಟ್ಟಿಯನ್ನು ಹಾಕಿಕೊಳ್ಳದೆ ವೇಳೆಯನ್ನು ಸದುಪಯೋಗ ಪಡಿಸಿ ಕೊಳ್ಳುವುದು ಸಾಧ್ಯವಲ್ಲದ ಮಾತು.} ವೇಳಾಪಟ್ಟಿಯನ್ನು ಹೇಗೆ ಹಾಕಿಕೊಳ್ಳ ಬೇಕು? ಮೊದಲು, ನೀನು ರಾತ್ರಿ ಮಲಗುವ ಹಾಗೂ ಬೆಳಗ್ಗೆ ಏಳುವ ಸಮಯ ವನ್ನು ಖಚಿತಪಡಿಸಿಕೊಂಡುಬಿಡಬೇಕು. ಏಕೆಂದರೆ ಮಲಗುವ, ಏಳುವ ವೇಳೆ ಯಲ್ಲಿ ವ್ಯತ್ಯಾಸವಾಯಿತೆಂದರೆ ನಿನ್ನ ವೇಳಾಪಟ್ಟಿಯನ್ನು ಸುಮ್ಮನೆ ಕಟ್ಟಿಡ ಬೇಕಾಗುತ್ತದೆ. ಹದಿಹರೆಯದ ನೀನು ರಾತ್ರಿ ಹತ್ತಕ್ಕೆ ಮಲಗಿ ಬೆಳಗ್ಗೆ ಐದಕ್ಕೆ ಏಳುವುದು ಅತ್ಯಂತ ಸೂಕ್ತ, ಸಮರ್ಪಕ. ದಿನವಿಡೀ ನಿನ್ನ ಮನಸ್ಸು ಶಾಂತವೂ ಸ್ಥಿರವೂ ಪ್ರಫುಲ್ಲವೂ ಆಗಿರಬೇಕಾದರೆ ರಾತ್ರಿಯಲ್ಲಿ ಮಾಡುವ \textbf{ನಿದ್ರೆ ಅತ್ಯಂತ ಪ್ರಾಮುಖ್ಯದ್ದು.} ಗಾಢನಿದ್ರೆಯೊಂದು ವರ. ಬೆಳಗ್ಗೆ ಐದರಿಂದ ರಾತ್ರಿ ಹತ್ತರ ವರೆಗೆ ನಿನಗೆ ದೊರೆಯುವ ಹದಿನೇಳು ಗಂಟೆಯ ಕಾಲವನ್ನು ನೀನು ಸದುಪಯೋಗಪಡಿಸಿಕೊಂಡು ಮೈ-ಮನಸ್ಸುಗಳನ್ನು ದುಡಿಸಿದ್ದೇ ಆದರೆ ರಾತ್ರಿ ದಿಂಬಿಗೆ ತಲೆಯಿಟ್ಟ ಕೂಡಲೇ ಗಾಢ ನಿದ್ರೆ ಹತ್ತುವುದು ಖಂಡಿತ.

ಬೆಳಗ್ಗೆ ಐದಕ್ಕೆ ಎದ್ದು ಶಾಲೆಗೆ ಹೋಗುವವರೆಗಿನ ವೇಳೆ ಹಾಗೂ ಶಾಲೆ ಯಿಂದ ಹಿಂದಿರುಗಿದ ಮೇಲೆ ರಾತ್ರಿ ಮಲಗುವವರೆಗಿನ ವೇಳೆ ನಿನ್ನದು. ಈ ಅವಧಿಯಲ್ಲಿ ನಿನ್ನ ವೈಯಕ್ತಿಕ ಪ್ರಾರ್ಥನೆ, ಅಧ್ಯಯನ ಹಾಗೂ ಇತರ ಕೆಲಸ ಕಾರ್ಯಗಳನ್ನು ಅಳವಡಿಸಿರುವ ವೇಳಾಪಟ್ಟಿಯೊಂದನ್ನು ತಯಾರಿಸಿಕೊಳ್ಳ ಬೇಕು. ಅದನ್ನು ನಿನ್ನ ಅಧ್ಯಾಪಕರಿಗೆ ತೋರಿಸಿ ಅವರ ಸಲಹೆ ಸೂಚನೆಗಳನ್ನು ಪಡೆಯಬಹುದು.

ಬೆಳಗ್ಗೆ ಎದ್ದು ಸ್ನಾನಾದಿಗಳನ್ನು ಮುಗಿಸಿದ ಮೇಲೆ ನಿನ್ನ ಶಕ್ತ್ಯನುಸಾರ ಹತ್ತು ನಿಮಿಷವೋ ಅರ್ಧಗಂಟೆಯೋ \textbf{ಸ್ತೋತ್ರಪಠನ ಪ್ರಾರ್ಥನೆ ಧ್ಯಾನ} ಇತ್ಯಾದಿಗಳ ಕಾರ್ಯವೊಂದು ನಿನ್ನ ವೇಳಾಪಟ್ಟಿಯಲ್ಲಿ ಅವಶ್ಯವಾಗಿ ಸೇರಿರಲಿ, ನಿನ್ನ ಮನಸ್ಸಿನ ಆರೋಗ್ಯಕ್ಕೆ, ಸಮಸ್ಥಿತಿಗೆ ಇದು ಅತ್ಯಂತ ಸಹಾಯಕ. ಹಾಗೆಯೇ ರಾತ್ರಿ ಮಲಗುವ ಸಮಯದಲ್ಲಿಯೂ ನಿನ್ನ ಇಷ್ಟ ದೇವತೆಯನ್ನು ಪ್ರಾರ್ಥಿಸಿ ಕೊಳ್ಳಬೇಕು: 

“ಹೇ ಭಗವನ್, ನಿನ್ನ ದಯೆಯಿಂದ ಇಂದಿನ ದಿನವನ್ನು ಸಾಧ್ಯವಾದಷ್ಟು ಪ್ರಾಮಾಣಿಕತೆಯಿಂದ ಸದುಪಯೋಗ ಪಡಿಸಿಕೊಂಡಿದ್ದೇನೆ. ಆದರೂ ಕೆಲವಾರು ಲೋಪದೋಷಗಳು ಇರಬಹುದು. ನಾಳೆಯ ದಿನ ಆ ಲೋಪದೋಷಗಳೂ ಆಗದಂತೆ ನನಗೆ ಶಕ್ತಿ-ವಿವೇಕ ಕೊಟ್ಟು ಮುನ್ನಡೆಸು.”

ಈ ಪ್ರಾರ್ಥನೆ ಹೃತ್ಪೂರ್ವಕವಾಗಿರಲಿ. ದೇವರು ಅದನ್ನು ಖಂಡಿತ ಈಡೇರಿ ಸುತ್ತಾನೆ. ಮತ್ತು ನಿನ್ನ ಮನಶ್ಶಕ್ತಿ ದಿನೇ ದಿನೇ ಹೆಚ್ಚುವುದನ್ನು ನೀನೇ ನೋಡ ಬಹುದು. (ಇದೇ ಪುಸ್ತಕದಲ್ಲಿರುವ ‘ಪ್ರಾರ್ಥಿಸಿರಿ! ಅಥವಾ, ಭಾವಿಸಿರಿ!’ ಎಂಬ ಲೇಖನದಲ್ಲಿ ಕೊಟ್ಟಿರುವ ಸೂಚನೆಗಳನ್ನೂ ಗಮನಿಸಿ ಅವುಗಳನ್ನು ಅಳವಡಿಸಿಕೊಂಡರೆ ಅತ್ಯುತ್ತಮ).

ಸ್ನಾನ ಎಂದೆನಲ್ಲ, ಇದನ್ನು ಮಾತ್ರ ಖಂಡಿತ \textbf{ಕಡೆಗಣಿಸಬೇಡ}. ಪ್ರತಿದಿನವೂ ತಲೆಯಿಂದ ಕಾಲಿನವರೆಗೂ ಬೆವರಿ ಬೆವರಿ, ಅದು ಗಾಳಿಗೆ ಆರಿ ಉಪ್ಪು ಹೆಪ್ಪುಗಟ್ಟಿರುತ್ತದೆ, ಜೊತೆಗೆ ಧೂಳು ಅಂಟಿಕೊಂಡಿರುತ್ತದೆ. ಇಂಥ ತಲೆ-ಮೈ ಯನ್ನು ತೊಳೆಯದಿದ್ದರೆ ಮನಸ್ಸು ಮಂದವಾಗಿ ಲವಲವಿಕೆಯನ್ನು ಕಳೆದು ಕೊಳ್ಳುತ್ತದೆ. ಆಗ, ಅಧ್ಯಯನ ಮಾಡುವ, ಅಧ್ಯಯನ ಮಾಡಿದ್ದನ್ನು ನೆನಪಿಟ್ಟು ಕೊಳ್ಳುವ ಸಾಮರ್ಥ್ಯ ಅರ್ಧಕ್ಕೆ ಅರ್ಧ ಕುಂಠಿತವಾಗುತ್ತದೆ.

ಬೆಳಗ್ಗೆ-ಸಂಜೆ ಎರಡು ಹೊತ್ತು ಸ್ನಾನ ಮಾಡುವ ಅಭ್ಯಾಸ ಬೆಳಸಿಕೊಳ್ಳಲು ಸಾಧ್ಯವಾದರೆ ಇನ್ನೂ ಒಳ್ಳೆಯದು. ತಣ್ಣೀರ ಸ್ನಾನ ಅತ್ಯುತ್ತಮ; ಸಾಧ್ಯವಾಗ ದಿದ್ದರೆ ಉಗುರು ಬೆಚ್ಚಗಿನ ನೀರು. ಬಿಸಿ ನೀರನ್ನು ಮಾತ್ರ ಬಳಸಲೇ ಬೇಡ. ತಣ್ಣೀರ ಸ್ನಾನವನ್ನು ರೂಢಿಸಿಕೊಳ್ಳಬೇಕೆಂದಿದ್ದರೆ ಸೆಕೆಗಾಲದಲ್ಲಿ ಪ್ರಾರಂಭಿಸು ವುದು ಸುಲಭ. ಈ ತಣ್ಣೀರ ಸ್ನಾನದಿಂದ ಎರಡು ಮಹಾ ಪ್ರಯೋಜನಗಳಿವೆ– ಮೊದಲನೆಯದಾಗಿ ಶರೀರವೂ ಚಟುವಟಿಕೆಯಿಂದಿರುತ್ತದೆ; ಮನಸ್ಸೂ ಚುರು ಕಾಗಿರುತ್ತದೆ. ಎರಡನೆಯದಾಗಿ ಬ್ರಹ್ಮಚರ್ಯಪಾಲನೆಗೆ ಅದು ಬಹಳ ಸಹಾಯಕಾರಿ.

೩. ಈಗ, ವೇಳಾಪಟ್ಟಿಗೆ ಅನುಸಾರವಾಗಿ ಅಧ್ಯಯನ ಮಾಡಬೇಕೆಂಬು ದೇನೋ ಸರಿಯೆ. ಆದರೆ ಇಲ್ಲಿ ಒಂದು ಸಣ್ಣ ಸಲಹೆ ನೀಡಲಿಚ್ಛಿಸುತ್ತೇನೆ. ನೋಡು, ನಿನ್ನ ತರಗತಿಯಲ್ಲಿ \textbf{ಅಂದಂದು ನಡೆಯುವ ಪಾಠಗಳನ್ನು ಮುಂಚಿತ ವಾಗಿಯೇ ಓದಿ ತಯಾರಾಗಿರಬೇಕು.} ಹೀಗೆ ಮಾಡಿದರೆ, ಅಧ್ಯಾಪಕರು ಪಾಠದ ವಿಷಯಗಳನ್ನು ವಿವರಿಸುವಾಗ ಮನಸ್ಸಿಗೆ ಅದು ಚೆನ್ನಾಗಿ ನಾಟುತ್ತದೆ. ನೀನೇ ಓದಿಕೊಳ್ಳುವಾಗ ನಿನಗೆ ಯಾವ ಅಂಶ ಅರ್ಥವಾಗಿರಲಿಲ್ಲವೋ ಅದು ನಿನಗೆ ತರಗತಿಯಲ್ಲಿ ಸ್ವಷ್ಟವಾಗುತ್ತದೆ. ಬಳಿಕ, ಮನೆಗೆ ಹಿಂದಿರುಗಿದ ಮೇಲೆ ರಾತ್ರಿ ಅಧ್ಯಯನದ ವೇಳೆಯಲ್ಲಿ, ಅಂದು ಮಾಡಿದ ಪಾಠ ಪ್ರವಚನಗಳನ್ನೊಮ್ಮೆ ಪುನರಾವರ್ತನೆ ಮಾಡಬೇಕು. ಹೀಗೆ ಪೂರ್ವತಯಾರಿ ಮಾಡಿಕೊಂಡು ಹೋಗು ವುದು, ಮನೆಗೆ ಬಂದು ಪುನರಾವರ್ತನೆ ಮಾಡುವುದು–ಇದನ್ನು ಕೇವಲ ಮೂರು ತಿಂಗಳು ನಡೆಸಿಕೊಂಡು ಬಾ. ಆಮೇಲೆ, ಅಷ್ಟಕ್ಕೇ ಅದೆಷ್ಟು ಪ್ರಯೋಜನವನ್ನು ಪಡೆದಿರುವೆ ಎಂಬುದನ್ನು ಕಂಡು ನೀನು ಬೆರಗಾಗದಿರ ಲಾರೆ. ಸ್ವಲ್ಪ ಇಚ್ಛಾಶಕ್ತಿಯನ್ನುಪಯೋಗಿಸಿ, ಎಂದರೆ ಸ್ವಲ್ಪ ಹಟ ತಂದು ಕೊಂಡು ವರ್ಷವಿಡೀ ಈ ಕ್ರಮವನ್ನು ಅನುಸರಿಸಿದಿಯಾದರೆ ಪಾಠಗಳೂ ‘ಬೋರ್​’ ಆಗುವುದಿಲ್ಲ, ಪರೀಕ್ಷೆಗಳೂ ಭಾರವಾಗುವುದಿಲ್ಲ.

ಆದರೆ, ಹೀಗೆ ಪ್ರತಿದಿನ ಪೂರ್ವ ತಯಾರಿ-ಪುನರಾವರ್ತನೆಯ ವ್ರತ ತೊಟ್ಟರೆ ಅಲ್ಲಿ ಇಲ್ಲಿ ಅಲೆದಾಡುವಂತಿಲ್ಲ, ಗಂಟೆಗಟ್ಟಲೆ ಟೀವಿಯ ಮುಂದೆ ಕುಳಿತಿರುವಂತಿಲ್ಲ, ಸ್ನೇಹಿತರನ್ನು ಕಟ್ಟಿಕೊಂಡು ಹರಟುತ್ತ ಕಾಲಕಳೆಯು ವಂತಿಲ್ಲ. ವಾರ್ಷಿಕ ಪರೀಕ್ಷೆ ಮುಗಿಯುವವರೆಗೂ ವ್ರತಧಾರಕನಂತೆ ಇರ ಬೇಕಾಗುತ್ತದೆ. ನಿಜಕ್ಕೂ ವಿದ್ಯಾಭ್ಯಾಸವೆಂದರೆ ಒಂದು ವ್ರತವೇ ಸರಿ. ಆದ್ದ ರಿಂದ ನೀನೊಬ್ಬ ವ್ರತಧಾರಕನೆಂದೇ ತಿಳಿದುಕೊ. ಸತ್ಯವಾಗಿ ಹೇಳುವುದಾದರೆ, ಅಧ್ಯಾಪಕರು ವಿದ್ಯಾದಾನದ ವ್ರತಧಾರಕರು, ವಿದ್ಯಾರ್ಥಿಗಳು ವಿದ್ಯಾಭ್ಯಾಸದ ವ್ರತಧಾರಕರು. ದೃಢ ಮನಸ್ಸಿನಿಂದ ವ್ರತಧಾರಣೆ ಮಾಡದಿದ್ದರೆ ಯಾವ ಮಹಾಕಾರ್ಯವೂ ಕೈಗೂಡದು.

ವೇಳಾಪಟ್ಟಿಗೆ ಸಂಬಂಧಿಸಿದಂತೆ ಇನ್ನೊಂದು ಮಾತು; ಸಾಮಾನ್ಯ ದಿನ ಚರಿಯ ವೇಳಾಪಟ್ಟಿಯಲ್ಲದೆ ವಿಶೇಷ ರಜಾದಿನಗಳಲ್ಲಿ ದೊರೆಯುವ ಅಧಿಕ ಸಮಯವನ್ನು ಸದುಪಯೋಗಪಡಿಸಿಕೊಳ್ಳಲು ವಿಶೇಷ ವೇಳಾಪಟ್ಟಿಯನ್ನು ಆಗಾಗ ಹಾಕಿಕೊಳ್ಳುತ್ತಿರಬೇಕು.

೪. ನಿನ್ನ ವೇಳಾಪಟ್ಟಿಯ ವಿಷಯ ಮನೆಮಂದಿಗೆ ತಿಳಿದಿದ್ದರೆ ಒಳ್ಳೆಯದು. ಇದರಿಂದ ಅವರು ನಿನ್ನನ್ನು ಅಧ್ಯಯನದ ವೇಳೆಯಲ್ಲಿ ಇತರ ಕೆಲಸ ಕಾರ್ಯ ಗಳಿಗೆ ಕರೆಯುವುದು ತಪ್ಪುತ್ತದೆ. ನೀನು ಒಮ್ಮೆ \textbf{ಅಧ್ಯಯನಕ್ಕೆ ಕುಳಿತೆಯೆಂದರೆ} ಕನಿಷ್ಠ ಪಕ್ಷ ಒಂದು ಗಂಟೆಯ ಕಾಲ ಅತ್ತಿತ್ತ ಅಲುಗದೆ, ಕಿಟಕಿಯ ಹೊರಗೆ ಕಣ್ಣು ಹಾಕದೆ, \textbf{ಓದುವುದರಲ್ಲೇ ತಲ್ಲೀನನಾಗಿರಬೇಕು.} ಇದು ಮೊದಮೊದಲು ಕಷ್ಟವಾಗಬಹುದು. ಆದರೂ ಹಟತೊಟ್ಟು ಅಭ್ಯಾಸಮಾಡಿದರೆ ನಿನ್ನ ಮೈ ಮನಸ್ಸುಗಳೆರಡು ನಿನ್ನಿಚ್ಛೆಯಂತೆ ನಡೆದುಕೊಳ್ಳಲಾರಂಭಿಸುತ್ತವೆ. ಒಂದು ಗಂಟೆಯ ಬಳಿಕ ಎದ್ದು ಹೊರಗೆ ಗಾಳಿಯಲ್ಲಿ ಅಡ್ಡಾಡಿ, ಒಂದು ಲೋಟ ನೀರು ಕುಡಿದು ಮತ್ತೆ ಓದಲು ಅಥವಾ ಬರೆಯಲು ಕುಳಿತುಕೊ. ಆಗಾಗ ನೀರು ಕುಡಿಯುವುದರಿಂದ ಚೆನ್ನಾಗಿ ರಕ್ತ ಸಂಚಾರವಾಗಿ ಮನಸ್ಸು ಲವಲವಿಕೆಗೊಳ್ಳುತ್ತದೆ.

ಓದುವಾಗ ಸಿಗುವ ಕಷ್ಟದ ಪದಗಳನ್ನು ಆಗಾಗಲೇ ಅರ್ಥಮಾಡಿಕೊಳ್ಳ ಬೇಕು. ಅದಕ್ಕೆ ಒಂದು ಒಳ್ಳೆಯ ಶಬ್ದಕೋಶ ಬಳಿಯಲ್ಲೇ ಇರಬೇಕು. ಒಂದೊಂದು ಶಬ್ದವನ್ನೂ ಸರಿಯಾದ ರೀತಿಯಲ್ಲಿ ಬಳಸಲು ಕಲಿತೆಯಾದರೆ, ನಿನ್ನ ಭಾಷಾಜ್ಞಾನ ವರ್ಧಿಸುವುದರ ಮೂಲಕ ನಿನ್ನ ಅಧ್ಯಯನ ನೂರುಮಡಿ ಹೆಚ್ಚು ಫಲಪ್ರದವಾಗುತ್ತದೆ. ಶಬ್ದಗಳ ಅರ್ಥ ತಿಳಿದಿದ್ದರೆ ವಿಷಯಗಳು ಸ್ಪಷ್ಟವಾಗಿ ಅರ್ಥವಾಗುತ್ತವೆ; ಆಗ ಇನ್ನಷ್ಟು ಓದಬೇಕು, ಮತ್ತಷ್ಟು ಓದ ಬೇಕು ಎಂಬ ಉತ್ಸಾಹ ಹುಟ್ಟಿಕೊಳ್ಳುತ್ತದೆ. ಪುನಃ ಪುನಃ ಓದುವುದರಿಂದ ಪಠ್ಯವಿಷಯಗಳು ಮನದಟ್ಟಾಗುತ್ತವೆ. ಜ್ಞಾಪಕಶಕ್ತಿಯನ್ನು ಹೆಚ್ಚಿಸಿಕೊಳ್ಳಲು ಉಪಾಯವೇನು ಎಂದು ಕೇಳುವ ವಿದ್ಯಾರ್ಥಿಗಳು ಬಹಳ. ಪಾಠವನ್ನು ಅರ್ಥ ಮಾಡಿಕೊಂಡು ಪುನಃ ಪುನಃ ಓದುವುದು-ಬರೆಯುವುದು ಅದಕ್ಕೆ ಪ್ರಧಾನ ಉಪಾಯ. ಇನ್ನೂ ಕೆಲವು ಉಪಾಯಗಳನ್ನು ಮುಂದೆ ತಿಳಿಸುತ್ತೇನೆ.

\textbf{ಮೇಜು-ಕುರ್ಚಿ}ಯ ಅನುಕೂಲತೆ ಇದೆ ತಾನೆ ನಿನಗೆ? ಇಲ್ಲದಿದ್ದರೆ ಕನಿಷ್ಠಪಕ್ಷ \textbf{ಪುಟ್ಟ ಡೆಸ್ಕಾದರೂ ಇರಬೇಕು.} ಡೆಸ್ಕಿನ ಮೇಲಿರುವ ಪುಸ್ತಕಕ್ಕೂ ನಿನ್ನ ಕಣ್ಣುಗಳಿಗೂ ಸಾಕಷ್ಟು ಅಂತರವಿರುವಂತೆ ನೋಡಿಕೊ. ಡೆಸ್ಕಿಗೆ ಅತಿ ಹತ್ತಿರದಲ್ಲಿ ಮುಖವಿಟ್ಟು ಓದುವುದರಿಂದ ಅಥವಾ ಬರೆಯುವುದರಿಂದ ಕಣ್ಣಿಗೂ ಮಿದುಳಿಗೂ ಬೇಗ ಆಯಾಸವುಂಟಾಗಿ, ಹೆಚ್ಚು ಹೊತ್ತು ಅಧ್ಯಯನ ಮಾಡುವ ಸಾಮರ್ಥ್ಯ ಕುಂಠಿತವಾಗುತ್ತದೆ.

ಇನ್ನು, ನೀನು ಬಳಸುವ \textbf{ಪೆನ್ನು-ಪೆನ್ಸಿಲು ಉತ್ತಮ ಮಟ್ಟದ್ದಾಗಿರುವಂತೆ ನೋಡಿಕೊ.} ಚೆನ್ನಾಗಿ ಬರೆಯುವ ಎರಡು ಪೆನ್ನುಗಳಿರುವುದು ಒಳ್ಳೆಯದು. ಅವನ್ನು ಇತರರಿಗೆ ಕೊಡಲೇ ಬೇಡ. ಪೆನ್ನನ್ನು ಹಿಡಿದುಕೊಂಡು ಬರೆಯುವ ರೀತಿ ಒಬ್ಬೊಬ್ಬರದು ಒಂದೊಂದು ತೆರ. ಹೀಗಿರುವಾಗ ನಿನ್ನ ಪೆನ್ನನ್ನು ಇತರರಿಗೆ ಕೊಟ್ಟರೆ ಏನಾಗಬಹುದೆಂದು ನೀನೇ ನೋಡಿಕೊ. ಪೆನ್ನು ನಿನ್ನ ಅನುಕೂಲಕ್ಕೆ ತಕ್ಕಂತೆ ಬರೆಯಬೇಕಾದರೂ ಕೆಲದಿನ ಬರೆದು ಬರೆದು ಅಭ್ಯಾಸ ವಾಗಬೇಕಾಗುತ್ತದೆ. ವೇಗವಾಗಿ ಬರೆದರೂ ಅಕ್ಷರಗಳು ಸುಂದರವಾಗಿರ ಬೇಕಾದರೆ ಅದು ಉತ್ತಮ ಪೆನ್ನಿನಿಂದಲೇ ಸಾಧ್ಯ.

ಉತ್ತರಪತ್ರಿಕೆಗಳು ಸುಂದರವೂ ಚೊಕ್ಕಟವೂ ಆಗಿರಬೇಕೆಂದು ಪರೀಕ್ಷಕರು ಅಪೇಕ್ಷಿಸುತ್ತಾರೆ. ಸುಂದರವಾಗಿದ್ದರೆ ಹೆಚ್ಚು ಅಂಕಗಳನ್ನು ಕೊಡಲು ಅವರ ಕೈ ಮುಂದಾಗುತ್ತದೆ; ಅಂದಗೆಟ್ಟರೆ ಉತ್ತರ ಸರಿಯಾಗಿದ್ದರೂ ಕಡಿಮೆ ಅಂಕಗಳೇ ಬೀಳುವ ಸಂಭವ! ಆದ್ದರಿಂದ ಬರವಣಿಗೆಯ ಕಡೆಗೆ ನೀನು ಸಾಕಷ್ಟು ಗಮನ ಕೊಡಲೇ ಬೇಕು. \textbf{ಉತ್ತಮ ಬರವಣಿಗೆಯಲ್ಲಿ ಐದು ಮುಖ್ಯಾಂಶಗಳು ಸೇರಿರುತ್ತವೆ: (೧) ಅಕ್ಷರಗಳು ಸುಂದರವಾಗಿರಬೇಕು, (೨) ಹೊಡೆದು ಬಡಿದು ಚಿತ್ತು ಮಾಡದೆ ಚೊಕ್ಕಟವಾಗಿರಬೇಕು, (೩) ತಪ್ಪುಗಳು \general{\eng{(spelling mistakes)}} ಇರಬಾರದು, (೪) ಸಾಲುಗಳು ನೇರವಾಗಿರಬೇಕು, (೫) ವಾಕ್ಯಗಳು ವ್ಯಾಕರಣಬದ್ಧವಾಗಿರ ಬೇಕು.}

ಅಧ್ಯಯನದಲ್ಲಿ ಕೇವಲ ಓದುವಿಕೆಯೊಂದೇ ಅಲ್ಲ, ಬರೆಯುವಿಕೆಯೂ ಸೇರಿಕೊಂಡಿದೆ ಎಂಬುದು ನಿನಗೆ ತಿಳಿದಿರಲಿ. ವಿವೇಚನೆಯಿಂದ ಓದುವುದು ಎಷ್ಟು ಪ್ರಾಮುಖ್ಯವೋ ಸಮರ್ಪಕವಾಗಿ ಬರೆಯುವುದೂ ಅಷ್ಟೇ ಪ್ರಾಮುಖ್ಯ. ನೀನು ಪರೀಕ್ಷೆಯಲ್ಲಿ ಪ್ರಶ್ನಪತ್ರಿಕೆಯಲ್ಲಿರುವ ಎಲ್ಲ ಪ್ರಶ್ನೆಗಳನ್ನೂ ಉತ್ತರಿ ಸಲೇ ಬೇಕು. ಆದರೆ ನಿಗದಿತ ವೇಳೆಯಲ್ಲಿ ಅವನ್ನೆಲ್ಲ ಬರೆದು ಮುಗಿಸಬೇಕಾಗು ತ್ತದೆ ಅಲ್ಲವೆ? ಅಂದಮೇಲೆ ವೇಗವಾಗಿ ಬರೆಯುವ ಅಭ್ಯಾಸ ಮಾಡಿಕೊಳ್ಳು ವುದು ಬೇಡವೆ? ಆದ್ದರಿಂದ \textbf{ಪ್ರತಿದಿನ ನೀನು ಒಂದಿಷ್ಟು ಬರವಣಿಗೆಯ ಕಾರ್ಯವನ್ನು ಮಾಡಲೇಬೇಕು.}

ಈ ಬರವಣಿಗೆಯ ಅಭ್ಯಾಸವನ್ನು ಹೇಗೆ ಮಾಡಬೇಕು ಎಂಬ ವಿಷಯವಾಗಿ ಒಂದೆರಡು ಮಾತನ್ನು ಹೇಳಿಬಿಡುತ್ತೇನೆ. ಮೊದಲನೆಯದಾಗಿ ಕಾಪಿ ಬರೆಯುವ \eng{(Copy writing)} ಅಭ್ಯಾಸ ಇಟ್ಟುಕೊ. ನಾನು ವಿದ್ಯಾರ್ಥಿಯಾಗಿದ್ದಾಗ ನಮ್ಮ ಅಧ್ಯಾಪಕರು ನಮ್ಮಿಂದ ಕಾಪಿ ಬರೆಯಿಸಿದ್ದು ಇಂದಿಗೂ ಚೆನ್ನಾಗಿ ನೆನಪಿದೆ. ಏಕೆಂದರೆ ಚೆನ್ನಾಗಿ ಏಟು ಕೊಟ್ಟು ಬರೆಸುತ್ತಿದ್ದರು. ಅಕ್ಷರ ಚೆನ್ನಾಗಿಲ್ಲದಿದ್ದರೆ ಏಟು, ತಪ್ಪಾಗಿ ಬರೆದಿದ್ದರೆ ಏಟು, ಸಾಲುಗಳು ಸೊಟ್ಟಗಿದ್ದರೆ ಏಟು, ಬರವಣಿಗೆ ಯಲ್ಲಿ ಒಂದೊಂದು ಅಕ್ಷರ ಒಂದೊಂದು ಗಾತ್ರದ್ದಾಗಿದ್ದರೆ ಏಟು, ಒಂದು ಶಬ್ದಕ್ಕೂ ಇನ್ನೊಂದು ಶಬ್ದಕ್ಕೂ ಸರಿಯಾದ ಅಂತರವಿರದಿದ್ದರೆ ಏಟು, ಅಕ್ಷರದ ಮೇಲೆ ಅಕ್ಷರ ಬಿದ್ದುಕೊಂಡಂತಿದ್ದರೆ ಏಟು, ಇವೆಲ್ಲದರ ಜೊತೆಗೆ ದಿನಾಲು ಬಳಸುವ ಆ ಕಾಪಿ ಪುಸ್ತಕ ಚೊಕ್ಕಟವಾಗಿಲ್ಲದಿದ್ದರೆ ಏಟು, ಇನ್ನು ಕಾಪಿ ಬರೆಯಲು ತಪ್ಪಿದ ದಿನವಂತೂ ಏಟೋ ಏಟು! ಹೀಗೆ ಏಟಿನ ಮೇಲೆ ಏಟು ತಿಂದೇ ನಾವು ಪಾಠ ಕಲಿತದ್ದು. ಏಟು ಕೊಟ್ಟ ಅಧ್ಯಾಪಕರ ಮೇಲೆ ಅಂದು ಕೋಪ ಬಂದದ್ದೂ ಉಂಟು, ಅವರ ಬೆನ್ನ ಹಿಂದೆ ಬೈದದ್ದೂ ಉಂಟು, ಆದರೆ ಇಂದು ಅವರನ್ನು ನೆನೆದಾಗಲೆಲ್ಲಾ ಹೃದಯ ಅವರಿಗೆ ಕೃತಜ್ಞತೆಯನ್ನರ್ಪಿ ಸುತ್ತದೆ.

ಆದರೆ ನೀನು ಕಾಪಿ ಬರೆಯುವುದೆಂದರೆ, ಪ್ರಾಥಮಿಕ ಶಾಲೆಯ ಮಕ್ಕಳಂತೆ ಮಾದರಿ ವಾಕ್ಯವೊಂದನ್ನು ಮತ್ತೆ ಮತ್ತೆ ಬರೆಯುವುದಲ್ಲ. \textbf{ಯಾವುದಾದರೂ ಪಾಠವನ್ನು ಮುಂದಿಟ್ಟುಕೊಂಡು ಅದನ್ನು ಬರೆಯುತ್ತ ಹೋಗು,} ಇದು ಕನ್ನಡದಂತೆಯೇ ಇಂಗ್ಲೀಷು, ಹಿಂದಿ ಭಾಷೆಯಲ್ಲೂ ನಡೆಯಬೇಕು. ಎಲ್ಲಾ ಭಾಷೆಗಳ ಪುಸ್ತಕದಿಂದ ಒಂದೆರಡು ಪ್ಯಾರಾಗಳನ್ನು ದಿನಾಲೂ ಬರೆಯುತ್ತಾ ಬಂದರಾಯಿತು. ಆದರೆ ಅಕ್ಷರ ಚೆನ್ನಾಗಿದೆಯೇ, ಸಾಲುಗಳು ಸರಿಯಾಗಿವೆಯೇ ಎಂಬುದನ್ನೆಲ್ಲ ನೀನೇ ನೋಡಿಕೊಳ್ಳಬೇಕು. ಬರೆದುದರಲ್ಲಿ ತಪ್ಪುಗಳಿವೆಯೆ ಎಂದು ಪುಸ್ತಕ ನೋಡಿ ಸರಿಪಡಿಸಿಕೊಳ್ಳಬೇಕು. ಕಂಡದ್ದನ್ನು ಕಂಡಂತೆಯೇ ಬರೆಯುವುದಕ್ಕೂ ಎಷ್ಟು ಎಚ್ಚರ ಬೇಕಾಗುತ್ತದೆ ಎಂಬುದು ನಿನಗೆ ಆಗ ತಿಳಿಯುತ್ತದೆ.

ಅದು ಕಾಪಿ ಬರೆಯುವ ವಿಚಾರವಾಯಿತು. ಇನ್ನು ನೀನು \textbf{ಸ್ವತಂತ್ರವಾಗಿ ಒಂದು ವಿಷಯದ ಕುರಿತು ಬರೆಯುವ ಸಾಮರ್ಥ್ಯವನ್ನು ಬೆಳಸಿಕೊಳ್ಳಬೇಕು.} ಇದಕ್ಕೆ ಪಠ್ಯಪುಸ್ತಕದ ಸಹಾಯವನ್ನು ಪಡೆದುಕೊಳ್ಳಬಹುದು. ಒಂದು ಪಾಠ ವನ್ನು ಆಮೂಲಾಗ್ರವಾಗಿ ಓದಿ, ಆ ವಿಷಯವನ್ನು ಸರಿಯಾಗಿ ಮನನ ಮಾಡಿಕೊ. ಬಳಿಕ ಪುಸ್ತಕವನ್ನು ಮುಚ್ಚಿಟ್ಟು, ಆ ವಿಷಯವನ್ನು ನಿನ್ನ ಸ್ವಂತ ವಾಕ್ಯಗಳಲ್ಲಿ ನಿನ್ನದೇ ಆದ ರೀತಿಯಲ್ಲಿ ಬರೆ. ಆಮೇಲೆ ಪುಸ್ತಕದಲ್ಲಿರುವ ವಿಷಯಕ್ಕೂ ನೀನು ಬರೆದಿರುವ ವಿಷಯಕ್ಕೂ ಹೊಂದಿಕೆಯಿದೆಯೇ ಎಂದು ನೋಡಿಕೊ. ಹಾಗೆಯೇ ಅಕ್ಷರಗಳು ಚೆನ್ನಾಗಿವೆಯೇ, ಪಂಕ್ತಿಗಳು ನೆಟ್ಟಗಿವೆಯೇ, ಚಿತ್ತುಗಳು ಎಷ್ಟಾಗಿವೆ ಎಂಬುದನ್ನೆಲ್ಲ ನೋಡಿಕೊ. ಮತ್ತು ಅಷ್ಟನ್ನು ಬರೆ ಯಲು ಎಷ್ಟು ಸಮಯ ಹಿಡಿಯಿತು ಎಂಬುದನ್ನು ನೋಡಿಕೊ. ಮುಂದಿನ ಸಲ ಇನ್ನೊಂದು ವಿಷಯವಾಗಿ ಬರೆಯುವಾಗ ಹಿಂದಿನ ಸಲದ ಎಲ್ಲ ಲೋಪ ದೋಷಗಳನ್ನು ಹೋಗಲಾಡಿಸಿಕೊ. ಈ ಬಗೆಯ ಅಭ್ಯಾಸದಿಂದ ಪರೀಕ್ಷೆಯಲ್ಲಿ ವೇಗವಾಗಿಯೂ ಸುಂದರವಾಗಿಯೂ ಚೊಕ್ಕಟವಾಗಿಯೂ ಬರೆಯಲು ಸಮರ್ಥನಾಗುವೆ.

\textbf{ಜ್ಞಾನಸಂಪಾದನೆಗಾಗಿ ಅಧ್ಯಯನ ಮಾಡುವುದು ಎಷ್ಟು ಮುಖ್ಯವೋ ಪರೀಕ್ಷೆಯಲ್ಲಿ ಯಶಸ್ವಿಯಾಗುವುದಕ್ಕಾಗಿ ಅಧ್ಯಯನ ಮಾಡುವುದೂ ಅಷ್ಟೇ ಮುಖ್ಯ.} ಜ್ಞಾನವಿದ್ದೂ ಪರೀಕ್ಷೆಯಲ್ಲಿ ನಪಾಸಾದರೆ ಏನು ಪ್ರಯೋಜನ? ಪರೀಕ್ಷೆಯಲ್ಲಿ ತೇರ್ಗಡೆಯಾಗಬೇಕಾದರೆ ಪರೀಕ್ಷೆಗೆ ಇಟ್ಟಿರುವ ಎಲ್ಲ ವಿಷಯ ಗಳ ಕಡೆಗೂ ಸಮಾನವಾಗಿ ಗಮನ ಹರಿಸಬೇಕಾಗುತ್ತದೆ. ನಿನಗೆ ಗಣಿತ ಹಾಗೂ ವಿಜ್ಞಾನ–ಇವುಗಳಲ್ಲಿ ಅಷ್ಟಾಗಿ ಆಸಕ್ತಿಯಿಲ್ಲ ಎಂದು ಬರೆದಿರುವೆ. ಆದರೆ ಪರೀಕ್ಷೆಗಾಗಿ ಆಸಕ್ತಿ ತಂದುಕೊಂಡು ಅವುಗಳ ಅಧ್ಯಯನ ನಡಸಬೇಕು. ಇನ್ನು ಕೆಲವು ವಿಷಯಗಳು ನಿನ್ನ ತಲೆಗೆ ಹತ್ತುವುದು ಸ್ವಲ್ಪ ಕಷ್ಟವಾಗಬಹುದು. ಅಂಥ ವಿಷಯಗಳಲ್ಲಿ ನೀನು ಇನ್ನೂ ಹೆಚ್ಚಿನ ಆಸಕ್ತಿ ವಹಿಸಬೇಕು. ಮತ್ತು ಅಧ್ಯಾಪಕರ ಅಥವಾ ಬುದ್ಧಿವಂತ ಹುಡುಗರ ಸಹಾಯವನ್ನು ಪಡೆದು, ಅರ್ಥವಾಗದಿದ್ದ ಅಂಶಗಳನ್ನು ಮನವರಿಕೆ ಮಾಡಿಕೊಳ್ಳಲು ಶ್ರಮಿಸಬೇಕು–ವಿದ್ಯಾಭ್ಯಾಸ ವೊಂದು ತಪಸ್ಸಲ್ಲವೆ? ಕಷ್ಟಪಡಬೇಕು.

ಕೆಲಕೆಲವು ಅಂಶಗಳನ್ನು \textbf{ಬಾಯಿಪಾಠ ಮಾಡಿಟ್ಟುಕೊಂಡಿರಬೇಕಾಗುತ್ತದೆ} ನಿಜ. ಆದರೆ ಪಾಠಕ್ಕೆ ಪಾಠವನ್ನೇ ಉರು ಹೊಡೆಯುವ ಕೆಲಸವನ್ನು ಮಾಡಲೇ ಬೇಡ. ಕೆಲವು ‘ಧೀರ’ರು ನೋಟ್ಸುಗಳನ್ನು–ಅದೂ ಅವರ ಸ್ವಂತದ್ದಲ್ಲ, ಯಾರಿಂದಲೋ ಗಿಟ್ಟಿಸಿಕೊಂಡದ್ದು–ಉರು ಹೊಡೆದು ಪಾಸಾಗಲೆತ್ನಿಸುತ್ತಾರೆ. ನೀನು ಮಾತ್ರ ಹಾಗೆ ಮಾಡಲು ಹೋಗಬೇಡ. ಪಾಠಗಳ ಒಂದೊಂದು ವಾಕ್ಯವನ್ನೂ, ಒಂದೊಂದು ಶಬ್ದವನ್ನೂ ಅರ್ಥಮಾಡಿಕೊಂಡೆಯಾದರೆ ಉರು ಹೊಡೆದದ್ದಕ್ಕಿಂತಲೂ ಹೆಚ್ಚು ಸ್ಪಷ್ಟವಾಗಿ ಆ ವಿಷಯಗಳು ನಿನ್ನ ಮನಸ್ಸಿನಲ್ಲಿ ಉಳಿದಿರುತ್ತವೆ.

ಪಾಠಗಳು ಮನಸ್ಸಿನಲ್ಲಿ ನಿಲ್ಲುವಂತೆ ಮಾಡಲು ಇನ್ನೊಂದು ಉಪಾಯ ಇದೆ. ಭಾನುವಾರವೇ ಮೊದಲಾದ ರಜಾ ದಿನಗಳಲ್ಲಿ ನೀನು ಮೂರ್ನಾಲ್ಕು ಬುದ್ಧಿವಂತ ವಿದ್ಯಾರ್ಥಿಸಹಪಾಠಿಗಳನ್ನು ಸೇರಿಸಿಕೊಂಡು ಸಾಮೂಹಿಕ ಅಧ್ಯ ಯನ, ಚರ್ಚೆ, ವಿಚಾರ ವಿನಿಮಯ ಮಾಡುತ್ತ ಬರಬೇಕು. ಇದರಿಂದ ನಿಮನಿಮಗೆ ಗೊತ್ತಿರುವ ಹೆಚ್ಚಿನ ವಿಚಾರಗಳು ಪರಸ್ಪರರಿಗೆ ತಿಳಿಯುವಂತಾಗು ವುದಲ್ಲದೆ, ಆ ಪಠ್ಯವಿಷಯಗಳನ್ನು ಚರ್ಚಿಸುವಾಗ ಅವುಗಳನ್ನು ಕಿವಿಗಳಿಂದ ಕೇಳುವುದರ ಮೂಲಕವೇ ಅವು ಎಷ್ಟೋ ಮಟ್ಟಿಗೆ ಮನದಟ್ಟಾಗುತ್ತವೆ. ಕಿವಿಯಿಂದ ಕೇಳುವುಕ್ಕೆ ಶ್ರವಣ ಎಂದು ಹೆಸರು. ಈ ಶ್ರವಣ ಎನ್ನುವುದು ಅತ್ಯಂತ ಪರಿಣಾಮಕಾರಿಯಾದುದು. ಎಷ್ಟೋ ಜನ ಸ್ತ್ರೀಪುರುಷರು ಹರಿಕಥೆ- ಪುರಾಣ ಪ್ರವಚನಾದಿಗಳನ್ನು ಕೇಳುತ್ತ ಕೇಳುತ್ತಲೇ ಹಲವಾರು ವಿಚಾರಗಳನ್ನು ತಿಳಿದುಕೊಂಡು ಬುದ್ಧಿವಂತರಾಗಿರುವುದನ್ನು ನೀನು ಕಾಣಬಹುದು. ಆದ್ದ ರಿಂದ ಪ್ರತಿಯೊಂದು ಪಠ್ಯವಿಷಯವನ್ನೂ ನಿನ್ನ ಸಹವಿದ್ಯಾರ್ಥಿಗಳೊಂದಿಗೆ ಚರ್ಚಿಸುವ ಪರಿಪಾಠವನ್ನೂ ಬೆಳಸಿಕೊ. ಇದರಿಂದ ಪರಸ್ಪರರಲ್ಲಿ ಉತ್ಸಾಹ ಹೆಚ್ಚಿ ಮನಸ್ಸು ಉಲ್ಲಾಸಭರಿತವಾಗಿ ಅಧ್ಯಯನ ಹೆಚ್ಚು ಫಲಪ್ರದವಾಗುವುದು ಖಂಡಿತ. ಆದರೆ ನಿಮ್ಮೊಳಗೆ ಹರಟೆ ಶುರುವಾಗದಂತೆ ನೋಡಿಕೊಳ್ಳಬೇಕು.

ಬಹಳ ಮುಖ್ಯವಾದ ಇನ್ನೊಂದು ವಿಚಾರ ಹೇಳುತ್ತೇನೆ ಕೇಳು. ನೀನು ನಿನ್ನ \textbf{ತರಗತಿಯಲ್ಲಿ ಮುಂದಿನ ಬೆಂಚಿನಲ್ಲಿ ಕುಳಿತುಕೊ} ಮತ್ತು ನಿನ್ನ ಕಣ್ಣು-ಕಿವಿ ಗಳೆರಡನ್ನೂ ಅಧ್ಯಾಪಕರಲ್ಲೇ ನೆಟ್ಟು, ಅವರು ಹೇಳುವುದನ್ನು ಗಮನವಿಟ್ಟು ಆಲಿಸು, ಆಗ ಅಧ್ಯಾಪಕರಿಗೂ ನಿನ್ನ ಶ್ರದ್ಧೆಯನ್ನು ಕಂಡು ನಿನ್ನ ಮೇಲೆ ಪ್ರೀತಿ-ವಿಶ್ವಾಸ ಹುಟ್ಟುತ್ತದೆ. ತಮ್ಮ ಪ್ರೀತಿ-ವಿಶ್ವಾಸಕ್ಕೆ ಪಾತ್ರನಾದ ನಿನ್ನ ಮೇಲೆ ಅವರು ವಿಶೇಷ ಗಮನ ಹರಿಯಿಸುತ್ತಾರೆ. ಹಾಗೆಯೇ ಇತರ ಅಧ್ಯಾಪಕರ ವಿಷಯದಲ್ಲಿ ತುಂಬ ವಿನಯ-ವಿಶ್ವಾಸದಿಂದ ನಡೆದುಕೊ. ಅಧ್ಯಾಪಕರ ಬಗ್ಗೆ ಇತರ ಹುಡುಗರು ಏನೇ ಟೀಕೆ ಮಾಡಿದರೂ ನೀನು ಅದರಿಂದ ಪ್ರಭಾವಿತನಾಗಿ ನಿನ್ನ ಮನಸ್ಸನ್ನು ಕೆಡಿಸಿಕೊಳ್ಳಬೇಡ. ನೀನು ಭಕ್ತಿ ಗೌರವಗಳಿಂದ ನಡೆದು ಕೊಂಡುದೇ ಆದರೆ ಎಂಥ ಬಿರುಸಿನ ಅಧ್ಯಾಪಕರೂ ನಿನ್ನ ವಿಷಯದಲ್ಲಿ ಮೃದುವಾಗುತ್ತಾರೆ.

ಕೆಲವು ವಿದ್ಯಾರ್ಥಿಗಳು ಕೇಳುವುದುಂಟು, “ಕೆಲಕೆಲವು ಅಧ್ಯಾಪಕರ ವೈಯ ಕ್ತಿಕ ಜೀವನವನ್ನು ಕಂಡರೆ ಅವರನ್ನು ಗೌರವದಿಂದ ಕಾಣಬೇಕೆಂದೆನಿಸುವುದೇ ಇಲ್ಲವಲ್ಲ?” ಎಂದು. ಆದರೆ ನಾನು ನಿನಗಾಗಿ ಒಂದು ಮಾತನ್ನು ಹೇಳುತ್ತೇನೆ ಕೇಳು–ನೀನು ಅಂಥ \textbf{ಅಧ್ಯಾಪಕರ ವೈಯಕ್ತಿಕ ಜೀವನವನ್ನು ನೋಡುವ ಗೊಡವೆಗೆ ಹೋಗಲೇ ಬೇಡ.} ಅಲ್ಲದೆ ಇತರ ಹುಡುಗರು ಆ ಸಂಬಂಧವಾಗಿ ಹೇಳುವುದನ್ನು ಕೇಳಲೂ ಬೇಡ. ನೀನು ಮಾತ್ರ \textbf{ಪ್ರತಿಯೊಬ್ಬ ಅಧ್ಯಾಪಕರ ಹೃದಯಲ್ಲೂ ಆ ಸಚ್ಚಿದಾನಂದ ಗುರುವೇ ಇದ್ದಾನೆ ಎಂಬ ಭಾವದಿಂದ ಅವರನ್ನು ಗೌರವದಿಂದ ಕಾಣು.} ನಿನಗೆ ಇದು ಕಷ್ಟವೆನಿಸಿದರೂ ವಿವೇಕಿಗಳ ಮಾತಿನ ಮೇಲೆ ವಿಶ್ವಾಸವಿಟ್ಟು ಹಾಗೆ ಮಾಡು. ಇದರಿಂದ ನಿನಗೆ ದೊರೆಯುವ ಪ್ರಯೋಜನ ಅದೆಷ್ಟು ಅಪಾರ ಎಂಬುದನ್ನು ಮುಂದೆ ನೀನೇ ನೋಡುವೆ.

ಇನ್ನು ಕೆಲವು ವಿದ್ಯಾರ್ಥಿಗಳು ಸುಮಾರು ಫೆಬ್ರುವರಿ-ಮಾರ್ಚ್ ತಿಂಗಳಲ್ಲಿ ನಮ್ಮ ಬಳಿ ಬಂದು “ಏಕಾಗ್ರತೆ ಸಂಪಾದಿಸಲು ಏನು ಮಾಡಬೇಕು” ಎಂದು ಕೇಳುವುದುಂಟು. ಹಾಗೆಯೇ “ಧ್ಯಾನ ಮಾಡುವುದು ಹೇಗೆ?” ಎಂದು ಕೇಳು ವವರೂ ಉಂಟು. ಇನ್ನು ಕೆಲವರು, “ಈ ಮನಸ್ಸನ್ನು ಹಿಡಿದಿಡುವುದು ಹೇಗೆ?” ಎಂದು ಕೇಳುತ್ತಾರೆ. ಇವರೆಲ್ಲರ ಪ್ರಶ್ನೆಯ ಹಿಂದಿರುವ ಭಾವವೇನೆಂದರೆ, ವರ್ಷವಿಡೀ ಓದದೆ ಒತ್ತಟ್ಟಿಗಿಟ್ಟ ಪಠ್ಯಪುಸ್ತಕಗಳ ವಿಷಯಗಳನ್ನು ದಿಢೀರನೆ ತಲೆಗೆ ತುಂಬಿಕೊಳ್ಳಲು ಏನಾದರೊಂದು ತಂತ್ರೋಪಾಯವಿದ್ದರೆ ತಿಳಿಸಿ ಎಂದು. ಇಂತಹ ಯಾವ ತಂತ್ರೋಪಾಯವೂ ಈ ಜಗತ್ತಿನಲ್ಲೇ ಇಲ್ಲ. ಮತ್ತೆ ಮತ್ತೆ ಓದುವುದು, ಓದಿದ್ದನ್ನೇ ಓದುವುದು, ಅರ್ಥಮಾಡಿಕೊಂಡು ಓದು ವುದು, ಓದಿದ್ದನ್ನು ನೆನಪು ಮಾಡಿಕೊಳ್ಳುವುದು, ಮರೆತಿದ್ದರೆ ಮತ್ತೆ ಓದು ವುದು–ಇದೇ ಜ್ಞಾಪಕ ಶಕ್ತಿಯ ಮರ್ಮ. \textbf{ಮನಸ್ಸನ್ನು ಏಕಾಗ್ರಗೊಳಿಸಲು ಅಭ್ಯಾಸವೇ ಪ್ರಧಾನ ಸಾಧನ} ಎನ್ನುತ್ತದೆ ಭಗವದ್ಗೀತೆ. ಅಭ್ಯಾಸವೆಂದರೇನು? ಪುನಃ ಪುನಃ ಪ್ರತಿದಿನ ಮಾಡುವ ಪ್ರಯತ್ನ. ನಿನ್ನ ಅಧ್ಯಯನವನ್ನು ಕ್ರಮಪ್ರಕಾರ ಪ್ರತಿದಿನ ಮಾಡುತ್ತ ಬಂದರೆ ಮನಸ್ಸು ಸಹಜವಾಗಿಯೇ ಏಕಾಗ್ರವಾಗುತ್ತದೆ.

ಮನಸ್ಸು ಏಕಾಗ್ರವಾಗಲು ಬೇಕಾದ \textbf{ಇನ್ನೊಂದು ಅತ್ಯಂತ ಮುಖ್ಯಾಂಶ ವೆಂದರೆ ಪ್ರೀತಿ.} ನೀನು ಯಾವುದನ್ನು ಪ್ರೀತಿಸುತ್ತೀಯೋ ಅದರಲ್ಲಿ ನಿನ್ನ ಮನಸ್ಸು ಏಕಾಗ್ರಗೊಂಡಿರುತ್ತದೆ. ಇದು ನಿಯಮ. ಆದ್ದರಿಂದ ನೀನು ನಿನ್ನ ಪಠ್ಯ ವಿಷಯಗಳಲ್ಲಿ ಪ್ರೀತಿಯಿಡಬೇಕು. ಆಗ ಅವುಗಳಲ್ಲಿ ಮನಸ್ಸನ್ನು ಏಕಾಗ್ರ ಗೊಳಿಸಿಕೊಂಡು ಅಧ್ಯಯನ ಮಾಡಲು ಸಾಧ್ಯವಾಗುತ್ತದೆ. 

ಇವಿಷ್ಟು ನಿನ್ನ ಅಧ್ಯಯನಕ್ಕೆ ನೇರವಾಗಿ ಸಂಬಂಧಪಟ್ಟ ವಿಚಾರಗಳಾಯಿತು. ಇನ್ನೂ ಕೆಲವು ಅಂಶಗಳ ಕಡೆಗೆ ನೀನು ಗಮನ ಹರಿಸಬೇಕಾದದ್ದಿದೆ. ಮೊದಲ ನೆಯದಾಗಿ ನಿನ್ನ ಊಟ ತಿಂಡಿಯ ವಿಷಯ. \textbf{ಹಿತವಾಗಿ ಮಿತವಾಗಿ ಕಾಲಕ್ಕೆ ಸರಿಯಾಗಿ ಊಟ} ಮಾಡುವುದರಿಂದ ಮನಸ್ಥಿತಿ ಸದಾ ಹದವಾಗಿರುತ್ತದೆ. ಅತಿಯಾಗಿ ಉಂಡರೆ ನಿದ್ರೆ, ಅತಿ ಕಡಿಮೆ ಉಂಡರೆ ನಿತ್ರಾಣ. ಕಾಲಕ್ಕೆ ಸರಿಯಾಗಿ ಊಟ ಮಾಡದಿದ್ದರೆ ಮನಸ್ಸು ಏರಿಳಿತಗಳಿಗೆ ಒಳಗಾಗುತ್ತದೆ.

ಇನ್ನು ವ್ಯಾಯಾಮದ ವಿಷಯ. ಆಟಗಳನ್ನಾಡುವವರು ಆಡಿಕೊಳ್ಳಲಿ. ನೀನು ಮಾತ್ರ ಒಂದೋ ಯೋಗಾಸನಗಳನ್ನು ಅಭ್ಯಾಸ ಮಾಡು, ಅಥವಾ ಅಂಗ ಸಾಧನೆ ಮಾಡು. ಕೇವಲ ಮುಕ್ಕಾಲು ಗಂಟೆಯಲ್ಲಿ ನಿನ್ನ ಶರೀರದ ಪ್ರತಿ ಯೊಂದು ಅಂಗಕ್ಕೂ ವ್ಯಾಯಾಮ ಸಿಗುವಂತಿದ್ದರಾಯಿತು. ಈ ವ್ಯಾಯಾಮದ ವಿಷಯದಲ್ಲಿ ‘ಹೆಚ್ಚಿಗೆ ಮಾಡಬೇಡ, ಮಾಡುವುದನ್ನು ಬಿಡಬೇಡ’ ಎಂಬ ನಿಯಮವನ್ನಿಟ್ಟುಕೊಂಡರೆ ಕ್ಷೇಮ.

ಯಾವಾಗಲೂ ಕುದಿಸಿ ಆರಿದ ನೀರನ್ನೇ ಕುಡಿಯುವ ಅಭ್ಯಾಸವಿಟ್ಟು ಕೊಂಡರೆ ಎಷ್ಟೋ ಕಾಯಿಲೆಗಳು ಹತ್ತಿರವೇ ಸುಳಿಯದಂತೆ ಮಾಡಬಹುದು. ನಿನ್ನೆಲ್ಲ ಪ್ರಗತಿಗೆ ದೊಡ್ಡ ಪ್ರತಿಬಂಧಕವೆಂದರೆ ಕಾಯಿಲೆಗಳು. ಶರೀರ- ಮನಸ್ಸುಗಳ ಹುಮ್ಮಸ್ಸನ್ನೇ ಅಪಹರಿಸಿಬಿಡುತ್ತವೆ ಅವುಗಳು. ಆದ್ದರಿಂದ ಎಚ್ಚರ!

ಇನ್ನೊಂದು ವಿಚಾರ. ಇದನ್ನು ನೀನು ಅತ್ಯಂತ ಮುಖ್ಯ ವಿಚಾರವೆಂದೇ ಭಾವಿಸಬಹುದು. ಅದೇನೆಂದು ಹೇಳಲೆ? \textbf{ಉತ್ಸಾಹ! ನಿನ್ನಲ್ಲಿ ಸದಾ ಉತ್ಸಾಹ ತುಂಬಿಕೊಂಡಿರಲಿ}–ಅಚ್ಚಳಿಯದ ಉತ್ಸಾಹ! ನಿತ್ಯೋತ್ಸಾಹ! ಜೀವನದ ಎಲ್ಲ ಕ್ಷೇತ್ರಗಳಲ್ಲೂ ನಮ್ಮನ್ನು ಜಯಶಾಲಿಯಾಗಿಸುವ ಕೀಲಿಕೈ ಈ ಉತ್ಸಾಹ. ನೀನೊಂದು ಉತ್ಸಾಹದ ಬುಗ್ಗೆಯೇ ಆಗಬೇಕು. ಸಂಸ್ಕೃತ ಪದವಾದ ಈ ಉತ್ಸಾಹಕ್ಕೆ ಕನ್ನಡದಲ್ಲಿ ಹುಮ್ಮಸ್ಸು, ಹುರುಪು ಎನ್ನಬಹುದು. ನಿನ್ನ ಹಿರಿಯರು ನಿನ್ನನ್ನು ಹುರಿದುಂಬಿಸುವುದರ ಮೂಲಕ ಈ ಉತ್ಸಾಹವನ್ನು ನಿನ್ನಲ್ಲಿ ಮೂಡಿಸ ಬೇಕು. ಆದರೆ ಹೆಚ್ಚಿನ ಹಿರಿಯರು ತಮ್ಮತಮ್ಮ ತಾಪತ್ರಯಗಳಲ್ಲೇ ಮುಳುಗಿ, ಅವರೇ ಇತರರಿಂದ ಪ್ರೋತ್ಸಾಹವನ್ನು ನಿರೀಕ್ಷಿಸುತ್ತಿರುವಾಗ ನಿನಗಾರು ನೀಡು ತ್ತಾರೆ ಉತ್ಸಾಹವನ್ನು? ಆದ್ದರಿಂದ ನೀನೇ ಅದನ್ನು ನಿನ್ನೊಳಗಿನಿಂದಲೇ ಹೊರ ಹೊಮ್ಮಿಸಿಕೊಳ್ಳಬೇಕು.

ಇದಕ್ಕೆ ಇನ್ನೊಂದು ಉಪಾಯವನ್ನು ಅನುಸರಿಸಬಹುದು–“ಈ ಸಲದ ಪರೀಕ್ಷೆಯಲ್ಲಿ ನನ್ನ ಸ್ನೇಹಿತರಿಗಿಂತಲೂ ಹೆಚ್ಚಿನ ಅಂಕಗಳನ್ನು ಪಡೆದು ಉತ್ತಮ ಶ್ರೇಣಿಯಲ್ಲಿ ತೇರ್ಗಡೆ ಹೊಂದುತ್ತೇನೆ” ಎಂದು ನಿನಗೆ ನೀನೇ ಶಪಥ ಮಾಡಿಕೊ. ಇದು ನಿಶ್ಚಯವಾಗಿಯೂ ನಿನ್ನೊಳಗಿರುವ ಛಲವನ್ನು ಬಡಿದೆಬ್ಬಿಸುತ್ತದೆ.

ಒಟ್ಟಿನಲ್ಲಿ ನೀನು ಉತ್ಸಾಹದಿಂದ ಪುಟಿಯುವ ಚೆಂಡಿನಂತಿರಬೇಕೇ ಹೊರತು, ನೆನೆಸಿದ ನಸೆ ನಸೆ ಅವಲಕ್ಕಿಯಂತಿರಬಾರದು. \textbf{ಸದಾ ಪ್ರಸನ್ನನಾಗಿರು.} ಅಳುಮೋರೆ ಅಥವಾ ಗಂಟು ಮುಖ ಹಾಕಿಕೊಂಡಿದ್ದರೆ ಅದು ಇರುವ ಅಲ್ಪ ಉತ್ಸಾಹವನ್ನು ಆರಿಸಿಬಿಡುತ್ತದೆ. ನಗುಮೊಗದಿಂದಿದ್ದರೆ ಅದು ಇರುವ ಅಲ್ಪ ಉತ್ಸಾಹವನ್ನು ವರ್ಧಿಸುತ್ತ ಅದಮ್ಯ ಚೈತನ್ಯದ ಪ್ರತಿರೂಪವೇ ನೀನಾಗುವೆ. ಹೀಗೆ, ಯಾವಾಗ ಈ ಉತ್ಸಾಹವನ್ನು ನೀನು ಕಾಪಾಡಿಕೊಳ್ಳುತ್ತೀಯೋ ಆಗ ನಿನ್ನ ಅಧ್ಯಯನವೆಲ್ಲ ಆನಂದಮಯವಾಗುತ್ತದೆ, ಯಶಸ್ವಿಯಾಗುತ್ತದೆ. ಆದ್ದ ರಿಂದ ದಿನೇ ದಿನೇ ನಿನ್ನ ಉತ್ಸಾಹದ ಚಿಲುಮೆ ಇನ್ನಷ್ಟು ಹೆಚ್ಚು ಚಿಮ್ಮುವಂತೆ ನೋಡಿಕೊ.

ಈ ಉತ್ಸಾಹಕ್ಕೆ ಸಂಬಂಧಿಸಿದಂತೆ ಇನ್ನೊಂದು ಅಂಶವನ್ನು ತಿಳಿದು ಕೊಳ್ಳಲೇ ಬೇಕು ನೀನು. ಮನಸ್ಸು ನಿದ್ರೆಯ ಜಾಡ್ಯವನ್ನು ಕಳೆದುಕೊಂಡು ಸಂಪೂರ್ಣ ಎಚ್ಚತ್ತುಕೊಂಡಿರುವಾಗ ಅದು ಚಟುವಟಿಕೆಯಿಂದ ಕೂಡಿ ಚುರುಕಾಗಿರುತ್ತದೆ. ಆದ್ದರಿಂದ ಸಹಜವಾಗಿ ಉತ್ಸಾಹಪೂರ್ಣವಾಗಿರುತ್ತದೆ. ಹಾಗೆಯೇ \textbf{ಉತ್ಸಾಹಪೂರ್ಣ} ವಾತಾವರಣದಲ್ಲಿ \textbf{ಮನಸ್ಸು} ತನ್ನ ನಿದ್ರೆಯ ಜಾಡ್ಯವನ್ನು ತ್ಯಜಿಸಿ \textbf{ಎಚ್ಚರವಾಗಿರುತ್ತದೆ.} ಈ ಅಂಶವನ್ನು ಗಮನಿಸಿದಾಗ ನೀನು ನಿನ್ನ ಉತ್ಸಾಹವನ್ನು ಕಾಪಾಡಿಕೊಂಡರೆ ಅದರ ಪರಿಣಾಮವಾಗಿ ನಿನ್ನ ಮನಸ್ಸು ಸದಾ ಜಾಗೃತವಾಗಿರುತ್ತದೆ ಎಂಬುದು ತಿಳಿಯುವುದಿಲ್ಲವೆ?

ಈ ಮನಸ್ಸನ್ನು ಎಚ್ಚರವಾಗಿಡುವುದಕ್ಕಾಗಿಯೇ ಎಷ್ಟೋ ವಿದ್ಯಾರ್ಥಿಗಳು ಗಂಟೆಗಂಟೆಗೂ ಟೀ-ಕಾಫೀ ಕುಡಿಯುವುದನ್ನು ನೀನು ನೋಡಿರಬಹುದು. ಆದರೆ ಈ ಕಾಫಿ-ಟೀ ಕ್ರಮೇಣ ಮೈಗೆ ಒಗ್ಗಿಹೋಗಿ ಅದನ್ನು ಕುಡಿದೇ ಗಾಢ ನಿದ್ರೆ ಮಾಡಿಬಿಡುತ್ತಾರೆ ಅವರು! ಆದ್ದರಿಂದ ಉನ್ನತ ಶ್ರೇಣಿಯಲ್ಲಿ ತೇರ್ಗಡೆ ಹೊಂದುವ ಮಹದಾಕಾಂಕ್ಷೆಯನ್ನಿಟ್ಟುಕೊಳ್ಳುವುದೇ ಉತ್ಸಾಹವನ್ನು ಕಾಪಾಡಿ ಕೊಳ್ಳಲು ಉತ್ತಮ ಉಪಾಯ. ನಿನ್ನ ವಿಷಯದಲ್ಲಿ ಮಹದಾಕಾಂಕ್ಷೆ ಎಂದ ರೇನು? ಇತರರಿಗಿಂತ ಅಧಿಕ ಅಂಕಗಳನ್ನು ಪಡೆದು ಅತ್ಯುತ್ತಮ ಶ್ರೇಣಿಯಲ್ಲಿ ತೇರ್ಗಡೆಹೊಂದುವ ಅಭಿಲಾಷೆ.

ನಿನ್ನ ಮಾರ್ಗದಲ್ಲಿ ಇನ್ನೊಂದು ವೈರಿ ಇದೆ ಎಂಬುದನ್ನು ನೀನು ಈಗಿ ನಿಂದಲೇ ತಿಳಿದಿರಬೇಕು. ಯಾವುದು ಗೊತ್ತೇನು ಆ ವೈರಿ? ಹೆದರಿಕೆ! ಪರೀಕ್ಷಾ ಭೀತಿ! ಈ ವೈರಿಯು ಸುಮಾರಾಗಿ ಎಲ್ಲ ವಿದ್ಯಾರ್ಥಿಗಳ ಮೇಲೂ ಆಕ್ರಮಣ ಮಾಡುತ್ತಿರುತ್ತದೆ. ದುರ್ಬಲ ವಿದ್ಯಾರ್ಥಿಗಳು ಈ ವೈರಿಯ ಹೊಡೆತವನ್ನು ತಾಳಲಾರದೆ ಜ್ವರ ಬೀಳುವುದುಂಟು. ವಾಂತಿ-ಭೇದಿಗೆ ಬಲಿಯಾಗುವುದುಂಟು. ಹಾಗೆಯೇ ಇನ್ನೂ ಏನೇನೋ ಶರೀರ-ಮನೋವಿಕಾರಗಳಿಗೆ ಗುರಿಯಾಗುವು ದುಂಟು. ಈ ಬಗೆಯ ಜ್ವರಕ್ಕೆ ಪರೀಕ್ಷಾ ಜ್ವರ ಎಂದೇ ಹೆಸರಾಗಿದೆ. ಈ ಜ್ವರವೇ ಮೊದಲಾದ ವಿವಿಧ ವಿಕಾರಗಳಿಗೆ ಮೂಲ ಕಾರಣ ಹೆದರಿಕೆಯೇ ಹೊರತು ಇನ್ನೇನೂ ಅಲ್ಲ. ಆದ್ದರಿಂದ ಈಗ ನಿನಗೆ ನೀನೇ ಒಂದು ಪ್ರಶ್ನೆ ಹಾಕಿಕೊ: “ಹೆದರಿಕೊಂಡರೆ ನನಗೆ ಸಿಗುವುದಾದರೂ ಏನು? ಕಾಯಿಲೆ. ಈ ಸೌಭಾಗ್ಯಕ್ಕೆ ನಾನೇಕೆ ಹೆದರಿಕೊಂಡು ಸಾಯಲಿ? ಇಲ್ಲ, ನಾನು \textbf{ಚೆನ್ನಾಗಿ ಅಧ್ಯಯನ ಮಾಡುತ್ತೇನೆ} ಮತ್ತು \textbf{ಧೈರ್ಯ}ದಿಂದ ಪರೀಕ್ಷೆಯಲ್ಲಿ ಬರೆಯುತ್ತೇನೆ” ಎಂದು.

ನಿಜ; ಪರೀಕ್ಷಾ ಭೀತಿಯಿಂದ ಪಾರಾಗಲು ಇಲ್ಲಿದೆ ಉಪಾಯ. ಯಾವುದು ಆ ಉಪಾಯ? ವರ್ಷದ ಪ್ರಾರಂಭದಿಂದಲೇ \textbf{ಶಿಸ್ತಿನಿಂದ ಅಧ್ಯಯನ} ಮಾಡು ವುದು. ಶಿಸ್ತಿನಿಂದ ಅಧ್ಯಯನ ಮಾಡುವವನು ಪರೀಕ್ಷೆಗೆ ಹೆದರಬೇಕಾಗಿಯೇ ಇಲ್ಲ. ಆದರೂ, ನೋಡು, ಮನುಷ್ಯನ ಸ್ವಭಾವದಲ್ಲಿ ಈ ಭೀತಿಯೆಂಬು ದೊಂದು ಸೇರಿಕೊಂಡೇ ಇರುತ್ತದೆ. ಆದ್ದರಿಂದ ನೀನು ಎಷ್ಟೇ ಅಧ್ಯಯನ ಮಾಡಿದರೂ ಈ ಭೀತಿ ನಿನ್ನನ್ನು ಕಾಡಬಹುದು. ಅಲ್ಲದೆ, ನಿನ್ನ ಶಾಲೆಯ ಇತರ ಸೋಮಾರಿ ವಿದ್ಯಾರ್ಥಿಗಳು, ‘ಅಯ್ಯೋ, ಪರೀಕ್ಷೆ ಹತ್ತಿರ ಬಂದೇ ಬಿಟ್ಟಿತು; ರಿವಿಷನ್ ಇನ್ನೂ ಮುಗಿದಿಲ್ಲ. ಏನು ಮಾಡಲಿ ಈಗ!’ ಎಂದು ಹೆದರಿಕೊಳ್ಳು ವುದನ್ನು ಕಂಡು ನಿನ್ನ ಮನಸ್ಸೂ ವಿನಾಕಾರಣ ಭಯಗ್ರಸ್ತವಾಗುವ ಸಂಭವ ವಿದೆ. ಆದರೆ, ತಿಳಿದುಕೊ, ಈ ಭಯಕ್ಕೇನಾದರೂ ನೀನು ಆಶ್ರಯ ಕೊಟ್ಟೆಯೋ ಅದು ನಿನ್ನನ್ನು ಸಾಯಿಸದಿರಬಹುದು, ಆದರೆ ನಿನ್ನ ಶರೀರ-ಮನಸ್ಸುಗಳ ಶಕ್ತಿಯನ್ನು ಉಡುಗಿಸುವುದು ಖಂಡಿತ. ಇದರ ಪರಿಣಾಮವಾಗಿ ಆಗುವುದಿಷ್ಟೆ –ಪರೀಕ್ಷೆಯಲ್ಲಿ ಬರೆಯುವ ವೇಳೆಗೆ ಸರಿಯಾಗಿ ನೀನು ಕಲಿತದ್ದೆಲ್ಲ ಮರವೆ ಯಾದಂತೆನಿಸುವುದು. ಒಂದು ಪ್ರಶ್ನೆಗೆ ಇನ್ನೊಂದು ಉತ್ತರವನ್ನು ಬರೆಯು ವಂತಾಗುವುದೂ ಈ ಹೆದರಿಕೆಯ ದೆಸೆಯಿಂದಲೇ.

ಆದ್ದರಿಂದ ಶಿಸ್ತುಬದ್ಧ ಅಧ್ಯಯನದೊಂದಿಗೆ, ಆತ್ಮವಿಶ್ವಾಸವನ್ನು ಬೆಳೆಸಿ ಕೊಳ್ಳುವುದರ ಮೂಲಕ, ಮನದ ಮೂಲೆಯಲ್ಲಿ ಮನೆ ಮಾಡಿಕೊಂಡಿರುವ ಭಯದ ಮನೋವೃತ್ತಿಯನ್ನು ನಿವಾರಿಸಿಕೊಳ್ಳಬೇಕಾದುದುದು ಮುಖ್ಯವಾಗು ತ್ತದೆ. ನೋಡಿದೆಯಾ, ಇಲ್ಲಿ ನೀನು ಇನ್ನೊಂದು ಹೊಸ ಶಬ್ದವನ್ನು ಕೇಳು ತ್ತಿರುವೆ! ಆತ್ಮವಿಶ್ವಾಸ! ಎಂದರೇನು? \textbf{ನಿನ್ನ ಶಕ್ತಿಯಲ್ಲಿ ನಿನಗೆ ನಂಬಿಕೆ;} ನೀನು ಮಾಡಿದ ಅಧ್ಯಯನದಲ್ಲಿ ನಿನಗೆ ನಂಬಿಕೆ! ಪರೀಕ್ಷೆಯಲ್ಲಿ ಶಾಂತ ಮನಸ್ಕನಾಗಿ ಸರಿಯಾದ ಉತ್ತರವನ್ನೇ ಬರೆಯುತ್ತೇನೆ ಎಂಬ ದೃಢವಿಶ್ವಾಸ –ಇದೇ ಆತ್ಮವಿಶ್ವಾಸ. ಈ ಆತ್ಮವಿಶ್ವಾಸವನ್ನು ಬೆಳೆಸಿಕೊಂಡೆಯಾದರೆ ಉತ್ಸಾಹ ಬರುತ್ತದೆ. ಹೆದರಿಕೆ ಹೋಗುತ್ತದೆ. 

ಸರಿ, ನೀನೀಗ ಹಲವಾರು ವಿಚಾರಗಳನ್ನು ತಿಳಿದಂತಾಯಿತು. ಇಷ್ಟು ವಿಚಾರಗಳನ್ನು ತಿಳಿದ ನಿನಗೆ ಯಶಸ್ಸು ಕಟ್ಟಿಟ್ಟದ್ದು. ಆದರೆ ಅಷ್ಟೇ ಅಲ್ಲ; ಇನ್ನೂ ಒಂದು ವಿಚಾರವನ್ನು ತಿಳಿಯುವುದು ಬಾಕಿಯಿದೆ. ನೀನು ಈ ಪತ್ರವನ್ನು ಆಗಾಗ ಓದಿಕೊಳ್ಳಬೇಕೆಂಬುದೇ ಆ ವಿಚಾರ. ನೀನು ನಿನ್ನ ಪಠ್ಯಪುಸ್ತಕಗಳನ್ನು ಅಧ್ಯಯನ ಮಾಡಲು ಆರಂಭಿಸುವುದಕ್ಕಿಂತ ಮೊದಲು ಈ ಪತ್ರವನ್ನು ಅಧ್ಯ ಯನ ಮಾಡಿ ಇದರಲ್ಲಿರುವ ಎಲ್ಲ ಅಂಶಗಳನ್ನೂ ಹೃದ್ಗತಮಾಡಿಕೊಳ್ಳಬೇಕು. ಹಾಗೆಯೇ, ಮುಂದೆ, ಇದರಲ್ಲಿ ನೀಡಿರುವ ಸಲಹೆಗಳಂತೆ ದಿನಂಪ್ರತಿಯೂ ನಡೆದುಕೊಳ್ಳುತ್ತಿರುವೆಯೋ ಇಲ್ಲವೋ ಎಂಬುದನ್ನೂ ನೋಡಿಕೊಳ್ಳುತ್ತಿರ ಬೇಕು. 

ಮುಂದಿನ ಪರೀಕ್ಷೆಯಲ್ಲಿ ನೀನು ಉನ್ನತ ಶ್ರೇಣಿಯಲ್ಲಿ ಉತ್ತೀರ್ಣನಾಗು ವಂತೆ ದೇವರು ನಿನಗೆ ಅನುಗ್ರಹ ಮಾಡಲಿ! 

ಹೃತ್ಪೂರ್ವಕ ಶುಭಾಶಯಗಳೊಂದಿಗೆ,

\titleauthor{ಸ್ವಾಮಿ ಪುರುಷೋತ್ತಮಾನಂದ}

