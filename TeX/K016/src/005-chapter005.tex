
\chapter{ಪ್ರಾರ್ಥಿಸಿರಿ! ಅಥವಾ, ಭಾವಿಸಿರಿ!}

ಜೂನ್ ಬಂತೆಂದರೆ ಶಾಲೆಗಳು ತೆರೆದು ಪಾಠಪ್ರವಚನಗಳು ಪ್ರಾರಂಭ. ಸುಮಾರು ಒಂದೂವರೆ ತಿಂಗಳ ರಜೆಯ ಮಜವನ್ನು ಅನುಭವಿಸಿದ ವಿದ್ಯಾರ್ಥಿ ಗಳು ಈಗ ಒಮ್ಮನಸ್ಸಿನಿಂದ ಅಧ್ಯಯನದಲ್ಲಿ ತೊಡಗಬೇಕು. ಸಾಮಾನ್ಯ ವಿದ್ಯಾರ್ಥಿಗಳು ಇನ್ನೂ ರಜಾದಿನಗಳ ಗುಂಗಿನಲ್ಲೇ ಇದ್ದರೂ ಉತ್ತಮ ವಿದ್ಯಾರ್ಥಿಗಳು ಮಾತ್ರ ಶ್ರದ್ಧೆಯಿಂದಲೇ ಅಧ್ಯಯನವನ್ನು ಪ್ರಾರಂಭಿಸುತ್ತಾರೆ. ಆದರೆ ಈ ಶ್ರದ್ಧಾವಂತ ವಿದ್ಯಾರ್ಥಿಗಳಲ್ಲಿ ಹಲವರನ್ನು ಪರೀಕ್ಷಾ ಭಯವೊಂದು ಹಿಡಿದುಕೊಂಡಿರುತ್ತದೆ. ಇನ್ನು ಕೆಲವರು ತಮ್ಮ ಹತ್ತಾರು ಪಠ್ಯವಿಷಯಗಳನ್ನು ಓದಿ ನೆನಪಿಟ್ಟುಕೊಳ್ಳಲು ಕಷ್ಟ ಪಡುತ್ತಿರುತ್ತಾರೆ. ಮತ್ತೆ ಕೆಲವರಿಗೆ ಪಾಠ ಸರಿಯಾಗಿ ಅರ್ಥವೇ ಆಗುತ್ತಿಲ್ಲವಲ್ಲ ಎಂಬ ಸಂಕಟ. ಇದೊಂದು ವಿಶ್ವ ವ್ಯಾಪಕ ಸಮಸ್ಯೆ. ಈ ಸಮಸ್ಯೆಗೆ ಪರಿಹಾರಗಳು ಇಲ್ಲದಿಲ್ಲ. ಆದರೆ ಆ ಪರಿಹಾರಗಳನ್ನು ಕಂಡುಕೊಳ್ಳಲು ತಾಳ್ಮೆಯಿಂದ ಆಲೋಚಿಸಬೇಕು. ಕೆಲವಾರು ಪರಿಹಾರೋಪಾಯಗಳನ್ನು ‘ಅಧ್ಯಯನದಲ್ಲಿ ಏಕಾಗ್ರತೆ’ ಹಾಗೂ ‘ವಿದ್ಯಾರ್ಥಿ ಗೊಂದು ಪತ್ರ’ ಎಂಬ ಲೇಖನಗಳಲ್ಲಿ ಅದಾಗಲೇ ಓದಿದ್ದೀರಿ. ಈಗ ಇನ್ನೂ ಒಂದು ಉಪಾಯ ಇಲ್ಲಿದೆ.

ಇದು ಅತ್ಯಂತ ಪರಿಣಾಮಕಾರಿ ಮತ್ತು ಒಂದು ದೃಷ್ಟಿಯಿಂದ ತುಂಬ ಸುಲಭ ಕೂಡ. ಆದರೆ ಇಲ್ಲಿ ಸೂಚಿಸಿದಂತೆ ತಪ್ಪದೆ ಮಾಡುತ್ತ ಬರಬೇಕು, ಅಷ್ಟೆ. ಯಾವುದು ಆ ಉಪಾಯ?–ಪ್ರಾರ್ಥನೆ. ಪ್ರಾರ್ಥನೆ ಎಂದಾಕ್ಷಣ ಅದು ಭಗವಂತನಿಗೇ ಎಂಬುದು ನಿಶ್ಚಯ. ಭಗವಂತ ಯಾರು? ಎಲ್ಲಿದ್ದಾನೆ? ಅವನು ನಮ್ಮ ಹೃದಯದೊಳಗೇ ಇದ್ದಾನೆ. ಅವನನ್ನು ರಾಮ ಕೃಷ್ಣ ಗಣೇಶ ಲಕ್ಷ್ಮಿ ಸರಸ್ವತಿ ಏಸುಕ್ರಿಸ್ತ ಇವೇ ಮೊದಲಾದ ಯಾವ ನಾಮರೂಪಗಳಿಂದ ಬೇಕಾ ದರೂ ಭಾವಿಸಬಹುದು. ಅದು ಅವರವರ ಇಷ್ಟ. ಅವನು ಸರ್ವಶಕ್ತ, ಅವನು ಸರ್ವವ್ಯಾಪಿ. ಅವನು ಸರ್ವಜ್ಞ, ಅವನು ಪರಮ ಕರುಣಾಮಯ. ಅವನು ಸರ್ವಶಕ್ತನಾದುದರಿಂದ ನಮ್ಮ ಪ್ರಾರ್ಥನೆಯನ್ನು ಈಡೇರಿಸಿಕೊಡಲು ಅವನಿ ಗೇನೇನೂ ಕಷ್ಟವಿಲ್ಲ. ಅವನು ಪರಮಕರುಣಾಮಯನಾದ್ದರಿಂದ ಪ್ರಾರ್ಥನೆಗೆ ಓಗೊಡದೆ ಇರಲಾರ. \textbf{ಆದರೆ ಭಗವಂತ ನಮ್ಮ ಪ್ರಾರ್ಥನೆಗೆ ಓಗೊಟ್ಟೇ ಗೊಡುತ್ತಾನೆ ಎಂಬ ವಿಷಯದಲ್ಲಿ ವಿಶ್ವಾಸ-ಶ್ರದ್ಧೆ ಇರಬೇಕಾದ್ದು ಬಹಳ ಮುಖ್ಯ.}

ಈಗ ವಿದ್ಯಾರ್ಥಿಗಳು ಏನೆಂದು ಪ್ರಾರ್ಥನೆ ಮಾಡಬೇಕು? ಯಾರಿಗೆ ಏನು ಬೇಕೋ ಅವರು ಅದಕ್ಕಾಗಿ ಪ್ರಾರ್ಥಿಸುವುದು ಲೋಕರೂಢಿ. ವಿದ್ಯಾರ್ಥಿಗಳಿಗೆ ವಿದ್ಯೆ ಬೇಕು, ಆ ವಿದ್ಯೆ ತಲೆಗೆ ಹತ್ತಬೇಕು, ಪರೀಕ್ಷೆಯಲ್ಲಿ ಪಾಸಾಗಬೇಕು, ಉತ್ತಮ ಶ್ರೇಣಿ ದೊರಕಬೇಕು–ಅಲ್ಲವೆ? ಆದ್ದರಿಂದ ಭಗವಂತನಲ್ಲಿ ಇದನ್ನೇ ಪ್ರಾರ್ಥಿಸಬೇಕು:

\begin{myquote}
ಹೇ ಭಗವನ್, ನಿನಗೆ ಭಕ್ತಿಪೂರ್ವಕ ಪ್ರಣಾಮಗಳು.\\
 ನನಗೆ ಹೆಚ್ಚೆಚ್ಚು ಓದಲು ಬುದ್ಧಿ ಕೊಡು;\\
 ಓದಿದ್ದನ್ನು ಅರ್ಥ ಮಾಡಿಕೊಳ್ಳಲು ಸಾಮರ್ಥ್ಯ ಕೊಡು;\\
 ಅರ್ಥವಾದದ್ದನ್ನು ನೆನಪಿಟ್ಟುಕೊಳ್ಳುವ ಶಕ್ತಿ ಕೊಡು;\\
 ನನ್ನನ್ನು ಉತ್ತಮ ವಿದ್ಯಾವಂತನನ್ನಾಗಿ ಮಾಡು.
\end{myquote}

\begin{myquote}
ಹೇ ದೇವಾ, ನಿನಗೆ ಶರಣಾಗಿದ್ದೇನೆ.\\
 ದಯಮಾಡಿ ನನ್ನೀ ಪ್ರಾರ್ಥನೆಯನ್ನು ಈಡೇರಿಸಿಕೊಡು.
\end{myquote}

ಪ್ರಾರ್ಥನೆ ಮಾಡುವಾಗ ಕಣ್ಣುಗಳನ್ನು ಮುಚ್ಚಿಕೊಳ್ಳುವುದು ಒಳ್ಳೆಯದು. ಏಕೆಂದರೆ ಕಣ್ಣುಗಳು ತೆರೆದಿರುವಾಗ ಹೊರಗಣ ಬೆಳಕು-ವಸ್ತುಗಳು ಕಾಣಿಸು ವುದರಿಂದ ಮನಸ್ಸು ಹೃದಯದೊಳಕ್ಕೆ ಸರಿಯಲು ಸಾಧ್ಯವಾಗುವುದಿಲ್ಲ. ಆದ್ದ ರಿಂದ ಕಣ್ಣುಗಳನ್ನು ಮುಚ್ಚಿಕೊಂಡು, ಅಂತರಂಗದೊಳಗಿರುವ ಪರಮಾತ್ಮ ನನ್ನು ಹೃತ್ಪೂರ್ವಕವಾಗಿ ಪ್ರಾರ್ಥಿಸಬೇಕು.

ವಿದ್ಯೆ ಬೇಕೆಂಬ ವಿದ್ಯಾರ್ಥಿಗಳು ರಾತ್ರಿ ಮಲಗಿಕೊಳ್ಳುವ ಮುನ್ನ ಈ ಪ್ರಾರ್ಥನೆಯನ್ನು ತಪ್ಪದೆ ಮೂರು ತಿಂಗಳ ಕಾಲ ಮಾಡಿನೋಡಲಿ. ಇದರ ಸತ್ಪರಿಣಾಮದ ಮಧುರ ಫಲವನ್ನು ಅವರು ಆ ಅಲ್ಪಾವಧಿಯಲ್ಲೇ ಅನು ಭವಿಸುವುದು ಖಂಡಿತ. ಆದರೆ ಪ್ರಾರ್ಥನೆ ಹೃತ್ಪೂರ್ವಕವಾಗಿರಬೇಕು, ಶ್ರದ್ಧೆ ಯಿಂದ ಕೂಡಿರಬೇಕು. ಏಕೆಂದರೆ \textbf{ಶ್ರದ್ಧೆಯೇ ಸಿದ್ಧಿಯ ಕೀಲಿಕೈ.}

ಭಗವಂತನು ನಮ್ಮ ಪ್ರಾರ್ಥನೆಯನ್ನು ಹೇಗೆ ನಡೆಸಿಕೊಡುತ್ತಾನೆ ಗೊತ್ತೆ? ಅವನು ಮನಸ್ಸನ್ನು ಅಧ್ಯಯನಕ್ಕೆ ಅಣಿಗೊಳಿಸಿಬಿಡುತ್ತಾನೆ. ಮನಸ್ಸಿನಲ್ಲಿ ಶಕ್ತಿ ಯನ್ನೂ ತೇಜಸ್ಸನ್ನೂ ತುಂಬುತ್ತಾನೆ. ಆಗ ಮನಸ್ಸು ಸಮರ್ಥವಾಗುತ್ತದೆ. ಸಮರ್ಥ ಮನಸ್ಸು ಸಾಧಿಸಲಾರದ್ದು ಯಾವುದಿದೆ? ಆದ್ದರಿಂದ ಪ್ರಾರ್ಥನೆ ಯೆಂಬ ಈ ಅದ್ಭುತ ಉಪಾಯದ ಪ್ರಯೋಜನವನ್ನು ಪ್ರತಿಯೊಬ್ಬ ವಿದ್ಯಾರ್ಥಿಯೂ ಹೆಚ್ಚೆಚ್ಚು ಪಡೆಯುವಂತಾಗಬೇಕು.

ಇನ್ನು ಕೆಲವು ವಿದ್ಯಾರ್ಥಿಗಳಿದ್ದಾರೆ. ಅವರಿಗೆ ದೇವರಲ್ಲಿ ನಂಬಿಕೆಯಿಲ್ಲ. ನಂಬುವುದಕ್ಕೆ ಇಷ್ಟವೂ ಇಲ್ಲ. ಅಂಥವರು ಏನು ಮಾಡಬೇಕು? ಈ ಪ್ರಶ್ನೆಗೂ ಉತ್ತರವಿದೆ. ಅಂಥವರು ತಮಗೆ ತಾವೇ ಹೀಗೆ ದೃಢವಾಗಿ ಹೇಳಿಕೊಳ್ಳಲಿ:

\begin{myquote}
“ನಾನು ಹೆಚ್ಚೆಚ್ಚು ಓದುತ್ತೇನೆ, ಹೆಚ್ಚೆಚ್ಚು ಅಧ್ಯಯನ ಮಾಡು ತ್ತೇನೆ. ಅಧ್ಯಯನ ಮಾಡಿದ್ದನ್ನು ಅರ್ಥಮಾಡಿಕೊಳ್ಳುತ್ತೇನೆ, ಮತ್ತು ಚೆನ್ನಾಗಿ ನೆನಪಿಟ್ಟುಕೊಳ್ಳುತ್ತೇನೆ. ಹೀಗೆ ಮಾಡುತ್ತ ನಾನು ಸಕಾಲದಲ್ಲಿ ಶ್ರೇಷ್ಠ ವಿದ್ಯಾವಂತನಾಗಿಯೇ ತೀರುತ್ತೇನೆ.”
\end{myquote}

ಇದು ಪ್ರಾರ್ಥನೆಯಲ್ಲ. ಇದೊಂದು ಭಾವನೆ. ಈ ರೀತಿ ಭಾವಿಸುತ್ತ ಭಾವಿಸುತ್ತಲೇ ವಿದ್ಯಾರ್ಥಿಯು ಸಕಾಲದಲ್ಲಿ ಶ್ರೇಷ್ಠ ವಿದ್ಯಾವಂತನಾಗುವುದು ಖಂಡಿತ. ಏಕೆಂದರೆ “ಯದ್ಭಾವಂ ತದ್ಭವತಿ” ಎಂಬಂತೆ, \textbf{ನಾವು ಹೇಗೆ ಭಾವಿಸುತ್ತೇವೆಯೋ ಹಾಗೆಯೇ ಆಗುತ್ತೇವೆ.} ಇದೊಂದು ಅನಿವಾರ್ಯ ನಿಯಮ. ಆದರೆ ಭಾವಿಸುವುದನ್ನು ಹೃತ್ಪೂರ್ವಕವಾಗಿ ಭಾವಿಸಬೇಕು, ಶ್ರದ್ಧೆ ಯಿಂದ ಭಾವಿಸಬೇಕು. ಇದು ಬಹಳ ಮುಖ್ಯ. ಏಕೆಂದರೆ ಪ್ರಾರ್ಥನೆಯೇ ಆಗಲಿ ಭಾವನೆಯೇ ಆಗಲಿ, ಶಕ್ತಿಯುತವಾಗುವುದು ಹಾಗೂ ಫಲಕಾರಿಯಾಗುವುದು ಶ್ರದ್ಧೆಯಿಂದ. ಈ ರಹಸ್ಯವನ್ನು ಪ್ರತಿಯೊಬ್ಬ ವಿದ್ಯಾರ್ಥಿಯೂ ಅರಿತಿರಬೇಕು.

ಈ ಪ್ರಾರ್ಥನೆಯನ್ನು ಅಥವಾ ಭಾವನೆಯನ್ನು ರಾತ್ರಿ ಮಲಗುವ ಹೊತ್ತಿಗೆ ಮಾಡಬೇಕೆಂಬುದರಲ್ಲಿ ಇನ್ನೊಂದು ರಹಸ್ಯವಿದೆ. ಮಲಗುವ ವೇಳೆಯಲ್ಲಿ ಮಾಡಿದ ಆಲೋಚನೆಗಳು, ಭಾವನೆಗಳು ಮನಸ್ಸಿನ ಆಳಕ್ಕೆ ಇಳಿಯುತ್ತವೆ. ಬೆಳಗ್ಗೆ ಏಳುವಾಗ ಅದೇ ಭಾವನೆಗಳೇ ಇನ್ನಷ್ಟು ಪುಷ್ಟಿಗೊಂಡು ವ್ಯಕ್ತ ವಾಗುತ್ತವೆ. ಒಂದೇ ಭಾವನೆಯನ್ನು ಕೆಲವಾರು ತಿಂಗಳ ಕಾಲ ಭಾವಿಸುತ್ತ ಬಂದಾಗ ಅದು ಅತ್ಯಂತ ಶಕ್ತಿಶಾಲಿಯಾಗಿ ವ್ಯಕ್ತವಾಗುತ್ತದೆ. ಇಂಥ ಶಕ್ತಿಶಾಲಿಯಾದ ಭಾವನೆಯಿಂದ ನಾವು ಏನನ್ನು ಬೇಕಾದರೂ ಸಾಧಿಸಲು ಸಮರ್ಥರಾಗುತ್ತೇವೆ. ಉದಾಹರಣೆಗೆ ಮಹಾಪುರುಷರನ್ನು ತೆಗೆದುಕೊಳ್ಳಿ. ನಮ್ಮಂತೆಯೇ ಮಾನವರಾದ ಅವರು ಮಹಾಪುರುಷರಾದದ್ದು ಹೇಗೆ? ಪ್ರಬಲ ಭಾವನಾಶಕ್ತಿಯಿಂದ, ಪ್ರಬಲ ಸಂಕಲ್ಪಶಕ್ತಿಯಿಂದ! ಅವರು ಉನ್ನತ, ಉದಾತ್ತ ಯೋಜನೆಗಳನ್ನು ಹಾಕಿಕೊಂಡು ಸದಾ ಭಾವಿಸುತ್ತ ಬಂದರು: “ನಾನು ಇಂತಿಂಥದನ್ನು ಮಾಡುತ್ತೇನೆ, ಮಾಡಿಯೇತೀರುತ್ತೇನೆ” ಎಂದು. ಕೊನೆ ಗೊಂದು ದಿನ ಅದನ್ನು ಮಾಡಿಯೇತೀರಿದರು, ಮಹಾ ಪುರುಷರೆನಿಸಿದರು.

ಆದ್ದರಿಂದ ವಿದ್ಯಾರ್ಥಿಯು ಭಾವಿಸುತ್ತಿರಲಿ:

\begin{myquote}
“ನಾನು ಹೆಚ್ಚೆಚ್ಚು ಓದುತ್ತೇನೆ, ಹೆಚ್ಚೆಚ್ಚು ಅಧ್ಯಯನ ಮಾಡು ತ್ತೇನೆ. ಅಧ್ಯಯನ ಮಾಡಿದ್ದನ್ನು ಅರ್ಥಮಾಡಿಕೊಳ್ಳುತ್ತೇನೆ, ಮತ್ತು ಚೆನ್ನಾಗಿ ನೆನಪಿಟ್ಟುಕೊಳ್ಳುತ್ತೇನೆ. ಹೀಗೆ ಮಾಡುತ್ತ ಸಕಾಲದಲ್ಲಿ ನಾನು ಶ್ರೇಷ್ಠ ವಿದ್ಯಾವಂತನಾಗಿಯೇ ತೀರುತ್ತೇನೆ.”
\end{myquote}

ಈ ಭಾವನೆಯ ಶಕ್ತಿ ಅದೆಷ್ಟೆಂದರೆ ಅದರ ಪ್ರಭಾವದಿಂದ ವಿದ್ಯಾರ್ಥಿಯು ಕಾರ್ಯೋನ್ಮುಖನಾಗುತ್ತಾನೆ... ಕಾರ್ಯತತ್ಪರನಾಗುತ್ತಾನೆ... ಕೃತಕೃತ್ಯ ನಾಗುತ್ತಾನೆ–ಎಂದರೆ ಒಂದು ದಿನ ಶ್ರೇಷ್ಠ ವಿದ್ಯಾವಂತನೆಂದು ಸನ್ಮಾನಿತ ನಾಗುತ್ತಾನೆ.

\delimiter

ಪ್ರಾರ್ಥನೆಯ ಮಹತ್ವವನ್ನು ಸಮರ್ಥಿಸಲು ಬೇರೆ ಮಾತು ಬೇಕಿಲ್ಲ. ಏಕೆಂದರೆ ಸ್ವತಃ ಪ್ರಾರ್ಥನೆ ಮಾಡುವವನಿಗೇ ಕೃತಾರ್ಥತೆಯ ಅನುಭವ ವಾಗುವುದಲ್ಲ!

\delimiter

