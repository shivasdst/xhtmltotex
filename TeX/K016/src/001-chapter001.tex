
\chapter{ಅಧ್ಯಯನದಲ್ಲಿ ಏಕಾಗ್ರತೆ}

ಯಶಸ್ಸಿನ ರಹಸ್ಯವಿರುವುದು ಏಕಾಗ್ರತೆಯಲ್ಲಿ ಎಂಬ ಸತ್ಯವನ್ನು ಅರಿತ ವರೇ ಜಾಣರು. ಈ ಏಕಾಗ್ರತೆ ಎಂಬುದು ಯೋಗಿಗಳಿಗೆ ಮಾತ್ರ ಅವಶ್ಯವಿರುವ ಸೊತ್ತು ಎಂದು ತಿಳಿಯುವುದು ಶುದ್ಧ ತಪ್ಪು. ಜೀವನದ ಪ್ರತಿಯೊಂದು ಕ್ಷೇತ್ರದಲ್ಲಿ ಕಾರ್ಯನಿರ್ವಹಿಸುವ \textbf{ಪ್ರತಿಯೋರ್ವರಿಗೂ ಈ ಏಕಾಗ್ರತೆ ಅತ್ಯ ಗತ್ಯ.} ಕಮ್ಮಾರರಲ್ಲಿ, ಕ್ಷೌರಿಕರಲ್ಲಿ, ಬಡಗಿಗಳಲ್ಲಿ, ಅಕ್ಕಸಾಲಿಗರಲ್ಲಿ, ನೇಯ್ಗೆ ಯವರಲ್ಲಿ, ಈ ಏಕಾಗ್ರತೆಯೆಂಬುದು ಸಹಜವಾಗಿಯೇ ರೂಢಮೂಲವಾಗಿರು ವುದನ್ನು ಕಾಣಬಹುದು. ಕಮ್ಮಾರನ ಸುತ್ತಿಗೆ ಗುರಿತಪ್ಪಿದರೆ ಕೈಮುರಿದು ಕೊಳ್ಳುವ ಸಂಭವ; ಕ್ಷೌರಿಕನ ಕತ್ತಿ ಸ್ವಲ್ಪ ಜಾರಿದರೂ ಚರ್ಮ ಕತ್ತರಿಸಿ ರಕ್ತ ಸುರಿಯುವ ಸಂಭವ; ಬಡಗಿಗೆ ತನ್ನ ಉಳಿಯ ಮೇಲೆ ಹಿಡಿತವಿಲ್ಲದಿದ್ದರೆ ಅವನ ಕಾಲ್ಬೆರಳಿಗೆ ಉಳಿಗಾಲವಿಲ್ಲ; ಅಕ್ಕಸಾಲಿಗಳ ಚಿನ್ನದ ಕೆಲಸವಂತೂ ಅತ್ಯಂತ ನಾಜೂಕಿನದು; ನೇಯ್ಗೆಯವರು ತಮ್ಮ ಲಾಳಿಗಳ ಮೇಲೆ ಸತತ ಕಣ್ಣಿಟ್ಟಿದ್ದರೆ ಮಾತ್ರ ಉತ್ತಮ ವಸ್ತ್ರಗಳು ಹೊರಬರಲು ಸಾಧ್ಯ.

ಆದರೆ ಇವರೆಲ್ಲರೂ ಪುಸ್ತಕಗಳನ್ನೋದಿಯೋ, ಭಾಷಣ ಕೇಳಿಯೋ ಏಕಾಗ್ರತೆಯನ್ನು ಅಭ್ಯಾಸ ಮಾಡಿದವರಲ್ಲ. ಪರಿಸ್ಥಿತಿಯ ಒತ್ತಡದಿಂದ ಇವರಿಗೆ ಏಕಾಗ್ರತೆ ಲಭಿಸಿದೆ. ಹಾಗಾದರೆ ಯಾವುದು ಆ ಪರಿಸ್ಥಿತಿ? ತಮ್ಮ ಕೈಕೆಲಸದಲ್ಲಿ ಸ್ವಲ್ಪ ಹೆಚ್ಚುಕಡಿಮೆಯಾದರೂ ಅಪಘಾತ ಸಂಭವಿಸಬಹುದೆಂಬ ಪರಿಸ್ಥಿತಿ. ಇವರೆಲ್ಲ ತಮ್ಮ ಕೆಲಸ-ಕಾರ್ಯಗಳ ಅವಧಿಯಲ್ಲಿ ಅಪಾಯಗಳ ಅಂಚಿನಲ್ಲೇ ಇರು ವವರು. ಆದ್ದರಿಂದ ಇವರು ತಮ್ಮ ಮನಸ್ಸನ್ನು ಬಿಗಿಹಿಡಿದುಕೊಂಡು ಅತ್ಯಂತ ಎಚ್ಚರಿಕೆಯಿಂದಲೆ ಕೆಲಸ ಮಾಡುತ್ತಿರಬೇಕು. ಈ ಮೂಲಕ ಇವರಿಗೆ ಒಂದು ಬಗೆಯ ಏಕಾಗ್ರತೆ–ಇವರ ವೃತ್ತಿಗೆ ಸಂಬಧಪಟ್ಟ ಏಕಾಗ್ರತೆ–ಸಿದ್ಧಿಸಿರುತ್ತದೆ. ಅಷ್ಟೇ ಅಲ್ಲ, ಈ ಬಗೆಯ ಏಕಾಗ್ರತೆ ವಂಶಪಾರಂಪರ್ಯವಾಗಿ ಹರಿದುಬಂದ ಉದಾಹರಣೆಗಳೂ ಹೇರಳವಾಗಿವೆ. ಅದು ಹೇಗೆಂದರೆ, ಕಬ್ಬಿಣದ ಕೆಲಸಗಳಿಗೆ ಸಂಬಂಧಿಸಿದ ವೃತ್ತಿಶಿಕ್ಷಣವನ್ನು ಪಡೆಯುವ ವಿದ್ಯಾರ್ಥಿಗಳಲ್ಲಿ ಕಮ್ಮಾರನ ಹುಡುಗ ಎಲ್ಲರಿಗಿಂತಲೂ ಹೆಚ್ಚು ಪ್ರಾವೀಣ್ಯ ತೋರಿಸುವುದು ಕಂಡುಬರು ತ್ತದೆ. ಹಾಗೆಯೇ ಈ ಮಾತು ಇತರ ಎಲ್ಲ ವಿಷಯಗಳಿಗೂ ಅನ್ವಯಿಸುತ್ತದೆ. ಆದರೆ ಅಲ್ಲಲ್ಲಿ ಒಬ್ಬೊಬ್ಬರು ಹೊಸ ವಿಷಯದಲ್ಲೂ ನೈಪುಣ್ಯ ತೋರ ಬಹುದು. ಆದರದು ತುಂಬಾ ಅಪರೂಪ.

ಇಲ್ಲಿಯವರೆಗೆ ನೋಡಿದ ಕೆಲವು ಅಂಶಗಳ ಆಧಾರದ ಮೇಲೆ ನಾವು ಇಷ್ಟನ್ನು ಸ್ಪಷ್ಟವಾಗಿ ತಿಳಿಯಬಹುದು, ಏನೆಂದರೆ ಸತತ ಅಭ್ಯಾಸದಿಂದ ಏಕಾ ಗ್ರತೆಯ ಸಿದ್ಧಿ. ಅರ್ಜುನ ಕೇಳಿದ ಪ್ರಶ್ನೆಗೆ ಶ್ರೀಕೃಷ್ಣ ನೀಡಿದ ಉತ್ತರವೂ ಇದೇ; ಅಭ್ಯಾಸದಿಂದ ಸಿದ್ಧಿ. ಅರ್ಜುನನ ಪ್ರಶ್ನೆಯೇನು? “ಕೃಷ್ಣ, ಈ ಮನಸ್ಸು ಬಹಳ ಚಂಚಲವಾದದ್ದು; ಸದಾ ಗಲಭೆಯೆಬ್ಬಿಸುತ್ತಲೇ ಇರುತ್ತದೆ. ಇದರ ಬಲವೂ ಕಡಿಮೆಯದೇನಲ್ಲ. ಇದನ್ನು ತಡೆದಿಡುವುದೂ ಒಂದೇ, ಗಾಳಿಯನ್ನು ಹಿಡಿದು ಕಟ್ಟಿಡುವುದೂ ಒಂದೇ. ಇಂತಹ ಮನಸ್ಸನ್ನು ಸ್ವಾಧೀನಕ್ಕೆ ತಂದುಕೊಳ್ಳುವುದು ಹೇಗೆ?” ಇದಕ್ಕೆ ಶ್ರೀಕೃಷ್ಣನ ಉತ್ತರ: “ನೀನೆನ್ನುವುದು ನಿಜ ಅರ್ಜುನ, ಈ ಮನಸ್ಸು ಅತ್ಯಂತ ಚಂಚಲವೆಂಬುದು ಸತ್ಯ, ಮತ್ತು ಅದನ್ನು ಹತೋಟಿಯ ಲ್ಲಿಟ್ಟಿರುವುದು ಕಷ್ಟವೆಂಬುದೂ ಸತ್ಯ. ಆದರೆ ಒಂದು ಮುಖ್ಯಾಂಶವನ್ನು ನಿನಗೆ ಹೇಳುತ್ತೇನೆ ಕೇಳು–\textbf{ಇಂತಹ ಚಂಚಲ ಮನಸ್ಸನ್ನೂ ಕೂಡ ಅಭ್ಯಾಸ ಬಲ ದಿಂದ ಹಾಗೂ ವೈರಾಗ್ಯದಿಂದ ನಿಗ್ರಹಿಸಿ ಹತೋಟಿಯಲ್ಲಿಡಲು ಸಾಧ್ಯ ವೆಂಬುದೂ ಸತ್ಯ.”}

ಅರ್ಜುನನ ಪ್ರಶ್ನೆ ಎಷ್ಟು ಸಹಜವಾಗಿದೆಯೋ ಅಷ್ಟೇ ಸರಳವಾಗಿದೆ ಶ್ರೀ ಕೃಷ್ಣನ ಉತ್ತರ. ಅಂತೂ ಈ ಚಂಚಲ ಮನಸ್ಸಿನ ಸಮಾಚಾರ ಇಂದು ನಿನ್ನೆಯದಲ್ಲ. ಹಿಂದಿನಿಂದಲೂ ಇದ್ದದ್ದೇ. ಆದರೆ ಈಗಿನ ಅನೀತಿ-ಅಶಿಸ್ತಿನ ಜೀವನದ ಪರಿಣಾಮವಾಗಿ ಅದು ಸ್ವಲ್ಪ ಹೆಚ್ಚಾಗಿರಬಹುದು. ಅಂತಹ ಧೀರ ಧರ್ಮಿಷ್ಟನಾದ ಅರ್ಜುನನ ಮನಸ್ಸೇ ಚಂಚಲವಾಗಿತ್ತೆನ್ನುವಾಗ ಇಂದಿನ ಷೋಕಿಲಾಲರ ಮನಸ್ಸು ಸ್ಥಿರವಾಗಿರುವುದುಂಟೆ?

ಮನಸ್ಸನ್ನು ತಮ್ಮ ಸ್ವಾಧೀನಕ್ಕೆ ತಂದುಕೊಳ್ಳಲೇಬೇಕೆಂಬವರು ಮೊಟ್ಟ ಮೊದಲನೆಯದಾಗಿ, ತಾವು ಎಂತಹ ಮನಸ್ಸಿನೊಂದಿಗೆ ಹೋರಾಡಬೇಕಾಗಿದೆ ಎಂಬುದರ ಸ್ವಷ್ಟ ಕಲ್ಪನೆಯನ್ನು ಹೊಂದಿರಬೇಕಾಗುತ್ತದೆ. ಈ ಮನಸ್ಸು ಮರ್ಕಟನಂತೆ ಚಂಚಲ, ಮದಭರಿತ ಸಲಗನಂತೆ ಬಲಿಷ್ಠ. ಅರ್ಜುನನೆನ್ನು ವಂತೆ, ಇದನ್ನು ನಿಯಂತ್ರಿಸುವುದೆಂದರೆ ಗಾಳಿಯನ್ನು ತಡೆಹಿಡಿದು ಕಟ್ಟಿಡುವ ಕೆಲಸ. ಮನಸ್ಸನ್ನು ಸಂಯಮಕ್ಕೆ ತಂದುಕೊಳ್ಳುವಲ್ಲಿ ಕೋತಿಯನ್ನು ಹಿಡಿ ಯುವ, ಸಲಗವನ್ನು ಪಳಗಿಸುವ ಚಾಣಾಕ್ಷತೆಯೇ ಬೇಕಾಗುತ್ತದೆ.

ಮನಸ್ಸನ್ನು ನಿಗ್ರಹಿಸುವ ಕೆಲಸ ಅತ್ಯಂತ ಕಷ್ಟ ಎಂದು ಅರ್ಜುನ ಹೇಳಿದಾಗ ಶ್ರೀಕೃಷ್ಣ, “ಏನಯ್ಯಾ, ಎಂತೆಂಥ ವೀರರನ್ನೇ ನಿಗ್ರಹಿಸಿದ ನಿನಗೆ ಈ ಮನ ಸ್ಸನ್ನು ನಿಗ್ರಹಿಸುವುದು ಯಾವ ಲೆಕ್ಕ? ಅಲ್ಲದೆ ಅದು ಎಷ್ಟಾದರೂ ನಿನ್ನ ಮನಸ್ಸೇ ಅಲ್ಲವೇನಯ್ಯ? ತುಂಬ ಸುಲಭವಾಗಿ ನಿಗ್ರಹಿಸಿಬಿಡಬಹುದು” ಎಂದು ಈ ರೀತಿಯಲ್ಲಿ ಹಗುರವಾಗಿ ಹೇಳಲಿಲ್ಲ. ಪರಿಸ್ಥಿತಿಯ ಗಾಂಭೀರ್ಯ ವನ್ನರಿತು ಹೇಳುತ್ತಾನೆ, “ನೀನೆನ್ನುವುದು ನಿಜ ಅರ್ಜುನ, ಈ ಮನಸ್ಸು ಅತ್ಯಂತ ಚಂಚಲವೆಂಬುದು ಸತ್ಯ ಮತ್ತು ಅದನ್ನು ಹತೋಟಿಯಲ್ಲಿಟ್ಟಿರುವುದು ಕಷ್ಟ ವೆಂಬುದೂ ಸತ್ಯ” ಎಂದು. ಶ್ರೀಕೃಷ್ಣ ಹೀಗೆ ಹೇಳಲು ಕಾರಣ–ಅವನು ಈ ಮನಸ್ಸಿನ ಸ್ವಭಾವವನ್ನು ಬಲ್ಲ.

ಈ ಜಗತ್ತಿನಲ್ಲಿ ಪ್ರತಿಯೊಂದು ಪ್ರಾಣಿಗೂ, ವಸ್ತುವಿಗೂ ಅದರದೇ ಆದ ಸ್ವಭಾವವಿರುತ್ತದೆ–ಗಾಳಿಗೆ ಬೀಸುವ ಸ್ವಭಾವ, ಬೆಂಕಿಗೆ ಸುಡುವ ಸ್ವಭಾವ, ನೀರಿಗೆ ಹರಿಯುವ ಸ್ವಭಾವ; ಹಾಗೆಯೇ ಮನಸ್ಸಿಗೆ ಕಂಡಕಂಡಲ್ಲಿ ಮೂಗು ಹಾಕುವ ಸ್ವಭಾವ, ಹುಚ್ಚೆದ್ದು ಕುಣಿಯುವ ಸ್ವಭಾವ, ಬಗೆಬಗೆಯ ಆಸೆಗಳನ್ನು ತಾಳುವ ಸ್ವಭಾವ, ಹಲವಿಧದ ಯೋಚನೆ ಮಾಡುವ ಸ್ವಭಾವ, ಹಲವು ಹದಿನೆಂಟು ಚಿಂತೆಗಳನ್ನಿಟ್ಟುಕೊಂಡು ಕೊರಗುವ ಸ್ವಭಾವ, ಆಶಾಗೋಪುರ ಕಟ್ಟುವ ಸ್ವಭಾವ, ಮಾಡಬೇಕಾದ ಕೆಲಸವೊಂದನ್ನು ಬಿಟ್ಟು ಉಳಿದೆಲ್ಲ ವಿಚಾರಗಳಲ್ಲೂ ತಲೆ ಹಾಕುವ ಸ್ವಭಾವ! ಸ್ವಭಾವತಃ ಚಂಚಲ ವಾಗಿರುವ ಇಂತಹ ಮನಸ್ಸನ್ನು ಇನ್ನಷ್ಟು ಚಂಚಲಗೊಳಿಸುವ ಪರಿಸರಗಳು ಸುತ್ತ-ಮುತ್ತ. ಇನ್ನು ಕೇಳುವುದೇನಿದೆ? ಕುಣಿದಾಡುವುದೊಂದೇ! ಆದ್ದರಿಂದ ಮನಸ್ಸನ್ನು ಸ್ವಾಧೀನದಲ್ಲಿಟ್ಟುಕೊಳ್ಳಬೇಕೆಂಬವರು ಚಂಚಲಗೊಳಿಸುವ ಪರಿಸರದಿಂದ ದೂರವಾಗಬೇಕು, ಪಾರಾಗಬೇಕು; ಎಂದರೆ ಊರು ಬಿಟ್ಟು ಹೋಗಬೇಕೆಂದಲ್ಲ, ಮನಸ್ಸನ್ನು ಅತ್ತ ಹರಿಯಗೊಡದಿರಬೇಕು. ಹೇಗೆ? ಇಲ್ಲಿ, ಇಂದ್ರಿಯಗಳ ಪಾತ್ರ ಬರುತ್ತದೆ. ಕಣ್ಣು, ಕಿವಿ, ನಾಲಗೆ, ಮೂಗು, ಚರ್ಮ –ಇವು ಮನಸ್ಸಿನ ವಾಹನಗಳು. ಒಂದು ಸುಂದರವಾದ ವಸ್ತುವನ್ನು ಕಣ್ಣು ಕಂಡರೆ ಸಾಕು, ಮನಸ್ಸು ಹಾರಿಹೋಗಿ ಅಲ್ಲಿ ಕುಳಿತಾಯಿತು. ಹೀಗೆಯೇ ಐದು ಇಂದ್ರಿಯಗಳೂ ಮನಸ್ಸನ್ನು ಎಲ್ಲೆಂದರಲ್ಲಿಗೆ ಕರೆದೊಯ್ಯುತ್ತಿರುತ್ತವೆ. ಆದ್ದ ರಿಂದ ಬುದ್ಧಿಯುಪಯೋಗಿಸಿ ಈ ಇಂದ್ರಿಯಗಳನ್ನು ನಮ್ಮ ವಶದಲ್ಲಿಟ್ಟಿರ ಬೇಕು. ಎಂದರೆ ನೋಡಬಾರದ್ದನ್ನು ನೋಡದಿರಬೇಕು, ಕೇಳಬಾರದ್ದನ್ನು ಕೇಳದಿರಬೇಕು, ತಿನ್ನಬಾರದ್ದನ್ನು ತಿನ್ನದಿರಬೇಕು, ಮಾಡಬಾರದ್ದನ್ನು ಮಾಡದಿರ ಬೇಕು. ಹೀಗೆ ಮಾಡುವುದಕ್ಕೆ ‘ದಮ’ ಎಂದು ಹೆಸರು. ಕೆಲವೊಮ್ಮೆ ಇಂದ್ರಿಯ ಗಳ ಸಹಾಯವಿಲ್ಲದೆಯೇ, ಈ ಮನಸ್ಸು ಸ್ವತಂತ್ರವಾಗಿ ತನಗೆ ಬೇಕೆಂಬಲ್ಲಿಗೆ ಹಾರಿಹೋಗಬಹುದು. ಆಗ ಬುದ್ಧಿಯ ಸಹಾಯದಿಂದ ಅದನ್ನು ಹಿಂದಿರುಗಿಸಿ ಎಳೆದು ತರಬೇಕು. ಹೀಗೆ ಮನಸ್ಸನ್ನು ನೇರವಾಗಿ ಸಂಯಮದಲ್ಲಿಡುವುದಕ್ಕೆ ‘ಶಮ’ ಎಂದು ಹೆಸರು.

ಮನಸ್ಸನ್ನು ಹತೋಟಿಯಲ್ಲಿಡಬೇಕಾದ ಆವಶ್ಯಕತೆಯಾದರೂ ಏನಿದೆ? –ಎಂಬ ಪ್ರಶ್ನೆಯೊಂದು ಏಳುತ್ತದೆ. ಅದಕ್ಕೊಂದು ಸರಿಯಾದ ಉತ್ತರವನ್ನು ಕಂಡುಕೊಳ್ಳಬೇಕು. ಉತ್ತರವಿಷ್ಟೆ: ನಮ್ಮ ಮನಸ್ಸು ನಮ್ಮ ಹತೋಟಿಯಲ್ಲಿ ದ್ದರೆ ಅದರ ಮೂಲಕ ಮಹತ್ಕಾರ್ಯಗಳನ್ನೇ ಸಾಧಿಸಿಬಿಡಬಹುದು. ಅದು ನಮ್ಮ ಹತೋಟಿಯಲ್ಲಿರದಿದ್ದರೆ ಮಾತ್ರ ಅದರಿಂದ ಸಾಮಾನ್ಯ ಕಾರ್ಯವನ್ನೂ ಮಾಡಿಸಲು ಸಾಧ್ಯವಾಗುವುದಿಲ್ಲ.

ನಿಜಕ್ಕೂ ನಮ್ಮ ಮನಸ್ಸಿಗೆ ಪ್ರಚಂಡ ದೈತ್ಯ ಶಕ್ತಿಯೇ ಇದೆ. ಆದರೂ ಹಲವರು ಎಷ್ಟೋ ವೇಳೆ, ಅಥವಾ ಜೀವನವಿಡೀ, ದುರ್ಬಲರಂತೆ ಇರುತ್ತಾರೆ. ಇದಕ್ಕೆ ಅವರ ಮನಶ್ಶಕ್ತಿಯು ನಾನಾರೀತಿಗಳಿಂದ ಹಂಚಿಹೋಗಿರುವುದೇ ಕಾರಣವಾಗಿದೆ. ಸೂರ್ಯಕಿರಣಗಳಿಗೆ ಸುಡುವ ಶಕ್ತಿ ಯಿದೆಯೆಂಬ ಸತ್ಯ ಎಲ್ಲರಿಗೂ ತಿಳಿದಿಲ್ಲ. ಕಾರಣ, ಅವು ಇಲ್ಲಿಯವರೆಗೂ ಎಲ್ಲಿಯೂ ಬೆಂಕಿ ಹೊತ್ತಿಸಿ ವಸ್ತುಗಳನ್ನು ಸುಡಲಿಲ್ಲವಲ್ಲ! ಆದರೆ ಅದೇ ಕಿರಗಣಗಳನ್ನು ಭೂತಕನ್ನಡಿಯ ಮೂಲಕ ಹಾಯಿಸಿ ಕಾಗದದ ಮೇಲೆ ಬಿಟ್ಟಾಗ ಬೆಂಕಿ ಹೊತ್ತಿಕೊಳ್ಳುವುದು ಕಾಣುತ್ತದೆ. ಆ ಕಿರಣಗಳಿಗೆ ಈಗ ಎಲ್ಲಿಂದ ಬಂತು ಈ ಸುಡುವ ಶಕ್ತಿ? ಅವುಗಳನ್ನು ಏಕತ್ರಗೊಳಿಸಿದ್ದರಿಂದ. ಹಿಂದೆ ಅವು ಹರಡಿಕೊಂಡಿದ್ದುವು, ಆದ್ದರಿಂದ ಶಾಖ ಮಾತ್ರ ಇತ್ತು. ಈಗ ಏಕತ್ರ ಗೊಂಡಾಗ ಧಗಧಗಿಸುವ ಅಗ್ನಿಯೇ ಉದ್ಭವಿಸಿತು. ಇದೇ ನಾವು ಗಮನಿಸ ಬೇಕಾದ ರಹಸ್ಯ. ನಮ್ಮ ಮನದಲ್ಲಿ ಸಹಜವಾಗಿಯೇ ಅಪಾರ ಶಕ್ತಿಯಿದೆ. ಆದರೆ ಅದು ಬೇಕಾದ-ಬೇಡವಾದ ಎಲ್ಲ ವಿಷಯಗಳಲ್ಲೂ ಹರಡಿಕೊಂಡಿರು ವುದರಿಂದ ನಮ್ಮಿಂದ ಸಾಮಾನ್ಯ ಕೆಲಸಕಾರ್ಯಗಳನ್ನು ಮಾಡಲು ಮಾತ್ರವೇ ಸಾಧ್ಯವಾಗುತ್ತದೆ. ಮಹತ್ಕಾರ್ಯವನ್ನೇನಾದರೂ ಸಾಧಿಸಬೇಕಾದರೆ \textbf{ಹರಡಿಕೊಂಡಿರುವ ಮನಶ್ಶಕ್ತಿಯನ್ನು ಒಗ್ಗೂಡಿಸಬೇಕಾಗುತ್ತದೆ.} ಮನಶ್ಶಕ್ತಿಯನ್ನು ಒಗ್ಗೂಡಿಸಬೇಕಾದರೆ ನಮ್ಮ ಮನಸ್ಸು ನಮ್ಮ ಅಧೀನದಲ್ಲಿರಬೇಕಾಗುತ್ತದೆ. ಕಂಡಕಂಡಕಡೆಗೆಲ್ಲ ಮಂಡೆಗೆಟ್ಟು ಓಡುವ ಮನಸ್ಸು ನಮ್ಮದಲ್ಲ. ಇಂದ್ರಿಯ ಗಳ ಆಹ್ವಾನಕ್ಕೆ ಓಗೊಟ್ಟು ವಿಷಯವಸ್ತುಗಳಲ್ಲಿ ಮುಳುಗಿರುವ ಮನಸ್ಸು ನಮ್ಮದಲ್ಲ. ನಮ್ಮದಲ್ಲದ ಮನಸ್ಸಿನಿಂದ ನಾವು ಏನು ತಾನೆ ಮಾಡಿಸಲು ಸಾಧ್ಯ?

ಪುಷಿಮುನಿಗಳು ಸತತ ಪ್ರಯತ್ನದಿಂದ ಮೊದಲು ಸಾಧಿಸಿದ್ದೇ ಮನ ಸ್ಸಂಯಮವನ್ನು. ಎಂದರೆ ತಮ್ಮ ಮನಸ್ಸನ್ನು ತಮ್ಮ ಹಿಡಿತಕ್ಕೆ ತಂದು ಕೊಂಡದ್ದು. ಬಳಿಕ ಆ ಮನಸ್ಸನ್ನು ಹಿಡಿದು ಏಕಾಗ್ರಗೊಳಿಸಿದಾಗ ಅದು ಸಕಲ ಯೋಗರಹಸ್ಯಗಳನ್ನೇ ಬಯಲಿಗೆಳೆಯಿತು. ದಿವ್ಯ ಜ್ಞಾನವನ್ನೇ ಪ್ರಾಪ್ತಿಮಾಡಿ ಕೊಟ್ಟಿತು.

ಸ್ವಾಮಿ ವಿವೇಕಾನಂದರು ಹೇಳುವಂತೆ, ಏಕಾಗ್ರಗೊಂಡ ಮನಸ್ಸು ಒಂದು ಸರ್ಚ್​ಲೈಟೇ ಸರಿ. ಸರ್ಚ್​ಲೈಟು ದೂರದ ಮೂಲೆಯಲ್ಲಿರುವ ವಸ್ತುವನ್ನೂ ದೃಗ್ಗೋಚರವಾಗುವಂತೆ ಮಾಡುತ್ತದೆ.

ಈಗ ಮನಸ್ಸನ್ನು ಏಕಾಗ್ರಗೊಳಿಸಬೇಕೆಂಬುದೇನೋ ಸರಿಯೆ, ಆದರೆ ಯಾವುದರ ಮೇಲೆ ಏಕಾಗ್ರಗೊಳಿಸಬೇಕು? ಈ ಪ್ರಶ್ನೆಗೆ, ಸರ್ವರೂ ಏಕಕಾಲ ದಲ್ಲಿ ಒಪ್ಪುವಂತಹ ಉತ್ತರ ಕೊಡಲು ಬರುವಂತಿಲ್ಲ. ಏಕೆಂದರೆ, ಮನಸ್ಸನ್ನು ಆತ್ಮಜ್ಯೋತಿಯ ಮೇಲೆ ಏಕಾಗ್ರಗೊಳಿಸಬೇಕು ಎಂದು ಹೇಳಿಬಿಟ್ಟರೆ, ಸರ್ವರೂ ಯೋಗಿಗಳಾಗಲು ಹೊರಟಿಲ್ಲವಲ್ಲ! ಮನಸ್ಸನ್ನು ಭಗವಂತನ ಮೇಲೆ ಏಕಾಗ್ರಗೊಳಿಸಬೇಕು ಎಂದುಬಿಟ್ಟರೆ ಸರ್ವರೂ ಭಕ್ತರಲ್ಲವಲ್ಲ! ಮನ ಸ್ಸನ್ನು ಪಾಠದ ಮೇಲೆ ಏಕಾಗ್ರಗೊಳಿಸಬೇಕು ಎಂದು ಬಿಟ್ಟರೆ, ಸರ್ವರೂ ಶಾಲಾವಿದ್ಯಾರ್ಥಿಗಳೇನಲ್ಲವಲ್ಲ! ಆದ್ದರಿಂದ ಅವರವರು \textbf{ತಮತಮಗೆ ಅಗತ್ಯ ವಾದ ವಿಷಯದ ಮೇಲೆ ಮನಸ್ಸನ್ನು ಏಕಾಗ್ರಗೊಳಿಸಬೇಕು ಎನ್ನುವುದೇ ಸರಿ.}

ಇಲ್ಲಿ ನಾವು, ವಿದ್ಯಾರ್ಥಿಗಳು ತಮ್ಮ ಪಠ್ಯವಿಷಯದ ಮೇಲೆ ಮನಸ್ಸನ್ನು ಏಕಾಗ್ರಗೊಳಿಸಬೇಕಾದ ಬಗೆಯನ್ನು ನೋಡೋಣ. ಏಕೆಂದರೆ, ಇದು ಮುಖ್ಯ ವಾಗಿ ವಿದ್ಯಾರ್ಥಿಗಳಿಗಾಗಿ ಬರೆದ ಪುಸ್ತಕ.

(೧) ಯೋಗಿಗೆ ಧ್ಯಾನಕ್ಕೆ ಕುಳಿತುಕೊಳ್ಳಲು ಒಂದು ಮೆತ್ತನೆಯ ಆಸನವಿರ ಬೇಕಾದಂತೆ, \textbf{ವಿದ್ಯಾರ್ಥಿಗಳಿಗೆ ತಮ್ಮ ಪುಸ್ತಕವನ್ನಿಟ್ಟುಕೊಂಡು ಆರಾಮವಾಗಿ ಓದಲು ಅನುಕೂಲಿಸುವಂತೆ ಮೇಜು-ಕುರ್ಚಿ ಇರಬೇಕು.} ವಿದ್ಯಾರ್ಥಿಯೂ ಹೆಚ್ಚು ಕಡಿಮೆ ಒಬ್ಬ ಯೋಗಿಯಂತೆಯೇ. ಸಹಸ್ರಾರು ಬಡವಿದ್ಯಾರ್ಥಿಗಳು ಈ ಮೇಜು ಕುರ್ಚಿಗೆಲ್ಲಿಗೆ ಹೋಗಬೇಕು–ಎಂಬ ಪ್ರಶ್ನೆಯನ್ನು ಇಲ್ಲಿ ಕೇಳ ಬಾರದು. “ಸರ್ ಎಂ. ವಿಶ್ವೇಶ್ವರಯ್ಯನವರು ವಿದ್ಯಾರ್ಥಿಯಾಗಿದ್ದಾಗ ಅವರಿ ಗೆಲ್ಲಿತ್ತು ಮೇಜು-ಕುರ್ಚಿ? ಬೀದಿಯ ಬೆಳಕಿನಲ್ಲಿ ಓದಿಯೇ ಅವರು ವಿಶ್ವ ವಿಖ್ಯಾತರಾಗಲಿಲ್ಲವೆ?” ಎಂಬ ವಾದವನ್ನು ಇಲ್ಲಿ ತರಬಾರದು; ಏಕೆಂದರೆ ಬೀದಿಯ ಬೆಳಕಿನಲ್ಲಿ ಓದಿದವರೆಲ್ಲರೂ ವಿಶ್ವೇಶ್ವರಯ್ಯ ಆಗಲಾರರು. ಈ ವಿಚಾರ ಹಾಗಿರಲಿ, ಮೇಜು-ಕುರ್ಚಿಯ ಅನುಕೂಲತೆಯಿಲ್ಲದವರು ಒಂದು ಸರಿಯಾದ ಡೆಸ್ಕನ್ನಾದರೂ ಇಟ್ಟುಕೊಳ್ಳಬೇಕು.

(೨) ಓದಲು ಅಥವಾ ಬರೆಯಲು ಕುಳಿತಾಗ \textbf{ಶರೀರ}. ತುಂಬ ಅಲುಗಾಡದೆ \textbf{ಸ್ಥಿರವಾಗಿರುವಂತೆ ನೋಡಿಕೊಳ್ಳಬೇಕು.} ಹೆಚ್ಚಿನ ವಿದ್ಯಾರ್ಥಿಗಳು ಬಗೆಬಗೆಯ ಚಿತ್ರವಿಚಿತ್ರ ಭಾವಭಂಗಿಗಗಳಲ್ಲಿ ಕುಳಿತು ಓದುವುದು ಕಂಡುಬರುತ್ತದೆ. ಏನೋ ಮಹಾ ಆಲೋಚಿಸುವವರಂತೆ ಕಣ್ಣುಗಳನ್ನು ಅತ್ತಿತ್ತ ಚಲಿಸುತ್ತ, ಪೆನ್ನನ್ನೋ, ಪೆನ್ಸಿಲನ್ನೋ ಬಾಯಿಗಿಟ್ಟುಕೊಂಡು ಕಚ್ಚುತ್ತ ಓದುವವರೂ ಉಂಟು, ಇನ್ನೂ ಏನೇನೋ ವಿನ್ಯಾಸಗಳೆಲ್ಲ ಇವೆ. ಇವೆಲ್ಲವೂ ಏಕಾಗ್ರತೆಗೆ ಮಾರಕ. ಪಾತ್ರೆ ಅಲುಗಾಡುತ್ತಿದ್ದರೆ ಅದರಲ್ಲಿರುವ ನೀರೂ ಅಲುಗಾಡುವಂತೆ, ಶರೀರ ವಿವಿಧ ಚಲನವಲನಗಳನ್ನು ಮಾಡುತ್ತಿದ್ದರೆ ಮನಸ್ಸೂ ಚಂಚಲವಾಗು ತ್ತದೆ. ಆದ್ದರಿಂದ ಗಂಭೀರವಾಗಿ ಸರಿಯಾದ ಕ್ರಮದಲ್ಲಿ ಕುಳಿತು ಓದಲು ಅಥವಾ ಬರೆಯಲು ತೊಡಗಬೇಕಾದದ್ದು ಪ್ರಮುಖ ಅಂಶ.

(೩)\textbf{ ಒಂದು ಸಲಕ್ಕೆ ಒಂದು ವಿಷಯವನ್ನು ಮಾತ್ರವೇ ಅಧ್ಯಯನಕ್ಕೆ ತೆಗೆದುಕೊಳ್ಳಬೇಕು} ಎಂಬುದನ್ನು ಪ್ರತ್ಯೇಕವಾಗಿ ಹೇಳಬೇಕಾಗಿಲ್ಲ. ಆದರೆ ಒಂದು ವಿಷಯವನ್ನು ಅಧ್ಯಯನ ಮಾಡಲು ಕುಳಿತರೆ ಕನಿಷ್ಠ ಪಕ್ಷ ಒಂದು ಗಂಟೆಯ ಕಾಲವಾದರೂ ಮನಸ್ಸು ಅದರಲ್ಲೇ ತನ್ಮಯವಾಗಿರುವಂತೆ ನೋಡಿ ಕೊಳ್ಳಬೇಕು. ಪುಸ್ತಕವನ್ನು ಸುಮ್ಮನೆ ಓದಿಕೊಂಡುಹೋದರೆ ಅಧ್ಯಯನ ಮಾಡಿದಂತಾಗುವುದಿಲ್ಲ. ಪುಸ್ತಕವನ್ನು ಸುಮ್ಮನೆ ಓದಿಕೊಂಡು ಹೋಗುವು ದಕ್ಕೂ, ಅಧ್ಯಯನ ಮಾಡುವುದಕ್ಕೂ ಇರುವ ವ್ಯತ್ಯಾಸವನ್ನು ತಿಳಿದಿರಬೇಕು. ಆದರೆ ಏಕಾಗ್ರತೆ ಮಾತ್ರ ಇವೆರಡಕ್ಕೂ ಸಮಾನವಾಗಿ ಅಗತ್ಯವಿರುವ ಅಂಶ ವೆನ್ನಿ. ಪುಸ್ತಕವನ್ನು ಸುಮ್ಮನೆ ಓದಿಕೊಂಡು ಹೋದಾಗ ಅದರ ಅಭಿಪ್ರಾಯ ವೇನೆಂದು ತಿಳಿಯುತ್ತದೆ, ಅಷ್ಟೇ. ಆದರೆ ಅದನ್ನೇ ಅಧ್ಯಯನ ಮಾಡಿದಾಗ ನಮ್ಮ ಮನಸ್ಸು ಅದರಲ್ಲಿ ವಿವರಿಸಲಾದ ವಿಷಯದ ಆಳಕ್ಕೆ ಹೋಗಿ ಅಂತ ರಾರ್ಥವನ್ನು ಅರಿಯುತ್ತದೆ. ಇದರಿಂದಾಗಿ ಅಂದು ಮಾಡಿದ ಅಧ್ಯಯನದ ವಿಷಯ ನಮ್ಮ ಸ್ವಾಧೀನಕ್ಕೆ ಬರುತ್ತದೆ ಮತ್ತು ಇದು ಮುಂದಿನ ವಿಷಯದ ಅಧ್ಯಯನಕ್ಕೆ ನೆರವಾಗುತ್ತದೆ. 

(೪)\textbf{ ಒಮ್ಮೆ ಅಧ್ಯಯನಕ್ಕೆ ಕುಳಿತೆವೆಂದರೆ ಒಂದು ಗಂಟೆಯ ಕಾಲ ಅದ ರಲ್ಲೇ ತೊಡಗಿರಬೇಕು} ಎಂಬುದನ್ನು ಇಲ್ಲಿ ವಿಶೇಷವಾಗಿ ಗಮನಿಸಬೇಕು. ಮನಸ್ಸು ಒಂದು ವಿಷಯವನ್ನು ಏಕಾಏಕಿ ಗ್ರಹಿಸಲು ಸಿದ್ಧವಿರುವುದಿಲ್ಲ. ಅಧ್ಯಯನಕ್ಕೆ ಕುಳಿತುಕೊಳ್ಳುವ ಮೊದಲು ನಾವು ಯಾವ ಕಾರ್ಯ ಮಾಡು ತ್ತಿದ್ದೆವೋ ಅಥವಾ ಯಾವ ಮಾತನ್ನಾಡುತ್ತಿದ್ದೆವೋ ಅಥವಾ ಇನ್ನಾವ ವಿಷಯ ವನ್ನು ಚಿಂತಿಸುತ್ತಿದ್ದೆವೋ ಆ ವಿಷಯವೇ ನಮ್ಮ ಮನಸ್ಸಿನಲ್ಲಿ ಇನ್ನೂ ಸುಳಿ ದಾಡುತ್ತಿರುತ್ತದೆ. ಆದ್ದರಿಂದ ಈಗ ಅಧ್ಯಯನಕ್ಕೆ ಕುಳಿತಾಗ ಮನಸ್ಸು ಅಣಿ ಯಾಗುವುದಕ್ಕೆ ಎಂಟು-ಹತ್ತು ನಿಮಿಷಗಳಾದರೂ ಬೇಕಾಗಬಹುದು, ಮತ್ತು ಮನಸ್ಸು ವಿಷಯದ ಆಳಕ್ಕೆ ಪ್ರವೇಶಿಸಿ ಅರಿಯಲು ತೊಡಗುವ ವೇಳೆಗೆ ಸರಕ್ಕನೆ ಅಲ್ಲಿಂದ ಎದ್ದರೆ ಏಕಾಗ್ರತೆ ನಷ್ಟವಾಗುತ್ತದೆ, ಅಧ್ಯಯನ ಭ್ರಷ್ಟವಾಗುತ್ತದೆ. ಆದ್ದರಿಂದ ಮನಸ್ಸು ಅಧ್ಯಯನದಲ್ಲಿ ಕೇಂದ್ರೀಕೃತವಾದಾಗ ಆ ಅವಕಾಶವನ್ನು ಉಪಯೋಗಿಸಿಕೊಂಡು ಇನ್ನೂ ಆಳಕ್ಕೆ ಮುಳುಗಬೇಕು. ಹೀಗೆ ಒಂದು ಗಂಟೆಯ ಕಾಲವಾದರೂ ಮನಸ್ಸು ಅಲ್ಲೇ ನಿಲ್ಲುವಂತೆ ಮಾಡಬೇಕು.

(೫) ಈ ಸಂದರ್ಭದಲ್ಲಿ ಯಾರಾದರೂ ಮನೆಮಂದಿ ಬಂದು ಯಾವುದೋ ಕೆಲಸಕ್ಕೆ ಕರೆಯಬಹುದು. ಆದ್ದರಿಂದ ಅವರಿಗೆ ಮೊದಲೇ ಹೇಳಿಟ್ಟಿರಬೇಕು– \textbf{‘ಒಂದು ಗಂಟೆಯ ಕಾಲ ಯಾರೂ ನನ್ನನ್ನು ಕರೆಯಬೇಡಿ’} ಎಂದು. ಏಕೆಂದರೆ, ಯಾರಾದರೂ ಬಂದು ಕರೆದಾರು ಎಂಬ ನಿರೀಕ್ಷೆ ಮನದಲ್ಲಿ ಕಿಂಚಿತ್ತು ಇದ್ದರೂ ಕೂಡ ಏಕಾಗ್ರತೆಯಿಂದ ಅಧ್ಯಯನ ಮಾಡಲು ಸಾಧ್ಯವಾಗುವುದಿಲ್ಲ.

(೬) ಇನ್ನು ಶಬ್ದಮಾಲಿನ್ಯದ ಸಮಸ್ಯೆಯೊಂದಿದೆ. ಹಳ್ಳಿಗಳೇ ಮೊದಲಾದ ಪ್ರಶಾಂತ ವಾತಾವರಣದಲ್ಲಿರುವ ವಿದ್ಯಾರ್ಥಿಗಳು ಈ ವಿಷಯದಲ್ಲಿ ಭಾಗ್ಯ ವಂತರು. ಆದರೆ ಪಟ್ಟಣಗಳಲ್ಲಿ, ಅದರಲ್ಲೂ ರಾಜಧಾನಿಗಳಲ್ಲಿ, ವಾಸವಾ ಗಿರುವ ವಿದ್ಯಾರ್ಥಿಗಳು ಅನಾವಶ್ಯಕ ಗದ್ದಲಗಳೊಂದಿಗೆ ಗುದ್ದಾಡುತ್ತಿರಬೇಕಾ ಗುತ್ತದೆ. ಈ ಗದ್ದಲಗಳೆಲ್ಲ ನಗರಗಳಲ್ಲಿ \textbf{ಅನಿವಾರ್ಯವಾದ್ದರಿಂದ ಅದಕ್ಕೇ ಹೊಂದಿಕೊಳ್ಳದೆ ಬೇರೆ ದಾರಿಯಿಲ್ಲ.} ಆದರೆ ಎಷ್ಟು ಹೊಂದಿಕೊಳ್ಳಬೇಕೆಂ ದರೂ ಧ್ವನಿವರ್ಧಕಗಳು ಅರಚಾಡಲು ಆರಂಭಿಸಿದುವೆಂದರೆ, ಅಧ್ಯಯನವೆಲ್ಲ ಅಧ್ವಾನವಾಗಿಬಿಡುತ್ತದೆ. ವಿನಾಯಕ ಚೌತಿ ಬಂತೆಂದರೆ ಒಂದು ತಿಂಗಳು ಕರ್ಣಕರ್ಕಶ, ಕನ್ನಡ ರಾಜ್ಯೋತ್ಸವ ಬಂದಾಗ ಇನ್ನೊಂದು ತಿಂಗಳು ಗುಲ್ಲೋ ಗುಲ್ಲು. ಶ್ರೀರಾಮನವಮಿಯ ತಿಂಗಳಲ್ಲಂತೂ ವಿದ್ಯಾರ್ಥಿಗಳ ಪಾಡು ಅವ ರಿಗೇ ಗೊತ್ತು. ಇವೆಲ್ಲ ನಗರದ ಕೆಲವು ನರಕಸದೃಶ ಪರಿಸ್ಥಿತಿಗಳು. ಸಮಾಜದ ವಿದ್ಯಾವಂತರು, ಶಿಕ್ಷಣತಜ್ಞರು ಈ ವಿಷಯದಲ್ಲಿ ಅಸಹಾಯಕರಾಗಿ ಕೈಚೆಲ್ಲಿ ಕುಳಿತಿರುವುದು ಕಂಡುಬರುತ್ತದೆ. ಶಿಕ್ಷಣತಜ್ಞರಿಗೆ ಅಧಿಕಾರಬಲವಿಲ್ಲವೆಂದು ತೋರುತ್ತದೆ. ಅಧಿಕಾರಬಲವಿರುವವರಿಗೆ ವಿವೇಕಪ್ರಜ್ಞೆಯಿರುವಂತೆ ಕಾಣು ತ್ತಿಲ್ಲ. ಇರಲಿ, ವಿದ್ಯಾರ್ಥಿಗಳು ಧ್ವನಿಮಾಲಿನ್ಯದಿಂದ ಪಾರಾಗಲು ಒಂದೇ ಒಂದು ದಾರಿಯೆಂದರೆ ತಾವು ಅತ್ಯಂತ ಉನ್ನತ ಮಟ್ಟದ ವಿದ್ಯಾವಂತರಾಗ ಬೇಕೆನ್ನುವ \textbf{ಹಂಬಲವನ್ನು ತೀವ್ರಗೊಳಿಸುವುದು.} ಮನದಲ್ಲಿ ಹಂಬಲವಿದ್ದರೆ ಹೊರಗಣ ಯಾವ ಗದ್ದಲವೂ ಕೇಳಿಸುವುದೇ ಇಲ್ಲ. ಉದಾಹರಣೆಗೆ ತಲೆ ಯಲ್ಲಿ ಚಿಂತೆಯೊಂದು ತುಂಬಿಕೊಂಡಿದ್ದರೆ ಪಕ್ಕದಲ್ಲಿ ತಮ್ಮಟೆ ಬಾರಿಸು ತ್ತಿದ್ದರೂ ಅದರ ಕಡೆಗೆ ನಮ್ಮ ಗಮನವಿರುವುದಿಲ್ಲ. ಪರೀಕ್ಷೆ ಸಮೀಪಿಸಿದಾಗ ಈ ಹಂಬಲ ಸಾಕಷ್ಟು ತೀವ್ರವಾಗುವುದುಂಟು. ಆದರೆ ಪರೀಕ್ಷೆಗೆ ಇನ್ನೂ ಸಾಕಷ್ಟು ಕಾಲಾವಕಾಶ ಇರುವಾಗಲೇ ಆ ತೀವ್ರಹಂಬಲವಿದ್ದರೆ ಏಕಾಗ್ರತೆ ಯನ್ನು ಚೆನ್ನಾಗಿ ಬೆಳೆಸಿಕೊಳ್ಳಬಹುದು. ದಿಢೀರನೆ ಏಕಾಗ್ರತೆಯನ್ನು ಬರಿಸಿ ಕೊಳ್ಳುವುದು ಎಂಥವರಿಂದಲೂ ಸಾಧ್ಯವಿಲ್ಲ ಎಂಬ ಸಂಗತಿಯನ್ನು ಗಮನ ದಲ್ಲಿಟ್ಟುಕೊಳ್ಳುವುದು ಒಳ್ಳೆಯದು.

(೭) ಅಧ್ಯಯನದಲ್ಲಿ ಏಕಾಗ್ರತೆಯನ್ನು ತಂದುಕೊಳ್ಳಲು ಇನ್ನೊಂದು ಉಪಾಯವನ್ನು ಸೇರಿಸಿಕೊಳ್ಳಬಹುದು–\textbf{ವಿಷಯದ ಕಡೆಗೆ ಚೆನ್ನಾಗಿ ಗಮನ ವಿಟ್ಟು ಓದುವುದೇ ಆ ಉಪಾಯ.} ಹೀಗೆ ಗಮನವಿರುವುದಕ್ಕೆ ‘ಅವಧಾನ’ ಎನ್ನುತ್ತಾರೆ. ಎಂದರೆ ಇದೇ ಮನಸ್ಸನ್ನು ಸಾಕಷ್ಟು ಎಚ್ಚರದ ಸ್ಥಿತಿಯಲ್ಲಿಟ್ಟು ಕೊಂಡರೆ ಮಾತ್ರ ಈ ‘ಅವಧಾನ’ ಸಿದ್ಧಿಸುತ್ತದೆ. ಹೆಚ್ಚಿನ ವಿದ್ಯಾರ್ಥಿಗಳು ಗಮನವಿಟ್ಟು ಅಧ್ಯಯನ ಮಾಡಲು ಅಸಮರ್ಥರಾಗಿರುತ್ತಾರೆ. ಏಕೆಂದರೆ, ಅವರ ಮನಸ್ಸು ಒಂದು ಬಗೆಯ ಸ್ವಪ್ನಾವಸ್ಥೆಯಲ್ಲಿಯೇ ಇರುತ್ತದೆ. ಅಥವಾ ಯಾವುದೋ ಒಂದು ಭಾವಲೋಕದಲ್ಲಿ ವಿಹರಿಸುತ್ತಿರುತ್ತದೆ. ಆದ್ದರಿಂದ ಯಾರ ಮನಸ್ಸು ಎಚ್ಚರದಿಂದಿರುತ್ತದೆಯೋ ಅವರಿಗೆ ಏಕಾಗ್ರತೆ ಸುಲಭವಾಗಿ ಸಿದ್ಧಿಸುತ್ತದೆ. ಹೀಗೆಂದಾಗ, ‘ಹಾಗಾದರೆ ಈ ಮನಸ್ಸನ್ನು ಎಚ್ಚರದ ಸ್ಥಿತಿ ಯಲ್ಲಿಡುವುದು ಹೇಗೆ?’ ಎಂಬ ಪ್ರಶ್ನೆ ಹುಟ್ಟಿಕೊಳ್ಳಲು ಸಾಧ್ಯವಿದೆ. ಇದಕ್ಕೆ ಕೆಲವು ಸಲಹೆಗಳನ್ನು ಕೊಡಬಹುದು.

(ಅ) ಕೆಲವು ಬಗೆಯ ಆಹಾರಪದಾರ್ಥಗಳನ್ನು ಸೇವಿಸಿದರೆ ನಿದ್ರೆ ಅಧಿಕ ವಾಗುವುದುಂಟು. ಅಂಥವುಗಳನ್ನು ವರ್ಜಿಸಬೇಕು.

(ಆ) ಶರೀರದ ಅಂಗಪ್ರತ್ಯಂಗವನ್ನೂ ಅತ್ಯಂತ ಶುಚಿಯಾಗಿಟ್ಟುಕೊಳ್ಳು ವುದರಿಂದ ಮನಸ್ಸು ಉಲ್ಲಾಸಭರಿತವಾಗಿರುತ್ತದೆ. ಜೊತೆಗೆ ಹಾಸಿಗೆ ಬಟ್ಟೆ ಯಿಂದ ಹಿಡಿದು ಧರಿಸುವ ಬಟ್ಟೆಬರೆಯವರೆಗೆ ಪ್ರತಿಯೊಂದೂ ಚೊಕ್ಕಟ ವಾಗಿರುವಂತೆ ಎಚ್ಚರಿಕೆ ವಹಿಸಬೇಕು.

(ಇ) ವಾಸದ ಕೋಣೆಯಲ್ಲಿರುವ ಹಾಗೂ ದೈನಂದಿನ ಬಳಕೆಯ \textbf{ಎಲ್ಲವಸ್ತುಗಳೂ ಅಚ್ಚುಕಟ್ಟಾಗಿರಬೇಕು.} ಈ ಅಚ್ಚುಕಟ್ಟುತನವು ಮನಸ್ಸಿನ ಎಚ್ಚರಸ್ಥಿತಿ ಯನ್ನು ಸಾರಿಹೇಳುತ್ತದೆ. ಹೆಚ್ಚಿನ ವಿದ್ಯಾರ್ಥಿಗಳಲ್ಲಿ ಕಂಡುಬರುವುದು ಎಚ್ಚರ ಗೇಡಿತನವೇ. ಅವರು ಉಪಯೋಗಿಸುವ ಪುಸ್ತಾಕಾದಿ ಸಕಲ ವಸ್ತುಗಳೂ ಅಸ್ತವ್ಯಸ್ತವಾಗಿರುವುದು ಆ ಎಚ್ಚರಗೇಡಿತನವನ್ನು ಎತ್ತಿತೋರಿಸುತ್ತದೆ.

(ಈ) ಶರೀರ-ವಸ್ತ್ರ-ವಸ್ತುಗಳನ್ನು ಚೊಕ್ಕಟವಾಗಿಯೂ ವ್ಯವಸ್ಥಿತವಾಗಿಯೂ ಇಟ್ಟುಕೊಳ್ಳಬೇಕಾದಂತೆಯೇ \textbf{ಮನಸ್ಸನ್ನೂ ಚೊಕ್ಕಟವಾಗಿಟ್ಟಿರಬೇಕು} ಎಂಬ ಮಹತ್ವದ ಸಂಗತಿಯೊಂದು ವಿದ್ಯಾರ್ಥಿಗಳಿಗೆ ತಿಳಿದಿರಬೇಕು. ಅಸಭ್ಯ, ಅಶ್ಲೀಲ ವಿಚಾರಗಳು ಮನಸ್ಸಿನೊಳಗೆ \textbf{ಪ್ರವೇಶ}ವಾಗಲೂ ಬಿಡಬಾರದೆಂಬ ವಿಷಯ ವಂತಿರಲಿ, ಹತ್ತಿರವೇ ಸುಳಿಯದಂತೆ ನೋಡಿಕೊಳ್ಳಬೇಕು. ಏಕೆಂದರೆ ಮನ ಸ್ಸನ್ನು ಕೆಡಿಸುವ ಮಹಾಮಾರಿಗಳು ಅವು. ಮನಸ್ಸು ನಿದ್ರಾವಸ್ಥೆಯಲ್ಲಿದ್ದರೂ ನಷ್ಟವಿಲ್ಲ, ಸ್ವಪ್ನಾವಸ್ಥೆಯಲ್ಲಿದ್ದರೂ ಬಾಧಕವಿಲ್ಲ; ಆದರೆ ಈ ಮನಸ್ಸು ಅಸಭ್ಯತೆ, ಅಶ್ಲೀಲತೆಗಳನ್ನು ಮೈಗೂಡಿಸಿಕೊಂಡಿತೆಂದರೆ ಏಕಾಗ್ರತೆಯ ಮಾತ ನ್ನೆಲ್ಲ ಕಟ್ಟಿಡಬೇಕಾಗುತ್ತದೆ. ಶುಚಿಯಾದ, ಪವಿತ್ರವಾದ ಮನಸ್ಸಿಗೆ ಏಕಾಗ್ರತೆ ಯೆಂಬುದು ಸುಲಭವಾಗಿ ಸಿದ್ಧಿಸುತ್ತದೆಯಾದ್ದರಿಂದ ಅಸಭ್ಯತೆ ಅಶ್ಲೀಲತೆ ಗಳನ್ನು ಬಳಿಯೇ ಸುಳಿಯಗೊಡದಿರಬೇಕು.

(ಉ) ‘ಒಂದು ಗಂಟೆಯ ಅವಧಿಯಲ್ಲಿ ಇಂತಿಷ್ಟು ವಿಷಯಗಳನ್ನು ಅಧ್ಯ ಯನ ಮಾಡಿ ಮುಗಿಸುತ್ತೇನೆ’ ಎಂದು ವಿದ್ಯಾರ್ಥಿಗಳು \textbf{ತಮಗೆ ತಾವೇ ನಿಬಂಧನೆ ಹಾಕಿಕೊಳ್ಳಬೇಕು} ಮತ್ತು ಆ ಅವಧಿಯೊಳಗೆ ಅದನ್ನು ಮುಗಿಸಲು ತವಕ ಪಡಬೇಕು. ಮೊದಮೊದಲು ಅವಧಿಯೊಳಗೆ ಮುಗಿಸಲು ಸಾಧ್ಯವಾಗದೆ ಹೋಗಬಹುದು. ಆದರೆ ತವಕದಿಂದ ಕೂಡಿದ ಈ ಪ್ರಯತ್ನದಲ್ಲಿ ಮನಸ್ಸು ಹೆಚ್ಚು ಎಚ್ಚರವಾಗಿರಲು ಸಾಧ್ಯವಾಗುವುದು ಖಂಡಿತ.

ಹೀಗೆ \textbf{ಮನಸ್ಸನ್ನು ಸದಾ ಎಚ್ಚರವಾಗಿಡಲು ಯತ್ನಿಸಿ}ದರೆ, ಅದು ಏಕಾಗ್ರತೆ ಯನ್ನು ಬೆಳೆಸಿಕೊಳ್ಳಲು ಬಹಳ ಮಟ್ಟಿಗೆ ಸಹಾಯವಾಗುತ್ತದೆ.

(೮) ಏಕಾಗ್ರತೆಯನ್ನು ಗಳಿಸಿಕೊಳ್ಳಬೇಕೆಂಬ ವಿದ್ಯಾರ್ಥಿಗಳು \textbf{ಹರಟೆ ಹೊಡೆ ಯುವುದನ್ನು ವಿಷದಂತೆ ವರ್ಜಿಸಬೇಕು.} ಆದರೆ ಅವರು ತಮ್ಮ ಪಠ್ಯವಿಷಯದ ಕುರಿತು ಸ್ನೇಹಿತರೊಡನೆ ಚರ್ಚಿಸುವುದರಿಂದ ಬಾಧಕವಿಲ್ಲ. ಬದಲಾಗಿ ಅದು ಸಾಧಕವೇ ಆಗುತ್ತದೆ. ಆದರೆ ನಿಷ್ಟ್ರಯೋಜಕ ಹರಟೆಯಿಂದ ಮನಸ್ಸಿನ ಶಕ್ತಿ ಬಹಳ ಮಟ್ಟಿಗೆ ಕುಂಠಿತಗೊಳ್ಳುತ್ತದೆ ಎಂಬುದು ಯೋಗಿಗಳು ಕಂಡುಕೊಂಡ ಸತ್ಯ. ಹರಟೆಯಿಂದ ಮನಸ್ಸಿನ ಶಿಸ್ತು-ಸುಸಂಬದ್ಧತೆ ಕೆಡುತ್ತದೆ. ಮನಸ್ಸು ದುರ್ಬಲಗೊಂಡಾಗ ಹಾಗೂ ಶಿಸ್ತನ್ನು ಕಳೆದುಕೊಂಡಾಗ ಏಕಾಗ್ರವಾಗಲು ಅಸಮರ್ಥವಾಗುತ್ತದೆ. ದುರ್ಬಲ ಶರೀರ-ಮನಸ್ಸಿನವರಿಗೆ ಏಕಾಗ್ರತೆ ಬಹುದೂರ.

(೯) ಏಕಾಗ್ರತೆಯಿಂದ ಓದಲು ಯತ್ನಿಸುವ ಕೆಲವು ವಿದ್ಯಾರ್ಥಿಗಳಿಗೆ ತಲೆ ನೋವು ಬರುವುದುಂಟು. ಇದಕ್ಕೆ ಪ್ರಧಾನ ಕಾರಣ ಅವರ ಮಿದುಳಿಗೆ ಶಕ್ತಿ ಯಿಲ್ಲದಿರುವುದು. ಮಿದುಳು ಬಲಗೊಳ್ಳುವಂತೆ \textbf{ಪೌಷ್ಟಿಕ ಆಹಾರದೊಂದಿಗೆ ಟಾನಿಕ್ಕು ಅಥವಾ ಚ್ಯವನಪ್ರಾಶಾದಿಗಳನ್ನು ಸೇವಿಸಬೇಕು.} ಹಿತವಾದ ಪ್ರಮಾಣ ದಲ್ಲಿ ಹಾಲು, ಬೆಣ್ಣೆ, ತುಪ್ಪ, ಪ್ರತಿದಿನ ಸೇವಿಸಬೇಕು. ಆದರೆ ಯಾವುದನ್ನೇ ಆಗಲಿ ಅತಿಯಾಗಿ ಸೇವಿಸಿದರೆ ಗತಿಗೆಡುವುದು ನಿಶ್ಚಿತ. ವೀರ್ಯಶಕ್ತಿಗಿಂತ ಶ್ರೇಷ್ಠವಾದ ಟಾನಿಕ್ಕು ಬೇರಾವುದೂ ಅಲ್ಲ. ಈ ವೀರ್ಯಶಕ್ತಿಯನ್ನು ಕಾಪಾಡಿ ಕೊಳ್ಳಲು ಅವರವರ ಅಂತರಂಗದಲ್ಲೇ ಇರುವ ಅಂತರ್ಯಾಮಿಯನ್ನು ಅಂತಃ ಕರಣಪೂರ್ವಕವಾಗಿ ಪ್ರಾರ್ಥಿಸಿಕೊಳ್ಳಬೇಕು. ಶರೀರ-ಮನಸ್ಸುಗಳನ್ನು ಸದಾ ಶುಚಿಯಾಗಿಟ್ಟಿರಬೇಕು. ಹುಲಿಕರಡಿಗಳಿಂದ ದೂರವಿರುವಂತೆ ವಿಲಾಸೀ ವಿದ್ಯಾರ್ಥಿಗಳ ಹಾಗೂ ಫಟಿಂಗ ಹುಡುಗರ ಸಹವಾಸಕ್ಕೆ ಸಿಕ್ಕಿಬೀಳದಂತೆ ಎಚ್ಚರಿಕೆಯಿಂದಿರಬೇಕು. ಆದರೆ ಅವರನ್ನು ನಿಂದಿಸಲೂ ಹೋಗಬಾರದು. ಹಾಗೆ ಮಾಡಿದರೆ ಅದು ಇನ್ನೊಂದು ಅನಾಹುತಕ್ಕೆ ಕಾರಣವಾಗುತ್ತದೆ.

(೧ಂ) ಏಕಾಗ್ರತೆಯನ್ನು ಸಿದ್ಧಿಸಿಕೊಳ್ಳಲು ಇನ್ನೊಂದು ಪ್ರಮುಖ ಸಾಧನ ವೆಂದರೆ \textbf{ಶ್ರದ್ಧೆ.} ಈ ಶ್ರದ್ಧೆ, ಹೊರಗಿನಿಂದ ಪಡೆಯುವ ವಸ್ತುವಲ್ಲ, ಒಳಗಿ ನಿಂದಲೇ ಬೆಳೆಯಬೇಕಾದ ಗುಣ. ಶ್ರದ್ಧೆಯೆಂಬ ಶಬ್ದವನ್ನೇನೋ ಎಲ್ಲರೂ ಒಂದಲ್ಲ ಒಂದು ಸಂದರ್ಭದಲ್ಲಿ ಕೇಳಿಯೇ ಇದ್ದಾರೆ. ಆದರೆ ಅದರ ಮರ್ಮ ವನ್ನರಿತವರು ಅತ್ಯಲ್ಪ ಮಂದಿ. ಅಂಥವರು ತಾವು ಹಿಡಿದ ಮಾರ್ಗದಲ್ಲಿ ಮುಂದುವರಿದು ತಮ್ಮ ಧ್ಯೇಯವನ್ನು ಸಿದ್ಧಿಸಿಕೊಂಡಿದ್ದಾರೆ. ಶ್ರದ್ಧೆಯ ಬಲವೇ ಅಂಥದು. ಅದು ಸಾಧಕನನ್ನು, ಎಂದರೆ ಪ್ರಯತ್ನಶೀಲನನ್ನು, ಗುರಿ ಮುಟ್ಟಿಸದೆ ಬಿಡುವುದೇ ಇಲ್ಲ. ಹೀಗೆ ಹೇಳಿದಾಗ, ಹಾಗಾದರೆ ಈ ಶ್ರದ್ಧೆ ಎಂದರೆ ಎಂತಹದು ಎಂಬ ಪ್ರಶ್ನೆ ಉದಿಸಿದರೆ ಆಶ್ಚರ್ಯವಿಲ್ಲ. ಶ್ರದ್ಧೆಯೆಂದರೆ ತಮ್ಮ ಶಕ್ತಿಯಲ್ಲಿ ತಮಗೆ ಇರುವ ನಂಬಿಕೆ. ಇದಕ್ಕೇ ಆತ್ಮವಿಶ್ವಾಸ ಎನ್ನುವುದು. ಒಬ್ಬನಿಗೆ ಸಾಕಷ್ಟು ತೋಳ್ಬಲವಿರಬಹುದು. ಆದರೆ ತನ್ನ ತೋಳ್ಬಲದಲ್ಲಿ ಅವನಿಗೇ ನಂಬಿಕೆಯಿಲ್ಲದೆ ಹೋದರೆ ಅವನಿಗೆ ಅದರಿಂದೇನೂ ಪ್ರಯೋಜನ ವಾಗದು. ಹನುಮಂತನಲ್ಲಿ ಸಾಗರಲಂಘನ ಮಾಡುವ ಶಕ್ತಿಯಿದ್ದರೂ ಅವ ನಿಗೇ ಅದರಲ್ಲಿ ನಂಬಿಕೆಯಿರಲಿಲ್ಲ. ಆದ್ದರಿಂದ ಸುಮ್ಮನೆ ಕುಳಿತಿದ್ದ. ಆದರೆ ಜಾಂಬವಂತ ಅವನಲ್ಲಿ ನಂಬಿಕೆ ತುಂಬಿ ಶಕ್ತಿಯ ಅರಿವು ಮಾಡಿಕೊಟ್ಟಾಗ ಒಂದೇ ನೆಗೆತಕ್ಕೆ ಲಂಕೆಯನ್ನು ತಲುಪಿದ.

\textbf{ಪ್ರತಿಯೊಬ್ಬ ವಿದ್ಯಾರ್ಥಿಯಲ್ಲೂ ಅಪೂರ್ವ ಜ್ಞಾನವನ್ನು ಪಡೆಯಬಲ್ಲ ಅಪಾರ ಶಕ್ತಿ ಅಡಗಿದೆ.} ಇದರಲ್ಲಿ ಸಂಶಯವೇ ಇಲ್ಲ. ಆದರೆ ವಿದ್ಯಾರ್ಥಿಯು ತನ್ನ ವಿಷಯದಲ್ಲಿ ತಾನೇ ಸಂಶಯ ತಾಳಬಾರದು, ಅಷ್ಟೆ. \textbf{ಶ್ರದ್ಧೆಯ ವೈರಿಯೇ ಸಂಶಯ.} ಸಂಶಯ ತಾಳಿದ ಕೂಡಲೇ ಅವನಲ್ಲಿ ಅಡಗಿರುವ ಶಕ್ತಿಯೆಲ್ಲ ಉಡುಗಿ ಹೋಗುತ್ತದೆ. ಆಮೇಲೆ ಪಶ್ಚಾತ್ತಾಪ ಪಡುತ್ತ ಕುಳಿತಿರುವುದೊಂದೇ ಅವನ ಹಣೆಬರಹವಾದೀತು. ಅಂತರಂಗದಲ್ಲಿರುವ ಜ್ಞಾನವನ್ನು ವ್ಯಕ್ತಗೊಳಿ ಸಲು ಬಹಿರಂಗದ ಅಧ್ಯಯನವೇ ದಾರಿ. ‘ನನ್ನೊಳಗಿರುವ ಜ್ಞಾನವನ್ನು ವೃದ್ಧಿ ಗೊಳಿಸಿಕೊಂಡೇ ತೀರುತ್ತೇನೆ’ ಎಂಬ ದೃಢಸಂಕಲ್ಪಮಾಡಿ ಶ್ರದ್ಧೆಯಿಂದ ಅಧ್ಯ ಯನಕ್ಕೆ ಕುಳಿತಾಗ ಮನಸ್ಸು ತನ್ನಷ್ಟಕ್ಕೆ ತಾನೇ ಏಕಾಗ್ರವಾಗುವ ವೈಖರಿಯನ್ನು ಅನುಭವಿಸಿಯೇ ನೋಡಬೇಕು.

ನಿಜಕ್ಕೂ ಈ ಶ್ರದ್ಧೆಯ ಮಹಿಮೆ ಅಪಾರ, ಆದ್ದರಿಂದ ಸದಾ ಹೇಳಿಕೊಳ್ಳು ತ್ತಿರಬೇಕು–‘ಓ ಶ್ರದ್ಧೆ! ನೀನಿದ್ದರೆ ನಾ ಗೆದ್ದೆ, ನೀ ಕೈಬಿಟ್ಟರೆ ನಾ ಬಿದ್ದೆ’.

(೧೧) ಏಕಾಗ್ರತೆಗೆ ಇನ್ನೊಂದು ಕೀಲಿಕೈ ಇದೆ–ಇದು ‘ಮಾಸ್ಟರ್ ಕೀ’. ಅದೇ \textbf{ಪ್ರೀತಿ.} ಎಲ್ಲಿ ನಮ್ಮ ಪ್ರೀತಿಯಿರುವುದೋ ಅಲ್ಲಿ ನಮ್ಮ ಮನಸ್ಸು ನಾಟಿರುತ್ತದೆ ಎಂಬುದು ಅನಿವಾರ್ಯ ನಿಯಮ. \textbf{ಪ್ರೀತಿ ಅಧಿಕವಾಗಿರುವಲ್ಲಿ ಮನಸ್ಸು ಗಾಢವಾಗಿ ಸಂಯೋಗಗೊಂಡಿರುತ್ತದೆ.} ಈ ಅಂಶವನ್ನು ಹೆಚ್ಚು ವಿವರಿಸಬೇಕಾಗಿಲ್ಲ. ಏಕೆಂದರೆ ಅವರವರ ಪ್ರೀತಿಯ ವಸ್ತುವಿನಲ್ಲಿ ಅವರವರ ಮನಸ್ಸು ನಾಟಿಕೊಂಡಿರುವುದನ್ನು ಅವರವರೇ ನೋಡಿಕೊಂಡರಾಯಿತು. ಆದ್ದ ರಿಂದ ವಿದ್ಯಾರ್ಥಿಯು ತನ್ನ ಅಧ್ಯಯನದ ವಿಷಯದಲ್ಲಿ ಪ್ರೀತಿ ತಾಳಬೇಕಾ ದದ್ದು ಅತ್ಯಂತ ಆವಶ್ಯಕ.

ಮೊದಮೊದಲು ಅಧ್ಯಯನದ ವಿಷಯದಲ್ಲಿ ಪ್ರೀತಿಯನ್ನು ಪ್ರಯತ್ನ ಪೂರ್ವಕವಾಗಿ ತಂದುಕೊಳ್ಳಬೇಕು. ಆಮೇಲಾಮೇಲೆ ವಿಷಯ ಅರ್ಥವಾಗ ತೊಡಗುವುದರಿಂದ ತಾನಾಗಿಯೇ ಅದರಲ್ಲಿ ಪ್ರೀತಿ ಉಂಟಾಗುತ್ತದೆ. ಪ್ರೀತಿ ಯುಂಟಾದಾಗ ಏಕಾಗ್ರತೆ ಸಿದ್ಧಿಸಿತೆಂದೇ ಅರ್ಥ.

ಶ್ರದ್ಧೆ ಮತ್ತು ಪ್ರೀತಿಯಿಂದ ಸಿದ್ಧಿಸಿಕೊಳ್ಳುವ ಏಕಾಗ್ರತೆಯೇ ಅತ್ಯಂತ ಸಹಜವಾದುದು. ಅಲ್ಲಿ ಶ್ರಮವಿರುವುದಿಲ್ಲ, ತಿಕ್ಕಾಟ-ತಿಣುಕಾಟಗಳಿರುವುದಿಲ್ಲ.

ಆದರೆ, ಇಲ್ಲಿ ವಿದ್ಯಾರ್ಥಿಗಳಿಗೆ ಒಂದು ವಿಶೇಷ ಸೂಚನೆಯನ್ನು ಕೊಡ ಬೇಕಾಗುತ್ತದೆ. ಏನೆಂದರೆ, ಅವರು ಮೇಲೆ ಹೇಳಿರುವಂತೆ ಹನ್ನೊಂದು ಅಂಶ ಗಳನ್ನು ಗಮನದಲ್ಲಿರಿಸಿಕೊಂಡು ಮುಂದುವರಿದರೂ ಅವರು ಅಧ್ಯಯನ ಮಾಡುವ ವಿಷಯವು ಸರಿಯಾಗಿ ಅರ್ಥವಾಗತೊಡಗುವವರೆಗೆ ಅವರಿಗೆ ಏಕಾ ಗ್ರತೆ ಸಿದ್ಧಿಸಿತು ಎಂದು ಹೇಳುವಂತಿಲ್ಲ. ಆದ್ದರಿಂದ, ‘ನನಗಿನ್ನೂ ಏಕಾಗ್ರತೆ ಏಕೆ ಸಿದ್ಧಿಸಿಲ್ಲ?’ ಎಂದು ಕೊರಗಬಾರದು. ಬದಲಾಗಿ ಪಾಠವನ್ನು ಸ್ಪಷ್ಟವಾಗಿ ಅರ್ಥಮಾಡಿಕೊಳ್ಳಲು \textbf{ತಾಳ್ಮೆಯಿಂದ ಓದುತ್ತ ಹೋಗಬೇಕು.} ಕ್ಲಿಷ್ಟ ಶಬ್ದಗಳ ಅರ್ಥವನ್ನು ಶಬ್ದಕೋಶದ ಸಹಾಯದಿಂದ ತಿಳಿದುಕೊಳ್ಳದೆ ಮುಂದೆ ಹೋಗಲೇಬಾರದು. ಹಾಗೆ ಅರ್ಥಮಾಡಿಕೊಳ್ಳದೆ ಓದಿದರೆ ಕೇವಲ ಸಮಯ ಹಾಳು, ಶಕ್ತಿನಷ್ಟ, ಪ್ರಯೋಜನ ಸೊನ್ನೆ. ಕ್ಲಿಷ್ಟ ಶಬ್ದಗಳನ್ನು ಅರ್ಥಮಾಡಿ ಕೊಳ್ಳ ದಿದ್ದರೆ ವಿಷಯ ಅರ್ಥವಾಗುವುದುಂಟೆ?

ಇಂದಿನ ವಿದ್ಯಾರ್ಥಿಗಳ ವಿದ್ಯಾಮಟ್ಟ, ಮುಖ್ಯವಾಗಿ ಲೇಖನಸಾಮರ್ಥ್ಯ, ಎಷ್ಟರ ಮಟ್ಟಿಗೆ ಕೆಳಗಿಳಿದಿದೆ ಎಂಬುದನ್ನು ತಿಳಿಯಲು ಅವರಿಂದ ಯಾವುದಾದ ರೊಂದು ವಿಷಯದ ಮೇಲೆ ಪ್ರಬಂಧ ಬರೆಯಿಸಿ ನೋಡಬೇಕು. ಒಂದೊಂದು ಪುಟದಲ್ಲೂ ತಪ್ಪು ಪದಪ್ರಯೋಗ ಹಾಗೂ ವಾಕ್ಯದೋಷಗಳು ಹೇರಳವಾಗಿ ರುತ್ತವೆ. ಇನ್ನು ಅಕ್ಷರದೋಷವನ್ನಂತೂ ನೋಡಲೇಬಾರದು! ವಿದ್ಯಾರ್ಥಿಯು ಎಷ್ಟೇ ಬುದ್ಧಿವಂತನಾಗಿದ್ದರೂ ಅಭ್ಯಾಸ ಮಾಡದಿದ್ದರೆ ಏನೂ ಸಿದ್ಧಿಸಲಾರದು.

ಆದ್ದರಿಂದಲೇ ಶ್ರೀಕೃಷ್ಣ ಅರ್ಜುನನಿಗೆ ಹೇಳುತ್ತಾನೆ, \textbf{“ಅಭ್ಯಾಸ ಮತ್ತು ವೈರಾಗ್ಯದಿಂದ ಮನಸ್ಸಿನ ಏಕಾಗ್ರತೆಯನ್ನು ಸಿದ್ಧಿಸಿಕೊಳ್ಳಬಹುದು”} ಎಂದು. ಅಭ್ಯಾಸವೆಂದರೆ ಪುನಃ ಪುನಃ ಮಾಡುವ ಪ್ರಯತ್ನ. ವಿದ್ಯಾರ್ಥಿಯೊಬ್ಬ ಐದು ವರ್ಷದ ಬಾಲಕನಾಗಿದ್ದಾಗ ಕಷ್ಟದಿಂದ ಕಾಗುಣಿತ ತಿದ್ದುತ್ತಿದ್ದವನು ಹಾಗೇ ಬರೆದೂ ಬರೆದೂ ಎಸ್ಸೆಸ್ಸೆಲ್ಸಿಯ ವೇಳೆಗೆ ಸರಾಗವಾಗಿ ಬರೆಯಲು ಸಾಧ್ಯ ವಾಯಿತಲ್ಲವೆ? ಇದೇ ಅಭ್ಯಾಸದ ಮಹತ್ವ. ಜಗತ್ತಿನಲ್ಲಿ ಎಂತೆಂಥ ಮಹಾ ಕಾರ್ಯಗಳೆಲ್ಲ ನಡೆದಿವೆ, ನಡೆಯುತ್ತಿವೆ; ಅವುಗಳನ್ನೆಲ್ಲ ಒಮ್ಮೆ ಭಾವಿಸಿ ನೋಡಬೇಕು; ಅವುಗಳ ಹಿನ್ನೆಲೆಯಲ್ಲಿ ಎಷ್ಟೊಂದು ಜನರ ಅವಿರತ ಅಭ್ಯಾಸ- ಪರಿಶ್ರಮ ಇದೆ! ಅಭ್ಯಾಸದಿಂದ ಸಾಧ್ಯವಾಗದುದೇ ಇಲ್ಲ. ಆದರೆ ಕ್ರಮವರಿತ ಶಿಸ್ತುಬದ್ಧ ಅಭ್ಯಾಸ ಅದಾಗಬೇಕು.

ಇನ್ನು ಶ್ರೀಕೃಷ್ಣ ಹೇಳಿದ ‘ವೈರಾಗ್ಯ’ದ ಅರ್ಥವೇನು? ಸಂನ್ಯಾಸವೆಂದೆ? ಅಲ್ಲ. \textbf{ವೈರಾಗ್ಯವೆಂದರೆ ಒಂದು ಉದ್ದೇಶವಿಟ್ಟುಕೊಂಡು ಅದನ್ನು ಈಡೇರಿಸಿ ಕೊಳ್ಳುವ ಪ್ರಯತ್ನದಲ್ಲಿ ತೊಡಗಿದಾಗ, ಇನ್ನಾವ ಪ್ರಲೋಭನೆ-ಆಕರ್ಷಣೆ ಗಳೂ ಮನಸೆಳೆಯದಂತೆ ನೋಡಿಕೊಳ್ಳುವುದು.} ಎಂದರೆ ಕೈಗೊಂಡ ಕಾರ್ಯ ದಲ್ಲಿ ಮಾತ್ರವೇ ಅನುರಾಗ, ಅದಕ್ಕೆ ವ್ಯತಿರಿಕ್ತವಾದ ವಿಷಯದಲ್ಲಿ ವಿರಾಗ –ಇದೇ ವೈರಾಗ್ಯ. ಸಂನ್ಯಾಸಿಗಳು ಆತ್ಮಸಾಕ್ಷಾತ್ಕಾರವನ್ನು ಜೀವನದ ಗುರಿ ಯಾಗಿಟ್ಟುಕೊಂಡಿರುವುದರಿಂದ ಜಗತ್ತಿನ ಇತರ ಎಲ್ಲ ಆಕರ್ಷಣೆಗಳನ್ನು ಬದಿಗೊತ್ತ ಬೇಕಾಯಿತು. ಸಾಮಾನ್ಯವಾಗಿ, ಅದನ್ನೇ ವೈರಾಗ್ಯವೆಂದು ಕರೆಯು ವುದರಿಂದ ಇಂದು ವೈರಾಗ್ಯವೆಂದರೆ ಸಂನ್ಯಾಸಿಯಾಗುವುದು, ಕಾಡಿಗೆ ಹೋಗು ವುದು ಎಂಬರ್ಥವೇ ಬರುತ್ತಿದೆ. ಆದರೆ ಅದು ಹಾಗಲ್ಲ. ಉದ್ದೇಶವನ್ನು ಸಿದ್ಧಿಸಿಕೊಳ್ಳುವ ಪ್ರಯತ್ನದಲ್ಲಿ ತೊಡಗಿರುವವರೆಲ್ಲರಲ್ಲೂ ಈ ವೈರಾಗ್ಯದ ಅಂಶ ಇದ್ದೇ ಇರುತ್ತದೆ. ವಿದ್ಯಾರ್ಥಿಗಳು ವಿದ್ಯಾಭ್ಯಾಸ ಮಾಡುತ್ತಾ ಜ್ಞಾನಾರ್ ಜನೆಯಲ್ಲಿ ತೊಡಗಿರುವಾಗ ಇತರ ಆಕರ್ಷಣೆಗಳು ಅವರ ಮನಸ್ಸನ್ನು ಸೆಳೆಯುವಂತಿದ್ದರೆ ವಿದ್ಯಾಭ್ಯಾಸ ಹೇಗೆ ಯಶಸ್ವಿಯಾಗಲು ಸಾಧ್ಯ? ಅದರಲ್ಲೂ ಏಕಾಗ್ರತೆಯನ್ನು ಅಭ್ಯಾಸಮಾಡುವವರಂತೂ ಇತರ ವಿಷಯಗಳೆಡೆಗೆ ಕಣ್ಣೆ ತ್ತಿಯೂ ನೋಡುವಂತಿಲ್ಲ.

ಏಕಾಗ್ರತೆಯಲ್ಲಿ ಆನಂದವಿದೆ, \textbf{ಏಕಾಗ್ರತೆಯಿಂದ ಸಿದ್ಧಿಯಿದೆ} ಎಂಬ ಸುದ್ದಿ ತಿಳಿದ ವಿದ್ಯಾರ್ಥಿಯು ಇನ್ನು ಸುಮ್ಮನಿರಬಾರದು. ಕಾರ್ಯತತ್ಪರನಾಗಬೇಕು.

\delimiter

ಮಾಡುವ ಅಧ್ಯಯನದಲ್ಲಿ ಮೈಮರೆತರೆ ಏಕಾಗ್ರತೆ ಸಿದ್ಧಿಸಿದೆ ಎಂದರ್ಥ.

\delimiter

ಶಿಸ್ತುಬದ್ಧ \textbf{ಯೋಗಾಸನಗಳು} ನರಮಂಡಲವನ್ನು ಬಲಗೊಳಿಸುವುದಲ್ಲದೆ ಚುರುಕಾಗಿಸುವುದರಿಂದ ಅವು \textbf{ಏಕಾಗ್ರತೆಗೆ ಬಹಳ ಸಹಾಯಕಾರಿ.}

\delimiter

ಜೋಗದ ಜಲಪಾತದಂತೆ ಯುವಕರ ಮನಸ್ಸು! ಬೃಹತ್ ಪ್ರಮಾಣದ ನೀರು ಎತ್ತರದ ಬೆಟ್ಟದಿಂದ ಧುಮುಧುಮಿಸಿ ಧುಮುಕಿ, ಎಲ್ಲೆಂದರಲ್ಲಿ ಹರಿ ದಾಡಿ, ಕೊನೆಗೂ ಯಾವ ಉಪಯೋಗಕ್ಕೂ ಬಾರದೆ ಸಮುದ್ರಕ್ಕೆ ಸೇರಿ ಹೇಳ ಹೆಸರಿಲ್ಲದಂತಾಗುತ್ತದೆ. ಆದರೆ ಆ ನೀರಿಗೆ ಅಣೆಕಟ್ಟು ಕಟ್ಟಿ, ಕಾಲುವೆಯಲ್ಲಿ ಹರಿಯಿಸಿ, ಹೊಲಗಳಿಗೆ ತಂದುಕೊಂಡಾಗ ಹಸನಾದ ಬೆಳೆ ನಳನಳಿಸುತ್ತದೆ.

ಹಾಗೆಯೇ, ಮನಬಂದಂತೆ ಕುಣಿದಾಡಿ, ವೃಥಾ ವ್ಯಯವಾಗುವ ಯುವಕರ ಅಶಿಕ್ಷಿತ ಮನಶ್ಶಕ್ತಿಗೆ ನಿಯಮಗಳ ಅಣೆಕಟ್ಟು ಕಟ್ಟಿ, ಶಿಸ್ತಿನ ಕಾಲುವೆ ತೋಡಿ, ಅದನ್ನು ಶಿಕ್ಷಣ ಕಲೆ ಸಾಹಿತ್ಯ ಕರ್ಮಕೌಶಲಗಳೆಂಬ ಹೊಲಗಳಲ್ಲಿ ಹರಿಯಿಸಿ ದಾಗ ಸಂಸ್ಕೃತಿಯೆಂಬ ಫಸಲು ಸಮೃದ್ಧವಾಗುತ್ತದೆ.

