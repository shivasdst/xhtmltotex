
\chapter{A Pramāṇa–Parīkṣā of Three–Dimensional Philology}\label{chapter8}

\section*{4.1 The Three Dimensions}

Although the term philology is seen throughout the works of Pollock, a theoretical framework and a methodology for understanding texts is only discussed in recent papers. This is primarily an academic necessity: since its academic birth in 1800 C.E., philology has slowly lost its predominance in the academia and is now a dying field. The effort to theorize is an attempt to save it from eventual death. Pollock’s philological theory is neither profound nor extensive as it ultimately amounts to \textit{interpretation} or superimposing alien models on Indian tradition. In fact, it is not even a theory as it does not follow his own practice, and is essentially \textit{kalpita} (consciously constructed to save philology). This “theory” of philology cannot be compared with even a basic text like Annambhaṭṭa’s \textit{Tarka–saṅgraha}, let alone the intellectual traditions of \textit{Vyākaraṇa}, \textit{Alaṅkāra} and \textit{Mīmāṁsā}. An analysis of the three–dimensional Pollockian philology shows that it is what it has always been: a branch of literary humanities and not a science. The definition and theory of philology are recent constructions even though Pollock has been “living life philologically” for four decades. Even the \textit{Language of the Gods in the World of Men} (2006) makes no attempt to theorize philology. \textit{Towards a Political Philology: D.D. Kosambi and Sanskrit} (2008) is the first paper where a definition is attempted. \textit{Future Philology: The Fate of a Soft Science in a Hard World} attempts to provide a theoretical basis for the three dimensions of philology, while the \textit{Philology in Three Dimensions} gives only two examples (the \textit{Rāmāyaṇa }and Kālidāsa’s \textit{Śākuntala}) as to how this theory is applied. Thus, the theory of three–dimensional philology covers no more than a few pages, which can be summarized in three sentences, or even three words. There are three planes or dimensions in understanding a Sanskrit (or any other) text.

\textbf{Plane 1: Historical (\textit{aitihāsika})} – This is the meaning according to the author of a Sanskrit text. This understanding can be supported by other historical evidences, but the text should always be the primary source. In this plane, non–traditional sources (Western theoretical models) should be avoided. All of this would be included in \textit{śabda–pramāṇa}.

\textbf{Plane 2: Traditional (\textit{sāmpradāyika})} – This is the meaning of the text according to tradition or the commentators. This has to be supported with textual evidence from the commentarial literature without comparison to non–traditional. \textit{Śabda–pramāṇa} is the basis for this plane of meaning. As mentioned before, all through his academic career, Pollock claims to be representing Indian tradition in his writings.

\textbf{Plane 3: Presentist (\textit{svecchā})} – This is the meaning of the text according to Pollock. In theory, this meaning should also be based on \textit{pramāṇa}–s, but in practice it is \textit{svecchā} (\textit{interpretation}). Every single paper refers to Western theorists (\textit{śabda–pramāṇa}) which is acceptable for this plane, but these theorists are not \textit{āpta}–s as they do not have direct knowledge of the societies they theorise. Thus, their theories are constantly changing as they are dependent on \textit{interpretation} which varies according to time and place. The “search for method” or theory has also occupied Pollock for around forty years and the only way to end this is to adhere to the principles of \textit{Nyāya–śāstra}.

Pollock claims to be an heir to the brilliant tradition of Sanskrit (that this heir can neither write or speak Sanskrit should not be considered a disqualification), but can he be considered an \textit{āpta}? The qualification of an \textit{āpta} was described earlier, but all the indications so far has shown that it would not be possible. This section will confirm the same.

Along planes 1 and 2, inferences would have to be made from the Sanskrit text (\textit{śabda–pramāṇa}) as perceptual knowledge is not possible for past events. Along plane 3, inferences can be from any one of the three: \textit{pratyakṣa}, \textit{upamāna }or \textit{śabda}. In the following two examples on the \textit{Rāmāyaṇa }and the \textit{Śākuntala}, we verify the \textit{pramāṇa}–s that are being used to establish the meaning of these texts along the three planes.


\section*{4.2 The Rāmāyaṇa along the Three Dimensions}

Pollock’s own view on the \textit{Rāmāyaṇa} is discussed first, followed by meaning along planes 1 and 2.

\subsection*{4.2.1 Plane 3: Presentist (svecchā)}

\begin{myquote}
Reading as presentists, upon Plane 3, we encounter the astonishing spectacle of an ancienttext repurposed as an anti–Muslim tract by Hindu fundamentalist politicians – which in 1992 led to the destruction of a mosque in Ayodhya, Rama’s putative birthplace, brought the nation–state to the brink of civil war and is still producing social upheaval. Such a reading prompts us to reflect on the presence of past texts in contemporary India, on this particular text’s malleability and availability for repurposing, and, not least, on our obligations (via Planes 1 and 2) to critically register the history of this malleability and repurposing. (2014c: 408)
\end{myquote}

Of the three planes, the presentist view is predominant, which is then used to interpret the historical and traditional planes. Pollock’s interpretation of the \textit{Rāmāyaṇa }illustrates this aspect. In “\textit{Rāmāyaṇa }and Political Imagination in India”, Pollock alludes to the repurposing of the \textit{Rāmāyaṇa }as an anti–Muslim tract. The introductory portion of this paper describes the events that led to the destruction of a mosque, but how Vālmīki’s\textit{ Rāmāyaṇa }played a part in this is not discussed. How was the \textit{Rāmāyaṇa }text repurposed? Pollock is directly perceiving (\textit{pratyakṣa}–\textit{pramāṇa}) this event that took place in 1992, but no \textit{pramāṇa} is shown to support the repurposing. There should be some section or chapter or even one \textit{śloka} in the \textit{Rāmāyaṇa }that actually lent itself to repurposing. Without showing any evidence as to how the text was repurposed, Pollock feels morally obliged to start a project to “critically register the history of this malleability and repurposing.” It should be noted that plane 3 reading prompts Pollock to look into the past in search of a political aspect embedded in the text. Interpreting the \textit{Rāmāyaṇa} along Planes 1 and 2 is the methodology used in carrying out this project. Thus, according to Pollock himself, even in the personalist (\textit{svecchā}) view, there is no \textit{śloka} (no \textit{śabda–pramāṇa}) that could be called political and was used for repurposing. How then did Pollock get this idea that the text was repurposed? This is answered later in this section.

In order to show that the \textit{Rāmāyaṇa} was written with a political intent, it is placed in a post–Ashokan time frame in the historical plane.


\subsection*{4.2.2 Plane 1: Historical (aitihāsika)}

In this plane, two aspects are vital: the chronology of the \textit{Rāmāyaṇa} in relation to Ashoka and its political nature. The chronology aspect is considered first.

\begin{myquote}
It [\textit{Rāmāyaṇa} of Vālmīki] has an obscure moment of genesis and an even more complex history of reception. Reading as a historicist upon Plane 1, one could argue (as I have argued) that the \textit{Rāmāyaṇa} belongs to the thought–world of \textbf{post–Ashokan} India. (2014c: 407)
\end{myquote}

The \textit{Rāmāyaṇa} is placed one or two centuries before the Common Era with no supporting textual evidence. From 1985 (post–\textit{Orientalism} period), motifs such as “obscure” and “complex” appear repeatedly in papers, and it means that Pollock is going to “see thorough” the texts and into the Indian mind, and perceive the political imagination of the Indians. Ashoka was an important figure in Indian history, and Vālmīki’s \textit{Rāmāyaṇa }is considered “the most important literary work,”; but neither the Ashokan inscriptions nor the \textit{ādikāvya} have made any references to each other. Since this plane represents the meaning (historicist) as intended by the author, any inference should be based primarily on the text with supporting evidence. Pollock, however, is unable to show any\textit{ śabda}–\textit{pramāṇa} from the text itself or any other historical evidence to prove that it is post–Ashokan. Moreover, his earlier paper contradicts this view.

\begin{myquote}
It was probably not much earlier than the seventh century B.C. that the major urban centres of aryan India came into existence, and yet during the composition of the \textit{Rāmāyaṇa} in \textbf{pre–Mauryan} times…(1986a:3) Vālmīki’s chronological relationship with the Mauryan king [Ashoka] is problematical.” (1986a: 24)
\end{myquote}

In the beginning of the introduction to \textit{Ayodhyākāṇḍa}, the composition of the \textit{ādikāvya} is placed before the time of Ashoka (pre–Mauryan). In the later part of the introduction, it is admitted that the chronological relation between the two cannot be conclusively established. Thus, this earlier view (1986) contradicts the later one (2014), but as noted before, contradictions are acceptable in Pollockian philology as theories (\textit{interpretation}) always supersede \textit{pramāṇa}–s. Such contradictions in the interpretation of the \textit{Rāmāyaṇa} were documented earlier in this \textit{prabandha}.

The political aspect is now discussed. Placing the \textit{Rāmāyaṇa }after Ashoka is very important to show that the \textit{ādikāvya} was political at the time of its composition.

\begin{myquote}
It bears the clear impress of Ashoka’s new quasi–Buddhist \textbf{political theology}, where power takes on a marked and unprecedented spiritualized dimension. (2014c: 407)
\end{myquote}

Ashoka’s political theology of power taking on spiritual dimension is not adequately explained in any of the earlier papers. An earlier paper (1986a: 23) quotes an Ashokan inscription which discusses the importance of \textit{dharma,} and the \textit{Rāmāyaṇa }is also shown to have similar discussions. But power taking on a spiritualized dimension is neither shown in the Ashokan inscriptions or in the \textit{Rāmāyaṇa}. Here, Ashoka is supposed to have propagated a “political theology.” But, once again, an earlier paper contradicts this view.

\begin{myquote}
The assumption that a religious plan – promoting a universalist Buddhism – underpinned Ashok’s quasi–universal empire has rightly been discarded in contemporary scholarship, and the new consensus emphasizes the \textbf{political–moral over religious} (let alone sectarian) nature of the doctrine of dhamma (\textit{dharma}). (2006b:181)
\end{myquote}

This passage from \textit{Empire and Imitation} is an example of how Pollockian theories are constantly changing. In this (2006) \textit{interpretation}, it is said that Ashok’s motive was not religious (theology), but rather a political–moral one, and thus there would be no possibility of power taking on a spiritual (or religious) dimension. This earlier \textit{interpretation} combines the political with the moral, while the later (2014) combines it with the spiritual (religious or theology). An even earlier (1984) \textit{interpretation} (pre–\textit{Orientalism} period) indicates the actual nature of the \textit{Rāmāyaṇa}. \textit{The Divine King in the Indian Epic} states that

\begin{myquote}
Although it goes without saying that in dealing with a literary text, we must not expect the discursive exposition of a \textit{śāstra}, still, even given these generic limitations, there is \textbf{some evidence} in the \textit{Rāmāyaṇa} enabling us to characterize its political–theological orientation. (1984b: 523)
\end{myquote}

Pollock cites three \textit{śloka}–s, two from \textit{Araṇyakāṇḍa} (3.1.17–18) and one from \textit{Ayodhyākāṇḍa} (2.95.4) as \textit{pramāṇa}–s, but he himself calls them as “mere inferential evidence.” \textit{Rājadharma} portion of \textit{Dharmaśāstra} and other texts are quoted, but not one \textit{śloka} from the entire \textit{Rāmāyaṇa} is given to us to show that the \textit{Rāmāyaṇa} is a text of political theology.In this paper on the \textit{Rāmāyaṇa}, it is said that there is only “some evidence” of its political orientation. Thus, this 1984 paper is before the “aestheticization of power” is propounded in response to Babri Masjid demolition. From 1993 onwards, the \textit{Rāmāyaṇa} becomes primarily a political text.

In the first or historical plane, which should represent Vālmīki’s view, two aspects needed to be proved: the \textit{Rāmāyaṇa} is post–Ashokan and that it is a political text. Neither has been proved as we are shown no evidence from the \textit{Rāmāyaṇa} or other historical sources. Hence, in plane 1, there is no \textit{śabda}–\textit{pramāṇa} to establish the \textit{Rāmāyaṇa} is either post–Ashokan or that it is a political text.


\subsection*{4.2.3 Plane 2: Traditional (sāmpradāyika)}

The plane 2 reading is important as it is supposed to represent the Indian viewpoint.From the standpoint of \textit{pramāṇa}–s, there is no difference between planes 1 and 2. The author of a Sanskrit text (plane 1) and (those of) its traditional commentaries (plane 2) are both included under \textit{śabda–pramāṇa}. In the second plane also, we would expect Pollock to closely follow and present whatever is being discussed in original Sanskrit texts and their commentaries.

\begin{myquote}
But virtually all accounts exclude the second plane of philology, the readings offered by tradition. (2014c: 402)
\end{myquote}

Pollock claims that scholars earlier had ignored the second plane or the traditional viewpoint. For example, Indologists such as WD Whitney ignored the tradition by discarding what the commentators had to say about the \textit{Veda}–s.

\begin{myquote}
Seeing the \textit{Rāmāyaṇa} on Plane 2 with the eyes of some traditional Hindu readers (other traditional readers, such as Jains and Buddhists, had other eyes to see with), we recognize the \textbf{presence of scripture}.
\end{myquote}

How did the traditional commentators perceive the text of Vālmīki’s \textit{Rāmāyaṇa}? They did not view it as a political text. This is an important admission and means that there is nothing in the commentaries (no\textit{ śabda}–\textit{pramāṇa}) that could be used to show that the \textit{Rāmāyaṇa }was political and that it was repurposed. There are several published and unpublished commentaries on the\textit{ādikāvya,} including Govindarāja’s\textit{Bhūṣaṇa}, Nageśabhaṭṭa’s \textit{Tilaka}, Maheśvaratīrtha’s \textit{Tattva–dīpikā}, Śivasahāya’s \textit{Rāmāyaṇa–śiromaṇi} and Mādhavayogīndra’s \textit{Kataka}. Tryambakamakhin, who was a minister in the court of Tanjavur King Sahaji, wrote an extensive commentary called \textit{Dharmākūta }in which the dharmic aspects are highlighted. Along with the unpublished commentaries, these would amount to a vast literature. Thus, in this vast commentarial literature on the \textit{Rāmāyaṇa}, Pollock admits there is no \textit{pramāṇa} to connect the \textit{Rāmāyaṇa }to the political. It should also be noted that even the Jains and the Buddhists did not view the\textit{ ādikāvya} as political.

\begin{myquote}
Such a vision of the text emerged only in the early second millennium, and even if derived from the older political theology, it is no \textbf{longer recognizably} such. For these readers, medieval theologian–commentators, the poem is an absolutely true record of God’s deeds on earth – Rama as avatar of the god Vishnu – a conviction that made the vernacular versions, above all the sixteenth–century Hindi adaptation \textit{Rāmcaritmānas}, among the most important religious texts of India. (2014c: 407)(\textit{emphasis ours})
\end{myquote}

It was claimed that the \textit{Rāmāyaṇa }was political in its genesis. But the \textit{interpretation} along plane 1 showed no \textit{śabda}–\textit{pramāṇa }to support this claim. In the vast literature (commentaries and regional adaptations) that has sprung from the \textit{Rāmāyaṇa }over a very long period, there is nothing in the texts themselves to show that it was political in nature. This again implies that there is no \textit{śabda}–\textit{pramāṇa} along plane 2 for saying that the \textit{Rāmāyaṇa }is a political text, i.e., the commentators and adaptors (Tulasidas and Kamban) never discuss the political nature. A text like \textit{Rāmcaritmānas} which can be taken to reflect the Indian “imagination” of that time has nothing political in it. Kamban’s\textit{ Rāmāyaṇa},which is considered a masterpiece in Tamil also did not view the \textit{ādikāvya }as political. The vernacular traditions follow the Sanskrit text with some adaptations, but across the whole country, the \textit{Rāmāyaṇa} was never considered political by the tradition.


\subsection*{4.2.4 Nirṇaya}

In the historical plane (plane 1), not even a single \textit{śloka} of Vālmīki’s was shown as \textit{śabda}–\textit{pramāṇa }to establish that the \textit{ādikāvya }was post–Ashokan and therefore political in its genesis. Again, in the traditional plane (plane 2), the commentators never viewed the \textit{ādikāvya} as a political text. Even along the personal plane (plane 3), Pollock fails to show how the text itself or a section or even a single \textit{śloka} was used for political repurposing. Moreover, contradictions between the earlier and later \textit{interpretations} have also been documented here.

On the first and second planes, which need to be established by \textit{śabda–pramāṇa}, Pollock provides no \textit{śloka}–s from the text or the commentaries to establish the inference that the\textit{ Rāmāyaṇa }is political. On the third plane, which is established by \textit{pratyakṣa–pramāṇa}, Pollock is unable to show how the text itself was repurposed. If an inference is not from \textit{pratyakṣa }or \textit{śabda}, then it necessarily has to be from \textit{upamāna }or comparison.

How then did Pollock get this idea that the text was repurposed? He himself states the source of this idea.

\begin{myquote}
One of the most suggestive features of this whole problematic, to my mind, is the very fact that imagining and representing the political present in twelfth–century India was enabled by a recuperation of the past...In a celebrated passage in \textit{\textbf{The Eighteenth Brumaire}}, Marx ascribes an almost law like character to this process of historical imitation: With the weight of the dead generations on their brain, the living "anxiously conjure up the spirits of the past to their service and borrow from them names, battle cries and costumes in order to present the new scene of world history."…I am clearly not prepared to extend the political imitative imaginary of twelfth–century India this far, but something \textbf{analogous}, I think, is taking place…(1993a: 279–280)
\end{myquote}

“Analogous” would mean \textit{upamāna }and Pollock establishes the political imagination in the \textit{Rāmāyaṇa }by comparing it with a theory described in \textit{The Eighteenth Brumaire}. This long passage shows the essence of Pollockian philology which is not different from what philology always was: \textit{interpretation} supersedes all \textit{pramāṇa}–s. A theory is forcefully applied to understand a Sanskrit text and even with no evidence from either the \textit{Rāmāyaṇa} or its commentaries and adaptations, it is now claimed that the \textit{ādikāvya }is political.

In a recent paper, Pollock claims that Swadeshi Indologists deny others the right to ask important questions about the history of classical culture such as

\begin{myquote}
how the \textit{Rāmāyaṇa}, the Sanskrit epic poem, relates to political discourse—a discourse that cannot be ignored \textbf{without doing violence to the work}—both at the period of its original composition and over the following centuries. (2016: 926)
\end{myquote}

It is claimed that we would be doing violence to the \textit{ādikāvya} if we do not interpret it as political. The contradictory \textit{interpretation} of the \textit{ādikāvya }is summarized for clarity.

\item 1984–86: No evidence that \textit{ādikāvya }is political. It is pre–Ashokan.

 \item 1993: It becomes a political text because of political events in India. Aestheticization of power is theorized from Western sources.

 \item 2006: Ashokan thought is political–moral and not political–theology.

 \item 2014: \textit{Ādikāvya }is political, post–Ashokan and Ashokan thought is political–theology.

 \item 2016: We are “doing violence to the work” if we don’t consider it political.

Our introduction repeatedly showed that philology was equated with \textit{interpretation}. Thus, theories are propounded even though they contradict\textit{ pramāṇa}–s and his own earlier views. The\textit{ ādikāvya} in interpreted in a way to fit a theory. In contrast, what Swadeshi Indologists want is that both questions and answers to be based on \textit{pramāṇa}–s, and not \textit{a priori} application of theory. Swadeshi Indology aims at real and deep scholarship along the lines of śāstric traditions, and not a superficial philology where \textit{interpretation} is always political. This interpretation of the \textit{ādikāvya }is sufficient to show that the political imagination of Pollockian mind has no basis in any \textit{pramāṇa}–s.

An analysis of \textit{Abhijñāna–śākuntala} in the following section would further substantiate this nature of philology.


\section*{4.3 Śākuntala along the Three Dimensions}

Pollock provides one more example to illustrate the method of comprehending a text along the three dimensions. The meaning of Kālidāsa’s \textit{Abhijñāna–śākuntala }along the three planes is now discussed.

\subsection*{4.3.1 Plane 3: Presentist (svecchā)}

\begin{myquote}
... \textit{Śākuntala}, the fifth–century Sanskrit drama by Kalidasa. And when you do teach this work to undergraduates, you inevitably encounter a third–plane reading that finds the play’s gender imbalance – especially Kalidasa’s enfeeblement of the heroine who appears so strong in the epic source, the \textit{Mahābhārata} – to be nearly fatal to any contemporary literary appreciation. (2014c: 408)(\textit{spellings as in the original})
\end{myquote}

This is an example of how the mind naturally works. Students study Kālidāsa’s \textit{kāvya} or \textit{nāṭaka} and understand it based on their own background. Present day students would look for gender equality and thus find Śakuntalā to be a weak character. Students read the text closely and assess the portrayal of the heroine Śakuntalā. Many contemporary students would feel the Kālidāsa’s description would not fit their own image of strong feminine characters. It should be noted that these students are not going beyond the text: the text is accepted as \textit{pramāṇa} and their interpretations are based on it. The pertinent issue for us is to see if the three–dimensional approach is possible at all. For the students, there is only one meaning of the text. Similarly, Pollock shows a singular meaning of a text (and that too without \textit{pramāṇa}–s) along the personal plane. The three–dimensions or planes is not a natural state of mind: entering the Indian mind and interpreting it is done only in Pollockian philology. This aspect is illustrated by Pollock’s own view of \textit{Abhijñāna–śākuntala}.

\begin{myquote}
‘Shakúntala’ is a ‘Maha·bhárata’ play, and ‘Rama’s Last Act’ seems designed as a ‘Ramáyana’ counterpart to, and competitor of, Kali·dasa’s masterpiece. Like the two epics the two plays share a deep resemblance. In their core they are stories about love, rejection, recovery, and \textbf{ultimately}—because this is the very reason behind the rejection— \textbf{political power and its perpetuation}…(2007b: 34) (emphasis ours) (diacritics as in the original)
\end{myquote}

The political nature of the \textit{Rāmāyaṇa }was considered established by applying a theory (from\textit{ The Eighteenth Brumaire}) and here also no \textit{pramāṇa}–s (or even theories) are shown to substantiate the political nature of this \textit{nāṭaka}. This \textit{interpretation} indicates that in Pollockian philology, any Sanskrit work can be interpreted as political even without a supporting Western theory. One should note that Pollock uses the word “political” in the Western sense and not in the sense of \textit{Rāja–dharma}.


\subsection*{4.3.2 Plane 2: Traditional (sāmpradāyika)}

\begin{myquote}
...tradition was concerned with the play’s emotional states, and above all with specifying the architecture of this aesthetic (how each state linked to the others). (2014c: 408)
\end{myquote}

According to Pollock’s own admission, tradition (there are numerous commentators on \textit{Abhijñāna–śākuntala}) never viewed this \textit{nāṭaka }as political. The enquiry into the meaning of the \textit{nāṭaka }is concerned primarily with \textit{Rasa}–s as far as the commentators are concerned. Pollock makes a distinction between \textit{kāvya} and \textit{nāṭaka}, and throughout his works,\textit{ kāvya }is the primary aesthetical tool used for propagating political thought. Kālidāsa (referred to no less than thirty times in \textit{The Language of the Gods} usually in the context of the political aesthetics) becomes a “canonical” author whose works imbibe the political canon of the cosmopolitan Sanskrit culture. Such a “canonical” author should have mentioned something about the political in his \textit{nāṭaka }also, but the traditional commentators find no such instances. Hence, for commentators, the \textit{nāṭaka }is an aesthetical production and not a political one. Pollock gives one more example:

\begin{myquote}
As a verse in circulation for centuries asserts, the essence of Sanskrit theatrical literature lies in \textit{Śākuntala}; the essence of all its seven acts lies in the fourth act; the essence of all the poems in the fourth act lies in the four verses spoken by Shakuntala’s foster–father, a forest–dwelling renunciant, as he bids her farewell to join her husband. For many traditional ‘sociological’ readers, Kalidasa’s play \textit{is about} the primal separation of fathers and daughters, a core fact of everyday Indian life. (2014c: 408)
\end{myquote}

A well–known Sanskrit \textit{śloka} is cited: “\textit{kāvyeṣu nāṭakaṁ ramyaṁ tatra ramyā śakuntalā | tatrāpi caturtho'ṅkaḥ tatra śloka–catuṣṭayam}.” This \textit{śloka} preserved in the oral tradition states that the essence of \textit{Abhijñāna–śākuntala }is captured in the four verses and the following is one of them.

\begin{myquote}
‘My heart is touched with sadness / since Shakuntala must go today, … / If a disciplined ascetic / suffers so deeply from love, / how do fathers bear the pain / of each daughter’s parting?’ (2014c: 408)
\end{myquote}

The four verses are about the separation of a father from his daughter. Thus, once again, the essence of this \textit{nāṭaka }is not considered political by tradition, but the parting of a daughter from a father, and the heroine is not viewed as representing political power as Pollock theorizes.

\begin{myquote}
Later traditional readers – with their own vernacular sociologies – include Goethe around 1800, an age and a world away, but still part of the history of reception. As the first Indian literary work made known to modern Europe, \textit{Śākuntala} constituted for Goethe a perfect embodiment of the purity of imagination from the infancy of the culture that ultimately (it was then believed) produced Europe. (2014c:408)
\end{myquote}

Goethe(date), a German literary giant, did not see this \textit{nāṭaka }as political, nor did any of the early German Indologists. How does Pollock know of Goethe’s understanding of the \textit{Śākuntala?} In other words, what is the \textit{pramāṇa} being used? It should only be by reading what Goethe had written (\textit{śabda–pramāṇa}) about it. There is no attempt to enter the mind of Goethe, and “exhume” what he actually felt, or how the Germans viewed themselves. This should be contrasted with Pollockian interpretation of this \textit{nāṭaka} where no \textit{śabda–pramāṇa }(evidence from commentaries) is provided to establish its political nature. Finally, Rabindranath Tagore’s (date) interpretation is contrasted with Pollock’s own.

\begin{myquote}
A century after Goethe, the Bengali poet Rabindranath Tagore read \textit{Śākuntala} in a way especially hard for contemporary readers to credit, since he insisted that the play’s central themes are sexual fall and redemption. It hardly matters that in the terms of the play itself the heroine’s behaviour is nowhere viewed as blameworthy. The new Victorian sensibilities of late–colonial Bengal and, perhaps more important, the need for a morality suited to the coming independence of India, led Tagore to find in the play – indeed, to see as its ultimate meaning – a concern with tempering freedom by restraint. (2014c: 408–409)
\end{myquote}

A century after Rabindranath Tagore, Pollock read the \textit{Śākuntala} (and all Indian literary traditions) in a way especially hard for traditional readers to credit, since he insisted that the play’s central themes are political power and its perpetuation. It hardly matters that in the terms of the play itself (and the commentaries) the central theme is nowhere viewed as political. The new post–\textit{Orientalism} sensibilities of the late–twentieth century humanities and, perhaps more important, the need for a theory suited to over–coming the imminent death of philology, led Pollock to find in the play – indeed, to see its ultimate meaning – a concern with tempering the aesthetics by the political.


\subsection*{4.3.3 Plane 1: Historical (aitihāsika)}

The historical meaning of a text should be primarily based on the text itself which can then be supported by other evidences such as inscriptions. All this would be included as \textit{śabda–pramāṇa} (textual authority).

…that recapitulates the design of the destiny that joined the \textbf{human–divine king} with the elemental power of nature in order to produce a son who will rule the four quarters of the world. This son is Bharata, whose eponym – Bharatavarsha (Clime of Bharata) is an ancient name of India – gestures toward the new imperial formation of the Gupta kings under whom Kalidasa wrote. (2014c: 409)

In the previous example, the \textit{Rāmāyaṇa }was placed after the time of Ashoka to show political connection and here Kālidāsa’s works are placed in the period of the Guptas for the same reason. Kālidāsa makes no mention of the Gupta kings, and Pollock provides no other \textit{pramāṇa}–s to substantiate this political–literary nexus. The meaning along this plane should have its basis in the original text, but no such \textit{pramāṇa} is provided. What then is the difference between the plane 1 and plane 3 reading? The plane 1 \textit{interpretation} about the human–divine king is drawn from the (unproven) thesis of an earlier paper \textit{The Divine King in the Indian Epic}.

…the conception of the divine nature of the earthly king, may \textbf{not always be easy to trace}, and if its precise meaning and significance must remain matters of \textbf{interpretation}, still, in the opinion of most contemporary scholars it manifests itself too plainly for us to deny that it was an important feature of this world view from as early as the time of the Vedas. (1984b:522)

The earlier paper uses the concept of human–divine king to interpretthe \textit{Rāmāyaṇa}, which is now extended to the \textit{Śākuntala}. It is admitted that this concept has little or no textual evidence in Sanskrit texts and thus one would have to interpret (or infer from Western sources) which is what Pollock does. More importantly, the reliance on contemporary scholars (the footnote names Spellman and Gonda) indicates the lack of \textit{śabda–pramāṇa }(textual evidence) in the Sanskrit texts themselves. Thus, even along plane 1, interpretation is more important than \textit{pramāṇa}–s. The meanings along the three planes are summarized here for clarity

\item Plane 3: the \textit{Śākuntala }is ultimately a political text: No \textit{pramāṇa} from the text is shown.

 \item Plane 2: the \textit{Śākuntala} is not seen as political – by the commentators – a traditional verse representing the oral tradition, Goethe and Rabindranath Tagore.

 \item Plane 1: the \textit{Śākuntala} is placed in the period of Guptas – so as to suggest political nexus, and the human–divine king is postulated citing Western sources.

As the plane 1 reading does not base itself on textual sources, it essentially is a plane 3 reading. Both represent only a personal view and thus a single dimension. In the introduction to\textit{Uttararāmacarita}, which is before the three–dimensional philology is theorized, Pollock conclusively states that the \textit{Śākuntala }is about “political power and its perpetuation” with scant regard for tradition. There is no evidence provided for the political nature of the \textit{Śākuntala} along the three planes. The political \textit{interpretation} of the \textit{Śākuntala }is similar to that of the \textit{Rāmāyaṇa} and is considered established without any textual evidence.


\subsection*{4.3.4 Nirṇaya}

The three dimensions of philology are summarised from the viewpoint of \textit{pramāṇa}–s:

\textbf{Plane 1: Historical} – In this plane, effort should be to get as close to the meaning as intended by the author of the text. The text itself should be the primary source which can then be supported by other sources such as inscriptions, etc., (all these would be \textit{śabda–pramāṇa}). A careful reading has shown us that along plane 1, nothing from within the Sanskrit texts would be shown as \textit{pramāṇa}. If \textit{śabda–pramāṇa} has not been shown and if we cannot use \textit{pratyakṣa–pramāṇa} as we cannot directly perceive events that took place long time ago, then only\textit{ upamāna–pramāṇa} could be used to validate the inferences along plane 1. Essentially, the meaning along this plane would only be a personal one and the historical plane is in essence the same as the personal (plane 3).

\textbf{Plane 2: Traditional} – In the second plane, the goal is to understand the meaning of the text as seen by the tradition (commentators). The primary source would have to be commentaries (including unpublished manuscripts) and can also be regional versions in the case of the \textit{Rāmāyaṇa }(Kamban, Tulasidas, etc.). All of these would come under \textit{śabda–pramāṇa }and in this plane also \textit{pratyakṣa–pramāṇa} would not be possible as we cannot directly perceive events that took place several hundred years ago. A careful reading of plane 2 in both the examples revels that Pollock admits that these two texts were not viewed as political.The \textit{Rāmāyaṇa }was viewed as a religious text and the \textit{Śākuntala} as an aesthetic one. Pollockian philology along this plane goes beyond the Sanskrit texts and tries to read the Indian mind or conscious. Thus, the meaning along this plane would also be a personal one with no textual basis.

\textbf{Plane 3: Presentist }– This represents a personal view, but one would expect this to be based on \textit{pramāṇa}–s, and not mere conjectures. In the first example concerning the \textit{Rāmāyaṇa}, some political events in India were directly connected with Vālmīki’s text without showing any \textit{pramāṇa}–s. Similarly, the \textit{Śākuntala }was also considered ultimately political with no supporting \textit{pramāṇa}. Even a personal view should have some basis in \textit{pramāṇa}–s: inferences drawn from Western \textit{theories} cannot be considered as valid, since these theories are what they are – unproved conjectures.

In conclusion, the three–dimensional philology is essentially one–dimensional personal view of Pollock. It does not follow his practice (section 4.6 establishes this aspect) and is “consciously constructed” or \textit{kalpita,} in an effort to save philology from its eventual death. Beginning with the 1985 paper and in every single paper that follows, Pollock is unable to show even a single \textit{pramāṇa }(from Sanskrit texts) to establish the thesis statements which are repeatedly substantiated by application of Western theories.


\section*{4.4 The Three Truths}

An analysis of the three–dimensional philology in the previous section has shown it to be after all one–dimensional. Even if the three dimensions are accepted, it gives raise to the possibility of three different meanings, and there arises a question as to which meaning would then be the correct one. The truth value of a text is discussed in several philological works and it is said that the meaning of a text along the three dimensions could have different truth values.

\begin{myquote}
And this suggests that there are three dimensions of meaning—the author’s, the tradition’s, and my own—and thus three forms, potentially \textbf{radically different forms, of textual truth}. (2016a: 16)
\end{myquote}

The question now is how to resolve the three possibly conflicting truths. Which one of the three should be the predominant one? Pollock states that the text’s

\begin{myquote}
…\textbf{one true meaning} can be nothing but the assemblage of all these other meanings, which we use in our different ways depending on the kind of sense we are aiming to make of the world: what the text may have meant to the first audience; what it meant to readers over time; what it means to me here and now. The balancing act required to read this way is indeed delicate, especially since we want to allow all \textbf{three to represent forms of truth}, while feeling no compulsion to rank or even to reconcile them. (2014c: 410)
\end{myquote}

It is claimed that the \textit{one true meaning} of a text results from putting together the truths that arise along the three planes. It was previously mentioned that there could be “radically different” forms of truth. If, along the three planes, three truths are opposed to each other, then a method of resolving this conflict would be needed. How does one assemble the three different truths into one? Which one of the three should be considered the strongest truth? No such method of assembling these truths into one is mentioned in any work on philology. One true meaning would also imply that the three dimensions are ultimately of no use. If we are aiming for one true meaning, what is the purpose of the three dimensions? Further, a contradiction is seen even between the first and the last sentences of the above passage. If we should not rank or reconcile the three truths, how then do we arrive at the “one true meaning?”

The “delicate balancing act” of Pollock becomes even more confused when he contradicts himself by proclaiming the opposite: one true meaning or absolute truth is not the aim.

\begin{myquote}
What we philologists should care about is what has been held to be true at those times and places, and why that was held to be true; we should not be concerned with what is \textbf{absolutely true}, since \textbf{absolute truth has no place} in the realm of making sense of texts. (2016b: 24)
\end{myquote}

Earlier “one true meaning” was the aim of philology, now “absolute truth” (or one true meaning) is discarded from the scope of philology! Even so, we still are left with three relative truths (for each of the three planes) that needs to be evaluated. For example, Pollock considers the \textit{Rāmāyaṇa} as primarily political, whereas the tradition does not. Even if we are not concerned with the absolute and the true meaning of the \textit{Rāmāyaṇa}, how do we reconcile the political and the apolitical? A \textit{pramāṇa parīkṣā} (in the previous section) of the two examples concerning the \textit{Rāmāyaṇa} and the \textit{Śākuntala }clearly showed a singular truth and an analysis of several papers later [section 4.6] will show the same. Even in theory, the meaning along the personal (plane 3) always prevails. Thus, if the three truths are allowed to coexist in our minds, there still would be a need for some method by which the “real meaning of the text” or its “one correct interpretation” is ascertained (see 2016d:921). Is such a method possible? The philological papers do not discuss any such method and in practice the three dimensions are non–existent and therefore a method is not even necessary.

\textit{\textbf{Nirṇaya}}

Pollock’s contradictory and amateur views on truth only goes to show philology’s untheorized nature.

\item The three truths are assembled to get the one true meaning. If so, then there is no question of three–dimensional philology.

 \item The same passage suggests that we should not rank them. How then do we get the one true meaning?

 \item Philology is not about one true meaning or absolute truth which contradicts [1].

Such contradicting statements imply that philology is ultimately not about truth. If philology is not about truth and is of a mere personal nature, what then does it have to do with freedom or ethics? A recent paper is deceptively titled \textit{Philology and Freedom} (2016b) as if to imply that philology properly cultivated would lead to freedom. But Pollockian philology does not lead us to freedom, and in fact does just the opposite, as an \textit{a priori} application of Western models can only bind (not only others but also Pollock).


\section*{4.5 The Five Avayava–s}

In the previous section, the theory of the three–dimensional philology was examined. In this part of the \textit{prabandha}, some works of Pollock are evaluated to see if they conform to the three–dimension theory of understanding texts. Does theory follow practice? Pollock’s most important claim is that he has been faithfully representing the Indian tradition throughout his academic life and that the three–dimensional philology actually theorizes his own practice over a period of four decades. To verify this claim, several papers are analysed using the \textit{pañcāvayava} method along \textit{abhyupagamasiddhānta} of \textit{Nyāya–śāstra} both of which are explained in this section.

\textit{\textbf{Nyāya–bhāṣya}}[p.30]

\begin{verse}
\textit{atha avayavāḥ |}
\end{verse}

Then, (the five) \textit{avayava}–s (are discussed).

\textbf{\textit{Nyāya–sūtra} 1.1.32}

\begin{verse}
\textit{pratijñā–hetūdāharaṇopanaya–nigamanāny avayavāḥ} |
\end{verse}

The five \textit{avayava}–s (components of an inference) are:

\item \textit{pratijñā} – the preliminary statement of the thesis.

 \item \textit{hetu} – the reason or the cause.

 \item \textit{udāharaṇa} – the exemplification.

 \item \textit{upanaya} – the application.

 \item \textit{nigamana} – the conclusion.

The \textit{Nyāya–bhāṣya}\supskpt{\footnote{[ 26 ] \textit{Nyāya–bhāṣya} states that the four \textit{pramāṇa}–s are collectively present in the five \textit{avayava}–s. The \textit{pratijñā} is \textit{āgama }or\textit{ śabda}. The \textit{hetu} is \textit{anumāna}. The \textit{udāharaṇa} is \textit{pratyakṣa}. The \textit{upanaya} is \textit{upamāna}. The \textit{nigamana} shows how all the four are used together. This is called \textit{parama} (highest or most complete) \textit{nyāya} and this is helpful in establishing the truth.}} explains the relationship between these five \textit{avayava}–s succinctly.

\textit{\textbf{Vivaraṇa}}

The section which discusses the five \textit{avayava}–s (components of an inference) is known as the \textit{nyāya–prakaraṇa}. The term \textit{nyāya} means the statement of the five \textit{avayava}–s as specified in the \textit{sūtra} (1.1.32). Commenting on the first \textit{sūtra}, Vātsyāyana calls this the highest form of \textit{nyāya}. \textit{Anumāna} is divided into two forms: \textit{svārtha} – inference meant for knowing any object or thesis for one’s own self; and \textit{parārtha} – inference for making others understand the thesis. Presenting in the form of the five \textit{avayava}–s clearly illustrates the \textit{pramāṇa}–s used by Pollock to support the thesis statements in his papers.

It is mentioned in the \textit{Nyāya–bhāṣya} that the five \textit{avayava}–s represents all the four \textit{pramāṇa}–s. The five \textit{avayava}–s are illustrated with the \textit{agni} (fire) and \textit{dhūma} (smoke) example which is the stock example in śāstric texts. The same method will be then be applied to evaluate the papers of Pollock.

This form of presentation is a reflection of the how the human mind works and is necessary here for examining the soundness of Pollock’s theories. Papers presented in this format would clearly show if theories are being substantiated from within the Indian tradition or simply from Western models.Therefore, when presented in the form of \textit{parārthaanumāna}, the thesis statements and conclusions can be known without ambiguity.

\subsection*{4.5.1 Pratijñā\supskpt{\protect\footnote{[ 27 ] \textit{Pratijñā} – \textit{Nyāyasūtra} 1.1.33

\begin{verse}
\textit{sādhyanirdeśaḥ pratijñā |}
\end{verse}

\textit{Pratijñā} (the thesis statement) is the \textit{nirdeśa} (the specific mention) of \textit{sādhya} (that which needs to be established or proved).

\textit{Vivaraṇa}

The \textit{Nyāyabhāṣya} explains that in the \textit{pratijñā}, the \textit{dharmī} (an object) is stated as being qualified by a \textit{dharma} which needs to be proved. The example given here is that \textit{\textbf{śabda}}\textbf{ (sound) is }\textit{\textbf{anitya}}\textbf{ (not eternal)}. \textit{Śabda} (\textit{dharmī}) is qualified by \textit{anitya} (\textit{dharma}) which needs to be proved. Śrī Phaṇibhuṣaṇa further adds that the word \textit{sādhya} is used in two senses. \textit{Nyāyabhāṣya} uses the word \textit{sādhya} to mean a \textit{dharmi} qualified by a \textit{dharma}. \textit{Sādhya} can also be taken in the sense of just the \textit{dharma} itself.}}}

(thesis statement) – “The hill has fire”. The statement “the hill has fire” is called \textit{āgama} and is the one that needs to be established.

\textbf{\textit{Parīkṣā} of 3D Philology}

\textit{Pratijñā} in 3D Philology is the thesis statement of Pollock that needs to be proved. Almost all papers will necessarily have a \textit{pratijñā }which is usually stated in the abstract or the introductory part of a paper.


\subsection*{4.5.2 Hetu\supskpt{\protect\footnote{[ 28 ] \textit{Hetu – Nyāya–sūtra} 1.1.34–35

\textit{udāharaṇa–sādharmyāt sādhyasādhanaṁ hetuḥ | tathā vaidharmyāt |}

By similarity with the \textit{udāharaṇa} (example cited), the \textit{sādhana} (establishment) of \textit{sādhya} (that which needs to be established) is \textit{hetu}. The \textit{hetu} is also the \textit{sādhana} of \textit{sādhya} through dissimilarity with the \textit{udāharaṇa}.

\textit{Vivaraṇa}

These two \textit{sutra}–s together define \textit{hetu}, the first by similarity and the second by dissimilarity. The \textit{Nyāyabhāṣya} explains that by showing similarity or dissimilarity with the \textit{udāharaṇa}, establishing the \textit{dharma} that needs to be established is \textit{hetu}. The example is that \textit{`} (sound) is \textit{anitya} (not eternal) \textbf{because it is produced or created}. What is produced is known to be \textit{anitya} as in the case of a pot. Here, the \textit{dharma} of “being produced” is common to both \textit{śabda} and the pot. \textit{Hetu} is also shown by dissimilarity: \textit{śabda} is \textit{anitya} because it is produced. That which is not created or produced such as\textit{ ātmā} is \textit{nitya} (eternal). In this example, the \textit{dharma} of “being produced” and “not being produced” are dissimilar to each other. \textit{Hetu} is the most important step in establishing the \textit{pratijñā} and thus has received the greatest attention in Indian śāstric traditions.}}}

(reason or cause) – [The hill has fire] because smoke is perceived on the hill. After observing the smoke, we infer fire on the hill because it is similar to the \textit{udāharaṇa} even though it (the fire) cannot be directly perceived.

\textbf{\textit{Parīkṣā} of 3D Philology}

In the papers of Pollock, \textit{hetu} is used to substantiate the thesis statements. Once the \textit{pratijñā} is known, then a careful study of the evidence provided by Pollock will show if the evidence shown for establishing the \textit{pratijñā} has similarity or dissimilarity to the \textit{udāharaṇa} cited. In all papers, the \textit{hetu} will be shown to be dissimilar to the \textit{udāharaṇa }cited.


\subsection*{4.5.3 Udāharaṇa\supskpt{\protect\footnote{[ 29 ] \textit{Udāharaṇa – Nyāya–sūtra} 1.1.36–37

\textit{sādhyasādharmyāt taddharmabhāvī dṛṣṭānta udāharaṇam | tadviparyayād vā viparītam |}

\textit{Udāharaṇa} is stating the \textit{dṛṣṭānta} (case or instance) which is similar (\textit{sādharmya}) to the \textit{sādhya} (that which needs to be proved) and possesses its \textit{dharma}.

Vivaraṇa

Śrī Phaṇibhuṣaṇa comments that \textit{udāharaṇa} will also be of two kinds (as in the case of \textit{hetu}) based on similarity and dissimilarity and the first type is illustrated as follows. \textit{Śabda} (sound) is \textit{anitya} (not eternal) because it is produced. Here, a pot is taken as the \textit{udāharaṇa}. The \textit{dharma} of being produced is in the pot also and thus it is not eternal. The \textit{dharma} of “being produced” is there both in \textit{śabda} and the pot and so they are similar. In the example of the pot, both the \textit{dharma}–s of “being produced” and “not eternal” are present.}}}

(exemplification) – We know from our perceptual experience that wherever there is smoke there is also fire. This will have been observed many times in various places. Without this experience, we will not be able to make any inference. If some trusted person has perceived regular the smoke–fire relation, then that person’s word is also authoritative. The basis of \textit{udāharaṇa }is perception either directly or indirectly.

\textbf{\textit{Parīkṣā} of 3D Philology}

The third step has its basis in \textit{pratyakṣa–pramāṇa} (or \textit{śabda}–\textit{pramāṇa} – here, the Sanskrit texts)and this is a vital step regarding Pollockian philology. Since he claims to be speaking for the Indian tradition, the pertinent question would be whether his theories are based on Sanskrit texts (perception would not be possible as the living tradition is denied). If the\textit{ udāharaṇa }cited is from Western sources, then it can be concluded that the basis of interpreting Indian tradition is a non–traditional source.


\subsection*{4.5.4 Upanaya\supskpt{\protect\footnote{[ 30 ] \textit{Upanaya – Nyāya–sūtra} 1.1.38

\textit{udāharaṇāpekṣaḥ tathety upasaṁhāro na tatheti vā sādhyasyopanayaḥ |}

When the \textit{sādhya }is characterized as similar to or dissimilar to (the\textit{ udāharaṇa}) and is determined by the\textit{ udāharaṇa }cited, then it is termed \textit{upanaya}.}}}

(application) – \textit{Upamāna} (analogy) is the basis for \textit{upanaya}. The statement that hill has fire (because of smoke) is similar to the \textit{udāharaṇa} which states that smoke is invariably related to fire.

\textbf{\textit{Parīkṣā} of 3D Philology}

\textit{Upanaya} is the fourth step and connects the thesis statement as being similar or dissimilar with the \textit{udāharaṇa} and has \textit{upamāna} (analogy) as its basis. Pollock uses the principle of analogy to Western models to establish all his theories. A careful study of his papers will show that in \textit{upanaya}, a Western model (\textit{udāharaṇa}) is connected to some aspect of Indian tradition (\textit{sādhya}).


\subsection*{4.5.5 Nigamana\supskpt{\protect\footnote{[ 31 ] \textit{Nigamana} – \textit{Nyāya–sūtra} 1.1.39

\textit{hetvapadeśāt pratijñāyāḥ punarvacanaṁ nigamanam |}

\textit{Nigamana} is the restating of the \textit{pratijñā} (thesis statement) along with the \textit{hetu} (cause).}}}

(conclusion) – Thus, because smoke has been perceived, we infer fire on the hill which is from our (or some trusted person’s) experience of directly perceiving smoke and fire invariably connected in many other places.

\textbf{\textit{Parīkṣā} of 3D Philology}

This final step is the conclusion, and it entails the restating of the thesis statement along with the reason. Thus, Pollock’s conclusion would always be derived from the fourth step of analogy (to Western models), and his claim of speaking for the Indian tradition would be flawed.


\subsection*{4.5.6 Summary}

All the five \textit{avayava}–s of a \textit{Nyāya} or inference have been defined and explained in this part of the \textit{prabandha}. To summarize:

\item \textit{Pratijñā} (involving the \textit{pramāṇa }viz.\textit{ āgama}/\textit{śabda}) – The statement of the \textit{sādhya }or the object that needs to be established.

 \item \textit{Hetu} (involving the \textit{pramāṇa }viz.\textit{ anumāna}) – can be similar or dissimilar to step 3.

 \item \textit{Udāharaṇa} (involving the \textit{pramāṇa }viz.\textit{ pratyakṣa}) – known from our own experience or a textual source.

 \item \textit{Upanaya} (involving the \textit{pramāṇa }viz.\textit{ upamāna}) – stating that step 1 is similar or dissimilar to step 3.

 \item \textit{Nigamana} (involving all the four \textit{pramāṇa–}s) – concluding by restating the four steps.

In Pollockian philology, \textit{hetu} (inference) will always be dissimilar to \textit{udāharaṇa,} and thus evidence shown for substantiating the thesis statement will always be fallacious.


\section*{4.6 A Chronological Parīkṣā of Pollock’s Works}

In this section, several works of Pollock are examined chronologically along the lines of the five \textit{avayava}–s. The early papers (pre–1985) on the\textit{Rāmāyaṇa} belong to the pre–Saidian period where Western models are readily applied to Sanskrit texts, and this aspect has been illustrated earlier. Edward Said’s \textit{Orientalism }had a devastating effect on humanities, and this forces Pollock to “exhume” Sanskrit texts for finding deeply embedded Western concepts on culture and power. This drastic shift in Pollock’s approach occurs in 1985, and thus we begin our analysis from this point.Our analysis in this section will show clearly that Pollock is “exhuming” such power structures by an application of Western theories; and the three–dimensional philology is in reality a single–dimensional one, with no effort whatsoever to represent the Indian tradition.

\subsection*{4.6.1 The Theory of Practice and the Practice of Theory in Indian Intellectual History (1985c)}

This 1985 paper is the first one in which, responding to Edward Said’s criticism, Pollock claims to be representing the Indian view point or “...how such things were understood in traditional India.”(Pollock 1985c: 511) All the later papers would have a similar claim, but intentions become clear when it is stated that he is “...interested primarily in exhuming a structure of signification”(Pollock 1985c: 501) from Sanskrit texts. The word “exhume” is also seen in the recent works on philology, and thus, Pollock’s aim over the last four decades has been to exhume power structures from Sanskrit texts. The main thesis of this paper of his is to show that \textit{śāstra}–s went from being a descriptive to prescriptive or normative process, thereby leading to the domination and control of Indian society.

\textbf{(a) \textit{Pratijñā }}

We examine in this section this \textit{pratijñā} of Pollock:

\begin{myquote}
In ancient India, however, there were special factors, which we shall examine, that contributed to \textbf{transforming \textit{śāstra} into a rigorously normative code}, enabling it to speak in an injunctive mood, with the authority appropriate to vedic \textit{vidhis}. (1985c: 504)...the transformation of śāstras from \textbf{descriptive catalogue to prescriptive system} set in. (1985c: 504)...I have suggested was a development from \textbf{descriptive catalogue to prescriptive plan}. (1985c: 511)
\end{myquote}

The “thesis statement that needs to be proved” is that this transformation actually took place and was recorded in the Indian tradition. What then were these “special factors”?

\textbf{(b) \textit{Hetu}}

The definition of \textit{śāstra }as given by various authors such as Kumārila and Patañjali is initially discussed, followed by an inquiry into the\textit{vidyā–sthāna}–s. In the beginning, \textit{śāstra}–s are accepted to be descriptive, but a “mutation is apparent already in the \textit{vedāṅga}–s themselves” as\textit{laukika} material is now included in it. However, no evidence (\textit{śabda}–\textit{pramāṇa}) from Sanskrit texts is shown to record this transformation. For example, are Kumārila and Patañjali discussing about this transformation? Thus, no \textit{hetu} (reason) is shown from Sanskrit texts to support this supposed transformation.

\textbf{(c) \textit{Udāharaṇa}}

We are expecting something from the Sanskrit texts (\textit{śabda–pramāṇa}) to substantiate the thesis, but no such \textit{pramāṇa} is shown. So, where then did Pollock get this thought?

\begin{myquote}
For here, on a scale probably unparalleled in the pre–modern world, we find a thorough transformation–adopting now Geertz's well–known dichotomy–of "\textbf{models of}" human activity into "\textbf{models for}," whereby texts that initially had shaped themselves to reality so as to make it "graspable," end by asserting the authority to shape reality to themselves. (1985c: 504)
\end{myquote}

Thus, the idea of transformation from descriptive (“models of”) to normative (“models for”) is from C. Geertz’s \textit{The Interpretation of Cultures}. Geertz’s theory is itself based on other theories, and thus they would have no basis in valid\textit{ pramāṇa}–s.

\textbf{(d) \textit{Upanaya}}

Such a transformation (as stated in the \textit{udāharaṇa} above) has occurred in the \textit{śāstra}–s also. The transformation hypothesized in the \textit{pratijñā}is similar to the one in the\textit{ udāharaṇa}.

\textbf{(e) \textit{Nigamana}}

Owing to the reason stated in \textit{hetu}, \textit{śāstra}–s have taken a prescriptive form from initially being descriptive.

\textbf{(f) \textit{Nirṇaya}}

The \textit{hetu} cited is dissimilar to the\textit{ udāharaṇa, }and thus the evidence shown is incorrect or non–existent. No śāstric text or even their commentaries discuss anything about this transformation. Thus, to establish this transformation, a Western theorist’s model is accepted as \textit{śabda–pramāṇa}. Even if we accept this, using Geertz’s work as \textit{śabda–pramāṇa} is not valid as his own theories (which are based on more theories) have not been established and thus remain a mere theory. The application of a theory clearly shows that the three–dimension philology is non–existent, and we only have a one–dimensional Pollockian view in this paper.

Moreover, if such a transformation did occur, then it should be asked as to when it happened. Did it happen instantaneously or over a period of time? In this paper, Pollock is careful not to mention any dates even in footnotes, and this should be contrasted with later works (such as the \textit{Language of the Gods}) where dates are vital to his theories. One should also note that this \textit{prabandha} uses the framework of \textit{Nyāya–śāstra }to analyse the works of the Western theorist. This is because \textit{Nyāya }is based on the observation of the world (descriptive) and this \textit{prabandha} documents how even Pollock naturally uses the \textit{pramāṇa}–s (incorrectly, though) to establish his own theories. This would not be possible if \textit{śāstra}–s had become prescriptive, as Pollock is independently writing his works.

This theory of the transformation of \textit{śāstra }from descriptive to prescriptive is applied to several later papers\supskpt{\footnote{[ 32 ] “The classical Indian \textit{śāstra }is a genre that supplanted the wholly descriptive texts of the earlier period (exemplified by the c.600–300 B.C.), and that gave way, in some measure, to a new descriptive–analytical scientific discourse…” (Pollock 1989c:302)}}, and thus the analysis above will be equally applicable to them.


\subsection*{4.6.2 Discourse of Power}

The second thesis statement in the above–mentioned paper is discussed here.

\textbf{(a) \textit{Pratijñā }}

\begin{myquote}
The theoretical discourse of \textit{śāstra} becomes in essence a practical discourse of power. (1985c: 516)
\end{myquote}

It is said that in the Indian tradition, practice follows theory and thus composing a\textit{ śāstra }would imply exercising political power and domination.

\textbf{(b) \textit{Hetu}}

Initially, the “mythogical self–understanding” or how the \textit{śāstra}–s came into existence is discussed, by citing texts such as \textit{Dharmaśāstra}–s, the \textit{Mahābhārata}, \textit{Kāmasūtra}, \textit{Kāvya–mīmāṁsā}, \textit{Matsya–purāṇa}, \textit{Caraka–saṁhitā}, etc. It is then stated that the transmission of a \textit{śāstra }can happen only from God, with \textit{Sūrya–siddhānta} being cited as the example. Here, we need to show that there is at least one place in the\textit{ śāstra}–s that discusses the connection between \textit{śāstra} and political discourse before we can infer and extend it to all cases. Again, we are looking for something within the Sanskrit tradition that would substantiate the thesis statement. Obviously, no such discussion is found in \textit{śāstra}–s such as \textit{Nyāya}, \textit{Vyākaraṇa}, \textit{Alaṅkāra}, etc.

Thus, this idea cannot be from traditional sources.

\textbf{(c) \textit{Udāharaṇa}}

The following paragraph provides Pollock with the necessary framework for interpreting the \textit{śāstra}–s.

\begin{myquote}
First, all contradiction between the model of cultural knowledge and actual cultural change is thereby at once \textbf{transmuted and denied}; creation is really re–creation, as the future is, in a sense, the past. Second, the living, social, historical, contingent tradition is naturalized, becoming as much a part of the order of things as the laws of nature themselves: Just as the social, historical phenomenon of language is viewed by Mīmāṁsā as natural and eternal, so the social dimension and \textbf{historicality of all cultural practices are eliminated} in the shastric paradigm. And finally, through such \textbf{denial of contradiction} and reification of tradition, the sectional interests of pre–modern India are universalized and valorized. (1985c: 516)
\end{myquote}

This passage of his discusses the framework that leads Pollock to conclude that\textit{ śāstra}–s are a practical discourse of power. Denial of the contradiction between theory (model) and practice, the naturalization of the śāstric tradition, the elimination of the history of cultural practices, etc., are topics that are not mentioned in the śāstrictexts. In the \textit{hetu} (reason), several \textit{śāstra}–s are quoted, but one should carefully note that the issues mentioned in this passage are not discussed in them. Since the tradition discusses no such topic, what is the source of these ideas? We are directed to the footnote (1985c: 516ff): “For the theoretical and terminological framework used here, see [Anthony] Giddens, \textit{Central Problems}, pp. 193–7”. Pollock borrows the idea of another theorist (a British sociologist) and directly applies it to the Indian intellectual traditions.

\textbf{(d) \textit{Upanaya}}

In Indian tradition also, composing a \textit{śāstra }meant exercising power (as stated in the \textit{udāharaṇa} above).

\textbf{(e) \textit{Nigamana}}

Therefore, because of the reasons stated in the \textit{hetu}, \textit{śāstra}–s in essence become a practical discourse of power.

\textbf{(f) \textit{Nirṇaya}}

The second important thesis of the 1985 paper, which logically follows the first thesis, is that even composing a \textit{śāstra} should be equated to an act of power and domination. Again, no evidence is shown from Sanskrit texts to establish that a “discourse of \textit{śāstra} becomes in essence a practical discourse of power.” The same can be said for cultural change being denied and historicity of all cultural practices being eliminated. Pollock is again using a theorist as \textit{śabda–pramāṇa} to establish this thesis.Thus, there is no three–dimensional philology seen here, only a single dimension personal \textit{interpretation}. This paper is perhaps the most important one in Pollock’s intellectual history and lays the foundation for later works. One should note how the paper begins:

\begin{myquote}
Students of Daniel Ingalls learned among other essential lessons how important it is to take account of \textbf{traditional categories and concepts} when attempting to understand the cultural achievements of ancient India… (1985c: 499).
\end{myquote}

Traditional viewpoint is never represented by Pollock and the only lesson that he seems to have learnt well is \textit{a priori} application of Western theory in understanding the Indian tradition.


\subsection*{4.6.3 Mīmāṁsā and the Problem of History in Traditional India (1989a)}

Ahistoricity had been hinted at in the 1985 paper on Indian intellectual traditions, and this one attempts to establish it. The supposed lack of history in India is a problem that has been plaguing Indologists and historians for a long time which is mentioned in the introduction of this paper.

\textbf{(a) \textit{Pratijñā }}

\begin{myquote}
\textbf{History}, consequently, seems not so much to be unknown in Sanskritic India as to be \textbf{denied}. (1989a: 603) …\textbf{History}, one might thus conclude, is not simply absent from or unknown to Sanskritic India; rather it \textbf{is denied} in favour of a model of "truth" that accorded history no epistemological value or social significance. (1989a: 610)
\end{myquote}

Pollock summarizes the views of various Indologists on historicity, disagrees with them, and puts forth his own thesis: History is denied in Indian Sanskrit tradition.

\textbf{(b) \textit{Hetu}}

\begin{myquote}
...a \textbf{confrontation with history} on the part of Mīmāṁsā, and the resulting limiting conditions on historiography imposed by the valuation of knowledge in general that Mīmāṁsā, the dominant orthodox discourse of traditional India, articulated. (1989a: 604) When the dominant hermeneutic of the Vedas \textbf{eliminated the possibility of historical referentiality}, any text seeking recognition of its truth claims – any text seeking to participate in brahmanical discourse at all – was required to exclude precisely this referential sphere. Discursive texts that came to be composed under the sign of the Veda \textbf{eliminated historical} referentiality and with it all possibility of historiography. As for the \textit{itihāsa} portions of Vedic literature and such works as the \textit{Mahābhārata} or \textit{Rāmāyaṇa}, these came to be interpreted in ways that \textbf{ignore or occlude their constitutive historicality}. (1989a: 604)
\end{myquote}

\textit{Mīmāṁsāśāstra} has a vast literature beginning with Jaimini’s \textit{Mīmāṁsā–sūtra} in twelve chapters containing more than two thousand \textit{sūtra}–s. The commentaries and independent works of \textit{Mīmāṁsā} in published form and unpublished manuscripts are innumerable. Śabarasvāmin’s \textit{Mīmāṁsā–bhāṣya} is the most important commentary on the \textit{Mīmāṁsā–sūtra}. In this vast literature, what we require is at least one \textit{sutra }(or even a single \textit{vākya}) that discusses the suppression of history. Since \textit{Mīmāṁsā} actually had a “confrontation with history,” then surely there should be a statement somewhere saying that history should be suppressed and anyone wanting to participate in the “brahmanical discourse” should also agree to suppress history. According to Pollock, “in a society that produced millions and millions of texts” (2001f: 441), where thousands of authors participated over a period of several thousand years, unanimity was achieved not only to suppress history, but to also make no mention of it in their own works! Going further, even the commentators of the \textit{Mahābhārata} and the \textit{Rāmāyaṇa }decided to participate in this grand project of suppression and thus we find no historical references in their works. Pollock discusses several Sanskrit works, but none of them have any reference to this act of suppression. If the idea of suppression is not found in the\textit{ Mīmāṁsā} texts, then tradition cannot be the source of this idea and it has to have an external source.

\textbf{(c) \textit{Udāharaṇa}}

Towards the end of the paper, Pollock himself gives the source of the idea of “suppression of history,” but we find it “suppressed” in the footnotes.

\begin{myquote}
The intractable antinomy of process and system is well described by Andrew Feenberg, \textit{Lukcacs, Marx and the Sources of Critical Theory}…My basic conclusion about the denial of history is at odds with the thesis of Claude Lefort (“Outline of the Genesis of Ideology in Modern Societies”) …which seeks to restrict such a denial to modern society. \textbf{My work has been much influenced}, however, by his characterization of ideology as a "sequence of representations which have the function of re–establishing the dimension of \textbf{'society without history'}…" (1989a: 610fn.)
\end{myquote}

Two texts mentioned in the above passage contain discussion about societies without history and are used as \textit{śabda–pramāṇa}–s. From the first work, it is concluded that Indian society privileges system over process: “the structure of the social order over the creative role of man in history” which leads to the denial of history. Using the second work, Pollock expands the denial of history to include traditional societies like India.

\textbf{(d) \textit{Upanaya}}

Therefore, the \textit{sādhya} (the denial of history in India) being similar to the \textit{udāharaṇa} (denial of history in certain societies according to theorists) cited leads us the conclude that history was denied in India also.

\textbf{(e) \textit{Nigamana}}

Therefore, history was denied in the Indian tradition because of \textit{Mīmāṁsāśāstra} since the thesis statement is similar to the \textit{udāharaṇa} (denial of history in certain societies according to theorists) cited.

\textbf{(f) \textit{Nirṇaya}}

This, perhaps, is the weakest paper of Pollock. No \textit{pramāṇa} is shown to prove how the whole tradition decided to participate in this activity of suppressing history. Even if this position is accepted, it should be asked why history was lacking in Buddhist and Jaina texts where the validity of the Vedas was denied. Vernacular literatures also were independent productions and one would have to ask as to why history was suppressed in them also. In any case, there is certainly no effort to even understand the Indian traditional view point. It should be noted that it is clearly stated that “my work has been much influenced…” by Western theorists. Thus, there can be no possibility of a three–dimensional philology and the result is only a personalist \textit{interpretation} and \textit{a priori} application of unproven Western models.

Lack of history in India is an important theme that reappears in later works and this would lead Pollock to conclude that no tradition (including Indian) can have a complete understanding of itself, and thus it is for him to “exhume” this history which has been denied or suppressed. It should also be noted that any later paper that refers to ahistoricity (the suppression of history) will have no basis in the Sanskrit texts themselves.


\subsection*{4.6.4 The Revelation of Tradition (2011g)}

In the subsequent papers on Indian intellectual traditions, \textit{theories from the previous papers are assumed to be “established”} even though they have no basis in Sanskrit texts. This paper aptly titled \textit{The Revelation of Tradition: śruti, smṛti and Discourse of Power} begins by stating that the previous papers have established the denial of history in \textit{śāstra}. This paper discusses the meaning of \textit{śruti} and \textit{smṛti}. In the beginning, \textit{śruti} was considered stronger than \textit{smṛti}, but by the time of Kumārila, they had both gained equal strength.

\textbf{(a) \textit{Pratijñā}}

\begin{myquote}
In some recent papers that consider the nature and role of \textit{śāstra }viewed as a genre, the character of the rules it articulates, and the \textbf{denial of history} its worldview entails, I have tried to clarify some of the ways in which social–cultural practices came to legitimated... (2011g: 41)
\end{myquote}

The thesis statement in this paper also is the denial of history in Indian \textit{śāstra}–s. The \textit{hetu} (reason) given in the previous paper was that \textit{Mīmāṁsā śāstra} eliminated references to history. A different \textit{hetu} is given in this paper.

\textbf{(b) \textit{Hetu}}

\begin{myquote}
When tradition and revelation are forced into convergence; when memory no longer bears the record of human achievement and ‘tradition’ no longer transmits the heritage of historical past, the understanding of culture and society...becomes impossible. (2011g: 58)
\end{myquote}

To substantiate ahistoricity, the reason provided is that \textit{śruti }(revelation) and \textit{smṛti} (tradition) were “forced into convergence.” In \textit{śāstra}–s, \textit{śruti} has greater authority than \textit{smṛti}. Pollock attempts to show that in course of time, \textit{smṛti }was forced to have the same value as that of\textit{ śruti}. No \textit{śabda}–\textit{pramāṇa} (Sanskrit texts) is shown to prove how this convergence would lead to the denial of history. For example, Kumārila, the author of \textit{Śloka–vārtika}, \textit{Tantra–vārtika} and \textit{Ṭupṭīkā}, should be saying something to the extent that once \textit{śruti} and \textit{smṛti }converge and have equal value, then denial of history is possible. However, Kumārila has not written anything on the denial of history in his voluminous works. Then, it is obvious that Pollock did not get the idea that the convergence of \textit{śruti} and \textit{smṛti} leads to the denial of history from the within Indian tradition.

\textbf{(c) \textit{Udāharaṇa}}

In the previous paper, two books were included as\textit{ śabda}–\textit{pramāṇa}–s for establishing ahistoricity. In this paper, Pollock cites one more source (as \textit{śabda}–\textit{pramāṇa}) which is the basis of this paper.

\begin{myquote}
The fact that these are sectional interests, and the legitimation by nature emerges from the competition and conflict over legitimacy discloses for us the dark face of the ‘revelation of tradition’. (2011g: 57)
\end{myquote}

Once again, the footnote leads us to P. \textit{Bourdieu’s Outline of a Theory of Practice} which provides the necessary framework for interpreting the relation between \textit{śruti }(revelation) and \textit{smṛti} (tradition). The cryptic passage above explains how a system (\textit{śruti}) interacts with human agency (\textit{smṛti}) and how they become legitimized (attain value). Pollock readily applies this theory to the \textit{Mīmāṁsā} tradition and from this infers that history has been denied: “…in the elite discourse of traditional India, there exists no cultural memory –\textit{ smṛti }– separate from the memory of the eternally given” (2011g: 57). The “dark face of the revelation of tradition” means all the negative aspects such as caste system, untouchability, etc.

\textbf{(d) \textit{Upanaya}}

Therefore, the \textit{sādhya} being similar to the \textit{udāharaṇa} cited, it is clear that there is denial of history in the Indian tradition also.

\textbf{(e) \textit{Nigamana}}

Thus, by the reasons stated (in the \textit{hetu}), history has not been transmitted by the tradition.

\textbf{(f) \textit{Nirṇaya}}

In this paper, ahistoricity is substantiated by interpreting the relation between \textit{śruti} and \textit{smṛti}, and this interpretation is based on how systems interact with human agency (referred to as practice theory). This amounts to \textit{a prio}ri application of theory and finding relevant data (\textit{Mīmāṁsā} texts) to fit it. We also find a direct criticism of \textit{Mīmāṁsā}: “…the reasons \textit{Mīmāṁsā} gives, and argues out with stunning acuity, are bad ones, that its logic of tradition is finally illogical...” (2011g: 57). If one were trying to represent tradition (as the three–dimensional philology claims), then there would be no reason to call \textit{Mīmāṁsā} arguments as illogical.

Thus, the method in this paper amounts to forcefully fitting a Western model (practice theory) with Indian data. Pollock is again interpreting Indian tradition with non–traditional sources and thus the three–dimensional philology is non–existent and what exists is a one–dimensional “forced” view of Pollock.

A revised version of this paper was published in 2005 (the earlier version was in 1997), and from this we can conclude that the three–dimensional philology is absent at least till 2005. These papers on the Indian śāstric traditions are important as they lay the foundation, and the theories “established” here are used in later works such as \textit{The Language of the Gods}. Therefore, all the later works will necessarily be based on Western theories and never on the Indian tradition. If a later work bases itself on any of these earlier papers, then it will necessarily mean the impossibility of the three–dimensional philology.

The (self–contradictory) nature of Pollockian \textit{interpretation} can be illustrated by one example. It was concluded in these papers that history was suppressed by \textit{Mīmāṁsā} and that everyone who participated in this tradition had to accept this paradigm. Melpputtūr Nārāyaṇa Bhaṭṭatīri was an important scholar of \textit{Mīmāṁsā} whose work on grammar is interpreted as restoring history to Sanskrit!

\begin{myquote}
Among the works of Melpputtūr Nārāyaṇa Bhaṭṭatīri (d. c. 1660), the most remarkable intellectual of seventeenth–century Kerala, is a small treatise, today almost wholly forgotten, called \textit{Apāṇinīyaprāmāṇyasādhana}...The upshot of his arguments goes beyond mere supplementation and is in fact radical, since what he is actually doing, however tacitly, is \textbf{restoring to Sanskrit at once its historicity} and its humanity. (2015g: 123)
\end{myquote}

Nārāyaṇa Bhaṭṭatīri has cowritten a work on\textit{Mīmāṁsā}titled \textit{Mānameyodaya }which, according to Pollock, could only be written if the intent was to suppress history. Since Nārāyaṇa Bhaṭṭatīri participated in the \textit{Mīmāṁsā}tradition, he must have actively suppressed history and all historical references must have been willingly deleted. This “suppressor of history” and thus its humanity, transforms into a champion of history by composing a small work on grammar. Such contradictions are common in Pollockian philology and it matters little that Nārāyaṇa Bhaṭṭatīri himself has nothing to say about history and humanity in his works.


\subsection*{4.6.5 Deep Orientalism? (1993a)}

There are two parts to this paper titled “”Deep Orientalism? Notes on Sanskrit and Power Beyond the Raj”. The second part dealing with \textit{Kṛtya–kalpataru }of Lakṣmīdhara is a continuation of the \textit{śāstra} theme.The \textit{Kṛtya–kalpataru}, a \textit{Dharma–nibandha}, is an extensive text on \textit{Dharmaśāstra} describing all aspects of dhārmic life.

\textbf{(a) \textit{Pratijñā }}

\begin{myquote}
As I try to argue above, \textbf{domination did not enter India with European colonialism}. Quite the contrary, gross asymmetries of power the systematic exclusion from access to \textbf{material and nonmaterial resources} of large sectors of the population appear to have characterized India in particular times and places over the last three millennia and have formed the background against which ideological power, intellectual and spiritual resistance, and many forms of \textbf{physical and psychological violence} crystallized. (1993a: 115)
\end{myquote}

One of the postulates of this paper is that the British had a “Brahmanizing tendency” and adopted the methods of domination previously followed by the Indian ruling elites.

\textbf{(b) \textit{Hetu}}

The \textit{Kṛtyakalpataru }of Lakṣmīdhara is taken as an example to substantiate the thesis statement. Since \textit{Dharma–nibandha} texts never discuss the allocation of “material and nonmaterial resources,” Pollock feels that

\begin{myquote}
Such vast intellectual output surely needs to be theorized in some way...These texts may well in part be "\textbf{models for}'' rather than "\textbf{models of…}” (1993a: 99)
\end{myquote}

The concept of “models for” and “models of” discussed previously [4.6.1] was shown to be substantiated from Western sources. Lakṣmīdhara’s discussionon \textit{sahamaraṇa} (sati) is cited, but even here women are allowed the option of asceticism. The exclusion of \textit{śūdra}–s and women is a much–discussed topic even before the arrival of Pollock. Again, we are specifically looking for some discussion in this \textit{Dharma–nibandha} that supports the thesis statement, but nothing is shown. Thus, the basis of such ideas cannot be Sanskrit texts.

\textbf{(c) \textit{Udāharaṇa}}

\begin{myquote}
\textbf{My thinking} about "power" and its maintenance has been informed by Giddens's analysis of power as the control of both "allocative" (\textbf{material}) resources and "authoritative'' (\textbf{including informational}) resources. When below I focus on \textbf{traditional }\textit{\textbf{vaidika }}\textbf{India}, I have in mind specifically Giddens…1979: 94ff., especially p. 162, where he argues for the primacy of "authority" over "allocation" in precapitalist societies… (1993a: 117fn.)
\end{myquote}

Once again, we are directed to a footnote where Giddens is used as \textit{śabda–pramāṇa} and his theories are directly applied to traditional \textit{vaidika} India. It should be noted the close resemblance of \textit{pratijñā} and \textit{udāharaṇa}. The concept of material and nonmaterial resources (borrowed from a Western Source) is directly applied to Indian tradition.

\textbf{(d) \textit{Upanaya}}

Therefore, the \textit{sādhya} (that needs to be proved) being similar to the \textit{udāharaṇa} cited, the discourse of power and domination is also seen in the \textit{Kṛtya–kalpataru} of Lakṣmīdhara.

\textbf{(e) \textit{Nigamana}}

Thus, as shown in the \textit{hetu}, domination existed in pre–colonial India and texts such as \textit{Kṛtya–kalpataru }of Lakṣmīdhara were meant to dominate and control the society.

\textbf{(f) \textit{Nirṇaya}}

It should be obvious that the three–dimensional philology is impossible here and only a one–dimensional personal interpretation is seen. This paper ends with the comment that “several friends [Indologists] commented on earlier drafts of this essay and attempted to check my excesses.” But Pollock forgets to ask the “marginalized” pundits. What would the marginalized pundits think about using Giddens to understand a text like \textit{Kṛtya–kalpataru}? Pollock has never bothered to ask traditional pundits to check his own excesses.


\subsection*{4.6.6 Literary History, Indian History, World History (1995c)}

This paper is significant because two important theories are first propounded here: beginning of writing in India and vernacularization of languages.

\textbf{(a) \textit{Pratijñā}}

\textit{Kāvya (Rāmāyaṇa)} was written.

\begin{myquote}
Or is it the fact that for the \textbf{first time such material was committed to writing}, a new communicative technology in the subcontinent not far antedating the “first poem” itself? (1995c: 118)
\end{myquote}

\textbf{(b) \textit{Hetu}}

Because\textit{ kāvya} represents something new.

\begin{myquote}
What actually are the traditional Indians and Romans saying when they call \textit{Rāmāyaṇa} the first poem or Livius the first poet? (1995c: 118)
\end{myquote}

\textbf{(c) \textit{Udāharaṇa}}

But the \textit{Rāmāyaṇa }itself clearly states that \textit{ādikāvya} was oral and not written (discussed in section 2.6) which is dismissed as fiction.

\begin{myquote}
On this last \textbf{interpretation}, the image of orality in the prelude of the Bālakāṇḍa becomes not a realist depiction but a nostalgic \textbf{“fiction” of written culture}, and the manuscript of the poem a record of just how difficult and discrepant such literization turned out to be. (1995c: 118)
\end{myquote}

The source of the idea of interpreting orality as “fiction” is hidden in the footnotes.

\begin{myquote}
See Irvine 1994: 431–35 (the story of Caedmon, long taken as a “case–history of an Anglo–Saxon oral singer,” is an exemplum of grammatical culture”). [By representing oral production in written discourse, orality becomes a significatum of textuality, an element of written discourse, or in Franz Bauml’s terms, a \textbf{“fiction” of written culture}. Irvine 1994: 432] (1995c: 139)
\end{myquote}

\textbf{(d) \textit{Upanaya}}

Therefore, the \textit{sādhya} (the \textit{Rāmāyaṇa} was written because of its newness) being similar to the \textit{udāharaṇa} (Irvine) cited, we say that the \textit{Rāmāyaṇa} was also written.

\textbf{(e) \textit{Nigamana}}

Therefore, from the traditional view that \textit{kāvya} represents some new, it is inferred that the \textit{Rāmāyaṇa }was written as it is similar to the Irvines’s (Franz Bauml’s) theory.

\textbf{(f) \textit{Nirṇaya}}

Once again, a Western theory is forced onto the Indian tradition. As discussed earlier, philology is \textit{interpretation} superseding even the \textit{pramāṇa}–s and this aspect is well illustrated in this paper. That \textit{kāvya} was something written is an important theory and is the basis of \textit{The Language of the Gods} (a similar passage to the one above occurs in p.78). Thus, this \textit{pañcāvayava} format can be applied to that thesis statement of The\textit{ Language of the Gods}. This example alone is sufficient to show that the three–dimensional philology is \textit{kalpita} or purposefully constructed and does not follow practice and which always has only one dimension of Pollockian meaning.


\subsection*{4.6.7 Bhoja's Śṛṅgāraprakāśa and the Problem of Rasa (1998a)}

This is the first paper to address the Indian tradition of \textit{Alaṅkāra }(aesthetics) and is also an important one in the intellectual history of Pollock. By now, the Indian śāstric traditions and the \textit{Rāmāyaṇa} have been “established” to have political aspects. Indian aesthetics will also be shown to have a social or political connotation.

\textbf{(a) \textit{Pratijñā}}

\begin{myquote}
The whole point of ŚP [\textit{Śṛṅgāraprakāśa}], for its part, is to discipline and correct the reading of Sanskrit literature, and by creating readers who thereby come to understand what they should or should not do in a peculiar lifeworld constituted by this literature, \textbf{its aims to create politically correct}\textbf{subject and subjectivities}…to help us develop a comprehensive moral imagination. Good readers make good subjects. (1998a: 141)
\end{myquote}

Use of such terms as “lifeworld” and “moral imagination” indicates an attempt to go beyond the texts and to understand the Indian mind.

\textbf{(b) \textit{Hetu}}

Bhoja’s extensive work discusses all topics regarding aesthetics. V.Raghavan’s edition and study are referred to, but Raghavan makes no political connection when discussing the text. Bhoja himself says that the purpose of composing the work is to enable readers to attain the four \textit{puruṣārtha}–s which is the common aim of every\textit{śāstra}. To support the thesis statement, we are expecting some discussion about creating politically correct subjects in Bhoja’s work, but Pollock admits that

\begin{myquote}
Even if Indian thinkers \textbf{do not often thematize the matter} and concentrate instead on the formal or language philosophical–dimensions of the literary…since in the last analysis the correct reading of Sanskrit literature requires a correct understanding of and subscription to a \textbf{larger social theory}. (1998a: 141)
\end{myquote}

Indian thinkers (even in other \textit{śāstra}–s) never seem to thematize the theories that Pollock wants them to and once again there is nothing in Bhoja’s \textit{Śṛṅgāraprakāśa} to directly support the thesis statement. What then is the source of this idea connecting aesthetic to a social theory?

\textbf{(c) \textit{Udāharaṇa}}

\begin{myquote}
…Terry Eagleton argues that “At the \textbf{very root of social relations lies the aesthetic}, source of all human bonding.” This is correct, I think, as far as it goes. But there are particular aesthetics for particular social relations. If we are to understand the \textbf{social world} of premodern Indian, we must understand this aesthetic, and no one is a better guide than Bhoja. (1998a: 141–2)
\end{myquote}

To understand the “social world” (this means going beyond the texts and entering the Indian mind), the aesthetic should be the guide. This idea is from a literary theorist as Bhoja neither discusses any social theory nor about the “political correctness” of the subjects.

\textbf{(d) \textit{Upanaya}}

The \textit{pratijñā} (the connection between aesthetics and the political) is similar to the \textit{udāharaṇa} cited and dissimilar to the \textit{hetu} (as Bhoja’s text has no reference to the political).

\textbf{(e) \textit{Nigamana}}

Thus, because of the reasons stated (\textit{hetu}), we conclude that Bhoja’s primary aim in composing \textit{Śṛṅgāra–prakāśa} is to create “politically correct subjects and subjectivities.”

\textbf{(f) \textit{Nirṇaya}}

Indian aesthetics is the third important topic, after \textit{śāstra}–s and the \textit{Rāmāyaṇa}, that is shown to have some form of political connection by an application of a theory from a Western source. The \textit{hetu} is obviously incorrect here. It is easy then to see the nonexistence of the three–dimensional philology in any of these papers. These three topics will be the building blocks for all subsequent papers and one can easily infer that only a single dimension of Pollockian \textit{interpretation} will be present in the later works also.


\subsection*{4.6.8 The Language of the Gods in the World of Men (2006a)}

An earlier paper titled \textit{Literary History, Indian History, World History }discussed two important theories about written literary cultures and vernacular languages. These two theories would be the basis of Pollock’s major literary work.

\textbf{(a) \textit{Pratijñā }}

Two important historical moments are supposed to have occurred in India and this work documents these two moments.

\begin{myquote}
This book is an attempt to understand two great moments of transformation in culture and power in premodern India. The first occurred around the beginning of the Common Era… [2016a: 1]
\end{myquote}

\textbf{(b) \textit{Hetu}}

\begin{myquote}
…when Sanskrit, long a sacred language restricted to religious practice, was \textbf{reinvented} as a code for literary and political expression. [2016a: 1]
\end{myquote}

Reinvention here means that Sanskrit was initially an oral (sacred) language and then became written in the form of \textit{kāvya}. This is considered a “great moment” which marks the beginning of the political expansion of Sanskrit.

\textbf{(c) \textit{Udāharaṇa}}

Sanskrit texts do not discuss any such historical moment nor is there any mention of transition to writing, so Pollock attempts to provide inscriptional evidence. Inscriptions also have nothing to say about this “great moment” or writing and the suppression of history in India would be cited as the reason for this silence. The \textit{ādikāvya }considers itself as oral and this has been discussed earlier [Chapter 2.6]. Discarding the utterance of Vālmīki as a sentimental “fiction of written culture,” Pollock interprets the \textit{ādikāvya }as written. The question then is, if the transition finds no mention in Sanskrit texts, where did Pollock get this idea? The \textit{upamāna} (comparison) is to “as the phenomenon has been described in the case of the \textit{chansons de geste}\supskpt{\footnote{[ 33 ] The \textit{chansons de geste }are a type of epic poems that are seen in French literature from the beginning of the eleventh century. Pollock also mentions \textit{Beowulf} in this context.}}” [2016a: 78].

\begin{myquote}
The texts of the \textit{\textbf{chansons de geste}} as well...far from being the consequence of a gradual literization of folk culture, represent primary literate products on the part of court literati of the twelfth and thirteenth centuries. Like the prologue of Vālmīki’s\textit{Rāmāyaṇa} (chapter 2.1), these works have been seen as “staging” an oral communicative situation with a comparable, almost wistful retrospection; the character of \textbf{orality in the texts themselves is artificial}...[2016a: 441]
\end{myquote}

\textbf{(d) \textit{Upanaya}}

The \textit{pratijñā }(the occurring of a great moment because of reinvention or the writing of the \textit{Rāmāyaṇa}) is similar to the \textit{udāharaṇa} (\textit{chansons de geste}) cited.

\textbf{(e) \textit{Nigamana}}

Therefore, by the reinvention of Sanskrit (because \textit{kāvya} represents newness and thus writing), it is inferred that the first great moment occurred.

\textbf{(f) \textit{Nirṇaya}}

The first great moment is substantiated from non–traditional sources and is in actuality a “fiction of the Pollockian mind” as it never occurred in Indian history. This unhistorical first moment is then used to support the spread of Sanskrit literary culture across South Asia. The spread of Sanskrit culture across not only southern, but also northern parts of Asia such as China and all the way to Japan is well known. There was also a westward movement of Sanskrit culture into Persia and beyond and in more recent times to Europe (around 1800 C.E). The vitality of Sanskrit can be grasped by the fact that even in recent times, it travelled (without any propaganda) to a distant place (Germany etc.) and infused life into the European notions of language. Wherever Sanskrit culture and language spread, it had a unique capacity to enrich the local traditions without disrupting them. The influence of Sanskrit on various countries and geographical regions is an area that needs to be studied and explained in greater detail. But in all the places it had travelled to over the last two–thousand years, there was no political aspect (in the Pollockian sense) to be seen. In fact, one can say that it was its inherently apolitical nature that made this possible. Unable to grasp the reason for Sanskrit’s mobility, once again, “cultural flow” is theorized from non–Indian sources.

\begin{myquote}
He [\textbf{anthropologist Ulf Hannerz}] rightly points out that \textbf{translocal processes} can add new materials for transformation by local cultures, and that there exists a kind of dialectic between micro and macro processes…the Sanskrit cosmopolis is a crucial case, since it displays to us the vast historical processes wherein the real condition of culture becomes visible, as the constant appropriation and localization of ever new \textbf{translocal flows.} (1996b: 246–47)
\end{myquote}

And so, the first great transformative moment in Indian history is inferred from\textit{ chansons de geste} and the mobility (translocal nature) of Sanskrit culture from an anthropological theory. Using an anthropological framework for explaining the mobility and translocal nature of Sanskrit implies the non–existence of the three–dimensional philology. But the dependence on anthropology is also bemusing given the fact that Pollock seems to have a poor view of it. In a recent paper, he opines that philology has a distinctive research method as opposed to anthropology and approvingly quotes John Comaroff who observes that anthropology is “a discipline that takes to doing work that could as well be done, and be done as well, by journalists.”(Pollock 2015i: 19) Thus, anthropology lacks a methodology and is more akin to journalism, but several important theories\supskpt{\footnote{[ 34 ] \textit{The Language of the Gods} has several references to anthropology and anthropologists: “...analogous to the old anthropological distinction between “little” and “great” traditions...” (2006a: 318). Anthropologist Clifford Geertz appears once again in this work after a period of twenty years (first referred to in the paper on Indian intellectual history – 1985c).}} of Pollock are based on anthropology (or journalism)!


\subsection*{4.6.9 On Grammar and the Political (2006)}

The connection between grammar and the political is another theory discussed in \textit{The Language of the Gods}. That \textit{śāstra}–s became prescriptive from initially being a descriptive model was substantiated from Western sources [see 4.6.1] in the 1985 paper on intellectual history. The prescriptive model is then extended to grammar which the kings would use to extend the Sanskrit cosmopolis.

\textbf{(a) \textit{Pratijñā }}

\begin{myquote}
The categories found in Western representation about the linkage between grammatical and political correctness are hardly alien to the conceptual universe of the Sanskrit cosmopolis. On the contrary, these categories could easily be filled with Indic material. Moreover, as the elite’s adoption of Sanskrit literary culture for the expression of political will shows, rulership and Sanskrit grammaticality and learning were more than merely associated; they were to some degree mutually constitutive. (2016a: 165)
\end{myquote}

Filling Western categories with Indic materials has been the method of philology for the last two–hundred years and is no different with Pollock. The thesis statement makes a connection between grammar and the political (kings).

\textbf{(b) \textit{Hetu}}

\begin{myquote}
\textbf{This is demonstrated by, among other things, the celebration of grammatical learning especially in kings, the royal patronage of such learning, and the competitive zeal among rulers everywhere to encourage grammatical creativity and adorn their courts with scholars who could exemplify it. (2016a: 167)}
\end{myquote}

Tradition considers \textit{Vyākaraṇa} as the foremost of all the \textit{śāstra}–s and there is a vast literature which includes both the Pāṇinian school and non–Pāṇinian traditions. One would expect at least some mention about the political in these texts. Celebration of grammar, dedication of grammars to kings and similar acts are indicative of the importance of traditional studies and the royal support it received, but we are specifically looking for a statement in the grammatical texts that connects it to the political such a this: “language has always been a companion of power.” If a similar statement is there in the “millions and millions” of Sanskrit texts available in printed and manuscript form, then based on it we could infer the connection between grammar and power. Not even a single line is found in the vast grammatical literature and one example would suffice. Bhoja has written a work on grammar titled\textit{Sarasvatī–kaṇṭhābharaṇa} and one on aesthetics called \textit{Śṛṅgāraprakāśa}(the first six chapters are concerned with grammar). Both these texts are good examples of a king who has himself written on the subject of grammar. If there was a grammar–political nexus, Bhoja would have surely mentioned it, but predictably both these works have no such references. Once grammar texts were written, Pollock supposes that kings made a great effort to circulate the text as far as possible. But Bhoja makes no such effort to circulate his grammar and the manuscripts of both these texts are few and found in only some libraries. If there is no such evidence in the Sanskrit texts about this nexus, what is the source of the connection between grammar and the political?

\textbf{(c) \textit{Udāharaṇa}}

\begin{myquote}
If thinkers in the West, on the threshold of modernity, were right to believe that “\textbf{language has always been the companion of power},” as the Castilian grammarian Antonio de Nebrija put in 1492 (2006a: 165) \textbf{Language was indeed the compañera of empire in the West}, and continuously so for almost two millennia before Nebrija declared it to be so. (2006a: 261)
\end{myquote}

Thus, Pollock draws from the European tradition and the work of Nebrija. One has to note that statements such as “language has always been the companion of power” are not seen in any Sanskrit or vernacular\supskpt{\footnote{[ 35 ] Regarding regional languages it is admitted that “No Indian text before modernity, whether political or grammatical, even acknowledges any conjuncture of these two elements.” (2016a: 481)}} grammatical text. It is also not found in a political text like the \textit{Arthaśāstra}. A textual philologist (one who closely works with original texts) would declare that no direct connection existed between grammar and power in Sanskrit tradition based on the lack of textual evidence. But Pollockian philology discards this lack of textual proof and \textit{interpretation} supersedes \textit{śabda–pramāṇa} (Sanskrit grammatical texts) to infer from Western sources the grammar–power nexus latent in Sanskrit texts.

\textbf{(d) \textit{Upanaya}}

Therefore, the \textit{sādhya} (the nexus between grammar and political power) is similar to the \textit{udāharaṇa} (Antonio de Nebrija’s work) cited.

\textbf{(e) \textit{Nigamana}}

Since grammar texts were composed and dedicated to kings, we infer that there was a nexus between grammar and power in Indian tradition, which is similar to how it happened in Europe as documented in Antonio de Nebrija’s work.

\textbf{(f) \textit{Nirṇaya}}

Two examples were given from \textit{TheLanguage of the Gods} to illustrate the nature of Pollockian philology. In both instances, only \textit{a priori} application of theory based on European history was used in evaluating important aspects of Indian tradition and thus the absence of both the first and second planes of the three–dimensional philology. This major literary work was published in 2006 even though many concepts were developed much earlier\supskpt{\footnote{[ 36 ] For example, the transformation of \textit{śāstra}–s from descriptive to prescriptive is discussed in the 1985 paper. The aestheticization of power was developed in the 1993 paper on the \textit{Rāmāyaṇa}.}}. There is no mention of the term three–dimensional philology anywhere, although it is claimed that the Indian traditional view is being presented. Thus, this major work is also based on a single dimensional personalist interpretation of Pollock


\section*{4.7 Summary of the Chronological Parīkṣā}

In this section, nine thesis statements from seven different works covering a period of twenty years (1985–2006) were presented in the format of the five \textit{avayava}–s to ascertain if the Indian tradition was faithfully represented at least along the second dimension. The result of this analysis clearly shows that each work consists of a single dimension of \textit{a priori} application of Western theory. The only conclusion is the complete nonexistence and the impossibility of the three–dimensional philology. The earlier pre–\textit{Orientalism} (= pre–1985) papers (discussed in section 2.6) were also shown to have only a single dimension of personalist \textit{interpretation}. But what claim did Pollock make about the three–dimensional philology?

\begin{myquote}
In the first instance my call for three–dimensional philology, aside from the \textbf{autobiographical reality it has for me}, derives from its \textbf{phenomenological reality}. This is how we do in fact read, whether or not we are fully aware of it and whether or not we try to suppress it. (2016: 24)
\end{myquote}

My own biographical reality has shown that Pollockian philology is only of a single (and always political) dimension. And a \textit{parīkṣā} based on the \textit{pramāṇa}–s has shown the “phenomenological reality” to be un–phenomenological and unreal. And what would Edward Said say about all of this? Pollock refers to Said in a recent paper.

\begin{myquote}
…Said’s \textit{Orientalism} arrived [in 1978]...While I welcomed the connection (or reconnection) of knowledge and power, \textbf{I was wounded} that Said was catching in his net even people like me, who thought of ourselves as \textbf{critical, not comprador, classicists}...that the precolonial past was effectively unknowable since all knowledge about it was entirely mediated by colonialism (2016d: 920)
\end{myquote}

Throughout his career, Pollock is unable to break free from Said’s net (of knowledge and power) and falsely claiming to be “wounded” shows once again that the philological papers are meant to deceive academic administrators in an attempt to save philology. Further, Pollock states that

\begin{myquote}
It may not be very fashionable to say so these days, but the lies and truths of texts must remain a prime object of any future philology. (2015b: 951)
\end{myquote}

Any future philology should be based on valid\textit{ pramāṇa}–s and presenting a paper along the lines of the five \textit{avayava}–s is necessary for testing the “lies and truths of texts.” The truth revealed by such an analysis is that Pollock is always a comprador classicist and never a critical one. In fact, he is not even a classicist, but only a theorist.

What is the three–dimensions–of–philology in theory, is in practice only a single dimension. But what claims were made about the three–dimensional philology?

\begin{myquote}
Plane 1 (historicism) helps us to better comprehend the nature, or natures, of human existence and the radical differences it has shown over time, that is, the vast variety of ways of being human. (2014c: 411)
\end{myquote}

But an analysis of the nine thesis statements has shown that Pollock is unable to comprehend the nature of human existence in India and the only (non–radical) difference is in the use of Western models and theories.

\begin{myquote}
Philology on Plane 2 (traditionism) helps us to \textbf{better understand and to develop patience for the views of others}, and so to expand the possibilities of human solidarity (this is the great value of reading a deep and distant past like India’s, since it is precisely the presence of a long and very unfamiliar history of reading and interpretation that lets \textbf{us exercise so effectively the virtues of the quest for understanding and solidarity})... (2014c: 411)
\end{myquote}

This passage illustrates the vast difference between Pollockian theory and practice. My own patient reading of his works has led me to the understanding that Pollock has not been able to free himself from theorizing. One has to admit that Pollock patiently collects the necessary Indian data to fit his Western models.

\begin{myquote}
Philology on Plane 3 (presentism) helps us to come to understand our own historicity and our relationship to all earlier historical interpretations, including the originary, and thereby to \textbf{gain a new humility} for the limits of our capacity to know and a new respect for the importance to keep trying. (2014c: 411)
\end{myquote}

This section and the \textit{prabandha} has shown that Pollock’s own historical framework will always be the basis of interpretation of India’s deep and distant past. Moreover, theorizing a three–dimensional philology while practicing a single–dimensional one only shows a lack of humility. We can conclude that the Indian past has so far remained unknown to Pollock and in all probability, will remain so in the future.

