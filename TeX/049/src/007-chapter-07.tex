
\chapter{Subject–wise Mapping of Pollock’s Works}\label{Anubandha2}

Each title is followed by the publication year which can be used to refer to the bibliography for complete details. Some bibliographical entries are not included in this mapping.

\section*{A2.1 A Pollock Reader}

A Pollock Reader would begin with these two works and reading these together shows Edward Said’s influence. The first is theory free while the second is heavily laden with theory.

(1985b) “Daniel Henry Holmes Ingalls”

(2015c) “Rice and Ragi: Remembering URA”


\section*{A2.2 Philological Papers}

(2019b) “Indian Philology: Edition, Interpretation, and Difference”

(2018a) “Small Philology and Large Philology”

(2017b) “Conundrums of Comparison

The Columbia Global Humanities Project”

(2016d) “Areas, Disciplines, and the Goals of Inquiry”

(2016b) “Philology and Freedom”

(2015i) “Liberating Philology”

(2015g) “What was Philology in Sanskrit”

(2015h) “Introduction” in \textit{World Philology}

(2015a) “\textit{Philologia Rediviva”}

(2014c) “Philology in Three Dimensions”

(2009b) “Future Philology: The Fate of a Soft Science in a Hard World”

(2008a) “Towards a Political Philology: D.D. Kosambi and Sanskrit”


\section*{A2.3 The Major Literary Work}

(2006a) \textit{The Language of the Gods in the World of Men}


\section*{A2.4 On Kāvya and Rasa}

(2012d) \textit{A Rasa Reader: Classical Indian Aesthetics} – “Introduction” (2016e)

“\textit{Rasa} after Abhinava”

(2012c) “\textit{Vyakti} and the History of \textit{Rasa”}

(2012a) “From \textit{Rasa} Seen to \textit{Rasa} Heard”

(2010b) “What was Bhaṭṭa Nāyaka Saying?”

(2009a) \textit{Rasamañjarī} and \textit{Rasataraṅginī} of Bhānudatta – “Introduction”

(2005b) “\textit{Ratnaśrījñāna”}

(1998a) “Bhoja's \textit{Śṛṅgāraprakāśa} and the Problem of \textit{Rasa}: A Historical Introduction”


\section*{A2.5 Early Modernity and Indian Intellectual Traditions}

(2008d) “Theory and Method in Indian Intellectual History”

(2008b) “Is there an Indian Intellectual History?”

(2007c) “The Problem of Early Modernity in the Sanskrit Intellectual Tradition”

(2005a) “The Ends of Man at the End of Premodernity”

(2002c) “Introduction: Working Papers on Sanskrit Knowledge–Systems on the Eve of Colonialism”

(2001a) “New Intellectuals in Seventeen Century India”


\section*{A2.6 Kannada}

(2004a) “A New Philology: From Norm–bound Practice to Practice–bound Norm in Kannada Intellectual History”


\section*{A2.7 Cosmopolitan and Vernacular}

(2013b) “Cosmopolitanism, Vernacularism, and Premodernity”

(2004c) “The Transformation of Culture–Power in Indo–Europe, 1000–1300”

(1998b) “India in the Vernacular Millennium: Literary Culture and Polity, 1000–1500”

(1996b) “The Sanskrit Cosmopolis, A.D. 300–1300: Transculturation, Vernacularization and the Question of Ideology”

(1996a) “Philology, Literature, Translation”

(1995c) “Literary History, Indian History, World History”

(1995b) “Literary History, Religion and Nation in South Asia: Introductory Note”


\section*{A2.8 Praśasti}

(2013a) “\textit{Praśasti} and Its Congeners: A Small Note on a Big Topic”

(1995a) “In Praise of Poets: On the History and Function of \textit{Kavipraśaṁsā”}

(1995d) “Public Poetry in Sanskrit”

(1995e) “Making History: Kalyāṇi, A.D. 1008”


\section*{A2.9 German Indology}

The first part of “Deep Orientalism?” is listed separately due to its importance and it continues with the \textit{Śāstra} theme of power and domination.

(2000b) “Indology, Power, and the Case of Germany”

(1993a) “Deep Orientalism?”


\section*{A2.10 On Śāstra}

“Deep Orientalism?” has two parts and the second is included here as it develops on ideas discussed in the earlier papers on \textit{śāstra}.

(1993a) “Deep Orientalism?” – \textit{Kṛtyakalpataru} of Lakṣmīdhara

(1990a) “From Discourse of Ritual to Discourse of Power in Sanskrit Culture”

(1989c) “Playing by the Rules: \textit{Śāstra} and Sanskrit Literature”

(1989b) “The Idea of \textit{Śāstra} in Indian Tradition”

(1989a) “\textit{Mīmāṁsā} and the Problem of History in Traditional India”

(1988) The Revelation of Tradition: \textit{Śruti}, \textit{Smṛti} and the Sanskrit Discourse of Power \endnote{ Regarding this paper, Pollock writes: “This is a corrected version of an essay originally published in S. Lienhard, I. Piovano 1997 (the essay was submitted to the editors in 1988 and reflects the scholarship up to that date).” (2005b:17)}

(1985c) “The Theory of Practice and Practice of Theory in Indian Intellectual History”


\section*{A2.11 Rāmāyaṇa}

Post–aestheticization of power

(2007b) \textit{Rama's Last Act (Uttararāmacarita) of Bhavabhūti} – “Introduction”

(1995f) “\textit{Rāmāyaṇa} and Public Discourse in Medieval India”

(1993b) “\textit{Rāmāyaṇa} and Political Imagination in India”

Pre–aestheticization of power

(1991b) \textit{The Rāmāyaṇa of Vālmīki: Volume III: Araṇyakāṇḍa} – “Introduction”

(1986b) \textit{The Rāmāyaṇa of Vālmīki: Volume II: Ayodhyākāṇḍa} – “Introduction”

(1985a) “\textit{Rākṣasas} and Others”

(1984b) “The Divine King in the Indian Epic”

(1984a) “The Rāmāyaṇa Text and the Critical Edition”

(1993a) “Some Lexical Problems in Vālmīki Rāmāyaṇa”

(1979a) “Text Critical Observations on Vālmīki Rāmāyaṇa”


\section*{A2.12 Translations}

The preface to \textit{A Rasa Reader} discusses the translation method: “The aim of producing as readable a sourcebook as possible has encouraged me to really try to translate the thought as well as the words of our authors. I therefore put into the translation what is implied in the text...”(Pollock 2016e:xviii) What the Indians thought “implicitly” would be the same as what the West thought explicitly and what our three–dimensional philologist has been thinking all his life: “But let me cut into the problem via the political–cultural dimension that has been my focus throughout.”(Pollock 1996b:239)

Post–aestheticization of power

(2016a) \textit{A Rasa Reader: Classical Indian Aesthetics} – Translation

(2009a) \textit{Rasamañjarīand Rasataraṅginī of Bhānudatta}

(2007b) \textit{Rama's Last Act (Uttararāmacarita) of Bhavabhūti}

(2004d) “The Meaning of \textit{Dharma} and the Relationship of the Two \textit{Mīmāṁsā}–s”

(2000a) \textit{Kālidāsa: Raghuvaṁśa} (Translation of Ch.7)

(1998a) “Bhoja's \textit{Śṛṅgāraprakāśa} and the Problem of \textit{Rasa}”: Annotated Translation

Pre–aestheticization of Power

(1991a) \textit{The Rāmāyaṇa of Vālmīki: An Epic of India, Volume III: Araṇyakāṇḍa}

(1986a) \textit{The Rāmāyaṇa of Vālmīki: An Epic of India, Volume II: Ayodhyākāṇḍa }


\section*{A2.13 Book Reviews}

There are several book reviews\endnote{ Some of the shorter reviews are: \textit{The Divyatattva of Raghunandana Bhaṭṭācārya: Ordeals in Classical Hindu Law} (1981), \textit{The Bridge to the Three Holy Cities: The Sāmānya–pragaṭṭaka Nārāyaṇa Bhaṭṭa's Tristhalīsetu} (1985), \textit{Perception, Knowledge, and Disbelief: A Study of Jayarāśi's Scepticism} (1987) and \textit{Translating the Orient: The Reception of Śākuntala in Nineteenth–CenturyEurope} (1991). The publication date of the reviewed book is given in brackets.} but only the four longer ones are included here.

(2012b) “Commentary on de Pee: Epicycles of Cathay”

(2011f) “Indian Philology and India’s Philology”

(2007a) “Pretextures of Time”

(1995g) “\textit{Genres Littéraires En Inde}”


\section*{A2.14 Edited Works}

These three works are edited with an introduction and an article. Edited works are important because Pollockian speculations are expanded by students and colleagues.

(2015h) “Introduction” in \textit{World Philology}

(2015g) “What was Philology in Sanskrit?”

(2011b) “Introduction” in \textit{Forms of Knowledge in Early Modern Asia}

(2011c) “The Languages of Science in Early–Modern India”

(2003b) “Introduction" in \textit{Literary Cultures in History}

(2003c) “Sanskrit Literary Culture from the Inside Out”


\section*{A2.15 Translation of Pollock’s Papers}

(2018d) “Philology and Freedom” – German translation

(2017c) “Introduction to World Philology” – Italian translation

(2015e) “\textit{Kritische Philologie”} – German translation of several\endnote{ Sections of Language of the Gods, Literary Cultures in History, The Cosmopolitan and Vernacular in History, Empire and Imitation, Comparison without Hegemony, Deep Orientalism (German Indology) and Future Philology are translated in this collection.} papers

(2014b) “Critical Philology” – German translation

(2011d) “Crisis in the Classics” – Tamil translation

(2009c, 2009b) “Future Philology?” – German and Russian translation

(2003d) “The Cosmopolitan Vernacular” – Kannada translation

(2002d) “Introduction” to \textit{Literary Cultures in History} – Italian translation

(1987a) “Introduction” to \textit{Ayodhyākāṇḍa} – abridged Spanish translation


\section*{A2.16 Miscellaneous Works}

(2015d) “State of Nature – \textit{Prakriti}”\textit{ –} trans. from the Kannada in collaboration with Ananthamurthy

(2015b) “The Alternative Classicism of Classical India”

(2014d) “Indian Classicity”\endnote{ This paper was presented in The Engelsberg Seminar 2013 held in Sweden. It was presented in the ‘Roads to Athens’ panel and the lecture was titled ‘Classicism: A Greek and Indian Comparison.’ The published paper is titled ‘Indian Classicity.’ See \url{http://axsonjohnsonfoundation.org/eng/project/civilisation}. Accessed 9th Sep., 2019. The revised version is published as ‘The Alternative Classicism of Classical India.’}

(2013a) “What is South Asian Knowledge Good for?”

(2011e) “Sanskrit Studies in the US”

(2011d) “Crisis in the Classics”

(2010a) “Comparison without Hegemony”

(2007e) “We need to find what we are not looking for”

(2008c) “The Real Classical Languages Debate”

(2006b) “Empire and Imitation”

(2001b) “The Death of Sanskrit”

(1990b) “Humanities in South Asian Studies”

(1977a) \textit{Ph. D thesis: Aspects of Versification in Sanskrit Lyric Poetry}

