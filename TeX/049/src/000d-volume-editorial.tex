
\chapter*{Volume Editorial}

“\textit{\underline{The greatest obstacle} to the understanding of the history of science is \underline{our inability to unload our minds of modern views} about the nature of the universe.}”

\begin{flushright}
– \textbf{Herbert Butterfield} (1900–1979)
\end{flushright}

“\textit{Great is the power of \underline{steady misrepresentation} – but the history of science shows how, fortunately, this power \underline{does not long endure.}}”

\begin{flushright}
– \textbf{Charles Darwin} (1809–1882)
\end{flushright}

“\textit{Science knows no “bare facts” at all; the “facts” that enter our knowledge are already viewed in a certain way, and are, therefore, essentially ideational.}”

\begin{flushright}
– \textbf{Paul K. Feyerabend }(1924–1994)
\end{flushright}

“\textit{You gotta say this for the White race – its self–confidence knows no bounds. Who else could go to a small island in the South Pacific where there is no poverty, no crime, no unemployment, no war and no worry – and call it a ‘primitive society’?}”

\begin{flushright}
– \textbf{Dick Gregory }(1932–2017)
\end{flushright}

\textit{atyuccair bhavati hi laghīyasāṁ dhārṣṭyam}

\begin{flushright}
– \textbf{Māgha} (c. 7th century)
\end{flushright}

It gives me pleasure to write a few words by way of introduction to the monograph “\textit{A Parīkṣā of Three–Dimensional Philology”} written by my dear student T. M. Narendran. The work sets out to analyse and examine the Three–Dimensional Philology (3DP) proposed by Prof. Sheldon Pollock (hereafter Pollock), in the light of and the spirit of the \textit{Nyāya–sūtra}–s of Gotama (and the commentary thereon (viz. \textit{Nyāya–bhāṣya} of Vātsyāyana)).

As Narendran has pointed out very well, Pollock has been seriously dwelling on Philology for quite a protracted period, and fairly steadily at that (see the Bibliography at the end of this monograph: See Pollock 1985, 1996a, 2004a, 2009b, 2009c, 2011f, 2014b, 2014c, 2015e, 2015f, 2015h, 2015i, 2016b, 2018b, 2019a and 2019b) – which is to say some 16 articles/books – written over a period of over 20 years. One can only expect that his ideas have progressively crystallised and taken a definite direction and shape. And this has to do with the interpretation of Indic – which is to say, essentially Sanskrit – texts.

In the typical traditional method, our author Dr. Narendran first states the \textit{pūrva–pakṣa} – which involves a faithful presentation of what the opponent claims; and then he states what the counter–arguments are, and finally what the net outcome (\textit{siddhānta}) would be. No work of Pollock, or for that matter, the work of any other Western/Eastern Indologist, has been subject to this kind of an indigenous/ traditionalist treatment. Hence the value and importance of Narendran’s work.

Of no small significance is the question: What is valid knowledge?; and how do we get to know what we know? This is precisely the concern of Nyāya–śāstra, the discipline of logic that examines things (\textit{artha–parīkṣaṇa}) through proper means of knowledge (\textit{pramāṇa}). Objects that can be scrutinised this way range from mundane things in the world right up to the concept of \textit{niśśreyasa} (another term for the highest good viz. \textit{mokṣa}). The ideas of Pollock make categorical claims to laying a theoretical foundation for understanding and interpreting Sanskrit texts.

The present work is planned adopting the typical procedure of a Nyāya text. We have three main chapters (Chh. 2, 3, \& 4) (preceded by an Introduction (Ch. 1), and succeeded by a Conclusion (Ch. 6)) The main chapters are respectively called \textit{uddeśa} (a mention of relevant terminology), \textit{lakṣaṇa} (framing definitions), and \textit{parīkṣā} (an examination of the issues). Expectably, Ch.s 2 \& 3 are relatively small (a little above 20 pp. each), and the \textit{parīkṣā} chapter, rather long (50 pp.). The Introductory chapter rightaway introduces the Nyāya scheme of \textit{tattva–jñāna} (ascertaining a truth), and \textit{nirṇaya} (arriving at a conclusion). The monograph concludes with two Appendices and a Bibliography – all of which cover the writings of Pollock from different perspectives.

It is only quite in order that an overview of the work be given here. Nyāya–śāstra is a \textit{pramāṇa–śāstra},a discipline of the valid means for obtaining knowledge. What needs to be known, and with what proper means – constitute its domain. As these are basic questions of every thinking man, Nyāya–śāstra will remain perpetually relevant. That the claims of Pollockian Philology are tall, and do not stand the acid–test of the Nyāya approach, will become clear towards the end of this book of Narendran..

The steps through which the issues have been handled are clarified by the author himself:

\item citation of the relevant \textit{Nyāya–sūtra} (and its \textit{bhāṣya}); followed by

 \item its translation and

 \item explanation of the same as needed;

 \item faithful statement of some key ideas of the 3DP (Three–Dimensional Philology) propounded by Pollock (and thus serving as the \textit{pūrva–pakṣa}) (the \textit{prima facie} view); and finally,

 \item the \textit{siddhānta}, final assessment of Pollock’s stand, after the examination of the \textit{pūrva–pakṣa –} leading to a \textit{nirṇaya} (conclusion) about their validity.

Pollock’s 3DP speaks of the Historical, the Traditional, and the Presentist as the three planes of Philology, which provide three perspectives in understanding any Sanskrit text. Our author examines whether there is concordance between this theory of Pollock and his own practice; and whether tracking the intellectual history of Pollock’s Philology indicates any internal corroborations/contradictions. The relationship between \textit{śāstra} and \textit{prayoga} in the East is one of diametrical opposition, claims Pollock, to what is usually found in the West. The obsession of the West that Indian \textit{śāstra}–s are not descriptive in nature but are prescriptive has landed itself in a fair degree of confusion. Misled themselves, Westerners mislead others with a finish, and with hardly any trepidation or compunction (which is not to say they do not make honest and hard attempts to mislead others even when they are not misled themselves).

Our author errs, in a way, when he says no grammar existed in the West that was even remotely comparable to Pāṇini’s \textit{Aṣṭādhyāyī} around the time Sanskrit was introduced to the West by Sir William Jones; the point being there is no such comparable work even to this day! The very birth of philology is due to the West discovering Sanskrit. “Philo–logy” literally means love of learning, but it originally meant analysing a text based on linguistic principles (which amounts more or less to say, the principles one learnt from Sanskrit). As Pollock himself admits, Franz Bopp, W. D. Whitney, F de Saussure, Maurice Bloomfield, and Noam Chomsky, who developed the historical, structural, and transformational linguistics, all knew Sanskrit, and what is more, their very inventions in linguistics owe much to Sanskrit. In strict adherence to his vocation of equivocation, Pollock portrays Sanskrit – which has had this degree of vitality – as one always on the threshold of mortality. Philology influenced disciplines ranging from anthropology to zoology.

Western languages regularly display the scourge of rapid semantic change, and “philology” was no exception. For some time it meant interpretation of texts, but later came to mean such interpretation as is based on grammar, textual criticism, and historical analysis. One characteristic of philology has however remained unchanged – viz. the imprecision of rules that could define its principles of interpretation. Even while identifying philology and interpretation, Pollock says that interpretation precedes and informs all other aspects of philology, including grammar and criticism. Narendran takes us through a conducted tour of the history of “philology” – through key figures such as Schlegel, Boeckh, Nietzsche, Auerbach, Gadamer, Paul de Man Dilthey and Edward Said.

“Interpretations can coexist and compete”, and “Interpretations can only be of a personal nature”: with such dicta or maxims, there can be no sound theory, or a proper discipline, of interpretation. Theory and interpretation are antithetical in the sense that theory is universal, and interpretation here, personal. Ultimately the 3DP is a 1DP – the one–dimensional philology (or better, one–dimensional–Pollockism). A dying humanistic science, philology is sought to be revived as Zukunfts–philologie (ZP) (Future Philology), and is sought to be made the \textit{basic} science of the humanities though lacking vitality and any sense of direction.

Narendran makes an important remark that “using Western political models to interpret Indian intellectual traditions is the basis of all the works beginning with the [Pollock’s] 1985 paper on \textit{śāstra}”. The basic premise of this ZP is that there were parallel and strikingly similar developments of philology in India, China, and the West. There are no limits to such wishful thinkings of Pollock. If science is erected upon rigorous objectivity, philology is founded on flimsy whimsicality. In 2011, Pollock speaks of a global theory of philology, but in 2015, he speaks of it as a discipline of making sense of texts, and as not about theory. Contradictions thus galore in Pollock as usual: apparently the West is practising philology for two centuries (\textit{sans} a basic definition of the field, though) on the one hand; and on the other, it is in India that philology has always been “...the queen of disciplines… the most sophisticated in the ancient world”; and again, philology never existed in India; in his 2009 writing, philology encompassed the three \textit{śāstra}–s (viz. Vyākaraṇa, Alaṅkāra, and Mīmāṁsā), but in the 2011 writing, they are all excluded. Where Pollock’s eyes are set is clearly indicated by his own admission viz. “philology has political projects to achieve”. His recent (2016) pronouncement on philology is also very telling: rather than a “heavyweight philological” theory, all he had to be content with is “lightweight autobiography”. There is indeed no better way to misinterpret Indian tradition than with this bastard science of philology.

Speaking of the Indian framework of interpretation, Narendran notes that the famed fourteen \textit{vidyā–sthāna}–s are also equally \textit{dharma–sthāna}–s, as clearly stated in \textit{Yājñavalkya–smṛti}. The Buddhist and Jaina traditions too toe the line of Hinduism in this regard reckoning however with their own revealed sources.

Blowing hot and cold is a steady wont with Pollock: Indian oral traditions are important, but also unimportant. \textit{Rāmāyaṇa} took a written form in 11th c. CE, and suddenly in the beginning of the first millennium, as it fits his newfound theory. \textit{Rāmāyaṇa} says it was taught orally, but Pollock \textit{knows} the statement is fictional. In 1984, the prelude in the \textit{Rāmāyaṇa} is “mostly authentic” as it is there in the critical edition, but later, it is “unquestionably later than the rest of the work”. Such reversal of opinions by Pollock is traced to the thinking of Pollock as anterior to and posterior to the book \textit{Orientalism} by Edward Said (1975). “Exhume power–structures from the texts” is the new \textit{mantra} of Pollock subsequent to Said.

The data of Vyākaraṇa being \textit{loka–siddha} (=agreed upon in, or derived from, the world) are in a way comparable to the \textit{pramāṇa}–s of Nyāya (which too are \textit{loka–siddha}). Seldom do grammatical texts in Sanskrit discuss political ideas/developments even to the slightest extent; if the political link is then to be “exhumed” therefrom, it can only be out of wafting the magic wand – of Western methods and sources.

Unlike Muslims who cite lines from their religious text as their war cry, did any Hindu, even a single Hindu, ever recite a single verse, a single line from the \textit{Rāmāyaṇa} in the context of the demolition of the disputed structure at Ayodhya? Why then should Pollock link the text with the event?: his article entitled “\textit{Rāmāyaṇa} and Political Imagination” promptly does it. And is there anything in the \textit{Rāmāyaṇa} asking of people to go about and act politically? Again, Pollock is aware of Sanskrit pundits teaching the \textit{śāstra}–s and writing \textit{kāvya}–s toeing the traditional line even to this day; and even \textit{vākyārtha–sabhā}–s are held as they used to be centuries ago; further some 20 million people in America alone practise Yoga, a tradition issuing from the pre–Chiristian era; Āyurveda is penetrating the Western medical establishment on no small a scale: yet Pollock portrays Sanskrit as dead! In both cases, then – of Sanskrit and the \textit{Rāmāyaṇa} – Pollock makes statements that are not validated by \textit{pratyakṣa}.

Indians defending themselves against the onslaughts of the invading hordes of Muslims is peculiarly portrayed by Pollock as an instance of repurposing the \textit{Rāmāyaṇa}, which as per him seeks to portray Muslims as demons and Rāma as the defending divine king. Pollock invokes temple–building, inscriptions, historiographical texts etc as supporting his fancied thesis, but then, by his own admission, “data are devilishly hard to find”. So we must presuppose newer kinds of textual practices and cultural processes that invite wider social inference, he argues. Well, none of the commentaries speak overtly or covertly of Islamic invasion, not a single temple ritual or record or inscription makes the slightest allusion to this; and yet Pollock has the cheek to posit the idea of the repurposing of the \textit{Rāmāyaṇa} against Islam as being \textit{carried out on a large scale around the country} – in spite of the fact that millions of texts/inscriptions are available, none with a shred of evidence. Nor do we find Kamban, Tulsidas and others who wrote in the regional languages repurposing the \textit{Rāmāyaṇa}.

Thus the \textit{anumāna}–s (inferences) of Pollock are fallacious too. In place of his “\textit{Rāmāyaṇa} and Political Imagination in India”, what would represent the true state of affairs is that the \textit{Rāmāyaṇa} is in India, and the political imagination is in the idle workshop namely Pollock’s brain. Pollock’s specialisation in logical fallacies extends to the inference of Sanskrit never being a spoken language which goes against many proofs, one among which is “\textit{yathā loke}” (“as people use it commonly”) occurring in Patañjali’s \textit{Mahābhāṣya} around 200 times. Always suspect a man grandiloquent with respect to his neat theory, yet impoverished in respect of facts: he is only working as per a pre–set agenda. As did he speak of repurposing the \textit{Rāmāyaṇa}, he speaks of the reinventing of Sanskrit. Speaking technically, and as Narendran shows in sufficient detail, every type of \textit{anumāna} has been violated by Pollock.

What applies to \textit{pratyakṣa} and \textit{anumāna} applies to the other two \textit{pramāṇa}–s viz. \textit{upamāna} and \textit{śabda} also. There is not a single \textit{pramāṇa} to which Pollock’s proclamations can squarely subscribe, singly or collectively, or even somewhat align with. Mere verbiage, it must be said in sum, consciously couched in an impenetrable language that Pollock’s is, does not add to the credit or merit of Pollock’s writings.

Narendran notes well the irony that while (the ancient) Vātsyāyana is accommodative of even \textit{mleccha}–s under \textit{āpta}–s – as long as they conform to the definition of an \textit{āpta,} that is– which is impartially the same for all (this modern scholar of pomp and polish), Pollock perceives on the contrary a “pre–form of racism in early India”, and “a biogenetic map of inequality”! This is indeed a premier example – of superciliousness and open racism on the part of a propounder of a fanciful racist theory, and the catholicity of outlook of a Hindu sage ages ago.

Pollock presents the “presentist interpretation of the \textit{Rāmāyaṇa”} – a veritable free–for–all, no–holds barred avenue. Positing an open validity of presentism can empower anyone to make anything out of anything.

There is nothing in the First Dimension, the Historical, the author’s – that could give a clue to this openness to or tendency towards Pollock’s fanciful repurposivism; there is nothing again in the second Dimension, the Traditional, the commentators’ – the dozens of them that we have – that could give a mildest hint of this vulnerability to repurposivism. There is a method in even madness, averred Shakespeare. In his 1986 paper, Vālmīki is pre–Mauryan, and in 2014 paper, Vālmīki is post–Aśokan! One could rather assert that it is Pollock who is an adept at repurposing, for, by pushing \textit{Rāmāyaṇa} into the post–Aśokan, he can impute politicality into the \textit{Rāmāyaṇa} at the very time of its composition. Pollock 2006 speaks of the consensus that emphasises the political–moral over religious nature of the doctrine of \textit{dhamma}; but suddenly Pollock 2014 speaks of the clear impress of Aśoka’s quasi–Buddhist political theology! Pollock delivers the verdict that power takes on a spiritualised dimension with Aśoka. None dare ask proofs for the same. And, typical of the intolerant brigade that brands all others who do not fall in line as intolerant, this presentist insists that those who do not consider the \textit{Rāmāyaṇa} as political are “doing violence to the work”!

What applies to Pollock’s indiscriminate inputs on the \textit{Ramāyaṇa}, apples equally to his dicta on Kālidāsa’s famous play \textit{Śākuntala}, wherein he (as too, a few of his own ilk) sees the “enfeeblement of the heroine who appears so strong in the epic source, the \textit{Mahābhārata}”; and he further perceives this to be “nearly fatal to any contemporary literary appreciation”. After reading Pollock, one comes to believe that not only beauty but even politics lies in the eyes of the beholder; and so, this play of Kālidāsa too is a story, which for him is one “ultimately”, of “political power and its perpetuation”. Narendran notes with a keen eye that what Pollock means by the word “political” is in no wise connected with \textit{Rāja–dharma}, but with what the word is in the Western sense. (And in the case of the \textit{Rāmāyaṇa}, his inspiration is from the Marxian “The Eighteenth Brumaire”, which work has little to do with the Indian context). One must not miss to look at the impunity and recklessness in interpretation: Pollock does not rely upon any commentary in the tradition on \textit{Śākuntala}. On the other hand, he invokes Goethe to impart his political flavour to the play. And the irony is that Goethe himself has not viewed the play politically; and what a stroke of genius that Pollock flashes: “the infancy of the culture that ultimately produced Europe”; all that one needs to pass off the nonsense is to add a cunning disclaimer “it was then believed”, which he does.

As usual, Pollock shows his inconsistency. He speaks of his idea multiply as – of “three truths”, of “one true meaning”, of “three forms of truth”; and of “absolute truth having no place”; and of “no compulsion to rank or even to reconcile them [the truths]”. If he also speaks of the three truths being radically different from one another; how are we then to handle them? Which one shall we discard and why? Why posit in the first place the three kinds, and then make a claim of “one true meaning” of a text?: Why this duplicity (or rather triplicity)? Pollock has been handling the theme of philology for over two decades, and he is not able to arrive at one result or one approach. What then was missed before the 3D Philology, and what now is gained by this 3D Philology? Is Pollock confused (or trying to confuse others) as to what he is against and why? Does his theory follow his own practice? What patterns, and what designs animate his ideas? And the irony of ironies is that he claims to be representing the Indian point of view, and how they were understood “in traditional India”! From Macdonell through Staal to Dalrymple, scholars have often stressed on the \textit{remarkable continuity} of the Indian tradition, and here is a scholar who goes to the extent of telling Indians that they do not after all know what they have been saying through millennia. He may be an outsider who follows none of their creeds, yet it is he who knows better how to “exhume” their living traditions, and unpack for them what they contain!

In the \textit{Rāmāyaṇa}, Rāma famously told Kaikeyī “Please, consider me to be a sage–like person. Without a second thought, here I am, leaving for the forest. My loyalty is to \textit{dharma} alone”. “\textit{viddhi mām ṛṣibhis tulyaṁ kevalaṁ dharmam āsthitam}” (\textit{Rāmāyaṇa} 2.19.20). Again, Duṣyanta’s repudiation of Śakuntalā was consequent to his own apprehensions of violating \textit{dharma}: “\textit{dāra–tyāgī bhavāmy āho parastrī–sparśa–pāṁsulaḥ}?!” (5.29). Both kings were utterly dedicated to \textit{dharma} all their life. That is what their respective authors have clearly shown (1DP), and their commentators have concurringly elucidated (2DP), but here is Pollock who has the effrontery to claim that neither the authors (Vālmīki and Kālidāsa) nor their commentators are in the know of things. It is as though he is claiming – but I know; for I have the “3DP”; I can exhume, and in fact I have exhumed the structure of signification – and it is all just politics, it is all a play for securing power! And the irony is that Rāma even said most clearly and directly: “\textit{nāham artha–paro devi}!”. Rāma may not have been \textit{artha–para}; but we have a modern interpreter who is most \textit{anartha–para}! \textit{“anarthāyaiva śabdaika–paro’tātparya–vij jaḍaḥ”.} He is called a \textit{jaḍa} who cannot grasp the purport; and he a \textit{kitava} who wilfully distorts. The \textit{anartha} imparted by the latter is more dangerous.

All key truths of the Indian tradition are for Pollock just “buried”, and for decades Prof. Pollock has been labouring in the burial ground exhuming “truth” after “truth” with great implements!: if he wants to aver that descriptive catalogues in \textit{śāstra}–s have become prescriptive plans, he does not need to consult any practising \textit{śāstrin} for all this; he can just invoke a more competent resource – Geertz’s dichotomy of “models of” and “models for” serves well! When Pollock asserts that things happened in India that happened nowhere else, and “on a scale unparalleled in the premodern world”, he invokes the so handy Western theoreticians only – none of whom lived in India, or practised some form of the Hindu way of life, or hardly knew Sanskrit to be able to speak authentically about its tradition – ones like Geertz or Giddens, or Marx or Machiavelli – to analyse and explain Indian society, and to make sweeping remarks about the same, even if some of these have not made comments specifically on India at all. All this in spite of the clear and explicit caution imparted by Ingalls, \textit{his own mentor} (what an irony) about India – as to “how important it is to \textit{take into account the traditional categories and concepts} when attempting to understand the cultural achievements of ancient India”. Unbridled, Pollock throws unto winds all words of caution, and attempts to bully his way into whatever arenas he chooses.

Pollock is never tired of his attacks on “the elimination of the possibility of historical referentiality” in Indian texts in general, in Mīmāṁsā in particular, and even more so in the Veda–s. Rather than in ramblings, he often indulges in categorical assertions that actually require a wealth of data to buttress, but a general poverty of evidence pervades Pollock’s writings. See his tenor: “any text” (– see how sweeping Pollock’s claims are – ) “any text seeking to participate in brahminical discourse at all” (– it is not just “any”, it is also “at all”–) “\textit{was required to exclude precisely this} referential sphere”. It is only a matter of wonder how Pollock regularly sacrifices accuracy on the altar of precision! A strong tone of precision is sufficient, he seems to think, to convey a feeling of accuracy. (Un)fortunately for Pollock, there indeed are scholars who admire his precise statements even if there is not a tinge of truth in them. If it was indeed demanded of texts to exclude – “ignore or occlude” as per his own explanation – their constitutive historicity, where is even a sample directive towards that end? Fond though he is of referring to, or discovering, or asserting the normative and prescriptive character of the \textit{śāstra}–s, seldom does he offer even stray or indirect pronouncements in the \textit{śāstra}–s to that effect. Are these then grand theories \textit{sans} foundational facts, after all? Narendran points out very well how Pollock refers to India as “a society that produced millions and millions of texts”, with the participation of thousands of authors for thousands of years, and yet if not one has been “allowed” to dwell on historical details, one must only imagine on what large a scale the suppression must have been.

And if this suppression issued from Vedic and Mīmāṁsā streams, one must ask, were Buddhists and Jains also compelled to conform to this – rebels and reformers as they are portrayed to be? Not a single voice of protest on this issue do we hear while we see that the various schools of thoughts were constantly at loggerheads on every issue – from the trivial to the transcendental, and unsparing in criticism. Do we find even a single royal decree directly or indirectly purporting to purge historical frameworks or even allusions – even while one encounters easily over a lakh of inscriptions? Not the slightest tinge or the faintest hint of it in even the vernacular literature that happened in course of time to “supplant” Sanskrit? Obviously, Pollock’s grand theory is being built on Feenberg’s (affiliated to the New Left) or some other model.

In another paper, Pollock posits yet another unfounded idea – that earlier, \textit{śruti} (revelation) was considered stronger than \textit{smṛti} (tradition), but latterly, they came to be treated on par; rather, they were forced into convergence. And as usual, not even a semblance of effort to provide facts from within the traditional fold: the framework of one P. Bourdieu is more important than any fact, after all. Bourdieu, a public intellectual, is influenced by Marx (so says Wikipedia, and he influenced Giddens, cited earlier by Pollock as another authority). Pollock is never tired of force–fitting one Western model after another, imposing one leftist mold after another, upon the Indian scheme of things.

Like a sinking man catching hold of a straw, Pollock tries to credit Melputtūr Nārāyaṇa Bhaṭṭatiri of 17th cent. Kerala, of restoring to Sanskrit at once its historicity and its humanity. This he does on the basis of a small grammatical work of Bhaṭṭatiri; but then, sadly for Pollock, the same author has also written a work on Mīmāṁsā that Pollock loves to hate, and worse, there is not a single, not even the most indirect, allusion to any suppression of historicity in the work.

How Pollock cannot overcome his temptation for producing novel theories – even if it means grafting and adjusting some Western theory or the other upon Indian heritage – is evidenced in his comment on Lakṣmīdhara’s vast work on \textit{dharma–śāstra}. Says Pollock: “such vast intellectual output surely needs to be theorized in some way”, and pat comes his own spontaneous suggestion: these texts may well in part be “models for” rather than “models of”. Issues of women and \textit{śūdra}–s are, of course, a ready and perpetual source of Hinduism–bashing for all Westerners, and Pollock is no exception. The power–angle is of course the recurring – or rather nagging – \textit{leitmotif} in all of the requiems that Pollock pens.

Pollock is not unaware that his imagination is running amok: he is kind enough to acknowledge that several of his own friends attempted to check his excesses! He would rather stoop to ask all and sundry in the West about his thoughts or doubts pertaining to a text, but come what may, would not deign to check with any traditional scholar who actually preserves and practises the precepts of the self–same texts. Pollock will take the trouble of searching everywhere – except where the tradition thrives/survives. Was the \textit{Rāmāyaṇa} taught orally, or written down first? Ask Irvine, rather than give a hearing to even Vālmīki: the obsession – of the West as the normative, and the prescriptive – is the narcissistic strain that courses through all of Pollockian corpus. It is not necessary to listen to Vālmīki, even to the most explicit of his statements; he is only to be interpreted, after all: he can fare no better than a poor “native informant”.

Similar is Pollock’s treatment of the massive work on aesthetics by name \textit{Śṛṅgāra Prakāśa} of Bhoja: it aims, says Pollock, “to create politically correct subject and subjectivities”. A \textit{magnum opus} on aesthetics, and not a word on politics? – well, this perhaps goes against the prescriptions of some Westerner. Aesthetics must also be a part of a larger social theory; and why? Because Terry Eagleton (who is influenced by Marx, who else) thinks so. So his model must be applied here. What must have happened in respect of the \textit{Rāmāyaṇa}?: well, something that happened with respect to \textit{chansons de geste}. For, what happened in Europe (12th century France, in this case) must have happened in India too; why not? How did Sanskrit indeed spread? Do not look for causes within the Sanskrit tradition; rather go by Hannerz. And why?: because, he is a Westerner (an anthropologist). This gives a window to how Pollock’s mind works.

Sanskrit has a vast grammatical literature; Bhoja, a great king, has written on grammar and aesthetics. So grammar must have been a political tool – even if you may find no trace of any political statement after rummaging through thousands of pages of his (or any) grammatical writing: How does Pollock link it then with politics? Well, some Westerner (by name Nebrija) has stated that “language was indeed the \textit{compañera} of the empire in the West”. What happened in Europe according to Nebrija \textit{must have} happened in India also. QED. That is Pollock.

Narendran’s conclusion is very revealing: “Pollock is always a comprador classicist and never a critical one. In fact, he is not even a classicist, but only a theorist”. Narendran has arrived at this after a careful and unprejudiced analysis of nine thesis–statements of the long career of Pollock. Pollock takes all the trouble – and it is not small – to collect all necessary data from the Indian tradition – to fit, ultimately, to some Western model.

Pollock asserts that his 3DP imparts humility; but the one common element that courses through all writings of Pollock is his uniform hubris instead. And hypocrisy to give company to hubris. Say one thing, do quite another: eulogise 3D Philology, but practise a single–dimension Philology viz. Pollockian Philology, as we saw earlier. This creed of Pollock brims with anything but humility (or at least the utter lack of it). This is Orwellian Double–Speak plus narcissism deep–rooted – the very antithesis of humility.

In the \textit{Rāmāyaṇa}, we have Mārīca (the demon who cheated Rāma) telling Rāvaṇa that he saw Rāma everywhere (so afraid was he of Rāma’s power). And almost in some similar strain, Pollock sees the play of power everywhere – be it poetry or poetics, or grammar or Mīmāṁsā.

It is by no means an easy task to wade through the convoluted impenetrable prose of the massive writings of Pollock. That Narendran has yet done the same with great tenacity and assiduity bespeaks of his inordinate patience and perseverance. The assessments of Narendran are free from any trace of bias. With his training under sound traditional scholars, he has spoken as a practising, hence dignified, knowledgeable insider, and therefore, his pronouncements show a sense of responsibility.

There is hardly any exaggeration when he says the “scholarly truth” (claimed to be in the writings of Pollock (and his ilk)) is far from both scholarship and truth. In order to have a truly liberating philology, philology itself must first be liberated from the clutches of the vested interests of the West.

K S Kannan

\delimiter

\begin{flushright}
“\textit{If you are anxious to shine\\ In the speculative line\\ As a man of thought profound\\ You must vent exotic gabble\\ That will flabbergast the rabble\\ By its awe–inspiring sound!”}\\\textbf{Bluff’s Old Sweet Song} (1937)
\end{flushright}

