
\chapter*{Author’s Words}\label{authorswords}

I was fortunate to hear the illuminating lectures of Prof. K. S. Kannan when I joined Karnataka Samskrit University as a Ph.D. student in the year 2013. Prof. Kannan introduced us to the world of Western Indology and to the works of Sri Rajiv Malhotra. It was then that I realized the importance of getting aware of, and responding to, the intellectual challenges posed by Western academia. He also showed us by example the necessity of maintaining high academic standards both in style and content. I am indebted to his guidance which has made this work possible.

Presenting a paper in the first Swadeshi Indology Conference was a great experience but even greater was meeting Sri Rajiv Malhotra. His scholarship, vision and understanding of global trends are unparalleled. He made me understand the hard work and sacrifice that is needed to respond to such global challenges. While reading his books and watching his videos, I realized the immense effort and sacrifice he himself has made over the last two decades and continues to make. More importantly, he considers all of this a \textit{yajña} and a duty that all of us owe to our country and civilization.

Smt. Vijaya Vishwanathan’s energy and dedication to our tradition is truly inspiring.

Smt. Shalini Puthiyedam read the draft and offered valuable comments which was greatly beneficial.

Sri Aditya Agarwal’s enthusiasm and attitude that I witnessed during the second Swadeshi Indology Conference are admirable.

Sri Sudarshan Therani’s pointed suggestions at crucial moments gave me confidence to continue in the direction I had taken. His intuitive grasp of both our tradition and Western Indology never fails to impress me.

Dr. H R Meera painstakingly corrected the innumerable mistakes and gently advised me on the need to maintain high standards. I have implemented her suggestions.

Vellayan Chettiar Trust’s support and funding of this work made it possible to focus on this task. My sincere thanks to them.

\begin{flushright}
T.M. Narendran\\ Vijayadaśamī, Vikāri Saṁvatsara\\ Mysuru.
\end{flushright}

