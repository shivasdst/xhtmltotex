
\chapter{Bibliography\supskpt{\protect\footnote{[ 45 ] This bibliography attempts to be comprehensive, but there still may be some papers that have not come to my notice. Not all reprints are included here. \textit{The Longman Anthology of World Literature} edited by David Damrosch includes contributions from Pollock which are not listed in this bibliography.}}}

\item Akshara K.V. (Tr.) (2003). \textit{Viśvātmaka Deśabhāṣe} [Kannada translation of “The Cosmopolitan Vernacular”]. Heggodu. Karnataka: Akshara Prakashana. (Reprint 2015.)

 \item Almqvist, Kurt and Linklater, Alexander. (Ed.s) (2014).\textit{ Civilisation: Perspectives from the Engelsberg Seminar 2013}. Stockholm: Axel and Margaret Ax:son Johnson Foundation.

 \item Apte, Harinarayana. (Ed.) (1903). \textit{Yājñavalkya Smṛti}. Poona: Anandashrama Publications.

 \item Árnason, Jóhann Páll., Eisenstadt, S. N., and Wittrock, Björn. (Ed.s) (2005). \textit{Axial Civilizations and World History}. Jerusalem Studies in Religion and Culture. Leiden: Brill.

 \item Breckenridge, Carol Appadurai., and van der Veer, Peter. (Ed.s) (1993). \textit{Orientalism and the Postcolonial Predicament: Perspectives on South Asia}. Philadelphia, PA: University of Pennsylvania Press.

 \item Breckenridge, Carol., Pollock, Sheldon., Bhabha, Homi K., and Chakrabarty, Dipesh. (Ed.s) (2002). \textit{Cosmopolitanism}. Durham: Duke University Press.

 \item Calhoun, Craig J., Cooper, Frederick., and Moore, Kevin W. (Ed.s) (2006). \textit{Lessons of Empire: Imperial Histories and American Power}. New York: New Press.

 \item Caws, Mary Ann., and Prendergast, Christopher. (Ed.s) (1994).\textit{ The HarperCollins World Reader}. New York, NY: HarperCollins.

 \item Channakeshava, B., and Rao, H. V. Nagaraja. (Ed.s) (1995). \textit{Ananda bharati: Dr. K. Krishnamoorthy felicitation volume}. Mysore: D.V.K. Murthy.

 \item Chevillard, Jean–Luc. (Ed.) (2004). \textit{South–Indian Horizons: Felicitation Volume for François Gros.} Pondicherry: Institut français de Pondichéry/école française d'extrême–orient.

 \item Chattopadhyaya, Debiprasad and Gangopadhyaya, Mrinalkanti. (1967–68). \textit{Nyaya Philosophy}. (Part 1\&2). Calcutta: Indian Studies: Past \& Present.

 \item Conrad, Sebastian., and Randeria, Shalini. (Ed.s) (2002). \textit{Jenseits des Eurozentrismus: postkoloniale Perspektiven in den Geschichts–und Kulturwissenschaften}. Frankfurt/New York, NY: Campus Verlag.

 \item Dallapiccola, Anna Libera. (Ed.) (1989). \textit{Śāstric Traditions in Indian Arts}, Vol.1. Stuttgart: Steiner Verlag Wiesbaden GMBH.

 \item d’Intino, Silvia and Guenzi, Caterina. (Ed.s) (2012).\textit{ Aux abords de la clairière: études indiennes et comparées en l’honneur de Charles Malamoud}. érudites de l’Ecole Pratique des Hautes Etudes. Paris: Brepols.

 \item d’Intino, Silvia and Pollock, Sheldon. (Ed.s) (2019). \textit{Enjeux de la philologie indienne: Traditions, éditions, traductions/transferts}. Paris: Institute de Civilisation Indienne, Collège de France.

 \item Eisenstadt, Shmuel N., Schluchter, Wolfgang., and Wittrock, Björn. (Ed.s) (2001). \textit{Public Spheres and Collective Identities. }New Brunswick: Transaction Publishers.

 \item Eliot, Simon., Nash, Andrew., and Willison, Ian. (Ed.s) (2007). \textit{History of the Book and Literary Cultures}. London: British Library.

 \item Gangopadhyaya, Mrinalkanti. (1972–76). \textit{Nyaya Philosophy}. (Part 3\&4). Calcutta: Indian Studies: Past \& Present.

 \item Garzilli, Enrica. (Ed.) (1996).\textit{ Translating, Translations, Translators: From India to West}. Harvard Oriental Series: Opera Minora, Vol.1. Cambridge: Harvard University Press.

 \item Goldman, Robert P. (Ed.) (1984). \textit{The Rāmāyaṇa of Vālmīki: An Epic of Ancient India, Vol I: Bālakāṇḍa. }Princeton, NJ: Princeton University Press.

 \item —. (Ed.s) (1986).\textit{ The Rāmāyaṇa of Vālmīki: An Epic of Ancient India, Vol II: Ayodhyākāṇḍa. }Translated by Sheldon I Pollock. Princeton, NJ: Princeton University Press.

 \item —. (Ed.) (1991). \textit{The Rāmāyaṇa of Vālmīki: An Epic of Ancient India, Vol III: Araṇyakāṇḍa.} Translated by Sheldon I. Pollock. Princeton, NJ: Princeton University Press.

 \item Gupta, Munilal (Tr.) (1951). \textit{Viṣṇu Purāṇa.} Gorakhpur: Gita Press.

 \item Haksar, A. N. D. (Ed.) (1995). \textit{Indian Horizons}: \textit{Glimpses of Sanskrit Literature}. 44(4). New Delhi: Indian Council for Cultural Relations.

 \item Houben, J. E. M. (Ed.) (1996). \textit{Ideology and the Status of Sanskrit: Contributions to the History of the Sanskrit Language}. Leiden: Brill.

 \item —., and Pollock, Sheldon. (2008).“Theory and Method in Indian Intellectual History”.\textit{ Journal of Indian Philosophy, }36. pp 531–532.

 \item Kielhorn, F. and Abhyankar, K. V. (Ed.s) (1985). \textit{Vyākaraṇa Mahābhāṣya} of Patañjali. Vol.1 Poona: Bhandarkar Oriental Research Institute.

 \item Klein, Barbro and Joas, Hans. (Ed.s) (2010).\textit{ The Benefit of Broad Horizons: Intellectual and Institutional Preconditions for a Global Social Science}. \textit{Festschriftfor Björn Wittrock on the Occasion of his 65th Birthday}. Leiden: Brill.

 \item Koenig, C., Muellner, L., Nagy, G., and Pollock, S. (Ed.s) (2016). \textit{The Art of Reading: From Homer to Paul Celan} of Bollack, Jean. Translated by S. Tarrow and C. Porter with B. King. Hellenic Studies Series 73. Washington, DC: Center for Hellenic Studies.

 \item Lienhard, S., and Piovano, I. (Ed.s) (1997). \textit{Lex et Litterae: Essays on Ancient Indian Law and Literature in Honour of Oscar Botto}. Turin: Edizioni dell’Orso.

 \item Macfie, Alexander Lyon. (Ed.) (2000).\textit{ Orientalism: A Reader}. New York, NY: New York University Press.

 \item \textit{\textbf{Mahābhāṣya}} of Patañjali. See Kielhorn and Abhyankar (1985).

 \item Moyn, Samuel and Sartori, Andrew. (Ed.s) (2013).\textit{ Global Intellectual History}. New York: Columbia University Press.

 \item \textit{\textbf{Nyāyadarśana}} of Gautama. See Tarkbagish (1917–19).

 \item \textit{\textbf{Nyāyadarśana}} of Gautama. See Thakur (1997).

 \item \textit{\textbf{Nyāya–pariśuddhi}} of Vedānta Deśika. See Tatacharya (2011).

 \item \textit{\textbf{Nyāya–rasāsvādinī.}} See Tatacharya (2011).

 \item Owen, Stephen., and Pollock, Sheldon (2018). “Sorting Out Babel: Literature and Its Changing Languages.” In Pollock and Elman (2018e). pp 165–95.

 \item Pollock, Sheldon. (1977a). \textit{Aspects of Versification in Sanskrit Lyric Poetry.} Vol. 61. American Oriental Series. New Haven, CT: American Oriental Society.

 \item —. (1979a). “Text Critical Observations on Vālmīki Rāmāyaṇa.” In Sinha (1979).. pp 317–24.

 \item —. (1983a). “Some Lexical Problems in Vālmīki Rāmāyaṇa.” \textit{Rtam: Journal of Akhila Bharatiya Sanskrit Parishad} XI–XV: 275–88. B. R. Saxsena Special Volume. 

 \item —. (1984a). “The \textit{Rāmāyaṇa} Text and the Critical Edition”. In Goldman (1984). pp 82–93.

 \item —. (1984b). “The Divine King in the Indian Epic.” \textit{Journal of American Oriental Society} 104(3). pp 505–28.

 \item —. (1984c). “Ātmānaṁ Mānuṣam Manye: \textit{Dharmākūtam} on the Divinity of Rāma.” \textit{Journal of Oriental Institute}, 33 (3–4). pp 231–243.

 \item —. (1985a). “Rākṣasas and Others.” \textit{Indologica Taurinensia}, \textit{Journal of the International Association of Sanskrit Studies}, 13. pp 263–81.

 \item —. (1985b). “Daniel Henry Holmes Ingalls.” \textit{Journal of American Oriental Society }105(3). pp 387–389.

 \item —. (1985c). “The Theory of Practice and Practice of Theory in Indian Intellectual History.” \textit{Journal of the American Oriental Society}, 105(3). pp 499–519.

 \item —. (1985d). “Rāma’s Madness.” \textit{Wiener Zeitschrift Für Die Kunde Südasiens}, (29). pp 43–56.

 \item —. (1986a). Translation of \textit{Rāmāyaṇa of Vālmīki: Ayodhyākāṇḍa. }See Goldman (1986).

 \item —. (1986b). “Introduction.” In Goldman (1986). pp. 1–76.

 \item —. (1987a). “Ideología y narrativa en el Rāmāyaṇa De Vālmīki.” \textit{Estudios de Asia y Africa}, 22 (3, julio). pp 336–54.

 \item —. (1989a). “Mīmāṁsā and the Problem of History in Traditional India.” \textit{Journal of the American Oriental Society}, 109(4). pp 603–610.

 \item —. (1989b). “The Idea of Śāstra in Indian Tradition.” In Dallapiccola (1989). pp 17–26.

 \item —. (1989c). “Playing by the Rules: \textit{Śāstra} and Sanskrit Literature.” In Dallapiccola (1989). pp 301–12.

 \item —. (1990a). “From Discourse of Ritual to Discourse of Power in Sanskrit Culture.” \textit{Journal of Ritual Studies}, 4(2). pp 315–345.

 \item —. (1990b). “Humanities in South Asian Studies.” \textit{Items: Social Science Research Council} 44, no. 4 (December). pp 81–82.

 \item —. (1991a). Translation of \textit{Rāmāyaṇa of Vālmīki: Araṇyakāṇḍa. }See Goldman (1991).

 \item —. (1991b). “Introduction.” In Goldman (1991). pp 3–84.

 \item —. (1993a). “Deep Orientalism? Notes on Sanskrit and power beyond the Raj.” In Breckenridge and van der Veer (1993). pp 76–133.

 \item —. (1993b). “\textit{Rāmāyaṇa} and Political Imagination in India.” \textit{The Journal of Asian Studies}, 52(2). pp 261–297.Reprint in, \textit{Debates in History}, ed. David N. Lorenzen. Delhi: Oxford U. Press, 2004.

 \item —. (1994a). “Early and Middle Period South Asia.” In Caws and Pendergast (1994). pp 487–599.

 \item —. (1995a). “In Praise of Poets: On the History and Function of \textit{Kavipraśaṁsā}.” In Channakeshava and Rao (1995). pp 443–457.

 \item —. (1995b). “Literary History, Religion and Nation in South Asia: Introductory Note.” \textit{Social Scientist}, 23 (10–12). pp 1–7.

 \item —. (1995c). “Literary History, Indian History, World History.” \textit{Social Scientist}, 23 (10–12). pp 112–142.

 \item —. (1995d). “Public poetry in Sanskrit.” In Haksar (1995). pp 85–107.

 \item —. (1995e). “Making History: Kalyāṇi, A.D. 1008.” In Srinivasan and Nagaraju (1995). pp 559–576.

 \item —. (1995f). “Rāmāyaṇa and Public Discourse in Medieval India.” In Vyas (1995). pp 141–158.

 \item —. (1995g). “Book Review of \textit{Genres Littéraires En Inde} by Nalini Balbir.” \textit{Journal of the American Oriental Society}, 115(4). pp 685–689.

 \item —. (1996a). “Philology, Literature, Translation.” In Garzilli (1996). pp 111–127.

 \item —. (1996b). “The Sanskrit Cosmopolis, A.D. 300–1300: Transculturation, Vernacularization and the Question of Ideology.” In Houben (1996). pp 197–247.

 \item —. (1997a). “The Revelation of Tradition: \textit{Śruti}, \textit{Smṛti} and the Sanskrit Discourse of Power.” In Lienhard and Piovano (1997). pp 395–418.

 \item —. (1998a). “Bhoja's \textit{Śṛṅgāraprakāśa} and the Problem of \textit{Rasa}: A Historical Introduction and Annotated Translation.” \textit{Asiatische Studien}, \textit{52}(1). pp 117–93.

 \item —. (1998b). "India in the Vernacular Millennium: Literary Culture and Polity, 1000–1500."\textit{Daedalus} 127.3. pp 41–74.

 \item —. (1998c). “The Cosmopolitan Vernacular.” \textit{The Journal of Asian Studies}, 57(1). pp 6–37.

 \item —. (2000a). “Kālidāsa: Raghuvaṁśa.” Translation of Chapter 7. In Sharma (2000). pp 421–431.

 \item —. (2000b). “Indology, Power, and the Case of Germany.” In Macfie (2000). pp 302–323.

 \item —. (2000c). “Cosmopolitan and Vernacular in History.” \textit{Public Culture}, 12(3). pp 591–625.

 \item —. (2000d). “Indian Knowledge Systems on the Eve of Colonialism.” \textit{Intellectual History Newsletter}, 22. pp 1–16.

 \item —. (2001a). “New Intellectuals in Seventeen Century India.” \textit{The Indian Economic and Social History Review}, 38(1). pp 4–31. (Reprint in \textit{Medieval Mentality}, ed. Eugenia Vanina and D. N. Jha. Delhi: Tulika, 2009. \textit{Mind over Matter: Essays on Mentalities in Medieval India}, eds. D.N. Jha and Eugenia Vanina, 228–61. New Delhi: Tulika Books, 2009)

 \item —. (2001b). "The Death of Sanskrit." \textit{Comparative Studies in Society and History} 43(02). pp 392–426.

 \item —. (2001c). “Social Aesthetic and Sanskrit Literary Theory.” \textit{Journal of Indian Philosophy}, 29. pp 197–229. (Reprint in \textit{Abhinavagupta: Reconsiderations, }eds. Makarand Paranjape and Sunthar Visuvalingam, 382–413. New Delhi: Samvad India Foundation, 2006.)

 \item —. (2001d). “India in the Vernacular Millennium: Literary Culture and Polity, 1000–1500.” In Eisenstadt \textit{et al} (2001). pp 41–74.

 \item —. (2002a). See Breckenridge \textit{et al} (2002).

 \item —. (2002b). “Introduction: Cosmopolitanisms.” In Breckenridge \textit{et al} (2002). pp 1–14.

 \item —. (2002c) “Introduction: Working Papers on Sanskrit Knowledge–Systems on the Eve of Colonialism.” \textit{Journal of Indian Philosophy}, 30. pp 431–439.

 \item —. (2002d). “Dalla storia letteraria alla cultura letteraria nella storia”. In Squarcini (2002). pp 55–73.

 \item —. (2002e). “Ex oriente nox: Indologie im nationalsozialistischen Staat”. In Conrad and Randeria (2002). pp 335–371.

 \item —. (Ed.) (2003a). \textit{Literary Cultures in History: Reconstructions from South Asia. }Berkeley: University of California Press. (Reprint, Delhi: Oxford U. Press, 2005.)

 \item —. (2003b). "Introduction." In Pollock (2003a). pp 1–36.

 \item —. (2003c). "Sanskrit Literary Culture from the Inside Out." In Pollock (2003a). pp 39–130.

 \item —. (2003d).See Akshara (2003).

 \item —. (2004a).\textit{ “A New Philology: From Norm–bound Practice to Practice–bound Norm in Kannada Intellectual History.” }In Chevillard (2004). pp. 389–406.

 \item —. (2004b). "Forms of Knowledge in Early Modern South Asia: Introduction.’ \textit{Comparative Studies of South Asia, Africa, and the Middle East}, 24(2). pp 19–21.

 \item —. (2004c). “The Transformation of Culture–Power in Indo–Europe, 1000–1300.” \textit{Medieval Encounters}, v.10(1–3). pp 247–278.

 \item —. (2004d). “The Meaning of \textit{Dharma} and the Relationship of the Two Mīmāmsās: Appayya Dīkṣita’s ‘Discourse on the Refutation of a Unified Knowledge System of Pūrvamīmāmsā and Uttaramīmāmsā.” \textit{Journal of Indian Philosophy} 32(5–6). pp 769–811.

 \item —. (2005a).\textit{ The Ends of Man at the End of Premodernity}. Amsterdam: Royal Netherlands Academy of Arts and Sciences.

 \item —. (2005b). "Ratnaśrījñāna." In Sharma (2005). pp 637–43.

 \item —. (2005c). “The Revelation of Tradition: \textit{Śruti}, \textit{Smṛti} and the Sanskrit Discourse of Power.” In Squarcini (2005). pp 17–37.

 \item —. (2005d). “Axialism and Empire.” In Arnason \textit{et al} (2005). pp 397–450.

 \item —. (2006a).\textit{ The Language of the Gods in the World of Men: Sanskrit, Culture, and Power in Premodern India}. Berkeley: University of California Press.

 \item —. (2006b). "Empire and Imitation." In Calhoun et al (2006). pp 175–188.

 \item —. (2006c). “Sanskrit Knowledge on the Eve of Colonialism and at the Dawn of Globalization.” \textit{Tattvabodha – Essays from the Lecture Series of the National Mission for }Manuscripts, 1. pp 29–47.

 \item —. (2006d). “Power and Culture Beyond Ideology and Identity.” In Sanders (2006). pp 277–287.

 \item —. (2007a). "Pretextures of Time." \textit{History and Theory}, 46(October). pp 364–381.

 \item —. (2007b).\textit{ Rama’s Last Act}. [=Bhavabhūti’s \textit{Uttararāmacarita}] Translated with an introduction. New York: New York University Press: JJC Foundation.

 \item —. (2007c). “The Problem of Early Modernity in the Sanskrit Intellectual Tradition.” \textit{International Association of Asian Studies Newsletter}, 43. pp 8–9.

 \item —. (2007d). "Literary Culture and Manuscript Culture in Precolonial India." In Eliot \textit{et al} (2007). pp 77–94.

 \item —. (2007e). “We Need to Find What We Are Not Looking For.” \textit{International Association of Asian Studies Newsletter}, 43. pp 1,4–5.

 \item —. (2008a). “Towards a Political Philology: D. D. Kosambi and Sanskrit.” \textit{Economic and Political Weekly}, 43(30). pp 52–59. (Revised version in \textit{D D Kosambi: Unsettling the Past}, ed. Meera Kosambi. Delhi: Permanent Black, 2013.)

 \item —. (2008b). “Is there an Indian Intellectual History? Introduction to ‘Theory and Method in Indian Intellectual History.’” \textit{Journal of Indian Philosophy} 36(5–6). pp 533–542.

 \item —. (2008c). “The Real Classical Languages Debate.” \textit{The Hindu,} 27 Nov. 2008.

 \item —. (2008d).See Houben (2008).

 \item —. (2009a).\textit{ Bouquet of Rasa }and\textit{ River of Rasa. }[=Bhānudatta’s \textit{Rasa–mañjarī} and \textit{Rasa–taraṅgiṇī}] Translated with an Introduction. New York: New York University Press: JJC Foundation\textit{.}

 \item —. (2009b). “Future Philology? The Fate of a Soft Science in a Hard World.” \textit{Critical Inquiry}, 35(4). pp 931–961. (German translation: \textit{Geschichte der Germanistik}, 35–36 (2009). pp 25–50; Russian translation: \textit{New Literary Observer }(Moscow) 110 (2011). pp 92–114.) 

 \item —. (2009c). “Zukunftsphilologie?” Herausgegeben von Christoph König und Marcel Lepper. \textit{Geschichte der Germanistik Mitteilungen}, 35–36. pp 25–50.

 \item —. (2010a). “Comparison without Hegemony.” In Klein and Joas (2010). pp 185–204.

 \item —. (2010b). “What was Bhaṭṭa Nāyaka Saying? The Hermeneutical Transformation of Indian Aesthetics.” In Pollock (2010c). pp 143–184.

 \item —. (Ed.) (2010c). \textit{Epic and Argument in Sanskrit Literary History: Essays in Honor of Robert P. Goldman}. New Delhi: Manohar.

 \item —. (Ed.) (2011a). \textit{Forms of Knowledge in Early Modern Asia: Explorations in the Intellectual History of India and Tibet,1500–1800, }Durham: Duke University Press.

 \item —. (2011b). “Introduction.” In Pollock (2011a). pp 1–16\textit{.}

 \item —. (2011c). “The Languages of Science in Early–Modern India.” In Pollock (2011a). pp 19–48.

 \item —. (2011d). “Crisis in the Classics.” \textit{ Social Research: An International Quarterly}, 78(1). pp 21–48. (Reprint, New Delhi: Rupa, 2012: 20–46; Tamil translation, \textit{Manarkeni} (Chennai) (June 2012): 47–56.)

 \item —. (2011e). "Sanskrit Studies in the US." In Tripathi (2011). pp 259–310. 

 \item —. (2011f). “Indian Philology and India’s Philology.” \textit{Journal Asiatique}, 299(1). pp 423–42.

 \item —. (2011g). “The Revelation of Tradition: \textit{Śruti}, \textit{Smṛti} and the Sanskrit Discourse of Power.” In Squarcini (2011). pp 41–61.

 \item —. (2012a). “From \textit{Rasa} Seen to \textit{Rasa} Heard.” In D’Intino and Guenzi (2012). pp 189–207.

 \item —. (2012b). “Commentary on De Pee: Epicycles of Cathay.” \textit{Fragments: Interdisciplinary Approaches to the Study of Ancient and Medieval Pasts,} 2. pp 68–75.

 \item —. (2012c). “\textit{Vyakti} and the History of \textit{Rasa}.” \textit{Saṁskṛtavimarśaḥ, Journal of the Rasthriya Sanskrit Sansthan} 6. pp 232–53. World Sanskrit Conference Special.

 \item —. (2012d). “\textit{Rasa} after Abhinava.” In Watanabe et al (2012). pp 431–445.

 \item —. (2013a). “\textit{Praśasti}and Its Congeners: A Small Note on a Big Topic.” In Sundareswaran (2013). pp 21–39.

 \item —. (2013b). “Cosmopolitanism, Vernacularism, and Premodernity.” In Moyn and Sartori (2013). pp 59–80.

 \item —. (2014a). “What is South Asian Knowledge Good for?”. \textit{South Asia Institute Papers – Beiträge des Südasien– Instituts Heidelberg}, 1. pp 1–22.

 \item —. (2014b). “Kritische Philologie.” \textit{Geschichte der Germanistik} (45–46). pp 5–12.

 \item —. (2014c).\textit{ “Philology in Three Dimensions.” Postmedieval: A Journal of Medieval Cultural Studies}, 5(4). pp 398–413.

 \item —. (2014d). “Indian Classicity.” In Almqvist and Linklater (2014). pp 61–69.

 \item —. (2015a). “Philologia Rediviva?”. \textit{American Academy of Arts and Sciences Bulletin}, 68(4). pp 34–36.

 \item —. (2015b). “The Alternative Classicism of Classical India.” \textit{Seminar} 671. Revised version of ‘Indian Classicity’. \textless \url{https://www.india–seminar.com/2015/666/666_sheldon_pollock.htm}\textgreater . Accessed on 15 November 2019.

 \item —. (2015c).\textit{ “}Rice and Ragi: Remembering URA.” Seminar 666. pp 16–20. \textless https://www.india–seminar.com/2015/666/666\_sheldon\_pollock.htm\textgreater . Accessed on 15 November 2019.

 \item —. (2015d). In collaboration with U. R. Ananthamurthy. “State of Nature (\textit{Prakriti})” translated from the Kannada. \textit{Seminar} 666. pp 77–82. \url{http://www.india–seminar.com/2015/666/666_u_r_ananthamurthy.htm}. Accessed on 15 November 2019.

 \item —. (2015e).\textit{Kritische Philologie: Essays zu Literatur, Sprache und Macht in Indien und Europa}. Herausgegeben von Christoph König. übersetzt von Brigitte Schöning. Philologien. Göttingen: Wallstein–Verlag.

 \item —. (2015g). “What was Philology in Sanskrit?” In Pollock \textit{et al.} (2015). pp 114–141.

 \item —. (2015h). “Introduction.” In Pollock \textit{et al} (2015f). pp 1–24.

 \item —. (2015i). “Liberating Philology.” \textit{Verge: Studies in Global Asias}, 1(1). pp 16–21.

 \item —. (2016a). \textit{A Rasa Reader: Classical Indian Aesthetics}. New York: Columbia University Press. (Reprint, 2017. Delhi: Permanent Black)

 \item —. (2016b). “Philology and Freedom.” \textit{Philological Encounters}, 1(1). pp 4–30.

 \item —. (2016c). “Why a Classical Library of India?” Introduction to The Murthy Classical Library of India. murtylibrary.com/why–a–classical–library–of–india.php. Accessed on Dec. 1, 2016.

 \item —. (2016d). “Areas, Disciplines, and the Goals of Inquiry.” \textit{The Journal of Asian Studies}, 75(4). pp 913–928.

 \item —. (2016e). “Introduction: An Intellectual History of Rasa.” In Pollock (2016a). pp1–45.

 \item —. (2016f). See Koenig et al (2016).

 \item —. (2017a). “The Columbia Global Humanities Project.” \textit{Comparative Studies of South Asia, Africa, and the Middle East.} 37(1). pp 113–16.

 \item —. (2017b). "Conundrums of Comparison," \textit{KNOW: A Journal on the Formation of Knowledge}, 1(2). pp 273–294.

 \item —. (2017c). “La filologia nel mondo.” \textit{Rivista di Filologia e di Istruzione Classica} 145(1). pp 221–51.

 \item —. (2018a). “Small Philology and Large Philology.” \textit{Comparative Studies of South Asia, Africa and the Middle East}, 38 (1). pp 122–127.

 \item —. (2018b). “What Should a Classical Library of India Be?”. \textit{Geschichte der Germanistik Historische Zeitschrift für die Philologien}, 53–54. pp 6–21.

 \item —. (2018c). “Wie Wir Lesen”. \textit{Geschichte der Germanistik}, 53–54. pp 62–75.

 \item —. (2018d). \textit{Philologie und Freiheit}. Aus dem Englischen von Reinhart Meyer–Kalkus. Berlin: Matthes \& Seitz.

 \item —. (2018g). See Owen and Pollock (2018).

 \item —. (2019a). See d’Intino and Pollock (2019).

 \item —. (2019b). “Indian Philology: Edition, Interpretation, and Difference.” In d’Intino and Pollock (2019). pp 3–45.

 \item —., Elman, Benjamin., and Chang, Ku–ming Kevin. (Ed.s) (2015f).\textit{ World Philology}. Cambridge: Harvard University Press.

 \item —. and Elman, Benjamin A. (Ed.s) (2018e). \textit{What China and India Once Were: The Pasts That May Shape the Global Future}. New York: Columbia University Press. (Reprint Delhi: Penguin Random House, 2018.)

 \item —. and Elman, Benjamin A. (2018f). “Introduction.” In Pollock and Elman (2018e). pp 1–24.

 \item Sanders, Seth L. (Ed.) (2006). \textit{Margins of Writing, Origins of Cultures}. Oriental Institute Seminars. Chicago: Oriental Institute of the University of Chicago.

 \item Sharma, Ramkaran. (Ed.) (2005). \textit{Encyclopedia of Indian Wisdom: Prof. Satya Vrat Shastri Felicitation Volume.} Delhi: Bharatiya Vidya Prakashan.

 \item Sharma, T. R. S. (Ed.) (2000). \textit{Ancient Indian Literature: An Anthology} Vol. 2. New Delhi: Sahitya Akademi.

 \item Squarcini, F. (Ed.) (2002). \textit{Verso l’India oltre l’India. Scritti e ricerche sulle tradizioni intellettuali sudasiatiche}. Milano: Mimesis.

 \item —. (Ed.) (2005). \textit{Boundaries, Dynamics and Construction of Traditions in South Asia}. Florence: Florence U. Press.

 \item —. (Ed.) (2011). \textit{Boundaries, Dynamics and Construction of Traditions in South Asia}. London: Anthem Press.

 \item Srinivasan, L. K. and Nagaraju, S. (Ed.s) (1995). \textit{Śrī Nāgābhinandanam: Dr. M.S. Nagaraja Rao Festschrift.}, Vol 2. Bangalore: M.S. Nagaraja Rao Felicitation Committee.

 \item Sundareswaran, N. K. (Ed.) (2013).\textit{ Rajamahima: C. Rajendran Congratulatory Volume}. Calicut University Sanskrit series No. 51. Calicut: University of Calicut Press.

 \item Tarkbagish, Pandit Srijukita Phanibhushan (Ed.) (1917–19). \textit{Nyāyadarśan O Bātsāyan Bhāṣya.} Vol.s 1–5. Kolkata: Vangiya Sahitya Parisad Mandir.

 \item Tatacharya, N. S. Ramanuja. (2011). \textit{Nyāyapariśuddhi} of Vedānta Deśika with \textit{Nyāya–sārāsvādinī} of N. S. Ramanuja Tatacharya. Vol.1. Chennai: Srirangam Srimad Andavan Ashramam.

 \item Thakur, Anantalal (Ed.) (1997). \textit{Gautamīya–nyāya–darśana} with the \textit{Bhāṣya} of Vātsyāyana. New Delhi: Indian Council of Philosophical Research.

 \item Tripathi, Radhavallabh. (Ed.) (2011). \textit{Sixty Years of Sanskrit Studies: Vol.2: Countries Other than India. }New Delhi: DK Printworld.

 \item \textit{\textbf{Viṣṇupurāṇa.}} See Gupta (1951).

 \item Vyas, R. T. (Ed.) (1995). \textit{Studies in Jaina Art and Iconography and Allied Subjects in Honour of U. P. Shah}. Baroda: Oriental Institute.

 \item Watanabe, Chikafumi., Desmarais, Michele., and Honda, Yoshichika. (Ed.s) (2012). \textit{Saṁskṛta–Sādhutā ‘Goodness of Sanskrit’: Studies in Honour of Professor Ashok N. Aklujkar}. New Delhi: D. K. Printworld.

 \item \textit{\textbf{Yājñavalkyasmṛti.}} See Apte (1903).

