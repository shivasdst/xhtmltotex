
\chapter{ಆದರ್ಶ ಗೃಹಸ್ಥ}

ಈ ದಿನ ನಾನು, ಈಗಾಗಲೇ ಪ್ರಕಟಿಸಿರುವಂತೆ ‘ಗೃಹಸ್ಥರು ಹಾಗೂ ಅವರ ಆಧ್ಯಾತ್ಮಿಕ ಜೀವನ’ ಎಂಬ ವಿಷಯದ ಮೇಲೆ ಅನೌಪಚಾರಿಕವಾಗಿ ಕೆಲವು ಮಾತುಗಳನ್ನಾಡುತ್ತೇನೆ.

ಶ್ರೀರಾಮಕೃಷ್ಣ ವಚನವೇದದಲ್ಲಿ ‘ಗೃಹಸ್ಥರಿಗೆ ಬುದ್ಧಿವಾದ’ ಇತ್ಯಾದಿ ಶೀರ್ಷಿಕೆಗಳನ್ನುಳ್ಳ ಅಧ್ಯಾಯಗಳು ಹಲವಾರಿವೆ. ಗೃಹಸ್ಥ ರಿಗೆ ಶ್ರೀರಾಮಕೃಷ್ಣರು ನೀಡಿರುವ ಬೋಧನೆ ಬಹಳ ಮಹತ್ವದ್ದು. ‘ಗೃಹಸ್ಥರು’ ಎಂದರೆ ಅದು ಜನಸಂಖ್ಯೆಯ ಶೇಕಡಾ ೯೯.೯ ಭಾಗಕ್ಕೆ ಅನ್ವಯಿಸುತ್ತದೆ. ಅವರು ಹೇಗೆ ಜೀವನ ನಡೆಸಬೇಕು ಎಂಬುದೊಂದು ದೊಡ್ಡ ಪ್ರಶ್ನೆ. ಸ್ವಾಮಿ ವಿವೇಕಾನಂದರು ತಮ್ಮ ಭಾಷಣದಲ್ಲಿ ಒಂದೆಡೆ ಹೇಳುತ್ತಾರೆ, ‘ಆಧ್ಯಾತ್ಮಿಕನಲ್ಲದ ಯಾವೊಬ್ಬನನ್ನೂ ನಾನು ಹಿಂದೂ ಎಂದು ಕರೆಯಲಾರೆ’ ಎಂದು. ಎಂಥ ಸುಂದರ ಭಾವನೆಯಿದು, ನೋಡಿ! ಹಿಂದೂಗಳು ‘ಆಧ್ಯಾ ತ್ಮಿಕ’ರಾಗಿರಬೇಕು; ಕೇವಲ ‘ಧಾರ್ಮಿಕ’ರಾಗಿರುವುದು ಬಹಳ ಸುಲಭ. ಹಣೆಯ ಮೇಲೆ ಗಂಧವನ್ನೋ, ವಿಭೂತಿಯನ್ನೋ ಹಚ್ಚಿ ಕೊಳ್ಳಿ–ನೀವು ಹಿಂದೂಗಳಾಗಿಬಿಡುತ್ತೀರಿ. ಶಿಲುಬೆ ಹಾಕಿಕೊಳ್ಳಿ– ಕ್ರೈಸ್ತರಾಗುತ್ತೀರಿ. ಟೊಪ್ಪಿಯ ಮೇಲೆ ಅರ್ಧಚಂದ್ರವನ್ನು ಧರಿಸಿ ದಿರೆಂದರೆ ಮುಸ್ಲಿಮರಾಗುತ್ತೀರಿ! ಹೀಗೆ, ಧಾರ್ಮಿಕರಾಗುವುದು ಸುಲಭ. ಆದರೆ ನಾವು ಆಧ್ಯಾತ್ಮಿಕ ವ್ಯಕ್ತಿಗಳಾಗಬೇಕೇ ಹೊರತು ಬರಿಯ ಧಾರ್ಮಿಕರಾಗುವುದಲ್ಲ.

ನಮಗೆ ಈ ಅಧ್ಯಾತ್ಮದ ಕಲ್ಪನೆ ದೊರಕುವುದು ಉಪನಿಷತ್ತು ಗಳಿಂದ. ಮಾನವನ ಸಹಜ ಸ್ವಭಾವವು ಮೂಲತಃ ಆಧ್ಯಾತ್ಮಿಕ ವಾದುದೆಂದು ಈ ಉಪನಿಷತ್ತುಗಳು ಸಾರುತ್ತವೆ. ದೇಹ-ಬುದ್ಧಿಗಳ ಈ ಸಂಮಿಶ್ರಣದ ಹಿಂದೆ ಚಿತ್ (ಪ್ರಜ್ಞೆ) ಹಾಗೂ ಸತ್ (ಸತ್ಯ) ಸ್ವರೂಪದ ಆತ್ಮವಸ್ತು ಇದೆ. ಅನಂತತೆಯೇ ನಮ್ಮ ನೈಜಲಕ್ಷಣ. ಇದೊಂದು ಮಹತ್ತರ ಸಂಶೋಧನೆ. ಇದನ್ನು ನಾವು ಮಾನವನ ಕುರಿತಾದ ‘ವೈಜ್ಞಾನಿಕ ಸತ್ಯ’ವೆಂದು ಕರೆಯುತ್ತೇವೆ. ಮಾನವರನ್ನು ಕೂಲಂಕಷವಾಗಿ ಅಧ್ಯಯನ ಮಾಡಿ ನಮ್ಮ ಉಪನಿಷತ್ ಕಾಲದ ಪುಷಿಗಳು ಈ ಮಹಾಸತ್ಯವನ್ನು ಕಂಡುಹಿಡಿದರು. ಕಠೋಪನಿಷತ್ತಿ ನಲ್ಲಿ ಯಮನು ನಚಿಕೇತನಿಗೆ ಮಾಡಿದ ಉಪದೇಶದಲ್ಲಿ ಹೇಳಿದೆ.

\begin{verse}
ಏಷ ಸರ್ವೇಷು ಭೂತೇಷು ಗೂಢೋಽತ್ಮಾ ನ ಪ್ರಕಾಶತೇ ।\\ ದೃಶ್ಯತೇ ತ್ವಗ್ರ್ಯಯಾ ಬುದ್ಧ್ಯಾ ಸೂಕ್ಷ್ಮಯಾ ಸೂಕ್ಷ್ಮದರ್ಶಿಭಿಃ ॥\hfill (ಕ. ಉ. ೩.೧೨)
\end{verse}

‘ಈ ಆತ್ಮನು ಸರ್ವಜೀವಿಗಳಲ್ಲೂ ಇದ್ದಾನೆ. ಆದರೆ ವ್ಯಕ್ತ ನಾಗದೆ ಮರೆಯಾಗಿದ್ದಾನೆ. ಯಾರು ಸೂಕ್ಷ್ಮ-ಸೂಕ್ಷ್ಮತರ ಸತ್ಯ ಗಳನ್ನು ಅರಿಯಲು ಬೇಕಾದ ಶಿಕ್ಷಣ ಪಡೆದಿದ್ದಾರೋ ಅಂಥವರು, ಏಕಾಗ್ರಗೊಂಡ ಅತ್ಯಂತ ಸೂಕ್ಷ್ಮಬುದ್ಧಿಯ ನೆರವಿನಿಂದ ಆತ್ಮನನ್ನು ಸಾಕ್ಷಾತ್ಕರಿಸಿಕೊಳ್ಳಲು ಸಾಧ್ಯ.’

ಈ ದಿವ್ಯ-ಪರಿಶುದ್ಧಚೈತನ್ಯವೂ, ನಿತ್ಯಶುದ್ಧ-ಅಮರವೂ ಆದ ಆತ್ಮವೆಂಬುದು ಪ್ರತಿಯೊಬ್ಬ ಮಾನವನಲ್ಲೂ ಇದೆ, ಇದು ಸತ್ಯ. ಆದರೆ ವ್ಯಕ್ತವಾಗಿಲ್ಲ, ಮರೆಯಾಗಿದೆ. ಎಂದರೆ ಅದು ಎಂದೆಂದಿಗೂ ಮರೆಯಾಗಿಯೇ ಇದ್ದುಬಿಡುವುದೇನು? ಇಲ್ಲ. ಹಿಂದಿನ ಮಹಾ ಪುಷಿಮುನಿಗಳು ಆತ್ಮವನ್ನು ಸಾಕ್ಷಾತ್ಕರಿಸಿಕೊಂಡಿದ್ದಾರೆ ಮತ್ತು ಪ್ರತಿಯೊಬ್ಬರೂ ಆತ್ಮವನ್ನು ಸಾಕ್ಷಾತ್ಕರಿಸಿಕೊಳ್ಳಬಹುದು. ಮಾನವರಲ್ಲಿ ಆ ಸಾಮರ್ಥ್ಯ ಅಂತರ್ಗತವಾಗಿಯೇ ಇದೆ. ನಮ್ಮ ಬುದ್ಧಿಗೆ ಸೂಕ್ಷ್ಮಸತ್ಯಗಳನ್ನು ಕಂಡುಕೊಳ್ಳುವ ತರಬೇತಿ ನೀಡಿ ದರೆ, ಸೂಕ್ಷ್ಮಗೊಂಡ ಬುದ್ಧಿ ಈ ಸೂಕ್ಷ್ಮಾತಿಸೂಕ್ಷ್ಮ ಸತ್ಯವಾದ ಆತ್ಮವನ್ನು ಕಂಡುಕೊಳ್ಳಬಲ್ಲದು.

ಇದು ಹೇಗೆ ಸಾಧ್ಯವೆಂಬುದಕ್ಕೆ ಇಂದಿನ ಪರಮಾಣು ವಿಜ್ಞಾನ ವೊಂದು ಉದಾಹರಣೆ. ತೀಕ್ಷ ್ಣಗೊಂಡ ಬುದ್ಧಿಯು ವಸ್ತುಗಳ ಸೂಕ್ಷ್ಮಸ್ವರೂಪವನ್ನು ಅರಿಯಬಲ್ಲದು. ಉದಾಹರಣೆಗೆ ವಸ್ತುವಿ ನಲ್ಲಿ ಶಕ್ತಿ ಅಡಕವಾಗಿದೆ ಎಂಬುದು ಆಧುನಿಕ ವಿಜ್ಞಾನಿಗಳ (ಮುಖ್ಯವಾಗಿ ಐನ್​ಸ್ಟೈನನ) ತೀಕ್ಷ್ಣಬುದ್ಧಿಯ ಅನ್ವೇಷಣೆ. ಹಿಂದೆ ನ್ಯೂಟನ್​ನ ಕಾಲದಲ್ಲಿ ಈ ಅನ್ವೇಷಣೆ ಆಗಿರಲಿಲ್ಲ. ಅಲ್ಲಿಯ ವರೆಗೂ ನ್ಯೂಟನ್​ನದೇ ಅತ್ಯಂತ ತೀಕ್ಷ ್ಣಬುದ್ಧಿ ಎಂದು ಹೆಸರಾ ಗಿತ್ತು. ಆದರೆ ಇಪ್ಪತ್ತನೇ ಶತಮಾನದಲ್ಲಿ ಆ ಬುದ್ಧಿ ಇನ್ನಷ್ಟು ತೀಕ್ಷ ್ಣಗೊಂಡಿತು. (ಅದರಿಂದಾಗಿ, ಅಣುವಿನಲ್ಲಿ ಶಕ್ತಿ ಅಡಕವಾ ಗಿರುವ ಮೂಲಭೂತ ಸಂಗತಿ ಗೋಚರವಾಯಿತು.) ಹಾಗೆಯೇ ಮಾನವನ ಹಿಂದಿರುವ ಸತ್ಯವನ್ನು ಪರಿಕಿಸುವಲ್ಲಿ ಎರಡು ಆಯಾಮಗಳಿವೆ: ಒಂದು, ಸಾಮಾನ್ಯಬುದ್ಧಿ; ಇನ್ನೊಂದು ಸೂಕ್ಷ್ಮ ಬುದ್ಧಿ. ನಮಗೆ ಆತ್ಮವಿಚಾರಕ್ಕೆ ಬೇಕಾಗಿರುವುದು ಅತಿ ಸೂಕ್ಷ್ಮ ಬುದ್ಧಿ. ಅದನ್ನು ಗಳಿಸುವುದು ಹೇಗೆ? ‘ಸೂಕ್ಷ್ಮದರ್ಶಿಭಿಃ’– ಯಾರು ಸೂಕ್ಷ್ಮಸತ್ಯಗಳೊಂದಿಗೆ, ಸೂಕ್ಷ್ಮತರ ಸತ್ಯಗಳೊಂದಿಗೆ, ವ್ಯವಹರಿಸುವುದನ್ನು ರೂಢಿಸಿಕೊಳ್ಳುತ್ತಾರೋ ಅಂಥವರು, ಒಳಗೆ ಅಡಕವಾಗಿರುವ ಸೂಕ್ಷ್ಮತಮ ಸತ್ಯವಾದ ಆತ್ಮವನ್ನು ಒಳಹೊಕ್ಕು ನೋಡಬಲ್ಲ ಅತಿಸೂಕ್ಷ್ಮಬುದ್ಧಿಯನ್ನು ಬೆಳೆಸಿಕೊಳ್ಳುತ್ತಾರೆ. ಅದೇ ಕಠೋಪನಿಷತ್ತಿನಲ್ಲಿ, ಆತ್ಮವನ್ನು ಅರಿಯುವತ್ತ ಮುನ್ನಡಿ ಇಡು ವಂತೆ ಆಜ್ಞಾಪಿಸುವ ಒಂದು ಭವ್ಯ ಉದ್ಗಾರವನ್ನು ಕಾಣುತ್ತೇವೆ. (೩.೧೪)

\begin{verse}
‘ಉತ್ತಿಷ್ಠತ ಜಾಗ್ರತ ಪ್ರಾಪ್ಯ ವರಾನ್ ನಿಬೋಧತ ।’
\end{verse}

‘ಏಳಿ, ಎಚ್ಚರಗೊಳ್ಳಿ ಮತ್ತು ಗುರಿಮುಟ್ಟುವವರೆಗೂ ನಿಲ್ಲ ದಿರಿ!’–ಈ ವಾಕ್ಯಕ್ಕೆ ಸ್ವಾಮಿ ವಿವೇಕಾನಂದರು ನೀಡಿದ ಭಾವಾರ್ಥವಿದು. ಈ ವಾಕ್ಯದ ಶಬ್ದಾರ್ಥ: ‘ಏಳಿ, ಎಚ್ಚರಗೊಳ್ಳಿ, ಮತ್ತು ಮಹಾತ್ಮರ ಬಳಿಸಾರಿ ಜ್ಞಾನವನ್ನು ಪಡೆದುಕೊಳ್ಳಿ!’ ಎಂಥ ಅದ್ಭುತ ಸಂದೇಶ!

ಗೃಹಸ್ಥರಾಗುವುದರೊಂದಿಗೆ ನಾವು ನಮ್ಮ ಜೀವನದ ಮತ್ತೊಂದು ಹಂತವನ್ನು ಪ್ರವೇಶಿಸುತ್ತೇವೆ. ವೇದಗಳ ಪ್ರಕಾರ, ಮಾನವನ ಜೀವಿತಾವಧಿ ಒಂದು ನೂರು ವರ್ಷಗಳು. ಶಂಕರಾ ಚಾರ್ಯರು ಬರೆಯುತ್ತಾರೆ: \textbf{‘ತಾವದ್ ಹಿ ಪುರುಷಸ್ಯ ಪರಮಾಯುಃ ನಿಬೋಧಿತಮ್​’}–‘ಮಾನವನ ಆಯುಷ್ಯದ ಪರಿಮಿತಿ ಇಷ್ಟು.’ ಈ ಜೀವಿತಾವಧಿಯನ್ನು ನಾವು ನಾಲ್ಕು ಭಾಗಗಳಾಗಿ ವಿಂಗಡಿ ಸುತ್ತೇವೆ: ಮೊದಲನೆಯದು, ವಿದ್ಯಾರ್ಥಿಯಾಗಿ, ಬ್ರಹ್ಮಚಾರಿ ಯಾಗಿ, ಅಧ್ಯಯನ ಮಾಡಿ ಜ್ಞಾನಾರ್ಜನೆ ಮಾಡುವ ಹಂತ. ನಾವು ಪಡೆದುಕೊಳ್ಳಬೇಕಾದ ಜ್ಞಾನ ಎಷ್ಟೊಂದಿದೆ! ಒಂದು ಮಾನವಶಿಶುವಿಗೆ ಶಿಕ್ಷಣ ಪಡೆದು ವಿದ್ಯಾವಂತನಾಗಲು ೨ಂ-೨೫ ವರ್ಷಬೇಕು. ಆದರೆ ಒಂದು ಪ್ರಾಣಿಯ ಮರಿಗೆ ಅಷ್ಟೊಂದು ಸಮಯ ಬೇಡ. ಕರು ಹುಟ್ಟಿದ ಒಂದು ಗಂಟೆಯೊಳಗೆಲ್ಲ ನೆಗೆದಾಡಲು ಕಲಿಯುತ್ತದೆ. ಒಂದು ಸ್ವಲ್ಪ ಅತ್ತಿತ್ತ ಓಡಾಡು ತ್ತದೆ–ಅಷ್ಟಕ್ಕೆ ಮುಗಿಯಿತು ಅದರ ಶಿಕ್ಷಣ. ಆದರೆ ಒಂದು ಮಾನವಶಿಶು ಸುಮಾರು ಇಪ್ಪತ್ತೈದು ವರ್ಷಗಳ ಕಾಲ ಶಿಕ್ಷಣ ಪಡೆಯುತ್ತಲೇ ಇರುತ್ತದೆ. ಇದೇ ಬ್ರಹ್ಮಚರ್ಯದ ಅವಧಿ. ಬ್ರಹ್ಮಚರ್ಯಕ್ಕೆಂದು ಒಂದು ಅವಧಿಯನ್ನು ಮೀಸಲು ಇಡುವ ಈ ಬಗೆಯ ಜೀವನವಿಶ್ಲೇಷಣೆಯನ್ನು ನಾವು ಬೇರೆ ಯಾವುದೇ ಜನಾಂಗದ ಸಾಹಿತ್ಯದಲ್ಲಿ ಕಾಣಲಾರೆವು. ಹೀಗೆ ಸುಮಾರು ಇಪ್ಪತ್ತೈದು ವರ್ಷ ವಯಸ್ಸಾದ ಮೇಲೆ, ಮುಂದಿನದು ಗೃಹಸ್ಥ ಜೀವನ. ಎಂದರೆ ಸ್ತ್ರೀ-ಪುರುಷರು ವಿವಾಹ ಮಾಡಿಕೊಂಡು, ಒಂದೋ ಎರಡೋ ಮಕ್ಕಳನ್ನು ಪಡೆದು ಸಂಸಾರ ನಡೆಸುವುದು. ಇದು ಪ್ರಕೃತಿಯಿಚ್ಛೆ. ಪ್ರಕೃತಿ ಮಾನವನಿಗೆ ಆದೇಶಿಸುತ್ತದೆ, ‘ಹೌದು, ನೀನು ಒಂದೋ ಎರಡೋ ಮಕ್ಕಳನ್ನು ಹೆರು.’ ಜೀವಶಾಸ್ತ್ರವನ್ನು ಓದಿದರೆ ತಿಳಿಯುತ್ತದೆ–ಸಂತಾನೋತ್ಪತ್ತಿ ಮಾಡುತ್ತ ತನ್ನ ವಂಶವನ್ನು ಉಳಿಸಿಕೊಳ್ಳಲಾರದ ಯಾವುದೇ ಒಂದು ಜೀವಜಾತಿಯನ್ನು ಪ್ರಕೃತಿ ಕಾಪಾಡಲಾರದು ಎಂದು. ಸಂತಾನೋತ್ಪತ್ತಿಯಿಲ್ಲದೆ ಸೃಷ್ಟಿಕಾರ್ಯ ಮುಂದುವರಿಯದು, ಜೀವವಿಕಾಸ ನಡೆಯದು. ಆದ್ದರಿಂದಲೆ ಮಾನವ ಜೀವನದಲ್ಲಿ ಇದೊಂದು ಹಂತ, ಒಂದು ಕರ್ತವ್ಯ–ವಿವಾಹವಾಗಿ ಒಂದೆರಡು ಮಕ್ಕಳನ್ನು ಹೆರುವ ಮೂಲಕ, ಮಾನವ ಜನಾಂಗದ ವಿಕಾಸವನ್ನು ಸಾಧಿಸುವುದರಲ್ಲಿ ಪ್ರಕೃತಿಯ ಉಪಕರಣವಾಗುವುದು. 

ಹಿಂದೆ, ಜಗತ್ತಿನಲ್ಲಿ ಅತಿಜನಸಂಖ್ಯೆಯ ಸಮಸ್ಯೆಯಿಲ್ಲದಿ ದ್ದಾಗ, ಒಬ್ಬರಿಗೆ ನೂರು ಮಕ್ಕಳಿರುತ್ತಿದ್ದರು. ಆದರೆ ಈಗ ಕಟ್ಟು ನಿಟ್ಟಾಗಿ ಒಂದು, ತಪ್ಪಿದರೆ ಎರಡು; ಅಷ್ಟು ಸಾಕು. ಇದಕ್ಕಿಂತ ಹೆಚ್ಚಾಗಿ ನಮಗೆ ಬೇಕಿಲ್ಲ. ಇರಲಿ, ಇದು ಒಂದು ಅಂಶ. ಆದರೆ ಈ ‘ಪ್ರಕೃತಿಯ ಕರ್ತವ್ಯ’ವನ್ನು ನಡೆಸುವುದರ ಜೊತೆಗೇ, ನಾವು ನಮ್ಮ ಆಧ್ಯಾತ್ಮಿಕ ಜೀವನವನ್ನೂ ಬೆಳೆಸಿಕೊಳ್ಳಬೇಕು. ಬ್ರಹ್ಮ ಚರ್ಯದ ಹಂತದಲ್ಲೇ ಪ್ರತಿಯೊಬ್ಬ ವ್ಯಕ್ತಿಯೂ–ಸ್ತ್ರೀಯಾಗಲಿ, ಪುರುಷನಾಗಲಿ–ತನ್ನೊಳಗೆ ಅಡಗಿರುವ ದಿವ್ಯತೆಯನ್ನು ಅರಿಯ ಲಾರಂಭಿಸುತ್ತಾನೆ. ದಿವ್ಯತೆಯು ನಮ್ಮಲ್ಲಿ ವ್ಯಕ್ತವಾದಂತೆಲ್ಲ ನಾವು, ಜನರನ್ನು ಪ್ರೀತಿಸುವುದು ಹೇಗೆ, ಜನರ ಸೇವೆ ಮಾಡು ವುದು ಹೇಗೆ, ಸಮಾಜದಲ್ಲಿ ಶಾಂತಿಯಿಂದ ಸಹಬಾಳ್ವೆ ನಡೆಸು ವುದು ಹೇಗೆ–ಇದನ್ನೆಲ್ಲ ಕಲಿಯುತ್ತೇವೆ. ‘ನಾನು ನಿನಗಾಗಿ ಏನು ಮಾಡಲಿ? ನಾನು ನಿನಗೆ ಹೇಗೆ ನೆರವಾಗಲಿ?’–ಹೀಗೆನ್ನುವ ಸಾಮರ್ಥ್ಯ ಬರುತ್ತದೆ. ನೋಡಿ, ಇದೆಂಥ ಅದ್ಭುತ ಭಾವನೆ– ನಾನು ಈ ಜಗತ್ತಿನಲ್ಲಿ ಏಕಾಂಗಿಯಲ್ಲ. ಇನ್ನೂ ಎಷ್ಟೋ ಜನ ಇಲ್ಲಿದ್ದಾರೆ. ಅವರೆಲ್ಲರೊಡನೆ ನಾನು ಹೇಗೆ ಸ್ನೇಹದಿಂದ ವ್ಯವಹರಿಸಲಿ?’ ಎಂಥ ಸುಂದರ ಕಲ್ಪನೆ! ನಾವಿರುವ ಈ ಸಮಾಜ ಕೇವಲ ಜನಜಂಗುಳಿಯಲ್ಲ, ಅದೊಂದು ಸುಸಂಬದ್ಧ ವ್ಯವಸ್ಥೆ. ಈ ಸುಸಂಬಂಧ ಏರ್ಪಡುವುದು ಜನರನ್ನು ಪ್ರೀತಿಸುವ, ಜನರಿಗೆ ಸೇವೆ ಸಲ್ಲಿಸುವ ಸಾಮರ್ಥ್ಯದಿಂದ. ತನ್ಮೂಲಕ ಪರಸ್ಪರ ಸಹಾಯ, ಪ್ರಗತಿ ಸಾಧ್ಯವಾಗಿ, ನಮ್ಮ ಜೀವನ ಸಂತೋಷಮಯ ವಾಗುತ್ತದೆ. ಗೀತೆ ಹೇಳುತ್ತದೆ (೩.೧೧):

\begin{verse}
‘ಪರಸ್ಪರಂ ಭಾವಯಂತಃ ಶ್ರೇಯಃ ಪರಮವಾಪ್ಸ್ಯಥ ।’
\end{verse}

‘ಪರಸ್ಪರ ಪ್ರೀತಿಸುವ ಹಾಗೂ ನೆರವಾಗುವ ಮೂಲಕ ಸಕಲರೂ ಅತ್ಯುಚ್ಚ ಸ್ಥಾನವನ್ನು ಹೊಂದಬಹುದು.’ ಮಹತ್ತರ ವಾದ ಶ್ರೇಯಸ್ಸು ಸಾಧ್ಯವಾಗಬೇಕಾದರೆ, ಪರಸ್ಪರ ನೆರವಾಗುವ, ಸಕಾರಾತ್ಮಕವಾಗಿ ವ್ಯವಹರಿಸುವ ಶಕ್ತಿ ನಮ್ಮಲ್ಲಿರಬೇಕು. ಇದೇ ಒಬ್ಬ ಆದರ್ಶಗೃಹಸ್ಥನ ಜೀವನಮಾರ್ಗ. ಗೃಹಸ್ಥನಿಗೆ ಈ ಮೂಲಕ ಮನಶ್ಶಾಂತಿ ದೊರೆಯುತ್ತದೆ; ಅಲ್ಲದೆ ಈ ಮಾರ್ಗ ದಲ್ಲೇ ಮುಂದುವರಿದು, ಸಂಸ್ಕೃತಿ ಮತ್ತು ನಾಗರಿಕತೆಗಳನ್ನು ಸಮೃದ್ಧಗೊಳಿಸುವಂಥ ಉತ್ತಮ ಮಕ್ಕಳನ್ನು ಪಡೆಯುತ್ತಾನೆ.

ಇಲ್ಲಿ, ಆಧುನಿಕ ಜೀವಶಾಸ್ತ್ರ ನಮಗೊಂದು ಗಹನವಾದ ಸತ್ಯ ವನ್ನು ತಿಳಿಸಿಕೊಡುತ್ತದೆ. ಅದು, ಮಾನವನ ವೈಶಿಷ್ಟ್ಯ; ಮಾನವ ಜೀವಿಗಳಿಗೂ ಮತ್ತು ಮಾನವಪೂರ್ವಮೃಗಗಳಿಗೂ ಇರುವ ವ್ಯತ್ಯಾಸ. ಪರಂಪರೆ ಎನ್ನುವುದು ಎಲ್ಲ ಪ್ರಾಣಿಗಳಲ್ಲೂ ಇರು ವಂಥದು. ಆದರೆ ಉಳಿದ ಪ್ರಾಣಿಗಳಿಗೆಲ್ಲ ಒಂದೇ ಒಂದು ಪರಂಪರೆಯಿದೆ. ಅದು ಕೇವಲ ಜೈವಿಕವಾದದ್ದು; ಎಂದರೆ ವಂಶವಾಹಿನಿಗಳ ಮೂಲಕ ಬರುವಂಥದ್ದು. ನಮಗೂ ಈ ಜೈವಿಕ ಪರಂಪರೆಯಿದೆ–ತಂದೆತಾಯಿಗಳಿಂದ ಈ ಶರೀರ ದೊರ ಕಿದೆ. ಆದರೆ ಇಷ್ಟೇ ಅಲ್ಲ. ನಮಗೆ ಮತ್ತೊಂದು ಪರಂಪರೆ ಇದೆ. ಅದನ್ನು ಸಾಂಸ್ಕೃತಿಕ ಪರಂಪರೆ ಎನ್ನುತ್ತಾರೆ. ಇಂದಿನ ಜೀವವಿಜ್ಞಾನ ತಿಳಿಸಿಕೊಡುತ್ತದೆ–ಸಂಸ್ಕೃತಿ ಎಂದರೆ ಸಂಚಿತ ಜ್ಞಾನ, ಒಗ್ಗೂಡಿಸಿದ ಅನುಭವ ಎಂದು. ಹಿಂದೆ ವೇದಕಾಲದ ಜನರು ಕೆಲವು ಅನುಭವಗಳನ್ನು ಪಡೆದರು. ಆದರೆ ಅವರ ನಿಧನ ದೊಂದಿಗೆ ಆ ಜ್ಞಾನ ಕಣ್ಮರೆಯಾಗಲಿಲ್ಲ. ಅವರು ಆ ಅನುಭವ ಗಳನ್ನು ಬರೆದಿಟ್ಟಿದ್ದು, ಅವೇ ವೇದಗಳಾದವು, ಅವು ಮುಂದಿನ ತಲೆಮಾರುಗಳ ಆಸ್ತಿಯಾದವು. ಹೀಗೆಯೆ ಸಾಹಿತ್ಯ, ಕಲೆ, ವಿಜ್ಞಾನ, ಧರ್ಮ, ತತ್ತ್ವಶಾಸ್ತ್ರಗಳು ಹೆಚ್ಚು ಹೆಚ್ಚಾಗಿ ಸಮೃದ್ಧ ವಾಗುತ್ತ ಮಾನವಸಂತತಿಯ ಪರಂಪರಾಗತ ಸ್ವತ್ತಾಗುತ್ತವೆ. ‘ಸಂಚಿತ ಸಂಸ್ಕೃತಿ’ಯೆಂದರೆ ಇದು. ಈ ಸಂಸ್ಕೃತಿ ಹೀಗೆ ವಿಸ್ತಾರ ವಾಗುತ್ತಲೇ ಇರುತ್ತದೆ. ಯಾವ ಪ್ರಾಣಿಗೂ ‘ಸಾಂಸ್ಕೃತಿಕ ಪರಂ ಪರೆ’ ಎಂಬುದಿಲ್ಲ. ಹೀಗೆ ಮಾನವಶಿಶು ತನ್ನ ತಂದೆತಾಯಿ ಗಳಿಂದ ಮತ್ತು ಹಿರಿಯರಿಂದ ಬರಿಯ ಜೈವಿಕ ಆನುವಂಶಿಕತೆ ಯನ್ನಷ್ಟೇ ಅಲ್ಲದೆ ಪೂರ್ವಕಾಲದ ಸಂಸ್ಕೃತಿಯನ್ನು ಪರಂಪರಾ ಗತವಾಗಿ ಪಡೆಯುತ್ತದೆ. ಮಾನವರಾದ ನಮಗೆ ಇಬ್ಬಗೆಯ ಪರಂಪರೆಯಿರುವುದರಿಂದ ಆ ಸಂಸ್ಕೃತಿಯನ್ನು ನಾವು ಬೆಳಸಿ, ವಿಶಾಲಗೊಳಿಸಿ ನಮ್ಮ ಸ್ವಂತ ಅನುಭವಗಳಿಂದ ಇನ್ನಷ್ಟು ಶ್ರೀಮಂತಗೊಳಿಸಿ ನಮ್ಮ ಮುಂದಿನ ಪೀಳಿಗೆಗೆ ನೀಡಬೇಕು. ಅದು ಗೃಹಸ್ಥರ ಜವಾಬ್ದಾರಿ.

ಭಾರತೀಯರಾದ ನಾವು ಕಳೆದ ಕೆಲವು ಶತಮಾನಗಳಿಂದ ಮರೆತಿದ್ದದ್ದು ಇದನ್ನೇ. ನಮ್ಮದು ನಿಂತ ನೀರಿನಂತಹ ಸಂಸ್ಕೃತಿ ಯಾಯಿತು. ಬದಲಾವಣೆಯಿಲ್ಲ, ಪ್ರಗತಿಯಿಲ್ಲ; ಹಲವು ನೂರು ವರ್ಷಗಳಿಂದ ಅದೇ ನಿರ್ಜೀವ ಸ್ಥಗಿತತೆ. ಅದಕ್ಕೆ ಹಿಂದಿನ ಅವಧಿಯಲ್ಲಿ, ಎಷ್ಟೋ ಕ್ಷೇತ್ರಗಳಲ್ಲಿ ಪ್ರಚಂಡ ಪ್ರಗತಿ ಯಾಗಿತ್ತು. ಆದರೆ ಹೇಗೋ ಏನೋ, ಕಳೆದ ಸಾವಿರ ವರ್ಷ ಗಳಲ್ಲಿ ನಾವು ಇನ್ನಷ್ಟು ಮತ್ತಷ್ಟು ಕುಸಿಯುತ್ತಲೇ ಬಂದೆವು. ಹೊರದೇಶಗಳ ಜನರೊಂದಿಗೆ ನಾವು ಸಂಪರ್ಕವಿಡಲಿಲ್ಲ. ‘ಮ್ಲೇಚ್ಛ’ ಎಂಬ ಪದವನ್ನೂ ಭಾವನೆಯನ್ನೂ ಹುಟ್ಟುಹಾಕಿ ಅವ ರನ್ನು ದೂರವಿಟ್ಟೆವು. ‘ಬೇರೆಲ್ಲ ರಾಷ್ಟ್ರೀಯರೂ ಮ್ಲೇಚ್ಛರು; ಮ್ಲೇಚ್ಛರನ್ನು ಮುಟ್ಟಬೇಡ; ನಮ್ಮ ದೇಶದ ಗಡಿಯನ್ನು ದಾಟ ಬೇಡ!’ ಈ ಭಾವನೆ ನಮ್ಮಲ್ಲಿ ತುಂಬಿತ್ತು. ಇದರ ಪರಿಣಾಮ ವೇನಾಯಿತು ಎಂಬುದನ್ನು ಸ್ವಾಮಿ ವಿವೇಕಾನಂದರು ಒಂದೇ ವಾಕ್ಯದಲ್ಲಿ ಸಂಗ್ರಹವಾಗಿ ಹೇಳುತ್ತಾರೆ: ‘ಯಾವಾಗ ಭಾರತವು “ಮ್ಲೇಚ್ಛ” ಶಬ್ದವನ್ನು ಸೃಷ್ಟಿಸಿ ಹೊರಜಗತ್ತಿನ ಸಂಪರ್ಕವನ್ನು ಕಡಿದುಕೊಂಡಿತೋ ಆಗಲೇ ಅದರ ದುರ್ವಿಧಿ ಇನ್ನು ಬದಲಾಗ ದಂತೆ ಗಟ್ಟಿಯಾಯಿತು.’

ಈ ಮನೋಭಾವದಿಂದಾಗಿ ನಾವು ಸಾಕಷ್ಟು ಕಷ್ಟ ಅನು ಭವಿಸಿದೆವು. ಇತರರೊಂದಿಗೆ ಸಂಪರ್ಕವಿಟ್ಟುಕೊಳ್ಳದೆ ಇದ್ದುದ ರಿಂದ, ಫ್ರಾನ್ಸ್​ನ ಬೂರ್ಬನ್ ರಾಜವಂಶಜರಂತೆ ಸ್ಥಗಿತ ರಾದೆವು. ಇತಿಹಾಸಕಾರರು ಬೂರ್ಬರ ಬಗ್ಗೆ ಹೇಳುತ್ತಾರೆ, ‘ಅವರು ಹೊಸದೇನನ್ನೂ ಕಲಿಯಲೂ ಇಲ್ಲ, ಹಳೆಯದೇನನ್ನೂ ಮರೆಯಲೂ ಇಲ್ಲ; ಆದ್ದರಿಂದಲೇ ಫ್ರಾನ್ಸಿನಲ್ಲಿ ರಕ್ತಕ್ರಾಂತಿ ಯಾದದ್ದು’ ಎಂದು. ಭಾರತೀಯರ ಬಗ್ಗೆಯೂ ಇಂಥದೇ ಮಾತು, ಹೆಸರಾಂತ ಅರಬ್ಬೀ ಪ್ರವಾಸಿ ಅಲ್ ಬೆರೂನಿಯಿಂದ ಬಂದಿದೆ. ಈತ ಹತ್ತನೇ ಶತಮಾನದಲ್ಲಿ ಘಸ್ನಿ ಮಹಮ್ಮದನ ಜೊತೆಯಲ್ಲಿ ಬಂದಿದ್ದ. ಇವನಿಗೆ ಸಂಸ್ಕೃತದ ಹಾಗೂ ಭಾರ ತೀಯ ತತ್ತ್ವಶಾಸ್ತ್ರದ ತಿಳಿವು ಸಾಕಷ್ಟಿತ್ತು. ಘಸ್ನಿ ಮಹಮ್ಮದನು ಭಾರತದ ಧನಸಂಪತ್ತನ್ನು ಸೂರೆ ಮಾಡಲು ಬಂದರೆ, ಅಲ್ ಬೆರೂನಿ ಇಲ್ಲಿಯ ತಾತ್ತ್ವಿಕ ಸಂಪತ್ತನ್ನು ಕೊಂಡೊಯ್ಯಲು ಬಂದಿದ್ದ. ‘ಅಲ್ ಬೆರೂನಿಯ ಭಾರತ’ ಎಂಬ ಪುಸ್ತಕದಲ್ಲಿ ಆತನ ಟೀಕೆಟಿಪ್ಪಣಿಗಳು ಕಾಣಸಿಗುತ್ತವೆ. ಅವನೆನ್ನುತ್ತಾನೆ: ‘ಈ ಭಾರತೀಯರಿಗೆ ಏನಾಗಿಬಿಟ್ಟಿತು? ಇವರ ಪೂರ್ವಜರು ಹೀಗೆ ಸಂಕುಚಿತ ಮನಸ್ಕರಾಗಿರಲಿಲ್ಲ! ಇವರು ಯಾರೊಂದಿಗೂ ಬೆರೆಯುವುದಿಲ್ಲ; ತಮ್ಮ ಜ್ಞಾನವನ್ನು ಯಾರಿಗೂ ಕೊಡುವುದಿಲ್ಲ; ಮತ್ತೊಬ್ಬರ ಜ್ಞಾನವನ್ನು ತಾವೂ ಪಡೆಯುವುದಿಲ್ಲ. ಇವರ ಪೂರ್ವಜರು ಹೀಗಿರಲಿಲ್ಲ!’

ಸ್ವಾಮಿ ವಿವೇಕಾನಂದರು ಹೇಳಿದ್ದೂ ಇದನ್ನೇ–ಸಾವಿರ ವರ್ಷಗಳ ಹಿಂದೆ ಅಲ್ ಬೆರೂನಿ ಹೇಳಿದ್ದನ್ನೇ. ಆದರೆ ಇಂದಿನ ಭಾರತ ಬದಲಾಗಿದೆ. ನಾವು ಹೆಚ್ಚು ತೆರೆದ ಮನಸ್ಸಿನವರಾಗಿ ದ್ದೇವೆ. ವಿದೇಶೀಯರೊಂದಿಗೆ ನಾವು ಭಾವನೆಗಳನ್ನು ವಿನಿ ಮಯಿಸಿಕೊಳ್ಳಬಲ್ಲೆವು. ಇಂದು ಜೀವಶಾಸ್ತ್ರ ತಿಳಿಸುತ್ತದೆ–ಮನು ಕುಲವೆಲ್ಲ ಒಂದೇ ಪ್ರಭೇದ ಎಂದು. ನೊಣದಂತಹ ಒಂದು ಕೀಟದಲ್ಲೇ ನೂರಾರು ಪ್ರಭೇದಗಳಿವೆ. ಆದರೆ ಮಾನವರದು ಒಂದೇ ಪ್ರಭೇದ–ಪರಸ್ಪರರಲ್ಲಿ ವಿಚಾರ ವಿನಿಮಯ ಮಾಡ ಬಲ್ಲ, ಸಂತಾನಾಭಿವೃದ್ಧಿ ಮಾಡಬಲ್ಲ ಏಕಜಾತಿ ನಮ್ಮದು. ಇದೆಷ್ಟು ಸುಂದರವಾದ ಕಲ್ಪನೆ! ನಿಜ, ಶಾರೀರಿಕವಾಗಿಯೇನೋ ನಾವೆಲ್ಲ ಒಂದೇ. ಆದರೆ ಅಷ್ಟು ಮಾತ್ರವಲ್ಲ, ಆಧ್ಯಾತ್ಮಿಕ ವಾಗಿಯೂ ನಾವೆಲ್ಲರೂ ಒಂದೇ ಎನ್ನುತ್ತದೆ ವೇದಾಂತ. ನನ್ನಲ್ಲೂ ನಿಮ್ಮಲ್ಲೂ ಸಕಲರಲ್ಲೂ ಒಂದೇ ಅನಂತಾತ್ಮ ಇದೆ ಎಂಬುದನ್ನು ಉಪನಿಷತ್ತುಗಳು ಕಂಡುಕೊಂಡವು. ಆಧ್ಯಾತ್ಮಿಕ ವಾಗಿ ನಾವೆಲ್ಲ ಒಂದೇ–ಈ ಸತ್ಯ ನಮಗೆ ಮನವರಿಕೆ ಯಾಗಬೇಕು. ದೈಹಿಕವಾಗಿಯೂ ನಾವು ಒಂದು. ಈ ದೃಷ್ಟಿ ಯಿಂದಲೇ ನಾವು ಭಾರತೀಯರು ನಮ್ಮ ಜೀವನವನ್ನು ರೂಪಿಸಿ ಕೊಂಡಿದ್ದುದು. ಆದರೆ ಮುಂದೆ ಸಂಕುಚಿತ ಬುದ್ಧಿ ಬಂತು; ನಾವು ಹೊರಗೆ ಹೋಗಿ ಹೊಸ ಬದಲಾವಣೆಗಳನ್ನು ಕಾಣಲಿಲ್ಲ. ತತ್ಫಲವಾಗಿ ವಿದೇಶೀ ಆಕ್ರಮಣಕಾರರಿಗೆ ಸೋತು ನಮ್ಮ ರಾಜಕೀಯ ಸ್ವಾತಂತ್ರ್ಯವನ್ನು ಕಳೆದುಕೊಂಡೆವು. ಅವರು ಬಂದೂಕು ಬಳಸಿದರೆ, ನಾವು ಬಿಲ್ಲು-ಬಾಣ ಹಿಡಿದು ಕಾದಿದೆವು. ಇತರೆಡೆಗಳಲ್ಲಿ ಯಾವ ಬದಲಾವಣೆಗಳಾಗಿವೆಯೆಂದು ನಮಗೆ ತಿಳಿದಿರದಿದ್ದ ಕಾರಣ, ನಾವು ಪ್ರತಿ ಬಾರಿಯೂ ಸೋತೆವು. ಈಗ ನಾವು ಅದರ ಪಾಠ ಕಲಿತಿದ್ದೇವೆ. ಕೊಳ್ಳಲೂ ನೀಡಲೂ ನಮ್ಮ ಮನಸ್ಸು ಸಿದ್ಧವಾಗಿ ತೆರೆದುಕೊಂಡಿದೆ. ಈ ಅಂಶವನ್ನು ಸ್ವಾಮಿ ವಿವೇಕಾನಂದರು ಪುನಃ ಪುನಃ ಒತ್ತಿ ಹೇಳುತ್ತಿದ್ದರು. 

ಇಂದಿನ ಗೃಹಸ್ಥ ಹಿಂದಿನವರಂತಲ್ಲ; ಆತ ಇತರರಿಗೆ ಕೊಡಲೂ ಬಲ್ಲ, ಅವರಿಂದ ಸ್ವೀಕರಿಸಲೂ ಬಲ್ಲ. ಒಂದು ಮಾನವೀಯ ಪ್ರಜ್ಞೆಯನ್ನು, ಮಾನವೀಯ ಸಂಸ್ಕೃತಿಯನ್ನು ಬೆಳೆ ಸಲು ಇದುವೇ ಮಾರ್ಗ. ಜಗತ್ತು ಇದೇ ನಿಟ್ಟಿನಲ್ಲಿ ಮುಂದುರಿಯುತ್ತಿದೆ; ಭಾರತವೂ ಅದಕ್ಕೆ ಕೊಡುಗೆ ನೀಡುತ್ತಿದೆ. ಇಂದಿನ ಗೃಹಸ್ಥನೂ ಹಾಗೆಯೇ. ಆತ ಸಂಕುಚಿತನಾಗಿ, ತಾನೇ ಬೇರೆಯಾಗಿ ಇರಲಾರ, ಇರಲೊಲ್ಲ. ಆದ್ದರಿಂದಲೇ ನಾವಿಂದು ಜಾತಿಯ ಕಟ್ಟುಪಾಡುಗಳನ್ನೂ ಇತರ ಕ್ಷುಲ್ಲಕ ಸಾಮಾಜಿಕ ಧೋರಣೆಗಳನ್ನೂ ಮುರಿದುಹಾಕುತ್ತಿದ್ದೇವೆ. ಬ್ರಿಟಿಷರ ಕಾಲದಲ್ಲಿ ಜಾತಿಬುದ್ಧಿ ಬಹಳ ಬಲವಾಗಿತ್ತು. ಒಬ್ಬ ಗವರ್ನರನೋ, ವೈಸರಾಯನೋ ತನ್ನನ್ನು ಕಾಣಲು ಬಂದರೆ ಭಾರತೀಯ ರಾಜ ನೊಬ್ಬನು ಆತನನ್ನು ಬರಮಾಡಿಕೊಂಡು ಮಾತನಾಡುತ್ತಿದ್ದ; ಅವನೊಡನೆ ಕೈಯನ್ನೂ ಕುಲುಕುತ್ತಿದ್ದ. ಆದರೆ ಬಂದ ಅತಿಥಿ ಗಳೆಲ್ಲ ಹೊರಟುಹೋದಮೇಲೆ, ಈ ರಾಜ ತನಗಾದ ಮೈಲಿಗೆ ಯನ್ನು ಹೋಗಲಾಡಿಸಿಕೊಳ್ಳಲು ಸ್ನಾನ ಮಾಡುತ್ತಿದ್ದ! ಆಗ ನಮ್ಮಲ್ಲಿದ್ದ ಸಣ್ಣತನ ಆ ತರಹದ್ದು. ಇಂದು ಅದೆಲ್ಲ ಹೋಗಿದೆ. ನಮ್ಮಲ್ಲಿ ಹೆಚ್ಚಿನ ಜನ ಮುಕ್ತಮನಸ್ಕರಾಗಿದ್ದಾರೆ. ನಮ್ಮ ಜೀವನವನ್ನು ಸರಿಯಾಗಿ ರೂಪಿಸಿಕೊಳ್ಳಲು ಈ ಕಾಲ ಅತ್ಯು ತ್ತಮವಾಗಿದೆ. ನಮ್ಮ ವೇದಾಂತವು ತಿಳಿಸಿರುವಂತಹ ಉನ್ನತವೂ ಸರ್ವವ್ಯಾಪಕವೂ ಆದ ಮಾನವತಾತತ್ವವನ್ನು ಹಾಗೂ ಗೀತೆ-ಉಪನಿಷತ್ತುಗಳು ನೀಡುವ ಅಧ್ಯಾತ್ಮವನ್ನು ನಮ್ಮ ಜೀವನ ಆಧರಿಸಿರಬೇಕು. ಈ ತತ್ತ್ವವಿಚಾರಗಳು ಪುರಾಣಗಳಲ್ಲೂ ಇವೆ; ಆದರೆ ಗೀತೆ-ಉಪನಿಷತ್ತುಗಳೇ ಮುಖ್ಯ ಮೂಲಗಳು. ಅವು ಹಿಂದೂಗಳಿಗೆ ಮಾತ್ರವಲ್ಲದೆ, ಪ್ರತಿಯೊಬ್ಬ ಮಾನವನಿಗಾಗಿ ಸಾರ್ವತ್ರಿಕವಾದ ಆಧ್ಯಾತ್ಮಿಕ ತತ್ತ್ವಗಳನ್ನು ಒಳಗೊಂಡಿವೆ. ಸಮಸ್ತ ಮಾನವತೆಯನ್ನು ದೃಷ್ಟಿಯಲ್ಲಿರಿಸಿಕೊಂಡೇ ಅವು ರಚಿತ ವಾಗಿವೆ.

\begin{verse}
‘ಶೃಣ್ವಂತು ವಿಶ್ವೇ ಅಮೃತಸ್ಯ ಪುತ್ರಾಃ।’
\end{verse}

‘ಓ ಸಮಸ್ತ ವಿಶ್ವದ ಅಮೃತಪುತ್ರರೇ! ಇಲ್ಲಿ ಕೇಳಿ!’

ಸಕಲ ಮಾನವರನ್ನು ಉದ್ದೇಶಿಸಿರುವ, ಶ್ವೇತಾಶ್ವತರ ಉಪನಿ ಷತ್ತಿನ ಈ ವಾಕ್ಯದ ಭಾಷೆಯನ್ನು ನೋಡಿ! ಶಿಕಾಗೋದ ಸರ್ವಧರ್ಮಸಮ್ಮೇಳನದಲ್ಲಿ (೧೮೯೩) ಸ್ವಾಮಿ ವಿವೇಕಾನಂದರು ಉದ್ಧರಿಸಿ ವಿವರಿಸಿದಾಗ, ಬಹು ಸಂಖ್ಯೆಯಲ್ಲಿ ನೆರೆದಿದ್ದ ಶ್ರೋತೃ ಗಳ ಮೇಲೆ ಪ್ರಚಂಡ ಪರಿಣಾಮವನ್ನುಂಟುಮಾಡಿದ ಶ್ಲೋಕ ಇದು. ‘ಇಲ್ಲಿ ಕೇಳಿ, ಎಲ್ಲೆಡೆಗಳಲ್ಲಿರುವ ಓ ಅಮೃತಪುತ್ರರೆ! ನೀವು ಪಾಪ ಸಂತಾನರಲ್ಲ. ಮನುಷ್ಯನನ್ನು ಪಾಪಿಯೆಂದು ಕರೆ ಯುವುದೇ ದೊಡ್ಡ ಪಾಪ; ಮಾನವತೆಗೇ ಅದೊಂದು ಶಾಶ್ವತ ಕಳಂಕ.’ ಇದು ಶಿಕಾಗೋ ಸಮ್ಮೇಳನದಲ್ಲಿ ಸ್ವಾಮೀಜಿ ಬಳಸಿದ ಭಾಷೆ. ಹೌದು, ಮಾನವನೆಂದರೆ ಅಮೃತತ್ವದ ಶಿಶು. ಸಹಸ್ರಾರು ವರ್ಷಗಳ ಹಿಂದೆ ಉಪನಿಷತ್ತುಗಳು ಸಾರಿದ್ದು ಇದನ್ನೇ. ಭಾರತದ ಒಳಗಾಗಲಿ ಹೊರಗಾಗಲಿ ಇದೇ ಮಾತು ಅನ್ವಯ ವಾಗುತ್ತದೆ. ಹಿಂದೂ, ಮುಸ್ಲಿಂ, ಕ್ರೈಸ್ತ, ನಾಸ್ತಿಕ, ಅಜ್ಞೇಯವಾದಿ –ಎಲ್ಲರೂ ಅಮೃತಪುತ್ರರೇ. ಮೂಲತಃ ನಾವು ಅಮರ ಆತ್ಮರು. ಅದೇ ನಮ್ಮ ನೈಜ ಪ್ರಕೃತಿ. \textbf{‘ತತ್ ತ್ವಂ ಅಸಿ’}–‘ನೀನು ಅದೇ ಆಗಿರುವೆ.’ ನೀವೆಲ್ಲರೂ ಆ ದಿವ್ಯ ಅವಿನಾಶೀ ಆತ್ಮವೆ. ಇದು ಛಾಂದೋಗ್ಯ ಉಪನಿಷತ್ತಿನ (೬.೯.೪) ಮಾತು. ಗೀತೆ- ಉಪನಿಷತ್ತುಗಳ ಈ ಬಗೆಯ ಮಹಾತತ್ತ್ವಗಳು ನಮಗೆ ಮಾರ್ಗ ದರ್ಶಕವಾಗಿವೆ. ಇವೇ ನಮ್ಮ ಮೂಲಭೂತ ಮಾರ್ಗದರ್ಶಿ ಗಳು; ಬೇರಾವುದೇ ಮಾರ್ಗದರ್ಶನವನ್ನು ಮೂಲಭೂತ ಎನ್ನ ಲಾಗದು. ಇದು ಒಂದು ‘ವೇದಾಂತಿಕ ಭಾರತ’ವನ್ನು ನಿರ್ಮಿ ಸುತ್ತದೆ. ಆ ಆದರ್ಶವನ್ನು ಶ್ರದ್ಧೆಯಿಂದ ಕೈಗೊಂಡದ್ದಾದರೆ ಭಾರತದಲ್ಲಿ ಒಂದು ಹೊಸ ಬಗೆಯ ಗೃಹಸ್ಥ ಜೀವನ ಬೆಳೆಯು ತ್ತದೆ–ಉತ್ಸಾಹಯುತವಾದ, ಶಕ್ತಿಭರಿತವಾದ, ಮಾನವೀಯ ದೃಷ್ಟಿಕೋನವನ್ನುಳ್ಳ ಜೀವನಶೈಲಿ ಅದು.

‘ಗೃಹಸ್ಥ’ನೆಂದರೆ–ಗೃಹದಲ್ಲಿ ವಾಸಿಸುವ ಪತಿ ಅಥವಾ ಪತ್ನಿ. ಅಂದರೆ ಆತನಾಗಲಿ ಆಕೆಯಾಗಲಿ ತಮ್ಮ ತಮ್ಮ ಮಕ್ಕಳ, ಮನೆಮಂದಿಯ ಯೋಗಕ್ಷೇಮವನ್ನಷ್ಟೇ ನೋಡಿಕೊಂಡು ಮನೆ ಯೊಳಗೇ ಇರಬೇಕೆಂದು ಅರ್ಥವೇ? ಇಲ್ಲ; ಹಾಗೇನಾದರೂ ಆಗಿದ್ದರೆ ಮನೆಯೊಂದು ಸೆರೆಮನೆಯಾಗಿಬಿಡುತ್ತಿತ್ತು! ಬದಲಾಗಿ, ಆತನೂ, ಆಕೆಯೂ ತಮ್ಮ ಇಡೀ ಸಮಾಜದ ಹಿತದಲ್ಲಿ ಆಸಕ್ತ ರಾಗಿರಬೇಕು. ಆದರೆ ಇಂದಿನ ಆಧುನಿಕ ಕಾಲದಲ್ಲಿ ಒಂದು ಹೊಸ ಪರಿಸ್ಥಿತಿ ಉದ್ಭವಿಸಿದೆ. ಈಗ ಭಾರತವೊಂದು ಬೃಹತ್ ಪ್ರಜಾಪ್ರಭುತ್ವ; ಇಲ್ಲೀಗ ರಾಜ-ಮಹಾರಾಜರಾಗಲಿ ಚಕ್ರವರ್ತಿ ಗಳಾಗಲಿ ಇಲ್ಲ. ರಾಷ್ಟ್ರದ ಸಾರ್ವಭೌಮತ್ವವಿರುವುದು ಪ್ರಜಾ ಕೋಟಿಯಲ್ಲಿ. ಆ ಪೌರಸ್ವಾತಂತ್ರ್ಯವೆ ಇಂದಿನ ಎಲ್ಲ ಗೃಹಸ್ಥರ –ಸ್ತ್ರೀಪುರುಷರೆಲ್ಲರ–ಲಕ್ಷಣ. ನಮ್ಮ ಪ್ರಜಾಸತ್ತೆ ನಿಂತಿರುವುದು ಅವರ ಭುಜಗಳ ಮೇಲೆ. ಪ್ರತಿಯೊಬ್ಬ ಗೃಹಸ್ಥನೂ ಒಂದು ರಾಷ್ಟ್ರೀಯ ಹೊಣೆಗಾರಿಕೆಯನ್ನು ಭಾವಿಸುವುದರ ಮೂಲಕ ಆ ಸ್ವಾತಂತ್ರ್ಯವನ್ನು ಹೆಚ್ಚು ಅರ್ಥವತ್ತಾಗಿಸಬೇಕು. ಅಂತಹ ಸ್ವತಂತ್ರ ಹಾಗೂ ಜವಾವ್ದಾರಿಯುತ ಪೌರರು ಮಾತ್ರವೇ ಗ್ರಾಮಪಂಚಾಯತಿಯಿಂದ ಮೊದಲುಗೊಂಡು ವಿಧಾನಸಭೆ ಹಾಗೂ ಸಂಸತ್ತಿನವರೆಗಿನ ನಮ್ಮ ವಿವಿಧ ರಾಜಕೀಯ ಸಂಸ್ಥೆ ಗಳನ್ನು, ಎಲ್ಲ ಬಗೆಯ ಸಹಕಾರ ಸಂಸ್ಥೆಗಳನ್ನು ಸಚೇತನ ಗೊಳಿಸಬಲ್ಲರು.

ನಮ್ಮ ಪ್ರತಿಯೊಬ್ಬ ಸ್ತ್ರೀಯ, ಪುರುಷನ, ರಾಷ್ಟ್ರೀಯ ಹೊಣೆ ಗಾರಿಕೆಗಳಲ್ಲಿ ಒಂದೆಂದರೆ ಸ್ವಾತಂತ್ರ್ಯಾನಂತರ ಹಿಡಿತವಿಲ್ಲದೆ ಬೆಳೆಯುತ್ತಿರುವ ಜನಸಂಖ್ಯಾ ಪ್ರಮಾಣವನ್ನು ನಿಯಂತ್ರಿಸು ವುದು. ಈ ಒಂದು ಅಂಶವೇ ನಮ್ಮೆಲ್ಲ ದಾರಿದ್ರ್ಯನಿರ್ಮೂಲನದ ಮತ್ತು ಲೋಕಶಿಕ್ಷಣದ ಕಾರ್ಯಕ್ರಮಗಳನ್ನು ನಿಷ್ಫಲಗೊಳಿಸು ತ್ತಿದೆ. ಆದಷ್ಟು ಬೇಗನೆ ನಾವು ಜನಸಂಖ್ಯೆಯ ಏರಿಕೆ ಮಟ್ಟವನ್ನು ಶೂನ್ಯಕ್ಕೆ ತರಿಸಬೇಕು. ಸ್ವಾತಂತ್ರ್ಯ ಬಂದು ಇಷ್ಟು ವರ್ಷ ಗಳಾದರೂ ನಾವು ಹಿಂದುಳಿದೇ ಇರುವುದು ಈ ಜನಸಂಖ್ಯಾ ಸ್ಪೋಟದಿಂದಾಗಿಯೇ. ಜನರು ಆಧ್ಯಾತ್ಮಿಕ ಹಾಗೂ ವೈದ್ಯಕೀಯ ವಿಧಾನಗಳನ್ನೆಲ್ಲ ಚೆನ್ನಾಗಿ ಬಳಸಿಕೊಂಡು, ತಾವೂ ಚಿಕ್ಕ ಕುಟುಂಬಗಳನ್ನು ಮಾಡಿಕೊಳ್ಳುವುದಲ್ಲದೆ, ಹಾಗೆ ಮಾಡುವಂತೆ ಇತರರನ್ನೂ ಮನವೊಲಿಸಬೇಕು. ಈ ದಿಸೆಯಲ್ಲಿ ಶ್ರಮಿಸು ತ್ತಿರುವ ಕೇಂದ್ರ ಹಾಗೂ ರಾಜ್ಯ ಸರ್ಕಾರಗಳಿಗೆ ಸಕ್ರಿಯವಾಗಿ ನೆರವಾಗುವ ಮೂಲಕ ತಮ್ಮ ಪೌರತ್ವದ ಜವಾಬ್ದಾರಿಯನ್ನು ಈಡೇರಿಸಬೇಕು.

ಇಲ್ಲಿಯವರೆಗೂ ನಮ್ಮ ಗೃಹಸ್ಥರಲ್ಲಿ ಕೆಲವು ದೌರ್ಬಲ್ಯಗಳಿ ದ್ದವು. ಅವುಗಳಲ್ಲಿ ಮುಖ್ಯವಾದದ್ದೆಂದರೆ, ಅವರಲ್ಲಿ ಮನೆ ಮಾಡಿದ್ದ ಅನೇಕ ಮೂಢನಂಬಿಕೆಗಳು. ಮೂಢನಂಬಿಕೆಗಳು ಸುಲಭವಾಗಿ ಸಂಸಾರಿಗಳ ಮನಸ್ಸನ್ನು ಪ್ರವೇಶಿಸಬಲ್ಲವು. ತಮ್ಮ ಹಳ್ಳಿಯ ಮೂಲಕ ಹಾದು ಹೋಗುತ್ತಿರುವ ಯಾವನೋ ಒಬ್ಬ ಸಾಧು ಏನೋ ಹೇಳಿದ; ಸರಿ, ಅಲ್ಲಿಯವರೆಲ್ಲ ಅದನ್ನು ನಂಬಿ ಬಿಡುತ್ತಾರೆ. ಜನರಲ್ಲಿ ಕಾಣಬರುವ ಭಯದ ಮೂಲವೂ ಇದೇ –ಮೌಢ್ಯ. ಒಂದು ಸಣ್ಣ ಅಹಿತಕರ ಘಟನೆ ಕೂಡ ಅವರಲ್ಲಿ ಭಯವನ್ನು ಪ್ರಚೋದಿಸಬಲ್ಲದು. ಭಾರತದಾದ್ಯಂತ ಜನಗಳ ಮನಸ್ಸಿನಲ್ಲಿ ಎಷ್ಟೋ ಮೂಢನಂಬಿಕೆಗಳು ತುಂಬಿಕೊಂಡಿರುವು ದನ್ನು ಕಾಣುತ್ತೇವೆ. ಧರ್ಮವೆಂದರೆ ಅವರ ಕಲ್ಪನೆಯಲ್ಲಿ ಹೆಚ್ಚಾಗಿ ಮಾಯಾಮಂತ್ರಗಳೇ ಹೊರತು ಅಧ್ಯಾತ್ಮವಲ್ಲ. ಮನೆಗೆ ಯಾರೋ ಬಂದರು; ಮಾರನೇ ದಿನ ಮಗುವಿಗೆ ಕಾಯಿಲೆ ಯಾಯಿತು. ಸರಿ; ‘ಓ, ಆ ಮನುಷ್ಯ ಬಂದದ್ದಕ್ಕೇ ನಮ್ಮ ಮಗು ವಿಗೆ ಈ ಕಾಯಿಲೆ’ ಎಂದು ಅದರ ತಾಯ್ತಂದೆಯರು ತೀರ್ಮಾ ನಿಸಿಯೂ ಆಯಿತು! ಇದರಲ್ಲಿ ವೈಜ್ಞಾನಿಕ ಭಾವನೆ ಇರಲಿ, ಒಂದಿಷ್ಟು ಸಾಮಾನ್ಯ ಪರಿಜ್ಞಾನವೂ ಇಲ್ಲ. ಇದೆಲ್ಲ ವಿಜ್ಞಾನ ವಿರುದ್ಧ, ಮೂಢನಂಬಿಕೆ. ಎರಡು ಸಂಗತಿಗಳು ಒಟ್ಟಿಗೆ ಸಂಭವಿಸಿದರೆ ಒಂದಕ್ಕೆ ಮತ್ತೊಂದು ಕಾರಣ ಎಂದು ಅರ್ಥೈಸಿ ಬಿಡುತ್ತಾರೆ. ಇದಕ್ಕೆ ‘ಕಾಕತಾಳೀಯನ್ಯಾಯ’ ಎನ್ನುತ್ತಾರೆ. ಆದರೆ ವೈಜ್ಞಾನಿಕ ತೀರ್ಮಾನಕ್ಕೆ ಬರಬೇಕಾದರೆ ಇನ್ನೊಂದು ವಿಧಾನ ವನ್ನೂ ಅನ್ವಯಿಸಿ ನೋಡಬೇಕು–ಅದು ‘ಭೇದದ ವಿಧಾನ.’ ಒಂದು ನಿರ್ದಿಷ್ಟ ಸಂದರ್ಭದಲ್ಲಿ, ಎರಡು ಅಂಶಗಳ ಪೈಕಿ ಒಂದನ್ನು ಹಿಂತೆಗೆದು ನೋಡಿ; ಆಗಲೂ ಪರಿಣಾಮ ಹಿಂದಿ ನಂತೆಯೆ ಇದ್ದರೆ ನಾವು ಮೊದಲು ಭಾವಿಸಿದ್ದು ತಪ್ಪೆಂದು ಸಾಬೀತಾಗುತ್ತದೆ.

ಈ ಕಾಕತಾಳೀಯನ್ಯಾಯವನ್ನೇ ನೆಚ್ಚಿಕೊಳ್ಳುವುದು ಹೇಗೆ ತಪ್ಪಾಗುತ್ತದೆ ಎಂದು ಒಂದು ಪುಸ್ತಕದಲ್ಲಿ ತಿಳಿಸಿದೆ. ಒಂದು ಊರಿನಲ್ಲಿ ಒಬ್ಬ ಮನುಷ್ಯ ಪ್ರತಿದಿನವೂ ಬೆಳಗ್ಗೆ ಏಳು ಗಂಟೆಗೆ ಅಂಚೆ ಕಚೇರಿಯಿಂದ ಹೊರಬರುತ್ತಿದ್ದನಂತೆ. ಆ ವೇಳೆಗೆ ಸರಿಯಾಗಿ ಸೂರ್ಯೋದಯವಾಗುತ್ತಿತ್ತು. ಇದನ್ನು ಕಂಡು ಕೆಲ ವರು ತೀರ್ಮಾನಿಸಿದರಂತೆ–‘ಓ, ಈತ ಹೊರಬರುವುದರಿಂದಲೆ ಸೂರ್ಯ ಹುಟ್ಟುತ್ತಿರುವುದು’ ಎಂದು! ಇದೆಂಥ ಅತಿ ಮೂರ್ಖ ತೀರ್ಮಾನ! ಇದರ ಸತ್ಯತೆಯನ್ನು ಪರೀಕ್ಷಿಸಬೇಕಾದರೆ, ಹೊರಗೆ ಬಾರದಂತೆ ಆ ವ್ಯಕ್ತಿಗೆ ಹೇಳಿ, ಏಳು ಗಂಟೆಗೆ ಸೂರ್ಯ ಹುಟ್ಟುತ್ತಾನೋ ಇಲ್ಲವೋ ನೋಡಬೇಕು. ಇದೇ ‘ಭೇದ ಮಾರ್ಗ’. ಇದಲ್ಲದೆ ‘ಆನುಷಂಗಿಕ ವ್ಯತ್ಯಯ’ ಹಾಗೂ ‘ಶೇಷದ ವಿಧಾನ’ಗಳೂ ಇವೆ. 

ಇವೆಲ್ಲವೂ, ನಿಜವಾದ ಕಾರ್ಯ-ಕಾರಣ ಸಂಬಂಧಗಳನ್ನು ಗುರುತಿಸಲು ಇರುವ ವೈಜ್ಞಾನಿಕ ವಿಧಾನಗಳು. ಆದರೆ ನಿತ್ಯ ಜೀವನದಲ್ಲಿ ಜನರು, ಅತಿದುರ್ಬಲ ವಿಧಾನವಾದ ಕಾಕ ತಾಳೀಯತೆಯನ್ನೇ ನೆಚ್ಚಿಕೊಳ್ಳುತ್ತಾರೆ. ಸರಿಯಾದ ಶಿಕ್ಷಣ, ಈ ದೋಷವನ್ನು ಕೆಲಮಟ್ಟಿಗೆ ಸರಿಪಡಿಸುತ್ತದೆ. ‘ತಿಳಿವಳಿಕೆ ಭಯ ವನ್ನು ಅಳಿಸುತ್ತದೆ’ ಎಂಬುದೊಂದು ಪ್ರಸಿದ್ಧ ಹೇಳಿಕೆ. ನಮ್ಮ ಜನಗಳಿಗೆ ವೈಜ್ಞಾನಿಕವಾಗಿ ಆಲೋಚಿಸುವ ಶಕ್ತಿಯೇನಾದರೂ ಬಂದರೆ, ಆಗ ನಮ್ಮ ಹಿಂದೂ ಜನಾಂಗ, ಭಾರತೀಯ ಸಮಾಜ, ಸ್ವಲ್ಪ ಬದಲಾಗುತ್ತದೆ. ಆಗ ಮಾತ್ರವೇ ನಮ್ಮ ಜನ ವೇದಾಂತ ವನ್ನು ಅರ್ಥ ಮಾಡಿಕೊಂಡು ಅದರಿಂದ ಲಾಭ ಪಡೆಯಬಲ್ಲರು. ವೇದಾಂತ ಅತ್ಯಂತ ವೈಜ್ಞಾನಿಕವಾದದ್ದು. ‘ವೈಜ್ಞಾನಿಕ’ ಎಂಬುದರ ಅರ್ಥವೇನು? ಯಾವುದು ಸತ್ಯದೊಂದಿಗೆ ಯಥಾವತ್ತಾಗಿ ವ್ಯವ ಹರಿಸುವುದೋ ಅದು ವೈಜ್ಞಾನಿಕ. ಸತ್ಯವನ್ನು ನಾವು ಕಂಡುಕೊಳ್ಳಬಲ್ಲೆವಷ್ಟೇ; ಅದನ್ನು ಸೃಷ್ಟಿಸಲಾರೆವು, ಬದಲಿಸ ಲಾರೆವು, ನಾಶಮಾಡಲಾರೆವು. ಅದನ್ನು ಕೇವಲ \textbf{ಗುರುತಿಸ}ಬಲ್ಲೆವು. ಆದಿಶಂಕರಾಚಾರ್ಯರು ತಮ್ಮ ಬ್ರಹ್ಮಸೂತ್ರಭಾಷ್ಯದಲ್ಲಿ. ಅದನ್ನು \textbf{‘ವಸ್ತುತಂತ್ರಜ್ಞಾನ’} ಎನ್ನುತ್ತಾರೆ. ಅವರು ಹೇಳುತ್ತಾರೆ: \textbf{‘ಕರ್ತುಂ, ಅಕರ್ತುಂ, ಅನ್ಯಥಾಕರ್ತುಂ ನ ಶಕ್ಯತೇ, ವಸ್ತುತಂತ್ರ ತ್ವಾತ್ ಏವ’}; ಮತ್ತೆ \textbf{‘ಬ್ರಹ್ಮಜ್ಞಾನಂ ವಸ್ತುತಂತ್ರಜ್ಞಾನಮ್​’} ಎಂದರೆ, ‘ಬ್ರಹ್ಮನ ಕುರಿತಾದ ಜ್ಞಾನವು ಅದಾಗಲೇ ಅಸ್ತಿತ್ವ ದಲ್ಲಿರುವ ಬ್ರಹ್ಮನ ಸತ್ಯವನ್ನು ಆಧರಿಸಿದೆ,’ ಎಂದರ್ಥ. ‘ಬೆಂಕಿ ಬಿಸಿಯಾಗಿದೆ’ ಎನ್ನುವುದೊಂದು ಸತ್ಯ, ಅದು \textbf{ಅಭಿಪ್ರಾಯ}ವಲ್ಲ. ಹೆಚ್ಚಿನ ಜನರಿಗೆ ಅಭಿಪ್ರಾಯಗಳು ಮಾತ್ರ ಇವೆ; ವಾಸ್ತವಾಂಶ ವೇನೆಂದು ಅವರಿಗೆ ತಿಳಿಯದು. ಶ್ರೀ ಶಂಕರಾಚಾರ್ಯರ ಮಾತನ್ನು ಪುನಃ ಉದ್ಧರಿಸುತ್ತೇನೆ: \textbf{‘ಬ್ರಹ್ಮಜ್ಞಾನಂ ವಸ್ತುತಂತ್ರ ಜ್ಞಾನಮ್​’.} ವೇದಾಂತವು ಇದನ್ನು \textbf{ಸತ್ಯ}ವೆಂಬಂತೆ ನಮ್ಮ ಮುಂದಿರಿಸುತ್ತದೆ: \textbf{ಅಭಿಪ್ರಾಯ}ವೆಂದಲ್ಲ. ಅದು ಬಳಸುವ ಶಬ್ದ \textbf{‘ವಸ್ತುತಂತ್ರಜ್ಞಾನಮ್​’}. ಎಂಥ ತಾಂತ್ರಿಕ \eng{(technical)} ಶಬ್ದ ನೋಡಿ! ‘ವಸ್ತು’ವೆಂದರೆ ಇರುವಂಥದ್ದು, ತಥ್ಯ. ‘ತಂತ್ರ’ ವೆಂದರೆ ‘ಆಧರಿಸಿದೆ’ ಎಂದು. ಒಟ್ಟಿನಲ್ಲಿ, ‘ಈಗಾಗಲೇ ಇರು ವಂಥ ವಾಸ್ತವಿಕತೆಯನ್ನು ಆಧರಿಸಿದಂತಹ ಜ್ಞಾನ’. ಎಂಥ ಸುಂದರ ಕಲ್ಪನೆ! ಆದ್ದರಿಂದ ಅವರೆನ್ನುತ್ತಾರೆ:

\begin{verse}
‘ಬ್ರಹ್ಮಜ್ಞಾನಂ ವಸ್ತುತಂತ್ರಜ್ಞಾನಮ್ ।\\ಕರ್ತುಂ ಅಕರ್ತುಂ ಅನ್ಯಥಾ ಕರ್ತುಂ ನ ಶಕ್ಯತೇ, ವಸ್ತುತಂತ್ರತ್ವಾದೇವ ।’
\end{verse}

‘ಬ್ರಹ್ಮಜ್ಞಾನವು ಅದಾಗಲೇ ಇರುವ ಬ್ರಹ್ಮದ ಸತ್ಯವನ್ನು ಆಧರಿಸಿದೆ. ಅದನ್ನು ಸೃಷ್ಟಿಸುವುದಾಗಲಿ, ಬದಲಿಸುವುದಾಗಲಿ, ನಾಶಪಡಿಸುವುದಾಗಲಿ ಸಾಧ್ಯವಿಲ್ಲ. ಏಕೆಂದರೆ ಅದೊಂದು ಅಸ್ತಿತ್ವದಲ್ಲಿರುವ ಸತ್ಯ’.

ಇದನ್ನೇ ವೈಜ್ಞಾನಿಕ ಸತ್ಯವೆನ್ನುವುದು. ಮತ್ತೊಂದು ಬಗೆಯ ಜ್ಞಾನವಿದೆ: \textbf{‘ಪುರುಷತಂತ್ರಜ್ಞಾನ’}. ‘ನಾನು ಸೋಮವಾರಗಳಂದು ಉಪವಾಸ ಮಾಡುತ್ತೇನೆ’ಎನ್ನಬಹುದು ನೀವು. ಅದು ನಿಮಗೆ ಬಿಟ್ಟದ್ದು. ಅದರಲ್ಲಿ ವಸ್ತುಶಃ ಅಥವಾ ಸಾರ್ವತ್ರಿಕ ಸತ್ಯವೇನಿಲ್ಲ. ಬದಲಾಗಿ ನೀವು ‘ನಾನು ಶನಿವಾರಗಳಂದು ಉಪವಾಸ ಮಾಡುತ್ತೇನೆ’ ಎಂದು ಬೇಕಾದರೂ ಹೇಳಬಹುದು. ನಿಮ್ಮಿಷ್ಟ! ಇದಕ್ಕೇ \textbf{‘ಪುರುಷತಂತ್ರಜ್ಞಾನ’} ಎನ್ನುವುದು. ಎಂದರೆ ‘ಸಂಬಂಧಿ ಸಿದ ವ್ಯಕ್ತಿಯನ್ನಾಧರಿಸಿದ ಸತ್ಯ’. ಬ್ರಹ್ಮಸೂತ್ರಭಾಷ್ಯದ ಆ ಭಾಗದ ಕೊನೆಯಲ್ಲಿ ಶಂಕರ ಭಗವತ್ಪಾದರು ಹೇಳುತ್ತಾರೆ:

\begin{verse}
‘ಆತ್ಮೈಕತ್ವವಿದ್ಯಾಪ್ರತಿಪತ್ತಯೇ ಸರ್ವೇ ವೇದಾಂತಾ ಆರಭ್ಯಂತೇ’
\end{verse}

‘ಎಲ್ಲ ಉಪನಿಷತ್ತುಗಳ ಉದ್ದಿಶ್ಯವೂ ನಮಗೆ ಆತ್ಮನ ಏಕತ್ವ ವನ್ನು ಮನಗಾಣಿಸುವುದೇ.’

ಇರುವುದೊಂದೇ–ಶುದ್ಧ ಚಿತ್ (ಪ್ರಜ್ಞೆ) ಸ್ವರೂಪನಾದ ಆತ್ಮ. ಪ್ರಜ್ಞೆಗೆ ನಾನಾತ್ವ ಅಥವಾ ಬಹುತ್ವವಿಲ್ಲ; ಅದು ಸದಾ ಏಕ. ಇದು ಉಪನಿಷತ್ತುಗಳ ಒಂದು ಅದ್ಭುತ ನುಡಿ. ಇಂದಿನ ಪರಮಾಣು ವಿಜ್ಞಾನಿ ಶ್ರೋಡಿಂಗರ್ ಹೇಳುವ ಈ ಮಾತು ಅದನ್ನು ಪುಷ್ಟೀಕರಿಸುತ್ತದೆ: ‘ಪ್ರಜ್ಞೆ ಏಕವೇ ಹೊರತು ಅನೇಕ ವಲ್ಲ’. ಅದು ಆಕಾಶದಂತೆ. ಆಕಾಶ ಒಂದೇ–ಕೋಣೆಯ ಒಳ ಗಾಗಲಿ ಹೊರಗಾಗಲಿ, ಇರುವುದೆಲ್ಲ ಒಂದೇ ಆಕಾಶ. ನಾವು ಅದನ್ನು ಕೋಣೆಯ ಮೂಲಕ ವಿಭಜಿಸಿದ್ದೇವೆ ಎಂದು ಬಾವಿಸ ಬಹುದು; ಆದರೆ ಅದು ಸಾಧ್ಯವಿಲ್ಲ. ಹಾಗೆಯೇ ಆತ್ಮನು ಶುದ್ಧ ಚಿತ್ ರೂಪದಲ್ಲಿ ಒಬ್ಬನೆ–ಒಂದೇ ಆಗಿ ನನ್ನ ನಿಮ್ಮ ಎಲ್ಲರ ಒಳಗೂ ಇದ್ದಾನೆ. ಇದೇ, ಉಪನಿಷತ್ತುಗಳು ವಿಶ್ವದೆಲ್ಲೆಡೆ ಇರುವ ಸರ್ವರಿಗೂ ತಿಳಿಸುವ ಬೋಧನೆ. ಎಂಥ ಗಹನ ತತ್ತ್ವ! ಈ ತತ್ತ್ವವನ್ನು ನಮ್ಮ ವೈಯಕ್ತಿಕ ಜೀವನಕ್ಕೆ–ಸಾಮೂಹಿಕ ಜೀವನಕ್ಕೆ–ಅನ್ವಯಿಸಿಕೊಂಡಾಗ ಇನ್ನೆಂಥ ಮಹತ್ತರ ಸಂಗತಿ ಗಳು ಸಿದ್ಧಿಸಬಹುದು! ಈಗೆಲ್ಲ ತಂತ್ರಜ್ಞಾನದ ಮೂಲಕ ಭೌತಿಕ (ಬಾಹ್ಯ) ಏಕತೆ ಸಾಧ್ಯವಾಗುತ್ತಿದೆ. ಇಂದು ನೀವು ಬಹುವೇಗ ವಾಗಿ ಪ್ರಯಾಣಿಸಬಹುದು. ಈಸ್ಟ್ ಇಂಡಿಯಾ ಕಂಪನಿಯ ಜನ ಇಂಗ್ಲೆಂಡಿನಿಂದ ಭಾರತಕ್ಕೆ ಬರಬೇಕಾದರೆ ಒಂದೂವರೆ ವರ್ಷ ತೆಗೆದುಕೊಳ್ಳುತ್ತಿದ್ದರು. ಆದರೆ ಈ ದಿನ ಕೇವಲ ಐದೋ ಹತ್ತೋ ತಾಸುಗಳಲ್ಲಿ ಈ ಅಂತರವನ್ನು ಕ್ರಮಿಸಬಹುದು. ಹಾಗೆಯೇ ಕಲ್ಪನೆಗಳ, ಭಾವನೆಗಳ ಸಂವಹನವೂ ಅಷ್ಟೆ: ಟೆಲಿ ಗ್ರಾಮ್​ಗಳಾದವು; ಈಗ ಫ್ಯಾಕ್ಸ್. ಒಂದೆರಡು ನಿಮಿಷದಲ್ಲೆಲ್ಲ ನಿಮ್ಮ ಸುದ್ದಿ ದೇಶಾಂತರವನ್ನು ತಲುಪಿರುತ್ತದೆ. ಆದರೆ ಇಷ್ಟು ಸಾಲದು; ಜನರ ಮನಸ್ಸು-ಹೃದಯಗಳೂ ಹತ್ತಿರವಾಗಬೇಕು. ಅದಿನ್ನೂ ಸಾಧ್ಯವಾಗಿಲ್ಲ. ಅದು ಸಾಧ್ಯವಾಗುವುದು, ಎಲ್ಲವನ್ನೂ ಒಗ್ಗೂಡಿಸುವ ವೇದಾಂತದ ಸಂದೇಶದಿಂದ. ವೇದಾಂತ ವೆಂಬುದು ಮಾನವ ವಿಜ್ಞಾನದ ತೀವ್ರ ಶೋಧನೆಯಿಂದ ಒಡ ಮೂಡಿದ ಫಲ. ಅದನ್ನು ಕೈಗೆತ್ತಿಕೊಳ್ಳಿ; ನೀವೇ ಪರೀಕ್ಷಿಸಿ ನೋಡಿ. ಉಪನಿಷತ್ ಪುಷಿಗಳು ಹೇಳಿದ ಮಾತು ಇದು:

\begin{verse}
‘ವೇದಾಹಮೇತಂ ಪುರುಷಂ ಮಹಾಂತಮ್ ।’
\end{verse}

‘ಸಾಂತ (ಅಲ್ಪ) ಮಾನವನ ಹಿಂದಿರುವ ಈ ಅನಂತ ಮಾನವನನ್ನು ನಾನು ಸಾಕ್ಷಾತ್ಕರಿಸಿಕೊಂಡಿದ್ದೇನೆ.’ ಮತ್ತು

\begin{verse}
‘ತಮೇವ ವಿದಿತ್ವಾ ಅತಿ ಮೃತ್ಯುಮೇತಿ\\ನಾನ್ಯಃ ಪಂಥಾ ವಿದ್ಯತೇ ಅಯನಾಯ–’
\end{verse}

‘ಆತನನ್ನು ಅರಿಯುವ ಮೂಲಕ ಮಾತ್ರವೇ ಮೃತ್ಯುವನ್ನೂ ವಿಭ್ರಮೆಯನ್ನೂ ಗೆಲ್ಲಲು ಸಾಧ್ಯ. ಮುಕ್ತಿಗೆ, ಧನ್ಯತೆಗೆ ಇದಲ್ಲದೆ ಬೇರೆ ದಾರಿಯಿಲ್ಲ.’

ನೀವು ಸಾವು-ಭ್ರಮೆಗಳನ್ನು ಮೀರಿ ಹೋಗಬಲ್ಲಿರಿ. ಆದರೆ ಈ ಸತ್ಯವನ್ನು ನೀವೇ ಸಾಕ್ಷಾತ್ಕರಿಸಿಕೊಳ್ಳಬೇಕು. ಬೇರೆ ಯಾರೋ ಸಾಕ್ಷಾತ್ಕಾರ ಮಾಡಿಕೊಂಡರೆ ನೀವು ಮಾಡಿಕೊಂಡಂತಾಗು ವುದಿಲ್ಲ. ‘ವಿವೇಕ ಚೂಡಾಮಣಿ’ಯಲ್ಲಿ ಶಂಕರಾಚಾರ್ಯರು ಹೇಳುತ್ತಾರೆ: ‘ನಿಮಗೆ ಹಸಿವಾಗಿದ್ದರೆ ನೀವೇ ತಿನ್ನಬೇಕು. ನಿಮ್ಮ ಪರವಾಗಿ ಬೇರೆ ಯಾರೋ ತಿಂದರೆ ನಿಮಗೇನೂ ಲಾಭವಿಲ್ಲ.’ ಆತ್ಮವೆಂಬ ಆ ಪರಮಸತ್ಯವನ್ನು ನೀವೇ ಕಂಡುಕೊಳ್ಳಬೇಕು. ವೇದಾಂತದಲ್ಲಿ ಇದನ್ನು ಮತ್ತೆ ಮತ್ತೆ ಹೇಳಲಾಗಿದೆ. 

ನಮಗೆ ನಮ್ಮ ಜೀವನದಲ್ಲಿ, ಕಲಿತು ಅಳವಡಿಸಿಕೊಳ್ಳಲು ಇಂತಹ ಸುಂದರ ವಿಚಾರಗಳೆಲ್ಲ ಇವೆ. ಆದರೆ ಕಳೆದ ಒಂದು ಸಾವಿರ ವರ್ಷಗಳಲ್ಲಿ ಇವುಗಳಲ್ಲಿ ಒಂದು ತುಣುಕನ್ನೂ ನಾವು ಮುಟ್ಟಲಿಲ್ಲ. ಏನೋ ಒಂದಿಷ್ಟು ಮೂಢ ನಂಬಿಕೆ, ಒಂದಿಷ್ಟು ಕಂತೆ ಪುರಾಣ–ಇಷ್ಟನ್ನೇ ಹಿಡಿದುಕೊಂಡೆವು. ಅದರಲ್ಲೂ ಕಂತೆ ಪುರಾಣಗಳೆಂದರೆ ನಮಗೆ ಬಹಳ ಪ್ರೀತಿ. ಭಾರತದಲ್ಲಿ ಸೃಷ್ಟಿ ಯಾಗಿರುವ ಇಂಥ ಸಾಹಿತ್ಯದ ಪ್ರಮಾಣ ಜಗತ್ತಿನಲ್ಲೇ ಅತ್ಯಧಿಕ. ವಿವೇಕಾನಂದರು ಹೇಳುತ್ತಿದ್ದರು, ನಮ್ಮ ಪುರಾಣಸಾಹಿತ್ಯದಿಂದ ಜಗತ್ತಿನ ಗ್ರಂಥಾಲಯಗಳನ್ನೆಲ್ಲ ತುಂಬಿಸಿಬಿಡಬಹುದು ಎಂದು. ಇರಲಿ, ಸ್ಪಲ್ಪ ಪುರಾಣ-ಪುಣ್ಯಕತೆ ಉಳಿದಿರಲಿ; ಅದರಲ್ಲೂ ಕೆಲವು ಅಂಶಗಳಿಗೆ ವೈಜ್ಞಾನಿಕ ಹಿನ್ನೆಲೆಯಿದೆ. ಆದರೆ ಉನ್ನತ ಹಂತಗಳಲ್ಲಿ ವಿಜ್ಞಾನವೇ ಪೌರಾಣಿಕವಾಗುತ್ತದೆ–ವಿಶೇಷತಃ ಖಗೋಳ ವಿಜ್ಞಾನದಲ್ಲಿ; ಆ ವಿಚಾರ ಬೇರೆ.

ನಮ್ಮ ಜನಗಳಲ್ಲಿ ಈ ವೈಜ್ಞಾನಿಕ ಮನೋಭಾವ, ವೈಜ್ಞಾನಿಕ ಪ್ರವೃತ್ತಿ ಬೆಳೆಯಬೇಕು; ತನ್ಮೂಲಕ ಅವರು ಪ್ರಶ್ನಿಸಿಕೊಳ್ಳು ವಂತಾಗಬೇಕು–‘ಈ ನನ್ನ ಅದ್ಭುತ ಜೀವನದಿಂದ ನಾನೇನು ಮಾಡಬಲ್ಲೆ? ನನ್ನೊಳಗೆ ಚೈತನ್ಯದ ನಿಧಿಯೊಂದು ಅಡಗಿದೆ; ಅದನ್ನು ಹೇಗೆ ಬಳಸಿಕೊಳ್ಳಲಿ?’ ಎಂದು. ಉಪನಿಷತ್ತು- ಗೀತೆಗಳಲ್ಲಿ ನಮಗೆ ಈ ಬಗ್ಗೆ ನಿರ್ದೇಶನ ಸಿಗುತ್ತದೆ. ಪ್ರತಿ ಯೊಬ್ಬನಲ್ಲೂ ಮೂರು ಬಗೆಯ ಶಕ್ತಿಗಳಿವೆ ಎಂದು ನಮ್ಮ ಶಾಸ್ತ್ರಗಳು ಹೇಳುತ್ತವೆ. ಮೊದಲನೆಯದು ‘ಬಾಹುಬಲ’,ಎಂದರೆ ಭೌತಿಕ ಶಕ್ತಿ. ಈ ಬಲ ತೀರ ಸಾಮಾನ್ಯವಾದದ್ದು. ರಾಕೆಟ್ಟು ಗಳಲ್ಲಿ ಇದರ ಕೋಟಿಪಾಲು ಶಕ್ತಿಯನ್ನು ಉತ್ಪಾದಿಸಿದ್ದೇವೆ. ಇವು ಮಾನವನನ್ನು ಚಂದ್ರನ ಮೇಲಕ್ಕೂ ಕಳಿಸಬಲ್ಲವು. ‘ವಾಯೇಜರ್​’ ನೌಕೆಯಂತೂ ಆಗಲೆ ಸೌರವ್ಯೂಹದ ಆಚೆಗೂ ಹೋಗಿದೆ. ಇದೆಲ್ಲ ‘ಬಾಹುಬಲ’ವೇ. ತಂತ್ರಜ್ಞಾನದ ಮೂಲಕ ಪ್ರಚಂಡವಾಗಿ ವೃದ್ಧಿಗೊಂಡಿರುವಂಥದು. ಎರಡನೆಯದು ‘ಬುದ್ಧಿಬಲ’. ನಾವು ವಿಶ್ವವಿದ್ಯಾನಿಲಯಕ್ಕೆ ಹೋಗುತ್ತೇವೆ, ಪುಸ್ತಕಗಳನ್ನು ಓದುತ್ತೇವೆ, ವಿಜ್ಞಾನ ಇತ್ಯಾದಿಯನ್ನೆಲ್ಲ ಕಲಿಯು ತ್ತೇವೆ, ತನ್ಮೂಲಕ ಜ್ಞಾನಶಕ್ತಿಯನ್ನು ಬುದ್ಧಿಬಲವನ್ನು ವರ್ಧಿಸಿ ಕೊಳ್ಳುತ್ತೇವೆ. ಆಯಿತು; ಆದರೆ ಇಷ್ಟೇನೇ? ‘ಬಲ’ವೆಂದರೆ ಮತ್ತಿ ನ್ನೇನಿದೆ? ಇಂದಿನ ನಮ್ಮ ಸಾಮಾನ್ಯ ತಿಳಿವಳಿಕೆ ಇಷ್ಟೇ– ಬಾಹುಬಲ, ಬುದ್ಧಿಬಲ. ಆದರೆ ನಮ್ಮ ಶಾಸ್ತ್ರಗಳು ಹೇಳುತ್ತವೆ: ಅಷ್ಟೇ ಅಲ್ಲ, ಅವುಗಳೊಂದಿಗೆ ‘ಆತ್ಮಬಲ’ವಿದೆ– ಆಧ್ಯಾತ್ಮಿಕ ಶಕ್ತಿಯಿದೆ. ಅದು ಪ್ರಚಂಡವಾದದ್ದು, ಸರಿಸಾಟಿಯಿಲ್ಲದ್ದು.

ಸರಿ, ಈ ಆತ್ಮಬಲದ ಬಗ್ಗೆ ನಾವು ಅರಿಯುವುದು ಹೇಗೆ? ಆಧುನಿಕ ಮಾನವ ಜನಾಂಗವು ಉತ್ತರ ಹುಡುಕಿಕೊಳ್ಳಬೇಕಾದ ಪ್ರಮುಖ ಸಮಸ್ಯೆ ಇದು. ಇಲ್ಲಿಯವರೆಗೂ–ಆತ್ಮಬಲದ ವರೆಗೂ–ನಾವು ಸುಗಮವಾಗಿ ಬಂದಿದ್ದೇವೆ. ಆದರೆ, ಭೌತಿಕ ಅಥವಾ ಇಂದ್ರಿಯಗಳ ಮಟ್ಟದ ಆಚೆಗೆ ನಮಗೆ ಏನೂ ಗೊತ್ತಿಲ್ಲ. ಈ ಬಗ್ಗೆ ಇಂದಿನ ವಿಜ್ಞಾನ ಏನೂ ಹೇಳಲಾರದು. ಆದರೆ ನಾವು ಪಡೆದುಕೊಳ್ಳಬೇಕಾದ ಅತಿ ಮುಖ್ಯಶಕ್ತಿ–ಆತ್ಮ ಬಲ–ಕಾದು ಕುಳಿತಿದೆ. ಮನುಷ್ಯನಿಗೆ ಎಲ್ಲ ನಿಟ್ಟಿನಿಂದಲೂ ದಿನ ದಿನವೂ ಹೊಸಹೊಸ ಪ್ರಲೋಭನೆಗಳು ಎದುರಾಗುತ್ತಿವೆ. ಅವನಿಗೆ ಅವುಗಳನ್ನು ಎದುರಿಸಲು ಶಕ್ತಿಯಿಲ್ಲವಾಗಿದೆ. ಬುದ್ಧಿ ಬಲಕ್ಕೆ ಆ ಸಾಮರ್ಥ್ಯವಿಲ್ಲ. ಅದು ಆತ್ಮಬಲದಿಂದ ಮಾತ್ರ ಸಾಧ್ಯ. ಸ್ವಲ್ಪ ಆತ್ಮಬಲವೊಂದಿದ್ದರೆ ಸಾಕು, ಎಲ್ಲ ಪ್ರಲೋಭನೆ ಗಳನ್ನೂ ನಾವು ಗೆಲ್ಲಬಲ್ಲೆವು. ಸಮಾಜದ ಗಣ್ಯವ್ಯಕ್ತಿಗಳು, ಸಾಮಾನ್ಯ ವ್ಯಕ್ತಿಗಳು ಎಲ್ಲರೂ ಪ್ರಲೋಭನೆಗಳಿಗೆ ಬಲಿಯಾಗು ತ್ತಿರುವುದರಿಂದ ಪ್ರತಿದಿನವೂ ನಮ್ಮ ಸಮಾಜ ದುಸ್ಥಿತಿಯನ್ನು ಅನುಭವಿಸುತ್ತಿದೆ. ಈ ಪ್ರಲೋಭನೆಗಳಿಂದ ಸ್ತ್ರೀಯರು-ಪುರು ಷರು, ಹೆಣ್ಣುಮಕ್ಕಳು-ಗಂಡುಮಕ್ಕಳು, ಎಲ್ಲರೂ ಕಷ್ಟಕ್ಕೀಡಾಗಿ ದ್ದಾರೆ. ಸಮಾಜಘಾತಕವಾದ ಈ ಪ್ರಲೋಭನೆಗಳನ್ನು ನಿರೋಧಿ ಸುವ ನೈತಿಕಬಲ ಕುಗ್ಗುತ್ತಿರುವುದನ್ನು ನಾವು ಎಲ್ಲೆಡೆಯೂ ಕಾಣು ತ್ತಿದ್ದೇವೆ. ಅದರಿಂದಾಗಿ ಹೆಚ್ಚು ಹೆಚ್ಚು ಸಾಮಾಜಿಕ ಸಮಸ್ಯೆ ಗಳನ್ನು ಎದುರಿಸಬೇಕಾಗಿ ಬರುತ್ತಿದೆ. ಆದ್ದರಿಂದ ನಮ್ಮ ಮನಸ್ಸುಗಳನ್ನೂ ಇಂದ್ರಿಯಗಳನ್ನೂ ಹಿಡಿತದಲ್ಲಿ ಇಟ್ಟು ಕೊಳ್ಳುವ ಸಲುವಾಗಿ ನಾವು ಕಿಂಚಿತ್ ಆತ್ಮಬಲವನ್ನು ಬೆಳೆಸಿ ಕೊಳ್ಳಬೇಕಾಗಿದೆ. ಸರಿಯಾಗಿ ನಿಯಂತ್ರಿಸದಿದ್ದರೆ ಈ ಇಂದ್ರಿಯ ಗಳು ಬಹಳಷ್ಟು ತೊಂದರೆ ಕೊಡಬಲ್ಲವು.

ನಮ್ಮ ಪುರಾತನ ಗುರುಗಳು ಒಂದು ಸುಂದರ ವಿಚಾರ ಧಾರೆಯನ್ನು ನಮ್ಮ ಮುಂದಿರಿಸಿದ್ದಾರೆ. ಪ್ರತಿಯೊಬ್ಬ ವ್ಯಕ್ತಿಗೂ ಆರು ಜನ ಶತ್ರುಗಳು–ಷಡ್ರಿಪುಗಳು–ಇದ್ದಾರೆ ಎಂದಿದ್ದಾರೆ ಅವರು. ಆ ಶತ್ರುಗಳು ನಮ್ಮ ಹೊರಗಡೆ ಇಲ್ಲ. ನಮ್ಮೊಳಗೇ ಇವೆ. ಅವು ಯಾವುವು? ಕಾಮ, ಕ್ರೋಧ, ಲೋಭ, ಮೋಹ, ಮದ, ಮಾತ್ಸರ್ಯ. ಇವೇ ಮನುಷ್ಯನಲ್ಲಿರುವ ಆರು ವೈರಿಗಳು: ಇದು ಮಾನವನ ಮನಸ್ಸಿನ ನಿಷ್ಕೃಷ್ಟವಾದ ವಿಶ್ಲೇಷಣೆ. ಎಲ್ಲ ತೊಂದರೆಗಳಿಗೂ ಕಾರಣ ಈ ಶತ್ರುಗಳೇ. ಮನುಷ್ಯರ ನಡುವಿನ ಬಾಂಧವ್ಯವನ್ನು ಕೆಡಿಸುತ್ತಿರುವ ಯಾವುದೇ ಸಮಸ್ಯೆಯನ್ನು ನೋಡಿ–ಅದು ಹೆಚ್ಚಿನ ವೇಳೆ, ಕಾಮ-ಕ್ರೋಧ-ಲೋಭ ಇವು ಗಳ ಪೈಕಿ ಒಂದು, ಎರಡು ಇಲ್ಲವೆ ಮೂರೂ ಅಂಶಗಳ ಒಕ್ಕೂಟದಿಂದ ಉಂಟಾದಂಥವು. ಇವನ್ನು ನಾವು ಅಂಕೆಯ ಲ್ಲಿಡಬೇಕು. ಆ ಕೆಲಸ ಮಾಡಬೇಕಾದದ್ದು ಯಾರು? ಮನಸ್ಸು ಆ ಕೆಲಸ ಮಾಡಬೇಕು. ಮನಸ್ಸಿರುವುದೇ ಅದಕ್ಕಾಗಿ. ಆದರೆ ಮನಸ್ಸು ದುರ್ಬಲವಾಗಿದ್ದರೆ ಆ ಶತ್ರುಗಳನ್ನು ನಿಯಂತ್ರಿಸುವ ಬದಲು ಅವನ್ನೇ ಅನುಸರಿಸುತ್ತದೆ. ಆಗ ಎಲ್ಲ ಪ್ರಲೋಭನೆಗಳಿಗೂ ನಾವು ಈಡಾಗುತ್ತೇವೆ. ಅನೇಕ ಜನರ ವಿಷಯದಲ್ಲಿ ಆಗುವುದು ಇದೇ. ನರವಿಜ್ಞಾನವನ್ನು ಓದಿದರೆ ನಮಗೆ ತಿಳಿಯುತ್ತದೆ. ನಮ್ಮ ಮೆದುಳಿನ ವ್ಯವಸ್ಥೆ ಇರುವುದು ಇಡೀ ಜ್ಞಾನೇಂದ್ರಿಯ ಸಮೂಹ ವನ್ನು ನಿಯಂತ್ರಿಸಲು ಎಂದು. ಮಾನವನಿಗೆ ಇದು ಜೀವ ವಿಕಾಸದ ವಿಶೇಷ ಉಡುಗೊರೆ. ಆದರೆ ಈ ಉನ್ನತ ಮಸ್ತಿಷ್ಕವು ಇಂದ್ರಿಯಸಮೂಹದ ಅಡಿಯಾಳಾದರೆ ನೈತಿಕ ಮೌಲ್ಯಗಳೆಲ್ಲ ಕೊಚ್ಚಿಹೋಗುತ್ತವೆ. ಆಗ ನಾಯಿ ತನ್ನ ಬಾಲವಾಡಿಸುವ ಬದಲು ಬಾಲವೇ ನಾಯಿಯನ್ನು ಆಡಿಸತೊಡಗುತ್ತದೆ. ಕ್ರಮೇಣ ಹೆಚ್ಚು ಹೆಚ್ಚು ಜನರಿಗೆ ಹೀಗಾಗುತ್ತ ಬರುತ್ತಿದೆ. ನಮ್ಮಲ್ಲಿ ಸಾಮಾಜಿಕ ಸಮಸ್ಯೆಗಳೂ ಸಂಕಷ್ಟಗಳೂ ಇಷ್ಟೊಂದು ವೃದ್ಧಿಯಾಗುತ್ತಿರು ವುದಕ್ಕೆ ಇದೇ ಕಾರಣ. ಆದ್ದರಿಂದ ಈ ಉನ್ನತ ಮಸ್ತಿಷ್ಕವೆಂಬ ನಿಯಂತ್ರಣ ವ್ಯವಸ್ಥೆ ಸ್ವತಂತ್ರವಾಗಬೇಕು; ಇಂದ್ರಿಯಸಮೂಹ ವನ್ನು ನಿಯಂತ್ರಿಸುವಷ್ಟು ಶಕ್ತವಾಗಬೇಕು. ಆಗ ಆ ಶತ್ರುಗಳು ತಲೆಯೆತ್ತಿ ವ್ಯಕ್ತಿಯನ್ನೂ ಸಮಾಜವನ್ನೂ ಕಾಡಲು ಆಗುವುದಿಲ್ಲ. ಅವುಗಳ ಹತೋಟಿ ಮಾನವನಿಂದ ಮಾತ್ರ ಸಾಧ್ಯ. ಗೀತೆಯ ಮೂರನೇ ಅಧ್ಯಾಯದಲ್ಲಿ ಈ ಕುರಿತಾಗಿ ಅರ್ಜುನ ಒಂದು ಪ್ರಶ್ನೆ ಕೇಳುತ್ತಾನೆ; ಅದಕ್ಕೆ ಶ್ರೀಕೃಷ್ಣ ಅತ್ಯಂತ ಅರ್ಥಗರ್ಭಿತವಾದ ಉತ್ತರ ಕೊಡುತ್ತಾನೆ. ಅರ್ಜುನನ ಪ್ರಶ್ನೆ ಇದು (೩.೩೬):

\begin{verse}
‘ಅಥ ಕೇನ ಪ್ರಯುಕ್ತೋಽಯಂ ಪಾಪಂ ಚರತಿ ಪೂರುಷಃ ।\\ಅನಿಚ್ಛನ್ನಪಿ ವಾರ್ಷ್ಣೇಯ ಬಲಾದಿವ ನಿಯೋಜಿತಃ ॥’
\end{verse}

‘ಓ ಕೃಷ್ಣ, ಮನುಷ್ಯನು ತನ್ನ ಇಚ್ಛೆಗೂ ವಿರುದ್ಧವಾಗಿ, ಯಾವುದೋ ಶಕ್ತಿಯ ಒತ್ತಡಕ್ಕೊಳಗಾದವನಂತೆ ದುಷ್ಕಾರ್ಯ ಮಾಡುತ್ತಾನಲ್ಲ ಅದು ಯಾವ ಶಕ್ತಿ?’ ಈ ಸಮಸ್ಯೆ ಪ್ರತಿಯೊಬ್ಬ ಮನುಷ್ಯನದೂ ಕೂಡ. ಇದಕ್ಕೆ ಶ್ರೀಕೃಷ್ಣ ಉತ್ತರಿಸುತ್ತಾನೆ:

‘ಕಾಮ, ಕ್ರೋಧ–ಇವೇ ಆ ಎರಡು ದುಷ್ಟಶಕ್ತಿಗಳು. ಅವು ನಿನ್ನನ್ನು ಮಣಿಸಿ, ನಿನ್ನಿಂದ ಅಕೃತ್ಯಗಳನ್ನೆಲ್ಲ ಮಾಡಿಸುತ್ತವೆ. ನೀನು ಅವುಗಳನ್ನು ಹತೋಟಿಗೆ ತಂದುಕೊಳ್ಳಬೇಕು. ಆ ಶಕ್ತಿ ನಿನಗಿದೆ.’ ಎಲ್ಲಿಂದ ಬರುತ್ತದೆ ಆ ಶಕ್ತಿ? ಅದೇ ಅಧ್ಯಾಯದ ಕೊನೆಯ ಕೆಲವು ಶ್ಲೋಕಗಳಲ್ಲಿ ಕೃಷ್ಣ ಉತ್ತರಕೊಡುತ್ತಾನೆ: \textbf{‘ಇಂದ್ರಿಯಾಣಿ ಪರಾಣ್ಯಾಹುಃ’} ‘ಇಂದ್ರಿಯಗಳು ಅತ್ಯಂತ ಸೂಕ್ಷ್ಮಗ್ರಾಹಿಗಳು ಹಾಗೂ ಅತಿ ಉಪಯುಕ್ತವಾದವುಗಳು.’ ಅವುಗಳ ಮೂಲಕ ನಾವು ಸುತ್ತಲಿನ ಜಗತ್ತನ್ನು ಗ್ರಹಿಸಬಲ್ಲೆವು. ಈ ಇಂದ್ರಿಯಗಳಿಗಿಂತ ಮಿಗಿಲಾದುದು ಮನಸ್ಸು–\textbf{‘ಇಂದ್ರಿ ಯೇಭ್ಯಃ ಪರಂ ಮನಃ’.} ಅನಂತರ, \textbf{‘ಮನಸಸ್ತು ಪರಾ ಬುದ್ಧಿಃ’}–ಮನಸ್ಸಿಗಿಂತಲೂ ಮಿಗಿಲಾದುದು ಬುದ್ಧಿ, ವಿವೇಚನಾ ಶಕ್ತಿ. ಇದು ಸರಿಯೋ ತಪ್ಪೋ ಈ ತಿಳಿವಳಿಕೆ ಬರುವುದು ಬುದ್ಧಿ ಮಟ್ಟದಲ್ಲಿ. ಈ ಬುದ್ಧಿಗಿಂತಲೂ ಮಿಗಿಲಾದುದು ಆತ್ಮ–\textbf{‘ಯೋ ಬುದ್ಧೇಃ ಪರತಸ್ತು ಸಃ’} ಆದ್ದರಿಂದ, \textbf{‘ಏವಂ ಬುದ್ಧೇಃ ಪರಂ ಬುದ್ಧ್ವಾ’} ಬುದ್ಧಿಗೂ ಮೀರಿದುದು ಅರ್ಥಾತ್, ನಿತ್ಯಶುದ್ಧನಾದ, ನಿತ್ಯಮುಕ್ತನಾದ, ನಿತ್ಯಬುದ್ಧನಾದ ಸದಾಜ್ಞಾನಿಯಾದ ಆತ್ಮನನ್ನು –ಆ ಸತ್ಯವನ್ನು ಅರಿಯಬೇಕು. ಆಗ ನಮಗೆ, ನಮ್ಮನ್ನು ಕಾಡುತ್ತಿರುವ ಕೀಳುಕಾಮನೆಗಳನ್ನು ನಿಯಂತ್ರಿಸಲು ಸಾಧ್ಯವಾಗು ತ್ತದೆ. ಆಮೇಲೆ ನಮ್ಮೊಳಗೆ ನಮಗೆ ಯಾವ ಶತ್ರುಗಳೂ ಇರು ವುದಿಲ್ಲ.

\begin{verse}
‘ಏವಂ ಬುದ್ಧೇಃ ಪರಂ ಬುದ್ಧ್ವಾ ಸಂಸ್ತಭ್ಯಾತ್ಮಾನಮಾತ್ಮನಾ ।\\ಜಹಿ ಶತ್ರುಂ ಮಹಾಬಾಹೋ ಕಾಮರೂಪಂ ದುರಾಸದಮ್ ॥’
\end{verse}

‘ಬುದ್ಧಿಗಿಂತ ಮಿಗಿಲಾದುದನ್ನು ಅರಿತುಕೊಂಡು, ಉನ್ನತ ಪ್ರಕೃತಿಯಿಂದ (ಆತ್ಮನಿಂದ) ಅಧಃಪ್ರಕೃತಿಯನ್ನು ನಿಯಂತ್ರಿಸುತ್ತ ಶತ್ರುವನ್ನು ಗೆಲ್ಲು. ಓ ಮಹಾಬಾಹು, ತೃಪ್ತಿಪಡಿಸಲು ದುಸ್ಸಾಧ್ಯ ವಾದ ಕಾಮವೆಂಬ ಶತ್ರುವನ್ನು ಗೆಲ್ಲು.’ ಕೋಟೆಯನ್ನು ವಶ ಪಡಿಸಿಕೊಳ್ಳುವಂತೆ ತನ್ನ ಸೇನೆಗೆ ಆಜ್ಞಾಪಿಸುವ ದಂಡನಾಯಕನ ಆದೇಶದಂತೆಯೇ ಇದೆ, ಇಲ್ಲಿ ಶ್ರೀಕೃಷ್ಣನ ಮಾತು: ‘ಶತ್ರುವನ್ನು ಗೆಲ್ಲು!’

ಆದ್ದರಿಂದ, ಇಂದಿನ ಗೃಹಸ್ಥರು ತಮಗೆ ತಾವೇ ಈ ಬಗೆಯ ತರಬೇತಿ ಕೊಟ್ಟುಕೊಳ್ಳುತ್ತ ಬಂದರೆ, ಇದೇ ನಿಟ್ಟಿನಲ್ಲಿ ಇತರ ರನ್ನೂ ಸೇರಿಸಿಕೊಂಡು ಶ್ರಮಿಸಿದರೆ, ಅವರೆಲ್ಲ ಶ್ರೇಷ್ಠಪ್ರಜೆ ಗಳಾಗಬಹುದು. ಪಾಶ್ಚಾತ್ಯರ ಚಿಂತನೆಯಲ್ಲಿ ಈ ಬಗೆಯ ತಿಳಿ ವಳಿಕೆಯಿಲ್ಲ, ಏಕೆಂದರೆ ಅವರು ಇಂದ್ರಿಯಗಳ ಹಂತದಲ್ಲೇ ನಿಂತುಬಿಡುತ್ತಾರೆ. ಮನಸ್ಸನ್ನು ಕೂಡ ಅವರು ಇಂದ್ರಿಯ ಸಮೂಹದ ಬಾಲಂಗೋಚಿ ಎಂದೇ ಭಾವಿಸುತ್ತಾರೆ. ಮೆದುಳಿನ ಬಗ್ಗೆ ತಿಳಿಸುವ ಯಾವುದೇ ಪುಸ್ತಕ ತೆಗೆದುಕೊಳ್ಳಿ, ಅದು ಮನಸ್ಸು ಎಂಬ ಪ್ರತ್ಯೇಕ ವಸ್ತುವೊಂದು ಇಲ್ಲ, ಇರುವುದು ಮೆದುಳು ಮಾತ್ರವೇ ಎಂದೇ ಹೇಳುತ್ತದೆ. ಆದರೆ ಈಗ ಎಷ್ಟೋ ನರ ವಿಜ್ಞಾನಿಗಳು ಅದನ್ನು ಒಪ್ಪುತ್ತಿಲ್ಲ. ಬದಲಾಗಿ ಮೆದುಳು ಕೇವಲ ಒಂದು ಭೌತಿಕ ಉಪಕರಣ, ಮನಸ್ಸು ಅದಕ್ಕಿಂತ ಮಿಗಿಲಾದುದು ಎಂಬ ಭಾರತೀಯ ದೃಷ್ಟಿಯ ಕಡೆಗೇ ಒಲವು ತೋರುತ್ತಿದ್ದಾರೆ. ಮನಸ್ಸಿಗಿಂತಲೂ ಉನ್ನತವಾದದ್ದು ಬುದ್ಧಿ–ವಿವೇಚನಾಶಕ್ತಿ. ಇದರ ಹಿನ್ನೆಲೆಯಲ್ಲಿರುವುದೇ ನಿತ್ಯಶುದ್ಧನೂ ಅಮರನೂ ಆದ, ಪರಮಸತ್ಯವಾದ ಆತ್ಮ. ಈಗ ಈ ಸತ್ಯವು ಕ್ರಮೇಣ ಪಾಶ್ಚಾತ್ಯ ಚಿಂತನೆಯೊಳಕ್ಕೆ ಪ್ರವೇಶಿಸುತ್ತಿದೆ. ಪ್ರಖ್ಯಾತ ನರವಿಜ್ಞಾನಿ ಸರ್ ಚಾರ್ಲ್ಸ್ ಶೆರಿಂಗ್​ಟನ್ ತಮ್ಮ ಪ್ರಸಿದ್ಧಪುಸ್ತಕ–ಪ್ರತಿಯೊಬ್ಬ ವೈದ್ಯಕೀಯ ವಿದ್ಯಾರ್ಥಿಯೂ ಇದನ್ನು ಓದಬೇಕು\eng{–‘Integrated Action of the Nervous System’ (}ನರಮಂಡಲದ ಸಂಯುಕ್ತ ಕಾರ್ಯಾಚರಣೆ) ಎಂಬುದರಲ್ಲಿ ಹೇಳುತ್ತಾರೆ: ‘ನರ ವ್ಯೂಹದ ವರ್ತನೆಯನ್ನು ವಿವರಿಸಲು ಒಂದೇ ಅಂಶ ಸಾಲದು. ಮೆದುಳು, ಮನಸ್ಸು ಈ ಎರಡೂ ಅಂಶಗಳೂ ಅಗತ್ಯ; ಮೆದು ಳೊಂದೇ ಸಾಲದು.’ ಇನ್ನೂ ಕೆಲವರು ಇದನ್ನೇ ಹೇಳುತ್ತಾರೆ \eng{‘Neurology and What Lies Beyond’ (}ನರವಿಜ್ಞಾನ ಮತ್ತು ಅದರಾಚೆಯ ದೃಶ್ಯ) ಎಂಬ ನನ್ನ ಪುಟ್ಟ ಪುಸ್ತಿಕೆಯಲ್ಲಿ ನಾನು ಅವುಗಳನ್ನು ಉದ್ಧರಿಸಿದ್ದೇನೆ.\footnote{
\begin{verse}
* ಹೈದರಾಬಾದಿನಲ್ಲಿ ನಡೆದ ‘ಅಖಿಲ ಭಾರತ ವಿಜ್ಞಾನ ಸಮ್ಮೇಳನ’ದಲ್ಲಿ ಸ್ವಾಮೀಜಿಯವರು ಮಾಡಿದ ಉದ್ಘಾಟನಾ ಭಾಷಣ. ಇದು ಮೊದಲು ಸಮ್ಮೇಳನದ ವತಿಯಿಂದ, ಅನಂತರ ಮುಂಬಯಿಯ ಭಾರತೀಯ ವಿದ್ಯಾಭವನದಿಂದ ಪುಸ್ತಕರೂಪದಲ್ಲಿ ಪ್ರಕಟವಾಯಿತು.
\end{verse}} ಮಾನವನ ವ್ಯಕ್ತಿತ್ವದ ಈ ಗಹನ ಆಯಾಮದ ಅರಿವು ವೇದಾಂತಕ್ಕಿದೆ. ಆದರೆ ಪಶ್ಚಿಮ ಜಗತ್ತಿನ್ನೂ ಮಾನವನನ್ನು ಆಳವಾಗಿ ಅಧ್ಯಯನಮಾಡಿ ಅದನ್ನು ಕಂಡು ಕೊಳ್ಳಬೇಕಾಗಿದೆ. ಈವರೆಗೂ ಪಾಶ್ಚಾತ್ಯರು ಅದನ್ನು ಅಧ್ಯಯನ ಮಾಡಿಲ್ಲ. ಅವರಲ್ಲಿ ಅಂಗರಚನಾವಿಜ್ಞಾನ, ಶರೀರವಿಜ್ಞಾನ, ನರವಿಜ್ಞಾನ ಇವೆಲ್ಲ ಇವೆ; ಆದರೆ ಅವುಗಳಾಚೆ ಅವರಿಗೆ ಗೊತ್ತಿಲ್ಲ. ಮನಶ್ಶಾಸ್ತ್ರವನ್ನು ಕೂಡ ಈಗಿನ್ನೂ ಆಳವಾಗಿ ಅರಿ ಯಲು ಯತ್ನಿಸುತ್ತಿದ್ದಾರಷ್ಟೆ. 

ಈ (೨0ನೇ) ಶತಮಾನದ ಆದಿಯಲ್ಲಿದ್ದ ಕಾರ್ಲ್ ಯೂಂಗ್ ಹಾಗೆ ಯತ್ನಿಸಿದವರಲ್ಲೊಬ್ಬ. \eng{‘Modern Man In Search of a Soul’ (}ಆಧುನಿಕ ಮಾನವ: ಆತ್ಮದ ಶೋಧನೆಯಲ್ಲಿ) ಎಂಬ ಪ್ರಸಿದ್ಧ ಪುಸ್ತಕವನ್ನು ಆತ ಬರೆದ. ಅವನೆನ್ನುತ್ತಾನೆ: ‘ನಮಗೆ ಒಂದು ಶರೀರವೇನೋ ಇದೆ, ಸಂತೋಷ. ಆದರೆ ನಮ್ಮ ಆತ್ಮ ವೆಲ್ಲಿದೆ? ನಾಗರಿಕತೆಯ ಶಿಥಿಲ ಅವಶೇಷಗಳಲ್ಲಿ ಅದು ಕಳೆದು ಹೋಗಿದೆ! ಆತ್ಮಶೋಧನೆಯನ್ನೇ ಆತ ಆ ಪುಸ್ತಕದಲ್ಲಿ ಬಣ್ಣಿಸಿ ರುವುದು. ಅದರಲ್ಲೊಂದು ಸುಂದರ ಪಂಕ್ತಿಯಿದೆ. ಅಲ್ಲಿ ಯೂಂಗ್ ಹೇಳುವಂತೆ, ಗೃಹಸ್ಥಜೀವನವನ್ನು ಎರಡು ಭಾಗ ಗಳಾಗಿ ವಿಂಗಡಿಸಬಹುದು. ವ್ಯಕ್ತಿಯೊಬ್ಬನು ಶಿಕ್ಷಣ ಪಡೆಯು ವುದು, ಉದ್ಯೋಗ ಹಿಡಿಯುವುದು, ವಿವಾಹವಾಗಿ ಸಂಸಾರ ಹೂಡುವುದು, ಸಮಾಜದಲ್ಲಿ ಹೆಸರು, ಅಂತಸ್ತುಗಳಿಸುವುದು ಇವೆಲ್ಲ ಮೊದಲ ಭಾಗ. ಇದನ್ನು ಯೂಂಗ್ ಯಶಸ್ಸು ಅಥವಾ ಸಾಧನೆ \eng{(Achievement)} ಎಂದು ಕರೆಯುತ್ತಾನೆ. ಆಮೇಲೆ, ಮಧ್ಯವಯಸ್ಸಿನ ನಂತರ ಶುರುವಾಗುವುದು ಜೀವನದ ಎರಡನೇ ಭಾಗ. ‘ಇದು ಮೊದಲನೆಯದಕ್ಕಿಂತ ಭಿನ್ನವಾಗಿರಬೇಕು; ನಿಮ್ಮ ಲೌಕಿಕ ಸಾಧನೆಯ ತತ್ತ್ವವನ್ನು ಇಲ್ಲಿ ತರಬೇಡಿ’ ಎನ್ನುತ್ತಾನೆ ಆತ. ‘ಈ ಭಾಗವನ್ನು ವ್ಯಕ್ತಿತ್ವದ ಬೆಳವಣಿಗೆ-ಪರಿಷ್ಕರಣಗಳಿಗೆ ಮೀಸಲಾಗಿರಿಸಬೇಕು. ಮಧ್ಯವಯಸ್ಸಾದ ಮೇಲೆ, ಕೇವಲ ಹಿಂದಿನ ‘ಸಾಧನೆ’ಗಳನ್ನೇ ಮುಂದುವರಿಸಬಾರದು. ಇಲ್ಲದಿದ್ದರೆ ಮಾನವನ ವ್ಯಕ್ತಿತ್ವದಲ್ಲಿ ನ್ಯೂನತೆಯುಂಟಾಗುತ್ತದೆ, ತಗ್ಗುಂಟಾ ಗುತ್ತದೆ. ಬದಲಾಗಿ ನಿಮ್ಮ ಆಂತರಿಕ ಜೀವನವನ್ನು ಬೆಳೆಸಿಕೊಳ್ಳಿ –ಇದೇ ನಿಮ್ಮ ಜೀವನದ ಉತ್ತರಾರ್ಧದ ಗುರಿಯಾಗಬೇಕು. ಸಿದ್ಧಿಗಳನ್ನು ಪಡೆಯುವ ಹೋರಾಟದಲ್ಲಿ ಇಷ್ಟು ಕಾಲ ಇದನ್ನು ಕಡೆಗಣಿಸಿದ್ದಿರಿ. ಇನ್ನು ಮುಂದೆ ಆ ಬಗ್ಗೆ ಗಮನಕೊಡಲು ನಿಮಗೆ ಸಮಯ ಒದಗಿದೆ.’ ಇವೆಲ್ಲ ಯೂಂಗ್​ನ ವಿಚಾರಧಾರೆ.

ಹಿಂದಿನಿಂದಲೂ ಭಾರತೀಯ ವಿಚಾರಧಾರೆ ಎತ್ತಿಹಿಡಿದಿರು ವುದು ಇದನ್ನೇ–ಒಂದು ನಿರ್ದಿಷ್ಟ ವಯಸ್ಸಿನಲ್ಲಿ ಗೃಹಸ್ಥನು ‘ವಾನಪ್ರಸ್ಥಿ’ಯಾಗಬೇಕು ಎಂಬುದನ್ನು. ಮನುಸ್ಮೃತಿಯಲ್ಲಿ ಬಹಳ ಸ್ವಾರಸ್ಯಕರವಾದ ಪ್ರಶ್ನೆಯೊಂದು ಬರುತ್ತದೆ: ‘ನಾನು ಯಾವಾಗ ಪ್ರಾಪಂಚಿಕ ಮಾರ್ಗವನ್ನು ಬಿಟ್ಟು ಆಧ್ಯಾತ್ಮಿಕ ಜೀವನ ದತ್ತ ಮನಸ್ಸು ಹರಿಸಬೇಕು?’ ಇದಕ್ಕೆ ಮನು ಉತ್ತರಿಸುತ್ತಾನೆ (೬.೨):

\begin{verse}
‘ಗೃಹಸ್ಥಸ್ತು ಯದಾ ಪಶ್ಯೇತ್ ವಲಿಪಲಿತಮಾತ್ಮನಃ ।\\ ಅಪತ್ಯಸ್ಯೈವ ಚಾಪತ್ಯಂ ತದಾರಣ್ಯಂ ಸಮಾಶ್ರಯೇತ್ ॥’
\end{verse}

‘ಯಾವಾಗ ಗೃಹಸ್ಥನು ತನ್ನ ನರೆಯುತ್ತಿರುವ ಕೂದಲನ್ನೂ, ಮೊಮ್ಮಗುವಿನ ಮುಖವನ್ನೂ ಕಾಣುತ್ತಾನೋ ಆಗ ಆತ ಅರಣ್ಯ ವಾಸಕ್ಕೆ ತೆರಳಬೇಕು.’

ಎಂಥ ಸುಂದರ ಮಾತು! ಕೂದಲು ಬೆಳ್ಳಗಾಗುತ್ತಿರುವು ದನ್ನು ಕಂಡಾಗ ನಮಗೆ ಗೊತ್ತಾಗುತ್ತದೆ–ವೃದ್ಧಾಪ್ಯ ಬರುತ್ತಿದೆ ಎಂದು; ನಮ್ಮ ಮೊಮ್ಮಗುವಿನ ಮುಖವನ್ನು ಕಂಡೆವೆಂದರೆ ಅದರ ಅರ್ಥ–ಪ್ರಕೃತಿಗೆ ನಮ್ಮ ಕರ್ತವ್ಯವನ್ನು ಸಲ್ಲಿಸಿದ್ದಾಯಿತು ಎಂದು. ಈಗ ಸಮಯ ಮುಗಿಯುತ್ತ ಬಂತು. ಇಷ್ಟು ದಿನ ನಾವೊಂದು ಸುಂದರ ಅಂಶವನ್ನು ಕಡೆಗಣಿಸಿದ್ದೆವು–ಅದು ನಮ್ಮದೇ ಆಂತರಿಕ ಪ್ರಗತಿ. ನಾವು ಲೌಕಿಕ ಸಾಧನೆ, ಹೆಸರು, ಕೀರ್ತಿ ಇವುಗಳಲ್ಲೇ ಮಗ್ನರಾಗಿದ್ದೆವು. ಇನ್ನು ಅದು ನಡೆಯುವು ದಿಲ್ಲ. ಆ ಕಡೆ ಗಮನ ಕಡಿಮೆ ಮಾಡಿ ನಮ್ಮ ಆಂತರಿಕ ಜೀವನ ವನ್ನು ಸಮೃದ್ಧಗೊಳಿಸುವತ್ತ ಮನಸ್ಸು ಹರಿಸಬೇಕು.

ಯೂಂಗ್ ಎಚ್ಚರಿಸುತ್ತಾನೆ: “ಈ ‘ಸಾಧನೆ’ಗಳ ಗೀಳನ್ನು ಜೀವನದ ದ್ವಿತೀಯ ಭಾಗಕ್ಕೂ ಕೊಂಡೊಯ್ಯುವ ಸ್ತ್ರೀಯರಾಗಲಿ ಪುರುಷರಾಗಲಿ ವ್ಯಕ್ತಿತ್ವದಲ್ಲಿ ನ್ಯೂನತೆಯನ್ನು ಅನುಭವಿಸು ತ್ತಾರೆ.” ಶಕ್ತಿ, ಸ್ಥೈರ್ಯ, ಆಂತರಿಕ ಸಮೃದ್ಧಿಯ ಭಾವ ಇವೆಲ್ಲ ಬರುವುದು ಆತ್ಮಜ್ಞಾನದಿಂದ. ‘ತನ್ನ ಆತ್ಮಸ್ವರೂಪದ ಕಿಂಚಿತ್ ಅರಿವೇ ಭಯವನ್ನು ನಾಶ ಮಾಡಿಬಿಡುತ್ತದೆ’ ಎಂದು ಗೀತೆ (ಎರಡನೇ ಅಧ್ಯಾಯ) ಹೇಳುತ್ತದೆ. ನಮ್ಮ ಲೌಕಿಕ ಜೀವನರಂಗ ದಲ್ಲೂ ಕೂಡ ನಮಗೆ ಕೆಲಮಟ್ಟಿಗೆ ಆಧ್ಯಾತ್ಮಿಕಶಕ್ತಿ ಒದಗಿ ಬರುತ್ತದೆ. ಆದರೆ ದ್ವಿತೀಯಾರ್ಧದಲ್ಲಿ ನಾವು ಅದರ ಕಡೆಗೆ ಇನ್ನಷ್ಟು ತೀವ್ರತೆಯಿಂದ ಮನಸ್ಸು ಕೊಡಲು ಸಾಧ್ಯ. ನಮ್ಮ ‘ವಾನಪ್ರಸ್ಥ’ ಹಾಗೂ ‘ಸಂನ್ಯಾಸ’ ಜೀವನತತ್ತ್ವವೂ ಇದೇ. ಒಟ್ಟಿನಲ್ಲಿ, ಜೀವನದ ಎರಡು ವಿಭಾಗಗಳು ಎಂಬ ಇದು ಎಂಥ ವಿನೂತನ ಕಲ್ಪನೆ, ನೋಡಿ! ಗೃಹಸ್ಥರು ಸಂತೋಷದ ಜೀವನ ನಡೆಸಬೇಕೆಂದಿದ್ದರೆ ಅವರು ಆಧ್ಯಾತ್ಮಿಕರಾಗಿರಲೇಬೇಕು. ಇದು ಸಾಧ್ಯವಾಗಬೇಕಾದರೆ ಒಬ್ಬನ ವ್ಯಕ್ತಿತ್ವ (ಆತ್ಮ) ವಿಕಾಸಗೊಂಡು ಇತರರನ್ನೂ ಅದರ ಪರಿಧಿಯೊಳಕ್ಕೆ ಸೇರಿಸಿಕೊಳ್ಳಬೇಕಾಗುತ್ತದೆ. ಈ ಆತ್ಮವಿಕಾಸವಿಲ್ಲದೆ ಹೋದರೆ, ವ್ಯಕ್ತಿ ತನ್ನ ಭೌತಿಕ ದೇಹ ಕ್ಕಷ್ಟೇ ಸೀಮಿತನಾಗಿ ಕುಂಠಿತಗೊಳ್ಳುತ್ತಾನೆ. ಅದೇಕೆ ಹಾಗೆ?– ಎಂದರೆ, ಪತಿ ಪತ್ನಿ ಇಬ್ಬರಿಗೂ ಪರಸ್ಪರರಲ್ಲಿ ಪ್ರೇಮವನ್ನು ಉಕ್ಕಿಸುವ ಸಾಮರ್ಥ್ಯವಿಲ್ಲದಂತಾಗಿ ಸದಾ ತಿಕ್ಕಾಟ ಶುರುವಾಗು ತ್ತದೆ. ವ್ಯಕ್ತಿಗಳಲ್ಲಿ ಆಧ್ಯಾತ್ಮಿಕ ಬೆಳವಣಿಗೆ ಇಲ್ಲದಿದ್ದಲ್ಲಿ ಘರ್ಷಣೆ ಖಚಿತ. ‘ಧಾರ್ಮಿಕರು’ ಎನ್ನಿಸಿಕೊಂಡ ಎಷ್ಟೋ ಸ್ತ್ರೀ ಪುರುಷರು ಪರಸ್ಪರ ಘರ್ಷಣೆಯಲ್ಲಿ ತೊಡಗಿರುವುದನ್ನು ಕಾಣು ತ್ತೇವೆ. ಎಷ್ಟೋ ಸಲ ಅತ್ತೆ ತೀರ ‘ಧಾರ್ಮಿಕ’ಳಾಗಿರುತ್ತಾಳೆ, ಆದರೂ ಸೊಸೆಯನ್ನು ಶೋಷಿಸುತ್ತಾಳೆ. ಏಕೆಂದರೆ ಆಕೆಯಲ್ಲಿ ಆಧ್ಯಾತ್ಮಿಕತೆ ಇರುವುದಿಲ್ಲ. ಸ್ವಲ್ಪ ಆಧ್ಯಾತ್ಮಿಕ ಬೆಳವಣಿಗೆ ಯಾದರೂ ಇವೆಲ್ಲ ಬದಲಾಗುತ್ತದೆ. 

ಶ್ರೀರಾಮಕೃಷ್ಣರು ಈ ಬಗ್ಗೆ ಬಹಳ ಸುಂದರವಾಗಿ ಹೇಳು ತ್ತಾರೆ: “ಈ ‘ನಾನು’ ಎನ್ನುವುದು ಕಾಯಿಯಾಗಿರುವವರೆಗೆ (‘ಅಪಕ್ವ ಅಹಂ’) ಸಮಾಜದ ಇತರ ‘ನಾನು’ಗಳೊಂದಿಗೆ ತಾಡಿ ಸುತ್ತಿರುತ್ತದೆ.” ಅಜ್ಞೇಯತಾವಾದಿ ಬರ್ಟ್ರಾಂಡ್ ರಸೆಲ್ ಹೇಳಿದ –ಕೆಲವರು ಬಿಲಿಯರ್ಡ್ ಆಟದ ಚೆಂಡುಗಳ ಹಾಗೆ; ಸದಾ ಒಬ್ಬರಿಗೊಬ್ಬರು ಬಡಿಯುತ್ತಿರುತ್ತಾರೆ ಎಂದು. ಬಿಲಿಯರ್ಡ್ ಚೆಂಡಿಗೆ ಇತರ ಚೆಂಡುಗಳೊಂದಿಗೆ ವಿರಸವಿಲ್ಲದೆ ಬದುಕಲು ಬರುವುದಿಲ್ಲ. ಶ್ರೀರಾಮಕೃಷ್ಣರು ಹೇಳುತ್ತಾರೆ–ಸಮಾಜದ ಇತರ ‘ಅಹಂ’ಗಳೊಡನೆ ಸಂತೋಷದಿಂದ ಇರಬೇಕಾದರೆ ಈ ‘ಅಲ್ಪ’ ಅಹಂ ವಿಕಾಸಗೊಳ್ಳಬೇಕು, ಬಲಿತು ‘ಪಕ್ವ ಅಹಂ’ ಆಗಬೇಕು ಎಂದು.

ಈಗ ನಮ್ಮ ಸಮಾಜದಲ್ಲಿ ಇಂಥ ಬಿಲಿಯರ್ಡ್ ಚೆಂಡುಗಳು ವಿಪರೀತವಾಗಿಬಿಟ್ಟಿವೆ; ರಾಜಕಾರಣ, ಆಡಳಿತ, ಕುಟುಂಬಗಳು, ಹೀಗೆ ಎಲ್ಲೆಲ್ಲೂ ತುಂಬಿಕೊಂಡು ಅವನ್ನೆಲ್ಲ ವಿರಸಮಯವಾಗಿ ಸಿವೆ. ಇಂಥ ಸಂಕುಚಿತ ವ್ಯಕ್ತಿಗಳ ಸಂಖ್ಯೆ ಅತಿಯಾಗಿ, ಸಾಂಸಾರಿಕ ಜೀವನವೂ ದುಸ್ಸಹವಾಗಿದೆ. ಆದರೆ ಕಿಂಚಿತ್ ಆಧ್ಯಾತ್ಮಿಕ ಬೆಳವಣಿಗೆ–ಆತ್ಮವಿಕಾಸ–ಸಾಧ್ಯವಾದರೆ, ಎಲ್ಲವೂ ಸುಗಮವಾಗಿ ಶಾಂತಿಮಯವಾಗಿಬಿಡುತ್ತವೆ; ಜೀವನ ಆನಂದಕರ ವಾಗುತ್ತದೆ, ಸಾರ್ಥಕವಾಗುತ್ತದೆ. ಈ ‘ಅಹಂ’ನ ವಿಕಸನದ ಬಗ್ಗೆ ಸರ್ ಜೂಲಿಯನ್ ಹಕ್ಸ್​ಲೀ ಆಧುನಿಕ ಜೀವಶಾಸ್ತ್ರದ ಪರಿಭಾಷೆ ಯಲ್ಲಿ ಪ್ರಸ್ತಾಪಿಸಿದ್ದರು. ಅವರು ಅಪಕ್ವ ಅಹಂ ಅನ್ನು ಸ್ವಾರ್ಥೀ ವ್ಯಕ್ತಿತ್ವ \eng{(Individuality)} ಎಂದೂ, ಪಕ್ವ ಅಹಂ ಅನ್ನು ನಿಸ್ಸ್ವಾರ್ಥೀವ್ಯಕ್ತಿತ್ವ \eng{(Personality)} ಎಂದೂ ಕರೆದರು. ಈ ಸ್ವಾರ್ಥೀವ್ಯಕ್ತಿತ್ವವು ನಿಸ್ಸ್ವಾರ್ಥೀವ್ಯಕ್ತಿತ್ವವಾಗಿ ವಿಕಸನಗೊಳ್ಳ ಬೇಕು.

ಈಗ ‘ವ್ಯಕ್ತಿ’ ಶಬ್ದದ ಅರ್ಥವೇನು? ಟೆಯ್ಲ್​ಹಾರ್ಡ್ ಡಿ ಶಾರ್ಡಿನ್ ಎಂಬಾತನ \eng{‘The Phenomenon of Life’} ಎಂಬ ಪುಸ್ತಕದ ಮುನ್ನುಡಿಯಲ್ಲಿ ಹಕ್ಸ್​ಲೀ ಹೇಳುತ್ತಾರೆ: ‘ಯಾರು ಪ್ರಜ್ಞಾಪೂರ್ವಕವಾಗಿ ಸಾಮಾಜಿಕ ಕ್ಷೇತ್ರಗಳಲ್ಲಿ ಪಾತ್ರವಹಿಸುವ ಮೂಲಕ ತಮ್ಮ ಪ್ರಕೃತಿದತ್ತವಾದ ಪ್ರತ್ಯೇಕ ಅಸ್ತಿತ್ವವನ್ನು ಮೀರುತ್ತಾರೋ ಅವರು ನಿಸ್ಸ್ವಾರ್ಥೀ ವ್ಯಕ್ತಿಗಳು.’ ಇಂಥವರಿಗೆ ಸಮಾಜದ ಇತರ ವ್ಯಕ್ತಿಗಳೊಂದಿಗೆ ಸಹಬಾಳ್ವೆಯ ಸಾಮರ್ಥ್ಯ ವಿರುತ್ತದೆ. ಸ್ವಾರ್ಥೀ ವ್ಯಕ್ತಿಗಳಿಗೆ \eng{(individuals)} ಇದು ಸಾಧ್ಯ ವಿಲ್ಲ. ಅವರು ಒಬ್ಬರಿಗೊಬ್ಬರು ಡಿಕ್ಕಿ ಹೊಡೆಯುತ್ತಾರಷ್ಟೆ. ಪತಿ ಪತ್ನಿ ಇಬ್ಬರೂ ಸ್ವಾರ್ಥೀವ್ಯಕ್ತಿಗಳಾದರೆ ಅವರಲ್ಲಿ ಸದಾ ಘರ್ಷಣೆ ನಡೆಯುತ್ತಿರುತ್ತದೆ. ಅವರಿಬ್ಬರೂ ನಿಸ್ಸ್ವಾರ್ಥೀವ್ಯಕ್ತಿಗಳಾದರೆ, ಸಂಸಾರದಲ್ಲಿ ಪರಿಪೂರ್ಣ ಶಾಂತಿ ನೆಲೆಸುತ್ತದೆ. ಇದನ್ನೇ ನಾವು ತಿಳಿದುಕೊಳ್ಳಬೇಕಾದದ್ದು. ಇದೆಲ್ಲ ಯಾರು ಬೇಕಾದರೂ ಸಾಧಿಸ ಬಹುದಾದದ್ದು. ಕಾಲೇಜುಶಿಕ್ಷಣದ ಜೊತೆಯಲ್ಲಿ ಪಡೆದುಕೊಳ್ಳ ಬೇಕಾದ ನೈಜಶಿಕ್ಷಣ ಇದು. ತನ್ಮೂಲಕ ನೀವು ಬೆಳೆಯುತ್ತೀರಿ. ನೀವು ನಿಮ್ಮ ಮಗುವಿಗೆ ಹೇಳಬಹುದು: ‘ನೋಡು, ನೀನು ಆಧ್ಯಾತ್ಮಿಕವಾಗಿ ಬೆಳೆದರೆ ತುಂಬ ಸಂತೋಷವಾಗಿರುತ್ತೀ’ ಎಂದು. ‘ಬೆಳವಣಿಗೆ’ ಎನ್ನುವ ಶಬ್ದ ಮಕ್ಕಳಿಗೆ ಪ್ರಿಯವಾದದ್ದು. ‘ನೀನು ಬೆಳೆಯಬೇಕು... ಬೇರೆ ಮಕ್ಕಳ ಜೊತೆಯಲ್ಲಿ ಜಗಳ ವಾಡಬೇಡ. ಅವರ ಜೊತೆಯಲ್ಲಿ ಸ್ನೇಹದಿಂದ ಇರು’–ಹೀಗೆ ಮಗುವಿಗೆ ಹೇಳುತ್ತ ಬಂದರೆ, ಅದರ ‘ಅಪಕ್ವ ಅಹಂ’ ಕ್ರಮೇಣ ‘ಪಕ್ವ ಅಹಂ’ ಆಗುತ್ತದೆ. ಈ ಶಿಕ್ಷಣವನ್ನು ತಾಯ್ತಂದೆಯರು ಮಕ್ಕಳಿಗೆ ಕೊಡಲೇಬೇಕು. ಆದರೆ ಈಗೆಲ್ಲ ಇದು ನಡೆಯುತ್ತಿಲ್ಲ. ಬದಲಾಗಿ ಇದಕ್ಕೆ ವಿರುದ್ಧವಾದದ್ದನ್ನೇ ಮಕ್ಕಳಿಗೆ ತುಂಬುತ್ತಾರೆ. ‘ನಿನ್ನ ಅಪಕ್ವ ಅಹಂ ಅನ್ನು ಚೆನ್ನಾಗಿ ಬಲಪಡಿಸಿಕೋ; ಸ್ವಾರ್ಥಿ ಯಾಗಿರು; ಯಾರನ್ನೂ ಲೆಕ್ಕಿಸಬೇಡ...’ ಸಾಮಾನ್ಯವಾಗಿ ಇಂಥ ಬೋಧನೆಯನ್ನೇ ನಾವೀಗ ನಮ್ಮ ಮಕ್ಕಳಿಗೆ ಕೊಡುತ್ತಿರು ವುದು. ಈ ‘ಅಪಕ್ವ ಅಹಂ’ ‘ಪಕ್ವ ಅಹಂ’ ಆಗುವುದರ ಮೂಲಕ ಆಧ್ಯಾತ್ಮಿಕ ಪ್ರಗತಿಯಾಗುವುದೊಂದು ಅದ್ಭುತ ಕಲ್ಪನೆ. ಸಾಂಸಾರಿಕ ಜೀವನ ಸುಖಮಯವಾಗಬೇಕಾದರೆ ಇದನ್ನು ದೃಷ್ಟಿ ಯಲ್ಲಿರಿಸಿಕೊಳ್ಳಲೇಬೇಕು. ಕಿಂಚಿತ್ ಆತ್ಮವಿಕಾಸವಾದರೂ ಶಾಂತಿಮಯ, ಸುಖಮಯ ಗೃಹಸ್ಥಜೀವನ ಸಿದ್ಧಿಸುತ್ತದೆ. ಸ್ತ್ರೀ ಯಾಗಲಿ, ಪುರುಷನಾಗಲಿ, ಜೀವನದಲ್ಲಿ ಉನ್ನತಿ ಸಾಧಿಸಲು ಸಾಧುವೋ, ತತ್ತ್ವ ಜ್ಞಾನಿಯೋ ಆಗಲು ಪ್ರಯತ್ನಿಸಬೇಕಿಲ್ಲ. ಆಧ್ಯಾತ್ಮಿಕ ಜೀವನವೆಂದರೆ ಅನುಭಾವಿಯಂತೆ ನಟಿಸುವುದಲ್ಲ. ನಮ್ಮಲ್ಲಿ ಆಧ್ಯಾತ್ಮಿಕ ಬೆಳವಣಿಗೆ ಶುದ್ಧರೂಪದಲ್ಲಿ ನಡೆದರೆ, ಅದು ಚಾರಿತ್ರ್ಯವಿಕಾಸ, ಗುಂಪಿನಲ್ಲಿ ಕೆಲಸ ಮಾಡುವ ಶಕ್ತಿ, ಪ್ರೀತಿ, ಸೇವೆ ಇವುಗಳ ಮೂಲಕ ವ್ಯಕ್ತಗೊಳ್ಳುತ್ತದೆ. ಸ್ವಾರ್ಥೀ ವ್ಯಕ್ತಿಗಳು ಸದಾ ಒಬ್ಬರ ಕಾಲನ್ನೊಬ್ಬರು ಎಳೆಯಲು ಪ್ರಯತ್ನಿ ಸುತ್ತಾರೆ. ಆದರೆ ‘ನಿಸ್ವಾರ್ಥೀವ್ಯಕ್ತಿ’ಗಳು ಎಂದಿಗೂ ಹಾಗೆ ಮಾಡಲಾರರು. ಒಟ್ಟಿಗೆ ಕೆಲಸ ಮಾಡುವುದು ಹೇಗೆನ್ನುವುದು ಅವರಿಗೆ ಗೊತ್ತು. ಈ ‘ಸ್ವಾರ್ಥೀವ್ಯಕ್ತಿತ್ವ’ ನಮ್ಮಲ್ಲಿ ಅತಿಯಾಗಿ ಇರುವುದರಿಂದಲೇ ಇಂದು ನಮಗೆ ಸಹದುಡಿಮೆಯ ಸಾಮರ್ಥ್ಯ ಇಲ್ಲವಾಗಿರುವುದು. ಮಾನವನ ವ್ಯಕ್ತಿತ್ವದ ಕೇಂದ್ರವಿರುವುದು ಆತ್ಮದಲ್ಲಿ ಎಂಬ ಅದ್ಭುತ ವಿನೂತನ ತತ್ತ್ವವನ್ನು ವೇದಾಂತ ಒಳಗೊಂಡಿದೆ. ಅದನ್ನು ಒಂದಿಷ್ಟೇ ಸಾಕ್ಷಾತ್ಕರಿಸಿಕೊಂಡರೂ ಸಾಕು–ನಮ್ಮ ಜೀವನ ಶ್ರೀಮಂತವಾಗುತ್ತದೆ, ಸುಂದರವಾಗು ತ್ತದೆ. ಶ್ರೀಕೃಷ್ಣನು ಗೀತೆಯಲ್ಲಿ (೨.೪೦) ಹೇಳುವ ಹಾಗೆ,

\begin{verse}
‘ಸ್ವಲ್ಪಮಪ್ಯಸ್ಯ ಧರ್ಮಸ್ಯ ತ್ರಾಯತೇ ಮಹತೋ ಭಯಾತ್ ।’
\end{verse}

‘ಈ ಬೋಧನೆಯ ಅತ್ಯಲ್ಪ ಅನುಷ್ಠಾನವೂ ಕೂಡ ಮಾನವ ನನ್ನು ಮಹಾಭಯದಿಂದ ಪಾರು ಮಾಡುತ್ತದೆ.’

ಸ್ವಾಮಿ ವಿವೇಕಾನಂದರು ತಮ್ಮ \eng{Vedanta in its Appli- cation to Modern Life (C. W. III, p. 237)} ಎಂಬ ಉಪ ನ್ಯಾಸದಲ್ಲಿ ಹೇಳುತ್ತಾರೆ:

“ಶಕ್ತಿ! ಉಪನಿಷತ್ತುಗಳ ಪುಟಪುಟದಲ್ಲೂ ನನಗೆ ಕಾಣುವುದು ಶಕ್ತಿ. ನಾವು ನೆನಪಿಡಬೇಕಾದ ಒಂದು ಅಮೋಘ ಸಂಗತಿ ಇದು. ಜೀವನದಲ್ಲಿ ನಾನು ಕಲಿತ ಒಂದು ಮಹತ್ತರ ಪಾಠ ಇದು. ಉಪನಿಷತ್ತು ಹೇಳುತ್ತದೆ: ‘ಶಕ್ತಿ! ಓ ಮನುಜ! ದುರ್ಬಲನಾಗ ಬೇಡ.’ ‘ಆದರೆ ನನ್ನಲ್ಲಿ ಸಹಜವಾಗಿಯೇ ದೌರ್ಬಲ್ಯಗಳಿಲ್ಲವೆ?’ ಎನ್ನುತ್ತಾನೆ ಆತ. ‘ನಿಜ’ ಉಪನಿಷತ್ತೆನ್ನುತ್ತದೆ, ‘ಆದರೆ ಮತ್ತಷ್ಟು ದುರ್ಬಲತೆಯಿಂದ ಅವು ಗುಣವಾಗುವುವೆ? ಕೊಳಕು ನೀರಿನಿಂದ ಕೊಳೆಯನ್ನು ತೊಳೆಯಲು ಸಾಧ್ಯವೆ? ಪಾಪ ಪಾಪವನ್ನು, ನಿಶ್ಶಕ್ತಿ ನಿಶ್ಶಕ್ತಿಯನ್ನು ತೊಡೆಯಲು ಶಕ್ಯವೆ? ಶಕ್ತಿ! ಓ ಮಾನವ, ಶಕ್ತಿ! ಎದ್ದು ನಿಲ್ಲು, ಶಕ್ತನಾಗು!’ ಹೌದು; ಜಗತ್ತಿನಲ್ಲಿ ‘ಅಭೀಃ’ ‘ನಿರ್ಭ ಯತೆ’ ಎಂಬ ಶಬ್ದ ಮತ್ತೆಮತ್ತೆ ಕಂಡುಬರುವುದು ಇದೊಂದೇ. ದೇವರಿಗಾಗಲಿ ಮನುಷ್ಯನಿಗಾಗಲಿ ಈ (‘ನಿರ್ಭಯ’ ಎಂಬ) ಗುಣವಾಚಕವನ್ನು ಅನ್ವಯಿಸಿರುವಂತಹ ಧರ್ಮಗ್ರಂಥ ಜಗತ್ತಿನಲ್ಲಿ ಇನ್ನಾವುದೂ ಇಲ್ಲ.”

ನಮ್ಮೊಳಗಿರುವ ಆಧ್ಯಾತ್ಮಿಕ ಶಕ್ತಿ ವ್ಯಕ್ತವಾಗುವುದು ಪ್ರೀತಿ, ಅನುಕಂಪೆ, ಸೇವಾಪರತೆ ಮೊದಲಾದ ಮಾನವೀಯ ಮೌಲ್ಯಗಳ ಮೂಲಕ. ‘ನಾನು ನಿನಗಾಗಿ ಏನು ಮಾಡಲಿ? ನಿನಗೆ ನಾನು ಹೇಗೆ ನೆರವಾಗಲಿ?...ಜನರು ತಮ್ಮೊಳಗೆ ಇಂತಹ ಆಧ್ಯಾತ್ಮಿಕ ಬೆಳವಣಿಗೆಯನ್ನು ತಂದುಕೊಂಡರೆ ನಮ್ಮ ಸಮಾಜದಲ್ಲಿ ಎಂಥ ಬದಲಾವಣೆಯಾಗಲಿಕ್ಕಿಲ್ಲ! ಇಲ್ಲಿಯವರೆಗೂ ಎಲ್ಲೋ ಒಬ್ಬಿಬ್ಬ ರನ್ನು ಬಿಟ್ಟರೆ, ನಮ್ಮಲ್ಲಿ ಇಂತಹ ಪ್ರಗತಿ ಆಗಿಯೇ ಇಲ್ಲ. ಆದರೆ ಹೆಚ್ಚಿನವರೆಲ್ಲ ಧಾರ್ಮಿಕರೇ–ದೇವಸ್ಥಾನಕ್ಕೆ ಹೋಗು ತ್ತಾರೆ; ಹಣೆಯ ಮೇಲೆ ಚಿಹ್ನೆಗಳನ್ನು ಧರಿಸುತ್ತಾರೆ; ಒಂದಿಷ್ಟು ಪೂಜೆ ವ್ರತ ಇತ್ಯಾದಿ ಮಾಡುತ್ತಾರೆ; ಆದರೆ ಎಂದೆಂದಿಗೂ ಅವರು ಏನಾಗಿದ್ದರೋ ಅದೇ ಆಗಿರುತ್ತಾರೆ: ಮನೆಯಲ್ಲಾಗಲಿ ಹೊರಗಿನ ಸಮಾಜದಲ್ಲಾಗಲಿ ಸದಾ ಎಡಸುಪೆಡಸು ಮಾಡಿ ಕೊಂಡೇ ಇರುತ್ತಾರೆ. ತಮ್ಮ ಈ ವಿವಾದಗಳ ಇತ್ಯರ್ಥಕ್ಕೆ ನ್ಯಾಯಾಲಯಗಳಿಗೆ ಹೋಗುತ್ತಾರೆ. ಹೀಗಾಗಿಯೇ ನಮ್ಮ ಕೋರ್ಟುಗಳು ಮೊಕದ್ದಮೆಗಳಿಂದ ತುಂಬಿಹೋಗಿರುವುದು. ಅವುಗಳಲ್ಲಿ ಅಸಂಖ್ಯಾತ ಕೇಸುಗಳು ಪರಿಹಾರಕ್ಕಾಗಿ ಕಾಯುತ್ತ ಬಿದ್ದಿವೆ. ಭಾರತದಲ್ಲಿ ಇರುವ ವ್ಯಾಜ್ಯಗಳ ಸಂಖ್ಯೆ ವಿಶ್ವದಲ್ಲೇ ಅತ್ಯಧಿಕ. ಇದೇಕೆ ಎಂದರೆ, ನಮಗೆ ಒಬ್ಬರ ಜೊತೆ ಒಬ್ಬರು ಮಾತನಾಡಿಕೊಂಡು ವಿವಾದವನ್ನು ಬಗೆಹರಿಸಿಕೊಳ್ಳಲು ಬರುವು ದಿಲ್ಲ. ಕಳೆದ ಒಂದು ಸಾವಿರ ವರ್ಷಗಳಿಂದಲೂ ನಮ್ಮಲ್ಲಿ ಮಾನವೀಯ ಸಂಬಂಧಗಳು ತೀರ ಕೆಳಮಟ್ಟದಲ್ಲಿವೆ.

ಇನ್ನು ಮುಂದಿನ ಆಧುನಿಕಕಾಲದಲ್ಲಿ ಅದೆಲ್ಲ ಬದಲಾಗು ತ್ತದೆ. ಅದೇ ಶ್ರೀರಾಮಕೃಷ್ಣರ ಮತ್ತು ಸ್ವಾಮಿ ವಿವೇಕಾನಂದರ ಸಂದೇಶ; ಅನುಷ್ಠಾನ ವೇದಾಂತದ ಸಂದೇಶ. ಸಂತೋಷ ದಾಯಕವಾದಂತಹ ಪರಸ್ಪರ ಮಾನವೀಯ ಸಂಬಂಧಗಳನ್ನು ಬೆಳೆಸಿಕೊಳ್ಳುವುದು ಹೇಗೆ ಎನ್ನುವ ಬಗ್ಗೆ ರಾಮಕೃಷ್ಣ- ವಿವೇಕಾ ನಂದ ಸಾಹಿತ್ಯದಲ್ಲಿ ಮಹತ್ತರವಾದ ಮಾರ್ಗೋಪಾಯಗಳು ಲಭ್ಯವಿವೆ. ಅವುಗಳ ಆಧಾರದ ಮೇಲೆಯೇ ನವಭಾರತವು ಬೆಳೆಯುವುದು. ನಮ್ಮ ಗೃಹೀಜನರು ಆತ್ಮಶ್ರದ್ಧಾವಂತರಾಗಿ– ತಮ್ಮ ಬಗ್ಗೆಯೂ ಪರರ ಬಗ್ಗೆಯೂ ವಿಶ್ವಾಸ ಹೊಂದಿದವರಾಗಿ –ಭಾರತದ, ಮತ್ತು ಇಡೀ ವಿಶ್ವದ, ಉತ್ತಮ, ಸಮರ್ಥ ಪ್ರಜೆ ಗಳಾಗುತ್ತಾರೆ. ಆತ್ಮಗೌರವವೊಂದು ಅತಿ ಮುಖ್ಯ ಅಂಶ– ನಮ್ಮಲ್ಲೀಗ ಇದರ ಕೊರತೆಯಿದೆ. ನಮ್ಮಲ್ಲೆಷ್ಟೋ ಜನ ಹೇಳು ತ್ತಿರುತ್ತಾರೆ: ‘ಅಯ್ಯೋ, ನಾನೊಬ್ಬ ಸಂಸಾರಿ. ನನ್ನಿಂದ ಏನಾ ದೀತು?’ ಆ ಭಾವನೆ ಹೋಗಬೇಕು. ಶ್ರೀರಾಮಕೃಷ್ಣರು ತಮ್ಮ ಗೃಹೀಭಕ್ತರಿಗೆ ಹೇಳುತ್ತಾರೆ: ‘ನೀವು ಸಂಸಾರಿಗಳಲ್ಲ. ನೀವು ಸಂಸಾರದಲ್ಲಿ ಜೀವಿಸಿದ್ದೀರಿ, ಆದರೆ ನಿಮ್ಮೊಳಗೆ ಸಂಸಾರ ಇರಬಾರದು.’ ಹೀಗಿದ್ದಾಗ ಮಾತ್ರವೇ ಇತರ ಮನೆಮಂದಿಯ ಹಾಗೂ ಸಾಮಾಜಿಕರ ಜೊತೆಯಲ್ಲಿ ನಾವು ಉತ್ತಮ ಬಾಂಧವ್ಯ ದಿಂದಿರಲಾದೀತು. ಆದ್ದರಿಂದ, ನಾವು ನಮ್ಮೊಳಗೆ ಸಂಸಾರ ಇರಲು ಅವಕಾಶ ಕೊಡುವುದು ಬೇಡ. ನಮ್ಮೊಳಗೆ ‘ನಾವು ಪ್ರಜಾಸತ್ತಾತ್ಮಕ ಭಾರತದ ಪ್ರಜೆಗಳು’ ಇಲ್ಲವೆ ‘ಭಗವಂತನ ಭಕ್ತರು ನಾವು’ ಎಂಬಂತಹ ಭಾವನೆ ಇರಬೇಕು. ಸಂಸಾರಕ್ಕೆ ಸಂಬಂಧಿಸಿದಂತೆ ಶ್ರೀರಾಮಕೃಷ್ಣರು ಈ ಉದಾಹರಣೆ ಕೊಡು ತ್ತಾರೆ: ದೋಣಿ ನೀರಿನ ಮೇಲಿರಬಹುದೇ ಹೊರತು ದೋಣಿ ಯಲ್ಲಿ ನೀರು ತುಂಬಿರಬಾರದು. ಹಾಗೆ ನೀರು ತುಂಬಿಕೊಂಡರೆ ದೋಣಿ ಮುಳುಗಿಹೋಗುತ್ತದೆ; ತನ್ನ ಉದ್ದೇಶವನ್ನು ಸಫಲ ಗೊಳಿಸಲು ಅಸಮರ್ಥವಾಗುತ್ತದೆ.

ಚರಿತ್ರೆ ಓದಿದರೆ ನಮಗೆ ಗೊತ್ತಾಗುತ್ತದೆ: ಹಿಂದೆ ನಾವು ವ್ಯಕ್ತಿ-ವ್ಯಕ್ತಿಗಳು, ಗುಂಪು-ಗುಂಪುಗಳು ಪರಸ್ಪರ ಜಗಳವಾಡು ತ್ತಲೇ ಇದ್ದೆವು; ಪರದೇಶದ ಆಕ್ರಮಣಕಾರರು ನಮ್ಮ ಈ ಸ್ವಭಾವವನ್ನೇ ಬಳಸಿಕೊಂಡು ನಮ್ಮನ್ನು ಮಣಿಸಿ ದೀರ್ಘಕಾಲ ತಮ್ಮ ಆಧಿಪತ್ಯ ನಡೆಸಿದರು, ಎಂದು. ಇವತ್ತಿಗೂ ನಾವು ಸಾಮಾಜಿಕ, ರಾಜಕೀಯ ಜೀವನಗಳಲ್ಲಿ ಸದಾ ಜಗಳವಾಡಿ ಕೊಂಡು, ತನ್ಮೂಲಕ ನಮ್ಮ ಪ್ರಜಾಸತ್ತೆಯನ್ನು ದುರ್ಬಲಗೊಳಿ ಸುತ್ತಿದ್ದೇವೆ. ರಾಷ್ಟ್ರದ ಹಿತಾಸಕ್ತಿಯ ಪ್ರಶ್ನೆಯಿರುವಾಗ ನಾವು ಒಬ್ಬೊರೊಡನೆ ಒಬ್ಬರು ಸಹಕರಿಸುವುದನ್ನು ಕಲಿಯಬೇಕು. ನಮ್ಮ ಕರ್ನಾಟಕ-ಮಹಾರಾಷ್ಟ್ರ ರಾಜ್ಯಗಳು, ತಮಿಳ್ನಾಡು- ಕರ್ನಾಟಕಗಳು, ತಾವು ವಿಭಿನ್ನ ದೇಶಗಳೋ ಎಂಬಂತೆ, ಕೊನೆ ಯಿಲ್ಲದ ಕಲಹದಲ್ಲಿ ತೊಡಗಿವೆ ಎಂದರೆ ಆ ಹಳೇ ವಾಸನೆ ಇನ್ನೂ ಹೋಗಿಲ್ಲ. ಯಾವಾಗ ವಿವೇಕಾನಂದ ಸಾಹಿತ್ಯ ನಮ್ಮ ಜನದ ಬಹುಭಾಗವನ್ನು ತಲುಪಿ ಅವರನ್ನು ಹುರಿದುಂಬಿಸು ತ್ತದೆಯೋ ಆಗ ಇಂಥದು ತಪ್ಪುತ್ತದೆ; ಯಾವ ಬಗೆಯ ಜನ ವಿಕಾಸ ಆಗಬೇಕೋ ಅದು ಆಗುತ್ತದೆ. ಅದು ಮೂಲಭೂತವಾದ ಮಾನುಷವಿಕಾಸ. ಕೇವಲ ಒಂದು ಡಿಗ್ರಿ ಗಳಿಸಿ ಒಳ್ಳೇ ಸಂಬಳ ಪಡೆಯುವುದಲ್ಲ; ಇದು ಸಾಲದು. ಒಬ್ಬ ಮಾನವನಾಗಿ ನಾನು ನಿಜಕ್ಕೂ ವಿಕಾಸಹೊಂದಿದ್ದೇನೆಯೇ? ನಮ್ಮೀ ಪ್ರಜಾಸತ್ತೆಗೆ ನಾನೊಂದು ಶಕ್ತಿಯ ಮೂಲವಾಗಿದ್ದೇನೆಯೆ? ನಾವು ಈ ಪ್ರಶ್ನೆ ಯನ್ನು ಕೇಳಿಕೊಳ್ಳಬೇಕು; ಸಕಾರಾತ್ಮಕ ಉತ್ತರ ಪಡೆಯಬೇಕು. ಆದ್ದರಿಂದಲೇ ನಮ್ಮ ಗೃಹಸ್ಥರು (ನಾನು ಹಿಂದೆ ಹೇಳಿದಂತೆ, ಶೇಕಡಾ ೯೯.೯ ರಷ್ಟು ಜನರು) ವಿವೇಕಾನಂದರ ಉಪದೇಶ ವನ್ನು, ಗೀತೆಯ ಅನುಷ್ಠಾನವೇದಾಂತವನ್ನು ಅನುಸರಿಸಿದರೆ ಆರೋಗ್ಯವಂತರೂ ಸಶಕ್ತರೂ ಆಗುತ್ತಾರೆ; ಮಾನವೀಯ ದೃಷ್ಟಿ ಕೋನವನ್ನು ಗಳಿಸುತ್ತಾರೆ. ತನ್ಮೂಲಕ ನಮ್ಮ ಇಡೀ ದೇಶ ಬಲ ಶಾಲಿಯೂ ಸೌಮ್ಯವೂ ಆಗುತ್ತದೆ.

ಈಗ ನಾಲ್ಕೈದು ವರ್ಷಗಳ ಹಿಂದೆ ನಾನು ‘ಗೃಹಸ್ಥ ಧರ್ಮ’ದ ಕುರಿತಾಗಿ ಹಿಂದಿಯಲ್ಲಿ ಉಪನ್ಯಾಸ ಮಾಡಲು ಬಿಹಾರದ ಛಾಪ್ರಾಗೆ ಹೋಗಿದ್ದೆ. (ಈ ಉಪನ್ಯಾಸವು ಪುಸ್ತಕ ರೂಪದಲ್ಲಿ ಎಷ್ಟೋ ಭಾಷೆಗಳಲ್ಲಿ ಬಂದಿದೆ.) ಅಲ್ಲಿ ನಾನು ಕಂಡದ್ದು ಒಂದು ದುರ್ಬಲ ಸಮಾಜವನ್ನು–ದಾರಿದ್ರ್ಯ, ಅಪರಾಧ ಮನೋವೃತ್ತಿ ತುಂಬಿರುವ ಜೀತ ಹಾಗೂ ಊಳಿಗ ಮಾನ್ಯ ಪದ್ಧತಿಗೆ ಒಗ್ಗಿಹೋಗಿರುವ ಜನರನ್ನು. ಅದೇಕೆ ಆ ಸಮಾಜ ಇಂಥ ಸ್ಥಿತಿಗೆ ಬಂದಿತ್ತೆಂದರೆ, ಒಂದು ಕಾಲದಲ್ಲಿ ಅಲ್ಲಿ ಬೌದ್ಧ ಸಂನ್ಯಾಸಿಗಳು ತುಂಬಿಹೋಗಿದ್ದರು–ಸಾವಿರಾರು ಸಂಖ್ಯೆ ಯಲ್ಲಿ. ಅವನತಿಯ ಅವಧಿಯಲ್ಲಿ ಆ ಸಂನ್ಯಾಸಿಗಳು ಸೋಮಾರಿ ಗಳಾಗಿ ಜೀವನ ನಡೆಸುತ್ತ ಅಪ್ರಯೋಜಕರಾದರು; ಸರಿ, ಸಂಸಾರಿಗಳೂ ಅವರನ್ನೇ ಅನುಕರಿಸಿದರು! ಕ್ರಮೇಣ ಆ ಸಮಾಜ ಇಂಥ ದುರವಸ್ಥೆಗೆ ಬಂದು ನಿಂತಿತು. ಆದ್ದರಿಂದ ಇಂದು ಗೃಹಸ್ಥಜೀವನ ಆರೋಗ್ಯಕರವಾಗಿ, ದೃಢವಾಗಿ ನೆಲೆನಿಂತು ಬೆಳೆಯಬೇಕಾದುದು ಅತ್ಯಗತ್ಯವಾಗಿದೆ. ಅದು ದೈವೀ ಗುಣಗಳಿಂದ, ಘನತೆಯಿಂದ ಕೂಡಿರಬೇಕು. ಇಂಥ ಗೃಹಸ್ಥ ಜೀವನವು ಕೇವಲ ಬಿಹಾರದಲ್ಲಲ್ಲ, ಇಡೀ ಭಾರತದಲ್ಲಿ ಕಾಣ ಸಿಗಬೇಕು. ಮನುಸ್ಮೃತಿಯಲ್ಲಿ ಗೃಹಸ್ಥನನ್ನು ಬಹುವಾಗಿ ಕೊಂಡಾಡಲಾಗಿದೆ. ನಮ್ಮ ಗೃಹಸ್ಥರು ಯಾವ ಆತ್ಮಗೌರವ ವನ್ನು, ದೃಢತೆಯನ್ನು, ಶಕ್ತಿಯನ್ನು ಕಳೆದ ಕೆಲವು ಶತಮಾನ ಗಳಲ್ಲಿ ಕಳೆದುಕೊಂಡಿದ್ದಾರೋ, ಆ ಗುಣಗಳನ್ನು ಎತ್ತಿ ಹಿಡಿ ಯುವ ಅದರ ಒಂದು ಶ್ಲೋಕ ಇದು:

\begin{verse}
‘ಯಸ್ಮಾತ್ ತ್ರಯೋಽಪ್ಯಾಶ್ರಮಿನೋ ಜ್ಞಾನೇನಾನ್ನೇನ ಚಾನ್ವಹಮ್ ।\\ಗೃಹಸ್ಥೇನೈವ ಧಾರ್ಯಂತೇ ತಸ್ಮಾತ್ ಜ್ಯೇಷ್ಠಾಶ್ರಮೀ ಗೃಹೀ ॥’
\end{verse}

ಗೃಹಸ್ಥನ ಹಿರಿಮೆಯನ್ನು ಇಲ್ಲಿ ಈ ಶಬ್ದಗಳಲ್ಲಿ ಹೇಳಿದೆ: \textbf{‘ತಸ್ಮಾತ್ ಜ್ಯೇಷ್ಠಾಶ್ರಮೀ ಗೃಹೀ’}–‘ಆದ್ದರಿಂದ ಗೃಹಸ್ಥಾ ಶ್ರಮವು ಅಗ್ರಗಣ್ಯವಾದದ್ದು.’ ಏಕೆ? \textbf{‘ತ್ರಯೋಪ್ಯಾಶ್ರಮಿನೋ ಜ್ಞಾನೇನಾನ್ನೇನ ಚಾನ್ವಹಂ ಗೃಹಸ್ಥೇನೈವ ಧಾರ್ಯಂತೇ’–} ‘ಬ್ರಹ್ಮಚರ್ಯ, ವಾನಪ್ರಸ್ಥ ಹಾಗೂ ಸಂನ್ಯಾಸ, ಈ ಮೂರೂ ಆಶ್ರಮಿಗಳೂ ಶಿಕ್ಷಣ-ಆಹಾರಗಳ ಮೂಲಕ ನಿರಂತರ ಪೋಷಣೆ ಪಡೆಯುವುದು ಗೃಹಸ್ಥರಿಂದಲೇ’. ಬ್ರಹ್ಮಚಾರಿಯೇನೂ ಸಂಪಾ ದನೆ ಮಾಡುವುದಿಲ್ಲ. ವಾನಪ್ರಸ್ಥಿಯೂ ಅಷ್ಟೆ, ಸಂನ್ಯಾಸಿಯೂ ಅಷ್ಟೆ. ಸಂಪಾದಿಸುವುದು ಗೃಹಸ್ಥರೇ. ಆ ಒಂದು ವರ್ಗದಿಂದ ಉಳಿದ ಮೂರೂ ವರ್ಗಗಳಿಗೆ ಅನ್ನ-ಶಿಕ್ಷಣ ದೊರೆಯುತ್ತದೆ. ಅದೇ ಗೃಹಸ್ಥರ ಹಿರಿಮೆ.

ನೋಡಿ, ಎಂಥ ಸುಂದರವಾದ ಮತ್ತು ಸತ್ಯವಾದ ಅಂಶ! ನಮ್ಮ ಗೃಹಸ್ಥರು ಇದನ್ನು ಮರೆತುಬಿಟ್ಟಿದ್ದರು. ಆ ಹಿರಿಮೆ ಯನ್ನು ಅವರು ಪುನಃ ಗೆದ್ದುಕೊಳ್ಳಬೇಕು. ಅದು ಗೃಹಸ್ಥರ ಜೀವನದಲ್ಲೊಂದು ಹೊಸ ಅಧ್ಯಾಯವಾಗುತ್ತದೆ. ಅವರು ಒಟ್ಟಾಗಿ ಶ್ರಮಿಸಿ, ಮಹತ್ಕಾರ್ಯಗಳನ್ನು ಸಾಧಿಸಬಹುದು. ಈ ಬಗೆಯ ಒಗ್ಗೂಡುವಿಕೆಯಿಂದ ನಮ್ಮ ಸಂಸತ್ತು, ವಿಧಾನಸಭೆ ಗಳು, ಪುರಸಭೆಗಳು, ಪಂಚಾಯತಿಗಳು–ಎಲ್ಲದರಲ್ಲೂ ಉತ್ಕ್ರಾಂತಿಯಾಗಿಬಿಡುತ್ತದೆ. ಅಂಥ ಏಕತೆ, ಸಮನ್ವಯ ಉಂಟಾ ಗಬೇಕು. ಈಗೇನಿದ್ದರೂ ನಮ್ಮ ಗೃಹಸ್ಥರು ತಮ್ಮ ನೆರೆಯವರಿಗೆ ಏನಾದರೂ ತೊಂದರೆ ಕೊಡುತ್ತಾರಷ್ಟೆ. ನನ್ನ ಮನೆಯನ್ನು ಗುಡಿಸಿಕೊಂಡರೆ, ಕಸವನ್ನು ಪಕ್ಕದ ಮನೆಯ ಮುಂದೆ ಹಾಕು ವುದು! ಭಾರತದಲ್ಲಿ ಎಲ್ಲ ಕಡೆಯೂ ಇದೇ ಪದ್ಧತಿ. ಒಳ್ಳೆಯ ದೇನಿದ್ದರೂ ನನಗೆ, ಕೆಟ್ಟದ್ದೇನಿದ್ದರೂ ಪರರಿಗೆ! ಈ ಕ್ಷುದ್ರ ಬುದ್ಧಿಯನ್ನು ಬಿಟ್ಟು ನಮ್ಮ ಜನ ಒಟ್ಟಿಗೆ ದುಡಿಯುವುದನ್ನು ಕಲಿಯಬೇಕು. 

‘ಶ್ರೀರಾಮಕೃಷ್ಣ ವಚನವೇದ’ದಲ್ಲಿ ಗೃಹೀಭಕ್ತರಿಗೆ ಬೋಧನೆ ನೀಡಿರುವ ಅನೇಕ ಅಧ್ಯಾಯಗಳಿವೆ. ಒಮ್ಮೆ ಒಬ್ಬ ಗೃಹಸ್ಥ ಕೇಳು ತ್ತಾನೆ, ‘ನಾವು ದೇವರನ್ನು ಸಾಕ್ಷಾತ್ಕರಿಸಿಕೊಳ್ಳಲು ಸಾಧ್ಯವೆ?’ ಶ್ರೀರಾಮಕೃಷ್ಣರೆನ್ನುತ್ತಾರೆ, ‘ಯಾಕಿಲ್ಲ? ಭಗವಂತ ನಿನ್ನೊಳಗೇ ಇದ್ದಾನೆ. ಆತ ನಿನ್ನಾತ್ಮದ ಆತ್ಮ. ಆದರೆ ನೀನು ನಿನ್ನ ಜೀವನ ದಲ್ಲಿ ಕೆಲವು ಬದಲಾವಣೆಗಳನ್ನು ತಂದುಕೊಳ್ಳಬೇಕು. ಆತನನ್ನು ಆಗ ಕಾಣಬಹುದು.’ ಹೀಗೆ, ಇನ್ನು ಮುಂದಕ್ಕೆ, ‘ಆಧ್ಯಾತ್ಮಿಕ ವಿಕಾಸ’ ಎನ್ನುವುದು ಮಾನವನ ಪ್ರಗತಿಯಲ್ಲಿ ಅತಿಮುಖ್ಯ ಶಬ್ದ ವಾಗುತ್ತದೆ. ಭೌತಿಕ ಹಾಗೂ ಬೌದ್ಧಿಕವಿಕಾಸದೊಂದಿಗೆ ಆಧ್ಯಾ ತ್ಮಿಕ ವಿಕಾಸಕ್ಕೆ ಒತ್ತುಕೊಡಬೇಕು. ‘ನಾನು ಆಧ್ಯಾತ್ಮಿಕವಾಗಿ ಬೆಳೆದಿದ್ದೇನೆಯೆ?–ಪ್ರತಿಯೊಬ್ಬನೂ ಈ ಪ್ರಶ್ನೆ ಹಾಕಿಕೊಳ್ಳ ಬೇಕು. ದೇವಸ್ಥಾನಕ್ಕೆ ಹೋಗಿ ಪೂಜೆ ಮಾಡಿಸಿ; ವಾಪಸು ಮನೆಗೆ ಬಂದು ನಿಮ್ಮನ್ನು ನೀವೇ ಪ್ರಶ್ನಿಸಿಕೊಳ್ಳಿ: ‘ನಾನು ಆಧ್ಯಾತ್ಮಿಕವಾಗಿ ಬೆಳೆದಿದ್ದೇನೆಯೆ?’ ದೇವಸ್ಥಾನಕ್ಕೆ ಹೋಗುವುದೇ ಆಗಲಿ ಇತರ ಧಾರ್ಮಿಕ ವಿಧಿ-ಆಚರಣೆಗಳೇ ಆಗಲಿ ಸಾರ್ಥಕವಾಗುವುದು ಈ ಒಂದು ಉದ್ದೇಶ ಫಲಿಸಿದಾಗ ಮಾತ್ರ. ನೀವು ಆಹಾರ ಸೇವಿಸುತ್ತೀರಿ: ಆದರೆ ಅದರಿಂದ ಶಾರೀರಕ ಬೆಳವಣಿಗೆ ಆಗ ದಿದ್ದರೆ ತಿಂದು ಏನು ಪ್ರಯೋಜನ? ಹಾಗೆಯೇ ಆಧ್ಯಾತ್ಮಿಕ ಜೀವನದಲ್ಲೂ. ನಾವು ಆಧ್ಯಾತ್ಮಿಕವಾಗಿ ವಿಕಾಸಗೊಳ್ಳಬೇಕು. ತತ್ತ್ವವನ್ನು ದೃಷ್ಟಿಯಲ್ಲಿರಿಸಿಕೊಳ್ಳಿ. ಆಗ ಗೃಹಸ್ಥಾಶ್ರಮವೊಂದು ರಮಣೀಯ ಅನುಭವವಾಗುತ್ತದೆ. ಇಂಥ ಗೃಹಸ್ಥಾಶ್ರಮದಿಂದ ಭಾರತದ ಉದ್ಧಾರವೂ ಆಗುತ್ತದೆ.

ವೇದಾಂತ, ಗೀತೆ ಮತ್ತು ಶ್ರೀರಾಮಕೃಷ್ಣರು ತಿಳಿಸಿ ಕೊಟ್ಟಂತಹ ಗೃಹಸ್ಥಜೀವನ ಇದೇ. ಹೀಗೆ ನಿಜವಾದ ಅರ್ಥದಲ್ಲಿ ಗೃಹಸ್ಥರಾಗುವಂತೆ ಶ್ರೀರಾಮಕೃಷ್ಣರು ನಿಮ್ಮನ್ನೆಲ್ಲ ಹರಸಲಿ.

