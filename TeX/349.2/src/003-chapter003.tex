

\newpage

~\thispagestyle{empty}

\newpage

\part{ರಾಜಯೋಗ}

\chapter{ಪ್ರಸ್ತಾವನೆ}

ಅನಾದಿ ಕಾಲದಿಂದಲೂ, ಮಾನವ ಜನಾಂಗದಲ್ಲಿ ಅದ್ಭುತ ಪವಾಡಗಳು ನಡೆದಿರುವುದು ವರದಿಯಾಗಿದೆ. ಕಂಗೊಳಿಸುತ್ತಿರುವ ವೈಜ್ಞಾನಿಕ ಪ್ರಭೆಯಲ್ಲಿ ಬಾಳುತ್ತಿರುವ ಆಧುನಿಕ ಸಮಾಜದಲ್ಲಿಯೂ, ಅಂತಹ ಘಟನೆಗಳು ನಡೆಯುತ್ತಿವೆ ಎಂಬುದನ್ನು ಹೇಳುವವರಿಗೆ ಬರಗಾಲವಿಲ್ಲ. ಇವುಗಳಲ್ಲಿ ಹೆಚ್ಚಿನವು ಅಜ್ಞಾನಿಗಳು. ಮೂಢನಂಬಿಕೆಯವರು ಅಥವಾ ವಂಚಕರಿಂದ ಬಂದಿರುವುವಾದ್ದರಿಂದ ನಂಬಿಕೆಗೆ ಅರ್ಹವಾಗಿಲ್ಲ. ಅನೇಕ ವೇಳೆ ನಾವು ಯಾವುದನ್ನು ಪವಾಡವೆನ್ನುವೆವೋ ಅದು ಅನುಕರಣೆ. ಆದರೆ ಅವರು ಅನುಕರಿಸುವುದು ಯಾವುದನ್ನು? ಯಾವುದನ್ನೂ ಸರಿಯಾಗಿ ವಿಚಾರಿಸದೆ ಅಲ್ಲಗಳೆಯುವುದು ಸರಳ ಅಥವಾ ವೈಜ್ಞಾನಿಕ ಸ್ವಭಾವದ ಗುರುತಲ್ಲ. ಕೇವಲ ತೋರಿಕೆಯ ವಿಜ್ಞಾನಿಗಳು, ಅನೇಕ ಅದ್ಭುತ ಮಾನಸಿಕ ಘಟನೆಗಳನ್ನು ಸರಿಯಾಗಿ ವಿವರಿಸಲಾಗದೆ, ಅವುಗಳ ಇರವನ್ನೆ ಕಡೆಗಣಿಸಲು ಯತ್ನಿಸುವರು. ತಮ್ಮ ಪ್ರಾರ್ಥನೆಯನ್ನು ಮೋಡಗಳಾಚೆ ಇರುವ ಯಾರೊ ದೇವತೆಗಳೋ ದೇವರೋ ಕೇಳಿದರು ಅಥವಾ ತಮ್ಮ ಪ್ರಾರ್ಥನೆಯನ್ನು ಕೇಳಿ ಜಗತ್ತಿನ ಗತಿಯನ್ನು ಬದಲಾಯಿಸಿದರು ಎಂಬುದನ್ನು ನಂಬುವ ಭಕ್ತರಿಗಿಂತಲೂ ಈ ವಿಜ್ಞಾನಿಗಳು ಹೆಚ್ಚು ನಿಂದಾರ್ಹರು. ಈ ರೀತಿಯ ಭಕ್ತರ ನಂಬಿಕೆಗೆ, ಅಂತಹ ದೇವರನ್ನೇ ನೆಚ್ಚಬೇಕೆಂದು ಬೋಧಿಸಿ ಈಗ ಅದು ಅವರ ಸಹಜಗುಣವಾಗುವಂತೆ ಮಾಡಿದ ದೋಷಪೂರ್ಣ ಶಿಕ್ಷಣಕ್ರಮ ಅಥವಾ ಅವರ ಅಜ್ಞಾನ ಕಾರಣವೆನ್ನಬಹುದು. ಆದರೆ ಈ ನುರಿತ ವಿಜ್ಞಾನಿಗಳು ಯಾವುದೇ ನೆವದಿಂದಲೂ ಈ ದೋಷಾರೋಪಣೆಯಿಂದ ತಪ್ಪಿಸಿಕೊಳ್ಳಲಾರರು. 

\vskip 5pt

ಸಾವಿರಾರು ವರ್ಷಗಳಿಂದ ಇಂತಹ ಅಸಾಮಾನ್ಯ ವಿಷಯಗಳನ್ನು ಅಧ್ಯಯನ ಮಾಡಿ, ಪರೀಕ್ಷಿಸಿ, ಸಾಮಾನ್ಯೀಕರಿಸಿರುವರು; ಮಾನವನ ಆಧ್ಯಾತ್ಮಿಕ ಸಾಮರ್ಥ್ಯವನ್ನೆಲ್ಲಾ ವಿಶ್ಲೇಷಣೆ ಮಾಡಿರುವರು; ಇಂತಹ ಶ್ರಮದ ಫಲವೇ ರಾಜಯೋಗ. ರಾಜಯೋಗವು ಕ್ಷಮಾಪಣೆಗೆ ಯೋಗ್ಯರಲ್ಲದ ಕೆಲವು ಆಧುನಿಕ ವಿಜ್ಞಾನಿಗಳಂತೆ, ವಿವರಿಸಲು ಅಸಾಧ್ಯವಾದ ವಿಷಯವನ್ನು ಇಲ್ಲವೆನ್ನುವುದಿಲ್ಲ. ಅದರ ಬದಲು, ಪವಾಡಗಳು, ನಮ್ಮ ಪ್ರಾರ್ಥನೆಗೆ ಉತ್ತರ, ನಮ್ಮ ನಂಬಿಕೆಯಲ್ಲಿರುವ ಮಹಾ ಸಾಮರ್ಥ್ಯ – ಇವುಗಳೆಲ್ಲ ವಾಸ್ತವಿಕವಾಗಿ ನಿಜವಾದರೂ ಇವುಗಳನ್ನೆಲ್ಲ ಯಾರೊ ಮುಗಿಲಿನ ಆಚೆ ಕುಳಿತಿರುವ ದೇವರೊ, ದೇವತೆಗಳೊ ಮಾಡುತ್ತಿರುವರು ಎಂಬ ವಿವರಣೆ ಅಷ್ಟು ಸಮರ್ಪಕವಾಗಿಲ್ಲ – ಎಂದು ಮೂಢನಂಬಿಕೆಯವರಿಗೆ ನಯವಾಗಿಯಾದರೂ ಖಂಡಿತವಾಗಿ ಹೇಳುತ್ತದೆ. ಪ್ರತಿಯೊಂದು ವ್ಯಕ್ತಿಯೂ ಮಾನವ ಜನಾಂಗದ ಹಿಂದೆ ಇರುವ ಅನಂತಶಕ್ತಿ ಮತ್ತು ಜ್ಞಾನ ಹರಿದುಬರುವುದಕ್ಕೆ ಒಂದು ಪ್ರಣಾಳ ವಿದ್ದಂತೆ ಎಂದು ರಾಜಯೋಗವು ಹೇಳುತ್ತದೆ. ಮಾನವನಲ್ಲಿ ಆಸೆ ಆಕಾಂಕ್ಷೆಗಳಿವೆ; ಅವನನ್ನು ಪೂರ್ಣನನ್ನಾಗಿ ಮಾಡುವ ಶಕ್ತಿಯೂ ಕೂಡ ಅವನಲ್ಲೇ ಇದೆ. ಎಂದಾದರೂ, ಎಲ್ಲಿಯಾದರೂ ಮಾನವನ ಬಯಕೆ ಅಥವಾ ಪ್ರಾರ್ಥನೆ ಈಡೇರಿದರೆ ಅದು ಈ ಅನಂತ ಸಾಗರದಿಂದ ಬಂದಿದೆಯೇ ಹೊರತು ಎಲ್ಲಿಯೋ ಮೋಡಗಳಾಚೆ ಇರುವ ಅತಿಪ್ರಾಕೃತ ವ್ಯಕ್ತಿಗಳಿಂದ ಬರಲಿಲ್ಲವೆಂದು ರಾಜಯೋಗವು ಬೋಧಿಸುವುದು. ಅತಿಪ್ರಾಕೃತ ದೇವರ ಭಾವನೆ ಮಾನವನ ಕ್ರಿಯೋತ್ತೇಜಕ ಶಕ್ತಿಯನ್ನೇನೋ ಸ್ವಲ್ಪ ಮಟ್ಟಿಗೆ ಜಾಗ್ರತಗೊಳಿಸಬಹುದು. ಆದರೆ ಅದು ಜನತೆಗೆ ಆಧ್ಯಾತ್ಮಿಕ ದುರ್ಬಲತೆಯನ್ನು ತರುವುದು, ಅಂಜಿಕೆಯನ್ನೂ ಮೂಢನಂಬಿಕೆಯನ್ನೂ ತುಂಬುವುದು. ಅದು ಮಾನವನ ಸಹಜ ದೌರ್ಬಲ್ಯದಲ್ಲಿ ನಂಬಿಕೆಯಿಡುವ ಹೀನಸ್ಥಿತಿಗೆ ಒಯ್ಯುತ್ತದೆ. ಯೋಗಿಯು ಅತಿಪ್ರಾಕೃತವೆಂಬುದಿಲ್ಲ, ಪ್ರಕೃತಿಯ ಸ್ಥೂಲ ಮತ್ತು ಸೂಕ್ಷ್ಮ ಅಭಿವ್ಯಕ್ತಿಗಳಿವೆ ಎನ್ನುತ್ತಾನೆ. ಸೂಕ್ಷ್ಮವು ಕಾರಣ, ಸ್ಥೂಲ ಪರಿಣಾಮ. ಸ್ಥೂಲವನ್ನು ನಾವು ಸುಲಭವಾಗಿ ಇಂದ್ರಿಯಗಳ ಮೂಲಕ ಗ್ರಹಿಸಬಹುದು. ಆದರೆ ಸೂಕ್ಷ್ಮವು ಅಷ್ಟು ಸುಲಭವಲ್ಲ. ರಾಜಯೋಗದ ಅಭ್ಯಾಸವು ಅತಿ ಸೂಕ್ಷ್ಮ ಗ್ರಹಣಶಕ್ತಿಯನ್ನು ಪಡೆಯಲು ಸಹಾಯ ಮಾಡುವುದು. 

\vskip 5pt

ಎಲ್ಲಾ ಸಾಂಪ್ರದಾಯಿಕ ಭಾರತೀಯ ತತ್ತ್ವಶಾಸ್ತ್ರಗಳಿಗೂ ಮತ್ತು ಸಿದ್ಧಾಂತಗಳಿಗೂ ಕೂಡ ಗುರಿ ಇರುವುದು ಒಂದೆ. ಅದೇ ಪೂರ್ಣತೆಯ ಮೂಲಕ ಮುಕ್ತಿಯನ್ನು ಪಡೆಯುವುದು. ಅದಕ್ಕೆ ಇರುವ ಮಾರ್ಗ ಯೋಗ. ಯೋಗವೆಂಬ ಪದಕ್ಕೆ ಅತಿ ವಿಶಾಲವಾದ ಅರ್ಥವಿದೆ. ಆದರೆ ಸಾಂಖ್ಯ ಮತ್ತು ವೇದಾಂತಗಳೆರಡೂ ಯೋಗ ಶಬ್ದಕ್ಕೆ ಯಾವುದೋ ಒಂದು ನಿರ್ದಿಷ್ಟ ಅರ್ಥವನ್ನು ಸೂಚಿಸುತ್ತವೆ.

\vskip 5pt

ರಾಜಯೋಗವೆಂದು ಕರೆಯಲ್ಪಡುವ ಯೋಗವೇ ಈ ಗ್ರಂಥದ ವಸ್ತು. ರಾಜಯೋಗಕ್ಕೆ ಪತಂಜಲಿಯ ಯೋಗಸೂತ್ರವೇ ಅತಿ ಮುಖ್ಯವಾದ ಪ್ರಮಾಣ ಮತ್ತು ಇದೇ ಅದರ ಪಠ್ಯಗ್ರಂಥ. ಉಳಿದ ದಾರ್ಶನಿಕರಿಗೆ ತಾತ್ತ್ವಿಕ ವಿಷಯಗಳಲ್ಲಿ ಕೆಲವು ವೇಳೆ ರಾಜಯೋಗದೊಂದಿಗೆ ಭಿನ್ನಾಭಿಪ್ರಾಯವಿದ್ದರೂ ಅದರ ಸಾಧನೆಯ ವಿಧಾನವನ್ನು ನಿರ್ಣಾಯಕವಾಗಿ ಒಪ್ಪುತ್ತಾರೆ. ಈ ಗ್ರಂಥದ ಮೊದಲನೆಯ ಭಾಗದಲ್ಲಿ, ನ್ಯೂಯಾರ್ಕ್​ ನಗರದಲ್ಲಿ ಈ ಗ್ರಂಥಕರ್ತನು ನೀಡಿದ ಕೆಲವು ಪ್ರವಚನಗಳಿವೆ. ಎರಡನೆಯದು ಸರಳ ಭಾಷ್ಯದಿಂದ ಕೂಡಿದ ಪತಂಜಲಿ ಯೋಗಸೂತ್ರದ ಭಾವಾನುವಾದ. ಸಾಧ್ಯವಾದ ಮಟ್ಟಿಗೆ ಪಾರಿಭಾಷಿಕ ಪದಗಳನ್ನು ಬಿಟ್ಟು ಸರಳವಾದ ಸಂವಾದ ಶೈಲಿಯಲ್ಲಿಡಲು ಪ್ರಯತ್ನಿಸಲಾಗಿದೆ. ಮೊದಲನೆಯ ಭಾಗದಲ್ಲಿ ಅಭ್ಯಾಸ ಮಾಡುವವರ ಸೌಕರ್ಯಕ್ಕಾಗಿ, ಸುಲಭವಾದ, ನಿರ್ದಿಷ್ಟವಾದ ಸಲಹೆಗಳನ್ನು ಕೊಟ್ಟಿದೆ. ಆದರೆ ಅವುಗಳಲ್ಲಿ ಕೆಲವು ಹೊರತು, ರಾಜಯೋಗದ ಉಳಿದ ಭಾಗವನ್ನು ಪ್ರತ್ಯಕ್ಷ ಒಬ್ಬ ಗುರುವಿನ ಸಹಾಯದಿಂದ ಮಾತ್ರ ಸುರಕ್ಷಿತವಾಗಿ ಕಲಿಯಬಹುದೆಂದು, ಮುಖ್ಯವಾಗಿ, ಮನಃಪೂರ್ವಕವಾಗಿ ಉಚ್ಚರಿಸುತ್ತೇವೆ. ಈ ಪ್ರಸಂಗವನ್ನು ಓದಿ, ಈ ವಿಷಯದ ಮೇಲೆ ಹೆಚ್ಚು ತಿಳಿದುಕೊಳ್ಳಬೇಕೆಂಬ ಬಯಕೆ ಉಂಟಾದರೆ, ಅದನ್ನು ಈಡೇರಿಸುವ ಗುರುವು ಸಿಕ್ಕದೆ ಇರಲಾರ.

ಪತಂಜಲಿಯ ಸಿದ್ಧಾಂತವು ಸಾಂಖ್ಯ ಸಿದ್ಧಾಂತದ ಆಧಾರದ ಮೇಲಿರುವುದು. ಇವೆರಡಕ್ಕೂ ಇರುವ ಭಿನ್ನಾಭಿಪ್ರಾಯ ಕಡಮೆ. ಬಹಳ ಮುಖ್ಯವಾದ ಎರಡು ಭಿನ್ನಾಭಿಪ್ರಾಯ ಯಾವುದೆಂದರೆ, ಮೊದಲನೆಯದು ಆದಿಗುರುವಿನಂತಿರುವ ಈಶ್ವರನನ್ನು ಪತಂಜಲಿಯು ಒಪ್ಪಿಕೊಳ್ಳುವನು. ಸಾಂಖ್ಯರು ಒಪ್ಪಿಕೊಳ್ಳುವ ದೇವರು, ಬಹುಮಟ್ಟಿಗೆ ಪೂರ್ಣತೆಯನ್ನು ಪಡೆದಿರುವ, ತಾತ್ಕಾಲಿಕವಾಗಿ ಒಂದು ಕಲ್ಪದ ಮೇಲ್ವಿಚಾರಣೆಯಲ್ಲಿರುವ ಪುರುಷ. ಎರಡನೆಯದೆ, ಯೋಗಿಯು ಆತ್ಮ ಅಥವಾ ಪುರುಷರಂತೆ ಮನಸ್ಸು ಕೂಡ ಸರ್ವವ್ಯಾಪಿಯೆಂದು ನಂಬುವನು. ಆದರೆ ಸಾಂಖ್ಯರು ಇದನ್ನು ನಂಬುವುದಿಲ್ಲ.

\begin{flushright}
\textbf{ಗ್ರಂಥಕರ್ತ}
\end{flushright}

\newpage

\thispagestyle{empty}

~

\vfill
\vskip 5cm


\begin{center}
\textbf{ಪ್ರತಿಯೊಂದು ಆತ್ಮನಲ್ಲಿಯೂ ದಿವ್ಯತೆ ಹುದುಗಿರುವುದು. ಈ ಸುಪ್ತವಾದ ದಿವ್ಯತೆಯನ್ನು ಬಾಹ್ಯ ಮತ್ತು ಆಂತರಿಕ ಪ್ರಕೃತಿಯ ನಿಗ್ರಹದಿಂದ ವ್ಯಕ್ತಪಡಿಸುವುದೇ ಜೀವನದ ಗುರಿ.}
\end{center}

\begin{center}
\textbf{ಇದನ್ನು ಕರ್ಮಯೋಗದಿಂದಾದಲಿ, ಭಕ್ತಿಯೋಗದಿಂದಾಗಲಿ, ರಾಜಯೋಗದಿಂದಾಗಲಿ, ಜ್ಞಾನಯೋಗದಿಂದಾಗಲಿ, ಯಾವುದಾದರೂ ಒಂದು ಮಾರ್ಗದಿಂದ ಅಥವಾ ಎಲ್ಲಾ ಮಾರ್ಗಗಳ ಸಂಯೋಗದಿಂದ ಆಗಲಿ ಸಾಧಿಸಿ ಮುಕ್ತರಾಗಿ.}
\end{center}

\begin{center}
\textbf{ಇದೇ ಧರ್ಮದ ಸರ್ವಸ್ವ. ಸಿದ್ಧಾಂತ, ನಂಬಿಕೆ, ಬಾಹ್ಯಾಚಾರ, ಪವಿತ್ರ ಗ್ರಂಥ, ದೇವಸ್ಥಾನ, ವಿಗ್ರಹ – ಇವೆಲ್ಲಾ ಗೌಣ.}
\end{center}

\begin{flushright}
\textbf{$-$ ಸ್ವಾಮಿ ವಿವೇಕಾನಂದ}
\end{flushright}
~

\vfill
\vskip 5cm

\chapter{ಪೀಠಿಕೆ}

ನಮ್ಮ ಜ್ಞಾನವೆಲ್ಲ ಅನುಭವವನ್ನು ಆಧರಿಸಿದೆ. ನಾವು ಯಾವುದನ್ನು ಅನುಮಾನ ಜ್ಞಾನವೆನ್ನುವೆವೊ, ಎಲ್ಲಿ ನಾವು ವಿಶೇಷದಿಂದ ಸಾಮಾನ್ಯಕ್ಕೆ ಹೋಗುವೆವೊ, ಅಥವಾ ಸಾಮಾನ್ಯದಿಂದ ವಿಶೇಷದ ಕಡೆ ಬರುವೆವೊ, ಅದಕ್ಕೆಲ್ಲ ಅನುಭವವೇ ತಳಹದಿಯಾಗಿದೆ. ನಿಶ್ಚಿತವಾದ ವೈಜ್ಞಾನಿಕ ಶಾಸ್ತ್ರಗಳಲ್ಲಿ ಜನರಿಗೆ ಸತ್ಯವು ಸುಲಭವಾಗಿ ಕಾಣುವುದು. ಏಕೆಂದರೆ ಅದು ಪ್ರತಿಯೊಬ್ಬ ಮಾನವನ ಪ್ರತ್ಯೇಕ ಅನುಭವದೊಂದಿಗೆ ತಾಳೆ ಹೊಂದುತ್ತದೆ. ನಾವು ಯಾವುದನ್ನಾದರೂ ನಂಬಬೇಕೆಂದು ವಿಜ್ಞಾನಿಯು ಹೇಳುವುದಿಲ್ಲ. ತನ್ನ ಸ್ವಂತ ಅನುಭವದ ಮತ್ತು ವಿಚಾರದ ಪರಿಣಾಮವಾಗಿ ಅವನಿಗೆ ಕೆಲವು ಫಲಿತಾಂಶಗಳು ಗೊತ್ತಿವೆ. ತನ್ನ ನಿರ್ಣಯಗಳನ್ನು ನಾವು ನಂಬಬೇಕೆಂದು ಅವನು ಹೇಳಿದಾಗ, ಮಾನವಕೋಟಿಯ ಕೆಲವು ಸಾರ್ವತ್ರಿಕ ಅನುಭವಗಳೊಂದಿಗೆ ಅವು ತಾಳೆ ಹೊಂದುತ್ತವೆಯೇ ಎಂಬುದನ್ನು ಪರಿಶೀಲಿಸಬೇಕೆಂದು ಅವನು ಹೇಳುತ್ತಾನೆ. ಪ್ರತಿಯೊಂದು ಖಚಿತವಾದ ವಿಜ್ಞಾನಕ್ಕೂ ಇಡೀ ಮಾನವ ಕೋಟಿಗೆ ಸಾಮಾನ್ಯವಾದ ಒಂದು ತಳಹದಿ ಇದೆ. ಆದಕಾರಣ ಅವುಗಳಿಂದ ಬಂದ ನಿರ್ಣಯ ಸರಿಯೋ, ತಪ್ಪೋ ಎಂಬುದನ್ನು ತಕ್ಷಣವೆ ನೋಡಬಹುದು. ಆಧ್ಯಾತ್ಮಿಕ ಪ್ರಪಂಚದಲ್ಲಿ ಅಂತಹ ಒಂದು ತಳಹದಿ ಇದೆಯೇ ಇಲ್ಲವೆ – ಎಂಬುದೆ ಈಗಿನ ಪ್ರಶ್ನೆ. ನಾನು ಈ ಪ್ರಶ್ನೆಗೆ ಇದೆ ಎಂತಲೂ, ಇಲ್ಲವೆಂತಲೂ ಉತ್ತರ ಹೇಳಬೇಕಾಗಿದೆ.

\vskip 5pt

ಸಾಧಾರಣವಾಗಿ ಜಗತ್ತಿನಲ್ಲಿ ಭೋಧಿಸಲ್ಪಡುತ್ತಿರುವ ಧರ್ಮಗಳು ಶ್ರದ್ಧೆಯನ್ನು ಅವಲಂಬಿಸಿವೆ. ಅನೇಕವು ಇವುಗಳಲ್ಲಿ ಕೇವಲ ವಿವಿಧ ಬಗೆಯ ಸಿದ್ಧಾಂತಗಳನ್ನು ಮಾತ್ರ ಒಳಗೊಂಡಿರುತ್ತವೆ. ಧರ್ಮಗಳು ಪರಸ್ಪರ ಹೋರಾಡುತ್ತಿರುವುದಕ್ಕೆ ಕಾರಣ ಇದೇ. ಪುನಃ ಈ ಸಿದ್ಧಾಂತಗಳು ನಿಂತಿರುವುದು ನಂಬಿಕೆಯ ಮೇಲೆ. ಒಬ್ಬನು ಮೋಡದಾಚೆ ಮಹಾವ್ಯಕ್ತಿಯೊಂದು ಕುಳಿತುಕೊಂಡು ಪ್ರಪಂಚವನ್ನು ಆಳುತ್ತಿರುವನು ಎನ್ನುತ್ತಾನೆ. ತನ್ನ ಪ್ರತಿಪಾದನೆಯ ಆಧಾರವೊಂದನ್ನು ಮಾತ್ರ ಅವಲಂಬಿಸಿ ಅದನ್ನು ನಂಬಬೇಕೆಂದು ನಮಗೆ ಹೇಳುತ್ತಾನೆ. ಅದರಂತೆಯೇ ನನ್ನಲ್ಲಿಯೂ ಸ್ವಂತ ಭಾವನೆಗಳು ಇರಬಹುದು; ಇತರರೂ ಅದನ್ನು ನಂಬಬೇಕೆಂದು ಹೇಳುತ್ತೇನೆ. ಅವರು ಕಾರಣ ಕೇಳಿದರೆ ನಾನು ಕೊಡಲಾರೆ. ಆದುದರಿಂದಲೆ ಇಂದಿನ ಕಾಲದಲ್ಲಿ ಧರ್ಮ ಮತ್ತು ತಾತ್ತ್ವಿಕ ದರ್ಶನಗಳಿಗೆ ಒಂದು ಕೆಟ್ಟ ಹೆಸರು ಬಂದಿರುವುದು. ಪ್ರತಿಯೊಬ್ಬ ವಿದ್ಯಾವಂತನೂ ಕೂಡ, “ಓ, ಈ ಧರ್ಮಗಳೆಲ್ಲ ಒಂದು ಸಿದ್ಧಾಂತದ ಕಂತೆ, ಅದನ್ನು ಪರೀಕ್ಷಿಸುವುದಕ್ಕೆ ಯಾವ ಒಂದು ಪ್ರಮಾಣವೂ ಇಲ್ಲ. ಪ್ರತಿಯೊಬ್ಬನೂ ತನಗೆ ತೋಚಿದುದನ್ನು ಬೋಧಿಸುತ್ತಿರುವನು” – ಎಂದು ಹೇಳುವಂತೆ ತೋರುವುದು. ಆದರೆ, ಎಲ್ಲಾ ಬೇರೆ ಬೇರೆ ಸಿದ್ಧಾಂತಗಳನ್ನೂ, ಬೇರೆ ಬೇರೆ ದೇಶಗಳಲ್ಲಿರುವ ಭಿನ್ನಭಿನ್ನ ಪಂಥಗಳ ಭಾವನೆಗಳನ್ನೂ ಆಳುತ್ತಿರುವ ಸರ್ವಸಾಮಾನ್ಯವಾದ ಒಂದು ನಂಬಿಕೆ ಧರ್ಮದಲ್ಲಿ ಇದ್ದೇ ಇರುವುದು. ಅವುಗಳ ಮೂಲಕ್ಕೆ ಹೋದರೆ ಅವೂ ಕೂಡ ಸರ್ವಸಾಮಾನ್ಯವಾದ ಅನುಭವದ ಮೇಲೆ ನಿಂತಿರುವುದು ನಮಗೆ ಕಂಡುಬರುತ್ತದೆ. 

\vskip 0.5cm

ಮೊದಲನೆಯದಾಗಿ, ಪ್ರಪಂಚದ ವಿವಿಧ ಧರ್ಮಗಳನ್ನು ನಾವು ವಿಶ್ಲೇಷಣೆ ಮಾಡಿದರೆ, ಅವನ್ನು ಎರಡು ವಿಧವಾಗಿ ವಿಂಗಡಿಸಬಹುದು: ಒಂದು ಧರ್ಮಗ್ರಂಥಗಳನ್ನುಳ್ಳದ್ದು, ಮತ್ತೊಂದು ಇಲ್ಲದೆ ಇರುವುದು. ಯಾವುದಕ್ಕೆ ಧರ್ಮಗ್ರಂಥವಿರುವುದೋ ಅದೇ ಅತ್ಯಂತ ಶಕ್ತಿಯುತವಾದದ್ದು ಮತ್ತು ಅದೇ ಹೆಚ್ಚು ಅನುಯಾಯಿಗಳನ್ನು ಹೊಂದಿರುವುದು. ಧರ್ಮಗ್ರಂಥಗಳು ಇಲ್ಲದೆ ಇರುವ ಧರ್ಮಗಳು ಬಹುಪಾಲು ಅಳಿದುಹೋಗಿವೆ. ಅವುಗಳಲ್ಲಿ ಹೊಸದಾಗಿರುವ ಕೆಲವಕ್ಕೆ ಇರುವ ಅನುಯಾಯಿಗಳು ಬಹಳ ಕಡಿಮೆ. ಆದರೂ ಇವೆಲ್ಲದರಲ್ಲಿಯೂ ನಮಗೆ ಒಂದು ಏಕಾಭಿಪ್ರಾಯ ಕಂಡುಬರುತ್ತದೆ. ಅದಾವುದೆಂದರೆ ಅವುಗಳು ಬೋಧಿಸುವ ಸತ್ಯಗಳು ನಿರ್ದಿಷ್ಟ ವ್ಯಕ್ತಿಗಳಿಗಾದ ಅನುಭವಗಳ ಫಲ–ಎಂಬುದು. ಕ್ರೈಸ್ತಮತೀಯನು, ಮತ್ತೊಬ್ಬರಿಗೆ “ನನ್ನ ಧರ್ಮವನ್ನು ನಂಬಿ; ಕ್ರಿಸ್ತನನ್ನು ನಂಬಿ; ಅವನು ದೇವರ ಅವತಾರವೆಂದು ನಂಬಿ; ದೇವರಲ್ಲಿ, ಆತ್ಮನಲ್ಲಿ ಮತ್ತು ಅದರ ಒಂದು ಉತ್ತಮ ಸ್ಥಿತಿಯಲ್ಲಿ ನಂಬಿಕೆಯಿಡಿ” ಎಂದು ಹೇಳುವನು. ಇದಕ್ಕೆ ಕಾರಣವನ್ನು ಕೇಳಿದರೆ ನಾನು ಅವನ್ನು ನಂಬುತ್ತೇನೆ ಎನ್ನುವನು. ಆದರೆ ನೀವು ಕ್ರೈಸ್ತಧರ್ಮದ ಮೂಲಕ್ಕೆ ಹೋದರೆ ಅದು ಒಂದು ವ್ಯಕ್ತಿಯ ಅನುಭವದ ಮೇಲೆ ನಿಂತಿದೆ ಎನ್ನುವುದು ನಿಮಗೆ ಗೊತ್ತಾಗುವುದು. ತಾನು ದೇವರನ್ನು ನೋಡಿದೆನು ಎಂದು ಕ್ರಿಸ್ತನು ಹೇಳಿದನು, ಅವನ ಶಿಷ್ಯರು ತಾವು ದೇವರನ್ನು ಅನುಭವಿಸಿದೆವೆಂದು ಹೇಳಿದರು. ಅದರಂತೆಯೇ ಬೌದ್ಧಧರ್ಮವು ಬುದ್ಧನ ಅನುಭವದ ಮೇಲೆ ನಿಂತಿದೆ. ಕೆಲವು ಸತ್ಯಗಳನ್ನು ಅವನು ಅನುಭವಿಸಿದನು; ಅವನ್ನು ಕಂಡನು ಮತ್ತು ಅವುಗಳೊಂದಿಗೆ ಸಂಬಂಧವನ್ನು ಬೆಳೆಸಿದನು; ಮತ್ತು ಆ ಸತ್ಯಗಳನ್ನೆ ಜಗತ್ತಿಗೆ ಬೋಧಿಸಿದನು. ಅದರಂತೆಯೇ ಹಿಂದೂಗಳೂ ಕೂಡ. ಅವರ ಶಾಸ್ತ್ರದಲ್ಲಿ ಬರುವ ಋಷಿಗಳು ಕೆಲವು ಸತ್ಯಗಳನ್ನು ತಾವು ಅನುಭವಿಸಿದೆವೆಂದು ಸಾರಿದರು ಮತ್ತು ಇವುಗಳನ್ನು ಜನರಿಗೆ ಬೋಧಿಸಿದರು. ಆದುದರಿಂದ ಪ್ರಪಂಚದ ಧರ್ಮಗಳೆಲ್ಲವೂ ಕೂಡ ಪ್ರತ್ಯಕ್ಷ ಅನುಭವವೆಂದು ಸರ್ವಸಾಮಾನ್ಯವಾದ ಮತ್ತು ವಜ್ರೋಪಮವಾದ ತಳಹದಿಯ ಮೇಲೆ ನಿಂತಿವೆ, ಎನ್ನುವುದು ಸ್ಪಷ್ಟವಾಗಿದೆ. ಬೋಧಕರೆಲ್ಲರೂ ದೇವರನ್ನು ಕಂಡರು; ತಮ್ಮಾತ್ಮವನ್ನು ಕಂಡರು, ತಮ್ಮ ಭವಿಷ್ಯವನ್ನು ಕಂಡರು, ತಮ್ಮ ಅನಂತತೆಯನ್ನು ದರ್ಶಿಸಿದರು. ತಾವು ಯಾವುದನ್ನು ಕಂಡರೋ ಅದನ್ನು ಬೋಧಿಸಿದರು. ಆದರೆ ಈ ಒಂದು ವ್ಯತ್ಯಾಸವಿದೆ. ಈಗಿರುವ ಅನೇಕ ಧರ್ಮಾವಲಂಬಿಗಳು, ಆ ಅನುಭವಗಳೆಲ್ಲ ಇಂದಿಗೆ ಸಾಧ್ಯವಿಲ್ಲ, ಆ ಧರ್ಮವನ್ನು ಸ್ಥಾಪಿಸಿದ ಕೆಲವರಿಗೆ ಮಾತ್ರ ಸಾಧ್ಯವಾಗಿತ್ತು ಎಂಬ ವಿಚಿತ್ರ ಧೋರಣೆಯನ್ನು ತಳೆದಿರುವರು. ಸದ್ಯಕ್ಕೆ ಈ ಅನುಭವಗಳೆಲ್ಲ ಪುರಾತನ ಪಳೆಯುಳಿಕೆಗಳಾಗಿವೆ; ಆದಕಾರಣ ನಾವು ಧರ್ಮವನ್ನು ನಂಬಿಕೆಯ ಮೇಲೆ ಸ್ವೀಕರಿಸಬೇಕು ಎನ್ನುವರು. ಆದರೆ ಇದನ್ನು ನಾನು ಸಂಪೂರ್ಣ\break ವಾಗಿ ವಿರೋಧಿಸುತ್ತೇನೆ. ಈ ಪ್ರಪಂಚದಲ್ಲಿ ಯಾವುದಾದರೊಂದು ಜ್ಞಾನಶಾಖೆಗೆ\break ಸಂಬಂಧಿಸಿದಂತೆ ಒಂದು ಅನುಭವವಿದ್ದರೆ, ಆ ಅನುಭವವು ಹಿಂದೆ ಕೋಟ್ಯಂತರ ವೇಳೆ ಸಾಧ್ಯವಾಗಿತ್ತು ಮತ್ತು ಮುಂದೆಯೂ ಯಾವಾಗಲೂ ಸಾಧ್ಯ ಎನ್ನುವುದು ಸಂಪೂರ್ಣವಾಗಿ ಯುಕ್ತಿಯುಕ್ತವಾಗಿದೆ. ಸಮಾನತೆ ಪ್ರಕೃತಿಯ ಕಠೋರ ನಿಯಮ. ಯಾವುದು ಒಮ್ಮೆ ನಡೆಯಿತೊ ಅದು ಯಾವಾಗಲೂ ನಡೆಯಬಲ್ಲುದು. 

\vskip 0.5cm

ಆದಕಾರಣ ಯೋಗಾಚಾರ್ಯರು, ಧರ್ಮವು ಹಿಂದಿನ ಕಾಲದವರ ಅನುಭವದ ಮೇಲೆ ಮಾತ್ರ ನಿಂತಿಲ್ಲ, ಯಾರಿಗೆ ಅದೇ ಅನುಭವವು ಆಗಿಲ್ಲವೋ ಅವರು ಧಾರ್ಮಿಕರಾಗ\break ಲಾರರು ಎಂದು ಸಾರುತ್ತಾರೆ. ಈ ಅನುಭವಗಳನ್ನು ಹೇಗೆ ಪಡೆಯಬೇಕೆಂದು ಬೋಧಿಸುವ ಶಾಸ್ತ್ರವೇ ಯೋಗ. ಧರ್ಮದ ಅನುಭವವನ್ನು ಪಡೆಯುವವರೆಗೆ ಅದರ ವಿಚಾರವಾಗಿ ಹೆಚ್ಚು ಮಾತನಾಡಿ ಪ್ರಯೋಜನವಿಲ್ಲ. ದೇವರ ಹೆಸರಿನಲ್ಲಿ ಏತಕ್ಕೆ ಇಷ್ಟೊಂದು ಅಶಾಂತಿ ಮತ್ತು ಕಲಹ? ದೇವರ ಹೆಸರಿನಲ್ಲಿ ನಡೆದಿರುವಷ್ಟು ರಕ್ತಪಾತ ಬೇರಾವ ಕಾರಣಗಳಿಂದಲೂ ನಡೆದಿಲ್ಲ. ಏಕೆಂದರೆ ಜನರು ಧರ್ಮದ ಮೂಲಕ ಹೋಗಲಿಲ್ಲ; ತಮ್ಮ ಪೂರ್ವಿಕರ ಆಚಾರ ವ್ಯವಹಾರಗಳಿಗೆ ಒಪ್ಪಿಗೆ ನೀಡುವುದರಲ್ಲಿಯೇ ತೃಪ್ತರಾದರು, ಉಳಿದವರು ಕೂಡ ಹಾಗೆಯೇ ಮಾಡಬೇಕೆಂದು ಬಯಸಿದರು. ಒಬ್ಬ ವ್ಯಕ್ತಿಯು ತಾನು ಆತ್ಮನ ಇರುವಿಕೆಯನ್ನು ಅನುಭವಿಸದೆ ತನಗೆ ಆತ್ಮವಿದೆ ಎಂದು ಹೇಳುವುದಕ್ಕೆ, ಅಥವಾ ತಾನು ದೇವರನ್ನು ಕಾಣದೆ ಅಂತಹ ದೇವರು ಇರುವನೆಂದು ಹೇಳುವುದಕ್ಕೆ ಅವನಿಗೆ ಯಾವ ಅಧಿಕಾರವಿದೆ? ದೇವರಿದ್ದರೆ ಅವನನ್ನು ನಾವು ನೋಡಬೇಕು; ಆತ್ಮವಿದ್ದರೆ, ಅದು ನಮಗೆ ಗೋಚರಿಸಬೇಕು. ಸಾಧ್ಯವಿಲ್ಲದೆ ಇದ್ದರೆ, ಅವುಗಳನ್ನು ನಂಬದಿರುವುದೇ ಮೇಲು. ಢಾಂಬಿಕನಾಗಿರುವುದಕ್ಕಿಂತ ಖಂಡಿತವಾಗಿ ನಾಸ್ತಿಕನಾಗಿರುವುದು ಮೇಲು. ಆಧುನಿಕ ವಿದ್ಯಾವಂತರ ಒಂದು ಪಂಗಡದವರ ದೃಷ್ಟಿಯಲ್ಲಿ ಧರ್ಮ, ಪರತತ್ತ್ವ, ಪರಮಾತ್ಮನೊಬ್ಬನು ಇರುವನು ಎಂದು ಅರಸುವಿಕೆ – ಇವೆಲ್ಲ ನಿರರ್ಥಕ. ಇನ್ನೊಂದು ಕಡೆ, ಅಲ್ಪವಿದ್ಯಾವಂತರ ದೃಷ್ಟಿಯಲ್ಲಿ ಇಂತಹ ಆಧ್ಯಾತ್ಮಿಕ ಸತ್ಯಗಳಿಗೆ ಯಾವ ಆಧಾರವೂ ಇಲ್ಲ; ಆದರೆ ಇದು ಪ್ರಪಂಚಕ್ಕೆ ಒಳ್ಳೆಯದನ್ನು ಮಾಡುವ ಕ್ರಿಯೋತ್ತೇಜನ ಶಕ್ತಿಯನ್ನು ಒದಗಿಸುತ್ತದೆ, ಇದೊಂದೇ ಅದರ ಪ್ರಯೋಜನ. ಮಾನವರು ದೇವರನ್ನು ನಂಬಿದರೆ ಅವರು ಉತ್ತಮರಾಗಬಹುದು, ನೀತಿವಂತರಾಗಬಹುದು; ಇದರಿಂದ ಜನಾಂಗವು ಉತ್ತಮವಾಗಬಹುದು. ಇಂತಹ ಅಭಿಪ್ರಾಯವನ್ನು ಅವರು ಇಟ್ಟುಕೊಂಡಿರುವುದಕ್ಕಾಗಿ ನಾವು ಅವರನ್ನು ದೂರುವುದಕ್ಕಾಗುವುದಿಲ್ಲ. ಏಕೆಂದರೆ ಅವರಿಗೆ ಸಿಕ್ಕುವ ಬೋಧನೆಯಿಷ್ಟೆ: ನಿಸ್ಸಾರವಾದ ಶಬ್ದಜಾಲವನ್ನು ನಂಬಿಕೊಂಡಿರುವುದು. ಮಾತನ್ನು ನಂಬಿ ಬಾಳುವುದನ್ನೇ ಅವರು ಕಲಿತಿರುವುದು. ಇದನ್ನು ಮಾಡಲು ಅವರಿಗೆ ಸಾಧ್ಯವೆ? ಅವರಿಗೆ ಏನಾದರೂ ಇದು ಸಾಧ್ಯವಾದರೆ, ಮಾನವನ ಸ್ವಭಾವದಲ್ಲಿ ನನಗೆ ಸ್ವಲ್ಪವಾದರೂ ಗೌರವ ಉಳಿಯುವುದಿಲ್ಲ. ಮಾನವನಿಗೆ ಸತ್ಯ ಬೇಕಾಗಿದೆ; ಪ್ರತ್ಯಕ್ಷವಾಗಿ ಸತ್ಯವನ್ನು ಅನುಭವಿಸಲು ಅವನು ಇಚ್ಛಿಸುವನು. ಅದು ಅವನಿಗೆ ಗೊತ್ತಾದ ಮೇಲೆ, ಸಾಕ್ಷಾತ್ಕಾರವಾದ ಮೇಲೆ, ತನ್ನ ಹೃದಯಾಂತರಾಳದಲ್ಲಿ ಅದನ್ನು ತಿಳಿದಮೇಲೆ ಮಾತ್ರ ಸಂಶಯಗಳೆಲ್ಲ ನಾಶವಾಗುವುವು; ಅಜ್ಞಾನದ ಕತ್ತಲೆ ಮಾಯವಾಗುವುದು; ಅವನ ಹೃದಯ ನಿರ್ಮಲವಾಗುವುದು – ಎಂದು ವೇದಗಳು ಸಾರುತ್ತವೆ. “ಹೇ ಅಮೃತಪುತ್ರರೆ, ಕೇಳಿ! ಸ್ವರ್ಗದಲ್ಲಿ ವಾಸಿಸುವವರೂ ಕೂಡ ಕೇಳಿ! ದಾರಿ ಸಿಕ್ಕಿತು! ಈ ಅಜ್ಞಾನದಿಂದ ಪಾರಾಗುವುದಕ್ಕೆ ಒಂದು ದಾರಿ ಇದೆ! ಅದೇ ಅಜ್ಞಾನಾತೀತನಾದ ಪರಮಾತ್ಮನ ಸಾಕ್ಷಾತ್ಕಾರದಿಂದ. ಬೇರಾವ ಮಾರ್ಗವೂ ಇಲ್ಲ. ”

\vskip 0.5cm

ರಾಜಯೋಗವು ಇಂತಹ ಪರಮಸತ್ಯವನ್ನು ಪಡೆಯಲು, ಅನುಷ್ಠಾನ–ಯೋಗ್ಯವಾದ, ವೈಜ್ಞಾನಿಕ ರೀತಿಯಿಂದ ಕೂಡಿದ ಒಂದು ಮಾರ್ಗವನ್ನು ಮಾನವನ ಮುಂದಿಡಲು ಯತ್ನಿಸುತ್ತದೆ. ಮೊದಲನೆಯದಾಗಿ, ಪ್ರತಿಯೊಂದು ಶಾಸ್ತ್ರವೂ ಕೂಡ ತನ್ನದೇ ಆದ ಒಂದು ಸಂಶೋಧನಾಕ್ರಮವನ್ನು ಹೊಂದಿರಬೇಕು. ನೀವು ಖಗೋಳಶಾಸ್ತ್ರಜ್ಞರಾಗಬೇಕಾದರೆ ಸುಮ್ಮನೆ ಕುಳಿತುಕೊಂಡು ಖಗೋಳಶಾಸ್ತ್ರ ಎಂದು ಅರಚುತ್ತಿದ್ದರೆ ಅದು ನಿಮಗೆ ಬರುವುದಿಲ್ಲ. ಇದರಂತೆಯೆ, ರಸಾಯನಶಾಸ್ತ್ರವೂ ಕೂಡ. ನಾವು ಒಂದು ನಿಯಮವನ್ನು ಅನುಸರಿಸಬೇಕು, ಒಂದು ಪ್ರಯೋಗ ಶಾಲೆಗೆ ಹೋಗಬೇಕು, ಅನೇಕ ವಸ್ತುಗಳನ್ನು ತೆಗೆದುಕೊಳ್ಳಬೇಕು ಅವುಗಳನ್ನು ಮಿಶ್ರಮಾಡಿ ಅವುಗಳೊಂದಿಗೆ ಪ್ರಯೋಗ ನಡೆಸಬೇಕು. ಇದರಿಂದ ನಮಗೆ ರಸಾಯನಶಾಸ್ತ್ರಜ್ಞಾನ ಬರುವುದು. ನೀವು ಖಗೋಳಶಾಸ್ತ್ರಜ್ಞರಾಗಬೇಕಾದರೆ, ಒಂದು ವೀಕ್ಷಣಾಲಯಕ್ಕೆ ಹೋಗಬೇಕು. ಒಂದು ದೂರದರ್ಶಕ ಯಂತ್ರವನ್ನು ತೆಗೆದುಕೊಳ್ಳಬೇಕು ಮತ್ತು ನಕ್ಷತ್ರಗಳನ್ನು ಮತ್ತು ಗ್ರಹಗಳನ್ನು ಪರೀಕ್ಷಿಸಬೇಕು. ಅನಂತರ ನೀವು ಖಗೋಳ ಶಾಸ್ತ್ರಜ್ಞರಾಗುವಿರಿ. ಪ್ರತಿಯೊಂದು ಶಾಸ್ತ್ರಕ್ಕೂ ಅದರದೇ ಒಂದು ಮಾರ್ಗವಿದೆ. ನಾನು ನಿಮಗೆ ಸಾವಿರಾರು ಧರ್ಮೋಪದೇಶಗಳನ್ನು ನೀಡಬಹುದು. ಅವನ್ನು ನೀವು ಅನುಷ್ಠಾನಕ್ಕೆ ತಂದಹೊರತು ಅವು ನಿಮ್ಮನ್ನು ಧಾರ್ಮಿಕರನ್ನಾಗಿ ಮಾಡಲಾರವು. ಎಲ್ಲಾ ದೇಶಗಳ ಎಲ್ಲಾ ಕಾಲದ, ಪರಿಶುದ್ಧರೂ ನಿಃಸ್ವಾರ್ಥರೂ ಆದ, ಪ್ರಪಂಚಕ್ಕೆ ಹಿತವನ್ನು ಮಾಡುವುದೊಂದಲ್ಲದೆ ಮತ್ತಾವ ಉದ್ದೇಶವನ್ನೂ ಹೊಂದಿರದ ಮಹಾ ಋಷಿಗಳ ಸತ್ಯವಚನಗಳಿವು. ಇಂದ್ರಿಯಗಳು ನಮಗೆ ತೋರಿಸುವುದಕ್ಕಿಂತ, ಅವುಗಳನ್ನು ಮೀರಿದ ಕೆಲವು ಸತ್ಯಗಳನ್ನು ತಾವು ಸಂದರ್ಶಿಸಿರುವೆವು, ಅವನ್ನು ಬೇಕಾದರೆ ಪರೀಕ್ಷೆ ಮಾಡಬಹುದು ಎಂದು ಅವರು ಸಾರುವರು. ಅವರು ನಮಗೆ, ಈ ಮಾರ್ಗವನ್ನು ಹಿಡಿದು ಸತ್ಯವಾಗಿ ಸಾಧನೆ ಮಾಡಿ, ಅನಂತರ ಈ ಅತೀಂದ್ರಿಯ ಸತ್ಯ ಗೋಚರಿಸದೇ ಇದ್ದರೆ, ನಾವು ಹೇಳುವ ಮಾತಿನಲ್ಲಿ ತಿರುಳಿಲ್ಲ ಎನ್ನುವುದರಲ್ಲಿ ಒಂದು ಸತ್ಯವಿದೆ ಎನ್ನುತ್ತಾರೆ. ಹಾಗೆ ಮಾಡುವುದಕ್ಕೆ ಮುಂಚೆ ಅವರ ಮಾತಿನ ಸತ್ಯವನ್ನು ನಾವು ಅಲ್ಲಗಳೆದರೆ ಅದು ತರ್ಕಬದ್ಧವಲ್ಲ. ಆದಕಾರಣ ನಾವು ಶ್ರದ್ಧೆಯಿಂದ ಅವರು ಹೇಳಿದ ರೀತಿ ಸಾಧನೆ ಮಾಡಬೇಕು. ಆಗ ಬೆಳಕು ಬರುವುದು. 

\vskip 0.5cm

ನಾವು ಜ್ಞಾನವನ್ನು ಪಡೆಯುವುದಕ್ಕಾಗಿ ಸಾಮಾನ್ಯೀಕರಣ ವಿಧಾನವನ್ನು ಉಪ\break ಯೋಗಿಸುತ್ತೇವೆ. ಈ ವಿಧಾನವು ಬಾಹ್ಯಪರಿಶೀಲನೆಯ ಮೇಲಿದೆ. ಮೊದಲು ನಾವು ವಿಷಯಗಳನ್ನು ಪರಿಶೀಲನೆ ಮಾಡುತ್ತೇವೆ. ಅನಂತರ ಅವುಗಳನ್ನು ಒಂದು ಸರ್ವಸಾಮಾನ್ಯ ನಿಯಮಕ್ಕೆ ತರುತ್ತೇವೆ. ಅನಂತರ ಅವುಗಳಿಂದ ನಿರ್ಣಯಗಳನ್ನೂ ಮೂಲತತ್ತ್ವಗಳನ್ನೂ ಪಡೆಯುತ್ತೇವೆ. ಅದರಂತೆಯೇ ಮೊದಲು ನಮ್ಮ ಮನಸ್ಸಿನಲ್ಲಿ ನಡೆಯುತ್ತಿರುವ ವಿಷಯಗಳನ್ನು ಪರಿಶೀಲನೆ ಮಾಡುವ ಶಕ್ತಿಯನ್ನು ಪಡೆಯುವತನಕ ಮಾನವನ ಮನಸ್ಸಿನ ಆಂತರಿಕ ಸ್ಥಿತಿಯ ಜ್ಞಾನವನ್ನು ಪಡೆಯಲಾಗುವುದಿಲ್ಲ. ಬಾಹ್ಯ ಪ್ರಪಂಚದೊಳಗೆ ಇರುವ ವಸ್ತುಗಳನ್ನು ನೋಡುವುದೇನೊ ಸುಲಭ. ಏತಕ್ಕೆಂದರೆ, ಇದಕ್ಕೋಸುಗ ಅನೇಕ ಯಂತ್ರಗಳನ್ನು ಕಂಡುಹಿಡಿದಿರುವರು. ಆದರೆ ಆಂತರಿಕ ಪ್ರಪಂಚದ ಪರಿಶೀಲನೆಯಲ್ಲಿ ನಮಗೆ ಸಹಾಯಮಾಡಲು ಇಂತಹ ಯಾವ ಯಂತ್ರಗಳೂ ಇಲ್ಲ. ನಿಜವಾದ ವೈಜ್ಞಾನಿಕ ಶಾಸ್ತ್ರ ಬೇಕಾದರೆ ನಾವು ಪರಿಶೀಲನೆ ಮಾಡಲೇಬೇಕೆಂದು ನಮಗೆ ಗೊತ್ತಿದೆ. ಸರಿಯಾದ ವಿಶ್ಲೇಷಣೆ ಇಲ್ಲದೆ ಯಾವ ಶಾಸ್ತ್ರವೂ ಕೂಡ ಯುಕ್ತಿ ಪೂರಿತವಾಗಲಾರದು–ಕೇವಲ ಒಂದು ಬಾಯಿಮಾತಿನ ಸಿದ್ಧಾಂತವಾಗುವುದು. ಈ ಕಾರಣದಿಂದಲೇ, ಪರಿಶೀಲನೆ ಮಾಡಲು ಮಾರ್ಗವನ್ನು ಕಂಡ ಕೆಲವರು ಮನಶ್ಯಾಸ್ತ್ರಜ್ಞರ ವಿನಾ ಇತರ ಮನಶ್ಯಾಸ್ತ್ರಜ್ಞರು ಅನಾದಿಯಿಂದಲೂ ತಮ್ಮೊಳಗೆ ಕಾದಾಡುತ್ತಿರುವರು. 

\vskip 5pt

ಮೊದಲನೆಯದಾಗಿ ರಾಜಯೋಗವು, ಅಂತರಂಗದ ಸ್ಥಿತಿಯನ್ನು ಪರೀಕ್ಷಿಸುವ ಮಾರ್ಗವೊಂದನ್ನು ನಮಗೆ ತೋರಿಸಿಕೊಡುತ್ತದೆ. ಇಲ್ಲಿ ಮನಸ್ಸೇ ಉಪಕರಣ. ಗ್ರಹಣಶಕ್ತಿಯನ್ನು ನಾವು ಸರಿಯಾಗಿ ರೂಢಿಸಿ ಆಂತರಿಕ ವಸ್ತುವಿನ ಮೇಲೆ ಬಿಟ್ಟಾಗ ಅದು ಮನಸ್ಸನ್ನು ವಿಶ್ಲೇಷಿಸಿ ಅದಕ್ಕೆ ಸಂಬಂಧಪಟ್ಟ ವಿಷಯಗಳನ್ನು ನಮಗೆ ಒದಗಿಸುವುದು. ನಮ್ಮ ಮನಸ್ಸಿನ ಶಕ್ತಿಯು ಬೆಳಕಿನ ಕಿರಣಗಳಂತೆ ಚದರಿಹೋಗಿದೆ. ಅದನ್ನು ಕೇಂದ್ರೀಕರಿಸಿದಾಗ ಪ್ರಕಾಶಮಾನವಾಗುತ್ತದೆ. ನಮ್ಮ ಜ್ಞಾನಾರ್ಜನೆಗೆ ಇದೊಂದೇ ದಾರಿ. ಪ್ರತಿಯೊಬ್ಬರೂ ಕೂಡ, ಅದನ್ನು ಬಾಹ್ಯ ಪ್ರಪಂಚದಲ್ಲಿ ಮತ್ತು ಆಂತರಿಕ ಪ್ರಪಂಚದಲ್ಲಿ ಉಪಯೋಗಿಸುತ್ತಿರುವರು. ಆದರೆ ಯಾವ ಸೂಕ್ಷ್ಮ ಪರಿಶೀಲನಾಶಕ್ತಿಯನ್ನು ವಿಜ್ಞಾನಿಯು ಬಾಹ್ಯವಸ್ತುವಿನ ಮೇಲೆ ಬೀರುತ್ತಾನೆಯೋ, ಅದನ್ನೇ ಮನಶ್ಯಾಸ್ತ್ರಜ್ಞನು ಆಂತರಿಕ ವಸ್ತುವಿನ ಮೇಲೆ ಹರಿಸಬೇಕು. ಇದಕ್ಕೆ ಬಹಳ ಅಭ್ಯಾಸ ಬೇಕು. ಜನನಾರಭ್ಯ ನಮಗೆ ಬಾಹ್ಯ ವಸ್ತುವನ್ನು ಲಕ್ಷಿಸುವಂತೆ ಬೋಧಿಸಿರುವರು; ಎಂದಿಗೂ ಕೂಡ ಆಂತರಿಕ ವಸ್ತುವಿನ ಕಡೆಗೆ ಇಲ್ಲ. ಅದಕ್ಕೋಸ್ಕರವಾಗಿಯೇ ಆಂತರಿಕ ವಸ್ತುರಚನೆಯನ್ನು ನೋಡುವ ಅಭ್ಯಾಸವನ್ನು ಅನೇಕರು ನಮ್ಮಲ್ಲಿ ಕಳೆದುಕೊಂಡಿರುವರು. ಹೊರಗೆ ಹೋಗುವ ಮನಸ್ಸನ್ನು ನಿಲ್ಲಿಸಿ, ಅದನ್ನು ಅಂತರ್ಮುಖಮಾಡಿ, ಆ ಶಕ್ತಿಯೆಲ್ಲವನ್ನೂ ಕೇಂದ್ರೀಕರಿಸಿ ಮನಸ್ಸಿನ ಸ್ಥಿತಿಯನ್ನು ಅರಿಯುವುದು, ಅದನ್ನು ವಿಶ್ಲೇಷಿಸುವುದು ಮತ್ತು ಅದರ ಗತಿಯನ್ನು ತಿರುಗಿಸುವುದು ಕಷ್ಟಸಾಧ್ಯವಾದ ಕೆಲಸ. ವೈಜ್ಞಾನಿಕ ರೀತಿಯಲ್ಲಿ ಈ ವಿಷಯವನ್ನು ಕಲಿಯಬೇಕಾದರೆ ಇದೊಂದೇ ಮಾರ್ಗ. 

\vskip 5pt

ಅಂತಹ ಜ್ಞಾನದಿಂದ ಪ್ರಯೋಜನವೇನು? ಮೊದಲನೆಯದಾಗಿ ಜ್ಞಾನವೇ ಆ ಜ್ಞಾನದ ಅತ್ಯುತ್ತಮವಾದ ಪ್ರಯೋಜನ. ಎರಡನೆಯದಾಗಿ ಅದರಿಂದ ಬೇರೆ ಪ್ರಯೋಜನವೂ ಇದೆ. ಇದು ನಮ್ಮ ದುಃಖವನ್ನೆಲ್ಲ ನಿವಾರಣೆ ಮಾಡುವುದು. ತನ್ನ ಸ್ವಂತ ಮನಸ್ಸನ್ನು ವಿಶ್ಲೇಷಿಸುವಾಗ, ಮಾನವನು ಎಂದಿಗೂ ನಾಶವಾಗದ, ನಿರಂತರವೂ ಪರಿಶುದ್ಧವೂ ಪರಿಪೂರ್ಣವೂ ಆದ ವಸ್ತುವನ್ನು ನೋಡುವುದು ಸಾಧ್ಯವಾದರೆ, ಅವನು ಇನ್ನು ಎಂದಿಗೂ ಕಷ್ಟದಲ್ಲಿ ನರಳಲಾರ, ದುಃಖದಲ್ಲಿರಲಾರ. ನಮ್ಮ ದುಃಖಗಳೆಲ್ಲ ಬರುವುದು ಅಂಜಿಕೆಯಿಂದ, ಈಡೇರದ ಬಯಕೆಗಳಿಂದ. ತಾನೆಂದಿಗೂ ಸಾಯುವುದಿಲ್ಲವೆಂದು ಮಾನವನಿಗೆ ಗೊತ್ತಾದರೆ, ಅನಂತರ ಅವನಿಗೆ ಸಾವಿನ ಭಯವೇ ಇರುವುದಿಲ್ಲ. ತಾನು ಪರಿಪೂರ್ಣನೆಂದು ತಿಳಿದಮೇಲೆ ಅವನಲ್ಲಿ ಯಾವ ಹೀನ ಆಸೆಯೂ ಇರುವುದಿಲ್ಲ. ಈ ಎರಡು ಕಾರಣಗಳು ಯಾವಾಗ ಇಲ್ಲವೋ, ಆಗ ದುಃಖವಿನ್ನೆಂದಿಗೂ ಇರುವುದಿಲ್ಲ. ಈ ದೇಹದಲ್ಲಿರುವಾಗಲೇ ಪೂರ್ಣಾನಂದವಿರುವುದು. 

ಈ ಜ್ಞಾನವನ್ನು ಪಡೆದುಕೊಳ್ಳಬೇಕಾದರೆ ಮಾನಸಿಕ ಏಕಾಗ್ರತೆ ಎಂಬ ಒಂದೇ ಮಾರ್ಗ ಇರುವುದು. ರಸಾಯನ ಶಾಸ್ತ್ರಜ್ಞನು ಪ್ರಯೋಗಶಾಲೆಯಲ್ಲಿ ತನ್ನ ಮಾನಸಿಕ ಶಕ್ತಿಯನ್ನೆಲ್ಲ ಕೇಂದ್ರೀಕರಿಸಿ ತಾನು ವಿಶ್ಲೇಷಣೆ ಮಾಡುತ್ತಿರುವ ವಸ್ತುಗಳ ಕಡೆ ಬಿಡುವನು. ಇದರಿಂದ ಅವುಗಳ ಗುಟ್ಟನ್ನು ಕಂಡುಹಿಡಿಯುವನು. ಖಗೋಳ ಶಾಸ್ತ್ರಜ್ಞನು ತನ್ನ ಮಾನಸಿಕ ಶಕ್ತಿಯನ್ನೆಲ್ಲ ಕೇಂದ್ರೀಕರಿಸಿ ದೂರದರ್ಶಕಯಂತ್ರದ ಮೂಲಕ ಆಕಾಶದ ಕಡೆ ತಿರುಗಿಸುವನು. ಸೂರ್ಯ–ಚಂದ್ರ–ನಕ್ಷತ್ರಗಳೆಲ್ಲ ತಮ್ಮ ರಹಸ್ಯವನ್ನು ಅವನಿಗೆ ಬಹಿರಂಗಪಡಿಸುವುವು. ನಾನು ಯಾವ ವಿಷಯದ ಮೇಲೆ ಮಾತಾಡುತ್ತಿರುವೆನೊ, ಆ ವಿಷಯದ ಮೇಲೆ ನಾನು ಹೆಚ್ಚು ಏಕಾಗ್ರನಾದಷ್ಟೂ ನಿಮಗೆ ಅದನ್ನು ಚೆನ್ನಾಗಿ ಹೇಳಬಲ್ಲೆ. ನೀವು ನನ್ನನ್ನು ಕೇಳುತ್ತಿರುವಿರಿ. ನೀವು ಅದರ ಮೇಲೆ ಹೆಚ್ಚು ಮನಸ್ಸನ್ನು ಇಟ್ಟಷ್ಟೂ ನಾನು ಹೇಳುವುದನ್ನು ಚೆನ್ನಾಗಿ ತಿಳಿದುಕೊಳ್ಳುವಿರಿ. 

ಮಾನಸಿಕ ಶಕ್ತಿಯ ಏಕಾಗ್ರತೆಯ ಮೂಲಕವಲ್ಲದೆ ಪ್ರಪಂಚದಲ್ಲಿರುವ ಸಕಲ ವಿದ್ಯೆಗಳನ್ನು ಹೇಗೆ ಪಡೆದರು? ಪ್ರಪಂಚವನ್ನು ಯಾವ ರೀತಿ ತಟ್ಟಬೇಕು, ಅದಕ್ಕೆ ಯಾವ ರೀತಿ ಆವಶ್ಯಕವಾದ ಪೆಟ್ಟನ್ನು ಕೊಡಬೇಕು ಎಂಬುದು ಗೊತ್ತಿದ್ದರೆ ಅದು ತನ್ನ ರಹಸ್ಯವನ್ನೆಲ್ಲ ನಮಗೆ ಕೊಡಲು ಸಿದ್ಧವಾಗಿರುವುದು. ಪೆಟ್ಟಿನ ಶಕ್ತಿ ಮತ್ತು ವೇಗ ಏಕಾಗ್ರತೆಯಿಂದ ಬರುವುದು, ಮಾನಸಿಕ ಶಕ್ತಿಗೆ ಮಿತಿಯೆಂಬುದಿಲ್ಲ. ಅದು ಯಾವುದೇ ವಿಷಯದ ಮೇಲೆ ಏಕಾಗ್ರವಾದಷ್ಟೂ ಹೆಚ್ಚು ಶಕ್ತಿ ಕೇಂದ್ರೀಕೃತವಾಗುವುದು, ಇದೇ ರಹಸ್ಯ. 

\vskip 5pt

ಬಾಹ್ಯವಸ್ತುಗಳ ಮೇಲೆ ಮನಸ್ಸನ್ನು ಏಕಾಗ್ರಗೊಳಿಸುವುದು ಸುಲಭ. ಮನಸ್ಸು\break ಸಾಧಾರಣವಾಗಿ ಹೊರಗೆ ಹೋಗುವುದು. ಆದರೆ ಎಲ್ಲಿ ಜ್ಞಾತೃ ಮತ್ತು ಜ್ಞೇಯ ಒಂದೇ ಆಗಿದೆಯೋ ಅಂತಹ ವಿಷಯಗಳಾದ ಧರ್ಮ, ಮನಶ್ಯಾಸ್ತ್ರ, ತತ್ತ್ವಶಾಸ್ತ್ರ, ಇವುಗಳಲ್ಲಿ ಅದು ಸುಲಭ ಸಾಧ್ಯವಲ್ಲ. ವಸ್ತು ಆಂತರಿಕವಾದುದು, ಮನಸ್ಸೇ ಆಜ್ಞೇಯ ವಸ್ತು. ಮನಸ್ಸನ್ನೇ ಪರೀಕ್ಷಿಸುವುದು ಅತ್ಯಾವಶ್ಯಕವಾಗಿದೆ; ಮನಸ್ಸು ಮನಸ್ಸನ್ನೇ ಪರೀಕ್ಷಿಸಬೇಕು. ಆತ್ಮಚಿಂತನೆ ಎಂಬ ಮಾನಸಿಕ ಶಕ್ತಿ ಇರುವುದು ನಮಗೆ ಗೊತ್ತಿದೆ. ನಾನು ನಿಮ್ಮೊಂದಿಗೆ ಮಾತನಾಡುತ್ತಿರುವೆನು, ಅದೇ ಸಮಯದಲ್ಲಿ ನಾನು ಬೇರೆಯವನಂತೆ, ಮಾತನಾಡುತ್ತಿರುವುದನ್ನು ಕೇಳುತ್ತಲೂ, ತಿಳಿಯುತ್ತಲೂ ಇರುವೆನು. ನೀವು ಏಕಕಾಲದಲ್ಲಿ ಕೆಲಸಮಾಡುತ್ತೀರಿ ಮತ್ತು ಆಲೋಚಿಸುತ್ತೀರಿ. ಅದೇ ಕಾಲದಲ್ಲಿ ನಿಮ್ಮ ಮನಸ್ಸಿನ ಅಂಶವೊಂದು ಬೇರೆಯಾಗಿ ನಿಂತು ನೀವು ಆಲೋಚಿಸುತ್ತಿರುವುದನ್ನು ನೋಡುತ್ತಿರುವುದು. ಮನಸ್ಸಿನ ಶಕ್ತಿಯನ್ನು ಕೇಂದ್ರೀಕರಿಸಿ ಅದರ ಮೇಲೆಯೇ ಅದನ್ನು ಬಿಡಬೇಕು. ಕೋರೈಸುತ್ತಿರುವ ಸೂರ್ಯನ ಕಿರಣಗಳೆದುರಿಗೆ ಕಗ್ಗತ್ತಲಿಂದ ತುಂಬಿದ ಸ್ಥಳ ತನ್ನ ರಹಸ್ಯವನ್ನು ತೋರುವಂತೆ, ಈ ಕೇಂದ್ರೀಕೃತವಾದ ಮನಸ್ಸು ತನ್ನ ಗಹನ ರಹಸ್ಯಗಳನ್ನೆಲ್ಲ ತೋರಿಸುವುದು. ಹೀಗೆ ನಿಜವಾದ ಆಧ್ಯಾತ್ಮಿಕ ನಂಬಿಕೆಗಳ ಮೂಲಕ್ಕೆ ಬರುವೆವು. ನಮಗೆ ಆತ್ಮವಿದೆಯೆ, ಜೀವನ ಕ್ಷಣಭಂಗುರವೇ, ಅಥವಾ ಸನಾತನವೇ, ಪ್ರಪಂಚದಲ್ಲಿ ದೇವರು ಇರುವನೆ, ಅಥವಾ ಇಲ್ಲವೆ ಎಂಬುದನ್ನು ನಾವು ಪ್ರತ್ಯಕ್ಷವಾಗಿ ನೋಡುವೆವು. ಇವು ನಮಗೆ ಗೋಚರಿಸುವುವು. ರಾಜಯೋಗ ಬೋಧಿಸಬೇಕೆಂದಿರುವುದೇ ಇದನ್ನು. ಮನಸ್ಸನ್ನು ಹೇಗೆ ಕೇಂದ್ರೀಕರಿಸುವುದು, ಮನಸ್ಸಿನ ರಹಸ್ಯತಮವಾದ ಪ್ರದೇಶಗಳನ್ನು ಹೇಗೆ ಕಂಡುಹಿಡಿಯುವುದು, ಅಲ್ಲಿ ಹುದುಗಿರುವ ವಿಷಯಗಳನ್ನು ಹೇಗೆ ಒಂದು ಸಾಮಾನ್ಯ ನಿಯಮಕ್ಕೆ ತರುವುದು, ಅನಂತರ ಅವುಗಳ ವಿಚಾರವಾಗಿ ಹೇಗೆ ನಮ್ಮದೇ ಆದ ನಿರ್ಣಯಕ್ಕೆ ಬರುವುದು–ಎಂಬುದೇ ರಾಜಯೋಗದ ಉಪದೇಶದ ಗುರಿ. ಆದಕಾರಣ ಇದು ನಮ್ಮನ್ನು ಎಂದಿಗೂ, ನಾವು ಆಸ್ತಿಕರೆ ಅಥವಾ ನಾಸ್ತಿಕರೆ, ಕ್ರೈಸ್ತರೆ, ಯೆಹೂದ್ಯರೆ ಅಥವಾ ಬೌದ್ಧರೆ ಎಂಬ ಪ್ರಶ್ನೆಯನ್ನು ಕೇಳುವುದೇ ಇಲ್ಲ. ನಾವೆಲ್ಲ ಮಾನವರು. ಅದೇ ಸಾಕು. ಧರ್ಮವನ್ನು ಹುಡುಕುವುದಕ್ಕೆ ಪ್ರತಿಯೊಬ್ಬ ಮಾನವನಿಗೂ ಒಂದು ಹಕ್ಕಿದೆ, ಶಕ್ತಿ ಇದೆ. ಪ್ರತಿಯೊಬ್ಬ ಮಾನವನಿಗೂ ಇದು ಏಕೆ ಎಂಬ ಪ್ರಶ್ನೆಯನ್ನು ಕೇಳುವ ಅಧಿಕಾರವಿದೆ. ಮತ್ತೆ ತಾನೇ ಪ್ರಯತ್ನಪಟ್ಟರೆ ಈ ಪ್ರಶ್ನೆಗೆ ಉತ್ತರ ಪಡೆಯಬಹುದು. 

\vskip 5pt

ರಾಜಯೋಗವನ್ನು ತಿಳಿದುಕೊಳ್ಳಬೇಕಾದರೆ ಯಾವ ನಂಬಿಕೆಯೂ ಅನಾವಶ್ಯಕ\break\ ವೆಂಬುದನ್ನು ತಿಳಿದುದಾಯಿತು. ನೀವಾಗಿ ಅದನ್ನು ಕಾಣುವವರೆಗೂ ಯಾವುದನ್ನೂ ನೀವು ನಂಬಬೇಕಾಗಿಲ್ಲ ಎಂಬುದೇ ಅದು ನಮಗೆ ಹೇಳುವುದು. ಸತ್ಯಕ್ಕೆ ಮತ್ತಾವ ಆಸರೆಯೂ ಬೇಕಾಗಿಲ್ಲ. ಜಾಗ್ರದವಸ್ಥೆಯಲ್ಲಿ ನಮಗೆ ಕಾಣುವ ವಸ್ತುವನ್ನು ಪ್ರತಿಪಾದಿಸುವುದಕ್ಕೆ ಯಾವುದಾದರೂ ಕಲ್ಪನೆಗಳ ಅಥವಾ ಕನಸುಗಳ ಆವಶ್ಯಕತೆ ಇದೆಯೆ? ಎಂದಿಗೂ ಇಲ್ಲ. ಈ ರಾಜಯೋಗದ ಅಧ್ಯಯನಕ್ಕೆ ಅನವರತ ಅಭ್ಯಾಸವೂ ಮತ್ತು ದೀರ್ಘಕಾಲವೂ ಹಿಡಿಯುತ್ತದೆ. ಈ ಸಾಧನೆಯ ಸ್ವಲ್ಪಭಾಗ ಶಾರೀರಿಕವಾದುದು, ಆದರೆ ಬಹುಭಾಗ ಮಾನಸಿಕವಾದುದು. ನಾವು ಮುಂದುವರಿದಂತೆ ಮನಸ್ಸಿಗೆ ಶರೀರದೊಂದಿಗೆ ಎಷ್ಟು ಹತ್ತಿರ ಸಂಬಂಧವಿದೆ ಎಂಬುದು ಗೊತ್ತಾಗುವುದು. ಮನಸ್ಸು ದೇಹದ ಒಂದು ಸೂಕ್ಷ್ಮ ಅವಸ್ಥೆ, ಅದು ಶರೀರದ ಮೇಲೆ ತನ್ನ ಪರಿಣಾಮವನ್ನು ಬೀರಬಲ್ಲದು ಎಂಬುದನ್ನು ನೀವು ನಂಬುವುದಾದರೆ, ದೇಹವೂ ಮನಸ್ಸಿನ ಮೇಲೆ ಪರಿಣಾಮ ಬೀರುತ್ತದೆ ಎಂಬುದು ಯುಕ್ತವಾಗಿಯೆ ಇದೆ. ದೇಹವು ಅಸ್ವಸ್ಥವಾಗಿದ್ದರೆ ಮನಸ್ಸು ಕೂಡ ಅಸ್ವಸ್ಥವಾಗುವುದು. ದೇಹವು ಆರೋಗ್ಯದಲ್ಲಿದ್ದರೆ ಮನಸ್ಸು ಕೂಡ ಆರೋಗ್ಯವಾಗಿ, ಬಲವಾಗಿರುವುದು. ಒಬ್ಬನು ಕೋಪ ತಾಳಿದರೆ ಮನಸ್ಸೂ ವಿಚಲಿತವಾಗುವುದು. ಮನಸ್ಸು ವಿಚಲಿತವಾದಾಗ ದೇಹವೂ ತೊಂದರೆಗೀಡಾಗುವುದು. ಬಹುಪಾಲು ಮಾನವರ ಮನಸ್ಸು ಅವರ ಶರೀರದ ವಶಕ್ಕೆ ಒಳಪಟ್ಟಿರುವುದು; ಏಕೆಂದರೆ ಅವರ ಮನಸ್ಸು ಇನ್ನೂ ಅಭಿವೃದ್ಧಿ ಆಗಲಿಲ್ಲ. ಬಹುಪಾಲು ಮಾನವ ಜನಾಂಗ ಪ್ರಾಣಿಗಳಿಗಿಂತ ಏನೂ ಬಹಳ ಮೇಲಿಲ್ಲ. ಇದೊಂದೆ ಅಲ್ಲ. ಅನೇಕ ವೇಳೆ ಅವರಲ್ಲಿರುವ ಮನಃಸ್ವಾಧೀನತೆ, ಕೆಳಮಟ್ಟದ ಪ್ರಾಣಿಗಳಿಗಿಂತ ಸ್ವಲ್ಪ ಮೇಲಿರಬಹುದು ಅಷ್ಟೆ. ನಮ್ಮ ಮನಸ್ಸಿನ ಮೇಲೆ ನಮಗೆ ಹಿಡಿತವಿರುವುದು ಬಹಳ ಸ್ವಲ್ಪ. ನಮ್ಮ ಮನಸ್ಸಿನ ಮೇಲೆ ಅಂತಹ ಒಂದು ಸ್ವಾಧೀನತೆಯನ್ನು ಪಡೆಯುವುದಕ್ಕೆ ನಾವು ಕೆಲವು ದೈಹಿಕ ಸಹಾಯಗಳನ್ನು ಸ್ವೀಕರಿಸಬೇಕು. ದೇಹವನ್ನು ನಾವು ಸಾಕಾದಷ್ಟು ಸ್ವಾಧೀನಕ್ಕೆ ಒಳಪಡಿಸಿಕೊಂಡಮೇಲೆ, ಮನಸ್ಸನ್ನು ನಿಗ್ರಹಿಸಲು ಪ್ರಯತ್ನಿಸಬಹದು. ಆಗ ಮನಸ್ಸು ನಮ್ಮ ಹಿಡಿತಕ್ಕೆ ಬಂದಮೇಲೆ ನಮ್ಮಿಚ್ಛೆಯಂತೆ ಅದರಿಂದ ಕೆಲಸ ಮಾಡಿಸಬಹುದು. ನಮ್ಮಿಚ್ಛೆಯಂತೆ ಅದರ ಶಕ್ತಿಯನ್ನು ಕೇಂದ್ರೀಕರಿಸಬಹುದು. 

\vskip 5pt

ರಾಜಯೋಗಿಯ ದೃಷ್ಟಿಯಲ್ಲಿ ಬಾಹ್ಯಜಗತ್ತು ಆಂತರಿಕ ಅಥವಾ ಸೂಕ್ಷ್ಮ ಜಗತ್ತಿನ ಸ್ಥೂಲರೂಪ. ಸೂಕ್ಷ್ಮ ಯಾವಾಗಲೂ ಕಾರಣ, ಸ್ಥೂಲ ಅದರ ಪರಿಣಾಮ. ಆದುದರಿಂದ ಆಂತರಿಕವು ಕಾರಣ, ಬಾಹ್ಯ ಜಗತ್ತು ಕಾರ್ಯ. ಇದೇ ರೀತಿಯಲ್ಲಿ ಸೂಕ್ಷ್ಮವಾಗಿರುವ ಆಂತರಿಕಶಕ್ತಿಯ ಸ್ಥೂಲಭಾವವೇ ಬಾಹ್ಯಶಕ್ತಿ. ಯಾರು ಆಂತರಿಕ ಶಕ್ತಿಯನ್ನು ಕಂಡುಹಿಡಿದಿರುವರೋ, ಅದನ್ನು ಹೇಗೆ ಉಪಯೋಗಿಸಬೇಕೆಂಬುದನ್ನು ತಿಳಿದಿರುವರೋ, ಅವರು ಪ್ರಕೃತಿಯೆಲ್ಲವನ್ನೂ ತಮ್ಮ ಸ್ವಾಧೀನಕ್ಕೆ ಒಳಪಡಿಸಿಕೊಳ್ಳುವರು. ಯೋಗಿಯು ಪ್ರಕೃತಿಯ ಅಧಿಪತಿಯಾಗಿ ತಾನು ಹೇಳಿದಂತೆ ಅದರ ಕೈಯಿಂದ ಮಾಡಿಸುವುದಕ್ಕಿಂತ ಕಡಮೆ ಸಾಹಸಕ್ಕೆ ಹೋಗುವುದಿಲ್ಲ. ನಾವು ಯಾವುದನ್ನು ಪ್ರಕೃತಿನಿಯಮಗಳೆಂದು ಕರೆಯುತ್ತೇವೆಯೋ ಅವಕ್ಕೆ ಬದ್ಧನಾಗದೆ ಇವುಗಳೆಲ್ಲವನ್ನೂ ಮೀರಿರುವ ಒಂದು ಸ್ಥಳವನ್ನು ಅವನು ಸೇರಲು ಇಚ್ಛಿಸುವನು. ಅವನು ಆಂತರಿಕ ಮತ್ತು ಬಾಹ್ಯ ಜಗತ್ತಿನ ಒಡೆಯನಾಗುವನು. ಮಾನವ ಜನಾಂಗದ ಸಂಸ್ಕೃತಿ ಮತ್ತು ಪ್ರಗತಿ ಎಂದರೆ ಪ್ರಕೃತಿನಿಗ್ರಹವೆಂದು ಮಾತ್ರ ಅರ್ಥ. 

\vskip 6pt

ಬೇರೆ ಬೇರೆ ಜನಾಂಗಗಳು ಬೇರೆ ಬೇರೆ ರೀತಿಯಿಂದ ಪ್ರಕೃತಿನಿಗ್ರಹಕ್ಕಾಗಿ ಯತ್ನಿಸುತ್ತವೆ. ಒಂದು ಸಮಾಜದಲ್ಲಿ, ಕೆಲವು ವ್ಯಕ್ತಿಗಳು ಬಾಹ್ಯಪ್ರಕೃತಿಯನ್ನೂ, ಮತ್ತೆ ಕೆಲವರು ಆಂತರಿಕ ಪ್ರಕೃತಿಯನ್ನೂ ನಿಗ್ರಹಿಸಲು ಯತ್ನಿಸುವಂತೆಯೇ; ಜನಾಂಗಗಳಲ್ಲಿ ಕೆಲವು ಆಂತರಿಕಪ್ರಕೃತಿಯ ನಿಗ್ರಹವನ್ನೂ, ಮತ್ತೆ ಕೆಲವು ಬಾಹ್ಯ ಪ್ರಕೃತಿಯ ನಿಗ್ರಹವನ್ನೂ ಬಯಸುತ್ತವೆ. ಕೆಲವರು ಆಂತರಿಕ ಪ್ರಕೃತಿಯ ನಿಗ್ರಹದಿಂದ ಎಲ್ಲವನ್ನೂ ನಿಗ್ರಹಿಸಬಲ್ಲೆವೆಂದು ಹೇಳುವರು. ಮತ್ತೆ ಕೆಲವರು ಬಾಹ್ಯ ಪ್ರಕೃತಿಯ ನಿಗ್ರಹದಿಂದ ಎಲ್ಲವನ್ನೂ ನಿಗ್ರಹಿಸಬಹುದೆಂದು ಹೇಳುವರು. ಇವೆರಡನ್ನೂ ಒಂದು ಅತಿರೇಕಕ್ಕೆ ತೆಗೆದುಕೊಂಡುಹೋದರೆ ಇಬ್ಬರೂ ಸರಿ. ಏಕೆಂದರೆ ಪ್ರಕೃತಿಯಲ್ಲಿ ಒಳಗೆ ಹೊರಗೆ ಎಂಬ ಭೇದವಿಲ್ಲ. ಇದು ಎಂದಿಗೂ ಇಲ್ಲದ ಒಂದು ಕಾಲ್ಪನಿಕ ಭೇದ ಭಾವನೆ. ಅಂತರ್ಮುಖಿಗಳು ಮತ್ತು ಬಹಿರ್ಮುಖಿಗಳು ಇಬ್ಬರೂ ಕೂಡ ತಮ್ಮ ಜ್ಞಾನದ ಪರಿಮಿತಿಯನ್ನು ಮೀರಿದರೆ ಒಂದು ಸ್ಥಳದಲ್ಲಿ ಸಂಧಿಸುವರು. ಭೌತಶಾಸ್ತ್ರಜ್ಞನು ತನ್ನ ಜ್ಞಾನದ ಮೇರೆಯನ್ನು ಮೀರಿದರೆ, ಅದು ಹೇಗೆ ಭೌತವನ್ನು ಮೀರಿಹೋಗುವಂತೆ ಕಂಡುಬರುವುದೊ, ಹಾಗೆಯೇ ತತ್ತ್ವಜ್ಞಾನಿಯೂ ಕೂಡ. ಯಾವುದನ್ನು ನಾವು ಜಡ ಮತ್ತು ಚೇತನ ಎಂದು ಕರೆಯುತ್ತೇವೆಯೊ ಅವುಗಳಿಗಿರುವ ಭಿನ್ನತೆ ಕೇವಲ ತೋರಿಕೆ ಮಾತ್ರ, ಸತ್ಯ ಒಂದೇ – ಎಂದು ಅವನು ಕಂಡುಕೊಳ್ಳುವನು. 

\vskip 6pt

ಯಾವ ಏಕದಿಂದ ಅನೇಕವು ಉಂಟಾಗಿದೆಯೋ, ಯಾವ ಏಕವು ಅನೇಕದಂತೆ ಇದೆಯೊ, ಆ ಏಕವನ್ನು ಕಂಡುಹಿಡಿಯುವುದೇ ಎಲ್ಲಾ ಶಾಸ್ತ್ರಗಳ ಗುರಿ. ಆಂತರಿಕ ಜಗತ್ತಿನಿಂದ ಹೊರಟು, ಅಲ್ಲಿರುವ ಪ್ರಕೃತಿಯನ್ನು ಪರೀಕ್ಷೆಮಾಡಿ ಅದರ ಮೂಲಕ ಬಾಹ್ಯ ಮತ್ತು ಅಂತರಂಗ ಎರಡನ್ನೂ ನಿಗ್ರಹಿಸಬೇಕೆಂದು ರಾಜಯೋಗವು ಹೇಳುತ್ತದೆ. ಇದು ಬಹಳ ಪುರಾತನವಾದ ಪ್ರಯತ್ನ. ಭರತಖಂಡವು ಇದರ ವಿಶೇಷ ಮೂಲಸ್ಥಾನವಾಗಿದೆ. ಆದರೆ ಬೇರೆ ಜನಾಂಗದವರು ಕೂಡ ಇದಕ್ಕಾಗಿ ಪ್ರಯತ್ನಪಟ್ಟಿರುವರು. ಪಾಶ್ಚಾತ್ಯದೇಶಗಳಲ್ಲಿ ಇದನ್ನು ರಹಸ್ಯವಿದ್ಯೆ ಎಂದು ಕರೆಯುತ್ತಿದ್ದರು. ಇದನ್ನು ಅಭ್ಯಾಸ ಮಾಡುವವರನ್ನು ಮಾಟಗಾತಿ, ಮಂತ್ರವಾದಿ ಎಂದು ಕರೆದು ಅವರನ್ನು ಜೀವಸಹಿತ ಸುಡುತ್ತಿದ್ದರು, ಇಲ್ಲವೆ ಕೊಲ್ಲುತ್ತಿದ್ದರು. ಭರತಖಂಡದಲ್ಲಿ ಇದು ಕಾರಣಾಂತರಗಳಿಂದ ಕೆಲವರ ಕೈಗೆ ಬಿತ್ತು. ಅವರು ಶೇಕಡ ತೊಂಬತ್ತರಷ್ಟು ಜ್ಞಾನವನ್ನು ಹಾಳುಮಾಡಿ ಉಳಿದುದನ್ನು ಒಂದು ದೊಡ್ಡ ರಹಸ್ಯವನ್ನಾಗಿ ಮಾಡಲು ಯತ್ನಿಸಿದರು. ಇಂದು ಪಾಶ್ಚಾತ್ಯ ದೇಶಗಳಲ್ಲಿ ಗುರುಗಳೆನ್ನಿಸಿಕೊಳ್ಳುವವರು ಅನೇಕರು ತಲೆಯೆತ್ತಿರುವರು. ಇವರು ಭಾರತೀಯರಿಗಿಂತಲೂ ಕೆಳಮಟ್ಟದವರು. ಏಕೆಂದರೆ, ಭಾರತೀಯರಿಗೆ ಸ್ವಲ್ಪ ತಿಳಿದಿತ್ತು, ಈ ಆಧುನಿಕ ಪಂಡಿತರಿಗೋ ಏನೂ ತಿಳಿಯದು. 

\vskip 6pt

ಈ ಯೋಗಶಾಸ್ತ್ರದಲ್ಲಿ ಯಾವುದು ಅತ್ಯಂತ ರಹಸ್ಯ ಎಂದು ತೋರುತ್ತದೆಯೊ ಅದನ್ನು ತಕ್ಷಣವೆ ತಿರಸ್ಕರಿಸಬೇಕು. ಶಕ್ತಿಯೇ ಜೀವನದ ಅತ್ಯುತ್ತಮ ಮಾರ್ಗದರ್ಶಿ. ಉಳಿದ ಎಲ್ಲಾ ಕಡೆಗಳಂತೆ ಧಾರ್ಮಿಕ ಕ್ಷೇತ್ರದಲ್ಲಿ ಕೂಡ ನಿಮ್ಮನ್ನು ಬಲಹೀನರನ್ನಾಗಿ ಮಾಡುವುದನ್ನೆಲ್ಲಾ ತಿರಸ್ಕರಿಸಿ. ಇವುಗಳೊಂದಿಗೆ ಯಾವ ಸಂಪರ್ಕವನ್ನೂ ಇಟ್ಟುಕೊಳ್ಳ ಬೇಡಿ. ಈ ರಹಸ್ಯ ವ್ಯವಹಾರಗಳು ಮಾನವನ ಬುದ್ಧಿಯನ್ನು ದುರ್ಬಲಗೊಳಿಸುತ್ತವೆ. ಇದು ಅದ್ಭುತವಾದ ಶಾಸ್ತ್ರಗಳಲ್ಲಿ ಒಂದಾದ ಯೋಗವನ್ನೇ ನಾಶಮಾಡಿದೆ. ನಾಲ್ಕು ಸಾವಿರ ವರುಷಗಳ ಹಿಂದೆ, ಇದನ್ನು ಕಂಡುಹಿಡಿದ ದಿನದಿಂದಲೇ ಇದನ್ನು ಪೂರ್ಣವಾಗಿ ವಿವರಿಸಿ, ಸಿದ್ಧಾಂತ ಮಾಡಿ ಜನರಿಗೆ ಬೋಧಿಸಿರುವರು. ಭಾಷ್ಯಕಾರರು ಆಧುನಿಕರಾದಷ್ಟೂ ಅವರಲ್ಲಿರುವ ತಪ್ಪು ಅಭಿಪ್ರಾಯ ಹೆಚ್ಚು. ಆದರೆ ಹೆಚ್ಚು ಹೆಚ್ಚು ಪುರಾತನಕಾಲದ ಶಾಸ್ತ್ರಜ್ಞನಾದರೆ ಆತನ ವಿಚಾರ ಹೆಚ್ಚು ಯುಕ್ತಿಪೂರಿತವಾಗಿರುವುದು ಒಂದು ಆಶ್ಚರ್ಯವೇ ಸರಿ. ಹೆಚ್ಚುಪಾಲು ಆಧುನಿಕ ಬರಹಗಾರರು ಏನೇನೊ ರಹಸ್ಯದ ವಿಚಾರವಾಗಿ ಮಾತನಾಡುವರು. ವಿಚಾರದ ಪೂರ್ಣಪ್ರಕಾಶ ಅದರ ಮೇಲೆ ಬೀಳುವಂತೆ ಮಾಡುವ ಬದಲು ಅದನ್ನು ಒಂದು ರಹಸ್ಯವನ್ನಾಗಿ ಮಾಡಿದ ಕೆಲವರ ಕರಗತವಾಯಿತು ಈ ಯೋಗ. ತಮ್ಮಲ್ಲಿಯೇ ಆ ಅಧಿಕಾರದ ಶಕ್ತಿ ಉಳಿಯಲಿ ಎಂಬ ಆಸೆಯಿಂದ ಪ್ರೇರಿತರಾಗಿ ಅವರು ಹೀಗೆ ಮಾಡಿದರು. 

\vskip 6pt

ಮೊದಲನೆಯದಾಗಿ, ನಾನು ನಿಮಗೆ ಹೇಳುವುದರಲ್ಲಿ ಯಾವುದೊಂದು ರಹಸ್ಯವೂ ಇಲ್ಲ. ನನಗೆ ತಿಳಿದಿರುವ ಅಲ್ಪವನ್ನು ನಾನು ನಿಮಗೆ ಹೇಳುತ್ತೇನೆ. ಎಲ್ಲಿಯವರೆಗೆ ಸಾಧ್ಯವೋ ಅಲ್ಲಿಯವರೆಗೆ ವಿಚಾರ ಮಾಡಿ, ನನಗೆ ಎಲ್ಲಿ ಅದು ಅಸಾಧ್ಯವೆಂದು ತೋರುತ್ತದೆಯೊ ಅಲ್ಲಿ ಶಾಸ್ತ್ರವು ಹೇಳುವುದನ್ನು ಮಾತ್ರ ಹೇಳುತ್ತೇನೆ. ಅಂಧಶ್ರದ್ಧೆ ಸರಿಯಲ್ಲ. ನಿಮ್ಮ ಸ್ವಂತ ವಿಚಾರವನ್ನು, ನಿರ್ಣಯವನ್ನು ನೀವು ಉಪಯೋಗಿಸಬೇಕು. ಈ ವಿಷಯಗಳು ಸಾಧ್ಯವೆ ಇಲ್ಲವೆ ಎಂಬುದನ್ನು ನೀವು ಸಾಧನೆ ಮಾಡಿ ನೋಡಬೇಕು. ನೀವು, ಬೇರೆ ವಿಜ್ಞಾನ ಶಾಸ್ತ್ರಗಳನ್ನು ಹೇಗೆ ಅಧ್ಯಯನ ಮಾಡುತ್ತೀರೊ ಹಾಗೆಯೇ ಇದನ್ನು ಕೂಡ ವ್ಯಾಸಂಗ ಮಾಡಬೇಕು. ಇದರಲ್ಲಿ ರಹಸ್ಯವೂ ಇಲ್ಲ, ಅಪಾಯವೂ ಇಲ್ಲ. ಸತ್ಯವಾದುದನ್ನು ಯಾವ ಅಂಜಿಕೆಯೂ ಇಲ್ಲದೆ ಎಲ್ಲರಿಗೂ ಘಂಟಾಘೋಷವಾಗಿ ಹೇಳಬೇಕು. ಇದನ್ನುರಹಸ್ಯವನ್ನಾಗಿ ಮಾಡುವ ಪ್ರಯತ್ನವೆಲ್ಲ ಬಹಳ ಅಪಾಯಕರವಾಗಿ ಪರಿಣಮಿಸುವುದು. 

ಇನ್ನು ನಾನು ಮುಂದುವರಿಯುವುದಕ್ಕೆ ಮುಂಚೆ ಯೋಗಕ್ಕೆ ಆಧಾರವಾಗಿರುವ ಸಾಂಖ್ಯದರ್ಶನವನ್ನು ನಿಮಗೆ ಹೇಳುತ್ತೇನೆ. ಸಾಂಖ್ಯದರ್ಶನದಲ್ಲಿ ಗ್ರಹಣಕ್ರಿಯೆಯ\break ವಿಧಾನವು ಹೀಗಿರುತ್ತದೆ. ಬಾಹ್ಯವಸ್ತುವಿನಿಂದ ಉಂಟಾದ ಸಂವೇದನೆಯನ್ನು ಆಯಾ ಬಾಹ್ಯೇಂದ್ರಿಯಗಳು ತಮಗೆ ಸಂಬಂಧಪಟ್ಟ ಮೆದುಳಿನ ಕೇಂದ್ರಕ್ಕೆ ಒಯ್ಯುತ್ತವೆ. ಈ ಕೇಂದ್ರಗಳು ಅದನ್ನು ಮನಸ್ಸಿಗೆ ಒಯ್ಯುತ್ತವೆ; ಮನಸ್ಸು ಬುದ್ಧಿಗೆ ಒಯ್ಯುತ್ತದೆ. ಬುದ್ಧಿಯಿಂದ ಪುರುಷನು ಅದನ್ನು ಸ್ವೀಕರಿಸುತ್ತಾನೆ. ಆಗ ನಮಗೆ ಬಾಹ್ಯವಸ್ತುವಿನ ಪರಿಚಯವಾಗುವುದು. ಅನಂತರ ಕರ್ಮೇಂದ್ರಿಯಗಳಿಗೆ ಆವಶ್ಯಕವಾದ ಕೆಲಸವನ್ನು ಮಾಡುವಂತೆ ಅಪ್ಪಣೆ ಕೊಡಲಾಗುವುದು. ಪುರುಷನ ವಿನಾ ಉಳಿದ ಈ ಎಲ್ಲವೂ ಜಡವಾದುವು. ಆದರೆ ಮನಸ್ಸು ಬಾಹ್ಯೇಂದ್ರಿಯಗಳಿಗಿಂತಲೂ ಸೂಕ್ಷ್ಮವಾದುದು. ಮನಸ್ಸು ಯಾವ ವಸ್ತುವಿನಿಂದ ನಿರ್ಮಿತವಾಗಿದೆಯೋ ಅದೇ ವಸ್ತುವಿನಿಂದಲೇ ಸೂಕ್ಷ್ಮ ವಸ್ತುಗಳಾದ ತನ್ಮಾತ್ರೆಗಳೂ ಉಂಟಾಗಿವೆ. ಇವೇ ಸ್ಥೂಲವಾಗಿ ಬಾಹ್ಯ ವಸ್ತುಗಳಾಗುವುವು. ಇದೇ ಸಾಂಖ್ಯರ ಮನಶ್ಯಾಸ್ತ್ರ. ಬುದ್ದಿಗೂ ಮತ್ತು ಹೊರಗಿನ ಜಡವಸ್ತುವಿಗೂ ಇರುವ ವ್ಯತ್ಯಾಸ ತರತಮದಲ್ಲಿ ಮಾತ್ರ. ಪುರುಷನೊಬ್ಬನು ಮಾತ್ರವೇ ಚೇತನನಾಗಿರುವನು. ಆತ್ಮನ ಕೈಯಲ್ಲಿ ಮನಸ್ಸೆಂಬುದು ಒಂದು ಕರಣದಂತಿದೆ. ಇದರ ಮೂಲಕ ಆತ್ಮನು ಬಾಹ್ಯವಸ್ತುವನ್ನು ಗ್ರಹಿಸುತ್ತಾನೆ. ಮನಸ್ಸು ಸದಾ ಚಂಚಲವಾಗಿರುವುದು. ಅದನ್ನು ಸರಿಯಾಗಿ ನಿಗ್ರಹಿಸಿದ ಮೇಲೆ, ಅದನ್ನು ಅನೇಕ ಇಂದ್ರಿಯಗಳಿಗೆ ಸಂಬಂಧಿಸಬಹುದು, ಒಂದೇ ಇಂದ್ರಿಯಕ್ಕೆ ಸಂಬಂಧಿಸಬಹುದು ಅಥವಾ ಯಾವ ಇಂದ್ರಿಯಕ್ಕೂ ಸಂಬಂಧಿಸದೇ ಇರಬಹುದು. ಉದಾಹರಣೆಗೆ: ಗಡಿಯಾರ ಹೊಡೆಯುವುದನ್ನು ಬಹಳ ಎಚ್ಚರಿಕೆಯಿಂದ ಕೇಳುತ್ತಿದ್ದರೆ ನನ್ನ ಕಣ್ಣು ತೆರೆದಿದ್ದರೂ ಕೂಡ ನಾನು ಬಹುಶಃ ಯಾವುದನ್ನೂ ನೋಡಲಾರೆ. ಮನಸ್ಸು ಕೇಳುವ ಇಂದ್ರಿಯದ ಕಡೆಗೆ ಇದ್ದುದರಿಂದ ನೋಡುವ ಕಡೆಗೆ ಇರಲಿಲ್ಲವೆಂಬುದನ್ನು ಇದು ತೋರುವುದು. ಆದರೆ ನಿಗ್ರಹಿಸಲ್ಪಟ್ಟ ಮನಸ್ಸು ಏಕಕಾಲದಲ್ಲಿ ಎಲ್ಲಾ ಇಂದ್ರಿಯಗಳ ಮೂಲಕವೂ ಗ್ರಹಿಸಬಲ್ಲದು. ತನ್ನ ಅಂತರಂಗವನ್ನು ತಾನೇ ಪರೀಕ್ಷಿಸುವ ಒಂದು ಶಕ್ತಿ ಅದಕ್ಕೆ ಇದೆ. ಈ ಶಕ್ತಿಯನ್ನು ಪಡೆಯಬೇಕೆಂದೇ ಯೋಗಿಯು ಆಶಿಸುವುದು. ಅವನು ಮನಸ್ಸಿನ ಶಕ್ತಿಯನ್ನು ಏಕಾಗ್ರಮಾಡಿ, ಅಂತರ್ಮುಖಮಾಡಿ ಅಲ್ಲಿ ಏನಾಗುತ್ತಿದೆ ಎಂಬುದನ್ನು ತಿಳಿಯಲು ಪ್ರಯತ್ನಿಸುವನು. ಇಲ್ಲಿ ಬರಿಯ ನಂಬಿಕೆ ಎಂಬುದಿಲ್ಲ; ಇದು ಕೆಲವು ತತ್ತ್ವಶಾಸ್ತ್ರಜ್ಞದ ನಿರ್ಣಾಯ. ಆಧುನಿಕ ಶರೀರವಿಜ್ಞಾನಿಗಳು ಸಾರುವುದು ಹೀಗೆ: ದೃಶ್ಯೇಂದ್ರಿಯವು ಕಣ್ಣಲ್ಲ. ಈ ದೃಶ್ಯೇಂದ್ರಿಯವು ಮೆದುಳಿನಲ್ಲಿರುವ ಒಂದು ನರಗಳ ಕೇಂದ್ರ. ಇದರಂತೆ ಪ್ರತಿಯೊಂದು ಇಂದ್ರಿಯವೂ ಕೂಡ. ಮೆದುಳು ಯಾವ ವಸ್ತುವಿನಿಂದ ಆಗಿದೆಯೊ ಅದರಿಂದಲೇ ಈ ಕೇಂದ್ರಗಳೂ ಆಗಿವೆ ಎನ್ನುತ್ತಾರೆ ಅವರು. ಸಾಂಖ್ಯಯೋಗವು ಹೇಳುವುದು ಕೂಡ ಇದನ್ನೇ. ಮೊದಲನೆಯದು ಶಾರೀರಿಕ ದೃಷ್ಟಿಯಿಂದ ಹೇಳಿದ್ದು, ಎರಡನೆಯದು ಮನಶ್ಯಾಸ್ತ್ರದ ದೃಷ್ಟಿಯಿಂದ. ಆದರೂ ಎರಡೂ ಒಂದೇ. ನಾವು ಸಂಶೋಧನೆ ಮಾಡುವ ಕ್ಷೇತ್ರವಿರುವುದು ಇದರಾಚೆ. 

ಎಲ್ಲ ವಿಧದ ಮಾನಸಿಕ ಸ್ಥಿತಿಗತಿಗಳನ್ನು ಪರೀಕ್ಷಿಸಬಲ್ಲ ಸೂಕ್ಷ್ಮ ಗ್ರಹಣಶಕ್ತಿಯನ್ನು ಪಡೆಯಬೇಕೆಂಬುದೇ ಯೋಗಿಯ ಉದ್ದೇಶ. ಎಲ್ಲಾ ಸ್ಥಿತಿಯ ಮಾನಸಿಕ ಗ್ರಹಣವಿರಬೇಕು. ಸಂವೇದನೆಯು ಹೇಗೆ ಚಲಿಸುತ್ತದೆ, ಮನಸ್ಸು ಅದನ್ನು ಹೇಗೆ ಸ್ವೀಕರಿಸುತ್ತದೆ, ಬುದ್ಧಿಯು ಅದನ್ನು ಪುರುಷನಿಗೆ ಹೇಗೆ ಕೊಡುತ್ತದೆ–ಎಂಬುದನ್ನೆಲ್ಲಾ ಒಬ್ಬನು ಗ್ರಹಿಸಬಹುದು. ಹೇಗೆ ಪ್ರತಿಯೊಂದು ವಿಜ್ಞಾನಕ್ಕೂ ಅದಕ್ಕೆ ಸಂಬಂಧಪಟ್ಟ ಸಿದ್ಧತೆಗಳು ಬೇಕೋ, ಹೇಗೆ ಅದನ್ನು ತಿಳಿದುಕೊಳ್ಳುವುದಕ್ಕೆ ಮುಂಚೆ ನಾವು ಅನುಸರಿಸಬೇಕಾದ ಒಂದು ಕ್ರಮವೂ ಇದೆಯೋ, ಅದರಂತೆ ರಾಜಯೋಗದ ವಿಷಯದಲ್ಲಿ ಕೂಡ. 

ನಮ್ಮ ಆಹಾರದಲ್ಲಿ ಕೆಲವು ನಿಯಮಗಳ ಆವಶ್ಯಕತೆ ಇದೆ. ನಮಗೆ ಅತ್ಯಂತ ಪರಿಶುದ್ಧವಾದ ಮನಸ್ಸನ್ನು ಕೊಡುವ ಆಹಾರವನ್ನೇ ಉಪಯೋಗಿಸಬೇಕು. ನೀವು ಪ್ರಾಣಿಗಳ ಪ್ರದರ್ಶನಾಲಯಕ್ಕೆ ಹೋಗಿ ನೋಡಿದರೆ ಒಡನೆಯೇ ಇದು ನಿಮಗೆ ಗೊತ್ತಾಗುವುದು. ಬಹಳ ದೊಡ್ಡ ಮೃಗವಾದ ಆನೆ ಶಾಂತವಾಗಿ ಸೌಮ್ಯವಾಗಿರುವುದು. ಸಿಂಹಗಳ ಮತ್ತು ಹುಲಿಗಳ ಬೋನಿನ ಕಡೆಗೆ ನೋಡಿದರೆ ಅವುಗಳ ಚಂಚಲತೆ ಕಾಣುವುದು. ಆಹಾರದಿಂದ ಎಷ್ಟು ವ್ಯತ್ಯಾಸವಾಗಿದೆ ಎಂಬುದನ್ನು ಇದು ತೋರುವುದು. ಈ ದೇಹದಲ್ಲಿ ಕೆಲಸ ಮಾಡುವ ಎಲ್ಲಾ ವಿಧದ ಶಕ್ತಿಗಳೂ ಆಹಾರದಿಂದ ಉತ್ಪನ್ನವಾಗಿವೆ. ಇದು ನಮಗೆ ಪ್ರತಿದಿನವೂ ಕಾಣುವುದು. ನೀವು ಉಪವಾಸ ಮಾಡಲು ಮೊದಲು ಮಾಡಿದರೆ, ಮೊದಲು ನಿಮ್ಮ ಶರೀರ ನಿರ್ಬಲವಾಗುವುದು. ಶಾರೀರಿಕ ಶಕ್ತಿ ಕ್ಷೀಣಿಸುವುದು. ಕೆಲವು ದಿನಗಳ ಅನಂತರ ಮಾನಸಿಕ ಶಕ್ತಿಯೂ ಕ್ಷೀಣಿಸುವುದು. ಮೊದಲು ಸ್ಮೃತಿ ಶಕ್ತಿ ನಿಲ್ಲುವುದು. ಅನಂತರ, ನಿರ್ದಿಷ್ಟವಾಗಿ ವಿಚಾರಮಾಡುವುದಿರಲಿ, ಯಾವ ಆಲೋಚನೆಯನ್ನೂ ಮಾಡಲಾಗದ ಸ್ಥಿತಿ ಬರುವುದು. ಆದಕಾರಣ ಮೊದಲು ಯಾವ ವಿಧದ ಆಹಾರವನ್ನು ಸೇವಿಸುತ್ತೇವೆ ಎಂಬುದನ್ನು ಗಮನದಲ್ಲಿಡಬೇಕು. ನಮಗೆ ಸಾಕಾದಷ್ಟು ಶಕ್ತಿ ಬಂದ ಮೇಲೆ, ಸಾಧನೆಯಲ್ಲಿ ಬಹಳ ಮುಂದುವರಿದ ಮೇಲೆ, ಆಹಾರದ ವಿಚಾರದಲ್ಲಿ ನಾವು ಅಷ್ಟು ಚೋಪಾನವಾಗಿರಬೇಕಾಗಿಲ್ಲ. ಸಸಿಯು ಬೆಳೆಯುತ್ತಿರುವಾಗ ಅದಕ್ಕೆ ಬೇಲಿಯನ್ನು ಹಾಕಬೇಕು. ಇಲ್ಲದೇ ಇದ್ದರೆ ಅದಕ್ಕೆ ಅಪಾಯ ತಟ್ಟಬಹುದು. ಆದರೆ ಅದು ಒಂದು ಮರವಾದ ಮೇಲೆ ಬೇಲಿಯನ್ನೆಲ್ಲಾ ತೆಗೆದುಬಿಡುವರು. ಆಗ ಅದಕ್ಕೆ ಬರುವ ಅಪಾಯಗಳನ್ನೆಲ್ಲಾ ಎದುರಿಸಲು ಸಾಕಾದಷ್ಟು ಶಕ್ತಿ ಇರುತ್ತದೆ. 

ಯೋಗಿಯು ಭೋಗ ಮತ್ತು ಕಠೋರ ತಪಸ್ಸು–ಈ ಅತಿರೇಕಗಳಿಂದ ಪಾರಾಗಬೇಕು. ಅವನು ಉಪವಾಸ ಮಾಡಕೂಡದು ಅಥವಾ ದೇಹವನ್ನು ದಂಡಿಸಲೂ ಕೂಡದು. ಯಾರು ಹೀಗೆ ಮಾಡುವರೋ ಅವರು ಯೋಗಿಗಳಾಗಲಾರರೆಂದು ಗೀತೆ ಬೋಧಿಸುವುದು. ಯಾರು ಉಪವಾಸ ಮಾಡುತ್ತಾರೋ, ಯಾರು ನಿದ್ರೆ ಮಾಡುವುದಿಲ್ಲವೊ, ಯಾರು ಅತಿ ಕೆಲಸ ಮಾಡುವರೋ, ಅಥವಾ ಯಾರು ಕೆಲಸವನ್ನೇ ಮಾಡುವುದಿಲ್ಲವೋ–ಇವರಲ್ಲಿ ಯಾರೂ ಯೋಗಿಗಳಾಗಲಾರರು.

\chapter{ಮೊದಲ ಮೆಟ್ಟಲುಗಳು}

ರಾಜಯೋಗವನ್ನು ಎಂಟು ಮೆಟ್ಟಲುಗಳಾಗಿ ವಿಭಾಗಿಸಿರುವರು. ಮೊದಲನೆಯದು ಯಮ: ಅಹಿಂಸೆ, ಸತ್ಯ, ಅಸ್ತೇಯ, ಬ್ರಹ್ಮಚರ್ಯ ಮತ್ತು ಅಪರಿಗ್ರಹ. ಎರಡನೆಯದು ನಿಯಮ: ಶುಚಿ, ತೃಪ್ತಿ, ತಪಸ್ಸು, ಅಧ್ಯಯನ ಮತ್ತು ಈಶ್ವರನಲ್ಲಿ ಶರಣಾಗತಿ. ಅನಂತರ ಆಸನ, ಅನಂತರ ಪ್ರಾಣಾಯಾಮ, ಪ್ರಾಣವನ್ನು ನಿಗ್ರಹಿಸುವುದು. ಮುಂದಿನದು ಪ್ರತ್ಯಾಹಾರ ಅಥವಾ ಇಂದ್ರಿಯಗಳನ್ನು ಅವಕ್ಕೆ ಸಂಬಂಧಪಟ್ಟ ವಸ್ತುಗಳಿಂದ ಹಿಂತೆಗೆದುಕೊಳ್ಳುವುದು. ಅನಂತರ ಧಾರಣ ಅಥವಾ ಮನಸ್ಸನ್ನು ಒಂದು ವಸ್ತುವಿನ ಮೇಲೆ ಏಕಾಗ್ರ ಮಾಡುವುದು. ಇದರನಂತರ ಬರುವುವು ಧ್ಯಾನ ಮತ್ತು ಸಮಾಧಿ. ನಮಗೆ ಕಾಣುವಂತೆ ಯಮ ಮತ್ತು ನಿಯಮಗಳು ನೈತಿಕ ಶಿಕ್ಷಣ. ಇವುಗಳ ತಳಹದಿ ಇಲ್ಲದೆ ಯಾವ ಯೋಗಾಭ್ಯಾಸವೂ ಜಯಪ್ರದವಾಗುವುದಿಲ್ಲ. ಯಮ ನಿಯಮಗಳಲ್ಲಿ ಚೆನ್ನಾಗಿ ಪಳಗಿದ ಮೇಲೆ ಯೋಗಿಯು ತನ್ನ ಸಾಧನೆಯ ಫಲವನ್ನು ಪಡೆಯುತ್ತಾನೆ. ಇವುಗಳಿಲ್ಲದೆ ಯೋಗವೆಂದಿಗೂ ಫಲಕಾರಿಯಾಗದು. ಯೋಗಿಯು ಮಾತು, ಕೃತಿ–ಆಲೋಚನೆಯಿಂದಲೂ ಕೂಡ–ಯಾರನ್ನೂ ಹಿಂಸಿಸಕೂಡದು. ದಯೆ ತೋರುವುದು ಮಾನವನಿಗೆ ಮಾತ್ರವಲ್ಲ; ಅದು ಈ ಹಂತವನ್ನೂ ಮೀರಿ ಹೋಗಿ ಇಡೀ ವಿಶ್ವವನ್ನೇ ಆಲಿಂಗನ ಮಾಡಬೇಕಾಗುವುದು. 

ಅನಂತರದ ಮೆಟ್ಟಲೇ ಆಸನ. ಉತ್ತಮ ಸ್ಥಿತಿಗೆ ಏರುವವರೆವಿಗೂ ಕೆಲವು ವಿಧದ ಶಾರೀರಿಕ ಮತ್ತು ಮಾನಸಿಕ ಸಾಧನೆಯನ್ನು ಪ್ರತಿದಿನ ಮಾಡಬೇಕು. ಆದ ಕಾರಣ ನಾವು ದೀರ್ಘಕಾಲ ಕುಳಿತುಕೊಳ್ಳಲು ಸಾಧ್ಯವಾಗುವ ಒಂದು ಆಸನವನ್ನು ಕಂಡುಹಿಡಿಯುವುದು ಅತ್ಯಾವಶ್ಯಕವಾಗಿದೆ. ಯಾರಿಗೆ ಯಾವ ಆಸನ ಸುಲಭವೋ ಅವರು ಅದನ್ನು ಆರಿಸಿಕೊಳ್ಳಬೇಕು. ಕುಳಿತುಕೊಂಡು ವಿಚಾರಮಾಡುವುದಕ್ಕೆ ಒಬ್ಬನಿಗೆ ಒಂದು ವಿಧವಾದ ಆಸನ ಬಹಳ ಸುಲಭವಾಗಿ ಕಾಣಬಹುದು. ಇದು ಬೇರೊಬ್ಬನಿಗೆ ಬಹಳ ಕಷ್ಟವಾಗಿ ಕಾಣುವುದು. ಮಾನಸಿಕ ಸ್ಥಿತಿಯನ್ನು ನಾವು ಅಧ್ಯಯನ ಮಾಡುತ್ತಿರುವಾಗ, ದೇಹದಲ್ಲಿ ಎಷ್ಟೋ ಕೆಲಸಗಳು ನಡೆಯುತ್ತ ಇರುತ್ತವೆ ಎಂಬುದು ನಮಗೆ ಅನಂತರ ಗೊತ್ತಾಗುವುದು. ಹರಿಯುತ್ತಿರುವ ನರಗಳ ಶಕ್ತಿಪ್ರವಾಹವನ್ನು ಹಿಂದಿನ ದಾರಿಯಿಂದ ತಪ್ಪಿಸಿ ಅವುಗಳಿಗೆ ಬೇರೆ ಮಾರ್ಗವನ್ನು ಕಲ್ಪಿಸಬೇಕು. ಆಗ ಹೊಸ ರೀತಿಯ ಸ್ಪಂದನ ಪ್ರಾರಂಭವಾಗುವುದು. ನಮ್ಮ ಇಡೀ ದೇಹರಚನೆಯೇ ಬದಲಾದಂತಾಗುತ್ತದೆ. ಆದರೆ ಕಾರ್ಯದ ಬಹುಮುಖ್ಯವಾದ ಭಾಗ ನಡೆಯುವುದು ಬೆನ್ನೆಲುಬಿನ ಮೂಲಕ. ಆದಕಾರಣ ನಮ್ಮ ಆಸನದ ಬಹಳ ಮುಖ್ಯವಾದ ಒಂದು ಅಂಶವೇ, ನೇರವಾಗಿ ಎದೆ, ಕತ್ತು ಮತ್ತು ತಲೆ – ಈ ಮೂರನ್ನು ಒಂದೇ ಸರಳರೇಖೆಯಲ್ಲಿ ಬರುವಂತೆ ಕುಳಿತುಕೊಳ್ಳಬೇಕು ಎಂಬುದು. ದೇಹದ ತೂಕವೆಲ್ಲವೂ ಪಕ್ಕೆಲುಬುಗಳ ಮೇಲೆ ನಿಲ್ಲಲಿ. ಅನಂತರ ಬೆನ್ನುಮೂಳೆ ನೇರವಾಗಿ, ಸುಲಭವಾದ ಮತ್ತು ಸಹಜವಾದ ಆಸನ ಸಿದ್ಧಿಸುವುದು. ಎದೆ ಬಗ್ಗಿರುವಾಗ ನೀವು ಯಾವ ಉದಾತ್ತ ವಿಷಯಗಳನ್ನೂ ಆಲೋಚನೆ ಮಾಡಲಾರಿರಿ ಎಂಬುದನ್ನು ಸುಲಭವಾಗಿ ನೋಡುವಿರಿ. ರಾಜಯೋಗದ ಈ ಭಾಗವು ಹಠಯೋಗವನ್ನು ಹೋಲುತ್ತದೆ. ಹಠಯೋಗವು ದೇಹಕ್ಕೆ ಸಂಬಂಧಪಟ್ಟ ವಿಷಯವನ್ನು ಮಾತ್ರ ಕುರಿತದ್ದು. ದೇಹವನ್ನು ದೃಢವಾಗಿ ಮಾಡುವುದೇ ಅದರ ಗುರಿ. ಇಲ್ಲಿ ಅದಕ್ಕೂ ನಮಗೂ ಏನೂ ಸಂಬಂಧವಿಲ್ಲ. ಏಕೆಂದರೆ ಅದರ ಸಾಧನೆ ಬಹಳ ಕಷ್ಟ. ಅದನ್ನು ಒಂದು ದಿನದಲ್ಲಿ ಕಲಿಯಲಾಗುವುದಿಲ್ಲ. ಅದೂ ಅಲ್ಲದೆ ಅದರಿಂದ ಆಧ್ಯಾತ್ಮಿಕ ಉನ್ನತಿಗೆ ಅಷ್ಟೇನೂ ಸಹಾಯವಾಗುವುದಿಲ್ಲ. ಇಂತಹ ಅನೇಕ ಅಭ್ಯಾಸಗಳನ್ನು, ದೇಹವನ್ನು ವಿವಿಧ ಭಂಗಿಯಲ್ಲಿ ಇರಿಸುವುದನ್ನು, ನೀವು ಡೆಲ್​ಸಾರ್ಟಿ ಮತ್ತು ಇನ್ನೂ ಇತರ ಶಿಕ್ಷಕರಲ್ಲಿ ನೋಡಬಹುದು. ಇದರ ಗುರಿ ಶಾರೀರಿಕವೇ ಹೊರತು ಮಾನಸಿಕವಲ್ಲ. ಮನುಷ್ಯನು ಸಂಪೂರ್ಣವಾಗಿ ನಿಗ್ರಹಿಸಲಾರದ ಯಾವ ಒಂದು ಮಾಂಸಖಂಡವೂ ಕೂಡ ಆತನ ದೇಹದಲ್ಲಿ ಇಲ್ಲ. ತನ್ನ ಇಚ್ಛೆಯಂತೆ ಹೃದಯವನ್ನು ಬೇಕಾದರೆ ನಿಲ್ಲಿಸಬಹುದು; ಅಥವಾ ಪುನಃ ಅದನ್ನು ಚಲಿಸುವಂತೆ ಮಾಡಬಹುದು. ಇದರಂತೆಯೇ ಅವನು ಪ್ರತಿಯೊಂದು ಅಂಗವನ್ನು ಕೂಡ ತನ್ನ ಸ್ವಾಧೀನಕ್ಕೆ ತರಬಹುದು. 

ಹಠಯೋಗದ ಒಂದು ಗುರಿ ಮಾನವರನ್ನು ದೀರ್ಘಾಯುಷಿಯನ್ನಾಗಿ ಮಾಡುವುದು. ಆರೋಗ್ಯವೇ ಅದಕ್ಕೆ ಮುಖ್ಯ. ಅದೇ ಅದರ ಗುರಿ, ತಾನು ಎಂದಿಗೂ ಕಾಯಿಲೆ ಬೀಳುವುದಿಲ್ಲವೆಂದು ಹಠಯೋಗಿ ಶಪಥ ಮಾಡುವನು. ಅದೂ ಅಲ್ಲದೆ ಅವನೆಂದಿಗೂ ಕಾಯಿಲೆ ಬೀಳುವುದೂ ಇಲ್ಲ. ಅವನು ಬಹಳ ಕಾಲ ಬದುಕುವನು. ನೂರು ವರ್ಷಗಳು ಬದುಕುವುದೇನೂ ಅಷ್ಟು ಹೆಚ್ಚಿನದಲ್ಲ. ಅವನಿಗೆ ನೂರೈವತ್ತು ವರುಷವಾದರೂ, ತಲೆಯ ಒಂದು ಕೂದಲೂ ಕೂಡ ನೆರೆಯದೆ ಉತ್ಸಾಹದಿಂದ ಕೂಡಿರುತ್ತಾನೆ. ಆದರೆ ಇದರ ಪ್ರಯೋಜನ ಇಷ್ಟೆ. ಆಲದ ಮರವು ಕೆಲವು ವೇಳೆ ಐದು ಸಾವಿರ ವರ್ಷ ಬದುಕುತ್ತದೆ. ಆದರೆ ಅದೊಂದು ಆಲದ ಮರ ಮಾತ್ರ. ಅದಕ್ಕಿಂತ ಹೆಚ್ಚಿನದೇನೂ ಅಲ್ಲ. ಆದಕಾರಣ ಮನುಷ್ಯನು ಬಹುಕಾಲ ಬದುಕಿದರೆ ಅವನೊಂದು ದೃಢಕಾಯನಾದ ಮೃಗ ಮಾತ್ರ. ಹಠಯೋಗಿಯು ಸಾಧಾರಣವಾದ ಒಂದೆರಡು ಅಭ್ಯಾಸಗಳು ಬಹಳ ಉಪಯೋಗಕಾರಿಗಳಾಗಿವೆ. ನಿಮ್ಮಲ್ಲಿ ಕೆಲವರಿಗೆ ತಲೆನೋವಿದ್ದರೆ, ಬೆಳಗ್ಗೆ ಎದ್ದ ಕೂಡಲೆ ತಣ್ಣೀರನ್ನು ಮೂಗಿನ ಮೂಲಕ ಸೆಳೆದು ಕೊಳ್ಳುವುದರಿಂದ ಅನುಕೂಲವಿದೆ. ಇಡೀ ದಿನ ನಿಮ್ಮ ಮೆದುಳು ತಂಪಾಗಿ ಶಾಂತವಾಗಿರುವುದು ಮತ್ತು ನಿಮಗೆ ಎಂದಿಗೂ ನೆಗಡಿ ಬರುವುದಿಲ್ಲ. ಇದನ್ನು ಮಾಡುವುದು ಬಹಳ ಸುಲಭ. ನಿಮ್ಮ ಮೂಗನ್ನು ನೀರಿನಲ್ಲಿಟ್ಟು ಅದನ್ನು ಒಳಗೆ ಸೆಳೆದುಕೊಳ್ಳಿ. ನಿಮ್ಮ ಗಂಟಲನ್ನು ನೀರನ್ನು ಸೆಳೆಯುವ ಯಂತ್ರವನ್ನಾಗಿ ಮಾಡಿ. 

\vskip 0.3cm

ಸ್ಥಿರವಾದ ಆಸನ ಸಿದ್ಧಿಸಿದ ಮೇಲೆ ಯೋಗಿಗಳು ಹೇಳುವ “ನಾಡಿಗಳ ಶುದ್ಧಿ” ಎಂಬ ಅಭ್ಯಾಸವನ್ನು ಮಾಡಬೇಕು. ಕೆಲವರು ಇದನ್ನು ರಾಜಯೋಗಕ್ಕೆ ಸೇರಿಲ್ಲವೆಂದು ತಿರಸ್ಕರಿಸುವರು. ಆದರೆ ಶಂಕರಾಚಾರ್ಯರಂತಹ ಶ್ರೇಷ್ಠ ಭಾಷ್ಯಕಾರರು ಇದನ್ನು ಹೇಳಿರುವುದರಿಂದ ನಾನು ಇದನ್ನು ಹೇಳುವುದು ಸೂಕ್ತವೆಂದು ತೋರುವುದು. ಶ್ವೇತಾಶ್ವತರ ಉಪನಿಷತ್ತಿನ ಭಾಷ್ಯದಿಂದ ಅವರ ಸ್ವಂತ ಸಲಹೆಯನ್ನು ನಿಮಗೆ ಹೇಳುತ್ತೇನೆ: “ಪ್ರಾಣಾಯಾಮದಿಂದ ಪರಿಶುದ್ಧವಾದ ಮನಸ್ಸು ಬ್ರಹ್ಮದಲ್ಲಿ ಏಕಾಗ್ರತೆಯನ್ನು ಪಡೆಯುವುದು. ಆದಕಾರಣವೇ ಪ್ರಾಣಾಯಾಮವನ್ನು ಹೇಳಿರುವುದು. ಮೊದಲು ನಾಡಿಗಳನ್ನು ಶುದ್ಧ ಮಾಡಬೇಕು. ಅನಂತರ ಪ್ರಾಣಾಯಾಮವನ್ನು ಅಭ್ಯಾಸಮಾಡುವ ಶಕ್ತಿ ಬರುವುದು. ಹೆಬ್ಬೆಟ್ಟಿನಿಂದ ಬಲಗಡೆಯ ಮೂಗನ್ನು ಮುಚ್ಚಿ, ಸಾಧ್ಯವಾದಷ್ಟು ಎಡಗಡೆಯ ಮೂಗಿನಿಂದ ಉಸಿರನ್ನು ಸೆಳೆದುಕೊಳ್ಳಿ. ಅನಂತರ ತತ್​ಕ್ಷಣವೇ ಎಡಗಡೆ ಮೂಗನ್ನು ಮುಚ್ಚಿಕೊಂಡು ಬಲಗಡೆಯ ಮೂಗಿನಿಂದ ಉಸಿರನ್ನು ಬಿಡಿ. ಪುನಃ ಸಾಧ್ಯವಾದಷ್ಟು ಬಲಗಡೆ ಮೂಗಿನಿಂದ ಉಸಿರನ್ನು ಎಳೆದು ಕೊಂಡು ಎಡಗಡೆಯ ಮೂಗಿನಿಂದ ಉಸಿರನ್ನು ಬಿಡಿ. ಇದನ್ನು ನಾಲ್ಕು ಘಂಟೆಗಳಿ\break\ ಗೊಂದಾವರ್ತಿ ಅಂದರೆ ಅರುಣೋದಯ, ಮಧ್ಯಾಹ್ನ, ಸಾಯಂಕಾಲ ಮತ್ತು ಅರ್ಧ ರಾತ್ರಿಯಲ್ಲಿ ಮೂರು ಸಾರಿ ಅಥವಾ ಐದು ಸಾರಿ ಮಾಡಿದರೆ, ಹದಿನೈದು ದಿನಗಳು ಅಥವಾ ಒಂದು ತಿಂಗಳಲ್ಲಿ ನಾಡಿಗಳ ಶುದ್ದಿ ಸಿದ್ಧಿಸುವುದು; ಅನಂತರ ಪ್ರಾಣಾಯಾಮ ಮೊದಲಾಗುವುದು. ”

ಅಭ್ಯಾಸ ಅತ್ಯಾವಶ್ಯಕ. ಗಂಟೆಗಳ ಕಾಲ ಕುಳಿತುಕೊಂಡು ನಾನು ಹೇಳುವುದನ್ನು ಕೇಳಬಹುದು. ಆದರೆ ನೀವು ಅಭ್ಯಾಸಮಾಡದೆ ಹೋದರೆ ಒಂದು ಹೆಜ್ಜೆಯೂ ಮುಂದುವರಿಯಲಾರಿರಿ. ಇದೆಲ್ಲಾ ಅಭ್ಯಾಸಬಲದ ಮೇಲೆ ನಿಂತಿದೆ. ಇವನ್ನು ನಾವೇ ಅನುಭವಿಸುವ ತನಕ ನಮಗೆ ಇವು ಅರ್ಥವಾಗುವುದಿಲ್ಲ. ನಾವೇ ಸ್ವಂತವಾಗಿ ಇವನ್ನು ನೋಡಬೇಕು; ಅನುಭವಿಸಬೇಕು; ಸುಮ್ಮನೆ ಸಿದ್ಧಾಂತಗಳನ್ನು ಮತ್ತು ವಿವರಣೆಗಳನ್ನು ಕೇಳುತ್ತಿದ್ದರೆ ಸಾಲದು. ಅಭ್ಯಾಸಕ್ಕೆ ಹಲವು ಅಡಚಣೆಗಳಿವೆ. ಮೊದಲನೆಯ ಅಡಚಣೆಯೇ ದೇಹದ ಅನಾರೋಗ್ಯ. ದೇಹವು ಆರೋಗ್ಯ ಸ್ಥಿತಿಯಲ್ಲಿ ಇಲ್ಲದೇ ಇದ್ದರೆ, ಅಭ್ಯಾಸಕ್ಕೆ ಆತಂಕ ಬರುವುದು. ಆದಕಾರಣ ನಾವು ದೇಹವನ್ನು ಆರೋಗ್ಯ ಸ್ಥಿತಿಯಲ್ಲಿಡಬೇಕು. ನಾವು ಸೇವಿಸುವ ಆಹಾರ, ಪಾನೀಯ, ಮಾಡುವ ಕೆಲಸ ಮುಂತಾದುವುಗಳಲ್ಲಿ ಬಹಳ ಜಾಗರೂಕರಾಗಿರಬೇಕು. ದೇಹವನ್ನು ದೃಢವಾಗಿಟ್ಟಿರುವುದಕ್ಕೆ ಯಾವಾಗಲೂ ಮಾನಸಿಕ ಶಕ್ತಿಯನ್ನು ಉಪಯೋಗಿಸಿ. ಸಾಧಾರಣವಾಗಿ ಇದನ್ನು ಕ್ರೈಸ್ತವಿಜ್ಞಾನವೆಂದು ಕರೆಯುತ್ತಾರೆ. ಇಷ್ಟೆ, ದೇಹದ ವಿಚಾರವಾಗಿ ಮತ್ತೇನೂ ಇಲ್ಲ. ಆರೋಗ್ಯವು ಯಾವುದೋ ಒಂದು ಗುರಿಗೆ ದಾರಿ ಮಾತ್ರ ಎಂಬುದನ್ನು ನಾವು ಮರೆಯಬಾರದು. ಆರೋಗ್ಯವೇ ನಮ್ಮ ಗುರಿಯಾಗಿದ್ದರೆ ನಾವು ಕೂಡ ಮೃಗಗಳಂತೆ ಆಗುವೆವು. ಮೃಗಗಳು ರೋಗದಿಂದ ನರಳುವುದು ಅಪರೂಪ. 

ಎರಡನೆಯ ಆತಂಕವೇ ಸಂಶಯ. ನಮಗೆ ಕಾಣದೆ ಇರುವ ವಸ್ತುವನ್ನು ನಾವು ಯಾವಾಗಲೂ ಅನುಮಾನಿಸುವೆವು. ಎಷ್ಟು ಪ್ರಯತ್ನಪಟ್ಟರೂ ಮಾನವನು ಮಾತಿನ ಮೇಲೆ ಜೀವಿಸಲಾರ. ಆದಕಾರಣ ಇಂತಹ ಆಧ್ಯಾತ್ಮಿಕ ವಸ್ತುಗಳ ವಿಚಾರದಲ್ಲಿ ಏನಾದರೂ ಸತ್ಯವಿದೆಯೇ ಎಂಬ ಅನುಮಾನ ನಮಗೆ ಬರುವುದು. ನಮ್ಮಲ್ಲಿ ಅತ್ಯುತ್ತಮರೂ ಕೂಡ ಕೆಲವು ವೇಳೆ ಅನುಮಾನಿಸುವರು. ಸಾಧನೆ ಮಾಡಿದರೆ ಕೆಲವು ದಿನಗಳಲ್ಲಿ ಸಾಧಕನಲ್ಲಿ ಉತ್ಸಾಹವನ್ನೂ ನಂಬಿಕೆಯನ್ನೂ ಕೊಡುವುದಕ್ಕೆ ಸಾಕಾಗುವಷ್ಟು ಆಧ್ಯಾತ್ಮಿಕ ಕ್ಷಣಿಕ ದರ್ಶನವಾಗುವುದು. ಯೋಗದರ್ಶನದ ಭಾಷ್ಯಕಾರರೊಬ್ಬರು ಹೀಗೆ ಹೇಳುತ್ತಾರೆ: “ನಮಗೆ ಒಂದು ಪ್ರಮಾಣ ಸಿಕ್ಕಿದರೆ ಅದು ಎಷ್ಟು ಅಲ್ಪವಾದರೂ, ಇಡೀ ಯೋಗದ ಉಪದೇಶದ ಮೇಲೆ ನಂಬಿಕೆಯನ್ನು ಹುಟ್ಟಿಸುವುದು. ” ಉದಾಹರಣೆಗೆ ಮೊದಲು ಕೆಲವು ತಿಂಗಳು ಸಾಧನೆಯಾದ ಮೇಲೆ ಮತ್ತೊಬ್ಬರ ಮನಸ್ಸಿನಲ್ಲಿರುವ ಆಲೋಚನೆಗಳನ್ನು ನೀವು ಓದಬಹುದೆಂದು ನಿಮಗೆ ತೋರುವುದು. ಚಿತ್ರಗಳ ರೂಪದಲ್ಲಿ ಅವುಗಳು ನಿಮಗೆ ಗೋಚರಿಸುವುವು. ಕೇಳಬೇಕೆಂಬ ಆಸೆಯ ಮೇಲೆ ನಿಮ್ಮ ಮನಸ್ಸನ್ನು ಏಕಾಗ್ರಮಾಡಿದರೆ, ಬಹುಶಃ ಬಹಳ ದೂರದಲ್ಲಿ ಏನೋ ಆಗುತ್ತಿರುವುದನ್ನು ನೀವು ಕೇಳಬಹುದು. ಮೊದಲು ಇಂತಹ ಕ್ಷಣಿಕ ನೋಟಗಳು ಸ್ವಲ್ಪವಾಗಿ ಬರುವುವು. ಆದರೆ ಶ್ರದ್ಧೆಯನ್ನೂ ನಂಬಿಕೆಯನ್ನೂ ಶಕ್ತಿಯನ್ನೂ ಕೊಡಲು ಇದು ಸಾಕು. ಉದಾಹರಣೆಗೆ ಮೂಗಿನ ಕೊನೆಯ ಮೇಲೆ ನಿಮ್ಮ ಆಲೋಚನೆಯನ್ನು ಏಕಾಗ್ರ ಮಾಡಿದರೆ, ಕೆಲವು ದಿನಗಳಲ್ಲಿ ಅತ್ಯುತ್ತಮವಾದ ಪರಿಮಳವನ್ನು ಅನುಭವಿಸುವಿರಿ. ಬಾಹ್ಯ ವಸ್ತುಗಳ ಸಂಬಂಧವಿಲ್ಲದ ಕೆಲವು ಮಾನಸಿಕ ಸಂವೇದನೆಗಳಿವೆ ಎಂಬುದನ್ನು ತೋರಲು ಇವು ಸಾಕು. ಆದರೆ ಇವು ಮಾರ್ಗಮಾತ್ರವೆಂಬುದನ್ನು ನಾವು ಗಮನದಲ್ಲಿಡಬೇಕು. ನಮ್ಮ ಸಾಧನೆಯ ಅಂತ್ಯಗುರಿ, ಅಂತಿಮ ಲಕ್ಷ್ಯ ಬಂಧನದಲ್ಲಿರುವ ಆತ್ಮನ ವಿಮೋಚನೆ, ಪ್ರಕೃತಿಯಿಂದ ಸಂಪೂರ್ಣ ಸ್ವಾಧೀನತೆ. ಇದಕ್ಕಿಂತ ಎಳ್ಳಷ್ಟೂ ಕಡಮೆಯಾಗಕೂಡದು. ಇದೇ ನಮ್ಮ ಗುರಿಯಾಗಬೇಕು. ನಾವು ಪ್ರಕೃತಿಯ ಗುಲಾಮರಲ್ಲ, ಅದನ್ನು ಆಳುವವರಾಗಬೇಕು. ದೇಹವಾಗಲೀ ಮನಸ್ಸಾಗಲೀ ನಮ್ಮ ಯಜಮಾನನಾಗಕೂಡದು. ದೇಹ ನನ್ನದು, ನಾನು ದೇಹದ ಆಳಲ್ಲ ಎಂಬುದನ್ನು ಮರೆಯಕೂಡದು. 

\vskip 0.3cm

ದೇವ ಮತ್ತು ದಾನವ ಇವರಿಬ್ಬರೂ ಮಹಾ ಋಷಿಯೊಬ್ಬನಿಂದ ಆತ್ಮನ ವಿಚಾರವನ್ನು ತಿಳಿಯುವುದಕ್ಕೆ ಹೋದರು. ಆತನ ಬಳಿಯಲ್ಲಿ ದೀರ್ಘಕಾಲ ಅಧ್ಯಯನ ಮಾಡಿದರು. ಕೊನೆಗೆ ಋಷಿಯು “ನೀವು ಯಾವುದನ್ನು ಹುಡುಕುತ್ತಿರುವಿರೋ ಅದೇ ನೀವು” ಎಂದು ಹೇಳಿದನು. ಇಬ್ಬರೂ ಕೂಡ ತಮ್ಮ ದೇಹವನ್ನೇ ಆತ್ಮನೆಂದು ತಿಳಿದರು. ತೃಪ್ತರಾಗಿ ತಮ್ಮ ಜನಗಳ ಬಳಿಗೆ ಹಿಂತಿರುಗಿದರು. “ಕಲಿಯಬೇಕಾದು ದೆಲ್ಲವನ್ನೂ ನಾವು ಕಲಿತೆವು. ತಿನ್ನಿರಿ, ಕುಡಿಯಿರಿ, ಸಂತೋಷವಾಗಿರಿ. ನಾವು ಆತ್ಮ. ಇದಕ್ಕಿಂತ ಹೆಚ್ಚಿನದು ಯಾವುದೂ ಇಲ್ಲ” ಎಂದು ಸಾರಿದರು. ಅಜ್ಞಾನವೇ ದಾನವನ ಸ್ವಭಾವ. ಅವನು ಅದಕ್ಕಿಂತ ಹೆಚ್ಚು ಏನೂ ವಿಚಾರ ಮಾಡಲಿಲ್ಲ. ತಾನೇ ದೇವರು ಮತ್ತು ಆತ್ಮನೆಂದರೆ ದೇಹವೆಂದು ತಿಳಿದು ಸಂಪೂರ್ಣವಾಗಿ ಸಮಾಧಾನ ಹೊಂದಿದನು. ಈತನಿಗಿಂತ ಪರಿಶುದ್ಧವಾದ ಸ್ವಭಾವ ದೇವನದು. ಆತನು ಕೂಡ ಮೊದಲು “ಈ ದೇಹವಾದ ನಾನೇ ಬ್ರಹ್ಮ; ಆದಕಾರಣ ಅದನ್ನು ದೃಢವಾಗಿ ಆರೋಗ್ಯವಾಗಿಟ್ಟು, ಚೆನ್ನಾದ ಬಟ್ಟೆಗಳನ್ನು ಹಾಕಿ, ಅದಕ್ಕೆ ಬೇಕಾದ ಭೋಗಗಳನ್ನೆಲ್ಲ ಕೊಡಬೇಕು” ಎಂದು ತಪ್ಪಾಗಿ ಭಾವಿಸಿದ್ದನು. ಆದರೆ ಸ್ವಲ್ಪ ದಿನಗಳಲ್ಲೆಲ್ಲ ತನ್ನ ಗುರುಗಳಾದ ಋಷಿಯ ಅಭಿಪ್ರಾಯವು ಇದಲ್ಲ ಇದಕ್ಕಿಂತ ಏನೋ ಉನ್ನತವಾಗಿರಬೇಕು ಎಂದು ಅವರು ಕಂಡುಕೊಂಡನು. ಅದಕ್ಕೆ ಪುನಃ ಅವರಲ್ಲಿಗೆ ಬಂದು “ಗುರುಗಳೇ, ಈ ದೇಹವೇ ಆತ್ಮನೆಂದು ನನಗೆ ಬೋಧಿಸಿದಿರೇನು? ಹಾಗಾದರೆ ಎಲ್ಲಾ ದೇಹಗಳೂ ಸಾಯುವುದು ನನಗೆ ಕಾಣುತ್ತದೆ. ಆತ್ಮನೆಂದಿಗೂ ನಾಶವಾಗಲಾರನು” ಎಂದನು. ಋಷಿಯು “ನೀನೇ ಅದು, ಅದನ್ನೇ ಹುಡುಕು,” ಎಂದು ಹೇಳಿದನು. ಅನಂತರ ದೇವನು ದೇಹವು ಕೆಲಸ ಮಾಡುವಂತೆ ಮಾಡುವ ಪ್ರಾಣವೇ ಆತ್ಮನಿರಬೇಕೆಂದು ಆಲೋಚಿಸಿದನು. ಆದರೆ ಕೆಲವು ಕಾಲವಾದ ಮೇಲೆ ಅವನು ಊಟ ಮಾಡಿದರೆ ಪ್ರಾಣ ಬಲವಾಗಿರುವುದನ್ನೂ, ಉಪವಾಸ ಮಾಡಿದರೆ ನಿರ್ಬಲವಾಗುವುದನ್ನೂ ತಿಳಿದನು. ಪುನಃ ಋಷಿಯ ಹತ್ತಿರ ಹೋಗಿ “ಗುರುಗಳೆ, ನೀವು ಪ್ರಾಣವನ್ನು ಆತ್ಮನೆಂದು ಹೇಳಿದಿರಾ?” ಎಂದು ಕೇಳಿದನು. ಅದಕ್ಕೆ ಋಷಿಯು “ನೀನೇ ಅದನ್ನು ಕಂಡುಹಿಡಿ” ಎಂದು ಹೇಳಿದನು. ಪುನಃ ಮನೆಗೆ ಹಿಂತಿರುಗಿ ಬಹುಶಃ ಆತ್ಮನು ಮನಸ್ಸಾಗಿರಬಹುದೆಂದು ತಿಳಿದನು. ಆದರೆ ಕೆಲವು ಕಾಲದಲ್ಲೇ, ಆಲೋಚನೆಗಳು ಒಮ್ಮೆ ಒಳ್ಳೆಯವು, ಒಮ್ಮೆ ಕೆಟ್ಟುವು – ಹೀಗೆ ಬಹುವಿಧವಾಗಿವೆ; ಮನಸ್ಸು ಅತಿ ಚಂಚಲವಾಗಿರುವುದರಿಂದ ಅದು ಆತ್ಮನಾಗಲಾರದೆಂದು ಕಂಡುಹಿಡಿದನು. ಪುನಃ ಋಷಿಯ ಹತ್ತಿರ ಹೋಗಿ “ಗುರುಗಳೇ ಮನಸ್ಸು ನನಗೆ ಆತ್ಮನೆಂದು ತೋರುವುದಿಲ್ಲ. ನೀವು ಹಾಗೆ ಹೇಳಿದಿರೇನು?” ಎಂದು ಕೇಳಿದನು. “ಇಲ್ಲ, ನೀನೇ ಅವನು; ಅದನ್ನು ಕಂಡುಹಿಡಿ” ಎಂದು ಹೇಳಿದನು. ದೇವನು ಕೊನೆಗೆ ಮನೆಗೆ ಹೋಗಿ ಆಲೋಚನಾತೀತವಾದ ಆತ್ಮನು ತಾನು ಎಂಬುದನ್ನು ಕಂಡುಹಿಡಿದನು. ಜನನ ಮರಣಾತೀತನವನು. ಕತ್ತಿ ಅವನನ್ನು ಕತ್ತರಿಸಲಾರದು. ಬೆಂಕಿ ಅವನನ್ನು ದಹಿಸಲಾರದು. ಅವನು ಆದಿ–ಅಂತ್ಯರಹಿತನು, ಅಚಲನು, ಅಗೋಚರನು, ಅವನೇ ಸರ್ವಜ್ಞ, ಸರ್ವಶಕ್ತಿಮಾನ್​. ಅವನು ದೇಹವೂ ಅಲ್ಲ, ಮನಸ್ಸೂ ಅಲ್ಲ. ಇವುಗಳನ್ನು ಮೀರಿದವನು. ಈ ಜ್ಞಾನದಿಂದ ಅವನಿಗೆ ತೃಪ್ತಿಯಾಯಿತು. ಆದರೆ, ಪಾಪ! ದಾನವನಿಗೆ ದೇಹಾಭಿಲಾಷೆ ಹೆಚ್ಚಾಗಿದ್ದುದರಿಂದ ಸತ್ಯವು ಗೋಚರಿಸಲಿಲ್ಲ.

\vskip 0.3cm

ಈ ಪ್ರಪಂಚದಲ್ಲಿ ಆಸುರೀ ಪ್ರಕೃತಿಯವರೇ ಹೆಚ್ಚು. ಆದರೆ ಕೆಲವರು ದೇವತೆಗಳು ಇಲ್ಲಿ ಕೂಡ ಇರುವರು. ಇಂದ್ರಿಯ ಭೋಗಶಕ್ತಿಯನ್ನು ಅಭಿವೃದ್ಧಿ ಮಾಡಬಲ್ಲ ಯಾವುದಾದರೊಂದು ಶಾಸ್ತ್ರವನ್ನು ಬೋಧಿಸಲು ಪ್ರಯತ್ನಪಟ್ಟರೆ ಅದನ್ನು ಅನುಸರಿಸಲು ಸಿದ್ಧರಾಗಿರುವ ಸಹಸ್ರಾರು ಜನರು ಸಿಕ್ಕುವರು. ಜೀವನದ ಪರಮ ಗುರಿಯನ್ನು ತೋರುತ್ತೇನೆಂದು ಪ್ರಯತ್ನಪಟ್ಟರೆ ಅದನ್ನು ಕೇಳುವವರೆಲ್ಲೋ ಅಲ್ಪ ಮಂದಿ. ಈ ಅತ್ಯುತ್ತಮ ವಿಷಯವನ್ನು ತಿಳಿದುಕೊಳ್ಳುವ ಶಕ್ತಿ ಎಲ್ಲೋ ಕೆಲವರಿಗಿದೆ. ಅದನ್ನು ಸಾಧನೆಮಾಡಬಲ್ಲ ಸಮಾಧಾನವಿರುವವರು ಅದಕ್ಕಿಂತ ಕಡಮೆ. ದೇಹವನ್ನು ಒಂದು ಸಾವಿರ ವರ್ಷಗಳು ಬದುಕುವಂತೆ ಮಾಡಿದರೂ ಕೊನೆಗೆ ಪ್ರತಿಫಲವೆಲ್ಲ ಒಂದೇ ಎಂದು ತಿಳಿದಿರುವವರು ಕೆಲವೆ ಮಂದಿ. ದೇಹವನ್ನು ಹಿಡಿದಿ ರಿಸಿರುವ ಶಕ್ತಿ ಹೋದರೆ ಅದು ನಾಶವಾಗಲೇಬೇಕು. ಬದಲಾವಣೆ ಹೊಂದದೆ ದೇಹವನ್ನು ಒಂದು ಕ್ಷಣ ಕಾಲವಾದರೂ ಇಡಬಲ್ಲ ಮಾನವನು ಎಂದೂ ಹುಟ್ಟಿಲ್ಲ. ಬದಲಾವಣೆಗಳ ಒಂದು ಶ್ರೇಣಿಗೆ ದೇಹವೆಂದು ಹೆಸರು. ಹೇಗೆ ನಿಮ್ಮ ಕಣ್ಣಮುಂದೆ ನದಿಯಲ್ಲಿ ನೀರು ಪ್ರತಿಕ್ಷಣವೂ ಹರಿದುಹೋಗುತ್ತಿದೆಯೋ ಮತ್ತು ಹೊಸ ನೀರು ಬರುತ್ತಿದೆಯೋ, ಆದರೂ ಕೂಡ ಹಿಂದಿನ ಆಕಾರವನ್ನೇ ತಾಳಿದೆಯೋ, ಅದರಂತೆಯೇ ದೇಹವೂ ಕೂಡ. ಆದರೂ ಕೂಡ ನಾವು ದೇಹವನ್ನು ದೃಢವಾಗಿ, ಆರೋಗ್ಯವಾಗಿಡ ಬೇಕು. ನಮಗೆ ಇರುವ ಅತ್ಯುತ್ತಮ ಯಂತ್ರವೇ ಇದು. 

\vskip 0.3cm

ಪ್ರಪಂಚದಲ್ಲಿರುವ ಎಲ್ಲಾ ದೇಹಗಳಿಗಿಂತಲೂ ಅತ್ಯುತ್ತಮವಾದುದು ಈ ಮಾನವ ದೇಹ. ಮಾನವನೇ ಅತ್ಯುತ್ತಮವಾದ ಜೀವಿ. ಎಲ್ಲಾ ಪ್ರಾಣಿಗಳಿಗಿಂತಲೂ, ಎಲ್ಲಾ ದೇವತೆಗಳಿಗಿಂತಲೂ ಅತ್ಯುತ್ತಮನಾದವನೇ ಮಾನವನು. ಮಾನವನಿಗಿಂತ ಅತ್ಯುತ್ತಮನಾರೂ ಇಲ್ಲ. ದೇವತೆಗಳೂ ಕೂಡ ಪುನಃ ಇಲ್ಲಿಗೆ ಬಂದು ಮಾನವ ದೇಹದ ಮೂಲಕ ಮೋಕ್ಷವನ್ನು ಸಂಪಾದಿಸಬೇಕು. ಮಾನವನಿಗೊಬ್ಬನಿಗೇ ಮೋಕ್ಷ ದೊರಕುವುದು. ದೇವತೆಗಳಿಗೂ ಕೂಡ ಇಲ್ಲ. ಯೆಹೂದ್ಯರ ಮತ್ತು ಮಹಮ್ಮದೀಯರ ಅಭಿಪ್ರಾಯದಂತೆ, ಈಶ್ವರನು, ದೇವತೆಗಳು ಮತ್ತು ಇನ್ನೂ ಇತರ ವಸ್ತುಗಳನ್ನೆಲ್ಲ ಸೃಷ್ಟಿಸಿ ಆದಮೇಲೆ ಮಾನವನನ್ನು ಸೃಷ್ಟಿಸಿ, ಅನಂತರ ಬಂದು ಮಾನವನಿಗೆ ನಮಸ್ಕಾರ ಮಾಡುವಂತೆ ದೇವತೆಗಳಿಗೆ ಹೇಳಿದನು. ಇಬ್ಲಿಸ್ಸನು ವಿನಾ ಉಳಿದವರೆಲ್ಲರೂ ಹೀಗೆ ಮಾಡಿದರು. ಅದಕ್ಕೆ ದೇವರು ಅವನಿಗೆ ಶಾಪ ಕೊಟ್ಟನು. ಅದರಿಂದ ಅವನು ಸೈತಾನನಾದನು. ಈ ರೂಪಕ ಕಥೆಯ ಹಿಂದೆ ದೊಡ್ಡ ಸತ್ಯವಿದೆ. ನಮಗೆ ಪ್ರಾಪ್ತವಾಗುವ ಜನ್ಮದಲ್ಲೆಲ್ಲಾ ಮಾನವ ಜನ್ಮವೇ ಅತ್ಯುತ್ತಮವಾದುದು. ಕೆಳಗಿನ ವರ್ಗದ ಪ್ರಾಣಿಗಳು ಮಂದಬುದ್ಧಿಯವು; ಹೆಚ್ಚು ಪಾಲು ತಮಸ್ಸಿನಿಂದ ಸೃಷ್ಟಿಯಾದವು ಅವು. ಬಹಳ ಉದಾತ್ತ ಭಾವನೆ ಪ್ರಾಣಿಗಳಿಗೆ ಇರಲಾರದು. ಅಥವಾ ದೇವತೆಗಳೂ ಕೂಡ ಮಾನವ ದೇಹವಿಲ್ಲದೆ ಮೋಕ್ಷವನ್ನು ನೇರವಾಗಿ ಸಾಧಿಸಲು ಸಾಧ್ಯವಿಲ್ಲ. ಅದರಂತೆಯೇ ಮಾನವ ಸಮಾಜದಲ್ಲಿ ಕೂಡ ಅತಿ ಬಡತನ, ಅತಿ ಐಶ್ವರ್ಯ ಇವೆರಡೂ ಉತ್ತಮ ಆತ್ಮವಿಕಾಸಕ್ಕೆ ಅಡಚಣೆ. ಪ್ರಪಂಚದ ಮಹಾಪುರುಷರು ಬರುವುದು ಮಧ್ಯಮವರ್ಗದವರಿಂದ. ಇಲ್ಲಿ ಶಕ್ತಿ ಒಂದೇ ಸಮನಾಗಿ ವಿಭಾಗವಾಗಿರುತ್ತದೆ, ಸಮತೂಕದಿಂದಿರುತ್ತದೆ. 

\vskip 0.3cm

ನಮ್ಮ ಪ್ರಸ್ತುತ ವಿಷಯಕ್ಕೆ ಬರೋಣ. ಮುಂದಿನ ಹಂತ ಪ್ರಾಣಾಯಾಮ ಅಥವಾ ಶ್ವಾಸೋಚ್ಛ್ವಾಸಗಳ ನಿಗ್ರಹ. ಮಾನಸಿಕ ಶಕ್ತಿಯ ಏಕಾಗ್ರತೆಗೂ ಇದಕ್ಕೂ ಏನು ಸಂಬಂಧವಿದೆ? ಈ ದೇಹವೆಂಬ ನಮ್ಮ ಯಂತ್ರಕ್ಕೆ ಉಸಿರು ಚಲನಾಶಕ್ತಿಯನ್ನು ಕೊಡುವ ಚಕ್ರದಂತೆ. ದೊಡ್ಡ ಯಂತ್ರದಲ್ಲಿ ಮೊದಲು ಫ್ಲೈವೀಲ್​ ಚಲಿಸುವುದು. ಈ ಚಲನೆಯು ಅನಂತರ ಹೆಚ್ಚು ಹೆಚ್ಚು ಸೂಕ್ಷ್ಮ ಭಾಗಗಳಿಗೂ ಪ್ರವೇಶಿಸಿ, ಕೊನೆಗೆ ಅತ್ಯಂತ ಸೂಕ್ಷ್ಮವಾದ ಗುಪ್ತ ಭಾಗಗಳನ್ನೂ ಚಲಿಸುವಂತೆ ಮಾಡುವುದು. ದೇಹದಲ್ಲಿರುವ ಪ್ರತಿಯೊಂದು ಭಾಗಕ್ಕೂ ಕ್ರಿಯೋತ್ತೇಜಕ ಶಕ್ತಿಯನ್ನು ಕೊಟ್ಟು ಅದನ್ನು ನಿಯಂತ್ರಿಸುವ ಚಕ್ರ ಉಸಿರು ಎಂಬುದು. 

ಒಂದಾನೊಂದು ಕಾಲದಲ್ಲಿ ಪ್ರಖ್ಯಾತನಾದ ದೊರೆಯೊಬ್ಬನಿಗೆ ಒಬ್ಬ ಮಂತ್ರಿ ಇದ್ದನು. ಅವನು ಒಮ್ಮೆ ರಾಜನ ಕೋಪಕ್ಕೆ ತುತ್ತಾದನು. ಆತನಿಗೆ ಶಿಕ್ಷೆಯನ್ನು ಕೊಡುವ ಸಲುವಾಗಿ ಎತ್ತರವಾದ ಗೋಪುರದ ಮೇಲೆ ಅವನನ್ನು ಬಂಧಿಸಿ ಎಂದು ಆಜ್ಞಾಪಿಸಿದನು. ರಾಜಾಧಿಕಾರಿಗಳು ಹಾಗೆಯೇ ಮಾಡಿದರು. ಅಲ್ಲಿಯೆ ನಾಶವಾಗಲೆಂದು ಮಂತ್ರಿಯನ್ನು ಬಿಟ್ಟರು.\break\ ಆತನಿಗೆ ಪತಿವ್ರತೆಯಾದ ಹೆಂಡತಿಯಿದ್ದಳು. ರಾತ್ರಿ ಗೋಪುರದ ಹತ್ತಿರ ಬಂದು ಮೇಲಿದ್ದ ಗಂಡನನ್ನು ಕರೆದು ತಾನು ಹೇಗೆ ಅವನಿಗೆ ಸಹಾಯಕಳಾಗಬಹುದೆಂದು ಕೇಳಿದಳು. ಮಂತ್ರಿಯು ಅವಳಿಗೆ ಮಾರನೆ ದಿನ ರಾತ್ರಿ, ಒಂದು ದೊಡ್ಡ ಹಗ್ಗ, ಗಟ್ಟಿಯಾದ ದಾರ, ಅದಕ್ಕಿಂತ ಸಣ್ಣದಾದ ದಾರ, ತೆಳ್ಳಗಿರುವ ರೇಷ್ಮೆ ನೂಲು, ಒಂದು ಜೀರುದುಂಬಿ, ಸ್ವಲ್ಪ ಜೇನುತುಪ್ಪ, ಇವುಗಳೆಲ್ಲವನ್ನೂ ತರುವಂತೆ ಹೇಳಿದನು. ವಿಧೇಯಳಾದ ಹೆಂಡತಿ ಆಶ್ಚರ್ಯಪಟ್ಟು ಗಂಡನ ಅಪ್ಪಣೆಯನ್ನು ಶಿರಸಾವಹಿಸಿ ಅವನು ಹೇಳಿದ ಸಾಮಾನುಗಳೆಲ್ಲವನ್ನೂ ಮಾರನೆ ದಿನ ರಾತ್ರಿ ತಂದಳು. ಗಂಡನು ಹೆಂಡತಿಗೆ ರೇಷ್ಮೆ ನೂಲನ್ನು ಜೀರುದುಂಬಿಗೆ ಬಿಗಿದು, ಅದರ ಮುಖದ ಮೇಲಿರುವ ಕೂದಲಿಗೆ ಜೇನುತುಪ್ಪವನ್ನು ಸವರಿ, ಮುಖವನ್ನು ಮೇಲು ಮಾಡಿ ನೇರವಾಗಿ ಗೋಪುರದ ಕಡೆಗೆ ಗೋಡೆಯ ಮೇಲೆ ಬಿಡುವಂತೆ ಹೇಳಿದನು. ಆಕೆಯು ಸಲಹೆಯಂತೆ ಮಾಡಿದಳು. ಜೀರುದುಂಬಿಯು ಗೋಡೆಯ ಮೇಲೆ ತನ್ನ ದೂರ ಪ್ರಯಾಣಕ್ಕೆ ಹೊರಟಿತು. ಮೇಲಿರುವ ಜೇನುತುಪ್ಪವನ್ನು ಮೂಸಿನೋಡುತ್ತ, ಅದನ್ನು ಪಡೆಯಲು ನಿಧಾನವಾಗಿ ಮೇಲಮೇಲಕ್ಕೆ ಏರಿ, ಕೊನೆಗೆ ಗೋಪುರದ ತುದಿಯನ್ನು ಸೇರಿತು. ಆಗ ಮಂತ್ರಿಯು ಜೀರುದುಂಬಿಯನ್ನು ಹಿಡಿದು ಕೊಂಡು ಅದಕ್ಕೆ ಕಟ್ಟಿದ್ದ ರೇಷ್ಮೆ ಎಳೆಯನ್ನು ತೆಗೆದುಕೊಂಡನು. ಆಗ ರೇಷ್ಮೆ ಎಳೆಯ ಇನ್ನೊಂದು ಕೊನೆಗೆ, ಅದಕ್ಕಿಂತ ಸ್ವಲ್ಪ ದಪ್ಪವಾಗಿರುವ ದಾರವನ್ನು ಕಟ್ಟೆಂದು ಹೇಳಿದನು. ಅದನ್ನು ಮೇಲಕ್ಕೆ ಎಳೆದಾದ ಮೇಲೆ ಅದರ ಕೊನೆಗೆ ದಪ್ಪ ದಾರವನ್ನು ಕಟ್ಟೆಂದು ಹೇಳಿದನು. ಅನಂತರ ಅದಕ್ಕೆ ಹಗ್ಗವನ್ನು ಕಟ್ಟು ಎಂದನು. ಅನಂತರ ಉಳಿದುದೆಲ್ಲ ಬಹಳ ಸುಲಭವಾಯಿತು. ಮಂತ್ರಿಯು ಹಗ್ಗದ ಮೂಲಕವಾಗಿ ಗೋಪುರದಿಂದ ಇಳಿದು ಸೆರೆಯಿಂದ ತಪ್ಪಿಸಿಕೊಂಡನು. ನಮ್ಮ ದೇಹದಲ್ಲಿ ಉಸಿರಿನ ಚಲನೆಯೇ “ರೇಷ್ಮೆಯ ನೂಲು.” ಅದನ್ನು ನಮ್ಮ ಸ್ವಾಧೀನಕ್ಕೆ ತರುವುದನ್ನು ಕಲಿತಮೇಲೆ ನರಗಳ ಶಕ್ತಿ ಎಂಬ ದಪ್ಪ ನೂಲಿನ ಸ್ವಾಧೀನವಾಗುವುದು. ಇದರಿಂದ ಇದಕ್ಕಿಂತಲೂ ದಪ್ಪನಾದ ಆಲೋಚನೆಯ ದಾರ ನಮ್ಮ ವಶವಾಗುವುದು; ಕೊನೆಗೆ ಪ್ರಾಣವೆಂಬ ದಾರ ನಮ್ಮ ಸ್ವಾಧೀನವಾಗುವುದು. ಅದನ್ನು ನಿಗ್ರಹಿಸುವುದನ್ನು ಕಲಿತರೆ ಮೋಕ್ಷ ಪ್ರಾಪ್ತವಾಗುವುದು. 

\vskip 0.3cm

ನಮ್ಮ ದೇಹದ ವಿಚಾರ ನಮಗೇನೂ ಗೊತ್ತಿಲ್ಲ. ಗೊತ್ತಾಗುವಂತೆಯೂ ಇಲ್ಲ. ಹೆಚ್ಚು ಎಂದರೆ ನಾವು ಒಂದು ಹೆಣವನ್ನು ತೆಗೆದುಕೊಂಡು ಚೂರುಚೂರಾಗಿ ಕತ್ತರಿಸಬಹುದು. ಮತ್ತೆ ಕೆಲವರು ಒಂದು ಬದುಕಿರುವ ಮೃಗವನ್ನು ತೆಗೆದುಕೊಂಡು ಅದರೊಳಗೆ ಏನಿವೆ ಎಂದು ಪರೀಕ್ಷೆ ಮಾಡುವುದಕ್ಕೆ ಅದನ್ನು ಕತ್ತರಿಸಬಹುದು. ಆದರೆ ಅದಕ್ಕೂ ನಮ್ಮ ದೇಹಕ್ಕೂ ಏನೂ ಸಂಬಂಧವಿಲ್ಲ. ನಮ್ಮ ದೇಹದ ವಿಚಾರವಾಗಿ ನಮಗೆ ತಿಳಿದಿರುವುದು ಬಹಳ ಸ್ವಲ್ಪ. ನಮಗೆ ಏಕೆ ಗೊತ್ತಾಗುವುದಿಲ್ಲ? ಏಕೆಂದರೆ ಒಳಗೆ ನಡೆಯುತ್ತಿರುವ ಅತ್ಯಂತ ಸೂಕ್ಷ್ಮ ಚಲನೆಗಳನ್ನು ಗ್ರಹಿಸುವಷ್ಟು ತೀಕ್ಷ್ಣ ಗಮನ ನಮಗಿಲ್ಲ. ನಮ್ಮ ಮನಸ್ಸು ಮತ್ತೂ ಸೂಕ್ಷ್ಮವಾಗಿ ದೇಹದೊಳಗೆ ಪ್ರವೇಶಿಸುವಂತಾದರೆ ಆಗ ನಮಗೆ ಅವು ತಿಳಿಯುತ್ತವೆ. ನಮಗೆ ಸೂಕ್ಷ್ಮಗ್ರಹಣ ಶಕ್ತಿ ಬೇಕಾದರೆ ಸ್ಥೂಲಗ್ರಹಣ ಶಕ್ತಿಯಿಂದ ಮೊದಲು ಮಾಡಬೇಕು. ಇಡೀ ಯಂತ್ರವನ್ನು ಚಲಿಸುವಂತೆ ಮಾಡುವ ಶಕ್ತಿಯನ್ನು ಸ್ವಾಧೀನ ಮಾಡಿಕೊಳ್ಳಬೇಕು. ಅದೇ ಪ್ರಾಣ. ಅದರ ಮುಖ್ಯ ಆವಿರ್ಭಾವವೇ ಉಸಿರು. ನಾವು ಉಸಿರಿನೊಂದಿಗೆ ನಿಧಾನವಾಗಿ ದೇಹವನ್ನು ಪ್ರವೇಶಿಸುವೆವು. ಅದು ದೇಹದಲ್ಲೆಲ್ಲ ಸಂಚರಿಸುವ ಸೂಕ್ಷ್ಮವಾದ ನರಗಳ ಶಕ್ತಿತರಂಗಗಳನ್ನು ಕಂಡುಹಿಡಿಯಲು ನಮಗೆ ಸಹಾಯ ಮಾಡುವುದು. ಅವು ನಮಗೆ ಗೋಚರವಾಗಿ, ಅದನ್ನು ನಾವು ತಿಳಿದಕೂಡಲೇ ದೇಹದ ಮೇಲೆ ಮತ್ತು ಅವುಗಳ ಮೇಲೆ ಸ್ವಾಧೀನತೆ ಬರುವುದು. ಈ ಬೇರೆ ಬೇರೆ ನರಗಳ ಶಕ್ತಿಪ್ರವಾಹವೇ ದೇಹವನ್ನು ಚಲಿಸುವಂತೆ ಮಾಡುವುದು. ಹೀಗೆ ಕೊನೆಗೆ ದೇಹವನ್ನು ಮತ್ತು ಮನಸ್ಸನ್ನು ಸಂಪೂರ್ಣ ನಿಗ್ರಹಿಸಿ ಅವನ್ನು ನಮ್ಮ ಸೇವಕರನ್ನಾಗಿ ಮಾಡಿಕೊಳ್ಳುವ ಸ್ಥಿತಿಗೆ ಬರುವೆವು. ಜ್ಞಾನವೇ ಶಕ್ತಿ; ನಮಗೆ ಈ ಶಕ್ತಿ ಬೇಕಾಗಿದೆ. ಆದ ಕಾರಣ ಪ್ರಾಣಾಯಾಮ ಅಥವಾ ಪ್ರಾಣದ ನಿಗ್ರಹವನ್ನು ಮೊದಲು ಮಾಡಬೇಕು. ಪ್ರಾಣಾಯಾಮ ಬಹಳ ದೊಡ್ಡ ವಿಷಯ. ಅದನ್ನು ಸಂಪೂರ್ಣವಾಗಿ ವಿವರಿಸಬೇಕಾದರೆ ಅನೇಕ ಅಧ್ಯಾಯಗಳು ಬೇಕಾಗುತ್ತವೆ. ಈ ವಿಷಯವನ್ನು ಸ್ವಲ್ಪ ಸ್ವಲ್ಪವಾಗಿ ತೆಗೆದುಕೊಳ್ಳೋಣ. 

\vskip 0.3cm

ಪ್ರತಿಯೊಂದು ಅಭ್ಯಾಸ, ಮತ್ತು ಅದರಿಂದ ನಮ್ಮ ದೇಹದಲ್ಲಿ ಯಾವ ಶಕ್ತಿಗಳು ಚುರುಕುಗೊಳ್ಳುತ್ತವೆ, ಎನ್ನುವುದಕ್ಕೆ ಕಾರಣವನ್ನು ನಾವು ಕ್ರಮೇಣ ನೋಡುತ್ತೇವೆ. ಇವುಗಳೆಲ್ಲ ನಮಗೆ ಬರುತ್ತವೆ. ಆದರೆ ಇದಕ್ಕೆ ನಿರಂತರ ಸಾಧನೆ ಬೇಕಾಗಿದೆ. ಸಾಧನೆಯಿಂದ ಪ್ರಮಾಣ ಸಿಗುತ್ತದೆ. ನಾನು ಹಲವು ಯುಕ್ತಿಗಳ ಮೂಲಕ ಈ ವಿಷಯವನ್ನು ಪ್ರಸ್ತಾಪಿಸಬಹುದು. ಆದರೆ ನೀವೇ ಇದನ್ನು ನಿಮ್ಮ ಜೀವನದಲ್ಲಿ ಪ್ರಯೋಗ ಮಾಡುವವರೆಗೆ ನಿಮಗೆ ಅವು ಪ್ರಮಾಣವಾಗಲಾರವು. ನಿಮ್ಮಲ್ಲಿ ಈ ಶಕ್ತಿ ಸಂಚಾರದ ಚಿಹ್ನೆ ನಿಮಗೆ ಗೊತ್ತಾದ ಕೂಡಲೇ ಸಂಶಯಗಳು ಮಾಯವಾಗುತ್ತವೆ. ಆದರೆ ಇದಕ್ಕೆ ಪ್ರತಿದಿನವೂ ತೀವ್ರ ಅಭ್ಯಾಸ ಮಾಡಬೇಕು. ಪ್ರತಿದಿನವೂ ಕಡಮೆ ಪಕ್ಷ ಎರಡು ಸಾರಿಯಾದರೂ ಅಭ್ಯಾಸ ಮಾಡಬೇಕು. ಇದಕ್ಕೆ ಬಹಳ ಅನುಕೂಲವಾದ ಕಾಲ ಪ್ರಾತಃಕಾಲ ಮತ್ತು ಸಾಯಂಕಾಲ. ರಾತ್ರಿ ಕಳೆದು ಹಗಲಾಗುತ್ತಿರುವಾಗ, ದಿನ ಕಳೆದು ರಾತ್ರಿಯಾಗುತ್ತಿರುವಾಗ ಸ್ವಭಾವತಃ ಒಂದು ವಿಧವಾದ ಶಾಂತಿ ಪ್ರಾಪ್ತವಾಗುವುದು. ಪ್ರಾತಃಕಾಲ ಮತ್ತು ಸಂಜೆ ಎರಡೂ ಶಾಂತಿ ಸಮಯ. ಆ ಸಮಯದಲ್ಲಿ ನಿಮ್ಮ ದೇಹವೂ ಕೂಡ ಶಾಂತವಾಗಿರುವುದು. ಈ ಬಾಹ್ಯ ಸ್ಥಿತಿಯ ಸ್ವಭಾವವನ್ನು ನಾವು ಉಪಯೋಗಿಸಿಕೊಂಡು ಆ ಸಮಯದಲ್ಲಿ ಸಾಧನೆಯನ್ನು ಪ್ರಾರಂಭಿಸಬೇಕು. ನೀವು ಅಭ್ಯಾಸ ಮಾಡಿದ ಹೊರತು ಊಟ ಮಾಡುವುದಿಲ್ಲವೆಂಬ ಒಂದು ನಿಯಮವನ್ನು ಮಾಡಿ. ಆಗ ಹಸಿವಿನ ಬಲವಂತವೇ ನಿಮ್ಮ ಸೋಮಾರಿತನವನ್ನು ಓಡಿಸುತ್ತದೆ. ಇಂಡಿಯಾ ದೇಶದಲ್ಲಿ ಮಕ್ಕಳಿಗೆ ಸಾಧನೆ ಅಥವಾ ಪೂಜೆ ಇವುಗಳನ್ನು ಮಾಡಿದ ಹೊರತು ಊಟ ಮಾಡಬಾರದೆಂದು ಕಲಿಸುತ್ತಾರೆ. ಕೆಲವು ದಿನಗಳಾದ ಮೇಲೆ ಇದು ಅವರ ಸಹಜ ಗುಣವಾಗುವುದು. ಸ್ನಾನ ಮಾಡಿ ಪೂಜೆ ಮಾಡಿದ ಹೊರತು ಹುಡುಗನಿಗೆ ಹಸಿವಾಗುವುದಿಲ್ಲ. 

\eject

ಸಾಧ್ಯವಾದರೆ ಈ ಅಭ್ಯಾಸಕ್ಕಾಗಿಯೇ ಒಂದು ಕೋಣೆಯನ್ನು ಪ್ರತ್ಯೇಕವಾಗಿಡುವುದು ಒಳ್ಳೆಯದು. ಆ ಕೋಣೆಯಲ್ಲಿ ಮಲಗಬೇಡಿ. ಅದನ್ನು ಪವಿತ್ರವಾಗಿಟ್ಟಿರಬೇಕು. ನೀವು ಸ್ನಾನ ಮಾಡಿ ದೇಹ ಮತ್ತು ಮನಸ್ಸುಗಳೆರಡೂ ಪವಿತ್ರವಾಗುವ ತನಕ ಅದನ್ನು ಪ್ರವೇಶಿಸಬೇಡಿ. ಯಾವಾಗಲೂ ಕೋಣೆಯಲ್ಲಿ ಹೂವುಗಳನ್ನು ಇಡಿ. ಯೋಗಿಯ ಸುತ್ತಲೂ ಇರಬೇಕಾದ ಅತ್ಯುತ್ತಮ ವಾತಾವರಣ ಇದು. ನಿಮಗೆ ಸಂತೋಷವನ್ನುಂಟುಮಾಡುವ ಕೆಲವು ಚಿತ್ರಗಳನ್ನು ಅಲ್ಲಿಡಿ. ಗಂಧದ ಕಡ್ಡಿಯನ್ನು ಬೆಳಿಗ್ಗೆ ಮತ್ತು ಸಾಯಂಕಾಲ ಉರಿಸಿ. ಆ ಕೋಣೆಯಲ್ಲಿ ಜಗಳ, ಕೋಪ ಮತ್ತು ಯಾವ ವಿಧದ ಅಪವಿತ್ರ ಆಲೋಚನೆಗಳೂ ಇಲ್ಲದಿರಲಿ. ನಿಮ್ಮಂತೆ ಇರುವವರಿಗೆ ಮಾತ್ರ ಒಳಗೆ ಪ್ರವೇಶಿಸುವುದಕ್ಕೆ ಅವಕಾಶವನ್ನು ಕೊಡಿ. ಅನಂತರ, ಕ್ರಮೇಣ ಆ ಕೋಣೆಯಲ್ಲಿ ಒಂದು ಪವಿತ್ರ ವಾತಾವರಣ ಸೃಷ್ಟಿಯಾಗುತ್ತದೆ. ನೀವು ಕಷ್ಟದಲ್ಲಿರುವಾಗ, ದುಃಖ, ಸಂಶಯ ಅಥವಾ ಇನ್ನು ಯಾವ ವಿಧದಿಂದಲಾದರೂ ನಿಮ್ಮ ಮನಸ್ಸು ಚಂಚಲವಾಗಿರುವಾಗ, ನೀವು ಆ ಕೋಣೆಗೆ ಹೋದರೆ ಸಾಕು; ಶಾಂತಿ ಲಭಿಸುವುದು. ಗುಡಿ ಮತ್ತು ಚರ್ಚುಗಳ ಹಿಂದಿರುವ ಭಾವನೆಯೇ ಇದು. ಕೆಲವು ಗುಡಿ ಮತ್ತು ಚರ್ಚುಗಳಲ್ಲಿ ಇದನ್ನು ಈಗಲೂ ಕಾಣುವಿರಿ. ಆದರೆ ಉಳಿದ ಬಹುಭಾಗದಲ್ಲಿ ಈ ಭಾವನೆ ಮಾಯವಾಗಿರುವುದು. ಇದರ ಅರ್ಥವೇನೆಂದರೆ ಪವಿತ್ರವಾದ ಆಲೋಚನಾ ತರಂಗವನ್ನು ಅಲ್ಲಿ ಹರಿಸುವುದರಿಂದ ಅದು ಪವಿತ್ರವಾಗುವುದು ಮತ್ತು ಆ ಪವಿತ್ರತೆ ಅಲ್ಲಿಯೇ ನೆಲೆಸಿರುವುದು. ಅದಕ್ಕೋಸ್ಕರವಾಗಿಯೇ ಯಾರಿಗೆ ಒಂದು ಪ್ರತ್ಯೇಕವಾಗಿ ಕೋಣೆಯನ್ನು ಕಾದಿರಿಸಲು ಸಾಧ್ಯವಿಲ್ಲವೋ ಅವರು ತಮಗೆ ಅನುಕೂಲವಾದೆಡೆಯಲ್ಲಿ ಅಭ್ಯಾಸ ಮಾಡಬಹುದು. ನೇರವಾಗಿ ಕುಳಿತುಕೊಳ್ಳಿ. ನೀವು ಮಾಡಬೇಕಾದ ಮೊದಲನೇ ಕೆಲಸವೇ ಸೃಷ್ಟಿಯ ಹಿತಕ್ಕಾಗಿ ಪವಿತ್ರ ಆಲೋಚನಾ ತರಂಗವನ್ನು ಕಳುಹಿಸುವುದು: “ಎಲ್ಲಾ ಜೀವಿಗಳೂ ಸುಖವಾಗಿರಲಿ; ಎಲ್ಲಾ ಜೀವಿಗಳೂ ಶಾಂತವಾಗಿರಲಿ; ಎಲ್ಲಾ ಜೀವಿಗಳೂ ಆನಂದವಾಗಿರಲಿ” ಎಂದು ಮನಸ್ಸಿನಲ್ಲಿ ಉಚ್ಚರಿಸಿ. ಹೀಗೆ ಪೂರ್ವ, ಪಶ್ಚಿಮ, ಉತ್ತರ ಮತ್ತು ದಕ್ಷಿಣ ದಿಕ್ಕಿನ ಕಡೆ ಮಾಡಿ. ನೀವು ಅದನ್ನು ಹೆಚ್ಚು ಮಾಡಿದಷ್ಟೂ ಅದರಿಂದ ನಿಮಗೆ ಅನುಕೂಲ ಕಂಡುಬರುವುದು. ನೀವು ಆರೋಗ್ಯವಾಗಿರಬೇಕಾದರೆ ಬಹಳ ಸುಲಭವಾದ ಮಾರ್ಗವೇ ಇತರರನ್ನು ಆರೋಗ್ಯದಲ್ಲಿರುವಂತೆ ಮಾಡುವುದು. ನಾವು ಸಂತೋಷವಾಗಿರಬೇಕಾದರೆ ಬಹಳ ಸುಲಭವಾದ ಮಾರ್ಗವೇ ಉಳಿದವರನ್ನು ಸಂತೋಷದಲ್ಲಿರುವಂತೆ ಮಾಡುವುದು. ಇದು ನಮಗೆ ಕೊನೆಗೆ ಗೊತ್ತಾಗುವುದು. ಇದನ್ನು ಮಾಡಿ ಆದ ಮೇಲೆ ಯಾರು ದೇವರನ್ನು ನಂಬುತ್ತಾರೋ ಅವರು ದೇವರನ್ನು ಪ್ರಾರ್ಥಿಸಬೇಕು–ಹಣಕ್ಕಾಗಿಯಾಗಲಿ, ಆರೋಗ್ಯಕ್ಕಾಗಿಯಾಗಲಿ, ಸ್ವರ್ಗಕ್ಕಾಗಿಯಾಗಲಿ ಅಲ್ಲ, ಜ್ಞಾನಕ್ಕಾಗಿ, ಅಂತರ್ಜ್ಯೋತಿಯ ವಿಕಾಸಕ್ಕಾಗಿ ಪ್ರಾರ್ಥಿಸಬೇಕು. ಉಳಿದ ಪ್ರಾರ್ಥನೆಗಳೆಲ್ಲ ಸ್ವಾರ್ಥತೆಯಿಂದ ಕೂಡಿವೆ. ಅನಂತರ ಮಾಡಬೇಕಾದ ಕೆಲಸವೇ ನಿಮ್ಮ ದೇಹದ ವಿಚಾರವಾಗಿ ಆಲೋಚಿಸುವುದು. ಅದನ್ನು ಆರೋಗ್ಯವಾಗಿ ದೃಢವಾಗಿಟ್ಟಿರಬೇಕು. ನಿಮಗೆ ಇರುವ ಅತ್ಯುತ್ತಮ ಯಂತ್ರವೇ ಇದು. ವಜ್ರದಂತೆ ಬಲವಾಗಿರುವುದು ಇದು, ಇದರ ಸಹಾಯದಿಂದ ಸಂಸಾರ ಸಾಗರವನ್ನು ದಾಟುತ್ತೇನೆ ಎಂದು ಭಾವಿಸಿ. ಬಲಹೀನರಿಗೆ ಎಂದಿಗೂ ಮುಕ್ತಿ ಲಭಿಸುವುದಿಲ್ಲ. ಎಲ್ಲಾ ನಿರ್ಬಲತೆಯನ್ನು ಕಿತ್ತೊಗೆಯಿರಿ. ದೇಹಕ್ಕೆ, ಅದು ದೃಢವಾಗಿರುವುದೆಂದು ಹೇಳಿ; ಮನಸ್ಸಿಗೆ ಅದು ಬಲವಾಗಿರುವುದೆಂದು ಹೇಳಿ. ಅಪರಿಮಿತ ಶ್ರದ್ಧೆ ಮತ್ತು ನಂಬಿಕೆ ನಿಮ್ಮಲ್ಲಿರಲಿ.

\chapter{ಪ್ರಾಣ}

ಪ್ರಾಣಾಯಾಮವೆಂದರೆ ಅನೇಕರು ಭಾವಿಸುವಂತೆ ಉಸಿರಿಗೆ ಸಂಬಂಧಪಟ್ಟಿದ್ದೇನೂ ಅಲ್ಲ. ಅದಕ್ಕೂ ಉಸಿರಿಗೂ ಸಂಬಂಧವಿರುವುದು ಎಲ್ಲಿಯೋ ಅಲ್ಪ ಸ್ವಲ್ಪ. ನಿಜವಾದ ಪ್ರಾಣಾಯಾಮಕ್ಕೆ ಪ್ರವೇಶಿಸಬೇಕಾದರೆ ಉಸಿರಾಡುವುದು ಅನೇಕ ಸಾಧನೆಗಳಲ್ಲಿ ಒಂದು ಪ್ರಾಣಾಯಾಮವೆಂದರೆ ಪ್ರಾಣದ ನಿಗ್ರಹ. ಭಾರತೀಯ ತತ್ತ್ವಶಾಸ್ತ್ರದ ಪ್ರಕಾರ ಪ್ರಪಂಚವೆಲ್ಲವೂ ಎರಡು ವಸ್ತುಗಳ ಸಂಯೋಗದಿಂದ ಆಗಿರುವುದು. ಅದರಲ್ಲಿ ಒಂದನ್ನು ಆಕಾಶವೆನ್ನುವರು. ಸರ್ವವ್ಯಾಪಿಯಾಗಿ ಸರ್ವಾಂತರಾಳದಲ್ಲಿಯೂ ಇರುವುದೇ ಇದು. ರೂಪವಿರುವ ಪ್ರತಿಯೊಂದೂ ಕೂಡ, ಸಂಯೋಗದ ಫಲವಾದ ಪ್ರತಿಯೊಂದೂ ಕೂಡ ಈ ಆಕಾಶದಿಂದ ಉತ್ಪತ್ತಿಯಾಗಿರುವುದು. ಆಕಾಶವೇ ಗಾಳಿಯಾಗುವುದು, ದ್ರವವಾಗುವುದು, ಘನವಾಗುವುದು; ಆಕಾಶವೇ ಸೂರ್ಯನಾಗುವುದು, ಪೃಥ್ವಿಯಾಗುವುದು, ಚಂದ್ರನಾಗುವುದು, ನಕ್ಷತ್ರಗಳಾಗುವುದು, ಧೂಮಕೇತುವಾಗುವುದು; ಆಕಾಶವೇ ಮಾನವ ದೇಹವಾಗುವುದು, ಪ್ರಾಣಿಗಳ ದೇಹವಾಗುವುದು, ಸಸ್ಯವಾಗುವುದು. ನಾವು ನೋಡುವ ಪ್ರತಿಯೊಂದು ಆಕಾರವೂ, ನಾವು ಅನುಭವಿಸುವ ಪ್ರತಿಯೊಂದು ವಿಷಯವೂ, ಪ್ರತಿಯೊಂದು ವಸ್ತುವೂ ಕೂಡ ಆಕಾಶವೇ. ಅದನ್ನು ನಾವು ನೋಡಲಾಗುವುದಿಲ್ಲ. ಅದು ಬಹಳ ಸೂಕ್ಷ್ಮವಾಗಿರುವುದರಿಂದ ಸಾಧಾರಣ ಇಂದ್ರಿಯ ಗೋಚರಕ್ಕೆ ಮೀರಿರುವುದು. ಅದು ಸ್ಥೂಲವಾಗಿ ಒಂದು ಆಕಾರವನ್ನು ಪಡೆದಾಗ ಮಾತ್ರ, ನಾವು ಅದನ್ನು ನೋಡಬಲ್ಲೆವು. ಸೃಷ್ಟಿಯ ಆದಿಯಲ್ಲಿ ಈ ಆಕಾಶವೊಂದೇ ಇರುವುದು. ಕಲ್ಪದ ಕೊನೆಯಲ್ಲಿ ಘನ, ದ್ರವ ಮತ್ತು ಅನಿಲ ರೂಪದಲ್ಲಿರುವುದೆಲ್ಲ ಪುನಃ ಆಕಾಶದಲ್ಲಿ ಐಕ್ಯವಾಗುವುವು. ಮತ್ತೊಂದು ಸೃಷ್ಟಿಯಲ್ಲಿ ಅವು ಆಕಾಶದಿಂದ ಹೊರಬರುವುವು. 

\vskip 0.2cm

ಯಾವ ಶಕ್ತಿಯಿಂದ ಈ ಆಕಾಶವು ಈ ಜಗದ್ರೂಪದಲ್ಲಿ ವ್ಯಕ್ತವಾಗುತ್ತದೆ? ಪ್ರಾಣಶಕ್ತಿಯ ಮೂಲಕ. ಹೇಗೆ ಆಕಾಶವು ಈ ಜಗತ್ತಿನ ಅನಂತವೂ ಸರ್ವ ವ್ಯಾಪಿಯೂ ಆದ ಮೂಲವಸ್ತುವಾಗಿರುವುದೊ, ಹಾಗೆಯೇ ಈ ಪ್ರಾಣವು ವಿಶ್ವಸೃಷ್ಟಿಯ ಹಿಂದಿರುವ ಅನಂತಶಕ್ತಿ. ಪ್ರತಿ ಕಲ್ಪದ ಆದಿ ಮತ್ತು ಅಂತ್ಯದಲ್ಲಿ ವಸ್ತುಗಳೆಲ್ಲ ಆಕಾಶವಾಗುವುವು; ಪ್ರಪಂಚದಲ್ಲಿರುವ ಶಕ್ತಿಯೆಲ್ಲ ಪ್ರಾಣದಲ್ಲಿ ಐಕ್ಯವಾಗುವುದು. ಮುಂದಿನ ಕಲ್ಪದಲ್ಲಿ ನಾವು ಶಕ್ತಿ ಎಂದು ಕರೆಯುವ ಪ್ರತಿಯೊಂದೂ ಈ ಪ್ರಾಣದಿಂದ ಜನಿಸುವುದು. ಚಲನೆಯಂತೆ ಅಭಿವ್ಯಕ್ತಿಗೊಳ್ಳುವುದೇ ಪ್ರಾಣ. ಪ್ರಾಣವೇ ಆಕರ್ಷಣ ಶಕ್ತಿಯಂತೆ ಇರುವುದು. ಅದೇ ಅಯಸ್​ಕಾಂತ ಶಕ್ತಿ. ಶಾರೀರಿಕ ಕ್ರಿಯೆಯಂತೆ ಅಭಿವ್ಯಕ್ತಿಗೊಳ್ಳುತ್ತಿರುವುದೂ ಅದೇ. ನರಗಳ ಶಕ್ತಿ ತರಂಗದಂತೆ ತೋರುವುದೂ ಅದೇ. ಆಲೋಚನಾ ಶಕ್ತಿಯಂತೆ ವ್ಯಕ್ತವಾಗುತ್ತಿರುವುದೂ ಅದೇ. ಆಲೋಚನೆಯ ಮಟ್ಟದಿಂದ ಹಿಡಿದು, ಕೆಳಗಿರುವ ಸ್ಥೂಲವಾದ ಜಡಶಕ್ತಿಯವರೆಗೆ ಇರುವ ಎಲ್ಲವೂ ಪ್ರಾಣದ ಅಭಿವ್ಯಕ್ತಿ. ಮಾನಸಿಕ ಮತ್ತು ಭೌತಿಕ ಕ್ಷೇತ್ರದ ವಿಧವಿಧವಾದ ಶಕ್ತಿಸಮೂಲವೆಲ್ಲದರ ಮೂಲರೂಪಕ್ಕೆ ಪ್ರಾಣವೆಂದು ಹೆಸರು. “ಇದೆಯೆಂತಲೂ ಅಲ್ಲ. ಇಲ್ಲವೆಂತಲೂ ಅಲ್ಲ. ಆ ಸ್ಥಿತಿಯಲ್ಲಿ ಅಂಧಕಾರದಿಂದ ಅಂಧಕಾರವು ಆವರಿಸಲ್ಪಟ್ಟಿತ್ತು. ಆಗ ಇದ್ದುದೇನು? ಆಕಾಶವು ಚಲನೆ ಇಲ್ಲದೆ ಇತ್ತು.” ಪ್ರಾಣದ ಸ್ಥೂಲ ಚಲನೆ ಇರಲಿಲ್ಲ, ಆದರೂ ಪ್ರಾಣವಿತ್ತು. ಈಗ ಗೋಚರಿಸುತ್ತಿರುವ ಶಕ್ತಿಯೆಲ್ಲ ಕಲ್ಪಾಂತದಲ್ಲಿ ಶಾಂತವಾಗಿ ಅವ್ಯಕ್ತವಾಗುವುದು. ಮುಂದಿನ ಸೃಷ್ಟಿಯ ಆದಿಯಲ್ಲಿ ಅವು ಪುನಃ ಜಾಗೃತವಾಗುವುವು. ಆಕಾಶದ ಮೇಲೆ ತಮ್ಮ ಪ್ರಭಾವವನ್ನು ಬೀರುವುವು. ಆಗ ವೈವಿಧ್ಯಗಳನ್ನು ಒಳಗೊಂಡ ಜಗತ್ತು ಆಕಾಶದಿಂದ ವಿಕಾಸವಾಗುವುದು. ಆಕಾಶವು ಬದಲಾಯಿಸಿದಂತೆ ಪ್ರಾಣವೂ ಕೂಡ ಭಿನ್ನಭಿನ್ನ ಶಕ್ತಿಯ ರೂಪವನ್ನು ಧರಿಸುವುದು. ಈ ಪ್ರಾಣದ ಜ್ಞಾನ ಮತ್ತು ಅದರ ನಿಗ್ರಹವನ್ನೇ ಪ್ರಾಣಾಯಾಮವೆನ್ನುವುದು. 

\vskip 0.2cm

ಅನಂತಶಕ್ತಿಯ ಆಗರಕ್ಕೆ ಇದು ನಮ್ಮನ್ನು ಒಯ್ಯುತ್ತದೆ. ಉದಾಹರಣೆಗೆ, ಒಬ್ಬನು ಪ್ರಾಣವೆಂದರೇನೆಂಬುದನ್ನು ಚೆನ್ನಾಗಿ ತಿಳಿದುಕೊಂಡು, ಅದನ್ನು ನಿಗ್ರಹಿಸುವುದನ್ನು ಕಲಿತರೆ ಪ್ರಪಂಚದಲ್ಲಿರುವ ಯಾವ ಶಕ್ತಿ ಅವನ ಕರಗತವಾಗಲಾರದು? ಸೂರ್ಯ ಮತ್ತು ನಕ್ಷತ್ರಗಳನ್ನು ಕೂಡ ಅವುಗಳ ಸ್ಥಾನದಿಂದ ಕದಲಿಸಬಲ್ಲ. ಸಣ್ಣ ಕಣದಿಂದ ಹಿಡಿದು ಅತ್ಯದ್ಭುತವಾದ ಸೂರ್ಯನವರೆಗೆ ಪ್ರಪಂಚದಲ್ಲಿರುವ ಎಲ್ಲಾ ವಸ್ತುಗಳನ್ನೂ ತನ್ನ ಸ್ವಾಧೀನಕ್ಕೆ ತರಬಲ್ಲ. ಏಕೆಂದರೆ ಪ್ರಾಣ ಅವನ ವಶವಾಗಿದೆ. ಇದೇ ಪ್ರಾಣಾಯಾಮದ ಕೊನೆ ಮತ್ತು ಗುರಿ. ಯೋಗಿಯು ಸಿದ್ಧನಾದ ಮೇಲೆ ತನ್ನ ಸ್ವಾಧೀನಕ್ಕೆ ಒಳಪಡದೆ ಇರುವುದು ಪ್ರಕೃತಿಯಲ್ಲಿ ಯಾವುದೂ ಇರುವುದಿಲ್ಲ. ದೇವತೆಗಳನ್ನು, ಗತಿಸಿದವರ ಆತ್ಮಗಳನ್ನು ಬರುವಂತೆ ಅವನು ಅಪ್ಪಣೆ ಮಾಡಿದರೆ ಅವರು ಬರುವರು. ಪ್ರಕೃತಿಯ ಶಕ್ತಿಗಳೆಲ್ಲ ಗುಲಾಮರಂತೆ ಅವನ ಅಣತಿಯನ್ನು ಪಾಲಿಸುವುವು. ಅಜ್ಞಾನಿಯು ಈ ಶಕ್ತಿಯನ್ನು ಯೋಗಿಯಲ್ಲಿ ನೋಡಿದಾಗ ಇದನ್ನು ಪವಾಡವೆಂದು ಕರೆಯುವನು. ಹಿಂದೂವಿನಲ್ಲಿರುವ ಒಂದು ವಿಶೇಷವಾದ ಗುಣವೇನೆಂದರೆ ಸಾಧ್ಯವಾದ ಮಟ್ಟಿಗೆ ಕಟ್ಟಕಡೆಯ ಸರ್ವಸಾಮಾನ್ಯ ತತ್ತ್ವವನ್ನು ಹುಡುಕಿ ಅದರ ವಿಶದ ವಿವರಣೆಯನ್ನು ಮುಂದಿನವರಿಗೆ ಬಿಟ್ಟುಬಿಡುವುದು. “ಯಾವುದನ್ನು ತಿಳಿದರೆ ನಾವು ಎಲ್ಲವನ್ನೂ ತಿಳಿಯಬಹುದೋ ಅದು ಯಾವುದು?” ಎಂಬ ಪ್ರಶ್ನೆ ವೇದದಲ್ಲಿ ಬರುವುದು. ಆದಕಾರಣ ಶಾಸ್ತ್ರಗಳನ್ನೆಲ್ಲ, ತತ್ತ್ವಗಳನ್ನೆಲ್ಲ ಏತಕ್ಕೆ ಬರೆದಿರುವುದೆಂದರೆ, ಯಾವುದನ್ನು ತಿಳಿದರೆ ಎಲ್ಲವನ್ನೂ ತಿಳಿಯುವೆವೋ ಅದನ್ನು ನಿರ್ದೇಶಿಸುವುದಕ್ಕೆ. ಚೂರು ಚೂರಾಗಿ, ಪ್ರತ್ಯೇಕವಾಗಿ, ಈ ಪ್ರಪಂಚವನ್ನು ತಿಳಿಯಬೇಕೆಂದು ಮನುಷ್ಯನು ಆಶಿಸಿದರೆ, ಪ್ರತಿಯೊಂದು ಕಣವನ್ನೂ ಬೇರೆಬೇರೆಯಾಗಿ ಪರೀಕ್ಷಿಸಬೇಕಾಗುವುದು. ಅದಕ್ಕೆ ಅನಂತಕಾಲ ಬೇಕಾಗುವುದರಿಂದ ಅವನು ಎಲ್ಲವನ್ನೂ ತಿಳಿಯಲಾರ. ಹಾಗಾದರೆ ನಮಗೆ ಜ್ಞಾನ ಹೇಗೆ ಸಿಕ್ಕುವುದು? ಪ್ರತ್ಯೇಕ ವಸ್ತುಗಳ ಮೂಲಕ ಎಲ್ಲವನ್ನೂ ತಿಳಿದುಕೊಳ್ಳುವುದು ಮನುಷ್ಯನಿಗೆ ಹೇಗೆ ಸಾಧ್ಯ? ಪ್ರತ್ಯೇಕ ವಸ್ತುಗಳ ಅಭಿವ್ಯಕ್ತಿಯ ಹಿಂದೆ ಒಂದು ಸರ್ವಸಾಮಾನ್ಯ ತತ್ತ್ವವಿದೆ ಎಂದು ಯೋಗಿಯು ಹೇಳುತ್ತಾನೆ, ಭಿನ್ನಭಿನ್ನವಾದ ಪ್ರತ್ಯೇಕ ವಸ್ತುಗಳ ಅಂತರಾಳದಲ್ಲಿ, ಸರ್ವಸಾಮಾನ್ಯವಾದ, ಅವ್ಯಕ್ತವಾದ, ಒಂದು ಮೂಲತತ್ತ್ವವಿದೆ. ಅದನ್ನು ನೀವು ತಿಳಿದರೆ ಎಲ್ಲವನ್ನೂ ತಿಳಿದಂತೆ. ವೇದದಲ್ಲಿ, ಈ ಪ್ರಪಂಚವನ್ನೆಲ್ಲ, ಒಂದು ಅನಂತ ಅದ್ವಿತೀಯ ಅಸ್ತಿತ್ವದ ಮೂಲತತ್ತ್ವಕ್ಕೆ ಇಳಿಸಿರುವರು. ಯಾರು ಆ ಮೂಲತತ್ತ್ವದ ಅಸ್ತಿತ್ವವನ್ನು ತಿಳಿದುಕೊಂಡಿರುವರೋ, ಅವರು ಪ್ರಪಂಚವನ್ನು ತಿಳಿದುಕೊಂಡಿರುವರು. ಅದರಂತೆಯೇ ಯೋಗಿಯು ಎಲ್ಲಾ ಶಕ್ತಿಗಳನ್ನೂ ಪ್ರಾಣವೆಂಬ ಒಂದು ಸಾಮಾನ್ಯ ಶಕ್ತಿಗೆ ಇಳಿಸಿರುವರು. ಯಾರು ಈ ಪ್ರಾಣವೇನೆಂಬುದನ್ನು ಚೆನ್ನಾಗಿ ತಿಳಿದುಕೊಂಡಿರುವರೋ, ಅವರು ಪ್ರಪಂಚದಲ್ಲಿರುವ ಮಾನಸಿಕ ಮತ್ತು ಭೌತಿಕ ಶಕ್ತಿಗಳನ್ನೆಲ್ಲ ತಿಳಿದುಕೊಂಡಿರುವರು. ಯಾರು ಪ್ರಾಣವನ್ನು ನಿಗ್ರಹಿಸಿರುವರೋ, ಅವರು ತಮ್ಮ ದೇಹವನ್ನು ಮತ್ತು ಇರುವ ಎಲ್ಲಾ ದೇಹಗಳನ್ನು ಸ್ವಾಧೀನಪಡಿಸಿಕೊಂಡಿರುವರು; ಏಕೆಂದರೆ ಎಲ್ಲ ಶಕ್ತಿಗಳ ಮೂಲವೇ ಪ್ರಾಣ. 

\vskip 0.2cm

ಪ್ರಾಣವನ್ನು ಹೇಗೆ ನಿಗ್ರಹಿಸಬೇಕೆಂಬುದೊಂದೇ ಪ್ರಾಣಾಯಾಮದ ಗುರಿ. ಇದಕ್ಕೆ ಸಂಬಂಧಪಟ್ಟ ಅಭ್ಯಾಸ, ಅಂಗಸಾಧನೆ ಇರುವುದೆಲ್ಲ ಈ ಒಂದನ್ನು ಮುಟ್ಟಲು. ಪ್ರತಿಯೊಬ್ಬನೂ ಕೂಡ ತಾನು ಇರುವ ಎಡೆಯಿಂದ ಮೊದಲು ಮಾಡಬೇಕು. ತನ್ನ ಸಮೀಪದಲ್ಲಿರುವ ವಸ್ತುವನ್ನು ಹೇಗೆ ಸ್ವಾಧೀನಕ್ಕೆ ತರಬೇಕೆಂಬುದನ್ನು ತಿಳಿದುಕೊಳ್ಳಬೇಕು. ಬಾಹ್ಯ ಪ್ರಪಂಚದಲ್ಲಿರುವ ಎಲ್ಲಾ ವಸ್ತುಗಳಿಗಿಂತಲೂ ಬಹಳ ಹತ್ತಿರದಲ್ಲಿರುವುದು ಈ ದೇಹ. ಅದಕ್ಕಿಂತ ಹತ್ತಿರದಲ್ಲಿರುವುದೇ ಮನಸ್ಸು. ಈ ದೇಹ ಮತ್ತು ಮನಸ್ಸುಗಳನ್ನು ಕೆಲಸ ಮಾಡುವಂತೆ ಪ್ರೇರೇಪಿಸುತ್ತಿರುವ ಪ್ರಾಣವೇ ಉಳಿದ ಎಲ್ಲಾ ಪ್ರಾಣಕ್ಕಿಂತಲೂ ಬಹಳ ಹತ್ತಿರದಲ್ಲಿರುವುದು. ನಮ್ಮೊಳಗೆ ಕೆಲಸ ಮಾಡುತ್ತಿರುವ ಆ ಕಿರಿಯ ಶಕ್ತಿತರಂಗವನ್ನು ನಾವು ನಿಗ್ರಹಿಸುವುದರಲ್ಲಿ ಜಯಶೀಲರಾದರೆ ಆಗ ನಾವು ಪ್ರಾಣವನ್ನೆಲ್ಲ ನಿಗ್ರಹಿಸಬಹುದೆಂದು ಆಶಿಸಬಹುದು. ಯಾವ ಯೋಗಿಯು ಇದನ್ನು ಮಾಡಿರುವನೋ ಅವನ್ನು ಪೂರ್ಣತೆಯನ್ನು ಪಡೆಯುತ್ತಾನೆ, ಅವನು ಇನ್ನು ಮತ್ತಾವ ಶಕ್ತಿಯ ಅಧೀನಕ್ಕೊ ಒಳಪಡುವುದಿಲ್ಲ. ಅವನನ್ನು ಬಹುಮಟ್ಟಿಗೆ ಸರ್ವಶಕ್ತ, ಸರ್ವಜ್ಞ ಎಂದೇ ಪರಿಗಣಿಸಬಹುದು. ಪ್ರಾಣವನ್ನು ನಿಗ್ರಹಿಸುವುದಕ್ಕೆ ಪ್ರಯತ್ನಪಟ್ಟ ಅನೇಕ ಪಂಗಡಗಳನ್ನು ನಾವು ಎಲ್ಲಾ ದೇಶದಲ್ಲಿಯೂ ನೋಡುವೆವು. ಈ ದೇಶದಲ್ಲಿ (ಅಮೇರಿಕಾದಲ್ಲಿ) ಮಾನಸಿಕ ಚಿಕಿತ್ಸಕರು \enginline{(Mind healers)}, ಶ್ರದ್ಧಾ ಚಿಕಿತ್ಸಕರು \enginline{(Faith healers)}, ಭೂತವಾದಿಗಳು \enginline{(Spirtualists)}, ಕ್ರೈಸ್ತವಿಜ್ಞಾನಿಗಳು \enginline{(Christian Scientists)}, ಸುಪ್ತಿ ಆವಾಹಕರು \enginline{(Hypnotists)} ಮುಂತಾದವರು ಇರುವರು. ಈ ಬೇರೆ ಬೇರೆ ಪಂಗಡಗಳನ್ನು ನಾವು ಪರೀಕ್ಷಿಸಿದರೆ, ಪ್ರತಿಯೊಂದರ ಹಿಂದೆಯೂ ಕೂಡ, ಅವರಿಗೆ ಗೊತ್ತಿರಲಿ ಇಲ್ಲದೇ ಇರಲಿ, ಪ್ರಾಣದ ನಿಗ್ರಹವಿದೆ. ಅವರ ಸಿದ್ಧಾಂತಗಳನ್ನೆಲ್ಲಾ ಭಟ್ಟಿ ಇಳಿಸಿದರೆ ಕೊನೆಗೆ ಉಳಿಯುವುದೇ ಇದು. ಈ ಒಂದು ಶಕ್ತಿಯನ್ನೇ ಅವರು ಉಪಯೋಗಿಸುತ್ತಿರುವರು. ಆದರೆ ಅವರಿಗೆ ಗೊತ್ತಿಲ್ಲ. ಅವರು ಕಾಣದೆ ಎಡವಿರುವುದು ಇದೇ ಶಕ್ತಿಯನ್ನು. ಅದರ ಸ್ವಭಾವವನ್ನು ತಿಳಿಯದೆ ಅವರು ಅದನ್ನು ಉಪ ಯೋಗಿಸುತ್ತಿರುವರು. ಯೋಗಿಯು ಕೂಡ ಇದೇ ಶಕ್ತಿಯನ್ನೇ ಉಪಯೋಗಿಸುತ್ತಿರುವನು. ಈ ಶಕ್ತಿ ಬರುವುದು ಪ್ರಾಣದಿಂದ. 

\vskip 0.2cm

ಪ್ರತಿಯೊಬ್ಬನಲ್ಲಿಯೂ ಪ್ರಾಣವು ಅತಿ ಮುಖ್ಯವಾದ ಶಕ್ತಿ. ಆಲೋಚನೆ ಪ್ರಾಣದ ಅತಿ ಸೂಕ್ಷ್ಮವಾದ ಮತ್ತು ಅತಿ ಮುಖ್ಯವಾದ ಕ್ರಿಯೆ. ನಮಗೆ ಕಾಣುವಂತೆ ಆಲೋಚನೆಯೆ ಸರ್ವವೂ ಆಗಿಲ್ಲ. ಬಹಳ ಕೆಳಗಿನ ಹಂತದಲ್ಲಿ ಕೆಲಸ ಮಾಡುವ ಹುಟ್ಟುಗುಣ \enginline{(Instinct)} ಅಥವಾ ಅಪ್ರಜ್ಞೆಯ ಮಾನಸಿಕ ಕ್ರಿಯೆಗಳಿವೆ. \enginline{(Unconscious thought)}. ಒಂದು ಸೊಳ್ಳೆ ಕಚ್ಚಿದರೆ ನಮ್ಮ ಕೈಗಳು ನಮ್ಮ ಅರಿವಿಲ್ಲದೆ ಸ್ವಭಾವತಃ ಅದನ್ನು ಅಟ್ಟುತ್ತವೆ. ಆಲೋಚನೆಯ ಒಂದು ವಿಧದ ಪ್ರಕಾಶವಿದು. ನಮ್ಮ ಅರಿವಿಲ್ಲದೆ ದೇಹದ ಮೂಲಕ ಆಗುವ ಕೆಲಸವೆಲ್ಲ ಈ ಆಲೋಚನಾ ಕ್ಷೇತ್ರಕ್ಕೆ ಸಂಬಂಧಪಟ್ಟಿರುವುದು. ಅನಂತರ ನಾವು ಪ್ರಜ್ಞಾಪೂರ್ವಕ ಮಾಡುವ ಆಲೋಚನೆಗಳಿವೆ. ನಾನು ಬೌದ್ಧಿಕವಾಗಿ ವಿಚಾರ ಮಾಡುತ್ತೇನೆ, ತೀರ್ಪು ಕೊಡುತ್ತೇನೆ. ಆಲೋಚನೆ ಮಾಡುತ್ತೇನೆ; ಒಂದು ವಸ್ತುವನ್ನು ಹಲವು ವಿಧದಿಂದ ಪರೀಕ್ಷಿಸುತ್ತೇನೆ. ಇಷ್ಟೇ ಇಲ್ಲದೆ, ಯುಕ್ತಿಗೆ ಒಂದು ಮೇರೆ ಇದೆ ಎಂಬುದೂ ನಮಗೆ ಗೊತ್ತಿದೆ. ಸ್ವಲ್ಪ ದೂರ ಮಾತ್ರ ಯುಕ್ತಿ ಹೋಗಬಲ್ಲುದು. ಅದರಾಚೆ ಹೋಗಲಾರದು. ಅದು ಕೆಲಸ ಮಾಡುವ ಕ್ಷೇತ್ರ ಕಿರಿದು. ಆದರೆ ಅದೇ ಕಾಲದಲ್ಲಿ ಈ ಸಣ್ಣ ವೃತ್ತದೊಳಗೆ ಹೊರಗಿನಿಂದ ಹಲವು ವಸ್ತುಗಳು ಬರುವುದು ನಮಗೆ ಕಂಡುಬರುತ್ತದೆ. ಪೃಥ್ವಿಯ ಪರಿಧಿಯೊಳಗೆ ಪ್ರವೇಶಿಸುವ ಧೂಮಕೇತುಗಳಂತೆ ಕೆಲವು ವಿಷಯಗಳು ಈ ಯುಕ್ತಿಯ ವೃತ್ತದೊಳಗೆ ನುಗ್ಗುವುವು. ನಮ್ಮ ಯುಕ್ತಿಯು ತನ್ನ ಮೇರೆಯನ್ನು ಮೀರಿ ಹೋಗಲಾಗದ್ದಿದರೂ, ಇವುಗಳು ನಮ್ಮ ಎಲ್ಲೆಯಾಚೆಯಿಂದ ಬರುವುದೆಂಬುದೇನೋ ನಿಜ. ಈ ಯುಕ್ತಿಯೆಂಬ ಸಣ್ಣ ವೃತ್ತದೊಳಗೆ ನುಗ್ಗುವ ವಿಷಯಗಳ ಮೂಲ ಕಾರಣವು ಈ ವೃತ್ತದ ಆಚೆ ಇರುವುದು. ಮನಸ್ಸು ಇದನ್ನೂ ಮೀರಿದ ಎತ್ತರದಲ್ಲಿ, ಎಂದರೆ ಪ್ರಜ್ಞಾತೀತ ಅವಸ್ಥೆಯಲ್ಲಿಯೂ \enginline{(Superconscious)} ಇರುವುದು. ಸಮಾಧಿ, ಪೂರ್ಣ ಏಕಾಗ್ರತೆ, ಅಥವಾ ಪ್ರಜ್ಞಾತೀತ ಎಂದು ಕರೆಯಲ್ಪಡುವ ಅವಸ್ಥೆಗೆ ಮನಸ್ಸು ಏರಿದಾಗ ಅದು ಯುಕ್ತಿಯನ್ನು ಮೀರಿಹೋಗುವುದು. ಆಗ ನಮ್ಮ ಸ್ವಾಭಾವಿಕ ಗುಣಕ್ಕಾಗಲೀ, ಯುಕ್ತಿಗಾಗಲೀ ಎಂದೂ ತೋರದ ಕೆಲವು ವಿಷಯಗಳು ಗೋಚರಿಸುವುವು. ದೇಹದಲ್ಲಿ ಕೆಲಸ ಮಾಡುವ ಸೂಕ್ಷ್ಮ ಶಕ್ತಿಗಳನ್ನು ಸ್ವಾಧೀನಕ್ಕೆ ತರುವುದನ್ನು ನಾವು ಕಲಿತರೆ, ಅದು ನಮ್ಮ ಮನಸ್ಸನ್ನು ಮುಂದಕ್ಕೆ ನೂಕಿ, ಇನ್ನೂ ಮೇಲಕ್ಕೆ ಹೋಗುವಂತೆ ಮಾಡುವುದು. ಮನಸ್ಸು ಆಗ ಪ್ರಜ್ಞಾತೀತ ಸ್ಥಿತಿಗೆ ಏರಿ, ಅಲ್ಲಿಂದ ಇದು ತನ್ನ ಕಾರ್ಯವನ್ನು ಮಾಡುವುದು. 

\vskip 0.2cm

ಪ್ರತಿಯೊಂದು ಅಸ್ತಿತ್ವದ ಹಂತದಲ್ಲಿಯೂ ಕೂಡ ಈ ಪ್ರಪಂಚದಲ್ಲಿರುವುದು ಒಂದೇ ಅವಿಚ್ಛಿನ್ನವಾದ ವಸ್ತು. ಭೌತಿಕವಾಗಿ ಪ್ರಪಂಚ ಒಂದು. ನಿಮಗೂ ಸೂರ್ಯನಿಗೂ ಏನೂ ವ್ಯತ್ಯಾಸವಿಲ್ಲ. ಇದನ್ನು ವಿರೋಧಿಸುವುದು ಒಂದು ಹುರುಳಿಲ್ಲದ ಭ್ರಾಂತಿ ಎನ್ನುವನು ವಿಜ್ಞಾನಿ. ಮೇಜಿಗೂ ನನಗೂ ಯಾವ ವಿಧವಾದ ವ್ಯತ್ಯಾಸವೂ ಇಲ್ಲ. ವಸ್ತುರಾಶಿಯಲ್ಲಿ ಮೇಜು ಒಂದು ಭಾಗ. ಅನಂತ ದ್ರವ್ಯ ರಾಶಿಯಲ್ಲಿ ಪ್ರತಿಯೊಂದು ಭಾಗವೂ ಕೂಡ ಒಂದೊಂದು ಸುಳಿಯಂತಿದೆ. ಅದರಲ್ಲಿ ಯಾವುದೂ ಸ್ಥಿರವಲ್ಲ. ಒಂದು ಹರಿಯುತ್ತಿರುವ ಪ್ರವಾಹದಲ್ಲಿ ಲಕ್ಷಾಂತರ ಸುಳಿಗಳಿವೆ. ಅವುಗಳಲ್ಲಿ ಪ್ರತಿಕ್ಷಣವೂ ಕೂಡ ಹೊಸ ನೀರು ಬರುವುದು, ಅದು ಸುಳಿಯಲ್ಲಿ ಕೆಲವು ಕ್ಷಣ ಸುತ್ತಾಡಿ, ಪುನಃ ಹೊಸ ನೀರಿಗೆ ಅವಕಾಶ ಕೊಟ್ಟು ಪ್ರವಾಹದಲ್ಲಿ ಬೆರೆತು ಹೋಗುವುದು. ಅದರಂತೆಯೇ ಇಡೀ ಜಗತ್ತು ಸದಾ ಬದಲಾಯಿಸುತ್ತಿರುವ ವಸ್ತುವಿನ ಪ್ರವಾಹ. ಅವುಗಳಲ್ಲಿ ಹಲವು ಆಕಾರಗಳೆಲ್ಲ ಅಷ್ಟು ಸುಳಿಗಳು. ಕೆಲವು ದ್ರವ್ಯರಾಶಿಯು ಮಾನವ ದೇಹವೆಂಬ ಸುಳಿಗೆ ಬರುತ್ತದೆ. ಅಲ್ಲಿ ಕೆಲವು ಕಾಲವಿದ್ದು ಬದಲಾವಣೆ ಹೊಂದಿ, ಅಲ್ಲಿಂದ ಪ್ರಾಣಿಯ ದೇಹಕ್ಕೆ ಹೋಗುವುದು. ಅಲ್ಲಿಂದ ಕೆಲವು ಕಾಲವಾದ ಮೇಲೆ ಖನಿಜವೆಂಬ ಮತ್ತೊಂದು ಸುಳಿಗೆ ಹೋಗುವುದು. ಅನವರತವೂ ಇದು ಬದಲಾಯಿಸುತ್ತಿರುವುದು. ಯಾವ ಒಂದು ವಸ್ತುವು ಕೂಡ ಒಂದೇ ಸಮನಾಗಿರುವುದಿಲ್ಲ. ನನ್ನ ದೇಹ, ನಿಮ್ಮ ದೇಹ ಎಂಬ ಭೇದವಿರುವುದು ಮಾತಿನಲ್ಲಿ ಮಾತ್ರ ವಿರಾಟ್​ ದ್ರವ್ಯರಾಶಿಯಲ್ಲಿ ಚಂದ್ರ, ಸೂರ್ಯ, ಮನುಷ್ಯ, ಭೂಮಿ, ಸಸ್ಯ, ಖನಿಜ ಇವುಗಳೆಲ್ಲ ಒಂದೊಂದು ಭಾಗ. ಯಾವುದೂ ಕೂಡ ಸ್ಥಿರವಾಗಿಲ್ಲ. ಎಲ್ಲವೂ ಚಲಿಸುತ್ತಿವೆ. ವಸ್ತು ಯಾವಾಗಲೂ ಮತ್ತೊಂದರೊಂದಿಗೆ ಬೆರೆಯುವುದು ಮತ್ತು ಅದರಿಂದ ಬೇರೆಯಾಗುವುದು. 

\vskip 0.2cm

ಮನಸ್ಸು ಕೂಡ ಅದರಂತೆಯೇ. ಆಕಾಶವೇ ವಸ್ತುವಿನ ಮೂಲಸ್ಥಿತಿ. ಪ್ರಾಣದ ಕ್ರಿಯೆಯು ಬಹಳ ಸೂಕ್ಷ್ಮವಾಗಿರುವಾಗ, ಸೂಕ್ಷ್ಮ ಸ್ಪಂದನದಲ್ಲಿರುವ ಈ ಆಕಾಶವನ್ನು ಮನಸ್ಸೆಂದು ಕರೆಯುವೆವು. ಅಲ್ಲಿಯೂ ಅದು ದ್ರವ್ಯರಾಶಿಯ ಒಂದು ಅವಿಭಾಜ್ಯ ಅಂಗವಾಗಿರುತ್ತದೆ. ಈ ಸೂಕ್ಷ್ಮ ಸ್ಪಂದನದ ಸಂಪರ್ಕವನ್ನು ನೀವು ಪಡೆದರೆ ಈ ವಿಶ್ವವೆಲ್ಲವು ಕೂಡ ಈ ಸೂಕ್ಷ್ಮ ಸ್ಪಂದನದಿಂದ ಆಗಿದೆ ಎಂಬುದು ಗೊತ್ತಾಗುತ್ತದೆ. ಕೆಲವೊಮ್ಮೆ ನಾವು ಇನ್ನೂ ಇಂದ್ರಿಯದ ಮೇರೆಯಲ್ಲಿ ಇರುವಾಗಲೇ ನಮ್ಮನ್ನು ಕೆಲವು ಔಷಧಿಗಳು ಆ ಸ್ಥಿತಿಗೆ ಕರೆದೊಯ್ಯಬಲ್ಲವು. ನಗುವಿನ ಅನಿಲಕ್ಕೆ \enginline{(Laughing gas)} ಪರವಶನಾದ ಸರ್​ ಹಂಫ್ರಿಡೇವಿಯ ಪ್ರಖ್ಯಾತವಾದ ಶೋಧನೆ ನಿಮ್ಮಲ್ಲಿ ಅನೇಕರಿಗೆ ಜ್ಞಾಪಕದಲ್ಲಿರಬಹುದು. ಆತ ಉಪನ್ಯಾಸ ಮಾಡುವಾಗ ಸ್ತಬ್ಧನಾಗಿ ನಿಂತುಬಿಟ್ಟನು. ಅನಂತರ ಜಗತ್ತೆಲ್ಲವೂ ಭಾವಮಯ ಎಂದನು. ತತ್ಕಾಲಕ್ಕೆ ಸ್ಥೂಲ ಸ್ಪಂದನಗಳು ನಿಂತು ಭಾವವೆಂಬ ಸೂಕ್ಷ್ಮಸ್ಪಂದನವೊಂದೇ ಅವನಿಗೆ ಇದ್ದಂತೆ ತೋರಿತು. ತನ್ನ ಸುತ್ತಲೂ ಕೂಡ ಅವನು ಸೂಕ್ಷ್ಮಸ್ಪಂದನವನ್ನು ಮಾತ್ರ ನೋಡಬಲ್ಲವನಾಗಿದ್ದನು. ಎಲ್ಲವೂ ಭಾವನಾಮಯವಾಯಿತು. ತಾನು ಮತ್ತು ಉಳಿದ ವಸ್ತುಗಳೆಲ್ಲ ಸೂಕ್ಷ್ಮ ಆಲೋಚನೆಯ ಸುಳಿಗಳಾದುವು. 

\vskip 0.2cm

ಹೀಗೆ ಆಲೋಚನಾ ಕ್ಷೇತ್ರದಲ್ಲಿ ಕೂಡ ಐಕ್ಯ ಇರುವುದು ನಮಗೆ ತೋರುತ್ತದೆ. ಕೊನೆಗೆ ಆತ್ಮನ ವಿಚಾರಕ್ಕೆ ಬಂದಾಗ ಅದು ಐಕಮಾತ್ರವೆನ್ನುವುದು ಗೊತ್ತೇ ಇದೆ. ವಸ್ತುವಿನ ಸ್ಥೂಲ ಮತ್ತು ಸೂಕ್ಷ್ಮ ಸ್ಪಂದನಗಳಾಚೆ ಇರುವುದು ಒಂದು. ವ್ಯಕ್ತವಾದ ಚಲನೆಯಲ್ಲಿ ಕೂಡ ಐಕ್ಯವೊಂದೇ ಇರುವುದು. ಈ ವಿಷಯವನ್ನು ನಾವು ಅಲ್ಲಗಳೆಯಲಾಗುವುದಿಲ್ಲ. ಆಧುನಿಕ ಭೌತಶಾಸ್ತ್ರವೂ ಕೂಡ ಜಗತ್ತಿನಲ್ಲಿರುವ ಶಕ್ತಿಯ ಮೊತ್ತವೆಲ್ಲ ಯಾವಾಗಲೂ ಒಂದೇ ಪರಿಮಾಣದಲ್ಲಿದೆ ಎಂಬುದನ್ನು ತೋರಿಸುವುದು. ಈ ಶಕ್ತಿಯ ಮೊತ್ತ ಎರಡು ವಿಧವಾಗಿರುವುದು ಎಂಬುದನ್ನು ಕೂಡ ಇದು ನಿದರ್ಶಿಸಿರುವುದು. ಇದು ಮೊದಲು ಗುಪ್ತವಾಗಿ ಶಾಂತವಾಗಿರುವುದು; ಅನಂತರ ನಮಗೆ ಗೋಚರಿಸುವ ಬಹುವಿಧ ಶಕ್ತಿರೂಪಧಾರಣೆ ಮಾಡುವುದು. ಅನಂತರ ಇದು ಪುನಃ ಸುಪ್ತಾವಸ್ಥೆಗೆ ಹೋಗುವುದು. ಅಲ್ಲಿಂದ ಪುನಃ ವ್ಯಕ್ತಾವಸ್ಥೆಗೆ ಬರುವುದು. ಹೀಗೆಯೇ ಎಂದೆಂದಿಗೂ ವಿಕಾಸವಾಗುತ್ತ ಸಂಕುಚಿತವಾಗುತ್ತ ಹೋಗುವುದು. ಹಿಂದೆ ಹೇಳಿದಂತೆ ಈ ಪ್ರಾಣದ ನಿಗ್ರಹವನ್ನೇ ಪ್ರಾಣಾಯಾಮವೆನ್ನುವುದು. 

\vskip 0.2cm

ದೇಹದಲ್ಲಿ ಸ್ಪಷ್ಟವಾಗಿ ನಮ್ಮ ಗಮನಕ್ಕೆ ಬರುವ ಪ್ರಾಣದ ಅಭಿವ್ಯಕ್ತಿಯೆ ಶ್ವಾಸಕೋಶಗಳ ಚಲನೆ. ಅದು ನಿಂತರೆ ದೇಹದಲ್ಲಿರುವ ಉಳಿದ ಎಲ್ಲಾ ಶಕ್ತಿಗಳ ಚಲನೆಯೂ ಖಂಡಿತವಾಗಿ ನಿಂತುಹೋಗುವುದು. ಆದರೆ ಈ ಚಲನೆ ನಿಂತು ಹೋದರೂ ಕೂಡ ದೇಹ ಬದುಕಿರುವಂತೆ ಮಾಡಬಲ್ಲವರು ಕೆಲವರು ಇರುವರು. ಅವರು ಅನೇಕ ದಿನಗಳು ನೆಲದಲ್ಲಿ ಹೂಳಿಸಿಕೊಂಡು ಉಸಿರಾಡದೆ ಜೀವಿಸಬಲ್ಲರು. ಸೂಕ್ಷ್ಮವನ್ನು ನಾವು ತಲುಪಬೇಕಾದರೆ ಸ್ಥೂಲದ ಸಹಾಯವನ್ನು ನಾವು ಪಡೆಯಬೇಕು. ಹೀಗೆಯೇ ನಮ್ಮ ಗುರಿಯು ಸಿಕ್ಕುವತನಕ ಅತ್ಯಂತ ಸೂಕ್ಷ್ಮತಮ ವಸ್ತುವಿನೆಡೆಗೆ ಹೋಗಬೇಕು. ಪ್ರಾಣಾಯಾಮವೆಂದರೆ ನಿಜವಾದ ಅರ್ಥದಲ್ಲಿ, ಶ್ವಾಸಕೋಶಗಳ ಚಲನೆಯನ್ನು ಸ್ವಾಧೀನಕ್ಕೆ ತರುವುದು. ಈ ಚಲನೆಗೂ ಉಸಿರಿಗೂ ಸಂಬಂಧವಿದೆ. ಉಸಿರು ಈ ಚಲನೆಗೆ ಕಾರಣವೆಂತಲ್ಲ; ಅದಕ್ಕೆ ಬದಲಾಗಿ, ಇದು ಉಸಿರಾಡುವುದಕ್ಕೆ ಕಾರಣ. ಈ ಚಲನೆಯು ಪಂಪಿನಂತೆ ಗಾಳಿಯನ್ನು ಒಳಗೆ ಸೆಳೆಯುತ್ತದೆ. ಪ್ರಾಣವು ಶ್ವಾಸಕೋಶಗಳನ್ನು ಚಲಿಸುವಂತೆ ಮಾಡುವುದು. ಈ ಚಲನೆಯು ವಾಯುವನ್ನು ಒಳಗೆ ಸೆಳೆಯುವುದು. ಆದಕಾರಣ ಪ್ರಾಣಾಯಾಮವೆಂದರೆ ಉಸಿರಾಡುವುದೆಂದಲ್ಲ ಅರ್ಥ, ಶ್ವಾಸಕೋಶಗಳನ್ನು ಚಲಿಸುವಂತೆ ಮಾಡುವ ಮಾಂಸಖಂಡಗಳನ್ನು ಸ್ವಾಧೀನಕ್ಕೆ ತೆಗೆದುಕೊಂಡು ಬರುವುದು. ನರಗಳ ಮೂಲಕ ಮಾಂಸಖಂಡಕ್ಕೆ ಹೋಗಿ, ಅಲ್ಲಿಂದ ಶ್ವಾಸಕೋಶಗಳಿಗೆ ಹೋಗಿ ಅದನ್ನು ಒಂದು ರೀತಿಯಲ್ಲಿ ಚಲಿಸುವಂತೆ ಮಾಡುವ ಆ ಶಕ್ತಿಯೇ ಪ್ರಾಣ. ಪ್ರಾಣಾಯಾಮದ ಅಭ್ಯಾಸದಲ್ಲಿ ನಾವು ಇದನ್ನು ಸ್ವಾಧೀನಕ್ಕೆ ತರಬೇಕಾಗಿದೆ. ಪ್ರಾಣವು ಸ್ವಾಧೀನವಾದ ಮೇಲೆ, ದೇಹದಲ್ಲಿರುವ ಉಳಿದ ಎಲ್ಲಾ ಪ್ರಾಣಗಳ ಕ್ರಿಯೆಗಳೂ ನಮ್ಮ ಸ್ವಾಧೀನಕ್ಕೆ ಬರುವುದು ತಕ್ಷಣವೇ ಗೊತ್ತಾಗುವುದು. ದೇಹದ ಪ್ರತಿಯೊಂದು ಮಾಂಸಖಂಡವನ್ನೂ ಕೂಡ ಸ್ವಾಧೀನಕ್ಕೆ ತರಬಲ್ಲ ಕೆಲವರನ್ನು ನಾನೇ ನೋಡಿರುವೆನು. ಇದು ಏಕೆ ಸಾಧ್ಯವಾಗಬಾರದು? ಕೆಲವು ಮಾಂಸಖಂಡಗಳ ಮೇಲೆ ನನ್ನ ಸ್ವಾಧೀನತೆ ಇದ್ದರೆ ಪ್ರತಿಯೊಂದು ಮಾಂಸ ಖಂಡದ ಮತ್ತು ಪ್ರತಿಯೊಂದು ನರದ ಮೇಲೆ ಏತಕ್ಕೆ ಸ್ವಾಧೀನತೆ ಇರಲಾರದು? ಇದರಲ್ಲಿ ಏನು ಅಸಾಧ್ಯವಿದೆ? ಸದ್ಯಕ್ಕೆ ಆ ಸ್ವಾಧೀನತೆ ತಪ್ಪಿದೆ. ಅದಕ್ಕೋಸುಗವೇ ಚಲನೆ ಅನೈಚ್ಛಿಕವಾಗಿದೆ. ನಮಗೆ ಇಚ್ಛೆ ಬಂದಂತೆ ನಾವು ಕಿವಿಗಳನ್ನು ತಿರುಗಿಸಲಾಗುವುದಿಲ್ಲ. ಆದರೆ ಇದನ್ನು ಪ್ರಾಣಿಗಳು ಮಾಡುವುದು ನಮಗೆ ಗೊತ್ತಿದೆ. ಅದನ್ನು ನಾವು ಬಳಸುವುದಿಲ್ಲ. ಅದಕ್ಕೇ ನಾವು ಆ ಶಕ್ತಿಯನ್ನು ಕಳೆದುಕೊಂಡಿರುವೆವು. ಅದನ್ನೇ ನಾವು ಅಟಾವಿಸಂ \enginline{(Atavism)} ಎನ್ನುವುದು. 

\vskip 0.2cm

ಯಾವ ಚಲನೆಯು ಸುಪ್ತವಾಗಿದೆಯೋ ಅದನ್ನು ಪುನಃ ವ್ಯಕ್ತಮಾಡಬಹುದೆಂಬುದು ನಮಗೆ ಗೊತ್ತಿದೆ. ಹೆಚ್ಚಿನ ಶ್ರಮ ಮತ್ತು ಅಭ್ಯಾಸದಿಂದ ಸುಪ್ತವಾಗಿರುವ ದೇಹದ ಕೆಲವು ಚಲನೆಗಳನ್ನು ಸಂಪೂರ್ಣವಾಗಿ ನಮ್ಮ ಸ್ವಾಧೀನಕ್ಕೆ ತರಬಹುದು. ಹೀಗೆ ವಿಚಾರ ಮಾಡಿನೋಡಿದರೆ ಅಸಾಧ್ಯವಾದುದು ಯಾವುದೂ ಇಲ್ಲವೆಂಬುದು ನಮಗೆ ತೋರುವುದು. ದೇಹದ ಪ್ರತಿಯೊಂದು ಭಾಗವನ್ನೂ ನಮ್ಮ ಸ್ವಾಧೀನಕ್ಕೆ ತರುವುದು ಸಾಧ್ಯ. ಇದನ್ನೇ ಯೋಗಿಯು ಪ್ರಾಣಯಾಮದ ಮೂಲಕ ಸಾಧಿಸುವನು. ಪ್ರಾಣಾಯಾಮದಲ್ಲಿ ವಾಯುವನ್ನು ಒಳಗೆ ಸೆಳೆದುಕೊಳ್ಳುವಾಗ ನಿಮ್ಮ ದೇಹದ ಆದ್ಯಂತವನ್ನೂ ಪ್ರಾಣದಿಂದ ತುಂಬಬೇಕೆಂಬುದನ್ನು ಬಹುಶಃ ನಿಮ್ಮಲ್ಲಿ ಕೆಲವರು ಓದಿರಬಹುದು. ಇಂಗ್ಲೀಷ್​ ಭಾಷಾಂತರದಲ್ಲಿ ಪ್ರಾಣವೆಂದರೆ ಉಸಿರು ಎಂದು ಇರುವುದರಿಂದ ಅದು ಹೇಗೆ ಸಾಧ್ಯವೆಂದು ನೀವು ಕೇಳಲು ಇಚ್ಛಿಸಬಹುದು. ತಪ್ಪು ಭಾಷಾಂತರಕಾರರದು. ದೇಹದ ಪ್ರತಿಯೊಂದು ಭಾಗವನ್ನೂ ಕೂಡ ಪ್ರಾಣಶಕ್ತಿಯಿಂದ ತುಂಬಬಹುದು. ಹೀಗೆ ಮಾಡುವುದು ನಿಮಗೆ ಸಾಧ್ಯವಾದರೆ ನೀವು ದೇಹವನ್ನೆಲ್ಲ ಸ್ವಾಧೀನಕ್ಕೆ ತರಬಹುದು. ದೇಹದಲ್ಲಿ ಕಾಣಿಸಿಕೊಳ್ಳುವ ಎಲ್ಲ ರೋಗ ಮತ್ತು ವೇದನೆಗಳನ್ನೂ ನಿಗ್ರಹಿಸಲು ಸಾಧ್ಯ. ಇಷ್ಟೇ ಇಲ್ಲದೆ ಇತರರ ದೇಹವನ್ನೂ ನೀವು ನಿಯಂತ್ರಿಸಬಲ್ಲಿರಿ. ಈ ಪ್ರಪಂಚದಲ್ಲಿ ಒಳ್ಳೆಯದು ಮತ್ತು ಕೆಟ್ಟದ್ದು ಎಲ್ಲವೂ ಸಾಂಕ್ರಾಮಿಕವಾದುದು. ನಿಮ್ಮ ದೇಹವು ಅಸ್ವಸ್ಥವಾಗಿದ್ದರೆ, ಉಳಿದವರಲ್ಲಿ ಕೂಡ ಅದೇ ಅಸ್ವಸ್ಥತೆಯನ್ನು ಉಂಟುಮಾಡುವ ಸಾಧ್ಯತೆಯಿದೆ. ನೀವು ದೃಢಕಾಯರಾಗಿ ಆರೋಗ್ಯವಾಗಿದ್ದರೆ ನಿಮ್ಮ ಹತ್ತಿರದಲ್ಲಿ ವಾಸಿಸುವುದು ಕೂಡ ದೃಢಕಾಯರಾಗಿ ಆರೋಗ್ಯದಿಂದಿರುವ ಸಂಭವವಿದೆ. ಆದರೆ ನೀವು ರೋಗಿಗಳಾಗಿ ನಿರ್ಬಲರಾಗಿದ್ದರೆ ಅವರೂ ಕೂಡ ಹಾಗೆಯೇ ಆಗುವ ಸಂಭವವಿದೆ. ಒಬ್ಬನು ಇನ್ನೊಬ್ಬನನ್ನು ಗುಣ ಮಾಡುವ ಸಂದರ್ಭದಲ್ಲಿ ಕೂಡ ತನ್ನ ಸ್ವಂತ ಆರೋಗ್ಯವನ್ನು ಮತ್ತೊಬ್ಬನಿಗೆ ಕೊಡುವುದೇ ಮೊದಲನೆಯ ಭಾವನೆ. ಇದೇ ಪೂರ್ವದಲ್ಲಿ ಗುಣ ಮಾಡುವ ರೀತಿಯಾಗಿತ್ತು. ತಿಳಿದೋ ಅಥವಾ ತಿಳಿಯದೆಯೋ ನಾವು ಆರೋಗ್ಯವನ್ನು ಮತ್ತೊಬ್ಬನಿಗೆ ಕೊಡಬಹುದು. ಬಲಾಢ್ಯನು ತನ್ನೊಂದಿಗೆ ಇರುವ ಬಲಹೀನನನ್ನು, ತನಗೆ ತಿಳಿದೋ ತಿಳಿಯದೆಯೋ ಸ್ವಲ್ಪ ಬಲಾಢ್ಯನನ್ನಾಗಿ ಮಾಡುತ್ತಾನೆ. ಅದನ್ನು ತಿಳಿದು ಮಾಡುವಾಗ ಆ ಕ್ರಿಯೆಯು ಉತ್ತಮವಾಗುವುದು ಮತ್ತು ತ್ವರಿತವೂ ಆಗುವುದು. ಅನಂತರ ನಮಗೆ ಕಂಡುಬರುವ ಸಂಗತಿ ಯಾವುದೆಂದರೆ, ಒಬ್ಬನು ತಾನೇ ಹೆಚ್ಚು ಆರೋಗ್ಯವಂತನಾಗದೆ ಇರಬಹುದು, ಆದರೆ ಮತ್ತೊಬ್ಬನಿಗೆ ಆರೋಗ್ಯವನ್ನು ಕೊಡಬಹುದು. ಅಂತಹ ಸಂದರ್ಭದಲ್ಲಿ ಮೊದಲನೆಯವನಿಗೆ ಪ್ರಾಣದ ಮೇಲೆ ಸ್ವಲ್ಪ ಸ್ವಾಧೀನತೆಯಿರುತ್ತದೆ. ತತ್ಕಾಲಕ್ಕೆ ತನ್ನ ಪ್ರಾಣವನ್ನು ಮೇಲಿನ ಸ್ಪಂದನಕ್ಕೆ ಏರಿಸಿ ಅದನ್ನು ಬೇರೊಬ್ಬನಿಗೆ ಕೊಡುವನು. 


ಇಂತಹ ಕೆಲಸವನ್ನು ದೂರದಿಂದ ಮಾಡಿದ ಕೆಲವು ಸನ್ನಿವೇಶಗಳಿವೆ. ಆದರೆ ನಿಜವಾಗಿ, ಅಂತರ ಎಂಬರ್ಥದಲ್ಲಿ ದೂರವೆಂಬುದಿಲ್ಲ. ಏನೂ ಇಲ್ಲದೆ ಇರುವ ಅಂತರವೆಲ್ಲಿದೆ? ನಿಮಗೂ ಸೂರ್ಯನಿಗೂ ಮಧ್ಯೆ ಏನಾದರೂ ಅಂತರ ಇದೆಯೇನು? ಇದೆಲ್ಲ ಒಂದು ಅಖಂಡವಾದ ದ್ರವ್ಯರಾಶಿ. ಸೂರ್ಯನು ಅದರಲ್ಲಿ ಒಂದು ಭಾಗ ನೀವು ಇನ್ನೊಂದು ಭಾಗ. ನದಿಯ ಒಂದು ಭಾಗಕ್ಕೂ ಮತ್ತೊಂದು ಭಾಗಕ್ಕೂ ಸಂಬಂಧವಿಲ್ಲದೆ ಇದೆಯೇನು? ಹಾಗಾದರೆ ಯಾವುದಾದರೊಂದು ಶಕ್ತಿಯು ಏತಕ್ಕೆ ಸಂಚರಿಸಬಾರದು? ಇದನ್ನು ವಿರೋಧಿಸುವುದಕ್ಕೆ ಯಾವ ಕಾರಣವೂ ಇಲ್ಲ. ದೂರದಿಂದ ಗುಣಮಾಡುವ ಕೆಲವು ಸಂದರ್ಭಗಳು ನಿಜವಾಗಿಯೂ ಸತ್ಯ. ಪ್ರಾಣವನ್ನು ಬಹಳ ದೂರಕ್ಕೆ ಬೇಕಾದರೆ ವರ್ಗಾಯಿಸಬಹುದು. ಆದರೆ ಇಂತಹ ವಿಷಯಗಳಲ್ಲಿ ಒಂದು ನಿಜವಿದ್ದರೆ ಮೋಸ ನೂರಾರು ಇವೆ. ಈ ವಿಧವಾಗಿ ಗುಣ ಮಾಡುವುದು ತಿಳಿದುಕೊಂಡಿರುವಷ್ಟು ಸುಲಭವೇನೂ ಅಲ್ಲ. ಇಂತಹ ಚಿಕಿತ್ಸೆಗಳಲ್ಲಿ ಸಾಧಾರಣವಾಗಿ ಅವರು ಮಾನವದೇಹದ ಸ್ವಾಭಾವಿಕ ಆರೋಗ್ಯಸ್ಥಿತಿಯನ್ನು ಉಪಯೋಗಿಸುವರು. ಆಂಗ್ಲೇಯ ವೈದ್ಯನು ಕಾಲರಾ ರೋಗಿಯ ಶುಶ್ರೂಷೆಗೆ ತನ್ನ ಔಷಧಿಯನ್ನು ಕೊಡುವನು. ಹೋಮಿಯೋಪತಿ ವೈದ್ಯರು ಬಂದು ರೋಗಿಗೆ ತನ್ನ ಔಷಧಿಯನ್ನು ಕೊಟ್ಟು ಬಹುಶಃ ಆಂಗ್ಲೇಯ ವೈದ್ಯನಿಗಿಂತ ಹೆಚ್ಚು ಗುಣಪಡಿಸಬಹುದು. ಏಕೆಂದರೆ ಅವನು ರೋಗಿಯನ್ನು ತೊಂದರೆ ಪಡಿಸುವುದಿಲ್ಲ. ಅದರ ಬದಲು ಪ್ರಕೃತಿಯು ತನ್ನ ಕೆಲಸ ಮಾಡುವಂತೆ ಅವಕಾಶ ಕಲ್ಪಿಸುವನು. ಶ್ರದ್ಧಾ ವೈದ್ಯನು ಇನ್ನೂ ಹೆಚ್ಚು ಗುಣ ಮಾಡುವನು. ಏಕೆಂದರೆ ಅವನು ತನ್ನ ಮಾನಸಿಕ ಶಕ್ತಿಯನ್ನು ಉಪಯೋಗಿಸಿ ನಂಬಿಕೆಯ ಮೂಲಕ ರೋಗಿಯಲ್ಲಿ ಸುಪ್ತವಾಗಿರುವ ಶಕ್ತಿಯನ್ನು ಜಾಗೃತ ಗೊಳಿಸುವನು. 



ಶ್ರದ್ಧಾವೈದ್ಯರು ಸಾಧಾರಣವಾಗಿ ಮಾಡುವ ಒಂದು ತಪ್ಪು ಇದೆ. ಅವರು, ನಂಬಿಕೆಯೇ ನೇರವಾಗಿ ರೋಗಿಯನ್ನು ಗುಣಪಡಿಸುತ್ತದೆ ಎಂದು ತಿಳಿಯುವರು. ಎಲ್ಲ ಸಂದರ್ಭಗಳಿಗೂ ಇದು ಅನ್ವಯಿಸುವುದಿಲ್ಲ. ಕೆಲವು ಖಾಯಿಲೆಗಳಿವೆ, ಅವುಗಳ ಸ್ವಭಾವವೇನೆಂದರೆ, ರೋಗಿಯು ತನಗೆ ಆ ಖಾಯಿಲೆಯೇ ಇಲ್ಲವೆಂದು ನಂಬುತ್ತಾನೆ. ಈ ಅಸಾಧ್ಯ ನಂಬಿಕೆಯೇ ರೋಗದ ಒಂದು ಚಿಹ್ನೆ. ಸಾಧಾರಣವಾಗಿ ಅವನು ಬೇಗ ಸಾಯುವನೆಂಬುದನ್ನು ಇದು ಸೂಚಿಸುತ್ತದೆ. ಅಂತಹ ಸಮಯಗಳಲ್ಲಿ ನಂಬಿಕೆಯು ಗುಣಪಡಿಸುವುದು ಎಂಬ ನಿಯಮವು ಸರಿ ಹೋಗುವುದಿಲ್ಲ. ನಂಬಿಕೆಯೊಂದೇ ಗುಣಪಡಿಸುವುದಿದ್ದರೆ ಇಂತಹ ರೋಗಿಗಳೂ ಗುಣಮುಖರಾಗುತ್ತಿದ್ದರು. ನಿಜವಾಗಿ ಗುಣವಾಗುವುದು ಪ್ರಾಣದಿಂದ. ಪರಿಶುದ್ಧನಾಗಿ, ಪ್ರಾಣವನ್ನು ನಿಗ್ರಹಿಸಿದವನಿಗೆ ಅದನ್ನು ಒಂದು ಸ್ಪಂದನಕ್ಕೆ ತರುವ ಶಕ್ತಿಯಿದೆ. ಇದನ್ನು ಮತ್ತೊಬ್ಬರಿಗೂ ವರ್ಗಾಯಿಸಿ ಅವರಲ್ಲಿಯೂ ಅದೇ ರೀತಿಯೆ ಉಂಟಾಗುವಂತೆ ಮಾಡಬಹುದು. ಪ್ರತಿದಿನದ ನಡವಳಿಕೆಯಲ್ಲಿ ನೀವು ಇದನ್ನು ನೋಡುತ್ತೀರಿ. ನಾನು ನಿಮ್ಮೊಂದಿಗೆ ಮಾತನಾಡುತ್ತಿರುವೆನು. ನಾನು ನಿಜವಾಗಿ ಏನು ಮಾಡಲು ಯತ್ನಿಸುತ್ತಿರುವೆನು? ಈಗ ನಾನು ನನ್ನ ಮನಸ್ಸನ್ನು ಒಂದು ಸ್ಪಂದನಾ ಸ್ಥಿತಿಗೆ ತರುತ್ತಿರುವೆನು. ನಾನು ಇದರಲ್ಲಿ ಯಶಸ್ವಿಯಾದಷ್ಟೂ ನಾನು ಹೇಳುವುದು ನಿಮ್ಮ ಮೇಲೆ ಹೆಚ್ಚು ಪರಿಣಾಮಕಾರಿ ಯಾಗುವುದು. ನಾನು ಹೆಚ್ಚು ಉತ್ಸಾಹಶಾಲಿಯಾದ ದಿನ ನೀವು ನನ್ನ ಉಪನ್ಯಾಸವನ್ನು ಹೆಚ್ಚು ಪ್ರೀತಿಸುತ್ತೀರಿ. ನನ್ನಲ್ಲಿ ಕಡಿಮೆ ಉತ್ಸಾಹವಿರುವ ದಿನ ಉಪನ್ಯಾಸದಲ್ಲಿ ಅಷ್ಟೇನೂ ಸ್ವಾರಸ್ಯವಿಲ್ಲದಿರುವುದು ನಿಮಗೆಲ್ಲಾ ಗೊತ್ತೇ ಇದೆ. 


\eject

ಜಗತ್ತನ್ನೇ ಅಲ್ಲಾಡಿಸಬಲ್ಲ ಅತ್ಯದ್ಭುತ ಇಚ್ಛಾಶಕ್ತಿಯುಳ್ಳವರು, ತಮ್ಮ ಪ್ರಾಣವನ್ನು ಅತಿ ಮೇಲಿನ ಸ್ಪಂದನಕ್ಕೆ ತರಬಲ್ಲರು. ಉಳಿದವರನ್ನು ಕ್ಷಣಕಾಲದಲ್ಲಿ ತಮ್ಮ ಸ್ವಾಧೀನಕ್ಕೆ ತರುವಷ್ಟು ಪ್ರಾಣಶಕ್ತಿ ಅವರಲ್ಲಿರುತ್ತದೆ. ಸಾವಿರಾರು ಜನ ಅವರಿಂದ ಆಕರ್ಷಿತರಾಗುತ್ತಾರೆ, ಅರ್ಧ ಪ್ರಪಂಚವೇ ಅವರು ಆಲೋಚಿಸಿದಂತೆ ಆಲೋಚಿಸುತ್ತದೆ. ಜಗದ ಪ್ರಖ್ಯಾತರಾದ ದೇವದೂತರಿಗೆ ಪ್ರಾಣದ ಮೇಲೆ ಅತ್ಯದ್ಭುತವಾದ ಸ್ವಾಧೀನತೆಯಿತ್ತು. ಇದು ಅವರಿಗೆ ಪ್ರಚಂಡ ಇಚ್ಛಾಶಕ್ತಿಯನ್ನು ಕೊಟ್ಟಿತು. ಇದೇ ಜನರ ಮನಸ್ಸನ್ನು ಸೂರೆಗೊಳ್ಳಲು ಸಹಾಯವಾಯಿತು. ಶಕ್ತಿಯ ಎಲ್ಲಾ ಅಭಿವ್ಯಕ್ತಿಗಳು ಬರುವುದು ಈ ಪ್ರಾಣದ ನಿಗ್ರಹದಿಂದ. ಇದರ ರಹಸ್ಯ ಜನರಿಗೆ ಗೊತ್ತಿಲ್ಲದೆ ಇರಬಹುದು. ಆದರೆ ಇದಕ್ಕೆ ಪ್ರಾಣನಿಗ್ರಹ ಕಾರಣ. ಕೆಲವು ವೇಳೆ ನಿಮ್ಮ ದೇಹದಲ್ಲಿಯೇ ಪ್ರಾಣವು ಯಾವುದೋ ಒಂದು ಭಾಗದ ಕಡೆಗೆ ಹೆಚ್ಚು ಹರಿಯುತ್ತದೆ. ಆಗ ಸಮತ್ವ ಬದಲಾಯಿಸುವುದು. ಪ್ರಾಣದ ಸಮತ್ವಕ್ಕೆ ಆತಂಕ ಬಂದಾಗ ರೋಗ ತಲೆದೋರುವುದು. ಹೆಚ್ಚಾಗಿರುವ ಪ್ರಾಣವನ್ನು ತಗ್ಗಿಸುವುದು, ಅಥವಾ ಕಡಮೆ ಪ್ರಾಣವಿರುವ ಕಡೆಗೆ ಹೆಚ್ಚು ಪ್ರಾಣವನ್ನು ಹರಿಸುವುದು – ಇದನ್ನೇ ರೋಗನಿವಾರಣೆಯೆಂದು ಕರೆಯುತ್ತೇವೆ. ಯಾವುದೊ ಒಂದು ಭಾಗದಲ್ಲಿ ಯಾವಾಗಲೂ ಇರಬೇಕಾದುದಕ್ಕಿಂತ ಸ್ವಲ್ಪ ಹೆಚ್ಚು ಅಥವಾ ಕಡಮೆ ಪ್ರಾಣವಿದೆ ಎಂದು ತಿಳಿಯುವುದೂ ಕೂಡ ಪ್ರಾಣಾಯಾಮ. ನಮ್ಮ ಭಾವನೆ ಎಷ್ಟು ಸೂಕ್ಷ್ಮವಾಗುವುದೆಂದರೆ ಕಾಲು ಅಥವಾ ಕೈಬೆರಳಿನಲ್ಲಿ ಪ್ರಾಣ ಕಡಮೆಯಿದ್ದರೆ ಅದೂ ಮನಸ್ಸಿಗೆ ವೇದ್ಯವಾಗುತ್ತದೆ, ಮತ್ತು ಅಲ್ಲಿಗೆ ಬೇಕಾಗುವಷ್ಟು ಪ್ರಾಣವನ್ನು ಕಳಿಸುವ ಶಕ್ತಿಯನ್ನೂ ಪಡೆಯುತ್ತದೆ. ಇವೇ ಪ್ರಾಣಾಯಾಮದ ಅನೇಕ ಕಾರ್ಯಗಳು. ಇವನ್ನು ಕ್ರಮೇಣ ನಿಧಾನವಾಗಿ ಕಲಿಯಬೇಕು. ನೀವು ನೋಡಿದಂತೆಯೇ ರಾಜ ಯೋಗದ ಮುಖ್ಯಗುರಿಯೇ ವಿವಿಧ ಸ್ತರಗಳಲ್ಲಿ ಪ್ರಾಣವನ್ನು ನಿಗ್ರಹಿಸಿ ಅದರ ಕಾರ್ಯವನ್ನು ನಿರ್ದೇಶಿಸುವುದು ಹೇಗೆ ಎಂಬುದನ್ನು ಬೋಧಿಸುವುದಾಗಿದೆ. ಮನುಷ್ಯನು ತನ್ನ ಶಕ್ತಿಯನ್ನೆಲ್ಲ ಏಕಾಗ್ರಗೊಳಿಸಿದಾಗ ದೇಹದಲ್ಲಿರುವ ಪ್ರಾಣವನ್ನು ಸ್ವಾಧೀನಕ್ಕೆ ತರುತ್ತಾನೆ. ಒಬ್ಬನು ಧ್ಯಾನ ಮಾಡುತ್ತಿರುವಾಗ ಪ್ರಾಣವನ್ನೇ ಏಕಾಗ್ರಗೊಳಿಸುತ್ತಿರುತ್ತಾನೆ. 

\vskip 0.2cm

ಸಾಗರದಲ್ಲಿ ಪರ್ವತದಷ್ಟು ಎತ್ತರವಾಗಿರುವ ಅಲೆಗಳಿವೆ. ಅನಂತರ ಸಣ್ಣ ಅಲೆಗಳು, ಅದಕ್ಕಿಂತ ಸಣ್ಣ ಅಲೆಗಳು ಮತ್ತು ನೀರಿನ ಗುಳ್ಳೆಗಳು ಇರುತ್ತವೆ. ಆದರೆ ಇವುಗಳ ಹಿಂದೆಲ್ಲ ಅನಂತ ಸಾಗರವಿದೆ. ಒಂದು ಕಡೆ ನೀರುಗುಳ್ಳೆಯು ಅನಂತ ಸಾಗರದೊಂದಿಗೆ ಸೇರಿದೆ, ಮತ್ತೊಂದು ಕಡೆ ಅಗಾಧವಾದ ಅಲೆಯೂ ಸಾಗರದೊಂದಿಗೆ ಸೇರಿದೆ. ಅದರಂತೆಯೇ ಒಬ್ಬನು ಅತ್ಯದ್ಭುತವಾದ ಪ್ರತಿಭಾ ಶಾಲಿಯಾಗಿರಬಹುದು; ಮತ್ತೊಬ್ಬನು ಸಣ್ಣ ನೀರಿನ ಗುಳ್ಳೆಗಳಂತೆ ಇರಬಹುದು. ಆದರೆ ಪ್ರತಿಯೊಬ್ಬರೂ ಕೂಡ, ಪ್ರತಿಯೊಂದು ಪ್ರಾಣಿಯೂ ಕೂಡ, ಎಲ್ಲಕ್ಕೂ ಆಜನ್ಮಸಿದ್ಧ ಹಕ್ಕಾದ, ಅನಂತ ಶಕ್ತಿಸಾಗರದೊಂದಿಗೆ ಸಂಬಂಧ ಪಡೆದಿದೆ. ಚೇತನ ವೆಲ್ಲಿರುವುದೋ ಅಲ್ಲೆಲ್ಲ ಅನಂತ ಶಕ್ತಿಸಾಗರ ಅದರ ಹಿಂದೆ ಇದೆ. ಅಣಬೆಯಂತಹ ಒಂದು ಅತಿ ಸೂಕ್ಷ್ಮ ಕ್ಷುದ್ರ ಜೀವಿಯು ಕೂಡ ಅನವರತವೂ ಆ ಅನಂತ ಶಕ್ತಿಯ ಆಗರದಿಂದ ಸತ್ತ್ವವನ್ನು ಹೀರಿಕೊಂಡು ಕ್ರಮೇಣ ನಿಧಾನವಾಗಿ ಆಕಾರವನ್ನು ಬದಲಾಯಿಸುತ್ತ, ಕಾಲಕ್ರಮದಲ್ಲಿ ಅದು ಮರವಾಗಿ, ಮೃಗವಾಗಿ, ಮಾನವನಾಗಿ ಕೊನೆಗೆ ದೇವರಾಗುವುದು. ಇದು ಲಕ್ಷಾಂತರ ವರ್ಷಗಳಲ್ಲಿ ಸಾಧಿಸಲ್ಪಡುತ್ತದೆ. ಆದರೆ ಕಾಲ ಎಂದರೇನು? ವೇಗ ಹೆಚ್ಚಿದರೆ, ಹೋರಾಟ ಬಲವಾದರೆ, ಕಾಲದ ದೀರ್ಘತೆ ಕುಗ್ಗುತ್ತದೆ. ಯಾವುದನ್ನು ಸಾಧಿಸುವುದಕ್ಕೆ ಸ್ವಾಭಾವಿಕವಾಗಿ ಹೆಚ್ಚು ಕಾಲ ಬೇಕಾಗುವುದೋ ಅದನ್ನು ನಮ್ಮ ಸಾಧನೆಯ ತೀವ್ರತೆಯಿಂದ ಕಡಮೆ ಮಾಡಬಹುದೆಂದು ಯೋಗಿಯು ಹೇಳುತ್ತಾನೆ. ಪ್ರಪಂಚದಲ್ಲಿರುವ ಅನಂತ ಶಕ್ತಿಯ ಆಗರದಿಂದ ಒಬ್ಬ ಮನುಷ್ಯನು ನಿಧಾನವಾಗಿ ಸತ್ತ್ವವನ್ನು ಸೆಳೆಯುತ್ತಿದ್ದರೆ ಬಹುಶಃ ದೇವನಾಗಲು ಅವನಿಗೆ ಲಕ್ಷ ವರ್ಷಗಳು ಹಿಡಿಯುತ್ತವೆ. ಅಲ್ಲಿಂದ ಇನ್ನೂ ಉತ್ತಮನಾಗಬೇಕಾದರೆ ಐದು ಲಕ್ಷ ವರುಷಗಳು ಬೇಕಾಗಬಹುದು. ಅನಂತರ ಅವನು ಸಿದ್ಧನಾಗಬೇಕಾದರೆ ಬಹುಶಃ ಐವತ್ತು ಲಕ್ಷ ವರುಷಗಳು ಬೇಕಾಗಬಹುದು. ಬೆಳವಣಿಗೆಯ ತೀವ್ರತೆಯನ್ನು ಹೆಚ್ಚಿಸಿದಷ್ಟೂ ಕಾಲ ಕಡಮೆಯಾಗುವುದು. ನಾವು ಕೈಲಾದಷ್ಟು ಕಷ್ಟಪಟ್ಟು ಈ ಪೂರ್ಣಾವಸ್ಥೆಯನ್ನು ಆರು ತಿಂಗಳಲ್ಲಿಯೋ ಅಥವಾ ಆರು ವರುಷಗಳಲ್ಲಿಯೋ ಪಡೆಯಲು ಏತಕ್ಕೆ ಸಾಧ್ಯವಿಲ್ಲ? ಇದಕ್ಕೆ ಯಾವ ಒಂದು ನಿರ್ಬಂಧವೂ ಇಲ್ಲ. ಯುಕ್ತಿಯು ಇದನ್ನು ತೋರುತ್ತದೆ. ಒಂದು ಯಂತ್ರವು ಒಂದು ನಿರ್ದಿಷ್ಟ ಮೊತ್ತದ ಕಲ್ಲಿದ್ದಿಲಿನಿಂದ ಗಂಟೆಗೆ ಎರಡು ಮೈಲಿ ಓಡಿದರೆ ಕಲ್ಲಿದ್ದಲನ್ನು ಜಾಸ್ತಿ ಹಾಕುವುದರಿಂದ ಅದೇ ಅಂತರವನ್ನು ಕಡಮೆ ಸಮಯದಲ್ಲಿ ಕ್ರಮಿಸುತ್ತದೆ. ಅದರಂತೆಯೇ ಜೀವನೂ ಕೂಡ ತನ್ನ ಸಾಧನೆಯ ಆಧಿಕ್ಯದಿಂದ ಈ ಜನ್ಮದಲ್ಲಿಯೇ ಏತಕ್ಕೆ ಸಿದ್ಧಿಯನ್ನು ಪಡೆಯಬಾರದು? ಎಲ್ಲಾ ಪ್ರಾಣಿಗಳೂ ಕಟ್ಟಕಡೆಗೆ ಗುರಿಯನ್ನು ಸೇರುತ್ತವೆ ಎಂಬುದು ನಮಗೆ ಗೊತ್ತಿದೆ. ಆದರೆ ಕೋಟ್ಯಂತರ ವರುಷಗಳಷ್ಟು ಕಾದು ಕುಳಿತುಕೊಳ್ಳಲು ಯಾರಿಗೆ ಸಹನೆ ಇದೆ? ಈ ಮಾನವದೇಹದಲ್ಲಿರುವಾಗಲೇ, ಇದೇ ದೇಹದಿಂದಲೇ ಗುರಿಯನ್ನು ಏತಕ್ಕೆ ತಕ್ಷಣವೇ ನಾವು ಸೇರಬಾರದು? ಆ ಅನಂತ ಜ್ಞಾನವನ್ನು, ಅನಂತ ಶಕ್ತಿಯನ್ನು ನಾವು ಏತಕ್ಕೆ ಪಡೆಯಬಾರದು?

\vskip 0.2cm

ನಿಧಾನವಾಗಿ ಒಂದು ಅವಸ್ಥೆಯಿಂದ ಮತ್ತೊಂದು ಅವಸ್ಥೆಗೆ ಮುಂದುವರಿಯುತ್ತಾ, ಇಡೀ ಮಾನವ ಜನಾಂಗವು ಮೋಕ್ಷವನ್ನು ಪಡೆಯುವತನಕ ಕಾದು ಕುಳಿತುಕೊಂಡಿರುವ ಬದಲು, ತತ್ತ್ವಗಳನ್ನು ಅನುಷ್ಠಾನದಲ್ಲಿ ಬರುವಂತೆ ಮಾಡುವ ಶಕ್ತಿಯನ್ನು ಹೆಚ್ಚಿಸಿ ಮುಕ್ತಿಯನ್ನು ಪಡೆಯುವ ಕಾಲವನ್ನು ಕಡಮೆ ಮಾಡುವ ವಿಧಾನವನ್ನು ಬೋಧಿಸುವುದಕ್ಕಾಗಿಯೇ ಯೋಗಶಾಸ್ತ್ರವಿರುವುದು. ಇದೇ ಯೋಗಿಯ ಗುರಿ. ದೇವದೂತರು, ಮಹರ್ಷಿಗಳು, ಮತ್ತು ತಪಸ್ವಿಗಳು ಏನು ಮಾಡಿದರು? ಒಂದು ಜೀವನದಲ್ಲಿ ಇಡೀ ಮಾನವ ಜನಾಂಗದ ಬಾಳನ್ನೇ ಬಾಳಿದರು. ಸಾಧಾರಣ ಮಾನವನು ಮುಕ್ತಿಯನ್ನು ಪಡೆಯುವುದಕ್ಕೆ ಹಲವು ಜನ್ಮಗಳು ನಡೆಯಬೇಕಾದ ದಾರಿಯನ್ನು ಒಂದೇ ಜನ್ಮದಲ್ಲಿ ಮುಗಿಸಿದರು. ಒಂದೇ ಜನ್ಮದಲ್ಲಿಯೇ ಅವರು ಪೂರ್ಣತೆಯನ್ನು ಪಡೆದರು. ಮತ್ತಾವುದನ್ನೂ ಅವರು ಆಲೋಚಿಸುವುದೇ ಇಲ್ಲ. ಬೇರಾವ ವಿಷಯಗಳನ್ನೂ ಅವರು ಒಂದು\break ಗಳಿಗೆಯೂ ಯೋಚಿಸುವುದೇ ಇಲ್ಲ. ಹೀಗೆ ಅವರ ದೂರ ಕಡಿಮೆಯಾಗುವುದು. ಏಕಾಗ್ರತೆಯೆಂದರೆ ಇದೇ. ಕಾಲವನ್ನು ಕಡಿಮೆ ಮಾಡುವುದಕ್ಕಾಗಿ, ಗಹನ ವಿಷಯಗಳನ್ನು ರಕ್ತಗತಮಾಡಿಕೊಳ್ಳುವ ಶಕ್ತಿಯನ್ನು ತೀವ್ರಗೊಳಿಸುವುದು. ಏಕಾಗ್ರತೆಯ ಶಕ್ತಿಯನ್ನು ಹೇಗೆ ಹೊಂದಬಹುದೆಂಬುದನ್ನು ಬೋಧಿಸುವ ಶಾಸ್ತ್ರವೇ ರಾಜಯೋಗ. 

ಭೂತವಾದಕ್ಕೂ ಪ್ರಾಣಾಯಾಮಕ್ಕೂ ಏನು ಸಂಬಂಧ? ಭೂತವಾದವೂ ಪ್ರಾಣಾಯಾಮದ ಒಂದು ರೂಪ. ದೇಹವನ್ನು ಬಿಟ್ಟು ಆತ್ಮಗಳು ಇರುವುದು ಸತ್ಯವಾದರೆ ನಾವು ನೋಡುವುದಕ್ಕೆ ಆಗದ, ಮುಟ್ಟುವುದಕ್ಕೆ ಆಗದ, ಅನುಭವಿಸುವುದಕ್ಕೆ ಆಗದ, ಸಹಸ್ರಾರು ಆತ್ಮಗಳು ನಮ್ಮ ಮಧ್ಯದಲ್ಲಿ ಇರಬಹುದು. ನಾವು ಅವುಗಳ ದೇಹಗಳ ಮೂಲಕ ನಡೆದಾಡುತ್ತ ಇದ್ದರೂ, ಅವು ನಮ್ಮನ್ನು ನೋಡದೆ, ಅನುಭವಿಸದೆ ಇರಬಹುದು. ಇದು ವೃತ್ತದೊಳಗೊಂದು ವೃತ್ತದಂತೆ, ವಿಶ್ವದೊಳಗೊಂದು ವಿಶ್ವದಂತೆ ಇದೆ. ನಮಗೆ ಪಂಚೇಂದ್ರಿಯಗಳಿವೆ. ಒಂದು ನಿರ್ದಿಷ್ಟವಾದ ಸ್ಪಂದನ ಸ್ಥಿತಿಯಲ್ಲಿ ನಮ್ಮ ಪ್ರಾಣವನ್ನು ವ್ಯಕ್ತಪಡಿಸುತ್ತೇವೆ. ಅದೇ ಸ್ಪಂದನ ಸ್ಥಿತಿಯಲ್ಲಿರುವ ಜೀವಿಗಳೆಲ್ಲ ಒಂದು ಮತ್ತೊಂದಕ್ಕೆ ಕಾಣುವುವು. ಆದರೆ, ಅದಕ್ಕಿಂತ ಮೇಲಿನ ಸ್ಪಂದನ ಸ್ಥಿತಿಯಲ್ಲಿರುವ ಜೀವಿಗಳಿದ್ದರೆ, ಅವುಗಳು ಕಾಣುವುದಿಲ್ಲ. ಜ್ಯೋತಿಯ ಪ್ರಕಾಶವನ್ನು ನಮ್ಮ ಕಣ್ಣಿಗೆ ಕಾಣದಷ್ಟು ಹೆಚ್ಚು ಮಾಡಬಹುದು. ಆದರೆ ಅಂತಹ ಜ್ಯೋತಿಗಳನ್ನು ಕೂಡ ನೋಡಬಲ್ಲ ಅತಿ ಸೂಕ್ಷ್ಮವಾದ ಕಣ್ಣುಗಳುಳ್ಳವರು ಇರಬಹುದು. ಜ್ಯೋತಿಯ ಪ್ರಕಾಶವು ಬಹಳ ಕಡಮೆಯಾದರೂ ಅದು ನಮಗೆ ಕಾಣುವುದಿಲ್ಲ. ಆದರೂ ಅಂತಹ ಬೆಳಕನ್ನೂ ಕೂಡ ನೋಡಬಲ್ಲ ಬೆಕ್ಕು ಮತ್ತು ಗೂಬೆಗಳಂತಹ ಪ್ರಾಣಿಗಳು ಇರಬಹುದು. ನಮ್ಮ ದೃಷ್ಟಿಗೆ ಗೋಚರವಾಗಿರುವ ಕ್ಷೇತ್ರವು ಪ್ರಾಣದಲ್ಲಿ ಒಂದು ಸ್ಪಂದನ ಸ್ಥಿತಿಗೆ ಸೇರಿರುವುದು. ಉದಾಹರಣೆಗೆ ಈಗ ನಮ್ಮ ವಾತಾವರಣವನ್ನು ತೆಗೆದುಕೊಳ್ಳಿ. ಇದರಲ್ಲಿ ಒಂದು ಪದರದ ಮೇಲೆ ಮತ್ತೊಂದು ಪದರವಿದೆ. ಭೂಮಿಗೆ ಹತ್ತಿರದಲ್ಲಿರುವ ಪದರವು ಅದಕ್ಕಿಂತ ಮೇಲೆ ಇರುವ ಪದರಕ್ಕಿಂತ ಸಾಂದ್ರವಾಗಿದೆ. ನೀವು ಮೇಲೆ ಹೋದಂತೆಲ್ಲ ವಾಯುಗುಣದ ಸಾಂದ್ರತೆಯು ಕ್ರಮೇಣ ಕಡಮೆಯಾಗುತ್ತಾ ಬರುವುದು. ಅಥವಾ ಸಾಗರದ ಉದಾಹರಣೆಯನ್ನು ತೆಗೆದುಕೊಳ್ಳಿ. ನೀವು ಆಳಕ್ಕೆ ಹೋದಂತೆ ನೀರಿನ ಒತ್ತಡ ಹೆಚ್ಚಾಗುವುದು. ಸಾಗರದ ಅಡಿಯಲ್ಲಿ ವಾಸಿಸುವ ಪ್ರಾಣಿಗಳು ಎಂದಿಗೂ ಮೇಲಕ್ಕೆ ಬರಲಾರವು. ಒಂದು ವೇಳೆ ಬಂದರೆ ಚೂರುಚೂರಾಗಿ ಒಡೆದುಹೋಗುವುವು. 

ಆಕಾಶದ ಮಹಾಸಾಗರದಂತೆ ಈ ವಿಶ್ವವನ್ನು ಚಿತ್ರಿಸಿಕೊಳ್ಳಿ. ಅದರಲ್ಲಿ ಪ್ರಾಣದ ಪ್ರಭಾವದಿಂದ ಬೇರೆಬೇರೆಯ ಸ್ಪಂದನಗಳನ್ನು ಹೊಂದಿದ ಅನೇಕ ಪದರಗಳು ಇರುವುವು. ಕೇಂದ್ರದಿಂದ ದೂರವಾದಷ್ಟೂ ಸ್ಪಂದನ ಕಡಮೆಯಾಗುವುದು. ಅದಕ್ಕೆ ಸಮೀಪವಾದಷ್ಟೂ ಸ್ಪಂದನ ಹೆಚ್ಚಾಗುವುದು. ಅದರಲ್ಲಿ ಒಂದೊಂದು ವಿಧದ ಸ್ಪಂದನ ಒಂದೊಂದು ಸ್ತರವಾಗುವುದು. ಅನಂತರ ಬೇರೆ ಬೇರೆ ಸ್ಪಂದನಗಳುಳ್ಳ ಕ್ಷೇತ್ರಗಳನ್ನೇ ಬೇರೆ ಬೇರೆ ಲೋಕಗಳಾಗಿ ವಿಂಗಡಿಸಿದೆ ಎಂದು ಇಟ್ಟುಕೊಳ್ಳೋಣ. ಕೆಲವು ಲಕ್ಷ ಮೈಲಿಗಳು ಒಂದು ಸ್ಪಂದನದ ಲೋಕ, ಅನಂತರ ಕೆಲವು ಲಕ್ಷ ಮೈಲಿಗಳು ಇನ್ನೊಂದು ಸ್ಪಂದನದ ಲೋಕ, ಇತ್ಯಾದಿ. ಆದುದರಿಂದ ಒಂದು ಸ್ಪಂದನದ ಕ್ಷೇತ್ರದಲ್ಲಿರುವವರು, ಅಲ್ಲಿರುವವರನ್ನೆಲ್ಲ ತಿಳಿಯಬಹುದು. ಆದರೆ ಅದರಿಂದ ಮೇಲಿರುವವರನ್ನು ಗುರುತಿಸಲಾರರು. ಆದರೂ ದೂರದರ್ಶಕ ಯಂತ್ರಗಳ, ಸೂಕ್ಷ್ಮದರ್ಶಕ ಯಂತ್ರಗಳ ಸಹಾಯದಿಂದ ನಮ್ಮ ದೃಷ್ಟಿಯ ಮೇರೆಯನ್ನು ಹೆಚ್ಚು ಮಾಡಿಕೊಳ್ಳುವಂತೆ ಯೋಗದ ಮೂಲಕ ಮತ್ತೊಂದು ಸ್ಪಂದನದ ಕ್ಷೇತ್ರಕ್ಕೆ ಬಂದು, ಅಲ್ಲಿ ಏನಾಗುತ್ತಿದೆ ಎಂಬುದನ್ನು ನೋಡಬಹುದು. ನಮಗೆ ಕಾಣಿಸದ ವ್ಯಕ್ತಿಗಳಿಂದ ಈ ಕೋಣೆ ತುಂಬಿದೆ ಎಂದು ಇಟ್ಟುಕೊಳ್ಳೋಣ. ಅವರ ಪ್ರಾಣ ಒಂದು ಸ್ಪಂದನದಲ್ಲಿದೆ, ನಮ್ಮ ಪ್ರಾಣ ಒಂದು ಸ್ಪಂದನದಲ್ಲಿದೆ. ಬಹುಶಃ ಅವರದು ಬಹಳ ವೇಗವಾದ ಸ್ಪಂದನವಾಗಿರಬಹುದು; ನಮ್ಮದು ಅದಕ್ಕೆ ವಿರೋಧವಾಗಿರಬಹುದು. ಅವರು ಮತ್ತು ನಾವು ಪ್ರಾಣದಿಂದಲೇ ಆದವರು, ಎಲ್ಲರೂ ಒಂದೇ ಪ್ರಾಣಸಾಗರದ ವಿವಿಧ ಭಾಗಗಳು, ನಾವು ಅವರು ಬೇರೆ ಆಗಿರುವುದು ಸ್ಪಂದನದ ತಾರತಮ್ಯದಿಂದ. ನಾನು ವೇಗವಾದ ಸ್ಪಂದನಕ್ಕೆ ಏರಿದರೆ ಈ ಸ್ತರ ತಕ್ಷಣವೇ ಬದಲಾಯಿಸುವುದು. ನಾನು ಇನ್ನು ನಿಮ್ಮನ್ನು ನೋಡುವುದಿಲ್ಲ. ನೀವು ಮಾಯವಾಗಿ ಅವರು ಕಾಣುತ್ತಾರೆ. ಇದು ನಿಜವೆಂದು ನಿಮ್ಮಲ್ಲಿ ಕೆಲವರಿಗೆ ಬಹುಶಃ ಗೊತ್ತಿರಬಹುದು. ಮೇಲಿನ ಸ್ಪಂದನಕ್ಕೆ ಮನಸ್ಸನ್ನು ಏರಿಸುವ ಈ ವಿಷಯವೆಲ್ಲ ಯೋಗ ಅಥವಾ ಸಮಾಧಿ ಎಂಬ ಒಂದು ಪದದಲ್ಲಿ ಅಡಗಿದೆ. ಈ ಉನ್ನತ ಸ್ಪಂದನದ ಸ್ಥಿತಿ, ಮಾನಸಿಕ ಅತೀಂದ್ರಿಯ ಅವಸ್ಥೆಗಳು ಇವೆಲ್ಲ ಸಮಾಧಿ ಎಂಬ ಒಂದು ಶಬ್ದದಲ್ಲಿ ಅಡಗಿದೆ. ಸಮಾಧಿಯ ಕೆಳಮಟ್ಟದ ಪದ್ಧತಿಯು ಈ ವ್ಯಕ್ತಿಗಳ ನೋಟವನ್ನು ನಮಗೆ ಒದಗಿಸುತ್ತದೆ. ಈ ವಿಧವಾದ ವ್ಯಕ್ತಿಗಳೆಲ್ಲ ಯಾವ ಒಂದು ವಸ್ತುವಿನಿಂದ ಆಗಿದ್ದಾರೆಯೋ, ಆ ಪರಮ ಸತ್ಯವನ್ನು ನೋಡುವುದೇ ಅತ್ಯುನ್ನತ ಸಮಾಧಿಯ ಅವಸ್ಥೆ. ಆ ಸತ್ಯವನ್ನು ತಿಳಿದರೆ ಪ್ರಪಂಚದಲ್ಲಿರುವ ಸತ್ಯವನ್ನೆಲ್ಲ ತಿಳಿದಂತೆ. 

ಆದಕಾರಣ ಭೂತವಾದದಲ್ಲಿ ಯಾವುವು ಸತ್ಯವಾಗಿರುವುವೋ ಅವೆಲ್ಲವೂ ಪ್ರಾಣಾಯಾಮದಲ್ಲಿ ಅಡಗಿವೆ. ಅದರಂತೆಯೇ ಎಲ್ಲಿಯಾದರೂ, ಯಾವ ಮತವೇ ಆಗಲಿ, ಯಾವ ಪಂಗಡವೇ ಆಗಲಿ, ಏನಾದರೂ ರಹಸ್ಯವನ್ನು ಹುಡುಕುತ್ತಿದ್ದರೆ, ಅವರು ನಿಜವಾಗಿ ಪ್ರಯತ್ನಿಸುತ್ತಿರುವುದು, ಈ ಪ್ರಾಣವನ್ನು ನಿಗ್ರಹಿಸುವ ಯೋಗವನ್ನು. ಎಲ್ಲಿಯಾದರೊಂದು ಅದ್ಭುತ ಶಕ್ತಿಪ್ರಭಾವವಿದ್ದರೆ ಅದು ಈ ಪ್ರಾಣದ ಆವಿರ್ಭಾವ. ಭೌತವಿಜ್ಞಾನಗಳನ್ನು ಕೂಡ ಪ್ರಾಣಾಯಾಮದಲ್ಲಿ ಸೇರಿಸಬಹುದು. ಆವಿಯ ಯಂತ್ರವನ್ನು ಯಾವುದು ಚಲಿಸುವುದು? ಆವಿಯ ಮೂಲಕ ಕೆಲಸ ಮಾಡುವ ಪ್ರಾಣ. ವಿದ್ಯುಚ್ಛಕ್ತಿ ಮುಂತಾದುವೆಲ್ಲ ಪ್ರಾಣವಲ್ಲದೆ ಮತ್ತೇನು? ಭೌತವಿಜ್ಞಾನ ಎಂದರೇನು? ಬಾಹ್ಯವಸ್ತುವಿನ ಮೂಲಕ ನಡೆಯುವ ಪ್ರಾಣಾಯಾಮ. ಮಾನಸಿಕ ಶಕ್ತಿಯಂತೆ ತೋರುವ ಪ್ರಾಣವನ್ನು ಮಾನಸಿಕ ಶಕ್ತಿಯ ಮೂಲಕವಾಗಿ ಮಾತ್ರ ನಿಗ್ರಹಿಸಬಹುದು. ಪ್ರಾಣದ ಬಾಹ್ಯ ಅಭಿವ್ಯಕ್ತಿಯನ್ನು ಬಾಹ್ಯ ಸಾಧನದ ಮೂಲಕ ನಿಗ್ರಹಿಸುವುದಕ್ಕೆ ಭೌತವಿಜ್ಞಾನವೆಂದು ಹೆಸರು. ಅಂತರಂಗದಲ್ಲಿ ಮಾನಸಿಕ ಶಕ್ತಿಯಾಗಿರುವ ಪ್ರಾಣವನ್ನು ಮಾನಸಿಕ ಶಕ್ತಿ ಮೂಲಕ ನಿಗ್ರಹಿಸುವುದೇ ರಾಜಯೋಗ.

\chapter{ಮಾನಸಿಕ ಪ್ರಾಣ}%%೨೭

ಯೋಗಿಗಳ ಅಭಿಪ್ರಾಯದಂತೆ ನಮ್ಮ ಬೆನ್ನುಮೂಳೆಯಲ್ಲಿ ಇಡ ಮತ್ತು ಪಿಂಗಳ ಎಂಬ ಎರಡು ನರಗಳ ಪ್ರವಾಹವಿದೆ. ಬೆನ್ನಿನ ಮೂಳೆಯ ಮಧ್ಯದಲ್ಲಿ ಸುಷುಮ್ನಾ ಎಂಬ ನಾಳೆ ಇದೆ. ಈ ನಾಲೆಯ ಕೆಳಭಾಗದಲ್ಲಿ ಯೋಗಿಗಳು ಕರೆಯುವ ಕುಂಡಲಿನಿಯ ಪದ್ಮವಿರುವುದು. ಅದು ತ್ರಿಕೋಣಾಕಾರವಾಗಿದೆ ಎಂದು ವರ್ಣಿಸುತ್ತಾರೆ. ಯೋಗಿಗಳ ಸಾಂಕೇತಿಕ ಭಾಷೆಯಲ್ಲಿ ಕುಂಡಲಿನಿ ಎಂಬ ಶಕ್ತಿ ಅಲ್ಲಿ ಸುಪ್ತವಾಗಿರುವುದು. ಕುಂಡಲಿನಿ ಜಾಗೃತವಾದ ಮೇಲೆ ಈ ಬೆನ್ನುಮೂಳೆಯ ಮಧ್ಯದಲ್ಲಿರುವ ನಾಲೆಯ ಮೂಲಕ ಮೇಲೆ ಹೋಗಲು ಯತ್ನಿಸುತ್ತದೆ. ಅದು ಹಂತ ಹಂತವಾಗಿ ಮೇಲೆ ಏರಿದಂತೆ ಮನಸ್ಸಿನ ಅನೇಕ ಪದರಗಳು ಕ್ರಮೇಣ ಪ್ರಕಾಶಕ್ಕೆ ಬರುವುವು. ಆಗ ಯೋಗಿಗೆ ನಾನಾ ವಿಧವಾದ ದರ್ಶನಗಳಾಗುವುವು. ಹಲವು ವಿಧದ ಶಕ್ತಿಗಳೂ ಪ್ರಾಪ್ತವಾಗುತ್ತವೆ. ಅದು ಮೆದುಳನ್ನು ಸೇರಿದ ಮೇಲೆ, ಯೋಗಿಯು ದೇಹದಿಂದ ಮತ್ತು ಮನಸ್ಸಿನಿಂದ ಸಂಪೂರ್ಣ ಬೇರೆಯಾಗಿ ಆತ್ಮ ಬಂಧನದಿಂದ ಮುಕ್ತವಾಗುವುದು. ಬೆನ್ನುಮೂಳೆಯು ವಿಚಿತ್ರವಾಗಿ ರಚಿತವಾಗಿದೆ ಎಂಬುದು ನಮಗೆ ಗೊತ್ತಿದೆ. ನಾವು ಎಂಟನ್ನು ಮಲಗಿಸಿದರೆ (\enginline{∞}) ಅದಕ್ಕೆ ಮಧ್ಯದಲ್ಲಿ ಸೇರಿದ ಎರಡು ಭಾಗಗಳಿವೆ. ನೀವು ಎಂಟರ ಮೇಲೆ ಎಂಟು ರಾಶಿ ಹಾಕಿರುವುದನ್ನು ಕಲ್ಪಿಸಿಕೊಂಡರೆ ಅದೇ ಬೆನ್ನು ಮೂಳೆಯಾಗುವುದು. ಎಡಗಡೆಯದೇ ಇಡ, ಬಲಗಡೆಯದೇ ಪಿಂಗಳ, ಮಧ್ಯದಲ್ಲಿ ಮೇಲಿನವರೆವಿಗೂ ಹರಿಯುವ ಟೊಳ್ಳು ನಾಳವೇ ಸುಷುಮ್ನ. ಮಿದುಳುಬಳ್ಳಿಯು ಬೆನ್ನು ಹುರಿಯಲ್ಲಿ ಕೊನೆಗೊಂಡ ಮೇಲೆ, ಅಲ್ಲಿಂದ ಕೆಳಗೆ ಸೂಕ್ಷ್ಮ ನರವು ಕೆಳಕ್ಕೆ ಇಳಿಯುತ್ತದೆ. ಅದರೊಳಗೆ ಅದಕ್ಕಿಂತಲೂ ಸೂಕ್ಷ್ಮವಾದ ಸುಷುಮ್ನಾ ಕಾಲುವೆಯು ಇರುವುದು. ತ್ರಿಕಾಸ್ಥಿ  ನರಜಾಲ ಎಂಬಲ್ಲಿ ಈ ಕಾಲುವೆ ಕೊನೆಗೊಳ್ಳುತ್ತದೆ. ಅದು ಆಧುನಿಕ ಶರೀರ ಶಾಸ್ತ್ರಜ್ಞರ ಅಭಿಪ್ರಾಯದಂತೆ ತ್ರಿಕೋಣಾಕಾರವಾಗಿದೆ. ಮಿದುಳು ಬಳ್ಳಿಯ ಉದ್ದಕ್ಕೂ ಬರುವ ಅನೇಕ ನರಗ್ರಂಥಿಗಳು ಯೋಗಿಯು ಹೇಳುವ ಕಮಲಗಳನ್ನು ಸೂಚಿಸುತ್ತವೆ. 


ಕೆಳಗಿರುವ ಮೂಲಾಧಾರದಿಂದ ಹಿಡಿದು ಮೇಲಿರುವ ಸಹಸ್ರದಳಪದ್ಮದವರೆಗೆ\break ಯೋಗಿಯು ಅನೇಕ ಚಕ್ರ ಅಥವಾ ಕಮಲಗಳನ್ನು ಕಲ್ಪಿಸಿಕೊಳ್ಳುತ್ತಾನೆ. ಆದಕಾರಣ ಈ ನರಗ್ರಂಥಿಗಳು ವಿವಿಧ ಚಕ್ರಗಳನ್ನು ಪ್ರತಿನಿಧಿಸುತ್ತವೆ ಎಂದು ಕಲ್ಪಿಸಿಕೊಂಡರೆ ಆಧುನಿಕ ಶರೀರಶಾಸ್ತ್ರದ ಪ್ರಕಾರ ಯೋಗಿಯ ಭಾವನೆಗಳನ್ನು ನಾವು ಸುಲಭವಾಗಿ ಗ್ರಹಿಸಬಹುದು. ನಮಗೆ ತಿಳಿದಿರುವಂತೆ ಈ ನರಗಳ ಪ್ರವಾಹದಲ್ಲಿ ಎರಡು ವಿಧ: ಒಂದು ಸುದ್ದಿಯನ್ನು ಮಿದುಳಿಗೆ ಒಯ್ಯುವುದು, ಮತ್ತೊಂದು ಮೆದುಳಿನಿಂದ ಆಯಾ ಭಾಗಗಳಿಗೆ ಸುದ್ದಿಯನ್ನು ತರುವುದು. ಅದರಲ್ಲಿ ಒಂದು ಜ್ಞಾನೇಂದ್ರಿಯ, ಮತ್ತೊಂದು ಕರ್ಮೇಂದ್ರಿಯ. ಒಂದು ಕೇಂದ್ರಾಕರ್ಷಕ, ಮತ್ತೊಂದು ಕೇಂದ್ರಾಪಕರ್ಷಕ. ಒಂದು ಮಿದುಳಿಗೆ ಸುದ್ದಿಯನ್ನು ಒಯ್ಯುವುದು. ಮತ್ತೊಂದು ಮಿದುಳಿನಿಂದ ಆಯಾ ಭಾಗಗಳಿಗೆ ಸುದ್ದಿಯನ್ನು ಒಯ್ಯುವುದು. ಈ ಕ್ರಿಯೆಗಳೆಲ್ಲ ಮಿದುಳಿಗೆ ಸಂಬಂಧ ಪಟ್ಟಿವೆ. ಮುಂದಿನ ವಿಷಯವನ್ನು ತಿಳಿದುಕೊಳ್ಳಬೇಕಾದರೆ ನಾವು ಹಲವು ಇತರ ವಿಷಯಗಳನ್ನು ಜ್ಞಾಪಕದಲ್ಲಿಡಬೇಕು. ಈ ಮಿದುಳುಬಳ್ಳಿ ಮಿದುಳಿನಲ್ಲಿ ಮೆಡುಲ \enginline{(Medulla)} ಎಂಬಲ್ಲಿ ಒಂದು ಗ್ರಂಥಿಯಂತೆ ಕೊನೆಗಾಣುವುದು. ಇದು ಮಿದುಳಿಗೆ ಸೇರಿಲ್ಲ; ಆದರೆ ಮಿದುಳಿನಲ್ಲಿರುವ ಒಂದು ದ್ರವದಲ್ಲಿ ತೇಲುತ್ತಿದೆ ಏಕೆಂದರೆ ಮಿದುಳಿಗೆ ಏನಾದರೂ ಪೆಟ್ಟುಬಿದ್ದರೆ, ಆ ವೇಗವು ದ್ರವದಲ್ಲಿ ವ್ಯಯವಾಗಿ ಮಿದುಳಿಗೆ ಏಟುಬೀಳದೆ ಇರುವುದು. ಈ ವಿಷಯವನ್ನು ನಾವು ಮುಖ್ಯವಾಗಿ ನೆನಪಿನಲ್ಲಿಡಬೇಕು. ಎರಡನೆಯದಾಗಿ, ಇರುವ ಚಕ್ರಗಳ ಪೈಕಿ ಮೂರನ್ನು, ಎಂದರೆ ಮೂಲಾಧಾರ (ಕೆಳಗಿರುವುದು) ಸಹಸ್ರಾರ (ಮಿದುಳಿನಲ್ಲಿರುವ ಸಹಸ್ರದಳದ ಪದ್ಮ ಮತ್ತು ಮಣಿಪೂರ (ಹೊಕ್ಕುಳಿನ ಸಮೀಪದಲ್ಲಿರುವುದು) – ಇವನ್ನು ಜ್ಞಾಪಕದಲ್ಲಿಡಬೇಕು. 



ಅನಂತರ ಭೌತಶಾಸ್ತ್ರದಿಂದ ನಾವು ಒಂದು ವಿಷಯವನ್ನು ತೆಗೆದುಕೊಳ್ಳೋಣ. ವಿದ್ಯುತ್ತು ಮತ್ತು ಅದಕ್ಕೆ ಸಂಬಂಧಪಟ್ಟ ಅನೇಕ ಶಕ್ತಿಗಳನ್ನು ನಾವು ಕೇಳಿರುವೆವು. ವಿದ್ಯುತ್​ ಶಕ್ತಿ ಎಂದರೆ ಏನೆಂದರೆ ಯಾರಿಗೂ ಗೊತ್ತಿಲ್ಲ. ಆದರೆ ಅದು ನಮಗೆ ತಿಳಿದಿರುವ ಮಟ್ಟಿಗೆ ಒಂದು ವಿಧವಾದ ಚಲನೆ. ಪ್ರಪಂಚದಲ್ಲಿ ಹಲವು ವಿಧದ ಚಲನೆಗಳಿವೆ. ಅದಕ್ಕೂ ವಿದ್ಯುತ್​ಶಕ್ತಿಗೂ ಏನು ವ್ಯತ್ಯಾಸ? ಬಹುಶಃ ಈ ಮೇಜು ಚಲಿಸುತ್ತಿದೆ ಎಂದು ಭಾವಿಸೋಣ –ಎಂದರೆ ಯಾವ ಕಣಗಳಿಂದ ಈ ಮೇಜು ಆಗಿದೆಯೊ ಅವುಗಳು ಅನೇಕ ಕಡೆ ಚಲಿಸುತ್ತಿವೆ. ಅವುಗಳೆಲ್ಲವನ್ನೂ ಒಂದು ಕಡೆ ಚಲಿಸುವಂತೆ ಮಾಡಿದರೆ ಅದೇ ವಿದ್ಯುತ್​ಶಕ್ತಿ. ವಿದ್ಯುತ್​ ಪ್ರವಾಹವು ಒಂದು ವಸ್ತುವಿನ ಕಣಗಳನ್ನು ಒಂದೇ ಕಡೆ ಹರಿಯುವಂತೆ ಮಾಡುವುದು. ಒಂದು ಕೋಣೆಯೊಳಗಿರುವ ವಾಯು ಕಣಗಳನ್ನೆಲ್ಲ ಒಂದು ಕಡೆ ಹೋಗುವಂತೆ ಮಾಡಿದರೆ, ಅದು ಆ ಕೋಣೆಯನ್ನೇ ವಿದ್ಯುತ್​ ಶಕ್ತಿಯನ್ನು ಉತ್ಪತ್ತಿ ಮಾಡುವ ಯಂತ್ರವನ್ನಾಗಿ ಮಾಡುವುದು. ಶರೀರಶಾಸ್ತ್ರದ ಮತ್ತೊಂದು ವಿಷಯವನ್ನು ನಾವು ಜ್ಞಾಪಕದಲ್ಲಿಡಬೇಕು. ಶ್ವಾಸೋಚ್ಛ್ವಾಸಗಳನ್ನು ವ್ಯವಸ್ಥೆಗೊಳಿಸುವ ಕೇಂದ್ರಕ್ಕೆ ನರಗಳ ಕೇಂದ್ರದ ಮೇಲೆ ಒಂದು ವಿಧದ ನಿಯಂತ್ರಣ ಇದೆ. 



ಲಯಬದ್ಧವಾಗಿ ಉಸಿರಾಡುವುದನ್ನು ಏತಕ್ಕೆ ಅಭ್ಯಾಸ ಮಾಡುತ್ತಾರೆ ಎಂಬುದನ್ನು ನಾವೀಗ ನೋಡೋಣ. ಮೊದಲನೆಯದಾಗಿ ಇದರಿಂದ ದೇಹದಲ್ಲಿರುವ ಕಣಗಳಿಗೆಲ್ಲಾ ಒಂದೇ ಕಡೆ ಹರಿಯುವ ಸ್ವಭಾವ ಬರುವುದು. ಮನಸ್ಸು ಇಚ್ಛಾಶಕ್ತಿಯಾಗಿ ಬದಲಾವಣೆ ಹೊಂದಿದ ಮೇಲೆ, ನರಪ್ರವಾಹವು ವಿದ್ಯುತ್​ ಶಕ್ತಿಗೆ ಸರಿಸಮನಾದ ಒಂದು ಶಕ್ತಿಯಾಗಿ ರೂಪಾಂತರ ಹೊಂದುವುದು. ಏಕೆಂದರೆ ವಿದ್ಯುತ್​ ಶಕ್ತಿಯ ಪ್ರಭಾವಕ್ಕೆ ಒಳಪಟ್ಟ ನರಗಳು ಧ್ರುವತ್ವವನ್ನು \enginline{(Polarity)} ಪಡೆಯುತ್ತವೆ. ಇಚ್ಛಾ ಶಕ್ತಿಯು ನರಗಳ ಶಕ್ತಿಯಾಗಿ ರೂಪಾಂತರ ಹೊಂದಿದ ಮೇಲೆ ಅದು ಒಂದು ವಿಧದ ವಿದ್ಯುತ್​ ಶಕ್ತಿಯಂತೆ ಬದಲಾವಣೆ ಹೊಂದುತ್ತದೆ ಎಂಬುದು ಇದರಿಂದ ಕಂಡುಬರುತ್ತದೆ. ದೇಹದ ಚಲನೆಗಳೆಲ್ಲ ಒಂದು ರೀತಿಯಾಗಿ ಲಯಬದ್ಧವಾದ ಮೇಲೆ, ದೇಹವು ಒಂದು ಪ್ರಚಂಡ ಇಚ್ಛಾಶಕ್ತಿಯ ಬ್ಯಾಟರಿಯಂತೆ ಆಗುವುದು. ಯೋಗಿಗೆ ಬೇಕಾಗುವುದೇ ಈ ಪ್ರಚಂಡ ಇಚ್ಛಾಶಕ್ತಿ. ಆದಕಾರಣ ಇದು ಶ್ವಾಸೋಚ್ಛ್ವಾಸ ಸಾಧನೆಯ ಶಾರೀರಿಕ ದೃಷ್ಟಿಯ ವಿವರಣೆ. ಇದು ನಮ್ಮ ದೇಹಕ್ಕೆ ಲಯಬದ್ಧವಾದ ಚಲನೆಯನ್ನು ತರುವುದು, ಮತ್ತು ಶ್ವಾಸೋಚ್ಛ್ವಾಸಗಳ ಕೇಂದ್ರದ ಮೂಲಕ ಉಳಿದ ಕೇಂದ್ರಗಳನ್ನು ಸ್ವಾಧೀನಕ್ಕೆ ತರಲು ನಮಗೆ ಸಹಾಯ ಮಾಡುತ್ತದೆ. ಪ್ರಾಣಾಯಾಮದ ಗುರಿಯು ಮೂಲಾಧಾರದಲ್ಲಿ ಸುಪ್ತವಾಗಿರುವ ಕುಂಡಲಿನಿ ಶಕ್ತಿಯನ್ನು ಜಾಗೃತಗೊಳಿಸುವುದು. 



ನಾವು ನೋಡುವ, ಕಲ್ಪಿಸುವ ಕನಸು ಕಾಣುವ ಎಲ್ಲವೂ ದೇಶಕ್ಕೆ ಸಂಬಂಧಿಸಿದೆ. ಈ ದೇಶವನ್ನೆ ಮಹಾಕಾಶ ಅಥವಾ ಭೌತಿಕ ಆಕಾಶ ಎಂದು ಕರೆಯುತ್ತಾರೆ. ಯೋಗಿಯು ಮತ್ತೊಬ್ಬರ ಆಲೋಚನೆಯನ್ನು ಓದುವಾಗ, ಅಥವಾ ಅತೀಂದ್ರಿಯ ವಸ್ತುವನ್ನು ನೋಡುವಾಗ, ಚಿತ್ತಾಕಾಶವೆಂಬ ಮತ್ತೊಂದು ಆಕಾಶದಲ್ಲಿ ನೋಡುವನು. ಗ್ರಹಣಶಕ್ತಿಯು ವಸ್ತುಹೀನವಾಗಿ, ಆತ್ಮವು ಸ್ವಯಂ ಜ್ಯೋತಿಯಿಂದ ಬೆಳಗಿದಾಗ ಅದಕ್ಕೆ ಚಿದಾಕಾಶ ಅಥವಾ ಜ್ಞಾನಾಕಾಶವೆಂದು ಹೆಸರು. ಕುಂಡಲಿನಿಯು ಜಾಗೃತವಾಗಿ ಸುಷುಮ್ನಾ ಕಾಲುವೆಯನ್ನು ಹೊಕ್ಕಮೇಲೆ ಗ್ರಹಣಶಕ್ತಿಯೆಲ್ಲ ಚಿತ್ತಾ ಕಾಶದಲ್ಲಿರುವುದು. ಮಿದುಳಿಗೆ ಪ್ರವೇಶಿಸುವ ಕಾಲುವೆಯ ಕೊನೆಯನ್ನು ಸೇರಿದ ಮೇಲೆ, ವಸ್ತುಹೀನ ಗ್ರಹಣಶಕ್ತಿಯು \enginline{(Objectless Perception)} ಚಿದಾಕಾಶದಲ್ಲಿರುವುದು. ವಿದ್ಯುತ್​ಶಕ್ತಿಯ ಉಪಮಾನವನ್ನು ತೆಗೆದುಕೊಳ್ಳೋಣ. ಮನುಷ್ಯನು ಅದನ್ನು ಒಂದು ತಂತಿಯ ಮೂಲಕ ಮಾತ್ರ ಕಳುಹಿಸಬಲ್ಲ. ಆದರೆ ವಿದ್ಯುತ್​ ಶಕ್ತಿಯ ಪ್ರಚಂಡ ಪ್ರವಾಹವನ್ನು ಕಳುಹಿಸುವುದಕ್ಕೆ ಪ್ರಕೃತಿಗೆ ಯಾವ ತಂತಿಯೂ ಬೇಕಾಗಿಲ್ಲ. ತಂತಿಯು ನಿಜವಾಗಿಯೂ ಆವಶ್ಯಕವಲ್ಲ. ಆದರೆ ನಮಗೆ ಅದು ಅನಿವಾರ್ಯವಾದುದರಿಂದ ನಾವು ಅದನ್ನು ಉಪಯೋಗಿಸಬೇಕಾಗಿದೆ ಎಂಬುದನ್ನು ಇದು ತೋರಿಸುತ್ತದೆ. 



ಇದೇ ರೀತಿ ದೇಹದ ಎಲ್ಲಾ ಚಲನೆಗಳೂ ಮತ್ತು ಸಂವೇದನೆಗಳೂ ಈ ನರತಂತುಗಳೆಂಬ ತಂತಿಯ ಮೂಲಕ ಮೆದುಳಿಗೆ ಹೋಗುತ್ತವೆ ಮತ್ತು ಮೆದುಳಿನಿಂದ ಹೊರಕ್ಕೆ ಬರುತ್ತವೆ. ಮಿದುಳು ಬಳ್ಳಿಯಲ್ಲಿರುವ ಜ್ಞಾನೇಂದ್ರಿಯ ಮತ್ತು ಕರ್ಮೇಂದ್ರಿಯಗಳಿಗೆ ಸಂಬಂಧಿಸಿದ ನರಗಳ ಭಾಗವೇ ಯೋಗಿಯು ಹೇಳುವ ಇಡಾ ಮತ್ತು ಪಿಂಗಳ. ಮುಖ್ಯವಾಗಿ ಈ ಮಾರ್ಗಗಳ ಮೂಲಕವೇ, ಮಿದುಳಿಗೆ ಹೋಗುವ ಮತ್ತು ಅಲ್ಲಿಂದ ಸುದ್ದಿಯನ್ನು ತರುವ ಪ್ರವಾಹವು ಹರಿಯುವುದು. ಆದರೆ ಮನಸ್ಸು ಯಾವ ತಂತಿಯೂ ಇಲ್ಲದೆ ಸುದ್ದಿಯನ್ನು ಏಕೆ ಕಳುಹಬಾರದು? ಅಥವಾ ಸ್ವೀಕರಿಸಬಾರದು? ಪ್ರಕೃತಿಯಲ್ಲಿ ಹೀಗಿರುವುದು ನಮಗೆ ಕಾಣುತ್ತದೆ. ನೀವು ಹಾಗೆ ಮಾಡಿದರೆ ಜಡವಸ್ತುವಿನ ಪಾಶದಿಂದ ಮುಕ್ತರಾದಂತೆ ಎಂದು ಯೋಗಿಯು ಹೇಳುತ್ತಾನೆ. ಅದನ್ನು ಹೇಗೆ ಮಾಡುವುದು? ಬೆನ್ನುಮೂಳೆಯ ಮಧ್ಯದಲ್ಲಿರುವ ಸುಷುಮ್ನಾ ಕಾಲುವೆಯಲ್ಲಿ ಶಕ್ತಿಯನ್ನು ಸಂಚರಿಸುವಂತೆ ಮಾಡಿದರೆ, ಸಮಸ್ಯೆ ಬಗೆಹರಿದಂತೆ ಆಯಿತು. ಮನಸ್ಸು ನರಗಳೆಂಬ ಜಾಲವನ್ನು ಮಾಡಿರುವುದು. ಅವುಗಳನ್ನು ಧ್ವಂಸ ಮಾಡಬೇಕು, ಕೆಲಸ ಮಾಡುವುದಕ್ಕೆ ಯಾವ ತಂತಿಗಳ ಆವಶ್ಯಕತೆಯೂ ಇರಕೂಡದು. ಆಗಲೇ ನಮಗೆ ಎಲ್ಲಾ ಜ್ಞಾನ ಬರುವುದು. ಇನ್ನೆಂದಿಗೂ ದೇಹದ ಸಂಬಂಧವಿರುವುದಿಲ್ಲ. ಅದಕ್ಕೇ ಸುಷುಮ್ನಾ ಕಾಲುವೆಯ ಸ್ವಾಧೀನತೆಯನ್ನು ಪಡೆದುಕೊಳ್ಳುವುದು ಅಷ್ಟು ಮುಖ್ಯ ತಂತಿಯಂತೆ ಕೆಲಸ ಮಾಡುವ ಯಾವ ನರಗಳ ಸಹಾಯವೂ ಇಲ್ಲದೆ, ಮಾನಸಿಕ ಶಕ್ತಿಯನ್ನು ಸುಷುಮ್ನಾ ಕಾಲುವೆಯ ಮೂಲಕ ಹಾಯಿಸಿದರೆ ಸಮಸ್ಯೆ ಬಗೆಹರಿದಂತೆ. ಅದನ್ನು ಸಾಧಿಸುವುದು ಕೂಡ ಸಾಧ್ಯವೆಂದು ಯೋಗಿಯು ಸಾರುತ್ತಾನೆ. 


ಸಾಧಾರಣ ಮನುಷ್ಯರಲ್ಲಿ ಈ ಸುಷಮ್ನಾ ಕಾಲುವೆಯ ಕೆಳಭಾಗವು ಮುಚ್ಚಿ ಹೋಗಿರುತ್ತದೆ. ಅದರ ಮೂಲಕ ಯಾವ ಕೆಲಸವೂ ಆಗುವುದಿಲ್ಲ. ಅದನ್ನು ತೆರೆಯುವ ಮತ್ತು ಅದರ ಮೂಲಕ ನರಗಳ ಶಕ್ತಿ ಸಂಚರಿಸುವಂತೆ ಮಾಡುವ ಒಂದು ಸಾಧನೆಯನ್ನು ಯೋಗಿಯು ಹೇಳುತ್ತಾನೆ. ಸಂವೇದನೆಯು ಒಂದು ಕೇಂದ್ರಕ್ಕೆ ಹೋದರೆ ಕೇಂದ್ರದಲ್ಲಿ ಪ್ರತಿಕ್ರಿಯೆಯಾಗುವುದು. ಸ್ವಯಂ ಚಾಲಕ ಕೇಂದ್ರಗಳಲ್ಲಿ \enginline{(Automatic Center)} ಈ ಪ್ರತಿಕ್ರಿಯೆಯ ನಂತರ ಚಲನೆಯುಂಟಾಗುವುದು. ಪ್ರಜ್ಞಾಕೇಂದ್ರದಲ್ಲಾದರೂ ಮೊದಲು ಅರಿವು, ಅನಂತರ ಕ್ರಿಯೆ ಅನುಸರಿಸುವುದು. ಎಲ್ಲಾ ಗ್ರಹಣವೂ ಹೊರಗಿನಿಂದ ಬಂದ ಕ್ರಿಯೆಯು ಪ್ರತಿಕ್ರಿಯೆ. ಹಾಗಾದರೆ ಸ್ವಪ್ನದಲ್ಲಿ ಹೇಗೆ ಗ್ರಹಣ ಉಂಟಾಗುವುದು? ಆಗ ಹೊರಗಿನಿಂದ ಯಾವ ಕ್ರಿಯೆಯೂ ನಡೆಯುವುದಿಲ್ಲ. ಆದಕಾರಣ ಇಂದ್ರಿಯ ಜನ್ಯ ಚಲನೆಗಳೆಲ್ಲ ಎಲ್ಲೋ ಒಂದು ಕಡೆ ಅಡಗಿವೆ. ಉದಾಹರಣೆಗೆ ನಾನೊಂದು ಪಟ್ಟಣವನ್ನು ನೋಡುತ್ತೇನೆ. ಪಟ್ಟಣಕ್ಕೆ ಸಂಬಂಧಪಟ್ಟ ಬಾಹ್ಯವಸ್ತುಗಳ ಮೂಲಕವಾಗಿ ಬಂದ ಸಂವೇದನೆಗಳ ಪ್ರತಿಕ್ರಿಯೆಯು ಪರಿಣಾಮವಾಗಿಯೇ, ನಮಗೆ ಪಟ್ಟಣವೆಂಬ ಭಾವನೆ ಬಂದಿರುವುದು. ಎಂದರೆ ಹೊರಗಿನಿಂದ ಸುದ್ದಿ ತರುವ ನರಗಳು ಮಿದುಳಿನಲ್ಲಿರುವ ಕಣಗಳಲ್ಲಿ ಒಂದು ಚಲನೆಯನ್ನುಂಟುಮಾಡಿದವು. ಆ ನರಗಳ ಚಲನೆಗೆ ಕಾರಣ ಹೊರಗಿರುವ ಪಟ್ಟಣದಲ್ಲಿರುವ ವಸ್ತುಗಳು. ಈಗ ನಾನು ಬಹಳ ದಿನಗಳಾದ ಮೇಲೆಯೂ ಆ ಪಟ್ಟಣವನ್ನು ಜ್ಞಾಪಿಸಿಕೊಳ್ಳಬಲ್ಲೆ. ಈ ನೆನಪೂ ಕೂಡ ಮೊದಲಿನ ಕ್ರಿಯೆಯಂತೆಯೇ, ಆದರೆ ಇದು ಅದಕ್ಕಿಂತ ಸೌಮ್ಯವಾದುದು. ಮಿದುಳಿನಲ್ಲಿ ನೆನಪೆಂಬ ಸೌಮ್ಯ ಚಲನೆಯನ್ನು ಉಂಟುಮಾಡಿದ ಕ್ರಿಯೆ ಎಲ್ಲಿದೆ? ನಿಜವಾಗಿಯೂ ಮೂಲ ಸಂವೇದನೆಯಿಂದ \enginline{(Primary Sensation)} ಅಲ್ಲ. ಆದಕಾರಣ ಎಲ್ಲೋ ಒಂದು ಕಡೆ ಈ ಸಂವೇದನೆಗಳೆಲ್ಲ ಸೇರಿವೆ. ಅವುಗಳ ಕ್ರಿಯೆಯಿಂದ ಸ್ವಪ್ನವೆಂಬ ಸೌಮ್ಯ ಪ್ರತಿಕ್ರಿಯೆ ಬರುವುದು. 

ಈ ಸಂವೇದನಾ ಸಂಸ್ಕಾರಗಳು ಎಲ್ಲಿ ಸೇರಿಕೊಂಡಿರುವುವೊ ಆ ಕೇಂದ್ರವೇ ಮೂಲಾಧಾರ. ಅದೇ ನಮ್ಮ ನೆನಪಿನ ಮೂಲ. ಕ್ರಿಯೆಯ ಗುಪ್ತ ಶಕ್ತಿಯೂ ಕುಂಡಲಿನಿಯೇ ಆಗಿರಬೇಕು. ಮಿದುಳಿನಿಂದ ಅಪ್ಪಣೆ ತೆಗೆದುಕೊಂಡು ಬರುವ ಶಕ್ತಿಯೂ ಕೂಡ ಬಹುಶಃ ಈ ಕೇಂದ್ರದಲ್ಲಿ ಅಡಗಿರುವಂತೆ ತೋರುವುದು. ಏಕೆಂದರೆ ದೀರ್ಘವ್ಯಾಸಂಗ ಅಥವಾ ಬಾಹ್ಯವಸ್ತುವಿನ ಮೇಲೆ ಮನಸ್ಸನ್ನು ಏಕಾಗ್ರಗೊಳಿಸಿದಾಗ ಮೂಲಾಧಾರದ ಚಕ್ರವಿರುವ ಭಾಗ ಬಿಸಿಯಾಗುತ್ತದೆ. ಈ ಸುಪ್ತವಾಗಿರುವ ಶಕ್ತಿಯನ್ನು ಜಾಗೃತಗೊಳಿಸಿ, ಕಾರ್ಯೋನ್ಮುಖ ಮಾಡಿ, ಸ್ವಪ್ರಜ್ಞೆಯಿಂದ ಅದನ್ನು ಸುಷುಮ್ನಾ ಕಾಲುವೆಯ ಮೂಲಕ ಮೇಲೆ ಹೋಗುವಂತೆ ಮಾಡಿದರೆ, ಅದು ಒಂದು ಚಕ್ರದಿಂದ ಮತ್ತೊಂದು ಚಕ್ರಕ್ಕೆ ಹೋಗುತ್ತಿರುವಾಗ, ಪ್ರಚಂಡ ಪ್ರತಿಕ್ರಿಯೆಯನ್ನು ಉಂಟುಮಾಡುವುದು. ಶಕ್ತಿಯ ಅಲ್ಪಭಾಗ ನರತಂತುವಿನ ಮೂಲಕ ಚಲಿಸಿ, ಆಯಾ ಕೇಂದ್ರಗಳಿಂದ ಪ್ರತಿಕ್ರಿಯೆಯನ್ನು ಉಂಟುಮಾಡಿದಾಗ ನಮಗೆ ಕನಸು ಅಥವಾ ಕಲ್ಪನೆ ಉಂಟಾಗುತ್ತದೆ. ದೀರ್ಘ ಅಂತರಂಗ ಧ್ಯಾನದ ಮೂಲಕ ಶೇಖರಿಸಲ್ಪಟ್ಟ ಅಪಾರ ಶಕ್ತಿಯು ಸುಷುಮ್ನಾ ಕಾಲುವೆಯ ಮೂಲಕ ಪ್ರಯಾಣ ಮಾಡಿ ಅನೇಕ ಚಕ್ರಗಳನ್ನು ಸೇರಿದಾಗ ಆಗುವ ಪ್ರತಿಕ್ರಿಯೆಯು ಪ್ರಚಂಡವಾದುದು. ಈ ಪ್ರತಿಕ್ರಿಯೆಯು ಕನಸು ಮತ್ತು ಕಲ್ಪನೆಗಳಿಗಿಂತ ಮೇಲಾದುದು. ಇಂದ್ರಿಯ ಗ್ರಹಣದಿಂದ ಆಗುವ ಪ್ರತಿಕ್ರಿಯೆ\break ಗಿಂತಲೂ ಬಹುಪಾಲು ತೀವ್ರವಾದುದು. ಇದು ಅತೀಂದ್ರಿಯ ಗ್ರಹಣ. ಎಲ್ಲಾ ಸಂವೇದನೆ\break ಗಳ ಕೇಂದ್ರವಾದ ಮಿದುಳನ್ನು ಇದು ತಲುಪಿದಾಗ ಮಿದುಳೆಲ್ಲವೂ ಇದಕ್ಕೆ ಪ್ರತಿಕ್ರಿಯಿಸುವಂತೆ ತೋರುವುದು. ಇದರ ಫಲವೇ ಪೂರ್ಣ ಅಂತರ್ಜ್ಯೋತಿ ಮತ್ತು ಆತ್ಮಾನುಭವ. ಒಂದು ಚಕ್ರದಿಂದ ಮತ್ತೊಂದು ಚಕ್ರಕ್ಕೆ ಈ ಕುಂಡಲಿನಿ ಶಕ್ತಿಯು ಪ್ರಯಾಣ ಮಾಡಿದಂತೆಲ್ಲ ಮನಸ್ಸಿನ ಅನೇಕ ಪದರಗಳು ಪ್ರಕಾಶಕ್ಕೆ ಬಂದಂತೆ ಆಗುವುವು. ಆಗ ಯೋಗಿಗೆ ಈ ಪ್ರಪಂಚದ ಸೂಕ್ಷ್ಮಾವಸ್ಥೆ ಅಥವಾ ಕಾರಣ ಪ್ರಪಂಚವು ಗೋಚರಿಸುವುದು. ಆಗ ಸೃಷ್ಟಿಯ ಕಾರಣವಾದ ಸಂವೇದನೆ ಮತ್ತು ಪ್ರತಿಕ್ರಿಯೆ ಇವುಗಳೆರಡನ್ನೂ ಅವನು ಯಥಾರ್ಥ ಸ್ಥಿತಿಯಲ್ಲಿ ನೋಡುತ್ತಾನೆ. ಆಗ ಜ್ಞಾನವೆಲ್ಲ ಅವನಿಗೆ ಲಭಿಸುವುದು. ಕಾರಣವು ಗೊತ್ತಾದ ಮೇಲೆ ಅದರ ಪರಿಣಾಮದ ಜ್ಞಾನ ಬಂದೇ ಬರುವುದು. 


ಆದಕಾರಣ ಪರಮಾರ್ಥಲಾಭಕ್ಕೆ, ಅತೀಂದ್ರಿಯ ದರ್ಶನಕ್ಕೆ, ಆತ್ಮಾನುಭಾವಕ್ಕೆ, ಕುಂಡಲಿನಿ ಶಕ್ತಿಯನ್ನು ಜಾಗೃತಮಾಡುವುದೊಂದೇ ದಾರಿ. ಇದರ ಜಾಗೃತಿ ನಾನಾ ವಿಧವಾಗಿ ಬರಬಹುದು. ಈಶ್ವರನ ಮೇಲೆ ಭಕ್ತಿ, ಸಿದ್ಧಿಯನ್ನು ಪಡೆದ ಮಹಾನುಭಾವರ ಕೃಪೆ, ಅಥವಾ ತಾತ್ತ್ವಿಕ ವಿಚಾರಶಕ್ತಿ –ಇವುಗಳ ಮೂಲಕವಾಗಿ ಇದು ಜಾಗೃತವಾಗಬಹುದು. ಎಲ್ಲಿಯಾದರೂ ನಾವು ಸಾಧಾರಣವಾಗಿ ಕರೆಯುವ ಅತೀಂದ್ರಿಯ ಶಕ್ತಿ ಅಥವಾ ಜ್ಞಾನ ಪ್ರಕಾಶವಾಗಿದ್ದರೆ, ಅಲ್ಲೆಲ್ಲ ಕುಂಡಲಿನಿ ಶಕ್ತಿಯ ಸ್ವಲ್ಪ ಭಾಗ ಸುಷುಮ್ನಾ ಕಾಲುವೆಗೆ ಸೇರಿರಬೇಕು. ಏಕೆಂದರೆ ಇಂತಹ ಬಹುಪಾಲು ಸನ್ನಿವೇಶದಲ್ಲಿ ಜನರು ತಿಳಿಯದೆಯೆ ಕುಂಡಲಿನಿಯ ಸ್ವಲ್ಪಾಂಶವನ್ನು ಮುಕ್ತಗೊಳಿಸುವ ಕೆಲವು ಸಾಧನೆಯನ್ನು ಮಾಡಿರುತ್ತಾರೆ. ಎಲ್ಲಾ ಪೂಜೆಗಳೂ, ನಮಗೆ ತಿಳಿದೋ ತಿಳಿಯದೆಯೋ, ಈ ಗುರಿಗೆ ಒಯ್ಯುತ್ತವೆ. ಪ್ರಾರ್ಥನೆಗೆ ಉತ್ತರ ಬರುತ್ತಿದೆ ಎನ್ನುವವನಿಗೆ ಇದು ತನ್ನ ಅಂತರಂಗದಿಂದಲೇ ಬರುತ್ತಿದೆ ಎನ್ನುವುದು ತಿಳಿಯದು. ಪ್ರಾರ್ಥನೆಯ ಮೂಲಕ ತನ್ನಿಲ್ಲಿಯೇ ಸುಪ್ತವಾಗಿರುವ ಅನಂತಶಕ್ತಿಯ ಅಂಶವನ್ನು ಜಾಗ್ರತಗೊಳಿಸಿರುತ್ತಾನೆ. ಹೀಗೆ ಅಜ್ಞಾನದಿಂದ ಅಂಜಿಕೆ ಸಂಕಟಗಳಿಗೀಡಾಗಿ ಮಾನವರು ಹಲವು ಹೆಸರುಗಳ ಮೂಲಕ ಯಾರನ್ನು ಪೂಜಿಸುತ್ತಾರೆಯೋ ಅದೇ ಪ್ರತಿಯೊಬ್ಬ ಮಾನವನಲ್ಲಿಯೂ ಸುಪ್ತವಾಗಿರುವ ನಿಜವಾದ ಶಕ್ತಿಯೆಂದು ಯೋಗಿಯು ಜಗತ್ತಿಗೆ ಸಾರುತ್ತಾನೆ. ಅದನ್ನು ನಾವು ಹೇಗೆ ಸಮೀಪಿಸುವುದೆಂದು ತಿಳಿದರೆ ಅದೇ ಅನಂತ ಕಲ್ಯಾಣದ ತವರೂರು. ರಾಜಯೋಗವು ಧರ್ಮದ ಒಂದು ವಿಜ್ಞಾನಶಾಸ್ತ್ರ. ಇದೇ ಎಲ್ಲಾ ಪೂಜೆಗಳ ಪ್ರಾರ್ಥನೆಯ ಬಾಹ್ಯಾಚಾರಗಳ ಮತ್ತು ಅದ್ಭುತಗಳ ವೈಚಾರಿಕ ಹಿನ್ನೆಲೆ.

\chapter{ಮಾನಸಿಕ ಪ್ರಾಣದ ಸ್ವಾಧೀನತೆ}%%೨೮

ನಾವೀಗ ಪ್ರಾಣಾಯಾಮದಲ್ಲಿ ಬರುವ ಸಾಧನೆಯನ್ನು ಕುರಿತು ವಿಚಾರ ಮಾಡಬೇಕು. ಯೋಗಿಗಳ ಅಭಿಪ್ರಾಯದಂತೆ ನಾವು ಮಾಡಬೇಕಾದ ಮೊದಲನೆಯ ಕೆಲಸವೇ ಶ್ವಾಸಕೋಶಗಳ ಚಲನೆಯನ್ನು ಸ್ವಾಧೀನಕ್ಕೆ ತರುವುದು. ನಾವು ಮಾಡಬೇಕಾಗಿರುವುದೇನೆಂದರೆ ನಮ್ಮ ಶರೀರದಲ್ಲಿ ಆಗುತ್ತಿರುವ ಸೂಕ್ಷ್ಮ ಚಲನೆಗಳನ್ನು ತಿಳಿಯಬೇಕು. ನಮ್ಮ ಮನಸ್ಸು ಬಾಹ್ಯಮುಖವಾಗಿದೆ, ಒಳಗೆ ನಡೆಯುತ್ತಿರುವ ಸೂಕ್ಷ್ಮ ಚಲನೆಯ ಅರಿವನ್ನು ಮರೆತಿದೆ. ಅವುಗಳನ್ನು ನಾವು ತಿಳಿಯಬಹುದು ಮತ್ತು ನಿಗ್ರಹಿಸಬಹುದು. ಈ ನರಗಳ ಶಕ್ತಿಯು ದೇಹದಲ್ಲೆಲ್ಲಾ ಸಂಚರಿಸಿ ಪ್ರತಿಯೊಂದು ಮಾಂಸಖಂಡಕ್ಕೂ ಜೀವವನ್ನು ಮತ್ತು ಚೈತನ್ಯವನ್ನು ಒದಗಿಸುತ್ತದೆ. ಆದರೆ ಅದು ನಮಗೆ ಗೋಚರವಾಗದು. ಅದನ್ನು ಬೇಕಾದರೆ ತಿಳಿಯುವಂತೆ ಮಾಡಿಕೊಳ್ಳಬಹುದೆಂದು ಯೋಗಿಯು ಹೇಳುತ್ತಾನೆ. ಅದು ಹೇಗೆ ಸಾಧ್ಯ? ಶ್ವಾಸಕೋಶಗಳ ಚಲನೆಯನ್ನು ನಿಗ್ರಹಿಸುವುದರಿಂದ. ಕೆಲವು ಕಾಲ ನಾವು ಅದನ್ನು ಸಾಧಿಸಿದ ಮೇಲೆ ಇತರ ಸೂಕ್ಷ್ಮ ಚಲನೆಗಳನ್ನು ನಮ್ಮ ಸ್ವಾಧೀನಕ್ಕೆ ತರುವುದು ಸಾಧ್ಯ. 

ಈಗ ಪ್ರಾಣಾಯಾಮದ ಅಭ್ಯಾಸಕ್ಕೆ ಬರುತ್ತೇವೆ. ನೇರವಾಗಿ ಕುಳಿತುಕೊಳ್ಳಬೇಕು. ದೇಹ ನೇರವಾಗಿರಬೇಕು. ಮಿದುಳು ಬಳ್ಳಿ ಕಶೇರುಕದೊಡನೆ ಕೂಡಿಕೊಂಡಿರದಿದ್ದರೂ, ಅದರ ಒಳಗೆ ಇರುವುದು. ನೀವು ಬಾಗಿ ಕುಳಿತರೆ ಅದಕ್ಕೆ ಅಡಚಣೆಯನ್ನು ತರುವಿರಿ. ಆದಕಾರಣ ಅದು ಆರಾಮವಾಗಿರಲಿ. ನೀವು ಯಾವಾಗಲಾದರೂ ಬಾಗಿ ಕುಳಿತುಕೊಂಡು ಧ್ಯಾನಮಾಡಲು ಪ್ರಯತ್ನಿಸಿದರೆ ನಿಮಗೆ ನೀವೇ ತೊಂದರೆ ಮಾಡಿಕೊಳ್ಳುತ್ತೀರಿ. ದೇಹದ ಮೂರು ಭಾಗಗಳು ಎಂದರೆ, ಎದೆ ಕತ್ತು ಮತ್ತು ತಲೆ ಇವು ಒಂದೇ ರೇಖೆಯಲ್ಲಿರಬೇಕು. ಸ್ವಲ್ಪ ಅಭ್ಯಾಸ ಮಾಡಿದರೆ ಇದು ನಮಗೆ ಉಸಿರಾಡುವಷ್ಟು ಸುಲಭವಾಗಿ ಬರುವುದು. ಎರಡನೆಯ ಕೆಲಸವೇ ನರಗಳನ್ನು ಸ್ವಾಧೀನಕ್ಕೆ ತರುವುದು. ಶ್ವಾಸೋಚ್ಛ್ವಾಸಗಳನ್ನು ವ್ಯವಸ್ಥೆಗೊಳಿಸುವ ನರಗಳ ಕೇಂದ್ರಕ್ಕೆ  ಉಳಿದ ನರಗಳ ಮೇಲೆ ಸ್ವಲ್ಪ ನಿಯಂತ್ರಣವಿರುವುದು. ಆದಕಾರಣ ಒಂದು ನಿಯಮಿತ ಉಸಿರಾಟ ಆವಶ್ಯಕವೆಂಬುದನ್ನು ನಾವು ಆಗಲೇ ಹೇಳಿರುವೆವು. ನಾವು ಸಾಧಾರಣವಾಗಿ ಉಸಿರಾಡುವುದನ್ನು ಉಸಿರಾಟವೆಂದು ಹೇಳಲಾಗುವುದಿಲ್ಲ. ಅದು ಬಹಳ ಅವ್ಯವಸ್ಥಿತವಾಗಿರುವುದು. ಮತ್ತೆ ಉಸಿರಾಡುವುದರಲ್ಲಿ ಸ್ತ್ರೀಯರಿಗೂ ಮತ್ತು ಪುರುಷರಿಗೂ ಕೆಲವು ನೈಸರ್ಗಿಕ ವ್ಯತ್ಯಾಸಗಳಿವೆ. 

ಮೊದಲನೆಯ ಅಭ್ಯಾಸವೇ ಒಂದು ನಿರ್ದಿಷ್ಟ ಅಳತೆಗೆ ಅನುಸಾರವಾಗಿ ಉಸಿರನ್ನು ಸೆಳೆದು ಬಿಡುವುದು. ಅದು ನಮ್ಮ ಶರೀರದಲ್ಲಿ ಒಂದು ಸಮನ್ವಯವನ್ನು ಉಂಟುಮಾಡುತ್ತದೆ. ನೀವು ಇದನ್ನು ಕೆಲವು ಕಾಲ ಅಭ್ಯಾಸಮಾಡಿದ ಮೇಲೆ “ಓಂ” ಅಥವಾ ಇನ್ನು ಯಾವುದಾದರೊಂದು ಪವಿತ್ರ ಮಂತ್ರವನ್ನು ಸೇರಿಸಿದರೆ ಉತ್ತಮ. ಇಂಡಿಯ ದೇಶದಲ್ಲಿ ನಾವು ಒಂದು, ಎರಡು, ಮೂರು, ನಾಲ್ಕು ಎಂದು ಎಣಿಸುವುದಕ್ಕೆ ಬದಲಾಗಿ ಕೆಲವು ಸಾಂಕೇತಿಕ ಪದಗಳನ್ನು ಉಪಯೋಗಿಸುತ್ತೇವೆ. ಅದಕ್ಕೋಸ್ಕರವೇ “ಓಂ” ಅಥವಾ ಮತ್ತಾವುದಾದರೊಂದು ಮಂತ್ರವನ್ನು ಪ್ರಾಣಾಯಾಮದೊಂದಿಗೆ ಉಪಯೋಗಿಸಬೇಕೆಂದು ನಿಮಗೆ ಹೇಳುತ್ತೇನೆ. ಆ ಪದ ಉಸಿರಿನೊಂದಿಗೆ ಲಯಬದ್ಧವಾಗಿ ಸಾಮರಸ್ಯದಿಂದ ಹರಿಯಲಿ. ಇದರಿಂದ ದೇಹವೆಲ್ಲ ಲಯಬದ್ಧವಾಗುತ್ತದೆ. ಆಗ ವಿಶ್ರಾಂತಿ ಎಂದರೆ ಏನೆಂಬುದು ಗೊತ್ತಾಗುವುದು. ಅದರೊಂದಿಗೆ ಹೋಲಿಸಿ ನೋಡಿದರೆ ನಿದ್ರೆ ವಿಶ್ರಾಂತಿಯಲ್ಲ. ಒಮ್ಮೆ ಈ ವಿಶ್ರಾಂತಿ ಬಂದರೆ ಬಹಳ ಬಳಲಿದ ನರಗಳೂ ಕೂಡ ಶಾಂತವಾಗುವುವು. ಇದಕ್ಕೆ ಮುಂಚೆ ನೀವು ನಿಜವಾಗಿಯೂ ವಿಶ್ರಾಂತಿಯನ್ನು ಪಡೆದಿರಲಿಲ್ಲವೆನ್ನುವುದು ನಿಮಗೆ ಆಗ ಗೊತ್ತಾಗುವುದು. 

ಈ ಸಾಧನೆಯ ಮೊದಲನೆಯ ಪರಿಣಾಮವೇ ಒಬ್ಬನ ಮುಖಭಾವದ ಬದಲಾವಣೆ. ಒರಟುತನ ಮುಖದಿಂದ ಮಾಯವಾಗುವುದು. ಶಾಂತ ಆಲೋಚನೆಗಳಿಂದ ಶಾಂತಿಯೇ ಅವನ ಮುಖದ ಮೇಲೆ ನಲಿಯುವುದು. ಅನಂತರ ಮಧುರವಾದ ಧ್ವನಿ ಬರುವುದು. ಕಿರುಚಲು ಧ್ವನಿಯಿಂದ ಕೂಡಿದ ಯೋಗಿಯನ್ನು ನಾನು ಇದುವರೆಗೂ ನೋಡಿಲ್ಲ. ಈ ಚಿಹ್ನೆಗಳು ಕೆಲವು ತಿಂಗಳ ಅಭ್ಯಾಸವಾದ ಮೇಲೆ ಬರುವುವು. ಮೇಲೆ ಹೇಳಿದಂತೆ ಉಸಿರಾಡುವುದನ್ನು ಕೆಲವು ದಿನಗಳು ಅಭ್ಯಾಸ ಮಾಡಿದ ಮೇಲೆ, ಅದಕ್ಕಿಂತ ಮುಂದಿನದನ್ನು ಅಭ್ಯಾಸ ಮಾಡಬೇಕು. ಎಡಗಡೆ ಮೂಗಿನ ಹೊಳ್ಳೆಯಿಂದ, ಇಡಾದ ಮೂಲಕ, ನಿಧಾನವಾಗಿ ವಾಯುವನ್ನು ಸೆಳೆದು ಶ್ವಾಸಕೋಶವನ್ನು ತುಂಬಿಸಿ, ಮತ್ತು ಅದೇ ಕಾಲದಲ್ಲಿ ನರಗಳ ಶಕ್ತಿಯ ಮೇಲೆ ನಿಮ್ಮ ಮನಸ್ಸನ್ನು ಏಕಾಗ್ರಗೊಳಿಸಿ. ನೀವು ಆ ನರಗಳ ಶಕ್ತಿಯನ್ನು ಬೆನ್ನು ಮೂಳೆಯ ಕಾಲುವೆಯ ಮೂಲಕ ಕೆಳಗೆ ಕಳುಹಿಸಿ, ಕುಂಡಲಿನಿಯ ವಾಸಸ್ಥಾನವಾದ ತ್ರಿಕೋಣಾಕಾರವಾದ ಮೂಲಾಧಾರ ಚಕ್ರದ ಮೇಲೆ ಬಲವಾಗಿ ಹೊಡೆಯುತ್ತಿರುವಂತೆ ಭಾವಿಸಿ. ಅನಂತರ ಅಲ್ಲಿ ಕೆಲವು ಕಾಲ ಶಕ್ತಿಯನ್ನು ನಿಲ್ಲಿಸಿ. ಪಿಂಗಳಮಾರ್ಗದ ಮೂಲಕ ಆ ನರಗಳ ಶಕ್ತಿಯನ್ನು ನಿಧಾನವಾಗಿ ಸೆಳೆದುಕೊಂಡಂತೆ ಪರಿಭಾವಿಸಿ ಉಸಿರನ್ನು ಬಲಹೊಳ್ಳೆಯ ಮೂಲಕ ಬಿಡಿ. ಇದನ್ನು ಅಭ್ಯಾಸ ಮಾಡುವುದು ಸ್ವಲ್ಪ ಕಷ್ಟ. ಇದಕ್ಕೆ ಸುಲಭವಾದ ಮಾರ್ಗವೇ, ಬಲಗಡೆಯ ಮೂಗನ್ನು ಬಲಗೈ ಹೆಬ್ಬೆಟ್ಟಿನ ಮೂಲಕ ಮುಚ್ಚಿಕೊಂಡು ಎಡಗಡೆಯ ಮೂಗಿನ ಮೂಲಕ ನಿಧಾನವಾಗಿ ಗಾಳಿಯನ್ನು ಸೆಳೆದುಕೊಳ್ಳಬೇಕು. ಅನಂತರ ಎರಡು ಮೂಗಿನ ಹೊಳ್ಳೆಗಳನ್ನೂ ಬೆರಳಿನಲ್ಲಿ ಮುಚ್ಚಿ, ಆ ಶಕ್ತಿ ಪ್ರವಾಹವನ್ನು ಕೆಳಗೆ ಕಳುಹಿಸಿ, ಅದು ಸುಷುಮ್ನಾ ಕಾಲುವೆಯ ಕೆಳಗಿನಲ್ಲಿರುವ ಮೂಲಾಧಾರವನ್ನು ಜಾಗೃತಗೊಳಿಸುತ್ತಿದೆಯೆಂದು ಭಾವಿಸಬೇಕು. ಅನಂತರ ಹೆಬ್ಬೆಟ್ಟನ್ನು ತೆಗೆದು ಬಲಗಡೆಯ ಮೂಗಿನ ಮೂಲಕ ಉಸಿರನ್ನು ಹೊರಗೆ ಬಿಡಬೇಕು. ಇದಾದ ಮೇಲೆ ಬಲಗಡೆಯ ಮೂಗಿನಿಂದ ಉಸಿರನ್ನು ಸೆಳೆದುಕೊಳ್ಳಿ. ಅನಂತರ ಎರಡನ್ನೂ ಹಿಂದಿನಂತೆ ಮುಚ್ಚಿ. ಹಿಂದೂಗಳು ಅಭ್ಯಾಸ ಮಾಡುವ ರೀತಿ ಈ ದೇಶದವರಿಗೆ (ಅಮೆರಿಕಾ) ಬಹಳ ಕಷ್ಟವಾಗಿ ಕಾಣಬಹುದು. ಏಕೆಂದರೆ ಅವರು ಬಾಲ್ಯದಿಂದಲೇ ಇದನ್ನು ಅಭ್ಯಾಸಮಾಡುವುದರಿಂದ ಅವರ ಶ್ವಾಸಕೋಶಗಳು ಇದಕ್ಕೆ ಸಿದ್ಧವಾಗಿರುತ್ತವೆ. ಇದನ್ನು ನಾಲ್ಕು ಕ್ಷಣಗಳಿಂದ ಮೊದಲು ಮಾಡಿ ಕ್ರಮೇಣ ಹೆಚ್ಚಿಸಬೇಕು. ನಾಲ್ಕು ಕ್ಷಣಗಳು ಉಸಿರನ್ನು ಸೆಳೆದುಕೊಳ್ಳಿ; ಹದಿನಾರು ಕ್ಷಣಗಳು ಉಸಿರನ್ನು ಇಟ್ಟುಕೊಳ್ಳಿ; ಎಂಟು ಕ್ಷಣ ಉಸಿರನ್ನು ಬಿಡಿ. ಇದು ಒಂದು ಪ್ರಾಣಾಯಾಮವಾಗುವುದು. ಅದೇ ಕಾಲದಲ್ಲಿ ತ್ರಿಕೋಣಾಕಾರವಾಗಿರುವ ಮೂಲಾಧಾರವನ್ನು ಆಲೋಚಿಸಿ. ಆ ಚಕ್ರದ ಮೇಲೆ ಮನಸ್ಸನ್ನು ಏಕಾಗ್ರಗೊಳಿಸಿ. ಕಲ್ಪನೆಯು ನಿಮಗೆ ಬಹಳ ಸಹಾಯ ಮಾಡಬಲ್ಲದು. ಇದಾದ ಮೇಲೆ ಅದೇ ಸಂಖ್ಯೆಯನ್ನು ಉಪಯೋಗಿಸಿ, ಒಳಗೆ ಉಸಿರನ್ನು ಸೆಳೆದುಕೊಂಡು ತಕ್ಷಣವೇ ಅದನ್ನು ನಿಧಾನವಾಗಿ ಹೊರಗೆ ಕಳುಹಿಸಿ, ಅಲ್ಲಿ ಅದನ್ನು ನಿಲ್ಲಿಸಬೇಕು. ಇದರಲ್ಲಿರುವ ವ್ಯತ್ಯಾಸವಿಷ್ಟೆ: ಮೊದಲನೆಯದರಲ್ಲಿ ಉಸಿರನ್ನು ಒಳಗೆ ನಿಲ್ಲಿಸಿತ್ತು. ಎರಡನೆಯದರಲ್ಲಿ ಅದನ್ನು ಹೊರಗೆ ನಿಲ್ಲಿಸಿದೆ. ಒಳಗೆ ಉಸಿರನ್ನು ನಿಲ್ಲಿಸುವ ಸಾಧನೆಯನ್ನು ಹೆಚ್ಚು ಅಭ್ಯಾಸ ಮಾಡಬಾರದು. ಬೆಳಗ್ಗೆ ನಾಲ್ಕು ಸಲ, ಸಾಯಂಕಾಲ ನಾಲ್ಕು ಸಲ ಮಾತ್ರ ಮಾಡಿ. ಅನಂತರ, ಕ್ರಮೇಣ ವೇಳೆಯನ್ನು ಮತ್ತು ಸಂಖ್ಯೆಯನ್ನು ಹೆಚ್ಚು ಮಾಡಬಹುದು. ಹಾಗೆ ಮಾಡಲು ನಿಮಗೆ ಶಕ್ತಿ ಇದೆ ಮತ್ತು ಹಾಗೆ ಮಾಡುವುದರಲ್ಲಿ ನಿಮಗೆ ಒಂದು ಉತ್ಸಾಹವಿದೆ ಎಂದು ನಿಮಗೆ ಕಾಣುವುದು. ಹೀಗೆ ಮಾಡುವುದಕ್ಕೆ ನಿಮಗೆ ಶಕ್ತಿ ಇದೆ ಎಂದು ಗೊತ್ತಾದ ಮೇಲೆ ಜೋಪಾನವಾಗಿ ನಾಲ್ಕರಿಂದ ಆರಕ್ಕೆ ಹೆಚ್ಚಿಸಿ. ಬಿಟ್ಟು ಬಿಟ್ಟು ಮಾಡುವುದರಿಂದ ನಿಮಗೆ ಅಪಾಯವಾಗಬಹುದು, ಪ್ರತಿದಿನ ಮಾಡಬೇಕು. 

ನರಗಳ ಶುದ್ಧಿಯ ವಿಷಯವಾಗಿ ಮೇಲೆ ಹೇಳಿದ ಮೂರು ಅಭ್ಯಾಸಗಳಲ್ಲಿ ಮೊದಲನೆಯದು ಮತ್ತು ಕೊನೆಯದು ಕಷ್ಟವೂ ಇಲ್ಲ, ಮತ್ತು ಅದರಿಂದ ಅಪಾಯವೂ ಇಲ್ಲ. ನೀವು ಮೊದಲನೆಯದನ್ನು ಹೆಚ್ಚು ಅಭ್ಯಾಸ ಮಾಡಿದಷ್ಟೂ ಹೆಚ್ಚು ಶಾಂತರಾಗುವಿರಿ. “ಓಂ” ಪದವನ್ನು ಧ್ಯಾನಿಸಿ. ನೀವು ಕೆಲಸ ಮಾಡುತ್ತಿರುವಾಗಲೂ ಕೂಡ ಇದನ್ನು ಅಭ್ಯಾಸ ಮಾಡಬಹುದು. ಹಾಗೆ ಮಾಡಿದಷ್ಟೂ ಉತ್ತಮ. ನೀವು ಕಷ್ಟಪಟ್ಟು ಅಭ್ಯಾಸ ಮಾಡಿದರೆ ಎಂದಾದರೊಂದು ದಿನ ಕುಂಡಲಿನೀ ಶಕ್ತಿ ಜಾಗೃತವಾಗುವುದು. ದಿನಕ್ಕೆ ಒಂದು ವೇಳೆಯೋ ಅಥವಾ ಎರಡು ವೇಳೆಯೋ ಯಾರು ಅಭ್ಯಾಸ ಮಾಡುತ್ತಾರೆಯೋ ಅವರಿಗೆ ತಕ್ಕಮಟ್ಟಿಗೆ ಶಾರೀರಿಕ ಮತ್ತು ಮಾನಸಿಕ ಶಾಂತಿಯೂ, ಮಧುರವಾದ ಧ್ವನಿಯೂ ಬರುವುವು. ಯಾರು ಅಭ್ಯಾಸದಲ್ಲಿ ಮುಂದುವರಿಯುತ್ತಾರೆಯೋ ಅವರಿಗೆ ಮಾತ್ರ ಕುಂಡಲಿನೀ ಶಕ್ತಿ ಜಾಗೃತವಾಗಿ ಪ್ರಕೃತಿಯೆಲ್ಲ ಬದಲಾಯಿಸುವುದು, ಜ್ಞಾನ ಭಂಡಾರ ತೆರೆಯುವುದು. ಜ್ಞಾನಾರ್ಜನೆಗೆ ಪುಸ್ತಕಗಳ ಸಹಾಯ ನಿಮಗೆ ಇನ್ನೇನೂ ಬೇಕಿಲ್ಲ. ನಿಮ್ಮ ಮನಸ್ಸೇ ಅನಂತ ಜ್ಞಾನವನ್ನೊಳಗೊಂಡ ಪುಸ್ತಕವಾಗುವುದು. ಬೆನ್ನುಮೂಳೆಗಳ ಮಧ್ಯದಲ್ಲಿರುವ ಸುಷುಮ್ನಾ ಕಾಲುವೆ, ಅದರ ಎರಡು ಪಾರ್ಶ್ವಗಳಲ್ಲಿಯೂ ಸಂಚರಿಸುವ ಇಡಾ ಮತ್ತು ಪಿಂಗಳಗಳೆಂಬ ಶಕ್ತಿ, ಇವುಗಳ ವಿಚಾರವನ್ನು ನಾನು ಆಗಲೇ ನಿಮಗೆ ಹೇಳಿರುತ್ತೇನೆ. ಈ ಮೂರು ಪ್ರತಿಯೊಂದು ಪ್ರಾಣಿಯಲ್ಲಿಯೂ ಇರುವುವು. ಯಾವ ಪ್ರಾಣಿಗೆ ಬೆನ್ನೆಲುಬು ಇದೆಯೋ ಅದಕ್ಕೆ ಈ ಮೂರು ಕ್ರಿಯೆಗಳೂ ಇರುತ್ತವೆ. ಸಾಧಾರಣ ಮಾನವರಲ್ಲಿ ಸುಷುಮ್ನಾ ಕಾಲುವೆ ಮುಚ್ಚಿರುವುದು. ಅದರ ಚಲನೆ ಅಷ್ಟು ಗೋಚರಿಸುವುದಿಲ್ಲ. ಆದರೆ ಉಳಿದ ಎರಡು, ಇಡಾ ಮತ್ತು ಪಿಂಗಳ, ದೇಹದ ನಾನಾ ಭಾಗಗಳಿಗೆ ಶಕ್ತಿಯನ್ನು ಒದಗಿಸುತ್ತವೆ ಎಂದು ಯೋಗಿಯು ಹೇಳುತ್ತಾನೆ. 

\eject

%%%361

ಯೋಗಿಯಲ್ಲಿ ಮಾತ್ರ ಸುಷುಮ್ನಾ ಕಾಲುವೆ ತೆರೆದಿದೆ. ಈ ಸುಷುಮ್ನಾನಾಳವು ತೆರೆದು ಶಕ್ತಿಯು ಮೇಲೇಳುವುದಕ್ಕೆ ಪ್ರಾರಂಭಿಸಿದಾಗ, ನಾವು ಇಂದ್ರಿಯಾತೀತರಾಗುವೆವು. ನಮ್ಮ ಮನಸ್ಸು ಅತೀಂದ್ರಿಯವಾಗಿ ಯುಕ್ತಿಗೆ ಮೀರಿದ ಪ್ರಜ್ಞಾತೀತ ಸ್ಥಿತಿಗೆ ಹೋಗುತ್ತದೆ. ಈ ಸುಷುಮ್ನಾ ಕಾಲುವೆಯನ್ನು ತೆರೆಯುವುದೇ ಯೋಗಿಯ ಪ್ರಥಮ ಕರ್ತವ್ಯ. ಯೋಗಿಯ ಅಭಿಪ್ರಾಯದ ಪ್ರಕಾರ ಈ ಸುಷಮ್ನಾಕಾಲುವೆಯಲ್ಲಿಯೇ ವಿವಿಧ ಕೇಂದ್ರಗಳಿರುವುವು. ಸಾಂಕೇತಿಕವಾಗಿ ಅವನ್ನು ಕಮಲಗಳು ಎಂದು ಕರೆಯುತ್ತಾರೆ. ಅತಿ ಕೆಳಗಿನದು ಬೆನ್ನೆಲುಬಿನ ತಳಭಾಗದಲ್ಲಿರುವುದು. ಅದನ್ನು ಮೂಲಾಧಾರವೆನ್ನುತ್ತಾರೆ. ಅದಕ್ಕಿಂತ ಮೇಲಿರುವುದು ಸ್ವಾಧಿಷ್ಠಾನ, ಮೂರನೆಯದು ಮಣಿಪೂರ, ನಾಲ್ಕನೆಯದು ಅನಾಹುತ, ಐದನೆಯದು ವಿಶುದ್ಧ, ಆರನೆಯದು ಆಜ್ಞಾ, ಮಿದುಳಿನಲ್ಲಿರುವ ಕೊನೆಯದೇ ಸಹಸ್ರಾರ, ಅಥವಾ ಸಹಸ್ರದಳದ ಪದ್ಮ. ಇವುಗಳಲ್ಲಿ ಸದ್ಯಕ್ಕೆ ನಾವು ಎರಡು ಚಕ್ರಗಳನ್ನು, ಅಂದರೆ ಮೊದಲನೆಯದಾದ ಮೂಲಾಧಾರ, ಕೊನೆಯದಾದ ಸಹಸ್ರಾರ ಇವುಗಳನ್ನು ಮಾತ್ರ ಗಮನಿಸಬೇಕು. ಶಕ್ತಿಯೆಲ್ಲವನ್ನೂ ಅದರ ಮೂಲಸ್ಥಾನವಾದ ಮೂಲಾಧಾರದಿಂದ ಸಹಸ್ರಾರಕ್ಕೆ ತರಬೇಕು. ದೇಹದಲ್ಲಿರುವ ಶಕ್ತಿಯಲ್ಲೆಲ್ಲ ಉತ್ತಮವಾದುದೆ ಓಜಸ್ಸು ಎಂದು ಯೋಗಿಗಳು ಹೇಳುತ್ತಾರೆ. ಈ ಓಜಸ್ಸು ಮಿದುಳಿನಲ್ಲಿ ಸಂಗ್ರಹವಾಗಿರುವುದು. ಎಷ್ಟು ಹೆಚ್ಚು ಓಜಸ್ಸು ಇರುವುದೋ, ಮನುಷ್ಯನು ಅಷ್ಟೂ ಬಲಾಢ್ಯನಾಗಿರುವನು, ಬುದ್ಧಿವಂತನಾಗಿರುವನು ಮತ್ತು ಹೆಚ್ಚು ಆತ್ಮಶಕ್ತಿಯಿಂದ ಕೂಡಿರುವನು. ಒಬ್ಬನು ಬಹಳ ಸುಂದರವಾದ ಭಾಷೆಗಳಲ್ಲಿ, ಸುಂದರವಾದ ಭಾವಗಳನ್ನು ವಿವರಿಸಬಹುದು. ಆದರೆ ಇದು ಜನರನ್ನು ಆಕರ್ಷಿಸುವುದಿಲ್ಲ. ಮತ್ತೊಬ್ಬನು ಚೆನ್ನಾಗಿ ಮಾತನಾಡುವುದೂ ಇಲ್ಲ; ಅದರಲ್ಲಿ ಸುಂದರವಾದ ಭಾವನೆಗಳೂ ಇರುವುದಿಲ್ಲ. ಆದರೂ ಕೂಡ ಅವನ ಪ್ರತಿಯೊಂದು ಚಲನೆಯೂ ಕೂಡ ಶಕ್ತಿಯಿಂದ ತುಂಬಿರುತ್ತದೆ. ಇದೇ ಓಜಸ್ಸಿನ ಶಕ್ತಿ. 

\vskip 6pt

ಪ್ರತಿಯೊಬ್ಬನಲ್ಲಿಯೂ ಕೂಡ ಸ್ವಲ್ಪ ಹೆಚ್ಚು ಕಡಮೆ ಓಜಸ್ಸು ಸಂಗ್ರಹವಾಗಿದೆ. ದೇಹದಲ್ಲಿ ಕೆಲಸ ಮಾಡುತ್ತಿರುವ ಎಲ್ಲಾ ಶಕ್ತಿಗಳೂ ಕೊನೆಗೆ ಓಜಸ್ಸಾಗಿ ಪರಿಣಮಿಸುವುವು. ಮಾಂಸಖಂಡಗಳ ಕ್ರಿಯೆಗಳಂತೆ ಕೆಲಸ ಮಾಡುತ್ತಿರುವ ಶಕ್ತಿಯೇ ಓಜಸ್ಸಾಗಿ ಪರಿಣಮಿಸುವುದು. ಲೈಂಗಿಕ ಭಾವನೆಗಳಲ್ಲಿ, ಲೈಂಗಿಕ ಶಕ್ತಿಯಾಗಿ ವ್ಯಕ್ತವಾಗುವ ಆ ಮಾನವಶಕ್ತಿಯನ್ನು ತಡೆದು ನಮ್ಮ ಸ್ವಾಧೀನಕ್ಕೆ ತಂದಾಗ ಅದು ಓಜಸ್ಸಾಗಿ ಪರಿಣಮಿಸುವುದು. ಇವುಗಳು ಮೂಲಾಧಾರದ ವಶದಲ್ಲಿರುವುದರಿಂದ ಯೋಗಿಯು ವಿಶೇಷವಾಗಿ ಅದಕ್ಕೆ ಗಮನ ಕೊಡುತ್ತಾನೆ. ಸಂಭೋಗಶಕ್ತಿಯನ್ನೆಲ್ಲ ಓಜಸ್ಸನ್ನಾಗಿ ಮಾಡಲು ಅವನು ಯತ್ನಿಸುತ್ತಾನೆ. ಬ್ರಹ್ಮಚರ್ಯ ವ್ರತವನ್ನು ಪರಿಪಾಲಿಸುವ ಸ್ತ್ರೀ ಮತ್ತು ಪುರುಷರು ಓಜಸ್ಸನ್ನು ಮೇಲೇಳುವಂತೆ ಮಾಡಿ ಆ ಶಕ್ತಿಯನ್ನು ಮೆದುಳಿನಲ್ಲಿ ಸಂಗ್ರಹಿಸಿಡಬಹುದು. ಆದಕಾರಣವೇ ಬ್ರಹ್ಮಚರ್ಯವನ್ನು ಅತ್ಯಂತ ಪವಿತ್ರವಾದ ಶೀಲವೆಂದು ಪರಿಗಣಿಸಿರುವುದು. ಬ್ರಹ್ಮಚರ್ಯವ್ರತವನ್ನು ಪರಿಪಾಲಿಸದೆ ಇದ್ದರೆ ಆಧ್ಯಾತ್ಮಿಕತೆ ಮಾಯವಾಗುವುದು, ಮಾನಸಿಕ ಪಟುತ್ವಕುಗ್ಗುವುದು, ನೈತಿಕಶಕ್ತಿ\break\ ತಗ್ಗುವುದು ಎಂದು ಮಾನವನಿಗೆ ವೇದ್ಯವಾಗುತ್ತದೆ. ಆದಕಾರಣವೆ ಆಧ್ಯಾತ್ಮಿಕ ಪ್ರತಿಭಾವಂತರಿಗೆ ಜನ್ಮವಿತ್ತ ಎಲ್ಲಾ ಧಾರ್ಮಿಕ ಸಂಸ್ಥೆಗಳೂ ಪೂರ್ಣ ಬ್ರಹ್ಮಚರ್ಯವನ್ನು ಒತ್ತಿ ಹೇಳುವುದು ನಮಗೆ ಕಂಡುಬರುವುದು. ಆದಕಾರಣವೆ ಮದುವೆಯಾಗುವುದನ್ನು ಬಿಟ್ಟು ಸಂನ್ಯಾಸಿಗಳಾಗುವರು. ಮನೋವಾಕ್ಕಾಯವಾಗಿ ಸಂಪೂರ್ಣ ಬ್ರಹ್ಮಚರ್ಯವ್ರತವನ್ನು ಪರಿಪಾಲಿಸಬೇಕು. ಅದಿಲ್ಲದೆ ಇದ್ದರೆ ರಾಜಯೋಗವನ್ನು ಅಭ್ಯಾಸ ಮಾಡುವುದು ಅಪಾಯಕರ. ಅದು ಅವರನ್ನು ಹುಚ್ಚರನ್ನಾಗಿ ಮಾಡಬಹುದು. ಜನರು ರಾಜಯೋಗವನ್ನು ಅಭ್ಯಾಸಮಾಡಿ ಜೊತೆಯಲ್ಲಿ ಕಳಂಕಜೀವನವನ್ನು ನಡೆಸಿದರೆ ಅವರು ಯೋಗಿಗಳು ಹೇಗೆ ಆಗಬಲ್ಲರು?

\chapter{ಪ್ರತ್ಯಾಹಾರ ಮತ್ತು ಧಾರಣ}

ಮುಂದಿನ ಮೆಟ್ಟಲೇ ಪ್ರತ್ಯಾಹಾರವೆನ್ನುವುದು. ಇದೇನು? ನಿಮಗೆ ಇಂದ್ರಿಯ ಜ್ಞಾನ ಬರುವುದು ಹೇಗೆ? ಮೊದಲು ಬಾಹ್ಯೇಂದ್ರಿಯಗಳಿವೆ, ಅನಂತರ ಮಿದುಳಿನ ಕೇಂದ್ರಗಳ ಮೂಲಕ ಕೆಲಸ ಮಾಡುತ್ತಿರುವ ಅಂತರಿಂದ್ರಿಯಗಳು, ಇದಾದ ಮೇಲೆ ಮನಸ್ಸು ಇರುವುದು. ಇವು ಕಲೆತು ಯಾವುದಾದರೊಂದು ಬಾಹ್ಯ ವಸ್ತುವಿನೊಂದಿಗೆ ಸಂಪರ್ಕಿಸಿದರೆ ಆ ವಸ್ತುವು ನಮಗೆ ಕಾಣುತ್ತದೆ. ಮನಸ್ಸನ್ನು ಏಕಾಗ್ರಮಾಡಿ ಒಂದೇ ಇಂದ್ರಿಯದ ಕಡೆ ಬಿಡುವುದು ಬಹಳ ಕಷ್ಟವಾದ ಕೆಲಸ. ಮನಸ್ಸು ಒಂದು ಗುಲಾಮನಂತೆ. 

\vskip 6pt

ಪ್ರಪಂಚದಲ್ಲೆಲ್ಲ “ಉತ್ತಮರಾಗಿ” “ಉತ್ತಮರಾಗಿ” “ಉತ್ತಮರಾಗಿ” ಎಂಬ ಬೋಧನೆಯನ್ನು ನಾವು ಕೇಳಿರುವೆವು. “ಕದಿಯಬೇಡ” “ಸುಳ್ಳನ್ನು ಹೇಳಬೇಡ” ಎಂಬುದನ್ನು ಕೇಳದೆ ಇರುವ ಮಗುವು ಪ್ರಪಂಚದಲ್ಲಿಯೇ ಇಲ್ಲವೆನ್ನಬಹುದು. ಆದರೆ ಇವನ್ನು ಹೇಗೆ ಸಾಧಿಸಬೇಕೆಂಬುದನ್ನು ಯಾರೂ ಹೇಳುವುದಿಲ್ಲ. ಮಾತಿನಿಂದ ಜನರಿಗೆ ಸಹಾಯವಾಗಲಾರದು. ಏತಕ್ಕೆ ಒಬ್ಬ ಕಳ್ಳನಾಗಬಾರದು? ಕದಿಯದೆ ಇರುವುದು ಹೇಗೆ–ಎಂಬುದನ್ನು ನಾವು ಅವನಿಗೆ ಕಲಿಸುವುದಿಲ್ಲ. ಸುಮ್ಮನೆ ಮಗುವಿಗೆ ಕದಿಯಬೇಡ ಎಂದು ಹೇಳುವೆವು. ಆ ಮಗುವಿಗೆ ತನ್ನ ಮನಸ್ಸನ್ನು ನಿಗ್ರಹಿಸುವುದು ಹೇಗೆ ಎಂಬುದನ್ನು ಕಲಿಸಿದಾಗ ಮಾತ್ರ ನಿಜವಾಗಿ ಸಹಾಯ ಮಾಡುತ್ತೇವೆ. ಆಂತರಿಕ ಮತ್ತು ಬಾಹ್ಯಕ್ರಿಯೆಗಳೆಲ್ಲ ಮನಸ್ಸು ಇಂದ್ರಿಯಗಳೆಂಬ ಕೇಂದ್ರದೊಂದಿಗೆ ಕಲೆತಾಗ ಮಾತ್ರ ನಡೆಯುವುವು. ಬೇಕೆಂತಲೋ ಬೇಡವೆಂತಲೋ ಮನಸ್ಸು ಆಯಾ ಇಂದ್ರಿಯಗಳೊಂದಿಗೆ ಕಲೆಯುವಂತೆ ಬಲಾತ್ಕರಿಸಲ್ಪಡುತ್ತದೆ. ಅದಕ್ಕೋಸ್ಕರವಾಗಿಯೇ ಜನರು ಮೂಢ ಕೆಲಸಗಳನ್ನು ಮಾಡಿ ವ್ಯಥೆ ಪಡುವರು. ಮನಸ್ಸು ಏನಾದರೂ ಅವರ ಸ್ವಾಧೀನದಲ್ಲಿ ಇದ್ದರೆ, ಅವರೆಂದಿಗೂ ಹಾಗೆ ಮಾಡುತ್ತಿರಲಿಲ್ಲ. ಮನಸ್ಸಿನ ನಿಗ್ರಹದ ಪರಿಣಾಮವಾಗಿ ಆಗುವುದೇನು? ಮನಸ್ಸು ಆಯಾ ಇಂದ್ರಿಯ ಕೇಂದ್ರ\break ಗಳೊಂದಿಗೆ ಬೆರೆಯುವುದಿಲ್ಲ. ಆಗ ಸ್ವಭಾವತಃ ಭಾವನೆ ಮತ್ತು ಇಚ್ಛೆಗಳೆರಡೂ ನಮ್ಮ ವಶದಲ್ಲಿರುತ್ತವೆ. ಇಲ್ಲಿಯವರೆವಿಗೂ ಇದು ಸ್ಪಷ್ಟವಾಗಿದೆ. ಇದು ಸಾಧ್ಯವೇ? ಇದು ಸಂಪೂರ್ಣವಾಗಿ ಸಾಧ್ಯ. ಇದನ್ನು ಈಗಿನ ಕಾಲದಲ್ಲಿ ನೋಡುತ್ತೀರಿ. ನಂಬಿಕೆಯಿಂದ ರೋಗವನ್ನು ಗುಣ ಮಾಡುವ ವೈದ್ಯರು ದುಃಖ, ಸಂಕಟ, ಪಾಪಗಳು ಇಲ್ಲವೆಂದು ಜನರು ಭಾವಿಸಬೇಕೆಂದು ಬೋಧಿಸುತ್ತಾರೆ. ಅವರ ತತ್ತ್ವ ಯುಕ್ತಿಪೂರ್ವಕವಾಗಿ ಇಲ್ಲದೇ ಇರಬಹುದು. ಆದರೆ ಇದು ಅವರು ತಿಳಿಯದೆ ಆಕಸ್ಮಿಕವಾಗಿ ಪಡೆದಿರುವ ಯೋಗದ ಒಂದು ಭಾಗ. ಕಷ್ಟವನ್ನು ಅದಿಲ್ಲವೆಂದು ಸಾಧಿಸುವುದರ ಮೂಲಕ ಅವರು ರೋಗಿಗಳನ್ನು ಗುಣಮುಖರನ್ನಾಗಿ ಮಾಡುವಾಗ, ಅವರು ನಿಜವಾಗಿಯೂ ಪ್ರತ್ಯಾಹಾರದ ವಿಧಾನವನ್ನೇ ಬಳಸಿರುತ್ತಾರೆ– ಅಂದರೆ, ಇಂದ್ರಿಯಗಳನ್ನು ನಿರ್ಲಕ್ಷಿಸುವಷ್ಟು ರೋಗಿಯ ಮನಸ್ಸನ್ನು ಬಲ ಪಡಿಸುತ್ತಾರೆ. ಅದರಂತೆಯೇ ಸುಪ್ತ್ಯಾವಾಹಕ \enginline{(hypnosis)} ವಿದ್ಯೆಯನ್ನು ಉಪಯೋಗಿಸುವವನು ಕೂಡ ತನ್ನ ಸಂಜ್ಞೆಗಳಿಂದ ರೋಗಿಯಲ್ಲಿ ಕೆಲವು ಕಾಲದ ಮಟ್ಟಿಗೆ ಒಂದು ವಿಧವಾದ ಅಸ್ವಾಭಾವಿಕವಾದ ಪ್ರತ್ಯಾಹಾರವನ್ನು ಉದ್ರೇಕಿಸುತ್ತಾನೆ. ಸುಪ್ತ್ಯಾವಾಹನೆಯ ಸಲಹೆಗಳು ದುರ್ಬಲ ಮನಸ್ಸಿನ ಮೇಲೆ ಮಾತ್ರ ಪರಿಣಾಮಕಾರಿಯಾಗುವುವು, ಸುಪ್ತ್ಯಾವಾಹಕನು ಎವೆಯಿಕ್ಕದೆ ನೋಡುವುದು ಅಥವಾ ಇನ್ನು ಯಾವುದಾದರೂ ಇಂತಹ ಕೆಲಸಗಳ ಮೂಲಕ ರೋಗಿಯ ಮನಸ್ಸನ್ನು ಒಂದು ಮೂಢಸ್ಥಿತಿಗೆ ತರುವ ತನಕ ಅವನ ಸಲಹೆಗಳು ಕೆಲಸ ಮಾಡುವುದಿಲ್ಲ. 

\vskip 6pt

ಸುಪ್ತ್ಯಾವಾಹನೆಗೆ ಅಥವಾ ಶ್ರದ್ಧಾಚಿಕಿತ್ಸೆಗೆ ಒಳಪಟ್ಟ ರೋಗಿಯ ಕೇಂದ್ರಗಳನ್ನು ವೈದ್ಯನು ನಿಯಂತ್ರಿಸುವುದು ಸರಿಯಲ್ಲ. ಏಕೆಂದರೆ ಕೊನೆಗೆ ಅದು ರೋಗಿಗೆ ಅಪಾಯವನ್ನು ಉಂಟುಮಾಡುತ್ತದೆ. ಏಕೆಂದರೆ ಅಂತಹ ಚಿಕಿತ್ಸೆ ಸ್ವಂತ ಇಚ್ಛೆಯಿಂದ ಮಿದುಳಿನ ಕೇಂದ್ರಗಳನ್ನು ನಿಯಂತ್ರಿಸುವ ಕ್ರಿಯೆ ಆಗಿರದೆ, ರೋಗಿಯ ಮನಸ್ಸಿಗೆ ಚಿಕಿತ್ಸಕರ ಇಚ್ಛಾಶಕ್ತಿಯು ಆಘಾತವನ್ನು ನೀಡಿ ಅದನ್ನು ಸ್ತಬ್ಧಗೊಳಿಸುವ ಕ್ರಿಯೆ ಆಗಿರುತ್ತದೆ. ಕುದುರೆಯನ್ನು ಪಳಗಿಸುವುದಕ್ಕಾಗಿ ಕಡಿವಾಣ ಮತ್ತು ಬಾಹುಬಲಗಳನ್ನು ಬಳಸುವುದನ್ನು ಬಿಟ್ಟು, ಇತರರು ಅದರ ತಲೆಯ ಮೇಲೆ ಬಲವಾಗಿ ಹೊಡೆಯುವಂತೆ ಮಾಡುವ ಹಾಗೆ ಅದು. ಈ ಪ್ರಕ್ರಿಯೆಗಳಲ್ಲಿ ಪ್ರತಿಯೊಂದರಲ್ಲಿಯೂ ರೋಗಿಯು ತನ್ನ ಬುದ್ಧಿಶಕ್ತಿಯಲ್ಲಿ ಸ್ವಲ್ಪ ಭಾಗವನ್ನು ಕಳೆದುಕೊಳ್ಳುತ್ತಾನೆ. ಕೊನೆಗೆ ಮನಸ್ಸು ಸಂಪೂರ್ಣವಾಗಿ ನಿಯಂತ್ರಣಶಕ್ತಿಯನ್ನು ಪಡೆದುಕೊಳ್ಳುವುದರ ಬದಲು, ಆಕಾರ ರಹಿತವೂ, ಶಕ್ತಿ ರಹಿತವೂ ಆಗಿಹೋಗುತ್ತದೆ. ರೋಗಿಯು ಕೊನೆಗೆ ಹುಚ್ಚಾಸ್ಪತ್ರೆಯ ದಾರಿಯನ್ನು ಹಿಡಿಯಬೇಕಾಗುತ್ತದೆ. 

\vskip 6pt

ಅನೈಚ್ಛಿಕವಾಗಿ ವ್ಯಕ್ತಿಯ ಸಮ್ಮತಿ ಇಲ್ಲದೆ ಆತನನ್ನು ವಶಮಾಡಿಕೊಳ್ಳುವ ನಮ್ಮ ಪ್ರತಿಯೊಂದು ಪ್ರಯತ್ನವೂ ಅನರ್ಥಕಾರಿ ಮಾತ್ರವಲ್ಲ ಅದರ ಉದ್ದೇಶವನ್ನೇ ನಾಶಮಾಡುವುದು. ಪ್ರತಿಯೊಬ್ಬ ಮಾನವನ ಗುರಿಯೂ ಕೂಡ ಸ್ವಾತಂತ್ರ್ಯ, ಜಡವಸ್ತು ಮತ್ತು ಆಲೋಚನೆಯ ಬಂಧನಗಳಿಂದ ಪಾರಾಗುವುದು; ಬಾಹ್ಯ ಮತ್ತು ಆಂತರಿಕ ಪ್ರಕೃತಿಯ ಮೇಲೆ ತನ್ನ ಪ್ರಭುತ್ವವನ್ನು ಸ್ಥಾಪಿಸುವುದು. ಮತ್ತೊಬ್ಬನಿಂದ ಬರುವ ಇಚ್ಛಾತರಂಗಕ್ಕೆ ನಾವು ವಶರಾದರೆ, ಅದು ಯಾವ ರೂಪದ್ದೇ ಆಗಲಿ – ಪ್ರತ್ಯಕ್ಷವಾಗಿ ನಮ್ಮ ಇಂದ್ರಿಯಗಳನ್ನು ನಿಗ್ರಹಿಸುವಂತೆ ಮಾಡುವುದಾಗಲಿ ಅಥವಾ ಪ್ರಜ್ಞೆಇಲ್ಲದೆ ಇರುವಾಗ ಅವುಗಳನ್ನು ನಿಗ್ರಹಿಸುವಂತೆ ಬಲವಂತ ಪಡಿಸುವುದಾಗಲಿ– ಆಗಲೇ ನಮ್ಮನ್ನು ಬಂಧಿಸಿರುವ ಪೂರ್ವಸಂಸ್ಕಾರಗಳ, ಪೂರ್ವ ಮೂಢನಂಬಿಕೆಗಳ ಭಾರೀ ಸರಪಣಿಗೆ ಮತ್ತೊಂದು ಕೊಂಡಿಯನ್ನು ಸೇರಿಸಿದಂತಿರುತ್ತದೆ. ಆದಕಾರಣ ಮತ್ತೊಬ್ಬರ ಪ್ರಭಾವಕ್ಕೆ ಒಳಗಾಗುವುದರ ವಿಚಾರದಲ್ಲಿ ಬಹಳ ಜೋಪಾನವಾಗಿರಿ. ತಿಳಿದೂ ತಿಳಿದೂ, ಮತ್ತೊಬ್ಬನನ್ನು ಹೇಗೆ ನಾಶಮಾಡುತ್ತೀರಿ ಎಂಬುದರಲ್ಲಿ ಬಹಳ ಜೋಪಾನವಾಗಿರಿ. ಅವರ ಪ್ರವೃತ್ತಿಯನ್ನು ಹೊಸದಿಕ್ಕಿನಲ್ಲಿ ಹರಿಯುವಂತೆ ಮಾಡುವುದರ ಮೂಲಕ ಅನೇಕರಿಗೆ ಸ್ವಲ್ಪಕಾಲದ ಮಟ್ಟಿಗಾದರೂ ಒಳ್ಳೆಯದು ಮಾಡುವುದರಲ್ಲಿ ಕೆಲವರು ಯಶಸ್ವಿಗಳಾಗಿರುವರು, ನಿಜ. ಆದರೆ ಅದೇ ಸಂದರ್ಭದಲ್ಲಿ ಅವರು ತಮ್ಮ ಸುತ್ತಲೂ ಹರಡುವ ಅನೈಚ್ಛಿಕ ಸಲಹೆಗಳಿಂದ ಸಹಸ್ರಾರು ಪುರುಷರನ್ನು ಮತ್ತು ಸ್ತ್ರೀಯರನ್ನು ಯಾವಾಗಲೂ ಮತ್ತೊಬ್ಬರ ಇಚ್ಛೆಗೆ ಒಳಗಾಗಿರುವ ಅನಾರೋಗ್ಯ ಸ್ಥಿತಿಗೆ ತಂದು, ಕೊನೆಗೆ ಅವರನ್ನು ನಿಸ್ತೇಜರನ್ನಾಗಿ ಮಾಡುತ್ತಾರೆ. ಆದ ಕಾರಣ ಯಾರೇ ಆಗಲಿ ತನ್ನನ್ನು ಸುಮ್ಮನೆ ನಂಬಿ ಎಂದು ಕೇಳುವನೋ ಅಥವಾ ತನ್ನ ಪ್ರಬಲವಾದ ಇಚ್ಛಾಶಕ್ತಿಯಿಂದ ಮತ್ತೊಬ್ಬನನ್ನು ಬಲವಂತದಿಂದ ತನ್ನ ಹಿಂದೆ ಎಳೆದುಕೊಂಡು ಹೋಗುವನೋ, ಅವನು ತಾನು ಇಚ್ಛಿಸದೇ ಇದ್ದರೂ ಮಾನವ ವರ್ಗಕ್ಕೆ ಹಿಂಸೆಯನ್ನು ಮಾಡುತ್ತಿರುವನು. 

\vskip 6pt

ಆದಕಾರಣ ನಿಮ್ಮ ಮನಸ್ಸನ್ನು ನೀವೇ ಉಪಯೋಗಿಸಿ, ದೇಹ ಮತ್ತು ಮನಸ್ಸುಗಳನ್ನು ನೀವೇ ಸ್ವಾಧೀನಕ್ಕೆ ತೆಗೆದುಕೊಂಡು ಬನ್ನಿ. ನೀವು ದುರ್ಬಲರಾದ ಮನಸ್ಸಿನವರಾಗುವುದಕ್ಕೆ ಮುಂಚೆ, ಯಾವ ಹೊರಗಿನ ಇಚ್ಛಾಶಕ್ತಿಯೂ ನಿಮ್ಮ ಮನಸ್ಸಿನ ಮೇಲೆ ಪರಿಣಾಮಕಾರಿಯಾಗಲಾರದು ಎಂಬುದನ್ನು ಜ್ಞಾಪಕದಲ್ಲಿಡಿ. ಆಲೋಚನೆ ಮಾಡದೆ, ತನ್ನನ್ನು ನಂಬಿ ಎನ್ನುವ ಎಲ್ಲರಿಂದಲೂ, ಅವರು ಎಷ್ಟೇ ಪ್ರಖ್ಯಾತರಾಗಿರಲಿ ಅಥವಾ ಒಳ್ಳೆಯವರಾಗಿರಲಿ, ದೂರವಿರಿ. ಪ್ರಪಂಚದಲ್ಲಿ ಕುಣಿದಾಡುವ ಕೂಗಾಡುವ ಎಷ್ಟೋ ಮತಪಂಥಗಳಿವೆ. ಅವು ಹಾಡು, ನರ್ತನ, ಬೋಧನೆಗಳ ಮೂಲಕ ಸಾಂಕ್ರಾಮಿಕ ರೋಗದಂತೆ ಹರಡುತ್ತವೆ. ಆವರೂ ಒಂದು ವಿಧಧ ಸುಪ್ತ್ಯಾವಾಹಕರು. ಅವರು ಭಾವಜೀವಿಗಳ ಮೇಲೆ ಕ್ಷಣಕಾಲ ತಮ್ಮ ಸ್ವಾಧೀನ ಶಕ್ತಿಯನ್ನು ಉಪಯೋಗಿಸುವರು. ಅಯ್ಯೋ! ಇದು ಕೊನೆಗೆ ಇಡೀ ಜನಾಂಗವನ್ನೇ ಹಾಳುಮಾಡುತ್ತದೆ! ವ್ಯಕ್ತಿ ಅಥವಾ ಜನಾಂಗ ಇಂತಹ ಬಾಹ್ಯ ಅನಾರೋಗ್ಯ ಪ್ರಯೋಗಗಳಿಂದ ಕೃತಕವಾಗಿ ಒಳ್ಳೆಯದಾಗಿರುವುದಕ್ಕಿಂತಲೂ ಕ್ರೂರವಾಗಿರುವುದು ಹಿತಕರ. ಜವಾಬ್ದಾರಿ ಇಲ್ಲದ ಆದರೂ ಒಳ್ಳೆಯದನ್ನು ಮಾಡಬೇಕೆಂದಿರುವ ಇಂತಹ ಧಾರ್ಮಿಕ ಮತಾಂಧರಿಂದ ಮಾನವ ಜನಾಂಗಕ್ಕೆ ಆಗುತ್ತಿರುವ ಅನ್ಯಾಯವನ್ನು ಆಲೋಚಿಸಿದರೆ ಬಹಳ ದುಃಖವಾಗುವುದು. ಕೀರ್ತನೆ, ಪ್ರಾರ್ಥನೆಗಳಿಂದ ಕೂಡಿದ ಅವರ ಸಲಹೆಗಳ ಮೂಲಕ ತಕ್ಷಣವೇ ಆಧ್ಯಾತ್ಮಿಕ ಶಿಖರಕ್ಕೆ ಏರಬಲ್ಲ ಮನಸ್ಸು, ಮತ್ತೊಬ್ಬರ ಅಡಿಯಾಳಾಗುತ್ತದೆ, ನಿಸ್ತೇಜವಾಗುತ್ತದೆ. ಆಗ ಇದು ಇತರರ ಎಂತಹ ದುಷ್ಟ ಸಲಹೆಗಳಿಗೂ ವಶವರ್ತಿಯಾಗುವ ಸ್ಥಿತಿಗೆ ಇಳಿಯುತ್ತದೆ. ಈ ವಿಷಯವನ್ನು ಅವರು ಅರಿಯರು. ಮಾನವರನ್ನು ಉತ್ತಮರನ್ನಾಗಿ ಮಾಡುವ ಒಂದು ಅದ್ಭುತಶಕ್ತಿ ತಮ್ಮಲ್ಲಿದೆ ಎಂದೂ, ಅದು ಎಲ್ಲೋ ಮುಗಿಲುಗಳಾಚೆ ಇರುವ ದೇವರು ತಮಗೆ ಸ್ವತಃ ಕೊಟ್ಟಿರುವನೆಂದೂ ಜಂಭಕೊಚ್ಚಿಕೊಳ್ಳುವ ಈ ಮೂರ್ಖರು, ತಾವು ಭಾವೀ ನಾಶದ, ಪಾತಕದ, ಹುಚ್ಚಿನ ಮತ್ತು ಸಾವಿನ ಬೀಜವನ್ನು ಬಿತ್ತುತ್ತಿರುವೆವು ಎಂಬುದನ್ನು ತಿಳಿಯರು. ಆದಕಾರಣ ನಿಮ್ಮ ಸ್ವಾತಂತ್ರ್ಯವನ್ನು ಅಪಹರಿಸುವ ಎಲ್ಲರ ವಿಷಯದಲ್ಲೂ ಜೋಪಾನವಾಗಿರಿ. ಅದು ಅಪಾಯಕರವೆಂದು ತಿಳಿಯಿರಿ. ನಿಮ್ಮ ಕೈಲಾದ ಮಟ್ಟಿಗೂ ಅದರಿಂದ ದೂರವಾಗಿರಿ. 

\newpage

ಯಾರು ಆಯಾ ಕೇಂದ್ರಗಳಲ್ಲಿ ತಮ್ಮ ಇಚ್ಛಾನುಸಾರ ಮನಸ್ಸನ್ನು ಇಡಬಲ್ಲರೊ, ಮತ್ತು ಅಲ್ಲಿಂದ ತೆಗೆಯಬಲ್ಲರೊ ಅವರು ಪ್ರತ್ಯಾಹಾರದಲ್ಲಿ ಜಯಶೀಲರಾಗಿರುವರು. ಪ್ರತ್ಯಾಹಾರವೆಂದರೆ ಮನಸ್ಸನ್ನು ಏಕಾಗ್ರಗೊಳಿಸುವುದು; ಹೊರಕ್ಕೆ ಹೋಗುವ ಮನಸ್ಸಿನ ಶಕ್ತಿಯನ್ನು ತಡೆಯುವುದು; ಇಂದ್ರಿಯಗಳ ಬಂಧನದಿಂದ ಅದನ್ನು ಬಿಡಿಸುವುದು. ಇದನ್ನು ಸಾಧಿಸಿದಾಗ ಮಾತ್ರ ನಾವು ನಿಜವಾಗಿಯೂ ಶೀಲವಂತರಾಗುವುದು. ಆಗ ಮಾತ್ರ ನಾವು ಮುಕ್ತಿಯೆಡೆಗೆ ನಡೆದಂತೆ. ಅದಕ್ಕೆ ಮೊದಲು ನಾವು ಕೇವಲ ಯಂತ್ರಗಳು. 

ಮನಸ್ಸನ್ನು ನಿಗ್ರಹಿಸುವುದು ಎಷ್ಟು ಕಷ್ಟ! ಇದನ್ನು ಒಂದು ಹುಚ್ಚು ಕಪಿಗೆ ಹೋಲಿಸಿರುವುದು ಚೆನ್ನಾಗಿದೆ. ಎಲ್ಲಾ ಕಪಿಗಳಂತೆ ಚಂಚಲ ಸ್ವಭಾವವುಳ್ಳ ಒಂದು ಕಪಿ ಇತ್ತು. ಇದೂ ಸಾಲದೆಂಬಂತೆ ಯಾರೋ ಒಬ್ಬರು ಅದಕ್ಕೆ ಬೇಕಾದಷ್ಟು ಹೆಂಡವನ್ನು ಕುಡಿಸಿದರು. ಇದರಿಂದ ಅದು ಇನ್ನೂ ಚಂಚಲವಾಯಿತು. ಆಮೇಲೆ ಅದನ್ನು ಚೇಳು ಕುಟುಕಿತು. ಮನುಷ್ಯನು ಚೇಳು ಕುಟುಕಿದರೆ ದಿನವೆಲ್ಲ ಕುಣಿದಾಡುವನು. ಪಾಪ! ಕಪಿಗೆ ತನ್ನ ಸ್ಥಿತಿಯನ್ನು ತಡೆಯಲು ತುಂಬಾ ಅಸಾಧ್ಯವಾಯಿತು. ಇರುವ ಕಷ್ಟವನ್ನು ಪೂರ್ತಿಮಾಡಬೇಕೆಂದು ಒಂದು ಪಿಶಾಚಿಯು ಬೇರೆ ಅದನ್ನು ಮೆಟ್ಟಿಕೊಂಡಿತು. ಆಗ ಕಪಿಯ ನಿಗ್ರಹಿಸಲದಳವಾದ ಚಂಚಲ ಸ್ವಭಾವವನ್ನು ಯಾವ ಭಾಷೆ ವರ್ಣಿಸಬಲ್ಲದು? ಆ ಕಪಿಯಂತೆ ಮನಸ್ಸು. ಸ್ವಭಾವತಃ ಅದು ಚಂಚಲ. ಅನಂತರ ಆಸೆಯೆಂಬ ಸುರೆಯನ್ನು ಕುಡಿದು ಚಂಚಲತೆಯನ್ನು ಇನ್ನೂ ಹೆಚ್ಚು ಮಾಡಿಕೊಳ್ಳುವುದು. ಆಸೆ ನಮ್ಮನ್ನು ಪ್ರವೇಶಿಸಿದ ಮೇಲೆ ಮತ್ತೊಬ್ಬರ ಜಯವನ್ನು ಸಹಿಸಲಾರದ ಅಸೂಯೆ ಎಂಬ ಚೇಳು ಬೇರೆ ಕುಟುಕುವುದು. ಕೊನೆಗೆ ತನ್ನ ಸಮಾನ ಯಾರೂ ಇಲ್ಲವೆಂದು ಆಲೋಚಿಸುವಂತೆ ಮಾಡುವ ಅಹಂಕಾರದ ಪಿಶಾಚಿ ಮೆಟ್ಟಿಕೊಳ್ಳುವುದು. ಇಂತಹ ಮನಸ್ಸನ್ನು ನಿಗ್ರಹಿಸುವುದು ಎಷ್ಟು ಕಷ್ಟ!

ಮೊದಲನೆ ಸಾಧನೆಯೇ ಕೆಲವು ಕಾಲ ಕುಳಿತುಕೊಂಡು ಮನಸ್ಸನ್ನು ತನ್ನ ಪಾಡಿಗೆ ತಾನು ಚಲಿಸುವಂತೆ ಬಿಡುವುದು. ಮನಸ್ಸು ಯಾವಾಗಲೂ ಚಂಚಲ ಸ್ವಭಾವವುಳ್ಳದ್ದು. ಆ ಕಪಿಯಂತೆ ಯಾವಾಗಲೂ ಓಡಾಡುತ್ತಿರುವುದು. ಕಪಿ ತನ್ನ ಕೈಲಾದಷ್ಟು ನೆಗೆಯಲಿ. ನೀವು ಸುಮ್ಮನೆ ನಿಂತು ಅದನ್ನು ನೋಡಿ. ಜ್ಞಾನವೇ ಶಕ್ತಿಯೆಂಬ ನಾಣ್ನುಡಿ ಇದೆ. ಇದು ನಿಜ. ಮನಸ್ಸು ಏನು ಮಾಡುತ್ತಿದೆ ಎಂಬುದು ಗೊತ್ತಾಗುವ ತನಕ, ಅದನ್ನು ನೀವು ನಿಗ್ರಹಿಸಲಾಗುವುದಿಲ್ಲ. ಲಗಾಮನ್ನು ಸಡಿಲಬಿಡಿ. ಅನೇಕ ಕೆಟ್ಟ ಆಲೋಚನೆಗಳು ಬರಬಹುದು. ಇಂತಹ ವಿಷಯಗಳನ್ನು ನೀವು ಆಲೋಚಿಸುವುದು ಹೇಗೆ ಸಾಧ್ಯವಾಯಿತೆಂದು ನಿಮಗೇ ಆಶ್ಚರ್ಯವಾಗಬಹುದು. ಆದರೆ ದಿನಕಳೆದಂತೆಲ್ಲಾ, ಮನಸ್ಸಿನ ಚಂಚಲ ಸ್ವಭಾವ ಕಡಮೆಯಾಗುತ್ತ ಬಂದು ಕೊನೆಗೆ ಶಾಂತವಾಗುವುದು ನಿಮಗೆ ಕಾಣುವುದು. ಮೊದಲಿನ ಕೆಲವು ತಿಂಗಳುಗಳಲ್ಲಿ ಅನೇಕ ಆಲೋಚನೆಗಳು ಮನಸ್ಸಿನಲ್ಲಿರುತ್ತವೆ. ಅನಂತರ ಅವು ಕಡಮೆಯಾದಂತೆ ತೋರುವುದು. ಇನ್ನೂ ಕೆಲವು ತಿಂಗಳಾದ ಮೇಲೆ ಆಲೋಚನೆ ಮತ್ತೂ ಕಡಮೆಯಾಗಿ ಕೊನೆಗೆ ಮನಸ್ಸು ಸಂಪೂರ್ಣ ನಮ್ಮ ಸ್ವಾಧೀನವಾಗುವುದು. ಆದರೆ ನಾವು ಪ್ರತಿದಿನವೂ ಸಾವಧಾನದಿಂದ ಅಭ್ಯಾಸ ಮಾಡಬೇಕು. ಆವಿ ಪ್ರವೇಶಿಸಿದೊಡನೆಯೇ ರೈಲ್ವೆ ಯಂತ್ರ ಕೆಲಸ ಮಾಡಬೇಕು. ನಮ್ಮೆದುರಿಗೆ ಒಂದು ವಸ್ತು ಕಂಡೊಡನೆಯೇ ಅದರ ಕಡೆ ಮನಸ್ಸು ಹೋಗುತ್ತದೆ. ಆದರೆ ಮನುಷ್ಯನು ತಾನು ಒಂದು ಯಂತ್ರವಲ್ಲ ಎಂಬುದನ್ನು ಖಚಿತ ಮಾಡಬೇಕಾದರೆ ಯಾವುದರ ಆಡಳಿತಕ್ಕೂ ತಾನು ಒಳಪಟ್ಟಿಲ್ಲ ಎಂಬುದನ್ನು ತೋರಿಸಬೇಕು. ಮನಸ್ಸನ್ನು ಆಯಾ ಇಂದ್ರಿಯಗಳ ಕೇಂದ್ರಕ್ಕೆ ಹೋಗದಂತೆ ತಡೆದು ನಮ್ಮ ಸ್ವಾಧೀನಕ್ಕೆ ತರುವುದೇ ಪ್ರತ್ಯಾಹಾರ. ಇದನ್ನು ಹೇಗೆ ಅಭ್ಯಾಸ ಮಾಡುವುದು? ಇದೇನು ಒಂದು ದಿನದಲ್ಲಿ ಸಾಧ್ಯವಾಗುವುದಿಲ್ಲ, ಬಹಳ ಕಷ್ಟವಾದ ಕೆಲಸ. ಅನೇಕ ವರುಷಗಳು ತಾಳ್ಮೆಯಿಂದ ನಾವು ಎಡೆಬಿಡದೆ ಪ್ರಯತ್ನ ಮಾಡಿದರೆ ಮಾತ್ರ ಯಶಸ್ವಿಯಾಗಬಹುದು. 

ಕೆಲವು ಕಾಲ ಪ್ರತ್ಯಾಹಾರವನ್ನು ಅಭ್ಯಾಸಮಾಡಿ ಆದಮೇಲೆ, ಮುಂದಿನ ಸಾಧನೆಯಾದ, ನಿರ್ದಿಷ್ಟ ವಸ್ತುವಿನ ಮೇಲೆ ಮನಸ್ಸನ್ನು ನಿಲ್ಲಿಸುವ ಧಾರಣವನ್ನು ತೆಗೆದುಕೊಳ್ಳಿ. ನಿರ್ದಿಷ್ಟ ಕೇಂದ್ರಗಳ ಮೇಲೆ ಮನಸ್ಸನ್ನು ನಿಲ್ಲಿಸುವುದೆಂದರೇನು ಅರ್ಥ? ಶರೀರದ ಬೇರೆ ಭಾಗಗಳನ್ನು ಬಿಟ್ಟು ಕೆಲವು ಭಾಗಗಳನ್ನು ಮಾತ್ರ ಭಾವಿಸುವಂತೆ ಮನಸ್ಸನ್ನು ಒತ್ತಾಯಿಸುವುದು. ಉದಾಹರಣೆಗೆ, ದೇಹದ ಉಳಿದ ಅವಯವಗಳನ್ನು ಬಿಟ್ಟು ಕೈಯನ್ನು ಮಾತ್ರವೇ ಆಲೋಚಿಸಿ, ಚಿತ್ತ ಅಥವಾ ಮನಸ್ಸು ಯಾವುದೋ ಒಂದು ಭಾಗಕ್ಕೆ ಸೀಮಿತವಾದಾಗ ಅದಕ್ಕೆ ಧಾರಣವೆಂದು ಹೆಸರು. ಈ ಧಾರಣ ಅನೇಕ ವಿಧವಾಗಿರುವುದು. ಇದರೊಂದಿಗೆ ಸ್ವಲ್ಪ ಕಲ್ಪನಾ ಶಕ್ತಿಯನ್ನು ಬೆರೆಸುವುದು ಮೇಲು. ಉದಾಹರಣೆಗೆ, ಹೃದಯದಲ್ಲಿ ಯಾವುದೋ ಒಂದು ಭಾಗವನ್ನು ಮನಸ್ಸು ಆಲೋಚಿಸುವಂತೆ ಮಾಡಬೇಕು. ಅದು ಬಹಳ ಕಷ್ಟ. ಅಲ್ಲಿ ಒಂದು ಕಮಲವನ್ನು ಚಿತ್ರಿಸಿಕೊಳ್ಳುವುದು ಸುಲಭ. ಆ ಕಮಲವು ಪ್ರಕಾಶಮಾನವಾಗಿರುವುದು ಎಂದು ಭಾವಿಸಿ. ಮನಸ್ಸನ್ನು ಅಲ್ಲಿಡಿ. ಇಲ್ಲದೇ ಇದ್ದರೆ ಮಿದುಳಿನಲ್ಲಿರುವ ಕಮಲವು ಪ್ರಕಾಶ ಮಾನವಾಗಿರುವುದೆಂದು ಯೋಚಿಸಿ. ಅಥವಾ ಸುಷುಮ್ನಾ ಕಾಲುವೆಯ ಮೇಲೆ ಇರುವ ಇನ್ನು ಇತರ ಚಕ್ರಗಳ ಮೇಲೆ ಮನಸ್ಸನ್ನಿರಿಸಿ. 

ಯೋಗಿಗಳು ಯಾವಾಗಲೂ ಅಭ್ಯಾಸ ಮಾಡಬೇಕು. ಯಾವಾಗಲೂ ಏಕಾಂಗಿಯಾಗಿ\break ರಲು ಪ್ರಯತ್ನಿಸಬೇಕು. ಅನ್ಯ ಸ್ವಭಾವದವರ ಸಂಗ ಮನಸ್ಸನ್ನು ಚಂಚಲಗೊಳಿಸುವುದು. ಹೆಚ್ಚು ಮಾತನಾಡಕೂಡದು, ಏಕೆಂದರೆ ಮಾತು ಮನಸ್ಸನ್ನು ಚಂಚಲಗೊಳಿಸುತ್ತದೆ. ಹೆಚ್ಚು ಕೆಲಸ ಮಾಡಕೂಡದು. ಹೆಚ್ಚು ಕೆಲಸ ಮನಸ್ಸನ್ನು ಚದುರಿಸುವುದು. ಹಗಲೆಲ್ಲ ಕಷ್ಟಪಟ್ಟು ಕೆಲಸ ಮಾಡಿದ ಮೇಲೆ ಮನಸ್ಸನ್ನು ನಿಗ್ರಹಿಸುವುದು ಅಸಾಧ್ಯ. ಮೇಲಿನ ನಿಯಮಗಳನ್ನು ಯಾರು ಪಾಲಿಸುತ್ತಾರೊ ಅವರು ಯೋಗಿಗಳಾಗುತ್ತಾರೆ. ಬಹಳ ಸ್ವಲ್ಪ ಮುಂದುವರಿದರೂ ಕೂಡ ಅದರಿಂದ ಎಷ್ಟೋ ಪ್ರಯೋಜನವಾಗುವುದು. ಇದೇ ಯೋಗದ ಮಹಿಮೆ. ಯಾರಿಗೂ ಇದು ತೊಂದರೆಯನ್ನು ಉಂಟುಮಾಡುವುದಿಲ್ಲ. ಎಲ್ಲರಿಗೂ ಪ್ರಯೋಜನಕಾರಿಯಾಗಿರುವುದು. ಮೊದಲನೆಯದಾಗಿ ಇದು ನರಗಳ ಉದ್ವೇಗವನ್ನು ತಗ್ಗಿಸುವುದು; ಶಾಂತಿಯನ್ನು ಕೊಡುವುದು. ವಸ್ತುಗಳನ್ನು ಸ್ಪಷ್ಟವಾಗಿ ನೋಡಲು ಸಹಕಾರಿಯಾಗುವುದು. ಸ್ವಭಾವ ಉತ್ತಮವಾಗುವುದು. ಆರೋಗ್ಯವೂ ಉತ್ತಮವಾಗುವುದು. ಮಂಜುಳವಾದ ಧ್ವನಿ ಮತ್ತು ಆರೋಗ್ಯ ಇವುಗಳು ಮೊದಲನೆ ಚಿಹ್ನೆಗಳು. ಧ್ವನಿಯಲ್ಲಿರುವ ಕುಂದುಕೊರತೆಗಳು ಬದಲಾಯಿಸುವುದು. ಬರುವ ಅನೇಕ ಪ್ರಯೋಜನಗಳಲ್ಲಿ ಮೊದಲನೆಯದು ಇದು. ಯಾರು ಕಷ್ಟಪಟ್ಟು ಅಭ್ಯಾಸ ಮಾಡುತ್ತಾರೋ ಅವರಿಗೆ ಇನ್ನೂ ಅನೇಕ ಚಿಹ್ನೆಗಳು ಕಾಣುವುವು. ಕೆಲವು ವೇಳೆ ದೂರದ ಘಂಟೆಗಳ ಧ್ವನಿಯಂತೆ, ಶಬ್ದಗಳು ಕಲೆತು ಒಂದು ನಿರಂತರ ಶಬ್ದ ಪ್ರವಾಹದಂತೆ ಕೇಳಿಸುತ್ತದೆ. ಕೆಲವು ವೇಳೆ ಕೆಲವು ವಸ್ತುಗಳು ಕಾಣುವುವು. ಬೆಳಕಿನ ಸಣ್ಣ ಕಣಗಳು ತೇಲುತ್ತಾ ಬರುತ್ತಾ ದೊಡ್ಡದಾಗುವುವು. ಇವುಗಳು ಕಂಡ ಮೇಲೆ ನೀವು ಚುರುಕಾಗಿ ಮುಂದುವರಿಯುತ್ತಿರುವಿರಿ ಎಂದು ತಿಳಿಯಿರಿ. 

ಯಾರಿಗೆ ತಾವು ಯೋಗಿಗಳಾಗಬೇಕು, ತಾವು ಕಷ್ಟಪಟ್ಟು ಅಭ್ಯಾಸ ಮಾಡಬೇಕು ಎಂದು ಆಸೆ ಇದೆಯೋ ಅವರು ಮೊದಲು ತಮ್ಮ ಆಹಾರದ ವಿಷಯದಲ್ಲಿ ಎಚ್ಚರಿಕೆ ವಹಿಸ\break ಬೇಕು. ಆದರೆ ಯಾರು ದೈನಿಕ ದಿನಚರಿಯಂತೆ ಸ್ವಲ್ಪ ಅಭ್ಯಾಸ ಮಾಡುತ್ತಾರೊ ಅವರು ತಮಗೆ ಬೇಕಾದುದನ್ನು ಉಣ್ಣಬಹುದು. ಆದರೆ ಅತಿಯಾಗಿ ಉಣ್ಣಬಾರದು. ಅಷ್ಟೇ. ಯಾರು ತ್ವರಿತವಾಗಿ ಮುಂದುವರಿಯಬೇಕು, ಕಷ್ಟಪಟ್ಟು ಸಾಧನೆ ಮಾಡಬೇಕು ಎಂದು ಬಯಸುವರೊ ಅವರಿಗೆ ಒಂದು ನಿಯಮಿತ ಆಹಾರ ಅತ್ಯಾವಶ್ಯಕ. ಕೆಲವು ತಿಂಗಳುಗಳು ಹಾಲಿನ ಮತ್ತು ದ್ವಿದಳ ಧಾನ್ಯಗಳ ಮೇಲೆ ಮಾತ್ರ ಜೀವಿಸುವುದು ಅವರಿಗೆ ಅನುಕೂಲವೆನಿಸುತ್ತದೆ. ದೇಹ ಸೂಕ್ಷ್ಮವಾಗುತ್ತ ಬಂದಂತೆಲ್ಲ ನಿಯಮದಲ್ಲಿ ಸ್ವಲ್ಪ ಅವ್ಯವಸ್ಥೆಯಾದರೂ ಅವರ ಸಮತ್ವಕ್ಕೆ ಭಂಗವುಂಟಾಗುವುದು. ಒಂದು ತುತ್ತು ಆಹಾರ ಹೆಚ್ಚು ಕಡಮೆಯಾದರೂ ಅವರ ಶರೀರಕ್ಕೆ ತೊಂದರೆಯಾಗುವುದು. ಸಂಪೂರ್ಣ ನಿಗ್ರಹ ಸಿದ್ಧಿಸುವವರೆಗೆ ಈ ನಿಯಮವನ್ನು ಪಾಲಿಸಬೇಕು. ಅನಂತರ ಏನು ಬೇಕಾದರೂ ಊಟ ಮಾಡಬಹುದು. 

ಚಿತ್ತೈಕಾಗ್ರತೆಯನ್ನು ಪ್ರಾರಂಭಿಸುವಾಗ ಒಂದು ಗುಂಡುಸೂಜಿ ಬಿದ್ದರೂ ಅದು ತಲೆಯಲ್ಲಿ ದೊಡ್ಡ ಸಿಡಿಲು ಹೊಡೆದಂತೆ ಕೇಳುವುದು. ಇಂದ್ರಿಯಗಳು ಸೂಕ್ಷ್ಮವಾದಷ್ಟೂ ಇಂದ್ರಿಯಗ್ರಹಣವೂ ಸೂಕ್ಷ್ಮವಾಗುವುದು. ಇವು ನಾವು ಸಾಗಿಹೋಗಬೇಕಾಗಿರುವ ಅನೇಕ ಘಟ್ಟಗಳು. ಯಾರು ಮರಳಿ ಮರಳಿ ಪ್ರಯತ್ನ ಪಡುವರೊ ಅವರು ಜಯಶೀಲರಾಗುವರು. ವಾದ ಮತ್ತು ನಮ್ಮ ಮನಸ್ಸನ್ನು ಚಂಚಲಪಡಿಸುವ ಇತರ ಹವ್ಯಾಸಗಳನ್ನು ತ್ಯಜಿಸಿ. ಕೇವಲ ಒಣಪಾಂಡಿತ್ಯದ ಕಸರತ್ತಿನಿಂದ ಏನು ಪ್ರಯೋಜನ? ಇದು ನಮ್ಮ ಮನಸ್ಸಿನ ಸಮತ್ವವನ್ನು ಕೆಡಿಸಿ ಅಶಾಂತಿಯನ್ನು ಮಾತ್ರ ತರುತ್ತದೆ. ಸೂಕ್ಷ್ಮ ಪ್ರಪಂಚದಲ್ಲಿರುವ ವಿಷಯಗಳನ್ನು ನಾವೇ ಕಣ್ಣಾರೆ ಅನುಭವಿಸಬೇಕು. ಇದಕ್ಕೆ ಮಾತು ಸಹಾಯ ಮಾಡುವುದೆ? ಆದಕಾರಣವೆ ಬರಿಯ ಮಾತನ್ನು ಬಿಡಿ. ಯಾರಿಗೆ ಸತ್ಯ ಸಾಕ್ಷಾತ್ಕಾರವಾಗಿದೆಯೋ ಅವರು ಬರೆದ ಕೆಲವು ಪುಸ್ತಕಗಳನ್ನು ಓದಿ. 

ಮುತ್ತಿನ ಕಪ್ಪೆಯಚಿಪ್ಪಿನಂತೆ ಇರಿ. ಸ್ವಾತೀ ನಕ್ಷತ್ರದ ಮಳೆ ಬೀಳುತ್ತಿರುವಾಗ, ಅದರ ಹನಿ ಕಪ್ಪೆಯಚಿಪ್ಪಿನ ಒಳಕ್ಕೆ ಬಿದ್ದರೆ ಆ ಹನಿ ಮುತ್ತಾಗುವುದು ಎಂಬ ಒಂದು ಕಥೆ ಇಂಡಿಯಾ ದೇಶದಲ್ಲಿ ಬಳಕೆಯಲ್ಲಿದೆ. ಕಪ್ಪೆಯ ಚಿಪ್ಪಿಗೆ ಇದು ಗೊತ್ತಿದೆ. ಅದಕ್ಕೇ ಸ್ವಾತೀ ನಕ್ಷತ್ರದಲ್ಲಿ ನೀರಿನ ಮೇಲೆ ಬಂದು ತನ್ನ ಚಿಪ್ಪನ್ನು ತೆರೆದುಕೊಂಡು, ಬೀಳುವ ಅಮೂಲ್ಯವಾದ ಹನಿಗೆ ಕಾಯುತ್ತ ತೇಲುತ್ತಿರುವುದು. ಅದರೊಳಗೆ ಹನಿಯೊಂದು ಬಿದ್ದ ತಕ್ಷಣವೇ ಮುಚ್ಚಿ, ನಿಧಾನವಾಗಿ ಅದನ್ನು ಮುತ್ತು ಮಾಡಲು ಕಡಲಿನ ಆಳಕ್ಕೆ ಹೋಗುತ್ತದೆ. ನಾವು ಅವುಗಳಂತೆ ಇರಬೇಕು. ಮೊದಲು ಕೇಳಬೇಕು, ಅನಂತರ ಅದನ್ನು ತಿಳಿದುಕೊಳ್ಳಬೇಕು. ಅನಂತರ ಎಲ್ಲಾ ಚಂಚಲತೆಯನ್ನು ತೊರೆದು ಎಲ್ಲಾ ಬಾಹ್ಯವಸ್ತುಗಳಿಂದಲೂ ನಮ್ಮ ಮನಸ್ಸನ್ನು ಸೆಳೆದು ನಮ್ಮಲ್ಲಿರುವ ಸತ್ಯವನ್ನು ಅಭಿವ್ಯಕ್ತಪಡಿಸುವುದರಲ್ಲಿ ನಿರತರಾಗಬೇಕು. ಒಂದು ವಿಷಯ ಹೊಸತಾಗಿದೆ ಎಂದು ಮೊದಲು ಅದನ್ನು ಸ್ವೀಕರಿಸಿ ಅನಂತರ ಅದಕ್ಕಿಂತ ಹೊಸದಾಗಿರುವುದು ಸಿಕ್ಕಿದಾಗ ಮೊದಲಿನದನ್ನು ತೊರೆಯುವುದರಿಂದ ಮನಸ್ಸಿನ ಶಕ್ತಿ ವ್ಯರ್ಥವಾಗುವುದು. ಈ ಒಂದು ಅಪಾಯದಿಂದ ದೂರವಿರಬೇಕು. ಯಾವುದಾದರೊಂದು ವಿಷಯವನ್ನು ತೆಗೆದುಕೊಂಡು ಅದನ್ನು ಸಾಧಿಸಿ, ಕೊನೆ ಮುಟ್ಟುವ ತನಕ ಬಿಡಬೇಡಿ. ಯಾವುದಾದರೂ ಒಂದು ಭಾವನೆಯ ವಿಷಯದಲ್ಲಿ ಹುಚ್ಚು ಹಿಡಿದಂತೆ ಯಾರು ಅದನ್ನು ತೀವ್ರವಾಗಿ ಹಿಡಿಯುವನೋ ಅವನಿಗೆ ಬೆಳಕು ಕಾಣುತ್ತದೆ. ಯಾರು ಇಲ್ಲಿ ಸ್ವಲ್ಪ ಅಲ್ಲಿ ಸ್ವಲ್ಪ ಪ್ರಯತ್ನಿಸುವರೊ ಅವರಿಗೆ ಏನೂ ಸಿದ್ಧಿಸುವುದಿಲ್ಲ. ಕ್ಷಣಕಾಲ ಅವರು ತಮ್ಮ ಮನಸ್ಸಿನ ಚಪಲತೆಯನ್ನು ತೀರಿಸಿಕೊಳ್ಳಬಹುದು. ಆದರೆ ಅದು ಅಲ್ಲಿಗೇ ಕೊನೆಗಾಣುವುದು. ಅವರು ಬಾಹ್ಯ ಪ್ರಕೃತಿಯ ದಾಸರಾಗುವರು. ಇಂದ್ರಿಯವನ್ನು ಮೀರಿ ಎಂದಿಗೂ ಹೋಗಲಾರರು. 

ಯಾರು ನಿಜವಾಗಿಯೂ ಯೋಗಿಗಳಾಗಬೇಕೆಂದು ಬಯಸುವರೋ ಅವರು ಎಲ್ಲವನ್ನೂ ಸ್ವಲ್ಪ ಸ್ವಲ್ಪ ರುಚಿನೋಡುವ ಅಭ್ಯಾಸವನ್ನು ಒಂದೇ ಸಲ ಬಿಟ್ಟು ಬಿಡಬೇಕು. ಒಂದು ಭಾವನೆಯನ್ನು ತೆಗೆದುಕೊಳ್ಳಿ. ಅದನ್ನು ನಿಮ್ಮ ಜೀವನದಲ್ಲಿ ಅಭ್ಯಾಸ ಮಾಡಿ, ಅದನ್ನೇ ಆಲೋಚನೆ ಮಾಡಿ. ಅದನ್ನೇ ಕನಸು ಕಾಣಿ. ಆ ಒಂದು ಭಾವನೆಗಾಗಿ ನಿಮ್ಮ ಬಾಳನ್ನೆಲ್ಲ ಸವೆಸಿ, ಮೆದುಳು, ಮಾಂಸಖಂಡಗಳು, ನರಗಳು ಮತ್ತು ನಿಮ್ಮ ದೇಹದ ಪ್ರತಿಯೊಂದು ಭಾಗವೂ ಕೂಡ ಆ ಭಾವದಿಂದ ತುಂಬಿ ತುಳುಕಾಡಲಿ. ಉಳಿದ ಆಲೋಚನೆಗಳನ್ನೆಲ್ಲ ಅವುಗಳ ಪಾಡಿಗೆ ಬಿಡಿ. ಜಯ ಪಡೆಯುವುದಕ್ಕೆ ಇದೊಂದೇ ದಾರಿ. ಮಹಾ ಆಧ್ಯಾತ್ಮಿಕ ವೀರರು ಆಗುವ ರೀತಿಯೆ ಇದು. ಉಳಿದವರು ಕೇವಲ ಮಾತನಾಡುವ ಯಂತ್ರಗಳು. ನಾವು ನಿಜವಾಗಿಯೂ ಮುಕ್ತರಾಗಬೇಕಾದರೆ, ಮತ್ತು ಇತರರನ್ನು ಬಂಧಮುಕ್ತರನ್ನಾಗಿ ಮಾಡಬೇಕಾದರೆ ನಾವು ಇನ್ನೂ ಆಳಕ್ಕೆ ಹೋಗಬೇಕು. ಅದಕ್ಕೆ ಮೊದಲನೆಯ ಸಲಹೆಯೇ ಮನಸ್ಸನ್ನು ಕದಲಿಸದೆ ಇರುವುದು. ಯಾರ ಆಲೋಚನೆಗಳು ನಮ್ಮ ಭಾವನೆಯನ್ನು ಚಂಚಲಗೊಳಿಸುತ್ತವೆಯೋ ಅವರೊಂದಿಗೆ ಬೆರೆಯಕೂಡದು. ಕೆಲವು ಸ್ಥಳ, ಕೆಲವು ವ್ಯಕ್ತಿ ಮತ್ತು ಕೆಲವು ಆಹಾರ ಜುಗುಪ್ಸೆಯನ್ನುಂಟುಮಾಡುವುದೆಂದು ನಿಮಗೆ ಗೊತ್ತಿದೆ. ಅವುಗಳಿಂದ ದೂರವಾಗಿ, ಯಾರು ಈ ಜೀವನದಲ್ಲಿ ತುತ್ತ ತುದಿಗೇರಬೇಕೆಂದು ಬಯಸುವರೊ ಅವರು ಒಳ್ಳೆಯ ಮತ್ತು ಕೆಟ್ಟ ಸಹವಾಸಗಳನ್ನೆಲ್ಲ ತ್ಯಜಿಸಬೇಕು. ಕಷ್ಟಪಟ್ಟು ಸಾಧನೆ ಮಾಡಿ. ನೀವು ಬದುಕಿದರೇನು ಸತ್ತರೇನು, ಚಿಂತೆಯಿಲ್ಲ. ಫಲಾಪೇಕ್ಷೆ ಇಲ್ಲದೆ ಕೆಲಸಕ್ಕೆ ಕೈಹಾಕಬೇಕು. ನೀವು ಧೈರ್ಯಶಾಲಿಗಳಾದರೆ ಆರು ತಿಂಗಳಲ್ಲಿ ಸಿದ್ಧ ಯೋಗಿಗಳಾಗುವಿರಿ. ಆದರೆ ಯಾರು ಇದನ್ನು ಸ್ವಲ್ಪ, ಜೊತೆಗೆ ಉಳಿದವುಗಳನ್ನೆಲ್ಲ ಸ್ವಲ್ಪ ಸ್ವಲ್ಪವಾಗಿ ಸ್ವೀಕರಿಸುತ್ತಾರೆಯೋ, ಅವರು ಎಂದಿಗೂ ಮುಂದುವರಿಯುವುದೇ ಇಲ್ಲ. ಸುಮ್ಮನೆ ಕೆಲವು ಉಪನ್ಯಾಸಗಳನ್ನು ಕೇಳಿದರೆ ಏನೂ ಪ್ರಯೋಜನವಿಲ್ಲ. ಯಾರು ತುಂಬಾ ತಾಮಸಿಗಳೋ, ಸೋಮಾರಿಗಳೋ, ದಡ್ಡರೋ, ಯಾರ ಮನಸ್ಸು ಯಾವ ಒಂದು ಆಲೋಚನೆಯ ಮೇಲೂ ನಿಲ್ಲಲಾರದೋ, ಬೇಜಾರನ್ನು ಪರಿಹರಿಸಿಕೊಳ್ಳುವುದಕ್ಕಾಗಿ ಮಾತ್ರ ಏನನ್ನಾದರೂ ಆಶಿಸುತ್ತಾರೋ, ಅಂತಹವರಿಗೆ ಧರ್ಮ, ತತ್ತ್ವ ಮುಂತಾದುವುಗಳೆಲ್ಲ ಕೇವಲ ಮನೋರಂಜನೆಯ ವಸ್ತುಗಳು. ಇವರೆಂದಿಗೂ ಛಲದಿಂದ ಒಂದು ಆದರ್ಶವನ್ನು ಹಿಡಿಯುವವರಲ್ಲ. ಇಂತಹವರು ಒಂದು ಉಪನ್ಯಾಸವನ್ನು ಕೇಳುತ್ತಾರೆ, ಬಹಳ ಚೆನ್ನಾಗಿದೆ ಎನ್ನುತ್ತಾರೆ, ಅನಂತರ ಮನೆಗೆ ಹೋಗಿ ಎಲ್ಲವನ್ನೂ ಮರೆಯುತ್ತಾರೆ. ಇದರಲ್ಲಿ ಜಯ ಹೊಂದಬೇಕಾದರೆ ಅತ್ಯದ್ಭುತವಾದ ಇಚ್ಛಾಶಕ್ತಿ ಛಲ ಇವುಗಳು ಬೇಕು. “ಸಾಗರವನ್ನೆ ಕುಡಿಯುತ್ತೇನೆ”, “ಪರ್ವತಗಳು ಧೂಳಿ ದೂಸರವಾಗುವುವು, ನನ್ನ ಇಚ್ಛಾಶಕ್ತಿಯಿಂದ” ಎನ್ನುವನು ಛಲಗಾರ. ಅಂತಹ ಶಕ್ತಿ ಇರಲಿ, ಅಂತಹ ಶಪಥವಿರಲಿ. ಕಷ್ಟಪಟ್ಟು ಸಾಧನೆಮಾಡಿ. ನೀವು ಗುರಿಯನ್ನು ಸೇರಿಯೇ ಸೇರುತ್ತೀರಿ.

\chapter{ಧ್ಯಾನ ಮತ್ತು ಸಮಾಧಿ}

\vskip 0.2cm

ನಮ್ಮ ಮನಸ್ಸನ್ನು ಏಕಾಗ್ರಗೊಳಿಸುವ ಯಾವ ಮುಖ್ಯ ಗುರಿಗೆ ರಾಜಯೋಗ ನಮ್ಮನ್ನು ಕರೆದೊಯ್ಯುವುದೋ ಅಂತಹ ಸೂಕ್ಷ್ಮ ಮಾರ್ಗವನ್ನು ಹೊರತು ಉಳಿದ ಎಲ್ಲಾ ಮೆಟ್ಟಲುಗಳನ್ನೂ ಸಂಕ್ಷೇಪವಾಗಿ ನೋಡಿದ್ದಾಯಿತು. ಯುಕ್ತಿಪೂರಿತವಾದುದೆಂದು ಕರೆಯಲ್ಪಡುವ ನಮ್ಮ ಜ್ಞಾನವೆಲ್ಲವೂ ಪ್ರಜ್ಞೆಗೆ ಅನ್ವಯಿಸುವುದೆಂದು ಮಾನವರಾದ ನಮಗೆ ತೋರುವುದು. ಎದುರಿಗಿರುವ ಮೇಜಿನ ಮತ್ತು ನಿಮ್ಮ ಪ್ರಜ್ಞೆ ನನಗಿರುವುದರಿಂದ ಮೇಜು ಮತ್ತು ನೀವು ಇಲ್ಲಿರುವಿರಿ ಎಂಬುದನ್ನು ತಿಳಿಯಲು ಅನುಕೂಲವಾಗಿದೆ. ಅದೇ ಸಮಯದಲ್ಲಿ ನನ್ನ ಪ್ರಜ್ಞೆಗೆ ಬರದ ನನ್ನ ಅಸ್ತಿತ್ವದ ಎಷ್ಟೋ ಅಂಶಗಳಿವೆ. ದೇಹದಲ್ಲಿರುವ ಬೇರೆ ಬೇರೆ ಅವಯವ, ಮಿದುಳಿನ ಬೇರೆ ಬೇರೆ ಭಾಗ, ಇವು ನಮ್ಮ ಪ್ರಜ್ಞೆಗೆ ದೂರವಾಗಿವೆ. 

\vskip 0.3cm

ನಾನು ಊಟ ಮಾಡುವಾಗ ಪ್ರಜ್ಞಾಪೂರ್ವಕವಾಗಿ ಊಟಮಾಡುವೆನು. ಅದನ್ನು ಅರಗಿಸಿಕೊಳ್ಳುವಾಗ ಅದನ್ನು ಅರಿವಿಲ್ಲದೆ ಮಾಡಿಕೊಳ್ಳುವೆನು. ಆಹಾರದಿಂದ ರಕ್ತ ತಯಾರಾಗುವುದು ನನ್ನ ಅರಿವಿಲ್ಲದೆ. ಆದರೂ ಕೂಡ ಇದನ್ನು ಮಾಡುತ್ತಿರುವವನು ನಾನು. ಈ ಒಂದು ದೇಹದಲ್ಲಿ ಅದನ್ನೆಲ್ಲ ಮಾಡುವ ಇಪ್ಪತ್ತು ಬೇರೆಬೇರೆ ಜನರಿಲ್ಲ. ಇದನ್ನೆಲ್ಲ ಮಾಡುವವನು ನಾನೆ, ಬೇರೆಯವರಲ್ಲ ಎಂಬುದು ಹೇಗೆ ಗೊತ್ತು? ಊಟ ಮಾಡುವುದು ಮತ್ತು ಅದನ್ನು ಅರಗಿಸಿಕೊಳ್ಳುವುದು ಮಾತ್ರ ನನ್ನ ಕೆಲಸ, ಆಹಾರದಿಂದ ದೇಹವನ್ನು ಪೋಷಿಸುವುದನ್ನು ನನಗೋಸ್ಕರವಾಗಿ ಮತ್ತೊಬ್ಬರು ಮಾಡುವ ಕೆಲಸವೆಂದು ವಾದಿಸಬಹುದು. ಆದರೆ ಅದು ಸಾಧ್ಯವಲ್ಲ. ಈಗ ನಮ್ಮ ಅರಿವಿಲ್ಲದೆ ಆಗುತ್ತಿರುವ ಬಹುಪಾಲು ಕ್ರಿಯೆಯನ್ನು ನಮ್ಮ ಅರಿವಿನ ಬೆಳಕಿಗೆ ತರಬಹುದೆಂಬುದನ್ನು ಪ್ರದರ್ಶಿಸಬಹುದು. ನಮ್ಮ ಸ್ವಾಧೀನವಿಲ್ಲದೆ ಹೃದಯ ಚಲಿಸುತ್ತದೆ. ನಮ್ಮಲ್ಲಿ ಯಾರೂ ಹೃದಯವನ್ನು ತಡೆಯಲಾರರು, ಅದು ತನ್ನಿಚ್ಛೆಯಂತೆ ನಡೆಯುವುದು. ಆದರೆ ಅಭ್ಯಾಸ ಬಲದಿಂದ ನಮ್ಮ ಹೃದಯವನ್ನು ಕೂಡ ಸ್ವಾಧೀನಕ್ಕೆ ತರಬಹುದು. ಅದು ನಮ್ಮಿಚ್ಛೆಯಂತೆ ವೇಗವಾಗಿ ಅಥವಾ ನಿಧಾನವಾಗಿ ಬಡಿಯುವಂತೆ ಮಾಡಬಹುದು. ಅದನ್ನು ಬೇಕಾದರೆ ನಿಲ್ಲಿಸುವವರೆಗೂ ಹೋಗಬಹುದು. ನಮ್ಮ ದೇಹದ ಬಹುಭಾಗವನ್ನು ನಮ್ಮ ಸ್ವಾಧೀನಕ್ಕೆ ತೆಗೆದುಕೊಂಡು ಬರಬಹುದು. ಇದು ಏನನ್ನು ತೋರುತ್ತದೆ? ನಮ್ಮ ಅರಿವಿನ ಕೆಳಗಿರುವ ಈ ಕ್ರಿಯೆ ಕೂಡ ನಡೆಯುತ್ತಿರುವುದು ನಮ್ಮಿಂದಲೇ. ಆದರೆ ಅದು ನಮ್ಮ ಅರಿವಿಲ್ಲದೇ ಆಗುತ್ತಿರುವುದು. ಆದಕಾರಣ ಮಾನವನ ಮನಸ್ಸು ಕೆಲಸ ಮಾಡುವ ಎರಡು ಸ್ತರಗಳಿವೆ ಎಂದ ಹಾಗಾಯಿತು. ಮೊದಲನೆಯದು ಅರಿವಿನಿಂದ ಮಾಡುವ ಸ್ತರ. ಅಲ್ಲಿ ಮಾಡುವ ಎಲ್ಲಾ ಕೆಲಸಗಳೂ ಅಹಂಕಾರದಿಂದ ಕೂಡಿವೆ. ಎರಡನೆಯದೆ ನಮ್ಮ ಅರಿವಿಲ್ಲದೆ ನಡೆಯುವ ಸ್ತರ. ಅಲ್ಲಿ ಯಾವ ಕೆಲಸದಲ್ಲಿ ಕೂಡ ನಾನು ಎಂಬ ಅಹಂಕಾರ ಇರುವುದಿಲ್ಲ. ನಾನು ಎಂಬ ಅಹಂಕಾರದಿಂದ ಕೂಡದೆ ಇರುವ ಮನಸ್ಸಿನ ಕೆಲಸವೆ ಪ್ರಜ್ಞಾರಹಿತ ಕೆಲಸ. ಅಹಂಭಾವದಿಂದ ಕೂಡಿದ ಕೆಲಸವೆ ಪ್ರಜ್ಞಾಪೂರ್ಣ ಕೆಲಸ. ಕೆಳಮಟ್ಟದ ಪ್ರಾಣಿಗಳಲ್ಲಿ ಈ ಪ್ರಜ್ಞಾರಹಿತ ಕೆಲಸವನ್ನು ಹುಟ್ಟುಗುಣ ಎನ್ನುತ್ತಾರೆ. ಮೇಲಿನ ವರ್ಗದ ಪ್ರಾಣಿಗಳಲ್ಲಿ, ಪ್ರಾಣಿಗಳಲ್ಲೆಲ್ಲಾ ಸರ್ವಶ್ರೇಷ್ಠನಾದ ಮಾನವನಲ್ಲಿ ಅರಿವಿನಿಂದ ಕೂಡಿದ ಕೆಲಸವಿದೆ. 

\vskip 0.3cm

ಆದರೆ ಇದು ಇಲ್ಲಿಗೇ ಕೊನೆಗಾಣುವುದಿಲ್ಲ. ಮನಸ್ಸು ಕೆಲಸ ಮಾಡುವ ಇನ್ನೂ ಉತ್ತಮ ಸ್ತರವೂ ಇದೆ. ಮನಸ್ಸು ಅರಿವಿನಾಚೆಯೂ ಹೋಗಬಲ್ಲದು. ಹೇಗೆ ನಮ್ಮ ಅರಿವಿಲ್ಲದ ಕೆಲಸ ನಮ್ಮ ಪ್ರಜ್ಞೆಯ ಕೆಳಗೆ ಇದೆಯೋ, ಅದರಂತೆಯೇ ನಮ್ಮ ಪ್ರಜ್ಞೆಯ ಮೇಲೆ ಇರುವ ಕೆಲಸವೂ ಇದೆ. ಇಲ್ಲಿಯೂ ಕೂಡ ನಾನು ಎಂಬ ಅಹಂಕಾರ ಭಾವನೆ ಇರುವುದಿಲ್ಲ. ಮಧ್ಯದ ಸ್ತರದಲ್ಲಿ ಮಾತ್ರ ಅಹಂಭಾವವಿರುತ್ತದೆ. ಮನಸ್ಸು ಅದಕ್ಕಿಂತ ಮೇಲೆ ಅಥವಾ ಕೆಳಗೆ ಇರುವಾಗ ಅಲ್ಲಿ ನಾನು ಎಂಬ ಭಾವನೆ ಇರುವುದಿಲ್ಲ. ಆದರೂ ಅದು ಕೆಲಸ ಮಾಡಬಲ್ಲದು. ಮನಸ್ಸು ನಾನು ಎಂಬ ಭಾವನೆಯನ್ನು ಮೀರಿದಾಗ ಅದಕ್ಕೆ ಸಮಾಧಿ ಅಥವಾ ಪ್ರಜ್ಞಾತೀತಾವಸ್ಥೆ ಎಂದು ಹೆಸರು. ಉದಾಹರಣೆಗೆ ಸಮಾಧಿಯಲ್ಲಿರುವವನು ಮೇಲಕ್ಕೆ ಏರುವ ಬದಲು ಕೆಳಗೆ ಜಾರಿಲ್ಲ ಎಂಬುದನ್ನು ಹೇಗೆ ತಿಳಿದುಕೊಳ್ಳಬೇಕು? ಎರಡು ಪ್ರಸಂಗಗಳಲ್ಲಿಯೂ ಕೆಲಸದೊಂದಿಗೆ ಅಹಂಕಾರದ ಭಾವನೆ ಇರುವುದಿಲ್ಲ. ಪರಿಣಾಮದಿಂದ, ಕೆಲಸದ ಫಲದಿಂದ ಯಾವುದು ಅಹಂಕಾರದ ಕೆಳಗೆ ಇರುವುದು, ಯಾವುದು ಅಹಂಕಾರದ ಮೇಲಿರುವುದು ಎಂಬುದನ್ನು ತಿಳಿದುಕೊಳ್ಳಬಹುದು. ಮನುಷ್ಯನು ಆಳ ನಿದ್ರೆಯಲ್ಲಿರುವಾಗ ಅರಿವಿನ ಕೆಳಗಿರುವ ಕ್ಷೇತ್ರವನ್ನು ಪ್ರವೇಶಿಸುತ್ತಾನೆ. ನಿದ್ರೆಯಲ್ಲಿರುವಾಗಲೂ ಅವನು ದೇಹದಿಂದ ಕೆಲಸ ಮಾಡುತ್ತಾನೆ, ಉಸಿರಾಡುತ್ತಾನೆ, ತನ್ನ ದೇಹವನ್ನು ಚಲಿಸುತ್ತಾನೆ. ಆದರೆ ಆಗ ಅವನಲ್ಲಿ ತಾನು ಮಾಡುತ್ತಿರುವೆನೆಂಬ ಅಹಂಭಾವವಿರುವುದಿಲ್ಲ. ಆಗ ಅವನಿಗೆ ಪ್ರಜ್ಞೆ ಇರುವುದಿಲ್ಲ. ನಿದ್ರೆಯಿಂದ ಎಚ್ಚೆತ್ತ ಮೇಲೆ, ಯಾರು ನಿದ್ರೆ ಮಾಡಲು ಹೋದನೋ ಅದೇ ಹಳೆಯ ಮನುಷ್ಯನಾಗಿರುತ್ತಾನೆ. ನಿದ್ರೆ ಮಾಡುವುದಕ್ಕೆ ಮುಂಚೆ ಅವನಲ್ಲಿದ್ದ ಜ್ಞಾನದ ಮೊತ್ತವು, ನಿದ್ರೆಯನ್ನು ತೀರಿಸಿಕೊಂಡು ಬಂದ ಮೇಲೆಯೂ ಹೆಚ್ಚದೆ, ಅಷ್ಟೇ ಇರುತ್ತದೆ. ಯಾವ ಜ್ಞಾನವೂ ಬರುವುದಿಲ್ಲ. ಆದರೆ ಒಬ್ಬನು ಸಮಾಧಿಗೆ ಹೋದರೆ ಅವನು ಹೋಗುವಾಗ ಮೂಢನಾಗಿದ್ದರೆ ಜ್ಞಾನಿಯಾಗಿ ಹಿಂತಿರುಗುವನು. 

\vskip 0.3cm

ಯಾವುದು ಈ ವ್ಯತ್ಯಾಸಕ್ಕೆ ಕಾರಣ? ಒಂದು ಸ್ಥಿತಿಯಲ್ಲಿ ಹೋದಂತೆಯೇ ಹಿಂತಿರುಗುತ್ತಾನೆ. ಮತ್ತೊಂದು ಸ್ಥಿತಿಯಲ್ಲಿ ಜ್ಞಾನಿಯಾಗಿ, ಧರ್ಮಾತ್ಮನಾಗಿ, ಋಷಿಯಾಗಿ, ಒಬ್ಬ ದೇವದೂತನಾಗಿ ಹಿಂತಿರುಗುತ್ತಾನೆ. ಅವನ ಶೀಲವೆಲ್ಲ ಪವಿತ್ರವಾಗುವುದು, ಜೀವನವೇ ಬದಲಾಯಿಸುವುದು, ಆತ್ಮೋದ್ಧಾರವಾಗುವುದು. ಇವು ಎರಡು ಪರಿಣಾಮಗಳು. ಈಗ ಫಲ ವ್ಯತ್ಯಾಸವಾಗಿರುವುದರಿಂದ, ಕಾರಣ ಕೂಡ ವ್ಯತ್ಯಾಸವಾಗಬೇಕಾಗಿದೆ. ಸಮಾಧಿಯಿಂದ ಹಿಂತಿರುಗುವಾಗ ಬರುವ ಜ್ಞಾನ, ಅರಿವಿಲ್ಲದ ಅವಸ್ಥೆಯಲ್ಲಿ ಬರುವ ಅಥವಾ ಜಾಗೃತಾವಸ್ಥೆಯಲ್ಲಿ ಯುಕ್ತಿ ಪೂರ್ಣವಾಗಿ ಬರುವ ಜ್ಞಾನಕ್ಕಿಂತ ಬಹಳ ಮೇಲುವರ್ಗಕ್ಕೆ \break ಸೇರಿದುದು. ಆದಕಾರಣ ಇದು ಪ್ರಜ್ಞಾತೀತವಾಗಿರಬೇಕು. ಸಮಾಧಿಯನ್ನು ಪ್ರಜ್ಞಾತೀತಾವಸ್ಥೆ ಎಂದು ಕರೆಯುವರು. 

\vskip 0.3cm

ಸಮಾಧಿ ಎಂಬುದರ ಸಂಕ್ಷೇಪಾರ್ಥವಿದು. ಅದನ್ನು ಹೇಗೆ ಉಪಯೋಗಿಸುವುದು? ಅದರ ಉಪಯೋಗ ಎಲ್ಲಿದೆ? ಯುಕ್ತಿ ಅಥವಾ ಜಾಗ್ರತಾವಸ್ಥೆಯಲ್ಲಿ ಕೆಲಸ ಮಾಡುವ ಮನಸ್ಸಿನ ಪರಿಧಿಯು ಬಹಳ ಕಿರಿದು. ಒಂದು ಸಣ್ಣ ವೃತ್ತದೊಳಗೆ ಮಾತ್ರ ಮನುಷ್ಯನ ಯುಕ್ತಿ ಸಂಚಾರ ಮಾಡಬಲ್ಲದು, ಅದನ್ನು ಮೀರಿ ಅದು ಹೋಗಲಾರದು. ಅದನ್ನು ಮೀರಿ ಹೋಗುವ ನಮ್ಮ ಪ್ರಯತ್ನವೆಲ್ಲವೂ ಅಸಾಧ್ಯ. ಆದರೂ ಮಾನವ ಕೋಟಿಯು ಪ್ರಿಯತಮವೆಂದು ಭಾವಿಸುವ ವಿಷಯವೆಲ್ಲವೂ ಯುಕ್ತಿಯೆಂಬ ಬೇಲಿಯಾಚೆ ಇರುವುವು. ಜನನಮರಣಾತೀತ ಆತ್ಮನಿರುವನೆ? ದೇವನಿರುವನೆ? ಪ್ರಪಂಚವನ್ನಾಳುವ ಯಾವುದಾದರೊಂದು ಪರಮಶಕ್ತಿ ಇದೆಯೆ, ಇಲ್ಲವೆ ಎಂಬ ಪ್ರಶ್ನೆಗಳೆಲ್ಲ ಯುಕ್ತಿಯ ಮೇರೆಯನ್ನು ಮೀರಿ ಇರುವುವು. ಈ ಪ್ರಶ್ನೆಗಳಿಗೆ ಯುಕ್ತಿಯು ಎಂದಿಗೂ ಉತ್ತರವನ್ನು ಕೊಡಲಾರದು. ಯುಕ್ತಿ ಹೇಳುವುದೇನು? “ನಾನು ಆಜ್ಞೇಯತಾವಾದಿ, ಈಶ್ವರನು ಇರುವನು ಅಥವಾ ಇಲ್ಲ ಎಂಬುದು ಗೊತ್ತಿಲ್ಲ” ಎಂದು ಅದು ಹೇಳುವುದು. ಆದರೂ ಈ ಪ್ರಶ್ನೆಗಳು ನಮಗೆ ಅತಿ ಮುಖ್ಯವಾದುವು. ಅವುಗಳಿಗೆ ಸರಿಯಾದ ಉತ್ತರ ದೊರಕದೇ ಇದ್ದರೆ ಜೀವನ ಉದ್ದೇಶಹೀನವಾಗುವುದು. ನಮ್ಮ ಧಾರ್ಮಿಕ ಸಿದ್ಧಾಂತಗಳೆಲ್ಲ, ನಮ್ಮ ನೈತಿಕ ದೃಷ್ಟಿಯೆಲ್ಲ, ಮಾನವ ಸ್ವಭಾವದಲ್ಲಿ ಒಳ್ಳೆಯದು ಮತ್ತು ಶ್ರೇಷ್ಠವಾದುವುಗಳೆಲ್ಲ, ಯುಕ್ತಿ ಎಂಬ ಕ್ಷೇತ್ರದ ಆಚೆಯಿಂದ ಬಂದ ಉತ್ತರದ ತಳಹದಿಯ ಮೇಲೆ ನಿಂತಿದೆ. ಆದಕಾರಣ ಈ ಪ್ರಶ್ನೆಗಳಿಗೆ ಉತ್ತರವನ್ನು ಪಡೆಯುವುದು ಬಹಳ ಮುಖ್ಯವಾಗಿದೆ. ಜೀವನ ಒಂದು ಕ್ಷಣಿಕ ಆಟವಾದರೆ, ಪ್ರಪಂಚ ಎಂಬುದು ಕಣಗಳ ಆಕಸ್ಮಿಕ ಸಂಯೋಗದಿಂದ ಉಂಟಾಗಿದ್ದರೆ, ನಾನು ಮತ್ತೊಬ್ಬರಿಗೆ ಒಳ್ಳೆಯದನ್ನು ಏಕೆ ಮಾಡಬೇಕು! ದಯೆ, ನ್ಯಾಯ, ಸಹೋದರಭಾವನೆ ಇವುಗಳೆಲ್ಲ ಏಕಿರಬೇಕು? ಪ್ರತಿಯೊಬ್ಬನಿಗೂ ಎಷ್ಟು ಸಾಧ್ಯವೋ ಅಷ್ಟು ಸುಖವನ್ನು ಹೀರುವುದೇ ಪ್ರಪಂಚದಲ್ಲಿ ಅತ್ಯುತ್ತಮವಾದ ಆದರ್ಶವಾಗುವುದು. ನೆಚ್ಚಿಗೆ ಎಂಬುದು ಇಲ್ಲದೇ ಇದ್ದರೆ, ನನ್ನ ಸಹೋದರನನ್ನು ನಾನು ಏಕೆ ಪ್ರೀತಿಸಬೇಕು? ಅವನ ಕೊರಳನ್ನೇಕೆ ಕತ್ತರಿಸ ಬಾರದು? ಈ ಜೀವನದಾಚೆ ಏನೂ ಇಲ್ಲದೆ ಇದ್ದರೆ, ಸ್ವಾತಂತ್ರವೆಂಬುದಿಲ್ಲದೆ ಕೇವಲ ಜಡ ನಿಯಮಗಳಿದ್ದರೆ, ನಾನು ಮಾತ್ರ ಇಲ್ಲಿ ಸುಖವಾಗಿರುವುದಕ್ಕೆ ಪ್ರಯತ್ನಿಸುತ್ತೇನೆ. ಈಗಿನ ಕಾಲದಲ್ಲಿ ಕೆಲವರು ನೀತಿಗೆ ಪ್ರಯೋಜನ ದೃಷ್ಟಿಯೇ ತಳಹದಿ ಎಂದು ಹೇಳುತ್ತಿರುವರು. ಈ ತಳಹದಿ ಯಾವುದು? ಸಾಧ್ಯವಾದಷ್ಟು ಹೆಚ್ಚು ಜನಕ್ಕೆ ಹೆಚ್ಚು ಸುಖವನ್ನು ಒದಗಿಸಿಕೊಡುವುದು. ನಾನೇಕೆ ಇದನ್ನು ಮಾಡಬೇಕು? ನನ್ನ ಬಯಕೆಗಳಿಗೆ ಸಹಾಯವಾಗುವ ಹಾಗಿದ್ದರೆ, ಸಾಧ್ಯವಾದಷ್ಟು ಹೆಚ್ಚು ಮಂದಿಗೆ ಸಾಧ್ಯವಾದಷ್ಟು ಹೆಚ್ಚು ದುಃಖವನ್ನು ನಾನೇಕೆ ಕೊಡಬಾರದು? ಪ್ರಯೋಜನ ದೃಷ್ಟಿಯುಳ್ಳವರು ಈ ಪ್ರಶ್ನೆಗೆ ಹೇಗೆ ಉತ್ತರ ಕೊಡುತ್ತಾರೆ? ಯಾವುದು ಸರಿ; ಯಾವುದು ತಪ್ಪು ಎಂಬುದು ನಿಮಗೆ ಹೇಗೆ ಗೊತ್ತು? ನಾನು ಸುಖದ ಆಸಕ್ತಿಗೆ ಒಳಗಾಗಿರುವೆನು. ಇದನ್ನು ನಾನು ಈಡೇರಿಸಿಕೊಳ್ಳುವೆನು, ಇದೇ ನನ್ನ ಸ್ವಭಾವ. ಇದಕ್ಕಿತ ಹೆಚ್ಚಿಗೆ ನನಗೇನೂ ಗೊತ್ತಿಲ್ಲ. ನನ್ನಲ್ಲಿ ಈ ಬಯಕೆಗಳಿವೆ, ಅವುಗಳನ್ನು ನಾನು ಪೂರ್ಣಮಾಡಿಕೊಳ್ಳಬೇಕು. ಅದಕ್ಕೆ ನೀವೇಕೆ ದೂರುವುದು? ಮಾನವ ಜೀವನ, ನೀತಿ, ಜನನ–ಮರಣಾತೀತ ಆತ್ಮ, ದೇವರು, ಪ್ರೀತಿ, ದಯೆ, ಒಳ್ಳೆಯದಾಗಿರುವುದು, ಎಲ್ಲಕ್ಕಿಂತ ಹೆಚ್ಚಾಗಿ ನಿಃಸ್ವಾರ್ಥಿಗಳಾಗಿರುವುದು–ಈ ಭಾವನೆಗಳೆಲ್ಲ ಎಲ್ಲಿಂದ ಬಂದವು?

\vskip 0.3cm

ಎಲ್ಲಾ ನೀತಿಗಳು, ಎಲ್ಲಾ ಮಾನವಕರ್ಮಗಳು, ಎಲ್ಲಾ ಮಾನವ ಆಲೋಚನೆಗಳು, ನಿಃಸ್ವಾರ್ಥತೆ ಎಂಬ ಒಂದು ಭಾವನೆಯ ಮೇಲೆ ನಿಂತಿವೆ. ಮಾನವ ಜೀವನದ ಇಡೀ ಭಾವನೆಯನ್ನು ನಿಃಸ್ವಾರ್ಥತೆ ಎಂಬ ಒಂದು ಪದದಲ್ಲಿಡಬಹುದು. ನಾವೇಕೆ ನಿಃಸ್ವಾರ್ಥಿಗಳಾಗಬೇಕು? ನನಗೆ ನಿಃಸ್ವಾರ್ಥನಾಗುವ ಆವಶ್ಯಕತೆಯೇನು? ಅದನ್ನು ಬಲಾತ್ಕರಿಸುವ ಶಕ್ತಿ ಯಾವುದು? ನಾನೊಬ್ಬ ವಿಚಾರವಾದಿ, ಪ್ರಯೋಜನ ದೃಷ್ಟಿಯುಳ್ಳವನು ಎಂದು ಜಂಭ ಕೊಚ್ಚಿಕೊಳ್ಳುತ್ತೀರಿ. ಆದರೆ ನಿಮ್ಮ ಪ್ರಯೋಜನ ದೃಷ್ಟಿಗೆ ಸಾಕಾದಷ್ಟು ಕಾರಣವನ್ನು ತೋರದೆ ಇದ್ದರೆ ನಿಮ್ಮನ್ನು ವಿಚಾರಹೀನರೆಂದು ಕರೆಯುತ್ತೇನೆ. ನಾನೇಕೆ ಸ್ವಾರ್ಥಪರನಾಗಬಾರದು ಎಂಬುದಕ್ಕೆ ಕಾರಣವನ್ನು ಹೇಳಿ. ಒಬ್ಬನಿಗೆ ನಿಃಸ್ವಾರ್ಥಪರನಾಗೆಂದು ಹೇಳುವುದು ಒಳ್ಳೆಯ ಕಾವ್ಯದಂತಿರಬಹುದು. ಆದರೆ ಕಾವ್ಯ ಯುಕ್ತಿಯಲ್ಲ, ನನಗೆ ಕಾರಣವನ್ನು ತೋರಿ. ನಾನೇಕೆ ನಿಃಸ್ವಾರ್ಥನಾಗಬೇಕು? ನಾನೇಕೆ ಒಳ್ಳೆಯವನಾಗಬೇಕು? ಯಾರೊ ಇಂತಿಂಥವರು ಹಾಗೆ ಹೇಳುತ್ತಾರೆ ಎಂದು ಹೇಳಿದರೆ ಅದನ್ನು ನಾನು ಒಪ್ಪುವುದಿಲ್ಲ. ಪ್ರಯೋಜನವೆಂದರೆ ಅತ್ಯಧಿಕ ಸುಖವನ್ನು ಪಡೆಯುವುದಾದರೆ ಸ್ವಾರ್ಥಪರನಾಗುವುದೇ ನನಗೆ ಪ್ರಯೋಜನಕರ. ಇದಕ್ಕೆ ಏನು ಉತ್ತರ? ಪ್ರಯೋಜನ ದೃಷ್ಟಿಯುಳ್ಳವನು ಇದಕ್ಕೆ ಉತ್ತರವನ್ನು ಎಂದಿಗೂ ಕೊಡಲಾರ. ಈ ಪ್ರಪಂಚವೆನ್ನುವುದು ಅನಂತಕಡಲಿನಲ್ಲಿ ಒಂದು ಬಿಂದು, ತುದಿಮೊದಲಿಲ್ಲದ ಸರಪಣಿಯಲ್ಲಿರುವ ಕೊಂಡಿ, ಎಂಬುದೇ ಇದಕ್ಕೆ ಉತ್ತರ. ಯಾರು ನಿಃಸ್ವಾರ್ಥತೆಯನ್ನು ಜನಾಂಗಕ್ಕೆ ಬೋಧಿಸಿದರೋ ಅವರಿಗೆ ಈ ಭಾವನೆ ಎಲ್ಲಿಂದ ಬಂತು? ಇದು ಹುಟ್ಟುಗುಣದಿಂದ ಬಂದಿರುವುದಲ್ಲ. ಬರೀ ಹುಟ್ಟುಗುಣವಿರುವ ಪ್ರಾಣಿಗಳಿಗೆ ಇದು ಗೊತ್ತಿಲ್ಲ. ಇದು ಯುಕ್ತಿಯೂ ಅಲ್ಲ. ಯುಕ್ತಿಗೆ ಈ ವಿಚಾರಗಳಾವುವೂ ತಿಳಿಯದು. ಆದರೆ ಅವುಗಳು ಎಲ್ಲಿಂದ ಬಂದುವು?

\vskip 0.3cm

ಚರಿತ್ರೆಯನ್ನು ಓದಿದರೆ ಜಗತ್ತಿನ ಎಲ್ಲಾ ಧಾರ್ಮಿಕ ಬೋಧಕರಲ್ಲಿಯೂ ಸಾಮಾನ್ಯವಾಗಿರುವ ಒಂದು ಸತ್ಯ ನಮಗೆ ತೋರುತ್ತದೆ. ಎಲ್ಲರೂ ತಮಗೆ ಕಂಡ ಸತ್ಯವು ಇಂದ್ರಿಯಾತೀತ ಪ್ರದೇಶದಿಂದ ಬಂದಿತೆಂದು ಪ್ರತಿಪಾದಿಸುವರು. ಹಲವರಿಗೆ ಆ ಸತ್ಯ ಎಲ್ಲಿಂದ ಬಂದಿತೆಂಬುದು ಗೊತ್ತಿಲ್ಲ ಅಷ್ಟೆ. ಉದಾಹರಣೆಗೆ ಒಬ್ಬನು, “ದೇವದೂತನು ರೆಕ್ಕೆಯ ಮನುಷ್ಯನಂತೆ ಬಂದು ‘ಹೇ ಮಾನವನೆ, ಕೇಳು ಇದೇ ಸಂದೇಶ’ ಎಂದಿರುವನು” ಎನ್ನುವನು. ಮತ್ತೊಬ್ಬನು, ತೇಜಸ್ವಿಯೊಬ್ಬನು ತನಗೆ ಕಾಣಿಸಿಕೊಂಡನೆಂದು ಹೇಳುವನು. ಮೂರನೆಯವನು, ತನ್ನ ಕನಸಿನಲ್ಲಿ ಪೂರ್ವಿಕನೊಬ್ಬನು ಬಂದು ಕೆಲವು ವಿಷಯಗಳನ್ನು ಹೇಳಿದನೆನ್ನುವರು. ಅದಕ್ಕಿಂತ ಹೆಚ್ಚಾಗಿ ಅವನಿಗೆ ಏನೂ ಗೊತ್ತಿಲ್ಲ. ಆದರೆ ಇದು ಮಾತ್ರ ಸಾಮಾನ್ಯವಾಗಿದೆ: ಅದೇ ಎಲ್ಲರೂ ತಮಗೆ ಈ ಜ್ಞಾನವು ಎಲ್ಲಿಯೋ ಅವ್ಯಕ್ತದಿಂದ ಬಂದಿತೆಂದು, ಅದು ಯುಕ್ತಿಮೂಲಕವಾಗಿ ಬಂದುದಲ್ಲವೆಂದು ಹೇಳುವುದು. ಯೋಗಶಾಸ್ತ್ರ ಏನನ್ನು ಬೋಧಿಸುವುದು? ಈ ಜ್ಞಾನವೆಲ್ಲ ಯುಕ್ತಿಯನ್ನು ಮೀರಿದ ಮತ್ತೊಂದು ಪ್ರದೇಶದಿಂದ ಬಂದಿತೆಂದು ಅದು ಪ್ರತಿಪಾದಿಸುವುದೇನೋ ನಿಜ; ಆದರೆ ಅವುಗಳೆಲ್ಲ ಬಂದುದು ಅವರ ಅಂತರಂಗದಿಂದಲೇ. 

\vskip 0.3cm

ಮನಸ್ಸಿಗೆ ಯುಕ್ತಿಯನ್ನು ಮೀರಿದ ಉನ್ನತಾವಸ್ಥೆಯೊಂದು ಇರುವುದು ಎಂದು ಯೋಗಿಯು ಬೋಧಿಸುವನು. ಆ ಮೇಲಿನ ಸ್ಥಿತಿಗೆ ಮನಸ್ಸು ಏರಿದಾಗ ಈ ತರ್ಕಾತೀತ ಜ್ಞಾನ ಬರುವುದು, ಅತಿಭೌತಿಕ ಮತ್ತು ಅಚಿಂತ್ಯಜ್ಞಾನ ಅವನಿಗೆ ಲಭಿಸುವುದು. ಈ ಯುಕ್ತಿಯನ್ನು ಮೀರಿಹೋಗುವ ಅವಸ್ಥೆಯು ಮನುಷ್ಯನ ಸ್ವಾಭಾವಿಕ ಸ್ಥಿತಿಯನ್ನು ಅತಿಕ್ರಮಿಸುವುದು. ಇದರ ರಹಸ್ಯ ತಿಳಿಯದೇ ಇರುವವನಿಗೂ ಕೆಲವು ವೇಳೆ ಅಕಸ್ಮಾತ್ತಾಗಿ ಈ ಅವಸ್ಥೆ ಉಂಟಾಗಬಹುದು. ಕೆಲವು ವೇಳೆ ಅವನು ಅದನ್ನು ಎಡವಿದಂತಿರುತ್ತದೆ. ಅವನು ಅದನ್ನು ಎಡವಿದಾಗ ಸಾಧಾರಣವಾಗಿ ಎಲ್ಲಿಯೊ ಹೊರಗಿನಿಂದ ಅದು ಬರುತ್ತದೆ ಎನ್ನುತ್ತಾನೆ. ಸ್ಫೂರ್ತಿ ಅಥವಾ ಅತೀಂದ್ರಿಯಜ್ಞಾನ ಎಲ್ಲಾ ದೇಶದಲ್ಲಿಯೂ ಒಂದೇ ಸಮನಾಗಿರಬಹುದು, ಆದರೆ ಬೇರೆ ಬೇರೆ ರೀತಿಯಲ್ಲಿ ಅದನ್ನು ವಿವರಿಸುತ್ತಾರೆ. ಒಂದು ಕಡೆ ದೇವದೂತನೊಬ್ಬನಿಂದ ಬಂದಿತೆಂದೂ, ಇನ್ನೊಂದು ಕಡೆ ದೇವತೆಗಳೊಬ್ಬರಿಂದ ಬಂದಿತೆಂದೂ, ಮತ್ತೊಂದು ಕಡೆ ಈಶ್ವರ ನಿಂದ ಬಂದಿತೆಂದೂ ನಾನಾ ವಿಧವಾಗಿ ವಿವರಿಸಬಹುದು. ಅಂದರೇನು? ಹಾಗೆಂದರೆ ಮನಸ್ಸು ಸ್ವಭಾವತಃ ಜ್ಞಾನವನ್ನು ತಂದಿತು. ಆದರೆ ಅದಕ್ಕೆ ಕಾರಣವನ್ನು ಕಂಡು ಹಿಡಿಯುವುದು ಮಾತ್ರ, ಆಯಾ ವ್ಯಕ್ತಿಗಳ ನಂಬಿಕೆ ಮತ್ತು ಬುದ್ಧಿವಂತಿಕೆಯ ಮೇಲೆ ನಿಂತಿದೆ. ಸತ್ಯಾಂಶವೇನೆಂದರೆ ಈ ಬೇರೆ ಬೇರೆ ವ್ಯಕ್ತಿಗಳು ಅತೀಂದ್ರಿಯ ಅವಸ್ಥೆಯನ್ನು ತಿಳಿಯದೆ ಎಡವಿದಂತಿರುವುದು. 

ಈ ಸ್ಥಿತಿಯನ್ನು ಈ ರೀತಿ ಎಡವಿ ಪಡೆಯುವುದರಿಂದ ಬಹಳ ಅಪಾಯವಿದೆ ಎಂದು ಯೋಗಿಯು ಹೇಳುತ್ತಾನೆ. ಅನೇಕ ವೇಳೆ ಅದನ್ನು ಪಡೆದವರು ಹುಚ್ಚರಾಗುವ ಅಪಾಯವೂ ಇದೆ. ಅದೂ ಅಲ್ಲದೆ ಯಾರೂ ಈ ಅತೀಂದ್ರಿಯ ಅವಸ್ಥೆಯನ್ನು ತಿಳಿಯದೆ ಎಡವಿದರೋ, ಅವರು ಎಷ್ಟೇ ದೊಡ್ಡವರಾಗಲೀ, ಅವರ ಜ್ಞಾನದೊಂದಿಗೆ ಸ್ವಲ್ಪ ಅಜ್ಞಾನ ಬೆರೆತೇ ಇರುತ್ತದೆ. ಅವರು ತಮ್ಮ ಚಿತ್ತವಿಕಾರಕ್ಕೆ ಅವಕಾಶವನ್ನು ಕಲ್ಪಿಸಿಕೊಂಡಿರುತ್ತಾರೆ. ಮಹಮ್ಮದನು ಒಂದು ದಿನ ಗ್ಯಾಬ್ರಿಯಲ್​ ದೇವದೂತನೊಬ್ಬನು ತನ್ನ ಗುಹೆಗೆ ಬಂದು, ಹರಾಕ್​ ಎಂಬ ದೇವಲೋಕದ ಕುದುರೆಯ ಮೇಲೆ ತನ್ನನ್ನು ಕೂಡಿಸಿಕೊಂಡು ದೇವಲೋಕಕ್ಕೆ ಹೋಗಿದ್ದನೆಂದು ಹೇಳುತ್ತಾನೆ. ಇಷ್ಟಾದರೂ ಕೂಡ ಮಹಮ್ಮದನು ಕೆಲವು ಅದ್ಭುತವಾದ ಸತ್ಯಗಳನ್ನು ನುಡಿದನು. ನೀವು ಖೊರಾನನ್ನು ಓದಿದರೆ, ಬಹಳ ಪ್ರಖ್ಯಾತವಾದ ಸತ್ಯವಾಣಿಗಳು ಮೂಢನಂಬಿಕೆಗಳೊಂದಿಗೆ ಬೆರೆತಿರುವುದು ಕಾಣುತ್ತವೆ. ಇವುಗಳನ್ನು ನೀವು ಹೇಗೆ ವಿವರಿಸುತ್ತೀರಿ? ಅವನು ಸ್ಫೂರ್ತಿಯನ್ನು ಪಡೆದಿದ್ದನು ಎಂಬುದೇನೋ ನಿಜ, ಆದರೆ ಅವನು ಆ ಸ್ಫೂರ್ತಿಯನ್ನು ಪಡೆದದ್ದು ಆಕಸ್ಮಿಕವಾಗಿ. ನಿಸ್ಸಂಶಯವಾಗಿಯೂ ಆತನು ತರಬೇತಿ ಹೊಂದಿದ ಯೋಗಿಯಾಗಿರಲಿಲ್ಲ. ಅದಕ್ಕೇ ತಾನು ಮಾಡುತ್ತಿರುವುದಕ್ಕೆ ಕಾರಣ ಅವನಿಗೆ ಗೊತ್ತಿರಲಿಲ್ಲ. ಮಹಮ್ಮದನು ಪ್ರಪಂಚಕ್ಕೆ ಮಾಡಿದ ಒಳ್ಳೆಯದನ್ನು ಆಲೋಚಿಸಿ, ಮತ್ತು ಅವನ ಮತಾಂಧತೆಯಿಂದ ಆದ ಮಹಾನಷ್ಟವನ್ನುಕುರಿತು ಆಲೋಚಿಸಿ! ಅವನ ಬೋಧನೆಯ ಪರಿಣಾಮವಾಗಿ ಲಕ್ಷಾಂತರ ಜನರು ಕೊಲ್ಲಲ್ಪಟ್ಟರು. ಸಾವಿರಾರು ತಾಯಂದಿರಿಗೆ ಮಕ್ಕಳ ವಿಯೋಗವಾಯಿತು. ಶಿಶುಗಳು ತಬ್ಬಲಿಗಳಾದರು, ದೇಶವೇ ಸರ್ವನಾಶವಾಯಿತು. ಲಕ್ಷಾಂತರ ಜನರು ಕೊಲೆ ಮಾಡಲ್ಪಟ್ಟರು. 

ಆದಕಾರಣವೆ ಮಹಮ್ಮದ್​ ಮುಂತಾದ ಪ್ರಖ್ಯಾತ ಬೋಧಕರ ಜೀವನವನ್ನು\break ವಿಮರ್ಶಿಸಿದರೆ ನಮಗೆ ಈ ಅಪಾಯ ತೋರುವುದು. ಪ್ರವಾದಿಗಳು ತಮ್ಮ ಭಾವವನ್ನು ಉದ್ರೇಕಗೊಳಿಸುವುದರ ಮೂಲಕ ಪ್ರಜ್ಞಾತೀತಾವಸ್ಥೆಗೆ ಏರಿದಾಗಲೆಲ್ಲಾ ಅಲ್ಲಿಂದ ಅವರು ಜಗತ್ತಿನ ಕಲ್ಯಾಣಕ್ಕಾಗಿ ಹಿಂದಿರುಗಿ ಬರುವಾಗ ಸತ್ಯವನ್ನು ಮಾತ್ರ ತರಲಿಲ್ಲ. ಜೊತೆಗೆ ಹಾನಿಕರವಾದ ಸ್ವಲ್ಪ ಮೂಢನಂಬಿಕೆಗಳನ್ನೂ ಅವರು ತಂದರು. ಮಾನವ ಜೀವನ ಎಂಬ ಅಸಂಬದ್ಧತೆಯ ರಾಶಿಯಿಂದ ಏನಾದರೂ ಸ್ವಲ್ಪ ಯುಕ್ತಿಯನ್ನು ಹೊಂದಬೇಕಾದರೆ ಯುಕ್ತಿಯನ್ನು ನಾವು ಮೀರಬೇಕು. ಆದರೆ ಅದನ್ನು ವೈಜ್ಞಾನಿಕ ರೀತಿಯಾದ, ನಿಯಮಿತ ಅಭ್ಯಾಸದ ಮೂಲಕ ನಿಧಾನವಾಗಿ ಮಾಡಬೇಕು. ಎಲ್ಲಾ ಮೂಢನಂಬಿಕೆಗಳನ್ನೂ ಕಿತ್ತೊಗೆಯಬೇಕು. ಅತೀಂದ್ರಿಯಾವಸ್ಥೆಯ ಅಧ್ಯಯನವನ್ನು ಮತ್ತೆ ಬೇರೆ ಯಾವುದಾದರೊಂದು ವಿಜ್ಞಾನಶಾಸ್ತ್ರವನ್ನು ಅಭ್ಯಾಸಮಾಡುವಂತೆ ಮಾಡಬೇಕು. ಯುಕ್ತಿಯ ಮೇಲೆ ನಮ್ಮ ತಳಹದಿಯನ್ನು ಕಟ್ಟಬೇಕು. ಎಲ್ಲಿಯವರೆವಿಗೂ ಅದು ನಮಗೆ ದಾರಿಯನ್ನು ತೋರಿಸಬಲ್ಲದೊ ಅಲ್ಲಿಯವರೆವಿಗೂ ಅದನ್ನು ನಾವು ಅನುಸರಿಸಬೇಕು. ಎಂದು ಯುಕ್ತಿಯು ಸೋಲುವುದೊ ಆಗ ಅದೇ ನಮಗೆ ಮೇಲಿನ ಸ್ಥಿತಿಗೆ ದಾರಿಯನ್ನು ತೋರುವುದು. ಯಾರಾದರೂ ತಾನು ಸ್ಫೂರ್ತಿ ಪಡೆದಿರುವೆನು ಎಂದು ಹೇಳಿ ಅಯುಕ್ತವಾಗಿ ಮಾತನಾಡಿದರೆ ಅದನ್ನು ತ್ಯಜಿಸಿ. ಅದೇತಕ್ಕೆ? ಏಕೆಂದರೆ ಹುಟ್ಟುಗುಣ, ಯುಕ್ತಿ, ಪ್ರಜ್ಞಾತೀತಸ್ಥಿತಿ ಅಥವಾ ಅಪ್ರಜ್ಞೆ, ಪ್ರಜ್ಞಾ ಪ್ರಜ್ಞೆಗೆ ಮೀರಿದ ಸ್ಥಿತಿಗಳು ಒಂದೇ ಮನಸ್ಸಿಗೆ ಸೇರಿದುವು. ಒಬ್ಬ ಮನುಷ್ಯನಲ್ಲಿ ಮೂರು ಮನಸ್ಸುಗಳಿಲ್ಲ. ಆದರೆ ಅದರ ಒಂದು ಸ್ಥಿತಿಯೇ ಮತ್ತೊಂದಾಗಿ ಬದಲಾವಣೆ ಹೊಂದುವುದು. ಹುಟ್ಟುಗುಣವೇ ಯುಕ್ತಿಯ ರೂಪವನ್ನು ತಾಳುವುದು. ಯುಕ್ತಿಯೇ ಪ್ರಜ್ಞಾತೀತಾವಸ್ಥೆಯಾಗುವುದು. ಆದಕಾರಣ ಯಾವ ಒಂದು ಅವಸ್ಥೆಯೂ ಕೂಡ ಮತ್ತೊಂದನ್ನು ವಿರೋಧಿಸುವುದಿಲ್ಲ. ನಿಜವಾದ ಅತೀಂದ್ರಿಯ ಜ್ಞಾನ ಯುಕ್ತಿಯನ್ನೆಂದಿಗೂ ವಿರೋಧಿಸುವುದಿಲ್ಲ. ಆದರೆ ಅದನ್ನು ಪೂರ್ತಿಗೊಳಿಸುವುದು. “ನಾನು ಧ್ವಂಸಮಾಡುವುದಕ್ಕೆ ಅಲ್ಲ ಬರುವುದು; ಪೂರ್ಣ ಮಾಡುವುದಕ್ಕೆ” ಎಂಬ ಮಹಾತ್ಮರ ವಾಣಿಯಂತೆ ಅತೀಂದ್ರಿಯ ಶಕ್ತಿಯು ಯಾವಾಗಲೂ ಯುಕ್ತಿಯನ್ನು ಪೂರ್ಣಗೊಳಿಸುವುದಕ್ಕೆ ಬರುವುದು, ಅದರೊಂದಿಗೆ ಸಾಮರಸ್ಯದಿಂದಿರುವುದು. 

ಪ್ರಜ್ಞಾತೀತಾವಸ್ಥೆ ಅಥವಾ ಸಮಾಧಿಗೆ, ವೈಜ್ಞಾನಿಕ ರೀತಿಯಲ್ಲಿ ಕರೆದುಕೊಂಡು ಹೋಗುವುದಕ್ಕಾಗಿಯೇ ಯೋಗದ ಹಲವು ಮೆಟ್ಟಲುಗಳು ಇರುವುದು. ಅದೂ ಅಲ್ಲದೆ ಎಲ್ಲಕ್ಕಿಂತ ಹೆಚ್ಚಾಗಿ ನಾವು ಇದನ್ನು ತಿಳಿದುಕೊಳ್ಳಬೇಕು: ಸ್ಫೂರ್ತಿಯನ್ನು ಪಡೆಯುವುದು ಪೂರ್ವಕಾಲದ ಮಹಾತ್ಮರಿಗೆ ಎಷ್ಟು ಸಾಧ್ಯವಿದ್ದಿತೋ ನಮಗೂ ಅಷ್ಟೇ ಸಾಧ್ಯ. ಆ ಮಹಾತ್ಮರೇನು ಅಸಾಧಾರಣ ವ್ಯಕ್ತಿಗಳಲ್ಲ. ಅವರು ಕೂಡ ನಮ್ಮಂತೆ ನಿಮ್ಮಂತೆಯೇ ಮನುಷ್ಯರು. ಅವರು ದೊಡ್ಡ ಯೋಗಿಗಳಾಗಿದ್ದರು. ಅವರು ಈ ಪ್ರಜ್ಞಾತೀತ ಸ್ಥಿತಿಯನ್ನು ಸಂಪಾದಿಸಿದ್ದರು. ನೀವು ಮತ್ತು ನಾವು ಕೂಡ ಇದನ್ನು ಪಡೆಯಬಹುದು. ಅವರೇನೂ ಅದ್ಭುತವಾದ ವ್ಯಕ್ತಿಗಳಲ್ಲ. ಒಬ್ಪರು ಆ ಅವಸ್ಥೆಯನ್ನು ಪಡೆದರು ಎಂಬುದೇ ಪ್ರತಿಯೊಬ್ಬರೂ ಕೂಡ ಆ ಅವಸ್ಥೆಯನ್ನು ಪಡೆಯಬಹುದು ಎಂಬುದನ್ನು ಸಾಧಿಸುವುದು. ಇದು ಎಲ್ಲರಿಗೂ ಸಾಧ್ಯ ಮಾತ್ರವಲ್ಲ, ಪ್ರತಿಯೊಬ್ಬರೂ ಕೂಡ ಕ್ರಮೇಣ ಆ ಅವಸ್ಥೆಯನ್ನು ಪಡೆಯಲೇಬೇಕು. ಧರ್ಮವೆಂಬುದೇ ಅದು. ಅನುಭವವೊಂದೇ ನಮಗಿರುವ ಗುರು. ನಮ್ಮ ಜೀವಮಾನವೆಲ್ಲ ನಾವು ಮಾತನಾಡಬಹುದು, ವಿಚಾರ ಮಾಡಬಹುದು. ಆದರೆ ಸ್ವಂತ ನಾವೇ ಅನುಭವಿಸುವವರೆಗೆ, ಧರ್ಮದ ಒಂದು ಮಾತಾದರೂ ಗೊತ್ತಾಗುವುದಿಲ್ಲ. ಕೆಲವು ವ್ಯೆದ್ಯಕೀಯ ಗ್ರಂಥಗಳನ್ನು ಕೊಟ್ಟ ಮಾತ್ರಕ್ಕೆ ನೀವು ಒಬ್ಬ ವ್ಯಕ್ತಿಯನ್ನು ಶಸ್ತ್ರಚಿಕಿತ್ಸಕನನ್ನಾಗಿ ಮಾಡಲಾರಿರಿ. ಒಂದು ದೇಶವನ್ನು ನೋಡಲು ನನಗಿರುವ ಕುತೂಹಲವನ್ನು ಭೂಪಟವನ್ನು ತೋರಿಸುವುದರಿಂದ ತೃಪ್ತಿಪಡಿಸಲಾರಿರಿ. ನನಗೆ ಪ್ರತ್ಯಕ್ಷ ಅನುಭವಬೇಕಾಗಿದೆ. ಭೂಪಟಗಳು ನಮ್ಮಲ್ಲಿ ಇನ್ನೂ ಹೆಚ್ಚು ಜ್ಞಾನವನ್ನು ಪಡೆಯಬೇಕು ಎನ್ನುವ ಕುತೂಹಲವನ್ನು ಹುಟ್ಟಿಸಬಹುದು. ಇದಕ್ಕಿಂತ ಹೆಚ್ಚಾಗಿ ಅದರಿಂದ ಏನೂ ಉಪಯೋಗವಿಲ್ಲ. ಕೇವಲ ಶಾಸ್ತ್ರಗಳನ್ನು ನೆಚ್ಚಿಕೊಂಡಿರುವುದು ನಮ್ಮನ್ನು ಹೀನಸ್ಥಿತಿಗೆ ತರುತ್ತದೆ. ದೇವರಿಗೆ ಸಂಬಂಧಪಟ್ಟ ಜ್ಞಾನವೆಲ್ಲವೂ ಯಾವುದೋ ಒಂದು ಧರ್ಮಗ್ರಂಥದಲ್ಲಿ ಅಡಗಿದೆ ಎಂದು ಹೇಳುವುದಕ್ಕಿಂತ ಘೋರವಾದ ಈಶ್ವರನಿಂದೆ ಇದೆಯೆ? ದೇವರನ್ನು ಸರ್ವಾಂತರ್ಯಾಮಿಯೆಂದು ಹೇಳಿದರೂ ಅವನನ್ನು ಒಂದು ಧರ್ಮಗ್ರಂಥದಲ್ಲಿ ಅಡಗಿಸುವ ಮನುಷ್ಯನ ಎದೆಗಾರಿಕೆ ಎಷ್ಟು! ಶಾಸ್ತ್ರದಲ್ಲಿ ಹೇಳಿದುದನ್ನು ನಂಬದಿದ್ದುದಕ್ಕಾಗಿ ಲಕ್ಷಾಂತರ ಜನರನ್ನು ಬಲಿ ಕೊಟ್ಟಿರುವರು. ಆ ಒಂದು ಸಣ್ಣ ಗ್ರಂಥದಲ್ಲಿ ದೇವರಿಗೆ ಸಂಬಂಧಪಟ್ಟ ವಿಷಯಗಳನ್ನೆಲ್ಲ ಅವರು ನೋಡಲಿಲ್ಲವೆನ್ನುವುದೇ ಅವರ ತಪ್ಪು! ಸಾವು ಮತ್ತು ಕೊಲೆಯೇನೊ ಈಗ ನಿಂತಿದೆ ನಿಜ. ಆದರೂ ಕೂಡ ಪ್ರಪಂಚವು ಶಾಸ್ತ್ರದ ನಂಬಿಕೆಯಲ್ಲಿ ಅತಿ ಘೋರವಾಗಿ ಬದ್ಧವಾಗಿದೆ. 

ಈ ಪ್ರಜ್ಞಾತೀತಾವಸ್ಥೆಯನ್ನು ವೈಜ್ಞಾನಿಕ ರೀತಿಯಲ್ಲಿ ತಲುಪಬೇಕಾದರೆ ನಾನು ಹೇಳುತ್ತಿರುವ ರಾಜಯೋಗದ ಹಲವು ಮೆಟ್ಟಿಲನ್ನು ದಾಟಿ ಹೋಗುವುದು ಅತ್ಯಾವಶ್ಯಕವಾಗಿದೆ. ಪ್ರತ್ಯಾಹಾರ ಧಾರಣಗಳಾದ ಮೇಲೆ ನಾವು ಧ್ಯಾನಕ್ಕೆ ಬರುತ್ತೇವೆ. ಕೆಲವು ಆಂತರಿಕ ಅಥವಾ ಬಾಹ್ಯ ವಸ್ತುವಿನ ಮೇಲೆ ಮನಸ್ಸನ್ನು ಕೇಂದ್ರೀಕರಿಸುವ ಅಭ್ಯಾಸ ಮಾಡಿದ ಮೇಲೆ, ಅದಕ್ಕೆ ಆ ವಸ್ತುವಿನೆಡೆಗೆ ನಿರಂತರವಾಗಿ ಹರಿಯುವ ಒಂದು ಶಕ್ತಿ ಬರುತ್ತದೆ. ಈ ಸ್ಥಿತಿಯನ್ನೆ ಧ್ಯಾನವೆನ್ನುವುದು. ಬಾಹ್ಯ ಇಂದ್ರಿಯ ಗ್ರಹಣವನ್ನು ತ್ಯಜಿಸುವಷ್ಟು ಮನಸ್ಸಿನ ಏಕಾಗ್ರತೆಯು ತೀವ್ರವಾಗಿ ಅಂತರಂಗ ಅಥವಾ ಅರ್ಥದಲ್ಲಿ ಮಾತ್ರ ಮನಸ್ಸು ತಲ್ಲೀನವಾದಾಗ ಅದನ್ನು ಸಮಾಧಿ ಎನ್ನುವುದು. ಧಾರಣ, ಧ್ಯಾನ, ಸಮಾಧಿ ಎಂಬ ಮೂರನ್ನು ಸಂಯಮವೆಂದು ಕರೆಯುತ್ತಾರೆ. ಅಂದರೆ ಮೊದಲು ವಸ್ತುವಿನ ಮೇಲೆ ಮನಸ್ಸನ್ನು ಕೇಂದ್ರೀಕರಿಸಬೇಕು; ಅನಂತರ ಸ್ವಲ್ಪ ಕಾಲದವರೆಗೆ ಏಕಾಗ್ರತೆಯಲ್ಲಿ ಮುಂದುವರಿಯಲು\break ಸಾಧ್ಯವಾಗಬೇಕು; ಅನಂತರ ಅನವರತ ಏಕಾಗ್ರತೆಯ ಮೂಲಕ, ವಸ್ತು ನಮಗೆ ಕಾಣುವಂತೆ ಮಾಡಿದ ಇಂದ್ರಿಯಗ್ರಹಣದ ಒಳಭಾಗದ ಮೇಲೆ ಮನಸ್ಸನ್ನು ಕೇಂದ್ರೀಕರಿಸುವುದು ಬಂದರೆ ಸರ್ವವೂ ಅಂತಹ ಮನಸ್ಸಿನ ಸ್ವಾಧೀನವಾಗುತ್ತದೆ. 

ಈ ಧ್ಯಾನಸ್ಥಿತಿಯೆ ಅಸ್ತಿತ್ವದ ಪರಮಾವಸ್ಥೆ. ಎಲ್ಲಿಯವರೆವಿಗೂ ಆಸೆ ಇರುವುದೊ ಅಲ್ಲಿಯವರೆವಿಗೂ ನಿಜವಾದ ಸಂತೋಷ ಬರುವುದಿಲ್ಲ. ಧ್ಯಾನಸ್ಥಿತಿಯಲ್ಲಿ ಸಾಕ್ಷಿಯಾಗಿ ವಸ್ತುಗಳನ್ನು ಪರೀಕ್ಷಿಸುವುದೇ ನಮಗೆ ನಿಜವಾದ ಸಂತೋಷವನ್ನು ಮತ್ತು ಸುಖವನ್ನು ಕೊಡುವುದು. ಪ್ರಾಣಿಗಳಿಗೆ ಅವುಗಳ ಸುಖ ಇಂದ್ರಿಯದಲ್ಲಿದೆ; ಮಾನವನಿಗೆ ಬುದ್ಧಿಯಲ್ಲಿದೆ; ಇಂತಹ ಧ್ಯಾನಸ್ಥಿತಿಗೆ ಏರಿದ ಜೀವನಿಗೆ ಮಾತ್ರ ಜಗತ್ತು ನಿಜವಾಗಿ ಸುಂದರವಾಗುವುದು. ಯಾರು ಏನನ್ನೂ ಆಶಿಸುವುದಿಲ್ಲವೋ, ಅವರಿಗೆ ಪ್ರಕೃತಿಯ ನಾನಾ ಬದಲಾವಣೆಗಳು ಸೌಂದರ್ಯ ಮತ್ತು ಭವ್ಯತೆಯ ದೃಶ್ಯಗಳಾಗಿ ಕಾಣಬರುತ್ತವೆ. 

\vskip 5pt

ಈ ಭಾವನೆಗಳನ್ನು ನಾವು ಧ್ಯಾನದಲ್ಲಿ ತಿಳಿಯಬೇಕಾಗಿದೆ. ನಾವೊಂದು ಶಬ್ದವನ್ನು ಕೇಳುತ್ತೇವೆ. ಮೊದಲು ಅದರ ಬಾಹ್ಯಸ್ಪಂದನ, ಎರಡನೆಯದಾಗಿ ಅದನ್ನು ಮಿದುಳಿಗೆ ಒಯ್ಯುವ ನರಗಳ ಚಲನೆ, ಮೂರನೆಯದು ಮಾನಸಿಕ ಪರಿವರ್ತನೆ ಇದರೊಂದಿಗೆ ಬಾಹ್ಯಸ್ಪಂದನದಿಂದ ಹಿಡಿದು ಮಾನಸಿಕ ಪರಿವರ್ತನೆಯವರೆವಿಗೂ ಆದ ಬದಲಾವಣೆಗಳಿಗೆ ಕಾರಣಭೂತವಾದ ವಸ್ತು ನಮಗೆ ಗೋಚರಿಸುವುದು. ಯೋಗದಲ್ಲಿ ಇವು ಮೂರಕ್ಕೆ ಶಬ್ದ, ಅರ್ಥ, ಜ್ಞಾನವೆಂದು ಹೆಸರು. ಶಾರೀರಿಕ ಶಾಸ್ತ್ರದ ಪ್ರಕಾರ ಇವುಗಳಿಗೆ ಬಾಹ್ಯಸ್ಪಂದನ, ಮಿದುಳು ಮತ್ತು ನರಗಳ ಚಲನೆ. ಮಾನಸಿಕ ಪ್ರತಿಕ್ರಿಯೆ ಎಂದು ಹೆಸರು. ಇವುಗಳು ಪ್ರತ್ಯೇಕವಾದ ಬದಲಾವಣೆಗಳಾದರೂ ಸ್ವಲ್ಪವೂ ವ್ಯತ್ಯಾಸವಿಲ್ಲದಂತೆ ಕಲೆತುಹೋಗಿವೆ. ನಿಜವಾಗಿಯೂ ನಮಗೆ ಇವಾವುದೂ ತೋರುವುದಿಲ್ಲ. ಬಾಹ್ಯ ವಸ್ತುವೆಂಬ ಒಟ್ಟು ಪರಿಣಾಮವನ್ನು ಮಾತ್ರ ಕಾಣುತ್ತೇವೆ. ಪ್ರತಿಯೊಂದು ಇಂದ್ರಿಯ ಗ್ರಹಣದಲ್ಲಿಯೂ ಈ ಮೂರು ಕ್ರಿಯೆಗಳೂ ಕಲೆತಿವೆ. ನಾವು ಅವುಗಳನ್ನು ಪ್ರತ್ಯೇಕವಾಗಿ ತಿಳಿದುಕೊಳ್ಳದೇ ಇರುವುದಕ್ಕೆ ಯಾವ ಕಾರಣಗಳೂ ಇಲ್ಲ. 

\vskip 5pt

ಹಿಂದಿನ ಸಿದ್ಧತೆಗಳ ಪರಿಣಾಮವಾಗಿ ಮನಸ್ಸು ದೃಢವಾಗಿ ನಿಗ್ರಹಕ್ಕೊಳಪಟ್ಟು, ಸೂಕ್ಷ್ಮ ಇಂದ್ರಿಯ ಗ್ರಹಣ ಶಕ್ತಿಯನ್ನು ಪಡೆದಮೇಲೆ ಅದನ್ನು ಧ್ಯಾನದಲ್ಲಿ ತೊಡಗಿಸಬೇಕು. ಧ್ಯಾನ ಸ್ಥೂಲ ವಸ್ತುಗಳಿಂದ ಮೊದಲಾಗಿ, ಕ್ರಮೇಣ ಸೂಕ್ಷ್ಮ ಸೂಕ್ಷ್ಮವಾಗಿ, ಅವಸ್ತುವಾಗುವವರೆಗೂ ಮೇಲಕ್ಕೆ ಏರಬೇಕು. ಸಂವೇದನೆಯ ಬಾಹ್ಯಕಾರಣಗಳನ್ನು ನೋಡುವುದರಲ್ಲಿ ಮನಸ್ಸನ್ನು ಮೊದಲು ನೇಮಿಸಬೇಕು. ಅನಂತರ ಆಂತರಿಕ ಚಲನೆಗಳು, ಅನಂತರ ತನ್ನದೇ ಪ್ರತಿಕ್ರಿಯೆ–ಇವುಗಳ ಮೇಲೆ ಮನಸ್ಸನ್ನು ನಿಲ್ಲಿಸಬೇಕು. ಸಂವೇದನೆಯ ಬಾಹ್ಯ ಕಾರಣಗಳನ್ನು ನೋಡುವುದರಲ್ಲಿ ಜಯಶೀಲರಾದ ಮೇಲೆ, ಎಲ್ಲಾ ಸೂಕ್ಷ್ಮವಸ್ತುಗಳನ್ನು ಮತ್ತು ರೂಪಗಳನ್ನು ನೋಡುವ ಶಕ್ತಿ ಮನಸ್ಸಿಗೆ ಬರುತ್ತದೆ. ಮನಸ್ಸು ತನ್ನಲ್ಲಿ ಆಗುತ್ತಿರುವ ಬದಲಾವಣೆಗಳನ್ನು ಮಾತ್ರ ಗ್ರಹಿಸುವುದನ್ನು ಕಲಿತ ಮೇಲೆ, ತನ್ನಲ್ಲಿ ಮತ್ತು ಇತರರಲ್ಲಿ ಇರುವ ಎಲ್ಲಾ ಮಾನಸಿಕ ತರಂಗಗಳು ಅವನ ಅಧೀನವಾಗುವುವು, ಅವು ಬಾಹ್ಯ ಕ್ರಿಯಾ ರೂಪದಲ್ಲಿ ವ್ಯಕ್ತವಾಗುವುದಕ್ಕೆ ಮುಂಚೆಯೇ ಅವನಿಗೆ ಅವು ತಿಳಿಯುವುವು. ಮಾನಸಿಕ ಪ್ರತಿಕ್ರಿಯೆಯನ್ನು ಮಾತ್ರ ಗ್ರಹಿಸುವುದನ್ನು ಕಲಿತ ಮೇಲೆ ಸರ್ವವಿಷಯಗಳ ಜ್ಞಾನವೂ ಯೋಗಿಯ ಅಧೀನವಾಗುವುದು. ನಾವು ತಿಳಿಯಬಲ್ಲ ಪ್ರತಿಯೊಂದು ವಸ್ತು ಮತ್ತು ಆಲೋಚನೆ ಈ ಪ್ರತಿಕ್ರಿಯೆಯ ಪರಿಣಾಮವಾಗಿದೆ. ಆಗ ತನ್ನ ಮನಸ್ಸಿನ ತಳಹದಿಯನ್ನೇ ಅವನು ನೋಡಿದಂತೆ ಆಗುವುದು ಮತ್ತು ಮನಸ್ಸು ಅವನ ಪೂರ್ಣ ಸ್ವಾಧೀನಕ್ಕೆ ಒಳಪಡುವುದು. ಆಗ ಯೋಗಿಗೆ ಅನೇಕ ಶಕ್ತಿಗಳು ಬರುತ್ತವೆ. ಇವುಗಳಲ್ಲಿ ಯಾವುದಾದರೂ ಒಂದರ ಚಪಲಕ್ಕೆ ಒಳಗಾದರೂ ಅವನು ಮುಂದುವರಿಯುವುದಕ್ಕೆ ಅಡ್ಡಿ ಉಂಟಾಗುವುದು. ಭೋಗಾನ್ವೇಷಣೆಯ ಕೆಟ್ಟ ಪರಿಣಾಮವಿದು. ಈ ಅದ್ಭುತ ಶಕ್ತಿಯನ್ನು ಕೂಡ ಅವನು ನಿರಾಕರಿಸುವಷ್ಟು ಶಕ್ತನಾಗಿದ್ದರೆ, ಮನಸ್ಸಿನ ಚಿತ್ತವೃತ್ತಿಯನ್ನು ಸಂಪೂರ್ಣ ನಿರೋಧಿಸಬಲ್ಲ ರಾಜಯೋಗದ ಗುರಿಯನ್ನು ಅವನು ಹೊಂದುತ್ತಾನೆ. ಆಗ ಮನಸ್ಸಿನ ಚಂಚಲತೆ ಮತ್ತು ಶಾರೀರಿಕ ಬದಲಾವಣೆಗಳಿಗೆ ಒಳಗಾಗದೆ ಅವನ ಆತ್ಮ ಪೂರ್ಣ ಸ್ವಯಂಪ್ರಕಾಶಮಾನವಾಗಿ ಬೆಳಗುವುದು. ಜ್ಞಾನಮಯವೂ, ಅಮೃತಮಯವೂ, ಸರ್ವವ್ಯಾಪಿಯೂ ಆದ ತನ್ನ ಈಗಿನ ಮತ್ತು ಎಂದೆಂದಿಗೂ ತನ್ನದೇ ಆದ ಸ್ವಭಾವ ವೇದ್ಯವಾಗುವುದು. 

\vskip 5pt

ಸಮಾಧಿ ಪ್ರತಿಯೊಬ್ಬ ಮಾನವನು ಮಾತ್ರವಲ್ಲ, ಪ್ರತಿಯೊಂದು ಪ್ರಾಣಿಯ ಆಜನ್ಮ ಸಿದ್ಧ ಹಕ್ಕು. ಕ್ಷುದ್ರ ಮೃಗದಿಂದ ಹಿಡಿದು ಪರಮದೇವನವರೆಗೆ ಪ್ರತಿಯೊಬ್ಬರೂ ಕೂಡ ಈ ಸ್ಥಿತಿಗೆ ಬರಲೇಬೇಕು. ಆಗ ಮಾತ್ರ ನಿಜವಾದ ಧರ್ಮ ಅವನಿಗೆ ಮೊದಲಾಗುವುದು. ಅಲ್ಲಿಯವರೆವಿಗೂ ನಾವು ಆ ಸ್ಥಿತಿಗಾಗಿ ಹೋರಾಡುವೆವು ಅಷ್ಟೆ. ಈಗ ನಮಗೂ ಮತ್ತು ನಾಸ್ತಿಕರಿಗೂ ಯಾವ ವ್ಯತ್ಯಾಸವೂ ಇಲ್ಲ. ಏಕೆಂದರೆ ನಮಗೆ ಯಾವ ಅನುಭವವೂ ಸಿಕ್ಕಿಲ್ಲ. ಮನೋ ಏಕಾಗ್ರತೆಯು ಈ ಮಹಾ ಅನುಭವದೆಡೆಗೆ ನಮ್ಮನ್ನು ಕರೆದೊಯ್ಯುವುದಕ್ಕಲ್ಲದೆ ಮತ್ತೇಕೆ ಇರುವುದು? ಸಮಾಧಿಯನ್ನು ಹೊಂದುವುದಕ್ಕಾಗಿರುವ ಪ್ರತಿಯೊಂದು ಮೆಟ್ಟಿಲನ್ನೂ ಕೂಡ ವೈಚಾರಿಕವಾಗಿ, ವೈಜ್ಞಾನಿಕವಾಗಿ ಅಧ್ಯಯನಮಾಡಿ ಸರಿಯಾಗಿ ವ್ಯವಸ್ಥೆಗೊಳಿಸಿರುವರು. ಶ್ರದ್ಧಾ ಪೂರ್ವಕವಾಗಿ ನಾವು ಅದನ್ನೇ ಅಭ್ಯಾಸ ಮಾಡಿದರೆ ಇಚ್ಛಿಸುವ ಗುರಿಯೆಡೆಗೆ ನಮ್ಮನ್ನು ನಿಜವಾಗಿಯೂ ಕರೆದೊಯ್ಯುವುದು. ಆಗ ದುಃಖ ಶಮನವಾಗುವುದು; ಕಷ್ಟ ಮಾಯವಾಗುವುದು; ಕರ್ಮ ಬೀಜ ಸೀಯುವುದು. ಜೀವನು ಎಂದೆಂದಿಗೂ ಮುಕ್ತನಾಗುವನು.

\chapter{ಸಂಕ್ಷೇಪ ರಾಜಯೋಗ}%%%೩೧

\centerline{\textbf{ಇದು ಕೂರ್ಮಪುರಾಣದಲ್ಲಿ ಬರುವ ರಾಜಯೋಗದ ಸಂಕ್ಷೇಪ ಸರಳಾನುವಾದ}}

ಯೋಗಾಗ್ನಿಯು ಮಾನವನನ್ನು ಬಂಧಿಸಿರುವ ಪಂಜರವನ್ನು ದಹಿಸುವುದು, ಜ್ಞಾನ ಪರಿಶುದ್ಧವಾಗುವುದು, ನಿರ್ವಾಣ ಪ್ರತ್ಯಕ್ಷ ಲಭಿಸುವುದು. ಯೋಗದಿಂದ ಜ್ಞಾನ ಲಭಿಸುವುದು, ಜ್ಞಾನವು ಯೋಗಿಗೆ ಸಹಾಯ ಮಾಡುವುದು. ಯಾರು ತಮ್ಮ ಜೀವನದಲ್ಲಿ ಯೋಗ ಮತ್ತು ಜ್ಞಾನವನ್ನು ಬೆರಸುತ್ತಾರೆಯೊ, ಅವರು ಈಶ್ವರನ ಪ್ರೀತಿಗೆ ಪಾತ್ರರಾಗುವರು. ಯಾರು ಮಹಾಯೋಗವನ್ನು ದಿನಕ್ಕೆ ಒಂದು ವೇಳೆ, ಎರಡು ವೇಳೆ, ಮೂರು ವೇಳೆ, ಅಭ್ಯಾಸ ಮಾಡುತ್ತಾರೆಯೋ ಅವರನ್ನು ದೇವತೆಗಳೆಂದು ತಿಳಿಯಿರಿ. ಯೋಗವನ್ನು ಎರಡು ಭಾಗವಾಗಿ ಮಾಡಿರುವರು. ಅದರಲ್ಲಿ ಒಂದಕ್ಕೆ ಅಭಾವವೆಂದೂ ಮತ್ತೊಂದಕ್ಕೆ ಮಹಾಯೋಗವೆಂದೂ ಹೆಸರು. ಗುಣಾತೀತವಾದ ಶೂನ್ಯವೆಂದು ಯಾರು ತಮ್ಮಾತ್ಮನನ್ನು ಧ್ಯಾನಿಸುವರೊ ಅದಕ್ಕೆ ಅಭಾವವೆಂದು ಹೆಸರು. ಯಾರು ತಮ್ಮಾತ್ಮವು ಆನಂದಮಯವೆಂದು, ಅಕಳಂಕವೆಂದು, ಬ್ರಹ್ಮೈಕ್ಯವೆಂದು ತಿಳಿಯುತ್ತಾರೋ ಅದಕ್ಕೆ ಮಹಾಯೋಗವೆಂದು ಹೆಸರು. ಯೋಗಿಗೆ ಎರಡು ಮಾರ್ಗಗಳಿಂದಲೂ ಆತ್ಮ ಸಾಕ್ಷಾತ್ಕಾರ ಲಭಿಸುವುದು. ನಾವು ಓದುವ ಮತ್ತು ಕೇಳುವ ಇತರ ಯೋಗಗಳನ್ನು ತನ್ನನ್ನೂ ಜಗತ್ತನ್ನೂ ಈಶ್ವರನೆಂದು ಕಾಣುವ ಅತ್ಯುತ್ತಮವಾದ ಮಹಾಯೋಗದೊಂದಿಗೆ ಹೋಲಿಸಲು ಯೋಗ್ಯವಲ್ಲ. ಯೋಗಗಳಲ್ಲೆಲ್ಲಾ ಇದೇ ಶ್ರೇಷ್ಠತಮವಾದುದು. 

\vskip 5pt

ಯಮ, ನಿಯಮ, ಆಸನ, ಪ್ರಾಣಾಯಾಮ, ಪ್ರತ್ಯಾಹಾರ, ಧಾರಣ, ಧ್ಯಾನ, ಸಮಾಧಿ ಎಂಬುವು ರಾಜಯೋಗದ ಮೆಟ್ಟಲು. ಯಮವೆಂದರೆ, ಅಹಿಂಸೆ, ಸತ್ಯ, ಅಸ್ತೇಯ, ಬ್ರಹ್ಮಚರ್ಯ, ಅಪರಿಗ್ರಹ ಇವು. ಇದರಿಂದ ಚಿತ್ತಶುದ್ಧಿಯಾಗುವುದು. ಯಾವ ಜೀವಜಂತುವಿಗೂ ಕೂಡ ಮನೋವಾಕ್ಕಾಯವಾಗಿ ನೋವನ್ನುಂಟುಮಾಡದೆ ಇರುವುದಕ್ಕೆ ಅಹಿಂಸೆ ಎಂದು ಹೆಸರು. ಅಹಿಂಸೆಗಿಂತ ಪರಮ ಧರ್ಮ ಮತ್ತಾವುದೂ ಇಲ್ಲ. ಮಾನವನಿಗೆ ಅಹಿಂಸಾ ದೃಷ್ಟಿಯಿಂದ ಎಲ್ಲಾ ಜೀವಜಂತುಗಳನ್ನು ನೋಡುವುದರಿಂದ ಬರುವ ಆನಂದಕ್ಕಿಂತ ಮಿಗಿಲಾದ ಆನಂದವು ಬೇರಾವುದೂ ಇಲ್ಲ. ಸತ್ಯದಿಂದ ನಮಗೆ ಕರ್ಮಫಲ ದೊರಕುವುದು. ಸತ್ಯದಿಂದ ನಮಗೆ ಎಲ್ಲವೂ ದೊರಕುವುದು, ಸತ್ಯದಲ್ಲೆ ಎಲ್ಲವೂ ನಿಂತಿರುವುದು. ಇದ್ದದ್ದನ್ನು ಇದ್ದಂತೆ ಹೇಳುವುದು ಸತ್ಯ. ಮತ್ತೊಬ್ಬರ ವಸ್ತುವನ್ನು ಕಳ್ಳತನದಿಂದಲಾದರೂ ಆಗಲಿ, ಬಲಾತ್ಕಾರದಿಂದಲಾದರೂ ಆಗಲಿ ತೆಗೆದುಕೊಳ್ಳದೆ ಇರುವುದೇ ಅಸ್ತೇಯಾ. ಎಲ್ಲಾ ಕಾಲದಲ್ಲಿಯೂ, ಎಲ್ಲಾ ಸ್ಥಿತಿಯಲ್ಲಿಯೂ ಮನೋವಾಕ್ಕಾಯವಾಗಿ ನಿರ್ಮಲವಾಗಿರುವುದೇ ಬ್ರಹ್ಮಚರ್ಯ. ಎಂತಹ ದುಃಸ್ಥಿತಿಯಲ್ಲಿದ್ದರೂ ಕೂಡ ಯಾರಿಂದಲೂ ಏನನ್ನೂ ಸ್ವೀಕರಿಸದೆ ಇರುವುದೇ ಅಪರಿಗ್ರಹ. ಇದರ ಅರ್ಥವೇನೆಂದರೆ ಒಬ್ಬ ಮತ್ತೊಬ್ಬರಿಂದ ದಾನವನ್ನು\break ಪಡೆದರೆ ಅವನ ಹೃದಯ ಮಲಿನವಾಗುವುದು, ಅವನು ನೀಚನಾಗುವನು, ತನ್ನ ಸ್ವಾತಂತ್ರ್ಯವನ್ನು ಕಳೆದುಕೊಂಡು ಆಸಕ್ತನಾಗಿ ಬಂಧನದಲ್ಲಿ ಬೀಳುವನು. 


ಯೋಗದಲ್ಲಿ ಜಯಶೀಲನಾಗಬೇಕಾದರೆ ಈ ಕೆಳಗಿನ ನಿಯಮಗಳು ಸಹಕಾರಿ: ತಪಸ್ಸು, ಸ್ವಾಧ್ಯಾಯ, ಸಂತೋಷ, ಶೌಚ ಮತ್ತು ಈಶ್ವರಪ್ರಣಿಧಾನ. ಉಪವಾಸ ಅಥವಾ ಬೇರೆ ವಿಧಾನದ ಮೂಲಕ ನಮ್ಮ ಶರೀರವನ್ನು ನಿಗ್ರಹಿಸುವುದಕ್ಕೆ ಶಾರೀರಿಕ ತಪಸ್ಸು ಎಂದು ಹೆಸರು. ದೇಹದಲ್ಲಿ ಸತ್ತ್ವಶುದ್ಧಿಯಾಗುವಂತಹ ವೇದ ಮತ್ತು ಇನ್ನಿತರ ಪವಿತ್ರ ಮಂತ್ರಗಳನ್ನು ಉಚ್ಚರಿಸುವುದಕ್ಕೆ ಸ್ವಾಧ್ಯಾಯವೆಂದು ಹೆಸರು. ಇದನ್ನು ಉಚ್ಚರಿಸುವುದಕ್ಕೆ ಮೂರು ವಿಧಗಳಿವೆ. ಮೊದಲನೆಯದು ಶಬ್ದದ ಮೂಲಕ; ಎರಡನೆಯದು ಕೇವಲ ಬಾಯಿ ಮಾತ್ರ ಚಲಿಸುವುದು, ಆದರೆ ಶಬ್ದ ಕೇಳಿಸುವುದಿಲ್ಲ; ಮೂರನೆಯದು, ಬಾಯಿ ಚಲಿಸದೆ ಅದರ ಅರ್ಥದಿಂದ ಕೂಡಿರುವುದಕ್ಕೆ ಮಾನಸಿಕ ಜಪವೆಂದು ಹೆಸರು. ಇದು ಅತ್ಯುತ್ತಮವಾದುದು. ಬಾಹ್ಯ ಮತ್ತು ಅಂತರಂಗದ ಶುದ್ಧಿಯೆಂಬ ಎರಡು ವಿಧದ ಶೌಚವನ್ನು ಯೋಗಿ ಹೇಳುವನು. ನೀರು, ಮಣ್ಣು, ಮತ್ತು ಇತರ ವಸ್ತುಗಳಿಂದ ದೇಹವನ್ನು ಶುದ್ಧಿಮಾಡುವುದು ಬಾಹ್ಯ ಶುದ್ಧಿ. ಸತ್ಯ ಮತ್ತು ಇನ್ನಿತರ ಒಳ್ಳೆಯ ಗುಣಗಳಿಂದ ಮನಸ್ಸನ್ನು ಶುದ್ಧಿ ಮಾಡುವುದು ಮಾನಸಿಕ ಪರಿಶುದ್ಧತೆ. ಎರಡೂ ಆವಶ್ಯಕ. ಅಂತರಂಗದಲ್ಲಿ ಮಾತ್ರ ಶುದ್ಧವಾಗಿ ಹೊರಗಡೆ ಕಶ್ಮಲದಿಂದ ಕೂಡಿದ್ದರೆ ಸಾಲದು. ಎರಡೂ ಸಿಕ್ಕದ ಪಕ್ಷಕ್ಕೆ ಅಂತರಂಗ ಶುದ್ಧಿಯೇ ಒಳ್ಳೆಯದು. ಆದರೆ ಬಾಹ್ಯ ಮತ್ತು ಅಂತರಂಗ ಶುದ್ಧಿ ಇಲ್ಲದೆ ಯಾರೂ ಯೋಗಿಗಳಾಗರು. ಪ್ರಾರ್ಥನೆ, ಧ್ಯಾನ, ಭಕ್ತಿ ಇವುಗಳ ಮೂಲಕ ದೇವರನ್ನು ಪೂಜಿಸಬೇಕು. 

ಯಮ ನಿಯಮದ ವಿಚಾರವನ್ನು ನಾವಾಗಲೇ ಹೇಳಿರುವೆವು. ಮುಂದಿನದೇ ಆಸನ. ಇದರ ವಿಚಾರವಾಗಿ ನಾವು ತಿಳಿದುಕೊಳ್ಳಬೇಕಾದುದೆಂದರೆ ದೇಹಕ್ಕೆ ಯಾವ ನಿರ್ಬಂಧವೂ ಇಲ್ಲದೆ ಇರುವುದು; ಎದೆ, ಭುಜ ಮತ್ತು ತಲೆಯನ್ನು ನೇರವಾಗಿ ನಿಲ್ಲಿಸಬೇಕು. ಅನಂತರ ಪ್ರಾಣಾಯಾಮ. ಪ್ರಾಣವೆಂದರೆ ಒಬ್ಬನ ದೇಹದಲ್ಲಿರುವ ಮುಖ್ಯವಾದ ಶಕ್ತಿ, ಆಯಾಮವೆಂದರೆ ನಿಗ್ರಹಿಸುವುದು ಎಂದು ಅರ್ಥ. ಪ್ರಾಣಾಯಾಮದಲ್ಲಿ ಮೂರು ವಿಧಗಳಿವೆ: ಬಹಳ ಸುಲಭವಾದುದು, ಮಧ್ಯಮ ತರಗತಿಯದು ಮತ್ತು ಉತ್ತಮ ತರಗತಿಯದು. ಪ್ರಾಣಾಯಾಮವನ್ನು ಉಸಿರನ್ನು ಸೆಳೆದುಕೊಳ್ಳುವುದು, ಅದನ್ನು ತಡೆಯುವುದು ಮತ್ತು ಬಿಡುವುದು, ಎಂದು ಮೂರು ವಿಭಾಗ ಮಾಡಬಹುದು. ಹನ್ನೆರಡು ಕ್ಷಣದಿಂದ ಮೊದಲು ಮಾಡುವುದು ಕೊನೆಯದರ್ಜೆಯದು: ಇಪ್ಪತ್ತುನಾಲ್ಕು ಕ್ಷಣದಿಂದ ಮೊದಲಾಗುವುದು ಮಧ್ಯಮವರ್ಗದ್ದು; ಮೂವತ್ತಾರು ಕ್ಷಣದಿಂದ ಮೊದಲು ಮಾಡುವುದು ಉತ್ತಮ ವರ್ಗದ್ದು. ಕೆಳದರ್ಜೆಯ ಪ್ರಾಣಯಾಮದಲ್ಲಿ ಬೆವರು ಸುರಿಯುವುದು; ಮಧ್ಯಮವರ್ಗದ ಪ್ರಾಣಯಾಮದಲ್ಲಿ ದೇಹ ಕಂಪಿಸುವುದು; ಯಾವಾಗ ದೇಹ ಹಗುರವಾಗಿ ಅನಂತರ ಪರಮಾನಂದದ ಪ್ರವಾಹ ಹರಿಯುವುದೋ ಅದೇ ಅತ್ಯುತ್ತಮವಾದ ಪ್ರಾಣಾಯಾಮ. ಗಾಯತ್ರಿ ಎಂಬ ಒಂದು ಮಂತ್ರವಿದೆ. ಇದು ವೇದದಲ್ಲಿರುವ ಒಂದು ಪವಿತ್ರ ಮಂತ್ರ: “ಈ ವಿಶ್ವವನ್ನು ಸೃಷ್ಟಿಸಿದಾತನ ಮಹಿಮೆಯನ್ನು ಕುರಿತು ನಾವು ಧ್ಯಾನಿಸುತ್ತೇವೆ. ಆತನು ನಮ್ಮ ಬುದ್ಧಿಯನ್ನು ಪ್ರಚೋದಿಸಲಿ.” ಮೊದಲು ಮತ್ತು ಕೊನೆಯಲ್ಲಿ ಓಂ ಎಂಬ ಪದವು ಇದರ ಜೊತೆಯಲ್ಲಿ ಸೇರಿರುತ್ತದೆ. ಒಂದು ಪ್ರಾಣಾಯಾಮದಲ್ಲಿ ಮೂರು ಗಾಯತ್ರಿಯನ್ನು\break\ ಉಚ್ಚರಿಸಿ. ಎಲ್ಲಾ ಗ್ರಂಥಗಳಲ್ಲಿಯೂ ಪ್ರಾಣಾಯಾಮವನ್ನು ರೇಚಕ (ಉಸಿರನ್ನು ಬಿಡುವುದು), ಪೂರಕ (ಉಸಿರನ್ನು ಸೆಳೆದುಕೊಳ್ಳುವುದು), ಕುಂಭಕ (ಉಸಿರನ್ನು ನಿಲ್ಲಿಸುವುದು) ಎಂದು ಮೂರು ವಿಧವಾಗಿ ವಿಭಾಗಿಸಿರುವರು. ಇಂದ್ರಿಯಗಳು ಹೊರಹೋಗಿ ಬಾಹ್ಯವಸ್ತುವಿನ ಸಂಪರ್ಕವನ್ನು ಪಡೆಯುತ್ತವೆ. ಅದನ್ನು ನಮ್ಮ ಇಚ್ಛಾಶಕ್ತಿಯ ಸ್ವಾಧೀನಕ್ಕೆ ತರುವುದೇ ಪ್ರತ್ಯಾಹಾರ, ಅಂದರೆ ತನ್ನೆಡೆಗೆ ಸೆಳೆದುಕೊಳ್ಳುವುದು. 

ಹೃದಯ ಕಮಲದಲ್ಲಿ ಅಥವಾ ಶಿರಸ್ಸಿನ ಕೇಂದ್ರದಲ್ಲಿ ಮನಸ್ಸನ್ನು ನೆಲೆಗೊಳಿಸುವುದಕ್ಕೆ ಧಾರಣವೆಂದು ಹೆಸರು. ಇದನ್ನು ಒಂದು ಸ್ಥಳದಲ್ಲಿ ನಿಲ್ಲಿಸಿ, ಅದನ್ನೇ ತಳಹದಿ ಮಾಡಿಕೊಂಡರೆ ಕೆಲವು ವಿಧದ ಮಾನಸಿಕ ತರಂಗಗಳು ಏಳುವುವು. ಬೇರೆ ಬಗೆಯ ಮಾನಸಿಕ ತರಂಗಗಳು ಇವನ್ನು ನಾಶಮಾಡಲಾರವು. ಕ್ರಮೇಣ ಇವು ಪ್ರಾಮುಖ್ಯವಾಗಿ, ಉಳಿದುವು ದೂರವಾಗಿ ಕ್ರಮೇಣ ಮಾಯವಾಗುವುವು. ಅನಂತರ ಉಳಿದ ಈ ಬಹುವಿಧದ ತರಂಗಗಳು ಐಕ್ಯವಾಗಿ ಕೊನೆಗೆ ಒಂದೇ ತರಂಗ ಮನಸ್ಸಿನಲ್ಲಿ ಉಳಿಯುವುದು. ಇದಕ್ಕೆ ಧ್ಯಾನವೆಂದು ಹೆಸರು. ಇದಕ್ಕೆ ಯಾವ ತಳಹದಿಯೂ ಇಲ್ಲದೆ ಮನಸ್ಸೆಲ್ಲ ಐಕ್ಯವಾಗಿ ಒಂದೇ ಅಲೆಯಾದಾಗ ಅದಕ್ಕೆ ಸಮಾಧಿ ಎಂದು ಹೆಸರು. ಆಗ ಸ್ಥಳ ಮತ್ತು ಕೇಂದ್ರಗಳ ಯಾವ ಸಹಾಯವೂ ಇಲ್ಲದೆ ಚಿಂತನೆಯ ಅರ್ಥ ಮಾತ್ರ ಉಳಿಯುವುದು. ಮನಸ್ಸನ್ನು ಒಂದು ಕೇಂದ್ರದ ಮೇಲೆ ಹನ್ನೆರಡು ಕ್ಷಣ ನಿಲ್ಲಿಸಿದರೆ ಅದು ಧಾರಣ. ಅಂತಹ ಹನ್ನೆರಡು ಧಾರಣ ಒಂದು ಧ್ಯಾನವಾಗುತ್ತದೆ. ಇಂತಹ ಹನ್ನೆರಡು ಧ್ಯಾನ ಒಂದು ಸಮಾಧಿಯಾಗುವುದು. 

ಬೆಂಕಿ ಇರುವ ಕಡೆ, ನೀರಿನಲ್ಲಿ, ತರಗೆಲೆಗಳು ಬಿದ್ದಿರುವ ನೆಲದ ಮೇಲೆ ಹುತ್ತವಿರುವ ಕಡೆ, ಕಾಡು ಮೃಗಗಳಿರುವೆಡೆ, ಅಪಾಯವಿರುವಲ್ಲಿ, ನಾಲ್ಕು ರಸ್ತೆಗಳು ಸಂಧಿಸುವಲ್ಲಿ, ಹೆಚ್ಚಾಗಿ ಗದ್ದಲವಿರುವ ಕಡೆ ಮತ್ತು ದುಷ್ಟ ಜನರು ಇರುವ ಕಡೆ ಯೋಗಾಭ್ಯಾಸ ಮಾಡಕೂಡದು. ಇದು ವಿಶೇಷವಾಗಿ ಇಂಡಿಯಾದೇಶಕ್ಕೆ ಹೆಚ್ಚು ಅನ್ವಯಿಸುವುದು. ದೇಹ ಸೋಮಾರಿಯಾದಾಗ ಅಥವಾ ಖಾಯಿಲೆಯಾದಾಗ, ಅಥವಾ ಮನಸ್ಸು ಬಹಳ ಸಂಕಟದಲ್ಲಿ ತೊಳಲುತ್ತಿರುವಾಗ ಇದನ್ನು ಅಭ್ಯಾಸ ಮಾಡಬೇಡಿ. ಜನರು ಬಂದು ತೊಂದರೆ ಕೊಡದೆ ಇರುವ ಏಕಾಂತ ಸ್ಥಳಕ್ಕೆ ಹೋಗಿ ಕಶ್ಮಲವಾದ ಸ್ಥಳವನ್ನು ಆರಿಸಿಕೊಳ್ಳಬೇಡಿ. ಸುಂದರವಾದ ಪ್ರಕೃತಿಯ ದೃಶ್ಯವನ್ನೊ, ಅಥವಾ ನಿಮ್ಮ ಮನೆಯಲ್ಲಿ ಅನುಕೂಲವಾಗಿರುವ ಒಂದು ಕೋಣೆಯನ್ನೊ ಆರಿಸಿ. ನೀವು ಅಭ್ಯಾಸ ಮಾಡುವುದಕ್ಕೆ ಮುಂಚೆ ಹಿಂದಿನ ಎಲ್ಲಾ ಯೋಗಿಗಳಿಗೆ, ನಿಮ್ಮ ಗುರುಗಳಿಗೆ, ಮತ್ತು ದೇವರಿಗೆ ನಮಸ್ಕಾರ ಮಾಡಿ, ಅನಂತರ ನಿಮ್ಮ ಸಾಧನೆಯನ್ನು ಪ್ರಾರಂಭಿಸಿ. 

ಧ್ಯಾನವೆಂದರೆ ಏನೆಂಬುದನ್ನು ಹೇಳಿದ್ದಾಯಿತು. ಯಾವುದರ ಮೇಲೆ ಧ್ಯಾನ ಮಾಡಬೇಕೆಂಬುದಕ್ಕೆ ಕೆಲವು ಉದಾಹರಣೆಗಳನ್ನು ಕೊಟ್ಟಿರುವರು. ನೇರವಾಗಿ ಕುಳಿತುಕೊಂಡು ನಿಮ್ಮ ಮೂಗಿನ ತುದಿಯನ್ನು ನೋಡಿ. ಇದು ಮನಸ್ಸನ್ನು ಹೇಗೆ ಏಕಾಗ್ರಗೊಳಿಸುತ್ತದೆ ಎಂಬುದನ್ನು ಕ್ರಮೇಣ ನೋಡುವೆವು. ನಯನೇಂದ್ರಿಯದ ಎರಡು ನರಗಳನ್ನು ನಿಗ್ರಹಿಸುವುದರಿಂದ, ಕರ್ಮೇಂದ್ರಿಯದ ಸ್ವಾಧೀನದಲ್ಲಿ ಮುಂದುವರಿದು ಸಾಧಕನು ತನ್ನ ಇಚ್ಛಾಶಕ್ತಿಯನ್ನು ನಿಗ್ರಹಿಸುತ್ತಾನೆ. ಧ್ಯಾನದ ಕೆಲವು ಉದಾಹರಣೆಗಳು ಇಲ್ಲಿವೆ. ತಲೆಯ ಮೇಲೆ ಕೆಲವು ಅಂಗುಲ ಎತ್ತರದಲ್ಲಿ ಒಂದು ಕಮಲವನ್ನು ಭಾವಿಸಿ. ಒಳ್ಳೆಯ ಶೀಲವೆ ಅದರ ಕೇಂದ್ರವೆಂತಲೂ, ಕಮಲದ ದಂಟನ್ನೇ ಜ್ಞಾನವೆಂತಲೂ ಭಾವಿಸಿಕೊಳ್ಳಿ. ಕಮಲದ ಅಷ್ಟದಳಗಳು ಯೋಗಿಯ ಅಷ್ಟಸಿದ್ಧಿಗಳು. ಒಳಗಿರುವ ಕೇಸರಗಳೇ ತ್ಯಾಗ. ಯೋಗಿಯು ಬಾಹ್ಯ ಸಿದ್ಧಿಯನ್ನು ನಿರಾಕರಿಸಿದರೆ ಸಾಯುಜ್ಯ ಪದವಿಗೆ ಬರುತ್ತಾನೆ. ಆದಕಾರಣ, ಕಮಲದ ಅಷ್ಟದಳ ಅಷ್ಟಸಿದ್ಧಿ, ಅದರ ಒಳಗಿರುವ ಕೇಸರಗಳು ತೀವ್ರತ್ಯಾಗ, ಇದೇ ಅಷ್ಟಸಿದ್ಧಿ ನಿರಾಕರಣೆ. ಅದರ ಒಳಗೆ ಸ್ವಯಂ ಜ್ಯೋತಿಯಿಂದ ಪ್ರಕಾಶಮಾನವಾಗಿರುವ ಸರ್ವಶಕ್ತವೂ, ಅವ್ಯಕ್ತವೂ, ವಾಗತೀತವೂ ಆದ ಓಂಕಾರ ಸ್ವರೂಪವನ್ನು ಆಲೋಚಿಸಿ, ಅದರ ಮೇಲೆ ಧ್ಯಾನ ಮಾಡಿ. ಇನ್ನೊಂದು ಬಗೆಯ ಧ್ಯಾನವನ್ನು ಕೊಟ್ಟಿದೆ. ಹೃದಯದಲ್ಲಿ ಆಕಾಶವನ್ನು ಕಲ್ಪಿಸಿ, ಅದರ ಮಧ್ಯದಲ್ಲಿ ಒಂದು ಜ್ಯೋತಿ ಉರಿಯುತ್ತಿದೆ ಎಂದು ಭಾವಿಸಿ. ಆ ಜ್ಯೋತಿಯನ್ನೇ ನಿಮ್ಮ ಆತ್ಮವೆಂದೂ, ಈ ಜ್ಯೋತಿಯೊಳಗೆ ಮತ್ತೊಂದು ಸ್ವಯಂ ಪ್ರಕಾಶಮಾನವಾದ ಜ್ಯೋತಿ ಇದೆ, ಅದೇ ನಿಮ್ಮ ಆತ್ಮದ ಆತ್ಮನಾದ ದೇವರೆಂದೂ ಭಾವಿಸಿ. ಹೃದಯದಲ್ಲಿ ಅದರ ಮೇಲೆ ಧ್ಯಾನಿಸಿ. ಬ್ರಹ್ಮಚರ್ಯ, ಅಹಿಂಸೆ, ಪರಮ ಶತ್ರುವನ್ನು ಕೂಡ ಕ್ಷಮಿಸುವುದು, ಸತ್ಯ, ದೇವರಲ್ಲಿ ಭಕ್ತಿ, ಇವುಗಳೆಲ್ಲ ಬೇರೆ ಬೇರೆ ವೃತ್ತಿಗಳು. ಈ ಶೀಲವೆಲ್ಲಾ ನಿಮ್ಮಲ್ಲಿ ಇನ್ನೂ ಪೂರ್ಣವಾಗಿಲ್ಲದೆ ಇದ್ದರೆ ಅಂಜಬೇಕಾಗಿಲ್ಲ. ಸಾಧನೆ ಮಾಡಿ, ಕ್ರಮೇಣ ಇವು ಬರುತ್ತವೆ. ಯಾರು ತಮ್ಮ ಆಸಕ್ತಿ, ಅಂಜಿಕೆ, ಕೋಪವನ್ನು ಬಿಡುವರೊ, ಯಾರು ದೇವರಿಗೆ ಶರಣಾಗಿರುವರೋ, ಯಾರ ಹೃದಯವು ಪರಿಶುದ್ಧವಾಗಿರುವುದೋ ಅವರು ಯಾವ ಬಯಕೆಯಿಂದ ಈಶ್ವರನ ಹತ್ತಿರ ಬಂದರೂ ಅವನು ಅವೆಲ್ಲವನ್ನೂ ಕರುಣಿಸುವನು. ಆದಕಾರಣ ಭಗವಂತನನ್ನು ಜ್ಞಾನ, ಭಕ್ತಿ ಅಥವಾ ತ್ಯಾಗ ಭಾವದಿಂದ ಪೂಜಿಸಿ. 

“ಯಾರು ಮತ್ತಾರನ್ನೂ ದ್ವೇಷಿಸುವುದಿಲ್ಲವೊ, ಯಾರು ಎಲ್ಲರ ಸ್ನೇಹಿತರೊ ಯಾರು ಎಲ್ಲರಿಗೂ ಅನುಕಂಪ ತೋರುವರೊ, ಯಾರಿಗೆ ಸ್ವಾರ್ಥವಿಲ್ಲವೊ, ಯಾರು ಅಹಂಕಾರದಿಂದ ಮುಕ್ತರಾಗಿರುವರೊ, ಯಾರು ಸುಖದುಃಖದಲ್ಲಿ ಸಮಬುದ್ಧಿಯುಳ್ಳವರೊ, ಯಾರು ಎಲ್ಲವನ್ನೂ ಸಹಿಸುವರೊ, ಯಾರು ನಿತ್ಯ ತೃಪ್ತರೊ, ನಿತ್ಯಯೋಗದಲ್ಲಿ ಕೂಡಿರುವರೊ, ಯಾರು ಜಿತಾತ್ಮರೊ, ಯಾರ ಮನಸ್ಸು ದೃಢವಾಗಿದೆಯೊ, ಯಾರ ಮನಸ್ಸು ಮತ್ತು ಬುದ್ಧಿ ನನ್ನಲ್ಲಿ ಲೀನವಾಗಿದೆಯೊ ಅವರೆ ನನ್ನ ಪ್ರಿಯ ಭಕ್ತರೆಂದು ತಿಳಿ. ಯಾರಿಂದ ಉಳಿದವರಿಗೆ ತೊಂದರೆಯಾಗುವುದಿಲ್ಲವೊ, ಯಾರು ಉಳಿದವರಿಂದ ತೊಂದರೆಗೆ ಒಳಗಾಗುವುದಿಲ್ಲವೊ, ಯಾರು ಸಂತೋಷ ದುಃಖ ಅಂಜಿಕೆ ಆತಂಕಗಳಿಂದ ದೂರವಾಗಿರುವರೊ, ಅಂತಹವರು ನನ್ನ ಭಕ್ತರು. ಯಾರು ಮತ್ತಾವುದನ್ನೂ ನೆಚ್ಚಿಕೊಂಡಿಲ್ಲವೊ, ಶುದ್ಧರಾಗಿ ಎಲ್ಲವನ್ನು ತೊರೆದು ಕೆಲಸದಲ್ಲಿ ನಿರತರಾಗಿರುವರೊ, ಯಾರು ಒಳ್ಳೆಯದಾಗಲಿ ಕೆಟ್ಟದಾಗಲಿ ಲೆಕ್ಕಿಸುವುದಿಲ್ಲವೊ, ಯಾರು ಎಂದಿಗೂ ವ್ಯಸನ ಪಡುವುದಿಲ್ಲವೊ, ಯಾರು ತನಗಾಗಿ ಎಲ್ಲ ಪ್ರಯತ್ನಗಳನ್ನು ತ್ಯಜಿಸಿರುವರೊ, ಯಾರು ಸ್ತುತಿನಿಂದೆಗಳನ್ನು ಸಮದೃಷ್ಟಿಯಿಂದ ನೋಡುವರೊ, ಮೌನವಾಗಿ, ವಿಚಾರಪರರಾಗಿ, ತಮಗೆ ಬಂದ ಅಲ್ಪದರಲ್ಲಿ ತೃಪ್ತರಾಗಿ, ಮನೆ ಇಲ್ಲದೆ ಜಗತ್ತನ್ನೇ ತಮ್ಮ ಮನೆಯನ್ನಾಗಿ ಮಾಡಿಕೊಂಡು ತಮ್ಮ ಯತ್ನದಲ್ಲಿ ದೃಢವಾಗಿರುವರೊ ಅವರೇ ನನ್ನ ಪ್ರಿಯ ಭಕ್ತರು.” ಅಂತಹವರು ಯೋಗಿಗಳಾಗುವರು. 

\vskip 0.3cm 

ಹಿಂದೆ ನಾರದರು ಎಂಬ ದೇವಋಷಿಗಳಿದ್ದರು. ಮನುಷ್ಯರಲ್ಲಿ ಹೇಗೆ ದೊಡ್ಡ ಯೋಗಿಗಳಿರುವರೊ ಅದರಂತೆ ದೇವತೆಗಳಲ್ಲಿಯೂ ಮಹಾಯೋಗಿಗಳಿರುವರು. ನಾರದರು ಒಬ್ಬ ಮಹಾಯೋಗಿಗಳಾಗಿದ್ದರು. ಅವರು ಎಲ್ಲಾ ಕಡೆಯಲ್ಲಿಯೂ ಸಂಚರಿಸುತ್ತಿದ್ದರು. ಒಂದು ದಿನ ಅವರು ಅರಣ್ಯದ ಮೂಲಕವಾಗಿ ಹೋಗುತ್ತಿದ್ದರು. ಅಲ್ಲಿ ಒಬ್ಬ ಋಷಿ ತಪಸ್ಸು ಮಾಡುತ್ತಿದ್ದ. ಅವನ ಸುತ್ತ ಹುತ್ತ ಬೆಳೆಯುವಷ್ಟು ದೀರ್ಘ ಕಾಲದಿಂದ ತಪಸ್ಸು ಮಾಡುತ್ತಿದ್ದ. ಅವನು ನಾರದರಿಗೆ “ನೀವು ಎಲ್ಲಿಗೆ ಹೋಗುತ್ತಿರುವಿರಿ?” ಎಂದು ಕೇಳಿದನು. ನಾರದರು “ನಾನು ವೈಕುಂಠದ ಕಡೆಗೆ ಹೋಗುತ್ತಿರುವೆನು” ಎಂದರು. “ಹಾಗಾದರೆ, ಎಂದು ದೇವರು ನನ್ನ ಮೇಲೆ ಕರುಣಿಸುತ್ತಾನೆ, ಎಂದು ನನಗೆ ಮುಕ್ತಿ ಸಿಕ್ಕುತ್ತದೆ ಎಂಬುದನ್ನು ಕೇಳಿ” ಎಂದನು. ಸ್ವಲ್ಪ ದೂರ ಹೋದಮೇಲೆ ನಾರದರು ಮತ್ತೊಬ್ಬನನ್ನು ನೋಡಿದರು. ಆತನು ಕುಣಿಯುತ್ತ, ಹಾಡುತ್ತ, ನಗೆದಾಡುತ್ತಿದ್ದನು. “ನಾರದರೆ, ನೀವು ಎಲ್ಲಿಗೆ ಹೋಗುತ್ತೀರಿ?” ಎಂದನು. ಅವನ ಧ್ವನಿ ಮತ್ತು ನಡತೆ ಬಹಳ ಒರಟಾಗಿತ್ತು. ನಾರದರು “ವೈಕುಂಠಕ್ಕೆ ಹೊರಟಿರುವೆ” ಎಂದು ಹೇಳಿದರು. “ಹಾಗಾದರೆ, ಎಂದು ನಾನು ಮುಕ್ತನಾಗುತ್ತೇನೆ, ಎಂಬುದನ್ನು ಕೇಳಿ” ಎಂದನು. ಕೆಲವು ಕಾಲದ ಮೇಲೆ ನಾರದರು ಪುನಃ ಅದೇ ದಾರಿಯಲ್ಲಿ ಬಂದರು. ಸುತ್ತಲೂ ಹುತ್ತ ಬೆಳೆದುಕೊಂಡು ಮಧ್ಯೆ ತಪಸ್ಸು ಮಾಡುತ್ತಿದ್ದವನು ಅಲ್ಲೆ ಇದ್ದನು. “ಏನು ನಾರದರೆ, ನನ್ನ ವಿಚಾರವಾಗಿ ದೇವರನ್ನು ಕೇಳಿದಿರಾ?” ಎಂದು ಕೇಳಿದನು ನಾರದರು “ದೇವರು ನಿನಗೆ ಇನ್ನು ನಾಲ್ಕು ಜನ್ಮದಲ್ಲಿ ಮುಕ್ತಿ ಸಿಗುತ್ತದೆ ಎಂದು ಹೇಳಿದನು” ಎಂದರು. ನಂತರ ಆ ಮನುಷ್ಯನು ಗೋಳಾಡಲು ಮೊದಲು ಮಾಡಿ, ಸುತ್ತಲೂ ಹುತ್ತ ಬೆಳೆಯುವ ತನಕ ನಾನು ತಪಸ್ಸು ಮಾಡಿದರೂ, ಇನ್ನೂ ನಾಲ್ಕು ಜನ್ಮಗಳು ಬೇಕೆ ನನ್ನ ಮುಕ್ತಿಗೆ, ಎಂದು ಪರಿತಪಿಸಿದನು. ನಾರದರು ಮತ್ತೊಬ್ಬ ಮನುಷ್ಯನ ಹತ್ತಿರ ಬಂದರು. “ಏನು, ನನ್ನ ಪ್ರಶ್ನೆಯನ್ನು ಕೇಳಿದಿರಾ?” ಎಂದನು. ಅದಕ್ಕೆ ನಾರದರು “ಹೌದು. ಈ ಹುಣಸೇಮರವನ್ನು ನೋಡಿದೆಯ, ಅಲ್ಲಿ ಎಷ್ಟು ಎಲೆಗಳಿವೆಯೋ ಅಷ್ಟು ವೇಳೆ ನೀನು ಜನ್ಮ ತಾಳಬೇಕು. ಅನಂತರ ನಿನಗೆ ಮುಕ್ತಿ ದೊರೆಯುವುದು” ಎಂದರು. ಅದಕ್ಕೆ ಆ ಮನುಷ್ಯನು ಸಂತೋಷದಿಂದ ಕುಣಿದಾಡುತ್ತ “ಏನು ಇಷ್ಟು ಅಲ್ಪ ಕಾಲದಲ್ಲಿ ನನಗೆ ಮೋಕ್ಷ ಸಿಕ್ಕುವುದೆ!” ಎಂದನು. “ಮಗು, ನೀನು, ಈ ಕ್ಷಣವೇ ಮುಕ್ತಜೀವಿ” ಎಂದಿತು ಒಂದು ಧ್ವನಿ. ಅವನ ಸತತ ಪ್ರಯತ್ನಕ್ಕೆ ಇದು ಪ್ರತಿಫಲ. ಇಷ್ಟೊಂದು ಜನ್ಮಗಳು ಅವನು ಪ್ರಯತ್ನ ಪಡುವುದಕ್ಕೆ ಸಿದ್ಧವಾಗಿದ್ದನು. ಯಾವುದೂ ಅವನ ಉತ್ಸಾಹಕ್ಕೆ ಭಂಗ ತರಲಿಲ್ಲ. ಆದರೆ ಮೊದಲನೆಯವನಿಗೆ ಮುಂದಿನ ನಾಲ್ಕು ಜನ್ಮಗಳೂ ಕೂಡ ಬೇಸರವಾಗಿತ್ತು. ಸಾವಿರಾರು ವರ್ಷಗಳಾದರೂ ಕಾಯಲು ಸಿದ್ಧನಾಗಿರುವ ಮನುಷ್ಯನ ಸತತ ಪ್ರಯತ್ನದಿಂದ ಮಾತ್ರ ಅತ್ಯುತ್ತಮ ಪ್ರತಿಫಲವು ದೊರಕುವುದು.

\chapter[ಪೀಠಿಕೆ : ಪಾತಂಜಲ ಯೋಗಸೂತ್ರಗಳು]{ಪಾತಂಜಲ ಯೋಗಸೂತ್ರಗಳು}

\begin{center}
{\LARGE \bf ಪೀಠಿಕೆ}
\end{center}
ಯೋಗಸೂತ್ರದ ಬಗೆ ಹೇಳುವುದಕ್ಕೆ ಮುಂಚೆ ಯೋಗಿಗಳ ಧರ್ಮ ಮತ್ತು ತತ್ತ್ವವೆಲ್ಲವೂ ನಿಂತಿರುವ ಒಂದು ದೊಡ್ಡ ಪ್ರಶ್ನೆಯನ್ನು ನಾನು ಚರ್ಚಿಸುತ್ತೇನೆ. ನಮ್ಮ ಈಗಿನ ವ್ಯಕ್ತ ಅವಸ್ಥೆಯ ಹಿಂದೆ, ಅವ್ಯಕ್ತ ಅವಸ್ಥೆ ಇದೆ, ನಾವು ಬಂದಿರುವುದು ಅದರಿಂದ, ಅವ್ಯಕ್ತವೇ ಪ್ರಪಂಚದಲ್ಲಿ ಅಭಿವ್ಯಕ್ತವಾಗುತ್ತಿದೆ ಮತ್ತು ನಾವು ಪುನಃ ಅವ್ಯಕ್ತಾವಸ್ಥೆಗೆ ಹಿಂತಿರುಗುವುದಕ್ಕಾಗಿಯೆ ಮುಂದುವರಿಯುತ್ತಿರುವುದು – ಎಂಬುದು ಪ್ರಪಂಚದ ಮಹಾಪುರುಷರ ಏಕಾಭಿಪ್ರಾಯ. ಬಾಹ್ಯ ಪ್ರಕೃತಿಯ ಸಂಶೋಧನೆಯಿಂದ ಇದನ್ನು ಮುಕ್ಕಾಲು ಪಾಲು ಆಗಲೇ ಸಪ್ರಮಾಣವಾಗಿ ತೋರಿರುವರು. ಇದನ್ನೇನೊ ಒಪ್ಪಿಕೊಳ್ಳೋಣ. ಆದರೆ ವ್ಯಕ್ತಾವಸ್ಥೆ ಉತ್ತಮವೊ ಅಥವಾ ಅವ್ಯಕ್ತಾವಸ್ಥೆಯೊ ಎಂಬುದು ಈಗಿನ ಪ್ರಶ್ನೆ. ವ್ಯಕ್ತಾವಸ್ಥೆಯೇ ಮಾನವನ ಪರಮಸ್ಥಿತಿ ಎಂದು ಆಲೋಚಿಸುವವರಿಗೆ ಬರಗಾಲವಿಲ್ಲ. ನಾವು ಅವ್ಯಕ್ತಾವಸ್ಥೆಯ ವ್ಯಕ್ತಸ್ವರೂಪ, ಅವ್ಯಕ್ತಕ್ಕಿಂತಲೂ ವ್ಯಕ್ತಾವಸ್ಥೆಯು ಉತ್ತಮಸ್ಥಿತಿ ಎಂಬುದು ಮಹಾ ಮೇಧಾವಿಗಳಾದ ಕೆಲವರ ಅಭಿಪ್ರಾಯ. ಅವ್ಯಕ್ತಾವಸ್ಥೆಯಲ್ಲಿ ಯಾವ ಗುಣಗಳೂ ಇರಲಾರವು, ಅದು ಜಡ, ಅಚೇತನ, ನಿರ್ಜೀವ, ಸಾವೆಂದು ಅವರು ಭಾವಿಸುವರು. ನಮ್ಮ ಕಣ್ಣೆದುರಿಗೆ ಇರುವ ಜೀವನವನ್ನು ಮಾತ್ರ ನಾವು ಅನುಭವಿಸಬಲ್ಲೆವು, ಅದಕ್ಕೇ ಅದನ್ನು ನಾವು ಬಿಡಕೂಡದು ಎಂಬುದು ಅವರ ಮತ. ಮೊದಲನೆಯದಾಗಿ ಜೀವನ ಸಮಸ್ಯೆಗಳನ್ನು ಎದುರಿಸುವ ಇತರ ವಿವರಣೆಗಳನ್ನು ವಿಮರ್ಶಿಸೋಣ. ಪುರಾತನ ಕಾಲದ ಒಂದು ಪರಿಹಾರವೇನೆಂದರೆ, ಮರಣಾನಂತರ ಮನುಷ್ಯನು ಹಿಂದಿನಂತೆಯೇ ಇರುತ್ತಾನೆ; ಅವನ ಕೆಟ್ಟ ಗುಣಗಳು ಕಳೆದು ಒಳ್ಳೆಯ ಗುಣಗಳೆಲ್ಲ ಅವನಲ್ಲಿ ಎಂದೆಂದಿಗೂ ಇರುತ್ತವೆ ಎಂಬುದು. ತಾರ್ಕಿಕ ರೀತಿಯಲ್ಲಿ ಹೇಳುವುದಾದರೆ, ಮಾನವನ ಗುರಿಯೆ ಪ್ರಪಂಚ ಎಂಬುದು ಈ ಅಭಿಪ್ರಾಯ. ಇದು ಸ್ವಲ್ಪ ಉತ್ತಮವಾಗಿ ದೋಷಗಳೆಲ್ಲ ಮಾಯವಾದರೆ ಅದನ್ನೇ ಅವರು ಸ್ವರ್ಗ ಅನ್ನುವುದು. ಈ ವಾದ ಅಸಂಬದ್ಧ ಮತ್ತು ಹುರುಳಿಲ್ಲದ್ದು. ಏಕೆಂದರೆ ಇದು ಹಾಗಿರಲಾರದು. ಒಳ್ಳೆಯದಿಲ್ಲದೆ ಕೆಟ್ಟದ್ದು ಇರುವುದಿಲ್ಲ, ಅಥವಾ ಕೆಟ್ಟದ್ದಿಲ್ಲದೆ ಒಳ್ಳೆಯದಿರುವುದಿಲ್ಲ. ಕೆಟ್ಟದ್ದು ಎಂಬುದು ಇಲ್ಲವೇ ಇಲ್ಲದೆ ಎಲ್ಲಾ ಒಳ್ಳೆಯದಾಗಿರುವ ಪ್ರಪಂಚದಲ್ಲಿ ವಾಸಿಸುವುದನ್ನು ಸಂಸ್ಕೃತದಲ್ಲಿ ಆಕಾಶಕುಸುಮವೆನ್ನುತ್ತಾರೆ. ಆಧುನಿಕ ಕಾಲದಲ್ಲಿ ಅನೇಕ ತತ್ತ್ವಜ್ಞರು ಮತ್ತೊಂದು ವಾದವನ್ನು ತರುವರು. ಅದೇ, ಮಾನವನ ಗುರಿ ಯಾವಾಗಲೂ ದೋಷವನ್ನು ತಿದ್ದಿಕೊಳ್ಳುವುದು, ಯಾವಾಗಲೂ ಗುರಿಯೆಡೆಗೆ ಪ್ರಯಾಣ ಮಾಡುವುದು, ಆದರೆ ಎಂದಿಗೂ ಗುರಿಯನ್ನು ಸೇರುವುದಲ್ಲ. ಈ ಹೇಳಿಕೆಯೂ ಕೂಡ ಮೊದಲ ನೋಟಕ್ಕೆ ಚೆನ್ನಾಗಿ ಕಂಡರೂ ದೋಷದಿಂದ ಕೂಡಿದೆ. ಏಕೆಂದರೆ ನೇರವಾಗಿ ಚಲನೆ ಎಂಬುದಿಲ್ಲ. ಚಲನೆಯೆಲ್ಲ ಒಂದು ವೃತ್ತಾಕಾರವಾಗಿರುತ್ತದೆ. ನೀವು ಒಂದು ಕಲ್ಲನ್ನು ತೆಗೆದುಕೊಂಡು ಅದನ್ನು ಆಕಾಶಕ್ಕೆ ಎಸೆಯಲು ಸಾಧ್ಯವಾದರೆ, ನಂತರ ನೀವು ಬದುಕಿದ್ದರೆ, ಆ ಕಲ್ಲಿಗೆ ಮತ್ತಾವ ಅಡಚಣೆಯೂ ಸಿಕ್ಕದೆ ಇದ್ದರೆ, ಆ ಕಲ್ಲು ಪುನಃ ನಿಮ್ಮ ಕೈಗೆ ಸರಿಯಾಗಿ ಬರುತ್ತದೆ. ಒಂದು ಸರಳರೇಖೆಯನ್ನು ಅನಂತವಾಗಿ ಎಳೆದರೆ ಅದು ಒಂದು ವೃತ್ತದಲ್ಲಿ ಕೊನೆಗಾಣಬೇಕು. ಆದಕಾರಣ ಮಾನವ ಗುರಿಯು ನಿಲ್ಲದೆ ಎಂದಿಗೂ ಮುಂದುವರಿಯುತ್ತಿರುವುದು ಎನ್ನುವುದು ದೋಷದಿಂದ ಕೂಡಿದೆ. ವಿಷಯಕ್ಕೆ ಸ್ವಲ್ಪ ಹೊರಗಾದರೂ ಕೂಡ, ನೀವು ಯಾರನ್ನೂ ದ್ವೇಷಿಸಕೂಡದು, ಎಲ್ಲರನ್ನೂ ಪ್ರೀತಿಸಬೇಕೆಂಬ ನೈತಿಕ ಸಿದ್ಧಾಂತವನ್ನು ಇದು ವಿವರಿಸುತ್ತದೆ. ಏಕೆಂದರೆ, ವಿದ್ಯುತ್​ ಶಕ್ತಿಯ ಮೇಲೆ ಆಧುನಿಕ ಸಿದ್ಧಾಂತವೇನೆಂದರೆ ಶಕ್ತಿಯು ಡೈನಮೋ ಬಿಟ್ಟು, ತನ್ನ ಸಂಚಾರವನ್ನು ಪೂರೈಸಿ ಪುನಃ ಡೈನಮೋ ಸೇರುವುದು. ಇದರಂತೆಯೇ ದ್ವೇಷ ಮತ್ತು ಪ್ರೀತಿಗಳೂ ಕೂಡ. ಅವುಗಳೂ ಕೂಡ ತಮ್ಮ ಮೂಲಸ್ಥಾನಕ್ಕೆ ಬರಬೇಕು. ಆದಕಾರಣವೆ, ನೀವು ಯಾರನ್ನೂ ದ್ವೇಷಿಸಬೇಡಿ. ಏಕೆಂದರೆ ಯಾವ ದ್ವೇಷ ನಿಮ್ಮಿಂದ ಬರುತ್ತದೆಯೋ ಅದು ಕೊನೆಗೆ ನಿಮ್ಮನ್ನೇ ಸೇರಬೇಕು. ನೀವು ಪ್ರೀತಿಸಿದರೆ, ಅದು ತನ್ನ ಸಂಚಾರವನ್ನು ಪೂರೈಸಿ ಕೊನೆಗೆ ಅದು ನಿಮ್ಮಲ್ಲಿಗೇ ಬರುವುದು. ಅದು ಸತ್ಯದಷ್ಟೆ ಪ್ರಮಾಣವಾದುದು. ಮನುಷ್ಯನಿಂದ ಹೊರಗೆ ಹೋಗುವ ಪ್ರತಿಯೊಂದು ದ್ವೇಷದ ಆಲೋಚನೆಯೂ ಕೂಡ ಅಷ್ಟೇ ವೇಗವಾಗಿ ಪುನಃ ಅವನಲ್ಲಿಗೆ ಬರುವುದು. ಇದನ್ನು ಯಾವುದೂ ತಡೆಯಲಾರದು. ಇದರಂತೆಯೇ ಪ್ರತಿಯೊಂದು ಪ್ರೀತಿಯ ಭಾವನೆಯೂ ಕೂಡ ಅವನಲ್ಲಿಗೆ ಹಿಂತಿರುಗುವುದು. 

ಬೇರೆ ಪ್ರಾಯೋಗಿಕ ದೃಷ್ಟಿಯಿಂದ ನೋಡಿದರೆ, ಕೊನೆ ಇಲ್ಲದ ಪ್ರಗತಿಯನ್ನು ಪ್ರತಿಪಾದಿಸುವ ಸಿದ್ಧಾಂತಗಳು ನಿಲ್ಲಲಾರವು. ಪ್ರಪಂಚದಲ್ಲಿರುವ ಪ್ರತಿಯೊಂದು ವಸ್ತುವೂ ನಾಶದಲ್ಲಿ ಕೊನೆಗಾಣಬೇಕು. ನಮ್ಮ ಪ್ರಯತ್ನಗಳು, ಅಂಜಿಕೆ, ಆಸೆ, ಆನಂದಗಳು ಕೊನೆಗೆ ನಮ್ಮನ್ನು ಎಲ್ಲಿ ಕರೆದೊಯ್ಯುವುವು? ನಾವೆಲ್ಲರೂ ಕೊನೆಗಾಣುವುದು ಸಾವಿನಲ್ಲಿ. ಇದರಷ್ಟು ನಿಶ್ಚಯ ಮತ್ತಾವುದೂ ಇಲ್ಲ. ಸರಳರೇಖೆಯಲ್ಲಿ ಚಲನೆ ಎಂಬುದೆಲ್ಲಿದೆ? ಈ ಅಂತ್ಯವಿಲ್ಲದ ಪ್ರಗತಿ ಎಂಬುದೆಲ್ಲಿದೆ? ಇದೆಲ್ಲ ಸ್ವಲ್ಪ ದೂರ ಹೋಗುವುದು. ನೋಡಿ, ಹೇಗೆ ನೀಹಾರಿಕೆಗಳಿಂದ ಸೂರ್ಯ, ಚಂದ್ರ, ತಾರಕೆಗಳು ಬಂದಿವೆ? ಪುನಃ ಅವುಗಳು ನಾಶವಾಗಿ ನೀಹಾರಿಕೆಗೇ ಹೋಗುತ್ತವೆ. ಇದೇ ಎಲ್ಲಾ ಕಡೆಯೂ ಆಗುತ್ತಿರುವುದು. ಸಸ್ಯವು ಭೂಮಿಯಿಂದ ವಸ್ತುವನ್ನು ಸ್ವೀಕರಿಸಿ, ಅನಂತರ ಅದನ್ನು ವಿಘಟಿಸಿ ಕೊನೆಗೆ ಅದಕ್ಕೇ ಹಿಂತಿರುಗಿಸುವುದು. ಪ್ರಪಂಚದಲ್ಲಿರುವ ಪ್ರತಿಯೊಂದು ಆಕಾರವೂ ಕೂಡ ಸುತ್ತಲಿರುವ ಕಣಗಳಿಂದ ಬಂದಿದೆ. ಕೊನೆಗೆ ಆ ಕಣಗಳಿಗೆ ಹೋಗುತ್ತದೆ. ಒಂದೇ ನಿಯಮ ಬೇರೆ ಬೇರೆ ಕಡೆಗಳಲ್ಲಿ ಬೇರೆ ಬೇರೆ ರೀತಿಯಲ್ಲಿ ಕೆಲಸ ಮಾಡುವುದು ಸಾಧ್ಯವಿಲ್ಲ. ನಿಯಮವು ಸಾಮಾನ್ಯವಾದುದು. ಇದಕ್ಕಿಂತ ಹೆಚ್ಚು ನಿಜವಾಗಿರುವುದು ಬೇರೆ ಇಲ್ಲ. ಇದು ಪ್ರಕೃತಿ ನಿಯಮವಾದರೆ ಇದು ಆಲೋಚನೆಗೂ ಅನ್ವಯಿಸುತ್ತದೆ. ಆಲೋಚನೆಯು ಕರಗಿ ತನ್ನ ಮೂಲಸ್ಥಾನಕ್ಕೆ ಹೋಗುವುದು. ನಮಗೆ ಇಚ್ಛೆ ಇರಲಿ ಇಲ್ಲದೇ ಇರಲಿ ನಾವು ಬಂದಲ್ಲಿಗೆ ಅಂದರೆ, ದೇವರೆಡೆಗೆ, ಅವ್ಯಕ್ತದೆಡೆಗೆ ಹೋಗಲೇಬೇಕು. ನಾವೆಲ್ಲ ಬಂದಿರುವುದು ದೇವರಿಂದ. ನಾವೆಲ್ಲ ಪುನಃ ಅಲ್ಲಿಗೆ ಹೋಗಲೇಬೇಕಾಗಿದೆ. ನೀವು ಅದನ್ನು ದೇವರು, ಅವ್ಯಕ್ತ, ಪ್ರಕೃತಿ ಎಂದು ಏನು\break\ ಬೇಕಾದರೂ ಕರೆಯಿರಿ. ಆದರೆ ಸತ್ಯವು ಎಲ್ಲಿಗೂ ಹೋಗುವುದಿಲ್ಲ. “ಯಾರಿಂದ ವಿಶ್ವವೆಲ್ಲ ಬಂದಿದೆಯೊ, ಯಾರಲ್ಲಿ ಹುಟ್ಟಿದ ಜೀವಿಗಳೆಲ್ಲ ಜೀವಿಸುತ್ತಿವೆಯೋ, ಕೊನೆಗೆ ಯಾರಲ್ಲಿಗೆ ಇವುಗಳೆಲ್ಲ ಸೇರುತ್ತವೆಯೋ,” ಅದೊಂದೇ ಸತ್ಯಾಂಶ. ಪ್ರಕೃತಿಯೂ ಈ ನಿಯಮವನ್ನೇ ಅನುಸರಿಸುತ್ತದೆ. ಒಂದು ಕ್ಷೇತ್ರದಲ್ಲಿ ಆಗುತ್ತಿರುವುದೇ ಎಲ್ಲಾ ಕ್ಷೇತ್ರಗಳಲ್ಲಿ ಆಗುವುವು. ಗ್ರಹಗಳ ಗತಿ ಹೀಗೆಯೆ. ಭೂಮಿ, ಮನುಷ್ಯ ಮತ್ತು ಇತರ ವಸ್ತುಗಳ ಗತಿಯೂ ಕೂಡ ಇದೇ. ಒಂದು ದೊಡ್ಡ ಅಲೆ ಅನೇಕ ಸಣ್ಣ ಅಲೆಗಳ ಮೊತ್ತ, ಅವುಗಳು ಲಕ್ಷಾಂತರ ಅಲೆಗಳಾಗಿರಬಹುದು. ಪ್ರಪಂಚದ ಇಡೀ ಜೀವನವೆ ಅನೇಕ ಲಕ್ಷಾಂತರ ಸಣ್ಣ ಜೀವಗಳ ಮೊತ್ತ. ಒಟ್ಟು ವಿಶ್ವದ ನಾಶವೆ ಲಕ್ಷಾಂತರ ಸಣ್ಣ ಜೀವಿಗಳ ಮೊತ್ತದ ನಾಶ. 

\vskip 0.3cm

ಈಗೊಂದು ಪ್ರಶ್ನೆ ಏಳುತ್ತದೆ. ದೇವರೆಡೆಗೆ ಹಿಂತಿರುಗುವುದು ಉತ್ತಮ ಅವಸ್ಥೆಯೆ, ಅಲ್ಲವೆ? ಯೋಗಶಾಸ್ತ್ರಜ್ಞರು ಅದು ನಿಜವಾಗಿಯೂ ಅತ್ಯುತ್ತಮಸ್ಥಿತಿ ಎಂದು ಸ್ಪಷ್ಟವಾಗಿ ಹೇಳುತ್ತಾರೆ. ಮಾನವನ ಈಗಿನ ಸ್ಥಿತಿ ಒಂದು ಹೀನ ಸ್ಥಿತಿ. ಪ್ರಪಂಚದಲ್ಲಿರುವ ಯಾವ ಧರ್ಮವೂ ಕೂಡ ಈ ಜೀವನವು ಅವನ ಉತ್ತಮ ಸ್ಥಿತಿ ಎಂದು ಒಪ್ಪುವುದಿಲ್ಲ. ಇದರ ಅರ್ಥವೇನೆಂದರೆ, ಮಾನವನ ಆದಿ ಪೂರ್ಣವಾದದ್ದು, ಮತ್ತು ಪರಿಶುದ್ಧವಾದದ್ದು. ಅವನು ಕ್ರಮೇಣ ಕ್ಷೀಣದೆಸೆಗೆ ಬರುತ್ತ ಕೊನೆಗೆ ಅವನು ಇನ್ನೂ ಹೆಚ್ಚು ಕ್ಷೀಣದೆಸೆಗೆ ಹೋಗಲು ಅಸಾಧ್ಯವಾದ ಒಂದು ಸ್ಥಿತಿಗೆ ಬರುತ್ತಾನೆ. ಆಗ ಅವನು ಪುನಃ ಮೇಲಕ್ಕೆ ಏಳುವ ಒಂದು ಸಮಯ ಬಂದೇ ಬರುವುದು. ಆಗ ವೃತ್ತ ಪೂರ್ತಿಯಾಗಲೇ ಬೇಕು. ಅವನು ಎಷ್ಟೇ ಕ್ಷೀಣದೆಸೆಗೆ ಇಳಿಯಲಿ, ಕೊನೆಗೆ ಅವನು ಮೇಲೇಳುವ ಸ್ಥಿತಿಗೆ ಬಂದೇ ಬರಬೇಕು, ತನ್ನ ಮೂಲ ಸ್ಥಾನವಾದ ದೇವರೆಡೆಗೆ ಹೋಗಲೇಬೇಕು. ಮೊದಲು ಮಾನವನು ಬರುವುದು ದೇವರಿಂದ, ಮಧ್ಯದಲ್ಲಿ ಮಾನವನಾಗುವನು, ಕೊನೆಗೆ ಪುನಃ ದೇವರೆಡೆಗೆ ಹೋಗುವನು. ದ್ವೈತ ರೀತಿಯಲ್ಲಿ ಹೇಳುವುದು ಹೀಗೆ. ಅದ್ವೈತ ರೀತಿಯಲ್ಲಿ ನರನೇ ದೇವರು, ಅವನಲ್ಲಿಗೇ ಅವನು ಪುನಃ ಹೋಗುತ್ತಾನೆ. ನಮ್ಮ ಈಗಿನ ಸ್ಥಿತಿಯೆ ಉತ್ತಮವಾದ ಸ್ಥಿತಿಯಾದರೆ, ಇಲ್ಲಿ ಇಷ್ಟೊಂದು ಕಷ್ಟ ದುಃಖಗಳಿವೆ ಏಕೆ? ಇದಕ್ಕೆ ಏಕೆ ಒಂದು ಅಂತ್ಯವಿದೆ? ಇದೇ ಉತ್ತಮ ಸ್ಥಿತಿಯಾದರೆ ಏತಕ್ಕೆ ಕೊನೆಗಾಣಬೇಕು? ಯಾವುದು ನಮ್ಮನ್ನು ಹೀನರನ್ನಾಗಿ ಮಾಡುತ್ತದೆಯೋ, ಯಾವುದು ನಮ್ಮನ್ನು ಕ್ಷಯಿಸುವಂತೆ ಮಾಡುತ್ತದೆಯೋ ಅದೆಂದಿಗೂ ಉತ್ತಮ ಸ್ಥಿತಿಯಾಗಿರಲಾರದು. ಇದೇಕೆ ಇಷ್ಟು ಅಯುಕ್ತಿಕರವಾಗಿರಬೇಕು? ಭಯಂಕರವಾಗಿರಬೇಕು? ಇದು ಉತ್ತಮ ಸ್ಥಿತಿಗೆ ಒಂದು ಮೆಟ್ಟಲು; ನಾವು ಮುಕ್ತರಾಗಬೇಕಾದರೆ ಈ ಸ್ಥಿತಿಯ ಮೂಲಕ ಹೋಗಬೇಕಾಗಿದೆ, ಎಂದರೆ ಇದು ಕ್ಷಮಾರ್ಹವಾದುದು. ನೀವು ಒಂದು ಬೀಜವನ್ನು ನೆಲದಲ್ಲಿ ಹಾಕಿದರೆ ಮೊದಲು ಕೊಳೆತು ಅದು ನಾಶವಾಗುತ್ತದೆ. ಅನಂತರ ಆ ನಾಶದಿಂದ ಸುಂದರವಾದ ಸಸಿ ಬರುವುದು. ಪ್ರತಿಯೊಂದು ಜೀವನೂ ಕೂಡ ದೇವರಾಗಬೇಕಾದರೆ ಮೊದಲು ನಾಶವಾಗಬೇಕು. ಆದಕಾರಣ ಇದರಿಂದ ಏನು ಗೊತ್ತಾಗುತ್ತದೆ ಎಂದರೆ, ನಾವು ಕರೆಯುವ ಮಾನವ ಜೀವನವೆಂಬ ಸ್ಥಿತಿಯಿಂದ ಎಷ್ಟು ಬೇಗ ತಪ್ಪಿಸಿಕೊಂಡರೆ ಅಷ್ಟು ಅನುಕೂಲ. ಏನು, ಆತ್ಮಹತ್ಯೆಯಿಂದ ನಾವು ಈ ಸ್ಥಿತಿಯಿಂದ ತಪ್ಪಿಸಿಕೊಳ್ಳಬೇಕೆ? ಎಂದಿಗೂ ಇಲ್ಲ, ಅದು ಇನ್ನೂ ಹದಗೆಡೆಸಿದಂತೆ ಆಗುತ್ತದೆ. ಆತ್ಮ ಹಿಂಸೆ ಅಥವಾ ಪ್ರಪಂಚವನ್ನು ದೂರುವುದು ನಾವು ಪ್ರಪಂಚದಿಂದ ತಪ್ಪಿಸಿಕೊಳ್ಳುವುದಕ್ಕೆ ದಾರಿಯಲ್ಲ. ನಿರಾಶೆಯ ಕಣಿವೆಯಲ್ಲಿ ನಾವು ಹಾದುಹೋಗಬೇಕಾಗಿದೆ. ನಾವು ಎಷ್ಟು ಬೇಗ ದಾಟಿದರೆ ಅಷ್ಟು ಒಳ್ಳೆಯದು. ಮಾನವ ಸ್ಥಿತಿಯೇ ಅತ್ಯುತ್ತಮಾವಸ್ಥೆಯಲ್ಲ ಎಂಬುದನ್ನು ನಾವು ಜ್ಞಾಪಕದಲ್ಲಿಟ್ಟುಕೊಂಡಿರಬೇಕು. 

\vskip 0.3cm

ಪರಮಾವಸ್ಥೆ ಎನ್ನುವ ಅವ್ಯಕ್ತಸ್ಥಿತಿಯು ಕೆಲವರು ಅಂಜುವಂತೆ ಕಲ್ಲು ಅಥವಾ ಸಸ್ಯಜಾತಿಯ ಜಡತನಕ್ಕೆ ಅನ್ವಯಿಸುವುದಿಲ್ಲ. ನಿಜವಾಗಿಯೂ ತಿಳಿದುಕೊಳ್ಳುವುದಕ್ಕೆ ಸ್ವಲ್ಪ ಕಷ್ಟವಾದ ವಿಭಾಗವಿದೆ. ಅವರ ದೃಷ್ಟಿಯಲ್ಲಿ ಎರಡು ಅವಸ್ಥೆ ಮಾತ್ರ ಇರುವುದು, ಒಂದು ಕಲ್ಲಿನಂತೆ ಅಚೇತನ, ಮತ್ತೊಂದು ಆಲೋಚನೆ. ಅಸ್ತಿತ್ವವನ್ನು ಎರಡು ಭಾಗ ಮಾತ್ರ ಮಾಡುವುದಕ್ಕೆ ಅವರಿಗೆ ಏನು ಹಕ್ಕಿದೆ? ಆಲೋಚನೆಗಿಂತಲೂ ಎಷ್ಟೋ ಪಾಲು ಉತ್ತಮವಾಗಿರುವ ಮತ್ತೊಂದು ಸ್ಥಿತಿ ಇಲ್ಲವೆ? ಜ್ಯೋತಿಯ ಸ್ಪಂದನ ಬಹಳ ಕಡಿಮೆಯಾದಾಗ ನಮಗೆ ಅದು ಕಾಣುವುದಿಲ್ಲ. ಅದು ಸ್ವಲ್ಪ ಜಾಸ್ತಿಯಾದಾಗ ಬೆಳಕಾಗಿ ಕಾಣುತ್ತದೆ. ಅದು ಮತ್ತೂ ಹೆಚ್ಚಾದರೆ ಕಾಣುವುದಿಲ್ಲ. ಅದು ನಮ್ಮ ಪಾಲಿಗೆ ಕತ್ತಲೆಯಂತೆ. ಕೊನೆಯ ಕತ್ತಲೆ ಮತ್ತು ಮೊದಲ ಕತ್ತಲೆ ಎರಡೂ ಒಂದೇ ಏನು? ಎಂದಿಗೂ ಇಲ್ಲ. ಅವು ಎರಡು ಧ್ರುವಗಳಷ್ಟು ದೂರದಲ್ಲಿವೆ. ಕಲ್ಲಿನ ಆಲೋಚನಾರಾಹಿತ್ಯವೂ ದೇವರ ಆಲೋಚನಾರಾಹಿತ್ಯವೂ ಒಂದೇ ಏನು? ಖಂಡಿತ ಅಲ್ಲ. ದೇವರು ಆಲೋಚಿಸುವುದಿಲ್ಲ. ಅವನು ತರ್ಕಿಸುವುದಿಲ್ಲ. ಅವನೇಕೆ ಆಲೋಚಿಸಬೇಕು? ಅವನಿಗೆ ಯಾವುದಾದರೂ ತಿಳಿಯದೆ ಇರುವ ವಸ್ತುವು ಇದೆಯೇನು ಅವನು ಆಲೋಚಿಸುವುದಕ್ಕೆ? ಕಲ್ಲಿಗೆ ಆಲೋಚನೆ ಮಾಡಲು ಸಾಧ್ಯವಿಲ್ಲ, ದೇವರು ಆಲೋಚನೆ ಮಾಡುವುದಿಲ್ಲ, ಇದೇ ವ್ಯತ್ಯಾಸ. ಈ ತತ್ತ್ವಜ್ಞಾನಿಗಳಿಗೆ ಆಲೋಚನೆಯನ್ನು ಮೀರಿಹೋಗುವುದು ತುಂಬಾ ಅಸಹ್ಯವೆನಿಸುತ್ತದೆ. ಅವರ ಕಣ್ಣಿಗೆ ಆಲೋಚನೆಯಿಂದಾಚೆ ಏನೂ ಕಾಣಿಸುವುದಿಲ್ಲ. 

\vskip 0.3cm

ಆಲೋಚನೆಯನ್ನು ಮೀರಿದ ಎಷ್ಟೋ ಉತ್ತಮ ಅವಸ್ಥೆಗಳಿವೆ. ನಿಜವಾಗಿಯೂ ಬುದ್ಧಿಯ ಆಚೆ ನಮಗೆ ಧಾರ್ಮಿಕ ಸ್ಥಿತಿಯ ಪ್ರಥಮ ಅವಸ್ಥೆ ಕಾಣುವುದು. ಆಲೋಚನೆ, ಬುದ್ಧಿ ಮತ್ತು ಯುಕ್ತಿಗಳನ್ನು ಮೀರಿ ಹೋದಾಗ ಮಾತ್ರ ನಾವು ದೇವರೆಡೆಗೆ ಮೊದಲನೆ ಮೆಟ್ಟಲನ್ನು ಇಟ್ಟಂತೆ ಇದೇ ಅಧ್ಯಾತ್ಮಿಕ ಜೀವನದ ಪ್ರಾರಂಭ. ಸಾಧಾರಣವಾಗಿ ನಾವು ಯಾವುದನ್ನು ಜೀವನವೆಂದು ಕರೆಯುತ್ತೇವೆಯೋ ಅದಿನ್ನೂ ಭ್ರೂಣ ಸ್ಥಿತಿಯಲ್ಲಿದೆ. 

\vskip 0.3cm

ಆಲೋಚನೆ ಮತ್ತು ಯುಕ್ತಿಗಳನ್ನು ಮೀರಿರುವ ಅವಸ್ಥೆಯು ನಿಜವಾಗಿಯೂ\break ಅತ್ಯುತ್ತಮಾವಸ್ಥೆ ಎಂಬುದಕ್ಕೆ ಪ್ರಮಾಣವೇನು ಎನ್ನುವುದೇ ಮುಂದಿನ ಪ್ರಶ್ನೆ. ಮೊದಲನೆಯದಾಗಿ ಜಗತ್ತಿನ ಎಲ್ಲಾ ಮಹಾಪುರುಷರು ಕೇವಲ ಬಾಯಿಮಾತಿನವರಿಗಿಂತಲೂ ಉತ್ತಮರು. ಪ್ರಪಂಚಕ್ಕೆ ದಾರಿ ತೋರಿದವರು ಅವರು. ಒಂದಾದರೂ ಸ್ವಾರ್ಥ ಆಲೋಚನೆಗಳನ್ನು\break ಮಾಡಿದವರಲ್ಲ. ಅಂತಹವರೆಲ್ಲರೂ ನಮ್ಮ ಜೀವನವನ್ನು, ವ್ಯಕ್ತಾವಸ್ಥೆಯ ಆಚೆ ಇರುವ ಅನಂತತೆಯನ್ನು ಸೇರುವಾಗ ದಾರಿಯಲ್ಲಿ ಸಿಕ್ಕುವ ಕ್ಷಣಿಕ ಅವಸ್ಥೆ ಎಂದು ಸಾರುತ್ತಾರೆ. ಎರಡನೆಯದಾಗಿ ಅವರು ಹೀಗೆ ಹೇಳುವುದು ಮಾತ್ರವಲ್ಲ; ಉಳಿದವರು ಕೂಡ ತಮ್ಮನ್ನು ಅನುಸರಿಸಲಿ ಎಂದು ದಾರಿ ತೋರುತ್ತಾರೆ; ಅದನ್ನು ವಿವರಿಸುತ್ತಾರೆ. ಮೂರನೆಯದಾಗಿ, ಬೇರೆ ದಾರಿಯಿಲ್ಲ, ಬೇರೊಂದು ವಿವರಣೆಯಿಲ್ಲ, ಇದಕ್ಕಿಂತ ಉತ್ತಮ ಸ್ಥಿತಿ ಬೇರೊಂದಿಲ್ಲ ಎಂಬುದನ್ನು ನಾವು ಒಪ್ಪಿಕೊಳ್ಳೋಣ. ಹಾಗಾದರೆ ನಾವು ಸದಾ ಸಂಸಾರ ಚಕ್ರದಲ್ಲಿ ಏಕೆ ಸುತ್ತುತ್ತಿರುವುದು? ಮತ್ತಾವ ಯುಕ್ತಿಯು ಪ್ರಪಂಚವನ್ನು ವಿವರಿಸಬಲ್ಲುದು? ಇದನ್ನು ಮೀರಿಹೋಗಲು ನಮಗೆ ಅಸಾಧ್ಯವಾದರೆ, ನಾವು ಮತ್ತಾವ ಪ್ರಶ್ನೆಯನ್ನೂ ಕೇಳಕೂಡದು ಎಂದರೆ, ಇಂದ್ರಿಯ ಗೋಚರ ಪ್ರಪಂಚವೆ ನಮ್ಮ ಜ್ಞಾನಕ್ಕೆ ಪರಮಾವಧಿಯಾಗುವುದು. ಇದನ್ನೇ ನಾವು ಆಜ್ಞೇಯತಾವಾದ ಎನ್ನುವುದು. ಆದರೆ ಇಂದ್ರಿಯಗಳ ಪ್ರಾಮಾಣ್ಯವನ್ನು ನಂಬುವುದಕ್ಕೆ ಕಾರಣವೇನು? ದಾರಿಯಲ್ಲಿ ಸುಮ್ಮನೆ ನಿಂತು ಸಾಯುವವನನ್ನು ನಾನು ನಿಜವಾಗಿಯೂ ಅಜ್ಞೇಯತಾವಾದಿ ಎನ್ನುತ್ತೇನೆ. ಯುಕ್ತಿಯೇ ಪರಮಾವಧಿಯಾದರೆ ಶೂನ್ಯತೆಯ ಈ ದಡದಲ್ಲಿ ನಮಗೆ ನಿಂತುಕೊಳ್ಳುವುದಕ್ಕೆ ಸ್ಥಳವೇ ಇರುವುದಿಲ್ಲ. ಹಣ, ಕೀರ್ತಿ ಇವನ್ನು ಬಿಟ್ಟು ಉಳಿದವುಗಳ ವಿಷಯದಲ್ಲಿ ಅವನು ಅಜ್ಞೇಯತಾವಾದಿಯಾದರೆ ಅವನು ಆಷಾಢ ಭೂತಿ. ಯುಕ್ತಿ ಎಂಬ ಪ್ರಚಂಡವಾದ ಕೋಟೆಯ ಆಚೆ ಇಣಕಿ ನೋಡಲು ನಮಗೆ ಸಾಧ್ಯವಿಲ್ಲ ಎಂಬುದನ್ನು ನಿಸ್ಸಂಶಯವಾಗಿ ತತ್ತ್ವಜ್ಞನಾದ ಕಾಂಟನು ಸಿದ್ಧಾಂತ ಮಾಡಿರುವನು. ಆದರೆ ಭಾರತೀಯ ತತ್ತ್ವಶಾಸ್ತ್ರಗಳ ಮೊದಲನೆ ಅಡಿಗಲ್ಲೆ ಅದು. ಜ್ಞಾನಿಯು ಯುಕ್ತಿಗೆ ಮೀರಿರುವುದನ್ನು ಹುಡುಕಲು ಸಾಹಸಪಡುವನು ಮತ್ತು ಅದರಲ್ಲಿ ಜಯಶೀಲನಾಗುವನು. ನಮ್ಮ ಈಗಿರುವ ಸ್ಥಿತಿಗೆ ವಿವರಣೆ ಸಿಕ್ಕುವುದು ಅಲ್ಲಿ ಮಾತ್ರ. ಈ ಸಂಸಾರದಾಚೆ ನಮ್ಮನ್ನು ಕರೆದೊಯ್ಯುವ ವಿದ್ಯೆಯನ್ನು ಅಧ್ಯಯನ ಮಾಡಿದರೆ ದೊರಕುವ ಫಲವೇ ಇದು. “ನೀನು ನಮ್ಮ ತಂದೆ, ಈ ಮಾಯೆಯ ಕಡಲಿನಾಚೆ ನಮ್ಮನ್ನು ಕರೆದೊಯ್ಯುವೆ.” ಇದೇ ಧರ್ಮದ ವಿಜ್ಞಾನ, ಉಳಿದವುಗಳಲ್ಲ.

\chapter{ಏಕಾಗ್ರತೆ, ಇದರ ಆಧ್ಯಾತ್ಮಿಕ ಪ್ರಯೋಜನ}%%೩೩


~

 ಅಥ ಯೋಗಾನುಶಾಸನಮ್​~॥ ೧~॥
 
 ಇನ್ನು ಏಕಾಗ್ರತೆಯನ್ನು ವಿವರಿಸುವುದು.


\medskip

ಯೋಗಶ್ಚಿತ್ತವೃತ್ತಿ ನಿರೋಧಃ~॥ ೨~॥

ಯೋಗವೆಂದರೆ ಚಿತ್ತದ ವೃತ್ತಿಗಳನ್ನು ನಿರೋಧಿಸುವುದು.

\medskip

ಇದಕ್ಕೆ ಸ್ವಲ್ಪ ಹೆಚ್ಚು ವಿವರಣೆ ಆವಶ್ಯಕ. ಚಿತ್ತವೆಂದರೇನು ಎಂಬುದನ್ನು ತಿಳಿದುಕೊಳ್ಳಬೇಕು. ನನಗೆ ಕಣ್ಣುಗಳಿವೆ. ಆದರೆ ಕಣ್ಣುಗಳು ನೋಡುವುದಿಲ್ಲ. ತಲೆಯಲ್ಲಿರುವ ಮಿದುಳಿನ ಕೇಂದ್ರವನ್ನು ತೆಗೆದುಬಿಟ್ಟರೆ ಕಣ್ಣು ಹೊರಗೆ ಇರಬಹುದು. ಅದರಲ್ಲಿ ದೃಷ್ಟಿಗೆ ಸಂಬಂಧಪಟ್ಟ ನರಗಳು ಪೂರ್ತಿ ಇರಬಹುದು; ಅದರ ಮೇಲೆ ವಸ್ತುವಿನ ಚಿತ್ರವೂ ಇರಬಹುದು; ಆದರೂ ಕಣ್ಣು ನೋಡಲಾರದು. ಆದಕಾರಣ ಕಣ್ಣು ಅಪ್ರಧಾನವಾದ ಕರಣ, ಇದೇ ನೋಟದ ಇಂದ್ರಿಯವಲ್ಲ. ದೃಷ್ಟಿಯ ಇಂದ್ರಿಯ ಮಿದುಳಿನಲ್ಲಿರುವ ಒಂದು ನರದ ಕೇಂದ್ರದಲ್ಲಿದೆ. ಎರಡು ಕಣ್ಣುಗಳೇ ಸಾಲದು. ಕೆಲವು ವೇಳೆ ಮನುಷ್ಯನು ಬಿಟ್ಟ ಕಣ್ಣುಗಳಿಂದ ನಿದ್ರಿಸುವನು. ಬೆಳಕಿದೆ, ಚಿತ್ರವಿದೆ. ಆದರೆ ಮೂರನೆಯದೊಂದು ಆವಶ್ಯಕವಿದೆ. ಮನಸ್ಸು ಇಂದ್ರಿಯದೊಂದಿಗೆ ಸೇರಬೇಕು. ಕಣ್ಣು ಬಾಹ್ಯಕರಣ; ನಮಗೆ ಮಿದುಳಿನ ಕೇಂದ್ರ ಮತ್ತು ಮನಸ್ಸಿನ ಕ್ರಿಯೆ ಎರಡೂ ಬೇಕು. ಗಾಡಿಗಳು ರಸ್ತೆಯ ಮೇಲೆ ಉರುಳುತ್ತವೆ. ಆದರೆ ನಿಮಗೆ ಅದು ಕೇಳುವುದಿಲ್ಲ. ಅದಕ್ಕೆ ಕಾರಣವೇನು? ಏಕೆಂದರೆ ನಿಮ್ಮ ಮನಸ್ಸು ಕೇಳುವ ಇಂದ್ರಿಯದೊಂದಿಗೆ ಸೇರಿಲ್ಲ. ಮೊದಲನೆಯದಾಗಿ ಬಾಹ್ಯಕರಣವಿದೆ; ಎರಡನೆಯದು ಮಿದುಳಿನಲ್ಲಿ ಅದಕ್ಕೆ ಸಂಬಂಧಪಟ್ಟ ನರಗಳ ಕೇಂದ್ರ; ಮೂರನೆಯದು ಇವೆರಡರೊಡನೆ ಸೇರಿರುವ ಮನಸ್ಸು. ಮನಸ್ಸು ಸಂವೇದನೆಯನ್ನು ಮತ್ತೂ ಮುಂದಕ್ಕೆ ತೆಗೆದುಕೊಂಡುಹೋಗಿ ನಿರ್ಧರಿಸುವ ಬುದ್ಧಿಗೆ ತೋರುತ್ತದೆ. ಆಗ ಅಲ್ಲಿ ಪ್ರತಿಕ್ರಿಯೆಯಾಗುವುದು. ಈ ಪ್ರತಿಕ್ರಿಯೆಯೊಂದಿಗೆ ನಾನೆಂಬ ಭಾವನೆ ಹೊಳೆಯುವುದು. ಅನಂತರ ಈ ಕ್ರಿಯೆ ಮತ್ತು ಪ್ರತಿಕ್ರಿಯೆಗಳ ಮಿಶ್ರವು ಪುರುಷನ ಮುಂದೆ ಇಡಲ್ಪಡುವುದು. ಅವನೇ ನಿಜವಾದ ಆತ್ಮನು. ಈ ಮಿಶ್ರಸ್ಥಿತಿಯಲ್ಲಿ ಅವನು ವಸ್ತುವನ್ನು ನೋಡುವನು. ಇಂದ್ರಿಯ, ಮನಸ್ಸು, ಬುದ್ಧಿ ಮತ್ತು ಅಹಂಕಾರ ಇವುಗಳೆಲ್ಲ ಅಂತಃಕರಣಕ್ಕೆ ಸಂಬಂಧಪಟ್ಟವು. ಇವುಗಳೆಲ್ಲವೂ ಚಿತ್ತದ ನಾನಾ ವಿಕಾರಗಳು. ಚಿತ್ತದ ಆಲೋಚನಾ ತರಂಗಗಳಿಗೆ ವೃತ್ತಿ ಎಂದು ಹೆಸರು. ಆಲೋಚನೆ ಎಂದರೇನು? ಆಕರ್ಷಣ ಮತ್ತು ವಿಕರ್ಷಣದಂತೆ ಅದೂ ಒಂದು ಶಕ್ತಿ. ಪ್ರಕೃತಿಯಲ್ಲಿರುವ ಅನಂತ ಶಕ್ತಿಯ ಗಣಿಯಿಂದ ಚಿತ್ತವೆಂಬ ಯಂತ್ರವು ಕೆಲವನ್ನು ತೆಗೆದುಕೊಂಡು ಅದನ್ನು ತನ್ನಲ್ಲಿ ಜೀರ್ಣಿಸಿ ಆಲೋಚನೆಯಂತೆ ಹೊರಗೆ ಕಳುಹಿಸುವುದು. ಶಕ್ತಿಯು ನಮಗೆ ಆಹಾರದಿಂದ ಬರುವುದು. ಆಹಾರದಿಂದಲೇ ದೇಹವು ಚಲನೆ ಮತ್ತು ಇನ್ನೂ ಇತರ ಕ್ರಿಯೆಗಳಿಗೆ ಶಕ್ತಿಯನ್ನು ಪಡೆಯುವುದು. ಉಳಿದ ಸೂಕ್ಷ್ಮಶಕ್ತಿಗಳನ್ನು ಆಲೋಚನೆಯ ರೂಪದಲ್ಲಿ ಹೊರಗೆ ಕಳುಹಿಸುವುದು. ಆದಕಾರಣ ಮನಸ್ಸೇ ಚೇತನವಲ್ಲವೆಂಬುದು ನಮಗೆ ಕಾಣುತ್ತದೆ. ಆದರೂ ಕೂಡ ಅದು ಚೇತನದಿಂದ ಕೂಡಿರುವಂತೆ ಇದೆ. ಇದಕ್ಕೆ ಕಾರಣವೇನು? ಏಕೆಂದರೆ ಇದರ ಹಿಂದೆ ಚೇತನಾತ್ಮನಿರುವನು. ನೀವೇ ಆ ಚೇತನಾತ್ಮರು. ಮನಸ್ಸು ನೀವು ಬಾಹ್ಯ ಪ್ರಪಂಚವನ್ನು ನೋಡುವುದಕ್ಕೆ ಇರುವ ಒಂದು ಯಂತ್ರ ಮಾತ್ರ. ಈ ಪುಸ್ತಕವನ್ನು ತೆಗೆದುಕೊಳ್ಳಿ. ಇದು ಒಂದು ಪುಸ್ತಕದಂತೆ ಹೊರಗೆ ಇಲ್ಲ. ಹೊರಗೆ ಇರುವುದು ನಮಗೆ ತಿಳಿಯದು ಮತ್ತು ಅಜ್ಞಾತವಾದುದು. ಅಜ್ಞಾತವಸ್ತುವು ನಮಗೆ ಒಂದು ಸೂಚನೆಯನ್ನು ಒದಗಿಸುವುದು. ಸೂಚನೆ ಮನಸ್ಸಿಗೆ ಒಂದು ಪೆಟ್ಟನ್ನು ಕೊಡುವುದು. ಮನಸ್ಸು ಪುಸ್ತಕದಂತೆ ಪ್ರತಿಕ್ರಿಯೆಯನ್ನುಂಟುಮಾಡುವುದು. ಇದು ಹೇಗೆಂದರೆ, ನೀರೊಳಗೆ ಒಂದು ಕಲ್ಲನ್ನು ಎಸೆದರೆ, ನೀರು ಅಲೆಯಂತೆ ಪ್ರತಿಕ್ರಿಯೆಯನ್ನುಂಟುಮಾಡುವುದು. ಮನಸ್ಸಿನ ಪ್ರತಿಕ್ರಿಯೆಗೆ ಯಾವುದು ಕಾರಣವೊ ಅದೇ ನಿಜವಾದ ಪ್ರಪಂಚ. ಪುಸ್ತಕ, ಆನೆ, ಅಥವಾ ಮನುಷ್ಯನ ರೂಪ ಹೊರಗೆ ಇಲ್ಲ. ನಮಗೆ ಗೊತ್ತಿರುವುದೆಲ್ಲ ಹೊರಗಿನ ಸೂಚನೆಗೆ ನಮ್ಮ ಮನಸ್ಸಿನ ಪ್ರತಿಕ್ರಿಯೆ ಮಾತ್ರ. “ಶಾಶ್ವತವಾಗಿ ಸಂವೇದನೆಯು ಸಾಧ್ಯವಾಗುವಂತೆ ಮಾಡುವುದೇ ದ್ರವ್ಯವಸ್ತು.” –ಎಂದು ಜಾನ್​ಸ್ಟುಯರ್ಟ್​ ಮಿಲ್ಲನೆ ಹೇಳಿರುವನು. ಸೂಚನೆಯೊಂದೇ ಹೊರಗೆ ಇರುವುದು. ಉದಾಹರಣೆಗೆ ಒಂದು ಮುತ್ತಿನ ಚಿಪ್ಪನ್ನು ತೆಗೆದುಕೊಳ್ಳಿ. ಮುತ್ತು ಹೇಗಾಗುತ್ತದೆ ಎಂದು ನಿಮಗೆ ಗೊತ್ತಿದೆ. ಯಾವುದಾದರೊಂದು ಪರೋಪಜೀವಿಯು ಹೊರಗಿನಿಂದ ಚಿಪ್ಪಿನ ಒಳಗೆ ಸೇರಿ ಅದಕ್ಕೆ ನವೆಯನ್ನುಂಟುಮಾಡುವುದು. ಚಿಪ್ಪಿನಲ್ಲಿರುವ ಪ್ರಾಣಿಯು ಒಂದು ವಿಧದ ಗಿಲಾಯಿಯನ್ನು ಸುತ್ತಲೂ ಸ್ರವಿಸುವುದು. ಇದೇ ಮುತ್ತಾಗುವುದು. ಈ ಅನುಭವದ ಪ್ರಪಂಚ ನಮ್ಮ ಭಾವನೆಯಿಂದ ಉತ್ಪನ್ನವಾದುದು, ಎಂದು ಹೇಳಬಹುದು. ನಿಜವಾದ ಪ್ರಪಂಚ ಅದರ ಮಧ್ಯದಲ್ಲಿರುವ ಒಂದು ಪರೋಪಜೀವಿ. ಸಾಧಾರಣ ಮನುಷ್ಯನಿಗೆ ಇದು ಗೊತ್ತಾಗುವುದಿಲ್ಲ. ಏಕೆಂದರೆ ಅವನು ಹಾಗೆ ಮಾಡಲು ಯತ್ನಿಸಿದಾಗ ತನ್ನಿಂದ ತನ್ನ ಗಿಲಾಯಿಯನ್ನು ಹೊರಗೆಡಹುವನು ಮತ್ತು ತನ್ನ ಗಿಲಾಯಿಯನ್ನು ಮಾತ್ರ ನೋಡುವನು. ಈಗ ನಮಗೆ ವೃತ್ತಿ ಎಂದರೇನೆಂಬುದು ಗೊತ್ತಾಯಿತು. ನಿಜವಾದ ಮನುಷ್ಯನು ಮನಸ್ಸಿನ ಹಿಂದೆ ಇರುವನು. ಮನಸ್ಸೆಂಬುದು ಆತನ ಕೈಯಲ್ಲಿ ಒಂದು ಉಪಕರಣ. ಆತನ ಚೇತನ ಮನಸ್ಸಿನ ಮೂಲಕವಾಗಿ ಜಿನುಗುತ್ತಿರುವುದು. ನೀವು ಮನಸ್ಸಿನ ಹಿಂದೆ ನಿಂತಾಗ ಮಾತ್ರ ಅದು ನಿಮಗೆ ತಿಳಿಯುವುದು. ಮನುಷ್ಯನು ಮನಸ್ಸನ್ನು ತ್ಯಜಿಸಿದಾಗ, ಅದು ಕೆಳಗೆ ಬಿದ್ದು ನುಚ್ಚು ನೂರಾಗಿ ಕೆಲಸಕ್ಕೆ ಬಾರದೆ ಹೋಗುವುದು. ಚಿತ್ತವೆಂದರೆ ಏನೆಂಬುದು ನಿಮಗೆ ಈಗ ಗೊತ್ತಾಯಿತು. ಚಿತ್ತವೆಂದರೆ ಮನಸ್ಸು; ವೃತ್ತಿ ಎಂದರೆ ಬಾಹ್ಯಕಾರಣಗಳು ಅದನ್ನು ಕಲಕಿದಾಗ ಮೇಲೇಳುವ ಅಲೆಗಳು. ಈ ವೃತ್ತಿಗಳೇ ನಮ್ಮ ಪ್ರಪಂಚ. 

ಕಡಲಿನಾಳವನ್ನು ನಾವು ನೋಡಲಾರೆವು. ಅದರ ಮೇಲಿನ ಭಾಗವು ಅಲೆಗಳಿಂದ ಕೂಡಿರುವುದೇ ಇದಕ್ಕೆ ಕಾರಣ. ಅಲೆಗಳು ಅಡಗಿ ನೀರು ಶಾಂತವಾದಾಗ ಮಾತ್ರ ತಳಭಾಗವನ್ನು ಕ್ಷಣಕಾಲ ನೋಡಲು ನಮಗೆ ಸಾಧ್ಯ. ನೀರು ಬಗ್ಗಡವಾಗಿದ್ದರೆ, ಅಥವಾ ಸದಾಕಾಲದಲ್ಲಿಯೂ ಕಲಕಿದ್ದರೆ, ತಳಭಾಗವನ್ನು ನೋಡಲಾಗುವುದಿಲ್ಲ. ಅದು ತಿಳಿಯಾಗಿ, ಅಲೆ ಇಲ್ಲವಾದಾಗ ತಳಭಾಗವನ್ನು ನೋಡಬಹುದು. ಕಡಲಿನ ತಳಭಾಗವೆ ನಮ್ಮ ನಿಜವಾದ ಆತ್ಮ; ಕಡಲೇಚಿತ್ತ, ಅಲೆಗಳೇ ವೃತ್ತಿ. ಅದೂ ಅಲ್ಲದೆ ಮನಸ್ಸು ಮೂರು ಅವಸ್ಥೆಗಳಲ್ಲಿ ಇರುವುದು. ಮೊದಲನೆಯದು ಅಜ್ಞಾನ ಅಥವಾ ತಾಮಸ ಪ್ರಕೃತಿ. ಇದು ಸಾಧಾರಣವಾಗಿ ದಡ್ಡರು ಮತ್ತು ಮೂರ್ಖರಲ್ಲಿ ಇರುವುದು. ಇದು ಯಾವಾಗಲೂ ಅಪಾಯವನ್ನು ತರುವುದಕ್ಕೆ ಕೆಲಸ ಮಾಡುವುದು. ಈ ಅವಸ್ಥೆಯಲ್ಲಿರುವ ಮನಸ್ಸಿಗೆ ಮತ್ತಾವ ಭಾವನೆಯೂ ಹೊಳೆಯುವುದಿಲ್ಲ. ಎರಡನೆಯದೆ ಚಟುವಟಿಕೆಯಿಂದ ಕೂಡಿದ ರಾಜಸಿಕ ಅವಸ್ಥೆ. ದರ್ಪ ಮತ್ತು ಭೋಗವೇ ಅದರ ಮೂಲ ಉದ್ದೇಶ. “ನಾನು ಬಲಾಢ್ಯನಾಗುತ್ತೇನೆ, ಉಳಿದವರನ್ನು ಆಳುತ್ತೇನೆ” ಎಂಬುದೇ ಅದರ ಉದ್ದೇಶ. ಮೂರನೆಯದೇ ಸಾತ್ತ್ವಿಕಾವಸ್ಥೆ. ಇದು ನಿರ್ಮಲತೆ ಮತ್ತು ಶಾಂತಿಯಿಂದ ಕೂಡಿರುವುದು. ಇಲ್ಲಿ ಅಲೆಗಳು ಅಡಗುವುವು, ಮಾನಸಿಕ ಸರೋವರದ ನೀರು ಶುಭ್ರವಾಗುವುದು. ಇದು ಜಡಾವಸ್ಥೆಯಲ್ಲ, ತೀವ್ರ ಚಟುವಟಿಕೆಯಿಂದ ಕೂಡಿದುದು. ಶಾಂತಿಯಿಂದ ಕೂಡಿದುದೆ ಶಕ್ತಿಯ ಅಭಿವ್ಯಕ್ತಿಯ ಪರಮಾವಸ್ಥೆ. ಚಟುವಟಿಕೆಯಿಂದ ಕೂಡಿರುವುದು ಸುಲಭ. ಲಗಾಮನ್ನು ಸಡಿಲಿಸಿ. ಕುದುರೆಗಳು ನಿಮ್ಮೊಂದಿಗೆ ಓಡಿ ಹೋಗುವುವು. ಇದನ್ನು ಯಾರು ಬೇಕಾದರೂ ಮಾಡಬಹುದು. ಆದರೆ ಯಾರು ಓಡುವ ಕುದುರೆಯನ್ನು ನಿಲ್ಲಿಸುತ್ತಾನೆಯೊ ಅವನೇ ಶಕ್ತನು. ಲಗಾಮನ್ನು ಸಡಿಲ ಬಿಡುವುದೇ ಅಥವಾ ಅದನ್ನು ಬಿಗಿಹಿಡಿಯುವುದೇ? ಯಾವುದಕ್ಕೆ ಹೆಚ್ಚು ಶಕ್ತಿ ಬೇಕು? ಶಾಂತಸ್ವಭಾವದವನು ಸೋಮಾರಿಯಲ್ಲ. ಸಾತ್ತ್ವಿಕಾವಸ್ಥೆಯನ್ನು ಸೋಮಾರಿತನ ಅಥವಾ ಮೌಢ್ಯ ಎಂದು ತಪ್ಪು ತಿಳಿದುಕೊಳ್ಳಬಾರದು. ಚಿತ್ತವೃತ್ತಿಯ ಮೇಲೆ ಯಾರಿಗೆ ಸ್ವಾಧೀನವಿದೆಯೊ ಅವನೇ ಶಾಂತಸ್ವಭಾವದವನು. ಚಟುವಟಿಕೆ, ಕೀಳುದರ್ಜೆಯ ಶಕ್ತಿಯ ಅಭಿವ್ಯಕ್ತಿ. ಶಾಂತಿಯೆ ಉತ್ತಮ ಶಕ್ತಿಯ ಚಿಹ್ನೆ. 

ಚಿತ್ತವು ಯಾವಾಗಲೂ ತನ್ನ ಸ್ವಾಭಾವಿಕ ಶುದ್ಧ ಸ್ಥಿತಿಗೆ ಹೋಗುವುದಕ್ಕೆ ಪ್ರಯತ್ನಿಸುತ್ತಿರುವುದು. ಆದರೆ ಇಂದ್ರಿಯಗಳು ಅದನ್ನು ಹೊರಗೆ ಸೆಳೆಯುತ್ತವೆ. ಇಂದ್ರಿಯವನ್ನು ನಿಗ್ರಹಿಸಿ ಹೊರಗೆ ಹೋಗುವ ಮನಸ್ಸಿನ ಸ್ವಭಾವವನ್ನು ನಿಲ್ಲಿಸಿ, ಅದನ್ನು ಚೇತನಾತ್ಮನ ಕಡೆ ತಿರುಗುವಂತೆ ಮಾಡುವುದೇ ಯೋಗದಲ್ಲಿ ಮೊದಲನೆ ಕೆಲಸ. ಏಕೆಂದರೆ ಹಾಗೆ ಮಾಡಿದರೆ ಮಾತ್ರ ಚಿತ್ತವು ಸರಿಯಾದ ದಾರಿಗೆ ಬರಲು ಸಾಧ್ಯ. 

ಅತ್ಯಂತ ಕ್ಷುದ್ರತಮ ಮೃಗದಿಂದ ಹಿಡಿದು, ಉತ್ತಮೋತ್ತಮ ಜೀವಿಯವರೆವಿಗೂ ಪ್ರತಿಯೊಂದು ಪ್ರಾಣಿಯಲ್ಲಿಯೂ ಚಿತ್ತವಿದ್ದರೂ ಮಾನವನಲ್ಲಿ ಮಾತ್ರ ಅದು ಬುದ್ಧಿಯಂತೆ ನಮಗೆ ತೋರುವುದು. ಮನಸ್ಸು ಬುದ್ಧಿಯ ರೂಪವನ್ನು ತಾಳುವವರೆವಿಗೂ, ಈ ದಾರಿಯಲ್ಲಿ ಹಿಂತಿರುಗಿ ಆತ್ಮನನ್ನು ಬಂಧನದಿಂದ ಬಿಡಿಸುವುದು ಸಾಧ್ಯವಿಲ್ಲ. ಮನಸ್ಸಿದ್ದರೂ ಹಸು ಅಥವಾ ನಾಯಿ ಮುಂತಾದುವುಗಳಿಗೆ ತಕ್ಷಣ ಮೋಕ್ಷವನ್ನು ಪಡೆಯುವುದು ಅಸಾಧ್ಯ. ಏಕೆಂದರೆ ಅವುಗಳ ಚಿತ್ತವು ಇನ್ನೂ ಬುದ್ಧಿಯ ರೂಪವನ್ನು ತಾಳಿರುವುದಿಲ್ಲ. 

\eject

ಚಿತ್ತವು ಈ ಕೆಳಗಿನ ರೂಪಗಳಲ್ಲಿ ಕಾಣಿಸಿಕೊಳ್ಳುವುದು; ಕ್ಷಿಪ್ತ, ಮೂಢ, ವಿಕ್ಷಿಪ್ತ, ಏಕಾಗ್ರ ಮತ್ತು ನಿರುದ್ಧ. ಕ್ಷಿಪ್ತಾವಸ್ಥೆಯು ಚಟುವಟಿಕೆಯ ಗುರುತು. ಸುಖ ಅಥವಾ ದುಃಖದಂತೆ ತೋರುವುದೇ ಅದರ ಸ್ವಭಾವ. ಮಂದಪ್ರವೃತ್ತಿಯೆ ಮೂಢಾವಸ್ಥೆ. ಕೇಡು ಮಾಡುವುದೆ ಅದರ ಸ್ವಭಾವ. ಮೂರನೆಯದು ದೇವತೆಗಳ; ಮೊದಲನೆ ಮತ್ತು ಎರಡನೆಯದು ಅಸುರರ ಸ್ವಭಾವ ಎಂದು ಭಾಷ್ಯಕಾರರು ಹೇಳುತ್ತಾರೆ. ಚಿತ್ತವು ತಾನು ಒಂದು ಕಡೆ ಸೇರಬೇಕೆಂದು ಪ್ರಯತ್ನಿಸಿದಾಗ ಅದನ್ನು ವಿಕ್ಷಿಪ್ತಾವಸ್ಥೆ ಎನ್ನುತ್ತಾರೆ. ಧ್ಯಾನದಲ್ಲಿ ಮನಸ್ಸು ನೆಲೆನಿಂತಾಗ ಅದು ಏಕಾಗ್ರತೆ. ಈ ಏಕಾಗ್ರತೆಯಿಂದ ಸಮಾಧಿಯುಂಟಾಗುತ್ತದೆ. ಇದು ನಿರುದ್ಧಾವಸ್ಥೆ. 

\vspace{-0.2cm}

\begin{verse}
\textbf{ತದಾ ದ್ರಷ್ಟುಃ ಸ್ವರೂಪೇಽವಸ್ಥಾನಮ್​~॥ ೩~॥}
\end{verse}

\vspace{-0.3cm}

\dsize{ಆ ಸಮಯದಲ್ಲಿ ಅಂದರೆ ನಿರುದ್ಧಾವಸ್ಥೆಯಲ್ಲಿ ಪುರುಷನು ತನ್ನ ನೈಜಸ್ಥಿತಿಯಲ್ಲಿರುವನು. }

\vspace{0.1cm}

ಅಲೆಯು ನಿಂತು ಸರೋವರ ಶಾಂತವಾದೊಡನೆ ಅದರ ತಳವು ನಮಗೆ ಕಾಣುವುದು. ಅದರಂತೆಯೇ ಮನಸ್ಸು ಕೂಡ. ಅದು ಶಾಂತವಾದಾಗ ನಮ್ಮ ಸ್ವರೂಪ ನಮಗೆ ಪರಿಚಯವಾಗುವುದು. ನಾವು ಇತರ ವಸ್ತುಗಳೊಂದಿಗೆ ಬೆರೆಯುವುದಿಲ್ಲ. ಆದರೆ ನಾವೇ ಬೇರೆಯಾಗುವೆವು. 

\vspace{-0.2cm}

\begin{verse}
ವೃತ್ತಿಸಾರೂಪ್ಯಮಿತರತ್ರ~॥ ೪~॥
\end{verse}

\vspace{-0.3cm}

\dsize{ಬೇರೆ ಕಾಲದಲ್ಲಿ (ಅಂದರೆ, ಧ್ಯಾನಾವಸ್ಥೆಯಲ್ಲಿಲ್ಲದಾಗ) ಆತ್ಮನು ಚಿತ್ತವೃತ್ತಿಗಳೊಂದಿಗೆ ಒಂದಾಗಿರುವನು. }

\vspace{0.1cm}

ಉದಾಹರಣೆಗೆ, ಯಾರೋ ಒಬ್ಬರು ನನ್ನನ್ನು ನಿಂದಿಸಿದರೆಂದು ಇಟ್ಟು ಕೊಳ್ಳೋಣ. ಇದು ನನ್ನ ಮನಸ್ಸಿನಲ್ಲಿ ವೃತ್ತಿಯನ್ನು ಉಂಟುಮಾಡುತ್ತದೆ. ನನ್ನನ್ನು ನಾನು ಅದರೊಂದಿಗೆ ಏಕೀಕರಿಸಿಕೊಳ್ಳುವೆನು. ಇದರ ಫಲವೇ ದುಃಖ. 

\vspace{-0.2cm}

\begin{verse}
ವೃತ್ತಯಃ ಪಂಚತಯ್ಯಃ ಕ್ಲಿಷ್ಟಾ ಅಕ್ಲಿಷ್ಟಾಃ~॥ ೫~॥
\end{verse}

\vspace{-0.3cm}

\dsize{ಐದು ವಿವಿಧ ವೃತ್ತಿಗಳಿರುವುವು. ಕೆಲವು ಕಷ್ಟವಾದವು ಮತ್ತು ಕೆಲವು ಕಷ್ಟವಿಲ್ಲದವು. }

\vspace{-0.1cm}

\begin{verse}
ಪ್ರಮಾಣ–ವಿಪರ್ಯಯ–ವಿಕಲ್ಪ–ನಿದ್ರಾ–ಸ್ಮೃತಯಃ~॥ ೬~॥
\end{verse}

\vspace{-0.3cm}

\dsize{ಇವುಗಳು ಯಥಾರ್ಥಜ್ಞಾನ, ವಿವೇಚನೆ ಇಲ್ಲದಿರುವುದು, ಮೋಹ, ನಿದ್ರೆ ಮತ್ತು ನೆನಪು ಎಂದು ಐದು ವಿಧ. }

\vspace{-0.1cm}

\begin{verse}
ಪ್ರತ್ಯಕ್ಷಾನುಮಾನಾಗಮಾಃ ಪ್ರಮಾಣಾನಿ~॥ ೭~॥
\end{verse}

\vspace{-0.3cm}

\dsize{ಪ್ರತ್ಯಕ್ಷ, ಅನುಮಾನ, ಆಪ್ತವಾಕ್ಯಗಳೇ ಪ್ರಮಾಣ. }

\vskip 6pt

ನಮ್ಮ ಇಂದ್ರಿಯ ಗ್ರಹಣ ಶಕ್ತಿಯಲ್ಲಿ ಎರಡು ಒಂದನ್ನೊಂದು ವಿರೋಧಿಸದೆ ಇದ್ದರೆ ಅದನ್ನು ಪ್ರಮಾಣವೆನ್ನುತ್ತೇವೆ. ನಾನು ಯಾವುದನ್ನೊ ಕೇಳುತ್ತೇನೆ. ನಾನು ಆಗಲೇ ಕೇಳಿರುವುದನ್ನು ಇದು ವಿರೋಧಿಸಿದರೆ, ಇದನ್ನು ನಂಬುವುದಿಲ್ಲ. ವಿಮರ್ಶೆ ಮಾಡಿ ನೋಡುತ್ತೇನೆ. ಮೂರು ವಿಧದ ಪ್ರಮಾಣಗಳಿವೆ. ನಮ್ಮ ಇಂದ್ರಿಯಗಳನ್ನು ಯಾವುದೂ ಮೋಹಗೊಳಿಸದೆ ಇದ್ದರೆ ನಾವು ಏನೇನನ್ನು ಇಂದ್ರಿಯಗಳ ಮೂಲಕ ಅನುಭವಿಸುತ್ತೇವೆಯೊ ಅದು ಪ್ರತ್ಯಕ್ಷ; ಅಂದರೆ ನಾನೇ ಸ್ವಂತವಾಗಿ ಏನನ್ನು ನೋಡುತ್ತೇನೆಯೋ ಅಥವಾ ಅನುಭವಿಸುತ್ತೇನೆಯೊ ಅದು ಒಂದು ಪ್ರಮಾಣ. ನಾನು ಪ್ರಪಂಚವನ್ನು ನೋಡುತ್ತೇನೆ. ಪ್ರಪಂಚವಿದೆ ಎನ್ನುವುದಕ್ಕೆ ಇದೇ ಸಾಕಾದಷ್ಟು ಪ್ರಮಾಣ. ಎರಡನೆಯದೆ ಅನುಮಾನ. ನಿಮಗೆ ಒಂದು ಚಿಹ್ನೆ ಕಾಣುತ್ತದೆ. ಅದರಿಂದ ನೀವು ಅದು ಸೂಚಿಸುವ ವಸ್ತುವನ್ನು ತಿಳಿಯುತ್ತೀರಿ. ಮೂರನೆಯದು ಆಪ್ತವಾಕ್ಯ, ಸತ್ಯವನ್ನು ಯಾರು ಸಾಕ್ಷಾತ್ಕಾರ ಮಾಡಿಕೊಂಡಿರುವರೋ, ಅಂತಹ ಯೋಗಿಯ ಮಾತು. ನಾವೆಲ್ಲರೂ ಕೂಡ ಜ್ಞಾನವನ್ನು ಸಂಪಾದಿಸುವುದಕ್ಕೆ ಪ್ರಯತ್ನ ಮಾಡುತ್ತಿರುವೆವು. ಆದರೆ ನಾನು ಮತ್ತು ನೀವು ಬಹಳ ಕಷ್ಟಪಟ್ಟು ವಿಚಾರವೆಂಬ ದೂರದ ಕಷ್ಟದ ಹಾದಿಯ ಮೂಲಕ ಬರಬೇಕಾಗಿದೆ. ಆದರೆ ನಿರ್ಮಲಚಿತ್ತರಾದ ಯೋಗಿಗಳು ಇದನ್ನು ಮೀರಿ ಹೋಗಿರುವರು. ಅವರೆದುರಿಗೆ ಭೂತ ಭವಿಷ್ಯತ್​ ವರ್ತಮಾನಗಳೆಲ್ಲಾ ಓದುವ ಒಂದು ಪುಸ್ತಕದಂತಿರುತ್ತದೆ. ನಾವು ಜ್ಞಾನಾರ್ಜನೆಗೆ ಪಡುವಷ್ಟು ಕಷ್ಟವನ್ನು ಅವರು ಪಡುವ ಆವಶ್ಯಕತೆ ಇರುವುದಿಲ್ಲ. ಅವರ ಮಾತೇ ಪ್ರಮಾಣ. ಸತ್ಯ ತಮ್ಮಲ್ಲಿಯೇ ಅವರಿಗೆ ಕಾಣಿಸುವುದು. ನಮ್ಮ ಶಾಸ್ತ್ರಕಾರರು ಈ ಗುಂಪಿಗೆ ಸೇರಿದವರು. ಆದಕಾರಣವೇ ಶಾಸ್ತ್ರಗಳು ಪ್ರಮಾಣ. ಬೇರೆ ದಾರ್ಶನಿಕರು ಆಪ್ತವಾಕ್ಯದ ಮೇಲೆ ದೀರ್ಘ ಚರ್ಚೆಯನ್ನು ನಡೆಸಿ ಅವರ ಮಾತಿನ ಪ್ರಮಾಣವೇನೆಂದು ಕೇಳುವರು. ಪ್ರತ್ಯಕ್ಷ ಅನುಭವವೇ ಅವರ ಮಾತಿಗೆ ಪ್ರಮಾಣ. ಏಕೆಂದರೆ ನಾನೇನು ನೋಡಿರುವೆನೊ ಅದು ಪ್ರಮಾಣ; ನೀವೇನು ನೋಡಿರುವಿರೋ ಅದು ಪ್ರಮಾಣ, ಇದು ಹಿಂದಿನ ಅನುಭವವನ್ನು ವಿರೋಧಿಸದೆ ಇದ್ದರೆ. ಇಂದ್ರಿಯಾತೀತವಾದ ಜ್ಞಾನವೊಂದಿದೆ. ಹಿಂದಿನ ಅನುಭವಗಳನ್ನು ಮತ್ತು ಯುಕ್ತಿಯನ್ನು ಇದು ಎಂದು ವಿರೋಧಿಸುವುದಿಲ್ಲವೋ ಆಗ ಇದೊಂದು ಪ್ರಮಾಣ. ಯಾವನೊ ಒಬ್ಬ ಹುಚ್ಚನು ಈ ಕೋಣೆಯೊಳಕ್ಕೆ ಬಂದು, ತನ್ನ ಸುತ್ತಲೂ ದೇವತೆಗಳು ಕಾಣಿಸುತ್ತಿರುವರು, ಎಂದು ಹೇಳಿಕೊಳ್ಳಬಹುದು. ಆದರೆ ಇದೊಂದು ಪ್ರಮಾಣವಾಗುವುದಿಲ್ಲ. ಮೊದಲನೆಯದಾಗಿ ಅದು ನಿಜವಾದ ಜ್ಞಾನವಾಗಿರಬೇಕು; ಎರಡನೆಯದಾಗಿ ಹಿಂದಿನ ಅನುಭವಗಳನ್ನು ವಿರೋಧಿಸಕೂಡದು; ಮೂರನೆಯದು ಅದನ್ನು ಹೇಳುವವನ ನಡತೆ ಶುದ್ಧವಾಗಿರಬೇಕು. ಒಬ್ಬರು ಹೇಳುವ ಉಪದೇಶದಷ್ಟು ಅವರ ಶೀಲವೇನೂ ಮುಖ್ಯವಲ್ಲ; ಅವರು ಹೇಳುವುದನ್ನು ಮೊದಲು ಕೇಳೋಣ ಎಂದು ಹೇಳುವವರಿದ್ದಾರೆ. ಬೇರೆ ವಿಷಯಗಳನ್ನು ಬೋಧಿಸುತ್ತಿರುವಾಗ ಇದು ಸತ್ಯವಿರಬಹುದು. ಒಬ್ಬನು ಕೆಟ್ಟಚಾಳಿಯವನಾಗಿರಬಹುದು, ಆದರೂ ಖಗೋಳಶಾಸ್ತ್ರದಲ್ಲಿ ಏನಾದರೂ ಹೊಸದನ್ನು ಕಂಡುಹಿಡಿಯಬಹುದು. ಆದರೆ ಧಾರ್ಮಿಕ ಪ್ರಪಂಚದಲ್ಲಿ ಹಾಗಲ್ಲ. ಏಕೆಂದರೆ ಅಶುದ್ಧರಾದ ಯಾರೂ ಆಧ್ಯಾತ್ಮಿಕ ಸತ್ಯಗಳನ್ನು ಪಡೆಯುವ ಶಕ್ತಿಯನ್ನು ಹೊಂದಲು ಸಾಧ್ಯವಿಲ್ಲ. ಆದಕಾರಣ ಆಪ್ತನೆಂದು ಹೇಳಿಕೊಳ್ಳುವವನಲ್ಲಿ ಮೊದಲು ನಾವು, ಆತ ನಿಃಸ್ವಾರ್ಥಪರನೆ, ಸಜ್ಜನನೆ ಎಂಬುದನ್ನು ನೋಡಬೇಕು. ಎರಡನೆಯದಾಗಿ ಅವನು ಇಂದ್ರಿಯಾತೀತನಾಗಿರಬೇಕು. ಮೂರನೆಯದಾಗಿ ಅವನ ಬೋಧನೆ ಹಿಂದಿನ ಅನುಭವವನ್ನು ವಿರೋಧಿಸಕೂಡದು. ಯಾವ ಹೊಸದಾಗಿ ಕಂಡುಹಿಡಿದ ಸತ್ಯವೂ ಕೂಡ ಹಿಂದಿನ ಸತ್ಯವನ್ನು ವಿರೋಧಿಸಕೂಡದು, ಆದರೆ ಅದರೊಡನೆ ಹೊಂದಿಕೊಳ್ಳಬೇಕು. ನಾಲ್ಕನೆಯದಾಗಿ ಆ ಸತ್ಯವನ್ನು, ಅದು ಸರಿಯೆ ಇಲ್ಲವೆ ಎಂಬುದನ್ನು ವಿಮರ್ಶಿಸಲು ಸಾಧ್ಯವಾಗಬೇಕು. “ತನಗೆ ಒಂದು ದೃಶ್ಯ ಕಂಡಿದೆ” ಎಂದು ಒಬ್ಬನು ಹೇಳಿ, ಅದನ್ನು ನನಗೆ ನೋಡಲು ಅಧಿಕಾರವಿಲ್ಲವೆಂದರೆ ನಾನು ಅವನನ್ನು ನಂಬುವುದಿಲ್ಲ. ಪ್ರತಿಯೊಬ್ಬನಿಗೂ ಕೂಡ ಸ್ವತಂತ್ರವಾಗಿ ಅದನ್ನು ನೋಡುವ ಅಧಿಕಾರವಿರಬೇಕು. ಯಾರು ಜ್ಞಾನವನ್ನು ಮಾರುತ್ತಾರೆಯೋ ಅವರೆಂದಿಗೂ ಆಪ್ತರಾಗುವುದಿಲ್ಲ. ಈ ಎಲ್ಲಾ ನಿಯಮಗಳನ್ನು ಕೂಡ ನಾವು ಅನುಸರಿಸಬೇಕು. ಮೊದಲನೆಯದಾಗಿ ಆಪ್ತನು ಸಜ್ಜನನಾಗಿರಬೇಕು, ಅವನಲ್ಲಿ ಸ್ವಾರ್ಥ ಇರಕೂಡದು. ಅವನಿಗೆ ಲಾಭ, ಕೀರ್ತಿಗಳ ಆಸೆ ಇರಕೂಡದು. ಎರಡನೆಯದಾಗಿ ಅವನು ಪ್ರಜ್ಞಾತೀತನೆಂಬುದನ್ನು ತೋರಿಸಬೇಕು. ಇಂದ್ರಿಯಗಳಿಂದ ಸಿಕ್ಕದ ಯಾವುದಾದರೊಂದು ಅನುಭವವನ್ನು ಕೊಡಬೇಕು ಮತ್ತು ಅದರಿಂದ ಪ್ರಪಂಚಕ್ಕೆ ಒಳ್ಳೆಯದಾಗಬೇಕು. ಮೂರನೆಯದಾಗಿ ಅದು ಉಳಿದ ಸತ್ಯಗಳನ್ನು ವಿರೋಧಿಸುವುದಿಲ್ಲ ಎಂಬುದನ್ನು ನಾವು ತಿಳಿಯಬೇಕು. ಉಳಿದ ಯಾವುದಾದರೂ ಸತ್ಯವನ್ನು ಅಲ್ಲಗಳೆದರೆ ತಕ್ಷಣವೇ ಅದನ್ನು ತ್ಯಜಿಸಿ. ನಾಲ್ಕನೆಯದಾಗಿ ಅವನು ಏನು ಹೇಳುತ್ತಾನೆಯೋ ಅದು ತನಗೆ ಮಾತ್ರ ಸಾಧ್ಯ ಎಂಬಂತಿರಬಾರದು. ಎಲ್ಲರೂ ಯಾವುದನ್ನು ಸಾಧಿಸಬಲ್ಲರೊ ಅಂಥದನ್ನು ಅವನೂ ಸಾಧಿಸುವಂತಿರಬೇಕು. ಮೂರು ವಿಧದ ಪ್ರಮಾಣಗಳಾವುವೆಂದರೆ ಪ್ರತ್ಯಕ್ಷ, ಅನುಮಾನ ಮತ್ತು ಆಪ್ತವಾಕ್ಯ.

\begin{verse}
ವಿಪರ್ಯಯೋ ಮಿಥ್ಯಾಜ್ಞಾನಮತದ್ರೂಪಪ್ರತಿಷ್ಠಮ್​~॥ ೮~॥
\end{verse}

\vspace{-0.3cm}

\dsize{ವಿಪರ್ಯಯವು ಅಸತ್ಯಜ್ಞಾನ. ಇದು ನೈಜ ಸ್ವರೂಪವನ್ನು ಅವಲಂಬಿಸಿರುವುದಿಲ್ಲ. }

\vspace{0.2cm}

ಮತ್ತೊಂದು ತರಗತಿಗೆ ಸೇರಿದ ವೃತ್ತಿಯೇ ಒಂದನ್ನು ಮತ್ತೊಂದಾಗಿ ತಿಳಿದುಕೊಳ್ಳುವುದರಿಂದ ಬರುವುದು–ಕಪ್ಪೆಯಚಿಪ್ಪನ್ನು ನೋಡಿ ಬೆಳ್ಳಿ ಎಂದು ಭ್ರಾಂತಿ ಪಟ್ಟಂತೆ.

\begin{verse}
ಶಬ್ದಜ್ಞಾನಾನುಪಾತೀ ವಸ್ತುಶೂನ್ಯೋ ವಿಕಲ್ಪಃ~॥ ೯~॥
\end{verse}

\vspace{-0.3cm}

\dsize{ಸಂಬಂಧಪಟ್ಟ ವಸ್ತು ಇಲ್ಲದಿದ್ದರೂ ಶಬ್ದಜ್ಞಾನದಿಂದ ಉಂಟಾಗುವ ವೃತ್ತಿಗೆ ವಿಕಲ್ಪವೆಂದು ಹೆಸರು. }

\vspace{0.2cm}

ಯಾರೊ ಒಂದು ಮಾತನ್ನು ಆಡುತ್ತಾರೆ. ಅದರ ಅರ್ಥವೇನು ತಿಳಿದುಕೊಳ್ಳುವುದಕ್ಕೆ ನಾವು ಕಾಯುವುದಿಲ್ಲ. ತಕ್ಷಣವೇ ಒಂದು ನಿರ್ಣಯಕ್ಕೆ ಬರುತ್ತೇವೆ. ಇದು ಚಿತ್ತದ ನಿರ್ಬಲತೆಯ ಕುರುಹು. ಈಗ ನಿಗ್ರಹದ ಸಿದ್ಧಾಂತವು ನಿಮಗೆ ಗೊತ್ತಾಗುವುದು. ನಿರ್ಬಲವಾದಷ್ಟೂ ಅವನಿಗೆ ನಿಗ್ರಹ ಕಡಮೆ. ಈ ಪ್ರಮಾಣದಿಂದ ಯಾವಾಗಲೂ ನೀವು ಪರೀಕ್ಷೆ ಮಾಡಿಕೊಳ್ಳಿ. ಕೋಪಗೊಂಡಾಗ ಅಥವಾ ದುಃಖ ಪಡುವಾಗ ವಿಮರ್ಶೆಮಾಡಿ ನೋಡಿ. ನಿಮಗೆ ಬಂದ ಯಾವುದೋ ಒಂದು ಸುದ್ಧಿಯು ಮನಸ್ಸಿನಲ್ಲಿ ಹೇಗೆ ವೃತ್ತಿಗಳೇಳುವಂತೆ ಮಾಡಿತು ಎಂಬುದನ್ನು ಆಲೋಚಿಸಿ.

\begin{verse}
ಅಭಾವ–ಪ್ರತ್ಯಯಾಲಂಬನಾ ವೃತ್ತಿರ್ನಿದ್ರಾ~॥ ೧೦~॥
\end{verse}

\vspace{-0.4cm}

\dsize{ಅಭಾವವೆಂಬ ಪ್ರತ್ಯಯವನ್ನು ಆಧರಿಸಿಕೊಳ್ಳುವುದು ನಿದ್ರೆಯೆಂಬ ವೃತ್ತಿ. }

\vspace{0.1cm}

ನಿದ್ರೆ ಮತ್ತು ಸ್ವಪ್ನವೇ ಇನ್ನೊಂದು ಬಗೆಯ ವೃತ್ತಿಗಳು. ನಾವು ಎದ್ದ ಮೇಲೆ ನಿದ್ದೆ ಮಾಡುತ್ತಿದ್ದೆವು ಎಂಬುದು ಗೊತ್ತಾಗುವುದು. ಇಂದ್ರಿಯ ಗ್ರಹಣದ ನೆನಪು ಮಾತ್ರ ನಿಮಗೆ ಇರಬಲ್ಲುದು. ನಮಗೆ ಗೋಚರವಿಲ್ಲದುದರ ನೆನಪು ಎಂದಿಗೂ ನಮಗೆ ಆಗಲಾರದು. ಪ್ರತಿಯೊಂದು ಪ್ರತಿಕ್ರಿಯೆಯೂ ಸರೋವರದಲ್ಲಿ ಒಂದು ಅಲೆಯಂತೆ. ಈಗ ನಾವು ನಿದ್ದೆ ಮಾಡುತ್ತಿದ್ದಾಗ ಮನಸ್ಸಿನಲ್ಲಿ ಅಲೆಗಳಿಲ್ಲದೆ ಇದ್ದರೆ ಅದಕ್ಕೆ ನಿಶ್ಚಿತವಾದ ಅಥವಾ ಅನಿಶ್ಚಿತವಾದ ಯಾವ ಇಂದ್ರಿಯ ಗ್ರಹಣವೂ ಇರುತ್ತಿರಲಿಲ್ಲ. ಆದಕಾರಣ ಅವುಗಳನ್ನು ನಾವು ಜ್ಞಾಪಿಸಿಕೊಳ್ಳುವುದಕ್ಕೇ ಆಗುತ್ತಿರಲಿಲ್ಲ. ನಿದ್ರೆಯ ನೆನಪು ಇರುವ ಕಾರಣವೆ, ನಾವು ಆ ಅವಸ್ಥೆಯಲ್ಲಿದ್ದಾಗ ಮನಸ್ಸಿನಲ್ಲಿ ಕೆಲವು ಅಲೆಗಳಿದ್ದವು ಎನ್ನುವುದಕ್ಕೆ ಸಾಕ್ಷಿ. ಸ್ಮೃತಿ ಎಂಬುದು ಮತ್ತೊಂದು ವಿಧದ ವೃತ್ತಿ. 

\vspace{-0.2cm}

\begin{verse}
ಅನುಭೂತವಿಷಯಾಸಂಪ್ರಮೋಷಃ ಸ್ಮೃತಿಃ~॥ ೧೧~॥
\end{verse}

\vspace{-0.4cm}

\dsize{ಅನುಭವಿಸಿದ ವಿಷಯಗಳು ಮಾಯವಾಗದೆ ಇದ್ದು (ಸಂಸ್ಕಾರಗಳ ಮೂಲಕ ಮತ್ತೆ ಅರಿವಿಗೆ ಬರುವುದು) ಸ್ಮೃತಿ. }

\vspace{0.1cm}

ಸ್ಮೃತಿಯು ಪ್ರತ್ಯಕ್ಷ, ವಿಪರ್ಯಯ ವಿಕಲ್ಪ ಮತ್ತು ನಿದ್ರೆಗಳಿಂದ ಬರುವುದು. ಉದಾಹರಣೆಗೆ, ನೀವು ಒಂದು ಮಾತನ್ನು ಕೇಳಿದಿರಿ ಎಂದು ಇಟ್ಟುಕೊಳ್ಳೋಣ. ಆ ಮಾತು ಚಿತ್ತವೆಂಬ ಸರೋವರಕ್ಕೆ ಒಂದು ಕಲ್ಲನ್ನು ಎಸೆದಂತೆ. ಇದು ಒಂದು ಅಲೆಯನ್ನು ಹುಟ್ಟಿಸುತ್ತದೆ, ಆ ಆಲೆ ಅನೇಕ ಅಲೆಗಳಿಗೆ ಕಾರಣವಾಗುತ್ತದೆ. ಇದೇ ಸ್ಮೃತಿ. ನಿದ್ರೆಯಲ್ಲಿಯೂ ಇದರಂತೆಯೇ. ನಿದ್ರೆ ಎನ್ನುವ ಅಲೆಯು ಚಿತ್ತವೆಂಬ ಸರೋವರದಲ್ಲಿ ಸ್ಮೃತಿ ಎಂಬ ಬೇರೆ ಅಲೆಗಳನ್ನು ಉಂಟು ಮಾಡಿದರೆ ಅದಕ್ಕೆ ಸ್ವಪ್ನವೆಂದು ಹೆಸರು. ಸ್ವಪ್ನವೆನ್ನುವುದು ಮತ್ತೊಂದು ವಿಧದ ಅಲೆ. ಜಾಗ್ರತಾವಸ್ಥೆಯಲ್ಲಿ ಅದಕ್ಕೆ ಸ್ಮೃತಿ ಎಂದು ಹೆಸರು. 

\vspace{-0.2cm}

\begin{verse}
ಅಭ್ಯಾಸವೈರಾಗ್ಯಾಭ್ಯಾಂ ತನ್ನಿರೋಧಃ~॥ ೧೨~॥
\end{verse}

\vspace{-0.3cm}

\dsize{ಅಭ್ಯಾಸ ಮತ್ತು ವೈರಾಗ್ಯಗಳಿಂದ ಅವುಗಳನ್ನು ನಿರೋಧಿಸಬಹುದು. }

\vspace{0.1cm}

ಮನಸ್ಸಿನಲ್ಲಿ ಆಸಕ್ತಿ ಇಲ್ಲದೆ ಇರಬೇಕಾದರೆ ಅದು ಸ್ಪಷ್ಟವಾಗಿರಬೇಕು, ಶುದ್ಧವಾಗಿರಬೇಕು ಮತ್ತು ಯುಕ್ತಿಪೂರಿತವಾಗಿರಬೇಕು. ನಾವು ಏತಕ್ಕೆ ಅಭ್ಯಾಸ ಮಾಡಬೇಕು? ಪ್ರತಿಯೊಂದು ಕ್ರಿಯೆಯೂ ಸರೋವರದ ಮೇಲೆ ಅದುರುತ್ತಿರುವ ಸ್ಪಂದನದಂತೆ. ಸ್ಪಂದನವೇನೋ ಅಳಿಸಿಹೋಗುವುದು, ಆದರೆ ಉಳಿಯುವುದೇನು? ಅವೇ ಸಂಸ್ಕಾರಗಳು, ನಮ್ಮ ಮನಸ್ಸಿನ ಮೇಲೆ ಆದ ಪರಿಣಾಮಗಳು. ಮನಸ್ಸಿನಲ್ಲಿ ಇಂತಹ ಅನೇಕ ಸಂಸ್ಕಾರಗಳು ಉಳಿದರೆ, ಅವುಗಳೆಲ್ಲ ಒಟ್ಟಿಗೆ ಕಲೆತು ಒಂದು ಚಾಳಿಯಾಗುವದು. “ಚಾಳಿ ನಮ್ಮ ಎರಡನೇ ಸ್ವಭಾವ” ಎಂಬ ನಾಣ್ನುಡಿಯಿರುವುದು. ಅದು ಮಾತ್ರವಲ್ಲ, ಅದೇ ನಮ್ಮ ಮೊದಲನೆ ಸ್ವಭಾವ ಮತ್ತು ಸ್ವಭಾವವೆಲ್ಲ ಕೂಡ. ನಮ್ಮ ಈಗಿನ ಸ್ಥಿತಿಗೆಲ್ಲ ಅಭ್ಯಾಸವೇ ಮುಖ್ಯಕಾರಣ. ನಮಗೆ ಇದು ಸಮಾಧಾನವನ್ನು ಕೊಡುತ್ತದೆ. ಇದು ಕೇವಲ ಒಂದು ಅಭ್ಯಾಸವೆಂದರೆ ನಾವು ಇದನ್ನು ಯಾವಾಗ ಬೇಕಾದರೂ ರೂಢಿಸಿಕೊಳ್ಳಬಹುದು ಅಥವಾ ಬಿಡಬಹುದು. ನಮ್ಮ ಮನಸ್ಸಿನಿಂದ ಹೋಗುತ್ತಿರುವ ಸ್ಪಂದನವೂ ತನ್ನ ಪರಿಣಾಮದ ಸಂಸ್ಕಾರವನ್ನು ಹಿಂದೆ ಬಿಡುವುದು. ಈ ಪರಿಣಾಮಗಳ ಮೊತ್ತವೇ ನಮ್ಮ ಶೀಲ. ಯಾವ ಅಲೆ ಅಲ್ಲಿ ಮುಖ್ಯವಾಗಿದೆಯೊ ಅವನು ಆ ಸ್ವಭಾವವನ್ನು ಧರಿಸುತ್ತಾನೆ. ಒಳ್ಳೆಯದಿದ್ದರೆ ಒಳ್ಳೆಯವನಾಗುತ್ತಾನೆ, ಕೆಟ್ಟದ್ದು ಇದ್ದರೆ ಕೆಟ್ಟವನಾಗುತ್ತಾನೆ, ಸಂತೋಷವಾಗಿರುವುದಿದ್ದರೆ ಸಂತೋಷವಾಗಿರುತ್ತಾನೆ. ಕೆಟ್ಟ ಅಭ್ಯಾಸಕ್ಕೆ ಚಿಕಿತ್ಸೆ, ಅದಕ್ಕೆ ವಿರೋಧವಾದ ಒಳ್ಳೆಯ ಅಭ್ಯಾಸವನ್ನು ರೂಢಿಸುವುದು. ಕೆಟ್ಟ ಪರಿಣಾಮಗಳನ್ನು ಬಿಟ್ಟು, ಎಲ್ಲಾ ಕೆಟ್ಟ ಅಭ್ಯಾಸಗಳನ್ನೂ ಒಳ್ಳೆಯ ಅಭ್ಯಾಸದ ಬಲದಿಂದ ನಿಗ್ರಹಿಸಬೇಕು. ಎಡೆಬಿಡದೆ ಒಳ್ಳೆಯದನ್ನು ಮಾಡುತ್ತ ಮಾಡುತ್ತ ಹೋಗಿ; ಒಳ್ಳೆಯದನ್ನು ಆಲೋಚಿಸುತ್ತ ಇರಿ. ನಮ್ಮ ಕೀಳು ಸ್ವಭಾವವನ್ನು ಅಡಗಿಸುವುದಕ್ಕೆ ಇದೊಂದು ದಾರಿ. ಯಾರನ್ನೂ ಕೆಟ್ಟವರೆಂದು ಹೇಳಬೇಡಿ, ಏಕೆಂದರೆ ಅವರು ಕೆಲವು ಅಭ್ಯಾಸಗಳ ಬಲದಿಂದ ಒಂದು ಶೀಲವನ್ನು ಪಡೆದಿದ್ದಾರೆ. ಈ ಶೀಲವನ್ನು ನಾವು ಬೇರೆ ಉತ್ತಮ ಅಭ್ಯಾಸ ಬಲದಿಂದ ತಿದ್ದಬಹುದು. ಶೀಲವೆಂದರೆ ಪುನರಾವರ್ತಿಸಿದ ಅಭ್ಯಾಸ. ಪುನಃ ಪುನಃ ಅಭ್ಯಾಸ ಮಾತ್ರ ನಮ್ಮ ಶೀಲವನ್ನು ತಿದ್ದಬಹುದು. 

\vspace{-0.2cm}

\begin{verse}
ತತ್ರ ಸ್ಥಿತೌ ಯತ್ನೋಭ್ಯಾಸಃ~॥ ೧೩~॥
\end{verse}

\vspace{-0.4cm}

\dsize{ಚಿತ್ತವೃತ್ತಿಗಳನ್ನು ಸಂಪೂರ್ಣ ಸ್ವಾಧೀನದಲ್ಲಿಡುವುದಕ್ಕಾಗಿ ಪ್ರಯತ್ನಿಸುವುದೇ ಅಭ್ಯಾಸ. }

\vspace{0.1cm}

ಅಭ್ಯಾಸವೆಂದರೇನು? ಮನಸ್ಸನ್ನು ಚಿತ್ತಾವಸ್ಥೆಯಲ್ಲಿ ನಿಗ್ರಹಿಸಿ ಅದನ್ನು ಅಲೆಗಳಾಗಿ ಹೋಗದಂತೆ ತಡೆಯುವುದು. 

\vspace{-0.2cm}

\begin{verse}
ಸ ತು ದೀರ್ಘಕಾಲನೈರಂತರ್ಯಸತ್ಕಾರಾಸೇವಿತೋ ದೃಢಭೂಮಿಃ~॥ ೧೪~॥
\end{verse}

\vspace{-0.4cm}

\dsize{ಅದು ದೀರ್ಘಕಾಲ ನಿರಂತರ ಪ್ರಯತ್ನ ಬಲದಿಂದ ಮತ್ತು (ನಾವು ಸೇರಬೇಕೆಂದಿರುವ ಗುರಿಯ ಮೇಲಿರುವ) ಪ್ರೇಮದಿಂದ  ದೃಢವಾಗುತ್ತದೆ. }

\vspace{0.1cm}

ಒಂದು ದಿನದಲ್ಲಿ ನಿಗ್ರಹ ಸಿದ್ಧಿಸುವುದಿಲ್ಲ. ದೀರ್ಘಕಾಲ ಸತತ ಪ್ರಯತ್ನ ಬೇಕು. 

\vspace{-0.2cm}

\begin{verse}
ದ್ರಷ್ಠಾನುಶ್ರವಿಕ ವಿಷಯ ವಿತೃಷ್ಣಸ್ಯ ವಶೀಕಾರಸಂಜ್ಞಾ ವೈರಾಗ್ಯಮ್​~॥ ೧೫~॥
\end{verse}

\vspace{-0.4cm}

\dsize{ಕೇಳಿದ ಮತ್ತು ನೋಡಿದ ವಿಷಯಗಳ ಆಸೆಯನ್ನು ತ್ಯಜಿಸಿ ವಿಷಯವಸ್ತುಗಳನ್ನು ನಿಗ್ರಹಿಸಲು ಇಚ್ಛಿಸಿದ ಪ್ರಯತ್ನದ ಪರಿಣಾಮವೆ ವೈರಾಗ್ಯ. }

\vspace{0.1cm}

ನಮ್ಮ ಕೆಲಸಗಳಿಗೆ ಮುಖ್ಯಕಾರಣ, ಮೊದಲನೆಯದಾಗಿ, ನಾವಾಗಿ ಏನನ್ನು ನೋಡು\break ವೆಯೋ ಅದು; ಎರಡನೆಯದು ಮತ್ತೊಬ್ಬರ ಅನುಭವ. ಈ ಎರಡು ಶಕ್ತಿಗಳೂ ಕೂಡ ಮನಸ್ಸಿನ ಸರೋವರದಲ್ಲಿ ಬಹುವಿಧದ ಅಲೆಗಳನ್ನು ಉಂಟು ಮಾಡುತ್ತವೆ. ಇಂತಹ ಶಕ್ತಿಯೊಡನೆ ಹೋರಾಡಿ ಮನಸ್ಸನ್ನು ನಿಗ್ರಹಿಸುವುದಕ್ಕೆ ವೈರಾಗ್ಯ ಎಂದು ಹೆಸರು. ಅವುಗಳ (ವಸ್ತುಗಳ) ತ್ಯಾಗ ನಮಗೆ ಬೇಕಾಗಿರುವುದು. ನಾನು ದಾರಿಯಲ್ಲಿ ಹೋಗುತ್ತಿರುವೆನು. ಒಬ್ಬನು ಬಂದು ನನ್ನ ಗಡಿಯಾರವನ್ನು ತೆಗೆದುಕೊಂಡು ಹೋಗುವನು. ಇದು ನನ್ನ ಸ್ವಂತ ಅನುಭವ. ನಾನು ಇದನ್ನು ಸ್ವತಃ ನೋಡುತ್ತೇನೆ. ನನ್ನ ಮನಸ್ಸಿನಲ್ಲಿ ಕೋಪದ ಅಲೆ ಏಳುವುದು ಹಾಗೆ ಆಗದಂತೆ ನೋಡಿಕೊಳ್ಳಿ. ಅದು ನಿಮಗೆ ಸಾಧ್ಯವಿಲ್ಲದೆ ಇದ್ದರೆ ನೀವು ಯಾವು ದಕ್ಕೂ ಪ್ರಯೋಜನವಿಲ್ಲ. ಅದು ನಿಮಗೆ ಸಾಧ್ಯವಾದರೆ ನಿಮಗೆ ವೈರಾಗ್ಯವಿದೆ. ಪ್ರಾಪಂಚಿಕ ವಿಷಯಗಳಲ್ಲಿ ಮಗ್ನರಾದವರು, ಇಂದ್ರಿಯಗಳ ಸುಖವೇ ಪರಮ ಸುಖವೆಂದು ನಮಗೆ ಬೋಧಿಸುತ್ತಾರೆ. ಇದು ಭಯಂಕರವಾದ ಪ್ರಲೋಭನೆ. ಇದನ್ನು ಧಿಕ್ಕರಿಸಿ. ಇದಕ್ಕಾಗಿ ಮನಸ್ಸು ಕದಲದಂತೆ ನೋಡಿಕೊಳ್ಳುವುದೇ ವೈರಾಗ್ಯ. ಅಂದರೆ ನಮ್ಮ ಅನುಭವದಿಂದ ಮತ್ತು ಮತ್ತೊಬ್ಬರ ಅನುಭವದಿಂದ ಏಳುವ ಈ ಎರಡು ವಿಧದ ಕ್ರಿಯೋತ್ತೇಜಕ ಶಕ್ತಿಯನ್ನು ತಡೆದು, ಮನಸ್ಸು ಅದರ ವಶವಾಗದಂತೆ ನೋಡಿಕೊಳ್ಳುವುದೇ ವೈರಾಗ್ಯ. ಅದು ನನ್ನ ಸ್ವಾಧೀನಕ್ಕೆ ಒಳಪಡಬೇಕು, ನಾನು ಅದರ ಅಧೀನಕ್ಕೆ ಒಳಪಡುವುದಲ್ಲ. ಇಂತಹ ಮಾನಸಿಕ ಶಕ್ತಿಗೆ ವೈರಾಗ್ಯವೆಂದು ಹೆಸರು. ಮುಕ್ತಿಗೆ ಇರುವುದು ವೈರಾಗ್ಯದ ದಾರಿಯೊಂದೇ. 

\vspace{-0.2cm}

\begin{verse}
ತತ್ಪರಂ ಪುರುಷಖ್ಯಾತೇರ್ಗುಣವೈತೃಷ್ಣ್ಯಮ್​~॥ ೧೬~॥
\end{verse}

\vspace{-0.4cm}

\dsize{ಗುಣಗಳನ್ನೂ ಯಾವುದು ತ್ಯಜಿಸುವುದೊ ಅದು ಪರಮ ವೈರಾಗ್ಯ. ಇದು ಪುರುಷನ ನೈಜ ಸ್ವರೂಪದ ಜ್ಞಾನದಿಂದ ಬರುತ್ತದೆ. }

ಗುಣಗಳಾಸೆಯ ಕಡೆಗೂ ಮನಸ್ಸು ಯಾವಾಗ ಹೋಗುವುದಿಲ್ಲವೋ ಅದೇ ಪರಮ ವೈರಾಗ್ಯದ ಚಿಹ್ನೆ. ಮೊದಲು ನಾವು ಪುರುಷ ಎಂದರೇನು ಮತ್ತು ಗುಣಗಳೆಂದರೆ ಏನು ಎಂಬುದನ್ನು ತಿಳಿದುಕೊಳ್ಳಬೇಕು. ಯೋಗಶಾಸ್ತ್ರದ ಪ್ರಕಾರ ಪ್ರಕೃತಿಯು ಮೂರು ಗುಣಗಳಿಂದ ಕೂಡಿದೆ: ತಮಸ್ಸು, ರಜಸ್ಸು ಮತ್ತು ಸತ್ತ್ವ ಎಂಬುವು. ಅವು ಪ್ರಪಂಚದಲ್ಲಿ ಅಜ್ಞಾನ ಅಥವಾ ಜಡತೆ, ಆಕರ್ಷಣ ಅಥವಾ ವಿಕರ್ಷಣ ಮತ್ತು ಸಮತ್ವ ಎಂಬ ಮೂರು ರೂಪಗಳನ್ನು ತಾಳಿವೆ. ಪ್ರಕೃತಿಯಲ್ಲಿರುವ ಪ್ರತಿಯೊಂದು ವಸ್ತು ಮತ್ತು ಅದರ ಎಲ್ಲಾ ವಿಧದ ಅಭಿವ್ಯಕ್ತಿಯೂ ಈ ಮೂರು ಗುಣಗಳ ಮಿಶ್ರದಿಂದ ಆದುವು. ಸಾಂಖ್ಯರು ಪ್ರಕೃತಿಯನ್ನು ಹಲವು ತತ್ತ್ವಗಳಾಗಿ ವಿಭಾಗಿಸಿರುವರು. ಮಾನವನ ಆತ್ಮ (ಪುರುಷ) ಇದನ್ನೆಲ್ಲ ಮೀರಿರುವುದು, ಪ್ರಕೃತಿಯನ್ನು ಮೀರಿರುವುದು. ಇದು ಸ್ವಪ್ರಕಾಶವೂ, ಪರಿಶುದ್ಧವೂ, ಪೂರ್ಣವೂ ಆಗಿರುವುದು. ಜಗತ್ತಿನಲ್ಲಿ ಕಾಣುವ ಎಲ್ಲಾ ವಿಧದ ಚೇತನವೂ ಪ್ರಕೃತಿಯ ಮೇಲೆ ಬಿದ್ದಿರುವ ಆತ್ಮನ ಪ್ರತಿಬಿಂಬ. ಪ್ರಕೃತಿಯು ಜಡವಾದುದು. ಪ್ರಕೃತಿ ಎಂದರೆ ಅದರಲ್ಲಿ ಮನಸ್ಸೂ ಕೂಡ ಸೇರಿದೆ ಎಂಬುದನ್ನು ನೀವು ಮರೆಯಬಾರದು. ಮನಸ್ಸು ಪ್ರಕೃತಿಯಲ್ಲಿರುವುದು. ಆಲೋಚನೆ ಪ್ರಕೃತಿಯಲ್ಲಿರುವುದು. ಜಗತ್ತಿನಲ್ಲಿ ಆಲೋಚನೆಯಿಂದ ಹಿಡಿದು ಅತಿ ಸ್ಥೂಲವಾದ ಜಡವಸ್ತುವಿನವರೆವಿಗೂ ಎಲ್ಲವೂ ಪ್ರಕೃತಿಯ ಅಭಿವ್ಯಕ್ತಿ. ಈ ಪ್ರಕೃತಿಯೇ ಮಾನವ ಆತ್ಮವನ್ನು ಮುಚ್ಚಿರುವುದು. ಪ್ರಕೃತಿಯ ಈ ಮುಸುಕನ್ನು ತೆಗೆದಮೇಲೆ ಆತ್ಮವು ಸ್ವಯಂ ಜ್ಯೋತಿಯಿಂದ ಪ್ರಕಾಶಿಸುವುದು. ೧೫ನೇ ಸೂತ್ರದಲ್ಲಿ ಹೇಳಿರುವ ವೈರಾಗ್ಯ (ವಸ್ತುಗಳ ಅಥವಾ ಪ್ರಕೃತಿಯ ಸ್ವಾಧೀನ) ಆತ್ಮನ ಸ್ವಯಂ ಜ್ಯೋತಿಯ ಪ್ರಕಾಶಕ್ಕೆ ಬಹಳ ಸಹಾಯಕವಾಗಿರುವುದು. ಮುಂದಿನ ಸೂತ್ರವು ಯೋಗಿಯ ಗುರಿಯಾದ ಸಮಾಧಿ ಅಂದರೆ ಸಂಪೂರ್ಣ ಏಕಾಗ್ರತೆಯನ್ನು ವಿವರಿಸುವುದು. 

\vspace{-0.3cm}

\begin{verse}
ವಿತರ್ಕವಿಚಾರಾನನ್ದಾಸ್ಮಿತಾನುಗಮಾತ್​ ಸಂಪ್ರಜ್ಞಾತಃ~॥ ೧೭~॥
\end{verse}

\vspace{-0.4cm}

\dsize{ಯುಕ್ತಿ, ವಿವೇಕ, ಆನಂದ ಮತ್ತು ವಿಶೇಷಣರಹಿತ ಅಹಂಕಾರ–ಇವುಗಳನ್ನು ಅನುಸರಿಸಿ ಉಂಟಾಗುವ ಸಮಾಧಿಯೇ ಸಂಪ್ರಜ್ಞಾತ. }

\vspace{0.1cm}

ಸಮಾಧಿಯನ್ನು ಎರಡು ವಿಧವಾಗಿ ವಿಭಾಗಿಸುವರು: ಒಂದು ಸಂಪ್ರಜ್ಞಾತ ಮತ್ತೊಂದು ಅಸಂಪ್ರಜ್ಞಾತ. ಸಂಪ್ರಜ್ಞಾತ ಸಮಾಧಿಯಲ್ಲಿ ಪ್ರಕೃತಿಯನ್ನು ಸ್ವಾಧೀನಕ್ಕೆ ತರಬಲ್ಲ ಎಲ್ಲಾ ಶಕ್ತಿಗಳೂ ಬರುತ್ತವೆ. ಇದು ನಾಲ್ಕು ವಿಧವಾಗಿರುವುದು. ಮೊದಲನೆಯದು ಸವಿತರ್ಕ. ಆಗ ಮನಸ್ಸು ಒಂದು ವಸ್ತುವನ್ನು ಉಳಿದ ವಸ್ತುಗಳಿಂದ ಪ್ರತ್ಯೇಕಿಸಿ ಪದೇ ಪದೇ ಅದನ್ನೇ ಧ್ಯಾನಿಸುವುದು. ಸಾಂಖ್ಯರ ಇಪ್ಪತ್ತೈದು ತತ್ತ್ವಗಳಲ್ಲಿ ಧ್ಯಾನಕ್ಕೆ ಎರಡು ವಿಧದ ವಸ್ತುಗಳು ಯೋಗ್ಯವಾಗಿವೆ: ೧. ಪ್ರಕೃತಿಯ ಇಪ್ಪತ್ತನಾಲ್ಕು ಅಚೇತನ ತತ್ತ್ವಗಳು. ೨. ಚೇತನಾತ್ಮಕ ಪುರುಷ. ಆಗಲೇ ಹೇಳಿದ ಸಾಂಖ್ಯಸಿದ್ಧಾಂತದ ಮೇಲೆ ಈ ಭಾಗ ನಿಂತಿರುವುದು. ನಿಮಗೆ ಜ್ಞಾಪಕವಿರುವಂತೆ ಅಹಂಕಾರ, ಇಚ್ಛೆ, ಬುದ್ಧಿ ಇವುಗಳಿಗೆಲ್ಲಾ ಸಾಮಾನ್ಯ ತಳಹದಿಯೇ ಚಿತ್ತ. ಇದರಿಂದ ಉಳಿದವುಗಳೆಲ್ಲ ಆಗಿವೆ. ಚಿತ್ತವು ಪ್ರಕೃತಿಯ ಶಕ್ತಿಯನ್ನು ತೆಗೆದುಕೊಂಡು ಆಲೋಚನೆಯಂತೆ ಹೊರಗೆಡಹುವುದು. ವಸ್ತು ಮತ್ತು ಶಕ್ತಿಗಳಿಗೂ ಸಾಮಾನ್ಯವಾದುದು ಮತ್ತೊಂದು ಇರಬೇಕು. ಇದನ್ನೇ ಅವ್ಯಕ್ತವೆನ್ನುವುದು–ಸೃಷ್ಟಿಗೆ ಮುಂಚೆ ಇನ್ನೂ ಪ್ರಕಾಶಕ್ಕೆ ಬರದ ಪ್ರಕೃತಿ. ಮುಂದಿನ ಸೃಷ್ಟಿಯಲ್ಲಿ ಪುನಃ ವ್ಯಕ್ತವಾಗಲು ಇಡೀ ಪ್ರಕೃತಿಯು ಕಲ್ಪಾಂತರದಲ್ಲಿ ಈ ಅವ್ಯಕ್ತವನ್ನು ಸೇರುವುದು. ಇದರಾಚೆ ಚೇತನಾತ್ಮಕವಾದ ಪುರುಷನಿರುವುದು. ಜ್ಞಾನವೇ ಶಕ್ತಿ. ನಮಗೆ ಒಂದು ವಸ್ತುವಿನ ಸ್ವಭಾವ ತಿಳಿದೊಡನೆಯೆ ಅದರ ಮೇಲೆ ಸ್ವಾಧೀನ ಬರುವುದು. ಇದರಂತೆಯೆ, ಮನಸ್ಸು ಬೇರೆ ಬೇರೆ ವಸ್ತುಗಳ ಮೇಲೆ ಧ್ಯಾನಮಾಡಿದಂತೆ ಅವುಗಳ ಮೇಲೆ ಅಧಿಕಾರವನ್ನು ಪಡೆಯುತ್ತದೆ. ನಾವು ಎಲ್ಲಿ ಬಾಹ್ಯ ಜಡವಸ್ತುವಿನ ಮೇಲೆ ಧ್ಯಾನ ಮಾಡುತ್ತೇವೆಯೊ ಅದಕ್ಕೆ ಸವಿತರ್ಕವೆಂದು ಹೆಸರು. ವಿತರ್ಕ ಎಂದರೆ ಪ್ರಶ್ನೆ; ಸವಿತರ್ಕ ಎಂದರೆ ಪ್ರಶ್ನೆ ಸಹಿತ ವಸ್ತುವಿನ ಮೇಲೆ ಧ್ಯಾನಮಾಡುವುದು–ಅವುಗಳ ನಿಜಸ್ಥಿತಿ ಮತ್ತು ಅವುಗಳಲ್ಲಿರುವ ಶಕ್ತಿಯನ್ನು ನಮಗೆ ಕೊಡುವಂತೆ ಕೇಳಿಕೊಂಡಂತೆ. ಆ ಶಕ್ತಿಯನ್ನು ಪಡೆಯುವುದರಿಂದ ಮುಕ್ತಿಯಿಲ್ಲ. ಭೋಗವನ್ನುಗಳಿಸುವುದಕ್ಕೋಸ್ಕರ ಜನರು ಇವನ್ನ ಹುಡುಕುವರು. ಈ ಪ್ರಪಂಚದಲ್ಲಿ ಸಂತೋಷವೆನ್ನುವುದಿಲ್ಲ; ಅದಕ್ಕೆ ಹುಡುಕುವುದೆಲ್ಲ ವ್ಯರ್ಥವೆನ್ನುವುದು ಬಹಳ ಹಳೆಯ ಬುದ್ಧಿವಾದ. ಆದರೂ ಮನುಷ್ಯನು ಕಲಿಯಲು ಬಹಳ ಕಷ್ಟಪಡುವನು. ಅವನು ಇದನ್ನು ಕಲಿತ ಮೇಲೆ ಪ್ರಪಂಚದಿಂದ ದೂರವಾಗುವನು; ಮುಕ್ತನಾಗುವನು. ಈ ಗುಪ್ತ ಶಕ್ತಿಗಳ ಸ್ವಾಧೀನತೆಯೆಲ್ಲ ಮಾನವನ ಪ್ರಾಪಂಚಿಕತೆಯನ್ನು ಹೆಚ್ಚಿಸುವುದು. ಕೊನೆಗೆ ಇವು ನಮ್ಮ ದುಃಖವನ್ನು ಉದ್ದೀಪನಗೊಳಿಸುತ್ತವೆ. ವಿಜ್ಞಾನಿಯಂತೆ ಪತಂಜಲಿಯು ಇಂತಹ ಶಕ್ತಿಸಾಧ್ಯತೆಯನ್ನು ನಮಗೆ ತೋರಿದರೂ, ಈ ಶಕ್ತಿಯ ಆಸೆಗೆ ಬೀಳಬಾರದೆಂದು ಹೇಳುವ ಅವಕಾಶವನ್ನು ಒಮ್ಮೆಯೂ ಕಳೆದುಕೊಳ್ಳುವುದಿಲ್ಲ. 

\newpage

ಅದೇ ಧ್ಯಾನದಲ್ಲಿ ವಸ್ತುಗಳನ್ನು ಕಾಲದೇಶಗಳಿಂದ ಬೇರ್ಪಡಿಸಲು ಯತ್ನಿಸಿ, ಅವುಗಳ ನಿಜಸ್ಥಿತಿಯ ಮೇಲೆ ಧ್ಯಾನಿಸುವುದಕ್ಕೆ ನಿರ್ವಿತರ್ಕ ಎಂದು ಹೆಸರು. ಧ್ಯಾನ ಮತ್ತೊಂದು ಮೆಟ್ಟಿಲು ಮುಂದೆ ಹೋಗಿ ತನ್ಮಾತ್ರಗಳನ್ನು ಧ್ಯೇಯ ವಸ್ತುವಾಗಿ ಮಾಡಿಕೊಂಡು, ಅವುಗಳು ಕಾಲ ದೇಶದಲ್ಲಿರುವುದು ಎಂದು ಆಲೋಚಿಸಿದರೆ ಅದು ಸವಿಚಾರ. ಅದೇ ಧ್ಯಾನದಲ್ಲಿ ಕಾಲ ದೇಶವನ್ನು ಬೇರ್ಪಡಿಸಿ ತನ್ಮಾತ್ರದ ನಿಜಸ್ಥಿತಿಯ ಮೇಲೆ ಆಲೋಚಿಸುವುದು ನಿರ್ವಿಚಾರ. ಮುಂದಿನ ಹಂತದಲ್ಲಿ ಸ್ಥೂಲವಸ್ತು ಮತ್ತು ಸೂಕ್ಷ್ಮತನ್ಮಾತ್ರಗಳನ್ನು ತೊರೆದು ಅಂತಃಕರಣವೇ ಧ್ಯೇಯವಸ್ತುವಾಗುತ್ತದೆ. ಅಂತಃಕರಣವು ರಜಸ್ಸು ಮತ್ತು ತಮೋಗುಣಗಳಿಂದ ಮುಕ್ತವಾದುದೆಂದು ಯೋಚಿಸಿದಾಗ ಅದಕ್ಕೆ ಸಾನಂದ ಸಮಾಧಿ ಎಂದು ಹೆಸರು. ಮನಸ್ಸೇ ಧ್ಯಾನದ ವಸ್ತುವಾದಾಗ, ಧ್ಯಾನವು ಅತ್ಯಂತ ಪಕ್ವವೂ ಏಕಾಗ್ರವೂ ಆದಾಗ, ಸ್ಥೂಲ ಸೂಕ್ಷ್ಮ ವಸ್ತುಗಳ ಭಾವನೆಗಳೆಲ್ಲ ಹೊರಟು ಹೋದಾಗ, ಅಹಂಕಾರವು ಎಲ್ಲ ವಸ್ತುಗಳಿಂದ ಬೇರೆಯಾಗಿ ಅದರ ಸಾತ್ತ್ವಿಕ ಸ್ಥಿತಿ ಮಾತ್ರ ಉಳಿದಾಗ ಅದು ಸಾಸ್ಮಿತ ಸಮಾಧಿ. ಯಾರು ಇದನ್ನು ಪಡೆದಿರುವರೋ ಅವರು ವಿದೇಹಾವಸ್ಥೆಯನ್ನು ಪಡೆದಿರುತ್ತಾರೆ. ಸ್ಥೂಲ ದೇಹಧಾರಿ ತಾನಲ್ಲವೆಂದು ಅವನು ಯೋಚಿಸಬಹುದು. ಅಂದರೆ ತಾನು ಸೂಕ್ಷ್ಮದೇಹಿ ಎಂಬುದನ್ನು ನೆನೆಯಲೇಬೇಕು. ಈ ಸ್ಥಿತಿಯಲ್ಲಿರುವವರು ಗುರಿಯನ್ನು ಮುಟ್ಟದೆ ಪ್ರಕೃತಿಯಲ್ಲಿ ಲಯವಾದಾಗ ಅವರನ್ನು ಪ್ರಕೃತಿಲಯಿಗಳು ಎನ್ನುವರು. ಆದರೆ ಯಾರು ಇಲ್ಲಿ ಕೂಡ ನಿಲ್ಲುವುದಿಲ್ಲವೋ ಅವರೇ ಗುರಿಯನ್ನು ಮುಟ್ಟುವರು. 

\vspace{-0.42cm}

\begin{verse}
ವಿರಾಮ ಪ್ರತ್ಯಯಾಭ್ಯಾಸಪೂರ್ವಃ ಸಂಸ್ಕಾರಶೇಷೋನ್ಯ~॥ ೧೮~॥
\end{verse}

\vspace{-0.57cm}

\dsize{ಎಲ್ಲಾ ಮಾನಸಿಕ ಕ್ರಿಯೆಗಳನ್ನು ನಿಲ್ಲಿಸುವ ನಿರಂತರ ಅಭ್ಯಾಸದ ಮೂಲಕ, ಚಿತ್ತವು ಅವ್ಯಕ್ತ ಸಂಸ್ಕಾರಗಳನ್ನು ಮಾತ್ರ ಯಾವಾಗ ಉಳಿಸಿಕೊಂಡಿರುವುದೋ ಆಗ ಮತ್ತೊಂದು ಸಮಾಧಿಯನ್ನು ಹೊಂದಬಹುದು. }

\vspace{0.05cm}

ಇದೇ ನಮಗೆ ಸ್ವಾತಂತ್ರ್ಯವನ್ನು ಕೊಡುವ ನಿರ್ದೋಷವಾದ ಪ್ರಜ್ಞಾತೀತವಾದ ಅಸಂಪ್ರ\break ಜ್ಞಾತ ಸಮಾಧಿ. ಮೊದಲನೆ ಸ್ಥಿತಿ ನಮಗೆ ಸ್ವಾತಂತ್ರ್ಯವನ್ನು ಕೊಡುವುದಿಲ್ಲ. ಆತ್ಮವನ್ನು ಬಂಧನದಿಂದ ಪಾರುಮಾಡುವುದಿಲ್ಲ. ಮನುಷ್ಯನು ಸಿದ್ಧಿಗಳನ್ನು ಪಡೆಯಬಹುದು. ಆದರೂ ಕೂಡ ಅವನು ಮತ್ತೊಮ್ಮೆ ಹಿಂದೆ ಬೀಳಬಹುದು. ಜೀವನು ಪ್ರಕೃತಿಯ ಪಾಶದಿಂದ ಮುಕ್ತನಾಗುವವರೆಗೆ ಅವನು ಅಪಾಯದಿಂದ ಪಾರಾಗನು. ಮಾರ್ಗವು ಬಹಳ ಸುಲಭವಾಗಿ ಕಂಡರೂ ಹಾಗೆ ಮಾಡುವುದು ಬಹಳ ಕಷ್ಟ. ಮನಸ್ಸಿನ ಮೇಲೆ ಧ್ಯಾನ ಮಾಡುವುದೇ ದಾರಿ, ಯಾವುದಾದರೂ ಆಲೋಚನೆ ಬಂದಾಗ ಅದವನ್ನು ಹೊಡೆದು ಅಟ್ಟಬೇಕು. ಮನಸ್ಸಿ\break ನೊಳಗೆ ಬೇರಾವ ಆಲೋಚನೆಯೂ ಬರದಂತೆ ಮಾಡಬೇಕು. ಅದನ್ನು ಒಂದು ಶೂನ್ಯ ಪ್ರದೇಶವನ್ನಾಗಿ ಮಾಡಬೇಕು. ಇದನ್ನು ನಾವು ನಿಜವಾಗಿಯೂ ಸಾಧಿಸಲು ಸಾಧ್ಯವಾದಾಗ, ತಕ್ಷಣವೇ ನಮಗೆ ಮುಕ್ತಿ ಲಭಿಸುತ್ತದೆ. ಸಾಧ್ಯವಾದಷ್ಟು ಅಭ್ಯಾಸವಿಲ್ಲದೇ ಜನರು ಮನಸ್ಸನ್ನು ಶೂನ್ಯಮಾಡಲು ಯತ್ನಿಸಿದಾಗ, ಮನಸ್ಸನ್ನು ಜಡಮಾಡಿ ಸೋಮಾರಿಯನ್ನಾಗಿ ಮಾಡುವ, ಅಜ್ಞಾನದ ಮೂಲವಾದ ತಮಸ್ಸಿನಿಂದ ಆವೃತ್ತರಾಗಿ, ತಾವು ಮನಸ್ಸನ್ನು ಶೂನ್ಯ ಮಾಡುತ್ತಿರು\-ವೆವು ಎಂಬ ತಪ್ಪು ಅಭಿಪ್ರಾಯವನ್ನು ಇಟ್ಟುಕೊಳ್ಳುವರು. ಅದನ್ನು ನಿಜವಾಗಿಯೂ ಸಾಧಿಸಬೇಕಾದರೆ, ನಾವು ಅದ್ಭುತವಾದ ಶಕ್ತಿಯನ್ನು ಮತ್ತು ಪ್ರಚಂಡ ನಿಗ್ರಹವನ್ನು ಪ್ರದರ್ಶಿಸಬೇಕಾಗುತ್ತದೆ. ಈ ಅತೀಂದ್ರಿಯ ಅಸಂಪ್ರಜ್ಞಾತ ಅವಸ್ಥೆಗೆ ತಲುಪಿದಾಗ ಸಮಾಧಿಯು ನಿರ್ಬೀಜವಾಗುವುದು. ಹಾಗೆಂದರೆ ಏನು ಅರ್ಥ? ಪ್ರಜ್ಞೆಯಿಂದ ಕೂಡಿದ ಏಕಾಗ್ರತೆಯಲ್ಲಿ ಮನಸ್ಸು ಚಿತ್ತದ ಅಲೆಗಳನ್ನು ಮಾತ್ರ ನಿಗ್ರಹಿಸಲು ಸಾಧ್ಯ. ಅಲ್ಲಿ ಭಾವನೆಯ ಅಲೆಗಳು ಸಂಸ್ಕಾರದ ಅವಸ್ಥೆಯಲ್ಲಿ ಇರುವುವು. ಕಾಲ ಬಂದಾಗ ಈ ಸಂಸ್ಕಾರಗಳೇ (ಬೀಜಗಳೇ) ಪುನಃ ಅಲೆಯಾಗುವುವು. ಆದರೆ ಈ ಸಂಸ್ಕಾರಗಳನ್ನೆಲ್ಲ ನಾಶಮಾಡಿದ ಮೇಲೆ, ಬಹುಪಾಲು ಮನಸ್ಸನ್ನೇ ನಾಶಮಾಡಿದ ಮೇಲೆ, ಸಮಾಧಿಯು ನಿರ್ಬೀಜವಾಗುವುದು. ಆಗ ಜನನ ಮರಣಗಳೆಂಬ ಸಂಸಾರದ ಮರವನ್ನು ಹುಟ್ಟಿಸುವ ಬೀಜಗಳಾವವೂ ಮನಸ್ಸಿನಲ್ಲಿ ಇರುವುದಿಲ್ಲ. 

ಮನಸ್ಸಿಲ್ಲದ, ಜ್ಞಾನವಿಲ್ಲದ, ಸ್ಥಿತಿ ಎಂತಹುದು ಎಂದು ಕೇಳಬಹುದು. ನೀವು ಯಾವುದನ್ನು ಜ್ಞಾನವೆನ್ನುತ್ತೀರೋ ಅದು ಜ್ಞಾನತೀತಾವಸ್ಥೆಗಿಂತ ಕೆಳಗಿನದು. ಎರಡು ಅತಿರೇಕಗಳೂ ಕೂಡ ನೋಡಲು ಒಂದೆಯಾಗಿ ತೋರುವುದು ಎಂಬುದನ್ನು ನೀವು ಮರೆಯಬಾರದು. ಆಕಾಶದಲ್ಲಿ ಬಹಳ ಕಡಮೆ ಸ್ಪಂದನವನ್ನು ಕತ್ತಲೆಯೆಂದು ತೆಗೆದುಕೊಂಡರೆ, ಅದರಲ್ಲಿ ಮಧ್ಯದ ಸ್ಥಿತಿಯು ಬೆಳಕಾಗುವುದು; ತೀವ್ರ ಸ್ಪಂದನವಿರುವುದು ಪುನಃ ಕತ್ತಲಾಗುವುದು. ಅದರಂತೆಯೇ ಅಜ್ಞಾನವು ಬಹಳ ಕೆಳಗಿನ ಸ್ಥಿತಿ; ಜ್ಞಾನವು ಮಧ್ಯದ ಸ್ಥಿತಿ. ಜ್ಞಾನಾತೀತವಾದ ಸ್ಥಿತಿಯೇ ಶ್ರೇಷ್ಠವಾದ ಸ್ಥಿತಿ. ಅತಿರೇಕಗಳೆರಡೂ ಒಂದರಂತೆಯೇ ಕಾಣುವುವು. ಜ್ಞಾನವೆನ್ನುವುದೇ ಸತ್ಯವಲ್ಲ. ಅದು ಒಂದು ಹೊಸದಾಗಿ ತಯಾರಿಸಲ್ಪಟ್ಟದ್ದು. ಒಂದು ಸಂಯೋಗ. ಅದೇ ಸತ್ಯವಲ್ಲ. 

ಈ ಮೇಲುತರದ ಏಕಾಗ್ರತೆಯನ್ನು ಪ್ರತಿದಿನವೂ ಅಭ್ಯಾಸ ಮಾಡುವುದರಿಂದಾಗುವ ಪ್ರಯೋಜನವೇನು? ಇದರಿಂದ ಹಳೆಯ ರಾಜಸಿಕ ಮತ್ತು ತಾಮಸಿಕ ಸಂಸ್ಕಾರಗಳೆಲ್ಲವೂ ನಾಶವಾಗುವುವು. ಅದರ ಜೊತೆಗೆ ಒಳ್ಳೆಯ ಸಂಸ್ಕಾರಗಳೂ ಕೂಡ ನಾಶವಾಗುವುವು. ಚಿನ್ನವನ್ನು ಅದರಲ್ಲಿರುವ ಕೊಳೆ ಮತ್ತು ಇನ್ನೂ ಇತರ ಲೋಹಗಳಿಂದ ಬೇರ್ಪಡಿಸುವುದಕ್ಕಾಗಿ ಉಪಯೋಗಿಸುವ ರಾಸಾಯನಿಕ ದ್ರವದಂತೆ ಇದು. ಚಿನ್ನದ ಅದುರನ್ನು ಇದರೊಂದಿಗೆ ಕಾಯಿಸಿದಾಗ ಕೊಳೆಯೊಂದಿಗೆ ಬೆರೆಸಿದ ರಸಾಯನ ದ್ರವ್ಯವೂ ಉರಿದು ಹೋಗುವುದು. ಇದರಂತೆಯೇ ಸತತವೂ ನಿಗ್ರಹಿಸುವ ಶಕ್ತಿಯು ಹಿಂದಿನ ಹಳೆಯ ಕೆಟ್ಟ ಸಂಸ್ಕಾರಗಳನ್ನೆಲ್ಲ ನಾಶಮಾಡುವುದು. ಜೊತೆಗೆ ಒಳ್ಳೆಯ ಸಂಸ್ಕಾರಗಳೂ ನಾಶವಾಗುವುವು. ಒಳ್ಳೆಯ ಮತ್ತು ಕೆಟ್ಟ ಸಂಸ್ಕಾರಗಳಾವುದರಿಂದಲೂ ಮಲಿನವಾಗದೆ ಸ್ವಯಂಜ್ಯೋತಿರೂಪನಾದ ಸರ್ವವ್ಯಾಪಿಯಾದ, ಸರ್ವಶಕ್ತನಾದ, ಸರ್ವಜ್ಞನಾದ ಆತ್ಮನು ತನ್ನ ಸಹಜಾವಸ್ಥೆಯಲ್ಲಿ ಉಳಿಯುವನು. ಆಗ ತನಗೆ ಹುಟ್ಟು ಸಾವುಗಳಿಲ್ಲವೆನ್ನುವುದು ಮಾನವನಿಗೆ ಗೊತ್ತಾಗುವುದು. ಸ್ವರ್ಗ ನರಕಗಳ ಆವಶ್ಯಕತೆಯೇ ಅವನಿಗೆ ಇರುವುದಿಲ್ಲ. ಅವನು ಎಂದೂ ಬಂದವನಲ್ಲ ಮತ್ತು ಹೋಗುವವನೂ ಇಲ್ಲ ಎಂಬುದು ಗೊತ್ತಾಗುವುದು. ಚಲಿಸುತ್ತ ಇದ್ದುದು ಪ್ರಕೃತಿ. ಇದರ ನೆರಳು ಆತ್ಮನ ಮೇಲೆ ಬೀಳುತ್ತಿತ್ತು. ಕನ್ನಡಿಯಿಂದ ಪ್ರತಿಬಿಂಬಿಸುವ ಬೆಳಕು ಗೋಡೆಯ ಮೇಲೆ ಬಿದ್ದು ಚಲಿಸುತ್ತಿದ್ದರೆ, ಗೋಡೆ ತಾನು ಚಲಿಸುತ್ತಿರುವೆನೆಂದು ಬುದ್ಧಿಯಿಲ್ಲದೆ ತಿಳಿದುಕೊಂಡಂತೆ ನಾವು. ಇದರಂತೆಯೇ ಅನೇಕ ರೂಪಗಳನ್ನು ಧರಿಸಿ ಚಲಿಸುತ್ತಿರುವುದು ಚಿತ್ತ. ಈ ಹಲವು ರೂಪಗಳೇ ನಾವೆಂದು ಭ್ರಮೆ ಪಡುತ್ತೇವೆ. ಈ ಭ್ರಾಂತಿ ಮಾಯವಾಗುವುದು. ಮುಕ್ತಾತ್ಮನು ಅಪ್ಪಣೆ ಮಾಡಿದಾಗ (ಪ್ರಾರ್ಥಿಸುವುದಲ್ಲ, ಬೇಡುವುದಲ್ಲ) ತನ್ನ ಬಯಕೆಗಳೆಲ್ಲ ತಕ್ಷಣವೇ ಈಡೇರುವುವು, ತನಗೆ ಏನು ಬೇಕೋ ಅದನ್ನು ಮಾಡಲು ಸಾಧ್ಯವಾಗುತ್ತದೆ. 

ಸಾಂಖ್ಯತತ್ತ್ವದಲ್ಲಿ ದೇವರಿಲ್ಲ. ಈ ಪ್ರಪಂಚಕ್ಕೆ ದೇವರಿರಲಾರ. ಏನಾದರೂ ಹಾಗೆ ಒಬ್ಬನು ಇದ್ದಿದ್ದರೆ ಆತ್ಮನಾಗಿರಬೇಕು. ಆತ್ಮನು ಬಂಧನದಲ್ಲಿರಬೇಕು ಅಥವಾ ಮುಕ್ತನಾಗಿರಬೇಕು. ಪ್ರಕೃತಿಯ ಬಂಧನಕ್ಕೊಳಗಾದ ಅಥವಾ ಅದರ ಅಧೀನಕ್ಕೆ ಒಳಪಟ್ಟ ಜೀವನು ಹೇಗೆ ಸೃಷ್ಟಿಸಬಲ್ಲನು? ಜೀವನೇ ಮತ್ತೊಂದರ ಅಧೀನದಲ್ಲಿರುವನು. ಅದಲ್ಲದೆ ಮುಕ್ತನಾಗಿದ್ದರೆ ಇದೆಲ್ಲವನ್ನು ಏತಕ್ಕೆ ಸೃಷ್ಟಿಸಬೇಕಿತ್ತು? ಅವನಿಗೆ ಯಾವ ಆಸೆಗಳೂ ಇಲ್ಲ. ಆದಕಾರಣ ಸೃಷ್ಟಿಸುವ ಆವಶ್ಯಕತೆಯೇ ಇರುವುದಿಲ್ಲ. ಎರಡನೆಯದಾಗಿ ಈಶ್ವರಸಿದ್ಧಾಂತ ಅಷ್ಟೊಂದು ಆವಶ್ಯಕವಲ್ಲ. ಪ್ರಕೃತಿಯು ಎಲ್ಲವನ್ನೂ ವಿವರಿಸಬಲ್ಲದು. ದೇವರಿಂದ ಪ್ರಯೋಜನವೇನು? ಅನೇಕ ಜನರು ಸಂಪೂರ್ಣಾವಸ್ಥೆಯನ್ನು ಸಮೀಪಿಸಿದ್ದರೂ ಪೂರ್ತಿ ಐಕ್ಯರಾಗುವುದಿಲ್ಲ. ಏಕೆಂದರೆ ಅವರು ಎಲ್ಲಾ ಸಿದ್ಧಿಗಳನ್ನೂ ಸಂಪೂರ್ಣ ತ್ಯಾಗ ಮಾಡಲಾರರು. ಪ್ರಕೃತಿಯ ಪ್ರಭುಗಳಾಗುವುದಕ್ಕಾಗಿ ಅವರ ಮನಸ್ಸು ಕೆಲಕಾಲ ಪ್ರಕೃತಿಯಲ್ಲಿ ಲೀನವಾಗಿರುವುದು. ಅಂತಹ ದೇವರುಗಳಿರುತ್ತಾರೆ ಎಂದು ಕಪಿಲರು ಹೇಳುತ್ತಾರೆ. ನಾವೆಲ್ಲರೂ ಕೂಡ ಅಂತಹ ದೇವರುಗಳಾಗುತ್ತೇವೆ. ಸಾಂಖ್ಯರ ಅಭಿಪ್ರಾಯದಲ್ಲಿ ನಿಜವಾಗಿಯೂ ವೇದದಲ್ಲಿ ಬರುವ ದೇವತೆಗಳೆಂದರೆ ಇಂತಹ ಮುಕ್ತ ಜೀವಿಗಳು. ಇವರನ್ನು ಮೀರಿದ ನಿತ್ಯಸ್ವತಂತ್ರನೂ ಆನಂದಮಯನೂ ಆದ ಒಬ್ಬ ಸೃಷ್ಟಿಸುವವನಿಲ್ಲ. ಇದು ಸಾಂಖ್ಯರ ವಾದಸರಣಿ. ಆದರೆ ಯೋಗಿಗಳು, “ಹಾಗಲ್ಲ ದೇವರೊಬ್ಬನು ಇರುವನು. ಎಲ್ಲಾ ಜೀವಿಗಳಿಂದಲೂ ಪ್ರತ್ಯೇಕವಾದ ಪರಮಾತ್ಮನಿರುವನು. ಅವನೇ ಸೃಷ್ಟಿಯ ನಿತ್ಯ ವಿಧಾತನು, ನಿತ್ಯಸ್ವತಂತ್ರನು. ಗುರುಗಳ ಗುರುವಾತ” ಎನ್ನುವರು. ಸಾಂಖ್ಯರು ಯಾರನ್ನು ಪ್ರಕೃತಿಯನ್ನು ಲೀನವಾಗುತ್ತಾರೆ ಎಂದು ಹೇಳುವರೋ ಅಂತಹವರೂ ಇರುತ್ತಾರೆ ಎಂದು ಯೋಗಿಗಳು ಒಪ್ಪುತ್ತಾರೆ. ಅವರು ಮುಕ್ತಿಯನ್ನು ಇನ್ನೂಗಳಿಸದವರು. ಗುರಿಯನ್ನು ಸೇರುವುದಕ್ಕೆ ಅವರಿಗೆ ಸ್ವಲ್ಪ ಅಡ್ಡಿಯಾದರೂ, ಪ್ರಪಂಚದ ಕೆಲವು ಭಾಗಕ್ಕೆ ಒಡೆಯರಾಗಿ ಉಳಿಯುವರು. 

\vspace{-0.1cm}

\begin{verse}
ಭವ–ಪ್ರತ್ಯಯೋ ವಿದೇಹ – ಪ್ರಕೃತಿಲಯಾನಾಮ್​~॥ ೧೯~॥
\end{verse}

\vspace{-0.4cm}

\dsize{(ಇದೇ ಸಮಾಧಿಯು ಪರಾವೈರಾಗ್ಯದಿಂದ ಕೂಡಿರದಿದ್ದರೆ) ದೇವತೆಗಳು ಪ್ರಕೃತಿಲಯರಿಗೆ ಪುನಃ ಆವಿರ್ಭವಿಸಲು ಕಾರಣವಾಗುತ್ತದೆ. }

\vspace{0.2cm}

ಹಿಂಧೂಧರ್ಮದಲ್ಲಿ ಬರುವ ದೇವತೆಗಳು ಕೆಲವು ಅಧಿಕಾರವನ್ನು ವಹಿಸಿಕೊಂಡಿರುವ\break ವರು ಹೆಸರು, ಅನೇಕ ಜೀವಿಗಳೂ ಆ ಅಧಿಕಾರವನ್ನು ಒಬ್ಬರಾದ ಮೇಲೆ ಒಬ್ಬರು ವಹಿಸುತ್ತಾರೆ. ಒಬ್ಬರೂ ಪರಿಪೂರ್ಣರಲ್ಲ. 

\newpage


\begin{verse}
ಶ್ರದ್ಧಾ–ವೀರ್ಯ–ಸ್ಮೃತಿ–ಸಮಾಧಿ–ಪ್ರಜ್ಞಾ–ಪೂರ್ವಕ ~ಇತರೇಷಾಮ್​~॥ ೨೦~॥
\end{verse}

\vspace{-0.4cm}

\dsize{ಇತರರಿಗೆ ಸಮಾಧಿಯು, ಶ್ರದ್ಧೆ, ವೀರ್ಯ, ಸ್ಮೃತಿ, ಏಕಾಗ್ರತೆ ಮತ್ತು ನಿತ್ಯವಸ್ತು ವಿವೇಕಗಳಿಂದ\break ಲಭಿಸುತ್ತದೆ. }

\vspace{0.1cm}

ದೇವತೆಗಳ ಪದವಿಯನ್ನು ಮತ್ತು ಆಯಾ ಕಲ್ಪಗಳನ್ನು ಆಳುವ ಅಧಿಕಾರವನ್ನು ಕೂಡ ತಿರಸ್ಕರಿಸಿದವರು ಇವರು. ಇವರಿಗೆ ಮೋಕ್ಷ ಸಿದ್ಧಿಸುವುದು. 

\vspace{-0.1cm}

\begin{verse}
ತೀವ್ರಸಂವೇಗಾನಾಮಾಸನ್ನಃ~॥ ೨೧~॥
\end{verse}

\vspace{-0.4cm}

\dsize{ಯಾರು ತೀವ್ರಸಾಧಕರೋ ಅವರಿಗೆ ಜಯವು ಬೇಗ ಲಭಿಸುವುದು. }

\vspace{-0.1cm}

\begin{verse}
ಮೃದುಮಧ್ಯಾಧಿಮಾತ್ರತ್ವಾತ್ತತೋಽಪಿ ವಿಶೇಷಃ~॥ ೨೨~॥
\end{verse}

\vspace{-0.4cm}

\dsize{ಮಂದ, ಮಧ್ಯ, ತೀವ್ರ ಪಥಗಳನ್ನು ಅನುಸರಿಸಿದಂತೆ ಅವರ ಜಯವೂ ಕೂಡ ಬದಲಾಯಿಸುವುದು. }

\vspace{-0.1cm}

\begin{verse}
ಈಶ್ವರಪ್ರಣಿಧಾನಾದ್ವಾ~॥ ೨೩~॥
\end{verse}

\vspace{-0.35cm}

\dsize{ಅಥವಾ ಈಶ್ವರನ ಮೇಲೆ ಇಡುವ ಭಕ್ತಿಯ ಮೂಲಕ. }

\vspace{-0.1cm}

\begin{verse}
ಕ್ಲೇಶಕರ್ಮವಿಪಾಕಾಶಯ್ಯೆರಪರಾಮೃಷ್ಟಃ ಪುರುಷವಿಶೇಷ ಈಶ್ವರಃ~॥ ೨೪~॥
\end{verse}

\vspace{-0.4cm}

\dsize{ಈಶ್ವರನು, ದುಃಖ, ಕರ್ಮ, ಅದರ ಫಲ ಮತ್ತು ಆಸೆಯಿಂದ ದೂರವಾದ ವಿಶೇಷ ಪುರುಷನು. }

\vspace{0.1cm}

ಪತಂಜಲಿಯ ಯೋಗಸಿದ್ಧಾಂತವು ಸಾಂಖ್ಯ ಸಿದ್ಧಾಂತದ ಮೇಲೆ ನಿಂತಿರುವುದು. ಆದರೆ ಸಾಂಖ್ಯರಲ್ಲಿ ಈಶ್ವರನಿಗೆ ಸ್ಥಾನವಿಲ್ಲ. ಯೋಗದಲ್ಲಿ ಈಶ್ವರನಿಗೆ ಸ್ಥಾನವಿದೆ ಎಂಬುದನ್ನು ನೆನಪಿನಲ್ಲಿಡಬೇಕು. ಆದರೆ ಯೋಗಿಗಳು ದೇವರ ವಿಚಾರವಾಗಿ ಸೃಷ್ಟಿ ಮುಂತಾದ ಭಾವನೆ\break ಗಳನ್ನು ಹೇಳುವುದಿಲ್ಲ. ಇಲ್ಲಿ ಈಶ್ವರ ಎಂದರೆ ಅವನು ಸೃಷ್ಟಿಕರ್ತನಲ್ಲ. ವೇದಗಳ ಪ್ರಕಾರ ಈಶ್ವರನು ವಿಶ್ವವನ್ನು ಸೃಷ್ಟಿಸುತ್ತಾನೆ. ವಿಶ್ವದಲ್ಲಿ ಒಂದು ಸಾಮರಸ್ಯ ಇರುವುದರಿಂದ ಅದು ಒಂದು ಇಚ್ಛೆಯ ಕೈವಾಡವಿರಬೇಕು. ಯೋಗಿಗಳು ದೇವರನ್ನು ಸಮರ್ಥಿಸಲು ಯತ್ನಿಸುವರು. ಆದರೆ ಆ ಸಿದ್ಧಾಂತಕ್ಕೆ ತಮ್ಮದೇ ಒಂದು ವಿಚಿತ್ರ ರೀತಿಯಿಂದ ಬರುವರು. ಅವರು ಹೀಗೆ ಹೇಳುತ್ತಾರೆ:

\vspace{-0.25cm}

\begin{verse}
ತತ್ರ ನಿರತಿಶಯಂ ಸರ್ವಜ್ಞಬೀಜಮ್​~॥ ೨೫~॥
\end{verse}

\vspace{-0.4cm}

%%%% 408
\dsize{ಯಾವ ಸರ್ವಜ್ಞತೆಯು ಉಳಿದವರಲ್ಲಿ ಬೀಜರೂಪವಾಗಿದೆಯೇ ಅದು ಅವನಲ್ಲಿ ಅನಂತವಾಗುವುದು. }

ಮನಸ್ಸು ಯಾವಾಗಲೂ ಎರಡು ಅತಿರೇಕಗಳ ಪ್ರಯಾಣ ಮಾಡಬೇಕು. ನೀವು ಪರಿಮಿತಿಯಿಂದ ಕೂಡಿದ ಅವಕಾಶವನ್ನು ಯೋಚಿಸಬಹುದು. ಆದರೆ ಈ ಮಿತಭಾವನೆ ನಿಮಗೆ ಆಕಾಶಕ್ಕೆ ಸಂಬಂಧಿಸಿದಂತೆ ಅಮಿತಭಾವನೆಯನ್ನು ಕೊಡುವುದು. ನಿಮ್ಮ ಕಣ್ಣನ್ನು ಮುಚ್ಚಿಕೊಂಡು ಮಿತಾಕಾಶವನ್ನು ಕಲ್ಪಿಸಿಕೊಳ್ಳುತ್ತಿರುವಾಗಲೇ ಅಪರಿಮಿತ ಆಕಾಶ ಅದಕ್ಕೆ ಹಿನ್ನೆಲೆಯಾಗುವುದು. ಇದರಂತೆಯೇ ಕಾಲವೂ ಕೂಡ. ನೀವು ಒಂದು ಕ್ಷಣವನ್ನು ಕುರಿತು ಯೋಚಿಸಿ. ಇದರೊಂದಿಗೆ ಅನಂತ ಕಾಲದ ಅನುಭವವೂ ಕೂಡ ಆಗುತ್ತದೆ. ಜ್ಞಾನವೂ ಕೂಡ ಇದರಂತೆ. ಜ್ಞಾನವು ಮಾನವನಲ್ಲಿ ಬೀಜ ಪ್ರಾಯವಾಗಿದೆ. ಆದರೆ ನಾವು ಇದನ್ನು ಆಲೋಚಿಸಬೇಕಾದರೆ ಸುತ್ತಲೂ ಅಪರಿಮಿತ ಜ್ಞಾನವನ್ನು ಕಲ್ಪಿಸಿಕೊಳ್ಳಬೇಕು. ಆದುದರಿಂದ ನಮ್ಮ ಮನಸ್ಸಿನ ಸ್ಥಿತಿಯೇ ಅನಂತ ಜ್ಞಾನವಿರುವುದನ್ನು ನಮಗೆ ತೋರುವುದು. ಯೋಗಿಗಳು ಇಂತಹ ಅಪರಿಮಿತ ಜ್ಞಾನವನ್ನೇ ದೇವರೆಂದು ಕರೆಯುವುದು.

\vspace{-0.2cm}

\begin{verse}
ಸ ಪೂರ್ವೇಷಾಮಪಿ ಗುರುಃ ಕಾಲೇನಾನವಚ್ಛೇದಾತ್​~॥ ೨೬~॥
\end{verse}

\vspace{-0.45cm}

\dsize{ಈಶ್ವರನು ಕಾಲದಿಂದ ಬಾಧಿತನಾಗದ ಕಾರಣ ಹಿಂದಿನ ಗುರುಗಳಿಗೂ ಗುರುವಾಗಿರುವನು. }

\vspace{0.2cm}

ಎಲ್ಲಾ ಜ್ಞಾನಗಳೂ ಕೂಡ ನಮ್ಮಲ್ಲಿರುವುದು ಎಂಬುದೇನೋ ನಿಜ. ಆದರೆ ಸುಪ್ತವಾಗಿರುವ ಜ್ಞಾನವನ್ನು ಮತ್ತೊಂದು ಜ್ಞಾನದಿಂದ ಹೊರಗೆ ತರಬೇಕಾಗಿದೆ. ಅದನ್ನು ತಿಳಿದುಕೊಳ್ಳುವ ಸಾಮರ್ಥ್ಯವನ್ನು ನಮ್ಮಲ್ಲಿದ್ದರೂ ಅದನ್ನು ನಾವು ವ್ಯಕ್ತಪಡಿಸಬೇಕಾಗಿದೆ. ಮತ್ತೊಂದು ಜ್ಞಾನದ ಮೂಲಕ ಈ ಜ್ಞಾನವನ್ನು ವ್ಯಕ್ತಮಾಡಲು ಸಾಧ್ಯವೆನ್ನುತ್ತಾನೆ ಯೋಗಿ. ಸತ್ತ ಜಡವಸ್ತು ಎಂದಿಗೂ ನಮ್ಮ ಜ್ಞಾನವನ್ನು ಹೊರಗೆ ತರುವುದಿಲ್ಲ. ಜ್ಞಾನದ ಕ್ರಿಯೆಯೇ ಜ್ಞಾನವನ್ನು ಹೊರಗೆ ತರಬಲ್ಲದು. ನಮ್ಮಲ್ಲಿರುವುದನ್ನು ವ್ಯಕ್ತದ ಬೆಳಕಿಗೆ ತರುವುದಕ್ಕಾಗಿ ತಿಳಿದಂತಹ ಜ್ಞಾನಿಗಳು ನಮ್ಮೊಡನಿರಬೇಕು. ಅದಕ್ಕಾಗಿಯೇ ಇಂತಹ ಜ್ಞಾನಿಗಳು ಯಾವಾಗಲೂ ಅತ್ಯಾವಶ್ಯಕವಾಗಿಬೇಕು. ಪ್ರಪಂಚದಲ್ಲಿ ಅವರೆಂದಿಗೂ ಇಲ್ಲದೇ ಇರಲಿಲ್ಲ. ಅವರಿಲ್ಲದೆ ಯಾವ ಜ್ಞಾನವೂ ನಮಗೆ ಬರುವುದಿಲ್ಲ. ದೇವರೆ ಎಲ್ಲಾ ಗುರುಗಳ ಗುರು. ಈ ಗುರುಗಳು ಎಷ್ಟೇ ದೊಡ್ಡವರಾಗಲೀ, ದೇವತೆಗಳಾಗಿರಲೀ, ದೇವದೂತರಾಗಲೀ, ಎಲ್ಲರೂ ಕಾಲವಶರಾಗಿರುವರು. ಆದರೆ ಈಶ್ವರನು ಇದನ್ನು ಮೀರಿದವನು. ಈ ಯೋಗಿಗಳ ಎರಡು ವಿಶೇಷ ಸಿದ್ಧಾಂತಗಳು: ಮೊದಲನೆಯದು ಪರಿಮಿತಿಯನ್ನು ಹೊಂದಿದ ವಸ್ತುವನ್ನು ನಾವು ಯೋಚಿಸಬೇಕಾದರೆ ಅಪರಿಮಿತವಾದುದನ್ನು ನಾವು ಯೋಚಿಸಲೇಬೇಕು. ಈ ಎರಡು ಮಾನಸಿಕ ಗ್ರಹಣಗಳಲ್ಲಿ ಒಂದು ನಿಜವಾದರೆ ಮತ್ತೊಂದು ನಿಜವಾಗಲೇಬೇಕು. ಏಕೆಂದರೆ ಮಾನಸಿಕ ಜ್ಞಾನದ ದೃಷ್ಟಿಯಿಂದ ಎರಡೂ ಒಂದೇ ದರ್ಜೆಗೆ ಸೇರಿದುವು. ಮನುಷ್ಯನಿಗೆ ಅಲ್ಪ ಜ್ಞಾನವಿದೆ ಎಂಬ ವಿಷಯವೇ, ದೇವರಿಗೆ ಅನಂತಜ್ಞಾನವಿದೆ ಎಂಬುದನ್ನು ತೋರುವುದು. ನಾನು ಒಂದನ್ನು ಸ್ವೀಕರಿಸಿದರೆ ಮತ್ತೊಂದನ್ನು ಏತಕ್ಕೆ ಸ್ವೀಕರಿಸಬಾರದು?\break ಎರಡನ್ನೂ ಸ್ವೀಕರಿಸಬೇಕು. ತಪ್ಪಿದರೆ ಎರಡನ್ನೂ ತಿರಸ್ಕರಿಸಬೇಕೆಂದು ಯುಕ್ತಿ ಬೋಧಿಸುವುದು. ಅಲ್ಪಜ್ಞಾನವುಳ್ಳ ಮನುಷ್ಯನಿರುವನು ಎಂದು ನಾನು ನಂಬಿದರೆ, ಅವನ ಹಿಂದೆ ಅಪರಿಮಿತ ಜ್ಞಾನವುಳ್ಳ ಮತ್ತೊಬ್ಬನಿರುವನು ಎಂಬುದನ್ನು ನಾನು ನಂಬಲೇಬೇಕು. ಎರಡನೆಯ ನಿರ್ಣಯವೆಂದರೆ ಯಾವ ಜ್ಞಾನವೂ ಗುರುವಿಲ್ಲದೆ ಬರುವುದಿಲ್ಲವೆಂಬುದು. ಕೆಲವು ಆಧುನಿಕ ತತ್ತ್ವಜ್ಞರು ಹೇಳುವಂತೆ ಮನುಷ್ಯನಲ್ಲಿ ವಿಕಾಸವಾಗುವ ಯಾವುದೋ ಒಂದು ಶಕ್ತಿ ಹುದುಗಿದೆ ಎಂಬುದೇನೋ ನಿಜ. ಆದರೆ ಅದನ್ನು ವ್ಯಕ್ತಗೊಳಿಸಬೇಕಾದರೆ ಒಂದು ವಾತಾವರಣ ಆವಶ್ಯಕ. ಗುರುಗಳಿಲ್ಲದೆ ಯಾವ ಜ್ಞಾನವೂ ನಮಗೆ ದೊರಕಲಾರದು. ಮಾನವ ಗುರುಗಳು, ದೇವಗುರುಗಳು ಮುಂತಾದವರಿದ್ದರೆ, ಅವರೆಲ್ಲರೂ ಕೂಡ ಮಿತಿಯುಳ್ಳವರು. ಅವರಿಗಿಂತ ಮುಂಚೆ ಯಾರು ಗುರುಗಳಾಗಿದ್ದರು? ಕಾಲಾತೀತನಾದ, ಆದಿ ಅಂತ್ಯ ರಹಿತನಾದ, ಸರ್ವಜ್ಞನಾದ ಒಬ್ಬ ಗುರುವನ್ನು ನಾವು ಒಪ್ಪಲೇ ಬೇಕಾಗುವುದು. ಆತನೇ ದೇವರು. 

\vspace{-0.25cm}

\begin{verse}
ತಸ್ಯ ವಾಚಕಃ ಪ್ರಣವಃ~॥ ೨೭~॥
\end{verse}

\vspace{-0.6cm}

\dsize{ಅವನನ್ನು ವ್ಯಕ್ತಪಡಿಸುವ ಅಕ್ಷರವು ‘ಓಂ’}

\vspace{0.2cm}

ನಿಮ್ಮ ಮನಸ್ಸಿನಲ್ಲಿರುವ ಪ್ರತಿಯೊಂದು ಭಾವನೆಯನ್ನೂ ಹೋಲುವ ಪದ ಒಂದು ಇದೆ. ಪದ ಮತ್ತು ಆಲೋಚನೆಗಳನ್ನು ಪ್ರತ್ಯೇಕಿಸಲಾಗುವುದಿಲ್ಲ. ಒಂದೇ ವಸ್ತುವಿನ ಹೊರಭಾಗವನ್ನು ನಾವು ಪದವೆಂದು ಕರೆಯುತ್ತೇವೆ. ಆಂತರಿಕ ಭಾಗವನ್ನು ಆಲೋಚನೆ ಎನ್ನುತ್ತೇನೆ. ನಾವು ಆಲೋಚನೆಯನ್ನು ಪದದಿಂದ ಪ್ರತ್ಯೇಕಿಸುವುದಕ್ಕೆ ಆಗುವುದಿಲ್ಲ. ಮನುಷ್ಯರು ಭಾಷೆಯನ್ನು ತಯಾರುಮಾಡಿದರು; ಕೆಲವು ಜನ ಕುಳಿತುಕೊಂಡು ಯಾವ ಪದವನ್ನು ಉಪಯೋಗಿಸಬೇಕು ಎಂಬುದನ್ನು ನಿಶ್ಚಯಿಸಿದರು ಎಂಬ ಭಾವನೆ ತಪ್ಪು ಎಂಬುದನ್ನು ಆಗಲೇ ಸಿದ್ಧಾಂತ ಮಾಡಿರುವರು. ಮನುಷ್ಯನು ಅಸ್ತಿತ್ವಕ್ಕೆ ಬಂದಾಗಿನಿಂದಲೂ ಪದ ಮತ್ತು ಭಾಷೆ ಇದ್ದುವು. ಪದಕ್ಕೂ ಭಾವನೆಗೂ ಏನು ಸಂಬಂಧವಿದೆ? ಪ್ರತಿ ಆಲೋಚನೆಗೆ ಪದ ಇದ್ದೇ ಇದೆ ಎಂದು ನಮಗೆ ತೋರಿದರೂ, ಆಯಾ ಆಲೋಚನೆಗೆ ಒಂದೇ ಪದ ಎಲ್ಲಾ ಭಾಷೆಯಲ್ಲಿಯೂ ಇರಬೇಕಾಗಿಲ್ಲ. ಇಪ್ಪತ್ತು ದೇಶಗಳಲ್ಲಿ ಆಲೋಚನೆ ಒಂದೇ ಇರಬಹುದು. ಆದರೂ ಭಾಷೆ ವ್ಯತ್ಯಾಸವಾಗಬಹುದು. ಪ್ರತಿಯೊಂದು ಭಾವನೆಯನ್ನು ನಾವು ವ್ಯಕ್ತಗೊಳಿಸಬೇಕಾದರೆ ನಮಗೆ ಒಂದು ಪದವಿರಬೇಕು. ಆದರೆ ಈ ಪದಗಳೆಲ್ಲಕ್ಕೂ ಒಂದೇ ಧ್ವನಿ ಇರಬೇಕಾಗಿಲ್ಲ. ಬೇರೆ ಬೇರೆ ದೇಶಗಳಲ್ಲಿ ಧ್ವನಿಗಳು ಬದಲಾಯಿಸುವುವು. ನಮ್ಮ ಭಾಷ್ಯಕಾರರು ಹೇಳುವುದೇನೆಂದರೆ, ಆಲೋಚನೆಗೂ ಪದಕ್ಕೂ ಇರುವ ಸಂಬಂಧ ಅತಿ ಸಹಜವಾದರೂ ಒಂದು ಭಾವಕ್ಕೆ ಒಂದೇ ಧ್ವನಿ ಇರಬೇಕು ಎನ್ನುವುದಿಲ್ಲ. ಈ ಧ್ವನಿಗಳು ಬದಲಾವಣೆಯಾಗುವುವು. ಆದರೂ ಧ್ವನಿಗೂ ಭಾವಕ್ಕೂ ಇರುವ ಸಂಬಂಧ ಸಹಜವಾದುದು. ಸಂಕೇತಕ್ಕೂ ಅದು ನಿರ್ದೇಶಿಸುವ ವಸ್ತುವಿಗೂ ನಿಜವಾದ ಸಂಬಂಧವಿದ್ದರೆ ಮಾತ್ರ ಧ್ವನಿಗೂ ಅದರ ಭಾವಕ್ಕೂ ಇರುವ ಸಂಬಂಧ ಒಳ್ಳೆಯದು ಎನ್ನಬಹುದು. ಹಾಗೆ ಆಗುವವರೆಗೂ ಆ ಸಂಕೇತವು ಸಾಮಾನ್ಯ ಬಳಕೆಗೆ ಬರುವುದಿಲ್ಲ. ಸಂಕೇತವು ತಾನು ನಿರ್ದೇಶಿಸುವುದನ್ನು ವ್ಯಕ್ತಗೊಳಿಸುತ್ತದೆ. ಉದ್ದೇಶಿಸಿದ ವಸ್ತು ಆಗಲೇ ಇದ್ದರೆ, ನಮ್ಮ ಅನುಭವದಿಂದ ಆ ಪದಸಂಕೇತ ಹಿಂದೆ ಅನೇಕ ವೇಳೆ ಅದೇ ವಸ್ತುವನ್ನು ವ್ಯಕ್ತಗೊಳಿಸಿದೆ ಎಂದು ನಮಗೆ ಗೊತ್ತಾದರೆ, ಆಗ ಪದಕ್ಕೂ ಮತ್ತು ವಸ್ತುವಿಗೂ ಸಂಬಂಧವಿದೆ ಎಂಬುದು ನಿಜ ಎನ್ನುತ್ತೇವೆ. ಆ ವಸ್ತು ಇಲ್ಲದೇ ಇದ್ದರೂ ಕೂಡ ಆ ಪದಗಳ ಮೂಲಕವಾಗಿ ಅದು ಉದ್ದೇಶಿಸುವ ವಸ್ತುವನ್ನು ತಿಳಿಯುವ ಸಹಸ್ರಾರು ಮಂದಿ ಇರುವರು. ಸಂಕೇತಕ್ಕೂ ಅದು ಉದ್ದೇಶಿಸುವ ವಸ್ತುವಿಗೂ ಒಂದು ಸಹಜ ಸಂಬಂಧವಿರಬೇಕು. ಆಗ ನಾವು ಆ ಸಂಕೇತವನ್ನು ಉಚ್ಚರಿಸಿದ ತಕ್ಷಣವೇ ಅದು ಉದ್ದೇಶಿಸಿದ ವಸ್ತು ನಮಗೆ ಹೊಳೆಯುತ್ತದೆ. ದೇವರೆಂಬ ಭಾವಕ್ಕೆ ಅನ್ವಯಿಸುವ ಪದವೇ “ಓಂ” ಎಂದು ಭಾಷ್ಯಕಾರರು ಹೇಳುತ್ತಾರೆ. ಆ ಪದಕ್ಕೆ ಅವರು ಇಷ್ಟು ಪ್ರಾಮುಖ್ಯ ಕೊಡುವುದು ಏತಕ್ಕೆ? ದೇವರೆಂಬ ಭಾವನೆಗೆ ನೂರಾರು ಇರುವ ಪದಗಳಿವೆ. ಒಂದು ಆಲೋಚನೆ ಸಾವಿರಾರು ಪದಗಳೊಂದಿಗೆ ಸಂಬಂಧ ಪಡೆದಿದೆ. ಅವುಗಳಲ್ಲಿ ಪ್ರತಿಯೊಂದು ಪದವೂ ಕೂಡ ದೇವರೆಂಬ ಭಾವನೆಗೆ ಸಂಜ್ಞೆಯಾಗಿ ನಿಲ್ಲುವುದು. ಅದೇನೋ ಒಳ್ಳೆಯದೆ. ಆದರೆ ಈ ಪದಗಳೆಲ್ಲಕ್ಕೂ ಒಂದು ಸಾಮಾನ್ಯತೆ ಇರಬೇಕು. ಯಾವುದು ಸರ್ವಸಾಮಾನ್ಯವಾದ ಸಂಕೇತವಾಗಿದೆಯೋ ಅದೇ ಉತ್ತಮೋತ್ತಮವಾದುದು. ಇದೊಂದೆ ಉಳಿದವುಗಳೆಲ್ಲಕ್ಕೂ ಪ್ರತಿನಿಧಿಯಾಗಬಲ್ಲದು. ಶಬ್ದವನ್ನು ಮಾಡುವಾಗ ಕಂಠನಾಳ ಮತ್ತು ನಾಲಗೆಯನ್ನು ಧ್ವನಿಯು ಹೊಮ್ಮುವುದಕ್ಕೆ ಸಹಕಾರಿಯಾದ ಫಲಕವನ್ನಾಗಿ ಮಾಡುತ್ತೇವೆ. ಎಲ್ಲಾ ಶಬ್ದಗಳಿಗೂ ಕೂಡ ಸಾಮಾನ್ಯ ಮೂಲವಾಗಿ, ಅತಿ ಸಹಜವಾಗಿರುವ ಯಾವುದಾದರೊಂದು ಶಬ್ದವಿದೆಯೆ? “ಓಂ” ಎಂಬುದು ಅಂತಹ ಶಬ್ದ. ಅದೇ ಎಲ್ಲಾ ಶಬ್ದಗಳಿಗೂ ಮೂಲ. ಮೊದಲನೆಯ ಅಕ್ಷರವಾದ “ಅ” ಎಂಬುದೇ (ಅ+ಉ+ಮ=ಓಂ) ಶಬ್ದಮೂಲ. ನಾಲಗೆಯ ಮತ್ತು ಗಂಟಲಿನ ಯಾವ ಭಾಗವನ್ನೂ ಮುಟ್ಟದೆ ಇದನ್ನು ಉಚ್ಚರಿಸುವೆವು. “ಮ” ಎಂಬುದು ಶಬ್ದದ ಕೊನೆ. ಇದನ್ನು ತುಟಿಯನ್ನು ಮುಚ್ಚಿ ಉಚ್ಚರಿಸುತ್ತೇವೆ. “ಉ” ಎಂಬ ಮಧ್ಯಾಕ್ಷರವು ಮೊದಲಿನಿಂದ ಹಿಡಿದು ಬಾಯಿಯ ಕೊನೆಯವರೆವಿಗೂ ಸಂಚರಿಸುವುದು. ಆದಕಾರಣ “ಓಂ” ಎಂಬುದು ಎಲ್ಲಾ ಶಬ್ದೋಚ್ಚಾರಣ ಕ್ರಿಯೆಗೂ ಮೂಲವಾಗಿದೆ. ಹೀಗಾಗಿ ಅದೇ ಸಹಜವಾದ ಸಂಜ್ಞೆಯಾಗಿರಬೇಕು; ಮತ್ತು ಉಳಿದ ಎಲ್ಲಾ ಶಬ್ದಗಳಿಗೂ ಮೂಲವಾಗಿರಬೇಕು. ಎಷ್ಟು ಶಬ್ದಗಳು ಸಾಧ್ಯವೋ ಅವೆಲ್ಲವನ್ನೂ ಉಚ್ಚರಿಸುವುದಕ್ಕೆ ಇದರಿಂದ ಸಾಧ್ಯ. ಈ ವೈಚಾರಿಕ ಹಿನ್ನೆಲೆ ಮಾತ್ರವಲ್ಲದೆ “ಓಂ” ಎಂಬ ಪದದೊಂದಿಗೆ ಭರತಖಂಡದ ಎಲ್ಲಾ ವಿಧದ ಧಾರ್ಮಿಕ ಭಾವನೆಗಳೂ ಹಾಸುಹೊಕ್ಕಾಗಿ ನೆಲಸಿವೆ. ಆದರೆ ಇದಕ್ಕೂ, ಅಮೇರಿಕ, ಇಂಗ್ಲೆಂಡ್​ ಮತ್ತು ಇನ್ನು ಇತರ ದೇಶಗಳಿಗೂ ಏನು ಸಂಬಂಧವಿದೆ? ಭರತಖಂಡದಲ್ಲಿ ಪ್ರತಿಯೊಂದು ಹೊಸ ಧಾರ್ಮಿಕ ಭಾವನೆಯೂ ವಿಕಾಸವಾಗುವ ಸಮಯದಲ್ಲಿ ಈ ಪದವನ್ನು ಅಲ್ಲಗಳೆಯದೆ ಇಟ್ಟುಕೊಂಡಿರುವರು. ಎಲ್ಲ ಭಗವದ್ಭಾವನೆಗಳನ್ನೂ ಇದು ಸೂಚಿಸುವಂತೆ ಮಾಡಿರುವರು. ಇದೇ ಇದರ ಉಪಯೋಗ. ದ್ವೈತಿಗಳು, ವಿಶಿಷ್ಟಾದ್ವೈತಿಗಳು ಮತ್ತು ನಾಸ್ತಿಕರೂ ಕೂಡ ಈ ಪದವನ್ನು ಬಳಸಿದರು. ಬಹುಪಾಲು ಮಾನವ ವರ್ಗಕ್ಕೆ ಅದು ಆಧ್ಯಾತ್ಮಿಕ ಆಕಾಂಕ್ಷೆಯ ಒಂದು ಗುರುತಾಗಿದೆ. ಉದಾಹರಣೆಗೆ ಆಂಗ್ಲ ಭಾಷೆಯ “ಗಾಡ್​” ಎಂಬ ಪದವನ್ನು ತೆಗೆದುಕೊಳ್ಳಿ. ಆದರೆ ಇದು ಒಂದು ಮಿತ ಭಾವನೆಗೆ ಮಾತ್ರ ಅನ್ವಯಿಸುತ್ತದೆ. ನೀವು ಅದನ್ನು ಮೀರಿಹೋಗಿ ದೇವರನ್ನು ಸಾಕಾರ, ನಿರಾಕಾರ, ಅಥವಾ ನಿರ್ಗುಣನನ್ನಾಗಿ ಮಾಡಬೇಕಾಗಿದೆ. ಇದರಂತೆಯೇ ಮಿಕ್ಕ ಭಾಷೆಗಳಲ್ಲಿ ದೇವರು ಎಂಬ ಭಾವನೆಗೆ ಇರುವ ಪದವೂ ಕೂಡ. ಪದ ಬಹಳ ಅಲ್ಪವನ್ನು ನಿರ್ದೇಶಿಸುವುದು. ಆದರೆ “ಓಂ” ಎಂಬ ಪದದಲ್ಲಿ ಎಲ್ಲಾ ವೈಶಿಷ್ಟ್ಯಗಳೂ ಇವೆ. ಆದಕಾರಣ ಇದು ಎಲ್ಲರ ಸ್ವೀಕಾರಕ್ಕೂ ಯೋಗ್ಯವಾಗಿದೆ.

\eject

\begin{verse}
ತಜ್ಜಪಸ್ತದರ್ಥಭಾವನಮ್​~॥ ೨೮~॥
\end{verse}

\vspace{-0.3cm}

\dsize{ಈ ಮಂತ್ರವನ್ನು ಜಪಮಾಡುವುದು ಮತ್ತು ಇತರ ಅರ್ಥದ ಮೇಲೆ ಧ್ಯಾನ ಮಾಡುವುದು (ಇದೇ ದಾರಿ). }

\vspace{0.1cm}

ನಾವೇಕೆ ಇದನ್ನು ಪುನಃ ಜಪ ಮಾಡಬೇಕು? ನಮ್ಮ ಸಂಸ್ಕಾರಗಳ ಮೊತ್ತ ಮನಸ್ಸಿನಲ್ಲಿರುವುದು ಎಂಬ ಸಿದ್ಧಾಂತವನ್ನು ನಾವು ಮರೆತಿಲ್ಲ. ಈ ಸಂಸ್ಕಾರಗಳು ಮನಸ್ಸಿನಲ್ಲಿದ್ದು ಕ್ರಮೇಣ ಅವ್ಯಕ್ತವಾಗುತ್ತ ಬರುತ್ತಿದ್ದರೂ ಇವು ಅಲ್ಲೇ ಇರುವುವು. ಸರಿಯಾದ ಅವಕಾಶ ಸಿಕ್ಕಿದೊಡನೆಯೆ ಇವು ವ್ಯಕ್ತವಾಗುವುವು. ಕಣಗಳ ಸ್ಪಂದನ ಎಂದಿಗೂ ನಿಲ್ಲುವುದಿಲ್ಲ. ಈ ಪ್ರಪಂಚವು ನಾಶವಾದ ಮೇಲೆ ದೊಡ್ಡ ತರಂಗಗಳೆಲ್ಲ ನಾಶವಾಗುತ್ತವೆ, ಸೂರ್ಯ, ಚಂದ್ರ, ತಾರೆಗಳು, ಪೃಥ್ವಿ, ಎಲ್ಲ ಕರಗಿ ಹೋಗುವುವು. ಆದರೆ ಕಣಗಳಲ್ಲಿ ಸ್ಪಂದನವು ಉಳಿಯುವುದು. ಬೃಹತ್​ ವಿಶ್ವದಲ್ಲಿ ಯಾವ ಕ್ರಿಯೆ ನಡೆಯುತ್ತಿದೆಯೋ ಅದೇ ಪ್ರತಿಯೊಂದು ಕಣದಲ್ಲಿಯೂ ನಡೆಯುತ್ತದೆ. ಅದರಂತೆಯೇ ಚಿತ್ತದ ಸ್ಪಂದನ ಶಾಂತವಾದರೂ, ಇದರ ಕಣಗಳ ಸ್ಪಂದನ ನಡೆಯುತ್ತಲೇ ಇರುತ್ತದೆ. ಪುನಃ ಅವಕ್ಕೆ ಅವಕಾಶ ಸಿಕ್ಕಿದಾಗ ಅವುಗಳೆಲ್ಲ ವ್ಯಕ್ತವಾಗುವುವು. ಪುನರಾವರ್ತನೆ ಎಂದರೆ ಏನೆಂಬುದು ನಮಗೆ ಈಗ ಗೊತ್ತಾಗುತ್ತದೆ. ಆಧ್ಯಾತ್ಮಿಕ ಸಂಸ್ಕಾರವನ್ನು ಪ್ರಚೋದಿಸುವ ಮಹಾಶಕ್ತಿ ಇದು. ನಮಗೆ ಒಂದು ಕ್ಷಣ ಲಭಿಸಿದ ಮಹಾತ್ಮರ ಸಂಗವು ನಮ್ಮ ಜೀವನದ ನಾವೆಯನ್ನು ಸಂಸಾರಸಾಗರದಿಂದ ದಾಟಿಸುವುದು. ಸಂಗದ ಮಹಾತ್ಮೆ ಇದು. ಆದಕಾರಣ “ಓಂ” ಮಂತ್ರವನ್ನು ಜಪ ಮಾಡುವುದು, ಅದರ ಅರ್ಥದ ಮೇಲೆ ಧ್ಯಾನಮಾಡುವುದು. ನಮ್ಮ ಮನಸ್ಸಿನಲ್ಲೇ ಸತ್ಸಂಗವನ್ನು ಕಲ್ಪಿಸಿಕೊಂಡಂತೆ. ಮೊದಲು ಅಧ್ಯಯನ ಮಾಡಿ. ನಂತರ ಧ್ಯಾನಮಾಡಿ. ಅದನ್ನು ಚೆನ್ನಾಗಿ ಅಧ್ಯಯನ ಮಾಡಿದಮೇಲೆ ಧ್ಯಾನಮಾಡಿ. ಹೀಗೆ ಮಾಡಿದರೆ ನಿಮಗೆ ಜ್ಞಾನ ಲಭಿಸುವುದು, ಅಂತರಾತ್ಮನು ನಿಮಗೆ ಪ್ರತ್ಯಕ್ಷ ನಾಗುವನು. 

ಆದರೆ ಒಬ್ಬನು “ಓಂ” ಮಂತ್ರವನ್ನು ಮತ್ತು ಅದರ ಅರ್ಥವನ್ನು ಕುರಿತು ಧ್ಯಾನ ಮಾಡಬೇಕು. ದುರ್ಜನರ ಸಂಗವನ್ನು ತ್ಯಜಿಸಿ. ಏಕೆಂದರೆ ಹಳೆಯ ಗಾಯದ ಮಚ್ಚೆ ಆಗಲೇ ನಿಮ್ಮಲ್ಲಿದೆ. ಅದು ಪುನಃ ಹುಣ್ಣಾಗುವುದಕ್ಕೆ ದುರ್ಜನರ ಸಂಗವೇ ಬೇಕಾಗಿರುವುದು. ಇದರಂತೆಯೇ ಸತ್​ಸಂಗ ನಮ್ಮಲ್ಲಿರುವ, ಆದರೆ ಅವ್ಯಕ್ತವಾಗಿರುವ, ಒಳ್ಳೆಯ ಸಂಸ್ಕಾರಗಳನ್ನು ಹೊರಗೆ ಬರುವಂತೆ ಮಾಡುವುದು. ಪ್ರಪಂಚದಲ್ಲಿ ಸತ್​ಸಂಗಕ್ಕಿಂತ ಮೇಲಾದುದು ಯಾವುದೂ ಇಲ್ಲ. ಆಗ ನಮ್ಮಲ್ಲಿರುವ ಒಳ್ಳೆಯ ಸಂಸ್ಕಾರಗಳು ಮೇಲಕ್ಕೆ ಬರುವುವು. 

%%\vspace{-0.3cm}

\begin{verse}
ತತಃ ಪ್ರತ್ಯಕ್​ ಚೇತನಾಧಿಗಮೋಽಪ್ಯಂತರಾಯಾಭಾವಶ್ಚ~॥ ೨೯~॥
\end{verse}

\vspace{-0.3cm}

\dsize{ಅದರಿಂದ ಆತ್ಮವಿಚಾರಜ್ಞಾನ ಲಭಿಸುವುದು ಮತ್ತು ಆತಂಕಗಳು ನಾಶವಾಗುವುವು. }

\vskip 0.2cm

ಪ್ರಣವಜಪದಿಂದ ಮನಸ್ಸು ಹೆಚ್ಚು ಹೆಚ್ಚು ಅಂತರ್ಮುಖವಾಗುತ್ತದೆ, ಹಾಗೂ ಮಾನಸಿಕ ಮತ್ತು ಭೌತಿಕ ಅಡಚಣೆಗಳೆಲ್ಲ ಮಾಯವಾಗಲು ತೊಡಗುತ್ತವೆ. ಯೋಗಿಗೆ ಬರುವ ಅಡಚಣೆಗಳಾವುವು?

\vspace{-0.3cm}

\begin{verse}
ವ್ಯಾಧಿ–ಸ್ತ್ಯಾನ–ಸಂಶಯ–ಪ್ರಮಾದಾಲಸ್ಯಾವಿರತಿ–ಭ್ರಾಂತಿ–\\ದರ್ಶನಾಲಬ್ಧಭೂಮಿಕತ್ವಾನವಸ್ಥಿತತ್ತ್ವಾನಿ ಚಿತ್ತವಿಕ್ಷೇಪಾಸ್ತೇಽಂತರಾಯಾಃ~॥~೩೦~॥
\end{verse}

\vspace{-0.32cm}

\dsize{ವ್ಯಾಧಿ. ಮಾನಸಿಕ ಜಾಡ್ಯ, ಸಂಶಯ, ಉತ್ಸಾಹ ಇಲ್ಲದೆ ಇರುವುದು, ಆಲಸ್ಯ, ಇಂದ್ರಿಯ ಭೋಗಾಸಕ್ತಿ, ಭ್ರಾಂತಿ, ಏಕಾಗ್ರತೆ ಸಿದ್ಧಿಸದೆ ಇರುವುದು, ಪಡೆದ ಸ್ಥಿತಿಯಿಂದ ಜಾರುವುದು–ಇವುಗಳೇ ಆತಂಕಗಳು. }

\vspace{0.2cm}

ವ್ಯಾಧಿ: ಸಂಸಾರಸಾಗರದ ಮೇಲೆ ಆಚೆಯ ದಡಕ್ಕೆ ಕರೆದೊಯ್ಯುವ ನಾವೆ ಈ ದೇಹ. ಇದನ್ನು ನಾವು ಜೋಪಾನವಾಗಿ ನೋಡಿಕೊಳ್ಳಬೇಕು. ಆರೋಗ್ಯವಿಲ್ಲದವರು ಎಂದಿಗೂ ಯೋಗಿಗಳಾಗಲಾರರು. ಮಾನಸಿಕ ಜಾಡ್ಯ ಒಂದು ವಿಷಯದ ಮೇಲೆ ಇರುವ ಸ್ಫೂರ್ತಿಯನ್ನೆಲ್ಲಾ ಹಾಳುಮಾಡುವುದು. ಸ್ಫೂರ್ತಿ ಇಲ್ಲದೆ ಇದ್ದರೆ ಅಭ್ಯಾಸ ಮಾಡುವುದಕ್ಕೆ ಇಚ್ಛೆಯಾಗಲೀ, ಶಕ್ತಿಯಾಗಲೀ ಇರುವುದಿಲ್ಲ. ನಮ್ಮ ಬುದ್ಧಿಯ ದೃಢತೆ ಎಷ್ಟು ನಿಶ್ಚಲವಾಗಿದ್ದರೂ ಕೂಡ, ದೂರದಿಂದ ಕೇಳುವುದು, ನೋಡುವುದು ಮುಂತಾದ ಅತೀಂದ್ರಿಯ ಅನುಭವ ಆಗುವವರೆಗೆ ಯೋಗದ ಸತ್ಯದಲ್ಲಿ ನಮಗೆ ಸಂಶಯ ಬರುವುದು. ಕ್ಷಣಿಕ ಅನುಭವ\break ದುರ್ಬಲವಾದ ಮನಸ್ಸನ್ನು ಪರಿಪುಷ್ಟಿಗೊಳಿಸಿ ಸಾಧನೆಯಲ್ಲಿ ಮುಂದುವರಿಯುವಂತೆ ಪ್ರೇರೇಪಿಸುವುದು. ಪಡೆದ ಸ್ಥಿತಿಯಿಂದ ಜಾರುವುದು–ಅಂದರೆ ಕೆಲವು ದಿನ ಅಥವಾ ವಾರ ಅಭ್ಯಾಸ ಮಾಡುವಾಗ ಮನಸ್ಸು ಶಾಂತವಾಗಿರುವುದು; ಸುಲಭವಾಗಿ ಮನಸ್ಸನ್ನು ಏಕಾಗ್ರ ಮಾಡಬಹುದು; ನೀವು ಬಹಳ ಬೇಗ ಮುಂದುವರಿಯುವಂತೆ ತೋರುವುದು. ಅಕಸ್ಮಾತ್ತಾಗಿ ಮುಂದುವರಿಯುವುದು ಒಂದು ದಿನ ನಿಲ್ಲುವುದು. ದಾರಿ ತಪ್ಪಿದಂತೆ ನಮಗೆ ತೋರುವುದು. ಆದರೂ ಬಿಡಬೇಡಿ, ಮುಂದುವರಿಯಿರಿ. ಇಂತಹ ಏಳುಬೀಳುಗಳ ಮೇಲೆಯೇ ಜಯವೆಲ್ಲ ನಿಂತಿರುವುದು. 

\vspace{-0.3cm}

\begin{verse}
ದುಃಖ–ದೌರ್ಮನಸ್ಯಾಂಗಮೇಜಯತ್ವ–ಶ್ವಾಸಪ್ರಶ್ವಾಸಾ ವಿಕ್ಷೇಪಸಹಭುವಃ~॥~೩೧~॥
\end{verse}

\vspace{-0.3cm}

\dsize{ದುಃಖ, ಮಾನಸಿಕ ಕೊರಗು, ದೇಹಕಂಪನ, ಅಕ್ರಮವಾಗಿ ಉಸಿರಾಡುವುದು–ಇವುಗಳೆಲ್ಲ ಮನಸ್ಸಿನ ಏಕಾಗ್ರತೆ ಸಿದ್ಧಿಸದೆ ಇರುವಾಗ ಬರುವುವು. }

\vspace{0.3cm}

ಮನಸ್ಸಿನ ಏಕಾಗ್ರತೆಯನ್ನು ಅಭ್ಯಾಸ ಮಾಡಿದಾಗಲೆಲ್ಲಾ ದೇಹಕ್ಕೆ ಮತ್ತು ಮನಸ್ಸಿಗೆ ಇದು ಶಾಂತಿಯನ್ನು ತರುವುದು. ಅಭ್ಯಾಸ ತಪ್ಪಿದಾಗ ಅಥವಾ ಮನಸ್ಸನ್ನು ಸಾಕಾದಷ್ಟು ನಿಗ್ರಹಿಸದಿದ್ದಾಗ ಈ ತೊಂದರೆಗಳು ಬರುವುವು. “ಓಂ” ಮಂತ್ರವನ್ನು ಜಪ ಮಾಡುವುದು, ಮತ್ತು ದೇವರಲ್ಲಿ ಶರಣಾಗತನಾಗುವುದರಿಂದ ಮನಸ್ಸು ದೃಢವಾಗುವುದು; ಅದಕ್ಕೆ ಒಂದು ಹೊಸ ಚೇತನ ಬರುವುದು. ನರಗಳ ಕಂಪನ ಸಾಮಾನ್ಯವಾಗಿ ಎಲ್ಲರಿಗೂ ಬರುವುದು. ಅದರ ಬಗ್ಗೆ ಚಿಂತಿಸಬೇಡಿ, ಅಭ್ಯಾಸ ಮಾಡುತ್ತ ಹೋಗಿ. ಅಭ್ಯಾಸ ಇದನ್ನು ಹೋಗಲಾಡಿಸುವುದು ಮತ್ತು ಆಸನ ಸ್ಥಿರವಾಗುವುದು. 


\begin{verse}
ತತ್​ ಪ್ರತಿಷೇಧಾರ್ಥಮೇಕತತ್ತ್ವಾಭ್ಯಾಸಃ~॥ ೩೨~॥
\end{verse}

\vspace{-0.33cm}

\dsize{ಇದನ್ನು ಹೋಗಲಾಡಿಸುವುದಕ್ಕೆ ಒಂದು ವಿಷಯವನ್ನು ಕುರಿತು ಅಭ್ಯಾಸ ಮಾಡಬೇಕು. }

\eject

ಕೆಲವು ಕಾಲ ಮನಸ್ಸು ಒಂದು ರೂಪವನ್ನು ತೆಗೆದುಕೊಳ್ಳುವಂತೆ ಮಾಡಿದರೆ ಈ ತೊಂದರೆಗಳು ಮಾಯವಾಗುವುವು. ಇದು ಸಾಮಾನ್ಯ ಬುದ್ಧಿವಾದ. ಮುಂದೆ ಬರುವ ಸೂತ್ರಗಳು ಇದನ್ನು ವಿಶಾಲಮಾಡಿ ಪ್ರತ್ಯೇಕವಾಗಿ ಹೇಳುತ್ತವೆ. ಒಂದೇ ಅಭ್ಯಾಸ ಎಲ್ಲರಿಗೂ ಸರಿಹೋಗದ ಕಾರಣ ಅನೇಕ ಸಾಧನೆಗಳನ್ನು ಹೇಳುತ್ತಾರೆ. ಪ್ರತಿಯೊಬ್ಬರೂ ಕೂಡ ಪ್ರತ್ಯಕ್ಷ ಅನುಭವದ ಮೂಲಕ ಯಾವುದು ತಮಗೆ ಹೆಚ್ಚು ಉಪಕಾರಿಯೋ ಅದನ್ನು ಕಂಡುಹಿಡಿಯಬೇಕು. 

\vspace{-0.1cm}

\begin{verse}
ಮೈತ್ರೀ–ಕರುಣಾ–ಮುದಿತೋಪೇಕ್ಷಾಣಾಂ ಸುಖ–\\ದುಃಖಪುಣ್ಯಾಪುಣ್ಯವಿಷಯಾಣಾಂ ಭಾವನಾತಶ್ಚಿತ್ತಪ್ರಸಾದನಮ್​~॥ ೩೩~॥
\end{verse}

\vspace{-0.4cm}

\dsize{ಸುಖ, ದುಃಖ, ಒಳ್ಳೆಯದು, ಕೆಟ್ಟದ್ದು, ಇವುಗಳ ವಿಷಯದಲ್ಲಿ ಸ್ನೇಹ, ಕರುಣೆ, ಸಂತೋಷ, ಉಪೇಕ್ಷೆ ಇವುಗಳನ್ನು ಕ್ರಮವಾಗಿ ತಾಳುವುದು. ಇವುಗಳಿಂದ ಚಿತ್ತ ಪ್ರಸನ್ನವಾಗುವುದು. }

\vskip 0.2cm

ನಮಗೆ ಇಂತಹ ನಾಲ್ಕು ಭಾವನೆಗಳು ಇರಬೇಕು. ಎಲ್ಲರಿಗೂ ನಾವು ಮಿತ್ರರಾಗಿರಬೇಕು. ದುಃಖದಲ್ಲಿರುವವರಿಗೆ ಕರುಣೆ ತೋರಬೇಕು. ಜನರು ಸಂತೋಷದಿಂದಿರುವಾಗ ನಾವು ಸಂತೋಷದಿಂದಿರಬೇಕು. ದುಷ್ಟರನ್ನು ಕಂಡರೆ ಉಪೇಕ್ಷೆ ತಾಳಬೇಕು. ನಮ್ಮೆದುರಿಗೆ ಬರುವ ಎಲ್ಲಾ ವಸ್ತುಗಳ ವಿಷಯದಲ್ಲಿಯೂ ಹೀಗೆ ಇರಬೇಕು. ಅದು ಸಂತೋಷದಿಂದ ಕೂಡಿದ್ದರೆ ಅದನ್ನು ಸೌಹಾರ್ದದಿಂದ ನೋಡಬೇಕು; ದುಃಖದಿಂದ ಕೂಡಿದ್ದರೆ ಅದನ್ನು ಸಹಾನುಭೂತಿಯಿಂದ ನೋಡಬೇಕು; ಅದು ಒಳ್ಳೆಯದಾಗಿದ್ದರೆ ನಾವು ಸಂತೋಷಪಡಬೇಕು; ಅದು ಕೆಟ್ಟದಿದ್ದರೆ ನಾವು ಮನಸ್ಸಿಗೆ ಹಚ್ಚಿಕೊಳ್ಳಬಾರದು. ಅನೇಕ ವಿಷಯಗಳು ಮನಸ್ಸಿಗೆ ಬರುವಾಗ ಇಂತಹ ಸ್ಥಿತಿಯಿಂದ ಮನಸ್ಸಿಗೆ ಶಾಂತಿ ದೊರಕುವುದು. ನಮ್ಮ ಅನುದಿನದ ಜೀವನದಲ್ಲಿರುವ ಮುಕ್ಕಾಲು ಪಾಲು ಕಷ್ಟಕ್ಕೆ ಕಾರಣ ನಾವು ಮನಸ್ಸನ್ನು ಈ ಸ್ಥಿತಿಯಲ್ಲಿ ಇರಿಸಲಾಗದೆ ಇರುವುದು. ಉದಾಹರಣೆಗೆ, ಒಬ್ಬನು ನಮಗೆ ಕೆಟ್ಟದ್ದನ್ನು ಮಾಡಿದರೆ ತಕ್ಷಣವೇ ನಾವು ಅವನಿಗೆ ಕೆಟ್ಟದ್ದನ್ನು ಮಾಡಲು ಬಯಸುತ್ತೇವೆ. ಕೆಟ್ಟದ್ದನ್ನು ಮಾಡಬೇಕೆಂಬ ಪ್ರತಿಯೊಂದು ಪ್ರತಿಕ್ರಿಯೆಯೂ ಕೂಡ ಚಿತ್ತವನ್ನು ನಾವು ಸ್ವಾಧೀನದಲ್ಲಿಡಲಾಗುವುದಿಲ್ಲ ಎಂಬುದನ್ನು ತೋರುವುದು. ಅದು ವಸ್ತುವಿನೆಡೆಗೆ ಅಲೆಯಂತೆ ತೆರಳುವುದು, ಆಗ ನಾವು ನಮ್ಮ ನಿಗ್ರಹವನ್ನು ಕಳೆದುಕೊಳ್ಳುವೆವು. ದ್ವೇಷ ಮತ್ತು ಕೆಟ್ಟ ಭಾವನೆಗಳಿಂದ ಕೂಡಿದ ಪ್ರತಿಯೊಂದು ಕ್ರಿಯೆಯೂ ಕೂಡ ಮನಸ್ಸಿಗೆ ಅಷ್ಟು ನಷ್ಟವನ್ನುಂಟು ಮಾಡುವುದು. ಪ್ರತಿಯೊಂದು ದ್ವೇಷದ ಆಲೋಚನೆ, ಕ್ರಿಯೆ ಅಥವಾ ಪ್ರತಿಕ್ರಿಯೆಯೇ ಆಗಲಿ ಇವನ್ನು ತಡೆದಷ್ಟೂ ನಮಗೆ ಒಳ್ಳೆಯದು. ಹೀಗೆ ನಿಗ್ರಹಿಸುವುದರಿಂದ ನಮಗೇನೂ ಕೇಡಾಗುವುದಿಲ್ಲ. ನಾವು ಆಶಿಸುವುದಕ್ಕಿಂತ ಹೆಚ್ಚಾಗಿ ನಮಗೆ ಒಳ್ಳೆಯದಾಗುವುದು. ಪ್ರತಿಯೊಂದು ಸಲವೂ ಬರುವ ಕೋಪ ಮತ್ತು ದ್ವೇಷದ ಭಾವನೆಯನ್ನು ತಡೆದಷ್ಟೂ, ಒಳ್ಳೆಯ ಶಕ್ತಿಯನ್ನು ತಮ್ಮ ಹಿತಕ್ಕೆ ಕೂಡಿಟ್ಟಂತೆ. ಇದೇ ಶಕ್ತಿ ಮುಂದೆ ಉತ್ತಮತರದ ತಪಶ್ಯಕ್ತಿಯಾಗಿ ಪರಿಣಮಿಸುವುದು. 

\newpage

\vspace{-0.3cm}

\begin{verse}
ಪ್ರಚ್ಛರ್ದನ–ವಿಧಾರಣಾಭ್ಯಾಂ ವಾ ಪ್ರಾಣಸ್ಯ~॥ ೩೪~॥
\end{verse}

\vspace{-0.33cm}

\dsize{ಪ್ರಾಣವನ್ನು ಹೊರಗೆ ಬಿಟ್ಟು ನಿಗ್ರಹಿಸುವುದರಿಂದ. }

\vspace{0.1cm}

ಇಲ್ಲಿ ಉಪಯೋಗಿಸಿರುವ ಪದ ಪ್ರಾಣ. ಪ್ರಾಣವೆಂದರೆ ಕೇವಲ ಉಸಿರು ಮಾತ್ರವಲ್ಲ. ವಿಶ್ವಶಕ್ತಿಯ ಹೆಸರಿದು. ವಿಶ್ವದಲ್ಲಿ ನಮಗೆ ಕಾಣಿಸುವ ಎಲ್ಲಾ ಶಕ್ತಿಯೂ, ಸಂಚರಿಸುವ, ಕೆಲಸಮಾಡುವ, ಅಥವಾ ಚೇತನದಿಂದ ಕೂಡಿದ ಸಕಲ ವಸ್ತು ಈ ಶಕ್ತಿಯ ಅಭಿವ್ಯಕ್ತಿ. ವಿಶ್ವದಲ್ಲಿರುವ ಶಕ್ತಿ ಸಮೂಲಕ್ಕೆ ಪ್ರಾಣವೆಂದು ಹೆಸರು. ಕಲ್ಪ ಪೂರ್ವದಲ್ಲಿ ಈ ಪ್ರಾಣವು ಬಹುಪಾಲು ತಟಸ್ಥಾವಸ್ಥೆಯಲ್ಲಿರುವುದು. ಕಲ್ಪ ಮೊದಲಾದ ಮೇಲೆ ಪ್ರಾಣವು ವ್ಯಕ್ತವಾಗುತ್ತದೆ. ಬಾಹ್ಯ ವಸ್ತುವಿನಲ್ಲಿರುವ ಚಲನೆ ಅಥವಾ ಮನುಷ್ಯರು ಮತ್ತು ಪ್ರಾಣಿವರ್ಗದಲ್ಲಿ ನರಗಳ ಚಲನೆಯಂತೆ ಯಾವ ಪ್ರಾಣವು ಕಾಣುತ್ತಿದೆಯೋ ಅದೇ ಪ್ರಾಣವೇ ಆಲೋಚನೆ ಇತ್ಯಾದಿಗಳಂತೆ ಕಾಣುತ್ತಿದೆ. ಪ್ರಪಂಚವೇ ಪ್ರಾಣ ಮತ್ತು ಆಕಾಶದ ಮಿಶ್ರಣ. ಅದರಂತೆಯೇ ನಮ್ಮ ದೇಹವೂ ಕೂಡ. ನೀವು ನೋಡುವ ಮತ್ತು ಅನುಭವಿಸುವ ಪ್ರತಿಯೊಂದು ವಸ್ತುವೂ ಬರುವುದು ಆಕಾಶದಿಂದ; ಎಲ್ಲಾ ವಿಧದ ಶಕ್ತಿ ಬರುವುದು ಪ್ರಾಣದಿಂದ. ಪ್ರಾಣವನ್ನು ಹೊರಗೆ ಬಿಟ್ಟು ನಿಗ್ರಹಿಸುವುದು ಎಂದರೆ, ಪ್ರಾಣಾಯಾಮವೆಂದು ಅರ್ಥ. ಯೋಗಶಾಸ್ತ್ರದ ಮೂಲ ಕರ್ತೃವಾದ ಪತಂಜಲಿಯು ಪ್ರಾಣಾಯಾಮದ ಮೇಲೆ ಹಲವು ವಿಧದ ಪ್ರತ್ಯೇಕ ಸಲಹೆಗಳನ್ನು ಕೊಡುವುದಿಲ್ಲ. ಆದರೆ ಅನಂತರ ಬಂದ ಕೆಲವು ಯೋಗಿಗಳು ಪ್ರಾಣಾಯಾಮದ ವಿಚಾರವಾಗಿ ಹಲವು ತಥ್ಯಗಳನ್ನು ಕಂಡುಹಿಡಿದು ಅದನ್ನೇ ಒಂದು ದೊಡ್ಡ ವಿಜ್ಞಾನವನ್ನಾಗಿ ಮಾಡಿರುವರು. ಪತಂಜಲಿಗೆ ಪ್ರಾಣಾಯಾಮ ಹಲವು ಮಾರ್ಗಗಳಲ್ಲಿ ಒಂದು, ಇದಕ್ಕೆ ಅಷ್ಟು ಪ್ರಾಧಾನ್ಯವನ್ನು ಕೊಡುವುದಿಲ್ಲ. ಅವನ ಅಭಿಪ್ರಾಯವೇನೆಂದರೆ, ನೀವು ಸುಮ್ಮನೆ ಉಸಿರನ್ನು ಬಿಟ್ಟು, ಮತ್ತೊಮ್ಮೆ ಉಸಿರನ್ನು ಸೆಳೆದುಕೊಂಡು ಸ್ವಲ್ಪ ಹೊತ್ತು ಅದನ್ನು ನಿಲ್ಲಿಸಿರುತ್ತೀರಿ, ಅಷ್ಟೆ. ಇದರಿಂದ ಮನಸ್ಸು ಸ್ವಲ್ಪ ಶಾಂತವಾಗುವುದು. ಆದರೆ ಪ್ರಾಣಾಯಾಮವೆಂಬ ವಿಜ್ಞಾನವು ಅನಂತರ ಇದರಿಂದ ಬೆಳೆಯಿತು ಎಂಬುದನ್ನು ನೋಡಿರುವಿರಿ. ಅನಂತರ ಬಂದ ಯೋಗಿಗಳು ಪ್ರಾಣಾಯಾಮದ ವಿಚಾರವಾಗಿ ಏನು ಹೇಳುತ್ತಾರೆ ಎಂಬುದನ್ನು ಸ್ವಲ್ಪ ಕೇಳೋಣ. ಇವುಗಳಲ್ಲಿ ಕೆಲವು ವಿಷಯವನ್ನು ಆಗಲೇ ನಿಮಗೆ ಹೇಳಿದ್ದೇವೆ. ಆದರೆ ಮತ್ತೊಮ್ಮೆ ಹೇಳಿದರೆ ಮನಸ್ಸಿನಲ್ಲಿ ನೆಲೆಸುವುದಕ್ಕೆ ಸಹಾಯಕವಾಗುವುದು. ಮೊದಲನೆಯದಾಗಿ ಪ್ರಾಣವೆಂದರೆ ಉಸಿರಲ್ಲ ಎಂಬುದನ್ನು ನೆನಪಿನಲ್ಲಿಡಬೇಕು. ಯಾವುದು ಶ್ವಾಸೋಚ್ಛ್ವಾಸಗಳನ್ನು ಮಾಡುವುದೋ, ಯಾವುದು ಶ್ವಾಸೋಚ್ಛ್ವಾಸಗಳ ಜೀವಾಳವೋ, ಅದೇ ಪ್ರಾಣ. ಪ್ರಾಣ ಎಂಬುದನ್ನು ಎಲ್ಲಾ ಇಂದ್ರಿಯಗಳಿಗೂ ಅನ್ವಯಿಸಲಾಗಿದೆ. ಅವುಗಳೆಲ್ಲವನ್ನೂ ಪ್ರಾಣವೆಂದು ಕರೆಯುತ್ತಾರೆ. ಮನಸ್ಸನ್ನು ಪ್ರಾಣವೆಂದು ಕರೆದಿದ್ದಾರೆ. ಆದಕಾರಣ ಪ್ರಾಣವೆಂದರೆ ಶಕ್ತಿ ಎಂದು ಅರ್ಥ. ಆದರೂ ಕೂಡ ನಾವು ಅದನ್ನು ಶಕ್ತಿ ಎಂದು ಕರೆಯುವುದಕ್ಕೆ ಆಗುವುದಿಲ್ಲ. ಏಕೆಂದರೆ ಶಕ್ತಿ ಎಂಬುದು ಪ್ರಾಣದ ಅಭಿವ್ಯಕ್ತಿ. ಎಲ್ಲ ವಿಧದ ಚಲನೆಯಲ್ಲಿಯೂ ಶಕ್ತಿಯಾಗಿ ವ್ಯಕ್ತವಾಗುತ್ತಿರುವುದು ಇದೇ. ಚಿತ್ತವು ಬಾಹ್ಯ ವಾತಾವರಣದಿಂದ ಪ್ರಾಣವನ್ನು ಒಳಗೆ ಸೆಳೆದುಕೊಳ್ಳುವ ಒಂದು ಯಂತ್ರ. ಇದು ದೇಹಪೋಷಣೆಗೆ, ಆಲೋಚನೆಗೆ, ಇಚ್ಛಾಶಕ್ತಿಗೆ ಬೇಕಾದ ಹಲವು ವಿಧದ ಶಕ್ತಿಗಳನ್ನು ಪ್ರಾಣದ ಮೂಲಕ ತಯಾರು ಮಾಡುವುದು. ಮೇಲೆ ಹೇಳಿದ ಪ್ರಾಣಾಯಾಮದ ಮೂಲಕ ದೇಹದಲ್ಲಿರುವ ನಾನಾ ವಿಧದ ನರಗಳ ಶಕ್ತಿಯನ್ನು ನಾವು ಸ್ವಾಧೀನಕ್ಕೆ ತರಬಹುದು. ಮೊದಲು ಅದನ್ನು ನಾವು ಕಂಡುಹಿಡಿಯುತ್ತೇವೆ, ಅನಂತರ ಸಾವಧಾನವಾಗಿ ನಾವು ಅವನ್ನು ಸ್ವಾಧೀನಕ್ಕೆ ತರುತ್ತೇವೆ. 

ಮಾನವ ದೇಹದಲ್ಲಿ ಮೂರು ಬಗೆಯ ಪ್ರಾಣಗಳಿವೆ ಎಂಬುದು ಅನಂತರ ಬಂದ ಯೋಗಿಗಳ ಅಭಿಪ್ರಾಯ. ಒಂದನ್ನು ಇಡಾ ಎಂತಲೂ, ಮತ್ತೊಂದನ್ನು ಪಿಂಗಳ ಎಂತಲೂ, ಮೂರನೆಯದನ್ನು ಸುಷುಮ್ನಾ ಎಂದೂ ಕರೆಯುತ್ತಾರೆ. ಅವರ ಅಭಿಪ್ರಾಯದ ಪ್ರಕಾರ ಬೆನ್ನುಮೂಳೆಯ ಬಲಭಾಗದಲ್ಲಿರುವುದು ಪಿಂಗಳ, ಎಡಭಾಗದಲ್ಲಿರುವುದು ಇಡಾ, ಮಧ್ಯಭಾಗದಲ್ಲಿರುವ ಬರಿದಾದ ನಾಲೆಯೇ ಸುಷುಮ್ನ ಇಡಾ ಮತ್ತು ಪಿಂಗಳಗಳೇ ಪ್ರತಿಯೊಬ್ಬನಲ್ಲಿ ಕೂಡ ಕೆಲಸ ಮಾಡುತ್ತಿರುವ ಶಕ್ತಿ. ಇದರ ಮೂಲಕ ನಾವು ಜೀವನದ ಎಲ್ಲಾ ಕೆಲಸವನ್ನೂ ನಡೆಸುತ್ತಿರುವೆವು. ಸುಷುಮ್ನಾ ಶಕ್ತಿ ಎಲ್ಲರಲ್ಲೂ ಇದೆ, ಎಲ್ಲರಿಗೂ ಅದನ್ನು ಪಡೆಯಲು ಸಾಧ್ಯ. ಆದರೆ ಯೋಗಿಯಲ್ಲಿ ಮಾತ್ರ ಅದು ಕೆಲಸ ಮಾಡುತ್ತಿರುವುದು. ಯೋಗ ದೇಹವನ್ನು ಬದಲಾಯಿಸುತ್ತದೆ ಎಂಬುದನ್ನು ನೀವು ಜ್ಞಾಪಕದಲ್ಲಿಡಬೇಕು. ನೀವು ಸಾಧನೆ ಮಾಡುತ್ತ ಹೋದರೆ ನಿಮ್ಮ ದೇಹ ಬದಲಾಯಿಸುವುದು. ಸಾಧನೆಗೆ ಮುಂಚೆ ಇದ್ದ ದೇಹವಲ್ಲ ಅನಂತರ ಇರುವುದು. ಈ ಹೇಳಿಕೆಯು ಯುಕ್ತಿಯುಕ್ತವಾದುದು. ಬೇಕಾದರೆ ಇದನ್ನು ವಿವರಿಸಬಹುದು. ಏಕೆಂದರೆ ನಮ್ಮಲ್ಲಿರುವ ಪ್ರತಿಯೊಂದು ಹೊಸ ಆಲೋಚನೆಯ ಕೂಡ ನಮ್ಮ ಮಿದುಳಿನಲ್ಲಿ ಒಂದು ಹೊಸ ಪ್ರಣಾಳಿಕೆಯನ್ನು ಮಾಡಿದಂತೆ. ಮಾನವನು ಬದಲಾವಣೆಯನ್ನು ಅತಿಯಾಗಿ ವಿರೋಧಿಸುವುದಕ್ಕೆ ಇದೇ ಕಾರಣ. ಆಗಲೇ ಇರುವ ದಾರಿಯಲ್ಲಿ ಮಾನವನ ಸ್ವಭಾವ ನಡೆಯುವುದಕ್ಕೆ ಇಚ್ಛಿಸುತ್ತದೆ. ಏಕೆಂದರೆ ಅದು ಸುಲಭ. ಉದಾಹರಣೆಗಾಗಿ, ಮನಸ್ಸನ್ನು ಒಂದು ಸೂಜಿ ಎಂತಲೂ, ಮಿದುಳು ಅದರ ಮುಂದಿರುವ ಮೃದುದ್ರವ್ಯ ಎಂತಲೂ ತಿಳಿದರೆ, ನಮ್ಮಲ್ಲಿರುವ ಪ್ರತಿಯೊಂದು ಹೊಸ ಆಲೋಚನೆಯೂ ಅಲ್ಲಿ ಒಂದು ಹೊಸ ಮಾರ್ಗವನ್ನು ಮಾಡಿದಂತೆ ಆಗುವುದು. ಈ ದಾರಿಯನ್ನು ಬೇರೆಯಾಗಿಡುವುದಕ್ಕೆ ಮಿದುಳಿನಲ್ಲಿರುವ ಬೂದು ದ್ರವ್ಯ \enginline{(Grey Matter)} ಬರದೇ ಇದ್ದರೆ ದಾರಿ ಮುಚ್ಚಿಕೊಳ್ಳುವುದು. ಈ ಬೂದು ದ್ರವ್ಯ ಇಲ್ಲದೇ ಇದ್ದರೆ ನಮಗೆ ಜ್ಞಾಪಕಶಕ್ತಿ ಇರುವುದಿಲ್ಲ. ಜ್ಞಾಪಕವೆಂದರೆ ಹಳೆಯ ರಸ್ತೆಯಲ್ಲಿ ಮತ್ತೆ ಹೋಗುವುದು, ಆಲೋಚನೆಯ ಮೇಲೆ ಪುನಃ ತಿದ್ದಿದಂತೆ. ಈಗ ನಿಮಗೆ ಗೊತ್ತಾಗಿರಬಹುದು. ಯಾರಾದರೂ ಸರ್ವರಿಗೂ ಸಾಮಾನ್ಯವಾಗಿ ಪರಿಚಿತವಾದ ಕೆಲವು ಭಾವನೆಗಳನ್ನು ತೆಗೆದುಕೊಂಡು ಅವನ್ನು ನಾನಾ ರೀತಿಯಲ್ಲಿ ಬೆರಸಿ ಒಂದು ವಿಷಯದ ಮೇಲೆ ಮಾತನಾಡಿದರೆ ಅದನ್ನು ತಿಳಿದುಕೊಳ್ಳುವುದು ಸುಲಭ. ಏಕೆಂದರೆ ಈ ದಾರಿ ಪ್ರತಿಯೊಬ್ಬರ ಮಿದುಳಿನಲ್ಲಿಯೂ ಇದೆ, ಇದು ಅವರಿಗೆ ಪುನಃ ಹೊಳೆಯಬೇಕು ಅಷ್ಟೆ. ಆದರೆ ಯಾವಾಗಲಾದರೂ ಹೊಸ ವಿಷಯ ಬಂದರೆ ಹೊಸ ರಸ್ತೆಯನ್ನು ಮಾಡಬೇಕು. ಆದಕಾರಣವೇ ಸುಲಭವಾಗಿ ಇದು ಗೊತ್ತಾಗುವುದಿಲ್ಲ. ಅದಕ್ಕೋಸ್ಕರವೇ ಮಿದುಳು (ಮನುಷ್ಯರ ಮಿದುಳು) ಹೊಸ ಭಾವನೆಗಳ ಪ್ರಭಾವಕ್ಕೆ ಬೀಳಲು ಒಪ್ಪುವುದಿಲ್ಲ, ಅವನ್ನು ಅಪ್ರಜ್ಞಾಪೂರ್ವಕವಾಗಿ ತಿರಸ್ಕರಿಸುವುದು. ಪ್ರಾಣವು ಹೊಸ ದಾರಿಯನ್ನು ಮಾಡುವುದಕ್ಕೆ ಪ್ರಯತ್ನಿಸುತ್ತದೆ. ಆದರೆ ಮಿದುಳು ಅದಕ್ಕೆ ಅವಕಾಶ ಕೊಡುವುದಿಲ್ಲ. ಬದಲಾವಣೆಯನ್ನು ವಿರೋಧಿಸುವವರ ಗುಟ್ಟು ಇದೇ. ಮಿದುಳಿನಲ್ಲಿ ಕಡಮೆ ದಾರಿ ಇದ್ದಷ್ಟೂ, ಪ್ರಾಣದ ಸೂಜಿ ರಸ್ತೆಯನ್ನು ಕಡಮೆ ಮಾಡಿದಷ್ಟೂ ಮಿದುಳು ಪ್ರತಿಗಾಮಿಯಾಗುವುದು, ಅದು ಅಷ್ಟೂ ಹೆಚ್ಚು ಹೆಚ್ಚಾಗಿ ಹೊಸ ಭಾವನೆಯೊಂದಿಗೆ ಹೋರಾಡುವುದು. ಮನುಷ್ಯನು ಹೆಚ್ಚು ಆಲೋಚನಾಪರನಾದಷ್ಟೂ, ಅಷ್ಟೂ ಹೆಚ್ಚು ದಾರಿಗಳು ಮಿದುಳಿನಲ್ಲಿ ಆಗುತ್ತವೆ. ಅವನು ಸುಲಭವಾಗಿ ಹೊಸ ಭಾವನೆಯನ್ನು ಸ್ವೀಕರಿಸಿ ಅರ್ಥಮಾಡಿಕೊಳ್ಳಬಲ್ಲ. ಪ್ರತಿಯೊಂದು ಹೊಸ ಭಾವನೆಯೂ ಅಷ್ಟೆ. ಮಿದುಳಿನಲ್ಲಿ ಹೊಸದೊಂದು ಮುದ್ರೆಯನ್ನು ಒತ್ತುವುದು, ಮಿದುಳಿನಲ್ಲಿ ಹೊಸ ದಾರಿಯನ್ನು ಮಾಡುವುದು. ಆದಕಾರಣವೆ, ಯೋಗಾಭ್ಯಾಸದ (ಇದು ಸಂಪೂರ್ಣ ಹೊಸ ಭಾವ ಮತ್ತು ಕ್ರಿಯೋತ್ತೇಜನ ಶಕ್ತಿಯಾಗಿರುವುದರಿಂದ) ಮೊದಲಲ್ಲಿ ಅಷ್ಟೊಂದು ಶಾರೀರಿಕ ಅಡಚಣೆ ಆಗುವುದು. ಆದಕಾರಣವೇ ಪ್ರಕೃತಿಯಲ್ಲಿ ಬಾಹ್ಯ ಪ್ರಪಂಚಕ್ಕೆ ಸೇರಿದ ಧಾರ್ಮಿಕ ಭಾವನೆಯನ್ನು ಅಷ್ಟು ಸುಲಭವಾಗಿ ಜನರು ಸ್ವೀಕರಿಸುವುದು ನಮಗೆ ಕಾಣುವುದು. ಆದರೆ ಮಾನವನ ಅಂತರಂಗಕ್ಕೆ ಸಂಬಂಧಪಟ್ಟ ತತ್ತ್ವ ಮತ್ತು ಮನಶ್ಯಾಸ್ತ್ರದ ಭಾಗವನ್ನು ಹೆಚ್ಚಾಗಿ ಕಡೆಗಣಿಸುವರು. 

ಈ ನಮ್ಮ ಜಗತ್ತಿನ ವಿವರಣೆಯನ್ನು ನಾವು ನೆನಪಿನಲ್ಲಿಡಬೇಕು, ಅನಂತ ಅಸ್ತಿತ್ವವು ನಮ್ಮ ಅರಿವಿನ ಸ್ತರಕ್ಕೆ ಚಾಚಿಕೊಂಡಿರುವುದೇ ಈ ಜಗತ್ತು. ಅನಂತತೆಯ ಅಲ್ಪಭಾಗ ನಮ್ಮ ಅರಿವಿಗೆ ತೋರುತ್ತಿದೆ. ಅದನ್ನೇ ನಾವು ನಮ್ಮ ಜಗತ್ತೆಂದು ಕರೆಯುವುದು. ಆದಕಾರಣ ಇದನ್ನು ಮೀರಿದ ಅನಂತತೆ ಇದೆ. ಜಗತ್ತು ಎಂದು ಕರೆಯುವ ಈ ಒಂದು ಚೂರು ಮತ್ತು ಇದನ್ನು ಮೀರಿರುವುದು ಎರಡನ್ನೂ ಧರ್ಮವು ತನ್ನ ಗಮನಕ್ಕೆ ತೆಗೆದುಕೊಳ್ಳಬೇಕು. ಯಾವ ಧರ್ಮ ಇದರಲ್ಲಿ ಒಂದನ್ನು ಮಾತ್ರ ಗಮನಕ್ಕೆ ತೆಗೆದುಕೊಳ್ಳುವುದೋ ಅದು ದೋಷಯುಕ್ತವಾದುದು. ಧರ್ಮವು ಆದಕಾರಣ ಇವೆರಡನ್ನೂ ನೋಡಿಕೊಳ್ಳಬೇಕು. ಯಾವ ಧರ್ಮ ಕಾಲದೇಶನಿಮಿತ್ತವೆಂಬ ಪಂಜರದೊಳಗೆ ಬದ್ಧವಾದ, ಮತ್ತು ನಮ್ಮ ಅರಿವಿನ ಸ್ತರಕ್ಕೆ ಸೀಮಿತವಾದ ಅನಂತವನ್ನು ಮಾತ್ರ ಗಮನಕ್ಕೆ ತೆಗೆದುಕೊಳ್ಳುವುದೋ ಅದು ನಮಗೆ ಚಿರಪರಿಚಿತವಾಗಿದೆ. ಏಕೆಂದರೆ ನಾವಾಗಲೇ ಅದರಲ್ಲಿರುವೆವು. ಅನಾದಿಕಾಲದಿಂದಲೂ ಕೂಡ ಪ್ರಪಂಚದ ಭಾವನೆ ನಮ್ಮಲ್ಲಿದೆ. ಅತೀಂದ್ರಿಯಾವಸ್ಥೆಗೆ ಸಂಬಂಧಪಟ್ಟ ಅನಂತತೆ ನಮಗೆ ಸಂಪೂರ್ಣ ಹೊಸದಾಗಿರುವುದು. ಅವುಗಳ ಮೂಲಕ ಭಾವಗಳು ಬಂದರೆ ಇಡೀ ನಮ್ಮ ಶರೀರದ ಸ್ವಾಸ್ಥ್ಯವನ್ನು ಕೆಡಿಸಿ ಹೊಸ ದಾರಿಯನ್ನು ಮಾಡುವುದು. ಆದಕಾರಣವೇ ಸಾಧಾರಣ ಜನರು ಯೋಗಾಭ್ಯಾಸ ಮಾಡುವಾಗ ಮೊದಲಲ್ಲಿ ತಮ್ಮ ಆರೋಗ್ಯವನ್ನು ಕೆಡಿಸಿಕೊಳ್ಳುವರು. ಸಾಧ್ಯವಾದಷ್ಟೂ ಈ ತೊಂದರೆಗಳನ್ನು ಕಡಮೆ ಮಾಡುವುದಕ್ಕಾಗಿ ಪತಂಜಲಿಯು ಹಲವು ಮಾರ್ಗಗಳನ್ನು ಹೇಳುತ್ತಾನೆ. ಅದರಲ್ಲಿ ಯಾವುದು ನಮಗೆ ಸರಿಯೋ ಅದನ್ನು ಅಭ್ಯಾಸ ಮಾಡಬಹುದು. 

%%\vspace{-0.3cm}

\begin{verse}
ವಿಷಯವತೀ ವಾ ಪ್ರವೃತ್ತಿರುತ್ಪನ್ನಾ ಮನಸಃ ಸ್ಥಿತಿಬಂಧಿನೀ~॥ ೩೫~॥
\end{verse}

\vspace{-0.4cm}

\dsize{ಅಸಾಧಾರಣವಾದ ಇಂದ್ರಿಯಾನುಭವಗಳನ್ನು ತರುವಂತಹ ಏಕಾಗ್ರತೆಯ ಮಾನಸಿಕವಾದ ದೀರ್ಘ ಪ್ರಯತ್ನಕ್ಕೆ ಸಹಕಾರಿ. }

\vspace{0.2cm}

ಇದು ಸಾಧಾರಣವಾಗಿ ಧಾರಣದಿಂದ ಬರುವುದು. ಮೂಗಿನ ಕೊನೆಯ ಮೇಲೆ ಮನಸ್ಸು ಏಕಾಗ್ರವಾದರೆ, ಕೆಲವು ದಿನಗಳಾದ ಮೇಲೆ ಅದ್ಭುತ ಸುವಾಸನೆಯ ಅನುಭವವಾಗುತ್ತದೆ, ಎಂದು ಯೋಗಿಗಳು ಹೇಳುತ್ತಾರೆ. ನಾಲಗೆಯ ಬುಡದಲ್ಲಿ ಏಕಾಗ್ರವಾದರೆ ಧ್ವನಿ ಕೇಳಲುಪಕ್ರಮಿಸುವುದು. ನಾಲಗೆಯ ಕೊನೆಯ ಮೇಲಾದರೆ ಅನೇಕ ರುಚಿಯಾದ ಪದಾರ್ಥಗಳ ಸವಿ ಉಂಟಾಗುತ್ತದೆ. ನಾಲಗೆಯ ಮಧ್ಯದಲ್ಲಿ ಆದರೆ ಯಾವುದೋ ಒಂದು ವಸ್ತುವಿನ ಸಂಯೋಗ ಉಂಟಾದಂತೆ ಅನುಭವವಾಗುತ್ತದೆ. ಗಂಟಲಿನ ಮೇಲೆ ಏಕಾಗ್ರಮಾಡಿದರೆ ವಿಚಿತ್ರ ವಸ್ತುಗಳನ್ನು ನೋಡಲು ಮೊದಲು ಮಾಡುತ್ತೇವೆ. ಮನಸ್ಸು ಚಂಚಲವಾಗಿರುವವನು ಈ ಯೋಗಸಾಧನೆಗಳನ್ನು ಮಾಡಬಯಸಿದರೆ, ಆದರೆ ಇವುಗಳ ಸತ್ಯದಲ್ಲಿ ಸಂದೇಹವಿದ್ದರೆ, ಸ್ವಲ್ಪ ಅಭ್ಯಾಸ ಮಾಡಿದರೆ ಇವು ಪ್ರಾಪ್ತವಾಗುವುವು. ಸಂದೇಹ ನಿವಾರಣೆಯಾಗಿ ದೃಢಪ್ರಯತ್ನದಲ್ಲಿ ಮುಂದುವರಿಯುತ್ತಾನೆ. 

\vspace{-0.15cm}

\begin{verse}
ವಿಶೋಕಾ ವಾ ಜ್ಯೋತಿಷ್ಮತೀ~॥ ೩೬~॥
\end{verse}

\vspace{-0.4cm}

\dsize{ಅಥವಾ ಶೋಕದೂರವಾದ ಉಜ್ವಲ ಬೆಳಕನ್ನು (ಧ್ಯಾನಿಸುವುದರ ಮೂಲಕ)–}

\vspace{0.2cm}

ಇದು ಒಂದು ಬಗೆಯ ಏಕಾಗ್ರತೆ. ಹೃದಯ ಕಮಲದ ದಳಗಳೆಲ್ಲ ಕೆಳಮುಖವಾಗಿವೆ; ಸುಷುಮ್ನಾ ಕಾಲುವೆ ಅಲ್ಲಿ ಹರಿಯುತ್ತಿದೆ ಎಂದು ಭಾವಿಸಿ. ಉಸಿರನ್ನು ತೆಗೆದುಕೊಂಡು ಹೊರಗೆ ಬಿಡುವಾಗ ಕಮಲವು ದಳಸಹಿತ ಮೇಲು ಮುಖವಾಗಿದೆ. ಅದರಲ್ಲಿ ಒಂದು ಸ್ವಯಂಪ್ರಭೆ ಇದೆ ಎಂದು ಭಾವಿಸಿ. ಅದರ ಮೇಲೆ ಧ್ಯಾನ ಮಾಡಿ. 

\vspace{-0.1cm}

\begin{verse}
ವೀತರಾಗವಿಷಯಂ ವಾ ಚಿತ್ತಮ್​~॥ ೩೭~॥
\end{verse}

\vspace{-0.4cm}

\dsize{ವಿಷಯ ಭೋಗಗಳ ಮೇಲಿರುವ ಆಸೆಯನ್ನು ಸಂಪೂರ್ಣ ತೊರೆದವರ ಹೃದಯದ (ಮೇಲೆ ಧ್ಯಾನಿಸುವುದರ ಮೂಲಕ)–}

\vspace{0.2cm}

ಪೂರ್ಣ ಅನಾಸಕ್ತರೆಂದು ತೋರುವ ಯಾರಾದರೂ ಪವಿತ್ರಾತ್ಮರನ್ನೊ ಮಹಾನುಭಾವರನ್ನೊ ಆರಿಸಿಕೊಂಡು ಅವರ ಹೃದಯದ ಮೇಲೆ ಧ್ಯಾನಿಸಿ. ಅವರ ಹೃದಯವೂ ಅನಾಸಕ್ತವಾದದ್ದು. ಅದರಿಂದ ಮನಸ್ಸಿಗೆ ಶಾಂತಿ ದೊರಕುತ್ತದೆ. ನಿಮಗೆ ಇದು ಮಾಡಲು ಸಾಧ್ಯವಿಲ್ಲದೆ ಇದ್ದರೆ ಬೇರೆ ದಾರಿ ಇದೆ:

\vspace{-0.1cm}

\begin{verse}
ಸ್ವಪ್ನನಿದ್ರಾಜ್ಞಾನಾಲಂಬನಂ ವಾ~॥ ೩೮~॥
\end{verse}

\vspace{-0.4cm}

\dsize{ಅಥವಾ ನಿದ್ರಾಕಾಲದಲ್ಲಿ ಬರುವ ಜ್ಞಾನದ ಮೇಲೆ ಧ್ಯಾನ ಮಾಡುವುದರ ಮೂಲಕ–}

\newpage

ಕೆಲವು ವೇಳೆ ದೇವತೆಗಳು ಬಂದು ತನ್ನ ಹತ್ತಿರ ಮಾತನಾಡುತ್ತಿದ್ದಂತೆ ಒಬ್ಬ ಕನಸು ಕಾಣುತ್ತಾನೆ. ಆಗ ಆಕಾಶದಲ್ಲಿ ತೇಲಿಬರುತ್ತಿರುವ ಸಂಗೀತವನ್ನು ಕೇಳಿ ಆನಂದಭರಿತನಾಗುವನು. ಸ್ವಪ್ನದಲ್ಲಿರುವ ತನಕ ಅವನು ಸಂತೋಷ ಸಾಗರದಲ್ಲಿರುವನು. ಅವನು ಜಾಗ್ರತನಾದ ಮೇಲೆ ಸ್ವಪ್ನವು ಮನಸ್ಸಿನ ಮೇಲೆ ಗಾಢವಾದ ಪರಿಣಾಮವನ್ನು ಉಂಟುಮಾಡಬಹುದು. ಈ ಸ್ವಪ್ನವನ್ನು ನಿಜವೆಂದು ಭಾವಿಸಿ, ಅದರ ಮೇಲೆ ಧ್ಯಾನಿಸಿ. ಇದು ನಿಮಗೆ ಸಾಧ್ಯವಿಲ್ಲದೇ ಇದ್ದರೆ, ನಿಮಗೆ ಸಮಾಧಾನವನ್ನು ಕೊಡುವ ಮತ್ತಾವುದಾದರೊಂದು ಪುಣ್ಯವಸ್ತುವಿನ ಮೇಲೆ ಧ್ಯಾನಿಸಿ. 

\vspace{-0.29cm}

\begin{verse}
ಯಥಾಽಭಿಮತಧ್ಯಾನಾದ್ವಾ~॥ ೩೯~॥
\end{verse}

\vspace{-0.4cm}

\dsize{ತಮಗೆ ಒಳ್ಳೆಯದೆಂದು ತೋರುವ ಯಾವುದಾದರೊಂದು ವಸ್ತುವಿನ ಮೇಲೆ ಧ್ಯಾನ ಮಾಡುವುದರ ಮೂಲಕ–}

\vspace{0.1cm}

ಅಂದರೆ ಯಾವುದಾದರೊಂದು ಕೆಟ್ಟ ವಿಷಯವನ್ನು ಕುರಿತು ಧ್ಯಾನಿಸು ಎಂದು ಅರ್ಥವಲ್ಲ. ಹೀಗೆ ಪವಿತ್ರವೆಂದು ತೋರುವ ಯಾವುದಾದರೊಂದು ವಸ್ತುವನ್ನು ಕುರಿತು, ನಿಮಗೆ ಅತ್ಯಂತ ಪ್ರಿಯಕರವಾದ ಸ್ಥಳವನ್ನು ಕುರಿತು, ಪ್ರಿಯಕರವಾದ ದೃಶ್ಯವನ್ನು ಕುರಿತು, ಪ್ರಿಯಕರವಾದ ಭಾವನೆಯನ್ನು ಕುರಿತು, ಅಥವಾ ಮನಸ್ಸಿನ ಏಕಾಗ್ರತೆಗೆ ಸಹಾಯಮಾಡುವ ಯಾವುದಾದರೊಂದು ವಸ್ತುವನ್ನು ಕುರಿತು ಧ್ಯಾನಿಸಿ. 

\vspace{-0.25cm}

\begin{verse}
ಪರಮಾಣು–ಪರಮಮಹತ್ತ್ವಾನ್ತೋಽಸ್ಯ ವಶೀಕಾರಃ~॥ ೪೦~॥
\end{verse}

\vspace{-0.4cm}

\dsize{ಹೀಗೆ ಮಾಡುವುದರಿಂದ ಯೋಗಿಯ ಮನಸ್ಸು ಪರಮಾಣುವಿನಿಂದ ಹಿಡಿದು ಅತಿ ಮಹತ್ತ್ವವಾದ ವಸ್ತುವಿನವರೆಗೆ ನಿರಾತಂಕವಾಗುವುದು. }

\vspace{0.1cm}

ಈ ಅಭ್ಯಾಸದಿಂದ ಮನಸ್ಸು ಅತ್ಯಂತ ಕ್ಷುದ್ರತಮವಾದ ವಸ್ತುವಿನಿಂದ ಅತಿ ಮಹೋನ್ನತ ವಸ್ತುವಿನವರೆಗೆ ಎಲ್ಲವನ್ನೂ ಅತಿ ಸುಲಭವಾಗಿ ಧ್ಯಾನಿಸಬಲ್ಲದು. ಹೀಗೆ ಮನಸ್ಸಿನ ವೃತ್ತಿಗಳು ನಿರ್ಬಲವಾಗುತ್ತವೆ. 

\vspace{-0.3cm}

\begin{verse}
ಕ್ಷೀಣವೃತ್ತೇರಭಿಜಾತಸ್ಯೇವ ಮಣೇರ್ಗ್ರಹೀತೃ–ಗ್ರಹಣ–\\ಗ್ರಾಹ್ಯೇಷು ತತ್ಸ್ಥತದಂಜನತಾ ಸಮಾಪತ್ತಿಃ~॥ ೪೧~॥
\end{verse}

\vspace{-0.4cm}

\dsize{ಚಿತ್ತಾವೃತ್ತಿಯು ತನ್ನ ವಶವಾದ ಯೋಗಿಯು ಜ್ಞಾನವನ್ನು, ಜ್ಞಾತಾ ಕರಣ ಮತ್ತು ಜ್ಞಾನವಸ್ತು (ಆತ್ಮ, ಮನಸ್ಸು ಮತ್ತು ಬಾಹ್ಯವಸ್ತು) ಇವುಗಳಲ್ಲಿ, ಬಣ್ಣದ ವಸ್ತುಗಳ ಮುಂದಿರುವ ಸ್ಫಟಿಕ ಮಣಿಯಂತೆ, ಏಕಾಗ್ರತೆಯನ್ನೂ ಸಾಮ್ಯವನ್ನೂ ಪಡೆಯುತ್ತಾನೆ. }

\vspace{0.1cm}

ಈ ನಿರಂತರ ಧ್ಯಾನದಿಂದ ಏನು ಪ್ರಯೋಜನವಾಗುತ್ತದೆ? ಹಿಂದಿನ ಸೂತ್ರಗಳಲ್ಲಿ ಪತಂಜಲಿಯು ಹೇಗೆ ಸೂಕ್ಷ್ಮ, ಸೂಕ್ಷ್ಮತರ, ಸೂಕ್ಷ್ಮಸ್ಥಿತಿಯ ಧ್ಯಾನಗಳನ್ನು ವಿವರಿಸಿರುವನು ಎಂಬುದನ್ನು ಜ್ಞಾಪಕದಲ್ಲಿಡಬೇಕು. ಸ್ಥೂಲ ವಸ್ತುವಿನ ಮೇಲೆ ಧ್ಯಾನ ಮಾಡುವಷ್ಟೇ ಸುಲಭವಾಗಿ ಸೂಕ್ಷ್ಮವಸ್ತುವಿನ ಮೇಲೆ ಧ್ಯಾನಮಾಡಲು ಸಾಧ್ಯವಾಗದೆ ಇದರ ಪ್ರಯೋಜನ. ಯೋಗಿಯು ಇಲ್ಲಿ ಮೂರು ವಿಷಯಗಳನ್ನು ನೋಡುತ್ತಾನೆ: ಗ್ರಹೀತೃ, ಗ್ರಹಣ ಮತ್ತು ಗ್ರಾಹ್ಯ ಅಂದರೆ ಆತ್ಮ, ಮನಸ್ಸು, ಮತ್ತು ಬಾಹ್ಯವಸ್ತು. ನಮಗೆ ಮೂರು ಧ್ಯಾನದ ವಸ್ತುಗಳಿವೆ. ಮೊದಲನೆಯದು ದೇಹ ಅಥವಾ ಬಾಹ್ಯ ಸ್ಥೂಲವಸ್ತುಗಳು, ಎರಡನೆಯದು ಸೂಕ್ಷ್ಮವಾದ ಮನಸ್ಸು, ಮೂರನೆಯದು ವಿಶೇಷಣಗಳಿಂದ ಕೂಡಿದ ಪುರುಷ, ನಿಜವಾದ ಪುರುಷನಲ್ಲ, ಅಂದರೆ ಅಹಂಕಾರ. ಅಭ್ಯಾಸದ ಬಲದಿಂದ ಯೋಗಿಯು ಯಾವಾಗ ಧ್ಯಾನ ಮಾಡಿದರೂ ಉಳಿದ ಎಲ್ಲಾ ಆಲೋಚನೆಗಳನ್ನೂ ಮರೆಯಬಲ್ಲ. ಯಾವುದರ ಮೇಲೆ ಧ್ಯಾನ ಮಾಡುತ್ತಾನೆಯೋ ಅದರ ತದಾತ್ಮ್ಯವನ್ನೇ ಪಡೆಯುತ್ತಾನೆ. ಧ್ಯಾನಮಾಡುವಾಗ ಅವನು ಒಂದು ಸ್ಫಟಿಕಮಣಿಯಂತೆ ಇರುತ್ತಾನೆ. ಮುಂದಿರುವ ಹೂವಿನ ಬಣ್ಣವನ್ನೇ ಸ್ಫಟಿಕಮಣಿ ತಾಳಿದಂತೆ ಕಾಣುವುದು. ಹೂವು ಕೆಂಪಾಗಿದ್ದರೆ ಸ್ಫಟಿಕ ಮಣಿಯೂ ಕೆಂಪಾಗಿ ಕಾಣುವುದು. ಹೂವು ನೀಲಿಯಾಗಿದ್ದರೆ ಸ್ಫಟಿಕಮಣಿಯೂ ಕೂಡ ನೀಲಿಬಣ್ಣವನ್ನೇ ತಾಳುವುದು. 

\vspace{-0.3cm}

\begin{verse}
ತತ್ರ ಶಬ್ಧಾರ್ಥಜ್ಞಾನವಿಕಲ್ಪೈಃ ಸಂಕೀರ್ಣಾ ಸವಿತರ್ಕಾ ಸಮಾಪತ್ತಿಃ~॥ ೪೨~॥
\end{verse}

\vspace{-0.4cm}

\dsize{ಶಬ್ದ, ಅರ್ಥ, ಜ್ಞಾನ–ಈ ವಿಕಲ್ಪಗಳಿಂದ ಕೂಡಿದ ಸಮಾಧಿಗೆ ಸವಿತರ್ಕ ಸಮಾಧಿ ಎಂದು ಹೆಸರು. }

\vspace{0.1cm}

ಇಲ್ಲಿ ಶಬ್ದವೆಂದರೆ ಸ್ಪಂದನ; ಅರ್ಥವೆಂದರೆ ಸ್ಪಂದನವನ್ನು ಒಳಗೆ ಒಯ್ಯುವ ನರಗಳ ಶಕ್ತಿ, ಜ್ಞಾನವೆಂದರೆ ಒಳಗೆ ಉಂಟಾಗುವ ಪ್ರತಿಕ್ರಿಯೆ. ಇಲ್ಲಿಯವರೆವಿಗೂ ನಾವು ಓದಿದ ಹಲವು ವಿಧದ ಧ್ಯಾನಗಳನ್ನು ಪತಂಜಲಿ ಸವಿತರ್ಕವೆನ್ನುತ್ತಾನೆ. ಅನಂತರ ಉತ್ತಮ ತರದ ಧ್ಯಾನಕ್ಕೆ ಸಲಹೆಗಳನ್ನು ಕೊಡುತ್ತಾನೆ. ಸವಿತರ್ಕದಲ್ಲಿ ವಸ್ತು ಮತ್ತು ಅದನ್ನು ನೋಡುವವರ ಭೇದ ಭಾವನೆಯನ್ನು ಉಳಿಸಿಕೊಳ್ಳುತ್ತೇವೆ. ಇದು ಶಬ್ದ ಅರ್ಥ ಜ್ಞಾನಗಳ ಮಿಶ್ರದಿಂದ ಆಗುವುದು. ಮೊದಲನೆಯದಾಗಿ ಶಬ್ದವೆಂಬ ಬಾಹ್ಯ ಸ್ಪಂದನವಿದೆ, ಇದನ್ನು ಮಿದುಳಿಗೆ ಒಯ್ಯುವ ಇಂದ್ರಿಯ ಶಕ್ತಿಗೆ ಅರ್ಥವೆಂದು ಹೆಸರು. ಇದಾದಮೇಲೆ ಚಿತ್ತದಲ್ಲಿ ಪ್ರತಿಕ್ರಿಯೆಯ ಅಲೆ ಏಳುತ್ತದೆ. ಅದೇ ಜ್ಞಾನ. ಆದರೆ ನಾವು ಜ್ಞಾನವೆನ್ನುವುದು ಈ ಮೂರರ ಮಿಶ್ರ. ಇಲ್ಲಿಯವರೆವಿಗೂ ಇರುವ ಎಲ್ಲಾ ಧ್ಯಾನದಲ್ಲಿಯೂ ಈ ಮಿಶ್ರವೇ ನಮ್ಮ ಧ್ಯಾನದ ಗುರಿಯಾಗಿರುವುದು. ಅನಂತರ ಬರುವ ಸಮಾಧಿಯು ಮೇಲಿನದು. 

\vspace{-0.3cm}

\begin{verse}
ಸ್ಮೃತಿಪರಿಶುದ್ಧೌಸ್ವರೂಪಶೂನ್ಯೇವಾರ್ಥಮಾತ್ರ ನಿರ್ಭಾಸಾ ನಿರ್ವಿತರ್ಕಾ~॥~೪೩~॥
\end{verse}

\vspace{-0.4cm}

\dsize{ಸ್ಮೃತಿಯು ಪರಿಶುದ್ಧವಾದ ಮೇಲೆ, ಅಥವಾ ಅದು ನಿರ್ಗುಣವಾಗಿ ಧ್ಯಾನಿಸುವ ವಸ್ತುವಿನ ಅರ್ಥವನ್ನು ಮಾತ್ರ ತಿಳಿಸುತ್ತಿದ್ದರೆ ಆ ಸಮಾಧಿಗೆ ನಿರ್ವಿತರ್ಕವೆಂದು ಹೆಸರು. }

\vspace{0.1cm}

ಶಬ್ದ, ಅರ್ಥ, ಜ್ಞಾನ ಇವುಗಳ ಧ್ಯಾನವನ್ನು ಅಭ್ಯಾಸ ಮಾಡಿದಮೇಲೆ, ಈ ಮೂರು ಮಿಶ್ರವಾಗದೇ ಇರುವ ಸ್ಥಿತಿಗೆ ಬರುತ್ತೇವೆ. ಮೊದಲು ಈ ಮೂರು ಏನೆಂಬುದನ್ನು ತಿಳಿದುಕೊಳ್ಳಲು ಯತ್ನಿಸೋಣ. ಇಲ್ಲಿ ಚಿತ್ತವಿದೆ ಎಂದು ಇಟ್ಟು ಕೊಳ್ಳೋಣ. ಚಿತ್ತವು ಒಂದು ಸರೋವರದಂತೆ ಎಂಬುದನ್ನು ಯಾವಾಗಲೂ ನೀವು ನೆನಪಿನಲ್ಲಿಡಬೇಕು. ಶಬ್ದವೆನ್ನುವುದು ಅದರ ಮೇಲೆ ಏಳುವ ಒಂದು ಅಲೆಯಂತೆ. ನಿಮ್ಮಲ್ಲಿ ಶಾಂತ ಸರೋವರವಿದೆ. ನಾನು ಹಸು ಎಂಬ ಪದವನ್ನು ಉಚ್ಚರಿಸುತ್ತೇನೆ. ಈ ಶಬ್ದ ನಿಮ್ಮ ಕಿವಿಗೆ ಬಿದ್ದ ತಕ್ಷಣವೇ ನಿಮ್ಮ ಚಿತ್ತದಲ್ಲಿ ಒಂದು ಅಲೆ ಏಳುತ್ತದೆ. ಆ ಅಲೆ ಹಸು ಎಂಬ ಭಾವನೆಯನ್ನು ತರುತ್ತದೆ. ಅದನ್ನೇ ನಾವು ಆಕಾರ ಅಥವಾ ಅರ್ಥವೆನ್ನುವುದು. ನಿಮಗೆ ತಿಳಿದಿರುವ ಹಸು ಎಂಬ ಭಾವನೆ ಬಾಹ್ಯ ಮತ್ತು ಅಂತರಂಗ ಶಬ್ದದ ಪರಿಣಾಮವಾಗಿ ಚಿತ್ತದಲ್ಲಿ ಆದ ಪ್ರತಿಕ್ರಿಯೆ. ಶಬ್ದದೊಂದಿಗೆ ಅಲೆಯೂ ಮಾಯವಾಗುತ್ತದೆ. ಅಲೆ ಶಬ್ದವಿಲ್ಲದೆ ಇರಲಾರದು. ನಾವು ಹಸುವನ್ನು ಆಲೋಚಿಸುತ್ತೇವೆ; ಆದರೆ ಶಬ್ದವನ್ನು ಕೇಳುವುದಿಲ್ಲ. ಇದು ಹೇಗೆ ಎಂದು ನೀವು ಕೇಳಬಹುದು. ನೀವೇ ಆ ಶಬ್ದವನ್ನು ಮಾಡುವಿರಿ. ಮನಸ್ಸಿನಲ್ಲಿ ಮೆಲ್ಲಗೆ ಹಸು ಎಂದು ಹೇಳುತ್ತೀರಿ, ಆಗ ಇದರೊಂದಿಗೆ ಅದಕ್ಕೆ ಸಂಬಂಧಪಟ್ಟ ಅಲೆಯೇಳುತ್ತದೆ. ಈ ಶಬ್ದದ ಪ್ರೇರಣೆಯಿಲ್ಲದೆ ಯಾವ ಅಲೆಯೂ ಏಳುವುದಿಲ್ಲ. ಹೊರಗಿನಿಂದ ಶಬ್ದವಿಲ್ಲದೇ ಇದ್ದರೆ ಒಳಗಿನಿಂದ ಶಬ್ದವಿರುತ್ತದೆ. ಶಬ್ದ ಮಾಯವಾದರೆ ಅಲೆಯೂ ಮಾಯವಾಗುವುದು. ಅನಂತರ ಉಳಿಯುವುದೇನು? ಪ್ರತಿಕ್ರಿಯೆಯ ಪರಿಣಾಮ ಉಳಿಯುವುದು. ಅದೇ ಜ್ಞಾನ. ನಮ್ಮ ಮನಸ್ಸಿನಲ್ಲಿ ಈ ಮೂರು ಅತಿ ನಿಕಟವಾಗಿ ಸಂಬಂಧ ಪಡೆದಿರುವುದರಿಂದ ನಾವು ಇವನ್ನು ವಿಭಾಗಿಸಲಾರೆವು. ಶಬ್ದ ಬಂದಾಗ ಇಂದ್ರಿಯ ಸ್ಪಂದಿಸುವುದು. ಇದರ ಪ್ರತಿಕ್ರಿಯೆಯಾಗಿ ಅಲೆ ಏಳುವುದು. ಇವು ಒಂದಾದ ಮೇಲೊಂದು ತಕ್ಷಣ ಬರುವುದರಿಂದ ಒಂದನ್ನು ಮತ್ತೊಂದರಿಂದ ಪ್ರತ್ಯೇಕವಾಗಿ ವಿಭಾಗಿಸಲು ಸಾಧ್ಯವಿಲ್ಲ. ಬಹಳ ಕಾಲ ಈ ಧ್ಯಾನಾಭ್ಯಾಸ ಮಾಡಿದ ಮೇಲೆ ಎಲ್ಲ ಸಂಸ್ಕಾರಗಳ ಆಶಯವಾದ ಸ್ಮೃತಿಯು ಪರಿಶುದ್ಧವಾಗುತ್ತದೆ. ಆಗ ಅವುಗಳನ್ನು ಒಂದರಿಂದ ಒಂದನ್ನು ಪ್ರತ್ಯೇಕಿಸಲು ಸಾಧ್ಯವಾಗುತ್ತದೆ. ಇದನ್ನೇ ನಿರ್ವಿತರ್ಕವೆನ್ನುವುದು, ಪ್ರಶ್ನಿಸದೆ ಮನಸ್ಸನ್ನು ಏಕಾಗ್ರಗೊಳಿಸುವುದು. 

\vspace{-0.3cm}

\begin{verse}
ಏತಯೈವ ಸವಿಚಾರಾ ನಿರ್ವಿಚಾರಾ ಚ ಸೂಕ್ಷ್ಮವಿಷಯಾ ವ್ಯಾಖ್ಯಾತಾ~॥ ೪೪~॥
\end{verse}

\vspace{-0.4cm}

\dsize{ಇದರಿಂದ ಸೂಕ್ಷ್ಮ ವಿಷಯಗಳನ್ನೊಳಗೊಂಡ ಸವಿಚಾರ ಮತ್ತು ನಿರ್ವಿಚಾರ ಧ್ಯಾನಗಳನ್ನು ವಿವರಿಸಿದಂತೆ ಆಯಿತು. }

\vspace{0.1cm}

ಹಿಂದಿನ ವಿಧಾನವೇ ಇಲ್ಲಿಗೆ ಅನ್ವಯಿಸುತ್ತದೆ. ಹಿಂದಿನ ಧ್ಯಾನದ ಗುರಿ ಸ್ಥೂಲವಾಗಿತ್ತು. ಇಲ್ಲಿ ಅದು ಸೂಕ್ಷ್ಮವಾಗಿದೆ. 

\vspace{-0.25cm}

\begin{verse}
ಸೂಕ್ಷ್ಮವಿಷಯತ್ವಂ ಚಾಲಿಂಗಪರ್ಯವಸಾನಮ್​~॥ ೪೫~॥
\end{verse}

\vspace{-0.4cm}

\dsize{ಸೂಕ್ಷ್ಮ ವಿಷಯಗಳು ಪ್ರಧಾನದೊಂದಿಗೆ ಕೊನೆಗಾಣುವುವು. }

\vspace{0.2cm}

ಸ್ಥೂಲವಸ್ತುವೇ ಭೂತಗಳು. ಎಲ್ಲಾ ವಸ್ತುಗಳೂ ಇದರಿಂದ ತಯಾರಾಗಿರುವುವು. ಸೂಕ್ಷ್ಮ ವಸ್ತು ಮೊದಲಾಗುವುದು ತನ್ಮಾತ್ರದಿಂದ. ಇಂದ್ರಿಯ, ಮನಸ್ಸು, ಅಹಂಕಾರ, ಚಿತ್ತ (ಎಲ್ಲಾ ಅಭಿವ್ಯಕ್ತಿಗೂ ಇದೇ ಮೂಲಕಾರಣ), ಸತ್ತ್ವ ರಜಸ್ಸು ತಮಸ್ಸುಗಳ ಸಮಾವಸ್ಥೆಯಾದ ಪ್ರಕೃತಿ ಅಥವಾ ಅವ್ಯಕ್ತ–ಇವುಗಳೆಲ್ಲವೂ ಕೂಡ ಸೂಕ್ಷ್ಮ ವಸ್ತುವಿನ ಗುಂಪಿಗೆ ಸೇರಿವೆ. ಪುರುಷ (ಅಂದರೆ ಆತ್ಮ) ಮಾತ್ರ ಇದರೊಂದಿಗೆ ಸೇರಿಲ್ಲ.

\begin{verse}
ತಾ ಏವ ಸಬೀಜಃ ಸಮಾಧಿಃ~॥ ೪೬~॥
\end{verse}

\vspace{-0.4cm}

\dsize{ಈ ಏಕಾಗ್ರತೆ ಬೀಜದಿಂದ ಕೂಡಿದೆ. }

\newpage

ಇವು ಪೂರ್ವಕರ್ಮದ ಬೀಜವನ್ನು ನಾಶಮಾಡುವುದಿಲ್ಲ. ಆದಕಾರಣ ಮೋಕ್ಷವನ್ನು ಕೊಡಲಾರವು. ಆದರೆ ಯೋಗಿಗೆ ಇವುಗಳಿಂದ ಪ್ರಯೋಜನವೇನೆಂಬುದನ್ನು ಮುಂದಿನ ಸೂತ್ರವು ಹೇಳುತ್ತದೆ. 

\vspace{-0.2cm}

\begin{verse}
ನಿರ್ವಿಚಾರ–ವೈಶಾರದ್ಯೇಽಧ್ಯಾತ್ಮಪ್ರಸಾದಃ~॥ ೪೭~॥
\end{verse}

\vspace{-0.4cm}

\dsize{ನಿರ್ವಿಚಾರಧ್ಯಾನವು ಪರಿಶುದ್ಧವಾದ ಮೇಲೆ ಚಿತ್ತವು ದೃಢ ಪ್ರತಿಷ್ಠವಾಗುತ್ತದೆ. }

\vspace{-0.1cm}

\begin{verse}
ಋತಂಭರಾ ತತ್ರ ಪ್ರಜ್ಞಾ~॥ ೪೮~॥
\end{verse}

\vspace{-0.4cm}

\dsize{ಅದರಲ್ಲಿರುವ ಜ್ಞಾನಕ್ಕೆ ‘ಋತಂಭರಾ’ (ಸತ್ಯದಿಂದ ಪೂರ್ಣವಾದುದು) ಎಂದು ಹೆಸರು. }

\vspace{0.2cm}

ಮುಂದಿನ ಸೂತ್ರ ಇದನ್ನು ವಿವರಿಸುತ್ತದೆ. 

\vspace{-0.2cm}

\begin{verse}
ಶ್ರುತಾನುಮಾನಪ್ರಜ್ಞಾಭ್ಯಾಮನ್ಯವಿಷಯಾ ವಿಶೇಷಾರ್ಥತ್ವಾತ್​~॥ ೪೯~॥
\end{verse}

\vspace{-0.4cm}

\dsize{ಶ್ರುತಿ ಅನುಮಾನಗಳಿಂದ ಪಡೆದ ಜ್ಞಾನವು ಸಾಮಾನ್ಯ ವಿಷಯಕ್ಕೆ ಸಂಬಂಧಪಟ್ಟಿರುವುದು. ಈಗ ತಾನೆ ಹೇಳಿದ ಸಮಾಧಿಯ ಮೂಲಕ ಬರುವ ಜ್ಞಾನವು ಅದಕ್ಕಿಂತ ಮೇಲಿನ ವರ್ಗದ್ದು. ಶ್ರುತಿ ಮತ್ತು ಅನುಮಾನಕ್ಕೆ ಪ್ರವೇಶವಿಲ್ಲದ ಕಡೆಯೂ ಕೂಡ ಇದು ಹೋಗಬಲ್ಲದು. }

\vspace{0.1cm}

ಸಾಧಾರಣ ವಸ್ತುವಿನ ಪ್ರಮಾಣವನ್ನು ನಾವು ಪ್ರತ್ಯಕ್ಷ, ಅನುಮಾನ ಮತ್ತು ಯೋಗ್ಯರ ಹೇಳಿಕೆಗಳಿಂದ ಪಡೆಯಬಹುದು. ಯೋಗ್ಯರಾದವರೆಂದರೆ, ಋಷಿಗಳು ಅಥವಾ ತಮ್ಮ ಅನುಭವವನ್ನು ವೇದದಲ್ಲಿ ಬರೆದ ಮಂತ್ರದ್ರಷ್ಟರು ಎಂದು ಯೋಗಿಯ ಅರ್ಥ. ಇಂತಹ ಅನುಭವವನ್ನು ಪಡೆದವರ ಆಪ್ತವಾಕ್ಯಗಳಾಗಿರುವುದೇ ಶ್ರುತಿ ಪ್ರಮಾಣ. ಆದರೂ ಕೂಡ ಶ್ರುತಿಯು ಮೋಕ್ಷವನ್ನು ಕೊಡಲಾರದೆಂದು ಅವರು ಹೇಳುತ್ತಾರೆ. ನಾವು ವೇದಗಳನ್ನೆಲ್ಲ ಓದಬಹುದು. ಆದರೂ ಸ್ವಲ್ಪವೂ ಲಾಭವನ್ನು ಪಡೆಯದೆ ಇರಬಹುದು. ಆದರೆ ನಾವು ಅದರ ಬೋಧನೆಯನ್ನು ಅಭ್ಯಾಸ ಮಾಡಿದಾಗ, ಎಲ್ಲಿ ಯುಕ್ತಿ, ಇಂದ್ರಿಯಗ್ರಹಣ, ಅನುಮಾನಗಳು ಪ್ರವೇಶಿಸಲಾರವೊ, ಎಲ್ಲಿ ಆಪ್ತವಾಕ್ಯ ಪ್ರಯೋಜನಕ್ಕೆ ಬರುವುದಿಲ್ಲ ಎಂದು ಶ್ರುತಿ ಸಾರುವುದೊ ಆ ಸ್ಥಿತಿಗೆ ನಾವು ಬರುತ್ತೇವೆ. 

ಸಾಕ್ಷಾತ್ಕಾರವೇ ನಿಜವಾದ ಧರ್ಮ, ಉಳಿದುವೆಲ್ಲ ಇದಕ್ಕೆ ಸಿದ್ಧತೆಗಳು. ಉಪನ್ಯಾಸ ಕೇಳುವುದು, ಪುಸ್ತಕ ಓದುವುದು, ಅಥವಾ ವಿಚಾರ ಮಾಡುವುದು, ಇವುಗಳೆಲ್ಲ ಕೇವಲ ಸಿದ್ಧತೆ ಮಾತ್ರ. ಇವೇ ಧರ್ಮವಲ್ಲ. ಬೌದ್ಧಿಕ ಆರೋಹಣ ಅವರೋಹಣಗಳು ಧರ್ಮವಲ್ಲ. ಹೇಗೆ ನಾವು ಇಂದ್ರಿಯಕ್ಕೆ ಸಂಬಂಧಪಟ್ಟ ವಸ್ತುವಿನ ಪ್ರತ್ಯಕ್ಷ ಅನುಭವವನ್ನು ಪಡೆಯಬಹುದೋ, ಅದರಂತೆಯೇ ಅದಕ್ಕಿಂತ ಹೆಚ್ಚಾಗಿ ನಾವು ಆಧ್ಯಾತ್ಮಿಕ ಸತ್ಯವನ್ನು ಗ್ರಹಿಸಬಹುದೆಂಬುದೇ ಯೋಗಿಗಳ ಮುಖ್ಯ ಅಭಿಪ್ರಾಯ. ಜೀವಾತ್ಮ ಪರಮಾತ್ಮರೆಂಬ ಧಾರ್ಮಿಕ ಸತ್ಯವನ್ನು ಬಾಹ್ಯೇಂದ್ರಿಯಗಳಿಂದ ನಾವು ಗ್ರಹಿಸಲಾರೆವು. ನನ್ನ ಕಣ್ಣಿನಿಂದ ದೇವರನ್ನು ನಾನು ನೋಡಲಾರೆ. ಅಥವಾ ಕೈಗಳಿಂದ ಅವನನ್ನು ಮುಟ್ಟಲು ಸಾಧ್ಯವಿಲ್ಲ. ಅತೀಂದ್ರಿಯ ವಸ್ತುವನ್ನು ನಾವು ಆಲೋಚಿಸುವುದು ಸಾಧ್ಯವಿಲ್ಲವೆಂಬುದು ನಮಗೆ ಗೊತ್ತಿದೆ. ಯಾವ ನಿರ್ಧಾರಕ್ಕೂ ಬರದ ಒಂದೆಡೆಯಲ್ಲಿ ಯುಕ್ತಿ ನಮ್ಮನ್ನು ತ್ಯಜಿಸುವುದು. ಸಾವಿರಾರು ವರ್ಷಗಳಿಂದ ಪ್ರಪಂಚದಲ್ಲಿ ನಡೆಯುತ್ತಿರುವಂತೆ ನಾವೂ ಕೂಡ ಬದುಕಿರುವವರೆಗೂ ವಿಚಾರದಲ್ಲಿ ಕಳೆಯಬಹುದು. ಇದರಿಂದ ಧಾರ್ಮಿಕ ಸತ್ಯವನ್ನು ಸಮರ್ಥಿಸುವುದಕ್ಕೆ ಅಥವಾ ವಿರೋಧಿಸುವುದಕ್ಕೆ ನಾವು ಅಸಮರ್ಥರಾಗುತ್ತೇವೆ. ಇದೇ ಅದರ ಪರಿಣಾಮ. ನಾವು ಪ್ರತ್ಯಕ್ಷವಾಗಿ ನೋಡಿದುದನ್ನು ತಳಹದಿಯಾಗಿ ಮಾಡಿಕೊಂಡು ಅದರ ಆಧಾರದ ಮೇಲೆ ನಾವು ವಿಚಾರ ಮಾಡುತ್ತೇವೆ. ಇದರಿಂದ ಇಂದ್ರಿಯ ಗ್ರಹಣದ ಮೇರೆಯಲ್ಲಿ ಮಾತ್ರ ಯುಕ್ತಿ ಕೆಲಸ ಮಾಡಬಲ್ಲದು ಎಂಬುದು ನಿಜವಾಯಿತು. ಇದನ್ನು ಮೀರಿ ಹೋಗಲಾರದು. ಆದಕಾರಣ ಆತ್ಮ ಸಾಕ್ಷಾತ್ಕಾರಕ್ಕೆ ಪ್ರವೇಶ ಇಂದ್ರಿಯ ಗ್ರಹಣಾತೀತವಾಗಿದೆ. ಮಾನವನು ಇಂದ್ರಿಯ ಅನುಭವಕ್ಕೆ ಅತೀತನಾಗಿ ಯುಕ್ತಿಯನ್ನೂ ಕೂಡ ಮೀರಿ ಹೋಗಬಲ್ಲನು ಎಂದು ಯೋಗಿಗಳು ಸಾರುತ್ತಾರೆ. ಮಾನವನಿಗೆ ತನ್ನ ಯುಕ್ತಿಯನ್ನು ಮೀರಿ ಹೋಗಬಲ್ಲ ಶಕ್ತಿ ಇದೆ. ಈ ಶಕ್ತಿ ಪ್ರತಿಯೊಬ್ಬ ಮಾನವನಲ್ಲಿಯೂ ಇದೆ. ಪ್ರತಿಯೊಂದು ಪ್ರಾಣಿಯಲ್ಲಿಯೂ ಇದೆ. ಯೋಗಾಭ್ಯಾಸದಿಂದ ಈ ಶಕ್ತಿ ಜಾಗ್ರತವಾಗುತ್ತದೆ. ಆಗ ಮಾನವನು ಯುಕ್ತಿಯ ಸಾಧಾರಣ ಮೇರೆಯನ್ನು ಮೀರುವನು. ಆಗ ಯುಕ್ತಿಯನ್ನು ಮೀರಿರುವ ಪರಮ ವಸ್ತುವನ್ನು ಸಂದರ್ಶಿಸುವನು. 

\vspace{-0.4cm}

\begin{verse}
ತಜ್ಜಃ ಸಂಸ್ಕಾರೋಽನ್ಯಸಂಸ್ಕಾರಪ್ರತಿಬನ್ಧೀ~॥ ೫೦~॥
\end{verse}

\vspace{-0.6cm}

\dsize{ಈ ಸಮಾಧಿಯಿಂದ ಬರುವ ಸಂಸ್ಕಾರವು ಉಳಿದ ಎಲ್ಲ ಸಂಸ್ಕಾರಗಳನ್ನೂ ತಡೆಯುವುದು. }

\vspace{0.2cm}

ಏಕಾಗ್ರತೆಯೊಂದೇ ಪ್ರಜ್ಞಾತೀತಾವಸ್ಥೆಗೆ ತಲುಪುವುದಕ್ಕೆ ಇರುವ ದಾರಿ ಎಂಬುದನ್ನು ಹಿಂದಿನ ಸೂತ್ರದಲ್ಲಿ ನೋಡಿದ್ದಾಯಿತು. ಮನಸ್ಸಿನ ಏಕಾಗ್ರತೆಯನ್ನು ಕೆಡಿಸುವುದು ಹಿಂದಿನ ಸಂಸ್ಕಾರಗಳು ಎಂಬುದೂ ಕೂಡ ನಮಗೆ ತಿಳಿಯಿತು. ನಿಮ್ಮ ಮನಸ್ಸನ್ನು ಕೇಂದ್ರೀಕರಿಸುವುದಕ್ಕೆ ಪ್ರಯತ್ನಿಸುತ್ತಿರುವಾಗ ನಿಮ್ಮ ಆಲೋಚನೆಗಳು ಚೆಲ್ಲಾಪಿಲ್ಲಿಯಾಗಿ ಹೋಗುವುದನ್ನು ನೀವೆಲ್ಲರೂ ನೋಡುತ್ತೀರಿ. ನೀವು ದೇವರನ್ನು ಧ್ಯಾನಮಾಡಲು ಪ್ರಯತ್ನಿಸುವ ಕಾಲವೆ ಸಂಸ್ಕಾರ ತಲೆ ಎತ್ತುವ ಸಮಯ. ಉಳಿದ ಸಮಯದಲ್ಲಿ ಇವು ಅಷ್ಟು ಪ್ರಬಲವಾಗಿರುವುದಿಲ್ಲ. ನಿಮಗೆ ಯಾವಾಗ ಅವು ಬೇಡವೊ ಆಗಲೆ ಅವು ಅಲ್ಲಿರುವುದಕ್ಕೆ ಸಿದ್ಧ. ನಿಮ್ಮ ಮನಸ್ಸನ್ನು ಮುತ್ತುವುದಕ್ಕೆ ತಮ್ಮ ಕೈಲಾದ ಪ್ರಯತ್ನವನ್ನೆಲ್ಲ ಮಾಡುವುವು. ಇದು ಏತಕ್ಕೆ ಹಾಗೆ ಇರಬೇಕು? ಮನಸ್ಸನ್ನು ಏಕಾಗ್ರಗೊಳಿಸುವ ಕಾಲದಲ್ಲೆ ಅವು ಏಕೆ ಹೀಗೆ ಆಗಬೇಕು? ಅದು ಏತಕ್ಕೆಂದರೆ ನೀವು ಸಂಸ್ಕಾರವನ್ನು ಅದುಮಲು ಪ್ರಯತ್ನಿಸುತ್ತೀರಿ. ಆದಕಾರಣವೆ ಅವು ತಮಗೆ ಸಾಧ್ಯವಾದಷ್ಟು ವಿರೋಧಿಸುತ್ತವೆ. ಬೇರೆ ಸಮಯದಲ್ಲಿ ಅವು ಅಷ್ಟು ವಿರೋಧಿಸುವುದಿಲ್ಲ. ಎಲ್ಲಿಯೋ ಚಿತ್ತದ ಗುಹೆಯಲ್ಲಿ ಹುಲಿಯಂತೆ ಅಡಗಿಕೊಂಡು ಮೇಲೆ ಬೀಳಲು ಸಿದ್ಧವಾಗಿರುವಂತಹ ಸಂಸ್ಕಾರಗಳು ಎಷ್ಟು ಹುದುಗಿಕೊಂಡಿವೆಯೋ! ನಮಗೆ ಬೇಕಾಗಿರುವ ಒಂದು ಆಲೋಚನೆ ಮಾತ್ರ ಮೇಲೇಳುವಂತೆ ಮಾಡಲು ಉಳಿದ ಆಲೋಚನೆಗಳನ್ನು ನಾವು ನಿಗ್ರಹಿಸಬೇಕು. ಅದರ ಬದಲು ಎಲ್ಲವೂ ಏಕಕಾಲದಲ್ಲಿ ಎದ್ದು ಬರಲು\break ಯತ್ನಿಸುತ್ತವೆ. ಮನಸ್ಸಿನ ಏಕಾಗ್ರತೆಗೆ ಅಡಚಣೆ ತರುವ ಸಂಸ್ಕಾರದ ಬಹು ವಿಧದ ಶಕ್ತಿಗಳಿವು. ಆದಕಾರಣ ಈಗ ತಾನೆ ಹೇಳಿದ ಸಮಾಧಿಯು ಅಭ್ಯಾಸಕ್ಕೆ ಅತ್ಯುತ್ತಮವಾದುದು. ಏಕೆಂದರೆ ಇದಕ್ಕೆ ಸಂಸ್ಕಾರವನ್ನು ಅಡಗಿಸುವ ಶಕ್ತಿ ಇದೆ. ಇಂತಹ ಧ್ಯಾನದ ಪರಿಣಾಮವಾಗಿ ಬರುವ ಸಂಸ್ಕಾರವು ಉಳಿದ ಸಂಸ್ಕಾರಗಳ ವೇಗವನ್ನು ತಡೆದು ಅವನ್ನು ನಿಗ್ರಹಿಸುವಷ್ಟು ಬಲವಾದುದು. 

\vspace{-0.25cm}

\begin{verse}
ತಸ್ಯಾಪಿ ನಿರೋಧೇ ಸರ್ವನಿರೋಧಾನ್ನಿರ್ಬೀಜಃ ಸಮಾಧಿಃ~॥~೫೧~॥
\end{verse}

\vspace{-0.42cm}

\dsize{(ಉಳಿದ ಎಲ್ಲಾ ಸಂಸ್ಕಾರವನ್ನು ಅಡಗಿಸುವ) ಈ ಸಂಸ್ಕಾರವನ್ನೂ ನಿರೋಧಿಸುವುದರಿಂದ ಎಲ್ಲಾ ಸಂಸ್ಕಾರವನ್ನು ನಿರೋಧಿಸಿದಂತೆ ಆಯಿತು. ಆಗ ನಿರ್ಬೀಜ ಸಮಾಧಿ ಬರುವುದು. }

\vspace{0.2cm}

ಆತ್ಮನನ್ನು ನೋಡುವುದೇ ನಮ್ಮ ಗುರಿ ಎಂಬುದು ನಿಮಗೆ ಗೊತ್ತಿದೆ. ಆತ್ಮನು ನಮಗೆ ಕಾಣುವುದಿಲ್ಲ. ಏಕೆಂದರೆ, ದೇಹ ಮತ್ತು ಮನಸ್ಸಿನಿಂದ ಕೂಡಿದ ಪ್ರಕೃತಿಯೊಂದಿಗೆ ಇದು ಬೆರೆತುಹೋಗಿದೆ. ಅಜ್ಞಾನಿ ತನ್ನ ದೇಹವನ್ನೇ ಆತ್ಮನೆಂದು ತಿಳಿಯುತ್ತಾನೆ. ಜ್ಞಾನಿ ತನ್ನ ಮನಸ್ಸನ್ನೇ ಆತ್ಮನೆಂದು ಭಾವಿಸುತ್ತಾನೆ. ಆದರೆ ಇಬ್ಬರೂ ಕೂಡ ತಪ್ಪು ತಿಳಿದುಕೊಂಡಿರುವರು. ಆತ್ಮನು ಇತರ ವಸ್ತುವಿನೊಂದಿಗೆ ಬೆರೆತು ಹೋಗುವಂತೆ ಯಾವುದು ಮಾಡುವುದು? ಚಿತ್ತದಲ್ಲಿ ನಾನಾ ಅಲೆಗಳೆದ್ದು ಆತ್ಮನನ್ನು ಮುತ್ತುವುವು. ಈ ಅಲೆಗಳ ಮೂಲಕ ಆತ್ಮನ ಮಂಕಾದ ಪ್ರತಿಬಿಂಬ ಮಾತ್ರ ನಮಗೆ ಕಾಣುವುದು. ಆದಕಾರಣ ಆತ್ಮನನ್ನು ಮುತ್ತಿದ ಅಲೆ ಕೋಪದ್ದಾಗಿದ್ದರೆ ಆತ್ಮನು ಕೋಪಗೊಂಡಂತೆ ಕಾಣುತ್ತಾನೆ, “ನನಗೆ ಕೋಪ ಬಂದಿದೆ” ಎಂದು ಆಗ ಹೇಳುತ್ತಾನೆ. ಅಲೆ ಪ್ರೀತಿಯಾಗಿದ್ದರೆ ಅದರಲ್ಲಿ ತನ್ನ ಪ್ರತಿಬಿಂಬ ಕಂಡು, “ನಾನು ಪ್ರೀತಿಸುತ್ತೇನೆ” ಎಂದು ಹೇಳುತ್ತಾನೆ. ಅಲೆ ನಿರ್ಬಲವಾಗಿದ್ದರೆ, ಆತ್ಮನು ಅದರಲ್ಲಿ ಪ್ರತಿಬಿಂಬಿಸಿದರೆ “ನಿರ್ಬಲರು ನಾವು” ಎಂದು ಯೋಚಿಸುತ್ತೇವೆ. ಆತ್ಮನನ್ನು ಮುಚ್ಚಿರುವ ಸಂಸ್ಕಾರದಿಂದ ಇಂತಹ ನಾನಾ ವಿಧದ ಭಾವನೆಗಳು ಬರುವುವು. ಚಿತ್ತದ ಸರೋವರದಲ್ಲಿ ಅಲೆಯೊಂದು ಇರುವ ತನಕ ಆತ್ಮನ ನೈಜಸ್ವರೂಪವನ್ನು ನಾವು ನೋಡಲಾರೆವು. ಅಲೆಗಳೆಲ್ಲ ಸಂಪೂರ್ಣ ಶಾಂತವಾಗುವವರೆಗೆ ನಮಗೆ ಆತ್ಮನ ನೈಜಸ್ವರೂಪ ಗೋಚರಿಸುವುದಿಲ್ಲ. ಅದಕ್ಕೋಸುಗವೇ ಪತಂಜಲಿ ಮೊದಲು ಅಲೆಯ ಅರ್ಥವನ್ನು ಹೇಳುತ್ತಾನೆ. ಎರಡನೆಯದಾಗಿ, ಅಲೆಗಳನ್ನೆಲ್ಲ ಅಡಗಿಸುವುದು ಹೇಗೆ ಎಂಬುದನ್ನು ಹೇಳುತ್ತಾನೆ. ಮೂರನೆಯದಾಗಿ, ಎಲ್ಲ ಅಲೆಗಳನ್ನೂ ನಿಗ್ರಹಿಸುವಂತಹ ಒಂದು ಅಲೆಯನ್ನು ಪ್ರಬಲವಾಗಿ ಮಾಡುವುದು ಹೇಗೆ ಎಂಬುದನ್ನು ಹೇಳುತ್ತಾನೆ. ಬೆಂಕಿ ಬೆಂಕಿಯನ್ನು ನುಂಗುವಂತೆ ಇದು. ಇದೇ ಆಲೋಚನೆ ಉಳಿದಮೇಲೆ ಅದನ್ನು ನಿಗ್ರಹಿಸುವುದು ಸುಲಭ. ಇದೂ ಹೋದಮೇಲೆ ಬರುವ ಸಮಾಧಿಗೆ ನಿರ್ಬೀಜವೆಂದು ಹೆಸರು. ಇಲ್ಲಿ ಯಾವ ಶೇಷವೂ ಉಳಿಯುವುದಿಲ್ಲ. ಆತ್ಮವು ಇಲ್ಲಿ ಸ್ವಯಂ ಜ್ಯೋತಿಪ್ರಕಾಶನು. ಆತ್ಮವು ಸಂಯೋಗವಲ್ಲ, ಪ್ರಪಂಚದಲ್ಲಿರುವ ನಿತ್ಯ ಸರಳವಸ್ತು ಅದೊಂದೇ ಎಂಬುದು ಆಗ ನಮಗೆ ತಿಳಿಯುವುದು. ಆದಕಾರಣ ಅದು ಹುಟ್ಟಲಾರದು, ಸಾಯಲಾರದು. ಅದು ಅಮೃತ, ಅವಿನಾಶಿ, ಚಿರಜಾಗ್ರತವಾದ ಚೇತನದ ಸಾರ.

\chapter{ಧ್ಯಾನ ಮತ್ತು ಅದರ ಅಭ್ಯಾಸ}%%೩೪

\begin{verse}
ತಪಸ್ಸ್ವಾಧ್ಯಾಯೇಶ್ವರ ಪ್ರಣಿಧಾನಾನಿ ಕ್ರಿಯಾಯೋಗಃ~॥ ೧~॥
\end{verse}

\vspace{-0.3cm}

\dsize{ತಪಸ್ಸು, ಸ್ವಾಧ್ಯಾಯ, ಈಶ್ವರನಲ್ಲಿ ಕರ್ಮಫಲಗಳನ್ನು ಅರ್ಪಿಸುವುದು–ಇವುಗಳಿಗೆ ಕ್ರಿಯಾಯೋಗವೆಂದು ಹೆಸರು. }

\vspace{0.2cm}

ಕಳೆದ ಅಧ್ಯಾಯದ ಕೊನೆಯಲ್ಲಿ ಹೇಳಿದ ಸಮಾಧಿಗಳನ್ನು ಹೊಂದುವುದು ಬಹಳ ಕಷ್ಟ. ಆದಕಾರಣ ನಾವು ಅದನ್ನು ನಿಧಾನವಾಗಿ ತೆಗೆದುಕೊಳ್ಳಬೇಕು. ಮೊದಲನೆಯ ಮೆಟ್ಟಲಿಗೆ ಕ್ರಿಯಾಯೋಗವೆಂದು ಹೆಸರು. ಪದಶಃ ಇದರ ಅರ್ಥ ಕರ್ಮವೆಂದು: ಯೋಗಕ್ಕೋಸುಗವಾಗಿ ಕರ್ಮವನ್ನು ಮಾಡುವುದು. ಇಂದ್ರಿಯಗಳು ಕುದುರೆಗಳು, ಮನಸ್ಸು ಲಗಾಮು, ಬುದ್ಧಿ ಸಾರಥಿ, ಆತ್ಮನು ರಥದಲ್ಲಿ ಕುಳಿತಿರುವವನು. ಕುದುರೆಗಳು ಬಲವಾಗಿದ್ದು ಲಗಾಮಿನ ಸೆಳೆತಕ್ಕೆ ಸಿಕ್ಕದೆ ಇದ್ದರೆ, ಸಾರಥಿಯಾದ ಬುದ್ಧಿಗೆ ಕುದುರೆಯನ್ನು ಹೇಗೆ ನಿಗ್ರಹಿಸಬೇಕೆಂದು ತಿಳಿಯದೆ ಇದ್ದರೆ, ಆಗ ರಥಕ್ಕೆ ಅಪಾಯ ಒದಗಬಹುದು. ಆದರೆ ಇಂದ್ರಿಯಗಳೆಂಬ ಕುದುರೆಯನ್ನು ಸರಿಯಾಗಿ ನಿಗ್ರಹಿಸಿದರೆ, ಬುದ್ಧಿಯೆಂಬ ಸಾರಥಿ ಮನಸ್ಸಿನ ಕಡಿವಾಣವನ್ನು ಸರಿಯಾಗಿ ಹಿಡಿದುಕೊಂಡಿದ್ದರೆ ರಥ ಗುರಿಯನ್ನು ಮುಟ್ಟುವುದು. ಆದಕಾರಣ ತಪಸ್ಸು ಎಂದರೇನು? ಇಂದ್ರಿಯಗಳು ಮತ್ತು ದೇಹವನ್ನು ನಡೆಸುವಾಗ ಲಗಾಮನ್ನು ಬಿಗಿಯಾಗಿ ಹಿಡಿದುಕೊಂಡಿರುವುದು; ಇಂದ್ರಿಯವು ಮನಸ್ಸಿಗೆ ತೋರಿದುದನ್ನು ಮಾಡಲು ಅವಕಾಶ ಕೊಡದೆ, ದೇಹ ಮತ್ತು ಇಂದ್ರಿಯವನ್ನು ನಮ್ಮ ಅಧೀನದಲ್ಲಿಟ್ಟುಕೊಂಡಿರುವುದು. ಸ್ವಾಧ್ಯಾಯ: ಇಲ್ಲಿ ಸ್ವಾಧ್ಯಾಯವೆಂದರೆ ಅರ್ಥವೇನು? ಕಾದಂಬರಿ ಕಥೆ ಪುಸ್ತಕಗಳನ್ನು ಓದುವುದಲ್ಲ–ಜೀವನ್ಮುಕ್ತಿಯನ್ನು ಬೋಧಿಸುವ ಪುಸ್ತಕಗಳನ್ನು ಓದುವುದು. ಸ್ವಾಧ್ಯಾಯವೆಂದರೆ ವಾಗ್ವಾದಪೂರಿತವಾದುದಲ್ಲ. ವಾಗ್ವಾದದ ಅವಧಿಯನ್ನು ಯೋಗಿ ಆವಾಗಲೆ ಪೂರೈಸಿರುವನು. ತನ್ನ ಅಭಿಪ್ರಾಯಗಳನ್ನು ಪುಷ್ಟಿಗೊಳಿಸುವುದಕ್ಕಾಗಿ ಅಧ್ಯಯನ ಮಾಡುವನು. ವಾದ ಮತ್ತು ಸಿದ್ಧಾಂತವೆಂಬ ಎರಡು ಬಗೆಯ ಶಾಸ್ತ್ರ ಜ್ಞಾನವಿರುವುದು. ಮನುಷ್ಯನಿಗೆ ಏನೂ ತಿಳಿಯದೇ ಇರುವಾಗ ಮೊದಲನೆಯದಾದ ವಾದವನ್ನು ಸ್ವೀಕರಿಸುವನು, ಪರ–ವಿರೋಧವಾದ ವಾದ ವಿವಾದಗಳಲ್ಲಿ ತೊಡಗುವನು. ಇದಾದ ಮೇಲೆ ಸಿದ್ಧಾಂತವೆಂದರೆ ಒಂದು ನಿರ್ಣಯಕ್ಕೆ ಬರುವುದನ್ನು ಅನುಸರಿಸುವನು. ಸುಮ್ಮನೆ ಸಿದ್ಧಾಂತಕ್ಕೆ ಬರುವುದು ಸಾಲದು. ಸಿದ್ಧಾಂತವನ್ನು ನಾವು ಪುಷ್ಟಿಗೊಳಿಸಬೇಕು. ಗ್ರಂಥಗಳು ಲೆಕ್ಕವಿಲ್ಲದಷ್ಟು ಇವೆ. ಆದರೆ ಇರುವ ಕಾಲ ಕಡಮೆ. ಆದಕಾರಣ, ಜ್ಞಾನದ ರಹಸ್ಯವೇ ಮುಖ್ಯವಾದುದನ್ನು ಆರಿಸಿಕೊಳ್ಳುವುದು. ಅದನ್ನು ಸ್ವೀಕರಿಸಿ ಅದರಂತೆ ಅಭ್ಯಾಸ ಮಾಡುವುದಕ್ಕೆ ಪ್ರಯತ್ನಿಸಿ. ಒಂದು ಹಳೆಯ ಕಥೆಯಿದೆ. ನೀವು ಒಂದು ಬಟ್ಟಲಲ್ಲಿ ಹಾಲು ಮತ್ತು ನೀರನ್ನು ಬೆರಸಿ ರಾಜಹಂಸದ ಮುಂದೆ ಇಟ್ಟರೆ ಅದು ಹಾಲನ್ನು ಮಾತ್ರ ತೆಗೆದುಕೊಂಡು ನೀರನ್ನು ಬಿಡುತ್ತದೆ. ಅದರಂತೆಯೇ ಶಾಸ್ತ್ರದಲ್ಲಿ ಯಾವುದು ಮುಖ್ಯವೋ, ಅದನ್ನು ಸ್ವೀಕರಿಸಿ, ಕೆಲಸಕ್ಕೆ ಬಾರದುದನ್ನು ಬಿಡಬೇಕು. ಬುದ್ಧಿಯ ಕಸರತ್ತು ಮೊದಲಲ್ಲಿ ಆವಶ್ಯಕ. ಯಾವುದನ್ನೂ ನಾವು ಕಣ್ಣು ಮುಚ್ಚಿ ಸ್ವೀಕರಿಸಬಾರದು. ಯೋಗಿಯು ವಾಗ್ವಾದದ ಸ್ಥಿತಿಯನ್ನು ದಾಟಿ ಒಂದು ಸಿದ್ಧಾಂತಕ್ಕೆ ಬಂದಿರುವನು. ಆ ಸಿದ್ಧಾಂತ ಬಂಡೆಯಂತೆ ಅಚಲ. ಈಗ ಅವನು ಮಾಡಬೇಕೆಂದು ಬಯಸುವುದೊಂದೆ:ಅದೇ ಸಿದ್ಧಾಂತಗಳನ್ನು ಮತ್ತೂ ಬಲಪಡಿಸುವುದು. ವಾದಿಸಬೇಡಿ. ಯಾರಾದರೂ ಬಲಾತ್ಕಾರದಿಂದ ನಿಮ್ಮನ್ನು ವಾದಕ್ಕೆ ಎಳೆದರೆ ಮೌನ ತಾಳಿ ಎನ್ನುವನು ಯೋಗಿ. ಯಾವ ವಾದಕ್ಕೂ ಉತ್ತರ ಹೇಳಬೇಡಿ. ಶಾಂತವಾಗಿ ಬೇರೆ ಕಡೆ ಹೋಗಿ. ಏಕೆಂದರೆ ವಾದಗಳು ಮನಸ್ಸಿನ ಸ್ಥಿರತೆಯನ್ನು ಕೆಡಿಸುವುವು ಅಷ್ಟೆ. ಬುದ್ಧಿಯನ್ನು ನಾವು ಸೂಕ್ಷ್ಮ ಮಾಡಬೇಕು. ಅದೊಂದೇ ನಮಗೆ ಬೇಕಾಗಿರುವುದು. ಪ್ರಯೋಜನವಿಲ್ಲದೆ ಬುದ್ಧಿಯನ್ನು ಏಕೆ ಅಸ್ವಸ್ಥಗೊಳಿಸುವಿರಿ? ಬುದ್ಧಿ ದುರ್ಬಲವಾದ ಯಂತ್ರ. ಇಂದ್ರಿಯ ಮೇರೆಯಲ್ಲಿರುವ ಜ್ಞಾನವನ್ನು ಮಾತ್ರ ನಮಗೆ ಒದಗಿಸಬಲ್ಲದು. ಯೋಗಿಯು ಇಂದ್ರಿಯವನ್ನು ಮೀರಿ ಹೋಗಲು ಇಚ್ಛಿಸುವನು. ಆದಕಾರಣ ಬುದ್ಧಿಯಿಂದ ಅವನಿಗೆ ಏನೂ ಪ್ರಯೋಜನವಿಲ್ಲ. ಯೋಗಿಗೆ ಇದು ಚೆನ್ನಾಗಿ ಗೊತ್ತಿದೆ. ಅದಕ್ಕೇ ಅವನು ಮೌನವಾಗಿರುವುದು, ವಾದಮಾಡುವುದಿಲ್ಲ. ವಾದ ಅವನ ಚಿತ್ತವನ್ನು ಅಸ್ಥಿರಗೊಳಿಸುವುದು, ಸಮತ್ವವನ್ನು ಕೆಡಿಸುವುದು. ಚಂಚಲತೆ ದುರ್ಬಲತೆಯ ಕುರುಹು. ಯುಕ್ತಿಯವಾದ ಮತ್ತು ಪರಿಶೋಧನೆಗಳೆಲ್ಲ ಸಾಧಾರಣವಾದುವು. ಇವುಗಳನ್ನೆಲ್ಲ ಮೀರಿದ ಎಷ್ಟೋ ವಿಷಯಗಳಿವೆ. ಶಾಲೆಯ ಹುಡುಗರ ವಾದಕ್ಕೆ ಮತ್ತು ಚರ್ಚಾ ಕೂಟಕ್ಕೆ ಅಲ್ಲ, ಇಡಿ ಜೀವನವಿರುವುದು. ಕರ್ಮಫಲಗಳನ್ನು ಈಶ್ವರಾರ್ಪಣ ಮಾಡುವುದು, ಎಂದರೆ, ನಾವು ನಿಂದೆ ಮತ್ತು ಹೊಗಳಿಕೆಗಳಾವುದನ್ನೂ ಇಚ್ಛಿಸದೆ ಎರಡನ್ನೂ ದೇವರಿಗೆ ಕೊಟ್ಟು ಶಾಂತವಾಗಿರುವುದು. 

\vspace{-0.15cm}

\begin{verse}
ಸಮಾಧಿ–ಭಾವನಾರ್ಥಃ ಕ್ಲೇಶ–ತನೂಕರಣಾರ್ಥಶ್ಚ~॥ ೨~॥
\end{verse}

\vspace{-0.4cm}

\dsize{ಇದು ಸಮಾಧಿಯ ಅಭ್ಯಾಸಕ್ಕೆ ಮತ್ತು ಕ್ಲೇಶಕಾರಕವಾದ ಆತಂಕಗಳನ್ನು ಕಡಮೆ ಮಾಡುವುದಕ್ಕೆ. }

\vspace{0.15cm}

ನಮ್ಮಲ್ಲಿ ಅನೇಕರು ಮನಸ್ಸನ್ನು ತುಂಟ ಹುಡುಗರಂತೆ ಮಾಡಿಕೊಳ್ಳುವೆವು. ಮನಸ್ಸಿಗೆ ಏನು ಇಚ್ಛೆಯಾಗುವುದೋ ಅದನ್ನು ಮಾಡಲು ಅವಕಾಶ ಕೊಡುವೆವು. ಮನಸ್ಸನ್ನು ನಿಗ್ರಹಿಸಿ ನಮ್ಮ ಸ್ವಾಧೀನಕ್ಕೆ ತರಬೇಕಾದರೆ ಕ್ರಿಯಾಯೋಗವನ್ನು ನಾವು ಯಾವಾಗಲೂ ಅಭ್ಯಾಸ ಮಾಡಬೇಕು. ಸಾಕಾದಷ್ಟು ನಿಗ್ರಹಶಕ್ತಿ ಇಲ್ಲದೇ ಇರುವುದರಿಂದ ಯೋಗಾಭ್ಯಾಸಕ್ಕೆ ಅನೇಕ ಆತಂಕಗಳು ಎದ್ದು ನಮಗೆ ತೊಂದರೆಯನ್ನು ಮಾಡುವುವು. ಕ್ರಿಯಾಯೋಗದ ಮೂಲಕ ಮನಸ್ಸು ಹೇಳಿದಂತೆ ಕೇಳದೆ ಅದನ್ನು ನಿಗ್ರಹಿಸುವುದರಿಂದ ನಾವು ಅವುಗಳಿಂದ ಪಾರಾಗಬಹುದು. 

\vspace{-0.2cm}

\begin{verse}
ಅವಿದ್ಯಾಽಸ್ಮಿತಾರಾಗದ್ವೇಷಾಭಿನಿವೇಶಾಃ ಪಂಚಕ್ಲೇಶಾಃ~॥ ೩~॥
\end{verse}

\vspace{-0.4cm}

\dsize{ಕ್ಲೇಶಕಾರಕವಾದ ಆತಂಕಗಳಾವುವುವೆಂದರೆ–ಅವಿದ್ಯೆ, ಅಹಂಕಾರ, ರಾಗ, ದ್ವೇಷ ಮತ್ತು ಜೀವನಾಸಕ್ತಿ. }

\vskip 0.2cm

ಇವೇ ಐದು ಕ್ಲೇಶಗಳು, ನಮ್ಮನ್ನು ಬಂಧಿಸುವ ಪಂಚಬಂಧನ. ಇವುಗಳಲ್ಲಿ ಅವಿದ್ಯೆಯೇ ಎಲ್ಲಕ್ಕೂ ಕಾರಣ, ಉಳಿದ ನಾಲ್ಕು ಅದರ ಪರಿಣಾಮ. ಇದೊಂದೆ ನಮ್ಮ ದುಃಖಕ್ಕೆಲ್ಲ ಕಾರಣ. ಮತ್ತಾವುದು ನಮ್ಮನ್ನು ದುಃಖಿಗಳನ್ನಾಗಿ ಮಾಡಬಲ್ಲದು? ನಿತ್ಯಾನಂದಮಯಸ್ವರೂಪ ಆತ್ಮ. ಅವಿದ್ಯೆ, ಭ್ರಮೆ ಮತ್ತು ಮೋಹವಿಲ್ಲದೆ ಮತ್ತಾವುದು ನಮ್ಮನ್ನು ದುಃಖಿಗಳನ್ನಾಗಿ ಮಾಡಬಹುದು? ಆತ್ಮನಿಗೆ ಸಂಬಂಧಪಟ್ಟ ವ್ಯಥೆಯೆಲ್ಲ ಭ್ರಮೆ. 

\vspace{-0.2cm}

\begin{verse}
ಅವಿದ್ಯಾ ಕ್ಷೇತ್ರಮುತ್ತರೇಷಾಂ ಪ್ರಸುಪ್ತ–ತನು–ವಿಚ್ಛಿನ್ನೋದಾರಾಣಾಮ್​~॥~೪~॥
\end{verse}

\vspace{-0.4cm}

\dsize{ಇವು ಅಪ್ರಕಟಿತವಾಗಿರಲಿ, ಕೃಶವಾಗಿರಲಿ, ಅಡಗಿಸಲ್ಪಟ್ಟಿರಲೀ ಅಥವಾ ವಿಕಾಸವಾಗಿರಲಿ ಎಲ್ಲಕ್ಕೂ ಅವಿದ್ಯೆಯೇ ಕ್ಷೇತ್ರ. }

\vskip 0.2cm

ಅಹಂಕಾರ, ರಾಗ, ದ್ವೇಷ ಮತ್ತು ಜೀವನಾಸಕ್ತಿಗೆ ಕಾರಣ ಅವಿದ್ಯೆ. ಈ ಸಂಸ್ಕಾರಗಳು ನಾನಾ ಅವಸ್ಥೆಗಳಲ್ಲಿರುತ್ತವೆ. ಕೆಲವು ವೇಳೆ ಅಪ್ರಕಟಿತವಾಗಿರುತ್ತವೆ. ಮಗುವಿನಂತೆ ಮುಗ್ಧ ಎಂಬುದನ್ನು ನೀವು ಕೇಳಿರಬಹುದು. ಆದರೂ ಆ ಮಗುವಿನಲ್ಲಿ ಮುಂದೆ ಕ್ರಮೇಣ ವ್ಯಕ್ತವಾಗುವ ದಾನವನ ಅಥವಾ ದೇವನ ಸ್ಥಿತಿ ಇರಬಹುದು. ಯೋಗಿಯಲ್ಲಿ ಪೂರ್ವಕರ್ಮದ ಪರಿಣಾಮವಾಗಿರುವ ಈ ಸಂಸ್ಕಾರ ಬಹಳ ಕೃಶವಾಗಿರಬಹುದು. ಇದನ್ನು ಯೋಗಿ ನಿಗ್ರಹಿಸಿ ಪುನಃ ಅವು ವ್ಯಕ್ತವಾಗದಂತೆ ಮಾಡಬಲ್ಲನು. ಅಡಗಿಸಲ್ಪಟ್ಟಿರುವುದೆಂದರೆ ಕೆಲವು ಕಾಲದವರೆಗೆ ಬಲವಾದ ಸಂಸ್ಕಾರವು ಕ್ಷೀಣವಾದ ಸಂಸ್ಕಾರವನ್ನು ಕೆಳಗೆ ತಳ್ಳಿರುವುದು ಎಂದು ಅರ್ಥ. ಆದರೆ ಬಲಾತ್ಕಾರದ ಶಕ್ತಿ ಕಡಮೆಯಾದ ಮೇಲೆ ಅದು ಪುನಃ ಮೇಲಕ್ಕೆ ಏಳುತ್ತದೆ. ಕೊನೆಯದಾದ ವಿಕಾಸಸ್ಥಿತಿ ಯಾವುದೆಂದರೆ–ಒಳ್ಳೆಯ ಅಥವಾ ಕೆಟ್ಟ ಸಂಸ್ಕಾರಗಳಿಗೆ ಸರಿಯಾದ ವಾತಾವರಣ ಸಿಕ್ಕಿದಾಗ ಅತ್ಯಂತ ಕ್ರಿಯಾಶೀಲವಾಗುವುದು. 

\vspace{-0.2cm}

\begin{verse}
ಅನಿತ್ಯಾಶುಚಿದುಃಖಾನಾತ್ಮಸು ನಿತ್ಯ–ಶುಚಿ–ಸುಖಾತ್ಮಖ್ಯಾತಿರವಿದ್ಯಾ~॥ ೫~॥
\end{verse}

\vspace{-0.4cm}

\dsize{ಅನಿತ್ಯವೂ, ಅಶುಚಿಯೂ, ದುಃಖಕಾರಕವೂ ಮತ್ತು ಅನಾತ್ಮವೂ ಆದುದನ್ನು ಕ್ರಮವಾಗಿ ನಿತ್ಯ, ಶುದ್ಧ, ಆನಂದಮಯ ಮತ್ತು ಆತ್ಮ ಎಂದು ತಪ್ಪಾಗಿ ತಿಳಿದುಕೊಳ್ಳುವುದೇ ಅಜ್ಞಾನ. }

\vskip 0.2cm

ಹಲವು ವಿಧದ ಸಂಸ್ಕಾರಗಳಿಗೆಲ್ಲ ಮೂಲವೊಂದೇ, ಅಜ್ಞಾನ. ಅಜ್ಞಾನವೆಂದರೇನೆಂಬು\break ದನ್ನು ನಾವು ಮೊದಲು ತಿಳಿದುಕೊಳ್ಳಬೇಕು. ಶುದ್ಧವೂ, ಸ್ವಯಂ ಪ್ರಕಾಶಮಾನವೂ, ನಿತ್ಯಾನಂದಮಯವೂ ಆದ ಆತ್ಮ ನಾವಲ್ಲ; ಬರಿಯ ದೇಹಧಾರಿಗಳು ನಾವೆಂದು ಭಾವಿಸುವುದೇ ಅಜ್ಞಾನ. ಮಾನವನನ್ನು ನಾವು ದೇಹವೆಂದು ತಿಳಿಯುತ್ತೇವೆ, ದೇಹವೆಂದೇ ನೋಡುತ್ತೇವೆ. ಇದು ಮಹಾಮೋಹ. 

\vspace{-0.2cm}

\begin{verse}
ದೃಗ್​ದರ್ಶನಶಕ್ತ್ಯೋರೇಕಾತ್ಮತೇವಾಸ್ಮಿತಾ~॥ ೬~॥
\end{verse}

\vspace{-0.4cm}

\dsize{ನೋಡುವವನನ್ನು ನೋಡುವ ಕರಣವೆಂದು ತಪ್ಪು ತಿಳಿದುಕೊಳ್ಳುವುದೇ ಅಹಂಕಾರ. }

\vskip 0.2cm

ನೋಡುವವನು ನಿಜವಾಗಿಯೂ, ಶುದ್ಧನೂ, ನಿತ್ಯಪವಿತ್ರನೂ, ಅನಂತನೂ,\break ಅದ್ಭುತನೂ ಆದ ಆತ್ಮ. ಇದೇ ನಿಜವಾದ ಆತ್ಮ. ನೋಡುವ ಕರಣಗಳಾವುವು? ಚಿತ್ತ, ಬುದ್ಧಿ (ನಿಶ್ಚಯಿಸುವ ಶಕ್ತಿ), ಮನಸ್ಸು ಮತ್ತು ಇಂದ್ರಿಯಗಳು. ಬಾಹ್ಯ ಪ್ರಪಂಚವನ್ನು ನಾವು ನೋಡುವುದಕ್ಕೆ ಇವು ಉಪಕರಣಗಳು. ಈ ಉಪಕರಣಗಳೊಂದಿಗೆ ಆತ್ಮನ ತಾದಾತ್ಮ್ಯವನ್ನು ನಾವು ಅಹಂಕಾರದ ಅಜ್ಞಾನವೆನ್ನುವುದು. “ನಾನು ಮನಸ್ಸು”, “ನಾನು ಆಲೋಚನೆ”, “ನಾನು ಕೋಪಗೊಂಡಿದ್ದೇನೆ” ಅಥವಾ “ನಾನು ಸಂತೋಷವಾಗಿರುವೆನು” ಎಂದು ನಾವು ಹೇಳುತ್ತೇವೆ. ನಾವು ಕೋಪಗೊಳ್ಳುವುದು ಹೇಗೆ? ಇನ್ನೊಬ್ಬರನ್ನು ದ್ವೇಷಿಸುವುದು ಹೇಗೆ? ಅವಿಕಾರಿಯಾದ ಆತ್ಮನೊಂದಿಗೆ ನಾವು ತಾದಾತ್ಮ್ಯವನ್ನು ಪಡೆಯಬೇಕು. ಅದು ಅವಿಕಾರಿಯಾದರೆ ಒಮ್ಮೆ ಸುಖದಲ್ಲಿ ಒಮ್ಮೆ ದುಃಖದಲ್ಲಿ ಹೇಗೆ ಇರಬಲ್ಲದು? ಅದು ನಿರಾಕಾರವೂ ಅನಂತವೂ ಸರ್ವವ್ಯಾಪಿಯೂ ಆಗಿರುವುದು. ಅದನ್ನು ಯಾವುದು ಬದಲಾಯಿಸಬಲ್ಲದು? ಅದು ಎಲ್ಲಾ ನಿಯಮವನ್ನು ಮೀರಿರುವುದು. ಅದನ್ನು ಬದಲಾಯಿಸಬಲ್ಲದು ಯಾವುದು? ಪ್ರಪಂಚದಲ್ಲಿ ಮತ್ತಾವುದೂ ಅದರ ಮೇಲೆ ಪರಿಣಾಮವನ್ನು ಉಂಟುಮಾಡಲಾರದು. ಆದರೂ ನಾವು ಅಜ್ಞಾನದಿಂದ ಚಿತ್ತವೆಂದು ಭಾವಿಸುತ್ತೇವೆ, ಸುಖಿಗಳು ದುಃಖಿಗಳೆಂದು ತಿಳಿಯುತ್ತೇವೆ. 

\vspace{-0.2cm}

\begin{verse}
ಸುಖಾನುಶಯೀ ರಾಗಃ~॥ ೭~॥
\end{verse}

\vspace{-0.4cm}

\dsize{ಸುಖದಲ್ಲಿ ನೆಲೆಗೊಂಡಿರುವುದೇ ರಾಗ. }

\vskip 0.2cm 

ಕೆಲವು ವಸ್ತುಗಳ ಮೇಲೆ ನಮಗೆ ಪ್ರೀತಿ. ಮನಸ್ಸು ಒಂದು ಪ್ರವಾಹದಂತೆ ಆ ವಸ್ತುವಿನ ಕಡೆಗೆ ಹರಿಯುವುದು. ಸುಖದ ಕೇಂದ್ರವನ್ನು ಅನುಸರಿಸುವುದನ್ನೇ ನಾವು ರಾಗವೆನ್ನುವುದು. ನಮಗೆ ಎಲ್ಲಿ ಸಂತೋಷ ತೋರುವುದಿಲ್ಲವೋ ಅಲ್ಲಿ ನಾವೆಂದಿಗೂ ಆಸಕ್ತರಾಗಿರುವುದಿಲ್ಲ. ಕೆಲವು ವೇಳೆ ಅತಿ ವಿಚಿತ್ರವಾದ ವಸ್ತುಗಳೆಲ್ಲ ನಮಗೆ ಸಂತೋಷ ತೋರುವುವು. ಆದರೆ ಅದರ ಹಿಂದೆ ಇರುವ ನಿಯಮ ಒಂದೇ. ಅದೇ, ನಮಗೆ ಎಲ್ಲಿ ಸಂತೋಷವೋ ಅಲ್ಲಿ ಆಸಕ್ತರು ನಾವು. 

\vspace{-0.2cm}

\begin{verse}
ದುಃಖಾನುಶಯೀ ದ್ವೇಷಃ~॥ ೮~॥
\end{verse}

\vspace{-0.4cm}

\dsize{ದುಃಖದಲ್ಲಿ ನೆಲೆಗೊಂಡಿರುವುದೇ ದ್ವೇಷ. }

\vskip 0.2cm 

ನಮಗೆ ಯಾವುದು ದುಃಖವನ್ನು ಕೊಡುತ್ತದೆಯೋ ಅದರಿಂದ ತಕ್ಷಣವೇ ಪಾರಾಗಲು ನಾವು ಯತ್ನಿಸುತ್ತೇವೆ. 

\vskip 0.1cm 

\vspace{-0.2cm}

\begin{verse}
ಸ್ವರಸವಾಹೀ ವಿದುಷೋಽಪಿ ತಥಾರೂಢೋಽಭಿನಿವೇಶಃ~॥ ೯~॥
\end{verse}

\vspace{-0.4cm}

\dsize{ಸ್ವಭಾವತಃ ಹರಿಯುವ, ಜ್ಞಾನಿಗಳಲ್ಲಿ ಕೂಡ ನಿಂತಿರುವುದೇ ಜೀವನಾಸಕ್ತಿ. }

\vskip 0.2cm 

ಜೀವನಾಸಕ್ತಿಯನ್ನು ನೀವು ಪ್ರತಿಯೊಂದು ಪ್ರಾಣಿಯಲ್ಲಿಯೂ ಕೂಡ ನೋಡುತ್ತೀರಿ. ಇದರ ಮೇಲೆ ಪುನರ್ಜನ್ಮದ ಸಿದ್ಧಾಂತವನ್ನು ಸ್ಥಾಪಿಸುವುದಕ್ಕೆ ಎಷ್ಟೋ ಪ್ರಯತ್ನ ನಡೆದಿದೆ. ಮಾನವನಿಗೆ ಜೀವನದ ಮೇಲೆ ಆಸಕ್ತಿ ಹೆಚ್ಚು. ಅದಕ್ಕೇ ಅವನು ಮತ್ತೊಂದು ಜೀವನವನ್ನು ಆಶಿಸುವುದು. ಈ ವಾದ ಅಷ್ಟೇನೂ ಹುರುಳಿಲ್ಲದ್ದು ಎಂದು ಹೇಳಬೇಕಾಗಿಲ್ಲ. ಆದರೆ ಒಂದು ವಿಚಿತ್ರ ಸಂಗತಿ ಏನೆಂದರೆ, ಪಾಶ್ಚಾತ್ಯ ದೇಶದಲ್ಲಿ ಈ ಜೀವನಾಸಕ್ತಿಯ ಭಾವನೆ, ಮಾನವನಿಗೆ ಮಾತ್ರ ಬಹುಶಃ ಪುನರ್ಜನ್ಮವಿದೆ ಎಂಬುದನ್ನು ತೋರುತ್ತದೆಯೇ ಹೊರತು ಪ್ರಾಣಿಗಳು ಇದಕ್ಕೆ ಸೇರುವುದಿಲ್ಲ. ಭರತಖಂಡದಲ್ಲಿ ಪೂರ್ವ ಅನುಭವ ಮತ್ತು ಪುನರ್ಜನ್ಮ ಇತ್ತು ಎಂಬುದಕ್ಕೆ ಜೀವನಾಸಕ್ತಿಯೊಂದು ವಾದ. ಉದಾಹರಣೆಗೆ, ನಮ್ಮ ಜ್ಞಾನವೆಲ್ಲ ಅನುಭವದಿಂದ ಬಂದಿರುವುದು ಸತ್ಯವಾಗಿದ್ದರೆ, ನಾವು ಯಾವುದನ್ನು ಅನುಭವಿಸಿಲ್ಲವೋ, ಅದನ್ನು ಆಲೋಚಿಸಲಾರೆವು ಅಥವಾ ತಿಳಿದುಕೊಳ್ಳಲಾರೆವು ಎನ್ನುವುದು ಸತ್ಯವಾಗಿರಬೇಕು. ಮೊಟ್ಟೆಯಿಂದ ಬಂದ ತಕ್ಷಣವೇ ಮರಿಯು ಆಹಾರವನ್ನು ಕುಕ್ಕಲು ಮೊದಲು ಮಾಡುತ್ತದೆ. ಅನೇಕ ವೇಳೆ ಕೋಳಿಗಳ ಹತ್ತಿರ ಶಾಖಕ್ಕೆ ಇಟ್ಟ ಬಾತಿನ ಮೊಟ್ಟೆಗಳಿಂದ ಮರಿ ಹೊರಗೆ ಬಂದೊಡನೆಯೇ ನೀರಿಗೆ ಓಡುವುದನ್ನು ನೋಡುತ್ತೇವೆ. ಮರಿಗಳು ನೀರಿಗೆ ಬೀಳುವುದೆಂದು ತಾಯಿ ಅಂಜುವುದು. ಅನುಭವ ಒಂದೇ ಜ್ಞಾನದ ಮೂಲವಾದರೆ ಕೋಳಿ ಮರಿಗಳು ಆಹಾರ ಕುಕ್ಕುವುದನ್ನು ಹೇಗೆ ಕಲಿತವು? ಬಾತಿನ ಮರಿಗಳು ಹೇಗೆ ಈಜುವುದನ್ನು ಕಲಿತವು? ಇದು ಒಂದು ಹುಟ್ಟುಗುಣವೆಂದರೆ ಅದರಲ್ಲಿ ಏನೂ ಅರ್ಥವಿಲ್ಲ. ಸುಮ್ಮನೆ ಒಂದು ಪದವನ್ನು ಉಚ್ಚರಿಸಿದಂತೆ. ಇದು ಒಂದು ವಿವರಣೆ ಅಲ್ಲ. ಹುಟ್ಟುಗುಣವೆಂದರೇನು? ನಮ್ಮಲ್ಲಿಯೇ ಅನೇಕ ಹುಟ್ಟು ಗುಣಗಳಿವೆ. ಉದಾಹರಣೆಗೆ, ನಿಮ್ಮಲ್ಲಿ ಅನೇಕ ಹೆಂಗಸರು ಪಿಯಾನೋ ನುಡಿಸುತ್ತಾರೆ. ನೀವು ಕಲಿತುಕೊಳ್ಳುವಾಗ ಎಷ್ಟು ಜೋಪಾನವಾಗಿ ಬಿಳಿ ಮತ್ತು ಕಪ್ಪು ಮನೆಗಳ ಮೇಲೆ ಕೈಯಿಡುತ್ತಿದ್ದಿರಿ ಎಂಬುದನ್ನು ಜ್ಞಾಪಿಸಿಕೊಳ್ಳಿ. ಆದರೆ ಈಗ ಅನೇಕ ವರುಷಗಳ ಅನುಭವದ ಮೇಲೆ ನಿಮ್ಮ ಸ್ನೇಹಿತರೊಂದಿಗೆ ಮಾತನಾಡುತ್ತಿರುವಾಗಲೂ ಕೈಗಳು ತಮ್ಮಷ್ಟಕ್ಕೆ ತಾವೇ ಆಡುತ್ತಿರುತ್ತವೆ. ಇದೊಂದು ಹುಟ್ಟುಗುಣವಾಗುತ್ತದೆ. ಯಾಂತ್ರಿಕವಾಗುತ್ತದೆ. ಆದರೆ ನಮಗೆ ತಿಳಿದಿರುವ ಮಟ್ಟಿಗೆ, ಈಗ ಯಾವ ಕೆಲಸಗಳು ನಮ್ಮ ಇಚ್ಛೆಯಿಲ್ಲದೆ ನಡೆಯುತ್ತವೆ ಎನ್ನುತ್ತೇವೆಯೊ ಅದೆಲ್ಲ ಕ್ಷೀಣಿಸಿದ ವಿಚಾರ ಯೋಗಿಯ ಭಾಷೆಯಲ್ಲಿ ಹುಟ್ಟುಗುಣವೆಂದರೆ ಅಂತರ್ಗತವಾಗಿರುವ ಯುಕ್ತಿ. ವಿಚಾರ ಅಂತರ್ಗತವಾಗುತ್ತದೆ. ಇವುಗಳೇ ಸ್ವೇಚ್ಛಾಸಂಸ್ಕಾರಗಳಾಗುತ್ತವೆ. ಆದಕಾರಣ ಪ್ರಪಂಚದಲ್ಲಿ ನಾವು ಯಾವುದನ್ನು ಹುಟ್ಟುಗುಣವೆನ್ನುತ್ತೇವೆಯೊ ಅವುಗಳೆಲ್ಲ ಅಂತರ್ಗತವಾದ ಯುಕ್ತಿ ಎಂದು ಹೇಳುವುದು ತರ್ಕಸಮ್ಮತವಾಗಿದೆ. ಯುಕ್ತಿಯು ಅನುಭವವಿಲ್ಲದೆ ಬರಲಾರದ ಕಾರಣ ಎಲ್ಲಾ ಹುಟ್ಟುಗುಣಗಳು ಕೂಡ ಪೂರ್ವ ಅನುಭವದ ಫಲ. ಕೋಳಿಮರಿಗಳು ಹದ್ದಿಗೆ ಅಂಜುತ್ತವೆ; ಬಾತಿನ ಮರಿಗಳಿಗೆ ನೀರು ಕಂಡರೆ ಪ್ರೀತಿ. ಇವು ಪೂರ್ವ ಅನುಭವದ ಫಲ. ಅನಂತರ ಏಳುವ ಪ್ರಶ್ನೆ, ಅನುಭವ ಯಾವುದೊ ಒಂದು ಜೀವಕ್ಕೆ ಸಂಬಂಧಿಸಿದುದೊ ಅಥವಾ ದೇಹಕ್ಕೋ ಎಂಬುದು. ಬಾತಿನ ಮರಿಗೆ ಬರುವ ಅನುಭವ ತನ್ನ ಪೂರ್ವಜರಿಂದ ಬಂದುದೇ ಅಥವಾ ತನ್ನ ಹಿಂದಿನ ಅನುಭವವೇ ಎನ್ನುವುದು. ಆಧುನಿಕ ವಿಜ್ಞಾನಿಗಳು, ದೇಹಕ್ಕೆ ಸಂಬಂಧ ಪಟ್ಟದ್ದು ಎನ್ನುತ್ತಾರೆ. ಯೋಗಿಗಳು ದೇಹದ ಮೂಲಕ ರವಾನಿಸಲ್ಪಟ್ಟ ಮಾನಸಿಕ ಅನುಭವ ಎನ್ನುತ್ತಾರೆ. ಇದನ್ನೇ ಪುನರ್ಜನ್ಮ ಸಿದ್ಧಾಂತವೆನ್ನುವುದು. 

ನಮ್ಮ ಜ್ಞಾನವನ್ನೆಲ್ಲ ಪ್ರತ್ಯಕ್ಷ ಅಥವಾ ಯುಕ್ತಿ, ಹುಟ್ಟುಗುಣಗಳೆಂದು ಬೇರೆ ಬೇರೆ ಹೆಸರಿನಿಂದ ಕರೆದರೂ ಅವು ಅನುಭವ ಎಂಬ ಒಂದು ದಾರಿಯ ಮೂಲಕ ಬರಬೇಕು–ಎನ್ನುವುದನ್ನು ನೋಡಿದೆವು. ನಾವು ಈಗ ಹುಟ್ಟುಗುಣವೆನ್ನುವುದೆಲ್ಲ, ಹಿಂದಿನ ಅನುಭವ ಕ್ಷಯಿಸಿ, ಹುಟ್ಟುಗುಣಕ್ಕೆ ರೂಪಾಂತರ ಹೊಂದಿದೆ. ಆ ಹುಟ್ಟುಗುಣವೇ ಮತ್ತೊಮ್ಮೆ ಯುಕ್ತಿಯಾಗಿ ವಿಕಾಸವಾಗುತ್ತದೆ. ಇದರಂತೆಯೇ ಪ್ರಪಂಚದಲ್ಲೆಲ್ಲ ಇರುವುದು. ಇದರ ತಳಹದಿಯ ಮೇಲೆ ಭಾರತ ದೇಶದಲ್ಲಿ ಪುನರ್ಜನ್ಮದ ಪರವಾಗಿ ಮುಖ್ಯವಾದ ವಾದವನ್ನು ಹೂಡಿರುವರು. ಪುನಃಪುನಃ ಉಂಟಾಗುವ ಅಂಜಿಕೆಯ ಅನುಭವಗಳು ನಮ್ಮನ್ನು ಜೀವನಕ್ಕೆ ಅಂಟಿಕೊಳ್ಳುವಂತೆ ಮಾಡುತ್ತವೆ. ಆದ್ದರಿಂದಲೇ ಮಗುವಿಗೆ ಹುಟ್ಟಿನಿಂದಲೇ ಭಯ ಎನ್ನುವುದು ಇರುತ್ತದೆ. ಏಕೆಂದರೆ ಹಿಂದಿನ ನೋವಿನ ಅನುಭವವು ಆ ಮಗುವಿನಲ್ಲಿ ಇರುವುದೇ ಅದಕ್ಕೆ ಕಾರಣ. ಇದನ್ನೇ ನಾವು ಜೀವನಾಸಕ್ತಿ ಎನ್ನುವುದು. ಈ ದೇಹ ನಾಶವಾಗುತ್ತದೆ ಎಂದು ತಿಳಿದ ದೊಡ್ಡ ಪಂಡಿತರಲ್ಲೂ ಕೂಡ, “ದುಃಖಿಸಬೇಡಿ, ಇಂತಹ ನೂರಾರು ದೇಹಗಳು ನಮಗೆ ಇದ್ದವು, ಆತ್ಮನು ಸಾಯುವುದಿಲ್ಲ” ಎನ್ನುವವರಲ್ಲಿ ಕೂಡ, ಪಾಂಡಿತ್ಯಪೋಷಿತ ದೃಢನಂಬಿಕೆ ಇದ್ದರೂ, ಜೀವನಾಸಕ್ತಿ ಇರುವುದನ್ನು ನೋಡುತ್ತೇವೆ. ಜೀವನಕ್ಕಾಗಿ ಏತಕ್ಕೆ ಇಷ್ಟೊಂದು ಆಸೆ? ಇದೊಂದು ಹುಟ್ಟುಗುಣವಾಗಿರುವುದು ನಮಗೆ ಕಾಣುತ್ತದೆ. ಯೋಗಿಯ ಮನಶ್ಯಾಸ್ತ್ರದ ಭಾಷೆಯಲ್ಲಿ ಹೇಳುವುದಾದರೆ, ಇದು ಸಂಸ್ಕಾರವಾಗಿದೆ. ಸೂಕ್ಷ್ಮವಾದ ಮತ್ತು ಗುಪ್ತವಾದ ಸಂಸ್ಕಾರಗಳು ನಮ್ಮ ಚಿತ್ತದಲ್ಲಿ ನಿದ್ರಿಸುತ್ತಿವೆ. ನಾವು ಹುಟ್ಟುಗುಣವೆನ್ನುವ ಈ ಸಾವಿನ ಅಂಜಿಕೆ ಸುಪ್ತವಾಗಿರುವ ಹಿಂದಿನ ಅನುಭವ. ಇವು ಚಿತ್ತದಲ್ಲಿವೆ. ಅಲ್ಲಿ ಇವು ಕೆಲಸ ಮಾಡದೆ ಇಲ್ಲ, ಗುಪ್ತವಾಗಿ ಕೆಲಸ ಮಾಡುವುವು. 

ಸ್ಥೂಲರೂಪದ ಚಿತ್ತವೃತ್ತಿಗಳನ್ನು ನಾವು ತಿಳಿಯಬಹುದು, ಮತ್ತು ಅವುಗಳನ್ನು ಸುಲಭವಾಗಿ ನಿಗ್ರಹಿಸಬಹುದು. ಆದರೆ ಸೂಕ್ಷ್ಮ ಚಿತ್ತವೃತ್ತಿಗಳನ್ನೇನು ಮಾಡಬೇಕು? ಅವುಗಳನ್ನು ಹೇಗೆ ನಿಗ್ರಹಿಸುವುದು? ನಾನು ಕೋಪಗೊಂಡಾಗ ನನ್ನ ಮನಸ್ಸೆಲ್ಲ ಒಂದು ಕೋಪದ ಅಲೆಯಾಗುವುದು. ನನಗೆ ಇದು ಕಾಣುವುದು, ಅದನ್ನು ಮುಟ್ಟಬಹುದು; ಸುಲಭವಾಗಿ ಅದನ್ನು ಎದುರಿಸಬಹುದು, ಅದರೊಂದಿಗೆ ಹೋರಾಡಬಹುದು. ಆದರೆ ನಾನು ಅದರ ಬೇರಿನ ಆಳಕ್ಕೆ ಹೋಗುವ ಪರಿಯಂತವೂ ಅದರೊಂದಿಗೆ ಜಯಪ್ರದವಾಗಿ ಹೋರಾಡಲಾರೆ. ಒಬ್ಬನು ನನಗೆ ಕಠಿಣವಾಗಿ ಮಾತನಾಡುತ್ತಾನೆ. ಆಗ ಉದ್ರೇಕವಾಗುವಂತೆ ನನಗೆ ಭಾಸವಾಗುವುದು. ಅವನು ಇನ್ನೂ ಕಠಿಣವಾದ ಮಾತನ್ನು ಮುಂದುವರಿಸುತ್ತಾನೆ. ಕೊನೆಗೆ ನಾನು ಸಂಪೂರ್ಣ ಕೋಪಗೊಂಡು ನನ್ನನ್ನು ಮರೆತು ಕೋಪದೊಂದಿಗೆ ತಾದಾತ್ಮ್ಯವನ್ನು ಹೊಂದುತ್ತೇನೆ. ಮೊದಲು ಅವನು ನನ್ನನ್ನು ನಿಂದಿಸಿದಾಗ, “ನಾನು ಕೋಪಗೊಳ್ಳುತ್ತಿದ್ದೇನೆ” ಎಂದು ಆಲೋಚಿಸಿದೆ. ಆಗ ಕೋಪ ಬೇರೆ ನಾನು ಬೇರೆ ಆಗಿದ್ದೆ. ಆದರೆ ನಾನು ಪೂರ್ಣ ಕೋಪಗೊಂಡಾಗ ನಾನು ಬರಿಯ ಕೋಪವೆ. ಈ ಭಾವನೆಗಳು ನಮ್ಮ ಮೇಲೆ ತಮ್ಮ ಪ್ರಭಾವವನ್ನು ಬೀರುತ್ತವೆ ಎಂದು ತಿಳಿಯುವುದಕ್ಕೆ ಮುಂಚೆಯೇ, ಇನ್ನೂ ಅಂಕುರಾವಸ್ಥೆಯಲ್ಲಿರುವಾಗಲೇ, ಸೂಕ್ಷ್ಮಾವಸ್ಥೆಯಲ್ಲಿರುವಾಗಲೇ ನಾವು ಇವನ್ನು ನಿಗ್ರಹಿಸಬೇಕು. ಬಹುಪಾಲು ಮನುಷ್ಯರಿಗೆ ಸುಪ್ತಾವಸ್ಥೆಯಿಂದ ಏಳುವ ಆಸೆಗಳ ಸೂಕ್ಷ್ಮಸ್ಥಿತಿ ಗೊತ್ತೇ ಇರುವುದಿಲ್ಲ.\break ಸರೋವರದ ಆಳದಿಂದ ಒಂದು ಗುಳ್ಳೆ ಏಳುತ್ತಿರುವಾಗ ನಾವು ಅದನ್ನು ನೋಡುವುದಿಲ್ಲ. ಅದು ಇನ್ನೇನು ನೀರಿನ ಮೇಲಕ್ಕೆ ಬಂದಿತು ಎನ್ನುವಾಗಲೂ ನೋಡುವುದಿಲ್ಲ. ಅದು ಒಡೆದು ಹೊಸ ತರಂಗಗಳನ್ನು ಎಬ್ಬಿಸಿದಾಗ ಮಾತ್ರ ಅದು ಅಲ್ಲಿದೆ ಎನ್ನುವುದು ನಮಗೆ ಗೊತ್ತಾಗುವುದು. ಅವು ಸೂಕ್ಷ್ಮಾವಸ್ಥೆಯಲ್ಲಿರುವಾಗ ಅವನ್ನು ನಿಗ್ರಹಿಸಿದರೆ ಮಾತ್ರ ತರಂಗಗಳೊಂದಿಗೆ ಜಯಪ್ರದವಾಗಿ ಹೋರಾಡಬಲ್ಲೆವು. ಅವು ಸ್ಥೂಲವಾಗುವುದಕ್ಕೆ ಮುಂಚೆ, ಅವನ್ನು ಹಿಡಿದು ಪಳಗಿಸುವ ತನಕ ಯಾವ ಪ್ರಲೋಭನೆಗಳನ್ನೂ ನಾವು ಸಂಪೂರ್ಣವಾಗಿ ಜಯಿಸುವ ನೆಚ್ಚಿಗೆ ಇಲ್ಲ. ಆಸೆಯನ್ನು ನಿಗ್ರಹಿಸಬೇಕಾದರೆ ಅದರ ಮೂಲವನ್ನು ನಿಗ್ರಹಿಸಬೇಕು ಆಗ ಮಾತ್ರ ನಾವು ಅದರ ಬೀಜವನ್ನು ದಹಿಸಬಲ್ಲೆವು. ಹುರಿದ ಬೀಜವನ್ನು ಬಿತ್ತಿದರೆ ಹೇಗೆ ಮೊಳಕೆ ಬರಲಾರದೋ ಹಾಗೆಯೇ ಈ ಆಸೆಗಳೂ ಕೂಡ ಪುನಃ ಮೊಳಕೆಯೊಡೆಯಲಾರವು. 

\vspace{-0.2cm}

\begin{verse}
ತೇ ಪ್ರತಿಪ್ರಸವಹೇಯಾಃ ಸೂಕ್ಷ್ಮಾಃ~॥ ೧೦~॥
\end{verse}

\vspace{-0.4cm}

\dsize{ಸೂಕ್ಷ್ಮವಾದ ಸಂಸ್ಕಾರಗಳನ್ನು ಕಾರಣಾವಸ್ಥೆಯಲ್ಲಿ ಲಯಗೊಳಿಸುವುದರ ಮೂಲಕ ನಿಗ್ರಹಿಸಬೇಕು. }

\vskip 0.2cm 

ಈ ಸೂಕ್ಷ್ಮ ಸಂಸ್ಕಾರಗಳು ಮುಂದೆ ಸ್ಥೂಲರೂಪದಲ್ಲಿ ಮೇಲೇಳುತ್ತವೆ. ಇಂತಹ ಸೂಕ್ಷ್ಮ ಸಂಸ್ಕಾರಗಳನ್ನು ಹೇಗೆ ನಾವು ನಿಗ್ರಹಿಸಬೇಕು? ಕಾರ್ಯವನ್ನು ಕಾರಣದಲ್ಲಿ ಲಯಗೊಳಿಸುವುದರ ಮೂಲಕ. ಕಾರ್ಯವಾದ ಚಿತ್ತವನ್ನು ಕಾರಣವಾದ ಅಸ್ಮಿತ ಅಥವಾ ಅಹಂಕಾರದಲ್ಲಿ ಲಯಗೊಳಿಸಿದಾಗ ಸೂಕ್ಷ್ಮ ಸಂಸ್ಕಾರಗಳೂ ಅದರೊಂದಿಗೆ ನಾಶವಾಗುತ್ತವೆ. ಧ್ಯಾನವು ಇವುಗಳನ್ನು ನಾಶಮಾಡಲಾರದು. 

\vspace{-0.2cm}

\begin{verse}
ಧ್ಯಾನಹೇಯಾಸ್ತದ್​ವೃತ್ತಯಃ~॥ ೧೧~॥
\end{verse}

\vspace{-0.4cm}

\dsize{ಧ್ಯಾನದ ಮೂಲಕ ಅವುಗಳ ಸ್ಥೂಲ ವೃತ್ತಿಗಳನ್ನು ತಿರಸ್ಕರಿಸಬೇಕು. }

\vskip 0.2cm 

ಈ ಅಲೆಗಳೇಳುವುದನ್ನು ನಿಗ್ರಹಿಸುವುದಕ್ಕೆ ಧ್ಯಾನ ಅತಿ ಸಹಕಾರಿ. ಧ್ಯಾನದ ಮೂಲಕ ಮನಸ್ಸು ಈ ಅಲೆಗಳನ್ನು ನಿಗ್ರಹಿಸುವಂತೆ ಮಾಡಬಹುದು. ಅನೇಕ ದಿನ. ಮಾಸ, ಸಂವತ್ಸರ ಧ್ಯಾನವನ್ನು ನಾವು ಅಭ್ಯಾಸ ಮಾಡುತ್ತಿದ್ದರೆ ಅದು ನಮ್ಮಲ್ಲಿ ಒಂದು ಸ್ವಭಾವವಾಗಿ ನಮಗೆ ಇಚ್ಛೆ ಇಲ್ಲದೆ ಇದ್ದರೂ ಬರುವಂತೆ ಆದರೆ, ಕೋಪ ಮತ್ತು ದ್ವೇಷವನ್ನು ನಾವು ನಿಗ್ರಹಿಸಬಹುದು. 

\vspace{-0.2cm}

\begin{verse}
ಕ್ಲೇಶಮೂಲಃ ಕರ್ಮಾಶಯೋ ದೃಷ್ಟಾದೃಷ್ಟಜನ್ಮವೇದನೀಯಃ~॥ ೧೨~॥
\end{verse}

\vspace{-0.4cm}

\dsize{ಕ್ಲೇಶವನ್ನೇ ಮೂಲವಾಗಿ ಪಡೆದಿರುವ ಕರ್ಮಾಶಯವು ಈಗಿನ ಮತ್ತು ಮುಂದಿನ ಜೀವನದ ಅನುಭವಗಳಿಗೆ ಕಾರಣವಾಗುತ್ತದೆ. }

\vskip 0.2cm 

ಕರ್ಮಾಶಯವೆಂದರೆ ಸಂಸ್ಕಾರಗಳ ಮೊತ್ತವೆಂದು ಅರ್ಥ. ನಾವು ಏನು ಕೆಲಸ ಮಾಡಿದರೂ ಚಿತ್ತದಲ್ಲಿ ಒಂದು ಅಲೆ ಏಳುವುದು. ಕೆಲಸವಾದ ಮೇಲೆ ಆ ಅಲೆ ಹೋಯಿತೆಂದು ನಾವು ಭಾವಿಸುವೆವು. ಹಾಗಲ್ಲ. ಅದು ಸೂಕ್ಷ್ಮರೂಪದಲ್ಲಿ ಅಲ್ಲೇ ಇರುತ್ತದೆ. ನಾವು ಕೆಲಸವನ್ನು ನೆನಸಿಕೊಳ್ಳುವುದಕ್ಕೆ ಯತ್ನಿಸಿದಾಗ ಮತ್ತೊಮ್ಮೆ ಅಲೆಯಾಗಿ ಮೇಲೇಳುವುದು. ಆದಕಾರಣ ಅದು ಅಲ್ಲೇ ಇತ್ತು. ಇಲ್ಲದೇ ಇದ್ದರೆ ಅದರ ನೆನಪು ಇರುತ್ತಿರಲಿಲ್ಲ. ಆದಕಾರಣ ಪ್ರತಿಯೊಂದು ಕೆಲಸ ಮತ್ತು ಆಲೋಚನೆ, ಅದು ಒಳ್ಳೆಯದಾಗಲೀ ಕೆಟ್ಟದಾಗಲೀ ನಮ್ಮ ಮನಸ್ಸಿನ ಅಂತರಾಳಕ್ಕೆ ಹೋಗಿ, ಸೂಕ್ಷ್ಮವಾಗಿ ಅಲ್ಲೇ ಶೇಖರಿಸಲ್ಪಡುವುವು. ಸುಖ ಮತ್ತು ದುಃಖ ಎರಡು ವಿಧದ ಆಲೋಚನೆಗಳು ಕೂಡ ಕ್ಲೇಶಗಳು. ಏಕೆಂದರೆ ಯೋಗಿಗಳ ಅಭಿಪ್ರಾಯದಲ್ಲಿ ಕೊನೆಗೆ ಇವೆರಡೂ ನಮಗೆ ದುಃಖವನ್ನು ತರುವುವು. ಇಂದ್ರಿಯಗಳ ಮೂಲಕವಾಗಿ ನಮಗೆ ಬರುವ ಸುಖವೆಲ್ಲ ಕೊನೆಗೆ ನಿಜವಾಗಿಯೂ ದುಃಖವನ್ನು ತಂದೇ ತರುವುದು. ಭೋಗವು ಭೋಗದ ದಾಹವನ್ನು ಹೆಚ್ಚಿಸುವುದು. ಅದರ ಪರಿಣಾಮವಾಗಿ ದುಃಖ ಪ್ರಾಪ್ತವಾಗುವುದು. ಮಾನವನ ಆಸೆಗೆ ಪಾರವಿಲ್ಲ. ಅವನು ಆಶಿಸುತ್ತಾ ಹೋಗುತ್ತಾನೆ. ಇನ್ನೂ ಹೆಚ್ಚು ಆಸೆಯನ್ನು ತೃಪ್ತಿಮಾಡಲಾರದ ಸ್ಥಿತಿಗೆ ಬಂದಾಗ ಅವನು ದುಃಖಿಯಾಗುತ್ತಾನೆ. ಆದಕಾರಣವೇ ಯೋಗಿಗಳು ನಮ್ಮ ಸಂಸ್ಕಾರಗಳ ಮೊತ್ತವನ್ನು, ಅದು ಒಳ್ಳೆಯದೇ ಆಗಲೀ, ಕೆಟ್ಟದ್ದೇ ಆಗಲಿ, ಕ್ಲೇಶ ಮೂಲವಾದ ಆತಂಕಗಳು ಎನ್ನುತ್ತಾರೆ. ಆತ್ಮಸ್ವಾರಾಜ್ಯವನ್ನು ಪಡೆಯುವ ಹಾದಿಗೆ ಇವು ಅಡ್ಡಿಯಾಗುತ್ತವೆ. 

\vspace{0.3cm}

ನಮ್ಮ ಎಲ್ಲಾ ಕ್ರಿಯೆಗಳ ಮೂಲವಾದ ಸಂಸ್ಕಾರವೂ ಹಾಗೆಯೆ. ಈ ಜನ್ಮದಲ್ಲಿಯೋ ಅಥವಾ ಮುಂದಿನ ಜನ್ಮದಲ್ಲಿಯೋ ಫಲವನ್ನು ತರುವ ಕಾರಣಗಳು ಇವು. ಕೆಲವು ಅಪೂರ್ವ ಸಂದರ್ಭಗಳಲ್ಲಿ ಸಂಸ್ಕಾರವು ಬಹಳ ಬಲವಾದಾಗ ಬೇಗ ಫಲವನ್ನು ಕೊಡುತ್ತವೆ.\break ಅಸಾಧಾರಣವಾದ ಕೆಟ್ಟ ಕೆಲಸ ಅಥವಾ ಒಳ್ಳೆಯ ಕೆಲಸ ಈ ಜನ್ಮದಲ್ಲೇ ಫಲವನ್ನು ಕೊಡುತ್ತವೆ. ಅದ್ಭುತವಾದ ಉತ್ತಮ ಸಂಸ್ಕಾರಗಳ ಶಕ್ತಿಯನ್ನು ಯಾರು ಈ ಜನ್ಮದಲ್ಲೆಗಳಿಸು\break ತ್ತಾರೋ ಅವರು ಕಾಲವಾಗಬೇಕಾಗಿಲ್ಲ; ಈ ಜನ್ಮದಲ್ಲಿಯೇ ತಮ್ಮ ದೇಹವನ್ನು ದೇವತೆಗಳ ದೇಹವನ್ನಾಗಿ ಮಾರ್ಪಡಿಸಬಲ್ಲರೆಂದು ಯೋಗಿ ತಿಳಿಸುತ್ತಾನೆ. ಇಂತಹ ಹಲವು ಪ್ರಸಂಗಗಳನ್ನು ಯೋಗಿಗಳು ತಮ್ಮ ಶಾಸ್ತ್ರದಲ್ಲಿ ಹೇಳುವರು. ಇವರು ತಮ್ಮ ದೇಹದ ವಸ್ತುವನ್ನೇ ಬದಲಾಯಿಸುವರು. ರೋಗರುಜಿನಗಳು ಬಾರದಂತೆ ಮತ್ತು ನಾವು ತಿಳಿದುಕೊಂಡಿರುವ ಸಾವು ಕೂಡ ಬಾರದಂತೆ ಅವರು ತಮ್ಮ ದೇಹದ ಕಣಗಳನ್ನು ಮಾರ್ಪಡಿಸಬಲ್ಲರು. ಇದು ಏತಕ್ಕೆ ಸಾಧ್ಯವಾಗಬಾರದು? ಶರೀರಶಾಸ್ತ್ರದ ರೀತಿ ಆಹಾರವೆಂದರೆ ಸೂರ್ಯನಿಂದ ಶಕ್ತಿಯನ್ನು ಹೀರಿಕೊಳ್ಳುವುದೆಂದು ಅರ್ಥ. ಈ ಶಕ್ತಿ ಸಸ್ಯವರ್ಗಕ್ಕೆ ತಲುಪುವುದು. ಪ್ರಾಣಿ ಸಸ್ಯವನ್ನು ತಿನ್ನುತ್ತದೆ. ಮನುಷ್ಯ ಪ್ರಾಣಿಯನ್ನು ತಿನ್ನುತ್ತಾನೆ. ಇದರ ರಹಸ್ಯವೇನೆಂದರೆ ನಾವು ಸೂರ್ಯನಿಂದ ಶಕ್ತಿಯನ್ನು ತೆಗೆದುಕೊಂಡು, ನಮ್ಮ ದೇಹದ ಭಾಗವನ್ನಾಗಿ ಮಾಡಿಕೊಳ್ಳುತ್ತೇವೆ. ಹೀಗಿರುವಾಗ ಶಕ್ತಿ ಸಂಗ್ರಹಕ್ಕೆ ಏಕೆ ಒಂದೇ ದಾರಿ ಇರಬೇಕು? ಸಸ್ಯಗಳು ಶಕ್ತಿಯನ್ನು ಸಂಗ್ರಹಿಸುವ ರೀತಿ ನಮ್ಮ ರೀತಿಯಂತೆ ಅಲ್ಲ. ಭೂಮಿಯು ಶಕ್ತಿ ಸಂಗ್ರಹಿಸುವುದಕ್ಕೂ, ನಾವು ಶಕ್ತಿ ಸಂಗ್ರಹಿಸುವುದಕ್ಕೂ ಎಷ್ಟೋ ವ್ಯತ್ಯಾಸವಿದೆ. ಆದರೆ ಎಲ್ಲವೂ ಒಂದಲ್ಲ ಒಂದು ವಿಧದಲ್ಲಿ ಶಕ್ತಿಯನ್ನು ಸಂಗ್ರಹಿಸುತ್ತವೆ. ಸಾಧಾರಣ ರೀತಿಯ ಆವಶ್ಯಕತೆಯೇ ಇಲ್ಲದೆ, ತಮಗೆ ಬೇಕಾದಷ್ಟು ಶಕ್ತಿಯನ್ನು ಮನಸ್ಸಿನ ಮೂಲಕವಾಗಿಯೇ ತಾವು ಸಂಗ್ರಹಿಸುತ್ತೇವೆ ಎಂದು ಯೋಗಿಗಳು ಹೇಳುತ್ತಾರೆ. ಜೇಡರಹುಳು ತನ್ನ ದೇಹದಿಂದಲೇ ತಂತುಗಳನ್ನು ಮಾಡಿ ಅದರಲ್ಲಿ ಬದ್ಧವಾಗಿ, ನೂಲಿನ ಮೇಲಲ್ಲದೆ ಬೇರೆ ಕಡೆ ಹೋಗದಂತೆ ನಾವೂ ಕೂಡ ನಮ್ಮಿಂದಲೇ ನರಗಳನ್ನು ಉತ್ಪತ್ತಿಮಾಡಿ ಅವುಗಳ ಮೂಲಕ ಅಲ್ಲದೆ ಬೇರೆ ವಿಧದಿಂದ ಕೆಲಸವನ್ನು ಮಾಡಲು ಸಾಧ್ಯವಿಲ್ಲದಂತೆ ಮಾಡಿಕೊಂಡಿರುವೆವು. ನಾವು ಅವುಗಳಿಂದ ಬದ್ಧರಾಗಬೇಕಾಗಿಲ್ಲ ಎಂದು ಯೋಗಿಗಳು ಹೇಳುತ್ತಾರೆ. 

ಅದರಂತೆಯೇ ನಾವು ವಿದ್ಯುಚ್ಛಕ್ತಿಯನ್ನು ಜಗತ್ತಿನ ಯಾವ ಮೂಲೆಗೆ ಬೇಕಾದರೂ ಕಳುಹಿಸಬಹುದು. ಆದರೆ ಅದನ್ನು ತಂತಿಯ ಮೂಲಕ ಕಳುಹಿಸಬೇಕು. ತಂತಿಯ ಸಹಾಯವಿಲ್ಲದೆ ಪ್ರಕೃತಿಯು ಪ್ರಚಂಡ ವಿದ್ಯುತ್​ ಪ್ರವಾಹವನ್ನು ಕಳುಹಿಸಬಲ್ಲದು. ನಾವು ಅದನ್ನೇ ಏಕೆ ಮಾಡಬಾರದು? ನಾವು ಮಾನಸಿಕ ವಿದ್ಯುಚ್ಛಕ್ತಿಯನ್ನು ಕಳುಹಿಸಬಹುದು. ನಾವು ಯಾವುದನ್ನು ಮನಸ್ಸು ಎನ್ನುವೆವೋ ಅದು ವಿದ್ಯುಚ್ಛಕ್ತಿಯಂತೆ. ಈ ನರಗಳ ಪ್ರವಾಹದಲ್ಲಿ ಸ್ವಲ್ಪ ವಿದ್ಯುಚ್ಛಕ್ತಿ ಪ್ರವಾಹವಿದೆ ಎಂಬುದೇನೊ ನಿಜ. ಏಕೆಂದರೆ ಇದು ಧ್ರುವದ ಕಡೆ ತಿರುಗುತ್ತದೆ ಮತ್ತು ವಿದ್ಯುಚ್ಛಕ್ತಿಯ ಎಲ್ಲಾ ಕಾರ್ಯ ವಿವರಗಳನ್ನು ಇದು ಸರಿಹೋಲುವುದು. ನಮ್ಮ ವಿದ್ಯುಚ್ಛಕ್ತಿಯನ್ನು ನರಗಳ ಶಾಖೆಯ ಮೂಲಕ ಮಾತ್ರ ಕಳುಹಿಸಬಹುದು. ನಮ್ಮ ಮಾನಸಿಕ ವಿದ್ಯುಚ್ಛಕ್ತಿಯನ್ನು ಇದರ ಸಹಾಯವಿಲ್ಲದೆ ನಾವೇಕೆ ಕಳುಹಿಸಬಾರದು? ಇದು ಸಂಪೂರ್ಣ ಸಾಧ್ಯ; ಮತ್ತು ಇದನ್ನು ಅನುಷ್ಠಾನಕ್ಕೆ ತರಬಹುದು. ಇದನ್ನು ಮಾಡಲು ನಿಮಗೆ ಸಾಧ್ಯವಾದಾಗ, ಪ್ರಪಂಚದಲ್ಲೆಲ್ಲ ಕೆಲಸ ಮಾಡಬಹುದು ಎಂದು ಯೋಗಿ ಹೇಳುತ್ತಾನೆ. ಯಾವ ದೇಹದ ಮೂಲಕವಾಗಿಯಾದರೂ, ಎಲ್ಲಿಂದ ಬೇಕಾದರೂ ನರಗಳ ಸಹಾಯ ವಿಲ್ಲದೆಯೆ ನೀವು ಕೆಲಸ ಮಾಡಬಹುದು. ಈ ನರಗಳ ಮೂಲಕವಾಗಿ ಜೀವನು ಕೆಲಸ ಮಾಡುತ್ತಿರುವಾಗ ಮನುಷ್ಯ ಬದುಕಿರುವನು ಎನ್ನುವೆವು. ಅವುಗಳು ನಿಂತಾಗ ಅವನು ಸತ್ತ ಎನ್ನುತ್ತಾರೆ. ಈ ನರಗಳ ಮೂಲಕವಾಗಿ ಮತ್ತು ಇವುಗಳಿಲ್ಲದೆ ಒಬ್ಬನಿಗೆ ಕೆಲಸ ಮಾಡಲು ಸಾಧ್ಯವಾದಾಗ, ಅವನಿಗೆ ಹುಟ್ಟು ಸಾವುಗಳು ಅರ್ಥವಿಲ್ಲದವುಗಳಾಗುತ್ತವೆ. ಪ್ರಪಂಚದಲ್ಲಿರುವ ದೇಹಗಳೆಲ್ಲ ತನ್ಮಾತ್ರದಿಂದಾಗಿವೆ. ತನ್ಮಾತ್ರಗಳ ಜೋಡಣೆಯಲ್ಲೆ ಒಂದು ದೇಹಕ್ಕೂ ಮತ್ತೊಂದು ದೇಹಕ್ಕೂ ಇರುವ ಭೇದ. ಅದನ್ನು ಜೋಡಿಸುವವರು ನೀವಾದರೆ, ಅದನ್ನು ಬೇಕಾದ ರೀತಿಯಲ್ಲಿ ನೀವು ಜೋಡಿಸಬಹುದು. ನೀವಲ್ಲದೆ ಈ ದೇಹವನ್ನು ಯಾರು ಮಾಡುವರು? ಆಹಾರವನ್ನು ಯಾರು ತಿನ್ನುವರು? ನಿಮಗಾಗಿ ಮತ್ತೊಬ್ಬನು ಊಟ ಮಾಡಿದರೆ ನೀವು ಹೆಚ್ಚು ಕಾಲ ಇರಲಾರಿರಿ. ಆಹಾರದಿಂದ ರಕ್ತವನ್ನು ಯಾರು ಮಾಡುವರು? ನಿಜವಾಗಿಯೂ ನೀವೆ. ರಕ್ತವನ್ನು ಶುದ್ಧಿ ಮಾಡಿ ನಾಳದ ಮೂಲಕ ಯಾರು ಕಳುಹಿಸುವರು? ನೀವೆ. ದೇಹದ ಯಜಮಾನರು ನಾವೆ. ದೇಹದಲ್ಲಿರುವವರು ನಾವೆ. ಮತ್ತೆ ದೇಹವನ್ನು ಹೊಸ ಶಕ್ತಿಯಿಂದ ತುಂಬುವುದನ್ನು ನಾವು ಮರೆತಿರುವೆವು ಅಷ್ಟೆ. ನಾವು ಯಂತ್ರವಾಗಿರುವೆವು, ಅಧೋಗತಿಗೆ ಇಳಿದಿರುವೆವು. ದೇಹದ ಕಣಗಳನ್ನು ಜೋಡಿಸುವ ರೀತಿಯನ್ನು ಮರೆತಿರುವೆವು. ಆದಕಾರಣ ನಾವು ಯಾವುದನ್ನು ಅನೈಚ್ಛಿಕವಾಗಿ ಮಾಡುತ್ತಿರುವೆವೋ ಅದನ್ನು ತಿಳಿದು ಮಾಡಬೇಕು. ಯಜಮಾನರು ನಾವು. ಈ ಜೋಡಣೆಯನ್ನು ನಾವು ನಿಯಂತ್ರಿಸಬೇಕು. ನಮಗೆ ಅದನ್ನು ಮಾಡಲು ಸಾಧ್ಯವಾದೊಡನೆಯೇ ನಮ್ಮ ಇಚ್ಛೆಯಂತೆ ಪುನಃ ಜೀವಶಕ್ತಿಯನ್ನು ತುಂಬಬಹುದು. ಆಗ ನಮಗೆ ಹುಟ್ಟು, ಸಾವು, ರೋಗಗಳು ಇರುವುದಿಲ್ಲ. 

%%\vspace{-0.3cm}

\begin{verse}
ಸತಿ ಮೂಲೇ ತದ್ವಿಪಾಕೋ ಜಾತ್ಯಾಯುರ್ಭೋಗಾಃ~॥ ೧೩~॥
\end{verse}

\vspace{-0.3cm}

\dsize{ಇದರ ಮೂಲವು ಅಲ್ಲಿರುವುದರಿಂದ ಫಲವು ಜಾತಿ, ಆಯುಸ್ಸು, ಸುಖದುಃಖಗಳ ಅನುಭವದ ರೂಪದಲ್ಲಿ ಬರುವುದು. }

\vskip 0.2cm 

ಕಾರಣಭೂತವಾದ ಸಂಸ್ಕಾರಗಳ ಬೇರು ಅಲ್ಲಿರುವುದರಿಂದ ಅವು ವಿಕಾಸವಾಗಿ ಕಾರ್ಯಗಳಾಗುವುವು. ಕಾರ್ಯವು ತಾನು ನಾಶಗೊಂಡು ಸೂಕ್ಷ್ಮವಾಗಿ ಮತ್ತೊಂದು ಕಾರ್ಯಕ್ಕೆ ಕಾರಣವಾಗುವುದು. ಮರದಲ್ಲಿ ಬೀಜ ಬಿಡುವುದು. ಅದು ಮತ್ತೊಂದು ಮರಕ್ಕೆ ಕಾರಣವಾಗುವುದು. ಹೀಗೆ ಅದು ಮುಂದುವರಿಯುವುದು. ನಮ್ಮ ಎಲ್ಲ ಕೆಲಸಗಳೂ, ಹೀಗೆ ಹಿಂದಿನ ಸಂಸ್ಕಾರಗಳ ಪರಿಣಾಮಗಳು. ಈ ಕೆಲಸಗಳು ಸಂಸ್ಕಾರಗಳಾಗಿ ಮುಂದಿನ ಕೆಲಸಗಳಿಗೆ ಕಾರಣಗಳಾಗುತ್ತವೆ. ಹೀಗೆ ನಾವು ಮುಂದುವರಿಯುತ್ತೇವೆ. ಆದಕಾರಣ ಈ ಸೂತ್ರ ಹೀಗೆನ್ನುವುದು: ಕಾರಣವು ಅಲ್ಲಿರುವುದರಿಂದ ಫಲವು ಜಾತಿಯ ರೂಪದಲ್ಲಿ ಬರಲೇಬೇಕು. ಒಬ್ಬನು ಮನುಷ್ಯನಾಗುತ್ತಾನೆ; ಒಬ್ಬನು ಪ್ರಾಣಿಯಾಗುತ್ತಾನೆ; ಒಬ್ಬನು ದೇವತೆಯಾಗುತ್ತಾನೆ, ಮತ್ತೊಬ್ಬನು ರಾಕ್ಷಸನಾಗುತ್ತಾನೆ. ಅನಂತರ ಜೀವನದಲ್ಲಿ ಕರ್ಮಕ್ಕೆ ಸೇರಿದ ಹಲವು ಫಲಗಳಿವೆ. ಒಬ್ಬನು ಐವತ್ತು ವರುಷ ಬದುಕುತ್ತಾನೆ. ಇನ್ನೊಬ್ಬನು ಹುಟ್ಟಿದ ಎರಡು ವರ್ಷಕ್ಕೇ ಕಾಲವಾಗುತ್ತಾನೆ. ಈ ವ್ಯತ್ಯಾಸಗಳೆಲ್ಲವೂ ಕರ್ಮಫಲದಿಂದ ಪ್ರೇರಿತವಾದುವು. ಒಬ್ಬನು ಸುಖಕ್ಕೋಸ್ಕರವಾಗಿಯೇ ಹುಟ್ಟಿದನೆಂದು ತೋರುತ್ತಾನೆ. ಅವನು ಕಾಡಿಗೆ ಹೋಗಿ ಅವಿತುಕೊಂಡರೂ ಸುಖ ಅಲ್ಲಿಯೂ ಅವನನ್ನು ಹುಡುಕಿಕೊಂಡು ಬರುವುದು. ಮತ್ತೊಬ್ಬನು ಎಲ್ಲಿ ಹೋದರೂ ದುಃಖವು ಅವನನ್ನು ಹಿಂಬಾಲಿಸುವುದು. ಎಲ್ಲವೂ ಅವನಿಗೆ ದುಃಖದಾಯಕವೇ ಆಗುತ್ತದೆ. ಎಲ್ಲವೂ ಅವರ ಕರ್ಮಫಲ. ಯೋಗಿಗಳ ಸಿದ್ಧಾಂತದ ಪ್ರಕಾರ ಪುಣ್ಯ ಕೆಲಸಗಳೆಲ್ಲ ಸುಖವನ್ನು ಕೊಡುತ್ತವೆ, ಪಾಪಕಾರ್ಯಗಳು ದುಃಖವನ್ನು ಕೊಡುತ್ತವೆ. ಯಾರು ಪಾಪ ಕೆಲಸಗಳನ್ನು ಮಾಡುತ್ತಾರೆಯೊ ಅವರು ದುಃಖದ ರೂಪದಲ್ಲಿ ಫಲವನ್ನು ಅನುಭವಿಸಿಯೇ ತೀರಬೇಕು. 

\vspace{-0.2cm}

\begin{verse}
ತೇ ಹ್ಲಾದಪರಿತಾಪಫಲಾಃ ಪುಣ್ಯಾಪುಣ್ಯಹೇತುತ್ವಾತ್​~॥ ೧೪~॥
\end{verse}

\vspace{-0.3cm}

\dsize{ಪಾಪಪುಣ್ಯಗಳಿಗೆ ತಕ್ಕಂತೆ ಅವು, ಸುಖದುಃಖ ರೂಪದ ಫಲವನ್ನು ಕೊಡುವುವು. }

\vspace{-0.1cm}

\begin{verse}
ಪರಿಣಾಮ–ತಾಪ–ಸಂಸ್ಕಾರದುಃಖೈರ್ಗುಣವೃತ್ತಿ\\ವಿರೋಧಾಚ್ಚ ದುಃಖಮೇವ ಸರ್ವಂ ವಿವೇಕಿನಃ~॥ ೧೫~॥
\end{verse}

\vspace{-0.3cm}

\dsize{ಏಕೆಂದರೆ ಪರಿಣಾಮ ರೂಪವಾಗಿ ಆಗಲಿ, ಅಥವಾ ಇರುವ ಸುಖವು ಮಾಯವಾಗಬಹುದು ಎಂಬ ಆಲೋಚನೆಯ ಮೂಲಕವೇ ಆಗಲಿ, ಅಥವಾ ಸುಖಸಂಸ್ಕಾರಗಳು ಹೊಸ ಆಸೆಗಳನ್ನು ತೋರುವುದರ ಮೂಲಕವಾಗಿ ಆಗಲಿ ಮತ್ತು ಗುಣಗಳಿಗೆ ವಿರೋಧವಾಗಿರುವುದರಿಂದಲೂ, ವಿವೇಕಿಗೆ ಎಲ್ಲವೂ ದುಃಖಮಯವಾಗಿರುವುವು. }

\vspace{0.1cm}

ವಿವೇಕ ಬುದ್ಧಿಯುಳ್ಳ ಮಾನವನು ಸುಖದುಃಖವೆನ್ನುವುದನ್ನೆಲ್ಲ ಆದ್ಯಂತವಾಗಿ ಪರಿಶೀಲಿಸುತ್ತಾನೆ, ಅವು ಎಲ್ಲಾ ಮಾನವರಿಗೆ ಬರುತ್ತವೆ ಎಂಬುದನ್ನು ತಿಳಿಯುತ್ತಾನೆ. ಒಂದು ಮತ್ತೊಂದನ್ನು ಹಿಂಬಾಲಿಸಿ ಅದರಲ್ಲಿ ಐಕ್ಯವಾಗುತ್ತದೆ ಎಂದು ಯೋಗಿಗಳು ಹೇಳುತ್ತಾರೆ. ಮಾನವರು ತಮ್ಮ ಜೀವನವನ್ನು ಎಂದಿಗೂ ಹಿಂದೆ ಸರಿಯುತ್ತಿರುವ ನೆರಳನ್ನು ಹಿಂಬಾಲಿಸುವುದರಲ್ಲಿ ಕಳೆಯುತ್ತಾರೆ. ಕೊನೆಗೂ ಅವರು ತಮ್ಮ ಆಸೆಯನ್ನು ಪೂರ್ಣ ಮಾಡಿಕೊಳ್ಳಲಾರರು. ಪ್ರಖ್ಯಾತನಾದ ಧರ್ಮರಾಜ ಒಮ್ಮೆ, ಜೀವನದಲ್ಲೆಲ್ಲ ಅತಿ ವಿಚಿತ್ರವಾದುದು ಯಾವುದೆಂದರೆ “ಪ್ರತಿ ಕ್ಷಣವೂ ನಮ್ಮ ಸುತ್ತಲೂ ಜನರು ಸಾಯುವುದನ್ನು ನೋಡುತ್ತೇವೆ, ಆದರೂ ಕೂಡ ನಾವು ಸಾಯುವುದಿಲ್ಲವೆಂದು ಯೋಚಿಸುತ್ತೇವೆ” ಎಂದನು. ಸುತ್ತಲೂ ಮೂಢರಿಂದ ಕೂಡಿಕೊಂಡಿದ್ದರೂ ನಾವು ಅದಕ್ಕೆ ಹೊರತಾದವರು, ನಾವು ಮಾತ್ರ ಪಂಡಿತರು ಎಂದು ಭಾವಿಸುತ್ತೇವೆ. ಎಲ್ಲ ಬಗೆಯ ಅಸ್ಥಿರ ಅನುಭವಗಳಿಂದ ಸುತ್ತುವರಿದಿದ್ದರೂ ನಮ್ಮ ಪ್ರೀತಿಯೊಂದೇ ಸ್ಥಿರವಾದುದೆಂದು ಯೋಚಿಸುತ್ತೇವೆ. ಇದು ಹೇಗೆ ಸಾಧ್ಯ? ಪ್ರೀತಿಯೂ ಕೂಡ ಸ್ವಾರ್ಥಪರವಾದದ್ದೇ, ಸತಿಪತಿಯರ ಪ್ರೀತಿ, ಸುತರು ಸ್ನೇಹಿತರ ಪ್ರೀತಿ ಕೊನೆಗೆ ಕ್ಷೀಣಿಸುವುದು, ಎಲ್ಲಾ ಸ್ವಾರ್ಥ ಎಂದು ಯೋಗಿಯು ಹೇಳುತ್ತಾನೆ. ಪ್ರಪಂಚದಲ್ಲಿ ಎಲ್ಲವನ್ನೂ ನಾಶವು ಆವರಿಸುವುದು, ಕೊನೆಗೆ ಪ್ರೇಮ ಸಹಿತವಾಗಿ ಎಲ್ಲವೂ ನಾಶವಾದ ಮೇಲೆ, ಯಾವುದರಲ್ಲಿಯೂ ಹುರುಳಿಲ್ಲ, ಕನಸಿನಂತೆ ಈ ಜಗತ್ತು ಎಂಬ ಭಾವನೆ ಮಿಂಚಿನಂತೆ ನಮ್ಮ ಮನಸ್ಸಿಗೆ ಹೊಳೆಯುವುದು. ಆಗ ವೈರಾಗ್ಯದ ಕ್ಷಣಿಕದರ್ಶನವಾಗುವುದು, ಆಗ ಇಂದ್ರಿಯಾತೀತ ವಸ್ತುವಿನ ಸುಳಿವು ಗೊತ್ತಾಗುವುದು. ಈ ಪ್ರಪಂಚವನ್ನು ತ್ಯಜಿಸುವುದರಿಂದಲೇ ನಮಗೆ ಆ ಇಂದ್ರಿಯಾತೀತ ವಸ್ತುವು ದೊರಕಬೇಕು. ಇದನ್ನೇ ನೆಚ್ಚಿಕೊಂಡಿದ್ದರೆ ನಮಗೆ ಎಂದಿಗೂ ಸಿಕ್ಕುವುದಿಲ್ಲ. ಇಂದ್ರಿಯ ಸುಖಲಾಲಸೆಗಳನ್ನು ತೊರೆಯದೆ ಮಹಾತ್ಮನಾದ ಯಾವ ಒಬ್ಬ ವ್ಯಕ್ತಿಯೂ ಜಗತ್ತಿಗೆ ಇದುವರೆಗೆ ಬಂದಿಲ್ಲ. ಪ್ರಕೃತಿಯಲ್ಲಿರುವ ಹಲವು ವಿಧದ ಶಕ್ತಿಗಳ ಘರ್ಷಣವೆ ನಮ್ಮ ದುಃಖಕ್ಕೆ ಮೂಲ. ಒಂದು ಒಂದು ಕಡೆ ಸೆಳೆದರೆ ಮತ್ತೊಂದು ಇನ್ನೊಂದು ಕಡೆ ಸೆಳೆಯುವುದು. ಹೀಗಾಗಿ ಶಾಶ್ವತವಾದ ಶಾಂತಿ ದುರ್ಲಭ\break ವಾಗುವುದು. 

\vspace{-0.2cm}

\begin{verse}
ಹೇಯಂ ದುಃಖಮನಾಗತಮ್​~॥ ೧೬~॥
\end{verse}

\vspace{-0.45cm}

\dsize{ಇನ್ನು ಮುಂದೆ ಬರುವ ದುಃಖವನ್ನು ನಿವಾರಿಸತಕ್ಕದ್ದು. }

\vspace{0.2cm}

ಕೆಲವು ಕರ್ಮವನ್ನು ಆಗಲೇ ನಾವು ಪೂರೈಸಿರುವೆವು; ಮತ್ತೆ ಕೆಲವನ್ನು ಈಗ ಅನುಭವಿಸುತ್ತಿರುವೆವು; ಮತ್ತೆ ಕೆಲವು ಮುಂದೆ ಫಲಿಸುವುದಕ್ಕೆ ಕಾದು ಕುಳಿತಿರುವೆವು. ಮೊದಲನೆಯದು ಹಿಂದಿನದು, ಆಗಲೇ ಆಗಿಹೋದುದು. ಎರಡನೆಯದನ್ನು ನಾವು ಅನುಭವಿಸಬೇಕು. ಯಾವುದು ಫಲ ಕೊಡುವುದಕ್ಕೆ ಕಾದು ಕುಳಿತಿದೆಯೋ ಅದನ್ನೇ ನಾವು ನಿಗ್ರಹಿಸಿ ಜಯಿಸಬಹುದು. ನಮ್ಮ ಮನಶ್ಯಕ್ತಿಯನ್ನೆಲ್ಲ ಆ ಗುರಿಯ ಕಡೆಗೆ ತಿರುಗಿಸಬೇಕು. ಸಂಸ್ಕಾರಗಳನ್ನು ಅವುಗಳ ಕಾರಣಾವಸ್ಥೆಗೆ ಲಯಗೊಳಿಸುವುದರ ಮೂಲಕ ಜಯಿಸಬೇಕು ಎಂದು ಪತಂಜಲಿ ಹೇಳಿರುವುದರ ಅರ್ಥ ಇದು. (೨. ೧೦)

\newpage

%%\vspace{-0.3cm}

\begin{verse}
ದ್ರಷ್ಟೃದೃಶ್ಯಯೋಃ ಸಂಯೋಗೋ ಹೇಯಹೇತುಃ~॥ ೧೭~॥
\end{verse}

\vspace{-0.4cm}

\dsize{ಯಾವುದು ನಿವಾರಿಸತಕ್ಕದ್ದೊ ಅದರ ಮೂಲವೇ ದೃಗ್​ ದೃಶ್ಯಗಳ ಸಂಯೋಗ. }

\vskip 0.2cm

ದೃಗ್​ ಯಾರು? ಮಾನವನ ಆತ್ಮ, ಪುರುಷ. ದೃಶ್ಯ ಯಾವುದು? ಮನಸ್ಸಿನಿಂದ ಮೊದಲಾಗಿ, ಸ್ಥೂಲ ವಸ್ತುವಿನವರೆಗೂ ಇರುವ ಇಡೀ ವಿಶ್ವ. ಸುಖದುಃಖಗಳೆಲ್ಲ ಬರುವುದು ಪುರುಷ ಮತ್ತು ಮನಸ್ಸಿನ ಸಂಯೋಗದಿಂದ. ಈ ದರ್ಶನದ ಪ್ರಕಾರ ಪುರುಷ ಪರಿಶುದ್ಧನು; ಪ್ರಕೃತಿಯೊಂದಿಗೆ ಸೇರಿದಾಗ, ಪ್ರತಿಬಿಂಬದ ಮೂಲಕವಾಗಿ ಅವನು ಸುಖದುಃಖಗಳನ್ನು ಅನುಭವಿಸುತ್ತಿರುವಂತೆ ತೋರುವನು ಎಂಬುದನ್ನು ನೀವು ಜ್ಞಾಪಕದಲ್ಲಿಟ್ಟಿರಬೇಕು. 

\vspace{-0.2cm}

\begin{verse}
ಪ್ರಕಾಶ–ಕ್ರಿಯಾ–ಸ್ಥಿತಿಶೀಲಂ ಭೂತೇಂದ್ರಿಯಾತ್ಮಕಂ\\ ಭೋಗಾಪವರ್ಗಾರ್ಥಂ ದೃಶ್ಯಮ್​~॥ ೧೮~॥
\end{verse}

\vspace{-0.4cm}

\dsize{ದೃಶ್ಯವು ಭೂತ ಮತ್ತು ಇಂದ್ರಿಯಗಳಿಂದ ಕೂಡಿದೆ. ಇದು ಪ್ರಕಾಶ, ಚಲನೆ ಮತ್ತು ಜಡತ್ವದಿಂದ ಕೂಡಿದೆ. ಇದು ಇರುವುದು ನೋಡುವವನ ಭೋಗಕ್ಕೆ ಮತ್ತು ಅವನ ಬಿಡುಗಡೆಗೆ. }

\vskip 0.2cm

ದೃಶ್ಯ, ಅಂದರೆ ಪ್ರಕೃತಿ, ಅದು ಪಂಚಭೂತಗಳು ಮತ್ತು ಇಂದ್ರಿಯಗಳಿಂದ ಕೂಡಿದೆ. ಭೂತವೆಂದರೆ ಪ್ರಕೃತಿಯಲ್ಲಿರುವ ಸ್ಥೂಲ ಸೂಕ್ಷ್ಮ ವಸ್ತುಗಳು. ಇಂದ್ರಿಯಗಳೆಂದರೆ\break ಜ್ಞಾನೇಂದ್ರಿಯ ಮತ್ತು ಮನಸ್ಸು. ಈ ದೃಶ್ಯವು ಸತ್ತ್ವ (ಪ್ರಕಾಶ) ರಜಸ್ಸು (ಕ್ರಿಯೆ) ಮತ್ತು ತಮಸ್ಸು (ಸ್ಥಿತಿ ಅಥವಾ ಜಡತ್ವ) – ಈ ಗುಣಗಳಿಂದ ಕೂಡಿದೆ. ಪ್ರಕೃತಿಯ ಗುರಿಯೇನು? ಪುರುಷನು ಅನುಭವವನ್ನು ಹೊಂದಬೇಕು. ಪುರುಷನು ಅದ್ಭುತವಾದ ತನ್ನ ದೈವಿಕ ಶಕ್ತಿಯನ್ನು ಮರೆತಂತೆ ತೋರುವನು. ಒಂದು ಕಥೆ ಇದೆ. ದೇವತೆಗಳ ದೊರೆಯಾದ ಇಂದ್ರನು ಒಮ್ಮೆ ಹಂದಿಯಾಗಿ ಕೆಸರಿನಲ್ಲಿ ಆಡುತ್ತಿದ್ದನು. ಒಂದು ಹೆಣ್ಣು ಹಂದಿ ಮತ್ತು ಬೇಕಾದಷ್ಟು ಮರಿಗಳೊಂದಿಗೆ ಅವನು ಆನಂದದಿಂದ ಇದ್ದನು. ಆಗ ಕೆಲವು ದೇವತೆಗಳು ಇವನ ದುಃಸ್ಥಿತಿಯನ್ನು ಕಂಡು ಹೀಗೆಂದರೆ: “ನೀನು ದೇವತೆಗಳ ಒಡೆಯ. ಎಲ್ಲಾ ದೇವತೆಗಳು ನಿನ್ನ ಕೈಕೆಳಗೆ ಇರುವರು. ನೀನೇಕೆ ಇಲ್ಲಿರುವುದು?” ಇಂದ್ರನು “ಚಿಂತೆ ಇಲ್ಲ, ನಾನು ಇಲ್ಲಿಯೇ ಸುಖ\break ವಾಗಿರುವೆನು. ನನ್ನ ಹೆಂಡತಿ ಮತ್ತು ಮಕ್ಕಳಿರುವಾಗ ನನಗೆ ಸ್ವರ್ಗಸುಖ ಬೇಕಾಗಿಲ್ಲ” ಎಂದನು. ಪಾಪ ದೇವತೆಗಳಿಗೆ ಏನು ಮಾಡಬೇಕೆಂದು ತೋಚದೆ ಹೋಯಿತು. ಅನಂತರ ಹಂದಿಗಳನ್ನೆಲ್ಲ ಒಂದಾದ ಮೇಲೊಂದನ್ನು ಕೊಲ್ಲಲು ಮೊದಲು ಮಾಡಿದರು. ಎಲ್ಲವೂ ಸತ್ತಮೇಲೆ ಇಂದ್ರ ಅಳತೊಡಗಿದನು. ಅನಂತರ ದೇವತೆಗಳು ಇಂದ್ರನ ಹಂದಿಯ ರೂಪವನ್ನು ಸೀಳಿದಾಗ, ಇಂದ್ರನು ಹೊರಗೆ ಬಂದು ತನ್ನ ಹಿಂದಿನ ಸ್ಥಿತಿ ತಿಳಿದ ಮೇಲೆ, “ಎಂತಹ ಘೋರ ಕನಸಿನಲ್ಲಿದ್ದೆ. ದೇವತೆಗಳ ರಾಜನಾದ ನಾನು ಹಂದಿಯಾಗಿ, ಹಂದಿಯ ಜನ್ಮವೊಂದು ಸಾರ್ಥಕವಾದ ಬಾಳು ಎಂದು ನಾನು ಎಣಿಸಿದ್ದೆನಲ್ಲ! ಅಷ್ಟೇ ಅಲ್ಲ, ಪ್ರಪಂಚವೆಲ್ಲವೂ ಹಂದಿಯಾಗಬೇಕೆಂದು ಬಯಸಿದ್ದೆನಲ್ಲ!” ಎಂದುಕೊಂಡನು. ಪುರುಷನು ಪ್ರಕೃತಿಯ ಸಂಪರ್ಕದಿಂದ ತಾನು ಪರಿಶುದ್ಧನೂ ಅನಂತನೂ ಎಂಬುದನ್ನು ಮರೆಯುವನು. ಪುರುಷನು ಪ್ರೀತಿಸುವುದಿಲ್ಲ, ಪ್ರೀತಿ ಅವನ ಸ್ಥಿತಿ. ಅವನು ಅಸ್ತಿತ್ವದಲ್ಲಿರುವನೆಂದಲ್ಲ, ಅಸ್ತಿತ್ವವೇ ಅವನು. ಜೀವನು ತಿಳಿದುಕೊಳ್ಳುವುದಿಲ್ಲ, ತಿಳಿವಳಿಕೆಯೇ ಅವನ ಸ್ವಭಾವ. ಆತ್ಮನನ್ನು ಕುರಿತು ಅದು ಪ್ರೀತಿಸುತ್ತದೆ, ಅದು ಇದೆ, ಅದು ತಿಳಿಯುತ್ತದೆ, ಎಂದು ಹೇಳುವುದು ತಪ್ಪು. ಪ್ರೀತಿ, ಇರುವಿಕೆ, ಜ್ಞಾನ, ಇವು ಪುರುಷನ ಗುಣಗಳಲ್ಲ. ಇವೇ ಪುರುಷನ ಸಾರ. ಈ ಗುಣಗಳು ಬೇರೊಂದು ವಸ್ತುವಿನ ಮೇಲೆ ಪ್ರತಿಬಿಂಬಿಸಿದಾಗ ಇವನ್ನು ಬೇಕಾದರೆ ಆಯಾ ವಸ್ತುಗಳ ಗುಣವೆಂದು ನೀವು ಕರೆಯಬಹುದು. ಆದರೆ ಇವು ಪುರುಷನ ಗುಣಗಳಲ್ಲ. ಅನಂತನೂ, ಜನನ ಮರಣಾತೀತನೂ, ಸ್ವಯಂ ಜ್ಯೋತಿ ಸ್ವರೂಪನೂ ಆದ, ಪರಮಪಾವನನಾದ ಆತ್ಮನ ಸಾರವೇ ಇವು. ಅವನು ತುಂಬ ಹೀನಸ್ಥಿತಿಗೆ ಬಂದಿರುವಂತೆ ತೋರುತ್ತದೆ. ‘ನೀನು ಹಂದಿಯಲ್ಲ’ ಎಂದು ಅದಕ್ಕೆ ಹೇಳಲು ಹೋದರೆ ಅದು ನಿಮ್ಮನ್ನು ಕಚ್ಚಲು ಬರುತ್ತದೆ. 

\vspace{0.3cm}

ಕನಸಿನ ಜಗತ್ತಾದ ಈ ಮಾಯೆಯಲ್ಲಿರುವ ನಾವೆಲ್ಲರೂ ಹೀಗೆ. ಇಲ್ಲಿ ಎಲ್ಲವೂ ದುಃಖ ಮತ್ತು ಅಳು. ಕೆಲವು ಚಿನ್ನದ ಚೆಂಡುಗಳು ಎಸೆಯಲ್ಪಟ್ಟಿವೆ, ಅದನ್ನು ತೆಗೆದುಕೊಳ್ಳುವುದಕ್ಕೆ ನೂಕು ನುಗ್ಗಾಟ. ನಿಯಮಗಳಿಂದ ನೀವೆಂದಿಗೂ ಬದ್ಧರಾಗಿಲ್ಲ. ಪ್ರಕೃತಿಯಲ್ಲಿ ನಿಮ್ಮನ್ನು ಬಂಧಿಸುವ ಸರಪಳಿಯೇ ಇಲ್ಲ. ಯೋಗಿ ಬೋಧಿಸುವುದೇ ಇದನ್ನು. ಇದನ್ನು ಕಲಿತುಕೊಳ್ಳುವುದಕ್ಕೆ ಸ್ವಲ್ಪ ಸಹನೆ ಇರಲಿ. ಪ್ರಕೃತಿಯ ಸಂಪರ್ಕದಿಂದ ಮನಸ್ಸಿನೊಡನೆ ಮತ್ತು ಪ್ರಪಂಚದೊಡನೆ ಐಕ್ಯವಾಗಿ ಪುರುಷನು ತಾನು ದುಃಖಿಯೆಂದು ತಿಳಿಯುತ್ತಾನೆ. ಅನಂತರ ಇದರಿಂದ ತಪ್ಪಿಸಿಕೊಳ್ಳಬೇಕಾದರೆ ಅನುಭವದ ಮೂಲಕ ಎಂದು ಯೋಗಿ ತೋರಿಸುತ್ತಾನೆ. ಈ ಅನುಭವಗಳೆಲ್ಲವನ್ನೂ ನೀವು ಪಡೆಯಬೇಕು. ಆದರೆ ಬೇಗ ಇದನ್ನು ಮುಗಿಸಿ. ನಾವು ಈ ಬಲೆಯಲ್ಲಿ ಬಿದ್ದಿರುವೆವು. ಇದರಿಂದ ತಪ್ಪಿಸಿಕೊಳ್ಳಬೇಕು. ಈ ಬಲೆಯಲ್ಲಿ ಬೀಳುವಂತೆ ನಾವು ಮಾಡಿ ಕೊಂಡಿರುವೆವು. ಅದರಿಂದ ಪಾರಾಗಲು ನಾವು ಯತ್ನಿಸಬೇಕು. ಆದಕಾರಣ ಸತಿ, ಪತಿ, ಸ್ನೇಹಿತರು, ಸುತರು ಎಂಬ ಪ್ರೇಮದ ಅನುಭವವನ್ನೆಲ್ಲ ಪಡೆಯಿರಿ. ನಿಮ್ಮ ನಿಜವಾದ ಸ್ವಭಾವವನ್ನು ನೀವು ಮರೆಯದೆ ಇದ್ದರೆ, ಇವುಗಳಿಂದ ಸುಲಭವಾಗಿ ನೀವು ಪಾರಾಗುವಿರಿ. ಇದು ಕೇವಲ ಕೆಳಗಿನ ಅವಸ್ಥೆ; ನಾವು ಇವುಗಳ ಮೂಲಕ ಹೋಗಬೇಕಾಗಿದೆ ಎಂಬುದನ್ನು ಮರೆಯಬೇಡಿ. ಸುಖದುಃಖಗಳ ಅನುಭವವೇ ಒಂದು ದೊಡ್ಡ ಗುರು. ಆದರೆ ಇದು ಕೇವಲ ಅನುಭವ ಮಾತ್ರ ಎಂಬುದನ್ನು ತಿಳಿಯಿರಿ. ಇದು ನಮ್ಮನ್ನು ಕ್ರಮೇಣ ಮೆಟ್ಟಲುಮೆಟ್ಟಲಾಗಿ ಕರೆದೊಯ್ಯುವುದು –ಪುರುಷನೊಬ್ಬನೇ ಪರಮ ವಸ್ತುವಾಗಿ, ವಿಶ್ವವೆಲ್ಲ ಅಲ್ಪವಾಗಿ, ಕಡಲಿನಲ್ಲಿರುವ ಒಂದು ಜಲಬಿಂದುವಿನಂತೆ ಅದು ತನ್ನ ಶೂನ್ಯತೆಯಿಂದ ತಾನೇ ಕಳಚಿಬೀಳುವ ಸ್ಥಿತಿಗೆ ನಮ್ಮನ್ನು ಕರೆದೊಯ್ಯುವುದು. ನಾನಾ ಆಭರಣಗಳನ್ನು ನಾವು ಪಡೆಯಬೇಕು. ಆದರೆ ಗುರಿಯನ್ನು ನಾವು ಎಂದಿಗೂ ಮರೆಯದಿರಬೇಕು. 

\vspace{-0.2cm}

\begin{verse}
ವಿಶೇಷಾವಿಶೇಷ–ಲಿಂಗಮಾತ್ರಾಲಿಂಗಾನಿ ಗುಣಪರ್ವಾಣಿ~॥ ೧೯~॥
\end{verse}

\vspace{-0.4cm}

\dsize{ವಿಶೇಷ, ಅವಿಶೇಷ, ನಿರ್ದೇಶಿತ ಮಾತ್ರ ಮತ್ತು ಅನಿರ್ದೇಶಿತ ಇವು ಗುಣಗಳ ಅವಸ್ಥೆಗಳು. }

\newpage

ನಾನು ನಿಮಗೆ ಹಿಂದೆ ಹೇಳಿದಂತೆ ಯೋಗವು ಸಾಂಖ್ಯತತ್ತ್ವವನ್ನು ಅವಲಂಬಿಸಿದೆ. ನಾನು ನಿಮಗೆ ಪುನಃ ಸಾಂಖ್ಯ ಸಿದ್ಧಾಂತದ ಸೃಷ್ಟಿವಾದವನ್ನು ಹೇಳುತ್ತೇನೆ. ಸಾಂಖ್ಯರ ದೃಷ್ಟಿಯಲ್ಲಿ ಪ್ರಕೃತಿಯೇ ಸೃಷ್ಟಿಯ ಉದಾದಾನ ಮತ್ತು ನಿಮಿತ್ತ ಕಾರಣ. ಪ್ರಕೃತಿಯಲ್ಲಿ ಸತ್ತ್ವ, ರಜಸ್ಸು, ತಮಸ್ಸು ಎಂಬ ಮೂರು ವಿಧದ ವಸ್ತುಗಳಿವೆ. ಕತ್ತಲೆಯಿಂದ ಕವಿದಿರುವುದು, ಅಜ್ಞಾನದಲ್ಲಿರುವುದು ಮತ್ತು ಭಾರವಾಗಿರುವುದೆಲ್ಲ ತಮಸ್ಸು. ರಜಸ್ಸೆ ಚಟುವಟಿಕೆ. ಸತ್ತ್ವವೇ ಶಾಂತಿ ಮತ್ತು ಜ್ಞಾನ. ಸೃಷ್ಟಿಗೆ ಮುಂಚಿನ ಪ್ರಕೃತಿಯನ್ನು ಅವರು ಅವ್ಯಕ್ತವೆನ್ನುವರು. ಅಲ್ಲಿ ನಾಮರೂಪಗಳ ಭೇದವಿಲ್ಲ. ಮೂರು ಗುಣಗಳೂ ಕೂಡ ಸಮತ್ವದಲ್ಲಿರುತ್ತವೆ. ಅನಂತರ ಅದರ ಸಮತ್ವಕ್ಕೆ ಕುಂದು ಬರುವುದು. ಆಗ ಗುಣಗಳು ಒಂದು ಮತ್ತೊಂದರೊಂದಿಗೆ ಹಲವು ವಿಧದಲ್ಲಿ ಸೇರುವುವು. ಆಗಲೆ ಜಗತ್ತು ಸೃಷ್ಟಿಯಾಗುವುದು. ಪ್ರತಿಯೊಬ್ಬ ಮಾನವನಲ್ಲಿಯೂ ಈ ಮೂರು ಗುಣಗಳಿವೆ. ಸತ್ತ್ವಗುಣ ಮೇಲಾದಾಗ ಜ್ಞಾನ, ರಜೋಗುಣ ಮೇಲಾದಾಗ ಚಟುವಟಿಕೆ, ತಮೋಗುಣ ಮೇಲಾದಾಗ ಕತ್ತಲೆ, ಆಯಾಸ, ಸೋಮಾರಿತನ ಮತ್ತು ಅಜ್ಞಾನ ಪ್ರಾಪ್ತವಾಗುವುವು. ಸಾಂಖ್ಯ ಸಿದ್ಧಾಂತದ ಪ್ರಕಾರ ಈ ಮೂರು ಗುಣಗಳನ್ನೊಳಗೊಂಡ ಸೃಷ್ಟಿಯ ಅತ್ಯುನ್ನತ ಅಭಿವ್ಯಕ್ತಿಯನ್ನು ಮಹತ್​ ಅಥವಾ ವಿಶ್ವವ್ಯಾಪಿ ಪ್ರಜ್ಞೆ ಎಂದು ಕರೆಯಲಾಗಿದೆ. ಪ್ರತಿಯೊಬ್ಬ ಮಾನವನ ಬುದ್ಧಿಯೂ ಕೂಡ ಇದರ ಒಂದು ಅಂಶವಾಗಿದೆ. ಸಾಂಖ್ಯರ ಮನಶ್ಯಾಸ್ತ್ರದಲ್ಲಿ ಮನಸ್ಸಿಗೂ ಬುದ್ಧಿಗೂ ಸ್ಪಷ್ಟವಾದ ವ್ಯತ್ಯಾಸವಿದೆ. ಮನಸ್ಸಿನ ಕೆಲಸ ಇಂದ್ರಿಯಗ್ರಹಣಗಳನ್ನು ಒಟ್ಟುಗೂಡಿಸಿ ಅದನ್ನು ಬುದ್ಧಿಗೆ (ವ್ಯಕ್ತಿಗತವಾದ ಮಹತ್​) ತೆಗೆದುಕೊಂಡು ಹೋಗುವುದು. ಬುದ್ಧಿಯು ಅದರ ಮೇಲೆ ನಿರ್ಧರಿಸುವುದು. ಮಹತ್​ನಿಂದ ಅಹಂಕಾರ ಬರುವುದು. ಅಹಂಕಾರದಿಂದ ತನ್ಮಾತ್ರಗಳು ಬರುವುವು. ಇವು ಒಂದರೊಡನೊಂದು ಕಲೆತು ಸ್ಥೂಲವಾದ ಬಾಹ್ಯಪ್ರಪಂಚವಾಗುವುದು. ಬುದ್ಧಿಯಿಂದ ಹಿಡಿದು ಕಲ್ಲಿನ ಬಂಡೆಯವರೆವಿಗೆ ಎಲ್ಲಾ ಆಗಿರುವುದು ಒಂದು ವಸ್ತುವಿನಿಂದ. ಅವುಗಳಿಗೆ ಇರುವ ವ್ಯತ್ಯಾಸವೆಲ್ಲ ಸ್ಥೂಲ ಸೂಕ್ಷ್ಮದ ಅಸ್ತಿತ್ವದಲ್ಲಿ ಎಂಬುದೇ ಸಾಂಖ್ಯರ ಸಿದ್ಧಾಂತ. ಸೂಕ್ಷ್ಮವು ಕಾರಣ, ಸ್ಥೂಲವು ಅದರಿಂದ ಆದ ಪರಿಣಾಮ. ಸಾಂಖ್ಯ ಸಿದ್ಧಾಂತದ ಪ್ರಕಾರ ಸೃಷ್ಟಿಯ ಆಚೆ ಇರುವವನು ಪುರುಷ. ಅದು ಜಡವಲ್ಲ. ಪುರುಷನು ಬುದ್ಧಿ ಅಥವಾ ತನ್ಮಾತ್ರಗಳನ್ನು ಹೋಲುವುದಿಲ್ಲ. ಮತ್ತಾವ ವಸ್ತುಗಳ ಸಂಯೋಗದಿಂದಲೂ ಆಗಿಲ್ಲದ ಕಾರಣ, ಪುರುಷನು ಜನನ–ಮರಣಾತೀತ ಎಂದು ಸಾಂಖ್ಯರು ವಾದಿಸುತ್ತಾರೆ. ಪರಸ್ಪರ ವಸ್ತುಗಳ ಸಂಯೋಗದಿಂದ ಯಾವುದು ಆಗಿಲ್ಲವೋ ಅದು ಸಾಯಲಾರದು. ಅಂತಹ ಪುರುಷರು ಅಥವಾ ಆತ್ಮರು ಅನಂತ ಸಂಖ್ಯೆಯಲ್ಲಿರುವರು. 

ಈಗ ಸೂತ್ರದಲ್ಲಿ ಹೇಳಿರುವ ಗುಣಗಳ ವಿವಿಧ ಅವಸ್ಥೆಗಳನ್ನು ಅರ್ಥಮಾಡಿ ಕೊಳ್ಳೋಣ. ನಾವು ತಿಳಿಯಬಲ್ಲ ಸ್ಥೂಲವಸ್ತುಗಳಿಗೆ ವಿಶೇಷ ವಸ್ತುಗಳೆಂದು ಹೆಸರು. ಅವಿಶೇಷವೆಂದರೆ ಸಾಧಾರಣ ಮನುಷ್ಯರಿಗೆ ಗೋಚರಿಸಲಾರದ ತನ್ಮಾತ್ರಗಳು. ಆದರೆ ಯೋಗವನ್ನು ನೀವು ಅಭ್ಯಾಸ ಮಾಡಿದರೆ ಕೆಲವು ಕಾಲದಮೇಲೆ ನಿಮ್ಮ ಇಂದ್ರಿಯ ಗ್ರಹಣಶಕ್ತಿ ಬಹಳ ಸೂಕ್ಷ್ಮವಾಗಿ ತನ್ಮಾತ್ರವನ್ನೂ ನೀವು ನೋಡಬಹುದೆಂದು ಯೋಗಿ ಹೇಳುತ್ತಾನೆ. ಉದಾಹರಣೆಗೆ\break\  ಪ್ರತಿಯೊಬ್ಬ ಮನುಷ್ಯನ ಸುತ್ತಲೂ ಒಂದು ಕಾಂತಿ ಇದೆ ಎಂಬುದನ್ನು ನೀವು ಕೇಳಿರಬಹುದು. ಪ್ರತಿಯೊಂದು ಜೀವಿಸಿರುವ ಪ್ರಾಣಿಯೂ ಕೂಡ ಒಂದು ವಿಧವಾದ ಕಾಂತಿಯನ್ನು ಹೊರಗೆಡಹುತ್ತದೆ. ಈ ಕಾಂತಿಯನ್ನು ನೋಡಬಹುದೆಂದು ಯೋಗಿ ಹೇಳುತ್ತಾನೆ. ನಾವೆಲ್ಲರೂ ಇದನ್ನು ನೋಡುವುದಿಲ್ಲ. ಆದರೆ ಹೂವು ಹೇಗೆ ನಿರಂತರವೂ ಸೂಕ್ಷ್ಮವಾದ ಕಣಗಳನ್ನು ಹೊರಗೆಸೆಯುವುದರಿಂದ ನಾವು ಅದರ ವಾಸನೆಯನ್ನು ಸೇವಿಸುತ್ತೇವೆಯೋ, ಹಾಗೆಯೇ ನಾವು ತನ್ಮಾತ್ರವನ್ನು ಹೊರಗೆಡಹುತ್ತೇವೆ. ನಮ್ಮ ಜೀವನದ ಪ್ರತಿದಿನದಲ್ಲಿಯೂ ಒಳ್ಳೆಯದನ್ನೋ ಕೆಟ್ಟುದನ್ನೋ ಒಂದು ಪ್ರಮಾಣದಲ್ಲಿ ಹೊರಗೆಡಹುತ್ತೇವೆ. ನಾವು ಹೋದ ಎಡೆಯಲ್ಲೆಲ್ಲ ವಾತಾವರಣ ಇದರಿಂದ ತುಂಬಿ ತುಳುಕಾಡುತ್ತದೆ. ಇದರಿಂದಾಗಿಯೇ ಮಾನವನಿಗೆ, ಅವನ ಅರಿವಿಲ್ಲದೆಯೇ, ದೇವಸ್ಥಾನಗಳನ್ನು, ಚರ್ಚುಗಳನ್ನು ಕಟ್ಟುವ ಭಾವನೆ ಬಂದದ್ದು. ದೇವರನ್ನು ಪೂಜಿಸುವುದಕ್ಕಾಗಿ ಮನುಷ್ಯ ಚರ್ಚುಗಳನ್ನು ಏಕೆ ಕಟ್ಟಬೇಕು? ಪರಮಾತ್ಮನನ್ನು ಎಲ್ಲಿ ಬೇಕಾದರೂ ಏಕೆ ಪೂಜಿಸಕೂಡದು? ಮನುಷ್ಯರಿಗೆ ಕಾರಣ ಗೊತ್ತಿಲ್ಲದೆ ಇದ್ದರೂ ಜನರು ಎಲ್ಲಿ ದೇವರನ್ನು ಪೂಜಿಸುತ್ತಾರೆಯೋ ಆ ಸ್ಥಳವು ಒಳ್ಳೆಯ ತನ್ಮಾತ್ರದಿಂದ ತುಂಬಿರುವುದು ಎಂದು ಅವರಿಗೆ ತಿಳಿದು ಬಂತು. ಪ್ರತಿ ದಿನವೂ ಜನರು ಅಲ್ಲಿಗೆ ಹೋಗುತ್ತಾರೆ. ಅಲ್ಲಿಗೆ ಹೆಚ್ಚು ಹೋದಷ್ಟೂ ಅವರು ಉತ್ತಮರಾಗುತ್ತಾರೆ. ಆ ಸ್ಥಳವೂ ಪವಿತ್ರವಾಗುತ್ತದೆ. ಯಾರಲ್ಲಿಯಾದರೂ ಹೆಚ್ಚು ಸಾತ್ತ್ವಿಕಗುಣ ಇಲ್ಲದೆ ಇದ್ದರೆ, ಅವನು ಅಲ್ಲಿಗೆ ಹೋದರೆ ಸ್ಥಳ ಅವನ ಮೇಲೆ ತನ್ನ ಪ್ರಭಾವವನ್ನು ಬೀರಿ, ಅವನಲ್ಲಿರುವ ಸತ್ತ್ವಗುಣವನ್ನು ಜಾಗ್ರತ ಮಾಡುತ್ತದೆ. ಆದಕಾರಣ ದೇವಸ್ಥಾನ, ತೀರ್ಥಸ್ಥಳ ಇವುಗಳ ಮಹಿಮೆಯೆಲ್ಲ ಇದರಲ್ಲಿದೆ. ಪುಣ್ಯಾತ್ಮರು ಅಲ್ಲಿಗೆ ಬರುವುದರ ಮೇಲೆ ಅದರ ಪವಿತ್ರತೆ ನಿಂತಿದೆ ಎಂಬುದನ್ನು ಮರೆಯಬಾರದು. ಮನುಷ್ಯನಲ್ಲಿರುವ ಕುಂದು ಏನೆಂದರೆ, ಅವನು ಮೊದಲಿನ ಅರ್ಥವನ್ನು ಮರೆತು ಕುದುರೆಯ ಮುಂದೆ ಗಾಡಿಯನ್ನು ಕಟ್ಟುತ್ತಾನೆ. ಆ ಸ್ಥಳವನ್ನು ಪವಿತ್ರವನ್ನಾಗಿ ಮಾಡಿದವರು ಮನುಷ್ಯರು. ಅನಂತರ ಪರಿಣಾಮವೇ ಕಾರಣವಾಗಿ ಮನುಷ್ಯರನ್ನು ಪವಿತ್ರರನ್ನಾಗಿ ಮಾಡಿತು. ಅಲ್ಲಿಗೆ ದುಷ್ಟರು ಮಾತ್ರ ಹೋದರೆ ಎಲ್ಲಾ ಸ್ಥಳದಂತೆ ಅದು ಅಪವಿತ್ರವಾಗುವುದು. ಒಂದು ಕಟ್ಟಡವನ್ನು ಚರ್ಚನ್ನಾಗಿ ಮಾಡುವುದು ಅಲ್ಲಿಗೆ ಹೋಗುವ ಜನರು, ಬರಿಯ ಕಟ್ಟಡವಲ್ಲ. ನಾವು ಯಾವಾಗಲೂ ಮರೆಯುವುದೇ ಅದನ್ನು. ಆದಕಾರಣವೇ ಸತ್ತ್ವಗುಣಾಧಿಕರಾದ ಋಷಿಗಳು ಮತ್ತು ಮಹಾತ್ಮರು ಹಗಲೂ ರಾತ್ರಿ ತಮ್ಮ ಶಕ್ತಿಯನ್ನು ಹೊರಗೆ ಕಳುಹಿಸಿ ವಾತಾವರಣವನ್ನು ಪವಿತ್ರವನ್ನಾಗಿ ಮಾಡಲು ಯತ್ನಿಸುವರು. ಪವಿತ್ರತೆಯೇ ರೂಪದಾಳಿದೆಯೇ ಎನ್ನುವಷ್ಟು ಒಬ್ಬನು ನಮಗೆ ಕಾಣಿಸಬಹುದು. ಯಾರು ಅವನ ಸಂಪರ್ಕವನ್ನು ಪಡೆಯುತ್ತಾರೆಯೋ ಅವರು ಪರಿಶುದ್ಧರಾಗುವರು. 

“ನಿರ್ದೇಶಿತ ಮಾತ್ರ” ಎಂದರೆ ಬುದ್ಧಿ ಎಂದು ಅರ್ಥ. ಪ್ರಕೃತಿಯ ಮೊದಲ ಅಭಿವ್ಯಕ್ತಿ ಬುದ್ಧಿ. ಇದರಿಂದ ಉಳಿದ ಅಭಿವ್ಯಕ್ತಿಗಳೆಲ್ಲ ಬರುವುವು. ಕೊನೆಯದೇ “ಅನಿರ್ದೇಶಿತ”. ಈ ವಿಷಯದಲ್ಲಿ ಆಧುನಿಕ ವಿಜ್ಞಾನ ಶಾಸ್ತ್ರಕ್ಕೂ ಧರ್ಮಗಳಿಗೂ ಬಹಳ ವ್ಯತ್ಯಾಸವಿರುವಂತೆ ತೋರುವುದು. ಪ್ರತಿಯೊಂದು ಧರ್ಮದಲ್ಲಿಯೂ ಕೂಡ ಜಗತ್ತು ಪ್ರಜ್ಞೆಯಿಂದ ಆದುದು ಎಂಬ ಭಾವನೆ ಇದೆ. ಈಶ್ವರ ಸಿದ್ಧಾಂತವನ್ನು, ಅದರ ಸಾಕಾರ ಭಾವನೆಯನ್ನು ಬಿಟ್ಟು ಮನಶ್ಯಾಸ್ತ್ರದ ರೀತಿ ತೆಗೆದುಕೊಂಡರೆ, ಸೃಷ್ಟಿಯ ಕ್ರಮದಲ್ಲಿ ಬರುವ ಮೊದಲನೆಯದೇ ಪ್ರಜ್ಞೆ, ಅನಂತರ ಉಳಿದ ಸ್ಥೂಲ ವಸ್ತುಗಳು. ಆಧುನಿಕ ತತ್ತ್ವಜ್ಞರು ಪ್ರಜ್ಞೆಯೇ ಕೊನೆ ಎನ್ನುತ್ತಾರೆ. ಅಚೇತನವಾದ ವಸ್ತುಗಳು ಕ್ರಮೇಣ ಪ್ರಾಣಿಗಳಾಗಿ ವಿಕಾಸವಾಗುವುವು. ಅನಂತರ ಪ್ರಾಣಿಗಳಿಂದ ಮನುಷ್ಯ ಬರುತ್ತಾನೆ. ಬುದ್ಧಿಯಿಂದ ಎಲ್ಲವೂ ಬರುವುದರ ಬದಲು, ಬುದ್ಧಿಯೇ ಕೊನೆಗೆ ಬರುವುದು ಎಂದು ಅವರ ಅಭಿಪ್ರಾಯ. ಧಾರ್ಮಿಕ ಮತ್ತು ವೈಜ್ಞಾನಿಕ ಸಿದ್ಧಾಂತಗಳು ತೋರಿಕೆಗೆ ವಿರೋಧವಾಗಿ ಕಂಡರೂ ಎರಡೂ ಸರಿ. ಆದಿ ಅಂತ್ಯ ವಿಲ್ಲದೆ ಎ–ಬಿ–ಎ–ಬಿ ಎಂಬ ಸರಪಳಿಯನ್ನು ತೆಗೆದುಕೊಳ್ಳೋಣ. ಇದರಲ್ಲಿ ಮೊದಲನೆಯದು ಯಾವುದು–ಎ ಅಥವಾ ಬಿ?–ಎಂಬುದೆ ಪ್ರಶ್ನೆ. ನೀವು ಸರಪಳಿಯನ್ನು ಎ–ಬಿ–ಎ–ಬಿ ಎಂದು ತೆಗೆದುಕೊಂಡರೆ ಆಗ ಎ ಮೊದಲು ಅನಂತರ ಬಿ. ಸರಪಳಿಯನ್ನು ಬಿ–ಎ–ಬಿ–ಎ ಎಂದು ತೆಗೆದುಕೊಂಡರೆ ಬಿ ಮೊದಲು ಅನಂತರ ಎ ಬರುವುದು. ಪ್ರಜ್ಞೆ ವಿಕಾಸವಾಗಿ ಸ್ಥೂಲವಾಗುವುದು; ಇದು ಪುನಃ ಪ್ರಜ್ಞೆ ಯಾಗುವುದು. ಹೀಗೆಯೇ ತುದಿ ಮೊದಲಿಲ್ಲದೆ ಹೋಗುವುದು. ಸಾಂಖ್ಯ ಮತ್ತು ಇತರ ಧರ್ಮಗಳು ಮೊದಲು ಪ್ರಜ್ಞೆಯನ್ನು ಇಡುತ್ತವೆ. ಆಗ ಸರಪಳಿ ಪ್ರಜ್ಞೆ– ವಸ್ತು–ಪ್ರಜ್ಞೆ–ವಸ್ತು–ವಾಗುವುದು. ವಿಜ್ಞಾನಿ ತನ್ನ ಬೆರಳನ್ನು ವಸ್ತುವಿನ ಮೇಲೆ ಇಡುವನು. ಆಗ ಅದು ವಸ್ತು–ಪ್ರಜ್ಞೆಯಾಗುವುದು. ಇಬ್ಬರೂ ಕೂಡ ಒಂದೇ ಸರಪಳಿಯನ್ನು ತೋರುವರು. ಆದರೂ ಭಾರತೀಯ ತತ್ತ್ವಶಾಸ್ತ್ರ, ವಸ್ತು ಮತ್ತು ಪ್ರಜ್ಞೆಯನ್ನು ಮೀರಿಹೋಗಿ, ಪ್ರಜ್ಞೆಯ ಆಚೆ ಇರುವ ಪುರುಷನನ್ನು ಕಾಣುವುದು. ಪ್ರಜ್ಞೆಯು ಪುರುಷನ ಪ್ರಭೆಯಿಂದ ಬೆಳಗುತ್ತಿದೆ. 

\vspace{-0.2cm}

\begin{verse}
ದ್ರಷ್ಟಾದೃಶಿಮಾತ್ರಃ ಶುದ್ಧೋಽಪಿ ಪ್ರತ್ಯಯಾನುಪಶ್ಯಃ~॥ ೨೦~॥
\end{verse}

\vspace{-0.5cm}

\dsize{ನೋಡುವವನು ಚೈತನ್ಯಮಾತ್ರನು. ಶುದ್ಧನಾದರೂ ಬುದ್ಧಿಯ ಉಪಾಧಿಯ ಮೂಲಕ ನೋಡುತ್ತಾನೆ. }

\vskip 0.2cm

ಇದು ಪುನಃ ಸಾಂಖ್ಯತತ್ತ್ವ. ಅದೇ ಸಿದ್ಧಾಂತದ ಪ್ರಕಾರ ಅತಿ ಕ್ಷುದ್ರವಾದ ವಸ್ತುವಿನಿಂದ ಹಿಡಿದು ಪ್ರಜ್ಞೆಯವರೆಗೆ ಎಲ್ಲವೂ ಪ್ರಕೃತಿ ಎಂಬುದನ್ನು ನಾವಾಗಲೇ ನೋಡಿದೆವು. ಪ್ರಕೃತಿಯಾಚೆ ಪುರುಷ ಇರುವನು. ನಿರ್ಗುಣನವನು. ಹಾಗಾದರೆ ಜೀವನು ಸುಖಿ ಮತ್ತು ದುಃಖಿಯಂತೆ ಏಕೆ ತೋರುತ್ತಾನೆ? ಪ್ರತಿಬಿಂಬದ ಮೂಲಕ. ಬಿಳಿಯ ಸ್ಫಟಿಕಮಣಿಯ ಹತ್ತಿರ ಕೆಂಪು ಹೂವನ್ನು ಇಟ್ಟರೆ ಸ್ಫಟಿಕಮಣಿ ಕೆಂಪಾಗಿ ತೋರುವುದು. ಅದರಂತೆಯೇ ಜೀವನದ ಸುಖ ದುಃಖಗಳ ತೋರಿಕೆಯೂ ಕೂಡ. ಜೀವನಿಗೆ ಯಾವ ಬಣ್ಣವೂ ಇಲ್ಲ, ಜೀವನು ಪ್ರಕೃತಿಯಿಂದ ಬೇರೆ. ಪ್ರಕೃತಿಯೇ ಒಂದು, ಜೀವನೇ ಒಂದು. ಎರಡಕ್ಕೂ ಶಾಶ್ವತವಾದ ಭೇದವಿದೆ. ಪ್ರಜ್ಞೆ ಒಂದು ಮಿಶ್ರವಸ್ತು. ಇದು ಅಭಿವೃದ್ಧಿಯಾಗುತ್ತದೆ, ಕ್ಷೀಣಿಸುತ್ತದೆ. ದೇಹ ಬದಲಾಯಿಸಿದಂತೆ ಇದೂ ಕೂಡ ವಿಕಾಸಕ್ಕೆ ಒಳಪಟ್ಟಿದೆ. ಇದರ ಸ್ವಭಾವ ಕೂಡ ಹೆಚ್ಚು ಕಡಮೆ ದೇಹದ ಸ್ವಭಾವದಂತೆ ಎಂದು ಸಾಂಖ್ಯರು ಹೇಳುತ್ತಾರೆ. ದೇಹಕ್ಕೆ ಉಗುರು ಹೇಗೋ ಹಾಗೆಯೇ ಪ್ರಜ್ಞೆಗೆ ದೇಹ. ಉಗುರು ದೇಹದ ಅಂಶ, ಆದರೆ ಅದನ್ನು ನಾವು ನೂರು ವೇಳೆ ಕತ್ತರಿಸಿದರೂ ದೇಹ ನಿಲ್ಲುವುದು. ಅದರಂತೆಯೇ ದೇಹವನ್ನು ಆಚೆಗೆ ಎಸೆದರೂ ಪ್ರಜ್ಞೆ ಯುಗಯುಗಾಂತರಗಳವರೆವಿಗೂ ಇರಬಲ್ಲದು. ಆದರೂ ಕೂಡ ಪ್ರಜ್ಞೆಯೂ ಅಮರವಾಗಲಾರದು. ಏಕೆಂದರೆ ಇದು ಏರಿಳಿತಗಳಿಗೆ ಒಳಪಟ್ಟಿದೆ. ವಿಕಾರಕ್ಕೆ ಒಳಪಟ್ಟಿರುವುದು ಅಮರವಾಗಲಾರದು. ನಿಜವಾಗಿಯೂ ಪ್ರಜ್ಞೆ ತಯಾರಿಸಲ್ಪಟ್ಟಿರುವುದು. ಇದರಾಚೆ ಯಾವುದೋ ಒಂದು ಇದೆ ಎಂದು ತೋರುವುದಕ್ಕೆ ಇದೊಂದೇ ಸಾಕು. ಇದು ಎಂದಿಗೂ ಸ್ವತಂತ್ರವಾಗಲಾರದು. ವಸ್ತುವಿಗೆ ಸಂಬಂಧಪಟ್ಟಿರುವುದೆಲ್ಲವೂ ಪ್ರಕೃತಿಯಲ್ಲಿದೆ. ಆದಕಾರಣ ಅದು ಎಂದೆಂದಿಗೂ ಬಂಧನದಲ್ಲಿರುವುದು. ಯಾರು ಸ್ವತಂತ್ರರಾಗಿರುವರು? ಸ್ವತಂತ್ರವಾಗಿರುವುದು ನಿಜವಾಗಿಯೂ ಕಾರ್ಯಕಾರಣಗಳಾಚೆ ಇರಬೇಕು. ಸ್ವಾತಂತ್ರ್ಯವೆನ್ನುವುದು ಒಂದು ಭ್ರಾಂತಿ ಎಂದು ನೀವು ಹೇಳಿದರೆ, ನಾನು ಬದ್ಧನೆಂಬುದೂ ಒಂದು ಭ್ರಾಂತಿ ಎಂದು ಹೇಳುತ್ತೇನೆ. ಮುಕ್ತಿ ಮತ್ತು ಬಂಧನವೆಂಬ ಎರಡರ ಭಾವನೆಯೂ ನಮ್ಮ ಮನಸ್ಸಿಗೆ ಬರುತ್ತದೆ. ಒಂದು ನಿಂತರೆ ಮತ್ತೊಂದೂ ನಿಲ್ಲುವುದು. ಒಂದು ಬಿದ್ದರೆ ಇನ್ನೊಂದು ಬೀಳುವುದು ಇವುಗಳೆ ಮುಕ್ತಿ ಬಂಧನಗಳೆಂಬ ನಮ್ಮ ಭಾವನೆಗಳು. ನಾವು ಒಂದು ಗೋಡೆಯ ಮೂಲಕವಾಗಿ ಹೋಗಬೇಕಾದರೆ, ನಮ್ಮ ತಲೆ ಗೋಡೆಗೆ ಹೊಡೆದಾಗ ಆ ಗೋಡೆಯಿಂದ ನಾವು ತಡೆಯಲ್ಪಟ್ಟಿರುವೆವು ಎಂಬುದನ್ನು ನೋಡುವೆವು. ಆದರೂ ಕೂಡ ನಮ್ಮಲ್ಲಿ ಒಂದು ಇಚ್ಛಾಶಕ್ತಿ ಇದೆ. ಅದನ್ನು ನಾವು ಎಲ್ಲಿಗೆ ಬೇಕಾದರೂ ಹರಿಸಬಹುದು ಎಂಬುದು ನಮಗೆ ತೋರುತ್ತದೆ. ಹೆಜ್ಜೆಹೆಜ್ಜೆಗೂ ವಿರುದ್ಧ ಭಾವನೆಗಳು ಬರುತ್ತವೆ. ಸ್ವತಂತ್ರರು ನಾವು ಎಂದು ನಂಬಬೇಕಾಗಿದೆ. ಆದರೂ ಪ್ರತಿಯೊಂದು ಹೆಜ್ಜೆಗೂ ನಾವು ಸ್ವತಂತ್ರರಲ್ಲವೆಂದು ನಮಗೆ ತೋರುತ್ತದೆ. ಒಂದು ಭಾವನೆ ಭ್ರಾಂತಿಯಾದರೆ ಮತ್ತೊಂದು ಭಾವನೆಯೂ ಭ್ರಾಂತಿ. ಒಂದು ಭಾವನೆ ಸತ್ಯವಾದರೆ ಮತ್ತೊಂದು ಭಾವನೆಯೂ ಸತ್ಯ. ಏಕೆಂದರೆ ಎರಡೂ ಕೂಡ ಒಂದೇ ತಳಹದಿಯಾದ ಪ್ರಜ್ಞೆಯ ಮೇಲೆ ನಿಂತಿರುವುವು. ಎರಡು ಕೂಡ ಸತ್ಯವೆಂದು ಯೋಗಿ ಹೇಳುತ್ತಾನೆ. ಪ್ರಜ್ಞೆ ಇರುವ ಪರಿಯಂತರವೂ ನಾವು ಬದ್ಧರು. ಆತ್ಮನ ದೃಷ್ಟಿಯಿಂದ ನಾವು ಮುಕ್ತಜೀವಿಗಳು. ಪುರುಷನೇ ಮಾನವನ ನಿಜವಾದ ಸ್ವಭಾವ. ಅವನು ಎಲ್ಲಾ ಕಾರ್ಯಕಾರಣ ನಿಯಮಗಳಾಚೆ ಇರುವನು. ಅವನ ಸ್ವಾತಂತ್ರ್ಯವೇ ಬುದ್ಧಿ, ಮನಸ್ಸು, ಮುಂತಾದ ನಾನಾ ತೆರೆಯ ಹಿಂದೆ ಜೀವಿಸುತ್ತಿರುವುದು. ಅವನ ಜ್ಯೋತಿಯೇ ಎಲ್ಲದರ ಹಿಂದೆ ಬೆಳಗುತ್ತಿರುವುದು. ಬುದ್ಧಿಗೆ ಸ್ವಯಂಪ್ರಭೆ ಇಲ್ಲ. ಪ್ರತಿಯೊಂದು ಇಂದ್ರಿಯಕ್ಕೂ ಕೂಡ ಮಿದುಳಿನಲ್ಲಿ ಪ್ರತ್ಯೇಕ ಕೇಂದ್ರವಿದೆ. ಎಲ್ಲಾ ಇಂದ್ರಿಯಗಳಿಗೂ ಒಂದೇ ಕೇಂದ್ರವಿಲ್ಲ. ಪ್ರತಿಯೊಂದು ಇಂದ್ರಿಯವೂ ಪ್ರತ್ಯೇಕವಾಗಿರುವುದು. ಎಲ್ಲಾ ಇಂದ್ರಿಯಗಳು ಏತಕ್ಕೆ ಒಂದುಗೂಡುತ್ತವೆ? ಅವುಗಳ ಐಕ್ಯತೆ ಎಲ್ಲಿಂದ ಬರುತ್ತದೆ? ಅದು ಮಿದುಳಿನಲ್ಲಿದ್ದರೆ ಕಣ್ಣು, ಕಿವಿ, ಮೂಗು ಮುಂತಾದ ಎಲ್ಲಾ ಇಂದ್ರಿಯಗಳಿಗೂ ಒಂದೇ ಕೇಂದ್ರವಿರಬೇಕಾಯಿತು. ಆದರೆ ಪ್ರತಿ ಇಂದ್ರಿಯಗಳಿಗೂ ಪ್ರತ್ಯೇಕ ಕೇಂದ್ರವಿದೆ ಎಂಬುದು ನಮಗೆ ಗೊತ್ತಿದೆ. ಮನುಷ್ಯನು ಏಕಕಾಲದಲ್ಲೆ ನೋಡುತ್ತಾನೆ ಮತ್ತು ಕೇಳುತ್ತಾನೆ. ಆದಕಾರಣ ಬುದ್ಧಿಯ ಹಿಂದೆ ಒಂದು ಐಕ್ಯತೆ ಇರಬೇಕು. ಬುದ್ಧಿ ಮಿದುಳಿಗೆ ಸೇರಿದೆ. ಬುದ್ಧಿಯ ಹಿಂದೆ ಏಕಮಾತ್ರ ಪುರುಷನು ಇರುವನು. ಅಲ್ಲಿ ಎಲ್ಲಾ ಸಂವೇದನೆಗಳು ಮತ್ತು ಇಂದ್ರಿಯಗ್ರಹಣಗಳು ಕಲೆತು ಒಂದಾಗುವುವು. ಆತ್ಮವೇ ನಾನಾವಿಧದ ಇಂದ್ರಿಯ ಗ್ರಹಣಗಳು ಕಲೆತು ಒಂದಾಗುವ ಕೇಂದ್ರ. ಆತ್ಮ ಸ್ವತಂತ್ರನು. ಆ ಸ್ವಾತಂತ್ರ್ಯವೇ ಪ್ರತಿ ಕ್ಷಣವೂ ನೀವು ಸ್ವತಂತ್ರರೆಂದು ಸಾರುವುದು. ಆದರೆ ನೀವು ಅದನ್ನು ತಪ್ಪು ಭಾವಿಸಿ ಪ್ರತಿಕ್ಷಣವೂ ಆ ಸ್ವಾತಂತ್ರ್ಯವನ್ನು ಬುದ್ಧಿ ಮತ್ತು ಮನಸ್ಸಿನೊಂದಿಗೆ ಬೆರೆಸುತ್ತೀರಿ. ಆ ಸ್ವಾತಂತ್ರ್ಯ ಬುದ್ಧಿಗೆ ಸೇರಿದುದೆಂದು ನೀವು ತಿಳಿಯಲು ಯತ್ನಿಸುತ್ತೀರಿ. ತತ್​ಕ್ಷಣವೇ ಬುದ್ಧಿ ಸ್ವತಂತ್ರವಲ್ಲವೆಂಬುದು ಗೊತ್ತಾಗುತ್ತದೆ. ಆ ಸ್ವಾತಂತ್ರ್ಯವನ್ನು ನೀವು ದೇಹಕ್ಕೆ ಅನ್ವಯಿಸುತ್ತೀರಿ. ತಕ್ಷಣವೇ ಪ್ರಕೃತಿ, ನೀವು ಮತ್ತೊಮ್ಮೆ ತಪ್ಪು ತಿಳಿದುಕೊಂಡಿರುವಿರಿ ಎನ್ನುವುದು. ಆದಕಾರಣವೆ ಏಕಕಾಲದಲ್ಲಿ ಮಿಶ್ರವಾದ ಸ್ವತಂತ್ರ ಮತ್ತು ಬಂಧನವೆಂಬ ಭಾವನೆ ಇರುವುದು. ಬಂಧನದಲ್ಲಿರುವುದು ಯಾವುದು, ಸ್ವಾತಂತ್ರ್ಯದಲ್ಲಿರುವುದು ಯಾವುದು, ಎಂಬ ಎರಡನ್ನು ಯೋಗಿಗಳು ವಿಭಜನೆ ಮಾಡುವರು. ಆಗ ಅವರ ಅಜ್ಞಾನ ಮಾಯವಾಗುವುದು. ಪುರುಷ ಮುಕ್ತನೆಂದು ಅವರಿಗೆ ಗೊತ್ತಾಗುವುದು. ಬುದ್ಧಿಯ ಮೂಲಕ ಪ್ರಕಾಶಿಸಿ ಬಂಧನದಲ್ಲಿ ನರಳುವ ಪ್ರಜ್ಞೆಯ ಮೂಲ ಸಾರವೇ ಪುರುಷ. 

\vspace{-0.2cm}

\begin{verse}
ತದರ್ಥ ಏವ ದೃಶ್ಯಸ್ಯಾತ್ಮಾ~॥ ೨೧~॥
\end{verse}

\vspace{-0.4cm}

\dsize{ದೃಶ್ಯದ ಸ್ವಭಾವವಿರುವುದು ಪುರುಷನಿಗಾಗಿ. }

\vspace{0.2cm}

ಪ್ರಕೃತಿಗೆ ಸ್ವಯಂಪ್ರಭೆ ಇಲ್ಲ. ಎಲ್ಲಿಯವರೆವಿಗೂ ಪುರುಷನು ಅದರಲ್ಲಿರುತ್ತಾನೆಯೋ ಅಲ್ಲಿಯವರೆವಿಗೂ ಅದು ಪ್ರಕಾಶಿಸುತ್ತದೆ. ಪ್ರತಿಬಿಂಬಿಸುವ ಚಂದ್ರನ ಬೆಳಕಿನಂತೆ ಇದು ಎರವಲಾಗಿ ಪಡೆದ ಪ್ರಕಾಶ. ಯೋಗಿಯ ಅಭಿಪ್ರಾಯದ ಪ್ರಕಾರ ಪ್ರಕೃತಿಯಲ್ಲಿರುವ ಅಭಿವ್ಯಕ್ತಿಯೆಲ್ಲ ಪ್ರಕೃತಿಯಿಂದಲೇ ಆದುವು. ಆದರೆ ಪುರುಷನನ್ನು ಮುಕ್ತನನ್ನಾಗಿ ಮಾಡುವುದಲ್ಲದೆ ಬೇರೆ ಗುರಿ ಪ್ರಕೃತಿಗೆ ಇಲ್ಲ. 

\vspace{-0.2cm}

\begin{verse}
ಕೃತಾರ್ಥಂ ಪ್ರತಿ ನಷ್ಟಮಪ್ಯನಷ್ಟಂ ತದನ್ಯಸಾಧಾರಣತ್ವಾತ್​~॥ ೨೨~॥
\end{verse}

\vspace{-0.45cm}

\dsize{ಗುರಿ ಸೇರಿದವರಿಗೆ ಇದು ನಾಶವಾದರೂ ಉಳಿದವರಿಗೆ ಇದು ಸಾಧಾರಣವಾಗಿರುವುದರಿಂದ ನಾಶವಾಗಿಲ್ಲ. }

\vspace{0.2cm}

ಪ್ರಕೃತಿಯಿಂದ ಪುರುಷನು ಬೇರೆ ಎಂಬುದನ್ನು ಅವನು (ಪುರುಷನು) ತಿಳಿದು ಕೊಳ್ಳುವಂತೆ ಮಾಡುವುದೇ ಪ್ರಕೃತಿಯ ಒಟ್ಟು ಕಾರ್ಯಕ್ರಮದ ಗುರಿ. ಪುರುಷನಿಗೆ ಇದು ತಿಳಿದರೆ ಪ್ರಕೃತಿಯ ಮೋಹಕ್ಕೆ ಒಳಗಾಗನು. ಯಾರು ಮುಕ್ತರಾಗಿರುವರೋ ಅವರಿಗೆ ಮಾತ್ರ ಪ್ರಕೃತಿ ಮಾಯವಾಗುತ್ತದೆ. ಯಾವಾಗಲೂ ಅಸಂಖ್ಯಾತ ಮಂದಿ ಉಳಿದೇ ಇರುವರು. ಅವರಿಗೆ ಪ್ರಕೃತಿ ಕೆಲಸ ಮಾಡುತ್ತದೆ. 

\vspace{-0.2cm}

\begin{verse}
ಸ್ವಸ್ಯಾಮಿಶಕ್ತ್ಯೋಃಸ್ವರೂಪೋಪಲಬ್ಧಿಹೇತುಃ ಸಂಯೋಗಃ~॥ ೨೩~॥
\end{verse}

\vspace{-0.45cm}

\dsize{ಅನುಭೂತ ವಸ್ತು ಮತ್ತು ಅದರ ಸ್ವಾಮಿ–ಇವರ ಸ್ವರೂಪದ ಸಾಕ್ಷಾತ್ಕಾರಕ್ಕೆ ಕಾರಣವಾಗುವುದೇ ಸಂಯೋಗ. }

\vspace{0.2cm}

ಈ ಸೂತ್ರದ ಪ್ರಕಾರ ಪುರುಷ ಮತ್ತು ಪ್ರಕೃತಿಗಳೆರಡೂ ಸಂಧಿಸಿದಾಗ ಅವುಗಳ ಶಕ್ತಿ ವ್ಯಕ್ತವಾಗುತ್ತದೆ. ಆಗ ಎಲ್ಲಾ ತೋರಿಕೆಗಳೂ ತ್ಯಜಿಸಲ್ಪಡುತ್ತವೆ. ಅಜ್ಞಾನವೇ ಈ ಸಂಯೋಗಕ್ಕೆ ಕಾರಣ. ನಮ್ಮ ಸುಖದುಃಖಗಳಿಗೆ ಕಾರಣ, ನಾವು ದೇಹದೊಂದಿಗೆ ಸೇರಿರುವುದೇ ಎಂಬುದನ್ನು ಹಗಲಿರುಳು ನೋಡುತ್ತೇವೆ. ದೇಹ ನಾನಲ್ಲ ಎಂದು ಸಂಪೂರ್ಣವಾಗಿ ತಿಳಿದಿದ್ದರೆ, ಶೀತ, ಉಷ್ಣ ಮತ್ತು ಇನ್ನು ಇತರ ಯಾವ ಅನುಭವವನ್ನೂ ನಾನು ಲೆಕ್ಕಿಸುವುದಿಲ್ಲ. ಈ ದೇಹ ಒಂದು ಸಂಯೋಗ. ನನಗೆ ಒಂದು ದೇಹವಿದೆ; ನಿಮಗೆ ಒಂದು ದೇಹವಿದೆ ಮತ್ತು ಸೂರ್ಯನ ದೇಹ ನನಗಿಂತ ಬೇರೆ ಎಂದು ಹೇಳುವುದು ಭ್ರಾಂತಿ. ಈ ವಿಶ್ವವೇ ವಸ್ತುವಿನ ಮಹಾಸಾಗರ. ಅದರಲ್ಲಿ ನೀವು ಒಂದು ಸಣ್ಣ ಕಣದ ಹೆಸರು, ನಾನು ಇನ್ನೊಂದು ಕಣ, ಮತ್ತು ಸೂರ್ಯನು ಮತ್ತೊಂದು ಕಣ. ಈ ದ್ರವ್ಯವಸ್ತು ಅನವರತವೂ ಬದಲಾಯಿಸು\break ತ್ತಿರುತ್ತದೆ ಎಂಬುದು ನಮಗೆ ಗೊತ್ತಿದೆ. ಒಂದು ದಿನ ಯಾವುದು ಸೂರ್ಯನಲ್ಲಿದ್ದಿತೋ, ಅದು ಮಾರನೆ ದಿನ ನಿಮ್ಮ ದೇಹದಲ್ಲಿರಬಹುದು. 

\vspace{-0.2cm}

\begin{verse}
ತಸ್ಯ ಹೇತುರವಿದ್ಯಾ~॥ ೨೪~॥
\end{verse}

\vspace{-0.4cm}

\dsize{ಅದಕ್ಕೆ ಕಾರಣ ಅವಿದ್ಯೆ. }

\vspace{0.2cm}

ಅಜ್ಞಾನದಿಂದ ನಾವು ಒಂದು ದೇಹಕ್ಕೆ ಸೇರಿಕೊಂಡಿರುವೆವು. ಇದರಿಂದ ನಾವು ದುಃಖಕ್ಕೆ ಈಡಾಗಿರುವೆವು. ನಮ್ಮನ್ನು ಸುಖಿಗಳನ್ನಾಗಿ ಅಥವಾ ದುಃಖಿಗಳನ್ನಾಗಿ ಮಾಡುವುದು ಮೂಢನಂಬಿಕೆ. ಅಜ್ಞಾನದಿಂದ ಹುಟ್ಟಿದ ಮೂಢನಂಬಿಕೆ ನಮ್ಮಲ್ಲಿ ಶೀತ, ಉಷ್ಣ, ಸುಖ, ದುಃಖಗಳೆಂಬ ಭಾವನೆಯನ್ನು ಪಡೆಯುವಂತೆ ಮಾಡುವುದು. ಈ ಮೂಢನಂಬಿಕೆಗಳನ್ನು ಮೀರಿಹೋಗುವುದು ನಮ್ಮ ಕರ್ತವ್ಯ. ಯೋಗಿ ಇದನ್ನು ಹೇಗೆ ಸಾಧಿಸುವುದು ಎಂಬುದನ್ನು ತೋರುತ್ತಾನೆ. ಕೆಲವು ಮಾನಸಿಕ ಅವಸ್ಥೆಯಲ್ಲಿದ್ದಾಗ ಒಬ್ಬ ಮನುಷ್ಯನನ್ನು ಸುಟ್ಟರೂ ಅವನಿಗೆ ನೋವು ತೋರುವುದಿಲ್ಲವೆಂಬುದನ್ನು ಪ್ರದರ್ಶಿಸಿರುವರು. ತೊಂದರೆ ಏನು ಎಂದರೆ ಇಂತಹ ಮಾನಸಿಕ ಸ್ಥಿತಿ ಒಮ್ಮೆ ಸುಂಟರಗಾಳಿಯಂತೆ ಬಂದು ಮರುಕ್ಷಣವೇ ಹೊರಟು ಹೋಗುವುದು. ಇದನ್ನು ಯೋಗದ ಮೂಲಕ ಪಡೆದರೆ, ದೇಹದಿಂದ ಆತ್ಮನ ಭಿನ್ನತೆಯನ್ನು ನಾವು ಶಾಶ್ವತವಾಗಿ ಪಡೆಯಬಹುದು. 

\vspace{-0.2cm}

\begin{verse}
ತದಭಾವಾತ್​ ಸಂಯೋಗಾಭಾವೋ ಹಾನಂ ತದ್ದೃಶೇಃ ಕೈವಲ್ಯಮ್​~॥ ೨೫~॥
\end{verse}

\vspace{-0.45cm}


\dsize{ಅಜ್ಞಾನವಿಲ್ಲದಿದ್ದರೆ ಸಂಯೋಗವೂ ಇರುವುದಿಲ್ಲ. ಇದರಿಂದಲೇ ನಾವು ತಪ್ಪಿಸಿಕೊಳ್ಳಬೇಕಾಗಿರುವುದು. ಅದೇ ದ್ರಷ್ಟೃವಿನ ಕೈವಲ್ಯಪದವಿ. }

\vskip 0.2cm

ಯೋಗಸಿದ್ಧಾಂತದ ಪ್ರಕಾರ, ಅಜ್ಞಾನವಶನಾಗಿ ಪುರುಷನು ಪ್ರಕೃತಿಯೊಂದಿಗೆ ಸೇರಿರುವನು. ಪ್ರಕೃತಿಯ ಸ್ವಾಧೀನದಿಂದ ತಪ್ಪಿಸಿಕೊಳ್ಳುವುದೇ ನಮ್ಮ ಗುರಿ. ಎಲ್ಲಾ ಧರ್ಮಗಳ ಗುರಿಯೂ ಇದೇ. ಪ್ರತಿಯೊಂದು ಆತ್ಮವೂ ಕೂಡ ಸ್ವಭಾವತಃ ದಿವ್ಯವಾದುದು. ಅಂತರಾಳದಲ್ಲಿರುವ ಈ ದಿವ್ಯತೆಯನ್ನು ಬಾಹ್ಯ ಮತ್ತು ಆಂತರಿಕ ಪ್ರಕೃತಿಗಳ ನಿಗ್ರಹದಿಂದ ವ್ಯಕ್ತ ಮಾಡುವುದೇ ಜೀವನದ ಗುರಿ. ಇದನ್ನು ಕರ್ಮ– ಭಕ್ತಿ–ಜ್ಞಾನ–ಯೋಗ ಇವುಗಳಲ್ಲಿ ಯಾವುದಾದರೂ ಒಂದರ ಮೂಲಕವಾಗಲೀ, ಅಥವಾ ಎಲ್ಲಾ ಮಾರ್ಗಗಳ ಮೂಲಕವಾಗಲೀ, ಸಾಧಿಸಿ ಮುಕ್ತರಾಗಿ, ಇದೇ ಧರ್ಮದ ಸರ್ವಸ್ವ. ಸಿದ್ಧಾಂತ, ಮತ, ಪೂಜೆ, ದೇವಸ್ಥಾನ, ವಿಗ್ರಹ–ಇವು ಅಷ್ಟೇನು ಪ್ರಾಮುಖ್ಯ ವಲ್ಲದ ವಿವರಗಳು. ಯೋಗಿಯು ಮನೋನಿಗ್ರಹದ ಮೂಲಕ ಈ ಗುರಿಯನ್ನು ಸೇರಲು ಯತ್ನಿಸುತ್ತಾನೆ. ಪ್ರಕೃತಿಯಿಂದ ಮುಕ್ತರಾಗುವವರೆವಿಗೂ ನಾವು ಗುಲಾಮರು. ಅದು ಹೇಳಿದಂತೆ ಕೇಳಬೇಕು. ಯಾರು ಮನಸ್ಸನ್ನು ಜಯಿಸುತ್ತಾರೆಯೊ ಅವರು ವಸ್ತುವನ್ನು ಜಯಿಸುತ್ತಾರೆ ಎಂದು ಯೋಗಿ ಹೇಳುತ್ತಾನೆ. ಆಂತರಿಕ ಪ್ರಕೃತಿ ಬಾಹ್ಯ ಪ್ರಕೃತಿಗಿಂತ ಉತ್ತಮವಾದುದು ಮತ್ತು ಅದನ್ನು ನಿಗ್ರಹಿಸುವುದು ಬಹಳ ಕಷ್ಟದ ಕೆಲಸ. ಆದಕಾರಣ ಯಾರು ತಮ್ಮ ಮನಸ್ಸನ್ನು ನಿಗ್ರಹಿಸುತ್ತಾರೆಯೋ ಅವರು ಜಗತ್ತನ್ನೇ ನಿಗ್ರಹಿಸುತ್ತಾರೆ. ಇದು ಅವನ ಗುಲಾಮನಾಗುತ್ತನೆ. ಈ ಸ್ವಾಧೀನತೆಯನ್ನು ಪಡೆಯುವ ಮಾರ್ಗವನ್ನು ರಾಜಯೋಗ ವಿವರಿಸುತ್ತದೆ. ನಮಗೆ ತೋರುವುದಕ್ಕಿಂತ ಉತ್ತಮತರದ ಪ್ರಕೃತಿಯ ಶಕ್ತಿಗಳನ್ನು ನಾವು ಅಡಗಿಸಬೇಕಾಗಿದೆ. ಈ ದೇಹವು ಮನಸ್ಸಿನ ಮೇಲಿರುವ ಹೊದಿಕೆ ಅಷ್ಟೆ. ಇವೆರಡೂ ಬೇರೆ ಅಲ್ಲ. ಇದು ಕಪ್ಪೆ ಮತ್ತು ಅದರ ಚಿಪ್ಪಿನಂತೆ.\break ಇವುಗಳು ಒಂದೇ ವಸ್ತುವಿನ ಎರಡು ಭಾಗಗಳು. ಕಪ್ಪೆಚಿಪ್ಪಿನಲ್ಲಿ ಒಳಗಿರುವ ಸತ್ತ್ವವು ಹೊರಗಿನಿಂದ ವಸ್ತುವನ್ನು ಸೇರಿಸಿ ಚಿಪ್ಪನ್ನು ತಯಾರು ಮಾಡುತ್ತದೆ. ಇದೇ ರೀತಿ ಮನಸ್ಸೆಂದು ಕರೆಯಲ್ಪಡುವ ಆಂತರಿಕ ಸೂಕ್ಷ್ಮಶಕ್ತಿಗಳು ಹೊರಗಿನಿಂದ ಸ್ಥೂಲವಸ್ತುಗಳನ್ನು ತೆಗೆದುಕೊಂಡು ದೇಹವೆಂಬ ಬಾಹ್ಯ ಚಿಪ್ಪನ್ನು ತಯಾರಿಸುತ್ತವೆ. ಆದಕಾರಣ ಅಂತರಾಳದಲ್ಲಿರುವ ವಸ್ತುವಿನ ಸ್ವಾಧೀನತೆ ಇದ್ದರೆ ಬಾಹ್ಯ ವಸ್ತುವಿನ ಮೇಲೆ ಸ್ವಾಧೀನತೆಯನ್ನು ಪಡೆಯುವುದು ಸುಲಭ. ಅದೂ ಅಲ್ಲದೆ ಈ ಶಕ್ತಿಗಳೇನೂ ಬೇರೆಯಲ್ಲ. ಕೆಲವು ಶಾರೀರಿಕ ಶಕ್ತಿ ಮತ್ತೆ ಕೆಲವು ಮಾನಸಿಕ ಶಕ್ತಿ ಎಂದು ಎರಡು ವಿಧಗಳಿಲ್ಲ. ಬಾಹ್ಯ ಪ್ರಪಂಚ ಸೂಕ್ಷ್ಮಪ್ರಪಂಚದ ಸ್ಥೂಲರೂಪವಾಗಿರುವಂತೆ ದೈಹಿಕ ಶಕ್ತಿಗಳು ಸೂಕ್ಷ್ಮಶಕ್ತಿಯ ಸ್ಥೂಲ ರೂಪಗಳಷ್ಟೆ. 

\vspace{-0.2cm}

\begin{verse}
ವಿವೇಕಖ್ಯಾತಿರವಿಪ್ಲವಾ ಹಾನೋಪಾಯಃ~॥ ೨೬~॥
\end{verse}

\vspace{-0.45cm}

\dsize{ನಿರಂತರವಾದ ವಿವೇಕದ ಅಭ್ಯಾಸವೇ ಅಜ್ಞಾನದ ನಾಶಕ್ಕೆ ಸಾಧನೆ. }

\vspace{0.2cm}

ಇದೇ ನಿಜವಾದ ಸಾಧನೆಯ ಗುರಿ – ನಿತ್ಯಾನಿತ್ಯ ವಸ್ತು ವಿವೇಕ, ಪುರುಷನು ಪ್ರಕೃತಿಯಲ್ಲವೆಂದು ತಿಳಿಯುವುದು. ಪುರುಷನು ವಸ್ತುವಲ್ಲ, ಮನಸ್ಸೂ ಅಲ್ಲ. ಅವನು ಪ್ರಕೃತಿ ಅಲ್ಲದ ಕಾರಣ ವಿಕಾರವಾಗುವ ಸಂಭವವಿಲ್ಲ. ವಿಕಾರವಾಗುವುದೇ ಪ್ರಕೃತಿಯ ಸ್ವಭಾವ. ನಿರಂತರ ಅಭ್ಯಾಸ ಬಲದಿಂದ ನಾವು ವಿಮರ್ಶಿಸಿದರೆ, ಅಜ್ಞಾನವು ನಾಶವಾಗಿ, ಪುರುಷನು ಸರ್ವವ್ಯಾಪಿಯಾಗಿ, ಸರ್ವಶಕ್ತನಾಗಿ, ಸ್ವಯಂ ಪ್ರಕಾಶದಿಂದ ಬೆಳಗುತ್ತಾನೆ. 

\vspace{-0.2cm}

\begin{verse}
ತಸ್ಯ ಸಪ್ತಧಾ ಪ್ರಾನ್ತಭೂಮಿಃ ಪ್ರಜ್ಞಾ~॥ ೨೭~॥
\end{verse}

\vspace{-0.45cm}

\dsize{ಅವನ ಜ್ಞಾನ ಏಳು ವಿಧದ ಅಂತಸ್ತಿನ ಮೇಲಿರುವುದು. }

\vspace{0.2cm}

ಈ ಜ್ಞಾನ ಬಂದಾಗ ಏಳು ಅಂತಸ್ತಿನಿಂದ ಕೂಡಿರುವಂತೆ ತೋರುವುದು. ಇವು ಒಂದಾದ ಮೇಲೊಂದು ಬರುವುವು. ಇವುಗಳಲ್ಲಿ ಯಾವುದಾದರೊಂದು ಮೊದಲಾದರೆ ನಮಗೆ ಜ್ಞಾನ ಬರುತ್ತಿದೆ ಎಂಬುದು ಗೊತ್ತಾಗುವುದು. ಮೊದಲನೆಯದೆ, ಯಾವುದು ನಮಗೆ ಗೊತ್ತಾಗಬೇಕೊ ಅದು ಗೊತ್ತಿದೆ ಎಂಬುದು. ಮನಸ್ಸಿನ ಅತೃಪ್ತಿ ನಿಲ್ಲುವುದು. ಜ್ಞಾನದಾಹ ನಮ್ಮಲ್ಲಿರುವಾಗ ಎಲ್ಲಿ ನಮಗೆ ಸತ್ಯ ತೋರುತ್ತದೆ ಎನಿಸುವುದೊ ಅದನ್ನು ಅಲ್ಲೆಲ್ಲ ಹುಡುಕಲು ಮೊದಲು ಮಾಡುವೆವು. ಜ್ಞಾನ ನಮ್ಮಲ್ಲಿಯೇ ಇರುವುದು. ಮತ್ತಾರೂ ನಮಗೆ ಸಹಾಯ ಮಾಡಲಾರರು, ನಮಗೆ ನಾವೇ ಸಹಾಯ ಮಾಡಿಕೊಳ್ಳಬೇಕು–ಎನ್ನುವುದನ್ನು ತಿಳಿದುಕೊಳ್ಳುವುದಕ್ಕೆ ಮೊದಲು ನಮ್ಮ ಸಂಶೋಧನೆಯೆಲ್ಲ ವ್ಯರ್ಥ. ನಿತ್ಯ ಅನಿತ್ಯ ವಸ್ತುವನ್ನು ನಾವು ಅಭ್ಯಾಸ ಮಾಡಿದ ಮೇಲೆ, ನಾವು ಸತ್ಯದೆಡೆಗೆ ಸಮೀಪಿಸುತ್ತಿರುವೆವು ಎಂಬುದಕ್ಕೆ ಮೊದಲನೆಯ ಗುರುತೇ ನಮ್ಮಲ್ಲಿರುವ ಅತೃಪ್ತಿಯ ಸ್ಥಿತಿ ಕೊನೆಗಾಣುವುದು. ಸತ್ಯ ನಮಗೆ ಸಿಕ್ಕಿದೆ. ನಮ್ಮಲ್ಲಿರುವುದು ಸತ್ಯವಲ್ಲದೆ ಬೇರೆ ಏನೂ ಅಲ್ಲ ಎಂಬುದು ನಿಸ್ಸಂದೇಹವಾಗುವುದು. ಆಗ (ಜ್ಞಾನ) ಸೂರ್ಯನು ಏಳುತ್ತಿರುವನು, ಬೆಳಕು ಹರಿಯುತ್ತಿದೆ ಎಂಬುದನ್ನು ತಿಳಿಯಬಹುದು. ಧೈರ್ಯ ತಾಳಿ ಗುರಿ ಮುಟ್ಟುವವರೆಗೂ ಪ್ರಯತ್ನ ಮಾಡಬೇಕು. ಎರಡನೆಯ ಹಂತವೆ ಎಲ್ಲಾ ವಿಧದ ವ್ಯಥೆಯ ಅಭಾವ. ಪ್ರಪಂಚದಲ್ಲಿ ಹೊರಗಿನದಾಗಲೀ, ಒಳಗಿನದಾಗಲೀ ಯಾವುದೂ ನಮಗೆ ವ್ಯಥೆಯನ್ನು ಕೊಡಲಾರದು. ಮೂರನೆಯದೆ ಪೂರ್ಣಜ್ಞಾನಸಿದ್ಧಿ. ಸರ್ವಜ್ಞತೆ ನಮ್ಮದಾಗುವುದು. ನಾಲ್ಕನೆಯದೆ ನಿತ್ಯಾನಿತ್ಯ ವಸ್ತುವಿವೇಕದ ಮೂಲಕ ಎಲ್ಲಾ ಕರ್ತವ್ಯಗಳ ಸಮಾಪ್ತಿಯನ್ನು ಹೊಂದುವುದು. ಅನಂತರ ಚಿತ್ತ ಸ್ವಾತಂತ್ರ್ಯ ಬರುವುದು. ಮನಸ್ಸಿನ ಎಲ್ಲಾ ಕಷ್ಟ, ಪ್ರಯತ್ನ, ಚಂಚಲತೆ–ಹೇಗೆ ಪರ್ವತದ ಮೇಲಿನಿಂದ ಕಣಿವೆಗೆ ಬಿದ್ದ ಕಲ್ಲು ಪುನಃ ಮೇಲೆ ಏರುವುದಿಲ್ಲವೋ ಹಾಗೆ–ತಟಸ್ಥವಾಗುವುವು. ಅನಂತರದ ಸ್ಥಿತಿಯಲ್ಲಿ ನಾವು ಇಚ್ಚಿಸಿದಾಗ ಚಿತ್ತವೇ ತನ್ನ ಕಾರಣ ಸ್ಥಿತಿಗೆ ಕರಗಿ ಹೋಗುವುದು. ಅದಕ್ಕೇ ವೇದ್ಯವಾಗುತ್ತದೆ. ಕೊನೆಯ ಹಂತದಲ್ಲಿ ನಮ್ಮ ಆತ್ಮನಲ್ಲಿ ನಾವು ಪ್ರತಿಷ್ಠಿತರಾಗಿರುವೆವೆಂದು ನಮಗೆ ತಿಳಿಯುವುದು. ಜಗತ್ತಿನಲ್ಲೆಲ್ಲ ನಾವು ಇದ್ದುದು ಏಕಾಕಿಯಾಗಿಯೇ. ನಮಗೂ ದೇಹ ಮನಸ್ಸಿಗೂ ಯಾವ ಸಂಬಂಧವೂ ಇರಲಿಲ್ಲ. ಅವು ನಮಗೆ ಸೇರಿಯೂ ಇರಲಿಲ್ಲ. ಅವು ತಮ್ಮ ಕೆಲಸವನ್ನು ತಾವು ಮಾಡುತ್ತಿದ್ದವು. ನಾವು ಅಜ್ಞಾನದಿಂದ ಅವುಗಳೊಂದಿಗೆ ಸೇರಿದೆವು. ಆದರೆ ನಾವು ಸರ್ವಶಕ್ತರಾಗಿ, ಸರ್ವವ್ಯಾಪಿಯಾಗಿ ನಿತ್ಯಾನಂದದಲ್ಲಿ ಏಕಾಕಿಗಳಾಗಿದ್ದೆವು. ನಮ್ಮ ಸ್ವಂತ ಆತ್ಮವೇ ಪರಿಶುದ್ಧವೂ ಪರಿಪಕ್ವವೂ ಆಗಿತ್ತು. ನಮಗೆ ಮತ್ತಾವುದೂ ಬೇಕಾಗಿರಲಿಲ್ಲ, ಏಕೆಂದರೆ ನಾವೆ ಸೌಖ್ಯ. ಈ ಜ್ಞಾನ ಮತ್ತಾವುದರ ಆಸರೆಯ ಮೇಲೂ ನಿಂತಿಲ್ಲ. ನಮ್ಮ ಜ್ಞಾನದ ಮುಂದೆ ಪ್ರಕಾಶವಾಗದೆ ಇರುವುದು ಜಗತ್ತಿನಲ್ಲಿ ಯಾವುದೂ ಇರುವುದಿಲ್ಲ ಎನ್ನುವುದು ನಮಗೆ ತಿಳಿದುಬರುತ್ತದೆ. ಇದೇ ಕೊನೆಯ ಸ್ಥಿತಿ. ಯೋಗಿ ಆಗ ಶಾಂತನಾಗುತ್ತಾನೆ, ಮೌನಿಯಾಗುತ್ತಾನೆ. ಮತ್ತಾವ ವ್ಯಥೆಯನ್ನು ಎಂದಿಗೂ ಪಡೆಯುವುದಿಲ್ಲ, ಎಂದಿಗೂ ಮೋಹಕ್ಕೆ ಒಳಗಾಗುವುದಿಲ್ಲ, ದುಃಖದ ಪಾಶಕ್ಕೆ ಬೀಳುವುದಿಲ್ಲ. ತಾನು ನಿತ್ಯಾನಂದಮಯ, ಪರಿಪೂರ್ಣ, ಸರ್ವಶಕ್ತ, ಎಂಬುದನ್ನು ತಿಳಿಯುತ್ತಾನೆ. 

\eject

\begin{verse}
ಯೋಗಾಂಗಾನುಷ್ಠಾನಾದಶುದ್ಧಿಕ್ಷಯೇ ಜ್ಞಾನದೀಪ್ತಿರಾವಿವೇಕಖ್ಯಾತೇಃ~॥ ೨೮~॥
\end{verse}

\vspace{-0.45cm}

\dsize{ಹಲವು ಯೋಗಾಂಗಗಳ ಅಭ್ಯಾಸ ಬಲದಿಂದ ಮಲಿನತೆ ನಾಶವಾಗಿ, ವಿವೇಕದವರೆವಿಗೂ ಜ್ಞಾನವು ಪ್ರಕಾಶಿಸುವುದು. }

\vspace{0.1cm}

ಈಗ ಅನುಷ್ಠಾನಕ್ಕೆ ಸಂಬಂಧಪಟ್ಟ ಜ್ಞಾನ ಬರುತ್ತದೆ. ನಾವು ಈಗ ತಾನೆ ಮಾತನಾಡುತ್ತಿದ್ದುದು ಬಹಳ ಮೇಲಿನ ವರ್ಗಕ್ಕೆ ಸೇರಿದುದು, ನಮ್ಮನ್ನು ಮೀರಿದುದು, ಬಹಳ ದೂರದಲ್ಲಿರುವುದು. ಆದರೆ ಇದು ನಮ್ಮ ಗುರಿ. ಶಾರೀರಿಕ ಮತ್ತು ಮಾನಸಿಕ ಸ್ವಾಧೀನತೆಯನ್ನು ಪಡೆಯುವುದು ಮೊದಲು ಆವಶ್ಯಕ. ಅನಂತರ ಆ ಆದರ್ಶದಲ್ಲಿ ನಮ್ಮ ಅನುಭವ ಸ್ಥಿರ\break ವಾಗುವುದು. ಗುರಿ ಗೊತ್ತಾದ ಮೇಲೆ ಉಳಿಯುವುದೇ ಅದನ್ನು ಸೇರುವ ರೀತಿಯನ್ನು ಅಭ್ಯಾಸ ಮಾಡುವುದು. 

\vspace{-0.3cm}

\begin{verse}
ಯಮ–ನಿಯಮಾಸನ–ಪ್ರಾಣಾಯಮ–ಪ್ರತ್ಯಾಹಾರ–\\ಧಾರಣಾ–ಧ್ಯಾನ–ಸಮಾಧಯೋಽಷ್ಟಾವಂಗಾನಿ~॥ ೨೯~॥
\end{verse}

\vspace{-0.4cm}

\dsize{ಯಮ, ನಿಯಮ, ಆಸನ, ಪ್ರಾಣಾಯಾಮ, ಪ್ರತ್ಯಾಹಾರ, ಧಾರಣ, ಧ್ಯಾನ ಮತ್ತು ಸಮಾಧಿಗಳೆಂಬುದು ಯೋಗದ ಅಷ್ಟಾಂಗಗಳು.}

\vspace{-0.2cm}

\begin{verse}
ಅಹಿಂಸಾ–ಸತ್ಯಾಸ್ತೇಯ–ಬ್ರಹ್ಮಚರ್ಯಾಪರಿಗ್ರಹಾ ಯಮಾಃ~॥ ೩೦~॥
\end{verse}

\vspace{-0.4cm}

\dsize{ಅಹಿಂಸೆ, ಸತ್ಯ, ಅಸ್ತೇಯ, ಬ್ರಹ್ಮಚರ್ಯ, ಅಪರಿಗ್ರಹ–ಇವು ಯಮ.}

\vspace{0.1cm}

ಯಾರು ನಿಜವಾಗಿ ಯೋಗಿಯಾಗಬೇಕೆಂದು ಬಯಸುವರೋ ಅವರು ಕಾಮವನ್ನು ತೊರೆಯಬೇಕು. ಆತ್ಮನಿಗೆ ಲಿಂಗವಿಲ್ಲ, ಕಾಮಭಾವನೆಯ ಹೀನ ದೃಷ್ಟಿಗೇಕೆ ಅದು ಬರಬೇಕು. ಈ ಭಾವನೆಗಳನ್ನು ಏತಕ್ಕೆ ತೊರೆಯಬೇಕೆಂಬುದು ನಮಗೆ ಅನಂತರ ಗೊತ್ತಾಗುವುದು. ಯಾರು ದಾನವನ್ನು ಸ್ವೀಕರಿಸುತ್ತಾರೆಯೊ ಅವರ ಮನಸ್ಸು ಕೊಡುವವರ ಅಧೀನಕ್ಕೆ ಒಳಗಾಗುತ್ತದೆ. ಆದಕಾರಣ ಸ್ವೀಕರಿಸುವವನು ಹೀನಸ್ಥಿತಿಗೆ ಇಳಿಯಬಹುದು. ದಾನವನ್ನು ಸ್ವೀಕರಿಸುವುದರಿಂದ ನಮ್ಮ ಮನಸ್ಸಿನ ಸ್ವಾತಂತ್ರ್ಯವನ್ನು ಕಳೆದುಕೊಂಡು ಅವರ ಆಳುಗಳಾಗುವ ಸಂಭವವಿದೆ. ಆದಕಾರಣ ಯಾರೂ ಯಾವ ದಾನವನ್ನೂ ಸ್ವೀಕರಿಸಬೇಡಿ. 

\vspace{-0.2cm}

\begin{verse}
ಏತೇ ಜಾತಿ–ದೇಶ–ಕಾಲ–ಸಮಾಯಾನವಚ್ಛಿನ್ನಾಃ ಸಾರ್ವಭೌಮಾ \\ಮಹಾವ್ರತಮ್​~॥~೩೧~॥
\end{verse}

\vspace{-0.4cm}

\dsize{ಇವು ದೇಶ, ಕಾಲ, ಸಮಯ, ಜಾತಿ, ನಿಯಮಗಳಿಂದ ದೂರವಾದ ಮಹಾ ವ್ರತಗಳು. }

\vspace{0.1cm}

ಅಹಿಂಸಾ, ಸತ್ಯ, ಅಸ್ತೇಯ, ಬ್ರಹ್ಮಚರ್ಯ, ಅಪರಿಗ್ರಹ ಇವುಗಳನ್ನು ಪ್ರತಿಯೊಬ್ಬ ಜೀವರೂ ಕೂಡ–ಅವರು ಯಾವ ದೇಶ ಅಥವಾ ಅಂತಸ್ತಿನಲ್ಲಿರಲಿ–ಎಲ್ಲಾ ಸ್ತ್ರೀಪುರುಷರು ಮತ್ತು ಹುಡುಗರು ಅಭ್ಯಾಸ ಮಾಡಬೇಕು. 

\vspace{-0.2cm}

\begin{verse}
ಶೌಚ–ಸಂತೋಷ–ತಪಃಸ್ವಾಧ್ಯಾಯೇಶ್ವರಪ್ರಣಿಧಾನಾನಿ ನಿಯಮಾಃ~॥ ೩೨~॥
\end{verse}

\vspace{-0.4cm}

\dsize{ಬಾಹ್ಯ ಮತ್ತು ಅಂತರಂಗ ನೈರ್ಮಲ್ಯ, ತೃಪ್ತಿ, ತಪಸ್ಸು, ಸ್ವಾಧ್ಯಾಯ, ಈಶ್ವರೋಪಾಸನೆ ಇವು ನಿಯಮ. }


\vfill\eject

ದೇಹವನ್ನು ಶುಚಿಯಾಗಿಟ್ಟಿರುವುದೇ ಬಾಹ್ಯ ಶೌಚ. ಕೊಳಕಾದವನು ಎಂದಿಗೂ ಯೋಗಿಯಾಗಲಾರ. ಆಂತರಿಕ ಶುಚಿಯೂ ಇರಬೇಕು. ಇದೂ ಸೂತ್ರ ೧, ೩೩ರಲ್ಲಿ ಹೇಳಿದ ಗುಣಗಳಿಂದ ಬರುವುದು. ಆಂತರಿಕ ಶುಚಿ ಬಾಹ್ಯ ಶುಚಿಗಿಂತ ಹೆಚ್ಚು ಮುಖ್ಯವಾದುದು. ಆದರೆ ಎರಡೂ ಆವಶ್ಯಕ. ಅಂತರಂಗ ಶುಚಿಯಿಲ್ಲದೆ ಬಾಹ್ಯ ಶುಚಿಯಿಂದ ಏನೂ ಪ್ರಯೋಜನವಿಲ್ಲ. 

\vspace{-0.2cm}

\begin{verse}
ವಿತರ್ಕಬಾಧನೇ ಪ್ರತಿಪಕ್ಷಭಾವನಮ್​~॥ ೩೩~॥
\end{verse}

\vspace{-0.6cm}

\dsize{ಯೋಗಕ್ಕೆ ಭಂಗವನ್ನು ತರುವ ಆಲೋಚನೆಗಳನ್ನು ತಡೆಯುವುದಕ್ಕಾಗಿ, ಅವಕ್ಕೆ ವಿರೋಧವಾಗಿರುವ ಭಾವನೆಗಳನ್ನು ತರಬೇಕು. }

\vspace{0.1cm}

ಈಗ ಹೇಳಿದ ಗುಣಗಳನ್ನು ಅಭ್ಯಾಸ ಮಾಡುವ ರೀತಿಯೇ ಇದು. ನಾವು ದೊಡ್ಡ ಅಲೆಯಿಂದ ಮೊದಲು ಮಾಡಿ ಕ್ರಮೇಣ ಸಣ್ಣ ಸಣ್ಣ ಅಲೆಯ ಕಡೆ ಬರಬೇಕು. ಉದಾಹರಣೆಗೆ ಮನಸ್ಸಿನಲ್ಲಿ ಕೋಪದ ಒಂದು ದೊಡ್ಡ ಅಲೆ ಇದೆ ಎಂದು ಇಟ್ಟುಕೊಳ್ಳೋಣ. ಅದನ್ನು ನಾವು ಹೇಗೆ ನಿಗ್ರಹಿಸುವುದು? ಅದನ್ನು ವಿರೋಧಿಸುವ ಮತ್ತೊಂದು ದೊಡ್ಡ ಅಲೆಯನ್ನು ಎಬ್ಬಿಸುವುದರಿಂದ. ಪ್ರೀತಿಯನ್ನು ಆಲೋಚಿಸಿ. ಕೆಲವು ವೇಳೆ ತಾಯಿಗೆ ತನ್ನ ಪತಿಯ ಮೇಲೆ ತುಂಬಾ ಕೋಪವಾಗಿದೆ. ಆ ಸಮಯದಲ್ಲಿ ಮಗು ಬರುವುದು. ಅದನ್ನು ಎತ್ತಿ ಮುದ್ದಿಸುವಳು. ಆಗ ಕೋಪದ ಹಳೆಯ ಅಲೆ ಮಾಯವಾಗಿ ಪ್ರೇಮದ ಹೊಸ ಅಲೆ ಏಳುವುದು. ಪ್ರೀತಿ ದ್ವೇಷಕ್ಕೆ ವಿರೋಧ. ಆದಕಾರಣ ನಾವು ಯಾವುದನ್ನು ತಿರಸ್ಕರಿಸಬೇಕೆಂದು ಇರುವೆವೋ, ಅದಕ್ಕೆ ವಿರೋಧವಾಗಿರುವ ಅಲೆಗಳನ್ನು ಎಬ್ಬಿಸುವುದರಿಂದ ಅದು ಸಾಧ್ಯವೆನ್ನುವುದು ನಮಗೆ ತೋರುತ್ತದೆ. ಅನಂತರ ನಮ್ಮ ಸೂಕ್ಷ್ಮ ಸ್ವಭಾವದಲ್ಲಿ ಅವನ್ನು ವಿರೋಧಿಸುವ ಸೂಕ್ಷ್ಮ ತರಂಗಗಳನ್ನು ಎಬ್ಬಿಸಿದರೆ ನಮ್ಮ ಮನಸ್ಸಿನ ಅಂತರಾಳದಲ್ಲಿ ಜಾಗ್ರತವಾಗಿರುವ ಕೋಪದ ಅಲೆಗಳನ್ನು ಅವು ತಡೆಯುತ್ತವೆ. ಸ್ವಭಾವ ಗುಣದಿಂದ ಪ್ರೇರಿತವಾದ ಕ್ರಿಯೆಗಳು ಐಚ್ಛಿಕ ಕ್ರಿಯೆಗಳಂತೆ ಮೊದಲಾಗಿ, ಕ್ರಮೇಣ ಸೂಕ್ಷ್ಮವಾದವುಗಳು. ಆದಕಾರಣ ಐಚ್ಛಿಕ ಚಿತ್ತದಲ್ಲಿ ಯಾವಾಗಲೂ ಒಳ್ಳೆಯ ಅಲೆಗಳನ್ನು ಎಬ್ಬಿಸುತ್ತಿದ್ದರೆ, ಅವು ನಮ್ಮ ಮನಸ್ಸಿನ ಅಂತರಾಳಕ್ಕೆ ಹೋಗಿ ಸೂಕ್ಷ್ಮವಾಗಿ ಹೀನ ಸಂಸ್ಕಾರದ ರೂಪದಲ್ಲಿರುವ ಅಲೆಯನ್ನು ವಿರೋಧಿಸುತ್ತವೆ. ಕದಿಯುವ ಆಲೋಚನೆ ಬಂದಾಗ ಕದಿಯದ ಆಲೋಚನೆ ಮಾಡಬೇಕು. ಇನ್ನೊಬ್ಬರಿಂದ ದಾನವನ್ನು ಸ್ವೀಕರಿಸುವ ಯೋಚನೆ ಬಂದಾಗ ಅದನ್ನು ಸ್ವೀಕರಿಸಿದ ಭಾವನೆಯನ್ನು ಆಲೋಚಿಸಬೇಕು. 

\vspace{-0.4cm}

\begin{verse}
ವಿತರ್ಕಾ ಹಿಂಸಾದಯಃ ಕೃತಕಾರಿತಾನುಮೋದಿತಾ ಲೋಭಕ್ರೋಧಮೋಹ–\\ಪೂರ್ವಕಾ ಮೃದುಮಧ್ಯಾಧಿಮಾತ್ರಾ ದುಃಖಾಜ್ಞಾನಾನಂತಫಲಾ ಇತಿ ಪ್ರತಿ–\\ಪಕ್ಷಭಾವನಮ್​~॥ ೩೪~॥
\end{verse}

\vspace{-0.5cm}

\dsize{ಹಿಂಸೆ, ಅಸತ್ಯ ಮುಂತಾದುವುಗಳೇ ಯೋಗ ಕಂಟಕಗಳು. ಇವುಗಳನ್ನು ತಾವೆ ಮಾಡಿರಬಹುದು ಅಥವಾ ಹಾಗೆ ಮಾಡುವಂತೆ ಮತ್ತೊಬ್ಬರನ್ನು ಪ್ರೇರೇಪಿಸಿರಬಹುದು ಅಥವಾ ಇತರರು ಮಾಡಿದುದನ್ನು ಅನುಮೋದಿಸಿರಬಹುದು; ಇವುಗಳು ಲೋಭ, ಕ್ರೋಧ, ಮೋಹಪೂರಿತವಾಗಿರಬಹುದು; ಅಲ್ಪವಾಗಿರಬಹುದು, ಮಧ್ಯತರವಾಗಿರಬಹುದು  ಅಥವಾ ತೀವ್ರವಾಗಿರಬಹುದು – ಎಲ್ಲವೂ ದುಃಖ ಮತ್ತು ಅಜ್ಞಾನಗಳಲ್ಲಿ ಕೊನೆಗಾಣುವುವು. ವಿರೋಧವನ್ನು ಆಲೋಚಿಸುವ ರೀತಿಯೇ ಇದು. }

\vspace{0.1cm}

ನಾವು ಒಂದು ಸುಳ್ಳು ಹೇಳಿದರೆ ಅಥವಾ ಇನ್ನೊಬ್ಬರಿಗೆ ಹೇಳುವಂತೆ ಪ್ರೇರೇಪಿಸಿದರೆ, ಅಥವಾ ಇನ್ನೊಬ್ಬರು ಹೇಳಿದ್ದನ್ನು ಸರಿ ಎಂದು ಒಪ್ಪಿಕೊಂಡರೆ ಅಷ್ಟೂ ಪಾತಕವೇ. ಅದು ಒಂದು ಸಣ್ಣ ಸುಳ್ಳಾದರೂ ಅದು ಸುಳ್ಳೆ. ಪ್ರತಿಯೊಂದು ಹೀನ ಆಲೋಚನೆಯೂ ಕೂಡ ನಮ್ಮ ಕಡೆ ತಿರುಗುವುದು. ನೀವು ಒಂದು ನಿರ್ಜನವಾದ ಗವಿಯೊಳಗೆ ಕುಳಿತು ಮಾಡಿದ ಕೆಟ್ಟ ಆಲೋಚನೆಗಳೂ ಕೂಡ ಶೇಖರಿಸಲ್ಪಟ್ಟ ಯಾವುದಾದರೊಂದು ದುಃಖದ ರೂಪದಲ್ಲಿ ಮಹಾವೇಗದಿಂದ ನಿಮ್ಮ ಕಡೆಗೆ ಹಿಂತಿರುಗುವುವು. ದ್ವೇಷ ಅಸೂಯೆಗಳನ್ನು ನೀವು ತೋರಿದರೆ ಅವು ಚಕ್ರಬಡ್ಡಿ ಸಹಿತ ನಿಮಗೆ ಹಿಂತಿರುಗಿ ಬರುವುವು. ಯಾವ ಶಕ್ತಿಯೂ ಅದನ್ನು ತಪ್ಪಿಸಲಾರದು. ಒಂದು ಸಲ ಅದನ್ನು ಚಲಿಸುವಂತೆ ಮಾಡಿದರೆ ಅದರ ಫಲವನ್ನು ಅನುಭವಿಸಬೇಕು. ಇದನ್ನು ನೆನಪಿನಲ್ಲಿಡುವುದು ನೀವು ಕೆಟ್ಟುದನ್ನು ಮಾಡುವುದನ್ನು ತಪ್ಪಿಸುತ್ತದೆ. 

\vspace{-0.2cm}

\begin{verse}
ಅಹಿಂಸಾಪ್ರತಿಷ್ಠಾಯಾಂ ತತ್ಸನ್ನಿಧೌ ವೈರತ್ಯಾಗಃ~॥ ೩೫~॥
\end{verse}

\vspace{-0.4cm}

\dsize{ಅಹಿಂಸೆಯಲ್ಲಿ ಪ್ರತಿಷ್ಠಿತವಾದ ಮೇಲೆ ಅವನೆದುರಿಗೆ ಇತರರ ದ್ವೇಷವೆಲ್ಲ ಶಾಂತವಾಗುವುದು. }

\vspace{0.1cm}

ಅಹಿಂಸಾ ವ್ರತವನ್ನು ಸ್ವೀಕರಿಸಿದರೆ ಅವನ ಮುಂದೆ ಸ್ವಭಾವತಃ ಉಗ್ರವಾದ ಮೃಗಗಳೂ ಶಾಂತವಾಗುವುವು. ಆ ಯೋಗಿಯ ಮುಂದೆ ಹುಲಿ ಮತ್ತು ಕುರಿಮರಿಗಳು ಜೊತೆಯಲ್ಲಿ ಆಟವಾಡುವುವು. ನೀವು ಅಂತಹ ಸ್ಥಿತಿಗೆ ಬಂದಾಗ ಮಾತ್ರ ಅಹಿಂಸಾ ವ್ರತದಲ್ಲಿ ಪೂರ್ಣವಾಗಿ ಸ್ಥಿರವಾಗಿರುವಿರಿ ಎಂದು ನಿಮಗೆ ಗೊತ್ತಾಗುವುದು. 

\vspace{-0.2cm}

\begin{verse}
ಸತ್ಯಪ್ರತಿಷ್ಠಾಯಾಂ ಕ್ರಿಯಾಫಲಾಶ್ರಯತ್ವಮ್​~॥ ೩೬~॥
\end{verse}

\vspace{-0.4cm}

\dsize{ಸತ್ಯದಲ್ಲಿ ಪ್ರತಿಷ್ಠನಾದ ಮೇಲೆ, ತನಗೆ ಮತ್ತು ಇತರರಿಗೆ ಕರ್ಮವಿಲ್ಲದೆ ಕರ್ಮ ಫಲವನ್ನು ಯೋಗಿ ಪಡೆಯಬಲ್ಲ. }

\vspace{0.2cm}

ಸತ್ಯಶಕ್ತಿ ನಿಮ್ಮಲ್ಲಿ ಸ್ಥಿರವಾದ ಮೇಲೆ ನೀವು ಕನಸಿನಲ್ಲಿ ಕೂಡ ಸುಳ್ಳನ್ನು ಹೇಳುವುದಿಲ್ಲ. ಮನೋವಾಕ್ಕಾಯವಾಗಿ, ನೀವು ಸತ್ಯವಾಗಿರುವಿರಿ. ನೀವು ಹೇಳುವುದೆಲ್ಲ ಸತ್ಯವಾಗುವುದು. ಪುಣ್ಯಶಾಲಿಯಾಗೆಂದು ನೀವು ಒಬ್ಬನನ್ನು ಹರಸಬಹುದು; ಅವನು ಪುಣ್ಯಶಾಲಿಯಾಗುವನು. ರೋಗಿಗೆ ಗುಣವಾಗು ಎಂದರೆ ಅವನು ತತ್​ಕ್ಷಣವೇ ಗುಣವಾಗುವನು. 

\vspace{-0.2cm}

\begin{verse}
ಅಸ್ತೇಯಪ್ರತಿಷ್ಠಾಯಾಂ ಸರ್ವರತ್ನೋಪಸ್ಥಾನಮ್​~॥ ೩೭~॥
\end{verse}

\vspace{-0.45cm}

\dsize{ಅಸ್ತೇಯದಲ್ಲಿ (ಕಳ್ಳತನ ಮಾಡದೆ ಇರುವುದರಲ್ಲಿ) ಪ್ರತಿಷ್ಠಿತನಾದ ಮೇಲೆ ಯೋಗಿಗೆ ಎಲ್ಲಾ ಐಶ್ವರ್ಯಗಳೂ ಪ್ರಾಪ್ತವಾಗುವುವು. }

\vspace{0.2cm}

ಪ್ರಕೃತಿಯಿಂದ ನೀವು ದೂರ ಹೋದಷ್ಟೂ ಅದು ನಿಮ್ಮನ್ನು ಅನುಸರಿಸುವುದು. ನೀವು ಅದನ್ನು ಲೆಕ್ಕಿಸದೆ ಇದ್ದರೆ ಅದು ನಿಮ್ಮ ಆಳಾಗುವುದು. 

\eject

\begin{verse}
ಬ್ರಹ್ಮಚರ್ಯಪ್ರತಿಷ್ಠಾಯಾಂ ವೀರ್ಯಲಾಭಃ~॥ ೩೮~॥
\end{verse}

\vspace{-0.4cm}

\dsize{ಬ್ರಹ್ಮಚರ್ಯದಲ್ಲಿ ಸ್ಥಿರವಾದ ಮೇಲೆ ವೀರ್ಯ ಲಾಭವಾಗುವುದು. }

\vspace{0.2cm}

ಬ್ರಹ್ಮಚರ್ಯದಿಂದ ಕೂಡಿದವರಿಗೆ ವಿಶೇಷ ಶಕ್ತಿಯೂ, ಅದ್ಭುತವಾದ ಇಚ್ಛಾ ಶಕ್ತಿಯೂ ಇರುವುದು. ಬ್ರಹ್ಮಚರ್ಯವಿಲ್ಲದೆ ಆಧ್ಯಾತ್ಮಿಕ ಶಕ್ತಿ ಇರುವುದು ಸಾಧ್ಯವಿಲ್ಲ. ಬ್ರಹ್ಮಚರ್ಯದ ಅಭ್ಯಾಸದಿಂದ ಮನುಷ್ಯರ ಮೇಲೆ ನಮ್ಮ ಪರಿಣಾಮವನ್ನು ಬೀರುವ ಅತ್ಯಂತ ಅಮೋಘ ಶಕ್ತಿ ಬರುವುದು. ಮಾನವರ ಧಾರ್ಮಿಕ ಮುಂದಾಳುಗಳು ಬ್ರಹ್ಮಚಾರಿಗಳಾಗಿದ್ದರು. ಅವರಿಗೆ ಶಕ್ತಿಯನ್ನು ನೀಡಿದ್ದೇ ಇದು. ಆದಕಾರಣವೇ ಯೋಗಿಯು ಬ್ರಹ್ಮಚಾರಿಯಾಗಿರಬೇಕು. 

\vspace{-0.2cm}

\begin{verse}
ಅಪರಿಗ್ರಹಸ್ಥೈರ್ಯೇ ಜನ್ಮಕಥನ್ತಾ ಸಂಬೋಧಃ~॥ ೩೯~॥
\end{verse}

\vspace{-0.4cm}

\dsize{ಅಪರಿಗ್ರಹದಲ್ಲಿ ಸ್ಥಿರನಾದ ಮೇಲೆ ಹಿಂದಿನ ಜನ್ಮಗಳ ನೆನಪು ಬರುವುದು. }

\vspace{0.2cm}

ಒಬ್ಬನು ಮತ್ತೊಬ್ಬನಿಂದ ದಾನವನ್ನು ಸ್ವೀಕರಿಸದೆ ಇದ್ದರೆ, ಇನ್ನೊಬ್ಬನಿಗೆ ಋಣಿಯಾಗದೆ ಸ್ವತಂತ್ರನಾಗಿರುವನು. ಅವನ ಮನಸ್ಸು ಪವಿತ್ರವಾಗುವುದು. ತೆಗೆದುಕೊಳ್ಳುವ ಪ್ರತಿಯೊಂದು ದಾನದಿಂದಲೂ ದಾನಿಯ ಪಾಪವನ್ನು ಸ್ವೀಕರಿಸುವ ಸಂಭವವಿದೆ. ಅವನು ದಾನವನ್ನು ಸ್ವೀಕರಿಸದೆ ಇದ್ದರೆ ಮನಸ್ಸು ಶುದ್ಧಿಯಾಗುವುದು. ಇದರಿಂದ ಬರುವ ಮೊದಲನೆ ಶಕ್ತಿಯೆ ಹಿಂದಿನ ಜನ್ಮಗಳ ನೆನಪು. ಆಗ ಮಾತ್ರ ಯೋಗಿ ತನ್ನ ಧ್ಯೇಯದಲ್ಲಿ ಸ್ಥಿರನಾಗಿರುವನು. ಅನೇಕ ವೇಳೆ ಜಗತ್ತಿಗೆ ಬಂದು ಹೋಗುತ್ತಿರುವುದು ಅವನಿಗೆ ತಿಳಿಯುವುದು. ಆಗ ಯೋಗಿ ತಾನು ಮತ್ತೊಮ್ಮೆ ಜಗತ್ತಿಗೆ ಹಿಂತಿರುಗಿ ಬರುವುದಿಲ್ಲ, ಪ್ರಕೃತಿಯ ಗುಲಾಮನಾಗುವುದಿಲ್ಲ ಎಂದು ಶಪಥ ಮಾಡುವನು. 

\vspace{-0.2cm}

\begin{verse}
ಶೌಚಾತ್ಸ್ವಾಂಗಜುಗುಪ್ಸಾ ಪರೈರಸಂಸರ್ಗಃ~॥ ೪೦~॥
\end{verse}

\vspace{-0.4cm}

\dsize{ಬಾಹ್ಯ ಮತ್ತು ಆಂತರಿಕ ಶೌಚದಲ್ಲಿ ಸ್ಥಿರನಾದ ಮೇಲೆ ತನ್ನ ದೇಹದ ಮೇಲೆ ಮತ್ತು ಪರರ ಸಂಬಂಧದೊಂದಿಗೆ ಜುಗುಪ್ಸೆ ಬರುವುದು. }

\vspace{0.2cm}

ದೇಹದಲ್ಲಿ ನಿಜವಾಗಿಯೂ ಆಂತರಿಕ ಮತ್ತು ಬಾಹ್ಯ ಶೌಚವಿದ್ದರೆ ಆಗ ದೇಹದ ಮೇಲೆ ಅಸಡ್ಡೆ ಬರುವುದು. ದೇಹವನ್ನು ಚೆನ್ನಾಗಿ ಇಟ್ಟಿರಬೇಕು ಎಂಬ ಭಾವನೆ ಹೋಗುವುದು. ಯಾರನ್ನು ಅತ್ಯಂತ ಸುಂದರನೆಂದು ಹೇಳುವರೋ ಅದರ ಹಿಂದೆ ಚೈತನ್ಯವಿಲ್ಲದೆ ಇದ್ದರೆ ಅದು ಕೇವಲ ಒಂದು ಮೃಗದೋಪಾದಿಯಲ್ಲಿ ಯೋಗಿಗೆ ಕಾಣುವುದು. ಜಗತ್ತು ಯಾವುದನ್ನು ಅತಿ ಸಾಮಾನ್ಯವೆಂದು ಹೇಳುವುದೋ ಅದರ ಹಿಂದೆ ಚೈತನ್ಯ ಬೆಳಗುತ್ತಿದ್ದರೆ ಅದೇ ದೈವಿಕವಾಗುವುದು. ದೇಹಾಸಕ್ತಿಯೇ ಮಾನವನ ಜೀವನದಲ್ಲಿರುವ ಘೋರದುಃಖ. ಆದಕಾರಣ ಶೌಚದಲ್ಲಿ ಸ್ಥಿರವಾದವನ ಮೊದಲನೆ ಗುರುತೆ, ತಾನು ದೇಹವೆಂದು ತಿಳಿಯಲು ಆಶಿಸುವುದಿಲ್ಲ ಎಂಬುದು. ಶುಚಿ ಬಂದಾಗಲೆ ನಾವು ದೇಹಭಾವನೆಯಿಂದ ಮುಕ್ತರಾಗುವುದು. 

\newpage

\vspace{-0.45cm}

\begin{verse}
ಸತ್ತ್ವಶುದ್ಧಿ–ಸೌಮನಸ್ಯೈಕಾಗ್ರ್ಯೇಂದ್ರಿಯಜಯಾತ್ಮದರ್ಶನಯೋಗ್ಯತ್ವಾನಿ ಚ~॥~೪೧~॥
\end{verse}

\vspace{-0.35cm}

\dsize{ಜೊತೆಗೆ ಸತ್ತ್ವಶುದ್ಧಿ, ಮನೋಲ್ಲಾಸ, ಏಕಾಗ್ರತೆ, ಇಂದ್ರಿಯನಿಗ್ರಹ, ಆತ್ಮ ಸಾಕ್ಷಾತ್ಕಾರಕ್ಕೆ ಅರ್ಹತೆ ಇವೆಲ್ಲ ಪ್ರಾಪ್ತವಾಗುವುವು. }

\vspace{0.1cm}


ಶೌಚದ ಅಭ್ಯಾಸದಿಂದ ಸತ್ತ್ವಶುದ್ಧಿಯಾಗುತ್ತದೆ. ಮನಸ್ಸು ಏಕಾಗ್ರವಾಗಿ ಉಲ್ಲಾಸದಿಂದ ಕೂಡಿರುವುದು. ನೀವು ಧಾರ್ಮಿಕರಾಗುತ್ತಿರುವಿರಿ ಎನ್ನುವುದಕ್ಕೆ ಮೊದಲನೆ ಗುರುತೆ ನೀವು ಹೆಚ್ಚು ಉಲ್ಲಾಸಭರಿತರಾಗಿರುವುದು. ಪೆಚ್ಚು ಮುಖವಾಗಿದ್ದರೆ ಅವನಿಗೆ ಏನೋ ಅಗ್ನಿಮಾಂದ್ಯವಿರಬಹುದೇ ಹೊರತು ಎಂದಿಗೂ ಅದು ಧಾರ್ಮಿಕತೆಯ ಕುರುಹಲ್ಲ. ಸಾತ್ತ್ವಿಕತೆಯ ಸ್ವಭಾವ ಆನಂದದಾಯಕವಾದುದು. ಸಾತ್ತ್ವಿಕನಿಗೆ ಎಲ್ಲವೂ ಸಂತೋಷದಾಯಕ. ಈ ಭಾವನೆ ಬಂದರೆ ನೀವು ಯೋಗದಲ್ಲಿ ಮುಂದುವರಿಯುತ್ತಿರುವಿರಿ ಎಂದು ತಿಳಿಯಿರಿ. ತಮಸ್ಸಿನಿಂದಲೇ ಎಲ್ಲಾ ದುಃಖ ಪ್ರಾಪ್ತವಾಗುವುದು. ಆದಕಾರಣ ಅದರಿಂದ ಪಾರಾಗಿ. ಅಳುಮೋರೆ ತಾಮಸಿಕತೆಯ ಫಲ. ಆಶಿಷ್ಟರೂ, ಧೃಢಿಷ್ಠರೂ, ಯುವಕರೂ, ಸಾಹಸಪ್ರಿಯರೂ ಮಾತ್ರ ಯೋಗಿಗಳಾಗಲು ಅರ್ಹರು. ಯೋಗಿಗೆ ಎಲ್ಲವೂ ಆನಂದ\break ಮಯ. ಅವನು ನೋಡುವ ಪ್ರತಿಯೊಂದು ಮಾನವನ ಮುಖವೂ ಮಂದಹಾಸವನ್ನು ತರುವುದು. ಗುಣಾಢ್ಯನ ಗುರುತು ಇದು. ದುಃಖ ಪಾಪದಿಂದಾದುದು, ಮತ್ತಾವುದರಿಂದಲೂ ಅಲ್ಲ. ಅಳುಮೋರೆಯನ್ನು ಕಟ್ಟಿಕೊಂಡು ನೀವೇನು ಮಾಡುತ್ತೀರಿ? ಅದು ಭಯಂಕರವಾದುದು. ಅಳುಮೋರೆ ಇದ್ದರೆ ಅಂದು ಹೊರಗೆ ಹೋಗಬೇಡಿ. ಕೋಣೆಯಲ್ಲೇ ಕುಳಿತುಕೊಳ್ಳಿ. ಈ ರೋಗವನ್ನು ಪ್ರಪಂಚಕ್ಕೆ ಹರಡುವುದಕ್ಕೆ ನಿಮಗೆ ಯಾವ ಅಧಿಕಾರವಿದೆ? ನಿಮ್ಮ ಮನಸ್ಸು ಸ್ಥಿರವಾದ ಮೇಲೆ ನಿಮಗೆ ದೇಹದ ಮೇಲೆ ಸ್ವಾಧೀನಕ್ಕೆ  ಬರುವುದು. ನೀವು ಈ ಯಂತ್ರಕ್ಕೆ ಗುಲಾಮರಾಗುವ ಬದಲು ಈ ಯಂತ್ರ ನಿಮಗೆ ಆಳಾಗುವುದು. ಯಂತ್ರವು ಆತ್ಮ ನನ್ನು ಹೀನ ಸ್ಥಿತಿಗೆ ಎಳೆಯುವ ಬದಲು ಅದರ ಪರಮಸಹಕಾರಿಯಾಗುವುದು. 

\vspace{-0.25cm}

\begin{verse}
ಸಂತೋಷಾದನುತ್ತಮಃ ಸುಖಲಾಭಃ~॥ ೪೨~॥
\end{verse}

\vspace{-0.4cm}

\dsize{ಸಂತೃಪ್ತಿಯಿಂದ ಉತ್ತಮ ಸುಖಲಾಭವಾಗುವುದು. }

\vspace{0.1cm}

\vspace{-0.25cm}
%%\vspace{-0.3cm}

\begin{verse}
ಕಾಯೇಂದ್ರಿಯಸಿದ್ಧಿರಶುದ್ಧಿಕ್ಷಯಾತ್ತಪಸಃ~॥ ೪೩~॥
\end{verse}

\vspace{-0.35cm}
%%\vspace{-0.3cm}

\dsize{ತಪಸ್ಸಿನ ಪ್ರಭಾವವು ತಕ್ಷಣವೇ ಗೊತ್ತಾಗುವುದು. ಗ್ರಹಣ ಶಕ್ತಿ ತೀವ್ರವಾಗಿ ಕೆಲವು ವೇಳೆ ದೂರದಿಂದ ನೋಡುವುದು ಮತ್ತು ಕೇಳುವುದು ಸಾಧ್ಯವಾಗುತ್ತದೆ. }

\vspace{-0.2cm}
%%\vspace{-0.3cm}

\begin{verse}
ಸ್ವಾಧ್ಯಾಯಾದಿಷ್ಟದೇವತಾಸಂಪ್ರಯೋಗಃ~॥ ೪೪~॥
\end{verse}

\vspace{-0.35cm}
%%\vspace{-0.3cm}

\dsize{ಮಂತ್ರೋಚ್ಛಾರಣೆಯಿಂದ ಇಷ್ಟದೇವತಾಸಿದ್ಧಿಯಾಗುವುದು. }

\vspace{0.1cm}

ನೀವು ಪಡೆಯಬೇಕೆಂದಿರುವ ಇಷ್ಟದೇವತೆ ಎಷ್ಟು ಮೇಲಿದ್ದರೆ ಅಷ್ಟೇ ಹೆಚ್ಚು ಸಾಧನೆ ಮಾಡಬೇಕಾಗುತ್ತದೆ. 

\vspace{-0.3cm}
%%\vspace{-0.3cm}

\vspace{0.1cm}

\begin{verse}
ಸಮಾಧಿಸಿದ್ಧಿರೀಶ್ವರಪ್ರಣಿಧಾನಾತ್​~॥ ೪೫~॥
\end{verse}

\vspace{0.05cm}

\vspace{-0.35cm}
%%\vspace{-0.3cm}

\dsize{ಈಶ್ವರಪ್ರಣಿಧಾನದಿಂದ ಸಮಾಧಿ ಬರುವುದು. }

\vspace{0.1cm}

ದೇವರಲ್ಲಿ ಶರಣಾಗತನಾಗುವುದರಿಂದ ಪರಿಪೂರ್ಣ ಸಮಾಧಿಯು ಸಿದ್ಧಿಸುತ್ತದೆ. 

\eject

\vspace{-0.35cm}
%%\vspace{-0.3cm}

\begin{verse}
ಸ್ಥಿರಸುಖಮಾಸನಮ್​~॥ ೪೬~॥
\end{verse}

\vspace{-0.35cm}
%%\vspace{-0.3cm}

\dsize{ಸ್ಥಿರವಾಗಿ ಸಂತೋಷದಾಯಕವಾಗಿರುವುದೇ ಆಸನ. }

\vspace{0.1cm}

ಈಗ ಆಸನದ ವಿಷಯ. ನಿಮಗೆ ಸ್ಥಿರ ಆಸನ ಸಿದ್ಧಿಸುವವರೆಗೂ ಪ್ರಾಣಾಯಾಮ ಮುಂತಾದವುಗಳನ್ನು ಮಾಡುವುದಕ್ಕೆ ಆಗುವುದಿಲ್ಲ. ಸ್ಥಿರಾಸನವೆಂದರೆ ನಿಮಗೆ ಆಗ ದೇಹದ ಭಾವನೆ ಇರುವುದಿಲ್ಲ. ಸಾಧಾರಣವಾಗಿ ನೀವು ಕೆಲಸ ನಿಮಿಷ ಕುಳಿತ ಒಡನೆಯೆ ಎಲ್ಲಾ ವಿಧದ ತೊಂದರೆಗಳೂ ಬರುವುವು. ಆದರೆ ಸ್ಥೂಲದೇಹದ ಭಾವನೆಯನ್ನು ಮೀರಿದರೆ ಆಗ ನಿಮಗೆ ದೇಹದ ಭಾವನೆಯು ಹೊರಟುಹೋಗುತ್ತದೆ. ನಿಮಗೆ ಸುಖದುಃಖಗಳು ಅನುಭವವಾಗುವುದಿಲ್ಲ. ಅನಂತರ ನೀವು ಎದ್ದರೆ ಎಷ್ಟೋ ವಿಶ್ರಾಂತಿ ಸಿಕ್ಕಿದಂತೆ ತೋರುವುದು. ದೇಹಕ್ಕೆ ನೀವು ಕೊಡುವ ಪೂರ್ಣ ವಿಶ್ರಾಂತಿ ಇದೊಂದೆ. ದೇಹವನ್ನು ಜಯಿಸಿ ಅದನ್ನು ಸ್ಥಿರವಾಗಿ ಇಡುವುದನ್ನು ಕಲಿತರೆ ನಿಮ್ಮ ಅಭ್ಯಾಸವೂ ಕೂಡ ಸ್ಥಿರವಾಗುವುದು. ಆದರೆ ನೀವು ದೇಹದಿಂದ ತೊಂದರೆಗೆ ಈಡಾದರೆ ನಿಮ್ಮ ನರಗಳೂ ಕೂಡ ಅಸ್ಥಿರವಾಗುತ್ತವೆ. ನೀವು ಮನಸ್ಸನ್ನು ಏಕಾಗ್ರ ಮಾಡಲಾರಿರಿ. 

\vspace{-0.2cm}

\begin{verse}
ಪ್ರಯತ್ನಶೈಥಿಲ್ಯಾನಂತಸಮಾಪತ್ತಿಭ್ಯಾಮ್​~॥ ೪೭~॥
\end{verse}

\vspace{-0.35cm}

\dsize{(ಸ್ವಾಭಾವಿಕವಾದ ಚಂಚಲತೆಯ) ಪ್ರವೃತ್ತಿಯನ್ನು ಕಡಮೆಮಾಡುವುದರಿಂದ ಮತ್ತು ಅನಂತದ ಮೇಲೆ ಧ್ಯಾನ ಮಾಡುವುದರಿಂದ ಆಸನ ಸ್ಥಿರವೂ ಶಾಂತವೂ ಆಗುವುದು. }

\vspace{0.1cm}

ಅನಂತವನ್ನು ಆಲೋಚಿಸುವುದರಿಂದ ನೀವು ಆಸನವನ್ನು ಸ್ಥಿರಗೊಳಿಸಬಹುದು. ಮೇರೆ ಇಲ್ಲದ ಅನಂತವನ್ನು ನಾವು ಯೋಚಿಸಲಾರೆವು. ಆದರೆ ಅನಂತ ಆಕಾಶವನ್ನು ಧ್ಯಾನಿಸಬಹುದು. 

\vspace{-0.2cm}

\begin{verse}
ತತೋ ದ್ವಂದ್ವಾನಭಿಘಾತಃ~॥ ೪೮~॥
\end{verse}

\vspace{-0.35cm}

\dsize{ಸ್ಥಿರ ಆಸನವನ್ನು ಪಡೆದಮೇಲೆ ದ್ವಂದ್ವಗಳು ಅಚಡಣೆಯನ್ನು ತರಲಾರವು. }

\vspace{0.1cm}

ದ್ವಂದ್ವಗಳು–ಎಂದರೆ ಒಳ್ಳೆಯದು, ಕೆಟ್ಟದ್ದು, ಶೀತ, ಉಷ್ಣ ಮತ್ತು ಉಳಿದ ದ್ವಂದ್ವಗಳಾವುವೂ ನಿಮಗೆ ತೊಂದರೆ ಕೊಡಲಾರವು. 

\vspace{-0.2cm}

\begin{verse}
ತಸ್ಮಿನ್​ ಸತಿ ಶ್ವಾಸಪ್ರಶ್ವಾಸಯೋರ್ಗತಿವಿಚ್ಛೇದಃ ಪ್ರಾಣಾಯಾಮಃ~॥ ೪೯~॥
\end{verse}

\vspace{-0.25cm}

\dsize{ಇದಾದ ನಂತರ ಉಚ್ಛ್ವಾಸನಿಃಶ್ವಾಸಗಳ ಕ್ರಿಯೆಯನ್ನು ನಿಗ್ರಹಿಸುವ ಸಾಧನೆ ಬರುವುದು. }

\vspace{0.1cm}

ಆಸನವು ಸ್ಥಿರವಾದಮೇಲೆ ಪ್ರಾಣದ ಸಂಚಾರವನ್ನು ತಡೆದು ನಿಗ್ರಹಿಸಬೇಕು. ಹೀಗೆ ದೇಹದ ಶಕ್ತಿಗಳನ್ನು ನಿಗ್ರಹಿಸುವ ಪ್ರಾಣಾಯಾಮಕ್ಕೆ ಬರುವೆವು. ಪ್ರಾಣವೆಂದರೆ ಅನೇಕ ವೇಳೆ ಉಸಿರೆಂದು ಭಾಷಾಂತರ ಮಾಡಿದ್ದರೂ ಇದು ಉಸಿರಲ್ಲ. ಇದು ವಿಶ್ವಶಕ್ತಿಯ ಮೊತ್ತ. ಪ್ರತಿ ದೇಹದಲ್ಲಿರುವ ಶಕ್ತಿಯೇ ಇದು. ಇದು ನಮಗೆ ಸ್ಪಷ್ಟವಾಗಿ ತೋರುವುದೇ ಶ್ವಾಸಕೋಶಗಳ ಚಲನೆಯಲ್ಲಿ. ಈ ಚಲನೆಯು ನಾವು ಶ್ವಾಸವನ್ನು ಒಳಗೆ ಸೆಳೆದು ಕೊಳ್ಳುವುದರಿಂದ ಆಗುವುದು. ಪ್ರಾಣಾಯಾಮದಲ್ಲಿ ನಿಗ್ರಹಿಸಬೇಕೆಂದಿರುವುದೇ ಇದನ್ನು.\break ಉಸಿರನ್ನು ನಿಗ್ರಹಿಸುವುದು ಪ್ರಾಣದ ನಿಗ್ರಹಕ್ಕೆ ಬಹಳ ಸುಲಭವಾದ ಮಾರ್ಗವೆಂದು\break ಅದರಿಂದ ಮೊದಲು ಮಾಡುತ್ತೇವೆ. 

\vspace{-0.3cm}

\begin{verse}
ಬಾಹ್ಯಾಭ್ಯಂತರಸ್ತಂಭವೃತ್ತಿಃ ದೇಶಕಾಲಸಂಖ್ಯಾಭಿಃ ಪರಿದೃಷ್ಟೋ\\
\hspace{6cm} ದೀರ್ಘಸೂಕ್ಷ್ಮಃ~॥~೫೦~॥
\end{verse}

\vspace{-0.35cm}

\dsize{ಈ ಪ್ರಾಣಾಯಾಮವು ಬಾಹ್ಯವಾಗಿರಬಹುದು ಅಥವಾ ಆಂತರಿಕವಾಗಿರಬಹುದು ಅಥವಾ ಚಲನರಹಿತವಾಗಿರಬಹುದು. ಅದು ದೇಶಕಾಲಸಂಖ್ಯೆಗಳಿಂದ ನಿಯಂತ್ರಿತವಾಗಿರುವುದು ಮತ್ತು ದೀರ್ಘ ಅಥವಾ ಸಂಕ್ಷೇಪವಾಗಿರುವುದು. }

\vspace{0.1cm}

ಪ್ರಾಣಾಯಾಮದ ಮೂರು ಚಲನೆಗಳೇ–ಉಸಿರನ್ನು ಸೆಳೆದುಕೊಳ್ಳುವುದು, ಉಸಿರನ್ನು ಬಿಡುವುದು, ಹಾಗೂ ಉಸಿರನ್ನು ಒಳಗೆ ನಿಲ್ಲಿಸುವುದು ಅಥವಾ ಹೊರಗಿನಿಂದ ಒಳಗೆ ಬರದಂತೆ ಮಾಡುವುದು. ಇವು ದೇಶ ಕಾಲಕ್ಕೆ ಅನುಗುಣವಾಗಿ ಬದಲಾವಣೆ ಹೊಂದುತ್ತವೆ. ದೇಶವೆಂದರೆ ದೇಹದ ಒಂದು ಭಾಗದಲ್ಲಿ ಪ್ರಾಣವನ್ನು ನಿಲ್ಲಿಸುವುದು ಎಂದು ಅರ್ಥ. ಪ್ರಾಣವನ್ನು ಒಂದು ಭಾಗದಲ್ಲಿ ಎಷ್ಟು ಕಾಲ ನಿಲ್ಲಿಸಬೇಕು ಎಂಬುದೇ ಕಾಲ. ಅದಕ್ಕೇ ಒಂದು ಚಲನೆಯನ್ನು ಎಷ್ಟು ಕ್ಷಣ ಮಾಡಬೇಕು; ಇನ್ನೊಂದು ಚಲನೆಯನ್ನು ಎಷ್ಟು ಕ್ಷಣ ಮಾಡಬೇಕು ಎನ್ನುವುದನ್ನು ಹೇಳುತ್ತಾರೆ. ಈ ಪ್ರಾಣಾಯಾಮದ ಪರಿಣಾಮವೇ ಅದ್ಭುತ ಕುಂಡಲಿನಿ ಶಕ್ತಿಯನ್ನು ಜಾಗ್ರತಗೊಳಿಸುವುದು. 

\vspace{-0.2cm}

\begin{verse}
ಬಾಹ್ಯಾಭ್ಯಂತರ ವಿಷಯಾಕ್ಷೇಪೀ ಚತುರ್ಥಃ~॥ ೫೧~॥
\end{verse}

\vspace{-0.4cm}

\dsize{ಇದು ನಾಲ್ಕನೇ ರೀತಿಯ ಪ್ರಾಣಾಯಾಮ, ಪ್ರಾಣವನ್ನು ಬಾಹ್ಯ ಅಥವಾ ಆಂತರಿಕ ವಸ್ತುವಿನ ಮೇಲೆ ನಿರ್ದೇಶಿಸುವುದು. }

\vspace{0.1cm}

ನಾಲ್ಕನೆಯದೇ ಪ್ರಾಣವನ್ನು ಯಾವುದಾದರೂ ಬಾಹ್ಯ ಅಥವಾ ಆಂತರಿಕ ವಸ್ತುವಿನಮೇಲೆ ನಿಲ್ಲಿಸಿ ನಿಗ್ರಹಿಸುವುದು. ಹಿಂದಿನ ಮೂರು ರೀತಿಯ ಪ್ರಾಣಾಯಾಮಗಳಂತಲ್ಲದೆ ಇಲ್ಲಿ ಕುಂಭಕವು ಸಾಧಿತವಾಗುತ್ತದೆ. ಅದು ಧೀರ್ಘ ಅಭ್ಯಾಸದಿಂದ ಸಾಧ್ಯ. ಅದರೊಂದಿಗೆ ಪ್ರಾಣವನ್ನು ಬಾಹ್ಯ ಮತ್ತು ಆಂತರಿಕ ವಸ್ತುವಿನ ಮೇಲೆ ನಿರ್ದೇಶಿಸಬಹುದು. 

\vspace{-0.2cm}

\begin{verse}
ತತಃ ಕ್ಷೀಯತೇ ಪ್ರಕಾಶಾವರಣಮ್​~॥ ೫೨~॥
\end{verse}

\vspace{-0.4cm}

\dsize{ಚಿತ್ತ ಜ್ಯೋತಿಯನ್ನು ಮುಸುಕಿದ ತೆರೆ ಇದರಿಂದ ವಿರಳವಾಗುವುದು. }

\vspace{0.1cm}

ಚಿತ್ತಕ್ಕೆ ಸ್ವಭಾವತಃ ಎಲ್ಲಾ ಜ್ಞಾನವೂ ಇರುವುದು. ಇದು ಸತ್ತ್ವಕಣಗಳಿಂದ ಆಗಿದೆ. ಆದರೆ ಇದರ ಮೇಲೆ ರಾಜಸಿಕ ಮತ್ತು ತಾಮಸಿಕ ಕಣಗಳು ಕುಳಿತಿರುವುವು. ಪ್ರಾಣಾಯಾಮದಿಂದ ಈ ಆವರಣವೂ ಜಾರುವುದು. 

\vspace{-0.2cm}

\begin{verse}
ಧಾರಣಾಸು ಚ ಯೋಗ್ಯತಾ ಮನಸಃ~॥ ೫೩~॥
\end{verse}

\vspace{-0.4cm}

\dsize{ಮನಸ್ಸು ಧಾರಣಕ್ಕೆ ಯೋಗ್ಯವಾಗುವುದು. }

\vspace{0.2cm}

ಈ ಆವರಣವು ಬಿದ್ದಮೇಲೆ ನಾವು ಮನಸ್ಸನ್ನು ಏಕಾಗ್ರಗೊಳಿಸಬಹುದು. 

%%\vspace{-0.3cm}

\begin{verse}
ಸ್ವಸ್ವವಿಷಯಾಸಂಪ್ರಯೋಗೇ ಚಿತ್ತಸ್ಯ ಸ್ವರೂಪಾನುಕಾರ\\ ಇವೇಂದ್ರಿಯಾಣಾಂ ಪ್ರತ್ಯಾಹಾರಃ~॥ ೫೪~॥
\end{verse}

\vspace{-0.3cm}

\dsize{ಇಂದ್ರಿಯಗಳನ್ನು ಒಳಗೆ ಸೆಳೆಯಬೇಕಾದರೆ ಅವು ತಮ್ಮ ವಸ್ತುವನ್ನು ತ್ಯಜಿಸಿ ಚಿತ್ತ ಸ್ವರೂಪವನ್ನು\break ಧರಿಸಬೇಕು. }

\vspace{0.2cm}

ಇಂದ್ರಿಯಗಳು ಚಿತ್ತದ ಬೇರೆ ಬೇರೆ ಅವಸ್ಥೆಗಳು. ನಾನು ಒಂದು ಪುಸ್ತಕವನ್ನು ನೋಡುತ್ತೇನೆ. ಆಕಾರ ಪುಸ್ತಕದಲ್ಲಿ ಇಲ್ಲ. ಅದು ಮನಸ್ಸಿನಲ್ಲಿದೆ. ಆಕಾರವನ್ನು ಮನಸ್ಸಿಗೆ ತರುವ ಒಂದು ವಸ್ತು ಹೊರಗೆ ಇದೆ. ನಿಜವಾದ ಆಕಾರ ಚಿತ್ತದಲ್ಲಿದೆ. ಇಂದ್ರಿಯಗಳು ತದೈಕ್ಯಭಾವವನ್ನು ತಾಳಿ, ತಮಗೆ ಯಾವ ವಸ್ತು ಬಂದರೂ ಅದರ ಆಕಾರವನ್ನು ಧರಿಸುತ್ತವೆ. ಇಂತಹ ಆಕಾರ ಧರಿಸದಂತೆ ನೀವು ಮನಸ್ಸನ್ನು ನಿಗ್ರಹಿಸಿದರೆ ಮನಸ್ಸು ಶಾಂತವಾಗುವುದು. ಇದನ್ನೇ ಪ್ರತ್ಯಾಹಾರವೆನ್ನುವುದು. 

\vspace{-0.2cm}

\begin{verse}
ತತಃ ಪರಮಾವಶ್ಯತೇಂದ್ರಿಯಾಣಾಮ್​~॥ ೫೫~॥
\end{verse}

\vspace{-0.4cm}

\dsize{ಅನಂತರ ಸಂಪೂರ್ಣ ಇಂದ್ರಿಯ ನಿಗ್ರಹವಾಗುವುದು. }

\vspace{0.2cm}

ಇಂದ್ರಿಯಗಳು ಬಾಹ್ಯವಸ್ತುವಿನ ರೂಪವನ್ನು ಧಾರಣೆ ಮಾಡುವುದನ್ನು ನಿಲ್ಲಿಸಿ. ಅವನ್ನು ಚಿತ್ತದಲ್ಲಿ ಐಕ್ಯವಾಗುವಂತೆ ಯೋಗಿಯು ಮಾಡಿದಮೇಲೆ ಸಂಪೂರ್ಣ ಇಂದ್ರಿಯನಿಗ್ರಹ ಕರಗತವಾಗುವುದು. ಇಂದ್ರಿಯಗಳು ಸಂಪೂರ್ಣವಾಗಿ ವಶವಾದ ಮೇಲೆ ಪ್ರತಿಯೊಂದು ಮಾಂಸಖಂಡ ಮತ್ತು ನರಗಳು ಸ್ವಾಧೀನದಲ್ಲಿರುತ್ತವೆ. ಏಕೆಂದರೆ ಇಂದ್ರಿಯಗಳೇ ಎಲ್ಲಾ ಸಂವೇದನೆಗಳಿಗೂ ಮತ್ತು ಕ್ರಿಯೆಗಳಿಗೂ ಕೇಂದ್ರವಾಗಿರುವುವು. ಇಂದ್ರಿಯಗಳು ಕರ್ಮೇಂದ್ರಿಯ ಮತ್ತು ಜ್ಞಾನೇಂದ್ರಿಯಗಳೆಂದು ವಿಭಾಗವಾಗಿರುವುವು. ಇಂದ್ರಿಯಗಳು ವಶವಾದ ಮೇಲೆ ಯೋಗಿ ಎಲ್ಲಾ ಭಾವನೆಗಳನ್ನು ಮತ್ತು ಕರ್ಮಗಳನ್ನು ನಿಗ್ರಹಿಸಬಲ್ಲ. ದೇಹವೆಲ್ಲ ಅವನ ವಶವಾಗುವುದು. ಆಗ ಮಾತ್ರ ಪ್ರಪಂಚದಲ್ಲಿ ಹುಟ್ಟಿದುದರಿಂದ ಒಬ್ಬನಿಗೆ ಸಂತೋಷವಾಗುವುದು. ಆಗ ಮಾತ್ರ ಅವನು ಸತ್ಯವಾಗಿ “ಧನ್ಯನಾದೆ, ನಾನು ಹುಟ್ಟಿ ಸಾರ್ಥಕವಾಯಿತು” ಎಂದು ಹೇಳಿಕೊಳ್ಳಬಹುದು. ಇಂದ್ರಿಯಗಳು ಸ್ವಾಧೀನತೆಗೆ ಬಂದಮೇಲೆ ನಿಜವಾಗಿ ಈ ದೇಹ ಎಷ್ಟು ಅದ್ಭುತವಾಗಿದೆ ಎಂಬುದು ಗೊತ್ತಾಗುವುದು.

\chapter{ಸಿದ್ಧಿಗಳು}%%೩೫

ಯೋಗಸಿದ್ಧಿಯನ್ನು ವಿವರಿಸುವ ಅಧ್ಯಾಯಕ್ಕೆ ಈಗ ಬಂದಿರುವೆವು. 

\vspace{-0.2cm}

\begin{verse}
ದೇಶಬಂಧಶ್ಚಿತ್ತಸ್ಯ ಧಾರಣಾ~॥ ೧~॥
\end{verse}

\vspace{-0.35cm}

\dsize{ಧಾರಣವೆಂದರೆ ಚಿತ್ತವನ್ನು ಯಾವುದಾದರೂ ಒಂದು ವಸ್ತುವಿನ ಮೇಲೆ ನಿಲ್ಲಿಸುವುದು. }

\vspace{0.2cm}

ಧಾರಣವೆಂದರೆ ಮನಸ್ಸನ್ನು ದೇಹದಲ್ಲಿಯೋ ಅಥವಾ ಹೊರಗಡೆಯೋ ಯಾವುದಾದರೊಂದು ವಸ್ತುವಿನ ಮೇಲೆ ನಿಲ್ಲಿಸಿ ಅದೇ ಸ್ಥಿತಿಯಲ್ಲಿಟ್ಟಿರುವುದು. 

\vspace{-0.2cm}

\begin{verse}
ತತ್ರ ಪ್ರತ್ಯಯೈಕತಾನತಾ ಧ್ಯಾನಮ್​~॥ ೨~॥
\end{verse}

\vspace{-0.4cm}

\dsize{ಆ ವಸ್ತುವಿನ ಮೇಲೆ ಮನಸ್ಸನ್ನು ನಿರಂತರವಾಗಿ ನಿಲ್ಲಿಸುವುದೇ ಧ್ಯಾನ. }

\vspace{0.2cm}

ಮನಸ್ಸು ಒಂದೇ ವಸ್ತುವನ್ನು ಅಥವಾ ಯಾವುದೇ ಭಾಗವನ್ನು, ಅಂದರೆ ನೆತ್ತಿ ಅಥವಾ ಹೃದಯವನ್ನು ಯೋಚಿಸಲು ಯತ್ನಿಸುವುದು. ಎಲ್ಲಾ ಭಾಗಗಳನ್ನೂ ಬಿಟ್ಟು ಆ ಭಾಗದಿಂದ ಮಾತ್ರ ಸಂವೇದನೆಗಳನ್ನು ಸ್ವೀಕರಿಸುವುದರಲ್ಲಿ ಮನಸ್ಸು ಜಯಿಸಿದರೆ ಅದನ್ನು ಧಾರಣವೆನ್ನುವುದು. ಅದೇ ಸ್ಥಿತಿಯಲ್ಲಿ ಮನಸ್ಸು ಕೆಲವು ಕಾಲ ನಿಂತರೆ ಅದೇ ಧ್ಯಾನ. 

\vspace{-0.2cm}

\begin{verse}
ತದೇವ ಅರ್ಥಮಾತ್ರನಿರ್ಭಾಸಂ ಸ್ವರೂಪಶೂನ್ಯಮಿವ ಸಮಾಧಿಃ~॥ ೩~॥
\end{verse}

\vspace{-0.4cm}

\dsize{ಎಲ್ಲ ಆಕಾರಗಳನ್ನು ತೊರೆದು, ಅರ್ಥದ ಮೇಲೆ ಮಾತ್ರ ಧ್ಯಾನ ಮಾಡಿದರೆ ಅದು ಸಮಾಧಿ. }

\vspace{0.2cm}

ಧ್ಯಾನದಲ್ಲಿ ಆಕಾರ ಅಥವಾ ಬಾಹ್ಯವಸ್ತುವಿನ ರೂಪವನ್ನು ತೊರೆದಮೇಲೆ ಇದು ಸಿದ್ಧಿಸುವುದು. ನಾನೀಗ ಒಂದು ಪುಸ್ತಕದ ಮೇಲೆ ಧ್ಯಾನ ಮಾಡುತ್ತಿರುವೆನು ಎಂದು ಇಟ್ಟುಕೊಳ್ಳೋಣ. ಪುಸ್ತಕದ ಮೇಲೆ ನನ್ನ ಮನಸ್ಸು ಕ್ರಮೇಣ ಏಕಾಗ್ರವಾಗುವುದರಲ್ಲಿ ಜಯಿಸಿ ಯಾವ ಆಕಾರದಿಂದಲೂ ವ್ಯಕ್ತವಾಗದ ಆಂತರಿಕ ಸಂವೇದನೆಗಳಿಂದ ಕೂಡಿದ ಅರ್ಥವನ್ನು ಮಾತ್ರ ತಿಳಿದರೆ, ಇಂತಹ ಧ್ಯಾನಕ್ಕೆ ಸಮಾಧಿ ಎಂದು ಹೆಸರು. 

\vspace{-0.2cm}

\begin{verse}
ತ್ರಯಮೇಕತ್ರ ಸಂಯಮಃ~॥ ೪~॥
\end{verse}

\vspace{-0.4cm}

\dsize{ಈ ಮೂರನ್ನೂ ಒಂದು ವಸ್ತುವಿಗಾಗಿ ಅಭ್ಯಾಸ ಮಾಡಿದಾಗ ಅದು ಸಂಯಮ. }

\vspace{0.2cm}

ಯಾವುದಾದರೂ ಒಂದು ವಸ್ತುವಿನ ಮೇಲೆ ತನ್ನ ಮನಸ್ಸನ್ನು ನಿರ್ದೇಶಿಸಿ, ಅಲ್ಲೇ ಕೇಂದ್ರೀಕರಿಸಿ, ದೀರ್ಘಕಾಲ ನಿಲ್ಲಿಸಿ ವಸ್ತುವನ್ನು ಒಳಭಾಗದಿಂದ ಬೇರ್ಪಡಿಸಿದರೆ ಇದು ಸಂಯಮ. ಧಾರಣ, ಧ್ಯಾನ, ಸಮಾಧಿ ಇವು ಒಂದಾದ ಮೇಲೊಂದು ಬರುವುವು. ವಸ್ತುವಿನ ಆಕಾರ ಮಾಯವಾಗಿ ಅದರ ಅರ್ಥ ಮಾತ್ರ ಉಳಿಯುತ್ತದೆ. 

\vspace{-0.2cm}

\begin{verse}
ತಜ್ಜಯಾತ್​ ಪ್ರಜ್ಞಾಲೋಕಃ~॥ ೫~॥
\end{verse}

\vspace{-0.4cm}

\dsize{ಅದನ್ನು ಜಯಿಸುವುದರ ಮೂಲಕ ಜ್ಞಾನವು ಲಭಿಸುವುದು. }

\vspace{0.2cm}


 
ಸಂಯಮದಲ್ಲಿ ಸಿದ್ಧಿ ಪಡೆದ ಮೇಲೆ ಎಲ್ಲಾ ಶಕ್ತಿಗಳೂ ಅವನ ವಶವಾಗುವುವು. ಇದೇ ಯೋಗಿಯ ಅತ್ಯುತ್ತಮವಾದ ಉಪಕರಣ. ಜ್ಞಾನದ ವಿಷಯಗಳು ಅನಂತವಾಗಿವೆ. ಅವು ಸ್ಥೂಲ, ಸ್ಥೂಲತರ, ಸ್ಥೂಲತಮ, ಸೂಕ್ಷ್ಮ, ಸೂಕ್ಷ್ಮತರ, ಸೂಕ್ಷ್ಮತಮ ಇತ್ಯಾದಿಯಾಗಿ ವಿಭಾಗಿಸಲ್ಪಟ್ಟಿರುವುವು. ಸಂಯಮವನ್ನು ಮೊದಲು ಸ್ಥೂಲ ವಸ್ತುವಿನ ಮೇಲೆ ಪ್ರಯೋಗಿಸಬೇಕು. ಸ್ಥೂಲವಸ್ತುವಿನ ಜ್ಞಾನ ಬಂದಮೇಲೆ ನಿಧಾನವಾಗಿ ಕ್ರಮೇಣ ಅದನ್ನು ಸೂಕ್ಷ್ಮ ವಸ್ತುಗಳ ಕಡೆಗೆ ತಿರುಗಿಸಬೇಕು. 

\vspace{-0.2cm}

\begin{verse}
ತಸ್ಯ ಭೂಮಿಷು ವಿನಿಯೋಗಃ~॥ ೬~॥
\end{verse}

\vspace{-0.4cm}

\dsize{ಅದನ್ನು ಹಂತಹಂತವಾಗಿ ಉಪಯೋಗಿಸಬೇಕು. }

\vspace{0.1cm}

ಬಹಳ ವೇಗವಾಗಿ ಹೋಗಲು ಪ್ರಯತ್ನಿಸಿದಂತೆ ಇರಲು ಇದೊಂದು ಎಚ್ಚರಿಕೆ. 

\vspace{-0.2cm}

\begin{verse}
ತ್ರಯಮಂತರಂಗಂ ಪೂರ್ವೇಭ್ಯಃ~॥ ೭~॥
\end{verse}

\vspace{-0.4cm}

\dsize{ಈ ಮೂರು ಹಿಂದಿನವುಗಳಿಗಿಂತ ಆಂತರಿಕವಾದವುಗಳು. }

\vspace{0.1cm}

ಇದಕ್ಕೆ ಮುಂಚೆ ಪ್ರತ್ಯಾಹಾರ, ಪ್ರಾಣಾಯಾಮ, ಆಸನ, ಯಮ ಮತ್ತು ನಿಯಮಗಳು ಇದ್ದುವು. ಇವು ಧ್ಯಾನ, ಧಾರಣ ಮತ್ತು ಸಮಾಧಿಯ ಹೊರಭಾಗ. ಇವುಗಳನ್ನು ಪಡೆದಮೇಲೆ ಒಬ್ಬನು ಸರ್ವಜ್ಞತ್ವ ಮತ್ತು ಸರ್ವಶಕ್ತಿತ್ವವನ್ನು ಪಡೆಯಬಹುದು. ಆದರೆ ಇದು ಮೋಕ್ಷವಾಗುವುದಿಲ್ಲ. ಈ ಮೂರು ಮನಸ್ಸನ್ನು ನಿರ್ವಿಕಲ್ಪ ಮಾಡಲಾರವು. ಪುನಃ ದೇಹಕ್ಕೆ ಕಾರಣವಾದ ಬೀಜಗಳು ಉಳಿಯುವುವು. ಯೋಗಿಯು ಹೇಳುವಂತೆ ಬೀಜವನ್ನು ಹುರಿದಾಗ ಮಾತ್ರ ಮುಂದೆ ಸಸಿಯಾಗುವ ಸ್ವಭಾವವನ್ನು ಕಳೆದುಕೊಳ್ಳುವುದು. ಈ ಸಿದ್ಧಿಗಳು ಬೀಜವನ್ನು ಹುರಿಯಲಾರವು. 

\vspace{-0.3cm}

\begin{verse}
ತದಪಿ ಬಹಿರಂಗಂ ನಿರ್ಬೀಜಸ್ಯ~॥ ೮~॥
\end{verse}

\vspace{-0.4cm}

\dsize{ಆದರೆ ಇವು ಕೂಡ ನಿರ್ಬೀಜ ಸಮಾಧಿಗೆ ಹೊರಗಿನವು. }

\vspace{0.1cm}

ನಿರ್ವಿಕಲ್ಪ ಸಮಾಧಿಯೊಂದಿಗೆ ಹೋಲಿಸಿ ನೋಡಿದರೆ ಇವುಗಳೂ ಕೂಡ ಹೊರಗಿನವು. ನಾವಿನ್ನೂ ಅತ್ಯುತ್ತಮವಾದ ಸಮಾಧಿಗೆ ಸೇರಿಲ್ಲ, ಇರುವುದು ಇನ್ನೂ ಕೆಳಗಿನ ಹಂತದಲ್ಲಿ. ಇಲ್ಲಿ ಜಗತ್ತು ನಾವು ನೋಡುವಂತೆಯೇ ಇನ್ನೂ ಇರುತ್ತದೆ. ಇಲ್ಲೇ ಈ ಸಿದ್ಧಿಗಳೆಲ್ಲ ಇರುವುದು. 

\vspace{-0.3cm}

\begin{verse}
ವ್ಯುತ್ಥಾನ–ನಿರೋಧಸಂಸ್ಕಾರಯೋರಭಿಭವ–ಪ್ರಾದುರ್ಭಾವೌ\\ನಿರೋಧಕ್ಷಣಚಿತ್ತಾನ್ವಯೋ ನಿರೋಧ–ಪರಿಣಾಮಃ~॥ ೯~॥
\end{verse}

\vspace{-0.35cm}

\dsize{ಮನಸ್ಸಿನ ಚಂಚಲ ಸ್ಥಿತಿಯ ಸಂಸ್ಕಾರಗಳನ್ನು ನಿರೋಧಿಸುವುದರ ಮೂಲಕ ಮತ್ತು ನಿರೋಧ ಸಂಸ್ಕಾರಗಳು ಮನಸ್ಸಿನಲ್ಲಿ ಏಳುವುದರ ಮೂಲಕ ಮನಸ್ಸು ಈ ನಿರೋಧ ಕ್ಷಣದಲ್ಲಿದ್ದರೆ ಆ ಸ್ಥಿತಿಯನ್ನು ನಿರೋಧ ಪರಿಣಾಮ ಎಂದು ಕರೆಯುತ್ತಾರೆ. }

\vspace{0.1cm}

ಅಂದರೆ, ಸಮಾಧಿಯ ಮೊದಲನೆ ಸ್ಥಿತಿಯಲ್ಲಿ ಮನಸ್ಸಿನ ವೃತ್ತಿಗಳು ನಿಗ್ರಹಿಸಲ್ಪಟ್ಟಿರುತ್ತವೆ, ಆದರೆ ಪೂರ್ಣವಾಗಿ ಅಲ್ಲ. ಏಕೆಂದರೆ ಅದು ಪೂರ್ಣವಾಗಿದ್ದರೆ ಅಲ್ಲಿ ವೃತ್ತಿಗಳೇ ಇರುತ್ತಿರಲಿಲ್ಲ. ಇಂದ್ರಿಯದ ಮೂಲಕ ಮನಸ್ಸನ್ನು ಹೊರಗೆ ಹೋಗುವಂತೆ ಬಲಾತ್ಕರಿಸುವ ಒಂದು ವೃತ್ತಿ ಇದ್ದರೆ, ಅದನ್ನು ಯೋಗಿಯು ನಿಗ್ರಹಿಸಲು ಪ್ರಯತ್ನಿಸಿದರೆ, ಆ ನಿಗ್ರಹವೇ ಒಂದು ವೃತ್ತಿಯಾಗುವುದು. ಒಂದು ಅಲೆ ಮತ್ತೊಂದು ಅಲೆಯಿಂದ ತಡೆಯಲ್ಪಡುತ್ತದೆ. ಆದಕಾರಣ ಅಲೆಗಳೆಲ್ಲ ಶಾಂತವಾದವು ಎಂದರೆ ಅದನ್ನು ಸಮಾಧಿ ಎಂದು ಹೇಳಲಾಗುವುದಿಲ್ಲ. ಏಕೆಂದರೆ ನಿಗ್ರಹವೇ ಒಂದು ಅಲೆಯಾಗುವುದು. ಆದರೂ ಮನಸ್ಸು ಚಂಚಲವಾಗಿ ಮೇಲೆ ಬರುವುದಕ್ಕಿಂತ, ಈ ಕೆಳಗಿನ ಸಮಾಧಿ ಮೇಲಿನ ಸಮಾಧಿಯ ಹತ್ತಿರದಲ್ಲಿದೆ. 

\vspace{-0.25cm}

\begin{verse}
ತಸ್ಯ ಪ್ರಶಾನ್ತವಾಹಿತಾ ಸಂಸ್ಕಾರಾತ್​~॥ ೧೦~॥
\end{verse}

\vspace{-0.42cm}

\dsize{ಅಭ್ಯಾಸಬಲದಿಂದ ಅದರ ಪ್ರವಾಹ ಏಕಪ್ರಕಾರವಾಗುವುದು. }

\vspace{0.1cm}

ಏಕಪ್ರಕಾರವಾಗಿ ಹರಿಯುವ ಮನಸ್ಸಿನ ಈ ನಿಗ್ರಹ ದಿನದಿನವೂ ಅಭ್ಯಾಸ ಮಾಡಿದಂತೆ ಸ್ಥಿರವಾಗುವುದು. ಮನಸ್ಸಿಗೆ ಚಿರಏಕಾಗ್ರತೆಯ ಸ್ವಭಾವ ಬರುವುದು. 

\vspace{-0.2cm}

\begin{verse}
ಸರ್ವಾರ್ಥತೈಕಾಗ್ರತಯೋಃ ಕ್ಷಯೋದಯೌ ಚಿತ್ತಸ್ಯ ಸಮಾಧಿಪರಿಣಾಮಃ~॥~೧೧~॥
\end{verse}

\vspace{-0.42cm}

\dsize{ಎಲ್ಲಾ ವಸ್ತುಗಳನ್ನು ತೆಗೆದುಕೊಳ್ಳುವುದು ಮತ್ತು ಒಂದು ವಸ್ತುವಿನ ಮೇಲೆ ಕೇಂದ್ರೀಕರಿಸುವುದು; ಇಂತಹ ಎರಡು ಶಕ್ತಿಯಲ್ಲಿ ಮೊದಲನೆಯದು ನಾಶವಾಗಿ ಎರಡನೆಯದು ಉದಿಸಿದಾಗ ಚಿತ್ತಕ್ಕೆ ಸಮಾಧಿ ಎಂಬ ಅವಸ್ಥೆ ಬರುವುದು.}

\vspace{0.1cm}

ಮನಸ್ಸು ಹಲವು ವಸ್ತುಗಳನ್ನು ಸ್ವೀಕರಿಸುವುದು. ಎಷ್ಟೋ ವಸ್ತುಗಳ ಹಿಂದೆ ಓಡುವುದು. ಅದು ಕೆಳಗಿನ ಸ್ಥಿತಿ. ಇದಕ್ಕಿಂತ ಉತ್ತಮ ಸ್ಥಿತಿ ಇದೆ. ಅಲ್ಲಿ ಎಲ್ಲವನ್ನೂ ಬಿಟ್ಟು ಒಂದನ್ನೆ ಮನಸ್ಸು ಸ್ವೀಕರಿಸುವುದು. ಇದರ ಪರಿಣಾಮವೆ ಸಮಾಧಿ. 

\vspace{-0.2cm}

\begin{verse}
ಶಾನ್ತೋದಿತೌ ತುಲ್ಯಪ್ರತ್ಯಯೌ ಚಿತ್ತಸ್ಯೈಕಾಗ್ರತಾ ಪರಿಣಾಮಃ~॥ ೧೨~॥
\end{verse}

\vspace{-0.42cm}

\dsize{ಭೂತ ಮತ್ತು ವರ್ತಮಾನಗಳ ಸಂಸ್ಕಾರಗಳು ಒಂದೇ ಸಮನಾಗಿದ್ದಾಗ ಅದು ಚಿತ್ತದ ಏಕಾಗ್ರತೆ. }

\vspace{0.2cm}

ಮನಸ್ಸು ಏಕಾಗ್ರವಾಗಿದೆ ಎಂಬುದು ನಮಗೆ ಹೇಗೆ ತಿಳಿಯಬೇಕು? ಎಂದರೆ ಅಲ್ಲಿ ಕಾಲದ ಭಾವನೆ ಮಾಯವಾಗುವುದು. ನಮಗೆ ಅರಿವಿಲ್ಲದೆ ಹೆಚ್ಚು ಸಮಯ ಕಳೆದಷ್ಟೂ ನಮ್ಮ ಮನಸ್ಸು ಹೆಚ್ಚು ಏಕಾಗ್ರವಾಗಿರುತ್ತದೆ. ಸಾಧಾರಣ ಸಮಯದಲ್ಲಿ ಒಂದು ಪುಸ್ತಕವನ್ನು ಆಸಕ್ತಿಯಿಂದ ಓದುತ್ತಿದ್ದರೆ ಕಾಲ ಓಡುತ್ತಿರುವುದು ನಮಗೆ ಗೊತ್ತಾಗುವುದೇ ಇಲ್ಲ. ನಾವು ಪುಸ್ತಕವನ್ನು ಇಟ್ಟಮೇಲೆ ಇಷ್ಟೊಂದು ಗಂಟೆಗಳು ಹೇಗೆ ಕಳೆದಿವೆ ಎಂಬುದನ್ನು ನೋಡಿದರೆ ನಮಗೆ ಆಶ್ಚರ್ಯವಾಗುವುದು. ಎಲ್ಲಾ ಕಾಲವು ವರ್ತಮಾನದಲ್ಲಿ ಬಂದು ನಿಲ್ಲುವುದು ಸ್ವಭಾವ. ಆದ್ದರಿಂದಲೇ ಭೂತ ಮತ್ತು ವರ್ತಮಾನಗಳೆರಡೂ ಒಂದೆಡೆ ನಿಂತಾಗ ಮನಸ್ಸು ಏಕಾಗ್ರವಾಗಿರುವುದು ಎಂದು ಹೇಳಿರುವುದು. \footnote{೯, ೧೧ ಮತ್ತು ೧೨ನೇ ಸೂತ್ರಗಳಲ್ಲಿ ಹೇಳಿರುವ ಮೂರು ಬಗೆಯ ಏಕಾಗ್ರತೆಗಳ ನಡುವೆ ಇರುವ ವ್ಯತ್ಯಾಸ ಇದು: ಮೊದಲನೆಯದರಲ್ಲಿ ಮನಸ್ಸಿನ ವಿಕ್ಷೇಪ ವೃತ್ತಿಗಳು ನಿಗ್ರಹಿಸಲ್ಪಟ್ಟಿರುತ್ತವೆ ಅಷ್ಟೆ; ಆದರೆ ಅವನ್ನು ಆಗತಾನೆ ಎದ್ದಿರುವ ನಿರೋಧ ಸಂಸ್ಕಾರಗಳ ಮೂಲಕ ನಾಶ ಮಾಡಿರುವುದಿಲ್ಲ. ಎರಡನೆಯದರಲ್ಲಿ ವಿಕ್ಷೇಪ ವೃತ್ತಿಗಳು ನಿರೋಧ ಸಂಸ್ಕಾರಗಳಿಂದ ಸಂಪೂರ್ಣವಾಗಿ ನಿಗ್ರಹಿಸಲ್ಪಟ್ಟಿರುತ್ತವೆ. ನಿರೋಧ ಸಂಸ್ಕಾರಗಳು ಸ್ಪಷ್ಟವಾಗಿ ವ್ಯಕ್ತವಾಗುವ ಸ್ಥಿತಿಯಲ್ಲಿರುತ್ತವೆ. ಮೂರನೆಯದೇ ಅತ್ಯುನ್ನತ ಸ್ಥಿತಿ. ಅಲ್ಲಿ ನಿಗ್ರಹದ ಪ್ರಶ್ನೆ ಇಲ್ಲ. ಒಂದೇ ಸ್ವರೂಪದ ಸಂಸ್ಕಾರಗಳು ಒಂದನ್ನೊಂದು ಎಡೆಬಿಡದೆ ಹಿಂಬಾಲಿಸುತ್ತವೆ. (ಸಂಪಾದಕ)}

\vspace{-0.3cm}

\begin{verse}
ಏತೇನ ಭೂತೇಂದ್ರಿಯೇಷು ಧರ್ಮಲಕ್ಷಣಾವಸ್ಥಾ ಪರಿಣಾಮಾ ವ್ಯಾಖ್ಯಾತಾಃ~॥~೧೩~॥
\end{verse}

\vspace{-0.4cm}

\dsize{ಸ್ಥೂಲ ಅಥವಾ ಸೂಕ್ಷ್ಮ ವಸ್ತುಗಳಿಗೆ ಮತ್ತು ಇಂದ್ರಿಯಗಳಿಗೆ ಸಂಬಂಧಿಸಿದಂತೆ ಆಕಾರ, ಕಾಲ ಮತ್ತು ಸ್ಥಿತಿ ಎಂಬ ಮೂರು ವಿಧದ ವಿಕಾರವನ್ನು ಇದರಿಂದ ವಿವರಿಸಿದಂತಾಯಿತು. }

\vskip 0.2cm

ರೂಪ, ಕಾಲ ಮತ್ತು ಸ್ಥಿತಿಗೆ ಸಂಬಂಧಿಸಿದಂತೆ ಚಿತ್ತದಲ್ಲಿ ಆಗುವ ವಿಕಾರಗಳನ್ನು ಸ್ಥೂಲ ಮತ್ತು ಸೂಕ್ಷ್ಮವಸ್ತುಗಳು ಹಾಗೂ ಇಂದ್ರಿಯಗಳ ಪರಿಣಾಮವೆಂದು ವಿವರಿಸಲಾಗಿದೆ. ಒಂದು ಚಿನ್ನದ ಮುದ್ದೆ ಇದೆ ಎಂದಿಟ್ಟುಕೊಳ್ಳಿ. ಅದು ಕಡಗ, ಮತ್ತೆ ಅನಂತರ ಓಲೆಯಾಗಿ ರೂಪಾಂತರಗೊಂಡರೆ ಅದು ರೂಪ ಪರಿಣಾಮ. ಇದನ್ನೇ ಕಾಲದ ದೃಷ್ಟಿಯಿಂದ ನೋಡಿದರೆ ಕಾಲಪರಿಣಾಮವಾಗುತ್ತದೆ. ಮತ್ತೆ ಕಡಗ ಅಥವಾ ಓಲೆ ಹೊಳೆಯುತ್ತಿರಬಹುದು ಅಥವಾ ಮಸಕಾಗಿರಬಹುದು, ದಪ್ಪ ಅಥವಾ ತೆಳ್ಳಗಿರಬಹುದು. ಇದು ಸ್ಥಿತಿಪರಿಣಾಮ. ೯, ೧೧ ಮತ್ತು ೧೨ನೇ ಸೂತ್ರಗಳಲ್ಲಿ ಹೇಳಿರುವಂತೆ ಮನಸ್ಸು ವೃತ್ತಿಯಾಗಿ ವಿಕಾರ ಹೊಂದುತ್ತಿರುವುದು. ಇದೇ ರೂಪ ಪರಿಣಾಮ. ಅದು ಭೂತ ವರ್ತಮಾನ ಭವಿಷ್ಯತ್ತಿನ ಕ್ಷಣಗಳಲ್ಲಿ ಹರಿಯುತ್ತಿರುವುದು. ಇದು ಕಾಲ ಪರಿಣಾಮ. ಯಾವುದೇ ಒಂದು ಸಮಯದಲ್ಲಿ ಉದಾಹರಣೆಗೆ ವರ್ತಮಾನದಲ್ಲಿ, ಸಂಸ್ಕಾರಗಳ ತೀವ್ರತೆಯ ವಿಷಯದಲ್ಲಿ ವ್ಯತ್ಯಾಸವಿರುತ್ತದೆ. ಇದು ಸ್ಥಿತಿ ಪರಿಣಾಮ. ಹಿಂದಿನ ಸೂತ್ರಗಳಲ್ಲಿ ಬೋಧಿಸಿದ ಏಕಾಗ್ರತೆಗಳು ಯೋಗಿಗೆ ಚಿತ್ತದ ಪರಿಣಾಮಗಳು ಮೇಲೆ ಇಚ್ಛಾಪೂರ್ವಕ ನಿಯಂತ್ರಣ ಶಕ್ತಿಯನ್ನು ನೀಡುತ್ತವೆ. ಇದರಿಂದ ಮಾತ್ರ (III, ೪)ರಲ್ಲಿ ಹೇಳಿದ ಸಂಯಮವನ್ನು ಪಡೆಯಲು ಸಾಧ್ಯ. 

\vspace{-0.2cm}

\begin{verse}
ಶಾನ್ತೋದಿತಾವ್ಯಪದೇಶ್ಯಧರ್ಮಾನುಪಾತೀ ಧರ್ಮೀ~॥ ೧೪~॥
\end{verse}

\vspace{-0.4cm}

\dsize{ವರ್ತಮಾನ, ಭೂತ, ಭವಿಷ್ಯತ್​ ಪರಿಣಾಮಗಳಿಂದ ವಿಕಾಸ ಹೊಂದುವುದು ಗುಣಿ. }

\vskip 0.2cm

ಅಂದರೆ ಗುಣಿಯೆ ಕಾಲ ಮತ್ತು ಸಂಸ್ಕಾರಗಳಿಗೆ ಸಿಕ್ಕಿ ವಿಕಾರಹೊಂದಿ ಅನವರತವೂ ವ್ಯಕ್ತಗೊಳ್ಳುತ್ತಿರುವ ದ್ರವ್ಯವಸ್ತು. 

\vspace{-0.2cm}

\begin{verse}
ಕ್ರಮಾನ್ಯತ್ವಂ ಪರಿಣಾಮಾನ್ಯತ್ವೇ ಹೇತುಃ~॥ ೧೫~॥
\end{verse}

\vspace{-0.4cm}

\dsize{ಬಿಡುವಿಲ್ಲದ ಪರಿಣಾಮವೆ ನಾನಾ ವಿಕಾಸಗಳಿಗೆ ಕಾರಣ. }

\vspace{-0.2cm}

\begin{verse}
ಪರಿಣಾಮತ್ರಯಸಂಯಮಾದತೀತಾನಾಗತಜ್ಞಾನಮ್​~॥ ೧೬~॥
\end{verse}

\vspace{-0.4cm}

\dsize{ಮೂರು ವಿಧದ ಪರಿಣಾಮಗಳ ಮೇಲೆ ಸಂಯಮವನ್ನು ಸಾಧಿಸುವುದರಿಂದ ಭೂತ ಮತ್ತು ಭವಿಷ್ಯತ್​ ಜ್ಞಾನ ಬರುವುದು. }

\vskip 0.2cm

ಸಂಯಮದ ಮೊದಲನೆ ವಿವರಣೆಯನ್ನು ನಾವು ಮರೆಯಬಾರದು. ಮನಸ್ಸು ಬಾಹ್ಯ ವಸ್ತುವನ್ನು ಬಿಟ್ಟು, ಅದರ ಆಂತರಿಕ ಸಂಸ್ಕಾರದೊಡನೆ ತಾದಾತ್ಮ್ಯವನ್ನು ತಾಳುವ ಸ್ಥಿತಿಗೆ ಬಂದಮೇಲೆ, ದೀರ್ಘಸಾಧನೆಯಿಂದ ಅದನ್ನು ಉಳಿಸಿಕೊಂಡು, ತಕ್ಷಣದಲ್ಲಿಯೇ ಆ ಸ್ಥಿತಿಗೆ ಬರುವುದಾದರೆ ಅದಕ್ಕೆ ಸಂಯಮವೆಂದು ಹೆಸರು. ಆ ಸ್ಥಿತಿಯಲ್ಲಿರುವ ವ್ಯಕ್ತಿಯು ಭೂತ ಭವಿಷ್ಯತ್ತುಗಳನ್ನು ತಿಳಿಯಬೇಕಾದರೆ ಸಂಸ್ಕಾರಗಳ ಬದಲಾವಣೆಯ ಮೇಲೆ (III, ೧೩) ಸಂಯಮ ಮಾಡಬೇಕು. ಕೆಲವು ಸಂಸ್ಕಾರಗಳು ಈಗ ಕೆಲಸ ಮಾಡುತ್ತಿವೆ, ಮತ್ತೆ ಕೆಲವು ತಮ್ಮ ಕೆಲಸ ಮುಗಿಸಿವೆ, ಇನ್ನು ಕೆಲವು ಮುಂದೆ ಕ್ರಿಯಾಶೀಲವಾಗುತ್ತವೆ. ಇವುಗಳ ಮೇಲೆ ಸಂಯಮ ಮಾಡುವುದರಿಂದ ಭೂತಭವಿಷ್ಯತ್ತನ್ನು ತಿಳಿಯಬಹುದು.

\vspace{-0.25cm}

\begin{verse}
ಶಬ್ಧಾರ್ಥಪ್ರತ್ಯಯಾನಾಮಿತರೇತರಾಧ್ಯಾಸಾತ್​‌ ಸಂಕರಸ್ತತ್ಪ್ರವಿಭಾಗಸಂಯಮಾತ್​ ಸರ್ವಭೂತರುತಜ್ಞಾನಮ್​~॥ ೧೭~॥
\end{verse}

\vspace{-0.4cm}

\dsize{ಸಾಧಾರಣವಾಗಿ ಕದಡಿಹೋಗಿರುವ ಶಬ್ದ ಅರ್ಥ ಜ್ಞಾನದ ಮೇಲೆ ಸಂಯಮ ಮಾಡಿದರೆ, ಎಲ್ಲಾ ಪ್ರಾಣಿಗಳ ಶಬ್ದದ ಅನುಭವವೂ ಬರುವುದು. }

\vskip 0.2cm

ಶಬ್ದವು ಬಾಹ್ಯಕಾರಣವನ್ನು ಪ್ರತಿನಿಧಿಸುತ್ತದೆ. ಅರ್ಥವು ಬಾಹ್ಯ ವಸ್ತುವಿನ ಅನುಭವವನ್ನು ಮಿದುಳಿಗೆ ತಿಳಿಸುವುದಕ್ಕಾಗಿ ಇಂದ್ರಿಯಗಳ ಕಾಲುವೆ ಮೂಲಕ ಪ್ರಯಾಣ ಮಾಡುವ ಆಂತರಿಕ ತರಂಗಗಳನ್ನು ಪ್ರತಿನಿಧಿಸುತ್ತದೆ. ಮನಸ್ಸಿನ ಪ್ರತಿಕ್ರಿಯೆಯು ಜ್ಞಾನವನ್ನು ಪ್ರತಿನಿಧಿಸುತ್ತದೆ. ಇದರಿಂದ ನಮಗೆ ಇಂದ್ರಿಯಗ್ರಹಣವಾಗುವುದು. ಈ ಮೂರರ ಮಿಶ್ರಣವೇ ನಮ್ಮ ವಿಷಯ ವಸ್ತುಗಳು. ನಾನು ಒಂದು ಶಬ್ದವನ್ನು ಕೇಳಿದೆ ಎಂದು ಇಟ್ಟುಕೊಳ್ಳೋಣ. ಮೊದಲು ಹೊರಗಿನ ಸ್ಪಂದನ, ಅನಂತರ ಶ್ರೋತ್ರೇಂದ್ರಿಯದಿಂದ ಮನಸ್ಸಿಗೆ ಒಯ್ಯುವ ಆಂತರಿಕ ಸಂವೇದನೆ, ಅನಂತರ ಮನಸ್ಸಿನಲ್ಲಿ ಪ್ರತಿಕ್ರಿಯೆಯಾಗಿ ನಾನು ಶಬ್ದವನ್ನು ತಿಳಿಯುತ್ತೇನೆ. ನನಗೆ ತಿಳಿದಿರುವ ಶಬ್ದವು ಸ್ಪಂದನ, ಸಂವೇದನೆ ಮತ್ತು ಪ್ರತಿಕ್ರಿಯೆ ಈ ಮೂರರ ಮಿಶ್ರಣ. ಸಾಧಾರಣವಾಗಿ ಈ ಮೂರು ಅವಿಭಾಜ್ಯವಾದುವುಗಳು. ಆದರೆ ಅಭ್ಯಾಸ ಬಲದಿಂದ ಯೋಗಿಯು ಇವನ್ನು ಪ್ರತ್ಯೇಕಿಸಬಹುದು. ಇದನ್ನು ಪಡೆದ ಮೇಲೆ ಅವನು ಯಾವುದಾದರೊಂದು ಶಬ್ದದ ಮೇಲೆ–ಅದು ಮನುಷ್ಯನು ಮಾಡಿದ್ದಾಗಿರಬಹುದು ಅಥವಾ ಪ್ರಾಣಿಗಳು ಮಾಡಿದ್ದಾಗಿರಬಹುದು–ಸಂಯಮ ಮಾಡಿದರೆ ಅದು ಯಾವ ಅರ್ಥವನ್ನು ವ್ಯಕ್ತಗೊಳಿಸುವುದಕ್ಕಾಗಿ ಇದೆಯೊ ಅದನ್ನು ಇವನು ತಿಳಿದುಕೊಳ್ಳುತ್ತಾನೆ. 

\vspace{-0.2cm}

\begin{verse}
ಸಂಸ್ಕಾರಸಾಕ್ಷಾತ್​‌ ಕರಣಾತ್​ ಪೂರ್ವಜಾತಿಜ್ಞಾನಮ್​~॥ ೧೮~॥
\end{verse}

\vspace{-0.4cm}

\dsize{ಸಂಸ್ಕಾರಗಳನ್ನು ಗ್ರಹಿಸುವುದರ ಮೂಲಕ ಪೂರ್ವಜನ್ಮದ ಅನುಭವ ಬರುವುದು. }

\vskip 0.2cm

ನಮಗುಂಟಾಗುವ ಪ್ರತಿಯೊಂದು ಅನುಭವವೂ ಚಿತ್ತವೃತ್ತಿ ರೂಪದಲ್ಲಿ ಬರುತ್ತದೆ. ಇದು ಶಾಂತವಾಗಿ ಅನಂತರ ಸೂಕ್ಷ್ಮ ಸೂಕ್ಷ್ಮವಾಗುತ್ತದೆ. ಆದರೆ ಎಂದಿಗೂ ಇದು ನಾಶವಾಗುವುದಿಲ್ಲ. ಇದು ಅಲ್ಲಿ ಸೂಕ್ಷ್ಮ ಸ್ಥಿತಿಯಲ್ಲಿರುವುದು. ಇದನ್ನು ನಾವು ಪುನಃ ಮೇಲಕ್ಕೆ ತಂದರೆ ಇದು ನೆನಪಾಗುವುದು. ಮನಸ್ಸಿನಲ್ಲಿರುವ ಪೂರ್ವ ಸಂಸ್ಕಾರದ ಮೇಲೆ ಯೋಗಿ ಸಂಯಮ ಮಾಡಿದರೆ, ತನ್ನ ಪೂರ್ವಜನ್ಮವೆಲ್ಲವನ್ನೂ ಜ್ಞಾಪಿಸಿಕೊಳ್ಳಲು ಮೊದಲುಮಾಡುವನು. 

%\newpage

\vspace{-0.3cm}


\begin{verse}
ಪ್ರತ್ಯಯಸ್ಯ ಪರಚಿತ್ತಜ್ಞಾನಮ್​~॥ ೧೯~॥
\end{verse}

\vspace{-0.4cm}

\dsize{ಮತ್ತೊಬ್ಬರ ದೇಹದ ಸಂಜ್ಞೆಗಳ ಮೇಲೆ ಸಂಯಮ ಮಾಡುವುದರಿಂದ ಅವರ ಮನಸ್ಸಿನ ಜ್ಞಾನ\break ಬರುವುದು. }

\vspace{0.1cm}

ಪ್ರತಿಯೊಬ್ಬರನ್ನು ಮತ್ತೊಬ್ಬರಿಂದ ಪ್ರತ್ಯೇಕಿಸಲ್ಪಡುವ ಕೆಲವು ಗುರುತು ಅವರ ದೇಹದ ಮೇಲೆ ಇದೆ ಎಂದು ಇಟ್ಟುಕೊಳ್ಳೋಣ. ಈ ಗುರುತಿನ ಮೇಲೆ ಯೋಗಿ ಸಂಯಮ ಮಾಡಿದರೆ ಮತ್ತೊಬ್ಬರ ಮನಸ್ಸಿನ ಸ್ವಭಾವ ಪರಿಚಯವಾಗುವುದು. 

\vspace{-0.2cm}

\begin{verse}
ನ ಚ ತತ್​ ಸಾಲಂಬನಂ ತಸ್ಯಾವಿಷಯೀಭೂತತ್ವಾತ್​~॥ ೨೦~॥
\end{verse}

\vspace{-0.4cm}

\dsize{ಆದರೆ ಅವರ ಮನಸ್ಸಿನಲ್ಲಿರುವುದು ತಿಳಿಯುವುದಿಲ್ಲ. ಅದು ಸಂಯಮದ ಗುರಿಯಲ್ಲ. }

\vspace{0.2cm}

ದೇಹದ ಮೇಲೆ ಸಂಯಮ ಮಾಡುವುದರಿಂದ ಮನಸ್ಸಿನಲ್ಲಿರುವುದು ತಿಳಿಯುವುದಿಲ್ಲ. ಇದಕ್ಕೆ ಎರಡು ವಿಧದ ಸಂಯಮ ಬೇಕಾಗುವುವು. ಮೊದಲನೆಯದು ದೇಹದ ಗುರುತಿನ ಮೇಲೆ, ಎರಡನೆಯದು ಮನಸ್ಸಿನ ಮೇಲೆ. ಆಗ ಯೋಗಿಗೆ ಇನ್ನೊಬ್ಬರ ಮನಸ್ಸಿನಲ್ಲಿರುವುದೆಲ್ಲ ತಿಳಿಯುವುದು. 

\vspace{-0.3cm}

\begin{verse}
ಕಾಯರೂಪಸಂಯಮಾತ್ತದ್ಗ್ರಾಹ್ಯಶಕ್ತಿ–ಸ್ತಂಭೇ ಚಕ್ಷುಃ\\ಪ್ರಕಾಶಾಸಂಯೋಗೇತರ್ಧಾನಮ್​ \hfill{॥ ೨೧~॥}
\end{verse}

\vspace{-0.4cm}

\dsize{ದೇಹದ ರೂಪದ ಮೇಲೆ ಸಂಯಮ ಮಾಡುವುದರಿಂದ ರೂಪದ ಗ್ರಾಹ್ಯಕತ್ವಕ್ಕೆ ಆತಂಕ ಉಂಟಾಗಿ, ಕಣ್ಣಿನಲ್ಲಿ ವ್ಯಕ್ತವಾಗುವ ಶಕ್ತಿಯನ್ನು ಬೇರ್ಪಡಿಸಿದ ಮೇಲೆ, ಯೋಗಿಯ ದೇಹವು ಕಣ್ಣಿಗೆ ಕಾಣದಂತೆ\break ಆಗುವುದು. }


\vspace{0.2cm}

ಕೋಣೆಯಲ್ಲಿ ನಿಂತಿದ್ದಂತೆಯೇ ಬೇಕಾದರೆ ಯೋಗಿ ಮಾಯವಾಗಬಹುದು. ಅವನು ನಿಜವಾಗಿಯೂ ಮಾಯವಾಗುವುದಿಲ್ಲ. ಆದರೆ ಅವನು ಯಾರಿಗೂ ಕಾಣಿಸುವುದಿಲ್ಲ. ಆಕಾರ ಮತ್ತು ದೇಹ ಬೇರ್ಪಡಿಸಿದಂತೆ ತೋರುತ್ತದೆ. ಆಕಾರ ಮತ್ತು ಅದು ಯಾವ ವಸ್ತುವಿನಿಂದ ಆಗಿದೆಯೋ ಅದನ್ನು ಬೇರ್ಪಡಿಸುವಷ್ಟು ಏಕಾಗ್ರತೆಯನ್ನು ಯೋಗಿ ಪಡೆದಾಗ ಮಾತ್ರ ಇದು ಸಾಧ್ಯವಾಗುವುದು ಎಂಬುದನ್ನು ನೆನಪಿನಲ್ಲಿಡಬೇಕು. ಆಗ ಅದರ ಮೇಲೆ ಸಂಯಮ ಮಾಡುವನು. ಆಕಾರವನ್ನು ನೋಡುವ ಶಕ್ತಿಗೆ ಆತಂಕ ಬರುವುದು. ಏಕೆಂದರೆ ಆಕಾರವನ್ನು ನೋಡುವ ಶಕ್ತಿಯು ಆಕಾರ ಮತ್ತು ಆಕಾರಕ್ಕೆ ಪ್ರಧಾನವಾದ ವಸ್ತು ಇವುಗಳ ಸಂಯೋಗದಿಂದ ಬರುವುದು. 

\vspace{-0.3cm}

\begin{verse}
ಏತೇನ ಶಬ್ದಾದ್ಯನ್ತರ್ಧಾನಮುಕ್ತಮ್​~॥ ೨೨~॥
\end{verse}

\vspace{-0.4cm}

\dsize{ಇದರಿಂದ, ಶಬ್ದದ ಮಾಯ ಅಥವಾ ಕಾಣಿಸದಂತೆ ಇರುವುದು ಎನ್ನುವುದನ್ನು ವಿವರಿಸಿದಂತೆ ಆಯಿತು. }

\vspace{-0.2cm}

\begin{verse}
ಸೋಪಕ್ರಮಂ ನಿರುಪಕ್ರಮಂ ಚ ಕರ್ಮ ತತ್​\\ಸಂಯಮಾದಪರಾನ್ತಜ್ಞಾನಮರಿಷ್ಟೇಭ್ಯೋ ವಾ \hfill{॥ ೨೩~॥}
\end{verse}

\vspace{-0.3cm}

\dsize{ಕರ್ಮದಲ್ಲಿ ಎರಡು ವಿಧ, ಬೇಗ ಫಲಿಸುವುದು ಮತ್ತು ನಿಧಾನವಾಗಿ ಫಲಿಸುವುದು. ಇವುಗಳ ಮೇಲೆ ಅಥವಾ ಅರಿಷ್ಟ ಎನ್ನುವ ಚಿಹ್ನೆಗಳ ಮೇಲೆ ಸಂಯಮ ಮಾಡುವುದರಿಂದ ಯೋಗಿಗೆ ತಾನು ದೇಹದಿಂದ ಬೇರೆಯಾಗುವ ಸರಿಯಾದ ಕಾಲ ಗೊತ್ತಾಗುವುದು. }

\vspace{0.1cm}

ಯೋಗಿ ತನ್ನ ಕರ್ಮಗಳ ಮೇಲೆ–ಈಗ ಫಲಿಸುತ್ತಿರುವ ಸಂಸ್ಕಾರ ಮತ್ತು ಮುಂದೆ ಫಲಿಸಲು ಕಾಯುತ್ತಿರುವ ಸಂಸ್ಕಾರ–ಇವುಗಳ ಮೇಲೆ ಸಂಯಮ ಮಾಡಿದರೆ, ಮುಂದೆ ಫಲಿಸಲು ಸಿದ್ಧವಾಗಿರುವ ಕರ್ಮಗಳ ಮೂಲಕ, ಎಂದು ತನ್ನ ದೇಹ ಬೀಳುವುದು ಎಂಬುದನ್ನು ನಿರ್ದಿಷ್ಟವಾಗಿ ತಿಳಿಯಬಲ್ಲನು. ಎಷ್ಟು ಗಂಟೆಗೆ ಎಷ್ಟು ನಿಮಿಷಕ್ಕೆ ತಾನು ಕಾಲವಾಗುತ್ತೇನೆ ಎಂಬುದನ್ನು ತಿಳಿಯುತ್ತಾನೆ. ಇನ್ನೇನು ಸಾವು ಸಮೀಪಿಸುತ್ತಿರುವುದು ಎಂಬ ಜ್ಞಾನ ಹಿಂದೂಗಳಿಗೆ ಬಹುಮುಖ್ಯವಾದುದು. ಏಕೆಂದರೆ ಸಾಯುವ ಕಾಲದಲ್ಲಿ ಮಾಡುವ ಆಲೋಚನೆ ಮುಂದಿನ ಜನ್ಮವನ್ನು ನಿಶ್ಚಯಿಸುವುದಕ್ಕೆ ಬಹಳ ಮುಖ್ಯವಾದುದು ಎಂದು ಗೀತೆ ಬೋಧಿಸುತ್ತದೆ. 

\vspace{-0.3cm}

\begin{verse}
ಮೈತ್ರ್ಯಾದಿಷು ಬಲಾನಿ~॥ ೨೪~॥
\end{verse}

\vspace{-0.4cm}

\dsize{ಸ್ನೇಹ, ದಯೆ, ಇವುಗಳ ಮೇಲೆ ಸಂಯಮ ಮಾಡುವುದರಿಂದ ಯೋಗಿಯು ಆಯಾ ಗುಣಗಳನ್ನು ಪಡೆದುಕೊಳ್ಳುತ್ತಾನೆ. }

\vspace{-0.2cm}

\begin{verse}
ಬಲೇಷು ಹಸ್ತಿಬಲಾದೀನಿ~॥ ೨೫~॥
\end{verse}

\vspace{-0.4cm}

\dsize{ಆನೆ ಮತ್ತು ಇತರ ವಸ್ತುಗಳ ಬಲದ ಮೇಲೆ ಸಂಯಮ ಮಾಡುವುದರಿಂದ ಅವುಗಳ ಶಕ್ತಿ ಯೋಗಿಗೆ ಬರುವುದು. }

\vspace{0.1cm}

ಈ ಸಂಯಮ ಯೋಗಿಗೆ ಸಿದ್ಧಿಯಾಗಿದ್ದರೆ, ಅವನಿಗೆ ಶಕ್ತಿ ಬೇಕಾದರೆ, ಆನೆಯ ಶಕ್ತಿಯ ಮೇಲೆ ಸಂಯಮ ಮಾಡಿ ಅದರ ಬಲವನ್ನು ಪಡೆಯುತ್ತಾನೆ. ಅನಂತ ಶಕ್ತಿ ಪ್ರತಿಯೊಬ್ಬನ ಸ್ವಾಧೀನದಲ್ಲಿಯೂ ಇರುವುದು. ಯೋಗಿ ಅದನ್ನು ಹೊಂದುವ ರೀತಿಯನ್ನು ಕಂಡುಹಿಡಿದುಕೊಳ್ಳುತ್ತಾನೆ. 

\vspace{-0.2cm}

\begin{verse}
ಪ್ರವೃತ್ತ್ಯಾಲೋಕನ್ಯಾಸಾತ್​ ಸೂಕ್ಷ್ಮ–ವ್ಯವಹಿತ–ವಿಪ್ರಕೃಷ್ಟಜ್ಞಾನಮ್​~॥ ೨೬~॥
\end{verse}

\vspace{-0.4cm}

\dsize{ಪ್ರಕಾಶಮಾನವಾದ ಜ್ಯೋತಿಯ ಮೇಲೆ ಸಂಯಮ ಮಾಡುವುದರಿಂದ ಸೂಕ್ಷ್ಮವಾಗಿರುವುದು, ಅಡಚಣೆಯಿಂದ ಕೂಡಿರುವುದು ಮತ್ತು ಬಹಳ ದೂರದಲ್ಲಿರುವುದು ಇವುಗಳ ಜ್ಞಾನ ಬರುವುದು. }

\vspace{0.1cm}

ಹೃದಯದಲ್ಲಿ ಪ್ರಕಾಶಮಾನವಾದ ಜ್ಯೋತಿಯ ಮೇಲೆ ಸಂಯಮ ಮಾಡುವುದರಿಂದ ಬಹಳ ದೂರದಲ್ಲಿರುವ ವಸ್ತುಗಳನ್ನು ನೋಡುತ್ತಾನೆ. ಬಹಳ ದೂರದಲ್ಲಿ ನಡೆಯುತ್ತಿರುವ ಘಟನೆ, ಪರ್ವತಾದಿ ಅಡಚಣೆಯಿಂದ ಕೂಡಿರುವುದು ಮತ್ತು ಅತಿ ಸೂಕ್ಷ್ಮವಾದ ವಿಷಯಗಳು, ಇವುಗಳನ್ನೆಲ್ಲಾ ನೋಡುತ್ತಾನೆ. 

\vspace{-0.2cm}

\begin{verse}
ಭುವನಜ್ಞಾನಂ ಸೂರ್ಯೇ ಸಂಯಮಾತ್​~॥ ೨೭~॥
\end{verse}

\vspace{-0.4cm}

\dsize{ಸೂರ್ಯನ ಮೇಲೆ ಸಂಯಮ ಮಾಡುವುದರಿಂದ ಪ್ರಪಂಚಜ್ಞಾನ ಬರುವುದು. }

\vspace{-0.2cm}

\begin{verse}
ಚಂದ್ರೇ ತಾರಾವ್ಯೂಹಜ್ಞಾನಮ್​~॥ ೨೮~॥
\end{verse}

\vspace{-0.4cm}

\dsize{ಚಂದ್ರನ ಮೇಲೆ ಸಂಯಮ ಮಾಡುವುದರಿಂದ ತಾರಾವ್ಯೂಹಗಳ ಜ್ಞಾನ ಬರುವುದು. }

\vspace{-0.2cm}

\begin{verse}
ಧ್ರುವೇ ತದ್ಗತಿಜ್ಞಾನಮ್​~॥ ೨೯~॥
\end{verse}

\vspace{-0.4cm}

\dsize{ಧ್ರುವ ನಕ್ಷತ್ರದ ಮೇಲೆ ಸಂಯಮ ಮಾಡುವುದರಿಂದ ತಾರಾಚಲನೆಗಳ ಜ್ಞಾನ ಬರುವುದು. }

\vfill\eject
\vspace{-0.2cm}

\begin{verse}
ನಾಭಿಚಕ್ರೇ ಕಾಯವ್ಯೂಹಜ್ಞಾನಮ್​~॥ ೩೦~॥
\end{verse}

\vspace{-0.4cm}

\dsize{ನಾಭಿಯ ಮೇಲೆ ಸಂಯಮ ಮಾಡುವುದರಿಂದ ದೇಹರಚನೆಯ ಜ್ಞಾನ ಬರುವುದು. }

\vspace{-0.2cm}

\begin{verse}
ಕಂಠಕೂಪೇ ಕ್ಷುತ್ಪಿಪಾಸಾನಿವೃತ್ಥಿಃ~॥ ೩೧~॥
\end{verse}

\vspace{-0.4cm}

\dsize{ಗಂಟಲಿನ ಕೆಳಭಾಗದ ಮೇಲೆ ಸಂಯಮ ಮಾಡುವುದರಿಂದ ಹಸಿವು ತೀರುವುದು. }

\vspace{0.1cm}

ಒಬ್ಬನಿಗೆ ಬಹಳ ಹಸಿವಾಗಿದ್ದರೆ ಗಂಟಲಿನ ಕೆಳಭಾಗದ ಮೇಲೆ ಸಂಯಮ ಮಾಡಿದರೆ ಹಸಿವು ನಿಲ್ಲುವುದು. 

\vspace{-0.2cm}

\begin{verse}
ಕೂರ್ಮನಾಡ್ಯಾಂ ಸ್ಥೈರ್ಯಮ್ ~॥ ೩೨~॥
\end{verse}

\vspace{-0.4cm}

\dsize{ಕೂರ್ಮನಾಡಿಯ ಮೇಲೆ ಸಂಯಮ ಮಾಡುವುದರಿಂದ ದೇಹದ ಸ್ಥಿರತೆ ಬರುವುದು. }

\vspace{0.1cm}

ಅವನು ಸಾಧನೆ ಮಾಡುತ್ತಿರುವಾಗ ದೇಹಕ್ಕೆ ತೊಂದರೆಯಾಗುವುದಿಲ್ಲ. 

\vspace{-0.2cm}

\begin{verse}
ಮೂರ್ಧಜ್ಯೋತಿಷಿ ಸಿದ್ಧದರ್ಶನಮ್​~॥ ೩೩~॥
\end{verse}

\vspace{-0.4cm}

\dsize{ತಲೆಯ ಮೇಲಿನಿಂದ ಬರುತ್ತಿರುವ ಜ್ಯೋತಿಯ ಮೇಲೆ ಸಂಯಮ ಮಾಡುವುದರಿಂದ ಸಿದ್ಧದರ್ಶನವಾಗುವುದು. }

\vspace{0.1cm}

ಸಿದ್ಧರು ಪ್ರೇತಗಳಿಗಿಂತ ಸ್ವಲ್ಪ ಮೇಲೆ ಇರುವರು. ಯೋಗಿ ತನ್ನ ಮನಸ್ಸನ್ನು ತಲೆಯ ಮೇಲೆ ಏಕಾಗ್ರ ಮಾಡಿದಾಗ ಸಿದ್ಧರನ್ನು ನೋಡುವನು. ಸಿದ್ಧರು ಎಂಬ ಪದ ಮುಕ್ತರಿಗೆ ಅನ್ವಯಿಸುವುದಿಲ್ಲ. ಅನೇಕ ವೇಳೆ ಇದನ್ನು ಮುಕ್ತರು ಎಂಬ ಅರ್ಥದಲ್ಲಿ ಉಪಯೋಗಿಸಲಾಗಿದೆ. 

\vspace{-0.2cm}

\begin{verse}
ಪ್ರಾತಿಭಾದ್ವಾ ಸರ್ವಮ್​~॥ ೩೪~॥
\end{verse}

\vspace{-0.4cm}

\dsize{ಪ್ರಾತಿಭೆಯಿಂದ ಜ್ಞಾನಗಳೆಲ್ಲಾ ಬರುವುವು. }

\vspace{0.1cm}

ಯಾರಿಗೆ ಪ್ರಾತಿಭೆ ಅಂದರೆ ಸ್ವಭಾವತಃ ಪರಿಶುದ್ಧತೆಯಿಂದ ಜ್ಞಾನಲಾಭವಾಗಿರುವುದೋ ಆತನು ಯಾವ ಸಂಯಮಗಳನ್ನೂ ಮಾಡದೆ ಎಲ್ಲಾ ಜ್ಞಾನಗಳನ್ನೂ ಪಡೆಯುವನು. ಉತ್ತಮ ಪ್ರಾತಿಭೆಯ ಸ್ಥಿತಿಗೆ ಮನುಷ್ಯನು ಹೋದ ಮೇಲೆ ಅವನಿಗೆ ಇಂತಹ ಜ್ಞಾನ ಲಭಿಸುವುದು. ಅವನಿಗೆ ಎಲ್ಲವೂ ತೋರಿಕೆಯಾಗಿದೆ. ಯಾವ ಸಂಯಮವನ್ನೂ ಮಾಡದೆ ಎಲ್ಲವೂ ಅವನಿಗೆ ಸ್ವಭಾವತಃ ಬರುವುದು. 

\vspace{-0.2cm}

\begin{verse}
ಹೃದಯೇ ಚಿತ್ತ–ಸಂವಿತ್​~॥ ೩೫~॥
\end{verse}

\vspace{-0.4cm}

\dsize{ಹೃದಯದಲ್ಲಿ ಸಂಯಮಮಾಡಿದರೆ ಮನಸ್ಸಿನ ಜ್ಞಾನ ಬರುವುದು. }

\vspace{-0.2cm}

\begin{verse}
ಸತ್ತ್ವಪುರುಷಯೋರತ್ಯಂತಾಸಂಕೀರ್ಣಯೋಃ~ ಪ್ರತ್ಯಯಾವಿಶೇಷಾದ್​ ಭೋಗಃ ಪರಾರ್ಥತ್ವಾತ್​ ಸ್ವಾರ್ಥಸಂಯಮಾತ್​ ಪುರುಷಜ್ಞಾನಮ್ \hfill{॥ ೩೬~॥}
\end{verse}

\vspace{-0.4cm}

\dsize{ಆತ್ಮ ಮತ್ತು ಸತ್ತ್ವ–ಇವು ಬೇರೆ ಎಂದು ವಿಮರ್ಶಿಸದೆ ಇರುವುದರಿಂದ ಭೋಗ ಪ್ರಾಪ್ತವಾಗುವುದು. ಯಾರ ಕೆಲಸ ಮತ್ತೊಬ್ಬನ ಭೋಗಕ್ಕಾಗಿ ಇದೆಯೊ ಆ ಸತ್ತ್ವವು ಆತ್ಮನಿಗಿಂತ ಅತ್ಯಂತ ಬೇರೆಯಾದುದು. ಸ್ವಾರ್ಥದ ಮೇಲೆ ಸಂಯಮ ಮಾಡಿದರೆ ಪುರುಷಜ್ಞಾನಲಾಭವಾಗುವುದು. }

\vfill\eject

ಜ್ಯೋತಿ ಮತ್ತು ಆನಂದದಿಂದ ಪೂರ್ಣವಾದ ಸತ್ತ್ವಕ್ರಿಯೆಗಳೆಲ್ಲ ಪ್ರಕೃತಿಯ ವಿಕಾಸಗಳು. ಇವುಗಳಿರುವುದು ಜೀವನಿಗಾಗಿ. ಸತ್ತ್ವಗುಣವು ಅಹಂಕಾರದಿಂದ ಮುಕ್ತವಾಗಿ, ಪುರುಷನ ಶುದ್ಧ ಜ್ಞಾನದಿಂದ ಪ್ರಕಾಶಮಾನವಾಗಿರುವಾಗ ಅದಕ್ಕೆ ಸ್ವಕೇಂದ್ರೀಕೃತ ಸ್ಥಿತಿ ಎಂದು ಹೆಸರು. ಏಕೆಂದರೆ ಆ ಸ್ಥಿತಿ ಎಲ್ಲಾ ಸಂಬಂಧಗಳಿಂದಲೂ ಮುಕ್ತವಾಗಿರುವುದು. 


\begin{verse}
ತತಃ ಪ್ರಾತಿಭಶ್ರಾವಣವೇದನಾದರ್ಶಾಸ್ವಾದವಾರ್ತಾ ಜಾಯನ್ತೇ~॥ ೩೭~॥
\end{verse}

\vspace{-0.3cm}

\dsize{ಅದರಿಂದ, ಪ್ರತಿಭೆಗೆ ಸೇರಿದ ಜ್ಞಾನ ಮತ್ತು ಕೇಳುವುದು, ಮುಟ್ಟುವುದು, ನೋಡುವುದು, ರುಚಿ ನೋಡುವುದು, ಮೂಸಿ ನೋಡುವುದು ಮುಂತಾದ ಅತೀಂದ್ರಿಯ ಜ್ಞಾನಗಳು ಪ್ರಾಪ್ತವಾಗುತ್ತವೆ. }



\begin{verse}
ತೇ ಸಮಾಧಾವುಪಸರ್ಗಾ ವ್ಯುತ್ಥಾನೇ ಸಿದ್ಧಯಃ~॥ ೩೮~॥
\end{verse}

\vspace{-0.3cm}

\dsize{ಇವುಗಳೆಲ್ಲ ಸಮಾಧಿಗೆ ಅಡಚಣೆಗಳು. ಆದರೆ ಪ್ರಾಪಂಚಿಕ ದೃಷ್ಟಿಯಿಂದ ಇವುಗಳೆಲ್ಲ ಸಿದ್ಧಿಗಳು. }

\vspace{0.2cm}

ಯೋಗಿಗೆ ಪ್ರಪಂಚದ ಭೋಗಜ್ಞಾನವು ಪುರುಷ ಮತ್ತು ಮನಸ್ಸಿನ ಸಂಯೋಗದಿಂದ ಬರುವುದು. ಪ್ರಕೃತಿ ಮತ್ತು ಆತ್ಮ ಬೇರೆ ಎಂಬ ಜ್ಞಾನದ ಮೇಲೆ ಸಂಯಮ ಮಾಡಿದರೆ, ಪುರುಷಜ್ಞಾನ ಲಾಭವಾಗುವುದು. ಇದರಿಂದ ವಿವೇಕ ಬರುವುದು. ವಿವೇಕ ಬಂದಮೇಲೆ ಪರಮ ಪವಿತ್ರವಾದ ಜ್ಯೋತಿಯ ಪ್ರತಿಭೆ ಬರುವುದು. ಶುದ್ಧಾತ್ಮನ ಜ್ಞಾನ, ಸ್ವಾತಂತ್ರ್ಯ ಇವುಗಳನ್ನೊಳಗೊಂಡ ಜೀವನದ ಪರಮ ಧ್ಯೇಯವನ್ನು ಮುಟ್ಟಲು ಇವು ಆತಂಕಗಳು. ಇವು ದಾರಿಯ ಮಧ್ಯದಲ್ಲಿ ಸಿಕ್ಕಿದಾಗ ಇವುಗಳನ್ನು ಯೋಗಿ ನಿರಾಕರಿಸಿದರೆ ಜೀವನದ ಪರಮ ಗುರಿಯನ್ನು ಸೇರುವನು. ಇವುಗಳ ಪ್ರಲೋಭನಕ್ಕೆ ಸಿಕ್ಕಿದರೆ ಅವನು ಮುಂದುವರಿಯುವುದು ಕೊನೆಗೊಂಡಂತೆಯೆ. 


\begin{verse}
ಬಂಧಕಾರಣಶೈಥಿಲ್ಯಾತ್​ ಪ್ರಚಾರಸಂವೇದನಾಚ್ಛ ಚಿತ್ತಸ್ಯ ಪರಶರೀರಾವೇಶಃ\\\hfill ॥ ೩೯ ॥
\end{verse}

\vspace{-0.3cm}

\dsize{ಚಿತ್ತದ ಬಂಧಕಾರಣವು ಸಡಿಲವಾದ ಮೇಲೆ, ಯೋಗಿಯು ಅದರ ಕರಣಗಳಾದ ಇಂದ್ರಿಯ ಕ್ರಿಯೆಯ ಜ್ಞಾನದ ಸಹಾಯದಿಂದ ಪರಕಾಯ ಪ್ರವೇಶ ಮಾಡುವನು. }

\vspace{0.1cm}

ಯೋಗಿಯು ತಾನೆ ಒಂದು ದೇಹದಲ್ಲಿ ಕೆಲಸ ಮಾಡುತ್ತಿರುವಾಗ, ಮತ್ತೊಂದು ಹೆಣವನ್ನು ಪ್ರವೇಶಮಾಡಿ, ಅದು ಎದ್ದುನಿಂತು ಚಲಿಸುವಂತೆ ಮಾಡಬಹುದು. ಅಥವಾ ಬದುಕಿರುವ ದೇಹವನ್ನು ಪ್ರವೇಶಮಾಡಿ, ಅವನ ಇಂದ್ರಿಯ ಮತ್ತು ಮನಸ್ಸುಗಳನ್ನು ತಡೆದು ತಾತ್ಕಾಲಿಕವಾಗಿ ಆ ದೇಹದ ಮೂಲಕ ಕೆಲಸಮಾಡಬಲ್ಲನು. ಪ್ರಕೃತಿ ಮತ್ತು ಪುರುಷರ ಜ್ಞಾನವನ್ನು ಪಡೆದ ಯೋಗಿ ಇದನ್ನು ಮಾಡುವನು. ಅವನು ಮತ್ತೊಂದು ದೇಹವನ್ನು ಪ್ರವೇಶಿಸಬೇಕಾದರೆ ಆ ದೇಹದ ಮೇಲೆ ಸಂಯಮ ಮಾಡಿ ಅದನ್ನು ಪ್ರವೇಶಿಸುತ್ತಾನೆ. ಏಕೆಂದರೆ ಯೋಗಿಯು ಹೇಳುವಂತೆ ಅವನ ಆತ್ಮ ಮಾತ್ರ ಸರ್ವವ್ಯಾಪಿಯಲ್ಲ, ಮನಸ್ಸೂ ಕೂಡ ಸರ್ವವ್ಯಾಪಿಯಾಗಿರುವುದು. ಇದು ವಿಶ್ವಮನಸ್ಸಿನ ಒಂದು ಚೂರು ಮಾತ್ರ. ಆದರೆ ಸದ್ಯಕ್ಕೆ ಅದು ದೇಹದಲ್ಲಿರುವ ನರಗಳ ಶಕ್ತಿಯ ಮೂಲಕ ಕೆಲಸ ಮಾಡಬಲ್ಲದು. ಆದರೆ ನರಗಳಿಂದ ಪಾರಾದ ಮೇಲೆ ಬೇರೆ ವಸ್ತುಗಳ ಮೂಲಕ ಅವನು ಕೆಲಸ ಮಾಡಬಲ್ಲನು. 


\newpage

\begin{verse}
ಉದಾನಜಯಾಜ್ಜಲಪಂಕಕಂಟಕಾದಿಷ್ವಸಂಗ ಉತ್ಕ್ರಾಂತಿಶ್ಚ~॥ ೪೦~॥
\end{verse}

\vspace{-0.4cm}

\dsize{ಉದಾನವೆಂಬ ನಾಡಿಯನ್ನು ಜಯಿಸಿದ ಮೇಲೆ ಯೋಗಿಯು ನೀರಿನಲ್ಲಿ ಅಥವಾ ಕೆಸರಿನಲ್ಲಿ ಮುಳುಗುವುದಿಲ್ಲ. ಮುಳ್ಳಿನ ಮೇಲೆ ನಡೆಯಬಲ್ಲ. ಇಚ್ಛೆ ಬಂದಾಗ ಸಾಯಬಲ್ಲ. }

\vspace{0.1cm}

ಶ್ವಾಸಕೋಶಗಳು ಮತ್ತು ದೇಹದ ಮೇಲಿನ ಭಾಗಕ್ಕೆ ಸಂಬಂಧಪಟ್ಟ ಚಲನೆಗಳನ್ನು ತನ್ನ ಸ್ವಾಧೀನದಲ್ಲಿಟ್ಟುಕೊಂಡಿರುವುದು ಉದಾನ. ಇದನ್ನು ನಿಗ್ರಹಿಸಿದ ಮೇಲೆ ಅವನು ಹಗುರವಾಗುವನು. ಅವನು ನೀರಿನಲ್ಲಿ ಮುಳುಗುವುದಿಲ್ಲ. ಕತ್ತಿಯ ಅಲಗಿನ ಮೇಲೆ ಮತ್ತು ಮುಳ್ಳಿನ ಮೇಲೆ ನಡೆಯಬಲ್ಲ. ಬೆಂಕಿಯಲ್ಲಿ ನಿಲ್ಲಬಲ್ಲ. ತನ್ನ ಇಚ್ಛೆ ಬಂದಾಗ ಈ ದೇಹವನ್ನು ಬಿಡಬಲ್ಲ. 

\vspace{-0.3cm}

\begin{verse}
ಸಮಾನಜಯಾತ್​ ಪ್ರಜ್ವಲನಮ್​~॥ ೪೧~॥
\end{verse}

\vspace{-0.4cm}

\dsize{ಸಮಾನದ ನಾಡಿಯನ್ನು ಜಯಿಸಿದಮೇಲೆ ಅವನು ಒಂದು ಜ್ಯೋತಿಯಿಂದ ಆವೃತ್ತನಾಗುವನು. }

\vspace{0.1cm}

ಅವನು ಇಚ್ಛೆಪಟ್ಟಾಗ ಜ್ಯೋತಿಯು ಅವನಿಂದ ಹೊರಹೊಮ್ಮುವುದು. 

\vspace{-0.2cm}

\begin{verse}
ಶ್ರೋತ್ರಾಕಾಶಯೋಃ ಸಂಬಂಧಸಂಯಮಾದ್ದಿವ್ಯಂ ಶ್ರೋತ್ರಮ್​~॥ ೪೨~॥
\end{verse}

\vspace{-0.4cm}

\dsize{ಆಕಾಶ ಮತ್ತು ಕಿವಿಯ ಸಂಬಂಧದ ಮೇಲೆ ಸಂಯಮ ಮಾಡುವುದರಿಂದ ಅತೀಂದ್ರಿಯ ಶಬ್ದಗಳು ಕೇಳುವುವು. }

\vspace{0.1cm}

ಆಕಾಶವಿದೆ, ಶಬ್ದವನ್ನು ಕೇಳುವುದಕ್ಕೆ ಉಪಕರಣವಾದ ಕಿವಿ ಇದೆ. ಇವುಗಳ ಮೇಲೆ ಸಂಯಮ ಮಾಡಿದರೆ ಯೋಗಿಯ ಕಿವಿಯು ಅತೀಂದ್ರಿಯವಾಗುವುದು. ಅವನು ಎಲ್ಲವನ್ನೂ ಕೇಳುತ್ತಾನೆ. ಕೆಲವು ಮೈಲಿಗಳಾಚೆ ಏನು ಮಾತನಾಡಿದರೂ, ಶಬ್ದ ಮಾಡಿದರೂ ಅದನ್ನು ಕೇಳಬಲ್ಲನು. 

\vspace{-0.3cm}

\begin{verse}
ಕಾಯಾಕಾಶಯೋಃ ಸಂಬಂಧಸಂಯಮಾಲ್ಲಘುತೂಲ\\ ಸಮಾಪತ್ತೇಶ್ಚಾಕಾಶಗಮನಮ್​~ \hfill{॥ ೪೩~॥}
\end{verse}

\vspace{-0.4cm}

\dsize{ಆಕಾಶ ಮತ್ತು ದೇಹದ ಸಂಬಂಧದ ಮೇಲೆ ಸಂಯಮ ಮಾಡುವುದರಿಂದ ಯೋಗಿಯು ಹತ್ತಿಯಂತೆ ಹಗುರವಾಗಿ ಆಕಾಶಕ್ಕೆ ಹಾರಬಲ್ಲನು. }

\vspace{0.1cm}

ಆಕಾಶವೇ ಈ ದೇಹದ ವಸ್ತು. ಆಕಾಶವೇ ಮತ್ತೊಂದು ರೀತಿಯಲ್ಲಿ ದೇಹವಾಗಿರುವುದು. ಯೋಗಿಯು ತನ್ನ ದೇಹದ ಆಕಾಶ ವಸ್ತುವಿನ ಮೇಲೆ ಸಂಯಮ ಮಾಡುವುದರಿಂದ, ದೇಹ ಆಕಾಶದಷ್ಟು ಹಗುರವಾಗಿ ಗಾಳಿಯ ಮೂಲಕ ಅವನು ಎಲ್ಲಿಗೆ ಬೇಕಾದರೂ ಹೋಗಬಲ್ಲನು. 

\vspace{-0.3cm}

\begin{verse}
ಬಹಿರಕಲ್ಪಿತಾ ವೃತ್ತಿರ್ಮಹಾವಿದೇಹಾ ತತಃ ಪ್ರಕಾಶಾವರಣಕ್ಷಯಃ~॥ ೪೪~॥
\end{verse}

\vspace{-0.4cm}

\dsize{ದೇಹದ ಹೊರಗೆ ಮಹಾವಿದೇಹವೆಂಬ ನಿಜವಾದ ಮಾನಸಿಕ ವೃತ್ತಿಯ ಮೇಲೆ ಸಂಯಮ ಮಾಡುವುದರಿಂದ ಜ್ಯೋತಿಯನ್ನು ಆವರಿಸಿರುವ ತೆರೆ ಮಾಯವಾಗುವುದು. }

\vspace{0.1cm}

ಮನಸ್ಸು ಮೌಢ್ಯದಿಂದ ತಾನು ಈ ದೇಹದಲ್ಲಿ ಕೆಲಸಮಾಡುತ್ತಿರುವೆನು ಎಂದು ತಿಳಿಯುವುದು. ಮನಸ್ಸು ಸರ್ವವ್ಯಾಪಿಯಾಗಿದ್ದರೆ, ನಾನೇಕೆ ಒಂದು ವಿಧದ ನರಗಳ ರಚನೆ ಯಲ್ಲಿ ಬದ್ಧನಾಗಿರಬೇಕು? ನನ್ನ ಅಹಂಕಾರವನ್ನು ಏಕೆ ಒಂದೇ ದೇಹದಲ್ಲಿ ಇಟ್ಟಿರಬೇಕು? ನಾನು ಏತಕ್ಕೆ ಹೀಗೆ ಇರಬೇಕು ಎನ್ನುವುದಕ್ಕೆ ಕಾರಣವಿಲ್ಲ. ಯೋಗಿಯು ತನಗೆ ಇಚ್ಛೆ ಬಂದೆಡೆಯಲ್ಲಿ ತನ್ನ ಅಹಂಕಾರವನ್ನು ಇಡಲು ಯತ್ನಿಸುತ್ತಾನೆ. ದೇಹದಲ್ಲಿ ಅಹಂಕಾರವಿಲ್ಲದೆ ಇರುವಾಗ ಏಳುವ ಅಲೆಗಳೆ, “ನಿಜವಾದ ವೃತ್ತಿಗಳು” ಅಥವಾ ಮಹಾವಿದೇಹ. ಈ ವೃತ್ತಿಗಳ ಮೇಲೆ ಸಂಯಮ ಮಾಡುವುದರಲ್ಲಿ ಜಯಶೀಲನಾದ ಮೇಲೆ ಜ್ಯೋತಿಯನ್ನು ಆವರಿಸಿದ ತೆರೆಯೆಲ್ಲ ಹೋಗುವುದು. ಎಲ್ಲಾ ವಿಧದ ಅಜ್ಞಾನ ಮತ್ತು ಅಂಧಕಾರ ಮಾಯವಾಗುವುವು. ಎಲ್ಲವೂ ಅವನಿಗೆ ಜ್ಞಾನದಿಂದ ಪೂರ್ಣವಾದಂತೆ ತೋರುವುದು. 

\vspace{-0.25cm}

\begin{verse}
ಸ್ಥೂಲ–ಸ್ವರೂಪ–ಸೂಕ್ಷ್ಮಾನ್ವಯಾರ್ಥವತ್ತ್ವಸಂಯಮಾದ್​ ಭೂತಜಯಃ~॥ ೪೫~॥
\end{verse}

\vspace{-0.4cm}

\dsize{ಸ್ಥೂಲ ಮತ್ತು ಸೂಕ್ಷ್ಮಭೂತಗಳು, ಅವುಗಳ ಪ್ರಧಾನ ಲಕ್ಷಣಗಳು, ಅವುಗಳಲ್ಲಿರುವ ಗುಣಗಳು ಮತ್ತು ಜೀವನ ಅನುಭವಕ್ಕೆ ಅವುಗಳ ಕೊಡುಗೆ–ಇವುಗಳ ಮೇಲೆ ಸಂಯಮ ಮಾಡಿದರೆ ಭೂತಜಯ ಸಿದ್ಧಿಸುವುದು. }

\vspace{0.1cm}

ಯೋಗಿಯು ಮೊದಲು ಭೂತದ ಸ್ಥೂಲಸ್ಥಿತಿ, ಅನಂತರ ಸೂಕ್ಷ್ಮಸ್ಥಿತಿಯ ಮೇಲೆ ಸಂಯಮ ಮಾಡುವನು. ಸಾಧಾರಣವಾಗಿ ಬೌದ್ಧರಲ್ಲಿ ಒಂದು ಪಂಗಡದವರು ಇದನ್ನು ಮಾಡುತ್ತಾರೆ. ಒಂದು ಚೂರು ಜೇಡಿಮಣ್ಣಿನ ಮೇಲೆ ಸಂಯಮ ಮಾಡುತ್ತಾರೆ, ಕ್ರಮೇಣ ಅದು ಯಾವ ಸೂಕ್ಷ್ಮವಸ್ತುವಿನಿಂದ ಆಗಿದೆಯೋ ಅದನ್ನು ನೋಡುತ್ತಾರೆ. ಅದರಲ್ಲಿರುವ ಎಲ್ಲಾ ಸೂಕ್ಷ್ಮವಸ್ತುಗಳನ್ನು ನೋಡಿದ ಮೇಲೆ ಅದು ಅವನ ಸ್ವಾಧೀನವಾಗುವುದು. ಇದರಂತೆಯೇ ಎಲ್ಲಾ ಭೂತಗಳೂ ಕೂಡ. ಯೋಗಿ ಇವುಗಳೆಲ್ಲವನ್ನೂ ಜಯಿಸಬಲ್ಲ. 

\vspace{-0.25cm}

\begin{verse}
ತತೋಽಣಿಮಾದಿಪ್ರಾದುರ್ಭಾವಃ ಕಾಯಸಮ್ಪತ್ತದ್ಧರ್ಮಾನಭಿಘಾತಶ್ಚ~॥ ೪೬~॥
\end{verse}

\vspace{-0.4cm}

\dsize{ಅದರಿಂದ ಅಣಿಮಾದಿ ಅಷ್ಟಸಿದ್ಧಿಗಳು, ದೇಹದ ಮಹತ್ವ, ದೈಹಿಕ ಗುಣಗಳ ಅವಿನಾಶತ್ವ ಮುಂತಾದುವು ಬರುವುವು. }

\vspace{0.1cm}

ಇದರಿಂದ ಯೋಗಿಯು ಅಷ್ಟಸಿದ್ಧಿಗಳನ್ನು ಪಡೆಯುವನು ಎಂದು ಅರ್ಥ. ಅವನು ಕಣದಷ್ಟು ಸಣ್ಣದಾಗಬಲ್ಲ, ಅಥವಾ ಮೇರುವಷ್ಟು ಎತ್ತರವಾಗಬಲ್ಲ, ಭೂಮಿಯಷ್ಟು ಭಾರವಾಗಬಲ್ಲ ಅಥವಾ ಗಾಳಿಯಷ್ಟು ಹಗುರವಾಗಬಲ್ಲ. ತನಗೆ ಬೇಕಾದುದನ್ನು ಜಯಿಸಬಲ್ಲ. ಸಿಂಹವು ಕುರಿಮರಿಯಂತೆ ಅವನ ಪಾದತಳದಲ್ಲಿ ಕುಳಿತುಕೊಳ್ಳುವುದು. ಇಚ್ಛಾಮಾತ್ರದಿಂದ ಅವನ ಬಯಕೆಗಳೆಲ್ಲ ಈಡೇರುವುವು. 

\vspace{-0.25cm}

\begin{verse}
ರೂಪಲಾವಣ್ಯ–ಬಲ–ವಜ್ರಸಂಹನನತ್ವಾನಿ ಕಾಯಸಂಪತ್​~॥ ೪೭~॥
\end{verse}

\vspace{-0.4cm}

\dsize{ರೂಪ, ಲಾವಣ್ಯ, ಬಲ, ವಜ್ರೋಪಮ ಕಠಿಣತೆ–ಇವು ಕಾಯಸಂಪತ್ತುಗಳು. }

\vspace{0.1cm}

ದೇಹ ಅವಿನಾಶಿಯಾಗುವುದು. ಅದಕ್ಕೆ ಯಾವುದೂ ಹಿಂಸೆಯನ್ನು ಕೊಡಲಾರದು. ಯೋಗಿಯು ಇಚ್ಛಿಸುವತನಕ ಯಾವುದೂ ಅದನ್ನು ನಾಶಮಾಡಲಾರದು. “ಕಾಲವೆಂಬ ದಂಡವನ್ನು ಭಂಗಿಸಿ ಈ ದೇಹದಲ್ಲಿ ವಾಸಿಸುತ್ತಾನೆ. ಅಂತಹ ಯೋಗಿಗೆ ರೋಗ, ಸಾವು, ವ್ಯಥೆಗಳು ಇನ್ನು ಇಲ್ಲ” ಎಂದು ವೇದದಲ್ಲಿ ಸಾರಿದೆ. 

\vspace{-0.2cm}

\begin{verse}
ಗ್ರಹಣಸ್ವರೂಪಾಸ್ಮಿತಾನ್ವಯಾರ್ಥವತ್ತ್ವ ಸಂಯಮಾದಿಂದ್ರಿಯಜಯಃ~॥ ೪೮~॥
\end{verse}

\vspace{-0.4cm}

\dsize{ವಸ್ತುತ್ವ, ವಿಷಯಗಳನ್ನು ಬೆಳಗುವ ಇಂದ್ರಿಯಗಳ ಶಕ್ತಿ, ಅಹಂಕಾರ ಮತ್ತು ಅವುಗಳಲ್ಲಿರುವ ಗುಣಗಳು ಹಾಗೂ ಅವುಗಳ ಅರ್ಥವತ್ವದ ಮೇಲೆ ಕ್ರಮೇಣ ಸಂಯಮ ಮಾಡುವುದರಿಂದ ಇಂದ್ರಿಯನಿಗ್ರಹ ಲಭಿಸುವುದು. }

\vspace{0.1cm}

ಬಾಹ್ಯವಸ್ತುವನ್ನು ಗ್ರಹಣಮಾಡುವಾಗ ಇಂದ್ರಿಯಗಳು ಮನಸ್ಸನ್ನು ಬಿಟ್ಟು ವಸ್ತುವಿನ ಕಡೆಗೆ ಹೋಗುವುವು. ಅನಂತರ ಜ್ಞಾನ ಉಂಟಾಗುವುದು. ಈ ಕ್ರಿಯೆಯ ಹಿಂದೆ ಅಹಂಕಾರ ಇದ್ದೇ ಇರುತ್ತದೆ. ಕ್ರಮೇಣ ಇವುಗಳ ಮೇಲೆ ಸಂಯಮ ಮಾಡುವ ಯೋಗಿಯು ಇಂದ್ರಿಯ ನಿಗ್ರಹವನ್ನು ಪಡೆಯುವನು. ನಿಮಗೆ ಕಾಣುವ ಅಥವಾ ತಿಳಿಯುವ ಯಾವುದಾದರೊಂದು ವಸ್ತುವನ್ನು ತೆಗೆದುಕೊಳ್ಳಿ:ಉದಾಹರಣೆಗೆ ಒಂದು ಪುಸ್ತಕ. ಮೊದಲು ಮನಸ್ಸನ್ನು ಅದರ ಮೇಲೆ ಏಕಾಗ್ರಗೊಳಿಸಿ; ಅನಂತರ ಪುಸ್ತಕದ ರೂಪದಲ್ಲಿರುವ ಜ್ಞಾನದ ಮೇಲೆ; ಅನಂತರ ಆ ಪುಸ್ತಕವನ್ನು ನೋಡುವ ಅಹಂಕಾರದ ಮೇಲೆ ಮನಸ್ಸನ್ನು ಏಕಾಗ್ರಗೊಳಿಸಿ. ಈ ಅಭ್ಯಾಸದಿಂದ ಇಂದ್ರಿಯಗಳೆಲ್ಲಾ ಜಯಿಸಲ್ಪಡುವುವು. 

\vspace{-0.2cm}

\begin{verse}
ತತೋ ಮನೋಜವಿತ್ವಂ ವಿಕರಣಭಾವಃ ಪ್ರಧಾನಜಯಶ್ಚ~॥ ೪೯~॥
\end{verse}

\vspace{-0.4cm}

\dsize{ಇದರಿಂದ ದೇಹಕ್ಕೆ ಮನಸ್ಸಿನಂತೆ ವೇಗವಾದ ಚಲನೆ, ದೇಹದಿಂದ ಸ್ವತಂತ್ರವಾದ ಇಂದ್ರಿಯಶಕ್ತಿ, ಪ್ರಕೃತಿಜಯ–ಇವುಗಳೆಲ್ಲ ಬರುವುವು. }

\vspace{0.1cm}

ಭೂತಜಯದಿಂದ ಕಾಯ ಸಂಪತ್ತು ಬರುವಂತೆ, ಇಂದ್ರಿಯಗಳ ಜಯದಿಂದ ಮೇಲೆ ಹೇಳಿದ ಶಕ್ತಿಗಳು ಬರುವುವು. 

\vspace{-0.3cm}

\begin{verse}
ಸತ್ತ್ವಪುರುಷಾನ್ಯತಾಖ್ಯಾತಿಮಾತ್ರಸ್ಯ ಸರ್ವಭಾವಾಧಿಷ್ಠಾತೃತ್ವಂ\\ ಸರ್ವಜ್ಞಾತೃತ್ವಂ ಚ~ \hfill{॥~೫೦~॥}
\end{verse}

\vspace{-0.4cm}

\dsize{ಸತ್ತ್ವ ಮತ್ತು ಪುರುಷ ವಿವೇಕ ಖ್ಯಾತಿಯ ಮೇಲೆ ಸಂಯಮ ಮಾಡಿದರೆ ಸರ್ವ ಶಕ್ತಿತ್ವ ಮತ್ತು ಸರ್ವಜ್ಞತ್ವಗಳು ಲಭಿಸುವುವು. }

\vspace{0.1cm}

ಪ್ರಕೃತಿಯನ್ನು ಗೆದ್ದಮೇಲೆ, ಪುರುಷ ಮತ್ತು ಪ್ರಕೃತಿಗೆ ಇರುವ ಭೇದವನ್ನು ತಿಳಿದು–ಪುರುಷನು ಅವಿನಾಶಿ, ಶುದ್ಧನು, ಪೂರ್ಣನು ಎಂದು ತಿಳಿದ ಮೇಲೆ ಸರ್ವ ಶಕ್ತಿತ್ವ ಮತ್ತು ತ್ರಿಕಾಲಜ್ಞಾನಗಳು ಬರುವುವು. 

\vspace{-0.3cm}

\begin{verse}
ತದ್ವೈರಾಗ್ಯಾದಪಿ ದೋಷಬೀಜಕ್ಷಯೇ ಕೈವಲ್ಯಮ್​~॥ ೫೧~॥
\end{verse}

\vspace{-0.4cm}

\dsize{ಈ ಶಕ್ತಿಗಳನ್ನೂ ಕೂಡ ಬಯಸದಿರುವುದರಿಂದ ಪಾಪಬೀಜ ನಾಶವಾಗುವುದು. ಇದೇ ನಮ್ಮನ್ನು ಕೈವಲ್ಯಪದವಿಗೆ ಒಯ್ಯುವುದು. }

\vspace{0.1cm}

ಅವನು ಏಕಾಕಿಯಾಗುತ್ತಾನೆ, ಮುಕ್ತನಾಗುತ್ತಾನೆ. ಸರ್ವಶಕ್ತಿತ್ವ ಮತ್ತು ಸರ್ವಜ್ಞತೆಗಳನ್ನು ಕೂಡ ತೊರೆದ ಮೇಲೆ ಸ್ವರ್ಗಸುಖ ಪ್ರಲೋಭಗಳನ್ನು ಕೂಡ ಜಯಿಸುತ್ತಾನೆ. ಯೋಗಿಯು ಇಂತಹ ಅದ್ಭುತ ಸಿದ್ಧಿಗಳನ್ನೆಲ್ಲ ನೋಡಿ ತ್ಯಜಿಸಿದ ಮೇಲೆ ಅವನು ಗುರಿಯನ್ನು ಸೇರುತ್ತಾನೆ. ಈ ಸಿದ್ಧಿಗಳೆಲ್ಲ ಏನು? ಕೇವಲ ತೋರಿಕೆ, ಅವುಗಳೇನು ಕನಸಿಗಿಂತ ಉತ್ತಮವಲ್ಲ, ಸರ್ವಶಕ್ತಿತ್ವವೂ ಕೂಡ ಒಂದು ಕನಸೇ. ಇದು ಮನಸ್ಸಿನ ಮೇಲೆ ನಿಂತಿದೆ. ಮನಸ್ಸು ಎಲ್ಲಿಯವರೆಗೂ ಇದೆಯೋ ಅಲ್ಲಿಯವರೆಗೂ ನಾವು ಇದನ್ನು ತಿಳಿದುಕೊಳ್ಳಬಹುದು. ಆದರೆ ನಮ್ಮ ಗುರಿ ಮನಸ್ಸನ್ನೂ ಮೀರಿರುವುದು. 

\vspace{-0.2cm}

\begin{verse}
ಸ್ಥಾನ್ಯುಪನಿಮಂತ್ರಣೇ ಸಂಗಸ್ಮಯಾಕರಣಂ ಪುನರನಿಷ್ಟಪ್ರಸಂಗಾತ್​~॥ ೫೨~॥
\end{verse}

\vspace{-0.4cm}

\dsize{ಪುನಃ ಬರುವ ಅನಿಷ್ಟಕ್ಕೋಸುಗವಾಗಿ ಯೋಗಿಯು ದೇವತೆಗಳ ಔಪಚಾರಿಕ ಮಾತಿಗೆ ಬೆರಗಾಗಕೂಡದು ಅಥವಾ ಆನಂದಿಸಕೂಡದು. }

\vspace{0.1cm}

ಇನ್ನೂ ಎಷ್ಟೋ ಅಪಾಯಗಳಿವೆ. ದೇವತೆಗಳು ಮತ್ತು ಇನ್ನೂ ಇತರ ವ್ಯಕ್ತಿಗಳು ಯೋಗಿಯನ್ನು ಪ್ರಲೋಭಿಸಲು ಬರುವರು. ಯಾರೂ ಪೂರ್ಣಸಿದ್ಧರಾಗಕೂಡದೆಂಬುದೆ ಅವರ ಬಯಕೆ. ನಮ್ಮಂತೆಯೆ ಅವರೂ ಅಸೂಯಾಪರರು. ಕೆಲವು ವೇಳೆ ನಮಗಿಂತ ಕೀಳು. ತಮ್ಮ ಸ್ಥಾನಚ್ಯುತಿಯ ಅಂಜಿಕೆ ಅವರಿಗೆ ಬಹಳ. ಮುಕ್ತಿಯನ್ನು ಪಡೆಯದ ಯೋಗಿಗಳು ಸತ್ತರೆ ದೇವತೆಗಳಾಗುತ್ತಾರೆ. ನೇರವಾದ ಹಾದಿಯನ್ನು ಬಿಟ್ಟು, ಯಾವುದಾದರೂ ಪಕ್ಕದ ಹಾದಿಯನ್ನು ಹಿಡಿದು, ಈ ಸಿದ್ಧಿಗಳನ್ನು ಪಡೆಯುತ್ತಾರೆ. ಆಗ ಅವರು ಪುನಃ ಹುಟ್ಟಬೇಕಾಗುತ್ತದೆ. ಈ ಪ್ರಲೋಭನಗಳನ್ನು ತಡೆಯುವಷ್ಟು ಯಾರು ಶಕ್ತರಾಗಿರುವರೋ, ಅವರು ನೇರವಾಗಿ ಗುರಿ ಸೇರಿ ಮುಕ್ತರಾಗುತ್ತಾರೆ. 

\vspace{-0.3cm}

\begin{verse}
ಕ್ಷಣತತ್​ಕ್ರಮಯೋಃ ಸಂಯಮಾದ್ವಿವೇಕಜಂ ಜ್ಞಾನಮ್​~॥ ೫೩~॥
\end{verse}

\vspace{-0.4cm}

\dsize{ಒಂದು ಕ್ಷಣಕಾಲದ ಮೇಲೆ, ಅದರ ಹಿಂದೆ ಮತ್ತು ಮುಂದೆ ಸಂಯಮ ಮಾಡುವುದರಿಂದ ವಿವೇಕ ಬರುವುದು. }

\vspace{0.1cm}

ದೇವತೆಗಳು, ಸ್ವರ್ಗ ಮತ್ತು ಸಿದ್ಧಿಗಳಿಂದ ನಾವು ಹೇಗೆ ತಪ್ಪಿಸಿಕೊಳ್ಳುವುದು? ವಿವೇಕದಿಂದ ಒಳ್ಳೆಯದಾವುದು ಕೆಟ್ಟದಾವುದು ಎಂಬುದನ್ನು ತಿಳಿದುಕೊಳ್ಳುವುದರಿಂದ. ಆದಕಾರಣವೇ ವಿವೇಕಶಕ್ತಿಯನ್ನು ಬಲಗೊಳಿಸುವುದಕ್ಕೆ ಒಂದು ಸಂಯಮವನ್ನು ಕೊಟ್ಟಿರುವುದು. ಒಂದು ಕ್ಷಣಕಾಲದಮೇಲೆ ಅದರ ಹಿಂದೆ ಏನಾಯಿತು ಮತ್ತು ಮುಂದೆ ಏನಾಗುವುದು ಎನ್ನುವುದರ ಮೇಲೆ ಸಂಯಮ ಮಾಡುವುದೇ ಅದು. 

\vspace{-0.3cm}

\begin{verse}
ಜಾತಿ–ಲಕ್ಷಣ–ದೇಶೈರನ್ಯತಾಽನವಚ್ಛೇದಾತ್ತುಲ್ಯಯೋಸ್ತತಃ ಪ್ರತಿಪತ್ತಿಃ~॥ ೫೪~॥
\end{verse}

\vspace{-0.4cm}

\dsize{ಜಾತಿ, ಲಕ್ಷಣ, ದೇಶ–ಇವುಗಳಿಂದ ಯಾವುದನ್ನು ವಿಮರ್ಶಿಸಲಾರೆವೊ, ಅವುಗಳನ್ನು ಕೂಡ ಈ ಸಂಯಮವು ವಿಮರ್ಶಿಸುವುದು. }

\vspace{0.1cm}

ನಾವು ಅನುಭವಿಸುವ ದುಃಖವು ಅಜ್ಞಾನದಿಂದ ಬರುವುದು. ಅದಕ್ಕೆ ಕಾರಣ ನಿತ್ಯಾನಿತ್ಯವಸ್ತು ವಿವೇಕವಿಲ್ಲದಿರುವುದು. ನಾವೆಲ್ಲರೂ ಕೆಟ್ಟದ್ದನ್ನು ಒಳ್ಳೆಯದೆಂದೂ ಕನಸನ್ನು ಸತ್ಯವೆಂದೂ ಭ್ರಮಿಸುತ್ತೇವೆ. ಆತ್ಮನೊಂದೇ ಸತ್ಯ. ನಾವು ಅದನ್ನು ಮರೆತಿರುವೆವು. ದೇಹವು ಒಂದು ಅಸತ್ಯವಾದ ಕನಸು. ನಾವೆಲ್ಲರೂ ದೇಹವೆಂದು ಭಾವಿಸುವೆವು. ಈ ಅವಿವೇಕವೇ ನಮ್ಮ ದುಃಖಕ್ಕೆ ಕಾರಣ. ಇದು ಅಜ್ಞಾನದಿಂದ ಬರುವುದು. ವಿವೇಕ ಬಂದಾಗ ನಮಗೆ ಶಕ್ತಿ ಬರುವುದು. ಆಗ ಮಾತ್ರ ದೇಹ, ಸ್ವರ್ಗ, ದೇವತೆಗಳು ಎಂಬ ಭಾವನೆಯನ್ನು ದೂರಮಾಡಬಹುದು. ಜಾತಿ, ಲಕ್ಷಣ, ದೇಶಗಳಿಂದ ಒಂದು ವಸ್ತುವನ್ನು ಪ್ರತ್ಯೇಕಿಸುವುದರಿಂದ ಅಜ್ಞಾನ ಹುಟ್ಟುವುದು. ಉದಾಹರಣೆಗೆ ಒಂದು ಹಸುವನ್ನು ತೆಗೆದುಕೊಳ್ಳಿ. ಹಸುವು ಜಾತಿಯಲ್ಲಿ ನಾಯಿಗಿಂತ ಬೇರೆಯಾಗಿದೆ. ಹಸುವಿನಲ್ಲೂ ನಾವು ಒಂದು ಹಸುವನ್ನು ಬೇರೆ ಹಸುವಿನಿಂದ ಹೇಗೆ ಬೇರ್ಪಡಿಸುತ್ತೇವೆ? ಲಕ್ಷಣದಿಂದ. ಎರಡು ವಸ್ತುಗಳೂ ಒಂದೇ ಸಮನಾಗಿದ್ದರೆ ಅವುಗಳ ಸ್ಥಳವ್ಯತ್ಯಾಸದಿಂದ ಅದನ್ನು ಬೇರ್ಪಡಿಸಬಹುದು. ವಸ್ತುಗಳು ಮಿಶ್ರವಾಗಿ ಹೋಗಿ, ಲಕ್ಷಣಗಳಿಂದ ಕೂಡ ನಾವು ಇದನ್ನು ಬೇರ್ಪಡಿಸದೆ ಇರುವಾಗ, ಮೇಲೆ ಹೇಳಿದ ಅಭ್ಯಾಸದಿಂದ ಪಡೆದ ವಿವೇಕ, ಅವುಗಳನ್ನು ವಿಂಗಡಿಸುವ ಕೌಶಲ್ಯವನ್ನು ಕೊಡುವುದು. ಪುರುಷನು ಶುದ್ಧನು, ಪೂರ್ಣನು, ಆತನೊಬ್ಬನೇ ಪ್ರಪಂಚದಲ್ಲಿರುವ “ಏಕವಸ್ತು”. ಯೋಗಿಗಳ ಪರಮಸಿದ್ಧಾಂತ ಇದರ ಮೇಲೆ ನಿಂತಿದೆ. ದೇಹ ಮತ್ತು ಮನಸ್ಸುಗಳು ಸಂಯೋಗಗಳು. ಆದರೂ ಕೂಡ ಅವುಗಳೊಂದಿಗೆ ನಾವು ತಾದಾತ್ಮ್ಯ ಭಾವವನ್ನು ತಾಳುತ್ತೇವೆ. ಅವುಗಳ ಪ್ರತ್ಯೇಕತೆ ಇಲ್ಲದಂತೆ ಆಗಿದೆ. ಅದೇ ದೊಡ್ಡ ತಪ್ಪು. ವಿವೇಕವು ಬಂದ ಮೇಲೆ ಪ್ರಪಂಚದಲ್ಲಿ ಶಾರೀರಿಕ ಮತ್ತು ಮಾನಸಿಕ ವಸ್ತುಗಳೆಲ್ಲವೂ ಕೂಡ ಸಂಯೋಗಗಳಾಗಿ ಕಾಣುವುವು. ಆದಕಾರಣ ಅವುಗಳು ಪುರುಷನಲ್ಲ. 

\vspace{-0.2cm}

\begin{verse}
ತಾರಕಂ ಸರ್ವವಿಷಯಂ ಸರ್ವಥಾವಿಷಯಮಕ್ರಮಂ ಚೇತಿ\\
 ವಿವೇಕಜಂ ಜ್ಞಾನಮ್​~\hfill{॥~೫೫~॥}
\end{verse}

\vspace{-0.4cm}

\dsize{ಎಲ್ಲಾ ವಸ್ತುಗಳನ್ನೂ, ಅವುಗಳ ಎಲ್ಲಾ ಸ್ಥಿತಿಯಲ್ಲಿಯೂ, ಒಂದೇ ಸಲ ತನ್ನ ಅಧೀನಕ್ಕೆ ತರುವುದೇ\break ಕಾಪಾಡುವ ವಿವೇಕ ಜ್ಞಾನ. }

\vspace{0.2cm}

ಕಾಪಾಡುವುದು ಏಕೆಂದರೆ, ಈ ಜ್ಞಾನವು ಯೋಗಿಯನ್ನು ಜನನ ಮರಣಗಳ ಸಾಗರವನ್ನು ದಾಟಿಸುತ್ತದೆ. ಸೂಕ್ಷ್ಮ ಮತ್ತು ಸ್ಥೂಲದಿಂದ ಕೂಡಿರುವ ಎಲ್ಲಾ ಪ್ರಕೃತಿಯೂ ಜ್ಞಾನದ ಅಧೀನದಲ್ಲಿರುವುದು. ಈ ಜ್ಞಾನಗ್ರಹಣ ಕ್ರಿಯೆಯಲ್ಲಿ ಒಂದು ಅನುಕ್ರಮವಿಲ್ಲ, ಏಕಕಾಲದಲ್ಲಿ ಅದು ಎಲ್ಲವನ್ನೂ ಗ್ರಹಿಸುತ್ತದೆ. 

\vspace{-0.2cm}

\begin{verse}
ಸತ್ತ್ವಪುರುಷಯೋಃ ಶುದ್ಧಿ ಸಾಮ್ಯೇ ಕೈವಲ್ಯಮಿತಿ~॥ ೫೬~॥
\end{verse}

\vspace{-0.4cm}

\dsize{ಸತ್ತ್ವ ಮತ್ತು ಪುರುಷನಲ್ಲಿ ಶುದ್ಧಿಸಾಮ್ಯವಿರುವುದರಿಂದ ಕೈವಲ್ಯ ಪದವಿ ಬರುವುದು. }

\vspace{0.2cm}

ದೇವತೆಗಳಿಂದ ಹಿಡಿದು ಒಂದು ಸಣ್ಣ ಕಣದವರೆವಿಗೂ, ಆತ್ಮವು ತಾನು ಯಾವುದರ ಅಧೀನದ ಮೇಲೆಯೂ ನಿಂತಿಲ್ಲವೆಂದು ತಿಳಿದರೆ ಅದೇ ಕೈವಲ್ಯ, ಮುಕ್ತಿ. ಈಗ ಶುದ್ಧ ಮತ್ತು ಅಶುದ್ಧಗಳ ಮಿಶ್ರವಾದ ಸತ್ತ್ವ (ಬುದ್ಧಿ) ಪುರುಷನಷ್ಟು ಪರಿಶುದ್ಧವಾದ ಮೇಲೆ ಇದು ಸಿದ್ಧಿಸುತ್ತದೆ. ಆಗ ಸತ್ತ್ವಗುಣವು ನಿರ್ವಿಶೇಷವೂ ಪರಮ ಪವಿತ್ರದ ಸಾರವೂ ಆದ ಪುರುಷನನ್ನು ಪ್ರತಿಬಿಂಬಿಸುವುದು.

\chapter{ಮುಕ್ತಿ}%%%36

\vspace{0.1cm}

\begin{verse}
ಜನ್ಮೌಷಧಿ–ಮಂತ್ರ–ತಪಃ–ಸಮಾಧಿಜಾಃ ಸಿದ್ಧಯಃ~॥ ೧~॥
\end{verse}

\vspace{-0.34cm}

\dsize{ಸಿದ್ಧಿಗಳು ಜನ್ಮ, ಔಷಧಿ, ಮಂತ್ರ, ತಪಸ್ಸು ಅಥವಾ ಏಕಾಗ್ರತೆಯಿಂದ ದೊರಕುತ್ತವೆ. }

\vspace{0.3cm}

ಕೆಲವು ವೇಳೆ ಜನರು ಸಿದ್ಧಿಯೊಂದಿಗೆ ಹುಟ್ಟುತ್ತಾರೆ. ಇದೇನೋ ಪೂರ್ವ ಜನ್ಮದಲ್ಲಿ ಸಂಪಾದಿಸಿದ ಶಕ್ತಿ; ತಮ್ಮ ಕರ್ಮಫಲವನ್ನು ಅನುಭವಿಸುವುದಕ್ಕಾಗಿ ಹುಟ್ಟಿರುವರು. ಸಾಂಖ್ಯತತ್ತ್ವದ ಕರ್ತೃವಾದ ಕಪಿಲ ಜನ್ಮಸಿದ್ಧ. ಅಂದರೆ ಈ ಸಿದ್ಧಿಯನ್ನು ಪಡೆದವನು ಎಂದು ಅರ್ಥ. 

\vspace{0.3cm}

ಈ ಸಿದ್ಧಿಗಳನ್ನು ಔಷಧದ ಮೂಲಕವಾಗಿ ಪಡೆಯಬಹುದು ಎಂದು ಯೋಗಿಗಳು ಹೇಳುತ್ತಾರೆ. ರಾಸಾಯನಿಕ ಶಾಸ್ತ್ರವು ಚಿನ್ನವನ್ನು ಮಾಡಲು ಪ್ರಯತ್ನಿಸುವ ವಿದ್ಯೆಯಿಂದ ಹುಟ್ಟಿತು ಎಂಬುದು ನಿಮಗೆಲ್ಲರಿಗೂ ಗೊತ್ತಿದೆ. ಮನುಷ್ಯರು, ಸ್ಪರ್ಶಮಣಿ, ಚಿರಂಜೀವತ್ವವನ್ನು ಕೊಡುವ ಅಮೃತ ಮುಂತಾದುವುಗಳನ್ನು ಹುಡುಕಲು ಪ್ರಯತ್ನಿಸಿದರು. ಇಂಡಿಯಾ ದೇಶದಲ್ಲಿ ರಾಸಾಯನಿಕರು ಎಂಬ ಒಂದು ಪಂಗಡವಿತ್ತು. ಅವರ ಅಭಿಪ್ರಾಯವೇನೆಂದರೆ, ಆದರ್ಶ, ಜ್ಞಾನ, ಆಧ್ಯಾತ್ಮಿಕತೆ, ಧರ್ಮ ಎಂಬುವುಗಳೆಲ್ಲ ಒಳ್ಳೆಯವು. ಆದರೆ ನಾವು ಇವುಗಳನ್ನು ದೇಹವೆಂಬ ಯಂತ್ರದ ಮೂಲಕವಾಗಿ ಮಾತ್ರ ಪಡೆಯಬಹುದು. ಆಗಿಂದಾಗ್ಗೆ ದೇಹವು ಲಯವಾಗುತ್ತಿದ್ದರೆ ನಾವು ಗುರಿಯನ್ನು ಸೇರಲು ಕಾಲವಿಳಂಬವಾಗುವುದು. ಉದಾಹರಣೆಗೆ, ಒಬ್ಬನು ಯೋಗವನ್ನು ಅಭ್ಯಾಸ ಮಾಡಬೇಕು, ಅಥವಾ ಆಧ್ಯಾತ್ಮಿಕ ಜೀವನದಲ್ಲಿ ಮುಂದುವರಿಯಬೇಕು ಎಂದು ಇಟ್ಟುಕೊಳ್ಳೋಣ. ಸ್ವಲ್ಪ ಮುಂದುವರಿಯುವುದಕ್ಕೆ ಮುಂಚೆಯೇ ಅವನು ಕಾಲವಾಗುತ್ತಾನೆ. ಅನಂತರ ಅವನು ಬೇರೆ ದೇಹವನ್ನು ಪಡೆದು ಮತ್ತೆ ಪ್ರಾರಂಭಿಸುತ್ತಾನೆ. ಅನಂತರ ಮತ್ತೆ ಸಾಯುತ್ತಾನೆ. ಹೀಗೆಯೇ ಮುಂದುವರಿದರೆ ಹುಟ್ಟುವುದು ಸಾಯುವುದರಲ್ಲಿಯೇ ಬಹಳ ಕಾಲ ವ್ಯಯವಾಗುತ್ತದೆ. ಜನನ ಮರಣಗಳಿಂದ ತಪ್ಪಿಸಿಕೊಳ್ಳುವಷ್ಟು ದೇಹ ಬಲವಾಗಿ ದೃಢಕಾಯವಾದರೆ ಆಧ್ಯಾತ್ಮಿಕ ಜೀವನದಲ್ಲಿ ಮುಂದುವರಿಯುವುದಕ್ಕೆ ಅಷ್ಟು ಹೆಚ್ಚು ಕಾಲ ಸಿಕ್ಕುತ್ತದೆ. ಅದಕ್ಕೆ ಈ ರಾಸಾಯನಿಕರು ದೇಹವನ್ನು ದೃಢವಾಗಿ ಮಾಡಿ ಎನ್ನುತ್ತಾರೆ, ದೇಹವನ್ನು ಸಾಯದಂತೆ ಮಾಡಲು ಸಾಧ್ಯವೆನ್ನುತ್ತಾರೆ. ಅವರ ಭಾವನೆ ಏನೆಂದರೆ, ಮನಸ್ಸು ದೇಹವನ್ನು ಸೃಷ್ಟಿಮಾಡಿದ್ದರೆ, ಅನಂತ ಶಕ್ತಿಗೆ ಪ್ರತಿಯೊಂದು ಮನಸ್ಸೂ ಕೂಡ ಒಂದು ತೂಬು ಎನ್ನುವುದು ಸತ್ಯವಾದರೆ, ಪ್ರತಿಯೊಂದು ತೂಬೂ ತನಗೆ ಬೇಕಾದಷ್ಟು ಶಕ್ತಿಯನ್ನು ಪಡೆಯುವುದು ಅಸಾಧ್ಯವಾದುದೇನೂ ಇಲ್ಲ. ಸದಾಕಾಲದಲ್ಲಿಯೂ ನಮ್ಮ ದೇಹವನ್ನು ಇಟ್ಟು ಕೊಂಡಿರುವುದಕ್ಕೆ ಏಕೆ ಸಾಧ್ಯವಿಲ್ಲ? ನಮಗೆ ಬೇಕಾದ ದೇಹವನ್ನು ನಾವೇ ತಯಾರು ಮಾಡಬೇಕು. ಈ ದೇಹ ನಾಶವಾದೊಡನೆಯೆ ಮತ್ತೊಂದು ದೇಹದೊಂದಿಗೆ ಹೋಗದೆ ಈ ಕ್ಷಣದಲ್ಲಿ ಇಲ್ಲೇ ಏತಕ್ಕೆ ತಯಾರು ಮಾಡಬಾರದು? ಈ ಸಿದ್ಧಾಂತವೇನೂ ಸರಿ. ಸತ್ತ ಮೇಲೆ ನಾವು ಇರುವುದು\break ನಿಜವಾದರೆ, ಬೇರೆ ದೇಹಗಳನ್ನು ಮಾಡುವುದು ನಿಜವಾದರೆ, ಈ ದೇಹವನ್ನು ಸಂಪೂರ್ಣ ನಾಶಮಾಡದೆ, ಇಲ್ಲೇ ಅದನ್ನು ಯಾವಾಗಲೂ ಬದಲಾಯಿಸುವುದರ ಮೂಲಕ ಇಲ್ಲೇ ದೇಹವನ್ನು ತಯಾರು ಮಾಡುವುದು ಏತಕ್ಕೆ ಸಾಧ್ಯವಾಗಬಾರದು? ಪಾದರಸ ಮತ್ತು ಗಂಧಕದಲ್ಲಿ ಏನೋ ಒಂದು ಅದ್ಭುತವಾದ ಶಕ್ತಿ ಹುದುಗಿದೆ. ಇದರಿಂದ ಮಾಡಿದ ಕೆಲವು ಔಷಧಗಳ ಸೇವನೆಯಿಂದ ವ್ಯಕ್ತಿಯು ತನಗೆ ಇಚ್ಛೆ ಬಂದಷ್ಟು ಕಾಲ ಜೀವಿಸಿರಬಹುದು ಎಂದು ಅವರು ನಂಬುತ್ತಿದ್ದರು. ಕೆಲವು ಔಷಧಿಗಳು ಆಕಾಶದಲ್ಲಿ ಹಾರಾಡುವುದು\break ಮುಂತಾದ ಶಕ್ತಿಗಳನ್ನು ಕೊಡುತ್ತವೆ ಎಂದು ಕೆಲವರು ನಂಬಿದ್ದರು. ಈಗಿನ ಕಾಲದ ಅನೇಕ ಔಷಧಿಗಳು, ಅದರಲ್ಲಿಯೂ ಔಷಧದಲ್ಲಿ ಲೋಹವನ್ನು ಉಪಯೋಗಿಸುವುದು, ರಾಸಾಯನಿಕರಿಂದ ನಮಗೆ ಬಂದದ್ದು. ಕೆಲವು ಯೋಗಿಗಳೆ ಪಂಗಡದವರು, ಅವರ ಮುಖ್ಯ ಗುರುಗಳು ತಮ್ಮ ಹಳೆಯ ದೇಹಗಳಲ್ಲೇ ಇನ್ನೂ ಜೀವಿಸಿರುವರು ಎಂದು ಹೇಳುತ್ತಾರೆ. ಯೋಗದ ಮುಖ್ಯ ಗುರುವಾದ ಪತಂಜಲಿ ಇದನ್ನು ನಿರಾಕರಿಸುವುದಿಲ್ಲ. 

\vspace{0.2cm}

ಮಂತ್ರಶಕ್ತಿ: ಮಂತ್ರಗಳೆಂಬ ಪವಿತ್ರವಾದ ಕೆಲವು ಪದಗಳಿವೆ. ಅವುಗಳನ್ನು ಸರಿಯಾಗಿ ಉಚ್ಚರಿಸಿದರೆ ಇಂತಹ ಅದ್ಭುತ ಸಿದ್ಧಿಯನ್ನು ಕೊಡುವ ಶಕ್ತಿ ಅವಕ್ಕೆ ಇದೆ. ಹಗಲು ರಾತ್ರೆ ನಾವು ಇಂತಹ ಪವಾಡ ರಾಶಿಯೊಳಗೆ ಜೀವಿಸುತ್ತಿರುವೆವು. ಅದರಲ್ಲಿ ಯಾವುದನ್ನೂ ನಾವು ಯೋಚಿಸುವುದೇ ಇಲ್ಲ. ಮನುಷ್ಯನ ಶಕ್ತಿಗೆ, ಮಂತ್ರ ಶಕ್ತಿಗೆ ಮತ್ತು ಮಾನಸಿಕ ಶಕ್ತಿಗೆ ಮಿತಿ ಇಲ್ಲ. 

\vspace{0.2cm}

ತಪಸ್ಸು: ಎಲ್ಲಾ ಧರ್ಮದವರೂ ತಪಸ್ಸು ಮತ್ತು ದೇಹದಂಡನೆಗಳನ್ನು ಅಭ್ಯಾಸಮಾಡಿರುವುದು ಕಾಣುತ್ತದೆ. ಇಂತಹ ಧಾರ್ಮಿಕ ಭಾವನೆಗಳಲ್ಲಿ ಹಿಂದುಗಳು ಯಾವಾಗಲೂ ಒಂದು ಅತಿರೇಕಕ್ಕೆ ಹೋಗಿರುವರು. ತಮ್ಮ ಕೈಗಳು ಒಣಗಿ ನಿಶ್ಚೇಷ್ಟಿತವಾಗುವವರೆಗೂ ಅದನ್ನು ಮೇಲಕ್ಕೆ ಎತ್ತಿ ಹಿಡಿದಿರುವವರು ನಿಮಗೆ ಸಿಕ್ಕುತ್ತಾರೆ. ಕಾಲು ಬಾತುಹೋಗಿ ಅದನ್ನು ಮತ್ತೆ ಮಡಿಸಲಾಗದಂತಾಗುವವರೆಗೂ ಹಗಲು ರಾತ್ರಿ ನಿಂತುಕೊಂಡೇ ಇರುವವರಿದ್ದಾರೆ. ಮತ್ತೆ ಅವರು ಬದುಕಿರುವ ತನಕ ಕುಳಿತುಕೊಳ್ಳಲಾರರು. ಹೀಗೆ ಮೇಲಕ್ಕೆ ಕೈ ಎತ್ತಿಕೊಂಡಿದ್ದವನನ್ನು ನಾನೊಮ್ಮೆ ನೋಡಿದೆ. ಪ್ರಾರಂಭದಲ್ಲಿ ಹೇಗೆನಿಸಿತೆಂದು ಅವನನ್ನು ಕೇಳಿದೆ. “ಓ! ಅದು ನರಕಯಾತನೆಯಾಗಿತ್ತು. ಒಂದು ನದಿಗೆ ಹೋಗಿ ನೀರಿನಲ್ಲಿರಬೇಕಾಗುತ್ತಿತ್ತು. ಆಗ ಮಾತ್ರ ಸ್ವಲ್ಪ ಯಾತನೆ ಕಡಮೆಯಾಗುತ್ತಿತ್ತು. ಒಂದು ತಿಂಗಳಾದ ಮೇಲೆ ಹೆಚ್ಚು ಕಷ್ಟವಾಗಲಿಲ್ಲ ಎಂದನು. ಇಂತಹ ಅಭ್ಯಾಸಗಳಿಂದ ಸಿದ್ಧಿಯನ್ನು ಪಡೆಯಬಹುದು. 

\vspace{0.2cm}

ಸಮಾಧಿ: ಇದು ನಿಜವಾದ ಯೋಗ. ಇದೇ ಈ ಶಾಸ್ತ್ರದ ಪ್ರಧಾನ ವಿಷಯ ಮತ್ತು ಅತ್ಯುನ್ನತ ಸಾಧನೆ. ಹಿಂದಿನವು ಗೌಣವಾದುವುಗಳು ಮತ್ತು ಅವುಗಳಿಂದ ಅತ್ಯುನ್ನತವಾದುದನ್ನು ಪಡೆಯಲಾಗುವುದಿಲ್ಲ. ಸಮಾಧಿಯಿಂದ ಯಾವುದನ್ನು ಬೇಕಾದರೂ–ಮಾನಸಿಕ, ನೈತಿಕ ಅಥವಾ ಆಧ್ಯಾತ್ಮಿಕವಾದ ಎಲ್ಲವನ್ನೂ ಪಡೆಯಬಹುದು. 

\newpage

\begin{verse}
ಜಾತ್ಯಂತರ ಪರಿಣಾಮಃ ಪ್ರಕೃತ್ಯಾಽಽಪೂರತ್​~॥ ೨~॥
\end{verse}

\vspace{-0.4cm}

\dsize{ಪ್ರಕೃತಿಯ ಒಳತುಂಬುವಿಕೆಯಿಂದ ಜಾತಿಯ ಬದಲಾವಣೆಯಾಗುವುದು. }

\vspace{0.2cm}

ಇಂತಹ ಸಿದ್ಧಿಗಳು. ಜನನದಿಂದ, ಕೆಲವು ವೇಳೆ ಔಷಧಿಯ ಮೂಲಕ, ಕೆಲವು ವೇಳೆ ತಪಸ್ಸಿನಿಂದ ಬರುತ್ತವೆ ಎಂಬ ಸಲಹೆಯನ್ನು ಪತಂಜಲಿ ಮುಂದಿಡುವನು. ಈ ದೇಹವನ್ನು ಎಷ್ಟು ದಿನಗಳು ಬೇಕಾದರೂ ನಾವು ಇಡಬಹುದು ಎಂಬುದನ್ನು ಆತನು ಒಪ್ಪುತ್ತಾನೆ. ಈಗ ಒಂದು ಜಾತಿಯು ಮತ್ತೊಂದಕ್ಕೆ ಬದಲಾವಣೆಯಾಗಬೇಕಾದರೆ ಕಾರಣವೇನು ಎಂಬುದನ್ನು ಹೇಳುತ್ತಾನೆ. ಇದು ಪ್ರಕೃತಿಪೂರಣದಿಂದ ಆಗುತ್ತದೆ. ಇದನ್ನು ಮುಂದಿನ ಸೂತ್ರದಲ್ಲಿ ವಿವರಿಸುವನು. 

\vspace{-0.1cm}

\begin{verse}
ನಿಮಿತ್ತಮಪ್ರಯೋಜಕಂ ಪ್ರಕೃತೀನಾಂ ವರಣಭೇದಸ್ತು ತತಃ ಕ್ಷೇತ್ರಿಕವತ್​~॥~೩~॥
\end{verse}

\vspace{-0.35cm}

\dsize{ಪ್ರಕೃತಿಯ ಬದಲಾವಣೆಗೆ ಒಳ್ಳೆಯ ಮತ್ತು ಕೆಟ್ಟ ಕರ್ಮಗಳೇ ಪ್ರತ್ಯಕ್ಷ ಕಾರಣಗಳಲ್ಲ. ಆದರೆ ಇವು ಪ್ರಕೃತಿಯ ವಿಕಾಸಕ್ಕೆ ಇರುವ ತೊಡಕುಗಳನ್ನು ನಾಶಮಾಡುತ್ತವೆ; ಹರಿಯುವ ನೀರಿಗೆ ಇರುವ ಅಡಚಣೆಯನ್ನು ರೈತನು ತೆಗೆಯುವುದರಿಂದ ನೀರು ಸ್ವಭಾವತಃ ಹರಿಯುವಂತೆ. }

\vspace{0.2cm}

ನೆಲವನ್ನು ಉಳುವುದಕ್ಕೆ ಬೇಕಾದ ನೀರು ಆಗಲೆ ಕಾಲುವೆಯಲ್ಲಿ ಇದೆ. ಆದರೆ ತೂಬು ಮುಚ್ಚಿದೆ. ರೈತನು ತೂಬನ್ನು ತೆರೆದರೆ ನೀರು ತನಗೆ ತಾನೆ ಆಕರ್ಷಣ ಶಕ್ತಿಯ ನಿಯಮವನ್ನು ಅನುಸರಿಸಿ ಹರಿಯುವುದು. ಎಲ್ಲಾ ಶಕ್ತಿ ಮತ್ತು ಪ್ರಗತಿ ಎಲ್ಲಾ ವ್ಯಕ್ತಿಗಳಲ್ಲಿಯೂ ಇರುವುವು. ಪೂರ್ಣತೆಯೇ ಮಾನವನ ಸ್ವಭಾವ. ಆದರೆ ಬಂಧನಕ್ಕೊಳಗಾಗಿ ತನ್ನ ನೈಜಸ್ಥಿತಿಯ ಪ್ರಕಾಶಕ್ಕೆ ತಡೆ ಬಂದಿರುವುದು. ಆ ತಡೆಯನ್ನು ಯಾರಾದರೂ ತೆಗೆದರೆ ಪ್ರಕೃತಿಯು ಅಲ್ಲಿ ಹರಿಯುವುದು. ವ್ಯಕ್ತಿಯು ಆಗಲೇ ಶಕ್ತಿಯನ್ನು ಪಡೆಯುವನು, ಆತಂಕವನ್ನು ತೆರೆದ ಒಡನೆಯೇ ಪ್ರಕೃತಿಯು ಅಲ್ಲಿ ಹರಿದು ಬರುವುದು. ಆಗ ನಾವು ಯಾರನ್ನು ದುಷ್ಟರು ಎನ್ನುವೆವೋ ಅವರು ಮಹಾತ್ಮ ರಾಗುವರು. ಪ್ರಕೃತಿಯು ನಮ್ಮನ್ನು ಪೂರ್ಣತೆಯ ಕಡೆ ನೂಕುತ್ತಿರುವುದು. ಕೊನೆಗೆ ಪ್ರಕೃತಿ ಎಲ್ಲರನ್ನೂ ಅಲ್ಲಿಗೆ ತಂದೇ ತರುತ್ತದೆ. ಧಾರ್ಮಿಕರಾಗಬೇಕೆನ್ನುವ ನಮ್ಮ ಎಲ್ಲಾ ಅಭ್ಯಾಸ ಮತ್ತು ಸಾಧನೆಗಳೂ ನಿಷೇಧಾತ್ಮಕ ಕ್ರಿಯೆಗಳು–ಇರುವ ಆತಂಕವನ್ನು ತೆಗೆದು, ನಮ್ಮ ಸ್ವಭಾವವಾದ, ಆಜನ್ಮಸಿದ್ಧ ಹಕ್ಕಾದ ಪವಿತ್ರತೆಯನ್ನು ಬರಮಾಡಿಕೊಳ್ಳುವುದು. 

\vspace{0.2cm}

ಇಂದು ಪುರಾತನ ಯೋಗಿಗಳ ವಿಕಾಸವಾದವನ್ನು ಆಧುನಿಕ ಸಂಶೋಧನೆಯ ಬೆಳಕಿನಲ್ಲಿ ಚೆನ್ನಾಗಿ ತಿಳಿದುಕೊಳ್ಳಬಹುದು. ಯೋಗಿಯ ಸಿದ್ಧಾಂತವು ಇದಕ್ಕಿಂತ ಉತ್ತಮವಾದ ವಿವರಣೆ. ಆಧುನಿಕರ ಅಭಿಪ್ರಾಯದ ಪ್ರಕಾರ ವಿಕಾರವಾದಕ್ಕೆ ಎರಡು ಕಾರಣಗಳಾದ ಲೈಂಗಿಕ ಆಯ್ಕೆ (Sexual Selection) ಮತ್ತು ಅತ್ಯಂತ ಅರ್ಹವಾದುದೇ ಉಳಿಯುವುದು (Survival of the Fittest) ಇವುಗಳು ಸಮರ್ಪಕವಾಗಿಲ್ಲ. ದೇಹಪೋಷಕ ಮತ್ತು ಕಾಮಪೂರಕ ಸ್ಪರ್ಧೆಯನ್ನು ತೊಡೆದು ಹಾಕುವಷ್ಟು ಮಾನವನ ಜ್ಞಾನವು ಒಂದು ವೇಳೆ ಮುಂದುವರಿದರೆ, ಆಧುನಿಕ ಅಭಿಪ್ರಾಯದ ಪ್ರಕಾರ, ಮಾನವನ ಏಳಿಗೆ ನಿಂತು ಜನಾಂಗವೇ ನಾಶವಾಗುವುದು. ಈ ಅಭಿಪ್ರಾಯವು ಪ್ರತಿಯೊಬ್ಬ ಕ್ರೂರನಿಗೂ ಅವನ ಅಂತರಾತ್ಮನ ವಾಣಿಯನ್ನು ಸ್ತಬ್ಧ ಮಾಡುವುದಕ್ಕೆ ಒಂದು ವಾದವನ್ನು ಒದಗಿಸಿಕೊಡುವುದು. ದೊಡ್ಡ ತತ್ತ್ವಜ್ಞರಂತೆ ತೋರಿಸಿಕೊಳ್ಳುತ್ತ ಎಲ್ಲಾ ದುಷ್ಟರನ್ನೂ ಮತ್ತು ಅನರ್ಹರನ್ನೂ ನಾಶಮಾಡಿ ಜನಾಂಗವನ್ನು ಉಳಿಸಬೇಕೆಂದು ಆಶಿಸುವ ಜನರಿಗೆ ಬರಗಾಲವಿಲ್ಲ. ಅನರ್ಹರು ಯಾರು ಎಂಬುದನ್ನು ನಿರ್ಣಯಿಸುವುದಕ್ಕೆ ಅವರೇ ನ್ಯಾಯಾಧೀಶರೇನು! ಆದರೆ ಪುರಾತನ ಮಹಾ ವಿಕಾಸವಾದಿಯಾದ ಪತಂಜಲಿಯಾದರೂ ಹೇಳುವುದು ಹೀಗೆ: ಆಗಲೆ ಪ್ರತಿಯೊಬ್ಬರಲ್ಲಿಯೂ ಅಂತರ್ಗತವಾಗಿರುವ ಪೂರ್ಣಾವಸ್ಥೆಯನ್ನು ವ್ಯಕ್ತಗೊಳಿಸುವುದೇ ನಿಜವಾದ ವಿಕಾಸವಾದದ ರಹಸ್ಯ. ಈ ಪೂರ್ಣತೆಗೆ ಅಡಚಣೆ ಬಂದೊದಗಿದೆ. ಆದರೆ ಅದರ ಹಿಂದೆ ಇರುವ ಅನಂತ ತರಂಗವು ವ್ಯಕ್ತವಾಗುವುದಕ್ಕೆ ಪ್ರಯತ್ನಿಸುತ್ತಿದೆ. ಈ ಸ್ಪರ್ಧೆ ಮತ್ತು ಹೋರಾಟ ನಮ್ಮ ಅಜ್ಞಾನದ ಪರಿಣಾಮ. ಏಕೆಂದರೆ ತೂಬನ್ನು ತೆಗೆದು ನೀರನ್ನು ಒಳಗೆ ಬಿಡಲು ಸರಿಯಾದ ಮಾರ್ಗ ನಮಗೆ ಗೊತ್ತಿಲ್ಲ. ಹಿಂದೆ ಇರುವ ಅನಂತ ತರಂಗ ವ್ಯಕ್ತವಾಗಲೇಬೇಕು. ಇದೇ ಎಲ್ಲಾ ಅಭಿವ್ಯಕ್ತಿಗಳಿಗೂ ಕಾರಣ. ಜೀವನೋಪಾಯ ಕಾಮತೃಪ್ತಿಯ ಸ್ಪರ್ಧೆಗಳೆಲ್ಲ, ಕ್ಷಣಿಕವಾದ ಅನಾವಶ್ಯಕವಾದ, ಅಜ್ಞಾನಜನ್ಯವಾದ ಬಾಹ್ಯ ಪರಿಣಾಮ. ಎಲ್ಲಾ ಸ್ಫರ್ಧೆಗಳು ನಿಂತ ಮೇಲೆಯೂ ಕೂಡ ನಮ್ಮ ಬೆನ್ನ ಹಿಂದೆ ಇರುವ ಪೂರ್ಣತೆ ಎಲ್ಲರೂ ಪೂರ್ಣವಾಗುವವರೆಗೆ ಮುಂದುವರಿಯುವಂತೆ ಪ್ರೇರೇಪಿಸುವುದು. ಆದಕಾರಣ ಈ ಸ್ಪರ್ಧೆಯು ಪ್ರಗತಿಗೆ ಆವಶ್ಯಕವೆಂದು ನಂಬುವುದಕ್ಕೆ ಕಾರಣವಿಲ್ಲ. ಪ್ರಾಣಿಯಲ್ಲಿ ಮನುಷ್ಯತ್ತ್ವವು ತುಳಿಯಲ್ಪಟ್ಟಿತ್ತು. ಆದರೆ ಬಾಗಿಲು ತೆರೆದೊಡನೆಯೆ ಮನುಷ್ಯ ಹೊರಗೆ ಬಂದನು. ಅದರಂತೆಯೇ ಮನುಷ್ಯನಲ್ಲಿ ಸುಪ್ತ ದೇವರು ಅಜ್ಞಾನದ ಸೆರೆಯಲ್ಲಿ ಬಂಧಿತನಾಗಿರುವನು. ಜ್ಞಾನ ಈ ಸೆರೆಯನ್ನು ಭೇದಿಸಿದ ಮೇಲೆ ದೇವನು ವ್ಯಕ್ತನಾಗುತ್ತಾನೆ. 

\begin{verse}
ನಿರ್ಮಾಣಚಿತ್ತಾನ್ಯಸ್ಮಿತಾಮಾತ್ರಾತ್​~॥ ೪~॥
\end{verse}

\vspace{-0.3cm}

\dsize{ಅಹಂಕಾರದಿಂದ ಮಾತ್ರ ಸೃಷ್ಟಿಸಿದ ಚಿತ್ತಗಳು ಬರುವುವು. }

\vspace{0.2cm}


ನಾವು ನಮ್ಮ ಪಾಪಪುಣ್ಯಕರ್ಮಗಳಿಗೆ ಅನುಸಾರವಾಗಿ ಅನುಭವಿಸುತ್ತೇವೆ ಎಂಬುದು ಕರ್ಮಸಿದ್ಧಾಂತ. ಮಾನವನಲ್ಲಿರುವ ಮಾಹಾತ್ಮ್ಯೆಯನ್ನು ಹೊಂದುವುದೇ ತತ್ತ್ವದ ಆದರ್ಶ. ಎಲ್ಲಾ ಶಾಸ್ತ್ರಗಳೂ ಮಾನವ ಜೀವನದ ಮಹಿಮೆಯನ್ನು ಸಾರುತ್ತವೆ ಮತ್ತು ಅದೇ ಧ್ವನಿಯಲ್ಲಿಯೇ ಕರ್ಮವನ್ನು ಬೋಧಿಸುತ್ತವೆ. ಒಳ್ಳೆಯ ಮತ್ತು ಕೆಟ್ಟ ಕರ್ಮಗಳು ತಮಗನುಗುಣವಾದ ಫಲವನ್ನು ತರುವುವು. ಆದರೆ ಆತ್ಮನು ಕೆಟ್ಟ ಮತ್ತು ಒಳ್ಳೆಯ ಕರ್ಮದ ಬಂಧನಕ್ಕೆ ಒಳಗಾದರೆ ಅವನು ಯಾವುದಕ್ಕೂ ಪ್ರಯೋಜನವಿಲ್ಲದವನಂತೆ ಆಗುವನು. ಪಾಪಕರ್ಮವು ಪುರುಷನ ನೈಜಸ್ವಭಾವ ಪ್ರಕಾಶಕ್ಕೆ ಆತಂಕವನ್ನು ತರುವುದು. ಒಳ್ಳೆಯ ಕರ್ಮವು ಆತಂಕವನ್ನು ನಿವಾರಿಸುವುದು. ಪುರುಷನ ಮಹಿಮೆಯು ಆಗ ಬೆಳಗುವುದು. ಪುರುಷನು ಮಾತ್ರ ಬದಲಾವಣೆ ಹೊಂದುವುದಿಲ್ಲ. ನೀವು ಏನು ಮಾಡಿದರೂ, ನಿಮ್ಮ ಮಹಿಮೆಯನ್ನು, ನಿಮ್ಮ ಪರಿಣಾಮಕ್ಕೂ ಸಿಕ್ಕುವುದಿಲ್ಲ. ಪೂರ್ಣತೆಯನ್ನು ಮರೆಮಾಡುವ ತೆರೆ ಮಾತ್ರ ಅದರ ಮೇಲೆ ಮುಸುಕುವುದು. 

ತಮ್ಮ ಕರ್ಮವನ್ನು ಬೇಗ ಮುಗಿಸುವುದಕ್ಕಾಗಿ ಯೋಗಿಗಳು ಕಾಯವ್ಯೂಹವನ್ನು ರಚಿಸುತ್ತಾರೆ. ಈ ದೇಹಗಳಿಗೆ ಬೇಕಾದ ಮನಸ್ಸನ್ನು ತಮ್ಮ ಅಹಂಕಾರದಿಂದ ಸೃಷ್ಟಿಸುವರು. ಇವುಗಳಿಗೆ ನಿರ್ಮಾಣ ಚಿತ್ತವೆಂದು ಹೆಸರು. 

\vspace{-0.2cm}

\begin{verse}
ಪ್ರವೃತ್ತಿಭೇದೇ ಪ್ರಯೋಜಕಂ ಚಿತ್ತಮೇಕಮನೇಕೇಷಾಮ್​~॥ ೫~॥
\end{verse}

\vspace{-0.4cm}

\dsize{ಭಿನ್ನಭಿನ್ನವಾದ ನಿರ್ಮಾಣ ಮನಸ್ಸಿನ ಕ್ರಿಯೆಗಳು ಬೇರೆಯಾದರೂ, ಇವೆಲ್ಲ ಮೊದಲ ಮನಸ್ಸಿನ ಅಧೀನದಲ್ಲಿರುವುವು. }

\vspace{0.2cm}

ಬೇರೆಬೇರೆ ದೇಹಗಳಲ್ಲಿ ಕೆಲಸ ಮಾಡುವ ಮನಸ್ಸಿಗೆ ನಿರ್ಮಾಣ ಮನಸ್ಸೆಂದೂ, ದೇಹಕ್ಕೆ ನಿರ್ಮಾಣ ದೇಹಗಳೆಂದೂ ಹೆಸರು–ಅಂದರೆ ತಯಾರಾದ ದೇಹಗಳು ಮತ್ತು ಮನಸ್ಸುಗಳು. ಭೌತವಸ್ತು ಮತ್ತು ಮನಸ್ಸು ಎಂದಿಗೂ ಖಾಲಿಯಾಗದ ಉಗ್ರಾಣದಂತೆ. ನೀವು ಯೋಗಿಗಳಾದರೆ ಅದನ್ನು ನಿಮ್ಮ ಅಧೀನಕ್ಕೆ ತರುವ ರಹಸ್ಯವನ್ನು ಕಲಿಯುತ್ತೀರಿ. ಯಾವಾಗಲೂ ಇದು ನಿಮ್ಮದಾಗಿತ್ತು. ಆದರೆ ನೀವು ಅದನ್ನು ಮರೆತಿದ್ದಿರಿ. ನೀವು ಯೋಗಿಗಳಾದಾಗ ಅದನ್ನು ಮತ್ತೆ ನೆನಪಿಸಿಕೊಳ್ಳುತ್ತೀರಿ. ಆಗ ಅದನ್ನು ಹೇಗೆ ಬೇಕಾದರೂ ಬಳಸಬಹುದು. ನಿರ್ಮಾಣ ಮನಸ್ಸು ತಯಾರಾದ ಮೂಲ ವಸ್ತುವಿನಿಂದಲೇ ಬ್ರಹ್ಮಾಂಡವೂ \enginline{(Macrocosm)} ಬಂದಿರುವುದು. ಮನಸ್ಸು ಮತ್ತು ಭೌತವಸ್ತು ಇವು ಬೇರೆಯಲ್ಲ, ಒಂದೇ ವಸ್ತುವಿನ ಭಿನ್ನಸ್ಥಿತಿಗಳು. ಅಹಂಕಾರವೇ ಮೂಲವಸ್ತು. ಇದೇ ಸೂಕ್ಷ್ಮಾವಸ್ಥೆ. ಇದರಿಂದ ಯೋಗಿಯ ನಿರ್ಮಾಣ ಮನಸ್ಸು ಮತ್ತು ನಿರ್ಮಾಣ ದೇಹಗಳು ತಯಾರಾಗುವುವು. ಆದಕಾರಣ ಯೋಗಿಯು ಪ್ರಕೃತಿಯ ಶಕ್ತಿ ರಹಸ್ಯವನ್ನು ಕಂಡುಹಿಡಿದ ಮೇಲೆ, ಅಹಂಕಾರದ ವಸ್ತುವಿನಿಂದ ಎಷ್ಟು ಬೇಕಾದರೂ ದೇಹವನ್ನೂ ಮತ್ತು ಮನಸ್ಸನ್ನೂ ನಿರ್ಮಾಣ ಮಾಡಬಹುದು. 

\vspace{-0.1cm}

\begin{verse}
ತತ್ರ ಧ್ಯಾನಜಮನಾಶಯಮ್​~॥ ೬~॥
\end{verse}

\vspace{-0.45cm}

\dsize{ಅನೇಕ ಚಿತ್ತಗಳಲ್ಲಿ ಯಾವುದನ್ನು ಸಮಾಧಿಯಿಂದ ಪಡೆಯುವೆವೋ, ಅದು ಆಶಾ ಹೀನವಾದದ್ದು. }

\vspace{0.2cm}

ಅನೇಕ ಜನರಲ್ಲಿ ಕಾಣುವ ಹಲವು ವಿಧದ ಮನಸ್ಸಿನಲ್ಲಿ ಯಾವುದು ಪೂರ್ಣ ಏಕಾಗ್ರತೆಯಿಂದ ಕೂಡಿದ ಸಮಾಧಿಯನ್ನು ಹೊಂದಿದೆಯೋ ಅದೇ ಅತ್ಯುತ್ತಮವಾದುದು: ಔಷಧಿ, ಮಂತ್ರ, ತಪಸ್ಸು ಮುಂತಾದುವುಗಳಿಂದ ಯಾರು ಸಿದ್ಧಿಯನ್ನು ಪಡೆದಿರುವರೋ ಅವರಲ್ಲಿ ಇನ್ನೂ ಆಸೆ ಇರುತ್ತದೆ. ಆದರೆ ಚಿತ್ತೈಕಾಗ್ರತೆಯಿಂದ ಯಾರು ಸಮಾಧಿಯನ್ನು ಪಡೆದಿರುತ್ತಾರೋ ಅವರೇ ಎಲ್ಲ ಆಸೆಗಳಿಂದ ಪಾರಾದವರು. 

\vspace{-0.1cm}

\begin{verse}
ಕರ್ಮಾಶುಕ್ಲಾಕೃಷ್ಣಂ ಯೋಗಿನಸ್ತ್ರಿವಿಧಮಿತರೇಷಾಮ್​~॥ ೭~॥
\end{verse}

\vspace{-0.35cm}

\dsize{ಕರ್ಮವು ಯೋಗಿಗೆ ಬಿಳಿಯೂ ಅಲ್ಲ, ಕಪ್ಪೂ ಅಲ್ಲ. ಉಳಿದವರಿಗೆ ಕರ್ಮವು ಬಿಳಿ, ಕಪ್ಪು ಮತ್ತು ಮಿಶ್ರವೆಂದು ಮೂರು ವಿಧವಾಗಿರುವುದು. }

\newpage 

ಮುಕ್ತಿಯನ್ನು ಪಡೆದ ಯೋಗಿಯನ್ನು, ಕೆಲಸಗಳು ಮತ್ತು ಕೆಲಸದಿಂದ ಉಂಟಾದ ಕರ್ಮಫಲಗಳು ಬಂಧಿಸಲಾರವು. ಏಕೆಂದರೆ ಅವನ ಕರ್ಮವು ಆಸೆಯಿಂದ ಪ್ರೇರೇಪಿತವಾದುದಲ್ಲ. ಅವನು ಸುಮ್ಮನೆ ಕೆಲಸ ಮಾಡುತ್ತಾನೆ. ಅವನು ಅದರಿಂದ ಒಳ್ಳೆಯದನ್ನು ಮಾಡುತ್ತಾನೆ. ಅವನು ಕರ್ಮಫಲವನ್ನು ಇಚ್ಛಿಸುವುದೂ ಇಲ್ಲ, ಅದು ಬರುವುದೂ ಇಲ್ಲ. ಅಂತಹ ಉತ್ತಮ ಸ್ಥಿತಿಗೆ ಸೇರದ ಸಾಧಾರಣ ಮನುಷ್ಯರ ಕರ್ಮಗಳು ಮೂರು ವಿಧ: ಕಪ್ಪು (ಪಾಪ) ಬಿಳಿಪು (ಪುಣ್ಯ) ಮತ್ತು ಮಿಶ್ರ. 

\vspace{-0.2cm}

\begin{verse}
ತತಸ್ತದ್ವಿಪಾಕಾನುಗುಣಾನಾಮೇವಾಭಿವ್ಯಕ್ತಿರ್ವಾಸನಾಮ್​~॥ ೮~॥
\end{verse}

\vspace{-0.35cm}

\dsize{ಪ್ರತಿಯೊಂದು ಸ್ಥಿತಿಯಲ್ಲಿಯೂ ಇಂತಹ ಮೂರು ವಿಧದ ಕರ್ಮಗಳಿಂದ ಆಯಾ ಸ್ಥಿತಿಗೆ ಸಂಬಂಧಪಟ್ಟ ಬಯಕೆಗಳು ಮಾತ್ರ ವ್ಯಕ್ತವಾಗುವುವು. (ಉಳಿದವು ತತ್ಕಾಲಕ್ಕೆ ಹಿಂದೆ ನಿಂತಿರುವುವು). }

\vspace{0.1cm}

ನಾನು ಒಳ್ಳೆಯದು ಕೆಟ್ಟದ್ದು ಮತ್ತು ಮಿಶ್ರವಾದ ಮೂರು ವಿಧ ಕರ್ಮಗಳನ್ನು ಮಾಡಿದೆ ಎಂದು ಇಟ್ಟುಕೊಳ್ಳೋಣ. ಬಹುಶಃ ನಾನು ಸತ್ತು ಸ್ವರ್ಗದಲ್ಲಿ ದೇವನಾಗಿ ಹುಟ್ಟುತ್ತೇನೆ. ದೇವನ ಶರೀರದಲ್ಲಿರುವ ಬಯಕೆಗಳು ಮಾನವಶರೀರದಲ್ಲಿರುವ ಬಯಕೆಗಳಂತೆ ಅಲ್ಲ. ದೇವನ ಶರೀರವು ಊಟ ಮಾಡುವುದಿಲ್ಲ ಮತ್ತು ನೀರು ಕುಡಿಯುವುದಿಲ್ಲ. ಆದರೆ ಇನ್ನೂ ಕಡಮೆಯಾಗದ ನನ್ನ ಪೂರ್ವ ಕರ್ಮಗಳ ಪರಿಣಾಮವಾಗಿ ಹುಟ್ಟುವ ಕುಡಿಯಬೇಕು, ತಿನ್ನಬೇಕು ಎಂಬ ಬಯಕೆ ನಾನು ದೇವನಾದಾಗ ಏನಾಗಬೇಕು? ನಾನು ದೇವನಾದಾಗ ಈ ಕರ್ಮಗಳು ಎಲ್ಲಿ ಹೋಗುತ್ತವೆ? ಬಯಕೆಗಳು ತಮಗೆ ತಕ್ಕ ವಾತಾವರಣ ಸಿಕ್ಕುವಾಗ ವ್ಯಕ್ತವಾಗುವುವು ಎಂಬುದೇ ಉತ್ತರ. ವಾತಾವರಣವು ಯಾವುದಕ್ಕೆ ಸರಿಯಾಗಿದೆಯೋ ಆ ಬಯಕೆಗಳು ವ್ಯಕ್ತವಾಗುವುವು. ಉಳಿದ ಬಯಕೆಗಳು ಸುಪ್ತವಾಗಿರುವುವು. ನಮಗೆ ಈ ಜನ್ಮದಲ್ಲಿ ಹಲವು ದೇವತೆಗಳ ಆಸೆ ಇದೆ; ಮಾನವನ ಆಸೆ ಇದೆ; ಮೃಗಗಳ ಆಸೆ ಇದೆ. ನಾನು ದೇವನ ಶರೀರವನ್ನು ಧಾರಣೆ ಮಾಡಿದರೆ ಒಳ್ಳೆಯ ಆಸೆಗಳು ಮಾತ್ರ ಮೇಲೆ ಬರುತ್ತವೆ; ಏಕೆಂದರೆ ವಾತಾವರಣವು ಅದಕ್ಕೆ ಮಾತ್ರ ಸರಿಯಾಗಿದೆ. ನಾನು ಒಂದು ಮೃಗದ ರೂಪವನ್ನು ಧಾರಣೆಮಾಡಿದರೆ, ಮೃಗೀಯ ಆಸೆಗಳು ಮಾತ್ರ ಮೇಲೆ ಬರುತ್ತವೆ; ಒಳ್ಳೆಯ ಸಂಸ್ಕಾರಗಳು ತಮ್ಮ ಸರದಿಗಾಗಿ ಕಾಯುತ್ತವೆ. ಇದು ಏನನ್ನು ತೋರುತ್ತದೆ? ವಾತಾವರಣದ ಸಹಾಯದಿಂದ ಆಸೆಯನ್ನು ತಡೆಗಟ್ಟಬಹುದು. ವಾತಾವರಣಕ್ಕೆ ತಕ್ಕ ಕರ್ಮ ಮಾತ್ರ ವ್ಯಕ್ತವಾಗುತ್ತದೆ. ವಾತಾವರಣ ಶಕ್ತಿಯು ಕರ್ಮವನ್ನು ನಿಗ್ರಹಿಸುವುದಕ್ಕೆ ಬಹಳ ಸಹಾಯಕವಾಗಬಲ್ಲದು ಎಂಬುದನ್ನು ಇದು ತೋರಿಸುತ್ತದೆ. 

\vspace{-0.2cm}

\begin{verse}
ಜಾತಿ–ದೇಶ–ಕಾಲ ವ್ಯವಹಿತಾನಾಮಪ್ಯಾನಂತರ್ಯಂ\\ ಸ್ಮೃತಿಸಂಸ್ಕಾರಯೋರೇಕರೂಪತ್ವಾತ್​~ \hfill{॥~೯~॥}
\end{verse}

\vspace{-0.35cm}

\dsize{ಜಾತಿದೇಶಕಾಲಗಳಿಂದ ಭೇದವಾಗಿದ್ದರೂ, ನೆನಪು ಮತ್ತು ಸಂಸ್ಕಾರದ ತಾದಾತ್ಮ್ಯ ಭಾವವಿರುವುದರಿಂದ ಆಸೆಯಲ್ಲಿ ಒಂದು ಅನುಕ್ರಮವಿದೆ. }

\vspace{0.1cm}

ಅನುಭವ ಸೂಕ್ಷ್ಮವಾಗಿ ಸಂಸ್ಕಾರಗಳಾಗುತ್ತದೆ. ಸಂಸ್ಕಾರವು ಜಾಗ್ರತವಾಗಿ ನೆನಪಾಗುತ್ತದೆ. ಇಲ್ಲಿ, ಸ್ಮೃತಿ ಎಂಬ ಪದದಲ್ಲಿ ಸಂಸ್ಕಾರಗಳಾಗಿ ಪರಿವರ್ತಿತವಾದ ಹಳೆಯ ಅನುಭವಗಳ ಅಪ್ರಜ್ಞಾಪೂರ್ವಕ ವ್ಯವಸ್ಥೆ ಮತ್ತು ಇಂದಿನ ಪ್ರಜ್ಞಾಪೂರ್ವಕ ಕ್ರಿಯೆ ಎಲ್ಲವೂ ಸೇರಿವೆ. ಪ್ರತಿಯೊಂದು ದೇಹದಲ್ಲಿಯೂ ಅದೇ ರೀತಿಯ ದೇಹದಲ್ಲಿ ಪಡೆದ ಸಂಸ್ಕಾರಗಳ ಗುಂಪು ಆ ದೇಹದಲ್ಲಿ ನಡೆಯುವ ಕ್ರಿಯೆಗಳಿಗೆ ಕಾರಣವಾಗುತ್ತದೆ; ಬೇರೆ ದೇಹದ ಅನುಭವ ತಡೆಯಲ್ಪಟ್ಟಿರುತ್ತದೆ. ಪ್ರತಿಯೊಂದು ದೇಹವೂ ಕೂಡ ಅದೇ ಜಾತಿಗೆ ಸೇರಿದ ಅನೇಕ ದೇಹದ ಪರಂಪರೆಯಂತೆ ಕೆಲಸ ಮಾಡುವುದು. ಹೀಗೆ ಆಸೆಯ ಕ್ರಮಕ್ಕೆ ಭಂಗ ಬರುವುದಿಲ್ಲ. 

\vspace{-0.2cm}

\begin{verse}
ತಾಸಾಮನಾದಿತ್ವಂ ಚಾಶಿಷೋ ನಿತ್ಯತ್ವಾತ್​~॥ ೧೦~॥
\end{verse}

\vspace{-0.35cm}

\dsize{ಸುಖದ ಹಂಬಲವು ನಿತ್ಯವಾಗಿರುವುದರಿಂದ ಆಸೆಗಳು ಅನಾದಿಯಾಗಿವೆ. }

\vspace{0.1cm}

ನಮ್ಮ ಅನುಭವವೆಲ್ಲ ಸುಖದ ಮೇಲಿನ ಆಸೆಯಿಂದ ಮೊದಲಾಗುವುದು. ಪ್ರತಿಯೊಂದು ಅನುಭವವೂ ಹಿಂದಿನ ಅನುಭವದ ಸ್ವಭಾವದ ಮೇಲೆ ನಿಂತಿರುವುದರಿಂದ ಅನುಭವಕ್ಕೆ ಆದಿಯಿಲ್ಲ. ಆದುದರಿಂದ ಆಸೆಗಳೂ ಅನಾದಿಯಾದುವು. 

\vspace{-0.17cm}

\begin{verse}
ಹೇತುಫಲಾಶ್ರಯಾಲಂಬನೈಃ ಸಂಗೃಹೀತತ್ವಾದೇಷಾಮಭಾವೇ ತದಭಾವಃ~॥~೧೧~॥
\end{verse}

\vspace{-0.35cm}

\dsize{ಹೇತು, ಫಲ, ಆಶ್ರಯ, ಅವಲಂಬನದಿಂದ ಕೂಡಿರುವುದರಿಂದ ಇವುಗಳಿಲ್ಲದೆ ಇದ್ದರೆ ಅದೂ ಇರುವುದಿಲ್ಲ. }

\vspace{0.1cm}

ಆಸೆಯು ಹೇತು ಮತ್ತು ಪರಿಣಾಮದಿಂದ ಕೂಡಿದೆ. ಮನಸ್ಸಿನಲ್ಲಿ ಒಂದು ಆಸೆಯನ್ನು ಎಬ್ಬಿಸಿದರೆ ಅದು ಪರಿಣಾಮವನ್ನು ಉಂಟುಮಾಡದೆ ಮಾಯವಾಗುವುದಿಲ್ಲ. ನಮ್ಮ ಚಿತ್ತವೃತ್ತಿಯಾದರೋ ಸಂಸ್ಕಾರವಾಗಿರುವ ಹಿಂದಿನ ಎಲ್ಲಾ ಬಯಕೆಗಳಿಗೂ ಆಶ್ರಯವಾಗಿ ದೊಡ್ಡ ಉಗ್ರಾಣದಂತೆ ಇರುವುದು. ಕರ್ಮಕ್ಷಯವಾಗುವವರೆವಿಗೂ ಅವು ನಾಶವಾಗುವುದಿಲ್ಲ. ಅಲ್ಲದೆ ಇಂದ್ರಿಯಗಳು ಬಾಹ್ಯವಸ್ತುಗಳನ್ನು ಸ್ವೀಕರಿಸುತ್ತಿರುವವರೆಗೂ ಹೊಸ ಆಸೆಗಳು ಏಳುತ್ತಲೇ ಇರುತ್ತವೆ. ನಮ್ಮ ಬಯಕೆಯ ಹೇತು, ಫಲ, ಆಶ್ರಯ, ಆಲಂಬನದಿಂದ ಪಾರಾದರೆ ಆಗ ಮಾತ್ರ ಅದು ಮಾಯವಾಗುವುವು. 

\vspace{-0.2cm}

\begin{verse}
ಅತೀತಾನಾಗತಂ ಸ್ವರೂಪತೋಽಸ್ತ್ಯಧ್ವಭೇದಾದ್ಧರ್ಮಾಣಾಮ್​~॥ ೧೨~॥
\end{verse}

\vspace{-0.35cm}

\dsize{ಗುಣಗಳಿಗೆ ಧರ್ಮಭೇದಗಳಿರುವುದರಿಂದ, ಭೂತ ಮತ್ತು ಭವಿಷ್ಯತ್ತು ತಮ್ಮ ಸ್ವಭಾವದ ಮೇಲೆ ನಿಂತಿರುವುದು. }

\vspace{0.1cm}

ಅನಸ್ತಿತ್ವದಿಂದ ಅಸ್ತಿತ್ವವು ಬರಲಾರದು. ಭೂತ ಭವಿಷ್ಯತ್ತುಗಳು ವ್ಯಕ್ತಾವಸ್ಥೆಯಲ್ಲಿರ\-ದಿದ್ದರೂ, ಸೂಕ್ಷ್ಮಸ್ಥಿತಿಯಲ್ಲಿ ಇರುತ್ತವೆ. 

\vspace{-0.15cm}

\begin{verse}
ತೇ ವ್ಯಕ್ತ–ಸೂಕ್ಷ್ಮಾ ಗುಣಾತ್ಮಾನಃ~॥ ೧೩~॥
\end{verse}

\vspace{-0.3cm}

\dsize{ಅವು ಗುಣಾತ್ಮಕವಾಗಿರುವುದರಿಂದ ವ್ಯಕ್ತವಾಗಿರುವುವು ಅಥವಾ ಸೂಕ್ಷ್ಮವಾಗಿರುವುವು. }

\vspace{0.1cm}

ಸತ್ತ್ವ, ರಜಸ್ಸು, ತಮಸ್ಸುಗಳೇ ಗುಣಗಳು. ಅವುಗಳ ವ್ಯಕ್ತಸ್ಥಿತಿಯೇ ನಮಗೆ ತೋರುವ ಜಗತ್ತು. ಈ ಗುಣಗಳ ಅಭಿವ್ಯಕ್ತಿಯಲ್ಲಿರುವ ಭೇದಭಾವದ ಮೇಲೆ ಭೂತ ಮತ್ತು ಭವಿಷ್ಯತ್ತು ನಿಂತಿವೆ. 

\newpage 

\begin{verse}
ಪರಿಣಾಮೈಕತ್ವಾದ್ವಸ್ತುತತ್ತ್ವಮ್​~॥ ೧೪~॥
\end{verse}

\vspace{-0.5cm}

\dsize{ಬದಲಾವಣೆಯ ಐಕ್ಯತೆಯಿಂದ ವಸ್ತುವಿನ ಐಕ್ಯತೆ ಇರುವುದು. }

\vspace{0.2cm}

ಮೂರು ಗುಣಗಳಿದ್ದರೂ, ಅವುಗಳ ಬದಲಾವಣೆ ಸರಿಸಮನಾಗಿರುವುದರಿಂದ ಎಲ್ಲಾ ವಸ್ತುಗಳಿಗೂ ಐಕ್ಯತೆಯಿದೆ. 


\begin{verse}
ವಸ್ತುಸಾಮ್ಯೇ ಚಿತ್ತಭೇದಾತ್ತಯೋರ್ವಿಭಕ್ತಃ ಪನ್ಥಾಃ~॥ ೧೫~॥
\end{verse}

\vspace{-0.5cm}

\dsize{ವಸ್ತು ಒಂದೇಯಾಗಿದ್ದರೂ ಇಂದ್ರಿಯಗ್ರಹಣ ಮತ್ತು ಬಯಕೆಗಳು ಮನಸ್ಸಿನ ಭೇದಕ್ಕೆ ತಕ್ಕಂತೆ ಬದಲಾವಣೆ ಹೊಂದುವುವು. }

\vspace{0.2cm}

ಅಂದರೆ, ನಮ್ಮ ಮನಸ್ಸುಗಳಿಂದ ಸ್ವತಂತ್ರವಾದ ವಿಷಯಜಗತ್ತು ಒಂದಿದೆ. ಇದು ಬೌದ್ಧ ವಿಜ್ಞಾನವಾದಕ್ಕೆ ವಿರುದ್ಧವಾದುದು. ಒಂದೇ ವಸ್ತುವು ಬೇರೆ ಬೇರೆ ಜನರು ಬೇರೆ ಬೇರೆ ರೀತಿಯಲ್ಲಿ ನೋಡುವುದರಿಂದ ಅದು ಯಾವುದೊ ಒಬ್ಬ ವ್ಯಕ್ತಿಯ ಕಲ್ಪನೆ ಮಾತ್ರವಲ್ಲ. \footnote{ಕೆಲವು ಆವೃತ್ತಿಗಳಲ್ಲಿ ಇನ್ನೊಂದು ಸೂತ್ರವಿದೆ:\\ನ ಚೈಕಚಿತ್ತತಂತ್ರಮ್​ ವಸ್ತುತದಪ್ರಮಾಣಕಮ್​ ತದಾ ಕಿಂ ಸ್ಯಾತ್​~॥\\ವಸ್ತುವು ಒಂದೇ ಮನಸ್ಸನ್ನು ಅವಲಂಬಿಸಿರುವುದು ಸಾಧ್ಯವಿಲ್ಲ. ಅದರ ಅಸ್ತಿತ್ವಕ್ಕೆ ಯಾವ ಪ್ರಮಾಣವೂ ಇಲ್ಲದೇ ಇರುವುದರಿಂದ ಅದು ಅಸತ್ಯವಾಗುತ್ತದೆ.

ಒಂದು ವಸ್ತುವಿನ ಅಸ್ತಿತ್ವಕ್ಕೆ ಅದು ನೀಡುವ ಅನುಭವವೊಂದೇ ಪ್ರಮಾಣವಾಗಿದ್ದರೆ, ಆಗ ಮನಸ್ಸು ಬೇರೆ ಯಾವುದರಲ್ಲೋ ತಲ್ಲಿನವಾಗಿದ್ದಾಗ ಅಥವಾ ಸಮಾಧಿಯಲ್ಲಿದ್ದಾಗ ಆ ವಸ್ತುವು ಯಾರಿಗೂ ಯಾವುದೇ ಅನುಭವವನ್ನು ನೀಡುವುದಿಲ್ಲ. ಹಾಗಾಗಿ ಅದು ಅಸ್ತಿತ್ವದಲ್ಲಿಲ್ಲ ಎಂದೇ ಹೇಳಬೇಕಾಗುತ್ತದೆ. ಇದು ಅಪೇಕ್ಷಣೀಯವಲ್ಲದ ತೀರ್ಮಾನ.}

\vspace{-0.2cm}

\begin{verse}
ತದುಪರಾಗಾಪೇಕ್ಷಿತ್ವಾಚ್ಚಿತ್ತಸ್ಯ ವಸ್ತುಜ್ಞಾತಾಜ್ಞಾತಮ್​~॥ ೧೬~॥
\end{verse}

\vspace{-0.5cm}

\dsize{ವಸ್ತುಗಳು ಮನಸ್ಸಿಗೆ ಜ್ಞಾತವಾಗಿರಬಹುದು ಅಥವಾ ಅಜ್ಞಾತವಾಗಿರಬಹುದು. ಅವು ಮನಸ್ಸಿಗೆ ಕೊಡುವ ಬಣ್ಣವನ್ನು ಅವಲಂಬಿಸುತ್ತದೆ. }

\vspace{0.2cm}

\vspace{-0.3cm}

\begin{verse}
ಸದಾ ಜ್ಞಾತಾಶ್ಚಿತ್ತವೃತ್ತಯಸ್ತತ್​ ಪ್ರಭೋಃ ಪುರುಷಸ್ಯಾಪರಿಣಾಮಿತ್ವಾತ್​~॥~೧೭~॥
\end{verse}

\vspace{-0.5cm}

\dsize{ಚಿತ್ತವೃತ್ತಿಗಳು ಯಾವಾಗಲೂ ಜ್ಞಾತವಾದುವು. ಏಕೆಂದರೆ ಮನಸ್ಸಿನ ಪ್ರಭುವಾದ ಪುರುಷನು\break ಅವಿಕಾರಿ. }

\vspace{0.2cm}

ಜಗತ್ತು ಮಾನಸಿಕವೂ ಹೌದು ಮತ್ತು ಭೌತಿಕವೂ ಹೌದು ಎನ್ನುವುದೇ ಈ ಸಿದ್ಧಾಂತದ ಸಾರ. ಎರಡೂ ಕೂಡ ನಿರಂತರ ಬದಲಾವಣೆಯನ್ನು ಹೊಂದುತ್ತಿರುವುವು. ಈ ಪುಸ್ತಕವೇನು? ನಿರಂತರ ಬದಲಾವಣೆ ಹೊಂದುತ್ತಿರುವ ಕಣದ ಸಮೂಹದ ಒಂದು ಗುಂಪು. ಒಂದು ಗುಂಪು ಹೋಗುತ್ತಿದೆ, ಮತ್ತೊಂದು ಗುಂಪು ಬರುತ್ತಿದೆ. ಇದೊಂದು ಸುಳಿ. ಆದರೆ ಐಕ್ಯತೆಯನ್ನು ಯಾವುದು ನೀಡುವುದು? ಇದನ್ನು ಅದೇ ಪುಸ್ತಕವನ್ನಾಗಿ ಯಾವುದು ಮಾಡುವುದು? ಬದಲಾವಣೆ ಲಯಬದ್ಧವಾಗಿದೆ. ಏಕರೀತಿಯಿಂದ ಬದಲಾವಣೆ ಹೊಂದು ತ್ತಿರುವ ಕಣಗಳು ನನ್ನ ಮನಸ್ಸಿಗೆ ಅನುಭವವನ್ನು ಕಳುಹಿಸುತ್ತಿವೆ. ಭಾಗಗಳು ಬದಲಾಯಿಸು ತ್ತಿದ್ದರೂ ಇವುಗಳೆಲ್ಲ ಕಲೆತು ಒಂದು ಅಖಂಡವಾದ ಚಿತ್ರವಾಗುತ್ತಿದೆ. ಮನಸ್ಸು ಸದಾ ಬದಲಾಯಿಸುತ್ತಿರುವುದು. ಮನಸ್ಸು ಮತ್ತು ದೇಹ ಒಂದೇ ದ್ರವ್ಯ ವಸ್ತುವಿನ ಎರಡು ಪದರಗಳು. ಎರಡೂ ಬೇರೆ ಬೇರೆ ವೇಗದಲ್ಲಿ ಚಲಿಸುತ್ತಿವೆ. ಸಾಪೇಕ್ಷ ದೃಷ್ಟಿಯಿಂದ ನೋಡಿದರೆ ಒಂದು ನಿಧಾನ, ಮತ್ತೊಂದು ವೇಗವಾಗಿರುವುದರಿಂದ ಈ ಎರಡು ಚಲನೆಗಳನ್ನೂ ನಾವು ಪ್ರತ್ಯೇಕಗೊಳಿಸಬಹುದು. ಉದಾಹರಣೆಗೆ, ಒಂದು ರೈಲುಬಂಡಿ ಓಡುತ್ತಿದೆ. ಜೊತೆಗೆ ಒಂದು ಗಾಡಿಯೂ ಚಲಿಸುತ್ತಿದೆ. ಇವುಗಳೆರಡರ ಚಲನೆಯನ್ನು ಕೂಡ ಸ್ವಲ್ಪ ಮಟ್ಟಿಗೆ ಕಂಡುಹಿಡಿಯುವುದು ಸಾಧ್ಯ. ಆದರೂ ಮತ್ತೊಂದು ಆವಶ್ಯಕ. ಮತ್ತೊಂದು ಚಲಿಸದೆ ಇರುವ ವಸ್ತುವಿದ್ದರೆ ಮಾತ್ರ ನಾವು ಚಲನೆಯನ್ನು ಗ್ರಹಿಸಬಹುದು. ಎರಡು ಮೂರು ವಸ್ತುಗಳು ಸಾಪೇಕ್ಷವಾಗಿ ಚಲಿಸುತ್ತಿದ್ದರೆ, ಮೊದಲು ಅತಿ ವೇಗವಾಗಿ ಹೋಗುವುದನ್ನು ಅನಂತರ ಅದಕ್ಕಿಂತ ನಿಧಾನವಾಗಿ ಹೋಗುವುದನ್ನು ಕಂಡುಹಿಡಿಯುತ್ತೇವೆ. ಮನಸ್ಸು ಗ್ರಹಿಸುವುದು ಹೇಗೆ? ಅದೂ ಕೂಡ ಚಲಿಸುತ್ತಿದೆ. ಅದಕ್ಕಿಂತ ನಿಧಾನವಾಗಿ ಚಲಿಸುವ ಮತ್ತೊಂದು ವಸ್ತುವಿಗೆ ನಾವು ಹೋಗಬೇಕು. ಹೀಗೆಯೇ ಅದಕ್ಕೆ ಕೊನೆಯಿರುವುದಿಲ್ಲ. ಆದಕಾರಣವೆ ನೀವು ಎಲ್ಲಿಯಾದರೂ ಒಂದೆಡೆಯಲ್ಲಿ ನಿಲ್ಲುವಂತೆ ಯುಕ್ತಿ ನಿರ್ಬಂಧಿಸುವುದು. ಎಂದೂ ಬದಲಾವಣೆಯಾಗದ ವಸ್ತುವನ್ನು ತಿಳಿಯುವುದರೊಂದಿಗೆ ಈ ಅನ್ವೇಷಣೆ ನಿಲ್ಲುತ್ತದೆ. ಬಿಡುವಿಲ್ಲದ ಬದಲಾವಣೆಯ ಹಿಂದೆ ಅವಿಕಾರಿಯಾದ, ನಿರ್ವರ್ಣನಾದ, ಶುದ್ಧನಾದ ಪುರುಷನಿರುವನು. ಈ ನಮ್ಮ ಭಾವನೆಯೆಲ್ಲ ಅದರ ಮೇಲೆ ಪ್ರತಿಬಿಂಬಿಸುವುದು ಅಷ್ಟೆ. ಮಾಯಾದೀಪವು ತೆರೆಯನ್ನು ವಿಕೃತಗೊಳಿಸದೆ ಅದರ ಮೇಲೆ ದೃಶ್ಯಗಳನ್ನು ವಿಸ್ತರಿಸುವಂತೆ. 

\vspace{-0.3cm}

\begin{verse}
ನ ತತ್​ ಸ್ವಾಭಾಸಂ ದೃಶ್ಯತ್ವಾತ್​~॥ ೧೮~॥
\end{verse}

\vspace{-0.43cm}

\dsize{ಮನಸ್ಸು ಒಂದು ದೃಶ್ಯವಾಗಿರುವುದರಿಂದ ಅದಕ್ಕೆ ಸ್ವಯಂಪ್ರಭೆ ಇಲ್ಲ. }

\vspace{0.1cm}

ಪ್ರಕೃತಿಯಲ್ಲಿ ಅದ್ಭುತವಾದ ಶಕ್ತಿ ವ್ಯಕ್ತವಾಗಿದೆ. ಆದರೆ ಇದಕ್ಕೆ ಸ್ವಯಂಪ್ರಭೆ ಇಲ್ಲ, ಮೂಲಚೈತನ್ಯವಿಲ್ಲ. ಪುರುಷನೊಬ್ಬನೇ ಸ್ವಯಂಪ್ರಭೆಯಿಂದ ಕೂಡಿರುವನು. ತನ್ನ ಬೆಳಕನ್ನು ಉಳಿದವುಗಳಿಗೆ ದಾನ ಮಾಡುವನು. ಎಲ್ಲಾ ಭೂತಗಳು ಮತ್ತು ಶಕ್ತಿಯ ಹಿಂದೆ ಚಲಿಸುತ್ತಿರುವುದೇ ಪುರುಷನ ಶಕ್ತಿ. 

\vspace{-0.3cm}

\begin{verse}
ಏಕಸಮಯೇ ಚೋಭಯಾನವಧಾರಣಮ್​~॥ ೧೯~॥
\end{verse}

\vspace{-0.43cm}

\dsize{ಏಕಸಮಯದಲ್ಲಿ ಎರಡು ವಿಷಯಗಳನ್ನು ಗ್ರಹಿಸುವುದು ಅದಕ್ಕೆ (ಮನಸ್ಸಿಗೆ) ಅಸಾಧ್ಯವಾಗಿರುವುದರಿಂದ. }

\vspace{0.2cm}

ಮನಸ್ಸಿಗೆ ಸ್ವಯಂ ಚೈತನ್ಯವಿದ್ದಿದ್ದರೆ, ಏಕಕಾಲದಲ್ಲಿ ತನ್ನನ್ನೂ ಮತ್ತು ತನ್ನ ವಸ್ತುವನ್ನೂ ಅದು ಗ್ರಹಿಸಬಹುದಾಗಿತ್ತು. ಆದರೆ ಅದಕ್ಕೆ ಇದು ಸಾಧ್ಯವಿಲ್ಲ. ನೀವು ಒಂದಕ್ಕೆ ಹೆಚ್ಚು ಗಮನ ಕೊಟ್ಟರೆ ಮತ್ತೊಂದು ನಿಮ್ಮ ಗಮನಕ್ಕೆ ಬರುವುದಿಲ್ಲ. ಮನಸ್ಸಿಗೆ ಸ್ವಯಂಚೈತನ್ಯವಿದ್ದರೆ, ತಾನು ಗ್ರಹಿಸಬಲ್ಲ ಭಾವನೆಗೆ ಒಂದು ಮೇರೆ ಇರುವುದಿಲ್ಲ. ಪುರುಷನು ಎಲ್ಲವನ್ನೂ ಏಕಕಾಲದಲ್ಲಿ ಗ್ರಹಿಸಬಲ್ಲ. ಆದಕಾರಣ ಪುರುಷನು ಸ್ವಯಂಚೈತನ್ಯನಾಗಿರುವನು; ಮನಸ್ಸು ಆಗಿಲ್ಲ. 

\vspace{-0.2cm}

\begin{verse}
ಚಿತ್ತಾನ್ತರದೃಶ್ಯೇ ಬುದ್ಧಿಬುದ್ಧೇರತಿಪ್ರಸಂಗಃ ಸ್ಮೃತಿಸಂಕರಶ್ಚ~॥ ೨೦~॥
\end{verse}

\vspace{-0.4cm}

\dsize{ಮತ್ತೊಂದು ಗ್ರಹಿಸುವ ಮನಸ್ಸನ್ನು ಒಪ್ಪಿಕೊಂಡರೆ, ಅಂತಹ ಊಹೆಗೆ ಕೊನೆಯೇ ಇಲ್ಲದಂತೆ ಆಗುವುದು ಮತ್ತು ಸ್ಮೃತಿಸಂಕರವೇ ಫಲವಾಗುವುದು. }

\vspace{0.2cm}

ಸಾಧಾರಣ ಮನಸ್ಸನ್ನು ತಿಳಿದುಕೊಳ್ಳುವ ಮತ್ತೊಂದು ಮನಸ್ಸಿದೆ ಎಂದು ಒಪ್ಪಿಕೊಂಡರೂ ಅದನ್ನು ತಿಳಿದುಕೊಳ್ಳಬೇಕಾದರೆ ಮತ್ತೊಂದು ಮನಸ್ಸು ಬೇಕಾಗುವುದು. ಆದಕಾರಣ ಇದಕ್ಕೆ ಕೊನೆಯೇ ಇಲ್ಲದಂತೆ ಆಗುವುದು. ಸ್ಮೃತಿಯ ಭಂಡಾರವಿಲ್ಲದೆ ಸ್ಮೃತಿ ಸಂಕರವಾಗುವುದು. 

\vspace{-0.3cm}

\begin{verse}
ಚಿತೇರಪ್ರತಿಸಂಕ್ರಮಾಯಾಸ್ತದಾಕಾರಪತ್ತೌ ಸ್ವಬುದ್ಧಿ ಸಂವೇದಮನ್​~॥ ೨೧~॥
\end{verse}

\vspace{-0.35cm}

\dsize{ಜ್ಞಾನದ ಸಾರನು (ಪುರುಷನು) ಅವಿಕಾರಿಯಾದುದರಿಂದ, ಮನಸ್ಸು ಪುರುಷನ ಆಕಾರವನ್ನು ತಾಳಿದಾಗ ಅದು ಪ್ರಜ್ಞೆಯಿಂದ ಕೂಡುವುದು. }

\vspace{0.2cm}

ಜ್ಞಾನವು ಪುರುಷನ ಗುಣವಲ್ಲ ಎಂಬುದನ್ನು ಚೆನ್ನಾಗಿ ವಿವರಿಸುವುದಕ್ಕೋಸ್ಕರ\break ಪತಂಜಲಿಯು ಇದನ್ನು ಹೇಳುತ್ತಾನೆ. ಮನಸ್ಸು ಪುರುಷನ ಹತ್ತಿರ ಬಂದಾಗ ಅವನು ಮನಸ್ಸಿನ ಮೇಲೆ ಪ್ರತಿಬಿಂಬಿಸಿದಂತೆ ತೋರುತ್ತದೆ. ಆಗ ಮನಸ್ಸು ತತ್ಕಾಲಕ್ಕೆ ತಿಳಿಯುವಂತಾಗಿ ಅದೇ ಪುರುಷನಂತೆ ಕಾಣುವುದು. 

\vspace{-0.2cm}

\begin{verse}
ದ್ರಷ್ಟೃ ದೃಶ್ಯೋಪರಕ್ತಂ ಚಿತ್ತಂ ಸರ್ವಾರ್ಥಮ್​~॥ ೨೨~॥
\end{verse}

\vspace{-0.35cm}

\dsize{ದೃಗ್​ ದೃಶ್ಯಗಳಿಂದ ಪರಿಣಾಮ ಹೊಂದುವುದರಿಂದ ಮನಸ್ಸಿಗೆ ಎಲ್ಲವನ್ನೂ ತಿಳಿದುಕೊಳ್ಳಲು ಸಾಧ್ಯವಾಗುವುದು. }

\vspace{0.2cm}

ಮನಸ್ಸಿನ ಒಂದು ಕಡೆ ಬಾಹ್ಯ ಪ್ರಪಂಚದ ದೃಶ್ಯವು ಪ್ರತಿಬಿಂಬಿಸುವುದು. ಮತ್ತೊಂದು ಕಡೆ ದೃಕ್​ ಪ್ರತಿಬಿಂಬಿಸುವುದು. ಆದಕಾರಣ ಮನಸ್ಸಿಗೆ ಸರ್ವಜ್ಞತೆ ಬರುವುದು. 

\vspace{-0.2cm}

\begin{verse}
ತದಸಂಖ್ಯೇಯವಾಸನಾಭಿಶ್ಚಿತ್ರಮಪಿ ಪರಾರ್ಥಂ ಸಂಹತ್ಯಕಾರಿತ್ವಾತ್​~॥ ೨೩~॥
\end{verse}

\vspace{-0.35cm}

\dsize{ಮನಸ್ಸು ತನ್ನ ಅಸಂಖ್ಯಾತ ಆಸೆಗಳಿಂದ ಮತ್ತೊಬ್ಬನಿಗೆ (ಪುರುಷನಿಗೆ) ಕೆಲಸ ಮಾಡುವುದು. ಏಕೆಂದರೆ ಇದು ಮತ್ತೊಂದರ ಸಂಗದಿಂದ ಕೆಲಸ ಮಾಡುವುದು. }

\vspace{0.2cm}

ಮನಸ್ಸು ಅನೇಕ ವಸ್ತುಗಳ ಸಂಯೋಗ. ಆದಕಾರಣ ಅದು ತನಗೋಸ್ಕರವಾಗಿ ಕೆಲಸ ಮಾಡಲಾರದು. ಪ್ರಪಂಚದಲ್ಲಿರುವ ಎಲ್ಲಾ ಸಂಯೋಗಗಳಿಗೂ ಕೂಡ ಹಾಗೆ ಮಿಶ್ರವಾಗುವುದಕ್ಕೆ ಒಂದು ಗುರಿ ಇದೆ. ಮತ್ತಾವುದೋ ಮೂರನೆ ವಸ್ತುವಿಗಾಗಿ ಇದು ನಡೆಯುವುದು. ಮನಸ್ಸಿನ ಈ ಸಂಯೋಗವು ಇರುವುದು ಪುರುಷನಿಗಾಗಿ. 

\vspace{-0.2cm}

\begin{verse}
ವಿಶೇಷದರ್ಶಿನ ಆತ್ಮಭಾವ–ಭಾವನಾನಿವೃತ್ತಿಃ~॥ ೨೪~॥
\end{verse}

\vspace{-0.38cm}

\dsize{ವಿಚಾರಶೀಲನಾದವನಿಗೆ ಮನಸ್ಸು ಆತ್ಮವೆಂಬ ಭಾವನೆ ತಪ್ಪುವುದು. }

\vspace{0.2cm}

ವಿಚಾರದಿಂದ ಯೋಗಿಗೆ ಮನಸ್ಸು ಪುರುಷನಲ್ಲವೆಂದು ಗೊತ್ತಾಗುವುದು. 

\vspace{-0.2cm}

\begin{verse}
ತದಾ ವಿವೇಕನಿಮ್ನಂ ಕೈವಲ್ಯಪ್ರಾಗ್ಭಾರಂ ಚಿತ್ತಮ್​~॥ ೨೫~॥
\end{verse}

\vspace{-0.34cm}

\dsize{ಅನಂತರ ವಿವೇಕದ ಕಡೆಗೆ ಹರಿಯುವುದರಿಂದ ಮನಸ್ಸು ತನ್ನ ಹಿಂದಿನ ಕೈವಲ್ಯ ಸ್ಥಿತಿಯನ್ನು\break ಪಡೆಯುವುದು. }

\vspace{0.1cm}

ಯೋಗಾಭ್ಯಾಸವು ವಿಚಾರ ಶಕ್ತಿ ಮತ್ತು ದೃಷ್ಟಿಪಟುತ್ವವನ್ನು ಕೊಡುವುದು. (ಅಜ್ಞಾನದ) ತೆರೆ ನಮ್ಮ ಕಣ್ಣಿನಿಂದ ಕೆಳಗೆ ಜಾರುವುದು. ಆಗ ವಸ್ತುವಿನ ಯಥಾರ್ಥ ಸ್ಥಿತಿಯನ್ನು ನಾವು ನೋಡುವೆವು. ಪ್ರಕೃತಿಯು ಸಂಯೋಗ. ಸಾಕ್ಷಿಯಾದ ಪುರುಷನಿಗೆ ಈ ದೃಶ್ಯವನ್ನು ತೋರುತ್ತಿದೆ. ಪ್ರಕೃತಿ ಪ್ರಭುವಲ್ಲ. ಪ್ರಕೃತಿಯ ವಿಧವಿಧವಾದ ಸಂಯೋಜನೆಗಳೆಲ್ಲ ಅಂತರಾಳದಲ್ಲಿ ಕುಳಿತಿರುವ ಪ್ರಭುವಾದ ಪುರುಷನಿಗೆ ತೋರುವ ದೃಶ್ಯಕ್ಕಾಗಿ ಎಂದು ಆಗ ನಮಗೆ ಗೊತ್ತಾಗುವುದು. ದೀರ್ಘ ಅಭ್ಯಾಸದಿಂದ ಈ ವಿವೇಕ ಬಂದಮೇಲೆ ಅಂಜಿಕೆ ನಿಲ್ಲುವುದು ಮತ್ತು ಮನಸ್ಸು ಕೈವಲ್ಯಪದವಿಯನ್ನು ಪಡೆಯುವುದು. 

\vspace{-0.25cm}

\begin{verse}
ತಚ್ಛಿದ್ರೇಷು ಪ್ರತ್ಯಯಾನ್ತರಾಣಿ ಸಂಸ್ಕಾರೇಭ್ಯಃ~॥ ೨೬~॥
\end{verse}

\vspace{-0.3cm}

\dsize{ಅದಕ್ಕೆ ಆತಂಕವನ್ನು ತರುವ ಆಲೋಚನೆಗಳು ಸಂಸ್ಕಾರದಿಂದ ಏಳುವುವು. }

\vspace{0.1cm}

ನಮಗೆ ಸಂತೋಷವನ್ನು ಉಂಟುಮಾಡುವುದಕ್ಕಾಗಿ ಕೆಲವು ಬಾಹ್ಯವಸ್ತುಗಳು ಬೇಕು ಎನ್ನುವ ನಂಬಿಕೆಯನ್ನು ಉಂಟುಮಾಡಲು ಏಳುವ ಎಲ್ಲಾ ಭಾವನೆಯೂ ಆತಂಕವನ್ನು ಉಂಟುಮಾಡುವುದು. ಪುರುಷನು ಸ್ವಭಾವತಃ ಆನಂದಮಯನು, ಮಂಗಳಕರನು, ಆದರೆ ಅಂತಹ ಜ್ಞಾನವು ಪೂರ್ವಸಂಸ್ಕಾರದಿಂದ ಆವರಿಸಲ್ಪಟ್ಚಿದೆ. ಈ ಸಂಸ್ಕಾರಗಳೆಲ್ಲ ಕ್ಷಯಿಸಬೇಕು. 

\vspace{-0.3cm}

\begin{verse}
ಹಾನಮೇಶಾಂ ಕ್ಲೇಶವದುಕ್ತಮ್​~॥ ೨೭~॥
\end{verse}

\vspace{-0.35cm}

\dsize{ಹಿಂದೆ ಹೇಳಿದಂತೆ (II, ೧೦) ಅಜ್ಞಾನ, ಅಹಂಕಾರಾದಿಗಳ ನಾಶದಂತೆಯೇ ಇದರ ನಾಶವೂ ಕೂಡ. }

\vspace{0.08cm}

\begin{verse}
ಪ್ರಸಂಖ್ಯಾನೇಽಪ್ಯಕುಸೀದಸ್ಯ ಸರ್ವಥಾ ವಿವೇಕಖ್ಯಾತೇರ್ಧರ್ಮಮೇಘಃ \\
\hspace{6cm} ಸಮಾಧಿಃ ॥ ೨೮ ॥
\end{verse}

\vspace{-0.3cm}

\dsize{ನಿಜವಾದ ವಿವೇಕಜ್ಞಾನವನ್ನು ಪಡೆಯುವಾಗಲೇ ಯಾರು ಫಲವನ್ನು ತ್ಯಜಿಸುತ್ತಾರೆಯೋ ಅವರಿಗೆ ಪೂರ್ಣ ವಿವೇಕದ ಪರಿಣಾಮವಾಗಿ ಧರ್ಮಮೇಘವೆಂಬ ಸಮಾಧಿ ಬರುವುದು. }

\vspace{0.1cm}

ಯೋಗಿಗೆ ಇಂತಹ ವಿವೇಕವು ಸಿದ್ಧಿಸಿದ ಮೇಲೆ, ಕಳೆದ ಅಧ್ಯಾಯದಲ್ಲಿ ಹೇಳಿದ ಶಕ್ತಿಗಳೆಲ್ಲ ಬರುವುವು. ಆದರೆ ನಿಜವಾದ ಯೋಗಿಯು ಇವೆಲ್ಲವನ್ನೂ ತಿರಸ್ಕರಿಸುತ್ತಾನೆ. ಅವನಿಗೆ ಧರ್ಮಮೇಘವೆಂಬ ಒಂದು ಅದ್ಭುತವಾದ ಜ್ಞಾನ, ಒಂದು ಪ್ರತ್ಯೇಕವಾದ ಕಾಂತಿ ಬರುವುದು. ಇತಿಹಾಸ ಪ್ರಸಿದ್ಧರಾದ ಮಹಾತ್ಮರಿಗೆಲ್ಲ ಇದು ಇರುವುದು. ಅವರು ಎಲ್ಲಾ ಜ್ಞಾನದ ತಳಹದಿಯನ್ನು ತಮ್ಮಲ್ಲಿಯೇ ಕಂಡು ಕೊಂಡರು. ಸತ್ಯ ಅವರಿಗೆ ಪ್ರತ್ಯಕ್ಷವಾಗಿತ್ತು. ಸಿದ್ಧಿಯ ಪ್ರಲೋಭವನ್ನು ತಿರಸ್ಕರಿಸಿದ ಮೇಲೆ ಶಾಂತಿ, ನಿಶ್ಚಲತೆ, ಪೂರ್ಣಪಾವಿತ್ರ್ಯ ಅವರ ಸ್ವಭಾವವಾಗುವುದು. 

\vspace{-0.2cm}

\begin{verse}
ತತಃ ಕ್ಲೇಶಕರ್ಮನಿವೃತ್ತಿಃ~॥ ೨೯~॥
\end{verse}

\vspace{-0.3cm}

\dsize{ಅದರಿಂದ ಕ್ಲೇಶ ಮತ್ತು ಕರ್ಮ ನಿವೃತ್ತಿಯಾಗುವುದು. }

\vfill\eject	

ಧರ್ಮಮೇಘ ಬಂದಮೇಲೆ, ಚ್ಯುತಿಯಾಗುವ ಅಂಜಿಕೆ ಇನ್ನು ಇರುವುದಿಲ್ಲ. ಯಾವ ಶಕ್ತಿಯೂ ಅವನನ್ನು ಕೆಳಗೆ ಎಳೆಯಲಾರದು. ಅವನಿಗೆ ಇನ್ನು ಯಾವ ಪಾಪವೂ ಇರುವುದಿಲ್ಲ. ಇನ್ನು ಯಾವ ದುಃಖವೂ ಇರುವುದಿಲ್ಲ. 

\vspace{-0.2cm}

\begin{verse}
ತದಾ ಸರ್ವಾವರಣಮಲಾಪೇತಸ್ಯ ಜ್ಞಾನಸ್ಯಾನಂತ್ಯಾತ್​  ಜ್ಞೇಯಮಲ್ಪಮ್​~॥~೩೦~॥
\end{verse}

\vspace{-0.3cm}

\dsize{ಆಗ ಜ್ಞಾನವು ಆವರಣ ಮತ್ತು ಮಲಿನತೆಯಿಂದ ಮುಕ್ತವಾಗಿ ಅನಂತವಾದ ಮೇಲೆ, ಜ್ಞೇಯವು ಅಲ್ಪವಾಗುವುದು. }

\vspace{0.1cm}

ಜ್ಞಾನವು ಅಲ್ಲೇ ಇರುವುದು; ಅದನ್ನು ಮರೆಮಾಡಿದ್ದ ತೆರೆ ಮಾತ್ರ ಸರಿಯುವುದು. ಒಂದು ಬೌದ್ಧಗ್ರಂಥವು, ಬುದ್ಧನೆಂದರೆ ಅದು ಒಂದು ಸ್ಥಿತಿಯ ಹೆಸರು, ಆಕಾಶದಂತೆ ವಿಶಾಲವಾಗಿರುವ ಅನಂತಜ್ಞಾನ ಎನ್ನುತ್ತದೆ. ಏಸುವು ಅದನ್ನು ಹೊಂದಿ ಕ್ರಿಸ್ತನಾದನು. ನೀವೆಲ್ಲರೂ ಕೂಡ ಆ ಸ್ಥಿತಿಯನ್ನು ಪಡೆಯುವಿರಿ. ಜ್ಞಾನವು ಅನಂತವಾಗಿ ಜ್ಞೇಯವು ಅಲ್ಪವಾಗುವುದು. ಜ್ಞಾನಕ್ಕೆ ವಿಷಯವಾದ ಇಡೀ ವಿಶ್ವವೂ ಪುರುಷನೆದುರಿಗೆ ತೃಣಸಮಾನವಾಗುವುದು. ಸಾಧಾರಣ ಮಾನವನು ತಾನು ಅತ್ಯಲ್ಪನೆಂದು ತಿಳಿಯುತ್ತಾನೆ. ಏಕೆಂದರೆ ಅವನಿಗೆ ಜ್ಞೇಯವು ಅನಂತವಾಗಿ ಕಾಣುವುದು.

\vspace{-0.3cm}

\begin{verse}
ತತಃ ಕೃತಾರ್ಥಾನಾಂ ಪರಿಣಾಮಕ್ರಮಸಮಾಪ್ತಿರ್ಗುಣಾನಾಮ್​~॥ ೩೧~॥
\end{verse}

\vspace{-0.33cm}

\dsize{ತಮ್ಮ ಗುರಿಯನ್ನು ಸೇರಿದಮೇಲೆ ಒಂದಾದ ಮೇಲೊಂದು ಬರುವ ಗುಣಗಳ ವಿಕಾರ ಮುಗಿಯುವುದು.}

\vspace{-0.2cm}

\begin{verse}
ಕ್ಷಣಪ್ರತಿಯೋಗೀ ಪರಿಣಾಮಾಪರಾಂತನಿರ್ಗ್ರಾಹ್ಯಃ ಕ್ರಮಃ~॥ ೩೨~॥
\end{verse}

\vspace{-0.33cm}

\dsize{ಕ್ಷಣಸಂಬಂಧದಿಂದ ಇರುವ ಬದಲಾವಣೆ ಮತ್ತು ಮತ್ತೊಂದು ಕೊನೆಯಲ್ಲಿ (ಬದಲಾವಣೆಯ ಅಂತ್ಯದಲ್ಲಿ) ಕಾಣಿಸಿಕೊಳ್ಳುವುದು – ಇದೇ ಕ್ರಮ. }

\vspace{0.1cm}

ಇಲ್ಲಿ ಪತಂಜಲಿಯು ಕ್ರಮ ಅಥವಾ ಕ್ಷಣಸಂಬಂಧದಿಂದ ಇರುವ ಬದಲಾವಣೆಯನ್ನು ಹೇಳುತ್ತಾನೆ. ನಾನು ಆಲೋಚಿಸುತ್ತಿರುವಾಗ ಎಷ್ಟೋ ಕ್ಷಣಗಳು ಕಳೆದು ಹೋಗುತ್ತವೆ. ಪ್ರತಿಯೊಂದು ಕ್ಷಣದಲ್ಲಿಯೂ ಭಾವನೆಯ ಬದಲಾವಣೆ ಇದೆ. ಆದರೆ ನಾನು ಈ ಬದಲಾವಣೆಗಳನ್ನೆಲ್ಲ ಒಂದು ಸರಣಿಯ ಅಂತ್ಯದಲ್ಲಿ ಕಾಣುತ್ತೇವೆ. ಇದನ್ನೇ ನಾವು ಕ್ರಮವೆನ್ನುವುದು. ಆದರೆ ಸರ್ವವ್ಯಾಪಿತ್ವವನ್ನು ಪಡೆದ ಮನಸ್ಸಿಗೆ ಕ್ರಮವಿಲ್ಲ. ಎಲ್ಲವೂ ಆ ಮನಸ್ಸಿಗೆ ವರ್ತಮಾನವಾಗುತ್ತವೆ. ಅದಕ್ಕೆ ವರ್ತಮಾನಕಾಲವೊಂದೆ ಇರುವುದು, ಭೂತ ಭವಿಷ್ಯತ್ತು ಇಲ್ಲ. ಕಾಲವು ಅದರ ಸ್ವಾಧೀನವಾಗುವುದು. ಜ್ಞಾನವೆಲ್ಲ ಕ್ಷಣ ಒಂದರಲ್ಲಿದೆ. ಎಲ್ಲವೂ ಒಂದು ಮಿಂಚಿನಂತೆ ಹೊಳೆಯುವುದು.

\vspace{-0.2cm}

\begin{verse}
ಪುರುಷಾರ್ಥಶೂನ್ಯಾನಾಂ ಗುಣಾನಾಂ ಪ್ರತಿಪ್ರಸವಃ ಕೈವಲ್ಯಂ \\ಸ್ವರೂಪಪ್ರತಿಷ್ಠಾ ವಾ ಚಿತಿಶಕ್ತೇರಿತಿ~\hfill{॥ ೩೩~॥}
\end{verse}

\vspace{-0.3cm}

\dsize{ಪುರುಷನಿಗಾಗಿ ಮಾಡಬೇಕಾದುದು ಏನೂ ಇಲ್ಲದೆ ಗುಣಗಳು ಕಾರಣದಲ್ಲಿ ಲಯವಾಗುವುದೇ ಕೈವಲ್ಯ. ಅಥವಾ ಪುರುಷನು ತನ್ನ ಸ್ವರೂಪದಲ್ಲಿ ಪ್ರತಿಷ್ಠಿತನಾಗುವುದೇ ಕೈವಲ್ಯ. }

\vfill\eject

ನಮ್ಮನ್ನು ಪ್ರೀತಿವಾತ್ಸಲ್ಯದಿಂದ ನೋಡುತ್ತಿದ್ದ ಪ್ರಕೃತಿ ಎಂಬ ದಾದಿಗೆ ತಾನಾಗಿ ನಿಃಸ್ವಾರ್ಥದಿಂದ ವಹಿಸಿಕೊಂಡಿದ್ದ ಕೆಲಸ ಮುಗಿಯಿತು. ತನ್ನ ನೈಜ ಸ್ವಭಾವವನ್ನು ಮರೆತಿದ್ದ ಜೀವನನ್ನು ತನ್ನ ಕೈಗಳಿಂದ ಮೃದುವಾಗಿ ಹಿಡಿದುಕೊಂಡಳೊ ಎಂಬುವಂತೆ, ಅವನಿಗೆ ಎಲ್ಲಾ ಅನುಭವಗಳನ್ನು ಮತ್ತು ವಸ್ತುವನ್ನು ತೋರಿ, ಅವನಲ್ಲಿ ಮಾಯವಾದ ಮಾಹಾತ್ಮ್ಯ ಪುನಃ ಬಂದು ತನ್ನ ಸ್ವಭಾವವನ್ನು ತಿಳಿದುಕೊಳ್ಳುವವರೆಗೆ, ಅವನನ್ನು ಅನೇಕ ವಿಧದಿಂದ ಮೇಲಮೇಲಕ್ಕೆ ತಂದಳು. ಅನಂತರ ದಾರಿಯಿಲ್ಲದ, ಜೀವನದ ಮರುಳುಕಾಡಿನಲ್ಲಿ ತಮ್ಮ ಗುರಿಯನ್ನು ಮರೆತ, ಇತರ ಮಕ್ಕಳಿಗೆ ದಾರಿ ತೋರುವುದಕ್ಕಾಗಿ ದಯಾಮಯಿಯಾದ ಅವಳು ಬಂದ ದಾರಿಯಿಂದ ಹಿಂತಿರುಗಿದಳು. ಕೊನೆಮೊದಲಿಲ್ಲದೆ ಅವಳು ಹೀಗೆ ಕೆಲಸ ಮಾಡುತ್ತಿರುವಳು. ಸುಖದುಃಖಗಳ ಮೂಲಕ, ಪಾಪಪುಣ್ಯಗಳ ಮೂಲಕ, ಅನಾದಿಯಾದ ಜೀವಿಗಳಿಂದ ತುಂಬಿರುವ ಈ ನದಿಯು, ಪರಿಪೂರ್ಣತೆಯ ಸಾಗರದೆಡೆಗೆ ಹರಿಯುತ್ತಿರುವುದು.

\begin{center}
ತಮ್ಮ ನೈಜಸ್ವಭಾವವನ್ನು ಅರಿತ ಮಹಾತ್ಮರಿಗೆ ಮಂಗಳ. \\ ಸರ್ವರಿಗೂ ಅವರ ಆಶೀರ್ವಾದದ ಬಲವಿರಲಿ. 
\end{center}

