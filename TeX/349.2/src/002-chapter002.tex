

\part{ಅನುಷ್ಠಾನ ವೇದಾಂತ}

\chapter{ಅಧ್ಯಾಯ ೧}

\begin{center}
\textbf{(ಲಂಡನ್ನಿನಲ್ಲಿ ೧೮೯೬ರ ನವೆಂಬರ್​ ೧೦ ರಂದು ನೀಡಿದ ಉಪನ್ಯಾಸ)}
\end{center}

ವೇದಾಂತ ತತ್ತ್ವದ ಅನುಷ್ಠಾನದ ವಿಚಾರವಾಗಿ ಏನಾದರೂ ಹೇಳಬೇಕೆಂದು ನನ್ನನ್ನು ಕೇಳಿಕೊಂಡಿರುವರು. ನಾನು ನಿಮಗೆ ಹೇಳಿದಂತೆ ಸಿದ್ಧಾಂತವೇನೋ ಬಹಳ ಚೆನ್ನಾಗಿದೆ. ಆದರೆ ಅದನ್ನು ನಾವು ಅನುಷ್ಠಾನಕ್ಕೆ ತರುವುದು ಹೇಗೆ? ಅದನ್ನು ಅನುಷ್ಠಾನಕ್ಕೆ ತರಲು ಸ್ವಲ್ಪವೂ ಸಾಧ್ಯವಿಲ್ಲದೇ ಇದ್ದರೆ ಯಾವ ಸಿದ್ಧಾಂತದಿಂದಲೂ ಏನೂ ಪ್ರಯೋಜನವಿಲ್ಲ. ಅದು ಒಂದು ಕೇವಲ ಬುದ್ಧಿವಂತಿಕೆಯ ಕಸರತ್ತಾಗ ಬಹುದು. ಆದಕಾರಣ ವೇದಾಂತವು ಧಾರ್ಮಿಕ ದೃಷ್ಟಿಯಿಂದ ಸಂಪೂರ್ಣ ಅನುಷ್ಠಾನ ಸಾಧ್ಯವಾಗಬೇಕು. ನಮ್ಮ ಜೀವನದ ಪ್ರತಿಯೊಂದು ಕಾರ್ಯಕ್ಷೇತ್ರದಲ್ಲಿಯೂ, ಅದನ್ನು ಅನುಷ್ಠಾನಕ್ಕೆ ತರಲು ನಮಗೆ ಸಾಧ್ಯವಾಗಬೇಕು. ಇದೊಂದೇ ಅಲ್ಲ. ಆಧ್ಯಾತ್ಮಿಕ ಜೀವನ ಮತ್ತು ವ್ಯಾವಹಾರಿಕ ಜೀವನಕ್ಕೆ ಇರುವ ಕಾಲ್ಪನಿಕ ಭೇದಭಾವ ಮಾಯವಾಗಬೇಕು. ಏಕೆಂದರೆ ವೇದಾಂತವು ಏಕತ್ವವನ್ನು, ಎಲ್ಲೆಡೆಯಲ್ಲಿಯೂ ಇರುವ ಏಕ ಜೀವನವನ್ನು ಉಪದೇಶಿಸುತ್ತದೆ. ಧರ್ಮದ ಆದರ್ಶವು ನಮ್ಮ ಜೀವನದ ಎಲ್ಲಾ ಕಾರ್ಯಕ್ಷೇತ್ರಗಳನ್ನು ಆವರಿಸಬೇಕು; ನಮ್ಮ ಆಲೋಚನೆಗಳೆಲ್ಲವನ್ನೂ ಅದು ಪ್ರವೇಶಿಸಬೇಕು; ಹೆಚ್ಚು ಹೆಚ್ಚಾಗಿ ಅನುಷ್ಠಾನದಲ್ಲಿ ಬಳಕೆಗೆ ಬರಬೇಕು. ನಾನು ಮುಂದುವರಿದಂತೆ ಕ್ರಮೇಣ ಇವುಗಳ ಅನುಷ್ಠಾನವನ್ನು ಕುರಿತಾಗಿ ಹೇಳುತ್ತೇನೆ. ಆದರೆ ಈ ಕೆಲವು ಮೂಲ ಉಪನ್ಯಾಸಗಳು ಮೊದಲು ಬೇಕಾಗಿವೆ. ಆದಕಾರಣ ಮೊದಲು ಸಿದ್ಧಾಂತವನ್ನು ತಿಳಿದುಕೊಂಡು, ಅನಂತರ ಅದು ಕಾನನಗಳಿಂದ ಹಿಡಿದು, ನಗರದ ಗಡಿಬಿಡಿಯ ದಾರಿಗಳಲ್ಲಿಯೂ ಕೂಡ ಹೇಗೆ ಅನುಷ್ಠಾನಕ್ಕೆ ಬಂದಿದೆ ಎಂಬುದನ್ನು ತಿಳಿದುಕೊಳ್ಳೋಣ. ಇದರಲ್ಲಿ ನಮಗೆ ಕಾಣುವ ಒಂದು ವಿಶೇಷ ಸಂಗತಿ ಏನೆಂದರೆ, ಇಲ್ಲಿಯ ಅನೇಕ ಮಹದಾಲೋಚನೆಗಳು ಕಾನನಾಂತರಗಳಲ್ಲಿ ವಿರಾಮದಿಂದ ಬಾಳುತ್ತಿದ್ದ ಮುನಿವರರಿಂದ ಬಂದವುಗಳಲ್ಲ. ನಾವು ಯಾರನ್ನು ಸ್ವಲ್ಪವೂ ವಿರಾಮವಿಲ್ಲದೆ ಅವಿಶ್ರಾಂತವಾಗಿ ದುಡಿಯುತ್ತಿದ್ದರು ಎಂದು ಹೇಳು ತ್ತೇವೆಯೋ ಅಂತಹ, ರಾಜ್ಯಭಾರ ಮಾಡುತ್ತಿದ್ದ, ಆಳರಸದಿಂದ ಬಂದಿವೆ.

ಶ್ವೇತಕೇತುವು ಆರುಣಿ ಋಷಿಯ ಮಗ. ಬಹುಶಃ ಅವನು ಕಾಡಿನಲ್ಲಿ ಹುಟ್ಟಿ ಬೆಳೆದವನು. ಪಾಂಚಾಲ ದೇಶಕ್ಕೆ ಹೋಗಿ ಪ್ರವಾಹನ ಜೈವಲಿ ಎಂಬ ರಾಜನ ಆಸ್ಥಾನದಲ್ಲಿ ಕಾಣಿಸಿಕೊಂಡನು. ರಾಜನು, “ಮರಣ ಕಾಲದಲ್ಲಿ ಜೀವಿಗಳು ಹೇಗೆ ಹೋಗುತ್ತವೆ ಎಂಬುದು ಗೊತ್ತೆ?” ಎಂದು ಕೇಳಿದನು. ಅದಕ್ಕೆ “ಇಲ್ಲ, ಮಹಾ ಶಯರೆ” ಎಂದು ಉತ್ತರ ಕೊಟ್ಟನು. ಅನಂತರ ರಾಜ, “ಅವರು ಪುನಃ ಹೇಗೆ ಹಿಂತಿ ರುಗುತ್ತಾರೆ ಎಂಬುದು ಗೊತ್ತಿದೆಯೆ?” ಎಂದನು. ಅದಕ್ಕೂ ಇಲ್ಲವೆಂದು ಉತ್ತರ ಕೊಟ್ಟನು. “ದೇವಯಾನ, ಪಿತೃಯಾನಗಳು ಗೊತ್ತಿವೆಯೆ?” ಎಂದು ಪ್ರಶ್ನಿಸಿದನು. ಅದಕ್ಕೂ ಇಲ್ಲ ಎಂದು ಹೇಳಿದನು. ಅನಂತರ ರಾಜನು ಬೇರೆ ಪ್ರಶ್ನೆಗಳನ್ನು ಕೇಳಿದನು. ಶ್ವೇತಕೇತುವಿಗೆ ಉತ್ತರ ಕೊಡಲು ಆಗಲಿಲ್ಲ. ಆದಕಾರಣ ದೊರೆ ಆತನಿಗೆ ನಿನಗೇನೂ ಗೊತ್ತಿಲ್ಲವೆಂದು ಹೇಳಿದನು. ಹುಡುಗನು ತಂದೆಯ ಸಮೀಪಕ್ಕೆ ಹಿಂತಿರುಗಿದನು. ತಂದೆಯು ಇಂತಹ ಪ್ರಶ್ನೆಗಳಿಗೆ ತಾನೂ ಉತ್ತರ ಕೊಡಲು ಆಗುತ್ತಿರಲಿಲ್ಲವೆಂದು ಒಪ್ಪಿಕೊಂಡನು. ಇಂತಹ ಪ್ರಶ್ನೆಗಳಿಗೆ ಉತ್ತರವನ್ನು ಹೇಳಬಾರದೆಂದು ಅಲ್ಲ, ಅಥವಾ ತನ್ನ ಹುಡುಗನಿಗೆ ಇಂತಹ ಪ್ರಶ್ನೆಗಳಿಗೆ ಉತ್ತರವನ್ನು ಕೊಡಬಾರದೆಂದೂ ಅಲ್ಲ, ಅಥವಾ ಮನಸ್ಸೂ ಇರಲಿಲ್ಲವೆಂದಲ್ಲ. ಆದರೆ ಅವನಿಗೆ ಈ ವಿಷಯಗಳು ತಿಳಿದಿರ ಲಿಲ್ಲ. ಆದಕಾರಣ ಶ್ವೇತಕೇತುವು ತಂದೆಯೊಂದಿಗೆ ರಾಜನಲ್ಲಿಗೆ ಮರಳಿದನು. ಇಬ್ಬರೂ, ತಮಗೆ ಆ ರಹಸ್ಯವನ್ನು ಹೇಳಿಕೊಡಬೇಕೆಂದು ಕೇಳಿಕೊಂಡರು. ಅದಕ್ಕೆ ರಾಜನು “ಇದುವರೆವಿಗೂ, ಈ ರಹಸ್ಯವು ರಾಜರಿಗೆ ಮಾತ್ರ ತಿಳಿದಿತ್ತು; ಬ್ರಾಹ್ಮಣ ರಿಗೆ ತಿಳಿದಿರಲಿಲ್ಲ” ಎಂದು ಹೇಳಿದನು. ಆದರೂ ಆತನು ಯಾವುದನ್ನು ಅವರು ತಿಳಿಯಬೇಕೆಂದು ಅಪೇಕ್ಷಿಸಿದರೋ, ಅದನ್ನು ಕಲಿಸಲು ಒಪ್ಪಿಕೊಂಡನು. ಅನೇಕ ಉಪನಿಷತ್ತುಗಳಲ್ಲಿ ಈ ವೇದಾಂತ ವಿಚಾರಗಳು, ಕೇವಲ ಕಾಡಿನಲ್ಲಿ ಮಾಡಿದ ತಪಸ್ಸಿನ ಫಲಮಾತ್ರವಲ್ಲ ಎನ್ನುವುದು ನಮಗೆ ತೋರುತ್ತದೆ. ಅದರಲ್ಲಿ ಬಹಳ ಮುಖ್ಯ ವಾದ ಭಾಗಗಳನ್ನು ಆಲೋಚನೆ ಮಾಡಿ ಅದನ್ನು ಬೋಧಿಸಿದವರು, ಜೀವನದ ಅನುದಿನದ ಕಾರ್ಯಕ್ರಮದಲ್ಲಿ ಸಿಕ್ಕಿ ಬಹಳ ಕಷ್ಟಪಟ್ಟು ಕೆಲಸ ಮಾಡುತ್ತಿದ್ದ ಮೇಧಾವಿ ಗಳು. ಲಕ್ಷಾಂತರ ಜನರನ್ನು ಆಳುತ್ತಿದ್ದ ನಿರಂಕುಶ ಚಕ್ರವರ್ತಿಗಿಂತ ಹೆಚ್ಚು ಕೆಲಸ ಉಳ್ಳವರನ್ನು ನಾವು ಕಲ್ಪಿಸಿಕೊಳ್ಳುವುದು ಬಹಳ ಕಷ್ಟ. ಆದರೂ ಇಂತಹ ರಾಜರು ಗಳಲ್ಲಿ ಕೆಲವರು ದೊಡ್ಡ ತತ್ತ್ವಜ್ಞಾನಿಗಳಾಗಿದ್ದರು.

ಈ ವೇದಾಂತವು ಅನುಷ್ಠಾನಯೋಗ್ಯವಾಗಿತ್ತು ಎಂಬುದನ್ನು ಎಲ್ಲವೂ ತೋರು ವುವು. ಕ್ರಮೇಣ ನಾವು ಭಗವದ್ಗೀತೆಯ ಕಾಲಕ್ಕೆ ಬರುತ್ತೇನೆ–ಬಹುಶಃ ಹಲವರು ಇದನ್ನು ಓದಿರಬಹುದು. ಇದು ವೇದಾಂತ ತತ್ತ್ವಶಾಸ್ತ್ರದ ಮೇಲೆ ಬರೆದ ಅತ್ಯುತ್ತಮ ಭಾಷ್ಯ. ಇಲ್ಲೊಂದು ವಿಚಿತ್ರವೇನೆಂದರೆ, ಗೀತೋಪದೇಶದ ಹಿನ್ನೆಲೆ ರಣರಂಗ. ಅಲ್ಲಿಯೇ ಶ‍್ರೀಕೃಷ್ಣನು ಅರ್ಜುನನಿಗೆ ತನ್ನ ತತ್ತ್ವವನ್ನು ಹೇಳುವುದು. ಅಲ್ಲಿ ಪ್ರತಿ ಯೊಂದು ಪುಟದಲ್ಲಿಯೂ ಕಂಗೊಳಿಸುವ ಸಿದ್ಧಾಂತವೆಂದರೆ ತೀವ್ರ ಚಟುವಟಿಕೆ ಆದರೆ ಇವುಗಳ ಅಂತರಾಳದಲ್ಲಿ ಅನಂತ ಶಾಂತಿ ಇದೆ. ಇದೇ ಕರ್ಮರಹಸ್ಯ. ಇದನ್ನು ಹೊಂದುವುದೇ ವೇದಾಂತದ ಗುರಿ. ನಾವು ಸೋಮಾರಿತನ ಎಂದು ಭಾವಿಸುವ ಅಕರ್ಮವೇ ನಿಜವಾಗಿಯೂ ಎಂದಿಗೂ ಗುರಿಯಾಗಲಾರದು. ಇಂತಹ ಅಕರ್ಮವೇ ಆದರ್ಶವಾಗಿದ್ದ ಪಕ್ಷದಲ್ಲಿ ನಮ್ಮ ಸುತ್ತಲಿರುವ ಗೋಡೆಗಳು ಬಹಳ ಮೇಧಾವಿಗಳಾ ಗಬೇಕಾಗಿತ್ತು; ಏಕೆಂದರೆ ಅವು ಕರ್ಮ ಮಾಡುವುದಿಲ್ಲ. ಮಣ್ಣಿನ ಮುದ್ದೆ, ಮರದ ತುಂಡು ಇವು ಪ್ರಪಂಚದ ಪರಮಶ್ರೇಷ್ಠರಾದ ಋಷಿಗಳಾಗಿರಬೇಕಾಗಿತ್ತು. ಏಕೆಂದರೆ ಅವು ಯಾವ ಕರ್ಮವನ್ನೂ ಮಾಡುತ್ತಿಲ್ಲ. ಅಥವಾ ಅಕರ್ಮವೇ ಆಸಕ್ತಿಯೊಂದಿಗೆ ಬೆರೆತರೆ, ಅದು ಕೂಡ ಕರ್ಮವಾಗುವುದಿಲ್ಲ. ವೇದಾಂತದ ಗುರಿಯಾದ ಕರ್ಮವು ಅನಂತರ ಶಾಂತಿಯೊಂದಿಗೆ ಕಲೆತಿರುವುದು. ಏನಾದರೂ ಆ ಶಾಂತಿಗೆ ಭಂಗ ಬರುವುದಿಲ್ಲ, ಮನಸ್ಸಿನ ಸಮತ್ವಕ್ಕೆ ತಡೆಯುಂಟಾಗುವುದಿಲ್ಲ. ಕೆಲಸ ಮಾಡುವುದಕ್ಕೆ ಅದೇ ಉತ್ತಮ ಸ್ಥಿತಿ ಎಂಬುದನ್ನು ಜೀವನದ ಅನುಭವದಿಂದ ತಿಳಿದುಕೊಂಡಿರುವೆವು.

ನಮಗೆ ಆಸೆ ಇಲ್ಲದೇ ಇದ್ದರೆ ಹೇಗೆ ಕೆಲಸ ಮಾಡಲು ಸಾಧ್ಯವೆಂದು ಅನೇಕರು ಹೇಳುತ್ತಾರೆ. ಕೆಲವು ವರುಷಗಳ ಹಿಂದೆ ನಾನೂ ಕೂಡ ಹಾಗೆ ಆಲೋಚಿಸಿದ್ದೆ. ಆದರೆ ನನಗೆ ವಯಸ್ಸಾದಂತೆಲ್ಲ, ನಾನು ಹೆಚ್ಚು ಅನುಭವಗಳನ್ನು ಪಡೆದಂತೆಲ್ಲ ಅದು ನಿಜವಲ್ಲವೆಂದು ತೋರುತ್ತಿದೆ. ಆಸಕ್ತಿ ಕಡಿಮೆಯಾದಷ್ಟೂ ಉತ್ತಮವಾಗಿ ಕೆಲಸಮಾಡುವೆವು. ಶಾಂತವಾಗಿದ್ದಷ್ಟೂ ನಮಗೆ ಉತ್ತಮ. ಹೆಚ್ಚು ಕೆಲಸವನ್ನು ಮಾಡುವುದೂ ನಮಗೆ ಸಾಧ್ಯವಾಗುತ್ತದೆ. ನಮ್ಮ ಭಾವನೆಯ ಉದ್ವೇಗವನ್ನು ಯಾವ ತಡೆಯೂ ಇಲ್ಲದೆ ಹರಿಯಲು ಬಿಟ್ಟರೆ ಅದರಿಂದ ಎಷ್ಟೋ ಶಕ್ತಿ ವ್ಯಯವಾಗುವುದು, ನಮ್ಮ ನರಗಳು ಬಲಹೀನವಾಗುವುವು, ಮನಸ್ಸಿಗೆ ಅಶಾಂತಿ ಉಂಟಾಗುವುದು. ಆಗ ಅತಿ ಅಲ್ಪ ಕಾರ್ಯವನ್ನು ಮಾತ್ರ ಸಾಧಿಸಬಹುದು. ಯಾವ ಶಕ್ತಿಯು ಕಾರ್ಯರೂಪದಲ್ಲಿ ಹೊರದೋರಬೇಕಾಗಿತ್ತೊ ಅದು ಯಾವುದಕ್ಕೂ ಪ್ರಯೋಜನವಿಲ್ಲದ ಉದ್ವೇಗದಲ್ಲಿ ಕೊನೆಗಾಣುವುದು. ಮನಸ್ಸು ಶಾಂತವಾಗಿ ಏಕಾಗ್ರವಾದಾಗ ಮಾತ್ರ ಅದರ ಎಲ್ಲಾ ಶಕ್ತಿಯನ್ನೂ ಒಳ್ಳೆಯ ಕಾರ್ಯರೂಪದಲ್ಲಿ ಉಪಯೋಗಿಸಬಹುದು. ಜಗತ್ತಿನಲ್ಲಿ ಜನ್ಮವೆತ್ತಿದ್ದ ಪ್ರಚಂಡ ಕಾರ್ಯೋನ್ಮುಖರಾದ ವ್ಯಕ್ತಿಗಳ ಜೀವನವನ್ನು ಓದಿದರೆ ಅವರು ಅಷ್ಟೇ ಅದ್ಭುತವಾದ ಶಾಂತಜೀವಿಗಳು ಆಗಿದ್ದರೆಂಬುದನ್ನು ನೋಡುತ್ತೇವೆ. ಯಾವುದರಿಂದಲೂ ಅವರ ಮನಸ್ಸಿನ ಸಮತ್ವಕ್ಕೆ ಭಂಗ ಬರುವಂತೆ ಇರಲಿಲ್ಲ. ಆದಕಾರಣವೆ ಯಾರು ಕೋಪ ತಾಳುವರೋ, ಅವರು ಎಂದಿಗೂ ಹೆಚ್ಚು ಕೆಲಸ ಮಾಡಲಾರರು. ಯಾರು ಏನಾದರೂ ಕೋಪಗೊಳ್ಳಲಾರರೊ ಅವರು ಎಷ್ಟೋ ಕೆಲಸಗಳನ್ನು ಸಾಧಿಸುತ್ತಾರೆ. ಯಾವ ಮನುಷ್ಯನು, ಕೋಪ ಅಸೂಯೆ ಮತ್ತು ಇನ್ನೂ ಇತರ ಯಾವುದಾದರೂ ಹೀನ ಗುಣಕ್ಕೆ ಬೀಳುತ್ತಾನೆಯೋ, ಅವನು ಕೆಲಸ ಮಾಡಲಾರ. ತನ್ನ ಮನಸ್ಸು ಚದುರಿಹೋಗುವುದೇ ಹೊರತು ಕಾರ್ಯತಃ ಏನೂ ಮಾಡಲಾರ. ಯಾರು ಶಾಂತರೋ, ಕ್ಷಮಾಜೀವಿಗಳೋ, ಎಲ್ಲರನ್ನೂ ಒಂದೇ ದೃಷ್ಟಿಯಿಂದ ನೋಡುವರೋ, ಸಮತ್ವಬುದ್ಧಿಯುಳ್ಳವರೋ, ಅವರೇ ಹೆಚ್ಚು ಪಾಲು ಕೆಲಸ ಮಾಡುವರು.

ವೇದಾಂತವು ನಮಗೆ ಆದರ್ಶವನ್ನು ಬೋಧಿಸುತ್ತದೆ. ನಮಗೆ ಗೊತ್ತಿರುವಂತೆ ಯಾವಾಗಲೂ ಆದರ್ಶವು ಇಂದಿನ ನಿಜಸ್ಥಿತಿಗಿಂತ, ಎಂದರೆ ಇಂದು ಯಾವುದು ಅನುಷ್ಠಾನದಲ್ಲಿ ಇದೆಯೋ, ಅದಕ್ಕಿಂತ ಮೇಲಿರುವುದು. ಮಾನವನಲ್ಲಿ ಎರಡು ಸ್ವಭಾವಗಳಿವೆ. ಒಂದು, ಆದರ್ಶವನ್ನು ನಮ್ಮ ಜೀವನದೊಂದಿಗೆ ರಾಜಿಮಾಡಿಕೊಳ್ಳು ವುದು. ಮತ್ತೊಂದು, ನಮ್ಮ ಜೀವನವನ್ನು ಆದರ್ಶದ ಮೇಲಿನ ಮಟ್ಟಕ್ಕೆ ಏರಿಸುವುದು. ಇದನ್ನು ತಿಳಿದುಕೊಳ್ಳುವುದು ಬಹಳ ಮುಖ್ಯ. ಏಕೆಂದರೆ, ಮೊದಲನೆಯ ಸ್ವಭಾವವೇ ನಮ್ಮ ಜೀವನವನ್ನು ಮೋಹಪಾಶಕ್ಕೆ ಕೆಡಹುವುದು. ನನಗೆ ಯಾವುದೋ ಒಂದು ವಿಧದ ಕೆಲಸ ಮಾಡುವುದಕ್ಕೆ ಬರುತ್ತದೆ ಎಂದು ಆಲೋಚಿಸುವೆನು. ಆ ಕೆಲಸದಲ್ಲಿ ಬಹುಶಃ ಮುಕ್ಕಾಲುಪಾಲು ಕೆಟ್ಟದ್ದು; ಬಹುಶಃ ಆ ಕೆಲಸಕ್ಕೆ ಕೋಪ, ಆಸೆ, ಸ್ವಾರ್ಥ, ಇವೇ ಉತ್ತೇಜನಕಾರಿ. ಈಗ ನನಗೆ ಯಾವುದಾದರೂ ಒಂದು ಆದರ್ಶವನ್ನು ಬೋಧಿಸುವುದಕ್ಕೆ ಯಾರಾದರೂ ಬಂದು, ಮೊದಲು ಸ್ವಾರ್ಥವನ್ನೂ, ಭೋಗಾಪೇಕ್ಷೆಯನ್ನೂ ತೊರೆಯಬೇಕೆಂದು ಬೋಧಿಸಿದರೆ, ಅದು ಅನುಷ್ಠಾನಕ್ಕೆ ತರಲು ಸಾಧ್ಯವಿಲ್ಲವೆಂದು ನಾನು ಯೋಚಿಸುತ್ತೇನೆ. ಆದರೆ ನನ್ನ ಸ್ವಾರ್ಥದೊಂದಿಗೆ ರಾಜಿಮಾಡಿಕೊಳ್ಳುವಂತಹ ಆದರ್ಶವನ್ನು ಯಾರಾದರೂ ತಂದರೆ ನನಗೆ ತುಂಬಾ ಸಂತೋಷ. ಅದನ್ನು ಸ್ವೀಕರಿಸಲು ಆನಂದದಿಂದ ಕುಣಿದಾಡುವೆನು. ಅದೇ ನನಗೆ ಬೇಕಾದ ಆದರ್ಶ. ಸಂಪ್ರದಾಯ ಎನ್ನುವ ಪದವನ್ನು ಅದು ಅನೇಕ ಅರ್ಥಗಳನ್ನು ಕೊಡುವಂತೆ ಹೇಗೆ ತಿರುಚಲಾಗಿದೆಯೋ, ಹಾಗೆಯೇ ಅನುಷ್ಠಾನವೆಂಬ ಪದವನ್ನೂ ಕೂಡ. “ನನ್ನ ಸಂಪ್ರದಾಯವು ಸತ್ಸಂಪ್ರದಾಯ”, “ನಿನ್ನ ಸಂಪ್ರದಾಯವು ದುಸ್ಸಂಪ್ರದಾಯ.” ಇದರಂತೆಯೇ ಅನುಷ್ಠಾನ ಕೂಡ. ನನಗೆ ಯಾವುದು ಅನು ಷ್ಠಾನಕ್ಕೆ ಯೋಗ್ಯವೆಂದು ತೋರುತ್ತದೆಯೋ, ಅದೊಂದೇ ಪ್ರಪಂಚದಲ್ಲಿ ಅನು ಷ್ಠಾನಕ್ಕೆ ಯೋಗ್ಯ. ನಾನೊಬ್ಬ ವ್ಯಾಪಾರಸ್ಥನಾದರೆ, ಪ್ರಪಂಚದಲ್ಲಿ ವ್ಯಾಪಾರವೊಂದೇ ಅನುಷ್ಠಾನಯೋಗ್ಯವೆಂದು ತಿಳಿಯುತ್ತೇನೆ. ನಾನು ಒಬ್ಬ ಕಳ್ಳನಾಗಿದ್ದರೆ, ಕಳ್ಳತನ ಮಾಡುವುದೊಂದೇ ಪ್ರಪಂಚದಲ್ಲಿ ಉತ್ತಮವಾದ ವ್ಯವಹಾರವೆಂದೂ, ಅದನ್ನು ಮಾಡದವರು ವ್ಯವಹಾರಚತುರರಲ್ಲವೆಂದೂ, ನಾನು ತಿಳಿಯುತ್ತೇನೆ. ನೋಡಿದಿರಾ! ಈ ಅನುಷ್ಠಾನವೆಂಬ ಪದವನ್ನು ನಾವುಗಳು ಯಾವುದನ್ನು ಮಾಡಬಲ್ಲೆವೋ, ಯಾವ ಕೆಲಸದ ಮೇಲೆ ನಮಗೆ ಪ್ರೀತಿಯೋ, ಅದಕ್ಕೆ ಉಪಯೋಗಿಸುತ್ತೇವೆ! ಆದಕಾರಣ ವೇದಾಂತವು ಎಷ್ಟು ಅನುಷ್ಠಾನ ಸಾಧ್ಯವಾದರೂ, ಅದು ಆದರ್ಶದ ದೃಷ್ಟಿಯಿಂದ ಎಂಬುದನ್ನು ತಿಳಿದುಕೊಳ್ಳಬೇಕೆಂದು ನಿಮಗೆ ಹೇಳುತ್ತೇನೆ. ಎಷ್ಟು ಮೇಲಿದ್ದರೂ, ವೇದಾಂತವು ಒಂದು ಅಸಾಧ್ಯವಾದ ಆದರ್ಶವನ್ನು ಬೋಧಿಸುವುದಿಲ್ಲ. ಅದು ಆದರ್ಶಕ್ಕೆ ಯೋಗ್ಯವಾದಷ್ಟು ಮೇಲಿರುವುದು. ಒಂದು ಮಾತಿನಲ್ಲಿ ಹೇಳಬೇಕಾದರೆ ಈ ಆದರ್ಶವೇ ನೀವು ಪವಿತ್ರರೆಂಬುದು. “ತತ್ತ್ವಮಸಿ” ಇದೇ ವೇದಾಂತ ಸಾರ. ಅನೇಕ ಶಾಖೋಪಶಾಖೆಗಳನ್ನು ನೋಡಿದ ಮೇಲೆ, ಬುದ್ಧಿಯ ಗರಡಿಯ ಮನೆಯಲ್ಲಿ ಪಳಗಿದ ಮೇಲೆ, ಮಾನವನ ಆತ್ಮವೇ ಶುದ್ಧವೆಂದೂ, ಸರ್ವಜ್ಞವೆಂದೂ ನಿಮಗೆ ಗೊತ್ತಾಗುವುದು. ಆತ್ಮನ ವಿಚಾರದಲ್ಲಿ ಜನನ ಮರಣ ಮುಂತಾದ ಮೂಢನಂಬಿಕೆ ಗಳನ್ನು ನಾವು ಬಳಸಿದರೆ ಅವುಗಳೆಲ್ಲ ಅರ್ಥಹೀನ. ಆತ್ಮನು ಎಂದೂ ಹುಟ್ಟಿಯೂ ಇರಲಿಲ್ಲ, ಸಾಯುವುದೂ ಇಲ್ಲ. ನಾವು ಇನ್ನೇನು ಸಾಯುವುದರಲ್ಲಿರುವೆವು, ಸಾಯು ವುದಕ್ಕೆ ಅಂಜುವೆವು, ಎಂಬ ಭಾವನೆಗಳೆಲ್ಲ ಮೂಢನಂಬಿಕೆಗಳು. ಇವುಗಳನ್ನು ಮಾಡಬಲ್ಲೆವು, ಅಥವಾ ಮಾಡಲಾರೆವು ಎಂಬ ಭಾವನೆಗಳೆಲ್ಲ ಮೂಢನಂಬಿಕೆಗಳು. ನಾವು ಯಾವುದನ್ನು ಬೇಕಾದರೂ ಮಾಡಬಲ್ಲೆವು. ಮೊದಲು ನಿಮ್ಮಲ್ಲಿ ಶ್ರದ್ಧೆಯನ್ನು ಇಡಿ ಎಂದು ವೇದಾಂತವು ಬೋಧಿಸುತ್ತದೆ. ಪ್ರಪಂಚದ ಕೆಲವು ಧರ್ಮಗಳು, ಯಾರಿಗೆ ತಮಗಿಂತ ಬೇರೆಯಾದ ದೇವರಲ್ಲಿ ನಂಬಿಕೆ ಇಲ್ಲವೋ ಅವರನ್ನು ನಾಸ್ತಿಕರೆಂದು ಸಾರುತ್ತವೆ. ಆತ್ಮದ ವೈಭವದಲ್ಲಿ ಯಾರಿಗೆ ನಂಬಿಕೆಯಿಲ್ಲವೋ ಅವರನ್ನು ವೇದಾಂತವು ನಾಸ್ತಿಕರೆಂದು ಕರೆಯುತ್ತದೆ. ಅನೇಕರಿಗೆ ಇದು ನಿಸ್ಸಂದೇಹವಾಗಿಯೂ ಭಯಂಕರ ವಾದ ಭಾವನೆ. ನಮ್ಮಲ್ಲಿ ಅನೇಕರು, ಈ ಆದರ್ಶವನ್ನು ಎಂದಿಗೂ ಸಾಧಿಸ ಲಾರೆವು ಎಂದು ತಿಳಿಯುವರು. ಆದರೆ ಪ್ರತಿಯೊಬ್ಬರೂ ಇದನ್ನು ಸಾಧಿಸುವುದಕ್ಕೆ ಸಾಧ್ಯ ಎಂಬುದಾಗಿ ವೇದಾಂತವು ಸಾರುತ್ತದೆ. ಗಂಡಸಾಗಲೀ, ಹೆಂಗಸಾಗಲೀ, ಮಕ್ಕಳಾಗಲೀ, ಜಾತಿಭೇದವಾಗಲೀ, ಅಥವಾ ಲಿಂಗಭೇದವಾಗಲೀ ಯಾವುದೂ, ಈ ಆದರ್ಶದ ಸಾಕ್ಷಾತ್ಕಾರಕ್ಕೆ ಅಡಚಣೆ ತರಲಾರದು. ಏಕೆಂದರೆ ಇದನ್ನು ಆಗಲೇ ಸಾಧಿಸಿ ಆಗಿದೆ. ಈ ಆದರ್ಶವು ಆಗಲೇ ಅಲ್ಲಿ ಇದೆ ಎಂಬುದನ್ನು ವೇದಾಂತವು ತೋರುತ್ತದೆ.

ವಿಶ್ವದಲ್ಲಿರುವ ಎಲ್ಲ ಶಕ್ತಿಗಳೂ ಆಗಲೇ ನಮ್ಮಲ್ಲಿರುವುವು. ನಮ್ಮ ಕಣ್ಣು ಗಳನ್ನು ಕೈಗಳಿಂದ ಮುಚ್ಚಿಕೊಂಡು ಕತ್ತಲೆ ಎಂದು ಅಳುವವರು ನಾವು. ನಿಮ್ಮ ಸುತ್ತಲೂ ಯಾವ ಕತ್ತಲೆಯೂ ಇಲ್ಲವೆಂಬುದನ್ನು ತಿಳಿಯಿರು. ಕೈಗಳನ್ನು ತೆಗೆಯಿರಿ. ಮೊದಲಿನಿಂದಲೇ ಇದ್ದ ಬೆಳಕು ಅಲ್ಲೇ ಇರುವುದು. ಕತ್ತಲೆ ಎಂದಿಗೂ ಇರಲಿಲ್ಲ. ನಿಶ್ಯಕ್ತಿ ಎಂದಿಗೂ ಇರಲಿಲ್ಲ. ಮೂಢರಾದ ನಾವು ಬಲಹೀನರೆಂದು ಅಳುವೆವು. ಮೂಢರಾದ ನಾವು ಅಶುದ್ಧರೆಂದು ಅಳುವೆವು. ಆದಕಾರಣ ವೇದಾಂತವು ಈ ಆದರ್ಶವು ಅನುಷ್ಠಾನ ಸಾಧ್ಯವೆಂದು ಮಾತ್ರ ಹೇಳುವುದಿಲ್ಲ; ಇದು ಯಾವಾಗಲೂ ಇದ್ದುದೇ ಹೀಗೆ ಎನ್ನುತ್ತದೆ. ಈ ಸತ್ಯವೇ, ಈ ಆದರ್ಶವೇ, ನಿಮ್ಮ ನಿಜವಾದ ಸ್ವಭಾವ. ನೀವು ನೋಡುವ ಉಳಿದವುಗಳೆಲ್ಲ ಅಸತ್ಯ, ಕಪಟ. ಎಂದು ನೀವು, “ನಾನೊಬ್ಬ ಹುಲು ಮನುಜ” ಎಂದು ಹೇಳುತ್ತೀರೋ, ಆಗ ನೀವು ಹೇಳುತ್ತಿರುವುದು ಸತ್ಯವಲ್ಲ. ನಿಮ್ಮನ್ನು ನೀವೇ ಕೆಲಸಕ್ಕೆ ಬಾರದ ದುರ್ಬಲರನ್ನಾಗಿ ಮಾಡಿಕೊಳ್ಳುತ್ತೀರಿ, ನೀಚರನ್ನಾಗಿ ಮಾಡಿಕೊಳ್ಳುತ್ತೀರಿ.

ವೇದಾಂತವು ಯಾವ ಪಾಪವನ್ನೂ ಒಪ್ಪಿಕೊಳ್ಳುವುದಿಲ್ಲ, ತಪ್ಪನ್ನು ಮಾತ್ರ ಒಪ್ಪಿಕೊಳ್ಳುವುದು. ನಾನು ನಿರ್ಬಲನೆಂದೂ, ಪಾಪಿಯೆಂದೂ, ಕೆಲಸಕ್ಕೆ ಬಾರದ ವಸ್ತುವೆಂದೂ, ನನಗೆ ಯಾವ ಶಕ್ತಿಯೂ ಇಲ್ಲವೆಂದೂ, ಅದನ್ನು ಮತ್ತು ಇದನ್ನು ಮಾಡುವುದಕ್ಕೆ ಆಗುವುದಿಲ್ಲವೆಂದೂ, ತಿಳಿದುಕೊಳ್ಳುವುದು ಅತ್ಯಂತ ಘೋರತಮವಾದ ತಪ್ಪೆಂದು ವೇದಾಂತವು ಸಾರುವುದು. ಈ ರೀತಿಯಲ್ಲಿ ನೀವು ಆಲೋಚನೆ ಮಾಡಿ ದಾಗಲೆಲ್ಲ, ನಿಮ್ಮನ್ನು ಆಗಲೇ ಬಂಧಿಸಿರುವ ಸರಪಣಿಗೆ ಮತ್ತೊಂದು ಕೊಂಡಿಯನ್ನು ಸೇರಿಸಿದಂತೆ ಆಗುವುದು; ನಮ್ಮ ಆತ್ಮವನ್ನು ಆವರಿಸಿಕೊಂಡಿರುವ ಮೋಹದ ತೆರೆಗೆ ಮತ್ತೊಂದನ್ನು ಸೇರಿಸಿದಂತೆ ಆಗುವುದು. ಆದಕಾರಣ ಯಾರು ತಾನು ನಿರ್ಬಲನೆಂದು ಆಲೋಚಿಸುತ್ತಾನೊ, ಯಾರು ತಾನು ಪಾಪಿ ಎಂದು ತಿಳಿದುಕೊಳ್ಳುತ್ತಾನೆಯೋ, ಅವನದು ತಪ್ಪು. ಅವನು ಒಂದು ಹೀನ ಆಲೋಚನೆಯನ್ನು ಜಗತ್ತಿಗೆ ಕೊಡುತ್ತಿರು ವನು. ಪ್ರಕೃತ ಜೀವನವನ್ನೂ, ಮೋಹಕ್ಕೆ ಒಳಗಾದ ಜೀವನವನ್ನೂ, ನಾವು ಬಾಳುತ್ತಿರುವ ಕಪಟಬದುಕನ್ನೂ ಆದರ್ಶದೊಂದಿಗೆ ರಾಜಿ ಮಾಡಿಕೊಳ್ಳುವ ಪ್ರಯತ್ನವನ್ನು ವೇದಾಂತವು ಒಪ್ಪುವುದಿಲ್ಲ ಎನ್ನುವುದನ್ನು ನಾವು ಮನಸ್ಸಿನಲ್ಲಿಟ್ಟು ಕೊಳ್ಳಬೇಕು. ಕಪಟ ಜೀವನವು ಮಾಯವಾಗಬೇಕು. ಅನಾದಿಯಿಂದಲೂ ಇರುವ ಸತ್ಯಜೀವನವು ಹೊರಗೆ ಬರಬೇಕು, ಕಂಗೊಳಿಸಬೇಕು. ಯಾವ ಮಾನವನೂ ಹೆಚ್ಚು ಹೆಚ್ಚು ಪವಿತ್ರನಾಗುವುದಿಲ್ಲ. ಹೆಚ್ಚು ಹೆಚ್ಚು ಪಾವಿತ್ರ್ಯವನ್ನು ವ್ಯಕ್ತಗೊಳಿಸಬೇಕು. ಮೋಹದ ತೆರೆಯು ಜಾರುತ್ತದೆ, ಆತ್ಮದ ಆಜನ್ಮ ಪಾವಿತ್ರ್ಯವು ಪ್ರಕಾಶಕ್ಕೆ ಬರುತ್ತದೆ. ಅನಂತ ಪಾವಿತ್ರ್ಯ, ಸ್ವಾತಂತ್ರ್ಯ, ಪ್ರೇಮ ಮತ್ತು ಶಕ್ತಿಗಳೆಲ್ಲ ಆಗಲೇ ನಮ್ಮಲ್ಲಿರುವುವು.

ಕಾನನಾಂತರಗಳಲ್ಲಿ, ಗಿರಿಗುಹೆಗಳಲ್ಲಿ ಇದನ್ನು ಸಾಕ್ಷಾತ್ಕಾರ ಮಾಡಿಕೊಳ್ಳುವುದು ಮಾತ್ರವಲ್ಲ; ಜೀವನದಲ್ಲಿ ಎಲ್ಲಾ ಕಾರ್ಯಕ್ಷೇತ್ರಗಳಲ್ಲಿರುವ ಜನರಿಂದಲೂ ಇದು ಸಾಧ್ಯವೆಂದು ವೇದಾಂತವು ಬೋಧಿಸುವುದು. ಈ ತತ್ತ್ವಗಳನ್ನು ಕಂಡುಹಿಡಿದವರು ಗುಹೆ ಕಾನನಗಳ ವಾಸಿಗಳೂ ಅಲ್ಲ; ಅಥವಾ ಸಾಧಾರಣ ಜನರಂತೆ ಜೀವನ ನಡೆಸುತ್ತಿದ್ದವರೂ ಅಲ್ಲವೆಂಬುದನ್ನು ನಾವು ನೋಡಿರುವೆವು. ಅವರು ಅತ್ಯುತ್ತಮ ವಾದ ಕರ್ಮಜೀವನವನ್ನು ನಡೆಸಿದವರೆಂದು ನಾವು ಪ್ರಮಾಣಸಹಿತ ಒಪ್ಪಲೇ ಬೇಕಾಗುತ್ತದೆ. ಅವರು ಸೇನೆಗೆ ಆಜ್ಞೆ ಮಾಡಬೇಕಾಗಿತ್ತು; ಸಿಂಹಾಸನದ ಮೇಲೆ ಕುಳಿತು ರಾಜ್ಯಭಾರ ಮಾಡಬೇಕಾಗಿತ್ತು; ಲಕ್ಷಾಂತರ ಜನರ ಹಿತರಕ್ಷಣೆಯನ್ನು ನೋಡಿಕೊಳ್ಳಬೇಕಾಗಿತ್ತು. ಇವೆಲ್ಲ ನಿರಂಕುಶ ಪ್ರಭುತ್ವದ ಕಾಲದಲ್ಲಿ, ರಾಜನು ಬಹುಪಾಲು ಕೇವಲ ಅಲಂಕಾರಪ್ರಾಯನಾಗಿರುವ ಇಂದಿನಂತಹ ಕಾಲದಲ್ಲಿ ಅಲ್ಲ. ಆದರೂ ಈ ಮಹದಾಲೋಚನೆಗಳನ್ನು ಚಿಂತಿಸುವುದಕ್ಕೆ, ಅವನ್ನು ಸಾಕ್ಷಾತ್ಕಾರ ಮಾಡಿ ಕೊಳ್ಳುವುದಕ್ಕೆ ಮತ್ತು ಮಾನವತೆಗೆ ಉಪದೇಶ ಮಾಡುವುದಕ್ಕೆ ಅವರಿಗೆ ಸಮಯ ಸಿಕ್ಕುತ್ತಿತ್ತು. ನಮ್ಮ ಜೀವನವು, ಅವರ ಜೀವನದೊಂದಿಗೆ ಹೋಲಿಸಿ ನೋಡಿದರೆ, ವಿರಾಮದಂತಿದೆ. ಹೀಗಿರುವಾಗ ನಮಗೆ ಅವರಿಗಿಂತ ಎಷ್ಟು ಹೆಚ್ಚು ಅಭ್ಯಾಸಕ್ಕೆ ಕಾಲವಿದೆ! ಅವರೊಂದಿಗೆ ಹೋಲಿಸಿ ನೋಡಿದರೆ ಸದಾಕಾಲವೂ ವಿರಾಮದಂತಿದೆ. ಮಾಡುವುದಕ್ಕೆ ಇರುವುದು ಬಹಳ ಸ್ವಲ್ಪ ಕೆಲಸ. ಅಂತಹ ಆದರ್ಶಗಳನ್ನು ಅಭ್ಯಾಸ ಮಾಡುವುದಕ್ಕೆ ಆಗುವುದಿಲ್ಲವೆಂದರೆ ನಮಗೆ ನಾಚಿಕೆಗೇಡು. ಪೂರ್ವಕಾಲದ ನಿರಂಕುಶ ಸಾರ್ವಭೌಮರೊಂದಿಗೆ ಹೋಲಿಸಿ ನೋಡಿದರೆ ನಮಗೆ ಇರುವ ಆವಶ್ಯಕತೆಗಳು ಬಹಳ ಸ್ವಲ್ಪ. ಕುರುಕ್ಷೇತ್ರದ ಸಮರಾಂಗಣದಲ್ಲಿ ಮಹಾ ಸೈನ್ಯದ ಮೇಲ್ವಿಚಾರಣೆ ನೋಡಿಕೊಳ್ಳುತ್ತಿದ್ದ ಅರ್ಜುನನೊಂದಿಗೆ ಹೋಲಿಸಿ ನೋಡಿದರೆ ನನ್ನ ಬಯಕೆಗಳು ಅತಿ ಅಲ್ಪ. ಆದರೂ ಆ ಸಮರದ ಗಡಿಬಿಡಿಯ ಮಧ್ಯದಲ್ಲಿ ಅತಿ ಗಹನ ತತ್ತ್ವವನ್ನು ಕುರಿತು ಮಾತನಾಡುವುದಕ್ಕೆ, ಮತ್ತು ಅದನ್ನು ಅನುಷ್ಠಾನಕ್ಕೆ ತರುವುದಕ್ಕೆ ಅವನಿಗೆ ಕಾಲ ಸಿಕ್ಕಿತು. ಹಿಂದಿನವರಿಗಿಂತ ಸ್ವತಂತ್ರರಾಗಿ, ಸುಲಭವಾದ ಜೀವನ ನಡೆಸುತ್ತಾ ಸೌಖ್ಯದಲ್ಲಿರುವ ನಮಗೆ ಅವರು ಮಾಡಿದಷ್ಟನ್ನಾದರೂ ಮಾಡುವುದಕ್ಕೆ ಈ ಜೀವನದಲ್ಲಿ ಸಾಧ್ಯವಾಗಬೇಕು. ನಮಗೆ ನಿಜವಾಗಿಯೂ ಕಾಲವನ್ನು ಒಳ್ಳೆಯದಕ್ಕೆ ಉಪಯೋಗಿಸಬೇಕೆಂಬ ಆಸೆ ಇದ್ದರೆ, ನಮ್ಮಲ್ಲಿ ಅನೇಕರಿಗೆ ನಮಗೆ ತಿಳಿದಿರು ವುದಕ್ಕಿಂತ ಹೆಚ್ಚು ಸಮಯವಿದೆ. ನಾವು ಪ್ರಯತ್ನಪಟ್ಟರೆ ನಮಗೆ ಇರುವ ಸ್ವಾತಂತ್ರ್ಯದಿಂದ ಇನ್ನೂರು ಆದರ್ಶಗಳ ಸಿದ್ಧಿಯನ್ನು ಪಡೆಯಬಹುದು. ಆದರೆ ನಾವು ಆದರ್ಶವನ್ನು ಈಗ ನಾವಿರುವ ಸ್ಥಿತಿಗೆ ಇಳಿಸಕೂಡದು. ನಾವು ಮಾಡಿರುವ ತಪ್ಪುಗಳನ್ನು ಒಪ್ಪಿಕೊಳ್ಳದೆ ಅವಕ್ಕೆ ನಾವು ಹೇಗೆ ನೆವಗಳನ್ನು ಕಲ್ಪಿಸಬೇಕೆಂಬುದನ್ನು ಕೆಲವರು ನಮಗೆ ಕಲಿಸುವರು. ಅವರ ಆದರ್ಶವೇ ನಮಗೆ ಬೇಕಾಗಿರುವುದೆಂದು ನಾವು ತಿಳಿಯುವೆವು. ಆದರೆ ಅದು ಹಾಗಲ್ಲ. ವೇದಾಂತವು ಇಂತಹದಾವುದನ್ನೂ ಬೋಧಿಸುವುದಿಲ್ಲ. ಇಂದು ನಾವಿರುವ ಸ್ಥಿತಿ ಆದರ್ಶದ ಸ್ಥಿತಿಗೇರಬೇಕು. ಇಂದಿನ ನಮ್ಮ ಜೀವನವನ್ನು ಅನಂತರ ಬಾಳುವೆಯಲ್ಲಿ ಬೆರೆಯುವಂತೆ ಮಾಡಬೇಕು.

ವೇದಾಂತದ ಏಕಮಾತ್ರ ಕೇಂದ್ರ ಆದರ್ಶವೇ ಈ ಏಕತ್ವ ಎಂಬುದನ್ನು ಯಾವಾಗಲೂ ಜ್ಞಾಪಕದಲ್ಲಿಟ್ಟುಕೊಳ್ಳಬೇಕು. ಯಾವುದರಲ್ಲಿಯೂ ಎರಡಿಲ್ಲ. ಎರಡು ಜೀವನಗಳಿಲ್ಲ ಅಥವಾ ಎರಡು ಬೇರೆ ಬೇರೆ ಪ್ರಪಂಚಗಳಿಗೆ ಬೇರೆ ಬೇರೆ ಜೀವನಗಳೂ ಇಲ್ಲ. ಸ್ವರ್ಗ ಮುಂತಾದುವುಗಳನ್ನು ಕುರಿತು ಹೇಳುವುದನ್ನು ನೀವು ನೋಡುತ್ತೀರಿ. ಆದರೆ ವೇದಋಷಿಗಳು ಅನಂತರ ತಮ್ಮ ತತ್ತ್ವದ ಪರಮ ಆದರ್ಶವನ್ನು ತಲುಪಿದಾಗ, ಅವುಗಳೆಲ್ಲವನ್ನು ಆಚೆಗೆಸೆಯುವರು. ಇರುವುದು ಒಂದೇ ಜೀವನ, ಒಂದೇ ಜಗತ್ತು ಒಂದೇ ಅಸ್ತಿತ್ವ. ಎಲ್ಲವೂ ಅದೊಂದೆ, ವ್ಯತ್ಯಾಸವಿರುವುದು ಪ್ರಮಾಣದಲ್ಲಲ್ಲದೆ ಸ್ವಭಾವದಲ್ಲಲ್ಲ. ನಮ್ಮ ಜೀವನದ ವ್ಯತ್ಯಾಸವು ವರ್ಗ ಭೇದದಲ್ಲಿ ಇಲ್ಲ. ಪ್ರಾಣಿಗಳು ಮಾನವನಿಂದ ಪ್ರತ್ಯೇಕವಾದುವು, ನಮ್ಮ ಆಹಾರಕ್ಕೋಸ್ಕರವಾಗಿಯೇ ದೇವರು ಅವನ್ನು ಸೃಷ್ಟಿಸಿರುವನು ಎಂಬ ಅಭಿಪ್ರಾಯಗಳನ್ನು ವೇದಾಂತವು ಸಂಪೂರ್ಣವಾಗಿ ವಿರೋಧಿಸುವುದು.

ಉದಾಹರಣೆಗಾಗಿ, ಪ್ರಾಣಿಗಳನ್ನು ಪರೀಕ್ಷಾರ್ಥವಾಗಿ ಕೊಲ್ಲುವುದನ್ನು ವಿರೋಧಿಸುವ ಒಂದು ಸಂಘವನ್ನು ಕೆಲವರು ಸ್ಥಾಪಿಸಿರುವರು. ಅವರ ಒಬ್ಬ ಸದಸ್ಯ ನನ್ನು ನಾನು ಕೇಳಿದೆ: “ಸಹೋದರನೆ, ಪ್ರಾಣಿಗಳನ್ನು ಆಹಾರಕ್ಕಾಗಿ ಕೊಲ್ಲುವುದು ನ್ಯಾಯ; ವೈಜ್ಞಾನಿಕ ಪರೀಕ್ಷಾರ್ಥವಾಗಿ ಒಂದೆರಡು ಪ್ರಾಣಿಗಳನ್ನು ಕೊಲ್ಲುವುದು ಅನ್ಯಾಯ ಎಂದು ಏತಕ್ಕೆ ತಿಳಿಯುತ್ತೀಯೆ?” ಅದಕ್ಕೆ ಆತನು, “ಪರೀಕ್ಷಾರ್ಥವಾಗಿ ಕೊಲ್ಲುವುದು ಮಹಾಪಾತಕ.ಆದರೆ ನಮ್ಮ ಆಹಾರಕ್ಕೋಸ್ಕರವಾಗಿಯೆ ದೇವರು ಪ್ರಾಣಿಗಳನ್ನು ಕೊಟ್ಟಿರುವುದು” ಎಂದನು. ಜೀವನದ ಏಕತ್ವದಲ್ಲಿ ಪ್ರಾಣಿಗಳೂ ಕೂಡ ಸೇರಿವೆ. ಮಾನವನ ಜೀವವು ಅಮರವಾದುದಾದರೆ ಪ್ರಾಣಿ ಜೀವವೂ ಅಮರವಾದುದು. ವ್ಯತ್ಯಾಸವಿರುವುದು ತರತಮದಲ್ಲಿ ಅಲ್ಲದೆ ವಸ್ತುವಿನಲ್ಲಿ ಅಲ್ಲ. ಜೀವಾಣು ಮತ್ತು ನಾನೂ ಒಂದೆ, ವ್ಯತ್ಯಾಸವಿರುವುದು ತರತಮದಲ್ಲಿ ಮಾತ್ರ. ಅತ್ಯುನ್ನತ ಜೀವನದ ದೃಷ್ಟಿಯಿಂದ ಈ ವ್ಯತ್ಯಾಸಗಳೆಲ್ಲವೂ ಮಾಯವಾಗುವುವು. ಮನುಷ್ಯನು ಸಣ್ಣ ಹುಲ್ಲಿಗೂ ಮರಕ್ಕೂ ಬೇಕಾದಷ್ಟೂ ವ್ಯತ್ಯಾಸವನ್ನು ಕಾಣಬಹುದು. ಆದರೆ ಬಹಳ ಮೇಲಕ್ಕೆ ಏರಿ ಹೋದರೆ ಮರ ಮತ್ತು ಹುಲ್ಲು ಹೆಚ್ಚು ಕಡಿಮೆ ಒಂದೇ ಸಮನಾಗಿ ತೋರುವುವು. ಅದರಂತೆಯೇ ಜೀವನದ ಅತ್ಯುತ್ತಮ ಆದರ್ಶದ ದೃಷ್ಟಿಯಿಂದ ಬಹಳ ತುಚ್ಛ ಪ್ರಾಣಿಯೂ ಒಂದೇ ಅತಿಶ್ರೇಷ್ಠ ಮಾನವನೂ ಒಂದೇ. ದೇವರಿರುವನೆಂದು ನೀವು ನಂಬಿದರೆ, ಅತಿ ಕ್ಷುದ್ರಪ್ರಾಣಿಗಳೂ ಮತ್ತು ಅತ್ಯುತ್ತಮ ಜೀವಿಯೂ ಒಂದೇ ಆಗಿರಬೇಕು. ಮನುಷ್ಯನೆಂಬ ತನ್ನ ಮಕ್ಕಳಿಗೆ ಪಕ್ಷಪಾತಿಯಾಗಿ, ಪ್ರಾಣಿಗಳೆಂಬ ತನ್ನ ಮಕ್ಕಳಿಗೆ ಕ್ರೂರನು ಆದರೆ ಅಂತಹ ದೇವರು ರಾಕ್ಷಸರಿ ಗಿಂತಲೂ ಕೀಳು.ಅಂತಹ ದೇವರನ್ನು ಪೂಜಿಸುವ ಬದಲು ನಾನು ನೂರು ಸಲವಾದರೂ ಪ್ರಾಣ ಬಿಡುವೆನು. ನನ್ನ ಇಡೀ ಜೀವನವೇ ಅಂತಹ ದೇವರೊಂದಿಗೆ ಒಂದು ಹೋರಾಟವಾಗುವುದು. ಆದರೆ ವ್ಯತ್ಯಾಸವಿಲ್ಲ. ಯಾರು ವ್ಯತ್ಯಾಸವಿದೆ ಎಂದು ಹೇಳುವರೋ, ಅವರು ಜವಾಬ್ದಾರಿ ಇಲ್ಲದ, ಏನೂ ತಿಳಿಯದ ಹೃದಯ ಹೀನರು. ಅನುಷ್ಠಾನವೆಂಬ ಪದವನ್ನು ಅಪಾರ್ಥದಲ್ಲಿ ಉಪಯೋಗಿಸಿರುವುದಕ್ಕೆ ಇದೊಂದು ಉದಾಹರಣೆ. ನಾನೇ ಶುದ್ಧ ಸಸ್ಯಾಹಾರಿಯಾಗದೇ ಇರಬಹುದು. ಆದರೆ ನನಗೆ ಅದರ ಆದರ್ಶ ಗೊತ್ತಿದೆ. ನಾನು ಮಾಂಸವನ್ನು ತಿನ್ನುವಾಗ ಅದು ತಪ್ಪು ಎಂಬುದು ನನಗೆ ಗೊತ್ತಿದೆ. ಕೆಲವು ಸಮಯಗಳಲ್ಲಿ ನಾನು ಅದನ್ನು ತಿನ್ನಲೇ ಬೇಕಾಗಿ ಬಂದರೂ ಅದು ಕ್ರೌರ್ಯ ಎಂಬುದು ನನಗೆ ಗೊತ್ತಿದೆ. ಆದರ್ಶವನ್ನು ನಾವು ಇಂದಿರುವದುಃಸ್ಥಿತಿಗೆ ಎಳೆದು ಈ ರೀತಿಯಲ್ಲಿ ನಮ್ಮ ದೌರ್ಬಲ್ಯವನ್ನು ಸಮರ್ಥಿಸಿ ಕೊಳ್ಳಕೂಡದು. ಆದರ್ಶವೆಂದರೆ ಮಾಂಸವನ್ನು ತಿನ್ನದೇ ಇರುವುದು, ಯಾವ ಪ್ರಾಣಿಗೂ ಹಿಂಸೆಯನ್ನು ಮಾಡದೇ ಇರುವುದು. ಏಕೆಂದರೆ ಎಲ್ಲಾ ಪ್ರಾಣಿ ಗಳೂ ನಮ್ಮ ಸಹೋದರರು. ನೀವು ಅವುಗಳನ್ನು ನಿಮ್ಮ ಸಹೋದರರೆಂಬ ಭಾವ ದಿಂದ ನೋಡುವುದಾದರೆ ಎಲ್ಲಾ ಜೀವಿಗಳ ಸಹೋದರತ್ವದ ಆದರ್ಶದೆಡೆಗೆ ನೀವು ಮುಂದುವರಿದಂತೆ ಆಗುತ್ತದೆ. ಇನ್ನು ಮಾನವ ಸಹೋದರತ್ವದ ವಿಚಾರವಾಗಿ ಹೇಳಬೇಕಾಗಿಯೇ ಇಲ್ಲ. ಇದು ಕೇವಲ ಮಕ್ಕಳಾಟ. ಈ ಆದರ್ಶವು ಅನೇಕರಿಗೆ ಸರಿಹೋಗುವುದಿಲ್ಲ ಎಂದು ನಿಮಗೆ ಸಾಧಾರಣವಾಗಿ ತಿಳಿಯುತ್ತದೆ. ಏಕೆಂದರೆ ಇದು ಇಂದು ನಾವಿರುವ ಸ್ಥಿತಿಯನ್ನು ತೊರೆದು ಇನ್ನು ಮೇಲಕ್ಕೆ, ಆದರ್ಶದೆಡೆಗೆ ಹೋಗಬೇಕೆಂದು ಹೇಳುತ್ತದೆ. ಜನರ ಈಗಿನ ನಡವಳಿಕೆಗೆ ಸರಿಹೋಗುವಂತಹ ಯಾವುದಾದರೂ ಆದರ್ಶವನ್ನು ತಂದರೆ ಅದನ್ನು ಅವರು ಸಂಪೂರ್ಣ ಅನುಷ್ಠಾನ ಯೋಗ್ಯವೆಂದು ತಿಳಿಯುತ್ತಾರೆ.

ಮಾನವ ಸ್ವಭಾವದಲ್ಲಿ ಬದಲಾವಣೆ ಬೇಡ, ಮುಂದಕ್ಕೆ ನಮಗೆ ಒಂದು ಹೆಜ್ಜೆಯನ್ನೂ ಇಡುವುದಕ್ಕೆ ಇಚ್ಛೆಯಿಲ್ಲ – ಎಂಬ ಅಭಿಪ್ರಾಯವು ಬಹಳ ಬಲವಾಗಿ ಬೇರೂರಿದೆ. ಹಿಮದಲ್ಲಿ ಘನೀಭೂತರಾದ ಜನರ ಬಗ್ಗೆ ನಾನು ಓದಿದ್ದೆ. ಮಾನವ ಸ್ವಭಾವವೂ ಅಂತಹುದೇ. ಹಿಮದಲ್ಲಿ ಸಿಕ್ಕಿರುವವರು ತಾವು ಇನ್ನೂ ನಿದ್ದೆಮಾಡ ಬೇಕೆಂದು ಹೇಳುವರು. ಅವರನ್ನು ನೀವು ಬಲಾತ್ಕಾರದಿಂದ ಎಳೆಯಲು ಪ್ರಯತ್ನ ಪಟ್ಟರೆ, “ನಿದ್ದೆಮಾಡೋಣ, ಹಿಮದಲ್ಲಿ ಮಲಗುವುದು ಎಷ್ಟೋ ಸಂತೋಷ” ಎನ್ನುವರು. ಆ ನಿದ್ದೆಯಲ್ಲೇ ಅವರು ಸಾಯುವರು. ಅದರಂತೆಯೇ ನಮ್ಮ ಸ್ವಭಾ ವವೂ ಕೂಡ. ಅದನ್ನು ನಮ್ಮ ಜೀವನದಲ್ಲೆಲ್ಲ ಮಾಡುತ್ತಿರುವೆವು. ಕಾಲಿನಿಂದ ಮೇಲಕ್ಕೆ ಶೀತವೇರುತ್ತ ಬರುತ್ತಿದ್ದರೂ, ನಾವು ನಿದ್ದೆಮಾಡಬೇಕೆಂದು ಬಯಸುವೆವು. ಆದಕಾರಣ ನೀವು ಆದರ್ಶದ ಸಿದ್ಧಿಗಾಗಿ ಹೋರಾಡಬೇಕು. ಯಾರಾದರೂ ಆ ಉನ್ನತ ಸ್ಥಿತಿಯಲ್ಲಿರುವ ಆದರ್ಶವನ್ನು, ನೀವಿರುವ ಸ್ಥಿತಿಗೆ ಎಳೆಯಲು ಯತ್ನಿಸಿದರೆ, ಆ ಉನ್ನತ ಧ್ಯೇಯವಿಲ್ಲದ ಧರ್ಮವನ್ನು ಬೋಧಿಸಿದರೆ, ಅವರ ಮಾತನ್ನು ಕೇಳಬೇಡಿ. ನನ್ನ ಭಾಗಕ್ಕೆ ಅದು ಕಾರ್ಯತಃ ಸಾಧ್ಯವಾಗುವ ಧರ್ಮವಲ್ಲ. ಆದರೆ ಯಾರಾದರೂ ಅತ್ಯಂತ ಉತ್ತಮ ಆದರ್ಶಗಳನ್ನೊಳಗೊಂಡ ಧರ್ಮವನ್ನು ಬೋಧಿಸಿದರೆ, ನಾನು ಅದಕ್ಕೆ ಸಿದ್ಧನಾಗಿರುವೆನು. ವಿಷಯಪ್ರಲೋಭನಗಳಿಗೆ ಮತ್ತು ಇಂದ್ರಿಯ ದೌರ್ಬಲ್ಯ ಗಳಿಗೆ ಸಮರ್ಥನೆಯನ್ನು ಯಾರಾದರೂ ಕೊಡುವಾಗ ಜೋಪಾನವಾಗಿರಿ. ಪಾಪ! ಇಂದ್ರಿಯ ಪ್ರಲೋಭನಗಳಿಗೆ ತುತ್ತಾದ, ವಿಚಾರ ಬುದ್ಧಿಯನ್ನು ಕಳೆದುಕೊಂಡ ನಮಗೆ, ಈ ಮಾರ್ಗವನ್ನು ಯಾರಾದರೂ ಬೋಧಿಸಿದರೆ ಇದನ್ನು ಅನುಸರಿಸಿ ನಾವೆಂದಿಗೂ ಮುಂದುವರಿಯಲಾರೆವು. ನಾನು ಇಂತಹ ಸ್ಥಿತಿಗಳನ್ನು ಎಷ್ಟೋ ನೋಡಿರುವೆನು. ನನಗೆ ಪ್ರಪಂಚದ ಸ್ವಲ್ಪ ಅನುಭವವಿದೆ. ನಾಯಿ ಬಣಬೆಗಳಂತೆ ಧರ್ಮ, ಪಂಥಗಳು ಬೆಳೆಯುವ ದೇಶ ಮತ್ತು ತಾಯ್ನಾಡು. ಪ್ರತಿವರುಷವೂ ಹೊಸ ಪಂಥಗಳು ಹುಟ್ಟುತ್ತಿರುವುವು. ಆದರೆ ನಾನು ಇದರಲ್ಲಿ ಒಂದನ್ನು ಗಮನಿಸಿರುವೆನು. ಯಾರು ಸತ್ಯಸಾಧಕರಿಗೂ ಇಂದ್ರಿಯ ಸುಖಾಭಿಲಾಷೆಗಳಿಗೆ ತುತ್ತಾದ ಮಾನವನಿಗೂ ರಾಜಿಮಾಡಿಸಲು ಒಪ್ಪುವುದಿಲ್ಲವೋ, ಅವರು ಮಾತ್ರ ಮುಂದುವರಿಯುವರು. ಎಲ್ಲಿ ಜೀವನದ ಅತ್ಯುಚ್ಚ ಧ್ಯೇಯಗಳೊಂದಿಗೆ ಇಂದ್ರಿಯ ಸುಖಗಳನ್ನು ಒಂದುಗೂಡಿಸುವ ತಪ್ಪು ಅಭಿಪ್ರಾಯವಿದೆಯೋ, ಎಲ್ಲಿ ದೇವನನ್ನು ಮಾನವನ ಅಧೋಗತಿಗೆ ಎಳೆಯಲು ಪ್ರಯತ್ನ ನಡೆದಿದೆಯೊ, ಅಲ್ಲೆಲ್ಲ ಕ್ಷೀಣದೆಸೆ ಪ್ರಾಪ್ತವಾಗುವುದು. ಮಾನವನನ್ನು ಇಂದ್ರಿಯ ಸುಖದ ಅಡಿಯಾಳನ್ನಾಗಿ ಮಾಡಕೂಡದು. ಅವನನ್ನು ದೇವತ್ವದ ಪೀಠಕ್ಕೆ ಏರಿಸಬೇಕು.

ಅದೇ ಕಾಲದಲ್ಲಿ ಪ್ರಶ್ನೆಯ ಇನ್ನೊಂದು ಭಾಗವಿದೆ. ಉಳಿದವರನ್ನು ನಾವು ನಿಕೃಷ್ಟದೃಷ್ಟಿಯಿಂದ ನೋಡಕೂಡದು. ನಾವೆಲ್ಲರೂ ಒಂದೇ ಗುರಿಯ ಕಡೆಗೆ ಪ್ರಯಾಣ ಮಾಡುತ್ತಿರುವೆವು. ಶಕ್ತಿಗೂ ದುರ್ಬಲತೆಗೂ ಇರುವ ಭಿನ್ನತೆ ತರತಮದಲ್ಲಿ. ಪಾಪಕ್ಕೂ ಪುಣ್ಯಕ್ಕೂ ಇರುವ ಭಿನ್ನತೆ ತರತಮದಲ್ಲಿ. ಸ್ವರ್ಗಕ್ಕೂ ನರಕಕ್ಕೂ ಇರುವ ಭಿನ್ನತೆ ತರತಮದಲ್ಲಿ. ಜೀವನಕ್ಕೂ ಮರಣಕ್ಕೂ ಇರುವ ಭಿನ್ನತೆ ತರತಮದಲ್ಲಿ. ಈ ಪ್ರಪಂಚದಲ್ಲಿರುವ ವ್ಯತ್ಯಾಸವೆಲ್ಲ ವಸ್ತುಗಳ ತರತಮದಲ್ಲಿದೆ, ವಸ್ತುಭೇದ ದಲ್ಲಿಲ್ಲ; ಏಕೆಂದರೆ ಏಕತ್ವವೇ ಪ್ರಪಂಚದಲ್ಲಿರುವ ಸಕಲ ವಸ್ತುಗಳ ಗೂಢ ರಹಸ್ಯ. ಆಲೋಚನೆಯಂತೆ ಕಾಣಲಿ, ಪ್ರಾಣದಂತೆ ಕಾಣಲಿ, ಆತ್ಮನಂತೆ ಕಾಣಲಿ ಅಥವಾ ದೇಹದಂತೆ ತೋರಲಿ, ಎಲ್ಲವೂ ಒಂದೆ, ವ್ಯತ್ಯಾಸವಿರುವುದು ಹೆಚ್ಚು ಕಡಿಮೆಯಲ್ಲಿ. ಆದಕಾರಣ ನಮ್ಮಷ್ಟು ಯಾರು ಅಭಿವೃದ್ಧಿಯಾಗಿಲ್ಲವೋ, ಅವರನ್ನು ನಿಕೃಷ್ಟದೃಷ್ಟಿ ಯಿಂದ ನೋಡಲು ನಮಗೆ ಯಾವ ಅಧಿಕಾರವೂ ಇಲ್ಲ. ಯಾರನ್ನೂ ಧಿಕ್ಕರಿಸಬೇಡಿ, ನಿಮಗೆ ಸಹಾಯ ಮಾಡಲು ಸಾಧ್ಯವಾದರೆ ಕೈ ಎತ್ತಿ. ಇಲ್ಲದೇ ಇದ್ದರೆ ಕೈಗಳನ್ನು ಜೋಡಿಸಿ, ನಿಮ್ಮ ಸಹೋದರನನ್ನು ಹರಸಿ. ಅವರ ದಾರಿಯನ್ನು ಅವರು ಹಿಡಿದು ನಡೆಯಲಿ. ಅವರನ್ನು ಕೆಳೆಗೆಳೆದು ಅಲ್ಲಗಳೆಯುವುದು ಕೆಲಸ ಮಾಡುವ ರೀತಿಯಲ್ಲ. ಈ ರೀತಿಯಲ್ಲಿ ಕೆಲಸವನ್ನು ಎಂದಿಗೂ ಸಾಧಿಸಲಾಗುವುದಿಲ್ಲ. ಇನ್ನೊಬ್ಬರನ್ನು ದೂರುವುದರಲ್ಲಿ ನಮ್ಮ ಶಕ್ತಿಯನ್ನೆಲ್ಲಾ ಕಳೆಯುವೆವು. ಇನ್ನೊಬ್ಬರಲ್ಲಿ ತಪ್ಪು ಕಂಡು ಹಿಡಿಯುವುದು, ಇನ್ನೊಬ್ಬರನ್ನು ಅಲ್ಲಗೆಳೆಯುವುದು, ಇವುಗಳೆಲ್ಲ ನಿರರ್ಥಕ. ಎಲ್ಲರೂ ನೋಡುತ್ತಿರುವುದು ಒಂದನ್ನೆ, ಹೆಚ್ಚು ಕಡಿಮೆ ಅನುಸರಿಸುತ್ತಿರುವುದು ಒಂದೇ ಆದರ್ಶವನ್ನು, ನಮ್ಮ ಭಿನ್ನತೆಗೆ ಇರುವ ಮುಕ್ಕಾಲುಪಾಲು ಕಾರಣವೆಲ್ಲ ಅಭಿವ್ಯಕ್ತಿಯ ವ್ಯತ್ಯಾಸದಲ್ಲಿ ಮಾತ್ರ ಎಂಬುದು ನಮಗೆ ಗೊತ್ತಾಗುತ್ತದೆ.

ಈಗ ಪಾಪವೆಂಬ ಭಾವನೆಯನ್ನು ತೆಗೆದುಕೊಳ್ಳೋಣ. ಈಗ ತಾನೆ ನಾನು ವೇದಾಂತದ ಭಾವನೆ ಏನು ಎಂಬುದನ್ನು ಹೇಳುತ್ತಿದ್ದೆ. ಅದಕ್ಕೆ ವಿರೋಧವಾದ ಭಾವನೆಯೆ ಮಾನವನನ್ನು ಪಾಪಿ ಎಂದು ಕರೆಯುವುದು. ಎರಡೂ ವಸ್ತುತಃ ಒಂದೆ. ಒಂದು ಅಸ್ತಿ ಮಾರ್ಗವನ್ನು ಅನುಸರಿಸುವುದು, ಮತ್ತೊಂದು ನಾಸ್ತಿಮಾರ್ಗವನ್ನು ಅನುಸರಿಸುವುದು. ಒಂದು ಮಾನವನಿಗೆ ಆತನ ಶಕ್ತಿಯನ್ನು ತೋರುವುದು, ಮತ್ತೊಂದು ಅವನ ದುರ್ಬಲತೆಯನ್ನು ತೋರುವುದು. ದುರ್ಬಲತೆ ಇರಬಹುದು, ಆದರೆ ಅದನ್ನು ಲೆಕ್ಕಿಸಬೇಡಿ, ನಾವು ಬೆಳೆಯಬೇಕು–ಎನ್ನುತ್ತದೆ ವೇದಾಂತ. ಮನುಷ್ಯನು ಹುಟ್ಟಿದ ಒಡನೆಯೇ ಖಾಯಿಲೆಯೂ ಹುಟ್ಟಿತು. ನಮ್ಮ ರೋಗ ವೇನೆಂಬುದನ್ನು ಹೇಳುವುದಕ್ಕೆ ಯಾರೂ ಬೇಕಾಗಿಲ್ಲ. ಸದಾಕಾಲದಲ್ಲಿಯೂ, ನಾವು ರೋಗದಲ್ಲಿ ನರಳುತ್ತಿರುವೆವು ಎಂದು ಆಲೋಚಿಸಿದರೆ ಅದರಿಂದ ಗುಣವಾಗುವು ದಿಲ್ಲ, ಔಷಧಿ ಅಗತ್ಯ. ನಾವು ಹೇಗೆ ಬೇಕಾದರೂ ಹೊರಗೆ ಮೆರೆಯಬಹುದು, ಬಾಹ್ಯ ಪ್ರಪಂಚದ ಮಟ್ಟಿಗೆ ನಾವು ಕಪಟಿಗಳಾಗಬಹುದು. ಆದರೆ ನಮ್ಮ ದೌರ್ಬಲ್ಯಗಳು ಎಷ್ಟು ಎಂಬುದು ನಮಗೆ ತಿಳಿದೇ ಇದೆ. ದೌರ್ಬಲ್ಯಗಳನ್ನು ಜ್ಞಾಪಿಸಿ ಕೊಳ್ಳುವುದರಿಂದ ಅಷ್ಟೇನೂ ಪ್ರಯೋಜನವಿಲ್ಲ. ಶಕ್ತಿಯನ್ನು ನೀಡಿ, ಸರ್ವದಾ ನಿರ್ಬಲತೆಯನ್ನು ಮೆಲಕು ಹಾಕುವುದರಿಂದ ಶಕ್ತಿ ಬರುವುದಿಲ್ಲ ಎನ್ನುತ್ತದೆ ವೇದಾಂತ. ದೌರ್ಬಲ್ಯಕ್ಕೆ ಪರಿಹಾರ ದೌರ್ಬಲ್ಯವನ್ನು ಕುರಿತು ಆಲೋಚಿಸುವುದಲ್ಲ, ಆದರೆ ಶಕ್ತಿಯನ್ನು ಕುರಿತು ವಿಚಾರಮಾಡುವುದು. ಜನರಿಗೆ ಆಗಲೇ ಅವರಲ್ಲಿರುವ ಶಕ್ತಿಯ ವಿಚಾರವಾಗಿ ಬೋಧನೆ ಮಾಡಿ. ವೇದಾಂತವು ಅವರನ್ನು ಪಾಪಿಗಳೆಂದು ಕರೆಯದೆ, ಇದಕ್ಕೆ ವಿರೋಧವಾಗಿ, “ನೀವು ಶುದ್ಧರು, ಪೂರ್ಣರು, ಯಾವುದನ್ನು ಪಾಪವೆನ್ನುವಿರೋ ಅದು ನಿಮಗೆ ಸೇರಿದ್ದಲ್ಲ” ಎನ್ನುವುದು. ಪಾಪವೆಂಬುದು ಆತ್ಮದ ಅತ್ಯಂದ ನೀಚ ಅಭಿವ್ಯಕ್ತಿ. ನಿಮ್ಮ ಆತ್ಮವನ್ನು ಉತ್ತಮರೀತಿಯಲ್ಲಿ ವ್ಯಕ್ತಪಡಿಸಿ. ಇದೊಂದು ನಾವು ಮನಸ್ಸಿನಲ್ಲಿಡಬೇಕಾದ ವಿಷಯ, ನಮ್ಮೆಲ್ಲರಿಗೂ ಇದು ಸಾಧ್ಯ. “ಇಲ್ಲ” “ಸಾಧ್ಯವಿಲ್ಲ” ಎಂದು ಎಂದೂ ಹೇಳಬೇಡಿ. ಏಕೆಂದರೆ ನಾವು ಅನಾದಿ, ಅನಂತರು. ನಮ್ಮ ಸಹಜಸ್ವಭಾವದೊಂದಿಗೆ ಹೋಲಿಸಿದರೆ ಕಾಲದೇಶಗಳೂ ಗಣನೆಗೆ ಬರುವುದಿಲ್ಲ. ಏನನ್ನಾದರೂ, ಯಾವುದನ್ನಾದರೂ ನೀವು ಮಾಡಬಲ್ಲಿರಿ. ನೀವು ಸರ್ವಶಕ್ತರು.

ಇವು ನೀತಿಯ ತತ್ತ್ವಗಳು. ಆದರೆ ಈಗ ನಾವು ಕೆಳಗೆ ಇಳಿದು ಅದನ್ನು ವಿಶದ ಪಡಿಸೋಣ. ಈ ವೇದಾಂತದ ತತ್ತ್ವಗಳನ್ನು ನಮ್ಮ ಅನುದಿನದ ಕಾರ್ಯರಂಗದಲ್ಲಿ, ನಗರ ಜೀವನದಲ್ಲಿ, ಹಳ್ಳಿಯ ಜೀವನದಲ್ಲಿ, ದೇಶದ ಜೀವನದಲ್ಲಿ, ಜನಾಂಗದ ಜೀವನದಲ್ಲಿ ಮತ್ತು ಪ್ರತಿಯೊಂದು ಜನಾಂಗದ ಜೀವನಾಂತರಾಳದಲ್ಲಿ, ಹೇಗೆ ಆಚರಣೆಗೆ ತರಬಹುದೆಂಬುದನ್ನು ನೋಡೋಣ. ಮಾನವನು ಯಾವ ದೇಶದಲ್ಲಾ ದರೂ ಇರಲಿ ಯಾವ ಕಾರ್ಯಕ್ಷೇತ್ರದಲ್ಲಾದರೂ ಇರಲಿ ಅವನಿಗೆ ಧರ್ಮವು ಸಹಾಯ ಮಾಡದೆ ಇದ್ದರೆ, ಅದರಿಂದ ಅಷ್ಟೇನೂ ಪ್ರಯೋಜನವಿಲ್ಲ. ಕೆಲವರಿಗೆ ಅದು ಒಂದು ಸಿದ್ಧಾಂತವಾಗಿ ಮಾತ್ರ ಉಳಿಯುವುದು. ಧರ್ಮವು ಮಾನವ ಜನಾಂಗಕ್ಕೆ ಸಹಕಾರಿಯಾಗಬೇಕಾದರೆ, ಮಾನವನು ಯಾವ ಸ್ಥಿತಿಯಲ್ಲಿದ್ದರೂ ಇರಲಿ, ಬಂಧನ ದಲ್ಲಿರಲಿ ಅಥವಾ ಸ್ವತಂತ್ರನಾಗಿರಲಿ, ಪಾಪದ ಆಳದಲ್ಲಿರಲಿ ಅಥವಾ ಪುಣ್ಯದ ಶಿಖರದಲ್ಲಿರಲಿ, ಎಲ್ಲಾ ಕಡೆಯಲ್ಲಿಯೂ ಒಂದೇ ಸಮನಾಗಿ ಅವನ ಸಹಾಯಕ್ಕೆ ಬರಲು ಸಿದ್ಧವಾಗಿರಬೇಕು. ಈ ವೇದಾಂತದ ತತ್ತ್ವಗಳು, ಈ ಧರ್ಮದ ಆದರ್ಶಗಳು, ಅಥವಾ ಇದನ್ನು ನೀವು ಮತ್ತಾವ ಹೆಸರಿನಿಂದಲಾದರೂ ಕರೆಯಿರಿ, ಇವು, ಈ ಮಹಾಕಾರ್ಯವನ್ನು ಸಾಧಿಸುವುದರಿಂದ ಮಾತ್ರ ಕೃತಕೃತ್ಯವಾಗುವುವು.

ಆತ್ಮಶ್ರದ್ಧೆಯ ಆದರ್ಶ ನಮಗೆ ಬಹಳ ಸಹಕಾರಿ. ನಮ್ಮಲ್ಲಿ ನಮಗೆ ನಂಬಿಕೆ ಯನ್ನು ಹೆಚ್ಚು ಬೋಧಿಸಿದ್ದರೆ, ಅದನ್ನು ಅನುಷ್ಠಾನಕ್ಕೆ ತಂದಿದ್ದರೆ, ನಮ್ಮಲ್ಲಿರುವ ಅನೇಕ ದೋಷಗಳು ಮತ್ತು ದುಃಖದ ಬಹುಭಾಗ ಮಾಯವಾಗುತ್ತಿದ್ದುವು ಎಂಬುದ ರಲ್ಲಿ ಸಂದೇಹವಿಲ್ಲ. ಮಾನವ ಇತಿಹಾಸದಲ್ಲೆಲ್ಲಾ ಪ್ರಖ್ಯಾತರಾದ ಸ್ತ್ರೀಪುರುಷರ ಜೀವನದಲ್ಲಿ, ಎಲ್ಲಾ ಕ್ರಿಯೋತ್ತೇಜನ ಶಕ್ತಿಗಿಂತಲೂ ಯಾವುದಾದರೊಂದು ಅತಿ ಪ್ರಾಮುಖ್ಯವಾಗಿದ್ದರೆ, ಅದು ಅವರ ಶಕ್ತಿಯಲ್ಲಿ ಅವರಿಗೆ ಇರುವ ಶ್ರದ್ಧೆ. ತಾವು ಪ್ರಖ್ಯಾತರಾಗಬೇಕೆಂಬ ಬಯಕೆಯೊಂದಿಗೆ ಜನ್ಮ ತಾಳಿ ಅವರು ಪ್ರಖ್ಯಾತ ರಾದರು. ಜನರು ಎಷ್ಟು ಕೆಳಗೆ ಸಾಧ್ಯವೋ ಅಷ್ಟು ಕೆಳಗೆ ಬೇಕಾದರೂ ಹೋಗಲಿ, ಅವರಿಗೆ ಜೀವನದಲ್ಲಿ ಒಂದು ಕಾಲ ಬಂದೇ ಬರುವುದು. ಆಗ ನಿರ್ವಾಹವಿಲ್ಲದೆ ನಿರಾಸೆಯಿಂದ ಪ್ರೇರೇಪಿತರಾಗಿಯೂ ಅವರು ಮೇಲೇಳುವ ಮಾರ್ಗವನ್ನು ಅನು ಸರಿಸುವರು. ಆಗ ತಮ್ಮಲ್ಲಿ ಶ್ರದ್ಧೆ ಇಡುವುದನ್ನು ಕಲಿತುಕೊಳ್ಳುವರು. ಅದನ್ನು ನಾವು ಮೊದಲಿನಿಂದಲೇ ತಿಳಿದಿರುವುದು ಹಿತಕಾರಿ. ನಮ್ಮಲ್ಲಿ ವಿಶ್ವಾಸವನ್ನು ಪಡೆಯು ವುದಕ್ಕೆ ಈ ಕಷ್ಟವನ್ನೆಲ್ಲಾ ಏತಕ್ಕೆ ಅನುಭವಿಸಬೇಕು? ಒಬ್ಬ ಮನುಷ್ಯನಿಗೂ ಮತ್ತೊಬ್ಬ ಮನುಷ್ಯನಿಗೂ ಇರುವ ವ್ಯತ್ಯಾಸವೆಲ್ಲ ತನ್ನಲ್ಲಿ ತನಗೆ ವಿಶ್ವಾಸವಿದೆಯೆ ಇಲ್ಲವೆ ಎಂಬುದರ ಮೇಲಿದೆ ಎಂದು ತೋರುವುದು. ಆತ್ಮ ವಿಶ್ವಾಸ ಎಲ್ಲವನ್ನೂ ಸಾಧಿಸಬಹುದು. ನನ್ನ ಜೀವನದಲ್ಲಿಯೇ ನಾನು ಅದನ್ನು ಅನುಭವಿಸುತ್ತಿರುವೆನು. ನಾನು ಯಾವಾಗಲೂ ಅದನ್ನು ಅನುಭವಿಸುತ್ತಿರುವೆನು. ನನಗೆ ವಯಸ್ಸು ಆದಂತೆಲ್ಲ ಆ ವಿಶ್ವಾಸ ಇನ್ನೂ ದೃಢವಾಗುತ್ತ ಬರುತ್ತಿದೆ. ಯಾರಿಗೆ ಆತ್ಮವಿಶ್ವಾಸ ಇಲ್ಲವೋ ಅವನು ನಾಸ್ತಿಕ. ಹಳೆಯ ಧರ್ಮಗಳು ಯಾರಿಗೆ ದೇವರಲ್ಲಿ ಭಕ್ತಿ ಇಲ್ಲವೋ ಅವರ ನನ್ನು ನಾಸ್ತಿಕನೆನ್ನುತ್ತಿದ್ದುವು. ನವೀನ ಧರ್ಮವು ಯಾರು ಆತ್ಮವಿಶ್ವಾಸಹೀನರೋ, ಅವರನ್ನು ನಾಸ್ತಿಕರೆಂದು ಸಾರುತ್ತದೆ. ಆದರೆ ಇದು ಸ್ವಾರ್ಥಪರತೆಯ ಶ್ರದ್ಧೆಯಲ್ಲ. ಏಕೆಂದರೆ ವೇದಾಂತವೆ ಏಕತ್ವದ ಸಿದ್ಧಾಂತ. ಅಂದರೆ ಎಲ್ಲರಲ್ಲೂ ವಿಶ್ವಾಸವಿಡ ಬೇಕೆಂದು ಆಗುವುದು. ಏಕೆಂದರೆ ನೀನೇ ಸರ್ವವೂ ಆಗಿರುವೆ. ಆತ್ಮಪ್ರೀತಿ ಎಂದರೆ ಎಲ್ಲರ ಮೇಲೂ ಪ್ರೀತಿ, ಪ್ರಾಣಿಗಳ ಮೇಲೂ ಪ್ರೀತಿ, ಎಲ್ಲದರ ವಿಷಯದಲ್ಲೂ ಪ್ರೀತಿ. ಏಕೆಂದರೆ ನಾವುಗಳೆಲ್ಲ ಒಂದು. ಈ ಮಹಾ ವಿಶ್ವಾಸವೇ ಜಗತ್ತನ್ನು ಮೇಲಕ್ಕೆ ತರುವುದು. ಇದರಲ್ಲಿ ನನಗೆ ಲವಲೇಶವೂ ಸಂದೇಹವಿಲ್ಲ. ಸತ್ಯವಾಗಿ ಯಾರು, “ನನ್ನ ವಿಚಾರವಾಗಿ ಎಲ್ಲವೂ ನನಗೆ ಗೊತ್ತಿದೆ” ಎಂದು ಹೇಳುತ್ತಾರೆಯೋ, ಅವರೇ ಶ್ರೇಷ್ಠ ಮಾನವರು. ನಮ್ಮ ದೇಹದ ಹಿಂದೆ ಎಷ್ಟೊಂದು ಶಕ್ತಿ ಅನೇಕ ರೂಪಗಳಲ್ಲಿ ಅವಿತುಕೊಂಡಿದೆ ಎಂಬುದು ನಮಗೆ ಗೊತ್ತೆ? ಮನುಷ್ಯನಲ್ಲಿರುವುದೆಲ್ಲವನ್ನು ಯಾವ ವಿಜ್ಞಾನಿ ಸಂಪೂರ್ಣವಾಗಿ ತಿಳಿದಿರುವನು? ಮನುಷ್ಯನು ಜಗತ್ತಿಗೆ ಕಾಲಿಟ್ಟು ಅನೇಕ ಲಕ್ಷಾಂತರ ವರುಷಗಳಾದುವು. ಆದರೂ ಅವನ ಅನಂತರ ಶಕ್ತಿಯಲ್ಲಿ ಒಂದು ಅಣು ಮಾತ್ರ ಪ್ರಕಾಶಕ್ಕೆ ಬಂದಿರುವುದು. ಆದಕಾರಣ ನಾವು ದುರ್ಬಲರೆಂದು ಹೇಳಿಕೊಳ್ಳ ಬಾರದು. ಮೇಲೆ ಕಾಣುವ ಹೀನ ಸ್ಥಿತಿಯ ಹಿಂದೆ ಯಾವ ಮಹಾ ಕಾರ್ಯವನ್ನು ಮಾಡುವ ಸಾಧ್ಯತೆಗಳು ಅಡಗಿರುವುವೋ ನಮಗೆ ಹೇಗೆ ಗೊತ್ತು? ನಿಮ್ಮಲ್ಲಿರು ವುದರಲ್ಲಿ ನಿಮಗೆ ಎಲ್ಲೋ ಸ್ವಲ್ಪ ಗೊತ್ತಿದೆ. ಏಕೆಂದರೆ ನಿಮ್ಮ ಹಿಂದೆ ಅನಂತ ಶಕ್ತಿಯ ಆ ಸಚ್ಚಿದಾನಂದದ ಸಾಗರವೇ ಇದೆ.

“ಮೊದಲು ಈ ಆತ್ಮದ ವಿಚಾರವನ್ನು ಕೇಳಬೇಕು.” ನೀವು ಆತ್ಮ ಎಂಬುದನ್ನು ಹಗಲು ರಾತ್ರಿ ಕೇಳಬೇಕು. ಹಗಲು ರಾತ್ರಿ ನಿಮ್ಮ ನಾಡಿಗಳೊಳಗೆ ಸೇರುವ ಪರಿಯಂತರವೂ ಪ್ರತಿಯೊಂದು ರಕ್ತದ ಬಿಂದುವಿನಲ್ಲಿ ಅನುರಣಿತವಾಗುವವರೆವಿಗೂ, ನಿಮ್ಮ ಮಾಂಸದಲ್ಲಿ ಮಾಂಸವಾಗಿ, ಮೂಳೆಯಲ್ಲಿ ಮೂಳೆಯಾಗುವ ಪರಿಯಂತರವೂ ಅದನ್ನು ಮನನ ಮಾಡಬೇಕು. “ನಾನು ಜನನ ಮರಣಾತೀತನು. ಆನಂದಮಯನು, ಸರ್ವಜ್ಞನು, ಸರ್ವಶಕ್ತನು, ಪರಮಪಾವನಾತ್ಮನು” ಎಂಬ ಆದರ್ಶದಿಂದ ಈ ದೇಹವೆಲ್ಲವೂ ತುಂಬಿ ತುಳುಕಾಡಲಿ. ಹಗಲಿರುಳು ಅದನ್ನೇ ವಿಚಾರಮಾಡಿ, ಅದು ನಮ್ಮ ದೇಹದಲ್ಲಿ ಐಕ್ಯವಾಗುವವರೆಗೆ ಅದನ್ನೇ ವಿಚಾರಮಾಡಿ. ಅದರ ಮೇಲೆ ಧ್ಯಾನಮಾಡಿ. ಇದರ ಪರಿಣಾಮವಾಗಿ ಧರ್ಮವು ಪ್ರಕಾಶಿಸುವುದು. “ಹೃದಯ ಪೂರ್ಣವಾದಾಗ ವದನ ತೆರೆಯುವುದು”. ಹೃದಯ ಪೂರ್ಣವಾಗಿ ಕೈಗಳು ಕೆಲಸವನ್ನು ಮಾಡುವುವು. ಆ ಆದರ್ಶವನ್ನು ನಿಮ್ಮ ಜೀವನದಲ್ಲಿ ನೆಲೆಗೊಳಿಸಿ. ನೀವು ಏನು ಕೆಲಸ ಮಾಡಿದರೂ ಅದನ್ನು ಕುರಿತು ಚೆನ್ನಾಗಿ ಆಲೋಚಿಸಿ. ಆಲೋಚನೆಯ ಶಕ್ತಿ ಮಾತ್ರದಿಂದ ನಿಮ್ಮ ಕಾರ್ಯವೆಲ್ಲಾ ಬೃಹದಾಕಾರ ತಾಳುವುದು, ರೂಪ ಬದಲಾಯಿಸುವುದು, ಎಲ್ಲಾ ಪುಣ್ಯ ಕಾರ್ಯವಾಗುವುದು. ಜಡಕ್ಕೆ ಸ್ವಲ್ಪ ಶಕ್ತಿ ಇದ್ದರೆ ಆಲೋಚನೆ ಸರ್ವಶಕ್ತಿಯುಳ್ಳದ್ದು. ಈ ಭಾವನೆಗಳು ನಮ್ಮ ಜೀವನದ ಮೇಲೆ ಪರಿಣಾಮವನ್ನುಂಟುಮಾಡುವಂತೆ ಮಾಡಿ. ನಿಮ್ಮ ಪ್ರಭುತ್ವ, ಸಾರ್ವಭೌಮತ್ವ, ಮತ್ತು ಕೀರ್ತಿಗಳಿಂದ ನಿಮ್ಮ ಜೀವನವನ್ನು ತುಂಬಿ. ದೇವರ ದಯೆಯಿಂದ ಯಾವ ಮೂಢನಂಬಿಕೆಗಳೂ ನಮ್ಮ ತಲೆಯನ್ನು ತುಂಬದೆ ಇರಲಿ! ದೇವರ ದಯೆಯಿಂದ ನಮ್ಮ ಜನ್ಮಾರಭ್ಯ ಈ ಮೂಢನಂಬಿಕೆಯ ಮತ್ತು ನಮ್ಮನ್ನು ಕುಗ್ಗಿಸುವ ದೌರ್ಬಲ್ಯ ಮತ್ತು ನೀಚತ್ವ–ಇಂತಹ ಭಾವಗಳ ವಾತಾವರಣದಲ್ಲಿ ನಾವು ಬೆಳೆಯದೇ ಇರಲಿ! ದೇವರ ದಯೆಯಿಂದ ಮಾನವ ವರ್ಗಕ್ಕೆ ಜೀವನದ ಪರಮೋತ್ಕೃಷ್ಟ ಆದರ್ಶಗಳನ್ನು ಪಡೆಯಲು ಸುಲಭವಾದ ಮಾರ್ಗವಿರಲಿ. ಆದರೆ ಮನುಷ್ಯನು ಇವುಗಳ ಮೂಲಕವಾಗಿಯೇ ಪ್ರಯಾಣಮಾಡಬೇಕಾಯಿತು. ನಿಮ್ಮ ಮುಂದಿನ ಪೀಳಿಗೆಯವರಿಗೆ ಮಾರ್ಗವನ್ನು ಅತಿ ಕಠಿಣ ಮಾಡಬೇಡಿ.

ಇವು, ಕೆಲವು ವೇಳೆ ಬೋಧಿಸುವುದಕ್ಕೆ ಭಯಂಕರವಾದ ಸಿದ್ಧಾಂತಗಳು. ಇವುಗಳ ವಿಚಾರವಾಗಿ ಅಂಜುವವರನ್ನೂ ನಾನು ಬಲ್ಲೆ. ಆದರೆ ಯಾರು ಇನ್ನು ಅನುಷ್ಠಾನಕ್ಕೆ ತರಬೇಕೆಂದು ಆಶಿಸುವರೋ, ಅವರು ಮೊದಲು ಕಲಿಯಬೇಕಾದುದು ಇದು. ನಿಮಗೆ ಅಥವಾ ಇನ್ನು ಯಾರಿಗಾದರೂ ಎಂದಿಗೂ ನೀವು ದುರ್ಬಲರೆಂದು ಹೇಳಬೇಡಿ. ಸಾಧ್ಯವಾದರೆ ಒಳ್ಳೆಯದನ್ನು ಮಾಡಿ. ಆದರೆ ಜಗತ್ತಿಗೆ ತೊಂದರೆಯನ್ನು ಕೊಡಬೇಡಿ. ಆತ್ಮನಿಂದೆ ಮಾಡಿಕೊಳ್ಳುವುದು, ಕಾಲ್ಪನಿಕ ದೇವರೆದುರಿಗೆ ಅಳುವುದು, ಬೇಡುವುದು, ಇಂತಹ ಅನೇಕ ಸಂಕುಚಿತ ಭಾವಗಳು ಮೌಢ್ಯ ಎಂಬುದು ನಿಮ್ಮ ಮನಸ್ಸಿನ ಅಂತರಾಳದಲ್ಲಿ ನಿಮಗೆ ಗೊತ್ತಿದೆ. ಈ ಪ್ರಾರ್ಥನೆ ಸಫಲವಾದ ಒಂದು ದೃಷ್ಟಾಂತ ವನ್ನಾದರೂ ನನಗೆ ಕೊಡಿ. ದೊರೆತ ಎಲ್ಲಾ ಉತ್ತರಗಳೂ ನಮ್ಮ ಅಂತರಾಳದಿಂದ ಹೊರಬಂದವು. ದೆವ್ವಗಳಿಲ್ಲವೆಂದು ನಮಗೆ ಗೊತ್ತಿದೆ. ಆದರೆ ಕತ್ತಲೆಯಲ್ಲಿ ನೀವು ಇದ್ದ ತಕ್ಷಣವೇ ಮನಸ್ಸು ಸ್ವಲ್ಪ ತಲ್ಲಣಿಸುವುದು. ಇದಕ್ಕೇನು ಕಾರಣವೆಂದರೆ, ನಮ್ಮ ಬಾಲ್ಯದಲ್ಲಿ ಇಂತಹ ಭಯಾನಕ ಭಾವಗಳನ್ನೆಲ್ಲಾ ತಲೆಗೆ ತುಂಬಿದ್ದುದು. ಆದರೆ ಇಂತಹ ಭಾವಗಳನ್ನು, ಸಮಾಜದ ಅಂಜಿಕೆಯಿಂದಲಾದರೂ ಆಗಲಿ, ಸಾರ್ವ ಜನಿಕರ ಅಭಿಪ್ರಾಯದ ಅಂಜಿಕೆಯಿಂದಾಗಲಿ, ಸ್ನೇಹಿತರ ದ್ವೇಷವನ್ನು ಗಳಿಸ ಬೇಕಾಗುವುದು ಎಂಬ ಅಂಜಿಕೆಯಿಂದಲಾದರೂ ಆಗಲಿ ಅಥವಾ ನಾವು ಬಹಳ ವಿಶ್ವಾಸದಿಂದ ಕಾಯ್ದುಕೊಂಡಿದ್ದ ಮೂಢನಂಬಿಕೆಗಳು ನಾಶವಾಗುತ್ತವೆಯೆಂದಾ ಗಲಿ, ಪರರಿಗೆ ಬೋಧಿಸಬೇಡಿ. ನೀವು ಇವೆಲ್ಲವನ್ನೂ ನಿಮ್ಮ ಅಂಕೆಯಲ್ಲಿ ಇಟ್ಟುಕೊಂಡಿರಿ. ವಿಶ್ವದ ಏಕತ್ವ ಮತ್ತು ಆತ್ಮಶ್ರದ್ಧೆ ಇವುಗಳಿಗಿಂತ ಹೆಚ್ಚಾಗಿ ಬೋಧಿಸುವುದು ಏನಿದೆ ಧರ್ಮದಲ್ಲಿ? ಕಳೆದ ಸಾವಿರಾರು ವರುಷಗಳಿಂದ ಮಾನವನು ಮಾಡಿದ ಎಲ್ಲಾ ಕ್ರಿಯೆಗಳೂ ಈ ಒಂದು ಗುರಿಯೆಡೆಗಾಗಿ ಸಾಗುವುದಕ್ಕಾಗಿ ಮಾಡಿದುವುಗಳಾಗಿವೆ. ಮಾನವ ಜನಾಂಗವು ಇಂದು ಕೂಡ ಅದಕ್ಕಾಗಿ ಪ್ರಯತ್ನಿ ಸುತ್ತಿರುವುದು. ಈಗ ನಿಮ್ಮ ಸರದಿ. ನಿಮಗೆ ಸತ್ಯ ಗೊತ್ತಿದೆ ಏಕೆಂದರೆ ಹಲವು ಕಡೆಗಳಲ್ಲಿ ಇದನ್ನು ಬೋಧಿಸಲಾಗಿದೆ. ತತ್ತ್ವಶಾಸ್ತ್ರ, ಮನಶ್ಯಾಸ್ತ್ರ ಮಾತ್ರವಲ್ಲ, ಭೌತವಿಜ್ಞಾನಗಳೂ ಕೂಡ ಇದನ್ನೇ ಸಾರಿವೆ. ಈ ವಿಶ್ವದ ಏಕತ್ವವನ್ನು ಒಪ್ಪಿಕೊಳ್ಳಲು ಅಂಜುವ ವಿಜ್ಞಾನಿ ಎಲ್ಲಿ ಇರುವನು? ಅನೇಕ ಜಗತ್ತುಗಳಿವೆ ಎಂದು ಹೇಳುವ ಧೈರ್ಯ ಯಾರಿಗಿದೆ? ಇವುಗಳೆಲ್ಲ ಮೂಢನಂಬಿಕೆ. ಇರುವ ಜೀವನ ಒಂದು. ಇರುವ ಜಗತ್ತು ಒಂದು. ಈ ಒಂದು ಜೀವನವೆ ಒಂದು ಜಗತ್ತೆ ನಮಗೆ ಅನೇಕವಾಗಿ ಕಾಣುತ್ತಿದೆ. ಈ ಅನೇಕ ಒಂದು ಸ್ವಪ್ನದಂತೆ. ನೀವು ಸ್ವಪ್ನದಲ್ಲಿರುವಾಗ ಒಂದು ಸ್ವಪ್ನ ಜಾರುವುದು. ಅನಂತರ ಮತ್ತೊಂದು ಬರುವುದು. ನೀವು ನಿಮ್ಮ ಕನಸಿನಲ್ಲಿ ಜೀವಿಸುವುದಿಲ್ಲ. ಕನಸುಗಳು ಒಂದಾದ ಮೇಲೊಂದು ಬರುವುದು. ಒಂದು ದೃಶ್ಯದ ಮೇಲೆ ಮತ್ತೊಂದು ದೃಶ್ಯವು ನಿಮ್ಮ ಮುಂದೆ ಬರುವುದು. ತೊಂಬತ್ತರಷ್ಟು ದುಃಖ, ಹತ್ತರಷ್ಟು ಸುಖವಿರುವ ಈ ಜಗತ್ತಿನಲ್ಲಿ ಬಹುಶಃ ಕೆಲವು ಕಾಲದ ಮೇಲೆ ಶೇಕಡ ತೊಂಬತ್ತರಷ್ಟು ಸುಖವಾಗಿ ಕಾಣಬಹುದು. ಅದನ್ನು ಆಗ ಸ್ವರ್ಗವೆನ್ನುತ್ತೇವೆ. ಇವುಗಳೆಲ್ಲ ಮಾಯವಾಗುವ ಒಂದು ಸಮಯವು ಸಂತನಿಗೆ ಬರುತ್ತದೆ. ಆಗ ಈ ಪ್ರಪಂಚವೇ ದೇವರಂತೆ ಕಾಣುವುದು. ತನ್ನಾತ್ಮವೇ ದೇವರಾಗುವುದು. ಆದಕಾರಣ ಅನೇಕ ಪ್ರಪಂಚಗಳೂ ಇಲ್ಲ, ಅನೇಕ ಜೀವನಗಳೂ ಇಲ್ಲ. ಈ ಅನೇಕವೆಲ್ಲವೂ ಆ ಏಕದ ಅಭಿವ್ಯಕ್ತಿಯೇ, ಆ ಏಕವೆ ಅನೇಕದಂತೆ–ವಸ್ತು, ಆತ್ಮ, ಆಲೋಚನೆ, ಮನಸ್ಸಿನಂತೆ ಕಾಣಿಸಿಕೊಳ್ಳುತ್ತದೆ. ಅನೇಕದಂತೆ ತೋರುತ್ತಿರುವುದು ಅದೊಂದೆ. ಆದಕಾರಣ ನಾವು ಮೊದಲು ಮಾಡಬೇಕಾದುದೆ ಆ ಸತ್ಯವನ್ನು ಬೋಧಿಸಿ ಕೊಳ್ಳುವುದು ಮತ್ತು ಇತರರಿಗೆ ಬೋಧಿಸುವುದು.

ಈ ಆದರ್ಶವು ಜಗತ್ತಿನಲ್ಲಿ ಅನುರುಣಿತವಾಗಲಿ, ಮೂಢನಂಬಿಕೆಗಳು ಮಾಯ ವಾಗಲಿ. ಯಾರು ನಿರ್ಬಲರೋ ಅವರಿಗೆ ಹೇಳಿ, ಬಿಡದೆ ಒತ್ತಿ ಒತ್ತಿ ಹೇಳಿ: ಶುದ್ಧಾತ್ಮನು ನೀನು. ಜಾಗ್ರತನಾಗು, ಏಳು. ಹೇ! ಸರ್ವಶಕ್ತನೆ, ನಿದ್ರೆ ನಿನಗೆ ತರವಲ್ಲ. ದುಃಖಿಯೆಂದೂ, ಹೀನನೆಂದೂ ನೀನು ಆಲೋಚಿಸದಿರು, ಸರ್ವಶಕ್ತನೆ, ಏಳು, ಜಾಗ್ರತನಾಗು. ನಿನ್ನ ಸ್ವಭಾವವನ್ನು ಪ್ರಕಟಪಡಿಸು. ನಾನು ಪಾಪಿಯೆಂದು ತಿಳಿಯುವುದು ನಿನಗೆ ತರವಲ್ಲ. ದುರ್ಬಲನು ನೀನೆಂದು ತಿಳಿಯುವುದೇ ನಿನಗೆ ತರವಲ್ಲ. ಇವನ್ನು ನಿಮ್ಮ ಮನಸ್ಸಿಗೆ ಹೇಳಿಕೊಳ್ಳಿ. ಪ್ರಪಂಚಕ್ಕೆ ಸಾರಿ. ಎಂತಹ ಒಂದು ಅನುಷ್ಠಾನ ಸಾಧ್ಯವಾದ ಫಲ ದೊರಕುತ್ತದೆಂಬುದನ್ನು ನೋಡಿ. ವಿದ್ಯು ದ್ವೇಗದಿಂದ ಹೇಗೆ ಎಲ್ಲವೂ ವ್ಯಕ್ತವಾಗುವುದೆಂಬುದನ್ನು, ಹೇಗೆ ಎಲ್ಲವೂ ಬದಲಾ ವಣೆಯಾಗುವುದೆಂಬುದನ್ನು ನೋಡಿ. ಇದನ್ನು ಮಾನವ ಜನಾಂಗಕ್ಕೆ ಸಾರಿ. ಅವರ ಶಕ್ತಿಯನ್ನು ಅವರಿಗೆ ತೋರಿ. ಆಗ ಅದನ್ನು ನಮ್ಮ ನಿತ್ಯಜೀವನದಲ್ಲಿ ಹೇಗೆ ಅನ್ವಯಿಸಿ ಕೊಳ್ಳಬೇಕೆಂಬುದನ್ನು ಕಲಿತುಕೊಳ್ಳುತ್ತೇವೆ.

ನಾವು ವಿವೇಕವೆಂಬುದನ್ನು ಬಳಸಬೇಕಾದರೆ, ನಮ್ಮ ಜೀವನದ ಪ್ರತಿಯೊಂದು ಗಳಿಗೆಯಲ್ಲೂ, ನಮ್ಮ ಪ್ರತಿಯೊಂದು ಕಾರ್ಯದಲ್ಲೂ ಯಾವುದು ಸರಿ, ಯಾವುದು ತಪ್ಪು ಎಂಬುದನ್ನು ವಿಮರ್ಶಿಸಬೇಕಾದರೆ, ಯಾವುದು ಸತ್ಯ ಯಾವುದು ಅಸತ್ಯ ಎಂಬುದನ್ನು ತಿಳಿಯಬೇಕಾದರೆ, ಸತ್ಯದ ಪರೀಕ್ಷೆ ನಮಗೆ ತಿಳಿದಿರಬೇಕು. ಅದೇ ಪರಿಶುದ್ಧತೆ ಮತ್ತು ಏಕತ್ವ. ಐಕ್ಯತೆಗೆ ಯಾವುದು ಸಹಾಯಮಾಡುವುದೊ ಅದೆಲ್ಲವೂ ಸತ್ಯ. ಪ್ರೀತಿಯೆ ಸತ್ಯ. ದ್ವೇಷವೇ ಅಸತ್ಯ, ದ್ವೇಷವೆ ಭಿನ್ನತೆಗೆ ಮೂಲ. ದ್ವೇಷವೆ ಒಬ್ಬನನ್ನು ಮತ್ತೊಬ್ಬನಿಂದ ಪ್ರತ್ಯೇಕಪಡಿಸುವುದು. ಆದಕಾರಣ ಇದು ತಪ್ಪು ಮತ್ತು ಅಸತ್ಯ. ಇದು ಬೇರ್ಪಡಿಸುವ ಶಕ್ತಿ. ಇದು ಪ್ರತ್ಯೇಕಗೊಳಿಸಿ ನಾಶ ಮಾಡುತ್ತದೆ.

ಪ್ರೇಮವು ಒಂದುಗೂಡಿಸುತ್ತದೆ, ಐಕ್ಯತೆಗೆ ಕಾರಣವಾಗುತ್ತದೆ. ತಾಯಿಯು ಮಕ್ಕಳೊಂದಿಗೆ, ಮನೆಗಳು ನಗರದೊಂದಿಗೆ, ಇಡೀ ವಿಶ್ವವು ಪ್ರಾಣಿಗಳೊಂದಿಗೆ ಒಂದಾಗುತ್ತದೆ. ಪ್ರೇಮವೇ ಅಸ್ತಿತ್ವ. ಅದೇ ಪ್ರತ್ಯಕ್ಷ ದೇವರು. ಇವೆಲ್ಲವೂ ಹೆಚ್ಚು ಕಡಿಮೆ ಅಭಿವ್ಯಕ್ತವಾಗುತ್ತಿರುವ ಪ್ರೇಮವೇ ವ್ಯತ್ಯಾಸವಿರುವುದು ತರತಮದಲ್ಲಿ ಮಾತ್ರ. ಎಲ್ಲಾ ಕಡೆಗಳಲ್ಲಿಯೂ ಆ ಪ್ರೇಮವೊಂದೇ ಬೆಳಗುತ್ತಿರುವುದು. ಆದಕಾರಣ ನಾವು ಮಾಡುವ ಪ್ರತಿಯೊಂದು ಕೆಲಸವನ್ನೂ ಅದು ಭಿನ್ನತೆಗೆ ಸಹಾಯ ಮಾಡುತ್ತದೆಯೆ ಅಥವಾ ಐಕ್ಯತೆಗೆ ಸಹಾಯ ಮಾಡುತ್ತದೆಯೇ ಎಂದು ನೋಡ ಬೇಕು. ಭಿನ್ನತೆಗೆ ಮೂಲವಾದರೆ ಅದನ್ನು ತ್ಯಜಿಸಬೇಕು. ಐಕ್ಯತೆಗೆ ಸಹಾಯ ವಾದರೆ ಅದು ಒಳ್ಳೆಯದೆಂಬುದು ನಿಜ. ಅದರಂತೆಯೇ ನಮ್ಮ ಆಲೋಚನೆಗಳೂ ಕೂಡ. ಅದು ಭಿನ್ನಗೊಳಿಸುವುದೋ ಅಥವಾ ಒಂದುಗೂಡಿಸುವಂತೆ ಮಾಡುವುದೊ, ಏಕ ಅಭಿಪ್ರಾಯವು ನೆಲೆಸುವಂತೆ ಮಾಡುವುದೊ, ನೋಡಬೇಕು. ಇದನ್ನು ಅದು ಸಾಧಿಸಿದರೆ ಸ್ವೀಕರಿಸೋಣ, ಇಲ್ಲದೇ ಇದ್ದರೆ ಘೋರಪಾಪದಂತೆ ಅದನ್ನು ಹೊರ ದೂಡೋಣ.

ನೀತಿಯ ಭಾವನೆ ನಮಗೆ ತಿಳಿಯಲಾರದ ಯಾವುದರ ಆಧಾರದ ಮೇಲೂ ನಿಂತಿಲ್ಲ. ನಮಗೆ ತಿಳಿಯಲಾರದ ಯಾವುದನ್ನೂ ನಮಗೆ ಅದು ಬೋಧಿಸುವುದಿಲ್ಲ. ಉಪನಿಷತ್ತಿನ ವಾಣಿಯಲ್ಲಿ, “ಯಾವ ದೇವರನ್ನು ನಾವು ತಿಳಿಯಲಾರದವನೆಂದು ಇಲ್ಲಿ ಪೂಜಿಸುತ್ತೇವೆಯೋ, ಅದೇ ದೇವರನ್ನೇ ನಾನು ನಿಮಗೆ ಬೋಧಿಸುತ್ತೇನೆ.” ನಿಮಗೆ ಏನು ತಿಳಿಯಬೇಕಾದರೂ ಅದು ಆತ್ಮನ ಮೂಲಕ. ನಾನು ಒಂದು ಕುರ್ಚಿಯನ್ನು ನೋಡುತ್ತೇನೆ. ಆದರೆ ನಾನು ಅದನ್ನು ನೋಡಬೇಕಾದರೆ ಮೊದಲು ನನ್ನನ್ನು ನಾನು ತಿಳಿಯಬೇಕು. ಅನಂತರ ಕುರ್ಚಿಯನ್ನು ಗ್ರಹಣ ಮಾಡುತ್ತೇನೆ. ನೀವು ನನಗೆ ಗೊತ್ತಾಗುವುದು, ಪ್ರಪಂಚ ನನಗೆ ತಿಳಿಯುವುದು ಆತ್ಮನಲ್ಲಿ, ಆತ್ಮನ ಮೂಲಕ. ಆದಕಾರಣ ಆತ್ಮನನ್ನು ತಿಳಿಯಲಾರದವನೆಂದು ಹೇಳುವುದರಲ್ಲಿ ಅರ್ಥವಿಲ್ಲ. ಆತ್ಮನನ್ನು ಬೇರೆ ಮಾಡಿದರೆ ಪ್ರಪಂಚವೇ ಮಾಯವಾಗುವುದು.ಎಲ್ಲಾ ಜ್ಞಾನ ಬರುವುದು ಆತ್ಮನಲ್ಲಿ, ಆತ್ಮನ ಮೂಲಕ. ಆದಕಾರಣ ಇದು ಎಲ್ಲಕ್ಕಿಂತ ಹೆಚ್ಚಾಗಿ ತಿಳಿದ ವಸ್ತು. ಯಾವುದನ್ನು “ನಾನು” ಎಂದು ಹೇಳುತ್ತೇನೆಯೋ ಅದೇ “ನೀನು” ನನ್ನ ಅಹಂಕಾರ ನಿಮ್ಮ ಅಹಂಕಾರವು ಹೇಗೆ ಆಗುವುದೆಂದು ನೀವು ಆಶ್ಚರ್ಯ ಪಡಬಹುದು.ಈ ಸಾಂತ ‘ನಾನು’ ಹೇಗೆ ಅನಂತವಾದೀತು – ಎಂದು ನೀವು ಆಶ್ಚರ್ಯ ಪಡಬಹುದು.ಆದರೆ ಸತ್ಯಾಂಶವೇ ಇದು. ಮಿತಿ ಇರುವುದು ಕೇವಲ ತೋರಿಕೆಗೆ. ಅನಂತ ಮಿತಿಯಾದಂತೆ ಕಾಣಿಸುವುದು. ಅದರಲ್ಲಿ ಅಲ್ಪಭಾಗ ನಾನೆಂಬ ಅಹಂಕಾರದಂತೆ ಕಾಣುತ್ತಿರುವುದು. ಅನಂತ ಎಂದಿಗೂ ಸಾಂತವಾಗ ಲಾರದು. ಇದು ಒಂದು ಕಲ್ಪನೆ. ಆದಕಾರಣ ನಮ್ಮಲ್ಲಿ ಪ್ರತಿಯೊಬ್ಬನಿಗೂ ಗಂಡಸಿಗೆ, ಹೆಂಗಸಿಗೆ ಅಥವಾ ಮಗುವಿಗೆ ಮತ್ತೆ ಎಲ್ಲಾ ಪ್ರಾಣಿಗಳಿಗೂ, ಈ ಆತ್ಮವು ತಿಳಿದಿರು ವುದು. ಆತ್ಮನ ಜ್ಞಾನವಿಲ್ಲದೆ ನಾವು ಬದುಕಲಾರೆವು, ಚಲಿಸಲಾರೆವು, ವ್ಯಕ್ತಿತ್ವ ವಿರಲಾರದು. ಈ ಪರಮಾತ್ಮನನ್ನು ನಾವು ತಿಳಿಯಲಾರದೆ ಉಸಿರಾಡಲಾರೆವು. ಒಂದು ಕ್ಷಣವಾದರೂ ಬದುಕಿರಲಾರೆವು. ಎಲ್ಲಾ ದೇವತೆಗಳಿಗಿಂತಲೂ ಹೆಚ್ಚು ಪರಿಚಿತನಾದವನು ವೇದಾಂತದ ದೇವರು. ಇದು ನಮ್ಮ ಕಲ್ಪನೆಯ ಫಲವಲ್ಲ.

ಇದೇ ಅನುಷ್ಠಾನಕ್ಕೆ ಯೋಗ್ಯವಾದ ದೇವರನ್ನು ಬೋಧಿಸುವ ವಿಧಾನ. ಇಲ್ಲದೇ ಇದ್ದರೆ ಬೇರೆ ಹೇಗೆ ಅನುಷ್ಠಾನ ಯೋಗ್ಯವಾದ ದೇವರನ್ನು ಬೋಧಿಸುವುದು? ನನ್ನ ಕಣ್ಣ ಮುಂದೆ ಕಾಣುವ ಸರ್ವವ್ಯಾಪ್ತಿಯಾದ, ನಮ್ಮ ಇಂದ್ರಿಯಗಳಿಗಿಂತಲೂ ಸತ್ಯವಾದ ದೇವರಿಗಿಂತಲೂ ಅನುಷ್ಠಾನಯೋಗ್ಯವಾದ ದೇವರು ಮತ್ತೊಂದು ಎಲ್ಲಿರುವುದು? ಏಕೆಂದರೆ ಸರ್ವವ್ಯಾಪಿಯಾದ, ಸರ್ವಶಕ್ತನಾದ, ನಿಮ್ಮ ಆತ್ಮದ ಆತ್ಮವಾದ ದೇವರೇ ನೀವು. ನೀವು ಅದಲ್ಲವೆಂದರೆ ನಾನು ಸುಳ್ಳು ಹೇಳಿದಂತೆ. ಸರ್ವಕಾಲದಲ್ಲಿಯೂ ನಾನು ಅದನ್ನು ಅನುಭವಿಸುತ್ತೇನೆಯೋ ಇಲ್ಲವೋ ಅದಿರಲಿ. ಅಂತೂ ನನಗೆ ಅದರ ಸತ್ಯ ಗೊತ್ತಿದೆ. ಆತ್ಮನೆ ಏಕ, ಎಲ್ಲದರ ಮೊತ್ತ, ಎಲ್ಲಾ ಜೀವನದ, ಎಲ್ಲಾ ಅಸ್ತಿತ್ವದ ಸತ್ಯ.

ವೇದಾಂತದ ನೈತಿಕ ಆದರ್ಶಗಳನ್ನು ವಿವರಿಸಬೇಕಾಗಿದೆ. ಆದಕಾರಣ ನಿಮಗೆ ಸ್ವಲ್ಪ ತಾಳ್ಮೆ ಇರಬೇಕು. ನಾನು ನಿಮಗೆ ಹೇಳಿದಂತೆ ವಿಶದವಾಗಿ ಈ ವಿಷಯವನ್ನು ತೆಗೆದುಕೊಂಡು ಕೂಲಂಕುಷವಾಗಿ ವಿಚಾರಿಸಬೇಕಾಗಿದೆ. ಬಹಳ ಕೆಳಗಿನ ಆದರ್ಶ ಗಳಿಂದ ವಿಚಾರಗಳು ಬೆಳೆದು, ಹೇಗೆ ಒಂದು ಏಕತ್ವದ ಮಹಾ ಆದರ್ಶವಾಗಿ ವಿಶ್ವ ಪ್ರೇಮದ ರೂಪವನ್ನು ತಾಳಿದೆ ಎಂಬುದನ್ನು ನೋಡಬೇಕು. ನಾವು ಅಪಾಯ ದಿಂದ ಪಾರಾಗಬೇಕಾದರೆ ಇದನ್ನು ತಿಳಿಯಬೇಕು. ಕೆಳಗಿನ ಮೆಟ್ಟಿಲಿನಿಂದಲೂ ಅದನ್ನು ರೂಢಿಸುವುದಕ್ಕೆ ಪ್ರಪಂಚಕ್ಕೆ ಸಮಯವಿಲ್ಲ. ನಾವು ಮೇಲೆ ನಿಂತು ನಮ್ಮ ತರುವಾಯ ಬರುವವರಿಗೆ ಸತ್ಯವನ್ನು ತೋರಿಸುವುದಕ್ಕೆ ಆಗದೇ ಇದ್ದರೆ ಪ್ರಯೋಜನವೇನು? ಆದಕಾರಣ ಅವುಗಳನ್ನು ಆಮೂಲಾಗ್ರವಾಗಿ ತಿಳಿದುಕೊಳ್ಳುವುದು ಉತ್ತಮ. ಬೌದ್ಧಿಕತೆ ಏನೂ ಅಷ್ಟು ಪ್ರಯೋಜನವಲ್ಲವೆಂದು ತಿಳಿದಿದ್ದರೂ ಮೊದಲು ಬುದ್ಧಿಗೆ ಸಂಬಂಧಪಟ್ಟ ಭಾಗವನ್ನು ಪರಿಶೀಲಿಸುವುದು ಅತ್ಯವಶ್ಯಕ. ಬಹಳ ಮುಖ್ಯವಾದುದು ಭಾವನೆ ಅಥವಾ ಹೃದಯ. ಹೃದಯದ ಮೂಲಕ ನಾವು ಭಗವಂತನನ್ನು ನೋಡುವುದು, ಬುದ್ಧಿಯ ಮೂಲಕವಲ್ಲ. ಬುದ್ಧಿ ದಾರಿಯನ್ನು ಗುರುತಿಸುವವನು, ನಮಗೋಸುಗವಾಗಿ ದಾರಿಯನ್ನು ಚೊಕ್ಕಟ ಮಾಡುವವನು. ಅಷ್ಟೇನು ಮುಖ್ಯವಾದ ಕೆಲಸಗಾರನಲ್ಲ. ಅದು ಪೋಲೀಸಿನವನಂತೆ. ಸಮಾಜದ ನಿರ್ವಹಣೆಗೆ ಪೋಲೀಸಿ ನವನು ಅನಿವಾರ್ಯ. ದೊಂಬಿಯನ್ನು ನಿಲ್ಲಿಸುವುದು, ತಪ್ಪು ಮಾಡಿದವರನ್ನು ಹಿಡಿಯುವುದು, ಇವುಗಳು ಮಾತ್ರ ಅವನ ಕೆಲಸ. ಬುದ್ಧಿ ಮಾಡಬೇಕಾದ ಕೆಲಸವೂ ಅಷ್ಟೆ. ಪಾಂಡಿತ್ಯಪೂರ್ಣವಾದ ಗ್ರಂಥವನ್ನು ಓದುವಾಗ, ಅವುಗಳೆಲ್ಲವನ್ನು ಗ್ರಹಿಸಿ ಆದ ಮೇಲೆ, “ಭಗವಂತನಿಗೆ ಜಯವಾಗಲಿ, ಸದ್ಯಕ್ಕೆ ನಾನು ಅವುಗಳಿಂದ ಪಾರಾಗಿರುವೆನು” ಎಂದು ನೀವು ಯೋಚಿಸುವಿರಿ. ಏಕೆಂದರೆ ಬುದ್ಧಿಗೆ ಕಣ್ಣಿಲ್ಲ. ತನ್ನಷ್ಟಕ್ಕೆ ತಾನೆ ಚಲಿಸಲಾರದು. ಅದಕ್ಕೆ ಕೈಗಳೂ ಇಲ್ಲ, ಕಾಲುಗಳೂ ಇಲ್ಲ. ವಿದ್ಯುತ್​ ಅಥವಾ ಮತ್ತಾವುದಾದರೂ ಶಕ್ತಿಗಿಂತ ವೇಗವಾಗಿ ಚಲಿಸಿ ಕೆಲಸಮಾಡುವುದೇ ಹೃದಯವಂತಿಕೆ. ನಿಮಗೆ ಹೃದಯವಂತಿಕೆ ಇದೆಯೇನು? ಅದೇ ಪ್ರಶ್ನೆ. ಹಾಗೆ ಹೃದಯವಂತಿಕೆ ಇರುವುದಾದರೆ ನೀವು ದೇವರನ್ನು ನೋಡಬಲ್ಲಿರಿ.ಇಂದು ನಿಮ್ಮಲ್ಲಿರುವ ಹೃದಯವಂತಿಕೆಯು ತೀವ್ರತರವಾಗುತ್ತದೆ.ಎಲ್ಲಕ್ಕೂ ಸ್ಪಂದಿಸುವ, ಎಲ್ಲದರಲ್ಲಿಯೂ ಏಕತ್ವವನ್ನು ಕಂಡು, ದೇವರನ್ನು ತನ್ನಲ್ಲಿ ಕಂಡು ಇತರರಲ್ಲಿ ಕಾಣುವ ಮಹೋನ್ನತ ಪೀಠಕ್ಕೆ ಕೊಂಡೊಯ್ಯುವುದೇ ಹೃದಯವಂತಿಕೆ. ಬುದ್ಧಿ ಅದನ್ನು ಎಂದಿಗೂ ಮಾಡಲಾರದು. “ಚಮತ್ಕಾರವಾಗಿ ಮಾತನಾಡುವ ನಾನಾ ವಿಧಾನಗಳು, ಶಾಸ್ತ್ರಗಳಿಗೆ ನಾನಾ ರೀತಿಯ ವ್ಯಾಖ್ಯಾನ ಮಾಡುವುದು ಇವು ಕೇವಲ ಪಂಡಿತರ ಮನೋರಂಜನೆಗೆ ಮಾತ್ರ. ಜೀವಿಯ ಮುಕ್ತಿಗಲ್ಲ.” (ವಿವೇಕಚೂಡಾಮಣಿ, ೫೮)

ನಿಮ್ಮಲ್ಲಿ ಯಾರು ಥಾಮಸ್​ ಎ. ಕೆಂಪಿಸ್​ನ “ಇಮಿಟೇಷನ್​ ಆಫ್​ ಕ್ರೈಸ್ಟ್​” (ಏಸುವಿನ ಹಾದಿಯಲ್ಲಿ) ಗ್ರಂಥವನ್ನು ಓದಿರುವಿರೋ ಅವರಿಗೆ ಆತನು ಹೇಗೆ ಪ್ರತಿ ಯೊಂದು ಪುಟದಲ್ಲಿಯೂ ಇದನ್ನು ಒತ್ತಿ ಹೇಳುತ್ತಾನೆ ಎಂಬುದು ಗೊತ್ತಿದೆ. ಪ್ರಪಂಚ ದಲ್ಲಿ ಪ್ರತಿಯೊಬ್ಬ ದೈವಭಕ್ತನೂ ಕೂಡ ಒತ್ತಿ ಹೇಳುವುದು ಇದನ್ನೇ. ಬುದ್ಧಿ ಆವಶ್ಯಕ. ಏಕೆಂದರೆ ಅದು ಇಲ್ಲದೇ ಇದ್ದರೆ ನಾವು ಅನೇಕ ಮೂಢ ನಂಬಿಕೆಗಳಿಗೆ ಬೀಳುತ್ತೇವೆ, ಎಷ್ಟೋ ತಪ್ಪುಗಳನ್ನು ಮಾಡುತ್ತೇವೆ. ಬುದ್ಧಿ ಅವನ್ನು ತಡೆಯುತ್ತದೆ. ಇದಕ್ಕಿಂತ ಹೆಚ್ಚಿನದನ್ನು ಆಶಿಸಬೇಡಿ. ಇದು ಅಚೇತನವಾದ, ಅಷ್ಟೇನೂ ಮುಖ್ಯವ ಲ್ಲದ ಸಹಾಯ. ನಿಜವಾದ ಸಹಾಯವೇ ಹೃದಯವಂತಿಕೆ, ಪ್ರೀತಿ. ಮತ್ತೊಬ್ಬರಿಗಾಗಿ ನೀವು ಸ್ಪಂದಿಸುತ್ತೀರೇನು? ನೀವು ಹಾಗೆ ಮಾಡಿದರೆ ಏಕತ್ವದಲ್ಲಿ ಮುಂದು ವರಿದಿರುವಿರಿ. ಇನ್ನೊಬ್ಬರಿಗಾಗಿ ನೀವು ಸ್ಪಂದಿಸದೆ ಇದ್ದರೆ, ನೀವು ಜಗತ್ತಿನ ಅತಿ ಶ್ರೇಷ್ಠ ಮೇಧಾವಿಯಾಗಿರಬಹುದು, ಆದರೆ ಅದರಿಂದ ನಿಮಗೆ ಏನೂ ಪ್ರಯೋಜನವಿಲ್ಲ. ನಿಮ್ಮದು ಕೇವಲ ಒಣಪಾಂಡಿತ್ಯ. ನೀವು ಮುಂದೆಯೂ ಹಾಗೆಯೇ ಇರುವಿರಿ. ನೀವು ಭಾವಜೀವಿಯಾದರೆ, ನಿಮಗೆ ಯಾವ ಪುಸ್ತಕವನ್ನು ಓದುವುದಕ್ಕೆ ಆಗದೆ ಇದ್ದರೂ, ಭಾಷಾಜ್ಞಾನವೇ ಇಲ್ಲದೇ ಇದ್ದರೂ, ಸರಿಯಾದ ಮಾರ್ಗದಲ್ಲಿ ರುವಿರಿ; ದೇವರು ನಿಮ್ಮವನು.

ಜಗತ್ತಿನ ಇತಿಹಾಸದಿಂದ ಮಹಾಪುರುಷರ ಶಕ್ತಿಯ ಮೂಲವು ಯಾವುದು ಎಂಬುದು ಗೊತ್ತಾಗಿಲ್ಲವೆ? ಅದು ಎಲ್ಲಿತ್ತು? ಅವರ ಬುದ್ಧಿಶಕ್ತಿಯಲ್ಲಿತ್ತೇನು? ಅವರಲ್ಲಿ ಯಾರಾದರೂ ಸೂಕ್ಷ್ಮತರ್ಕದ ಮೇಲೆ ಅಥವಾ ತತ್ತ್ವಗ್ರಂಥದ ಮೇಲೆ ಚಮತ್ಕಾರವಾದ ಪುಸ್ತಕವನ್ನು ಬರೆದರೆ? ಒಬ್ಬರೂ ಇಲ್ಲ. ಅವರು ಎಲ್ಲೋ ಸ್ವಲ್ಪ ಮಾತನಾಡಿದರು. ಕ್ರಿಸ್ತನಂತೆ ನೀವು ಇತರರಿಗಾಗಿ ಮರುಗಬಲ್ಲವರಾದರೆ ಕ್ರಿಸ್ತರಾಗುವಿರಿ. ಬುದ್ಧನಂತೆ ಹಾಗೆ ಮಾಡಿದರೆ ಬುದ್ಧರಾಗುವಿರಿ. ಹೃದಯವಂತಿಕೆಯು ಜೀವಶಕ್ತಿ ಮತ್ತು ಪೌರುಷ. ಇದಿಲ್ಲದೆ ಬುದ್ಧಿಯ ಕಸರತ್ತು ಎಷ್ಟಿದ್ದರೂ ದೇವರನ್ನು ಸೇರುವುದಕ್ಕೆ ಆಗುವುದಿಲ್ಲ. ಬುದ್ಧಿಶಕ್ತಿಯು ಸ್ವಂತ ಚಾಲನ ಶಕ್ತಿಯಿಲ್ಲದ ಅಂಗಾಂಗ ಗಳಂತೆ. ಭಾವವು ಪ್ರವೇಶಿಸಿ ಚಾಲನೆ ನೀಡಿದಾಗಲೇ ಅವು ಚಲಿಸಿ ಬೇರೆ ವಸ್ತುವಿನ ಮೇಲೆ ಪರಿಣಾಮವನ್ನು ಉಂಟುಮಾಡುವುವು. ಜಗತ್ತಿನಲ್ಲೆಲ್ಲಾ ಆಗುತ್ತಿರುವುದು ಹೀಗೆಯೇ ಹಾಗೆಯೇ, ನೀವು ಯಾವಾಗಲೂ ಜ್ಞಾಪಕದಲ್ಲಿ ಇಟ್ಟುಕೊಂಡಿರಬೇಕಾದ ವಿಷಯ ಇದು. ವೇದಾಂತದ ನೀತಿಯಲ್ಲಿ ಇದು ಒಂದು ಬಹಳ ಅನುಷ್ಠಾನ ಯೋಗ್ಯವಾದ ವಿಷಯ. ಏಕೆಂದರೆ ನೀವೆಲ್ಲರೂ ಮಹಾಪುರುಷರು, ಮಹಾ ಪುರುಷರಾಗಲೇಬೇಕು ಎಂಬುದೇ ವೇದಾಂತದ ಬೋಧನೆ. ನಿಮ್ಮ ನಡತೆಗೆ ಪ್ರಮಾಣವು ಶಾಸ್ತ್ರವಲ್ಲ, ನೀವೆ ಶಾಸ್ತ್ರಕ್ಕೆ ಪ್ರಮಾಣ. ಒಂದು ಗ್ರಂಥ ಸತ್ಯವನ್ನು ಬೋಧಿಸುತ್ತದೆ ಎಂಬುದು ಹೇಗೆ ಗೊತ್ತು? ಏಕೆಂದರೆ ನೀವೇ ಸತ್ಯ, ನೀವು ಅದನ್ನು ಭಾವಿಸುತ್ತೀರಿ. ವೇದಾಂತವು ಹೇಳುವುದೇ ಇದನ್ನು. ಜಗತ್ತಿನ ಕ್ರಿಸ್ತ ಮತ್ತು ಬುದ್ಧರಿಗೆ ಪ್ರಮಾಣ ಯಾವುದು? ನಾನು ಮತ್ತು ನೀವು ಅವರಂತೆ ಅನುಭವಗಳನ್ನು ಪಡೆಯುವುದೇ, ಅವರು ನಿಜವೆಂದು ನಾನು ಮತ್ತು ನೀವು ಕಾಣುವುದೇ ಪ್ರಮಾಣ. ಅವರ ಆತ್ಮದ ಮಹತ್ವಕ್ಕೆ ನಮ್ಮ ಆತ್ಮದ ಮಹತ್ವವೇ ಸಾಕ್ಷಿ. ನಿಮ್ಮ ದೈವತ್ವವೇ ಅವರಿಗೆ ಪ್ರಮಾಣ. ನೀವು ಒಬ್ಬ ಮಹಾಪುರುಷರಲ್ಲದೇ ಇದ್ದರೆ ದೇವರ ವಿಚಾರವಾಗಿ ಸತ್ಯವಾದುದು ಯಾವುದೂ ಇಲ್ಲ. ನೀವು ದೇವರಲ್ಲದೆ ಇದ್ದರೇ ಮತ್ತಾವ ದೇವರೂ ಎಂದೂ ಇರಲಿಲ್ಲ, ಮುಂದೆಯೂ ಇರುವುದಿಲ್ಲ. ಇದೇ ನಾವು ಅನುಸರಿಸಬೇಕಾದ ಆದರ್ಶವೆಂದು ವೇದಾಂತ ಸಾರುವುದು. ನಮ್ಮಲ್ಲಿ ಪ್ರತಿಯೊಬ್ಬರೂ ಕೂಡ ದೇವಾಂಶ ಪುರುಷರಾಗಬೇಕು. ನೀವು ಆಗಲೇ ಅದಾಗಿರುವಿರಿ. ಅದನ್ನು ನೀವು ತಿಳಿದುಕೊಳ್ಳಬೇಕಷ್ಟೆ. ಆತ್ಮನಿಗೆ ಅಸಾಧ್ಯವಾದುದು ಏನಾದರೂ ಇದೆ ಎಂದು ಎಂದೂ ತಿಳಿಯಬೇಡಿ. ಹಾಗೆ ಹೇಳುವುದೇ ನಾಸ್ತಿಕತೆ. ಪಾಪವೇನಾದರೂ ಇದ್ದರೆ, ನೀವು ನಿರ್ಬಲರೆಂದೂ, ಉಳಿದವರು ನಿರ್ಬಲರೆಂದೂ ಹೇಳುವುದೇ ಮಹಾ ಪಾಪ.

\chapter{ಅಧ್ಯಾಯ ೨}

\begin{center}
\textbf{(ಲಂಡನ್ನಿನಲ್ಲಿ ೧೮೯೬ನೇ ನವೆಂಬರ್​ ೧೨ರಂದು ನೀಡಿದ ಉಪನ್ಯಾಸ)}
\end{center}

ಬಾಲಕನೊಬ್ಬನಿಗೆ ಜ್ಞಾನವು ಹೇಗೆ ಬಂದಿತು ಎಂಬ ವಿಚಾರವಾಗಿ ಛಾಂದೋಗ್ಯ ಉಪನಿಷತ್ತಿನಲ್ಲಿರುವ ಒಂದು ಹಳೆಯ ಕಥೆಯನ್ನು ಹೇಳುತ್ತೇನೆ. ಕಥೆಯೇನೋ ಅಷ್ಟು ನಯವಾಗಿಲ್ಲ. ಆದರೆ ಅದರಲ್ಲಿ ಒಂದು ತತ್ತ್ವ ಅಡಗಿದೆ ಎಂಬುದನ್ನು ನಾವು ನೋಡುತ್ತೇವೆ. ಸಣ್ಣ ಬಾಲಕನೊಬ್ಬನು ತನ್ನ ತಾಯಿಗೆ, ನಾನು ವೇದವನ್ನು ಕಲಿಯಲು ಹೋಗುತ್ತೇನೆ, ನನ್ನ ತಂದೆಯ ಹೆಸರು ಮತ್ತು ಜಾತಿಯನ್ನು ಹೇಳು ಎಂದನು. ತಾಯಿಯು ಶಾಸ್ತ್ರಪ್ರಕಾರ ಮದುವೆಯಾದವಳಲ್ಲ. ಭಾರತ ದೇಶದಲ್ಲಿ ಅವಿವಾಹಿತ ಸ್ತ್ರೀಯ ಮಗುವನ್ನು ಜಾತಿಭ್ರಷ್ಠನೆನ್ನುತ್ತಾರೆ. ಸಮಾಜವು ಅವನನ್ನು ಸ್ವೀಕರಿಸುವುದಿಲ್ಲ. ವೇದಾಧ್ಯಯನಕ್ಕೆ ಅವನು ಅರ್ಹನಲ್ಲ. ಆದಕಾರಣ, ಪಾಪ ತಾಯಿ, “ಮಗು, ನನಗೆ ನಿನ್ನ ವಂಶದ ಹೆಸರು ಗೊತ್ತಿಲ್ಲ. ನಾನು ಕೆಲಸದಲ್ಲಿದ್ದೆ. ಅನೇಕ ಕಡೆ ಕೆಲಸ ಮಾಡಿದೆ, ನನಗೆ ನಿನ್ನ ತಂದೆ ಯಾರೋ ಗೊತ್ತಿಲ್ಲ. ಆದರೆ ನನ್ನ ಹೆಸರು ಜಬಾಲಾ. ನಿನ್ನ ಹೆಸರು ಸತ್ಯಕಾಮ” ಎಂದಳು. ಆ ಹುಡುಗನು ಒಬ್ಬ ಋಷಿಯ ಹತ್ತಿರ ಹೋಗಿ ವಿದ್ಯಾರ್ಥಿಯಂತೆ ತನ್ನನ್ನು ಸ್ವೀಕರಿಸಬೇಕೆಂದು ಕೇಳಿ ಕೊಂಡನು. ಋಷಿ, “ನಿನ್ನ ತಂದೆಯ ಹೆಸರು ಏನು? ನಿನ್ನ ಜಾತಿ ಯಾವುದು?” ಎಂದು ಪ್ರಶ್ನೆ ಮಾಡಿದನು. ತಾಯಿಯಿಂದ ಏನನ್ನು ಕೇಳಿದ್ದನೊ ಅದನ್ನು ಹುಡುಗನು ಪುನಃ ಹೇಳಿದನು. ಋಷಿಯು ತಕ್ಷಣವೆ “ಬ್ರಾಹ್ಮಣನಲ್ಲದೆ ಮತ್ತಾರೂ ತನ್ನ ಶೀಲಕ್ಕೆ ಕಳಂಕ ತರುವ ಸತ್ಯವನ್ನು ಹೇಳನು. ನೀನು ಬ್ರಾಹ್ಮಣ, ನಾನು ನಿನಗೆ ಉಪದೇಶ ಕೊಡುತ್ತೇನೆ, ನೀನು ಸತ್ಯವನ್ನು ತ್ಯಜಿಸಿಲ್ಲ” ಎಂದನು. ಹೀಗೆ ಆ ಹುಡುಗನನ್ನು ತನ್ನ ಹತ್ತಿರ ಇಟ್ಟುಕೊಂಡು ಅವನಿಗೆ ವಿದ್ಯೆಯನ್ನು ಕಲಿಸಿದನು.

ಹಿಂದೆ ಭರತಖಂಡದಲ್ಲಿದ್ದ ಕೆಲವು ವಿಚಿತ್ರ ವಿದ್ಯಾಭ್ಯಾಸ ಕ್ರಮಗಳಿಗೆ ಈಗ ಬರುತ್ತೇವೆ. ಗುರುವು ಸತ್ಯಕಾಮನಿಗೆ ನಾನೂರು ನಿರ್ಬಲವಾದ ಬಡಕಲ ಆಕಳನ್ನು ಕೊಟ್ಟು ಅವನ್ನು ನೋಡಿಕೊಳ್ಳುವಂತೆ ಹೇಳಿ ಕಾಡಿಗೆ ಕಳುಹಿಸಿದನು. ಹುಡುಗನು ಅಲ್ಲಿ ಹೋಗಿ ಕೆಲವು ದಿನಗಳು ಇದ್ದನು. ಗುರುವು ಶಿಷ್ಯನಿಗೆ ದನಗಳು ಒಂದು ಸಾವಿರದ ಸಂಖ್ಯೆಯನ್ನು ತಲುಪಿದಾಗ ಹಿಂದಿರುಗುವಂತೆ ಹೇಳಿದ್ದನು. ಕೆಲವು ವರುಷ ಗಳ ತರುವಾಯ ಒಂದು ದಿನ ದೊಡ್ಡ ಎತ್ತೊಂದು ಸತ್ಯಕಾಮನಿಗೆ, “ನಾವೀಗ ಒಂದು ಸಾವಿರವಾಗಿರುವೆವು. ನಿನ್ನ ಗುರುವಿನ ಬಳಿಗೆ ನಮ್ಮನ್ನು ಕರೆದುಕೊಂಡು ಹೋಗು. ನಾನು ನಿನಗೆ ಸ್ವಲ್ಪ ಬ್ರಹ್ಮದ ವಿಚಾರವನ್ನು ಹೇಳುತ್ತೇನೆ” ಎಂದಿತು. ಸತ್ಯಕಾಮನು, “ಹೇಳಿ ಸ್ವಾಮಿ” ಎಂದನು. ಆಗ ಎತ್ತು, “ಪೂರ್ವದಿಕ್ಕು ದೇವರ ಒಂದು ಅಂಶ. ದಕ್ಷಿಣ ಪಶ್ಚಿಮ ಉತ್ತರ ದಿಕ್ಕುಗಳೂ ದೇವರ ಅಂಶಗಳೇ. ಪ್ರಧಾನವಾದ ಈ ಮುಖ್ಯ ದಿಕ್ಕುಗಳು ಬ್ರಹ್ಮನ ನಾಲ್ಕು ಭಾಗಗಳು. ಅಗ್ನಿಯು ಬ್ರಹ್ಮನ ವಿಚಾರವಾಗಿ ನಿನಗೆ ಸ್ವಲ್ಪ ಹೇಳುತ್ತದೆ” ಎಂದಿತು. ಆಗಿನ ಕಾಲದಲ್ಲಿ ಅಗ್ನಿಯು ಒಂದು ಪವಿತ್ರವಾದ ಚಿಹ್ನೆಯಾಗಿತ್ತು. ಪ್ರತಿಯೊಬ್ಬವಿದ್ಯಾರ್ಥಿಯೂ ಬೆಂಕಿಯನ್ನು ತಂದು ಅದಕ್ಕೆ ಆಹುತಿಯನ್ನು ಕೊಡಬೇಕಾಗಿತ್ತು. ಮಾರನೆಯ ದಿನ ಸತ್ಯಕಾಮನು ತನ್ನ ಗುರುವಿನ ಮನೆಗೆ ಹೊರಟನು. ಸಾಯಂಕಾಲ ತನ್ನ ಅಹ್ನಿಕವನ್ನು ತೀರಿಸಿದ ಮೇಲೆ ಅಗ್ನಿಯನ್ನು ಪೂಜಿಸಿ ಅದರ ಹತ್ತಿರ ಕುಳಿತುಕೊಂಡಿರುವಾಗ ಬೆಂಕಿಯಿಂದ “ಓ, ಸತ್ಯಕಾಮ” ಎಂಬ ಧ್ವನಿ ಕೇಳಿಸಿತು. ಸತ್ಯಕಾಮನು, “ಮಾತನಾಡಿ ಸ್ವಾಮಿ” ಎಂದನು (ಸ್ಯಾಮುಯಲ್ಲನು ರಹಸ್ಯವಾದ ಧ್ವನಿಯನ್ನು ಹೇಗೆ ಕೇಳಿದನು ಎಂಬ ಸಂಗತಿಯನ್ನು ನೀವು ಬೈಬಲಿನಲ್ಲಿ ಓದಿರಬಹುದು). “ಓ ಸತ್ಯಕಾಮ, ನಾನು ನಿನಗೆ ಸ್ವಲ್ಪ ಬ್ರಹ್ಮದ ವಿಷಯವನ್ನು ತಿಳಿಸುವುದಕ್ಕೆ ಬಂದಿರುವೆನು. ಈ ಪೃಥ್ವಿಯು ಆ ಬ್ರಹ್ಮದ ಒಂದು ಭಾಗ; ಅಂತರಿಕ್ಷ ಮತ್ತು ಸ್ವರ್ಗಗಳು ಅದರ ಒಂದು ಅಂಶ. ಸಾಗರವೂ ಕೂಡ ಅದರ ಒಂದು ಅಂಶ.” ಅನಂತರ ಅಗ್ನಿಯು ಯಾವುದೋ ಒಂದು ಹಕ್ಕಿಯು ಅವನಿಗೆ ಬ್ರಹ್ಮದ ವಿಚಾರವನ್ನು ಸ್ವಲ್ಪ ತಿಳಿಸುವು ದೆಂದಿತು. ಸತ್ಯಕಾಮನು ತನ್ನ ಪಯಣವನ್ನು ಮುಂದುವರಿಸಿದನು. ಮಾರನೆ ದಿನ ಅವನು ತನ್ನ ಸಂಜೆಯ ಹೋಮ ಮುಂತಾದವುಗಳನ್ನು ತೀರಿಸಿ ಆದ ಮೇಲೆ ಒಂದು ಹಂಸ ಬಂದಿತು. “ನಾನು ನಿನಗೆ ಬ್ರಹ್ಮದ ವಿಚಾರವನ್ನು ಸ್ವಲ್ಪ ತಿಳಿಸುತ್ತೇನೆ. ನೀನು ಪೂಜಿಸುವ ಈ ಅಗ್ನಿಯು ಬ್ರಹ್ಮದ ಒಂದು ಅಂಶ, ಸೂರ್ಯನು ಅದರ ಒಂದು ಅಂಶ, ಚಂದ್ರನು ಅದರ ಒಂದು ಅಂಶ, ಮಿಂಚು ಅದರ, ಒಂದು ಅಂಶ. ಮದ್ಗು ಎಂಬ ಹಕ್ಕಿಯು ಅದರ ವಿಚಾರವಾಗಿ ಹೆಚ್ಚು ಹೇಳುವುದು” ಎಂದಿತು. ಮಾರನೆ ದಿನ ಸಾಯಂಕಾಲ ಆ ಪಕ್ಷಿ ಬಂದಿತು. ಹಿಂದಿನಂತೆಯೇ ಧ್ವನಿ ಸತ್ಯಕಾಮನಿಗೆ ಕೇಳಿಸಿತು: “ಬ್ರಹ್ಮದ ವಿಚಾರವಾಗಿ ನಾನು ನಿನಗೆ ಸ್ವಲ್ಪ ಹೇಳುತ್ತೇನೆ. ಪ್ರಾಣ ಬ್ರಹ್ಮದ ಒಂದು ಭಾಗ. ನೋಟ ಒಂದು ಭಾಗ. ಕೇಳುವುದು ಒಂದು ಭಾಗ. ಮನಸ್ಸು ಒಂದು ಭಾಗ”. ಅನಂತರ ಹುಡುಗನು ಗುರುವಿನ ಸಮೀಪಕ್ಕೆ ಹೋಗಿ ಭಕ್ತಿಪೂರ್ವಕವಾಗಿ ಅವರಿಗೆ ವಂದಿಸಿದನು. ಶಿಷ್ಯನ ಮುಖ್ಯವನ್ನು ನೋಡಿದ ತಕ್ಷಣವೇ ಗುರು, “ಸತ್ಯಕಾಮ, ಬ್ರಹ್ಮ ಜ್ಞಾನಿಯಂತೆ ನಿನ್ನ ಮುಖ ಕಂಗೊಳಿಸುತ್ತಿದೆ. ಹಾಗಾದರೆ ಯಾರು ನಿನಗೆ ಕಲಿಸಿದರು?” ಎಂದು ಕೇಳಿದನು. “ಮನುಷ್ಯರಲ್ಲದ ಬೇರೆ ಜೀವಿಗಳು” ಎಂದನು ಸತ್ಯಕಾಮ. “ಆದರೆ ಪೂಜ್ಯರೆ, ನೀವು ನನಗೆ ಬೋಧಿಸಬೇಕು. ಏಕೆಂದರೆ ಗುರುವಿನಿಂದ ಕಲಿತ ವಿದ್ಯೆಯೊಂದೆ ಶ್ರೇಯೋಭಿವೃದ್ಧಿಗೆ ದಾರಿ ಎಂದು ನಿಮ್ಮಂತಹ ಮಹಾತ್ಮರಿಂದ ಕೇಳಿರುವೆನು” ಎಂದನು. ಗುರುಗಳು ದೇವತೆಗಳಿಂದ ಪ್ರಾಪ್ತವಾದ ಅದೇ ಬ್ರಹ್ಮಜ್ಞಾನವನ್ನು ಶಿಷ್ಯನಿಗೆ ಬೋಧಿಸಿದರು. “ಏನನ್ನೂ ಬಿಡಲಿಲ್ಲ, ಏನನ್ನೂ ಬಿಡಲಿಲ್ಲ.”

ಈಗ ಎತ್ತು ಬೆಂಕಿ ಮತ್ತು ಹಕ್ಕಿಗಳು ಬೋಧಿಸಿರುವ ಈ ದೃಷ್ಟಾಂತ ಕಥೆಗಳ ಅರ್ಥ ಏನೇ ಇರಲಿ, ಆಗಿನ ಕಾಲದಲ್ಲಿ ಆಲೋಚನೆಯ ರೀತಿ ಮತ್ತು ಅದು ಯಾವ ದಿಕ್ಕಿನಲ್ಲಿ ಹೋಗುತ್ತಿತ್ತು ಎಂಬುದನ್ನು ನಾವು ನೋಡುವೆವು. ಮಹದಾ ಲೋಚನೆಯ ಬೀಜರೂಪವನ್ನು ನಾವಿಲ್ಲಿ ನೋಡುವೆವು. ಈ ಧ್ವನಿಗಳೆಲ್ಲವೂ ನಮ್ಮೊ ಳಗೇ ಇರುವುವು. ಈ ಸತ್ಯಗಳು ನಮಗೆ ಚೆನ್ನಾಗಿ ಅರ್ಥವಾದಂತೆಲ್ಲ, ಈ ಧ್ವನಿ ನಮ್ಮ ಹೃದಯದಲ್ಲಿಯೇ ಇರುವುದು ಎಂದು ಗೊತ್ತಾಗುವುದು. ಸತ್ಯವನ್ನೇ ಕೇಳು ತ್ತಿರುವುದಾಗಿ ಶಿಷ್ಯನಿಗೆ ಅರ್ಥವಾಗಿತ್ತು. ಆದರೆ ಅವನು ಕೊಟ್ಟ ವಿವರಣೆ ಸರಿಯಾಗಿರಲಿಲ್ಲ. ಧ್ವನಿ ಯಾವಾಗಲೂ ತನ್ನಲ್ಲಿಯೇ ಇದ್ದರೂ ಅದು ಹೊರಗಿನಿಂದ ಬರುತ್ತಿತ್ತೆಂದು ವ್ಯಾಖ್ಯಾನ ಮಾಡುತ್ತಿದ್ದನು. ನಮಗೆ ಸಿಕ್ಕುವ ಎರಡನೆಯ ಅಭಿಪ್ರಾಯವೆ, ಆತ್ಮಜ್ಞಾನವನ್ನು ಕಾರ್ಯಕಾರಿಯನ್ನಾಗಿ ಮಾಡುವುದು. ಧರ್ಮದ ಅನುಷ್ಠಾನ ಸಾಧ್ಯತೆಗಳನ್ನು ಈ ಜಗತ್ತು ಯಾವಾಗಲೂ ಹುಡುಕುತ್ತಿದೆ. ಈ ಕಥೆಗಳಲ್ಲಿ ಇವು ಕ್ರಮೇಣ ಹೇಗೆ ಹೆಚ್ಚು ಹೆಚ್ಚಾಗಿ ಅಭ್ಯಾಸ ಸಾಧ್ಯವಾಗುತ್ತಿತ್ತು ಎಂಬುದನ್ನು ನೋಡುವೆವು. ಶಿಷ್ಯನಿಗೆ ಪರಿಚಯವಾದ ಎಲ್ಲ ವಸ್ತುಗಳ ಮೂಲಕವೂ ಸತ್ಯವು ತೋರಿಸಲ್ಪಟ್ಟಿತು. ಅವರು ಪೂಜಿಸುತ್ತಿದ್ದ ಅಗ್ನಿಯೂ ಬ್ರಹ್ಮ, ಪೃಥ್ವಿ ಬ್ರಹ್ಮನ ಒಂದು ಭಾಗ ಇತ್ಯಾದಿ.

ಅನಂತರದ ಕಥೆಯೇ ಸತ್ಯಕಾಮನ ಶಿಷ್ಯನಾದ ಉಪಕೋಸಲ ಕಾಮಲಾಯ ನನದು. ಅವನಿಂದ ಉಪದೇಶವನ್ನು ಪಡೆಯಲು ಹೋಗಿ, ಕಾಮಲಾಯನನು ಅಲ್ಲಿಯೇ ಗುರುವಿನೊಂದಿಗೆ ಕೆಲವು ಕಾಲವಿದ್ದನು. ಒಮ್ಮೆ ಸತ್ಯಕಾಮನು ಎಲ್ಲಿಗೋ ಪ್ರಯಾಣ ಮಾಡಿದನು. ಶಿಷ್ಯನು ಬಹಳ ಖಿನ್ನನಾದನು. ಗುರುಪತ್ನಿ ಬಂದು ಶಿಷ್ಯನನ್ನು ಏತಕ್ಕೆ ಊಟಮಾಡುತ್ತಿಲ್ಲವೆಂದು ಪ್ರಶ್ನಿಸಿದಾಗ, “ನನಗೆ ತುಂಬ ದುಃಖವಾಗಿದೆ. ಊಟ ಮಾಡಲಾರೆ” ಎಂದು ಉತ್ತರ ಕೊಟ್ಟನು. ಅವನು ಪೂಜಿಸುತ್ತಿದ್ದ ಬೆಂಕಿಯಿಂದ ಧ್ವನಿಯೊಂದು, “ಈ ಪ್ರಾಣವು ಬ್ರಹ್ಮ, ಬ್ರಹ್ಮನೇ ಆಕಾಶ, ಬ್ರಹ್ಮನೇ ಆನಂದ. ಬ್ರಹ್ಮನನ್ನು ತಿಳಿ” ಎಂದಿತು. ಹುಡುಗನು, “ಸ್ವಾಮಿ, ಪ್ರಾಣವೂ ಬ್ರಹ್ಮವೆಂದೇನೋ ಗೊತ್ತಾಗುತ್ತದೆ. ಆದರೆ ಅದು ಆಕಾಶ ಮತ್ತು ಆನಂದವೆನ್ನುವುದು ಗೊತ್ತಾಗು ವುದಿಲ್ಲ” ಎಂದನು.ಆಗ ಬೆಂಕಿಯು, ಆಕಾಶ ಆನಂದಗಳೆಂಬ ಎರಡು ಪದಗಳೂ ಒಂದೇ ವಸ್ತುವನ್ನು, ಎಂದರೆ ಹೃದಯದಲ್ಲಿರುವ ಸಚೇತನವಾದ ಆಕಾಶ (ಶುದ್ಧ ಚೈತನ್ಯ)ವನ್ನು ಸೂಚಿಸುತ್ತದೆ ಎಂದಿತು. ಬ್ರಹ್ಮನೇ ಪ್ರಾಣವೆಂದೂ, ಹೃದಯಾಕಾಶ ವೆಂದೂ ಬೋಧಿಸಿತು. ಅನಂತರ ಬೆಂಕಿ, “ನೀನು ಪೂಜಿಸುವ ಪೃಥ್ವಿ, ಅನ್ನ, ಬೆಂಕಿ, ಸೂರ್ಯ ಇವುಗಳೆಲ್ಲ ಬ್ರಹ್ಮನ ಆಕಾರಗಳು. ಸೂರ್ಯನಲ್ಲಿ ಕಾಣುವ ಪುರುಷನೂ ನಾನೆ. ಯಾರಿಗೆ ಇದು ಗೊತ್ತಿದೆಯೋ, ಯಾರು ಅವನ ಮೇಲೆ ಧ್ಯಾನ ಮಾಡುತ್ತಾನೆಯೊ, ಅವನ ಪಾಪವೆಲ್ಲ ಕ್ಷೀಣಿಸುವುದು. ದೀರ್ಘಾಯುಷಿಯಾಗಿ ಸುಖವಾಗಿ ಬಾಳುವನು. ಯಾರು ದಶದಿಕ್ಕುಗಳಲ್ಲಿರುವನೋ, ಯಾರು ಚಂದ್ರನಲ್ಲಿ, ನಕ್ಷತ್ರದಲ್ಲಿ, ಜಲದಲ್ಲಿ ಇರುವನೋ, ಅವನೇ ನಾನು. ಯಾರು ಪ್ರಾಣದಲ್ಲಿ, ಆಕಾಶದಲ್ಲಿ, ಸ್ವರ್ಗದಲ್ಲಿ ಮತ್ತು ಜ್ಯೋತಿಯಲ್ಲಿ ವಾಸಿಸುವನೊ ಅವನೇ ನಾನು” ಎಂದು ಬೋಧಿಸಿತು. ಇಲ್ಲಿಯೂ ಕೂಡ ಅದೇ ಅನುಷ್ಠಾನ ಧರ್ಮದ ಭಾವವನ್ನೇ ನೋಡುತ್ತೇವೆ.ಅವರು ಪೂಜಿಸುತ್ತಿದ್ದ ಬೆಂಕಿ ಸೂರ್ಯ ಚಂದ್ರ ಮತ್ತು ಅವರಿಗೆ ಪರಿಚಿತವಾದ ಧ್ವನಿ ಇವೇ ಕಥಾವಸ್ತುವಾಗಿ, ಅವಕ್ಕೆ ಒಂದು ಉನ್ನತವಾದ ಅರ್ಥವನ್ನು ಕೊಟ್ಟು ವಿವರಿಸುತ್ತಾರೆ. ಇದೇ ನಿಜವಾಗಿಯೂ ವೇದಾಂತದ ಅನುಷ್ಠಾನ ಭಾಗ. ಇದು ಪ್ರಪಂಚವನ್ನು ನಾಶಮಾಡುವುದಿಲ್ಲ; ಆದರೆ ಅದನ್ನು ವಿವರಿಸುತ್ತದೆ. ವ್ಯಕ್ತಿತ್ವವನ್ನು ನಾಶಮಾಡುವುದಿಲ್ಲ, ಆದರೆ ಅದನ್ನು ವಿವರಿಸುತ್ತದೆ. ಮಾನವನ ವೈಶಿಷ್ಟ್ಯವನ್ನು ನಾಶಮಾಡುವುದಿಲ್ಲ, ಆದರೆ ಅವನ ನಿಜವಾದ ವೈಶಿಷ್ಟ್ಯವನ್ನು ತೋರಿ ಅದನ್ನು ವಿವರಿಸುತ್ತದೆ. ತಮ್ಮ ಕಣ್ಣ ಎದುರಿಗೆ ಕಾಣುವ ಜಗತ್ತು ಕೆಲಸಕ್ಕೆ ಬಾರದುದೆಂದೂ, ಮತ್ತು ಅದಕ್ಕೆ ಅಸ್ತಿತ್ವವಿಲ್ಲವೆಂದೂ ಹೇಳದೆ, ಪ್ರಪಂಚ ನಿಮ್ಮನ್ನು ನೋಯಿಸದೆ ಇರಬೇಕಾದರೆ ಅದು ಏನೆಂಬುದನ್ನು ತಿಳಿದುಕೊಳ್ಳಿ ಎನ್ನುತ್ತದೆ. ಉಪಕೋಸಲನಿಗೆ ಆ ಧ್ವನಿಯು, ಅವನು ಪೂಜಿಸುತ್ತಿದ್ದ ಬೆಂಕಿ, ಸೂರ್ಯ, ಚಂದ್ರ, ವಿದ್ಯುತ್​ ಮತ್ತು ಇನ್ನೂ ಇತರ ವಸ್ತುಗಳೆಲ್ಲ ತಪ್ಪೆಂದು ಹೇಳಲಿಲ್ಲ. ಆದರೆ ಯಾವ ಶಕ್ತಿಯು ಸೂರ್ಯ, ಚಂದ್ರ ಬೆಂಕಿ ವಿದ್ಯುತ್​ ಪೃಥ್ವಿ ಇವುಗಳಲ್ಲಿ ದೆಯೋ ಅದೇ ಶಕ್ತಿಯೇ ತನ್ನಲ್ಲಿದೆ ಎಂದಿತು. ಇದರಿಂದ ಉಪಕೋಸಲನಿಗೆ ಇವುಗಳೆಲ್ಲವೂ ರೂಪಾಂತರವನ್ನು ಹೊಂದಿದಂತೆ ತೋರಿದವು. ಮೊದಲು ಯಾವ ಬೆಂಕಿ ಆಹುತಿಯನ್ನು ಸ್ವೀಕರಿಸುವ ಕೇವಲ ವಾಸ್ತವಿಕ ಬೆಂಕಿಯಾಗಿದ್ದಿತೋ, ಅದು ರೂಪಾಂತರ ಹೊಂದಿ ದೇವರಾಯಿತು. ಭೂಮಿಯು ರೂಪಾಂತರ ಹೊಂದಿತು. ಜೀವನವು ರೂಪಾಂತರ ಹೊಂದಿತು. ಸೂರ್ಯ ಚಂದ್ರ ತಾರಾವಳಿಗಳು ಮಿಂಚು ಮತ್ತು ಸಕಲವೂ ರೂಪಾಂತರ ಹೊಂದಿ ದೈವತ್ವವನ್ನು ತಾಳಿದವು. ಅವುಗಳ ನಿಜಸ್ವರೂಪವು ತಿಳಿಯಿತು. ಸಕಲದರಲ್ಲಿಯೂ ದೇವರನ್ನು ನೋಡುವುದು, ವಸ್ತು ಗಳು ನಮಗೆ ಕಾಣುವಂತೆ ನೋಡದೆ ಅವುಗಳ ಅಂತರಾಳದಲ್ಲಿರುವ ಸತ್ಯಾಂಶವನ್ನು ಮನಗಾಣಿಸುವುದೇ ವೇದಾಂತದ ಗುರಿ. ಅನಂತರ ಉಪನಿಷತ್ತಿನಲ್ಲಿ ಮತ್ತೊಂದು ಬೋಧನೆ ಬರುತ್ತದೆ. “ಯಾರು ನಮ್ಮ ಕಣ್ಣುಗಳ ಮೂಲಕ ಪ್ರಕಾಶಿಸುವನೋ ಅವನೇ ಬ್ರಹ್ಮ. ಅವನೇ ಸೌಂದರ್ಯಮಯನು. ಅವನೇ ಕಾಂತಿಮಯನು. ಈ ಪ್ರಪಂಚದಲ್ಲೆಲ್ಲ ಪ್ರಕಾಶಿಸುತ್ತಿರುವವನೂ ಅವನೆ.” ಕಣ್ಣಿನಲ್ಲಿರುವ ಬೆಳಕು ಎಂಬುದರ ಅರ್ಥ, ಪರಿಶುದ್ಧಾತ್ಮರಿಗೆ ಬರುವ ಒಂದು ಬಗೆಯ ವಿಚಿತ್ರವಾದ ಕಾಂತಿ ಎಂದು. ವ್ಯಕ್ತಿಯು ಪರಿಶುದ್ಧನಾದ ಮೇಲೆ ಆತನ ಕಣ್ಣಿನಲ್ಲಿ ಅಂತಹ ಕಾಂತಿ ಕಂಗೊಳಿಸು ತ್ತದೆ ಎಂದು ಭಾಷ್ಯಕಾರರೊಬ್ಬರು ಹೇಳುತ್ತಾರೆ. ಆ ಕಾಂತಿಯು ಸರ್ವಾಂತರ್ ಯಾಮಿಯಾದ ಅಂತರಾತ್ಮನದು. ಆ ಕಾಂತಿಯು ಗ್ರಹಗಳಲ್ಲಿ, ತಾರಾವಳಿ ಗಳಲ್ಲಿ ಸೂರ್ಯನಲ್ಲಿ ಪ್ರಕಾಶಿಸುತ್ತಿರುವುದು.

ಈ ಪುರಾತನ ಉಪನಿಷತ್ತುಗಳಲ್ಲಿ ಜನನ, ಮರಣ ಮುಂತಾದ ವಿಷಯಗಳಿಗೆ ಸಂಬಂಧಪಟ್ಟ ಮತ್ತೊಂದು ಸಿದ್ಧಾಂತವನ್ನು ನಿಮಗೆ ಓದಿ ಹೇಳುತ್ತೇನೆ. ಬಹುಶಃ ನಿಮಗೆ ಅದರಲ್ಲಿ ಆಸಕ್ತಿ ಇರಬಹುದು. ಶ್ವೇತಕೇತುವು ಪಾಂಚಾಲ ದೇಶದ ರಾಜನ ಬಳಿಗೆ ಹೋದನು. ರಾಜನು, “ಮಾನವರು ಕಾಲವಾದ ಮೇಲೆ ಎಲ್ಲಿಗೆ ಹೋಗುತ್ತಾರೆ ಎಂಬುದು ನಿನಗೆ ಗೊತ್ತೇ? ಅವರು ಹೇಗೆ ಹಿಂದಕ್ಕೆ ಬರುತ್ತಾರೆ ಎಂಬುದು ನಿನಗೆ ಗೊತ್ತಿದೆಯೆ? ಬೇರೆ ಲೋಕವು ಏತಕ್ಕೆ ತುಂಬಿಹೋಗುವುದಿಲ್ಲ ಎಂಬುದು ನಿನಗೆ ಗೊತ್ತಿದೆಯೆ?” ಎಂದು ಪ್ರಶ್ನೆ ಮಾಡಿದನು. ಹುಡುಗನು ತನಗೆ ಗೊತ್ತಿಲ್ಲವೆಂದು ಹೇಳಿದನು. ಅನಂತರ ಅವನು ತಂದೆಯ ಬಳಿಗೆ ಹೋಗಿ ಅದೇ ಪ್ರಶ್ನೆಗಳನ್ನು ಕೇಳಿದನು. ತಂದೆಯು “ನನಗೂ ಗೊತ್ತಿಲ್ಲ” ಎಂದನು. ಅನಂತರ ಅವನು ರಾಜನ ಬಳಿಗೆ ಹೋದನು. ಈ ಜ್ಞಾನ ಬ್ರಾಹ್ಮಣರಿಗೆ ಎಂದಿಗೂ ಗೊತ್ತಿರಲಿಲ್ಲವೆಂದೂ, ಅದು ರಾಜರಿಗೆ ಮಾತ್ರ ತಿಳಿದಿತ್ತೆಂದೂ ರಾಜನು ಹೇಳಿದನು. ಆದಕಾರಣವೆ ರಾಜರು ದೇಶವನ್ನು ಆಳುತ್ತಿದ್ದರು. ಆತನು ಕೆಲವು ಕಾಲ ರಾಜನ ಸೇವೆ ಮಾಡಿದನು. ಕೊನೆಗೆ ರಾಜನು ಅದನ್ನು ಬೋಧಿಸುತ್ತೇನೆಂದನು. “ಓ ಗೌತಮ, ಆ ಪರಲೋಕವೆ ಅಗ್ನಿ. ಸೂರ್ಯನೆ ಅದಕ್ಕೆ ಸೌದೆ, ಕಿರಣಗಳೇ ಧೂಮ, ಹಗಲೇ ಉರಿ, ಚಂದ್ರನೇ ತನಿಗೆಂಡ, ತಾರಾವಳಿಗಳೆ ಕಿಡಿಗಳು. ಈ ಅಗ್ನಿಯಲ್ಲಿ ದೇವತೆಗಳು ಶ್ರದ್ಧೆಯ ಆಹುತಿಯನ್ನು ಅರ್ಪಿಸುವರು. ಈ ಆಹುತಿಯಿಂದ ಸೋಮರಾಜನು ಜನಿಸುವನು.” ಈ ರೀತಿಯಲ್ಲಿ ಅವನು ಹೇಳುತ್ತಾ ಹೋಗುವನು. “ಆ ಸಣ್ಣ ಅಗ್ನಿಗೆ ನೀನು ಆಹುತಿಯನ್ನು ಕೊಡಬೇಕಾಗಿಲ್ಲ. ಈ ವಿಶ್ವವೇ ಆ ಅಗ್ನಿ. ಈ ಆಹುತಿಯು, ಈ ಪೂಜಯು ಅನವರತವೂ ನಡೆಯುತ್ತಿರುವುದು. ದೇವತೆಗಳು, ಯಕ್ಷರು, ಮಾನವರು ಎಲ್ಲರೂ ಅದನ್ನು ಪೂಜಿಸುತ್ತಿರುವರು. ಮಾನವ ದೇಹವೆ ಅಗ್ನಿಯ ಅತ್ಯುತ್ತಮವಾದ ಚಿಹ್ನೆ.” ಇಲ್ಲಿಯೂ ಕೂಡ ಆದರ್ಶವು ಕಾರ್ಯಕಾರಿಯಾಗುತ್ತಿರುವುದನ್ನು ನಾವು ನೋಡು ತ್ತೇವೆ. ಚರಾಚರ ವಸ್ತುಗಳಲ್ಲೆಲ್ಲಾ ಬ್ರಹ್ಮನು ಸಂಚರಿಸುತ್ತಿರುವನು. ಈ ಕಥೆಗಳ ಲ್ಲೆಲ್ಲಾ ಅಂತರ್ಗತವಾಗಿರುವ ತತ್ತ್ವವೇನೆಂದರೆ ನಾವು ಕಂಡುಹಿಡಿದ ಚಿಹ್ನೆಗಳು ಒಳ್ಳೆಯದಾಗಿರಬಹುದು, ಮತ್ತು ನಮಗೆ ಸಹಾಯ ಮಾಡಬಹುದು; ಆದರೆ ನಾವು ಕಂಡುಹಿಡಿಯಬಹುದಾದ ಚಿಹ್ನೆಗಳೆಲ್ಲಕ್ಕಿಂತಲೂ ಉತ್ತಮವಾದುದೊಂದು ಇದೆ ಎನ್ನುವುದು. ದೇವರ ಪೂಜೆಗೆ ಅನುಕೂಲವಾಗಲೆಂದು ನೀವೊಂದು ವಿಗ್ರಹವನ್ನು ಕಂಡುಹಿಡಿಯಬಹುದು. ಅದಕ್ಕಿಂತ ಉತ್ತಮವಾದ ಚಿಹ್ನೆಯೊಂದು ಜೀವಂತ ನಾಗಿರುವ ಮನುಷ್ಯನೇ. ದೇವರನ್ನು ಪೂಜಿಸುವುದಕ್ಕೋಸುಗ ನೀವೊಂದು ಗುಡಿಯನ್ನು ಕಟ್ಟಬಹುದು. ಅದು ಸುಂದರವಾಗಿರಬಹುದು. ಆದರೆ ಅದಕ್ಕಿಂತ ಮೇಲಾದ, ಅದ ಕ್ಕಿಂತ ಮತ್ತೂ ಉತ್ತಮವಾದ ಮಾನವ ದೇಹವು ಈಗಾಗಲೇ ಇರುವುದು.

ವೇದಗಳಲ್ಲಿ ಜ್ಞಾನಕಾಂಡ, ಕರ್ಮಕಾಂಡಗಳೆಂಬ ಎರಡು ಭಾಗವಿರುವುದು ನಿಮಗೆ ಗೊತ್ತಿದೆ. ಕೆಲವು ಕಾಲಾನಂತರ ಕರ್ಮಕಾಂಡವು ಬಹಳ ವೃದ್ಧಿಯಾಗಿ ತೊಡಕಾಯಿತು. ಅವುಗಳನ್ನು ಬಿಡಿಸುವುದೇ ಬಹಳ ಕಷ್ಟವಾಯಿತು. ಆದಕಾರಣ ಉಪನಿಷತ್ತುಗಳಲ್ಲಿ ಕರ್ಮಕಾಂಡದ ಹೆಚ್ಚು ಭಾಗವನ್ನೇ ವಜಾಮಾಡಿರುವುದು ನಮಗೆ ತೋರುತ್ತದೆ. ಆದರೆ ನಿಧಾನವಾಗಿ, ಅವುಗಳನ್ನು ವಿವರಿಸಿ ಬದಲಾವಣೆಗಳನ್ನು ತಂದಿರುವರು. ಪೂರ್ವ ಕಾಲದಲ್ಲಿ ಈ ಆಹುತಿ ಮತ್ತು ಬಲಿಗಳ ಇರುವುದನ್ನೇ ನಾವು ನೋಡುತ್ತೇವೆ. ಅನಂತರ ತತ್ತ್ವಜ್ಞರು ಬಂದರು. ಆಧುನಿಕ ಸಮಾಜ ಸುಧಾರಕ ರಂತೆ ನಿಷೇಧ ಮಾರ್ಗವನ್ನು ಅವಲಂಬಿಸದೆ, ಮೂಢಮತಿಗಳ ಕೈಯಿಂದ, ಅವರು ಪೂಜಿಸುತ್ತಿದ್ದ ಚಿಹ್ನೆಗಳನ್ನು ಸುಮ್ಮನೆ ಕಿತ್ತುಕೊಳ್ಳುವುದರ ಬದಲು, ಅವುಗಳ ಸ್ಥಾನವನ್ನು ಆಕ್ರಮಿಸಲು ಉತ್ತಮವಾದ ಆದರ್ಶವನ್ನು ಅವರು ಕೊಟ್ಟರು. “ಇಲ್ಲಿ ಅಗ್ನಿಯ ಚಿಹ್ನೆಯೊಂದು ಇದೆ. ಬಹಳ ಒಳ್ಳೆಯದು. ಆದರೆ ಇಲ್ಲಿ ಪೃಥ್ವಿಯ ಮತ್ತೊಂದು ಚಿಹ್ನೆ ಇದೆ. ಎಷ್ಟು ಮಹೋತ್ತುಂಗವಾದ ಸುಂದರ ಚಿಹ್ನೆ ಇದು! ಸಣ್ಣ ಗುಡಿಯೊಂದು ಇಲ್ಲಿದೆ. ಆದರೆ ಇಡೀ ವಿಶ್ವವೇ ಒಂದು ದೇವಸ್ಥಾನ. ಮನುಷ್ಯನು ತನಗೆ ಇಚ್ಛೆ ಬಂದೆಡೆಯಲ್ಲಿ ಅರಾಧಿಸಬಹುದು. ಜಗತ್ತಿನಲ್ಲಿ ಮಾನವನು ಬಿಡಿಸಿರುವ ಅನೇಕ ವಿಚಿತ್ರ ಚಿತ್ರಗಳಿವೆ, ಮತ್ತು ಯಜ್ಞವೇದಿಕೆಗಳಿವೆ. ಉಸಿರಾಡುವ ಸಚೇತನವಾದ ಮಾನವ ದೇಹವೇ ಸರ್ವೋತ್ಕೃಷ್ಟವಾದ ವೇದಿಕೆ. ಈ ವೇದಿಕೆಗೆ ಪೂಜೆ ಸಲ್ಲಿಸುವುದೇ ನಿರ್ಜೀವವಾದ ಚಿಹ್ನೆಗಳನ್ನು ಪೂಜಿಸುವುದಕ್ಕಿಂತ ಅನಂತಾನಂತ ಪಾಲು ಉತ್ತಮ.”

ಈಗ ನಾವೊಂದು ವಿಚಿತ್ರವಾದ ಸಿದ್ಧಾಂತಕ್ಕೆ ಬರುತ್ತೇವೆ. ಅದರಲ್ಲಿ ಬಹುಭಾಗ ನನಗೇ ಅರ್ಥವಾಗುವುದಿಲ್ಲ. ನಿಮಗೆ ಏನಾದರೂ ಸ್ವಲ್ಪ ಅರ್ಥವಾಗುವಂತೆ ಇದ್ದರೆ ಇದನ್ನು ಓದುತ್ತೇನೆ. ಧ್ಯಾನದಿಂದ ಯಾರು ಪವಿತ್ರರಾಗಿ ಜ್ಞಾನವನ್ನು ಗಳಿಸುವರೋ, ಅವರು ಸತ್ತಮೇಲೆ ಮೊದಲು ಬೆಳಕಿಗೆ ಹೋಗುತ್ತಾರೆ. ಅನಂತರ ಬೆಳಕಿನಿಂದ ಹಗಲಿಗೆ, ಹಗಲಿನಿಂದ ಶುಕ್ಲಪಕ್ಷಕ್ಕೆ, ಅಲ್ಲಿಂದ ಉತ್ತರಾಯಣಕ್ಕೆ, ಅಲ್ಲಿಂದ ವರುಷಕ್ಕೆ, ವರುಷದಿಂದ ಸೂರ್ಯನಿಗೆ, ಸೂರ್ಯನಿಂದ ಚಂದ್ರನಿಗೆ, ಚಂದ್ರನಿಂದ ಮಿಂಚಿಗೆ ಹೋಗುತ್ತಾರೆ. ಮಿಂಚಿನ ಲೋಕಕ್ಕೆ ಹೋದಾಗ ಒಬ್ಬ ಮಾನವನಲ್ಲದ ವ್ಯಕ್ತಿಯನ್ನು ಸಂದರ್ಶಿಸುವರು. ಆತನು ಇವರನ್ನು ಸಾಕ್ಷಾತ್​ ಬ್ರಹ್ಮನಲ್ಲಿಗೆ ಕೊಂಡೊಯ್ಯುವನು. ಇದು ದೇವಯಾನದ ಮಾರ್ಗ. ಋಷಿಗಳು ಮತ್ತು ಧರ್ಮಾತ್ಮರು ಕಾಲವಾದರೆ ಅವರು ಈ ಮಾರ್ಗದಲ್ಲಿ ಹೋಗುವರು. ಪುನಃ ಹಿಂತಿರುಗುವುದಿಲ್ಲ. ಈ ಮಾಸ ವರುಷ ಮುಂತಾದುವುಗಳೆಲ್ಲ ಏನೆಂದು ಯಾರಿಗೂ ಚೆನ್ನಾಗಿ ಅರ್ಥವಾಗುವಂತಿಲ್ಲ. ಪ್ರತಿಯೊಬ್ಬರೂ ತಮ್ಮದೇ ಆದ ಅರ್ಥವನ್ನು ಅದಕ್ಕೆ ಕೊಡುವರು. ಮತ್ತೆ ಕೆಲವರು ಅದಕ್ಕೇನೂ ಅರ್ಥವೇ ಇಲ್ಲ ಎನ್ನುವರು. ಚಂದ್ರಲೋಕಕ್ಕೆ ಮತ್ತು ಸೂರ್ಯಲೋಕಕ್ಕೆ ಹೋಗುವುದೆಂದರೇನು? ಮಿಂಚಿನ ಲೋಕಕ್ಕೆ ಹೋದ ಮೇಲೆ ಜೀವಾತ್ಮನಿಗೆ ಸಹಾಯಮಾಡಲು ಬರುವ ವ್ಯಕ್ತಿ ಎಂದರೆ ಏನೆಂದು ಯಾರಿಗೂ ಗೊತ್ತಿಲ್ಲ. ಚಂದ್ರಲೋಕವು ಜೀವಿಗಳಿರುವ ಸ್ಥಳ ಎಂಬ ಒಂದು ಅಭಿಪ್ರಾಯ ಹಿಂದೂಗಳಲ್ಲಿದೆ. ಅಲ್ಲಿಂದ ಜೀವ ಹೇಗೆ ಬಂತೆಂದು ನಾವು ನೋಡುತ್ತೇವೆ. ಯಾರಿಗೆ ಜ್ಞಾನ ಪ್ರಾಪ್ತಿಯಾಗಿಲ್ಲವೋ, ಆದರೆ ಯಾರು ಈ ಜೀವನದಲ್ಲಿ ಒಳ್ಳೆಯ ಕರ್ಮಗಳನ್ನು ಮಾಡಿರುವರೋ, ಅವರು ಕಾಲವಾದ ಮೇಲೆ ಮೊದಲು ಧೂಮಕ್ಕೆ ಹೋಗುತ್ತಾರೆ. ಅಲ್ಲಿಂದ ರಾತ್ರಿಗೆ, ಅಲ್ಲಿಂದ ಕೃಷ್ಣಪಕ್ಷಕ್ಕೆ ಅಲ್ಲಿಂದ ದಕ್ಷಿಣಾಯನಕ್ಕೆ, ಅಲ್ಲಿಂದ ಪಿತೃಲೋಕಕ್ಕೆ, ಅಲ್ಲಿಂದ ಆಕಾಶಕ್ಕೆ, ಅಲ್ಲಿಂದ ಚಂದ್ರಲೋಕಕ್ಕೆ ಹೋಗುತ್ತಾರೆ. ಅಲ್ಲಿ ದೇವತೆಗಳ ಆಹಾರವಾಗಿ, ಅನಂತರ ಅವರಲ್ಲಿ ಹುಟ್ಟಿ ತಮ್ಮ ಪುಣ್ಯ ಕರ್ಮ ಸವೆಯುವವರೆಗೂ ಅಲ್ಲಿರುತ್ತಾರೆ. ತಮ್ಮ ಪುಣ್ಯ ಕರ್ಮ ತೀರಿದ ಮೇಲೆ ಅದೇ ದಾರಿಯಿಂದ ಭೂಮಿಗೆ ಹಿಂತಿರುಗುವರು. ಮೊದಲು ಆಕಾಶ, ನಂತರ ಗಾಳಿ, ನಂತರ ಧೂಮ, ನಂತರ ಹಿಮ, ನಂತರ ಮೋಡವಾಗಿ ಮಳೆಹನಿಯಂತೆ ಭೂಮಿಗೆ ಬರುತ್ತಾರೆ. ಮನುಷ್ಯರು ತಿನ್ನುವ ಆಹಾರದಲ್ಲಿ ಪ್ರವೇಶಿಸಿ ನಂತರ ಕೊನೆಗೆ ಅವರ ಮಕ್ಕಳಾಗಿ ಹುಟ್ಟುವರು. ಯಾರು ಸುಕರ್ಮಿಗಳೊ, ಅವರು ಉತ್ತಮ ವಂಶದಲ್ಲಿ ಜನಿಸುವರು. ಯಾರದು ಕುಕರ್ಮವೊ ಅವರು ಹೀನವಂಶದಲ್ಲಿ, ಕೆಲವು ವೇಳೆ ಪ್ರಾಣಿಗಳ ಹೊಟ್ಟೆಯಲ್ಲಿ ಕೂಡ, ಜನಿಸುವರು. ಒಂದು ಸಮನಾಗಿ ಪ್ರಾಣಿಗಳು ಈ ಪ್ರಪಂಚಕ್ಕೆ ಬಂದು ಹೋಗುತ್ತಿವೆ. ಆದಕಾರಣವೆ ಇದು ಖಾಲಿಯಾಗಿಯೂ ಇಲ್ಲ ಭರ್ತಿ ಯಾಗಿಯೂ ಇಲ್ಲ.

ಇವುಗಳಿಂದಲೂ ಕೂಡ ನಮಗೆ ಅನೇಕ ವಿಷಯಗಳು ಗೊತ್ತಾಗುತ್ತವೆ. ಕೊನೆಗೆ ಇದು ಬಹುಶಃ ಚೆನ್ನಾಗಿ ತಿಳಿಯಬಹುದು. ಇವು ಏನೆಂದು ನಾವು ಸ್ವಲ್ಪ ಊಹಿಸ ಬಹುದು.ಕೊನೆಯದು, ಎಂದರೆ, ಸ್ವರ್ಗದಿಂದ ಈ ಪ್ರಪಂಚಕ್ಕೆ ಹೇಗೆ ಹಿಂತಿರುಗು ವರು ಎಂಬುದು ಮೊದಲನೆ ಭಾಗಕ್ಕಿಂತ ಸ್ವಲ್ಪ ಸ್ಪಷ್ಟವಾಗಿದೆ. ಆದರೆ ಇದರ ಒಟ್ಟು ಅಭಿಪ್ರಾಯ ಹೀಗೆಂದು ತೋರುತ್ತದೆ. ಏನೆಂದರೆ ಬ್ರಹ್ಮ ಸಾಕ್ಷಾತ್ಕಾರವಿಲ್ಲದೆ ನಿತ್ಯಸ್ವರ್ಗವಿಲ್ಲ ಎಂಬುದು. ಯಾರು ಬ್ರಹ್ಮನನ್ನು ಸಾಕ್ಷಾತ್ಕಾರ ಮಾಡಿ ಕೊಂಡಿಲ್ಲವೋ, ಆದರೆ (ಫಲಾಪೇಕ್ಷೆಯಿಂದ) ಯಾರು ಒಳ್ಳೆಯ ಕರ್ಮಗಳನ್ನು ಮಾಡುತ್ತಾರೋ ಅವರು ಸತ್ತಮೇಲೆ ಯಾವುದಾದರೂ ಮಾರ್ಗವಾಗಿ ಸ್ವರ್ಗಕ್ಕೆ ಹೋಗಿ, ನಾವು ಇಲ್ಲಿ ಹುಟ್ಟುವಂತೆ ದೇವತೆಗಳಿಗೆ ಮಕ್ಕಳಾಗಿ ಹುಟ್ಟಿ, ತಮ್ಮ ಪುಣ್ಯಕರ್ಮಗಳ ಫಲ ಸವೆಯುವವರೆಗೆ ಅಲ್ಲಿರುತ್ತಾರೆ. ಇವುಗಳಿಂದ ಬಹಳ ಮುಖ್ಯವಾದ ವೇದಾಂತ ಸಿದ್ಧಾಂತವೊಂದು ಮೂಡುತ್ತದೆ: ನಾಮರೂಪಗಳುಳ್ಳ ಪ್ರತಿಯೊಂದೂ ನಶ್ವರ; ಈ ಜಗತ್ತೂ ಕ್ಷಣಿಕ. ಏಕೆಂದರೆ ಇಲ್ಲಿ ನಾಮರೂಪಗಳಿವೆ. ಇದರಂತೆಯೆ, ಸ್ವರ್ಗವು ಕೂಡಾ ಕ್ಷಣಿಕವಾಗಿರಬೇಕು, ಏಕೆಂದರೆ ಅಲ್ಲಿಯೂ ಕೂಡ ನಾಮರೂಪಗಳಿವೆ. ನಿತ್ಯಸ್ವರ್ಗ ಎಂಬುದು ಅಸಂಬದ್ಧವಾದುದು. ಏಕೆಂದರೆ ನಾಮರೂಪಗಳುಳ್ಳ ಪ್ರತಿಯೊಂದು ವಸ್ತುವೂ ಕಾಲದಲ್ಲಿ ಹುಟ್ಟಿ, ಕಾಲದಲ್ಲಿದ್ದು, ಕಾಲದಲ್ಲಿ ಲಯವಾಗಬೇಕು. ಇದು ವೇದಾಂತದ ಸುನಿಶ್ಚಿತಾರ್ಥವಾದ ಸಿದ್ಧಾಂತ. ಅದಕ್ಕೋಸ್ಕರವೆ ಸ್ವರ್ಗವನ್ನು ತ್ಯಜಿಸುವರು.

ಸಂಹಿತೆಗಳಲ್ಲಿ ಬರುವ ಸ್ವರ್ಗಲೋಕವೆಂಬ ಭಾವನೆ ಕ್ರೈಸ್ತರಲ್ಲಿ ಮತ್ತು ಮಹಮ್ಮದೀಯರಲ್ಲಿ ರೂಢಿಯಾಗಿರುವಂತೆ ಅನಂತವಾದದ್ದು. ಮಹಮ್ಮದೀಯರು ಇದನ್ನು ಮತ್ತಷ್ಟು ಸ್ಥೂಲವಾಗಿ ಮಾಡಿರುವರು. ಸುಂದರವಾದ ಉದ್ಯಾನವನಗಳು ಅಲ್ಲಿವೆಯೆಂದೂ, ನದಿಗಳು ಹರಿಯುತ್ತಿವೆಯೆಂದೂ ಹೇಳುತ್ತಾರೆ. ಅರೇಬಿಯದ ಮರಳು ಕಾಡಿನವರು ನೀರನ್ನು ಬಯಸುವರು. ಆದಕಾರಣವೆ ಮಹಮ್ಮದೀಯರು ಸ್ವರ್ಗದಲ್ಲಿ ಬೇಕಾದಷ್ಟು ನೀರಿರುವುದೆಂದು ಕಲ್ಪಿಸಿಕೊಳ್ಳುವರು. ಪ್ರತಿ ವರುಷವೂ ಆರು ತಿಂಗಳು ಮಳೆ ಇರುವ ದೇಶದಲ್ಲಿ ನಾನು ಹುಟ್ಟಿದವನು. ಸ್ವರ್ಗವನ್ನು ಅಷ್ಟೇನೂ ಮಳೆಯಿಲ್ಲದ ದೇಶವಾಗಿ ನಾನು ಭಾವಿಸಬೇಕಾಗುವುದು. ಅದರಂತೆಯೇ ಇಂಗ್ಲೀಷಿನವರೂ ಕೂಡ. ಸಂಹಿತೆಯಲ್ಲಿ ಬರುವ ಸ್ವರ್ಗಲೋಕ ಅನಂತವಾದುದು. ಇಲ್ಲಿಂದ ಅಲ್ಲಿಗೆ ಹೋದವರಿಗೆ ಸುಂದರವಾದ ದೇಹಗಳಿರುವುವು. ತಮ್ಮ ಪಿತೃಗಳೊಂದಿಗೆ ಅವರು ಅಲ್ಲಿ ವಾಸಿಸುವರು. ಅನಂತರ ಅವರು ಯಾವಾಗಲೂ ಸುಖವಾಗಿರುವರು. ಅಲ್ಲಿ ಅವರು ತಮ್ಮ ತಂದೆ, ತಾಯಿ, ಮಕ್ಕಳು ಮತ್ತು ಇತರ ನಂಟರನ್ನು ನೋಡುವರು. ಇಲ್ಲಿನಂತೆ ಅಲ್ಲಿಯೂ ಕೂಡ ಜೀವನ ನಡೆಸುವರು. ಆದರೆ ಇಲ್ಲಿಗಿಂತ ಹೆಚ್ಚು ಸೌಖ್ಯವೇನೋ ಅಲ್ಲಿರುವುದು, ಅಷ್ಟೆ. ಈ ಜೀವನದಲ್ಲಿ ಸುಖಕ್ಕೆ ಇರುವ ಆತಂಕಗಳು ಮತ್ತು ಕಷ್ಟಗಳು ಮಾಯವಾಗಿ ಕೇವಲ ಸುಖವು ಮಾತ್ರ ಉಳಿದಿರುತ್ತದೆ. ಈ ಸ್ಥಿತಿ ಸುಖವೆಂದು ಮಾನವನು ಎಷ್ಟೇ ಭಾವಿಸಿದರೂ ಸತ್ಯವೇ ಬೇರೆ ಮತ್ತು ಸೌಖ್ಯವೇ ಬೇರೆ. ಸತ್ಯದ ಪರಮಾವಧಿಯನ್ನು ನಾವು ಮುಟ್ಟುವವರೆವಿಗೂ ಸತ್ಯವು ಯಾವಾಗಲೂ ಸುಖಕರವಾಗಿರುವುದಿಲ್ಲ. ಯಾವಾಗಲೂ ಬದಲಾವಣೆ ಬೇಡವೆನ್ನುವುದು ಮಾನವನ ಸ್ವಭಾವ. ಅದು ಏನಾದರೊಂದನ್ನು ಮಾಡುವುದು. ಅದನ್ನು ಮಾಡಿದ ಮೇಲೆ ಅದರಿಂದ ತಪ್ಪಿಸಿಕೊಳ್ಳುವುದಕ್ಕೆ ಕಷ್ಟವಾಗುತ್ತದೆ. ಮನಸ್ಸು ಹೊಸ ವಿಚಾರಗಳನ್ನು ಸ್ವೀಕರಿಸುವುದಿಲ್ಲ, ಏಕೆಂದರೆ ಅವುಗಳಿಂದ ತೊಂದರೆ ಹೆಚ್ಚು.

ಉಪನಿಷತ್ತುಗಳಲ್ಲಿ ಇದರ ದಾರಿಯೇ ಬೇರೆಯಾಗಿರುವುದು ನಮಗೆ ಕಾಣುತ್ತದೆ. ಮಾನವನು ಕಾಲಾನಂತರ ನೆಂಟರಿಷ್ಟರೊಡನೆ ವಾಸಿಸುವ ಸ್ವರ್ಗ ನಿತ್ಯವಾಗದು. ಏಕೆಂದರೆ ನಾಮರೂಪಗಳುಳ್ಳ ಪ್ರತಿಯೊಂದು ವಸ್ತುವೂ ನಾಶವಾಗಲೇಬೇಕು. ನಾಮರೂಪಗಳಿಂದ ಕೂಡಿದ ಸ್ವರ್ಗವಿದ್ದರೆ ಕಾಲಾನಂತರದಲ್ಲಿ ಅದು ನಾಶವಾಗಲೇ ಬೇಕು. ಕೋಟ್ಯಂತರ ವರುಷಗಳು ಅದು ಇರಬಹುದು. ಆದರೆ ಅದು ನಾಶವಾಗುವ ಸಮಯ ಒಂದು ಬಂದೇ ಬರುವುದು. ಈ ಭಾವನೆಯೊಂದಿಗೆ ಮತ್ತೊಂದು ಬಂದಿತು. ಜೀವಿಗಳು ಪ್ರಪಂಚಕ್ಕೆ ಪುನಃ ಹಿಂತಿರುಗಬೇಕು. ತಮ್ಮ ಪುಣ್ಯ ಕರ್ಮಫಲವನ್ನು ಅನುಭವಿಸುವ ಲೋಕ ಸ್ವರ್ಗ ಇದರ ಫಲ ತೀರಿದ ಮೇಲೆ ಪ್ರಪಂಚದ ಬಾಳಿಗೆ ಮತ್ತೊಮ್ಮೆ ಹಿಂತಿರುಗಬೇಕು. ಇವುಗಳಿಂದ ಒಂದು ವಿಷಯ ಸ್ಪಷ್ಟವಾಗುತ್ತದೆ. ಬಹಳ ಹಿಂದಿನ ಕಾಲದಲ್ಲೆ ಮಾನವನಿಗೆ ಕಾರ್ಯಕಾರಣ ಸಂಬಂಧದ ತತ್ತ್ವ ವೇದ್ಯ ವಾಗಿತ್ತು. ಕಾಲಾನಂತರದಲ್ಲಿ ನಮ್ಮ ತತ್ತ್ವಜ್ಞರು ಅದನ್ನು ತತ್ತ್ವ ಮತ್ತು ತಾರ್ಕಿಕ ವೇಷದಲ್ಲಿ ತರುವುದನ್ನು ನೋಡುವೆವು. ಆದರೆ ಇಲ್ಲದರೊ ಅದು ಬಾಲಭಾಷೆ ಯಲ್ಲಿದೆ. ಈ ಪುಸ್ತಕಗಳನ್ನು ಓದುವುದರಿಂದ ನೀವು ಇನ್ನೊಂದು ವಿಚಾರವನ್ನು ಗಮನಿಸಬಹುದು. ಇವೆಲ್ಲಾ ಅಂತರಂಗವೇದ್ಯವಾದುವು.ಇವು ಕಾರ್ಯಕಾರಿಯಾಗ ಬಹುದೆ ಎಂದು ನನ್ನನ್ನು ನೀವು ಕೇಳಿದರೆ, ಇವು ಮೊದಲು ಕಾರ್ಯಕಾರಿಯಾಯಿತು, ಅನಂತರ ತತ್ತ್ವವಾಯಿತು ಎನ್ನುವುದೇ ನನ್ನ ಉತ್ತರ. ಮೊದಲು ಇವು ವೇದ್ಯವಾಗಿ ಸಾಕ್ಷಾತ್ಕಾರವಾದುವು, ಅನಂತರ ಇವನ್ನು ಬರವಣಿಗೆಯಲ್ಲಿ ಇಟ್ಟರೆಂಬುದನ್ನು ನೀವೇ ನೋಡಬಹುದು. ಪೂರ್ವಕಾಲದ ಜಿಜ್ಞಾಸುಗಳೊಂದಿಗೆ ಪ್ರಪಂಚ ಮಾತನಾಡಿತು, ಹಕ್ಕಿಗಳು ಮಾತನಾಡಿದವು, ಮೃಗಗಳು ಮಾತನಾಡಿದವು, ಸೂರ್ಯಚಂದ್ರರು ಮಾತನಾಡಿದರು. ಕ್ರಮೇಣ ಸ್ವಲ್ಪವಾಗಿ ವಿಷಯವನ್ನು ತಿಳಿದು ಪ್ರಕೃತಿಯ ಅಂತ ರಾಳಕ್ಕೆ ಪ್ರವೇಶಿಸಿದರು. ಆಲೋಚನೆಯಿಂದ ಅಲ್ಲ, ತರ್ಕ ಪಾಂಡಿತ್ಯಗಳಿಂದ ಅಲ್ಲ, ಇಂದಿನ ಕಾಲದಲ್ಲಿ ರೂಢಿಯಾಗಿರುವ, ಇನ್ನೊಬ್ಬನಿಂದ ವಿಚಾರಗಳನ್ನು ಕದ್ದು ಒಂದು ದೊಡ್ಡ ಪುಸ್ತಕವನ್ನು ಬರೆಯುವುದರಿಂದ ಅಲ್ಲ. ಅಂತಹ ಋಷಿಗಳ ಒಂದು ನುಡಿಯನ್ನು ತೆಗೆದುಕೊಂಡು ಒಂದು ದೀರ್ಘ ಉಪನ್ಯಾಸವನ್ನು ಮಾಡುವ ನನ್ನಂತೆಯೂ ಅಲ್ಲ. ಸಾವಧಾನವಾದ ಪರೀಕ್ಷೆಯಿಂದಲೂ ಮತ್ತು ಅರಸುವಿಕೆ ಯಿಂದಲೂ ಅವರು ಸತ್ಯವನ್ನು ಕಂಡುಹಿಡಿದರು. ಅವರ ಮುಖ್ಯವಾದ ಮಾರ್ಗವೇ ಅಭ್ಯಾಸ. ಆದಕಾರಣ ಅದರಂತೆಯೇ ಯಾವಾಗಲೂ ಇರಬೇಕು. ಧರ್ಮವು ಯಾವಾಗಲೂ ಒಂದು ಕಾರ್ಯಕಾರಿಯಾದ ವಿಜ್ಞಾನ ಶಾಸ್ತ್ರ. ಬರಿಯ ಬಾಯಿಮಾತಿನ ಸಿದ್ಧಾಂತಗಳ ಧರ್ಮ ಹಿಂದೆ ಎಂದೂ ಇರಲಿಲ್ಲ, ಮುಂದೆ ಎಂದೂ ಇರಲಾರದು. ಮೊದಲು ಅನುಷ್ಠಾನ, ಅನಂತರ ಸಿದ್ಧಾಂತ. ಜೀವಿಗಳು ಸ್ವರ್ಗ ಲೋಕ ದಿಂದ ಹಿಂತಿರುಗಿ ಬರುತ್ತಾರೆ ಎಂಬ ಭಾವನೆ ಆಗಲೆ ಇದೆ. ಫಲಾಪೇಕ್ಷೆಯಿಂದ ಯಾರು ಸತ್ಯರ್ಮಗಳನ್ನು ಮಾಡುತ್ತಾರೊ ಅವರಿಗೆ ಫಲವೇನೋ ಸಿಕ್ಕುತ್ತದೆ, ಆದರೆ ಫಲವು ಶಾಶ್ವತವಲ್ಲ. ಕಾರಣದ ಪರಿಣಾಮಕ್ಕೆ ತಕ್ಕಂತೆ ಕಾರ್ಯ ಎಂಬ ಕಾರ್ಯಕಾರಣಗಳ ಸಂಬಂಧವನ್ನು ಚೆನ್ನಾಗಿ ಹೇಳಲಾಗಿದೆ.ಕಾರಣವಿದ್ದಂತೆ ಪರಿಣಾಮವೂ ಕೂಡ. ಕಾರಣವು ಮಿತಿಯಾದುದರಿಂದ ಫಲವೂ ಮಿತಿಯಿಂದಲೇ ಕೂಡಿರಬೇಕು. ಕಾರಣವು ಅನಂತವಾಗಿದ್ದರೆ ಕಾರ್ಯವೂ ಕೂಡ ಅನಂತವಾಗಬಲ್ಲದು. ಆದರೆ ಸತ್ಕರ್ಮ ಮುಂತಾದ ಕಾರಣಗಳೆಲ್ಲ ಅಂತವಾದುವು. ಆದಕಾರಣ ಇವು ಅನಂತ ಪರಿಣಾಮವನ್ನು ಉಂಟುಮಾಡಲಾರವು.

ಈಗ ನಾವು ಪ್ರಶ್ನೆಯ ಬೇರೊಂದು ಭಾಗಕ್ಕೆ ಬರುವೆವು.ಏಕೆ ಅನಂತ ಸ್ವರ್ಗವಿರಲಾರದೋ, ಅದೇ ಕಾರಣದಿಂದಲೇ ಅನಂತವಾದ ನರಕವೂ ಇರಲಾರದು. ನನ್ನ ಜೀವನದ ಪ್ರತಿಯೊಂದು ಕ್ಷಣದಲ್ಲಿಯೂ ಪಾಪಕರ್ಮಗಳನ್ನು ಮಾಡುವ ದುರಾತ್ಮನು ನಾನೆಂದು ಭಾವಿಸೋಣ. ನನ್ನ ಅನಂತ ಜೀವನಕ್ಕೆ ಹೋಲಿಸಿ ನೋಡಿದರೆ, ಇಂದಿನ ನನ್ನ ಕ್ಷಣಿಕ ಬಾಳು ಗಣನೆಗೆ ಬಾರದು. ಏನಾದರೊಂದು ಅನಂತ ನರಕದ ಶಿಕ್ಷೆ ಇದ್ದರೆ, ಅಂತವುಳ್ಳ ಕಾರಣಗಳಿಂದ ಅನಂತದ ಪರಿಣಾಮವಾಯಿ ತೆಂದು ಆಗುತ್ತದೆ. ಅದೆಂದಿಗೂ ಹಾಗೆ ಆಗಲಾರದು. ನನ್ನ ಜೀವನವೆಲ್ಲ ಒಳ್ಳೆಯ ದನ್ನು ಮಾಡುತ್ತಿದ್ದರೂ, ನನಗೆ ಅನಂತ ಸ್ವರ್ಗ ಲಭಿಸಲಾರದು. ಇದೂ ಕೂಡ ಹಿಂದಿನ ದೋಷವನ್ನೇ ಸೂಚಿಸುತ್ತದೆ. ಯಾರಿಗೆ ಸತ್ಯ ಸಾಕ್ಷಾತ್ಕಾರವಾಗಿದೆಯೋ, ಯಾರಿಗೆ ಅದು ತಿಳಿದಿದೆಯೋ ಅವರಿಗೆ ಅನ್ವಯಿಸುವ ಮೂರನೆಯ ದಾರಿ ಇದೆ. ಈ ಮಾಯಾ ಮೋಹದಿಂದ ತಪ್ಪಿಸಿಕೊಳ್ಳುವುದಕ್ಕೆ ಇದೊಂದೇ ಹಾದಿ. ಸತ್ಯಾನುಭವ ವೆಂದರೇನೆಂಬುದನ್ನು ಉಪನಿಷತ್ತು ಬೋಧಿಸುವುದು.

ಅಂದರೆ ಒಳಿತು ಕೆಡುಕುಗಳು ಯಾವುದನ್ನೂ ನಾವು ಗುರುತಿಸಕೂಡದು. ಸರ್ವವೂ ಆತ್ಮನಿಂದ ಬರುವುದರಿಂದ, ಆತ್ಮನು ಪ್ರತಿಯೊಂದರಲ್ಲಿಯೂ ಇರುವನು ಎಂದು ತಿಳಿಯಬೇಕು. ಪ್ರಪಂಚವು ಇಲ್ಲವೆಂದು ಸಾಧಿಸಬೇಕು. ಪ್ರಪಂಚವು ನಮ್ಮ ಪಾಲಿಗೆ ಇಲ್ಲದಾಂತಾಗಬೇಕು. ಸ್ವರ್ಗದಲ್ಲಿ ಮತ್ತು ನರಕದಲ್ಲಿ, ಮರಣ ದಲ್ಲಿ ಮತ್ತು ಜೀವನದಲ್ಲಿ ದೇವರನ್ನು ನೋಡಬೇಕು. ನಾನು ನಿಮಗೆ ಓದಿದ ಶ್ರುತಿವಾಕ್ಯದ ಅರ್ಥಸರಣಿ ಇದು ಪೃಥ್ವಿಯು ದೇವರ ಒಂದು ಚಿಹ್ನೆ, ಆಕಾಶವು ದೇವರು, ನಾವಿರುವ ಸ್ಥಳವು ದೇವರು, ಎಲ್ಲವೂ ಬ್ರಹ್ಮಮಯವಾಗಿದೆ. ಇದನ್ನು ನಾವು ನೋಡಬೇಕು, ಇದನ್ನು ನಾವು ಸಾಕ್ಷಾತ್ಕಾರ ಮಾಡಿಕೊಳ್ಳಬೇಕು. ಸುಮ್ಮನೆ ಮಾತ ನಾಡುವುದಲ್ಲ ಅಥವಾ ಆಲೋಚನೆ ಮಾಡುವುದಲ್ಲ.ಇದರ ತಾರ್ಕಿಕ ನಿರ್ಣಯದ ಪರಿಣಾಮವೇನೆಂದರೆ, ಎಂದು ಆತ್ಮನಿಗೆ ಎಲ್ಲವೂ ಬ್ರಹ್ಮಮಯವೆಂದು ತಿಳಿಯುವುದೋ ಆಗ ಅದು ಸ್ವರ್ಗಕ್ಕೆ ಹೋಗಲಿ, ಅಥವಾ ಇನ್ನು ಎಲ್ಲಿಯಾದರೂ ಹೋಗಲಿ, ಪುನಃ ಅದು ಭೂಮಿಯಲ್ಲಿ ಹುಟ್ಟಲಿ ಅಥವಾ ಸ್ವರ್ಗದಲ್ಲಿ ಹುಟ್ಟಲಿ ಇವಾವುದನ್ನೂ ಗಣನೆಗೆ ತೆಗೆದುಕೊಳ್ಳುವುದಿಲ್ಲ. ಆತ್ಮನ ಪಾಲಿಗೆ ಇವುಗಳಾವುದಕ್ಕೂ ಅರ್ಥವಿಲ್ಲ. ಏಕೆಂದರೆ ಅದಕ್ಕೆ ಎಲ್ಲಾ ಸ್ಥಳವೂ ಸಮನಾಗಿ ಕಾಣುವುದು.ಪ್ರತಿ ಯೊಂದು ಸ್ಥಳವೂ ದೇವರಿರುವ ಗುಡಿಯಾಗಿದೆ. ಎಲ್ಲಾ ಪ್ರದೇಶವೂ ಪುಣ್ಯ ಸ್ಥಳವಾಗಿದೆ. ಸ್ವರ್ಗವಾಗಲಿ ನರಕವಾಗಲಿ ಅಥವಾ ಇನ್ನು ಯಾವ ಸ್ಥಳವೇ ಆಗಲಿ ಅದು ನೋಡುವುದು ಬ್ರಹ್ಮ ಒಂದನ್ನೆ. ಒಳಿತು ಕೆಡಕುಗಳಿಲ್ಲ; ಜನನ ಮರಣಗಳಿಲ್ಲ; ಅನಂತ ಬ್ರಹ್ಮವೊಂದೇ ಇರುವುದು.

ವೇದಾಂತದ ದೃಷ್ಟಿಯಿಂದ ಇಂತಹ ಒಂದು ಅನುಭವವನ್ನು ಪಡೆದ ಮೇಲೆ ಮಾನವನು ಮುಕ್ತನಾಗುವನು. ಈ ಪ್ರಪಂಚದಲ್ಲಿ ಜೀವಿಸಲು ಯೋಗ್ಯನಾದವನು ಅವನೊಬ್ಬನೇ. ಉಳಿದವರು ಅಲ್ಲ. ಯಾರು ಕೆಟ್ಟದನ್ನು ನೋಡುತ್ತಾರೊ ಅವರು ಪ್ರಪಂಚದಲ್ಲಿ ಹೇಗೆ ಬಾಳುವರು? ಅವರ ಬಾಳೊಂದು ದುಃಖದ ರಾಶಿ. ಯಾರು ಅಪಾಯವನ್ನು ನೋಡುತ್ತಾರೋ ಅವರ ಬಾಳೊಂದು ದುಃಖದ ರಾಶಿ. ಯಾರು ಸಾವನ್ನು ನೋಡುತ್ತಾರೋ ಅವರ ಜೀವನ ಒಂದು ದುಃಖದ ರಾಶಿ. ಯಾರಿಗೆ ಸತ್ಯಸಾಕ್ಷಾತ್ಕಾರವಾಗಿದೆಯೊ, ಯಾರಿಗೆ ಎಲ್ಲಾ ವ್ಯಕ್ತಿಗಳಲ್ಲಿಯೂ ಅದೇ ಸತ್ಯವು ಗೋಚರಿಸುವುದೋ ಅವನು ಮಾತ್ರ ಈ ಪ್ರಪಂಚದಲ್ಲಿ ಬಾಳಬಲ್ಲ. ಅವನು ಮಾತ್ರ “ನಾನು ಈ ಜೀವನದಲ್ಲಿ ಸುಖವಾಗಿದ್ದೇನೆ” ಎಂದು ಹೇಳಬಲ್ಲ. ಅದಿರಲಿ, ನರಕದ ಭಾವನೆ ವೇದದಲ್ಲೆಲ್ಲೂ ಬರುವುದಿಲ್ಲವೆಂದು ನಾನು ಹೇಳುತ್ತೇನೆ. ಅದು ಬಹಳ ಕಾಲವಾದ ಮೇಲೆ ಪುರಾಣಗಳಲ್ಲಿ ಬರುತ್ತದೆ. ವೇದಗಳ ದೃಷ್ಟಿಯಿಂದ ನಮ್ಮ ಶಿಕ್ಷೆಯ ಪರಮಾವಧಿ ಎಂದರೆ ಈ ಪ್ರಪಂಚಕ್ಕೆ ಮತ್ತೊಮ್ಮೆ ಬರುವುದು. ಮತ್ತೊಂದು ಅವಕಾಶ ಈ ಪ್ರಪಂಚದಲ್ಲಿ ನಮಗೆ ಸಿಕ್ಕುವುದು. ಮೊದಲಿನಿಂದಲೇ ಭಾವನೆಗಳು ಅವ್ಯಕ್ತದೆಡೆಗೆ ತಿರುಗುವುದನ್ನು ನೋಡಬಹುದು. ಶಿಕ್ಷೆ ಬಹುಮಾನಗಳೆಂಬ ಭಾವನೆಗಳು ಬಹಳ ಸ್ಥೂಲವಾದುವು. ಇವು ನಮ್ಮಂತೆ ಒಬ್ಬನನ್ನು ಪ್ರೀತಿಸಿ ಮತ್ತೊಬ್ಬನನ್ನು ದ್ವೇಷಿಸುವ ಮಾನವ ದೇವತೆಗಳಿಗೆ ಮಾತ್ರ ಸರಿಹೋಗುವುವು. ಅಂತಹ ದೇವರಿದ್ದರೆ ಮಾತ್ರ ಬಹುಮಾನ ಮತ್ತು ಶಿಕ್ಷೆಗೆ ಅವಕಾಶವಿದೆ. ಸಂಹಿತೆಯಲ್ಲಿ ಅಂತಹ ದೇವರಿದ್ದನು. ಅಲ್ಲಿ ಭಯವು ಆಗಲೇ ಪ್ರವೇಶಿಸುವುದನ್ನು ನೋಡುವೆವು. ನಾವು ಉಪನಿಷತ್ತಿಗೆ ಬಂದೊಡನೆಯೇ ಅದು ಮಾಯವಾಗುವುದು. ನಿರಾಕಾರವು ಅದರ ಸ್ಥಾನವನ್ನು ಆಕ್ರಮಿಸುತ್ತದೆ. ನಿರಾಕಾರ ಭಾವವನ್ನು ತಿಳಿದುಕೊಳ್ಳುವುದು ಸಾಧಾರಣವಾಗಿ ಮಾನವನಿಗೆ ಕಷ್ಟ. ಏಕೆಂದರೆ ಅವನು ಯಾವಾಗಲೂ ಆಕಾರವನ್ನೇ ನೆಚ್ಚಿಕೊಂಡಿರುವನು. ಬಹಳ ಆಲೋಚನಾಪರರು ಎಂದು ಹೆಸರಾದವರು ಕೂಡ ದೇವರ ಅವ್ಯಕ್ತ ಭಾವನೆಗೆ ಅಂಜುವರು. ಆದರೆ ನನಗೇನೋ ದೇವರನ್ನು ಮಾನವನಂತೆ ಭಾವಿಸುವುದು ಅರ್ಥವಿಲ್ಲದಂತೆ ತೋರುವುದು. ಬದುಕಿರುವ ದೇವರೋ ಅಥವಾ ಸತ್ತ ದೇವರೋ, ಯಾವುದು ಉತ್ತಮವಾದ ಆದರ್ಶ? ಯಾರೂ ನೋಡದ ಯಾರೂ ತಿಳಿಯದ ದೇವರೋ ಅಥವಾ ತಿಳಿದ ದೇವರೋ?

ಈ ನಿರ್ಗುಣ ದೇವರೆ ಸಚೇತನ ದೇವರು. ಅದು ಒಂದು ತತ್ತ್ವ. ಸುಗುಣ ಮತ್ತು ನಿರ್ಗುಣ ದೇವರಿಗೆ ಇರುವ ವ್ಯತ್ಯಾಸ ಇದು–ಸುಗುಣ ದೇವರು ಮನುಷ್ಯ ಮಾತ್ರ, ನಿರ್ಗುಣ ದೇವನಾದರೋ, ಅವನೇ ಇಂದ್ರಾದಿ ದೇವತೆಗಳು, ಅವನೇ ಮಾನವನು, ಅವನೇ ಪ್ರಾಣಿ. ಅದು ಮಾತ್ರವಲ್ಲದೆ ನಮಗೆ ಕಾಣದ ಇನ್ನು ಏನೇನೋ ಆಗಿರುವನು. ಏಕೆಂದರೆ ನಿರ್ಗುಣದಲ್ಲಿ ಸಾಕಾರಗಳೆಲ್ಲ ಸೇರಿವೆ. ಸೃಷ್ಟಿಯಲ್ಲಿರುವ ಸಕಲ ವಸ್ತುಗಳ ಮೊತ್ತ ಅವನು. ಇದು ಮಾತ್ರವಲ್ಲ, ಇದನ್ನು ಕೂಡ ಮೀರಿರುವನು. “ಹೇಗೆ ಅಗ್ನಿಯೊಂದು ಜಗತ್ತಿಗೆ ಬಂದು ನಾನಾ ರೂಪಗಳಲ್ಲಿ ಪ್ರಕಾಶಿಸುತ್ತಿದ್ದರೂ, ಅದು ಹೇಗೆ ಇವುಗಳೆಲ್ಲವನ್ನು ಅಧಿಕ ಪಾಲು ಮೀರಿದೆಯೊ” ಅದರಂತೆಯೇ ನಿರ್ಗುಣವೂ ಕೂಡ.

ನಾವು ಒಬ್ಬ ಜೀವಂತ ದೇವರನ್ನು ಪೂಜಿಸಬೇಕು. ನನ್ನ ಜೀವನದಲ್ಲೆಲ್ಲಾ ನಾನು ದೇವರಲ್ಲದೆ ಬೇರೆ ಯಾವುದನ್ನೂ ನೋಡಿಲ್ಲ. ನೀವೂ ಕೂಡ ಅದರಂತೆಯೆ. ನೀವು ಈ ಕುರ್ಚಿಯನ್ನು ನೋಡಬೇಕಾದರೆ ಮೊದಲು ದೇವರನ್ನು ನೋಡಬೇಕು. ಅನಂತರ ಅವನಿಂದ, ಅವನ ಮೂಲಕ ಮಾತ್ರ ನೀವು ಕುರ್ಚಿಯನ್ನು ನೋಡ ಬಹುದು. “ನಾನಿದ್ದೇನೆ” ಎಂದು ಸಾರುತ್ತಾ ಅವನು ಎಲ್ಲಾ ಕಡೆಯಲ್ಲಿಯೂ ಇರುವನು. “ನಾನಿದ್ದೇನೆ” ಎಂದು ನೀವು ಹೇಳಿದ ತಕ್ಷಣವೇ ನಿಮಗೆ ಅಸ್ತಿತ್ವದ ಪ್ರಜ್ಞೆ ಇದೆ. ನಮ್ಮ ಹೃದಯ ಕಮಲದಲ್ಲಿ ಮತ್ತು ಪ್ರತಿಯೊಂದು ಸಚೇತನವಾದ ಪ್ರಾಣಿಯ ಅಂತರಾಳದಲ್ಲಿ ಅದನ್ನು ನೋಡದೆ ಮತ್ತೆಲ್ಲಿಗೆ ಅವನನ್ನು ನೋಡುವುದಕ್ಕೆ ಹೋಗು ವುದು? “ಪುರುಷನೆ ನೀನು, ಸ್ತ್ರೀಯು ನೀನೆ, ಕುಮಾರಿಯೂ ನೀನೆ, ಕುಮಾರನು ನೀನೆ. ಕೋಲೂರಿಕೊಂಡು ನಡುಗುತ್ತ ನಡೆಯುವ ಮುದುಕನು ನೀನೆ, ನವಶಕ್ತಿ ಉತ್ಸಾಹದಿಂದ ಧೈರ್ಯವಾಗಿ ಚಲಿಸುವ ಯುವಕನು ನೀನೆ.” ಇರುವ ಸಕಲವೂ ನೀನೆ. ಪ್ರಪಂಚದಲ್ಲಿ ಸತ್ಯವಾಗಿರುವ ಏಕಮಾತ್ರ ಸಚೇತನವಾದ ಅದ್ಭುತವಾದ ಬ್ರಹ್ಮನು ನೀನೇ. ಎಲ್ಲಿಯೋ ಯಾರೂ ನೋಡದ, ತೆರೆಯ ಹಿಂದೆ ಇರುವ, ಸಂಪ್ರದಾಯಶೀಲರ ದೇವರ ಭಾವಕ್ಕೆ ಇದು ವಿರೋಧವಾಗಿ ತೋರುತ್ತದೆ. ಅವರು ಹೇಳಿದಂತೆ ನಡೆದರೆ, ಅವರ ನಿಂದೆಯನ್ನೆಲ್ಲಾ ಸಾವಧಾನವಾಗಿ ಕೇಳಿದರೆ, ನಾವು ಅವರು ತೋರುವ ದಾರಿಯಲ್ಲಿ ನಡೆದರೆ ಪೂಜಾರಿಗಳು ಮಾತ್ರ ಭರವಸೆ ಕೊಡುತ್ತಾರೆ, ನಾವು ಕಾಲವಾದ ಮೇಲೆ, ದೇವರನ್ನು ಪ್ರತ್ಯಕ್ಷವಾಗಿ ಕಾಣುವಂತೆ ಮಾಡುವ ಒಂದು ಅಪ್ಪಣೆ ಚೀಟಿಯನ್ನು ಕೊಡುತ್ತಾರೆ! ಈ ಸ್ವರ್ಗದ ಕಟ್ಟುಕಥೆಗಳೆಲ್ಲ ಕೆಲಸಕ್ಕೆ ಬಾರದ ಪೂಜಾರಿಗಳ ಕಲ್ಪನೆಯಲ್ಲದೆ ಮತ್ತೇನು?

ನಿರ್ಗುಣ ಭಾವನೆ ಧ್ವಂಸಕಾರಿಯೇನೋ ನಿಜ. ಇದು, ಪೂಜಾರಿ ಚರ್ಚು ಗುಡಿ ಇವರ ಕೈಯಿಂದ ವ್ಯಾಪಾರವೆಲ್ಲವನ್ನು ಕಸಿದುಕೊಳ್ಳುವುದು. ಭರತಖಂಡದಲ್ಲಿ ಈಗ ಕ್ಷಾಮಗಾಲ. ಆದರೆ ಅಲ್ಲಿ ಕೆಲವು ಗುಡಿಗಳಲ್ಲಿ ಒಂದು ವೇಳೆ ಯಾರಾದರೂ ಒಬ್ಬ ರಾಜನನ್ನು ಸೆರೆಹಿಡಿದು, ಅಪಾರವಾದ ಐಶ್ವರ್ಯವನ್ನು ಕೊಟ್ಟರೆ ಮಾತ್ರ ಅವನನ್ನು ಬಿಡುತ್ತೇವೆ ಎಂದರೆ ಅಷ್ಟನ್ನು ತೆತ್ತು ಅವನನ್ನು ಬಿಡಿಸಿಕೊಂಡು ಹೋಗಬಹುದಾದಷ್ಟು ಬೆಲೆ ಬಾಳುವ ಆಭರಣಗಳು ಅಲ್ಲಿವೆ. ಜನರಿಗೆ ಈ ನಿರಾಕಾರ ಭಾವನೆಯನ್ನು ಪೂಜಾರಿಗಳು ಬೋಧಿಸಿದರೆ, ಅವರ ಕಸಬಿಗೇ ಸಂಚಕಾರ ಬರುವುದು. ಆದರೂ ಕೂಡ ನಾವು ಪೂಜಾರಿಗಳನ್ನು ಹೊರತುಪಡಿಸಿ, ನಿಃಸ್ವಾರ್ಥತೆಯಿಂದ ಈ ಭಾವನೆಗಳನ್ನು ಬೋಧಿಸಬೇಕು. ನೀವೂ ದೇವರು, ನಾನೂ ದೇವರು. ಯಾರು ಯಾರ ಅಣಿತಿಯನ್ನು ಪಾಲಿಸಬೇಕು, ಯಾರು ಯಾರನ್ನು ಪೂಜಿಸಬೇಕು? ದೇವರ ಅತ್ಯುತ್ತಮ ದೇಗುಲ ನೀವು; ಬೇರಾವ ಗುಡಿಯನ್ನೋ, ವಿಗ್ರಹವನ್ನೋ, ಗ್ರಂಥ ವನ್ನೋ ಪೂಜಿಸುವ ಬದಲು ನಿಮ್ಮನ್ನು ನಾನು ಪೂಜಿಸುತ್ತೇನೆ. ಕೆಲವರೇತಕ್ಕೆ ತಮ್ಮ ಆಲೋಚನೆಯಲ್ಲಿ ಇಷ್ಟು ವಿರೋಧಾಭಾಸದಿಂದ ಕೂಡಿಕೊಂಡಿರುವರು? ಅವರು ನಮ್ಮ ಕೈಗಳಿಂದ ನುಣಚಿಕೊಳ್ಳುವ ಮೀನಿನಂತೆ ಇರುವರು. ತಾವು ಸಂಪೂರ್ಣ ಕಾರ್ಯಕಾರಿ ದೃಷ್ಟಿಯವರೆಂದು ಹೇಳಿಕೊಳ್ಳುವರು. ಅದು ಒಳ್ಳೆಯದೆ. ಆದರೆ ಇಲ್ಲಿ ನಿಮ್ಮನ್ನು ಪೂಜಿಸುವುದಕ್ಕಿಂತ ಹೆಚ್ಚು ಕಾರ್ಯಕಾರಿ ಯಾವುದು? ನಾನು ನಿಮ್ಮನ್ನು ನೋಡುತ್ತೇನೆ, ನಾನು ನಿಮ್ಮನ್ನು ತಿಳಿಯುತ್ತೇನೆ ಮತ್ತು ನೀವು ದೇವರೆಂದು ನನಗೆ ಗೊತ್ತಿದೆ. ಮಹಮ್ಮದೀಯರು “ಅಲ್ಲ” ನಲ್ಲದೆ ಬೇರೆ ದೇವರಿಲ್ಲ ಎನ್ನುತ್ತಾರೆ. ವೇದಾಂತವಾದರೋ, ದೇವರಲ್ಲದ ವಸ್ತು ಮತ್ತಾವುದೂ ಇಲ್ಲ ಎನ್ನುತ್ತದೆ. ನಿಮ್ಮಲ್ಲಿ ಅನೇಕರನ್ನು ಇದು ಅಂಜಿಸಬಹುದು. ಆದರೆ ಕ್ರಮೇಣ ನಿಮಗೆ ಇದು ಗೊತ್ತಾಗು ತ್ತದೆ. ಜೀವಂತನಾಗಿರುವ ದೇವರು ನಿಮ್ಮಲ್ಲಿರುವನು. ಆದರೂ ಕೆಲಸಕ್ಕೆ ಬಾರದ ಕಟ್ಟುಕಥೆ ನಂಬಿ ಚರ್ಚು ಗುಡಿ ಇವನ್ನು ಕಟ್ಟುತ್ತಿರುವಿರಿ. ಮನುಜದೇಹದಲ್ಲಿರುವ ಜೀವಾತ್ಮವೊಂದೆ ನಾವು ಪೂಜಿಸಬೇಕಾದ ವಸ್ತು. ಎಲ್ಲ ಪ್ರಾಣಿಗಳ ದೇಹವೂ ಕೂಡ ದೇವರ ಗುಡಿಯೇನೊ ಹೌದು. ಆದರೆ ಮಾನವನದು ಅತ್ಯುತ್ತಮವಾದುದು. ಕಟ್ಟಡಗಳಲ್ಲಿ ತಾಜ್​ ಮಹಲ್​ ಇದ್ದಂತೆ. ಅದರಲ್ಲಿ ನನಗೆ ಪೂಜಿಸಲು ಆಗದೆ ಇದ್ದರೆ ಉಳಿದ ಎಂತಹ ದೇವಸ್ಥಾನಗಳಿಂದಲೂ ಪ್ರಯೋಜನವಿಲ್ಲ. ಪ್ರತಿಯೊಂದು ಮಾನವ ದೇಹದ ಗುಡಿಯಲ್ಲೂ ಕುಳಿತಿರುವ ದೇವರು ನನಗೆ ತಿಳಿದ ಒಡನೆ, ಪ್ರತಿಯೊಬ್ಬ ಮಾನವನೆದುರು ಭಕ್ತಿಯಿಂದ ನಮ್ರನಾಗಿ ನಿಂತುಕೊಂಡು, ಅವನಲ್ಲಿ ದೇವರನ್ನು ಕಂಡೊಡನೆಯೆ, ಬಂಧನದಿಂದ ಮುಕ್ತನಾಗುವೆನು. ನನ್ನನ್ನು ಬಂಧಿಸಿದುದೆಲ್ಲ ಮಾಯವಾಗುವುದು, ನಾನು ಮುಕ್ತಾತ್ಮನಾಗುವೆನು.

ಇದೊಂದೆ ಎಲ್ಲಾ ಪೂಜೆಗಿಂತಲೂ ಹೆಚ್ಚು ಕಾರ್ಯಕಾರಿಯಾದುದು. ಸಿದ್ಧಾಂತ ಮತ್ತು ಊಹೆಗೂ ಇದಕ್ಕೂ ಸಂಬಂಧವಿಲ್ಲ. ಆದರೂ ಇದು ಅನೇಕರನ್ನು ಅಂಜಿಸು ತ್ತದೆ. ಅವರು, ಇದು ಸರಿಯಲ್ಲವೆನ್ನುವರು. ಎಲ್ಲಿಯೋ ಸ್ವರ್ಗದಲ್ಲಿರುವ ದೇವರು ಯಾರಿಗೊ ತಾನು ದೇವರೆಂದು ಹೇಳಿದನು ಎಂಬ ತಮ್ಮ ಮುತ್ತಾತಂದಿರಿಂದ ಕೇಳಿದ ಪುರಾತನ ಭಾವನೆಗಳನ್ನು ಮೆಲುಕು ಹಾಕುತ್ತಾರೆ. ಅಂದಿನ ಕಾಲದಿಂದಲೂ ನಮಗೆ ಇರುವುದು ಬರಿಯ ಸಿದ್ಧಾಂತಗಳೆ. ಅವರ ದೃಷ್ಟಿಯಿಂದ ಇದು ಕಾರ್ಯಕಾರಿ, ನಮ್ಮ ಭಾವನೆಗಳು ಅಲ್ಲ! ಪ್ರತಿಯೊಬ್ಬನಿಗೂ ತನ್ನದೇ ಒಂದು ದಾರಿ ಇರಬೇಕೆಂದು ವೇದಾಂತ ಹೇಳುತ್ತದೆ. ಇದರಲ್ಲಿ ಸಂದೇಹವೇನೂ ಇಲ್ಲ. ಆದರೆ ದಾರಿಯೇ ಗುರಿಯಲ್ಲ. ಸ್ವರ್ಗದಲ್ಲಿ ಇರುವ ದೇವರನ್ನು ಪೂಜಿಸುವುದು ಮುಂತಾದುವು ಕೆಟ್ಟ ದ್ದೇನೂ ಅಲ್ಲ. ಅದೆಲ್ಲ ಸತ್ಯದೆಡೆಗೆ ಕೊಂಡೊಯ್ಯುವ ಮೆಟ್ಟಲು, ಅದೇ ಸತ್ಯವಲ್ಲ. ಅವು ಹಿತಕಾರಿ ಮತ್ತು ಸುಂದರವಾಗಿವೆ. ಕೆಲವು ಅದ್ಭುತ ಭಾವನೆಗಳೂ ಅದರಲ್ಲಿವೆ. ಆದರೆ ಪ್ರತಿಯೊಂದು ಕಡೆಯೂ ವೇದಾಂತ ಹೇಳುತ್ತದೆ: “ನನ್ನ ಸ್ನೇಹಿತನೆ, ಯಾರನ್ನು ನೀನು ತಿಳಿಯದವನೆಂದು ಪೂಜಿಸುತ್ತಿರುವೆಯೋ, ಅವನೇ ನಾನೆಂದು ನಾನು ಪೂಜಿಸುತ್ತೇನೆ. ಯಾರನ್ನು ನೀನು ತಿಳಿಯಲಾರದವನೆಂದು ತಿಳಿದು ಜಗದಲ್ಲಿ ಅರಸು ತ್ತಿರುವೆಯೋ, ಅವನು ಸದಾಕಾಲದಲ್ಲಿಯೂ ನಿನ್ನಲ್ಲಿರುವನು. ಅವನಿಂದ ಮಾತ್ರ ನೀನು ಬದುಕಿರುವೆ, ಆತನೇ ಪ್ರಪಂಚದ ನಿತ್ಯಸಾಕ್ಷಿ.” “ವೇದಗಳು ಯಾರನ್ನು ಪೂಜಿಸುತ್ತವೆಯೋ, ಇಲ್ಲಿ ಅದಕ್ಕಿಂತಲೂ ಹೆಚ್ಚಾಗಿ ನಿತ್ಯವಾದ ‘ಅಹಂ’ ಎಂಬಲ್ಲಿ ಯಾರು ಯಾವಾಗಲೂ ಇರುವನೋ, ಯಾರಿರುವುದರಿಂದ ಪ್ರಪಂಚ ಇರುವುದೋ, ಅವನೇ ಸೃಷ್ಟಿಯ ಬೆಳಕು ಮತ್ತು ಜೀವ. ‘ಅಹಂ’ ಎಂಬುದು ನಿನ್ನಲ್ಲಿರದೇ ಇದ್ದರೆ, ನೀನು ಸೂರ್ಯನನ್ನು ನೋಡುತ್ತಿರಲಿಲ್ಲ. ಎಲ್ಲವೂ ಕತ್ತಲೆಯ ಮೊತ್ತವಾಗುತ್ತಿತ್ತು. ಅವನು ಪ್ರಕಾಶಿಸುವುದರಿಂದ ನೀನು ಪ್ರಪಂಚವನ್ನು ನೋಡುವೆ.”

ಸಾಧಾರಣವಾಗಿ ಒಂದು ಪ್ರಶ್ನೆಯನ್ನು ಕೇಳುವರು. ಇದು ನಮ್ಮನ್ನು ಬೇಕಾದಷ್ಟು ಕಷ್ಟಕ್ಕೆ ಈಡುಮಾಡುವುದು. ನಮ್ಮಲ್ಲಿ ಪ್ರತಿಯೊಬ್ಬರೂ, ನಾನೇ ದೇವರು, ನಾನು ಏನು ಮಾಡುತ್ತೇನೆಯೋ, ಆಲೋಚಿಸುತ್ತೇನೆಯೋ ಅದೆಲ್ಲವೂ ಒಳ್ಳೆಯ ದಾಗಿರಬೇಕು; ಏಕೆಂದರೆ ದೇವರು ಎಂದಿಗೂ ಕೇಡನ್ನು ಮಾಡ ಬಯಸುವುದಿಲ್ಲ. ಎನ್ನಲು ಮೊದಲು ಮಾಡಬಹುದು. ಮೊದಲನೆಯದಾಗಿ ಈ ವಿಪರೀತ ತರ್ಕವನ್ನು ಸರಿಯೆಂದು ಒಪ್ಪಿಕೊಂಡರೂ, ಇತರ ಅಭಿಪ್ರಾಯಗಳಲ್ಲಿ ಅಪಾಯವಿಲ್ಲವೆಂದು ನೀವು ಸಕಾರಣವಾಗಿ ತೋರುವಿರೇನು? ತಮ್ಮಿಂದ ಬೇರೆಯಾದ ಸ್ವರ್ಗದಲ್ಲಿರುವ ದೇವರನ್ನು ಅವರು ಪೂಜಿಸುವರು, ಅವನಿಗೆ ಅಂಜುವರು. ಅವರು ಭಯದಿಂದ ನಡುಗುತ್ತಲೇ ಜನಿಸಿರುವರು, ಮತ್ತು ಜೀವಮಾನ ಪರ್ಯಂತ ಭಯದಿಂದ ನಡುಗುತ್ತಲೇ ಇರುವರು. ಇದರಿಂದ ಪ್ರಪಂಚ ಬಹಳ ಉತ್ತಮವಾಯಿತೇನು? ಜಗತ್ತಿನ ಮಹಾನ್​ ಕರ್ಮಪಟುಗಳು ಮತ್ತು ಮಹಾನ್​ ನೀತಿಶಾಲಿಗಳು ಸಗುಣ ದೇವರ ಉಪಾಸಕರೋ ಅಥವಾ ನಿರ್ಗುಣ ದೇವರ ಉಪಾಸಕರೋ? ನಿಜವಾಗಿಯೂ ನಿರ್ಗುಣ ದೇವರ ಉಪಾಸಕರು. ಅಂಜಿಕೆಯಿಂದ ನೀತಿ ಹೇಗೆ ಬೆಳೆಯುವುದು? ಎಂದಿಗೂ ಇಲ್ಲ. “ಎಲ್ಲಿ ಒಬ್ಬನು ಇನ್ನೊಬ್ಬನನ್ನು ನೋಡುತ್ತಾನೋ, ಎಲ್ಲಿ ಒಬ್ಬನು ಇನ್ನೊಬ್ಬನು ಹೇಳುವುದನ್ನು ಕೇಳುತ್ತಾನೋ ಅದೇ ಮಾಯೆ. ಎಲ್ಲಿ ಒಬ್ಬನು ಇನ್ನೊಬ್ಬನನ್ನು ನೋಡುವುದಿಲ್ಲವೋ, ಒಬ್ಬನು ಇನ್ನೊಬ್ಬನು ಹೇಳುವುದನ್ನು ಕೇಳುವುದಿಲ್ಲವೊ, ಎಲ್ಲಿ ಎಲ್ಲವೂ ಆತ್ಮಮಯವಾಗಿರುವುದೋ ಅಲ್ಲಿ ಯಾರು ಯಾರನ್ನು ನೋಡುವರು? ಯಾರು ಯಾರನ್ನು ಕಂಡುಹಿಡಿಯುವರು?” ಎಲ್ಲವೂ ಅವನೆ, ಮತ್ತು ಅದೇ ಕಾಲದಲ್ಲಿ ಎಲ್ಲವೂ ನಾನೆ. ಆತ್ಮವು ಪರಿಶುದ್ಧವಾಗಿದೆ. ಆಗ ಮಾತ್ರ ಪ್ರೇಮವೆಂದರೇನೆಂಬುದು ನಮಗೆ ಗೊತ್ತಾಗುವುದು. ಪ್ರೇಮವು ಎಂದಿಗೂ ಅಂಜಿಕೆ ಯಿಂದ ಉದಯಿಸುವುದಿಲ್ಲ. ಪ್ರೇಮಕ್ಕೆ ಮೂಲ ಸ್ವಾತಂತ್ರ್ಯ ನಿಜವಾಗಿಯೂ ನೀವು ಪ್ರಪಂಚವನ್ನು ಪ್ರೀತಿಸುವುದಕ್ಕೆ ಪ್ರಯತ್ನಿಸಿದಾಗ ಮಾತ್ರ, ಸೋದರತ್ವ ಮತ್ತು ಮಾನವತೆ ಎಂದರೇನೆಂದು ಗೊತ್ತಾಗುವುದು. ಅದಕ್ಕೆ ಮುಂಚೆ ಅಲ್ಲ.

ಆದಕಾರಣ ನಿರ್ಗುಣ ಭಾವನೆ ಪ್ರಪಂಚದಲ್ಲಿ ಅತ್ಯಂತ ಭೀಕರವಾದ ಪಾಪಕ್ಕೆ ದಾರಿಯಾಗುತ್ತದೆ ಎನ್ನುವುದು ಸರಿಯಲ್ಲ. ಮಿಕ್ಕ ಸಿದ್ಧಾಂತಗಳು ಪಾಪಕ್ಕೆ ದಾರಿಯಾಗಲಿಲ್ಲವೇನು? ಮತಭ್ರಾಂತಿಯಿಂದ ರಕ್ತದ ಕಾಲುವೆಯನ್ನು ಹರಿಸಿ ಮಾನವರನ್ನು ಒಬ್ಬರು ಇನ್ನೊಬ್ಬರನ್ನು ಚೂರು ಚೂರು ಮಾಡುವಂತೆ ಪ್ರೇರೇಪಿಸಲಿಲ್ಲವೇನು? “ನನ್ನ ದೇವರೆ ಸರ್ವೋತ್ತಮವಾದ ದೇವರು. ಇಬ್ಬರೂ ಹೋರಾಡಿ ಇದನ್ನು ನಿಷ್ಕರ್ಷಿಸೋಣ.” ಪ್ರಪಂಚದಲ್ಲೆಲ್ಲಾ ದ್ವೈತದ ಪರಿಣಾಮವೇ ಇದು. ಬನ್ನಿ ವಿಶಾಲವಾದ ಯಾವ ಆತಂಕವೂ ಇಲ್ಲದ ಹಗಲ ಬೆಳಕಿಗೆ, ಚಿಕ್ಕ ಚಿಕ್ಕ ಸಂದಿಗೊಂದಿಗಳಿಂದ ಹೊರಗೆ ಬನ್ನಿ. ಅನಂತಾತ್ಮವು ಹೇಗೆ ಸಣ್ಣ ಬಿಲದಲ್ಲಿ ಹುಟ್ಟಿ ಸಾಯುವುದಕ್ಕೆ ಒಪ್ಪುವುದು? ಬನ್ನಿ ಹೊರಗೆ ಬೆಳಕಿನ ಜಗತ್ತಿಗೆ. ಜಗತ್ತಿನಲ್ಲಿರುವ ಸಕಲವೂ ನಿಮ್ಮದು. ನಿಮ್ಮ ಬಾಹುಗಳನ್ನು ನೀಡಿ ಪ್ರೇಮದಿಂದ ಆಲಂಗಿಸಿ. ಹೀಗೆ ಮಾಡಬೇಕೆಂದು ನಿಮಗೆ ಎಂದಾದರೂ ಅನ್ನಿಸಿದ್ದರೆ ನಿಮಗೆ ದೇವರ ಅನುಭವ ದೊರೆತಂತೆ.

ಬುದ್ಧದೇವನು ಹೇಗೆ ಅದ್ಭುತವಾದ, ಅನಂತವಾದ ಪ್ರೇಮದ ಆಲೋಚನೆ ಯನ್ನು ನಾಲ್ಕು ದಿಕ್ಕುಗಳಿಗೆ, ಮೇಲೆ ಕೆಳಗೆ, ವಿಶ್ವವೆಲ್ಲವೂ ಇದರಿಂದ ತುಂಬಿ ತುಳು ಕಾಡುವಂತೆ ಕಳುಹಿಸಿದನು ಎಂಬುದನ್ನು ಅವನ ಉಪದೇಶದಲ್ಲಿ ನೀವು ಓದಿರುವಿರಿ. ಅಂತಹ ವಿಶ್ವಾನುಕಂಪ ನಿಮ್ಮ ಎದೆಯಲ್ಲಿದ್ದರೆ ಆಗ ನಿಜವಾದ ವ್ಯಕ್ತಿತ್ವ ನಿಮ್ಮಲ್ಲಿರುವುದು. ವಿಶ್ವವೇ ಒಂದು ವಿರಾಟ್​ ವ್ಯಕ್ತಿ. ಅಲ್ಪ ವಸ್ತುಗಳನ್ನು ತೊರೆಯಿರಿ. ಮಹತ್ತಿಗಾಗಿ ಅಲ್ಪವನ್ನು ತ್ಯಾಗಮಾಡಿ. ಕ್ಷುದ್ರ ಭೋಗಗಳನ್ನು ಸಚ್ಚಿದಾನಂದ ಕ್ಕೋಸುಗ ತೊರೆಯಿರಿ. ಇವೆಲ್ಲವೂ ನಿಮ್ಮದು. ನಿರಾಕಾರವು ಸಾಕಾರಗಳನ್ನು ಒಳಗೊಂಡಿದೆ. ಆದಕಾರಣ ದೇವರು ಏಕಕಾಲದಲ್ಲಿ ಸಾಕಾರ ಮತ್ತು ನಿರಾಕಾರ. ಅನಂತವಾದ ನಿರಾಕಾರವಾದ ಮಾನವನು ಆಕಾರದಂತೆ ವ್ಯಕ್ತಗೊಳ್ಳುತ್ತಿರುವನು. ಅನಂತವಾದ ನಾವು ಸಾಂತವಾಗಿ ಸಂಕೋಚವಾದಂತೆ ತೋರುವೆವು. ಅನಂತವೆ ನಿಜವಾದ ಸ್ವರೂಪವೆಂದು ವೇದಾಂತ ಸಾರುತ್ತದೆ. ಅದು ಎಂದಿಗೂ ಮಾಯವಾಗು ವುದಿಲ್ಲ. ಅದು ಎಂದೆಂದಿಗೂ ಚಿರಸ್ಥಾಯಿಯಾಗಿರುವುದು. ನಮ್ಮ ಕರ್ಮದಿಂದ ನಮಗೆ ನಾವೇ ಮಿತಿಯನ್ನು ಕಲ್ಪಿಸಿಕೊಳ್ಳುತ್ತಿರುವೆವು. ಆ ಕರ್ಮವೇ ನಮ್ಮ ಕೊರಳಿಗೆ ಕಟ್ಟಿದ ಸರಪಳಿಯಂತೆ ನಮ್ಮನ್ನು ಈ ಸ್ಥಿತಿಗೆ ತಂದಿರುವುದು. ಆ ಸರಪಳಿಯನ್ನು ಕಿತ್ತುಹಾಕಿ ಸ್ವತಂತ್ರರಾಗಿ, ನಿಯಮವೆಂಬುದನ್ನು ನಿಮ್ಮ ಕಾಲಿನಿಂದ ತುಳಿಯಿರಿ. ಮಾನವ ಪ್ರಕೃತಿಯಲ್ಲಿ ನಿಯಮವೆಂಬುದಿಲ್ಲ, ಹಣೆಯ ಬರಹವೆಂಬುದಿಲ್ಲ, ಅದೃಷ್ಟ ವೆಂಬುದಿಲ್ಲ. ಅನಂತದಲ್ಲಿ ನಿಯಮ ಹೇಗೆ ಬರುವುದು? ಸ್ವಾತಂತ್ರ್ಯವೇ ಅದರ ಪಲ್ಲವಿ. ಅದರ ಸ್ವಭಾವವೇ ಸ್ವಾತಂತ್ರ್ಯ. ಸ್ವಾತಂತ್ರ್ಯವೇ ಅದರ ಜನ್ಮಸಿದ್ಧ ಹಕ್ಕು. ಮೊದಲು ಸ್ವತಂತ್ರರಾಗಿ. ಅನಂತರ ನಿಮಗೆ ಬೇಕಾದಷ್ಟು ಆಕಾರಗಳನ್ನು ಇಟ್ಟುಕೊಳ್ಳಿ. ಅನಂತರ ರಂಗಭೂಮಿಗೆ ಬಂದು ಭಿಕ್ಷುಕ ಪಾತ್ರವನ್ನು ನಟಿಸುವ ಪಾತ್ರಧಾರಿಯಂತೆ ನಾವು ಆಡುತ್ತೇವೆ. ದಾರಿಯಲ್ಲಿ ನಡೆಯುವ ನಿಜವಾದ ಭಿಕ್ಷುಕನೊಂದಿಗೆ ಅವನನ್ನು ಹೋಲಿಸಿ ನೋಡಿ. ಎರಡು ಸಂದರ್ಭಗಳಲ್ಲೂ ಬಹುಶಃ ದೃಶ್ಯ ಒಂದೆ, ಮಾತು ಒಂದೆ ಆಗಿರುವುದು. ಆದರೂ ಎಷ್ಟು ವ್ಯತ್ಯಾಸ! ಒಬ್ಬನು ತನ್ನ ಭಿಕ್ಷುಕನ ಪಾತ್ರದಿಂದ ಆನಂದಿಸುವನು. ಮತ್ತೊಬ್ಬನು ಅದರಿಂದ ದುಃಖಿಸುವನು. ಈ ಅಂತರಕ್ಕೆ ಕಾರಣವೇನು? ಒಬ್ಬನು ಮುಕ್ತ, ಮತ್ತೊಬ್ಬ ಬದ್ಧ. ಪಾತ್ರಧಾರಿಗೆ ತನ್ನ ಭಿಕ್ಷಾವೃತ್ಥಿ ನಿಜವಲ್ಲವೆಂದೂ, ಕೇವಲ ಆಟಕ್ಕಾಗಿ ನಟಿಸಿರುವೆನೆಂದೂ ಗೊತ್ತಿದೆ. ಆದರೆ ನಿಜವಾದ ಭಿಕ್ಷುಕನಾದರೋ ಅದೇ ತನ್ನ ನಿಜಸ್ವಭಾವವೆಂದೂ, ತನಗೆ ಅದು ಬೇಕಾಗಲೀ, ಬೇಡವಾಗಲೀ, ಸಹಿಸಬೇಕೆಂದೂ ಗೊತ್ತಿದೆ. ಇದೇ ನಿಯಮ. ಎಲ್ಲಿಯವರೆಗೆ ನಮಗೆ ಸಹಜ ಸ್ಥಿತಿಯ ಜ್ಞಾನ ಇರುವುದಿಲ್ಲವೋ ಅಲ್ಲಿಯವರೆಗೂ ನಾವು ಭಿಕ್ಷುಕರು. ಪ್ರಕೃತಿಯ ಪ್ರತಿಯೊಂದು ಶಕ್ತಿಯೂ ನಮ್ಮನ್ನು ನಡುಗಿಸುತ್ತದೆ. ಪ್ರಕೃತಿಯಲ್ಲಿರುವ ಪ್ರತಿಯೊಂದೂ ನಮ್ಮನ್ನು ಗುಲಾಮರನ್ನಾಗಿ ಮಾಡುವುದು. ಸಹಾಯಕ್ಕೋಸುಗವಾಗಿ ಪ್ರಪಂಚದಲ್ಲೆಲ್ಲಾ ಮೊರೆ ಇಡುವೆವು. ಆದರೆ ಸಹಾಯ ನಮಗೆ ಎಂದಿಗೂ ಬರುವುದಿಲ್ಲ. ಕಾಲ್ಪನಿಕ ದೇವರನ್ನು ನಾವು ಬೇಡುವೆವು. ಆದರೂ ಸಹಾಯ ಬರುವುದಿಲ್ಲ. ಇಷ್ಟಾದರೂ, ಸಹಾಯ ಬರುತ್ತದೆ ಎಂದು ನಾವು ನೆಚ್ಚುತ್ತೇವೆ. ಹೀಗೆ ಅಳುವುದು, ಗೋಳಿಡುವುದು ಇದರಲ್ಲಿ ಒಂದು ಜೀವನ ವ್ಯರ್ಥವಾಯಿತು. ಇದೇ ಆಟ ಪುನಃ ನಡೆಯುತ್ತಿರುತ್ತದೆ.

ಸ್ವತಂತ್ರರಾಗಿ, ಯಾರಿಂದಲೂ ಏನನ್ನೂ ನಿರೀಕ್ಷಿಸಬೇಡಿ. ಇದುವರೆಗೆ ನಿಮಗೆ ಆಗಿರುವುದನ್ನು ಪರೀಕ್ಷೆಮಾಡಿ ನೋಡಿದರೆ, ಎಂದೆಂದಿಗೂ ಬಾರದ ಇನ್ನೊಬ್ಬರ ಸಹಾಯಕ್ಕಾಗಿ ನೀವು ನೆಚ್ಚಿ ಕುಳಿತುದು ನಿರರ್ಥಕವೆಂಬುದು ನಿಮಗೆ ಗೊತ್ತಾಗುವು ದರಲ್ಲಿ ನನಗೆ ಸಂದೇಹವಿಲ್ಲ. ಬಂದ ಸಹಾಯವೆಲ್ಲವೂ ಕೂಡ ನಿಮ್ಮಿಂದಲೇ. ನೀವು ಯಾವುದಕ್ಕೆ ಪ್ರಯತ್ನಪಟ್ಟಿರೋ ಅದರ ಪ್ರತಿಫಲ ಮಾತ್ರ ನಿಮಗೆ ಸಿಕ್ಕಿತು. ಆದರೂ ಕೂಡ ನೀವು ಯಾವಾಗಲೂ ಹೊರಗಿನ ಸಹಾಯಕ್ಕೆ ಕಾದು ಕುಳಿತುದು ಆಶ್ಚರ್ಯ. ಶ‍್ರೀಮಂತನ ಮನೆ ಅಂಗಳದಲ್ಲಿ ಜನರು ಯಾವಾಗಲೂ ಕಿಕ್ಕಿರಿದು ನೆರೆ ದಿರುವರು. ಆದರೆ ಪರೀಕ್ಷೆ ಮಾಡಿದರೆ ಯಾವಾಗಲೂ ಅದೇ ಜನರು ಇರುವುದಿಲ್ಲ. ಬಂದವರು ಶ‍್ರೀಮಂತನಿಂದ ತಮಗೆ ಏನಾದರೂ ಸಿಕ್ಕುವುದೆಂದು ಆಶಿಸುವರು. ಆದರೆ ಅವರಿಗೆ ಎಂದಿಗೂ ಸಿಕ್ಕುವುದಿಲ್ಲ. ಹಾಗೆಯೇ ಎಂದಿಗೂ ಸಿದ್ಧಿಸದ ಫಲವನ್ನು ಯಾವಾಗಲೂ ನೆಚ್ಚುವುದರಲ್ಲೆ ನಮ್ಮ ಜೀವನವೆಲ್ಲ ವ್ಯರ್ಥವಾಗಿದೆ. ನೆಚ್ಚಿಗೆಯನ್ನು ತ್ಯಜಿಸಿ ಎನ್ನುವುದು ವೇದಾಂತ. ನೀವು ಏತಕ್ಕೆ ಆಶಿಸುವುದು? ನಿಮಗೆ ಆಗಲೇ ಸರ್ವವಸ್ತುಗಳೂ ಸಿಕ್ಕಿವೆ. ಇಲ್ಲ, ನೀವೇ ಸರ್ವವಸ್ತುಗಳೂ ಆಗಿರುವಿರಿ. ನೀವು ಯಾವುದಕ್ಕೆ ನೆಚ್ಚಿಕೊಂಡಿರುವಿರಿ? ರಾಜನೊಬ್ಬನು ಹುಚ್ಚನಾಗಿ ದೇಶದಲ್ಲೆಲ್ಲಾ ರಾಜ ನನ್ನು ಹುಡುಕಾಡಲು ಅಲೆದಾಡಿದರೆ ಅವನಿಗೆ ಎಂದಿಗೂ ಸಿಕ್ಕುವುದಿಲ್ಲ. ಏಕೆಂದರೆ ಅವನೇ ರಾಜನಾಗಿರುವನು. ತನ್ನ ದೇಶದಲ್ಲಿ ಪ್ರತಿಯೊಂದು ಪಟ್ಟಣಕ್ಕೂ, ಗ್ರಾಮಕ್ಕೂ ಹೋಗಿ, ಪ್ರತಿಯೊಂದು ಮನೆಯಲ್ಲಿಯೂ ಅಳುತ್ತ ಹುಡುಕಬಹುದು. ಆದರೆ ಅವನಿಗೆ ಎಂದಿಗೂ ಸಿಕ್ಕುವುದಿಲ್ಲ, ಏಕೆಂದರೆ ಅವನೇ ರಾಜನಾಗಿರುವನು. ನಾವೇ ದೇವರೆಂದು ತಿಳಿದು ಈ ಹುಚ್ಚು ಹುಡುಕಾಟವನ್ನು ತೊರೆಯುವುದು ಮೇಲು. ನಾವೇ ದೇವರೆಂದು ತಿಳಿದರೆ ಸಂತೋಷಿಸುತ್ತೇವೆ, ತೃಪ್ತಿಪಡುತ್ತೇವೆ. ಈ ಹುಚ್ಚು ಹುಡುಕಾಟಗಳನ್ನೆಲ್ಲ ತೊರೆದು, ರಂಗಭೂಮಿಯ ಮೇಲೆ ನಟನು ಮಾಡುವಂತೆ, ಜಗತ್ತಿನಲ್ಲಿ ನಿಮ್ಮ ಪಾಲಿಗೆ ಬಂದ ಪಾತ್ರವನ್ನು ಅಭಿನಯಿಸಿ.

ಆಗ ಒಟ್ಟು ನೋಟವೇ ಬದಲಾಯಿಸುವುದು. ಜಗತ್ತು ಒಂದು ನಿತ್ಯ ಸೆರೆಮನೆಯಾಗದೆ ಆಟದ ಮೈದಾನವಾಗುವುದು. ಪರಸ್ಪರ ಪೈಪೋಟಿಯಿಲ್ಲದೆ ಅನವರತವೂ ವಸಂತಋತು, ಅರಳಿದ ಅಲರು, ಹಾರುವ ಬಣ್ಣ ಬಣ್ಣದ ಚಿಟ್ಟೆಗಳಿಂದ ತುಂಬಿದ ಆನಂದದ ತವರೂರಾಗುವುದು. ಮೊದಲು ಯಾವುದು ನರಕವಾಗಿತ್ತೋ ಆ ಜಗತ್ತೇ ಸ್ವರ್ಗವಾಗುವುದು. ಬದ್ಧ ಜೀವಿಗಳ ಕಣ್ಣಿಗೆ ಇದು ನರಕ ಯಾತನೆಯ ಮನೆ. ಆದರೆ ಮುಕ್ತ ಜೀವಿಗಳಿಗೆ ಕಾಣುವುದೇ ಬೇರೆ ಬಗೆ. ಈ ಜೀವನವೇ ವಿಶ್ವ ಜೀವನ. ಸ್ವರ್ಗ ಮುಂತಾದ ಲೋಕಗಳೆಲ್ಲ ಇಲ್ಲೇ ಇವೆ. ಮಾನವನನ್ನು ಹೋಲುವ ದೇವತೆಗಳೆಲ್ಲ ಇಲ್ಲಿರುವರು. ದೇವತೆಗಳು ತಮ್ಮಂತೆ ಮನುಷ್ಯನನ್ನು ಸೃಷ್ಟಿಸಲಿಲ್ಲ. ಆದರೆ ಮನುಷ್ಯನು ತನ್ನಂತೆ ದೇವತೆಗಳನ್ನು ಸೃಷ್ಟಿಸಿದನು. ಇಲ್ಲಿದೆ ದೇವತೆಗಳ ಮೂಲರೂಪ. ಇಂದ್ರನಿಲ್ಲಿರುವನು, ವರುಣನಿಲ್ಲಿರುವನು ಮತ್ತು ವಿಶ್ವದ ಉಳಿದ ದೇವತೆಗಳೆಲ್ಲರೂ ಇಲ್ಲಿರುವರು. ನಮ್ಮ ಅಲ್ಪ ಪ್ರತಿರೂಪಗಳನ್ನು ನಾವು ಕಲ್ಪಿಸಿ ಕೊಳ್ಳುತ್ತಿರುವೆವು. ಈ ದೇವತೆಗಳ ಮೂಲರೂಪ ನಾವೆ. ಪೂಜಾಯೋಗ್ಯವಾದ, ಸತ್ಯವಾದ, ಏಕಮಾತ್ರ ದೇವತೆಗಳು ನಾವೆ. ವೇದಾಂತದ ರೀತಿಯೆ ಇದು. ಇದೇ ಅದರ ಕಾರ್ಯಕಾರಿತ್ವ. ನಾವು ಸ್ವತಂತ್ರರಾದ ಮೇಲೆ ಹುಚ್ಚರಂತೆ ಅಲೆಯಬೇಕಾ ಗಿಲ್ಲ, ಸಮಾಜವನ್ನು ನಿರಾಕರಿಸಿ, ಕಾಡಿನಲ್ಲೋ ಗುಹೆಯಲ್ಲೋ ಸಾಯುವುದಕ್ಕೆ ಹೋಗಬೇಕಾಗಿಲ್ಲ. ನಾವು ಎಲ್ಲಿದ್ದೆವೋ ಅಲ್ಲಿಯೇ ಇರುತ್ತೇವೆ. ಆದರೆ ಎಲ್ಲ ವನ್ನೂ ಆಮೂಲಾಗ್ರವಾಗಿ ತಿಳಿದುಕೊಳ್ಳುತ್ತೇವೆ. ಈಗಿನ ನೋಟವೇ ಆಗಲೂ ಇರುತ್ತದೆ. ಆದರೆ ಬೇರೆಯ ಅರ್ಥದಲ್ಲಿ. ನಮಗೆ ಇನ್ನೂ ಪ್ರಪಂಚದ ರಹಸ್ಯ ಗೊತ್ತಿಲ್ಲ. ಸ್ವಾತಂತ್ರ್ಯದ ಮೂಲಕ ಮಾತ್ರ ಅದು ಏನೆಂಬುದನ್ನು ನೋಡುತ್ತೇವೆ, ಅದರ ನಿಜತ್ವವನ್ನು ತಿಳಿಯುತ್ತೇವೆ. ನಿಯಮ, ಅದೃಷ್ಟ, ಗ್ರಹಚಾರ ಮುಂತಾದು ವುಗಳೆಲ್ಲ ಸ್ವರೂಪದ ಅತ್ಯಲ್ಪ ಭಾಗವಾಗಿದೆ ಎಂಬುದು ಆಗ ನಮಗೆ ಗೊತ್ತಾಗುತ್ತದೆ. ಇವುಗಳೆಲ್ಲ ಒಂದು ಭಾಗ ಮಾತ್ರ. ಆದರೆ ಉಳಿದ ಬೇರೆ ಭಾಗದಲ್ಲಿ ನಿತ್ಯ ಸ್ವಾತಂತ್ರ್ಯವಿದೆ. ನಮಗೆ ಇದು ಗೊತ್ತಿರಲಿಲ್ಲ. ಅದಕ್ಕೋಸ್ಕರವಾಗಿಯೇ ನಾವು ಪಾಪದಿಂದ ತಪ್ಪಿಸಿಕೊಳ್ಳಬೇಕೆಂದು, ಬೇಟೆಯವರಿಂದ ಅಟ್ಟಿಸಿಕೊಂಡು ಬಂದ ಮೊಲವು ಮರಳಿನಲ್ಲಿ ತನ್ನ ಮುಖವನ್ನು ಮುಚ್ಚಿಕೊಳ್ಳುವಂತೆ ಮಾಡುತ್ತಿದ್ದೆವು. ಮೋಹದಿಂದ ನಾವು ನಮ್ಮ ನೈಜತೆಯನ್ನು ಮರೆಯುವುದಕ್ಕೆ ಪ್ರಯತ್ನಿಸುತ್ತಿದ್ದೆವು. ಆದರೂ ಅದು ಸಾಧ್ಯವಾಗಲಿಲ್ಲ. ಅದು ಯಾವಾಗಲೂ ನಮ್ಮನ್ನು ಕರೆಯುತ್ತಿತ್ತು. ದೇವದೇವತೆಗಳನ್ನು ಹುಡುಕುವುದು, ಹೊರಗಿನ ಪ್ರಪಂಚದಲ್ಲಿ ಸ್ವತಂತ್ರರಾಗಲು ಪ್ರಯತ್ನಿಸುವುದು, ಇವುಗಳೆಲ್ಲ ನಮ್ಮ ನೈಜಸ್ಥಿತಿಯ ಅರಸುವಿಕೆ. ನಮಗೆ ಕೇಳುತ್ತಿದ್ದ ಧ್ವನಿಯನ್ನು ತಪ್ಪು ಪ್ರಭಾವಿಸಿಕೊಂಡೆವು. ಅದು ಬೆಂಕಿಯಿಂದ ಅಥವಾ ದೇವರಿಂದ, ಸೂರ್ಯ–ಚಂದ್ರ ನಕ್ಷತ್ರಾವಳಿಗಳಿಂದ ಬಂದಿತೆಂದು ನಾವು ಆಲೋಚಿಸಿದೆವು. ಆದರೆ ಕೊನೆಗೆ ಅದು ನಮ್ಮ ಅಂತರಂಗದಿಂದಲೇ ಬಂದಿತೆಂದು ಗೊತ್ತಾಗಿದೆ. ನಿಜ ಸ್ವಾತಂತ್ರ್ಯವನ್ನು ಬೋಧಿಸುತ್ತಿರುವ ಈ ಶಾಶ್ವತಧ್ವನಿ ನಮ್ಮ ಹೃದಯಾಂತ ರಾಳದಲ್ಲಿದೆ. ಅನವರತವೂ ಇದರ ಗಾನವಾಗುತ್ತಿದೆ. ಈ ಆತ್ಮಗಾನದ ಒಂದು ಅಂಶವೇ ಪೃಥ್ವಿಯಾಗಿರುವುದು, ನಿಯಮವಾಗಿರುವುದು, ವಿಶ್ವವಾಗಿರುವುದು. ಆದರೆ ಇದು ಯಾವಾಗಲೂ ನಮ್ಮದಾಗಿತ್ತು, ಮತ್ತು ನಮ್ಮದಾಗಿಯೇ ಇರುವುದು. ಒಂದು ಮಾತಿನಲ್ಲಿ ಹೇಳುವುದಾದರೆ, ವೇದಾಂತದ ಆದರ್ಶವೇ, ಮಾನವನ ನೈಜಸ್ಥಿತಿಯನ್ನು ಅರಿಯುವುದು. ನಿನ್ನ ಕಣ್ಣೆದುರಿಗೆ ಇರುವ ದೇವರಾದ ಸಹೋದರ ಮಾನವನನ್ನೇ ಪೂಜಿಸಲು ಆಗದೆ ಇದ್ದರೆ, ಅವ್ಯಕ್ತನಾದ ದೇವರನ್ನು ನೀನು ಹೇಗೆ ಪೂಜಿಸಬಲ್ಲೆ ಎಂಬುದೇ ವೇದಾಂತದ ಸಂದೇಶ.

ಬೈಬಲ್ಲು ಏನು ಹೇಳುತ್ತದೆ ಎಂಬುದು ನಿಮಗೆ ನೆನಪಿಲ್ಲವೆ? “ನೀನು ನೋಡಿದ ಸಹೋದರನನ್ನೇ ಪ್ರೀತಿಸಲಾಗದೆ ಇದ್ದರೆ, ನೋಡದ ದೇವರನ್ನು ಹೇಗೆ ಪ್ರೀತಿಸ ಬಲ್ಲೆ?” ಮಾನವಮುಖಗಳಲ್ಲಿ ನೀವು ದೇವರನ್ನು ಕಾಣದೆ ಇದ್ದರೆ ಎಲ್ಲೋ ಮೋಡಗಳ ಮಧ್ಯದಲ್ಲಿ, ಜಡವಾದ ಅಚೇತನವಾದ ವಸ್ತುವಿನಿಂದ ಮಾಡಿದ ವಿಗ್ರಹದಲ್ಲಿ ಅಥವಾ ನಿಮ್ಮ ಕಾಲ್ಪನಿಕ ಪುರಾಣಗಳಲ್ಲಿ ಹೇಗೆ ಅವನನ್ನು ನೋಡಬಲ್ಲಿರಿ? ಪುರುಷರಲ್ಲಿ ಮತ್ತು ಸ್ತ್ರೀಯರಲ್ಲಿ ದೇವರನ್ನು ನೋಡಲು ನೀವು ಮೊದಲು ಮಾಡಿದ ದಿನದಿಂದ ನಿಮ್ಮನ್ನು ಆಧ್ಯಾತ್ಮಿಕ ವ್ಯಕ್ತಿಗಳೆಂದು ಕರೆಯುತ್ತೇನೆ. ಬಲದ ಕೆನ್ನೆಗೆ ಹೊಡೆದರೆ ಎಡದ ಕೆನ್ನೆಯನ್ನು ತೋರಿಸುವುದು ಎಂದರೆ ಏನೆಂಬುದು ಆಗ ನಿಮಗೆ ಗೊತ್ತಾಗುವುದು. ನೀವು ಎಂದು ಮಾನವನನ್ನು ದೇವರಂತೆ ಕಾಣುತ್ತೀರೊ ಅಂದಿನಿಂದ ಎಲ್ಲವನ್ನೂ, ದುಷ್ಟ ವ್ಯಾಘ್ರವನ್ನು ಕೂಡ ಪ್ರೇಮದಿಂದ ಬರ ಮಾಡಿಕೊಳ್ಳುತ್ತೀರಿ. ನಿಮಗೆ ಏನು ಪ್ರಾಪ್ತವಾದರೂ, ಅದೆಲ್ಲವೂ ನಿತ್ಯಾನಂದ ಮಯನಾದ ಭಗವಂತನೆ. ಆತನೆ ನಮಗೆ ನಾನಾ ರೂಪಿನಲ್ಲಿ, ತಂದೆಯಂತೆ, ತಾಯಿಯಂತೆ, ಸ್ನೇಹಿತನಂತೆ, ಮಗುವಿನಂತೆ, ಕಾಣಿಸಿಕೊಳ್ಳುತ್ತಿರುವನು. ನಮ್ಮೊಂದಿಗೆ ಆಟವಾಡುತ್ತಿರುವ ನಮ್ಮಾತ್ಮವೇ ಅವನು.

ಹೇಗೆ ಮಾನವರೊಂದಿಗೆ ಇರುವ ನಮ್ಮ ಸಂಬಂಧವನ್ನು ಪವಿತ್ರವಾಗಿ ಮಾಡಬಹುದೊ, ಅದರಂತೆಯೇ ದೇವರೊಂದಿಗೆ ನಮಗೆ ಇರುವ ಸಂಬಂಧವು ಯಾವುದಾದರೊಂದು ಮಾನವಿಕ ಸಂಬಂಧವನ್ನು ಧರಿಸಬಹುದು. ದೇವರನ್ನು ನಾವು ನಮ್ಮ ತಾಯಿ, ತಂದೆ, ಸ್ನೇಹಿತ, ಪ್ರಿಯತಮ ಮುಂತಾದ ದೃಷ್ಟಿಯಿಂದ ನೋಡ ಬಹುದು. ದೇವರನ್ನು ತಾಯಿಯೆಂದು ಕರೆಯುವುದು, ತಂದೆಯೆಂದು ಕರೆಯುವು ದಕ್ಕಿಂತ ಮೇಲಾದುದು. ಇದಕ್ಕಿಂತ ಸ್ನೇಹಿತನೆಂದು ಕರೆಯುವುದು ಮತ್ತೂ ಮೇಲಾ ದುದು. ಸರ್ವೋತ್ತಮವಾದುದೆ ಪ್ರಿಯತಮನೆಂಬುದು. ಇವುಗಳ ಪರಮಾವಧಿಯೆ ಪ್ರಿಯನು ಪ್ರೇಯಸಿಯಲ್ಲಿ ಭಿನ್ನತೆಯನ್ನು ನೋಡದೆ ಇರುವುದು. ಒಬ್ಬ ಪ್ರಿಯನು ಬಂದು ತನ್ನ ಪ್ರೇಯಸಿ ಇರುವ ಕೋಣೆಯ ಬಾಗಿಲನ್ನು ತಟ್ಟಿದ ಒಂದು ಪರ್ಸಿಯ ದೇಶದ ಹಳೆಯ ಕತೆ ಬಹುಶಃ ನಿಮಗೆ ಜ್ಞಾಪಕವಿರಬಹುದು. ಪ್ರೇಯಸಿ ನೀನಾರೆಂದು ಪ್ರಶ್ನೆ ಮಾಡಿದಳು. ಅದಕ್ಕೆ ‘ನಾನು’ ಎಂದು ಉತ್ತರ ಕೊಟ್ಟನು. ಪ್ರತ್ಯುತ್ತರ ಬರಲಿಲ್ಲ. ಎರಡನೆಯ ಸಲ ಪ್ರಿಯನು “ನಾನು ಇಲ್ಲಿರುವುದು” ಎಂದು ಹೇಳಿದನು. ಆದರೂ ಬಾಗಿಲು ತೆರೆಯಲಿಲ್ಲ. ಮೂರನೇ ಸಲ ಅವನು ಬಂದನು, ಒಳಗಿನಿಂದ ಧ್ವನಿಯು “ಯಾರವರು?” ಎಂದು ಪ್ರಶ್ನೆ ಮಾಡಿತು. ಆತನು “ನಾನೇ ನೀನು, ಎನ್ನ ಪ್ರಿಯ ತಮಳೆ” ಎಂದು ಉತ್ತರ ಕೊಟ್ಟನು. ಆಗ ಬಾಗಿಲು ತೆರೆಯಿತು. ಇದರಂತೆಯೇ ದೇವರಿಗೂ ನಮಗೂ ಇರುವ ಸಂಬಂಧವೂ ಕೂಡ. ಸಮಸ್ತದಲ್ಲಿಯೂ ಆತನು ಇರುವನು. ಆತನೇ ಸಮಸ್ತವೂ ಕೂಡ. ಪ್ರತಿಯೊಬ್ಬ ಪುರುಷನೂ ಸ್ತ್ರೀಯೂ ಕೂಡ ಪ್ರತ್ಯಕ್ಷನಾದ, ಆನಂದಮಯನಾದ, ಸಚೇತನ ದೇವರು. ದೇವರು ಅಜ್ಞಾತನೆಂದು ಹೇಳುವವರು ಯಾರು? ಆತನನ್ನು ಹುಡುಕಬೇಕೆಂದು ಹೇಳುವವರು ಯಾರು? ಶಾಶ್ವತವಾದ ದೇವರನ್ನು ನಾವು ನೋಡಿರುವೆವು. ಶಾಶ್ವತವಾಗಿಯೂ ನಾನು ಅವನಲ್ಲಿ ವಾಸಿಸುತ್ತಿರುವೆನು. ಎಲ್ಲೆಡೆಯಲ್ಲಿಯೂ ಆತನು ನಿತ್ಯ ಪರಿಚಿತನು, ನಿತ್ಯಪೂಜಿತನು.

ಅನಂತರ ಮಿಕ್ಕ ಪೂಜಾವಿಧಾನಗಳು ತಪ್ಪಲ್ಲವೆಂಬ ಮತ್ತೊಂದು ಭಾವನೆ ಬರುವುದು. ನಾವು ಜ್ಞಾಪಕದಲ್ಲಿಟ್ಟುಕೊಳ್ಳಬೇಕಾದ ಬಹಳ ಮುಖ್ಯವಾದ ಒಂದು ಅಂಶವೇ, ಬಾಹ್ಯಾಚಾರದ ಮೂಲಕ, ಎಷ್ಟೇ ಒರಟಾದ ವಿಗ್ರಹದ ಮೂಲಕ ದೇವರನ್ನು ಪೂಜಿಸುವವರು ಕೂಡ ತಪ್ಪು ಮಾರ್ಗದಲ್ಲಿ ಇಲ್ಲ ಎನ್ನುವುದು. ಸತ್ಯದಿಂದ ಸತ್ಯದೆಡೆಗೆ ಪ್ರಯಾಣ ಇದು. ಸಣ್ಣ ಸತ್ಯದಿಂದ ದೊಡ್ಡ ಸತ್ಯದೆಡೆಗೆ ಪ್ರಯಾಣ. ಕತ್ತಲೆ ಎಂದರೆ, ಕಡಿಮೆ ಬೆಳಕು, ಕೆಟ್ಟದು ಎಂದರೆ ಕಡಿಮೆ ಒಳ್ಳೆಯದು. ಮಲಿನತೆ ಎಂದರೆ ಕಡಿಮೆ ಪಾವನತೆ. ನಾವು ಬಂದ ದಾರಿಯಲ್ಲೇ ಅವರೂ ನಡೆಯುತ್ತಿರುವರು ಎಂಬ ಸಹಾನುಭೂತಿಯಿಂದ, ಪ್ರೇಮದಿಂದ ಇನ್ನೊಬ್ಬರನ್ನು ನೋಡಬೇಕೆನ್ನುವುದನ್ನು ನಾವು ಯಾವಾಗಲೂ ಜ್ಞಾಪಕದಲ್ಲಿಟ್ಟಿರಬೇಕು. ನೀವೀಗ ಮುಕ್ತರಾದರೆ, ಎಲ್ಲರೂ ಈಗಲೋ ಸ್ವಲ್ಪ ತಡವಾಗಿಯೋ ಮುಕ್ತರಾಗುವರೆಂಬುದನ್ನು ತಿಳಿಯಬೇಕು. ನೀವು ಮುಕ್ತರಾದರೆ ಅಶಾಶ್ವತತೆಯನ್ನು ಹೇಗೆ ಕಾಣುವಿರಿ? ನೀವು ನಿಜವಾಗಿಯೂ ಪರಿಶುದ್ಧರಾಗಿದ್ದರೆ ಹೊರಗೆ ಕಳಂಕವನ್ನು ಹೇಗೆ ನೋಡುತ್ತೀರಿ? ಏಕೆಂದರೆ ಯಾವುದು ಅಂತರಂಗದಲ್ಲಿದೆಯೋ ಅದೇ ಬಾಹ್ಯದಲ್ಲಿದೆ. ನಮ್ಮಲ್ಲಿ ಮಲಿನತೆಯಿಲ್ಲದೆ ಹೊರಗೆ ಮಲಿನತೆಯನ್ನು ನೋಡಲಾರೆವು. ಇದು ವೇದಾಂತದ ಒಂದು ಅನುಷ್ಠಾನ ಭಾಗ. ನಾವೆಲ್ಲರೂ ನಮ್ಮ ಜೀವನದಲ್ಲಿ ಇದನ್ನು ಅಭ್ಯಾಸ ಮಾಡುವುದಕ್ಕೆ ಪ್ರಯತ್ನಿಸುತ್ತೇವೆ ಎಂದು ತಿಳಿಯುತ್ತೇನೆ. ಇವುಗಳನ್ನು ಅನುಷ್ಠಾನಕ್ಕೆ ತರುವುದೇ ನಮ್ಮ ಜೀವನದ ಗುರಿ. ನಮಗೆ ಇದರಿಂದ ಆಗುವ ದೊಡ್ಡ ಪ್ರಯೋಜನವೆಂದರೆ, ಅತೃಪ್ತಿ ಅಶಾಂತಿಗಳಿಲ್ಲದೆ ಶಾಂತಿಯಿಂದ, ತೃಪ್ತಿಯಿಂದ ಕೆಲಸ ಮಾಡುತ್ತೇವೆ. ಏಕೆಂದರೆ, ಸತ್ಯ ನಮ್ಮಲ್ಲಿದೆ ಎಂಬುದು ನಮಗೆ ಗೊತ್ತು. ಅದು ನಮ್ಮ ಜನ್ಮಸಿದ್ಧ ಹಕ್ಕಾಗಿದೆ. ಅದನ್ನು ನಾವು ಪ್ರಕಾಶಕ್ಕೆ ತರಬೇಕಾಗಿದೆ, ಅದನ್ನು ಸುಸ್ಪಷ್ಟಗೊಳಿಸಬೇಕಾಗಿದೆ.

\chapter{ಅಧ್ಯಾಯ ೩}

\begin{center}
\textbf{(೧೮೯೬ನೆಯ ನವೆಂಬರ್​ ೧೭ ರಂದು ಲಂಡನ್ನಿನಲ್ಲಿ ನೀಡಿದ ಉಪನ್ಯಾಸ)}
\end{center}

ಛಾಂದೋಗ್ಯ ಉಪನಿಷತ್ತಿನಲ್ಲಿ ನಾರದಋಷಿಯು ಸನತ್ಕುಮಾರನ ಬಳಿಗೆ ಹೋಗಿ ಅನೇಕ ಪ್ರಶ್ನೆಗಳನ್ನು ಕೇಳಿದ ಭಾಗವನ್ನು ಈಗ ನಿಮಗೆ ಓದಿ ಹೇಳುತ್ತೇನೆ. ಅವುಗಳಲ್ಲಿ ಒಂದು ಪ್ರಶ್ನೆ ವಸ್ತುಗಳ ಸ್ಥಿತಿಗೆ ಧರ್ಮವು ಕಾರಣವೇ ಎನ್ನುವುದು. ಸನತ್ಕುಮಾರನು ಆತನಿಗೆ ಈ ಜಗತ್ತಿಗಿಂತ ಮೇಲಾದುದು ಮತ್ತೊಂದಿದೆ, ಅದಕ್ಕಿಂತ ಮೇಲಾದುದು ಮತ್ತೊಂದಿದೆ ಎಂದು ಹೇಳಿ ಅವನನ್ನು ಮೆಟ್ಟಿಲು ಮೆಟ್ಟಲಾಗಿ ಆಕಾಶತತ್ತ್ವದ ಪರಿಯಂತರವೂ ಕರೆದುಕೊಂಡು ಬರುವನು. ಆಕಾಶವು ಜ್ಯೋತಿಗಿಂತ ಮೇಲಾದುದು. ಏಕೆಂದರೆ ಸೂರ್ಯ ಚಂದ್ರ ನಕ್ಷತ್ರಾವಳಿಗಳು, ಮಿಂಚು ಇವೆಲ್ಲ ಇರುವುದು ಆಕಾಶದಲ್ಲಿ. ನಾವು ವಾಸಮಾಡುವುದು ಆಕಾಶದಲ್ಲಿ, ಸಾಯುವುದು ಆಕಾಶದಲ್ಲಿ. ಆಗ ಅದಕ್ಕಿಂತ ಮೇಲಾದುದು ಏನಾದರೂ ಇದೆಯೆ ಎಂದು ಪ್ರಶ್ನೆ ಏಳುತ್ತದೆ. ಆಗ ಸನತ್ಕುಮಾರನು ಅದಕ್ಕಿಂತ ಮೇಲಾದುದು ಪ್ರಾಣ ಎಂದು ಹೇಳುತ್ತಾನೆ. ವೇದಾಂತದ ದೃಷ್ಟಿಯಿಂದ ಪ್ರಾಣವೇ ಜೀವತತ್ತ್ವ. ಇದು ಆಕಾಶದಂತೆ ಒಂದು ಸರ್ವವ್ಯಾಪಿಯಾದ ತತ್ತ್ವ. ದೇಹದಲ್ಲಿಯಾಗಲೀ ಅಥವಾ ಇನ್ನು ಎಲ್ಲಿಯಾದರೂ ಆಗಲಿ, ಎಲ್ಲಾ ಚಲನೆಯೂ ಪ್ರಾಣದ ಕೆಲಸ. ಇದು ಆಕಾಶಕ್ಕಿಂತ ಮೇಲಾದುದು. ಇದರ ಮೂಲಕ ಎಲ್ಲವೂ ಜೀವಿಸುತ್ತದೆ. ಪ್ರಾಣವು ತಾಯಿಯಲ್ಲಿದೆ, ತಂದೆಯಲ್ಲಿದೆ, ಅಕ್ಕ ತಂಗಿಯರಲ್ಲಿದೆ, ಗುರುಗಳಲ್ಲಿದೆ. ಪ್ರಾಣವೇ ಜ್ಞಾತೃ.

ಶ್ವೇತಕೇತುವು ತನ್ನ ತಂದೆಯನ್ನು ಸತ್ಯದ ವಿಚಾರವಾಗಿ ಕೇಳಿದ ಮತ್ತೊಂದು ಭಾಗವನ್ನು ಈಗ ನಿಮಗೆ ಓದಿ ಹೇಳುತ್ತೇನೆ. ತಂದೆಯು ಆತನಿಗೆ ಬೇರೆ ಬೇರೆ ವಿಷಯಗಳನ್ನು ಹೇಳಿ, “ಇವುಗಳಲ್ಲೆಲ್ಲಾ ಯಾವುದು ಸೂಕ್ಷ್ಮಕಾರಣವೊ, ಯಾವುದರಿಂದ ಇವೆಲ್ಲ ಆಗಿವೆಯೊ, ಅದೇ ಸರ್ವ, ಅದೇ ಸತ್ಯ! ಓ ಶ್ವೇತಕೇತು, ನೀನೇ ಅದು” ಎಂದು ಹೇಳಿ ಪೂರೈಸುವನು. ಅನಂತರ ಅನೇಕ ಉದಾಹರಣೆಗಳನ್ನು ಹೇಳುವನು. “ಓ ಶ್ವೇತಕೇತು, ಜೇನುಹುಳವು ಮಕರಂದವನ್ನು ಬೇರೆ ಬೇರೆ ಹುವುಗಳಿಂದ ಬಂದಿರುವುದು ತಿಳಿಯದೆ ಇರುವಂತೆ, ನಾವು ಕೂಡ ಅಂತಹ ಅತ್ಯುತ್ತಮ ಸ್ಥಿತಿಗೆ ಬಂದ ಮೇಲೆ ನಾವು ಹೇಗೆ ಬಂದೆವೆಂಬುದು ನಮಗೆ ತಿಳಿಯದು. ಯಾವುದು ಅಂತಹ ಸೂಕ್ಷ್ಮವಾದ ಸಾರವೋ ಅದೇ ಪ್ರತಿಯೊಂದು ವಸ್ತುವಿನ ಆತ್ಮ. ಅದೇ ಸತ್ಯ, ಅದೇ ಆತ್ಮ, ಓ ಶ್ವೇತಕೇತು, ನೀನೇ ಅದು.” ನದಿಗಳು ಸಮುದ್ರಕ್ಕೆ ಹರಿದುಹೋಗುವ ಮತ್ತೊಂದು ಉದಾಹರಣೆಯನ್ನು ಕೊಡುವನು. “ಸಾಗರದಲ್ಲಿದ್ದಾಗ ನದಿಗಳೆಲ್ಲ, ನಾವು ಬೇರೆ ಬೇರೆ ನದಿಗಳಾಗಿದ್ದೆವು ಎಂದು ಹೇಗೆ ತಿಳಿಯವೊ ಅದರಂತೆಯೇ ನಾವು ಕೂಡ ಅಂತಹ ಸತ್ಯದಿಂದ ಬಂದ ಮೇಲೆ ನಾವು ಅದೇ ಎಂಬುದು ನಮಗೆ ತಿಳಿಯದು. ಓ, ಶ್ವೇತಕೇತು, ನೀನೇ ಅದು.” ಹೀಗೆ ಅವನ ಉಪದೇಶ ಮುಂದೆ ಸಾಗುವುದು.

ಜ್ಞಾನದ ಎರಡು ನಿಯಮಗಳಿವೆ. ಮೊದಲನೆಯದು ವಿಶೇಷವಾದುದನ್ನು ಸಾಧಾರಣವಾದುದಕ್ಕೆ ಹೋಲಿಸಿ, ಅನಂತರ ಸಾಧಾರಣವಾದುದನ್ನು ಸರ್ವಸಾಧಾರಣ ವಾದುದಕ್ಕೆ ಹೋಲಿಸುವುದರ ಮೂಲಕ ತಿಳಿಯುವುದು. ಎರಡನೆಯ ನಿಯಮವೆ, ವಿವರಣೆ ಯಾವ ವಸ್ತುವಿಗೆ ಬೇಕೋ ಅದನ್ನು ಸಾಧ್ಯವಾದ ಮಟ್ಟಿಗೆ ವಸ್ತುವಿನ ಸ್ವಭಾವದ ಮೂಲಕವೆ ವಿವರಿಸುವುದು. ಮೊದಲನೆಯ ನಿಯಮವನ್ನು ತೆಗೆದು ಕೊಂಡರೆ, ನಮ್ಮ ಜ್ಞಾನವೆಲ್ಲ ಹೆಚ್ಚು ಹೆಚ್ಚು ವರ್ಗೀಕರಣದಲ್ಲಿದೆ ಎಂಬುದನ್ನೂ, ಹೀಗೆ ಮಾಡಿ ಉನ್ನತದಿಂದ ಉನ್ನತ ಸ್ತರಕ್ಕೆ ಹೋಗುತ್ತೇವೆ ಎಂಬುದನ್ನೂ ನೋಡುತ್ತೇವೆ. ಯಾವಾಗಲಾದರೂ ಏನಾದರೂ ಒಂಟಿಯಾಗಿ ನಡೆದರೆ ನಮಗೆ ಅಸಮಾಧಾನ. ಆ ಸಂಗತಿಯು ಮತ್ತೆ ಮತ್ತೆ ಆಗುತ್ತದೆ ಎಂಬುದನ್ನು ತೋರಿಸಿದಾಗ ನಮಗೆ ತೃಪ್ತಿಯಾಗಿ ಅದನ್ನು ಒಂದು ನಿಯಮ ಎಂದು ಕರೆಯುತ್ತೇವೆ. ಒಂದು ಸೇಬಿನ ಹಣ್ಣು ಕೆಳಗೆ ಬೀಳುವುದನ್ನು ನೋಡಿದಾಗ ನಮಗೆ ಅತೃಪ್ತಿಯಾಗುವುದು. ಆದರೆ ಎಲ್ಲಾ ಸೇಬಿನ ಹಣ್ಣುಗಳೂ ಬೀಳುವುದನ್ನು ನೋಡಿದಾಗ ತೃಪ್ತಿಯಾಗಿ ಅದನ್ನು ಆಕರ್ಷಣ ಶಕ್ತಿಯ ನಿಯಮವೆಂದು ಹೇಳುತ್ತೇವೆ. ವಸ್ತುಸ್ಥಿತಿಯೇನೆಂದರೆ, ನಾವು ವಿಶೇಷವಾದುದರಿಂದ ಸಾಮಾನ್ಯವಾದುದನ್ನು ತರ್ಕಿಸಿ ತಿಳಿಯುತ್ತೇವೆ.

ನಾವು ಧಾರ್ಮಿಕ ವಿಷಯಗಳನ್ನು ವಿಚಾರ ಮಾಡುವಾಗ ಈ ವೈಜ್ಞಾನಿಕ ರೀತಿಯನ್ನು ಬಳಸಬೇಕು. ಅದೇ ನಿಯಮವೇ ಇಲ್ಲಿಯೂ ಕೂಡ ಸರಿಹೋಗುತ್ತದೆ. ನಿಜವಾಗಿಯೂ ಇದೇ ನಿಯಮವನ್ನೇ ಇಂದಿನವರೆಗೂ ನಾವು ಅನುಸರಿಸಿರುವುದು. ನಿಮಗೆ ನಾನು ಅನುವಾದ ಮಾಡಿ ಹೇಳುತ್ತಿರುವ ಪುಸ್ತಕಗಳನ್ನು ಓದಿದಾಗ ಒಂದು ಅಂಶ ಸ್ಪಷ್ಟವಾಗುತ್ತದೆ. ನಮ್ಮ ವಾದಸರಣಿಯಲ್ಲಿ ವಿಶೇಷದಿಂದ ಸಾಮಾನ್ಯಕ್ಕೆ ತಲುಪುವ ಪ್ರಾಚೀನ ತತ್ತ್ವವನ್ನು ಅನುಸರಿಸುತ್ತಿದ್ದೇವೆ ಎಂಬುದೇ ಅದು. ದೇವತೆಗಳು ಹೇಗೆ ಒಂದು ಸಾಮಾನ್ಯ ನಿಯಮದಲ್ಲಿ ಐಕ್ಯರಾದರೆಂಬುದನ್ನು ನಾವು ನೋಡಿದೆವು. ಅದರಂತೆಯೆ ಸೃಷ್ಟಿಯ ಭಾವನೆಯಲ್ಲಿ ಕೂಡ ಆ ಪುರಾತನ ತತ್ತ್ವಜ್ಞಾನಿಗಳು ಮೇಲುಮೇಲಕ್ಕೆ ಹೋಗುತ್ತಿರುವುದನ್ನು ನೋಡುವೆವು. ಸೂಕ್ಷ್ಮವಸ್ತುವಿನಿಂದ ಸೂಕ್ಷ್ಮತರವಾದ, ಹೆಚ್ಚು ವ್ಯಾಪಿತಯುಳ್ಳ ವಸ್ತುವಿಗೆ ಹೋಗುತ್ತಾರೆ. ಅಲ್ಲಿಂದ ಮತ್ತೂ ಸೂಕ್ಷ್ಮತರವಾದ ಸರ್ವಸಾಮಾನ್ಯವಾದ ಆಕಾಶಕ್ಕೆ ಹೋಗುತ್ತಾರೆ. ಅಲ್ಲಿಂದಲೂ ಕೂಡ ಅವರು ಸರ್ವವ್ಯಾಪಿಯಾದ ಶಕ್ತಿ ಅಥವಾ ಪ್ರಾಣಕ್ಕೆ ಹೋಗುತ್ತಾರೆ. ಇವುಗಳೆಲ್ಲದರ ಅಂತರಾಳದಲ್ಲಿ ಕೂಡ ಒಂದು ಮತ್ತೊಂದರಿಂದ ಬೇರೆಯಲ್ಲವೆಂಬ ನಿಯಮವು ಹರಿಯುತ್ತಿರುವುದು. ಆಕಾಶವೆ ಸೂಕ್ಷ್ಮತರ ಪ್ರಾಣವಾಗಿರುವುದು. ಅಥವಾ ಸೂಕ್ಷ್ಮ ಪ್ರಾಣವೇ ಮೂರ್ತರೂಪವನ್ನು ಪಡೆದು ಆಕಾಶವಾಗುವುದು, ಮತ್ತು ಆಕಾಶವು ಇನ್ನೂ ಸ್ಥೂಲವಾಗುವುದು, ಇತ್ಯಾದಿ.

ಸಗುಣ ದೇವರ ಸಾಮಾನ್ಯೀಕರಣದ ಮತ್ತೊಂದು ವಿಚಾರವಿದೆ. ಈ ಸಾಮಾನ್ಯ ತತ್ತ್ವವನ್ನು ಅವರು ಹೇಗೆ ಕಂಡುಹಿಡಿದರು ಎಂಬುದನ್ನೂ ಮತ್ತು ಅದನ್ನು ನಮ್ಮ ಪ್ರಜ್ಞಗಳೆಲ್ಲದರ ಮೊತ್ತವೆಂದು ಕರೆದರು ಎಂಬುದನ್ನೂ ನಾವು ನೋಡಿದ್ದೇವೆ. ಆದರೆ ಇಲ್ಲಿ ಒಂದು ಕಷ್ಟ ತಲೆದೋರುತ್ತದೆ. ಈ ಸಾಮಾನ್ಯೀಕರಣ ಅಸಂಪೂರ್ಣ ವಾಗಿದೆ. ಪ್ರಕೃತಿಯ ಅರ್ಧಭಾಗವಾದ ಪ್ರಜ್ಞೆಯನ್ನು ಮಾತ್ರ ತೆಗೆದುಕೊಂಡು ಅದರಲ್ಲಿ ಸರ್ವಸಾಮಾನ್ಯ ತತ್ತ್ವವನ್ನು ಹುಡುಕಿ, ಉಳಿದ ಅರ್ಧವನ್ನು ಗಮನಕ್ಕೇ ತೆಗೆದುಕೊಂಡಿಲ್ಲ. ಆದಕಾರಣ ಇದು ಮೊದಲನೆಯದಾಗಿ ದೋಷಪೂರ್ಣ ಸಾಮಾನ್ಯೀಕರಣ. ಇದರಲ್ಲಿ ಮತ್ತೊಂದು ಕುಂದಿದೆ. ಅದು ಎರಡನೆಯ ನಿಯಮಕ್ಕೆ ಸಂಬಂಧಪಟ್ಟದ್ದು. ಪ್ರತಿಯೊಂದು ವಸ್ತುವನ್ನೂ ಅದರ ಸ್ವಭಾವದ ಮೂಲಕವಾಗಿಯೆ ವಿವರಿಸಬೇಕು. ನೆಲದ ಮೇಲೆ ಬೀಳುವ ಸೇಬಿನ ಹಣ್ಣನ್ನು ಯಾವುದೋ ಒಂದು ಭೂತ ನೆಲಕ್ಕೆ ಸೆಳೆಯಿತೆಂದು ನಂಬುವ ಕೆಲವರಿರಬಹುದು. ಆದರೆ ಆಕರ್ಷಣ ಶಕ್ತಿಯ ನಿಯಮವೆ ಇದಕ್ಕೆ ಕಾರಣ ಎಂಬುದು ಆ ಘಟನೆಯ ವಿವರಣೆ. ಇದೂ ಅಷ್ಟೊಂದು ತೃಪ್ತಿದಾಯಕವಾದ ವಿವರಣೆ ಅಲ್ಲವೆಂದು ಗೊತ್ತಿದ್ದರೂ ಕೂಡ ಇದು ಮೊದಲನೆಯ ವಿವರಣೆಗಿಂತಲೂ ಎಷ್ಟೋ ಮೇಲು. ಏಕೆಂದರೆ ಇದು ವಸ್ತುವಿನ ಸ್ವಭಾವದಿಂದಲೇ ಬಂದಿರುವುದು. ಮೊದಲನೆಯ ವಿವರಣೆಯಲ್ಲಿ ಒಂದು ಬಾಹ್ಯ ಕಾರಣವನ್ನು ಊಹಿಸಲಾಗಿದೆ. ಆದಕಾರಣ ನಮ್ಮ ಅನುಭವದ ಎಲ್ಲೆಯೊಳಗೆ, ವಸ್ತುವಿನ ಸ್ವಭಾವದ ಮೇಲೆ ನಿಂತ ವಿವರಣೆಯು ವೈಜ್ಞಾನಿಕವಾದುದು; ಯಾವ ವಿವರಣೆ ಬಾಹ್ಯ ವಸ್ತುವಿನ ಸಹಕಾರವನ್ನು ಕೋರುವುದೋ ಅದು ವೈಜ್ಞಾನಿಕವಾದುದಲ್ಲ.

ಆದಕಾರಣ ಸಗುಣ ದೇವರು ಸೃಷ್ಟಿಗೆ ಕಾರಣ ಎಂಬ ವಿವರಣೆಯು ಆ ಪರೀಕ್ಷೆಯಲ್ಲಿ ತೇರ್ಗಡೆಯಾಗಬೇಕು. ಆ ದೇವರಿಗೆ ಪ್ರಕೃತಿಯೊಂದಿಗೆ ಯಾವ ಸಂಬಂಧವೂ ಇಲ್ಲದೆ, ಅವನು ಪ್ರಕೃತಿಯ ಹೊರಗೆ ಇದ್ದು, ಆತನ ಆಜ್ಞೆಯಂತೆ ಶೂನ್ಯದಿಂದ ಈ ಸೃಷ್ಟಿಯು ಬಂದಿದ್ದರೆ, ಇದು ಅತ್ಯಂತ ಅವೈಜ್ಞಾನಿಕವಾದ ಸಿದ್ಧಾಂತವಾಗುತ್ತದೆ. ಪ್ರಾಚೀನ ಕಾಲದಿಂದಲೂ ಸಗುಣ ದೇವರನ್ನು ನಂಬಿದ ಪ್ರತಿಯೊಂದು ಧರ್ಮದ ದುರ್ಬಲ ಅಂಶವೂ ಇದೇ. ಏಕದೇವ ವಾದ ಅಥವಾ ಸಗುಣಬ್ರಹ್ಮ ವಾದ ಎಂದು ಯಾವುದನ್ನು ಕರೆಯುತ್ತಾರೆಯೋ, ಅದರಲ್ಲಿ ಈ ಎರಡು ಕೊರತೆಗಳು ಕಾಣುತ್ತವೆ. ಸಾಕಾರ ದೇವರಲ್ಲಿ ಮಾನವರಲ್ಲಿರುವ ಸುಗುಣಗಳನ್ನೆಲ್ಲ ಅಧಿಕವಾಗಿ ಗುಣಾಕಾರ ಮಾಡಿದಂತೆ ಕಾಣುತ್ತದೆ. ತನ್ನ ಇಚ್ಛೆಯ ಪ್ರಕಾರ ವಿಶ್ವವನ್ನು ಶೂನ್ಯದಿಂದ ಅವನು ಸೃಷ್ಟಿಸಿದನು, ಆದರೂ ಆತನು ಅದರಿಂದ ಬೇರೆಯಾಗಿರುವನು, ಎಂದು ಹೇಳಲಾಗಿದೆ. ಇದು ನಮ್ಮನ್ನು ಎರಡು ಕಷ್ಟಗಳಿಗೆ ಗುರಿಮಾಡುತ್ತದೆ.

ನಮಗೆ ಕಂಡಂತೆ ಇದು ಸಮರ್ಪಕವಾದ ಸಾಮಾನ್ಯೀಕರಣವಲ್ಲ. ಎರಡನೆ ಯದಾಗಿ ಇದು ಪ್ರಕೃತಿಯಿಂದ ಪ್ರಕೃತಿಯ ವಿವರಣೆಯಲ್ಲ. ಕಾರ್ಯವು ಕಾರಣವಲ್ಲ, ಕಾರಣವು ಕಾರ್ಯದಿಂದ ಸಂಪೂರ್ಣ ಬೇರೆ ಎಂದು ಇದು ಹೇಳುತ್ತದೆ. ಆದರೆ ಮಾನವನ ಜ್ಞಾನವೆಲ್ಲವೂ ಕಾರ್ಯವು ಕಾರಣದ ಮತ್ತೊಂದು ರೂಪವೆಂದು ಹೇಳುತ್ತದೆ. ಆಧುನಿಕ ವೈಜ್ಞಾನಿಕ ಸಂಶೋಧನೆಯು ದಿನಕ್ರಮೇಣ ಇದನ್ನು ಸಮರ್ಥಿಸುತ್ತದೆ. ಎಲ್ಲರ ಸಮ್ಮತಿಯನ್ನು ಪಡೆದ ಅತ್ಯಾಧುನಿಕವಾದ ನಿಯಮವೆಂದರೆ ವಿಕಾಸವಾದ. ಕಾರ್ಯವು. ಕಾರಣದ ರೂಪಾಂತರವೆನ್ನುವುದು ಅದರ ಮುಖ್ಯ ತತ್ತ್ವ. ಕಾರ್ಯವು ಕಾರಣದ ಒಂದು ಪುನರ್ರಚನೆ, ಕಾರಣವೆ ಕಾರ್ಯದ ರೂಪಾಂತರವನ್ನು ಹೊಂದುವುದು. ಶೂನ್ಯದಿಂದ ಸೃಷ್ಟಿ ಎಂಬುದು ಆಧುನಿಕ ವಿಜ್ಞಾನಿಗಳ ಅಪಹಾಸ್ಯಕ್ಕೆ ಗುರಿಯಾಗಿದೆ.

ಈಗ, ಧರ್ಮವು ಈ ಪರೀಕ್ಷೆಗಳನ್ನು ಎದುರಿಸಬಲ್ಲುದೆ? ಈ ಎರಡು ಪರೀಕ್ಷೆ ಯಲ್ಲೂ ತೇರ್ಗಡೆಯಾಗಬಲ್ಲ ಧಾರ್ಮಿಕ ನಿಯಮಗಳಿದ್ದರೆ, ಆಧುನಿಕ ವಿಚಾರಪರ ರಿಗೆ ಸ್ವೀಕಾರ ಯೋಗ್ಯವಾಗಿರುತ್ತವೆ. ಮತ್ತಾವ ಸಿದ್ಧಾಂತವನ್ನೇ ಆಗಲೀ, ಪಾದ್ರಿಗಳ, ಚರ್ಚಿನ ಅಥವಾ ಪುಸ್ತಕದ ಬಲದ ಮೇಲೆ ನಂಬಿ ಎಂದು ಹೇಳಿದರೆ ಇಂದಿನವರು ಅದನ್ನು ಒಪ್ಪಿಕೊಳ್ಳಲಾರರು. ಇದರ ಪರಿಣಾಮವೆ ಒಂದು ಅನಿಷ್ಟ ಅಪನಂಬಿಕೆಯ ರಾಶಿ. ಹೊರಗಡೆ ನಂಬಿಕೆಯನ್ನು ನಟಿಸುವವರು ಹೃದಯದ ಅಂತರಾಳದಲ್ಲಿ ಕೂಡ ಬೇಕಾದಷ್ಟು ಅಪನಂಬಿಕೆ ಇದೆ. ಉಳಿದವರು ಧರ್ಮವೆಂದರೆ ಅದು ಒಂದು ಪುರೋಹಿತರ ಕಾರುಬಾರೆಂದು ತಿಳಿದು ಅದರಿಂದ ದೂರವಾಗಿರುವರು.

ಧರ್ಮವನ್ನು ಒಂದು ರೀತಿಯಲ್ಲಿ ಒಂದು ಜನಾಂಗದ ಬಾಹ್ಯ ಆಚರಣೆ ಎಂಬ ಹೀನ ಸ್ಥಿತಿಗೆ ಇಂದು ತಂದಿರುವರು. ನಮ್ಮ ಸಮಾಜದಲ್ಲಿ ಹಿಂದೆ ಕೆಲಸಮಾಡಿ ಉಳಿದಿರುವ ಒಂದು ಅತ್ಯುತ್ತಮವಾದ ಅಂಗ ಧರ್ಮವೆಂಬುದು. ಆದುದರಿಂದ ಇದು ಇರಲಿ. ಆದರೆ ತಮ್ಮ ತಾತಂದಿರಲ್ಲಿ ಧರ್ಮಕ್ಕೆ ಇದ್ದ ಅತ್ಯಾವಶ್ಯಕತೆ ಈಗಿನವರಲ್ಲಿ ಮಾಯವಾಗಿದೆ. ಅವರಿಗೆ ಇದು ತಮ್ಮ ವಿಚಾರದೃಷ್ಟಿಗೆ ಸಮರ್ಪಕವಾದಂತೆ ಕಾಣು ವುದಿಲ್ಲ. ಇಂತಹ ಸಾಕಾರ ದೇವರು, ಇಂತಹ ಸೃಷ್ಟಿ, ಪ್ರತಿಯೊಂದು ಧರ್ಮ ದಲ್ಲಿಯೂ ಕಂಡುಬರುವ ಏಕದೇವ ವಾದ–ಈ ಭಾವನೆಗಳು ಹೆಚ್ಚು ಕಾಲ ತಮ್ಮ ಕಾಲ ಮೇಲೆ ನಿಂತುಕೊಳ್ಳಲಾರವು. ಭರತಖಂಡದಲ್ಲಿಯಾದರೋ ಇವು ಬೌದ್ಧರ ದೆಸೆಯಿಂದ ತಲೆ ಎತ್ತದೇ ಹೋದವು. ಹಿಂದಿನ ಕಾಲದಲ್ಲಿ ಬೌದ್ಧರು ಜಯವನ್ನು ಪಡೆದದ್ದು ಈ ಒಂದು ವಿಚಾರದ ಮೇಲೆ. ಪ್ರಕೃತಿಗೆ ಅನಂತಶಕ್ತಿ ಇದೆ, ಪ್ರಕೃತಿಯು ತನ್ನ ಕಾರ್ಯಗಳನ್ನೆಲ್ಲ ತಾನೆ ನಡೆಸಿಕೊಳ್ಳಬಲ್ಲದು ಎಂಬುದನ್ನು ಒಪ್ಪಿಕೊಂಡರೆ, ಪ್ರಕೃತಿಗಿಂತ ಮತ್ತೊಂದು ಬೇರೆ ಇದೆ ಎಂದು ಒಪ್ಪಿಕೊಳ್ಳುವುದು ಅನಾವಶ್ಯಕ ವೆಂಬುದನ್ನು ತೋರಿದರು. ಆತ್ಮನೂ ಕೂಡ ಅನಾವಶ್ಯಕ ಎಂದು ಬೌದ್ಧರು ಹೇಳಿದರು.

ಗುಣ ಮತ್ತು ಗುಣಿಗಳ ಚರ್ಚೆ ಬಹಳ ಹಳೆಯದು. ಕೆಲವು ವೇಳೆ ಹಳೆಯ ಮೂಢ ನಂಬಿಕೆಗಳು ಇಂದಿಗೂ ಕೂಡ ಇವೆ ಎಂಬುದನ್ನು ನೀವು ನೋಡಬಹುದು. ಮಧ್ಯಯುಗದಲ್ಲಿ ಮತ್ತು ಅದಾದ ಮೇಲೆಯೂ ಕೂಡ ಈ ಚರ್ಚೆ ಇದ್ದಿತೆಂದು ಹೇಳುವುದಕ್ಕೆ ವ್ಯಸನವಾಗುವುದು. ಉದ್ದ, ಅಗಲ, ದಪ್ಪವೆಂಬ ಗುಣಗಲು ನಾವು ಅಚೇತವೆಂದು ಕರೆಯುವ ಗುಣಿಗೆ ಸೇರಿದವೆ? ಗುಣಗಳು ಇರಲಿ, ಇಲ್ಲದಿರಲಿ, ಅದರ ಹಿಂದೆ ಗುಣಿ ಒಂದು ಇದೆಯೆ? ಇವು ಚರ್ಚೆಯ ಒಂದು ಮುಖ್ಯ ವಿಷಯವಾಗಿ ತ್ತೆಂಬುದನ್ನು ಓದಿರುವೆವು. ಅದಕ್ಕೆ ನಮ್ಮ ಬೌದ್ಧರು “ಅಂತಹ ಒಂದು ಗುಣಿ ಇದೆ ಎಂದು ಹೇಳುವುದಕ್ಕೆ ಸಾಕಾದಷ್ಟು ಆಧಾರವಿಲ್ಲ. ಗುಣಗಳೊಂದೇ ಇರುವುದು. ಇದರ ಆಚೆ ಮತ್ತೇನೊ ಇರುವುದು ಎನ್ನುವುದು ನಿಮಗೇನೂ ಕಾಣುವುದಿಲ್ಲ” ಎನ್ನುವರು. ಮುಕ್ಕಾಲುಪಾಲು ಆಧುನಿಕ ಆಯತಾವಾದಿಗಳ ನಿಲುವು ಇದು. ಗುಣ ಗುಣಿಗಳ ಚರ್ಚೆಯೆ ಮುಂದುವರಿದರೆ ವ್ಯಕ್ತ ಅವ್ಯಕ್ತಗಳ ಚರ್ಚೆಯ ರೂಪವನ್ನು ತಾಳುವುದು. ಇಲ್ಲಿ ಅನವರತವೂ ಚಲಿಸುತ್ತಿರುವ ವ್ಯಕ್ತ ಸೃಷ್ಟಿ ಇದೆ. ಇದರ ಹಿಂದೆ ಅವಿಕಾರವಾದ ಯಾವುದೋ ವಸ್ತುವಿದೆ. ಈ ವ್ಯಕ್ತ ಅವ್ಯಕ್ತವೆಂಬ ದ್ವಂದ್ವ ಅಸ್ತಿತ್ವವನ್ನು ಕೆಲವರು ನಿಜವೆಂದು ನಂಬುವರು. ಉಳಿದವರು ಇನ್ನೂ ಉತ್ತಮ ಕಾರಣಗಳನ್ನು ನೀಡಿ ನಿಮಗೆ ಎರಡನ್ನೂ ಒಪ್ಪಿಕೊಳ್ಳುವುದಕ್ಕೆ ಸಾಕಾದಷ್ಟು ಪ್ರಮಾಣವಿಲ್ಲ; ಏಕೆಂದರೆ ನಾವು ನೋಡುವುದು, ಅನುಭವಿಸುವುದು, ಆಲೋಚಿ ಸುವುದು ಎಲ್ಲಾ ವ್ಯಕ್ತವಾದುದು ಮಾತ್ರ; ವ್ಯಕ್ತದಾಚೆ ಮತ್ತೊಂದು ಏನೋ ಇದೆ ಎಂದು ಹೇಳಲು ನಿಮಗೆ ಅಧಿಕಾರವಿಲ್ಲ ಎನ್ನುವರು. ಇದಕ್ಕೆ ಉತ್ತರವೇ ಇಲ್ಲ. ನಿಮಗೆ ಸಿಕ್ಕುವ ಏಕಮಾತ್ರ ಉತ್ತರ ಅದ್ವೈತವೇದಾಂತ ಸಿದ್ಧಾಂತದಿಂದ ಮಾತ್ರ. ಇರುವುದು ಒಂದು ಮಾತ್ರ ಎನ್ನುವುದು ಸತ್ಯ. ಅದು ವ್ಯಕ್ತವಾಗಿರಬಹುದು ಅಥವಾ ಅವ್ಯಕ್ತವಾಗಿರಬಹುದು. ಎರಡು ವಸ್ತುಗಳಿವೆ, ಒಂದು ವಿಕಾರವಾದುದು–ಮತ್ತೊಂದು ಅವಿಕಾರವಾದುದು ಎನ್ನುವುದು ಸತ್ಯವಲ್ಲ. ಇರುವ ಒಂದೇ ಒಂದು ವಸ್ತು ವಿಕಾರವಾದಂತೆ ತೋರುವುದು. ಆದರೆ ಅದು ಅವಿಕಾರವಾದುದು. ನಾವು ದೇಹ, ಮನಸ್ಸು, ಆತ್ಮ, ಹಲವು ಇವೆ ಎಂದು ತಿಳಿಯಲು ಮೊದಲು ಮಾಡಿರುವೆವು. ಆದರೆ ನಿಜವಾಗಿಯೂ ಇರುವುದು ಒಂದೆ. ಆ ಒಂದೇ ನಾನಾ ರೂಪಗಳಲ್ಲಿ ಕಾಣಿಸು ತ್ತಿರುವುದು. ಅದ್ವೈತಿಗಳ ಪ್ರಖ್ಯಾತವಾದ ಉದಾಹರಣೆಯಾದ “ಹಾವಿನಂತೆ ಕಾಣುವ ಹಗ್ಗ” ವನ್ನು ತೆಗೆದುಕೊಳ್ಳಿ. ಕೆಲವರು ಕತ್ತಲೆಯಲ್ಲೋ ಅತವಾ ಇನ್ನು ಯಾವುದಾದರೂ ಕಾರಣದಿಂದಲೋ ಹಗ್ಗವನ್ನು ಹಾವೆಂದು ಭ್ರಮಿಸುವರು. ಆದರೆ ನಮಗೆ ತಿಳಿವಳಿಕೆ ಬಂದಾಗ ಹಾವು ಮಾಯವಾಗಿ ಹಗ್ಗವು ಕಾಣುವುದು. ಈ ಉದಾಹರಣೆಯಿಂದ ನಮಗೆ, ಹಾವು ಮನಸ್ಸಿನಲ್ಲಿದ್ದಾಗ ಹಗ್ಗ ಮಾಯವಾಗಿರುವುದು, ಹಗ್ಗ ಇದ್ದಾಗ ಹಾವು ಮಾಯವಾಗುವುದು ಕಾಣುತ್ತದೆ. ನಮ್ಮ ಸುತ್ತಲೂ ನಾವು ವ್ಯಕ್ತವನ್ನೇ ಯಾವಾಗ ನೋಡುವೋವೋ ಆಗ ಅವ್ಯಕ್ತ ಮಾಯವಾಗಿದೆ, ಅವಿಕಾರಿ ಯಾದ ಅವ್ಯಕ್ತವನ್ನು ಯಾವಾಗ ನೋಡುವೆವೋ, ಆಗ ವ್ಯಕ್ತ ಮಾಯವಾಗಿದೆ ಎಂದು ಸ್ವಭಾವತಃ ನಿರ್ಮಯವಾಗುವುದು. ವಸ್ತುಸತ್ತಾವಾದಿಗಳ ಮತ್ತು ಭಾವಸತ್ತಾವಾದಿಗಳ ಸ್ಥಿತಿಗಳನ್ನು ನಾವು ಈಗ ಚೆನ್ನಾಗಿ ತಿಳಿದುಕೊಳ್ಳುವೆವು. ವಸ್ತುಸತ್ತಾವಾದಿ ವ್ಯಕ್ತವನ್ನು ಮಾತ್ರ ನೋಡುವನು. ಭಾವಸತ್ತಾವಾದಿ ಅವ್ಯಕ್ತವನ್ನು ನೋಡುವನು. ಭಾವಸತ್ತಾ ವಾದಿಗೆ, ಸತ್ಯವಾಗಿಯೂ ನಿಷ್ಕಪಟವಾದ ಭಾವಸತ್ತಾವಾದಿಗೆ, ವಿಕಾರಗಳನ್ನು ಮೀರಬಲ್ಲ ಗ್ರಹಣಶಕ್ತಿ ಉಳ್ಳವನಿಗೆ, ಈ ವಿಕಾರವಾದ ಪ್ರಪಂಚ ಮಾಯವಾಗಿದೆ. ಇದೆಲ್ಲ ಭ್ರಮೆ, ನಿಜವಾಗಿಯೂ ಇರುವುದು ಒಂದೇ ಎಂದು ಹೇಳುವುದಕ್ಕೆ ಅವನಿಗೆ ಅಧಿಕಾರವಿದೆ. ವಸ್ತುಸತ್ತಾವಾದಿಯು ಪರಿವರ್ತನಶೀಲವಾದುದನ್ನು ನೋಡುತ್ತಾನೆ. ಅವನ ಪಾಲಿಗೆ ಅವಿಕಾರಿಯಾದುದು ಇಲ್ಲ. ಕಾಣುವ ಈ ಎಲ್ಲವೂ ಸತ್ಯ ಎಂದು ಹೇಳುವ ಅಧಿಕಾರ ಅವನಿಗೆ ಉಂಟು.

ಈ ತತ್ತ್ವಶಾಸ್ತ್ರದ ಪರಿಣಾಮವೇನಾಯಿತು? ಅದೇ, ಸಗುಣ ದೇವರ ಭಾವನೆ ಸಾಲದೆಂಬುದು. ಇದಕ್ಕಿಂತ ಮೇಲಿರುವ ನಿರ್ಗುಣ ಬ್ರಹ್ಮಕ್ಕೆ ಹೋಗಬೇಕು. ನಾವು ತೆಗೆದುಕೊಳ್ಳಬಹುದಾದ ಮುಂದಿನ ತರ್ಕಮಾರ್ಗವೇ ಇದು. ಇದರಿಂದ ಸಗುಣ ದೇವರ ಭಾವನೆ ನಾಶವಾಗುತ್ತದೆ ಎಂದು ಅಲ್ಲ. ಸಗುಣ ದೇವರಿಲ್ಲ ಎನ್ನುವುದಕ್ಕೆ ಸಾಕಾದಷ್ಟು ಪ್ರಮಾಣವನ್ನು ನಾವು ಒದಗಿಸುತ್ತೇವೆ ಎಂದೂ ಅಲ್ಲ. ಆದರೆ ಸಗುಣ ದೇವರನ್ನು ವಿವರಿಸಬೇಕಾದರೆ ನಾವು ನಿರ್ಗುಣಕ್ಕೆ ಹೋಗಲೇಬೇಕು. ಏಕೆಂದರೆ ನಿರ್ಗುಣವು ಸಗುಣಕ್ಕಿಂತ ಉನ್ನತ ಮಟ್ಟದ ಸಾಮಾನ್ಯೀಕರಣ. ನಿರ್ಗುಣವೊಂದೆ ಅನಂತವಾಗಬಲ್ಲದು, ಸಗುಣವೂ ಸಾಂತವಾದದ್ದು. ಹೀಗೆ ನಾವು ಸಗುಣವನ್ನು ಉಳಿಸಿಕೊಳ್ಳುತ್ತೇವೆ. ಅದನ್ನು ನಾಶ ಮಾಡುವುದಿಲ್ಲ. ನಿರ್ಗುಣದೇವರ ಭಾವನೆಗೆ ಬಂದರೆ ಸಗುಣ ದೇವರಿಗೆ ಉಳಿವಿಲ್ಲ; ನಿರ್ಗುಣ ಮಾನವನ ಭಾವನೆಗೆ ಬಂದರೆ, ಮಾನವನ ಪೃಥಕ್​ಭಾವ ನಾಶವಾಗುವುದು ಎಂಬ ಸಂಶಯ ಅನೇಕ ವೇಳೆ ಬರುವುದು. ಆದರೆ ವ್ಯಕ್ತಿತ್ವವನ್ನು ಉಳಿಸಿಕೊಳ್ಳುವುದೇ ಹೊರತು ಅದನ್ನು ಧ್ವಂಸ ಮಾಡುವುದಲ್ಲ ವೇದಾಂತದ ಗುರಿ. ವ್ಯಕ್ತಿಯ ವೈಶಿಷ್ಟ್ಯವನ್ನು ಸರ್ವವ್ಯಾಪಿಯಾದ ವಸ್ತುವಿನೊಂದಿಗೆ ಹೋಲಿಸಿದಲ್ಲದೆ, ಈ ವ್ಯಕ್ತಿತ್ವವೇ ನಿಜವಾಗಿಯೂ ಸರ್ವವ್ಯಾಪಿಯಾದುದೆಂದು ತೋರಿಸಿದಲ್ಲದೆ ಅದನ್ನು ನಾವು ಸಮರ್ತಿಸುವುದಕ್ಕೆ ಆಗುವುದಿಲ್ಲ. ಪ್ರಪಂಚ ದಲ್ಲಿರುವ ಸಕಲ ವಸ್ತುಗಳಿಗಿಂತಲೂ ವ್ಯಕ್ತಿಯು ಪ್ರತ್ಯೇಕವೆಂದು ಭಾವಿಸಿದರೆ ಅದು ಒಂದು ನಿಮಿಷವೂ ಕೂಡ ಇರಲಾರದು. ಅಂತಹ ಸ್ಥಿತಿ ಎಂದಿಗೂ ಇರಲಿಲ್ಲ.

ಎರಡನೆಯದಾಗಿ ವಿವರಣೆಯು ವಸ್ತುವಿನ ಸ್ವಭಾವದಿಂದಲೇ ಬರಬೇಕು ಎಂಬ ಎರಡನೆಯ ನಿಯಮವನ್ನು ಪ್ರಯೋಗಿಸುವುದರ ಮೂಲಕ, ಮತ್ತೂ ಧೈರ್ಯವಾದ, ತಿಳಿದುಕೊಳ್ಳಲು ಅತಿ ಗಹನವಾದ ಭಾಗಕ್ಕೆ ಬರುವೆವು. ಅದರ ಸಾರಾಂಶವಿಷ್ಟೆ: ಆ ನಿರಾಕಾರ ಸತ್ಯವು, ನಮ್ಮ ಅತ್ಯುತ್ತಮವಾದ ಸರ್ವಸಾಮಾನ್ಯತೆಯು ನಮ್ಮಲ್ಲಿಯೇ ಇರುವುದು ಮತ್ತು ನಾವೇ ಅದು. “ಓ, ಶ್ವೇತಕೇತು, ನೀನೇ ಅದು,” ನೀನೇ ಆ ನಿರಾಕಾರವಾದ ಸತ್ಯ. ಯಾವ ದೇವರಿಗೋಸ್ಕರ ನೀನು ಪ್ರಪಂಚದಲ್ಲಿ ಹುಡುಕು ತ್ತಿದ್ದೆಯೋ, ಅದೇ ಯಾವಾಗಲೂ ನೀನಾಗಿದೆ. ನೀನು ಎಂದರೆ ವ್ಯಕ್ತಿ ದೃಷ್ಟಿಯಿಂದ ಅಲ್ಲ, ವ್ಯಕ್ತಿತ್ವ ಭಾವರಹಿತ ದೃಷ್ಟಿಯಿಂದ. ನಾವು ಈಗ ನೋಡುತ್ತಿರುವ ವ್ಯಕ್ತ ಮಾನವನು, ವ್ಯಕ್ತೀಕರಣಗೊಂಡವನು. ಆದರೆ ಸತ್ಯವೇನೆಂದರೆ ಅವನು ವ್ಯಕ್ತಿತ್ವ ರಹಿತ ನಾದವನು. ನಾವು ವ್ಯಕ್ತಿತ್ವವನ್ನು ತಿಳಿದುಕೊಳ್ಳಬೇಕಾದರೆ ಅದನ್ನು ವ್ಯಕ್ತಿತ್ವಭಾವ ರಹಿತವಾದುದರೊಂದಿಗೆ ಹೋಲಿಸಬೇಕು, ವಿಶೇಷವನ್ನು ನಾವು ಸರ್ವಸಾಮಾನ್ಯಕ್ಕೆ ಹೋಲಿಸಬೇಕು. ಆ ಅವ್ಯಕ್ತವೇ ಸತ್ಯ. ಅದೇ ಮಾನವನ ಆತ್ಮ.

ಇವುಗಳಿಗೆ ಸಂಬಂಧಪಟ್ಟ ಅನೇಕ ಪ್ರಶ್ನೆಗಳಿರುತ್ತವೆ, ಮುಂದುವರಿದಂತೆಲ್ಲ ಅವುಗಳಿಗೆ ಸಮಾಧಾನ ಹೇಳಲು ಪ್ರಯತ್ನಿಸುತ್ತೇನೆ. ಅನೇಕ ತೊಂದರೆಗಳು ಮೇಲೇ ಳುತ್ತವೆ. ಆದರೆ ಮೊದಲು ಅದ್ವೈತದ ದೃಷ್ಟಿಯಿಂದ ಚೆನ್ನಾಗಿ ತಿಳಿದುಕೊಳ್ಳೋಣ. ವ್ಯಕ್ತಿದೃಷ್ಟಿಯಿಂದ ನಾವು ಬೇರೆಯಾಗಿರುವಂತೆ ಕಾಣುವೆವು. ಆದರೆ ಸತ್ಯವು ಒಂದು. ಆ ಏಕದಿಂದ ನಾವು ಬೇರೆಯಾಗಿರುವೆವೆಂದು ಎಷ್ಟು ಕಡಿಮೆ ಆಲೋಚಿಸಿದರೆ ಅಷ್ಟು ಮೇಲು. ಪೂರ್ಣದಿಂದ ನಾವು ಬೇರೆ ಎಂದು ಎಷ್ಟು ಹೆಚ್ಚು ಆಲೋಚಿಸುವೆವೋ ಅಷ್ಟು ದುಃಖಕ್ಕೆ ಈಡಾಗುವೆವು. ಈ ಅದ್ವೈತ ತತ್ತ್ವದ ಮೂಲಕ ನಾವು ನೀತಿಯ ತಳಹದಿಗೆ ಬರುತ್ತೇವೆ. ಮತ್ತೆಲ್ಲಿಂದಲೂ ಯಾವ ನೀತಿಯೂ ಸಿಕ್ಕುವುದಿಲ್ಲವೆಂದು ನಾನು ಧೈರ್ಯವಾಗಿ ಹೇಳಬಲ್ಲೆ. ಅತ್ಯಂತ ಹಳೆಯ ನೀತಿಯ ಭಾವನೆ ಯಾವುದಾ ಗಿತ್ತು ಎಂದರೆ, ನೀತಿಯು ಯಾರೋ ಒಬ್ಬರ ಅಥವಾ ಹಲವರ ಇಚ್ಛೆ ಎಂಬುದು. ಆದರೆ ಈಗಿನ ಕಾಲದಲ್ಲಿ ಅದನ್ನು ಒಪ್ಪುವವರು ಬಹಳ ಕಡಿಮೆ, ಏಕೆಂದರೆ ಅದು ಆಂಶಿಕವಾದ ಸಾಮಾನ್ಯೀಕರಣ. ಇವನ್ನು ಮತ್ತು ಅವನ್ನು ಮಾಡಬಾರದು, ಏಕೆಂದರೆ ವೇದ ಹಾಗೆ ಹೇಳುತ್ತದೆ ಎಂದು ಹಿಂದು ಹೇಳುತ್ತಾನೆ. ಆದರೆ ಕ್ರೈಸ್ತನು ವೇದದ ಪ್ರಮಾಣವನ್ನು ಒಪ್ಪುವುದಿಲ್ಲ. ನೀನು ಇವನ್ನು ಅವನ್ನು ಮಾಡಬಾರದು, ಬೈಬಲ್​ ಹಾಗೆ ಹೇಳುತ್ತದೆ ಎಂದು ಕ್ರೈಸ್ತನು ಹೇಳುತ್ತಾನೆ. ಯಾರಿಗೆ ಬೈಬಲಿನಲ್ಲಿ ನಂಬಿಕೆ ಇಲ್ಲವೋ ಅವರ ಮೇಲೆ ಇದು ತನ್ನ ಅಧಿಕಾರವನ್ನು ಚಲಾಯಿಸಲಾರದು. ಈ ಎಲ್ಲಾ ವಿಭಾಗಗಳನ್ನು ಕೂಡ ಆವರಿಸುವಷ್ಟು ವಿಸ್ತಾರವಾದ ಸಿದ್ಧಾಂತವು ನಮಗೆ ಬೇಕಾಗಿದೆ. ಹೇಗೆ ಸಾಕಾರ ಸೃಷ್ಟಿಕರ್ತನನ್ನು ನಂಬುವುದಕ್ಕೆ ಲಕ್ಷಾಂತರ ಜನರು ಸಿದ್ಧರಾಗಿದ್ದರೋ, ಅದರಂತೆಯೇ ಪ್ರಪಂಚದಲ್ಲಿ ಸಾವಿರಾರು ಜನ ಅತ್ಯಂತ ಬುದ್ಧಿವಂತರು ಅಂತಹ ಸಾಕಾರ ಸೃಷ್ಟಿಕರ್ತನು ನಮಗೆ ತೃಪ್ತಿಕರವಾಗಿಲ್ಲವೆಂದು ಭಾವಿಸಿದವರಿದ್ದರು. ಅದಕ್ಕಿಂತ ಉತ್ತಮವಾದುದನ್ನು ಆಶಿಸಿದರು. ಎಲ್ಲಿ ಇಂತಹ ವಿಚಾರಪರರನ್ನು ಸೇರಿಸಿಕೊಳ್ಳುವಷ್ಟು ವಿಸ್ತಾರವಾಗಿ ಧರ್ಮವಿರಲಿಲ್ಲವೋ ಅಲ್ಲಿ ಅದರ ಪರಿಣಾಮವಾಗಿ ಸಮಾಜದ ಅತಿ ಪ್ರತಿಭಾವಂತರು ಧರ್ಮಕ್ಕೆ ಬಾಹಿರರಾಗಿ ಉಳಿದರು. ಇಂದಿನ ಕಾಲದಲ್ಲಿ ಅದರಲ್ಲೂ ಯೂರೋಪಿನಲ್ಲಿ ಇದು ಅತ್ಯಂತ ಸ್ಪಷ್ಟವಾಗಿ ಕಂಡುಬರುತ್ತದೆ.

ಇಂತಹ ವಿಚಾರಪರರನ್ನು ಒಳಗೊಳ್ಳಬೇಕಾದರೆ ಧರ್ಮವು ಸಾಕಾದಷ್ಟು ವಿಶಾಲವಾಗಿರಬೇಕು. ಧರ್ಮವು ಸಾರುವ ಎಲ್ಲವನ್ನೂ ವಿಚಾರ ದೃಷ್ಟಿಯಿಂದ ಪರೀಕ್ಷಿಸಬೇಕು. ಯುಕ್ತಿದೃಷ್ಟಿಗೆ ತಾವು ಬದ್ಧವಲ್ಲವೆಂದು ಧರ್ಮಗಳು ಏತಕ್ಕೆ ಹೇಳಿಕೊಳ್ಳಬೇಕು? ಅದು ಯಾರಿಗೂ ತಿಳಿಯದು. ಯುಕ್ತಿಪ್ರಮಾಣವನ್ನು ನಾವು ತೆಗೆದುಕೊಳ್ಳದೇ ಇದ್ದರೆ, ಧಾರ್ಮಿಕ ವಿಷಯದಲ್ಲಿಯ ಕೂಡ ಸರಿಯಾದ ನಿರ್ಣಯವಿರಲಾರದು. ಒಂದು ಧರ್ಮವು ಅತಿ ತಿರಸ್ಕರಣೀಯವಾದುದನ್ನು ಆಜ್ಞಾಪಿಸಬಹುದು. ಉದಾಹರಣೆಗೆ ಇಸ್ಲಾಂ ಧರ್ಮವು ತಮ್ಮ ಧರ್ಮಕ್ಕೆ ಸೇರದ ವರೆಲ್ಲರನ್ನೂ ಕೊಲ್ಲುವುದನ್ನು ಒಪ್ಪುತ್ತದೆ. “ನಾಸ್ತಿಕರು ಮಹಮ್ಮದೀಯರಾಗದೆ ಇದ್ದರೆ ಅವರನ್ನು ಕೊಲ್ಲಿ” ಎಂದು ಖುರಾನಿನಲ್ಲೆ ಸ್ಪಷ್ಟವಾಗಿ ಹೇಳಿದೆ. ಅವರನ್ನು ಕತ್ತಿಗೆ ಬಲಿ ಕೊಡಬೇಕು, ಬೆಂಕಿಗೆ ಹಾಕಬೇಕು. ಮಹಮ್ಮದೀಯರಿಗೆ ಇದು ಸರಿಯಲ್ಲವೆಂದರೆ ಅವರು, “ಅದು ನಿಮಗೆ ಹೇಗೆ ಗೊತ್ತು? ಅದು ಒಳ್ಳೆಯದಲ್ಲವೆಂದು ನಿಮಗೆ ಹೇಗೆ ಗೊತ್ತು? ನಮ್ಮ ಗ್ರಂಥ ಅದು ಸರಿ ಎಂದು ಸಾರುತ್ತದೆ” ಎನ್ನುತ್ತಾರೆ. ನಿಮ್ಮ ಗ್ರಂಥ (ಬೈಬಲ್​) ಖುರಾನಿಗಿಂತ ಹಳೆಯದು ಎಂದರೆ, ಬೌದ್ಧರು ಬಂದು ತಮ್ಮ ಗ್ರಂಥ ಅದಕ್ಕಿಂತಲೂ ಪುರಾತನವಾದುದೆಂದು ಹೇಳುತ್ತಾರೆ. ಅನಂತರ ಹಿಂದು ಬಂದು ತನ್ನ ಗ್ರಂಥವೇ ಎಲ್ಲದಕ್ಕಿಂತಲೂ ಅತಿ ಪುರಾತನ ಗ್ರಂಥವೆನ್ನುತ್ತಾನೆ. ಆದುದರಿಂದ ಗ್ರಂಥವನ್ನು ಆಧಾರವಾಗಿ ಇಟ್ಟು ಕೊಳ್ಳುವುದರಿಂದ ಪ್ರಯೋಜನವಿಲ್ಲ. ನೀವು ಒಂದನ್ನು ಮತ್ತೊಂದರೊಂದಿಗೆ ಹೋಲಿಸಲು ಇರುವ ಪ್ರಮಾಣ ಯಾವುದು? ನೀವು (ಎಂದರೆ ಕ್ರೈಸ್ತರು) ಕ್ರಿಸ್ತನು ಗುಡ್ಡದ ಮೇಲೆ ಮಾಡಿದ ಉಪದೇಶವನ್ನು ನೋಡಿ ಎನ್ನಬಹುದು. ಮಹಮ್ಮದೀ ಯರು ಖುರಾನಿನಲ್ಲಿರುವ ನೀತಿಯನ್ನು ನೋಡಿ ಎಂದು ಉತ್ತರ ಕೊಡುವರು. ಎರಡರಲ್ಲಿ ಯಾವುದು ಸರಿ ಎಂದು ನಿಶ್ಚಯಿಸುವ ಮಧ್ಯಸ್ಥಗಾರನಾರು ಎಂದು ಮಹಮ್ಮದೀಯರು ಕೇಳುತ್ತಾರೆ. ಅವರಿಬ್ಬರಿಗೂ ಇರುವ ವ್ಯಾಜ್ಯದಲ್ಲಿ ಖುರಾ ನಾಗಲೀ ಬೈಬಲ್​ ಆಗಲೀ ಮಧ್ಯಸ್ಥಿಕೆಗಾರನ ಸ್ಥಾನವನ್ನು ವಹಿಸಲಾರದು. ಒಂದು ಸ್ವತಂತ್ರವಾದ ಪ್ರಮಾಣ ಅಲ್ಲಿರಬೇಕು. ಅದು ಯಾವ ಗ್ರಂಥವೂ ಆಗಲಾರದು. ಸರ್ವರಿಗೂ ಸಾಮಾನ್ಯವಾದುದು ಯಾವುದಾದರೂ ಆಗಬೇಕು. ಯುಕ್ತಿಗಿಂತ ಸರ್ವಸಾಮಾನ್ಯವಾದುದು ಯಾವುದಿದೆ? ಯುಕ್ತಿ ಅಷ್ಟೇನು ಬಲವಾದುದಲ್ಲ, ಸತ್ಯವನ್ನರಿಯಲು ನಮಗೆ ಅದು ಯಾವಾಗಲೂ ಸಹಾಯ ಮಾಡುವುದಿಲ್ಲ, ಅನೇಕ ವೇಳೆ ಇದು ತಪ್ಪು ಮಾಡುತ್ತದೆ ಎನ್ನುತ್ತಾರೆ. ಆದಕಾರಣ ನಿರ್ಣಯವೇನೆಂದರೆ ಚರ್ಚಿನ ಅಧಿಕಾರವನ್ನು ನಂಬಲೇಬೇಕು! ಹೀಗೆಂದು ಒಬ್ಬ ರೋಮನ್​ ಕೆಥೊಲಿಕ್​ ನನಗೆ ಹೇಳಿದನು. ಆದರೆ ಅವನ ಯುಕ್ತಿ ನನಗೆ ಸರಿ ತೋರಲಿಲ್ಲ. ಯುಕ್ತಿಯೇ ಅಷ್ಟು ನಿರ್ಬಲವಾಗಿದ್ದರೆ ಅದಕ್ಕಿಂತ ಹೆಚ್ಚು ನಿರ್ಬಲರು ಪಾದ್ರಿಗಳ ಗುಂಪು ಎಂದು ನಾನು ಹೇಳುತ್ತೇನೆ. ಅವರ ಅಭಿಪ್ರಾಯವನ್ನು ನಾನು ಸ್ವೀಕರಿಸುವುದಿಲ್ಲ. ನಾನು ನನ್ನ ಯುಕ್ತಿಯನ್ನು ತೊರೆಯುವುದಿಲ್ಲ. ಏಕೆಂದರೆ ಅದು ಎಷ್ಟೇ ನಿರ್ಬಲವಾಗಿದ್ದರೂ, ಅದರ ಮೂಲಕ ಸತ್ಯವನ್ನು ತಿಳಿಯುವುದಕ್ಕೆ ಸ್ವಲ್ಪ ಅವಕಾಶವಿದೆ. ಆದರೆ ಬೇರೆಯ ಮಾರ್ಗದಲ್ಲಿಯಾದರೋ ಅಂತಹ ಒಂದು ಭರವಸೆಯೇ ಇಲ್ಲ.

ಆದಕಾರಣ ನಾವು ಯುಕ್ತಿಯನ್ನು ಅನುಸರಿಸಬೇಕು ಮತ್ತು ಅದನ್ನು ಅನುಸರಿಸಿ ಯಾವ ನಿರ್ಧಾರಕ್ಕೂ ಬಾರದವರಿಗೆ ಸಹಾನುಭೂತಿಯನ್ನು ತೋರಬೇಕು. ಯುಕ್ತಿಯನ್ನು ಅನುಸರಿಸಿ ನಾಸ್ತಿಕರಾಗುವುದು, ಕಣ್ಣು ಮುಚ್ಚಿಕೊಂಡು, ಇತರರು ಹೇಳಿದ್ದರ ಆಧಾರದ ಮೇಲೆ ಇಪ್ಪುತ್ತು ಕೋಟಿ ದೇವತೆಗಳಲ್ಲಿ ನಂಬಿಕೆಯಿಡು ವುದಕ್ಕಿಂತಲೂ ಎಷ್ಟೋ ಮೇಲಾದದ್ದು. ನಮಗೆ ಬೇಕಾಗಿರುವುದು ಪ್ರಗತಿ, ಬೆಳ ವಣಿಗೆ, ಸಾಕ್ಷಾತ್ಕಾರ. ಯಾವ ಸಿದ್ಧಾತಗಳೂ ಎಂದಿಗೂ ಮಾನವನನ್ನು ಉತ್ತಮ ನನ್ನಾಗಿ ಮಾಡಿಲ್ಲ. ಎಷ್ಟು ಗ್ರಂಥರಾಶಿಯಾದರೂ ನಮ್ಮನ್ನು ಪವಿತ್ರರನ್ನಾಗಿ ಮಾಡಲು ಸಹಾಯ ಮಾಡಲಾರದು. ಸತ್ಯ ಸಾಕ್ಷಾತ್ಕಾರ ಒಂದರಲ್ಲಿಯೇ ಶಕ್ತಿ ಇರುವುದು. ಅದಕ್ಕೆ ಬೇಕಾದ ಶಕ್ತಿ ನಮ್ಮಲ್ಲಿಯೇ ಇರುವುದು. ಅದು ಬರುವುದು ವಿಚಾರದಿಂದ. ಮಾನವರು ವಿಚಾರಪರರಾಗಲಿ. ಒಂದು ಮಣ್ಣಿನ ಮುದ್ದೆ ಎಂದಿಗೂ ವಿಚಾರ ಮಾಡುವುದಿಲ್ಲ. ಆದರೆ ಅದು ಒಂದು ಮಣ್ಣಿನ ಮುದ್ದೆಯಾಗಿಯೇ ಉಳಿಯು ವುದು. ಮಾನವನು ಆಲೋಚನಾಜೀವಿಯಾಗಿರುವುದೇ ಆತನ ಹಿರಿಮೆ. ಈ ವಿಷಯದಲ್ಲಿ ಮಾತ್ರ ಅವನು ಮೃಗಗಳಿಗಿಂತ ಬೇರೆಯಾಗಿರುವನು. ನನಗೆ ವಿಚಾರ ದಲ್ಲಿ ನಂಬಿಕೆಯುಂಟು ಮತ್ತು ವಿಚಾರ ಪಥವನ್ನೇ ಅನುಸರಿಸುತ್ತೇನೆ. ಅಧಿಕಾರ ವಾದದ \enginline{(anthority)} ದುರುಪಯೋಗವನ್ನು ನಾನು ಬೇಕಾದಷ್ಟು ನೋಡಿರುವೆನು. ಏಕೆಂದರೆ ಅಧಿಕಾರವಾದವು ತನ್ನ ಪರಾಕಾಷ್ಠೆಯನ್ನು ಮುಟ್ಟಿದ ದೇಶದಲ್ಲಿ ನಾನು ಜನ್ಮ ತಾಳಿದವನು.

ಸೃಷ್ಟಿಯು ವೇದಗಳಿಂದ ಬಂದಿದೆ ಎಂದು ಹಿಂದುಗಳು ನಂಬುತ್ತಾರೆ. ಒಂದು ಹಸುವಿದೆ ಎಂದು ನಿಮಗೆ ಹೇಗೆ ಗೊತ್ತು? ಏಕೆಂದರೆ ವೇದದಲ್ಲಿ ಹಸುವೆಂಬ ಪದವಿದೆ. ಹೊರಗೆ ಒಬ್ಬ ಮನುಷ್ಯನು ಇರುವನು ಎಂಬುದು ಹೇಗೆ ಗೊತ್ತು? ಏಕೆಂದರೆ ಮನುಷ್ಯನೆಂಬ ಪದ ವೇದದಲ್ಲಿದೆ, ಅದು ಅಲ್ಲಿ ಇಲ್ಲದೇ ಇದ್ದರೆ ಹೊರಗೆ ಮನುಷ್ಯರೇ ಇರುತ್ತಿರಲಿಲ್ಲ. ಅವರು ಹೇಳುವುದೇ ಇದು. ಅಷ್ಟು ಬಲವಾಗಿ ವೇದ ಪ್ರಾಮಾಣ್ಯಕ್ಕೆ ಅವರು ಅಂಟಿಕೊಳ್ಳುತ್ತಾರೆ! ನಾನು ಅಧ್ಯಯನ ಮಾಡಿದಂತೆ ಅವರು ವೇದಗಳನ್ನು ಅಧ್ಯಯನ ಮಾಡಿಲ್ಲ. ಅತ್ಯಂತ ಪ್ರತಿಭಾಶಾಲಿಗಳಲ್ಲಿ ಕೆಲವರು ಅದನ್ನು ತೆಗೆದುಕೊಂಡು, ಅದರ ಸುತ್ತಲೂ ಅತ್ಯಂತ ಸೂಕ್ಷ್ಮವಾದ ಯುಕ್ತಿಪೂರಿತ ಸಿದ್ಧಾಂತ ಗಳನ್ನು ನೇಯ್ದಿರುವರು. ಅದನ್ನು ಅವರು ಕೂಲಂಕಷವಾಗಿ ವಿಚಾರಿಸಿರುವರು. ಒಂದು ತತ್ತ್ವವು ಸಂಪ್ರದಾಯ ಮಾತ್ರವಲ್ಲ, ಸಾವಿರಾರು ವರುಷಗಳಿಂದಲೂ ಸಹ ಸ್ರಾರು ಪ್ರತಿಭಾವಂತರು ಈ ತತ್ತ್ವ ನಿರ್ಣಯಕ್ಕೋಸುಗವಾಗಿ ತಮ್ಮ ಜೀವನವನ್ನುರ್ ಪಿಸಿರುವರು. ಪ್ರಮಾಣದ ಶಕ್ತಿ ಹೀಗಿರುವುದು. ಅದರಲ್ಲಿರುವ ಅಪಾಯಗಳೆಷ್ಟು! ಅದು ಮಾನವನ ಬೆಳೆವಣಿಗೆಗೆ ಆತಂಕವನ್ನು ತರುವುದು. ನಾವು ಬೆಳೆಯಬೇಕೆಂಬು ದನ್ನು ಎಂದಿಗೂ ಮರೆಯಕೂಡದು. ಎಲ್ಲಾ ಸಾಪೇಕ್ಷ ಸತ್ಯದ ವಿಷಯದಲ್ಲಿಯೂ ಕೂಡ ಸತ್ಯಕ್ಕಿಂತ ಹೆಚ್ಚಾಗಿ ನಮಗೆ ಬೇಕಾಗಿರುವುದು ಅನುಷ್ಠಾನ. ಅದೇ ನಮ್ಮ ಜೀವನ.

ಅದ್ವೈತ ಸಿದ್ಧಾಂತಕ್ಕೆ ಈ ಒಂದು ಗುಣವಿದೆ: ನಾವು ವಿಚಾರಿಸಬಹುದಾದ ಧಾರ್ಮಿಕ ತತ್ತ್ವಗಳಲ್ಲೆಲ್ಲ ಇದು ಬಹಳ ಯುಕ್ತಿಪೂರಿತವಾದುದು. ಉಳಿದ ಬೇರೆ ಸಿದ್ಧಾಂತಗಳು, ಅಸಂಪೂರ್ಣವಾದ, ಅಲ್ಪವಾದ, ಸಾಕಾರ ದೇವರ ಭಾವಗಳು ಎಲ್ಲವೂ ಯುಕ್ತಿಪೂರಿತವಾಗಿಲ್ಲ. ಆದರೂ ಅದ್ವೈತದಲ್ಲಿ ಈ ಒಂದು ಮಹತ್ವವಿದೆ. ಅದೇ ಅಸಂಪೂರ್ಣವಾದ ದೇವರ ಭಾವಗಳೆಲ್ಲವನ್ನೂ ಅವು ಮಾನವನಿಗೆ ಉಪಯೋಗಕಾರಿ ಎಂದು ತಿಳಿದು ತನ್ನಲ್ಲಿ ಸೇರಿಸಿಕೊಳ್ಳುವುದು. ಕೆಲವರು ಈ ಸಾಕಾರ ವಿವರಣೆಯು ಅತಾರ್ಕಿಕವಾದುದೆನ್ನಬಹುದು. ಆದರೆ ಅದು ಸಮಾಧಾನವನ್ನು ನೀಡುತ್ತದೆ. ಜನರಿಗೆ ಸಮಾಧಾನದಾಯಕವಾದ ಒಂದು ಧರ್ಮ ಬೇಕಾಗಿದೆ. ಅವರಿಗೆ ಇದು ಅತ್ಯಾವಶ್ಯಕವೆನ್ನುವುದು ನಮಗೆ ತಿಳಿಯುತ್ತದೆ. ಸತ್ಯದ ಪೂರ್ಣ ಜ್ಯೋತಿಯನ್ನು ಪಡೆಯುವವರು ಅಲ್ಪಮಂದಿ. ಅದರಂತೆ ಬಾಳುವವರು ಮತ್ತೂ ಅಲ್ಪಮಂದಿ. ಆದಕಾರಣ ಇಂತಹ ಸಮಾಧಾನಕರ ಧರ್ಮವಿರುವುದು ಅತ್ಯಾವಶ್ಯಕ. ಅನೇಕ ಜೀವಿಗಳನ್ನು ಇದು ಉತ್ತಮರನ್ನಾಗಿ ಮಾಡುತ್ತದೆ. ಯಾರ ಆಲೋಚನಾಕ್ಷೇತ್ರ ಅತಿ ಕಿರಿದಾಗಿರುವುದೋ, ಅದನ್ನು ಕಟ್ಟಲು ಅಲ್ಪ ಸಾಮಗ್ರಿ ಸಾಕೋ, ಅಂತಹ ಸಂಕುಚಿತ ಬುದ್ಧಿವಂತರು ಆಲೋಚನಾ ಕ್ಷೇತ್ರದಲ್ಲಿ ಅಂತರಿಕ್ಷಕ್ಕೆ ಹಾರುವ ಸಾಹಸ ವನ್ನು ಮಾಡಲಾರರು. ಅವರ ಭಾವನೆಗಳು ಸಣ್ಣ ಸಣ್ಣ ದೇವತೆಗಳದ್ದು ಆಗಲಿ, ವಿಗ್ರಹಾದಿ ಬಾಹ್ಯ ಚಿಹ್ನೆಗಳದ್ದು ಆಗಲಿ, ಎಲ್ಲವೂ ಬಹಳ ಒಳ್ಳೆಯದು ಮತ್ತು ಅವರಿಗೆ ಬಹಳ ಸಹಾಯಕವಾಗಿರುವುವು. ಆದರೆ ನೀವು ನಿರ್ಗುಣ ಸತ್ಯವನ್ನು ಚೆನ್ನಾಗಿ ತಿಳಿದುಕೊಳ್ಳಬೇಕು. ಏಕೆಂದರೆ ಅದರಿಂದ ಮತ್ತು ಅದರ ಮೂಲಕ ಮಾತ್ರ ಉಳಿದವುಗಳನ್ನು ವಿವರಿಸಬಹುದು. ಉದಾಹರಣೆಗೆ ಸಗುಣ ದೇವರನ್ನು ತೆಗೆದು ಕೊಳ್ಳಿ. ನಿರ್ಗುಣ ದೇವರನ್ನು ನಂಬುವವರು, ಉದಾಹರಣೆಗೆ ಜಾನ್​ ಸ್ಟುಯರ್ಟ್​ ಮಿಲ್ಲನು, ಸಗುಣ ದೇವರು ಅಸಾಧ್ಯ ಮತ್ತು ಅದನ್ನು ನಾವು ಪ್ರಮಾಣೀಕರಿಸುವುದು ಅಸಾಧ್ಯ ಎಂದು ಹೇಳಬಹುದು. ಸಗುಣ ದೇವರನ್ನು ಪ್ರಮಾಣೀಕರಿಸುವುದಕ್ಕೆ ಆಗುವುದಿಲ್ಲವೆಂದು ನಾನು ಒಪ್ಪುತ್ತೇನೆ. ಆದರೆ ಮಾನವ ಬುದ್ಧಿಯ ಮೂಲಕ ಸಿಕ್ಕುವ ನಿರ್ಗುಣ ಸತ್ಯದ ಅತ್ಯುತ್ತಮ ಪ್ರತಿಬಿಂಬವೇ ಸಗುಣ. ಜಗತ್ತು ನಿರ್ಗುಣ ಪ್ರರ ಬ್ರಹ್ಮನ ಭಿನ್ನ ಅಭಿವ್ಯಕ್ತಿಗಳಲ್ಲದೆ ಮತ್ತೇನು? ಅದು ನಮ್ಮೆದುರಿಗೆ ಇರುವ ಒಂದು ಪುಸ್ತಕ ದಂತೆ. ಪ್ರತಿಯೊಬ್ಬರೂ ಅದನ್ನು ಓದುವುದಕ್ಕೆ ತಮ್ಮ ಬುದ್ಧಿಯನ್ನು ತಂದಿರುವರು. ಪ್ರತಿಯೊಬ್ಬರೂ ಅದನ್ನು ತಾವೇ ಓದಿಕೊಳ್ಳಬೇಕು. ಪ್ರತಿಯೊಬ್ಬರ ತಿಳಿವಳಿಕೆ ಯಲ್ಲಿಯೂ ಸರ್ವಸಾಮಾನ್ಯವಾದ ಕೆಲವು ವಿಷಯಗಳಿವೆ. ಆದಕಾರಣ ಕೆಲವು ವಿಷಯಗಳು ಮಾನವನ ಬುದ್ಧಿಗೆ ಒಂದೇ ಸಮನಾಗಿರುವಂತೆ ತೋರುವುದು. ನಾನು ಮತ್ತು ನೀವು ಇಬ್ಬರೂ ಒಂದು ಕುರ್ಚಿಯನ್ನು ನೋಡುವುದು ನಮ್ಮಿಬ್ಬರ ಮನಸ್ಸಿಗೂ ಒಂದು ಸಾಮಾನ್ಯವಾದ ವಸ್ತುವಿದೆ ಎಂಬುದಕ್ಕೆ ಪ್ರಮಾಣ. ಬೇರೊಂದು ಇಂದ್ರಿಯ ದಿಂದ ಕೂಡಿದ ಮತ್ತೊಬ್ಬನು ಬಂದ ಎಂದು ಇಟ್ಟುಕೊಳ್ಳೋಣ: ಅವನಿಗೆ ಕುರ್ಚಿ ಕಾಣಿಸುವುದೇ ಇಲ್ಲ. ಆದರೆ ಒಂದೇ ಸಮನಾದ ಉಳಿದ ಎಲ್ಲಾ ಮನಸ್ಸುಗಳೂ ಒಂದೇ ವಸ್ತುವನ್ನು ನೋಡುವುವು. ಆದಕಾರಣ ಈ ಜಗತ್ತೇ ಏಕವಾದ ಬ್ರಹ್ಮ. ಅದು ಅವಿಕಾರಿಯಾದದ್ದು, ಅವ್ಯಕ್ತವಾದದ್ದು, ವ್ಯಕ್ತವಾದುವೆಲ್ಲ ಅದರ ಭಿನ್ನ ದೃಷ್ಟಿಗಳು. ವ್ಯಕ್ತವಾದ ಎಲ್ಲವೂ ಮಿತವಾದುದೆಂದೂ ನಮಗೆ ಮೊದಲೇ ತಿಳಿದಿದೆ. ನಾವು ನೋಡುವ, ಆಲೋಚಿಸುವ ಮತ್ತು ಅನುಭವಿಸುವ ಎಲ್ಲವೂ ಸಾಕಾರ ದೇವರೂ ಕೂಡ, ಸಾಂತ ಅಭಿವ್ಯಕ್ತಿ. ಕಾರ್ಯಕಾರಣ ಸಂಬಂಧದ ಭಾವಿರುವುದೇ ವ್ಯಕ್ತ ಜಗತ್ತಿನಲ್ಲಿ. ಸಹಜವಾಗಿಯೇ ಸೃಷ್ಟಿಕರ್ತನಾದ ದೇವರನ್ನು ಸಾಂತವೆಂದೇ ನಾವು ಆಲೋಚಿಸಬೇಕಾಗುತ್ತದೆ. ಆದರೂ ಕೂಡ ಅವನು ನಿರಾಕಾರವಾದ ಅನಂತವಾದ ಬ್ರಹ್ಮವೇ. ಈ ಜಗತ್ತು ಕೂಡ ನಮ್ಮ ಬುದ್ಧಿಯಿಂದ ಗ್ರಹಿಸಲ್ಪಟ್ಟ ಆ ನಿರ್ಗುಣ ಬ್ರಹ್ಮವೇ ಆಗಿದೆ. ಪ್ರಪಂಚದಲ್ಲಿ ಯಾವುದು ಸತ್ಯವೋ ಅದೇ ಆ ನಿರಾಕಾರ ಬ್ರಹ್ಮ. ಆಕಾರ ಮತ್ತು ಕಲ್ಪನೆಗಳೆಲ್ಲ ನಮ್ಮ ಬುದ್ದಿಯಿಂದ ಅವಕ್ಕೆ ಬಂದಿವೆ. ಈ ಮೇಜಿನಲ್ಲಿ ಯಾವುದು ಸತ್ಯವಾಗಿದೆಯೋ ಅದೇ ಆ ಸತ್ತು. ಮೇಜಿನ ಮತ್ತು ಉಳಿದ ಆಕಾರಗಳೆಲ್ಲ ನಮ್ಮ ಬುದ್ಧಿಯಿಂದ ಅದಕ್ಕೆ ಕೊಡಲ್ಪಟ್ಟಿವೆ.

ಈಗ, ಉದಾಹರಣೆಗೆ, ವ್ಯಕ್ತ ವಸ್ತುವಿನ ವಿಶೇಷಣವಾದ ಚಲನೆಯನ್ನು ಅವ್ಯಕ್ತಕ್ಕೆ ಆರೋಪಿಸುವುದಕ್ಕೆ ಆಗುವುದಿಲ್ಲ. ವಿಶ್ವದಲ್ಲಿರುವ ಪ್ರತಿಯೊಂದು ಚೂರು ಕೂಡ, ಪ್ರತಿಯೊಂದು ಕಣವು ಕೂಡ ಅನವರತವೂ ಚಲಿಸುತ್ತಿದೆ, ವಿಕಾರವಾಗುತ್ತಿದೆ. ಆದರೆ ಒಟ್ಟಿನಲ್ಲಿ ವಿಶ್ವವು ಅವಿಕಾರಿಯಾದುದು. ಏಕೆಂದರೆ ಚಲನೆ ಅಥವಾ ವಿಕಾರ ಸಾಪೇಕ್ಷವಾದುದು, ಚಲಿಸದೇ ಇರುವ ಮತ್ತಾವುದಾದರೊಂದು ವಸ್ತುವಿಗೆ ಹೋಲಿಸಿದಾಗ ಮಾತ್ರ ಒಂದು ವಸ್ತು ಚಲಿಸುತ್ತಿದೆ ಎಂದು ಆಲೋಚಿಸಬಹುದು. ಚಲನೆಯನ್ನು ಅರ್ಥಮಾಡಿಕೊಳ್ಳಬೇಕಾದರೆ ಎರಡು ವಸ್ತುಗಳು ಇರಬೇಕು. ಇಡೀ ವಿಶ್ವವನ್ನೇ ನಾವು ಒಂದು ಏಕವಾಗಿ ತೆಗೆದುಕೊಂಡರೆ ಅದು ಚಲಿಸಲಾರದು. ಯಾವುದರೊಂದಿಗೆ ಹೋಲಿಸಿದರೆ ಅದು ಚಲಿಸುವುದು? ಅದು ವಿಕಾರಹೊಂದುತ್ತದೆ ಎಂದು ಹೇಳಲಾಗುವುದಿಲ್ಲ.ಯಾವುದರೊಂದಿಗೆ ಹೋಲಿಸಿದಾಗ ಅದು ಬದಲಾಯಿಸುವುದು? ಆದಕಾರಣ ಪೂರ್ಣವೇ ಬ್ರಹ್ಮ. ಆದರೆ ಅದರಂತರಾಳದಲ್ಲಿ ಪ್ರತಿಯೊಂದು ಕಣ ಕೂಡ ನಿರಂತರ ಪ್ರವಾಹದಲ್ಲಿರುವುದು, ಬದಲಾಯಿಸು ತ್ತಿರುವುದು. ಅದು ಏಕ ವೇಳೆಯಲ್ಲಿ ವಿಕಾರಿಯಾದುದು ಮತ್ತು ಅವಿಕಾರಿಯಾದುದು. ಅದು ನಿರಾಕರಾವೂ ಹೌದು ಸಾಕಾರವೂ ಹೌದು. ಸೃಷ್ಟಿ, ಚಲನೆ, ದೇವರು ಇವುಗಳ ವಿಷಯದಲ್ಲಿ ನಮ್ಮ ಭಾವನೆಯೇ ಇದು. “ತತ್ತ್ವಮಸಿ” ಎಂಬುದರ ಅರ್ಥವೇ ಇದು. ನಿರಾಕಾರವು ಸಾಕಾರವನ್ನು ನಿರಾಕರಿಸುವ ಬದಲು ಅದನ್ನು ವಿವರಿಸುತ್ತದೆ. ನಿರಪೇಕ್ಷವು ಸಾಪೇಕ್ಷವನ್ನು ಕೆಳಕ್ಕೆ ಇಳಿಸುವುದರ ಬದಲು ಅದನ್ನು, ನಮ್ಮ ಬುದ್ಧಿ ಭಾವಗಳಿಗೆ ಪೂರ್ಣವಾಗಿ ತೃಪ್ತಿಯಾಗುವಂತೆ ವಿವರಿಸುತ್ತದೆ. ವಿಶ್ವದಲ್ಲಿರುವ ಸಾಕಾರ ದೇವರು ಮತ್ತು ಇನ್ನು ಉಳಿದವುಗಳೆಲ್ಲವೂ ಕೂಡ ನಮ್ಮ ಮನಸ್ಸಿನ ಮೂಲಕ ನೋಡಿದ ನಿರ್ಗುಣ ಬ್ರಹ್ಮವೆ. ನಮ್ಮ ವಸ್ತುವನ್ನು ಮೀರಿಹೋದ ಮೇಲೆ, ನಮ್ಮ ಕ್ಷುದ್ರ ವ್ಯಕ್ತಿತ್ವ ಅಳಿಸಿಹೋದಮೇಲೆ ನಾವು ಬ್ರಹ್ಮನೊಂದಿಗೆ ಒಂದಾಗುವೆವು. “ತತ್ತ್ವಮಸಿ” ಎಂಬುದರ ಅರ್ಥವೆ ಇದು. ಏಕೆಂದರೆ ನಮ್ಮ ನಿಜ ಸ್ವರೂಪವನ್ನು, ಅನಂತತೆಯನ್ನು ನಾವು ತಿಳಿದುಕೊಳ್ಳಬೇಕು.

ಅಂತವುಳ್ಳ, ವ್ಯಕ್ತವಾದ ವ್ಯಕ್ತಿಯು, ತಾನು ಎಲ್ಲಿಂದ ಬಂದೆ ಎಂಬುದನ್ನು ಮರೆತು, ತಾನು ಸಂಪೂರ್ಣವಾಗಿ ಪ್ರತ್ಯೇಕಗೊಂಡವನು ಎಂದು ತಿಳಿಯುವನು. ವ್ಯಕ್ತರೂಪಧಾರಣೆ ಮಾಡಿದ ನಾವು, ವಿಭಿನ್ನರಾದ ನಾವು, ನಮ್ಮ ನಿಜತ್ವವನ್ನು ಮರೆಯುವೆವು. ಈ ಭಿನ್ನತೆಗಳನ್ನು ತೊರೆದುಬಿಡಬೇಕೆಂದು ಅದ್ವೈತಿಗಳು ಹೇಳು ವುದಿಲ್ಲ. ಆದರೆ ಭಿನ್ನತೆಗಳೇನೆಂಬುದನ್ನು ನಾವು ತಿಳಿದುಕೊಳ್ಳಬೇಕು. ನಿಜವಾಗಿಯೂ ನಾವು ಅನಂತಬ್ರಹ್ಮ. ನಮ್ಮ ಅನೇಕ ವ್ಯಕ್ತಿತ್ವಗಳು, ಆ ಅದ್ವಿತೀಯವಾದ ಸತ್ಯವು ಪ್ರಕಾಶಿಸುತ್ತಿರುವ ಅನೇಕ ಮಾರ್ಗಗಳಾಗಿವೆ. ನಾವು ವಿಕಾಸವೆಂದು ಕರೆಯುವ ಈ ಬದಲಾವಣೆಗಳ ಮೊತ್ತವು, ಆತ್ಮವು ಹೆಚ್ಚುಹೆಚ್ಚಾಗಿ ತನ್ನ ಅನಂತ ಶಕ್ತಿಯನ್ನು ಹೊರಪಡಿಸುವುದರಿಂದ ಬಂದಿದೆ. ನಾವು ಈ ಅನಂತತೆಯ ಕಡಲಿನ ಮಧ್ಯದಲ್ಲಿ ಎಲ್ಲಿಯೂ ನಿಲ್ಲುವುದಕ್ಕೆ ಆಗುವುದಿಲ್ಲ. ನಮ್ಮಲ್ಲಿರುವ ಸಚ್ಚಿದಾನಂದವು ಅನಂತವಾ ಗದೆ ಇರದು. ಅನಂತ ಶಕ್ತಿ, ಅನಂತ ಅಸ್ತಿತ್ವ, ಆನಂದಗಳು ನಮ್ಮವು. ಅವನ್ನು ನಾವು ಗಳಿಸಬೇಕಾಗಿಲ್ಲ, ಅವು ನಮ್ಮ ಸ್ವಂತ ವಸ್ತುಗಳು. ಅವನ್ನು ನಾವು ಪ್ರಕಾಶ ಪಡಿಸಬೇಕಷ್ಟೆ.

ತಿಳಿದುಕೊಳ್ಳುವುದಕ್ಕೆ ಅತಿ ಕಷ್ಟಸಾಧ್ಯವಾದ ಅದ್ವೈತದ ಮುಖ್ಯಭಾವನೆ ಇದು. ನನ್ನ ಬಾಲ್ಯದಿಂದಲೂ ಸುತ್ತಲಿರುವ ಎಲ್ಲರೂ ನಿರ್ಬಲತೆಯನ್ನು ಬೋಧಿಸಿರುವರು. ನಾನು ಹುಟ್ಟಿದಾಗಿನಿಂದಲೂ ನನ್ನನ್ನು ಒಬ್ಬ ಅಬಲನೆಂದು ಹೇಳಿರುವರು. ಸಧ್ಯದಲ್ಲಿ ನನ್ನ ಶಕ್ತಿಯನ್ನು ನಾನು ತಿಳಿದುಕೊಳ್ಳುವುದು ಅತಿ ಕಷ್ಟವಾಗಿದೆ. ಆದರೆ ವಿಶ್ವೇಷಣೆ ಯಿಂದ, ವಿಚಾರದಿಂದ ನನ್ನ ಸ್ವಂತ ಶಕ್ತಿಯ ಜ್ಞಾನ ನನಗೆ ಬರುವುದು. ಅನುಭವ ದಿಂದ ನಾನದನ್ನು ತಿಳಿದುಕೊಳ್ಳುತ್ತೇನೆ. ಜಗತ್ತಿನಲ್ಲಿರುವ ಜ್ಞಾನವೆಲ್ಲ ಎಲ್ಲಿಂದ ಬಂದಿತು? ಅದು ನಮ್ಮಲ್ಲಿಯೇ ಇತ್ತು. ಯಾವ ಜ್ಞಾನ ಹೊರಗೆ ಇತ್ತು? ಯಾವುದೂ ಇಲ್ಲ. ಜ್ಞಾನವು ಜಡವಸ್ತುವಿನಲ್ಲಿ ಇರಲಿಲ್ಲ. ಸರ್ವಕಾಲದಲ್ಲಿಯೂ ಅದು ಮಾನವ ನಲ್ಲಿತ್ತು. ಯಾರು ಎಂದಿಗೂ ಜ್ಞಾನವನ್ನು ಸೃಷ್ಟಿಸಲಿಲ್ಲ. ಮಾನವನು ತನ್ನ ಅಂತರಂಗ ದಿಂದಲೇ ಅದನ್ನು ತರುವನು. ಅದು ಅಲ್ಲಿಯೆ ಇರುವುದು. ಅನೇಕ ಎಕರೆಗಳಲ್ಲಿ ಹಬ್ಬಿರುವ ಆ ದೊಡ್ಡದೊಂದು ಆಲದ ಮರ ಬಹುಶಃ ಸಾಸುವೆ ಕಾಳಿನ ಎಂಟನೇ ಒಂದು ಭಾಗದಷ್ಟು ಕೂಡ ದೊಡ್ಡದಾಗಿಲ್ಲದ ಅದರ ಸಣ್ಣ ಬೀಜದಲ್ಲಿತ್ತು. ಆ ಶಕ್ತಿಯ ಮೊತ್ತವೆಲ್ಲ ಅಲ್ಲಿ ಅಡಗಿತ್ತು. ಪ್ರಚಂಡ ಮೇಧಾಶಕ್ತಿ ಜೀವಾಣುವಿನಲ್ಲಿ ಸುಪ್ತವಾಗಿರುವುದು ನಮಗೆ ಗೊತ್ತಿದೆ. ಅನಂತ ಶಕ್ತಿ ಏಕೆ ಅಲ್ಲಿ ಇರಬಾರದು? ಅದು ಹಾಗಿರುವುದೆಂದು ನಮಗೆ ಗೊತ್ತಿದೆ. ಆ ಸತ್ಯ ವಿರೋಧಾಭಾಸದಂತೆ ನಮಗೆ ತೋರಬಹುದು. ಆದರೆ ಅದು ಸತ್ಯ. ನಾವೆಲ್ಲರೂ ಜೀವಾಣುವಿನಿಂದ ಬಂದಿರುವೆವು. ಈಗ ನಮ್ಮಲ್ಲಿರುವ ಎಲ್ಲಾ ಶಕ್ತಿಯೂ ಅಲ್ಲಿ ಸುಪ್ತವಾಗಿರುವುದು. ಅದೆಲ್ಲ ಆಹಾರದಿಂದ ಬಂದಿತೆಂದು ಹೇಳಲಾಗುವುದಿಲ್ಲ. ಆಹಾರವನ್ನು ನೀವು ಬೆಟ್ಟದಂತೆ ರಾಶಿ ಹಾಕಿದರೆ ಏನು ಶಕ್ತಿ ಉದ್ಭವಿಸುತ್ತದೆ? ಆ ಶಕ್ತಿ ಸುಪ್ತ ರೀತಿಯಲ್ಲಿ ಇತ್ತು. ಆದರೂ ಅಲ್ಲಿ ಶಕ್ತಿ ಇತ್ತು. ಅದರಂತೆಯೆ ಮಾನವನಿಗೆ ತಿಳಿಯಲಿ ತಿಳಿಯದೆ ಇರಲಿ ಅನಂತಶಕ್ತಿ ಅವನಾತ್ಮನಲ್ಲಿದೆ. ಅದರ ಅರಿವು ನಮಗೆ ಆದಾಗ ಅದು ಜಾಗ್ರತವಾಗು ವುದು. ನಿಧಾನವಾಗಿ ಈ ಅನಂತ ಮಹಾಮಹಿಮನು ಸುಜಾಗ್ರತನಾಗಿ, ತನ್ನ ಶಕ್ತಿಯ ಪರಿಚಯ ಹೊಂದಿ ಮೇಲೇಳುವಂತೆ ಕಾಣುತ್ತಿರುವುದು. ಅವನ ಪ್ರಜ್ಞೆ ವಿಶಾಲ ವಾಗುತ್ತ ಆಗುತ್ತ ಅವನ ಬಂಧನಗಳು ಹೆಚ್ಚು ಹೆಚ್ಚು ಸಡಿಲವಾಗುತ್ತಿವೆ, ಸರಪಣಿಗಳು ಮುರಿಯುತ್ತಿವೆ. ತನ್ನ ಅನಂತ ಶಕ್ತಿಯ, ಅನಂತ ಜ್ಞಾನದ ಪೂರ್ಣ ಪರಿಚಯವಾಗಿ ತನ್ನ ಕಾಲಮೇಲೆ ಧೈರ್ಯದಿಂದ ನಿಲ್ಲುವ ದಿನ ಬರುವುದು ನಿಜ. ಅಂತಹ ಪವಿತ್ರ ವಾದ ಗುರಿಯೆಡೆಗೆ ಎಲ್ಲರೂ ಬೇಗ ಹೋಗಲು ನಾವುಗಳೆಲ್ಲರೂ ಸಹಾಯ ಮಾಡೋಣ.

\chapter{ಅಧ್ಯಾಯ ೪}

\begin{center}
\textbf{(೧೮೯೬ರ ನವೆಂಬರ್​ ೧೮ ರಂದು ಲಂಡನ್ನಿನಲ್ಲಿ ನೀಡಿದ ಉಪನ್ಯಾಸ)}
\end{center}

ಇದುವರೆಗೂ ನಾವು ಸಮಷ್ಟಿಯನ್ನು ಗಮನಿಸುತ್ತಿದ್ದೆವು. ಇಂದು ವ್ಯಷ್ಟಿಗೂ ಸಮಷ್ಟಿಗೂ ವೇದಾಂತದ ರೀತಿಯಾಗಿ ಇರುವ ಸಂಬಂಧವನ್ನು ನಿಮ್ಮ ಮುಂದಿಡಲು ಪ್ರಯತ್ನಿಸುವೆನು. ನಾವು ನೋಡಿದಂತೆ ದ್ವೈತ ಸಿದ್ಧಾಂತದ ಪ್ರಕಾರ ಪ್ರತಿಯೊಂದು ಜೀವರಿಗೂ ಸ್ಪಷ್ಟ ಲಕ್ಷಣಗಳುಳ್ಳ ಮಿತವಾದ, ಪ್ರತ್ಯೇಕವಾದ ಒಂದು ಆತ್ಮವಿದೆ. ಪ್ರತಿ ವ್ಯಕ್ತಿಯಲ್ಲಿಯೂ ಇರುವ ಪ್ರತ್ಯೇಕವಾದ ಆತ್ಮನ ವಿಚಾರವಾಗಿ ಹಲವು ಸಿದ್ಧಾಂತಗಳಿವೆ. ಆದರೆ ಮುಖ್ಯವಾದ ಚರ್ಚೆ ಹಳೆಯ ವೇದಾಂತಿಗಳಿಗೂ ಹಳೆಯ ಬೌದ್ಧರಿಗೂ ನಡೆಯುತ್ತಿತ್ತು. ವೇದಾಂತಿಗಳು ಪರಿಪೂರ್ಣವಾದ ಆತ್ಮವಿದೆ ಎಂದು ನಂಬುತ್ತಿದ್ದರು. ಬೌದ್ಧರು ಅಂತಹ ಆತ್ಮನ ಅಸ್ತಿತ್ವವನ್ನು ಸಂಪೂರ್ಣವಾಗಿ ವಿರೋಧಿ ಸುತ್ತಿದ್ದರು. ನಾನು ನಿಮಗೆ ಹಿಂದೆ ಹೇಳಿದಂತೆ ಯೂರೋಪಿನಲ್ಲಿ ನಿಮ್ಮಲ್ಲಿರುವ ಗುಣ ಮತ್ತು ಗುಣಿಗಳ ಚರ್ಚೆಯಂತೆ ಇದೆ ಇದು. ಒಂದು ಪಕ್ಷದವರು ಗುಣಗಳ ಹಿಂದೆ ಗುಣಿ ಎಂಬುದೊಂದು ಇದೆ, ಇದರಲ್ಲಿ ಗುಣವೆಲ್ಲ ಅಂತರ್ಗತವಾಗಿದೆ ಎಂತಲೂ; ಮತ್ತೊಂದು ಪಕ್ಷದವರು, ಅಂತಹ ಗುಣಿ ಇಲ್ಲವೆಂದೂ ಅದು ಅನಾ ವಶ್ಯಕವೆಂದೂ, ಗುಣಗಳು ತಮಗೆ ತಾವೆ ಇರಬಹುದೆಂದೂ ವಾದಿಸುತ್ತಿದ್ದರು. ಆತ್ಮನ ವಿಚಾರವಾಗಿ ಬಹಳ ಪುರಾತನ ಸಿದ್ಧಾಂತವು ತಾದಾತ್ಮ್ಯವಾದದ ಮೇಲೆ ನಿಂತಿರುವುದು. “ನಾನೆ ನಾನು.” – ನಿನ್ನೆಯ ನಾನೆ ಇಂದಿನ ನಾನು, ಇಂದಿನ ನಾನೆ ನಾಳೆಯ ನಾನಾಗುವುದು. “ನಿನ್ನೆಯ ದೇಹ ಎಷ್ಟೇ ಬದಲಾವಣೆಯಾಗುತ್ತಿದ್ದರೂ, ಹಿಂದಿನ ನಾನೆ ಇಂದಿನ ನಾನು ಎಂದು ನಂಬುತೇನೆ.” ಮಿತವಾದ ಆದರೂ ಸಂಪೂರ್ಣವಾದ ಪ್ರತ್ಯೇಕ ಆತ್ಮ ವು ಇದೆ ಎಂದು ನಂಬಿದವರು ಮುಖ್ಯವಾದ ವಾದ ಇದು ಎಂದು ಕಾಣುತ್ತದೆ.

ಆದರೆ ಪ್ರಾಚೀನ ಬೌದ್ಧರಾದರೋ ಇಂತಹ ಊಹೆ ಅನಾವಶ್ಯಕವಾದುದೆಂದು ಸಾರಿದರು. ತಮಗೆ ತಿಳಿದಿರುವುದೆಲ್ಲ ಮತ್ತು ಮುಂದೆ ತಿಳಿಯಬಹುದಾದುದೆಲ್ಲ ಈ ಬದಲಾವಣೆಗಳು ಮಾತ್ರ ಎಂಬ ವಾದವನ್ನು ತಂದರು. ವಿಕಾರಹೊಂದದ ಮತ್ತು ಅವಿಕಾರಿಯಾದ ಒಂದು ವಸ್ತುವನ್ನು ಊಹಿಸುವುದು ನಿಷ್ಟ್ರಯೋಜನ ಎಂದೂ ಅಂತಹ ಒಂದು ವಿಕಾರವಾಗದ ವಸ್ತು ಒಂದು ವೇಳೆ ಇದ್ದರೂ, ಅದನ್ನು ನಾವು ತಿಳಿಯ ಲಾರೆವು, ಅದರ ವಿಚಾರದಲ್ಲಿ ಯಾವುದನ್ನೂ ಗ್ರಹಿಸುವುದಕ್ಕೆ ನಮಗೆ ಸಾಧ್ಯವಿಲ್ಲ ಎಂದೂ ಬೌದ್ಧರು ವಾದಿಸಿದರು. ಸದ್ಯದಲ್ಲಿ ಯೂರೋಪಿನಲ್ಲಿ ಅಂತಹ ಚರ್ಚೆ ನಡೆಯುತ್ತಿದೆ. ಒಂದು ಪಕ್ಷದಲ್ಲಿ ಧರ್ಮಾಭಿಮಾನಿಗಳು ಮತ್ತು ಭಾವ ಸತ್ತಾವಾದಿಗಳಿರುವರು. ಇನ್ನೊಂದು ಪಕ್ಷದಲ್ಲಿ ಅಸ್ತಿವಾದಿಗಳೂ ಆಜ್ಞೆಯತಾ ವಾದಿಗಳೂ ಇರುವರು. ಒಂದು ಪಕ್ಷದವರು ಬದಲಾವಣೆಯಾಗದ ಯಾವುದೋ ಒಂದು ವಸ್ತು ಹಿಂದೆ ಇದೆ ಎಂದು ನಂಬುವರು. ಅವರ ಇತ್ತೀಚಿನ ಪ್ರತಿನಿಧಿ ಹರ್​ಬರ್ಟ್​ ಸ್ಪೆನ್ಸರ್​. ಆಧುನಿಕ ಕಾಮ್ಟೆ ಅನುಯಾಯಿಗಳು ಮತ್ತು ಆಜ್ಞೇಯತಾ ವಾದಿಗಳು ಇನ್ನೊಂದು ಪಕ್ಷವನ್ನು ಪ್ರತಿನಿಧಿಸುವರು. ಕೆಲವು ವರ್ಷಗಳ ಹಿಂದೆ ಹರ್ಬರ್ಟ್​ ಸ್ಪೆನ್ಸ್​ರ್​ ಮತ್ತು ಫ್ರೆಡರಿಕ್​ ಹ್ಯಾರಿಸ್​ ಇವರಿಬ್ಬರ ನಡುವೆ ನಡೆದ ಚರ್ಚೆಯಲ್ಲಿ ನಿಮ್ಮಲ್ಲಿ ಕುತೂಹಲಿಗಳಾಗಿದ್ದವರು, ಅದು ಆ ಹಳೆಯ ವಾದವೇ ಎಂಬುದನ್ನು ಕಂಡುಹಿಡಿಯಬಹುದು. ಒಂದು ಪಕ್ಷದವರು ಬದಲಾಯಿಸುವ ಗುಣಗಳ ಹಿಂದೆ ಬದಲಾಯಿಸದ ಒಂದು ಗುಣಿ ಇದೆ ಎನ್ನುವರು. ಮತ್ತೊಂದು ಪಕ್ಷದವರು ಅಂತಹ ಊಹೆ ಅನಗತ್ಯವಾದುದು ಎನ್ನುವರು. ಒಂದು ಪಕ್ಷದವರು ಅವಿಕಾರಿಯಾದ ವಸ್ತುವಿನ ಗ್ರಹಣವಿಲ್ಲದೆ ವಿಕಾರವನ್ನು ನಾವು ಗ್ರಹಿಸಲಾರೆವು ಎನ್ನುವರು. ಮತ್ತೊಂದು ಪಕ್ಷದವರು ಅದು ಕೇವಲ ಅನಗತ್ಯವಾದುದೆಂದು ಹೇಳುವರು. ವಿಕಾರವಾದುದನ್ನು ಮಾತ್ರ ನಾವು ಗ್ರಹಿಸಲು ಸಾಧ್ಯ. ಅವಿಕಾರಿಯಾದುದನ್ನು ನಾವು ಗ್ರಹಿಸಲಾರೆವು, ಅನುಭವಿಸಲಾರೆವು ಅಥವಾ ನೋಡಲಾರೆವು.

ಭಾರತಖಂಡದಲ್ಲಿ ಬಹಳ ಹಿಂದಿನ ಕಾಲದಲ್ಲಿ ಇದಕ್ಕೆ ಉತ್ತರ ಸಿಕ್ಕಲಿಲ್ಲ. ಏಕೆಂದರೆ ಗುಣಗಳ ಹಿಂದಿರುವ ಮತ್ತು ಗುಣಗಳಲ್ಲದ ಗುಣಿಯನ್ನು ಊಹಿಸುವುದು ಮತ್ತು ಅದನ್ನು ಸತ್ಯವೆಂದು ಸಾಧಿಸುವುದು ಅಸಾಧ್ಯ. ತಾದಾತ್ಮ್ಯ ದೃಷ್ಟಿಯಿಂದಲೂ, ಮತ್ತು ನೆನಪಿನ ಆಧಾರದ ಮೇಲಿನ ವಾದವಾದ – ನಿನ್ನೆಯ ನಾನೆ ಇಂದಿನ ನಾನು; ಏಕೆಂದರೆ ನನಗೆ ಅದು ಜ್ಞಾಪಕದಲ್ಲಿದೆ, ಆದಕಾರಣ ನಾನೊಂದು ಹಿಂದೆ ಮತ್ತು ಇಂದು ಇರುವ ವಸ್ತು– ಎಂಬುದನ್ನು ಯಾರೂ ಪ್ರಮಾಣ ಸಹಿತ ಸಿದ್ಧ ಪಡಿಸುವುದುದಕ್ಕಾಗುವುದಿಲ್ಲ. ಸಾಧಾರಣವಾಗಿ ಅವರು ತರುವ ಮತ್ತೊಂದು ಕುತರ್ಕ ಕೇವಲ ಪದಗಳ ಆಡಂಬರ ಅಷ್ಟೆ. ಉದಾಹರಣೆಗೆ ಮುಂದೆ ಬರುವ ಅನೇಕ ವಾಕ್ಯಗಳನ್ನು ತೆಗೆದುಕೊಳ್ಳಬಹುದು. “ನಾನು ಮಾತನಾಡುತ್ತೇನೆ” “ನಾನು ಹೋಗುತ್ತೇನೆ” “ನಾನು ಕನಸು ಕಾಣುತ್ತೇನೆ” “ನಾನು ಮಲಗುತ್ತೇನೆ” “ನಾನು ಸಂಚರಿಸುತ್ತೇನೆ” ಇತ್ಯಾದಿ. ಇಲ್ಲಿ ಮಾಡುತ್ತೇನೆ, ಹೋಗುತ್ತೇನೆ, ಕನಸು ಕಾಣುತ್ತೇನೆ ಮುಂತಾದ ಕ್ರಿಯಾಪದಗಳು ಬದಲಾಯಿಸುತ್ತಿರುವುದನ್ನು ನೀವು ನೋಡಬಹುದು. ಆದರೆ ಬದಲಾವಣೆಯಾಗದ ಪದ ಯಾವುದೆಂದರೆ “ನಾನು” ಎಂಬುದು. ಆದಕಾರಣ ಅವರು ನಾನು ಎಂಬುದು ಅವಿಕಾರಿಯಾದುದೆಂದೂ, ಪ್ರತ್ಯೇಕವಾದುದೆಂದೂ, ವಿಕಾರವೆಲ್ಲ ದೇಹಕ್ಕೆ ಅನ್ವಯಿಸುವುದೆಂದೂ ಹೇಳುತ್ತಾರೆ. ಈ ವಾದ ತೋರಿಕೆಗೆ ಬಹಳ ತೃಪ್ತಿಕರವಾಗಿ ಕಂಡರೂ, ಇದು ನಿಂತಿರುವುದು ಪದಗಳ ಆಟದ ಮೇಲೆ. ನಾನು ಹಾಗೂ ಕಾಣುವುದು, ಹೋಗುವುದು ಮತ್ತು ಕನಸು ಕಾಣುವುದು ಬರವಣಿಗೆಯಲ್ಲಿ ಬೇರೆ ಬೇರೆಯಾಗಿ ಕಂಡರೂ, ಮನಸ್ಸಿನಲ್ಲಿ ಎಂದಿಗೂ ಅದನ್ನು ಬೇರ್ಪಡಿಸಲಾಗುವುದಿಲ್ಲ.

ನಾನು ಊಟ ಮಾಡುವಾಗ ನಾನೇ ಊಟ ಮಾಡುವಂತೆ ತಿಳಿಯುವೆನು, ನಾನು ಊಟಮಾಡುವ ಕ್ರಿಯೆಯೊಂದಿಗೆ ಸಾರೂಪ್ಯ ಭಾವವನ್ನು ಹೊಂದುವೆನು. ನಾನು ಓಡುವಾಗ, ನಾನು ಮತ್ತು ಓಡುವುದು ಎಂಬ ಎರಡು ಬೇರೆ ಬೇರೆ ವಸ್ತುಗಳಿಲ್ಲ. ಆದಕಾರಣ ತಾದಾತ್ಮ್ಯವಾದ ಅಷ್ಟೇನು ಬಲವಾಗಿರುವಂತೆ ತೋರುವುದಿಲ್ಲ. ನೆನಪಿನ ಮೇಲೆ ನಿಂತಿರುವ ಮತ್ತೊಂದು ವಾದವೂ ಕೂಡ ಅಷ್ಟೇ ದುರ್ಬಲವಾದುದು. ನೆನಪಿನ ಮೇಲೆ ತಾದಾತ್ಮ್ಯವು ನಿಂತಿದ್ದರೆ, ನಾನು ಮರೆತ ಹಲವು ವಿಷಯಗಳು ತಾದಾತ್ಮ್ಯದಿಂದ ಮಾಯವಾದಂತೆಯೆ.ಜನರು, ಕೆಲವು ಸ್ಥಿತಿಗಳಲ್ಲಿ, ತಮ್ಮ ಇಡೀ ಹಿಂದನ್ನು ಮರೆಯುವರು. ಹುಚ್ಚಿನ ಕೆಲವು ಅವಸ್ಥೆಗಳಲ್ಲಿ ವ್ಯಕ್ತಿಯು ತಾನು ಗಾಜಿನಿಂದ ಮಾಡಲ್ಪಟ್ಟವನೆಂದೊ, ಅಥವಾ ತಾನು ಪ್ರಾಣಿಯೆಂದೊ ತಿಳಿದು ಕೊಳ್ಳುತ್ತಾನೆ. ವ್ಯಕ್ತಿಯ ಅಸ್ತಿತ್ವವು ನೆನಪನ್ನು ಆಶ್ರಯಿಸಿದ್ದರೆ ಅವನು ಗಾಜಾಗಿದ್ದಾನೆ. ಹಾಗಲ್ಲದೇ ಇರುವುದರಿಂದ ಆತ್ಮದ ಅಸ್ತಿತ್ವವನ್ನು ನೆನಪಿನಂತಹ ಅಭದ್ರ ಅಂಶದ ನೆರವಿನಿಂದ ಪ್ರಮಾಣೀಕರಿಸುವುದಕ್ಕೆ ಆಗುವುದಿಲ್ಲ. ಹೀಗೆ ಗುಣಗಳಿಗಿಂತ ಪ್ರತ್ಯೇಕವಾದ ಪೂರ್ಣವಾದ ಆತ್ಮವೊಂದಿದೆ ಎಂಬುದನ್ನು ಪ್ರಮಾಣೀಕರಿಸಲಾಗುವುದಿಲ್ಲ, ಮತ್ತು ಗುಣಗಳ ಮೊತ್ತವನ್ನು ಆರೋಪಿಸಬಹುದಾದ ಮಿತವಾದ ಅಸ್ತಿತ್ವವೊಂದನ್ನು ಪ್ರಮಾಣೀಕರಿಸುವುದೂ ಸಾಧ್ಯವಿಲ್ಲ.

ಈ ವಿಷಯದಲ್ಲಿ ಪ್ರಾಚೀನ ಬೌದ್ಧರ ವಾದವು ಪ್ರಬಲವಾಗಿರುವಂತೆ ಕಾಣುತ್ತದೆ. ಗುಣಗಳ ಸಮೂಹಕ್ಕಿಂತ ಬೇರೆಯಾದ ಯಾವುದೂ ನಮಗೆ ತಿಳಿಯದು ಮತ್ತು ತಿಳಿಯಲಾರೆವು.ಅವರ ಪ್ರಕಾರ ಆತ್ಮವೆಂದರೆ ಸಂವೇದನೆ ಮತ್ತು ಭಾವನೆಗಳೆಂದು ಕರೆಯಲ್ಪಡುವ ಗುಣಗಳ ಒಂದು ಮೊತ್ತ, ಹಾಗೂ ಇದು ನಿರಂತರವಾಗಿ ಬದಲಾಗುತ್ತಿರುವುದು.

ಅದ್ವೈತಿಯ ಆತ್ಮಸಿದ್ಧಾಂತವು ಎರಡು ಪಕ್ಷಗಳಿಗೂ ರಾಜಿಮಾಡಿಸುತ್ತದೆ. ಅದ್ವೈತಿಯ ದೃಷ್ಟಿ ಇದು: ಗುಣಗಳಿಂದ ಗುಣಿಯು ಬೇರೆ ಇದೆ ಎಂದು ಆಲೋಚನೆ ಮಾಡುವುದಕ್ಕೆ ನಮಗೆ ಆಗುವುದಿಲ್ಲವೆಂಬುದೇನೋ ನಿಜ. ವಿಕಾರ ಮತ್ತು ಅವಿಕಾರ ವನ್ನು ನಾವು ಏಕಕಾಲದಲ್ಲಿ ಆಲೋಚಿಸಲು ಸಾಧ್ಯವಿಲ್ಲ. ಆದರೆ ಯಾವುದು ಗುಣಿಯೋ ಅದೇ ಗುಣ. ಗುಣ ಮತ್ತು ಗುಣಿ ವಿಭಿನ್ನವಾದ ಎರಡು ವಸ್ತುಗಳಲ್ಲ. ಅವಿಕಾರವಾದುದೆ ವಿಕಾರವಾದಂತೆ ನಮಗೆ ತೋರುತ್ತಿರುವುದು. ವಿಶ್ವದ ಅವಿಕಾರ ವಸ್ತುವು ವಿಕಾರವಸ್ತುವಿನಿಂದ ಬೇರೆಯಲ್ಲ. ಅವ್ಯಕ್ತವು ವ್ಯಕ್ತಕ್ಕಿಂತ ಬೇರೆಯಲ್ಲ, ಆದರೆ ಅವ್ಯಕ್ತವೇ ವ್ಯಕ್ತವಾಗಿರುವುದು. ಅವಿಕಾರವಾದ ಆತ್ಮವೊಂದಿದೆ; ಮತ್ತು ನಮ್ಮ ಚಿಂತನೆ, ಭಾವನೆ, ಹಾಗೂ ದೇಹವೂ ಕೂಡ, ನಾವು ಬೇರೆ ದೃಷ್ಟಿಯಿಂದ ನೋಡಿದ ಈ ಆತ್ಮವೆ. ನಮಗೆ ದೇಹಗಳಿವೆ, ಆತ್ಮಗಳಿವೆ, ಎಂದು ಹೇಳುವ ಒಂದು ಅಭ್ಯಾಸ ನಮಗೆ ಬಂದಿರುವುದು. ನಿಜವಾಗಿಯೂ ಇರುವುದು ಒಂದೇ ಒಂದು.

ನಾನು ದೇಹವೆಂದು ಭಾವಿಸಿದಾಗ, ನಾನು ದೇಹಮಾತ್ರ; ನಾನು ದೇಹವಲ್ಲದೆ ಬೇರೆ ಎಂದು ಹೇಳುವುದಕ್ಕೆ ಅರ್ಥವಿಲ್ಲ. ನಾನು ಆತ್ಮವೆಂದು ಆಲೋಚಿಸಿದಾಗ ದೇಹಬುದ್ಧಿ ಮಾಯವಾಗುತ್ತದೆ, ದೇಹದ ಅರಿವು ಉಳಿಯುವುದಿಲ್ಲ. ದೇಹದ ಭಾವನೆ ಮಾಯವಾಗುವವರೆಗೂ, ಯಾರಿಗೂ, ಆತ್ಮದ ಭಾವನೆ ಬರುವುದಿಲ್ಲ. ಗುಣಗಳ ಭಾವನೆ ಮಾಯವಾಗುವವರೆಗೂ ಯಾರಿಗೂ ಗುಣಿಯ ಭಾವನೆ ಬರುವುದಿಲ್ಲ.

ಅದ್ವೈತ ವೇದಾಂತದ ರಜ್ಜು ಸರ್ಪ ಭ್ರಮೆಯ ಹಳೆಯ ಉದಾಹರಣೆಯು ವಿಷಯವನ್ನು ಮತ್ತೂ ಸ್ಪಷ್ಟಪಡಿಸುವುದು. ಮನುಷ್ಯನು ಹಗ್ಗವನ್ನು ಹಾವೆಂದು ಭ್ರಮಿಸಿದಾಗ ಹಗ್ಗ ಮಾಯವಾಗಿರುತ್ತದೆ. ಅದನ್ನು ಹಗ್ಗವೆಂದು ತಿಳಿದಾಗ ಹಾವು ಮಾಯವಾರುತ್ತಗಿದೆ. ಹಗ್ಗ ಮಾತ್ರ ಉಳಿಯುತ್ತದೆ. ಎರಡು ಮೂರು ವಸ್ತುಗಳು ಇವೆ ಎಂಬ ಭಾವನೆ ಅಪೂರ್ಣವಾದ ಅಂಕಿ ಅಂಶಗಳನ್ನು ಆಶ್ರಯಿಸಿ ಆಲೋಚಿಸುವುದರ ಮೂಲಕ ಬರುವುದು. ಇಂತಹ ಭಾವನೆಯನ್ನು ಪುಸ್ತಕಗಳಲ್ಲಿ ಓದುತ್ತೇವೆ, ಇಲ್ಲವೆ ಮತ್ತೊಬ್ಬರಿಂದ ಕೇಳಿ ನಮಗೆ ನಿಜವಾಗಿಯೂ ದೇಹ ಮತ್ತು ಆತ್ಮಗಳೆಂಬ ಎರಡು ವಸ್ತುಗಳ ಜ್ಞಾನವು ಇದೆ ಎಂಬ ಮೋಹಕ್ಕೆ ಒಳಗಾಗುತ್ತೇವೆ. ಆದರೆ ಅಂತಹ ಜ್ಞಾನ ನಿಜವಾಗಿ ಎಂದಿಗೂ ಇಲ್ಲ.ನಮ್ಮ ಜ್ಞಾನವು ದೇಹ ಅಥವಾ ಆತ್ಮಕ್ಕೆ ಸಂಬಂಧಿಸಿದುದು. ಇದನ್ನು ಸಿದ್ಧಾಂತ ಪಡಿಸುವುದಕ್ಕೆ ಯಾವ ಯುಕ್ತಿ ಪ್ರಮಾಣವೂ ಬೇಕಾಗಿಲ್ಲ. ನಮ್ಮ ಮನಸ್ಸಿನಲ್ಲೆ ಇದನ್ನು ಹೋಲಿಸಿ ನೋಡಬಹುದು.

ನಾನೊಂದು ನಿರ್ದೇಹನಾದ ಆತ್ಮ ವಸ್ತು ಎಂದು ಯೋಚಿಸಲು ಪ್ರಯತ್ನಪಟ್ಟು ನೋಡಿ, ಅದು ಅತ್ಯಂತ ಕಷ್ಟವೆಂಬುದು ನಿಮಗೆ ಗೊತ್ತಾಗುತ್ತದೆ. ಹೀಗೆ ಮಾಡುವುದ ರಲ್ಲಿ ಜಯಶೀಲರಾದ ಅಲ್ಪಮಂದಿಗೆ ತಾವು ಆತ್ಮವೆಂದು ತಿಳಿದ ಕಾಲದಲ್ಲಿ ದೇಹದ ಆಲೋಚನೆ ಇರುವುದಿಲ್ಲ. ಕೆಲವರು ದೀರ್ಘಧ್ಯಾನ, ಸ್ವಸಮ್ಮೋಹಿನೀ ವಿದ್ಯೆಯ ಪ್ರಭಾವ, ಸನ್ನಿ ಅಥವಾ ಕೆಲವು ಔಷಧಿಗಳ ಪ್ರಯೋಗದ ಪರಿಣಾಮ ಇವುಗಳಿಂದಾಗಿ ಕೆಲವು ವೇಳೆ ಅತಿ ವಿಚಿತ್ರವಾದ ಮಾನಸಿಕ ಸ್ಥಿತಿಯಲ್ಲಿರುವುದನ್ನು ನೀವು ಬಹುಶಃ ಕೇಳಿರಬಹುದು ಅಥವಾ ನೋಡಿರಬಹುದು. ಅಂತರಂಗದಲ್ಲಿರುವ ಯಾವುದೋ ವಸ್ತುವನ್ನು ಅವರು ನೋಡುತ್ತಿದ್ದಾಗ ಬಾಹ್ಯ ವಸ್ತು ಅವರ ಪಾಲಿಗೆ ಮಾಯವಾಗಿತ್ತು ಎಂಬುದನ್ನು ಅವರ ಅನುಭವಗಳಿಂದ ತಿಳಿಯಬಹುದು. ಇರುವುದೆಲ್ಲ ಒಂದೇ ವಸ್ತು ಎಂಬುದನ್ನು ಇದು ತೋರುವುದು. ಆ ಏಕವೆ ಅನಂತ ರೂಪಗಳಲ್ಲಿ ಕಾಣುತ್ತಿರುವುದು. ಈ ಅನೇಕವೇ ಕಾರ್ಯಕಾರಣ ಸಂಬಂಧಕ್ಕೆ ಕಾರಣ. ಕಾರ್ಯಕಾರಣಗಳ ಸಂಬಂಧವು ವಿಕಾಸಕ್ಕೆ ಸೇರಿದುದು ಒಂದು ಇನ್ನೊಂದಾಗುವುದು. ಕೆಲವು ವೇಳೆ ಕಾರಣವು ಮಾಯವಾದಂತೆ ತೋರಿ ಅದರ ಸ್ಥಳದಲ್ಲಿ ಕಾರ್ಯವು ನಿಲ್ಲುವುದು. ಆತ್ಮವು ದೇಹಕ್ಕೆ ಕಾರಣವಾದರೆ, ಆತ್ಮವು ಸದ್ಯಕ್ಕೆ ಮಾಯವಾದಂತೆ ತೋರಿ ದೇಹ ಮಾತ್ರ ನಿಲ್ಲುವುದು; ದೇಹವು ಮಾಯವಾದಾಗ ಆತ್ಮವು ನಿಲ್ಲುವುದು. ಇದು ದೇಹ ಮತ್ತು ಆತ್ಮಗಳೆರಡನ್ನೂ ಊಹಿಸುವುದಕ್ಕೆ ವಿರೋಧವಾಗಿ ಹೂಡಿದ ಬೌದ್ಧರ ವಾದಸರಣಿಗೆ ಸರಿಯಾದ ಉತ್ತರವಾಗಿದೆ. ಈ ಸಿದ್ಧಾಂತವು ದೇಹ ಮತ್ತು ಆತ್ಮಗಳೆಂಬ ದ್ವೈತ ಭಾವವನ್ನು ನಿಷೇಧಿಸಿ, ಗುಣ ಮತ್ತು ಗುಣಿಗಳು ಒಂದೆ ಎಂದೂ ಒಂದೇ ವಸ್ತುವೇ ನಾನಾ ರೂಪಗಳಲ್ಲಿ ಕಾಣುತ್ತಿದೆ ಎಂದು ತೋರಿಸುತ್ತದೆ.

ಅವಿಕಾರವಸ್ತುವಿನ ಭಾವನೆಯನ್ನು ಪೂರ್ಣದೃಷ್ಟಿಯಿಂದ ಮಾತ್ರ ಸಿದ್ಧಾಂತ ಪಡಿಸಬಹುದೇ ಹೊರತು ಎಂದಿಗೂ ಅಂಶ ದೃಷ್ಟಿಯಿಂದ ಅಲ್ಲ ಎಂಬುದನ್ನು ನಾವಾಗಲೇ ನೋಡಿರುವೆವು. ಅಂಶದ ಭಾವನೆ ಬರುವುದೇ ಬದಲಾವಣೆ ಮತ್ತು ಚಲನೆಯ ಭಾವನೆಯಿಂದ. ಅಂತವುಳ್ಳ ಪ್ರತಿಯೊಂದು ವಸ್ತುವನ್ನೂ ನಾವು ತಿಳಿದುಕೊಳ್ಳಬಹುದು. ಏಕೆಂದರೆ ಅಂತವುಳ್ಳ ವಸ್ತುಗಳು ವಿಕಾರವಾಗಬಲ್ಲವು. ಪೂರ್ಣವಾದರೊ ವಿಕಾರವಾಗದೆ ಇರಬೇಕು. ಏಕೆಂದರೆ ಇದಲ್ಲದೆ ಯಾವ ವಸ್ತುವಿನ ಹೋಲಿಕೆಯ ಮೇಲೆ ಇದು ಬದಲಾಯಿಸುವುದಕ್ಕೆ ಸಾಧ್ಯವೋ ಅಂತಹ ವಸ್ತುವೇ ಇಲ್ಲ. ಬದಲಾವಣೆಯಾಗದೆ ಇರುವ ಅಥವಾ ಕಡಿಮೆ ಬದಲಾವಣೆಯಾಗುವ ವಸ್ತು ವಿನೊಂದಿಗೆ ಹೋಲಿಸಿದಾಗ ಮಾತ್ರ ಬದಲಾವಣೆ ಸಾಧ್ಯ.

ಆದಕಾರಣ ಅದ್ವೈತ ದೃಷ್ಟಿಯಿಂದ ಆತ್ಮವು ಸರ್ವವ್ಯಾಪಿಯಾದುದು, ಅವಿಕಾರಿ ಯಾದುದು ಮತ್ತು ಅಮೃತವಾದುದು ಎಂಬುದನ್ನು ಸಾಧ್ಯವಾದ ಮಟ್ಟಿಗೆ ತೋರಿಸ ಬಹುದು. ಆದರೆ ಕಷ್ಟವಿರುವುದು ಪ್ರತ್ಯೇಕ ಆತ್ಮವಿದೆ ಎಂಬ ವಿಷಯದಲ್ಲಿ ಮಾತ್ರ. ಆದರೆ ನಮ್ಮ ಮೇಲೆ ಅಷ್ಟೊಂದು ಅಧಿಕಾರವುಳ್ಳ, ಯಾವ ಸಿದ್ಧಾಂತದ ಮೂಲಕ ನಾವುಗಳೆಲ್ಲ ಹಾದು ಹೋಗಬೇಕೋ ಅಂತಹ, ಮಿತವಾದ ಪ್ರತ್ಯೇಕಾತ್ಮನಲ್ಲಿ ನಂಬಿಕೆ ಉಳ್ಳ, ಆ ಹಳೆಯ ದ್ವೈತಸಿದ್ಧಾಂತಗಳನ್ನು ನಾವು ಏನು ಮಾಡೋಣ?

ಪೂರ್ಣದೊಂದಿಗೆ ಹೋಲಿಸಿದಾಗ ಮಾತ್ರ ನಾವು ಅಮರರು ಎಂಬುದನ್ನು ನೋಡಿರುವೆವು. ಆದರೆ ಬರುವ ಕಷ್ಟವೇ ಇದು. ನಾವು ಪೂರ್ಣದ ಭಾಗವಾಗಿ ಅಮರರಾಗಲು ಆಶಿಸುತ್ತೇವೆ. ನಾವು ಅನಂತ, ಅದೇ ನಮ್ಮ ನಿಜವಾದ ವ್ಯಕ್ತಿತ್ವ ಎಂಬುದನ್ನು ನೋಡಿರುವೆವು. ಆದರೆ ನಾವು ಈ ಅಲ್ಪಾತ್ಮನನ್ನೇ ನಮ್ಮ ನೈಜ ವ್ಯಕಿತ್ತ್ವವಾಗಿ ಮಾಡಲು ಬಯಸುವೆವು. ನಮ್ಮ ಅನುದಿನದ ಅನುಭವದಲ್ಲಿ ಈ ಅಲ್ಪಾತ್ಮವೇ ನಮ್ಮ ವ್ಯಕ್ತಿತ್ವ. ಆದರೆ ಅದು ಎಡೆಬಿಡದೆ ವಿಕಾರ ಹೊಂದುತ್ತಿರುವ ಆತ್ಮ ಎಂದು ನಮಗೆ ತಿಳಿದಾಗ ಅದರ ಪಾಡೇನಾಗುವುದು? ಅದು ಒಂದೇ ಆದರೂ ಒಂದೆ ಅಲ್ಲ. ನಿನ್ನೆಯ ನಾನೆ ಇಂದಿನ ನಾನು. ಆದರೂ ಅದು ಅಲ್ಲ. ಏಕೆಂದರೆ ಅದು ಸ್ವಲ್ಪ ಬದಲಾಯಿಸಿರುವುದು. ಈ ಬದಲಾವಣೆಯ ಅಂತರಾಳದಲ್ಲಿ ಒಂದು ವಸ್ತುವಿದೆ ಎಂಬ ದ್ವೈತ ಭಾವನೆಯಿಂದ ಪಾರಾಗಿ ವಿಕಾಸವಾದದ ಆತ್ಯಾಧುನಿಕ ಭಾವವನ್ನು ನಾವು ಸ್ವೀಕರಿಸಿದರೆ, “ನಾನು” ಎಂಬುದು ಅನುಗಾಲವೂ ವಿಕಾರ ವಾಗುತ್ತ ವಿಕಾಸವಾಗುತ್ತಿರುವ ವಸ್ತುವೆಂದು ತೋರುತ್ತದೆ.

ಮಾನವನು ಬಸವನ ಹುಳುವಿನಿಂದ ವಿಕಾಸವಾಗಿರುವುದು ಸತ್ಯವಾದರೆ, ಬಸವನಹುಳುವಿನ ವ್ಯಕ್ತಿತ್ವವೂ ಮಾನವನ ವ್ಯಕ್ತಿತ್ವವೂ ಒಂದೇ ಆದರೆ, ಬಸವನ ಹುಳು ಇನ್ನೂ ಹೆಚ್ಚು ವಿಕಾಸವಾಗಬೇಕು, ಅಷ್ಟೆ. ಬಸವನ ಹುಳುವಿನಿಂದ ಹಿಡಿದು ಮಾನವನವರೆಗಿನ ವಿಕಾಸವು ಅನಂತತೆಯ ಕಡೆಗಿನ ನಿರಂತರ ವಿಕಾಸವೇ ಆಗಿದೆ. ಆದಕಾರಣ ಈ ಜೀವಾತ್ಮನನ್ನು ಅನಂತತೆಯ ಕಡೆಗೆ ಅನವರತವಾಗಿ ವಿಕಾಸವಾಗು ತ್ತಿರುವ ವ್ಯಕ್ತಿ ಎನ್ನಬಹುದು. ಅನಂತವನ್ನು ಸೇರಿದಾಗ ಮಾತ್ರ ಸಂಪೂರ್ಣ ವ್ಯಕ್ತಿತ್ವ ವನ್ನು ಪಡೆಯಬಹುದು. ಆದರೆ ಅದಕ್ಕೆ ಮೊದಲು ಅದು ಸದಾ ವಿಕಾಸವಾಗುತ್ತಿರುವ, ಅಭಿವೃದ್ಧಿಯಾಗುತ್ತಿರುವ ವ್ಯಕ್ತಿ. ಅದ್ವೈತ ಸಿದ್ಧಾಂತದ ಒಂದು ಅಪೂರ್ವವಾದ ಮುಖ್ಯ ಲಕ್ಷಣವೇ ಹಿಂದಿನ ಸಿದ್ಧಾಂತಗಳೆಲ್ಲವನ್ನೂ ಸಮನ್ವಯ ಮಾಡುವುದು. ಅನೇಕ ಸಲ ಇದು ತತ್ತ್ವಕ್ಕೆ ಬಹಳ ಸಹಾಯಮಾಡಿದೆ. ಕೆಲವು ಸಲ ಅದಕ್ಕೆ ಕುಂದನ್ನೂ ತಂದಿದೆ. ನಮ್ಮ ಹಿಂದಿನ ತತ್ತ್ವಶಾಸ್ತ್ರಜ್ಞರಿಗೆ ನಿಮ್ಮ ವಿಕಾಸವಾದ, ಎಂದರೆ ಬೆಳವಣಿಗೆಯು ಮೆಟ್ಟಲು ಮೆಟ್ಟಲಾಗಿ ಕ್ರಮೇಣ ಆಗುವುದು ಎಂಬುದು ತಿಳಿದಿತ್ತು. ಈ ಅಂಶವನ್ನು ಗುರುತಿಸಿದುದರಿಂದಾಗಿಯೇ ಹಿಂದಿನಿಂದ ಬಂದ ಸಿದ್ಧಾಂತಗಳೆಲ್ಲ ವನ್ನು ಸಮನ್ವಯ ಮಾಡಲು ಅವರಿಗೆ ಸಾಧ್ಯವಾಯಿತು. ಆದಕಾರಣ ಹಿಂದಿನಿಂದ ಬಂದ ಯಾವ ಭಾವನೆಯನ್ನು ಅಲ್ಲಗಳೆಯಲಿಲ್ಲ. ಅನವರತ ವಿಕಾಸವಾಗುತ್ತಿರುವ ಬೆಳವಣಿಗೆ ಬೌದ್ಧರಿಗೆ ಗೋಚರವಾಗಲಿಲ್ಲ ಮತ್ತು ಅದನ್ನು ತಿಳಿಯಲು ಆಸೆಯೂ ಇರಲಿಲ್ಲ. ಇದೇ ಬೌದ್ಧಧರ್ಮದ ಕುಂದು. ಆದಕಾರಣ ಅದು ಆದರ್ಶವನ್ನು ಸೇರುವುದಕ್ಕೆ ಇರುವ ಹಿಂದಿನ ವಿವಿಧ ಹಂತಗಳನ್ನು ಸಮನ್ವಯಗೊಳಿಸುವ ಗೋಜಿಗೆ ಹೋಗಲಿಲ್ಲ. ಅಹಿತಕರವೆಂದೂ ನಿಷ್ಟ್ರ ಯೋಜಕವೆಂದೂ ಅವನ್ನು ತ್ಯಜಿಸಿತು.

ಇಂತಹ ಧಾರ್ಮಿಕ ಪ್ರವೃತ್ತಿ ಬಹಳ ಕೇಡಿಗೆ ಕಾರಣ. ಮಾನವನಿಗೆ ಒಂದು ಸೊಗಸಾದ ಉತ್ತಮ ಭಾವನೆ ಸಿಗುತ್ತದೆ. ಅನಂತರ ತಾನು ತ್ಯಜಿಸಿದ ಭಾವನೆಗಳನ್ನು ನಿಕೃಷ್ಟ ದೃಷ್ಟಿಯಿಂದ ಕಾಣುವನು. ತಕ್ಷಣವೇ ಅವು ಅನಾವಶ್ಯಕವಾಗಿತ್ತೆಂದೂ, ಕೇಡಿಗೆ ಕಾರಣವೆಂದೂ ನಿರ್ಧಾರಕ್ಕೆ ಬರುವನು. ಆದರೆ ಆತನ ಈಗಿನ ದೃಷ್ಟಿಯಿಂದ ಅವು ಎಷ್ಟೇ ಒರಟಾಗಿ ಕಂಡರೂ, ಅವನಿಗೆ ಹಿಂದಿನದೂ ಬಹಳ ಉಪಕಾರಿಯಾ ಗಿದ್ದವು. ಇಂದಿನ ಸ್ಥಿತಿಗೆ ಬರುವುದಕ್ಕೆ ಅವು ಬಹಳ ಸಹಾಯಕವಾಗಿದ್ದವು. ನಾವೆಲ್ಲರೂ ಇದೇ ರೀತಿಯಲ್ಲಿ ಅಭಿವೃದ್ಧಿಯಾಗಬೇಕು. ಮೊದಲು ಒರಟಾದ ಭಾವಗಳಲ್ಲಿದ್ದು, ಅವುಗಳಿಂದ ಸಹಾಯವನ್ನು ಪಡೆದು, ಅದಕ್ಕಿಂತ ಉತ್ತಮ ಸ್ಥಿತಿಗೆ ಬರಬೇಕೆಂಬುದನ್ನು ಅವನು ಯೋಚಿಸುವುದೇ ಇಲ್ಲ. ಆದಕಾರಣ ಅದ್ವೈತವು ಪುರಾತನ ಸಿದ್ಧಾಂತ ಗಳೊಂದಿಗೆ ಸೌಹಾರ್ದದಿಂದಿರುವುದು. ಅದ್ವೈತವು ತನಗೆ ಹಿಂದೆ ಇದ್ದ ದ್ವೈತ ಮುಂತಾದ ಸಿದ್ಧಾಂತಗಳೆಲ್ಲವನ್ನೂ ಸ್ವೀಕರಿಸುವುದು; ಅವುಗಳಿಗೆ ಕೇವಲ ಭುಜತಟ್ಟಿ ಉತ್ತೇಜನ ಉತ್ತೇಜನ ಕೊಡುವ ರೀತಿಯಿಂದ ಅಲ್ಲ, ಅವುಗಳೆಲ್ಲವೂ ಏಕಸತ್ಯದ ಅಭಿವ್ಯಕ್ತಿಗಳೆಂದು, ತಾನು ಯಾವ ನಿರ್ಣಯಕ್ಕೆ ಬಂದಿದೆಯೋ ಅದೇ ನಿರ್ಣಯಕ್ಕೆ ಅವು ಕೂಡ ಒಯ್ಯುವುವು ಎಂಬ ದೃಢ ವಿಶ್ವಾಸದಿಂದ ಅವುಗಳನ್ನು ಅದ್ವೈತವು ಒಪ್ಪಿಕೊಳ್ಳುವುದು.

ಮಾನವ ಜನಾಂಗವು ಹತ್ತಿಹೋಗಬೇಕಾಗಿರುವ ಈ ಮೆಟ್ಟಲುಗಳೆಲ್ಲವನ್ನೂ ನಿಂದಿಸದೆ ಆಶೀರ್ವದಿಸಿ ಕಾಪಾಡಬೇಕು. ಆದಕಾರಣ ಈ ದ್ವೈತ ಸಿದ್ಧಾಂತಗಳಾ ವುದನ್ನೂ ತಿರಸ್ಕರಿಸಿಲ್ಲ ಅಥವಾ ಅವನ್ನು ಆಚೆಗೆ ಎಸೆದಿಲ್ಲ. ಯಾವುದನ್ನೂ ಕಳೆಯದೆ ಎಲ್ಲವನ್ನೂ ವೇದಾಂತದಲ್ಲಿಟ್ಟಿರುವರು. ಪರಿಮಿತಿಯಿಂದ ಕೂಡಿದ್ದರೂ ತನ್ನಲ್ಲಿಯೇ ಪೂರ್ಣವಾಗಿರುವ ಪ್ರತ್ಯೇಕಾತ್ಮನಿಗೆ ವೇದಾಂತದಲ್ಲಿ ಸ್ಥಳವಿದೆ.

ದ್ವೈತಸಿದ್ಧಾಂತಗಳ ರೀತಿಯಲ್ಲಿ ಮನುಷ್ಯನು ಕಾಲವಾಗಿ ಇತರ ಲೋಕಗಳಿಗೆ ಹೋಗುತ್ತಾನೆ. ವೇದಾಂತದಲ್ಲಿ ಇಂತಹ ಭಾವನೆಗಳನ್ನು ಸಂಪೂರ್ಣವಾಗಿ ಕಳೆಯದೆ ಇಟ್ಟಿರುವರು. ಅದ್ವೈತಸಿದ್ಧಾಂತದಲ್ಲಿ ಬೆಳವಣಿಗೆಯನ್ನು ಒಪ್ಪಿಕೊಂಡಮೇಲೆ ಈ ಸಿದ್ಧಾಂತಗಳು ಸತ್ಯವನ್ನು ಆಂಶಿಕವಾಗಿ ಪ್ರತಿನಿಧಿಸುತ್ತವೆ ಎಂದು ತಿಳಿದು ಅವಕ್ಕೆ ತಕ್ಕ ಸ್ಥಾನವನ್ನು ಕೊಟ್ಟಿರುವರು.

ದ್ವೈತ ದೃಷ್ಟಿಯಿಂದ ಈ ಜಗತ್ತು ಜಡವಸ್ತು ಅಥವಾ ಪ್ರಾಣದಿಂದ ಸೃಷ್ಟಿ ಯಾಗಿದೆ ಎಂದು ಮಾತ್ರ ನೋಡಲು ಸಾಧ್ಯ, ಯಾವುದೋ ಒಂದು ಇಚ್ಛೆಯ ಲೀಲೆ ಎಂದು ನೋಡಬಹುದು. ಆ ಇಚ್ಛೆಯು ಜಗತ್ತಿನಿಂದ ಬೇರೆ ಎಂದು ಮಾತ್ರ ನೋಡಲು ಸಾಧ್ಯ. ಆದಕಾರಣ ಅಂತಹ ದೃಷ್ಟಿಯಿಂದ ಮಾನವನು ತನ್ನನ್ನು ದೇಹ ಆತ್ಮಗಳ ದ್ವಂದ್ವಮಿಶ್ರವೆಂದು ನೋಡಲೇಬೇಕು. ಈ ಆತ್ಮವು ಮಿತಿಯಿಂದ ಕೂಡಿದ್ದರೂ, ತನ್ನ ಮಟ್ಟಿಗೆ ಪೂರ್ಣವಾಗಿರುವುದು. ಅಂತಹ ಮಾನವನ ಅಮರತ್ವ ಮತ್ತು ಭವಿಷ್ಯ ಜೀವನದ ಭಾವನೆಗಳು ಆತನು ಆತ್ಮದ ವಿಷಯದಲ್ಲಿ ತಾಳಿರುವ ಭಾವನೆಗೆ ಅನುಗುಣವಾಗಿಯೇ ಇರುವುವು. ವೇದಾಂತದಲ್ಲಿ ಈ ಭಾವನೆಗಳೆಲ್ಲವನ್ನೂ ಇಟ್ಟಿರುವರು. ಆದಕಾರಣ ದ್ವೈತಿಗಳ ಕೆಲವು ಸಾಮಾನ್ಯವಾದ ಭಾವನೆಗಳನ್ನು ನಿಮ್ಮ ಮುಂದಿಡುವುದು ಆವಶ್ಯಕ. ಈ ಸಿದ್ಧಾಂತದ ರೀತಿ ನಮಗೆ ಒಂದು ದೇಹವೇನೊ ಇರುವುದು. ಇದರ ಹಿಂದೆ ಅವರು ಕರೆವ ಸೂಕ್ಷ್ಮ ಶರೀರವಿದೆ. ಈ ಸೂಕ್ಷ್ಮ ಶರೀರವೂ ಕೂಡ ಭೌತ ವಸ್ತುವಿನಿಂದ ಆಗಿದೆ, ಆದರೆ ಇದು ಸೂಕ್ಷ್ಮ ಅಷ್ಟೇ. ಪ್ರತ್ಯಕ್ಷವಾದ ರೂಪಕ್ಕೆ ಬರಲು ಸಿದ್ಧವಾಗಿರುವ ನಮ್ಮ ಎಲ್ಲಾ ಕರ್ಮಗಳ, ಸಂಸ್ಕಾರಗಳ ನಿಧಿ ಇದು. ನಾವು ಮಾಡುವ ಪ್ರತಿಯೊಂದು ಆಲೋಚನೆ ಮತ್ತು ಕ್ರಿಯೆ ಕೆಲವು ಕಾಲದ ಮೇಲೆ ಸೂಕ್ಷ್ಮವಾಗಿ ಬೀಜರೂಪಕ್ಕೆ ಬಂದಂತೆ ತೋರುವುವು. ಸೂಕ್ಷ್ಮ ದೇಹದಲ್ಲಿ ಅವ್ಯಕ್ತವಾಗಿದ್ದು, ಅವು, ಕೆಲವು ಕಾಲದ ಅನಂತರ ಮೇಲೆದ್ದು ಫಲವನ್ನು ಕೊಡುವುವು. ಈ ಫಲಗಳೇ ಮನುಷ್ಯನ ಜೀವನದ ಸ್ಥಿತಿಯನ್ನು ನಿರ್ಧರಿಸುವುದು. ಹೀಗೆ ಮನುಷ್ಯನು ತನ್ನ ಸ್ಥಿತಿಗೆ ತಾನೇ ಕಾರಣನಾಗುವನು. ಮಾನವನು ತಾನಾಗಿಯೇ ಮಾಡಿಕೊಳ್ಳುವ ಬಂಧನಗಳಲ್ಲದೆ ಬೇರೆ ಯಾವುದ ರಿಂದಲೂ ಬದ್ಧನಾಗಿಲ್ಲ. ನಮ್ಮ ಆಲೋಚನೆ, ಮಾತು ಮತ್ತು ಕಾರ್ಯಗಳೆಲ್ಲ, ಒಳ್ಳೆಯದಕ್ಕೊ ಕೆಟ್ಟದಕ್ಕೊ, ನಮ್ಮ ಸುತ್ತ ನಾವು ಬೀಸಿಕೊಳ್ಳುವ ಬಲೆಯ ದಾರಗಳು. ಒಮ್ಮೆ ನಾವೊಂದು ಶಕ್ತಿಯನ್ನು ಚಲಿಸುವಂತೆ ಮಾಡಿದರೆ ಅದರ ಪೂರ್ಣ ಪರಿಣಾಮವನ್ನು ನಾವು ಪಡೆಯಲೇಬೇಕು. ಇದೇ ಕರ್ಮ ನಿಯಮ. ಸೂಕ್ಷ್ಮ ಶರೀರದ ಹಿಂದೆ ಜೀವಾತ್ಮನಿರುವನು. ಜೀವಾತ್ಮನ ಆಕಾರ ಮತ್ತು ಪ್ರಮಾಣಗಳ ವಿಚಾರವಾಗಿ ಹಲವು ಚರ್ಚೆಗಳಿವೆ. ಕೆಲವರ ದೃಷ್ಟಿಯಲ್ಲಿ ಅದು ಕಣದಂತೆ ಬಹಳ ಸಣ್ಣದು, ಮತ್ತೆ ಕೆಲವರ ದೃಷ್ಟಿಯಲ್ಲಿ ಅದು ಅಷ್ಟು ಸಣ್ಣದಲ್ಲ, ಮತ್ತೆ ಕೆಲವರ ದೃಷ್ಟಿಯಿಂದ ಅದು ಬಹಳ ದೊಡ್ಡದು–ಇತ್ಯಾದಿ ಹಲವು ಅಭಿಪ್ರಾಯಗಳಿವೆ. ಜೀವಾತ್ಮನು ಪರಮಾತ್ಮನ ಒಂದು ಅಂಶ. ಇದೂ ಕೂಡ ಅದರಂತೆ ಅನಾದ್ಯನಂತ. ಅನಾದಿಕಾಲದಿಂದಲೂ ಇದು ಇದೆ. ಮುಂದೆ ಅಂತ್ಯವಿಲ್ಲದೆ ಇರುವುದು. ತನ್ನ ನೈಜಸ್ವರೂಪವಾದ ಪಾವಿತ್ರ್ಯವನ್ನು ಪ್ರದರ್ಶಿಸುವುದಕ್ಕೋಸುಗವಾಗಿ ಹಲವು ಆಕಾರಗಳ ಮೂಲಕ ಇದು ಹಾದುಹೋಗುತ್ತಿದೆ. ಪಾವಿತ್ರ್ಯದ ಅಭಿವ್ಯಕ್ತಿಗೆ ಆತಂಕವನ್ನು ತಂದೊಡ್ಡುವ ಪ್ರತಿಯೊಂದು ಕ್ರಿಯೆಯೂ ಪಾಪ. ಅದರಂತೆಯೇ ಆಲೋಚನೆ ಗಳು ಕೂಡ. ಜೀವಾತ್ಮನು ವಿಕಾಸವಾಗುವುದಕ್ಕೆ ತನ್ನ ನೈಜ ಸ್ವಭಾವವನ್ನು ಬೆಳಗಲು ಸಹಾಯಮಾಡುವ ಪ್ರತಿಯೊಂದು ಕ್ರಿಯೆ ಮತ್ತು ಆಲೋಚನೆ ಕೂಡ ಪುಣ್ಯ. ಅತ್ಯಂತ ಒರಟಾದ ದ್ವೈತಿಗಳಿಂದ ಹಿಡಿದು ಬಹಳ ಮುಂದುವರಿದ ಅದ್ವೈತಿ ಗಳವರೆಗೂ ಎಲ್ಲರೂ ಒಪ್ಪಿಕೊಳ್ಳುವ ಸಾಮಾನ್ಯ ಸಿದ್ಧಾಂತವೇ, ಆತ್ಮನ ಎಲ್ಲಾ ಶಕ್ತಿ ಮತ್ತು ಸಾಧ್ಯತೆಗಳು ಆಗಲೆ ಅವನಲ್ಲಿವೆ, ಹೊರಗಿನಿಂದ ಬರುವುದಿಲ್ಲ, ಎನ್ನುವುದು.ಆಗಲೆ ಇವು ಆತ್ಮನಲ್ಲಿ ಅವ್ಯಕ್ತ ರೂಪದಲ್ಲಿರುವುವು. ಇವನ್ನು ವ್ಯಕ್ತ ಮಾಡುವುದಕ್ಕೋಸುಗವಾಗಿಯೆ ನಮ್ಮ ಜೀವನದ ಕಾರ್ಯಗಳೆಲ್ಲ ಇರುವುದು.

ಅವರಲ್ಲಿ ಪುನರ್ಜನ್ಮ ಸಿದ್ಧಾಂತವೂ ಇದೆ. ಜೀವಾತ್ಮವು ಈ ಜಗದಲ್ಲೋ ಅಥವಾ ಇನ್ನೂ ಯಾವುದಾದರೂ ಜಗತ್ತಿನಲ್ಲಿಯೋ ಈ ಜನ್ಮವಾದ ಮೇಲೆ ಮತ್ತೊಂದು ಜನ್ಮ, ಅದಾದ ಮೇಲೆ ಇನ್ನೊಂದು ಜನ್ಮವನ್ನು ಪಡೆಯುತ್ತಾ ಹೋಗುವುದು ಎನ್ನುವರು. ಆದರೆ ಈ ಜಗತ್ತಿಗೆ ಮೊದಲ ಸ್ಥಾನವನ್ನುಕೊಡುವರು. ಏಕೆಂದರೆ ಇದೇ ಇರುವ ಜಗತ್ತುಗಳಲ್ಲೆಲ್ಲ ನಮ್ಮ ಸಾಧನೆಗೆ ಅತ್ಯುತ್ತಮವಾದುದೆಂದು ಅವರ ಅಭಿಪ್ರಾಯ. ಉಳಿದ ಲೋಕಗಳಲ್ಲಿ ದುಃಖವು ಇಲ್ಲಿಗಿಂತ ಕಡಿಮೆ ಎಂಬುದು ಅವರ ಮತ. ಆದರೆ, ಅದಕ್ಕೋಸ್ಕರವೆ ಅಲ್ಲಿ ಉತ್ತಮ ವಿಷಯಗಳನ್ನು ಆಲೋಚಿಸು ವುದಕ್ಕೆ ಅವಕಾಶ ಕಡಿಮೆ. ಈ ಜಗತ್ತಿನಲ್ಲಿ ಸುಖವು ಸ್ವಲ್ಪ ಇದ್ದು ಕಷ್ಟವು ಹೆಚ್ಚಾಗಿರುವುದರಿಂದ ಜೀವನು ಎಂದಾದರೊಮ್ಮೆ ಜಾಗ್ರತನಾಗಿ ಮುಕ್ತನಾಗಲು ಆಶಿಸುವನು. ಈ ಪ್ರಪಂಚದಲ್ಲಿ ಅತಿ ಶ‍್ರೀಮಂತರಿಗೆ ಹೇಗೆ ಒಳ್ಳೆಯ ವಿಷಯಗಳನ್ನು ಆಲೋಚಿಸಲು ಅವಕಾಶ ಬಹಳ ಕಡಿಮೆ ಇದೆಯೋ, ಅದರಂತೆಯೇ ಸ್ವರ್ಗದಲ್ಲಿರುವ ಜೀವನಿಗೂ ಅಭಿವೃದ್ಧಿಗೆ ಅವಕಾಶ ಕಡಿಮೆ. ಇದರ ಅವಸ್ಥೆಯು ಶ‍್ರೀಮಂತನಂತೆ. ಆದರೆ ಅವನಿಗಿಂತಲೂ ಸುಖ ಮತ್ತೂ ಹೆಚ್ಚು ಅಷ್ಟೆ. ಜೀವನಿಗೆ ಅಲ್ಲಿ ಯಾವ ರೋಗ ಭಯವೂ ಇಲ್ಲದ ಒಂದು ಸೂಕ್ಷ್ಮದೇಹವಿರುವುದು. ಹಸಿವು ದಾಹಗಳ ಭಾಧೆ ಇದಕ್ಕೆ ಇಲ್ಲ. ಇದರ ಬಯಕೆಗಳೆಲ್ಲ ಪೂರ್ಣವಾಗುತ್ತವೆ. ಭೋಗದ ಮೇಲೆ ಭೋಗ ವನ್ನು ಅನುಭವಿಸುತ್ತ ಜೀವವು ಅಲ್ಲಿ ವಾಸಿಸುವುದು. ಹೀಗೆ ತನ್ನ ನೈಜ ಸ್ವಭಾವವನ್ನು ಅದು ಮರೆಯುವುದು. ಎಷ್ಟು ಭೋಗಗಳಿದ್ದರೂ ಅಭಿವೃದ್ಧಿಗೆ ಅವಕಾಶವಿರುವ ಕೆಲವು ಉತ್ತಮ ಲೋಕಗಳಿವೆ. ಆತ್ಮವು ದೇವರೊಂದಿಗೆ ಶಾಶ್ವತವಾಗಿ ವಾಸಿಸುವ ಅತ್ಯುನ್ನತ ಸ್ವರ್ಗವೇ ಕೆಲವು ದ್ವೈತಿಗಳ ಗುರಿ. ಅಲ್ಲಿ ಅವರಿಗೆ ಸುಂದರವಾದ ದೇಹ ಗಳಿರುತ್ತವೆ. ಸಾವು, ರೋಗ ಮತ್ತು ಪಾಪದ ಸುಳಿವು ಕೂಡ ಅವರಿಗೆ ಅಲ್ಲಿ ಇರು ವುದಿಲ್ಲ. ಅಲ್ಲಿ ಅವರ ಬಯಕೆಗಳೆಲ್ಲ ಪೂರ್ಣವಾಗುವುವು. ಕೆಲವು ವೇಳೆ ಅವರಲ್ಲಿ ಕೆಲವರು ಈ ಪ್ರಪಂಚಕ್ಕೆ ಹಿಂತಿರುಗಿ ಬಂದು ಮಾನವನಿಗೆ ದೇವರೆಡೆಗೆ ದಾರಿ ತೋರಲು ಉಪದೇಶ ಮಾಡುವುದಕ್ಕಾಗಿ ದೇಹಧಾರಣೆ ಮಾಡುವರು. ಇಂತಹವರೆ ಪ್ರಪಂಚದ ಆಚಾರ್ಯಪುರುಷರು. ಅವರಾಗಲೇ ಮುಕ್ತರಾಗಿ ಅತ್ಯುತ್ತಮ ಲೋಕದಲ್ಲಿ ದೇವರೊಂದಿಗೆ ವಾಸಿಸುತ್ತಿದ್ದರು. ಆದರೆ ದುಃಖದಲ್ಲಿ ನರಳುತ್ತಿರುವ ಮಾನವರಿಗಾಗಿ ಅವರೆದೆಯಲ್ಲಿ ಪ್ರೇಮ ಮತ್ತು ಸಹಾನುಭೂತಿ ಜಿನುಗುತ್ತಿದ್ದುವು. ಆದಕಾರಣ ಮಾನವನಿಗೆ ಮೋಕ್ಷಮಾರ್ಗವನ್ನು ಬೋಧಿಸುವುದಕ್ಕಾಗಿ ಪುನಃ ಅವತಾರ ಮಾಡಿದರು.

ಇದೇ ನಮ್ಮ ಪರಮ ಆದರ್ಶವಾಗಲಾರದೆಂದು ಅದ್ವೈತಿಗಳು ಹೇಳುವುದು ನಿಮಗೆ ಗೊತ್ತೇ ಇದೆ. ದೇಹಾತೀತರಾಗುವುದೇ ನಮ್ಮ ಗುರಿಯಾಗಬೇಕು. ಆದರ್ಶವು ಸಾಂತವಾಗಿರಲಾರದು. ಅನಂತಕ್ಕಿಂತ ಸ್ವಲ್ಪ ಕಡಿಮೆಯಾದರೂ ಅದು ನಮ್ಮ ಆದರ್ಶ ವಾಗಲಾರದು, ಮತ್ತು ಅನಂತವಾದ ದೇಹವಿರಲು ಸಾಧ್ಯವಿಲ್ಲ. ದೇಹವು ಸಾಂತದಿಂದ ಬಂದಿರುವ ಕಾರಣ ಅದು ಎಂದಿಗೂ ಸಾಧ್ಯವಾಗುವುದಿಲ್ಲ. ಅನಂತ ಆಲೋಚನೆ ಕೂಡ ಇರಲಾರದು, ಏಕೆಂದರೆ ಆಲೋಚನೆ ಹುಟ್ಟುವುದೇ ಸಾಂತದಲ್ಲಿ. ನಾವು ದೇಹದ ಆಚೆ, ಆಲೋಚನೆಯ ಆಚೆ ಕೂಡ ಹೇಗಬೇಕೆಂದು ಅದ್ವೈತ ಸಾರುವುದು. ಅದ್ವೈತ ಸಿದ್ಧಾಂತದ ಪ್ರಕಾರ ಈ ಸ್ವಾತಂತ್ರ್ಯವನ್ನು ನಾವು ಹೊಸದಾಗಿ ಗಳಿಸಬೇಕಾಗಿಲ್ಲ, ಅದು ಆಗಲೇ ನಮ್ಮದಾಗಿರುವುದು ಎಂಬುದನ್ನು ನಾವು ಆಗಲೆ ನೋಡಿರುವೆವು. ನಾವು ಅದನ್ನು ಮರೆತು ಅಲ್ಲಗಳೆಯುವೆವು ಅಷ್ಟೆ. ನಾವು ಪೂರ್ಣತೆ ಯನ್ನು ಹೊಂದಬೇಕಾಗಿಲ್ಲ. ಆದಾಗಲೆ ನಮ್ಮಲ್ಲಿರುವುದು. ಅಮರತ್ವ ಮತ್ತು ಆನಂದವನ್ನು ನಾವು ಹೊಸದಾಗಿ ಗಳಿಸಬೇಕಾಗಿಲ್ಲ. ಆದಾಗಲೇ ನಮ್ಮಲ್ಲಿರುವುದು. ತ್ರಿಕಾಲದಲ್ಲಿಯೂ ಅದು ನಮ್ಮದಾಗಿರುವುದು.

ನೀವು ಸ್ವತಂತ್ರರೆಂದು ದೃಢವಾಗಿ ಸಾರಲು ಪ್ರಯತ್ನಪಟ್ಟರೆ, ಈ ಕ್ಷಣವೆ ನೀವು ಮುಕ್ತರು. ನೀವು ಬದ್ಧರೆಂದು ಹೇಳಿದರೆ ಬದ್ಧರಾಗಿಯೆ ಇರುವಿರಿ. ಇದೆನ್ನೇ ಅದ್ವೈತವು ಧೈರ್ಯದಿಂದ ಸಾರುವುದು. ನಾನು ನಿಮಗೆ ದ್ವೈತಿಗಳ ಭಾವನೆಯನ್ನು ಹೇಳಿರುವೆನು. ಯಾವುದನ್ನು ಬೇಕಾದರೂ ನೀವು ಸ್ವಿಕರಿಸಬಹುದು.

ವೇದಾಂತದ ಗಹನ ಆದರ್ಶಗಳನ್ನು ತಿಳಿದುಕೊಳ್ಳುವುದು ಕಷ್ಟ. ಜನರು ಯಾವಾ ಗಲೂ ಇದರ ವಿಚಾರವಾಗಿ ಕಾದಾಡುತ್ತಿರುವರು. ಅತಿ ದೊಡ್ಡ ಆತಂಕವೆ ಇದು: ಕೆಲವು ಭಾವನೆಗಳು ಇವರ ವಶವಾದರೆ, ಉಳಿದ ಭಾವನೆಗಳನ್ನು ನಿಷೇಧಿಸಿ ಅವು ಗಳೊಂದಿಗೆ ಹೋರಾಡುವರು. ನಿಮಗೆ ಸರಿಯಾದುದನ್ನು ನೀವು ಸ್ವೀಕರಿಸಿ. ಉಳಿದ ವರಿಗೆ ಯಾವುದು ಅವಶ್ಯಕವೋ ಅವರು ಅದನ್ನು ಸ್ವೀಕರಿಸಲಿ. ಪ್ರತ್ಯೇಕವಾಗಿರುವ ವ್ಯಷ್ಟಿ ವ್ಯಕ್ತಿತ್ವವನ್ನು ನೀವು ಇಟ್ಟುಕೊಂಡಿರಬೇಕೆಂದು ಬಯಸಿದರೆ ಅದರಂತೆ ಇರಿ. ಬಯಕೆಗಳನ್ನೆಲ್ಲಾ ಇಟ್ಟುಕೊಳ್ಳಿ. ಇವುಗಳನ್ನು ಹೊಂದಿ ಸಂತೋಷವಾಗಿ, ಸಮಾಧಾನ ದಿಂದ ಇರಿ. ಮಾನವೀಯ ಅನುಭವವು ಚೆನ್ನಾಗಿದ್ದರೆ, ನಿಮಗೆ ಬೇಕಾಗುವ ತನಕ ಅದನ್ನು ಇಟ್ಟುಕೊಳ್ಳಿ. ನೀವು ಹಾಗೆ ಮಾಡಬಹುದು. ನಿಮ್ಮ ಅದೃಷ್ಟ ನಿಮ್ಮಗಳ ಕೈಯಲ್ಲಿರುವುದು. ನಿಮ್ಮ ವ್ಯಕ್ತಿತ್ವವನ್ನು ತ್ಯಜಿಸುವಂತೆ ಯಾರೂ ಬಲಾತ್ಕಾರ ಮಾಡಲಾರರು. ಎಷ್ಟು ಕಾಲ ನೀವು ಆಶಿಸುವಿರೋ ಅಷ್ಟು ಕಾಲ ನೀವು ಮನುಷ್ಯ ರಾಗಿರುವಿರಿ. ಯಾರೂ ನಿಮಗೆ ಅಡ್ಡಿಯನ್ನು ತರಲಾರರು. ನೀವು ದೇವತೆ ಗಳಾಗಬೇಕೆಂದು ಆಲೋಚಿಸಿದರೆ ನೀವು ದೇವತೆಗಳಾಗುವಿರಿ. ಅದೇ ಪ್ರಕೃತಿ ನಿಯಮ. ದೇವತೆಗಳೂ ಕೂಡ ಆಗಲು ಇಚ್ಛಿಸದವರು ಕೆಲವರು ಇರಬಹುದು. ಅವರದು ಒಂದು ಘೋರ ಭಾವನೆ ಎಂದು ತಿಳಿಯುವುದಕ್ಕೆ ನಿಮಗೆ ಏನು ಅಧಿಕಾರವಿದೆ? ನಿಮ್ಮ ಹಣದಲ್ಲಿ ನೂರು ಪೌಂಡನ್ನು ಕಳೆದುಕೊಳ್ಳುವುದಕ್ಕೆ ನಿಮಗೆ ತುಂಬಾ ಅಂಜಿಕೆಯಾಗಬಹುದು. ಆದರೆ ಪ್ರಪಂಚದಲ್ಲಿ ತಮ್ಮ ಸರ್ವಸ್ವ ಹೋದರೂ ಕಣ್ಣನ್ನು ಕೂಡ ಮಿಟಕಿಸದವರು ಕೆಲವರು ಇರಬಹುದು. ಅಂತಹ ವ್ಯಕ್ತಿಗಳು ಹಿಂದೆ ಇದ್ದರು; ಈಗಲೂ ಕೂಡ ಇರುವರು. ನಿಮ್ಮಂತೆಯೆ ಇನ್ನೊಬ್ಬರನ್ನು ಅಳೆಯಲು ಏಕೆ ಸಾಹಸ ಪಡುವಿರಿ? ನಿಮಗಿರುವ ಬಂಧನಗಳನ್ನು ನೀವು ಬಾಚಿ ತಬ್ಬಿಕೊಂಡಿರುವಿರಿ. ಈ ಪ್ರಾಪಂಚಿಕ ಕ್ಷುದ್ರ ವಿಷಯಗಳು ನಿಮ್ಮ ಜೀವನದ ಗುರಿಯ ಪರಾಮಾವಧಿ ಯಾಗಿರಬಹುದು. ನೀವು ಬೇಕಾದರೆ ಅವನ್ನು ಸ್ವೀಕರಿಸಬಹುದು. ನೀವು ಬಯಸಿದಂತೆ ನಿಮಗೆ ಆಗುವುದು. ಆದರೆ ಸತ್ಯವನ್ನು ಕಂಡ ಕೆಲವರು ಇರುವರು. ಈ ಬಂಧನಗಳಲ್ಲಿ ಅವರು ಬಾಳಲಾರರು. ಇವುಗಳೇನೆಂಬುದು ಅವರಿಗೆ ಗೊತ್ತಾಗಿದೆ. ಇವುಗಳನ್ನು ಮೀರಿ ಹೋಗಬೇಕೆನ್ನುವರು. ಸಮಸ್ತ ಸುಖಭೋಗಗಳಿಂದ ತುಂಬಿದ ಜಗತ್ತು ಅವರ ಪಾಲಿಗೆ ಒಂದು ಕೆಸರಿನ ಗುಂಡಿಯಂತೆ. ನಿಮ್ಮ ಭಾವನೆಗಳಿಂದ ಅವರನ್ನು ಬಂಧಿಸಲು ಏಕೆ ಇಚ್ಛಿಸುವಿರಿ? ಈ ಸ್ವಭಾವದಿಂದ ನೀವು ಒಮ್ಮೆಯೇ ಪಾರಾಗಬೇಕು. ಪ್ರತಿಯೊಬ್ಬರಿಗೂ ಇರುವುದಕ್ಕೆ ಒಂದು ಸ್ಥಳವನ್ನು ಕೊಡಿ.

ದಕ್ಷಿಣ ಸಮುದ್ರ ದ್ವೀಪಗಳ ಸಮೀಪದಲ್ಲಿ ಬಿರುಗಾಳಿಗೆ ಸಿಕ್ಕಿದ ಕೆಲವು ಹಡಗುಗಳ ಕಥೆಯನ್ನು ನಾನು ಒಮ್ಮೆ ಓದಿದೆ. ಇಲಸ್ಟ್ರೇಟೆಡ್​ ಲಂಡನ್​ ನ್ಯೂಸ್​ನಲ್ಲಿ ಅದರ ಒಂದು ಚಿತ್ರವಿತ್ತು. ಇಂಗ್ಲೀಷಿನವರ ಹಡಗೊಂದು ಉಳಿದು ಉಳಿದ ಎಲ್ಲಾ ಹಡಗುಗಳೂ ನಾಶವಾದುವು. ಅದು ಈ ಪ್ರಚಂಡ ಬಿರುಗಾಳಿಯಲ್ಲಿ ಮುಂದು ವರಿಯುತ್ತಿತ್ತು. ಮುಳುಗುತ್ತಿದ್ದ ಹಡಗಿನಲ್ಲಿದ್ದ ಜನರು ಮೇಲಿನ ಅಂತಸ್ತಿನ ಮೇಲೆ ನಿಂತು ಮುಂದೆ ಹೋಗುತ್ತಿದ್ದ ಹಡಗಿನ ಜನರನ್ನು ನೋಡಿ ಹುರಿದುಂಬಿಸುತ್ತಿದ್ದ ಒಂದು ಚಿತ್ರವಿತ್ತು. ಅದರಂತೆ ಧೈರ್ಯವಾಗಿರಿ, ಉದಾರಿಗಳಾಗಿರಿ. ನೀವಿರುವ ಹಳ್ಳಕ್ಕೆ ಮತ್ತೊಬ್ಬರನ್ನು ಎಳೆಯಬೇಡಿ. ಮತ್ತೊಂದು ಮೂಢ ಭಾವನೆ ಯಾವುದೆಂದರೆ, ನಮ್ಮ ಅಲ್ಪ ವ್ಯಕ್ತಿತ್ವವನ್ನು ಕಳೆದುಕೊಂಡರೆ, ನೀತಿಯೆ ಇರುವುದಿಲ್ಲ, ಮಾನವ ಕುಲದ ಉದ್ಧಾರವೇ ಆಗುವುದಿಲ್ಲ, ಎಂಬುದು. ಇದುವರೆಗೂ ಏನೋ ಎಲ್ಲರೂ ಮಾನವ ಜನಾಂಗಕ್ಕಾಗಿ ಬಹಳ ಸಾಯುತ್ತಿದ್ದಂತೆ! ದೇವರು ನಿಮಗೆ ಒಳ್ಳೆಯದು ಮಾಡಲಿ. ಪ್ರತಿಯೊಂದು ದೇಶದಲ್ಲಿಯೂ ಇನ್ನೂರು ಜನ ಸ್ತ್ರೀ ಪುರುಷರು ಸತ್ಯವಾಗಿ ಮಾನವನ ಕಲ್ಯಾಣವನ್ನು ಕುರಿತು ಚಿಂತಿಸಿದರೆ ಸ್ವರ್ಣಯುಗ ಐದೇ ದಿನದಲ್ಲಿ ಉದಿಸುತ್ತಿತ್ತು. ನಾವು ಮಾನವ ಜನಾಂಗಕ್ಕಾಗಿ ಹೇಗೆ ಪ್ರಾಣ ಕೊಡುತ್ತಿರುವೆವು ಎಂಬುದು ನಮಗೆ ಗೊತ್ತಿದೆ. ಇವುಗಳೆಲ್ಲ ಒಣ ವೇದಾಂತ, ಮತ್ತೇನೂ ಅಲ್ಲ. ಯಾರು ತಮ್ಮ ಅಲ್ಪ ವ್ಯಕ್ತಿತ್ವದ ಸಲುವಾಗಿ ಸ್ವಲ್ಪವೂ ಗಮನ ಕೊಡುವುದಿಲ್ಲವೋ, ಅವರೇ ಮಾನವ ವರ್ಗದ ಅತ್ಯುತ್ತಮ ಉಪಕಾರಿಗಳು. ಎಷ್ಟು ಹೆಚ್ಚಾಗಿ ಪುರುಷರು ಮತ್ತು ಸ್ತ್ರೀಯರು ತಮ್ಮ ಸಲುವಾಗಿ ಆಲೋಚಿಸುತ್ತಾರೆಯೋ, ಮತ್ತೊಬ್ಬರಿಗೆ ಉಪಕಾರ ಮಾಡುವುದು ಅಷ್ಟು ಕಡಿಮೆ. ಒಂದು ನಿಃಸ್ವಾರ್ಥತೆ, ಮತ್ತೊಂದು ಸ್ವಾರ್ಥ. ಅಲ್ಪಭೋಗಗಳ ದಾಸನಾಗಿ ಅಂತಹ ಸ್ಥಿತಿ ಮತ್ತೆ ಮತ್ತೆ ಬರಬೇಕೆಂದು, ಅದು ಮುಂದೆಯೂ ಇರಬೇಕೆಂದು ಆಶಿಸುವುದು ಶುದ್ಧ ಸ್ವಾರ್ಥ. ಈ ಭಾವನೆಗಳು ಸತ್ಯಾಭಿಲಾಷೆಯಿಂದ ಬಂದುದಲ್ಲ, ಮತ್ತೊಬ್ಬರ ಉಪಕಾರಕ್ಕೋಸುಗವಾಗಿಯೂ ಈ ಭಾವಗಳು ಬರಲಿಲ್ಲ. “ಎಲ್ಲವೂ ನನಗಿರಲಿ. ನಾನು ಇನ್ನೊಬ್ಬನಿಗೆ ಲೆಕ್ಕಿಸುವುದಿಲ್ಲ” ಎಂಬ ಮಾನವನ ಹೃದಯದಲ್ಲಿರುವ ಸ್ವಾರ್ಥ ಭಾವನೆಯೇ ಇದಕ್ಕೆ ಮುಖ್ಯ ಕಾರಣ. ನನಗೆ ತೋರುವುದು ಹೀಗೆ. ಒಂದು ಸಣ್ಣ ಮೃಗಕ್ಕೆ ಉಪಕಾರ ಆಗುವ ಹಾಗಿದ್ದರೆ, ತಮ್ಮ ನೂರು ಜನ್ಮಗಳನ್ನಾದರೂ ತೆರಬಲ್ಲವರಾಗಿದ್ದ, ಪೂರ್ವಕಾಲದ ಮಹಾಮಹಿಮರಾದ ಋಷಿಗಳಂತಹ ದೇವದೂತರನ್ನು, ಧಾರ್ಮಿಕ ವ್ಯಕ್ತಿಗಳನ್ನು ನಾನು ನೋಡಬಯಸುವೆನು! ನೀತಿಯನ್ನು, ಪರೋಪಕಾರದ ವಿಚಾರವನ್ನು ನೀವು ಮಾತನಾಡುವಿರಲ್ಲ! ಕೆಲಸಕ್ಕೆ ಬಾರದ ಇಂದಿನ ಕಾಲದ ಮಾತು! ಸಾಕಾರ ದೇವರಲ್ಲಿ ಆಗಲಿ ಅಥವಾ ಜೀವಾತ್ಮನಲ್ಲಿ ಆಗಲಿ ನಂಬಿಕೆ ಇಡದ ಅದರ ವಿಚಾರವನ್ನು ಕೂಡ ಎತ್ತದ, ಅತಿ ಧರ್ಮಾತ್ಮರಾದಂಥ ಬುದ್ಧನಂತಹ ವ್ಯಕ್ತಿಗಳನ್ನು ನಾನು ನೋಡಬಯಸುವೆನು. ಸಂಪೂರ್ಣ ಆಜ್ಞೇಯತಾವಾದಿಯಾಗಿದ್ದ ನವನು. ಆದರೂ ಕೂಡ ಯಾರಿಗೆ ಬೇಕಾದರೂ ತನ್ನ ಜೀವನವನ್ನು ಧಾರೆ ಎರೆಯಲು ಸಿದ್ಧನಾಗಿದ್ದನು. ಸರ್ವರ ಹಿತಕ್ಕೋಸುಗ ತಾನು ಬದುಕಿರುವವರೆಗೂ ದುಡಿದನು. ಎಲ್ಲರ ಹಿತಚಿಂತನೆಯೇ ಅವನ ಬಯಕೆಯಾಗಿತ್ತು. ಬಹುಜನರ ಹಿತಕ್ಕೆ, ಬಹುಜನರ ಸುಖಕ್ಕೆ ಅವನು ಜನ್ಮವೆತ್ತಿದನು ಎಂದು ಅವನ ಜೀವನ ಚರಿತ್ರೆಕಾರರು ಸರಿಯಾಗಿ ವರ್ಣಿಸಿರುವರು. ತನ್ನ ಸ್ವಂತ ಮುಕ್ತಿಯನ್ನು ಸಂಪಾದಿಸಲೋಸುಗ ತಪಸ್ಸು ಮಾಡಲು ಕಾಡಿಗೆ ಹೋದವನಲ್ಲ. ಪ್ರಪಂಚ ದುಃಖದ ಕಾಡುಕಿಚ್ಚಿನಲ್ಲಿ ಸಿಕ್ಕಿದೆ, ಅದರಿಂದ ಪಾರಾಗಲು ಒಂದು ದಾರಿಯನ್ನು ಕಂಡುಹಿಡಿಯಬೇಕೆಂಬುದೇ ಅವನ ಬಯಕೆ ಯಾಗಿತ್ತು. ಈ ಪ್ರಪಂಚದಲ್ಲಿ ಏತಕ್ಕೆ ಎಷ್ಟೊಂದು ದುಃಖವಿದೆ ಎಂಬುದೇ ಅವನ ಜೀವನವನ್ನೆಲ್ಲ ಆವರಿಸಿದ ಒಂದು ಪ್ರಶ್ನೆಯಾಗಿತ್ತು. ಬುದ್ಧದೇವನಷ್ಟು ನಾವು ನೀತಿವಂತರೆಂದು ನೀವು ಭಾವಿಸುತ್ತೀರೇನು?

ಮನುಷ್ಯನು ಸ್ವಾರ್ಥಪರನಾದಷ್ಟೂ ಅಧರ್ಮಿಯಾಗುವನು. ಅದರಂತೆಯೇ ಜನಾಂಗಗಳೂ ಕೂಡ. ತನ್ನ ಸುಖವೇ ಪರಮಾವಧಿ ಎಂದು ತಿಳಿದುಕೊಂಡ ಜನಾಂಗವು ಪ್ರಪಂಚದಲ್ಲೆಲ್ಲ ಅತಿಕ್ರೂರವಾದುದು ಮತ್ತು ಹೀನವಾದುದು. ಅರೇಬಿಯದ ಪ್ರವಾದಿಯಿಂದ ಸ್ಥಾಪಿತವಾದ ಧರ್ಮದ ಅನುಯಾಯಿಗಳಷ್ಟು ಹೆಚ್ಚಾಗಿ ಈ ದ್ವೈತದಲ್ಲಿ ನಂಬಿಕೆಯಿಟ್ಟಿದ್ದ ಮತ್ತೊಬ್ಬರಿಲ್ಲ. ಇತರರನ್ನು ಕಂಡರೆ ಇವರಷ್ಟು ನಿರ್ದಯರಾಗಿ ವರ್ತಿಸಿದ, ಇವರಷ್ಟು ರಕ್ತದ ಕಾಲುವೆಯನ್ನು ಹರಿಸಿದ ಮತ್ತೊಂದು ಧರ್ಮವಿಲ್ಲ. ಯಾರು ಈ ಉಪದೇಶಗಳನ್ನು ನಂಬುವುದಿಲ್ಲವೋ ಅವರನ್ನು ಕೊಲ್ಲಬೇಕೆಂಬ ಸಿದ್ಧಾಂತ ಖೊರಾನಿನಲ್ಲಿಯೇ ಇದೆ. ಅಂತಹವನನ್ನು ಕೊಲ್ಲುವುದೊಂದು ದಯೆಯಿಂದ ಪ್ರೇರಿತವಾದ ಪರೋಪಕಾರ! ಚಿರ ಯೌವನ ದಲ್ಲಿರುವ ಸ್ತ್ರೀಯರು ಮತ್ತು ಎಲ್ಲಾ ವಿಧದ ವಿಷಯಭೋಗಗಳಿಂದ ತುಂಬಿ ತುಳುಕಾಡುತ್ತಿರುವ ಸ್ವರ್ಗಕ್ಕೆ ನಿಸ್ಸಂಶಯವಾಗಿ ಹೋಗಲು, ಮಹಮ್ಮದೀಯ ಧರ್ಮದಲ್ಲಿ ನಂಬದವರನ್ನು ಕೊಲ್ಲುವುದೊಂದೆ ನೇರವಾದ ಹಾದಿ. ಇಂತಹ ನಂಬಿಕೆಗಳ ಪರಿಣಾಮವಾಗಿ ಎಷ್ಟೊಂದು ರಕ್ತ ಹರಿದಿದೆ ಎಂದು ನೀವೇ ಆಲೋಚಿಸಿ ನೋಡಿ!

ಕ್ರೈಸ್ತಧರ್ಮದಲ್ಲಿ ಒರಟುತನವಿರುವುದು ಬಹಳ ಸ್ವಲ್ಪ. ಶುದ್ಧ ಕ್ರೈಸ್ತಧರ್ಮಕ್ಕೂ ವೇದಾಂತಕ್ಕೂ ವ್ಯತ್ಯಾಸವಿರುವುದು ಬಹಳ ಸ್ವಲ್ಪ. ಅಲ್ಲಿಯೂ ಏಕತ್ವದ ಭಾವನೆ ಇದೆ. ಆದರೆ ಕ್ರಿಸ್ತನು ಸಾಧಾರಣ ಜನರಿಗೆ ತಿಳಿಯಲು ಅನುಕೂಲವಾಗುವಂತೆ ಕೆಲವು ಸ್ಥೂಲಭಾವನೆಗಳನ್ನು ಕೊಟ್ಟು, ಕ್ರಮೇಣ ಅವರನ್ನು ಪರಮಾದರ್ಶದೆಡೆಗೆ ಒಯ್ಯುವುದಕ್ಕೋಸುಗ ದ್ವೈತ ಭಾವನೆಗಳನ್ನು ಉಪದೇಶಿಸಿದನು. “ಸ್ವರ್ಗದಲ್ಲಿರುವ ನಮ್ಮ ತಂದೆ” ಎಂದು ಬೋಧಿಸಿದ ದೇವದೂತನೇ “ನಾನು ಮತ್ತು ನನ್ನ ತಂದೆಯೂ ಒಂದೆ” ಎಂದು ಬೋಧಿಸಿದನು. “ನಾನು ಮತ್ತು ನನ್ನ ತಂದೆ ಒಂದೆ” ಎಂಬ ಭಾವಕ್ಕೆ ಹೋಗಬೇಕಾದರೆ “ಸ್ವರ್ಗದಲ್ಲಿರುವ ತಂದೆ” ಎಂಬ ಭಾವನೆಯ ಮಾರ್ಗದ ಮೂಲಕ ಎಂಬುದು ಆತನಿಗೆ ಗೊತ್ತಿತ್ತು. ಕ್ರೈಸ್ತಧರ್ಮದಲ್ಲಿ ಪ್ರೇಮ ಮತ್ತು ಆಶೀರ್ವಾದವೆ ಇತ್ತು. ಆದರೆ ಎಂದು ಒರಟುತನ ಅದರಲ್ಲಿ ತಲೆಹಾಕಿತ್ತೋ, ಅಂದು ಅದು ಅಧೋಗತಿಗಿಳಿದು, ಅರೇಬಿಯದ ಪ್ರವಾದಿಯ ಧರ್ಮಕ್ಕಿಂತ ಏನೂ ಉತ್ತಮವಾಗಲಿಲ್ಲ. ಈ ಅಲ್ಪಾತ್ಮನಿಗಾಗಿ ಹೋರಾಡುವುದು, ಅದನ್ನೇ ಅಪ್ಪಿಕೊಂಡಿ ರುವುದು, ಈ ಜನ್ಮದಲ್ಲಿ ಮಾತ್ರವಲ್ಲ, ಸತ್ತ ಮೇಲೆಯೂ ಹಿಂದಿನಂತೆ ಇರಬೇಕೆಂದು ಬಯಸುವುದು ನಿಜವಾಗಿಯೂ ಒರಟುತನ. ಇದನ್ನೇ ಅವರು ನಿಃ ಸ್ವಾರ್ಥತೆ ಎಂದು ಸಾರುವುದು, ನೀತಿಯ ತಳಹದಿ ಎನ್ನುವುದು. ಇದು ನೀತಿಯ ತಳಹದಿಯಾದರೆ ದೇವರು ನಮ್ಮನ್ನು ಕಾಪಾಡಲಿ! ಆದರೆ ಇದೇ ಒಂದು ಆಶ್ಚರ್ಯ – ಇನ್ನೂ ಅಲ್ಪಮತಿಗಳಾದ ಪುರುಷರು ಮತ್ತು ಸ್ತ್ರೀಯರು ಈ ಅಲ್ಪಾತ್ಮನು ಇಲ್ಲದೇ ಇದ್ದರೆ ನೀತಿಯೆಲ್ಲ ನಾಶವಾಗುವುದೆಂದು ತಿಳಿಯುವರು. ಈ ಕ್ಷುದ್ರಾತ್ಮನು ನಾಶವಾದ ಮೇಲೆ ಮಾತ್ರ ನೀತಿಯು ನಿಲ್ಲುವುದಾದರೆ ಅವರು ಸ್ತಂಭೀಭೂತರಾಗಿ ನಿಲ್ಲುವರು. ಎಲ್ಲಾ ಹಿತ ಚಿಂತನೆಯ ಮತ್ತು ಧಾರ್ಮಿಕ ಭಾವನೆಯ ಸಂಕೇತ ಪದವೆ ‘ನಾನಲ್ಲ’ ‘ನೀನು’ ಎಂಬುದು. ಸ್ವರ್ಗವಿದೆಯೋ ನರಕವಿದೆಯೋ ಯಾರಿಗೆ ಬೇಕು? ಆತ್ಮನಿರುವನೊ ಇಲ್ಲವೊ ಅದರಿಂದ ಏನು? ಶಾಶ್ವತವಾದೊಂದು ವಸ್ತುವಿದೆಯೋ ಇಲ್ಲವೋ ಯಾರಿಗೆ ಬೇಕು? ನಮ್ಮ ಕಣ್ಣೆದುರಿಗೆ ಪ್ರಪಂಚವಿದೆ. ಅದು ದುಃಖದಿಂದ ತುಂಬಿ ತುಳುಕಾಡುತ್ತಿದೆ. ಬುದ್ಧನಂತೆ ಅದರಲ್ಲಿ ಪ್ರವೇಶಿಸಿ ದುಃಖವನ್ನು ಕಡಿಮೆಮಾಡಲು ಪ್ರಯತ್ನಿಸಿ. ಇಲ್ಲದೇ ಇದ್ದರೆ ಆ ಮಹಾ ಪ್ರಯತ್ನದಲ್ಲಿ ನಿಮ್ಮ ಪ್ರಾಣವನ್ನು ಕೊಡಿ. ನಿಮ್ಮನ್ನು ನೀವು ಮರೆಯಿರಿ; ನೀವು ಆಸ್ತಿಕರೊ ನಾಸ್ತಿಕರೊ, ಆಜ್ಞೇಯತಾವಾದಿಗಳೊ, ವೇದಾಂತಿಗಳೊ, ಕ್ರೈಸ್ತರೊ, ಮಹಮ್ಮದೀಯರೊ, ಯಾರಾದರಾಗಲಿ–ಕಲಿಯಬೇಕಾದ ಮೊದಲ ಪಾಠವಿದು. ಮೊದಲು ಕ್ಷುದ್ರಾತ್ಮನ ನಾಶ. ಅನಂತರ ಅಲ್ಲಿ ನಿಜವಾದ ಆತ್ಮನ ಸ್ಥಾಪನೆ. ಎಲ್ಲರೂ ನಿಸ್ಸಂದೇಹವಾಗಿ ಕಲಿಯಬೇಕಾದ ಪಾಠ ಇದು.

ಎರಡು ಶಕ್ತಿಗಳು ಒಂದರ ಪಕ್ಕದಲ್ಲಿ ಮತ್ತೊಂದು ಸಮಾನಾಂತರ ದೂರದಲ್ಲಿ ಕೆಲಸ ಮಾಡುತ್ತಿರುವುವು. ಮೊದಲನೆಯದು ನಾನು ಎನ್ನುವುದು, ಎರಡನೆಯದು ನಾನಲ್ಲ ಎನ್ನುವುದು. ಈ ಶಕ್ತಿ ಮನುಷ್ಯರಲ್ಲಿ ಮಾತ್ರವಲ್ಲ ಪ್ರಾಣಿಗಳಲ್ಲೂ ತೋರು ವುದು; ಪ್ರಾಣಿಗಳಲ್ಲಿ ಮಾತ್ರವಲ್ಲ ಕ್ಷುತ್ರತಮ ಕೀಟಗಳಲ್ಲಿಯೂ ಇರುವುದು. ಮಾನವನ ಬಿಸಿರಕ್ತವನ್ನು ತನ್ನ ನಖಗಳಿಂದ ಬಗೆವ ಹುಲಿಯು ಕೂಡ ತನ್ನ ಮರಿಗಳ ಸಂರಕ್ಷಣಾರ್ಥವಾಗಿ ತನ್ನ ಪ್ರಾಣವನ್ನು ಬಲಿಕೊಡುವುದು. ತನ್ನ ನೆರೆಯವರ ಪ್ರಾಣ ವನ್ನು ಕೂಡ ತೆಗೆಯಲು ಹಿಂಜರಿಯದ ದುರ್ಭಾಗ್ಯನು, ಉಪವಾಸದಿಂದ ನರಳು ತ್ತಿರುವ ತನ್ನ ಹೆಂಡತಿ ಮಕ್ಕಳನ್ನು ಸಂರಕ್ಷಿಸಲು ಯಾವ ಅನುಮಾನವೂ ಇಲ್ಲದೆ ಬಹುಶಃ ತನ್ನ ಪ್ರಾಣವನ್ನು ಕೊಡುವನು. ಆದಕಾರಣ ಸೃಷ್ಟಿಯಲ್ಲೆಲ್ಲ ಈ ಎರಡು ಶಕ್ತಿಗಳು ಅಕ್ಕಪಕ್ಕದಲ್ಲಿಯೇ ಕೆಲಸಮಾಡುತ್ತಿರುವುವು. ಎಲ್ಲಿ ನೀವು ಒಂದನ್ನು ನೋಡುತ್ತೀರೋ, ಅಲ್ಲಿ ನೀವು ಮತ್ತೊಂದನ್ನು ನೋಡುತ್ತೀರಿ. ಒಂದು ಸ್ವಾರ್ಥ, ಮತ್ತೊಂದು ನಿಃಸ್ವಾರ್ಥತೆ. ಒಂದು ಆರ್ಜನೆ, ಮತ್ತೊಂದು ತ್ಯಾಗ. ಒಂದು ಸ್ವೀಕರಿಸು ವುದು, ಮತ್ತೊಂದು ದಾನ ಮಾಡುವುದು. ಕ್ಷುದ್ರತಮ ಕ್ರಿಮಿಯಿಂದ ಹಿಡಿದು ಪವಿತ್ರತಮ ಮಹರ್ಷಿಯ ವರೆಗೆ, ವಿಶ್ವವು ಈ ಎರಡು ಶಕ್ತಿಗಳ ಆಟದ ಭೂಮಿ. ಯಾರೂ ಇದನ್ನು ತೋರಬೇಕಾಗಿಲ್ಲ. ಎಲ್ಲರಿಗೂ ಇದು ಸ್ವತಃ ವೇದ್ಯ.

ಈ ಎರಡು ಶಕ್ತಿಗಳಲ್ಲಿ ಯಾವುದೋ ಒಂದನ್ನು ಮಾತ್ರ ತನ್ನ ಕೆಲಸದ ತಳಹದಿಯನ್ನಾಗಿ ಮಾಡಿಕೊಳ್ಳುವುದಕ್ಕೆ ಮತ್ತು ಅದರ ಆಧಾರದ ಮೇಲೆಯೇ ಪ್ರಪಂಚದ ವಿಕಾಸವನ್ನೆ ವಿವರಿಸುವುದಕ್ಕೆ ಯಾವ ಜನಾಂಗಕ್ಕೆ, ಕೋಮಿಗೆ ಅಧಿಕಾರವಿದೆ? ಕಾಮ, ಹೋರಾಟ, ಸ್ಪರ್ಧೆ ಇವುಗಳೇ ವಿಶ್ವದಲ್ಲಿರುವ ಕ್ರಿಯೆಗೆಲ್ಲ ಮೂಲವೆನ್ನುವುದಕ್ಕೆ ಅವರಿಗೆ ಯಾವ ಅಧಿಕಾರವಿದೆ? ಇವುಗಳು ಇವೆ ಎನ್ನುವುದನ್ನು ನಾವು ಅಲ್ಲಗಳೆಯುವುದಿಲ್ಲ. ಆದರೆ ಮತ್ತೊಂದು ಶಕ್ತಿ ಇಲ್ಲವೆನ್ನುವುದಕ್ಕೆ ಯಾವ ಅಧಿಕಾರವಿದೆ? ಪ್ರೇಮ, ನಾನಲ್ಲವೆನ್ನುವುದು, ತ್ಯಾಗ, ಇವೇ ಪ್ರಪಂಚದಲ್ಲಿರುವ ವಾಸ್ತವಿಕ ಶಕ್ತಿ ಎಂಬುದನ್ನು ಯಾರಾದರೂ ನಿರಾಕರಿಸಬಲ್ಲರೆ? ಮತ್ತೊಂದೇ ಪ್ರೇಮ ಶಕ್ತಿಯ ತಪ್ಪು ಬಳಕೆ. ಪ್ರೇಮದ ಶಕ್ತಿಯು ಸ್ಪರ್ಧೆಯನ್ನು ತರುವುದು. ಸ್ಪರ್ಧೆಯ ಮೂಲ ಪ್ರೇಮದಲ್ಲಿದೆ. ಪಾಪಕ್ಕೆ ಮೂಲ ತ್ಯಾಗದಲ್ಲಿದೆ. ಪಾಪದ ಕರ್ತೃವೇ ಪುಣ್ಯ. ಕೊನೆಯ ಗುರಿಯು ಕೂಡ ಪುಣ್ಯವೆ. ದಾರಿ ತಪ್ಪಿದ ಪುಣ್ಯಶಕ್ತಿಯೇ ಪಾಪ. ಇನ್ನೊಬ್ಬನನ್ನು ಖೂನಿ ಮಾಡುವ ಮನುಷ್ಯನಿಗೆ ತನ್ನ ಮಗುವಿನ ಮೇಲಿನ ಪ್ರೇಮ ಬಹುಶಃ ಹಾಗೆ ಮಾಡುವಂತೆ ಪ್ರೇರೇಪಿಸಿರಬಹುದು. ಅವನ ಪ್ರೇಮವು ಪ್ರಪಂಚದಲ್ಲಿರುವ ಕೋಟಿ ಮಾನವರನ್ನು ತೊರೆದು ಆ ಮಗು ಒಂದರ ಮೇಲೆ ಕೇಂದ್ರೀಕೃತವಾಗಿ ಮಿತವಾಗಿ ರುವುದು. ಸಾಂತವೋ ಅನಂತವೋ ಅದು ಯಾವುದಾದರೇನು? ಅಲ್ಲಿರುವುದು ಒಂದೇ, ಅದೇ ಪ್ರೇಮ.

ಆದಕಾರಣ ವಿಶ್ವಪ್ರವರ್ತಕಶಕ್ತಿ, ಅದು ಯಾವ ರೀತಿಯಲ್ಲಿ ಪ್ರಕಟವಾದರೂ, ಒಂದು ಅತ್ಯಮೋಘವಾದ ವಸ್ತು. ಅದೇ ನಿಃಸ್ವಾರ್ಥ, ತ್ಯಾಗ, ಪ್ರೇಮ, ಸತ್ಯ. ಅದೊಂದೆ ವಿಶ್ವದಲ್ಲಿ ಚಿರಜಾಗ್ರತ ಶಕ್ತಿ. ಆದಕಾರಣವೆ ವೇದಾಂತಗಳಾದ ನಾವು ಏಕತ್ವವನ್ನು ಪ್ರತಿಪಾದಿಸುವುದು. ನಾವು ಈ ವಿವರಣೆಯನ್ನು ಪ್ರತಿಪಾದಿಸುತ್ತೇವೆ. ಏಕೆಂದರೆ ಪ್ರಪಂಚಕ್ಕೆ ಎರಡು ಕಾರಣಗಳನ್ನು ಒಪ್ಪಲಾರೆವು. ಆ ಸುಂದರ ಅತ್ಯ ಮೋಘ ಪ್ರೇಮ, ಮಿತಿಗೊಳಗಾಗಿ ಪಾಪದಂತೆ, ಕ್ರೌರ್ಯದಂತೆ, ಕಾಣುವುದೆಂದು ಒಪ್ಪಿಕೊಂಡರೆ, ಪ್ರೇಮಶಕ್ತಿ ಒಂದರಿಂದ ಜಗತ್ತನ್ನೆಲ್ಲ ವಿವರಿಸಿದಂತೆ ಆಗುತ್ತದೆ. ಇಲ್ಲದೆ ಇದ್ದರೆ ಪ್ರಪಂಚಕ್ಕೆ ಎರಡು ಕಾರಣಗಳಿವೆ ಎಂಬುದನ್ನು ನಾವು ಒಪ್ಪಬೇಕಾ ಗುತ್ತದೆ. ಒಂದು ಪಾಪ, ಒಂದು ಪುಣ್ಯ, ಒಂದು ಪ್ರೇಮ ಮತ್ತೊಂದು ದ್ವೇಷ. ಯಾವುದು ಹೆಚ್ಚು ಯುಕ್ತಿಯುಕ್ತವಾದುದು? ನಿಜವಾಗಿಯೂ ಏಕಶಕ್ತಿಯ ಸಿದ್ಧಾಂತ.

ಇನ್ನು ಬಹುಶಃ ದ್ವೈತಕ್ಕೆ ಸೇರದ ವಿಷಯಗಳನ್ನಾಲೋಚಿಸಲು ಮುಂದುವರಿ ಯೋಣ. ದ್ವೈತದೊಂದಿಗೆ ನಾನು ದೀರ್ಘಕಾಲವಿರಲಾರೆನು. ನೀತಿ ಮತ್ತು ತ್ಯಾಗದ ಮಹೋತ್ತಮ ಆದರ್ಶಗಳು ಅತಿ ಗಹನವಾದ ತತ್ತ್ವಸಿದ್ಧಾಂತದೊಂದಿಗೆ ಹೊಂದಿ ಕೊಳ್ಳಬಲ್ಲದೆಂಬುದನ್ನು ತೋರುವುದೇ ನನ್ನ ಉದ್ದೇಶ. ನೀತಿಗೋಸುಗ ನಿಮ್ಮ ತತ್ತ್ವವನ್ನು ಕೆಳಗೆ ಇಳಿಸಬೇಕಾಗಿಲ್ಲ. ಆದರೆ ನೀತಿಯ ನಿಜವಾದ ತಳಹದಿ ಸಿಕ್ಕಬೇಕಾದರೆ ಅತ್ಯುತ್ತಮ ತತ್ತ್ವ ಮತ್ತು ವೈಜ್ಞಾನಿಕ ಸಿದ್ಧಾಂತ ಬೇಕಾಗಿದೆ. ಮಾನವನ ಜ್ಞಾನವು ಮಾನವನ ಯೋಗಕ್ಷೇಮಕ್ಕೆ ವಿರೋಧವಾಗಿಲ್ಲ. ಅದಲ್ಲದೆ ಜೀವನದ ಎಲ್ಲಾ ಕಾರ್ಯಕ್ಷೇತ್ರಗಳಲ್ಲಿ ನಮ್ಮನ್ನು ಕಾಪಾಡುವುದು ಜ್ಞಾನವೊಂದೇ. ಜ್ಞಾನದಲ್ಲೆ ಪೂಜೆ ಇದೆ. ಹೆಚ್ಚು ನಮಗೆ ತಿಳಿದಷ್ಟು ಅದರಿಂದ ನಮಗೆ ಒಳ್ಳೆಯದು. ತೋರಿಕೆಗೆ ಪಾಪರೂಪದಲ್ಲಿ ಕಾಣುವಂತೆ ಮಾಡುವುದೇ ಅನಂತವನ್ನು ಸಾಂತವಾಗಿ ಮಾಡುವ ಪ್ರಯತ್ನ ಎಂದು ವೇದಾಂತ ಸಾರುತ್ತದೆ. ಸಣ್ಣ ಸಣ್ಣ ಕಾಲುವೆಯೊಳಗೆ ಹೋಗಿ ಸಾಂತವಾಗಿ ಪಾಪದಂತೆ ತೋರುವ ಪ್ರೇಮವೆ, ಕ್ರಮೇಣ ಮತ್ತೊಂದು ಕಡೆಯಿಂದ ಹರಿದು ಹೊರಬಂದು ದೇವರಂತೆ ಪ್ರಕಾಶಿಸುವುದು. ಈ ಎಲ್ಲಾ ತೋರಿಕೆಯ ಪಾಪಕ್ಕೆ ಕಾರಣವು ನಮ್ಮಲ್ಲಿಯೇ ಇರುವುದೆಂದು ವೇದಾಂತ ಹೇಳುತ್ತದೆ. ಯಾವ ಅತಿಪ್ರಾಕೃತ ದೇವತೆಗಳನ್ನೂ ದೂರಬೇಡಿ, ಹತಾಶರೂ ವಿಶಾದಿಗಳೂ ಆಗಬೇಡಿ, ಅಥವಾ ಯಾರಾದರೂ ಬಂದು ನಮ್ಮ ಸಹಾಯಕ್ಕೆ ಕೈ ನೀಡಿದ ಹೊರತು ಮುಕ್ತರಾಗುವುದಕ್ಕೆ ಸಾಧ್ಯವಿಲ್ಲವೆನ್ನುವ ಸ್ಥಳದಲ್ಲಿ ನಾವು ಇರುವೆವು ಎಂದೂ ಯೋಚಿಸಬೇಡಿ. ಹಾಗೆಂದಿಗೂ ಇಲ್ಲ. ನಾವು ರೇಶ್ಮೆಯ ಹುಳುವಿನಂತೆ ಇರುವೆವು. ನಮ್ಮ ಸತ್ತ್ವದಿಂದಲೇ ನಾವು ನೂಲನ್ನು ಎಳೆಯುವೆವು, ಗೂಡನ್ನು ಹೆಣೆಯುವೆವು. ಕಾಲಕ್ರಮದಲ್ಲಿ ಅದರಲ್ಲಿ ಬಂಧಿತರಾಗುವೆವು. ಆದರೆ ಇದು ಎಂದೆಂದಿಗೂ ಇರುವುದಿಲ್ಲ. ನಾವು ಆಧ್ಯಾತ್ಮಿಕ ಅನುಭವವನ್ನು ಆ ಗೂಡಿನಲ್ಲಿ ಪಡೆದು ಮುಕ್ತರಾಗಿ ಚಿಟ್ಟೆಯಂತೆ ಹೊರಗೆ ಬರುವೆವು. ಈ ಕರ್ಮದ ಬಲೆಯನ್ನು ನಾವೇ ನಮ್ಮ ಸುತ್ತಲೂ ಹೆಣೆದುಕೊಂಡಿರುವೆವು. ಅಜ್ಞಾನದಿಂದ ನಾವು ಬದ್ಧರೆಂದು ಭ್ರಮಿಸಿ ಸಹಾಯಕ್ಕಾಗಿ ಕಂಬನಿಗರೆದು ಅಳುವೆವು. ಆದರೆ ಸಹಾಯವು ಹೊರಗಿನಿಂದ ಬರುವುದಿಲ್ಲ. ಅದು ನಮ್ಮಿಂದಲೇ ಬರುವುದು. ಪ್ರಪಂಚದಲ್ಲಿರುವ ದೇವರಿಗೆಲ್ಲ ಗೋಳಿಡಿ. ನಾನೂ ಅನೇಕ ವರ್ಷಗಳು ಗೋಳಿಟ್ಟೆ, ನನಗೆ ಸಹಾಯ ಕೊನೆಗೆ ದೊರಕಿತು. ಆದರೆ ಆ ಸಹಾಯ ಬಂದುದು ನನ್ನ ಒಳಗಿನಿಂದಲೆ. ಭ್ರಾಂತಿಯಿಂದ ನಾನು ಏನನ್ನು ಮಾಡಿದ್ದೆನೋ ಅದನ್ನು ಬಿಡಬೇಕಾಯಿತು. ಅದೊಂದೇ ದಾರಿ. ನನ್ನ ಸುತ್ತಲೂ ಬೀಸಿ ಕೊಂಡಿರುವ ಬಲೆಯನ್ನು ನಾನೇ ಕತ್ತರಿಸಬೇಕಾಯಿತು. ಇದನ್ನು ಮಾಡುವು ದಕ್ಕೆ ಶಕ್ತಿ ನಮ್ಮಲ್ಲಿಯೇ ಇರುವುದು. ಸನ್ಮಾರ್ಗದಲ್ಲಿಯೋ ಅಥವಾ ದುರ್ಮಾರ್ಗ ದಲ್ಲಿಯೋ ಹೋದ ನನ್ನ ಯಾವ ಆಶಯವೂ ವ್ಯರ್ಥವಾಗಿಲ್ಲವೆಂಬುದು ನಿಸ್ಸಂಶಯವಾಗಿದೆ. ಆದರೆ ನನ್ನ ಪೂರ್ವ ಪಾಪ ಪುಣ್ಯಗಳ ಕರ್ಮದ ಪ್ರತಿಫಲ ನಾನು. ನಾನು ಜೀವನದಲ್ಲಿ ಎಷ್ಟೋ ತಪ್ಪನ್ನು ಮಾಡಿರುವೆ. ಆದರೆ ಇದನ್ನು ಜ್ಞಾಪಕದಲ್ಲಿಡಿ. ಹಿಂದಿನ ಯಾವ ಒಂದು ತಪ್ಪೂ ಇಲ್ಲದೆ ನಾನು ಇಂದಿನಂತೆ ಆಗುತ್ತಿರಲಿಲ್ಲ. ಆದಕಾರಣ ಹಾಗೆ ತಪ್ಪನ್ನು ಮಾಡಿದುದು ನನಗೆ ಸಮಾಧಾನ ಕರವಾಗಿದೆ. ನೀವು ಮನೆಗೆ ಹೋಗಿಬೇಕೆಂದು ತಪ್ಪುಗಳನ್ನು ಮಾಡಿ ಎಂದು ನಿಮಗೆ ಹೇಳುವುದಿಲ್ಲ. ಈ ರೀತಿಯಲ್ಲಿ ನನ್ನನ್ನು ತಪ್ಪು ತಿಳಿದುಕೊಳ್ಳಬೇಡಿ. ಹಿಂದೆ ಮಾಡಿದ ತಪ್ಪಿಗಾಗಿ ಸಪ್ಪೆ ಮುಖ ಹಾಕಿಕೊಳ್ಳಬೇಡಿ, ಕೊನೆಗೆ ಎಲ್ಲವೂ ಸರಿಯಾಗುವುದೆಂದು ತಿಳಿಯಿರಿ. ಹಾಗಲ್ಲದೆ ಬೇರೆ ಆಗಲಾರದು, ಏಕೆಂದರೆ ಒಳ್ಳೆಯದೇ ನಮ್ಮ ಸ್ವಭಾವ, ಪಾವಿತ್ರ್ಯವೇ ನಮ್ಮ ಸ್ವಭಾವ. ಸ್ವಭಾವವನ್ನು ಎಂದಿಗೂ ನಾಶ ಮಾಡುವುದಕ್ಕೆ ಆಗುವುದಿಲ್ಲ. ನಮ್ಮ ಮೂಲ ಸ್ವಭಾವ ಯಾವಾಗಲೂ ಒಂದೇ ಸಮನಾಗಿರುವುದು.

ನಾವು ತಿಳಿದುಕೊಳ್ಳಬೇಕಾಗಿರುವುದು ಇದು: ನಾವು ಯಾವುದನ್ನು ತಪ್ಪು ಮತ್ತು ಪಾಪವೆನ್ನುವೆವೊ ಬಲಹೀನರಾಗಿರುವುದರಿಂದ ನಾವು ಅದನ್ನು ಮಾಡುತ್ತೇವೆ. ನಾವು ಅಜ್ಞಾನಿಗಳಾದ ಕಾರಣವೆ ಬಲಹೀನರಾಗಿರುವುದು. ಅದನ್ನು ತಪ್ಪೆಂದು ಕರೆಯುವುದು ಒಳ್ಳೆಯದು. ಪಾಪವೆಂಬ ಪದ ಹಿಂದಿನ ಕಾಲದಲ್ಲಿ ಬಹಳ ಒಳ್ಳೆಯ ಪದವಾಗಿ ಕಂಡರೂ, ನನಗೆ ಅಂಜಿಕೆಯಾಗುವ ಯಾವುದೋ ಒಂದು ವಾಸನ ಇದೆ ಅದಕ್ಕೆ. ನಮ್ಮನ್ನು ಯಾರು ಅಜ್ಞಾನಿಗಳನ್ನಾಗಿ ಮಾಡುತ್ತಾರೆ? ನಾವೆ ನಮ್ಮ ಕಣ್ಣಿನ ಮೇಲೆ ಕೈ ಇಟ್ಟುಕೊಂಡು ಕತ್ತಲೆ ಎಂದು ಅಳುತ್ತೇವೆ. ಕೈಗಳನ್ನು ತೆಗೆಯಿರಿ. ಬೆಳಕು ಅಲ್ಲೇ ಇರುವುದು. ಆತ್ಮದ ಸ್ವಯಂ ಜ್ಯೋತಿಯ ಬೆಳಕು ಯಾವಾಗಲೂ ಇದ್ದೇ ಇರುವುದು. ನಿಮ್ಮ ಆಧುನಿಕ ವಿಜ್ಞಾನಿಗಳು ಏನೆನ್ನುತ್ತಾರೆ ನಿಮಗೆ ತಿಳಿಯದೆ? ವಿಕಾಸಕ್ಕೆ ಕಾರಣ ವೇನು? ಆಸೆ. ಪ್ರಾಣಿಯು ಏನನ್ನೊ ಮಾಡಬೇಕೆಂದು ಇಚ್ಛಿಸುವುದು, ಆದರೆ ವಾತಾವರಣ ಸರಿಯಾಗಿಲ್ಲ. ಆದಕಾರಣ ಹೊಸ ದೇಹವನ್ನು ಪಡೆಯುತ್ತದೆ. ಹೊಸ ದೇಹಕ್ಕೆ ಕಾರಣ ಯಾವುದು? ಪ್ರಾಣಿಯೆ, ಅದರ ಇಚ್ಛೆಯೆ ಕಾರಣ. ಕ್ಷುದ್ರತಮ ಅಣುವಿನಿಂದ ನೀವು ವಿಕಾಸಹೊಂದಿರುವಿರಿ. ನಿಮ್ಮ ಇಚ್ಛಾಶಕ್ತಿಯನ್ನು ಪ್ರಯೋಗಿಸಿ, ಅದು ನಿಮ್ಮನ್ನು ಇನ್ನೂ ಮೇಲೆ ಕರೆದೊಯ್ಯುವುದು. ಇಚ್ಛಾಶಕ್ತಿ ಸರ್ವಶಕ್ತಿಮಾನ್​. ಅದು ಸರ್ವಶಕ್ತವಾದರೆ ನಾನು ಎಲ್ಲವನ್ನೂ ಏಕೆ ಮಾಡಬಾರದು ಎಂದು ಕೇಳ ಬಹುದು. ಆದರೆ ನೀವು ನಿಮ್ಮ ಅಲ್ಪ ವ್ಯಕ್ತಿತ್ವವನ್ನು ಮಾತ್ರ ಕುರಿತು ಯೋಚಿಸು ತ್ತಿರುವಿರಿ. ಕ್ಷುದ್ರತಮ ಕೀಟದ ಸ್ಥಿತಿಯಿಂದ ಹಿಡಿದು ಮಾನವ ಜನ್ಮದವರೆವಿಗೂ ಆದ ನಿಮ್ಮ ವಿಕಾಸವನ್ನು ನೀವೆ ಪರಿಗಣಿಸಿ. ಈ ಅವಸ್ಥೆಗಳನ್ನೆಲ್ಲ ಮಾಡಿದವರಾರು? ನಿಮ್ಮ ಇಚ್ಛಾಶಕ್ತಿಯೆ. ಹಾಗಾದರೆ ಅದು ಸರ್ವಶಕ್ತಿಮಾನ್​ ಎಂಬುದನ್ನು ನಿರಾಕರಿಸುವಿರಾ ನೀವು? ಇಷ್ಟು ಮೇಲಿನ ಸ್ಥಿತಿಯವರೆವಿಗೂ ಬರುವಂತೆ ಮಾಡಿದುದು ಇದಕ್ಕಿಂತಲೂ ಮೇಲೆ ಹೋಗುವಂತೆ ಮಾಡಬಲ್ಲದು. ನಿಮಗೆ ಬೇಕಾಗಿರುವುದು ಇಚ್ಛಾಶಕ್ತಿಯನ್ನು ವೃದ್ಧಿಗೊಳಿಸುವ ಶೀಲ.

“ನಿಮ್ಮ ಸ್ವಭಾವ ಕೆಟ್ಟದು; ನೀವು ಹೋಗಿ ಹರಕು ಬಟ್ಟೆ ಹಾಕಿಕೊಂಡು ಮೈಗೆ ಬೂದಿ ಬಳಿದುಕೊಂಡು, ಬದುಕಿರುವ ತನಕ ಕುಳಿತು ಅಳಬೇಕು, ಏಕೆಂದರೆ ಹಿಂದೆ ನೀವು ತಪ್ಪು ದಾರಿ ಹಿಡಿದಿರಿ” –ಎಂದು ನಾನು ಉಪದೇಶಿಸಿದರೆ ನಿಮಗೆ ಯಾವ ವಿಧವಾದ ಸಹಾಯವೂ ಆಗುವುದಿಲ್ಲ. ಅಲ್ಲದೆ ನಿಮ್ಮನ್ನು ಮತ್ತೂ ಬಲಹೀನ ರನ್ನಾಗಿ ಮಾಡಲು ದಾರಿ ತೋರಿದಂತಾಗುತ್ತದೆ. ಸಾವಿರಾರು ವರುಷಗಳಿಂದ ಈ ಕೋಣೆಯಲ್ಲಿ ಕತ್ತಲೆ ಕವಿದಿದ್ದರೆ ನೀವು ಬಂದು “ಅಯ್ಯೊ, ಕತ್ತಲೆ” ಎಂದು ಗೋಳಾಡಿದರೆ ಕತ್ತಲೆ ಹೋಗುವುದೆ? ಒಂದು ಬೆಂಕಿಯ ಕಡ್ಡಿಯನ್ನು ಗೀರಿದರೆ ಬೆಳಕು ತಕ್ಷಣವೇ ಬರುವುದು. “ಅಯ್ಯೋ ನಾನು ಪಾಪ ಮಾಡಿರುವೆ; ಅನೇಕ ತಪ್ಪುಗಳನ್ನು ಮಾಡಿರುವೆ” ಎಂದು ನೀವು ಬದುಕಿರುವ ತನಕ ಆಲೋಚಿಸುತ್ತಿದ್ದರೆ ಅದರಿಂದ ಏನು ಒಳ್ಳೆಯದಾಗುವುದು? ಪಾಪಿಗಳೆಂದು ನಮಗೆ ಹೇಳಲು ಯಾವ ಭೂತವು ಬೇಕಾಗಿಲ್ಲ. ಜ್ಞಾನಜ್ಯೋತಿಯನ್ನು ತಂದರೆ ಪಾಪವು ಮರುಕ್ಷಣವೇ ಮಾಯವಾಗುವುದು. ಮೊದಲು ನಿಮ್ಮ ನಡತೆಯನ್ನು ತಿದ್ದಿಕೊಳ್ಳಿ. ನಿತ್ಯಶುದ್ಧ ವಾದ, ಪ್ರಕಾಶಮಾನವಾದ, ನಿಮ್ಮ ಸ್ವಯಂಜ್ಯೋತಿಯನ್ನು ವ್ಯಕ್ತಗೊಳಿಸಿ, ನೀವು ನೋಡುವ ಎಲ್ಲರಲ್ಲಿಯೂ ಅದು ವ್ಯಕ್ತವಾಗುವಂತೆ ಮಾಡಿ. ಅತಿ ಪಾಪಾತ್ಮನ ಅಂತರಾಳದಲ್ಲಿಯೂ ಇರುವ ಪರಿಶುದ್ಧಾತ್ಮನನ್ನು ನೋಡುವ ಸ್ಥಿತಿಗೆ ನಾವೆಲ್ಲರೂ ಬಂದಿದ್ದರೆ, ಅವನನ್ನು ನೋಡಿ ನಿಂದಿಸುವ ಬದಲು, “ಹೇ ಸ್ವಯಂ ಪ್ರಕಾಶಾತ್ಮನೆ, ಜಾಗ್ರತನಾಗು! ಹೇ, ನಿತ್ಯಶುದ್ಧನೆ ಏಳು! ಜನನ ಮರಣಾತೀತನೆ ಏಳು! ಸರ್ವಶಕ್ತನೆ ಏಳು! ನಿನ್ನ ನೈಜಸ್ವಭಾವವನ್ನು ವ್ಯಕ್ತಪಡಿಸು. ಈ ಕ್ಷುದ್ರರೂಪುಗಳು ನಿನಗೆ ತರವಲ್ಲ” ಎನ್ನಬಹುದಾಗಿತ್ತು. ಅದ್ವೈತಿಯು ಬೋಧಿಸುವ ಅತ್ಯುತ್ತಮ ಪ್ರಾರ್ಥನೆಯೆ ಇದು: ನಮ್ಮಲ್ಲಿ ಅನವರತವೂ ಇರುವ ಅಂತರಾತ್ಮನ ನಿಜ ಸ್ವಭಾವವನ್ನು ಮರೆಯದಿರುವುದು, ಯಾವಾಗಲೂ ಅದನ್ನು ಅನಂತವೆಂತಲೂ ಸರ್ವಶಕ್ತಿಮಾನ್​ ಎಂತಲೂ, ನಿತ್ಯ ಶುದ್ಧನು, ನಿತ್ಯೋಪಕಾರಿಯು, ನಿಃಸ್ವಾರ್ಥನು, ಸರ್ವ ಉಪಾಧಿ ವರ್ಜಿತನು ಎಂದೂ ತಿಳಿಯುವುದು. ಅದರ ಸ್ವಭಾವವು ಸ್ವಾರ್ಥಹೀನವಾದ ಕಾರಣ ಅದು ಬಲಾಢ್ಯವಾದುದು, ನಿರ್ಭಯವಾದುದು. ಏಕೆಂದರೆ ಅಂಜಿಕೆ ಬರುವುದು ಸ್ವಾರ್ಥಕ್ಕೆ ಮಾತ್ರ. ತನಗೋಸ್ಕರವಾಗಿ ಯಾರು ಏನನ್ನೂ ಆಶಿಸುವುದಿಲ್ಲವೋ, ಅವನು ಅಂಜುವುದು ಯಾರಿಗೆ? ಅವನನ್ನು ಅಂಜಿಸುವುದು ಯಾವುದು? ಅವನಿಗೆ ಸಾವಿನಿಂದ ಏನು ಭಯ? ಪಾಪದಿಂದ ಯಾವ ಭಯ? ನಾವು ಅದ್ವೈತಿಗಳಾದರೆ ಈ ಕ್ಷಣದಿಂದಲೇ ನಮ್ಮ ಹಳೆಯ ವ್ಯಕ್ತಿತ್ವ ಮಾಯವಾಗಿಹೋಯಿತು ಎಂದು ಆಲೋಚಿಸಬೇಕು. ಹಳೆಯ ಶ್ರಿಮಾನ್​, ಶ‍್ರೀಮತಿ, ಕುಮಾರಿ ಮುಂತಾದುವು ಗಳೆಲ್ಲ ಹೋಯಿತು. ಅವುಗಳೆಲ್ಲ ಮೂಢನಂಬಿಕೆಗಳಾಗಿದ್ದವು. ಉಳಿಯುವುದೇ, ನಿತ್ಯಶುದ್ಧವಾದ ಅಪೌರುಷೇಯವಾದ ಸರ್ವಶಕ್ತಿ ಮತ್ತು ಸರ್ವಜ್ಞ. ಅನಂತರ ಎಲ್ಲಾ ಅಂಜಿಕೆಗಳೂ ಅದರಿಂದ ಪಲಾಯನವಾಗುವುವು. ಸರ್ವಶಕ್ತನನ್ನು ಯಾರು ನೋಯಿಸಬಲ್ಲರು? ನಮ್ಮಲ್ಲಿರುವ ಬಲಹೀನತೆಯೆಲ್ಲ ಮಾಯೆ. ನಮ್ಮ ಸಹೋದರರಲ್ಲಿ ಈ ಆತ್ಮಜ್ಞಾನವನ್ನು ಎಚ್ಚರಗೊಳಿಸುವುದೊಂದೆ ನಮ್ಮ ಕೆಲಸ. ಅವರೂ ಕೂಡ ಮುಕ್ತಾತ್ಮರೆಂದು ಆಗ ನಮಗೆ ಕಾಣುತ್ತದೆ. ಆದರೆ ಅವರಿಗೆ ಅದು ಗೊತ್ತಿಲ್ಲ. ನಾವು ಅವರಿಗೆ ಇದನ್ನು ಬೋಧಿಸಬೇಕು. ಅವರ ಅನಂತಾತ್ಮವನ್ನು ಜಾಗ್ರಗೊಳಿಸುವುದಕ್ಕೆ ನಾವು ಸಹಾಯ ಮಾಡಬೇಕು. ಇಡೇ ಇಡೀ ಜಗತಿನಲ್ಲೆಲ್ಲಾ ಅತ್ಯಾವಶ್ಯಕವಾದುದೆಂದು ನನಗೆ ಕಾಣುತ್ತದೆ. ಈ ಸಿದ್ಧಾಂತ ಬಹಳ ಪುರಾತನ ವಾದುದು. ಬಹುಶಃ ಅನೇಕ ಪರ್ವತಗಳಿಗಿಂತಲೂ ಹಳೆಯದು. ಸತ್ಯವೆಲ್ಲ ನಿತ್ಯವಾದುದು. ಸತ್ಯ ಯಾರ ಸ್ವಂತ ಆಸ್ತಿಯೂ ಅಲ್ಲ. ಯಾವ ಜನಾಂಗವೇ ಆಗಲೀ, ಯಾವ ವ್ಯಕ್ತಿಯೇ ಆಗಲೀ, ಪ್ರತ್ಯೇಕವಾಗಿ ತನ್ನ ಹಕ್ಕನ್ನು ಅದರ ಮೇಲೆ ಸ್ಥಾಪಿಸ ಲಾರದು.ಎಲ್ಲಾ ಜೀವಿಗಳ ನೈಜಸ್ವಭಾವವೇ ಸತ್ಯ. ಅದಕ್ಕೋಸುಗ ಯಾರು ಪ್ರತ್ಯೇಕ ವಾದ ಹಕ್ಕನ್ನು ಸ್ಥಾಪಿಸಬಲ್ಲರು? ಆದರೆ ಅದನ್ನು ಅನುಷ್ಠಾನದಲ್ಲಿ ಬರುವಂತೆ ಮಾಡಬೇಕು. ಅದನ್ನು ಸುಲಭವಾಗಿ ತಿಳಿಯುವಂತೆ ಮಾಡಬೇಕು. ಉತ್ತಮ ಸತ್ಯ ಯಾವಾಗಲೂ ಅತಿ ಸುಲಭ. ಅದು ಸಮಾಜದ ಅಂಗಾಂಗಳಲ್ಲಿ ಪ್ರವಹಿಸುವಂತೆ ಆಗಬೇಕು. ಪ್ರೌಢಪಂಡಿತರ ಮತ್ತು ಪಾಮರರ, ಪುರುಷರ ಮತ್ತು ಸ್ತ್ರೀಯರ, ಆಬಾಲ ವೃದ್ಧರ ಸರ್ವ ಸಾಮಾನ್ಯ ಆಸ್ತಿಯಾಗಬೇಕು. ತರ್ಕಶಾಸ್ತ್ರದ ಶಾಖೋಪ ಶಾಖೆ, ತತ್ತ್ವಶಾಸ್ತ್ರದ ಸಿದ್ಧಾಂತದ ಹೊರೆ, ಪುರಾಣ, ಮತ್ತು ಬಾಹ್ಯಾಚಾರ ನಿಯಮಗಳು ಒಂದು ಕಾಲದಲ್ಲಿ ಚೆನ್ನಾಗಿದ್ದಿರಬಹುದು. ಆದರೆ ವಿಷಯವನ್ನು ಸುಲಭ ಗ್ರಾಹ್ಯವಾಗಿ ಮಾಡಲು ಪ್ರಯತ್ನಿಸೋಣ, ಪ್ರತಿ ಜೀವರಲ್ಲಿರುವ ಸತ್ಯವೇ ಪೂಜಾವಸ್ತುವಾಗಿ, ಪ್ರತಿ ಮಾನವನೂ ಉಪಾಸಕನಾಗುವ ಪುಣ್ಯಯುಗ ಉದಯಿಸುವಂತೆ ಮಾಡೋಣ.

