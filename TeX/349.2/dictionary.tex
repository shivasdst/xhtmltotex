\sethyphenation{kannada}{
ಅ
ಅಂಕಿ
ಅಂಕುರಾ-ವಸ್ಥೆ-ಯಲ್ಲಿ
ಅಂಕುರಾ-ವಸ್ಥೆ-ಯಲ್ಲಿ-ರು-ವಾಗಲೇ
ಅಂಕುರಾ-ವಸ್ಥೆ-ಯಲ್ಲೇ
ಅಂಕೆ-ಯಲ್ಲಿ
ಅಂಕೆ-ಯಿಂದ
ಅಂಗ
ಅಂಗಡಿ-ಯಲ್ಲಿ-ರುವ
ಅಂಗನ್ಯಾಸ
ಅಂಗ-ಲಾಚಿ
ಅಂಗಳ-ದಲ್ಲಿ
ಅಂಗ-ವನ್ನು
ಅಂಗ-ವಲ್ಲವೆ
ಅಂಗ-ವಾಗಿ-ರುತ್ತದೆ
ಅಂಗವು
ಅಂಗ-ಸಾಧನೆ
ಅಂಗ-ಸೌಲಭ್ಯ-ವಿಲ್ಲ
ಅಂಗಾಂಗ
ಅಂಗಾಂಗ-ಗ-ಳನ್ನು
ಅಂಗಾಂಗಳಲ್ಲಿ
ಅಂಗುಲ
ಅಂಗೋಪಾಂಗ-ಗಳಲ್ಲೆಲ್ಲ
ಅಂಜದ
ಅಂಜದ-ವರು
ಅಂಜ-ಬೇಕಾ-ಗಿಲ್ಲ
ಅಂಜಿ
ಅಂಜಿಕೆ
ಅಂಜಿಕೆ-ಗಳೂ
ಅಂಜಿಕೆ-ಗಳೆಲ್ಲ
ಅಂಜಿಕೆಯ
ಅಂಜಿಕೆ-ಯನ್ನು
ಅಂಜಿಕೆ-ಯನ್ನೂ
ಅಂಜಿಕೆ-ಯಾಗ-ಬಹುದು
ಅಂಜಿಕೆ-ಯಾಗುವ
ಅಂಜಿಕೆ-ಯಾಗು-ವಂತಹ
ಅಂಜಿಕೆ-ಯಾ-ಯಿತು
ಅಂಜಿಕೆ-ಯಿಂದ
ಅಂಜಿಕೆ-ಯಿಂದ-ಲಾದರೂ
ಅಂಜಿಕೆ-ಯಿಂದಾ-ಗಲಿ
ಅಂಜಿಕೆಯೂ
ಅಂಜಿಕೆ-ಯೆಲ್ಲ
ಅಂಜಿದ
ಅಂಜಿಸ-ಬಹುದು
ಅಂಜಿಸಿ
ಅಂಜಿಸಿ-ದರೆ
ಅಂಜಿಸು
ಅಂಜಿಸು-ತ್ತಿತ್ತು
ಅಂಜಿಸು-ವುದು
ಅಂಜು
ಅಂಜು-ತ್ತವೆ
ಅಂಜು-ತ್ತಿರು-ವೆವು
ಅಂಜುವ
ಅಂಜು-ವಂತ-ಹದೂ
ಅಂಜು-ವಂತೆ
ಅಂಜು-ವರು
ಅಂಜು-ವ-ವ-ರನ್ನೂ
ಅಂಜು-ವು-ದಾರಿಗೆ
ಅಂಜು-ವುದು
ಅಂಜು-ವೆಯ
ಅಂಜು-ವೆವು
ಅಂಟಿ
ಅಂಟಿ-ಕೊಂಡಿ-ರುವ
ಅಂಟಿ-ಕೊಂಡಿ-ರೂಲ್ಲದು
ಅಂಟಿ-ಕೊಳ್ಳುತ್ತಾರೆ
ಅಂಟಿ-ಕೊಳ್ಳುವ
ಅಂಟಿ-ಕೊಳ್ಳು-ವಂತೆ
ಅಂತ
ಅಂತಃ
ಅಂತಃಕರ-ಣಕ್ಕೆ
ಅಂತಃಕರ-ಣವೇ
ಅಂತ-ಮುರ್ಖಿ-ಯಾಗಿ
ಅಂತರ
ಅಂತ-ರಂಗ
ಅಂತ-ರಂಗಕ್ಕೆ
ಅಂತ-ರಂಗದ
ಅಂತ-ರಂಗ-ದಲ್ಲಿ
ಅಂತ-ರಂಗ-ದಲ್ಲಿ-ದೆಯೋ
ಅಂತ-ರಂಗ-ದಲ್ಲಿ-ರುವ
ಅಂತ-ರಂಗ-ದಿಂದಲೇ
ಅಂತ-ರಂಗ-ವನ್ನು
ಅಂತ-ರಂಗ-ವೇದ್ಯ-ವಾದು-ವು-ಇವು
ಅಂತ-ರಕ್ಕೆ
ಅಂತರ-ತಮ
ಅಂತರದ
ಅಂತರ-ವನ್ನು
ಅಂತರ-ವಿದೆ
ಅಂತರ-ವಿದ್ದರೂ
ಅಂತರ-ವಿಲ್ಲದೆ
ಅಂತರ-ವೆಲ್ಲಿದೆ
ಅಂತ-ರಾತ್ಮನ
ಅಂತ-ರಾತ್ಮ-ನದು
ಅಂತ-ರಾತ್ಮನು
ಅಂತ-ರಾತ್ಮಲ್ಲಿ-ರುವ
ಅಂತ-ರಾತ್ಮ-ವಾದ
ಅಂತ-ರಾತ್ಮವು
ಅಂತ-ರಾರ್ಥ
ಅಂತ-ರಾಳ
ಅಂತ-ರಾಳಕ್ಕೆ
ಅಂತ-ರಾಳ-ದಲ್ಲಿ
ಅಂತ-ರಾಳ-ದಲ್ಲಿಯೂ
ಅಂತ-ರಾಳ-ದಲ್ಲಿ-ರುವ
ಅಂತ-ರಾಳ-ದಲ್ಲೆಲ್ಲಾ
ಅಂತ-ರಾಳ-ದಿಂದ
ಅಂತ-ರಾಳ-ದಿಂದಲೇ
ಅಂತ-ರಿಂದ್ರಿಯ-ಗಳು
ಅಂತ-ರಿಕ್ಷ
ಅಂತ-ರಿಕ್ಷಕ್ಕೆ
ಅಂತ-ರಿಕ್ಷದ
ಅಂತರ್
ಅಂತರ್ಗತ
ಅಂತರ್ಗತ-ಭಾ-ವನೆ
ಅಂತರ್ಗತ-ವಾಗಿದೆ
ಅಂತರ್ಗತ-ವಾ-ಗಿ-ರುವ
ಅಂತರ್ಗತ-ವಾಗು-ತ್ತದೆ
ಅಂತರ್ಗತ-ವಾದ
ಅಂತರ್ಜ-ಗತ್ತನ್ನು
ಅಂತರ್ಜ-ಗತ್ತಿನ
ಅಂತರ್ಜೋ-ತಿಯ
ಅಂತರ್ಜ್ಯೋತಿ
ಅಂತರ್ದೃಷ್ಟಿ-ಉಳ್ಳ-ವ-ರಿಗೆ
ಅಂತರ್ದೃಷ್ಟಿ-ಯುಳ್ಳ
ಅಂತರ್ನಿ-ಹಿತ-ವಾಗಿರ
ಅಂತರ್ಮುಖ
ಅಂತರ್ಮುಖತ್ವ-ದಿಂದ
ಅಂತರ್ಮುಖ-ನಾಗು-ವು-ದಿಲ್ಲ
ಅಂತರ್ಮುಖ-ಮಾಡಿ
ಅಂತರ್ಮುಖ-ವಾ-ಗಿತ್ತು
ಅಂತರ್ಮುಖ-ವಾಗು-ತ್ತದೆ
ಅಂತರ್ಮುಖ-ವಾದು-ದನ್ನು
ಅಂತರ್ಮುಖ-ವಾ-ಯಿತು
ಅಂತರ್ಮುಖಿ-ಗಳು
ಅಂತರ್ಯಾಮಿ
ಅಂತರ್ಯಾಮಿಯ
ಅಂತರ್ಯಾಮಿ-ಯಾಗಿ
ಅಂತರ್ಯಾಮಿ-ಯಾಗಿ-ರುವನು
ಅಂತರ್ವಾಣಿ
ಅಂತ-ವಾದುವು
ಅಂತ-ವುಳ್ಳ
ಅಂತ-ವುಳ್ಳದ್ದು
ಅಂತಸ್ತಿನ
ಅಂತಸ್ತಿನಲ್ಲಿ-ರಲಿ-ಎಲ್ಲಾ
ಅಂತಸ್ತಿನ-ವ-ರಲ್ಲಿ
ಅಂತಸ್ತಿನಿಂದ
ಅಂತಸ್ತು-ಗಳು
ಅಂತಸ್ಥ
ಅಂತಹ
ಅಂತಹ-ವನ
ಅಂತಹ-ವ-ನನ್ನು
ಅಂತಹ-ವ-ನಾದ
ಅಂತಹ-ವ-ನೊಬ್ಬನೇ
ಅಂತಹ-ವರ
ಅಂತಹ-ವ-ರನ್ನು
ಅಂತಹ-ವ-ರಿಗೆ
ಅಂತಹ-ವರು
ಅಂತಹ-ವರೂ
ಅಂತಹ-ವ-ರೆಲ್ಲರೂ
ಅಂತ-ಹುದೇ
ಅಂತಿಮ
ಅಂತೂ
ಅಂತ್ಯ
ಅಂತ್ಯ-ಗಳಿಲ್ಲ
ಅಂತ್ಯ-ಗಳಿಲ್ಲದ
ಅಂತ್ಯ-ಗುರಿ
ಅಂತ್ಯ-ದಲ್ಲಿ
ಅಂತ್ಯ-ವಲ್ಲ
ಅಂತ್ಯ-ವಾಯಿ-ತೇನು
ಅಂತ್ಯ-ವಿದೆ
ಅಂತ್ಯ-ವಿರ-ಬೇಕು
ಅಂತ್ಯ-ವಿರ-ಲೇ-ಬೇಕು
ಅಂತ್ಯ-ವಿಲ್ಲ
ಅಂತ್ಯ-ವಿಲ್ಲದ
ಅಂತ್ಯ-ವಿಲ್ಲದೆ
ಅಂತ್ಯವೂ
ಅಂತ್ಯವೆ
ಅಂತ್ಯವೇ
ಅಂಥ
ಅಂಥ-ದನ್ನು
ಅಂದರೆ
ಅಂದ-ರೇನು
ಅಂದಿನ
ಅಂದಿ-ನಿಂದ
ಅಂದು
ಅಂದೇ
ಅಂಧ-ಕಾರ
ಅಂಧ-ಕಾರದ
ಅಂಧ-ಕಾರ-ದಿಂದ
ಅಂಧ-ಕಾರವು
ಅಂಧ-ಕಾರ-ವೆಲ್ಲ
ಅಂಧಶ್ರದ್ಧೆ
ಅಂಶ
ಅಂಶ-ಗಳ
ಅಂಶ-ಗ-ಳನ್ನು
ಅಂಶ-ಗಳಿದ್ದರೆ
ಅಂಶ-ಗಳಿವೆ
ಅಂಶ-ಗಳು
ಅಂಶ-ಗಳೆಂಬು-ದನ್ನು
ಅಂಶ-ಗಳೇ
ಅಂಶದ
ಅಂಶ-ದಲ್ಲಿ
ಅಂಶ-ವನ್ನಾ-ದರೂ
ಅಂಶ-ವನ್ನು
ಅಂಶ-ವಾಗಿದೆ
ಅಂಶ-ವಾಗು-ವುದು
ಅಂಶ-ವಿರು-ವುದು
ಅಂಶವು
ಅಂಶವೂ
ಅಂಶವೇ
ಅಂಶ-ವೊಂದು
ಅಂಹ-ಕಾರವು
ಅಎಮಓಂ
ಅಕ
ಅಕರ್ಮ-ದಲ್ಲಿ-ರು-ವುದು
ಅಕರ್ಮವೇ
ಅಕರ್ಮಿ-ಯಾಗ-ಬಲ್ಲೆ
ಅಕಳಂಕ
ಅಕಳಂಕನು
ಅಕಸ್ಮಾತ್ತಾಗಿ
ಅಕ್ಕ
ಅಕ್ಕ-ಪಕ್ಕದಲ್ಲಿಯೇ
ಅಕ್ಟೋಬರ್
ಅಕ್ರಮ-ವಾಗಿ
ಅಕ್ಲಿಷ್ಠಾಃ
ಅಕ್ಷರ-ವಾದ
ಅಕ್ಷ-ರವು
ಅಕ್ಷಿಪ-ಟದ
ಅಕ್ಷಿಪಟ-ಲದ
ಅಖಂಡ
ಅಖಂಡ-ಅ-ನಂತಸ್ವ-ರೂಪ-ದವ
ಅಖಂಡ-ವಾಗಿ
ಅಖಂಡ-ವಾದ
ಅಗತ್ಯ
ಅಗತ್ಯ-ವಾಗಿದೆ
ಅಗಲ
ಅಗಾಧ-ವಾದ
ಅಗೋಚರ
ಅಗೋಚರನು
ಅಗ್ನಿ
ಅಗ್ನಿ-ಕುಂಡ-ದಿಂದ
ಅಗ್ನಿಗೆ
ಅಗ್ನಿ-ಮಾಂದ್ಯ-ವಿರ-ಬಹುದೇ
ಅಗ್ನಿಯ
ಅಗ್ನಿ-ಯನ್ನು
ಅಗ್ನಿ-ಯಲ್ಲಿ
ಅಗ್ನಿಯು
ಅಗ್ನಿಯೂ
ಅಗ್ನಿ-ಯೊಂದು
ಅಚಡಣೆ-ಯನ್ನು
ಅಚ-ತನ
ಅಚಲ
ಅಚ-ಲನು
ಅಚಲನೂ
ಅಚಲ-ವಲ್ಲ
ಅಚಲ-ವಾಗಿರು-ವು-ದನ್ನು
ಅಚಲ-ವಾದ
ಅಚಲ-ವಾದು-ದನ್ನು
ಅಚಲ-ವಾ-ದುದು
ಅಚಿಂತ್ಯಜ್ಞಾನ
ಅಚೇ-ತನ
ಅಚೇ-ತನದ
ಅಚೇ-ತನ-ದಲ್ಲಿ
ಅಚೇ-ತನ-ವಾದ
ಅಚೇತ-ವೆಂದು
ಅಚ್ಚಳಿ-ಯದೆ
ಅಜ
ಅಜ್ಞರು
ಅಜ್ಞಾತ
ಅಜ್ಞಾತ-ನೆಂದು
ಅಜ್ಞಾತ-ವಾಗಿ-ರ-ಬಹುದು
ಅಜ್ಞಾನ
ಅಜ್ಞಾನಕ್ಕೆ
ಅಜ್ಞಾನ-ಗಳಲ್ಲಿ
ಅಜ್ಞಾನ-ಜನಿತ
ಅಜ್ಞಾನ-ಜನ್ಯ-ವಾದ
ಅಜ್ಞಾನದ
ಅಜ್ಞಾನ-ದಂತಾ-ಗಲಿ
ಅಜ್ಞಾನ-ದಲ್ಲಿ
ಅಜ್ಞಾನ-ದಲ್ಲಿ-ರುವುದು
ಅಜ್ಞಾನ-ದಲ್ಲಿ-ರು-ವೆವು
ಅಜ್ಞಾನ-ದಿಂದ
ಅಜ್ಞಾನ-ವನ್ನು
ಅಜ್ಞಾನ-ವಶ-ನಾಗಿ
ಅಜ್ಞಾ-ನ-ವಿಲ್ಲ-ದಿದ್ದರೆ
ಅಜ್ಞಾನವು
ಅಜ್ಞಾನ-ವೆನ್ನು-ವುದು
ಅಜ್ಞಾನ-ವೆಲ್ಲ
ಅಜ್ಞಾನವೇ
ಅಜ್ಞಾನಾಂಧ-ಕಾರ-ದಲ್ಲಿ
ಅಜ್ಞಾನಾ-ತೀತ-ನಾದ
ಅಜ್ಞಾನಿ
ಅಜ್ಞಾನಿ-ಗಳನ್ನಾಗಿ
ಅಜ್ಞಾನಿ-ಗಳಾ-ಗಿದ್ದರು
ಅಜ್ಞಾನಿ-ಗ-ಳಾದ
ಅಜ್ಞಾನಿ-ಗಳಿಗೆ
ಅಜ್ಞಾನಿ-ಗಳು
ಅಜ್ಞಾನಿ-ಗಳೆಲ್ಲ
ಅಜ್ಞಾನಿ-ಯನ್ನು
ಅಜ್ಞಾನಿಯು
ಅಜ್ಞೇ-ಯತಾ
ಅಜ್ಞೇ-ಯತಾ-ವಾದಿ
ಅಜ್ಞೇ-ಯತಾ-ವಾದಿ-ಗಳು
ಅಜ್ಞೇ-ಯತಾ-ವಾದಿ-ಯಾದರೆ
ಅಜ್ಞೇ-ಯವೂ
ಅಜ್ಞೇಯ-ವೆಂದು
ಅಟಾವಿಸಂ
ಅಟ್ಟ-ಬೇಕು
ಅಟ್ಟಿಸಿ-ಕೊಂಡು
ಅಟ್ಟುತ್ತವೆ
ಅಟ್ಟುತ್ತಿದ್ದರೆ
ಅಡಕ-ಮಾಡಿ-ದ್ದಾರೆ
ಅಡಗಿ
ಅಡಗಿ-ಕೊಂಡು
ಅಡ-ಗಿತ್ತು
ಅಡ-ಗಿದೆ
ಅಡಗಿ-ರದೆ
ಅಡಗಿ-ರುವಂತೆ
ಅಡಗಿ-ರುವುದು
ಅಡಗಿ-ರು-ವುವೋ
ಅಡ-ಗಿವೆ
ಅಡಗಿ-ಸ-ಬೇಕಾ-ಗಿದೆ
ಅಡಗಿ-ಸಲ್ಪಟ್ಟಿ-ರಲೀ
ಅಡಗಿ-ಸಲ್ಪಟ್ಟಿರು-ವುದೆಂದರೆ
ಅಡಗಿ-ಸುವ
ಅಡಗಿ-ಸುವು-ದಕ್ಕೂ
ಅಡಗಿ-ಸು-ವು-ದಕ್ಕೆ
ಅಡಗಿ-ಸು-ವುದು
ಅಡ-ಗು-ವುವು
ಅಡಚಣೆ
ಅಡಚಣೆ-ಗಳಾ-ವುವು
ಅಡಚಣೆ-ಗಳಿವೆ
ಅಡಚಣೆ-ಗಳು
ಅಡಚಣೆ-ಗಳೆಲ್ಲ
ಅಡಚಣೆ-ಯನ್ನು
ಅಡಚಣೆ-ಯಿಂದ
ಅಡಚಣೆ-ಯಿದೆ
ಅಡಚಣೆಯೂ
ಅಡಚಣೆಯೇ
ಅಡಿ
ಅಡಿ-ಗಲ್ಲೆ
ಅಡಿಗೆ
ಅಡಿಗೆಗೆ
ಅಡಿ-ಯಲ್ಲಿ
ಅಡಿ-ಯಾಳನ್ನಾಗಿ
ಅಡಿ-ಯಾಳಾ-ಗುತ್ತದೆ
ಅಡಿ-ಯಾಳಾಗು-ವನು
ಅಡ್ಡಿ
ಅಡ್ಡಿ-ಯನ್ನು
ಅಡ್ಡಿ-ಯಾಗಿ
ಅಡ್ಡಿ-ಯಾಗುತ್ತವೆ
ಅಡ್ಡಿ-ಯಾದರೂ
ಅಡ್ಡಿ-ಯಿಲ್ಲ
ಅಣ-ಕಿಸು-ತ್ತಿರುವ
ಅಣಕಿ-ಸುವ
ಅಣತಿ-ಯನ್ನು
ಅಣಬೆಯಂತಹ
ಅಣಿಕಿಸ-ಬಹುದು
ಅಣಿತಿ-ಯನ್ನು
ಅಣಿಮಾದಿ
ಅಣಿ-ಯಾಗುತ್ತವೆ
ಅಣು
ಅಣು-ಗಳನ್ನು
ಅಣು-ವಾ-ಗು-ವುದು
ಅಣುವಿ-ಗಿಂತ
ಅಣುವಿ-ನಿಂದ
ಅತವಾ
ಅತಾರ್ಕಿಕ-ವಾದು-ದೆನ್ನ-ಬಹುದು
ಅತಿ
ಅತಿಕ್ರಮಿ-ಸಲು
ಅತಿಕ್ರಮಿಸಿ
ಅತಿಕ್ರಮಿಸಿ-ರುವ
ಅತಿಕ್ರಮಿ-ಸು-ವುದು
ಅತಿಕ್ರೂರ
ಅತಿಕ್ರೂರ-ವಾ-ದುದು
ಅತಿಕ್ಷುದ್ರ
ಅತಿ-ಗಹನ-ವಾದ
ಅತಿ-ಗೌರವ-ದಿಂದ
ಅತಿ-ಘೋರ
ಅತಿ-ಥಿ-ಗಳು
ಅತಿ-ಥಿಯೂ
ಅತಿ-ಪವಿತ್ರ
ಅತಿಪ್ರಾ-ಕೃತ
ಅತಿಪ್ರಾ-ಕೃತ-ವೆಂಬು-ದಿಲ್ಲ
ಅತಿಪ್ರಾ-ಕೃತಿಕ
ಅತಿಪ್ರಿಯ
ಅತಿಪ್ರಿಯ-ವಾದ
ಅತಿ-ಭೌತಿಕ
ಅತಿ-ಮಾನುಷ
ಅತಿ-ಮೂರ್ಖ-ನಿಗೂ
ಅತಿ-ಯಾಗಿ
ಅತಿ-ಯಾದ
ಅತಿ-ರೇಕಕ್ಕೆ
ಅತಿ-ರೇಕ-ಗಳ
ಅತಿ-ರೇಕ-ಗಳಿಂದ
ಅತಿ-ರೇಕ-ಗಳು
ಅತಿ-ರೇಕ-ಗಳೂ
ಅತಿ-ರೇಕ-ಗಳೆ-ರಡೂ
ಅತಿ-ರೇಕ-ದಿಂದ
ಅತಿ-ವಿಚಿತ್ರ-ವಾದ
ಅತಿ-ವಿನಯ-ದಿಂದ
ಅತಿಶ್ರೇಷ್ಠ
ಅತಿಶ್ರೇಷ್ಠ-ತಮ
ಅತಿ-ಸರಳ-ವಾಗಿ
ಅತಿ-ಸೂಕ್ಷ್ಮ-ವಾದ
ಅತಿ-ಹೀನ
ಅತಿ-ಹೇಯ-ವಾದ
ಅತೀಂದ್ರಿಯ
ಅತೀಂದ್ರಿಯಜ್ಞಾನ
ಅತೀಂದ್ರಿ-ಯದ
ಅತೀಂದ್ರಿಯ-ವಾಗಿ
ಅತೀಂದ್ರಿಯ-ವಾ-ಗು-ವುದು
ಅತೀಂದ್ರಿಯಾ-ವಸ್ಥೆಗೆ
ಅತೀಂದ್ರಿಯಾ-ವಸ್ಥೆಯ
ಅತೀತ
ಅತೀತದ
ಅತೀತ-ನಾಗಿ
ಅತೀತ-ನಾದ-ವನೊಬ್ಬ-ನನ್ನು
ಅತೀ-ತನು
ಅತೀತ-ರಾಗಿರು-ವರೊ
ಅತೀತ-ವಾಗಿದೆ
ಅತೀತ-ವಾಗಿರು
ಅತೀತ-ವಾಗಿ-ರುವ
ಅತೀತ-ವಾಗಿ-ರು-ವಂತೆ
ಅತೀತ-ವಾಗಿ-ರು-ವುದರ
ಅತೀತ-ವಾಗಿ-ರು-ವುದು
ಅತೀತ-ವಾಗಿ-ರು-ವುದೋ
ಅತೀತ-ವಾಣಿ
ಅತೀತ-ವಾದ
ಅತೀತ-ವಾದು-ದನ್ನು
ಅತೀತ-ವಾದುದು
ಅತೀತಾ-ನಾಗತಂ
ಅತೃಪ್ತಿ
ಅತೃಪ್ತಿಯ
ಅತೃಪ್ತಿ-ಯಾ-ಗಲಿ
ಅತೃಪ್ತಿ-ಯಾಗು-ವುದು
ಅತೃಪ್ತಿ-ಯಾ-ದಾಗ
ಅತೃಪ್ತಿ-ಯಾ-ಯಿತು
ಅತ್ತ
ಅತ್ತ-ಕಡೆ
ಅತ್ತು
ಅತ್ಮ-ಸಾಕ್ಷಾತ್ಕಾರ
ಅತ್ಯ
ಅತ್ಯಂತ
ಅತ್ಯಂದ
ಅತ್ಯದ್ಭುತ
ಅತ್ಯದ್ಭುತ-ವಾದ
ಅತ್ಯ-ಧಿಕ
ಅತ್ಯ-ಮೋಘ-ವಾದ
ಅತ್ಯಲ್ಪ
ಅತ್ಯಲ್ಪ-ನೆಂದು
ಅತ್ಯ-ವಶ್ಯಕ
ಅತ್ಯಾ
ಅತ್ಯಾಧುನಿಕ
ಅತ್ಯಾಧುನಿಕ-ವಾದ
ಅತ್ಯಾ-ನಂದ
ಅತ್ಯಾ-ವಶ್ಯಕ
ಅತ್ಯಾ-ವಶ್ಯಕತೆ
ಅತ್ಯಾ-ವಶ್ಯಕ-ವಾಗಿ
ಅತ್ಯಾ-ವಶ್ಯಕ-ವಾಗಿದೆ
ಅತ್ಯಾ-ವಶ್ಯಕ-ವಾದು-ದೆಂದು
ಅತ್ಯಾ-ವಶ್ಯಕ-ವೆನ್ನು-ವುದು
ಅತ್ಯುಚ್ಚ
ಅತ್ಯುತ್ತಮ
ಅತ್ಯುತ್ತಮ-ನ-ವರೆಗೆ
ಅತ್ಯುತ್ತಮ-ನಾದ-ವನೇ
ಅತ್ಯುತ್ತಮ-ನಾರೂ
ಅತ್ಯುತ್ತಮರೂ
ಅತ್ಯುತ್ತಮ-ವಾದ
ಅತ್ಯುತ್ತಮ-ವಾದುದು
ಅತ್ಯುತ್ತಮ-ವಾದು-ದೆಂದು
ಅತ್ಯುತ್ತಮ-ಸ್ಥಿತಿ
ಅತ್ಯುತ್ತಮಾ-ವಸ್ಥೆ
ಅತ್ಯುತ್ತಮಾ-ವಸ್ಥೆ-ಯಲ್ಲ
ಅತ್ಯುತ್ಮ
ಅತ್ಯುನ್ನತ
ಅತ್ಯುನ್ನತ-ವಾದು-ದನ್ನು
ಅಥ
ಅಥವಾ
ಅದ
ಅದಕ್ಕಾಗಿ
ಅದಕ್ಕಾಗಿಯೇ
ಅದಕ್ಕಿಂತ
ಅದಕ್ಕಿಂತಲೂ
ಅದಕ್ಕೂ
ಅದಕ್ಕೆ
ಅದಕ್ಕೆಲ್ಲ
ಅದಕ್ಕೇ
ಅದಕ್ಕೇನೂ
ಅದಕ್ಕೊಂದು
ಅದಕ್ಕೋ
ಅದಕ್ಕೋಸುಗ
ಅದಕ್ಕೋಸುಗವೇ
ಅದಕ್ಕೋಸ್ಕರ-ವಾಗಿಯೇ
ಅದಕ್ಕೋಸ್ಕ-ರವೆ
ಅದಕ್ಕೋಸ್ಕ-ರವೇ
ಅದನ್ನು
ಅದನ್ನೂ
ಅದನ್ನೆ
ಅದನ್ನೆಂದಿಗೂ
ಅದನ್ನೆಲ್ಲ
ಅದನ್ನೇ
ಅದರ
ಅದರಂತ-ರಾಳ-ದಲ್ಲಿ
ಅದ-ರಂತೆ
ಅದ-ರಂತೆಯೆ
ಅದ-ರಂತೆಯೇ
ಅದರ-ಡಿಗೆ
ಅದ-ರದೇ
ಅದರಲ್ಲಾ-ದರೂ
ಅದ-ರಲ್ಲಿ
ಅದ-ರಲ್ಲಿಯೂ
ಅದ-ರಲ್ಲಿಯೇ
ಅದ-ರಲ್ಲಿರು
ಅದ-ರಲ್ಲಿ-ರುವ
ಅದ-ರಲ್ಲಿ-ರು-ವು-ದೆಂದು
ಅದ-ರಲ್ಲಿ-ರು-ವೆವು
ಅದ-ರಲ್ಲಿಲ್ಲ
ಅದ-ರಲ್ಲಿವೆ
ಅದ-ರಲ್ಲೂ
ಅದ-ರಲ್ಲೆ
ಅದ-ರಷ್ಟು
ಅದರಾಚೆ
ಅದ-ರಿಂದ
ಅದ-ರಿಂದಲೂ
ಅದ-ರಿಂದಲೇ
ಅದರೆ-ಡೆಗೆ
ಅದ-ರೊಂದಿಗೆ
ಅದ-ರೊಡನೆ
ಅದ-ರೊಳಕ್ಕೆ
ಅದ-ರೊಳಗೆ
ಅದಲ್ಲದ
ಅದಲ್ಲದೆ
ಅದಲ್ಲ-ವೆಂದರೆ
ಅದ-ವನ್ನು
ಅದಷ್ಟೆ
ಅದಾ
ಅದಾ-ಗಲೆ
ಅದಾಗಿ-ರದೆ
ಅದಾಗಿ-ರು-ವಿರಿ
ಅದಾದ
ಅದಾ-ವುದು
ಅದಾ-ವುದೆಂದರೆ
ಅದಿನ್ನೂ
ಅದಿ-ರಲಿ
ಅದಿಲ್ಲದೆ
ಅದಿಲ್ಲ-ವೆಂದು
ಅದು
ಅದು-ಮಲು
ಅದು-ರನ್ನು
ಅದುರು-ತ್ತಿರುವ
ಅದೂ
ಅದೃಶ್ಯ
ಅದೃಶ್ಯ-ವಾಗು-ವುದು
ಅದೃಷ್ಟ
ಅದೃಷ್ಟಕ್ಕೆ
ಅದೃಷ್ಟ-ವಂತ-ರಿಗೇನೋ
ಅದೃಷ್ಟ-ವನ್ನು
ಅದೃಷ್ಟ-ವಾದಿ
ಅದೃಷ್ಟ-ವೆಲ್ಲಿದೆ
ಅದೆ
ಅದೆಂದರೆ
ಅದೆಂದಿಗ
ಅದೆಂದಿಗೂ
ಅದೆಂದೂ
ಅದೆಲ್ಲ
ಅದೆಲ್ಲವೂ
ಅದೆಲ್ಲಾ
ಅದೆಲ್ಲೊ
ಅದೆಷ್ಟು
ಅದೇ
ಅದೇಕೆ
ಅದೇ-ತಕ್ಕೆ
ಅದೇ-ನಾಗು-ವುದು
ಅದೇನು
ಅದೇನೂ
ಅದೇ-ನೆಂದರೆ
ಅದೇನೊ
ಅದೇನೋ
ಅದೊಂದು
ಅದೊಂದೆ
ಅದೊಂದೇ
ಅದ್ಭುತ
ಅದ್ಭುತ-ಗಳ
ಅದ್ಭುತನೂ
ಅದ್ಭುತ-ವಾಗಿ
ಅದ್ಭುತ-ವಾಗಿದೆ
ಅದ್ಭುತ-ವಾದ
ಅದ್ಭುತ-ವಾದುದು
ಅದ್ಭುತ-ವಾದುವು
ಅದ್ಭುತ-ಶಕ್ತಿ
ಅದ್ಭುತ-ಶಕ್ತಿ-ಯುಳ್ಳ
ಅದ್ರವ್ಯ
ಅದ್ವಿ-ತೀಯ
ಅದ್ವಿ-ತೀಯ-ವಾದ
ಅದ್ವಿ-ತೀಯ-ವಾದುದು
ಅದ್ವೈತ
ಅದ್ವೈತಕ್ಕೆ
ಅದ್ವೈತದ
ಅದ್ವೈತ-ದರ್ಶನದ
ಅದ್ವೈತ-ದಲ್ಲಿ
ಅದ್ವೈತ-ಭಾವನೆ
ಅದ್ವೈತ-ಭಾವನೆ-ಯಿಂದ
ಅದ್ವೈತ-ವನ್ನು
ಅದ್ವೈತ-ವಾದ-ನಿರಾ-ಕಾರ
ಅದ್ವೈತವು
ಅದ್ವೈತ-ವೆಂದರೆ
ಅದ್ವೈತ-ವೇದಾಂತ
ಅದ್ವೈತ-ಸಿದ್ಧಾಂತದ
ಅದ್ವೈತ-ಸಿದ್ಧಾಂತ-ದಲ್ಲಿ
ಅದ್ವೈತಾನು-ಯಾಯಿ-ಗಳು
ಅದ್ವೈತಿ
ಅದ್ವೈತಿ-ಗಳ
ಅದ್ವೈತಿ-ಗಳಾ-ದರೆ
ಅದ್ವೈತಿ-ಗಳು
ಅದ್ವೈತಿಗೆ
ಅದ್ವೈತಿಯ
ಅದ್ವೈತಿ-ಯಾದು-ದ-ರಿಂದ
ಅದ್ವೈತಿಯು
ಅಧಮ
ಅಧರ್ಮ
ಅಧರ್ಮಿ-ಯಾಗು-ವನು
ಅಧಿಕ
ಅಧಿಕ-ವಾಗಿ
ಅಧಿಕಾರ
ಅಧಿಕಾರ-ಇವು-ಗಳ
ಅಧಿಕಾರದ
ಅಧಿಕಾರ-ವನ್ನು
ಅಧಿಕಾರ-ವಾ-ಗಲಿ
ಅಧಿಕಾರ-ವಾ-ದವು
ಅಧಿಕಾರ-ವಿದೆ
ಅಧಿಕಾರ-ವಿದ್ದರೆ
ಅಧಿಕಾರ-ವಿರ-ಬೇಕು
ಅಧಿಕಾರ-ವಿರ-ಲಿಲ್ಲ
ಅಧಿಕಾರ-ವಿರು-ವುದು
ಅಧಿಕಾರ-ವಿಲ್ಲ
ಅಧಿಕಾರ-ವಿಲ್ಲ-ವೆಂದರೆ
ಅಧಿಕಾರ-ವುಳ್ಳ
ಅಧಿಕಾರವೂ
ಅಧಿಕಾರ-ಸ್ತಾನ-ಈ
ಅಧಿಕಾರಿ
ಅಧಿಪತಿ-ಯಾಗಿ
ಅಧಿಪತಿ-ಯಾದ
ಅಧೀ
ಅಧೀನಕ್ಕೆ
ಅಧೀನಕ್ಕೊ
ಅಧೀನ-ಗೊಳಿಸ-ಬಹುದು
ಅಧೀನದ
ಅಧೀನ-ದಲ್ಲಿಟ್ಟು
ಅಧೀನ-ದಲ್ಲಿದೆ
ಅಧೀನ-ದಲ್ಲಿಯೂ
ಅಧೀನ-ದಲ್ಲಿರು
ಅಧೀನ-ದಲ್ಲಿರು-ವುದು
ಅಧೀನ-ದಲ್ಲಿರು-ವುವು
ಅಧೀನ-ದಲ್ಲಿವೆ
ಅಧೀನ-ವಾ-ಗು-ವುದು
ಅಧೀನ-ವಾ-ಗು-ವುವು
ಅಧೋ-ಗತಿ-ಗಿಳಿದು
ಅಧೋ-ಗ-ತಿಗೆ
ಅಧೋ-ಗ-ತಿಯ
ಅಧೋ-ಗತಿ-ಯಲ್ಲಿ-ರುವುದು
ಅಧೋಸ್ಥಿತಿಗೆ
ಅಧ್ಯಯನ
ಅಧ್ಯಯ-ನಕ್ಕೆ
ಅಧ್ಯಯ-ನದ
ಅಧ್ಯಯನ-ಮಾಡಿ
ಅಧ್ಯಯನ-ಮಾಡುವ
ಅಧ್ಯಯನ-ವಾದ
ಅಧ್ಯಯನ-ವಾದುವು
ಅಧ್ಯಾತ್ಮ
ಅಧ್ಯಾತ್ಮ-ಜ್ಞಾನ
ಅಧ್ಯಾತ್ಮಿಕ
ಅಧ್ಯಾಯ
ಅಧ್ಯಾ-ಯಕ್ಕೆ
ಅಧ್ಯಾಯ-ಗಳ
ಅಧ್ಯಾಯ-ಗ-ಳನ್ನು
ಅಧ್ಯಾಯ-ಗಳು
ಅಧ್ಯಾ-ಯದ
ಅಧ್ಯಾಯ-ದಲ್ಲಿ
ಅಧ್ಯಾಸ
ಅನ
ಅನಂತ
ಅನಂತ-ಕಡಲಿ-ನಲ್ಲಿ
ಅನಂತ-ಕಲ್ಯಾಣ
ಅನಂತ-ಕಲ್ಯಾಣ-ಗುಣ-ಗಣಿ
ಅನಂತ-ಕಾಲ
ಅನಂತ-ಕಾಲದ
ಅನಂತ-ಕಾಲ-ದಲ್ಲಿ
ಅನಂತ-ಕಾಲ-ವಿದೆ
ಅನಂತಕ್ಕಿಂತ
ಅನಂತ-ಗಳು
ಅನಂತ-ಜ್ಞಾನ
ಅನಂತ-ಜ್ಞಾನ-ವಿದೆ
ಅನಂತ-ಜ್ಞಾನವು
ಅನಂತತಾ
ಅನಂತತೆ
ಅನಂತತೆ-ಗಾಗಿ
ಅನಂತ-ತೆಯ
ಅನಂತ-ತೆ-ಯನ್ನು
ಅನಂತ-ತೆ-ಯಲ್ಲಿ
ಅನಂತ-ತೆ-ಯೆ-ಡೆಗೆ
ಅನಂತದ
ಅನಂತ-ದ-ಯಾ-ಸಿಂಧು-ವಾದ
ಅನಂತ-ದಲ್ಲಿ
ಅನಂತ-ದಿಂದ
ಅನಂತ-ದೆ-ಡೆಗೆ
ಅನಂತ-ದೊಂದಿಗೆ
ಅನಂತ-ನಾಗಿ-ರುವನೊ
ಅನಂತ-ನಾದ
ಅನಂತನು
ಅನಂತನೂ
ಅನಂತಪ್ರೀತಿ
ಅನಂತಪ್ರೇಮ-ದೊಂದಿಗೆ
ಅನಂತಬ್ರಹ್ಮ
ಅನಂತ-ಭಾವ-ನೆಗೆ
ಅನಂತರ
ಅನಂತ-ರದ
ಅನಂತ-ರವೆ
ಅನಂತ-ರವೇ
ಅನಂತ-ರಾದ
ಅನಂತರು
ಅನಂತ-ವನ್ನು
ಅನಂತ-ವಲ್ಲ
ಅನಂತವಾ
ಅನಂತ-ವಾಗ-ಬಲ್ಲದು
ಅನಂತ-ವಾಗ-ಬೇಕು
ಅನಂತ-ವಾಗಿ
ಅನಂತ-ವಾಗಿದೆ
ಅನಂತ-ವಾಗಿ-ದ್ದರೆ
ಅನಂತ-ವಾಗಿಯೇ
ಅನಂತ-ವಾಗಿರ-ಬೇಕು
ಅನಂತ-ವಾಗಿರು
ಅನಂತ-ವಾಗಿ-ರು-ವು-ದ-ರಿಂದ
ಅನಂತ-ವಾಗಿವೆ
ಅನಂತ-ವಾಗು-ವುದು
ಅನಂತ-ವಾದ
ಅನಂತ-ವಾ-ದದ್ದು
ಅನಂತ-ವಾದರೆ
ಅನಂತ-ವಾ-ದೀತು
ಅನಂತ-ವಾದು-ದನ್ನು
ಅನಂತ-ವಾ-ದುದು
ಅನಂತವು
ಅನಂತವೂ
ಅನಂತವೆ
ಅನಂತ-ವೆಂತಲೂ
ಅನಂತ-ವೆಂದು
ಅನಂತ-ವೆಂಬ
ಅನಂತವೇ
ಅನಂತವೊ
ಅನಂತವೋ
ಅನಂತವ್ಯೂಹ-ಗಳು
ಅನಂತ-ಶಕ್ತಿ
ಅನಂತ-ಶಕ್ತಿಯ
ಅನಂತ-ಶಕ್ತಿ-ಯನ್ನು
ಅನಂತ-ಶಕ್ತಿ-ಯನ್ನೆಲ್ಲ
ಅನಂತ-ಸಾಗರ
ಅನಂತಾ
ಅನಂತಾ-ಕಾಶ-ದಂತೆ
ಅನಂತಾತ್ಮ
ಅನಂತಾತ್ಮ-ನಾದ
ಅನಂತಾತ್ಮನೂ
ಅನಂತಾತ್ಮ-ವನ್ನು
ಅನಂತಾತ್ಮ-ವಾಗ-ಲಾರದು
ಅನಂತಾತ್ಮ-ವಾಗಿ
ಅನಂತಾತ್ಮವು
ಅನಂತಾ-ನಂತ
ಅನಂತಾ-ನಂದ
ಅನಂತಾ-ನಂದ-ವನ್ನೇ
ಅನಂತಾ-ನಂದವು
ಅನಂತಾ-ನಂದ-ವೇ-ಆದರೆ
ಅನಕ್ಷ
ಅನಗತ್ಯ-ವಾ-ದುದು
ಅನಗತ್ಯ-ವಾದು-ದೆಂದು
ಅನರ್ಥ
ಅನರ್ಥ-ಕಾರಿ
ಅನರ್ಥ-ಗಳಿಂದ
ಅನರ್ಹ-ರನ್ನೂ
ಅನರ್ಹರು
ಅನ-ವರತ
ಅನ-ವರತ-ವಾಗಿ
ಅನ-ವರ-ತವೂ
ಅನ-ವರತೂ
ಅನಸ್ತಿತ್ವ-ದಿಂದ
ಅನಾ
ಅನಾಗರಿಕ
ಅನಾಗರಿಕರ
ಅನಾಗರಿಕ-ವಾಗಿ-ರಲಿ
ಅನಾತ್ಮವೂ
ಅನಾಥಾಲಯ-ಗಳು
ಅನಾದಿ
ಅನಾದಿ-ಕಾಲದ
ಅನಾದಿ-ಕಾಲ-ದಿಂದಲೂ
ಅನಾದಿ-ಯಾಗಿರು-ವಂತೆ
ಅನಾದಿ-ಯಾಗಿವೆ
ಅನಾದಿ-ಯಾದ
ಅನಾದಿ-ಯಾದುವು
ಅನಾದಿ-ಯಿಂದಲೂ
ಅನಾದ್ಯ-ನಂತ
ಅನಾರೋಗ್ಯ
ಅನಾವಶ್ಯಕ
ಅನಾವಶ್ಯಕ-ವಾಗಿತ್ತೆಂದೂ
ಅನಾವಶ್ಯಕ-ವಾದ
ಅನಾವಶ್ಯಕ-ವಾದು-ದೆಂದು
ಅನಾಸಕ್ತ
ಅನಾಸಕ್ತ-ರೆಂದು
ಅನಾಸಕ್ತ-ವಾ-ದದ್ದು
ಅನಾಸಕ್ತಿ
ಅನಾಹುತ
ಅನಿತ್ಯ
ಅನಿತ್ಯ-ದಲ್ಲಿ
ಅನಿತ್ಯ-ವಸ್ತು
ಅನಿತ್ಯವೂ
ಅನಿತ್ಯಾ-ಶುಚಿ-ದುಃಖಾನಾತ್ಮಸು
ಅನಿರೀಕ್ಷಿತ
ಅನಿರ್ದೇಶಿತ
ಅನಿಲ
ಅನಿಲಕ್ಕೆ
ಅನಿಲ-ವಾ-ಗು-ವುವು
ಅನಿ-ವಾರ್ಯ
ಅನಿ-ವಾರ್ಯ-ವಾಗಿದೆ
ಅನಿ-ವಾರ್ಯ-ವಾದುದ
ಅನಿಶ್ಚಿತ-ವಾದ
ಅನಿಷ್ಟ
ಅನಿಷ್ಟಕ್ಕೋಸುಗ-ವಾಗಿ
ಅನಿ-ಸಿಕೆ
ಅನಿಸು-ತ್ತದೆ
ಅನೀತಿ
ಅನೀತಿಯೇ
ಅನು
ಅನು-ಕಂಪ
ಅನು-ಕಂಪ-ವಿತ್ತು
ಅನು-ಕರ-ಣಕ್ಕೆ
ಅನುಕರಣೆ
ಅನುಕರಿಸಿ-ದು-ದಕ್ಕೆ
ಅನುಕರಿಸು-ವುದ-ರಲ್ಲಿ
ಅನುಕರಿ-ಸು-ವುದು
ಅನುಕಾಲ-ವಾದ
ಅನು-ಕೂಲ
ಅನು-ಕೂಲ-ಕ್ಕಾಗಿ
ಅನು-ಕೂಲತೆ-ಯಿದೆ
ಅನು-ಕೂಲ-ವಾಗ-ಲೆಂದು
ಅನು-ಕೂಲ-ವಾಗಿದೆ
ಅನು-ಕೂಲ-ವಾ-ಗಿ-ರುವ
ಅನು-ಕೂಲ-ವಾಗು-ವಂತೆ
ಅನು-ಕೂಲ-ವಾದ
ಅನು-ಕೂಲ-ವಾ-ದೆಡೆ-ಯಲ್ಲಿ
ಅನು-ಕೂಲ-ವಿತ್ತು
ಅನು-ಕೂಲ-ವಿದೆ
ಅನು-ಕೂಲ-ವೆನಿಸು-ತ್ತದೆ
ಅನುಕ್ರಮ
ಅನುಕ್ರಮ-ದಲ್ಲಿ
ಅನುಕ್ರಮ-ವಿದೆ
ಅನುಕ್ರಮ-ವಿಲ್ಲ
ಅನು-ಗಾಲವೂ
ಅನುಗುಣ-ವಾಗಿ
ಅನುಗುಣ-ವಾಗಿಯೇ
ಅನುಚಿತ-ವಾಗಿ-ದ್ದರೂ
ಅನುಚಿತ-ವಾಗಿರು
ಅನು-ದಿನ
ಅನು-ದಿನದ
ಅನುಭವ
ಅನುಭವಕ್ಕೆ
ಅನುಭವ-ಗಳ
ಅನುಭವ-ಗ-ಳನ್ನು
ಅನುಭವ-ಗಳಿಂದ
ಅನುಭವ-ಗಳಿಗೆ
ಅನುಭವ-ಗಳು
ಅನುಭವ-ಗಳೂ
ಅನುಭವ-ಗಳೆಲ್ಲ
ಅನುಭವ-ಗಳೆಲ್ಲ-ವನ್ನೂ
ಅನುಭವ-ಗಳೊಂದಿಗೆ
ಅನುಭವದ
ಅನುಭವ-ದಂತೆ
ಅನುಭವ-ದಲ್ಲಿ
ಅನುಭವ-ದಿಂದ
ಅನುಭವ-ದೆ-ಡೆಗೆ
ಅನುಭವ-ದೊಂದಿಗೆ
ಅನುಭವ-ವನ್ನು
ಅನುಭವ-ವನ್ನೆಲ್ಲ
ಅನುಭವ-ವನ್ನೇ
ಅನುಭವ-ವಾಗುತ್ತದೆ
ಅನುಭವ-ವಾಗು-ವು-ದಿಲ್ಲ
ಅನುಭವ-ವಾ-ದರೆ
ಅನುಭವ-ವಿದೆ
ಅನುಭವ-ವಿದ್ದರೆ
ಅನುಭವ-ವಿಲ್ಲದೆ
ಅನುಭವವು
ಅನುಭವವೂ
ಅನುಭವ-ವೆಂದು
ಅನುಭವ-ವೆಲ್ಲ
ಅನುಭವವೇ
ಅನುಭವ-ವೇದ್ಯ-ವಾದ
ಅನುಭವ-ವೊಂದೇ
ಅನುಭವಿ
ಅನುಭವಿಸ
ಅನುಭವಿ-ಸದೆ
ಅನುಭವಿ-ಸ-ಬಲ್ಲ-ವ-ನನ್ನು
ಅನುಭವಿ-ಸ-ಬಲ್ಲವು
ಅನುಭವಿ-ಸ-ಬಲ್ಲೆವು
ಅನುಭವಿ-ಸ-ಬಹುದು
ಅನುಭವಿ-ಸ-ಬೇಕಾ
ಅನುಭವಿ-ಸ-ಬೇಕಾ-ಗಿಲ್ಲವೊ
ಅನುಭವಿ-ಸ-ಬೇಕಾ-ಗು-ವುದು
ಅನುಭವಿ-ಸ-ಬೇಕು
ಅನುಭವಿ-ಸ-ಲಾರೆವು
ಅನುಭವಿ-ಸಲು
ಅನುಭವಿ-ಸ-ಲೇ-ಬೇಕು
ಅನುಭವಿಸಿ
ಅನುಭವಿ-ಸಿದ
ಅನುಭವಿ-ಸಿ-ದನು
ಅನುಭವಿ-ಸಿ-ದೆ-ವೆಂದು
ಅನುಭವಿ-ಸಿಯೇ
ಅನುಭವಿ-ಸಿರ
ಅನುಭವಿ-ಸಿ-ರ-ಬೇಕು
ಅನುಭವಿ-ಸಿ-ರು-ವು-ದ-ರಿಂದ
ಅನುಭವಿಸು
ಅನುಭವಿ-ಸುತ್ತ
ಅನುಭವಿ-ಸುತ್ತಿ-ರುವ
ಅನುಭವಿ-ಸುತ್ತಿ-ರು-ವಂತೆ
ಅನುಭವಿ-ಸುತ್ತಿ-ರು-ವೆನು
ಅನುಭವಿ-ಸುತ್ತಿ-ರು-ವೆವು
ಅನುಭವಿ-ಸುತ್ತೇ-ನೆಯೊ
ಅನುಭವಿ-ಸುತ್ತೇ-ನೆಯೋ
ಅನುಭವಿ-ಸುತ್ತೇವೆ
ಅನುಭವಿ-ಸುತ್ತೇವೆಯೊ
ಅನುಭವಿ-ಸುವ
ಅನುಭವಿ-ಸು-ವನು
ಅನುಭವಿ-ಸು-ವರು
ಅನುಭವಿ-ಸು-ವರೊ
ಅನುಭವಿ-ಸು-ವ-ವರೆಗೆ
ಅನುಭವಿ-ಸು-ವಿರಿ
ಅನುಭವಿ-ಸು-ವು-ದಕ್ಕಾಗಿ
ಅನುಭವಿ-ಸು-ವು-ದಕ್ಕೆ
ಅನುಭವಿ-ಸು-ವು-ದಿಲ್ಲ
ಅನುಭವಿ-ಸು-ವುದು
ಅನುಭವಿ-ಸು-ವೆವು
ಅನು-ಭಾವದ
ಅನುಭಾವಿಗೆ
ಅನು-ಭೂತ
ಅನು-ಭೂತ-ವಿಷ-ಯಾ-ಸಂಪ್ರ-ಮೋಷಃ
ಅನು-ಮತಿ
ಅನು-ಮಾನ
ಅನು-ಮಾ-ನಕ್ಕೆ
ಅನು-ಮಾನ-ಗಳಿಂದ
ಅನು-ಮಾನ-ಗಳು
ಅನು-ಮಾನ-ವಿದೆ
ಅನು-ಮಾನ-ವಿಲ್ಲ
ಅನು-ಮಾನವೂ
ಅನು-ಮಾನಾಸ್ಪದ-ವಾಗಿ-ರ-ಬಹುದು
ಅನು-ಮಾನಿಸಿ
ಅನು-ಮಾನಿಸು-ವಂತೆ
ಅನು-ಮಾನಿಸು-ವರು
ಅನು-ಮಾನಿ-ಸು-ವೆವು
ಅನು-ಮೋದಿ-ಸಿದ್ದಾರೆ
ಅನು-ಮೋದಿಸಿ-ರ-ಬಹುದು
ಅನು-ಯಾಯಿ
ಅನು-ಯಾಯಿ-ಗ-ಳನ್ನು
ಅನು-ಯಾಯಿ-ಗಳಷ್ಟು
ಅನು-ಯಾಯಿ-ಗಳಾ-ಗಿ-ರು-ವ-ವರು
ಅನು-ಯಾಯಿ-ಗಳಿಂದ
ಅನು-ಯಾಯಿ-ಗಳು
ಅನು-ರಣಿತ-ವಾಗಿ
ಅನು-ರಣಿತ-ವಾಗು-ವ-ವರೆ-ವಿಗೂ
ಅನು-ರುಣಿತ-ವಾ-ಗಲಿ
ಅನು-ವಂಶಿಕ
ಅನು-ವಂಶಿಕ-ವಾಗಿ
ಅನು-ವಂಶಿಕ-ವೆಂದು
ಅನು-ವಾದ
ಅನುಷ್ಟಾನ
ಅನುಷ್ಠಾನ
ಅನುಷ್ಠಾನ-ಇವು
ಅನುಷ್ಠಾ-ನಕ್ಕೆ
ಅನುಷ್ಠಾನದ
ಅನುಷ್ಠಾನ-ದಲ್ಲಿ
ಅನುಷ್ಠಾನ-ಯೋಗ್ಯ
ಅನುಷ್ಠಾನ-ಯೋಗ್ಯ-ವಾಗಿ
ಅನುಷ್ಠಾನ-ಯೋಗ್ಯ-ವಾ-ಗಿತ್ತು
ಅನುಷ್ಠಾನ-ಯೋಗ್ಯ-ವಾದ
ಅನುಷ್ಠಾನ-ಯೋಗ್ಯ-ವಾದುದು
ಅನುಷ್ಠಾನ-ಯೋಗ್ಯ-ವೆಂದು
ಅನುಷ್ಠಾನ-ವನ್ನು
ಅನುಷ್ಠಾನ-ವಾಗ-ಲಿಲ್ಲ
ಅನುಷ್ಠಾನ-ವೆಂಬ
ಅನುಷ್ಠಾನಿಕ
ಅನು-ಸರಿಸ
ಅನು-ಸರಿ-ಸ-ಬೇಕಾ-ಗಿದೆ
ಅನು-ಸರಿ-ಸ-ಬೇಕಾ-ಗಿಲ್ಲ
ಅನು-ಸರಿ-ಸ-ಬೇಕಾದ
ಅನು-ಸರಿ-ಸ-ಬೇಕು
ಅನು-ಸರಿ-ಸಲಿ
ಅನು-ಸರಿ-ಸಲು
ಅನು-ಸರಿಸಿ
ಅನು-ಸರಿಸಿ-ಕೊಂಡು
ಅನು-ಸರಿಸಿ-ದಂತೆ
ಅನು-ಸರಿಸಿ-ದರೆ
ಅನು-ಸರಿಸಿ-ದಾಗ
ಅನು-ಸರಿಸಿ-ರು-ವುದು
ಅನು-ಸರಿಸು
ಅನು-ಸರಿಸು-ತ್ತದೆ
ಅನು-ಸರಿ-ಸುತ್ತಾ
ಅನು-ಸರಿ-ಸುತ್ತಾರೆ
ಅನು-ಸರಿ-ಸುತ್ತಿದ್ದೇವೆ
ಅನು-ಸರಿ-ಸುತ್ತಿ-ರು-ವರು
ಅನು-ಸರಿ-ಸುತ್ತಿ-ರುವುದು
ಅನು-ಸರಿ-ಸುತ್ತಿ-ರು-ವೆವು
ಅನು-ಸರಿ-ಸುತ್ತೇನೆ
ಅನು-ಸರಿ-ಸುವ
ಅನು-ಸರಿ-ಸು-ವನು
ಅನು-ಸರಿ-ಸು-ವರು
ಅನು-ಸರಿ-ಸು-ವುದನ್ನೇ
ಅನು-ಸರಿ-ಸು-ವುದು
ಅನು-ಸಾರ-ವಾಗಿ
ಅನೃತ
ಅನೃತ-ವೆಂದು
ಅನೃ-ತವೇ
ಅನೇಕ
ಅನೇ-ಕತೆ
ಅನೇಕ-ದಂತೆ
ಅನೇಕ-ದಂತೆ-ವಸ್ತು
ಅನೇಕ-ರನ್ನು
ಅನೇಕ-ರಿಗೆ
ಅನೇ-ಕರು
ಅನೇಕ-ವನ್ನು
ಅನೇಕ-ವಾಗಿ
ಅನೇಕ-ವಾಗಿದೆ
ಅನೇಕ-ವಾ-ಗಿಲ್ಲ
ಅನೇಕ-ವಾ-ಗು-ವುದು
ಅನೇ-ಕವು
ಅನೇಕ-ವೆಲ್ಲವೂ
ಅನೇ-ಕವೇ
ಅನೇಕ-ವೇಳೆ
ಅನೈಚ್ಛಿಕ
ಅನೈಚ್ಛಿಕ-ವಾಗಿ
ಅನೈಚ್ಛಿಕ-ವಾಗಿದೆ
ಅನ್ನ
ಅನ್ನಿ-ಸಿದ್ದರೆ
ಅನ್ನಿ-ಸುತ್ತದೆ
ಅನ್ನು-ವುದು
ಅನ್ಯ
ಅನ್ಯಥಾ
ಅನ್ಯ-ಥಾ-ರೀ-ತಿಗೆ
ಅನ್ಯ-ದೇವ-ತಾ-ರಾಧನೆ-ಯಲ್ಲಿ
ಅನ್ಯ-ಭಾಷೆ-ಯಲ್ಲಿ
ಅನ್ಯ-ರನ್ನು
ಅನ್ಯ-ರಿಗೆ
ಅನ್ಯ-ರಿಲ್ಲ
ಅನ್ಯರು
ಅನ್ಯಾಯ
ಅನ್ಯಾಯ-ವನ್ನು
ಅನ್ಯೋನ್ಯ
ಅನ್ವ-ಯಿಸ-ಬಾ-ರದು
ಅನ್ವಯಿ-ಸ-ಬೇಕು
ಅನ್ವ-ಯಿಸ-ಲಾ-ಗಿದೆ
ಅನ್ವ-ಯಿಸ-ಲಾಗು-ವು-ದಿಲ್ಲ
ಅನ್ವ-ಯಿಸಿ
ಅನ್ವ-ಯಿಸಿ-ದರೆ
ಅನ್ವಯಿ-ಸುತ್ತದೆ
ಅನ್ವ-ಯಿಸುತ್ತೀರಿ-ತಕ್ಷಣವೇ
ಅನ್ವಯಿ-ಸುವ
ಅನ್ವಯಿ-ಸು-ವಂತೆ
ಅನ್ವಯಿ-ಸು-ವು-ದಕ್ಕೆ
ಅನ್ವಯಿ-ಸುವು-ದ-ರಿಂದ
ಅನ್ವಯಿ-ಸು-ವು-ದಲ್ಲ
ಅನ್ವಯಿ-ಸು-ವು-ದಿಲ್ಲ
ಅನ್ವಯಿ-ಸು-ವುದು
ಅನ್ವಯಿ-ಸು-ವುದೆ
ಅನ್ವಯಿ-ಸು-ವುದೆಂದು
ಅನ್ವಯಿ-ಸು-ವುದೆಂದೂ
ಅನ್ವೇಷಣ-ಯೊಂದೆ
ಅನ್ವೇಷಣಾ
ಅನ್ವೇಷಣೆ
ಅನ್ವೇಷಣೆಯ
ಅನ್ವೇಷಣೆಯು
ಅಪ-ಕರ್ಷಣ
ಅಪ-ಕಾರ
ಅಪ-ನಂಬಿಕೆ
ಅಪ-ನಂಬಿ-ಕೆಯ
ಅಪ-ಮಾನ
ಅಪರಿ
ಅಪರಿಗ್ರಹ
ಅಪರಿಗ್ರಹ-ಇವು
ಅಪರಿಗ್ರಹ-ದಲ್ಲಿ
ಅಪರಿಗ್ರಹಸ್ಥೈರ್ಯೇ
ಅಪರಿ-ಮಿತ
ಅಪರಿ-ಮಿತ-ವಾದು-ದನ್ನು
ಅಪ-ರೂಪ
ಅಪ-ವಾದ
ಅಪ-ವಾದವೇ
ಅಪವಿತ್ರ
ಅಪವಿತ್ರ-ವಾಗು-ವುದು
ಅಪ-ಹರಿಸು-ತ್ತಿರು-ವೆವು
ಅಪ-ಹರಿ-ಸುವ
ಅಪ-ಹರಿ-ಸು-ವು-ದಕ್ಕೆ
ಅಪ-ಹಾಸ್ಯಕ್ಕೆ
ಅಪಾ
ಅಪಾಯ
ಅಪಾಯ-ಕರ
ಅಪಾಯ-ಕರ-ವಾಗಿ
ಅಪಾಯ-ಕರ-ವಾದ
ಅಪಾಯ-ಕ-ರವೆ
ಅಪಾಯ-ಕರ-ವೆಂದು
ಅಪಾಯ-ಗಳನ್ನೆಲ್ಲಾ
ಅಪಾಯ-ಗಳಿಂದ
ಅಪಾಯ-ಗಳಿವೆ
ಅಪಾಯ-ಗಳೆಷ್ಟು
ಅಪಾಯ-ದಿಂದ
ಅಪಾಯ-ವನ್ನು
ಅಪಾಯ-ವಾಗ-ಬಹುದು
ಅಪಾಯ-ವಿದೆ
ಅಪಾಯ-ವಿ-ರ-ಬಹುದು
ಅಪಾಯ-ವಿರು-ವಲ್ಲಿ
ಅಪಾಯ-ವಿಲ್ಲ-ವೆಂದು
ಅಪಾಯವೂ
ಅಪಾರ
ಅಪಾರ-ವಾದ
ಅಪಾರ-ಶಕ್ತಿ
ಅಪಾರ್ಥ-ದಲ್ಲಿ
ಅಪೂರ್ಣ-ತೆಯ
ಅಪೂರ್ಣ-ತೆ-ಯನ್ನು
ಅಪೂರ್ಣ-ನಾಗು-ವನು
ಅಪೂರ್ಣರು
ಅಪೂರ್ಣ-ವಾದ
ಅಪೂರ್ವ
ಅಪೂರ್ವ-ವಾದ
ಅಪೇಕ್ಷಿಸಿ-ದರೋ
ಅಪೇಕ್ಷಿಸು-ವ-ವರು
ಅಪೇಕ್ಷಿ-ಸು-ವು-ದಿಲ್ಲ
ಅಪೋಹ
ಅಪೌರುಷೇಯ
ಅಪೌರುಷೇಯ-ವಾದ
ಅಪ್ಪಣೆ
ಅಪ್ಪ-ಣೆಯೂ
ಅಪ್ಪಳಿಸಿ
ಅಪ್ಪಿ-ಕೊಂಡಿ
ಅಪ್ಪಿ-ಕೊಂಡಿದ್ದ
ಅಪ್ಪಿ-ಕೊಂಡಿ-ರುವ
ಅಪ್ಪಿ-ಕೊಂಡಿ-ರುವನು
ಅಪ್ರಕ-ಟಿತ-ವಾಗಿ-ರಲಿ
ಅಪ್ರಕ-ಟಿತ-ವಾಗಿ-ರುತ್ತವೆ
ಅಪ್ರಜ್ಞಾ-ಪೂರ್ವಕ
ಅಪ್ರಜ್ಞಾ-ಪೂರ್ವ-ಕ-ವಾಗಿ
ಅಪ್ರಜ್ಞೆ
ಅಪ್ರಜ್ಞೆಯ
ಅಪ್ರ-ಧಾನ-ವಾದ
ಅಪ್ರ-ಬುದ್ಧ
ಅಬಲ-ನೆಂದು
ಅಬಾಧಿತ
ಅಬ್ಬರ-ವಿದೆ
ಅಭದ್ರ
ಅಭವ್ಯಕ್ತಿ
ಅಭಾವ
ಅಭಾವಪ್ರತ್ಯಯಾ-ಲಂಬನಾ
ಅಭಾವ-ವೆಂದು
ಅಭಾವ-ವೆಂದೂ
ಅಭಾವ-ವೆಂಬ
ಅಭಿ
ಅಭಿ-ನ-ಯಿಸಿ
ಅಭಿಪ್ರಾಯ
ಅಭಿಪ್ರಾಯ-ಅ-ವಿ-ಕಾರಿ-ಯಾ-ದುದು
ಅಭಿಪ್ರಾ-ಯಕ್ಕೆ
ಅಭಿಪ್ರಾಯ-ಗಳ
ಅಭಿಪ್ರಾಯ-ಗ-ಳನ್ನು
ಅಭಿಪ್ರಾಯ-ಗಳಲ್ಲಿ
ಅಭಿಪ್ರಾಯ-ಗಳಿವೆ
ಅಭಿಪ್ರಾಯ-ಗಳಿ-ವೆ-ಒಂದು
ಅಭಿಪ್ರಾಯ-ಗಳು
ಅಭಿಪ್ರಾಯದ
ಅಭಿಪ್ರಾಯ-ದಂತೆ
ಅಭಿಪ್ರಾಯ-ದಲ್ಲಿ
ಅಭಿಪ್ರಾಯ-ದೊಂದಿಗೆ
ಅಭಿಪ್ರಾಯ-ವನ್ನು
ಅಭಿಪ್ರಾಯ-ವಿ-ದೆಯೋ
ಅಭಿಪ್ರಾಯವು
ಅಭಿಪ್ರಾಯವೆ
ಅಭಿಪ್ರಾಯ-ವೇನೆಂದರೆ
ಅಭಿಪ್ರಾಯವ್ಕು
ಅಭಿ-ಮಾನ
ಅಭಿ-ಮಾನ-ವಿದೆ
ಅಭಿ-ಮಾನ-ವಿದ್ದರೂ
ಅಭಿ-ಮುಖ-ವಾಗಿ
ಅಭಿ-ವೃದ್ಧಿ
ಅಭಿ-ವೃದ್ಧಿಗೆ
ಅಭಿ-ವೃದ್ಧಿ-ಯಲ್ಲಿವೆ
ಅಭಿ-ವೃದ್ಧಿ-ಯಾಗ-ಬೇಕು
ಅಭಿ-ವೃದ್ಧಿ-ಯಾಗಿ
ಅಭಿ-ವೃದ್ಧಿ-ಯಾಗಿಲ್ಲವೋ
ಅಭಿ-ವೃದ್ಧಿ-ಯಾ-ಗುತ್ತದೆ
ಅಭಿ-ವೃದ್ಧಿ-ಯಾಗುತ್ತಿದೆ
ಅಭಿ-ವೃದ್ಧಿ-ಯಾಗುತ್ತಿ-ರುವ
ಅಭಿ-ವೃದ್ಧಿ-ಯಾಗುತ್ತಿವೆ
ಅಭಿ-ವೃದ್ಧಿ-ಯಾಗು-ವುದೂ
ಅಭಿ-ವೃದ್ಧಿ-ಯಾ-ರುತ್ತದೆ
ಅಭಿವ್ಯಕ್ತ
ಅಭಿವ್ಯಕ್ತ-ಗೊಳಿ-ಸು-ವುದು
ಅಭಿವ್ಯಕ್ತ-ಪಡಿಸು-ವು-ದ-ರಲ್ಲಿ
ಅಭಿವ್ಯಕ್ತ-ವಾಗುತ್ತಿದೆ
ಅಭಿವ್ಯಕ್ತ-ವಾಗುತ್ತಿ-ರುವ
ಅಭಿವ್ಯಕ್ತ-ವಾದ
ಅಭಿವ್ಯಕ್ತಿ
ಅಭಿವ್ಯಕ್ತಿ-ಗ-ಳನ್ನು
ಅಭಿವ್ಯಕ್ತಿ-ಗಳಲ್ಲದೆ
ಅಭಿವ್ಯಕ್ತಿ-ಗಳಿಗೂ
ಅಭಿವ್ಯಕ್ತಿ-ಗಳು
ಅಭಿವ್ಯಕ್ತಿ-ಗ-ಳೆಂದು
ಅಭಿವ್ಯಕ್ತಿ-ಗಳೆಲ್ಲ
ಅಭಿವ್ಯಕ್ತಿ-ಗಳೇನು
ಅಭಿವ್ಯಕ್ತಿಗೂ
ಅಭಿವ್ಯಕ್ತಿಗೆ
ಅಭಿವ್ಯಕ್ತಿ-ಗೊಳ್ಳುತ್ತಿ-ರುವುದೂ
ಅಭಿವ್ಯಕ್ತಿ-ಗೊಳ್ಳು-ವುದೇ
ಅಭಿವ್ಯಕ್ತಿಯ
ಅಭಿವ್ಯಕ್ತಿ-ಯನ್ನು
ಅಭಿವ್ಯಕ್ತಿ-ಯಲ್ಲಿ
ಅಭಿವ್ಯಕ್ತಿ-ಯಲ್ಲಿಯೇ
ಅಭಿವ್ಯಕ್ತಿ-ಯಲ್ಲಿ-ರುವ
ಅಭಿವ್ಯಕ್ತಿ-ಯಾ-ಗಿದೆ
ಅಭಿವ್ಯಕ್ತಿಯು
ಅಭಿವ್ಯಕ್ತಿಯೂ
ಅಭಿವ್ಯಕ್ತಿಯೆ
ಅಭಿವ್ಯಕ್ತಿ-ಯೆಂದೂ
ಅಭಿವ್ಯಕ್ತಿ-ಯೆಲ್ಲ
ಅಭಿವ್ಯಕ್ತಿಯೇ
ಅಭಿವ್ಯಕ್ತಿಯೊ
ಅಭಿ-ಶಾಪ
ಅಭೇದ್ಯ-ಕೋಟೆ-ಯಂತೆ
ಅಭೇದ್ಯ-ವಾದ
ಅಭೇದ್ಯ-ವಾದರೂ
ಅಭ್ಯಾಸ
ಅಭ್ಯಾಸಕ್ಕಾಗಿಯೇ
ಅಭ್ಯಾಸಕ್ಕೆ
ಅಭ್ಯಾಸ-ಗಳ
ಅಭ್ಯಾಸ-ಗಳನ್ನು
ಅಭ್ಯಾಸ-ಗಳನ್ನೂ
ಅಭ್ಯಾಸ-ಗಳಲ್ಲಿ
ಅಭ್ಯಾಸ-ಗಳಿಂದ
ಅಭ್ಯಾಸ-ಗಳು
ಅಭ್ಯಾ-ಸದ
ಅಭ್ಯಾಸ-ದಲ್ಲಿ
ಅಭ್ಯಾಸ-ದಿಂದ
ಅಭ್ಯಾಸ-ಬಲದ
ಅಭ್ಯಾಸ-ಬಲ-ದಿಂದ
ಅಭ್ಯಾಸ-ಮಾಡದೆ
ಅಭ್ಯಾಸ-ಮಾಡಿ
ಅಭ್ಯಾಸ-ಮಾಡಿದ
ಅಭ್ಯಾಸ-ಮಾಡಿ-ರು-ವುದು
ಅಭ್ಯಾಸ-ಮಾಡುವ
ಅಭ್ಯಾಸ-ಮಾಡು-ವಂತೆ
ಅಭ್ಯಾಸ-ಮಾಡು-ವು-ದ-ರಿಂದ
ಅಭ್ಯಾಸ-ವನ್ನು
ಅಭ್ಯಾಸ-ವಾಗಿದೆ
ಅಭ್ಯಾಸ-ವಾದ
ಅಭ್ಯಾಸ-ವಿಲ್ಲದೇ
ಅಭ್ಯಾಸವು
ಅಭ್ಯಾಸವೂ
ಅಭ್ಯಾಸ-ವೆಂದರೆ
ಅಭ್ಯಾಸ-ವೆಂದ-ರೇನು
ಅಭ್ಯಾಸವೇ
ಅಭ್ಯಾಸ-ವೈ-ರಾಗ್ಯಾಭ್ಯಾಂ
ಅಭ್ಯುದಯ
ಅಭ್ಯುದ-ಯದ
ಅಭ್ಯುದ-ಯವು
ಅಮರ
ಅಮರತ್ವ
ಅಮರತ್ವಕ್ಕೆ
ಅಮರತ್ವದ
ಅಮರತ್ವ-ವನ್ನು
ಅಮರತ್ವ-ವಲ್ಲ
ಅಮರ-ರಾ-ಗಲು
ಅಮರರು
ಅಮರ-ವಾಗ-ಲಾರದು
ಅಮರ-ವಾದು-ದಾದರೆ
ಅಮರ-ವಾ-ದುದು
ಅಮಾನುಷ
ಅಮಿತ-ಭಾವ-ನೆ-ಯನ್ನು
ಅಮಿತ-ವಲ್ಲವೇ
ಅಮಿಶ್ರ-ವಾದುದು
ಅಮೂರ್ತ
ಅಮೂರ್ತ-ಭಾವ-ನೆಯು
ಅಮೂರ್ತೀ-ಕರ-ಣ-ಗೊಂಡ
ಅಮೂಲ್ಯ-ವಾದ
ಅಮೃತ
ಅಮೃ-ತತ್ತ್ವ
ಅಮೃ-ತತ್ತ್ವಂ
ಅಮೃ-ತತ್ವ
ಅಮೃತತ್ವದ
ಅಮೃ-ತತ್ವ-ವನ್ನು
ಅಮೃತ-ನಾಗು-ವನು
ಅಮೃತ-ಪುತ್ರರೆ
ಅಮೃತ-ಪುತ್ರರೆಂದು
ಅಮೃತ-ಪುತ್ರರೇ
ಅಮೃತ-ಮ-ಯವೂ
ಅಮೃ-ತರು
ಅಮೃತ-ವಾ-ಗಿ-ರುವುದು
ಅಮೃತ-ವಾ-ದುದು
ಅಮೃತಾತ್ಮರೂ
ಅಮೆರಿಕ
ಅಮೆರಿ-ಕದ
ಅಮೆರಿಕ-ದಲ್ಲಿ
ಅಮೆರಿಕಾ
ಅಮೆರಿಕಾ-ದಲ್ಲಿ
ಅಮೆರಿಕಾ-ದಲ್ಲಿ-ರುವಷ್ಟೆ
ಅಮೇರಿಕ
ಅಮೇರಿ-ಕದ
ಅಮೇರಿಕಾ
ಅಮೇರಿಕಾ-ದಲ್ಲಿ
ಅಮೋಘ
ಅಮೋಘ-ವಾದ
ಅಮೋಘ-ಶಾಸ್ತ್ರ
ಅಯಸ್ಕಾಂತ
ಅಯುಕ್ತ
ಅಯುಕ್ತ-ವಾಗಿ
ಅಯುಕ್ತಿ-ಕರ-ವಾಗಿ-ರ-ಬೇಕು
ಅಯುಕ್ತಿ-ಯಾಗು-ವುದು
ಅಯೋಗ್ಯ-ರಾಗಿರು-ವೆವು
ಅಯೋಗ್ಯ-ರೆಂದು
ಅಯ್ಯಾ
ಅಯ್ಯೊ
ಅಯ್ಯೋ
ಅರಗಿಳಿ-ಗಳಂತೆ
ಅರ-ಗಿಸಿ-ಕೊಳ್ಳು-ವಾಗ
ಅರ-ಗಿಸಿ-ಕೊಳ್ಳು-ವುದು
ಅರಚಿ-ಕೊಳ್ಳುತ್ತೇವೆ
ಅರ-ಚಿತು
ಅರಚುತ್ತಿತ್ತು
ಅರಚುತ್ತಿದ್ದರೆ
ಅರಚು-ವುದು
ಅರಣ್ಯ
ಅರಣ್ಯದ
ಅರಣ್ಯ-ದಲ್ಲಿ
ಅರಣ್ಯಾಶ್ರಮ-ದಲ್ಲಿ
ಅರ-ಮನೆಯ
ಅರಳಿದ
ಅರಸ-ಬೇಕು
ಅರಸಲು
ಅರಸಿ
ಅರಸು
ಅರಸು-ತ್ತಿದ್ದನೊ
ಅರಸು-ತ್ತಿ-ರು-ವನು
ಅರಸುತ್ತಿ-ರು-ವರೋ
ಅರಸುತ್ತಿ-ರುವು
ಅರಸು-ವಂತೆ
ಅರಸು-ವರು
ಅರಸು-ವ-ವನು
ಅರಸು-ವಿಕೆ
ಅರಸು-ವು-ದಕ್ಕೆ
ಅರಸು-ವುದು
ಅರಸು-ವು-ದೆಲ್ಲಿ
ಅರಸು-ವುದೇ
ಅರಾಧಿಸ-ಬಹುದು
ಅರಿಗಿಳಿ-ಗಳೂ
ಅರಿತ
ಅರಿ-ತರು
ಅರಿ-ತರೆ
ಅರಿ-ತಾಗ
ಅರಿತಿ-ರುವನೊ
ಅರಿತಿಲ್ಲವೊ
ಅರಿತು
ಅರಿತುಕೊ
ಅರಿತು-ಕೊಳ್ಳಿ
ಅರಿತೆ
ಅರಿತೇ
ಅರಿತೊ
ಅರಿಯ
ಅರಿ-ಯದೆ
ಅರಿಯ-ದೆಂದು
ಅರಿಯ-ದೆಯೊ
ಅರಿಯನು
ಅರಿಯ-ಬೇಕಾ-ದರೆ
ಅರಿ-ಯರು
ಅರಿಯ-ಲಾರರು
ಅರಿಯ-ಲಿಚ್ಛಿ-ಸು-ವನು
ಅರಿ-ಯಲು
ಅರಿಯಳು
ಅರಿಯು
ಅರಿ-ಯುತ್ತ
ಅರಿಯುತ್ತಾನೆ
ಅರಿಯುತ್ತೇವೆಯೋ
ಅರಿಯು-ವಂತೆ
ಅರಿಯು-ವನು
ಅರಿಯು-ವರೋ
ಅರಿಯು-ವುದು
ಅರಿಯು-ವೆನು
ಅರಿ-ವನ್ನು
ಅರಿ-ವನ್ನುಂಟು-ಮಾಡುವ
ಅರಿ-ವಾಗ-ದಿ-ರುವ
ಅರಿ-ವಾಗಿ-ದೆಯೋ
ಅರಿ-ವಾ-ಗು-ವುದು
ಅರಿ-ವಾಗು-ವುದೇ
ಅರಿ-ವಾ-ದರೆ
ಅರಿ-ವಿಗೆ
ಅರಿ-ವಿನ
ಅರಿವಿ-ನಾಚೆಯೂ
ಅರಿ-ವಿ-ನಿಂದ
ಅರಿ-ವಿಲ್ಲ
ಅರಿ-ವಿಲ್ಲದ
ಅರಿ-ವಿಲ್ಲದೆ
ಅರಿ-ವಿಲ್ಲ-ದೆಯೇ
ಅರಿ-ವಿಲ್ಲದೇ
ಅರಿವು
ಅರಿವುಂಟಾ-ಗು-ವುದು
ಅರಿವೇ
ಅರಿಷ್ಟ
ಅರುಣೋದಯ
ಅರೆ-ವಾಸಿ
ಅರೇಬಿ-ಯದ
ಅರೇಬಿಯಾ
ಅರ್ಜುನ-ನಿಗೆ
ಅರ್ಜುನ-ನೊಂದಿಗೆ
ಅರ್ಥ
ಅರ್ಥಕ್ಕೆ
ಅರ್ಥ-ಗಳ
ಅರ್ಥ-ಗ-ಳನ್ನು
ಅರ್ಥ-ಗಳೂ
ಅರ್ಥದ
ಅರ್ಥ-ದಂತೆ
ಅರ್ಥ-ದಲ್ಲಿ
ಅರ್ಥ-ದಲ್ಲೇ
ಅರ್ಥ-ದಿಂದ
ಅರ್ಥ-ಮಾಡಿ
ಅರ್ಥ-ಮಾಡಿ-ಕೊಳ್ಳದೆ
ಅರ್ಥ-ಮಾಡಿ-ಕೊಳ್ಳ-ಬಲ್ಲ
ಅರ್ಥ-ಮಾಡಿ-ಕೊಳ್ಳ-ಬೇಕಾ-ದರೆ
ಅರ್ಥ-ಮಾಡಿ-ಕೊಳ್ಳ-ಬೇಕು
ಅರ್ಥ-ಮಾಡಿ-ಕೊಳ್ಳ-ಲಾರ
ಅರ್ಥ-ಮಾಡಿ-ಕೊಳ್ಳ-ಲೆಂದು
ಅರ್ಥ-ಮಾಡಿ-ಕೊಳ್ಳುತ್ತಾನೆ
ಅರ್ಥ-ಮಾಡಿ-ಕೊಳ್ಳುತ್ತೇವೆ
ಅರ್ಥ-ಮಾಡಿ-ಕೊಳ್ಳು-ವು-ದಕ್ಕೆ
ಅರ್ಥ-ಮಾಡಿ-ಕೊಳ್ಳು-ವುದು
ಅರ್ಥ-ಮಾಡಿ-ಕೊಳ್ಳು-ವೆನು
ಅರ್ಥ-ಮಾಡಿ-ಕೊಳ್ಳೋಣ
ಅರ್ಥ-ಮಾತ್ರ-ನಿರ್ಭಾಸಂ
ಅರ್ಥ-ವತ್ವದ
ಅರ್ಥ-ವನ್ನು
ಅರ್ಥ-ವಲ್ಲ
ಅರ್ಥ-ವಾಗ
ಅರ್ಥ-ವಾ-ಗಿತ್ತು
ಅರ್ಥ-ವಾಗು
ಅರ್ಥ-ವಾಗುತ್ತದೆ
ಅರ್ಥ-ವಾಗು-ವಂತಿಲ್ಲ
ಅರ್ಥ-ವಾಗು-ವಂತೆ
ಅರ್ಥ-ವಾಗು-ವು-ದಿಲ್ಲ
ಅರ್ಥ-ವಾ-ಗು-ವುದು
ಅರ್ಥ-ವಾ-ದಂತೆಲ್ಲ
ಅರ್ಥ-ವಿತ್ತೆಂಬು-ದನ್ನು
ಅರ್ಥ-ವಿದೆ
ಅರ್ಥ-ವಿಲ್ಲ
ಅರ್ಥ-ವಿಲ್ಲದ
ಅರ್ಥ-ವಿಲ್ಲ-ದಂತೆ
ಅರ್ಥ-ವಿಲ್ಲ-ದವು
ಅರ್ಥವು
ಅರ್ಥವೂ
ಅರ್ಥವೆ
ಅರ್ಥ-ವೆಂದರೆ
ಅರ್ಥ-ವೆಂದು
ಅರ್ಥ-ವೆನ್ನು-ವುದು
ಅರ್ಥವೇ
ಅರ್ಥ-ವೇನು
ಅರ್ಥ-ವೇನೆಂದರೆ
ಅರ್ಥ-ಸರಣಿ
ಅರ್ಥ-ಹೀನ
ಅರ್ಧ
ಅರ್ಧಕ್ಕಿಂತ
ಅರ್ಧ-ಗಂಟೆ
ಅರ್ಧ-ಗಂಟೆಯ
ಅರ್ಧ-ಗಂಟೆ-ಯಲ್ಲಿ
ಅರ್ಧ-ಜಾಗ್ರ-ತಾ-ವಸ್ಥೆಯ
ಅರ್ಧ-ನಿದ್ರೆಯ
ಅರ್ಧ-ಭಾಗ-ವಾದ
ಅರ್ಧ-ವನ್ನು
ಅರ್ಪಿ-ಸಿದ
ಅರ್ಪಿ-ಸಿದರೆ
ಅರ್ಪಿ-ಸುವ
ಅರ್ಪಿ-ಸುವರು
ಅರ್ಪಿ-ಸು-ವುದು-ಇವು-ಗಳಿಗೆ
ಅರ್ಹತೆ
ಅರ್ಹ-ನಲ್ಲ
ಅರ್ಹ-ನಾದ
ಅರ್ಹರು
ಅರ್ಹ-ವಾ-ಗಿಲ್ಲ
ಅರ್ಹ-ವಾದುದೇ
ಅಲಂಕಾರ
ಅಲಂಕಾರ-ಗಳಿಂದ
ಅಲಂಕಾರಪ್ರಾಯ-ನಾ-ಗಿ-ರುವ
ಅಲಂಕಾರಿಕ
ಅಲಂಕೃತ-ವಾ-ದಂತೆ
ಅಲ-ಗಿನ
ಅಲಭ್ಯ
ಅಲ-ರಂತೆ
ಅಲರು
ಅಲು-ಗಾಡಿ-ಸದು
ಅಲು-ಗಾಡು
ಅಲೆ
ಅಲೆ-ಗಳ
ಅಲೆ-ಗ-ಳನ್ನು
ಅಲೆ-ಗ-ಳನ್ನೂ
ಅಲೆ-ಗಳನ್ನೆಲ್ಲ
ಅಲೆ-ಗಳಾ-ಗಿ-ರ-ಬಹುದು
ಅಲೆ-ಗಳಿಂದ
ಅಲೆ-ಗಳಿಗೆ
ಅಲೆ-ಗಳಿದ್ದವು
ಅಲೆ-ಗಳಿಲ್ಲದೆ
ಅಲೆ-ಗಳಿವೆ
ಅಲೆ-ಗಳು
ಅಲೆ-ಗಳೆ
ಅಲೆ-ಗಳೆದ್ದು
ಅಲೆ-ಗಳೆಲ್ಲ
ಅಲೆ-ಗಳೇ
ಅಲೆ-ಗಳೇ-ಳು-ವು-ದನ್ನು
ಅಲೆಗೂ
ಅಲೆಗ್ಸಾಂಡ್ರಿಯ
ಅಲೆ-ದಾಡಿ-ದರೆ
ಅಲೆ-ದಾಡುತ್ತಿದ್ದೆ
ಅಲೆ-ದಾಡುತ್ತಿದ್ದೆವು
ಅಲೆ-ದಾಡುತ್ತಿ-ರುವುದು
ಅಲೆಯ
ಅಲೆ-ಯಂತೆ
ಅಲೆ-ಯನ್ನು
ಅಲೆ-ಯ-ಬೇಕಾ
ಅಲೆ-ಯಾಗಿ
ಅಲೆ-ಯಾಗು-ವು-ದಕ್ಕೆ
ಅಲೆ-ಯಾ-ಗು-ವುದು
ಅಲೆ-ಯಾ-ಗು-ವುವು
ಅಲೆ-ಯಾ-ದಾಗ
ಅಲೆ-ಯಿಂದ
ಅಲೆಯು
ಅಲೆಯೂ
ಅಲೆ-ಯೆಂಬ
ಅಲೆ-ಯೇಳುತ್ತದೆ
ಅಲೆ-ಯೊಂದು
ಅಲೌಕಿ-ಕದ
ಅಲ್ಪ
ಅಲ್ಪ-ಕಾಲ-ದಲ್ಲಿ
ಅಲ್ಪಕ್ಷೇತ್ರ-ದಲ್ಲಿ
ಅಲ್ಪ-ಜೀವ-ನವು
ಅಲ್ಪಜ್ಞಾನ-ವುಳ್ಳ
ಅಲ್ಪ-ದ-ರಲ್ಲಿ
ಅಲ್ಪ-ಭಾಗ
ಅಲ್ಪ-ಭಾ-ವನೆ
ಅಲ್ಪ-ಭೋಗ-ಗಳ
ಅಲ್ಪ-ಮಂದಿ
ಅಲ್ಪ-ಮಂದಿಗೆ
ಅಲ್ಪ-ಮತಿ-ಗ-ಳಾದ
ಅಲ್ಪರು
ಅಲ್ಪ-ವನ್ನು
ಅಲ್ಪ-ವಸ್ತು-ಗ-ಳನ್ನು
ಅಲ್ಪ-ವಾಗಿ
ಅಲ್ಪ-ವಾಗಿ-ರ-ಬಹುದು
ಅಲ್ಪ-ವಾಗಿ-ರು-ವುದೊ
ಅಲ್ಪ-ವಾ-ಗು-ವುದು
ಅಲ್ಪ-ವಾದ
ಅಲ್ಪ-ವಾ-ದರೂ
ಅಲ್ಪ-ವಾದುದು
ಅಲ್ಪ-ವಿದ್ಯಾ
ಅಲ್ಪ-ಸುಖ
ಅಲ್ಪಾತ್ಮ
ಅಲ್ಪಾತ್ಮ-ನನ್ನೇ
ಅಲ್ಪಾತ್ಮ-ನಿ-ಗಾಗಿ
ಅಲ್ಪಾತ್ಮನು
ಅಲ್ಪಾತ್ಮವೇ
ಅಲ್ಲ
ಅಲ್ಲ-ಗಳೆ
ಅಲ್ಲ-ಗಳೆ-ದರೆ
ಅಲ್ಲ-ಗಳೆದು
ಅಲ್ಲ-ಗಳೆಯ
ಅಲ್ಲ-ಗಳೆ-ಯ-ಬಲ್ಲರು
ಅಲ್ಲ-ಗಳೆ-ಯ-ಬೇಡಿ
ಅಲ್ಲ-ಗಳೆ-ಯ-ಲಿಲ್ಲ
ಅಲ್ಲ-ಗಳೆ-ಯುತ್ತವೆ
ಅಲ್ಲ-ಗಳೆ-ಯುವ
ಅಲ್ಲ-ಗಳೆ-ಯು-ವಂತಿಲ್ಲ
ಅಲ್ಲ-ಗಳೆ-ಯು-ವರು
ಅಲ್ಲ-ಗಳೆ-ಯುವು
ಅಲ್ಲ-ಗಳೆ-ಯುವು-ದ-ರಿಂದ
ಅಲ್ಲ-ಗಳೆ-ಯುವು-ದಿಲ್ಲ
ಅಲ್ಲ-ಗಳೆ-ಯುವುದು
ಅಲ್ಲ-ಗಳೆ-ಯು-ವೆವು
ಅಲ್ಲ-ಗೆಳೆದು
ಅಲ್ಲ-ಗೆಳೆಯು-ವುದು
ಅಲ್ಲದ
ಅಲ್ಲ-ದುದು
ಅಲ್ಲದೆ
ಅಲ್ಲಲ್ಲಿ
ಅಲ್ಲ-ವೆಂದು
ಅಲ್ಲ-ವೆಂಬು-ದನ್ನು
ಅಲ್ಲ-ವೆ-ಯೋಗ-ಶಾಸ್ತ್ರಜ್ಞರು
ಅಲ್ಲಾ
ಅಲ್ಲಾ-ಡಿಸ-ಬಲ್ಲ
ಅಲ್ಲಾ-ಡಿ-ಸು-ವುದು
ಅಲ್ಲಿ
ಅಲ್ಲಿಂದ
ಅಲ್ಲಿಂದಲೂ
ಅಲ್ಲಿಂದಲೇ
ಅಲ್ಲಿಗೆ
ಅಲ್ಲಿ-ಗೆಯೇ
ಅಲ್ಲಿಗೇ
ಅಲ್ಲಿಡಿ
ಅಲ್ಲಿತ್ತು
ಅಲ್ಲಿದೆ
ಅಲ್ಲಿದೆ-ತಾನು
ಅಲ್ಲಿದ್ದರೆ
ಅಲ್ಲಿದ್ದು
ಅಲ್ಲಿನ್ನು
ಅಲ್ಲಿಯ
ಅಲ್ಲಿ-ಯ-ವರೆಗೂ
ಅಲ್ಲಿ-ಯ-ವರೆಗೆ
ಅಲ್ಲಿ-ಯ-ವರೆ-ವಿಗೂ
ಅಲ್ಲಿಯೂ
ಅಲ್ಲಿಯೆ
ಅಲ್ಲಿಯೇ
ಅಲ್ಲಿ-ರ-ಬೇಕು
ಅಲ್ಲಿ-ರುತ್ತಾರೆ
ಅಲ್ಲಿ-ರುವ
ಅಲ್ಲಿ-ರುವ-ವರನ್ನೆಲ್ಲ
ಅಲ್ಲಿ-ರು-ವು-ದಕ್ಕೆ
ಅಲ್ಲಿ-ರು-ವು-ದ-ರಿಂದ
ಅಲ್ಲಿ-ರುವುದು
ಅಲ್ಲಿವೆ
ಅಲ್ಲಿವೆ-ಯೆಂದೂ
ಅಲ್ಲೆ
ಅಲ್ಲೆಲ್ಲ
ಅಲ್ಲೇ
ಅಲ್ಲೇ-ನಿದೆ
ಅಲ್ಲೊಂದು
ಅಲ್ಲೊಬ್ಬ
ಅಳ-ತೆಗೆ
ಅಳ-ತೊಡಗಿ-ದನು
ಅಳ-ಬೇಕಾ-ಗಿಯೇ
ಅಳ-ಬೇಕಾ-ಗಿಲ್ಲ
ಅಳ-ಬೇಕು
ಅಳಲೇ-ಬೇಕು
ಅಳವಡಿಸಿ-ಕೊಂಡು
ಅಳವಡಿಸಿ-ಕೊಳ್ಳು
ಅಳಿ-ದಾಗ
ಅಳಿದು
ಅಳಿದು-ಹೋಗಿವೆ
ಅಳಿಯ-ಬೇಕಾ-ಗಿಲ್ಲ
ಅಳಿಯ-ಬೇಕು
ಅಳಿಯುತ್ತಿದೆ
ಅಳಿಯು-ವುದು
ಅಳಿಯು-ವುವು
ಅಳಿವ
ಅಳಿವು
ಅಳಿಸಿ-ಹೋ-ಗು-ವುದು
ಅಳಿಸಿ-ಹೋದ-ಮೇಲೆ
ಅಳು
ಅಳು-ಇವು-ಗಳಿಂದ
ಅಳುತ್ತ
ಅಳುತ್ತದೆ
ಅಳುತ್ತಾ
ಅಳುತ್ತಿ
ಅಳುತ್ತೇವೆ
ಅಳು-ಮೋರೆ
ಅಳು-ಮೋರೆ-ಯನ್ನು
ಅಳು-ವ-ವರು
ಅಳು-ವ-ವ-ರೆಲ್ಲ
ಅಳು-ವಿರಿ
ಅಳು-ವು-ದಿಲ್ಲ
ಅಳು-ವುದು
ಅಳುವೆ
ಅಳು-ವೆವು
ಅಳೆಯ
ಅಳೆಯ-ಬೇಕಾ-ಗಿದೆ
ಅಳೆಯ-ಬೇಕೇ
ಅಳೆ-ಯಲು
ಅಳೆಯು
ಅಳೆಯು-ವೆವು
ಅವ
ಅವ-ಕಾಶ
ಅವ-ಕಾಶ-ಕೊಡು-ವುದು
ಅವ-ಕಾಶ-ಮಾಡಿ
ಅವ-ಕಾಶ-ವನ್ನು
ಅವ-ಕಾಶ-ವನ್ನೇ
ಅವ-ಕಾಶ-ವಾ-ಗು-ವುದು
ಅವ-ಕಾಶ-ವಾ-ದಷ್ಟು
ಅವ-ಕಾಶ-ವಿದೆ
ಅವ-ಕಾಶ-ವಿ-ರುವ
ಅವ-ಕಾಶ-ವಿ-ರುವುದು
ಅವ-ಕಾಶ-ವಿಲ್ಲ
ಅವ-ಕಾಶವೇ
ಅವಕ್ಕೂ
ಅವಕ್ಕೆ
ಅವಕ್ಕೇ
ಅವಗಾ-ಹ-ನೆಗೆ
ಅವ-ಗುಣ-ಗ-ಳನ್ನು
ಅವತಾರ
ಅವತಾರ-ಗ-ಳನ್ನೂ
ಅವತಾರ-ದ-ವರೆಗೆ
ಅವತಾರ-ವೆಂದು
ಅವಧಿ
ಅವಧಿ-ಯನ್ನು
ಅವನ
ಅವನತಿ
ಅವನ-ತಿಗೆ
ಅವನ-ತಿಯ
ಅವನ-ತಿ-ಯಾಗಿ-ರ-ಬೇಕು
ಅವನ-ತಿ-ಯಿಂದ
ಅವ-ನದು
ಅವನ-ದೇನು
ಅವ-ನನ್ನು
ಅವ-ನಲ್ಲಿ
ಅವ-ನಲ್ಲಿಗೆ
ಅವ-ನಲ್ಲಿಗೇ
ಅವ-ನಲ್ಲಿದ್ದ
ಅವ-ನಲ್ಲಿ-ರುವ
ಅವ-ನಲ್ಲಿ-ರು-ವೆವು
ಅವ-ನಲ್ಲಿವೆ
ಅವ-ನಲ್ಲೇ
ಅವನವು
ಅವನಾ-ಗಲೆ
ಅವನಾತ್ಮನಲ್ಲಿದೆ
ಅವ-ನಿಂದ
ಅವ-ನಿಂದಲೇ
ಅವನಿ-ಗಾಗಿ
ಅವನಿ-ಗಿಂತ
ಅವನಿ-ಗಿಂತಲೂ
ಅವನಿ-ಗಿನ್ನು
ಅವ-ನಿಗೆ
ಅವನಿಗೇ-ತಕ್ಕೆ
ಅವನಿ-ರುವನು
ಅವನಿಲ್ಲದ
ಅವನು
ಅವನೂ
ಅವನೆ
ಅವ-ನೆಂದಿಗೂ
ಅವ-ನೆಂದು
ಅವ-ನೆಂದೂ
ಅವನೆ-ಡೆಗೆ
ಅವ-ನೆ-ದು-ರಿಗೆ
ಅವನೆಲ್ಲಿಗೆ
ಅವನೆಲ್ಲೊ
ಅವನೇ
ಅವನೇಕೆ
ಅವನೇ-ತಕ್ಕೆ
ಅವನೇನು
ಅವ-ನೊಂದಿಗೆ
ಅವನೊಂದು
ಅವ-ನೊಬ್ಬ
ಅವನೊಬ್ಬನೆ
ಅವನೊಬ್ಬನೇ
ಅವನ್ನು
ಅವನ್ನೂ
ಅವನ್ನೆ
ಅವನ್ನೆಲ್ಲ
ಅವ-ಯವ
ಅವ-ಯವ-ಗಳ
ಅವ-ಯವ-ಗ-ಳನ್ನು
ಅವರ
ಅವ-ರಂತೆ
ಅವರದು
ಅವರ-ದೊಂದು
ಅವ-ರನ್ನು
ಅವರನ್ನೆಲ್ಲ
ಅವ-ರಲ್ಲಿ
ಅವ-ರಲ್ಲಿಗೆ
ಅವ-ರಲ್ಲಿಯೂ
ಅವ-ರಲ್ಲಿ-ರುತ್ತದೆ
ಅವರಲ್ಲಿ-ರುವ
ಅವರ-ವರ
ಅವ-ರಷ್ಟು
ಅವ-ರಾಗಲೆ
ಅವ-ರಾಗಲೇ
ಅವರಾ-ಡುವ
ಅವರಿ
ಅವ-ರಿಂದ
ಅವರಿ-ಗಿಂತ
ಅವ-ರಿಗೂ
ಅವ-ರಿಗೆ
ಅವರಿ-ಗೆಲ್ಲ
ಅವರಿ-ಗೇನು
ಅವರಿತ್ತ
ಅವರಿನ್ನು
ಅವರಿಬ್ಬ-ರಿಗೂ
ಅವರಿಲ್ಲದೆ
ಅವ-ರಿಳಿಯು-ವರು
ಅವರೀಗ
ಅವರು
ಅವರೂ
ಅವರೆ
ಅವರೆಂದಿಗೂ
ಅವ-ರೆಂದೂ
ಅವರೆ-ಡನ್ನೂ
ಅವರೆಡೂ
ಅವ-ರೆದು-ರಿಗೆ
ಅವ-ರೆದೆ-ಯಲ್ಲಿ
ಅವ-ರೆಲ್ಲ
ಅವ-ರೆಲ್ಲರೂ
ಅವರೆಲ್ಲಿಯೂ
ಅವರೇ
ಅವರೇನೂ
ಅವ-ರೊಂದಿಗೆ
ಅವರೊ-ಡನೆ
ಅವರೊಬ್ಬ
ಅವರೋ-ಹಣ
ಅವರೋ-ಹಣ-ದಿಂದ
ಅವಲಂಬನ-ದಿಂದ
ಅವಲಂಬಿ-ಸದೆ
ಅವಲಂಬಿಸಿ
ಅವಲಂಬಿ-ಸಿದೆ
ಅವಲಂಬಿಸಿ-ರ-ಲಿಲ್ಲ
ಅವಲಂಬಿಸಿ-ರು-ವು-ದಿಲ್ಲ
ಅವಲಂಬಿ-ಸಿವೆ
ಅವಲಂಬಿ-ಸುತ್ತದೆ
ಅವ-ಲೋಕಿಸುತ್ತಿ-ರು-ವೆವು
ಅವಳ
ಅವ-ಳನ್ನು
ಅವಳಿಗೆ
ಅವಳು
ಅವಶೇಷ
ಅವಶ್ಯಕ
ಅವಶ್ಯಕ-ವಾದ
ಅವಶ್ಯಕವೋ
ಅವಶ್ಯವಿದ್ದಾಗ
ಅವಸಾ-ನದ
ಅವಸ್ತು-ವಾಗು-ವ-ವರೆಗೂ
ಅವಸ್ಥೆ
ಅವಸ್ಥೆ-ಗ-ಳನ್ನು
ಅವಸ್ಥೆ-ಗಳನ್ನೆಲ್ಲ
ಅವಸ್ಥೆ-ಗಳಲ್ಲಿ
ಅವಸ್ಥೆ-ಗಳಲ್ಲಿ-ರುತ್ತವೆ
ಅವಸ್ಥೆ-ಗಳಿವೆ
ಅವಸ್ಥೆ-ಗಳು
ಅವಸ್ಥೆಗೆ
ಅವಸ್ಥೆಯ
ಅವಸ್ಥೆ-ಯನ್ನು
ಅವಸ್ಥೆ-ಯಲ್ಲಿ
ಅವಸ್ಥೆ-ಯಲ್ಲಿತ್ತು
ಅವಸ್ಥೆ-ಯಲ್ಲಿದ್ದಾಗ
ಅವಸ್ಥೆ-ಯಲ್ಲಿಯೂ
ಅವಸ್ಥೆ-ಯಲ್ಲಿ-ರುವ
ಅವಸ್ಥೆ-ಯಿಂದ
ಅವಸ್ಥೆಯು
ಅವಸ್ಥೆಯೂ
ಅವಸ್ಥೆಯೆ
ಅವಾ-ವುವೂ
ಅವಿ
ಅವಿ-ಕಸ-ವಾಗಿದ್ದ
ಅವಿಕ-ಸಿತ
ಅವಿಕ-ಸಿತ-ಗೊಂಡ
ಅವಿಕ-ಸಿತ-ನಾ-ಗಿ-ರುವ-ವನೂ
ಅವಿಕ-ಸಿತ-ವಾಗಿದ್ದು
ಅವಿಕಾರ
ಅವಿಕಾರ-ವಸ್ತು-ವಿನ
ಅವಿಕಾರ-ವಾದ
ಅವಿಕಾರ-ವಾದುದು
ಅವಿಕಾರ-ವಾದುದೆ
ಅವಿ-ಕಾರಿ
ಅವಿ-ಕಾರಿ-ಯಲ್ಲ
ಅವಿ-ಕಾರಿ-ಯಾಗದೆ
ಅವಿ-ಕಾರಿ-ಯಾದ
ಅವಿ-ಕಾರಿ-ಯಾ-ದದ್ದು
ಅವಿ-ಕಾರಿ-ಯಾ-ದರೆ
ಅವಿ-ಕಾರಿ-ಯಾದು-ದನ್ನು
ಅವಿ-ಕಾರಿ-ಯಾದು-ದ-ರಿಂದ
ಅವಿ-ಕಾರಿ-ಯಾ-ದುದು
ಅವಿ-ಕಾರಿ-ಯಾದು-ದೆಂದೂ
ಅವಿಕಾ-ಸದ
ಅವಿಚ್ಛಿನ್ನ-ವಾದ
ಅವಿ-ತು-ಕೊಂಡರೂ
ಅವಿ-ತು-ಕೊಂಡಿದೆ
ಅವಿದ್ಯಾ
ಅವಿದ್ಯಾ-ವಂತರ
ಅವಿದ್ಯಾಸ್ಮಿತಾ-ರಾಗ್ವೇಷಾಭಿನಿ-ವೇಶಾಃ
ಅವಿದ್ಯೆ
ಅವಿದ್ಯೆಈ
ಅವಿದ್ಯೆಯೇ
ಅವಿನಾ-ಶತ್ವ
ಅವಿನಾಶಿ
ಅವಿನಾಶಿ-ಗಳು
ಅವಿನಾಶಿ-ಯಾಗು-ವುದು
ಅವಿನ್ನೂ
ಅವಿ-ಭಾಜ್ಯ
ಅವಿ-ಭಾಜ್ಯ-ವಾದು-ವು-ಗಳು
ಅವಿರ್ಭಾವ-ಗಳೆಲ್ಲ
ಅವಿ-ವಾಹಿತ
ಅವಿವೇ-ಕದ
ಅವಿವೇ-ಕವೇ
ಅವಿಶೇಷ
ಅವಿಶೇಷ-ವೆಂದರೆ
ಅವಿಶ್ರಾಂತ-ವಾಗಿ
ಅವು
ಅವು-ಗಳ
ಅವು-ಗಳಂತೆ
ಅವು-ಗಳನ್ನು
ಅವು-ಗಳಲ್ಲಿ
ಅವು-ಗಳಲ್ಲಿ-ರುವ
ಅವು-ಗಳಲ್ಲೆಲ್ಲಾ
ಅವು-ಗಳಾ-ವುವೂ
ಅವು-ಗಳಿಂದ
ಅವು-ಗಳಿ-ಗಾಗಿ
ಅವು-ಗಳಿ-ಗಿರುವ
ಅವು-ಗಳಿಗೆ
ಅವು-ಗಳಿಗೇ
ಅವು-ಗಳು
ಅವು-ಗಳೂ
ಅವು-ಗಳೆಲ್ಲ
ಅವು-ಗಳೆಲ್ಲ-ದರ
ಅವು-ಗಳೆಲ್ಲ-ರನ್ನೂ
ಅವು-ಗಳೆಲ್ಲ-ವನ್ನು
ಅವು-ಗಳೆಲ್ಲ-ವನ್ನೂ
ಅವು-ಗಳೆಲ್ಲವೂ
ಅವು-ಗಳೇನು
ಅವು-ಗಳೊಂದಿಗೆ
ಅವೂ
ಅವೆ-ರ-ಡನ್ನೂ
ಅವೆ-ರಡೂ
ಅವೆಲ್ಲ
ಅವೆಲ್ಲ-ವನ್ನೂ
ಅವೆಲ್ಲವೂ
ಅವೆಲ್ಲಾ
ಅವೆಷ್ಟು
ಅವೇ
ಅವೇಕೆ
ಅವೈಜ್ಞಾನಿಕ-ವಾದ
ಅವೈದಿಕ
ಅವ್ಯಕ್ತ
ಅವ್ಯಕ್ತ-ಇವು-ಗಳೆಲ್ಲವೂ
ಅವ್ಯಕ್ತಕ್ಕೆ
ಅವ್ಯಕ್ತ-ಗಳ
ಅವ್ಯಕ್ತ-ದಿಂದ
ಅವ್ಯಕ್ತ-ದೆ-ಡೆಗೆ
ಅವ್ಯಕ್ತ-ನಾದ
ಅವ್ಯಕ್ತ-ವನ್ನು
ಅವ್ಯಕ್ತ-ವಾಗಿ
ಅವ್ಯಕ್ತ-ವಾಗಿದ್ದು
ಅವ್ಯಕ್ತ-ವಾಗಿ-ರ-ಬಹುದು
ಅವ್ಯಕ್ತ-ವಾಗುತ್ತ
ಅವ್ಯಕ್ತ-ವಾಗು-ವುದು
ಅವ್ಯಕ್ತ-ವಾದ
ಅವ್ಯಕ್ತ-ವಾ-ದದ್ದು
ಅವ್ಯಕ್ತವು
ಅವ್ಯಕ್ತವೂ
ಅವ್ಯಕ್ತ-ವೆಂಬ
ಅವ್ಯಕ್ತ-ವೆನ್ನು-ವರು
ಅವ್ಯಕ್ತ-ವೆನ್ನು-ವುದು-ಸೃಷ್ಟಿಗೆ
ಅವ್ಯಕ್ತವೇ
ಅವ್ಯಕ್ತಸ್ಥಿತಿಯು
ಅವ್ಯಕ್ತಾ-ವಸ್ಥೆಗೆ
ಅವ್ಯಕ್ತಾ-ವಸ್ಥೆಯ
ಅವ್ಯಕ್ತಾ-ವಸ್ಥೆ-ಯಲ್ಲಿ
ಅವ್ಯಕ್ತಾ-ವಸ್ಥೆಯೊ
ಅವ್ಯಯ
ಅವ್ಯವಸ್ಥಿತ-ವಾಗಿ-ರುವುದು
ಅವ್ಯ-ವಸ್ಥೆ-ಯಾದರೂ
ಅಶಾಂತ-ಗೊಳಿ-ಸು-ವು-ದಿಲ್ಲವೊ
ಅಶಾಂತರೊ
ಅಶಾಂತಿ
ಅಶಾಂತಿ-ಗಳಿಲ್ಲದೆ
ಅಶಾಂತಿ-ಯನ್ನು
ಅಶಾಶ್ವತ-ತೆ-ಯನ್ನು
ಅಶುಚಿಯೂ
ಅಶುದ್ಧ-ಗಳ
ಅಶುದ್ಧತೆ-ಗಳೆಲ್ಲ
ಅಶುದ್ಧ-ನಾಗಿ-ರುವುದು
ಅಶುದ್ಧ-ರಾದ
ಅಶುದ್ಧರು
ಅಶುದ್ಧ-ರೆಂದು
ಅಶುದ್ಧ-ವಾಗು-ವು-ದೆಂದು
ಅಶುದ್ಧ-ವಾದು-ದನ್ನು
ಅಷ್ಟ-ದಳ
ಅಷ್ಟ-ದಳ-ಗಳು
ಅಷ್ಟನ್ನು
ಅಷ್ಟ-ರಲ್ಲಿ
ಅಷ್ಟ-ಸಿದ್ಧಿ
ಅಷ್ಟ-ಸಿದ್ಧಿ-ಗಳನ್ನು
ಅಷ್ಟ-ಸಿದ್ಧಿ-ಗಳು
ಅಷ್ಟಾಂಗ-ಗಳು
ಅಷ್ಟು
ಅಷ್ಟೂ
ಅಷ್ಟೆ
ಅಷ್ಟೆ-ದೇವತೆ-ಗಳ
ಅಷ್ಟೇ
ಅಷ್ಟೇನು
ಅಷ್ಟೇನೂ
ಅಷ್ಟೊಂದು
ಅಸಂಖ್ಯಾತ
ಅಸಂಜಸ-ವಾ-ಗಿ-ರು-ವಂತೆ
ಅಸಂಜಸ-ವಾದ
ಅಸಂಪೂರ್ಣ
ಅಸಂಪೂರ್ಣ-ವಾದ
ಅಸಂಪ್ರಜ್ಞಾತ
ಅಸಂಬದ್ಧ
ಅಸಂಬದ್ಧತೆ
ಅಸಂಬದ್ಧ-ತೆಯ
ಅಸಂಬದ್ಧ-ವಾಗಿತ್ತು
ಅಸಂಬದ್ಧ-ವಾದುದು
ಅಸಂಭವ
ಅಸಂಸ್ಕೃತ
ಅಸಡ್ಡೆ
ಅಸತ್ಯ
ಅಸತ್ಯಜ್ಞಾನ
ಅಸತ್ಯ-ದಲ್ಲಿ
ಅಸತ್ಯ-ದಿಂದ
ಅಸತ್ಯ-ವನ್ನು
ಅಸತ್ಯ-ವಾಗ-ಬೇಕು
ಅಸತ್ಯ-ವಾದ
ಅಸದ-ಳ-ವಾದ
ಅಸದ-ಳ-ವಾ-ಯಿತು
ಅಸಮಂಜಸ
ಅಸಮಂಜಸ-ತೆ-ಯನ್ನು
ಅಸಮಂಜಸವೇ
ಅಸ-ಮರ್ಥ-ರಾಗುತ್ತೇವೆ
ಅಸ-ಮರ್ಥ-ವಾ-ದರೆ
ಅಸಮಾ-ಧಾನ
ಅಸಹಜ
ಅಸಹನೀಯ
ಅಸಹಾಯ-ಕರೂ
ಅಸಹ್ಯ
ಅಸಹ್ಯ-ಕರ
ಅಸಹ್ಯ-ಕರ-ವಾಗಿ
ಅಸಹ್ಯ-ವಾದ
ಅಸಹ್ಯ-ವೆನಿ-ಸುತ್ತದೆ
ಅಸಾ-ಧಾರಣ
ಅಸಾ-ಧಾರಣ-ವಾದ
ಅಸಾಧು-ವನ್ನೂ
ಅಸಾಧು-ವಿನ
ಅಸಾಧ್ಯ
ಅಸಾಧ್ಯ-ವಾಗಿರು-ವುದ-ರಿಂದ
ಅಸಾಧ್ಯ-ವಾಗು-ವಷ್ಟು
ಅಸಾಧ್ಯ-ವಾದ
ಅಸಾಧ್ಯ-ವಾ-ದರೆ
ಅಸಾಧ್ಯ-ವಾದುದು
ಅಸಾಧ್ಯ-ವಾ-ಯಿತು
ಅಸಾಧ್ಯ-ವಿದೆ
ಅಸಾಧ್ಯ-ವೆಂದು
ಅಸಾ-ಮಾನ್ಯ
ಅಸೀಮ
ಅಸುಖಿ-ಯನ್ನಾಗಿ
ಅಸುಖಿ-ಯಾಗಿರು-ವೆವು
ಅಸುರರ
ಅಸುರರು
ಅಸೂ-ಯಾದಿ-ಹೀನ-ಭಾವ-ತರಂಗ-ಗಳಿಂದ
ಅಸೂಯಾ-ಪರರು
ಅಸೂಯೆ
ಅಸೂಯೆ-ಗಳನ್ನು
ಅಸೂಯೆ-ಪ-ಡು-ವುದು
ಅಸೂಯೆ-ಯನ್ನೂ
ಅಸೂಯೆಯೂ
ಅಸೂಯೆ-ಯೆಲ್ಲ
ಅಸ್ತಿ
ಅಸ್ತಿತ್ವ
ಅಸ್ತಿತ್ವ-ಇವು
ಅಸ್ತಿತ್ವಕ್ಕೂ
ಅಸ್ತಿತ್ವಕ್ಕೆ
ಅಸ್ತಿತ್ವದ
ಅಸ್ತಿತ್ವ-ದಂತೆಯೊ
ಅಸ್ತಿತ್ವ-ದಲ್ಲಿ
ಅಸ್ತಿತ್ವ-ದಲ್ಲಿ-ರುವ-ನೆಂದಲ್ಲ
ಅಸ್ತಿತ್ವ-ದಿಂದ
ಅಸ್ತಿತ್ವ-ದೊ-ಡನೆ
ಅಸ್ತಿತ್ವ-ವನ್ನು
ಅಸ್ತಿತ್ವ-ವನ್ನೇ
ಅಸ್ತಿತ್ವ-ವಾಗಿದ್ದಾನೆ
ಅಸ್ತಿತ್ವ-ವಾ-ಗು-ವುದು
ಅಸ್ತಿತ್ವ-ವಿದೆ
ಅಸ್ತಿತ್ವ-ವಿದೆಯೊ
ಅಸ್ತಿತ್ವ-ವಿಲ್ಲ
ಅಸ್ತಿತ್ವ-ವಿಲ್ಲ-ವೆಂದೂ
ಅಸ್ತಿತ್ವವು
ಅಸ್ತಿತ್ವವೂ
ಅಸ್ತಿತ್ವ-ವೆಂದಾಗಲಿ
ಅಸ್ತಿತ್ವ-ವೆಲ್ಲ
ಅಸ್ತಿತ್ವ-ವೆಲ್ಲಾ
ಅಸ್ತಿತ್ವವೇ
ಅಸ್ತಿತ್ವ-ವೊಂದನ್ನು
ಅಸ್ತಿತ್ವ-ವೊಂದು
ಅಸ್ತಿತ್ವ-ಸಾ-ಗರದ
ಅಸ್ತಿ-ನಾಸ್ತಿ-ಗಳ
ಅಸ್ತಿ-ನಾಸ್ತಿ-ಗಳಿಲ್ಲದೆ
ಅಸ್ತಿ-ವಾದಿ-ಗಳೂ
ಅಸ್ತೇಯ
ಅಸ್ತೇಯ-ದಲ್ಲಿ
ಅಸ್ತೇಯಪ್ರತಿಷ್ಠಾಯಾಂ
ಅಸ್ತೇಯಾ
ಅಸ್ತ್ರ-ಗಳಿವೆ
ಅಸ್ತ್ರ-ಗಳೆ
ಅಸ್ಥಿರ
ಅಸ್ಥಿರ-ಗೊಳಿ-ಸು-ವುದು
ಅಸ್ಥಿರ-ವಾಗುತ್ತವೆ
ಅಸ್ಪಷ್ಟ
ಅಸ್ಪಷ್ಟತೆ
ಅಸ್ಪಷ್ಟ-ವಾಗಿ
ಅಸ್ಪಷ್ಟ-ವಾಗಿದೆ
ಅಸ್ಮಿತ
ಅಸ್ವ-ತಂತ್ರ-ರೆಂಬು-ದನ್ನು
ಅಸ್ವಸ್ಥ-ಗೊಳಿ-ಸು-ವಿರಿ
ಅಸ್ವಸ್ಥತೆ-ಯನ್ನು
ಅಸ್ವಸ್ಥ-ವಾಗಿದ್ದರೆ
ಅಸ್ವಸ್ಥ-ವಾ-ಗು-ವುದು
ಅಸ್ವಾ-ಭಾವಿ-ಕ-ವಾದ
ಅಹಂ
ಅಹಂಕಾ
ಅಹಂಕಾರ
ಅಹಂಕಾ-ರ-ಇವು-ಗ-ಳನ್ನು
ಅಹಂಕಾ-ರ-ಇವು-ಗಳಿಂದ
ಅಹಂಕಾ-ರ-ಜನಿತ
ಅಹಂಕಾ-ರದ
ಅಹಂಕಾ-ರ-ದಂತೆ
ಅಹಂಕಾ-ರ-ದಲ್ಲಿ
ಅಹಂಕಾ-ರ-ದಿಂದ
ಅಹಂಕಾ-ರ-ಪಡ-ಬೇಡಿ
ಅಹಂಕಾ-ರ-ಪ-ಡಲು
ಅಹಂಕಾ-ರ-ವನ್ನು
ಅಹಂಕಾ-ರ-ವಿಲ್ಲದೆ
ಅಹಂಕಾ-ರವು
ಅಹಂಕಾ-ರವೆ
ಅಹಂಕಾ-ರವೇ
ಅಹಂಕಾ-ರಾದಿ-ಗಳ
ಅಹಂಭಾವ-ದಿಂದ
ಅಹಂಭಾವ-ವಿ-ರುತ್ತದೆ
ಅಹಂಭಾವ-ವಿರು-ವು-ದಿಲ್ಲ
ಅಹಿಂಸಾ
ಅಹಿಂಸಾಪ್ರತಿಷ್ಠಾಯಾಂ
ಅಹಿಂಸಾ-ಸತ್ಯಾಸ್ತೇಯಬ್ರಹ್ಮ-ಚರ್ಯಾ-ಪರಿಗ್ರಹಾ
ಅಹಿಂಸೆ
ಅಹಿಂಸೆ-ಗಿಂತ
ಅಹಿಂಸೆ-ಯಲ್ಲಿ
ಅಹಿಂಸೋಪಾಸ-ಕರು
ಅಹಿತ-ಕರ-ವೆಂದೂ
ಅಹ್ನಿಕ-ವನ್ನು
ಅಹ್ರಿ-ಮಾನ್
ಆ
ಆಂಗ್ಲ
ಆಂಗ್ಲ-ಭಾಷಾನು-ವಾದ-ವನ್ನು
ಆಂಗ್ಲೇಯ
ಆಂಗ್ಲೇಯನು
ಆಂತರಿಕ
ಆಂತರಿಕಪ್ರ-ಕೃತಿಯ
ಆಂತರಿಕ-ವಾಗಿರ
ಆಂತರಿಕ-ವಾದ-ವು-ಗಳು
ಆಂತರಿಕ-ವಾ-ದುದು
ಆಂತರಿ-ಕವು
ಆಂತರಿಕ-ಶಕ್ತಿಯ
ಆಂತರ್ಯ
ಆಂತರ್ಯ-ದಲ್ಲಿ
ಆಂತರ್ಯ-ದಲ್ಲಿ-ರುವ
ಆಂತರ್ಯ-ದಲ್ಲೂ
ಆಂದೋಳನ
ಆಂದೋಳನ-ವನ್ನು
ಆಂದೋಳನಾಕ್ಷೇತ್ರ-ದಲ್ಲಿ
ಆಂಶಿಕ-ದೃಷ್ಟಿ
ಆಂಶಿಕ-ವಾಗಿ
ಆಂಶಿಕ-ವಾದ
ಆಕರ್ಷಣ
ಆಕರ್ಷಣೀಯ-ವಾದ
ಆಕರ್ಷಣೆ
ಆಕರ್ಷ-ಣೆಗೆ
ಆಕರ್ಷ-ಣೆಯ
ಆಕರ್ಷಿತ-ರಾ-ಗುತ್ತಾರೆ
ಆಕರ್ಷಿತ-ರಾಗು-ವೆವು
ಆಕರ್ಷಿತ-ರಾ-ದಂತೆ
ಆಕರ್ಷಿತ-ವಾಗಿ
ಆಕರ್ಷಿತ-ವಾದು-ದನ್ನು
ಆಕರ್ಷಿ-ಸಲ್ಪ-ಡು-ವುದು
ಆಕರ್ಷಿಸಿ
ಆಕರ್ಷಿ-ಸುತ್ತಿ-ರುವನು
ಆಕರ್ಷಿ-ಸುತ್ತಿ-ರುವುದು
ಆಕರ್ಷಿ-ಸುವ
ಆಕರ್ಷಿ-ಸು-ವು-ದಿಲ್ಲ
ಆಕರ್ಷಿ-ಸು-ವುವು
ಆಕರ್ಷಿ-ಸು-ವೆವು
ಆಕ-ಳನ್ನು
ಆಕಸ್ಮಿಕ
ಆಕಸ್ಮಿಕ-ವಾಗಿ
ಆಕಾಂಕ್ಷೆ
ಆಕಾಂಕ್ಷೆ-ಗಳನ್ನೆಲ್ಲ
ಆಕಾಂಕ್ಷೆ-ಗಳಲ್ಲಿ
ಆಕಾಂಕ್ಷೆ-ಗಳಿ-ರು-ವುದು
ಆಕಾಂಕ್ಷೆ-ಗಳು
ಆಕಾಂಕ್ಷೆ-ಗಳು-ಇ-ವೆಲ್ಲಾ
ಆಕಾಂಕ್ಷೆಯ
ಆಕಾರ
ಆಕಾರಕ್ಕೂ
ಆಕಾ-ರಕ್ಕೆ
ಆಕಾರ-ಗಳ
ಆಕಾರ-ಗಳನ್ನಿತ್ತು
ಆಕಾರ-ಗ-ಳನ್ನು
ಆಕಾರ-ಗಳಲ್ಲಿ
ಆಕಾರ-ಗಳಾಗಿ
ಆಕಾರ-ಗಳು
ಆಕಾರ-ಗಳೆ-ರಡೂ
ಆಕಾರ-ಗಳೆಲ್ಲ
ಆಕಾರ-ಗಳೆಲ್ಲಾ
ಆಕಾರದ
ಆಕಾರ-ದಂತೆ
ಆಕಾರ-ದಲ್ಲಿ
ಆಕಾರ-ದಲ್ಲಿದೆ
ಆಕಾರ-ದಿಂದ
ಆಕಾರ-ದಿಂದಲೂ
ಆಕಾರ-ವನ್ನು
ಆಕಾರ-ವನ್ನೇ
ಆಕಾರ-ವಾಗಿದೆ
ಆಕಾರ-ವಿದೆ
ಆಕಾರ-ವಿ-ರುವ
ಆಕಾರ-ವಿ-ರು-ವು-ದಕ್ಕೆಲ್ಲ
ಆಕಾರ-ವಿಲ್ಲ
ಆಕಾರ-ವಿಲ್ಲ-ದುದು
ಆಕಾರವು
ಆಕಾರವೂ
ಆಕಾರ-ಹೀನ-ವಾದುದು
ಆಕಾಶ
ಆಕಾಶ-ಕುಸುಮ-ವೆನ್ನುತ್ತಾರೆ
ಆಕಾಶಕ್ಕಿಂತ
ಆಕಾಶಕ್ಕೆ
ಆಕಾಶ-ತತ್ತ್ವಕ್ಕೆ
ಆಕಾಶ-ತತ್ತ್ವದ
ಆಕಾಶ-ತತ್ತ್ವವು
ಆಕಾಶದ
ಆಕಾಶ-ದಂತೆ
ಆಕಾಶ-ದಲ್ಲಿ
ಆಕಾಶ-ದಷ್ಟು
ಆಕಾಶ-ದಿಂದ
ಆಕಾಶ-ದೊಂದಿಗೆ
ಆಕಾಶ-ವನ್ನು
ಆಕಾಶ-ವಾ-ಗು-ವುದು
ಆಕಾಶ-ವಾ-ಗು-ವುವು
ಆಕಾಶ-ವಿದೆ
ಆಕಾಶವು
ಆಕಾಶವೂ
ಆಕಾಶವೆ
ಆಕಾಶ-ವೆಂದು
ಆಕಾಶ-ವೆನ್ನು-ವರು
ಆಕಾಶವೇ
ಆಕಾಶ-ವೊಂದೇ
ಆಕಾಶ-ಸಿದ್ಧಾಂತಕ್ಕಿಂತ
ಆಕೃತಿ-ಯನ್ನು
ಆಕೆಯ
ಆಕೆಯು
ಆಕ್ರಂದನ
ಆಕ್ರಮಿ-ಸಲು
ಆಕ್ರಮಿಸಿ
ಆಕ್ರಮಿಸಿದೆ
ಆಕ್ರಮಿಸಿ-ರುವನು
ಆಕ್ರಮಿಸು-ತ್ತದೆ
ಆಕ್ರಮಿಸುವ
ಆಕ್ರಮಿಸು-ವುದು
ಆಕ್ಷೇಪಣೀಯ
ಆಕ್ಷೇಪಣೆ
ಆಕ್ಷೇಪಣೆ-ಗಳಿಗೆ
ಆಕ್ಷೇಪಣೆ-ಗಳು
ಆಕ್ಷೇಪ-ಣೆಗೆ
ಆಕ್ಷೇಪಣೆ-ಯನ್ನು
ಆಕ್ಷೇಪ-ಣೆಯೂ
ಆಕ್ಷೇಪ-ಣೆಯೆ
ಆಕ್ಷೇಪ-ಣೆಯೇ
ಆಕ್ಷೇ-ಪಿಸಿ-ರು-ವರು
ಆಕ್ಷೇಪಿ-ಸು-ವು-ದಿಲ್ಲ
ಆಗ
ಆಗ-ಕೂಡದು
ಆಗದ
ಆಗ-ದಂತೆ
ಆಗ-ದಷ್ಟು
ಆಗದೆ
ಆಗದೇ
ಆಗ-ಬಲ್ಲರು
ಆಗ-ಬೇಕಾ-ಗಿದೆ
ಆಗ-ಬೇಕು
ಆಗ-ಬೇಡಿ
ಆಗ-ಮ-ನಕ್ಕೆ
ಆಗ-ಮಾತ್ರ
ಆಗ-ರಕ್ಕೆ
ಆಗ-ರ-ದಿಂದ
ಆಗ-ಲಾ-ದರೂ
ಆಗ-ಲಾರದು
ಆಗ-ಲಾರಿರಿ
ಆಗಲಿ
ಆಗಲಿ-ಇದರ
ಆಗ-ಲಿಲ್ಲ
ಆಗಲೀ
ಆಗಲು
ಆಗಲೂ
ಆಗಲೆ
ಆಗಲೇ
ಆಗಲೇ-ಬೇಕು
ಆಗಾರ
ಆಗಿ
ಆಗಿಂದಾಗ್ಗೆ
ಆಗಿತ್ತು
ಆಗಿದೆ
ಆಗಿದೆಯೊ
ಆಗಿದೆಯೋ
ಆಗಿದ್ದರೂ
ಆಗಿದ್ದರೆ
ಆಗಿದ್ದ-ರೆಂಬು-ದನ್ನು
ಆಗಿದ್ದೆ
ಆಗಿನ
ಆಗಿ-ರದೆ
ಆಗಿ-ರ-ಬೇಕಾ-ಗಿತ್ತು
ಆಗಿ-ರ-ಬೇಕು
ಆಗಿ-ರಲಿ
ಆಗಿ-ರ-ಲಿಲ್ಲ
ಆಗಿರಿ
ಆಗಿರು
ಆಗಿ-ರುತ್ತದೆ
ಆಗಿ-ರುವ
ಆಗಿ-ರು-ವನು
ಆಗಿ-ರು-ವನೋ
ಆಗಿ-ರು-ವರು
ಆಗಿ-ರು-ವಿರಿ
ಆಗಿ-ರು-ವು-ದನ್ನು
ಆಗಿ-ರು-ವು-ದನ್ನೆಲ್ಲ
ಆಗಿ-ರು-ವು-ದ-ರಿಂದ
ಆಗಿ-ರು-ವುದು
ಆಗಿ-ರು-ವುದೊ
ಆಗಿ-ರುವೆ
ಆಗಿ-ರು-ವೆನು
ಆಗಿಲ್ಲ
ಆಗಿಲ್ಲದ
ಆಗಿಲ್ಲವೊ
ಆಗಿಲ್ಲವೋ
ಆಗಿವೆ
ಆಗಿ-ವೆಯೊ
ಆಗಿ-ಹೋಗಿರು-ವುವು
ಆಗಿ-ಹೋಗು-ತ್ತದೆ
ಆಗಿ-ಹೋ-ದುದು
ಆಗು
ಆಗುತ್ತ
ಆಗುತ್ತದೆ
ಆಗುತ್ತವೆ
ಆಗುತ್ತಾನೆ
ಆಗುತ್ತಿ
ಆಗುತ್ತಿತ್ತು
ಆಗುತ್ತಿದೆ
ಆಗುತ್ತಿರ-ಲಿಲ್ಲ
ಆಗುತ್ತಿರ-ಲಿಲ್ಲ-ವೆಂದು
ಆಗುತ್ತಿರು
ಆಗುತ್ತಿ-ರುತ್ತದೆ
ಆಗುತ್ತಿ-ರುವ
ಆಗುತ್ತಿರು-ವು-ದನ್ನು
ಆಗುತ್ತಿರು-ವುದು
ಆಗುತ್ತಿರು-ವುದೇ
ಆಗುತ್ತಿವೆ
ಆಗುತ್ತೇವೆ
ಆಗುವ
ಆಗು-ವಂತಿಲ್ಲ
ಆಗು-ವನು
ಆಗು-ವರು
ಆಗು-ವ-ವರೆಗೂ
ಆಗು-ವ-ವರೆಗೆ
ಆಗು-ವು-ದನ್ನು
ಆಗು-ವು-ದಿಲ್ಲ
ಆಗು-ವು-ದಿಲ್ಲ-ಅದು
ಆಗು-ವು-ದಿಲ್ಲವೆ
ಆಗು-ವು-ದಿಲ್ಲ-ವೆಂದರೆ
ಆಗು-ವು-ದಿಲ್ಲ-ವೆಂದು
ಆಗು-ವು-ದಿಲ್ಲ-ವೆಂದೂ
ಆಗು-ವು-ದಿಲ್ಲ-ವೆಂಬು-ದೇನೋ
ಆಗು-ವುದು
ಆಗು-ವು-ದೆಂದು
ಆಗು-ವುದೇ
ಆಗು-ವುದೇನು
ಆಗು-ವುದೋ
ಆಗು-ವುವು
ಆಗು-ವೆವು
ಆಘಾತ-ವನ್ನು
ಆಘಾತ-ವಾ-ಗು-ವುದು
ಆಚರಣೆ
ಆಚರಣೆ-ಗಳೇ
ಆಚರ-ಣೆಗೆ
ಆಚರಣೆ-ಯನ್ನು
ಆಚರಣೆ-ಯಲ್ಲಿ-ರುವ
ಆಚರಿ-ಸಲು
ಆಚರಿ-ಸುತ್ತಿದ್ದರು
ಆಚರಿ-ಸುವರೋ
ಆಚಾರ
ಆಚಾರ-ಗ-ಳನ್ನು
ಆಚಾರ-ಗಳಲ್ಲಿ
ಆಚಾರ-ಗಳುಳ್ಳ
ಆಚಾರದ
ಆಚಾರ-ದಲ್ಲಿಯೂ
ಆಚಾರ-ಪರ-ನಿಗೆ
ಆಚಾರ-ವಂತ
ಆಚಾರ-ವನ್ನಾ-ಗಲೀ
ಆಚಾರ-ವನ್ನು
ಆಚಾರವೂ
ಆಚಾರ್ಯ-ಪುರುಷರು
ಆಚೆ
ಆಚೆಗೆ
ಆಚೆ-ಗೆಸೆ-ಯಿರಿ
ಆಚೆ-ಗೆಸೆಯು-ವರು
ಆಚೆ-ಗೊಯ್ಯು-ವುದೊ
ಆಚೆಯ
ಆಚೆ-ಯಿಂದ
ಆಚೆಯೂ
ಆಜನ್ಮ
ಆಜನ್ಮ-ಸಿದ್ಧ
ಆಜನ್ಮ-ಸಿದ್ಧ-ಹಕ್ಕು
ಆಜ್ಞಾ
ಆಜ್ಞಾ-ಪಿಸ-ಬಹುದು
ಆಜ್ಞಾಪಿಸಿ-ದನು
ಆಜ್ಞೆ
ಆಜ್ಞೆ-ಯಂತೆ
ಆಜ್ಞೆ-ಯತಾ
ಆಜ್ಞೆ-ಯತಾ-ವಾದಿ
ಆಜ್ಞೆ-ಯಿಂದ
ಆಜ್ಞೇ
ಆಜ್ಞೇಯ
ಆಜ್ಞೇ-ಯತಾ
ಆಜ್ಞೇಯ-ತಾ-ವಾದ
ಆಜ್ಞೇಯ-ತಾ-ವಾದಿ-ಗಳಾ-ಗ-ಲಾರೆವು
ಆಜ್ಞೇಯ-ತಾ-ವಾದಿ-ಗಳಿಗೂ
ಆಜ್ಞೇಯ-ತಾ-ವಾದಿ-ಗಳು
ಆಜ್ಞೇಯ-ತಾ-ವಾದಿ-ಗಳೊ
ಆಜ್ಞೇಯ-ತಾ-ವಾದಿಯ
ಆಜ್ಞೇಯ-ತಾ-ವಾದಿ-ಯಾಗಿದ್ದ
ಆಜ್ಞೇಯ-ತಾ-ವಾದಿಯು
ಆಜ್ಞೇಯ-ವಾದ-ವಲ್ಲ
ಆಟ
ಆಟಕ್ಕಾಗಿ
ಆಟದ
ಆಟ-ವನ್ನು
ಆಟವಾಡುತ್ತಿ-ರುವ
ಆಟವಾಡು-ವುವು
ಆಟ-ವಾ-ದರೆ
ಆಡಂಬರ
ಆಡಂಬ-ರದ
ಆಡ-ಬಯ-ಸುತ್ತೇವೆ
ಆಡಮ್ನ
ಆಡಳಿ-ತಕ್ಕೂ
ಆಡಿ
ಆಡಿದ
ಆಡಿ-ದರೆ
ಆಡಿಲ್ಲ
ಆಡಿ-ಸಿ-ಕೊಳ್ಳು-ವು-ದಕ್ಕೆ
ಆಡುತ್ತಾರೆ
ಆಡುತ್ತಿ
ಆಡುತ್ತಿ-ರುತ್ತವೆ
ಆಡುತ್ತೇವೆ
ಆಡುವ
ಆಡು-ವರು
ಆಡುವಾಗ
ಆಣ-ತಿಯ
ಆಣೆ
ಆಣೆ-ಗಳ
ಆತ
ಆತಂಕ
ಆತಂಕ-ಗ-ಳನ್ನು
ಆತಂಕ-ಗಳಾ-ವುವು-ವೆಂದರೆ-ಅ-ವಿದ್ಯೆ
ಆತಂಕ-ಗಳಿಂದ
ಆತಂಕ-ಗಳು
ಆತಂಕ-ಗಳೆಲ್ಲ-ದ-ರಿಂದ
ಆತಂಕ-ವನ್ನು
ಆತಂಕ-ವಾಗಿದೆ
ಆತಂಕ-ವಿಲ್ಲ
ಆತಂಕವೂ
ಆತಂಕವೆ
ಆತಂಕವೇ
ಆತನ
ಆತ-ನನ್ನು
ಆತ-ನಾಡುತ್ತಿ-ರುವೆ
ಆತ-ನಿಗೆ
ಆತನು
ಆತನೆ
ಆತನೇ
ಆತ-ನೊಬ್ಬನೇ
ಆತುರ
ಆತುರ-ಪಡ-ಬೇಕಾ-ಗಿಲ್ಲ
ಆತ್ಮ
ಆತ್ಮಕ್ಕೆ
ಆತ್ಮ-ಗಳ
ಆತ್ಮ-ಗ-ಳನ್ನು
ಆತ್ಮ-ಗಳಿವೆ
ಆತ್ಮ-ಗಳು
ಆತ್ಮ-ಗಳೆಂಬ
ಆತ್ಮ-ಗಳೆ-ರ-ಡನ್ನೂ
ಆತ್ಮ-ಗಳೆಲ್ಲ
ಆತ್ಮ-ಗಾ-ನದ
ಆತ್ಮ-ಚಿಂತನೆ
ಆತ್ಮಜ್ಞಾನ
ಆತ್ಮಜ್ಞಾನ-ದಿಂದ
ಆತ್ಮಜ್ಞಾನ-ವನ್ನು
ಆತ್ಮ-ತತ್ತ್ವ-ವನ್ನು
ಆತ್ಮದ
ಆತ್ಮ-ದಲ್ಲಿ
ಆತ್ಮನ
ಆತ್ಮ-ನಂತೆ
ಆತ್ಮ-ನಂತೆಯೇ
ಆತ್ಮ-ನನ್ನು
ಆತ್ಮ-ನನ್ನೇ
ಆತ್ಮ-ನಲ್ಲ
ಆತ್ಮ-ನಲ್ಲಿ
ಆತ್ಮ-ನಲ್ಲಿದೆ
ಆತ್ಮ-ನಲ್ಲಿಯೂ
ಆತ್ಮ-ನಲ್ಲಿ-ರುವ
ಆತ್ಮ-ನಲ್ಲಿ-ರುವಾಗ
ಆತ್ಮ-ನಲ್ಲಿ-ರುವುದು
ಆತ್ಮ-ನಾಗ-ಲಾರ-ದೆಂದು
ಆತ್ಮ-ನಾಗಿ-ರ-ಬೇಕು
ಆತ್ಮ-ನಾಗು-ವನು
ಆತ್ಮ-ನಾದ
ಆತ್ಮ-ನಿಂದ
ಆತ್ಮ-ನಿಂದೆ
ಆತ್ಮ-ನಿ-ಗಿಂತ
ಆತ್ಮ-ನಿಗೆ
ಆತ್ಮ-ನಿ-ಗೇನೂ
ಆತ್ಮ-ನಿರ-ಬೇಕೆಂದು
ಆತ್ಮ-ನಿ-ರುವನು
ಆತ್ಮ-ನಿ-ರುವನೆ
ಆತ್ಮ-ನಿ-ರುವನೊ
ಆತ್ಮನು
ಆತ್ಮನೂ
ಆತ್ಮನೆ
ಆತ್ಮ-ನೆಂದರೆ
ಆತ್ಮ-ನೆಂದಿಗೂ
ಆತ್ಮ-ನೆಂದು
ಆತ್ಮ-ನೆ-ಡೆಗೆ
ಆತ್ಮ-ನೆದು-ರಿಗೆ
ಆತ್ಮನೇ
ಆತ್ಮ-ನೊಂದಿಗೆ
ಆತ್ಮ-ನೊಂದೇ
ಆತ್ಮಪ್ರೀತಿ
ಆತ್ಮ-ಭಾವ-ಭಾವ-ನಾ-ನಿ-ವೃತ್ತಿಃ
ಆತ್ಮ-ಭಿ-ಮಾನದ
ಆತ್ಮ-ಮಯ-ವಾ-ಗಿ-ರುವುದೋ
ಆತ್ಮರು
ಆತ್ಮ-ವನ್ನು
ಆತ್ಮ-ವನ್ನೇ
ಆತ್ಮ-ವಲ್ಲ
ಆತ್ಮ-ವಾಗ-ಲಾರದು
ಆತ್ಮ-ವಾ-ಗು-ವುದು
ಆತ್ಮ-ವಾದ
ಆತ್ಮ-ವಿಕಾಸಕ್ಕೆ
ಆತ್ಮ-ವಿ-ಚಾರಜ್ಞಾನ
ಆತ್ಮ-ವಿದೆ
ಆತ್ಮ-ವಿದೆಯೆ
ಆತ್ಮ-ವಿದ್ದರೆ
ಆತ್ಮ-ವಿಲ್ಲ
ಆತ್ಮ-ವಿಶ್ವಾಸ
ಆತ್ಮ-ವಿಶ್ವಾಸ-ಹೀನರೋ
ಆತ್ಮವು
ಆತ್ಮವೂ
ಆತ್ಮವೆ
ಆತ್ಮ-ವೆಂದರೆ
ಆತ್ಮ-ವೆಂದ-ರೇನು
ಆತ್ಮ-ವೆಂದು
ಆತ್ಮ-ವೆಂದೂ
ಆತ್ಮ-ವೆಂಬ
ಆತ್ಮ-ವೆನ್ನುವ
ಆತ್ಮ-ವೆನ್ನುವುದು
ಆತ್ಮವೇ
ಆತ್ಮ-ವೊಂದಿದೆ
ಆತ್ಮ-ವೊಂದೆ
ಆತ್ಮ-ಶಕ್ತಿ-ಯಿಂದ
ಆತ್ಮಶ್ರದ್ಧೆ
ಆತ್ಮಶ್ರದ್ಧೆಯ
ಆತ್ಮ-ಸಾಕ್ಷಾ
ಆತ್ಮ-ಸಾಕ್ಷಾತ್ಕಾರ
ಆತ್ಮ-ಸಾಕ್ಷಾತ್ಕಾರದ
ಆತ್ಮ-ಸಾಕ್ಷಾತ್ಕಾರ-ದಿಂದ
ಆತ್ಮ-ಸಾಕ್ಷಾತ್ಕಾರ-ವನ್ನು
ಆತ್ಮ-ಸಾಕ್ಷಾತ್ಕಾರ-ವೆಂಬ
ಆತ್ಮ-ಸಿದ್ಧಾಂತವು
ಆತ್ಮಸ್ವ-ರೂಪ-ನಾದ
ಆತ್ಮ-ಹತ್ಯ-ದಿಂದ
ಆತ್ಮ-ಹತ್ಯೆ
ಆತ್ಮ-ಹತ್ಯೆ-ಯನ್ನು
ಆತ್ಮ-ಹತ್ಯೆ-ಯಿಂದ
ಆತ್ಮ-ಹಾನಿ-ಕರ
ಆತ್ಮ-ಹಾನಿ-ಕರ-ವಾದ
ಆತ್ಮಾನಂದ-ದಲ್ಲಿ
ಆತ್ಮಾನು-ಭವ-ಒಂದು
ಆತ್ಮಾನು-ಭಾವಕ್ಕೆ
ಆತ್ಮಾಭಿ-ಮಾನದ
ಆತ್ಮೀಯ
ಆತ್ಮೋದ್ಧಾರ-ವಾ-ಗು-ವುದು
ಆತ್ಯನ
ಆತ್ಯಾಧು-ನಿಕ
ಆದ
ಆದಂತೆ
ಆದಂತೆಲ್ಲ
ಆದ-ಕಾರಣ
ಆದ-ಕಾರಣವೆ
ಆದ-ಕಾರಣವೇ
ಆದದ್ದು
ಆದ-ಮೇಲೂ
ಆದ-ಮೇಲೆ
ಆದರೂ
ಆದರೆ
ಆದರ್ಶ
ಆದರ್ಶಕ್ಕಾಗಿ
ಆದರ್ಶಕ್ಕೆ
ಆದರ್ಶ-ಗಳ
ಆದರ್ಶ-ಗಳನ್ನು
ಆದರ್ಶ-ಗಳನ್ನೊಳ-ಗೊಂಡ
ಆದರ್ಶ-ಗಳಿಂದ
ಆದರ್ಶ-ಗಳು
ಆದರ್ಶದ
ಆದರ್ಶ-ದಲ್ಲಿ
ಆದರ್ಶ-ದಷ್ಟು
ಆದರ್ಶ-ದಿಂದ
ಆದರ್ಶ-ದೆಡೆಗೆ
ಆದರ್ಶ-ದೊಂದಿಗೆ
ಆದರ್ಶ-ಭಾವನೆ-ಗಳು
ಆದರ್ಶ-ಮನುಷ್ಯ
ಆದರ್ಶ-ವನ್ನು
ಆದರ್ಶ-ವಾಗ-ಲಾರ-ದೆಂದು
ಆದರ್ಶ-ವಾಗಿ
ಆದರ್ಶ-ವಾಗಿದ್ದ
ಆದರ್ಶ-ವಾ-ಗು-ವುದು
ಆದರ್ಶ-ವಾದ
ಆದರ್ಶ-ವಿದೆ
ಆದರ್ಶ-ವಿದ್ದಷ್ಟೂ
ಆದರ್ಶ-ವಿರ-ಬೇಕು
ಆದರ್ಶ-ವಿರುವ
ಆದರ್ಶ-ವಿರು-ವುದು
ಆದರ್ಶ-ವಿಲ್ಲದೆ
ಆದರ್ಶವು
ಆದರ್ಶವೂ
ಆದರ್ಶ-ವೆಂದರೆ
ಆದರ್ಶ-ವೆಂದು
ಆದರ್ಶ-ವೆನ್ನು-ವುದು
ಆದರ್ಶವೇ
ಆದರ್ಶ-ಶೋಧನೆ
ಆದರ್ಶ-ಸಾಧನೆ-ಯನ್ನು
ಆದ-ವನು
ಆದ-ವರು
ಆದ-ವರೂ
ಆದವು
ಆದವು-ಗಳಲ್ಲ
ಆದಷ್ಟು
ಆದಾಗ
ಆದಾಗಲೆ
ಆದಾಗಲೇ
ಆದಿ
ಆದಿ-ಅಂತ್ಯ-ಗಳ
ಆದಿ-ಅಂತ್ಯ-ಗಳಿಲ್ಲ
ಆದಿ-ಅಂತ್ಯ-ರಹಿತ
ಆದಿ-ಅಂತ್ಯ-ರಹಿತನು
ಆದಿ-ಗುರು-ವಿನಂತಿ-ರುವ
ಆದಿ-ಮಾನವ
ಆದಿ-ಮಾನವ-ನಿಂದ
ಆದಿಯ
ಆದಿ-ಯನ್ನು
ಆದಿ-ಯಲ್ಲಿ
ಆದಿ-ಯಲ್ಲಿದ್ದ
ಆದಿ-ಯಲ್ಲಿಯೂ
ಆದಿ-ಯಿಂದಲೂ
ಆದಿ-ಯಿಲ್ಲ
ಆದಿಯು
ಆದುದ
ಆದು-ದನ್ನು
ಆದು-ದ-ರಿಂದ
ಆದು-ದ-ರಿಂದಲೆ
ಆದುದು
ಆದು-ದೊಂದು
ಆದುನಿಕ
ಆದುವು
ಆದುವು-ಆದರೆ
ಆದೃಷ್ಟಿ-ಯಿಂದ
ಆದೊ-ಡನೆ
ಆದೊ-ಡನೆಯೇ
ಆದ್ದ
ಆದ್ದ-ರಿಂದ
ಆದ್ದ-ರಿಂದಲೇ
ಆದ್ಯಂತ
ಆದ್ಯಂತ-ರಹಿತ
ಆದ್ಯಂತ-ವನ್ನೂ
ಆದ್ಯಂತ-ವಾಗಿ
ಆಧರಿಸಿ-ಕೊಳ್ಳು-ವುದು
ಆಧರಿಸಿದೆ
ಆಧಾರ
ಆಧಾರದ
ಆಧಾರ-ಭೂತ-ವನ್ನಾಗಿ
ಆಧಾರ-ಭೂತ-ವಾದ
ಆಧಾರ-ವನ್ನು
ಆಧಾರ-ವಾಗಿ
ಆಧಾರ-ವಾಗಿ-ರುವ
ಆಧಾರ-ವಿದೆ
ಆಧಾರ-ವಿಲ್ಲ
ಆಧಾರ-ವಿಲ್ಲದೆ
ಆಧಾರವೂ
ಆಧಾರ-ವೇನು
ಆಧಾರ-ವೊಂದನ್ನು
ಆಧಿಕ್ಯ-ದಿಂದ
ಆಧು
ಆಧು-ನಿಕ
ಆಧು-ನಿಕರ
ಆಧು-ನಿಕ-ರನ್ನು
ಆಧು-ನಿಕ-ರಲ್ಲಿ
ಆಧು-ನಿಕ-ರಷ್ಟೇ
ಆಧು-ನಿಕ-ರಾ-ದಷ್ಟೂ
ಆಧು-ನಿಕ-ರಿಗೆ
ಆಧು-ನಿಕರು
ಆಧು-ನಿಕ-ವಾ-ಗಲಿ
ಆಧ್ಯಾ
ಆಧ್ಯಾತ್ಮ
ಆಧ್ಯಾತ್ಮ-ದಲ್ಲಿದೆ
ಆಧ್ಯಾತ್ಮ-ವಿದೆ
ಆಧ್ಯಾತ್ಮಿಕ
ಆಧ್ಯಾತ್ಮಿಕಕ್ಕೇ
ಆಧ್ಯಾತ್ಮಿಕತೆ
ಆಧ್ಯಾತ್ಮಿಕ-ತೆಯ
ಆಧ್ಯಾತ್ಮಿಕ-ತೆಯಿಂದ
ಆಧ್ಯಾತ್ಮಿಕ-ತೆಯು
ಆಧ್ಯಾತ್ಮಿಕ-ಭಾವನೆ
ಆಧ್ಯಾತ್ಮಿಕ-ವಾಗಿ
ಆಧ್ಯಾತ್ಮಿಕ-ವಾಗಿ-ಯಾಗಲಿ
ಆಧ್ಯಾತ್ಮಿಕ-ವಾದ
ಆಧ್ಯಾತ್ಮಿಕ-ವೀರರು
ಆನಂದ
ಆನಂದಕ್ಕಿಂತ
ಆನಂದ-ಗಳ
ಆನಂದ-ಗಳು
ಆನಂದ-ಗಳೆಂಬ
ಆನಂದದ
ಆನಂದ-ದಲ್ಲಿ
ಆನಂದ-ದಾಯಕ-ವಾದುದು
ಆನಂದ-ದಿಂದ
ಆನಂದ-ಪರ-ವಶತೆ
ಆನಂದ-ಭರಿತ-ನಾಗು-ವನು
ಆನಂದ-ಮಯ
ಆನಂದ-ಮಯ-ನಾದ
ಆನಂದ-ಮಯನು
ಆನಂದ-ಮಯನೂ
ಆನಂದ-ಮಯ-ವೆಂದು
ಆನಂದ-ವನ್ನು
ಆನಂದ-ವಾಗಲಿ
ಆನಂದ-ವಾಗಿ
ಆನಂದ-ವಾಗಿದೆ
ಆನಂದ-ವಿದೆಯೋ
ಆನಂದ-ವಿಲ್ಲದ
ಆನಂದವು
ಆನಂದ-ವೆನ್ನು-ವುದು
ಆನಂದ-ವೆಲ್ಲಾ
ಆನಂದವೇ
ಆನಂದಿಸ-ಕೂಡದು
ಆನಂದಿಸ-ಬಲ್ಲರು
ಆನಂದಿಸ-ಬಲ್ಲವು
ಆನಂದಿಸ-ಬಹುದು
ಆನಂದಿಸ-ಬಹು-ದೆಂದು
ಆನಂದಿಸ-ಬೇಕು
ಆನಂದಿಸ-ಬೇಡಿ
ಆನಂದಿಸಿ-ದವನು
ಆನಂದಿಸು-ತ್ತಿರು-ವನು
ಆನಂದಿ-ಸುವ
ಆನಂದಿಸು-ವನು
ಆನಂದಿಸು-ವರು
ಆನಂದಿಸುವೆ
ಆನು-ವಂಶಿಕ
ಆನೆ
ಆನೆಯ
ಆನೆ-ಯನ್ನು
ಆಪೀಮು
ಆಪೋಶನ
ಆಪ್ತನು
ಆಪ್ತ-ನೆಂದು
ಆಪ್ತ-ರಾಗು-ವುದಿಲ್ಲ
ಆಪ್ತವಾಕ್ಯ
ಆಪ್ತವಾಕ್ಯ-ಗಳಾ-ಗಿರು-ವುದೇ
ಆಪ್ತವಾಕ್ಯ-ಗಳೇ
ಆಪ್ತ-ವಾಕ್ಯದ
ಆಫ್
ಆಫ್ರಿಕಾ
ಆಫ್ರಿಕಾ-ದಲ್ಲಿ
ಆಬಾಲ
ಆಭರಣ-ಗಳನ್ನು
ಆಭರಣ-ಗಳು
ಆಮೂಲಾಗ್ರ-ವಾಗಿ
ಆಮೇಲೆ
ಆಯತಾ-ವಾದಿ-ಗಳ
ಆಯವ
ಆಯ-ವುದೂ
ಆಯಸ್ಕಾಂತ
ಆಯಾ
ಆಯಾಮ-ವೆಂದರೆ
ಆಯಾಯ
ಆಯಾಸ
ಆಯಿತು
ಆಯಿತುಈ
ಆಯಿತೆಂಬು-ದನ್ನು
ಆಯುಧ-ವಾಗಿ
ಆಯುಸ್ಸು
ಆಯ್ಕೆ
ಆರಂಭ
ಆರಂಭ-ವಾಗು-ವುದು
ಆರಕ್ಕೆ
ಆರಣ್ಯಕ
ಆರನೆ-ಯದು
ಆರಾಧಕ-ರಾಗಿದ್ದರು
ಆರಾಧನೆ
ಆರಾಧ-ನೆಗೆ
ಆರಾಧನೆ-ಯನ್ನು
ಆರಾಧಿಸ-ಬೇಕಾ-ಗಿಲ್ಲ
ಆರಾಧಿಸ-ಬೇಕೆಂದು
ಆರಾಧಿ-ಸಲು
ಆರಾಧಿಸಿ-ದರೆ
ಆರಾಧಿಸಿ-ರುವರು
ಆರಾಧಿಸು
ಆರಾಧಿಸು-ವವರು
ಆರಾಮ-ವಾಗಿ
ಆರಾಮ-ವಾಗಿ-ರಲಿ
ಆರಿಸಿ
ಆರಿಸಿ-ಕೊಂಡರೆ
ಆರಿಸಿ-ಕೊಂಡು
ಆರಿಸಿ-ಕೊಳ್ಳ-ಬೇಕು
ಆರಿಸಿ-ಕೊಳ್ಳ-ಬೇಡಿ
ಆರಿಸಿ-ಕೊಳ್ಳು
ಆರಿಸಿ-ಕೊಳ್ಳು-ವರು
ಆರಿಸಿ-ಕೊಳ್ಳು-ವರೋ
ಆರಿಸಿ-ಕೊಳ್ಳು-ವುದು
ಆರಿಸಿದ
ಆರಿಸಿ-ದರೆ
ಆರು
ಆರುಣಿ
ಆರು-ನೂರು
ಆರೆವು
ಆರೋಗ್ಯ
ಆರೋಗ್ಯ-ಕರ
ಆರೋಗ್ಯಕ್ಕಾಗಿ-ಯಾ-ಗಲಿ
ಆರೋಗ್ಯ-ದಲ್ಲಿ
ಆರೋಗ್ಯ-ದಲ್ಲಿದ್ದರೆ
ಆರೋಗ್ಯ-ದಲ್ಲಿ-ರು-ವಂತೆ
ಆರೋಗ್ಯ-ದಿಂದಿ-ರುವ
ಆರೋಗ್ಯ-ವನ್ನು
ಆರೋಗ್ಯ-ವಾಗಿ
ಆರೋಗ್ಯ-ವಾಗಿಡ
ಆರೋಗ್ಯ-ವಾಗಿದ್ದರೆ
ಆರೋಗ್ಯ-ವಾಗಿ-ರ-ಬೇಕಾ-ದರೆ
ಆರೋಗ್ಯ-ವಾಗಿ-ರು-ವ-ವನು
ಆರೋಗ್ಯ-ವಿಲ್ಲ-ದ-ವರು
ಆರೋಗ್ಯವು
ಆರೋಗ್ಯವೂ
ಆರೋಗ್ಯವೇ
ಆರೋಗ್ಯ-ಶಾಲಿ-ಯಾದ
ಆರೋಗ್ಯಸ್ಥಿತಿ-ಯನ್ನು
ಆರೋಪ
ಆರೋಪಿ
ಆರೋಪಿ-ಸ-ಬಹು-ದಾದ
ಆರೋಪಿ-ಸಲು
ಆರೋಪಿಸಿ-ರುವ
ಆರೋಪಿಸು-ವನು
ಆರೋಪಿಸು-ವು-ದಕ್ಕೆ
ಆರೋಪಿಸು-ವು-ದ-ರಲ್ಲಿ
ಆರೋಪಿಸು-ವುದು
ಆರೋಹಣ
ಆರೋಹಣ-ದಿಂದ
ಆರ್ಜನೆ
ಆರ್ತ-ರಾಗಿ
ಆರ್ನಾಲ್ಡ್
ಆರ್ಭಟ
ಆರ್ಯ
ಆರ್ಯನು
ಆರ್ಯ-ಪಂಗಡ-ಗಳಲ್ಲಿ
ಆರ್ಯ-ಮ-ಹರ್ಷಿ-ಗಳು
ಆರ್ಯರ
ಆರ್ಯ-ರಿಗೆ
ಆರ್ಯ-ಸಾ-ಹಿತ್ಯದ
ಆಲಂಗಿಸಿ
ಆಲಂಬನ-ದಿಂದ
ಆಲದ
ಆಲದ-ಮರ
ಆಲದ-ಮ-ರವು
ಆಲಸ್ಯ
ಆಲಿಂಗನ
ಆಲಿ-ಸಿದ
ಆಲಿ-ಸಿದೆ
ಆಲಿಸು-ವಂತಹ
ಆಲೆ
ಆಲೋ
ಆಲೋ-ಚನಾ
ಆಲೋ-ಚನಾಕ್ಷೇತ್ರ
ಆಲೋ-ಚನಾ-ಜೀವಿ-ಯಾಗಿ-ರು-ವುದೇ
ಆಲೋ-ಚನಾ-ಪರ-ನಾ-ದಷ್ಟೂ
ಆಲೋ-ಚನಾ-ಪರರು
ಆಲೋ-ಚನಾ-ಪರ-ರೊಂದಿಗೆ
ಆಲೋ-ಚನಾ-ರಾ-ಹಿತ್ಯವೂ
ಆಲೋ-ಚನಾ-ಶಕ್ತಿಯು
ಆಲೋ-ಚನೆ
ಆಲೋ-ಚನೆ-ಗಳ
ಆಲೋ-ಚನೆ-ಗ-ಳನ್ನು
ಆಲೋ-ಚನೆ-ಗ-ಳನ್ನೂ
ಆಲೋ-ಚನೆ-ಗಳನ್ನೆಲ್ಲ
ಆಲೋ-ಚನೆ-ಗಳನ್ನೇ
ಆಲೋ-ಚನೆ-ಗಳಿ-ಗಿಂತ
ಆಲೋ-ಚನೆ-ಗಳಿವೆ
ಆಲೋ-ಚನೆ-ಗಳು
ಆಲೋ-ಚನೆ-ಗಳೂ
ಆಲೋ-ಚನೆ-ಗಳೆಲ್ಲ-ವನ್ನೂ
ಆಲೋ-ಚನೆ-ಗಿಂತಲೂ
ಆಲೋ-ಚ-ನೆಗೂ
ಆಲೋ-ಚ-ನೆಗೆ
ಆಲೋ-ಚನೆಯ
ಆಲೋ-ಚನೆ-ಯಂತೆ
ಆಲೋ-ಚನೆ-ಯನ್ನು
ಆಲೋ-ಚನೆ-ಯನ್ನೂ
ಆಲೋ-ಚನೆ-ಯಲ್ಲಿ
ಆಲೋ-ಚನೆ-ಯಾಗಲಿ
ಆಲೋ-ಚನೆ-ಯಿಂದ
ಆಲೋ-ಚನೆ-ಯಿಂದಾಚೆ
ಆಲೋ-ಚ-ನೆಯು
ಆಲೋ-ಚ-ನೆಯೂ
ಆಲೋ-ಚ-ನೆಯೆ
ಆಲೋ-ಚ-ನೆಯೇ
ಆಲೋಚಿ
ಆಲೋ-ಚಿಸ-ದಿರು
ಆಲೋ-ಚಿಸದೆ
ಆಲೋ-ಚಿಸ-ಬಹುದು
ಆಲೋ-ಚಿಸ-ಬೇಕಾ-ಗುತ್ತದೆ
ಆಲೋ-ಚಿಸ-ಬೇಕಾ-ದರೆ
ಆಲೋ-ಚಿಸ-ಬೇಕು
ಆಲೋ-ಚಿಸ-ಬೇಕೆಂದು
ಆಲೋ-ಚಿಸ-ಬೇಡಿ
ಆಲೋ-ಚಿಸ-ಲಾರ-ದ-ವರೂ
ಆಲೋ-ಚಿಸ-ಲಾರೆವು
ಆಲೋ-ಚಿಸಲು
ಆಲೋ-ಚಿಸಿ
ಆಲೋ-ಚಿಸಿದ
ಆಲೋ-ಚಿಸಿ-ದಂತೆ
ಆಲೋ-ಚಿಸಿ-ದನು
ಆಲೋ-ಚಿಸಿ-ದರೆ
ಆಲೋ-ಚಿಸಿ-ದಾಗ
ಆಲೋ-ಚಿಸಿದೆ
ಆಲೋ-ಚಿಸಿ-ದೆನೊ
ಆಲೋ-ಚಿಸಿ-ದೆವು
ಆಲೋ-ಚಿಸಿದ್ದೆ
ಆಲೋ-ಚಿಸಿ-ರು-ವರು
ಆಲೋ-ಚಿಸು
ಆಲೋ-ಚಿಸುತ್ತ
ಆಲೋ-ಚಿಸು-ತ್ತದೆ
ಆಲೋ-ಚಿಸುತ್ತಾ
ಆಲೋ-ಚಿಸು-ತ್ತಾನೊ
ಆಲೋ-ಚಿಸು-ತ್ತಾರೆಯೋ
ಆಲೋ-ಚಿಸು-ತ್ತಿದ್ದರೆ
ಆಲೋ-ಚಿಸುತ್ತಿ-ರು-ವಾಗ
ಆಲೋ-ಚಿಸುತ್ತಿರು-ವು-ದನ್ನು
ಆಲೋ-ಚಿಸುತ್ತೀರಿ
ಆಲೋ-ಚಿಸುತ್ತೇ-ನೆಯೋ
ಆಲೋ-ಚಿಸುವ
ಆಲೋ-ಚಿಸು-ವಂತೆ
ಆಲೋ-ಚಿಸು-ವನೊ
ಆಲೋ-ಚಿಸು-ವರು
ಆಲೋ-ಚಿಸು-ವಾಗ
ಆಲೋ-ಚಿಸು-ವು-ದಕ್ಕೆ
ಆಲೋ-ಚಿಸು-ವು-ದನ್ನು
ಆಲೋ-ಚಿಸು-ವುದರ
ಆಲೋ-ಚಿಸು-ವು-ದ-ರಿಂದ
ಆಲೋ-ಚಿಸು-ವು-ದಲ್ಲ
ಆಲೋ-ಚಿಸು-ವು-ದಿಲ್ಲ
ಆಲೋ-ಚಿ ಸು-ವು-ದಿಲ್ಲವೊ
ಆಲೋ-ಚಿಸು-ವು-ದಿಲ್ಲವೋ
ಆಲೋ-ಚಿಸು-ವುದು
ಆಲೋ-ಚಿಸು-ವುದೇ
ಆಲೋ-ಚಿಸು-ವೆನು
ಆಲೋ-ಚಿಸು-ವೆವೋ
ಆಲ್ಲಿರ-ಲಿಲ್ಲ
ಆಳ
ಆಳಕ್ಕೆ
ಆಳ-ದಲ್ಲಿ-ರಲಿ
ಆಳ-ದಿಂದ
ಆಳರಸ-ದಿಂದ
ಆಳರಸರ
ಆಳಲ್ಲ
ಆಳ-ವಾಗಿ
ಆಳಾ-ಗು-ವುದು
ಆಳಿ
ಆಳಿದೆ
ಆಳಿಸಿ-ಕೊಳ್ಳದೆ
ಆಳು-ಗಳಾ-ಗುವ
ಆಳುತ್ತಿದ್ದ
ಆಳುತ್ತಿದ್ದ-ನೆಂದು
ಆಳುತ್ತಿದ್ದರು
ಆಳುತ್ತಿದ್ದರೆ
ಆಳುತ್ತಿ-ರುವ
ಆಳುತ್ತಿ-ರುವನು
ಆಳುತ್ತಿ-ರುವ-ನೆಂಬ
ಆಳುತ್ತೇನೆ
ಆಳುವ
ಆಳು-ವರು
ಆಳುವ-ವನು
ಆಳುವ-ವನೆ
ಆಳುವ-ವ-ನೊಬ್ಬ
ಆಳುವ-ವ-ರಾಗ-ಬೇಕು
ಆಳು-ವು-ದನ್ನು
ಆಳು-ವು-ದಿಲ್ಲ
ಆಳು-ವುದು
ಆಳ್ವಿಕೆ-ಯಲ್ಲಿ
ಆಳ್ವಿಕೆ-ಯಲ್ಲಿ-ರುವ
ಆವರಣ
ಆವರಣ-ವನ್ನು
ಆವರ-ಣವು
ಆವರ-ಣವೂ
ಆವರಿಸ
ಆವರಿಸ-ಬೇಕು
ಆವರಿಸ-ಲ್ಪಟ್ಚಿದೆ
ಆವರಿಸಿ
ಆವರಿಸಿ-ಕೊಂಡಿ-ರುವ
ಆವರಿಸಿದ
ಆವರಿಸಿ-ರುವ
ಆವರಿಸು-ವಷ್ಟು
ಆವರಿಸು-ವು-ದಿಲ್ಲ
ಆವರಿಸು-ವುದು
ಆವಶ್ಯಕ
ಆವಶ್ಯಕ-ಇಲ್ಲಿ
ಆವಶ್ಯಕತೆ
ಆವಶ್ಯಕತೆ-ಗಳು
ಆವಶ್ಯಕತೆ-ಯನ್ನು
ಆವಶ್ಯಕತೆಯೂ
ಆವಶ್ಯಕ-ತೆಯೇ
ಆವಶ್ಯಕ-ತೆ-ಯೇನು
ಆವಶ್ಯಕ-ವಲ್ಲ
ಆವಶ್ಯಕ-ವಲ್ಲ-ವಾ-ದರೂ
ಆವಶ್ಯಕ-ವಾಗಿ
ಆವಶ್ಯಕ-ವಾಗಿದ್ದ
ಆವಶ್ಯಕ-ವಾದ
ಆವಶ್ಯಕ-ವಿತ್ತು
ಆವಶ್ಯಕ-ವಿದೆ
ಆವಶ್ಯಕ-ವಿಲ್ಲ
ಆವಶ್ಯಕ-ವೆಂದು
ಆವಶ್ಯಕ-ವೆಂಬು-ದನ್ನು
ಆವಾಗ
ಆವಾಗಲೆ
ಆವಾ-ಹ-ಕರು
ಆವಿ
ಆವಿಗೇ
ಆವಿಯ
ಆವಿ-ಯಂತೆ
ಆವಿ-ಯಾಗು-ವುದು
ಆವಿ-ಯಿಂದ
ಆವಿರ್ಭವಿ-ಸಲು
ಆವಿರ್ಭವಿ-ಸು-ವುವು
ಆವಿರ್ಭಾವ
ಆವಿರ್ಭಾವ-ಗ-ಳಾದರೂ
ಆವಿರ್ಭಾವ-ಗಳು
ಆವಿರ್ಭಾವ-ದಂತೆ
ಆವಿರ್ಭಾ-ವವೇ
ಆವಿರ್ಭೂತ-ರಾದ
ಆವಿಷ್ಕಾರ-ಗಳಂತಲ್ಲದೆ
ಆವಿಷ್ಕಾರ-ಗಳು
ಆವಿಷ್ಕಾರ-ದಲ್ಲಿದೆಯೋ
ಆವೃತ-ವಾಗಿ
ಆವೃತ-ವಾ-ಗಿ-ರುವುದು
ಆವೃತ-ವಾದ
ಆವೃತ್ತ-ನಾಗು-ವನು
ಆವೃತ್ತ-ರಾಗಿ
ಆಶಯ
ಆಶಯ-ವಾದ
ಆಶ-ಯವು
ಆಶ-ಯವೂ
ಆಶಾ
ಆಶಾ-ಜನಕ-ವಾದುದೂ
ಆಶಾ-ಭರಿತ
ಆಶಾ-ವಾದ
ಆಶಾ-ವಾದ-ದಿಂದ
ಆಶಾ-ವಾದವೂ
ಆಶಾ-ವಾದಿ
ಆಶಾ-ವಾದಿ-ಗಳು
ಆಶಾ-ವಾದಿ-ಯಾಗು
ಆಶಿಷ್ಟರೂ
ಆಶಿಸ-ಬಲ್ಲರು
ಆಶಿಸ-ಬಹುದು
ಆಶಿಸ-ಬಾರದು
ಆಶಿಸ-ಬೇಡಿ
ಆಶಿಸಿ
ಆಶಿಸಿ-ದರು
ಆಶಿಸಿ-ದರೆ
ಆಶಿಸು
ಆಶಿ-ಸುತ್ತಾ
ಆಶಿ-ಸುತ್ತಾರೋ
ಆಶಿ-ಸುತ್ತಿ-ರುವನು
ಆಶಿಸುತ್ತೇನೆ
ಆಶಿಸು-ತ್ತೇವೆ
ಆಶಿ-ಸುವ
ಆಶಿಸು-ವಂತೆ
ಆಶಿಸು-ವನು
ಆಶಿಸು-ವರು
ಆಶಿಸು-ವರೊ
ಆಶಿಸು-ವರೋ
ಆಶಿಸು-ವಾಗ
ಆಶಿಸು-ವಿರೋ
ಆಶಿಸು-ವು-ದಕ್ಕಿಂತ
ಆಶಿಸು-ವು-ದಿಲ್ಲ
ಆಶಿಸು-ವು-ದಿಲ್ಲವೋ
ಆಶಿಸು-ವುದು
ಆಶೀರ್ವದಿ-ಸಲಿ
ಆಶೀರ್ವದಿಸಿ
ಆಶೀರ್ವದಿಸಿ-ದ-ವರು
ಆಶೀರ್ವಾದ
ಆಶೀರ್ವಾದದ
ಆಶೀರ್ವಾದಪ್ರಾಯ-ವಾ-ಗಿ-ರುವುದು
ಆಶೀರ್ವಾ-ದವೆ
ಆಶೀರ್ವಾ-ವನ್ನು
ಆಶೆ
ಆಶೆಯ
ಆಶೆ-ಯನ್ನು
ಆಶ್ಚರ್ಯ
ಆಶ್ಚರ್ಯ-ಪಟ್ಟನು
ಆಶ್ಚರ್ಯ-ಪಟ್ಟು
ಆಶ್ಚರ್ಯ-ವಾಗ-ಬಹುದು
ಆಶ್ಚರ್ಯ-ವಾಗಿ
ಆಶ್ಚರ್ಯ-ವಾ-ಗು-ವುದು
ಆಶ್ಚರ್ಯ-ವಾಯಿತು
ಆಶ್ಚರ್ಯವೇ
ಆಶ್ಚರ್ಯ-ವೇ-ನಿದೆ
ಆಶ್ಚರ್ಯ-ವೇನೂ
ಆಶ್ಯ-ಕತೆ-ಯಿಲ್ಲ
ಆಶ್ರಮ
ಆಶ್ರಯ
ಆಶ್ರಯ-ವಾಗಿ
ಆಶ್ರಯ-ವಾದ
ಆಶ್ರಯ-ವಿಲ್ಲದೆ
ಆಶ್ರ-ಯವೇ
ಆಶ್ರ-ಯಿಸ-ಬೇಕು
ಆಶ್ರ-ಯಿಸಿ
ಆಶ್ರ-ಯಿಸಿ-ಕೊಂಡಿದೆ
ಆಶ್ರ-ಯಿಸಿದ್ದರೆ
ಆಶ್ರಯಿ-ಸಿವೆ
ಆಶ್ರಯಿ-ಸುತ್ತಾರೋ
ಆಷಾಢ
ಆಷಾಢ-ಭೂತಿಯ
ಆಸಕ್ತ-ನಾಗಿ
ಆಸಕ್ತನು
ಆಸಕ್ತ-ರಾಗಿ
ಆಸಕ್ತ-ರಾಗಿದ್ದ
ಆಸಕ್ತ-ರಾಗಿದ್ದರು
ಆಸಕ್ತ-ರಾಗಿದ್ದರೂ
ಆಸಕ್ತ-ರಾಗಿ-ರು-ವು-ದಿಲ್ಲ
ಆಸಕ್ತ-ರಾ-ಗಿ-ರು-ವೆವು
ಆಸಕ್ತರು
ಆಸಕ್ತಿ
ಆಸಕ್ತಿಗೆ
ಆಸಕ್ತಿಯ
ಆಸಕ್ತಿ-ಯನ್ನು
ಆಸಕ್ತಿ-ಯಿಂದ
ಆಸಕ್ತಿ-ಯಿದ್ದರೆ
ಆಸಕ್ತಿ-ಯೊಂದಿಗೆ
ಆಸನ
ಆಸ-ನದ
ಆಸನ-ವನ್ನು
ಆಸ-ನವು
ಆಸರೆ
ಆಸರೆಯ
ಆಸರೆಯೂ
ಆಸುರೀ
ಆಸುರೀದ್ವೇಷವು
ಆಸೆ
ಆಸೆ-ಗಳ
ಆಸೆ-ಗ-ಳನ್ನು
ಆಸೆ-ಗ-ಳನ್ನೂ
ಆಸೆ-ಗಳನ್ನೆಲ್ಲಾ
ಆಸೆ-ಗಳನ್ನೇ
ಆಸೆ-ಗಳಿಂದ
ಆಸೆ-ಗಳಿ-ಗಾಗಿ
ಆಸೆ-ಗಳಿದ್ದರೆ
ಆಸೆ-ಗಳು
ಆಸೆ-ಗಳೂ
ಆಸೆ-ಗಳೆಂಬು-ದನ್ನು
ಆಸೆ-ಗಳೆಲ್ಲ-ವನ್ನೂ
ಆಸೆಗೆ
ಆಸೆಯ
ಆಸೆ-ಯನ್ನು
ಆಸೆ-ಯನ್ನೆಲ್ಲ
ಆಸೆ-ಯಲ್ಲಿ
ಆಸೆ-ಯಿಂದ
ಆಸೆಯು
ಆಸೆಯೂ
ಆಸೆಯೆ
ಆಸೆ-ಯೆಂಬ
ಆಸೆ-ಯೆಲ್ಲ
ಆಸ್ತಿ
ಆಸ್ತಿಕ
ಆಸ್ತಿ-ಕ-ರಲ್ಲಿ
ಆಸ್ತಿ-ಕರು
ಆಸ್ತಿ-ಕರೆ
ಆಸ್ತಿ-ಕರೊ
ಆಸ್ತಿತ್ವ
ಆಸ್ತಿ-ಯನ್ನು
ಆಸ್ತಿ-ಯಲ್ಲ
ಆಸ್ತಿ-ಯಲ್ಲಿ
ಆಸ್ತಿ-ಯಾಗ-ಬೇಕು
ಆಸ್ತಿಯೂ
ಆಸ್ಥಾನ-ದಲ್ಲಿ
ಆಸ್ಪತ್ರೆಗೆ
ಆಸ್ಪದ-ಕೊಡುತ್ತ
ಆಹಾರ
ಆಹಾರಕ್ಕಾಗಿ
ಆಹಾರಕ್ಕೋಸ್ಕರ-ವಾಗಿಯೆ
ಆಹಾರಕ್ಕೋಸ್ಕರ-ವಾಗಿಯೇ
ಆಹಾರ-ಗಳನ್ನು
ಆಹಾರದ
ಆಹಾರ-ದಲ್ಲಿ
ಆಹಾರ-ದಿಂದ
ಆಹಾರ-ದಿಂದಲೇ
ಆಹಾರ-ವನ್ನು
ಆಹಾರ-ವನ್ನೇ
ಆಹಾರ-ವಾಗಿ
ಆಹಾರ-ವೆಂದರೆ
ಆಹುತಿ
ಆಹುತಿ-ಯನ್ನು
ಆಹುತಿ-ಯಾಗುತ್ತಿ-ರುವುದು
ಆಹುತಿ-ಯಿಂದ
ಆಹುತಿಯು
ಆ್ಯಡಮ್
ಆ್ಯಡಮ್ನ
ಇಂಗ್ಲೀಷಿನ-ವರ
ಇಂಗ್ಲೀಷಿನ-ವರೂ
ಇಂಗ್ಲೀಷ್
ಇಂಗ್ಲೆಂಡಿನ
ಇಂಗ್ಲೆಂಡಿ-ನಲ್ಲಿ
ಇಂಗ್ಲೆಂಡಿನಲ್ಲಿ-ರು-ವಿರಿ
ಇಂಗ್ಲೆಂಡಿ-ನಲ್ಲೂ
ಇಂಗ್ಲೆಂಡ್
ಇಂಡಿಯ
ಇಂಡಿಯನ್ನನು
ಇಂಡಿಯಾ
ಇಂಡಿಯಾ-ದೇಶಕ್ಕೆ
ಇಂತಹ
ಇಂತಹ-ದಾ-ವು-ದನ್ನೂ
ಇಂತಹದು
ಇಂತಹ-ವ-ನನ್ನು
ಇಂತಹ-ವ-ರಿಂದ
ಇಂತಹ-ವರು
ಇಂತಹ-ವರೆ
ಇಂತಹ-ವ-ರೆಲ್ಲರೂ
ಇಂತಹು-ದೆಂದು
ಇಂತಿಂಥ-ವರು
ಇಂಥ
ಇಂಥದೆ
ಇಂಥಿಂಥ-ವರ
ಇಂದಿಗೂ
ಇಂದಿಗೆ
ಇಂದಿನ
ಇಂದಿ-ನಂತಹ
ಇಂದಿ-ನಂತೆ
ಇಂದಿನದು
ಇಂದಿನ-ವರು
ಇಂದಿನ-ವರೆಗೂ
ಇಂದಿ-ನಿಂದ
ಇಂದಿ-ರುವ-ದುಃಸ್ಥಿತಿಗೆ
ಇಂದು
ಇಂದೊ
ಇಂದೋ
ಇಂದ್ರ
ಇಂದ್ರನ
ಇಂದ್ರ-ನಲ್ಲ
ಇಂದ್ರ-ನಿಲ್ಲಿ-ರುವನು
ಇಂದ್ರನು
ಇಂದ್ರಾದಿ
ಇಂದ್ರಿಗ್ರಹ-ಣ-ಗಳು
ಇಂದ್ರಿಯ
ಇಂದ್ರಿ-ಯಕ್ಕೂ
ಇಂದ್ರಿ-ಯಕ್ಕೆ
ಇಂದ್ರಿಯ-ಗಳ
ಇಂದ್ರಿಯ-ಗ-ಳನ್ನು
ಇಂದ್ರಿಯ-ಗಳಲ್ಲ
ಇಂದ್ರಿಯ-ಗಳಲ್ಲಿ
ಇಂದ್ರಿಯ-ಗಳಲ್ಲಿದೆ
ಇಂದ್ರಿಯ-ಗಳಿಂದ
ಇಂದ್ರಿಯ-ಗಳಿ-ಗಿಂತಲೂ
ಇಂದ್ರಿಯ-ಗಳಿಗೂ
ಇಂದ್ರಿಯ-ಗಳಿಗೆ
ಇಂದ್ರಿಯ-ಗಳಿ-ಗೊಂದು
ಇಂದ್ರಿಯ-ಗಳಿ-ರ-ಬೇಕು
ಇಂದ್ರಿಯ-ಗಳಿವೆ
ಇಂದ್ರಿಯ-ಗಳು
ಇಂದ್ರಿಯ-ಗಳೂ
ಇಂದ್ರಿಯ-ಗಳೆಂಬ
ಇಂದ್ರಿಯ-ಗಳೆಲ್ಲಾ
ಇಂದ್ರಿಯ-ಗಳೇ
ಇಂದ್ರಿಯ-ಗಳೊಂದಿಗೆ
ಇಂದ್ರಿಯ-ಗ್ರಹಣ
ಇಂದ್ರಿಯ-ಗ್ರಹಣ-ಗ-ಳನ್ನು
ಇಂದ್ರಿಯ-ಗ್ರಹಣ-ಗಳು
ಇಂದ್ರಿಯ-ಗ್ರಹಣದ
ಇಂದ್ರಿಯ-ಗ್ರಹಣ-ವಾ-ಗು-ವುದು
ಇಂದ್ರಿಯ-ಗ್ರಹಣವು
ಇಂದ್ರಿಯ-ಗ್ರಹಣವೂ
ಇಂದ್ರಿಯ-ಜನ್ಮ
ಇಂದ್ರಿ-ಯದ
ಇಂದ್ರಿಯ-ದಲ್ಲಿದೆ
ಇಂದ್ರಿಯ-ದೊಂದಿಗೆ
ಇಂದ್ರಿಯ-ನಿಗ್ರಹ
ಇಂದ್ರಿಯ-ಭೋಗ
ಇಂದ್ರಿಯ-ವನ್ನು
ಇಂದ್ರಿಯ-ವಲ್ಲ
ಇಂದ್ರಿಯ-ವಿದೆ
ಇಂದ್ರಿಯ-ವಿದ್ದರೆ
ಇಂದ್ರಿಯ-ವಿದ್ದಿದ್ದರೆ
ಇಂದ್ರಿ-ಯವು
ಇಂದ್ರಿ-ಯವೂ
ಇಂದ್ರಿಯ-ಶಕ್ತಿ
ಇಂದ್ರಿಯ-ಸಂವೇದ-ನೆಯೂ
ಇಂದ್ರಿಯ-ಸುಖ
ಇಂದ್ರಿಯ-ಸುಖ-ಗಳಲ್ಲೇ
ಇಂದ್ರಿಯ-ಸುಖದ
ಇಂದ್ರಿಯ-ಸುಖ-ಭೋಗ
ಇಂದ್ರಿಯ-ಸುಖ-ಭೋಗ-ಗಳಿಂದ
ಇಂದ್ರಿಯಾ
ಇಂದ್ರಿಯಾ-ತೀತ
ಇಂದ್ರಿಯಾ-ತೀತ-ನಾಗಿ-ರ-ಬೇಕು
ಇಂದ್ರಿಯಾ-ತೀತ-ವಾದ
ಇಂದ್ರಿಯಾ-ತೀತ-ವಾದು-ದನ್ನು
ಇಂದ್ರಿಯಾ-ನು-ಭವ-ಗ-ಳನ್ನು
ಇಂದ್ರಿಯಾ-ನು-ಭವ-ದಲ್ಲೇ
ಇಚ್ಚಿಸಿ
ಇಚ್ಚಿಸಿ-ದಾಗ
ಇಚ್ಚಿಸುತ್ತಾರೆ
ಇಚ್ಚಿ-ಸುತ್ತಿದ್ದರೊ
ಇಚ್ಛಾ
ಇಚ್ಛಾತ-ರಂಗಕ್ಕೆ
ಇಚ್ಛಾನು-ಸಾರ
ಇಚ್ಛಾ-ಪೂರ್ವಕ
ಇಚ್ಛಾ-ಮರ-ಣಿ-ಗಳಾ-ಗಿದ್ದರು
ಇಚ್ಛಾ-ಮಾತ್ರ-ದಿಂದ
ಇಚ್ಛಾ-ಶಕ್ತಿ
ಇಚ್ಛಾ-ಶಕ್ತಿ-ಗ-ಳನ್ನು
ಇಚ್ಛಾ-ಶಕ್ತಿಗೆ
ಇಚ್ಛಾ-ಶಕ್ತಿಯ
ಇಚ್ಛಾ-ಶಕ್ತಿ-ಯನ್ನು
ಇಚ್ಛಾ-ಶಕ್ತಿ-ಯಿಂದ
ಇಚ್ಛಾ-ಶಕ್ತಿಯು
ಇಚ್ಛಾ-ಶಕ್ತಿ-ಯುಳ್ಳ-ವರು
ಇಚ್ಛಾ-ಶಕ್ತಿಯೂ
ಇಚ್ಛಾ-ಶಕ್ತಿಯೆ
ಇಚ್ಛಿಸ-ದ-ವರು
ಇಚ್ಛಿ-ಸದೆ
ಇಚ್ಛಿ-ಸದೇ
ಇಚ್ಛಿಸ-ಬಹುದು
ಇಚ್ಛಿಸ-ಬೇಕು
ಇಚ್ಛಿ-ಸಿದ
ಇಚ್ಛಿಸಿ-ದರು
ಇಚ್ಛಿಸಿ-ದರೂ
ಇಚ್ಛಿ-ಸಿದರೆ
ಇಚ್ಛಿಸಿ-ದಷ್ಟು
ಇಚ್ಛಿಸಿ-ದ್ದನು
ಇಚ್ಛಿಸು-ತ್ತದೆ
ಇಚ್ಛಿಸು-ತ್ತಿದೆ
ಇಚ್ಛಿಸು-ತ್ತಿದ್ದ-ರೆಂಬು-ದನ್ನು
ಇಚ್ಛಿಸು-ತ್ತೇವೆ
ಇಚ್ಛಿ-ಸುವ
ಇಚ್ಛಿಸು-ವನು
ಇಚ್ಛಿಸು-ವನು-ಆದ-ಕಾರಣ
ಇಚ್ಛಿಸು-ವರು
ಇಚ್ಛಿಸು-ವಿರಿ
ಇಚ್ಛಿಸು-ವು-ದಿಲ್ಲ
ಇಚ್ಛಿಸು-ವುದು
ಇಚ್ಛಿಸು-ವುದೂ
ಇಚ್ಛೆ
ಇಚ್ಛೆ-ಗಳೆ-ರಡೂ
ಇಚ್ಛೆಗೆ
ಇಚ್ಛೆ-ಪಟ್ಟರೆ
ಇಚ್ಛೆ-ಪಟ್ಟಾಗ
ಇಚ್ಛೆಯ
ಇಚ್ಛೆ-ಯಂತೆ
ಇಚ್ಛೆ-ಯನ್ನು
ಇಚ್ಛೆ-ಯನ್ನೇ
ಇಚ್ಛೆ-ಯಾ-ಗಲೀ
ಇಚ್ಛೆ-ಯಾಗಿ
ಇಚ್ಛೆ-ಯಾಗುತ್ತದೆ
ಇಚ್ಛೆ-ಯಾಗು-ವುದು
ಇಚ್ಛೆ-ಯಾಗು-ವುದೋ
ಇಚ್ಛೆ-ಯಿಂದ
ಇಚ್ಛೆ-ಯಿಲ್ಲ
ಇಚ್ಛೆ-ಯಿಲ್ಲದೆ
ಇಚ್ಛೆಯು
ಇಚ್ಛೆಯೆ
ಇಚ್ಛೆಯೇ
ಇಟ್ಟ
ಇಟ್ಟಂತೆ
ಇಟ್ಟನು
ಇಟ್ಟ-ಮೇಲೆ
ಇಟ್ಟರು
ಇಟ್ಟರೆ
ಇಟ್ಟ-ರೆಂಬು-ದನ್ನು
ಇಟ್ಟಷ್ಟೂ
ಇಟ್ಟಿದ್ದರು
ಇಟ್ಟಿದ್ದಾರೆ
ಇಟ್ಟಿ-ರ-ಬೇಕು
ಇಟ್ಟಿ-ರು-ವರು
ಇಟ್ಟು
ಇಟ್ಟು-ಕೊಂಡಿ
ಇಟ್ಟು-ಕೊಂಡಿದ್ದ
ಇಟ್ಟು-ಕೊಂಡಿರ-ಬಹು-ದಾ-ಗಿತ್ತು
ಇಟ್ಟು-ಕೊಂಡಿರ-ಬಹುದು
ಇಟ್ಟು-ಕೊಂಡಿರ-ಬೇಕಾದ
ಇಟ್ಟು-ಕೊಂಡಿರ-ಬೇಕು
ಇಟ್ಟು-ಕೊಂಡಿರ-ಬೇಕೆಂದು
ಇಟ್ಟು-ಕೊಂಡಿ-ರಲಿ
ಇಟ್ಟು-ಕೊಂಡಿ-ರಲು
ಇಟ್ಟು-ಕೊಂಡಿರಿ
ಇಟ್ಟು-ಕೊಂಡಿ-ರುವ
ಇಟ್ಟು-ಕೊಂಡಿ-ರು-ವರು
ಇಟ್ಟು-ಕೊಂಡಿ-ರುವುದ
ಇಟ್ಟು-ಕೊಂಡು
ಇಟ್ಟು-ಕೊಳ್ಳ
ಇಟ್ಟು-ಕೊಳ್ಳಲಿ
ಇಟ್ಟು-ಕೊಳ್ಳಿ
ಇಟ್ಟು-ಕೊಳ್ಳುವ
ಇಟ್ಟು-ಕೊಳ್ಳು-ವರು
ಇಟ್ಟು-ಕೊಳ್ಳು-ವು-ದಕ್ಕಾಗಿ
ಇಟ್ಟು-ಕೊಳ್ಳೋಣ
ಇಟ್ಟು-ಕೊಳ್ಳೋಣಸ್ವಲ್ಪ
ಇಡ
ಇಡದ
ಇಡ-ಬಲ್ಲ
ಇಡ-ಬಹುದು
ಇಡ-ಬಹು-ದೆಂದು
ಇಡಲಿ
ಇಡಲಿಕ್ಕಾ-ಗದ
ಇಡಲು
ಇಡಲ್ಪ-ಡು-ವುದು
ಇಡಾ
ಇಡಾದ
ಇಡಿ
ಇಡೀ
ಇಡುತ್ತವೆ
ಇಡುವ
ಇಡುವನು
ಇಡು-ವಿರಿ
ಇಡುವು-ದಕ್ಕಾಗಿ
ಇಡು-ವು-ದಕ್ಕೆ
ಇಡು-ವು-ದನ್ನು
ಇಡು-ವುವು
ಇಡು-ವೆವು
ಇಡೇ
ಇಣಕಿ
ಇತರ
ಇತರರ
ಇತರ-ರ-ದಲ್ಲ
ಇತರ-ರನ್ನು
ಇತರ-ರಲ್ಲಿ
ಇತರ-ರಿ-ಗಲ್ಲ
ಇತರ-ರಿ-ಗಾಗಿ
ಇತರ-ರಿ-ಗಿಲ್ಲ
ಇತರ-ರಿಗೂ
ಇತರ-ರಿಗೆ
ಇತರರು
ಇತರರೂ
ಇತರ-ರೊಂದಿಗೆ
ಇತ-ರೇಷಾಮ್
ಇತಿ
ಇತಿ-ಹಾಸ
ಇತಿ-ಹಾಸದ
ಇತಿ-ಹಾಸ-ದಲ್ಲಿ
ಇತಿ-ಹಾಸ-ದಲ್ಲೆಲ್ಲ
ಇತಿ-ಹಾಸ-ದಲ್ಲೆಲ್ಲಾ
ಇತಿ-ಹಾಸ-ದಿಂದ
ಇತಿ-ಹಾಸವು
ಇತಿ-ಹಾಸ-ವೆಲ್ಲ
ಇತ್ತ
ಇತ್ತ-ಕಡೆ
ಇತ್ತೀಚಿನ
ಇತ್ತು
ಇತ್ತೆಂದು
ಇತ್ತೊ
ಇತ್ತೋ
ಇತ್ಯರ್ಥ
ಇತ್ಯರ್ಥ-ಮಾಡ-ಲಾಗ-ಲಿಲ್ಲ
ಇತ್ಯರ್ಥ-ವಾ-ಗಿಲ್ಲ
ಇತ್ಯಾತ್ಮಕ
ಇತ್ಯಾತ್ಮಕ-ವಾ-ದುದು
ಇತ್ಯಾದಿ
ಇತ್ಯಾದಿ-ಗಳಂತೆ
ಇತ್ಯಾದಿ-ಗಳಿವೆ
ಇತ್ಯಾದಿ-ಯಾಗಿ
ಇದ
ಇದಕ್ಕಾಗಿ
ಇದಕ್ಕಾಗಿಯೇ
ಇದಕ್ಕಿಂತ
ಇದಕ್ಕಿಂತಲೂ
ಇದಕ್ಕಿತ
ಇದಕ್ಕೂ
ಇದಕ್ಕೆ
ಇದಕ್ಕೇನು
ಇದಕ್ಕೋಸುಗ
ಇದನು
ಇದನ್ನು
ಇದನ್ನೂ
ಇದನ್ನೆ
ಇದನ್ನೆಲ್ಲ
ಇದನ್ನೆಲ್ಲಾ
ಇದನ್ನೇ
ಇದನ್ನೇನೊ
ಇದರ
ಇದ-ರಂತೆ
ಇದ-ರಂತೆಯೆ
ಇದ-ರಂತೆಯೇ
ಇದ-ರ-ನಂತರ
ಇದ-ರಲ್ಲಿ
ಇದ-ರಲ್ಲಿದೆ
ಇದ-ರಲ್ಲಿ-ರುವ
ಇದ-ರಲ್ಲೆ
ಇದ-ರಲ್ಲೆಲ್ಲ
ಇದ-ರಲ್ಲೆಲ್ಲಾ
ಇದ-ರಷ್ಟು
ಇದ-ರಾಚೆ
ಇದ-ರಿಂದ
ಇದ-ರಿಂದಲೂ
ಇದ-ರಿಂದಲೇ
ಇದ-ರಿಂದಾ-ಗಿಯೇ
ಇದ-ರೊಂದಿಗೆ
ಇದ-ರೊ-ಡನೆ
ಇದ-ರೊ-ಳಗೆ
ಇದಲ್ಲ
ಇದಲ್ಲದೆ
ಇದಾದ
ಇದಾದ-ಮೇಲೆ
ಇದಾ-ವುದೊ
ಇದಿರಿ-ಸು-ವರು
ಇದಿರಿ-ಸು-ವಷ್ಟು
ಇದಿಲ್ಲದ
ಇದಿಲ್ಲದೆ
ಇದಿಷ್ಟು
ಇದಿಷ್ಟೇ
ಇದು
ಇದು-ವರೆಗೂ
ಇದು-ವರೆಗೆ
ಇದು-ವರೆ-ವಿಗೂ
ಇದು-ವರೆ-ವಿಗೆ
ಇದು-ಸರ್ವವ್ಯಾಪಿ
ಇದು-ಸು-ಗುಣ
ಇದೂ
ಇದೆ
ಇದೆಂದಿಗೂ
ಇದೆ-ಅದೇ
ಇದೆ-ಆ-ದರೂ
ಇದೆನ್ನೇ
ಇದೆಯೆ
ಇದೆ-ಯೆಂತಲೂ
ಇದೆಯೇ
ಇದೆ-ಯೇನು
ಇದೆಯೊ
ಇದೆಯೋ
ಇದೆಲ್ಲ
ಇದೆಲ್ಲಕ್ಕಿಂತ
ಇದೆಲ್ಲ-ವನ್ನು
ಇದೆಲ್ಲ-ವನ್ನೂ
ಇದೆಲ್ಲಾ
ಇದೆಲ್ಲೊ
ಇದೆಷ್ಟು
ಇದೇ
ಇದೇ-ಕಾಲ-ವನ್ನು
ಇದೇಕೆ
ಇದೇ-ತಕ್ಕೆ
ಇದೇ-ನಾ-ದರೂ
ಇದೇನು
ಇದೇನೂ
ಇದೇನೊ
ಇದೇನೋ
ಇದೇ-ಯೇನು
ಇದೊಂದು
ಇದೊಂದೆ
ಇದೊಂದೇ
ಇದ್ದ
ಇದ್ದಂತೆ
ಇದ್ದದ್ದನ್ನು
ಇದ್ದನು
ಇದ್ದರು
ಇದ್ದರೂ
ಇದ್ದರೆ
ಇದ್ದರೇ
ಇದ್ದ-ವರು
ಇದ್ದವು
ಇದ್ದಷ್ಟು
ಇದ್ದಷ್ಟೂ
ಇದ್ದಾಗ
ಇದ್ದಾದ
ಇದ್ದಾನೆ
ಇದ್ದಿತು
ಇದ್ದಿ-ತೆಂದು
ಇದ್ದಿದ್ದರೆ
ಇದ್ದಿ-ರ-ಬಹುದು
ಇದ್ದಿ-ರ-ಬೇಕು
ಇದ್ದಿರ-ಬೇಕೆಂಬು-ದನ್ನು
ಇದ್ದೀರಿ
ಇದ್ದು
ಇದ್ದು-ದಕ್ಕೆ
ಇದ್ದು-ದನ್ನು
ಇದ್ದು-ದನ್ನೆಲ್ಲಾ
ಇದ್ದು-ದ-ರಲ್ಲಿ
ಇದ್ದು-ದ-ರಿಂದ
ಇದ್ದುದು
ಇದ್ದುದೇ
ಇದ್ದು-ದೇನು
ಇದ್ದುವು
ಇದ್ದೆ
ಇದ್ದೆವು
ಇದ್ದೇ
ಇದ್ದೇವೆ
ಇದ್ಧಾನೆ
ಇನ್ನರ್ಧ
ಇನ್ನಾ-ವುದೊ
ಇನ್ನಿ-ತರ
ಇನ್ನು
ಇನ್ನು-ಮುಂದೆ
ಇನ್ನು-ಮೇಲೆ
ಇನ್ನೂ
ಇನ್ನೂರು
ಇನ್ನೆಂದಿಗೂ
ಇನ್ನೆಲ್ಲಿಯ
ಇನ್ನೆಲ್ಲೊ
ಇನ್ನೇನು
ಇನ್ನೇನೂ
ಇನ್ನೇನೊ
ಇನ್ನೊಂದರ
ಇನ್ನೊಂದಾ-ಗು-ವುದು
ಇನ್ನೊಂದು
ಇನ್ನೊಬ್ಬ
ಇನ್ನೊಬ್ಬನ
ಇನ್ನೊಬ್ಬ-ನನ್ನು
ಇನ್ನೊಬ್ಬ-ನಿಂದ
ಇನ್ನೊಬ್ಬ-ನಿಗೆ
ಇನ್ನೊಬ್ಬನು
ಇನ್ನೊಬ್ಬರ
ಇನ್ನೊಬ್ಬ-ರನ್ನು
ಇನ್ನೊಬ್ಬ-ರಲ್ಲಿ
ಇನ್ನೊಬ್ಬ-ರಿಂದ
ಇನ್ನೊಬ್ಬ-ರಿ-ಗಾಗಿ
ಇನ್ನೊಬ್ಬ-ರಿಗೆ
ಇನ್ನೊಬ್ಬರು
ಇಪ್ಪತ್ತ-ನಾಲ್ಕು
ಇಪ್ಪತ್ತು
ಇಪ್ಪತ್ತೈದು
ಇಪ್ಪುತ್ತು
ಇಬ್ಬರ
ಇಬ್ಬ-ರನ್ನೂ
ಇಬ್ಬರು
ಇಬ್ಬರೂ
ಇಬ್ಲಿಸ್ಸನು
ಇಮಿಟೇಷನ್
ಇರ
ಇರ-ಕೂಡದು
ಇರದಿದ್ದ
ಇರ-ದಿದ್ದರೆ
ಇರದು
ಇರದೆ
ಇರದೇ
ಇರ-ಬಲ್ಲ
ಇರ-ಬಲ್ಲದು
ಇರ-ಬಲ್ಲುದು
ಇರ-ಬಲ್ಲೆವು
ಇರ-ಬಹುದು
ಇರ-ಬಹು-ದುಜ್ಯೋತಿಯ
ಇರ-ಬಹು-ದೆಂದೂ
ಇರ-ಬಾ-ರದು
ಇರ-ಬೇಕಾ-ಗಿತ್ತು
ಇರ-ಬೇಕಾ-ಗಿದೆ
ಇರ-ಬೇಕಾ-ಗಿಲ್ಲ
ಇರ-ಬೇಕಾ-ಗುತ್ತಿತ್ತು
ಇರ-ಬೇಕಾ-ಗು-ವುದು
ಇರ-ಬೇಕಾದ
ಇರ-ಬೇಕಾ-ದರೆ
ಇರ-ಬೇಕಾ-ದು-ದಕ್ಕಿಂತ
ಇರ-ಬೇಕಾ-ಯಿತು
ಇರ-ಬೇಕು
ಇರ-ಬೇಕೆಂದು
ಇರ-ಲಾರ
ಇರ-ಲಾರದು
ಇರ-ಲಾರವು
ಇರ-ಲಾರಿರಿ
ಇರ-ಲಾರೆ
ಇರಲಿ
ಇರ-ಲಿಲ್ಲ
ಇರ-ಲಿಲ್ಲ-ವೆಂದಲ್ಲ
ಇರ-ಲಿಲ್ಲ-ವೆಂಬು-ದನ್ನು
ಇರಲು
ಇರಲೇ
ಇರಲೇ-ಬೇಕಾ-ಗು-ವುದು
ಇರಲೇ-ಬೇಕು
ಇರ-ವನ್ನೆ
ಇರ-ವನ್ನೇ
ಇರವು
ಇರವೆ
ಇರವೇ
ಇರಿ
ಇರಿ-ಸ-ಲಾ-ಗದೆ
ಇರಿಸು
ಇರು
ಇರುತ್ತದೆ
ಇರುತ್ತವೆ
ಇರುತ್ತಾನೆ
ಇರುತ್ತಾರೆ
ಇರುತ್ತಿ
ಇರುತ್ತಿತ್ತು
ಇರುತ್ತಿದ್ದವು
ಇರುತ್ತಿ-ರ-ಲಿಲ್ಲ
ಇರುತ್ತೇವೆ
ಇರುವ
ಇರು-ವಂತೆ
ಇರು-ವಂಥದು
ಇರುವ-ತನಕ
ಇರುವದು
ಇರುವನು
ಇರುವನೆ
ಇರುವ-ನೆಂದು
ಇರುವ-ನೆಂಬುದು
ಇರುವನೋ
ಇರು-ವರು
ಇರುವ-ವ-ನಿಗೂ
ಇರುವ-ವನು
ಇರುವ-ವನೆ
ಇರುವ-ವ-ನೆಂದು
ಇರುವ-ವರ
ಇರುವ-ವ-ರಿಗೆ
ಇರುವ-ವರಿದ್ದಾರೆ
ಇರು-ವ-ವರೆಗೆ
ಇರು-ವಾಗ
ಇರು-ವಾ-ಗಲೂ
ಇರು-ವಾ-ಗಲೇ
ಇರು-ವಿಕೆ
ಇರು-ವಿಕೆ-ಯನ್ನು
ಇರು-ವಿರಿ
ಇರು-ವು-ದಕ್ಕಿಂತಲೂ
ಇರು-ವು-ದಕ್ಕೆ
ಇರು-ವು-ದಕ್ಕೆಲ್ಲ
ಇರು-ವು-ದನ್ನು
ಇರು-ವುದನ್ನೇ
ಇರು-ವು-ದ-ರಲ್ಲಿ
ಇರು-ವು-ದ-ರಿಂದ
ಇರುವು-ದಾದರೆ
ಇರು-ವು-ದಿಲ್ಲ
ಇರು-ವು-ದಿಲ್ಲವೋ
ಇರು-ವುದು
ಇರು-ವು-ದೆಂದು
ಇರುವು-ದೆಂಬ
ಇರುವು-ದೆಲ್ಲ
ಇರು-ವುದೇ
ಇರುವುದೊ
ಇರುವು-ದೊಂದು
ಇರು-ವುದೊಂದೆ
ಇರುವು-ದೊಂದೇ
ಇರುವುದೋ
ಇರು-ವುವು
ಇರುವೆ
ಇರುವೆ-ಗಿಂತ
ಇರು-ವೆನು
ಇರು-ವೆನೊ
ಇರು-ವೆವು
ಇರು-ವೆವೋ
ಇರೋಣ-ವೆಂದು
ಇಲಸ್ಟ್ರೇಟೆಡ್
ಇಲ್ಲ
ಇಲ್ಲಈ
ಇಲ್ಲದ
ಇಲ್ಲ-ದಂತೆ
ಇಲ್ಲ-ದಂತೆಯೆ
ಇಲ್ಲ-ದರೊ
ಇಲ್ಲ-ದ-ವನು
ಇಲ್ಲ-ದಾಂತಾಗ-ಬೇಕು
ಇಲ್ಲ-ದಾಗ
ಇಲ್ಲ-ದಿದ್ದರೂ
ಇಲ್ಲ-ದಿದ್ದರೆ
ಇಲ್ಲ-ದಿ-ರಲಿ
ಇಲ್ಲ-ದಿ-ರು-ವುದು
ಇಲ್ಲ-ದು-ದ-ರಿಂದ
ಇಲ್ಲದೆ
ಇಲ್ಲದೇ
ಇಲ್ಲ-ನಮ್ಮ
ಇಲ್ಲ-ಯಾವ
ಇಲ್ಲ-ವಾ-ದರೆ
ಇಲ್ಲ-ವಾ-ದಾಗ
ಇಲ್ಲವೆ
ಇಲ್ಲ-ವೆಂತಲೂ
ಇಲ್ಲ-ವೆಂದಲ್ಲ
ಇಲ್ಲ-ವೆಂದು
ಇಲ್ಲ-ವೆಂದೂ
ಇಲ್ಲ-ವೆಂಬು-ದನ್ನು
ಇಲ್ಲ-ವೆಂಬುದು
ಇಲ್ಲ-ವೆನ್ನ-ಬಹುದು
ಇಲ್ಲ-ವೆನ್ನು-ವರು
ಇಲ್ಲ-ವೆನ್ನು-ವು-ದಕ್ಕೆ
ಇಲ್ಲ-ವೆನ್ನು-ವು-ದಿಲ್ಲ
ಇಲ್ಲ-ವೆನ್ನು-ವುದು
ಇಲ್ಲವೇ
ಇಲ್ಲವೊ
ಇಲ್ಲವೋ
ಇಲ್ಲಿ
ಇಲ್ಲಿಂದ
ಇಲ್ಲಿ-ಗಿಂತ
ಇಲ್ಲಿಗೆ
ಇಲ್ಲಿಗೇ
ಇಲ್ಲಿದೆ
ಇಲ್ಲಿದೆಯೊ
ಇಲ್ಲಿನ
ಇಲ್ಲಿ-ನಂತೆ
ಇಲ್ಲಿಯ
ಇಲ್ಲಿ-ಯ-ವರೆಗೂ
ಇಲ್ಲಿ-ಯ-ವರೆಗೆ
ಇಲ್ಲಿ-ಯ-ವರೆ-ವಿಗೂ
ಇಲ್ಲಿಯೂ
ಇಲ್ಲಿಯೇ
ಇಲ್ಲಿ-ರ-ಬೇಕು
ಇಲ್ಲಿ-ರುವ
ಇಲ್ಲಿ-ರುವಂತೆಯೇ
ಇಲ್ಲಿ-ರು-ವರು
ಇಲ್ಲಿ-ರುವ-ವ-ರೆಲ್ಲ
ಇಲ್ಲಿ-ರು-ವಿರಿ
ಇಲ್ಲಿ-ರುವುದು
ಇಲ್ಲಿ-ರುವುದೇ
ಇಲ್ಲಿವೆ
ಇಲ್ಲೆ
ಇಲ್ಲೆಲ್ಲ
ಇಲ್ಲೇ
ಇಲ್ಲೊಂದು
ಇಲ್ಲೊಬ್ಬ
ಇಲ್ಲೋ
ಇಳಿ-ತಕ್ಕೂ
ಇಳಿತ-ವಿದೆ
ಇಳಿದ
ಇಳಿ-ದಿರು
ಇಳಿ-ದಿರು-ವೆವು-ದೇಹದ
ಇಳಿದು
ಇಳಿ-ಮುಖ-ವಾಗಿ
ಇಳಿಯ-ಬಹುದು
ಇಳಿಯ-ಬೇಕು
ಇಳಿ-ಯಲಿ
ಇಳಿಯುತ್ತದೆ
ಇಳಿಯು-ವನು
ಇಳಿಯು-ವರು
ಇಳಿಯು-ವಿರಿ
ಇಳಿಯು-ವುದು
ಇಳಿಯು-ವೆನು
ಇಳಿಸ-ಕೂಡದು
ಇಳಿಸ-ಬೇಕಾ-ಗಿಲ್ಲ
ಇಳಿಸಿ-ದರೆ
ಇಳಿಸಿ-ರು-ವರು
ಇಳಿ-ಸು-ವುದರ
ಇವಕ್ಕಿಂತ
ಇವಕ್ಕೂ
ಇವಕ್ಕೆ
ಇವಕ್ಕೆಲ್ಲ
ಇವನ
ಇವ-ನನ್ನು
ಇವನಿ-ಗಿಂತ
ಇವ-ನಿಗೆ
ಇವನು
ಇವನೇ
ಇವನ್ನ
ಇವನ್ನು
ಇವನ್ನೆ
ಇವನ್ನೆಲ್ಲ
ಇವನ್ನೆಲ್ಲಾ
ಇವನ್ನೇ
ಇವರ
ಇವ-ರನ್ನು
ಇವ-ರಷ್ಟು
ಇವ-ರಿಂದ
ಇವ-ರಿಗೆ
ಇವರಿಬ್ಬರ
ಇವ-ರಿಬ್ಬರೂ
ಇವರು
ಇವರೆಂದಿಗೂ
ಇವರೆಡೂ
ಇವ-ರೆಲ್ಲ
ಇವ-ರೆಲ್ಲ-ರಿಗೂ
ಇವರೇ
ಇವಾ
ಇವಾ-ವು-ದನ್ನೂ
ಇವಾ-ವುದೂ
ಇವಾ-ವುವೂ
ಇವಿಷ್ಟೇ
ಇವು
ಇವು-ಗಳ
ಇವು-ಗಳಂತೆ
ಇವು-ಗ-ಳನ್ನು
ಇವು-ಗಳನ್ನೆಲ್ಲ
ಇವು-ಗಳನ್ನೆಲ್ಲಾ
ಇವು-ಗಳನ್ನೊಳ-ಗೊಂಡ
ಇವು-ಗಳಲ್ಲಿ
ಇವು-ಗಳಲ್ಲಿತ್ತು
ಇವು-ಗಳಲ್ಲಿಯೂ
ಇವು-ಗಳಲ್ಲೆಲ್ಲ
ಇವು-ಗಳಲ್ಲೆಲ್ಲಾ
ಇವು-ಗಳಾ-ವು-ದಕ್ಕೂ
ಇವು-ಗಳಾ-ವು-ದನ್ನೂ
ಇವು-ಗಳಾ-ವುವೂ
ಇವು-ಗಳಿಂದ
ಇವು-ಗಳಿಂದಲೂ
ಇವು-ಗಳಿಂದಾಗಿ
ಇವು-ಗಳಿ-ಗಿಂತ
ಇವು-ಗಳಿಗೆ
ಇವು-ಗಳಿ-ಗೆಲ್ಲ
ಇವು-ಗಳಿ-ಗೆಲ್ಲಾ
ಇವು-ಗಳಿ-ರು-ವುದು
ಇವು-ಗಳಿಲ್ಲದೆ
ಇವು-ಗಳು
ಇವು-ಗಳೂ
ಇವು-ಗಳೆ
ಇವು-ಗಳೆ-ರ-ಡನ್ನೂ
ಇವು-ಗಳೆ-ರಡರ
ಇವು-ಗಳೆಲ್ಲ
ಇವು-ಗಳೆಲ್ಲ-ದರ
ಇವು-ಗಳೆಲ್ಲ-ವನ್ನು
ಇವು-ಗಳೆಲ್ಲ-ವನ್ನೂ
ಇವು-ಗಳೆಲ್ಲವೂ
ಇವು-ಗಳೇ
ಇವು-ಗಳೇ-ನೆಂಬುದು
ಇವು-ಗಳೊಂದಿಗೆ
ಇವೆ
ಇವೆಯೆ
ಇವೆ-ಯೆಂದು
ಇವೆಯೊ
ಇವೆ-ರಡಕ್ಕೂ
ಇವೆ-ರ-ಡನ್ನೂ
ಇವೆ-ರಡರ
ಇವೆ-ರಡ-ರಲ್ಲಿ
ಇವೆ-ರಡ-ರೊ-ಡನೆ
ಇವೆ-ರಡು
ಇವೆ-ರಡೂ
ಇವೆಲ್ಲ
ಇವೆಲ್ಲ-ದರ
ಇವೆಲ್ಲ-ದರಲ್ಲಿಯೂ
ಇವೆಲ್ಲ-ವನ್ನೂ
ಇವೆಲ್ಲ-ವು-ಗಳಲ್ಲಿಯೂ
ಇವೆಲ್ಲವೂ
ಇವೆಲ್ಲಾ
ಇವೇ
ಇವೇಂದ್ರಿ-ಯಾಣಾಂ
ಇವೇನು
ಇಷ್ಟ
ಇಷ್ಟ-ದೇವ-ತಾ-ಸಿದ್ಧಿ-ಯಾ-ಗು-ವುದು
ಇಷ್ಟ-ದೇ-ವತೆ
ಇಷ್ಟ-ವಾಗು-ವು-ದಿಲ್ಲ
ಇಷ್ಟ-ವಾದ
ಇಷ್ಟ-ವಿಲ್ಲ
ಇಷ್ಟಾ-ದರೂ
ಇಷ್ಟು
ಇಷ್ಟೂ
ಇಷ್ಟೆ
ಇಷ್ಟೇ
ಇಷ್ಟೊಂದು
ಇಸವಿ
ಇಸ್ಲಾಂ
ಇಹ-ಪರ-ಗಳೆ-ರಡೂ
ಇಹ-ಲೋಕದ
ಈ
ಈಕೆಳ
ಈಕ್ಷಿಸಿ-ದಾಗ
ಈಗ
ಈಗ-ತಾನೆ
ಈಗಲೂ
ಈಗಲೆ
ಈಗಲೋ
ಈಗಾ-ಗಲಿ
ಈಗಾ-ಗಲೆ
ಈಗಾ-ಗಲೇ
ಈಗಿ
ಈಗಿನ
ಈಗಿ-ನ-ವ-ರಲ್ಲಿ
ಈಗಿ-ನ-ವರೆಗೂ
ಈಗಿ-ನಷ್ಟು
ಈಗಿ-ರುವ
ಈಗಿ-ರು-ವಂತೆ
ಈಗಿ-ರು-ವು-ದಕ್ಕಿಂತ
ಈಗಿ-ರು-ವು-ದನ್ನು
ಈಗಿ-ರುವುದು
ಈಗೊಂದು
ಈಚಿನ
ಈಚಿ-ನದು
ಈಜ-ಬಲ್ಲದು
ಈಜಿಪ್ಟಿನ
ಈಜಿಪ್ಟಿನ-ವರು
ಈಜಿಪ್ಟ್
ಈಜಿ-ರ-ಲಿಲ್ಲ
ಈಜು
ಈಜು-ವು-ದನ್ನು
ಈಡಾ
ಈಡಾ-ಗ-ಬಾ-ರದು
ಈಡಾ-ಗಿ-ರು-ವೆವು
ಈಡಾ-ಗು-ವೆವು
ಈಡಾ-ದಂತೆ
ಈಡು-ಮಾಡು-ವುದು
ಈಡೇ-ರದ
ಈಡೇರಿ-ದರೆ
ಈಡೇ-ರಿಸಿ-ಕೊಳ್ಳು-ವೆನು
ಈಡೇರಿ-ಸುವ
ಈಡೇರು-ವು-ದಿಲ್ಲ
ಈಡೇ-ರು-ವುವು
ಈತ
ಈತನಿ-ಗಿಂತ
ಈತನು
ಈತನೇ
ಈಥರ್
ಈಯು-ವು-ದಲ್ಲ
ಈಶ್ವರ
ಈಶ್ವರಃ
ಈಶ್ವರನ
ಈಶ್ವರ-ನನ್ನು
ಈಶ್ವರ-ನಲ್ಲಿ
ಈಶ್ವರ-ನಿಂದನೆ
ಈಶ್ವರ-ನಿಂದಲೇ
ಈಶ್ವರ-ನಿಂದೆ
ಈಶ್ವರ-ನಿಗೆ
ಈಶ್ವರನು
ಈಶ್ವರ-ನೆಂದು
ಈಶ್ವರ-ನೆ-ಡೆಗೆ
ಈಶ್ವರನೇ
ಈಶ್ವ-ರನ್ನು
ಈಶ್ವರಪ್ರಣಿ-ಧಾನ
ಈಶ್ವರಪ್ರಣಿ-ಧಾನ-ದಿಂದ
ಈಶ್ವರಪ್ರಣಿ-ಧಾನಾಧ್ಯಾ
ಈಶ್ವರ-ಸಿದ್ಧಾಂತ
ಈಶ್ವರಾರ್ಪಣ
ಈಶ್ವರೋ
ಉ
ಉಂಟಾಗ-ದಂತೆ
ಉಂಟಾಗ-ಬಹುದು
ಉಂಟಾಗಿ
ಉಂಟಾಗಿ-ದೆಯೋ
ಉಂಟಾಗಿದ್ದರೆ
ಉಂಟಾ-ಗಿವೆ
ಉಂಟಾ-ಗುತ್ತದೆ
ಉಂಟಾ-ಗುವ
ಉಂಟಾಗು-ವಂತೆ
ಉಂಟಾಗು-ವುದು
ಉಂಟಾಗು-ವುದು-ಆಗ
ಉಂಟಾಗು-ವುವು
ಉಂಟಾದ
ಉಂಟಾ-ದಂತೆ
ಉಂಟಾ-ದರೆ
ಉಂಟು
ಉಂಟು-ಮಾಡದೆ
ಉಂಟು-ಮಾಡ-ಬಲ್ಲದು
ಉಂಟು-ಮಾಡ-ಬಲ್ಲವು
ಉಂಟು-ಮಾಡ-ಬಹುದು
ಉಂಟು-ಮಾಡ-ಲಾರದು
ಉಂಟು-ಮಾಡ-ಲಾರವು
ಉಂಟು-ಮಾ-ಡಲು
ಉಂಟು-ಮಾಡಿದ
ಉಂಟು-ಮಾಡಿ-ದಾಗ
ಉಂಟು-ಮಾಡುತ್ತದೆ
ಉಂಟು-ಮಾಡುತ್ತ-ದೆ-ನೀವು
ಉಂಟು-ಮಾಡುವ
ಉಂಟು-ಮಾಡು-ವು-ದಕ್ಕಾಗಿ
ಉಂಟು-ಮಾಡು-ವು-ದಿಲ್ಲ
ಉಂಟು-ಮಾಡು-ವುದು
ಉಂಟು-ಮಾಡು-ವುವು
ಉಕ್ಕಿ
ಉಕ್ಕೇ-ರು-ವುದು
ಉಗುರು
ಉಗ್ರ-ವಾದ
ಉಗ್ರಾಣ
ಉಗ್ರಾಣ-ದಂತೆ
ಉಚರಿ
ಉಚ್ಚ
ಉಚ್ಚ-ಮಾನ-ವರು
ಉಚ್ಚ-ರಿಸಿ
ಉಚ್ಚ-ರಿಸಿದ
ಉಚ್ಚ-ರಿಸಿ-ದಂತೆ
ಉಚ್ಚ-ರಿಸಿ-ದರೆ
ಉಚ್ಚ-ರಿ-ಸುತ್ತಿದ್ದ-ನೆಂದು
ಉಚ್ಚ-ರಿಸುತ್ತೇನೆ
ಉಚ್ಚ-ರಿ-ಸುತ್ತೇವೆ
ಉಚ್ಚ-ರಿ-ಸು-ವು-ದಕ್ಕೆ
ಉಚ್ಚ-ರಿ-ಸು-ವೆವು
ಉಚ್ಚ-ವರ್ಗಕ್ಕೆ
ಉಚ್ಛ
ಉಚ್ಛ-ರಿಸು
ಉಚ್ಛ-ವರ್ಗದ
ಉಚ್ಛಸ್ಥಿತಿ-ಯಲ್ಲಿದ್ದೆ-ವೆಂಬು-ದನ್ನು
ಉಚ್ಛ್ರಾಯ
ಉಚ್ಛ್ರಾಯಸ್ಥಿತಿ-ಯಿಂದ
ಉಚ್ಛ್ವಾಸ
ಉಚ್ಛ್ವಾಸ-ನಿಃಶ್ವಾಸ-ಗಳ
ಉಚ್ಛ್ವಾಸ-ನಿಃಶ್ವಾಸ-ಗಳೊಂದಿಗೆ
ಉಜ್ಜಿ
ಉಜ್ವಲ
ಉಜ್ವಲ-ವಾದ
ಉಟ್ಟು-ಕೊಳ್ಳ-ಬೇಕಾ-ಗು-ವುದು
ಉಡುಗಿ-ಹೋ-ದಾಗ
ಉಡು-ಗು-ವುದು
ಉಡುಪಿ-ನಲ್ಲಿ
ಉಡುಪು-ಗ-ಳನ್ನು
ಉಡು-ವು-ದಕ್ಕೆ
ಉಣ್ಣ-ಬಹುದು
ಉಣ್ಣ-ಬಾ-ರದು
ಉತ್ಕಟ-ವಾ-ಗಿ-ರುವುದು
ಉತ್ಕ್ರಾಂತಿಶ್ಚ
ಉತ್ತಮ
ಉತ್ತಮ-ಗೊಂಡಿವೆ
ಉತ್ತಮ-ತರ
ಉತ್ತಮ-ತರದ
ಉತ್ತಮ-ನಾಗ-ಕೂಡದು
ಉತ್ತಮ-ನಾಗ-ಬೇಕಾ-ದರೆ
ಉತ್ತಮ-ನಾ-ದರೆ
ಉತ್ತಮ-ನೀವು
ಉತ್ತಮ-ಪದ
ಉತ್ತಮ-ರನ್ನಾಗಿ
ಉತ್ತಮ-ರಾಗ-ಬಹುದು
ಉತ್ತಮ-ರಾ-ಗಲು
ಉತ್ತಮ-ರಾಗಿ
ಉತ್ತಮ-ರಾ-ಗುತ್ತಾರೆ
ಉತ್ತಮ-ರೀತಿ-ಯಲ್ಲಿ
ಉತ್ತ-ಮರು
ಉತ್ತಮ-ವಲ್ಲ
ಉತ್ತಮ-ವಾಗ-ಬಹುದು
ಉತ್ತಮ-ವಾಗ-ಲಿಲ್ಲ
ಉತ್ತಮ-ವಾಗಿ
ಉತ್ತಮ-ವಾ-ಗಿತ್ತೊ
ಉತ್ತಮ-ವಾಗಿ-ರುವ
ಉತ್ತಮ-ವಾ-ಗು-ವುದು
ಉತ್ತಮ-ವಾದ
ಉತ್ತಮ-ವಾದು-ದನ್ನು
ಉತ್ತಮ-ವಾದುದು
ಉತ್ತಮ-ವಾದುದೆ
ಉತ್ತಮ-ವಾದು-ದೊಂದು
ಉತ್ತಮ-ವಾಯಿ-ತೇನು
ಉತ್ತಮವೊ
ಉತ್ತಮಸ್ಥಿತಿ
ಉತ್ತಮೋತ್ತಮ
ಉತ್ತಮೋತ್ತಮವಾ
ಉತ್ತರ
ಉತ್ತರ-ಕೊಟ್ಟರು
ಉತ್ತರಕ್ಕಾಗಿ
ಉತ್ತರ-ಗಳ
ಉತ್ತರ-ಗ-ಳನ್ನು
ಉತ್ತರ-ಗಳಿಂದ
ಉತ್ತರ-ಗಳೂ
ಉತ್ತರದ
ಉತ್ತರ-ದಲ್ಲಿ
ಉತ್ತರ-ದಲ್ಲಿಯೂ
ಉತ್ತರ-ಮೊದ-ಲ-ನೆ-ಯ-ದಾಗಿ
ಉತ್ತರ-ವನ್ನು
ಉತ್ತರ-ವಲ್ಲ
ಉತ್ತರ-ವಾಗಿ
ಉತ್ತರ-ವಾಗಿದೆ
ಉತ್ತರ-ವಿತ್ತಳು
ಉತ್ತ-ರವೂ
ಉತ್ತ-ರವೇ
ಉತ್ತ-ರಾಯ-ಣಕ್ಕೆ
ಉತ್ತುಂಗ
ಉತ್ತೇ-ಜನ
ಉತ್ತೇ-ಜನ-ಕಾರಿ
ಉತ್ತೇ-ಜನ-ಕಾರಿ-ಯಾದ
ಉತ್ತೇಜಿ-ಸು-ವಂಥದು
ಉತ್ಪತ್ತಿ
ಉತ್ಪತ್ತಿ-ಮಾಡಿ
ಉತ್ಪತ್ತಿ-ಯಾಗಿರು
ಉತ್ಪನ್ನ-ವಾಗಿವೆ
ಉತ್ಪನ್ನ-ವಾ-ದುದು
ಉತ್ಪಾದಿ-ಸ-ಲಾರೆವು
ಉತ್ಪ್ರೇಕ್ಷೆ-ಯಿಂದ
ಉತ್ಸಾಹ
ಉತ್ಸಾ-ಹಕ್ಕೆ
ಉತ್ಸಾಹ-ದಿಂದ
ಉತ್ಸಾಹ-ಭರಿತ-ವಾದ
ಉತ್ಸಾಹ-ವನ್ನು
ಉತ್ಸಾಹ-ವನ್ನೂ
ಉತ್ಸಾಹ-ವಿದೆ
ಉತ್ಸಾಹ-ವಿ-ರುವ
ಉತ್ಸಾಹ-ಶಾಲಿ-ಯಾದ
ಉದಯಾಸ್ತ-ಗಳು
ಉದಯಿ
ಉದಯಿ-ಸ-ಲಿಲ್ಲ
ಉದ-ಯಿಸಿ
ಉದ-ಯಿ-ಸಿತು
ಉದ-ಯಿಸಿದ
ಉದ-ಯಿಸಿ-ದವು
ಉದ-ಯಿಸು
ಉದಯಿ-ಸುತ್ತವೆ
ಉದಯಿ-ಸು-ವಂತೆ
ಉದಯಿ-ಸು-ವು-ದಿಲ್ಲ
ಉದಯಿ-ಸು-ವುದು
ಉದಾ
ಉದಾತ್ತ
ಉದಾತ್ತ-ತತ್ತ್ವ-ವನ್ನು
ಉದಾತ್ತ-ಭಾವ-ನೆ-ಗಳಿಂದ
ಉದಾ-ದಾನ
ಉದಾನ
ಉದಾ-ನ-ಜಯಾಜ್ಜಲಪಂಕಕಂಟಕಾದಿಷ್ಟ-ಸಂಗ
ಉದಾನ-ವೆಂಬ
ಉದಾರ
ಉದಾ-ರ-ತೆ-ಯನ್ನು
ಉದಾ-ರ-ವಾಗಿ
ಉದಾ-ರ-ವಾಗಿ-ರ-ಬೇಕು
ಉದಾರಿ-ಗಳಾ-ಗಿರಿ
ಉದಾ-ಹರಣೆ
ಉದಾ-ಹರ-ಣೆ-ಗ-ಳನ್ನು
ಉದಾ-ಹರ-ಣೆ-ಗಾಗಿ
ಉದಾ-ಹರ-ಣೆಗೆ
ಉದಾ-ಹರ-ಣೆಯ
ಉದಾ-ಹರ-ಣೆ-ಯನ್ನು
ಉದಾ-ಹರ-ಣೆ-ಯಾದ
ಉದಾ-ಹರ-ಣೆ-ಯಿಂದ
ಉದಾ-ಹರ-ಣೆಯು
ಉದಾ-ಹರಿ-ತ-ವಾಗುತ್ತಿದೆ
ಉದಾ-ಹ-ರಿಸಿ-ದರೆ
ಉದಾ-ಹ-ರಿಸಿ-ರು-ವರು
ಉದಾ-ಹರಿ-ಸುತ್ತದೆ
ಉದಾ-ಹ-ರಿಸುತ್ತೇನೆ
ಉದಿಸಿ-ದಾಗ
ಉದಿಸುತ್ತಿತ್ತು
ಉದುರಿ-ಹೋ-ಗು-ವುದು
ಉದ್ದ
ಉದ್ದ-ಅಗಲ-ಗಾತ್ರ-ಗಳಿಲ್ಲದ
ಉದ್ದಕ್ಕೂ
ಉದ್ದೀಪನ
ಉದ್ದೀಪನ-ಗೊಳಿ-ಸುತ್ತವೆ
ಉದ್ದೇಶ
ಉದ್ದೇಶ-ಗ-ಳನ್ನು
ಉದ್ದೇಶದ
ಉದ್ದೇಶ-ದಿಂದ
ಉದ್ದೇಶ-ದಿಂದಲೇ
ಉದ್ದೇಶ-ವನ್ನು
ಉದ್ದೇಶ-ವನ್ನೂ
ಉದ್ದೇಶ-ವನ್ನೇ
ಉದ್ದೇಶವೂ
ಉದ್ದೇಶವೇ
ಉದ್ದೇಶ-ಹೀನ-ವಾ-ಗು-ವುದು
ಉದ್ದೇಶಿಸಿ
ಉದ್ದೇಶಿಸಿದ
ಉದ್ದೇಶಿಸುತ್ತಿಲ್ಲ
ಉದ್ದೇಶಿ-ಸುವ
ಉದ್ಧರಿಸ-ಬಲ್ಲ
ಉದ್ಧ-ರಿಸಿ-ದನು
ಉದ್ಧಾರ
ಉದ್ಧಾ-ರಕ್ಕೆ
ಉದ್ಧಾರ-ಮಾಡು-ವು-ದಿಲ್ಲ
ಉದ್ಧಾರ-ವಾಗ-ಬೇಕಾದ
ಉದ್ಧಾರ-ವಾಗು-ವರು
ಉದ್ಧಾರವೇ
ಉದ್ಭವಿ-ಸಿ-ದವೊ
ಉದ್ಭವಿಸು
ಉದ್ಭವಿ-ಸುತ್ತದೆ
ಉದ್ಯಮ-ವಿದೆ
ಉದ್ಯಾನ-ವನ-ಗಳು
ಉದ್ಯಾನ-ವನ-ವಿರು-ವು-ದಿಲ್ಲ
ಉದ್ರೇಕ-ಗೊಳಿ-ಸು-ವುದರ
ಉದ್ರೇಕ-ವಾಗು-ವಂತೆ
ಉದ್ರೇಕಿಸುತ್ತಾನೆ
ಉದ್ವಿಗ್ನ-ರನ್ನಾಗಿ
ಉದ್ವೇಗ
ಉದ್ವೇಗ-ಗ-ಳನ್ನು
ಉದ್ವೇಗ-ಗಳಿಂದ
ಉದ್ವೇಗ-ಜೀವಿ
ಉದ್ವೇಗ-ದಲ್ಲಿ
ಉದ್ವೇಗ-ಪರ-ರಾಗಿದ್ದರು
ಉದ್ವೇಗ-ಪರ-ವಶ-ವಾ-ಗು-ವುದು
ಉದ್ವೇಗ-ಯುಕ್ತ-ವಾದ
ಉದ್ವೇಗ-ವನ್ನು
ಉನ್ನತ
ಉನ್ನತ-ತರ
ಉನ್ನತ-ದಿಂದ
ಉನ್ನತ-ವಾಗಿ-ರ-ಬೇಕು
ಉನ್ನತ-ವಾದ
ಉನ್ನ-ತಾ-ವಸ್ಥೆ-ಯೊಂದು
ಉನ್ನತಿ
ಉನ್ನ-ತಿಗೆ
ಉನ್ನತಿ-ಯನ್ನು
ಉನ್ಮತ್ತ
ಉನ್ಮತ್ತ-ನಾಗು-ವೆನು
ಉನ್ಮತ್ತ-ರಾಗ-ಬೇಕು
ಉನ್ಮತ್ತ-ರಾಗಿ
ಉಪ
ಉಪ-ಕರಣ
ಉಪ-ಕರ-ಣಕ್ಕಿಂತ
ಉಪ-ಕರಣ-ಗಳು
ಉಪ-ಕರ-ಣದ
ಉಪ-ಕರಣ-ವಾದ
ಉಪ-ಕರ-ಣವೂ
ಉಪ-ಕಾರ
ಉಪ-ಕಾರಕ್ಕೋಸುಗ-ವಾಗಿಯೂ
ಉಪ-ಕಾರ-ಮಾಡಿ
ಉಪ-ಕಾರಿ-ಗಳು
ಉಪ-ಕಾರಿಯಾ
ಉಪ-ಕಾರಿಯೋ
ಉಪ-ಕೋಸಲ
ಉಪ-ಕೋಸಲ-ನಿಗೆ
ಉಪ-ಕ್ರಮಿ-ಸಿದ
ಉಪ-ಕ್ರಮಿ-ಸಿದೆ
ಉಪ-ಕ್ರಮಿಸು-ವನು
ಉಪಟಳಕ್ಕೆ
ಉಪ-ದೇಶ
ಉಪ-ದೇಶ-ಗಳನ್ನು
ಉಪ-ದೇಶದ
ಉಪ-ದೇಶ-ದಂತೆಯೇ
ಉಪ-ದೇಶ-ದಲ್ಲಿ
ಉಪ-ದೇಶ-ದಷ್ಟು
ಉಪ-ದೇಶ-ವನ್ನು
ಉಪ-ದೇಶ-ವೆಂದರೆ
ಉಪ-ದೇಶಿ-ಸಲಾ-ಗಿದೆ
ಉಪ-ದೇಶಿಸಿ-ದನು
ಉಪ-ದೇಶಿಸಿ-ದರೆ
ಉಪ-ದೇಶಿ-ಸುತ್ತದೆ
ಉಪದ್ರವ
ಉಪ-ನಿಷತ್ತಿ
ಉಪ-ನಿಷತ್ತಿ-ಗಿಂತ
ಉಪ-ನಿಷತ್ತಿಗೆ
ಉಪ-ನಿಷತ್ತಿನ
ಉಪ-ನಿಷತ್ತಿ-ನಲ್ಲಿ
ಉಪ-ನಿಷತ್ತಿ-ನಲ್ಲಿ-ರುವ
ಉಪ-ನಿಷತ್ತಿ-ನಲ್ಲಿವೆ
ಉಪ-ನಿಷತ್ತಿ-ನಿಂದ
ಉಪ-ನಿಷತ್ತು
ಉಪ-ನಿಷತ್ತು-ಗಳ
ಉಪ-ನಿಷತ್ತು-ಗಳನ್ನು
ಉಪ-ನಿಷತ್ತು-ಗಳಲ್ಲಿ
ಉಪ-ನಿಷತ್ತು-ಗಳಲ್ಲಿಯೂ
ಉಪ-ನಿಷತ್ತು-ಗಳಲ್ಲೆಲ್ಲಾ
ಉಪ-ನಿಷತ್ತೆಂದು
ಉಪನಿಷ್ತು-ಗಳಲ್ಲೆಲ್ಲ
ಉಪನ್ಯಾಸ
ಉಪನ್ಯಾಸ-ಗಳನ್ನು
ಉಪನ್ಯಾಸ-ಗಳು
ಉಪನ್ಯಾಸದ
ಉಪನ್ಯಾಸ-ದಲ್ಲಿ
ಉಪನ್ಯಾಸ-ವನ್ನು
ಉಪನ್ಯಾಸ-ವಾಗು-ವುದು
ಉಪ-ಮಾನ
ಉಪ-ಮಾನ-ವನ್ನು
ಉಪ-ಮಾನವು
ಉಪ-ಯೋಗ
ಉಪ-ಯೋಗ-ಕರ-ವಾಗು-ವಂತಹ
ಉಪ-ಯೋಗ-ಕಾರಿ
ಉಪ-ಯೋಗ-ಕಾರಿ-ಗಳಾ-ಗಿವೆ
ಉಪ-ಯೋಗ-ವನ್ನು
ಉಪ-ಯೋಗ-ವಿಲ್ಲ
ಉಪ-ಯೋಗ-ವೇನೂ
ಉಪ-ಯೋಗಿ
ಉಪ-ಯೋಗಿಸ
ಉಪ-ಯೋಗಿ-ಸದೆ
ಉಪ-ಯೋಗಿಸ-ಬಹುದು
ಉಪ-ಯೋಗಿಸ-ಬೇಕಾ-ಗಿದೆ
ಉಪ-ಯೋಗಿಸ-ಬೇಕು
ಉಪ-ಯೋಗಿಸ-ಬೇಕುಈ
ಉಪ-ಯೋಗಿಸ-ಬೇಕೆಂದು
ಉಪ-ಯೋಗಿಸ-ಬೇಕೆಂಬ
ಉಪ-ಯೋಗಿಸ-ಬೇಕೆಂಬು-ದನ್ನು
ಉಪ-ಯೋಗಿಸ-ಲಾಗಿದೆ
ಉಪ-ಯೋಗಿಸ-ಲಿಲ್ಲ
ಉಪ-ಯೋಗಿಸಿ
ಉಪ-ಯೋಗಿಸಿ-ಕೊಂಡು
ಉಪ-ಯೋಗಿಸಿ-ಕೊಳ್ಳ-ಬೇಕೆಂಬುದು
ಉಪ-ಯೋಗಿಸಿದ
ಉಪ-ಯೋಗಿಸಿ-ದರೆ
ಉಪ-ಯೋಗಿಸಿ-ರುವ
ಉಪ-ಯೋಗಿಸಿ-ರು-ವರು
ಉಪ-ಯೋಗಿಸಿ-ರುವು-ದಕ್ಕೆ
ಉಪ-ಯೋಗಿಸು
ಉಪ-ಯೋಗಿಸು-ತ್ತಿರು-ವರು
ಉಪ-ಯೋಗಿಸು-ತ್ತೀರಿ
ಉಪ-ಯೋಗಿಸು-ತ್ತೇವೆ
ಉಪ-ಯೋಗಿ-ಸುವ
ಉಪ-ಯೋಗಿಸು-ವನು
ಉಪ-ಯೋಗಿಸು-ವರು
ಉಪ-ಯೋಗಿಸು-ವವನು
ಉಪ-ಯೋಗಿಸು-ವುದಿಲ್ಲ
ಉಪ-ಯೋಗಿಸು-ವುದು
ಉಪ-ವಾಸ
ಉಪ-ವಾಸ-ದಿಂದ
ಉಪಾ-ದಾನ
ಉಪಾ-ದಾನ-ಕಾರಣ
ಉಪಾಧಿ
ಉಪಾಧಿ-ಗಳಿಂದ
ಉಪಾ-ಧಿಯ
ಉಪಾಯ
ಉಪಾಸಕ-ನಾಗುವ
ಉಪಾಸ-ಕರ
ಉಪಾಸ-ಕರು
ಉಪಾಸ-ಕರೋ
ಉಪಾಸ್ಯದ
ಉಪೇಕ್ಷೆ
ಉಬ್ಬಿಸಿ
ಉಭಯ
ಉರಿ
ಉರಿದು
ಉರಿ-ಯುತ್ತಿದೆ
ಉರಿ-ಯುತ್ತಿ-ರುವ
ಉರಿಸಿ
ಉರುಳಿ
ಉರುಳಿ-ರು-ವುವು
ಉರು-ಳುತ್ತವೆ-ಆ-ದರೆ
ಉರುಳುತ್ತಿದೆ
ಉರುಳು-ರುಳಿ
ಉಲ್ಲಾಸ-ದಿಂದ
ಉಲ್ಲಾಸ-ಭರಿತ-ರಾಗಿ-ರು-ವುದು
ಉಳಿದ
ಉಳಿದ-ಮೇಲೆ
ಉಳಿ-ದರು
ಉಳಿ-ದರೂ
ಉಳಿ-ದರೆ
ಉಳಿ-ದವ
ಉಳಿದ-ವ-ರನ್ನು
ಉಳಿದ-ವ-ರಲ್ಲ
ಉಳಿದ-ವ-ರಲ್ಲಿ
ಉಳಿದ-ವ-ರಿಂದ
ಉಳಿದ-ವ-ರಿಗೆ
ಉಳಿದ-ವರು
ಉಳಿದ-ವ-ರೆಲ್ಲ
ಉಳಿದ-ವ-ರೆಲ್ಲಾ
ಉಳಿ-ದವು
ಉಳಿದ-ವು-ಗ-ಳನ್ನು
ಉಳಿದ-ವು-ಗಳನ್ನೆಲ್ಲ
ಉಳಿದ-ವು-ಗಳಲ್ಲ
ಉಳಿದ-ವು-ಗಳಿಗೆ
ಉಳಿದ-ವು-ಗಳೆಲ್ಲ
ಉಳಿದ-ವು-ಗಳೆಲ್ಲಕ್ಕೂ
ಉಳಿದ-ವು-ಗಳೆಲ್ಲವೂ
ಉಳಿದ-ವೆಲ್ಲ
ಉಳಿ-ದಾಗ
ಉಳಿ-ದಿತ್ತು
ಉಳಿ-ದಿದೆ
ಉಳಿದಿ-ರ-ಬಹುದು
ಉಳಿದಿ-ರುತ್ತದೆ
ಉಳಿದಿ-ರುವ
ಉಳಿದಿ-ರು-ವು-ದ-ರಿಂದ
ಉಳಿದಿ-ರುವುದು
ಉಳಿ-ದಿವೆ
ಉಳಿದು
ಉಳಿದು-ದನ್ನು
ಉಳಿ-ದುದು
ಉಳಿದು-ದೆಲ್ಲ
ಉಳಿ-ದುವು
ಉಳಿದು-ವೆಲ್ಲ
ಉಳಿ-ದೆಲ್ಲ
ಉಳಿದೇ
ಉಳಿ-ಯಲಿ
ಉಳಿ-ಯಿತು
ಉಳಿಯು
ಉಳಿಯುತ್ತದೆ
ಉಳಿ-ಯುವ
ಉಳಿಯು-ವನು
ಉಳಿಯು-ವರು
ಉಳಿ-ಯು-ವು-ದಿಲ್ಲ
ಉಳಿಯು-ವುದು
ಉಳಿಯು-ವುದು-ಇ-ದಕ್ಕೆ
ಉಳಿಯು-ವುದೆ
ಉಳಿಯು-ವುದೆಂಬು-ದನ್ನು
ಉಳಿಯು-ವುದೇ
ಉಳಿಯು-ವುದೇನು
ಉಳಿಯು-ವುದೊ
ಉಳಿಯು-ವುದೋ
ಉಳಿಯು-ವುವು
ಉಳಿ-ವಿಲ್ಲ
ಉಳಿವು
ಉಳಿವು-ದೆಲ್ಲವೂ
ಉಳಿಸ-ಬೇಕೆಂದು
ಉಳಿಸಿ-ಕೊಂಡಿ-ರು-ವುದೋ
ಉಳಿಸಿ-ಕೊಂಡು
ಉಳಿಸಿ-ಕೊಳ್ಳುತ್ತೇವೆ
ಉಳಿಸಿ-ಕೊಳ್ಳು-ವುದೇ
ಉಳು-ವು-ದಕ್ಕೆ
ಉಳ್ಳ
ಉಳ್ಳದ್ದು
ಉಳ್ಳ-ವ-ನಿಗೆ
ಉಳ್ಳ-ವ-ರನ್ನು
ಉಷ್ಣ
ಉಸಿ-ರನ್ನಾಗಿ
ಉಸಿ-ರನ್ನು
ಉಸಿ-ರಲ್ಲ
ಉಸಿ-ರಾಗು-ವಂತೆ
ಉಸಿರಾಟ
ಉಸಿರಾಟ-ವೆಂದು
ಉಸಿರಾಡದೆ
ಉಸಿರಾಡ-ಲಾರೆವು
ಉಸಿರಾಡು
ಉಸಿರಾಡುತ್ತಾನೆ
ಉಸಿರಾಡುತ್ತಿದ್ದರು
ಉಸಿರಾ-ಡುವ
ಉಸಿರಾಡು-ವು-ದಕ್ಕೆ
ಉಸಿರಾಡು-ವು-ದನ್ನು
ಉಸಿರಾಡು-ವು-ದ-ರಲ್ಲಿ
ಉಸಿರಾಡು-ವುದು
ಉಸಿರಾಡು-ವುದು-ಇವು
ಉಸಿರಾಡು-ವುದೆಂದಲ್ಲ
ಉಸಿರಾಡು-ವೆವು
ಉಸಿ-ರಿಗೂ
ಉಸಿ-ರಿಗೆ
ಉಸಿರಿನ
ಉಸಿರಿ-ನೊಂದಿಗೆ
ಉಸಿರು
ಉಸಿರುವ
ಉಸಿರೂ
ಉಸಿ-ರೆಂದು
ಉಸುರುತ್ತಲೇ
ಉಸು-ರು-ವುದು
ಊಟ
ಊಟಕ್ಕೆ
ಊಟ-ಮಾಡ-ಬಲ್ಲರು
ಊಟ-ಮಾಡಿ-ದರೆ
ಊಟ-ಮಾಡಿ-ದು-ದನ್ನು
ಊಟ-ಮಾಡುತ್ತಿಲ್ಲ-ವೆಂದು
ಊಟ-ಮಾಡುವ
ಊಟ-ಮಾಡು-ವೆನು
ಊಟ-ಮಾಡು-ವೆವು
ಊಟ-ವಿಲ್ಲದೆ
ಊನ-ವಾಗುತ್ತಿತ್ತು
ಊನ-ವಾ-ದರೆ
ಊರಿಂದ
ಊರಿಗೆ
ಊರಿ-ನಲ್ಲಿ
ಊಹಿಸ
ಊಹಿಸ-ಬಹುದು
ಊಹಿಸ-ಬೇಕಾ-ಗಿ-ಬರು-ವುದೆಂಬು-ದನ್ನು
ಊಹಿಸ-ಬೇಕಾ-ಗುತ್ತದೆ
ಊಹಿ-ಸ-ಬೇಕು
ಊಹಿಸ-ಬೇಕೆಂಬ
ಊಹಿಸ-ಬೇಡಿ
ಊಹಿಸ-ಲಾ-ಗಿದೆ
ಊಹಿಸ-ಲಾರೆವು
ಊಹಿಸಿ
ಊಹಿಸಿ-ಕೊಳ್ಳು-ವಂತೆ
ಊಹಿ-ಸಿದ
ಊಹಿಸಿದ್ದನೊ
ಊಹಿಸುತ್ತೀರಿ
ಊಹಿಸುತ್ತೇನೆ
ಊಹಿಸುತ್ತೇನೆ-ಮಾನ-ವನ
ಊಹಿ-ಸು-ವು-ದಕ್ಕೆ
ಊಹಿಸು-ವು-ದ-ರಿಂದ
ಊಹಿ-ಸು-ವುದು
ಊಹಿ-ಸು-ವೆವು
ಊಹೆ
ಊಹೆ-ಗಳು
ಊಹೆಗೂ
ಊಹೆಗೆ
ಊಹೆಯ
ಊಹೆ-ಯಂತೆ
ಋಎಂದ
ಋಗ್ವೇದ
ಋಗ್ವೇದದ
ಋಗ್ವೇದ-ದಲ್ಲಿ
ಋಣಿ-ಗಳೆಲ್ಲ
ಋಣಿ-ಯಾಗದೆ
ಋತಂಭರಾ
ಋಷಿ
ಋಷಿ-ಗಳ
ಋಷಿ-ಗಳಂತಹ
ಋಷಿ-ಗಳಾ-ಗಿ-ರ-ಬೇಕಾ-ಗಿತ್ತು
ಋಷಿ-ಗಳು
ಋಷಿಯ
ಋಷಿ-ಯನ್ನು
ಋಷಿಯು
ಋಷಿ-ಯೊಬ್ಬ-ನಿಂದ
ಎ
ಎಂಜಿನ್ನಿ-ನಿಂದ
ಎಂಟನೇ
ಎಂಟನ್ನು
ಎಂಟರ
ಎಂಟು
ಎಂತಲೂ
ಎಂತಹ
ಎಂತಹುದು
ಎಂದ
ಎಂದಂತೆ
ಎಂದನು
ಎಂದ-ನು-ಆಗ
ಎಂದರು
ಎಂದರೆ
ಎಂದ-ರೇನು
ಎಂದ-ರೇ-ನೆಂದು
ಎಂದ-ರೇ-ನೆಂಬುದು
ಎಂದಲ್ಲ
ಎಂದಳು
ಎಂದಷ್ಟೆ
ಎಂದಾ
ಎಂದಾ-ಗಲಿ
ಎಂದಾ-ಗಿತ್ತು
ಎಂದಾ-ಗುತ್ತದೆ
ಎಂದಾ-ದರೂ
ಎಂದಾದ-ರೊಂದು
ಎಂದಾ-ದ-ರೊಮ್ಮೆ
ಎಂದಾ-ಯಿತು
ಎಂದಿ
ಎಂದಿ-ಗಿಂತ
ಎಂದಿಗೂ
ಎಂದಿಟ್ಟು-ಕೊಳ್ಳಿ
ಎಂದಿತು
ಎಂದಿದೆ
ಎಂದಿ-ನಂತೆ
ಎಂದಿ-ನಂತೆಯೆ
ಎಂದಿ-ನಿಂದಲೂ
ಎಂದಿರಿ
ಎಂದಿ-ರುವನು
ಎಂದು
ಎಂದು-ಕೊಂಡ
ಎಂದು-ಕೊಂಡನು
ಎಂದು-ಕೊಂಡೆ
ಎಂದು-ಕೊಳ್ಳಿ
ಎಂದು-ಗೊತ್ತಾಗು
ಎಂದೂ
ಎಂದೆ
ಎಂದೆಂದಿಗೂ
ಎಂದೆಂದೂ
ಎಂದೇ
ಎಂದೇನೂ
ಎಂದೊ
ಎಂಬ
ಎಂಬಂತಹ
ಎಂಬಂತಿರ-ಬಾ-ರದು
ಎಂಬಂತೆ
ಎಂಬರ್ಥ-ದಲ್ಲಿ
ಎಂಬಲ್ಲಿ
ಎಂಬು
ಎಂಬುದ
ಎಂಬು-ದಕ್ಕೆ
ಎಂಬು-ದನ್ನು
ಎಂಬು-ದನ್ನೂ
ಎಂಬು-ದರ
ಎಂಬು-ದ-ರಲ್ಲಿ
ಎಂಬು-ದಾಗಿ
ಎಂಬು-ದಿಲ್ಲ
ಎಂಬು-ದಿಷ್ಟನ್ನೇ
ಎಂಬುದು
ಎಂಬು-ದು-ಆದ-ಕಾರಣ
ಎಂಬುದೂ
ಎಂಬುದೆ
ಎಂಬು-ದೆಲ್ಲ
ಎಂಬು-ದೆಲ್ಲಾ
ಎಂಬು-ದೆಲ್ಲಿದೆ
ಎಂಬುದೇ
ಎಂಬು-ದೇನೊ
ಎಂಬು-ದೇನೋ
ಎಂಬು-ದೊಂದಿದೆ
ಎಂಬು-ದೊಂದು
ಎಂಬು-ವಂತೆ
ಎಂಬುವು
ಎಂಬು-ವು-ಗಳೆಲ್ಲ
ಎಂಬು-ವು-ದನ್ನು
ಎಂಭತ್ತು
ಎಕರೆ-ಗಳಲ್ಲಿ
ಎಚ್ಚರ-ಗೊಳಿ-ಸು-ವುದೊಂದೆ
ಎಚ್ಚರಗೊಳ್ಳುವ
ಎಚ್ಚರಿಕೆ
ಎಚ್ಚರಿಕೆ-ಯಿಂದ
ಎಚ್ಚರಿ-ಸು-ವು-ದಕ್ಕೆ
ಎಚ್ಚೆತ್ತ
ಎಡ-ಗಡೆ
ಎಡ-ಗಡೆಯ
ಎಡ-ಗಡೆ-ಯದೇ
ಎಡದ
ಎಡ-ಭಾಗ-ದಲ್ಲಿ-ರುವುದು
ಎಡವಿ
ಎಡವಿ-ದಂತಿ-ರು-ವುದು
ಎಡವಿ-ದರೋ
ಎಡವಿ-ದಾಗ
ಎಡ-ವಿರು
ಎಡಹು-ವನು
ಎಡಹು-ವನೊ
ಎಡಹು-ವವ-ನನ್ನು
ಎಡೆ-ಕೊಟ್ಟವು
ಎಡೆಗೆ
ಎಡೆ-ಬಿಡದೆ
ಎಡೆ-ಯಲ್ಲೆಲ್ಲ
ಎಡೆ-ಯಿಂದ
ಎಡ್ವಿನ್
ಎಣಿ-ಸಿದ್ದೆ-ನಲ್ಲ
ಎಣಿ-ಸು-ವು-ದಕ್ಕೆ
ಎಣ್ಣೆ
ಎಣ್ಣೆ-ಯನ್ನು
ಎತ್ತ
ಎತ್ತದ
ಎತ್ತದೇ
ಎತ್ತ-ರ-ದಲ್ಲಿ
ಎತ್ತ-ರ-ವಾಗ-ಬಲ್ಲ
ಎತ್ತ-ರ-ವಾ-ಗಿ-ರುವ
ಎತ್ತರ-ವಾದ
ಎತ್ತಿ
ಎತ್ತಿ-ಕೊಂಡಿದ್ದವ-ನನ್ನು
ಎತ್ತು
ಎತ್ತುವ
ಎತ್ತೊಂದು
ಎದುರಿ
ಎದುರಿ-ಗಿ-ರುವ
ಎದು-ರಿಗೆ
ಎದುರಿ-ಸ-ಬಲ್ಲುದೆ
ಎದುರಿ-ಸ-ಬಹುದು
ಎದುರಿ-ಸ-ಬೇಕಾ-ಗಿದೆ
ಎದುರಿ-ಸ-ಬೇಕಾ-ಗು-ವುದು
ಎದುರಿ-ಸಲು
ಎದುರಿ-ಸುವ
ಎದೆ
ಎದೆ-ಗಾರಿಕೆ
ಎದೆ-ಯಲ್ಲಿದ್ದರೆ
ಎದ್ದ
ಎದ್ದರೆ
ಎದ್ದಿತು
ಎದ್ದಿವೆ
ಎದ್ದು
ಎದ್ದು-ನಿಂತು
ಎದ್ದು-ನಿಲ್ಲಿ
ಎನಿ-ಸುತ್ತದೆ
ಎನಿ-ಸು-ವುದು
ಎನಿಸು-ವುದೊ
ಎನ್ನ
ಎನ್ನ-ಬಹು-ದಾ-ಗಿತ್ತು
ಎನ್ನ-ಬಹುದು
ಎನ್ನ-ಬೇಕಾ-ಗುತ್ತದೆ
ಎನ್ನ-ಬೇಕಾ-ಗು-ವುದು
ಎನ್ನ-ಬೇಡಿ
ಎನ್ನ-ಲಾಗು-ವು-ದಿಲ್ಲ
ಎನ್ನ-ಲಾರೆವು
ಎನ್ನಲು
ಎನ್ನಿ
ಎನ್ನಿ-ಸಿ-ಕೊಳ್ಳುವ
ಎನ್ನಿ-ಸಿತು
ಎನ್ನಿ-ಸು-ವುದು
ಎನ್ನು
ಎನ್ನುತ್ತದೆ
ಎನ್ನುತ್ತಾನೆ
ಎನ್ನುತ್ತಾರೆ
ಎನ್ನುತ್ತೇನೆ
ಎನ್ನುತ್ತೇವೆ
ಎನ್ನುತ್ತೇವೆಯೊ
ಎನ್ನುವ
ಎನ್ನು-ವಂತೆ
ಎನ್ನು-ವ-ದೇನೋ
ಎನ್ನು-ವನು
ಎನ್ನು-ವ-ರಲ್ಲ
ಎನ್ನು-ವರು
ಎನ್ನು-ವ-ರುಆ
ಎನ್ನು-ವರೊ
ಎನ್ನು-ವ-ವ-ನಂತೆ
ಎನ್ನು-ವ-ವನು
ಎನ್ನು-ವ-ವರ
ಎನ್ನು-ವ-ವ-ರಲ್ಲಿ
ಎನ್ನು-ವ-ವರು
ಎನ್ನು-ವಷ್ಟು
ಎನ್ನು-ವಾ-ಗಲೂ
ಎನ್ನು-ವಿರಿ
ಎನ್ನು-ವಿರೊ
ಎನ್ನುವು
ಎನ್ನು-ವು-ದಕ್ಕೆ
ಎನ್ನು-ವು-ದನ್ನು
ಎನ್ನು-ವು-ದರ
ಎನ್ನು-ವು-ದ-ರಲ್ಲಿ
ಎನ್ನು-ವು-ದ-ರಿಂದ
ಎನ್ನು-ವು-ದಿಲ್ಲ
ಎನ್ನು-ವುದು
ಎನ್ನು-ವು-ದು-ಆ-ಗಲೆ
ಎನ್ನು-ವು-ದು-ಆ-ದರೆ
ಎನ್ನು-ವುದೂ
ಎನ್ನು-ವು-ದೆಲ್ಲಾ
ಎನ್ನು-ವುದೇ
ಎನ್ನು-ವು-ದೇನೊ
ಎನ್ನು-ವು-ದೇನೋ
ಎನ್ನು-ವುವು
ಎನ್ನು-ವೆನು
ಎನ್ನು-ವೆವು
ಎನ್ನು-ವೆವೊ
ಎನ್ನು-ವೆವೋ
ಎನ್ನೋಣ
ಎಬಿಎಬಿ
ಎಬ್ಬಿಸಿ-ದರೆ
ಎಬ್ಬಿಸಿ-ದಾಗ
ಎಬ್ಬಿ-ಸುತ್ತಿದ್ದರೆ
ಎಬ್ಬಿಸು-ವು-ದ-ರಿಂದ
ಎರಕ-ದಲ್ಲಿ
ಎರಡ
ಎರಡಕ್ಕೂ
ಎರ-ಡನೆ
ಎರಡ-ನೆಯ
ಎರಡ-ನೆ-ಯ-ದನ್ನು
ಎರಡ-ನೆ-ಯ-ದ-ರಲ್ಲಿ
ಎರಡ-ನೆ-ಯ-ದಾಗಿ
ಎರಡ-ನೆ-ಯದು
ಎರಡ-ನೆ-ಯದೆ
ಎರಡನೇ
ಎರಡನ್ನು
ಎರ-ಡನ್ನೂ
ಎರಡರ
ಎರಡ-ರಲ್ಲಿ
ಎರಡ-ರಲ್ಲಿಯೂ
ಎರಡ-ರಷ್ಟು
ಎರಡಲ್ಲ
ಎರಡಿ-ರ-ಲಿಲ್ಲ
ಎರಡಿಲ್ಲ
ಎರಡು
ಎರಡು-ಕೋಟಿಗೆ
ಎರಡೂ
ಎರಡೆ-ರಡು
ಎರವ-ಲಾಗಿ
ಎರೆ-ದರೆ
ಎರೆ-ಯಲು
ಎಲೆ-ಗಳಿ-ವೆಯೋ
ಎಲ್ಲ
ಎಲ್ಲಕ್ಕಿಂತ
ಎಲ್ಲಕ್ಕಿಂದ
ಎಲ್ಲಕ್ಕೂ
ಎಲ್ಲದಕ್ಕಿಂತಲೂ
ಎಲ್ಲ-ದಕ್ಕೂ
ಎಲ್ಲದರ
ಎಲ್ಲ-ದ-ರಲ್ಲಿಯೂ
ಎಲ್ಲದ-ರಲ್ಲೂ
ಎಲ್ಲದ-ರಿಂದಲೂ
ಎಲ್ಲದ-ರೊ-ಳಗೆ
ಎಲ್ಲರ
ಎಲ್ಲ-ರನ್ನೂ
ಎಲ್ಲ-ರಲ್ಲಿ
ಎಲ್ಲ-ರಲ್ಲಿಯೂ
ಎಲ್ಲ-ರಲ್ಲಿ-ರುವ
ಎಲ್ಲ-ರಲ್ಲೂ
ಎಲ್ಲ-ರಿಂದಲೂ
ಎಲ್ಲ-ರಿ-ಗಿಂತ
ಎಲ್ಲ-ರಿಗೂ
ಎಲ್ಲರೂ
ಎಲ್ಲ-ವನ್ನು
ಎಲ್ಲ-ವನ್ನೂ
ಎಲ್ಲವೂ
ಎಲ್ಲಾ
ಎಲ್ಲಿ
ಎಲ್ಲಿಂದ
ಎಲ್ಲಿಗೂ
ಎಲ್ಲಿಗೆ
ಎಲ್ಲಿಗೋ
ಎಲ್ಲಿತ್ತು
ಎಲ್ಲಿದೆ
ಎಲ್ಲಿದ್ದರೂ
ಎಲ್ಲಿದ್ದೆವೋ
ಎಲ್ಲಿಯ
ಎಲ್ಲಿ-ಯ-ವರೆ
ಎಲ್ಲಿ-ಯ-ವರೆಗೂ
ಎಲ್ಲಿ-ಯ-ವರೆಗೆ
ಎಲ್ಲಿ-ಯ-ವರೆ-ವಿಗೂ
ಎಲ್ಲಿ-ಯಾ-ದರೂ
ಎಲ್ಲಿ-ಯಾದ-ರೊಂದು
ಎಲ್ಲಿಯೂ
ಎಲ್ಲಿಯೊ
ಎಲ್ಲಿಯೋ
ಎಲ್ಲಿ-ರುವನೊ
ಎಲ್ಲಿ-ರುವುದು
ಎಲ್ಲಿವೆ
ಎಲ್ಲಿವೆಯೊ
ಎಲ್ಲೆ
ಎಲ್ಲೆ-ಡೆ-ಯಲ್ಲಿಯೂ
ಎಲ್ಲೆ-ಡೆ-ಯಲ್ಲೂ
ಎಲ್ಲೆ-ಯನ್ನು
ಎಲ್ಲೆ-ಯಲ್ಲಿ
ಎಲ್ಲೆ-ಯಾಚೆ-ಯಿಂದ
ಎಲ್ಲೆ-ಯೊಳ-ಗಿದೆ
ಎಲ್ಲೆ-ಯೊ-ಳಗೆ
ಎಲ್ಲೆಲ್ಲಿಯೂ
ಎಲ್ಲೆಲ್ಲೂ
ಎಲ್ಲೊ
ಎಲ್ಲೋ
ಎಳೆ-ದರೆ
ಎಳೆ-ದಾದ
ಎಳೆದು
ಎಳೆದು-ಕೊಂಡು
ಎಳೆದು-ಕೊಂಡು-ಹೋ-ಯಿತು
ಎಳೆದು-ತಂದಾಗ
ಎಳೆಯ
ಎಳೆ-ಯದೆ
ಎಳೆ-ಯನ್ನು
ಎಳೆಯ-ಬೇಡಿ
ಎಳೆಯ-ಲಾರದು
ಎಳೆ-ಯಲು
ಎಳೆ-ಯುವ
ಎಳೆ-ಯು-ವೆವು
ಎಳ್ಳಷ್ಟೂ
ಎವೆಯಿಕ್ಕದೆ
ಎಷ್ಟರ-ಮಟ್ಟಿಗೆ
ಎಷ್ಟಿದ್ದರೂ
ಎಷ್ಟು
ಎಷ್ಟೇ
ಎಷ್ಟೊಂದು
ಎಷ್ಟೋ
ಎಸೆ-ದಂತೆ
ಎಸೆ-ದರೂ
ಎಸೆ-ದರೆ
ಎಸೆ-ದಿಲ್ಲ
ಎಸೆದು
ಎಸೆಯ-ಬಲ್ಲದು
ಎಸೆಯ-ಬೇಕೆ
ಎಸೆ-ಯಲು
ಎಸೆಯಲ್ಪಟ್ಟಿವೆ
ಎಸೆ-ಯುತ್ತಾ
ಎಸೆಯುತ್ತಿ-ರು-ವೆವು
ಎಸೆ-ಯುವ
ಎಸೆಯು-ವು-ದ-ರಲ್ಲಿ
ಎಸೆ-ಯೋಣ
ಏಕ
ಏಕಕಂಠ-ದಿಂದ
ಏಕ-ಕಾಲ
ಏಕ-ಕಾಲ-ದಲ್ಲಿ
ಏಕ-ಕಾಲ-ದಲ್ಲೆ
ಏಕತಾ-ಭಾವ-ನೆ-ಯನ್ನು
ಏಕತೆ
ಏಕ-ತೆಗೆ
ಏಕ-ತೆಯ
ಏಕತೆ-ಯನ್ನು
ಏಕ-ತೆಯೆ
ಏಕತ್ವ
ಏಕತ್ವದ
ಏಕತ್ವ-ದಲ್ಲಿ
ಏಕತ್ವ-ವನ್ನು
ಏಕತ್ವವೇ
ಏಕದ
ಏಕ-ದಂತೆ
ಏಕ-ದಿಂದ
ಏಕ-ದೆ-ಡೆಗೆ
ಏಕದೇವ
ಏಕ-ನಾಗಿ
ಏಕ-ಪಕ್ಷ-ದ-ವರೇ
ಏಕಪ್ರ-ಕಾರ-ವಾಗಿ
ಏಕಪ್ರ-ಕಾರ-ವಾ-ಗು-ವುದು
ಏಕಮ-ತೀಯ-ರಲ್ಲ
ಏಕ-ಮತೀ-ಯರು
ಏಕ-ಮಾತ್ರ
ಏಕ-ಮುಖ-ವಾದ
ಏಕ-ಮೇವ
ಏಕ-ರೀತಿ-ಯಿಂದ
ಏಕ-ವನ್ನು
ಏಕ-ವಸ್ತು
ಏಕ-ವಾಗಿ
ಏಕ-ವಾದ
ಏಕ-ವಾ-ದುದು
ಏಕವು
ಏಕವೆ
ಏಕ-ವೆಂದು
ಏಕವೇ
ಏಕ-ಶಕ್ತಿಯ
ಏಕ-ಸತ್ಯದ
ಏಕ-ಸ-ಮಯ-ದಲ್ಲಿ
ಏಕ-ಸ-ಮಯೇ
ಏಕಾಂಗಿ
ಏಕಾಂತ
ಏಕಾಕಿ-ಗಳಾ-ಗಿದ್ದೆವು
ಏಕಾಕಿ-ಯಾ-ಗಿಯೇ
ಏಕಾಕಿ-ಯಾಗುತ್ತಾನೆ
ಏಕಾಗ್ರ
ಏಕಾಗ್ರ-ಗೊಳಿ-ಸ-ಬಹುದು
ಏಕಾಗ್ರ-ಗೊಳಿಸಿ
ಏಕಾಗ್ರ-ಗೊಳಿ-ಸಿ-ದರೆ
ಏಕಾಗ್ರ-ಗೊಳಿ-ಸಿ-ದಾಗ
ಏಕಾಗ್ರ-ಗೊಳಿ-ಸುವ
ಏಕಾಗ್ರ-ಗೊಳಿ-ಸು-ವುದು
ಏಕಾಗ್ರತೆ
ಏಕಾಗ್ರ-ತೆ-ಗಳು
ಏಕಾಗ್ರ-ತೆಗೂ
ಏಕಾಗ್ರ-ತೆಗೆ
ಏಕಾಗ್ರ-ತೆಯ
ಏಕಾಗ್ರ-ತೆ-ಯನ್ನು
ಏಕಾಗ್ರ-ತೆ-ಯನ್ನೂ
ಏಕಾಗ್ರ-ತೆ-ಯಲ್ಲಿ
ಏಕಾಗ್ರ-ತೆ-ಯಿಂದ
ಏಕಾಗ್ರ-ತೆಯು
ಏಕಾಗ್ರ-ತೆ-ಯೆಂದರೆ
ಏಕಾಗ್ರ-ತೆ-ಯೊಂದೇ
ಏಕಾಗ್ರ-ನಾ-ದಷ್ಟೂ
ಏಕಾಗ್ರ-ಮಾಡಿ
ಏಕಾಗ್ರ-ಮಾಡಿ-ದರೆ
ಏಕಾಗ್ರ-ವಾಗಿ
ಏಕಾಗ್ರ-ವಾಗಿದೆ
ಏಕಾಗ್ರ-ವಾಗಿ-ರುತ್ತದೆ
ಏಕಾಗ್ರ-ವಾಗಿ-ರು-ವು-ದ-ರಿಂದ
ಏಕಾಗ್ರ-ವಾ-ಗಿ-ರುವುದು
ಏಕಾಗ್ರ-ವಾಗು
ಏಕಾಗ್ರ-ವಾಗು-ವು-ದಿಲ್ಲ
ಏಕಾಗ್ರ-ವಾಗು-ವುದು
ಏಕಾಗ್ರ-ವಾ-ದರೆ
ಏಕಾಗ್ರ-ವಾ-ದಷ್ಟೂ
ಏಕಾಗ್ರ-ವಾ-ದಾಗ
ಏಕಾಗ್ರವೂ
ಏಕಾಭಿ
ಏಕಾಭಿಪ್ರಾಯ
ಏಕಿ-ರ-ಬೇಕು
ಏಕಿಷ್ಟು
ಏಕೀಕ-ರಿಸಿ-ಕೊಳ್ಳು-ವೆನು
ಏಕೆ
ಏಕೆಂದರೆ
ಏಕೋ
ಏಟಿಗೆ
ಏಟು
ಏಟು-ಬೀಳದೆ
ಏತಕ್ಕೆ
ಏತಕ್ಕೆಂದರೆ
ಏತಯೈವ
ಏತೇ
ಏತೇನ
ಏದು
ಏದು-ಸಿ-ರನ್ನು
ಏನನ್ನಾ-ದರೂ
ಏನನ್ನು
ಏನನ್ನೂ
ಏನನ್ನೊ
ಏನನ್ನೋ
ಏನಾ
ಏನಾ-ಗ-ಬಹುದು
ಏನಾ-ಗ-ಬೇಕು
ಏನಾ-ಗ-ಬೇಕೆಂದು
ಏನಾ-ಗಿತ್ತು
ಏನಾ-ಗಿ-ರುವನೋ
ಏನಾ-ಗಿ-ರು-ವೆವೊ
ಏನಾಗುತ್ತಾನೆ
ಏನಾ-ಗುತ್ತಿತ್ತು
ಏನಾ-ಗುತ್ತಿದೆ
ಏನಾ-ಗುತ್ತಿದೆಯೊ
ಏನಾ-ಗು-ವುದು
ಏನಾ-ಗು-ವುದೆಂದರೆ
ಏನಾ-ಗು-ವುದೆಂಬು-ದನ್ನು
ಏನಾ-ಗುವು-ದೆಂಬುದು
ಏನಾ-ದರೂ
ಏನಾ-ದರೂ-ರಹಸ್ಯ-ವನ್ನು
ಏನಾ-ದರೊಂದನ್ನು
ಏನಾ-ದ-ರೊಂದು
ಏನಾ-ದು-ವೆಂಬು-ದನ್ನು
ಏನಾ-ಯಿತು
ಏನಿದೆ
ಏನಿರು-ವುದೊ
ಏನಿವೆ
ಏನಿ-ವೆಯೊ
ಏನು
ಏನೂ
ಏನೆಂದರೆ
ಏನೆಂದು
ಏನೆಂಬು-ದನ್ನು
ಏನೆಂಬುದು
ಏನೆನ್ನುತ್ತಾರೆ
ಏನೇ
ಏನೇ-ನನ್ನು
ಏನೇನು
ಏನೇನೊ
ಏನೇನೋ
ಏನೊ
ಏನೋ
ಏರದು
ಏರ-ಬಲ್ಲ
ಏರ-ಬೇಕು
ಏರಲು
ಏರಿ
ಏರಿಗೂ
ಏರಿದ
ಏರಿ-ದರೆ
ಏರಿ-ದಾಗ
ಏರಿ-ದಾಗ-ಲೆಲ್ಲಾ
ಏರಿದೆ
ಏರಿ-ಳಿತ
ಏರಿ-ಳಿತ-ಗಳ
ಏರಿ-ಳಿತ-ಗಳಿಗೆ
ಏರಿ-ಸ-ಬೇಕು
ಏರಿ-ಸ-ಲಿಲ್ಲ
ಏರಿಸಿ
ಏರಿಸು
ಏರಿ-ಸುವ
ಏರಿ-ಸು-ವುದು
ಏರುತ್ತದೆ
ಏರುತ್ತಾರೆ
ಏರುತ್ತಿದೆಯೊ
ಏರುವ
ಏರು-ವರು
ಏರು-ವ-ವರೆ-ವಿಗೂ
ಏರು-ವು-ದಿಲ್ಲವೋ
ಏರು-ವೆವು
ಏರ್ಪಡ-ಬೇಕಾ-ದರೂ
ಏಳಿ
ಏಳಿಗೆ
ಏಳಿ-ಗೆಗೆ
ಏಳು
ಏಳುತ್ತದೆ
ಏಳುತ್ತಲೇ
ಏಳುತ್ತವೆ
ಏಳುತ್ತಾನೆ
ಏಳುತ್ತಿರು
ಏಳುತ್ತಿರು-ವನು
ಏಳುತ್ತಿ-ರು-ವೆವು
ಏಳುತ್ವ
ಏಳು-ಬೀಳು-ಗಳ
ಏಳು-ಬೀಳು-ಗಳಿಲ್ಲ
ಏಳುವ
ಏಳು-ವನು
ಏಳು-ವರು
ಏಳು-ವು-ದಕ್ಕೆ
ಏಳು-ವುದರ
ಏಳು-ವು-ದಿಲ್ಲ
ಏಳು-ವುದು
ಏಳು-ವುವು
ಏವ
ಏಷ್ಯಾ
ಏಷ್ಯಾ-ಖಂಡದ
ಏಷ್ಯಾ-ಖಂಡವೇ
ಏಷ್ಯಾದ
ಏಷ್ಯಾ-ದಲ್ಲಿ
ಏಸುಕ್ರಿಸ್ತ
ಏಸುವಿನ
ಏಸುವು
ಐಕ-ಮಾತ್ರ-ವೆನ್ನು-ವುದು
ಐಕ್ಯ
ಐಕ್ಯ-ಗೊಳಿ-ಸ-ಬಹುದು
ಐಕ್ಯತೆ
ಐಕ್ಯ-ತೆಗೆ
ಐಕ್ಯ-ತೆ-ಯನ್ನು
ಐಕ್ಯ-ತೆ-ಯಿಂದ
ಐಕ್ಯ-ತೆ-ಯಿದೆ
ಐಕ್ಯ-ನಾಗು-ವನು
ಐಕ್ಯ-ರಾಗು-ವರು
ಐಕ್ಯ-ರಾಗು-ವು-ದಿಲ್ಲ
ಐಕ್ಯ-ರಾ-ದ-ರೆಂಬು-ದನ್ನು
ಐಕ್ಯ-ವಾ-ಗಲು
ಐಕ್ಯ-ವಾಗಿ
ಐಕ್ಯ-ವಾಗಿದೆ
ಐಕ್ಯ-ವಾಗು
ಐಕ್ಯ-ವಾಗುತ್ತದೆ
ಐಕ್ಯ-ವಾಗು-ವಂತೆ
ಐಕ್ಯ-ವಾಗು-ವ-ವರೆಗೆ
ಐಕ್ಯ-ವಾ-ಗು-ವುದು
ಐಕ್ಯ-ವಾ-ಗು-ವುವು
ಐಕ್ಯವೇ
ಐಕ್ಯ-ವೊಂದೇ
ಐಚ್ಛಿಕ
ಐತಿಹಾ-ಸಿಕ
ಐದ-ನೆ-ಯದು
ಐದು
ಐದು-ಸಾ-ವಿರ
ಐದೇ
ಐರೋಪ್ಯ
ಐರೋಪ್ಯ-ರಲ್ಲಿ
ಐರೋಪ್ಯರು
ಐವತ್ತು
ಐಶ್ವರ್ಯ
ಐಶ್ವರ್ಯ-ಗಳೂ
ಐಶ್ವರ್ಯದ
ಐಶ್ವರ್ಯ-ದಲ್ಲಿ
ಐಶ್ವರ್ಯ-ದಿಂದ
ಐಶ್ವರ್ಯ-ಭೋಗ-ಗಳಲ್ಲಿ
ಐಶ್ವರ್ಯ-ವಂತನ
ಐಶ್ವರ್ಯ-ವಂತ-ನಿದ್ದ
ಐಶ್ವರ್ಯ-ವನ್ನು
ಐಶ್ವರ್ಯ-ವಾ-ಗಲಿ
ಒಂಟಿ-ಕಾ-ಲಿನ
ಒಂಟಿ-ಯಾಗಿ
ಒಂದ
ಒಂದಕ್ಕೆ
ಒಂದಕ್ಕೇ
ಒಂದನ್ನು
ಒಂದನ್ನೂ
ಒಂದನ್ನೆ
ಒಂದನ್ನೊಂದು
ಒಂದರ
ಒಂದ-ರಂತೆಯೇ
ಒಂದ-ರಲ್ಲಿ
ಒಂದ-ರಲ್ಲಿದೆ
ಒಂದ-ರಲ್ಲಿಯೇ
ಒಂದ-ರಲ್ಲೆ
ಒಂದ-ರಷ್ಟು
ಒಂದ-ರಿಂದ
ಒಂದರೆ
ಒಂದ-ರೊಡನೊಂದು
ಒಂದಲ್ಲ
ಒಂದಾಗ-ಬೇಕು
ಒಂದಾಗಿ
ಒಂದಾಗಿ-ರುವನು
ಒಂದಾಗಿ-ರು-ವುದೋ
ಒಂದಾಗು-ತ್ತದೆ
ಒಂದಾಗುವ
ಒಂದಾಗು-ವುದು
ಒಂದಾಗು-ವುವು
ಒಂದಾಗು-ವೆವು
ಒಂದಾದ
ಒಂದಾದ-ಮೇಲೊಂದ-ರಂತೆ
ಒಂದಾ-ದರೂ
ಒಂದಾದರೆ
ಒಂದಾನೊಂದು
ಒಂದಿದೆ
ಒಂದು
ಒಂದು-ಕೂಡಿಸಿ
ಒಂದು-ಗೂಡಿಸಿದ
ಒಂದು-ಗೂಡಿ-ಸುತ್ತದೆ
ಒಂದು-ಗೂಡಿ-ಸುವ
ಒಂದು-ಗೂಡಿ-ಸು-ವಂತೆ
ಒಂದು-ಗೂಡುತ್ತವೆ
ಒಂದು-ಮಿಶ್ರಣ
ಒಂದೆ
ಒಂದೆ-ಎಲ್ಲಿ-ಯ-ವರೆಗೆ
ಒಂದೆಡೆ
ಒಂದೆ-ಡೆ-ಯಲ್ಲಿ
ಒಂದೆ-ಯಾಗಿ
ಒಂದೆ-ರಡು
ಒಂದೇ
ಒಂದೇ-ಒಂದೇ
ಒಂದೇ-ಯಾಗಿದ್ದರೂ
ಒಂದೊಂದ-ರಂತೆ
ಒಂದೊಂದು
ಒಂಬತ್ತು
ಒಗ್ಗಟ್ಟಿ-ನಿಂದ
ಒಗ್ಗಿ-ಹೋ-ದರೆ
ಒಟ್ಟಿಗೆ
ಒಟ್ಟಿ-ನಲ್ಲಿ
ಒಟ್ಟು
ಒಟ್ಟು-ಗೂಡಿಸಿ
ಒಡಂಬಡಿಕೆ
ಒಡಂಬಡಿಕೆ-ಗಳ
ಒಡಂಬಡಿಕೆ-ಯಲ್ಲಿ
ಒಡನೆ
ಒಡ-ನೆಯೆ
ಒಡ-ನೆಯೇ
ಒಡ-ಹುಟ್ಟಿ-ದ-ವರಿ-ಗಿಂತ
ಒಡೆದ
ಒಡೆದು
ಒಡೆದು-ಹೋ-ಗು-ವುವು
ಒಡೆಯ
ಒಡೆಯ-ನಾಗು-ವನು
ಒಡೆಯ-ರಾಗಿ
ಒಣ
ಒಣಗಿ
ಒಣಗಿದ
ಒಣಗಿ-ಸದು
ಒಣಗಿ-ಹೋ-ಗು-ವುವು
ಒಣ-ಪಾಂಡಿತ್ಯ
ಒಣ-ಪಾಂಡಿತ್ಯದ
ಒತ್ತಡ
ಒತ್ತ-ಬೇಕು
ಒತ್ತಾ-ಯಿಸು-ವುದು
ಒತ್ತಿ
ಒತ್ತು-ವುದು
ಒದಗ-ಬಹುದು
ಒದಗಿ-ದರೆ
ಒದಗಿಸ
ಒದಗಿಸ-ಬಲ್ಲದು
ಒದ-ಗಿಸಿ-ಕೊಡು-ವುದು
ಒದ-ಗಿಸಿ-ದುವು
ಒದಗಿ-ಸುತ್ತದೆ
ಒದಗಿ-ಸುತ್ತದೆ-ಆ-ದರೆ
ಒದಗಿಸುತ್ತವೆ
ಒದಗಿ-ಸುತ್ತೇವೆ
ಒದಗಿ-ಸುವಷ್ಟರ
ಒದಗಿ-ಸು-ವುದು
ಒದ-ಗುವ
ಒದ-ಗು-ವುದು
ಒದೆಯು-ವನು
ಒಪ್ಪದೆ
ಒಪ್ಪ-ಬೇಕಾ
ಒಪ್ಪ-ಬೇಕು
ಒಪ್ಪ-ಲಾರೆವು
ಒಪ್ಪಲೇ
ಒಪ್ಪಲೇ-ಬೇಕಾ-ಗುತ್ತದೆ
ಒಪ್ಪಿ
ಒಪ್ಪಿ-ಕೊಂಡ
ಒಪ್ಪಿ-ಕೊಂಡನು
ಒಪ್ಪಿ-ಕೊಂಡ-ಮೇಲೆ
ಒಪ್ಪಿ-ಕೊಂಡರೂ
ಒಪ್ಪಿ-ಕೊಂಡರೆ
ಒಪ್ಪಿ-ಕೊಂಡಷ್ಟೇ
ಒಪ್ಪಿ-ಕೊಂಡಿ-ರುವನು
ಒಪ್ಪಿ-ಕೊಂಡಿ-ರು-ವರು
ಒಪ್ಪಿ-ಕೊಂಡಿ-ರು-ವೆವೊ
ಒಪ್ಪಿ-ಕೊಂಡಿವೆ
ಒಪ್ಪಿ-ಕೊಂಡು
ಒಪ್ಪಿ-ಕೊಳ್ಳ
ಒಪ್ಪಿ-ಕೊಳ್ಳದೆ
ಒಪ್ಪಿ-ಕೊಳ್ಳ-ಬಹುದು
ಒಪ್ಪಿ-ಕೊಳ್ಳ-ಬೇಕಾ-ಗಿದೆ
ಒಪ್ಪಿ-ಕೊಳ್ಳ-ಬೇಕಾ-ಗು-ವುದು
ಒಪ್ಪಿ-ಕೊಳ್ಳ-ಬೇಕು
ಒಪ್ಪಿ-ಕೊಳ್ಳ-ಬೇಕೆಂಬ
ಒಪ್ಪಿ-ಕೊಳ್ಳ-ಲಾರರು
ಒಪ್ಪಿ-ಕೊಳ್ಳ-ಲಿಚ್ಛಿ-ಸು-ವು-ದಿಲ್ಲ
ಒಪ್ಪಿ-ಕೊಳ್ಳಲು
ಒಪ್ಪಿ-ಕೊಳ್ಳ-ಲೇ-ಬೇಕಾ-ಯಿತು
ಒಪ್ಪಿ-ಕೊಳ್ಳಿ
ಒಪ್ಪಿ-ಕೊಳ್ಳುತ್ತೇನೆ
ಒಪ್ಪಿ-ಕೊಳ್ಳುತ್ತೇವೆ
ಒಪ್ಪಿ-ಕೊಳ್ಳುವ
ಒಪ್ಪಿ-ಕೊಳ್ಳು-ವಂತೆ
ಒಪ್ಪಿ-ಕೊಳ್ಳು-ವನು
ಒಪ್ಪಿ-ಕೊಳ್ಳು-ವರು
ಒಪ್ಪಿ-ಕೊಳ್ಳು-ವರೊ
ಒಪ್ಪಿ-ಕೊಳ್ಳು-ವು-ದಕ್ಕೆ
ಒಪ್ಪಿ-ಕೊಳ್ಳು-ವು-ದ-ರಿಂದ
ಒಪ್ಪಿ-ಕೊಳ್ಳು-ವು-ದಿಲ್ಲ
ಒಪ್ಪಿ-ಕೊಳ್ಳು-ವುದು
ಒಪ್ಪಿ-ಕೊಳ್ಳು-ವೆವು
ಒಪ್ಪಿ-ಕೊಳ್ಳೋಣ
ಒಪ್ಪಿಗೆ
ಒಪ್ಪಿ-ಗೆಯ
ಒಪ್ಪಿ-ಸ-ಬೇಕು
ಒಪ್ಪಿ-ಸುತ್ತೇನೆ
ಒಪ್ಪಿ-ಸು-ವುದು-ಬುದ್ಧಿಯು
ಒಪ್ಪುತ್ತದೆ
ಒಪ್ಪುತ್ತಲೇ
ಒಪ್ಪುತ್ತಾನೆ
ಒಪ್ಪುತ್ತಾರೆ
ಒಪ್ಪುತ್ತಿ-ರ-ಲಿಲ್ಲ
ಒಪ್ಪುತ್ತೇನೆ
ಒಪ್ಪುತ್ತೇವೆ
ಒಪ್ಪುವ
ಒಪ್ಪುವಂತೆಯೇ
ಒಪ್ಪು-ವರು
ಒಪ್ಪುವ-ವರು
ಒಪ್ಪುವ-ವರೆಗೆ
ಒಪ್ಪು-ವು-ದಿಲ್ಲ
ಒಪ್ಪು-ವು-ದಿಲ್ಲ-ಇದು
ಒಪ್ಪು-ವು-ದಿಲ್ಲವೋ
ಒಪ್ಪು-ವುದು
ಒಬ್ಪರು
ಒಬ್ಬ
ಒಬ್ಬನ
ಒಬ್ಬ-ನನ್ನು
ಒಬ್ಬ-ನಾಗ-ಬಹು-ದೆಂದು
ಒಬ್ಬ-ನಾ-ದರೂ
ಒಬ್ಬ-ನಿ-ಗಿಂತ
ಒಬ್ಬ-ನಿಗೆ
ಒಬ್ಬ-ನಿ-ರುವನು
ಒಬ್ಬನು
ಒಬ್ಬನೂ
ಒಬ್ಬನೇ
ಒಬ್ಬರ
ಒಬ್ಬ-ರನ್ನೊಬ್ಬರು
ಒಬ್ಬ-ರಾದ
ಒಬ್ಬ-ರಿಗೂ
ಒಬ್ಬ-ರಿಗೆ
ಒಬ್ಬರು
ಒಬ್ಬರೂ
ಒಬ್ಬರೇ
ಒಬ್ಬ-ರೊಡನೊಬ್ಬರು
ಒಬ್ಬಳು
ಒಬ್ಬೊಬ್ಬ
ಒಬ್ಬೊಬ್ಬರ
ಒಮ್ಮತ
ಒಮ್ಮೆ
ಒಮ್ಮೆಯೂ
ಒಮ್ಮೆಯೇ
ಒಯ್ದಂತೆ
ಒಯ್ದರು
ಒಯ್ದಾ-ಯಿತು
ಒಯ್ದಿವೆ
ಒಯ್ಯ
ಒಯ್ಯದೆ
ಒಯ್ಯ-ಬೇಕಾ-ಗಿದೆ
ಒಯ್ಯ-ಬೇಕಾ-ದರೆ
ಒಯ್ಯ-ಬೇಕು
ಒಯ್ಯಲ್ಪಡುತ್ತಿ-ರುವೆ
ಒಯ್ಯಲ್ಪಡು-ವುವು
ಒಯ್ಯಿ
ಒಯ್ಯಿರಿ
ಒಯ್ಯುತ್ತದೆ
ಒಯ್ಯುತ್ತವೆ
ಒಯ್ಯುತ್ತಿ-ರುವುದು
ಒಯ್ಯುವ
ಒಯ್ಯು-ವನು
ಒಯ್ಯು-ವರು
ಒಯ್ಯುವ-ವರೆಗೂ
ಒಯ್ಯು-ವಷ್ಟು
ಒಯ್ಯುವು-ದಕ್ಕೋಸುಗ
ಒಯ್ಯುವುದು
ಒಯ್ಯು-ವುವು
ಒಯ್ಯು-ವುವೋ
ಒರಟಾಗಿ
ಒರಟಾ-ಗಿತ್ತು
ಒರಟಾದ
ಒರಟು
ಒರಟು-ತನ
ಒರಟು-ತನ-ವಿ-ರು-ವುದು
ಒಲವು
ಒಳ
ಒಳ-ಕೊಂಡ
ಒಳ-ಕೊಳ್ಳು-ವಂತಿರ
ಒಳಕ್ಕೆ
ಒಳಗಾಗ
ಒಳಗಾ-ಗದು
ಒಳಗಾ-ಗದೆ
ಒಳಗಾ-ಗನು
ಒಳಗಾಗ-ಬೇಕು
ಒಳ-ಗಾಗಿ
ಒಳ-ಗಾಗಿದ್ದೇವೆ
ಒಳ-ಗಾಗಿ-ರುವ
ಒಳ-ಗಾಗಿ-ರು-ವೆನು
ಒಳಗಾ-ಗಿವೆ
ಒಳಗಾಗು
ಒಳಗಾ-ಗುತ್ತದೆ
ಒಳಗಾಗುತ್ತೇವೆ
ಒಳಗಾಗು-ವುದರ
ಒಳ-ಗಾಗು-ವು-ದಿಲ್ಲ
ಒಳ-ಗಾಗು-ವು-ದಿಲ್ಲವೊ
ಒಳಗಾ-ಗು-ವುದು
ಒಳಗಾಗು-ವುದೊ
ಒಳಗಾದ
ಒಳಗಾ-ದರೂ
ಒಳಗಾ-ದರೆ
ಒಳಗಾ-ದಾಗ
ಒಳ-ಗಿನ
ಒಳಗಿ-ನ-ದಾಗಲೀ
ಒಳ-ಗಿ-ನಿಂದ
ಒಳ-ಗಿ-ನಿಂದಲೆ
ಒಳ-ಗಿ-ರುವ
ಒಳಗೆ
ಒಳ-ಗೊಂಡ
ಒಳ-ಗೊಂಡಿದೆ
ಒಳ-ಗೊಂಡಿರುತ್ತವೆ
ಒಳಗೊಳ್ಳ-ಬೇಕಾ-ದರೆ
ಒಳಗೊಳ್ಳುತ್ತವೆ
ಒಳಗೊಳ್ಳುವ
ಒಳ-ತುಂಬು-ವಿಕೆ-ಯಿಂದ
ಒಳ-ಪಟ್ಟ
ಒಳ-ಪಟ್ಟದ್ದು
ಒಳ-ಪಟ್ಟಿದೆ
ಒಳ-ಪಟ್ಟಿ-ರುವು-ದನ್ನೆಲ್ಲಾ
ಒಳ-ಪಟ್ಟಿ-ರು-ವುದು
ಒಳ-ಪಟ್ಟಿ-ರು-ವುವು
ಒಳ-ಪಟ್ಟಿಲ್ಲ
ಒಳಪಡದೆ
ಒಳ-ಪಡ-ಬೇಕು
ಒಳ-ಪಡಿ-ಸ-ಬಹುದು
ಒಳ-ಪಡಿ-ಸಿ-ಕೊಂಡ-ಮೇಲೆ
ಒಳ-ಪಡಿ-ಸಿ-ಕೊಳ್ಳು
ಒಳ-ಪಡು-ವು-ದಲ್ಲ
ಒಳ-ಪಡು-ವು-ದಿಲ್ಲ-ಅವ-ನನ್ನು
ಒಳ-ಪ-ಡು-ವುದು
ಒಳ-ಭಾಗದ
ಒಳ-ಭಾಗ-ದಿಂದ
ಒಳಿತಿ-ನಿಂದ
ಒಳಿತು
ಒಳ್ಳೆ
ಒಳ್ಳೆಯ
ಒಳ್ಳೆ-ಯ-ದಕ್ಕಾಗಿ
ಒಳ್ಳೆ-ಯ-ದಕ್ಕಿಂತ
ಒಳ್ಳೆ-ಯ-ದಕ್ಕೆ
ಒಳ್ಳೆ-ಯ-ದಕ್ಕೊ
ಒಳ್ಳೆ-ಯ-ದನ್ನು
ಒಳ್ಳೆ-ಯ-ದನ್ನೋ
ಒಳ್ಳೆ-ಯ-ದರ
ಒಳ್ಳೆ-ಯ-ದರಂತಿ-ರುವ
ಒಳ್ಳೆ-ಯ-ದ-ರಂತೆ
ಒಳ್ಳೆ-ಯ-ದಲ್ಲ-ವೆಂದು
ಒಳ್ಳೆ-ಯ-ದಾಗ-ಬಹುದು
ಒಳ್ಳೆ-ಯ-ದಾಗ-ಬೇಕು
ಒಳ್ಳೆ-ಯ-ದಾಗಲಿ
ಒಳ್ಳೆ-ಯ-ದಾಗಲೀ
ಒಳ್ಳೆ-ಯ-ದಾಗಿದ್ದರೆ
ಒಳ್ಳೆ-ಯ-ದಾಗಿ-ರ-ಬಹುದು
ಒಳ್ಳೆ-ಯ-ದಾ-ಗಿ-ರುವ
ಒಳ್ಳೆ-ಯ-ದಾ-ಗಿ-ರುವು-ದಕ್ಕಿಂತಲೂ
ಒಳ್ಳೆ-ಯ-ದಾ-ಗಿ-ರುವುದು
ಒಳ್ಳೆ-ಯ-ದಾಗು
ಒಳ್ಳೆ-ಯ-ದಾಗುತ್ತದೆ
ಒಳ್ಳೆ-ಯ-ದಾಗು-ವುದು
ಒಳ್ಳೆ-ಯ-ದಾಗು-ವು-ದೆಂದು
ಒಳ್ಳೆ-ಯ-ದಾದರೆ
ಒಳ್ಳೆ-ಯ-ದಾ-ವುದು
ಒಳ್ಳೆ-ಯ-ದಿದ್ದರೆ
ಒಳ್ಳೆ-ಯ-ದಿರು-ವು-ದಿಲ್ಲ
ಒಳ್ಳೆ-ಯ-ದಿಲ್ಲದೆ
ಒಳ್ಳೆ-ಯದು
ಒಳ್ಳೆ-ಯ-ದುಂಟೊ
ಒಳ್ಳೆ-ಯದೂ
ಒಳ್ಳೆ-ಯದೆ
ಒಳ್ಳೆ-ಯ-ದೆಂದು
ಒಳ್ಳೆ-ಯ-ದೆಂದೂ
ಒಳ್ಳೆ-ಯ-ದೆಂಬುದು
ಒಳ್ಳೆ-ಯದೇ
ಒಳ್ಳೆ-ಯ-ವ-ನಾಗ-ಬೇಕು
ಒಳ್ಳೆ-ಯ-ವ-ನಾಗಿ-ರ-ಬಹುದು
ಒಳ್ಳೆ-ಯ-ವ-ನಾಗುತ್ತಾನೆ
ಒಳ್ಳೆ-ಯ-ವನು
ಒಳ್ಳೆ-ಯ-ವ-ರಲ್ಲಿ
ಒಳ್ಳೆ-ಯ-ವ-ರಾಗಿ
ಒಳ್ಳೆ-ಯ-ವ-ರಾಗಿ-ಬಿಟ್ಟರೆ
ಒಳ್ಳೆ-ಯ-ವ-ರಾಗಿ-ರಲಿ
ಒಳ್ಳೆ-ಯ-ವ-ರಾಗಿ-ರು-ವರು
ಒಳ್ಳೆ-ಯ-ವರು
ಒಳ್ಳೆ-ಯವು
ಒಳ್ಳೆ-ಯವೆ
ಓ
ಓಂ
ಓಂಕಾರ
ಓಕ್
ಓಜಸ್ಸನ್ನಾಗಿ
ಓಜಸ್ಸನ್ನು
ಓಜಸ್ಸಾಗಿ
ಓಜಸ್ಸಿನ
ಓಜಸ್ಸು
ಓಡ-ಬಲ್ಲೆ
ಓಡ-ಬೇಕು
ಓಡಾಡುತ್ತಿ-ರುವುದು
ಓಡಿ
ಓಡಿತು
ಓಡಿದ
ಓಡಿ-ದರೆ
ಓಡಿದೆ
ಓಡಿನ್ನಿ
ಓಡಿ-ಸ-ಬೇಕೆಂದೆ
ಓಡಿ-ಸಿ-ದರೆ
ಓಡಿ-ಸುತ್ತದೆ
ಓಡುತ್ತಿದೆ
ಓಡುತ್ತಿ-ರುವುದು
ಓಡುವ
ಓಡು-ವಾಗ
ಓಡು-ವುದನ್ನು
ಓಡು-ವುದು
ಓಡು-ವೆವು
ಓತಪ್ರೋತ-ನಾಗಿ-ರುವನು
ಓದ-ಬಹುದು
ಓದ-ಬಹು-ದೆಂದು
ಓದ-ಬೇಕೆಂದಿ-ರುವ
ಓದಲು
ಓದಿ
ಓದಿ-ಕೊಳ್ಳ-ಬೇಕು
ಓದಿದ
ಓದಿ-ದರೆ
ಓದಿ-ದವ-ರಿಗೆ
ಓದಿ-ದಾಗ
ಓದಿದೆ
ಓದಿದ್ದೆ
ಓದಿ-ಯಾದ
ಓದಿರ-ಬಹುದು
ಓದಿ-ರುವಿರಿ
ಓದಿ-ರುವಿರೋ
ಓದಿ-ರುವೆವು
ಓದಿಲ್ಲ
ಓದಿ-ಹೇಳುತ್ತೇನೆ
ಓದು
ಓದುತ್ತಿದ್ದರೆ
ಓದುತ್ತಿರು
ಓದುತ್ತಿರು-ವಾಗ
ಓದುತ್ತಿ-ರು-ವಿರಿ
ಓದುತ್ತಿರು-ವೆನು
ಓದುತ್ತೀರಿ
ಓದುತ್ತೇನೆ
ಓದುತ್ತೇವೆ
ಓದುವ
ಓದು-ವಾಗ
ಓದು-ವು-ದಕ್ಕೆ
ಓದು-ವು-ದ-ರಲ್ಲಿ
ಓದು-ವು-ದ-ರಿಂದ
ಓದು-ವು-ದಲ್ಲ-ಜೀವನ್ಮುಕ್ತಿ-ಯನ್ನು
ಓದು-ವುದು
ಓದು-ವೆವು
ಓಪಿಯಂ
ಓಪಿಯಂಮಯ-ವಾದ
ಓಲಾಡುತ್ತಿದೆ-ಆಧು-ನಿಕ
ಓಲೆ
ಓಲೆ-ಗರಿ
ಓಲೆ-ಗರಿಯ
ಓಲೆ-ಯಾಗಿ
ಔದ್ಧತ್ಯ
ಔಪ-ಚಾರಿಕ
ಔಷಧ
ಔಷಧ-ಗಳ
ಔಷಧದ
ಔಷಧ-ದಲ್ಲಿ
ಔಷಧಾಲಯ
ಔಷಧಾಲಯ-ಗಳು
ಔಷಧಿ
ಔಷಧಿ-ಗಳ
ಔಷಧಿ-ಗಳು
ಔಷ-ಧಿಯ
ಔಷಧಿ-ಯನ್ನು
ಕಂಗೊಳಿ-ಸ-ಬೇಕು
ಕಂಗೊಳಿಸು
ಕಂಗೊಳಿ-ಸುತ್ತಿದೆ
ಕಂಗೊಳಿ-ಸುತ್ತಿ-ರ-ಬಹುದು
ಕಂಗೊಳಿ-ಸುತ್ತಿ-ರುವ
ಕಂಗೊಳಿ-ಸುತ್ತಿ-ರುವನು
ಕಂಗೊಳಿ-ಸುವ
ಕಂಟಕ-ಗಳು
ಕಂಟಕಪ್ರಾಯ-ವಾಗಿ
ಕಂಠಕೂಪೇ
ಕಂಠನಾಳ
ಕಂಡ
ಕಂಡಂ
ಕಂಡಂತೆ
ಕಂಡದ್ದನ್ನು
ಕಂಡನು
ಕಂಡರು
ಕಂಡರೂ
ಕಂಡರೆ
ಕಂಡರೋ
ಕಂಡ-ವನು
ಕಂಡಾಗ
ಕಂಡಿದೆ
ಕಂಡಿದ್ದರೆ
ಕಂಡಿರ
ಕಂಡಿ-ರು-ವರು
ಕಂಡಿಲ್ಲ
ಕಂಡಿಲ್ಲ-ವಾ-ದರೂ
ಕಂಡು
ಕಂಡು-ಕೊಂಡ
ಕಂಡು-ಕೊಂಡನು
ಕಂಡು-ಕೊಂಡರು
ಕಂಡು-ಕೊಂಡಾಗ
ಕಂಡು-ಕೊಂಡಿದ್ದರೂ
ಕಂಡು-ಕೊಂಡಿ-ರುವನು
ಕಂಡು-ಕೊಂಡು
ಕಂಡು-ಕೊಳ್ಳ-ಬೇಕು
ಕಂಡು-ಕೊಳ್ಳು
ಕಂಡು-ಕೊಳ್ಳುತ್ತಾರೆ
ಕಂಡು-ಕೊಳ್ಳು-ವನು
ಕಂಡು-ಕೊಳ್ಳು-ವರು
ಕಂಡು-ಬಂತು
ಕಂಡು-ಬ-ರುತ್ತದೆ
ಕಂಡು-ಬರುತ್ತವೆ
ಕಂಡು-ಬ-ರುವ
ಕಂಡು-ಬರು-ವು-ದಿಲ್ಲ
ಕಂಡು-ಬ-ರುವುದು
ಕಂಡು-ಬ-ರುವುದೊ
ಕಂಡು-ಹಿಡಿ
ಕಂಡು-ಹಿಡಿದ
ಕಂಡು-ಹಿಡಿ-ದನು
ಕಂಡು-ಹಿಡಿ-ದರು
ಕಂಡು-ಹಿಡಿ-ದ-ರು-ಒಂದ
ಕಂಡು-ಹಿಡಿ-ದ-ವರು
ಕಂಡು-ಹಿಡಿ-ದಷ್ಟೂ
ಕಂಡು-ಹಿಡಿ-ದಾಗ
ಕಂಡು-ಹಿಡಿ-ದಿ-ರು-ವರು
ಕಂಡು-ಹಿಡಿ-ದಿ-ರು-ವರೋ
ಕಂಡು-ಹಿಡಿ-ದಿ-ರು-ವೆನು
ಕಂಡು-ಹಿಡಿ-ದಿಲ್ಲ
ಕಂಡು-ಹಿಡಿದು
ಕಂಡು-ಹಿಡಿ-ದು-ಕೊಳ್ಳುತ್ತಾನೆ
ಕಂಡು-ಹಿಡಿಯ
ಕಂಡು-ಹಿಡಿ-ಯ-ಬಹು-ದಾದ
ಕಂಡು-ಹಿಡಿ-ಯ-ಬಹುದು
ಕಂಡು-ಹಿಡಿ-ಯ-ಬೇಕು
ಕಂಡು-ಹಿಡಿ-ಯ-ಬೇಕೆಂಬುದೇ
ಕಂಡು-ಹಿಡಿ-ಯ-ಲಾರಿರಿ
ಕಂಡು-ಹಿಡಿ-ಯಲು
ಕಂಡು-ಹಿಡಿ-ಯಿರಿ
ಕಂಡು-ಹಿಡಿಯು
ಕಂಡು-ಹಿಡಿ-ಯುತ್ತಾ
ಕಂಡು-ಹಿಡಿ-ಯುತ್ತಾನೆ
ಕಂಡು-ಹಿಡಿ-ಯುತ್ತೇವೆ
ಕಂಡು-ಹಿಡಿ-ಯು-ವನು
ಕಂಡು-ಹಿಡಿ-ಯು-ವರು
ಕಂಡು-ಹಿಡಿ-ಯು-ವಿರಿ
ಕಂಡು-ಹಿಡಿ-ಯು-ವು-ದ-ರಿಂದ
ಕಂಡು-ಹಿಡಿ-ಯು-ವು-ದ-ರಿಂದಲೂ
ಕಂಡು-ಹಿಡಿ-ಯು-ವುದು
ಕಂಡು-ಹಿಡಿ-ಯು-ವುದೇ
ಕಂಡೆ
ಕಂಡೊ-ಡ-ನೆಯೆ
ಕಂಡೊಡ-ನೆಯೇ
ಕಂತೆ
ಕಂತೆ-ಪುರಾಣ
ಕಂದಾಯ
ಕಂಪನ
ಕಂಪ-ವಿ-ರಲಿ
ಕಂಪಿ-ಸು-ವುದು
ಕಂಬನಿ-ಗರೆದು
ಕಂಬಳಿಯ
ಕಗ್ಗತ್ತಲಿಂದ
ಕಚೇರಿ-ಗಳು
ಕಚ್ಚಲು
ಕಚ್ಚಿ-ದರೆ
ಕಚ್ಚಿದೆ
ಕಚ್ಚುತ್ತಿದೆ
ಕಟು
ಕಟು-ಕನ
ಕಟ್ಟ-ಕ-ಡೆಗೆ
ಕಟ್ಟ-ಕಡೆಯ
ಕಟ್ಟಡ
ಕಟ್ಟಡ-ಗಳಲ್ಲಿ
ಕಟ್ಟಡ-ವನ್ನು
ಕಟ್ಟಡ-ವಲ್ಲ
ಕಟ್ಟ-ಬಹುದು
ಕಟ್ಟ-ಬೇಕು
ಕಟ್ಟ-ಲಿಲ್ಲ
ಕಟ್ಟಲು
ಕಟ್ಟಿ
ಕಟ್ಟಿ-ಕೊಂಡು
ಕಟ್ಟಿದ
ಕಟ್ಟಿ-ದಂತೆ
ಕಟ್ಟಿದ್ದ
ಕಟ್ಟಿ-ಸ-ಬೇಕು
ಕಟ್ಟು
ಕಟ್ಟು-ಕಥೆ
ಕಟ್ಟು-ಕಥೆ-ಗಳೆಲ್ಲ
ಕಟ್ಟು-ಕಥೆ-ಯನ್ನು
ಕಟ್ಟುತ್ತದೆ
ಕಟ್ಟುತ್ತಾನೆ
ಕಟ್ಟುತ್ತಿದ್ದರು
ಕಟ್ಟುತ್ತಿ-ರು-ವಿರಿ
ಕಟ್ಟು-ನಿಟ್ಟು
ಕಟ್ಟುವ
ಕಟ್ಟು-ವು-ದಿಲ್ಲ
ಕಟ್ಟು-ವುದು
ಕಟ್ಟೆಂದು
ಕಠಿಣ
ಕಠಿಣ-ತೆ-ಇವು
ಕಠಿಣ-ವಾಗ-ಲ-ಬಲ್ಲದು
ಕಠಿಣ-ವಾದ
ಕಠಿಣ-ವಾದಿ
ಕಠಿ-ನವೇ
ಕಠೋಪ
ಕಠೋಪ-ನಿಷತ್
ಕಠೋಪ-ನಿಷತ್ತು
ಕಠೋರ
ಕಠೋರ-ವಾಗಿ
ಕಡಗ
ಕಡಮೆ
ಕಡಮೆ-ಮಾಡು-ವು-ದ-ರಿಂದ
ಕಡಮೆ-ಯಾಗ-ಕೂಡದು
ಕಡಮೆ-ಯಾಗದ
ಕಡಮೆ-ಯಾಗಿ
ಕಡಮೆ-ಯಾಗುತ್ತ
ಕಡಮೆ-ಯಾಗುತ್ತದೆ
ಕಡಮೆ-ಯಾಗುತ್ತಾ
ಕಡಮೆ-ಯಾಗುತ್ತಿತ್ತು
ಕಡಮೆ-ಯಾಗುತ್ತಿದೆ
ಕಡಮೆ-ಯಾಗುತ್ತಿ-ರ-ಬೇಕು
ಕಡಮೆ-ಯಾಗುತ್ತಿವೆ
ಕಡಮೆ-ಯಾ-ಗು-ವುದು
ಕಡಮೆ-ಯಾದ
ಕಡಮೆ-ಯಾ-ದಂತೆ
ಕಡಮೆ-ಯಾ-ದರೂ
ಕಡಮೆ-ಯಾ-ಯಿತು-ಈಗ
ಕಡಮೆ-ಯಿದ್ದರೆ
ಕಡ-ಲಿನ
ಕಡಲಿ-ನಲ್ಲಿ-ರುವ
ಕಡಲಿ-ನಾಚೆ
ಕಡಲಿ-ನಾಳ-ವನ್ನು
ಕಡಲೇ-ಚಿತ್ತ
ಕಡಿ-ದರೆ
ಕಡಿಮೆ
ಕಡಿಮೆ-ಅ-ವನು
ಕಡಿಮೆ-ಮಾ-ಡಲು
ಕಡಿಮೆ-ಯಲ್ಲಿ
ಕಡಿಮೆ-ಯಾಗ-ಬೇಕು
ಕಡಿಮೆ-ಯಾಗಿ
ಕಡಿಮೆ-ಯಾಗುತ್ತಾ
ಕಡಿಮೆ-ಯಾಗುತ್ತಿದೆ
ಕಡಿಮೆ-ಯಾಗುತ್ತಿದ್ದರೆ
ಕಡಿಮೆ-ಯಾಗು-ವು-ದಿಲ್ಲ
ಕಡಿಮೆ-ಯಾ-ಗು-ವುದು
ಕಡಿಮೆ-ಯಾ-ದರೂ
ಕಡಿಮೆ-ಯಾ-ದಷ್ಟೂ
ಕಡಿಮೆ-ಯಾ-ದಾಗ
ಕಡಿಮೆ-ಯಿಂದ
ಕಡಿಮೆಯೂ
ಕಡಿ-ಯುವ
ಕಡಿವಾಣ
ಕಡಿವಾಣ-ವನ್ನು
ಕಡು-ಮೂಢ-ನಂಬಿಕೆ
ಕಡೆ
ಕಡೆ-ಗಣಿ-ಸಲು
ಕಡೆ-ಗಣಿ-ಸಿದ-ವರೇ
ಕಡೆ-ಗಣಿ-ಸು-ವರು
ಕಡೆ-ಗಳಂತೆ
ಕಡೆ-ಗಳಲ್ಲಿ
ಕಡೆ-ಗಳಲ್ಲಿಯೂ
ಕಡೆ-ಗಳಲ್ಲೂ
ಕಡೆ-ಗಳಿಂದ
ಕಡೆ-ಗಳಿಂದಲೂ
ಕಡೆ-ಗಿನ
ಕಡೆಗೂ
ಕಡೆಗೆ
ಕಡೆಗೇ
ಕಡೆಯ
ಕಡೆ-ಯಲ್ಲಿ
ಕಡೆ-ಯಲ್ಲಿಯೂ
ಕಡೆ-ಯಿಂದ
ಕಡೆಯೂ
ಕಡ್ಡಿ-ಯನ್ನು
ಕಣ
ಕಣ-ಕಣಕ್ಕೂ
ಕಣಕ್ಕೆ
ಕಣ-ಗ-ಲನ್ನು
ಕಣ-ಗಳ
ಕಣ-ಗ-ಳನ್ನು
ಕಣ-ಗಳನ್ನೆಲ್ಲ
ಕಣ-ಗಳಲ್ಲಿ
ಕಣ-ಗಳಾಗಿ
ಕಣ-ಗಳಾ-ಗು-ವುವು
ಕಣ-ಗಳಿಂದ
ಕಣ-ಗಳಿಗೆ
ಕಣ-ಗಳಿ-ಗೆಲ್ಲಾ
ಕಣ-ಗಳು
ಕಣದ
ಕಣ-ದಂತೆ
ಕಣ-ದಲ್ಲಿಯೂ
ಕಣ-ದ-ವರೆ-ವಿಗೂ
ಕಣ-ದಷ್ಟು
ಕಣ-ದಿಂದ
ಕಣ-ರಾಶಿ
ಕಣ-ವನ್ನಾ-ಗಲಿ
ಕಣ-ವನ್ನೂ
ಕಣವು
ಕಣವೂ
ಕಣ-ವೊಂದೂ
ಕಣಾದಿ-ಗಳೆಂಬ
ಕಣಿವೆಗೆ
ಕಣಿವೆ-ಯಲ್ಲಿ
ಕಣ್ಣ
ಕಣ್ಣನ್ನು
ಕಣ್ಣನ್ನೊ
ಕಣ್ಣ-ಮುಂದೆ
ಕಣ್ಣಲ್ಲ
ಕಣ್ಣಾರೆ
ಕಣ್ಣಿಗೆ
ಕಣ್ಣಿನ
ಕಣ್ಣಿ-ನಲ್ಲಿ
ಕಣ್ಣಿನಲ್ಲಿ-ರುವ
ಕಣ್ಣಿ-ನಿಂದ
ಕಣ್ಣಿಲ್ಲ
ಕಣ್ಣೀರಿ
ಕಣ್ಣು
ಕಣ್ಣು-ಗಳ
ಕಣ್ಣು-ಗಳನ್ನೆಲ್ಲ
ಕಣ್ಣು-ಗಳಿ
ಕಣ್ಣು-ಗಳಿಂದ
ಕಣ್ಣು-ಗಳಿದ್ದರೂ
ಕಣ್ಣು-ಗಳಿವೆ
ಕಣ್ಣು-ಗಳಿ-ವೆಯೊ
ಕಣ್ಣು-ಗಳು
ಕಣ್ಣು-ಗಳುಳ್ಳ-ವರು
ಕಣ್ಣು-ಗಳೇ
ಕಣ್ಣು-ಮುಚ್ಚು-ವು-ದ-ರಿಂದ
ಕಣ್ಣು-ಮೇಲೆ
ಕಣ್ಣೆ
ಕಣ್ಣೆ-ದು-ರಿಗೆ
ಕಣ್ತೆರೆದು
ಕಣ್ತೆರೆ-ಯುವ
ಕಣ್ಮರೆ-ಯಾ-ಗಿಲ್ಲ-ವೆಂಬು-ದನ್ನು
ಕಣ್ಮರೆ-ಯಾಗು-ವರು
ಕತೆ
ಕತೆಯ
ಕತೆ-ಯನ್ನು
ಕತ್ತನ್ನು
ಕತ್ತರಿಸ
ಕತ್ತರಿಸ-ಬಲ್ಲುದು
ಕತ್ತರಿಸ-ಬಹುದು
ಕತ್ತರಿಸ-ಬೇಕಾ-ಯಿತು
ಕತ್ತರಿಸ-ಲಾರದು
ಕತ್ತ-ರಿಸಿ-ಕೊಳ್ಳು-ವಿರಿ
ಕತ್ತ-ರಿಸಿ-ದರೂ
ಕತ್ತ-ರಿಸಿ-ದರೆ
ಕತ್ತ-ರಿಸಿ-ದಾಗಲೂ
ಕತ್ತಲಾ-ಗು-ವುದು
ಕತ್ತಲು-ಬೆಳ-ಕು-ಗಳ
ಕತ್ತಲೆ
ಕತ್ತಲೆ-ಗಳ
ಕತ್ತಲೆಯ
ಕತ್ತಲೆ-ಯಂತೆ-ಕೊ-ನೆಯ
ಕತ್ತಲೆ-ಯಲ್ಲಿ
ಕತ್ತಲೆ-ಯಲ್ಲೋ
ಕತ್ತಲೆ-ಯಿಂದ
ಕತ್ತಲೆಯೂ
ಕತ್ತಲೆ-ಯೆಂದು
ಕತ್ತಿ
ಕತ್ತಿ-ಕೋವಿ-ಗಳ
ಕತ್ತಿಗೆ
ಕತ್ತಿಯ
ಕತ್ತು
ಕಥಾ-ವಸ್ತು-ವಾಗಿ
ಕಥೆ
ಕಥೆ-ಗಳ
ಕಥೆ-ಗಳಲ್ಲಿ
ಕಥೆ-ಗಳಲ್ಲೆಲ್ಲಾ
ಕಥೆ-ಗಳು
ಕಥೆಯ
ಕಥೆ-ಯನ್ನು
ಕಥೆ-ಯಲ್ಲ
ಕಥೆ-ಯಲ್ಲಿ
ಕಥೆ-ಯಲ್ಲಿ-ರುವ
ಕಥೆ-ಯಿಂದ
ಕಥೆ-ಯಿದೆ
ಕಥೆಯೇ
ಕಥೆ-ಯೇನೋ
ಕಥೆ-ಯೊಂದಿದೆ
ಕಥೆ-ಯೊಂದು
ಕದ-ಡಿ-ಹೋ-ಗಿ-ರುವ
ಕದ-ನ-ಗಳ
ಕದ-ಲ-ದಂತೆ
ಕದ-ಲಿ-ಸದೆ
ಕದ-ಲಿಸ-ಬಲ್ಲ
ಕದಿ-ಯದ
ಕದಿ-ಯದೆ
ಕದಿಯ-ಬೇಡ
ಕದಿಯುತ್ತೇವೆ
ಕದಿಯು-ವು-ದ-ರಿಂದಲೂ
ಕದ್ದರೆ
ಕದ್ದು
ಕನ-ಸನ್ನು
ಕನ-ಸನ್ನೆಲ್ಲಾ
ಕನ-ಸಾಗಿ
ಕನ-ಸಾಗಿರ
ಕನ-ಸಾದ
ಕನ-ಸಿ-ಗಿಂತ
ಕನ-ಸಿನ
ಕನ-ಸಿ-ನಂದೆ
ಕನ-ಸಿ-ನಲ್ಲಿ
ಕನ-ಸಿನಲ್ಲಿದ್ದೆ
ಕನ-ಸಿನಲ್ಲಿ-ಯಾ-ದರೂ
ಕನ-ಸಿ-ನಿಂದ
ಕನಸು
ಕನ-ಸು-ಗಳ
ಕನ-ಸು-ಗಳು
ಕನ-ಸು-ಣಿ-ಯಾ-ಗು-ವುದು
ಕನಸೆ
ಕನ-ಸೆಂದು
ಕನಸೇ
ಕನಿಷ್ಠ
ಕನಿಷ್ಠ-ವಾದ
ಕನಿಷ್ಠ-ವಾ-ದಾಗ
ಕನ್ನಡಿ-ಯನ್ನು
ಕನ್ನಡಿ-ಯಲ್ಲಿ
ಕನ್ನಡಿ-ಯಿಂದ
ಕನ್ಫೂಷಿಯಸ್ನ
ಕನ್ಯೆ-ಯರು
ಕಪಟ
ಕಪಟ-ಬದು-ಕನ್ನೂ
ಕಪಟಿ-ಗಳಾ-ಗ-ದಿರೋಣ
ಕಪಟಿ-ಗಳಾ-ಗ-ಬಹುದು
ಕಪಟಿ-ಗಳಿಗೆ
ಕಪಿ
ಕಪಿ-ಗಳಂತೆ
ಕಪಿಗೆ
ಕಪಿಯ
ಕಪಿ-ಯಂತೆ
ಕಪಿಲ
ಕಪಿ-ಲರು
ಕಪ್ಪಗೊ
ಕಪ್ಪು
ಕಪ್ಪೂ
ಕಪ್ಪೆ
ಕಪ್ಪೆ-ಚಿಪ್ಪಿನಲ್ಲಿ
ಕಪ್ಪೆಯ
ಕಪ್ಪೆ-ಯ-ಚಿಪ್ಪಿನ
ಕಪ್ಪೆ-ಯ-ಚಿಪ್ಪಿನಂತೆ
ಕಬ್ಬಿ-ಣದ
ಕಮಲ
ಕಮಲ-ಗ-ಳನ್ನು
ಕಮಲ-ಗಳು
ಕಮ-ಲದ
ಕಮಲ-ದಲ್ಲಿ
ಕಮಲ-ವನ್ನು
ಕಮಲವು
ಕರ-ಗತ-ವಾಗ-ಲಾರದು
ಕರ-ಗತ-ವಾ-ಗು-ವುದು
ಕರ-ಗತ-ವಾ-ಯಿತು
ಕರ-ಗಳ
ಕರ-ಗಳು
ಕರಗಿ
ಕರ-ಗಿ-ದಾಗ
ಕರಣ
ಕರ-ಣ-ಗ-ಳಾದ
ಕರ-ಣ-ಗಳಾ-ವುವು
ಕರ-ಣ-ದಂತಿದೆ
ಕರ-ಣವು
ಕರ-ಣ-ವೆಂದು
ಕರ-ಣವೇ
ಕರನ್ಯಾಸ
ಕರ-ವಾಗಿದೆ
ಕರ-ವಾದು-ವಲ್ಲವೊ
ಕರು-ಣಾ-ಪೂರಿತ
ಕರು-ಣಿಸಿ
ಕರು-ಣಿ-ಸಿದ
ಕರು-ಣಿಸಿ-ದನು
ಕರು-ಣಿಸುತ್ತಾನೆ
ಕರು-ಣಿ-ಸುತ್ತಿದ್ದರು
ಕರು-ಣಿ-ಸುತ್ತಿ-ರುವನು
ಕರು-ಣಿ-ಸು-ವನು
ಕರುಣೆ
ಕರು-ಣೆಗೆ
ಕರು-ಹಾಕ-ಲಾರದ
ಕರೆ
ಕರೆ-ತಂದು
ಕರೆದ
ಕರೆ-ದರು
ಕರೆ-ದರೂ
ಕರೆ-ದಿದ್ದಾರೆ
ಕರೆ-ದಿ-ರು-ವರು
ಕರೆದು
ಕರೆ-ದು-ಕೊಂಡು
ಕರೆ-ದು-ಕೊಳ್ಳಿ
ಕರೆ-ದೊಯ್ಯ-ಬಲ್ಲವು
ಕರೆ-ದೊಯ್ಯುವ
ಕರೆ-ದೊಯ್ಯುವು-ದಕ್ಕಲ್ಲದೆ
ಕರೆ-ದೊಯ್ಯುವುದು
ಕರೆ-ದೊಯ್ಯುವುದೋ
ಕರೆ-ದೊಯ್ಯು-ವುವು
ಕರೆ-ದೊಯ್ಯುವೆ
ಕರೆ-ಯದೆ
ಕರೆ-ಯ-ಬಹುದು
ಕರೆ-ಯ-ಬ-ಹುದೇ
ಕರೆ-ಯ-ಬೇಕೆಂದು
ಕರೆ-ಯಲಾ-ಗಿದೆ
ಕರೆ-ಯ-ಲಾರಿರಿ
ಕರೆ-ಯ-ಲಾರೆವು
ಕರೆ-ಯಲ್ಪ-ಡುವ
ಕರೆ-ಯಿರಿ
ಕರೆಯು
ಕರೆ-ಯುತ್ತ
ಕರೆ-ಯುತ್ತದೆ
ಕರೆ-ಯುತ್ತಾ
ಕರೆ-ಯುತ್ತಾರೆ
ಕರೆ-ಯುತ್ತಾ-ರೆಯೋ
ಕರೆ-ಯುತ್ತಿತ್ತು
ಕರೆ-ಯುತ್ತಿದ್ದರು
ಕರೆ-ಯುತ್ತಿವೆ
ಕರೆ-ಯುತ್ತೀ-ರಲ್ಲ
ಕರೆ-ಯುತ್ತೀರೋ
ಕರೆ-ಯುತ್ತೇನೆ
ಕರೆ-ಯುತ್ತೇನೆಯೋ
ಕರೆ-ಯುತ್ತೇವೆ
ಕರೆ-ಯುತ್ತೇವೆಯೊ
ಕರೆ-ಯುತ್ತೇವೆಯೋ
ಕರೆ-ಯುವ
ಕರೆ-ಯು-ವಂತೆ
ಕರೆ-ಯು-ವನು
ಕರೆ-ಯು-ವರು
ಕರೆ-ಯು-ವಳು
ಕರೆ-ಯುವು
ಕರೆ-ಯು-ವು-ದಕ್ಕೆ
ಕರೆ-ಯು-ವು-ದರ
ಕರೆ-ಯು-ವು-ದಿಲ್ಲ
ಕರೆ-ಯು-ವುದು
ಕರೆ-ಯು-ವೆವು
ಕರೆ-ಯೊಂದೆ
ಕರೆವ
ಕರೆ-ಸಿ-ಕೊಂಡ-ವನೆ
ಕರ್ತವ್ಯ
ಕರ್ತವ್ಯ-ಗಳ
ಕರ್ತವ್ಯ-ಭಾವನೆ
ಕರ್ತವ್ಯ-ವನ್ನು
ಕರ್ತವ್ಯ-ವಾ-ಗಿತ್ತು
ಕರ್ತವ್ಯ-ವಿದೆ
ಕರ್ತವ್ಯ-ವೆಂದು
ಕರ್ತೃ-ಗಳು
ಕರ್ತೃ-ವಾದ
ಕರ್ತೃವೇ
ಕರ್ಮ
ಕರ್ಮ-ಕಷ್ಟ-ಗಳು
ಕರ್ಮ-ಕಾಂಡ
ಕರ್ಮ-ಕಾಂಡ-ಗಳೆಂಬ
ಕರ್ಮ-ಕಾಂಡದ
ಕರ್ಮ-ಕಾಂಡವು
ಕರ್ಮ-ಕಾಂಡವೇ
ಕರ್ಮಕ್ಕಾಗಿ
ಕರ್ಮಕ್ಕೆ
ಕರ್ಮಕ್ಷಯ-ವಾಗು
ಕರ್ಮ-ಗಳ
ಕರ್ಮ-ಗ-ಳನ್ನು
ಕರ್ಮ-ಗಳಿಂದ
ಕರ್ಮ-ಗಳಿಗೆ
ಕರ್ಮ-ಗಳು
ಕರ್ಮ-ಗಳೇ
ಕರ್ಮ-ಜೀವ-ನ-ವನ್ನು
ಕರ್ಮ-ತಂತು-ವಿ-ನಿಂದ
ಕರ್ಮದ
ಕರ್ಮ-ದಲ್ಲಿ
ಕರ್ಮ-ದಿಂದ
ಕರ್ಮ-ನಿರತ-ರಾ-ದರೆ
ಕರ್ಮ-ನಿಷ್ಠೆ-ಯಿಂದ
ಕರ್ಮ-ಪಟು
ಕರ್ಮ-ಪಟು-ಗಳು
ಕರ್ಮ-ಪಟು-ಗಳೆಂಬು-ದನ್ನು
ಕರ್ಮ-ಪರಿ-ಣಾಮವೂ
ಕರ್ಮ-ಫಲ
ಕರ್ಮ-ಫಲ-ಗಳ
ಕರ್ಮ-ಫಲ-ಗ-ಳನ್ನು
ಕರ್ಮ-ಫಲ-ಗಳು
ಕರ್ಮ-ಫಲ-ಗಳೆ
ಕರ್ಮ-ಫಲ-ದಿಂದ
ಕರ್ಮ-ಫಲ-ವನ್ನು
ಕರ್ಮ-ಬಲೆ-ಯನ್ನು
ಕರ್ಮ-ಭೂಮಿ
ಕರ್ಮ-ಭೂಮಿಯೇ
ಕರ್ಮ-ಮಾ-ಡಲು
ಕರ್ಮ-ಮಾಡು-ವುದು
ಕರ್ಮ-ಯೋಗ
ಕರ್ಮ-ಯೋಗ-ದಿಂದಾದಲಿ
ಕರ್ಮ-ಯೋಗಿ
ಕರ್ಮ-ಯೋಗಿಯ
ಕರ್ಮ-ರಹಸ್ಯ
ಕರ್ಮ-ವನ್ನು
ಕರ್ಮ-ವನ್ನೂ
ಕರ್ಮ-ವನ್ನೇ
ಕರ್ಮ-ವಾಗು-ವು-ದಿಲ್ಲ
ಕರ್ಮ-ವಿಲ್ಲದೆ
ಕರ್ಮವು
ಕರ್ಮವೂ
ಕರ್ಮ-ವೆಂದು
ಕರ್ಮ-ವೆಂಬ
ಕರ್ಮವೇ
ಕರ್ಮ-ವೇಗ
ಕರ್ಮ-ಸಿದ್ಧಾಂತ
ಕರ್ಮಾಶ-ಯವು
ಕರ್ಮಾಶಯ-ವೆಂದರೆ
ಕರ್ಮಾಶಯೋ
ಕರ್ಮಾಶುಕ್ಲಾ-ಕೃಷ್ಣಂ
ಕರ್ಮಿ
ಕರ್ಮೇಂದ್ರಿಯ
ಕರ್ಮೇಂದ್ರಿಯ-ಗಳಿಗೆ
ಕರ್ಮೇಂದ್ರಿ-ಯದ
ಕಲಕ-ಬೇಡಿ
ಕಲಕಿ-ದಾಗ
ಕಲ-ಕಿದ್ದರೆ
ಕಲಸು
ಕಲಹ
ಕಲಹ-ಗಳಿದ್ದೇ
ಕಲಹಾತ್ಮಕ
ಕಲಾಪ-ಗಳು
ಕಲಿತ
ಕಲಿತ-ಮೇಲೆ
ಕಲಿ-ತರು
ಕಲಿ-ತರೆ
ಕಲಿ-ತರೊ
ಕಲಿ-ತವು
ಕಲಿತಿ-ರು-ವುದು
ಕಲಿತು-ಕೊಳ್ಳ-ಬೇಕಾ-ಗಿದೆ
ಕಲಿತು-ಕೊಳ್ಳ-ಬೇಕು
ಕಲಿತು-ಕೊಳ್ಳುತ್ತೇವೆ
ಕಲಿತು-ಕೊಳ್ಳು-ವರು
ಕಲಿತು-ಕೊಳ್ಳು-ವಾಗ
ಕಲಿತು-ಕೊಳ್ಳು-ವು-ದಕ್ಕೆ
ಕಲಿತು-ಕೊಳ್ಳು-ವು-ದಿಲ್ಲ
ಕಲಿತು-ಕೊಳ್ಳು-ವುದು
ಕಲಿತೆವು
ಕಲಿಯ
ಕಲಿ-ಯದ
ಕಲಿಯ-ಬಹು-ದೆಂದು
ಕಲಿಯ-ಬೇಕಾ-ಗಿದೆ
ಕಲಿಯ-ಬೇಕಾದ
ಕಲಿಯ-ಬೇಕಾ-ದರೆ
ಕಲಿಯ-ಬೇಕಾದು
ಕಲಿಯ-ಬೇಕಾ-ದುದು
ಕಲಿಯ-ಬೇಕು
ಕಲಿಯ-ಲಾಗು
ಕಲಿ-ಯಲು
ಕಲಿಯುತ್ತೀರಿ
ಕಲಿಯು-ವನು
ಕಲಿಯು-ವು-ದಕ್ಕೆ
ಕಲಿಯು-ವೆವು
ಕಲಿಯು-ವೆವೊ
ಕಲಿಸ-ಬಲ್ಲಿರಿ
ಕಲಿಸ-ಲಾರರು
ಕಲಿ-ಸಲು
ಕಲಿಸಿ-ದನು
ಕಲಿಸಿ-ದರು
ಕಲಿಸಿ-ದಾಗ
ಕಲಿಸಿ-ರು-ವರು
ಕಲಿಸುತ್ತಾರೆ
ಕಲಿಸು-ವರು
ಕಲಿ-ಸು-ವು-ದಿಲ್ಲ
ಕಲು
ಕಲು-ಷಿತ
ಕಲು-ಷಿತ-ವಾಗುತ್ತ
ಕಲೆ
ಕಲೆ-ತಾಗ
ಕಲೆ-ತಿ-ರು-ವುದು
ಕಲೆ-ತಿವೆ
ಕಲೆತು
ಕಲೆ-ತು-ಹೋಗಿವೆ
ಕಲೆ-ದರೆ
ಕಲೆ-ಯು-ವಂತೆ
ಕಲೆಸಿ
ಕಲೋಪಾಸ-ಕನ
ಕಲ್ಪ
ಕಲ್ಪ-ಗ-ಳನ್ನು
ಕಲ್ಪದ
ಕಲ್ಪ-ದಲ್ಲಿ
ಕಲ್ಪ-ದ-ವರೆಗೂ
ಕಲ್ಪನಾ
ಕಲ್ಪ-ನಾ-ಜೀವಿ-ಯಲ್ಲ
ಕಲ್ಪನೆ
ಕಲ್ಪ-ನೆ-ಗಳ
ಕಲ್ಪ-ನೆ-ಗಳಲ್ಲಿ
ಕಲ್ಪ-ನೆ-ಗಳಿ-ಗಿಂತ
ಕಲ್ಪ-ನೆ-ಗಳೆಲ್ಲ
ಕಲ್ಪ-ನೆಗೂ
ಕಲ್ಪ-ನೆಗೆ
ಕಲ್ಪ-ನೆಯ
ಕಲ್ಪ-ನೆ-ಯಲ್ಲದೆ
ಕಲ್ಪ-ನೆ-ಯಲ್ಲಿ
ಕಲ್ಪ-ನೆ-ಯಾ-ಗುತ್ತದೆ
ಕಲ್ಪ-ನೆಯು
ಕಲ್ಪ-ವಾದ
ಕಲ್ಪವು
ಕಲ್ಪ-ವೆಂದು
ಕಲ್ಪಾಂತ-ದಲ್ಲಿ
ಕಲ್ಪಾಂತರ-ದಲ್ಲಿ
ಕಲ್ಪಾಂತ್ಯ-ದಲ್ಲಿ
ಕಲ್ಪಿಸಬಲ್ಲದೆಂದು
ಕಲ್ಪಿಸ-ಬಲ್ಲರು
ಕಲ್ಪಿ-ಸ-ಬೇಕು
ಕಲ್ಪಿಸ-ಬೇಕೆಂಬು-ದನ್ನು
ಕಲ್ಪಿ-ಸಲು
ಕಲ್ಪಿಸಿ
ಕಲ್ಪಿಸಿ-ಕೊಂಡ
ಕಲ್ಪಿಸಿ-ಕೊಂಡಂತೆ
ಕಲ್ಪಿಸಿ-ಕೊಂಡರೆ
ಕಲ್ಪಿಸಿ-ಕೊಂಡಾಗ
ಕಲ್ಪಿಸಿ-ಕೊಂಡಿ-ರ-ಬೇಕು
ಕಲ್ಪಿಸಿ-ಕೊಂಡಿ-ರುತ್ತಾರೆ
ಕಲ್ಪಿಸಿ-ಕೊಂಡಿ-ರುವ
ಕಲ್ಪಿಸಿ-ಕೊಂಡಿಲ್ಲದೆ
ಕಲ್ಪಿಸಿ-ಕೊಂಡು
ಕಲ್ಪಿಸಿ-ಕೊಳ್ಳ-ಬಹುದು
ಕಲ್ಪಿಸಿ-ಕೊಳ್ಳ-ಬೇಕು
ಕಲ್ಪಿಸಿ-ಕೊಳ್ಳ-ಲಾರರು
ಕಲ್ಪಿಸಿ-ಕೊಳ್ಳ-ಲಾರೆವೊ
ಕಲ್ಪಿಸಿ-ಕೊಳ್ಳಿ
ಕಲ್ಪಿಸಿ-ಕೊಳ್ಳುತ್ತವೆ
ಕಲ್ಪಿಸಿ-ಕೊಳ್ಳುತ್ತಾನೆ
ಕಲ್ಪಿಸಿ-ಕೊಳ್ಳುತ್ತಿ-ರು-ವಾ-ಗಲೇ
ಕಲ್ಪಿಸಿ-ಕೊಳ್ಳುತ್ತಿ-ರು-ವೆವು
ಕಲ್ಪಿಸಿ-ಕೊಳ್ಳುವ
ಕಲ್ಪಿಸಿ-ಕೊಳ್ಳು-ವಂತೆ
ಕಲ್ಪಿಸಿ-ಕೊಳ್ಳು-ವನು
ಕಲ್ಪಿಸಿ-ಕೊಳ್ಳು-ವರು
ಕಲ್ಪಿಸಿ-ಕೊಳ್ಳು-ವು-ದಕ್ಕೆ
ಕಲ್ಪಿಸಿ-ಕೊಳ್ಳು-ವುದು
ಕಲ್ಪಿಸಿ-ಕೊಳ್ಳೋಣ
ಕಲ್ಪಿ-ಸಿವೆ
ಕಲ್ಪಿ-ಸುವ
ಕಲ್ಪಿ-ಸುವನು
ಕಲ್ಪಿ-ಸು-ವುದು
ಕಲ್ಯಾಣ
ಕಲ್ಯಾಣ-ಕಾರಿ
ಕಲ್ಯಾಣಕ್ಕಾಗಿ
ಕಲ್ಯಾಣಕ್ಕೆ
ಕಲ್ಯಾಣ-ಗುಣ-ಗಣಿ
ಕಲ್ಯಾಣದ
ಕಲ್ಯಾಣ-ವನ್ನು
ಕಲ್ಲನ್ನು
ಕಲ್ಲಿಗೆ
ಕಲ್ಲಿದ್ದ-ಲನ್ನು
ಕಲ್ಲಿದ್ದ-ಲಿನ
ಕಲ್ಲಿದ್ದಿಲಿ-ನಿಂದ
ಕಲ್ಲಿನ
ಕಲ್ಲಿನಂತೆ
ಕಲ್ಲಿನಲ್ಲಿ
ಕಲ್ಲು
ಕಲ್ಲು-ಗಳು-ಇ-ವನ್ನೆಲ್ಲ
ಕಲ್ಲು-ಸಕ್ಕರೆ
ಕಲ್ಲೊಂದು
ಕಳಂಕ
ಕಳಂಕ-ಜೀವ-ನ-ವನ್ನು
ಕಳಂಕ-ವನ್ನು
ಕಳಚಿ
ಕಳಚಿ-ಬೀ-ಳುವ
ಕಳಿತು-ಕೊಳ್ಳ-ಬೇಕಾದ
ಕಳಿ-ಸುವ
ಕಳು-ವಾಗಿದೆ
ಕಳುಹ-ಬಾ-ರದು
ಕಳುಹಿಸ
ಕಳುಹಿಸ-ಬಲ್ಲ
ಕಳುಹಿಸ-ಬಹುದು
ಕಳುಹಿಸ-ಬಾ-ರದು
ಕಳುಹಿಸ-ಬೇಕು
ಕಳುಹಿ-ಸಲೂ
ಕಳುಹಿಸಿ
ಕಳುಹಿ-ಸಿದ
ಕಳುಹಿಸಿ-ದನು
ಕಳುಹಿ-ಸುತ್ತದೆ
ಕಳುಹಿಸುತ್ತಿವೆ
ಕಳುಹಿಸು-ವರು
ಕಳುಹಿ-ಸು-ವು-ದಕ್ಕೆ
ಕಳುಹಿಸು-ವುದು
ಕಳೆದ
ಕಳೆ-ದಂತೆ
ಕಳೆ-ದರೆ
ಕಳೆ-ದಷ್ಟೂ
ಕಳೆ-ದಿವೆ
ಕಳೆದು
ಕಳೆ-ದು-ಕೊಂಡ
ಕಳೆ-ದು-ಕೊಂಡರೂ
ಕಳೆ-ದು-ಕೊಂಡರೆ
ಕಳೆ-ದು-ಕೊಂಡಿ-ರು-ವರು
ಕಳೆ-ದು-ಕೊಂಡಿ-ರು-ವೆವು
ಕಳೆ-ದು-ಕೊಂಡಿಲ್ಲ
ಕಳೆ-ದು-ಕೊಂಡಿಲ್ಲವೋ
ಕಳೆ-ದು-ಕೊಂಡು
ಕಳೆ-ದು-ಕೊಂಡೆ
ಕಳೆ-ದು-ಕೊಂಡೆವು
ಕಳೆ-ದು-ಕೊಂಡೆ-ವೆಂದು
ಕಳೆ-ದು-ಕೊಳ್ಳ-ಬೇಕು
ಕಳೆ-ದು-ಕೊಳ್ಳುತ್ತಾನೆ
ಕಳೆ-ದು-ಕೊಳ್ಳು-ವಂತೆ
ಕಳೆ-ದು-ಕೊಳ್ಳು-ವನು
ಕಳೆ-ದು-ಕೊಳ್ಳು-ವರು
ಕಳೆ-ದು-ಕೊಳ್ಳು-ವರೊ
ಕಳೆ-ದು-ಕೊಳ್ಳು-ವ-ವರೆಗೆ
ಕಳೆ-ದು-ಕೊಳ್ಳು-ವು-ದಕ್ಕೆ
ಕಳೆ-ದು-ಕೊಳ್ಳು-ವು-ದನ್ನು
ಕಳೆ-ದು-ಕೊಳ್ಳು-ವು-ದಿಲ್ಲ
ಕಳೆ-ದು-ಕೊಳ್ಳು-ವುದು
ಕಳೆ-ದು-ಕೊಳ್ಳು-ವೆವು
ಕಳೆ-ದು-ದನ್ನು
ಕಳೆ-ದುವು
ಕಳೆ-ದು-ಹೋದ
ಕಳೆ-ದು-ಹೋ-ದವು
ಕಳೆದೆ
ಕಳೆಯ
ಕಳೆ-ಯದೆ
ಕಳೆ-ಯಿರಿ
ಕಳೆ-ಯುತ್ತಾರೆ
ಕಳೆ-ಯು-ವೆವು
ಕಳ್ಳ
ಕಳ್ಳ-ತನ
ಕಳ್ಳ-ತನ-ದಿಂದ-ಲಾ-ದರೂ
ಕಳ್ಳ-ನಾಗ-ಬಾ-ರದು
ಕಳ್ಳ-ನಾಗಿದ್ದರೆ
ಕಳ್ಳ-ನಿಲ್ಲ
ಕಳ್ಳನು
ಕಳ್ಳ-ನೊಬ್ಬ
ಕವಚ
ಕವನ-ಗಳಲ್ಲಿ
ಕವನ-ದಲ್ಲಿ
ಕವನ-ವನ್ನು
ಕವಲೊಡೆಯು-ವುದು
ಕವಿ
ಕವಿ-ಗಳ
ಕವಿ-ಗಳು
ಕವಿಗೆ
ಕವಿ-ತಾ-ವಾಣಿ-ಯಲ್ಲಿ-ಡ-ಬಹುದು
ಕವಿದ
ಕವಿ-ದಿದ್ದರೆ
ಕವಿ-ದಿ-ರು-ವುದು
ಕಶೇರುಕ-ದೊ-ಡನೆ
ಕಶ್ಮಲ-ದಿಂದ
ಕಶ್ಮಲ-ವಾದ
ಕಷ್ಟ
ಕಷ್ಟ-ಕರ-ವಾದ
ಕಷ್ಟಕ್ಕೆ
ಕಷ್ಟ-ಗಳಿಗೆ
ಕಷ್ಟ-ಗಳಿವೆ
ಕಷ್ಟ-ಗಳು
ಕಷ್ಟದ
ಕಷ್ಟ-ದಲ್ಲಿ
ಕಷ್ಟ-ದಿಂದ
ಕಷ್ಟ-ನಷ್ಟ-ಗ-ಳನ್ನು
ಕಷ್ಟ-ಪಟ್ಟು
ಕಷ್ಟ-ಪಡ-ಬಹುದು
ಕಷ್ಟ-ಪಡ-ಬೇಕಾ-ಗು-ವುದು
ಕಷ್ಟ-ಪಡು
ಕಷ್ಟ-ವನ್ನು
ಕಷ್ಟ-ವನ್ನೆಲ್ಲಾ
ಕಷ್ಟ-ವಾಗ-ಲಿಲ್ಲ
ಕಷ್ಟ-ವಾಗಿ
ಕಷ್ಟ-ವಾಗಿದೆ
ಕಷ್ಟ-ವಾಗುತ್ತದೆ
ಕಷ್ಟ-ವಾದ
ಕಷ್ಟ-ವಾ-ದರೆ
ಕಷ್ಟ-ವಾ-ದವು
ಕಷ್ಟ-ವಾ-ಯಿತು
ಕಷ್ಟ-ವಿ-ರ-ಬೇಕು
ಕಷ್ಟ-ವಿ-ರು-ವುದು
ಕಷ್ಟ-ವಿಲ್ಲ-ದವು
ಕಷ್ಟವು
ಕಷ್ಟವೂ
ಕಷ್ಟ-ವೆಂಬುದು
ಕಷ್ಟ-ವೆನ್ನು-ವುದು
ಕಷ್ಟವೇ
ಕಷ್ಟ-ಸಾಧ್ಯ-ವಾದ
ಕಸಬಿಗೇ
ಕಸರತ್ತಾಗ
ಕಸರತ್ತಿ-ನಿಂದ
ಕಸರತ್ತು
ಕಸರತ್ತೂ
ಕಸ-ವನ್ನು
ಕಸಿದು-ಕೊಳ್ಳು-ವುದು
ಕಹಿ
ಕಹಿ-ಯಾ-ಗಲಿ
ಕಾಂಟನು
ಕಾಂತ-ತೆಯ
ಕಾಂತಿ
ಕಾಂತಿ-ಮಯನು
ಕಾಂತಿಯ
ಕಾಂತಿ-ಯನ್ನು
ಕಾಂತಿಯು
ಕಾಂತಿ-ಯುಕ್ತ
ಕಾಗದ
ಕಾಗದ-ಗಳ
ಕಾಗದದ
ಕಾಡಿ
ಕಾಡಿಗೆ
ಕಾಡಿ-ನಲ್ಲಿ
ಕಾಡಿ-ನಲ್ಲಿ-ರುವ
ಕಾಡಿ-ನಲ್ಲೇ
ಕಾಡಿ-ನಲ್ಲೋ
ಕಾಡಿ-ನ-ವರು
ಕಾಡು
ಕಾಡು-ಕಿಚ್ಚಿ-ನಲ್ಲಿ
ಕಾಡು-ಜ-ನರ
ಕಾಡುತ್ತಿದೆ
ಕಾಡುತ್ತಿ-ರುವ
ಕಾಡು-ಮನುಷ್ಯ
ಕಾಡು-ಮನುಷ್ಯನ
ಕಾಡು-ವಂತೆ
ಕಾಡು-ಹಂದಿ
ಕಾಡು-ಹಂದಿ-ಗ-ಳನ್ನು
ಕಾಡು-ಹಂದಿ-ಯನ್ನು
ಕಾಡ್ಗಿಚ್ಚಿ-ನಂತೆ
ಕಾಣ
ಕಾಣದ
ಕಾಣ-ದಂತೆ
ಕಾಣ-ದಷ್ಟು
ಕಾಣ-ದಿದ್ದರೂ
ಕಾಣದು
ಕಾಣದೆ
ಕಾಣ-ಬ-ರುತ್ತದೆ
ಕಾಣ-ಬರುತ್ತವೆ
ಕಾಣ-ಬರು-ವು-ದಿಲ್ಲ
ಕಾಣ-ಬಲ್ಲ
ಕಾಣ-ಬಹುದು
ಕಾಣ-ಲಾರರೊ
ಕಾಣಲಿ
ಕಾಣ-ಲಿಲ್ಲ
ಕಾಣಲು
ಕಾಣ-ವುವು
ಕಾಣಿ
ಕಾಣಿರಿ
ಕಾಣಿ-ಸದ
ಕಾಣಿ-ಸ-ದಂತೆ
ಕಾಣಿ-ಸ-ಬಹುದು
ಕಾಣಿ-ಸಿ-ಕೊಂಡನು
ಕಾಣಿ-ಸಿ-ಕೊಂಡ-ನೆಂದು
ಕಾಣಿ-ಸಿ-ಕೊಳ್ಳುತ್ತದೆ
ಕಾಣಿ-ಸಿ-ಕೊಳ್ಳುತ್ತಲೇ
ಕಾಣಿ-ಸಿ-ಕೊಳ್ಳುತ್ತಿ-ರುವನು
ಕಾಣಿ-ಸಿ-ಕೊಳ್ಳುವ
ಕಾಣಿ-ಸಿ-ಕೊಳ್ಳು-ವುದು
ಕಾಣಿಸು
ಕಾಣಿ-ಸುತ್ತದೆ
ಕಾಣಿ-ಸುತ್ತಿದೆ
ಕಾಣಿ-ಸುತ್ತಿ-ರು-ವಂತೆ
ಕಾಣಿ-ಸುತ್ತಿ-ರು-ವರು
ಕಾಣಿ-ಸುವ
ಕಾಣಿ-ಸು-ವನು
ಕಾಣಿ-ಸು-ವು-ದಿಲ್ಲ
ಕಾಣಿ-ಸು-ವುದು
ಕಾಣಿ-ಸು-ವುದೇ
ಕಾಣು
ಕಾಣುತ್ತದೆ
ಕಾಣುತ್ತದೆಯೋ
ಕಾಣುತ್ತವೆ
ಕಾಣುತ್ತಾನೆ
ಕಾಣುತ್ತಾರೆ
ಕಾಣುತ್ತಿತ್ತು
ಕಾಣುತ್ತಿದೆ
ಕಾಣುತ್ತಿದೆಯೋ
ಕಾಣುತ್ತಿದ್ದ
ಕಾಣುತ್ತಿದ್ದು
ಕಾಣು-ತ್ತಿರ-ಬಹುದು
ಕಾಣು-ತ್ತಿರ-ಲಿಲ್ಲ
ಕಾಣು-ತ್ತಿರುವ
ಕಾಣು-ತ್ತಿರು-ವುದು
ಕಾಣು-ತ್ತಿರು-ವುದೆಲ್ಲ
ಕಾಣುತ್ತಿಲ್ಲ-ವೆಂದು
ಕಾಣುತ್ತೀರೊ
ಕಾಣುತ್ತೇನೆ
ಕಾಣುತ್ತೇವೆ
ಕಾಣುವ
ಕಾಣು-ವಂತೆ
ಕಾಣು-ವನು
ಕಾಣು-ವರು
ಕಾಣು-ವರೋ
ಕಾಣು-ವ-ವರೆಗೂ
ಕಾಣು-ವಿರಿ
ಕಾಣುವು
ಕಾಣು-ವು-ದಿಲ್ಲ
ಕಾಣು-ವು-ದಿಲ್ಲವೆ
ಕಾಣು-ವುದು
ಕಾಣು-ವು-ದು-ಆಲೋ
ಕಾಣು-ವು-ದು-ಇದ-ರಂತೆಯೆ
ಕಾಣು-ವು-ದುಪ್ರತಿ
ಕಾಣು-ವು-ದೆಂದು
ಕಾಣು-ವುದೇ
ಕಾಣು-ವುವು
ಕಾಣು-ವೆವು
ಕಾಣುಸು
ಕಾತರ-ಗೊಂಡಿದೆ
ಕಾತರ-ರಾಗಿದ್ದರು
ಕಾತರ-ರಾಗಿ-ರು-ವಿರಿ
ಕಾತರ-ವಾ-ಗಿ-ರುವುದು
ಕಾತ-ರಿಸುತ್ತಿ-ರು-ವರು
ಕಾತುರ-ರಾಗಿ-ರು-ವರು
ಕಾದಂಬರಿ
ಕಾದಾಡಿ
ಕಾದಾಡುತ್ತಿ-ರು-ವರು
ಕಾದಾಡುತ್ತಿವೆ
ಕಾದಾ-ಡುವ
ಕಾದಿರಿ-ಸಲು
ಕಾದು
ಕಾದು-ಕೊಂಡಿದೆ
ಕಾನನ-ಗಳ
ಕಾನನ-ಗಳಿಂದ
ಕಾನನಾಂತರ-ಗಳಲ್ಲಿ
ಕಾಪಟ್ಯ-ವಿದೆ
ಕಾಪಾಡ-ಬೇಕು
ಕಾಪಾ-ಡಲಿ
ಕಾಪಾಡಿ-ಕೊಂಡು
ಕಾಪಾ-ಡುವ
ಕಾಪಾಡು-ವುದು
ಕಾಮ
ಕಾಮಕ್ರೋಧಾದಿ-ಗ-ಳನ್ನು
ಕಾಮ-ಗ-ಳನ್ನು
ಕಾಮ-ತೃಪ್ತಿಯ
ಕಾಮ-ಪೂರಕ
ಕಾಮ-ಭಾವ-ನೆಯ
ಕಾಮ-ಲಾಯ
ಕಾಮ-ಲಾಯ-ನನು
ಕಾಮ-ಲೋಭ-ಗಳು
ಕಾಮ-ವನ್ನು
ಕಾಮಾನ-ನು-ಯಂತಿ
ಕಾಮಾಲೆ
ಕಾಮ್ಟೆ
ಕಾಯ
ಕಾಯ-ಬೇಕಾ-ಗು-ವುದು
ಕಾಯ-ಬೇಕಾ-ಯಿತು
ಕಾಯ-ರೂಪ-ಸಂಯಮಾತ್ತದ್ಗ್ರಾಹ್ಯ-ಶಕ್ತಿಸ್ತಂಭೇ
ಕಾಯಲು
ಕಾಯವ್ಯೂಹಜ್ಞಾನಮ್
ಕಾಯವ್ಯೂಹ-ವನ್ನು
ಕಾಯ-ಸಂಪತ್
ಕಾಯ-ಸಂಪತ್ತು-ಗಳು
ಕಾಯ-ಸಮ್ಪತ್ತದ್ಧರ್ಮಾನಭಿಘಾತಶ್ಚ
ಕಾಯಾ-ಕಾಶಯೋಃ
ಕಾಯಿಲೆ
ಕಾಯಿಸಿ-ದಾಗ
ಕಾಯುತ್ತ
ಕಾಯುತ್ತವೆ
ಕಾಯುತ್ತಿತ್ತು
ಕಾಯುತ್ತಿದ್ದ
ಕಾಯುತ್ತಿ-ರುವ
ಕಾಯುತ್ತಿ-ರುವನು
ಕಾಯುತ್ತಿ-ರುವೆ
ಕಾಯುವು-ದಕ್ಕಾ-ಗಲಿ
ಕಾಯು-ವು-ದಿಲ್ಲ
ಕಾಯೇಂದ್ರಿಯ-ಸಿದ್ಧಿ-ರ-ಶುದ್ಧಿಕ್ಷಯಾತ್ತಪಸಃ
ಕಾಯೋಣ
ಕಾಯ್ದಿರಿ-ಸಲಾ-ಗಿದೆ
ಕಾಯ್ದು-ಕೊಂಡಿದ್ದ
ಕಾಯ್ದು-ಕೊಂಡಿ-ರು-ವಂತೆ
ಕಾಯ್ದು-ಕೊಂಡಿವೆ
ಕಾರಣ
ಕಾರ-ಣ-ಇವು
ಕಾರ-ಣ-ಕರ್ತ-ನನ್ನಾಗಿ
ಕಾರ-ಣಕ್ಕಿಂತ
ಕಾರ-ಣಕ್ಕೆ
ಕಾರ-ಣ-ಗಳ
ಕಾರ-ಣ-ಗ-ಳನ್ನು
ಕಾರ-ಣ-ಗಳಲ್ಲಿ
ಕಾರ-ಣ-ಗಳಾ-ಗುತ್ತವೆ
ಕಾರ-ಣ-ಗ-ಳಾದ
ಕಾರ-ಣ-ಗಳಿಂದ
ಕಾರ-ಣ-ಗಳಿಂದಲೂ
ಕಾರ-ಣ-ಗಳಿಲ್ಲ
ಕಾರ-ಣ-ಗಳಿವೆ
ಕಾರ-ಣ-ಗಳಿ-ವೆಯೆ
ಕಾರ-ಣ-ಗಳಿ-ವೆಯೇ
ಕಾರ-ಣ-ಗಳು
ಕಾರ-ಣ-ಗಳೂ
ಕಾರ-ಣ-ಗಳೆ
ಕಾರ-ಣ-ಗಳೆಲ್ಲ
ಕಾರ-ಣದ
ಕಾರ-ಣ-ದಂತೆ
ಕಾರ-ಣ-ದಂತೆಯೆ
ಕಾರ-ಣ-ದಲ್ಲಿ
ಕಾರ-ಣ-ದಿಂದ
ಕಾರ-ಣ-ದಿಂದಲೇ
ಕಾರ-ಣ-ದಿಂದಲೋ
ಕಾರ-ಣ-ದಿಂದಾಗಿ
ಕಾರ-ಣ-ನಾಗಿ
ಕಾರ-ಣ-ನಾಗು-ವನು
ಕಾರ-ಣ-ನೆಂದು
ಕಾರ-ಣ-ಭೂತ-ವಾದ
ಕಾರ-ಣ-ರ-ಹಿತ-ವಾದ
ಕಾರ-ಣ-ವನ್ನು
ಕಾರ-ಣ-ವಲ್ಲ
ಕಾರ-ಣ-ವಾಗಿ
ಕಾರ-ಣ-ವಾಗಿ-ರ-ಬೇಕು
ಕಾರ-ಣ-ವಾಗು
ಕಾರ-ಣ-ವಾಗುತ್ತದೆ
ಕಾರ-ಣ-ವಾ-ಗು-ವುದು
ಕಾರ-ಣ-ವಾದ
ಕಾರ-ಣ-ವಾ-ದರೂ
ಕಾರ-ಣ-ವಾ-ದರೆ
ಕಾರ-ಣ-ವಿತ್ತು
ಕಾರ-ಣ-ವಿದೆ
ಕಾರ-ಣ-ವಿದೆಯೆ
ಕಾರ-ಣ-ವಿ-ರ-ಬೇಕು
ಕಾರ-ಣ-ವಿಲ್ಲ
ಕಾರ-ಣ-ವಿಲ್ಲದೆ
ಕಾರ-ಣವು
ಕಾರ-ಣವೂ
ಕಾರ-ಣವೆ
ಕಾರ-ಣ-ವೆಂತಲ್ಲ
ಕಾರ-ಣ-ವೆಂದರೆ
ಕಾರ-ಣ-ವೆಂದೂ
ಕಾರ-ಣ-ವೆನ್ನ-ಬಹುದು
ಕಾರ-ಣ-ವೆಲ್ಲ
ಕಾರ-ಣವೇ
ಕಾರ-ಣ-ವೇನು
ಕಾರ-ಣವೊ
ಕಾರ-ಣಾಂತರ-ಗಳಿಂದ
ಕಾರ-ಣಾ-ವಸ್ಥೆಗೆ
ಕಾರ-ಣಾ-ವಸ್ಥೆ-ಯಲ್ಲಿ
ಕಾರ-ಣಾ-ವಸ್ಥೆ-ಯಿಂದ
ಕಾರ-ವಾ-ಗಿ-ರುವ
ಕಾರಿ
ಕಾರಿ-ಯಾ-ಗು-ವುವು
ಕಾರುಬಾ-ರೆಂದು
ಕಾರ್ಯ
ಕಾರ್ಯ-ಕಲಾಪ-ಗಳಲ್ಲಿಯೂ
ಕಾರ್ಯ-ಕಾರಣ
ಕಾರ್ಯ-ಕಾರ-ಣ-ಇವು-ಗಳೆಂದ-ರೇನು
ಕಾರ್ಯ-ಕಾರ-ಣ-ಗಳ
ಕಾರ್ಯ-ಕಾರ-ಣ-ಗಳಾಚೆ
ಕಾರ್ಯ-ಕಾರ-ಣ-ಗಳು
ಕಾರ್ಯ-ಕಾರ-ಣ-ಗಳೂ
ಕಾರ್ಯ-ಕಾರ-ಣ-ಜ-ಗತ್ತಿನ
ಕಾರ್ಯ-ಕಾರ-ಣ-ನಿಯ-ಮಕ್ಕೆ
ಕಾರ್ಯ-ಕಾರ-ಣವೂ
ಕಾರ್ಯ-ಕಾರಿ
ಕಾರ್ಯ-ಕಾರಿತ್ವ
ಕಾರ್ಯ-ಕಾರಿ-ಯನ್ನಾಗಿ
ಕಾರ್ಯ-ಕಾರಿ-ಯಾಗ
ಕಾರ್ಯ-ಕಾರಿ-ಯಾಗುತ್ತಿರು-ವು-ದನ್ನು
ಕಾರ್ಯ-ಕಾರಿ-ಯಾದ
ಕಾರ್ಯ-ಕಾರಿ-ಯಾ-ದುದು
ಕಾರ್ಯ-ಕಾರಿ-ಯಾ-ಯಿತು
ಕಾರ್ಯಕ್ಕೆ
ಕಾರ್ಯಕ್ರಮಕ್ಕೆ
ಕಾರ್ಯಕ್ರಮದ
ಕಾರ್ಯಕ್ರಮ-ದಲ್ಲಿ
ಕಾರ್ಯಕ್ಷೇತ್ರ
ಕಾರ್ಯಕ್ಷೇತ್ರಕ್ಕೆ
ಕಾರ್ಯಕ್ಷೇತ್ರ-ಗ-ಳನ್ನು
ಕಾರ್ಯಕ್ಷೇತ್ರ-ಗಳಲ್ಲಿ
ಕಾರ್ಯಕ್ಷೇತ್ರ-ಗಳಲ್ಲಿಯೂ
ಕಾರ್ಯಕ್ಷೇತ್ರ-ಗಳಲ್ಲಿ-ರುವ
ಕಾರ್ಯಕ್ಷೇತ್ರ-ದಲ್ಲಾ-ದರೂ
ಕಾರ್ಯಕ್ಷೇತ್ರ-ದಲ್ಲಿಯೂ
ಕಾರ್ಯಕ್ಷೇತ್ರ-ದಿಂದ
ಕಾರ್ಯ-ಗಳ
ಕಾರ್ಯ-ಗಳನ್ನೆಲ್ಲ
ಕಾರ್ಯ-ಗಳಾ-ಗು-ವುವು
ಕಾರ್ಯ-ಗಳಿಗೆ
ಕಾರ್ಯ-ಗಳು
ಕಾರ್ಯ-ಗಳೆಲ್ಲ
ಕಾರ್ಯತಃ
ಕಾರ್ಯದ
ಕಾರ್ಯ-ದಲ್ಲಿ
ಕಾರ್ಯ-ದಲ್ಲೂ
ಕಾರ್ಯ-ದಿಂದ
ಕಾರ್ಯ-ರಂಗ-ದಲ್ಲಿ
ಕಾರ್ಯ-ರೂಪಕ್ಕೆ
ಕಾರ್ಯ-ರೂಪ-ದಲ್ಲಿ
ಕಾರ್ಯ-ವನ್ನು
ಕಾರ್ಯ-ವನ್ನೆಲ್ಲಾ
ಕಾರ್ಯ-ವಲಯ
ಕಾರ್ಯ-ವಲ್ಲ
ಕಾರ್ಯ-ವಾಗಿದ್ದರೆ
ಕಾರ್ಯ-ವಾಗು
ಕಾರ್ಯ-ವಾಗು-ವುದು
ಕಾರ್ಯ-ವಾದ
ಕಾರ್ಯವು
ಕಾರ್ಯವೂ
ಕಾರ್ಯ-ವೆಲ್ಲಾ
ಕಾರ್ಯವೇ
ಕಾರ್ಯೋನ್ಮುಖ
ಕಾರ್ಯೋನ್ಮುಖ-ರಾದ
ಕಾಲ
ಕಾಲ-ಕಳೆ-ದಂತೆ
ಕಾಲ-ಕಾಲಕ್ಕೆ
ಕಾಲಕ್ಕೂ
ಕಾಲಕ್ಕೆ
ಕಾಲಕ್ರಮ-ದಲ್ಲಿ
ಕಾಲಕ್ರಮೇಣ
ಕಾಲ-ಗಣ-ತಿಗೆ
ಕಾಲ-ಗಳಲ್ಲಿ
ಕಾಲ-ಗಳಿಗೆ
ಕಾಲದ
ಕಾಲ-ದ-ಮೇಲೆ
ಕಾಲ-ದಲ್ಲಿ
ಕಾಲ-ದಲ್ಲಿದ್ದು
ಕಾಲ-ದಲ್ಲಿಯೂ
ಕಾಲ-ದಲ್ಲಿ-ರಲಿ
ಕಾಲ-ದಲ್ಲಿಲ್ಲ
ಕಾಲ-ದಲ್ಲೂ
ಕಾಲ-ದಲ್ಲೆ
ಕಾಲ-ದಲ್ಲೇ
ಕಾಲ-ದ-ವರ
ಕಾಲ-ದ-ವರೆಗೂ
ಕಾಲ-ದ-ವರೆಗೆ
ಕಾಲ-ದಿಂದ
ಕಾಲ-ದಿಂದಲೂ
ಕಾಲ-ದೇಶ
ಕಾಲ-ದೇಶ-ಕಾರ್ಯ-ಕಾರ-ಣ-ಗಳಾಚೆ
ಕಾಲ-ದೇಶ-ಗಳ
ಕಾಲ-ದೇಶ-ಗಳಾಚೆ
ಕಾಲ-ದೇಶ-ಗಳಿಂದ
ಕಾಲ-ದೇಶ-ಗಳು
ಕಾಲ-ದೇಶ-ಗಳೂ
ಕಾಲ-ದೇಶ-ನಿ-ಮಿತ್ತ
ಕಾಲ-ಪರಿ-ಣಾಮ-ವಾಗುತ್ತದೆ
ಕಾಲ-ಪುರುಷ
ಕಾಲ-ಮೇಲೆ
ಕಾಲರಾ
ಕಾಲ-ವನ್ನು
ಕಾಲ-ವಶ-ರಾಗಿ-ರು-ವರು
ಕಾಲ-ವಾಗ-ಬೇಕಾ-ಗಿಲ್ಲ
ಕಾಲ-ವಾ-ಗಲಿ
ಕಾಲ-ವಾಗಿ
ಕಾಲ-ವಾಗುತ್ತಾನೆ
ಕಾಲ-ವಾಗುತ್ತಾ-ನೆಈ
ಕಾಲ-ವಾಗುತ್ತೇನೆ
ಕಾಲ-ವಾಗು-ವರು
ಕಾಲ-ವಾಗು-ವರೇ
ಕಾಲ-ವಾದ
ಕಾಲ-ವಾ-ದರು
ಕಾಲ-ವಾ-ದರೂ
ಕಾಲ-ವಾ-ದರೆ
ಕಾಲ-ವಿದೆ
ಕಾಲ-ವಿದ್ದನು
ಕಾಲ-ವಿದ್ದು
ಕಾಲ-ವಿ-ರುತ್ತಿ-ರ-ಲಿಲ್ಲ
ಕಾಲ-ವಿಲ್ಲದೆ
ಕಾಲ-ವಿಳಂಬ-ವಾ-ಗು-ವುದು
ಕಾಲವು
ಕಾಲವೂ
ಕಾಲವೆ
ಕಾಲ-ವೆಂಬ
ಕಾಲ-ವೆಲ್ಲಿ
ಕಾಲವೇ
ಕಾಲ-ವೊಂದು
ಕಾಲ-ಹರಣ
ಕಾಲಾ-ತೀತ
ಕಾಲಾ-ನಂತರ
ಕಾಲಾ-ನಂತ-ರ-ದಲ್ಲಿ
ಕಾಲಿಗೆ
ಕಾಲಿಟ್ಟು
ಕಾಲಿನ
ಕಾಲಿ-ನಿಂದ
ಕಾಲು
ಕಾಲು-ಗಳ
ಕಾಲು-ಗಳೂ
ಕಾಲುವೆ
ಕಾಲು-ವೆಯ
ಕಾಲು-ವೆ-ಯನ್ನು
ಕಾಲು-ವೆ-ಯಲ್ಲಿ
ಕಾಲು-ವೆಯು
ಕಾಲು-ವೆ-ಯೊ-ಳಗೆ
ಕಾಲೇಜಿಗೂ
ಕಾಲೇಜಿನ
ಕಾಲೇನಾನವಚ್ಛೇದಾತ್
ಕಾಲ್ಪ-ನಿಕ
ಕಾಳಿನ
ಕಾವ್ಯ
ಕಾವ್ಯದ
ಕಾವ್ಯ-ದಂತಿರ-ಬಹುದು
ಕಾವ್ಯ-ಮಯ-ವಾದ
ಕಾವ್ಯ-ರೂಪ-ದಲ್ಲಿ
ಕಾವ್ಯ-ವನ್ನು
ಕಾಶ-ದಲ್ಲಿರು-ವುದು
ಕಾಶಿ
ಕಾಶಿ-ಯಲ್ಲಿ
ಕಿಂಚಿತ್
ಕಿಕ್ಕಿರಿದು
ಕಿಟಕಿ-ಯನ್ನು
ಕಿಡಿ-ಗಳು
ಕಿಡಿ-ಗಳೇಳು-ವಂತೆ
ಕಿಡಿ-ಯೊಂದು
ಕಿತ್ತಳೆ-ಹಣ್ಣು
ಕಿತ್ತು
ಕಿತ್ತು-ಕೊಳ್ಳು-ವುದರ
ಕಿತ್ತು-ಹಾಕಿ
ಕಿತ್ತೊಗೆದು
ಕಿತ್ತೊ-ಗೆಯ-ಬೇಕು
ಕಿತ್ತೊಗೆ-ಯಿರಿ
ಕಿರಚಿ-ಕೊಂಡಿತು
ಕಿರಣ
ಕಿರ-ಣಕ್ಕೆ
ಕಿರಣ-ಗಳಂತೆ
ಕಿರಣ-ಗಳನ್ನೆಲ್ಲಾ
ಕಿರಣ-ಗಳಿಗೆ
ಕಿರಣ-ಗಳೆ-ದು-ರಿಗೆ
ಕಿರಣ-ಗಳೇ
ಕಿರಣದ
ಕಿರಿ-ದಾಗಿ-ರುವುದೋ
ಕಿರಿದು
ಕಿರಿಯ
ಕಿರು-ಮೀನೊಂದು
ಕಿವಿ
ಕಿವಿ-ಗ-ಳನ್ನು
ಕಿವಿಗೆ
ಕಿವಿಯ
ಕಿವಿಯು
ಕೀಟ-ಗಳಲ್ಲಿಯೂ
ಕೀಟದ
ಕೀಟ-ದಿಂದ
ಕೀರ್ತನೆ
ಕೀರ್ತಿ
ಕೀರ್ತಿ-ಗಳ
ಕೀರ್ತಿ-ಗಳಿಂದ
ಕೀರ್ತಿ-ಯನ್ನು
ಕೀರ್ತಿ-ಯಿಲ್ಲ
ಕೀಳಾಗಿ
ಕೀಳಾಗುತ್ತಿಲ್ಲ
ಕೀಳಾದ
ಕೀಳು
ಕೀಳು-ಅಂತಹ
ಕೀಳು-ದರ್ಜೆಯ
ಕೀಳುಪ್ರಾಣಿ-ಯಿಂದ
ಕೀಳು-ಮನುಷ್ಯ-ನಿಂದ
ಕೀಳು-ಮೃಗ-ದಿಂದ
ಕೀಳು-ವರ್ಗಕ್ಕೆ
ಕೀಳು-ವಿಷ-ಯ-ಗ-ಳನ್ನು
ಕುಂಟ
ಕುಂಡ-ಲಿನಿ
ಕುಂಡಲಿ-ನಿಯ
ಕುಂಡಲಿ-ನಿಯು
ಕುಂಡಲಿ-ನಿಯೇ
ಕುಂಡ-ಲಿನೀ
ಕುಂದನ್ನೂ
ಕುಂದಿದೆ
ಕುಂದು
ಕುಂದು-ಕೊರತೆ-ಗಳು
ಕುಂದೂ
ಕುಂಭಕ
ಕುಂಭ-ಕವು
ಕುಕರ್ಮವೊ
ಕುಕ್ಕಲು
ಕುಕ್ಕು-ವು-ದನ್ನು
ಕುಗ್ಗಿ-ಸುವ
ಕುಗ್ಗಿ-ಸು-ವು-ದಿಲ್ಲ
ಕುಗ್ಗಿ-ಸು-ವುವು
ಕುಗ್ಗುತ್ತದೆ
ಕುಗ್ಗುತ್ತಾ
ಕುಗ್ಗುತ್ತಿದೆ
ಕುಟುಂಬ
ಕುಟುಕಿತು
ಕುಟುಕಿ-ದರೆ
ಕುಟುಕು-ವುದು
ಕುಡಿ-ದರೆ
ಕುಡಿದು
ಕುಡಿಯ
ಕುಡಿಯ-ಬೇಕು
ಕುಡಿಯ-ಬೇಕೆಂದು
ಕುಡಿ-ಯಲು
ಕುಡಿ-ಯಿರಿ
ಕುಡಿಯು
ಕುಡಿಯುತ್ತೇನೆ
ಕುಡಿ-ಯುವ
ಕುಡಿಯು-ವರು
ಕುಡಿಯು-ವಾ-ಗಲೂ
ಕುಡಿಯು-ವು-ದಕ್ಕೆ
ಕುಡಿಯು-ವು-ದನ್ನು
ಕುಡಿಯು-ವು-ದ-ರಲ್ಲಿದೆ
ಕುಡಿ-ಯು-ವು-ದಿಲ್ಲ
ಕುಡಿಯು-ವುದು
ಕುಡಿಸಿ-ದರು
ಕುಡುಕ
ಕುಡು-ಕನೊ
ಕುಣಿ-ದಾಡುತ್ತ
ಕುಣಿದಾ-ಡುವ
ಕುಣಿದಾ-ಡುವನು
ಕುಣಿದಾಡು-ವುದು
ಕುಣಿದಾಡು-ವೆನು
ಕುಣಿದು
ಕುಣಿ-ಯುತ್ತ
ಕುತರ್ಕ
ಕುತೂ-ಹಲ-ವನ್ನು
ಕುತೂಹಲಿ-ಗಳಾ-ಗ-ಬಹುದು
ಕುತೂಹಲಿ-ಗಳಾ-ಗಿದ್ದ-ವರು
ಕುದಿ-ಯಲು
ಕುದಿಯು-ವು-ದಕ್ಕೆ
ಕುದುರೆ
ಕುದುರೆ-ಗ-ಳನ್ನು
ಕುದುರೆ-ಗಳು
ಕುದುರೆ-ಗಾಡಿ
ಕುದುರೆಯ
ಕುದುರೆ-ಯನ್ನು
ಕುಪಿತ-ನಾಗಿ
ಕುಪಿತ-ನಾದ
ಕುಮಾರನು
ಕುಮಾರಿ
ಕುಮಾರಿಯೂ
ಕುರಿ
ಕುರಿ-ಗಳಂತೆ
ಕುರಿ-ಗಳು
ಕುರಿ-ಗಳೇ
ಕುರಿತ
ಕುರಿ-ತಂತೆ
ಕುರಿ-ತದ್ದು
ಕುರಿ-ತಾಗಿ
ಕುರಿ-ತಾದ
ಕುರಿತು
ಕುರಿ-ಮಂದೆ-ಯನ್ನು
ಕುರಿ-ಮಂದೆ-ಯಲ್ಲಿ
ಕುರಿ-ಮಂದೆ-ಯಲ್ಲಿದ್ದ
ಕುರಿ-ಮರಿ
ಕುರಿ-ಮರಿ-ಗಳಿಗೆ
ಕುರಿ-ಮರಿ-ಗಳು
ಕುರಿ-ಮ-ರಿಗೆ
ಕುರಿ-ಮರಿ-ಯಂತೆ
ಕುರಿ-ಮರಿ-ಸಿಂಹ
ಕುರಿ-ಯಂತೆ
ಕುರಿ-ಯಲ್ಲ
ಕುರುಕ್ಷೇತ್ರದ
ಕುರುಡ
ಕುರುಡನ
ಕುರುಹಲ್ಲ
ಕುರುಹು
ಕುರುಹೇ
ಕುರೂಪಿ
ಕುರ್ಚಿ
ಕುರ್ಚಿ-ಗಿಂತಲೂ
ಕುರ್ಚಿಗೆ
ಕುರ್ಚಿನ್ನು
ಕುರ್ಚಿಯ
ಕುರ್ಚಿ-ಯನ್ನು
ಕುರ್ಚಿ-ಯಾ-ಗಿಯೇ
ಕುಲಗೋತ್ರ-ಗಳಿಲ್ಲ
ಕುಲದ
ಕುಲದ-ವರಿ-ಗಿಂತ
ಕುಲ-ದೇವ-ತೆ-ಗಳ
ಕುಲ-ಭಾ-ವನೆ
ಕುಲುವೆಗೆ
ಕುಳಿತ
ಕುಳಿತರೆ
ಕುಳಿತಾಗ
ಕುಳಿತಿ-ದೆಯೋ
ಕುಳಿತಿರು
ಕುಳಿತಿ-ರುವ
ಕುಳಿತಿರು-ವರು
ಕುಳಿತಿರು-ವು-ದನ್ನು
ಕುಳಿತಿರು-ವುವು
ಕುಳಿತಿರು-ವೆವು
ಕುಳಿತು
ಕುಳಿತು-ಕೊಂಡಿತು
ಕುಳಿತು-ಕೊಂಡಿ-ರುವ
ಕುಳಿತು-ಕೊಂಡಿ-ರು-ವಾಗ
ಕುಳಿತು-ಕೊಂಡಿ-ರು-ವಿರಿ-ನೀವು
ಕುಳಿತು-ಕೊಂಡು
ಕುಳಿತು-ಕೊಳ್ಳ
ಕುಳಿತು-ಕೊಳ್ಳ-ಬೇಕು
ಕುಳಿತು-ಕೊಳ್ಳಲು
ಕುಳಿತು-ಕೊಳ್ಳಲೂ
ಕುಳಿತು-ಕೊಳ್ಳಿ
ಕುಳಿತು-ಕೊಳ್ಳು-ವರು
ಕುಳಿತು-ಕೊಳ್ಳು-ವುದು
ಕುಳಿತುಕೋ
ಕುಳಿತುದು
ಕುಶಲಿ-ಯಾಗಿ-ರ-ಬೇಕು
ಕುಸಿಯುತ್ತಿದೆ
ಕುಸಿಯು-ವುದು
ಕುಸುಮ-ವನ್ನು
ಕೂಗಲು
ಕೂಗಾಟಕ್ಕೆ
ಕೂಗಾಡಿ
ಕೂಗಾಡುವ
ಕೂಗಿ-ಕೊಳ್ಳು-ವುದು
ಕೂಟಕ್ಕೆ
ಕೂಟಸ್ಥ
ಕೂಡ
ಕೂಡ-ಅ-ವರು
ಕೂಡದು
ಕೂಡದೆ
ಕೂಡ-ಯಾ-ರನ್ನೂ
ಕೂಡಲೆ
ಕೂಡಲೇ
ಕೂಡಾ
ಕೂಡಿ
ಕೂಡಿ-ಕೊಂಡಿರ
ಕೂಡಿ-ಕೊಂಡಿ-ರು-ವರು
ಕೂಡಿಟ್ಟ
ಕೂಡಿಟ್ಟಂತೆ
ಕೂಡಿದ
ಕೂಡಿ-ದ-ವ-ರಿಗೆ
ಕೂಡಿ-ದುದು
ಕೂಡಿ-ದುದೆ
ಕೂಡಿದೆ
ಕೂಡಿದ್ದರೂ
ಕೂಡಿದ್ದರೆ
ಕೂಡಿದ್ದುವು
ಕೂಡಿ-ರ-ದಿದ್ದರೆ
ಕೂಡಿ-ರ-ಬೇಕು
ಕೂಡಿ-ರುತ್ತಾನೆ
ಕೂಡಿ-ರುವ
ಕೂಡಿ-ರು-ವಂತೆ
ಕೂಡಿ-ರುವನು
ಕೂಡಿ-ರು-ವರೊ
ಕೂಡಿ-ರು-ವು-ದಕ್ಕೆ
ಕೂಡಿ-ರು-ವು-ದ-ರಿಂದ
ಕೂಡಿ-ರುವುದು
ಕೂಡಿ-ರುವುದೇ
ಕೂಡಿವೆ
ಕೂಡಿ-ಸಿ-ಕೊಂಡು
ಕೂಡಿ-ಸಿದ
ಕೂಡು-ವುದು
ಕೂದ-ಲಿಗೆ
ಕೂದಲೂ
ಕೂಪ-ದಲ್ಲಿದ್ದಿರಿ
ಕೂರ್ಮ-ನಾಡಿಯ
ಕೂರ್ಮ-ನಾಡ್ಯಾಂ
ಕೂರ್ಮ-ಪುರಾಣ-ದಲ್ಲಿ
ಕೂಲಂಕಷ
ಕೂಲಂಕಷ-ವಾಗಿ
ಕೂಲಂಕುಷ
ಕೂಲಂಕುಷ-ವಾಗಿ
ಕೂಳಿ-ಗಾಗಿ
ಕೃತ
ಕೃತ-ಕ-ವಾಗಿ
ಕೃತ-ಕಾರಿ-ತಾನು-ಮೋದಿತ
ಕೃತ-ಕೃತ್ಯ-ವಾ-ಗು-ವುವು
ಕೃತಘ್ನ-ರಾಗಿ
ಕೃತಾರ್ಥಂ
ಕೃತಾರ್ಥಾನಾಂ
ಕೃತಿ-ಆಲೋ-ಚನೆ-ಯಿಂದಲೂ
ಕೃತಿಶ್ರೇಣಿ
ಕೃತ್ಯ-ಗಳಿಗೆ
ಕೃಪಣ
ಕೃಪಾ-ನಂದ-ವನ್ನು
ಕೃಪೆ
ಕೃಶ
ಕೃಶ-ವಾಗಿ-ರಲಿ
ಕೃಷ್ಣ
ಕೃಷ್ಣ-ಪಕ್ಷಕ್ಕೆ
ಕೆಂದ್ರ-ಗಳೇ
ಕೆಂಪಗೊ
ಕೆಂಪಾಗಿ
ಕೆಂಪಾಗಿದ್ದರೆ
ಕೆಂಪಿಸ್ನ
ಕೆಂಪು
ಕೆಗೆ
ಕೆಚ್ಚು
ಕೆಟ್ಟ
ಕೆಟ್ಟ-ಕರ್ಮ-ವಿದ್ದರೆ
ಕೆಟ್ಟ-ಚಾಳಿ-ಯವ-ನಾಗಿ-ರ-ಬಹುದು
ಕೆಟ್ಟ-ದಕ್ಕೆ
ಕೆಟ್ಟ-ದಕ್ಕೊ
ಕೆಟ್ಟ-ದನ್ನು
ಕೆಟ್ಟ-ದರ
ಕೆಟ್ಟ-ದ-ರಂತೆ
ಕೆಟ್ಟ-ದ-ರಲ್ಲಿ
ಕೆಟ್ಟ-ದಾಗಲಿ
ಕೆಟ್ಟ-ದಾಗಲೀ
ಕೆಟ್ಟ-ದಾ-ವುದು
ಕೆಟ್ಟ-ದಿದ್ದರೆ
ಕೆಟ್ಟ-ದಿಲ್ಲದೆ
ಕೆಟ್ಟದು
ಕೆಟ್ಟದ್ದಕ್ಕಿಂತ
ಕೆಟ್ಟದ್ದಕ್ಕೆ
ಕೆಟ್ಟದ್ದನ್ನು
ಕೆಟ್ಟದ್ದನ್ನೆಲ್ಲ
ಕೆಟ್ಟದ್ದರ
ಕೆಟ್ಟದ್ದ-ರಂತೆ
ಕೆಟ್ಟದ್ದ-ರಿಂದ
ಕೆಟ್ಟದ್ದಾಗ-ಬಹುದು
ಕೆಟ್ಟದ್ದಾ-ಗುತ್ತದೆ
ಕೆಟ್ಟದ್ದಿಲ್ಲದೆ
ಕೆಟ್ಟದ್ದು
ಕೆಟ್ಟದ್ದೂ
ಕೆಟ್ಟದ್ದೇ
ಕೆಟ್ಟ-ವ-ನಲ್ಲ
ಕೆಟ್ಟ-ವ-ನಾಗುತ್ತಾನೆ
ಕೆಟ್ಟ-ವನಾ-ದರೆ
ಕೆಟ್ಟ-ವನು
ಕೆಟ್ಟ-ವ-ರೆಂದು
ಕೆಟ್ಟವು
ಕೆಟ್ಟು-ದನ್ನು
ಕೆಟ್ಟು-ದನ್ನೋ
ಕೆಟ್ಟು-ದರ
ಕೆಟ್ಟುದು
ಕೆಟ್ಟುವು
ಕೆಡಕು-ಗಳಿಲ್ಲ
ಕೆಡಹು-ವುದು
ಕೆಡಿಸಿ
ಕೆಡಿಸಿ-ಕೊಳ್ಳ-ಬೇಡಿ
ಕೆಡಿಸಿ-ಕೊಳ್ಳು-ವರು
ಕೆಡಿಸಿ-ಕೊಳ್ಳು-ವಿರಿ
ಕೆಡಿಸು
ಕೆಡಿಸುತ್ತಿ-ರು-ವರು
ಕೆಡಿಸು-ವರು
ಕೆಡಿ-ಸು-ವುದು
ಕೆಡಿ-ಸು-ವುವು
ಕೆಡುಕು-ಗಳು
ಕೆಥೊಲಿಕ್
ಕೆನ್ನೆಗೆ
ಕೆನ್ನೆ-ಯನ್ನು
ಕೆರಳಿ
ಕೆರಳಿಸಿ
ಕೆರಳಿ-ಸುವ
ಕೆರೆಗೆ
ಕೆರೆ-ಯಲ್ಲಿ
ಕೆರೆ-ಯಷ್ಟು
ಕೆಲ-ಕಾಲ
ಕೆಲ-ವಕ್ಕೆ
ಕೆಲ-ವನ್ನು
ಕೆಲ-ವರ
ಕೆಲವ-ರನ್ನು
ಕೆಲವ-ರಲ್ಲಿ
ಕೆಲವ-ರಿಂದ
ಕೆಲ-ವರಿ-ಗಿದೆ
ಕೆಲವ-ರಿಗೂ
ಕೆಲ-ವ-ರಿಗೆ
ಕೆಲ-ವರಿ-ರ-ಬಹುದು
ಕೆಲ-ವರಿ-ರು-ವರು
ಕೆಲ-ವರು
ಕೆಲ-ವರೇ-ತಕ್ಕೆ
ಕೆಲವು
ಕೆಲವೆ
ಕೆಲವೊಮ್ಮೆ
ಕೆಲಸ
ಕೆಲಸಕ್ಕೆ
ಕೆಲಸಕ್ಕೆ-ಬಾ-ರದ
ಕೆಲಸ-ಗಳ
ಕೆಲಸ-ಗ-ಳನ್ನು
ಕೆಲಸ-ಗಳಿಗೆ
ಕೆಲಸ-ಗಳು
ಕೆಲಸ-ಗಳೂ
ಕೆಲಸ-ಗಳೆಲ್ಲ
ಕೆಲಸ-ಗಾರ-ನಲ್ಲ
ಕೆಲ-ಸದ
ಕೆಲಸ-ದಲ್ಲಿ
ಕೆಲಸ-ದಲ್ಲಿದ್ದೆ
ಕೆಲಸ-ದಿಂದ
ಕೆಲಸ-ದೊಂದಿಗೆ
ಕೆಲಸ-ಮಾಡ-ಬಲ್ಲನು
ಕೆಲಸ-ಮಾಡ-ಬೇಕಾ-ಗಿದೆ
ಕೆಲಸ-ಮಾಡ-ಬೇಕಾ-ಗಿಲ್ಲ
ಕೆಲಸ-ಮಾಡ-ಬೇಕಾದ
ಕೆಲಸ-ಮಾಡ-ಬೇಕು
ಕೆಲಸ-ಮಾಡಿ
ಕೆಲಸ-ಮಾಡಿದ
ಕೆಲಸ-ಮಾಡಿ-ದರೂ
ಕೆಲಸ-ಮಾಡುತ್ತಿ-ರುವ
ಕೆಲಸ-ಮಾಡುತ್ತಿ-ರುವ-ವರು
ಕೆಲಸ-ಮಾಡುತ್ತಿ-ರು-ವಿರೊ
ಕೆಲಸ-ಮಾಡುತ್ತಿ-ರು-ವುವು
ಕೆಲಸ-ಮಾಡುತ್ತಿ-ರು-ವೆನು
ಕೆಲಸ-ಮಾಡುತ್ತೀರಿ
ಕೆಲಸ-ಮಾಡುವ
ಕೆಲಸ-ಮಾಡು-ವುದು
ಕೆಲಸ-ಮಾಡು-ವುದೇ
ಕೆಲಸ-ಮಾಡು-ವುದೇನೊ
ಕೆಲಸ-ಮಾಡು-ವುವು-ಅಂತಹ
ಕೆಲಸ-ಮಾಡು-ವೆವು
ಕೆಲಸ-ಮಾಡೋಣ
ಕೆಲಸ-ವನ್ನು
ಕೆಲಸ-ವನ್ನೂ
ಕೆಲಸ-ವಲ್ಲ
ಕೆಲಸ-ವಾದ
ಕೆಲಸ-ವಾ-ದರೂ
ಕೆಲಸ-ವಿದೆ
ಕೆಲಸವು
ಕೆಲಸವೂ
ಕೆಲಸವೆ
ಕೆಲಸ-ವೆಂದ-ರೇನು
ಕೆಲಸ-ವೆಂದು
ಕೆಲಸ-ವೆಲ್ಲ
ಕೆಲಸ-ವೆಲ್ಲಾ
ಕೆಲಸವೇ
ಕೆಳ
ಕೆಳಕ್ಕೆ
ಕೆಳ-ಗಿನ
ಕೆಳ-ಗಿನದು
ಕೆಳ-ಗಿನ-ದು-ಎ-ರಡು
ಕೆಳ-ಗಿನ-ರುವ
ಕೆಳ-ಗಿನ-ವ-ರಲ್ಲಿ
ಕೆಳ-ಗಿ-ನಿಂದ
ಕೆಳ-ಗಿ-ರುವ
ಕೆಳ-ಗಿ-ರುವ-ವರು
ಕೆಳ-ಗಿ-ರುವುದು
ಕೆಳಗೆ
ಕೆಳ-ದರ್ಜೆ
ಕೆಳ-ಭಾಗದ
ಕೆಳ-ಭಾಗ-ದಲ್ಲಿ
ಕೆಳ-ಭಾಗವು
ಕೆಳ-ಮಟ್ಟದ
ಕೆಳ-ಮಟ್ಟದ-ವರು
ಕೆಳ-ಮುಖ
ಕೆಳ-ವರ್ಗದ
ಕೆಳೆ-ಗೆಳೆದು
ಕೆಸರಿನ
ಕೆಸರಿ-ನಲ್ಲಿ
ಕೇಂದ್ರ
ಕೇಂದ್ರ-ಕೆಕ
ಕೇಂದ್ರಕ್ಕೆ
ಕೇಂದ್ರಕ್ಕೇಕೆ
ಕೇಂದ್ರ-ಗಳ
ಕೇಂದ್ರ-ಗ-ಳನ್ನು
ಕೇಂದ್ರ-ಗಳಲ್ಲಿ
ಕೇಂದ್ರ-ಗಳಿಂದ
ಕೇಂದ್ರ-ಗಳಿ-ರು-ವುವು
ಕೇಂದ್ರ-ಗಳು
ಕೇಂದ್ರ-ಗಳೊಂದಿಗೆ
ಕೇಂದ್ರದ
ಕೇಂದ್ರ-ದಂತೆ
ಕೇಂದ್ರ-ದಲ್ಲಿ
ಕೇಂದ್ರ-ದಲ್ಲಿದೆ
ಕೇಂದ್ರ-ದಲ್ಲಿ-ರುವ-ವನೆ
ಕೇಂದ್ರ-ದಿಂದ
ಕೇಂದ್ರ-ದೊಂದಿಗೆ
ಕೇಂದ್ರ-ದೊ-ಡನೆ
ಕೇಂದ್ರ-ಭಾ-ವನೆ
ಕೇಂದ್ರ-ವನ್ನು
ಕೇಂದ್ರ-ವಾದ
ಕೇಂದ್ರ-ವಾ-ದರೆ
ಕೇಂದ್ರ-ವಿದೆ
ಕೇಂದ್ರ-ವಿರ-ಬೇಕಾ-ಯಿತು
ಕೇಂದ್ರ-ವಿಲ್ಲ
ಕೇಂದ್ರವು
ಕೇಂದ್ರವೂ
ಕೇಂದ್ರ-ವೆಂತಲೂ
ಕೇಂದ್ರ-ವೆನ್ನು-ವನು
ಕೇಂದ್ರ-ವೆಲ್ಲಿ-ರುವುದು
ಕೇಂದ್ರವೇ
ಕೇಂದ್ರಾ-ಕರ್ಷಕ
ಕೇಂದ್ರಾಪ-ಕರ್ಷಕ
ಕೇಂದ್ರೀ-ಕರಿಸ-ಬಹುದು
ಕೇಂದ್ರೀ-ಕರಿ-ಸ-ಬೇಕು
ಕೇಂದ್ರೀಕ-ರಿಸಿ
ಕೇಂದ್ರೀಕ-ರಿಸಿ-ದಾಗ
ಕೇಂದ್ರೀ-ಕರಿ-ಸುವ
ಕೇಂದ್ರೀ-ಕರಿ-ಸುವನು
ಕೇಂದ್ರೀ-ಕರಿ-ಸುವು-ದಕ್ಕೆ
ಕೇಂದ್ರೀ-ಕರಿ-ಸು-ವುದು
ಕೇಂದ್ರೀ-ಕೃತ-ವಾಗಿ
ಕೇಂದ್ರೀ-ಕೃತ-ವಾಗು
ಕೇಂದ್ರೀ-ಕೃತ-ವಾದ
ಕೇಂದ್ರೀ-ಕೃತ-ವಾ-ದಂತೆ
ಕೇಡನ್ನು
ಕೇಡಲ್ಲ
ಕೇಡಾಗು-ವು-ದಿಲ್ಲ
ಕೇಡಿಗೆ
ಕೇಡು
ಕೇಳ
ಕೇಳ-ಕೂಡದು
ಕೇಳ-ದಿದ್ದರೆ
ಕೇಳದೆ
ಕೇಳ-ಬಲ್ಲನು
ಕೇಳ-ಬಹು-ದಾ-ದರೂ
ಕೇಳ-ಬಹುದು
ಕೇಳ-ಬೇಕಾ-ದರೆ
ಕೇಳ-ಬೇಕು
ಕೇಳ-ಬೇಡಿ
ಕೇಳಲಿ
ಕೇಳಲು
ಕೇಳ-ಲು-ಪಕ್ರಮಿ-ಸು-ವುದು
ಕೇಳ-ಲೇ-ಬೇಕಾ-ಗಿದೆ
ಕೇಳಿ
ಕೇಳಿ-ಕೊಂಡ
ಕೇಳಿ-ಕೊಂಡಂತೆ
ಕೇಳಿ-ಕೊಂಡನು
ಕೇಳಿ-ಕೊಂಡರು
ಕೇಳಿ-ಕೊಂಡಿ-ರು-ವರು
ಕೇಳಿ-ಕೊಳ್ಳುತ್ತೇನೆ
ಕೇಳಿತು
ಕೇಳಿದ
ಕೇಳಿ-ದ-ನಂತೆ
ಕೇಳಿ-ದನು
ಕೇಳಿ-ದರು
ಕೇಳಿ-ದರೆ
ಕೇಳಿ-ದರೇ
ಕೇಳಿ-ದಳು
ಕೇಳಿ-ದಾಗ
ಕೇಳಿ-ದಿರಾ
ಕೇಳಿ-ದಿರಿ
ಕೇಳಿದೆ
ಕೇಳಿ-ದೆನೋ
ಕೇಳಿ-ದೊ-ಡ-ನೆಯೆ
ಕೇಳಿದ್ದ
ಕೇಳಿದ್ದನೊ
ಕೇಳಿದ್ದೀರಿ
ಕೇಳಿದ್ದೆ
ಕೇಳಿದ್ದೇನೆ
ಕೇಳಿದ್ದೇವೆ
ಕೇಳಿ-ಬ-ರು-ವುದು
ಕೇಳಿ-ರ-ಬಹುದು
ಕೇಳಿರು
ಕೇಳಿ-ರು-ವನು
ಕೇಳಿ-ರು-ವರು
ಕೇಳಿ-ರು-ವರೋ
ಕೇಳಿ-ರು-ವ-ವರು
ಕೇಳಿ-ರು-ವಿರಿ
ಕೇಳಿ-ರು-ವು-ದನ್ನು
ಕೇಳಿ-ರು-ವೆನು
ಕೇಳಿ-ರು-ವೆವು
ಕೇಳಿಲ್ಲ
ಕೇಳಿ-ಸ-ಲಿಲ್ಲ
ಕೇಳಿ-ಸಿತು
ಕೇಳಿ-ಸುತ್ತದೆ
ಕೇಳಿ-ಸು-ವು-ದಿಲ್ಲ
ಕೇಳು
ಕೇಳುತ್ತಲೂ
ಕೇಳುತ್ತಾನೆ
ಕೇಳುತ್ತಾನೋ
ಕೇಳುತ್ತಾರೆ
ಕೇಳುತ್ತಿದ್ದ
ಕೇಳುತ್ತಿದ್ದರೂ
ಕೇಳುತ್ತಿದ್ದರೆ
ಕೇಳುತ್ತಿರುತ್ತಾನೆ
ಕೇಳುತ್ತಿ-ರುವನು
ಕೇಳುತ್ತಿ-ರು-ವಿರಿ
ಕೇಳುತ್ತಿ-ರು-ವಿರೊ
ಕೇಳುತ್ತಿ-ರು-ವು-ದ-ರಿಂದ
ಕೇಳುತ್ತಿ-ರು-ವೆನು
ಕೇಳುತ್ತಿ-ರು-ವೆವು
ಕೇಳುತ್ತೀರಿ
ಕೇಳುತ್ತೇನೆ
ಕೇಳುತ್ತೇವೆ
ಕೇಳುವ
ಕೇಳು-ವನು
ಕೇಳು-ವನೋ
ಕೇಳು-ವರು
ಕೇಳು-ವ-ವ-ರೆಲ್ಲೋ
ಕೇಳು-ವಿರಿ
ಕೇಳು-ವು-ದಕ್ಕೂ
ಕೇಳು-ವು-ದಕ್ಕೆ
ಕೇಳು-ವು-ದಕ್ಕೇನೋ
ಕೇಳು-ವು-ದ-ರಿಂದ
ಕೇಳು-ವು-ದಿಲ್ಲ
ಕೇಳು-ವು-ದಿಲ್ಲವೊ
ಕೇಳು-ವುದು
ಕೇಳು-ವುದೂ
ಕೇಳು-ವುದೇ
ಕೇಳು-ವುವು
ಕೇಳು-ವೆವು
ಕೇಳೋಣ
ಕೇವಲ
ಕೇವಲ-ವಾದ
ಕೇಸರ-ಗಳು
ಕೇಸರ-ಗಳೇ
ಕೈ
ಕೈಕಟ್ಟಿ
ಕೈಕೆ-ಳಗೆ
ಕೈಗ-ಳನ್ನು
ಕೈಗಳಲ್ಲಿ
ಕೈಗಳಿಂದ
ಕೈಗಳಿದ
ಕೈಗಳು
ಕೈಗಳೂ
ಕೈಗೆ
ಕೈಗೊಳ್ಳುತ್ತದೆ
ಕೈಬಿಟ್ಟಿ-ರು-ವರು
ಕೈಬೆರಳಿ-ನಲ್ಲಿ
ಕೈಬೆರಳು-ಗಳಿಂದ
ಕೈಯನ್ನು
ಕೈಯನ್ನೊ
ಕೈಯಲ್ಲಿ
ಕೈಯಲ್ಲಿದ್ದಿತೋ
ಕೈಯಲ್ಲಿಯೂ
ಕೈಯಲ್ಲಿ-ರುವುದು
ಕೈಯಲ್ಲಿ-ರುವುದೋ
ಕೈಯಿಂದ
ಕೈಯಿಡುತ್ತಿದ್ದಿರಿ
ಕೈಯ್ಯಲ್ಲಿ
ಕೈಯ್ಯಲ್ಲಿದ್ದ
ಕೈಲಾದ
ಕೈಲಾ-ದಷ್ಟು
ಕೈವಲ್ಯ
ಕೈವಲ್ಯಂ
ಕೈವಲ್ಯ-ಪ-ದವಿ
ಕೈವಲ್ಯ-ಪದ-ವಿಗೆ
ಕೈವಲ್ಯ-ಪ-ದವಿ-ಯನ್ನು
ಕೈವಲ್ಯಪ್ರಾಗ್ಭಾವಂ
ಕೈವಲ್ಯ-ಮಿತಿ
ಕೈವಲ್ಯಮ್
ಕೈವಾಡ-ವಿ-ರ-ಬೇಕು
ಕೈಹಾಕ-ಬೇಕು
ಕೊಂಚ
ಕೊಂಚ-ಕಾಲ
ಕೊಂಡನು
ಕೊಂಡರು
ಕೊಂಡರೂ
ಕೊಂಡರೆ
ಕೊಂಡಾಗ
ಕೊಂಡಿ
ಕೊಂಡಿ-ಗಳ
ಕೊಂಡಿ-ಗಳಂತೆ
ಕೊಂಡಿ-ಗ-ಳನ್ನು
ಕೊಂಡಿ-ಗಳಿವೆ
ಕೊಂಡಿತು
ಕೊಂಡಿದ್ದರೂ
ಕೊಂಡಿದ್ದೇವೆ
ಕೊಂಡಿ-ಯನ್ನು
ಕೊಂಡಿರು
ಕೊಂಡಿ-ರುವ
ಕೊಂಡಿ-ರು-ವನೊ
ಕೊಂಡಿ-ರು-ವರು
ಕೊಂಡಿ-ರು-ವುದು
ಕೊಂಡಿ-ರು-ವೆವು
ಕೊಂಡಿಲ್ಲ
ಕೊಂಡಿಲ್ಲವೋ
ಕೊಂಡು
ಕೊಂಡೊಯ್ಯುವ
ಕೊಂಡೊಯ್ಯು-ವನು
ಕೊಂಡೊಯ್ಯುವುದೇ
ಕೊಂದ
ಕೊಂದು
ಕೊಂದು-ಕೊಳ್ಳ-ಬಲ್ಲೆನೆ
ಕೊಂಬು
ಕೊಂಬೆ-ಗಳೆಲ್ಲಕ್ಕೂ
ಕೊಚ್ಚಿ-ಕೊಂಡರೂ
ಕೊಚ್ಚಿ-ಕೊಂಡು
ಕೊಚ್ಚಿ-ಕೊಂಡು-ಹೋ-ದಳು
ಕೊಚ್ಚಿ-ಕೊಳ್ಳ-ಬಹುದು
ಕೊಚ್ಚಿ-ಕೊಳ್ಳುತ್ತೀರಿ
ಕೊಚ್ಚಿ-ಕೊಳ್ಳುವ
ಕೊಟ್ಟ
ಕೊಟ್ಟನು
ಕೊಟ್ಟರು
ಕೊಟ್ಟರೂ
ಕೊಟ್ಟರೆ
ಕೊಟ್ಟಷ್ಟು
ಕೊಟ್ಟಾ-ದರೂ
ಕೊಟ್ಟಿತು
ಕೊಟ್ಟಿದೆ
ಕೊಟ್ಟಿದ್ದರು
ಕೊಟ್ಟಿದ್ದೀರಿ
ಕೊಟ್ಟಿ-ರುವನು
ಕೊಟ್ಟಿ-ರುವ-ನೆಂದೂ
ಕೊಟ್ಟಿ-ರು-ವರು
ಕೊಟ್ಟಿ-ರು-ವುದು
ಕೊಟ್ಟಿಲ್ಲ
ಕೊಟ್ಟು
ಕೊಟ್ಟು-ಕೊಳ್ಳ-ಬಲ್ಲೆನೆ
ಕೊಟ್ಟು-ಬಿಡಿ
ಕೊಟ್ಟೆವು
ಕೊಡ
ಕೊಡದೆ
ಕೊಡ-ಬಲ್ಲ
ಕೊಡ-ಬಲ್ಲ-ದೆಂದು
ಕೊಡ-ಬಲ್ಲಿರಾ
ಕೊಡ-ಬಹು-ದಾದ
ಕೊಡ-ಬಹುದು
ಕೊಡ-ಬಾ-ರದು
ಕೊಡ-ಬಾರ-ದೆಂದೂ
ಕೊಡ-ಬೇಕಾ-ಗಿತ್ತು
ಕೊಡ-ಬೇಕಾ-ಗಿದೆ
ಕೊಡ-ಬೇಕಾ-ಗಿಲ್ಲ
ಕೊಡ-ಬೇಕಾ-ಯಿತು
ಕೊಡ-ಬೇಕು
ಕೊಡ-ಬೇಕೆಂದಿ-ರು-ವಿರಿ
ಕೊಡ-ಬೇಡಿ
ಕೊಡ-ಲಾಗ-ಲಿಲ್ಲ
ಕೊಡ-ಲಾ-ಗಿದೆ
ಕೊಡ-ಲಾ-ಗು-ವುದು
ಕೊಡ-ಲಾರ
ಕೊಡ-ಲಾರದು
ಕೊಡ-ಲಾರ-ದು-ಯುಕ್ತಿ
ಕೊಡ-ಲಾರ-ದೆಂದು
ಕೊಡ-ಲಾರರು
ಕೊಡ-ಲಾರವು
ಕೊಡ-ಲಾರೆ
ಕೊಡಲಿ-ಆತ್ಮ-ವೆಂಬ
ಕೊಡಲಿಚ್ಛಿ-ಸು-ವನು
ಕೊಡಲಿ-ಯಿಂದ
ಕೊಡ-ಲಿಲ್ಲ
ಕೊಡಲು
ಕೊಡಲೇ-ಬೇಕು
ಕೊಡಲ್ಪಟ್ಟಿವೆ
ಕೊಡಿ
ಕೊಡು
ಕೊಡು-ಗೆ-ಇವು-ಗಳ
ಕೊಡುತ್ತದೆ
ಕೊಡುತ್ತದೆ-ಎಂಬು-ದನ್ನೆಲ್ಲಾ
ಕೊಡುತ್ತದೆಯೋ
ಕೊಡುತ್ತವೆ
ಕೊಡುತ್ತಾನೆ
ಕೊಡುತ್ತಾರೆ
ಕೊಡುತ್ತಿದ್ದ
ಕೊಡುತ್ತಿದ್ದರು
ಕೊಡುತ್ತಿ-ರ-ಲಿಲ್ಲ
ಕೊಡುತ್ತಿರು
ಕೊಡುತ್ತಿರುವ
ಕೊಡುತ್ತಿರುವೆ
ಕೊಡುತ್ತಿರು-ವೆನು
ಕೊಡುತ್ತಿರು-ವೆವು
ಕೊಡುತ್ತಿಲ್ಲ
ಕೊಡುತ್ತೀರಿ
ಕೊಡುತ್ತೇನೆ
ಕೊಡುವ
ಕೊಡು-ವಂತಿ-ರ-ಬೇಕು
ಕೊಡು-ವಂತೆ
ಕೊಡು-ವನು
ಕೊಡು-ವರು
ಕೊಡು-ವ-ವ-ನದು
ಕೊಡು-ವ-ವನು
ಕೊಡು-ವ-ವರ
ಕೊಡು-ವಾಗ
ಕೊಡು-ವಿರಿ
ಕೊಡುವು
ಕೊಡು-ವು-ದಕ್ಕೆ
ಕೊಡು-ವು-ದ-ರಲ್ಲೇ
ಕೊಡು-ವು-ದ-ರಿಂದ
ಕೊಡು-ವು-ದಾ-ವುದು
ಕೊಡು-ವು-ದಿಲ್ಲ
ಕೊಡು-ವು-ದಿಲ್ಲವೋ
ಕೊಡು-ವುದು
ಕೊಡು-ವುದೂ
ಕೊಡು-ವುದೆ
ಕೊಡು-ವು-ದೆಂದು
ಕೊಡು-ವುದೇ
ಕೊಡು-ವುವು
ಕೊಡುವೆ
ಕೊಡು-ವೆನು
ಕೊಡು-ವೆವು
ಕೊನೆ
ಕೊನೆ-ಗಳೂ
ಕೊನೆ-ಗಾ-ಣದ
ಕೊನೆ-ಗಾಣ-ಬೇಕು
ಕೊನೆ-ಗಾಣ-ಬೇಕು-ಆದ-ಕಾರಣ
ಕೊನೆ-ಗಾಣಿ-ಸು-ವುದು
ಕೊನೆ-ಗಾಣು
ಕೊನೆ-ಗಾಣು-ವು-ದಿಲ್ಲ
ಕೊನೆ-ಗಾಣು-ವುದು
ಕೊನೆ-ಗಾಣು-ವುದೇ
ಕೊನೆ-ಗಾಣು-ವುವು
ಕೊನೆಗೂ
ಕೊನೆಗೆ
ಕೊನೆ-ಗೊಂಡ
ಕೊನೆ-ಗೊಂಡಂತೆ
ಕೊನೆ-ಗೊಂಡಂತೆಯೆ
ಕೊನೆ-ಗೊಂಡರೆ
ಕೊನೆ-ಗೊಳ್ಳುತ್ತದೆ
ಕೊನೆ-ಗೊಳ್ಳು-ವುವು
ಕೊನೆ-ಮೊದ-ಲಿಲ್ಲ
ಕೊನೆ-ಮೊದ-ಲಿಲ್ಲದ
ಕೊನೆ-ಮೊದ-ಲಿಲ್ಲದೆ
ಕೊನೆಯ
ಕೊನೆ-ಯ-ದಾಗಿ
ಕೊನೆ-ಯ-ದಾದ
ಕೊನೆ-ಯದು
ಕೊನೆ-ಯದೇ
ಕೊನೆ-ಯನ್ನು
ಕೊನೆ-ಯಲ್ಲಿ
ಕೊನೆ-ಯ-ವರೆಗೂ
ಕೊನೆ-ಯ-ವರೆಗೆ
ಕೊನೆ-ಯ-ವರೆ-ವಿಗೂ
ಕೊನೆ-ಯಿರು-ವು-ದಿಲ್ಲ
ಕೊನೆಯು
ಕೊನೆಯೇ
ಕೊರಗು
ಕೊರಡು-ಗ-ಳನ್ನು
ಕೊರತೆ
ಕೊರತೆ-ಗಳು
ಕೊರಳನ್ನೇಕೆ
ಕೊರಳಿಗೆ
ಕೊಲೆ
ಕೊಲೆ-ಪಾತಕಿ
ಕೊಲೆ-ಪಾತ-ಕಿ-ಗ-ಳನ್ನು
ಕೊಲೆ-ಪಾತ-ಕಿ-ಯಲ್ಲಿಯೂ
ಕೊಲೆ-ಮಾಡ-ಬಹು-ದೆಂಬುದು
ಕೊಲೆ-ಯಾಗ-ಬೇಕು
ಕೊಲೆ-ಯೇನೊ
ಕೊಲ್ಲದೆ
ಕೊಲ್ಲ-ಬಲ್ಲುದು
ಕೊಲ್ಲ-ಬೇಕೆಂದು
ಕೊಲ್ಲ-ಬೇಕೆಂಬ
ಕೊಲ್ಲ-ಲಾರೆ
ಕೊಲ್ಲಲು
ಕೊಲ್ಲಲ್ಪಟ್ಟರು
ಕೊಲ್ಲಿ
ಕೊಲ್ಲಿ-ಸಿ-ಕೊಳ್ಳು-ವ-ವನು
ಕೊಲ್ಲಿ-ಸಿ-ಕೊಳ್ಳು-ವು-ದಿಲ್ಲ
ಕೊಲ್ಲು
ಕೊಲ್ಲುತ್ತಿದ್ದರು
ಕೊಲ್ಲುತ್ತೇನೆ
ಕೊಲ್ಲುತ್ತೇವೆ
ಕೊಲ್ಲುವ
ಕೊಲ್ಲು-ವಂತಹ
ಕೊಲ್ಲು-ವರು
ಕೊಲ್ಲು-ವ-ವನು
ಕೊಲ್ಲು-ವು-ದಕ್ಕೆ
ಕೊಲ್ಲು-ವು-ದನ್ನು
ಕೊಲ್ಲು-ವು-ದಾಗಿ
ಕೊಲ್ಲು-ವುದು
ಕೊಲ್ಲು-ವು-ದೊಂದು
ಕೊಲ್ಲು-ವುದೊಂದೆ
ಕೊಲ್ಲು-ವೆನು
ಕೊಳಕಾದ-ವನು
ಕೊಳ-ಲಿನ
ಕೊಳುತ್ತ
ಕೊಳೆ
ಕೊಳೆತು
ಕೊಳೆ-ಯೊಂದಿಗೆ
ಕೊಳ್ಳ-ಕೂಡದು
ಕೊಳ್ಳ-ಬಲ್ಲದೆಂಬು-ದನ್ನು
ಕೊಳ್ಳ-ಬಲ್ಲರು
ಕೊಳ್ಳ-ಬಹು-ದಾದ
ಕೊಳ್ಳ-ಬೇಕು
ಕೊಳ್ಳ-ಬೇಕೆಂಬು-ದನ್ನು
ಕೊಳ್ಳ-ಲಾರರು
ಕೊಳ್ಳಲು
ಕೊಳ್ಳಿ
ಕೊಳ್ಳುತ್ತಾನೆ
ಕೊಳ್ಳುತ್ತಿ-ರು-ವೆವು
ಕೊಳ್ಳುತ್ತೀರಿ
ಕೊಳ್ಳುತ್ತೇನೆ
ಕೊಳ್ಳುವ
ಕೊಳ್ಳು-ವಂತೆ
ಕೊಳ್ಳು-ವನು
ಕೊಳ್ಳು-ವ-ವನೂ
ಕೊಳ್ಳು-ವ-ವರ
ಕೊಳ್ಳು-ವು-ದಕ್ಕೂ
ಕೊಳ್ಳು-ವು-ದಕ್ಕೆ
ಕೊಳ್ಳು-ವು-ದ-ರಿಂದ
ಕೊಳ್ಳು-ವು-ದಿಲ್ಲ
ಕೊಳ್ಳು-ವುದು
ಕೊಳ್ಳೋಣ
ಕೋಟಿ
ಕೋಟಿ-ಗಟ್ಟಲೆ
ಕೋಟಿ-ಗಿಂತ
ಕೋಟಿಗೆ
ಕೋಟಿ-ಗೊಬ್ಬನೂ
ಕೋಟಿ-ಪಾಲು
ಕೋಟಿಯ
ಕೋಟಿ-ಯಲ್ಲಿ
ಕೋಟಿಯು
ಕೋಟೆ
ಕೋಟೆಯ
ಕೋಟೆ-ಯಿಂದ
ಕೋಟ್ಯಂತರ
ಕೋಡಿ-ಯನ್ನು
ಕೋಣೆ
ಕೋಣೆಗೆ
ಕೋಣೆಯ
ಕೋಣೆ-ಯನ್ನು
ಕೋಣೆ-ಯನ್ನೇ
ಕೋಣೆ-ಯನ್ನೊ
ಕೋಣೆ-ಯಲ್ಲಿ
ಕೋಣೆ-ಯಲ್ಲಿ-ರು-ವುದು
ಕೋಣೆ-ಯಲ್ಲೇ
ಕೋಣೆ-ಯಿಂದ
ಕೋಣೆ-ಯೊಳ
ಕೋಣೆ-ಯೊಳಕ್ಕೆ
ಕೋನ-ಗಳು
ಕೋಪ
ಕೋಪಕ್ಕೆ
ಕೋಪ-ಗೊಂಡಂತೆ
ಕೋಪ-ಗೊಂಡಾಗ
ಕೋಪ-ಗೊಂಡಿದ್ದೇನೆ
ಕೋಪ-ಗೊಳ್ಳ-ಲಾರರೊ
ಕೋಪ-ಗೊಳ್ಳುತ್ತಿದ್ದೇನೆ
ಕೋಪ-ಗೊಳ್ಳು-ವುದು
ಕೋಪದ
ಕೋಪ-ದೊಂದಿಗೆ
ಕೋಪದ್ದಾಗಿದ್ದರೆ
ಕೋಪ-ವನ್ನು
ಕೋಪ-ವಾಗಿದೆ
ಕೋಪವು
ಕೋಪವೆ
ಕೋಪಿಸಿ-ಕೊಳ್ಳು-ವುದೇ
ಕೋಮಿಗೆ
ಕೋಮಿನ
ಕೋಮು
ಕೋರುತ್ತೇನೆ
ಕೋರು-ವುದೋ
ಕೋರೈಸುತ್ತಿ-ರುವ
ಕೋರ್ಟು
ಕೋಲನ್ನು
ಕೋಲಿಗೆ
ಕೋಲಿ-ನಿಂದ
ಕೋಲೂರಿ
ಕೋಲೂರಿ-ಕೊಂಡು
ಕೋಳಿ
ಕೋಳಿ-ಗಳ
ಕೋಳಿ-ಮರಿ
ಕೋಳಿ-ಮರಿ-ಗಳು
ಕೌಶಲ
ಕೌಶಲಂ
ಕೌಶಲ್ಯ-ವನ್ನು
ಕ್ಕಾಗಿ
ಕ್ಕಿಂತ
ಕ್ಕಿಂತಲೂ
ಕ್ಕೊಂದು
ಕ್ಕೋಸುಗ
ಕ್ಯಾಥೊಲಿಕ್
ಕ್ಯಾಮರಾ
ಕ್ಯಾಮೆರಾ
ಕ್ಯಾಲಿಫೋರ್ನಿಯಾ
ಕ್ರಂದನಕ್ಕಾ-ಗಲಿ
ಕ್ರಮ
ಕ್ರಮಃ
ಕ್ರಮಕ್ಕೆ
ಕ್ರಮ-ಗಳಿಗೆ
ಕ್ರಮ-ದಂತೆಯೇ
ಕ್ರಮ-ದಲ್ಲಿ
ಕ್ರಮ-ಬದ್ಧ-ವಾದ
ಕ್ರಮ-ವನ್ನನು-ಸರಿ-ಸು-ವುದು
ಕ್ರಮ-ವಾಗಿ
ಕ್ರಮ-ವಿ-ಪರ್ಯಯ
ಕ್ರಮ-ವಿಲ್ಲ
ಕ್ರಮ-ವಿಲ್ಲದೆ
ಕ್ರಮವು
ಕ್ರಮವೂ
ಕ್ರಮ-ವೆನ್ನು-ವುದು
ಕ್ರಮಾನ್ಯತ್ವಂ
ಕ್ರಮಿ-ಸುತ್ತದೆ
ಕ್ರಮೇಣ
ಕ್ರಿಮಿ
ಕ್ರಿಮಿ-ಯಿಂದ
ಕ್ರಿಯಾ
ಕ್ರಿಯಾತ್ಮಕ-ವಾದ
ಕ್ರಿಯಾ-ಪದ-ಗಳು
ಕ್ರಿಯಾ-ಫಲಾಶ್ರಯತ್ವಮ್
ಕ್ರಿಯಾ-ಯೋಗಃ
ಕ್ರಿಯಾ-ಯೋಗದ
ಕ್ರಿಯಾ-ಯೋಗ-ವನ್ನು
ಕ್ರಿಯಾ-ಯೋಗ-ವೆಂದು
ಕ್ರಿಯಾ-ರೂಪ
ಕ್ರಿಯಾ-ವಿಧಿ
ಕ್ರಿಯಾ-ವಿಧಿ-ಗ-ಳನ್ನು
ಕ್ರಿಯಾ-ವಿಧಿ-ಗಳು
ಕ್ರಿಯಾ-ಶಕ್ತಿ-ಯನ್ನು
ಕ್ರಿಯಾ-ಶೀಲ-ವಾಗುತ್ತವೆ
ಕ್ರಿಯಾ-ಶೀಲ-ವಾ-ಗು-ವುದು
ಕ್ರಿಯೆ
ಕ್ರಿಯೆ-ಗಳ
ಕ್ರಿಯೆ-ಗಳಂತೆ
ಕ್ರಿಯೆ-ಗ-ಳನ್ನು
ಕ್ರಿಯೆ-ಗ-ಳನ್ನೂ
ಕ್ರಿಯೆ-ಗಳಿಗೂ
ಕ್ರಿಯೆ-ಗಳಿಗೆ
ಕ್ರಿಯೆ-ಗಳಿವೆ
ಕ್ರಿಯೆ-ಗಳು
ಕ್ರಿಯೆ-ಗಳು-ಇ-ರುವ
ಕ್ರಿಯೆ-ಗಳೂ
ಕ್ರಿಯೆ-ಗಳೆಲ್ಲ
ಕ್ರಿಯೆ-ಗಿಂತಲೂ
ಕ್ರಿಯೆಗೂ
ಕ್ರಿಯೆಗೆ
ಕ್ರಿಯೆ-ಗೆಲ್ಲ
ಕ್ರಿಯೆಯ
ಕ್ರಿಯೆ-ಯಂತೆಯೇ
ಕ್ರಿಯೆ-ಯನ್ನು
ಕ್ರಿಯೆ-ಯಲ್ಲಿ
ಕ್ರಿಯೆ-ಯಲ್ಲಿಯೂ
ಕ್ರಿಯೆ-ಯಿಂದ
ಕ್ರಿಯೆಯು
ಕ್ರಿಯೆಯೂ
ಕ್ರಿಯೆಯೇ
ಕ್ರಿಯೆ-ಯೊಂದಿಗೆ
ಕ್ರಿಯೋ
ಕ್ರಿಯೋತ್ತೇ-ಜಕ
ಕ್ರಿಯೋತ್ತೇ-ಜಕ-ಶಕ್ತಿ
ಕ್ರಿಯೋತ್ತೇ-ಜನ
ಕ್ರಿಶ
ಕ್ರಿಸ್ತ
ಕ್ರಿಸ್ತನ
ಕ್ರಿಸ್ತ-ನಂತೆ
ಕ್ರಿಸ್ತ-ನನ್ನು
ಕ್ರಿಸ್ತ-ನಾ-ದನು
ಕ್ರಿಸ್ತನು
ಕ್ರಿಸ್ತ-ನೆನ್ನು-ವರೊ
ಕ್ರಿಸ್ತ-ರಾಗು-ವಿರಿ
ಕ್ರೀಡಾ-ಭೂಮಿ-ಯಾ-ಗು-ವುದು-ಅಪಾಯ
ಕ್ರೂರ-ನಾಗಿ-ರ-ಲಿಲ್ಲ
ಕ್ರೂರ-ನಾದ
ಕ್ರೂರ-ನಿಗೂ
ಕ್ರೂರನು
ಕ್ರೂರನೊ
ಕ್ರೂರ-ಪರಾಕ್ರಮ
ಕ್ರೂರ-ವಾಗಿ
ಕ್ರೂರ-ವಾ-ಗಿ-ರುವುದು
ಕ್ರೂರಿ
ಕ್ರೂರಿ-ಗಳನ್ನಾಗಿ
ಕ್ರೈಸ್ಟ್
ಕ್ರೈಸ್ತ
ಕ್ರೈಸ್ತ-ಧರ್ಮ
ಕ್ರೈಸ್ತ-ಧರ್ಮಕ್ಕೂ
ಕ್ರೈಸ್ತ-ಧರ್ಮದ
ಕ್ರೈಸ್ತ-ಧರ್ಮ-ದಲ್ಲಿ
ಕ್ರೈಸ್ತ-ಧರ್ಮ-ದಲ್ಲಿ-ರುವ
ಕ್ರೈಸ್ತ-ಧರ್ಮ-ವನ್ನು
ಕ್ರೈಸ್ತ-ನಿಲ್ಲ
ಕ್ರೈಸ್ತನು
ಕ್ರೈಸ್ತ-ಬೌದ್ಧ-ವೇದಾಂತಿ-ಗಳೆಲ್ಲಾ
ಕ್ರೈಸ್ತ-ಭಕ್ತ
ಕ್ರೈಸ್ತ-ಮ-ತೀಯನು
ಕ್ರೈಸ್ತರ
ಕ್ರೈಸ್ತ-ರಲ್ಲಿ
ಕ್ರೈಸ್ತ-ರಾ-ದರೆ
ಕ್ರೈಸ್ತ-ರಿಗೆ
ಕ್ರೈಸ್ತರು
ಕ್ರೈಸ್ತರೂ
ಕ್ರೈಸ್ತರೆ
ಕ್ರೈಸ್ತ-ರೇನೊ
ಕ್ರೈಸ್ತರೊ
ಕ್ರೈಸ್ತ-ವಿಜ್ಞಾನ-ವೆಂದು
ಕ್ರೈಸ್ತ-ವಿಜ್ಞಾ-ನಿ-ಗಳು
ಕ್ರೈಸ್ತೇ-ತರ
ಕ್ರೋಧ
ಕ್ರೌರ್ಯ
ಕ್ರೌರ್ಯ-ದಂತೆ
ಕ್ರೌರ್ಯ-ವೇ-ತಕ್ಕೆ
ಕ್ಲಿಷ್ಟಾ
ಕ್ಲೇಶ
ಕ್ಲೇಶ-ಕರ್ಮ-ನಿ-ವೃತ್ತಿಃ
ಕ್ಲೇಶ-ಕರ್ಮ-ವಿಪಾ-ಕಾಶಯ್ಯೆರ-ಪರಾಮೃಷ್ಟಃ
ಕ್ಲೇಶ-ಕಾರ-ಕ-ವಾದ
ಕ್ಲೇಶ-ಗಳು
ಕ್ಲೇಶ-ತನೂ-ಕರ-ಣಾರ್ಥಶ್ಚ
ಕ್ಲೇಶ-ದುಕ್ತಮ್
ಕ್ಲೇಶ-ಮೂಲಃ
ಕ್ಲೇಶ-ವನ್ನೇ
ಕ್ಲೇಶಾಃ
ಕ್ಷಣ
ಕ್ಷಣ-ಕಾಲ
ಕ್ಷಣ-ಕಾಲದ
ಕ್ಷಣ-ಕಾಲ-ದ-ಮೇಲೆ
ಕ್ಷಣ-ಕಾಲ-ದಲ್ಲಿ
ಕ್ಷಣ-ಕಾಲ-ವಾ-ದರೂ
ಕ್ಷಣಕ್ಕೇ
ಕ್ಷಣ-ಗಳ
ಕ್ಷಣ-ಗಳಲ್ಲಿ
ಕ್ಷಣ-ಗಳಿಂದ
ಕ್ಷಣ-ಗಳು
ಕ್ಷಣ-ತತ್ಕ್ರಮಯೋಃ
ಕ್ಷಣ-ದಂತೆ
ಕ್ಷಣ-ದಲ್ಲಿ
ಕ್ಷಣ-ದಲ್ಲಿದ್ದರೆ
ಕ್ಷಣ-ದಲ್ಲಿಯೂ
ಕ್ಷಣ-ದಿಂದ
ಕ್ಷಣ-ದಿಂದಲೇ
ಕ್ಷಣಪ್ರತಿ-ಯೋಗೀ
ಕ್ಷಣ-ಭಂಗು-ರವೇ
ಕ್ಷಣ-ವನ್ನು
ಕ್ಷಣ-ವಾ-ದರೂ
ಕ್ಷಣವೂ
ಕ್ಷಣವೆ
ಕ್ಷಣವೇ
ಕ್ಷಣ-ಸಂಬಂಧ-ದಿಂದ
ಕ್ಷಣಿಕ
ಕ್ಷಣಿಕ-ದರ್ಶನ-ವಾ-ಗು-ವುದು
ಕ್ಷಣಿಕ-ವಾಗಿ-ರ-ಬೇಕು
ಕ್ಷಣಿಕ-ವಾದ
ಕ್ಷಮಾ-ಜೀವಿ-ಗಳೋ
ಕ್ಷಮಾಪ-ಣೆಗೆ
ಕ್ಷಮಾಪಣೆ-ಯನ್ನು
ಕ್ಷಮಾರ್ಹ-ವಲ್ಲ
ಕ್ಷಮಾರ್ಹ-ವಾ-ದುದು
ಕ್ಷಮಿ-ಸದೆ
ಕ್ಷಮಿಸ-ಬಲ್ಲ
ಕ್ಷಮಿ-ಸು-ವು-ದಕ್ಕೆ
ಕ್ಷಮಿ-ಸು-ವುದು
ಕ್ಷಮೆ
ಕ್ಷಮೆಗೆ
ಕ್ಷಯ
ಕ್ಷಯ-ಗಳಿಗೆ
ಕ್ಷಯ-ಗಳಿವೆ
ಕ್ಷಯವು
ಕ್ಷಯಿ-ಸ-ಬೇಕು
ಕ್ಷಯಿಸಿ
ಕ್ಷಯಿಸು-ವಂತೆ
ಕ್ಷಯಿ-ಸು-ವು-ದಿಲ್ಲ
ಕ್ಷಯಿ-ಸು-ವುದು
ಕ್ಷಾಮಗಾಲ
ಕ್ಷಿಪ್ತ
ಕ್ಷಿಪ್ತಾ-ವಸ್ಥೆಯು
ಕ್ಷೀಣ-ದೆಸೆ
ಕ್ಷೀಣ-ದೆಸೆಗೆ
ಕ್ಷೀಣ-ವಾಗಿ
ಕ್ಷೀಣ-ವಾಣಿ
ಕ್ಷೀಣ-ವಾದ
ಕ್ಷೀಣ-ವೃತ್ತೇರಭಿಜಾ-ತಸ್ಯೇವ
ಕ್ಷೀಣಿ-ಸಿದ
ಕ್ಷೀಣಿ-ಸುತ್ತದೆ
ಕ್ಷೀಣಿ-ಸು-ವುದು
ಕ್ಷೀಯತೇ
ಕ್ಷುತ್ಪಿಪಾಸಾನಿವೃತ್ಥಿಃ
ಕ್ಷುತ್ರ-ತಮ
ಕ್ಷುದ್ರ
ಕ್ಷುದ್ರ-ತಮ
ಕ್ಷುದ್ರ-ತಮ-ಕೀಟ-ದಿಂದ
ಕ್ಷುದ್ರ-ತಮ-ವಾದ
ಕ್ಷುದ್ರ-ದೇಹಿ-ಗಳು
ಕ್ಷುದ್ರಪ್ರಾಣಿ-ಗಳೂ
ಕ್ಷುದ್ರ-ಭಾ-ವನೆ
ಕ್ಷುದ್ರ-ರೂಪು-ಗಳು
ಕ್ಷುದ್ರ-ವಾದ
ಕ್ಷುದ್ರಾತ್ಮನ
ಕ್ಷುದ್ರಾತ್ಮನು
ಕ್ಷೇತ್ರ
ಕ್ಷೇತ್ರಕ್ಕೆ
ಕ್ಷೇತ್ರ-ಗಳ
ಕ್ಷೇತ್ರ-ಗಳಲ್ಲಿ
ಕ್ಷೇತ್ರ-ಗಳಿಗೂ
ಕ್ಷೇತ್ರದ
ಕ್ಷೇತ್ರ-ದಲ್ಲಿ
ಕ್ಷೇತ್ರ-ದಲ್ಲಿರು
ಕ್ಷೇತ್ರ-ದಲ್ಲಿ-ರು-ವ-ವರು
ಕ್ಷೇತ್ರ-ದಲ್ಲೇ
ಕ್ಷೇತ್ರ-ದಿಂದ
ಕ್ಷೇತ್ರ-ಮುತ್ತರೇಷಾಂ
ಕ್ಷೇತ್ರ-ವನ್ನು
ಕ್ಷೇತ್ರ-ವಿ-ರು-ವುದು
ಕ್ಷೇತ್ರವು
ಕ್ಷೇತ್ರವೇ
ಕ್ಷೇತ್ರಿಕವತ್
ಕ್ಷೋಭೆಗೆ
ಖಂಡದ
ಖಂಡ-ದಲ್ಲಿ
ಖಂಡ-ವನ್ನು
ಖಂಡಿತ
ಖಂಡಿತ-ವಾಗಿ
ಖಗೋಳ
ಖಗೋಳ-ಶಾಸ್ತ್ರ
ಖಗೋಳ-ಶಾಸ್ತ್ರಜ್ಞ
ಖಗೋಳ-ಶಾಸ್ತ್ರಜ್ಞ-ರಾಗ-ಬೇಕಾ-ದರೆ
ಖಗೋಳ-ಶಾಸ್ತ್ರ-ದಲ್ಲಿ
ಖಗೋಳ-ಶಾಸ್ತ್ರ-ವನ್ನು
ಖಚಿತ
ಖಚಿತ-ವಾಗಿ
ಖಚಿತ-ವಾದ
ಖಡ್ಗ-ಗಳಲ್ಲದೆ
ಖಡ್ಗ-ವನ್ನು
ಖನಿಜ
ಖನಿಜ-ವೆಂಬ
ಖರ್ಚು
ಖರ್ಚು-ದೇವಸ್ಥಾನ-ಗಳಲ್ಲಿ
ಖಾಯಿಲೆ-ಗಳಿವೆ
ಖಾಯಿಲೆ-ಯಾ-ದಾಗ
ಖಾಯಿಲೆಯೂ
ಖಾಯಿಲೆಯೇ
ಖಾಲಿ-ಯಾಗದ
ಖಾಲಿ-ಯಾಗಿಯೂ
ಖಾಲಿ-ಯಾ-ಗು-ವುದು
ಖಿನ್ನನಾ-ದನು
ಖುರಾ
ಖುರಾ-ನನ್ನು
ಖುರಾ-ನಿ-ಗಿಂತ
ಖುರಾ-ನಿ-ನಲ್ಲಿ
ಖುರಾ-ನಿ-ನಲ್ಲಿ-ರುವ
ಖುರಾ-ನಿ-ನಲ್ಲೆ
ಖುರಾನ್
ಖೂನಿ
ಖೊರಾ-ನನ್ನು
ಖೊರಾನಿನಲ್ಲಿಯೇ
ಖ್ಯಾತಿಯ
ಗಂಗಾ
ಗಂಗಾ-ನದಿಯ
ಗಂಟನ್ನು
ಗಂಟ-ಲನ್ನು
ಗಂಟ-ಲಿನ
ಗಂಟೆ
ಗಂಟೆ-ಗಳ
ಗಂಟೆ-ಗಳು
ಗಂಟೆಗೆ
ಗಂಟೆ-ಯಲ್ಲಿ
ಗಂಡ
ಗಂಡನ
ಗಂಡ-ನನ್ನು
ಗಂಡನು
ಗಂಡ-ಸಾ-ಗಲೀ
ಗಂಡ-ಸಿಗೆ
ಗಂಡಸು
ಗಂಡ-ಹೆಂಡಿ-ರನ್ನು
ಗಂಡ-ಹೆಂಡಿ-ರಲ್ಲಿ
ಗಂಧ
ಗಂಧಕ
ಗಂಧದ
ಗಂಧರ್ವ
ಗಂಧರ್ವ-ಲೋಕ-ವೆಂಬ
ಗಂಧಾಸ್ವಾದ-ನಕ್ಕೆ
ಗಂಭೀರ-ವಾಗಿ
ಗಗನ-ಇ-ವು-ಗಳೆಲ್ಲ
ಗಟ್ಟಿ-ಯಾಗಿ
ಗಟ್ಟಿ-ಯಾದ
ಗಡಿ-ಬಿಡಿಯ
ಗಡಿ-ಯಾರ
ಗಡಿ-ಯಾರ-ವನ್ನು
ಗಡ್ಡ
ಗಣ-ನೆಗೆ
ಗಣನೆಗೇ
ಗಣಿ
ಗಣಿತ
ಗಣಿ-ತದ
ಗಣಿ-ತ-ಶಾಸ್ತ್ರದ
ಗಣಿ-ಯಿಂದ
ಗತಿ
ಗತಿ-ಗಳಿವೆ
ಗತಿಗೆ
ಗತಿಯ
ಗತಿ-ಯನ್ನು
ಗತಿ-ಯಿಲ್ಲ
ಗತಿಯೂ
ಗತಿ-ಸಿದ
ಗತಿ-ಸಿದ-ವರ
ಗದ
ಗದೆ
ಗದ್ದಲ
ಗದ್ದಲ-ವಿ-ರುವ
ಗದ್ದಲ-ವಿಲ್ಲದೆ
ಗಬೇಕಾ-ಗಿತ್ತು
ಗಮನ
ಗಮ-ನಕ್ಕೆ
ಗಮನಕ್ಕೇ
ಗಮನ-ದಲ್ಲಿ-ಡ-ಬೇಕು
ಗಮನ-ವೆಲ್ಲ
ಗಮನಿಸ
ಗಮನಿಸ-ದಿದ್ದರೂ
ಗಮನಿ-ಸದೆ
ಗಮನಿಸ-ಬಹುದು
ಗಮನಿಸ-ಬೇಕಾ-ಗಿಲ್ಲ
ಗಮನಿ-ಸ-ಬೇಕು
ಗಮನಿಸ-ಬೇಡಿ
ಗಮನಿಸಿ
ಗಮನಿಸಿ-ದರು
ಗಮನಿಸಿ-ರು-ವೆನು
ಗಮನಿ-ಸುತ್ತಿದ್ದೆವು
ಗಮನಿ-ಸು-ವು-ದಿಲ್ಲ
ಗಮನಿ-ಸೋಣ
ಗರ
ಗರ-ಡಿಯ
ಗರು
ಗರ್ಜಿ-ಸಿತು
ಗರ್ಭಿಣಿ
ಗಲಿ
ಗಲೂ
ಗಲೇ
ಗಳ
ಗಳಂತಲ್ಲದೆ
ಗಳಂತೆ
ಗಳನ್ನು
ಗಳನ್ನುಳ್ಳದ್ದು
ಗಳನ್ನೂ
ಗಳನ್ನೆಲ್ಲ
ಗಳನ್ನೆಲ್ಲಾ
ಗಳನ್ನೇ
ಗಳಲ್ಲ
ಗಳಲ್ಲಿ
ಗಳಲ್ಲಿಯೂ
ಗಳಲ್ಲಿಯೆ
ಗಳಲ್ಲಿ-ರುವ
ಗಳಲ್ಲೆ
ಗಳ-ವರೆಗೂ
ಗಳ-ವರೆಗೆ
ಗಳಾ-ಗ-ಬೇಕೆಂದು
ಗಳಾಗಿ
ಗಳಾ-ಗುತ್ತವೆಪ್ರಪಂಚ-ದಲ್ಲಿ-ರುವ
ಗಳಾ-ಗು-ವರು
ಗಳಾ-ಗು-ವ-ವರೆಗೆ
ಗಳಾ-ಗು-ವುದು
ಗಳಾ-ಗು-ವೆವು
ಗಳಾದ
ಗಳಾದರೂ
ಗಳಾದಾಗ
ಗಳಿಂದ
ಗಳಿಂದಲೂ
ಗಳಿ-ಗಾಗಿ
ಗಳಿಗೂ
ಗಳಿಗೆ
ಗಳಿ-ಗೆ-ಯಲ್ಲೂ
ಗಳಿ-ಗೆಯೂ
ಗಳಿದ್ದವು
ಗಳಿ-ರ-ಬಹುದು
ಗಳಿ-ರುತ್ತವೆ
ಗಳಿವು
ಗಳಿವೆ
ಗಳಿ-ವೆ-ಒಂದು
ಗಳಿ-ವೆಯೊ
ಗಳಿಸ
ಗಳಿ-ಸದ
ಗಳಿ-ಸ-ಬಹುದು
ಗಳಿ-ಸ-ಬೇಕಾ-ಗಿಲ್ಲ
ಗಳಿ-ಸ-ಬೇಕೆನ್ನು-ವುದು
ಗಳಿ-ಸುತ್ತಾರೋ
ಗಳಿ-ಸುವರೋ
ಗಳಿ-ಸು-ವುದಕ್ಕೋಸ್ಕರ
ಗಳಿ-ಸು-ವುದೂ
ಗಳು
ಗಳೂ
ಗಳೆಂದರೆ
ಗಳೆಂದು
ಗಳೆಂಬ
ಗಳೆ-ಯ-ಲಾಗು-ವು-ದಿಲ್ಲ
ಗಳೆ-ಯು-ವುದು
ಗಳೆ-ರಡ-ರಿಂದಲೂ
ಗಳೆ-ರಡೂ
ಗಳೆಲ್ಲ
ಗಳೆಲ್ಲ-ವನ್ನೂ
ಗಳೆಲ್ಲಾ
ಗಳೊಂದಿಗೆ
ಗಳೋ
ಗವಿ-ಯೊ-ಳಗೆ
ಗಹನ
ಗಹನ-ಭಾವ-ನೆಗೆ
ಗಹನ-ವಾದ
ಗಹ-ನವೂ
ಗಾಗಲಿ
ಗಾಗಿ
ಗಾಗು-ವು-ದಿಲ್ಲ
ಗಾಜಾಗಿದ್ದಾನೆ
ಗಾಜಿನ
ಗಾಜಿ-ನಂತೆ
ಗಾಜಿನ-ಲೋಟ-ವನ್ನು
ಗಾಜಿ-ನಲ್ಲಿ
ಗಾಜಿ-ನಿಂದ
ಗಾಜು
ಗಾಡಿ
ಗಾಡಿ-ಗಳು
ಗಾಡಿ-ಯನ್ನು
ಗಾಡಿಯೂ
ಗಾಡ್
ಗಾಢ-ವಾಗಿ
ಗಾಢ-ವಾಗುತ್ತ
ಗಾಢ-ವಾದ
ಗಾತ್ರ
ಗಾದರೂ
ಗಾನಲ-ಹರಿ
ಗಾನ-ವಾಗುತ್ತಿದೆ
ಗಾಬರಿ-ಯಾಗಿ
ಗಾಮಿ-ಯಾ-ಗು-ವುದು
ಗಾಯ
ಗಾಯ-ಗಳು
ಗಾಯ-ಗೊಂಡರೂ
ಗಾಯತ್ರಿ-ಯನ್ನು
ಗಾಯದ
ಗಾಯ-ಮಾಡಿ-ಕೊಳ್ಳ
ಗಾಯ-ವನ್ನು
ಗಾಯವು
ಗಾರುಡಿ
ಗಾಲಿ-ಗಳು
ಗಾಳಿ
ಗಾಳಿ-ಬಂದಾಗ
ಗಾಳಿ-ಮೋಡ-ಗಳ
ಗಾಳಿಯ
ಗಾಳಿ-ಯನ್ನು
ಗಾಳಿ-ಯಷ್ಟು
ಗಾಳಿ-ಯಾ-ಗು-ವುದು
ಗಾಳಿಯು
ಗಿಂತ
ಗಿಂತಲೂ
ಗಿಡ
ಗಿಡ-ಮರ-ಗ-ಳನ್ನು
ಗಿಡ-ವನ್ನು
ಗಿತ್ತು
ಗಿದೆ
ಗಿದ್ದವು
ಗಿನ
ಗಿನ್ನು
ಗಿರಿ
ಗಿರಿ-ಗಳು
ಗಿರಿ-ಗಿಟ್ಟೆಯ
ಗಿರಿ-ಗುಹೆ-ಗಳಲ್ಲಿ
ಗಿರಿ-ಶಿಖರ-ಗಳು
ಗಿರುವ
ಗಿರು-ವುದೇ-ನೆಂದರೆ
ಗಿಲಾಯಿ-ಯನ್ನು
ಗಿಲ್ಲ
ಗಿವೆ
ಗೀತೆ
ಗೀತೆ-ಗಳು
ಗೀತೆ-ಯಲ್ಲಿ
ಗೀತೋಪ-ದೇಶದ
ಗೀರಿ-ದರೆ
ಗುಂಡಿ-ಯಂತೆ
ಗುಂಡು-ಸೂಜಿ
ಗುಂಪಿಗೆ
ಗುಂಪಿನ-ವರು
ಗುಂಪಿನ-ವರೆ
ಗುಂಪು
ಗುಂಪು-ಗಳಾಗಿ
ಗುಂಪೂ
ಗುಟ್ಟನ್ನು
ಗುಟ್ಟು
ಗುಟ್ಟೆಲ್ಲ
ಗುಡಿ
ಗುಡಿ-ಗಳಲ್ಲಿ
ಗುಡಿ-ಯನ್ನು
ಗುಡಿ-ಯನ್ನೋ
ಗುಡಿ-ಯಲ್ಲೂ
ಗುಡಿ-ಯಾ-ಗಿದೆ
ಗುಡಿ-ಯೇನೊ
ಗುಡಿ-ಯೊಂದು
ಗುಡಿಸು
ಗುಡುಗು
ಗುಡ್ಡದ
ಗುಣ
ಗುಣಕ್ಕಾ-ಗಲೀ
ಗುಣಕ್ಕೆ
ಗುಣ-ಗಲು
ಗುಣ-ಗಳ
ಗುಣ-ಗ-ಳನ್ನು
ಗುಣ-ಗ-ಳನ್ನೂ
ಗುಣ-ಗಳಲ್ಲ
ಗುಣ-ಗಳಲ್ಲದ
ಗುಣ-ಗಳಾ-ಸೆಯ
ಗುಣ-ಗಳಿಂದ
ಗುಣ-ಗಳಿಂದಾ-ದುದು
ಗುಣ-ಗಳಿ-ಗಿಂತ
ಗುಣ-ಗಳಿಗೆ
ಗುಣ-ಗಳಿದ್ದರೂ
ಗುಣ-ಗಳಿ-ರು-ವುವೊ
ಗುಣ-ಗಳಿವೆ
ಗುಣ-ಗಳು
ಗುಣ-ಗಳೂ
ಗುಣ-ಗಳೆಲ್ಲ
ಗುಣ-ಗಳೆಲ್ಲಾ
ಗುಣ-ಗಳೊಂದೇ
ಗುಣ-ದಿಂದ
ಗುಣ-ಪಡಿ-ಸ-ಬಹುದು
ಗುಣ-ಪಡಿ-ಸುತ್ತದೆ
ಗುಣ-ಪಡಿ-ಸುವು-ದಿದ್ದರೆ
ಗುಣ-ಪಡಿ-ಸು-ವುದು
ಗುಣ-ಪರ್ವಾಣಿ
ಗುಣ-ಮಯೀ
ಗುಣ-ಮಾಡುವ
ಗುಣ-ಮುಖ-ನಾಗು-ವನು
ಗುಣ-ಮುಖ-ರನ್ನಾಗಿ
ಗುಣ-ಮುಖ-ರಾಗುತ್ತಿದ್ದರು
ಗುಣ-ವಂತ-ನಾದ
ಗುಣ-ವನ್ನು
ಗುಣ-ವಲ್ಲ
ಗುಣ-ವಾಗು
ಗುಣ-ವಾಗು-ವನು
ಗುಣ-ವಾಗುವು
ಗುಣ-ವಾಗು-ವುದು
ಗುಣ-ವಾಚ-ಕದ
ಗುಣ-ವಿಕಾರ-ಗಳಿಗೆ
ಗುಣ-ವಿದೆ
ಗುಣ-ವೆಂದು
ಗುಣ-ವೆಲ್ಲ
ಗುಣ-ವೇನು
ಗುಣ-ವೇನೆಂದರೆ
ಗುಣ-ಶಕ್ತಿ-ಗಳೆಲ್ಲ
ಗುಣಾ-ಕಾರ
ಗುಣಾಢ್ಯನ
ಗುಣಾ-ತೀತ-ವಾದ
ಗುಣಾತ್ಮಕ-ವಾಗಿರು-ವುದ-ರಿಂದ
ಗುಣಾತ್ಮಾನಃ
ಗುಣಾನಾಂ
ಗುಣಾ-ಶ್ರಯ-ನಾದ
ಗುಣಿ
ಗುಣಿ-ಗಳ
ಗುಣಿ-ಗಳು
ಗುಣಿಗೆ
ಗುಣಿಯ
ಗುಣಿ-ಯನ್ನು
ಗುಣಿಯು
ಗುಣಿಯೆ
ಗುಣಿಯೋ
ಗುಣಿಸಿ-ದರೂ
ಗುತ್ತದೆ
ಗುತ್ತಾರೆ
ಗುಪ್ತ
ಗುಪ್ತ-ಗಾಮಿ-ಯಾಗಿ-ರು-ವುದು
ಗುಪ್ತ-ವಾಗಿ
ಗುಪ್ತ-ವಾದ
ಗುರಿ
ಗುರಿ-ಇ-ವಕ್ಕೆ
ಗುರಿ-ಗಾಗಿ
ಗುರಿಗೆ
ಗುರಿ-ಮಾಡುತ್ತದೆ
ಗುರಿಯ
ಗುರಿ-ಯನ್ನು
ಗುರಿ-ಯಲ್ಲ
ಗುರಿ-ಯಾಗ-ಬೇಕು
ಗುರಿ-ಯಾಗ-ಲಾರದು
ಗುರಿ-ಯಾ-ಗಿದೆ
ಗುರಿ-ಯಾಗಿದ್ದರೆ
ಗುರಿ-ಯಾದ
ಗುರಿ-ಯಾ-ದರೆ
ಗುರಿಯು
ಗುರಿಯೂ
ಗುರಿಯೆ
ಗುರಿ-ಯೆ-ಡೆ-ಗಾಗಿ
ಗುರಿ-ಯೆ-ಡೆಗೆ
ಗುರಿ-ಯೆ-ಡೆಗೇ
ಗುರಿ-ಯೆಲ್ಲ
ಗುರಿಯೇ
ಗುರಿ-ಯೇನು
ಗುರು
ಗುರುಃ
ಗುರು-ಗಳ
ಗುರು-ಗಳಲ್ಲಿದೆ
ಗುರು-ಗಳಾ-ಗಿದ್ದರು
ಗುರು-ಗಳಿಗೂ
ಗುರು-ಗಳಿಗೆ
ಗುರು-ಗಳಿಲ್ಲದೆ
ಗುರು-ಗಳು
ಗುರು-ಗಳೆ
ಗುರು-ಗಳೆನ್ನಿಸಿ-ಕೊಳ್ಳು-ವ-ವರು
ಗುರು-ಗಳೇ
ಗುರು-ತಲ್ಲ
ಗುರು-ತಾ-ಗಿದೆ
ಗುರು-ತಿನ
ಗುರು-ತಿಸ-ಕೂಡದು
ಗುರು-ತಿಸ-ಬಹುದು
ಗುರು-ತಿ-ಸ-ಬೇಕು-ನನ್ನನ್ನು
ಗುರು-ತಿಸ-ಲಾರರು
ಗುರು-ತಿಸಿ-ದು-ದ-ರಿಂದಾ-ಗಿಯೇ
ಗುರು-ತಿಸು-ವ-ವನು
ಗುರುತು
ಗುರುತೆ
ಗುರುತೇ
ಗುರುತ್ವಾ-ಕರ್ಷಣ
ಗುರುತ್ವಾ-ಕರ್ಷಣೆ
ಗುರುತ್ವಾ-ಕರ್ಷಣೆಗೆ
ಗುರುತ್ವಾ-ಕರ್ಷಣೆ-ಯನ್ನು
ಗುರು-ಪತ್ನಿ
ಗುರು-ವನ್ನು
ಗುರು-ವರ್ಯರೂ
ಗುರು-ವಾತ
ಗುರು-ವಾದ
ಗುರು-ವಿನ
ಗುರು-ವಿ-ನಂತೆ
ಗುರು-ವಿ-ನಿಂದ
ಗುರು-ವಿ-ನೊಂದಿಗೆ
ಗುರು-ವಿಲ್ಲದೆ
ಗುರುವು
ಗುಲಾಬಿ
ಗುಲಾಮ-ನಂತೆ
ಗುಲಾಮ-ನಾಗುತ್ತನೆ
ಗುಲಾಮ-ನಾಗು-ವು-ದಿಲ್ಲ
ಗುಲಾಮ-ರಂತೆ
ಗುಲಾಮ-ರನ್ನಾಗಿ
ಗುಲಾಮ-ರಲ್ಲ
ಗುಲಾಮ-ರಾಗುತ್ತಿ-ರು-ವರು
ಗುಲಾಮ-ರಾಗುವ
ಗುಲಾ-ಮರು
ಗುಲಾಮ-ರೆಂದು
ಗುಳ್ಳೆ
ಗುಳ್ಳೆ-ಗಳಂತೆ
ಗುಳ್ಳೆ-ಗಳು
ಗುಳ್ಳೆ-ಯಂತೆ
ಗುವು-ದಕ್ಕಾಗಿ
ಗುವುದು
ಗುವುವು
ಗುಹೆ
ಗುಹೆಗೆ
ಗುಹೆಯ
ಗುಹೆ-ಯಲ್ಲಿ
ಗುಹೆ-ಯಲ್ಲೋ
ಗುಹೆಯು
ಗೂಡನ್ನು
ಗೂಡ-ಬಹುದು
ಗೂಡಿ-ನಲ್ಲಿ
ಗೂಡಿಸ-ಲಾರದ
ಗೂಡಿಸಿ-ರುವ
ಗೂಡು-ಮಾಡಿ
ಗೂಢ
ಗೂಬೆ-ಗಳಂತಹ
ಗೃಹಸ್ಥರಿ-ಗೆಲ್ಲ
ಗೆಡಹು-ವುದು
ಗೆದ್ದ
ಗೆದ್ದ-ಮೇಲೆ
ಗೆದ್ದೆವೊ
ಗೆಯೆ
ಗೆಲುವು
ಗೆಲ್ಲ
ಗೆಲ್ಲದೆ
ಗೆಲ್ಲ-ಬಹುದು
ಗೆಲ್ಲ-ಬೇಕು
ಗೆಲ್ಲುವ
ಗೆಲ್ಲು-ವುದು
ಗೆಳೆ-ಯರ
ಗೇಕೆ
ಗೇನೂ
ಗೊಂಡ
ಗೊಂಡದ್ದು
ಗೊಂಡ-ವನು
ಗೊಂಡು
ಗೊಂದಲ
ಗೊಂದಲ-ಗಳ
ಗೊಂದಲ-ದಲ್ಲಿ
ಗೊಂದಲ-ವಾ-ಗಿ-ರುವುದು
ಗೊಂದಾ-ವರ್ತಿ
ಗೊಣಗಾಡ-ಬೇಡಿ
ಗೊತ್ತಾಗ
ಗೊತ್ತಾಗದೆ
ಗೊತ್ತಾಗ-ಬೇಕು
ಗೊತ್ತಾಗ-ಬೇಕೊ
ಗೊತ್ತಾಗ-ಲಿಲ್ಲ
ಗೊತ್ತಾಗಲು
ಗೊತ್ತಾಗಿದೆ
ಗೊತ್ತಾಗಿ-ರ-ಬಹುದು
ಗೊತ್ತಾಗಿಲ್ಲವೆ
ಗೊತ್ತಾಗು
ಗೊತ್ತಾಗುತ್ತದೆ
ಗೊತ್ತಾಗುತ್ತವೆ
ಗೊತ್ತಾಗುತ್ತಿದೆ
ಗೊತ್ತಾಗುವ
ಗೊತ್ತಾಗು-ವಂತೆಯೂ
ಗೊತ್ತಾಗುವು
ಗೊತ್ತಾಗು-ವು-ದಿಲ್ಲ
ಗೊತ್ತಾಗು-ವುದು
ಗೊತ್ತಾಗು-ವು-ದುಗ್ರಹಣ
ಗೊತ್ತಾಗು-ವುದೇ
ಗೊತ್ತಾದ
ಗೊತ್ತಾ-ದರೆ
ಗೊತ್ತಾಯಿತು
ಗೊತ್ತಿತ್ತು
ಗೊತ್ತಿದೆ
ಗೊತ್ತಿದೆ-ಆ-ದರೆ
ಗೊತ್ತಿದೆಯೆ
ಗೊತ್ತಿದೆಯೊ
ಗೊತ್ತಿದೆಯೋ
ಗೊತ್ತಿದ್ದರೂ
ಗೊತ್ತಿದ್ದರೆ
ಗೊತ್ತಿ-ರ-ಬಹುದು
ಗೊತ್ತಿ-ರ-ಬೇಕು
ಗೊತ್ತಿ-ರಲಿ
ಗೊತ್ತಿ-ರ-ಲಿಲ್ಲ
ಗೊತ್ತಿ-ರ-ಲಿಲ್ಲ-ವೆಂದೂ
ಗೊತ್ತಿ-ರುವ
ಗೊತ್ತಿ-ರು-ವಂತೆ
ಗೊತ್ತಿ-ರುವನು
ಗೊತ್ತಿ-ರುವ-ನೆಂದು
ಗೊತ್ತಿ-ರುವು-ದೆಲ್ಲ
ಗೊತ್ತಿಲ್ಲ
ಗೊತ್ತಿಲ್ಲದೆ
ಗೊತ್ತಿಲ್ಲವೆ
ಗೊತ್ತಿಲ್ಲ-ವೆಂದು
ಗೊತ್ತಿವೆ
ಗೊತ್ತಿವೆಯೆ
ಗೊತ್ತು
ಗೊತ್ತು-ಇವು
ಗೊತ್ತು-ಗುರಿ-ಯಿಲ್ಲದ
ಗೊತ್ತು-ಗುರಿ-ಯಿಲ್ಲದೆ
ಗೊತ್ತೆ
ಗೊತ್ತೇ
ಗೊಳಿ-ಸ-ಬೇಕಾ-ದರೆ
ಗೊಳಿ-ಸಲು
ಗೊಳಿಸಿ
ಗೊಳಿ-ಸಿ-ರುತ್ತಾನೆ
ಗೊಳಿ-ಸಿ-ರು-ವರು
ಗೊಳಿ-ಸುತ್ತದೆ
ಗೊಳಿ-ಸುತ್ತಿ-ರುತ್ತಾನೆ
ಗೊಳಿ-ಸುವ
ಗೊಳಿ-ಸುವನು
ಗೊಳಿ-ಸು-ವುದು
ಗೊಳಿ-ಸು-ವುವು
ಗೋಚರ
ಗೋಚ-ರಕ್ಕೆ
ಗೋಚರ-ನನ್ನಾಗಿ
ಗೋಚರವಾ
ಗೋಚರ-ವಾ-ಗದು
ಗೋಚರ-ವಾ-ಗ-ಬಹುದು
ಗೋಚರ-ವಾಗ-ಲಿಲ್ಲ
ಗೋಚರ-ವಾಗಿ
ಗೋಚರ-ವಾ-ಗಿ-ರುವ
ಗೋಚರ-ವಾ-ಗು-ವಷ್ಟು
ಗೋಚರ-ವಾ-ಗು-ವುದು
ಗೋಚರ-ವಿಲ್ಲ-ದು-ದರ
ಗೋಚರಿಸ
ಗೋಚರಿ-ಸದೇ
ಗೋಚರಿಸ-ಲಾರದ
ಗೋಚರಿಸ-ಲಿಲ್ಲ
ಗೋಚ-ರಿಸುತ್ತಿ-ರುವ
ಗೋಚ-ರಿಸುವ
ಗೋಚರಿ-ಸುವನು
ಗೋಚರಿ-ಸು-ವು-ದಿಲ್ಲ
ಗೋಚರಿ-ಸು-ವುದು
ಗೋಚರಿ-ಸು-ವುದೋ
ಗೋಚರಿ-ಸು-ವುವು
ಗೋಚರಿ-ಸು-ವುವು-ಕೇಳ
ಗೋಜಿಗೆ
ಗೋಡೆ
ಗೋಡೆ-ಗಳಿಗೆ
ಗೋಡೆ-ಗಳು
ಗೋಡೆಗೆ
ಗೋಡೆಯ
ಗೋಡೆ-ಯಲ್ಲಿ-ರುವನು
ಗೋಪು-ರದ
ಗೋಪುರ-ದಿಂದ
ಗೋಳಾಡ-ಬೇಕಾ-ಗಿಲ್ಲ
ಗೋಳಾ-ಡಲು
ಗೋಳಾಡಿ-ದರೆ
ಗೋಳಿಟ್ಟೆ
ಗೋಳಿಡಿ
ಗೋಳಿ-ಡು-ವುದು
ಗೋವಿನಾ-ಕಾರ
ಗೌಣ
ಗೌಣ-ವಾದು-ವು-ಗಳು
ಗೌತಮ
ಗೌರವ
ಗೌರವ-ದಿಂದ
ಗೌರವ-ವಿದೆ-ಆ-ದರೆ
ಗೌರ-ವಾರ್ಹ
ಗೌರವಾಶ್ಚರ್ಯ-ಗಳಿಂದ
ಗೌರವಿಸು
ಗೌರವಿ-ಸು-ವು-ದಿಲ್ಲ
ಗ್ಯಾಬ್ರಿಯಲ್
ಗ್ರಂಥ
ಗ್ರಂಥ-ಕರ್ತ
ಗ್ರಂಥ-ಕರ್ತನು
ಗ್ರಂಥ-ಕರ್ತರು
ಗ್ರಂಥ-ಗಳ
ಗ್ರಂಥ-ಗ-ಳನ್ನು
ಗ್ರಂಥ-ಗಳಲ್ಲಿ
ಗ್ರಂಥ-ಗಳಲ್ಲಿಯೂ
ಗ್ರಂಥ-ಗಳಿ-ಗಿಂತಲೂ
ಗ್ರಂಥ-ಗಳು
ಗ್ರಂಥದ
ಗ್ರಂಥ-ದಲ್ಲಿ
ಗ್ರಂಥ-ಪಾಠ-ಗ-ಳನ್ನು
ಗ್ರಂಥ-ರಾಶಿ-ಯಾ-ದರೂ
ಗ್ರಂಥ-ವನ್ನು
ಗ್ರಂಥವೂ
ಗ್ರಂಥ-ವೆನ್ನುತ್ತಾನೆ
ಗ್ರಂಥವೇ
ಗ್ರಂಥಿ-ಯಂತೆ
ಗ್ರಹ
ಗ್ರಹ-ಗಳ
ಗ್ರಹ-ಗ-ಳನ್ನು
ಗ್ರಹ-ಗಳಲ್ಲಿ
ಗ್ರಹ-ಗಳಿಂದ
ಗ್ರಹ-ಗಳು
ಗ್ರಹ-ಚಾರ
ಗ್ರಹಣ
ಗ್ರಹ-ಣಕ್ಕೆ
ಗ್ರಹ-ಣಕ್ರಿಯೆಯ
ಗ್ರಹ-ಣ-ಗಳಲ್ಲಿ
ಗ್ರಹ-ಣದ
ಗ್ರಹ-ಣ-ದಲ್ಲಿಯೂ
ಗ್ರಹ-ಣ-ದಿಂದ
ಗ್ರಹ-ಣ-ಮಾಡು-ವಾಗ
ಗ್ರಹ-ಣ-ವಾ-ಗು-ವುದು
ಗ್ರಹ-ಣ-ವಿ-ರ-ಬೇಕು
ಗ್ರಹ-ಣ-ವಿಲ್ಲದೆ
ಗ್ರಹ-ಣವೂ
ಗ್ರಹ-ಣ-ಶಕ್ತಿ
ಗ್ರಹ-ಣ-ಶಕ್ತಿ-ಯನ್ನು
ಗ್ರಹ-ಣ-ಶಕ್ತಿಯು
ಗ್ರಹ-ಣ-ಶಕ್ತಿ-ಯೆಲ್ಲ
ಗ್ರಹ-ಣಸ್ವ-ರೂಪಾಸ್ಮಿತಾನ್ವ-ಯಾರ್ಥವತ್ತ್ವ
ಗ್ರಹಣಾ
ಗ್ರಹ-ತಾರಾ
ಗ್ರಹ-ನಕ್ಷತ್ರಾದಿ-ಗಳಾ-ವುವೂ
ಗ್ರಹಾ-ವಳಿ-ಗ-ಳನ್ನು
ಗ್ರಹಾ-ವಳಿ-ಗಳು
ಗ್ರಹಿ-ಕೆಯ
ಗ್ರಹಿಸ
ಗ್ರಹಿಸದ
ಗ್ರಹಿಸ-ಬಲ್ಲ
ಗ್ರಹಿಸ-ಬಲ್ಲದು
ಗ್ರಹಿಸ-ಬಲ್ಲನು
ಗ್ರಹಿಸ-ಬಲ್ಲರೊ
ಗ್ರಹಿಸ-ಬಲ್ಲುದೊ
ಗ್ರಹಿಸ-ಬಲ್ಲೆವು
ಗ್ರಹಿಸ-ಬಹು-ದಾ-ಗಿತ್ತು
ಗ್ರಹಿಸ-ಬಹುದು
ಗ್ರಹಿಸ-ಬಹು-ದೆಂಬುದೇ
ಗ್ರಹಿಸ-ಬೇಕು
ಗ್ರಹಿಸ-ಲಾರದು
ಗ್ರಹಿಸ-ಲಾರೆವು
ಗ್ರಹಿಸ-ಲಿಲ್ಲ
ಗ್ರಹಿಸಲು
ಗ್ರಹಿಸಲ್ಪಟ್ಟ
ಗ್ರಹಿಸಿ
ಗ್ರಹಿಸಿದ
ಗ್ರಹಿಸಿ-ದೆವು
ಗ್ರಹಿಸಿದ್ದೇ
ಗ್ರಹಿಸು
ಗ್ರಹಿ-ಸುತ್ತದೆ
ಗ್ರಹಿಸುತ್ತಾನೆ
ಗ್ರಹಿಸುತ್ತಿ-ರು-ವೆವು
ಗ್ರಹಿಸುವ
ಗ್ರಹಿಸು-ವಷ್ಟು
ಗ್ರಹಿ-ಸು-ವು-ದಕ್ಕೆ
ಗ್ರಹಿಸು-ವು-ದನ್ನು
ಗ್ರಹಿಸು-ವುದರ
ಗ್ರಹಿಸು-ವುದು
ಗ್ರಹೀತೃ
ಗ್ರಾಮ
ಗ್ರಾಮಕ್ಕೂ
ಗ್ರಾಹಕ
ಗ್ರಾಹ್ಯ
ಗ್ರಾಹ್ಯ-ಕತ್ವಕ್ಕೆ
ಗ್ರಾಹ್ಯ-ವಾಗಿ
ಗ್ರಾಹ್ಯೇಷು
ಗ್ರೀಕ್
ಗ್ಲಾಸು
ಘಂಟಾಘೋಷ-ವಾಗಿ
ಘಂಟೆ-ಗಳ
ಘಂಟೆ-ಗಳಿ
ಘಟನಾ
ಘಟನೆ
ಘಟನೆ-ಗಳ
ಘಟನೆ-ಗ-ಳನ್ನು
ಘಟನೆ-ಗಳಿಂದ
ಘಟನೆ-ಗಳಿಗೆ
ಘಟನೆ-ಗಳು
ಘಟ-ನೆಗೂ
ಘಟ-ನೆಯ
ಘಟ್ಟ-ಗಳು
ಘಟ್ಟಿ-ಗಳೇ
ಘಟ್ಟಿ-ಯಾದ
ಘನ
ಘನತೆ
ಘನ-ದಂತೆ
ಘನ-ವಸ್ತು-ಗಳೆಲ್ಲ
ಘನ-ವಸ್ತು-ವಿನ
ಘನ-ವಾ-ಗು-ವುದು
ಘನೀ-ಭೂತ-ರಾದ
ಘರ್ಷ-ಣವೆ
ಘರ್ಷಣೆ
ಘರ್ಷ-ಣೆಗೆ
ಘರ್ಷಣೆ-ಯನ್ನು
ಘಾತಕ್ಕೆ
ಘೋರ
ಘೋರ-ತಮ
ಘೋರ-ತಮ-ವಾದ
ಘೋರ-ದುಃಖ
ಘೋರ-ಪಾಪ-ದಂತೆ
ಘೋರ-ವಾಗಿ
ಘೋರ-ವಾದ
ಘೋರ-ವೈಮನಸ್ಯ-ವನ್ನು
ಘೋಷಿ-ಸುತ್ತಿ-ರಲಿ
ಘೋಷಿಸುತ್ತಿ-ರುವ
ಚ
ಚಂಚಲ
ಚಂಚಲ-ಗೊಳಿ-ಸುತ್ತವೆಯೋ
ಚಂಚಲ-ಗೊಳಿ-ಸು-ವುದು
ಚಂಚಲ-ಚಿತ್ತ-ರಾ-ಗಿ-ರು-ವೆವು
ಚಂಚ-ಲತೆ
ಚಂಚಲ-ತೆಯ
ಚಂಚಲ-ತೆ-ಯನ್ನು
ಚಂಚಲ-ತೆ-ಹೇಗೆ
ಚಂಚಲ-ಪಡಿ-ಸುವ
ಚಂಚಲ-ವಾಗಿ
ಚಂಚಲ-ವಾಗಿ-ರು-ವ-ವನು
ಚಂಚಲ-ವಾಗಿ-ರು-ವಾಗ
ಚಂಚಲ-ವಾಗಿ-ರು-ವು-ದ-ರಿಂದ
ಚಂಚಲ-ವಾ-ಗಿ-ರುವುದು
ಚಂಚಲ-ವಾ-ಯಿತು
ಚಂಡ
ಚಂದ್ರ
ಚಂದ್ರ-ಇವು-ಗಳಲ್ಲಿ
ಚಂದ್ರನ
ಚಂದ್ರ-ನಲ್ಲಿ
ಚಂದ್ರ-ನಾ-ಗು-ವುದು
ಚಂದ್ರ-ನಿಂದ
ಚಂದ್ರ-ನಿಗೂ
ಚಂದ್ರ-ನಿಗೆ
ಚಂದ್ರನು
ಚಂದ್ರನೇ
ಚಂದ್ರ-ರನ್ನು
ಚಂದ್ರ-ಲೋಕಕ್ಕೂ
ಚಂದ್ರ-ಲೋ-ಕಕ್ಕೆ
ಚಂದ್ರ-ಲೋಕವು
ಚಂದ್ರ-ವಂಶ-ಗಳಿಗೆ
ಚಂದ್ರ-ವಂಶದ
ಚಂದ್ರೇ
ಚಕಮುಕಿ
ಚಕ್ರ
ಚಕ್ರಕ್ಕೆ
ಚಕ್ರ-ಗಳ
ಚಕ್ರ-ಗ-ಳನ್ನು
ಚಕ್ರ-ಗಳಿಗೆ
ಚಕ್ರದ
ಚಕ್ರ-ದಂತೆ
ಚಕ್ರ-ದಲ್ಲಿ
ಚಕ್ರ-ದಲ್ಲಿ-ದೆಯೊ
ಚಕ್ರ-ದಿಂದ
ಚಕ್ರ-ದೊ-ಳಗೆ
ಚಕ್ರ-ಬಡ್ಡಿ
ಚಕ್ರ-ವನ್ನು
ಚಕ್ರ-ವರ್ತಿ
ಚಕ್ರ-ವರ್ತಿ-ಗಿಂತ
ಚಕ್ರ-ವರ್ತಿಗೆ
ಚಕ್ರ-ವರ್ತಿ-ಯಂತಿ-ರುವ
ಚಕ್ರ-ವರ್ತಿಯು
ಚಕ್ರ-ವಾದ
ಚಕ್ರ-ವಿ-ರುವ
ಚಕ್ರಾಧಿಪತ್ಯ
ಚಕ್ಷುಃ
ಚಟು
ಚಟು-ವಟಿಕೆ
ಚಟು-ವಟಿಕೆ-ಗಳ
ಚಟು-ವಟಿಕೆ-ಗಳಿಗೆ
ಚಟು-ವಟಿಕೆ-ಗಳು
ಚಟು-ವಟಿಕೆ-ಗೆಲ್ಲ
ಚಟು-ವಟಿಕೆಯ
ಚಟು-ವಟಿಕೆ-ಯಿಂದ
ಚತುರ
ಚತುರನು
ಚತುರ-ರಾಗಿದ್ದರೂ
ಚತುರರು
ಚತುರ್ಥಃ
ಚದರಿ-ಹೋಗಿದೆ
ಚದು-ರಿಸು
ಚದುರಿ-ಹೋಗು-ವುದೇ
ಚನೆ
ಚನೆ-ಗಳು
ಚನೆಯ
ಚಪಲಕ್ಕೆ
ಚಪ-ಲತೆ-ಯನ್ನು
ಚಮತ್ಕಾರ-ವಾಗಿ
ಚಮತ್ಕಾರ-ವಾಗಿದೆ
ಚಮತ್ಕಾರ-ವಾದ
ಚರಮ
ಚರಾಚರ
ಚರಿತ್ರಾಂಶ
ಚರಿತ್ರೆ
ಚರಿತ್ರೆ-ಕಾರರು
ಚರಿತ್ರೆಯ
ಚರಿತ್ರೆ-ಯನ್ನು
ಚರಿತ್ರೆ-ಯಿಂದ
ಚರಿತ್ರೆ-ಯೆಲ್ಲ
ಚರಿತ್ರೆಯೇ
ಚರ್ಚನ್ನಾಗಿ
ಚರ್ಚನ್ನೆ
ಚರ್ಚಾ
ಚರ್ಚಾ-ಕೂಟ-ದಲ್ಲಿ
ಚರ್ಚಾಸ್ಪದ-ವಾದ
ಚರ್ಚಿ
ಚರ್ಚಿಗೆ
ಚರ್ಚಿನ
ಚರ್ಚಿ-ನಲ್ಲಿ
ಚರ್ಚಿ-ನಲ್ಲಿಯೂ
ಚರ್ಚಿ-ಸಿ-ರು-ವರು
ಚರ್ಚಿ-ಸುತ್ತಿದ್ದಾಗ
ಚರ್ಚಿ-ಸುತ್ತಿ-ರುವುದು
ಚರ್ಚಿ-ಸುತ್ತೇನೆ
ಚರ್ಚಿ-ಸೋಣ
ಚರ್ಚು
ಚರ್ಚು-ಗಳ
ಚರ್ಚು-ಗ-ಳನ್ನು
ಚರ್ಚು-ಗಳಲ್ಲಿ
ಚರ್ಚೆ
ಚರ್ಚೆ-ಗಳಿವೆ
ಚರ್ಚೆ-ಗಳು
ಚರ್ಚೆಗೆ
ಚರ್ಚೆಯ
ಚರ್ಚೆ-ಯಂತೆ
ಚರ್ಚೆ-ಯನ್ನು
ಚರ್ಚೆ-ಯಲ್ಲಿ
ಚರ್ಚೆ-ಯಾ-ಗಿದೆ
ಚರ್ಚೆ-ಯಾದ
ಚರ್ಚೆಯೆ
ಚರ್ಮದ
ಚರ್ಮ-ವನ್ನು
ಚಲನ-ರ-ಹಿತ-ವಾಗಿ-ರ-ಬಹುದು
ಚಲನವಲನ-ಗಳಲ್ಲಿ
ಚಲನವಲನ-ಗಳೂ
ಚಲನಾ-ಶಕ್ತಿ
ಚಲನೂ
ಚಲನೆ
ಚಲನೆ-ಗ-ಳನ್ನು
ಚಲನೆ-ಗ-ಳನ್ನೂ
ಚಲನೆ-ಗಳಿವೆ
ಚಲನೆ-ಗಳೂ
ಚಲನೆ-ಗಳೆಲ್ಲ
ಚಲನೆ-ಗಳೇ-ಉಸಿ-ರನ್ನು
ಚಲ-ನೆಗೂ
ಚಲ-ನೆಗೆ
ಚಲನೆಯ
ಚಲನೆ-ಯಂತೆ
ಚಲನೆ-ಯನ್ನು
ಚಲನೆ-ಯನ್ನುಂಟು-ಮಾಡಿ-ದವು
ಚಲನೆ-ಯಲ್ಲಿ
ಚಲನೆ-ಯಲ್ಲಿಯೂ
ಚಲ-ನೆಯು
ಚಲನೆ-ಯುಂಟಾ-ಗು-ವುದು
ಚಲ-ನೆಯೂ
ಚಲನೆ-ಯೆಲ್ಲ
ಚಲ-ನೆಯೇ
ಚಲನೆ-ಯೊಂದಿಗೆ
ಚಲಾಯಿಸ-ಲಾರದು
ಚಲಿ-ಸದ
ಚಲಿ-ಸದೆ
ಚಲಿ-ಸದೇ
ಚಲಿಸ-ಬಲ್ಲದು
ಚಲಿಸ-ಬಹುದು
ಚಲಿಸ-ಬೇಕಾ-ದರೆ
ಚಲಿಸ-ಬೇಕೆಂಬುದು
ಚಲಿಸ-ಲಾರದು
ಚಲಿಸ-ಲಾರೆವು
ಚಲಿಸಿ
ಚಲಿಸಿ-ದರೆ
ಚಲಿಸು
ಚಲಿ-ಸುತ್ತ
ಚಲಿ-ಸುತ್ತದೆ
ಚಲಿ-ಸುತ್ತಾನೆ
ಚಲಿ-ಸುತ್ತಿದೆ
ಚಲಿ-ಸುತ್ತಿದೆ-ಇ-ವೆಲ್ಲ
ಚಲಿ-ಸುತ್ತಿದ್ದರೆ
ಚಲಿ-ಸುತ್ತಿರು
ಚಲಿ-ಸುತ್ತಿರುವ
ಚಲಿ-ಸುತ್ತಿರು-ವನು
ಚಲಿ-ಸುತ್ತಿರು-ವುದರ
ಚಲಿ-ಸುತ್ತಿರು-ವುದು
ಚಲಿ-ಸುತ್ತಿರು-ವುದೇ
ಚಲಿ-ಸುತ್ತಿಲ್ಲ
ಚಲಿ-ಸುತ್ತಿವೆ
ಚಲಿ-ಸುತ್ತೀರಿ
ಚಲಿ-ಸುತ್ತೇನೆ
ಚಲಿ-ಸುವ
ಚಲಿಸು-ವಂತೆ
ಚಲಿಸು-ವತೆ
ಚಲಿಸು-ವಾಗ
ಚಲಿಸು-ವು-ದನ್ನು
ಚಲಿಸು-ವುದ-ರಲ್ಲೆಲ್ಲಾ
ಚಲಿ-ಸು-ವು-ದಿಲ್ಲ
ಚಲಿ-ಸು-ವುದು
ಚಲ್ಲಾಟಕ್ಕೆ
ಚಳಿ-ಗಳಿಗೆ
ಚಳಿಗಾಲ-ದಲ್ಲಿ
ಚಳಿ-ಯಿಂದ
ಚಾಚಿ-ಕೊಂಡಿ-ರು-ವುದೇ
ಚಾರಿತ್ರಿಕ
ಚಾರಿತ್ರ್ಯವೇ
ಚಾರಿತ್ರ್ಯ-ಶುದ್ಧಿ
ಚಾರ್ಯ-ರಲ್ಲಿ
ಚಾರ್ಯರು
ಚಾರ್ವಾಕ-ರಿಗೆ
ಚಾರ್ವಾಕ-ರಿದ್ದರು
ಚಾಲಕ
ಚಾಲ-ಕರು
ಚಾಲಕ-ಶಕ್ತಿ
ಚಾಲಕ-ಶಕ್ತಿ-ಗಳಾ-ಗಿವೆ
ಚಾಲನ
ಚಾಲನೆ
ಚಾಲಿ
ಚಾಲಿಂಗ-ಪರ್ಯವ-ಸಾ-ನಮ್
ಚಾಳಿ
ಚಾಳಿ-ಯಾ-ಗಿದೆ
ಚಾವಟಿಯ
ಚಾವಟಿ-ಸರಪಳಿ-ಗಳು
ಚಾಶಿಷೋ
ಚಿಂತ-ಕರು
ಚಿಂತನೆ
ಚಿಂತನೆ-ಗ-ಳನ್ನು
ಚಿಂತನೆ-ಗಳಿಂದ
ಚಿಂತನೆ-ಗಳಿ-ರು-ವಂತೆ
ಚಿಂತನೆ-ಗಳು
ಚಿಂತನೆಯ
ಚಿಂತನೆ-ಯಿಂದಲೇ
ಚಿಂತನೆಯೂ
ಚಿಂತಿಸ-ಬೇಡಿ
ಚಿಂತಿಸ-ಲಾರದ
ಚಿಂತಿಸಿ
ಚಿಂತಿಸಿ-ದರೆ
ಚಿಂತಿಸಿ-ದಷ್ಟೂ
ಚಿಂತಿ-ಸುತ್ತಿದೆ
ಚಿಂತಿ-ಸುತ್ತಿ-ರ-ಲಾರನು
ಚಿಂತಿ-ಸು-ವು-ದಕ್ಕೆ
ಚಿಂತಿಸು-ವು-ದನ್ನು
ಚಿಂತೆ
ಚಿಂತೆ-ಯಿಲ್ಲ
ಚಿಂದಿ-ಬಟ್ಟೆ-ಯನ್ನು
ಚಿಕಾಗೊ
ಚಿಕಾಗೋ
ಚಿಕಿತ್ಸ-ಕರ
ಚಿಕಿತ್ಸ-ಕರು
ಚಿಕಿತ್ಸಾಲ-ಯ-ಗಳು
ಚಿಕಿತ್ಸೆ
ಚಿಕಿತ್ಸೆ-ಗಳಲ್ಲಿ
ಚಿಕಿತ್ಸೆ-ಯಾಗ-ಬಲ್ಲುದು
ಚಿಕಿತ್ಸೆಯು
ಚಿಕ್ಕ
ಚಿಕ್ಕಪ್ಪ
ಚಿಟ್ಟೆ-ಗಳಿಂದ
ಚಿಟ್ಟೆ-ಯಂತೆ
ಚಿತಿ-ಶಕ್ತೇರಿತಿ
ಚಿತೇರಪ್ರತಿ-ಸಂಕ್ರ-ಮಾಯಾಸ್ತದಾ-ಕಾರ-ಪತ್ತೌ
ಚಿತ್
ಚಿತ್ಆ-ನಂದ
ಚಿತ್ಆ-ನಂದ-ಗಳೆಂಬುವು
ಚಿತ್ತ
ಚಿತ್ತಂ
ಚಿತ್ತಕ್ಕೆ
ಚಿತ್ತ-ಗಳಲ್ಲಿ
ಚಿತ್ತ-ಗಳು
ಚಿತ್ತದ
ಚಿತ್ತ-ದಲ್ಲಿ
ಚಿತ್ತ-ದಲ್ಲಿದೆ
ಚಿತ್ತ-ದಲ್ಲಿವೆ
ಚಿತ್ತನ್ನೇ
ಚಿತ್ತ-ಭೇದಾತ್ತಯೋರ್ವಿ-ಭಕ್ತಃ
ಚಿತ್ತ-ಮೇಕಮನೇಕೇಷಾಮ್
ಚಿತ್ತಮ್
ಚಿತ್ತ-ವನ್ನು
ಚಿತ್ತ-ವಾ-ಗು-ವುದು
ಚಿತ್ತ-ವಿ-ಕಾ-ರಕ್ಕೆ
ಚಿತ್ತ-ವಿಕ್ಷೇಪಾಸ್ತೇಣಿನ್ತ-ರಾಯಾಃ
ಚಿತ್ತ-ವಿದೆ
ಚಿತ್ತ-ವಿದ್ದರೂ
ಚಿತ್ತವು
ಚಿತ್ತ-ವೃತ್ತಿ
ಚಿತ್ತ-ವೃತ್ತಿ-ಗ-ಳನ್ನು
ಚಿತ್ತ-ವೃತ್ತಿ-ಗಳು
ಚಿತ್ತ-ವೃತ್ತಿಯ
ಚಿತ್ತ-ವೃತ್ತಿ-ಯಾ-ದರೋ
ಚಿತ್ತ-ವೃತ್ಥಿ-ಗಳನ್ನೇನು
ಚಿತ್ತ-ವೆಂದರೆ
ಚಿತ್ತವೆಂದ-ರೇನು
ಚಿತ್ತವೆಂದು
ಚಿತ್ತವೆಂಬ
ಚಿತ್ತವೇ
ಚಿತ್ತ-ಸಂವಿತ್
ಚಿತ್ತ-ಸರೋ-ವರ-ದಲ್ಲಿ
ಚಿತ್ತಸ್ಯ
ಚಿತ್ತಸ್ಯೈಕಾಗ್ರತಾ
ಚಿತ್ತಸ್ವ-ರೂಪಾನು-ಕಾರ
ಚಿತ್ತಾ
ಚಿತ್ತಾ-ಕಾಶ-ವೆಂಬ
ಚಿತ್ತಾ-ಗಲಿ
ಚಿತ್ತಾನ್ತರ-ದೃಶ್ಯೇ
ಚಿತ್ತಾ-ವಸ್ಥೆ-ಯಲ್ಲಿ
ಚಿತ್ತಾ-ವೃತ್ತಿಯು
ಚಿತ್ತಿನ
ಚಿತ್ತು
ಚಿತ್ತೇ
ಚಿತ್ತೈಕಾಗ್ರತೆ-ಯನ್ನು
ಚಿತ್ತೈಕಾಗ್ರತೆ-ಯಿಂದ
ಚಿತ್ರ
ಚಿತ್ರ-ಗಳ
ಚಿತ್ರ-ಗಳನ್ನು
ಚಿತ್ರ-ಗಳನ್ನೆಲ್ಲ
ಚಿತ್ರ-ಗಳಿವೆ
ಚಿತ್ರ-ಗಳೆ
ಚಿತ್ರ-ದಿಂದ
ಚಿತ್ರ-ವನ್ನು
ಚಿತ್ರ-ವನ್ನೆಲ್ಲಾ
ಚಿತ್ರ-ವಾಗಿ
ಚಿತ್ರ-ವಾಗುತ್ತಿದೆ
ಚಿತ್ರ-ವಿತ್ತು
ಚಿತ್ರ-ವಿದೆ
ಚಿತ್ರವು
ಚಿತ್ರವೂ
ಚಿತ್ರವೆ
ಚಿತ್ರ-ವೆಲ್ಲ
ಚಿತ್ರಿಸಿ-ಕೊಳ್ಳಿ
ಚಿತ್ರಿಸಿ-ಕೊಳ್ಳು-ವುದು
ಚಿತ್ರಿಸುತ್ತಿ-ರು-ವೆವು
ಚಿತ್ರಿಸು-ವರು
ಚಿತ್ಶಕ್ತಿಯ
ಚಿತ್ಸ್ವ-ರೂಪ
ಚಿದಾ-ಕಾಶ
ಚಿದಾ-ಕಾಶ-ದಲ್ಲಿ
ಚಿದಾನಂದ-ಸಾ-ಗರ-ವನ್ನು
ಚಿನ್ನಕ್ಕಾಗಿ
ಚಿನ್ನದ
ಚಿನ್ನ-ವನ್ನು
ಚಿನ್ನ-ವಿದೆ
ಚಿಪ್ಪನ್ನು
ಚಿಪ್ಪಿಗೆ
ಚಿಪ್ಪಿನ
ಚಿಪ್ಪಿನಂತೆ
ಚಿಪ್ಪಿ-ನಲ್ಲಿ-ರುವ
ಚಿರ
ಚಿರಂಜೀವತ್ವ-ವನ್ನು
ಚಿರ-ಏಕಾಗ್ರ-ತೆಯ
ಚಿರ-ಜಾಗ್ರತ
ಚಿರ-ಜಾಗ್ರತ-ವಾದ
ಚಿರ-ನೂ-ತನ-ವಾಗಿದೆ
ಚಿರ-ಪರಿ-ಚಿತ
ಚಿರಸ್ಥಾಯಿ-ಯಾಗಿ-ರು-ವುದು
ಚಿಸದೆ
ಚಿಸಲು
ಚಿಸುತ್ತಿ-ರುವ-ನು-ಆ-ದರೆ
ಚಿಸು-ವ-ವ-ರಿಗೆ
ಚಿಹ್ನೆ
ಚಿಹ್ನೆ-ಗಳ
ಚಿಹ್ನೆ-ಗಳದ್ದು
ಚಿಹ್ನೆ-ಗಳನ್ನು
ಚಿಹ್ನೆ-ಗಳು
ಚಿಹ್ನೆ-ಗಳೆಲ್ಲಕ್ಕಿಂತಲೂ
ಚಿಹ್ನೆಯ
ಚಿಹ್ನೆ-ಯನ್ನು
ಚಿಹ್ನೆ-ಯಲ್ಲ
ಚಿಹ್ನೆ-ಯಾಗಿತ್ತು
ಚಿಹ್ನೆ-ಯಾಗಿದೆ
ಚಿಹ್ನೆ-ಯಾಗಿ-ದೆಯೋ
ಚಿಹ್ನೆಯೇ
ಚಿಹ್ನೆ-ಯೊಂದು
ಚೀಟಿ-ಯನ್ನು
ಚೀನಾ
ಚೀಲದ
ಚುಂಬನ-ದಲ್ಲಿರು-ವನು
ಚುಂಬಿಸು-ವಲ್ಲಿ
ಚುಚ್ಚುತ್ತಿ-ರುವ
ಚುರು-ಕಾಗಿ
ಚುರುಕು-ಗೊಳ್ಳು-ತ್ತವೆ
ಚೂರನ್ನು
ಚೂರಾಗಿ
ಚೂರು
ಚೂರು-ಗಳಂತೆ
ಚೂರು-ಚೂ-ರಾಗಿ
ಚೂರು-ಚೂ-ರಾಗು-ವು-ದ-ರಲ್ಲಿ
ಚೂರು-ಮಾಡಿ-ದರೂ
ಚೆಂಡನ್ನು
ಚೆಂಡಿನ
ಚೆಂಡಿ-ನಂತೆ
ಚೆಂಡು
ಚೆಂಡು-ಗಳು
ಚೆತನ್ಯ-ವಿಲ್ಲದೆ
ಚೆನ್ನಾಗಿ
ಚೆನ್ನಾಗಿದೆ
ಚೆನ್ನಾಗಿದ್ದರೆ
ಚೆನ್ನಾಗಿದ್ದಿ-ರ-ಬಹುದು
ಚೆನ್ನಾಗಿದ್ದು
ಚೆನ್ನಾಗಿರು
ಚೆನ್ನಾಗಿ-ರುವ
ಚೆನ್ನಾಗಿ-ರು-ವುದು
ಚೆನ್ನಾದ
ಚೆಲ್ಲಾಟ
ಚೆಲ್ಲಾಪಿಲ್ಲಿ-ಯಾಗಿ
ಚೆಲ್ಲುತ್ತಾ
ಚೇತನ
ಚೇತನದ
ಚೇತನ-ದಿಂದ
ಚೇತನ-ನಾ-ಗಿ-ರುವನು
ಚೇತನ-ವನ್ನು
ಚೇತನ-ವಲ್ಲ-ವೆಂಬುದು
ಚೇತನವು
ಚೇತನವೂ
ಚೇತನವೇ
ಚೇತನ-ಹೀನ-ವಾಗಿದೆ
ಚೇತನಾತ್ಮಕ
ಚೇತನಾತ್ಮಕ-ವಾದ
ಚೇತನಾತ್ಮನ
ಚೇತನಾತ್ಮನಿ-ರುವನು
ಚೇತನಾತ್ಮರು
ಚೇತನಾಧಿಗಮೋಪ್ಯಂತ-ರಾಯಾ-ಭಾವಶ್ಚ
ಚೇತಿ
ಚೇಳು
ಚೈತನ್ಯ
ಚೈತನ್ಯಕ್ಕೆ
ಚೈತನ್ಯದ
ಚೈತನ್ಯ-ದಂತೆ
ಚೈತನ್ಯ-ಮಾತ್ರನು
ಚೈತನ್ಯ-ವನ್ನು
ಚೈತನ್ಯ-ವಿದೆ
ಚೈತನ್ಯ-ವಿದ್ದರೆ
ಚೈತನ್ಯ-ವಿದ್ದಿದ್ದರೆ
ಚೈತನ್ಯ-ವಿಲ್ಲ
ಚೈತನ್ಯವು
ಚೈತನ್ಯವೇ
ಚೈತನ್ಯಸ್ವ-ರೂಪ-ನಾದ
ಚೈನಾ
ಚೊಂಬು
ಚೊಕ್ಕಟ
ಚೋಪಾನ-ವಾಗಿ-ರ-ಬೇಕಾ-ಗಿಲ್ಲ
ಚೋಭ-ಯಾನ-ವ-ಧಾರ-ಣಮ್
ಚೌಕಟ್ಟನ್ನು
ಚ್ಛ್ವಾಸ
ಚ್ಯುತಿ-ಯಾಗುವ
ಛಲ
ಛಲ-ಗಾರ
ಛಲ-ದಿಂದ
ಛಾಂದೋಗ್ಯ
ಛಾಯೆ
ಛಾಯೆ-ಯಂತೆ
ಛಾಯೆ-ಯನ್ನು
ಛಿದ್ರಿಸಿ
ಛೇದಿ-ಸದು
ಛೇದಿ-ಸಲು
ಛೇದಿಸಿ
ಛೇದಿ-ಸುವ
ಜಂಭ
ಜಂಭ-ಕೊಚ್ಚಿ-ಕೊಳ್ಳುವ
ಜಗತಿ-ನಲ್ಲೆಲ್ಲಾ
ಜಗತ್ತನ್ನು
ಜಗತ್ತನ್ನೂ
ಜಗತ್ತನ್ನೆಲ್ಲ
ಜಗತ್ತನ್ನೇ
ಜಗತ್ತಾಗ-ಬೇಕು
ಜಗತ್ತಾ-ಗಲಿ
ಜಗತ್ತಾಗಲೀ
ಜಗತ್ತಾ-ಗಿದೆ
ಜಗತ್ತಾದ
ಜಗತ್ತಾಯಿತು
ಜಗತ್ತಿ-ಗಾಗಿ
ಜಗತ್ತಿ-ಗಿಂತ
ಜಗತ್ತಿಗೆ
ಜಗತ್ತಿದೆ
ಜಗತ್ತಿನ
ಜಗತ್ತಿನಂತೆ
ಜಗತ್ತಿನಲ್ಲಿ
ಜಗತ್ತಿನಲ್ಲಿತ್ತು
ಜಗತ್ತಿನಲ್ಲಿಯೂ
ಜಗತ್ತಿನಲ್ಲಿಯೋ
ಜಗತ್ತಿನಲ್ಲಿ-ರುವ
ಜಗತ್ತಿನಲ್ಲಿ-ರುವುದು
ಜಗತ್ತಿನಲ್ಲೆಲ್ಲ
ಜಗತ್ತಿನಲ್ಲೆಲ್ಲಾ
ಜಗತ್ತಿನಿಂದ
ಜಗತ್ತಿನೊ-ಳಗೆ
ಜಗತ್ತು
ಜಗತ್ತು-ಗಳಲ್ಲೆಲ್ಲ
ಜಗತ್ತು-ಗಳಿವೆ
ಜಗತ್ತೂ
ಜಗತ್ತೆ
ಜಗತ್ತೆಂದು
ಜಗತ್ತೆಲ್ಲವೂ
ಜಗತ್ತೇ
ಜಗದ
ಜಗ-ದಲ್ಲಿ
ಜಗ-ದಲ್ಲೋ
ಜಗದ್ರೂಪ-ದಲ್ಲಿ
ಜಗಳ
ಜಗಳವೇ
ಜಗ್ಗ-ದವ-ನಾ-ದರೆ
ಜಟಿಲ
ಜಟಿಲ-ತೆ-ಯನ್ನು
ಜಟಿಲ-ವಾದ
ಜಟಿಲವೂ
ಜಡ
ಜಡಕ್ಕೆ
ಜಡ-ಜ-ಗತ್ತಿನ
ಜಡ-ಜಗತ್ತು
ಜಡತ-ನಕ್ಕೆ
ಜಡತೆ
ಜಡತ್ವ
ಜಡತ್ವ-ದಿಂದ
ಜಡ-ಮಾಡಿ
ಜಡ-ವನ್ನು
ಜಡ-ವಲ್ಲ
ಜಡ-ವಸ್ತು
ಜಡ-ವಸ್ತು-ವನ್ನು
ಜಡ-ವಸ್ತು-ವಿನ
ಜಡ-ವಸ್ತು-ವಿ-ನಲ್ಲಿ
ಜಡ-ವಸ್ತು-ವಿನ-ವರೆ-ವಿಗೂ
ಜಡ-ವಸ್ತು-ವಿ-ನಿಂದ
ಜಡ-ವಾಗಿದೆ
ಜಡ-ವಾದ
ಜಡ-ವಾದ-ದಿಂದ
ಜಡ-ವಾದ-ವನ್ನು
ಜಡ-ವಾ-ದವು
ಜಡ-ವಾದಿ
ಜಡ-ವಾದಿ-ಗಳಿಗೆ
ಜಡ-ವಾದಿ-ಗಳು
ಜಡ-ವಾದಿಗೂ
ಜಡ-ವಾದಿಯ
ಜಡ-ವಾದಿಯೂ
ಜಡ-ವಾದಿಯೇ
ಜಡ-ವಾ-ದುದು
ಜಡ-ವಾದುವು
ಜಡ-ಶಕ್ತಿ-ಯ-ವರೆಗೆ
ಜಡಾ-ವಸ್ಥೆ-ಯಲ್ಲ
ಜನ
ಜನ-ಕ-ನಂದಿತ್ತ
ಜನಕ್ಕೆ
ಜನ-ಗಳ
ಜನ-ತೆಗೆ
ಜನನ
ಜನ-ನಕ್ಕೆ
ಜನನ-ದಿಂದ
ಜನನ-ಮರಣ-ಗಳ
ಜನನ-ಮರಣ-ಗಳಲ್ಲಿ
ಜನನ-ಮರಣ-ಗಳಲ್ಲಿ-ರು-ವುದು
ಜನನ-ಮರಣ-ಗಳಿಗೆ
ಜನನ-ಮರಣ-ಗಳಿಲ್ಲ
ಜನನ-ಮರಣ-ಗಳು
ಜನನ-ಮರಣಾ
ಜನನ-ಮರಣಾ-ತೀತ
ಜನನ-ಮರಣಾ-ತೀ-ತನು
ಜನನ-ವಾ-ಗಲಿ
ಜನನವೂ
ಜನನ-ವೆನ್ನು-ವೆವು
ಜನ-ನಾರಭ್ಯ
ಜನಪ್ರಿಯ-ತೆ-ಯನ್ನು
ಜನರ
ಜನ-ರಂತೆ
ಜನ-ರನ್ನು
ಜನ-ರಲ್ಲಿ
ಜನ-ರಿಂದಲೂ
ಜನರಿ-ಗಿತ್ತು
ಜನ-ರಿಗೆ
ಜನ-ರಿರು-ವರು
ಜನ-ರಿರು-ವರೆ
ಜನ-ರಿರು-ವರೋ
ಜನ-ರಿಲ್ಲ
ಜನರು
ಜನರೂ
ಜನ-ರೆದು-ರಿಗೆ
ಜನ-ರೆಲ್ಲ
ಜನ-ರೆಲ್ಲಾ
ಜನರೇ
ಜನರೇಕೆ
ಜನ-ವರಿ
ಜನ-ವಾಗು-ವುದಿಲ್ಲ
ಜನ-ವಾಗು-ವುದು
ಜನ-ಸಂಖ್ಯೆ
ಜನ-ಸಾಧಾ-ರಣಕ್ಕೆ
ಜನ-ಸಾಮಾನ್ಯರ
ಜನ-ಸಾಮಾನ್ಯ-ರಿಗೆ
ಜನ-ಸಾಮಾ-ನ್ಯರು
ಜನಾಂಗ
ಜನಾಂಗಕ್ಕಾಗಿ
ಜನಾಂಗಕ್ಕೂ
ಜನಾಂಗಕ್ಕೆ
ಜನಾಂಗ-ಗಳ
ಜನಾಂಗ-ಗಳ-ವರು
ಜನಾಂಗ-ಗಳು
ಜನಾಂಗ-ಗಳೂ
ಜನಾಂಗದ
ಜನಾಂಗ-ದಲ್ಲಿ
ಜನಾಂಗ-ದ-ವ-ರನ್ನು
ಜನಾಂಗ-ದ-ವರು
ಜನಾಂಗ-ವನ್ನು
ಜನಾಂಗ-ವನ್ನೇ
ಜನಾಂಗವು
ಜನಾಂಗ-ವೆಂದ-ರೇನು
ಜನಾಂಗವೇ
ಜನಿಕರ
ಜನಿತ
ಜನಿಸ-ಬಹು-ದೆಂದು
ಜನಿ-ಸಿತು
ಜನಿಸಿತೆನ್ನು-ವರು
ಜನಿ-ಸಿದ
ಜನಿಸಿ-ದವು
ಜನಿ-ಸಿದೆ
ಜನಿಸಿ-ರು-ವರು
ಜನಿಸುತ್ತಿವೆ
ಜನಿ-ಸು-ವನು
ಜನಿಸು-ವರು
ಜನಿ-ಸು-ವು-ದಿಲ್ಲ
ಜನಿ-ಸು-ವುದು
ಜನ್ಮ
ಜನ್ಮ-ಕಥನ್ತಾ
ಜನ್ಮಕ್ಕೆ
ಜನ್ಮ-ಗಳ
ಜನ್ಮ-ಗಳನ್ನಾ-ದರೂ
ಜನ್ಮ-ಗ-ಳನ್ನು
ಜನ್ಮ-ಗಳಲ್ಲಾ-ದರೂ
ಜನ್ಮ-ಗಳು
ಜನ್ಮ-ಗಳೂ
ಜನ್ಮತಃ
ಜನ್ಮ-ತಾಳಿ-ರು-ವೆವು
ಜನ್ಮ-ದಲ್ಲಿ
ಜನ್ಮ-ದಲ್ಲಿಯೇ
ಜನ್ಮ-ದಲ್ಲಿಯೋ
ಜನ್ಮ-ದಲ್ಲೆ
ಜನ್ಮ-ದಲ್ಲೆಲ್ಲಾ
ಜನ್ಮ-ದಲ್ಲೇ
ಜನ್ಮ-ದ-ವರೆ-ವಿಗೂ
ಜನ್ಮ-ಧಾರಣ
ಜನ್ಮ-ವನ್ನು
ಜನ್ಮ-ವಾದ
ಜನ್ಮ-ವಿತ್ತ
ಜನ್ಮವು
ಜನ್ಮ-ವೆತ್ತಿದ
ಜನ್ಮ-ವೆತ್ತಿದನು
ಜನ್ಮ-ವೆತ್ತಿದರು
ಜನ್ಮ-ವೆತ್ತಿದ್ದ
ಜನ್ಮ-ವೆತ್ತಿರ-ರು-ವುದು
ಜನ್ಮವೇ
ಜನ್ಮ-ವೊಂದು
ಜನ್ಮ-ಸಿದ್ಧ
ಜನ್ಮಾರಭ್ಯ
ಜನ್ಮೌಷಧಿ-ಮಂತ್ರ-ತಪಃಸಮಾ-ಧಿಜಾಃ
ಜನ್ಯ
ಜಪ
ಜಪ-ಮಾಡು-ವುದು
ಜಪ-ವೆಂದು
ಜಬಾಲಾ
ಜಯ
ಜಯಕ್ಕೆ
ಜಯ-ದಿಂದ
ಜಯಪ್ರದ
ಜಯಪ್ರದ-ನಾ-ಗಿ-ರುವ
ಜಯಪ್ರದ-ನಾಗು-ವನು
ಜಯಪ್ರದ-ರಾ-ಗಿಲ್ಲ
ಜಯಪ್ರದ-ವಾಗಿ
ಜಯಪ್ರದ-ವಾಗು-ವು-ದಿಲ್ಲ
ಜಯ-ವನ್ನು
ಜಯ-ವಾ-ಗಲಿ
ಜಯವು
ಜಯವೂ
ಜಯ-ವೆಲ್ಲ
ಜಯ-ಶಾಲಿ
ಜಯ-ಶಾಲಿ-ಗಳಾ-ಗ-ಲಿಲ್ಲ
ಜಯ-ಶೀಲ
ಜಯ-ಶೀಲ-ನಾಗ-ಬೇಕಾ-ದರೆ
ಜಯ-ಶೀಲ-ನಾಗು-ವನು
ಜಯ-ಶೀಲ-ನಾದ
ಜಯ-ಶೀಲ-ರಾಗು-ವರು
ಜಯ-ಶೀಲ-ರಾದ
ಜಯಿಸ-ಬಲ್ಲ
ಜಯಿಸ-ಬಹುದು
ಜಯಿ-ಸ-ಬೇಕು
ಜಯಿಸಲ್ಪಡು-ವುವು
ಜಯಿಸಿ
ಜಯಿಸಿದ
ಜಯಿಸಿ-ದ-ಮೇಲೆ
ಜಯಿಸಿ-ದರೆ
ಜಯಿಸು
ಜಯಿಸುತ್ತಾರೆ
ಜಯಿಸುತ್ತಾರೆಯೊ
ಜಯಿಸುವ
ಜಯಿಸು-ವುದರ
ಜಯಿ-ಸು-ವೆವು
ಜರ್ಝರಿತ-ರಾಗುತ್ತಿ-ರು-ವೆವು
ಜರ್ಝರಿತ-ವಾಗಿ
ಜರ್ಮ-ನರು
ಜರ್ಮ-ನಿಯ
ಜರ್ಮನಿ-ಯಲ್ಲಿ
ಜರ್ಮನ್
ಜಲ
ಜಲ-ದಲ್ಲಿ
ಜಲ-ಬಿಂದು-ವಿ-ನಂತೆ
ಜವಾ
ಜವಾಬ್ದಾರರು
ಜವಾಬ್ದಾರಿ
ಜವಾಬ್ದಾರಿ-ಯನ್ನೆಲ್ಲ
ಜವಾಬ್ದಾರಿ-ಯಿಲ್ಲ-ದ-ವರು
ಜಾಗರೂ-ಕತೆ-ಯಿಂದ
ಜಾಗರೂಕ-ರಾಗಿ-ರ-ಬೇಕು
ಜಾಗರೂಕ-ರಾಗಿರಿ
ಜಾಗೃತ
ಜಾಗೃತ-ಗೊಳಿ-ಸ-ಬೇಕಾ-ಗಿದೆ
ಜಾಗೃತ-ಗೊಳಿಸಿ
ಜಾಗೃತ-ಗೊಳಿ-ಸುತ್ತಿದೆ
ಜಾಗೃತ-ಗೊಳಿ-ಸು-ವುದು
ಜಾಗೃತ-ಮಾಡು-ವು-ದೊಂದೇ
ಜಾಗೃತ-ರಾಗಿ
ಜಾಗೃತ-ರಾಗಿ-ರು-ವ-ವರು
ಜಾಗೃತ-ವಾಗ-ಬಹುದು
ಜಾಗೃತ-ವಾಗಿ
ಜಾಗೃತ-ವಾ-ಗು-ವುವು
ಜಾಗೃತ-ವಾದ
ಜಾಗೃತ-ವಾ-ದಾಗ
ಜಾಗೃ-ತಾ-ವಸ್ಥೆ-ಯಲ್ಲಿ
ಜಾಗೃತಿ
ಜಾಗೃತಿ-ಯನ್ನು
ಜಾಗ್ರ-ಗೊಳಿ-ಸು-ವು-ದಕ್ಕೆ
ಜಾಗ್ರತ
ಜಾಗ್ರತ-ಗೊಳಿ-ಸ-ಬಹುದು
ಜಾಗ್ರತ-ಗೊಳಿ-ಸು-ವುದು
ಜಾಗ್ರತ-ನಾಗಿ
ಜಾಗ್ರತ-ನಾಗಿ-ರುವನೊ
ಜಾಗ್ರತ-ನಾಗು
ಜಾಗ್ರತ-ನಾದ
ಜಾಗ್ರತ-ರಾಗಿ
ಜಾಗ್ರತ-ರಾಗು-ವರು
ಜಾಗ್ರತ-ವಾಗಿ
ಜಾಗ್ರತ-ವಾಗಿ-ರುವ
ಜಾಗ್ರತ-ವಾಗು
ಜಾಗ್ರತ-ವಾಗುತ್ತದೆ
ಜಾಗ್ರತ-ವಾಗು-ವುದು
ಜಾಗ್ರ-ತಾ-ವಸ್ಥೆ
ಜಾಗ್ರ-ತಾ-ವಸ್ಥೆಯ
ಜಾಗ್ರ-ತಾ-ವಸ್ಥೆ-ಯಲ್ಲಿ
ಜಾಗ್ರತ್
ಜಾಗ್ರದ-ವಸ್ಥೆ-ಯಲ್ಲಿ
ಜಾಡಿ
ಜಾಡಿ-ಯ-ವ-ನಿಗೆ
ಜಾಡ್ಯ
ಜಾಡ್ಯದ
ಜಾಡ್ಯ-ದಂತೆ
ಜಾಣನೂ
ಜಾತಿ
ಜಾತಿಗೆ
ಜಾತಿ-ದೇಶ-ಕಾಲ
ಜಾತಿ-ದೇಶ-ಕಾಲ-ಗಳಿಂದ
ಜಾತಿ-ದೇಶ-ಕಾಲ-ಸ-ಮಾಯಾ-ನವಚ್ಛಿನ್ನಾಃ
ಜಾತಿ-ಭೇದ-ವಾ-ಗಲೀ
ಜಾತಿಭ್ರಷ್ಠನೆನ್ನುತ್ತಾರೆ
ಜಾತಿಯ
ಜಾತಿ-ಯನ್ನು
ಜಾತಿ-ಯಲ್ಲಿ
ಜಾತಿಯು
ಜಾತಿ-ಲಕ್ಷ-ಣ-ದೇಶೈರನ್ಯತಾ-ನವಚ್ಛೇದಾತ್ತುಲ್ಯಯೋಸ್ತತಃ
ಜಾತ್ಯಂತರ
ಜಾತ್ಯಾಯುರ್ಭೋಗಾಃ
ಜಾನ್
ಜಾನ್ಸ್ಟು-ಯರ್ಟ್
ಜಾಯನ್ತೇ
ಜಾರ
ಜಾರ-ತೂಷ್ಟ್ರ
ಜಾರಿಗೆ
ಜಾರಿ-ಯಲ್ಲಿತ್ತು
ಜಾರಿ-ಯಲ್ಲಿ-ರ-ಬೇಕು
ಜಾರಿ-ಯಲ್ಲಿರು
ಜಾರಿ-ಯಲ್ಲಿ-ರುವ
ಜಾರಿಲ್ಲ
ಜಾರುತ್ತದೆ
ಜಾರು-ವುದು
ಜಾರು-ವುದು-ಅಂದರೆ
ಜಾರು-ವುದು-ಇವು-ಗಳೇ
ಜಾಲ-ವನ್ನು
ಜಾಸ್ತಿ
ಜಾಸ್ತಿ-ಯಾ-ದಾಗ
ಜಾಸ್ತಿ-ಯಾ-ದಾಗಲೂ
ಜಿಂಕೆ
ಜಿಂಕೆ-ಯಂತೆ
ಜಿಗುಪ್ಸೆ-ಯಿಂದ
ಜಿಜ್ಞಾಸು-ಗಳೊಂದಿಗೆ
ಜಿಠ್ಡಿಟ್ಝ್ಠಠಿಜಿಟ್ಞ
ಜಿತಾತ್ಮರೊ
ಜಿತೇಂದ್ರಿಯನು
ಜಿತೇಂದ್ರಿ-ಯರೊ
ಜಿನುಗುತ್ತಿದ್ದುವು
ಜಿನುಗುತ್ತಿ-ರುವುದು
ಜೀರುದುಂಬಿ
ಜೀರುದುಂಬಿಗೆ
ಜೀರುದುಂಬಿ-ಯನ್ನು
ಜೀರುದುಂಬಿಯು
ಜೀರ್ಣಿಸಿ
ಜೀರ್ಣಿಸಿ-ಕೊಂಡು
ಜೀರ್ಣಿಸಿ-ಕೊಳ್ಳ-ಲಾರದ
ಜೀವ
ಜೀವಂತ
ಜೀವಂತ-ನಾ-ಗಿ-ರುವ
ಜೀವಂತ-ವಾಗಿಟ್ಟಿರುತ್ತವೆ
ಜೀವಂತ-ವಾ-ಗು-ವುದು
ಜೀವ-ಕಣ
ಜೀವ-ಕಣ-ದಲ್ಲಿ
ಜೀವ-ಕೋಶ-ದಲ್ಲಿ
ಜೀವಕ್ಕೆ
ಜೀವ-ಗಳ
ಜೀವ-ಗಳು
ಜೀವ-ಜಂತು-ಗ-ಳನ್ನು
ಜೀವ-ಜಂತು-ವಿಗೂ
ಜೀವ-ಜಗತ್ತು-ಗಳು
ಜೀವ-ತತ್ತ್ವ
ಜೀವದ
ಜೀವ-ದಲ್ಲಿ-ರುವ
ಜೀವ-ದಲ್ಲಿ-ರುವುದೇ
ಜೀವ-ದಾನ
ಜೀವ-ದಿಂದ
ಜೀವದ್ರವ್ಯ-ದಲ್ಲಿ
ಜೀವದ್ರವ್ಯ-ದಿಂದ
ಜೀವನ
ಜೀವ-ನಕ್ಕಾಗಿ
ಜೀವ-ನಕ್ಕಿಂತ
ಜೀವ-ನಕ್ಕೂ
ಜೀವ-ನಕ್ಕೆ
ಜೀವ-ನಕ್ರಿಯೆ
ಜೀವ-ನ-ಗಳಿಲ್ಲ
ಜೀವ-ನ-ಗಳೂ
ಜೀವ-ನದ
ಜೀವ-ನ-ದಂತೆ
ಜೀವ-ನ-ದಲ್ಲಿ
ಜೀವ-ನ-ದಲ್ಲಿಯೇ
ಜೀವ-ನ-ದಲ್ಲಿ-ರುವ
ಜೀವ-ನ-ದಲ್ಲೆ
ಜೀವ-ನ-ದಲ್ಲೆಲ್ಲ
ಜೀವ-ನ-ದಲ್ಲೆಲ್ಲಾ
ಜೀವ-ನ-ದಲ್ಲೇ
ಜೀವ-ನ-ದಾಚೆ
ಜೀವ-ನ-ದಿಂದ
ಜೀವ-ನ-ದಿ-ಗಳಲ್ಲಿ
ಜೀವ-ನ-ದೊಂದಿಗೆ
ಜೀವ-ನನ್ನು
ಜೀವ-ನಪ್ರವಾ-ಹದ
ಜೀವ-ನ-ಭಾವ-ನೆ-ಗಳ
ಜೀವ-ನ-ವನ್ನು
ಜೀವ-ನ-ವನ್ನುರ್
ಜೀವ-ನ-ವನ್ನೂ
ಜೀವ-ನ-ವನ್ನೆ
ಜೀವ-ನ-ವನ್ನೆಲ್ಲ
ಜೀವ-ನ-ವನ್ನೆಲ್ಲಾ
ಜೀವ-ನ-ವನ್ನೇಕೆ
ಜೀವ-ನ-ವಿದೆ
ಜೀವ-ನ-ವಿ-ರು-ವುದು
ಜೀವ-ನವು
ಜೀವ-ನವೂ
ಜೀವ-ನವೆ
ಜೀವ-ನ-ವೆಂದರೆ
ಜೀವ-ನ-ವೆಂದು
ಜೀವ-ನ-ವೆಂಬ
ಜೀವ-ನ-ವೆಲ್ಲ
ಜೀವ-ನವೇ
ಜೀವ-ನ-ವೇನು
ಜೀವ-ನ-ಸಾರವೇ
ಜೀವನಾ
ಜೀವ-ನಾಂತ-ರಾಳ-ದಲ್ಲಿ
ಜೀವ-ನಾದ್ಯಂತವೂ
ಜೀವ-ನಾ-ಧಾರ
ಜೀವ-ನಾ-ನು-ಭವ
ಜೀವ-ನಾ-ಸಕ್ತಿ
ಜೀವ-ನಾ-ಸಕ್ತಿಯ
ಜೀವ-ನಾ-ಸಕ್ತಿ-ಯನ್ನು
ಜೀವ-ನಾ-ಸಕ್ತಿ-ಯೊಂದು
ಜೀವ-ನಿ-ಗಾಗಿ
ಜೀವ-ನಿಗೂ
ಜೀವ-ನಿಗೆ
ಜೀವನು
ಜೀವನೂ
ಜೀವನೇ
ಜೀವ-ನೋಪಾಯ
ಜೀವನ್ಮುಕ್ತ
ಜೀವನ್ಮುಕ್ತ-ರಾಗು-ವರು
ಜೀವ-ಮಾನ
ಜೀವ-ರಲ್ಲಿ-ರುವ
ಜೀವ-ರಿಗೂ
ಜೀವರು
ಜೀವರೂ
ಜೀವ-ವನ್ನು
ಜೀವ-ವಿ-ದೆಯೋ
ಜೀವವು
ಜೀವವೂ
ಜೀವ-ವೆನ್ನುವ
ಜೀವ-ಶಕ್ತಿ
ಜೀವ-ಶಕ್ತಿ-ಯನ್ನು
ಜೀವ-ಸ-ಹಿತ
ಜೀವಾ-ಣವು
ಜೀವಾಣಿ-ಎಲ್ಲವೂ
ಜೀವಾಣಿ-ವಿ-ನಲ್ಲಿತ್ತೋ
ಜೀವಾಣು
ಜೀವಾಣು-ವಿಗೆ
ಜೀವಾಣು-ವಿ-ನಲ್ಲಿ
ಜೀವಾಣು-ವಿ-ನಿಂದ
ಜೀವಾತ್ಮ
ಜೀವಾತ್ಮನ
ಜೀವಾತ್ಮ-ನನ್ನು
ಜೀವಾತ್ಮ-ನಲ್ಲಿ
ಜೀವಾತ್ಮ-ನಿಗೆ
ಜೀವಾತ್ಮ-ನಿ-ರುವನು
ಜೀವಾತ್ಮನು
ಜೀವಾತ್ಮ-ರಾಜ-ನಿಗೆ
ಜೀವಾತ್ಮವು
ಜೀವಾತ್ಮ-ವೊಂದೆ
ಜೀವಾಳ
ಜೀವಾಳ-ಇದು
ಜೀವಾಳವೋ
ಜೀವಾ-ವಧಿ
ಜೀವಾ-ಸಕ್ತಿ
ಜೀವಾಸಕ್ತಿಗೆ
ಜೀವಿ
ಜೀವಿ-ಗನ್ನೂ
ಜೀವಿ-ಗಳ
ಜೀವಿ-ಗಳನ್ನಾಗಿ
ಜೀವಿ-ಗ-ಳನ್ನು
ಜೀವಿ-ಗಳಲ್ಲಿ
ಜೀವಿ-ಗಳಲ್ಲಿಯೂ
ಜೀವಿ-ಗಳಲ್ಲೂ
ಜೀವಿ-ಗಳಿಂದ
ಜೀವಿ-ಗಳಿಂದಲೂ
ಜೀವಿ-ಗಳಿಗೆ
ಜೀವಿ-ಗಳಿದ್ದರೆ
ಜೀವಿ-ಗಳಿ-ರುವ
ಜೀವಿ-ಗಳು
ಜೀವಿ-ಗಳೂ
ಜೀವಿ-ಗಳೆಲ್ಲ
ಜೀವಿ-ಗಳೋ
ಜೀವಿಗೂ
ಜೀವಿಗೆ
ಜೀವಿಯ
ಜೀವಿ-ಯಂತೆ
ಜೀವಿ-ಯನ್ನಾಗಿ
ಜೀವಿ-ಯನ್ನು
ಜೀವಿ-ಯಾಗ-ಬಹುದು
ಜೀವಿ-ಯಾ-ಗಲಿ
ಜೀವಿ-ಯಾ-ದರೆ
ಜೀವಿಯು
ಜೀವಿಯೂ
ಜೀವಿ-ಸ-ಬಲ್ಲರು
ಜೀವಿ-ಸ-ಬಾ-ರದು
ಜೀವಿ-ಸ-ಬೇಕಾ-ದರೆ
ಜೀವಿ-ಸ-ಲಾರ
ಜೀವಿ-ಸಲು
ಜೀವಿಸಿ
ಜೀವಿ-ಸಿರ
ಜೀವಿ-ಸಿ-ರುತ್ತಿತ್ತು
ಜೀವಿ-ಸಿ-ರುವ
ಜೀವಿ-ಸಿ-ರು-ವರು
ಜೀವಿ-ಸಿ-ರುವುದು
ಜೀವಿ-ಸಿ-ರು-ವೆವು
ಜೀವಿಸು
ಜೀವಿ-ಸುತ್ತದೆ
ಜೀವಿ-ಸುತ್ತಿ
ಜೀವಿ-ಸುತ್ತಿ-ರು-ವ-ವನು
ಜೀವಿ-ಸುತ್ತಿ-ರುವುದು
ಜೀವಿ-ಸುತ್ತಿ-ರು-ವೆವು
ಜೀವಿ-ಸುತ್ತಿ-ವೆಯೋ
ಜೀವಿ-ಸು-ವನು
ಜೀವಿ-ಸು-ವರು
ಜೀವಿ-ಸು-ವು-ದಿಲ್ಲ
ಜೀವಿ-ಸು-ವುದು
ಜೀವಿ-ಸು-ವುವು
ಜೀವಿ-ಸು-ವೆನು
ಜುಗುಪ್ಸೆ
ಜುಗುಪ್ಸೆ-ಇವು-ಗಳಿಂದ
ಜುಗುಪ್ಸೆ-ಯನ್ನುಂಟು-ಮಾಡು-ವು-ದೆಂದು
ಜೇಡರ
ಜೇಡರ-ಹುಳು
ಜೇಡವು
ಜೇಡಿ-ಮಣ್ಣಿನ
ಜೇನು
ಜೇನು-ತುಪ್ಪ
ಜೇನು-ತುಪ್ಪ-ವನ್ನು
ಜೇನು-ಹುಳವು
ಜೈನ
ಜೈವಲಿ
ಜೊತೆ
ಜೊತೆಗೆ
ಜೊತೆ-ಜೊತೆ-ಯಲ್ಲಿಯೇ
ಜೊತೆ-ಜೊತೆ-ಯಾಗಿ
ಜೊತೆ-ಯನ್ನು
ಜೊತೆ-ಯಲ್ಲಿ
ಜೊತೆ-ಯಲ್ಲಿಯೇ
ಜೊತೆ-ಯಲ್ಲೇ
ಜೋಡಣೆ-ಯನ್ನು
ಜೋಡಣೆ-ಯಲ್ಲೆ
ಜೋಡಿಸ
ಜೋಡಿಸ-ಬೇಕು
ಜೋಡಿಸಲ್ಪಟ್ಟಿವೆ
ಜೋಡಿಸಿ
ಜೋಡಿ-ಸುವ
ಜೋಡಿ-ಸುವ-ವರು
ಜೋಡಿ-ಸು-ವು-ದಕ್ಕೆ
ಜೋಡಿ-ಸು-ವುದು
ಜೋಡಿ-ಸು-ವೆನು
ಜೋಪಾನ-ವಾಗಿ
ಜೋಪಾನ-ವಾಗಿರಿ
ಜ್ಞಯಮಲ್ಪಮ್
ಜ್ಞಾತ
ಜ್ಞಾತ-ವಾಗಿ-ರ-ಬಹುದು
ಜ್ಞಾತ-ವಾದುವು
ಜ್ಞಾತಾ
ಜ್ಞಾತಾಶ್ಚಿತ್ತ-ವೃತ್ತ-ಯಸ್ತತ್
ಜ್ಞಾತೃ
ಜ್ಞಾತೃ-ವಾ-ಗಿ-ರು-ವೆವು
ಜ್ಞಾನ
ಜ್ಞಾನಈ
ಜ್ಞಾನ-ಕಾಂಡ
ಜ್ಞಾನಕ್ಕಾಗಿ
ಜ್ಞಾನಕ್ಕಿಂತ
ಜ್ಞಾನಕ್ಕೂ
ಜ್ಞಾನಕ್ಕೆ
ಜ್ಞಾನ-ಗಳ
ಜ್ಞಾನ-ಗ-ಳನ್ನೂ
ಜ್ಞಾನ-ಗಳು
ಜ್ಞಾನ-ಗಳೂ
ಜ್ಞಾನ-ಗಳೆಲ್ಲಾ
ಜ್ಞಾನಗ್ರಹಣ
ಜ್ಞಾನಜ್ಯೋತಿ
ಜ್ಞಾನಜ್ಯೋತಿ-ಗಾಗಿ
ಜ್ಞಾನಜ್ಯೋತಿ-ಯನ್ನು
ಜ್ಞಾನ-ತ-ರಂಗ
ಜ್ಞಾನ-ತೀ-ತಾ-ವಸ್ಥೆ-ಗಿಂತ
ಜ್ಞಾನದ
ಜ್ಞಾನ-ದಲ್ಲಿ
ಜ್ಞಾನ-ದಲ್ಲಿ-ರುವ
ಜ್ಞಾನ-ದಲ್ಲೆ
ಜ್ಞಾನ-ದಾಹ
ಜ್ಞಾನ-ದಿಂದ
ಜ್ಞಾನ-ದೀಪ್ತಿರಾ-ವಿವೇಕಖ್ಯಾತೇಃ
ಜ್ಞಾನ-ದೃಷ್ಟಿ-ಯಿಂದ
ಜ್ಞಾನ-ದೊಂದಿಗೆ
ಜ್ಞಾನ-ಮ-ಯವೂ
ಜ್ಞಾನಮ್
ಜ್ಞಾನ-ಯೋಗ
ಜ್ಞಾನ-ಯೋ-ಗಕ್ಕೆ
ಜ್ಞಾನ-ಯೋಗ-ದಿಂದಾ-ಗಲಿ
ಜ್ಞಾನ-ಯೋಗಿ
ಜ್ಞಾನ-ರಾಶಿಯೇ
ಜ್ಞಾನ-ಲಾಭ
ಜ್ಞಾನ-ವನ್ನು
ಜ್ಞಾನ-ವನ್ನೂ
ಜ್ಞಾನ-ವನ್ನೆಲ್ಲ
ಜ್ಞಾನ-ವನ್ನೇ
ಜ್ಞಾನ-ವನ್ನೊಳ
ಜ್ಞಾನ-ವಸ್ತು
ಜ್ಞಾನ-ವಾಗಿ-ರ-ಬೇಕು
ಜ್ಞಾನ-ವಾ-ಹಕ-ವೆಂದು
ಜ್ಞಾನ-ವಿದೆ
ಜ್ಞಾನ-ವಿದೆಯೊ
ಜ್ಞಾನ-ವಿರು-ವು-ದನ್ನು
ಜ್ಞಾನ-ವಿ-ರು-ವುದು
ಜ್ಞಾನ-ವಿಲ್ಲ
ಜ್ಞಾನ-ವಿಲ್ಲದ
ಜ್ಞಾನ-ವಿಲ್ಲದೆ
ಜ್ಞಾನವು
ಜ್ಞಾನ-ವುಳ್ಳ
ಜ್ಞಾನವೂ
ಜ್ಞಾನವೆ
ಜ್ಞಾನ-ವೆಂದರೆ
ಜ್ಞಾನ-ವೆಂದು
ಜ್ಞಾನ-ವೆನ್ನುತ್ತೀರೋ
ಜ್ಞಾನ-ವೆನ್ನು-ವುದು
ಜ್ಞಾನ-ವೆನ್ನು-ವು-ದೆಲ್ಲ
ಜ್ಞಾನ-ವೆನ್ನು-ವುದೇ
ಜ್ಞಾನ-ವೆನ್ನುವೆವೊ
ಜ್ಞಾನ-ವೆಲ್ಲ
ಜ್ಞಾನ-ವೆಲ್ಲವೂ
ಜ್ಞಾನ-ವೆಲ್ಲಾ
ಜ್ಞಾನವೇ
ಜ್ಞಾನ-ವೊಂದೇ
ಜ್ಞಾನ-ಶಾಖೆಗೆ
ಜ್ಞಾನ-ಶೂನ್ಯ-ವಾಗಿ
ಜ್ಞಾನ-ಸಾರ-ವಾದ
ಜ್ಞಾನಸ್ಯಾ-ನಂತ್ಯಾತ್
ಜ್ಞಾನಾ
ಜ್ಞಾನಾ-ಕಾಂಕ್ಷಿಯಾ-ಗಿ-ರುವ
ಜ್ಞಾನಾ-ಕಾಶ-ವೆಂದು
ಜ್ಞಾನಾಜ್ಞಾನ-ಗಳ
ಜ್ಞಾನಾ-ತೀತ-ವಾದ
ಜ್ಞಾನಾರ್ಜನೆ
ಜ್ಞಾನಾರ್ಜ-ನೆಗೆ
ಜ್ಞಾನಾರ್ಜನೆ-ಯಾ-ಗು-ವುದು-ಇ-ದೊಂದೇ
ಜ್ಞಾನಿ
ಜ್ಞಾನಿ-ಗಳಲ್ಲಿ
ಜ್ಞಾನಿ-ಗಳು
ಜ್ಞಾನಿ-ಗಳೆಂದಿಗೂ
ಜ್ಞಾನಿಗೆ
ಜ್ಞಾನಿ-ಯಂತೆ
ಜ್ಞಾನಿ-ಯಾಗಿ
ಜ್ಞಾನಿಯು
ಜ್ಞಾನೇಂದ್ರಿಯ
ಜ್ಞಾನೇಂದ್ರಿಯ-ಗ-ಳೆಂದು
ಜ್ಞಾಪಕ
ಜ್ಞಾಪ-ಕಕ್ಕೆ
ಜ್ಞಾಪ-ಕದ
ಜ್ಞಾಪಕ-ದಲ್ಲಿ
ಜ್ಞಾಪಕ-ದಲ್ಲಿಟ್ಟಿ-ರ-ಬೇಕು
ಜ್ಞಾಪಕ-ದಲ್ಲಿಟ್ಟು-ಕೊಂಡಿ-ರ-ಬೇಕು
ಜ್ಞಾಪಕ-ದಲ್ಲಿಟ್ಟು-ಕೊಂಡು
ಜ್ಞಾಪಕ-ದಲ್ಲಿಟ್ಟು-ಕೊಳ್ಳ-ಬೇಕಾದ
ಜ್ಞಾಪಕ-ದಲ್ಲಿಟ್ಟು-ಕೊಳ್ಳ-ಬೇಕು
ಜ್ಞಾಪಕ-ದಲ್ಲಿಡ
ಜ್ಞಾಪಕ-ದಲ್ಲಿ-ಡ-ಬೇಕು
ಜ್ಞಾಪಕ-ದಲ್ಲಿಡಿ
ಜ್ಞಾಪಕ-ದಲ್ಲಿದೆ
ಜ್ಞಾಪಕ-ದಲ್ಲಿ-ರ-ಬಹುದು
ಜ್ಞಾಪಕ-ವಾಗು-ವು-ದಿಲ್ಲವೆ
ಜ್ಞಾಪಕ-ವಿ-ರ-ಬಹುದು
ಜ್ಞಾಪಕ-ವಿರು-ವಂತೆ
ಜ್ಞಾಪಕ-ವಿಲ್ಲ
ಜ್ಞಾಪಕ-ಶಕ್ತಿ
ಜ್ಞಾಪಕ-ಶಕ್ತಿ-ಯನ್ನು
ಜ್ಞಾಪಕ-ಶಕ್ತಿ-ಯಲ್ಲಿಯೂ
ಜ್ಞಾಪಿಸಿ
ಜ್ಞಾಪಿಸಿ-ಕೊಂಡು
ಜ್ಞಾಪಿಸಿ-ಕೊಳ್ಳ-ಬಲ್ಲೆ
ಜ್ಞಾಪಿಸಿ-ಕೊಳ್ಳ-ಬಲ್ಲೆವೆ
ಜ್ಞಾಪಿಸಿ-ಕೊಳ್ಳ-ಬೇಕು
ಜ್ಞಾಪಿಸಿ-ಕೊಳ್ಳ-ಲಾರೆವು
ಜ್ಞಾಪಿಸಿ-ಕೊಳ್ಳಲು
ಜ್ಞಾಪಿಸಿ-ಕೊಳ್ಳಿ
ಜ್ಞಾಪಿಸಿ-ಕೊಳ್ಳು-ವು-ದಕ್ಕೇ
ಜ್ಞಾಪಿಸುತ್ತೇನೆ
ಜ್ಞೇಯ
ಜ್ಞೇಯಕ್ಕಿಂತ
ಜ್ಞೇಯವು
ಜ್ಞೇಯವೂ
ಜ್ಞೇಯ-ವೆಂದು
ಜ್ಯೋತಿ
ಜ್ಯೋತಿ-ಗ-ಳನ್ನು
ಜ್ಯೋತಿ-ಗಿಂತ
ಜ್ಯೋತಿಪ್ರ-ಕಾಶನು
ಜ್ಯೋತಿಯ
ಜ್ಯೋತಿ-ಯನ್ನು
ಜ್ಯೋತಿ-ಯನ್ನೇ
ಜ್ಯೋತಿ-ಯಲ್ಲಿ
ಜ್ಯೋತಿ-ಯಿಂದ
ಜ್ಯೋತಿಯು
ಜ್ಯೋತಿಯೆ
ಜ್ಯೋತಿಯೇ
ಜ್ಯೋತಿ-ಯೊ-ಳಗೆ
ಜ್ಯೋತಿಷ್ಮತೀ
ಜ್ಯೋತಿಷ್ಯ
ಟಿಬೆಟ್
ಟಿಬೆಟ್ಟಿ-ನಲ್ಲಿ
ಟಿಬೆಟ್ಟಿನ-ವ-ರಿಗೆ
ಟೀಕಿ-ಸಲು
ಟೀಕಿಸಿ
ಟೀಕಿ-ಸುತ್ತಾ
ಟೀಕಿಸುತ್ತಿಲ್ಲ
ಟೀಕಿಸು-ವರು
ಟೀಕಿಸು-ವುದೆ
ಟೆಂಡಲ್
ಟೊಳ್ಳು
ಟ್ಟಿರು-ವು-ದಕ್ಕೆ
ಟ್ಯಾಂಟಲಸ್ಸನ
ಟ್ಯಾಂಟಲಸ್ಸ-ನಂತೆ
ಠಕ್ಕ-ತನ
ಡನೆಯೆ
ಡಲಿ
ಡಲು
ಡಲೇ-ಬೇಕು
ಡಾಯ್ಸನ್
ಡಾರ್ವಿನ್
ಡುತ್ತೇವೆ
ಡುವುದು
ಡೆಲ್ಸಾರ್ಟಿ
ಡೇವಿಯ
ಡೈನಮೋ
ಡೈನಮೋ-ದಿಂದ
ಢಾಂಬಿಕ-ನಾಗಿ
ಣಾಮ-ದಿಂದ
ತಂಗಾಳಿ
ತಂಗಿಯ-ರಲ್ಲಿದೆ
ತಂತಿ-ಗಳ
ತಂತಿಯ
ತಂತಿ-ಯಂತೆ
ತಂತಿಯು
ತಂತಿಯೂ
ತಂತು
ತಂತು-ಗ-ಳನ್ನು
ತಂತು-ಗಳೆಂಬ
ತಂತು-ವನ್ನು
ತಂತುವು
ತಂತ್ರ
ತಂತ್ರ-ಗ-ಳನ್ನು
ತಂದ
ತಂದಂತೆ
ತಂದದ್ದಾ-ಯಿತು
ತಂದರು
ತಂದರೂ
ತಂದರೆ
ತಂದಳು
ತಂದ-ವ-ನಿಗೆ
ತಂದ-ವನು
ತಂದ-ಹೊರತು
ತಂದಾಗ
ತಂದಿತು
ತಂದಿದೆ
ತಂದಿದ್ದರೆ
ತಂದಿ-ರುವ
ತಂದಿ-ರು-ವರು
ತಂದಿ-ರುವುದು
ತಂದಿಲ್ಲ
ತಂದು
ತಂದೆ
ತಂದೆ-ಗ-ಳನ್ನು
ತಂದೆ-ಗಳಿಗೆ
ತಂದೆ-ಗಳಿಲ್ಲ
ತಂದೆ-ಗಿಂತ
ತಂದೆಗೆ
ತಂದೆ-ತಾಯಿ-ಗಳ
ತಂದೆ-ತಾಯಿ-ಗಳು
ತಂದೆಯ
ತಂದೆ-ಯಂತೆ
ತಂದೆ-ಯನ್ನು
ತಂದೆ-ಯಲ್ಲಿದೆ
ತಂದೆ-ಯಲ್ಲಿದ್ದರೆ
ತಂದೆ-ಯಲ್ಲಿ-ರುವುದು
ತಂದೆ-ಯ-ವರೆಗೂ
ತಂದೆಯು
ತಂದೆಯೂ
ತಂದೆ-ಯೆಂದು
ತಂದೆ-ಯೊಂದಿಗೆ
ತಂದೇ
ತಂದೊಡ್ಡುವ
ತಂಪಾಗಿ
ತಕ್ಕ
ತಕ್ಕಂತೆ
ತಕ್ಕ-ಮಟ್ಟಿಗೆ
ತಕ್ಕಷ್ಟು
ತಕ್ಷಣ
ತಕ್ಷಣ-ದಲ್ಲಿಯೇ
ತಕ್ಷಣವೆ
ತಕ್ಷಣವೇ
ತಗಲಿದ್ದರೆ
ತಗುಲಿ
ತಗ್ಗಿ-ದರೆ
ತಗ್ಗಿಸ-ಬಹುದು
ತಗ್ಗಿಸಿ
ತಗ್ಗಿ-ಸುವು-ದಕ್ಕಾಗಿ
ತಗ್ಗಿ-ಸು-ವು-ದಕ್ಕೆ
ತಗ್ಗಿ-ಸು-ವುದು
ತಗ್ಗು
ತಗ್ಗು-ವುದು
ತಚ್ಛಿದ್ರೇಷು
ತಜ್ಜಃ
ತಜ್ಜಪಸ್ತದರ್ಥ-ಭಾವ-ನಮ್
ತಜ್ಜಯಾತ್
ತಟಸ್ಥಾ-ವಸ್ಥೆ-ಯಲ್ಲಿರು
ತಟ್ಟ-ಬಹುದು
ತಟ್ಟ-ಬೇಕು
ತಟ್ಟಿದ
ತಟ್ಟಿ-ದರೆ
ತಟ್ಟಿ-ದಾಗ
ತಡ
ತಡ-ಮಾ-ಡಲು
ತಡ-ವಾಗಿಯೋ
ತಡೆ
ತಡೆ-ಗಟ್ಟ-ಬಹುದು
ತಡೆ-ಗಟ್ಟು-ವಂತೆ
ತಡೆ-ಗಟ್ಟು-ವು-ದಕ್ಕೆ
ತಡೆ-ದಷ್ಟೂ
ತಡೆದು
ತಡೆ-ಯದೇ
ತಡೆ-ಯನ್ನು
ತಡೆ-ಯ-ಲಾರ
ತಡೆ-ಯ-ಲಾರದು
ತಡೆ-ಯ-ಲಾರರು
ತಡೆ-ಯಲು
ತಡೆ-ಯಲ್ಪಟ್ಟಿ-ರುತ್ತದೆ
ತಡೆ-ಯಲ್ಪಟ್ಟಿರು-ವೆವು
ತಡೆ-ಯಲ್ಪಡುತ್ತದೆ
ತಡೆ-ಯುಂಟಾಗು-ವು-ದಿಲ್ಲ
ತಡೆ-ಯುತ್ತದೆ
ತಡೆ-ಯುತ್ತವೆ
ತಡೆ-ಯುವು-ದಕ್ಕಾಗಿ
ತಡೆ-ಯು-ವುದು
ತಡೆಯೂ
ತಣ್ಣೀ-ರನ್ನು
ತತಃ
ತತಃಪ್ರಾತಿಭಶ್ರಾವಣ-ವೇದ-ನಾದರ್ಶಾಸ್ವಾದ-ವಾರ್ತಾ
ತತಸ್ತದ್ವಿಪಾಕಾನು-ಗುಣಾ-ನಾ-ಮೇವಾಭಿವ್ಯಕ್ತಿರ್ವಾಸ-ನಾಮ್
ತತೋ
ತತೋ-ಣಿಮಾದಿಪ್ರಾ-ದುರ್ಭಾವಃ
ತತ್
ತತ್ಕಾಲಕ್ಕೆ
ತತ್ಕ್ಷಣವೇ
ತತ್ತ್ವ
ತತ್ತ್ವಕ್ಕೂ
ತತ್ತ್ವಕ್ಕೆ
ತತ್ತ್ವ-ಗಳ
ತತ್ತ್ವ-ಗ-ಳನ್ನು
ತತ್ತ್ವ-ಗಳನ್ನೆಲ್ಲ
ತತ್ತ್ವ-ಗಳಲ್ಲಿ
ತತ್ತ್ವ-ಗಳಲ್ಲೆಲ್ಲ
ತತ್ತ್ವ-ಗಳಾಗಿ
ತತ್ತ್ವ-ಗಳಿವೆ
ತತ್ತ್ವ-ಗಳು
ತತ್ತ್ವಗ್ರಂಥದ
ತತ್ತ್ವಜ್ಞ
ತತ್ತ್ವಜ್ಞ-ನಾದ
ತತ್ತ್ವಜ್ಞ-ರಂತೆ
ತತ್ತ್ವಜ್ಞ-ರಾ-ಗಲು
ತತ್ತ್ವಜ್ಞರು
ತತ್ತ್ವಜ್ಞಾನಿ
ತತ್ತ್ವಜ್ಞಾನಿ-ಗಳಾ-ಗಿದ್ದರು
ತತ್ತ್ವಜ್ಞಾನಿ-ಗಳಿಗೆ
ತತ್ತ್ವಜ್ಞಾನಿ-ಗಳಿ-ರು-ವರು
ತತ್ತ್ವಜ್ಞಾನಿ-ಗಳು
ತತ್ತ್ವಜ್ಞಾನಿಯೂ
ತತ್ತ್ವದ
ತತ್ತ್ವ-ದಲ್ಲಿ
ತತ್ತ್ವ-ದಲ್ಲೆಲ್ಲೂ
ತತ್ತ್ವ-ದಿಂದ
ತತ್ತ್ವ-ಭಾ-ಗಕ್ಕೆ
ತತ್ತ್ವ-ಭಾ-ವನೆ
ತತ್ತ್ವ-ಮಸಿ
ತತ್ತ್ವ-ವನ್ನಾ-ಗಲೀ
ತತ್ತ್ವ-ವನ್ನಾಗಿ
ತತ್ತ್ವ-ವನ್ನು
ತತ್ತ್ವ-ವಾ-ಗಿ-ರುವನು
ತತ್ತ್ವ-ವಾಗು-ವನು
ತತ್ತ್ವ-ವಾಗು-ವು-ದನ್ನು
ತತ್ತ್ವ-ವಾ-ಯಿತು
ತತ್ತ್ವ-ವಿದೆ
ತತ್ತ್ವವು
ತತ್ತ್ವ-ವೆನ್ನು-ವರು
ತತ್ತ್ವ-ವೇನೆಂದರೆ
ತತ್ತ್ವ-ಶಕ್ತಿ-ಯನ್ನು
ತತ್ತ್ವ-ಶಾಸ್ತ್ರ
ತತ್ತ್ವ-ಶಾಸ್ತ್ರ-ಗಳ
ತತ್ತ್ವ-ಶಾಸ್ತ್ರ-ಗಳಿಗೂ
ತತ್ತ್ವ-ಶಾಸ್ತ್ರಜ್ಞದ
ತತ್ತ್ವ-ಶಾಸ್ತ್ರಜ್ಞ-ರಿಗೆ
ತತ್ತ್ವ-ಶಾಸ್ತ್ರಜ್ಞರು
ತತ್ತ್ವ-ಶಾಸ್ತ್ರದ
ತತ್ತ್ವ-ಶಾಸ್ತ್ರ-ದಿಂದ
ತತ್ತ್ವ-ಶಾಸ್ತ್ರವು
ತತ್ತ್ವ-ಸಂಪ್ರ-ದಾಯ-ಗಳಾ-ವುವೂ
ತತ್ತ್ವ-ಸಿದ್ಧಾಂತ-ದೊಂದಿಗೆ
ತತ್ತ್ವಾನ್ವೇಷಣೆ-ಯಲ್ಲಿ
ತತ್ಪರಂ
ತತ್ರ
ತತ್ವದ
ತತ್ವ-ವನ್ನು
ತತ್ಸನ್ನಿಧೌ
ತತ್ಸ್ಥ-ತದಂಜನತಾ
ತಥಾರೂಢೋಭಿನಿ-ವೇಶಃ
ತಥ್ಯ-ಗ-ಳನ್ನು
ತದ-ನಂತರ
ತದನ್ಯ-ಸಾ-ಧಾರ-ಣತ್ವಾತ್
ತದಪಿ
ತದ-ಭಾವಃ
ತದ-ಭಾ-ವಾತ್
ತದರ್ಥ
ತದ-ಸಂಖ್ಯೇಯ-ವಾಸ-ನಾಭಿಶ್ಚಿತ್ರ-ಮಪಿ
ತದಾ
ತದಾಮ್ಯ-ವನ್ನೇ
ತದುಪ-ರಾಗಾಪೇಕ್ಷಿತ್ವಾಚ್ಚಿತ್ತಸ್ಯ
ತದೇವ
ತದೈಕ್ಯ-ಭಾವ-ವನ್ನು
ತದ್ಗತಿಜ್ಞಾನಮ್
ತದ್ದೃಶೇಃ
ತದ್ವಿಪಾಕೋ
ತದ್ವೈ-ರಾಗ್ಯಾದಪಿ
ತನಕ
ತನ-ಗಾಗಿ
ತನಗೆ
ತನಗೇ
ತನ-ಗೋಸ್ಕರ-ವಾಗಿ
ತನ-ದಿಂದ
ತನಿಗೆಂಡ
ತನ್ನ
ತನ್ನಂತೆ
ತನ್ನದೇ
ತನ್ನನ್ನು
ತನ್ನನ್ನೂ
ತನ್ನನ್ನೇ
ತನ್ನಲ್ಲಿ
ತನ್ನಲ್ಲಿದೆ
ತನ್ನಲ್ಲಿಯೇ
ತನ್ನಲ್ಲಿ-ರುವ
ತನ್ನಲ್ಲಿ-ರು-ವರು
ತನ್ನಲ್ಲಿ-ರು-ವು-ದಕ್ಕಿಂತ
ತನ್ನಲ್ಲಿ-ರುವುದು
ತನ್ನಷ್ಟಕ್ಕೆ
ತನ್ನಾತ್ಮ-ನಲ್ಲಿ
ತನ್ನಾತ್ಮ-ವಲ್ಲದೆ
ತನ್ನಾತ್ಮವೇ
ತನ್ನಿ
ತನ್ನಿಂದ
ತನ್ನಿಚ್ಛೆ-ಯಂತೆ
ತನ್ನಿರೋಧಃ
ತನ್ನಿಲ್ಲಿಯೇ
ತನ್ನೆ-ಡೆಗೆ
ತನ್ನೊಂದಿಗೆ
ತನ್ನೊ-ಡನೆ
ತನ್ಮಯ-ನಾಗು-ವನು
ತನ್ಮಯ-ರಾಗು-ವರೊ
ತನ್ಮಾತ್ರ
ತನ್ಮಾತ್ರ-ಗಳ
ತನ್ಮಾತ್ರ-ಗ-ಳನ್ನು
ತನ್ಮಾತ್ರ-ಗಳು
ತನ್ಮಾತ್ರದ
ತನ್ಮಾತ್ರ-ದಿಂದ
ತನ್ಮಾತ್ರ-ದಿಂದಾ-ಗಿವೆ
ತನ್ಮಾತ್ರ-ವನ್ನೂ
ತನ್ಮಾತ್ರೆ-ಗಳೂ
ತಪಶ್ಯಕ್ತಿ-ಯಾಗಿ
ತಪಸ್ವಿ-ಗಳು
ತಪಸ್ಸನ್ನು
ತಪಸ್ಸಿನ
ತಪಸ್ಸಿ-ನಿಂದ
ತಪಸ್ಸು
ತಪಸ್ಸುಈ
ತಪಸ್ಸ್ವಾದ್ಯಾಯೇಶ್ವರ
ತಪ್ಪನ್ನು
ತಪ್ಪನ್ನೆಲ್ಲ
ತಪ್ಪಲ್ಲ
ತಪ್ಪಲ್ಲ-ವೆಂಬ
ತಪ್ಪಾಗಿ
ತಪ್ಪಾಗಿ-ರ-ಬಹು-ದಾ-ದರೂ
ತಪ್ಪಿ
ತಪ್ಪಿ-ಗಾಗಿ
ತಪ್ಪಿದ
ತಪ್ಪಿ-ದಂತೆ
ತಪ್ಪಿ-ದರೆ
ತಪ್ಪಿ-ದಾಗ
ತಪ್ಪಿದೆ
ತಪ್ಪಿ-ನಿಂದ
ತಪ್ಪಿ-ಸ-ಲಾರದು
ತಪ್ಪಿ-ಸಲು
ತಪ್ಪಿಸಿ
ತಪ್ಪಿ-ಸಿ-ಕೊಂಡನು
ತಪ್ಪಿ-ಸಿ-ಕೊಂಡರೆ
ತಪ್ಪಿ-ಸಿ-ಕೊಳ್ಳ-ಬೇಕಾ-ಗಿ-ರುವುದು
ತಪ್ಪಿ-ಸಿ-ಕೊಳ್ಳ-ಬೇಕಾ-ದರೆ
ತಪ್ಪಿ-ಸಿ-ಕೊಳ್ಳ-ಬೇಕಾ-ಯಿತು
ತಪ್ಪಿ-ಸಿ-ಕೊಳ್ಳ-ಬೇಕು
ತಪ್ಪಿ-ಸಿ-ಕೊಳ್ಳ-ಬೇಕೆ
ತಪ್ಪಿ-ಸಿ-ಕೊಳ್ಳ-ಬೇಕೆಂದು
ತಪ್ಪಿ-ಸಿ-ಕೊಳ್ಳ-ಲಾರರು
ತಪ್ಪಿ-ಸಿ-ಕೊಳ್ಳು-ವಷ್ಟು
ತಪ್ಪಿ-ಸಿ-ಕೊಳ್ಳು-ವು-ದಕ್ಕೆ
ತಪ್ಪಿ-ಸಿ-ಕೊಳ್ಳು-ವುದು
ತಪ್ಪಿ-ಸಿ-ಕೊಳ್ಳು-ವುದೇ
ತಪ್ಪಿ-ಸುತ್ತದೆ
ತಪ್ಪಿ-ಸುವ
ತಪ್ಪಿ-ಸು-ವು-ದಕ್ಕೆ
ತಪ್ಪಿ-ಹೋಗು-ವರು
ತಪ್ಪು
ತಪ್ಪು-ಗ-ಳನ್ನು
ತಪ್ಪು-ಗಳಿಗೆ
ತಪ್ಪು-ಗಳು
ತಪ್ಪು-ವುದು
ತಪ್ಪೂ
ತಪ್ಪೆ
ತಪ್ಪೆಂದರೆ
ತಪ್ಪೆಂದು
ತಪ್ಪೋ
ತಬ್ಬಲಿ-ಗಳಾ-ದರು
ತಬ್ಬಲಿಯ
ತಬ್ಬಲಿ-ಯಾದು-ದ-ರಿಂದ
ತಬ್ಬಿ-ಕೊಂಡಿ-ರುವ
ತಬ್ಬಿ-ಕೊಂಡಿ-ರು-ವಿರಿ
ತಬ್ಬು-ವಷ್ಟು
ತಮ
ತಮ-ಗನು-ಗುಣ-ವಾದ
ತಮ-ಗಲ್ಲದೆ
ತಮಗಿ
ತಮ-ಗಿಂತ
ತಮಗೆ
ತಮಗೇ
ತಮಳೆ
ತಮ-ವಾದ
ತಮಸ್ಸಿ-ನಿಂದ
ತಮಸ್ಸಿ-ನಿಂದಲೇ
ತಮಸ್ಸು
ತಮಸ್ಸು-ಗಳ
ತಮಸ್ಸು-ಗಳೇ
ತಮೋ-ಗುಣ
ತಮೋ-ಗುಣ-ಗಳಿಂದ
ತಮ್ಮ
ತಮ್ಮಂತೆ
ತಮ್ಮದೆ
ತಮ್ಮದೇ
ತಮ್ಮನ್ನು
ತಮ್ಮನ್ನೇ
ತಮ್ಮಲ್ಲಿ
ತಮ್ಮಲ್ಲಿದೆ
ತಮ್ಮಲ್ಲಿನ
ತಮ್ಮಲ್ಲಿಯೇ
ತಮ್ಮಷ್ಟಕ್ಕೆ
ತಮ್ಮಾತ್ಮ-ನನ್ನು
ತಮ್ಮಾತ್ಮ-ವನ್ನು
ತಮ್ಮಾತ್ಮವು
ತಮ್ಮಿಂದ
ತಮ್ಮೊ-ಳಗೆ
ತಮ್ಮೊ-ಳಗೇ
ತಯಾ-ರಾಗಿರು
ತಯಾರಾ-ಗು-ವುದು
ತಯಾರಾ-ಗು-ವುವು
ತಯಾ-ರಾದ
ತಯಾರಿ
ತಯಾರಿ-ಸಲ್ಪಟ್ಟದ್ದು
ತಯಾ-ರಿಸಿ
ತಯಾರಿ-ಸುತ್ತವೆ
ತಯಾರಿ-ಸುವ-ವನ
ತಯಾರಿ-ಸು-ವುದು
ತಯಾರು
ತಯಾರು-ಮಾಡಿ-ದರು
ತಯಾರು-ಮಾಡುತ್ತಿ-ರು-ವರು
ತಯಾರು-ಮಾಡು-ವು-ದನ್ನು
ತರ
ತರಂಗ
ತರಂಗ-ಗ-ಳನ್ನು
ತರಂಗ-ಗಳಿಗೆ
ತರಂಗ-ಗಳು
ತರಂಗ-ಗಳೆಲ್ಲ
ತರಂಗ-ಗಳೊಂದಿಗೆ
ತರಂಗ-ದಂತೆ
ತರಂಗ-ವನ್ನು
ತರಂತಿ
ತರ-ಗಂವು
ತರ-ಗತಿ
ತರ-ಗ-ತಿಗೆ
ತರ-ಗತಿ-ಯದು
ತರ-ಗೆಲೆ-ಗಳು
ತರ-ತಮ-ದಲ್ಲಲ್ಲದೆ
ತರ-ತಮ-ದಲ್ಲಿ
ತರ-ತಮ-ದಲ್ಲಿದೆ
ತರ-ತಮ-ದಲ್ಲಿ-ಯಲ್ಲದೆ
ತರದ
ತರ-ಬಲ್ಲ
ತರ-ಬಲ್ಲದು
ತರ-ಬಲ್ಲರು
ತರ-ಬಹುದು
ತರ-ಬಹು-ದೆಂಬು-ದನ್ನು
ತರ-ಬ-ಹುದೇ
ತರ-ಬಾ-ರದು
ತರ-ಬೇಕಾ-ಗಿದೆ
ತರ-ಬೇಕಾ-ದರೆ
ತರ-ಬೇಕಾ-ದುದು
ತರ-ಬೇಕು
ತರ-ಬೇಕೆಂದು
ತರ-ಬೇಕೆಂಬು-ದನ್ನು
ತರ-ಬೇಡಿ
ತರ-ಬೇತಿ
ತರ-ಲಾರದು
ತರ-ಲಾರರು
ತರ-ಲಾರವು
ತರ-ಲಿಲ್ಲ
ತರಲು
ತರ-ವಲ್ಲ
ತರ-ವಾದ
ತರಿಸು-ವಂತೆ
ತರುಣ
ತರುಣನು
ತರುಣ-ರಿಗೆ
ತರುತ್ತದೆ
ತರುತ್ತವೆ
ತರುತ್ತಾನೆ
ತರುತ್ತಾನೋ
ತರುತ್ತಿ-ರು-ವೆನು
ತರುತ್ತೇನೆ
ತರುತ್ತೇವೆ
ತರುವ
ತರು-ವಂತಹ
ತರು-ವಂತೆ
ತರುವನು
ತರು-ವರು
ತರು-ವರು-ಆ-ದರೆ
ತರುವ-ವನು
ತರು-ವ-ವರು
ತರು-ವಷ್ಟು
ತರುವಾಯ
ತರು-ವಿರಿ
ತರುವು-ದಕ್ಕಾಗಿ
ತರುವು-ದಕ್ಕೂ
ತರು-ವು-ದಕ್ಕೆ
ತರು-ವು-ದನ್ನು
ತರು-ವು-ದ-ರಿಂದ
ತರು-ವು-ದಿಲ್ಲ
ತರು-ವುದು
ತರುವುದೆ
ತರು-ವುದೇ
ತರು-ವುವು
ತರು-ವೆನು
ತರುವೆಯಾ
ತರ್ಕ
ತರ್ಕದ
ತರ್ಕ-ಬದ್ಧ-ವಲ್ಲ
ತರ್ಕ-ಬದ್ಧ-ವಾಗಿದೆ
ತರ್ಕ-ಬದ್ಧ-ವಾದ
ತರ್ಕ-ಮಾರ್ಗವೇ
ತರ್ಕ-ವನ್ನು
ತರ್ಕ-ವಿದೆ
ತರ್ಕವು
ತರ್ಕವೂ
ತರ್ಕವೇ
ತರ್ಕ-ಶಾಸ್ತ್ರದ
ತರ್ಕ-ಸಮ್ಮತ
ತರ್ಕ-ಸಮ್ಮತ-ವಾಗಿದೆ
ತರ್ಕ-ಸಿದ್ಧಾಂತ-ದಿಂದ
ತರ್ಕಾ
ತರ್ಕಿಸಿ
ತರ್ಕಿ-ಸು-ವು-ದಿಲ್ಲ
ತಲುಪ-ಬೇಕಾ-ದರೆ
ತಲುಪಿ-ದಾಗ
ತಲುಪುವ
ತಲುಪು-ವು-ದಕ್ಕೆ
ತಲುಪು-ವುದು
ತಲೆ
ತಲೆ-ಕೆಡಿಸಿ
ತಲೆಗೆ
ತಲೆ-ದೋರಿತು
ತಲೆ-ದೋ-ರುತ್ತದೆ
ತಲೆ-ದೋ-ರು-ವುದು
ತಲೆ-ನೋವಿದ್ದರೆ
ತಲೆಯ
ತಲೆ-ಯನ್ನು
ತಲೆ-ಯಲ್ಲಿ
ತಲೆ-ಯಿಂದ
ತಲೆ-ಯೆತ್ತಿ
ತಲೆ-ಯೆತ್ತಿ-ರು-ವರು
ತಲೆ-ಯೆಲ್ಲ
ತಲೆ-ಹರಟೆ
ತಲೆ-ಹಾಕಿತ್ತೋ
ತಲ್ಲಣ-ಗೊಳಿ-ಸಿ-ದುವು
ತಲ್ಲಣಿ-ಸು-ವುದು
ತಲ್ಲೀನ-ವಾ-ಗು-ವುವು
ತಲ್ಲೀನ-ವಾದ
ತಲ್ಲೀನ-ವಾ-ದಾಗ
ತಳಪಾಯ-ವಿಲ್ಲದ
ತಳ-ಭಾಗ-ದಲ್ಲಿ-ರುವುದು
ತಳ-ಭಾಗ-ವನ್ನು
ತಳ-ಭಾಗವೆ
ತಳವು
ತಳ-ಹದಿ
ತಳ-ಹದಿಗೆ
ತಳ-ಹದಿಯ
ತಳ-ಹದಿ-ಯನ್ನಾಗಿ
ತಳ-ಹದಿ-ಯನ್ನು
ತಳ-ಹದಿ-ಯಾಗಿ
ತಳ-ಹದಿ-ಯಾ-ಗಿದೆ
ತಳ-ಹದಿ-ಯಾದ
ತಳ-ಹದಿ-ಯಾ-ದರೆ
ತಳ-ಹದಿಯೂ
ತಳ-ಹದಿಯೆ
ತಳ-ಹದಿಯೇ
ತಳೆ-ದಿರು-ವರು
ತಳ್ಳಿ
ತಳ್ಳಿ-ರುವುದು
ತವಕಿ-ಸುವುದು
ತವರು
ತವರೂ-ರಾ-ಗು-ವುದು
ತವರೂರು
ತಸ್ಮಿನ್
ತಸ್ಯ
ತಸ್ಯಾಪಿ
ತಸ್ಯಾ-ವಿಷ-ಯೀ-ಭೂತತ್ವಾತ್
ತಾ
ತಾಂಡವವಾಡು-ವು-ದಕ್ಕೆ
ತಾಂಡವ-ವಾ-ಡು-ವುದು
ತಾಂಡವವಾಡು-ವುದೋ
ತಾಕಿತು
ತಾಜ್
ತಾಡಿತ-ರಾಗಿ
ತಾತಂದಿ-ರಲ್ಲಿ
ತಾತ್ಕಾಲಿಕ
ತಾತ್ಕಾಲಿಕ-ವಾಗಿ
ತಾತ್ಕಾಲಿಕ-ವಾಗಿ-ಯಾ-ದರೂ
ತಾತ್ತ್ವಿಕ
ತಾತ್ತ್ವಿಕ-ನಿಗೆ
ತಾತ್ತ್ವಿಕ-ನಿ-ರುವನು
ತಾತ್ತ್ವಿಕನೆ
ತಾತ್ತ್ವಿಕರ
ತಾತ್ತ್ವಿಕ-ರನ್ನು
ತಾತ್ತ್ವಿಕ-ರಲ್ಲಿ
ತಾತ್ತ್ವಿಕರು
ತಾತ್ತ್ವಿಕರೂ
ತಾತ್ತ್ವಿಕ-ರೆಲ್ಲ
ತಾತ್ತ್ವಿಕ-ರೆಲ್ಲಾ
ತಾತ್ತ್ವಿಕವೂ
ತಾತ್ತ್ವಿಕ-ಶಕ್ತಿ
ತಾತ್ವಿಕ
ತಾತ್ವಿಕರ
ತಾತ್ವಿಕ-ರೊಬ್ಬರು
ತಾತ್ಸಾರದ
ತಾದಾತ್ಮ್ಯ
ತಾದಾತ್ಮ್ಯ-ದಿಂದ
ತಾದಾತ್ಮ್ಯ-ವನ್ನು
ತಾದಾತ್ಮ್ಯ-ವಾದ
ತಾದಾತ್ಮ್ಯ-ವಾದದ
ತಾದಾತ್ಮ್ಯವು
ತಾನಲ್ಲದೆ
ತಾನಲ್ಲ-ವೆಂದು
ತಾನಾಗಿ
ತಾನಾಗಿಯೇ
ತಾನಿ-ರುವ
ತಾನು
ತಾನೂ
ತಾನೆ
ತಾನೆಂದಿಗೂ
ತಾನೇ
ತಾನೊಂದು
ತಾಮಸ
ತಾಮಸಿ
ತಾಮ-ಸಿಕ
ತಾಮಸಿ-ಕ-ತೆಯ
ತಾಯಂದಿ-ರಿಗೆ
ತಾಯಿ
ತಾಯಿ-ಗಳಿಂದ
ತಾಯಿ-ಗಳು
ತಾಯಿಗೆ
ತಾಯಿಯ
ತಾಯಿ-ಯಂತೆ
ತಾಯಿ-ಯನ್ನು
ತಾಯಿ-ಯಲ್ಲಿದೆ
ತಾಯಿ-ಯಾಗಿ-ರು-ವುದು
ತಾಯಿ-ಯಿಂದ
ತಾಯಿಯು
ತಾಯಿ-ಯೆಂದು
ತಾಯಿ-ಯೆ-ಡೆಗೆ
ತಾಯಿಯೇ
ತಾಯ್ನಾಡು
ತಾರಕಂ
ತಾರಕೆ-ಗಳು
ತಾರ-ತಮ್ಯ
ತಾರ-ತಮ್ಯ-ತೆ-ಯಿಂದ
ತಾರ-ತಮ್ಯ-ದಿಂದ
ತಾರದು
ತಾರದೆ
ತಾರ-ದೆಂದು
ತಾರಾ-ಖ-ಚಿತ
ತಾರಾ-ಚ-ಲನೆ-ಗಳ
ತಾರಾ-ವಳಿ
ತಾರಾ-ವಳಿ-ಗಳು
ತಾರಾ-ವಳಿ-ಗಳೆ
ತಾರಾವ್ಯೂಹ-ಗಳ
ತಾರಾವ್ಯೂಹಜ್ಞಾನಮ್
ತಾರೆ
ತಾರೆ-ಗಳಂತೆ
ತಾರೆ-ಗಳು
ತಾರ್ಕಿಕ
ತಾರ್ಕಿಕ-ರೀತಿ-ಯಲ್ಲಿ
ತಾರ್ಕಿಕ-ವಾ-ಗಿ-ರುವುದು
ತಾಳ
ತಾಳ-ಬೇಕು
ತಾಳಿ
ತಾಳಿತು
ತಾಳಿ-ದಂತೆ
ತಾಳಿ-ದರೆ
ತಾಳಿ-ದ-ವನು
ತಾಳಿ-ದವು
ತಾಳಿ-ದಾಗ
ತಾಳಿದೆ
ತಾಳಿ-ದೆಯೋ
ತಾಳಿ-ರುವ
ತಾಳಿ-ರು-ವು-ದಿಲ್ಲ
ತಾಳಿವೆ
ತಾಳುತ್ತೇವೆ
ತಾಳುವ
ತಾಳು-ವರೋ
ತಾಳು-ವ-ವರೆ-ವಿಗೂ
ತಾಳು-ವು-ದಕ್ಕೆ
ತಾಳು-ವುದು
ತಾಳು-ವುದೂ
ತಾಳು-ವುವು
ತಾಳೆ
ತಾಳ್ಮೆ
ತಾಳ್ಮೆ-ಯಿಂದ
ತಾವು
ತಾವೆ
ತಾವೇ
ತಾವೊಬ್ಬರೆ
ತಾಸಾಮ-ನಾದಿತ್ವಂ
ತಿಂಗಳ
ತಿಂಗಳಲ್ಲಿ
ತಿಂಗಳಲ್ಲಿಯೋ
ತಿಂಗ-ಳಾದ
ತಿಂಗಳಿ-ನ-ವರೆಗೆ
ತಿಂಗಳಿ-ನಿಂದ
ತಿಂಗಳು
ತಿಂಗಳು-ಗಟ್ಟಲೆ
ತಿಂಗಳು-ಗಳಲ್ಲಿ
ತಿಂಗಳು-ಗಳು
ತಿಂಡಿ-ಗ-ಳನ್ನು
ತಿಂದು
ತಿಂದೊ-ಡ-ನೆಯೆ
ತಿಂದೋ
ತಿಕ್ಕಾಟ-ಗಳ
ತಿದ್ದ-ಬಹುದು
ತಿದ್ದಿ-ಕೊಳ್ಳಿ
ತಿದ್ದಿ-ಕೊಳ್ಳು-ವುದು
ತಿದ್ದಿ-ದಂತೆ
ತಿನ್ನದಿ-ರಲಿ
ತಿನ್ನದೆ
ತಿನ್ನದೇ
ತಿನ್ನ-ಬಹುದು
ತಿನ್ನ-ಬೇಕು
ತಿನ್ನಲು
ತಿನ್ನಲೇ
ತಿನ್ನಿ
ತಿನ್ನಿರಿ
ತಿನ್ನುತ್ತದೆ
ತಿನ್ನುತ್ತಲೆ
ತಿನ್ನುತ್ತಾನೆ-ಇದರ
ತಿನ್ನುತ್ತಿ
ತಿನ್ನುತ್ತಿ-ರುವನು
ತಿನ್ನುವ
ತಿನ್ನುವರು
ತಿನ್ನುವಾಗ
ತಿನ್ನುವುಗು
ತಿನ್ನು-ವು-ದಕ್ಕೆ
ತಿನ್ನು-ವು-ದನ್ನು
ತಿನ್ನು-ವು-ದ-ರಲ್ಲಿದೆ
ತಿನ್ನುವು-ದ-ರಲ್ಲೇ
ತಿನ್ನು-ವು-ದಿಲ್ಲ
ತಿನ್ನುವುದು
ತಿರಸ್ಕರ-ಣೀಯ-ವಾದು-ದನ್ನು
ತಿರಸ್ಕರಿ-ಸ-ಬೇಕು
ತಿರಸ್ಕರಿಸ-ಬೇಕೆಂದು
ತಿರಸ್ಕ-ರಿಸಿ
ತಿರಸ್ಕ-ರಿಸಿದ
ತಿರಸ್ಕ-ರಿಸಿ-ದ-ವರು
ತಿರಸ್ಕರಿ-ಸಿಲ್ಲ
ತಿರಸ್ಕ-ರಿಸು
ತಿರಸ್ಕ-ರಿಸು-ವರು
ತಿರಸ್ಕ-ರಿಸು-ವುದು
ತಿರಸ್ಕಾರ
ತಿರಸ್ಕಾರ-ಭಾವ-ದಿಂದ
ತಿರುಗಿ
ತಿರುಗಿತು
ತಿರುಗಿ-ಸ-ಬೇಕು
ತಿರುಗಿ-ಸ-ಲಾಗು
ತಿರು-ಗಿಸಿ
ತಿರುಗಿ-ಸಿ-ದರೆ
ತಿರುಗಿ-ಸು-ವನು
ತಿರುಗಿ-ಸು-ವುದು
ತಿರು-ಗುತ್ತದೆ
ತಿರುಗುತ್ತಿ-ರುವುದು
ತಿರುಗು-ವಂತೆ
ತಿರುಗು-ವು-ದನ್ನು
ತಿರು-ಗು-ವುದು
ತಿರುಚ-ಲಾಗಿ-ದೆಯೋ
ತಿರುಳಿಲ್ಲ
ತಿರುಳಿಲ್ಲದ
ತಿರುಳು
ತಿರೋ-ಗತಿ
ತಿರೋ-ಭಾವವಿ
ತಿಳಿ
ತಿಳಿದ
ತಿಳಿ-ದಂತಹ
ತಿಳಿ-ದಂತೆ
ತಿಳಿ-ದ-ಕೂಡಲೇ
ತಿಳಿ-ದನು
ತಿಳಿ-ದ-ಮೇಲೆ
ತಿಳಿ-ದರು
ತಿಳಿ-ದರೆ
ತಿಳಿ-ದ-ವನು
ತಿಳಿ-ದ-ವನೂ
ತಿಳಿ-ದ-ವರು
ತಿಳಿ-ದಷ್ಟು
ತಿಳಿ-ದಾಗ
ತಿಳಿ-ದಾ-ಯಿತು
ತಿಳಿ-ದಿತ್ತು
ತಿಳಿ-ದಿತ್ತೆಂದೂ
ತಿಳಿ-ದಿದೆ
ತಿಳಿ-ದಿದೆಯೋ
ತಿಳಿ-ದಿದ್ದರೂ
ತಿಳಿ-ದಿದ್ದರೆ
ತಿಳಿ-ದಿದ್ದೀರಿ
ತಿಳಿ-ದಿರ
ತಿಳಿ-ದಿರ-ಬೇಕು
ತಿಳಿ-ದಿರಲಿ
ತಿಳಿ-ದಿರ-ಲಿಲ್ಲ
ತಿಳಿ-ದಿರು
ತಿಳಿ-ದಿರುವ
ತಿಳಿ-ದಿರು-ವಂತೆ
ತಿಳಿ-ದಿರು-ವನು
ತಿಳಿ-ದಿರು-ವರೋ
ತಿಳಿ-ದಿರು-ವ-ವರು
ತಿಳಿ-ದಿ-ರು-ವು-ದಕ್ಕಿಂತ
ತಿಳಿ-ದಿರು-ವುದು
ತಿಳಿ-ದಿರು-ವುದೂ
ತಿಳಿ-ದಿರು-ವು-ದೆಲ್ಲ
ತಿಳಿ-ದಿರು-ವೆವು
ತಿಳಿ-ದಿಲ್ಲ
ತಿಳಿದು
ತಿಳಿದು-ಕೊಂಡ
ತಿಳಿದು-ಕೊಂಡಂತೆ
ತಿಳಿದು-ಕೊಂಡದ್ದನ್ನು
ತಿಳಿದು-ಕೊಂಡರು
ತಿಳಿದು-ಕೊಂಡರೆ
ತಿಳಿದು-ಕೊಂಡಿದ್ದ-ನೆಂದು
ತಿಳಿದು-ಕೊಂಡಿ-ರ-ಬಹುದು
ತಿಳಿದು-ಕೊಂಡಿರು
ತಿಳಿದು-ಕೊಂಡಿ-ರುವ
ತಿಳಿದು-ಕೊಂಡಿ-ರು-ವರು
ತಿಳಿದು-ಕೊಂಡಿ-ರು-ವರೋ
ತಿಳಿದು-ಕೊಂಡಿ-ರು-ವಷ್ಟು
ತಿಳಿದು-ಕೊಂಡಿ-ರು-ವಿರಿ
ತಿಳಿದು-ಕೊಂಡಿ-ರು-ವು-ದಕ್ಕಿಂತ
ತಿಳಿದು-ಕೊಂಡಿ-ರುವೆ
ತಿಳಿದು-ಕೊಂಡಿ-ರು-ವೆವು
ತಿಳಿದು-ಕೊಂಡು
ತಿಳಿದು-ಕೊಳ್ಳ
ತಿಳಿದು-ಕೊಳ್ಳದೇ
ತಿಳಿದು-ಕೊಳ್ಳ-ಬಲ್ಲರು
ತಿಳಿದು-ಕೊಳ್ಳ-ಬಹು-ದಾದ
ತಿಳಿದು-ಕೊಳ್ಳ-ಬಹುದು
ತಿಳಿದು-ಕೊಳ್ಳ-ಬಹು-ದು-ಮನುಷ್ಯನು
ತಿಳಿದು-ಕೊಳ್ಳ-ಬಾ-ರದು
ತಿಳಿದು-ಕೊಳ್ಳ-ಬೇ-ಕಷ್ಟೆ
ತಿಳಿದು-ಕೊಳ್ಳ-ಬೇಕಾ-ಗಿ-ರುವುದು
ತಿಳಿದು-ಕೊಳ್ಳ-ಬೇಕಾದ
ತಿಳಿದು-ಕೊಳ್ಳ-ಬೇಕಾ-ದರೆ
ತಿಳಿದು-ಕೊಳ್ಳ-ಬೇಕಾ-ದುದೆಂದರೆ
ತಿಳಿದು-ಕೊಳ್ಳ-ಬೇಕು
ತಿಳಿದು-ಕೊಳ್ಳ-ಬೇಕೆಂದು
ತಿಳಿದು-ಕೊಳ್ಳ-ಬೇಕೆಂಬ
ತಿಳಿದು-ಕೊಳ್ಳ-ಬೇಡಿ
ತಿಳಿದು-ಕೊಳ್ಳ-ಲಾರದು
ತಿಳಿದು-ಕೊಳ್ಳ-ಲಾರನು
ತಿಳಿದು-ಕೊಳ್ಳ-ಲಾರೆವು
ತಿಳಿದು-ಕೊಳ್ಳಲು
ತಿಳಿದು-ಕೊಳ್ಳಿ
ತಿಳಿದು-ಕೊಳ್ಳು
ತಿಳಿದು-ಕೊಳ್ಳುತ್ತಾನೆ
ತಿಳಿದು-ಕೊಳ್ಳುತ್ತಾ-ನೆಯೋ
ತಿಳಿದು-ಕೊಳ್ಳುತ್ತೇನೆ
ತಿಳಿದು-ಕೊಳ್ಳುತ್ತೇವೆ
ತಿಳಿದು-ಕೊಳ್ಳುವ
ತಿಳಿದು-ಕೊಳ್ಳು-ವನು
ತಿಳಿದು-ಕೊಳ್ಳು-ವರು
ತಿಳಿದು-ಕೊಳ್ಳು-ವ-ವರು
ತಿಳಿದು-ಕೊಳ್ಳು-ವ-ವರೆಗೆ
ತಿಳಿದು-ಕೊಳ್ಳು-ವಿರಿ
ತಿಳಿದು-ಕೊಳ್ಳು-ವು-ದಕ್ಕೆ
ತಿಳಿದು-ಕೊಳ್ಳು-ವು-ದ-ರಿಂದ
ತಿಳಿದು-ಕೊಳ್ಳು-ವು-ದಿಲ್ಲ
ತಿಳಿದು-ಕೊಳ್ಳು-ವುದು
ತಿಳಿದು-ಕೊಳ್ಳು-ವುದೆ
ತಿಳಿದು-ಕೊಳ್ಳು-ವುದೇ
ತಿಳಿದು-ಕೊಳ್ಳು-ವೆವು
ತಿಳಿದು-ಕೊಳ್ಳು-ವೆವೊ
ತಿಳಿದು-ಕೊಳ್ಳೋಣ
ತಿಳಿದು-ದಾ-ಯಿತು
ತಿಳಿದು-ಪುರುಷನು
ತಿಳಿದು-ಬಂದಿದೆ
ತಿಳಿದು-ಬ-ರುತ್ತದೆ
ತಿಳಿದು-ಬ-ರುವ
ತಿಳಿದೂ
ತಿಳಿದೇ
ತಿಳಿದೊ
ತಿಳಿ-ದೊ-ಡ-ನೆಯೆ
ತಿಳಿ-ದೊ-ಡ-ನೆಯೇ
ತಿಳಿದೋ
ತಿಳಿಯ
ತಿಳಿ-ಯದ
ತಿಳಿ-ಯ-ದ-ವನೂ
ತಿಳಿ-ಯ-ದ-ವ-ನೆಂದು
ತಿಳಿ-ಯದು
ತಿಳಿ-ಯ-ದುದೂ
ತಿಳಿ-ಯದೆ
ತಿಳಿ-ಯ-ದೆಯೆ
ತಿಳಿ-ಯ-ದೆಯೊ
ತಿಳಿ-ಯ-ದೆಯೋ
ತಿಳಿ-ಯದೇ
ತಿಳಿ-ಯ-ಬಯ-ಸು-ವೆನು
ತಿಳಿ-ಯ-ಬಲ್ಲ
ತಿಳಿ-ಯ-ಬಲ್ಲನು
ತಿಳಿ-ಯ-ಬಲ್ಲೆ
ತಿಳಿ-ಯ-ಬಲ್ಲೆವೋ
ತಿಳಿ-ಯ-ಬಹು-ದಾದು-ದೆಲ್ಲ
ತಿಳಿ-ಯ-ಬಹುದು
ತಿಳಿ-ಯ-ಬಹುದೋ
ತಿಳಿ-ಯ-ಬೇಕಾ-ಗಿದೆ
ತಿಳಿ-ಯ-ಬೇಕಾ-ದರೂ
ತಿಳಿ-ಯ-ಬೇಕಾ-ದರೆ
ತಿಳಿ-ಯ-ಬೇಕು
ತಿಳಿ-ಯ-ಬೇಕೆಂದು
ತಿಳಿ-ಯ-ಬೇಡಿ
ತಿಳಿ-ಯರು
ತಿಳಿ-ಯ-ಲಾರ
ತಿಳಿ-ಯ-ಲಾರದ
ತಿಳಿ-ಯ-ಲಾರ-ದ-ವ-ನೆಂದು
ತಿಳಿ-ಯ-ಲಾರದೆ
ತಿಳಿ-ಯ-ಲಾರಿರಿ
ತಿಳಿ-ಯ-ಲಾರಿರಿ-ಅ-ದನ್ನು
ತಿಳಿ-ಯ-ಲಾರೆವು
ತಿಳಿ-ಯ-ಲಾರೆ-ವು-ಅ-ವರ
ತಿಳಿ-ಯಲಿ
ತಿಳಿ-ಯ-ಲಿಚ್ಛಿಸಿ-ದನು
ತಿಳಿ-ಯ-ಲಿಚ್ಛಿ-ಸು-ವನು
ತಿಳಿ-ಯ-ಲಿಚ್ಛಿ-ಸು-ವುದು
ತಿಳಿ-ಯ-ಲಿಚ್ಛಿ-ಸುವೆ
ತಿಳಿ-ಯಲು
ತಿಳಿ-ಯಲೂ
ತಿಳಿ-ಯ-ಲೆತ್ನಿ-ಸು-ವುದು
ತಿಳಿ-ಯವೊ
ತಿಳಿ-ಯ-ಹೇಳಿ-ದರೆ
ತಿಳಿ-ಯ-ಹೊರ-ಡರು
ತಿಳಿ-ಯಾಗಿ
ತಿಳಿ-ಯಿತು
ತಿಳಿ-ಯಿರಿ
ತಿಳಿ-ಯಿರು
ತಿಳಿಯು
ತಿಳಿ-ಯುತ್ತದೆ
ತಿಳಿ-ಯುತ್ತಲೂ
ತಿಳಿ-ಯುತ್ತವೆ
ತಿಳಿ-ಯುತ್ತಾನೆ
ತಿಳಿ-ಯುತ್ತಾನೊ
ತಿಳಿ-ಯುತ್ತಾರೆ
ತಿಳಿ-ಯುತ್ತಾರೋ
ತಿಳಿ-ಯುತ್ತೀಯೆ
ತಿಳಿ-ಯುತ್ತೀರಿ
ತಿಳಿ-ಯುತ್ತೇನೆ
ತಿಳಿ-ಯುತ್ತೇವೆ
ತಿಳಿ-ಯುವ
ತಿಳಿ-ಯು-ವಂತಾಗಿ
ತಿಳಿ-ಯು-ವಂತೆ
ತಿಳಿ-ಯು-ವಂತೆಯೇ
ತಿಳಿ-ಯು-ವನು
ತಿಳಿ-ಯು-ವರು
ತಿಳಿ-ಯು-ವ-ವರೆಗೂ
ತಿಳಿ-ಯು-ವ-ವರೆಗೆ
ತಿಳಿ-ಯು-ವಿರಿ
ತಿಳಿ-ಯುವು
ತಿಳಿ-ಯು-ವು-ದಕ್ಕೆ
ತಿಳಿ-ಯು-ವು-ದ-ರಿಂದ
ತಿಳಿ-ಯು-ವು-ದ-ರೊಂದಿಗೆ
ತಿಳಿ-ಯು-ವು-ದಿಲ್ಲ
ತಿಳಿ-ಯು-ವುದು
ತಿಳಿ-ಯು-ವುದೂ
ತಿಳಿ-ಯು-ವುದೇ
ತಿಳಿ-ಯು-ವು-ದೇನೋ
ತಿಳಿ-ಯು-ವುದೋ
ತಿಳಿ-ಯು-ವುವು
ತಿಳಿ-ಯು-ವೆನು
ತಿಳಿ-ಯು-ವೆವು
ತಿಳಿ-ಯು-ವೆವೋ
ತಿಳಿ-ವಳಿಕೆ
ತಿಳಿ-ವಳಿಕೆಯೇ
ತಿಳಿ-ಸುತ್ತದೆ
ತಿಳಿ-ಸುತ್ತಾನೆ
ತಿಳಿ-ಸುತ್ತಿದ್ದರೆ
ತಿಳಿ-ಸುತ್ತೇನೆ
ತಿಳಿ-ಸುವ
ತಿಳಿ-ಸುವು
ತಿಳಿ-ಸುವು-ದಕ್ಕಾಗಿ
ತಿಳಿ-ಸು-ವು-ದಕ್ಕೆ
ತಿಳಿ-ಸು-ವುದು
ತಿಳು-ವಳಿಕೆ
ತಿವಿ-ದನು
ತೀಕ್ಷ್ಣ
ತೀಕ್ಷ್ಣಪ್ರಭೆ-ಯನ್ನು
ತೀತ
ತೀತ-ನ-ವನು
ತೀತ-ವಾಗಿದೆ
ತೀತ-ವಾದ
ತೀತ-ವೆಂಬುದು
ತೀಯ
ತೀರಕ್ಕೆ
ತೀರ-ದಲ್ಲಿ
ತೀರ-ಬೇಕು
ತೀರಿ
ತೀರಿದ
ತೀರಿಸಿ
ತೀರಿ-ಸಿ-ಕೊಂಡು
ತೀರಿ-ಸಿ-ಕೊಳ್ಳ-ಬಹುದು
ತೀರಿ-ಸಿದ
ತೀರಿ-ಹೋದ
ತೀರು-ವುದು
ತೀರ್ಥ-ಶಾಸ್ತ್ರ-ಯಾಗಾದಿ-ಗಳು
ತೀರ್ಥಸ್ಥಳ
ತೀರ್ಪು
ತೀವ್ರ
ತೀವ್ರ-ಗಾಮಿ-ಗಳನ್ನೆಲ್ಲ
ತೀವ್ರ-ಗೊಳಿ-ಸು-ವುದು
ತೀವ್ರ-ತರ-ವಾಗುತ್ತ-ದೆ-ಎಲ್ಲಕ್ಕೂ
ತೀವ್ರ-ತೆಯ
ತೀವ್ರ-ತೆ-ಯನ್ನು
ತೀವ್ರ-ತೆ-ಯಲ್ಲಿ
ತೀವ್ರ-ತೆ-ಯಿಂದ
ತೀವ್ರತ್ಯಾಗ
ತೀವ್ರ-ವಲ್ಲದೆ
ತೀವ್ರ-ವಾಗಿ
ತೀವ್ರ-ವಾಗಿದೆ
ತೀವ್ರ-ವಾಗಿಯೋ
ತೀವ್ರ-ವಾಗಿ-ರ-ಬಹುದು
ತೀವ್ರ-ವಾದ
ತೀವ್ರ-ವಾ-ದರೆ
ತೀವ್ರ-ವಾದು-ದು-ಇದು
ತೀವ್ರ-ಸಂವೇಗಾನಾ-ಮಾಸನ್ನಃ
ತೀವ್ರ-ಸಾಧ-ಕರೋ
ತು
ತುಂಟ
ತುಂಡು
ತುಂಬ
ತುಂಬದೆ
ತುಂಬ-ಬಹುದು
ತುಂಬ-ಬೇಕೆಂಬು-ದನ್ನು
ತುಂಬಾ
ತುಂಬಿ
ತುಂಬಿ-ತುಳು-ಕಾಡುತ್ತಿದೆ
ತುಂಬಿ-ತುಳು-ಕಾಡುತ್ತಿ-ರುವ
ತುಂಬಿದ
ತುಂಬಿದೆ
ತುಂಬಿದ್ದುದು
ತುಂಬಿ-ರುತ್ತದೆ
ತುಂಬಿ-ರುವ
ತುಂಬಿ-ರುವುದು
ತುಂಬಿವೆ
ತುಂಬಿಸಿ
ತುಂಬಿ-ಹೋಗಿದೆ
ತುಂಬಿ-ಹೋಗಿವೆ
ತುಂಬಿ-ಹೋಗು-ವು-ದಿಲ್ಲ
ತುಂಬು-ವು-ದನ್ನು
ತುಂಬು-ವುದು
ತುಂಬು-ವೆವು
ತುಚ್ಛ
ತುಟಿ-ಯನ್ನು
ತುತ್ತ
ತುತ್ತ-ತುದಿ-ಯಲ್ಲಿ
ತುತ್ತ-ತುದಿ-ಯಲ್ಲಿದ್ದಿರಿ
ತುತ್ತಾಗಿ
ತುತ್ತಾಗು-ವೆವು
ತುತ್ತಾದ
ತುತ್ತಾ-ದನು
ತುತ್ತು
ತುತ್ತೂ
ತುದಿ
ತುದಿ-ಗೇರ-ಬೇಕೆಂದು
ತುದಿ-ಮೊದ-ಲಿಲ್ಲದ
ತುದಿ-ಯನ್ನು
ತುಮು-ಲದ
ತುಮುಲ-ದಲ್ಲಿ-ರು-ವೆವು
ತುರಾಯಿ
ತುರ್ಕಿ
ತುಲನಾತ್ಮಕ
ತುಲ್ಯಪ್ರತ್ಯಯೌ
ತುಳಿ-ತಕ್ಕೆ
ತುಳಿ-ತಕ್ಕೊಳ-ಗಾಗಿ-ರು-ವರು
ತುಳಿಯಲ್ಪಟ್ಟಿತ್ತು
ತುಳಿ-ಯಿರಿ
ತುಳು
ತುಳುಕಾ
ತುಳು-ಕಾ-ಡಲಿ
ತುಳು-ಕಾ-ಡಿ-ದಾಗ
ತುಳು-ಕಾ-ಡುತ್ತದೆ
ತುಳು-ಕಾಡುತ್ತಿದೆ
ತುಳು-ಕಾಡುತ್ತಿ-ರುವ
ತೂಕ-ವೆಲ್ಲವೂ
ತೂಗಾಡುತ್ತಿ-ರು-ವೆವು
ತೂಬನ್ನು
ತೂಬು
ತೂಬೂ
ತೂರಿ-ಕೊಳ್ಳಿ
ತೃಣ-ಸ-ಮಾನ-ವಾ-ಗು-ವುದು
ತೃಪ್ತ-ಕರ-ವಾದ
ತೃಪ್ತ-ನಾಗ-ಲಾರ
ತೃಪ್ತ-ನಾಗಿ
ತೃಪ್ತ-ನಾಗಿ-ರುವೆ
ತೃಪ್ತ-ನಾಗು-ವನು
ತೃಪ್ತ-ರಾಗಿ
ತೃಪ್ತ-ರಾಗಿ-ರು-ವರು
ತೃಪ್ತ-ರಾಗಿ-ರು-ವರೋ
ತೃಪ್ತ-ರಾ-ದರು
ತೃಪ್ತರೊ
ತೃಪ್ತಿ
ತೃಪ್ತಿ-ಕರ-ವಾಗಿ
ತೃಪ್ತಿ-ಕರ-ವಾ-ಗಿತ್ತು
ತೃಪ್ತಿ-ಕರ-ವಾಗಿಲ್ಲ-ವೆಂದು
ತೃಪ್ತಿ-ಗಾಗಿ
ತೃಪ್ತಿ-ದಾಯ-ಕ-ವಾದ
ತೃಪ್ತಿ-ಪಡಿ-ಸ-ಬಲ್ಲ
ತೃಪ್ತಿ-ಪಡಿ-ಸ-ಲಾರಿರಿ
ತೃಪ್ತಿ-ಪಡಿ-ಸುವ
ತೃಪ್ತಿ-ಪ-ಡುತ್ತೇವೆ
ತೃಪ್ತಿ-ಮಾಡ-ಲಾರದ
ತೃಪ್ತಿ-ಯನ್ನು
ತೃಪ್ತಿ-ಯಾಗ-ಲಿಲ್ಲ
ತೃಪ್ತಿ-ಯಾಗಿ
ತೃಪ್ತಿ-ಯಾಗುವ
ತೃಪ್ತಿ-ಯಾಗು-ವಂತೆ
ತೃಪ್ತಿ-ಯಾಗು-ವುದು
ತೃಪ್ತಿ-ಯಾ-ಯಿತು
ತೃಪ್ತಿ-ಯಿಂದ
ತೆಂದು
ತೆಗೆದ
ತೆಗೆ-ದ-ಮೇಲೆ
ತೆಗೆ-ದರೆ
ತೆಗೆದು
ತೆಗೆ-ದು-ಕೊಂಡನು
ತೆಗೆ-ದು-ಕೊಂಡರು
ತೆಗೆ-ದು-ಕೊಂಡರೆ
ತೆಗೆ-ದು-ಕೊಂಡಿ
ತೆಗೆ-ದು-ಕೊಂಡಿ-ರು-ವರು
ತೆಗೆ-ದು-ಕೊಂಡಿಲ್ಲ
ತೆಗೆ-ದು-ಕೊಂಡು
ತೆಗೆ-ದು-ಕೊಂಡು-ಹೋಗಿ
ತೆಗೆ-ದು-ಕೊಂಡು-ಹೋ-ದರೆ
ತೆಗೆ-ದು-ಕೊಂಡೆ
ತೆಗೆ-ದು-ಕೊಳ್ಳದೆ
ತೆಗೆ-ದು-ಕೊಳ್ಳದೇ
ತೆಗೆ-ದು-ಕೊಳ್ಳ-ಬಹು-ದಾದ
ತೆಗೆ-ದು-ಕೊಳ್ಳ-ಬಹುದು
ತೆಗೆ-ದು-ಕೊಳ್ಳ-ಬೇಕು
ತೆಗೆ-ದು-ಕೊಳ್ಳಿ
ತೆಗೆ-ದು-ಕೊಳ್ಳಿ-ಇ-ದ-ರಲ್ಲಿ
ತೆಗೆ-ದು-ಕೊಳ್ಳಿ-ಉದಾ-ಹರ-ಣೆಗೆ
ತೆಗೆ-ದು-ಕೊಳ್ಳಿಪ್ರತಿ-ಯೊಬ್ಬರೂ
ತೆಗೆ-ದು-ಕೊಳ್ಳುತ್ತಾನೆ
ತೆಗೆ-ದು-ಕೊಳ್ಳುತ್ತೀರಿ
ತೆಗೆ-ದು-ಕೊಳ್ಳುವ
ತೆಗೆ-ದು-ಕೊಳ್ಳು-ವಂತೆ
ತೆಗೆ-ದು-ಕೊಳ್ಳು-ವನು
ತೆಗೆ-ದು-ಕೊಳ್ಳು-ವು-ದಕ್ಕೆ
ತೆಗೆ-ದು-ಕೊಳ್ಳು-ವು-ದಿಲ್ಲ
ತೆಗೆ-ದು-ಕೊಳ್ಳು-ವುದು
ತೆಗೆ-ದು-ಕೊಳ್ಳು-ವುದೋ
ತೆಗೆ-ದು-ಕೊಳ್ಳೋಣ
ತೆಗೆ-ದು-ಬಿಟ್ಟರೆ
ತೆಗೆ-ದು-ಬಿಡು-ವರು
ತೆಗೆ-ದು-ಹಾಕ-ಬೇಕಾ-ಗುತ್ತದೆ
ತೆಗೆ-ಯ-ಬಲ್ಲರೊ
ತೆಗೆ-ಯ-ಬೇಕು
ತೆಗೆ-ಯ-ಬೇಕೆಂದು
ತೆಗೆ-ಯ-ಲಾರೆವು
ತೆಗೆ-ಯಲು
ತೆಗೆ-ಯಿರಿ
ತೆಗೆ-ಯುವ
ತೆಗೆ-ಯು-ವನು
ತೆಗೆ-ಯು-ವರು
ತೆಗೆ-ಯು-ವು-ದ-ರಿಂದ
ತೆಗೆ-ಯುವುದು
ತೆತ್ತು
ತೆಪ್ಪಗೆ
ತೆಯೇ
ತೆರ-ಬಲ್ಲ-ವ-ರಾಗಿದ್ದ
ತೆರಳು-ವುದು
ತೆರಳು-ವೆವು
ತೆರೆ
ತೆರೆ-ಗ-ಳನ್ನು
ತೆರೆಗೆ
ತೆರೆದ
ತೆರೆ-ದರೆ
ತೆರೆ-ದಳು
ತೆರೆ-ದಾಗ
ತೆರೆ-ದಿಡು-ವೆನು
ತೆರೆ-ದಿಡು-ವೆನು-ಹಿಂದಿನ
ತೆರೆ-ದಿದೆ
ತೆರೆ-ದಿದ್ದರೂ
ತೆರೆ-ದಿ-ರು-ವೆನು
ತೆರೆದು
ತೆರೆ-ದು-ಕೊಂಡು
ತೆರೆ-ದೊ-ಡ-ನೆಯೆ
ತೆರೆಯ
ತೆರೆ-ಯಂತೆ
ತೆರೆ-ಯನ್ನು
ತೆರೆ-ಯ-ಲಿಲ್ಲ
ತೆರೆ-ಯಿತು
ತೆರೆ-ಯಿದೆ
ತೆರೆಯು
ತೆರೆ-ಯು-ವನು
ತೆರೆ-ಯು-ವುದು
ತೆರೆ-ಯು-ವುದೇ
ತೆರೆ-ಯೆಲ್ಲ
ತೆಳ್ಳಗಿ-ರ-ಬಹುದು
ತೆಳ್ಳ-ಗಿ-ರುವ
ತೆಳ್ಳೆಗಾಗುತ್ತಾ
ತೆವಳುತ್ತ
ತೆವಳುತ್ತಿ-ರುವ
ತೇ
ತೇಜಸ್ವಿ-ಗ-ಳಾದ
ತೇಜಸ್ವಿ-ಯೊಬ್ಬನು
ತೇಪೆ
ತೇಪೆಯ
ತೇರು-ವುದು
ತೇರ್ಗಡೆ-ಯಾಗ-ಬಲ್ಲ
ತೇರ್ಗಡೆ-ಯಾಗ-ಬೇಕು
ತೇಲಾಡಿ-ರು-ವುವು
ತೇಲಿ
ತೇಲಿ-ಬರುತ್ತಿ-ರುವ
ತೇಲಿ-ಹೋಗು
ತೇಲುತ್ತಾ
ತೇಲುತ್ತಿದೆ
ತೇಲುತ್ತಿದ್ದರೂ
ತೇಲುತ್ತಿ-ರುವುದು
ತೊಂದರೆ
ತೊಂದರೆ-ಕೊಡ-ಬೇಡಿ
ತೊಂದರೆ-ಗ-ಳನ್ನು
ತೊಂದರೆ-ಗಳಿಗೆ
ತೊಂದರೆ-ಗಳಿದ್ದರೂ
ತೊಂದರೆ-ಗಳು
ತೊಂದರೆ-ಗಳು-ಇವು-ಗಳ
ತೊಂದರೆ-ಗಳೂ
ತೊಂದರೆ-ಗೀಡಾ-ಗು-ವುದು
ತೊಂದರೆಗೆ
ತೊಂದರೆ-ಯನ್ನು
ತೊಂದರೆ-ಯಾಗು-ವು-ದಿಲ್ಲ
ತೊಂದರೆ-ಯಾಗು-ವು-ದಿಲ್ಲವೊ
ತೊಂದರೆ-ಯಾ-ಗು-ವುದು
ತೊಂದರೆ-ಯಾ-ದರೆ
ತೊಂದರೆ-ಯಿಂದ
ತೊಂದರೆ-ಯೆಂದರೆ
ತೊಂಬತ್ತ-ರಷ್ಟು
ತೊಂಬತ್ತು
ತೊಟ್ಟಿಲಿ-ನಂತೆ
ತೊಟ್ಟಿಲು-ಗಳು
ತೊಟ್ಟಿಲು-ಗಳೇ
ತೊಡಕಾ-ಯಿತು
ತೊಡಕು-ಗ-ಳನ್ನು
ತೊಡ-ಗದೆ
ತೊಡಗಿ
ತೊಡಗುತ್ತವೆ
ತೊಡಗು-ವನು
ತೊಡೆದು
ತೊಡೆದು-ಹಾಕಿ
ತೊರೆದ
ತೊರೆದ-ಮೇಲೆ
ತೊರೆದ-ವರ
ತೊರೆದು
ತೊರೆದು-ಬಿಡ-ಬೇಕೆಂದು
ತೊರೆ-ಯದೆ
ತೊರೆಯ-ಬೇಕು
ತೊರೆಯ-ಬೇಕೆಂದು
ತೊರೆಯ-ಬೇಕೆಂಬುದು
ತೊರೆ-ಯಿರಿ
ತೊರೆಯು-ವ-ವರೆಗೂ
ತೊರೆಯು-ವು-ದ-ರಿಂದ
ತೊರೆ-ಯು-ವು-ದಿಲ್ಲ
ತೊರೆಯು-ವುದು
ತೊಲಗ-ಬೇಕು
ತೊಲಗಿ
ತೊಲಗಿ-ದಷ್ಟೂ
ತೊಲಗು
ತೊಲ-ಗು-ವುದು
ತೊಳಲುತ್ತಿ-ರು-ವರು
ತೊಳಲುತ್ತಿ-ರುವಾಗ
ತೊಳಲು-ವು-ದಕ್ಕಿಂತ
ತೊಳಲು-ವು-ದಿಲ್ಲ
ತೊಳೆದೊ-ಡ-ನೆಯೆ
ತೋಚದೆ
ತೋಚಿದು
ತೋಚಿದು-ದನ್ನು
ತೋಟ-ವೆಂದು
ತೋಯಿ-ಸಿಲ್ಲ
ತೋಯಿ-ಸು-ವುದು-ಇದ
ತೋರದ
ತೋರದೆ
ತೋರ-ಬಲ್ಲ
ತೋರ-ಬಲ್ಲೆ
ತೋರ-ಬಹುದು
ತೋರ-ಬೇಕಾ-ಗಿಲ್ಲ
ತೋರ-ಬೇಕು
ತೋರಲಿ
ತೋರ-ಲಿಲ್ಲ
ತೋರಲು
ತೋರಿ
ತೋರಿಕೆ
ತೋರಿ-ಕೆ-ಗಳೂ
ತೋರಿ-ಕೆಗೆ
ತೋರಿ-ಕೆಯ
ತೋರಿ-ಕೆ-ಯದು
ತೋರಿ-ಕೆ-ಯಾ-ಗಿದೆ-ಯಾವ
ತೋರಿ-ಕೆಯೂ
ತೋರಿತು
ತೋರಿತ್ತೊ
ತೋರಿದ
ತೋರಿ-ದಂತಾ-ಗುತ್ತದೆ
ತೋರಿ-ದಂತಾಯಿತು
ತೋರಿ-ದಂತೆ
ತೋರಿ-ದರು
ತೋರಿ-ದರೂ
ತೋರಿ-ದರೆ
ತೋರಿ-ದ-ವರು
ತೋರಿ-ದವು
ತೋರಿ-ದಷ್ಟು
ತೋರಿ-ದಾಗ
ತೋರಿ-ದು-ದನ್ನು
ತೋರಿ-ಬ-ರು-ವುದು
ತೋರಿ-ರು-ವರು
ತೋರಿಸ
ತೋರಿ-ಸ-ಬಲ್ಲದೊ
ತೋರಿ-ಸ-ಬಹುದು
ತೋರಿ-ಸ-ಬೇಕಾ-ದರೆ
ತೋರಿ-ಸ-ಬೇಕು
ತೋರಿ-ಸ-ಲಿಲ್ಲ
ತೋರಿ-ಸಲು
ತೋರಿ-ಸಲ್ಪಟ್ಟಿತು
ತೋರಿಸಿ
ತೋರಿ-ಸಿ-ಕೊಡುತ್ತದೆ
ತೋರಿ-ಸಿ-ದನು
ತೋರಿ-ಸಿ-ದಲ್ಲದೆ
ತೋರಿ-ಸಿ-ದಾಗ
ತೋರಿ-ಸಿದ್ದಾರೆ
ತೋರಿ-ಸುತ್ತದೆ
ತೋರಿ-ಸುತ್ತಾನೆ
ತೋರಿ-ಸುತ್ತಿದೆ
ತೋರಿ-ಸುವ
ತೋರಿ-ಸುವ-ವರಿಲ್ಲದೆ
ತೋರಿ-ಸು-ವು-ದಕ್ಕಿಂತ
ತೋರಿ-ಸು-ವು-ದಕ್ಕೆ
ತೋರಿ-ಸು-ವು-ದ-ರಿಂದ
ತೋರಿ-ಸು-ವುದು
ತೋರು
ತೋರುತ್ತದೆ
ತೋರುತ್ತದೆ-ಒಂದು
ತೋರುತ್ತದೆಯೇ
ತೋರುತ್ತದೆಯೊ
ತೋರುತ್ತದೆಯೋ
ತೋರುತ್ತಾ
ತೋರುತ್ತಾನೆ
ತೋರುತ್ತಾರೆ
ತೋರುತ್ತಿ
ತೋರುತ್ತಿದೆ
ತೋರುತ್ತಿ-ದೆಯೋ
ತೋರುತ್ತಿ-ರುವ
ತೋರುತ್ತಿ-ರುವುದು
ತೋರುತ್ತೇನೆ
ತೋರುತ್ತೇನೆಂದು
ತೋರುವ
ತೋರು-ವಂತೆ
ತೋರುವನು
ತೋರು-ವರು
ತೋರು-ವರೊ
ತೋರು-ವ-ವನೂ
ತೋರು-ವಿ-ರೇನು
ತೋರು-ವು-ದಕ್ಕಾಗಿ
ತೋರು-ವು-ದಕ್ಕಿಂತ
ತೋರು-ವು-ದಕ್ಕೆ
ತೋರು-ವುದರ
ತೋರು-ವು-ದಿಲ್ಲ
ತೋರು-ವು-ದಿಲ್ಲ-ವೆಂಬು
ತೋರು-ವು-ದಿಲ್ಲವೋ
ತೋರು-ವುದು
ತೋರು-ವುದೆ
ತೋರು-ವುದೆಂದರೆ
ತೋರು-ವುದೇ
ತೋರು-ವುದೊ
ತೋರು-ವುವು
ತೋರು-ವೆವು
ತೋಳದ
ತೋಳ-ದಂತೆ
ತೋಳ-ದಷ್ಟು
ತೌರೂ-ರಾದ
ತೌರೂರಿ-ನಲ್ಲಿಯೂ
ತೌರೂರು
ತ್ಕಾರ
ತ್ಕಾರದ
ತ್ತದೆ
ತ್ತಲೂ
ತ್ತವೆ
ತ್ತಾನೆ
ತ್ತಾನೆಯೋ
ತ್ತಾರೆ
ತ್ತಾರೆಯೋ
ತ್ತಿತ್ತು
ತ್ತಿದೆ
ತ್ತಿದೆವ್ಯಕ್ತಿತ್ವವು
ತ್ತಿದ್ದು-ದನ್ನೂ
ತ್ತಿದ್ದೆಯೋ
ತ್ತಿದ್ದೇವೆ
ತ್ತಿರ-ಲಿಲ್ಲ
ತ್ತಿರುವ
ತ್ತಿರು-ವನು
ತ್ತಿರು-ವರು
ತ್ತಿರು-ವರೊ
ತ್ತಿರು-ವಾಗ
ತ್ತಿರು-ವಿರಿ
ತ್ತಿರು-ವು-ದನ್ನು
ತ್ತಿರು-ವು-ದಾಗಿ
ತ್ತಿರು-ವುದು
ತ್ತಿರು-ವುವು
ತ್ತಿರು-ವೆಯೋ
ತ್ತಿರು-ವೆವು
ತ್ತಿಲ್ಲ
ತ್ತಿವೆ
ತ್ತೀರಿ
ತ್ತೆಂಬು-ದನ್ನು
ತ್ತೇಜನ-ಕಾರಿ-ಯಾಗಿ-ರ-ಬೇಕು
ತ್ತೇನೆ
ತ್ತೇವೆ
ತ್ತೇವೆಯೊ
ತ್ತೇವೆಯೋ
ತ್ಮಕ-ವಾಗಿ
ತ್ಮನು
ತ್ಮಿಕ
ತ್ಯಜಿಸ-ಕೂಡದು
ತ್ಯಜಿ-ಸದೆ
ತ್ಯಜಿಸ-ಬಾ-ರದು
ತ್ಯಜಿಸ-ಬೇಕಾ-ಗಿಲ್ಲ
ತ್ಯಜಿಸ-ಬೇಕಾ-ಗುತ್ತದೆ
ತ್ಯಜಿಸ-ಬೇಕಾ-ದುದು
ತ್ಯಜಿಸ-ಬೇಕಾ-ಯಿತು
ತ್ಯಜಿ-ಸ-ಬೇಕು
ತ್ಯಜಿಸ-ಬೇಕೆಂದು
ತ್ಯಜಿಸ-ಬೇಡಿ
ತ್ಯಜಿಸ-ಲಾರಿರಿ
ತ್ಯಜಿಸ-ಲಾರೆವು
ತ್ಯಜಿ-ಸಲು
ತ್ಯಜಿ-ಸಲ್ಪಡುತ್ತವೆ
ತ್ಯಜಿಸಿ
ತ್ಯಜಿ-ಸಿತು
ತ್ಯಜಿ-ಸಿದ
ತ್ಯಜಿಸಿ-ದರೆ
ತ್ಯಜಿಸಿ-ದಾಗ
ತ್ಯಜಿಸಿ-ರುತ್ತದೆ
ತ್ಯಜಿಸಿ-ರು-ವರೊ
ತ್ಯಜಿಸಿ-ರುವೆ
ತ್ಯಜಿ-ಸಿಲ್ಲ
ತ್ಯಜಿಸು
ತ್ಯಜಿ-ಸುವ
ತ್ಯಜಿಸು-ವಂತೆ
ತ್ಯಜಿಸು-ವನು
ತ್ಯಜಿಸು-ವರು
ತ್ಯಜಿಸು-ವಷ್ಟು
ತ್ಯಜಿಸು-ವಾಗ
ತ್ಯಜಿಸು-ವಿಕೆಯು
ತ್ಯಜಿ-ಸುವು
ತ್ಯಜಿ-ಸು-ವು-ದಕ್ಕೆ
ತ್ಯಜಿಸು-ವು-ದ-ರಿಂದಲೇ
ತ್ಯಜಿ-ಸು-ವು-ದಿಲ್ಲ
ತ್ಯಜಿ-ಸು-ವು-ದಿಲ್ಲವೊ
ತ್ಯಜಿಸು-ವುದು
ತ್ಯಜಿಸು-ವು-ದೆಂದ-ರೇನು
ತ್ಯಜಿಸು-ವುದೊ
ತ್ಯಜಿಸು-ವೆನು
ತ್ಯಜಿಸು-ವೆವು
ತ್ಯಾಗ
ತ್ಯಾಗದ
ತ್ಯಾಗ-ದಲ್ಲಿ
ತ್ಯಾಗ-ದಲ್ಲಿದೆ
ತ್ಯಾಗ-ಮಾ-ಡಲಿ
ತ್ಯಾಗ-ಮಾಡಿ
ತ್ಯಾಗವು
ತ್ಯಾಗವೂ
ತ್ಯಾಗ-ವೆನ್ನು-ವುದು
ತ್ಯಾಗವೇ
ತ್ರಯಮಂತ-ರಂಗಂ
ತ್ರಯಮೇಕತ್ರ
ತ್ರಿಕಾಲಜ್ಞಾನ-ಗಳು
ತ್ರಿಕಾಲ-ದಲ್ಲಿಯೂ
ತ್ರಿಕಾಶ್ಥಿ
ತ್ರಿಕೋಣಾ
ತ್ರಿಕೋಣಾ-ಕಾರ-ವಾಗಿದೆ
ತ್ರಿಕೋಣಾ-ಕಾರ-ವಾದ
ತ್ವರಿ-ತ-ವಾಗಿ
ತ್ವರಿ-ತವೂ
ಥಾಮಸ್
ದಂಗೆ
ದಂಗೆ-ಗಾರ
ದಂಗೆ-ಗಾರರು
ದಂಟನ್ನೇ
ದಂಡ-ವನ್ನು
ದಂಡಿ
ದಂಡೆತ್ತಿ
ದಂತಾ-ಗಲಿ
ದಂತಿ-ರುತ್ತದೆ
ದಂತೆ
ದಂತೆಯೆ
ದಂತೆಯೇ
ದಂತೆಯೊ
ದಂತೆಲ್ಲಾ
ದಕ್ಕಾಗಿ
ದಕ್ಕಿಂತ
ದಕ್ಕು-ವುದೆಂದೇ
ದಕ್ಕೂ
ದಕ್ಕೆ
ದಕ್ಷಿಣ
ದಕ್ಷಿಣಾಯ-ನಕ್ಕೆ
ದಗ್ಧ-ವಾದ
ದಡಕ್ಕೆ
ದಡದ
ದಡ-ದಲ್ಲಿ
ದಡ-ಮೀರಿ
ದಡ್ಡ
ದಡ್ಡ-ನಾಗಿದ್ದನೆ
ದಡ್ಡ-ರಂತೆ
ದಡ್ಡರು
ದಡ್ಡರೋ
ದನ
ದನ-ಗಳು
ದನ-ಮಾಂಸ-ಭಕ್ಷಕ-ರಾದ
ದನು
ದನ್ನು
ದನ್ನೆಲ್ಲಾ
ದಪ್ಪ
ದಪ್ಪ-ನಾದ
ದಪ್ಪ-ವಾ-ಗಿ-ರುವ
ದಪ್ಪ-ವೆಂಬ
ದಬ್ಬಾಳಿ-ಕೆಗೆ
ದಬ್ಬು-ವರು
ದಯಾ
ದಯಾ-ಮಯ
ದಯಾ-ಮಯ-ನಾದ
ದಯಾ-ಮಯಿ-ಯಾದ
ದಯಾ-ಮೂರ್ತಿ
ದಯೆ
ದಯೆ-ತೋ-ರಲು
ದಯೆಯ
ದಯೆ-ಯನ್ನು
ದಯೆ-ಯಿಂದ
ದರ
ದರತ್ವಕ್ಕೆ
ದರಲ್ಲಿ
ದರಿಂದ
ದರಿದ್ರ-ರಾಗುತ್ತಾ
ದರಿದ್ರ-ರಿಗೆ
ದರು
ದರೂ
ದರೂ-ಮಾನ-ಸಿಕ
ದರೆ
ದರೊ-ಳಗೆ
ದರ್ಜೆಗೆ
ದರ್ಜೆಯ
ದರ್ಜೆ-ಯ-ದು-ಇಪ್ಪತ್ತು-ನಾಲ್ಕು
ದರ್ಜೆ-ಯಲ್ಲಿ
ದರ್ಪ
ದರ್ಶಕ
ದರ್ಶನ
ದರ್ಶನಕ್ಕಾಗಿ
ದರ್ಶ-ನಕ್ಕೆ
ದರ್ಶನ-ಗ-ಳನ್ನು
ದರ್ಶನ-ಗಳಲ್ಲಿ
ದರ್ಶನ-ಗಳಲ್ಲಿಯೂ
ದರ್ಶನ-ಗಳಾ-ಗಿದ್ದರೂ
ದರ್ಶನ-ಗಳಾ-ಗು-ವುವು
ದರ್ಶನ-ಗಳಿಗೆ
ದರ್ಶನ-ಗಳು
ದರ್ಶನದ
ದರ್ಶನ-ವನ್ನು
ದರ್ಶನ-ವಾಗುತ್ತಿದ್ದುದು
ದರ್ಶನ-ವಾ-ಗು-ವುದು
ದರ್ಶನವು
ದರ್ಶನವೂ
ದರ್ಶನ-ಶಾಸ್ತ್ರವು
ದರ್ಶನಾಲಬ್ಧ-ಭೂಮಿ-ಕತ್ವಾನವಸ್ಥಿ-ತತ್ತ್ವಾನಿ
ದರ್ಶಿಸಿ
ದರ್ಶಿ-ಸಿ-ದರು
ದಲ್ಲಿ
ದಲ್ಲಿ-ಡ-ಬೇಕು
ದಲ್ಲಿದೆ
ದಲ್ಲಿಯೂ
ದಲ್ಲಿಯೋ
ದಲ್ಲಿ-ರಲಿ
ದಲ್ಲಿ-ರುವ
ದಲ್ಲಿ-ರುವನು
ದಲ್ಲಿಲ್ಲ
ದಲ್ಲಿವೆ
ದಲ್ಲೆ
ದಲ್ಲೆಲ್ಲಾ
ದಳ-ಗಳೆಲ್ಲ
ದಳ-ಸ-ಹಿತ
ದಳು
ದವ-ಡೆ-ಯಲ್ಲಿ-ರು-ವಾ-ಗಲೂ
ದವ-ರಿಂದಲೇ
ದವರು
ದಶ-ದಿಕ್ಕು-ಗಳಲ್ಲಿ-ರುವನೋ
ದಷ್ಟು
ದಷ್ಟೂ
ದಹಿಸ
ದಹಿ-ಸದು
ದಹಿಸ-ಬಲ್ಲೆವು
ದಹಿಸ-ಲಾರದು
ದಹಿಸುತ್ತಿ-ರುವ
ದಹಿ-ಸು-ವುದು
ದಾಂಡಿ-ನಿಂದ
ದಾಂಡು
ದಾಕ್ಷಿಣಾತ್ಯ
ದಾಖಲೆ-ಗಳಲ್ಲಿ
ದಾಖಲೆ-ಗೊಳಿ-ಸಿ-ದರು
ದಾಗ
ದಾಗ-ಲೆಲ್ಲ
ದಾಗಿ
ದಾಗಿ-ರ-ಬೇಕು
ದಾಚೆ
ದಾಟಲು
ದಾಟಿ
ದಾಟಿ-ದರೆ
ದಾಟಿ-ಸುತ್ತದೆ
ದಾಟಿ-ಸು-ವುದು
ದಾಟು
ದಾಟುತ್ತೇನೆ
ದಾಟು-ವರು
ದಾಟು-ವು-ದಕ್ಕೆ
ದಾದ
ದಾದರೂ
ದಾದರೆ
ದಾದ-ರೊಂದು
ದಾದಿಗೆ
ದಾನ
ದಾನ-ದಿಂದಲೂ
ದಾನ-ಮಾಡದೆ
ದಾನ-ಮಾಡ-ಬೇಕಾ-ಗಿತ್ತು-ಆತ
ದಾನವ
ದಾನ-ವನ
ದಾನ-ವ-ನಿಗೆ
ದಾನ-ವನ್ನು
ದಾನ-ವನ್ನೂ
ದಾನ-ವಾಗಿ
ದಾನಿ
ದಾನಿಯ
ದಾಯಕ
ದಾಯ-ಗಳ
ದಾರ
ದಾರ-ಗಳು
ದಾರ-ವನ್ನು
ದಾರಿ
ದಾರಿ-ಗಳಲ್ಲಿಯೂ
ದಾರಿ-ಗಳು
ದಾರಿ-ಗಾಗಿ
ದಾರಿಗೆ
ದಾರಿ-ತೋ-ರಲು
ದಾರಿದ್ರ್ಯ
ದಾರಿಯ
ದಾರಿ-ಯನ್ನು
ದಾರಿ-ಯಲ್ಲಿ
ದಾರಿ-ಯಲ್ಲಿಯೂ
ದಾರಿ-ಯಲ್ಲೆ
ದಾರಿ-ಯಲ್ಲೇ
ದಾರಿ-ಯಾಗ-ಲಿಲ್ಲ-ವೇನು
ದಾರಿ-ಯಾ-ಗುತ್ತದೆ
ದಾರಿ-ಯಿಂದ
ದಾರಿ-ಯಿಲ್ಲದ
ದಾರಿಯೇ
ದಾರಿ-ಯೊಂದೇ
ದಾರ್ಢ್ಯವಿ-ರು-ವಾಗ
ದಾರ್ಶ-ನಿಕ-ನೊಬ್ಬ
ದಾರ್ಶ-ನಿಕರ
ದಾರ್ಶ-ನಿಕ-ರಲ್ಲಿ
ದಾರ್ಶ-ನಿಕ-ರಾ-ದರೆ
ದಾರ್ಶ-ನಿಕ-ರಿಗೆ
ದಾರ್ಶ-ನಿಕರು
ದಾಳ-ಗ-ಳನ್ನು
ದಾಳ-ಗಳಿವೆ
ದಾಳ-ದಂತೆ
ದಾಳ-ದಲ್ಲಿ
ದಾಸ-ನಾಗಿ
ದಾಸ-ರಲ್ಲ
ದಾಸ-ರಾಗು-ವರು
ದಾಸರಾ-ದುದು
ದಾಸರು
ದಾಹ-ಗಳ
ದಾಹ-ಗಳಿಲ್ಲ
ದಾಹ-ದಿಂದ
ದಾಹ-ವನ್ನು
ದಿಂದ
ದಿಂದಲೇ
ದಿಕ್ಕನ್ನು
ದಿಕ್ಕಿನ
ದಿಕ್ಕಿ-ನಲ್ಲಿ
ದಿಕ್ಕಿ-ನಿಂದಲೂ
ದಿಕ್ಕು-ಗಳಿಗೆ
ದಿಕ್ಕು-ಗಳು
ದಿಕ್ಕು-ಗಳೂ
ದಿಗ್ಬ್ರಮೆಯುಂಟಾಗ-ಬಹುದು
ದಿಗ್ಭ್ರಾಂತ-ರಾ-ಗುತ್ತಾರೆ
ದಿಗ್ವಿ-ಜಯದ
ದಿಗ್ವಿ-ಜಯ-ವನ್ನು
ದಿಣ್ಣೆ-ಗಳಿಲ್ಲ
ದಿದ್ದರೂ
ದಿದ್ದರೆ
ದಿನ
ದಿನಂಪ್ರತಿ
ದಿನ-ಕಳೆ
ದಿನಕ್ಕೆ
ದಿನಕ್ರಮೇಣ
ದಿನ-ಗಟ್ಟಲೆ
ದಿನ-ಗಳ
ದಿನ-ಗಳಲ್ಲಿ
ದಿನ-ಗಳಲ್ಲೆಲ್ಲ
ದಿನ-ಗಳ-ವರೆಗೆ
ದಿನ-ಗಳಾ
ದಿನ-ಗ-ಳಾದ
ದಿನ-ಗಳು
ದಿನ-ಚರಿ-ಯಂತೆ
ದಿನದ
ದಿನ-ದಲ್ಲಿ
ದಿನ-ದಿಂದ
ದಿನ-ದಿಂದಲೇ
ದಿನ-ದಿನವೂ
ದಿನವು
ದಿನವೂ
ದಿನ-ವೆಲ್ಲ
ದಿನ-ವೆಲ್ಲಾ
ದಿನೇ-ದಿನೇ
ದಿರು-ವರು
ದಿಲ್ಲ
ದಿವ್ಯ
ದಿವ್ಯತೆ
ದಿವ್ಯ-ತೆ-ಯನ್ನು
ದಿವ್ಯ-ತೆ-ವನ್ನು
ದಿವ್ಯ-ದರ್ಶನದ
ದಿವ್ಯಪ್ರಭೆಯ
ದಿವ್ಯ-ವಾ-ದುದು
ದೀನ
ದೀನ-ಮಾನ-ವನು
ದೀನ-ರಾ-ಗಿ-ರು-ವೆವು
ದೀನ-ರಿಗೆ
ದೀನ-ವ-ರಿಗೆ
ದೀಪ
ದೀಪಕ್ಕೆ
ದೀಯರ
ದೀರ್ಘ
ದೀರ್ಘ-ಕಾಲ
ದೀರ್ಘ-ಕಾಲ-ದಿಂದ
ದೀರ್ಘ-ಕಾಲ-ನೈರಂತರ್ಯಸತ್ಕಾರಾಸೇವಿತೋ
ದೀರ್ಘ-ಕಾಲ-ವಿರ-ಲಾರೆನು
ದೀರ್ಘ-ಕಾಲವೂ
ದೀರ್ಘ-ಜೀವನ
ದೀರ್ಘತೆ
ದೀರ್ಘಧ್ಯಾನ
ದೀರ್ಘ-ವರು-ಷ-ಗಳಾ-ಗಿದ್ದವು
ದೀರ್ಘವ್ಯಾ-ಸಂಗ
ದೀರ್ಘ-ಸಾಧನೆ-ಯಿಂದ
ದೀರ್ಘ-ಸೂಕ್ಷ್ಮಃ
ದೀರ್ಘಾಯುಷಿ-ಯನ್ನಾಗಿ
ದೀರ್ಘಾಯುಷಿ-ಯಾಗಿ
ದೀರ್ಘಾಯುಸ್ಸು
ದುಃಖ
ದುಃಖ-ಕರ
ದುಃಖ-ಕಾರ-ಕವೂ
ದುಃಖಕ್ಕಿಂತಲೂ
ದುಃಖಕ್ಕೂ
ದುಃಖಕ್ಕೆ
ದುಃಖಕ್ಕೆಲ್ಲ
ದುಃಖಕ್ಕೆಲ್ಲಾ
ದುಃಖಕ್ಷೇತ್ರವೂ
ದುಃಖ-ಗಳ
ದುಃಖ-ಗಳಿಂದ
ದುಃಖ-ಗಳಿಗೂ
ದುಃಖ-ಗಳಿ-ಗೆಲ್ಲಾ
ದುಃಖ-ಗಳಿವೆ
ದುಃಖ-ಗಳೆಂಬ
ದುಃಖ-ಗಳೆ-ರಡಕ್ಕೂ
ದುಃಖ-ಗಳೆಲ್ಲ
ದುಃಖ-ಗಳೆಲ್ಲಾ
ದುಃಖ-ಗೊಂಡು
ದುಃಖದ
ದುಃಖ-ದಂತೆ
ದುಃಖ-ದರ್ಶನ
ದುಃಖ-ದಲ್ಲಿ
ದುಃಖ-ದಲ್ಲಿಯೂ
ದುಃಖ-ದಲ್ಲಿರ
ದುಃಖ-ದಲ್ಲಿ-ರುವ-ವ-ರಿಗೆ
ದುಃಖ-ದಲ್ಲಿ-ರು-ವೆವು
ದುಃಖ-ದಾಯ-ಕವೇ
ದುಃಖ-ದಿಂದ
ದುಃಖ-ದಿಂದಲೂ
ದುಃಖ-ದೌರ್ಮನಸ್ಯಾಂಗಮೇ-ಜಯತ್ವಶ್ವಾಸಪ್ರಶ್ವಾಸಾ
ದುಃಖ-ನೋವು-ಗಳು
ದುಃಖ-ಪಡು-ವು-ದಕ್ಕೆ
ದುಃಖ-ಪಡು-ವುದೂ
ದುಃಖ-ಪುಣ್ಯಾ-ಪುಣ್ಯ-ವಿಷ-ಯಾಣಾಂ
ದುಃಖ-ಮ-ನಾಗ-ತಮ್
ದುಃಖ-ಮಯ
ದುಃಖ-ಮಯ-ವನ್ನಾಗಿ
ದುಃಖ-ಮಯ-ವಾ-ಗಿ-ರು-ವುವು
ದುಃಖ-ಮ-ಯವೂ
ದುಃಖ-ಮೇವ
ದುಃಖ-ವನ್ನು
ದುಃಖ-ವನ್ನೂ
ದುಃಖ-ವನ್ನೆಲ್ಲ
ದುಃಖ-ವಾಗಿದೆ
ದುಃಖ-ವಾಗಿದ್ದರೂ
ದುಃಖ-ವಾಗುತ್ತಿದೆ
ದುಃಖ-ವಾ-ಗು-ವುದು
ದುಃಖ-ವಿದೆ
ದುಃಖ-ವಿನ್ನೆಂದಿಗೂ
ದುಃಖ-ವಿಲ್ಲದ
ದುಃಖ-ವಿಲ್ಲ-ದಿ-ರುವ
ದುಃಖವು
ದುಃಖವೂ
ದುಃಖ-ವೆಂದ-ರೇನು
ದುಃಖ-ವೆನ್ನ-ಬಹುದು
ದುಃಖ-ವೆಲ್ಲ
ದುಃಖ-ವೆಲ್ಲಾ
ದುಃಖ-ವೆಲ್ಲಿ-ಯದು
ದುಃಖವೇ
ದುಃಖ-ಶೋಕಾ-ತೀತ
ದುಃಖಾಜ್ಞಾನಾ-ನಂತ-ಫಲಾ
ದುಃಖಾನುಶಯೀ
ದುಃಖಿ-ಗಳ
ದುಃಖಿ-ಗಳನ್ನಾಗಿ
ದುಃಖಿ-ಗಳಾ-ಗಿ-ರು-ವು-ದಕ್ಕೆ
ದುಃಖಿ-ಗಳಾ-ಗಿ-ರು-ವೆವು
ದುಃಖಿ-ಗಳು
ದುಃಖಿ-ಗ-ಳೆಂದು
ದುಃಖಿ-ಯಂತೆ
ದುಃಖಿ-ಯಾಗಿಯೂ
ದುಃಖಿ-ಯಾಗುತ್ತಾನೆ
ದುಃಖಿ-ಯಾಗುತ್ತೇನೆ
ದುಃಖಿ-ಯೆಂದು
ದುಃಖಿ-ಯೆಂದೂ
ದುಃಖಿಸ-ಬೇಡಿ
ದುಃಖಿ-ಸುತ್ತದೆ
ದುಃಖಿ-ಸು-ವನು
ದುಃಖಿ-ಸುವ-ವ-ರಲ್ಲಿ
ದುಃಖಿ-ಸು-ವು-ದಕ್ಕೆ
ದುಃಖಿ-ಸು-ವು-ದಿಲ್ಲ
ದುಃಖಿ-ಸು-ವುದು
ದುಃಸ್ಥಿತಿ-ಯನ್ನು
ದುಃಸ್ಥಿತಿ-ಯಲ್ಲಿದ್ದರೂ
ದುಡಿ-ತಕ್ಕೆ
ದುಡಿ-ದನು
ದುಡಿ-ದಿರು
ದುಡಿ-ದಿರು-ವಷ್ಟು
ದುಡಿ-ದಿಲ್ಲ
ದುಡಿ-ಯುತ್ತಿದ್ದರು
ದುದು
ದುರ
ದುರತ್ಯಯಾ
ದುರ-ದೃಷ್ಟ-ಕರ
ದುರ-ದೃಷ್ಟದ
ದುರ-ದೃಷ್ಟವಶಾತ್
ದುರಾತ್ಮನು
ದುರಿಗೆ
ದುರು-ಪ-ಯೋಗ-ವನ್ನು
ದುರ್ಜ-ನರ
ದುರ್ದೆಸೆ
ದುರ್ಬ
ದುರ್ಬಲ
ದುರ್ಬ-ಲ-ಕಾರಕ
ದುರ್ಬ-ಲ-ಕಾರಿ-ಯಾದ
ದುರ್ಬ-ಲ-ಗೊಳಿ-ಸುತ್ತವೆ
ದುರ್ಬ-ಲ-ಗೊಳಿ-ಸು-ವಿರಿ
ದುರ್ಬ-ಲ-ಗೊಳಿ-ಸು-ವುದು
ದುರ್ಬ-ಲ-ಗೊಳಿ-ಸು-ವುದೊ
ದುರ್ಬ-ಲತೆ
ದುರ್ಬ-ಲ-ತೆ-ಗಳು
ದುರ್ಬ-ಲ-ತೆಗೂ
ದುರ್ಬ-ಲ-ತೆಗೆ
ದುರ್ಬ-ಲ-ತೆಯ
ದುರ್ಬ-ಲ-ತೆ-ಯನ್ನು
ದುರ್ಬ-ಲ-ತೆಯೇ
ದುರ್ಬ-ಲನು
ದುರ್ಬ-ಲ-ರನ್ನಾಗಿ
ದುರ್ಬ-ಲ-ರಾಗಿ-ರು-ವವ-ರನ್ನು
ದುರ್ಬ-ಲ-ರಾಗು-ವುರು
ದುರ್ಬ-ಲ-ರಾದ
ದುರ್ಬ-ಲ-ರಾದು
ದುರ್ಬ-ಲ-ರಾದು-ದ-ರಿಂದ
ದುರ್ಬ-ಲ-ರಿಗೆ
ದುರ್ಬ-ಲರು
ದುರ್ಬ-ಲ-ರೆಂದು
ದುರ್ಬ-ಲರೊ
ದುರ್ಬ-ಲ-ವಾಗಿ
ದುರ್ಬ-ಲ-ವಾಗಿದೆ
ದುರ್ಬ-ಲ-ವಾಗಿಯೂ
ದುರ್ಬ-ಲ-ವಾ-ಗು-ವುದು
ದುರ್ಬ-ಲ-ವಾದ
ದುರ್ಬ-ಲ-ವಾ-ದರೂ
ದುರ್ಬ-ಲ-ವಾ-ದಾಗ
ದುರ್ಬ-ಲ-ವಾದುದು
ದುರ್ಭಾಗ್ಯನು
ದುರ್ಮಾರ್ಗ
ದುರ್ಲಭ-ವಾ-ಗು-ವುದು
ದುರ್ವಿಷ-ಯ-ಗಳಿ-ಗಿಂತ
ದುಷ್ಕರ್ಮಿ-ಗಳೋ
ದುಷ್ಟ
ದುಷ್ಟ-ತನ
ದುಷ್ಟನ
ದುಷ್ಟ-ನಿಗೆ
ದುಷ್ಟನು
ದುಷ್ಟ-ಪೂ-ಜಾರಿ-ಗಳ
ದುಷ್ಟ-ರನ್ನು
ದುಷ್ಟ-ರನ್ನೂ
ದುಷ್ಟ-ರಾ-ಗಿ-ರು-ವರೊ
ದುಷ್ಟರು
ದುಷ್ಟ-ವಾ-ದವು
ದುಸ್ಸಂಪ್ರ-ದಾಯ
ದುಸ್ಸಹನೀಯ
ದುಸ್ಸಾಧ್ಯ
ದುಸ್ಸಾಧ್ಯ-ವಾಗಿದೆ
ದೂಡೋಣ
ದೂರ
ದೂರಕ್ಕೆ
ದೂರದ
ದೂರ-ದಕ್ಕಿಂತ
ದೂರ-ದರ್ಶಕ
ದೂರ-ದರ್ಶಕ-ಯಂತ್ರದ
ದೂರ-ದಲ್ಲಿ
ದೂರ-ದಲ್ಲಿದೆ
ದೂರ-ದಲ್ಲಿದ್ದಾನೆ
ದೂರ-ದಲ್ಲಿ-ರುವ
ದೂರ-ದಲ್ಲಿ-ರುವುದು
ದೂರ-ದ-ವರೆಗೆ
ದೂರ-ದಿಂದ
ದೂರ-ಬೇಡಿ
ದೂರ-ಮಾಡ-ಬಹುದು
ದೂರ-ಮಾಡ-ಬೇಕು
ದೂರ-ವಾ-ಗನು
ದೂರ-ವಾಗಿ
ದೂರ-ವಾಗಿ-ರಲು
ದೂರ-ವಾಗಿರಿ
ದೂರ-ವಾ-ಗಿ-ರು-ವರು
ದೂರ-ವಾ-ಗಿ-ರು-ವರೊ
ದೂರ-ವಾಗಿವೆ
ದೂರ-ವಾಗು-ವನು
ದೂರ-ವಾ-ಗು-ವುವು
ದೂರ-ವಾದ
ದೂರ-ವಾ-ದಷ್ಟೂ
ದೂರ-ವಿದ್ದಷ್ಟೂ
ದೂರ-ವಿ-ರ-ಬೇಕು
ದೂರ-ವಿರಿ
ದೂರ-ವೆಂಬು-ದಿಲ್ಲ
ದೂರಿ
ದೂರಿ-ಕೊಳ್ಳ-ಬೇಕು
ದೂರಿ-ಕೊಳ್ಳು-ವುದು
ದೂರು
ದೂರುತ್ತ
ದೂರುತ್ತಾನೆ
ದೂರುತ್ತಾರೆ
ದೂರುವ
ದೂರು-ವರೊ
ದೂರು-ವುದಕ್ಕಾಗು-ವು-ದಿಲ್ಲ
ದೂರು-ವು-ದಕ್ಕಿಂತ
ದೂರು-ವು-ದಕ್ಕೆ
ದೂರು-ವು-ದ-ರಲ್ಲಿ
ದೂರು-ವು-ದಿಲ್ಲ
ದೂರು-ವುದು
ದೂರು-ವು-ದೊಂದು
ದೂಷಿತ-ನಾಗು-ವು-ದಿಲ್ಲ
ದೂಷಿ-ಸಲ್ಪಡು-ವುದೆಂದರೇ-ನೆಂದೂ
ದೂಸರ
ದೃಕ್
ದೃಗ್
ದೃಗ್ದರ್ಶನ-ಶಕ್ತೋರೇಕಾತ್ಮತೈ-ವಾಸ್ಮಿತಾ
ದೃಗ್ದೃಶ್ಯ-ಗಳೆಂಬ
ದೃಢ
ದೃಢ-ಕಾಯ-ನಾದ
ದೃಢ-ಕಾಯ-ರಾಗಿ
ದೃಢ-ಕಾಯ-ವಾ-ದರೆ
ದೃಢತೆ
ದೃಢ-ನಂಬಿಕೆ
ದೃಢಪ್ರ-ಯತ್ನ-ದಲ್ಲಿ
ದೃಢ-ಭೂಮಿಃ
ದೃಢ-ವಾಗ-ಬಲ್ಲದು
ದೃಢ-ವಾಗಿ
ದೃಢ-ವಾ-ಗಿತ್ತು
ದೃಢ-ವಾಗಿ-ದೆಯೊ
ದೃಢ-ವಾಗಿದ್ದರೆ
ದೃಢ-ವಾಗಿದ್ದು
ದೃಢ-ವಾ-ಗಿ-ರು-ವರೊ
ದೃಢ-ವಾ-ಗಿ-ರುವುದು
ದೃಢ-ವಾಗಿ-ರು-ವು-ದೆಂದು
ದೃಢ-ವಾಗಿವೆ
ದೃಢ-ವಾಗುತ್ತ
ದೃಢ-ವಾಗುತ್ತದೆ
ದೃಢ-ಸಂಕಲ್ಪ
ದೃಶ್ಯ
ದೃಶ್ಯಕ್ಕಾಗಿ
ದೃಶ್ಯ-ಗಳ
ದೃಶ್ಯ-ಗ-ಳನ್ನು
ದೃಶ್ಯ-ಗಳಾಗಿ
ದೃಶ್ಯ-ಗಳಿಂದ
ದೃಶ್ಯ-ಗಳು
ದೃಶ್ಯತ್ವಾತ್
ದೃಶ್ಯದ
ದೃಶ್ಯಮ್
ದೃಶ್ಯ-ವನ್ನು
ದೃಶ್ಯ-ವನ್ನೇ
ದೃಶ್ಯ-ವನ್ನೊ
ದೃಶ್ಯ-ವಲ್ಲದೆ
ದೃಶ್ಯ-ವಾಗಿ-ರು-ವು-ದ-ರಿಂದ
ದೃಶ್ಯ-ವಿಲ್ಲ
ದೃಶ್ಯವು
ದೃಶ್ಯ-ವೆಲ್ಲ
ದೃಶ್ಯವೇ
ದೃಶ್ಯಸ್ಯಾತ್ಮಾ
ದೃಶ್ಯೇಂದ್ರಿ-ಯವು
ದೃಶ್ಯೋಪ-ರಕ್ತಂ
ದೃಷ್ಟವಶ-ದಿಂದ
ದೃಷ್ಟಾಂತ
ದೃಷ್ಟಾದೃಷ್ಟ-ಜನ್ಮ-ವೇ-ದನೀಯಃ
ದೃಷ್ಟಿ
ದೃಷ್ಟಿ-ಗ-ಳನ್ನೂ
ದೃಷ್ಟಿ-ಗಳನ್ನೆಲ್ಲ
ದೃಷ್ಟಿ-ಗಳಲ್ಲಿ
ದೃಷ್ಟಿ-ಗಳು
ದೃಷ್ಟಿಗೆ
ದೃಷ್ಟಿ-ಪಟುತ್ವ-ವನ್ನು
ದೃಷ್ಟಿ-ಭೇದ
ದೃಷ್ಟಿಯ
ದೃಷ್ಟಿ-ಯನ್ನು
ದೃಷ್ಟಿ-ಯಲ್ಲಿ
ದೃಷ್ಟಿ-ಯಲ್ಲೆ
ದೃಷ್ಟಿ-ಯ-ವನು
ದೃಷ್ಟಿ-ಯ-ವ-ರಿ-ಗೆಲ್ಲಾ
ದೃಷ್ಟಿ-ಯ-ವ-ರೆಂದು
ದೃಷ್ಟಿ-ಯಿಂದ
ದೃಷ್ಟಿ-ಯಿಂದ-ಲಾ-ದರೂ
ದೃಷ್ಟಿ-ಯಿಂದಲೂ
ದೃಷ್ಟಿಯು
ದೃಷ್ಟಿ-ಯುಳ್ಳ-ವನು
ದೃಷ್ಟಿ-ಯುಳ್ಳ-ವರು
ದೃಷ್ಟಿಯೆ
ದೃಷ್ಟಿ-ಯೆಲ್ಲ
ದೃಷ್ಟಿಯೇ
ದೃಷ್ಟಿ-ವಾದಿಗೆ
ದೆಂದಿತು
ದೆಂದು
ದೆಂಬುದು
ದೆಂಬುದೆ
ದೆಯೊ
ದೆಯೋ
ದೆಲ್ಲ
ದೆಲ್ಲ-ವನ್ನೂ
ದೆಲ್ಲಾ
ದೆಲ್ಲಿಗೆ
ದೆವ್ವ
ದೆವ್ವ-ಗಳಿಲ್ಲ-ವೆಂದು
ದೆವ್ವದ
ದೆವ್ವವೂ
ದೆವ್ವ-ವೆಂದು
ದೆಸೆ
ದೆಸೆ-ಯಿಂದ
ದೇಗುಲ
ದೇಗುಲ-ದಲ್ಲಿ-ರುವ
ದೇಗುಲವೇ
ದೇದೀಪ್ಯ-ಮಾನ
ದೇವ
ದೇವ-ಋಷಿ-ಗಳಿದ್ದರು
ದೇವ-ಗುರು-ಗಳು
ದೇವತಾ
ದೇವತೆ
ದೇವತೆ-ಗಳ
ದೇವತೆ-ಗಳಂತೆಯೇ
ದೇವತೆ-ಗಳದ್ದು
ದೇವತೆ-ಗಳನ್ನು
ದೇವತೆ-ಗಳನ್ನೂ
ದೇವತೆ-ಗಳಲ್ಲಿ
ದೇವತೆ-ಗಳಲ್ಲಿಯೂ
ದೇವತೆ-ಗಳ-ವರೆಗೆ
ದೇವತೆ-ಗಳಾಗಿ
ದೇವತೆ-ಗಳಾ-ಗಿ-ರಲೀ
ದೇವತೆ-ಗಳಾ-ಗುತ್ತಾರೆ
ದೇವತೆ-ಗಳಾ-ಗು-ವರು
ದೇವತೆ-ಗಳಾ-ಗು-ವಿರಿ
ದೇವತೆ-ಗಳಾ-ಗು-ವುದು
ದೇವತೆ-ಗಳಾ-ದರು
ದೇವತೆ-ಗಳಿಂದ
ದೇವತೆ-ಗಳಿ-ಗಿಂತ
ದೇವತೆ-ಗಳಿ-ಗಿಂತಲೂ
ದೇವತೆ-ಗಳಿಗೂ
ದೇವತೆ-ಗಳಿಗೆ
ದೇವತೆ-ಗಳಿಗೇ
ದೇವತೆ-ಗಳಿದ್ದರೆ
ದೇವತೆ-ಗಳು
ದೇವತೆ-ಗಳೂ
ದೇವತೆ-ಗಳೆಂದರೆ
ದೇವತೆ-ಗ-ಳೆಂದು
ದೇವತೆ-ಗಳೆಂಬ
ದೇವತೆ-ಗಳೆಲ್ಲ
ದೇವತೆ-ಗಳೆ-ಲ್ಲರೂ
ದೇವತೆ-ಗಳೊ
ದೇವತೆ-ಗಳೊಬ್ಬ-ರಿಂದ
ದೇವತೆ-ಗಳೋ
ದೇವತೆ-ಯಾಗು-ತ್ತಾನೆ
ದೇವತೆಯೂ
ದೇವತ್ವದ
ದೇವ-ದೂತ
ದೇವ-ದೂತ-ನಂತೆ
ದೇವ-ದೂತ-ನನ್ನು
ದೇವ-ದೂತ-ನಾಗಿ
ದೇವ-ದೂ-ತನು
ದೇವ-ದೂತನೇ
ದೇವ-ದೂತ-ನೊಬ್ಬ-ನಿಂದ
ದೇವ-ದೂತರ
ದೇವ-ದೂತ-ರನ್ನು
ದೇವ-ದೂತ-ರಾ-ಗಲೀ
ದೇವ-ದೂತ-ರಿಂದ
ದೇವ-ದೂತ-ರಿಗೆ
ದೇವ-ದೂತರು
ದೇವ-ದೂತರೊ
ದೇವ-ದೇವ-ತೆ-ಗ-ಳನ್ನು
ದೇವ-ದೇವ-ತೆ-ಗಳು
ದೇವ-ದೇವ-ನೆಂದು
ದೇವನ
ದೇವ-ನದು
ದೇವ-ನನ್ನು
ದೇವ-ನಾಗ-ಲಾರ
ದೇವ-ನಾ-ಗಲು
ದೇವ-ನಾಗಿ
ದೇವ-ನಾಗಿ-ರುವನು
ದೇವ-ನಾಗು-ವನು
ದೇವ-ನಾ-ದರೋ
ದೇವ-ನಾ-ದಾಗ
ದೇವನಿ
ದೇವನು
ದೇವನೂ
ದೇವ-ಮಾನ-ವ-ರಿಗೆ
ದೇವ-ಯಾನ
ದೇವ-ಯಾನದ
ದೇವರ
ದೇವ-ರಂತೆ
ದೇವ-ರಂತೆಯೆ
ದೇವ-ರದು
ದೇವ-ರದೇ
ದೇವ-ರನ್ನಾಗಿ
ದೇವ-ರನ್ನು
ದೇವ-ರನ್ನೂ
ದೇವ-ರನ್ನೇ
ದೇವ-ರಲ್ಲದ
ದೇವ-ರಲ್ಲದೆ
ದೇವ-ರಲ್ಲಿ
ದೇವ-ರಲ್ಲೋ
ದೇವ-ರಾಗ-ಬೇಕಾ-ದರೆ
ದೇವ-ರಾಗಲಿ
ದೇವ-ರಾಗಿರ-ಲಿಲ್ಲ
ದೇವ-ರಾಗಿರು-ವುದು
ದೇವ-ರಾಗು-ವನು
ದೇವ-ರಾಗು-ವವನೂ
ದೇವ-ರಾಗು-ವುದು
ದೇವ-ರಾಗು-ವುದು-ಇದು
ದೇವ-ರಾದ
ದೇವ-ರಾಯಿತು
ದೇವ-ರಿಂದ
ದೇವರಿ-ಗಾಗಿ
ದೇವರಿ-ಗಿಂತಲೂ
ದೇವ-ರಿಗೂ
ದೇವ-ರಿಗೆ
ದೇವರಿ-ಗೆಲ್ಲ
ದೇವ-ರಿಗೋಸ್ಕರ
ದೇವ-ರಿದ್ದನು
ದೇವ-ರಿದ್ದರೆ
ದೇವ-ರಿರ-ಲಾರ
ದೇವ-ರಿರುವ
ದೇವ-ರಿರು-ವನು
ದೇವ-ರಿರುವ-ನು-ಅವನೆ
ದೇವ-ರಿರು-ವನೆ
ದೇವ-ರಿರುವ-ನೆಂದು
ದೇವ-ರಿಲ್ಲ
ದೇವ-ರಿಲ್ಲದ
ದೇವರು
ದೇವರು-ಗಳನ್ನು
ದೇವರು-ಗಳಿರು-ಗಳಾ-ಗುತ್ತೇವೆ
ದೇವರು-ಗಳಿ-ರುತ್ತಾರೆ
ದೇವರು-ಗಳು
ದೇವರು-ವಿಶ್ವದ
ದೇವರೂ
ದೇವ-ರೂಪ-ದಲ್ಲಿವೆ
ದೇವರೆ
ದೇವ-ರೆಂದು
ದೇವ-ರೆಂದೂ
ದೇವ-ರೆಂಬ
ದೇವ-ರೆಂಬುದು
ದೇವ-ರೆಂಬು-ವನು
ದೇವ-ರೆಡೆಗೆ
ದೇವ-ರೆದು-ರಿಗೆ
ದೇವ-ರೆಲ್ಲ-ದ-ರಲ್ಲಿ
ದೇವ-ರೆಲ್ಲಿ
ದೇವರೇ
ದೇವರೊ
ದೇವ-ರೊಂದಿಗೆ
ದೇವ-ರೊಂದು
ದೇವ-ರೊಬ್ಬ-ನಿಗೆ
ದೇವ-ರೊಬ್ಬನು
ದೇವ-ರೊಬ್ಬನೆ
ದೇವರೋ
ದೇವ-ಲೋ-ಕಕ್ಕೆ
ದೇವ-ಲೋಕದ
ದೇವ-ವಾಣಿ
ದೇವಸ್ಥಾನ
ದೇವಸ್ಥಾ-ನಕ್ಕೆ
ದೇವಸ್ಥಾನ-ಗ-ಳನ್ನು
ದೇವಸ್ಥಾನ-ಗಳಿಂದಲೂ
ದೇವಸ್ಥಾನದ
ದೇವಸ್ಥಾನ-ದಲ್ಲಿಲ್ಲ
ದೇವಸ್ಥಾನ-ದಿಂದ
ದೇವಾಂಶ
ದೇವಾಲಯ
ದೇವಾಸುರರು
ದೇವೇಂದ್ರ
ದೇವೇಂದ್ರ-ನಾಗು-ವನು
ದೇವೋಕ್ತ
ದೇಶ
ದೇಶ-ಇವು-ಗಳಿಂದ
ದೇಶ-ಕಾಲ
ದೇಶ-ಕಾಲ-ಕಾರ್ಯ
ದೇಶ-ಕಾಲ-ಗಳಲ್ಲಿಯೂ
ದೇಶ-ಕಾಲ-ಸಂಖ್ಯಾಭಿಃ
ದೇಶ-ಕಾಲ-ಸಂಖ್ಯೆ-ಗಳಿಂದ
ದೇಶಕ್ಕೂ
ದೇಶಕ್ಕೆ
ದೇಶ-ಗಳ
ದೇಶ-ಗಳ-ಕಾರ್ಯ
ದೇಶ-ಗ-ಳನ್ನು
ದೇಶ-ಗಳಲ್ಲಿ
ದೇಶ-ಗಳಲ್ಲಿಯೂ
ದೇಶ-ಗಳಲ್ಲಿ-ರುವ
ದೇಶ-ಗಳಿಂದ
ದೇಶ-ಗಳಿ-ಗಿಂತ
ದೇಶ-ಗಳಿಗೂ
ದೇಶ-ಗಳು
ದೇಶ-ಗಳೆಲ್ಲ
ದೇಶ-ಗಳೊಂದಿಗೆ
ದೇಶದ
ದೇಶ-ದಲ್ಲಾ
ದೇಶ-ದಲ್ಲಾ-ಗಲಿ
ದೇಶ-ದಲ್ಲಿ
ದೇಶ-ದಲ್ಲಿಯೂ
ದೇಶ-ದಲ್ಲಿಯೇ
ದೇಶ-ದಲ್ಲಿ-ರುವ
ದೇಶ-ದಲ್ಲಿ-ರು-ವಂತೆ
ದೇಶ-ದಲ್ಲಿ-ರುವ-ವರು
ದೇಶ-ದಲ್ಲಿ-ರುವುದು
ದೇಶ-ದಲ್ಲೂ
ದೇಶ-ದಲ್ಲೆ
ದೇಶ-ದಲ್ಲೆಲ್ಲಾ
ದೇಶ-ದ-ವ-ನೊಬ್ಬ
ದೇಶ-ದ-ವ-ರಿಗೆ
ದೇಶ-ದ-ವರು
ದೇಶ-ದಿಂದ
ದೇಶ-ಬಂಧಶ್ಚಿತ್ತಸ್ಯ
ದೇಶ-ಭಕ್ತಿ
ದೇಶ-ಭಕ್ತಿಯ
ದೇಶ-ವನ್ನು
ದೇಶ-ವನ್ನೂ
ದೇಶ-ವನ್ನೆ
ದೇಶ-ವಾ-ಗಲಿ
ದೇಶ-ವಾಗಿ
ದೇಶ-ವಿದೆ
ದೇಶವು
ದೇಶವೂ
ದೇಶ-ವೆಂದರೆ
ದೇಶ-ವೆಂದ-ರೇ-ನೆಂಬುದು
ದೇಶವೇ
ದೇಶಾ-ತೀತ-ವಾದ
ದೇಹ
ದೇಹ-ಕಂಪನ
ದೇಹ-ಕಣ-ಗಳ
ದೇಹಕ್ಕಿಂತ
ದೇಹಕ್ಕೂ
ದೇಹಕ್ಕೆ
ದೇಹಕ್ಕೋ
ದೇಹ-ಗಳ
ದೇಹ-ಗ-ಳನ್ನು
ದೇಹ-ಗಳಲ್ಲಿ
ದೇಹ-ಗಳಲ್ಲೇ
ದೇಹ-ಗಳಿ-ಗಿಂತ
ದೇಹ-ಗಳಿ-ಗಿಂತಲೂ
ದೇಹ-ಗಳಿಗೆ
ದೇಹ-ಗಳಿದ್ದು
ದೇಹ-ಗಳಿ-ರು-ವುವು
ದೇಹ-ಗಳಿವೆ
ದೇಹ-ಗಳು
ದೇಹ-ಗಳೂ
ದೇಹ-ಗಳೆಂದೂ
ದೇಹ-ಗಳೆಲ್ಲ
ದೇಹ-ಚ-ಲ-ನೆಗೆ
ದೇಹದ
ದೇಹ-ದಂಡನೆ-ಗ-ಳನ್ನು
ದೇಹ-ದಂತೆ
ದೇಹ-ದಲ್ಲಿ
ದೇಹ-ದಲ್ಲಿದ್ದರೆ
ದೇಹ-ದಲ್ಲಿದ್ದು
ದೇಹ-ದಲ್ಲಿ-ಯಾ-ಗಲೀ
ದೇಹ-ದಲ್ಲಿಯೂ
ದೇಹ-ದಲ್ಲಿಯೇ
ದೇಹ-ದಲ್ಲಿಯೋ
ದೇಹ-ದಲ್ಲಿ-ರ-ಬಹುದು
ದೇಹ-ದಲ್ಲಿ-ರಲು
ದೇಹ-ದಲ್ಲಿ-ರುವ
ದೇಹ-ದಲ್ಲಿ-ರುವನು
ದೇಹ-ದಲ್ಲಿ-ರುವ-ವರು
ದೇಹ-ದಲ್ಲಿ-ರುವಾ-ಗಲೇ
ದೇಹ-ದಲ್ಲಿವೆ
ದೇಹ-ದಲ್ಲೂ
ದೇಹ-ದಲ್ಲೆ
ದೇಹ-ದಲ್ಲೆಲ್ಲ
ದೇಹ-ದಲ್ಲೆಲ್ಲಾ
ದೇಹ-ದಾರಿ-ಯಾಗಿ
ದೇಹ-ದಿಂದ
ದೇಹ-ದಿಂದಲೇ
ದೇಹ-ದೊಂದಿಗೆ
ದೇಹ-ದೊ-ಳಗೆ
ದೇಹ-ಧಾರಣ
ದೇಹ-ಧಾರಣೆ
ದೇಹ-ಧಾರಿ
ದೇಹ-ಪೋಷಕ
ದೇಹ-ಪೋಷ-ಣೆಗೆ
ದೇಹ-ಬಲ
ದೇಹ-ಬುದ್ಧಿ
ದೇಹ-ಭಾವ-ನೆ-ಯನ್ನು
ದೇಹ-ಭಾವ-ನೆ-ಯಿಂದ
ದೇಹ-ಮನಸ್ಸು-ಗಳ
ದೇಹ-ಮನಸ್ಸು-ಗ-ಳನ್ನು
ದೇಹ-ಮನಸ್ಸು-ಗಳಿ-ಗಿಂತ
ದೇಹ-ಮನಸ್ಸು-ಗಳು
ದೇಹ-ಮಾತ್ರ
ದೇಹ-ರ-ಚನೆಯ
ದೇಹ-ರ-ಚ-ನೆಯೇ
ದೇಹ-ವನ್ನಾಗಿ
ದೇಹ-ವನ್ನಾ-ದರೂ
ದೇಹ-ವನ್ನು
ದೇಹ-ವನ್ನೂ
ದೇಹ-ವನ್ನೆಲ್ಲ
ದೇಹ-ವನ್ನೆಲ್ಲಾ
ದೇಹ-ವನ್ನೇ
ದೇಹ-ವಲ್ಲ
ದೇಹ-ವಲ್ಲದ
ದೇಹ-ವಲ್ಲದೆ
ದೇಹ-ವಾ-ಗಲಿ
ದೇಹ-ವಾ-ಗಲೀ
ದೇಹ-ವಾಗಿ
ದೇಹ-ವಾಗಿದೆ
ದೇಹ-ವಾಗಿ-ರ-ಲಿಲ್ಲ
ದೇಹ-ವಾಗು-ವು-ದಿಲ್ಲ
ದೇಹ-ವಾ-ಗು-ವುದು
ದೇಹ-ವಾದ
ದೇಹ-ವಿದೆ
ದೇಹ-ವಿದೆಯೋ
ದೇಹ-ವಿರಲು
ದೇಹ-ವಿರು-ವು-ದಿಲ್ಲ
ದೇಹ-ವಿಲ್ಲದೆ
ದೇಹವು
ದೇಹವೂ
ದೇಹವೆ
ದೇಹ-ವೆಂದು
ದೇಹ-ವೆಂದೇ
ದೇಹ-ವೆಂಬ
ದೇಹ-ವೆಲ್ಲ
ದೇಹ-ವೆಲ್ಲವೂ
ದೇಹವೇ
ದೇಹ-ವೇನೊ
ದೇಹ-ಶಕ್ತಿ
ದೇಹ-ಸುಖ-ವನ್ನು
ದೇಹಾ-ತೀತ-ರಾಗು-ವುದೇ
ದೇಹಾ-ಭಿಲಾಷೆ
ದೇಹಾ-ಸಕ್ತಿಯೇ
ದೈತ್ಯರು
ದೈನಂದಿನ
ದೈನಿಕ
ದೈವತ್ವ-ವನ್ನು
ದೈವತ್ವವೇ
ದೈವ-ಭಕ್ತನೂ
ದೈವಿಕ
ದೈವಿಕ-ವಾ-ಗು-ವುದು
ದೈವೀ
ದೈವೀ-ಕರಣ
ದೈವೀ-ದೃಷ್ಟಿ-ಯಿಂದ
ದೈವೀ-ಭಾ-ವನೆ
ದೈವೀ-ಮಾಯೆ
ದೈವೀಸ್ವ-ರೂಪಕ್ಕೆ
ದೈವೀಸ್ವ-ರೂಪ-ವನ್ನು
ದೈವೇಚ್ಛೆ
ದೈಹಿಕ
ದೈಹಿಕ-ವಾಗಿ
ದೊಂದಿಗೆ
ದೊಂದು
ದೊಂಬಿ-ಯನ್ನು
ದೊಡನೆ
ದೊಡ್ಡ
ದೊಡ್ಡ-ದಲ್ಲ
ದೊಡ್ಡ-ದಾ-ಗಿತ್ತು
ದೊಡ್ಡ-ದಾ-ಗಿಲ್ಲದ
ದೊಡ್ಡ-ದಾಗುತ್ತ
ದೊಡ್ಡ-ದಾ-ಗು-ವುವು
ದೊಡ್ಡ-ದಾ-ಯಿತು
ದೊಡ್ಡದು
ದೊಡ್ಡ-ದು-ಇತ್ಯಾದಿ
ದೊಡ್ಡದೊ
ದೊಡ್ಡ-ದೊಂದು
ದೊಡ್ಡ-ಮರ
ದೊಡ್ಡ-ವ-ನಾದ
ದೊಡ್ಡ-ವನು
ದೊಡ್ಡ-ವನೊ
ದೊಡ್ಡ-ವ-ರಾ-ಗಲೀ
ದೊರ-ಕದಿ-ರುವ
ದೊರ-ಕದು
ದೊರ-ಕದೆ
ದೊರ-ಕದೇ
ದೊರ-ಕದೊ
ದೊರಕ-ಬಹು-ದಾದ
ದೊರಕ-ಬಹುದು
ದೊರಕ-ಬಹು-ದೆಂದು
ದೊರಕ-ಬೇಕಾ-ದರೆ
ದೊರಕ-ಬೇಕು
ದೊರಕ-ಬೇಕೆಂದು
ದೊರಕ-ಲಾರ
ದೊರಕ-ಲಾರದು
ದೊರಕಲಿ
ದೊರಕ-ಲಿಲ್ಲ
ದೊರ-ಕಲು
ದೊರಕಿತು
ದೊರಕಿಯೇ
ದೊರಕಿಲ್ಲ
ದೊರಕು
ದೊರಕುತ್ತದೆ
ದೊರಕುತ್ತದೆಂಬು-ದನ್ನು
ದೊರಕುತ್ತವೆ
ದೊರಕುವ
ದೊರಕು-ವಂತೆಯೂ
ದೊರಕು-ವನು
ದೊರಕು-ವರು
ದೊರಕು-ವ-ವರೆಗೆ
ದೊರಕು-ವು-ದಿಲ್ಲ
ದೊರಕು-ವುದು
ದೊರಕು-ವುದೆ
ದೊರಕು-ವುದೋ
ದೊರಕು-ವುವು
ದೊರೆ
ದೊರೆತ
ದೊರೆ-ತಂತೆ
ದೊರೆ-ತನು
ದೊರೆ-ತರೆ
ದೊರೆ-ತಿ-ರ-ಬಹುದು
ದೊರೆ-ತಿ-ರು-ವರು
ದೊರೆ-ತಿಲ್ಲ
ದೊರೆ-ತಿ-ವೆಯೋ
ದೊರೆ-ಯದೆ
ದೊರೆ-ಯ-ಬಹುದು
ದೊರೆ-ಯಾದ
ದೊರೆ-ಯು-ವು-ದಿಲ್ಲ
ದೊರೆ-ಯು-ವುದು
ದೊರೆ-ಯೊಬ್ಬ-ನಿಗೆ
ದೋಚಲು
ದೋಚಿ-ಕೊಳ್ಳಿ
ದೋಚುತ್ತಿ-ರು-ವೆವು
ದೋಣಿ-ಗಳು
ದೋಣಿ-ಯನ್ನು
ದೋಣಿ-ಯಲ್ಲಿಟ್ಟೂ
ದೋಣಿ-ಯಲ್ಲಿ-ರುವಾಗ
ದೋಷ
ದೋಷಕ್ಕೆ
ದೋಷ-ಗಳ
ದೋಷ-ಗ-ಳನ್ನು
ದೋಷ-ಗಳಾ-ವುವೂ
ದೋಷ-ಗಳಿಂದ
ದೋಷ-ಗಳಿಗೆ
ದೋಷ-ಗಳು
ದೋಷ-ಗಳೆಲ್ಲ
ದೋಷ-ಗಳೇ-ಳು-ವುವು
ದೋಷ-ದರ್ಶನ
ದೋಷ-ದಿಂದ
ದೋಷ-ಪೂರ್ಣ
ದೋಷ-ಪೂರ್ಣ-ವಾಗಿದೆ
ದೋಷ-ಬೀಜಕ್ಷಯೇ
ದೋಷ-ಯುಕ್ತ-ವಾ-ಗಿ-ರುವುದು
ದೋಷ-ಯುಕ್ತ-ವಾ-ದುದು
ದೋಷ-ರಾಶಿ
ದೋಷ-ವನ್ನು
ದೋಷ-ವನ್ನೇ
ದೋಷ-ವಲ್ಲದೆ
ದೋಷ-ವಿದೆ
ದೋಷ-ವಿದ್ದರೆ
ದೋಷ-ವಿಲ್ಲ
ದೋಷವು
ದೋಷವೂ
ದೋಷಾರೋಪಣೆ-ಯಿಂದ
ದೌರ್ಜನ್ಯ-ವನ್ನು
ದೌರ್ಜನ್ಯ-ವೆಂದು
ದೌರ್ಬಲ್ಯ
ದೌರ್ಬಲ್ಯಕ್ಕೆ
ದೌರ್ಬಲ್ಯ-ಗ-ಳನ್ನು
ದೌರ್ಬಲ್ಯ-ಗಳಿಗೆ
ದೌರ್ಬಲ್ಯ-ಗಳು
ದೌರ್ಬಲ್ಯದ
ದೌರ್ಬಲ್ಯ-ದಲ್ಲಿ
ದೌರ್ಬಲ್ಯ-ವನ್ನು
ದೌರ್ಬಲ್ಯ-ವಲ್ಲದೆ
ದ್ದನು
ದ್ದನ್ನು
ದ್ದರು
ದ್ದರೂ
ದ್ದಾಗ
ದ್ದಾನೆ
ದ್ದಾರೆಯೋ
ದ್ದುವು
ದ್ದೇನೂ
ದ್ದೇವೆ
ದ್ಭುತ-ವಾದ
ದ್ರವ
ದ್ರವಕ್ಕೆ
ದ್ರವ-ದಂತೆ
ದ್ರವ-ದಲ್ಲಿ
ದ್ರವ-ವಾ-ಗು-ವುದು
ದ್ರವ್ಯ
ದ್ರವ್ಯ-ಗಳ
ದ್ರವ್ಯ-ಗಳಿಂದ
ದ್ರವ್ಯ-ಗಳು
ದ್ರವ್ಯದ
ದ್ರವ್ಯ-ದಂತೆ
ದ್ರವ್ಯ-ದಲ್ಲಿ-ರುವ
ದ್ರವ್ಯ-ದಿಂದ
ದ್ರವ್ಯ-ರಾಶಿ
ದ್ರವ್ಯ-ರಾಶಿಯ
ದ್ರವ್ಯ-ರಾಶಿ-ಯಲ್ಲಿ
ದ್ರವ್ಯ-ರಾಶಿಯು
ದ್ರವ್ಯ-ವನ್ನಾ-ಗಲಿ
ದ್ರವ್ಯ-ವನ್ನು
ದ್ರವ್ಯ-ವಲ್ಲ
ದ್ರವ್ಯ-ವಸ್ತು
ದ್ರವ್ಯ-ವಸ್ತುವೂ
ದ್ರವ್ಯ-ವಿಲ್ಲದೆ
ದ್ರವ್ಯವು
ದ್ರವ್ಯವೂ
ದ್ರವ್ಯವೆ
ದ್ರವ್ಯ-ವೆಂದ-ರೇನು
ದ್ರವ್ಯವೇ
ದ್ರವ್ಯಾರ್ಜನೆ-ಯಾ-ಗಲಿ
ದ್ರಷ್ಟಾದೃಶಿ-ಮಾತ್ರಃ
ದ್ರಷ್ಟುಃ
ದ್ರಷ್ಟೃ
ದ್ರಷ್ಟೃ-ದೃಶ್ಯಯೋಃ
ದ್ರಷ್ಠಾನುಶ್ರ-ವಿಕ
ದ್ರಷ್ವೃ-ವಿನ
ದ್ವಂದ್ವ
ದ್ವಂದ್ವ-ಗಳಾ-ವುವೂ
ದ್ವಂದ್ವ-ಗಳು
ದ್ವಂದ್ವ-ಗಳು-ಎಂದರೆ
ದ್ವಂದ್ವ-ಮಿಶ್ರ-ವೆಂದು
ದ್ವಂದ್ವ-ವನ್ನು
ದ್ವಂದ್ವವೇ
ದ್ವಂದ್ವಾನಭಿಘಾತಃ
ದ್ವಾರ-ಗಳಂತಿವೆ
ದ್ವಾರ-ಗ-ಳನ್ನು
ದ್ವಿದಳ
ದ್ವಿರೂಪ-ತೆ-ಯನ್ನು
ದ್ವೀಪ-ಗಳ
ದ್ವೇಗ-ದಿಂದ
ದ್ವೇಷ
ದ್ವೇಷಃ
ದ್ವೇಷಕ್ಕೂ
ದ್ವೇಷಕ್ಕೆ
ದ್ವೇಷ-ಗಳ
ದ್ವೇಷ-ಗಳು
ದ್ವೇಷದ
ದ್ವೇಷ-ವನ್ನು
ದ್ವೇಷವೆ
ದ್ವೇಷ-ವೆಲ್ಲ
ದ್ವೇಷವೇ
ದ್ವೇಷಾಸೂಯೆ-ಗಳ
ದ್ವೇಷಿಸ-ಕೂಡದು
ದ್ವೇಷಿಸ-ಬೇಡಿ
ದ್ವೇಷಿಸುತ್ತೇನೆ
ದ್ವೇಷಿ-ಸುವ
ದ್ವೇಷಿ-ಸುವ-ವ-ರಾರು
ದ್ವೇಷಿ-ಸು-ವಷ್ಟು
ದ್ವೇಷಿ-ಸು-ವು-ದಿಲ್ಲ
ದ್ವೇಷಿ-ಸು-ವು-ದಿಲ್ಲವೊ
ದ್ವೇಷಿ-ಸು-ವುದು
ದ್ವೈತ
ದ್ವೈತಕ್ಕೆ
ದ್ವೈತ-ತತ್ತ್ವದ
ದ್ವೈತದ
ದ್ವೈತ-ದಲ್ಲಿ
ದ್ವೈತ-ದಿಂದ
ದ್ವೈತ-ದೊಂದಿಗೆ
ದ್ವೈತ-ಧರ್ಮ-ಗಳಲ್ಲಿ
ದ್ವೈತ-ಧರ್ಮ-ವಿದ್ದರೆ
ದ್ವೈತ-ಭಾ-ವನೆ
ದ್ವೈತ-ವನ್ನು
ದ್ವೈತ-ವಲ್ಲದುದು
ದ್ವೈತ-ವೇದಾಂತಿ-ಗಳು
ದ್ವೈತ-ಸಿದ್ಧಾಂತ-ಗಳ
ದ್ವೈತ-ಸಿದ್ಧಾಂತ-ಗ-ಳನ್ನು
ದ್ವೈತ-ಸಿದ್ಧಾಂತ-ಗಳಲ್ಲಿಯೂ
ದ್ವೈತ-ಸಿದ್ಧಾಂತ-ಗಳು
ದ್ವೈತಿ
ದ್ವೈತಿ-ಗಳ
ದ್ವೈತಿ-ಗ-ಳನ್ನು
ದ್ವೈತಿ-ಗಳಿಂದ
ದ್ವೈತಿ-ಗಳಿ-ಗಿಂತ
ದ್ವೈತಿ-ಗಳಿಗೆ
ದ್ವೈತಿ-ಗಳಿ-ರು-ವರು
ದ್ವೈತಿ-ಗಳಿ-ರು-ವುರು
ದ್ವೈತಿ-ಗಳು
ದ್ವೈತಿ-ಗಳೆ
ದ್ವೈತಿ-ಗ-ಳೆಂದು
ದ್ವೈತಿ-ಗಳೊಂದಿಗೆ
ದ್ವೈತಿ-ಗಳೊ-ಡನೆ
ದ್ವೈತಿ-ಗಿಂತ
ದ್ವೈತಿಗೆ
ದ್ವೈತಿಯ
ದ್ವೈತಿ-ಯನ್ನು
ಧಕ್ಕೆ
ಧನ
ಧನಿಕ-ರಿಗೆ
ಧನ್ಯ-ನಾದೆ
ಧನ್ಯ-ವಾದ
ಧನ್ಯ-ವಾದ-ಗಳು
ಧನ್ಯ-ವಾದ-ವನ್ನು
ಧಮ್ಮ
ಧರಿಸ-ದಂತೆ
ಧರಿಸ-ಬಲ್ಲ
ಧರಿಸ-ಬಹುದು
ಧರಿ-ಸ-ಬೇಕು
ಧರಿ-ಸಲು
ಧರಿಸಿ
ಧರಿಸಿ-ದನು
ಧರಿಸಿ-ರುವುದು
ಧರಿ-ಸುತ್ತದೆ
ಧರಿಸುತ್ತವೆ
ಧರಿಸುತ್ತಾನೆ
ಧರಿ-ಸು-ವನು
ಧರಿಸು-ವರು
ಧರಿ-ಸು-ವುದು
ಧರಿಸು-ವುದೋ
ಧರ್ಮ
ಧರ್ಮಂಧತೆ-ಯಾ-ಗು-ವುದು
ಧರ್ಮಕ್ಕಾ-ದರೂ
ಧರ್ಮಕ್ಕಿಂತ
ಧರ್ಮಕ್ಕೂ
ಧರ್ಮಕ್ಕೆ
ಧರ್ಮಕ್ಕೇ
ಧರ್ಮ-ಗಳ
ಧರ್ಮ-ಗ-ಳನ್ನು
ಧರ್ಮ-ಗಳನ್ನೆಲ್ಲಾ
ಧರ್ಮ-ಗಳಲ್ಲಿ
ಧರ್ಮ-ಗಳಲ್ಲಿಯೂ
ಧರ್ಮ-ಗಳಲ್ಲೆಲ್ಲಾ
ಧರ್ಮ-ಗಳಿಂದ
ಧರ್ಮ-ಗಳಿಗೂ
ಧರ್ಮ-ಗಳಿ-ಗೆಲ್ಲಾ
ಧರ್ಮ-ಗಳು
ಧರ್ಮ-ಗಳೂ
ಧರ್ಮ-ಗಳೆಲ್ಲ
ಧರ್ಮ-ಗಳೆಲ್ಲವೂ
ಧರ್ಮ-ಗಳೇನೋ
ಧರ್ಮಗ್ರಂಥ
ಧರ್ಮಗ್ರಂಥ-ಗಳು
ಧರ್ಮಗ್ರಂಥ-ಗಳೂ
ಧರ್ಮಗ್ರಂಥ-ದಲ್ಲಿ
ಧರ್ಮಗ್ರಂಥ-ವಿರು
ಧರ್ಮದ
ಧರ್ಮ-ದಲ್ಲಿ
ಧರ್ಮ-ದಲ್ಲಿಯೂ
ಧರ್ಮ-ದಲ್ಲಿ-ರುವ
ಧರ್ಮ-ದಲ್ಲಿ-ರುವು-ದೆಲ್ಲ
ಧರ್ಮ-ದಲ್ಲೆಲ್ಲ
ಧರ್ಮ-ದ-ವರೂ
ಧರ್ಮ-ದಷ್ಟು
ಧರ್ಮ-ದಿಂದ
ಧರ್ಮಧ್ವಂಸ-ಕ-ರಾಗುವ
ಧರ್ಮಪ್ರೇರಣೆ
ಧರ್ಮ-ಭೇದ-ಗಳಿ-ರು-ವು-ದ-ರಿಂದ
ಧರ್ಮ-ಮಾರ್ಗಕ್ಕೆ
ಧರ್ಮ-ಮೂಲ-ದಿಂದ
ಧರ್ಮ-ಮೇಘ
ಧರ್ಮ-ಮೇಘ-ವೆಂಬ
ಧರ್ಮ-ಲಕ್ಷ-ಣಾವಸ್ಥಾ
ಧರ್ಮ-ವನ್ನು
ಧರ್ಮ-ವನ್ನೂ
ಧರ್ಮ-ವಲ್ಲ
ಧರ್ಮ-ವಲ್ಲದೆ
ಧರ್ಮ-ವಾಗ-ಲಾರದು
ಧರ್ಮ-ವಾ-ಗಲಿ
ಧರ್ಮ-ವಾ-ಗಿತ್ತು
ಧರ್ಮ-ವಾಗು-ವು-ದಿಲ್ಲ
ಧರ್ಮ-ವಿದ್ದರೆ
ಧರ್ಮ-ವಿ-ರ-ಲಿಲ್ಲವೋ
ಧರ್ಮ-ವಿ-ರು-ವುದು
ಧರ್ಮ-ವಿಲ್ಲ
ಧರ್ಮವು
ಧರ್ಮವೂ
ಧರ್ಮವೆ
ಧರ್ಮ-ವೆಂದರೆ
ಧರ್ಮ-ವೆಂಬುದು
ಧರ್ಮ-ವೆಂಬುದೇ
ಧರ್ಮ-ವೆನ್ನು-ವುದು
ಧರ್ಮವೇ
ಧರ್ಮ-ಶಾಸ್ತ್ರ
ಧರ್ಮ-ಶಾಸ್ತ್ರಕ್ಕೆ
ಧರ್ಮ-ಶಾಸ್ತ್ರ-ಗಳೆಲ್ಲ
ಧರ್ಮ-ಸಾಮ-ರಸ್ಯದ
ಧರ್ಮಾತ್ಮ-ನಾಗಿ
ಧರ್ಮಾತ್ಮರಾದಂಥ
ಧರ್ಮಾತ್ಮರು
ಧರ್ಮಾಧಿ-ಕಾರಿ-ಗಳ
ಧರ್ಮಾನು
ಧರ್ಮಾಭಿ-ಮಾನಿ-ಗಳು
ಧರ್ಮಾರ್ಥದ
ಧರ್ಮಾವಲಂಬಿ-ಗಳು
ಧರ್ಮೀ
ಧರ್ಮೋದಯ
ಧರ್ಮೋಪ-ದೇಶ-ಗ-ಳನ್ನು
ಧಾನ-ಗಳಿಂದ
ಧಾನ-ವನ್ನು
ಧಾನ್ಯ-ಗಳ
ಧಾನ್ಯ-ದಲ್ಲಿ-ರು-ವರು
ಧಾರಣ
ಧಾರ-ಣಕ್ಕೆ
ಧಾರ-ಣ-ಗ-ಳಾದ
ಧಾರ-ಣ-ದಿಂದ
ಧಾರ-ಣ-ಮಾಡಿ
ಧಾರ-ಣ-ಮಾಡುತ್ತಿದೆ
ಧಾರ-ಣ-ಮಾಡು-ವುದು
ಧಾರ-ಣ-ವನ್ನು
ಧಾರ-ಣ-ವೆಂದರೆ
ಧಾರ-ಣ-ವೆಂದು
ಧಾರ-ಣ-ವೆನ್ನು-ವುದು
ಧಾರಣಾ
ಧಾರ-ಣಾಧ್ಯಾನ-ಸಮಾ-ಧಯೋಽಷ್ಟಾವಂಗಾನಿ
ಧಾರ-ಣಾಸು
ಧಾರಣೆ
ಧಾರ-ಣೆ-ಮಾಡಿ-ದರೆ
ಧಾರಿ-ಗಳು
ಧಾರೆ
ಧಾರೆ-ಯೆರೆಯ-ಲಾರದು
ಧಾರ್ಮಿ
ಧಾರ್ಮಿಕ
ಧಾರ್ಮಿ-ಕ-ನಾಗ-ಬಲ್ಲ
ಧಾರ್ಮಿ-ಕ-ನಾಗ-ಬೇಕಾ-ದರೆ
ಧಾರ್ಮಿ-ಕ-ನಿಗೂ
ಧಾರ್ಮಿ-ಕ-ನಿಯಮ
ಧಾರ್ಮಿ-ಕ-ರನ್ನಾಗಿ
ಧಾರ್ಮಿ-ಕ-ರಲ್ಲ-ವೆಂದು
ಧಾರ್ಮಿ-ಕ-ರಾಗ-ಬೇಕೆನ್ನುವ
ಧಾರ್ಮಿ-ಕ-ರಾಗ-ಲಾರೆವು
ಧಾರ್ಮಿ-ಕ-ರಾ-ಗಲು
ಧಾರ್ಮಿ-ಕ-ರಾಗುತ್ತಿ-ರು-ವಿರಿ
ಧಾರ್ಮಿ-ಕ-ರಾಗು-ವಿರಿ
ಧಾರ್ಮಿ-ಕರು
ಧಾರ್ಮಿ-ಕ-ರೆಂದು
ಧಾರ್ಮಿ-ಕ-ಶಕ್ತಿಯು
ಧಾಳಿ
ಧಾವಿ-ಸಲೇ-ಬೇಕು
ಧಾವಿಸು
ಧಾವಿ-ಸುತ್ತಿದೆ
ಧಾವಿಸುತ್ತಿ-ರುವ
ಧಾವಿಸುತ್ತಿ-ರು-ವರು
ಧಾವಿಸುತ್ತಿ-ರುವುದು
ಧಾವಿಸುತ್ತಿ-ರು-ವೆವು
ಧಾವಿಸುತ್ತಿವೆ
ಧಾವಿಸು-ವಂತೆ
ಧಾವಿ-ಸು-ವು-ದಕ್ಕೆ
ಧಾವಿ-ಸು-ವೆವು
ಧಿಕ್ಕರಿಸ-ಬೇಡಿ
ಧಿಕ್ಕ-ರಿಸಿ
ಧೀರ
ಧೀರನ
ಧೀರ-ನುಡಿ
ಧೀರ-ನುಡಿ-ಗ-ಳನ್ನು
ಧೀರ-ನೊಬ್ಬನು
ಧೀರ-ರಂತೆ
ಧೀರ-ರಾಗಿ
ಧೀರ-ರಾಗು-ವರು
ಧೀರ-ರಿ-ರು-ವರು
ಧೀರರು
ಧೀರ-ವಾಣಿ
ಧೀರಾ
ಧೀರ್ಘ
ಧೂಮ
ಧೂಮ-ಕೇತು-ಗಳಂತೆ
ಧೂಮ-ಕೇತು-ವಾ-ಗು-ವುದು
ಧೂಮಕ್ಕೆ
ಧೂರ್ತ-ತನ-ವನ್ನು
ಧೂಳಿ
ಧೂಳಿ-ನಿಂದ
ಧೂಳಿಯ
ಧೂಳೀ-ಕಣ-ಗಳಾಗುತ್ತಿ-ರು-ವುವು
ಧೂಳೀ-ಕರ-ಣ-ಗಳಾಗಿ
ಧೃಢಿಷ್ಠರೂ
ಧೈರ್ಯ
ಧೈರ್ಯ-ಗಳಿವೆ
ಧೈರ್ಯ-ದಿಂದ
ಧೈರ್ಯ-ವನ್ನು
ಧೈರ್ಯ-ವಾಗಿ
ಧೈರ್ಯ-ವಾಗಿರ
ಧೈರ್ಯ-ವಾಗಿರಿ
ಧೈರ್ಯ-ವಾದ
ಧೈರ್ಯ-ವಿದ್ದರೆ
ಧೈರ್ಯ-ವಿದ್ದ-ವನು
ಧೈರ್ಯ-ವಿ-ರ-ಲಿಲ್ಲ
ಧೈರ್ಯ-ವುಳ್ಳ
ಧೈರ್ಯ-ವೆಲ್ಲ
ಧೈರ್ಯ-ಶಾಲಿ
ಧೈರ್ಯ-ಶಾಲಿ-ಗಳಾ-ಗಿದ್ದರು
ಧೈರ್ಯ-ಶಾಲಿ-ಗ-ಳಾದರೆ
ಧೈರ್ಯ-ಶಾಲಿ-ಗಳು
ಧೋರಣೆ-ಯನ್ನು
ಧ್ಯಾನ
ಧ್ಯಾನಕ್ಕೆ
ಧ್ಯಾನ-ಗ-ಳನ್ನು
ಧ್ಯಾನ-ಜಮ-ನಾಶ-ಯಮ್
ಧ್ಯಾನದ
ಧ್ಯಾನ-ದಲ್ಲಿ
ಧ್ಯಾನ-ದಲ್ಲಿಯೂ
ಧ್ಯಾನ-ದಿಂದ
ಧ್ಯಾನ-ಮಗ್ನ-ನಾಗುತ್ತೇನೆ
ಧ್ಯಾನ-ಮಾಡ-ಬೇಕು
ಧ್ಯಾನ-ಮಾ-ಡಲು
ಧ್ಯಾನ-ಮಾಡಿ
ಧ್ಯಾನ-ಮಾಡಿ-ದಂತೆ
ಧ್ಯಾನ-ಮಾಡು
ಧ್ಯಾನ-ಮಾಡು-ವುದು
ಧ್ಯಾನ-ಮಾಡು-ವುದು-ಅವು-ಗಳ
ಧ್ಯಾನಮ್
ಧ್ಯಾನ-ವನ್ನು
ಧ್ಯಾನ-ವಾಗು
ಧ್ಯಾನವು
ಧ್ಯಾನ-ವೆಂದರೆ
ಧ್ಯಾನ-ವೆಂದು
ಧ್ಯಾನ-ವೆನ್ನು-ವುದು-ಬಾಹ್ಯ
ಧ್ಯಾನಸ್ಥಿತಿಗೆ
ಧ್ಯಾನಸ್ಥಿತಿಯೆ
ಧ್ಯಾನ-ಹೇಯಾಸ್ತದ್ವೃತ್ತಯಃ
ಧ್ಯಾನಾಭ್ಯಾಸ
ಧ್ಯಾನಾ-ವಸ್ಥೆ-ಯಲ್ಲಿಲ್ಲ-ದಾಗ
ಧ್ಯಾನಿಸ-ಬಲ್ಲದು
ಧ್ಯಾನಿಸ-ಬಹುದು
ಧ್ಯಾನಿ-ಸ-ಬೇಕು
ಧ್ಯಾನಿಸಿ
ಧ್ಯಾನಿಸು
ಧ್ಯಾನಿಸುತ್ತೇವೆ
ಧ್ಯಾನಿ-ಸುವ
ಧ್ಯಾನಿಸು-ವರೊ
ಧ್ಯಾನಿ-ಸು-ವು-ದಕ್ಕೆ
ಧ್ಯಾನಿಸು-ವುದರ
ಧ್ಯಾನಿಸು-ವುದು
ಧ್ಯೇಯ
ಧ್ಯೇಯ-ಗ-ಳನ್ನು
ಧ್ಯೇಯ-ಗಳಿಂದ
ಧ್ಯೇಯ-ಗಳೊಂದಿಗೆ
ಧ್ಯೇಯದ
ಧ್ಯೇಯ-ದಲ್ಲಿ
ಧ್ಯೇಯ-ವನ್ನು
ಧ್ಯೇಯ-ವಸ್ತು-ವಾಗುತ್ತದೆ
ಧ್ಯೇಯ-ವಿಲ್ಲದ
ಧ್ಯೇಯವೆ
ಧ್ರುವ
ಧ್ರುವ-ಗಳಂತೆ
ಧ್ರುವ-ಗಳಷ್ಟು
ಧ್ರುವತ್ವ-ವನ್ನು
ಧ್ರುವದ
ಧ್ರುವ-ದಷ್ಟು
ಧ್ರುವ-ಮಧ್ರುವೇಷ್ಟಿಹ
ಧ್ರುವೇ
ಧ್ವಂಸ
ಧ್ವಂಸ-ಕಾರಿ-ಯೇನೋ
ಧ್ವಂಸದ
ಧ್ವಂಸ-ಮಾಡ-ಬೇಡಿ
ಧ್ವಂಸ-ಮಾಡಿ
ಧ್ವಂಸ-ಮಾಡುತ್ತಿಲ್ಲ
ಧ್ವಂಸ-ಮಾಡುತ್ತಿವೆ
ಧ್ವಂಸ-ಮಾಡು-ವು-ದಕ್ಕೆ
ಧ್ವಂಸ-ಮಾಡು-ವುದು
ಧ್ವಂಸ-ವಾಗಿ
ಧ್ವಂಸ-ವಾ-ಗು-ವುದು
ಧ್ವನಿ
ಧ್ವನಿ-ಗ-ಳನ್ನು
ಧ್ವನಿ-ಗಳು
ಧ್ವನಿ-ಗಳೆಲ್ಲವೂ
ಧ್ವನಿಗೂ
ಧ್ವನಿಯ
ಧ್ವನಿ-ಯಂತೆ
ಧ್ವನಿ-ಯನ್ನು
ಧ್ವನಿ-ಯಲ್ಲಿ
ಧ್ವನಿ-ಯಲ್ಲಿಯೇ
ಧ್ವನಿ-ಯಲ್ಲಿ-ರುವ
ಧ್ವನಿ-ಯಿಂದ
ಧ್ವನಿಯು
ಧ್ವನಿಯೂ
ಧ್ವನಿ-ಯೆ-ಡೆಗೆ
ಧ್ವನಿ-ಯೊಂದು
ನ
ನಂಟ-ರನ್ನು
ನಂತರ
ನಂತೆ
ನಂದ-ವನ್ನು
ನಂಬ
ನಂಬದ
ನಂಬ-ದ-ವ-ರನ್ನು
ನಂಬ-ದಾಗ
ನಂಬ-ದಿದ್ದು-ದಕ್ಕಾಗಿ
ನಂಬ-ದಿರು
ನಂಬ-ದಿರು-ವುದೇ
ನಂಬದೆ
ನಂಬ-ಬಹುದು
ನಂಬ-ಬೇಕಾ-ಗಿದೆ
ನಂಬ-ಬೇಕಾ-ಗಿಲ್ಲ
ನಂಬ-ಬೇಕು
ನಂಬ-ಬೇಕೆಂದು
ನಂಬ-ಲಾರ
ನಂಬ-ಲಾರೆವು
ನಂಬ-ಲಿಲ್ಲ
ನಂಬಲೇ
ನಂಬ-ಲೇ-ಬೇಕು
ನಂಬಿ
ನಂಬಿಕೆ
ನಂಬಿ-ಕೆ-ಗಳ
ನಂಬಿ-ಕೆ-ಗಳಿಗೆ
ನಂಬಿ-ಕೆ-ಗಳು
ನಂಬಿ-ಕೆಗೆ
ನಂಬಿ-ಕೆಯ
ನಂಬಿ-ಕೆ-ಯನ್ನು
ನಂಬಿ-ಕೆ-ಯನ್ನೂ
ನಂಬಿ-ಕೆ-ಯಲ್ಲ
ನಂಬಿ-ಕೆ-ಯಲ್ಲಿ-ರುವ
ನಂಬಿ-ಕೆ-ಯಿಟ್ಟಿದ್ದ
ನಂಬಿ-ಕೆ-ಯಿಡಿ
ನಂಬಿ-ಕೆ-ಯಿಡು
ನಂಬಿ-ಕೆ-ಯಿಡುವ
ನಂಬಿ-ಕೆ-ಯಿದೆ
ನಂಬಿ-ಕೆ-ಯಿಲ್ಲ
ನಂಬಿ-ಕೆ-ಯಿಲ್ಲವೋ
ನಂಬಿ-ಕೆಯು
ನಂಬಿ-ಕೆ-ಯುಂಟು
ನಂಬಿ-ಕೆಯೂ
ನಂಬಿ-ಕೆಯೇ
ನಂಬಿ-ಕೆ-ಯೊಂದೇ
ನಂಬಿ-ಕೊಂಡಿ-ರು-ವುದು
ನಂಬಿ-ಗೆ-ಯುಳ್ಳ-ವರು
ನಂಬಿದ
ನಂಬಿ-ದರೆ
ನಂಬಿ-ದ-ವರು
ನಂಬಿದ್ದರು
ನಂಬಿದ್ದರೆ
ನಂಬು
ನಂಬು-ತೇನೆ
ನಂಬುತ್ತಾನೆ
ನಂಬುತ್ತಾರೆ
ನಂಬುತ್ತಾರೊ
ನಂಬುತ್ತಾರೋ
ನಂಬುತ್ತಿದ್ದರು
ನಂಬುತ್ತೇನೆ
ನಂಬುತ್ತೇವೆ
ನಂಬುವ
ನಂಬು-ವಂತ-ಹ-ವರಿ-ರ-ಬಹುದು
ನಂಬು-ವಂತೆ
ನಂಬು-ವನು
ನಂಬು-ವರು
ನಂಬು-ವ-ರು-ಅ-ವನೇ
ನಂಬು-ವರೋ
ನಂಬು-ವ-ವರು
ನಂಬು-ವು-ದಕ್ಕೆ
ನಂಬು-ವು-ದಲ್ಲ
ನಂಬು-ವು-ದಾದರೆ
ನಂಬು-ವು-ದಿಲ್ಲ
ನಂಬು-ವು-ದಿಲ್ಲವೊ
ನಂಬು-ವು-ದಿಲ್ಲವೋ
ನಂಬು-ವುದು
ನಂಬು-ವುದೇ
ನಂಬು-ವುದೊ
ನಂಬು-ವೆನು
ನಂಬು-ವೆವು
ನಕ್ಕರೆ
ನಕ್ಕು
ನಕ್ಕೆ
ನಕ್ಷತ್ರ
ನಕ್ಷತ್ರ-ಗಳ
ನಕ್ಷತ್ರ-ಗಳಂತೆ
ನಕ್ಷತ್ರ-ಗ-ಳನ್ನು
ನಕ್ಷತ್ರದ
ನಕ್ಷತ್ರ-ದಲ್ಲಿ
ನಕ್ಷತ್ರ-ಮಂಡಲ
ನಕ್ಷತ್ರಾವಳಿ
ನಕ್ಷತ್ರಾವಳಿ-ಗಳಿಂದ
ನಕ್ಷತ್ರಾವಳಿ-ಗಳು
ನಕ್ಷತ್ರಾವಳಿ-ಗಳೆಲ್ಲಾ
ನಕ್ಷೆ-ಯನ್ನು
ನಖ-ಗ-ಳನ್ನು
ನಖ-ಗಳಿಂದ
ನಗ-ಬಹುದು
ನಗ-ಬೇಕಾ-ಗು-ವುದು
ನಗ-ಬೇಕು
ನಗರ
ನಗರ-ಗಳು
ನಗ-ರದ
ನಗರ-ದಲ್ಲಿ
ನಗರ-ದಲ್ಲಿ-ರುವ
ನಗರ-ದೊಂದಿಗೆ
ನಗಲೂ
ನಗಿ-ಸುವ
ನಗುತ್ತೇನೆ
ನಗುವ
ನಗು-ವಂತೆ
ನಗು-ವರು
ನಗುವ-ವ-ರೆಲ್ಲ
ನಗುವಿನ
ನಗು-ವಿರಿ
ನಗು-ವುದು
ನಗು-ವೆವು
ನಗೆ-ದಾಡುತ್ತಿದ್ದನು
ನಗೆಯ-ಲಾರರು
ನಗೆ-ಯಿಂದ
ನಚಿಕೇತ
ನಚಿಕೇ-ತನ
ನಚಿಕೇತ-ನನ್ನು
ನಚಿಕೇತ-ನಿಗೆ
ನಚಿಕೇತ-ನೆಂಬ
ನಟನು
ನಟಿಸಿ-ರುವೆ-ನೆಂದೂ
ನಟಿ-ಸುವ
ನಟಿ-ಸುವ-ವರು
ನಟಿ-ಸು-ವು-ದಿಲ್ಲ
ನಡತೆ
ನಡ-ತೆಗೆ
ನಡತೆ-ಯನ್ನು
ನಡತೆ-ಯಲ್ಲಿ
ನಡವಳಿ-ಕೆಗೆ
ನಡ-ವಳಿಕೆ-ಯಲ್ಲಿ
ನಡುಗಿ-ಸುತ್ತದೆ
ನಡುಗುತ್ತ
ನಡುಗುತ್ತಲೇ
ನಡುವಣ
ನಡುವೆ
ನಡೆ
ನಡೆ-ತೆ-ಗಳಿಂದ
ನಡೆದ
ನಡೆ-ದಂತೆ
ನಡೆ-ದರು
ನಡೆ-ದರೆ
ನಡೆ-ದಾಡುತ್ತ
ನಡೆ-ದಿದೆ
ನಡೆ-ದಿದೆಯೊ
ನಡೆ-ದಿದ್ದರೂ-ಅವು-ಗಳಲ್ಲಿ-ರುವ
ನಡೆ-ದಿ-ರ-ಬೇಕು
ನಡೆ-ದಿರು-ವಷ್ಟು
ನಡೆ-ದಿ-ರು-ವುದು
ನಡೆ-ದಿಲ್ಲ
ನಡೆ-ದಿವೆ
ನಡೆ-ದು-ಕೊಂಡು
ನಡೆ-ಯದು
ನಡೆ-ಯ-ಬಲ್ಲ
ನಡೆ-ಯ-ಬಲ್ಲುದು
ನಡೆ-ಯ-ಬೇಕಾ-ಗಿದೆ
ನಡೆ-ಯ-ಬೇಕಾದ
ನಡೆ-ಯಲಿ
ನಡೆ-ಯಲು
ನಡೆ-ಯಿತೊ
ನಡೆಯು
ನಡೆ-ಯುತ್ತ
ನಡೆ-ಯುತ್ತದೆ
ನಡೆ-ಯುತ್ತಲೇ
ನಡೆ-ಯುತ್ತವೆ
ನಡೆ-ಯುತ್ತಿತ್ತು
ನಡೆ-ಯುತ್ತಿದೆ
ನಡೆ-ಯುತ್ತಿದೆಯೋ
ನಡೆ-ಯುತ್ತಿ-ರುತ್ತದೆ
ನಡೆ-ಯುತ್ತಿ-ರುವ
ನಡೆ-ಯುತ್ತಿ-ರು-ವಂತೆ
ನಡೆ-ಯುತ್ತಿ-ರು-ವರು
ನಡೆ-ಯುತ್ತಿ-ರುವುದು
ನಡೆ-ಯುತ್ತಿವೆ
ನಡೆ-ಯುವ
ನಡೆ-ಯು-ವಂತೆ
ನಡೆ-ಯು-ವರು
ನಡೆ-ಯು-ವು-ದಕ್ಕೆ
ನಡೆ-ಯು-ವು-ದಿಲ್ಲ
ನಡೆ-ಯು-ವುದು
ನಡೆ-ಯು-ವು-ದೆಲ್ಲ-ವನ್ನು
ನಡೆ-ಯು-ವುವು
ನಡೆ-ಯೋಣ
ನಡೆ-ಸ-ಬೇಕು
ನಡೆಸಿ
ನಡೆ-ಸಿ-ಕೊಳ್ಳ-ಬಲ್ಲದು
ನಡೆ-ಸಿದ
ನಡೆ-ಸಿ-ದರೆ
ನಡೆ-ಸಿ-ದ-ವ-ರೆಂದು
ನಡೆಸು
ನಡೆ-ಸುತ್ತಾ
ನಡೆ-ಸುತ್ತಿದ್ದ-ವರೂ
ನಡೆ-ಸುತ್ತಿದ್ದುದು
ನಡೆ-ಸುತ್ತಿ-ರು-ವೆವು
ನಡೆ-ಸುವ
ನಡೆ-ಸು-ವರು
ನಡೆ-ಸು-ವ-ವನು
ನಡೆ-ಸು-ವಾಗ
ನದಾಗಲೀ
ನದಿ
ನದಿ-ಗಳ
ನದಿ-ಗಳಾ-ಗಿದ್ದೆವು
ನದಿ-ಗಳಿಂದ
ನದಿ-ಗಳು
ನದಿ-ಗಳೆಲ್ಲ
ನದಿಗೆ
ನದಿಯ
ನದಿ-ಯನ್ನು
ನದಿ-ಯಲ್ಲಿ
ನದಿ-ಯಷ್ಟು
ನದಿಯು
ನದು
ನನ-ಗನ್ನಿಸು
ನನ-ಗಾಗಿ
ನನ-ಗಿಂತ
ನನಗಿ-ರಲಿ
ನನ-ಗಿ-ರುವ
ನನಗಿ-ರು-ವು-ದ-ರಿಂದ
ನನಗೂ
ನನಗೆ
ನನಗೇ
ನನ-ಗೇನು
ನನ-ಗೇನೂ
ನನಗೇನೋ
ನನಗೊಂದು
ನನಗೋಸ್ಕರ-ವಾಗಿ
ನನದು
ನನ್ನ
ನನ್ನಂತಹ
ನನ್ನಂತೆ
ನನ್ನಂತೆಯೂ
ನನ್ನಂತೆಯೇ
ನನ್ನ-ದಲ್ಲ
ನನ್ನದು
ನನ್ನದೂ
ನನ್ನ-ದೆಂಬುದು
ನನ್ನನ್ನು
ನನ್ನನ್ನೇ
ನನ್ನಲ್ಲಿ
ನನ್ನಲ್ಲಿದೆ
ನನ್ನಲ್ಲಿಯೂ
ನನ್ನಲ್ಲಿರ
ನನ್ನಲ್ಲಿ-ರು-ವುದು
ನನ್ನಾಗಿ
ನನ್ನಾತ್ಮಾದ
ನನ್ನಿಂದ
ನನ್ನು
ನನ್ನೆ-ಡೆಗೆ
ನನ್ನೆ-ದುರಿ-ಗಿ-ರುವ
ನನ್ನೊ-ಡ-ನೆಯೆ
ನಪುಂಸ-ಕನೆ
ನಮ
ನಮ-ಗದು
ನಮ-ಗಾಗಿ
ನಮ-ಗಾಗು-ವು-ದಿಲ್ಲ
ನಮ-ಗಿಂತ
ನಮಗಿಂದ
ನಮಗಿಂದು
ನಮ-ಗಿ-ರುವ
ನಮ-ಗಿಲ್ಲ
ನಮಗೀಗ
ನಮಗುಂಟಾ-ಗುವ
ನಮಗೂ
ನಮಗೆ
ನಮ-ಗೆಲ್ಲ
ನಮ-ಗೆಲ್ಲಾ
ನಮಗೇ
ನಮ-ಗೇನು
ನಮ-ಗೇನೂ
ನಮಗೊಂದು
ನಮ-ಗೋಸುಗ-ವಾಗಿ
ನಮಸ್ಕಾರ
ನಮ್ಮ
ನಮ್ಮಂತಹ
ನಮ್ಮಂತೆ
ನಮ್ಮಂತೆಯೆ
ನಮ್ಮಂತೆಯೇ
ನಮ್ಮ-ದಲ್ಲವೊ
ನಮ್ಮ-ದಾ-ಗಿತ್ತು
ನಮ್ಮ-ದಾ-ಗಿಯೇ
ನಮ್ಮ-ದಾಗಿ-ರು-ವುದು
ನಮ್ಮ-ದಾ-ಗು-ವುದು
ನಮ್ಮದು
ನಮ್ಮದೇ
ನಮ್ಮನ್ನು
ನಮ್ಮನ್ನೆ
ನಮ್ಮನ್ನೆಲ್ಲ
ನಮ್ಮನ್ನೆಲ್ಲಾ
ನಮ್ಮಲ್ಲಿ
ನಮ್ಮಲ್ಲಿದೆ
ನಮ್ಮಲ್ಲಿದ್ದರೂ
ನಮ್ಮಲ್ಲಿಯೇ
ನಮ್ಮಲ್ಲಿ-ರ-ಬೇಕು
ನಮ್ಮಲ್ಲಿರು
ನಮ್ಮಲ್ಲಿ-ರುವ
ನಮ್ಮಲ್ಲಿ-ರು-ವಂತಹ
ನಮ್ಮಲ್ಲಿ-ರು-ವಾಗ
ನಮ್ಮಲ್ಲಿ-ರು-ವುದು
ನಮ್ಮಲ್ಲಿ-ರು-ವುವು
ನಮ್ಮಲ್ಲೇ
ನಮ್ಮವು
ನಮ್ಮಷ್ಟು
ನಮ್ಮಾತ್ಮವೇ
ನಮ್ಮಿಂದ
ನಮ್ಮಿಂದಲೇ
ನಮ್ಮಿಚ್ಛೆ-ಯಂತೆ
ನಮ್ಮಿಬ್ಬರ
ನಮ್ಮೆ
ನಮ್ಮೆ-ದುರಿ-ಗಿ-ರುವ
ನಮ್ಮೆ-ದು-ರಿಗೆ
ನಮ್ಮೆಲ್ಲರ
ನಮ್ಮೆಲ್ಲ-ರಿಗೂ
ನಮ್ಮೊ
ನಮ್ಮೊಂದಿಗೆ
ನಮ್ಮೊ-ಡನಿರ
ನಮ್ಮೊ-ಳಗೆ
ನಮ್ರ-ನಾಗಿ
ನಯನೇಂದ್ರಿ-ಯದ
ನಯ-ವಾಗಿ
ನಯ-ವಾ-ಗಿಲ್ಲ
ನರ
ನರಕ
ನರ-ಕ-ಕುಂಡ-ವಾ-ಗು-ವುದು
ನರ-ಕಕ್ಕೂ
ನರ-ಕಕ್ಕೆ
ನರ-ಕ-ಗಳ
ನರ-ಕದ
ನರ-ಕ-ದಂತೆ
ನರ-ಕ-ದಲ್ಲಿ
ನರ-ಕ-ಯಾ-ತನೆ
ನರ-ಕ-ವಾ-ಗಲಿ
ನರ-ಕ-ವಾಗಿತ್ತೋ
ನರ-ಕ-ವಿ-ದೆಯೋ
ನರ-ಕ-ವಿರ-ಲಾರದು
ನರ-ಕವೂ
ನರ-ಗಳ
ನರ-ಗ-ಳನ್ನು
ನರ-ಗಳಿಂದ
ನರ-ಗಳಿವೆ
ನರ-ಗಳು
ನರ-ಗಳೂ
ನರ-ಗಳೆಂಬ
ನರ-ಗಳೇ
ನರಗ್ರಂಥಿ-ಗಳು
ನರ-ಜಾಲ
ನರ-ತಂತು-ವಿನ
ನರದ
ನರನೇ
ನರಪ್ರವಾಹವು
ನರ-ಭಕ್ಷಕ
ನರ-ಭಕ್ಷ-ಕರ
ನರ-ಮಾಂಸ-ಭಕ್ಷಣೆಯ
ನರ-ಳ-ಲಾರ
ನರಳಿ
ನರ-ಳಿಯೋ
ನರಳು
ನರ-ಳುತ್ತಿ-ರುವ
ನರ-ಳುತ್ತಿ-ರು-ವೆವು
ನರ-ಳುವ
ನರ-ಳು-ವರು
ನರ-ಳು-ವಾಗ
ನರ-ಳು-ವುದು
ನರ-ವನ್ನು
ನರವು
ನರ್ತನ
ನಲಿಯು-ವುದು
ನಲ್ಲದೆ
ನಲ್ಲಾಗುತ್ತಿ-ರುವ
ನಲ್ಲಿ
ನಲ್ಲಿ-ಡ-ಬೇಕು
ನಲ್ಲಿತ್ತು
ನಲ್ಲಿಯೂ
ನಲ್ಲಿ-ರುವ
ನಲ್ಲಿ-ರುವಾಗ
ನವ
ನವನು
ನವ-ರಾಗು-ವು-ದಕ್ಕೆ
ನವ-ಶಕ್ತಿ
ನವಾನ್ವೇಷಣೆ-ಗಳನ್ನು
ನವಿಲ್ಲ
ನವೀನ
ನವೆಂಬರ್
ನವೆ-ಯನ್ನುಂಟು-ಮಾಡು-ವುದು
ನಶಿ-ಸುತ್ತದೆ
ನಶ್ವರ
ನಶ್ವರ-ದಲ್ಲಿ
ನಶ್ವರ-ವಾದ
ನಷ್ಟ
ನಷ್ಟ-ಮಪ್ಯ-ನಷ್ಟಂ
ನಷ್ಟ-ವನ್ನುಂಟು
ನಷ್ಟ-ವಲ್ಲ
ನಷ್ಟ-ವಾದುವು
ನಷ್ಟ-ವಿಲ್ಲ
ನಷ್ಟವೇ
ನಾಂತರ-ದಲ್ಲಿ
ನಾಕ
ನಾಗರಿ-ಕ-ತೆಯ
ನಾಗರಿ-ಕ-ತೆಯೆ
ನಾಗರಿ-ಕ-ವಾಗಿ-ರ-ಬಹುದು
ನಾಗಲೀ
ನಾಗಲು
ನಾಗಿ
ನಾಗಿದ್ದರೆ
ನಾಗಿದ್ದಾಗ
ನಾಗಿ-ರ-ಬಹುದು
ನಾಗಿ-ರ-ಬೇಕು
ನಾಗಿ-ರುವ
ನಾಗಿಲ್ಲ
ನಾಗುತ್ತಾನೆ
ನಾಗು-ವನು
ನಾಚಿ-ಕೆಗೇಡು
ನಾಚಿಕೆ-ಯಿಲ್ಲದೆ
ನಾಜೂ-ಕಾಗಿ
ನಾಜೂ-ಕಾಗಿ-ರುತ್ತದೆ
ನಾಜೂಕಾದ
ನಾಜೂಕು
ನಾಡ-ಲಾರೆವು
ನಾಡಲು
ನಾಡಿ-ಗಳ
ನಾಡಿ-ಗ-ಳನ್ನು
ನಾಡಿ-ಗಳೊ-ಳಗೆ
ನಾಡಿದ್ದು
ನಾಡಿ-ಯನ್ನು
ನಾಡುತ್ತಾನೆ
ನಾಡು-ವು-ದಲ್ಲ
ನಾಣ್ನುಡಿ
ನಾಣ್ನುಡಿ-ಯಿ-ರು-ವುದು
ನಾಣ್ಯದ
ನಾದ
ನಾದಕ್ಕೆ
ನಾದರೂ
ನಾದ-ವನು
ನಾನ-ದನ್ನು
ನಾನರಿತೆ
ನಾನಲ್ಲ
ನಾನಲ್ಲದೆ
ನಾನಲ್ಲ-ವೆನ್ನು-ವುದು
ನಾನಾ
ನಾನಾ-ಗಲಿ
ನಾನಾ-ಗು-ವುದು
ನಾನಾ-ರಿಗೆ
ನಾನಾ-ವಿಧದ
ನಾನಿದ್ದೇನೆ
ನಾನಿನ್ನು
ನಾನಿನ್ನೂ
ನಾನೀಗ
ನಾನು
ನಾನೂ
ನಾನೂರು
ನಾನೆ
ನಾನೆಂದಿಗೂ
ನಾನೆಂದು
ನಾನೆಂದೂ
ನಾನೆಂಬ
ನಾನೆಂಬು-ದರ
ನಾನೆಂಬುದು
ನಾನೆಂಬುವ
ನಾನೆಷ್ಟು
ನಾನೇ
ನಾನೇಕೆ
ನಾನೇನು
ನಾನೊಂದು
ನಾನೊಬ್ಬ
ನಾನೊಮ್ಮೆ
ನಾಭಿ-ಚಕ್ರೇ
ನಾಭಿಯ
ನಾಮ
ನಾಮ-ರೂಪ
ನಾಮ-ರೂಪ-ಗಳ
ನಾಮ-ರೂಪ-ಗ-ಳನ್ನು
ನಾಮ-ರೂಪ-ಗಳಾಗಿ
ನಾಮ-ರೂಪ-ಗಳಿಂದ
ನಾಮ-ರೂಪ-ಗಳಿವೆ
ನಾಮ-ರೂಪ-ಗಳು
ನಾಮ-ರೂಪ-ಗಳುಳ್ಳ
ನಾಯಿ
ನಾಯಿ-ಗಳು
ನಾಯಿ-ಗಿಂತ
ನಾಯಿಗೆ
ನಾಯಿ-ಮರಿ-ಗಳಂತೆ
ನಾಯಿಯ
ನಾಯಿ-ಯಂತೆ
ನಾಯಿ-ಯನ್ನು
ನಾಯಿ-ಯಷ್ಟು
ನಾಯಿಯೆ
ನಾರದ
ನಾರದ-ಋಷಿಯು
ನಾರದ-ನನ್ನು
ನಾರ-ದನು
ನಾರದ-ರಿಗೆ
ನಾರ-ದರು
ನಾರ-ದರೆ
ನಾರಿ-ಯರು
ನಾರುತ್ತಿ-ರುವ
ನಾರುವ
ನಾರು-ವುದು
ನಾರ್ವೆ-ದೇಶದ
ನಾರ್ವೆಯ-ವರು
ನಾಲ-ಗೆಯ
ನಾಲಗೆ-ಯನ್ನು
ನಾಲೆಯ
ನಾಲೆಯೇ
ನಾಲ್ಕ-ನೆಯ-ದ-ರಲ್ಲಿ
ನಾಲ್ಕ-ನೆ-ಯ-ದಾಗಿ
ನಾಲ್ಕ-ನೆ-ಯದು
ನಾಲ್ಕನೆ-ಯದೆ
ನಾಲ್ಕ-ನೆ-ಯದೇ
ನಾಲ್ಕನೇ
ನಾಲ್ಕ-ರಿಂದ
ನಾಲ್ಕು
ನಾಳದ
ನಾಳವು
ನಾಳವೇ
ನಾಳೆ
ನಾಳೆಯ
ನಾಳೆಯೇ
ನಾಳೆಯೊ
ನಾಳೆಯೋ
ನಾವಲ್ಲ
ನಾವಷ್ಟು
ನಾವಾ-ಗಲೇ
ನಾವಾಗಿ
ನಾವಿಂದು
ನಾವಿತ್ತ
ನಾವಿನ್ನು
ನಾವಿನ್ನೂ
ನಾವಿ-ರುವ
ನಾವಿಲ್ಲಿ
ನಾವೀಗ
ನಾವು
ನಾವು-ಗಳು
ನಾವು-ಗಳೆಲ್ಲ
ನಾವು-ಗಳೆಲ್ಲರೂ
ನಾವೂ
ನಾವೆ
ನಾವೆಂದಿಗೂ
ನಾವೆಂದು
ನಾವೆ-ಯನ್ನು
ನಾವೆಲ್ಲ
ನಾವೆಲ್ಲರೂ
ನಾವೆಲ್ಲಾ
ನಾವೆಷ್ಟೇ
ನಾವೇ
ನಾವೇಕೆ
ನಾವೇ-ನನ್ನೂ
ನಾವೇ-ನಾ-ಗಿ-ರು-ವೆವೊ
ನಾವೇನು
ನಾವೇನೂ
ನಾವೇನೊ
ನಾವೊಂದು
ನಾಶ
ನಾಶಕ್ಕೆ
ನಾಶ-ಗಳಿಲ್ಲ
ನಾಶ-ಗೊಂಡು
ನಾಶ-ಗೊಳಿ-ಸು-ವಂತೆ
ನಾಶದ
ನಾಶ-ದಂತೆಯೇ
ನಾಶ-ದಲ್ಲಿ
ನಾಶ-ದಿಂದ
ನಾಶ-ಮಾಡದೆ
ನಾಶ-ಮಾಡ-ಬಲ್ಲದು
ನಾಶ-ಮಾಡ-ಬಲ್ಲೆ
ನಾಶ-ಮಾಡ-ಬೇಡಿ
ನಾಶ-ಮಾಡ-ಲಾಗು-ವು-ದಿಲ್ಲ
ನಾಶ-ಮಾಡ-ಲಾರದು
ನಾಶ-ಮಾಡ-ಲಾರವು
ನಾಶ-ಮಾಡ-ಲಾರಿರಿ
ನಾಶ-ಮಾಡ-ಲಾರಿರೊ
ನಾಶ-ಮಾಡ-ಲಾರೆ
ನಾಶ-ಮಾಡ-ಲಾರೆವು
ನಾಶ-ಮಾ-ಡಲು
ನಾಶ-ಮಾಡಿ
ನಾಶ-ಮಾಡಿದ
ನಾಶ-ಮಾಡಿದೆ
ನಾಶ-ಮಾಡಿವೆ
ನಾಶ-ಮಾಡುತ್ತದೆ
ನಾಶ-ಮಾಡುತ್ತವೆ
ನಾಶ-ಮಾಡುತ್ತೀರಿ
ನಾಶ-ಮಾಡುನು
ನಾಶ-ಮಾಡುವ
ನಾಶ-ಮಾಡು-ವರು
ನಾಶ-ಮಾಡು-ವಿರಿ
ನಾಶ-ಮಾಡು-ವು-ದಿಲ್ಲ
ನಾಶ-ಮಾಡು-ವುದು
ನಾಶ-ವಾಗ
ನಾಶ-ವಾಗದ
ನಾಶ-ವಾಗದು
ನಾಶ-ವಾಗದೆ
ನಾಶ-ವಾಗ-ಬೇಕು
ನಾಶ-ವಾಗ-ಲಾರದು
ನಾಶ-ವಾಗ-ಲಾರನು
ನಾಶ-ವಾಗ-ಲೂ-ಬಹುದು
ನಾಶ-ವಾ-ಗಲೇ
ನಾಶ-ವಾಗ-ಲೇ-ಬೇಕು
ನಾಶ-ವಾಗಿ
ನಾಶ-ವಾಗಿದೆ
ನಾಶ-ವಾ-ಗಿಲ್ಲ
ನಾಶ-ವಾಗು
ನಾಶ-ವಾಗುತ್ತದೆ
ನಾಶ-ವಾಗುತ್ತವೆ
ನಾಶ-ವಾಗುತ್ತ-ವೆ-ಯೆಂದಾ
ನಾಶ-ವಾಗುತ್ತಿವೆ
ನಾಶ-ವಾಗುವ
ನಾಶ-ವಾಗು-ವರು
ನಾಶ-ವಾಗು-ವು-ದ-ರಿಂದ
ನಾಶ-ವಾಗು-ವು-ದಿಲ್ಲ
ನಾಶ-ವಾ-ಗು-ವುದು
ನಾಶ-ವಾ-ಗು-ವುದು-ಈ-ಗಿ-ರುವ
ನಾಶ-ವಾಗು-ವು-ದೆಂದು
ನಾಶ-ವಾ-ಗು-ವುವು
ನಾಶ-ವಾದ
ನಾಶ-ವಾ-ದಂತೆ
ನಾಶ-ವಾ-ದರೂ
ನಾಶ-ವಾ-ದರೆ
ನಾಶ-ವಾದುವು
ನಾಶ-ವಾದೊ-ಡ-ನೆಯೆ
ನಾಶವು
ನಾಶವೂ
ನಾಶವೆ
ನಾಶ-ವೆಂದರೆ
ನಾಶ-ವೆನ್ನು-ವುದು
ನಾಶವೇ
ನಾಸ್ತಿ
ನಾಸ್ತಿಕ
ನಾಸ್ತಿ-ಕತೆ
ನಾಸ್ತಿ-ಕ-ನಾ-ಗಿ-ರುವುದು
ನಾಸ್ತಿ-ಕನು
ನಾಸ್ತಿ-ಕ-ನೆನ್ನುತ್ತಿದ್ದುವು
ನಾಸ್ತಿ-ಕ-ರಾ-ಗು-ವುದು
ನಾಸ್ತಿ-ಕ-ರಿಗೂ
ನಾಸ್ತಿ-ಕರು
ನಾಸ್ತಿ-ಕರೂ
ನಾಸ್ತಿ-ಕರೆ
ನಾಸ್ತಿ-ಕ-ರೆಂದು
ನಾಸ್ತಿ-ಕರೇ
ನಾಸ್ತಿ-ಕರೊ
ನಾಸ್ತಿ-ಗಳಿಲ್ಲದ
ನಾಸ್ತಿತ್ವವೂ
ನಾಸ್ತಿ-ಮಾರ್ಗ-ವನ್ನು
ನಿಂತ
ನಿಂತರೆ
ನಿಂತಾಗ
ನಿಂತಿದೆ
ನಿಂತಿದ್ದಂತೆಯೇ
ನಿಂತಿದ್ದರೆ
ನಿಂತಿದ್ದೆವು
ನಿಂತಿ-ರ-ಬೇಕು
ನಿಂತಿರು
ನಿಂತಿ-ರುವ
ನಿಂತಿರು-ವನು
ನಿಂತಿರು-ವುದನ್ನೇ
ನಿಂತಿರು-ವುದು
ನಿಂತಿರು-ವುದೇ
ನಿಂತಿರು-ವುವು
ನಿಂತಿರು-ವೆವು
ನಿಂತಿಲ್ಲ
ನಿಂತಿಲ್ಲ-ವೆಂದು
ನಿಂತಿವೆ
ನಿಂತು
ನಿಂತು-ಕೊಂಡು
ನಿಂತು-ಕೊಂಡೇ
ನಿಂತು-ಕೊಳ್ಳ-ಲಾರವು
ನಿಂತು-ಕೊಳ್ಳಿ
ನಿಂತು-ಬಿಟ್ಟನು
ನಿಂತು-ಹೋ-ಗು-ವುದು
ನಿಂತೊ-ಡನೆ
ನಿಂದ
ನಿಂದಾರ್ಹರು
ನಿಂದಿ-ಸದೆ
ನಿಂದಿಸ-ಬೇಕಾ-ಗಿಲ್ಲ
ನಿಂದಿಸಿ-ದರೂ
ನಿಂದಿಸಿ-ದರೆ
ನಿಂದಿಸಿ-ದ-ರೆಂದು
ನಿಂದಿಸಿ-ದಾಗ
ನಿಂದಿಸುತ್ತಾನೆ
ನಿಂದಿ-ಸುವ
ನಿಂದೆ
ನಿಂದೆಯ
ನಿಂದೆ-ಯನ್ನೆಲ್ಲಾ
ನಿಂದೆ-ಯಲ್ಲವೇ
ನಿಃ
ನಿಃಶ್ವಾಸ-ಗಳಂತೆ
ನಿಃಸ್ವಾರ್ಥ
ನಿಃಸ್ವಾರ್ಥತೆ
ನಿಃಸ್ವಾರ್ಥ-ತೆ-ಯನ್ನು
ನಿಃಸ್ವಾರ್ಥ-ತೆ-ಯಿಂದ
ನಿಃಸ್ವಾರ್ಥ-ದಂತಹ
ನಿಃಸ್ವಾರ್ಥ-ದಿಂದ
ನಿಃಸ್ವಾರ್ಥ-ನಾಗ
ನಿಃಸ್ವಾರ್ಥ-ನಾಗುವ
ನಿಃಸ್ವಾರ್ಥನು
ನಿಃಸ್ವಾರ್ಥ-ಪರ-ನಾಗೆಂದು
ನಿಃಸ್ವಾರ್ಥ-ಪರನೆ
ನಿಃಸ್ವಾರ್ಥಪ್ರೇಮ-ವೆಂದು
ನಿಃಸ್ವಾರ್ಥರೂ
ನಿಃಸ್ವಾರ್ಥ-ವೊಂದೇ
ನಿಃಸ್ವಾರ್ಥಿ-ಗಳಾ-ಗ-ಬೇಕು
ನಿಃಸ್ವಾರ್ಥಿ-ಗಳಾ-ಗಿ-ರು-ವುದುಈ
ನಿಃಸ್ವಾರ್ಥಿ-ಯಾದ
ನಿಕ
ನಿಕ-ಟತೆ-ಯನ್ನು
ನಿಕ-ಟ-ವಾಗಿ
ನಿಕ-ಟ-ಸಂಬಂಧ-ವಿದೆ
ನಿಕೃಷ್ಟ
ನಿಕೃಷ್ಟ-ದೃಷ್ಟಿ
ನಿಕೃಷ್ಟ-ದೃಷ್ಟಿ-ಯಿಂದ
ನಿಗೂ
ನಿಗೆ
ನಿಗೇನೂ
ನಿಗ್ರ
ನಿಗ್ರಹ
ನಿಗ್ರ-ಹಕ್ಕೆ
ನಿಗ್ರ-ಹಕ್ಕೊಳ-ಪಟ್ಟು
ನಿಗ್ರ-ಹದ
ನಿಗ್ರ-ಹ-ದಲ್ಲಿ
ನಿಗ್ರ-ಹ-ದಿಂದ
ನಿಗ್ರ-ಹ-ವನ್ನು
ನಿಗ್ರ-ಹ-ವನ್ನೂ
ನಿಗ್ರ-ಹ-ವನ್ನೇ
ನಿಗ್ರ-ಹ-ವಾ-ಗು-ವುದು
ನಿಗ್ರ-ಹ-ವಿದೆ
ನಿಗ್ರ-ಹವೇ
ನಿಗ್ರ-ಹ-ಶಕ್ತಿ
ನಿಗ್ರಹಿಸ
ನಿಗ್ರಹಿಸ-ದಿದ್ದಾಗ
ನಿಗ್ರಹಿ-ಸದೆ
ನಿಗ್ರಹಿಸ-ಬಲ್ಲ
ನಿಗ್ರಹಿಸ-ಬಲ್ಲೆ-ವೆಂದು
ನಿಗ್ರಹಿಸ-ಬಹುದು
ನಿಗ್ರಹಿಸ-ಬಹು-ದೆಂದು
ನಿಗ್ರಹಿಸ-ಬೇಕಾ-ದರೆ
ನಿಗ್ರಹಿಸ-ಬೇಕು
ನಿಗ್ರಹಿಸ-ಬೇಕೆಂದಿ
ನಿಗ್ರಹಿಸ-ಬೇಕೆಂದು
ನಿಗ್ರಹಿಸ-ಬೇಕೆಂಬು-ದೊಂದೇ
ನಿಗ್ರಹಿಸ-ಲ-ದಳ-ವಾದ
ನಿಗ್ರಹಿಸ-ಲಾಗು-ವು-ದಿಲ್ಲ
ನಿಗ್ರಹಿಸ-ಲಾರದ
ನಿಗ್ರಹಿ-ಸಲು
ನಿಗ್ರಹಿಸಲ್ಪಟ್ಟ
ನಿಗ್ರಹಿಸಿ
ನಿಗ್ರಹಿ-ಸಿದ
ನಿಗ್ರಹಿಸಿ-ದರೆ
ನಿಗ್ರಹಿಸಿ-ದ-ವ-ನಿಗೆ
ನಿಗ್ರಹಿಸಿ-ರು-ವರೋ
ನಿಗ್ರಹಿಸು
ನಿಗ್ರಹಿಸುತ್ತಾನೆ
ನಿಗ್ರಹಿಸುತ್ತಾರೆ
ನಿಗ್ರಹಿಸುತ್ತಾರೆಯೋ
ನಿಗ್ರಹಿ-ಸುವ
ನಿಗ್ರಹಿಸು-ವಂತಹ
ನಿಗ್ರಹಿಸು-ವಂತೆ
ನಿಗ್ರಹಿಸು-ವಷ್ಟು
ನಿಗ್ರಹಿ-ಸು-ವು-ದಕ್ಕೆ
ನಿಗ್ರ-ಹಿಸು-ವು-ದನ್ನು
ನಿಗ್ರಹಿಸು-ವು-ದ-ರಲ್ಲಿ
ನಿಗ್ರಹಿಸು-ವು-ದ-ರಿಂದ
ನಿಗ್ರಹಿಸು-ವುದು
ನಿಗ್ರಹಿಸು-ವುದೇ
ನಿಜ
ನಿಜತ್ವ-ವನ್ನು
ನಿಜ-ವಲ್ಲ
ನಿಜ-ವಲ್ಲ-ವೆಂದು
ನಿಜ-ವಲ್ಲ-ವೆಂದೂ
ನಿಜ-ವಾಗ-ಲೇ-ಬೇಕು
ನಿಜ-ವಾಗಿ
ನಿಜ-ವಾಗಿಯೂ
ನಿಜ-ವಾಗಿ-ರ-ಬಹುದು
ನಿಜ-ವಾ-ಗಿ-ರುವುದು
ನಿಜ-ವಾಗುತ್ತಿತ್ತು
ನಿಜ-ವಾದ
ನಿಜ-ವಾ-ದರೂ
ನಿಜ-ವಾ-ದರೆ
ನಿಜ-ವಾ-ಯಿತು
ನಿಜ-ವಿದ್ದರೆ
ನಿಜ-ವಿರ
ನಿಜ-ವೆಂದು
ನಿಜ-ವೆಂಬು-ದನ್ನು
ನಿಜವೇ
ನಿಜಸ್ಥಿತಿ
ನಿಜಸ್ಥಿತಿ-ಗಿಂತ
ನಿಜಸ್ಥಿತಿಯ
ನಿಜಸ್ವಭಾ-ವ-ವೆಂದೂ
ನಿಜಸ್ವ-ರೂಪವು
ನಿಟ್ಟು-ಸಿರು-ಬಿಟ್ಟ
ನಿತ್ಯ
ನಿತ್ಯ-ಜೀವ-ನ-ದಲ್ಲಿ
ನಿತ್ಯ-ತೃಪ್ತ-ರಾಗಿ
ನಿತ್ಯತ್ವಾತ್
ನಿತ್ಯ-ಪರಿ-ಪೂರ್ಣವೂ
ನಿತ್ಯ-ಪವಿತ್ರನೂ
ನಿತ್ಯ-ಪೂಜಿ-ತನು
ನಿತ್ಯ-ಪೂರ್ಣ
ನಿತ್ಯ-ಮುಕ್ತ
ನಿತ್ಯ-ಮುಕ್ತತೆ
ನಿತ್ಯ-ಮುಕ್ತ-ನಿಗೆ
ನಿತ್ಯ-ಮುಕ್ತ-ನೆನ್ನುವು
ನಿತ್ಯ-ಮುಕ್ತ-ರಾಗಿ-ರ-ಬೇಕು
ನಿತ್ಯ-ಮುಕ್ತರು
ನಿತ್ಯ-ಮುಕ್ತ-ವಾಗಿದ್ದರೆ
ನಿತ್ಯ-ಯೋಗ
ನಿತ್ಯ-ವನ್ನು
ನಿತ್ಯ-ವಸ್ತು
ನಿತ್ಯ-ವಾಗದು
ನಿತ್ಯ-ವಾಗಿ-ರು-ವು-ದ-ರಿಂದ
ನಿತ್ಯ-ವಾ-ಗಿ-ರು-ವುವು
ನಿತ್ಯ-ವಾದ
ನಿತ್ಯ-ವಾದುದು
ನಿತ್ಯ-ಶುಚಿ-ಸುಖಾತ್ಮಖ್ಯಾತಿ-ರ-ವಿದ್ಯಾ
ನಿತ್ಯ-ಶುದ್ಧ
ನಿತ್ಯ-ಶುದ್ಧತೆ
ನಿತ್ಯ-ಶುದ್ಧನು
ನಿತ್ಯ-ಶುದ್ಧನೆ
ನಿತ್ಯ-ಶುದ್ಧ-ಮಕ್ತಸ್ವಭಾ-ವ-ದ-ವನು
ನಿತ್ಯ-ಶುದ್ಧರು
ನಿತ್ಯ-ಶುದ್ಧರೂ
ನಿತ್ಯ-ಶುದ್ಧ-ವಾದ
ನಿತ್ಯ-ಸತ್ಯ-ವಾದ
ನಿತ್ಯ-ಸಾಕ್ಷಿ
ನಿತ್ಯ-ಸುಖಿ-ಗಳಾಗಿ
ನಿತ್ಯಸ್ವ-ತಂತ್ರ
ನಿತ್ಯಸ್ವ-ತಂತ್ರನು
ನಿತ್ಯಸ್ವ-ತಂತ್ರ-ನು-ಗುರು-ಗಳ
ನಿತ್ಯಸ್ವರ್ಗ
ನಿತ್ಯಸ್ವರ್ಗ-ವಿಲ್ಲ
ನಿತ್ಯಾ-ನಂದ
ನಿತ್ಯಾ-ನಂದ-ದಲ್ಲಿ
ನಿತ್ಯಾ-ನಂದ-ಮ-ಯವೂ
ನಿತ್ಯಾ-ನಂದ-ಮಯಸ್ವ-ರೂಪ
ನಿತ್ಯಾ-ನಂದ-ವಾಗಿ
ನಿತ್ಯಾ-ನಿತ್ಯ
ನಿತ್ಯಾ-ನಿತ್ಯ-ವಸ್ತು
ನಿತ್ಯೋಪ-ಕಾರಿಯು
ನಿದರ್ಶಿಸಿ
ನಿದಿಧ್ಯಾಸನ
ನಿದಿಷ್ಟ-ವಾದ
ನಿದ್ದೆ
ನಿದ್ದೆ-ಮಾಡ
ನಿದ್ದೆ-ಮಾಡ-ಬೇಕೆಂದು
ನಿದ್ದೆ-ಮಾಡೋಣ
ನಿದ್ದೆ-ಯಲ್ಲೇ
ನಿದ್ರಾ-ಕಾಲ-ದಲ್ಲಿ
ನಿದ್ರಿ-ಸುತ್ತಿದೆ
ನಿದ್ರಿಸುತ್ತಿ-ರು-ವಾ-ಗಲೂ
ನಿದ್ರಿಸುತ್ತಿವೆ
ನಿದ್ರಿ-ಸು-ವನು
ನಿದ್ರೆ
ನಿದ್ರೆ-ಗಳಿಂದ
ನಿದ್ರೆಯ
ನಿದ್ರೆ-ಯನ್ನು
ನಿದ್ರೆ-ಯಲ್ಲಿ
ನಿದ್ರೆ-ಯಲ್ಲಿಯೂ
ನಿದ್ರೆ-ಯಿಂದ
ನಿದ್ರೆ-ಯೆಂಬ
ನಿಧಾನ
ನಿಧಾನ-ವಾಗಿ
ನಿಧಾನ-ವಾಗಿಯೊ
ನಿಧಿ
ನಿಧಿ-ಯನ್ನು
ನಿನ-ಗಾಗಿ
ನಿನ-ಗಿಂತ
ನಿನಗೆ
ನಿನ-ಗೇನೂ
ನಿನ್ನ
ನಿನ್ನದು
ನಿನ್ನ-ದೆಂಬುದೆ
ನಿನ್ನನ್ನು
ನಿನ್ನಲ್ಲಿ
ನಿನ್ನಲ್ಲಿ-ರದೇ
ನಿನ್ನಲ್ಲಿ-ರುವ
ನಿನ್ನಲ್ಲಿ-ರುವನು
ನಿನ್ನಲ್ಲಿ-ರುವುದು
ನಿನ್ನಲ್ಲೇ
ನಿನ್ನವು
ನಿನ್ನೆ
ನಿನ್ನೆ-ಡೆಗೆ
ನಿನ್ನೆಯ
ನಿಪುಣ-ರಾಗು-ವಂತಹ
ನಿಪುಣ-ರಾ-ದರೆ
ನಿಪುಣರು
ನಿಪುಣರೊ
ನಿಪುಣ-ವಾಗು-ವು-ದೆಂದು
ನಿಮ-ಗಾಗಿ
ನಿಮ-ಗಿ-ರುವ
ನಿಮ-ಗಿಲ್ಲ
ನಿಮಗೂ
ನಿಮಗೆ
ನಿಮ-ಗೆಲ್ಲ
ನಿಮಗೆಲ್ಲ-ರಿಗೂ
ನಿಮ-ಗೆಲ್ಲಾ
ನಿಮಗೇ
ನಿಮಗೇ-ನಂತೆ
ನಿಮ-ಗೇನೂ
ನಿಮಗೊಂದು
ನಿಮಿತ್ತ
ನಿಮಿತ್ತ-ಕಾರಣ
ನಿಮಿತ್ತ-ಗಳಿಂದ
ನಿಮಿತ್ತ-ಗಳು
ನಿಮಿತ್ತ-ಮಪ್ರಯೋ-ಜಕಂ
ನಿಮಿತ್ತ-ವಾಗಿ
ನಿಮಿಷ
ನಿಮಿ-ಷಕ್ಕೆ
ನಿಮಿಷ-ಗಳ-ವರೆಗೂ
ನಿಮಿಷ-ವಾ-ದರೂ
ನಿಮಿ-ಷವೂ
ನಿಮ್ಮ
ನಿಮ್ಮಂತಹ
ನಿಮ್ಮಂತೆ
ನಿಮ್ಮಂತೆಯೆ
ನಿಮ್ಮಂತೆಯೇ
ನಿಮ್ಮ-ಗಳ
ನಿಮ್ಮ-ದಕ್ಕಿಂತ
ನಿಮ್ಮ-ದ-ರಂತೆಯೇ
ನಿಮ್ಮ-ದಾ-ಗಿತ್ತು
ನಿಮ್ಮ-ದಾ-ಗು-ವುದು
ನಿಮ್ಮದು
ನಿಮ್ಮದೇ
ನಿಮ್ಮನ್ನು
ನಿಮ್ಮನ್ನೆಲ್ಲಾ
ನಿಮ್ಮನ್ನೇ
ನಿಮ್ಮಲ್ಲಿ
ನಿಮ್ಮಲ್ಲಿಗೇ
ನಿಮ್ಮಲ್ಲಿದೆ
ನಿಮ್ಮಲ್ಲಿ-ರಲಿ
ನಿಮ್ಮಲ್ಲಿರು
ನಿಮ್ಮಲ್ಲಿ-ರುವ
ನಿಮ್ಮಲ್ಲಿ-ರು-ವನು
ನಿಮ್ಮಲ್ಲಿ-ರು-ವುದು
ನಿಮ್ಮಲ್ಲೆ
ನಿಮ್ಮ-ವ-ನಿಗೆ
ನಿಮ್ಮ-ವನು
ನಿಮ್ಮಾತ್ಮ
ನಿಮ್ಮಿಂದ
ನಿಮ್ಮಿಂದಲೇ
ನಿಮ್ಮೆ-ದು-ರಿಗೆ
ನಿಮ್ಮೊಂದಿಗೆ
ನಿಮ್ಮೊ-ಡನೆ
ನಿಯಂತ್ರಣ
ನಿಯಂತ್ರ-ಣ-ವಿ-ರು-ವುದು-ಆದ-ಕಾರಣ
ನಿಯಂತ್ರ-ಣ-ಶಕ್ತಿ-ಯನ್ನು
ನಿಯಂತ್ರಿತ-ವಾ-ಗಿ-ರುವುದು
ನಿಯಂತ್ರಿಸ-ಬಲ್ಲಿರಿ
ನಿಯಂತ್ರಿ-ಸ-ಬೇಕು
ನಿಯಂತ್ರಿ-ಸುವ
ನಿಯಂತ್ರಿ-ಸು-ವುದು
ನಿಯಮ
ನಿಯಮ-ಇ-ವನ್ನು
ನಿಯಮಕ್ಕೂ
ನಿಯಮಕ್ಕೆ
ನಿಯಮ-ಗಳ
ನಿಯಮ-ಗ-ಳನ್ನು
ನಿಯಮ-ಗಳಲ್ಲಿ
ನಿಯಮ-ಗಳಾಚೆ
ನಿಯಮ-ಗಳಿಂದ
ನಿಯಮ-ಗಳಿಗೆ
ನಿಯಮ-ಗಳಿದ್ದರೆ
ನಿಯಮ-ಗಳಿವೆ
ನಿಯಮ-ಗಳು
ನಿಯಮ-ಗಳೊ-ಡನೆ
ನಿಯಮದ
ನಿಯಮ-ದಂತೆಯೊ
ನಿಯಮ-ದಲ್ಲಿ
ನಿಯಮ-ದಿಂದ
ನಿಯಮ-ಬದ್ಧ-ವಾಗಿದೆ
ನಿಯಮ-ವನ್ನು
ನಿಯಮ-ವನ್ನೂ
ನಿಯಮ-ವನ್ನೇ
ನಿಯಮ-ವಾ-ಗಿ-ರುವುದು
ನಿಯಮ-ವಾ-ದರೆ
ನಿಯಮ-ವಿ-ದೆಯೊ
ನಿಯಮವು
ನಿಯಮವೂ
ನಿಯಮವೆ
ನಿಯಮ-ವೆಂದರೆ
ನಿಯಮ-ವೆಂದು
ನಿಯಮ-ವೆಂಬು-ದನ್ನು
ನಿಯಮ-ವೆಂಬು-ದಿಲ್ಲ
ನಿಯಮ-ವೆನ್ನುವ
ನಿಯಮವೇ
ನಿಯಮ-ಶಕ್ತಿ
ನಿಯಮಾಃ
ನಿಯಮಾನು
ನಿಯಮಾನು-ಸಾರ
ನಿಯಮಾ-ವಳಿ-ಗ-ಳನ್ನು
ನಿಯಮಾ-ವಳಿ-ಗಳು
ನಿಯಮಾ-ವಳಿ-ಗಳೆಲ್ಲಾ
ನಿಯಮಾ-ವಳಿಗೆ
ನಿಯಮಿತ
ನಿಯು
ನಿರಂಕುಶ
ನಿರಂತರ
ನಿರಂತರ-ವಾಗಿ
ನಿರಂತರ-ವಾಗಿ-ರುವ
ನಿರಂತರ-ವಾದ
ನಿರಂತ-ರವೂ
ನಿರತ-ನಾಗಿ-ರುತ್ತಾನೆ
ನಿರತ-ನಾಗಿ-ರುವನು
ನಿರತ-ನಾಗಿ-ರುವ-ವ-ನಿಗೂ
ನಿರತ-ನಾದ-ವರು
ನಿರತ-ರಾಗ-ಬೇಕು
ನಿರತ-ರಾಗಿ
ನಿರತ-ರಾಗಿದ್ದು-ದನ್ನು
ನಿರತ-ರಾಗಿ-ರಲಿ
ನಿರತ-ರಾಗಿ-ರು-ವರು
ನಿರತ-ರಾಗಿ-ರು-ವರೊ
ನಿರತ-ವಾಗಿದೆ
ನಿರತ-ವಾಗಿ-ರು-ವು-ದ-ರಿಂದ
ನಿರತಿಶಯಂ
ನಿರಪೇಕ್ಷ
ನಿರಪೇಕ್ಷ-ನಾ-ಗಿ-ರುವನೊ
ನಿರಪೇಕ್ಷ-ವಸ್ತು-ಗಳು
ನಿರಪೇಕ್ಷ-ವಾಗಿಯೆ
ನಿರಪೇಕ್ಷ-ವಾದ
ನಿರಪೇಕ್ಷ-ವಾ-ದದ್ದು
ನಿರಪೇಕ್ಷ-ವಾದುದು
ನಿರಪೇಕ್ಷವು
ನಿರಪೇಕ್ಷವೂ
ನಿರರ್ಥಕ
ನಿರರ್ಥಕ-ವಾಗು-ವು-ದ-ರಲ್ಲಿ
ನಿರರ್ಥಕ-ವಾ-ಗು-ವುದು
ನಿರರ್ಥಕ-ವೆಂಬುದು
ನಿರಾ-ಕರ
ನಿರಾ-ಕರಣೆ
ನಿರಾ-ಕರ-ಣೆಯ
ನಿರಾ-ಕರ-ಣೆಯೆ
ನಿರಾ-ಕರಾವೂ
ನಿರಾ-ಕರಿಸ-ಬಲ್ಲರೆ
ನಿರಾ-ಕರಿಸಿ
ನಿರಾ-ಕರಿಸಿ-ದರೆ
ನಿರಾ-ಕರಿಸು-ತ್ತೇವೆ
ನಿರಾ-ಕರಿ-ಸುವ
ನಿರಾ-ಕರಿಸು-ವಷ್ಟು
ನಿರಾ-ಕರಿಸು-ವಿರಾ
ನಿರಾ-ಕರಿಸು-ವು-ದಿಲ್ಲ
ನಿರಾ-ಕಾರ
ನಿರಾ-ಕಾರ-ದೇವರು
ನಿರಾ-ಕಾರ-ನಾಗು-ವನು
ನಿರಾ-ಕಾರನು
ನಿರಾ-ಕಾರ-ವಾದ
ನಿರಾ-ಕಾರ-ವಾದುದು
ನಿರಾ-ಕಾರವು
ನಿರಾ-ಕಾರವೂ
ನಿರಾ-ಕಾರ-ವೆಂದು
ನಿರಾತಂಕ-ವಾ-ಗು-ವುದು
ನಿರಾ-ಧಾರ-ವಾದ
ನಿರಾಶ
ನಿರಾಶ-ನಾಗಿ
ನಿರಾಶ-ರಾಗು-ವುರು
ನಿರಾಶ-ವಾದವೂ
ನಿರಾಶಾ-ಜನ-ಕ-ವಲ್ಲ
ನಿರಾಶಾ-ದಾಯ-ಕ-ವಲ್ಲ
ನಿರಾಶಾ-ವಾದವೂ
ನಿರಾಶಾ-ವಾದಿ
ನಿರಾಶಾ-ವಾದಿ-ಯಾ-ಗು-ವುದು
ನಿರಾಶೆ-ಗಳ
ನಿರಾಶೆಯ
ನಿರಾಶೆ-ಯಾಗಿ-ರ-ಬಹುದು
ನಿರಾಶೆ-ಯಿಂದ
ನಿರಾಸೆ-ಯಿಂದ
ನಿರೀಕ್ಷಿಸ-ಬೇಡಿ
ನಿರೀಕ್ಷಿಸಿ-ದರೂ
ನಿರೀಕ್ಷಿಸಿ-ದಷ್ಟು
ನಿರೀಕ್ಷಿಸುತ್ತವೆ
ನಿರೀಕ್ಷಿ-ಸು-ವುದು
ನಿರುದ್ಧ
ನಿರುದ್ಧಾ-ವಸ್ಥೆ
ನಿರುದ್ಧಾ-ವಸ್ಥೆ-ಯಲ್ಲಿ
ನಿರುಪಕ್ರಮಂ
ನಿರೂಪಣೆ
ನಿರೂಪಿಸು-ವು-ದಕ್ಕೆ
ನಿರೋಧ
ನಿರೋಧಃ
ನಿರೋಧಕ್ಷಣ-ಚಿತ್ತಾನ್ವಯೋ
ನಿರೋಧ-ಪರಿ-ಣಾಮಃ
ನಿರೋಧಿಸ-ಬಲ್ಲ
ನಿರೋಧಿಸ-ಬಹುದು
ನಿರೋಧಿಸಿ-ದಂತೆ
ನಿರೋಧಿಸು
ನಿರೋಧಿಸು-ವುದರ
ನಿರೋಧಿಸು-ವುದು
ನಿರೋಧೇ
ನಿರ್ಗುಣ
ನಿರ್ಗುಣಕ್ಕೆ
ನಿರ್ಗುಣ-ದಲ್ಲಿ
ನಿರ್ಗುಣ-ದೇವರ
ನಿರ್ಗುಣ-ನನ್ನಾಗಿ
ನಿರ್ಗುಣ-ನ-ವನು
ನಿರ್ಗುಣ-ವಾಗಿ
ನಿರ್ಗುಣವು
ನಿರ್ಗುಣವೂ
ನಿರ್ಗುಣ-ವೊಂದೆ
ನಿರ್ಜನ
ನಿರ್ಜೀವ
ನಿರ್ಜೀವ-ವಾದ
ನಿರ್ಜೀವ-ವಾ-ದರೆ
ನಿರ್ಣಯ
ನಿರ್ಣಯಕ್ಕೆ
ನಿರ್ಣಯಕ್ಕೋಸುಗ-ವಾಗಿ
ನಿರ್ಣಯ-ಗ-ಳನ್ನು
ನಿರ್ಣಯ-ಗ-ಳನ್ನೂ
ನಿರ್ಣಯದ
ನಿರ್ಣಯ-ವನ್ನು
ನಿರ್ಣಯ-ವಾ-ಗು-ವುದು
ನಿರ್ಣಯ-ವಿರ-ಲಾರದು
ನಿರ್ಣಯವೇ
ನಿರ್ಣಯ-ವೇ-ನೆಂದರೆ
ನಿರ್ಣ-ಯಿಸಿ-ದರೆ
ನಿರ್ಣ-ಯಿಸು-ವು-ದಕ್ಕೆ
ನಿರ್ಣಾಯ
ನಿರ್ಣಾಯ-ಕ-ವಾಗಿ
ನಿರ್ದಯ-ರಾಗಿ
ನಿರ್ದ-ಯರು
ನಿರ್ದಯ-ವಾಗಿ
ನಿರ್ದಿಷ್ಟ
ನಿರ್ದಿಷ್ಟ-ವಾಗಿ
ನಿರ್ದಿಷ್ಟ-ವಾದ
ನಿರ್ದಿಷ್ಟ-ವಾ-ಯಿತು
ನಿರ್ದೇಶಿತ
ನಿರ್ದೇಶಿ-ಯಾ-ದರೂ
ನಿರ್ದೇಶಿಸಿ
ನಿರ್ದೇಶಿಸುತ್ತವೆ
ನಿರ್ದೇಶಿ-ಸುವ
ನಿರ್ದೇಶಿ-ಸುವು-ದಕ್ಕೆ
ನಿರ್ದೇಶಿ-ಸುವು-ದನ್ನು
ನಿರ್ದೇಶಿ-ಸು-ವುದು
ನಿರ್ದೇಶಿ-ಸು-ವುವು
ನಿರ್ದೇಸಿ-ಬಹುದು
ನಿರ್ದೇಹ-ನಾದ
ನಿರ್ದೇಹಿ
ನಿರ್ದೇಹಿ-ಯಾ-ದರೂ
ನಿರ್ದೋಷ-ವಾದ
ನಿರ್ಧರಿ-ಸಲು
ನಿರ್ಧರಿಸಲ್ಪಟ್ಟ
ನಿರ್ಧರಿಸಲ್ಪಟ್ಟಿದೆ-ದುಃಖ
ನಿರ್ಧರಿಸಿ-ದರೆ
ನಿರ್ಧರಿಸು-ತ್ತೇವೆ
ನಿರ್ಧರಿ-ಸುವ
ನಿರ್ಧರಿಸು-ವುದು
ನಿರ್ಧರಿಸು-ವುವು
ನಿರ್ಧರಿಸು-ವೆವು
ನಿರ್ಧಾರ
ನಿರ್ಧಾರಕ್ಕೂ
ನಿರ್ಧಾ-ರಕ್ಕೆ
ನಿರ್ಧಾರ-ದಿಂದ
ನಿರ್ಧಾರ-ವುಳ್ಳ-ವರು
ನಿರ್ನಾಮ-ವಾಗಿ
ನಿರ್ನಾಮ-ವಾ-ಗು-ವುದು
ನಿರ್ನಾಮ-ವಾದ
ನಿರ್ಬಂಧವೂ
ನಿರ್ಬಂಧಿ-ಸು-ವುದು
ನಿರ್ಬಲ-ತೆಯ
ನಿರ್ಬಲ-ತೆ-ಯನ್ನು
ನಿರ್ಬಲ-ನೆಂದು
ನಿರ್ಬಲ-ನೆಂದೂ
ನಿರ್ಬಲ-ರಾಗಿದ್ದರೆ
ನಿರ್ಬಲರು
ನಿರ್ಬಲ-ರೆಂದೂ
ನಿರ್ಬಲರೋ
ನಿರ್ಬಲ-ವಾಗಿದ್ದರೂ
ನಿರ್ಬಲ-ವಾಗಿದ್ದರೆ
ನಿರ್ಬಲ-ವಾಗುತ್ತವೆ
ನಿರ್ಬಲ-ವಾಗು-ವು-ದನ್ನೂ
ನಿರ್ಬಲ-ವಾಗು-ವು-ದಿಲ್ಲ
ನಿರ್ಬಲ-ವಾ-ಗು-ವುದು
ನಿರ್ಬಲ-ವಾದ
ನಿರ್ಬಲ-ವಾ-ದಷ್ಟೂ
ನಿರ್ಬೀಜ
ನಿರ್ಬೀಜ-ವಾ-ಗು-ವುದು
ನಿರ್ಬೀಜಸ್ಯ
ನಿರ್ಭಯತೆ
ನಿರ್ಭಯ-ನಾಗು-ವನು
ನಿರ್ಭಯ-ವಾ-ದುದು
ನಿರ್ಭಾಸಾ
ನಿರ್ಮಯ-ವಾ-ಗು-ವುದು
ನಿರ್ಮಲ-ಚಿತ್ತ-ರಾದ
ನಿರ್ಮ-ಲತೆ
ನಿರ್ಮಲ-ವಾಗಿ
ನಿರ್ಮಲ-ವಾಗಿ-ರು-ವುದೇ
ನಿರ್ಮಲ-ವಾ-ಗು-ವುದು
ನಿರ್ಮಾಣ
ನಿರ್ಮಾ-ಣಕ್ಕೆ
ನಿರ್ಮಾಣ-ಚಿತ್ತಾನ್ಯಸ್ಮಿತಾ-ಮಾತ್ರಾತ್
ನಿರ್ಮಾ-ಣದ
ನಿರ್ಮಿತ-ವಾಗಿದೆ
ನಿರ್ಮಿತ-ವಾಗಿ-ದೆಯೋ
ನಿರ್ಮಿ-ಸದೆ
ನಿರ್ಮಿ-ಸದೇ
ನಿರ್ಮಿಸ-ಬೇಕಾ-ದರೆ
ನಿರ್ಮಿ-ಸಲು
ನಿರ್ಮಿಸಿ
ನಿರ್ಮಿಸಿ-ಕೊಳ್ಳಿ
ನಿರ್ಮೂಲ
ನಿರ್ಮೂಲ-ಮಾಡಿ
ನಿರ್ಮೂಲ-ವಾ-ದರೆ
ನಿರ್ಲಕ್ಷಿಸು-ವಷ್ಟು
ನಿರ್ವರ್ಣ-ನಾದ
ನಿರ್ವಹ-ಣೆಗೆ
ನಿರ್ವಹಿಸ-ಲಾರರು
ನಿರ್ವಾಣ
ನಿರ್ವಾಣ-ವನ್ನು
ನಿರ್ವಾಹ-ವಿಲ್ಲದೆ
ನಿರ್ವಿ-ಕಲ್ಪ
ನಿರ್ವಿ-ಚಾರ
ನಿರ್ವಿ-ಚಾರಧ್ಯಾನವು
ನಿರ್ವಿ-ಚಾರ-ವೈಶಾ-ರದ್ಯೇಧ್ಯಾತ್ಮಪ್ರಸಾದಃ
ನಿರ್ವಿ-ತರ್ಕ
ನಿರ್ವಿ-ತರ್ಕ-ವೆಂದು
ನಿರ್ವಿ-ತರ್ಕಾ
ನಿರ್ವಿ-ವಾದ-ವಾಗಿ
ನಿರ್ವಿ-ವಾದ-ವಾದ
ನಿರ್ವಿ-ವಾದ-ವಾದುದು
ನಿರ್ವಿ-ಶೇ-ಷವೂ
ನಿಲು-ಕದ
ನಿಲು-ಕಿದೆ
ನಿಲುಕುವ
ನಿಲುಕು-ವಂತೆ
ನಿಲುಕು-ವು-ದಿಲ್ಲವೊ
ನಿಲುವು
ನಿಲ್ಲದೆ
ನಿಲ್ಲ-ಬಲ್ಲ
ನಿಲ್ಲ-ಬಹುದು
ನಿಲ್ಲ-ಬೇಕಾ-ಗಿದೆ
ನಿಲ್ಲ-ಬೇಕಾದ
ನಿಲ್ಲ-ಬೇಡಿ
ನಿಲ್ಲ-ಲಾರ
ನಿಲ್ಲ-ಲಾರದು
ನಿಲ್ಲ-ಲಾರದೋ
ನಿಲ್ಲ-ಲಾರರು
ನಿಲ್ಲ-ಲಾರವು
ನಿಲ್ಲಲಿ
ನಿಲ್ಲಿ
ನಿಲ್ಲಿ-ಸ-ಬಹುದು
ನಿಲ್ಲಿ-ಸ-ಬೇಕಾ-ದರೆ
ನಿಲ್ಲಿ-ಸ-ಬೇಕು
ನಿಲ್ಲಿ-ಸ-ಲಾರರು
ನಿಲ್ಲಿಸಿ
ನಿಲ್ಲಿ-ಸಿತ್ತು
ನಿಲ್ಲಿ-ಸಿ-ದರೆ
ನಿಲ್ಲಿ-ಸಿದೆ
ನಿಲ್ಲಿ-ಸಿ-ರುತ್ತೀರಿ
ನಿಲ್ಲಿಸು
ನಿಲ್ಲಿ-ಸುತ್ತಾನೆ
ನಿಲ್ಲಿ-ಸುತ್ತಾ-ನೆಯೊ
ನಿಲ್ಲಿ-ಸುವ
ನಿಲ್ಲಿ-ಸು-ವ-ವರೆಗೂ
ನಿಲ್ಲಿ-ಸು-ವುದು
ನಿಲ್ಲಿ-ಸು-ವುದೆಂದ-ರೇನು
ನಿಲ್ಲಿ-ಸು-ವುದೇ
ನಿಲ್ಲುತ್ತದೆ
ನಿಲ್ಲುತ್ತವೆ
ನಿಲ್ಲುತ್ತಿ-ರ-ಲಿಲ್ಲ
ನಿಲ್ಲುತ್ತೇನೆ
ನಿಲ್ಲುವ
ನಿಲ್ಲು-ವಂತೆ
ನಿಲ್ಲು-ವರು
ನಿಲ್ಲು-ವು-ದಕ್ಕೆ
ನಿಲ್ಲುವು-ದಾದರೆ
ನಿಲ್ಲು-ವು-ದಿಲ್ಲ
ನಿಲ್ಲು-ವು-ದಿಲ್ಲವೋ
ನಿಲ್ಲು-ವುದು
ನಿಲ್ಲು-ವುದೋ
ನಿಲ್ಲು-ವುವು
ನಿಲ್ಲು-ವೆನು
ನಿಲ್ಲೋಣ
ನಿವಾರಣೆ
ನಿವಾರ-ಣೆಗೆ
ನಿವಾರ-ಣೆ-ಯಾಗಿ
ನಿವಾರ-ಣೋಪಾಯ
ನಿವಾರ-ಣೋಪಾಯ-ವನ್ನು
ನಿವಾರಿಸ
ನಿವಾರಿಸ-ತಕ್ಕದ್ದು
ನಿವಾರಿಸ-ತಕ್ಕದ್ದೊ
ನಿವಾರಿಸ-ಬೇಕಾ
ನಿವಾರಿಸ-ಬೇಕು
ನಿವಾರಿ-ಸಲು
ನಿವಾ-ರಿಸು
ನಿವಾರಿ-ಸುತ್ತಾ
ನಿವೃತ್ತಿ
ನಿವೃತ್ತಿ-ಯಾ-ಗು-ವುದು
ನಿವೃತ್ತಿ-ಯಿಂದ
ನಿವೇದಿಸಿ-ದಾಗ
ನಿಶ್ಚಯ
ನಿಶ್ಚಯ-ವನ್ನು
ನಿಶ್ಚಯ-ವನ್ನೂ
ನಿಶ್ಚಯ-ವಾಗಿ
ನಿಶ್ಚ-ಯಿಸಿ-ದರು
ನಿಶ್ಚಯಿ-ಸುವ
ನಿಶ್ಚಯಿ-ಸುವು-ದಕ್ಕೆ
ನಿಶ್ಚ-ಲತೆ
ನಿಶ್ಚಲ-ವಾಗಿದ್ದರೂ
ನಿಶ್ಚಿತ-ವಾಗಿದೆ
ನಿಶ್ಚಿತ-ವಾದ
ನಿಶ್ಚೇಷ್ಟಿತ
ನಿಶ್ಯಕ್ತಿ
ನಿಶ್ಯಬ್ದ-ವಾಗಿ-ರ-ಬೇಕು-ಇಲ್ಲದೆ
ನಿಷತ್ತಿನ
ನಿಷತ್ತು-ಗಳು
ನಿಷೇಧ
ನಿಷೇಧ-ರೂಪ-ವಾ-ಗಿತ್ತು
ನಿಷೇಧಾತ್ಮಕ
ನಿಷೇಧಿಸಿ
ನಿಷೇಧಿಸು-ವಂತ-ಹದು
ನಿಷೇಧಿ-ಸು-ವು-ದಿಲ್ಲ
ನಿಷ್ಕಪಟ-ವಾದ
ನಿಷ್ಕಪಟಿ
ನಿಷ್ಕಪಟಿ-ಗಳು
ನಿಷ್ಕರ್ಷಿ-ಸ-ಬೇಕು
ನಿಷ್ಕರ್ಷಿಸ-ಲಾಗು-ವು-ದಿಲ್ಲ
ನಿಷ್ಕರ್ಷಿ-ಸು-ವುದು
ನಿಷ್ಕರ್ಷಿ-ಸೋಣ
ನಿಷ್ಕರ್ಷೆ
ನಿಷ್ಕೃಷ್ಟತೆ-ಯಿಂದ
ನಿಷ್ಟ್ರ
ನಿಷ್ಟ್ರ-ಯೋ-ಜಕ
ನಿಷ್ಟ್ರ-ಯೋ-ಜಕ-ವಾದ
ನಿಷ್ಟ್ರ-ಯೋ-ಜನ
ನಿಷ್ಠುರ-ರಾಗಿ
ನಿಷ್ಣಾ-ತರು
ನಿಷ್ಪಕ್ಷ-ಪಾತಿ
ನಿಷ್ಪಲ-ವಾ-ಗು-ವುದು
ನಿಷ್ಪಲ-ವಾಗು-ವು-ದೆಂದೂ
ನಿಷ್ಪ್ರಯೋ-ಜಕ
ನಿಸ್ತೇಜ-ರನ್ನಾಗಿ
ನಿಸ್ತೇಜ-ವಾಗುತ್ತದೆ
ನಿಸ್ಪಂದಿತ-ವಾಗಿ
ನಿಸ್ಸಂದೇಹ
ನಿಸ್ಸಂದೇಹ-ವಾಗಿ
ನಿಸ್ಸಂದೇಹ-ವಾಗಿಯೂ
ನಿಸ್ಸಂದೇಹ-ವಾಗು
ನಿಸ್ಸಂಶಯ-ವಾಗಿ
ನಿಸ್ಸಂಶಯ-ವಾಗಿದೆ
ನಿಸ್ಸಂಶಯ-ವಾಗಿಯೂ
ನಿಸ್ಸಾರ-ವಾದ
ನಿಹಿತ-ವಾ-ಗಿತ್ತು
ನೀಗದೆ
ನೀಚ
ನೀಚತ್ವ-ಇಂತಹ
ನೀಚ-ನಾಗು-ವನು
ನೀಚ-ನಿಂದ
ನೀಚನೋ
ನೀಚ-ರನ್ನಾಗಿ
ನೀಚ-ರಿಗೆ
ನೀಚ-ವಾದುದೆ
ನೀಡದಿ
ನೀಡ-ಬಲ್ಲ-ದೆಂಬುದು
ನೀಡ-ಬಹು-ದಾದ
ನೀಡ-ಬಹುದು
ನೀಡ-ಬೇಕು
ನೀಡ-ಬೇಕೆಂದು
ನೀಡ-ಲಾರದು
ನೀಡ-ಲಾರಿರೊ
ನೀಡ-ಲಾರೆ
ನೀಡ-ಲಿಲ್ಲ
ನೀಡಲು
ನೀಡಿ
ನೀಡಿದ
ನೀಡಿ-ದಾ-ಗಲೇ
ನೀಡಿದ್ದೇ
ನೀಡುತ್ತದೆ
ನೀಡುತ್ತವೆ
ನೀಡುತ್ತಿವೆ
ನೀಡುವ
ನೀಡು-ವಂತ-ಹ-ದಿ-ರ-ಬೇಕು
ನೀಡು-ವಂತೆ
ನೀಡುವನು
ನೀಡು-ವು-ದಕ್ಕೆ
ನೀಡು-ವು-ದ-ರಲ್ಲಿಯೇ
ನೀಡುವುದಾ-ವುದೆಂದರೆ
ನೀಡು-ವು-ದಿಲ್ಲ
ನೀಡು-ವುದು
ನೀಡು-ವುದೋ
ನೀಡು-ವುವು
ನೀತಿ
ನೀತಿ-ಕಥೆ-ಗಳಲ್ಲಿ
ನೀತಿ-ಗಳು
ನೀತಿಗೆ
ನೀತಿ-ಗೋಸುಗ
ನೀತಿ-ದಾಯ-ಕ-ವಾಗಿದೆ
ನೀತಿ-ನಿಯಮ
ನೀತಿ-ನಿಯ-ಮಾ-ವಳಿ-ಗಳು
ನೀತಿಯ
ನೀತಿ-ಯನ್ನು
ನೀತಿ-ಯನ್ನೂ
ನೀತಿ-ಯಲ್ಲಿ
ನೀತಿಯು
ನೀತಿಯೂ
ನೀತಿಯೆ
ನೀತಿ-ಯೆಲ್ಲ
ನೀತಿಯೇ
ನೀತಿ-ವಂತ
ನೀತಿ-ವಂತ-ನಾದ
ನೀತಿ-ವಂತನೂ
ನೀತಿ-ವಂತ-ರನ್ನಾಗಿ
ನೀತಿ-ವಂತ-ರಲ್ಲ
ನೀತಿ-ವಂತ-ರಾಗ
ನೀತಿ-ವಂತ-ರಾಗ-ಬಲ್ಲೆವು
ನೀತಿ-ವಂತ-ರಾಗ-ಬಹುದು
ನೀತಿ-ವಂತ-ರಾಗಿ
ನೀತಿ-ವಂತ-ರಾಗಿ-ರ-ಬೇಕು
ನೀತಿ-ವಂತ-ರಾಗಿ-ರ-ಲಾರಿರಿ
ನೀತಿ-ವಂತ-ರಾಗು-ವೆವು
ನೀತಿ-ವಂತರು
ನೀತಿ-ವಂತ-ರೆಂದು
ನೀತಿ-ಶಾಲಿ-ಗಳು
ನೀತಿ-ಶಾಸ್ತ್ರ-ಗಳ
ನೀನಲ್ಲ
ನೀನಾಗಿದೆ
ನೀನಾ-ರೆಂದು
ನೀನು
ನೀನೂ
ನೀನೆ
ನೀನೆಂದು
ನೀನೇ
ನೀನೇಕೆ
ನೀನೊಬ್ಬ
ನೀನೊಬ್ಬನೇ
ನೀರನ್ನು
ನೀರಿ-ಗಾಗಿ
ನೀರಿಗೆ
ನೀರಿನ
ನೀರಿ-ನಂತೆ
ನೀರಿ-ನಲ್ಲಿ
ನೀರಿನಲ್ಲಿಟ್ಟು
ನೀರಿನಲ್ಲಿ-ರ-ಬೇಕಾ-ಗುತ್ತಿತ್ತು
ನೀರಿ-ನಿಂದ
ನೀರಿರು-ವು-ದೆಂದು
ನೀರು
ನೀರು-ಗುಳ್ಳೆಯು
ನೀರೆಲ್ಲ
ನೀರೆಲ್ಲಿ
ನೀರೊ-ಳಗೆ
ನೀರ್ಗಲ್ಲ
ನೀರ್ಗಲ್ಲಿಗೆ
ನೀಲಾ-ಕಾಶ
ನೀಲಿ
ನೀಲಿ-ಬಣ್ಣ-ವನ್ನೇ
ನೀಲಿಯ
ನೀಲಿ-ಯಾ-ಕಾಶ
ನೀಲಿ-ಯಾಗಿದ್ದರೆ
ನೀವಲ್ಲದೆ
ನೀವಲ್ಲವೆ
ನೀವಾ-ಗಲಿ
ನೀವಾ-ಗಲೆ
ನೀವಾ-ಗಲೇ
ನೀವಾಗಿ
ನೀವಾ-ದರೆ
ನೀವಿ-ರುವ
ನೀವೀಗ
ನೀವು
ನೀವು-ಗಳೆಲ್ಲ
ನೀವೂ
ನೀವೆ
ನೀವೆಂದಿಗೂ
ನೀವೆಲ್ಲ
ನೀವೆಲ್ಲರೂ
ನೀವೆಲ್ಲಾ
ನೀವೇ
ನೀವೇಕೆ
ನೀವೇನು
ನೀವೊಂದು
ನೀವೊಬ್ಬರೆ
ನೀಹಾರಿ-ಕೆ-ಗಳಿಂದ
ನೀಹಾರಿ-ಕೆ-ಗಳು
ನೀಹಾರಿ-ಕೆಗೇ
ನೀಹಾರಿ-ಕೆ-ಯಿಂದ
ನುಂಗು-ವಂತೆ
ನುಗ್ಗಾಟ
ನುಗ್ಗುವ
ನುಗ್ಗು-ವುವು
ನುಚ್ಚು
ನುಚ್ಚು-ನೂರು
ನುಡಿ-ದನು
ನುಡಿ-ಯನ್ನು
ನುಡಿ-ಯನ್ನೂ
ನುಡಿ-ಸುತ್ತಾರೆ
ನುಣಚಿ-ಕೊಳ್ಳುವ
ನುರಿತ
ನುಸುಳಿ
ನೂಕಲು
ನೂಕಿ
ನೂಕು
ನೂಕುತ್ತಿ-ರುವ
ನೂಕುತ್ತಿ-ರುವುದು
ನೂಕು-ವನು
ನೂತನ
ನೂರ-ರಷ್ಟು
ನೂರಾಗಿ
ನೂರಾಗಿಲ್ಲವೊ
ನೂರಾರು
ನೂರು
ನೂರು-ಮಡಿ
ನೂರೈವತ್ತು
ನೂಲನ್ನು
ನೂಲಿನ
ನೂಲು
ನೃತ್ಯ
ನೆಂಟರಿಷ್ಟ-ರನ್ನು
ನೆಂಟರಿಷ್ಟ-ರೊ-ಡನೆ
ನೆಂದು
ನೆಗಡಿ
ನೆಗೆ-ಯಲಿ
ನೆಗೆ-ಯಿತು
ನೆಗೆ-ಯುತ್ತ
ನೆಗೆ-ಯು-ವುದು
ನೆಚ್ಚ-ಬೇಕೆಂದು
ನೆಚ್ಚಿ
ನೆಚ್ಚಿ-ಕೊಂಡಿದ್ದರೆ
ನೆಚ್ಚಿ-ಕೊಂಡಿ-ರ-ಬಲ್ಲಿರಿ
ನೆಚ್ಚಿ-ಕೊಂಡಿ-ರು-ವನು
ನೆಚ್ಚಿ-ಕೊಂಡಿ-ರು-ವಿರಿ
ನೆಚ್ಚಿ-ಕೊಂಡಿ-ರು-ವುದು
ನೆಚ್ಚಿ-ಕೊಂಡಿಲ್ಲವೊ
ನೆಚ್ಚಿಗೆ
ನೆಚ್ಚಿ-ಗೆ-ಯನ್ನು
ನೆಚ್ಚಿರು
ನೆಚ್ಚುತ್ತೇವೆ
ನೆಚ್ಚು-ವುದ-ರಲ್ಲೆ
ನೆತ್ತಿ
ನೆತ್ತಿ-ಗೇರಿ
ನೆದು-ರಿಗೆ
ನೆನಪನ್ನು
ನೆನಪಾ-ಗುತ್ತದೆ
ನೆನಪಾ-ಗು-ವುದು
ನೆನಪಿ
ನೆನಪಿಗೆ
ನೆನಪಿನ
ನೆನಪಿ-ನಂತಹ
ನೆನಪಿ-ನಂತೆ
ನೆನಪಿ-ನಲ್ಲಿ
ನೆನಪಿ-ನಲ್ಲಿಟ್ಟಿ-ರ-ಬೇಕು
ನೆನಪಿ-ನಲ್ಲಿ-ಡ-ಬೇಕು
ನೆನಪಿ-ನಲ್ಲಿ-ಡು-ವುದು
ನೆನಪಿಲ್ಲ
ನೆನಪಿಲ್ಲವೆ
ನೆನಪಿಲ್ಲವೋ
ನೆನಪಿ-ಸಿ-ಕೊಳ್ಳುತ್ತೀರಿ
ನೆನಪು
ನೆನಪೂ
ನೆನಪೆಂಬ
ನೆನಪೆಲ್ಲ
ನೆನಸಿ-ಕೊಳ್ಳು-ವು-ದಕ್ಕೆ
ನೆನೆಯ-ಲೇ-ಬೇಕು
ನೆಮ್ಮದಿ
ನೆಯ
ನೆಯ-ದಾಗಿ
ನೆಯದು
ನೆಯು
ನೆರ-ಳನ್ನು
ನೆರಳಿ-ನಂತೆ
ನೆರಳು
ನೆರಳು-ಬೆಳ-ಕು-ಗಳಂತೆ
ನೆರ-ವಿ-ನಿಂದ
ನೆರೆ
ನೆರೆ-ಯದೆ
ನೆರೆ-ಯವರ
ನೆಲಕ್ಕೆ
ನೆಲದ
ನೆಲ-ದಲ್ಲಿ
ನೆಲ-ದಿಂದ
ನೆಲ-ದೊ-ಳಗೆ
ನೆಲ-ವನ್ನು
ನೆಲವು
ನೆಲ-ಸಮ
ನೆಲಸಿ-ದರು
ನೆಲ-ಸಿವೆ
ನೆಲಸು-ವಂತೆ
ನೆಲಸು-ವರು
ನೆಲ-ಸು-ವು-ದಕ್ಕೆ
ನೆಲೆ
ನೆಲೆ-ಗೊಂಡಿ-ರು-ವುದೇ
ನೆಲೆ-ಗೊಳಿಸಿ
ನೆಲೆ-ಗೊಳಿಸು
ನೆಲೆ-ನಿಂತಾಗ
ನೆಲೆ-ಯನ್ನು
ನೆಲೆ-ಸಿ-ರು-ವುದು
ನೆಲೆ-ಸು-ವಂತೆ
ನೆಲೆ-ಸು-ವು-ದಕ್ಕೆ
ನೆಲೆ-ಸು-ವುದು
ನೆವ-ಗ-ಳನ್ನು
ನೆವ-ದಿಂದಲೂ
ನೇ
ನೇತಿ
ನೇಮಿ-ಸ-ಬೇಕು
ನೇಯ್ದಿ-ರು-ವರು
ನೇರ-ಮಾಡ-ಲಾರದು
ನೇರ-ವಾಗಿ
ನೇರ-ವಾಗಿ-ರ-ಬೇಕು
ನೇರ-ವಾ-ಗು-ವುದು
ನೇರ-ವಾದ
ನೇವಿಯಾ
ನೈಜ
ನೈಜ-ಗುಣ-ವಾದ
ನೈಜ-ತೆಗೆ
ನೈಜ-ತೆ-ಯನ್ನು
ನೈಜ-ವಾಗಿ
ನೈಜಸ್ಥಿತಿ
ನೈಜಸ್ಥಿತಿಗೆ
ನೈಜಸ್ಥಿತಿಯ
ನೈಜಸ್ಥಿತಿ-ಯನ್ನು
ನೈಜಸ್ಥಿತಿಯೇ
ನೈಜಸ್ವ-ಭಾವ
ನೈಜಸ್ವ-ಭಾವ-ವನ್ನು
ನೈಜಸ್ವಭಾ-ವ-ವಾದ
ನೈಜಸ್ವಭಾ-ವವೇ
ನೈಜಸ್ವ-ರೂಪ
ನೈಜಸ್ವ-ರೂಪ-ವನ್ನು
ನೈಜಸ್ವ-ರೂಪ-ವಾದ
ನೈತಿಕ
ನೈತಿಕ-ಭಾ-ವನೆ
ನೈತಿಕ-ವಾಗಿ
ನೈತಿಕ-ಶಕ್ತಿ
ನೈಯಾಯಿಕ
ನೈರ್ಮಲ್ಯ
ನೈಸರ್ಗಿಕ
ನೊಬ್ಬನು
ನೋಟ
ನೋಟಕ್ಕೆ
ನೋಟ-ಗಳು
ನೋಟದ
ನೋಟ-ವನ್ನು
ನೋಟವೇ
ನೋಟ-ವೊಂದು
ನೋಡ
ನೋಡ-ಕೂಡದು
ನೋಡದ
ನೋಡದೆ
ನೋಡ-ಬಯ-ಸುವನು
ನೋಡ-ಬಯ-ಸು-ವೆನು
ನೋಡ-ಬಲ್ಲ
ನೋಡ-ಬಲ್ಲನೊ
ನೋಡ-ಬಲ್ಲಿರಿ
ನೋಡ-ಬಲ್ಲಿರಿ-ಇಂದು
ನೋಡ-ಬಲ್ಲೆವು
ನೋಡ-ಬಹು-ದಾ-ಗಿತ್ತು
ನೋಡ-ಬಹು-ದಾ-ದರೂ
ನೋಡ-ಬಹುದು
ನೋಡ-ಬಹು-ದು-ನಮಗೆ
ನೋಡ-ಬಹು-ದೆಂದು
ನೋಡ-ಬಾ-ರದು-ಇಡೀ
ನೋಡ-ಬೇಕಾ-ಗಿದೆ
ನೋಡ-ಬೇಕಾ-ಗಿ-ದೆ-ಯಲ್ಲ
ನೋಡ-ಬೇಕಾದ
ನೋಡ-ಬೇಕಾ-ದರೆ
ನೋಡ-ಬೇಕು
ನೋಡ-ಬೇಕೆನ್ನು-ವು-ದನ್ನು
ನೋಡ-ಬೇಡಿ
ನೋಡ-ಲಾಗು-ವು-ದಿಲ್ಲ
ನೋಡ-ಲಾರ
ನೋಡ-ಲಾರ-ದು-ಆದ-ಕಾರಣ
ನೋಡ-ಲಾರರು
ನೋಡ-ಲಾರಿರಿ
ನೋಡ-ಲಾರೆ
ನೋಡ-ಲಾರೆವು
ನೋಡ-ಲಾರೆ-ವುಗ್ರಹಿಸ
ನೋಡ-ಲಾರೆವೊ
ನೋಡಲಿ
ನೋಡಲಿಚ್ಛಿಸು-ವೆನು
ನೋಡ-ಲಿಲ್ಲ-ವೆನ್ನು-ವುದೇ
ನೋಡಲು
ನೋಡ-ಲೆತ್ನಿ-ಸು-ವುದು
ನೋಡ-ಲೇ-ಬೇಕಾ-ಗಿತ್ತು
ನೋಡ-ಲೇ-ಬೇಕಾ-ಗಿದೆ
ನೋಡಲೇ-ಬೇಕು
ನೋಡಿ
ನೋಡಿ-ಕೊಳ್ಳ-ಬೇಕಾ-ಗಿತ್ತು
ನೋಡಿ-ಕೊಳ್ಳ-ಬೇಕು
ನೋಡಿ-ಕೊಳ್ಳಿ
ನೋಡಿ-ಕೊಳ್ಳುತ್ತಿದ್ದ
ನೋಡಿ-ಕೊಳ್ಳುತ್ತಿದ್ದವು
ನೋಡಿ-ಕೊಳ್ಳು-ವಂತೆ
ನೋಡಿ-ಕೊಳ್ಳು-ವು-ದನ್ನು
ನೋಡಿ-ಕೊಳ್ಳು-ವು-ದಿಲ್ಲ
ನೋಡಿ-ಕೊಳ್ಳು-ವುದೇ
ನೋಡಿ-ಕೋಳ್ಳ-ಬೇಕು
ನೋಡಿಜ್ಞಾನವೇ
ನೋಡಿತು
ನೋಡಿದ
ನೋಡಿ-ದಂತೆ
ನೋಡಿ-ದಂತೆಯೇ
ನೋಡಿ-ದರು
ನೋಡಿ-ದರೂ
ನೋಡಿ-ದರೆ
ನೋಡಿ-ದಾಗ
ನೋಡಿ-ದಿರಾ
ನೋಡಿ-ದು-ದನ್ನು
ನೋಡಿದೆ
ನೋಡಿ-ದೆನು
ನೋಡಿ-ದೆಯ
ನೋಡಿ-ದೆವು
ನೋಡಿ-ದೊ-ಡ-ನೆಯೆ
ನೋಡಿದ್ದರೆ
ನೋಡಿದ್ದಾ-ಯಿತು
ನೋಡಿದ್ದೇನೆ
ನೋಡಿದ್ದೇವೆ
ನೋಡಿಯೂ
ನೋಡಿ-ರ-ಬಹುದು
ನೋಡಿ-ರು-ವರೊ
ನೋಡಿ-ರು-ವರೋ
ನೋಡಿ-ರು-ವಿರಾ
ನೋಡಿ-ರು-ವಿರಿ
ನೋಡಿ-ರು-ವಿರೊ
ನೋಡಿ-ರು-ವಿರೋ
ನೋಡಿ-ರು-ವೆನು
ನೋಡಿ-ರು-ವೆವು
ನೋಡಿಲ್ಲ
ನೋಡು
ನೋಡುತ್ತ
ನೋಡುತ್ತದೆ
ನೋಡುತ್ತಾ
ನೋಡುತ್ತಾನೆ
ನೋಡುತ್ತಾನೋ
ನೋಡುತ್ತಾರೆ
ನೋಡುತ್ತಾರೊ
ನೋಡುತ್ತಾರೋ
ನೋಡುತ್ತಿದ್ದ
ನೋಡುತ್ತಿದ್ದಾಗ
ನೋಡುತ್ತಿದ್ದಿರಿ
ನೋಡುತ್ತಿದ್ದೆ
ನೋಡುತ್ತಿ-ರ-ಲಿಲ್ಲ
ನೋಡುತ್ತಿರುತ್ತೇವೆ
ನೋಡುತ್ತಿ-ರುವ
ನೋಡುತ್ತಿ-ರು-ವಂತೆ
ನೋಡುತ್ತಿ-ರುವನು
ನೋಡುತ್ತಿ-ರು-ವರೊ
ನೋಡುತ್ತಿ-ರು-ವಿರಿ
ನೋಡುತ್ತಿ-ರುವುದು
ನೋಡುತ್ತಿ-ರು-ವೆನು
ನೋಡುತ್ತಿ-ರು-ವೆನುಈ
ನೋಡುತ್ತಿ-ರು-ವೆವು
ನೋಡುತ್ತೀರಿ
ನೋಡುತ್ತೀರಿ-ನಂಬಿ-ಕೆ-ಯಿಂದ
ನೋಡುತ್ತೀರೋ
ನೋಡುತ್ತೇನೆ
ನೋಡುತ್ತೇನೆಯೋ
ನೋಡುತ್ತೇವೆ
ನೋಡುತ್ತೇವೆ-ಅ-ವರು
ನೋಡುತ್ತೇವೆಯೋ
ನೋಡುವ
ನೋಡು-ವಂತೆ
ನೋಡು-ವಂತೆಯೇ
ನೋಡು-ವನು
ನೋಡು-ವರು
ನೋಡು-ವ-ರು-ಅ-ವನು
ನೋಡು-ವರೊ
ನೋಡು-ವರೋ
ನೋಡು-ವ-ವ-ನನ್ನು
ನೋಡು-ವ-ವನು
ನೋಡು-ವ-ವನೆ
ನೋಡು-ವ-ವರ
ನೋಡು-ವಾಗ
ನೋಡು-ವಿರಾ
ನೋಡು-ವಿರಿ
ನೋಡು-ವಿರೋ
ನೋಡುವು
ನೋಡು-ವು-ದಕ್ಕಿಂತ
ನೋಡು-ವು-ದಕ್ಕಿಂತಲೂ
ನೋಡು-ವು-ದಕ್ಕೆ
ನೋಡು-ವು-ದನ್ನು
ನೋಡು-ವು-ದ-ರಲ್ಲಿ
ನೋಡು-ವು-ದ-ರಿಂದ
ನೋಡು-ವು-ದಾದರೆ
ನೋಡು-ವು-ದಿಲ್ಲ
ನೋಡು-ವು-ದಿಲ್ಲವೆ
ನೋಡು-ವು-ದಿಲ್ಲವೋ
ನೋಡು-ವುದು
ನೋಡು-ವುದೇ
ನೋಡು-ವು-ದೇನೊ
ನೋಡು-ವುವು
ನೋಡುವೆ
ನೋಡು-ವೆನು
ನೋಡು-ವೆನೊ
ನೋಡು-ವೆವು
ನೋಡು-ವೆ-ವು-ಧರ್ಮ-ದಿಂದ
ನೋಡು-ವೆವೊ
ನೋಡು-ವೆವೋ
ನೋಡು-ವೋವೋ
ನೋಡೋಣ
ನೋಯಿ-ಸದೆ
ನೋಯಿಸ-ಬಲ್ಲರು
ನೋವನ್ನು
ನೋವನ್ನುಂಟು-ಮಾಡದೆ
ನೋವಿ-ಗಿಂತ
ನೋವಿ-ಗಿಂತಲೂ
ನೋವಿನ
ನೋವು
ನೋವೂ
ನ್ನರು
ನ್ನಾಗಲಿ
ನ್ನಾಗಿ
ನ್ನಾಗಿಯೂ
ನ್ನೆಲ್ಲಾ
ನ್ನೊಳ-ಗೊಂಡ
ನ್ಯಾಯ
ನ್ಯಾಯ-ದಲ್ಲಿ
ನ್ಯಾಯ-ಪರ
ನ್ಯಾಯ-ಮೂರ್ತಿ-ಯಂತಿ-ರುವ
ನ್ಯಾಯ-ಸಮ್ಮತ-ವಾದು-ದೆಲ್ಲ
ನ್ಯಾಯಾಧೀಶ-ರೇನು
ನ್ಯಾಸ
ನ್ಯೂನತೆ
ನ್ಯೂನತೆ-ಯಿದೆ-ಅದೇ
ನ್ಯೂಯಾರ್ಕಿನ
ನ್ಯೂಯಾರ್ಕಿ-ನಲ್ಲಿ
ನ್ಯೂಯಾರ್ಕ್
ನ್ಯೂಸ್ನಲ್ಲಿ
ಪಂಗಡ
ಪಂಗಡಕ್ಕೂ
ಪಂಗಡಕ್ಕೆ
ಪಂಗಡ-ಗ-ಳನ್ನು
ಪಂಗಡ-ಗಳಲ್ಲಿ
ಪಂಗಡ-ಗಳಾ-ಗಲಿ
ಪಂಗಡ-ಗಳಾ-ವುವೂ
ಪಂಗಡ-ಗಳಿರು
ಪಂಗಡ-ಗಳಿ-ರು-ವು-ದ-ರಿಂದ
ಪಂಗಡ-ಗಳಿವೆ
ಪಂಗಡ-ಗಳು
ಪಂಗಡದ
ಪಂಗಡ-ದಲ್ಲಿ-ರುವ
ಪಂಗಡ-ದ-ವ-ನಾ-ದರೂ
ಪಂಗಡ-ದ-ವರ
ಪಂಗಡ-ದ-ವ-ರಿಗೆ
ಪಂಗಡ-ದ-ವರು
ಪಂಗಡ-ದ-ವರೂ
ಪಂಗಡ-ವನ್ನು
ಪಂಗಡ-ವಿತ್ತು
ಪಂಗಡವು
ಪಂಗಡವೇ
ಪಂಚತಯ್ಯಃ
ಪಂಚ-ಬಂಧನ
ಪಂಚ-ಭೂತ-ಗಳ
ಪಂಚ-ಭೂತ-ಗಳಾಗಿ
ಪಂಚ-ಭೂತ-ಗಳಿಂದ
ಪಂಚ-ಭೂತ-ಗಳು
ಪಂಚೇಂದ್ರಿಯ
ಪಂಚೇಂದ್ರಿಯ-ಗಳ
ಪಂಚೇಂದ್ರಿಯ-ಗ-ಳನ್ನು
ಪಂಚೇಂದ್ರಿಯ-ಗಳಲ್ಲಿದೆ
ಪಂಚೇಂದ್ರಿಯ-ಗಳಿಗೆ
ಪಂಚೇಂದ್ರಿಯ-ಗಳಿವೆ
ಪಂಜನ್ನು
ಪಂಜರ-ದೊ-ಳಗೆ
ಪಂಜರ-ವನ್ನು
ಪಂಜು
ಪಂಡಿತ
ಪಂಡಿತ-ಪಾ-ಮರರ
ಪಂಡಿ-ತರ
ಪಂಡಿತ-ರಲ್ಲೂ
ಪಂಡಿತ-ರಿಗೋ
ಪಂಡಿ-ತರು
ಪಂಥ
ಪಂಥಕ್ಕೆ
ಪಂಥ-ಗಳ
ಪಂಥ-ಗ-ಳನ್ನು
ಪಂಥ-ಗಳಲ್ಲಿ
ಪಂಥ-ಗಳಲ್ಲೆಲ್ಲಾ
ಪಂಥ-ಗಳ-ವರೂ
ಪಂಥ-ಗಳಿಗೆ
ಪಂಥ-ಗಳಿವೆ
ಪಂಥ-ಗಳಿ-ವೆಯೊ
ಪಂಥ-ಗಳು
ಪಂಥ-ಗಳೂ
ಪಂಥ-ಗಳೆಲ್ಲ
ಪಂಥದ
ಪಂಥ-ದ-ವರು
ಪಂಥ-ದಿಂದಲೂ
ಪಂಥ-ವನ್ನು
ಪಂಥ-ವಾಗಿ-ರ-ಲಿಲ್ಲ
ಪಂಥ-ವಾ-ದೊ-ಡನೆಯೇ
ಪಂಥವು
ಪಂಪಿ-ನಂತೆ
ಪಕ್ಕದ
ಪಕ್ಕ-ದಲ್ಲಿ
ಪಕ್ಕದಲ್ಲಿಯೆ
ಪಕ್ಕ-ಪಕ್ಕ-ದಲ್ಲೆ
ಪಕ್ಕೆಲುಬು-ಗಳ
ಪಕ್ರಮಿಸಿದೆ
ಪಕ್ವವೂ
ಪಕ್ಷ
ಪಕ್ಷಕ್ಕೆ
ಪಕ್ಷ-ಗಳಿಗೂ
ಪಕ್ಷ-ಗಳಿವೆ
ಪಕ್ಷದ
ಪಕ್ಷ-ದಲ್ಲಿ
ಪಕ್ಷ-ದ-ವ-ರಿಗೂ
ಪಕ್ಷ-ದ-ವರು
ಪಕ್ಷ-ಪಾತ-ವೆಂದು
ಪಕ್ಷ-ಪಾತಿ
ಪಕ್ಷ-ಪಾತಿ-ಯಾಗಿ
ಪಕ್ಷ-ಭಾವ-ನಮ್
ಪಕ್ಷ-ವನ್ನು
ಪಕ್ಷ-ವನ್ನೂ
ಪಕ್ಷ-ವಾದ
ಪಕ್ಷವು
ಪಕ್ಷಿ
ಪಕ್ಷಿ-ಗಳಲ್ಲಿ
ಪಟ-ಗ-ಳನ್ನು
ಪಟ-ವನ್ನು
ಪಟುತ್ವಕುಗ್ಗು
ಪಟುತ್ವ-ವನ್ನು
ಪಟ್ಟಂತೆ
ಪಟ್ಟಣ
ಪಟ್ಟ-ಣಕ್ಕೂ
ಪಟ್ಟ-ಣಕ್ಕೆ
ಪಟ್ಟ-ಣ-ವನ್ನು
ಪಟ್ಟ-ಣ-ವೆಂಬ
ಪಟ್ಟ-ದುದು
ಪಟ್ಟದ್ದು
ಪಟ್ಟರೂ
ಪಟ್ಟರೆ
ಪಟ್ಟಿ-ರು-ವರು
ಪಟ್ಟಿ-ರು-ವುದು
ಪಟ್ಟಿ-ರುವು-ದೆಲ್ಲವೂ
ಪಟ್ಟಿ-ವೆ-ಮುಂದಿನ
ಪಟ್ಟು-ಕೊಂಡು
ಪಠ್ಯಗ್ರಂಥ
ಪಡ-ಬಹು-ದು-ಆ-ದರೆ
ಪಡ-ಬಹು-ದುಈ
ಪಡ-ಬೇಕಾ-ಗಿಲ್ಲ
ಪಡ-ಬೇಕಾ-ಗು-ವು-ದೆಂದು
ಪಡ-ಬೇಕು
ಪಡ-ಬೇಕೆಂದು
ಪಡಿ-ಸ-ಬ-ಹುದೇ
ಪಡಿ-ಸ-ಬೇ-ಕಷ್ಟೆ
ಪಡಿ-ಸ-ಬೇಕಾ-ದರೆ
ಪಡಿ-ಸ-ಲಾಗು-ವು-ದಿಲ್ಲ
ಪಡಿ-ಸ-ಲಾರದು
ಪಡಿ-ಸಿ-ಕೊಳ್ಳು-ವುದು
ಪಡಿ-ಸುತ್ತಾ
ಪಡಿ-ಸುತ್ತಾರೆ
ಪಡಿ-ಸು-ವು-ದಕ್ಕೆ
ಪಡಿ-ಸುವು-ದಾಗಲಿ
ಪಡಿ-ಸು-ವು-ದಿಲ್ಲ
ಪಡಿ-ಸು-ವುದು-ದಕ್ಕಾಗು-ವು-ದಿಲ್ಲ
ಪಡಿ-ಸೋಣ
ಪಡುತ್ತಾನೆ
ಪಡುತ್ತಾರೆ
ಪಡುತ್ತಿದ್ದರೂ
ಪಡುತ್ತಿ-ರುವ
ಪಡುತ್ತೇವೆ
ಪಡುವ
ಪಡು-ವಂತಹ
ಪಡುವನು
ಪಡುವರು
ಪಡುವರೊ
ಪಡುವ-ವ-ರೆಲ್ಲ
ಪಡು-ವಷ್ಟು
ಪಡುವಾಗ
ಪಡು-ವಿರಿ
ಪಡು-ವುದಕ್ಕೆ
ಪಡುವು-ದಿಲ್ಲವೊ
ಪಡು-ವುದು
ಪಡೆ
ಪಡೆದ
ಪಡೆ-ದಂತೆಲ್ಲ
ಪಡೆ-ದದ್ದು
ಪಡೆದ-ಮೇಲೆ
ಪಡೆ-ದರು
ಪಡೆ-ದರೂ
ಪಡೆ-ದರೆ
ಪಡೆ-ದವನು
ಪಡೆ-ದವರ
ಪಡೆ-ದವರು
ಪಡೆ-ದಾಗ
ಪಡೆದಿ
ಪಡೆ-ದಿದೆ
ಪಡೆ-ದಿದೆ-ಚೇತನ
ಪಡೆ-ದಿದ್ದನು
ಪಡೆ-ದಿದ್ದಾನೊ
ಪಡೆ-ದಿದ್ದಾರೆ
ಪಡೆ-ದಿದ್ದೆವು
ಪಡೆ-ದಿದ್ದೇ-ನೆಯೋ
ಪಡೆ-ದಿರ-ಲಿಲ್ಲ-ವೆನ್ನು-ವುದು
ಪಡೆ-ದಿರಲೇ-ಬೇಕು
ಪಡೆ-ದಿರು-ತ್ತಾರೆ
ಪಡೆ-ದಿರು-ತ್ತಾರೋ
ಪಡೆ-ದಿರುವ
ಪಡೆ-ದಿರು-ವರು
ಪಡೆ-ದಿರು-ವರೊ
ಪಡೆ-ದಿರು-ವರೋ
ಪಡೆ-ದಿರು-ವುದ-ರಿಂದ
ಪಡೆ-ದಿರು-ವುದು
ಪಡೆ-ದಿವೆ
ಪಡೆದು
ಪಡೆದು-ಕೊಂಡದ್ದು
ಪಡೆದು-ಕೊಂಡಿ-ರುವು-ದೆಂದು
ಪಡೆದು-ಕೊಳ್ಳ-ಬಹುದು
ಪಡೆದು-ಕೊಳ್ಳ-ಬೇಕಾ-ದರೆ
ಪಡೆದು-ಕೊಳ್ಳುತ್ತಾನೆ
ಪಡೆದು-ಕೊಳ್ಳು-ವರು
ಪಡೆದು-ಕೊಳ್ಳು-ವು-ದರ
ಪಡೆದು-ಕೊಳ್ಳು-ವುದು
ಪಡೆ-ದುದು
ಪಡೆಯ
ಪಡೆ-ಯದ
ಪಡೆ-ಯದೆ
ಪಡೆಯ-ಬಲ್ಲ
ಪಡೆಯ-ಬಹು
ಪಡೆಯ-ಬಹುದು
ಪಡೆಯ-ಬಹುದೋ
ಪಡೆಯ-ಬಾರದು
ಪಡೆಯ-ಬೇಕಾ-ಗಿದೆ
ಪಡೆಯ-ಬೇಕಾ-ಗಿಲ್ಲ
ಪಡೆಯ-ಬೇಕಾದ
ಪಡೆಯ-ಬೇಕಾ-ದರೆ
ಪಡೆಯ-ಬೇಕು
ಪಡೆಯ-ಬೇಕೆಂದಿ-ರುವ
ಪಡೆಯ-ಬೇಕೆಂದು
ಪಡೆಯ-ಬೇಕೆಂದೇ
ಪಡೆಯ-ಬೇಕೆಂಬ
ಪಡೆಯ-ಬೇಕೆಂಬುದೇ
ಪಡೆಯ-ಲಾಗು-ವುದಿಲ್ಲ
ಪಡೆಯ-ಲಾರರು
ಪಡೆಯ-ಲಾರವು
ಪಡೆಯ-ಲಾರೆವು
ಪಡೆ-ಯಲಿ
ಪಡೆ-ಯಲು
ಪಡೆಯ-ಲೆತ್ನಿಸಿ-ದರೆ
ಪಡೆ-ಯಲೇ-ಬೇಕಾ-ಗಿದೆ
ಪಡೆ-ಯಲೇ-ಬೇಕು
ಪಡೆ-ಯಲೇ-ಬೇಕೋ
ಪಡೆ-ಯಿರಿ
ಪಡೆಯು
ಪಡೆ-ಯುತ್ತ
ಪಡೆ-ಯುತ್ತದೆ
ಪಡೆ-ಯುತ್ತವೆ
ಪಡೆ-ಯುತ್ತಾ
ಪಡೆ-ಯುತ್ತಾನೆ
ಪಡೆ-ಯುತ್ತಾ-ರೆಯೋ
ಪಡೆ-ಯುತ್ತಿದ್ದರು
ಪಡೆ-ಯುತ್ತೇವೆ
ಪಡೆ-ಯುವ
ಪಡೆ-ಯು-ವಂತೆ
ಪಡೆ-ಯು-ವನು
ಪಡೆ-ಯು-ವರು
ಪಡೆ-ಯು-ವರೊ
ಪಡೆ-ಯು-ವ-ವನು
ಪಡೆ-ಯು-ವ-ವರು
ಪಡೆ-ಯು-ವ-ವರೆಗೂ
ಪಡೆ-ಯು-ವ-ವರೆಗೆ
ಪಡೆ-ಯು-ವಾ-ಗಲೇ
ಪಡೆ-ಯು-ವಿರಿ
ಪಡೆ-ಯು-ವು-ದಕ್ಕಾಗಿ
ಪಡೆ-ಯು-ವು-ದಕ್ಕೂ
ಪಡೆ-ಯು-ವು-ದಕ್ಕೆ
ಪಡೆ-ಯು-ವು-ದ-ರಿಂದ
ಪಡೆ-ಯು-ವು-ದಾದರೆ
ಪಡೆ-ಯು-ವು-ದಿಲ್ಲ
ಪಡೆ-ಯು-ವುದು
ಪಡೆ-ಯು-ವುದೂ
ಪಡೆ-ಯು-ವುದೇ
ಪಡೆ-ಯು-ವೆವೋ
ಪತಂಗವು
ಪತಂಜಲಿ
ಪತಂಜ-ಲಿಗೆ
ಪತಂಜಲಿಯ
ಪತಂಜಲಿ-ಯಾ-ದರೂ
ಪತಂಜಲಿಯು
ಪತನದ
ಪತಿ
ಪತಿತ
ಪತಿ-ತ-ನಾ-ದನು
ಪತಿ-ತಾ-ವಸ್ಥೆ
ಪತಿಯ
ಪತಿ-ಯಲ್ಲಿ
ಪತಿವ್ರತೆ-ಯಾದ
ಪತ್ನಿ-ಯೊಂದಿಗೆ
ಪಥ-ಗ-ಳನ್ನು
ಪಥ-ಗಳಿಗೂ
ಪಥ-ವನ್ನೇ
ಪದ
ಪದಕ್ಕೂ
ಪದಕ್ಕೆ
ಪದ-ಗಳ
ಪದ-ಗಳಂತೆ
ಪದ-ಗ-ಳನ್ನು
ಪದ-ಗಳಾ-ಗಿವೆ
ಪದ-ಗಳಿಗೆ
ಪದ-ಗಳಿಲ್ಲ
ಪದ-ಗಳಿವೆ
ಪದ-ಗಳು
ಪದ-ಗಳೂ
ಪದ-ಗಳೆಲ್ಲಕ್ಕೂ
ಪದ-ಗಳೊಂದಿಗೆ
ಪದಚ್ಯುತ-ನನ್ನಾಗಿ
ಪದಚ್ಯುತ-ರನ್ನಾಗಿ
ಪದ-ತಳ-ದಲ್ಲಿ
ಪದದ
ಪದ-ದಲ್ಲಿ
ಪದ-ದಲ್ಲಿ-ಡ-ಬಹುದು
ಪದ-ದಿಂದ
ಪದ-ದೊಂದಿಗೆ
ಪದ-ರಕ್ಕಿಂತ
ಪದ-ರ-ಗಳು
ಪದ-ರದ
ಪದ-ರ-ಪದ-ರ-ಗಳಾಗಿ
ಪದ-ರ-ವಿದೆ
ಪದ-ರವು
ಪದ-ವನ್ನು
ಪದ-ವನ್ನೂ
ಪದ-ವನ್ನೇ
ಪದ-ವನ್ನೇ-ತಕ್ಕೆ
ಪದ-ವಲ್ಲದೆ
ಪದ-ವಾಗಿ
ಪದವಿ
ಪದವಿ-ಗಳ
ಪದವಿ-ಗಳು
ಪದ-ವಿಗೆ
ಪದ-ವಿದೆ
ಪದವಿ-ಯನ್ನು
ಪದ-ವಿ-ರ-ಬೇಕು
ಪದವು
ಪದವೂ
ಪದವೆ
ಪದ-ವೆಂದು
ಪದವೇ
ಪದಶಃ
ಪದ-ಸಂಕೇತ
ಪದಾರ್ಥ-ಗಳ
ಪದಾರ್ಥ-ಗಳಿ-ರುವ
ಪದಾರ್ಥ-ವನ್ನು
ಪದೇ
ಪದೇ-ಪದೇ
ಪದ್ಧತಿ
ಪದ್ಧ-ತಿಯ
ಪದ್ಧತಿ-ಯನ್ನು
ಪದ್ಧ-ತಿಯು
ಪದ್ಮ
ಪದ್ಮ-ದ-ವರೆಗೆ
ಪದ್ಮ-ವಿ-ರು-ವುದು
ಪನ್ಥಾಃ
ಪಬಾವಿಸಿ
ಪಯಣ-ವನ್ನು
ಪರ
ಪರಂಜ್ಯೋತಿಯ
ಪರಂಜ್ಯೋತಿಯೇ
ಪರಂಜ್ಯೋತಿ-ಯೊಂದೆ
ಪರಂಪರೆ-ಗ-ಳನ್ನು
ಪರಂಪರೆ-ಯಂತೆ
ಪರಂಪರೆ-ಯನ್ನು
ಪರಂಪರೆ-ಯಲ್ಲಿ
ಪರ-ಕಾಯ
ಪರ-ಚಿತ್ತಜ್ಞಾನಮ್
ಪರ-ಣಾಮ
ಪರ-ತತ್ತ್ವ
ಪರಬ್ರಹ್ಮ
ಪರಬ್ರಹ್ಮನ
ಪರಬ್ರಹ್ಮ-ನಂತೆ
ಪರಬ್ರಹ್ಮ-ನನ್ನು
ಪರಬ್ರಹ್ಮ-ನಲ್ಲಿ
ಪರಬ್ರಹ್ಮ-ನೊಂದಿಗೆ
ಪರಬ್ರಹ್ಮ-ವಲ್ಲ
ಪರಬ್ರಹ್ಮ-ವಾಗು-ವು-ದಿಲ್ಲ
ಪರಬ್ರಹ್ಮ-ವಿದೆ
ಪರಬ್ರಹ್ಮವು
ಪರಬ್ರಹ್ಮವೇ
ಪರಮ
ಪರಮ-ದೇವನ-ವರೆಗೆ
ಪರಮ-ಪಾವನ-ನಾದ
ಪರಮ-ಪಾವ-ನಾತ್ಮನು
ಪರಮ-ಶ್ರೇಷ್ಠ-ರಾದ
ಪರಮ-ಸತ್ಯ-ಗಳಿವೆ
ಪರಮ-ಸತ್ಯ-ವನ್ನು
ಪರಮ-ಸಹ-ಕಾರಿ-ಯಾಗು-ವುದು
ಪರಮ-ಸಿದ್ಧಾಂತ
ಪರಮ-ಸುಖ-ವನ್ನು
ಪರಮ-ಸ್ಥಿತಿ
ಪರಮಾ
ಪರ-ಮಾಣು-ಪರಮ-ಮಹತ್ತ್ವಾನ್ತೋಸ್ಯ
ಪರ-ಮಾಣುವಿ-ನಿಂದ
ಪರ-ಮಾತ್ಮ
ಪರ-ಮಾತ್ಮನ
ಪರ-ಮಾತ್ಮ-ನನ್ನು
ಪರ-ಮಾತ್ಮ-ನಲ್ಲಿ
ಪರ-ಮಾತ್ಮ-ನಿ-ರುವನು
ಪರ-ಮಾತ್ಮನು
ಪರ-ಮಾತ್ಮನೆ
ಪರ-ಮಾತ್ಮ-ನೊಬ್ಬನು
ಪರ-ಮಾತ್ಮ-ರೆಂಬ
ಪರ-ಮಾ-ದರ್ಶ-ಗಳನ್ನು
ಪರ-ಮಾ-ದರ್ಶ-ದೆಡೆಗೆ
ಪರ-ಮಾ-ಧಿಪತ್ಯ-ದಲ್ಲಿ
ಪರ-ಮಾ-ನಂದದ
ಪರ-ಮಾರ್ಥ-ಲಾಭಕ್ಕೆ
ಪರಮಾ-ವಧಿ
ಪರಮಾ-ವಧಿ-ಯನ್ನು
ಪರಮಾ-ವಧಿ-ಯಾ-ದರೆ
ಪರಮಾ-ವಧಿಯೆ
ಪರಮಾ-ವಶ್ಯತೇಂದ್ರಿಯಾ-ಣಾಮ್
ಪರಮಾ-ವಸ್ಥೆ
ಪರಮಾ-ಸಕ್ತಿ
ಪರ-ಮೇಶ್ವರ
ಪರ-ಮೋತ್ಕೃಷ್ಟ
ಪರರ
ಪರ-ರಿಗೆ
ಪರ-ಲೋ-ಕಕ್ಕೆ
ಪರ-ಲೋಕ-ಚಿಂತನೆ
ಪರ-ಲೋಕವೆ
ಪರ-ವಶ-ನಾದ
ಪರ-ವಶ-ರಾಗಿ
ಪರ-ವಾಗಿ
ಪರ-ವಿರೋ-ಧ-ವಾದ
ಪರ-ಶರೀರಾ-ವೇಶಃ
ಪರಸ್ಪರ
ಪರಾಕಾಷ್ಠೆ
ಪರಾಕಾಷ್ಠೆಗೆ
ಪರಾಕಾಷ್ಠೆ-ಯನ್ನು
ಪರಾಕ್ರಮ-ಶಾಲಿ
ಪರಾಕ್ರಮಿ-ಗಳಾ-ಗಿದ್ದರು
ಪರಾಚಃ
ಪರಾತ್ಪರ
ಪರಾ-ಮಾ-ವಧಿ
ಪರಾರಿಯಾ-ಯಿತು
ಪರಾರ್ಥಂ
ಪರಾರ್ಥತ್ವಾತ್
ಪರಾ-ವೈ-ರಾಗ್ಯ-ದಿಂದ
ಪರಿ
ಪರಿ-ಗಣಿ-ಸದೆ
ಪರಿ-ಗಣಿ-ಸ-ಬಹುದು
ಪರಿ-ಗಣಿ-ಸ-ಬೇಕು
ಪರಿ-ಗಣಿಸಿ
ಪರಿ-ಗಣಿ-ಸಿ-ರು-ವುದು
ಪರಿ-ಗಣಿಸು
ಪರಿ-ಗಣಿ-ಸು-ವರು
ಪರಿ-ಚಯ
ಪರಿ-ಚಯ-ವಾಗಿ
ಪರಿ-ಚಯ-ವಾಗಿದೆ
ಪರಿ-ಚಯ-ವಾ-ಗು-ವುದು
ಪರಿ-ಚಯ-ವಾದ
ಪರಿ-ಚ-ಯವೆ
ಪರಿ-ಚಿತ-ನಾದ-ವನು
ಪರಿ-ಚಿತನು
ಪರಿ-ಚಿತ-ವಾದ
ಪರಿ-ಣತ-ವಾದ
ಪರಿ-ಣಮಿ
ಪರಿ-ಣಮಿ-ಸು-ವುದು
ಪರಿ-ಣಾಮ
ಪರಿ-ಣಾಮಃ
ಪರಿ-ಣಾಮ-ಕಾರಿ
ಪರಿ-ಣಾಮ-ಕಾರಿ-ಯಾಗ-ಲಾರದು
ಪರಿ-ಣಾಮಕ್ಕೂ
ಪರಿ-ಣಾಮಕ್ಕೆ
ಪರಿ-ಣಾಮಕ್ರಮ-ಸಮಾಪ್ತಿರ್ಗುಣಾ-ನಾಮ್
ಪರಿ-ಣಾಮ-ಗಳ
ಪರಿ-ಣಾಮ-ಗ-ಳನ್ನು
ಪರಿ-ಣಾಮ-ಗಳಿಂದ
ಪರಿ-ಣಾಮ-ಗಳು
ಪರಿ-ಣಾಮ-ಗಳು-ಮನಸ್ಸಿ-ನಲ್ಲಿ
ಪರಿ-ಣಾಮ-ತಾಪ-ಸಂಸ್ಕಾರ-ದುಃಖೈರ್ಗುಣ-ವೃತ್ತಿ
ಪರಿ-ಣಾಮತ್ರಯ-ಸಂಯಮಾದ-ತೀತಾ-ನಾಗ-ತಜ್ಞಾನಮ್
ಪರಿ-ಣಾಮದ
ಪರಿ-ಣಾಮ-ದಲ್ಲಿ
ಪರಿ-ಣಾಮ-ದಿಂದ
ಪರಿ-ಣಾಮ-ವನ್ನು
ಪರಿ-ಣಾಮ-ವನ್ನುಂಟು-ಮಾಡು-ವಂತೆ
ಪರಿ-ಣಾಮ-ವನ್ನುಂಟು-ಮಾಡು-ವುದೊ
ಪರಿ-ಣಾಮ-ವಲ್ಲದೆ
ಪರಿ-ಣಾಮ-ವಾಗ
ಪರಿ-ಣಾಮ-ವಾಗಿ
ಪರಿ-ಣಾಮ-ವಾಗಿದೆ
ಪರಿ-ಣಾಮ-ವಾಗಿಯೆ
ಪರಿ-ಣಾಮ-ವಾಗಿಯೇ
ಪರಿ-ಣಾಮ-ವಾಗಿ-ರುವ
ಪರಿ-ಣಾಮ-ವಾಗು-ವುದು
ಪರಿ-ಣಾಮ-ವಾಯಿ
ಪರಿ-ಣಾಮ-ವಿದು
ಪರಿ-ಣಾಮವು
ಪರಿ-ಣಾಮವೂ
ಪರಿ-ಣಾಮವೆ
ಪರಿ-ಣಾಮ-ವೆಂದು
ಪರಿ-ಣಾಮ-ವೆನ್ನು-ವುದು
ಪರಿ-ಣಾಮವೇ
ಪರಿ-ಣಾಮ-ವೇ-ನಾ-ಯಿತು
ಪರಿ-ಣಾಮ-ವೇನು
ಪರಿ-ಣಾಮ-ವೇನೆಂದರೆ
ಪರಿ-ಣಾಮಾ
ಪರಿ-ಣಾಮಾನ್ಯತ್ವೇ
ಪರಿ-ಣಾಮಾ-ಪರಾಂತನಿರ್ಗ್ರಾಹ್ಯಃ
ಪರಿ-ಣಾಮೈಕತ್ವಾದ್ವಸ್ತು-ತತ್ತ್ವಮ್
ಪರಿ-ತಪಿಸಿ-ದನು
ಪರಿ-ತಪಿಸು
ಪರಿ-ದೃಷ್ಟೋ
ಪರಿ-ಧಿಯು
ಪರಿ-ಧಿ-ಯೊ-ಳಗೆ
ಪರಿ-ಪಾಲಿ-ಸದೆ
ಪರಿ-ಪಾಲಿ-ಸ-ಬೇಕು
ಪರಿ-ಪುಷ್ಟಿ
ಪರಿ-ಪೂರ್ಣ
ಪರಿ-ಪೂರ್ಣತೆ
ಪರಿ-ಪೂರ್ಣ-ತೆಯ
ಪರಿ-ಪೂರ್ಣ-ತೆ-ಯನ್ನು
ಪರಿ-ಪೂರ್ಣ-ತೆ-ಯೆಂಬ
ಪರಿ-ಪೂರ್ಣ-ನೆಂದು
ಪರಿ-ಪೂರ್ಣ-ರಲ್ಲ
ಪರಿ-ಪೂರ್ಣ-ರಾಗ
ಪರಿ-ಪೂರ್ಣ-ರಾ-ಗಲು
ಪರಿ-ಪೂರ್ಣ-ರಾಗಿ
ಪರಿ-ಪೂರ್ಣ-ರಾಗು-ವು-ದಲ್ಲ
ಪರಿ-ಪೂರ್ಣ-ರಾ-ದರೆ
ಪರಿ-ಪೂರ್ಣರು
ಪರಿ-ಪೂರ್ಣರೂ
ಪರಿ-ಪೂರ್ಣ-ವಾಗಿದ್ದರೆ
ಪರಿ-ಪೂರ್ಣ-ವಾಗಿ-ರು-ವು-ದ-ರಿಂದ
ಪರಿ-ಪೂರ್ಣ-ವಾದ
ಪರಿ-ಪೂರ್ಣವೂ
ಪರಿ-ಮಳ-ವನ್ನು
ಪರಿ-ಮಾಣ-ದಲ್ಲಿದೆ
ಪರಿ-ಮಿತ
ಪರಿ-ಮಿತ-ನಾಗಿ-ರು-ವು-ದ-ರಿಂದ
ಪರಿ-ಮಿತ-ವಾದ
ಪರಿ-ಮಿತಿ
ಪರಿ-ಮಿ-ತಿಗೆ
ಪರಿ-ಮಿತಿಯ
ಪರಿ-ಮಿತಿ-ಯನ್ನು
ಪರಿ-ಮಿತಿ-ಯಲ್ಲಿ-ರು-ವೆವೊ
ಪರಿ-ಮಿತಿ-ಯಿಂದ
ಪರಿ-ಮಿತಿ-ಯುಳ್ಳ
ಪರಿ-ಯಂತ-ರವೂ
ಪರಿ-ಯಂತವೂ
ಪರಿ-ವರ್ತ-ನಗೊಳ್ಳು-ವುದು
ಪರಿ-ವರ್ತ-ನ-ವಾ-ಗು-ವುದು
ಪರಿ-ವರ್ತ-ನ-ಶೀಲ-ವಾದು-ದನ್ನು
ಪರಿ-ವರ್ತನೆ
ಪರಿ-ವರ್ತ-ನೆ-ಗೊಂಡು
ಪರಿ-ವರ್ತ-ನೆ-ಯ-ವರೆ-ವಿಗೂ
ಪರಿ-ವರ್ತಿ-ತ-ವಾದ
ಪರಿ-ಶೀಲ-ನಾ-ಶಕ್ತಿ-ಯನ್ನು
ಪರಿ-ಶೀ-ಲನೆ
ಪರಿ-ಶೀ-ಲನೆ-ಯಲ್ಲಿ
ಪರಿ-ಶೀಲಿಸ-ಬೇಕೆಂದು
ಪರಿ-ಶೀಲಿಸಿ-ದರೆ
ಪರಿ-ಶೀಲಿಸಿ-ರುವ
ಪರಿ-ಶೀಲಿಸುತ್ತಾನೆ
ಪರಿ-ಶೀಲಿಸು-ವ-ವರು
ಪರಿ-ಶೀಲಿ-ಸು-ವುದು
ಪರಿ-ಶುದ್ಧ
ಪರಿ-ಶುದ್ಧತೆ
ಪರಿ-ಶುದ್ಧ-ತೆ-ಯಲ್ಲಿ
ಪರಿ-ಶುದ್ಧ-ತೆ-ಯಿಂದ
ಪರಿ-ಶುದ್ಧ-ನನ್ನಾಗಿ
ಪರಿ-ಶುದ್ಧ-ನಾಗಿ
ಪರಿ-ಶುದ್ಧ-ನಾಗಿದ್ದ
ಪರಿ-ಶುದ್ಧ-ನಾಗು-ವ-ವರೆಗೆ
ಪರಿ-ಶುದ್ಧ-ನಾದ
ಪರಿ-ಶುದ್ಧ-ನಾದೊ-ಡ-ನೆಯೆ
ಪರಿ-ಶುದ್ಧನು
ಪರಿ-ಶುದ್ಧನೂ
ಪರಿ-ಶುದ್ಧ-ರಾಗಿದ್ದರೆ
ಪರಿ-ಶುದ್ಧ-ರಾಗು
ಪರಿ-ಶುದ್ಧ-ರಾಗು-ವರು
ಪರಿ-ಶುದ್ಧರು
ಪರಿ-ಶುದ್ಧರೂ
ಪರಿ-ಶುದ್ಧ-ರೆಂದು
ಪರಿ-ಶುದ್ಧರೋ
ಪರಿ-ಶುದ್ಧ-ಳಾದ
ಪರಿ-ಶುದ್ಧ-ವಾಗಿ
ಪರಿ-ಶುದ್ಧ-ವಾಗಿದೆ
ಪರಿ-ಶುದ್ಧ-ವಾಗಿ-ರುವು-ದ-ರಲ್ಲೆಲ್ಲ
ಪರಿ-ಶುದ್ಧ-ವಾಗಿ-ರು-ವು-ದಿಲ್ಲವೊ
ಪರಿ-ಶುದ್ಧ-ವಾಗಿ-ರು-ವುದೋ
ಪರಿ-ಶುದ್ಧ-ವಾಗುತ್ತದೆ
ಪರಿ-ಶುದ್ಧ-ವಾ-ಗು-ವುದು
ಪರಿ-ಶುದ್ಧ-ವಾದ
ಪರಿ-ಶುದ್ಧ-ವಾ-ದದ್ದು
ಪರಿ-ಶುದ್ಧವೂ
ಪರಿ-ಶುದ್ಧಾತ್ಮ-ನನ್ನು
ಪರಿ-ಶುದ್ಧಾತ್ಮ-ರಾಗು-ವ-ವರೆಗೆ
ಪರಿ-ಶುದ್ಧಾತ್ಮ-ರಿಗೆ
ಪರಿ-ಶುದ್ಧಾತ್ಮರು
ಪರಿ-ಶುದ್ಧಾತ್ಮ-ವಿದೆ
ಪರಿ-ಶೋಧನೆ-ಗಳೆಲ್ಲ
ಪರಿಶ್ರಮ-ವಿದೆ
ಪರಿ-ಸರ-ಗಳು
ಪರಿಸ್ಥಿತಿ
ಪರಿಸ್ಥಿತಿ-ಗ-ಳನ್ನೂ
ಪರಿಸ್ಥಿತಿ-ಯನ್ನು
ಪರಿಸ್ಥಿತಿ-ಯನ್ನೆಲ್ಲ
ಪರಿಸ್ಥಿತಿ-ಯಿಂದ
ಪರಿಸ್ಥಿತಿಯು
ಪರಿಸ್ಥಿತಿ-ಯೊಂದಿಗೆ
ಪರಿ-ಹರಿ-ಸ-ಬಹುದು
ಪರಿ-ಹರಿ-ಸ-ಲಾರ
ಪರಿ-ಹರಿ-ಸ-ಲಾರದು
ಪರಿ-ಹರಿ-ಸ-ಲಾರೆವು
ಪರಿ-ಹರಿ-ಸಲು
ಪರಿ-ಹ-ರಿಸಿ-ಕೊಳ್ಳು-ವು-ದಕ್ಕಾಗಿ
ಪರಿ-ಹ-ರಿಸುವ
ಪರಿ-ಹರಿ-ಸು-ವು-ದಕ್ಕೆ
ಪರಿ-ಹಾರ
ಪರಿ-ಹಾರಕ್ಕಾಗಿ
ಪರಿ-ಹಾ-ರಕ್ಕೆ
ಪರಿ-ಹಾರದ
ಪರಿ-ಹಾರ-ಮಾರ್ಗ
ಪರಿ-ಹಾರ-ವನ್ನು
ಪರಿ-ಹಾರ-ವಲ್ಲ
ಪರಿ-ಹಾರ-ವಾಗ-ಲಾರದು
ಪರಿ-ಹಾರ-ವಿದೆ
ಪರಿ-ಹಾರ-ವಿಲ್ಲ
ಪರಿ-ಹಾರ-ವೆಲ್ಲಿ
ಪರಿ-ಹಾರವೇ
ಪರಿ-ಹಾರ-ವೇನೆಂದರೆ
ಪರಿ-ಹಾರೋಪಾಯ
ಪರಿ-ಹಾರೋಪಾಯ-ಗ-ಳನ್ನು
ಪರಿ-ಹಾಸ್ಯ
ಪರೀಕ್ಷಕ
ಪರೀಕ್ಷ-ಕನು
ಪರೀಕ್ಷ-ಕರ
ಪರೀಕ್ಷಾರ್ಥ-ವಾಗಿ
ಪರೀಕ್ಷಿಸ-ಬಲ್ಲ
ಪರೀಕ್ಷಿಸ-ಬೇಕಾ-ಗು-ವುದು
ಪರೀಕ್ಷಿ-ಸ-ಬೇಕು
ಪರೀಕ್ಷಿಸಿ
ಪರೀಕ್ಷಿಸಿ-ದರೆ
ಪರೀಕ್ಷಿಸಿ-ದಾಗ
ಪರೀಕ್ಷಿ-ಸುತ್ತಿದ್ದರು
ಪರೀಕ್ಷಿಸುತ್ತಿ-ರು-ವೆನು
ಪರೀಕ್ಷಿಸುತ್ತೇನೆ
ಪರೀಕ್ಷಿ-ಸುವ
ಪರೀಕ್ಷಿ-ಸುವು-ದಕ್ಕೆ
ಪರೀಕ್ಷಿ-ಸು-ವುದು
ಪರೀಕ್ಷಿ-ಸು-ವುದೇ
ಪರೀಕ್ಷಿ-ಸು-ವೆವು
ಪರೀಕ್ಷಿ-ಸೋಣ
ಪರೀಕ್ಷೆ
ಪರೀಕ್ಷೆ-ಗ-ಳನ್ನು
ಪರೀಕ್ಷೆ-ಮಾಡಿ
ಪರೀಕ್ಷೆ-ಯಲ್ಲಿ
ಪರೀಕ್ಷೆ-ಯಿಂದ
ಪರೀಕ್ಷೆ-ಯಿಂದಲೂ
ಪರು-ಶುದ್ಧ
ಪರು-ಶುದ್ಧರು
ಪರೆಗೆ
ಪರೈರಸಂಸರ್ಗಃ
ಪರೋಪ-ಕಾರ
ಪರೋಪ-ಕಾರದ
ಪರೋಪ-ಜೀವಿ
ಪರೋಪ-ಜೀವಿಯು
ಪರ್ಯಂತ
ಪರ್ಯಂತರ
ಪರ್ಯವ-ಸಾನ-ವಾಗು-ವುದು
ಪರ್ಯವ-ಸಾನ-ವಾಯಿತು
ಪರ್ಯಾಯ
ಪರ್ಯಾ-ಲೋಚಿಸ-ಬೇಕು
ಪರ್ಯಾ-ಲೋಚಿಸಿ-ದರೆ
ಪರ್ವತ-ಗಳಿ-ಗಿಂತಲೂ
ಪರ್ವತ-ಗಳು
ಪರ್ವ-ತದ
ಪರ್ವತ-ದಷ್ಟು
ಪರ್ವತ-ರಾಶಿಯೆ
ಪರ್ವ-ತಾದಿ
ಪರ್ಷಿಯ
ಪರ್ಷಿಯನ್
ಪರ್ಷಿಯನ್ನರ
ಪರ್ಷಿಯನ್ನರು
ಪರ್ಷಿಯಾ
ಪರ್ಷಿಯಾ-ದಲ್ಲಿ
ಪರ್ಷಿಯಾ-ದೇಶ-ವನ್ನು
ಪರ್ಸಿಯ
ಪಲಾಯನ-ವಾಗು-ವುದು
ಪಲಾಯನ-ವಾಗು-ವುವು
ಪಲ್ಲವಿ
ಪಲ್ಲವಿ-ಯಂತೆ
ಪಳಗಿದ
ಪಳ-ಗಿಲ್ಲವೊ
ಪಳಗಿ-ವೆಯೋ
ಪಳಗಿ-ಸುವ
ಪಳಗಿ-ಸುವು-ದಕ್ಕಾಗಿ
ಪಳೆಯುಳಿಕೆ-ಗಳಾ-ಗಿವೆ
ಪವಾಡ
ಪವಾಡ-ಗಳು
ಪವಾಡ-ವೆಂದು
ಪವಾಡ-ವೆನ್ನು-ವೆವೋ
ಪವಿತ್ರ
ಪವಿತ್ರ-ಗೊಳಿಸಿ
ಪವಿತ್ರ-ತಮ
ಪವಿತ್ರತೆ
ಪವಿತ್ರ-ತೆಯ
ಪವಿತ್ರ-ತೆಯನ್ನು
ಪವಿತ್ರ-ತೆಯೇ
ಪವಿತ್ರದ
ಪವಿತ್ರ-ನಾಗು-ವು-ದಿಲ್ಲ
ಪವಿತ್ರನು
ಪವಿತ್ರ-ರನ್ನಾಗಿ
ಪವಿತ್ರ-ರಾಗಿ
ಪವಿತ್ರ-ರೆಂದು
ಪವಿತ್ರ-ರೆಂಬುದು
ಪವಿತ್ರ-ವನ್ನಾಗಿ
ಪವಿತ್ರ-ವಾಗಿ
ಪವಿತ್ರ-ವಾಗಿಟ್ಟಿರ
ಪವಿತ್ರ-ವಾಗು
ಪವಿತ್ರ-ವಾಗುತ್ತದೆ
ಪವಿತ್ರ-ವಾಗುವ
ಪವಿತ್ರ-ವಾಗು-ವುದು
ಪವಿತ್ರ-ವಾದ
ಪವಿತ್ರ-ವಾ-ದವು
ಪವಿತ್ರ-ವಾದುದು
ಪವಿತ್ರ-ವೆಂದು
ಪವಿತ್ರ-ವೆಂದೂ
ಪವಿತ್ರಾ
ಪವಿತ್ರಾತ್ಮ-ರನ್ನೊ
ಪಶು-ಪಕ್ಷಿ-ಗಳೆ
ಪಶ್ಚಿಮ
ಪಾಂಚಾಲ
ಪಾಂಡಿತ್ಯ
ಪಾಂಡಿತ್ಯ-ಗಳಿಂದ
ಪಾಂಡಿತ್ಯ-ಪೂರ್ಣ-ವಾದ
ಪಾಂಡಿತ್ಯ-ಪೋ-ಷಿತ
ಪಾಂಡಿತ್ಯ-ವಿದ್ದರೂ
ಪಾಠ
ಪಾಠಕ್ಕೆ
ಪಾಠ-ವನ್ನು
ಪಾಠ-ವಿದು
ಪಾಡಿಗೆ
ಪಾಡು
ಪಾಡೇನಾ-ಗು-ವುದು
ಪಾತಂಜಲ
ಪಾತ-ಕದ
ಪಾತ-ಕವೇ
ಪಾತ-ಕಿ-ಯಲ್ಲಿ-ರುವ
ಪಾತ-ಕಿಯೋ
ಪಾತಾಳ
ಪಾತಿವ್ರತ್ಯದ
ಪಾತಿವ್ರತ್ಯ-ದಲ್ಲಿ
ಪಾತ್ರ
ಪಾತ್ರ-ದಿಂದ
ಪಾತ್ರ-ಧಾರಿ-ಗಳು
ಪಾತ್ರ-ಧಾ-ರಿಗೆ
ಪಾತ್ರ-ಧಾರಿ-ಯಂತೆ
ಪಾತ್ರ-ರಾಗು-ವರು
ಪಾತ್ರ-ವನ್ನು
ಪಾತ್ರೆ
ಪಾತ್ರೆಯ
ಪಾತ್ರೆ-ಯಂತೆ
ಪಾತ್ರೆ-ಯನ್ನು
ಪಾತ್ರೆ-ಯಲ್ಲಿಯೂ
ಪಾತ್ರೆ-ಯಲ್ಲಿ-ರುವ
ಪಾದತಳ-ದಲ್ಲಿ
ಪಾದರಸ
ಪಾದವು
ಪಾದಿ-ಸುತ್ತವೆ
ಪಾದಿ-ಸುತ್ತಾರೆ
ಪಾದ್ರಿ
ಪಾದ್ರಿ-ಗಳ
ಪಾದ್ರಿ-ಗಳು
ಪಾದ್ರಿಗೆ
ಪಾದ್ರಿಯು
ಪಾದ್ರಿ-ಯೊಂದಿಗೆ
ಪಾನೀಯ
ಪಾಪ
ಪಾಪ-ಇ-ವೆಲ್ಲ
ಪಾಪ-ಕರ್ಮ-ಗಳ
ಪಾಪ-ಕರ್ಮ-ಗ-ಳನ್ನು
ಪಾಪ-ಕರ್ಮದ
ಪಾಪ-ಕರ್ಮವು
ಪಾಪ-ಕರ್ಮವೇ
ಪಾಪ-ಕಾರ್ಯ-ಗ-ಳನ್ನು
ಪಾಪ-ಕಾರ್ಯ-ಗಳು
ಪಾಪ-ಕೃತ್ಯ-ಗ-ಳನ್ನು
ಪಾಪ-ಕೃತ್ಯ-ದಿಂದ
ಪಾಪಕ್ಕೂ
ಪಾಪಕ್ಕೆ
ಪಾಪ-ಗ-ಳನ್ನು
ಪಾಪ-ಗಳು
ಪಾಪ-ಗಳೆಲ್ಲಕ್ಕೂ
ಪಾಪದ
ಪಾಪ-ದಂತೆ
ಪಾಪ-ದಲ್ಲಿ
ಪಾಪ-ದಿಂದ
ಪಾಪ-ದಿಂದಲೂ
ಪಾಪ-ದಿಂದಾ-ದುದು
ಪಾಪ-ಪುಣ್ಯ
ಪಾಪ-ಪುಣ್ಯ-ಕರ್ಮ-ಗಳಿಗೆ
ಪಾಪ-ಪುಣ್ಯ-ಗಳ
ಪಾಪ-ಪುಣ್ಯ-ಗಳಾಚೆ
ಪಾಪ-ಪುಣ್ಯ-ಗಳಿಂದ
ಪಾಪ-ಪುಣ್ಯ-ಗಳಿಗೆ
ಪಾಪ-ಪುಣ್ಯ-ಗಳೆ-ರಡೂ
ಪಾಪ-ಪುರುಷ
ಪಾಪ-ಬೀಜ
ಪಾಪ-ರೂಪ-ದಲ್ಲಿ
ಪಾಪ-ವನ್ನು
ಪಾಪ-ವನ್ನೂ
ಪಾಪ-ವನ್ನೆಲ್ಲಾ
ಪಾಪ-ವಿದೆ
ಪಾಪ-ವಿರ-ಬಲ್ಲದು
ಪಾಪ-ವಿಲ್ಲದೆ
ಪಾಪವು
ಪಾಪವೂ
ಪಾಪ-ವೆಂಬ
ಪಾಪ-ವೆಂಬುದು
ಪಾಪ-ವೆನ್ನು-ವಿರೋ
ಪಾಪ-ವೆನ್ನುವೆವೊ
ಪಾಪ-ವೆಲ್ಲ
ಪಾಪ-ವೇ-ನಾ-ದರೂ
ಪಾಪಾತ್ಮನ
ಪಾಪಿ
ಪಾಪಿ-ಗಳ
ಪಾಪಿ-ಗಳಂತೆ
ಪಾಪಿ-ಗಳಿಗೆ
ಪಾಪಿ-ಗಳು
ಪಾಪಿ-ಗ-ಳೆಂದು
ಪಾಪಿ-ಪುಣ್ಯ-ವಂತರ
ಪಾಪಿ-ಪುಣ್ಯ-ವಂತ-ರಲ್ಲಿ
ಪಾಪಿ-ಯಲ್ಲಿ
ಪಾಪಿ-ಯೆಂದು
ಪಾಪಿ-ಯೆಂದೂ
ಪಾಮರರ
ಪಾರಮಾರ್ಥಿಕ
ಪಾರ-ವಿಲ್ಲ
ಪಾರಾಗ
ಪಾರಾ-ಗದೆ
ಪಾರಾ-ಗನು
ಪಾರಾಗ-ಬಹುದು
ಪಾರಾಗ-ಬೇಕಾ
ಪಾರಾಗ-ಬೇಕಾ-ಗಿದೆ
ಪಾರಾಗ-ಬೇಕಾದ
ಪಾರಾಗ-ಬೇಕಾ-ದರೆ
ಪಾರಾಗ-ಬೇಕು
ಪಾರಾಗ-ಬೇಕೆಂದು
ಪಾರಾ-ಗಲು
ಪಾರಾಗಿ
ಪಾರಾಗಿ-ರು-ವೆನು
ಪಾರಾ-ಗಿ-ರು-ವೆವು
ಪಾರಾಗು-ವನು
ಪಾರಾಗು-ವರು
ಪಾರಾಗು-ವಿರಿ
ಪಾರಾಗು-ವು-ದಕ್ಕೆ
ಪಾರಾ-ಗು-ವುದು
ಪಾರಾಗು-ವುದೆಂದ-ರೇನು
ಪಾರಾಗು-ವುದೇ
ಪಾರಾದ
ಪಾರಾ-ದರೆ
ಪಾರಾದ-ವರು
ಪಾರಿಭಾಷಿಕ
ಪಾರಿವಾಳದ
ಪಾರು
ಪಾರು-ಮಾ-ಡಲು
ಪಾರು-ಮಾಡು-ವು-ದಿಲ್ಲ
ಪಾರ್ಶ್ವ-ಗಳಲ್ಲಿಯೂ
ಪಾರ್ಸೀ
ಪಾಲ-ನೆಗೆ
ಪಾಲನೆಯ
ಪಾಲನ್ನು
ಪಾಲಾ-ಗದ
ಪಾಲಾಗು-ವರು
ಪಾಲಾಗು-ವು-ದಕ್ಕೆ
ಪಾಲಾ-ದರೆ
ಪಾಲಿಗೆ
ಪಾಲಿ-ಸ-ಬೇಕು
ಪಾಲಿ-ಸುತ್ತಾರೊ
ಪಾಲಿ-ಸುವ
ಪಾಲಿ-ಸು-ವುವು
ಪಾಲು
ಪಾವನ
ಪಾವ-ನತೆ
ಪಾವಿತ್ರ್ಯ
ಪಾವಿತ್ರ್ಯ-ಗಳೆಲ್ಲ
ಪಾವಿತ್ರ್ಯದ
ಪಾವಿತ್ರ್ಯ-ವನ್ನು
ಪಾವಿತ್ರ್ಯವು
ಪಾವಿತ್ರ್ಯವೇ
ಪಾಶಕ್ಕೆ
ಪಾಶ-ದಿಂದ
ಪಾಶಮ್
ಪಾಶ್ಚಾತ್ಯ
ಪಾಶ್ಚಾತ್ಯ-ದೇಶ-ಗಳಲ್ಲಿ
ಪಾಶ್ಚಾತ್ಯ-ನಷ್ಟೇ
ಪಾಶ್ಚಾತ್ಯ-ರಲ್ಲಿ
ಪಾಶ್ಚಾತ್ಯರು
ಪಾಸನೆ
ಪಿಂಗಳ
ಪಿಂಗಳ-ಗಳೆಂಬ
ಪಿಂಗಳ-ಗಳೇ
ಪಿಂಗಳ-ಮಾರ್ಗದ
ಪಿಂಡಾಂಡದ
ಪಿತೃ
ಪಿತೃ-ಗಳ
ಪಿತೃ-ಗ-ಳನ್ನು
ಪಿತೃ-ಗಳಲ್ಲಿ
ಪಿತೃ-ಗಳಿ-ರುವೆ-ಡೆಗೆ
ಪಿತೃ-ಗಳು
ಪಿತೃ-ಗಳೊಂದಿಗೆ
ಪಿತೃ-ಗಳೊ-ಡನೆ
ಪಿತೃ-ಪೂಜೆ
ಪಿತೃ-ಪೂಜೆಯ
ಪಿತೃ-ಪೂಜೆ-ಯಲ್ಲಿದೆ
ಪಿತೃ-ಪೂಜೆಯೇ
ಪಿತೃ-ಯಾನ
ಪಿತೃ-ಯಾನ-ಗಳು
ಪಿತೃ-ಲೋಕ
ಪಿತೃ-ಲೋ-ಕಕ್ಕೆ
ಪಿಯಾನೋ
ಪಿರಮಿಡ್ಡು-ಗ-ಳನ್ನು
ಪಿಶಾಚಿ
ಪಿಶಾಚಿಯು
ಪಿಸಿ-ರು-ವರು
ಪೀಠಕ್ಕೆ
ಪೀಠಿಕೆ
ಪೀಡಿಸ-ಬೇಡ
ಪೀಳಿ-ಗೆಯ-ವ-ರಿಗೆ
ಪುಟ
ಪುಟ-ಗಳಿವೆ
ಪುಟ-ಗಳು
ಪುಟ-ದಲ್ಲಿಯೂ
ಪುಟ-ವನ್ನು
ಪುಟ-ವಾದ
ಪುಡಿಪುಡಿ
ಪುಡಿ-ಯಾಗಿ
ಪುಣ್ಯ
ಪುಣ್ಯ-ಕರ್ಮ-ಗಳ
ಪುಣ್ಯಕ್ಕೂ
ಪುಣ್ಯ-ಗಳ
ಪುಣ್ಯ-ತಮ
ಪುಣ್ಯದ
ಪುಣ್ಯ-ಯುಗ
ಪುಣ್ಯ-ವಂತ-ರನ್ನಾಗಿ
ಪುಣ್ಯ-ವನ್ನು
ಪುಣ್ಯ-ವಸ್ತು-ವಿನ
ಪುಣ್ಯವೆ
ಪುಣ್ಯ-ಶಕ್ತಿಯೇ
ಪುಣ್ಯ-ಶಾಲಿ-ಯಾಗು-ವನು
ಪುಣ್ಯ-ಶಾಲಿ-ಯಾಗೆಂದು
ಪುಣ್ಯಾತ್ಮ
ಪುಣ್ಯಾತ್ಮ-ನ-ವರೆಗೆ
ಪುಣ್ಯಾತ್ಮರು
ಪುಣ್ಯಾ-ಪುಣ್ಯ-ಹೇತುತ್ವಾತ್
ಪುತ್ರರೆನ್ನಿ
ಪುನಃ
ಪುನಃಪುನಃ
ಪುನಃಸೃಷ್ಟಿ-ಯಾ-ದರೆ
ಪುನರ-ನಿಷ್ಟಪ್ರ-ಸಂಗಾತ್
ಪುನರಾ-ವರ್ತನೆ
ಪುನರಾ-ವರ್ತ-ನೆ-ಯಾಗುತ್ತಿ-ರುವುದು
ಪುನರಾ-ವರ್ತಿ-ಸಿದ
ಪುನರಾ-ವೃತ್ತಿ
ಪುನರಾ-ವೃತ್ತಿ-ಯಾಗುತ್ತಿ-ರು-ವುವು
ಪುನರ್ಜನ್ಮ
ಪುನರ್ಜನ್ಮದ
ಪುನರ್ಜನ್ಮ-ವನ್ನು
ಪುನರ್ಜನ್ಮ-ವಾ-ಗಲಿ
ಪುನರ್ಜನ್ಮ-ವಿದೆ
ಪುನರ್ಜನ್ಮಾದಿ-ಗಳು-ಇ-ವೆಲ್ಲಾ
ಪುನರ್ರ-ಚನೆ
ಪುರಾಣ
ಪುರಾಣ-ಕವಿ
ಪುರಾ-ಣಕ್ಕೆ
ಪುರಾಣ-ಗಳ
ಪುರಾಣ-ಗ-ಳನ್ನು
ಪುರಾಣ-ಗಳಲ್ಲಿ
ಪುರಾಣ-ಗಳಲ್ಲೇ-ನಾ-ದರೂ
ಪುರಾಣ-ಗಳು
ಪುರಾಣ-ಗಳು-ಇಲ್ಲಿ
ಪುರಾಣದ
ಪುರಾಣ-ದಲ್ಲಿ
ಪುರಾಣ-ವನ್ನಾ-ಗಲೀ
ಪುರಾಣ-ವನ್ನು
ಪುರಾಣ-ವಲ್ಲ
ಪುರಾಣ-ವಿದೆ
ಪುರಾಣ-ವಿದೆಯೆ
ಪುರಾ-ತನ
ಪುರಾ-ತನ-ಕಾಲದ
ಪುರಾ-ತನ-ವಾದ
ಪುರಾ-ತನ-ವಾದುದು
ಪುರಾ-ತನ-ವಾದು-ದೆಂದು
ಪುರುಷ
ಪುರುಷಖ್ಯಾತೇರ್ಗುಣ-ವೈತೃಷ್ಣ್ಯಮ್
ಪುರುಷಜ್ಞಾನ
ಪುರುಷಜ್ಞಾನ-ಲಾಭ-ವಾ-ಗು-ವುದು
ಪುರುಷಜ್ಞಾ-ಮನ್
ಪುರುಷನ
ಪುರುಷ-ನಂತೆ
ಪುರುಷ-ನನ್ನು
ಪುರುಷ-ನಲ್ಲ
ಪುರುಷ-ನಲ್ಲ-ವೆಂದು
ಪುರುಷ-ನಲ್ಲಿ
ಪುರುಷ-ನ-ವರೆಗೆ
ಪುರುಷ-ನಷ್ಟು
ಪುರುಷ-ನಿ-ಗಾಗಿ
ಪುರುಷ-ನಿಗೆ
ಪುರುಷ-ನಿ-ರುವನು
ಪುರುಷ-ನಿ-ರು-ವುದು
ಪುರುಷನು
ಪುರುಷನೂ
ಪುರುಷನೆ
ಪುರುಷನೇ
ಪುರುಷನೊ
ಪುರುಷ-ನೊಬ್ಬನು
ಪುರುಷ-ನೊಬ್ಬನೇ
ಪುರುಷರ
ಪುರುಷ-ರಂತೆ
ಪುರುಷ-ರನ್ನು
ಪುರುಷ-ರಲ್ಲಿ
ಪುರುಷ-ರಾಗ-ಬೇಕು
ಪುರುಷ-ರಾ-ಗಲೇ-ಬೇಕು
ಪುರುಷರು
ಪುರುಷರೆ
ಪುರುಷ-ವಿಶೇಷ
ಪುರುಷಸ್ಯಾ-ಪರಿ-ಣಾಮಿತ್ವಾತ್
ಪುರುಷಾರ್ಥ-ಶೂನ್ಯಾನಾಂ
ಪುರೋಗಾಮಿ-ಯಾದ
ಪುರೋ-ಹಿತ
ಪುರೋ-ಹಿತರ
ಪುರೋ-ಹಿತರು
ಪುರೋ-ಹಿತರೂ
ಪುರೋ-ಹಿತ-ವರ್ಗ-ವಿ-ರ-ಲಿಲ್ಲ
ಪುಷ್ಟಿ-ಗೊಳಿ-ಸ-ಬೇಕು
ಪುಷ್ಟಿ-ಗೊಳಿ-ಸುವು
ಪುಷ್ಪ-ವೆನ್ನಿ
ಪುಸ್ತಕ
ಪುಸ್ತಕ-ಗಳ
ಪುಸ್ತಕ-ಗ-ಳನ್ನು
ಪುಸ್ತಕ-ಗಳಲ್ಲಿ
ಪುಸ್ತ-ಕದ
ಪುಸ್ತಕ-ದಂತಿ-ರುತ್ತದೆ
ಪುಸ್ತಕ-ದಂತೆ
ಪುಸ್ತಕ-ದಲ್ಲಿ
ಪುಸ್ತಕ-ದೊ-ಡನೆ
ಪುಸ್ತಕ-ವನ್ನಾಗಿ
ಪುಸ್ತಕ-ವನ್ನು
ಪುಸ್ತಕ-ವನ್ನೂ
ಪುಸ್ತಕ-ವಾ-ಗು-ವುದು
ಪುಸ್ತಕ-ವೇನು
ಪುಸ್ತ-ಕಾಲ-ಯ-ದಲ್ಲಿ
ಪೂಜಯು
ಪೂಜಾ-ಯೋಗ್ಯ-ವಾದ
ಪೂಜಾರಿ
ಪೂಜಾರಿ-ಗಳ
ಪೂಜಾರಿ-ಗ-ಳನ್ನು
ಪೂಜಾರಿ-ಗಳು
ಪೂಜಾ-ವಸ್ತು-ವಾಗಿ
ಪೂಜಾ-ವಿಧಾನ-ಗಳು
ಪೂಜಿಸ-ಕೂಡದು
ಪೂಜಿಸ-ಬಲ್ಲೆ
ಪೂಜಿಸ-ಬೇಕಾದ
ಪೂಜಿ-ಸ-ಬೇಕು
ಪೂಜಿ-ಸಲು
ಪೂಜಿಸಿ
ಪೂಜಿ-ಸಿದ
ಪೂಜಿಸಿ-ದರೆ
ಪೂಜಿಸು
ಪೂಜಿಸುತ್ತವೆಯೋ
ಪೂಜಿಸುತ್ತಾರೆಯೋ
ಪೂಜಿ-ಸುತ್ತಿದ್ದ
ಪೂಜಿ-ಸುತ್ತಿದ್ದರೂ
ಪೂಜಿ-ಸುತ್ತಿದ್ದರೋ
ಪೂಜಿ-ಸುತ್ತಿದ್ದೆವು
ಪೂಜಿಸುತ್ತಿ-ರು-ವರು
ಪೂಜಿಸುತ್ತಿ-ರು-ವೆಯೋ
ಪೂಜಿಸುತ್ತೇನೆ
ಪೂಜಿ-ಸುತ್ತೇವೆಯೋ
ಪೂಜಿ-ಸುವ
ಪೂಜಿಸು-ವರು
ಪೂಜಿಸು-ವ-ವರು
ಪೂಜಿಸು-ವಾಗ
ಪೂಜಿಸು-ವು-ದಕ್ಕಾಗಿ
ಪೂಜಿಸು-ವು-ದಕ್ಕಿಂತ
ಪೂಜಿಸು-ವುದಕ್ಕೋಸುಗ
ಪೂಜಿಸು-ವುದು
ಪೂಜೆ
ಪೂಜೆ-ಗಳ
ಪೂಜೆ-ಗಳೂ
ಪೂಜೆ-ಗಿಂತಲೂ
ಪೂಜೆಗೆ
ಪೂಜೆ-ಮಾಡುವ
ಪೂಜೆ-ಯಂತೆ
ಪೂಜೆ-ಯಾ-ಗಲಿ
ಪೂಜೆ-ಯಿಂದ
ಪೂಜ್ಯ-ತಮ
ಪೂಜ್ಯತ-ಮರು
ಪೂಜ್ಯರೆ
ಪೂರಕ
ಪೂರಿತ-ವಾಗ-ಲಾರ-ದು-ಕೇವಲ
ಪೂರೈ-ಸ-ಬೇಕು
ಪೂರೈಸಿ
ಪೂರೈಸಿ-ಕೊಂಡು
ಪೂರೈಸಿ-ಕೊಳ್ಳು-ವ-ವರೆಗೆ
ಪೂರೈ-ಸಿದ
ಪೂರೈಸಿ-ದಂತೆ
ಪೂರೈಸಿ-ರುವನು
ಪೂರೈಸಿ-ರು-ವೆವು
ಪೂರೈಸುತ್ತಿ-ರು-ವುವು
ಪೂರೈ-ಸು-ವನು
ಪೂರೈ-ಸು-ವುದು
ಪೂರ್ಣ
ಪೂರ್ಣ-ಗೊಳಿ-ಸುವ
ಪೂರ್ಣ-ಗೊಳಿ-ಸುವು-ದಕ್ಕೆ
ಪೂರ್ಣ-ಗೊಳಿ-ಸುವು-ದಿಲ್ಲ
ಪೂರ್ಣ-ಗೊಳಿ-ಸು-ವುದು
ಪೂರ್ಣಜ್ಞಾನ-ಸಿದ್ಧಿ
ಪೂರ್ಣ-ತೃಪ್ತಿ-ಯನ್ನು
ಪೂರ್ಣತೆ
ಪೂರ್ಣ-ತೆಗೆ
ಪೂರ್ಣ-ತೆಯ
ಪೂರ್ಣ-ತೆ-ಯನ್ನು
ಪೂರ್ಣ-ತೆ-ಯನ್ನೂ
ಪೂರ್ಣ-ತೆ-ಯೆ-ಡೆಗೆ
ಪೂರ್ಣ-ತೆಯೇ
ಪೂರ್ಣದ
ಪೂರ್ಣ-ದಿಂದ
ಪೂರ್ಣ-ದೃಷ್ಟಿ-ಯಿಂದ
ಪೂರ್ಣ-ದೊಂದಿಗೆ
ಪೂರ್ಣ-ನನ್ನಾಗಿ
ಪೂರ್ಣ-ನಾಗು-ವನು
ಪೂರ್ಣ-ನಾಗುವು
ಪೂರ್ಣನು
ಪೂರ್ಣ-ಪಾವಿತ್ರ್ಯ
ಪೂರ್ಣಪ್ರ-ಕಾಶ
ಪೂರ್ಣ-ಮಾಡ-ಲಾರವು
ಪೂರ್ಣ-ಮಾಡಿ-ಕೊಳ್ಳ
ಪೂರ್ಣ-ಮಾಡು-ವು-ದಿಲ್ಲ
ಪೂರ್ಣ-ರಾಗ-ಬೇಕಾ-ದರೆ
ಪೂರ್ಣ-ರಾಗಿದ್ದೀರಿ
ಪೂರ್ಣ-ರಾಗು-ವರು
ಪೂರ್ಣರು
ಪೂರ್ಣ-ವಲ್ಲ
ಪೂರ್ಣ-ವಾಗ-ಬೇಕಾ-ಗಿದೆ
ಪೂರ್ಣ-ವಾಗಿ
ಪೂರ್ಣ-ವಾಗಿದ್ದರೆ
ಪೂರ್ಣ-ವಾಗಿ-ರುವ
ಪೂರ್ಣ-ವಾಗಿ-ರುವುದು
ಪೂರ್ಣ-ವಾಗಿಲ್ಲದೆ
ಪೂರ್ಣ-ವಾಗುತ್ತ
ಪೂರ್ಣ-ವಾಗುತ್ತವೆ
ಪೂರ್ಣ-ವಾಗು-ವಂತೆ
ಪೂರ್ಣ-ವಾಗು-ವ-ವರೆಗೆ
ಪೂರ್ಣ-ವಾಗು-ವು-ದಿಲ್ಲ
ಪೂರ್ಣ-ವಾ-ಗು-ವುದು
ಪೂರ್ಣ-ವಾ-ಗು-ವುವು
ಪೂರ್ಣ-ವಾದ
ಪೂರ್ಣ-ವಾ-ದಂತೆ
ಪೂರ್ಣ-ವಾ-ದದ್ದು
ಪೂರ್ಣ-ವಾ-ದರೊ
ಪೂರ್ಣ-ವಾ-ದಾಗ
ಪೂರ್ಣ-ವಾದುದು
ಪೂರ್ಣವೂ
ಪೂರ್ಣವೇ
ಪೂರ್ಣ-ಸ-ಮತ್ವ
ಪೂರ್ಣ-ಸಿದ್ಧ-ರಾಗ-ಕೂಡ
ಪೂರ್ಣಾತ್ಮ
ಪೂರ್ಣಾತ್ಮ-ನ-ವರೆಗೆ
ಪೂರ್ಣಾತ್ಮ-ನಿಗೆ
ಪೂರ್ಣಾತ್ಮ-ರಾ-ದಾಗ
ಪೂರ್ಣಾ-ನಂದ-ವಿ-ರು-ವುದು
ಪೂರ್ಣಾ-ವಸ್ಥೆ-ಯನ್ನು
ಪೂರ್ತಿ
ಪೂರ್ತಿ-ಗೊಳಿ-ಸು-ವುದು
ಪೂರ್ತಿ-ಮಾಡ-ಬೇಕೆಂದು
ಪೂರ್ತಿ-ಯಾಗಲೇ
ಪೂರ್ತಿ-ಯಾಗಿ
ಪೂರ್ತಿ-ಯಾ-ಯಿತು
ಪೂರ್ವ
ಪೂರ್ವ-ಕರ್ಮ
ಪೂರ್ವ-ಕರ್ಮದ
ಪೂರ್ವ-ಕ-ವಾಗಿ
ಪೂರ್ವಕಾ
ಪೂರ್ವ-ಕಾ-ಲದ
ಪೂರ್ವಗ್ರಹ-ಗ-ಳನ್ನು
ಪೂರ್ವ-ಜನ್ಮದ
ಪೂರ್ವ-ಜನ್ಮ-ವೆಲ್ಲ-ವನ್ನೂ
ಪೂರ್ವ-ಜರ
ಪೂರ್ವ-ಜ-ರಿಂದ
ಪೂರ್ವ-ಜರು
ಪೂರ್ವ-ಜಾತಿಜ್ಞಾನಮ್
ಪೂರ್ವದ
ಪೂರ್ವ-ದಲ್ಲಿ
ಪೂರ್ವ-ದಿಕ್ಕು
ಪೂರ್ವ-ವಾಸ-ನೆ-ಗಳ
ಪೂರ್ವ-ಸಂಸ್ಕಾರ-ಗಳ
ಪೂರ್ವ-ಸಂಸ್ಕಾರ-ಗಳೊಂದಿಗೆ
ಪೂರ್ವ-ಸಂಸ್ಕಾರ-ದಿಂದ
ಪೂರ್ವ-ಸಂಸ್ಕಾರ-ಫಲ
ಪೂರ್ವಾ-ಚಾರ-ಬದ್ಧರು
ಪೂರ್ವಿಕ
ಪೂರ್ವಿಕರ
ಪೂರ್ವಿಕರು
ಪೂರ್ವೇಭ್ಯಃ
ಪೂರ್ವೇಷಾ-ಮಪಿ
ಪೃಥಕ್ಭಾವ
ಪೃಥ್ವಿ
ಪೃಥ್ವಿಗೆ
ಪೃಥ್ವಿಯ
ಪೃಥ್ವಿ-ಯಲ್ಲಿ
ಪೃಥ್ವಿ-ಯಲ್ಲಿ-ರುವ
ಪೃಥ್ವಿ-ಯಾಗಿ-ರು-ವುದು
ಪೃಥ್ವಿ-ಯಾ-ಗು-ವುದು
ಪೃಥ್ವಿಯು
ಪೃಥ್ವಿ-ಸೂರ್ಯ-ಚಂದ್ರ-ತಾರೆ-ಗಳಂತೆ
ಪೃಥ್ವೀ-ಮಾತೆ
ಪೆಚ್ಚು
ಪೆಟ್ಟನ್ನು
ಪೆಟ್ಟಲ್ಲ
ಪೆಟ್ಟಿಗೆ-ಯಂತೆ
ಪೆಟ್ಟಿನ
ಪೆಟ್ಟು
ಪೆಟ್ಟು-ಗಳ
ಪೆಟ್ಟು-ಗ-ಳನ್ನು
ಪೆಟ್ಟು-ಬಿದ್ದರೆ
ಪೆಟ್ಟು-ಬಿದ್ದಾಗ
ಪೆಟ್ಟೂ
ಪೆಟ್ಟೊಂದು
ಪೈ
ಪೈಕಿ
ಪೈಗಳ
ಪೈಪೋಟಿ-ಯಿಲ್ಲದೆ
ಪೊಳ್ಳು
ಪೋಟೋ
ಪೋಟೋ-ಗ-ಳನ್ನು
ಪೋಟೋ-ಗಳಿ-ಗಿಂತ
ಪೋಪ-ರನ್ನು
ಪೋಲೀಸಿ
ಪೋಲೀಸಿ-ನವ-ನಂತೆ
ಪೋಲೀಸಿ-ನ-ವನು
ಪೋಲು
ಪೋಷಣೆ-ಯಲ್ಲ
ಪೋಷಣೆ-ಯಲ್ಲಿ
ಪೋಷಾಕು
ಪೋಷಿ-ಸುವಳು
ಪೋಷಿಸು-ವಷ್ಟು
ಪೋಷಿಸು-ವು-ದನ್ನು
ಪೌಂಡನ್ನು
ಪೌಂಡಿ-ನಷ್ಟು
ಪೌಂಡು-ಗಳು
ಪೌರಾಣಿಕ
ಪೌರುಷ
ಪೌರುಷ-ವಂತ-ರಿಲ್ಲ
ಪೌರುಷ-ವನ್ನು
ಪ್ಯಾಸಡೇ-ನಾದ
ಪ್ರಕಟ-ಪಡಿಸು
ಪ್ರಕಟ-ವಾಗಲು
ಪ್ರಕಟ-ವಾದರೂ
ಪ್ರಕಟ-ವಾದೀತು
ಪ್ರಕ-ಟಿತ-ವಾಗಿವೆ
ಪ್ರಕಾರ
ಪ್ರಕಾಶ
ಪ್ರಕಾಶಕ್ಕೆ
ಪ್ರಕಾಶಕ್ರಿಯಾಸ್ಥಿತಿ-ಶೀಲಂ
ಪ್ರಕಾಶ-ದಿಂದ
ಪ್ರಕಾಶ-ಪೂರ್ಣ
ಪ್ರಕಾಶ-ಮಾನ-ನಾದು-ದ-ರಿಂದ
ಪ್ರಕಾಶ-ಮಾನ-ವಾ-ಗಿ-ರುವ
ಪ್ರಕಾಶ-ಮಾನ-ವಾ-ಗಿ-ರುವಾಗ
ಪ್ರಕಾಶ-ಮಾನ-ವಾ-ಗಿ-ರುವುದು
ಪ್ರಕಾಶ-ಮಾನ-ವಾಗುತ್ತದೆ
ಪ್ರಕಾಶ-ಮಾನ-ವಾದ
ಪ್ರಕಾಶ-ಮಾನವೂ
ಪ್ರಕಾಶ-ವನ್ನು
ಪ್ರಕಾಶ-ವಾಗದೆ
ಪ್ರಕಾಶ-ವಾಗಿದ್ದರೆ
ಪ್ರಕಾಶ-ವಿದು
ಪ್ರಕಾಶವು
ಪ್ರಕಾಶವೇ
ಪ್ರಕಾಶಾತ್ಮನೆ
ಪ್ರಕಾಶಾ-ವರ-ಣಕ್ಷಯಃ
ಪ್ರಕಾಶಾ-ವರ-ಣಮ್
ಪ್ರಕಾಶಾ-ಸಂಯೋ-ಗೇ-ತರ್ಧಾನಮ್
ಪ್ರಕಾಶಿತ-ವಾಗ-ಲಾರದು
ಪ್ರಕಾಶಿ-ಸದು
ಪ್ರಕಾಶಿಸ-ಬೇಕಾ-ಗಿದೆ
ಪ್ರಕಾಶಿಸಿ
ಪ್ರಕಾಶಿಸು
ಪ್ರಕಾಶಿಸುತ್ತದೆ
ಪ್ರಕಾಶಿಸುತ್ತಿದ್ದರೂ
ಪ್ರಕಾಶಿಸುತ್ತಿ-ರುವ
ಪ್ರಕಾಶಿಸುತ್ತಿ-ರುವ-ವನೂ
ಪ್ರಕಾಶಿಸುತ್ತಿ-ರುವುದು
ಪ್ರಕಾಶಿ-ಸುವ
ಪ್ರಕಾಶಿಸು-ವನೋ
ಪ್ರಕಾಶಿಸು-ವು-ದ-ರಿಂದ
ಪ್ರಕಾಶಿಸು-ವುದು
ಪ್ರಕಾಶಿಸು-ವುದು-ಅ-ವನ
ಪ್ರಕಾಶಿಸು-ವುದೇ
ಪ್ರಕೃತ
ಪ್ರಕೃತಿ
ಪ್ರಕೃತಿ-ಗಳ
ಪ್ರಕೃತಿ-ಗಳೆ-ರಡೂ
ಪ್ರಕೃತಿ-ಗಿಂತ
ಪ್ರಕೃ-ತಿಗೆ
ಪ್ರಕೃತಿ-ಜಯ-ಇ-ವು-ಗಳೆಲ್ಲ
ಪ್ರಕೃತಿಜ್ಞಾನ-ವಿ-ರು-ವುದು
ಪ್ರಕೃತಿ-ನಿಗ್ರ-ಹಕ್ಕಾಗಿ
ಪ್ರಕೃತಿ-ನಿಗ್ರ-ಹ-ವೆಂದು
ಪ್ರಕೃತಿ-ನಿಯ-ಮ-ಗ-ಳೆಂದು
ಪ್ರಕೃತಿ-ಪೂಜೆ-ಯಿಂದ
ಪ್ರಕೃತಿ-ಪೂರ-ಣ-ದಿಂದ
ಪ್ರಕೃತಿಯ
ಪ್ರಕೃತಿ-ಯಂತೆ
ಪ್ರಕೃತಿ-ಯನ್ನು
ಪ್ರಕೃತಿ-ಯನ್ನೂ
ಪ್ರಕೃತಿ-ಯಲ್ಲ
ಪ್ರಕೃತಿ-ಯಲ್ಲ-ವೆಂದು
ಪ್ರಕೃತಿ-ಯಲ್ಲಿ
ಪ್ರಕೃತಿ-ಯಲ್ಲಿದೆ
ಪ್ರಕೃತಿ-ಯಲ್ಲಿನ
ಪ್ರಕೃತಿ-ಯಲ್ಲಿ-ರುವ
ಪ್ರಕೃತಿ-ಯಲ್ಲಿ-ರುವುದು
ಪ್ರಕೃತಿ-ಯಲ್ಲಿವೆ
ಪ್ರಕೃತಿ-ಯ-ವ-ರನ್ನು
ಪ್ರಕೃತಿ-ಯ-ವ-ರಿ-ಗೆಲ್ಲ
ಪ್ರಕೃತಿ-ಯ-ವರಿ-ರು-ವರು
ಪ್ರಕೃತಿ-ಯ-ವರೇ
ಪ್ರಕೃತಿ-ಯಾ-ಗಲಿ
ಪ್ರಕೃತಿ-ಯಾಚೆ
ಪ್ರಕೃತಿ-ಯಾಚೆ-ಯಿ-ರು-ವುದು
ಪ್ರಕೃತಿ-ಯಿಂದ
ಪ್ರಕೃತಿ-ಯಿಂದಲೇ
ಪ್ರಕೃತಿಯು
ಪ್ರಕೃತಿಯೂ
ಪ್ರಕೃತಿಯೆ
ಪ್ರಕೃತಿ-ಯೆಂಬ
ಪ್ರಕೃತಿ-ಯೆಲ್ಲ
ಪ್ರಕೃತಿ-ಯೆಲ್ಲ-ವನ್ನೂ
ಪ್ರಕೃತಿಯೇ
ಪ್ರಕೃತಿ-ಯೊಂದಿಗೆ
ಪ್ರಕೃತಿ-ಯೊಡ್ಡುವ
ಪ್ರಕೃತಿ-ಲಯ-ರಿಗೆ
ಪ್ರಕೃತಿ-ಲಯಾ-ನಾಮ್
ಪ್ರಕೃತಿ-ಲಯಿ-ಗಳು
ಪ್ರಕೃತೀನಾಂ
ಪ್ರಕೃತ್ಯಾ-ಪೂರತ್
ಪ್ರಕ್ರಿಯೆ-ಗಳಲ್ಲಿ
ಪ್ರಖ್ಯಾತ
ಪ್ರಖ್ಯಾತ-ನಾದ
ಪ್ರಖ್ಯಾತ-ರಾಗ-ಬೇಕೆಂಬ
ಪ್ರಖ್ಯಾತ-ರಾದ
ಪ್ರಖ್ಯಾತ-ವಾದ
ಪ್ರಗತಿ
ಪ್ರಗ-ತಿಗೆ
ಪ್ರಗ-ತಿಯ
ಪ್ರಗತಿ-ಯನ್ನು
ಪ್ರಗತಿ-ಯಲ್ಲ
ಪ್ರಗ-ತಿಯೂ
ಪ್ರಚಂಡ
ಪ್ರಚಂಡ-ವಾದ
ಪ್ರಚಂಡ-ವಾದುದು
ಪ್ರಚಲಿತ-ವಾಗಿದೆ
ಪ್ರಚಾರ
ಪ್ರಚಾ-ರಕ್ಕೆ
ಪ್ರಚಾರ-ಗೊಳಿ-ಸುವರು
ಪ್ರಚಾರಣಾ-ಯಂತ್ರ
ಪ್ರಚಾರ-ದಲ್ಲಿ
ಪ್ರಚಾರ-ದಲ್ಲಿ-ರುವ
ಪ್ರಚಾರ-ಮಾಡಲು
ಪ್ರಚಾರ-ಮಾಡು-ವುದು
ಪ್ರಚಾರ-ವಾಗ-ಬೇಕಾದ
ಪ್ರಚಾರ-ವಾದ
ಪ್ರಚಾರ-ಸಂವೇದ-ನಾಚ್ಛ
ಪ್ರಚೋ-ದಿಸಲಿ
ಪ್ರಚೋ-ದಿಸು-ತ್ತಿರು-ವುದು
ಪ್ರಚೋ-ದಿಸುವ
ಪ್ರಚೋ-ದಿಸು-ವುದು
ಪ್ರಚೋ-ದಿಸು-ವುವು
ಪ್ರಚ್ಛರ್ದನವಿ-ಧಾರ-ಣಾಭ್ಯಾಂ
ಪ್ರಜಾಪ್ರಭುತ್ವದ
ಪ್ರಜೆ-ಗಳಿ-ರು-ವರು
ಪ್ರಜ್ಞ-ಗಳೆಲ್ಲ-ದರ
ಪ್ರಜ್ಞ-ಯಿಂದ
ಪ್ರಜ್ಞಾ
ಪ್ರಜ್ಞಾ-ಕೇಂದ್ರ-ದಲ್ಲಾ-ದರೂ
ಪ್ರಜ್ಞಾ-ತೀತ
ಪ್ರಜ್ಞಾ-ತೀತದ
ಪ್ರಜ್ಞಾ-ತೀತ-ನೆಂಬು-ದನ್ನು
ಪ್ರಜ್ಞಾ-ತೀತ-ವಾದ
ಪ್ರಜ್ಞಾ-ತೀತಸ್ಥಿತಿ
ಪ್ರಜ್ಞಾ-ತೀ-ತಾ-ವಸ್ಥೆ
ಪ್ರಜ್ಞಾ-ತೀ-ತಾ-ವಸ್ಥೆಗೆ
ಪ್ರಜ್ಞಾ-ತೀ-ತಾ-ವಸ್ಥೆ-ಯನ್ನು
ಪ್ರಜ್ಞಾ-ತೀ-ತಾ-ವಸ್ಥೆ-ಯಾ-ಗು-ವುದು
ಪ್ರಜ್ಞಾ-ಪೂರ್ಣ
ಪ್ರಜ್ಞಾ-ಪೂರ್ವಕ
ಪ್ರಜ್ಞಾ-ಪೂರ್ವ-ಕ-ವಾಗಿ
ಪ್ರಜ್ಞಾ-ರ-ಹಿತ
ಪ್ರಜ್ಞಾ-ಲೋಕಃ
ಪ್ರಜ್ಞಾಸ್ತ-ರಕ್ಕೆ
ಪ್ರಜ್ಞೆ
ಪ್ರಜ್ಞೆಗೆ
ಪ್ರಜ್ಞೆಯ
ಪ್ರಜ್ಞೆ-ಯನ್ನು
ಪ್ರಜ್ಞೆ-ಯ-ವರೆಗೆ
ಪ್ರಜ್ಞೆ-ಯಿಂದ
ಪ್ರಜ್ಞೆಯು
ಪ್ರಜ್ಞೆಯೂ
ಪ್ರಜ್ಞೆಯೇ
ಪ್ರಜ್ವಲ-ನಮ್
ಪ್ರಣವಃ
ಪ್ರಣವಜಪ-ದಿಂದ
ಪ್ರಣಾಳ
ಪ್ರಣಾಳಿಕೆ-ಯನ್ನು
ಪ್ರಣಿ-ಧಾನಾನಿ
ಪ್ರತಾಪ
ಪ್ರತಿ
ಪ್ರತಿ-ಕೊಂಡಿಯೂ
ಪ್ರತಿಕ್ರಿಯಾ
ಪ್ರತಿಕ್ರಿ-ಯಿಸು-ವಂತೆ
ಪ್ರತಿಕ್ರಿಯೆ
ಪ್ರತಿಕ್ರಿಯೆ-ಇವು-ಗಳ
ಪ್ರತಿಕ್ರಿಯೆ-ಗಳ
ಪ್ರತಿಕ್ರಿಯೆಗೆ
ಪ್ರತಿಕ್ರಿಯೆಯ
ಪ್ರತಿಕ್ರಿಯೆ-ಯನ್ನು
ಪ್ರತಿಕ್ರಿಯೆ-ಯನ್ನುಂಟು-ಮಾಡು-ವುದು
ಪ್ರತಿಕ್ರಿಯೆ-ಯಲ್ಲಿ
ಪ್ರತಿಕ್ರಿಯೆ-ಯಾಗಿ
ಪ್ರತಿಕ್ರಿಯೆ-ಯಾ-ಗು-ವುದು
ಪ್ರತಿಕ್ರಿಯೆ-ಯಿಂದ
ಪ್ರತಿಕ್ರಿಯೆಯು
ಪ್ರತಿಕ್ರಿಯೆಯೂ
ಪ್ರತಿಕ್ರಿಯೆಯೇ
ಪ್ರತಿಕ್ರಿಯೆ-ಯೊಂದಿಗೆ
ಪ್ರತಿಕ್ಷಣ
ಪ್ರತಿಕ್ಷಣ-ದಲ್ಲಿಯೂ
ಪ್ರತಿಕ್ಷಣವೂ
ಪ್ರತಿಜ್ಞೆ
ಪ್ರತಿ-ದಿನ
ಪ್ರತಿ-ದಿನದ
ಪ್ರತಿ-ದಿನ-ದಲ್ಲಿಯೂ
ಪ್ರತಿ-ದಿನವೂ
ಪ್ರತಿ-ದೇಶ-ದಲ್ಲಿಯೂ
ಪ್ರತಿ-ನಿಧಿ
ಪ್ರತಿ-ನಿಧಿ-ಯಾಗ-ಬಲ್ಲದು
ಪ್ರತಿ-ನಿಧಿ-ಯಾ-ಗಿದೆ
ಪ್ರತಿ-ನಿಧಿ-ಯಾಗಿ-ದೆಯೊ
ಪ್ರತಿ-ನಿಧಿ-ಸುತ್ತದೆ
ಪ್ರತಿ-ನಿಧಿ-ಸುತ್ತವೆ
ಪ್ರತಿ-ನಿಧಿ-ಸು-ವರು
ಪ್ರತಿ-ಪಕ್ಷ-ದ-ವರ
ಪ್ರತಿ-ಪಕ್ಷ-ಭಾವ-ನಮ್
ಪ್ರತಿ-ಪತ್ತಿಃ
ಪ್ರತಿ-ಪಾದನೆ
ಪ್ರತಿ-ಪಾದನೆಯ
ಪ್ರತಿ-ಪಾದನೆ-ಯಾ-ಗಿದೆ
ಪ್ರತಿ-ಪಾದಿ
ಪ್ರತಿ-ಪಾದಿಸ
ಪ್ರತಿ-ಪಾದಿಸಿ
ಪ್ರತಿ-ಪಾದಿ-ಸಿ-ದರು
ಪ್ರತಿ-ಪಾದಿ-ಸಿ-ದುದು
ಪ್ರತಿ-ಪಾದಿ-ಸುತ್ತವೆ
ಪ್ರತಿ-ಪಾದಿ-ಸುತ್ತೇವೆ
ಪ್ರತಿ-ಪಾದಿ-ಸುವ
ಪ್ರತಿ-ಪಾದಿ-ಸುವರು
ಪ್ರತಿ-ಪಾದಿ-ಸುವು-ದ-ರಲ್ಲಿ
ಪ್ರತಿ-ಪಾದಿ-ಸು-ವುದು
ಪ್ರತಿ-ಪಾದಿ-ಸು-ವುದೇನೋ
ಪ್ರತಿಪ್ರಸವಃ
ಪ್ರತಿಪ್ರಸವಹೇಯಾಃ
ಪ್ರತಿ-ಫಲ
ಪ್ರತಿ-ಫಲ-ವನ್ನು
ಪ್ರತಿ-ಫಲವು
ಪ್ರತಿ-ಫಲ-ವೆಂಬು-ದನ್ನು
ಪ್ರತಿ-ಫಲ-ವೆಲ್ಲ
ಪ್ರತಿ-ಫಲಾಪೇಕ್ಷೆಯೂ
ಪ್ರತಿ-ಬಿಂಬ
ಪ್ರತಿ-ಬಿಂಬ-ಕದ
ಪ್ರತಿ-ಬಿಂಬ-ಕವು
ಪ್ರತಿ-ಬಿಂಬ-ಗ-ಳನ್ನು
ಪ್ರತಿ-ಬಿಂಬ-ಗಳು
ಪ್ರತಿ-ಬಿಂಬದ
ಪ್ರತಿ-ಬಿಂಬ-ವನ್ನು
ಪ್ರತಿ-ಬಿಂಬ-ವನ್ನೂ
ಪ್ರತಿ-ಬಿಂಬವು
ಪ್ರತಿ-ಬಿಂಬವೇ
ಪ್ರತಿ-ಬಿಂಬಿತ-ವಾಗಿ
ಪ್ರತಿ-ಬಿಂಬಿತ-ವಾಗಿ-ವೆ-ಸತ್
ಪ್ರತಿ-ಬಿಂಬಿತ-ವಾಗು-ವಂತೆ
ಪ್ರತಿ-ಬಿಂಬಿಸ-ಬಲ್ಲಿರಿ
ಪ್ರತಿ-ಬಿಂಬಿಸಿ-ದಂತೆ
ಪ್ರತಿ-ಬಿಂಬಿಸಿ-ದರೆ
ಪ್ರತಿ-ಬಿಂಬಿಸಿ-ದಾಗ
ಪ್ರತಿ-ಬಿಂಬಿ-ಸುವ
ಪ್ರತಿ-ಬಿಂಬಿ-ಸು-ವುದು
ಪ್ರತಿಭಾ
ಪ್ರತಿ-ಭಾ-ವಂತ
ಪ್ರತಿ-ಭಾ-ವಂತ-ರಿಗೆ
ಪ್ರತಿ-ಭಾ-ವಂತರು
ಪ್ರತಿ-ಭಾ-ಶಾಲಿ-ಗಳಲ್ಲಿ
ಪ್ರತಿ-ಭಾ-ಶಾಲಿ-ಯಾಗಿ-ರ-ಬೇಕು
ಪ್ರತಿಭೆ
ಪ್ರತಿ-ಭೆಗೆ
ಪ್ರತಿ-ಭೆಯ
ಪ್ರತಿ-ಯಾಗಿ
ಪ್ರತಿ-ಯೊಂದಕ್ಕೂ
ಪ್ರತಿ-ಯೊಂದನ್ನು
ಪ್ರತಿ-ಯೊಂದನ್ನೂ
ಪ್ರತಿ-ಯೊಂದರ
ಪ್ರತಿ-ಯೊಂದರಲ್ಲಿಯೂ
ಪ್ರತಿ-ಯೊಂದು
ಪ್ರತಿ-ಯೊಂದೂ
ಪ್ರತಿ-ಯೊದು
ಪ್ರತಿ-ಯೊಬ್ಬ
ಪ್ರತಿ-ಯೊಬ್ಬನ
ಪ್ರತಿ-ಯೊಬ್ಬ-ನಲ್ಲಿ
ಪ್ರತಿ-ಯೊಬ್ಬ-ನಲ್ಲಿಯೂ
ಪ್ರತಿ-ಯೊಬ್ಬ-ನಿಗೂ
ಪ್ರತಿ-ಯೊಬ್ಬನು
ಪ್ರತಿ-ಯೊಬ್ಬನೂ
ಪ್ರತಿ-ಯೊಬ್ಬರ
ಪ್ರತಿ-ಯೊಬ್ಬ-ರನ್ನು
ಪ್ರತಿ-ಯೊಬ್ಬ-ರನ್ನೂ
ಪ್ರತಿ-ಯೊಬ್ಬ-ರಲ್ಲಿಯೂ
ಪ್ರತಿ-ಯೊಬ್ಬ-ರಿಗೂ
ಪ್ರತಿ-ಯೊಬ್ಬರೂ
ಪ್ರತಿ-ಯೊಬ್ಬರೊ
ಪ್ರತಿ-ಯೊಬ್ಬ-ವಿದ್ಯಾರ್ಥಿಯೂ
ಪ್ರತಿ-ರೂಪ
ಪ್ರತಿ-ರೂಪಕ್ಕೆ
ಪ್ರತಿ-ರೂಪ-ಗ-ಳನ್ನು
ಪ್ರತಿ-ರೂಪದ
ಪ್ರತಿ-ರೂಪವು
ಪ್ರತಿ-ವರುಷ
ಪ್ರತಿ-ವರು-ಷವೂ
ಪ್ರತಿ-ಷೇ-ಧಾರ್ಥಮೇಕ-ತತ್ತ್ವಾಭ್ಯಾಸಃ
ಪ್ರತಿಷ್ಟ-ವಾಗುತ್ತದೆ
ಪ್ರತಿಷ್ಟಿತ-ನಾಗು-ವುದೇ
ಪ್ರತಿಷ್ಠ-ನಾದ
ಪ್ರತಿಷ್ಠಾ-ವಂತ-ರಿಗೆ
ಪ್ರತಿಷ್ಠಿತ-ನಾದ
ಪ್ರತಿಷ್ಠಿ-ತರಾ-ಗಿ-ರುವೆ-ವೆಂದು
ಪ್ರತಿಷ್ಠಿತ-ವಾ-ಗಿ-ರುವುದು
ಪ್ರತಿಷ್ಠಿತ-ವಾದ
ಪ್ರತಿಷ್ಠಿ-ವಾ-ಗಿ-ರುವುದು
ಪ್ರತಿಷ್ಠೆ
ಪ್ರತಿ-ಸಲ
ಪ್ರತಿ-ಸಲವೂ
ಪ್ರತಿ-ಸಾರಿಯೂ
ಪ್ರತೀಕ
ಪ್ರತ್ಯಕ್
ಪ್ರತ್ಯಕ್ಷ
ಪ್ರತ್ಯಕ್ಷ-ದೇವ-ರಂತೆ
ಪ್ರತ್ಯಕ್ಷ-ನಾಗು-ವ-ನೆಂದರೆ
ಪ್ರತ್ಯಕ್ಷ-ನಾದ
ಪ್ರತ್ಯಕ್ಷ-ವಾಗಿ
ಪ್ರತ್ಯಕ್ಷ-ವಾ-ಗಿತ್ತು
ಪ್ರತ್ಯಕ್ಷ-ವಾಗಿದೆ
ಪ್ರತ್ಯಕ್ಷ-ವಾದ
ಪ್ರತ್ಯಕ್ಷಾನು-ಮಾನಾಗಮಾಃ
ಪ್ರತ್ಯಯ-ವನ್ನು
ಪ್ರತ್ಯಯಸ್ಯ
ಪ್ರತ್ಯಯಾ-ನುಪಶ್ಯಃ
ಪ್ರತ್ಯಯಾನ್ತರಾಣಿ
ಪ್ರತ್ಯಯಾಭ್ಯಾಸ-ಪೂರ್ವಃ
ಪ್ರತ್ಯಯಾ-ವಿಶೇಷಾದ್
ಪ್ರತ್ಯಯೈಕತಾನತಾ
ಪ್ರತ್ಯಾ-ಹಾರ
ಪ್ರತ್ಯಾ-ಹಾರಃ
ಪ್ರತ್ಯಾ-ಹಾರದ
ಪ್ರತ್ಯಾ-ಹಾರ-ದಲ್ಲಿ
ಪ್ರತ್ಯಾ-ಹಾರ-ವನ್ನು
ಪ್ರತ್ಯಾ-ಹಾರ-ವೆಂದರೆ
ಪ್ರತ್ಯಾ-ಹಾರ-ವೆನ್ನು-ವುದು
ಪ್ರತ್ಯುತ್ತರ
ಪ್ರತ್ಯೇಕ
ಪ್ರತ್ಯೇಕ-ಗೊಂಡ-ವನು
ಪ್ರತ್ಯೇಕ-ಗೊಳಿ-ಸ-ಬಹುದು
ಪ್ರತ್ಯೇಕ-ಗೊಳಿಸಿ
ಪ್ರತ್ಯೇಕತಾ
ಪ್ರತ್ಯೇ-ಕತೆ
ಪ್ರತ್ಯೇಕ-ತೆಯ
ಪ್ರತ್ಯೇಕ-ಪಡಿ-ಸು-ವುದು
ಪ್ರತ್ಯೇಕ-ಭಾ-ವನೆ
ಪ್ರತ್ಯೇಕ-ಭಾವ-ನೆ-ಇವು-ಗಳಿಂದ
ಪ್ರತ್ಯೇಕ-ವಾಗಿ
ಪ್ರತ್ಯೇಕ-ವಾಗಿ-ರುವ
ಪ್ರತ್ಯೇಕ-ವಾಗಿ-ರುವುದು
ಪ್ರತ್ಯೇಕ-ವಾದ
ಪ್ರತ್ಯೇಕ-ವಾದು-ದೆಂದೂ
ಪ್ರತ್ಯೇಕ-ವಾದುವು
ಪ್ರತ್ಯೇಕ-ವೆಂದು
ಪ್ರತ್ಯೇಕಾತ್ಮ-ನಲ್ಲಿ
ಪ್ರತ್ಯೇಕಾತ್ಮ-ನಿಗೆ
ಪ್ರತ್ಯೇಕಿಸ-ಬಹುದು
ಪ್ರತ್ಯೇಕಿಸ-ಲಾ-ಗದ
ಪ್ರತ್ಯೇಕಿಸ-ಲಾಗ-ದಷ್ಟು
ಪ್ರತ್ಯೇಕಿಸ-ಲಾಗು-ವು-ದಿಲ್ಲ
ಪ್ರತ್ಯೇಕಿ-ಸಲು
ಪ್ರತ್ಯೇಕಿ-ಸಲ್ಪ-ಡುವ
ಪ್ರತ್ಯೇಕಿಸಿ
ಪ್ರತ್ಯೇಕಿ-ಸು-ವು-ದಕ್ಕೆ
ಪ್ರತ್ಯೇಕಿಸು-ವು-ದ-ರಿಂದ
ಪ್ರಥಮ
ಪ್ರದಕ್ಷಿಣೆ-ಯನ್ನು
ಪ್ರದರ್ಶನಾಲ-ಯಕ್ಕೆ
ಪ್ರದರ್ಶಿ-ಸ-ಬಹುದು
ಪ್ರದರ್ಶಿ-ಸ-ಬೇಕಾ-ಗುತ್ತದೆ
ಪ್ರದರ್ಶಿ-ಸ-ಬೇಕು
ಪ್ರದರ್ಶಿ-ಸಿ-ರು-ವರು
ಪ್ರದರ್ಶಿ-ಸುವು-ದಕ್ಕೋಸುಗ-ವಾಗಿ
ಪ್ರದ-ವಾದ
ಪ್ರದೇಶ
ಪ್ರದೇ-ಶ-ಗ-ಳನ್ನು
ಪ್ರದೇ-ಶ-ದಿಂದ
ಪ್ರದೇ-ಶ-ವನ-ನಾಗಿ
ಪ್ರದೇ-ಶವೂ
ಪ್ರಧಾನ
ಪ್ರಧಾನ-ಜಯಶ್ಚ
ಪ್ರಧಾನ-ದೊಂದಿಗೆ
ಪ್ರಧಾನ-ವಾದ
ಪ್ರಪಂಚ
ಪ್ರಪಂಚಕ್ಕಾಗಿ
ಪ್ರಪಂಚಕ್ಕೂ
ಪ್ರಪಂಚಕ್ಕೆ
ಪ್ರಪಂಚಕ್ಕೇ
ಪ್ರಪಂಚ-ಗಳ
ಪ್ರಪಂಚ-ಗಳಿಗೆ
ಪ್ರಪಂಚ-ಗಳೂ
ಪ್ರಪಂಚಜ್ಞಾನ
ಪ್ರಪಂಚದ
ಪ್ರಪಂಚ-ದಂತೆ
ಪ್ರಪಂಚ-ದಂತೆಯೇ
ಪ್ರಪಂಚ-ದಲ್ಲಿ
ಪ್ರಪಂಚ-ದಲ್ಲಿಯೂ
ಪ್ರಪಂಚ-ದಲ್ಲಿಯೇ
ಪ್ರಪಂಚ-ದಲ್ಲಿ-ರ-ಬೇಕು
ಪ್ರಪಂಚ-ದಲ್ಲಿ-ರುವ
ಪ್ರಪಂಚ-ದಲ್ಲಿ-ರುವ-ವ-ರನ್ನು
ಪ್ರಪಂಚ-ದಲ್ಲಿ-ರುವ-ವರು
ಪ್ರಪಂಚ-ದಲ್ಲಿ-ರುವುದು
ಪ್ರಪಂಚ-ದಲ್ಲಿ-ರುವು-ದೆಲ್ಲ
ಪ್ರಪಂಚ-ದಲ್ಲಿ-ರು-ವೆವು
ಪ್ರಪಂಚ-ದಲ್ಲೆ
ಪ್ರಪಂಚ-ದಲ್ಲೆಲ್ಲ
ಪ್ರಪಂಚ-ದಲ್ಲೆಲ್ಲಾ
ಪ್ರಪಂಚ-ದಾಚೆ
ಪ್ರಪಂಚ-ದಿಂದ
ಪ್ರಪಂಚ-ದೊಂದಿಗೆ
ಪ್ರಪಂಚ-ದೊ-ಡನೆ
ಪ್ರಪಂಚ-ದೊ-ಳಗೆ
ಪ್ರಪಂಚ-ವನ್ನಾ-ಳುವ
ಪ್ರಪಂಚ-ವನ್ನು
ಪ್ರಪಂಚ-ವನ್ನೆಲ್ಲ
ಪ್ರಪಂಚ-ವನ್ನೇ
ಪ್ರಪಂಚ-ವನ್ನೇಕೆ
ಪ್ರಪಂಚ-ವಿದೆ
ಪ್ರಪಂಚ-ವಿಲ್ಲ
ಪ್ರಪಂಚವು
ಪ್ರಪಂಚವೂ
ಪ್ರಪಂಚವೆ
ಪ್ರಪಂಚ-ವೆಂದ-ರೇನು
ಪ್ರಪಂಚ-ವೆಂಬ
ಪ್ರಪಂಚ-ವೆನ್ನು-ವುದು
ಪ್ರಪಂಚ-ವೆಲ್ಲ
ಪ್ರಪಂಚ-ವೆಲ್ಲವೂ
ಪ್ರಪಂಚ-ವೆಲ್ಲಾ
ಪ್ರಪಂಚವೇ
ಪ್ರಪಂಚ್ದಲ್ಲಿ
ಪ್ರಪಂಡ
ಪ್ರಪದ್ಯನ್ತೇ
ಪ್ರಪೂರ್ವಕ
ಪ್ರಬಲ-ವಾಗಿ
ಪ್ರಬಲ-ವಾ-ಗಿ-ರು-ವಂತೆ
ಪ್ರಬಲ-ವಾಗಿ-ರು-ವು-ದಿಲ್ಲ
ಪ್ರಬಲ-ವಾ-ಗಿ-ರುವುದು
ಪ್ರಬಲ-ವಾಗುತ್ತ
ಪ್ರಬಲ-ವಾಗುತ್ತಿದ್ದಾರೆ
ಪ್ರಬಲ-ವಾ-ಗು-ವುದು
ಪ್ರಬಲ-ವಾದ
ಪ್ರಬಲ-ವಾದುದು
ಪ್ರಬುದ್ಧ-ವಾಗಿರು
ಪ್ರಭಾವ
ಪ್ರಭಾವಕ್ಕೂ
ಪ್ರಭಾ-ವಕ್ಕೆ
ಪ್ರಭಾವ-ದಿಂದ
ಪ್ರಭಾವನ
ಪ್ರಭಾವ-ವನ್ನು
ಪ್ರಭಾವ-ವ-ವನ್ನು
ಪ್ರಭಾವವು
ಪ್ರಭಾವವೂ
ಪ್ರಭಾವಿಸಿ-ಕೊಂಡೆವು
ಪ್ರಭು-ಗಳಾ
ಪ್ರಭು-ಗಳು
ಪ್ರಭುತ್ವ
ಪ್ರಭುತ್ವದ
ಪ್ರಭುತ್ವ-ವನ್ನು
ಪ್ರಭು-ಭೋ-ಜನ
ಪ್ರಭು-ವಲ್ಲ
ಪ್ರಭು-ವಾದ
ಪ್ರಭೆ
ಪ್ರಭೆ-ಯನ್ನು
ಪ್ರಭೆ-ಯಲ್ಲಿ
ಪ್ರಭೆ-ಯಿಂದ
ಪ್ರಭೆ-ಯೊಂದೇ
ಪ್ರಭೋಃ
ಪ್ರಮಾಣ
ಪ್ರಮಾ-ಣಕ್ಕೆ
ಪ್ರಮಾಣ-ಗಳ
ಪ್ರಮಾಣ-ಗಳಲ್ಲಿ
ಪ್ರಮಾಣ-ಗಳಾ-ವು-ವೆಂದರೆ
ಪ್ರಮಾಣ-ಗಳಿವೆ
ಪ್ರಮಾಣ-ಗಳು
ಪ್ರಮಾ-ಣದ
ಪ್ರಮಾಣ-ದಲ್ಲಲ್ಲದೆ
ಪ್ರಮಾಣ-ದಲ್ಲಿ
ಪ್ರಮಾಣ-ದಲ್ಲಿತ್ತು
ಪ್ರಮಾಣ-ದಿಂದ
ಪ್ರಮಾಣ-ವನ್ನು
ಪ್ರಮಾಣ-ವಾಗ-ಲಾರವು
ಪ್ರಮಾಣ-ವಾಗಿದೆ
ಪ್ರಮಾಣ-ವಾಗು-ವು-ದಿಲ್ಲ
ಪ್ರಮಾಣ-ವಾ-ದುದು
ಪ್ರಮಾಣ-ವಿ-ಪರ್ಯಯ-ವಿ-ಕಲ್ಪ-ನಿದ್ರಾಸ್ಮೃತಯಃ
ಪ್ರಮಾಣ-ವಿಲ್ಲ
ಪ್ರಮಾಣ-ವಿಲ್ಲದೆ
ಪ್ರಮಾ-ಣವು
ಪ್ರಮಾ-ಣವೂ
ಪ್ರಮಾಣ-ವೆನ್ನುತ್ತೇವೆ
ಪ್ರಮಾಣ-ವೇನು
ಪ್ರಮಾಣ-ವೇ-ನೆಂದು
ಪ್ರಮಾಣ-ಸ-ಹಿತ
ಪ್ರಮಾಣಾನಿ
ಪ್ರಮಾಣೀ-ಕರಿ-ಸಲಾ-ಗಿದೆ
ಪ್ರಮಾಣೀ-ಕರಿಸ-ಲಾಗು-ವು-ದಿಲ್ಲ
ಪ್ರಮಾಣೀ-ಕರಿ-ಸು-ವು-ದಕ್ಕೆ
ಪ್ರಮಾಣೀ-ಕರಿ-ಸು-ವುದು
ಪ್ರಮಾಣೀಕ-ರಿಸು-ವುದೂ
ಪ್ರಮುಖ-ವಲ್ಲದ
ಪ್ರಮೇಯ
ಪ್ರಮೇಯ-ಗಳು
ಪ್ರಮೇಯ-ವಿದೆ
ಪ್ರಯತ್ನ
ಪ್ರಯತ್ನಕ್ಕೆ
ಪ್ರಯತ್ನ-ಗ-ಳನ್ನು
ಪ್ರಯತ್ನ-ಗಳಲ್ಲಿ
ಪ್ರಯತ್ನ-ಗಳಾ-ಗಿವೆ
ಪ್ರಯತ್ನ-ಗಳಿಂದ
ಪ್ರಯತ್ನ-ಗಳು
ಪ್ರಯತ್ನ-ಗಳೂ
ಪ್ರಯತ್ನ-ಗಳೆಲ್ಲ
ಪ್ರಯತ್ನದ
ಪ್ರಯತ್ನ-ದಲ್ಲಿ
ಪ್ರಯತ್ನ-ದಿಂದ
ಪ್ರಯತ್ನ-ಪಟ್ಟ
ಪ್ರಯತ್ನ-ಪಟ್ಟರು
ಪ್ರಯತ್ನ-ಪಟ್ಟರೂ
ಪ್ರಯತ್ನ-ಪಟ್ಟರೆ
ಪ್ರಯತ್ನ-ಪಟ್ಟಾಗ
ಪ್ರಯತ್ನ-ಪಟ್ಟಿರೋ
ಪ್ರಯತ್ನ-ಪಟ್ಟಿಲ್ಲ
ಪ್ರಯತ್ನ-ಪಟ್ಟಿವೆ
ಪ್ರಯತ್ನ-ಪಟ್ಟು
ಪ್ರಯತ್ನ-ಪ-ಡುವರು
ಪ್ರಯತ್ನ-ಪಡು-ವು-ದಿಲ್ಲವೊ
ಪ್ರಯತ್ನ-ಪಡು-ವೆವು
ಪ್ರಯತ್ನ-ವನ್ನು
ಪ್ರಯತ್ನ-ವನ್ನೆಲ್ಲ
ಪ್ರಯತ್ನ-ವಾ-ಗಿತ್ತು
ಪ್ರಯತ್ನವೂ
ಪ್ರಯತ್ನ-ವೆಲ್ಲ
ಪ್ರಯತ್ನ-ವೆಲ್ಲವೂ
ಪ್ರಯತ್ನ-ವೆಲ್ಲಾ
ಪ್ರಯತ್ನ-ವೇಕೆ
ಪ್ರಯತ್ನ-ಶೈಥಿಲ್ಯಾ-ನಂತ-ಸಮಾ-ಪತ್ತಿಭ್ಯಾಮ್
ಪ್ರಯತ್ನಿ
ಪ್ರಯತ್ನಿ-ಸಬ-ಹದು
ಪ್ರಯತ್ನಿ-ಸ-ಬಹುದು
ಪ್ರಯತ್ನಿ-ಸ-ಬೇಕಾದ
ಪ್ರಯತ್ನಿ-ಸ-ಬೇಕು
ಪ್ರಯತ್ನಿ-ಸ-ಬೇಡವೇ
ಪ್ರಯತ್ನಿ-ಸಲಾ-ಗಿದೆ
ಪ್ರಯತ್ನಿಸಿ
ಪ್ರಯತ್ನಿ-ಸಿತು
ಪ್ರಯತ್ನಿ-ಸಿದ
ಪ್ರಯತ್ನಿ-ಸಿ-ದಂತೆ
ಪ್ರಯತ್ನಿ-ಸಿ-ದನು
ಪ್ರಯತ್ನಿ-ಸಿ-ದರು
ಪ್ರಯತ್ನಿ-ಸಿ-ದರೆ
ಪ್ರಯತ್ನಿ-ಸಿ-ದಾಗ
ಪ್ರಯತ್ನಿ-ಸಿ-ದಾಗ-ಲೆಲ್ಲಾ
ಪ್ರಯತ್ನಿ-ಸಿದ್ದರು-ಈ-ಗಲೂ
ಪ್ರಯತ್ನಿಸು
ಪ್ರಯತ್ನಿ-ಸುತ್ತದೆ
ಪ್ರಯತ್ನಿ-ಸುತ್ತಿದೆ
ಪ್ರಯತ್ನಿಸುತ್ತಿದ್ದರೆ
ಪ್ರಯತ್ನಿಸುತ್ತಿದ್ದೆವು
ಪ್ರಯತ್ನಿ-ಸುತ್ತಿರು
ಪ್ರಯತ್ನಿ-ಸುತ್ತಿರುವ
ಪ್ರಯತ್ನಿ-ಸುತ್ತಿರು-ವನು
ಪ್ರಯತ್ನಿ-ಸುತ್ತಿರು-ವರು
ಪ್ರಯತ್ನಿ-ಸುತ್ತಿರು-ವ-ರೆಂದೂ
ಪ್ರಯತ್ನಿ-ಸುತ್ತಿರು-ವಾಗ
ಪ್ರಯತ್ನಿ-ಸುತ್ತಿರು-ವುದು
ಪ್ರಯತ್ನಿ-ಸುತ್ತಿರು-ವೆವು
ಪ್ರಯತ್ನಿಸುತ್ತಿವೆ
ಪ್ರಯತ್ನಿಸುತ್ತೀರಿ
ಪ್ರಯತ್ನಿಸುತ್ತೇನೆ
ಪ್ರಯತ್ನಿ-ಸುತ್ತೇವೆ
ಪ್ರಯತ್ನಿ-ಸುವ
ಪ್ರಯತ್ನಿ-ಸು-ವಂತೆ
ಪ್ರಯತ್ನಿ-ಸು-ವನು
ಪ್ರಯತ್ನಿ-ಸು-ವರೊ
ಪ್ರಯತ್ನಿ-ಸು-ವು-ದನ್ನು
ಪ್ರಯತ್ನಿ-ಸು-ವುದು
ಪ್ರಯತ್ನಿ-ಸು-ವುದೇ
ಪ್ರಯತ್ನಿ-ಸು-ವೆನು
ಪ್ರಯತ್ನಿ-ಸು-ವೆವು
ಪ್ರಯತ್ನಿ-ಸೋಣ
ಪ್ರಯಾಣ
ಪ್ರಯಾ-ಣಕ್ಕೆ
ಪ್ರಯಾಣ-ದಲ್ಲಿ
ಪ್ರಯಾಣ-ಮಾಡ-ಬೇಕಾ-ಯಿತು
ಪ್ರಯಾಣ-ಮಾಡುತ್ತಿ-ರುವುದು
ಪ್ರಯಾಣ-ಮಾಡು-ವಾಗ
ಪ್ರಯಾಣ-ವನ್ನು
ಪ್ರಯಾಣ-ವನ್ನೆ
ಪ್ರಯೋ
ಪ್ರಯೋಗ
ಪ್ರಯೋ-ಗದ
ಪ್ರಯೋಗ-ವನ್ನೇಕೆ
ಪ್ರಯೋ-ಗ-ಶಾಲೆ-ಯಲ್ಲಿ
ಪ್ರಯೋ-ಗಿ-ಸ-ಬೇಕು
ಪ್ರಯೋ-ಗಿಸಿ
ಪ್ರಯೋ-ಗಿ-ಸು-ವುದರ
ಪ್ರಯೋಜ
ಪ್ರಯೋ-ಜಕಂ
ಪ್ರಯೋ-ಜನ
ಪ್ರಯೋ-ಜ-ನ-ಕರ
ಪ್ರಯೋ-ಜ-ನ-ಕಾರಿ
ಪ್ರಯೋ-ಜ-ನ-ಕಾರಿ-ಯಾಗಿ
ಪ್ರಯೋ-ಜ-ನ-ಕಾರಿ-ಯಾಗಿ-ರು-ವುದು
ಪ್ರಯೋ-ಜ-ನಕ್ಕೆ
ಪ್ರಯೋ-ಜ-ನ-ಗಳಲ್ಲಿ
ಪ್ರಯೋ-ಜ-ನದ
ಪ್ರಯೋ-ಜ-ನ-ದೃಷ್ಟಿಯ
ಪ್ರಯೋ-ಜ-ನ-ದೃಷ್ಟಿ-ಯ-ವರು
ಪ್ರಯೋ-ಜ-ನ-ವನ್ನು
ಪ್ರಯೋ-ಜ-ನ-ವಲ್ಲ
ಪ್ರಯೋ-ಜ-ನ-ವಲ್ಲ-ವೆಂದು
ಪ್ರಯೋ-ಜ-ನ-ವಾಗುತ್ತದೆ
ಪ್ರಯೋ-ಜನ-ವಾಗು-ವು-ದಿಲ್ಲ
ಪ್ರಯೋ-ಜ-ನ-ವಿದೆ
ಪ್ರಯೋ-ಜ-ನ-ವಿಲ್ಲ
ಪ್ರಯೋ-ಜ-ನ-ವಿಲ್ಲದ
ಪ್ರಯೋ-ಜ-ನ-ವಿಲ್ಲ-ದ-ವ-ನಂತೆ
ಪ್ರಯೋ-ಜ-ನ-ವಿಲ್ಲದೆ
ಪ್ರಯೋ-ಜ-ನ-ವುಂಟು
ಪ್ರಯೋ-ಜ-ನವೂ
ಪ್ರಯೋ-ಜ-ನವೆ
ಪ್ರಯೋ-ಜ-ನ-ವೆಂದರೆ
ಪ್ರಯೋ-ಜ-ನ-ವೆಂಬು-ದಿಲ್ಲ
ಪ್ರಯೋ-ಜ-ನವೇ
ಪ್ರಯೋ-ಜ-ನ-ವೇನು
ಪ್ರರ
ಪ್ರಲಾಪ
ಪ್ರಲೋ
ಪ್ರಲೋ-ಭ-ಗ-ಳನ್ನು
ಪ್ರಲೋ-ಭನ-ಕಾರಿ
ಪ್ರಲೋ-ಭ-ನಕ್ಕೆ
ಪ್ರಲೋ-ಭನ-ಗ-ಳನ್ನು
ಪ್ರಲೋ-ಭನ-ಗಳಿಗೆ
ಪ್ರಲೋ-ಭನೆ
ಪ್ರಲೋ-ಭನೆ-ಗ-ಳನ್ನೂ
ಪ್ರಲೋ-ಭ-ವನ್ನು
ಪ್ರಲೋ-ಭಿ-ಸಲು
ಪ್ರಳಯ
ಪ್ರಳಯ-ಗಳು
ಪ್ರಳ-ಯದ
ಪ್ರಳಯ-ವನ್ನು
ಪ್ರಳಯ-ವಾ-ಯಿತು
ಪ್ರಳ-ಯವು
ಪ್ರಳಯಾ-ನಂತರ
ಪ್ರಳಯಾ-ನಂದರ
ಪ್ರವಚನ-ಗಳಿವೆ
ಪ್ರವಹಿಸು-ವಂತೆ
ಪ್ರವಾದಿ-ಗಳು
ಪ್ರವಾದಿಯ
ಪ್ರವಾದಿ-ಯಿಂದ
ಪ್ರವಾಹ
ಪ್ರವಾ-ಹಕ್ಕೆ
ಪ್ರವಾಹ-ಗಳಾಗಿ
ಪ್ರವಾ-ಹದ
ಪ್ರವಾಹ-ದಂತೆ
ಪ್ರವಾಹ-ದಲ್ಲಿ
ಪ್ರವಾಹ-ದಲ್ಲಿ-ರುವ
ಪ್ರವಾಹ-ದಲ್ಲಿ-ರುವುದು
ಪ್ರವಾಹ-ದಿಂದ
ಪ್ರವಾಹನ
ಪ್ರವಾಹ-ವನ್ನು
ಪ್ರವಾಹ-ವಿದೆ
ಪ್ರವಾಹವು
ಪ್ರವೃತ್ತಿ
ಪ್ರವೃತ್ತಿ-ಭೇದೇ
ಪ್ರವೃತ್ತಿ-ಯನ್ನು
ಪ್ರವೃತ್ತಿ-ಯವ
ಪ್ರವೃತ್ತಿ-ಯುಳ್ಳ-ವನು
ಪ್ರವೃತ್ತಿಯೇ
ಪ್ರವೃತ್ತಿ-ರುತ್ಪನ್ನಾ
ಪ್ರವೃತ್ತ್ಯಾ-ಲೋಕನ್ಯಾಸಾತ್
ಪ್ರವೃದ್ಧ-ಮಾ-ನಕ್ಕೆ
ಪ್ರವೇಶ
ಪ್ರವೇ-ಶ-ಮಾಡಿ
ಪ್ರವೇ-ಶ-ವಿಲ್ಲದ
ಪ್ರವೇ-ಶಿಸ
ಪ್ರವೇ-ಶಿಸ-ಬೇಕಾ-ದರೆ
ಪ್ರವೇ-ಶಿಸ-ಬೇಕು
ಪ್ರವೇ-ಶಿಸ-ಬೇಡಿ
ಪ್ರವೇ-ಶಿ-ಸಲಿ
ಪ್ರವೇ-ಶಿಸಿ
ಪ್ರವೇ-ಶಿ-ಸಿದ
ಪ್ರವೇ-ಶಿಸಿ-ದರು
ಪ್ರವೇ-ಶಿಸಿ-ದೊಡ-ನೆಯೇ
ಪ್ರವೇ-ಶಿಸಿದ್ದರೆ
ಪ್ರವೇ-ಶಿಸಿ-ರುವನು
ಪ್ರವೇ-ಶಿಸುತ್ತಾನೆ
ಪ್ರವೇ-ಶಿ-ಸುವ
ಪ್ರವೇ-ಶಿ-ಸುವಂತಾ-ದರೆ
ಪ್ರವೇ-ಶಿ-ಸುವು-ದಕ್ಕೆ
ಪ್ರವೇ-ಶಿ-ಸುವು-ದನ್ನು
ಪ್ರವೇ-ಶಿ-ಸು-ವುದು
ಪ್ರವೇ-ಶಿ-ಸು-ವುದೋ
ಪ್ರವೇ-ಶಿ-ಸು-ವೆವು
ಪ್ರವೇ-ಸಿಸ-ಬೇಕಾ-ದರೆ
ಪ್ರಶಾಂತ-ಚಿತ್ತ
ಪ್ರಶಾನ್ತವಾ-ಹಿತ್ಯಾ
ಪ್ರಶ್ನಿ-ಸದೆ
ಪ್ರಶ್ನಿಸ-ಬಹುದು
ಪ್ರಶ್ನಿಸ-ಬೇಡ
ಪ್ರಶ್ನಿ-ಸಲೇ-ಬೇಕಾ-ಗು-ವುದು
ಪ್ರಶ್ನಿಸಿ
ಪ್ರಶ್ನಿಸಿ-ಕೊಳ್ಳು-ವುದೇ
ಪ್ರಶ್ನಿಸಿ-ದನು
ಪ್ರಶ್ನಿಸಿ-ದರು
ಪ್ರಶ್ನಿಸಿ-ದಾಗ
ಪ್ರಶ್ನಿ-ಸುತ್ತಲೇ
ಪ್ರಶ್ನಿ-ಸುವ
ಪ್ರಶ್ನಿ-ಸುವರು
ಪ್ರಶ್ನಿ-ಸುವು-ದಕ್ಕೆ
ಪ್ರಶ್ನಿ-ಸು-ವುದನ್ನೇ
ಪ್ರಶ್ನಿ-ಸುವು-ದಿಲ್ಲ
ಪ್ರಶ್ನಿ-ಸು-ವುದು
ಪ್ರಶ್ನಿ-ಸು-ವೆವು
ಪ್ರಶ್ನೆ
ಪ್ರಶ್ನೆ-ಗಳ
ಪ್ರಶ್ನೆ-ಗ-ಳನ್ನು
ಪ್ರಶ್ನೆ-ಗಳಲ್ಲಿ
ಪ್ರಶ್ನೆ-ಗಳಿಗೆ
ಪ್ರಶ್ನೆ-ಗಳಿ-ರುತ್ತವೆ
ಪ್ರಶ್ನೆ-ಗಳು
ಪ್ರಶ್ನೆ-ಗಳೆದ್ದುವು
ಪ್ರಶ್ನೆ-ಗಳೆಲ್ಲ
ಪ್ರಶ್ನೆ-ಗಳೇಳು
ಪ್ರಶ್ನೆ-ಗಳೇ-ಳುತ್ತವೆ
ಪ್ರಶ್ನೆ-ಗಳೇ-ಳು-ವುವು
ಪ್ರಶ್ನೆಗೆ
ಪ್ರಶ್ನೆಯ
ಪ್ರಶ್ನೆ-ಯಂತೆಯೇ
ಪ್ರಶ್ನೆ-ಯನ್ನು
ಪ್ರಶ್ನೆ-ಯನ್ನೂ
ಪ್ರಶ್ನೆ-ಯನ್ನೇ
ಪ್ರಶ್ನೆ-ಯಲ್ಲಿ
ಪ್ರಶ್ನೆ-ಯಷ್ಟು
ಪ್ರಶ್ನೆ-ಯಾ-ಗಿತ್ತು
ಪ್ರಶ್ನೆ-ಯಿಂದ
ಪ್ರಶ್ನೆಯು
ಪ್ರಶ್ನೆಯೂ
ಪ್ರಶ್ನೆಯೆ
ಪ್ರಶ್ನೆ-ಯೆಂದರೆ
ಪ್ರಶ್ನೆಯೇ
ಪ್ರಶ್ನೆ-ಯೇನೋ
ಪ್ರಶ್ನೆ-ಯೊಂದು
ಪ್ರಸಂಖ್ಯಾನೇಪ್ಯಕುಸೀದಸ್ಯ
ಪ್ರಸಂಗ
ಪ್ರಸಂಗ-ಗ-ಳನ್ನು
ಪ್ರಸಂಗ-ಗಳಲ್ಲಿಯೂ
ಪ್ರಸಂಗ-ಗಳು
ಪ್ರಸಂಗ-ದಿಂದ
ಪ್ರಸಂಗ-ವನ್ನು
ಪ್ರಸಂಗವೇ
ಪ್ರಸನ್ನ
ಪ್ರಸರಿ-ಸಿ-ದರೆ
ಪ್ರಸಿದ್ಧ-ರಾದ
ಪ್ರಸಿದ್ಧರು
ಪ್ರಸುಪ್ತ-ತನು-ವಿಚ್ಛಿನ್ನೋ-ದಾರಾ-ಣಾಮ್
ಪ್ರಸ್ತಾಪಿಸ-ಬಹುದು
ಪ್ರಸ್ತಾಪಿಸುತ್ತಿ-ರುವ
ಪ್ರಸ್ತಾ-ವನೆ
ಪ್ರಸ್ತುತ
ಪ್ರಾಂಗ-ಣಕ್ಕೆ
ಪ್ರಾಕೃತಿಕ
ಪ್ರಾಖ್ಯಾತ
ಪ್ರಾಚೀನ
ಪ್ರಾಚೀನ-ಕಾಲ-ದಲ್ಲಿ
ಪ್ರಾಚೀನ-ಕಾಲ-ದಿಂದಲೂ
ಪ್ರಾಚೀನ-ತಮ
ಪ್ರಾಚೀನ-ವಾದ
ಪ್ರಾಚೀನ-ವೆಂಬು-ದನ್ನು
ಪ್ರಾಚ್ಯ
ಪ್ರಾಣ
ಪ್ರಾಣಕ್ಕಿಂತ
ಪ್ರಾಣಕ್ಕಿಂತಲೂ
ಪ್ರಾಣಕ್ಕೆ
ಪ್ರಾಣ-ಗಳ
ಪ್ರಾಣ-ಗಳಿಂದ
ಪ್ರಾಣ-ಗಳಿವೆ
ಪ್ರಾಣ-ಗಳು
ಪ್ರಾಣ-ತತ್ತ್ವದ
ಪ್ರಾಣ-ತತ್ತ್ವ-ವನ್ನು
ಪ್ರಾಣದ
ಪ್ರಾಣ-ದಂತೆ
ಪ್ರಾಣ-ದಲ್ಲಿ
ಪ್ರಾಣ-ದಿಂದ
ಪ್ರಾಣ-ದಿಂದಲೇ
ಪ್ರಾಣ-ನಿಗ್ರಹ
ಪ್ರಾಣ-ಪಕ್ಷಿ
ಪ್ರಾಣಪ್ರತಿಷ್ಠೆ-ಯನ್ನು
ಪ್ರಾಣ-ಬಿ-ಡುವನು
ಪ್ರಾಣ-ಯಾಮದ
ಪ್ರಾಣ-ಯಾಮ-ದಲ್ಲಿ
ಪ್ರಾಣ-ವನ್ನು
ಪ್ರಾಣ-ವನ್ನೆಲ್ಲ
ಪ್ರಾಣ-ವನ್ನೇ
ಪ್ರಾಣ-ವಲ್ಲದೆ
ಪ್ರಾಣ-ವಾ-ಗಿ-ರುವುದು
ಪ್ರಾಣ-ವಾ-ಗು-ವುದು
ಪ್ರಾಣ-ವಿತ್ತು
ಪ್ರಾಣ-ವಿದೆ
ಪ್ರಾಣ-ವಿ-ರುವ
ಪ್ರಾಣವು
ಪ್ರಾಣವೂ
ಪ್ರಾಣವೆ
ಪ್ರಾಣ-ವೆಂದರೆ
ಪ್ರಾಣ-ವೆಂದರೇ-ನೆಂಬು-ದನ್ನು
ಪ್ರಾಣ-ವೆಂದು
ಪ್ರಾಣ-ವೆಂದೂ
ಪ್ರಾಣ-ವೆಂಬ
ಪ್ರಾಣ-ವೆಲ್ಲ
ಪ್ರಾಣ-ವೆಲ್ಲಾ
ಪ್ರಾಣವೇ
ಪ್ರಾಣ-ವೇ-ನೆಂಬು-ದನ್ನು
ಪ್ರಾಣ-ಶಕ್ತಿ
ಪ್ರಾಣ-ಶಕ್ತಿಯ
ಪ್ರಾಣ-ಸಾ-ಗರದ
ಪ್ರಾಣಸ್ಯ
ಪ್ರಾಣಾ
ಪ್ರಾಣಾ-ಪಾಯ-ವನ್ನೂ
ಪ್ರಾಣಾ-ಯಾಮ
ಪ್ರಾಣಾ-ಯಾಮಃ
ಪ್ರಾಣಾ-ಯಾಮಕ್ಕೂ
ಪ್ರಾಣಾ-ಯಾಮಕ್ಕೆ
ಪ್ರಾಣಾ-ಯಾಮದ
ಪ್ರಾಣಾ-ಯಾಮ-ದಲ್ಲಿ
ಪ್ರಾಣಾ-ಯಾಮ-ದಿಂದ
ಪ್ರಾಣಾ-ಯಾಮ-ದೊಂದಿಗೆ
ಪ್ರಾಣಾ-ಯಾಮ-ವನ್ನು
ಪ್ರಾಣಾ-ಯಾಮ-ವಾ-ಗು-ವುದು
ಪ್ರಾಣಾ-ಯಾ-ಮವು
ಪ್ರಾಣಾ-ಯಾಮ-ವೆಂದರೆ
ಪ್ರಾಣಾ-ಯಾಮ-ವೆಂದು
ಪ್ರಾಣಾ-ಯಾಮ-ವೆಂಬ
ಪ್ರಾಣಾ-ಯಾಮ-ವೆನ್ನು-ವುದು
ಪ್ರಾಣಿ
ಪ್ರಾಣಿ-ಗಳ
ಪ್ರಾಣಿ-ಗಳಂತೆ
ಪ್ರಾಣಿ-ಗ-ಳನ್ನು
ಪ್ರಾಣಿ-ಗಳಲ್ಲಿ
ಪ್ರಾಣಿ-ಗಳಲ್ಲೂ
ಪ್ರಾಣಿ-ಗಳಲ್ಲೆಲ್ಲಾ
ಪ್ರಾಣಿ-ಗಳಾಗಿ
ಪ್ರಾಣಿ-ಗಳಿಂದ
ಪ್ರಾಣಿ-ಗಳಿ-ಗಿಂತ
ಪ್ರಾಣಿ-ಗಳಿ-ಗಿಂತಲೂ
ಪ್ರಾಣಿ-ಗಳಿಗೂ
ಪ್ರಾಣಿ-ಗಳಿಗೆ
ಪ್ರಾಣಿ-ಗಳು
ಪ್ರಾಣಿ-ಗಳೂ
ಪ್ರಾಣಿ-ಗಳೆ
ಪ್ರಾಣಿ-ಗಳೆಂಬ
ಪ್ರಾಣಿ-ಗಳೊಂದಿಗೆ
ಪ್ರಾಣಿ-ಗಿಂತ
ಪ್ರಾಣಿಗೂ
ಪ್ರಾಣಿಗೆ
ಪ್ರಾಣಿಯ
ಪ್ರಾಣಿ-ಯನ್ನು
ಪ್ರಾಣಿ-ಯಲ್ಲಿ
ಪ್ರಾಣಿ-ಯಲ್ಲಿಯೂ
ಪ್ರಾಣಿ-ಯ-ವರೆಗೆ
ಪ್ರಾಣಿ-ಯಾಗುತ್ತಾನೆ
ಪ್ರಾಣಿ-ಯಾ-ಗು-ವುದು
ಪ್ರಾಣಿ-ಯಿಂದ
ಪ್ರಾಣಿಯು
ಪ್ರಾಣಿಯೂ
ಪ್ರಾಣಿಯೆ
ಪ್ರಾಣಿ-ಯೆಂದೊ
ಪ್ರಾಣಿ-ವರ್ಗಕ್ಕೆ
ಪ್ರಾಣಿ-ವರ್ಗದ
ಪ್ರಾಣಿ-ವರ್ಗ-ದಲ್ಲಿ
ಪ್ರಾಣಿ-ವರ್ಗ-ದಲ್ಲೆಲ್ಲಾ
ಪ್ರಾಣಿ-ಹತ್ಯೆಗೆ
ಪ್ರಾತಃಕಾಲ
ಪ್ರಾತಿಭಾದ್ವಾ
ಪ್ರಾತಿಭೆ
ಪ್ರಾತಿ-ಭೆಯ
ಪ್ರಾತಿಭೆ-ಯಿಂದ
ಪ್ರಾಧಾನ್ಯ
ಪ್ರಾಧಾನ್ಯ-ವನ್ನು
ಪ್ರಾಧ್ಯಾಪಕ
ಪ್ರಾಧ್ಯಾಪ-ಕರು
ಪ್ರಾನ್ತ-ಭೂಮಿಃ
ಪ್ರಾಪಂಚಿಕ
ಪ್ರಾಪಂಚಿಕ-ತೆಯ
ಪ್ರಾಪಂಚಿಕ-ತೆ-ಯನ್ನು
ಪ್ರಾಪ್ತ-ವಾಗ-ಬಹುದು
ಪ್ರಾಪ್ತ-ವಾಗು
ಪ್ರಾಪ್ತ-ವಾಗುತ್ತದೆ
ಪ್ರಾಪ್ತ-ವಾಗುತ್ತವೆ
ಪ್ರಾಪ್ತ-ವಾಗುವ
ಪ್ರಾಪ್ತ-ವಾಗು-ವುದು
ಪ್ರಾಪ್ತ-ವಾಗು-ವುವು
ಪ್ರಾಪ್ತ-ವಾದ
ಪ್ರಾಪ್ತ-ವಾ-ದರೂ
ಪ್ರಾಪ್ತ-ವಾ-ಯಿತು
ಪ್ರಾಪ್ತಿ
ಪ್ರಾಪ್ತಿ-ಯಾ-ಗಿಲ್ಲವೋ
ಪ್ರಾಬಲ್ಯ-ದಿಂದ
ಪ್ರಾಮಾಣಿ-ಕತೆ-ಯನ್ನೂ
ಪ್ರಾಮಾಣಿಕ-ರಲ್ಲ
ಪ್ರಾಮಾಣಿ-ಕರಾ-ಗಿ-ರು-ವರೋ
ಪ್ರಾಮಾಣಿ-ಕರು
ಪ್ರಾಮಾಣ್ಯಕ್ಕೆ
ಪ್ರಾಮಾಣ್ಯ-ವನ್ನು
ಪ್ರಾಮುಖ್ಯ
ಪ್ರಾಮುಖ್ಯ-ವನ್ನು
ಪ್ರಾಮುಖ್ಯ-ವಾಗಿ
ಪ್ರಾಮುಖ್ಯ-ವಾಗಿದ್ದರೆ
ಪ್ರಾಮುಖ್ಯವೂ
ಪ್ರಾಯ
ಪ್ರಾಯ-ವನ್ನು
ಪ್ರಾಯ-ವಾಗಿದೆ
ಪ್ರಾಯವೇ
ಪ್ರಾಯೋ-ಗಿಕ
ಪ್ರಾರಂಬ-ದಲ್ಲಿ
ಪ್ರಾರಂಭ
ಪ್ರಾರಂಭದ
ಪ್ರಾರಂಭ-ದಲ್ಲಿ
ಪ್ರಾರಂಭ-ದಿಂದಲೂ
ಪ್ರಾರಂಭ-ದೊ-ಡನೆ
ಪ್ರಾರಂಭ-ವಾಗಿ
ಪ್ರಾರಂಭ-ವಾಗಿತ್ತಲ್ಲದೆ
ಪ್ರಾರಂಭ-ವಾಗಿದೆ
ಪ್ರಾರಂಭ-ವಾಗಿಲ್ಲವೆಂದೇ
ಪ್ರಾರಂಭ-ವಾಗುತ್ತದೆ
ಪ್ರಾರಂಭ-ವಾ-ಗು-ವುದು
ಪ್ರಾರಂಭ-ವಾಗು-ವು-ದೆಂದು
ಪ್ರಾರಂಭ-ವಾಗು-ವುದೊ
ಪ್ರಾರಂಭ-ವಾ-ಗು-ವುವು
ಪ್ರಾರಂಭ-ವಾ-ದಂತೆಯೇ
ಪ್ರಾರಂಭ-ವಾ-ದವು
ಪ್ರಾರಂಭ-ವಾ-ಯಿತು
ಪ್ರಾರಂಭಿಸ-ಬಾ-ರದು
ಪ್ರಾರಂಭಿಸ-ಬೇಕಾ-ಗಿದೆ
ಪ್ರಾರಂಭಿ-ಸ-ಬೇಕು
ಪ್ರಾರಂಭಿಸಿ
ಪ್ರಾರಂಭಿಸಿ-ದಂದಿ-ನಿಂದಲೂ
ಪ್ರಾರಂಭಿಸಿ-ದನು
ಪ್ರಾರಂಭಿಸಿ-ದವು
ಪ್ರಾರಂಭಿಸಿ-ದಾಗ
ಪ್ರಾರಂಭಿಸಿ-ದಾಗಲೇ
ಪ್ರಾರಂಭಿಸುತ್ತಾನೆ
ಪ್ರಾರಂಭಿಸು-ವರು
ಪ್ರಾರಂಭಿಸು-ವಾಗ
ಪ್ರಾರಂಭಿ-ಸು-ವುದು
ಪ್ರಾರ್ಥ
ಪ್ರಾರ್ಥನೆ
ಪ್ರಾರ್ಥ-ನೆ-ಗಳಿಂದ
ಪ್ರಾರ್ಥ-ನೆ-ಗಳೆಲ್ಲ
ಪ್ರಾರ್ಥ-ನೆ-ಗಾಗಿ
ಪ್ರಾರ್ಥ-ನೆಗೆ
ಪ್ರಾರ್ಥ-ನೆಯ
ಪ್ರಾರ್ಥ-ನೆ-ಯನ್ನು
ಪ್ರಾರ್ಥ-ನೆ-ಯಾಗ-ಬೇಕೆನ್ನು-ವುದು
ಪ್ರಾರ್ಥ-ನೆಯೆ
ಪ್ರಾರ್ಥ-ಯಂತೇ
ಪ್ರಾರ್ಥಿಸ-ಬಹುದು
ಪ್ರಾರ್ಥಿ-ಸ-ಬೇಕು
ಪ್ರಾರ್ಥಿ-ಸ-ಬೇಕು-ಹಣಕ್ಕಾಗಿ-ಯಾ-ಗಲಿ
ಪ್ರಾರ್ಥಿಸಿ
ಪ್ರಾರ್ಥಿ-ಸುತ್ತಿದ್ದ
ಪ್ರಾರ್ಥಿ-ಸುತ್ತಿದ್ದರೆ
ಪ್ರಾರ್ಥಿ-ಸುತ್ತಿದ್ದಿರೊ
ಪ್ರಾರ್ಥಿಸುತ್ತೇನೆ
ಪ್ರಾರ್ಥಿ-ಸುವ
ಪ್ರಾರ್ಥಿ-ಸು-ವು-ದಲ್ಲ
ಪ್ರಾಸ-ಗಳಿಂದ
ಪ್ರಿಯ
ಪ್ರಿಯ-ಕರ-ವಾದ
ಪ್ರಿಯ-ತಮ
ಪ್ರಿಯ-ತಮ-ನೆಂಬುದು
ಪ್ರಿಯ-ತಮ-ಳನ್ನು
ಪ್ರಿಯ-ತಮ-ಳೆಂದು
ಪ್ರಿಯ-ತಮ-ವಾ-ಗಿಲ್ಲ
ಪ್ರಿಯ-ತಮ-ವೆಂದು
ಪ್ರಿಯನು
ಪ್ರಿಯ-ವಾಗ-ಬಹುದು
ಪ್ರಿಯ-ವಾದ
ಪ್ರಿಯ-ವಾದ-ವರೊಬ್ಬರು
ಪ್ರಿಯ-ವಾದು-ದಲ್ಲ
ಪ್ರೀತಿ
ಪ್ರೀತಿ-ಕರ
ಪ್ರೀತಿ-ಗಳೂ
ಪ್ರೀತಿಗೆ
ಪ್ರೀತಿ-ಪಾತ್ರ-ರಿ-ರು-ವರು
ಪ್ರೀತಿ-ಪಾತ್ರರು
ಪ್ರೀತಿ-ಪಾತ್ರರೋ
ಪ್ರೀತಿಯ
ಪ್ರೀತಿ-ಯನ್ನು
ಪ್ರೀತಿ-ಯಾಗಿದ್ದರೆ
ಪ್ರೀತಿ-ಯಾ-ದರೂ
ಪ್ರೀತಿ-ಯಿಂದ
ಪ್ರೀತಿಯೂ
ಪ್ರೀತಿಯೆ
ಪ್ರೀತಿ-ಯೆಲ್ಲಾ
ಪ್ರೀತಿಯೇ
ಪ್ರೀತಿ-ಯೊಂದೇ
ಪ್ರೀತಿಯೋ
ಪ್ರೀತಿ-ವಾತ್ಸಲ್ಯ-ದಿಂದ
ಪ್ರೀತಿಸ
ಪ್ರೀತಿ-ಸದೆ
ಪ್ರೀತಿ-ಸ-ಬಲ್ಲ
ಪ್ರೀತಿ-ಸ-ಬಲ್ಲನು
ಪ್ರೀತಿ-ಸ-ಬಲ್ಲಳು
ಪ್ರೀತಿ-ಸ-ಬೇಕು
ಪ್ರೀತಿ-ಸ-ಬೇಕೆಂಬ
ಪ್ರೀತಿ-ಸ-ಬೇಕೆಂಬು-ದನ್ನು
ಪ್ರೀತಿ-ಸ-ಲಾ-ಗದೆ
ಪ್ರೀತಿ-ಸಲು
ಪ್ರೀತಿಸಿ
ಪ್ರೀತಿ-ಸಿದ
ಪ್ರೀತಿ-ಸಿ-ದರು
ಪ್ರೀತಿ-ಸಿ-ದರೆ
ಪ್ರೀತಿಸು
ಪ್ರೀತಿ-ಸುತ್ತದೆ
ಪ್ರೀತಿ-ಸುತ್ತಾ-ನೆಯೊ
ಪ್ರೀತಿ-ಸುತ್ತಿ-ರುವ
ಪ್ರೀತಿ-ಸುತ್ತೀರಿ
ಪ್ರೀತಿ-ಸುತ್ತೇನೆ
ಪ್ರೀತಿ-ಸುವ
ಪ್ರೀತಿ-ಸು-ವನು
ಪ್ರೀತಿ-ಸು-ವ-ನು-ಅ-ವನು
ಪ್ರೀತಿ-ಸು-ವರು
ಪ್ರೀತಿ-ಸು-ವಳು
ಪ್ರೀತಿ-ಸು-ವ-ವ-ನಿಗೆ
ಪ್ರೀತಿ-ಸು-ವಾಗ
ಪ್ರೀತಿ-ಸು-ವಿರೋ
ಪ್ರೀತಿ-ಸು-ವು-ದಕ್ಕೆ
ಪ್ರೀತಿ-ಸು-ವು-ದೆಲ್ಲ
ಪ್ರೆಸ್ಬಿಟೇರಿ-ಯನ್ನಾಗಿ
ಪ್ರೇಕ್ಷ-ಕನೂ
ಪ್ರೇಕ್ಷ-ಕರ
ಪ್ರೇಕ್ಷ-ಕರು
ಪ್ರೇತ
ಪ್ರೇತ-ಗಳಿ-ಗಿಂತ
ಪ್ರೇಮ
ಪ್ರೇಮಕ್ಕೆ
ಪ್ರೇಮ-ಚಿಲುಮೆ
ಪ್ರೇಮದ
ಪ್ರೇಮ-ದಲ್ಲಿ
ಪ್ರೇಮ-ದಲ್ಲಿದೆ
ಪ್ರೇಮ-ದಿಂದ
ಪ್ರೇಮ-ನಿಧಿ-ಯಾದ
ಪ್ರೇಮ-ವನ್ನು
ಪ್ರೇಮ-ವಲ್ಲ
ಪ್ರೇಮ-ವಿದ್ದರೂ
ಪ್ರೇಮವು
ಪ್ರೇಮವೆ
ಪ್ರೇಮ-ವೆಂದ-ರೇ-ನೆಂಬುದು
ಪ್ರೇಮ-ವೆಂಬ
ಪ್ರೇಮವೇ
ಪ್ರೇಮ-ವೊಂದೇ
ಪ್ರೇಮ-ಶಕ್ತಿ
ಪ್ರೇಮಸ್ವ-ರೂಪನು
ಪ್ರೇಮಾ-ದರ್ಶ
ಪ್ರೇಮಾಮೃತ-ವನ್ನು
ಪ್ರೇಮೇಶ್ವರ-ನಂತೆ
ಪ್ರೇಮೇಶ್ವರ-ನಾದ
ಪ್ರೇಯಸಿ
ಪ್ರೇಯಸಿ-ಯಲ್ಲಿ
ಪ್ರೇಯಸ್ಸನ್ನು
ಪ್ರೇಯಸ್ಸಿ-ಗಿಂತ
ಪ್ರೇಯಸ್ಸು
ಪ್ರೇಯಸ್ಸು-ಗಳೆ-ರಡೂ
ಪ್ರೇರಣೆ
ಪ್ರೇರಣೆ-ಗಳಿವೆ
ಪ್ರೇರಣೆ-ಯಿಲ್ಲದೆ
ಪ್ರೇರಣೆಯೂ
ಪ್ರೇರಿತ
ಪ್ರೇರಿತ-ನಾಗಿ
ಪ್ರೇರಿತ-ರಾಗಿ
ಪ್ರೇರಿತ-ರಾಗಿಲ್ಲವೊ
ಪ್ರೇರಿತ-ರಾಗುತ್ತೇವೆ
ಪ್ರೇರಿತ-ರಾದ-ವ-ನಿಗೆ
ಪ್ರೇರಿತ-ರಾದ-ವ-ರಿಗೆ
ಪ್ರೇರಿತ-ವಾದ
ಪ್ರೇರೇಪಣಾ
ಪ್ರೇರೇ-ಪಿತ-ರಾಗಿಯೂ
ಪ್ರೇರೇ-ಪಿತ-ರಾ-ಗಿ-ರು-ವರೊ
ಪ್ರೇರೇ-ಪಿತ-ವಾಗಿವೆ
ಪ್ರೇರೇ-ಪಿತ-ವಾದು-ದಲ್ಲ
ಪ್ರೇರೇಪಿಸ-ಲಿಲ್ಲ-ವೇನು
ಪ್ರೇರೇಪಿಸಿ-ದರೆ
ಪ್ರೇರೇಪಿಸಿ-ರ-ಬಹುದು
ಪ್ರೇರೇಪಿಸುತ್ತವೆ
ಪ್ರೇರೇಪಿಸುತ್ತಿ-ರುವ
ಪ್ರೇರೇಪಿಸುತ್ತಿ-ರುವುದು
ಪ್ರೇರೇಪಿ-ಸು-ವುದು
ಪ್ರೇರೇಪಿಸು-ವುದೋ
ಪ್ರೋತ್ಸಾಹ
ಪ್ರೋತ್ಸಾಹಿಸಿ-ದರೆ
ಪ್ರೌಢ-ಪಂಡಿ-ತರ
ಪ್ರ್ಮೇದಿಂದ
ಪ್ಲೇಗಿನ
ಪ್ಲೇಟೊ
ಫಲ
ಫಲ-ಎಂಬುದು
ಫಲ-ಕವನ್ನಾಗಿ
ಫಲ-ಕಾರಿ
ಫಲ-ಕಾರಿ-ಯಾಗದು
ಫಲ-ಕಾರಿ-ಯಾ-ಗಿಲ್ಲ
ಫಲ-ಗ-ಳನ್ನು
ಫಲ-ಗಳಿವೆ
ಫಲ-ಗಳೇ
ಫಲದ
ಫಲ-ದಿಂದ
ಫಲಪ್ರಾಪ್ತಿ-ಯಾ-ಗು-ವುದು
ಫಲ-ಮಾತ್ರ-ವಲ್ಲ
ಫಲ-ವನ್ನು
ಫಲ-ವಲ್ಲ
ಫಲ-ವಾಗಿ
ಫಲ-ವಾ-ಗು-ವುದು
ಫಲ-ವಾದ
ಫಲವು
ಫಲವೂ
ಫಲವೇ
ಫಲ-ವೇನೋ
ಫಲಾಪೇಕ್ಷೆ
ಫಲಾಪೇಕ್ಷೆ-ಯಿಂದ
ಫಲಾಪೇಕ್ಷೆ-ಯಿಲ್ಲದೆ
ಫಲಿತಾಂಶ-ಗಳು
ಫಲಿ-ಸಲು
ಫಲಿಸುತ್ತಿ-ರುವ
ಫಲಿ-ಸು-ವು-ದಕ್ಕೆ
ಫಲಿ-ಸು-ವುದು
ಫಿರಂಗಿ-ಯನ್ನು
ಫಿಲಿಪಿನೋ
ಫುಟ್ಪೌಂಡು
ಫೇರಿಸ್
ಫೋಟೋ
ಫೋಟೋ-ಗ-ಳನ್ನು
ಫೋಟೋ-ಗಳು
ಫ್ರೆಡರಿಕ್
ಫ್ರೊಫೆ-ಸರ್
ಫ್ಲೈವೀಲ್
ಬಂಡೆಯ
ಬಂಡೆ-ಯಂತೆ
ಬಂತು
ಬಂತೆಂದು
ಬಂತೆಂಬ
ಬಂತೆಂಬು-ದನ್ನು
ಬಂದ
ಬಂದಂತೆ
ಬಂದಂತೆಲ್ಲ
ಬಂದದ್ದು
ಬಂದದ್ದೇ
ಬಂದನು
ಬಂದ-ಮೇಲೆ
ಬಂದರು
ಬಂದರೂ
ಬಂದರೆ
ಬಂದಲ್ಲಿಗೆ
ಬಂದ-ವ-ನಲ್ಲ
ಬಂದ-ವರು
ಬಂದ-ವಲ್ಲವೆ
ಬಂದವು
ಬಂದ-ವು-ಗಳಲ್ಲ
ಬಂದವೋ
ಬಂದಷ್ಟು
ಬಂದಾಗ
ಬಂದಾಗಲೆ
ಬಂದಾಗಿ-ನಿಂದಲೂ
ಬಂದಿ
ಬಂದಿತು
ಬಂದಿ-ತುಈ
ಬಂದಿ-ತೆಂದು
ಬಂದಿ-ತೆಂದೂ
ಬಂದಿ-ತೆಂಬುದು
ಬಂದಿತೊ
ಬಂದಿದೆ
ಬಂದಿ-ದೆಯೇ
ಬಂದಿ-ದೆಯೊ
ಬಂದಿ-ದೆಯೋ
ಬಂದಿದ್ದರೆ
ಬಂದಿದ್ದಾರೆ
ಬಂದಿ-ರ-ಬಹುದು
ಬಂದಿ-ರ-ಬೇಕು
ಬಂದಿ-ರ-ಲಾರದು
ಬಂದಿ-ರ-ಲಿಲ್ಲ
ಬಂದಿರಿ
ಬಂದಿ-ರುವ
ಬಂದಿ-ರು-ವಂತೆ
ಬಂದಿ-ರುವನು
ಬಂದಿ-ರುವನೊ
ಬಂದಿ-ರು-ವರು
ಬಂದಿ-ರು-ವರೊ
ಬಂದಿ-ರು-ವು-ದ-ರಿಂದ
ಬಂದಿ-ರು-ವು-ದಲ್ಲ
ಬಂದಿ-ರುವುದು
ಬಂದಿ-ರುವು-ದು-ಎಂದರೆ
ಬಂದಿ-ರುವು-ವಾದ್ದ-ರಿಂದ
ಬಂದಿ-ರು-ವೆನು
ಬಂದಿ-ರು-ವೆವು
ಬಂದಿಲ್ಲ
ಬಂದಿವೆ
ಬಂದು
ಬಂದು-ದರ
ಬಂದು-ದಲ್ಲ
ಬಂದು-ದಲ್ಲ-ವೆಂದು
ಬಂದುದು
ಬಂದುದೇ
ಬಂದುವು
ಬಂದು-ವು-ಗಳು
ಬಂದುವೋ
ಬಂದು-ಹೋಗಿದೆ
ಬಂದು-ಹೋಗುತ್ತಿದೆ
ಬಂದೆ
ಬಂದೆ-ಡೆ-ಯಲ್ಲಿ
ಬಂದೆವು
ಬಂದೆ-ವೆಂಬುದು
ಬಂದೇ
ಬಂದೊ
ಬಂದೊ-ಡ-ನೆಯೆ
ಬಂದೊ-ಡನೆಯೇ
ಬಂದೊ-ದ-ಗಿದೆ
ಬಂದೋದಗು-ವು-ದನ್ನು
ಬಂಧ-ಕಾರ-ಣವು
ಬಂಧ-ಕಾರ-ಣ-ಶೈಥಿಲ್ಯಾತ್
ಬಂಧನ
ಬಂಧ-ನಕ್ಕೆ
ಬಂಧನಕ್ಕೊಳ-ಗಾಗಿ
ಬಂಧನಕ್ಕೊಳಗಾದ
ಬಂಧನ-ಗ-ಳನ್ನು
ಬಂಧನ-ಗಳನ್ನೆಲ್ಲಾ
ಬಂಧನ-ಗಳಲ್ಲದೆ
ಬಂಧನ-ಗಳಲ್ಲಿ
ಬಂಧನ-ಗಳಿಂದ
ಬಂಧನ-ಗಳು
ಬಂಧನ-ಗಳೆಂಬ
ಬಂಧನ-ದಲ್ಲಿ
ಬಂಧನ-ದಲ್ಲಿ-ರ-ಬೇಕು
ಬಂಧನ-ದಲ್ಲಿ-ರುವ
ಬಂಧನ-ದಲ್ಲಿ-ರುವುದು
ಬಂಧನ-ದಲ್ಲಿ-ರು-ವೆವು
ಬಂಧನ-ದಿಂದ
ಬಂಧನ-ದೆ-ಡೆಗೆ
ಬಂಧನ-ವನ್ನು
ಬಂಧನವು
ಬಂಧನ-ವೆಂಬ
ಬಂಧ-ಮುಕ್ತ-ರನ್ನಾಗಿ
ಬಂಧಿ-ತನಾ-ಗದೆ
ಬಂಧಿತ-ನಾ-ಗಿ-ರುವನು
ಬಂಧಿತ-ರಾಗು-ವೆವು
ಬಂಧಿತ-ವಾದ
ಬಂಧಿಸ-ಬಾ-ರದು
ಬಂಧಿಸ-ಲಾರದು
ಬಂಧಿಸ-ಲಾರವು
ಬಂಧಿ-ಸಲು
ಬಂಧಿ-ಸಲೆತ್ನಿ-ಸುವ
ಬಂಧಿಸಲ್ಪಟ್ಟಿತ್ತು
ಬಂಧಿಸಿ
ಬಂಧಿಸಿ-ಡಲು
ಬಂಧಿಸಿ-ದು-ದೆಲ್ಲ
ಬಂಧಿಸಿ-ರ-ಲಿಲ್ಲ
ಬಂಧಿಸಿ-ರುವ
ಬಂಧಿ-ಸುವ
ಬಂಧಿ-ಸು-ವಂತ-ಹದು
ಬಂಧಿ-ಸುವ-ವ-ರಾರು
ಬಂಧಿ-ಸು-ವು-ದಿಲ್ಲ
ಬಂಧಿ-ಸು-ವುದು
ಬಂಧು
ಬಂಧು-ಗಳ
ಬಕೆಟ್ಟು
ಬಗುಳಿ
ಬಗುಳಿ-ದರೆ
ಬಗುಳು-ವುದು
ಬಗೆ
ಬಗೆ-ಬ-ಗೆಯ
ಬಗೆಯ
ಬಗೆ-ಯನ್ನು
ಬಗೆ-ಯಲ್ಲಿ
ಬಗೆವ
ಬಗೆ-ಹರಿ
ಬಗೆ-ಹರಿ-ದಂತೆ
ಬಗೆ-ಹರಿ-ದಂತೆಯೆ
ಬಗೆ-ಹರಿ-ದಿ-ರು-ವುದು
ಬಗೆ-ಹರಿ-ಯದೆ
ಬಗೆ-ಹರಿ-ಯು-ವುದು
ಬಗೆ-ಹರಿ-ಯು-ವುದೆ
ಬಗೆ-ಹರಿ-ಸದು
ಬಗೆ-ಹರಿ-ಸದೆ
ಬಗೆ-ಹರಿ-ಸ-ಬಲ್ಲ
ಬಗೆ-ಹರಿ-ಸ-ಬೇಕಾ-ಗಿದೆ
ಬಗೆ-ಹರಿ-ಸ-ಬೇಕಾದ
ಬಗೆ-ಹರಿ-ಸ-ಲಾ-ಗದ
ಬಗೆ-ಹರಿ-ಸ-ಲಾರದು
ಬಗೆ-ಹರಿ-ಸ-ಲಿಲ್ಲ
ಬಗೆ-ಹರಿ-ಸಲು
ಬಗೆ-ಹ-ರಿಸಿ-ರು-ವರು
ಬಗೆ-ಹ-ರಿಸುವ
ಬಗೆ-ಹ-ರಿಸು-ವು-ದಕ್ಕೆ
ಬಗೆ-ಹ-ರಿಸು-ವುದು
ಬಗೆ-ಹ-ರಿಸು-ವುದೇ
ಬಗ್ಗಡ-ವಾಗಿದ್ದರೆ
ಬಗ್ಗಿ-ರು-ವಾಗ
ಬಗ್ಗೆ
ಬಗ್ಗೆಯೂ
ಬಚ್ಚ-ಲಿನ
ಬಚ್ಚಿ-ಡಲು
ಬಟ್ಟ-ಲಲ್ಲಿ
ಬಟ್ಟ-ಲಿನ
ಬಟ್ಟಲು
ಬಟ್ಟೆ
ಬಟ್ಟೆ-ಕಟ್ಟಿ
ಬಟ್ಟೆ-ಗ-ಳನ್ನು
ಬಟ್ಟೆ-ಬರೆ-ಯನ್ನು
ಬಡ
ಬಡಕಲ
ಬಡ-ತನ
ಬಡವ
ಬಡ-ವರು
ಬಡ-ವಾಗುತ್ತವೆ
ಬಡಿಯು
ಬಣಬೆ-ಗಳಂತೆ
ಬಣ್ಣ
ಬಣ್ಣದ
ಬಣ್ಣ-ವನ್ನು
ಬಣ್ಣ-ವನ್ನೇ
ಬಣ್ಣವೂ
ಬತ್ತಿ-ಹೋಗು-ವುದು
ಬತ್ತು
ಬದ
ಬದಲಾ
ಬದಲಾ-ಗತ್ತಿ-ರುವುದು
ಬದ-ಲಾಗದ
ಬದ-ಲಾಗದೆ
ಬದ-ಲಾಗ-ಬಹುದು
ಬದ-ಲಾಗಿ
ಬದ-ಲಾಗಿದೆ
ಬದ-ಲಾಗಿರ-ಬಹುದು
ಬದ-ಲಾಗು
ಬದಲಾ-ಗುತ್ತದೆ
ಬದಲಾ-ಗುತ್ತಿದೆ
ಬದಲಾ-ಗುತ್ತಿ-ರುವ
ಬದಲಾ-ಗುತ್ತಿ-ರು-ವರು
ಬದಲಾ-ಗುತ್ತಿರು-ವು-ದನ್ನು
ಬದಲಾ-ಗುತ್ತಿ-ರುವುದು
ಬದಲಾ-ಗುವ
ಬದಲಾ-ಗು-ವಂತೆ
ಬದಲಾ-ಗು-ವುದರ
ಬದಲಾ-ಗು-ವು-ದ-ರಿಂದ
ಬದ-ಲಾಗು-ವು-ದಿಲ್ಲ
ಬದಲಾ-ಗು-ವುದು
ಬದಲಾ-ಗು-ವುವು-ಆ-ದರೆ
ಬದಲಾ-ದಂತಾ-ಗುತ್ತದೆ
ಬದಲಾ-ಯಿಸ
ಬದಲಾ-ಯಿಸ-ಕೂಡದು
ಬದಲಾ-ಯಿಸದ
ಬದಲಾ-ಯಿಸದೆ
ಬದಲಾ-ಯಿಸ-ಬಹುದು
ಬದಲಾ-ಯಿಸ-ಲಾರದು
ಬದಲಾ-ಯಿಸ-ಲಾರವು
ಬದಲಾ-ಯಿಸ-ಲಿಲ್ಲ
ಬದಲಾ-ಯಿ-ಸಲು
ಬದಲಾ-ಯಿಸಲೇ
ಬದಲಾ-ಯಿಸಿ
ಬದಲಾ-ಯಿಸಿ-ದಂತೆ
ಬದಲಾ-ಯಿಸಿ-ದರು
ಬದಲಾ-ಯಿಸಿ-ದರೆ
ಬದಲಾ-ಯಿಸಿ-ರು-ವುದು
ಬದಲಾ-ಯಿಸು
ಬದಲಾ-ಯಿ-ಸುತ್ತ
ಬದಲಾ-ಯಿಸುತ್ತದೆ
ಬದಲಾ-ಯಿ-ಸುತ್ತಾ
ಬದಲಾ-ಯಿಸುತ್ತಿದೆ
ಬದಲಾ-ಯಿಸುತ್ತಿದ್ದರೂ
ಬದಲಾ-ಯಿಸುತ್ತಿ-ರುತ್ತದೆ
ಬದಲಾ-ಯಿಸುತ್ತಿರುತ್ತವೆ
ಬದ-ಲಾ-ಯಿಸುತ್ತಿ-ರುವ
ಬದ-ಲಾ-ಯಿಸುತ್ತಿ-ರುವು-ದನ್ನು
ಬದ-ಲಾ-ಯಿಸುತ್ತಿ-ರುವುದು
ಬದ-ಲಾ-ಯಿಸುತ್ತಿ-ರು-ವುವು
ಬದಲಾ-ಯಿಸುವ
ಬದಲಾ-ಯಿಸು-ವಂತ-ಹದು
ಬದಲಾ-ಯಿಸು-ವು-ದಕ್ಕೆ
ಬದಲಾ-ಯಿಸು-ವುದರ
ಬದಲಾ-ಯಿಸು-ವು-ದ-ರಿಂದ
ಬದಲಾ-ಯಿಸು-ವು-ದಿಲ್ಲ-ವೆಂದು
ಬದಲಾ-ಯಿಸು-ವುದು
ಬದಲಾ-ಯಿಸು-ವೆವು
ಬದ-ಲಾವಣೆ
ಬದ-ಲಾವಣೆ-ಗಳ
ಬದ-ಲಾವಣೆ-ಗಳನ್ನು
ಬದ-ಲಾವಣೆ-ಗಳನ್ನೆಲ್ಲ
ಬದ-ಲಾವಣೆ-ಗಳಾ
ಬದ-ಲಾವಣೆ-ಗಳಿಗೆ
ಬದ-ಲಾವಣೆ-ಗಳು
ಬದ-ಲಾವಣೆ-ಗಳೆಲ್ಲ
ಬದ-ಲಾವಣೆಗೂ
ಬದ-ಲಾವ-ಣೆಗೆ
ಬದ-ಲಾವಣೆಯ
ಬದ-ಲಾವಣೆ-ಯನ್ನು
ಬದ-ಲಾವಣೆ-ಯನ್ನೇ
ಬದ-ಲಾವಣೆ-ಯಲ್ಲದೆ
ಬದ-ಲಾವಣೆ-ಯಾಗ
ಬದ-ಲಾವಣೆ-ಯಾಗದ
ಬದ-ಲಾವಣೆ-ಯಾಗದೆ
ಬದ-ಲಾವಣೆ-ಯಾಗಿ
ಬದ-ಲಾವಣೆ-ಯಾಗುತ್ತಿದ್ದರೂ
ಬದ-ಲಾವಣೆ-ಯಾಗುವ
ಬದ-ಲಾವಣೆ-ಯಾಗು-ವುದು
ಬದ-ಲಾವಣೆ-ಯಾಗು-ವುವು
ಬದ-ಲಾವಣೆ-ಯಾದರೆ
ಬದ-ಲಾವಣೆ-ಯಿಂದ
ಬದ-ಲಾವ-ಣೆಯೂ
ಬದಲು
ಬದುಕ-ಬಲ್ಲೆವು
ಬದುಕ-ಲಾರೆವು
ಬದುಕಲಿ
ಬದುಕಿ
ಬದುಕಿ-ದರೆ
ಬದುಕಿ-ದರೇನು
ಬದು-ಕಿದ್ದ
ಬದುಕಿದ್ದರೆ
ಬದು-ಕಿನ
ಬದು-ಕಿನಿಂದ
ಬದುಕಿಯೂ
ಬದುಕಿ-ರ-ಲಾರೆವು
ಬದುಕಿರು
ಬದುಕಿ-ರುವ
ಬದುಕಿ-ರುವಂತೆ
ಬದುಕಿ-ರುವನು
ಬದುಕಿ-ರುವರು
ಬದುಕಿ-ರುವರೋ
ಬದುಕಿ-ರು-ವ-ವರೆಗೂ
ಬದುಕಿ-ರು-ವಾ-ಗಲೇ
ಬದುಕಿ-ರು-ವಿರೊ
ಬದುಕಿ-ರು-ವು-ದನ್ನು
ಬದುಕಿ-ರುವೆ
ಬದುಕಿವೆ
ಬದುಕು
ಬದುಕು-ಇದೇ
ಬದುಕುತ್ತದೆ
ಬದುಕುತ್ತಾನೆ
ಬದುಕುತ್ತಿದ್ದರು
ಬದುಕುತ್ತೇವೆ
ಬದುಕುವ
ಬದುಕು-ವಂತೆ
ಬದುಕು-ವನು
ಬದುಕು-ವು-ದಕ್ಕೆ
ಬದುಕು-ವುದೇನೂ
ಬದುಕು-ವುವು
ಬದ್ದರಲ್ಲ
ಬದ್ದ-ರೆಂದು
ಬದ್ದರೊ
ಬದ್ಧ
ಬದ್ಧ-ನಲ್ಲ
ಬದ್ಧ-ನಾ-ಗದೆ
ಬದ್ಧ-ನಾಗ-ಲಾರ
ಬದ್ಧ-ನಾಗಿ
ಬದ್ಧ-ನಾಗಿ-ರ-ಬೇಕು
ಬದ್ಧ-ನಾಗಿ-ರುವನು
ಬದ್ಧ-ನಾಗಿ-ರು-ವೆನು
ಬದ್ಧ-ನಾ-ಗಿಲ್ಲ
ಬದ್ಧ-ನಾಗಿಲ್ಲವೋ
ಬದ್ಧ-ನಾಗು-ವನು
ಬದ್ಧ-ನಾಗು-ವನೋ
ಬದ್ಧ-ನಾ-ದಂತೆ
ಬದ್ಧ-ನೆಂದು
ಬದ್ಧ-ನೆಂಬುದೂ
ಬದ್ಧಭ್ರುಕುಟಿ-ಯಿಂದ
ಬದ್ಧ-ರಲ್ಲ
ಬದ್ಧ-ರಾಗ-ಬೇಕಾ-ಗಿಲ್ಲ
ಬದ್ಧ-ರಾಗಿಯೆ
ಬದ್ಧ-ರಾ-ಗಿಯೇ
ಬದ್ಧ-ರಾಗಿ-ರಲೇ-ಬೇಕು
ಬದ್ಧ-ರಾ-ಗಿ-ರು-ವಿರೊ
ಬದ್ಧ-ರಾ-ಗಿ-ರು-ವೆವು
ಬದ್ಧ-ರಾ-ಗಿಲ್ಲ
ಬದ್ಧ-ರಾಗು-ವು-ದಿಲ್ಲ
ಬದ್ಧರು
ಬದ್ಧ-ರೆಂದು
ಬದ್ಧ-ವಲ್ಲ-ವೆಂದು
ಬದ್ಧ-ವಾಗದ
ಬದ್ಧ-ವಾಗ-ಲಾರದು
ಬದ್ಧ-ವಾಗಿ
ಬದ್ಧ-ವಾಗಿದೆ
ಬದ್ಧ-ವಾಗಿದ್ದರೆ
ಬದ್ಧ-ವಾಗಿ-ರಲಿಲ್ಲ-ವೆಂಬುದೆ
ಬದ್ಧ-ವಾಗಿ-ರುವು-ದೆಲ್ಲಾ
ಬದ್ಧ-ವಾಗುತ್ತಿತ್ತು
ಬದ್ಧ-ವಾದ
ಬನ್ನಿ
ಬಪೆಂಕಿಯ
ಬಯಕೆ
ಬಯಕೆ-ಗಳಂತೆ
ಬಯಕೆ-ಗಳನ್ನೆಲ್ಲಾ
ಬಯಕೆ-ಗಳಿಂದ
ಬಯಕೆ-ಗಳಿಗೆ
ಬಯಕೆ-ಗಳಿವೆ
ಬಯಕೆ-ಗಳು
ಬಯಕೆ-ಗಳೆಲ್ಲ
ಬಯ-ಕೆಯ
ಬಯಕೆ-ಯಾ-ಗಿತ್ತು
ಬಯಕೆ-ಯಿಂದ
ಬಯಕೆ-ಯೊಂದಿಗೆ
ಬಯಲಭ್ರಾಂತಿ
ಬಯ-ಲಿಗೆ
ಬಯ-ಸದಿ-ರು-ವು-ದ-ರಿಂದ
ಬಯಸರೊ
ಬಯಸಿ-ದಂತೆ
ಬಯಸಿ-ದರು
ಬಯಸಿ-ದರೆ
ಬಯ-ಸಿದ್ದೆ-ನಲ್ಲ
ಬಯಸುತ್ತವೆ
ಬಯಸುತ್ತಾರೆ
ಬಯ-ಸುತ್ತೇವೆ
ಬಯ-ಸುವ
ಬಯ-ಸುವನು
ಬಯ-ಸುವನೊ
ಬಯ-ಸುವರು
ಬಯ-ಸು-ವರೊ
ಬಯ-ಸುವರೋ
ಬಯ-ಸುವ-ವನು
ಬಯ-ಸು-ವು-ದಕ್ಕೆ
ಬಯ-ಸು-ವು-ದಿಲ್ಲ
ಬಯ-ಸು-ವುದು
ಬಯ-ಸು-ವುದೊಂದೆ-ಅದೇ
ಬಯ-ಸು-ವೆವು
ಬರ
ಬರಗಾಲ-ವಿಲ್ಲ
ಬರದ
ಬರ-ದಂತೆ
ಬರ-ದಿದ್ದರೆ
ಬರದೇ
ಬರ-ಬರುತ್ತಾ
ಬರ-ಬಲ್ಲರು
ಬರ-ಬಹು-ದಾದ
ಬರ-ಬಹುದು
ಬರ-ಬಹು-ದೆಂದು
ಬರ-ಬೇಕಾ-ಗಿದೆ
ಬರ-ಬೇಕು
ಬರ-ಬೇಕು-ಎನ್ನು-ವು-ದನ್ನು
ಬರ-ಬೇಕೆಂದು
ಬರ-ಬೇಕೆಂಬು-ದನ್ನು
ಬರ-ಮಾಡಿ-ಕೊಳ್ಳು-ವುದು
ಬರ-ಲಾರದ
ಬರ-ಲಾರದು
ಬರ-ಲಾರದೋ
ಬರ-ಲಾರ-ವು-ಒಂದು
ಬರಲಿ
ಬರಲಿ-ರುವ
ಬರ-ಲಿಲ್ಲ
ಬರ-ಲಿಲ್ಲ-ವೆಂದು
ಬರಲು
ಬರಲೇ
ಬರಲೇ-ಬೇಕಾ-ಗು-ವುದು
ಬರಲೇ-ಬೇಕು
ಬರ-ವಣಿಗೆ-ಯಲ್ಲಿ
ಬರ-ವನ್ನು
ಬರಹ-ಗಾರರು
ಬರ-ಹದ
ಬರಹವೂ
ಬರಹ-ವೆಂಬು-ದಿಲ್ಲ
ಬರಿ
ಬರಿ-ದಾದ
ಬರಿಯ
ಬರೀ
ಬರು
ಬರುತ್ತ
ಬರುತ್ತದೆ
ಬರುತ್ತದೆಯೋ
ಬರುತ್ತಲೂ
ಬರುತ್ತವೆ
ಬರುತ್ತಾ
ಬರುತ್ತಾನೆ
ಬರುತ್ತಾರೆ
ಬರುತ್ತಿ
ಬರುತ್ತಿತ್ತೆಂದು
ಬರುತ್ತಿದೆ
ಬರುತ್ತಿ-ದೆಯೋ
ಬರುತ್ತಿದ್ದ
ಬರುತ್ತಿದ್ದರು
ಬರುತ್ತಿದ್ದರೂ
ಬರುತ್ತಿದ್ದರೆ
ಬರುತ್ತಿ-ರ-ಲಿಲ್ಲ
ಬರುತ್ತಿ-ರುವ
ಬರುತ್ತಿ-ರು-ವರು
ಬರುತ್ತಿ-ರುವುದು
ಬರುತ್ತಿ-ರು-ವುವು
ಬರುತ್ತಿ-ರುವೆ
ಬರುತ್ತಿವೆ
ಬರುತ್ತೇನೆ-ಬಹುಶಃ
ಬರುತ್ತೇವೆ
ಬರುವ
ಬರು-ವಂತಹ
ಬರು-ವಂತೆ
ಬರುವನು
ಬರುವ-ನೆನ್ನು-ವುದು
ಬರು-ವರು
ಬರುವ-ವ-ರಾರು
ಬರುವ-ವ-ರಿಗೆ
ಬರು-ವ-ವರು
ಬರು-ವ-ವರೆಗೆ
ಬರು-ವಾಗ
ಬರು-ವಿರಿ
ಬರುವು
ಬರು-ವು-ದಕ್ಕಿಂತ
ಬರು-ವು-ದಕ್ಕೆ
ಬರು-ವು-ದನ್ನು
ಬರುವು-ದನ್ನೆಲ್ಲಾ
ಬರುವು-ದರ
ಬರು-ವು-ದ-ರಲ್ಲಿ
ಬರು-ವು-ದ-ರಿಂದ
ಬರುವು-ದಾಗಲಿ
ಬರುವು-ದಾ-ದರೂ
ಬರುವು-ದಾದರೆ
ಬರು-ವು-ದಿಲ್ಲ
ಬರು-ವು-ದಿಲ್ಲ-ಆ-ದರೆ
ಬರು-ವು-ದಿಲ್ಲಧ್ಯಾನಸ್ಥಿತಿ-ಯಲ್ಲಿ
ಬರು-ವು-ದಿಲ್ಲ-ವೆಂದು
ಬರು-ವು-ದಿಲ್ಲ-ವೆಂಬುದು
ಬರು-ವುದು
ಬರು-ವುದು-ಎಂಬುದೇ
ಬರು-ವುದು-ಕಪ್ಪೆ-ಯ-ಚಿಪ್ಪನ್ನು
ಬರು-ವುದು-ಕಿರುಚಲು
ಬರುವುದೂ
ಬರುವುದೆ
ಬರು-ವು-ದೆಂದು
ಬರುವು-ದೆಂಬುದು
ಬರುವು-ದೆಂಬುದೇನೋ
ಬರುವು-ದೆನ್ನು-ವನು
ಬರು-ವು-ದೆಲ್ಲಿಗೆ
ಬರು-ವುದೇ
ಬರುವುದೋ
ಬರು-ವುವು
ಬರು-ವುವು-ಆಗ
ಬರುವು-ವೆಂದು
ಬರು-ವೆವು
ಬರು-ವೆವು-ಏಕೆ
ಬರು-ವೆವೊ
ಬರೆ
ಬರೆದ
ಬರೆ-ದರೆ
ಬರೆ-ದಿಟ್ಟರು
ಬರೆ-ದಿಟ್ಟರೆ
ಬರೆ-ದಿ-ರುವ
ಬರೆ-ದಿರು-ವುದೆಂದರೆ
ಬರೆ-ದುದು
ಬರೆಯ
ಬರೆ-ಯುತ್ತಿದ್ದರು
ಬರೆ-ಯು-ವರು
ಬರೆ-ಯು-ವು-ದ-ರಿಂದ
ಬರೋಣ
ಬಲ
ಬಲ-ಗಡೆ
ಬಲ-ಗಡೆಯ
ಬಲ-ಗಡೆ-ಯದೇ
ಬಲಗೈ
ಬಲ-ಗೊಳಿ-ಸು-ವು-ದಕ್ಕೆ
ಬಲ-ಗೊಳಿ-ಸು-ವುದೇ
ಬಲದ
ಬಲ-ದಿಂದ
ಬಲ-ಪಡಿ-ಸು-ವುದು
ಬಲಪ್ರ-ಯೋಗಕ್ಕೂ
ಬಲ-ಭಾಗ-ದಲ್ಲಿ-ರುವುದು
ಬಲ-ಯುತ-ವಾಗಿ
ಬಲ-ಯುತ-ವಾಗಿದೆ
ಬಲ-ಯುತ-ವಾಗಿ-ದೆಯೋ
ಬಲ-ಯುತ-ವಾಗಿಯೂ
ಬಲ-ಯುತ-ವಾದ
ಬಲ-ವಂತ
ಬಲ-ವಂತ-ದಿಂದ
ಬಲ-ವಂತವೇ
ಬಲ-ವನ್ನು
ಬಲ-ವಾಗಿ
ಬಲ-ವಾಗಿದೆ
ಬಲ-ವಾಗಿದ್ದು
ಬಲ-ವಾ-ಗಿ-ರು-ವಂತೆ
ಬಲ-ವಾಗಿ-ರು-ವು-ದನ್ನೂ
ಬಲ-ವಾ-ಗಿ-ರುವುದು
ಬಲ-ವಾಗಿ-ರು-ವು-ದೆಂದು
ಬಲ-ವಾದ
ಬಲ-ವಾ-ದರೆ
ಬಲ-ವಾ-ದಾಗ
ಬಲ-ವಾದು-ದಲ್ಲ
ಬಲ-ವಾ-ದುದು
ಬಲ-ವಾ-ಯಿತು
ಬಲ-ವಿ-ರಲಿ
ಬಲ-ಶಾಲಿ
ಬಲ-ಶಾಲಿ-ಗಳು
ಬಲ-ಶಾಲಿ-ಯಾಗು-ವ-ವರೆಗೆ
ಬಲ-ಹೀನ
ಬಲ-ಹೀನ-ತೆ-ಯೆಲ್ಲ
ಬಲ-ಹೀನ-ನನ್ನು
ಬಲ-ಹೀನ-ರನ್ನಾಗಿ
ಬಲ-ಹೀನ-ರಾಗಿ-ರು-ವು-ದ-ರಿಂದ
ಬಲ-ಹೀನ-ರಾಗಿ-ರು-ವುದು
ಬಲ-ಹೀನ-ರಿಗೆ
ಬಲ-ಹೀನ-ರೆಂದು
ಬಲ-ಹೀನ-ವಾ-ಗು-ವುವು
ಬಲ-ಹೊಳ್ಳೆಯ
ಬಲಾಢ್ಯ
ಬಲಾಢ್ಯ-ನನ್ನಾಗಿ
ಬಲಾಢ್ಯ-ನಾ-ಗಿ-ರುವನು
ಬಲಾಢ್ಯ-ನಾಗುತ್ತೇನೆ
ಬಲಾಢ್ಯನು
ಬಲಾಢ್ಯ-ರಾಗ-ಬಹುದು
ಬಲಾಢ್ಯ-ರಾಗಿ
ಬಲಾಢ್ಯ-ರಾಗಿ-ರು-ವಿರಾ
ಬಲಾಢ್ಯ-ರೆಂದು
ಬಲಾಢ್ಯ-ವಾ-ದುದು
ಬಲಾತ್ಕರಿ-ಸಲ್ಪಡುತ್ತದೆ
ಬಲಾತ್ಕ-ರಿಸಿ
ಬಲಾತ್ಕ-ರಿಸಿ-ದರೆ
ಬಲಾತ್ಕ-ರಿಸುತ್ತಿವೆ
ಬಲಾತ್ಕರಿ-ಸುವ
ಬಲಾತ್ಕರಿ-ಸು-ವುದು
ಬಲಾತ್ಕಾರ
ಬಲಾತ್ಕಾರದ
ಬಲಾತ್ಕಾರ-ದಿಂದ
ಬಲಾತ್ಕಾರ-ದಿಂದ-ಲಾ-ದರೂ
ಬಲಾತ್ಕಾರ-ವಾಗಿ
ಬಲಾನಿ
ಬಲಿ
ಬಲಿ-ಕೊಡು-ವುದು
ಬಲಿ-ಗಳ
ಬಲಿ-ಯಾಗುತ್ತಿ-ರು-ವರು
ಬಲಿಷ್ಠ
ಬಲೆ-ಗಳಾ-ಗಿವೆ
ಬಲೆಗೆ
ಬಲೆಯ
ಬಲೆ-ಯನ್ನು
ಬಲೆ-ಯಲ್ಲಿ
ಬಲೇಷು
ಬಲ್ಲ
ಬಲ್ಲದು
ಬಲ್ಲನು
ಬಲ್ಲ-ರೆಂದು
ಬಲ್ಲರೊ
ಬಲ್ಲ-ವ-ನಾಗಿದ್ದನು
ಬಲ್ಲಿ-ದರು
ಬಲ್ಲಿರಿ
ಬಲ್ಲೆ
ಬಲ್ಲೆನೆ
ಬಳಕೆ
ಬಳ-ಕೆಗೆ
ಬಳಕೆ-ಯಲ್ಲಿದೆ
ಬಳಗ
ಬಳಲಿದ
ಬಳಸ-ಬಹುದು
ಬಳಸ-ಬೇಕಾ-ದರೆ
ಬಳ-ಸ-ಬೇಕು
ಬಳಸಿ
ಬಳಸಿ-ದರು
ಬಳಸಿ-ದರೆ
ಬಳಸಿದ್ದಾರೆ
ಬಳಸಿ-ರುತ್ತಾರೆ
ಬಳ-ಸಿಲ್ಲ
ಬಳಸು
ಬಳಸುತ್ತಾರೆ
ಬಳ-ಸುತ್ತಿದ್ದರು
ಬಳಸುತ್ತಿ-ರು-ವರು
ಬಳಸುತ್ತೇನೆ
ಬಳ-ಸು-ವು-ದಿಲ್ಲ
ಬಳ-ಸು-ವೆವು
ಬಳಿಕ
ಬಳಿಗೆ
ಬಳಿದು-ಕೊಂಡು
ಬಳಿ-ಯಲ್ಲಿ
ಬಳ್ಳಿ
ಬಳ್ಳಿಯ
ಬಳ್ಳಿ-ಯಲ್ಲಿ-ರುವ
ಬಸ-ವನ
ಬಸ-ವನ-ಹುಳ
ಬಸ-ವನ-ಹುಳು
ಬಸ-ವನ-ಹುಳು-ವಿನ
ಬಸ-ವನ-ಹುಳು-ವಿ-ನಲ್ಲಿ
ಬಸ-ವನ-ಹುಳು-ವಿ-ನಿಂದ
ಬಹಳ
ಬಹಳ-ಕಾಲ
ಬಹಳ-ವಾಗಿ
ಬಹಳ-ವಾಗಿದೆ
ಬಹಿ-ರಂಗ
ಬಹಿ-ರಂಗಂ
ಬಹಿ-ರಂಗ-ಪಡಿ-ಸು-ವುವು
ಬಹಿ-ರಂಗ-ವಾಗಿ
ಬಹಿರ-ಕಲ್ಪಿತಾ
ಬಹಿರ್ಮುಖ-ಮಾಡಿ-ರು-ವು-ದ-ರಿಂದ
ಬಹಿರ್ಮುಖ-ವಾಗಿ
ಬಹಿರ್ಮುಖಿ-ಗಳು
ಬಹಿಷ್ಕ-ರಿಸಿ-ದು-ದನ್ನು
ಬಹಿಷ್ಕಾರ
ಬಹು
ಬಹು-ಕಾಲ
ಬಹು-ಜ-ನರ
ಬಹು-ಜ-ನರು
ಬಹುತ್ವ-ದಲ್ಲಿ
ಬಹು-ದಾ-ಗಿತ್ತು
ಬಹುದು
ಬಹು-ದು-ಕೊ-ನೆ-ಯದು
ಬಹು-ದೂರ
ಬಹು-ದೂರ-ದಲ್ಲಿ-ರುವುದು
ಬಹು-ದೂರ-ವಿ-ರ-ಬಹುದು
ಬಹುದೆ
ಬಹು-ದೆಂಬು-ದನ್ನು
ಬಹುದೊ
ಬಹುದೋ
ಬಹು-ಪಾ-ಲನ್ನು
ಬಹು-ಪಾಲು
ಬಹು-ಭಾಗ
ಬಹು-ಭಾಗ-ದಲ್ಲಿ
ಬಹು-ಭಾಗ-ವನ್ನು
ಬಹು-ಮಂದಿ
ಬಹು-ಮಟ್ಟಿಗೆ
ಬಹು-ಮಾನ
ಬಹು-ಮಾನ-ಗಳೆಂಬ
ಬಹು-ಮಾನದ
ಬಹು-ಮುಖ್ಯ
ಬಹು-ಮುಖ್ಯ-ವಾದ
ಬಹು-ಮುಖ್ಯ-ವಾದುದು
ಬಹು-ವಾಗಿ-ಮಾಡಿ
ಬಹು-ವಿಧ
ಬಹು-ವಿಧದ
ಬಹು-ವಿಧ-ವಾಗಿವೆ
ಬಹುಶಃ
ಬಾಂಧವ್ಯಕ್ಕಾಗಿ
ಬಾಂಧವ್ಯಕ್ಕಿಂತಲೂ
ಬಾಂಧವ್ಯಕ್ಕೆ
ಬಾಂಧವ್ಯ-ವನ್ನು
ಬಾಂಧವ್ಯವು
ಬಾಗದ
ಬಾಗ-ಬೇಕು
ಬಾಗಿ
ಬಾಗಿಲ
ಬಾಗಿ-ಲನ್ನು
ಬಾಗಿ-ಲಿನ
ಬಾಗಿ-ಲಿ-ನಲ್ಲೆ
ಬಾಗಿಲು
ಬಾಗುತ್ತೇನೆ
ಬಾಗು-ವುದು
ಬಾಚಿ
ಬಾಡು-ವುದು
ಬಾತಿನ
ಬಾತಿ-ಮರಿ-ಗಳಿಗೆ
ಬಾತು-ಹೋಗಿ
ಬಾಧಿತ-ನಾಗದ
ಬಾಧಿತ-ನಾಗು-ವು-ದಿಲ್ಲ
ಬಾಯಾರಿಕೆ
ಬಾಯಾರಿ-ಕೆ-ಯಾ-ಗಿದೆ
ಬಾಯಿ
ಬಾಯಿಂದ
ಬಾಯಿಗೆ
ಬಾಯಿ-ಮಾ-ತಿನ
ಬಾಯಿ-ಮಾ-ತಿನ-ವರಿ
ಬಾಯಿಯ
ಬಾಯಿ-ಯನ್ನೂ
ಬಾಯಿ-ಯಿಂದ
ಬಾರದ
ಬಾರ-ದಂತೆ
ಬಾರದ-ವ-ರಿಗೆ
ಬಾರದ-ವು-ಗಳೆ
ಬಾರದವೂ
ಬಾರದು
ಬಾರದು-ದನ್ನು
ಬಾರದುದು
ಬಾರದು-ದೆಂದೂ
ಬಾರದೆ
ಬಾರಿ
ಬಾರಿ-ಸು-ವುದು
ಬಾಲ
ಬಾಲಕ
ಬಾಲ-ಕನು
ಬಾಲ-ಕ-ನೊಬ್ಬ-ನಿಗೆ
ಬಾಲ-ಕ-ನೊಬ್ಬನು
ಬಾಲ-ಭಾಷೆ
ಬಾಲ-ಭಾಷೆಗೂ
ಬಾಲ-ಭಾಷೆ-ಯಂತೆ
ಬಾಲಾಸ್ತೇ
ಬಾಲ್ಯ
ಬಾಲ್ಯದ
ಬಾಲ್ಯ-ದಲ್ಲಿ
ಬಾಲ್ಯ-ದಿಂದಲೂ
ಬಾಲ್ಯ-ದಿಂದಲೇ
ಬಾಲ್ಯ-ವಿ-ವಾ-ಹಕ್ಕೆ
ಬಾಲ್ಯ-ವಿವಾಹವು
ಬಾಲ್ಯಾರಭ್ಯ
ಬಾಲ್ಯಾ-ವಸ್ಥೆ
ಬಾಲ್ಯಾ-ವಸ್ಥೆಯ
ಬಾಲ್ಯಾ-ವಸ್ಥೆ-ಯ-ದರ
ಬಾಲ್ಯಾ-ವಸ್ಥೆ-ಯನ್ನು
ಬಾಳ
ಬಾಳನ್ನು
ಬಾಳನ್ನೆಲ್ಲ
ಬಾಳನ್ನೇ
ಬಾಳ-ಬಲ್ಲ
ಬಾಳ-ಬಲ್ಲರು
ಬಾಳ-ಬೇಕಾ-ಗಿದೆ
ಬಾಳ-ಬೇಕು
ಬಾಳ-ಬೇಕೆಂಬ
ಬಾಳ-ಲಾರರು
ಬಾಳಲು
ಬಾಳಲೆ
ಬಾಳ-ಲೆಂದು
ಬಾಳಿ
ಬಾಳಿಗೆ
ಬಾಳಿ-ದಂತೆ
ಬಾಳಿ-ದರು
ಬಾಳಿದ್ದಾರೆ
ಬಾಳಿನ
ಬಾಳು
ಬಾಳುತ್ತಿದ್ದ
ಬಾಳುತ್ತಿ-ರುವ
ಬಾಳುತ್ತಿ-ರುವನೊ
ಬಾಳುತ್ತಿ-ರುವುದು
ಬಾಳುತ್ತಿ-ರು-ವೆವು
ಬಾಳುತ್ತೇವೆ
ಬಾಳುವ
ಬಾಳು-ವಂತೆ
ಬಾಳು-ವನು
ಬಾಳು-ವರು
ಬಾಳು-ವ-ವರು
ಬಾಳುವಿ
ಬಾಳು-ವು-ದಕ್ಕಿಂತ
ಬಾಳು-ವು-ದಕ್ಕೆ
ಬಾಳು-ವುದನ್ನೇ
ಬಾಳು-ವುದು
ಬಾಳು-ವೆ-ಗಾಗಿ
ಬಾಳು-ವೆಯ
ಬಾಳು-ವೆ-ಯಲ್ಲಿ
ಬಾಳೂ
ಬಾಳೇ
ಬಾಳೊಂದು
ಬಾಳೋಣ
ಬಾವನೆ-ಗಳ
ಬಾವನೆ-ಗಳು
ಬಾಹಿರ-ರಲ್ಲ
ಬಾಹಿರ-ರಾಗಿ
ಬಾಹು-ಗ-ಳನ್ನು
ಬಾಹು-ಬಲ-ಗ-ಳನ್ನು
ಬಾಹ್ಯ
ಬಾಹ್ಯ-ಕರಣ
ಬಾಹ್ಯ-ಕರ-ಣ-ವಿದೆ
ಬಾಹ್ಯ-ಕವಚ
ಬಾಹ್ಯ-ಕಾರ-ಣ-ಗ-ಳನ್ನು
ಬಾಹ್ಯ-ಕಾರ-ಣ-ಗಳು
ಬಾಹ್ಯ-ಕಾರ-ಣ-ದಿಂದ
ಬಾಹ್ಯಕ್ರಿಯೆ-ಗಳೆಲ್ಲ
ಬಾಹ್ಯ-ಜ-ಗತ್ತನ್ನು
ಬಾಹ್ಯ-ಜ-ಗತ್ತಿನಲ್ಲಿ
ಬಾಹ್ಯ-ಜ-ಗತ್ತಿನಷ್ಟೇ
ಬಾಹ್ಯ-ಜಗತ್ತು
ಬಾಹ್ಯ-ದಲ್ಲಿದೆ
ಬಾಹ್ಯ-ದಲ್ಲಿ-ರುವು-ದನ್ನೆಲ್ಲಾ
ಬಾಹ್ಯ-ದಿಂದ
ಬಾಹ್ಯ-ದೃಶ್ಯ-ಗಳ
ಬಾಹ್ಯ-ದೇಹಕ್ಕಿಂತ
ಬಾಹ್ಯ-ದೇಹ-ವನ್ನು
ಬಾಹ್ಯ-ಪರಿ-ಶೀ-ಲನೆಯ
ಬಾಹ್ಯಪ್ರಕ-ಟಿತ-ವಾ-ಗು-ವುದು
ಬಾಹ್ಯಪ್ರ-ಕೃತಿ-ಯನ್ನೂ
ಬಾಹ್ಯಪ್ರಪಂಚ-ದಲ್ಲಿ
ಬಾಹ್ಯಪ್ರಪಂಚ-ವಾ-ಗು-ವುದು
ಬಾಹ್ಯ-ಮುಖ-ವಾಗಿದೆ
ಬಾಹ್ಯ-ಯಂತ್ರ
ಬಾಹ್ಯ-ಯಂತ್ರ-ದೊ-ಡನೆ
ಬಾಹ್ಯ-ವಸ್ತು
ಬಾಹ್ಯ-ವಸ್ತು-ಗಳ
ಬಾಹ್ಯ-ವಸ್ತು-ಗ-ಳನ್ನು
ಬಾಹ್ಯ-ವಸ್ತು-ಗಳಿಂದಲೂ
ಬಾಹ್ಯ-ವಸ್ತು-ಗಳು
ಬಾಹ್ಯ-ವಸ್ತು-ವನ್ನು
ಬಾಹ್ಯ-ವಸ್ತು-ವಿನ
ಬಾಹ್ಯ-ವಸ್ತು-ವಿ-ನಿಂದ
ಬಾಹ್ಯ-ವಸ್ತುವೂ
ಬಾಹ್ಯ-ವಾಗಿ-ರ-ಬಹುದು
ಬಾಹ್ಯ-ವಾದ
ಬಾಹ್ಯ-ವಿಶ್ವ-ವೆಲ್ಲ
ಬಾಹ್ಯ-ಶಕ್ತಿ
ಬಾಹ್ಯ-ಸಾಧನೆ-ಗಳನ್ನೆಲ್ಲಾ
ಬಾಹ್ಯಸ್ಪಂದನ
ಬಾಹ್ಯಸ್ಪಂದನ-ದಿಂದ
ಬಾಹ್ಯಸ್ವಾ-ತಂತ್ರ್ಯದ
ಬಾಹ್ಯಾ-ಕಾ-ರಕ್ಕೆ
ಬಾಹ್ಯಾ-ಚಾರ
ಬಾಹ್ಯಾ-ಚಾರ-ಗಳ
ಬಾಹ್ಯಾ-ಚಾರದ
ಬಾಹ್ಯಾಭ್ಯಂತರಸ್ತಂಭ-ವೃತ್ತಿಃ
ಬಾಹ್ಯೇಂದ್ರಿಯ-ಗಳಿಂದ
ಬಾಹ್ಯೇಂದ್ರಿಯ-ಗಳಿ-ಗಿಂತಲೂ
ಬಾಹ್ಯೇಂದ್ರಿಯ-ಗಳಿವೆ
ಬಾಹ್ಯೇಂದ್ರಿಯ-ಗಳು
ಬಿ
ಬಿಂದು
ಬಿಂದು-ವಿಗೂ
ಬಿಂದು-ವಿ-ನಲ್ಲಿ
ಬಿಂದುವೂ
ಬಿಂಬ-ಗಳು
ಬಿಎಂಬುದೆ
ಬಿಎಬಿಎ
ಬಿಗಿದ
ಬಿಗಿದು
ಬಿಗಿ-ಯಾಗಿ
ಬಿಗಿಯು-ವುದು
ಬಿಗಿ-ಹಿಡಿ-ಯು-ವುದೇ
ಬಿಟ್ಟ
ಬಿಟ್ಟನು
ಬಿಟ್ಟರು
ಬಿಟ್ಟರೂ
ಬಿಟ್ಟರೆ
ಬಿಟ್ಟಾಗ
ಬಿಟ್ಟಿ-ರು-ವರು
ಬಿಟ್ಟು
ಬಿಟ್ಟು-ಕೊಡು
ಬಿಟ್ಟು-ಬಿ-ಡು-ವುದು
ಬಿಟ್ಟು-ಹೋಗಿ
ಬಿಡ
ಬಿಡ-ಕೂಡದು
ಬಿಡದೆ
ಬಿಡ-ಬಲ್ಲ
ಬಿಡ-ಬಹುದು
ಬಿಡ-ಬಾ-ರದು
ಬಿಡ-ಬೇಕಾ-ಗಿದೆ
ಬಿಡ-ಬೇಕಾ-ಗಿಲ್ಲ
ಬಿಡ-ಬೇಕಾ-ಯಿತು
ಬಿಡ-ಬೇಕು
ಬಿಡ-ಬೇಡಿ
ಬಿಡ-ಲಿಲ್ಲ
ಬಿಡಲು
ಬಿಡಿ
ಬಿಡಿ-ಸಲು
ಬಿಡಿ-ಸಿ-ಕೊಂಡು
ಬಿಡಿ-ಸಿ-ರುವ
ಬಿಡಿಸು
ಬಿಡಿ-ಸು-ವುದು
ಬಿಡಿ-ಸು-ವುದೇ
ಬಿಡು-ಗಡೆ
ಬಿಡುಗ-ಡೆಗೆ
ಬಿಡುತ್ತದೆ
ಬಿಡುತ್ತಾ
ಬಿಡುತ್ತೇವೆ
ಬಿಡುವ
ಬಿಡು-ವಂತೆ
ಬಿಡುವನು
ಬಿಡು-ವರೊ
ಬಿಡುವಾಗ
ಬಿಡು-ವಿರಿ
ಬಿಡು-ವಿಲ್ಲದ
ಬಿಡು-ವಿಲ್ಲದೆ
ಬಿಡು-ವು-ದಿಲ್ಲ
ಬಿಡು-ವುದು
ಬಿಡು-ವುದೆಂಬು-ದನ್ನು
ಬಿಡು-ವುದೇ
ಬಿಡು-ವೆನು
ಬಿತ್ತ-ಬೇಕಾ
ಬಿತ್ತಿ
ಬಿತ್ತಿದ
ಬಿತ್ತಿ-ದಂತೆ
ಬಿತ್ತಿ-ದಷ್ಟು
ಬಿತ್ತಿ-ದು-ದನ್ನು
ಬಿತ್ತಿಲ್ಲ
ಬಿತ್ತು
ಬಿತ್ತುತ್ತಿ-ರು-ವೆವು
ಬಿದ್ದ
ಬಿದ್ದನು
ಬಿದ್ದ-ಮೇಲೆ
ಬಿದ್ದರೂ
ಬಿದ್ದರೆ
ಬಿದ್ದವು
ಬಿದ್ದಾಗ
ಬಿದ್ದಿದೆ
ಬಿದ್ದಿದ್ದರೂ
ಬಿದ್ದಿರ
ಬಿದ್ದಿರ-ಲಿಲ್ಲ
ಬಿದ್ದಿರು
ಬಿದ್ದಿರುವ
ಬಿದ್ದು
ಬಿದ್ದು-ಹೋ-ದರೆ
ಬಿನ್ನ-ಭಿನ್ನ
ಬಿರು-ಗಾಳಿಗೆ
ಬಿರು-ಗಾಳಿ-ಯಲ್ಲಿ
ಬಿಲ-ದಲ್ಲಿ
ಬಿಳಲಿ-ನಿಂದ
ಬಿಳಿ
ಬಿಳಿ-ಕೊಂಡಿ
ಬಿಳಿಪು
ಬಿಳಿಯ
ಬಿಳಿಯೂ
ಬಿಸಿ-ಯಾ-ಗುತ್ತದೆ
ಬಿಸಿ-ರಕ್ತ-ವನ್ನು
ಬಿಸಿಲು
ಬೀಗದಕೈ
ಬೀಜ
ಬೀಜ-ಇ-ವನ್ನು
ಬೀಜಕ್ಕೆ
ಬೀಜ-ಗಳಾ-ವ-ವವೂ
ಬೀಜ-ಗಳು
ಬೀಜ-ಗಳೇ
ಬೀಜದ
ಬೀಜ-ದಲ್ಲಿ
ಬೀಜ-ದಲ್ಲಿತ್ತು
ಬೀಜ-ದಲ್ಲೇ
ಬೀಜ-ದಿಂದ
ಬೀಜ-ರೂಪಕ್ಕೆ
ಬೀಜ-ರೂಪ-ವನ್ನು
ಬೀಜ-ರೂಪ-ವಾಗಿ-ದೆಯೇ
ಬೀಜ-ವನ್ನು
ಬೀಜವು
ಬೀದಿಯ
ಬೀರ-ಬಲ್ಲದು
ಬೀರ-ಲಾರದು
ಬೀರ-ಲಾರದೋ
ಬೀರಿ
ಬೀರಿದ
ಬೀರಿ-ರುತ್ತದೆ
ಬೀರುತ್ತದೆ
ಬೀರುತ್ತವೆ
ಬೀರುತ್ತಾ-ನೆಯೋ
ಬೀರುತ್ತಿತ್ತು
ಬೀರುತ್ತಿದೆ
ಬೀರುತ್ತಿದ್ದ
ಬೀರುತ್ತಿ-ರುವ
ಬೀರುತ್ತೇವೆ
ಬೀರುವ
ಬೀರುವನು
ಬೀರು-ವುದು
ಬೀರುವುದೊ
ಬೀರು-ವುವು
ಬೀಳ
ಬೀಳ-ಬಹುದು
ಬೀಳ-ಬಾರ-ದೆಂದು
ಬೀಳ-ಬೇಕು
ಬೀಳ-ಬೇಡಿ
ಬೀಳಲು
ಬೀಳು
ಬೀಳು-ಗಳಂತೆ
ಬೀಳುತ್ತ
ಬೀಳುತ್ತದೆ
ಬೀಳುತ್ತಾ-ನೆಯೋ
ಬೀಳುತ್ತಿತ್ತು
ಬೀಳುತ್ತಿದೆ
ಬೀಳುತ್ತಿ-ರುವ
ಬೀಳುತ್ತಿ-ರುವಾಗ
ಬೀಳುತ್ತಿ-ರು-ವುವು
ಬೀಳುತ್ತೇವೆ
ಬೀಳುವ
ಬೀಳು-ವಂತೆ
ಬೀಳು-ವನು
ಬೀಳು-ವರು
ಬೀಳು-ವಳು
ಬೀಳು-ವ-ವರೆಗೆ
ಬೀಳುವು
ಬೀಳು-ವು-ದನ್ನು
ಬೀಳು-ವು-ದಿಲ್ಲ
ಬೀಳು-ವು-ದಿಲ್ಲ-ಅಂದರೆ
ಬೀಳು-ವು-ದಿಲ್ಲ-ವೆಂದು
ಬೀಳು-ವುದು
ಬೀಳು-ವುದೂ
ಬೀಳು-ವು-ದೆಂದು
ಬೀಳು-ವುದೊ
ಬೀಳು-ವುವು
ಬೀಳು-ವೆನು
ಬೀಸಿ
ಬೀಸಿ-ಕೊಂಡಿ-ರು-ವೆವು
ಬೀಸಿ-ಕೊಳ್ಳುವ
ಬೀಸುತ್ತಿದೆ
ಬೀಸು-ವುದು
ಬುಡ-ದಲ್ಲಿ
ಬುಡ-ವಿಲ್ಲ
ಬುಡ-ವಿಲ್ಲ-ವೆಂದು
ಬುದ್ದಿಗೂ
ಬುದ್ದಿಯ
ಬುದ್ದಿ-ಯಿಂದ
ಬುದ್ಧ
ಬುದ್ಧ-ದೇವ-ನಷ್ಟು
ಬುದ್ಧ-ದೇ-ವನು
ಬುದ್ಧನ
ಬುದ್ಧ-ನಂತಹ
ಬುದ್ಧ-ನಂತೆ
ಬುದ್ಧ-ನನ್ನು
ಬುದ್ಧ-ನಲ್ಲಿ
ಬುದ್ಧ-ನಿಗೆ
ಬುದ್ಧನು
ಬುದ್ಧ-ನೆಂದರೆ
ಬುದ್ಧ-ನೆನ್ನು-ವರೊ
ಬುದ್ಧ-ನೊಂದಿಗೆ
ಬುದ್ಧ-ರಂತಹ
ಬುದ್ಧ-ರಾಗು-ವಿರಿ
ಬುದ್ಧ-ರಿಗೆ
ಬುದ್ಧಿ
ಬುದ್ಧಿ-ಕೆಟ್ಟ-ವರು
ಬುದ್ಧಿ-ಗಳ
ಬುದ್ಧಿ-ಗಳು
ಬುದ್ಧಿ-ಗಾಗಿ
ಬುದ್ಧಿಗೂ
ಬುದ್ಧಿಗೆ
ಬುದ್ಧಿ-ಬುದ್ಧೇರತಿಪ್ರ-ಸಂಗಃ
ಬುದ್ಧಿಯ
ಬುದ್ಧಿ-ಯಂತೆ
ಬುದ್ಧಿ-ಯನ್ನು
ಬುದ್ಧಿ-ಯಲ್ಲಿ
ಬುದ್ಧಿ-ಯಲ್ಲಿದೆ
ಬುದ್ಧಿ-ಯ-ವ-ರಾಗಿರ
ಬುದ್ಧಿ-ಯಿಂದ
ಬುದ್ಧಿ-ಯಿಲ್ಲದೆ
ಬುದ್ಧಿಯು
ಬುದ್ಧಿ-ಯುಳ್ಳ
ಬುದ್ಧಿ-ಯುಳ್ಳ-ವರು
ಬುದ್ಧಿಯೂ
ಬುದ್ಧಿ-ಯೆಂಬ
ಬುದ್ಧಿಯೇ
ಬುದ್ಧಿ-ವಂತ-ನಾಗ-ಲಾರದು
ಬುದ್ಧಿ-ವಂತ-ನಾದ
ಬುದ್ಧಿ-ವಂತ-ರಾಗಿ
ಬುದ್ಧಿ-ವಂತರು
ಬುದ್ಧಿ-ವಂತರೆ
ಬುದ್ಧಿ-ವಂತ-ಳಾದ
ಬುದ್ಧಿ-ವಂತಿಕೆಯ
ಬುದ್ಧಿ-ವಾದ
ಬುದ್ಧಿ-ಶಕ್ತಿ-ಯನ್ನು
ಬುದ್ಧಿ-ಶಕ್ತಿ-ಯಲ್ಲಿ
ಬುದ್ಧಿ-ಶಕ್ತಿ-ಯಲ್ಲಿತ್ತೇನು
ಬುದ್ಧಿ-ಶಕ್ತಿಯು
ಬುದ್ಧಿ-ಶಾಲಿ
ಬೂಟ್ಟಿಗೆ
ಬೂಟ್ಸಿಗೆ
ಬೂದಿ
ಬೂದು
ಬೃಹತ್
ಬೃಹದಾ-ಕಾರ
ಬೃಹದಾ-ಕಾರದ
ಬೃಹದಾ-ಕಾರ-ವನ್ನು
ಬೃಹದಾ-ಕಾರ-ವಾದದ್ದನ್ನು
ಬೆಂಕಿ
ಬೆಂಕಿಗೆ
ಬೆಂಕಿಯ
ಬೆಂಕಿ-ಯನ್ನು
ಬೆಂಕಿ-ಯಲ್ಲಿ
ಬೆಂಕಿ-ಯಾಗಿದ್ದಿತೋ
ಬೆಂಕಿ-ಯಿಂದ
ಬೆಂಕಿ-ಯಿದೆ
ಬೆಂಕಿಯು
ಬೆಂಬಲಕ್ಕಿದೆ
ಬೆಕ್ಕಿನ
ಬೆಕ್ಕು
ಬೆಕ್ಕು-ಗಳು
ಬೆಟ್ಟ-ಗಳು
ಬೆಟ್ಟ-ಗಳೇ-ಳು-ವುವು
ಬೆಟ್ಟದ
ಬೆಟ್ಟ-ದಂತೆ
ಬೆಟ್ಟ-ದಿಂದು
ಬೆಟ್ಟ-ವನ್ನು
ಬೆಟ್ಟವು
ಬೆಣ್ಣೆ-ಯನ್ನು
ಬೆತ್ತ-ವಿಲ್ಲದ
ಬೆನ್ನ
ಬೆನ್ನಟ್ಟಿ
ಬೆನ್ನಿನ
ಬೆನ್ನು
ಬೆನ್ನು-ಮೂಳೆ
ಬೆನ್ನು-ಮೂಳೆ-ಗಳ
ಬೆನ್ನು-ಮೂಳೆಯ
ಬೆನ್ನು-ಮೂಳೆ-ಯಲ್ಲಿ
ಬೆನ್ನು-ಮೂಳೆಯು
ಬೆನ್ನೆಲುಬಿನ
ಬೆನ್ನೆಲುಬು
ಬೆರಗಾಗ-ಕೂಡದು
ಬೆರ-ಳನ್ನು
ಬೆರಳಿ-ನಲ್ಲಿ
ಬೆರಳು
ಬೆರಸಿ
ಬೆರಸುತ್ತಾರೆಯೊ
ಬೆರೆ-ತರೆ
ಬೆರೆತಿ-ರ-ಬಹುದು
ಬೆರೆತಿ-ರು-ವುದು
ಬೆರೆತು
ಬೆರೆತು-ಹೋಗಿದೆ
ಬೆರೆತೇ
ಬೆರೆಯ-ಕೂಡದು
ಬೆರೆಯು-ವಂತೆ
ಬೆರೆ-ಯು-ವು-ದಿಲ್ಲ
ಬೆರೆಯು-ವುದು
ಬೆರೆ-ಸಿದ
ಬೆರೆಸುತ್ತೀರಿ
ಬೆರೆ-ಸು-ವುದು
ಬೆಲೆ
ಬೆಲೆ-ಯಿಲ್ಲ
ಬೆಲೆಯೇ
ಬೆಳ
ಬೆಳ-ಕನ್ನು
ಬೆಳ-ಕನ್ನೂ
ಬೆಳ-ಕಾಗಿ
ಬೆಳ-ಕಾ-ಗು-ವುದು
ಬೆಳ-ಕಿಗೆ
ಬೆಳ-ಕಿದೆ
ಬೆಳ-ಕಿನ
ಬೆಳ-ಕಿ-ನಂತೆ
ಬೆಳ-ಕಿ-ನಲ್ಲಿ
ಬೆಳ-ಕಿ-ನಿಂದ
ಬೆಳಕು
ಬೆಳಕೂ
ಬೆಳಕೇ
ಬೆಳ-ಗದು
ಬೆಳ-ಗ-ಲಾರ
ಬೆಳ-ಗಲು
ಬೆಳಗಿ
ಬೆಳ-ಗಿ-ದಾಗ
ಬೆಳ-ಗುತ್ತಾನೆ
ಬೆಳ-ಗುತ್ತಿದೆ
ಬೆಳ-ಗುತ್ತಿದ್ದರೆ
ಬೆಳ-ಗುತ್ತಿ-ರುವ
ಬೆಳ-ಗುತ್ತಿ-ರುವನು
ಬೆಳ-ಗುತ್ತಿ-ರುವುದು
ಬೆಳ-ಗುವ
ಬೆಳ-ಗುವನು
ಬೆಳ-ಗು-ವುದು
ಬೆಳ-ಗು-ವುದೋ
ಬೆಳಗ್ಗೆ
ಬೆಳ-ದಿದ್ದನೆಂದೂ
ಬೆಳದೇ
ಬೆಳ-ವಣಿಗೆ
ಬೆಳ-ವಣಿಗೆಗೂ
ಬೆಳ-ವಣಿಗೆಗೆ
ಬೆಳ-ವಣಿಗೆಯ
ಬೆಳ-ವಣಿಗೆ-ಯನ್ನು
ಬೆಳ-ವಣಿಗೆ-ಯಲ್ಲಿ
ಬೆಳ-ವಣಿಗೆ-ಯಲ್ಲೆಲ್ಲಾ
ಬೆಳ-ವಣಿಗೆಯು
ಬೆಳ-ವಣಿಗೆಯೂ
ಬೆಳ-ಸಿ-ಕೊಂಡು
ಬೆಳಿಗ್ಗೆ
ಬೆಳಿದಿ
ಬೆಳಿ-ದಿರು
ಬೆಳಿ-ಸಿರು-ವೆವು
ಬೆಳೆ
ಬೆಳೆದ
ಬೆಳೆ-ದಂತೆ
ಬೆಳೆ-ದರೂ
ಬೆಳೆ-ದ-ವನು
ಬೆಳೆ-ದಷ್ಟೂ
ಬೆಳೆ-ದಿರು-ವಂತೆ
ಬೆಳೆ-ದಿ-ರುವನು
ಬೆಳೆ-ದಿ-ರು-ವರು
ಬೆಳೆ-ದಿರು-ವೆವು
ಬೆಳೆದು
ಬೆಳೆ-ದು-ಕೊಂಡು
ಬೆಳೆ-ದುವು
ಬೆಳೆ-ಯದೇ
ಬೆಳೆ-ಯ-ಬಲ್ಲ
ಬೆಳೆ-ಯ-ಬೇಕು-ಎನ್ನುತ್ತದೆ
ಬೆಳೆ-ಯ-ಬೇಕೆಂಬು
ಬೆಳೆ-ಯ-ಬೇಕೆಂಬುದು
ಬೆಳೆ-ಯ-ಲಾರದು
ಬೆಳೆ-ಯಿತು
ಬೆಳೆ-ಯುತ್ತದೆ
ಬೆಳೆ-ಯುತ್ತಾ
ಬೆಳೆ-ಯುತ್ತಿತ್ತು
ಬೆಳೆ-ಯುತ್ತಿ-ರುವಾಗ
ಬೆಳೆ-ಯುತ್ತೇವೆ
ಬೆಳೆ-ಯುವ
ಬೆಳೆ-ಯು-ವಂತೆ
ಬೆಳೆ-ಯುವಂಥದ್ದು
ಬೆಳೆ-ಯು-ವನು
ಬೆಳೆ-ಯು-ವಷ್ಟು
ಬೆಳೆ-ಯು-ವು-ದಿಲ್ಲ
ಬೆಳೆ-ಯುವುದು
ಬೆಳೆ-ವಣಿಗೆಗೆ
ಬೆಳೆ-ಸ-ಬೇಕು
ಬೆಳೆ-ಸಿ-ಕೊಂಡಾಗ
ಬೆಳೆ-ಸಿ-ದನು
ಬೆಳ್ಳಿ
ಬೆವರು
ಬೇಕಾ
ಬೇಕಾ-ಗ-ಬಹುದು
ಬೇಕಾ-ಗ-ಬಹು-ದು-ಅ-ನಂತರ
ಬೇಕಾ-ಗಲೀ
ಬೇಕಾಗಿ
ಬೇಕಾ-ಗಿತ್ತು
ಬೇಕಾ-ಗಿದೆ
ಬೇಕಾ-ಗಿದ್ದರೆ
ಬೇಕಾ-ಗಿದ್ದಾರೆ
ಬೇಕಾ-ಗಿದ್ದುವು
ಬೇಕಾ-ಗಿ-ರ-ಲಿಲ್ಲ
ಬೇಕಾ-ಗಿ-ರುವ
ಬೇಕಾ-ಗಿ-ರುವುದು
ಬೇಕಾ-ಗಿ-ರುವು-ದೆಂದು
ಬೇಕಾ-ಗಿಲ್ಲ
ಬೇಕಾ-ಗಿವೆ
ಬೇಕಾಗು
ಬೇಕಾ-ಗುತ್ತದೆ
ಬೇಕಾ-ಗುತ್ತವೆ
ಬೇಕಾ-ಗುವ
ಬೇಕಾ-ಗು-ವಷ್ಟು
ಬೇಕಾ-ಗು-ವು-ದ-ರಿಂದ
ಬೇಕಾ-ಗು-ವು-ದಿಲ್ಲ
ಬೇಕಾ-ಗು-ವುದು
ಬೇಕಾ-ಗು-ವುದೇ
ಬೇಕಾ-ಗು-ವುದೋ
ಬೇಕಾ-ಗು-ವುವು
ಬೇಕಾದ
ಬೇಕಾ-ದರೂ
ಬೇಕಾ-ದರೆ
ಬೇಕಾ-ದಷ್ಟು
ಬೇಕಾ-ದಷ್ಟೂ
ಬೇಕಾ-ದು-ದನ್ನು
ಬೇಕಾ-ದು-ದನ್ನೆಲ್ಲ
ಬೇಕಾ-ದುದಿಷ್ಟೆ
ಬೇಕಾ-ಯಿತು
ಬೇಕಿಲ್ಲ
ಬೇಕು
ಬೇಕೆ
ಬೇಕೆಂತಲೋ
ಬೇಕೆಂದಿ-ರುವ
ಬೇಕೆಂದು
ಬೇಕೆಂದೂ
ಬೇಕೆಂಬ
ಬೇಕೆಂಬುದು
ಬೇಕೆನ್ನುವ-ವ-ನಿಗೆ
ಬೇಕೊ
ಬೇಕೋ
ಬೇಗ
ಬೇಗನೆ
ಬೇಗ-ವಾಗಿಯೊ
ಬೇಗ-ವಾಗಿಯೋ
ಬೇಗ-ಸೋ-ಲನ್ನು
ಬೇಜಾ-ರನ್ನು
ಬೇಟೆ
ಬೇಟೆಗೆ
ಬೇಟೆ-ಯವ-ರಿಂದ
ಬೇಟೆ-ಯಾಡ-ಬಯ-ಸು-ವಂತೆ
ಬೇಡ
ಬೇಡ-ಬಹುದು
ಬೇಡ-ಬಾರ-ದೆಂದು
ಬೇಡ-ಬೇಕು-ದೇವ-ತೆ-ಗಳೊ
ಬೇಡ-ವಾ-ಗಲೀ
ಬೇಡ-ವೆಂತಲೋ
ಬೇಡ-ವೆನ್ನು-ವುದು
ಬೇಡವೊ
ಬೇಡಿ
ಬೇಡಿ-ಕೊಂಡನು
ಬೇಡಿ-ಕೊಳ್ಳು-ವನು
ಬೇಡಿ-ದರೆ
ಬೇಡುತ್ತಾ
ಬೇಡುತ್ತಿ-ರುವನು
ಬೇಡುತ್ತೇನೆ
ಬೇಡು-ವು-ದನ್ನು
ಬೇಡು-ವು-ದಲ್ಲ
ಬೇಡು-ವುದು
ಬೇಡು-ವೆವು
ಬೇಯಿಸಿದ
ಬೇರಾವ
ಬೇರಾ-ವು-ದನ್ನೂ
ಬೇರಾವು-ದನ್ನೊ
ಬೇರಾ-ವು-ದ-ರಿಂದಲೂ
ಬೇರಾ-ವುದೂ
ಬೇರಿನ
ಬೇರು
ಬೇರು-ಬಿಟ್ಟಿ
ಬೇರೂರಿದೆ
ಬೇರೆ
ಬೇರೆ-ಬೇರೆ
ಬೇರೆ-ಬೇರೆಯ
ಬೇರೆ-ಬೇರೆ-ಯಾ-ಗಿದೆ
ಬೇರೆ-ಬೇರೆ-ಯಾಗಿ-ರು-ವು-ದ-ರಿಂದ
ಬೇರೆಯ
ಬೇರೆ-ಯಲ್ಲ
ಬೇರೆ-ಯಲ್ಲ-ವೆಂಬ
ಬೇರೆ-ಯ-ವ-ನಂತೆ
ಬೇರೆ-ಯ-ವ-ರಲ್ಲ
ಬೇರೆ-ಯಾಗಿ
ಬೇರೆ-ಯಾಗಿ-ಡು-ವು-ದಕ್ಕೆ
ಬೇರೆ-ಯಾ-ಗಿದೆ
ಬೇರೆ-ಯಾಗಿ-ದೆಯೆ
ಬೇರೆ-ಯಾ-ಗಿಯೇ
ಬೇರೆ-ಯಾಗಿ-ರ-ಲಿಲ್ಲ
ಬೇರೆ-ಯಾಗಿ-ರು-ವಂತೆ
ಬೇರೆ-ಯಾಗಿ-ರುವನು
ಬೇರೆ-ಯಾಗಿ-ರು-ವುದು
ಬೇರೆ-ಯಾಗಿ-ರು-ವೆನೊ
ಬೇರೆ-ಯಾಗಿ-ರುವೆ-ವೆಂದು
ಬೇರೆ-ಯಾಗುವ
ಬೇರೆ-ಯಾಗು-ವುದು
ಬೇರೆ-ಯಾಗು-ವೆವು
ಬೇರೆ-ಯಾದ
ಬೇರೆ-ಯಾ-ದಂತೆ
ಬೇರೆ-ಯಾ-ದರೂ
ಬೇರೆ-ಯಾ-ದುದು
ಬೇರೊಂದರ
ಬೇರೊಂದಿಲ್ಲ
ಬೇರೊಂದು
ಬೇರೊಬ್ಬ
ಬೇರೊಬ್ಬ-ನಿಂದ
ಬೇರೊಬ್ಬ-ನಿಗೆ
ಬೇರ್ಪಡಿ-ಸದೆ
ಬೇರ್ಪಡಿ-ಸ-ಬಹುದು
ಬೇರ್ಪಡಿ-ಸ-ಲಾಗು-ವು-ದಿಲ್ಲ
ಬೇರ್ಪಡಿ-ಸ-ಲಾಗು-ವುದೇ
ಬೇರ್ಪಡಿ-ಸಲು
ಬೇರ್ಪಡಿಸಿ
ಬೇರ್ಪಡಿ-ಸಿದ
ಬೇರ್ಪಡಿ-ಸಿ-ದಂತೆ
ಬೇರ್ಪಡಿ-ಸುತ್ತೇವೆ
ಬೇರ್ಪಡಿ-ಸುವ
ಬೇರ್ಪಡಿ-ಸು-ವಷ್ಟು
ಬೇರ್ಪಡಿ-ಸುವು-ದಕ್ಕಾಗಿ
ಬೇರ್ಪಡಿ-ಸುವು-ದಾ-ವುದ
ಬೇಲಿ
ಬೇಲಿ-ಯನ್ನೆಲ್ಲಾ
ಬೇಲಿ-ಯಾಚೆ
ಬೇಲಿ-ಯೊ-ಳಗೆ
ಬೇಸರ
ಬೇಸರ-ವಾ-ಗಿತ್ತು
ಬೇಸಿಗೆ-ಯಲ್ಲಿ
ಬೈಬ-ಲನ್ನು
ಬೈಬಲಿ-ನಲ್ಲಿ
ಬೈಬಲ್
ಬೈಬಲ್ಲನ್ನು
ಬೈಬಲ್ಲಿನ
ಬೈಬಲ್ಲು
ಬೈಯ್ಯದೆ
ಬೊಂತೆ
ಬೊಗಸೆ-ಮಣ್ಣಿನಿಂದಾದು-ದಲ್ಲ
ಬೋಧಕ
ಬೋಧ-ಕನು
ಬೋಧ-ಕರ
ಬೋಧಕ-ರಲ್ಲಿಯೂ
ಬೋಧ-ಕರು
ಬೋಧಕ-ರೆಲ್ಲರೂ
ಬೋಧನೆ
ಬೋಧನೆ-ಗ-ಳನ್ನು
ಬೋಧನೆ-ಗಳು
ಬೋಧನೆ-ಗಳೆಲ್ಲಾ
ಬೋಧ-ನೆಯ
ಬೋಧನೆ-ಯನ್ನು
ಬೋಧನೆ-ಯಿಂದ
ಬೋಧನೆ-ಯಿಂದಲೂ
ಬೋಧನೆ-ಯಿಷ್ಟೆ
ಬೋಧ-ನೆಯೇ
ಬೋಧಿ
ಬೋಧಿ-ವೃಕ್ಷದ
ಬೋಧಿಸ
ಬೋಧಿ-ಸ-ತೊಡಗಿ-ದರು
ಬೋಧಿ-ಸ-ಬಾ-ರದು
ಬೋಧಿ-ಸ-ಬೇಕು
ಬೋಧಿ-ಸ-ಬೇಕೆಂದಿ
ಬೋಧಿ-ಸ-ಬೇಕೆಂದಿ-ರು-ವುದೊ
ಬೋಧಿ-ಸ-ಬೇಡಿ
ಬೋಧಿ-ಸ-ಲಾ-ಗಿದೆ
ಬೋಧಿ-ಸ-ಲಾರಿರಿ
ಬೋಧಿ-ಸ-ಲಿಲ್ಲ
ಬೋಧಿ-ಸಲು
ಬೋಧಿಸಿ
ಬೋಧಿ-ಸಿತು
ಬೋಧಿ-ಸಿದ
ಬೋಧಿ-ಸಿ-ದಂತೆ
ಬೋಧಿ-ಸಿ-ದನು
ಬೋಧಿ-ಸಿ-ದರು
ಬೋಧಿ-ಸಿ-ದರೆ
ಬೋಧಿ-ಸಿ-ದರೋ
ಬೋಧಿ-ಸಿ-ದ-ವರು
ಬೋಧಿ-ಸಿ-ದಿ-ರೇನು
ಬೋಧಿ-ಸಿ-ದುದೆ
ಬೋಧಿ-ಸಿದ್ದರೆ
ಬೋಧಿ-ಸಿ-ರುವ
ಬೋಧಿ-ಸಿ-ರು-ವರು
ಬೋಧಿ-ಸಿಲ್ಲ
ಬೋಧಿಸು
ಬೋಧಿ-ಸುತ್ತದೆ
ಬೋಧಿ-ಸುತ್ತವೆ
ಬೋಧಿ-ಸುತ್ತಾರೆ
ಬೋಧಿ-ಸುತ್ತಿತ್ತು
ಬೋಧಿ-ಸುತ್ತಿದ್ದರೊ
ಬೋಧಿ-ಸುತ್ತಿ-ರುವ
ಬೋಧಿ-ಸುತ್ತಿ-ರುವನು
ಬೋಧಿ-ಸುತ್ತಿ-ರುವಾಗ
ಬೋಧಿ-ಸುತ್ತೀರಿ
ಬೋಧಿ-ಸುತ್ತೇನೆ
ಬೋಧಿ-ಸುತ್ತೇನೆಂದನು
ಬೋಧಿ-ಸುವ
ಬೋಧಿ-ಸು-ವನು
ಬೋಧಿ-ಸು-ವರು
ಬೋಧಿ-ಸು-ವ-ವರ
ಬೋಧಿ-ಸು-ವಾಗ
ಬೋಧಿ-ಸು-ವು-ದಕ್ಕಾಗಿ
ಬೋಧಿ-ಸು-ವು-ದಕ್ಕಾಗಿಯೇ
ಬೋಧಿ-ಸು-ವು-ದಕ್ಕೆ
ಬೋಧಿ-ಸು-ವು-ದ-ರಿಂದ
ಬೋಧಿ-ಸು-ವು-ದಾ-ಗಿದೆ
ಬೋಧಿ-ಸು-ವು-ದಿಲ್ಲ
ಬೋಧಿ-ಸು-ವುದು
ಬೋಧಿ-ಸು-ವುದೇ
ಬೋಧಿ-ಸು-ವುದೇನು
ಬೋಧಿ-ಸು-ವು-ದೊಂದು
ಬೋನಿನ
ಬೌದ್ಧ
ಬೌದ್ಧಗ್ರಂಥವು
ಬೌದ್ಧ-ದಾರ್ಶ-ನಿಕ
ಬೌದ್ಧ-ದಾರ್ಶ-ನಿಕರ
ಬೌದ್ಧ-ಧರ್ಮದ
ಬೌದ್ಧ-ಧರ್ಮ-ದಲ್ಲಿ
ಬೌದ್ಧ-ಧರ್ಮವು
ಬೌದ್ಧ-ಮ-ತಕ್ಕೆ
ಬೌದ್ಧರ
ಬೌದ್ಧ-ರನ್ನಾಗಿ
ಬೌದ್ಧ-ರನ್ನು
ಬೌದ್ಧ-ರಲ್ಲಿ
ಬೌದ್ಧ-ರಾ-ದರು
ಬೌದ್ಧ-ರಾ-ದರೋ
ಬೌದ್ಧ-ರಿಗೂ
ಬೌದ್ಧ-ರಿಗೆ
ಬೌದ್ಧ-ರಿ-ರು-ವರು
ಬೌದ್ಧರು
ಬೌದ್ಧರೂ
ಬೌದ್ಧರೆ
ಬೌದ್ಧಿಕ
ಬೌದ್ಧಿಕತೆ
ಬೌದ್ಧಿಕ-ವಾಗಿ
ಬೌದ್ಧಿಕ-ಸುಖ-ವಾಗಿ
ಬ್ದಾರಿ-ಯನ್ನೆಲ್ಲಾ
ಬ್ಯಾಟರಿ-ಯಂತೆ
ಬ್ಯಾಬಿಲೋ-ನಿಯ-ರಲ್ಲಿಯೂ
ಬ್ಯಾಬಿಲೋ-ನಿಯಾ
ಬ್ರಮಿಸು-ವೆನು
ಬ್ರಹ್ಮ
ಬ್ರಹ್ಮಕ್ಕೆ
ಬ್ರಹ್ಮ-ಚರ್ಯ
ಬ್ರಹ್ಮ-ಚರ್ಯದ
ಬ್ರಹ್ಮ-ಚರ್ಯ-ದಲ್ಲಿ
ಬ್ರಹ್ಮ-ಚರ್ಯ-ದಿಂದ
ಬ್ರಹ್ಮ-ಚರ್ಯಪ್ರತಿಷ್ಠಾಯಾಂ
ಬ್ರಹ್ಮ-ಚರ್ಯ-ವನ್ನು
ಬ್ರಹ್ಮ-ಚರ್ಯ-ವಿಲ್ಲದೆ
ಬ್ರಹ್ಮ-ಚರ್ಯವ್ರತ-ವನ್ನು
ಬ್ರಹ್ಮ-ಚಾರಿ-ಗಳಾ-ಗಿದ್ದರು
ಬ್ರಹ್ಮ-ಚಾರಿ-ಯಾಗಿ-ರ-ಬೇಕು
ಬ್ರಹ್ಮಜ್ಞಾನ-ವನ್ನು
ಬ್ರಹ್ಮದ
ಬ್ರಹ್ಮ-ದಲ್ಲಿ
ಬ್ರಹ್ಮನ
ಬ್ರಹ್ಮ-ನನ್ನು
ಬ್ರಹ್ಮ-ನಲ್ಲಿ
ಬ್ರಹ್ಮ-ನಲ್ಲಿಗೆ
ಬ್ರಹ್ಮ-ನಲ್ಲಿತ್ತು
ಬ್ರಹ್ಮ-ನಿಂದ
ಬ್ರಹ್ಮ-ನಿಗೆ
ಬ್ರಹ್ಮನು
ಬ್ರಹ್ಮ-ನೆಂದು
ಬ್ರಹ್ಮ-ನೆ-ಡೆಗೆ
ಬ್ರಹ್ಮನೇ
ಬ್ರಹ್ಮ-ನೊಂದಿಗೆ
ಬ್ರಹ್ಮ-ಭಾ-ವನೆ
ಬ್ರಹ್ಮ-ಮಯ-ವಾಗಿದೆ
ಬ್ರಹ್ಮ-ಮಯ-ವೆಂದು
ಬ್ರಹ್ಮ-ಲೋಕ
ಬ್ರಹ್ಮ-ಲೋಕಕ್ಕೂ
ಬ್ರಹ್ಮ-ಲೋ-ಕಕ್ಕೆ
ಬ್ರಹ್ಮ-ಲೋಕ-ದ-ವರೆ-ಗಿನ
ಬ್ರಹ್ಮ-ವನ್ನು
ಬ್ರಹ್ಮವು
ಬ್ರಹ್ಮವೆ
ಬ್ರಹ್ಮ-ವೆಂದೇನೋ
ಬ್ರಹ್ಮವೇ
ಬ್ರಹ್ಮ-ವೊಂದೇ
ಬ್ರಹ್ಮಾಂಡ
ಬ್ರಹ್ಮಾಂಡಕ್ಕೆ
ಬ್ರಹ್ಮಾಂಡದ
ಬ್ರಹ್ಮಾಂಡ-ದಲ್ಲಿ
ಬ್ರಹ್ಮಾಂಡ-ವನ್ನು
ಬ್ರಹ್ಮಾಂಡವೂ
ಬ್ರಹ್ಮಾಂಡ-ವೆಲ್ಲ
ಬ್ರಹ್ಮಾಂಡ-ವೆಲ್ಲಾ
ಬ್ರಹ್ಮೈಕ್ಯ-ವೆಂದು
ಬ್ರಾಹ್ಮಣ
ಬ್ರಾಹ್ಮಣ-ಗಳಲ್ಲಿ
ಬ್ರಾಹ್ಮಣ-ನಲ್ಲದೆ
ಬ್ರಾಹ್ಮಣ-ಭಾಗ-ದಲ್ಲಿ
ಬ್ರಾಹ್ಮಣ-ರಿಗೆ
ಬ್ರಾಹ್ಮಣರು
ಬ್ರಾಹ್ಮಣರೇ
ಭಂಗ
ಭಂಗ-ವನ್ನು
ಭಂಗ-ವುಂಟಾ-ಗು-ವುದು
ಭಂಗಿ-ಯಲ್ಲಿ
ಭಂಗಿಸಿ
ಭಂಡಾರ
ಭಂಡಾ-ರಕ್ಕೆ
ಭಂಡಾರ-ವನ್ನು
ಭಕ್ತನ
ಭಕ್ತ-ನಿಗೆ
ಭಕ್ತ-ನಿ-ರುವನು
ಭಕ್ತರ
ಭಕ್ತ-ರಿ-ಗಿಂತಲೂ
ಭಕ್ತ-ರಿಗೆ
ಭಕ್ತರು
ಭಕ್ತ-ರೆಂದು
ಭಕ್ತಿ
ಭಕ್ತಿಗೂ
ಭಕ್ತಿಜ್ಞಾನ-ಯೋಗ
ಭಕ್ತಿ-ಪೂರ್ವ-ಕ-ವಾಗಿ
ಭಕ್ತಿಯ
ಭಕ್ತಿ-ಯನ್ನು
ಭಕ್ತಿ-ಯಿಂದ
ಭಕ್ತಿ-ಯುಕ್ತ-ವಾಗಿ-ರ-ಬೇಕು
ಭಕ್ತಿ-ಯೋಗ
ಭಕ್ತಿ-ಯೋಗ-ದಿಂದಾ-ಗಲಿ
ಭಕ್ತಿ-ಯೋಗ-ವಿದೆ
ಭಕ್ತಿ-ಯೋಗಿ
ಭಕ್ಷಿ-ಸುವ
ಭಗ
ಭಗ-ವಂತ
ಭಗ-ವಂತನ
ಭಗ-ವಂತ-ನನ್ನು
ಭಗ-ವಂತ-ನಲ್ಲಿ
ಭಗ-ವಂತ-ನಿಂದ
ಭಗ-ವಂತ-ನಿ-ಗಾಗಿ
ಭಗ-ವಂತ-ನಿಗೆ
ಭಗ-ವಂತ-ನಿ-ರುವನು
ಭಗ-ವಂತನು
ಭಗ-ವಂತನೆ
ಭಗ-ವಂತ-ನೆಂದೇ
ಭಗ-ವಂತ-ನೆಂಬ
ಭಗ-ವಂತನೇ
ಭಗ-ವಂತ-ನೊ-ಡನೆ
ಭಗ-ವಂತ-ನೊಬ್ಬನೇ
ಭಗವತ್ಸಾಕ್ಷಾತ್ಕಾರ
ಭಗವತ್ಸಾಕ್ಷಾತ್ಕಾ-ರಕ್ಕೆ
ಭಗವತ್ಸ್ವ-ರೂಪ-ವನ್ನು
ಭಗ-ವದನ್ವೇಷಣೆ-ಗಾಗಿ
ಭಗವ-ದಾರಾಧನೆ
ಭಗ-ವದ್ಗೀತೆ
ಭಗ-ವದ್ಗೀತೆಯ
ಭಗ-ವದ್ಭಾವ-ನೆ-ಗ-ಳನ್ನೂ
ಭಗ-ವದ್ಭಾವ-ನೆಯ
ಭಗ-ವನ್ಮಯ-ವಾಗಿ
ಭಗ-ವಾನ್
ಭಟ್ಟಿ
ಭದ್ರ-ವಾಗಿ
ಭದ್ರವಾದ
ಭನೆ
ಭಯ
ಭಯಂಕರ
ಭಯಂಕರ-ವಾಗಿ
ಭಯಂಕರ-ವಾಗಿದೆ
ಭಯಂಕರ-ವಾಗಿ-ರ-ಬೇಕು
ಭಯಂಕರ-ವಾ-ಗಿ-ರುವುದು
ಭಯಂಕರ-ವಾದ
ಭಯಂಕರ-ವಾದುದು
ಭಯಂಕರ-ವಾದುದೇ
ಭಯ-ದಿಂದ
ಭಯ-ಭಕ್ತಿ-ಗ-ಳನ್ನು
ಭಯ-ವಿಲ್ಲ
ಭಯವು
ಭಯವೂ
ಭಯವೇ
ಭಯ-ವೇಕೆ
ಭಯಾನಕ
ಭಯಾನ-ಕ-ವಾ-ಗಿ-ರುವುದು
ಭಯಾನ-ಕ-ವಾದ
ಭಯಾನ-ಕ-ವಾದುದೂ
ಭರತ
ಭರತ-ಖಂಡ
ಭರತ-ಖಂಡದ
ಭರತ-ಖಂಡ-ದಲ್ಲಿ
ಭರತ-ಖಂಡ-ದಲ್ಲಿದ್ದ
ಭರತ-ಖಂಡ-ದಲ್ಲಿ-ಯಾ-ದರೋ
ಭರತ-ಖಂಡ-ದಲ್ಲಿ-ರುವ
ಭರತ-ಖಂಡ-ದಲ್ಲೂ
ಭರತ-ಖಂಡ-ದಲ್ಲೇ
ಭರತ-ಖಂಡ-ದಷ್ಟು
ಭರತ-ಖಂಡ-ವನ್ನು
ಭರತ-ಖಂಡವು
ಭರ-ದಿಂದ
ಭರವಸೆ
ಭರವಸೆ-ಇ-ವೆಲ್ಲ
ಭರವಸೆ-ಗಳೆಲ್ಲ
ಭರವಸೆಯ
ಭರವಸೆಯೆ
ಭರವಸೆಯೇ
ಭರ್ತಿ
ಭವದ
ಭವ-ನವು
ಭವಪ್ರತ್ಯಯೋ
ಭವ-ರೋ-ಗಕ್ಕೆ
ಭವ-ವನ್ನೂ
ಭವ-ಸಾ-ಗರ-ದಲ್ಲಿ
ಭವಿ-ಷತ್ತು-ಗ-ಳನ್ನು
ಭವಿಷ್ಯ
ಭವಿಷ್ಯಕ್ಕೆ
ಭವಿಷ್ಯ-ಗ-ಳನ್ನು
ಭವಿಷ್ಯತ್
ಭವಿಷ್ಯತ್ತಿನ
ಭವಿಷ್ಯತ್ತು
ಭವಿಷ್ಯತ್ತು-ಗ-ಳನ್ನು
ಭವಿಷ್ಯತ್ತು-ಗಳು
ಭವಿಷ್ಯದ
ಭವಿಷ್ಯ-ದಲ್ಲಿ
ಭವಿಷ್ಯ-ದಲ್ಲೂ
ಭವಿಷ್ಯ-ಧರ್ಮ-ಗಳು
ಭವಿಷ್ಯ-ವನ್ನಾ-ದರೂ
ಭವಿಷ್ಯ-ವನ್ನು
ಭವಿಷ್ಯವು
ಭವಿ-ಸಿ-ರುವನು
ಭವಿ-ಸು-ವೆನು
ಭವ್ಯ
ಭವ್ಯ-ತೆ-ಗಳು
ಭವ್ಯ-ತೆಯ
ಭವ್ಯ-ಭಾವ-ನೆ-ಗಳು
ಭವ್ಯ-ವಾಗಿ
ಭವ್ಯ-ವಾಗಿ-ರುವ
ಭವ್ಯ-ವಾಗಿವೆ
ಭವ್ಯ-ವಾದ
ಭವ್ಯ-ವಾ-ದಷ್ಟೂ
ಭವ್ಯ-ವಾದು-ದನ್ನು
ಭವ್ಯ-ವಾದು-ದೆಲ್ಲ
ಭವ್ಯವೂ
ಭಸ್ಮೀ-ಭೂತ-ಮಾಡುವ
ಭಾಗ
ಭಾಗಕ್ಕಿಂತ
ಭಾಗಕ್ಕೂ
ಭಾಗಕ್ಕೆ
ಭಾಗ-ಗಳ
ಭಾಗ-ಗ-ಳನ್ನು
ಭಾಗ-ಗ-ಳನ್ನೂ
ಭಾಗ-ಗಳಲ್ಲಿ
ಭಾಗ-ಗಳಾ-ವುವೂ
ಭಾಗ-ಗಳಿಗೂ
ಭಾಗ-ಗಳಿಗೆ
ಭಾಗ-ಗಳಿ-ರ-ಬೇಕಾ-ಗುತ್ತದೆ
ಭಾಗ-ಗಳಿಲ್ಲ-ದಿದ್ದರೆ
ಭಾಗ-ಗಳಿವೆ
ಭಾಗ-ಗಳು
ಭಾಗದ
ಭಾಗ-ದಲ್ಲಿ
ಭಾಗ-ದ-ವ-ರೆಲ್ಲರೂ
ಭಾಗ-ದಷ್ಟು
ಭಾಗ-ದಿಂದ
ಭಾಗ-ವನ್ನಾಗಿ
ಭಾಗ-ವನ್ನು
ಭಾಗ-ವನ್ನೂ
ಭಾಗ-ವನ್ನೇ
ಭಾಗ-ವಹಿ-ಸುವ
ಭಾಗ-ವಾಗ-ಬಹುದು
ಭಾಗ-ವಾಗಿ
ಭಾಗ-ವಾಗಿದೆ
ಭಾಗ-ವಾ-ಗು-ವುದು
ಭಾಗ-ವಾದ
ಭಾಗ-ವಿದೆ
ಭಾಗ-ವಿ-ರುವ
ಭಾಗ-ವಿ-ರು-ವುದು
ಭಾಗ-ವಿಲ್ಲ-ದುದು
ಭಾಗವು
ಭಾಗವೂ
ಭಾಗವೇ
ಭಾಜ್ಯ
ಭಾಧೆ
ಭಾನು
ಭಾನು-ವಾರ
ಭಾರ
ಭಾರತ
ಭಾರ-ತ-ಖಂಡ-ದಲ್ಲಿ
ಭಾರ-ತದ
ಭಾರ-ತ-ದಲ್ಲಿ
ಭಾರ-ತ-ದಲ್ಲಿದ್ದಾಗ
ಭಾರ-ತ-ದಲ್ಲಿ-ರುವ
ಭಾರ-ತೀಯ
ಭಾರ-ತೀಯರ
ಭಾರ-ತೀಯ-ರಿ-ಗಿಂತಲೂ
ಭಾರ-ತೀಯ-ರಿಗೆ
ಭಾರ-ವನ್ನು
ಭಾರ-ವಾಗ-ಬಲ್ಲ
ಭಾರ-ವಾ-ಗಿ-ರುವು-ದೆಲ್ಲ
ಭಾರೀ
ಭಾವ
ಭಾವಕ್ಕೂ
ಭಾವಕ್ಕೆ
ಭಾವ-ಗಳ
ಭಾವ-ಗ-ಳನ್ನು
ಭಾವ-ಗಳನ್ನೆಲ್ಲಾ
ಭಾವ-ಗಳನ್ನೊಳ-ಗೊಂಡ
ಭಾವ-ಗಳಲ್ಲಿದ್ದು
ಭಾವ-ಗಳಿಗೆ
ಭಾವ-ಗಳು
ಭಾವ-ಗಳೆಲ್ಲ-ವನ್ನೂ
ಭಾವ-ಗಳೊಂದಿಗೆ
ಭಾವ-ಚಿತ್ರ
ಭಾವ-ಚಿತ್ರ-ವನ್ನು
ಭಾವ-ಜೀವಿ
ಭಾವ-ಜೀವಿ-ಗಳ
ಭಾವ-ಜೀವಿ-ಗಳಿಗೆ
ಭಾವ-ಜೀವಿ-ಗಳು
ಭಾವ-ಜೀ-ವಿಗೆ
ಭಾವ-ಜೀವಿ-ಯಾದ
ಭಾವ-ಜೀವಿ-ಯಾ-ದರೆ
ಭಾವ-ದವ-ರನ್ನೆಲ್ಲಾ
ಭಾವ-ದಿಂದ
ಭಾವನಾ
ಭಾವ-ನಾ-ತ-ರಂಗ-ಗಳಿವೆ
ಭಾವ-ನಾ-ತಶ್ತಿತ್ತಪ್ರಸಾದ-ನಮ್
ಭಾವ-ನಾ-ಮಯ-ವಾ-ಯಿತು
ಭಾವ-ನಾ-ವಿಕಾಸಕ್ಕೆ
ಭಾವನೆ
ಭಾವ-ನೆ-ಗಳ
ಭಾವ-ನೆ-ಗ-ಳನ್ನು
ಭಾವ-ನೆ-ಗ-ಳನ್ನೂ
ಭಾವ-ನೆ-ಗಳನ್ನೆಲ್ಲಾ
ಭಾವ-ನೆ-ಗಳನ್ನೇ
ಭಾವ-ನೆ-ಗಳಲ್ಲಿ
ಭಾವ-ನೆ-ಗಳಿಂದ
ಭಾವ-ನೆ-ಗಳಿ-ಗಿಂತ
ಭಾವ-ನೆ-ಗಳಿಗೆ
ಭಾವ-ನೆ-ಗಳಿ-ರ-ಬಹುದು
ಭಾವ-ನೆ-ಗಳಿ-ರಲಿ
ಭಾವ-ನೆ-ಗಳು
ಭಾವ-ನೆ-ಗಳುಳ್ಳ
ಭಾವ-ನೆ-ಗಳುಳ್ಳ-ವರು
ಭಾವ-ನೆ-ಗಳೂ
ಭಾವ-ನೆ-ಗ-ಳೆಂದು
ಭಾವ-ನೆ-ಗಳೆಂಬ
ಭಾವ-ನೆ-ಗಳೆಲ್ಲ
ಭಾವ-ನೆ-ಗಳೆಲ್ಲ-ವನ್ನೂ
ಭಾವ-ನೆ-ಗಳೊಂದಿಗೆ
ಭಾವ-ನೆಗೂ
ಭಾವ-ನೆಗೆ
ಭಾವ-ನೆಯ
ಭಾವ-ನೆ-ಯಂತೆ
ಭಾವ-ನೆ-ಯನ್ನು
ಭಾವ-ನೆ-ಯನ್ನೂ
ಭಾವ-ನೆ-ಯಲ್ಲ
ಭಾವ-ನೆ-ಯಲ್ಲಿ
ಭಾವ-ನೆ-ಯಲ್ಲಿಯೂ
ಭಾವ-ನೆ-ಯಲ್ಲೇ
ಭಾವ-ನೆ-ಯಾ-ದರೆ
ಭಾವ-ನೆ-ಯಿಂದ
ಭಾವ-ನೆಯು
ಭಾವ-ನೆಯೂ
ಭಾವ-ನೆಯೆ
ಭಾವ-ನೆ-ಯೆಂದರೆ
ಭಾವ-ನೆ-ಯೆಲ್ಲ
ಭಾವ-ನೆ-ಯೆಲ್ಲಾ
ಭಾವ-ನೆಯೇ
ಭಾವ-ನೆ-ಯೇನೊ
ಭಾವ-ನೆ-ಯೊಂದಿಗೆ
ಭಾವ-ನೆ-ಯೊ-ಡನೆ
ಭಾವ-ನೆ-ಸೃಷ್ಟಿ-ಯಲ್ಲಿ-ರುವ
ಭಾವ-ಮಯ
ಭಾವರ
ಭಾವ-ರ-ಹಿತ
ಭಾವ-ವನ್ನು
ಭಾವ-ವನ್ನೇ
ಭಾವ-ವಲ್ಲದೆ
ಭಾವ-ವಿ-ರು-ವು-ದ-ರಿಂದ
ಭಾವವು
ಭಾವ-ವೆಂಬ
ಭಾವ-ಸತ್ತಾ
ಭಾವ-ಸತ್ತಾ-ವಾದವೂ
ಭಾವ-ಸತ್ತಾ-ವಾದಿ
ಭಾವ-ಸತ್ತಾ-ವಾದಿ-ಗಳ
ಭಾವ-ಸತ್ತಾ-ವಾದಿಗೆ
ಭಾವ-ಸಾ-ಗರ-ವಾ-ಗು-ವುದು
ಭಾವಾನು
ಭಾವಿ
ಭಾವಿ-ರು-ವುದೇ
ಭಾವಿ-ಸದ
ಭಾವಿ-ಸದೆ
ಭಾವಿ-ಸದ್ದೆ
ಭಾವಿ-ಸ-ಬಹುದು
ಭಾವಿ-ಸ-ಬಾ-ರದು
ಭಾವಿ-ಸ-ಬೇಕಾ-ಗು-ವುದು
ಭಾವಿ-ಸ-ಬೇಕು-ಅ-ನಂತರ
ಭಾವಿ-ಸ-ಬೇಕೆಂದು
ಭಾವಿ-ಸ-ಬೇಡಿ
ಭಾವಿ-ಸಯೇ
ಭಾವಿ-ಸಲಾ-ಗಿತ್ತು
ಭಾವಿ-ಸಲಾ-ಗಿದೆ
ಭಾವಿ-ಸಲೂ
ಭಾವಿ-ಸಲೇ-ಬೇಕಾ-ಗುತ್ತದೆ
ಭಾವಿಸಿ
ಭಾವಿ-ಸಿ-ಕೊಂಡಿದ್ದನೋ
ಭಾವಿ-ಸಿ-ಕೊಳ್ಳಿ
ಭಾವಿ-ಸಿತು
ಭಾವಿ-ಸಿದ
ಭಾವಿ-ಸಿ-ದಂತೆ
ಭಾವಿ-ಸಿ-ದರು
ಭಾವಿ-ಸಿ-ದರೂ
ಭಾವಿ-ಸಿ-ದರೆ
ಭಾವಿ-ಸಿ-ದ-ವ-ರಿದ್ದರು
ಭಾವಿ-ಸಿ-ದಾಗ
ಭಾವಿ-ಸಿ-ದಿರಿ
ಭಾವಿ-ಸಿದೆ
ಭಾವಿ-ಸಿ-ದೊ-ಡನೆ
ಭಾವಿ-ಸಿದ್ದನು
ಭಾವಿ-ಸಿದ್ದನೊ
ಭಾವಿ-ಸಿದ್ದರು
ಭಾವಿ-ಸಿದ್ದರೆ
ಭಾವಿ-ಸಿದ್ದಿರಿ
ಭಾವಿ-ಸಿದ್ದಿರೊ
ಭಾವಿ-ಸಿದ್ದೀರಾ
ಭಾವಿ-ಸಿದ್ದೆ
ಭಾವಿ-ಸಿದ್ದೆನೊ
ಭಾವಿ-ಸಿದ್ದೇವೆ
ಭಾವಿ-ಸಿ-ರು-ವರೋ
ಭಾವಿ-ಸಿ-ರುವಿ-ರೇನು
ಭಾವಿ-ಸಿ-ರು-ವೆವು
ಭಾವಿಸು
ಭಾವಿ-ಸುತ್ತಾನೆ
ಭಾವಿ-ಸುತ್ತಿದ್ದ
ಭಾವಿ-ಸುತ್ತಿದ್ದನೊ
ಭಾವಿ-ಸುತ್ತಿದ್ದರು
ಭಾವಿ-ಸುತ್ತಿದ್ದೇವೆಯೋ
ಭಾವಿ-ಸುತ್ತೀರಿ
ಭಾವಿ-ಸುತ್ತೀ-ರೇನು
ಭಾವಿ-ಸುತ್ತೇನೆ
ಭಾವಿ-ಸುತ್ತೇವೆ
ಭಾವಿ-ಸುವ
ಭಾವಿ-ಸು-ವಂತೆ
ಭಾವಿ-ಸು-ವನು
ಭಾವಿ-ಸು-ವನೊ
ಭಾವಿ-ಸು-ವರು
ಭಾವಿ-ಸು-ವರೊ
ಭಾವಿ-ಸು-ವ-ವ-ನಲ್ಲಿಯೂ
ಭಾವಿ-ಸು-ವಿರಾ
ಭಾವಿ-ಸು-ವಿರಿ
ಭಾವಿ-ಸು-ವಿ-ರೇನು
ಭಾವಿ-ಸು-ವು-ದಿಲ್ಲ
ಭಾವಿ-ಸು-ವುದು
ಭಾವಿ-ಸು-ವುದೇ
ಭಾವಿ-ಸು-ವೆವು
ಭಾವಿ-ಸು-ವೆವೊ
ಭಾವಿ-ಸೋಣ
ಭಾವೀ
ಭಾವು-ಕರಾ-ಗಿ-ರುವು
ಭಾವೈಕ್ಯ-ವನ್ನು
ಭಾಷಾಂತರ
ಭಾಷಾಂತರ-ಕಾರ-ರದು
ಭಾಷಾಂತರ-ದಲ್ಲಿ
ಭಾಷಾಜ್ಞಾನವೇ
ಭಾಷಾ-ವೈ-ವಿಧ್ಯ
ಭಾಷೆ
ಭಾಷೆ-ಗಳಲ್ಲಿ
ಭಾಷೆಗೂ
ಭಾಷೆಯ
ಭಾಷೆ-ಯಂತೆ
ಭಾಷೆ-ಯನ್ನು
ಭಾಷೆ-ಯಲ್ಲಿ
ಭಾಷೆ-ಯಲ್ಲಿ-ಡ-ಬಹುದು
ಭಾಷೆ-ಯಲ್ಲಿಯೂ
ಭಾಷೆ-ಯಲ್ಲಿ-ರುವ್ಚ್ಟಛಿಚ್ಠಿಜಿಟ್ಞ
ಭಾಷೆ-ಯಲ್ಲೂ
ಭಾಷೆಯೂ
ಭಾಷ್ಯ
ಭಾಷ್ಯ-ಕಾರರು
ಭಾಷ್ಯ-ಕಾರ-ರೊಬ್ಬರು
ಭಾಷ್ಯ-ದಿಂದ
ಭಾಸ-ವಾಗುತ್ತದೆ
ಭಾಸ-ವಾ-ಗು-ವುದು
ಭಾಸ-ವಾ-ಯಿತು
ಭಾಹ್ಯಾಭ್ಯಂತರ
ಭಿಕಾರಿ-ಗಳು
ಭಿಕ್ಷಾವೃತ್ಥಿ
ಭಿಕ್ಷುಕ
ಭಿಕ್ಷು-ಕನ
ಭಿಕ್ಷುಕ-ನಾ-ದರೋ
ಭಿಕ್ಷುಕ-ನೊಂದಿಗೆ
ಭಿಕ್ಷು-ಕರು
ಭಿನ್ನ
ಭಿನ್ನ-ಗೊಳಿ-ಸು-ವುದೋ
ಭಿನ್ನತೆ
ಭಿನ್ನ-ತೆ-ಗ-ಳನ್ನು
ಭಿನ್ನ-ತೆ-ಗಳೇ-ನೆಂಬು-ದನ್ನು
ಭಿನ್ನ-ತೆಗೆ
ಭಿನ್ನ-ತೆ-ಯನ್ನು
ಭಿನ್ನ-ತೆ-ಯೆಲ್ಲ
ಭಿನ್ನ-ತೆಯೇ
ಭಿನ್ನ-ಭಿನ್ನ
ಭಿನ್ನ-ಭಿನ್ನ-ವಾದ
ಭಿನ್ನ-ಮತ
ಭಿನ್ನ-ರೀತಿಯ
ಭಿನ್ನಸ್ಥಿತಿ-ಗಳು
ಭಿನ್ನಾಭಿ
ಭಿನ್ನಾಭಿಪ್ರಾಯ
ಭಿನ್ನಾಭಿಪ್ರಾಯ-ಗಳಿಂದ
ಭಿನ್ನಾಭಿಪ್ರಾಯ-ಗಳಿಗೆ
ಭಿನ್ನಾಭಿಪ್ರಾಯ-ಗಳಿದ್ದರೂ
ಭಿನ್ನಾಭಿಪ್ರಾಯ-ಗಳು
ಭಿನ್ನಾಭಿಪ್ರಾಯ-ವಿದ್ದರೂ
ಭೀಕರ
ಭೀಕರ-ವಾದ
ಭೀಮ-ವೃಕ್ಷ-ವಾ-ಗು-ವುದು
ಭುಕ್ತಯೇ
ಭುಜ
ಭುಜ-ತಟ್ಟಿ
ಭುಜದ
ಭುವನಜ್ಞಾನಂ
ಭೂತ
ಭೂತ-ಕಾಲಕ್ಕೆ
ಭೂತ-ಕಾಲದ
ಭೂತ-ಗಳಲ್ಲಿ
ಭೂತ-ಗಳು
ಭೂತ-ಗಳೂ
ಭೂತ-ಜಯ
ಭೂತ-ಜಯಃ
ಭೂತ-ಜಯ-ದಿಂದ
ಭೂತದ
ಭೂತಪ್ರೇತ-ಗಳ
ಭೂತಪ್ರೇತ-ಗಳನ್ನೆಲ್ಲ
ಭೂತಪ್ರೇತ-ಗಳಾ-ಗು-ವರು
ಭೂತಪ್ರೇತ-ಗಳು
ಭೂತಪ್ರೇತ-ಗಳೊ
ಭೂತಪ್ರೇ-ತಾದಿ-ಗಳು
ಭೂತ-ಭ-ವಿಷ್ಯ-ಗ-ಳನ್ನು
ಭೂತ-ಭ-ವಿಷ್ಯತ್ತನ್ನು
ಭೂತ-ಭ-ವಿಷ್ಯತ್ತು-ಗ-ಳನ್ನು
ಭೂತ-ವಾಗಿ-ದೆ-ಅದ್ವೈತ
ಭೂತ-ವಾದ
ಭೂತ-ವಾ-ದಕ್ಕೂ
ಭೂತ-ವಾದ-ದಲ್ಲಿ
ಭೂತ-ವಾದವೂ
ಭೂತ-ವಾದಿ-ಗಳು
ಭೂತವು
ಭೂತ-ವೆಂದರೆ
ಭೂತಿ
ಭೂತೇಂದ್ರಿಯಾತ್ಮಕಂ
ಭೂತೇಂದ್ರಿಯೇಷು
ಭೂಪಟ-ಗಳು
ಭೂಪಟ-ವನ್ನು
ಭೂಪಟ-ವೇನೊ
ಭೂಮಿ
ಭೂಮಿ-ಕೆ-ಯಲ್ಲಿದೆ
ಭೂಮಿಗೂ
ಭೂಮಿಗೆ
ಭೂಮಿ-ತಾಯಿ
ಭೂಮಿಯ
ಭೂಮಿ-ಯನ್ನು
ಭೂಮಿ-ಯಲ್ಲ
ಭೂಮಿ-ಯಲ್ಲಿ
ಭೂಮಿ-ಯಲ್ಲಿ-ರು-ವಿರಿ
ಭೂಮಿ-ಯಷ್ಟು
ಭೂಮಿ-ಯಾಯಿತೋ
ಭೂಮಿ-ಯಿಂದ
ಭೂಮಿಯು
ಭೂಮಿಷು
ಭೂಲೋ-ಕಕ್ಕೆ
ಭೂಸ್ವರ್ಗ-ಗಳು
ಭೇದ
ಭೇದಕ್ಕೆ
ಭೇದ-ದಲ್ಲಿ
ಭೇದ-ಭಾವ
ಭೇದ-ಭಾವದ
ಭೇದ-ವನ್ನು
ಭೇದ-ವಾಗಿದ್ದರೂ
ಭೇದ-ವಿದೆ
ಭೇದ-ವಿ-ರು-ವುದು
ಭೇದ-ವಿಲ್ಲ
ಭೇದಿಸ-ಬೇಕೆಂಬ
ಭೇದಿಸ-ಲಾ-ಗದ
ಭೇದಿಸಿ
ಭೇದಿಸಿ-ಕೊಂಡು
ಭೇದಿ-ಸಿದ
ಭೇದಿ-ಸು-ವು-ದಕ್ಕೆ
ಭೇದಿಸು-ವು-ದ-ರಲ್ಲಿ
ಭೋಕ್ತೃ
ಭೋಗ
ಭೋಗಃ
ಭೋಗಕ್ಕಾಗಿ
ಭೋಗಕ್ಕಿಂತ
ಭೋಗಕ್ಕೆ
ಭೋಗ-ಗಳ
ಭೋಗ-ಗ-ಳನ್ನು
ಭೋಗ-ಗಳನ್ನೆಲ್ಲ
ಭೋಗ-ಗಳಿದ್ದರೂ
ಭೋಗಜ್ಞಾನವು
ಭೋಗದ
ಭೋಗ-ದಲ್ಲಿ
ಭೋಗ-ದಿಂದ
ಭೋಗ-ವನ್ನು
ಭೋಗ-ವಲಯ
ಭೋಗ-ವಸ್ತು-ಗಳು
ಭೋಗವು
ಭೋಗವೇ
ಭೋಗ-ಶಕ್ತಿ-ಯನ್ನು
ಭೋಗ-ಸಾಮಗ್ರಿ-ಗಳಿ-ರ-ಬಾ-ರದು
ಭೋಗ-ಸುಖ-ವನ್ನೆ
ಭೋಗಾನ್ವೇಷಣೆಯ
ಭೋಗಾಪ-ವರ್ಗಾರ್ಥಂ
ಭೋಗಾಪೇಕ್ಷೆ-ಯನ್ನೂ
ಭೋಗಾ-ಸಕ್ತಿ
ಭೋಗಿಸ-ಬೇಕೆಂಬ
ಭೋಗಿಸಿ
ಭೋಗಿಸು-ವಂತೆ
ಭೋಗಿ-ಸು-ವು-ದಕ್ಕೆ
ಭೋಗೇಚ್ಛೆ
ಭೋಜನ
ಭೋಧಿ-ಸಲ್ಪಡುತ್ತಿ-ರುವ
ಭೌತ
ಭೌತಜ್ಞಾನಾ-ಭಿ-ವೃದ್ಧಿ-ಯಿಂದ
ಭೌತದ್ರವ್ಯ
ಭೌತದ್ರವ್ಯ-ದಿಂದ
ಭೌತದ್ರವ್ಯ-ವಿದೆ
ಭೌತದ್ರವ್ಯ-ವೆನ್ನುವೆವೊ
ಭೌತ-ವನ್ನು
ಭೌತ-ವಸ್ತು
ಭೌತ-ವಸ್ತು-ವನ್ನು
ಭೌತ-ವಸ್ತು-ವಿ-ನಲ್ಲಿ
ಭೌತ-ವಸ್ತುವು
ಭೌತ-ವಸ್ತು-ವೆಂದರೆ
ಭೌತ-ವಸ್ತು-ವೊಂದಿದೆ
ಭೌತ-ವಾದಿ-ಗಳು
ಭೌತ-ವಿಜ್ಞಾನ
ಭೌತ-ವಿಜ್ಞಾನ-ಗಳನ್ನು
ಭೌತ-ವಿಜ್ಞಾನ-ಗಳೂ
ಭೌತ-ವಿಜ್ಞಾನ-ವೆಂದು
ಭೌತ-ಶಾಸ್ತ್ರಜ್ಞನು
ಭೌತ-ಶಾಸ್ತ್ರದ
ಭೌತ-ಶಾಸ್ತ್ರ-ದಿಂದ
ಭೌತ-ಶಾಸ್ತ್ರವೂ
ಭೌತಿಕ
ಭೌತಿಕ-ವಾಗಿ
ಭೌತಿಕವೂ
ಭ್ರಮಿಸಿ
ಭ್ರಮಿಸಿ-ದಾಗ
ಭ್ರಮಿಸು
ಭ್ರಮಿ-ಸುತ್ತೇವೆ
ಭ್ರಮಿಸು-ವರು
ಭ್ರಮೆ
ಭ್ರಮೆಯ
ಭ್ರಮೆ-ಯನ್ನು
ಭ್ರಮೆ-ಯಿಲ್ಲ
ಭ್ರಷ್ಟ-ನಾಗಿ-ರುವನು
ಭ್ರಾಂತ
ಭ್ರಾಂತ-ನಾದ
ಭ್ರಾಂತ-ರಾಗು-ವರು
ಭ್ರಾಂತ-ರಾಗು-ವು-ದಿಲ್ಲ
ಭ್ರಾಂತರು
ಭ್ರಾಂತಿ
ಭ್ರಾಂತಿಗೆ
ಭ್ರಾಂತಿ-ಗೊಳಗಾದ
ಭ್ರಾಂತಿ-ಪಡು-ವೆವು
ಭ್ರಾಂತಿ-ಭಾವನೆ-ಗಳನ್ನು
ಭ್ರಾಂತಿಯ
ಭ್ರಾಂತಿ-ಯನ್ನು
ಭ್ರಾಂತಿ-ಯಲ್ಲ
ಭ್ರಾಂತಿ-ಯಲ್ಲಿ
ಭ್ರಾಂತಿ-ಯಲ್ಲಿರ-ಲಿಲ್ಲ
ಭ್ರಾಂತಿ-ಯಲ್ಲಿರು-ವಿರಿ
ಭ್ರಾಂತಿ-ಯಾಗಿರ
ಭ್ರಾಂತಿ-ಯಿಂದ
ಭ್ರಾಂತಿ-ಯಿಂದಲೇ
ಭ್ರಾಂತಿಯು
ಭ್ರಾಂತಿಯೆ
ಭ್ರಾಂತಿ-ಯೆಂದು
ಭ್ರಾಂತಿಯೇ
ಭ್ರಾಂತಿ-ರೂಪದ
ಭ್ರೂಣ
ಭ್ರೂಣಸ್ಥಿತಿ-ಯಲ್ಲಿರ-ಬೇಕು
ಮ
ಮಂಕಾಗಿ
ಮಂಕಾದ
ಮಂಕಾ-ಯಿತು
ಮಂಕು
ಮಂಗ-ಮಾಯ-ವಾಗು-ವುದು
ಮಂಗಳ
ಮಂಗಳ-ಕರನು
ಮಂಗಳ-ಕಾರಿ
ಮಂಗಳ-ವಾಗು-ವು-ದಿಲ್ಲ
ಮಂಗಳಾ-ರತಿ
ಮಂಜಿ-ನಿಂದ
ಮಂಜು
ಮಂಜುಳ-ವಾದ
ಮಂಡಲ-ದಿಂದ
ಮಂಡಿಸ-ಲಾರೆವು
ಮಂಡಿಸಿರು-ವರು
ಮಂಡಿ-ಸುವ
ಮಂತ್ರ
ಮಂತ್ರ-ಗಳನ್ನು
ಮಂತ್ರ-ಗಳಲ್ಲಿ
ಮಂತ್ರ-ಗಳೆಂಬ
ಮಂತ್ರ-ದಲ್ಲಿ
ಮಂತ್ರ-ದೃಷ್ಟಿ-ಯುಳ್ಳ-ವರು
ಮಂತ್ರದ್ರಷ್ಟ
ಮಂತ್ರದ್ರಷ್ಟರು
ಮಂತ್ರ-ವನ್ನು
ಮಂತ್ರ-ವಾದಿ
ಮಂತ್ರ-ವಿದೆ-ಇದು
ಮಂತ್ರ-ಶಕ್ತಿ
ಮಂತ್ರಿ
ಮಂತ್ರಿ-ಯನ್ನು
ಮಂತ್ರಿಯು
ಮಂತ್ರೋಚ್ಛಾರಣೆ-ಯಿಂದ
ಮಂದ
ಮಂದ-ಗಮನ-ದಿಂದ
ಮಂದಪ್ರ-ವೃತ್ತಿಯೆ
ಮಂದ-ಬುದ್ಧಿ-ಯವು
ಮಂದ-ಬುದ್ಧಿ-ಯುಳ್ಳ-ವರು
ಮಂದ-ಹಾಸ-ದಿಂದ
ಮಂದ-ಹಾಸ-ವನ್ನು
ಮಂದಾನಿ-ಲನು
ಮಂದಾ-ವಾಗಿಯೋ
ಮಂದಿ
ಮಂದಿಗೆ
ಮಂದಿಯ
ಮಂದಿ-ಯಾ-ದರೂ
ಮಕರಂದ-ವನ್ನು
ಮಕ್ಕಳ
ಮಕ್ಕಳಂತೆ
ಮಕ್ಕ-ಳನ್ನು
ಮಕ್ಕ-ಳನ್ನೂ
ಮಕ್ಕಳಲ್ಲಿ
ಮಕ್ಕಳಾ-ಗಲೀ
ಮಕ್ಕಳಾಗಿ
ಮಕ್ಕಳಾಗಿದ್ದಾಗ
ಮಕ್ಕಳಾಟ
ಮಕ್ಕಳಾಟ-ವನ್ನು
ಮಕ್ಕ-ಳಾದ
ಮಕ್ಕಳಿಗೂ
ಮಕ್ಕಳಿಗೆ
ಮಕ್ಕಳಿ-ರು-ವಾಗ
ಮಕ್ಕಳು
ಮಕ್ಕಳೂ
ಮಕ್ಕ-ಳೆಂದು
ಮಕ್ಕಳೊಂದಿಗೆ
ಮಗ
ಮಗನ
ಮಗ-ನನ್ನು
ಮಗ-ನಿಗೆ
ಮಗನಿದ್ದ
ಮಗಳು
ಮಗು
ಮಗು-ವನ್ನು
ಮಗು-ವಾಗಿದ್ದಾಗ
ಮಗು-ವಾ-ಗಿ-ರುವಾಗ
ಮಗು-ವಾಗು-ವು-ದಿಲ್ಲ
ಮಗು-ವಾ-ಗು-ವುದು
ಮಗುವಿ
ಮಗುವಿ-ಗಾದರೂ
ಮಗು-ವಿಗೆ
ಮಗು-ವಿದೆ
ಮಗು-ವಿನ
ಮಗುವಿ-ನಂತೆ
ಮಗುವಿ-ನಲ್ಲಿ
ಮಗುವಿ-ನಲ್ಲಿದೆ
ಮಗುವಿ-ನಲ್ಲಿ-ರು-ವುದು
ಮಗುವಿ-ನಲ್ಲಿ-ರು-ವುವು
ಮಗುವು
ಮಗುವೂ
ಮಗ್ನ-ರಾಗಿದ್ದರೂ
ಮಗ್ನರಾ-ದ-ವರು
ಮಚ್ಚೆ
ಮಟ್ಟಕ್ಕೆ
ಮಟ್ಟದ
ಮಟ್ಟ-ದಲ್ಲಿ
ಮಟ್ಟದಲ್ಲಿದೆ
ಮಟ್ಟದಿಂದ
ಮಟ್ಟಿ-ಗಾದರೂ
ಮಟ್ಟಿಗೂ
ಮಟ್ಟಿಗೆ
ಮಠ-ಗಳಲ್ಲಿ
ಮಡಕೆ-ಯಲ್ಲಿ
ಮಡಕೆ-ಯಷ್ಟು
ಮಡಿ-ಯು-ವುದು
ಮಡಿಸ-ಲಾಗ-ದಂತಾಗು-ವವರೆಗೂ
ಮಣಿ
ಮಣಿ-ಗಳನ್ನು
ಮಣಿ-ಗಳಲ್ಲಿ-ರುವ
ಮಣಿ-ಗಳು
ಮಣಿ-ಗಳೊ-ಳಗಿಂದ
ಮಣಿ-ಪೂರ
ಮಣಿ-ಯಂತೆ
ಮಣಿಯೂ
ಮಣೇರ್ಗ್ರಹೀತೃಗ್ರಹಣ
ಮಣ್ಣನ್ನು
ಮಣ್ಣನ್ನೆಲ್ಲಾ
ಮಣ್ಣಿನ
ಮಣ್ಣಿನಿಂದ
ಮಣ್ಣು
ಮತ
ಮತ-ಗಳಿರ-ಬೇಕು
ಮತ-ಗಳಿ-ರುವ
ಮತ-ತತ್ತ್ವ-ವೆಲ್ಲ-ವನ್ನು
ಮತದ
ಮತ-ದಂತೆ
ಮತ-ಪಂಥ-ಗಳಿ-ವೆ-ಅವು
ಮತಭ್ರಾಂತ
ಮತಭ್ರಾಂತ-ನಿಗೂ
ಮತಭ್ರಾಂತ-ರಿಗೆ
ಮತಭ್ರಾಂತ-ರಿ-ರು-ವರು
ಮತಭ್ರಾಂತರು
ಮತಭ್ರಾಂತಿ
ಮತಭ್ರಾಂತಿ-ಯನ್ನು
ಮತಭ್ರಾಂತಿ-ಯಿಂದ
ಮತ-ವನ್ನು
ಮತವೇ
ಮತಸ್ಥಾಪಕ
ಮತಾಂಧತೆ-ಯನ್ನು
ಮತಾಂಧತೆ-ಯಿಂದ
ಮತಾಂಧ-ರಿಂದ
ಮತಾನು-ಯಾಯಿ-ಗಳನ್ನೆಲ್ಲಾ
ಮತಾನು-ಯಾಯಿ-ಗಳಲ್ಲಿ
ಮತಿಗೆಟ್ಟು
ಮತ್ತಷ್ಟು
ಮತ್ತಾ-ರನ್ನೂ
ಮತ್ತಾರೂ
ಮತ್ತಾವ
ಮತ್ತಾವು
ಮತ್ತಾ-ವುದ
ಮತ್ತಾವು-ದನ್ನು
ಮತ್ತಾವು-ದನ್ನೂ
ಮತ್ತಾವು-ದನ್ನೊ
ಮತ್ತಾವು-ದರ
ಮತ್ತಾ-ವು-ದ-ರಿಂದಲೂ
ಮತ್ತಾವು-ದಾ-ದರೂ
ಮತ್ತಾವು-ದಾದ-ರೊಂದು
ಮತ್ತಾ-ವುದು
ಮತ್ತಾ-ವುದೂ
ಮತ್ತಾ-ವುದೊ
ಮತ್ತಾ-ವುದೋ
ಮತ್ತು
ಮತ್ತೂ
ಮತ್ತೆ
ಮತ್ತೆಲ್ಲರೂ
ಮತ್ತೆಲ್ಲಿ
ಮತ್ತೆಲ್ಲಿಂದಲೂ
ಮತ್ತೆಲ್ಲಿಗೆ
ಮತ್ತೆಲ್ಲಿದೆ
ಮತ್ತೆಲ್ಲಿಯೂ
ಮತ್ತೆಲ್ಲಿಯೋ
ಮತ್ತೆಲ್ಲೂ
ಮತ್ತೆಲ್ಲೊ
ಮತ್ತೇಕೆ
ಮತ್ತೇ-ನನ್ನು
ಮತ್ತೇ-ನನ್ನೊ
ಮತ್ತೇನು
ಮತ್ತೇನೂ
ಮತ್ತೇನೊ
ಮತ್ತೇನೋ
ಮತ್ತೊಂದಕ್ಕೆ
ಮತ್ತೊಂದನ್ನು
ಮತ್ತೊಂದನ್ನೂ
ಮತ್ತೊಂದರ
ಮತ್ತೊಂದ-ರಲ್ಲಿ
ಮತ್ತೊಂದರ-ವರು
ಮತ್ತೊಂದ-ರಿಂದ
ಮತ್ತೊಂದ-ರೊಂದಿಗೆ
ಮತ್ತೊಂದಾಗಿ
ಮತ್ತೊಂದಿದೆ
ಮತ್ತೊಂದಿಲ್ಲ
ಮತ್ತೊಂದು
ಮತ್ತೊಂದೂ
ಮತ್ತೊಂದೇ
ಮತ್ತೊದು
ಮತ್ತೊಬ್ಬ
ಮತ್ತೊಬ್ಬನ
ಮತ್ತೊಬ್ಬ-ನಂತೆ
ಮತ್ತೊಬ್ಬ-ನನ್ನು
ಮತ್ತೊಬ್ಬ-ನಿಂದ
ಮತ್ತೊಬ್ಬ-ನಿ-ಗಿಂತ
ಮತ್ತೊಬ್ಬ-ನಿಗೂ
ಮತ್ತೊಬ್ಬ-ನಿಗೆ
ಮತ್ತೊಬ್ಬ-ನಿ-ರುವನು
ಮತ್ತೊಬ್ಬನು
ಮತ್ತೊಬ್ಬ-ನೊ-ಡನೆ
ಮತ್ತೊಬ್ಬರ
ಮತ್ತೊಬ್ಬ-ರನ್ನು
ಮತ್ತೊಬ್ಬ-ರಿಂದ
ಮತ್ತೊಬ್ಬ-ರಿ-ಗಾಗಿ
ಮತ್ತೊಬ್ಬ-ರಿಗೂ
ಮತ್ತೊಬ್ಬ-ರಿಗೆ
ಮತ್ತೊಬ್ಬ-ರಿಲ್ಲ
ಮತ್ತೊಬ್ಬರು
ಮತ್ತೊಬ್ಬ-ರೊ-ಡನೆ
ಮತ್ತೊಮ್ಮೆ
ಮದ-ದಲ್ಲಿ
ಮದುವೆ
ಮದುವೆ-ಮಾಡಿ
ಮದುವೆ-ಯನ್ನು
ಮದುವೆ-ಯಾಗ-ದ-ವರು
ಮದುವೆ-ಯಾಗದೆ
ಮದುವೆ-ಯಾಗಿ
ಮದುವೆ-ಯಾಗುವ
ಮದುವೆ-ಯಾಗು-ವರು
ಮದುವೆ-ಯಾಗು-ವು-ದನ್ನು
ಮದುವೆ-ಯಾದ-ವಳಲ್ಲ
ಮದುವೆ-ಯಿಲ್ಲ
ಮದ್ಗು
ಮದ್ಯಪಾನ
ಮದ್ಯಪಾನ-ಇವು-ಗಳಲ್ಲಿ
ಮಧುರ
ಮಧುರ-ವಾದ
ಮಧ್ಯ
ಮಧ್ಯ-ತರ-ವಾಗಿ-ರ-ಬಹುದು
ಮಧ್ಯದ
ಮಧ್ಯ-ದಲ್ಲಿ
ಮಧ್ಯ-ದಲ್ಲಿಯೂ
ಮಧ್ಯ-ದಲ್ಲಿ-ರುವ
ಮಧ್ಯ-ದಲ್ಲಿ-ರುವುದು
ಮಧ್ಯ-ದಿಂದೆದ್ದು
ಮಧ್ಯ-ಭಾಗ-ದಲ್ಲಿ-ರುವ
ಮಧ್ಯ-ಭೂಮಿ
ಮಧ್ಯಮ
ಮಧ್ಯ-ಮ-ವರ್ಗದ
ಮಧ್ಯ-ಮ-ವರ್ಗ-ದ-ವ-ರಿಂದ
ಮಧ್ಯ-ಯುಗ-ದಲ್ಲಿ
ಮಧ್ಯಸ್ಥ-ಗಾರ-ನಾರು
ಮಧ್ಯಸ್ಥಿಕೆ-ಗಾರನ
ಮಧ್ಯಾಕ್ಷ-ರವು
ಮಧ್ಯಾಹ್ನ
ಮಧ್ಯೆ
ಮನ
ಮನಃ
ಮನಃಪೂರ್ವ-ಕ-ವಾಗಿ
ಮನಃಸ್ವಾಧೀ-ನತೆ
ಮನ-ಗಂಡರು
ಮನ-ಗಂಡರೆ
ಮನ-ಗಂಡಿ-ರು-ವರೊ
ಮನಗಾಣಿಸು-ವುದೇ
ಮನನ
ಮನನ-ಮಾಡಿ
ಮನ-ಬಂದೆತ
ಮನ-ವನ್ನು
ಮನ-ವರಿಕೆ
ಮನಶ್ಯಕ್ತಿ
ಮನಶ್ಯಕ್ತಿ-ಯನ್ನೆಲ್ಲ
ಮನಶ್ಯಾಸ್ತ್ರ
ಮನಶ್ಯಾಸ್ತ್ರಕ್ಕೆ
ಮನಶ್ಯಾಸ್ತ್ರಜ್ಞನು
ಮನಶ್ಯಾಸ್ತ್ರಜ್ಞರ
ಮನಶ್ಯಾಸ್ತ್ರಜ್ಞರು
ಮನಶ್ಯಾಸ್ತ್ರದ
ಮನಶ್ಯಾಸ್ತ್ರ-ದಲ್ಲಿ
ಮನಶ್ಶಾಸ್ತ್ರಜ್ಞರು
ಮನಸಃ
ಮನಸ್ತಾಪ
ಮನಸ್ತಾಪ-ಗಳಿ-ಗೆಲ್ಲ
ಮನಸ್ಸನ್ನಿ-ರಿಸಿ
ಮನಸ್ಸನ್ನು
ಮನಸ್ಸನ್ನೂ
ಮನಸ್ಸನ್ನೆಲ್ಲ
ಮನಸ್ಸನ್ನೇ
ಮನಸ್ಸಾ-ಗಲಿ
ಮನಸ್ಸಾ-ಗಲೀ
ಮನಸ್ಸಾಗಿ-ರ-ಬಹು-ದೆಂದು
ಮನಸ್ಸಿ
ಮನಸ್ಸಿ-ಗಿಂತ
ಮನಸ್ಸಿಗೂ
ಮನಸ್ಸಿಗೆ
ಮನಸ್ಸಿದೆ
ಮನಸ್ಸಿದ್ದರೂ
ಮನಸ್ಸಿನ
ಮನಸ್ಸಿ-ನಂತೆ
ಮನಸ್ಸಿ-ನಲ್ಲ-ರುವ
ಮನಸ್ಸಿ-ನಲ್ಲಿ
ಮನಸ್ಸಿ-ನಲ್ಲಿಟ್ಟು
ಮನಸ್ಸಿ-ನಲ್ಲಿಟ್ಟು-ಕೊಂಡಿ-ರ-ಬಹುದು
ಮನಸ್ಸಿ-ನಲ್ಲಿ-ಡ-ಬೇಕಾದ
ಮನಸ್ಸಿ-ನಲ್ಲಿದೆ
ಮನಸ್ಸಿ-ನಲ್ಲಿ-ದೆ-ಇ-ವೆಲ್ಲಾ
ಮನಸ್ಸಿ-ನಲ್ಲಿದ್ದಾಗ
ಮನಸ್ಸಿ-ನಲ್ಲಿದ್ದು
ಮನಸ್ಸಿ-ನಲ್ಲಿ-ರುತ್ತವೆ
ಮನಸ್ಸಿ-ನಲ್ಲಿ-ರುವ
ಮನಸ್ಸಿ-ನಲ್ಲಿ-ರುವುದು
ಮನಸ್ಸಿ-ನಲ್ಲಿ-ರುವು-ದೆಲ್ಲ
ಮನಸ್ಸಿ-ನಲ್ಲಿ-ರು-ವುವು
ಮನಸ್ಸಿ-ನಲ್ಲಿವೆ
ಮನಸ್ಸಿ-ನಲ್ಲೆ
ಮನಸ್ಸಿ-ನಲ್ಲೇ
ಮನಸ್ಸಿ-ನಾಚೆ
ಮನಸ್ಸಿ-ನಿಂದ
ಮನಸ್ಸಿ-ನೊಂದಿಗೆ
ಮನಸ್ಸಿ-ನೊ-ಡನೆ
ಮನಸ್ಸಿ-ನೊ-ಳಗೆ
ಮನಸ್ಸಿಲ್ಲದ
ಮನಸ್ಸು
ಮನಸ್ಸು-ಗಳ
ಮನಸ್ಸು-ಗ-ಳನ್ನು
ಮನಸ್ಸು-ಗಳಲ್ಲದೆ
ಮನಸ್ಸು-ಗಳಿಂದ
ಮನಸ್ಸು-ಗಳಿ-ಗಿಂತ
ಮನಸ್ಸು-ಗಳಿಗೆ
ಮನಸ್ಸು-ಗಳಿಲ್ಲ
ಮನಸ್ಸು-ಗಳು
ಮನಸ್ಸು-ಗಳೂ
ಮನಸ್ಸು-ಗಳೆ-ರಡೂ
ಮನಸ್ಸು-ಗಳೇ
ಮನಸ್ಸುಳ್ಳ
ಮನಸ್ಸೂ
ಮನಸ್ಸೆ
ಮನಸ್ಸೆಂದು
ಮನಸ್ಸೆಂದೂ
ಮನಸ್ಸೆಂಬ
ಮನಸ್ಸೆಂಬುದು
ಮನಸ್ಸೆನ್ನುವ
ಮನಸ್ಸೆಲ್ಲ
ಮನಸ್ಸೆಲ್ಲಾ
ಮನಸ್ಸೇ
ಮನಸ್ಸೊಂದೆ
ಮನು
ಮನುಜ
ಮನು-ಜ-ದೇಹ-ದಲ್ಲಿ-ರುವ
ಮನು-ಜ-ರನ್ನೂ
ಮನು-ಜ-ರಿಗೆ
ಮನು-ವಿಗೆ
ಮನು-ವೆಂಬ
ಮನುಷ್ಯ
ಮನುಷ್ಯತ್ವವು
ಮನುಷ್ಯತ್ವದ
ಮನುಷ್ಯನ
ಮನುಷ್ಯ-ನಂತೆ
ಮನುಷ್ಯ-ನನ್ನು
ಮನುಷ್ಯ-ನನ್ನೆ
ಮನುಷ್ಯ-ನಲ್ಲ
ಮನುಷ್ಯ-ನಲ್ಲಿ
ಮನುಷ್ಯ-ನಲ್ಲಿ-ರುವ
ಮನುಷ್ಯ-ನಲ್ಲಿ-ರುವು-ದೆಲ್ಲ-ವನ್ನು
ಮನುಷ್ಯ-ನಲ್ಲೊ
ಮನುಷ್ಯ-ನಾಗ-ಬೇಕು
ಮನುಷ್ಯ-ನಾ-ಗಲಿ
ಮನುಷ್ಯ-ನಾಗಿರು
ಮನುಷ್ಯ-ನಾಗಿ-ರು-ವನೋ
ಮನುಷ್ಯ-ನಾಗು
ಮನುಷ್ಯ-ನಾಗುವ
ಮನುಷ್ಯ-ನಾಗು-ವುದು
ಮನುಷ್ಯ-ನಾಗು-ವುದು-ನೀವು
ಮನುಷ್ಯ-ನಾ-ದರೊ
ಮನುಷ್ಯ-ನಿಂದ
ಮನುಷ್ಯ-ನಿ-ಗಿಂತ
ಮನುಷ್ಯ-ನಿಗೂ
ಮನುಷ್ಯ-ನಿಗೆ
ಮನುಷ್ಯ-ನಿ-ರುವನು
ಮನುಷ್ಯ-ನಿ-ರುವ-ವರೆಗ
ಮನುಷ್ಯನು
ಮನುಷ್ಯನೂ
ಮನುಷ್ಯನೆ
ಮನುಷ್ಯ-ನೆಂದು
ಮನುಷ್ಯ-ನೆಂಬ
ಮನುಷ್ಯನೇ
ಮನುಷ್ಯ-ನೊಬ್ಬನೆ
ಮನುಷ್ಯರ
ಮನುಷ್ಯ-ರಂತೆ
ಮನುಷ್ಯ-ರನ್ನು
ಮನುಷ್ಯ-ರನ್ನೂ
ಮನುಷ್ಯ-ರಲ್ಲದ
ಮನುಷ್ಯ-ರಲ್ಲಿ
ಮನುಷ್ಯ-ರಲ್ಲಿ-ರುವ
ಮನುಷ್ಯ-ರಲ್ಲೆಲ್ಲ
ಮನುಷ್ಯ-ರಾಗ-ಬೇಕು
ಮನುಷ್ಯ-ರಾಗಿದ್ದು
ಮನುಷ್ಯ-ರಾಗು-ವರು
ಮನುಷ್ಯ-ರಾ-ಗು-ವುವು
ಮನುಷ್ಯ-ರಾ-ದರೆ
ಮನುಷ್ಯ-ರಾ-ದೆವು
ಮನುಷ್ಯ-ರಿಗೆ
ಮನುಷ್ಯ-ರಿಬ್ಬರೂ
ಮನುಷ್ಯರು
ಮನುಷ್ಯ-ರೆಂಬುದೆ
ಮನುಷ್ಯ-ರೆಲ್ಲ
ಮನುಷ್ಯರೇ
ಮನೆ
ಮನೆ-ಗಳ
ಮನೆ-ಗಳು
ಮನೆಗೂ
ಮನೆಗೆ
ಮನೆಯ
ಮನೆ-ಯಂತೆ
ಮನೆ-ಯನ್ನಾಗಿ
ಮನೆ-ಯನ್ನು
ಮನೆ-ಯಲ್ಲಿ
ಮನೆ-ಯಲ್ಲಿಯೂ
ಮನೆ-ಯಲ್ಲಿ-ರುವ
ಮನೆ-ಯ-ವರು
ಮನೆ-ಯಿಂದ
ಮನೋ
ಮನೋ-ಜವಿತ್ವಂ
ಮನೋ-ನಿಗ್ರಹ
ಮನೋ-ನಿಗ್ರ-ಹದ
ಮನೋ-ಭೀಷ್ಟ-ವಾ-ಗಿತ್ತು
ಮನೋ-ರಂಜ-ನೆಗೆ
ಮನೋ-ರಂಜ-ನೆಯ
ಮನೋ-ಲೋಕ-ದಲ್ಲಿ
ಮನೋಲ್ಲಾಸ
ಮನೋ-ವಾಕ್ಕಾಯ-ವಾಗಿ
ಮನೋ-ವಿಶ್ಲೇಷ-ಣೆಯ
ಮನೋ-ವಿಶ್ಲೇಷಣೆ-ಯಿಂದ
ಮನೋ-ಹರ
ಮನ್ನಣೆ
ಮಬ್ಬು
ಮಬ್ಬು-ಮಬ್ಬಾ-ಗಿದೆ
ಮಮ
ಮಮ-ಕಾರ
ಮಮ-ಕಾರ-ಗಳು
ಮಮ-ಕಾರದ
ಮಮ-ತೆಯೂ
ಮಮ್ಮಿ
ಮಮ್ಮಿ-ಗಳಂತೆ
ಮಯ
ಮಯ-ನಾದ
ಮಯ-ವಾದ-ವು-ಗಳು
ಮಯಾ
ಮರ
ಮರಕ್ಕೂ
ಮರಕ್ಕೆ
ಮರ-ಗಳನ್ನೂ
ಮರಣ
ಮರಣಕ್ಕೂ
ಮರಣಕ್ಕೆ
ಮರಣ-ಗಳ
ಮರಣ-ಗಳನ್ನು
ಮರಣ-ಗಳಿಂದ
ಮರಣ-ಗಳಿಲ್ಲ
ಮರಣ-ಗಳು
ಮರಣ-ಗಳೂ
ಮರಣದ
ಮರಣ-ದಲ್ಲಿ
ಮರಣ-ದಿಂದ
ಮರಣ-ದೆಡೆಗೆ
ಮರಣ-ಭಯ
ಮರಣ-ರಹಿತ-ವಾಗಿರ-ಬೇಕು
ಮರಣ-ವನ್ನು
ಮರಣ-ವಿದೆ
ಮರಣ-ವಿಲ್ಲದ
ಮರಣ-ವಿಲ್ಲ-ದಂತೆ
ಮರಣವು
ಮರಣವೂ
ಮರಣ-ವೆಂಬ
ಮರಣ-ವೆನ್ನು-ವೆವು
ಮರಣಾ
ಮರಣಾ-ತೀತ
ಮರಣಾ-ತೀ-ತನು
ಮರಣಾ-ತೀತನೂ
ಮರಣಾ-ತೀತನೆ
ಮರಣಾ-ನಂತರ
ಮರಣಾ-ನಂತ-ರವೂ
ಮರದ
ಮರ-ದಂತೆಯೇ
ಮರ-ದಲ್ಲಿ
ಮರ-ದಿಂದಾ-ಗು-ವುದು
ಮರಳಿ
ಮರ-ಳಿಗೆ
ಮರ-ಳಿ-ದನು
ಮರ-ಳಿನ
ಮರ-ಳಿ-ನಲ್ಲಿ
ಮರ-ಳಿ-ನಿಂದ
ಮರಳು
ಮರ-ಳು-ಕಾಡಿನ
ಮರ-ಳು-ಗಾಡಿ-ನಲ್ಲಿ
ಮರ-ಳು-ಗಾಡು
ಮರ-ಳು-ವುದು
ಮರ-ಳು-ವುವು
ಮರ-ವನ್ನು
ಮರ-ವಾಗಿ
ಮರ-ವಾಗುವ
ಮರ-ವಾಗು-ವು-ದಿಲ್ಲ
ಮರ-ವಾ-ಗು-ವುದು
ಮರ-ವಾದ
ಮರವು
ಮರ-ವೆಲ್ಲ
ಮರವೇ
ಮರಿ
ಮರಿ-ಗಳ
ಮರಿ-ಗ-ಳನ್ನು
ಮರಿ-ಗಳು
ಮರಿ-ಗಳೊಂದಿಗೆ
ಮರಿಗೆ
ಮರಿಯು
ಮರೀಚಿಕೆ
ಮರೀಚಿಕೆ-ಯನ್ನೇ
ಮರೀಚಿಕೆ-ಯಲ್ಲಿ
ಮರುಕ್ಷಣವೇ
ಮರುಗ-ಬಲ್ಲ-ವ-ರಾ-ದರೆ
ಮರುಗುವ
ಮರು-ಗು-ವುದು
ಮರು-ಭೂಮಿ
ಮರು-ಭೂಮಿ-ಯಂತೆ
ಮರುಳ
ಮರುಳು-ಕಾಡಿ-ನಲ್ಲಿ
ಮರೆತ
ಮರೆ-ತಂತೆ
ಮರೆತಿದೆ-ಅವು-ಗ-ಳನ್ನು
ಮರೆತಿದ್ದ
ಮರೆತಿದ್ದಿರಿ
ಮರೆತಿರು-ವೆವು
ಮರೆ-ತಿಲ್ಲ
ಮರೆತು
ಮರೆತು-ಹೋಗಿ
ಮರೆ-ಮಾಡಿದ
ಮರೆ-ಮಾಡಿದ್ದ
ಮರೆ-ಮಾಡುವ
ಮರೆ-ಮಾಡು-ವು-ದಕ್ಕೆ
ಮರೆ-ಮಾಡು-ವು-ದಲ್ಲ
ಮರೆಯ-ಕೂಡದು
ಮರೆ-ಯದಿ-ರ-ಬೇಕು
ಮರೆ-ಯದಿ-ರು-ವುದು
ಮರೆ-ಯದೆ
ಮರೆಯ-ಬಲ್ಲ
ಮರೆಯ-ಬಾ-ರದು
ಮರೆಯ-ಬೇಡಿ
ಮರೆ-ಯಲು
ಮರೆ-ಯಿರಿ
ಮರೆಯು
ಮರೆಯು-ವನು
ಮರೆಯು-ವರು
ಮರೆಯು-ವು-ದಕ್ಕೆ
ಮರೆಯು-ವುದು
ಮರೆಯು-ವುದೇ
ಮರೆಯು-ವೆವು
ಮರ್ತ್ಯ
ಮರ್ತ್ಯನು
ಮರ್ತ್ಯನೇ
ಮರ್ತ್ಯ-ಭಾಗ-ವೆಲ್ಲ
ಮರ್ತ್ಯ-ಲೋಕ
ಮರ್ತ್ಯ-ಲೋಕಕ್ಕಿಂತಲೂ
ಮರ್ತ್ಯ-ಲೋಕ-ವನ್ನು
ಮರ್ತ್ಯ-ಲೋಕವೆ
ಮಲ
ಮಲಗ-ಬೇಡಿ
ಮಲ-ಗಲು
ಮಲಗಿ-ಕೊಂಡಿದ್ದ
ಮಲಗಿದ್ದಾಗ
ಮಲ-ಗಿಸಿ-ದರೆ
ಮಲಗುತ್ತೇನೆ
ಮಲ-ಗು-ವುದು
ಮಲಿನತೆ
ಮಲಿನ-ತೆ-ಯನ್ನು
ಮಲಿನ-ತೆ-ಯಿಂದ
ಮಲಿನ-ತೆ-ಯಿಲ್ಲದೆ
ಮಲಿನ-ವಾಗ-ದಂತೆ
ಮಲಿನ-ವಾಗದೆ
ಮಲಿನ-ವಾ-ಗು-ವುದು
ಮಳೆ
ಮಳೆ-ಗಾಲ-ದಲ್ಲಿ
ಮಳೆಗೆ
ಮಳೆಯ
ಮಳೆ-ಯ-ದಂತೆ
ಮಳೆ-ಯಲ್ಲಿ
ಮಳೆ-ಯಿಲ್ಲದ
ಮಳೆಯೂ
ಮಳೆ-ಹನಿ-ಯಂತೆ
ಮಳೆ-ಹನಿ-ಯಾಗಿ
ಮಸ-ಕಾಗಿ-ರ-ಬಹುದು
ಮಸೀದಿಗೆ
ಮಹತ್
ಮಹತ್ತರ-ವಾ-ದುದು
ಮಹತ್ತರ-ವಾದು-ದೇನೋ
ಮಹತ್ತಾ-ದುದು
ಮಹತ್ತಿ-ಗಾಗಿ
ಮಹತ್ತಿನ
ಮಹತ್ತಿ-ನಿಂದ
ಮಹತ್ತು
ಮಹತ್ತೇ
ಮಹತ್ತ್ವ-ವಾದ
ಮಹತ್ನಿಂದ
ಮಹತ್ವ
ಮಹತ್ವಕ್ಕೆ
ಮಹತ್ವ-ಪೂರ್ಣ-ವಾದ
ಮಹತ್ವ-ಪೂರ್ಣ-ವಾದು-ದೆಲ್ಲ
ಮಹತ್ವ-ವಿದೆ
ಮಹತ್ವವೇ
ಮಹದಾ
ಮಹದಾ-ನಂದ-ವಿದೆ
ಮಹದಾ-ಲೋ-ಚನೆ-ಗ-ಳನ್ನು
ಮಹದಾ-ಲೋ-ಚನೆ-ಗಳು
ಮಹದೈಶ್ವರ್ಯ-ವಾ-ಗು-ವುದು
ಮಹಮ್ಮ
ಮಹಮ್ಮ-ದನು
ಮಹಮ್ಮದೀ
ಮಹಮ್ಮ-ದೀಯ
ಮಹಮ್ಮ-ದೀ-ಯ-ನಲ್ಲವೋ
ಮಹಮ್ಮ-ದೀ-ಯ-ನಾದ
ಮಹಮ್ಮ-ದೀ-ಯ-ನಾ-ದರೆ
ಮಹಮ್ಮ-ದೀ-ಯರ
ಮಹಮ್ಮ-ದೀ-ಯ-ರನ್ನು
ಮಹಮ್ಮ-ದೀ-ಯ-ರಲ್ಲಿ
ಮಹಮ್ಮ-ದೀ-ಯ-ರಾ-ಗದೆ
ಮಹಮ್ಮ-ದೀ-ಯ-ರಿಗೆ
ಮಹಮ್ಮ-ದೀ-ಯ-ರಿ-ರು-ವರು
ಮಹಮ್ಮ-ದೀ-ಯರು
ಮಹಮ್ಮ-ದೀ-ಯರೂ
ಮಹಮ್ಮ-ದೀ-ಯರೊ
ಮಹಮ್ಮದ್
ಮಹರ್ಷಿ-ಗಳು
ಮಹರ್ಷಿಯ
ಮಹರ್ಷಿಯು
ಮಹಲ್
ಮಹಾ
ಮಹಾ-ಕಲ್ಯಾಣಪ್ರದ-ವಾದ
ಮಹಾ-ಕಾರ್ಯ-ಗ-ಳನ್ನು
ಮಹಾ-ಕಾರ್ಯ-ವನ್ನು
ಮಹಾ-ಕಾವ್ಯ
ಮಹಾ-ಕಾಶ
ಮಹಾ-ಕೇಂದ್ರ
ಮಹಾ-ಚಕ್ರದ
ಮಹಾ-ಚಿಂತನೆ-ಗಳೆಲ್ಲ
ಮಹಾಜ್ಞಾನಿ
ಮಹಾತ್ಮ
ಮಹಾತ್ಮ-ನಾಗ-ಬೇಕಾ-ದರೆ
ಮಹಾತ್ಮ-ನಾದ
ಮಹಾತ್ಮರ
ಮಹಾತ್ಮ-ರನ್ನು
ಮಹಾತ್ಮ-ರಿಂದ
ಮಹಾತ್ಮ-ರಿಗೆ
ಮಹಾತ್ಮ-ರಿ-ಗೆಲ್ಲ
ಮಹಾತ್ಮ-ರಿಗೆಲ್ಲ-ರಿಗೂ
ಮಹಾತ್ಮ-ರಿ-ಗೆಲ್ಲಾ
ಮಹಾತ್ಮರು
ಮಹಾತ್ಮ-ರೇನು
ಮಹಾತ್ಮೆ
ಮಹಾ-ದೋಷ-ವಿದೆ
ಮಹಾ-ಧರ್ಮ-ಗಳೆಲ್ಲ
ಮಹಾ-ಧರ್ಮವೂ
ಮಹಾ-ಧೀರ-ನಾದ
ಮಹಾ-ನದಿ-ಗಳು
ಮಹಾ-ನದಿಯ
ಮಹಾ-ನದಿಯು
ಮಹಾ-ನಷ್ಟ-ವನ್ನು-ಕುರಿತು
ಮಹಾ-ನೀತಿ-ಯನ್ನು
ಮಹಾನು
ಮಹಾ-ನು-ಭಾವ
ಮಹಾ-ನು-ಭಾವ-ರನ್ನೊ
ಮಹಾನ್
ಮಹಾ-ಪರ್ವತ-ಗಳು
ಮಹಾ-ಪಾತಕ
ಮಹಾ-ಪಾತ-ಕ-ಆ-ದರೆ
ಮಹಾ-ಪಾಪ-ದಿಂದ
ಮಹಾ-ಪುರುಷ
ಮಹಾ-ಪುರುಷ-ನಿಗೆ
ಮಹಾ-ಪುರುಷರ
ಮಹಾ-ಪುರುಷ-ರನ್ನೂ
ಮಹಾ-ಪುರುಷ-ರಲ್ಲದೇ
ಮಹಾ-ಪುರುಷರು
ಮಹಾ-ಪುರುಷ-ರೆಲ್ಲ
ಮಹಾಪ್ರ-ಪಾತದ
ಮಹಾಪ್ರೇರಣೆಯು
ಮಹಾ-ಭಾ-ವನೆ
ಮಹಾ-ಭಾವ-ನೆ-ಗಳು
ಮಹಾ-ಭಾವ-ವಿದೆ
ಮಹಾ-ಮಹಿ-ಮನು
ಮಹಾ-ಮ-ಹಿಮ-ರಾದ
ಮಹಾ-ಮಹೀಮ-ನಾ-ದರೂ
ಮಹಾ-ಮೋಹ
ಮಹಾ-ಯಾಗ-ವನ್ನು
ಮಹಾ-ಯೋಗ-ದೊಂದಿಗೆ
ಮಹಾ-ಯೋಗ-ವನ್ನು
ಮಹಾ-ಯೋಗ-ವೆಂದು
ಮಹಾ-ಯೋಗ-ವೆಂದೂ
ಮಹಾ-ಯೋಗಿ-ಗಳಾ-ಗಿದ್ದರು
ಮಹಾ-ಯೋಗಿ-ಗಳಿ-ರು-ವರು
ಮಹಾ-ರಾಜ
ಮಹಾ-ವಿ-ದೇಹ
ಮಹಾ-ವಿ-ದೇಹ-ವೆಂಬ
ಮಹಾ-ವೇಗ-ದಿಂದ
ಮಹಾವ್ಯಕ್ತಿ
ಮಹಾವ್ಯಕ್ತಿ-ಗಳ
ಮಹಾವ್ಯಕ್ತಿ-ಗಳು
ಮಹಾವ್ಯಕ್ತಿಯ
ಮಹಾವ್ಯಕ್ತಿ-ಯೊಂದು
ಮಹಾವ್ರ-ತಮ್
ಮಹಾ-ಶಕ್ತಿ
ಮಹಾ-ಸತ್ಯಕ್ಕೆ
ಮಹಾ-ಸತ್ಯ-ಗಳು
ಮಹಾ-ಸತ್ಯದ
ಮಹಾ-ಸಾ-ಗರ
ಮಹಾ-ಸಾ-ಗರ-ವನ್ನು
ಮಹಿ-ಮಾನ್ವಿತ
ಮಹಿಮಾ-ಪೂರ್ಣ-ವಾಗಿ
ಮಹಿಮಾ-ಮಯ
ಮಹಿಮಾ-ಮಯ-ವಾದುವು
ಮಹಿ-ಮಾ-ವಂತ
ಮಹಿಮೆ
ಮಹಿಮೆ-ಯನ್ನು
ಮಹಿಮೆ-ಯಲ್ಲಿ
ಮಹಿಮೆಯು
ಮಹಿಮೆ-ಯೆಲ್ಲ
ಮಹೋತ್ತಮ
ಮಹೋತ್ತುಂಗ-ವಾದ
ಮಹೋನ್ನತ
ಮಹ್ಮ-ದೀ-ಯರು
ಮಾಂಸ
ಮಾಂಸ-ಖಂಡ
ಮಾಂಸ-ಖಂಡಕ್ಕೂ
ಮಾಂಸ-ಖಂಡಕ್ಕೆ
ಮಾಂಸ-ಖಂಡ-ಗಳ
ಮಾಂಸ-ಖಂಡ-ಗ-ಳನ್ನು
ಮಾಂಸ-ಖಂಡ-ಗಳು
ಮಾಂಸ-ಖಂಡದ
ಮಾಂಸ-ಖಂಡ-ವನ್ನೂ
ಮಾಂಸ-ಖಂಡವೂ
ಮಾಂಸ-ಗ-ಳನ್ನು
ಮಾಂಸದ
ಮಾಂಸ-ದಲ್ಲಿ
ಮಾಂಸ-ವನ್ನು
ಮಾಂಸ-ವಾಗಿ
ಮಾಂಸಾ-ಹಾರ
ಮಾಟಗಾತಿ
ಮಾಡ
ಮಾಡ-ಕೂಡದು
ಮಾಡ-ತಕ್ಕ-ವರು
ಮಾಡ-ದ-ವರು
ಮಾಡ-ದಿ-ರು-ವುದು
ಮಾಡದೆ
ಮಾಡದೇ
ಮಾಡ-ಬಯಸಿ-ದರೆ
ಮಾಡ-ಬಲ್ಲ
ಮಾಡ-ಬಲ್ಲದು
ಮಾಡ-ಬಲ್ಲ-ದು-ಆ-ದರೆ
ಮಾಡ-ಬಲ್ಲ-ದು-ಇ-ದಾದ
ಮಾಡ-ಬಲ್ಲನು
ಮಾಡ-ಬಲ್ಲರು
ಮಾಡ-ಬಲ್ಲ-ವರು
ಮಾಡ-ಬಲ್ಲಿರಿ
ಮಾಡ-ಬಲ್ಲುದು
ಮಾಡ-ಬಲ್ಲೆ
ಮಾಡ-ಬಲ್ಲೆವು
ಮಾಡ-ಬಲ್ಲೆವೋ
ಮಾಡ-ಬಹುದು
ಮಾಡ-ಬಹು-ದು-ಜ-ನರು
ಮಾಡ-ಬಹು-ದೆಂದು
ಮಾಡ-ಬಹುದೊ
ಮಾಡ-ಬಾ-ರದು
ಮಾಡ-ಬಾರ-ದೆಂದು
ಮಾಡ-ಬೇಕಾ
ಮಾಡ-ಬೇಕಾ-ಗಿತ್ತು
ಮಾಡ-ಬೇಕಾ-ಗಿದೆ
ಮಾಡ-ಬೇಕಾ-ಗುತ್ತದೆ
ಮಾಡ-ಬೇಕಾ-ಗು-ವುದು
ಮಾಡ-ಬೇಕಾದ
ಮಾಡ-ಬೇಕಾ-ದರೆ
ಮಾಡ-ಬೇಕಾ-ದುದಿಷ್ಟೆ
ಮಾಡ-ಬೇಕಾ-ದುದು
ಮಾಡ-ಬೇಕಾ-ದುದೆ
ಮಾಡ-ಬೇಕಾ-ಯಿತು
ಮಾಡ-ಬೇಕು
ಮಾಡ-ಬೇಕು-ಇ-ದ-ರಲ್ಲಿ
ಮಾಡ-ಬೇಕು-ಸುಳ್ಳು
ಮಾಡ-ಬೇಕೆಂದಿ-ರುವ
ಮಾಡ-ಬೇಕೆಂದಿ-ರು-ವೆನು
ಮಾಡ-ಬೇಕೆಂದು
ಮಾಡ-ಬೇಕೆಂಬ
ಮಾಡ-ಬೇಕೆಂಬು-ದಕ್ಕೆ
ಮಾಡ-ಬೇಡ
ಮಾಡ-ಬೇಡಿ
ಮಾಡ-ಲಾ-ಗದ
ಮಾಡ-ಲಾರ
ಮಾಡ-ಲಾರದು
ಮಾಡ-ಲಾರ-ದು-ದನ್ನು
ಮಾಡ-ಲಾರ-ದೇನು
ಮಾಡ-ಲಾರರು
ಮಾಡ-ಲಾರವು
ಮಾಡ-ಲಾರಿರಿ
ಮಾಡ-ಲಾರೆ
ಮಾಡ-ಲಾರೆವು
ಮಾಡಲಿ
ಮಾಡ-ಲಿಚ್ಛಿ-ಸು-ವನು
ಮಾಡ-ಲಿಲ್ಲ
ಮಾಡಲು
ಮಾಡ-ಲೆತ್ನಿಸಿ-ದಾಗ
ಮಾಡ-ಲೇ-ಬೇಕು
ಮಾಡ-ಲೇ-ಬೇಕೆಂದು
ಮಾಡಲ್ಪಟ್ಟರು
ಮಾಡಲ್ಪಟ್ಟ-ವನೆಂದೊ
ಮಾಡಿ
ಮಾಡಿ-ಅ-ನಂತರ
ಮಾಡಿ-ಕೊಂಡ
ಮಾಡಿ-ಕೊಂಡಂತೆ
ಮಾಡಿ-ಕೊಂಡರು
ಮಾಡಿ-ಕೊಂಡರೆ
ಮಾಡಿ-ಕೊಂಡ-ವರು
ಮಾಡಿ-ಕೊಂಡಾಗ
ಮಾಡಿ-ಕೊಂಡಿದ್ದರೆ
ಮಾಡಿ-ಕೊಂಡಿದ್ದು
ಮಾಡಿ-ಕೊಂಡಿ-ರು-ವರು
ಮಾಡಿ-ಕೊಂಡಿ-ರು-ವರೋ
ಮಾಡಿ-ಕೊಂಡಿ-ರು-ವ-ವನು
ಮಾಡಿ-ಕೊಂಡಿ-ರು-ವೆವು
ಮಾಡಿ-ಕೊಂಡು
ಮಾಡಿ-ಕೊಂಡೆವು
ಮಾಡಿ-ಕೊಡ-ಬೇಕೆಂದು
ಮಾಡಿ-ಕೊಡುತ್ತದೆ
ಮಾಡಿ-ಕೊಳ್ಳ-ಬಲ್ಲರೊ
ಮಾಡಿ-ಕೊಳ್ಳ-ಬಹು
ಮಾಡಿ-ಕೊಳ್ಳ-ಬಹುದು
ಮಾಡಿ-ಕೊಳ್ಳ-ಬಾ-ರದು
ಮಾಡಿ-ಕೊಳ್ಳ-ಬೇಕಾ-ಗಿದೆ
ಮಾಡಿ-ಕೊಳ್ಳ-ಬೇಕು
ಮಾಡಿ-ಕೊಳ್ಳ-ಬೇಕು-ಎನ್ನು-ವು-ದನ್ನು
ಮಾಡಿ-ಕೊಳ್ಳ-ಲಾರರು
ಮಾಡಿ-ಕೊಳ್ಳಲು
ಮಾಡಿ-ಕೊಳ್ಳಿ
ಮಾಡಿ-ಕೊಳ್ಳು
ಮಾಡಿ-ಕೊಳ್ಳುತ್ತೀರಿ
ಮಾಡಿ-ಕೊಳ್ಳುತ್ತೇವೆ
ಮಾಡಿ-ಕೊಳ್ಳು-ಲಾರೆವು
ಮಾಡಿ-ಕೊಳ್ಳುವ
ಮಾಡಿ-ಕೊಳ್ಳು-ವಂತೆ
ಮಾಡಿ-ಕೊಳ್ಳು-ವರು
ಮಾಡಿ-ಕೊಳ್ಳು-ವು-ದಕ್ಕೆ
ಮಾಡಿ-ಕೊಳ್ಳು-ವು-ದ-ರಲ್ಲಿ
ಮಾಡಿ-ಕೊಳ್ಳು-ವುದು
ಮಾಡಿ-ಕೊಳ್ಳು-ವೆನು
ಮಾಡಿ-ಕೊಳ್ಳು-ವೆವು
ಮಾಡಿ-ಕೊಳ್ಳೋಣ
ಮಾಡಿತು
ಮಾಡಿತೋ
ಮಾಡಿದ
ಮಾಡಿ-ದಂತೆ
ಮಾಡಿ-ದಂತೆಲ್ಲ
ಮಾಡಿ-ದನು
ಮಾಡಿ-ದ-ಮೇಲೆ
ಮಾಡಿ-ದರು
ಮಾಡಿ-ದರೂ
ಮಾಡಿ-ದರೆ
ಮಾಡಿ-ದ-ರೆಂಬುದು
ಮಾಡಿ-ದಳು
ಮಾಡಿ-ದ-ವ-ರನ್ನು
ಮಾಡಿ-ದ-ವ-ರಲ್ಲ
ಮಾಡಿ-ದ-ವ-ರಾರು
ಮಾಡಿ-ದ-ವರು
ಮಾಡಿ-ದಷ್ಟನ್ನಾ-ದರೂ
ಮಾಡಿ-ದಷ್ಟು
ಮಾಡಿ-ದಷ್ಟೂ
ಮಾಡಿ-ದಾಗ
ಮಾಡಿ-ದಾಗ-ಲೆಲ್ಲಾ
ಮಾಡಿ-ದಿರಿ
ಮಾಡಿ-ದು-ದನ್ನು
ಮಾಡಿ-ದು-ದ-ರಿಂದ
ಮಾಡಿ-ದುದು
ಮಾಡಿ-ದುವು-ಗಳಾ-ಗಿವೆ
ಮಾಡಿದೆ
ಮಾಡಿ-ದೆನೊ
ಮಾಡಿ-ದೆ-ವು-ಈಗ
ಮಾಡಿ-ದೆವೊ
ಮಾಡಿದ್ದ
ಮಾಡಿದ್ದರೂ
ಮಾಡಿದ್ದಾಗಿ-ರ-ಬಹುದು
ಮಾಡಿದ್ದಾಗಿ-ರ-ಬಹು-ದು-ಸಂಯಮ
ಮಾಡಿದ್ದೆನೋ
ಮಾಡಿ-ನೋಡಿ-ದರೆ
ಮಾಡಿಯೂ
ಮಾಡಿರ
ಮಾಡಿ-ರ-ಬಹುದು
ಮಾಡಿ-ರ-ಲಾರರು
ಮಾಡಿ-ರುತ್ತಾರೆ
ಮಾಡಿ-ರುವ
ಮಾಡಿ-ರುವನು
ಮಾಡಿ-ರುವ-ನೆಂದು
ಮಾಡಿ-ರುವನೋ
ಮಾಡಿ-ರು-ವರು
ಮಾಡಿ-ರು-ವರೋ
ಮಾಡಿ-ರುವ-ವನು
ಮಾಡಿ-ರು-ವಷ್ಟು
ಮಾಡಿ-ರುವುದು
ಮಾಡಿ-ರುವೆ
ಮಾಡಿ-ರು-ವೆವು
ಮಾಡಿ-ರು-ವೆವೊ
ಮಾಡಿಲ್ಲ
ಮಾಡಿವೆ
ಮಾಡಿ-ಸ-ಬಹುದು
ಮಾಡಿ-ಸ-ಬೇಕು
ಮಾಡಿ-ಸು-ವು-ದಕ್ಕಿಂತ
ಮಾಡು
ಮಾಡುತ್ತ
ಮಾಡುತ್ತದೆ
ಮಾಡುತ್ತ-ದೆಯೆ
ಮಾಡುತ್ತ-ದೆಯೇ
ಮಾಡುತ್ತ-ದೆಯೋ
ಮಾಡುತ್ತವೆ
ಮಾಡುತ್ತ-ವೆಯೋ
ಮಾಡುತ್ತಾ
ಮಾಡುತ್ತಾನೆ
ಮಾಡುತ್ತಾ-ನೆಯೊ
ಮಾಡುತ್ತಾ-ನೆಯೋ
ಮಾಡುತ್ತಾರೆ
ಮಾಡುತ್ತಾ-ರೆಯೊ
ಮಾಡುತ್ತಾ-ರೆಯೋ
ಮಾಡುತ್ತಾರೊ
ಮಾಡುತ್ತಾರೋ
ಮಾಡುತ್ತಿ
ಮಾಡುತ್ತಿದೆ
ಮಾಡುತ್ತಿದ್ದ
ಮಾಡುತ್ತಿದ್ದನು
ಮಾಡುತ್ತಿದ್ದರೂ
ಮಾಡುತ್ತಿದ್ದರೆ
ಮಾಡುತ್ತಿದ್ದ-ವನು
ಮಾಡುತ್ತಿದ್ದವು
ಮಾಡುತ್ತಿದ್ದಾಗ
ಮಾಡುತ್ತಿದ್ದು
ಮಾಡುತ್ತಿದ್ದೆವು
ಮಾಡುತ್ತಿ-ರ-ಬಹುದು
ಮಾಡುತ್ತಿ-ರ-ಲಿಲ್ಲ
ಮಾಡುತ್ತಿ-ರುವ
ಮಾಡುತ್ತಿ-ರುವನು
ಮಾಡುತ್ತಿ-ರು-ವರು
ಮಾಡುತ್ತಿ-ರು-ವರೋ
ಮಾಡುತ್ತಿ-ರುವಳು
ಮಾಡುತ್ತಿ-ರುವ-ವನು
ಮಾಡುತ್ತಿ-ರುವ-ವರು
ಮಾಡುತ್ತಿ-ರುವಾಗ
ಮಾಡುತ್ತಿ-ರು-ವಾ-ಗಲೂ
ಮಾಡುತ್ತಿ-ರು-ವಿರಿ
ಮಾಡುತ್ತಿ-ರುವು-ದಕ್ಕೆ
ಮಾಡುತ್ತಿರು-ವು-ದನ್ನು
ಮಾಡುತ್ತಿ-ರುವುದು
ಮಾಡುತ್ತಿ-ರುವುದೇ
ಮಾಡುತ್ತಿ-ರು-ವುವು
ಮಾಡುತ್ತಿ-ರು-ವೆನು
ಮಾಡುತ್ತಿ-ರುವೆ-ನೆಂಬ
ಮಾಡುತ್ತಿ-ರು-ವೆವು
ಮಾಡುತ್ತಿ-ರು-ವೆವೋ
ಮಾಡುತ್ತಿಲ್ಲ
ಮಾಡುತ್ತಿವೆ
ಮಾಡುತ್ತೀರಾ
ಮಾಡುತ್ತೀರಿ
ಮಾಡುತ್ತೀರೊ
ಮಾಡುತ್ತೇನೆ
ಮಾಡುತ್ತೇನೆಯೋ
ಮಾಡುತ್ತೇವೆ
ಮಾಡುತ್ತೇವೆಯೊ
ಮಾಡುವ
ಮಾಡು-ವಂತೆ
ಮಾಡು-ವನು
ಮಾಡು-ವನೆ
ಮಾಡು-ವರು
ಮಾಡು-ವರೊ
ಮಾಡು-ವರೋ
ಮಾಡು-ವ-ವನ
ಮಾಡು-ವ-ವ-ನಿಗೆ
ಮಾಡು-ವ-ವನು
ಮಾಡು-ವವ-ರನ್ನು
ಮಾಡು-ವ-ವ-ರಲ್ಲ
ಮಾಡು-ವ-ವರು
ಮಾಡು-ವ-ವರೆಗೆ
ಮಾಡು-ವಷ್ಟೇ
ಮಾಡು-ವಾಗ
ಮಾಡು-ವಿರಿ
ಮಾಡುವು
ಮಾಡು-ವುದ
ಮಾಡು-ವು-ದಕ್ಕಾಗಿ
ಮಾಡು-ವು-ದಕ್ಕಾಗಿಯೆ
ಮಾಡು-ವು-ದಕ್ಕಿಂತ
ಮಾಡು-ವು-ದಕ್ಕೂ
ಮಾಡು-ವು-ದಕ್ಕೆ
ಮಾಡು-ವು-ದಕ್ಕೋಸುಗ-ವಾಗಿಯೆ
ಮಾಡು-ವು-ದನ್ನು
ಮಾಡು-ವು-ದನ್ನೆಲ್ಲಾ
ಮಾಡು-ವು-ದರ
ಮಾಡು-ವು-ದ-ರಿಂದ
ಮಾಡು-ವು-ದಲ್ಲ
ಮಾಡು-ವು-ದಲ್ಲ-ಇದರ
ಮಾಡು-ವು-ದಲ್ಲದೆ
ಮಾಡು-ವು-ದಾ-ಗಲಿ
ಮಾಡು-ವು-ದಾ-ದರೆ
ಮಾಡು-ವು-ದಾ-ವುದು
ಮಾಡು-ವು-ದಿಲ್ಲ
ಮಾಡು-ವು-ದಿಲ್ಲವೊ
ಮಾಡು-ವು-ದಿಲ್ಲ-ವೋಇವ-ರಲ್ಲಿ
ಮಾಡು-ವುದು
ಮಾಡು-ವುದೂ
ಮಾಡು-ವುದೆ
ಮಾಡು-ವುದೇ
ಮಾಡು-ವು-ದೇನು
ಮಾಡು-ವುದೊ
ಮಾಡು-ವು-ದೊಂದಲ್ಲದೆ
ಮಾಡು-ವು-ದೊಂದೇ
ಮಾಡು-ವುದೋ
ಮಾಡು-ವುವು
ಮಾಡು-ವೆನು
ಮಾಡು-ವೆವು
ಮಾಡು-ವೆವೊ
ಮಾಡೋಣ
ಮಾಡ್ಡ್ತಿ-ರುವನು
ಮಾತ
ಮಾತ-ನಾಡ
ಮಾತ-ನಾಡ-ಕೂಡದು
ಮಾತ-ನಾಡ-ತೊಡಗಿದ
ಮಾತ-ನಾಡದ
ಮಾತ-ನಾಡದೆ
ಮಾತ-ನಾಡ-ಬಹುದು
ಮಾತ-ನಾಡ-ಬೇಕು
ಮಾತ-ನಾಡ-ಬೇಡಿ
ಮಾತ-ನಾ-ಡಲು
ಮಾತ-ನಾಡಿ
ಮಾತ-ನಾಡಿತು
ಮಾತ-ನಾಡಿದ
ಮಾತ-ನಾಡಿ-ದರು
ಮಾತ-ನಾಡಿ-ದರೂ
ಮಾತ-ನಾಡಿ-ದರೆ
ಮಾತ-ನಾಡಿ-ದವು
ಮಾತ-ನಾಡಿ-ದಾಗ
ಮಾತ-ನಾಡಿ-ರುವೆ
ಮಾತ-ನಾಡು
ಮಾತ-ನಾಡುತ್ತವೆ
ಮಾತ-ನಾಡುತ್ತಾನೆ
ಮಾತ-ನಾಡುತ್ತಿ
ಮಾತ-ನಾಡುತ್ತಿದ್ದ
ಮಾತ-ನಾಡುತ್ತಿದ್ದಂತೆ
ಮಾತ-ನಾಡುತ್ತಿದ್ದುದು
ಮಾತ-ನಾಡುತ್ತಿ-ರು-ವಾ-ಗಲೂ
ಮಾತ-ನಾಡುತ್ತಿರು-ವು-ದನ್ನು
ಮಾತ-ನಾಡುತ್ತಿ-ರುವೆ
ಮಾತ-ನಾಡುತ್ತಿ-ರು-ವೆನು
ಮಾತ-ನಾಡುತ್ತೀರಿ
ಮಾತ-ನಾಡುತ್ತೇನೆ
ಮಾತ-ನಾಡುವ
ಮಾತ-ನಾಡು-ವರು
ಮಾತ-ನಾಡು-ವ-ವನು
ಮಾತ-ನಾಡು-ವ-ವರ
ಮಾತ-ನಾಡು-ವಿ-ರಲ್ಲ
ಮಾತ-ನಾಡು-ವು-ದಕ್ಕೆ
ಮಾತ-ನಾಡು-ವು-ದ-ರಲ್ಲಿ
ಮಾತ-ನಾಡು-ವು-ದಿಲ್ಲ
ಮಾತ-ನಾಡು-ವುದು
ಮಾತ-ನಾಡು-ವುದೂ
ಮಾತನ್ನು
ಮಾತಲ್ಲ
ಮಾತಾಡು
ಮಾತಾಡುತ್ತಿ-ರುವ
ಮಾತಾಡುತ್ತಿ-ರು-ವೆನೊ
ಮಾತಾ-ದರೂ
ಮಾತಾ-ನಾಡ
ಮಾತಾಯಿತು
ಮಾತಿಗೂ
ಮಾತಿಗೆ
ಮಾತಿನ
ಮಾತಿ-ನಲ್ಲಿ
ಮಾತಿನಿಂದ
ಮಾತು
ಮಾತು-ಕತೆ-ಯಲ್ಲಿ
ಮಾತು-ಗ-ಳನ್ನು
ಮಾತು-ಗಳಲ್ಲಿಯೂ
ಮಾತೂ
ಮಾತೆಲ್ಲ
ಮಾತೇ
ಮಾತ್ರ
ಮಾತ್ರಕ್ಕೆ
ಮಾತ್ರ-ದಿಂದ
ಮಾತ್ರ-ವಲ್ಲ
ಮಾತ್ರ-ವಲ್ಲದೆ
ಮಾತ್ರ-ವಾಗಿದೆ
ಮಾತ್ರ-ವಾ-ಯಿತು
ಮಾತ್ರವೆ
ಮಾತ್ರವೇ
ಮಾತ್ರೆ-ಗಳುಳ್ಳ
ಮಾದಕದ್ರವ್ಯ
ಮಾನ
ಮಾನ-ಗಳ
ಮಾನ-ನಿಗೆ
ಮಾನವ
ಮಾನ-ವ-ಕರ್ಮ-ಗಳು
ಮಾನ-ವ-ಕುಲವೇ
ಮಾನ-ವ-ಕೃತ
ಮಾನ-ವ-ಕೊಟಿಯ
ಮಾನ-ವ-ಕೋಟಿ
ಮಾನ-ವ-ಕೋಟಿಗೆ
ಮಾನ-ವ-ಕೋಟಿಗೇ
ಮಾನ-ವ-ಕೋಟಿಯ
ಮಾನ-ವ-ಕೋಟಿಯು
ಮಾನ-ವ-ಕೋಟಿಯೂ
ಮಾನ-ವ-ಕೋಟಿ-ಯೊಂದಿಗೆ
ಮಾನ-ವ-ಜನಾಂಗದ
ಮಾನ-ವ-ಜೀವನ
ಮಾನ-ವ-ಜೀವಿಯ
ಮಾನ-ವತೆ
ಮಾನ-ವ-ತೆಗೆ
ಮಾನ-ವ-ತೆ-ಯನ್ನು
ಮಾನ-ವ-ದೇಹದ
ಮಾನ-ವ-ದೇಹ-ದಲ್ಲಿ-ರುವ
ಮಾನ-ವ-ದೇಹ-ದಲ್ಲಿ-ರುವಾ-ಗಲೇ
ಮಾನ-ವನ
ಮಾನ-ವ-ನಂತೆ
ಮಾನ-ವ-ನದು
ಮಾನ-ವ-ನನ್ನು
ಮಾನ-ವ-ನನ್ನೇ
ಮಾನ-ವ-ನಲ್ಲದ
ಮಾನ-ವ-ನಲ್ಲದೆ
ಮಾನ-ವ-ನಲ್ಲಿ
ಮಾನ-ವ-ನಲ್ಲಿಯೂ
ಮಾನ-ವ-ನಲ್ಲಿ-ರುವ
ಮಾನ-ವ-ನ-ವರೆ-ಗಿನ
ಮಾನ-ವ-ನ-ವರೆಗೆ
ಮಾನ-ವ-ನಾ-ಗಲಿ
ಮಾನ-ವ-ನಾಗಿ
ಮಾನ-ವ-ನಾಗುವ
ಮಾನ-ವ-ನಾಗು-ವನು
ಮಾನ-ವ-ನಿಂದ
ಮಾನ-ವ-ನಿ-ಗಿಂತ
ಮಾನ-ವ-ನಿಗೂ
ಮಾನ-ವ-ನಿಗೆ
ಮಾನ-ವ-ನಿಗೊಬ್ಬನಿಗೇ
ಮಾನ-ವನು
ಮಾನ-ವನೂ
ಮಾನ-ವನೆ
ಮಾನ-ವ-ನೆ-ದುರು
ಮಾನ-ವನೇ
ಮಾನ-ವ-ನೊಬ್ಬನೇ
ಮಾನ-ವ-ಮುಖ-ಗಳಲ್ಲಿ
ಮಾನ-ವ-ಮೃಗ-ಗಳಲ್ಲಿ
ಮಾನ-ವರ
ಮಾನ-ವ-ರಂತೆ
ಮಾನ-ವ-ರನ್ನಾಗಿ
ಮಾನ-ವ-ರನ್ನು
ಮಾನ-ವ-ರಲ್ಲಿ
ಮಾನ-ವ-ರಲ್ಲಿ-ರುವ
ಮಾನ-ವ-ರಾಗ
ಮಾನ-ವ-ರಾಗ-ಬೇಕು
ಮಾನ-ವ-ರಾಗಿ
ಮಾನ-ವ-ರಾಗಿದ್ದು
ಮಾನ-ವ-ರಾಗು-ವರು
ಮಾನ-ವ-ರಾದ
ಮಾನ-ವ-ರಿ-ಗಾಗಿ
ಮಾನ-ವ-ರಿ-ಗಿಂತ
ಮಾನ-ವ-ರಿಗೂ
ಮಾನ-ವ-ರಿಗೆ
ಮಾನ-ವರು
ಮಾನ-ವ-ರೆಲ್ಲ
ಮಾನ-ವರೇ
ಮಾನ-ವ-ರೊಂದಿಗೆ
ಮಾನ-ವವ್ಯಕ್ತಿ
ಮಾನ-ವ-ಶಕ್ತಿ-ಯನ್ನು
ಮಾನ-ವ-ಶರೀರ-ದಲ್ಲಿ-ರುವ
ಮಾನ-ವ-ಸಹೋ-ದರತ್ವದ
ಮಾನ-ವ-ಹೃದಯ-ವನ್ನು
ಮಾನ-ವಾಗಿ-ರು-ವು-ದೆಂದು
ಮಾನ-ವಿಕ
ಮಾನ-ವಿದೆ
ಮಾನ-ವೀಯ
ಮಾನ-ವೀಯತೆ
ಮಾನ-ವೆಲ್ಲ
ಮಾನ-ಸಿಕ
ಮಾನ-ಸಿಕ-ವಲ್ಲ
ಮಾನ-ಸಿಕ-ವಾಗಿ
ಮಾನ-ಸಿಕ-ವಾಗಿ-ಯಾ-ಗಲಿ
ಮಾನ-ಸಿಕ-ವಾದ
ಮಾನ-ಸಿಕ-ವಾದುದು
ಮಾನ-ಸಿಕವೂ
ಮಾನ್ಯ
ಮಾಮೇವ
ಮಾಯ
ಮಾಯ-ವಾಗ
ಮಾಯ-ವಾಗದೆ
ಮಾಯ-ವಾಗ-ಬಹುದು
ಮಾಯ-ವಾಗ-ಬೇಕು
ಮಾಯ-ವಾ-ಗಲು
ಮಾಯ-ವಾಗಿ
ಮಾಯ-ವಾ-ಗಿತ್ತು
ಮಾಯ-ವಾಗಿದೆ
ಮಾಯ-ವಾಗಿ-ದೆಯೊ
ಮಾಯ-ವಾಗಿರ
ಮಾಯ-ವಾಗಿ-ರುತ್ತದೆ
ಮಾಯ-ವಾ-ಗಿ-ರುವುದು
ಮಾಯ-ವಾ-ಗಿಲ್ಲ
ಮಾಯ-ವಾ-ಗಿಲ್ಲವೊ
ಮಾಯ-ವಾಗಿ-ಹೋ-ಯಿತು
ಮಾಯ-ವಾಗು
ಮಾಯ-ವಾಗುತ್ತದೆ
ಮಾಯ-ವಾಗುತ್ತವೆ
ಮಾಯ-ವಾಗುತ್ತಿದೆ
ಮಾಯ-ವಾಗುತ್ತಿದ್ದುವು
ಮಾಯ-ವಾಗುವ
ಮಾಯ-ವಾಗು-ವಂತೆ
ಮಾಯ-ವಾಗು-ವಂತೆಯೂ
ಮಾಯ-ವಾಗು-ವದು
ಮಾಯ-ವಾಗು-ವ-ವರೆಗೂ
ಮಾಯ-ವಾಗು-ವು-ದಿಲ್ಲ
ಮಾಯ-ವಾಗು-ವುದು
ಮಾಯ-ವಾಗು-ವುದು-ಎಲ್ಲಾ
ಮಾಯ-ವಾಗು-ವುವು
ಮಾಯ-ವಾಗು-ವುವೋ
ಮಾಯ-ವಾದ
ಮಾಯ-ವಾ-ದಂತೆ
ಮಾಯ-ವಾ-ದಂತೆ-ಯೆ-ಜ-ನರು
ಮಾಯ-ವಾ-ದನು
ಮಾಯ-ವಾ-ದರೆ
ಮಾಯ-ವಾ-ದಾಗ
ಮಾಯ-ವಾ-ದಾಗಲೆ
ಮಾಯ-ವಾ-ದೊ-ಡನೆ
ಮಾಯ-ವಾ-ಯಿತು
ಮಾಯ-ವಾರುತ್ತ-ಗಿದೆ
ಮಾಯ-ಸಿದ್ದಾಂತ
ಮಾಯ-ಸಿದ್ಧಾಂತ-ವನ್ನು
ಮಾಯಾ
ಮಾಯಾ-ದೀಪವು
ಮಾಯಾ-ಮೇತಾಂ
ಮಾಯಾ-ವರ-ಣ-ದೊ-ಳಗೆ
ಮಾಯಾ-ವಾದ
ಮಾಯಾ-ವಾದದ
ಮಾಯಾ-ವಾದಿ
ಮಾಯಾ-ಸಿದ್ಧಾಂತ
ಮಾಯಾ-ಸಿದ್ಧಾಂತ-ವನ್ನು
ಮಾಯೆ
ಮಾಯೆಗೆ
ಮಾಯೆಯ
ಮಾಯೆ-ಯನ್ನು
ಮಾಯೆ-ಯಲ್ಲಿ
ಮಾಯೆ-ಯಲ್ಲಿದೆ
ಮಾಯೆ-ಯಲ್ಲಿ-ರುವ
ಮಾಯೆ-ಯಲ್ಲಿವೆ
ಮಾಯೆ-ಯಲ್ಲೆ
ಮಾಯೆ-ಯಿಂದ
ಮಾಯೆಯು
ಮಾಯೆಯೇ
ಮಾರನೆ
ಮಾರ-ನೆಯ
ಮಾರಲು
ಮಾರುತ
ಮಾರುತ್ತಾರೆಯೋ
ಮಾರು-ವ-ವನು
ಮಾರುವ-ವನೆ
ಮಾರು-ವುದು
ಮಾರು-ಹೋ-ಗಿ-ರು-ವರೊ
ಮಾರ್ಗ
ಮಾರ್ಗಕ್ಕೆ
ಮಾರ್ಗ-ಗಳ
ಮಾರ್ಗ-ಗ-ಳನ್ನು
ಮಾರ್ಗ-ಗಳಲ್ಲಿ
ಮಾರ್ಗ-ಗಳಾ-ಗಿವೆ
ಮಾರ್ಗ-ಗಳಿಂದ
ಮಾರ್ಗ-ಗಳಿಂದಲೂ
ಮಾರ್ಗ-ಗಳು
ಮಾರ್ಗದ
ಮಾರ್ಗ-ದರ್ಶಕ
ಮಾರ್ಗ-ದರ್ಶಕ-ರಾಗ-ಬಹುದು
ಮಾರ್ಗ-ದರ್ಶಕ-ರಾದ
ಮಾರ್ಗ-ದರ್ಶಕರು
ಮಾರ್ಗ-ದರ್ಶಿ
ಮಾರ್ಗ-ದರ್ಶಿ-ಯಾಗ-ಲಿಚ್ಛಿ-ಸುವ
ಮಾರ್ಗ-ದಲ್ಲಿ
ಮಾರ್ಗ-ದಲ್ಲಿ-ಯಾ-ದರೋ
ಮಾರ್ಗ-ದಲ್ಲಿ-ರುವ
ಮಾರ್ಗ-ದಲ್ಲಿ-ರುವನು
ಮಾರ್ಗ-ದಲ್ಲಿ-ರು-ವರು
ಮಾರ್ಗ-ದಿಂದ
ಮಾರ್ಗ-ಮಧ್ಯದ
ಮಾರ್ಗ-ಮಾತ್ರ-ವೆಂಬು-ದನ್ನು
ಮಾರ್ಗ-ವನ್ನು
ಮಾರ್ಗ-ವನ್ನೂ
ಮಾರ್ಗ-ವನ್ನೆಲ್ಲ
ಮಾರ್ಗ-ವಾಗಿ
ಮಾರ್ಗ-ವಾ-ಗು-ವುದು
ಮಾರ್ಗ-ವಿದು
ಮಾರ್ಗ-ವಿದೆ
ಮಾರ್ಗ-ವಿದೆಯೆ
ಮಾರ್ಗ-ವಿ-ರಲಿ
ಮಾರ್ಗ-ವಿಲ್ಲ
ಮಾರ್ಗವು
ಮಾರ್ಗವೂ
ಮಾರ್ಗ-ವೆಂದು
ಮಾರ್ಗ-ವೆನ್ನು-ವುದು
ಮಾರ್ಗವೇ
ಮಾರ್ಗ-ವೊಂದನ್ನು
ಮಾರ್ಜಾಲ
ಮಾರ್ಪಡಿಸ
ಮಾರ್ಪಡಿ-ಸ-ಬಲ್ಲರು
ಮಾರ್ಪಡಿಸು
ಮಾರ್ಪ-ಡು-ವುದು
ಮಾರ್ಪಾಟು
ಮಾರ್ಪಾಡನ್ನು
ಮಾರ್ಪಾಡು-ಗ-ಳನ್ನು
ಮಾರ್ಮಾನ್
ಮಾವ
ಮಾವು
ಮಾಸ
ಮಾಹಾತ್ಮ್ಯ
ಮಾಹಾತ್ಮ್ಯೆ-ಯನ್ನು
ಮಿಂಚಿಗೆ
ಮಿಂಚಿನ
ಮಿಂಚಿ-ನಂತೆ
ಮಿಂಚಿ-ನಿಂದ
ಮಿಂಚು
ಮಿಕ್ಕ
ಮಿಕ್ಕ-ವು-ಗಳು
ಮಿಗಿ-ಲಾಗಿ
ಮಿಗಿಲಾ-ಗಿದೆ
ಮಿಗಿ-ಲಾಗಿ-ರುವನು
ಮಿಗಿ-ಲಾದ
ಮಿಗಿ-ಲಾ-ದುದು
ಮಿಗಿಲು
ಮಿಟಕಿಸ-ದ-ವರು
ಮಿಠಾಯಿ
ಮಿಡಿಯುತ್ತಿದೆ
ಮಿತ
ಮಿತ-ಭಾ-ವನೆ
ಮಿತ-ವಾಗ-ಬಲ್ಲದು
ಮಿತ-ವಾಗಿ
ಮಿತ-ವಾಗಿದೆ
ಮಿತ-ವಾದ
ಮಿತ-ವಾದು-ದೆಂದೂ
ಮಿತಾ-ಕಾಶ-ವನ್ನು
ಮಿತಿ
ಮಿತಿ-ಗ-ಳನ್ನು
ಮಿತಿಗೆ
ಮಿತಿ-ಗೊಳ-ಗಾಗಿ
ಮಿತಿ-ಮೀರಿ
ಮಿತಿಯ
ಮಿತಿ-ಯನ್ನು
ಮಿತಿ-ಯಲ್ಲಿ-ರ-ಲಾರದು
ಮಿತಿ-ಯಾ-ದಂತೆ
ಮಿತಿ-ಯಾದು-ದ-ರಿಂದ
ಮಿತಿ-ಯಿಂದ
ಮಿತಿ-ಯಿಂದಲೇ
ಮಿತಿ-ಯಿಂದಾಗಿ
ಮಿತಿ-ಯುಳ್ಳ-ವರು
ಮಿತಿಯೂ
ಮಿತಿ-ಯೆಂಬು-ದಿಲ್ಲ
ಮಿತ್ರ-ರಾಗಿ
ಮಿತ್ರಶತ್ರು-ಗಳಿಲ್ಲ
ಮಿಥ್ಯಾಜ್ಞಾನ-ಮತದ್ರೂಪಪ್ರತಿಷ್ಠಮ್
ಮಿಥ್ಯೆ
ಮಿಥ್ಯೆಗೆ
ಮಿಥ್ಯೆ-ಯಾಗಿದ್ದರೆ
ಮಿದು-ಳನ್ನು
ಮಿದುಳಿಗೂ
ಮಿದುಳಿಗೆ
ಮಿದುಳಿದೆ
ಮಿದುಳಿನ
ಮಿದುಳಿ-ನಲ್ಲಿ
ಮಿದುಳಿನಲ್ಲಿದ್ದರೆ
ಮಿದುಳಿನಲ್ಲಿನ
ಮಿದುಳಿನಲ್ಲಿಯೂ
ಮಿದುಳಿನಲ್ಲಿ-ರುವ
ಮಿದುಳಿ-ನಿಂದ
ಮಿದುಳಿ-ನೊಂದಿಗೆ
ಮಿದುಳು
ಮಿದುಳು-ಬಳ್ಳಿ
ಮಿದುಳು-ಬಳ್ಳಿಯು
ಮಿದುಳೆಲ್ಲವೂ
ಮಿದುಳೇ
ಮಿನುಗುತ್ತಿ-ರುವ
ಮಿಲನ-ವಾಗ-ಬೇಕು
ಮಿಲ್ಲನು
ಮಿಲ್ಲನೆ
ಮಿಶ್ರ
ಮಿಶ್ರಣ
ಮಿಶ್ರ-ಣ-ಎಲ್ಲಿ
ಮಿಶ್ರ-ಣ-ದಂತೆ
ಮಿಶ್ರ-ಣ-ವಲ್ಲದೆ
ಮಿಶ್ರ-ಣ-ವಾಗಿ-ರಲಿಲ್ಲವೊ
ಮಿಶ್ರ-ಣವೇ
ಮಿಶ್ರ-ದಿಂದ
ಮಿಶ್ರ-ದೃಶ್ಯ-ಗ-ಳನ್ನು
ಮಿಶ್ರ-ಮಾಡಿ
ಮಿಶ್ರ-ಮಾಡು-ವುವು
ಮಿಶ್ರ-ವಸ್ತು
ಮಿಶ್ರ-ವಾಗದೇ
ಮಿಶ್ರ-ವಾಗಿ
ಮಿಶ್ರ-ವಾಗಿ-ರುವ
ಮಿಶ್ರ-ವಾಗಿ-ರುವ-ವರು
ಮಿಶ್ರ-ವಾಗುತ್ತಿ-ರುವ
ಮಿಶ್ರ-ವಾಗು-ವು-ದಕ್ಕೆ
ಮಿಶ್ರ-ವಾದ
ಮಿಶ್ರವು
ಮಿಶ್ರ-ವೆಂದು
ಮಿಶ್ರವೇ
ಮಿಶ್ರಸ್ಥಿತಿ-ಯಲ್ಲಿ
ಮೀನಿ-ನಂತೆ
ಮೀನು
ಮೀನೊಂದು
ಮೀಮಾಂಸ
ಮೀರ-ಬಲ್ಲ
ಮೀರ-ಬೇಕು
ಮೀರಿ
ಮೀರಿದ
ಮೀರಿ-ದರೆ
ಮೀರಿ-ದ-ವನು
ಮೀರಿ-ದ-ವರು
ಮೀರಿ-ದಾಗ
ಮೀರಿ-ದುದು
ಮೀರಿ-ದುದೂ
ಮೀರಿದೆ
ಮೀರಿ-ದೆಯೊ
ಮೀರಿದ್ದರೆ
ಮೀರಿ-ರುವ
ಮೀರಿ-ರುವನು
ಮೀರಿ-ರು-ವಿರಿ
ಮೀರಿ-ರು-ವು-ದನ್ನು
ಮೀರಿ-ರು-ವು-ದ-ರಿಂದ
ಮೀರಿ-ರುವುದು
ಮೀರಿ-ರು-ವುದೇನೊ
ಮೀರಿಸಿ
ಮೀರಿ-ಹೋಗ-ಬಲ್ಲ
ಮೀರಿ-ಹೋಗ-ಲಾರದು
ಮೀರಿ-ಹೋಗಲು
ಮೀರಿ-ಹೋಗಿ
ಮೀರಿ-ಹೋಗಿದ್ದರೆ
ಮೀರಿ-ಹೋಗು
ಮೀರಿ-ಹೋ-ಗುತ್ತದೆ
ಮೀರಿ-ಹೋಗುವ
ಮೀರಿ-ಹೋಗು-ವುದು
ಮೀರಿ-ಹೋದ
ಮೀರಿ-ಹೋ-ದರು
ಮೀರುವನು
ಮೀರುವು-ದಾ-ವುದೂ
ಮೀರು-ವುದು
ಮೀಸ-ಲಾ-ಗದೆ
ಮೀಸ-ಲಾದ
ಮೀಸ-ಲಾ-ದದ್ದು
ಮೀಸಲು
ಮೀಸೆ
ಮೀಸೆ-ಯಿಲ್ಲ
ಮುಂಚಿನ
ಮುಂಚೆ
ಮುಂಚೆಯೇ
ಮುಂಜೆ
ಮುಂತಾಗಿ
ಮುಂತಾದ
ಮುಂತಾದ-ವನ್ನು
ಮುಂತಾದ-ವರಿದ್ದರೆ
ಮುಂತಾದ-ವರು
ಮುಂತಾ-ದವು
ಮುಂತಾದ-ವು-ಗಳ
ಮುಂತಾದ-ವು-ಗ-ಳನ್ನು
ಮುಂತಾದ-ವೆಲ್ಲ
ಮುಂತಾದು
ಮುಂತಾದು-ವನ್ನು
ಮುಂತಾದು-ವನ್ನೆಲ್ಲ
ಮುಂತಾದು-ವಾ-ವುವೂ
ಮುಂತಾ-ದುವು
ಮುಂತಾದು-ವು-ಗ-ಳನ್ನು
ಮುಂತಾದು-ವು-ಗಳಲ್ಲಿ
ಮುಂತಾದು-ವು-ಗಳಿಂದ
ಮುಂತಾದು-ವು-ಗಳಿಗೆ
ಮುಂತಾದು-ವು-ಗಳೆಲ್ಲ
ಮುಂತಾದು-ವು-ಗಳೇ
ಮುಂತಾದು-ವೆಲ್ಲ
ಮುಂದಕ್ಕೆ
ಮುಂದಾಳು
ಮುಂದಿ
ಮುಂದಿ-ಡಲು
ಮುಂದಿಡು
ಮುಂದಿ-ಡು-ವನು
ಮುಂದಿ-ಡು-ವುದು
ಮುಂದಿ-ಡು-ವೆನು
ಮುಂದಿನ
ಮುಂದಿ-ನ-ದನ್ನು
ಮುಂದಿ-ನದು
ಮುಂದಿ-ನದೇ
ಮುಂದಿ-ನ-ವ-ರಿಗೆ
ಮುಂದಿ-ರುವ
ಮುಂದಿ-ರುವು-ದನ್ನು
ಮುಂದು
ಮುಂದು-ವರಿ
ಮುಂದು-ವರಿದ
ಮುಂದು-ವರಿ-ದಂತೆ
ಮುಂದು-ವರಿ-ದಂತೆಲ್ಲ
ಮುಂದು-ವರಿ-ದರೂ
ಮುಂದು-ವರಿ-ದರೆ
ಮುಂದು-ವರಿ-ದಷ್ಟೂ
ಮುಂದು-ವರಿ-ದಿರು-ವೆವು
ಮುಂದು-ವರಿದು
ಮುಂದು-ವರಿಯ
ಮುಂದು-ವರಿ-ಯ-ಬೇಕು
ಮುಂದು-ವರಿ-ಯ-ಬೇಕೆಂದು
ಮುಂದು-ವರಿ-ಯ-ಲಾರಿರಿ
ಮುಂದು-ವರಿ-ಯ-ಲಾರೆವು
ಮುಂದು-ವರಿ-ಯಲು
ಮುಂದು-ವರಿ-ಯಿರಿ
ಮುಂದು-ವರಿ-ಯುತ್ತಾ
ಮುಂದು-ವರಿ-ಯುತ್ತಾನೆ
ಮುಂದು-ವರಿ-ಯುತ್ತಿದೆ
ಮುಂದು-ವರಿ-ಯುತ್ತಿ-ರುವ
ಮುಂದು-ವರಿ-ಯುತ್ತಿ-ರು-ವಿರಿ
ಮುಂದು-ವರಿ-ಯುತ್ತಿ-ರುವುದು
ಮುಂದು-ವರಿ-ಯುತ್ತೇವೆ
ಮುಂದು-ವರಿ-ಯು-ವಂತೆ
ಮುಂದು-ವರಿ-ಯು-ವರು
ಮುಂದು-ವರಿ-ಯು-ವು-ದಕ್ಕೆ
ಮುಂದು-ವರಿ-ಯು-ವುದು
ಮುಂದು-ವರಿ-ಯು-ವುದು-ನಮ್ಮ
ಮುಂದು-ವರಿ-ಯು-ವುದೇ
ಮುಂದು-ವರಿ-ಯು-ವುವು
ಮುಂದು-ವರಿ-ಸಿ-ದನು
ಮುಂದು-ವರಿ-ಸಿ-ದರೆ
ಮುಂದು-ವರಿ-ಸುತ್ತಾನೆ
ಮುಂದೆ
ಮುಂದೆಯೂ
ಮುಂದೊ
ಮುಂಬೆಳ-ಕಿ-ನಲ್ಲಿ
ಮುಕ್ಕಾಲು
ಮುಕ್ಕಾಲು-ಪಾಲು
ಮುಕ್ತ
ಮುಕ್ತ-ಗೊಳಿ
ಮುಕ್ತ-ಜೀವಿ
ಮುಕ್ತ-ಜೀವಿ-ಗಳು
ಮುಕ್ತ-ನನ್ನಾಗಿ
ಮುಕ್ತ-ನಾ-ಗಲು
ಮುಕ್ತ-ನಾಗಿದ್ದರೆ
ಮುಕ್ತ-ನಾಗಿದ್ದೆ
ಮುಕ್ತ-ನಾಗಿ-ರ-ಬೇಕು
ಮುಕ್ತ-ನಾ-ಗಿ-ರುವನೊ
ಮುಕ್ತ-ನಾ-ಗಿ-ರುವೆ
ಮುಕ್ತ-ನಾಗು
ಮುಕ್ತ-ನಾಗುತ್ತಾನೆ
ಮುಕ್ತ-ನಾಗುತ್ತೇನೆ
ಮುಕ್ತ-ನಾಗು-ವನು
ಮುಕ್ತ-ನಾಗು-ವ-ವರೆಗೆ
ಮುಕ್ತ-ನಾಗು-ವೆನು
ಮುಕ್ತ-ನಾದ
ಮುಕ್ತ-ನೆಂದು
ಮುಕ್ತ-ನೆನ್ನು-ವರೊ
ಮುಕ್ತನೇ
ಮುಕ್ತಯೇ
ಮುಕ್ತ-ರಾಗ-ಬೇಕಾ-ದರೆ
ಮುಕ್ತ-ರಾಗ-ಬೇಕು
ಮುಕ್ತ-ರಾಗ-ಲಾರಿರಿ
ಮುಕ್ತ-ರಾ-ಗಲು
ಮುಕ್ತ-ರಾಗಿ
ಮುಕ್ತ-ರಾಗಿ-ರು-ವರೊ
ಮುಕ್ತ-ರಾಗಿ-ರು-ವರೋ
ಮುಕ್ತ-ರಾಗುವ
ಮುಕ್ತ-ರಾಗು-ವ-ರೆಂಬು-ದನ್ನು
ಮುಕ್ತ-ರಾಗು-ವ-ವರೆಗೂ
ಮುಕ್ತ-ರಾಗು-ವ-ವರೆ-ವಿಗೂ
ಮುಕ್ತ-ರಾಗು-ವು-ದಕ್ಕೆ
ಮುಕ್ತ-ರಾಗು-ವುದು
ಮುಕ್ತ-ರಾಗು-ವೆವು
ಮುಕ್ತ-ರಾ-ದಂತೆ
ಮುಕ್ತ-ರಾ-ದರೆ
ಮುಕ್ತ-ರಾ-ದಾಗ
ಮುಕ್ತ-ರಿಗೆ
ಮುಕ್ತರು
ಮುಕ್ತ-ರೆಂದು
ಮುಕ್ತ-ವಾಗಿ
ಮುಕ್ತ-ವಾಗಿದ್ದರೆ
ಮುಕ್ತ-ವಾಗು
ಮುಕ್ತ-ವಾಗುವ
ಮುಕ್ತ-ವಾಗು-ವುದು
ಮುಕ್ತ-ವಾಗು-ವುದೇನೊ
ಮುಕ್ತ-ವಾದ
ಮುಕ್ತ-ವಾದು-ದೆಂದು
ಮುಕ್ತ-ವೆಂದು
ಮುಕ್ತಾತ್ಮ
ಮುಕ್ತಾತ್ಮ-ನ-ವರೆಗೆ
ಮುಕ್ತಾತ್ಮ-ನಾಗು-ವೆನು
ಮುಕ್ತಾತ್ಮ-ನಾದ
ಮುಕ್ತಾತ್ಮ-ನಿಗೂ
ಮುಕ್ತಾತ್ಮ-ನಿಗೆ
ಮುಕ್ತಾತ್ಮನು
ಮುಕ್ತಾತ್ಮ-ರೆಂದು
ಮುಕ್ತಿ
ಮುಕ್ತಿ-ಕೊಳ್ಳ-ಬಹುದು
ಮುಕ್ತಿ-ಗಲ್ಲ
ಮುಕ್ತಿಗೆ
ಮುಕ್ತಿಯ
ಮುಕ್ತಿ-ಯನ್ನು
ಮುಕ್ತಿ-ಯಲ್ಲ
ಮುಕ್ತಿ-ಯೆ-ಡೆಗೆ
ಮುಖ
ಮುಖ-ಗಳ
ಮುಖ-ಗಳು
ಮುಖ-ಗಳೂ
ಮುಖ-ಚರ್ಯೆ
ಮುಖದ
ಮುಖ-ದಿಂದ
ಮುಖ-ದೆ-ಡೆಗೆ
ಮುಖ-ಭಾವದ
ಮುಖ-ವನ್ನು
ಮುಖ-ವಯ-ವಾದ
ಮುಖ-ವಾಗಿ
ಮುಖ-ವಾಗಿದೆ
ಮುಖ-ವಾಗಿದ್ದರೆ
ಮುಖವೂ
ಮುಖ್ಯ
ಮುಖ್ಯ-ಕಾರಣ
ಮುಖ್ಯ-ಗುರಿಯೇ
ಮುಖ್ಯ-ಭಾ-ವನೆ
ಮುಖ್ಯವ
ಮುಖ್ಯ-ವನ್ನು
ಮುಖ್ಯ-ವಲ್ಲ
ಮುಖ್ಯ-ವಾಗಿ
ಮುಖ್ಯ-ವಾ-ಗಿತ್ತು
ಮುಖ್ಯ-ವಾಗಿದೆ
ಮುಖ್ಯ-ವಾಗಿ-ದೆಯೊ
ಮುಖ್ಯ-ವಾಗಿ-ರು-ವುದೆ
ಮುಖ್ಯ-ವಾದ
ಮುಖ್ಯ-ವಾದು-ದನ್ನು
ಮುಖ್ಯ-ವಾದುದು
ಮುಖ್ಯ-ವಾದುವು
ಮುಖ್ಯವೇ
ಮುಖ್ಯವೋ
ಮುಖ್ಯಾಭಿಪ್ರಾಯ
ಮುಗಿದ
ಮುಗಿ-ದರೆ
ಮುಗಿ-ದಾಗ
ಮುಗಿದು
ಮುಗಿ-ಯಿತು
ಮುಗಿಯು
ಮುಗಿಯು-ವುದು
ಮುಗಿ-ಲಿನ
ಮುಗಿಲಿ-ನಂತೆ
ಮುಗಿ-ಲಿ-ನಿಂದ
ಮುಗಿಲು-ಗಳಾಚೆ
ಮುಗಿಲು-ಗಳು
ಮುಗಿಸಿ
ಮುಗಿಸಿ-ದರು-ಒಂದೇ
ಮುಗಿ-ಸಿವೆ
ಮುಗಿಸುತ್ತಿ-ರು-ವೆನು
ಮುಗಿಸುತ್ತಿ-ರು-ವೆವು
ಮುಗಿ-ಸುವು-ದಕ್ಕಾಗಿ
ಮುಗಿ-ಸು-ವೆವು
ಮುಗ್ಧ
ಮುಚ್ಚಲು
ಮುಚ್ಚಲೆತ್ನಿಸ
ಮುಚ್ಚಿ
ಮುಚ್ಚಿ-ಕೊಂಡು
ಮುಚ್ಚಿ-ಕೊಳ್ಳು-ವಂತೆ
ಮುಚ್ಚಿ-ಕೊಳ್ಳು-ವುದು
ಮುಚ್ಚಿದ
ಮುಚ್ಚಿ-ದರೆ
ಮುಚ್ಚಿದೆ
ಮುಚ್ಚಿದ್ದನು
ಮುಚ್ಚಿ-ರ-ಬಹುದು
ಮುಚ್ಚಿ-ರುವ
ಮುಚ್ಚಿ-ರುವುದು
ಮುಚ್ಚಿ-ರುವೆ
ಮುಚ್ಚಿ-ಸುವ
ಮುಚ್ಚು
ಮುಚ್ಚುವ
ಮುಚ್ಚು-ವಳು
ಮುಚ್ಚು-ವೆವು
ಮುಟ್ಟ
ಮುಟ್ಟದೆ
ಮುಟ್ಟ-ಲಾರವು
ಮುಟ್ಟಲು
ಮುಟ್ಟಿ
ಮುಟ್ಟಿದ
ಮುಟ್ಟಿ-ದರೆ
ಮುಟ್ಟಿ-ದವು
ಮುಟ್ಟಿದೆ
ಮುಟ್ಟಿರು
ಮುಟ್ಟುತ್ತೀರಿ
ಮುಟ್ಟುವ
ಮುಟ್ಟು-ವರು
ಮುಟ್ಟುವ-ವರೆಗೂ
ಮುಟ್ಟು-ವ-ವರೆ-ವಿಗೂ
ಮುಟ್ಟು-ವು-ದಕ್ಕೆ
ಮುಟ್ಟು-ವುದು
ಮುಡಿಪಾಗಿಟ್ಟು
ಮುಡುಪು
ಮುತ್ತಾ-ಗು-ವುದು
ಮುತ್ತಾ-ತಂದಿ-ರಿಂದ
ಮುತ್ತಿದ
ಮುತ್ತಿನ
ಮುತ್ತಿ-ರಲಿ
ಮುತ್ತು
ಮುತ್ತು-ವು-ದಕ್ಕೆ
ಮುತ್ತು-ವುವು
ಮುದು-ಕನು
ಮುದ್ದಿ-ಸುವಳು
ಮುದ್ದೆ
ಮುದ್ದೆ-ಗಳಂತೆ
ಮುದ್ದೆ-ಯಾ-ಗಿಯೇ
ಮುದ್ರಿತ-ವಾಗಿ-ದೆಯೋ
ಮುದ್ರೆ-ಯನ್ನು
ಮುನಿ-ವರ-ರಿಂದ
ಮುನ್ನಡೆ-ಯನ್ನೂ
ಮುರಿದುಹೋಕ-ಬಲ್ಲ
ಮುರಿಯುತ್ತಿವೆ
ಮುಲ್ಲರ್
ಮುಳುಗಿ
ಮುಳುಗಿ-ಸಿತು
ಮುಳುಗುತ್ತಿದ್ದ
ಮುಳುಗು-ವು-ದಿಲ್ಲ
ಮುಳ್ಳಿನ
ಮುಷ್ಟಿ-ಯಿಂದ
ಮುಸು-ಕನ್ನು
ಮುಸು-ಕಿದ
ಮುಸುಕು-ವುದು
ಮೂಗನ್ನು
ಮೂಗಿನ
ಮೂಗಿ-ನಿಂದ
ಮೂಗು
ಮೂಡಿತು
ಮೂಡಿಸ-ಬಹುದು
ಮೂಡುತ್ತದೆ
ಮೂಡುವು
ಮೂಢ
ಮೂಢ-ನಂಬಿಕೆ
ಮೂಢ-ನಂಬಿ-ಕೆ-ಗಳ
ಮೂಢ-ನಂಬಿ-ಕೆ-ಗ-ಳನ್ನು
ಮೂಢ-ನಂಬಿ-ಕೆ-ಗ-ಳನ್ನೂ
ಮೂಢ-ನಂಬಿ-ಕೆ-ಗಳಲ್ಲಿ
ಮೂಢ-ನಂಬಿ-ಕೆ-ಗಳಾ-ಗಿದ್ದವು
ಮೂಢ-ನಂಬಿ-ಕೆ-ಗಳಿಂದ
ಮೂಢ-ನಂಬಿ-ಕೆ-ಗಳು
ಮೂಢ-ನಂಬಿ-ಕೆ-ಗಳು-ಇವು-ಗಳಿಂದ
ಮೂಢ-ನಂಬಿ-ಕೆ-ಗಳೂ
ಮೂಢ-ನಂಬಿ-ಕೆಗೆ
ಮೂಢ-ನಂಬಿ-ಕೆಯ
ಮೂಢ-ನಂಬಿ-ಕೆ-ಯಂತೆ
ಮೂಢ-ನಂಬಿ-ಕೆ-ಯನ್ನು
ಮೂಢ-ನಂಬಿ-ಕೆ-ಯನ್ನೂ
ಮೂಢ-ನಂಬಿ-ಕೆ-ಯ-ವ-ರಿಗೆ
ಮೂಢ-ನಂಬಿ-ಕೆ-ಯ-ವರು
ಮೂಢ-ನಂಬಿ-ಕೆ-ಯಿಂದ
ಮೂಢ-ಭಕ್ತಿ-ಯೇನೋ
ಮೂಢ-ಮತಿ-ಗಳ
ಮೂಢ-ರಾದ
ಮೂಢ-ರಿಂದ
ಮೂಢರು
ಮೂಢಸ್ಥಿತಿಗೆ
ಮೂಢಾ-ಚಾರದ
ಮೂಢಾ-ವಸ್ಥೆ
ಮೂರಕ್ಕೆ
ಮೂರನೆ
ಮೂರ-ನೆಯ
ಮೂರನೆ-ಯ-ದನ್ನು
ಮೂರನೆ-ಯ-ದ-ರಲ್ಲಿ
ಮೂರ-ನೆ-ಯ-ದಾಗಿ
ಮೂರನೆ-ಯ-ದಾದ
ಮೂರನೆ-ಯದು
ಮೂರನೆ-ಯದೆ
ಮೂರನೆ-ಯದೇ
ಮೂರನೆ-ಯ-ದೊಂದು
ಮೂರನೆ-ಯ-ವನು
ಮೂರನೇ
ಮೂರನ್ನು
ಮೂರನ್ನೂ
ಮೂರರ
ಮೂರು
ಮೂರು-ಸಾ-ವಿರ
ಮೂರ್ಖ
ಮೂರ್ಖ-ತನ
ಮೂರ್ಖನ
ಮೂರ್ಖ-ನಂತೆ
ಮೂರ್ಖ-ರಲ್ಲಿ
ಮೂರ್ಖ-ರಿ-ರು-ವರು
ಮೂರ್ಖರು
ಮೂರ್ತ-ರೂಪ-ವನ್ನು
ಮೂರ್ತಿ-ಯಾದ
ಮೂರ್ತೀಕ-ರಿಸುತ್ತಿ-ರು-ವಿರಿ
ಮೂರ್ಧಜ್ಯೋತಿಷಿ
ಮೂಲ
ಮೂಲಕ
ಮೂಲ-ಕ-ವಲ್ಲ
ಮೂಲ-ಕ-ವಲ್ಲದೆ
ಮೂಲ-ಕ-ವಾ-ಗಲಿ
ಮೂಲ-ಕ-ವಾ-ಗಲೀ
ಮೂಲ-ಕ-ವಾಗಿ
ಮೂಲ-ಕ-ವಾಗಿ-ಯಾ-ದರೂ
ಮೂಲ-ಕ-ವಾಗಿಯೆ
ಮೂಲ-ಕ-ವಾಗಿಯೇ
ಮೂಲ-ಕವೂ
ಮೂಲ-ಕವೆ
ಮೂಲ-ಕವೇ
ಮೂಲ-ಕಾರಣ
ಮೂಲ-ಕಾರ-ಣಕ್ಕೆ
ಮೂಲ-ಕಾರ-ಣ-ವನ್ನೇ
ಮೂಲಕ್ಕೆ
ಮೂಲಕ್ಕೇ
ಮೂಲಗೆ
ಮೂಲಗ್ರಂಥ-ಗಳು
ಮೂಲ-ಚೈ-ತನ್ಯ-ವಿಲ್ಲ
ಮೂಲ-ತತ್ತ್ವಕ್ಕೆ
ಮೂಲ-ತತ್ತ್ವ-ಗ-ಳನ್ನೂ
ಮೂಲ-ತತ್ತ್ವ-ಗಳೇ
ಮೂಲ-ತತ್ತ್ವದ
ಮೂಲ-ತತ್ತ್ವ-ವನ್ನು
ಮೂಲ-ತತ್ತ್ವ-ವಿದೆ
ಮೂಲ-ಪುರುಷರ
ಮೂಲ-ಪುರುಷರು
ಮೂಲ-ಭೂತ
ಮೂಲ-ಭೂತ-ಗಳಿಗೆ
ಮೂಲ-ರೂಪ
ಮೂಲ-ರೂಪಕ್ಕೆ
ಮೂಲ-ವನ್ನು
ಮೂಲ-ವಸ್ತು
ಮೂಲ-ವಸ್ತು-ವಾಗಿ-ರು-ವುದೊ
ಮೂಲ-ವಸ್ತು-ವಿರ-ಬೇಕೆಂಬ
ಮೂಲ-ವಸ್ತುವೂ
ಮೂಲ-ವಾಗಿ
ಮೂಲ-ವಾಗಿದೆ
ಮೂಲ-ವಾಗಿ-ರ-ಬೇಕು
ಮೂಲ-ವಾದ
ಮೂಲ-ವಾ-ದರೆ
ಮೂಲ-ವಿದೆ
ಮೂಲವು
ಮೂಲವೂ
ಮೂಲವೆ
ಮೂಲ-ವೆಂದು
ಮೂಲ-ವೆನ್ನು-ವರು
ಮೂಲ-ವೆನ್ನು-ವು-ದಕ್ಕೆ
ಮೂಲವೇ
ಮೂಲವೊ
ಮೂಲ-ವೊಂದೇ
ಮೂಲ-ಸಿದ್ಧಾಂತ-ಗಳಲ್ಲಿ
ಮೂಲಸ್ಥಾ-ನಕ್ಕೆ
ಮೂಲಸ್ಥಾನ-ವಾಗಿದೆ
ಮೂಲಸ್ಥಾನ-ವಾದ
ಮೂಲಸ್ಥಾನವೂ
ಮೂಲಸ್ಥಿತಿ
ಮೂಲಾ-ಧಾರ
ಮೂಲಾ-ಧಾರ-ಅದೇ
ಮೂಲಾ-ಧಾರದ
ಮೂಲಾ-ಧಾರ-ದಲ್ಲಿ
ಮೂಲಾ-ಧಾರ-ದಿಂದ
ಮೂಲಾ-ಧಾರ-ವನ್ನು
ಮೂಲಾ-ಧಾರ-ವೆನ್ನುತ್ತಾರೆ
ಮೂಲಾ-ವಸ್ಥೆಗೆ
ಮೂಲೆ
ಮೂಲೆ-ಗಳಿಗೆ
ಮೂಲೆಗೆ
ಮೂಲೆ-ಗೊತ್ತಿ
ಮೂಲೆ-ಮೂಲೆಗೂ
ಮೂಲೆ-ಯಲ್ಲಿ
ಮೂಲೆ-ಯಿಂದ
ಮೂಲೇ
ಮೂಳೆ
ಮೂಳೆಯ
ಮೂಳೆ-ಯಲ್ಲಿ
ಮೂಳೆ-ಯಾಗುವ
ಮೂಳೆ-ಯಾಗು-ವುದು
ಮೂವತ್ತಾರು
ಮೂವತ್ತು
ಮೂಸಿ
ಮೂಸಿ-ನೋಡುತ್ತ
ಮೃಗ
ಮೃಗಕ್ಕೆ
ಮೃಗ-ಗಳ
ಮೃಗ-ಗಳಂತೆ
ಮೃಗ-ಗಳಾಗಿ
ಮೃಗ-ಗಳಾ-ಗು-ವರು
ಮೃಗ-ಗಳಿ-ಗಿಂತ
ಮೃಗ-ಗಳಿಗೆ
ಮೃಗ-ಗಳಿ-ರುವೆಡೆ
ಮೃಗ-ಗಳು
ಮೃಗ-ಗಳೂ
ಮೃಗ-ಗಳೊ
ಮೃಗದ
ಮೃಗ-ದಲ್ಲಿ-ರ-ಬಹುದು
ಮೃಗ-ದ-ವರೆಗೆ
ಮೃಗ-ದಿಂದ
ಮೃಗ-ದೇಹ-ದಲ್ಲಿ
ಮೃಗ-ದೋ-ಪಾದಿ-ಯಲ್ಲಿ
ಮೃಗ-ವನ್ನು
ಮೃಗ-ವಾಗಿ
ಮೃಗ-ವಾದ
ಮೃಗ-ಸ-ಮಾನ-ರಾಗಿ
ಮೃಗೀಯ
ಮೃಗೀಯತೆ
ಮೃತ್ತಿಕೆ-ಯನ್ನು
ಮೃತ್ತಿಕೆ-ಯಿಂದ
ಮೃತ್ಯು
ಮೃತ್ಯು-ಗಳ
ಮೃತ್ಯು-ಪಾಶ-ದಿಂದ
ಮೃತ್ಯು-ಬಲೆ-ಯಲ್ಲಿ
ಮೃತ್ಯು-ವನ್ನು
ಮೃತ್ಯು-ವಶ-ರಾಗ-ಬೇಕು
ಮೃತ್ಯು-ವಶ-ರಾಗು-ವರು
ಮೃತ್ಯು-ವಿಗೆ
ಮೃತ್ಯು-ವಿನ
ಮೃತ್ಯು-ವಿ-ನಲ್ಲಿ
ಮೃತ್ಯು-ವಿ-ನಿಂದ
ಮೃತ್ಯುವು
ಮೃತ್ಯುವೇ
ಮೃತ್ಯು-ಶೋಕ-ಗಳಿಲ್ಲದ
ಮೃತ್ಯೋ
ಮೃದುದ್ರವ್ಯ
ಮೃದು-ಮಧುರ-ಭಾವನೆ-ಗಳನ್ನು
ಮೃದು-ಮಧ್ಯಾಧಿ-ಮಾತ್ರ
ಮೃದು-ಮಧ್ಯಾಧಿ-ಮಾತ್ರತ್ವಾತ್ತತೋಪಿ
ಮೃದು-ಮಾಡು-ವಷ್ಟು
ಮೃದು-ವಾಗಿ
ಮೆಚ್ಚ-ಲಿಲ್ಲ
ಮೆಚ್ಚಿ-ದನು
ಮೆಚ್ಚುತ್ತೇನೆ
ಮೆಚ್ಚು-ವರು
ಮೆಟ್ಟ-ಲನ್ನು
ಮೆಟ್ಟ-ಲಾಗಿ
ಮೆಟ್ಟ-ಲಿಗೆ
ಮೆಟ್ಟಲು
ಮೆಟ್ಟಲು-ಗ-ಳನ್ನೂ
ಮೆಟ್ಟಲು-ಗಳಾಗಿ
ಮೆಟ್ಟಲು-ಗಳು
ಮೆಟ್ಟಲು-ಗಳೆಲ್ಲ-ವನ್ನೂ
ಮೆಟ್ಟಲು-ಮೆಟ್ಟ-ಲಾಗಿ
ಮೆಟ್ಟಲೇ
ಮೆಟ್ಟಿ
ಮೆಟ್ಟಿ-ಕೊಂಡಿದೆ
ಮೆಟ್ಟಿ-ಲನ್ನು
ಮೆಟ್ಟಿ-ಲನ್ನೂ
ಮೆಟ್ಟಿ-ಲಿ-ನಿಂದಲೂ
ಮೆಟ್ಟಿಲು
ಮೆಡುಲ
ಮೆತ್ತಿ-ಕೊಂಡಿ-ರುವ
ಮೆತ್ತಿದ
ಮೆದು-ಳನ್ನು
ಮೆದುಳಿ
ಮೆದುಳಿಗೆ
ಮೆದುಳಿನ
ಮೆದುಳಿ-ನಲ್ಲಿ
ಮೆದುಳಿ-ನಲ್ಲಿ-ರುವ
ಮೆದುಳಿ-ನಿಂದ
ಮೆದುಳು
ಮೆರೆಯ-ಬಹುದು
ಮೆಲಕು
ಮೆಲುಕು
ಮೆಲುಕುತ್ತಿದ್ದ
ಮೆಲುದನಿ
ಮೆಲ್ಲಗೆ
ಮೇಘಕ್ಕೆ
ಮೇಘ-ಮಾಲೆ-ಯಂತೆ
ಮೇಚಿನ
ಮೇಜನ್ನು
ಮೇಜಿಗೂ
ಮೇಜಿನ
ಮೇಜಿ-ನಲ್ಲಿ
ಮೇಜು
ಮೇಧಾ-ವಂತ
ಮೇಧಾವಿ
ಮೇಧಾವಿ-ಗಳಾ
ಮೇಧಾವಿ-ಗ-ಳಾದ
ಮೇಧಾವಿ-ಗಳು
ಮೇಧಾವಿ-ಯಾಗಿ-ರ-ಬಹುದು
ಮೇಧಾ-ಶಕ್ತಿ
ಮೇರು-ವಷ್ಟು
ಮೇರೆ
ಮೇರೆ-ಗಳ
ಮೇರೆ-ಯನ್ನು
ಮೇರೆ-ಯಲ್ಲಿ
ಮೇರೆ-ಯಲ್ಲಿ-ರುವ
ಮೇಲಕ್ಕೆ
ಮೇಲ-ಮೇಲಕ್ಕೆ
ಮೇಲಲ್ಲ
ಮೇಲಲ್ಲದೆ
ಮೇಲಾ
ಮೇಲಾ-ಗಿ-ರ-ಬೇಕು
ಮೇಲಾ-ಗಿ-ರುವನು
ಮೇಲಾ-ಗಿ-ರು-ವಿರಾ
ಮೇಲಾ-ಗುತ್ತಿತ್ತು
ಮೇಲಾ-ಗುತ್ತಿದೆ
ಮೇಲಾಗು-ವು-ದಿಲ್ಲ
ಮೇಲಾದ
ಮೇಲಾ-ದದ್ದು
ಮೇಲಾ-ದರೆ
ಮೇಲಾ-ದಾಗ
ಮೇಲಾ-ದುದು
ಮೇಲಾ-ಯಿತು
ಮೇಲಿತ್ತು
ಮೇಲಿದೆ
ಮೇಲಿದ್ದ
ಮೇಲಿದ್ದರೂ
ಮೇಲಿದ್ದರೆ
ಮೇಲಿನ
ಮೇಲಿನಂದ
ಮೇಲಿನ-ದಕ್ಕೆ
ಮೇಲಿನದು
ಮೇಲಿನ-ವರೆ-ವಿಗೂ
ಮೇಲಿ-ನಿಂದ
ಮೇಲಿ-ರ-ಬಹುದು
ಮೇಲಿ-ರುವ
ಮೇಲಿ-ರುವ-ವ-ನಿಗೂ
ಮೇಲಿರು-ವವ-ರನ್ನು
ಮೇಲಿ-ರುವುದು
ಮೇಲಿಲ್ಲ
ಮೇಲು
ಮೇಲು-ತರದ
ಮೇಲು-ಮೇಲಕ್ಕೆ
ಮೇಲು-ವರ್ಗಕ್ಕೆ
ಮೇಲೂ
ಮೇಲೆ
ಮೇಲೆಂದು
ಮೇಲೆ-ಅದು
ಮೇಲೆ-ಈಗ
ಮೇಲೆತ್ತು-ವುದು
ಮೇಲೆದ್ದಾಗ
ಮೇಲೆದ್ದು
ಮೇಲೆ-ಬಿಡ-ಬೇಕು
ಮೇಲೆಯೂ
ಮೇಲೆಯೆ
ಮೇಲೆಯೇ
ಮೇಲೇ
ಮೇಲೇರಿ
ಮೇಲೇ-ಳುತ್ತವೆ
ಮೇಲೇ-ಳುತ್ತಿದೆ
ಮೇಲೇ-ಳುವ
ಮೇಲೇ-ಳು-ವಂತೆ
ಮೇಲೇ-ಳು-ವು-ದಕ್ಕೆ
ಮೇಲೇ-ಳು-ವುದು
ಮೇಲೇ-ಳು-ವೆನು
ಮೇಲೊಂದನ್ನು
ಮೇಲೊಂದು
ಮೇಲೊಬ್ಬರು
ಮೇಲೋ-ಗರ-ವಾ-ಯಿತು
ಮೇಲ್ಮಟ್ಟದ
ಮೇಲ್ವಿ-ಚಾರಣೆ
ಮೇಲ್ವಿ-ಚಾರ-ಣೆ-ಯಲ್ಲಿ-ರುವ
ಮೈಗೆ
ಮೈತ್ರೀ-ಕರು-ಣಾಮುದಿತೋಪೇಕ್ಷಾಣಾಂ
ಮೈತ್ರ್ಯಾದಿಷು
ಮೈದಾನ-ವಾ-ಗು-ವುದು
ಮೈಲಿ
ಮೈಲಿ-ಗಳ
ಮೈಲಿ-ಗಳ-ವರೆಗೆ
ಮೈಲಿ-ಗಳಷ್ಟು
ಮೈಲಿ-ಗಳಾಚೆ
ಮೈಲಿ-ಗಳು
ಮೈಲಿ-ದೂರ
ಮೈಸೂರು
ಮೊಟ್ಟೆ
ಮೊಟ್ಟೆ-ಯಿಂದ
ಮೊತ್ತ
ಮೊತ್ತದ
ಮೊತ್ತ-ದಂತೆ
ಮೊತ್ತ-ದೊಂದಿಗೆ
ಮೊತ್ತ-ವನ್ನು
ಮೊತ್ತ-ವಷ್ಠೆ
ಮೊತ್ತ-ವಾಗುತ್ತಿತ್ತು
ಮೊತ್ತವು
ಮೊತ್ತವೂ
ಮೊತ್ತವೆ
ಮೊತ್ತ-ವೆಂದು
ಮೊತ್ತ-ವೆಲ್ಲ
ಮೊತ್ತವೇ
ಮೊದ
ಮೊದಲ
ಮೊದ-ಲನೆ
ಮೊದ-ಲ-ನೆಯ
ಮೊದ-ಲ-ನೆ-ಯ-ದನ್ನು
ಮೊದ-ಲ-ನೆ-ಯ-ದ-ರಲ್ಲಿ
ಮೊದ-ಲ-ನೆ-ಯ-ದಾಗಿ
ಮೊದ-ಲ-ನೆ-ಯ-ದಾದ
ಮೊದ-ಲ-ನೆ-ಯದು
ಮೊದ-ಲ-ನೆ-ಯದೆ
ಮೊದ-ಲ-ನೆ-ಯ-ದೆಂದರೆ
ಮೊದ-ಲ-ನೆ-ಯದೇ
ಮೊದ-ಲ-ನೆ-ಯ-ವ-ನಿಗೆ
ಮೊದ-ಲ-ನೆ-ಯ-ವನು
ಮೊದ-ಲ-ನೆ-ಯ-ವರು
ಮೊದ-ಲನೇ
ಮೊದ-ಲಲ್ಲಿ
ಮೊದಲಾ
ಮೊದ-ಲಾಗಿ
ಮೊದ-ಲಾ-ಗು-ವುದು
ಮೊದ-ಲಾದ
ಮೊದ-ಲಿಗೆ
ಮೊದ-ಲಿನ
ಮೊದ-ಲಿನ-ದಕ್ಕಿಂತಲೂ
ಮೊದ-ಲಿನ-ದನ್ನು
ಮೊದ-ಲಿ-ನಿಂದ
ಮೊದ-ಲಿ-ನಿಂದಲೂ
ಮೊದ-ಲಿ-ನಿಂದಲೇ
ಮೊದ-ಲಿಲ್ಲದೆ
ಮೊದಲು
ಮೊದ-ಲು-ಮಾಡು-ವನು
ಮೊದಲೂ
ಮೊದಲೆ
ಮೊದ-ಲೆದ್ದ
ಮೊದಲೇ
ಮೊಬ್ಬು
ಮೊಮ್ಮಕ್ಕ
ಮೊರೆ
ಮೊಲ-ದಂತೆ
ಮೊಲವು
ಮೊಳಕೆ
ಮೊಳಕೆ-ಯೊಡೆಯ-ಲಾರವು
ಮೊಳ-ಗಿವೆ
ಮೊಳೆ-ಯಿತು
ಮೋಕ್ಷ
ಮೋಕ್ಷಕ್ಕೆ
ಮೋಕ್ಷ-ಮಾರ್ಗ-ವನ್ನು
ಮೋಕ್ಷ-ವನ್ನು
ಮೋಕ್ಷ-ವಾಗು-ವು-ದಿಲ್ಲ
ಮೋಘ
ಮೋಟು
ಮೋಡ
ಮೋಡ-ಗಳ
ಮೋಡ-ಗಳಾಚೆ
ಮೋಡ-ಗಳು
ಮೋಡ-ವಾಗಿ
ಮೋಡವು
ಮೋಡ-ವೊಂದು
ಮೋಸ
ಮೋಸ-ಗಾರರು
ಮೋಸ-ಗೊಂಡು
ಮೋಸದ
ಮೋಸ-ದಿಂದ
ಮೋಸ-ಮಾಡುವ
ಮೋಸ-ಮಾಡು-ವರು
ಮೋಸಸ್ನನ್ನು
ಮೋಸ-ಹೋಗಿ
ಮೋಹ
ಮೋಹಕ್ಕೆ
ಮೋಹ-ಗೊಳಿ-ಸದು
ಮೋಹ-ಗೊಳಿ-ಸದೆ
ಮೋಹದ
ಮೋಹ-ದಿಂದ
ಮೋಹ-ಪಾಶಕ್ಕೆ
ಮೋಹ-ಪೂರಿತ-ವಾಗಿ-ರ-ಬಹುದು
ಮೋಹ-ವಿಲ್ಲ
ಮೋಹ-ವಿಲ್ಲದೆ
ಮೌಢ್ಯ
ಮೌಢ್ಯ-ಗಳನ್ನೆಲ್ಲ
ಮೌಢ್ಯ-ದಿಂದ
ಮೌಢ್ಯ-ವೆನ್ನು-ವರು
ಮೌನ
ಮೌನ-ವಾಗಿ
ಮೌನ-ವಾಗಿದ್ದರೆ
ಮೌನ-ವಾದ
ಮೌನಿ-ಯಾಗುತ್ತಾನೆ
ಮ್ಯಾಕ್ಸ್
ಯ
ಯಂತಿ
ಯಂತೆ
ಯಂತ್ರ
ಯಂತ್ರಕ್ಕೆ
ಯಂತ್ರ-ಗಳ
ಯಂತ್ರ-ಗ-ಳನ್ನು
ಯಂತ್ರ-ಗಳಿಂದ
ಯಂತ್ರ-ಗಳು
ಯಂತ್ರ-ಗಳೂ
ಯಂತ್ರ-ಜಾಲ-ವಿದೆ
ಯಂತ್ರದ
ಯಂತ್ರ-ದಲ್ಲಿ
ಯಂತ್ರ-ವನ್ನಾಗಿ
ಯಂತ್ರ-ವನ್ನು
ಯಂತ್ರ-ವಲ್ಲ
ಯಂತ್ರವು
ಯಂತ್ರವೇ
ಯಕ್ಕೂ
ಯಕ್ತಿ-ಯುಕ್ತ-ವಲ್ಲ
ಯಕ್ಷರು
ಯಜ-ಮಾನ-ನಾಗ-ಕೂಡ-ದು-ದೇಹ
ಯಜ-ಮಾನರು
ಯಜ್ಞ-ಕಾಲ-ದಲ್ಲಿ
ಯಜ್ಞ-ದಲ್ಲಿ-ರುವನು
ಯಜ್ಞ-ವನ್ನು
ಯಜ್ಞ-ವೆಂಬುದು
ಯಜ್ಞ-ವೇದಿಕೆ-ಗ-ಳನ್ನು
ಯಜ್ಞ-ವೇದಿಕೆ-ಗಳಿವೆ
ಯತಾ-ವಾದಿ
ಯತಿ
ಯತಿ-ಗಳಿಗೆ
ಯತಿ-ಗಳು
ಯತ್ನ
ಯತ್ನ-ದಲ್ಲಿ
ಯತ್ನಿಸ-ಕೂಡದು
ಯತ್ನಿ-ಸ-ಬೇಕು
ಯತ್ನಿಸ-ಬೇಡಿ
ಯತ್ನಿಸಿ
ಯತ್ನಿ-ಸಿದ
ಯತ್ನಿಸಿ-ದರು
ಯತ್ನಿಸಿ-ದರೂ
ಯತ್ನಿಸಿ-ದರೆ
ಯತ್ನಿಸಿ-ದಾಗ
ಯತ್ನಿಸಿ-ದೊ-ಡ-ನೆಯೆ
ಯತ್ನಿಸು
ಯತ್ನಿ-ಸುತ್ತದೆ
ಯತ್ನಿಸುತ್ತವೆ
ಯತ್ನಿಸುತ್ತಾನೆ
ಯತ್ನಿಸುತ್ತಾರೆ
ಯತ್ನಿ-ಸುತ್ತಿದೆ
ಯತ್ನಿಸುತ್ತಿದ್ದ
ಯತ್ನಿಸುತ್ತಿದ್ದರೊ
ಯತ್ನಿಸುತ್ತಿದ್ದುದು
ಯತ್ನಿಸುತ್ತಿದ್ದೇವೆ
ಯತ್ನಿಸುತ್ತಿ-ರುವ
ಯತ್ನಿಸುತ್ತಿ-ರುವನು
ಯತ್ನಿಸುತ್ತಿ-ರು-ವರು
ಯತ್ನಿಸುತ್ತಿ-ರು-ವಿರಿ
ಯತ್ನಿಸುತ್ತಿ-ರುವುದೇ
ಯತ್ನಿಸುತ್ತಿ-ರು-ವೆನು
ಯತ್ನಿಸುತ್ತಿ-ರು-ವೆವು
ಯತ್ನಿಸುತ್ತಿವೆ
ಯತ್ನಿಸುತ್ತೀರಿ
ಯತ್ನಿಸುತ್ತೇನೆ
ಯತ್ನಿ-ಸುತ್ತೇವೆ
ಯತ್ನಿ-ಸುವ
ಯತ್ನಿಸು-ವಂತೆಯೇ
ಯತ್ನಿಸು-ವನು
ಯತ್ನಿಸು-ವರು
ಯತ್ನಿಸು-ವಾಗ
ಯತ್ನಿ-ಸು-ವು-ದಿಲ್ಲ
ಯತ್ನಿಸು-ವುದು
ಯತ್ನಿಸು-ವುದೇ
ಯತ್ನಿಸು-ವೆನು
ಯತ್ನಿ-ಸೋಣ
ಯತ್ನೋಭ್ಯಾಸಃ
ಯಥಾಭಿ-ಮತಧ್ಯಾನಾದ್ವಾ
ಯಥಾರ್ಥ
ಯಥಾರ್ಥಜ್ಞಾನ
ಯಥಾರ್ಥಸ್ಥಿತಿ
ಯದಾಗಿ
ಯದಾ-ದಷ್ಟೂ
ಯದು
ಯದೆ
ಯನ್ನಾಗಿ
ಯನ್ನು
ಯನ್ನೂ
ಯನ್ನೇ
ಯಮ
ಯಮ-ಧರ್ಮ-ರಾಯ
ಯಮ-ನಿಗೆ
ಯಮ-ನಿಯ-ಮಾಸನಪ್ರಾಣಾ-ಯಮಪ್ರತ್ಯಾ-ಹಾರ
ಯಮನು
ಯಮ-ರಾಜನ
ಯಮ-ಲೋ-ಕಕ್ಕೆ
ಯಮ-ವೆಂದರೆ
ಯಮಾಃ
ಯರ
ಯರು
ಯಲು
ಯಲ್ಲ
ಯಲ್ಲಿ
ಯಲ್ಲಿ-ಡು-ವಿರೊ
ಯಲ್ಲಿದೆ
ಯಲ್ಲಿಯೂ
ಯಲ್ಲಿಯೇ
ಯಲ್ಲಿ-ರ-ದಿದ್ದರೂ
ಯಲ್ಲಿ-ರುವ
ಯಲ್ಲಿ-ರುವನು
ಯಲ್ಲಿ-ರುವಾಗ
ಯಲ್ಲಿವೆ
ಯಲ್ಲೂ
ಯವ-ನ-ರಲ್ಲಿ
ಯವ-ನಿಗೂ
ಯವ-ರಾಗಿ-ರ-ಬಹುದು
ಯವು
ಯವೂ
ಯಶಸ್ವಿ-ಗಳಾ-ಗಿ-ರು-ವರು
ಯಶಸ್ವಿ-ಯಾಗ-ಬಹುದು
ಯಶಸ್ವಿ-ಯಾಗಿ
ಯಶಸ್ವಿಯಾ-ದಷ್ಟೂ
ಯಶಸ್ಸು
ಯಹೂದ್ಯ-ನಾ-ಗಿ-ರು-ವಂತೆ
ಯಹೂದ್ಯನು
ಯಹೋವ
ಯಹೋ-ವನ
ಯಹೋ-ವ-ನಿಗೆ
ಯಹೋ-ವನೂ
ಯಾಂತ್ರಿಕ-ವಾಗುತ್ತದೆ
ಯಾಗ
ಯಾಗ-ದಲ್ಲಿ
ಯಾಗ-ಬಹುದು
ಯಾಗ-ಯಜ್ಞ-ಗಳ
ಯಾಗ-ಯಜ್ಞ-ಗಳಿಂದ
ಯಾಗ-ಯಜ್ಞಾದಿ-ಗಳೇ
ಯಾಗಲಿ
ಯಾಗ-ಲಿಲ್ಲ
ಯಾಗ-ವನ್ನು
ಯಾಗಿ
ಯಾಗಿತ್ತು
ಯಾಗಿದೆ
ಯಾಗಿದ್ದರೆ
ಯಾಗಿಯೂ
ಯಾಗಿ-ರ-ಬಹುದು
ಯಾಗಿ-ರ-ಲಾರದು
ಯಾಗಿ-ರಲು
ಯಾಗಿ-ರು-ವುದು
ಯಾಗಿಲ್ಲ
ಯಾಗುತ್ತದೆ
ಯಾಗುತ್ತಾ
ಯಾಗು-ವದು
ಯಾಗು-ವನು
ಯಾಗು-ವು-ದನ್ನು
ಯಾಗು-ವುದು
ಯಾತನಾ-ಪೂರ್ಣ-ವಾಗಿಯೂ
ಯಾತ-ನಾಮ-ಯ-ವಾ-ಗಿ-ರುವುದು
ಯಾತನೆ
ಯಾತನೆ-ಗಳು
ಯಾತನೆಯ
ಯಾತನೆ-ಯನ್ನು
ಯಾತನೆ-ಯನ್ನುಂಟು
ಯಾತನೆ-ಯಾ-ಗಲಿ
ಯಾದ
ಯಾದದ್ದು
ಯಾದರೂ
ಯಾದರೆ
ಯಾದ-ವ-ಗಿರಿ
ಯಾದ-ವನು
ಯಾದವು
ಯಾದುದು
ಯಾಮ
ಯಾಮ-ಗಾಯತ್ರಿ
ಯಾಮದ
ಯಾಮ-ದಲ್ಲಿ
ಯಾಮಿ-ಯಾದ
ಯಾಯಿ-ಗಳು
ಯಾರ
ಯಾರದು
ಯಾರನ್ನು
ಯಾರನ್ನೂ
ಯಾರಲ್ಲಿ
ಯಾರಲ್ಲಿಗೆ
ಯಾರಲ್ಲಿ-ಯಾ-ದರೂ
ಯಾರಲ್ಲಿಯೂ
ಯಾರ-ವರು
ಯಾರಾದ-ರಾ-ಗಲಿ
ಯಾರಾದ-ರಾಗ-ಲಿ-ಕಲಿಯ-ಬೇಕಾದ
ಯಾರಾ-ದರೂ
ಯಾರಿ
ಯಾರಿಂದ
ಯಾರಿಂದಲೂ
ಯಾರಿ-ಗಾಗಿ
ಯಾರಿ-ಗಾ-ದರೂ
ಯಾರಿ-ಗಿದೆ
ಯಾರಿಗೂ
ಯಾರಿಗೆ
ಯಾರಿಗೊ
ಯಾರಿಬ್ಬರೂ
ಯಾರಿ-ರು-ವು-ದ-ರಿಂದ
ಯಾರು
ಯಾರೂ
ಯಾರೇ
ಯಾರೊ
ಯಾರೊಂದಿಗೆ
ಯಾರೋ
ಯಾವ
ಯಾವ-ನನ್ನು
ಯಾವ-ನಾ-ದರೂ
ಯಾವ-ನಿಗೆ
ಯಾವನು
ಯಾವನೊ
ಯಾವಾ
ಯಾವಾಗ
ಯಾವಾ-ಗ-ಲಾ-ದರೂ
ಯಾವಾ-ಗಲೂ
ಯಾವು
ಯಾವುದ
ಯಾವು-ದಕ್ಕಾಗಿ
ಯಾವು-ದಕ್ಕೂ
ಯಾವು-ದಕ್ಕೆ
ಯಾವು-ದನ್ನಾ-ದರೂ
ಯಾವು-ದನ್ನು
ಯಾವು-ದನ್ನೂ
ಯಾವು-ದನ್ನೊ
ಯಾವು-ದರ
ಯಾವು-ದ-ರದು
ಯಾವು-ದ-ರಲ್ಲಿಯೂ
ಯಾವು-ದ-ರಿಂದ
ಯಾವು-ದ-ರಿಂದಲೂ
ಯಾವು-ದ-ರಿಂದಾ-ಯಿತು
ಯಾವು-ದ-ರೊಂದಿಗೆ
ಯಾವುದಾ
ಯಾವು-ದಾದ
ಯಾವು-ದಾದರೂ
ಯಾವು-ದಾದ-ರೇನು
ಯಾವು-ದಾದ-ರೊಂದು
ಯಾವು-ದಿದೆ
ಯಾವು-ದಿರು-ವುದೋ
ಯಾವುದು
ಯಾವು-ದುಎ
ಯಾವುದೂ
ಯಾವುದೆಂದರೆ
ಯಾವುದೆಂದರೆ-ಒಳ್ಳೆಯ
ಯಾವು-ದೆಲ್ಲ
ಯಾವುದೇ
ಯಾವುದೊ
ಯಾವು-ದೊಂದನ್ನೂ
ಯಾವು-ದೊಂದರ
ಯಾವು-ದೊಂದು
ಯಾವು-ದೊಂದೂ
ಯಾವುದೋ
ಯಾವುವು
ಯಾವುವೂ
ಯಿಂದ
ಯಿಂದಲೂ
ಯಿತು
ಯಿದೆ
ಯಿರುತ್ತದೆ
ಯಿಲ್ಲ
ಯಿಲ್ಲದೆ
ಯಿಸಿ
ಯಿಸು-ವುವು
ಯು
ಯುಕ್ತ-ವಾಗಿ
ಯುಕ್ತ-ವಾಗಿಯೆ
ಯುಕ್ತಿ
ಯುಕ್ತಿ-ಗಳ
ಯುಕ್ತಿ-ಗ-ಳನ್ನು
ಯುಕ್ತಿ-ಗಳು
ಯುಕ್ತಿ-ಗಾ-ಗಲೀ
ಯುಕ್ತಿ-ಗಿಂತ
ಯುಕ್ತಿಗೆ
ಯುಕ್ತಿ-ಜನಿತ-ವಾದು-ದಲ್ಲ
ಯುಕ್ತಿ-ದೃಷ್ಟಿಗೆ
ಯುಕ್ತಿ-ಪೂರಿತ
ಯುಕ್ತಿ-ಪೂರಿತ-ವಾಗಿ
ಯುಕ್ತಿ-ಪೂರಿತ-ವಾಗಿ-ರ-ಬೇಕು
ಯುಕ್ತಿ-ಪೂರಿತ-ವಾಗಿ-ರು-ವುದು
ಯುಕ್ತಿ-ಪೂರಿತ-ವಾ-ಗಿಲ್ಲ
ಯುಕ್ತಿ-ಪೂರಿತ-ವಾ-ದುದು
ಯುಕ್ತಿ-ಪೂರ್ವ-ಕ-ವಾಗಿ
ಯುಕ್ತಿ-ಪೂರ್ವ-ಕ-ವಾದ
ಯುಕ್ತಿಪ್ರಮಾಣ-ವನ್ನು
ಯುಕ್ತಿ-ಬದ್ಧ
ಯುಕ್ತಿ-ಬದ್ಧ-ವಾಗಿ
ಯುಕ್ತಿ-ಬದ್ಧ-ವಾಗಿ-ರುವಂತಿದೆ
ಯುಕ್ತಿ-ಬದ್ಧ-ವಾದ
ಯುಕ್ತಿ-ಮೂಲ-ಕ-ವಾಗಿ
ಯುಕ್ತಿಯ
ಯುಕ್ತಿ-ಯನ್ನು
ಯುಕ್ತಿ-ಯನ್ನೂ
ಯುಕ್ತಿ-ಯನ್ನೆಂದಿಗೂ
ಯುಕ್ತಿ-ಯಲ್ಲ
ಯುಕ್ತಿ-ಯಾಗಿ
ಯುಕ್ತಿ-ಯಿಂದ
ಯುಕ್ತಿಯು
ಯುಕ್ತಿ-ಯುಕ್ತ-ವಲ್ಲ
ಯುಕ್ತಿ-ಯುಕ್ತ-ವಾಗಿದೆ
ಯುಕ್ತಿ-ಯುಕ್ತ-ವಾಗಿ-ದೆಯೊ
ಯುಕ್ತಿ-ಯುಕ್ತ-ವಾದ
ಯುಕ್ತಿ-ಯುಕ್ತ-ವಾದುದು
ಯುಕ್ತಿಯೂ
ಯುಕ್ತಿ-ಯೆಂಬ
ಯುಕ್ತಿ-ಯೆಲ್ಲ
ಯುಕ್ತಿಯೇ
ಯುಕ್ತಿ-ವಂತ-ರಿಗೆ
ಯುಕ್ತಿ-ಸಮ್ಮತ-ವಾಗಿಯೂ
ಯುಕ್ತಿ-ಸರಣಿ
ಯುಗ-ಚಕ್ರದ
ಯುಗದ
ಯುಗ-ಯುಗಾಂತರ-ಗಳ-ವರೆ-ವಿಗೂ
ಯುತ್ತಾ
ಯುತ್ತಿ-ರು-ವಿರಿ
ಯುತ್ತಿ-ರುವುದು
ಯುದ್ಧ-ಗಳೆಲ್ಲ
ಯುಧಿಷ್ಠಿರ
ಯುವ
ಯುವಕ
ಯುವ-ಕನ
ಯುವ-ಕ-ನನ್ನು
ಯುವ-ಕ-ನಾ-ಗು-ವುದು
ಯುವ-ಕನು
ಯುವ-ಕ-ನೊಬ್ಬ
ಯುವ-ಕ-ನೊಬ್ಬನು
ಯುವ-ಕರೂ
ಯುವತಿ
ಯುವ-ತಿ-ಯಾದ
ಯುವಷ್ಟು
ಯುವು-ದಿಲ್ಲ
ಯೂನಿ-ವರ್ಸಲಿಸ್ಟ್
ಯೂಫ್ರೇಟಿಸ್
ಯೂರೋಪಿನ
ಯೂರೋಪಿ-ನಲ್ಲಿ
ಯೂರೋಪು
ಯೂರೋಪ್
ಯೂವು-ದೊಂದೂ
ಯೆಂದು
ಯೆಲ್ಲ
ಯೆಹೂದ್ಯರ
ಯೆಹೂದ್ಯ-ರಂತೆ
ಯೆಹೂದ್ಯರು
ಯೆಹೂದ್ಯರೂ
ಯೆಹೂದ್ಯರೆ
ಯೇ
ಯೇಸುವು
ಯೊಂದಿಗೆ
ಯೊಂದು
ಯೊಂದೂ
ಯೊಬ್ಬ
ಯೊಬ್ಬ-ರಲ್ಲಿಯೂ
ಯೊಬ್ಬ-ರಿಗೂ
ಯೊಬ್ಬರೂ
ಯೋಗ
ಯೋಗಕ್ಕೂ
ಯೋಗಕ್ಕೆ
ಯೋಗಕ್ಕೋಸುಗ-ವಾಗಿ
ಯೋಗಕ್ಷೇಮಕ್ಕೆ
ಯೋಗ-ಗಳ
ಯೋಗ-ಗ-ಳನ್ನು
ಯೋಗ-ಗಳನ್ನೆಲ್ಲ
ಯೋಗ-ಗಳಲ್ಲಿ
ಯೋಗ-ಗಳಲ್ಲೆಲ್ಲಾ
ಯೋಗ-ಗಳಿಗೂ
ಯೋಗ-ಗಳೂ
ಯೋಗದ
ಯೋಗ-ದರ್ಶನದ
ಯೋಗ-ದಲ್ಲಿ
ಯೋಗ-ದಿಂದ
ಯೋಗ-ವನ್ನು
ಯೋಗ-ವನ್ನೇ
ಯೋಗವು
ಯೋಗ-ವೆಂದರೆ
ಯೋಗ-ವೆಂದಿಗೂ
ಯೋಗ-ವೆಂದು
ಯೋಗ-ವೆಂಬ
ಯೋಗವೇ
ಯೋಗ-ಶಾಸ್ತ್ರ
ಯೋಗ-ಶಾಸ್ತ್ರದ
ಯೋಗ-ಶಾಸ್ತ್ರ-ದಲ್ಲಿ
ಯೋಗ-ಶಾಸ್ತ್ರ-ವಿರು-ವುದು
ಯೋಗಶ್ಚಿತ್ತ-ವೃತ್ತಿ
ಯೋಗ-ಸಹಿತ-ವಾಗಿರ-ಬೇಕು
ಯೋಗ-ಸಾಧನೆ-ಗಳನ್ನು
ಯೋಗ-ಸಿದ್ಧಾಂತದ
ಯೋಗ-ಸಿದ್ಧಾಂತವು
ಯೋಗ-ಸಿದ್ಧಿ-ಯನ್ನು
ಯೋಗ-ಸೂತ್ರ-ಗಳು
ಯೋಗ-ಸೂತ್ರದ
ಯೋಗ-ಸೂತ್ರವೇ
ಯೋಗಾಂಗ-ಗಳ
ಯೋಗಾಂಗಾನುಷ್ಠಾ-ನಾದ-ಶುದ್ಧಿಕ್ಷಯೇ
ಯೋಗಾಗ್ನಿಯು
ಯೋಗಾ-ಚಾರ್ಯರು
ಯೋಗಾನುಶಾಸ-ನಮ್
ಯೋಗಾಭ್ಯಾಸ
ಯೋಗಾಭ್ಯಾಸಕ್ಕೆ
ಯೋಗಾಭ್ಯಾ-ಸದ
ಯೋಗಾಭ್ಯಾಸವು
ಯೋಗಾಭ್ಯಾಸವೂ
ಯೋಗಿ
ಯೋಗಿ-ಗಳ
ಯೋಗಿ-ಗಳಾ
ಯೋಗಿ-ಗಳಾ-ಗ-ಬೇಕು
ಯೋಗಿ-ಗಳಾ-ಗ-ಬೇಕೆಂದು
ಯೋಗಿ-ಗಳಾ-ಗ-ಲಾರರು
ಯೋಗಿ-ಗಳಾ-ಗ-ಲಾರ-ರೆಂದು
ಯೋಗಿ-ಗಳಾ-ಗಲು
ಯೋಗಿ-ಗಳಾ-ಗಿದ್ದರು
ಯೋಗಿ-ಗಳಾ-ಗು-ವಿರಿ
ಯೋಗಿ-ಗಳಾ-ದರೆ
ಯೋಗಿ-ಗಳಾ-ದಾಗ
ಯೋಗಿ-ಗಳಿಗೆ
ಯೋಗಿ-ಗಳಿ-ರು-ವರೊ
ಯೋಗಿ-ಗಳು
ಯೋಗಿ-ಗಳೆ
ಯೋಗಿಗೆ
ಯೋಗಿ-ನಸ್ತ್ರಿ-ವಿಧ-ಮಿತ-ರೇಷಾಮ್
ಯೋಗಿಯ
ಯೋಗಿ-ಯನ್ನು
ಯೋಗಿ-ಯಲ್ಲಿ
ಯೋಗಿ-ಯ-ವರೆಗೆ
ಯೋಗಿ-ಯಾಗ-ಬೇಕೆಂದು
ಯೋಗಿ-ಯಾಗ-ಲಾರ
ಯೋಗಿ-ಯಾಗಿ-ರ-ಲಿಲ್ಲ
ಯೋಗಿಯು
ಯೋಗಿ-ಸುತ್ತಿ-ರು-ವರು
ಯೋಗಿ-ಸು-ವುದು
ಯೋಗ್ಯ
ಯೋಗ್ಯತಾ
ಯೋಗ್ಯ-ತೆಗೆ
ಯೋಗ್ಯ-ತೆ-ಯನ್ನು
ಯೋಗ್ಯ-ನಲ್ಲ
ಯೋಗ್ಯ-ನಾದ-ವನು
ಯೋಗ್ಯರ
ಯೋಗ್ಯ-ರಲ್ಲ
ಯೋಗ್ಯ-ರಲ್ಲದ
ಯೋಗ್ಯ-ರಾಗ-ಬೇಕಾ-ದರೆ
ಯೋಗ್ಯ-ರಾ-ದಂತಹ-ವರು
ಯೋಗ್ಯ-ರಾದ-ವರೆಂದರೆ
ಯೋಗ್ಯರೊ
ಯೋಗ್ಯ-ವಲ್ಲ
ಯೋಗ್ಯ-ವಲ್ಲದ
ಯೋಗ್ಯ-ವಾಗಿದೆ
ಯೋಗ್ಯ-ವಾಗಿ-ರುತ್ತವೆ
ಯೋಗ್ಯ-ವಾ-ಗಿ-ರು-ವಂತಹ
ಯೋಗ್ಯ-ವಾಗಿವೆ
ಯೋಗ್ಯ-ವಾಗು-ವಂತೆ
ಯೋಗ್ಯ-ವಾ-ಗು-ವುದು
ಯೋಗ್ಯ-ವಾದ
ಯೋಗ್ಯ-ವಾದ-ದೆಲ್ಲ
ಯೋಗ್ಯ-ವಾ-ದಷ್ಟು
ಯೋಗ್ಯ-ವಾದು-ದನ್ನು
ಯೋಗ್ಯ-ವೆಂದು
ಯೋಚನೆ
ಯೋಚಿಸ-ಬಹುದು
ಯೋಚಿಸ-ಬೇಕಾ-ದರೆ
ಯೋಚಿ-ಸ-ಬೇಕು
ಯೋಚಿಸ-ಬೇಡಿ
ಯೋಚಿಸ-ಲಾರ
ಯೋಚಿಸ-ಲಾರೆವು
ಯೋಚಿ-ಸಲು
ಯೋಚಿ-ಸಲೆತ್ನಿಸಿ
ಯೋಚಿ-ಸಲೇ
ಯೋಚಿಸಿ
ಯೋಚಿಸಿ-ದಾಗ
ಯೋಚಿ-ಸಿದೆ
ಯೋಚಿಸಿದ್ದೆನೊ
ಯೋಚಿಸು
ಯೋಚಿಸುತ್ತಲೇ
ಯೋಚಿ-ಸುತ್ತಿ-ರುವನು
ಯೋಚಿಸುತ್ತೇನೆ
ಯೋಚಿ-ಸುತ್ತೇವೆ
ಯೋಚಿಸುವ
ಯೋಚಿಸು-ವಷ್ಟು
ಯೋಚಿಸು-ವಿರಿ
ಯೋಚಿ-ಸು-ವು-ದಿಲ್ಲ
ಯೋಚಿಸು-ವುದೇ
ಯೋಚಿಸು-ವೆವು
ಯೋಜಕ-ವೆಂದೂ
ಯೋಜನೆ
ಯೋಜನೆ-ಗಳೆಲ್ಲ
ಯೋಜನೆ-ಯನ್ನು
ಯೋಜನೆ-ಯಿದೆ
ಯೋಜ-ನೆಯೆ
ಯೋಣ
ಯೌವನ
ಯೌವ-ನದ
ಯೌವನ-ದಲ್ಲಿ
ರ
ರಂಗ
ರಂಗ-ಭೂಮಿ
ರಂಗ-ಭೂಮಿಗೆ
ರಂಗ-ಭೂಮಿಯ
ರಂಜ-ನೆಯೇ
ರಂತೆ
ರಂತೆಯೆ
ರಂತೆಯೇ
ರಂದು
ರಂಧ್ರ
ರಂಧ್ರ-ವಿದೆ
ರಕ್ಕಸ-ರಂತೆ
ರಕ್ತ
ರಕ್ತ-ಗತ-ಮಾಡಿ-ಕೊಳ್ಳುವ
ರಕ್ತದ
ರಕ್ತ-ದಲ್ಲಿ
ರಕ್ತ-ದಾಹ
ರಕ್ತ-ನಾಳ-ಗಳಲ್ಲೆಲ್ಲ
ರಕ್ತ-ಪಾತ
ರಕ್ತ-ವನ್ನು
ರಕ್ತವು
ರಕ್ತ-ಸಂಬಂಧಿ
ರಕ್ಷಕ
ರಕ್ಷಕ-ರಾಗು-ವರು
ರಕ್ಷಣೆ
ರಕ್ಷಣೆಗೆ
ರಕ್ಷಿ-ಸ-ಬೇಕು
ರಕ್ಷಿಸ-ಲಾರದು
ರಕ್ಷಿ-ಸಲು
ರಕ್ಷಿಸಿ-ಕೊಂಡಿದ್ದರೆ
ರಕ್ಷಿಸಿ-ದನು
ರಕ್ಷಿ-ಸಿಲ್ಲ
ರಕ್ಷಿ-ಸು-ವು-ದಕ್ಕೆ
ರಚನಾತ್ಮಕ-ವಾಗಿ
ರಚನಾತ್ಮಕ-ವಾ-ದುದು
ರಚನಾ-ಸಿದ್ಧಾಂತವು
ರಚನೆ
ರಚನೆ-ಯಲ್ಲಿ
ರಚಿತ-ವಾಗಿದೆ
ರಚಿಸ
ರಚಿಸಿ-ದರು
ರಚಿಸಿ-ದರೆ
ರಚಿಸುತ್ತಾರೆ
ರಚಿ-ಸುತ್ತಿದೆಯೊ
ರಚಿ-ಸುತ್ತಿ-ರುವುದು
ರಚಿಸು-ವಂತಹ
ರಚಿಸು-ವುದೊ
ರಜಸ್ಸು
ರಜಸ್ಸೆ
ರಜೋ-ಗುಣ
ರಜ್ಜು
ರಣ
ರಣ-ರಂಗ
ರಣ-ರಂಗ-ದಲ್ಲಿ
ರಣ-ವಾಗಿ
ರಣಾಂಗಣ
ರಥ
ರಥಕ್ಕೆ
ರಥ-ದಲ್ಲಿ
ರಥಿ
ರದ
ರನ್ನಾಗಿ
ರನ್ನು
ರನ್ನೂ
ರಬಹುದು
ರಬೇಕು
ರಭಸಕ್ಕೆ
ರಮ್ಯ
ರಲಿ
ರಲಿಲ್ಲ
ರಲ್ಲವೊ
ರಲ್ಲಿ
ರವಾನಿಸಲ್ಪಟ್ಟ
ರವಿ-ಕಿರಣ-ದಿಂದ
ರಶ್ಮಿಯ
ರಸ-ವನ್ನು
ರಸ-ವನ್ನೆಲ್ಲಾ
ರಸಾಯನ
ರಸಾಯ-ನ-ಶಾಸ್ತ್ರಜ್ಞನು
ರಸಾಯ-ನ-ಶಾಸ್ತ್ರಜ್ಞಾನ
ರಸಾಯ-ನ-ಶಾಸ್ತ್ರವೂ
ರಸ್ತೆ-ಗಳು
ರಸ್ತೆಗೆ
ರಸ್ತೆಯ
ರಸ್ತೆ-ಯನ್ನು
ರಸ್ತೆ-ಯಲ್ಲಿ
ರಸ್ಥ-ರಿಗೆ
ರಹಸ್ಯ
ರಹಸ್ಯಕ್ಕೆ
ರಹಸ್ಯ-ಗ-ಳನ್ನು
ರಹಸ್ಯ-ಗಳನ್ನೆಲ್ಲ
ರಹಸ್ಯ-ತಮ
ರಹಸ್ಯದ
ರಹಸ್ಯ-ದೊ-ಡನೆ
ರಹಸ್ಯ-ವನ್ನಾಗಿ
ರಹಸ್ಯ-ವನ್ನು
ರಹಸ್ಯ-ವನ್ನೆಲ್ಲ
ರಹಸ್ಯ-ವಾಗಿ
ರಹಸ್ಯ-ವಾದ
ರಹಸ್ಯ-ವಿದ್ಯೆ
ರಹಸ್ಯವು
ರಹಸ್ಯವೂ
ರಹಸ್ಯವೇ
ರಹಸ್ಯಾ
ರಹಿತ
ರಹಿತ-ನಾದ
ರಹಿತನೋ
ರಹಿತ-ವಾದು-ದ-ರೊಂದಿಗೆ
ರಹಿತವೂ
ರಾಕ್ಷಸ-ನಾಗುತ್ತಾನೆ
ರಾಕ್ಷಸರ
ರಾಕ್ಷಸ-ರನ್ನು
ರಾಕ್ಷಸರಿ
ರಾಕ್ಷಸರಿದ್ದರು
ರಾಕ್ಷಸೀ
ರಾಕ್ಷಸೀ-ಕೃತ್ಯ
ರಾಕ್ಷಸೀಯ
ರಾಗ
ರಾಗಃ
ರಾಗ-ಬೇಕಾ-ದರೆ
ರಾಗ-ಬೇಡಿ
ರಾಗ-ಲಾರರು
ರಾಗ-ವೆನ್ನು-ವುದು
ರಾಗಿ
ರಾಗಿದ್ದರೂ
ರಾಗಿ-ರಲಿ
ರಾಗಿ-ರು-ವರು
ರಾಗಿ-ರು-ವಿರಿ
ರಾಗಿ-ರು-ವುದು
ರಾಗಿ-ರು-ವೆವು
ರಾಗುತ್ತಾರೆ
ರಾಗು-ವರು
ರಾಗು-ವಿರಿ
ರಾಗು-ವೆವು
ರಾಜ
ರಾಜ-ಕೀಯ
ರಾಜನ
ರಾಜ-ನಂತೆ
ರಾಜ-ನನ್ನು
ರಾಜ-ನಲ್ಲಿಗೆ
ರಾಜ-ನಾ-ಗಿ-ರುವನು
ರಾಜ-ನಾಗು
ರಾಜ-ನಾದ
ರಾಜನು
ರಾಜ-ನೊಬ್ಬನು
ರಾಜ-ಯೋಗ
ರಾಜ-ಯೋ-ಗಕ್ಕೆ
ರಾಜ-ಯೋಗದ
ರಾಜ-ಯೋಗ-ದಲ್ಲಿ
ರಾಜ-ಯೋಗ-ದಿಂದಾಗಲಿ
ರಾಜ-ಯೋಗ-ದೊಂದಿಗೆ
ರಾಜ-ಯೋಗ-ವನ್ನು
ರಾಜ-ಯೋಗವು
ರಾಜ-ಯೋಗ-ವೆಂದು
ರಾಜ-ಯೋಗಿ
ರಾಜ-ಯೋಗಿಯ
ರಾಜ-ರಿಗೆ
ರಾಜರು
ರಾಜ-ಸಿಕ
ರಾಜ-ಹಂಸದ
ರಾಜಾಧಿ-ಕಾರಿ-ಗಳು
ರಾಜಿ
ರಾಜಿ-ಮಾ-ಡಲು
ರಾಜಿ-ಮಾಡಿ-ಕೊಳ್ಳದೆ
ರಾಜಿ-ಮಾಡಿ-ಕೊಳ್ಳು
ರಾಜಿ-ಮಾಡಿ-ಕೊಳ್ಳು-ವಂತಹ
ರಾಜಿ-ಮಾಡಿ-ಸಲು
ರಾಜಿ-ಮಾಡಿ-ಸುತ್ತದೆ
ರಾಜ್ಯ
ರಾಜ್ಯ-ದೊ-ಳಗೆ
ರಾಜ್ಯ-ಭಾರ
ರಾಜ್ಯ-ವನ್ನಾ-ದರೂ
ರಾತ್ಮ
ರಾತ್ರಿ
ರಾತ್ರಿಗೆ
ರಾತ್ರಿ-ಯಲ್ಲಿ
ರಾತ್ರಿ-ಯಾಗುತ್ತಿ-ರುವಾಗ
ರಾತ್ರೆ
ರಾದರು
ರಾದರೆ
ರಾಮ-ಕೃಷ್ಣ
ರಾರಾಜಿ-ಸುತ್ತಿ-ರುವನು
ರಾರಾಜಿ-ಸುತ್ತಿ-ರುವುದೊ
ರಾರು
ರಾಳಕ್ಕೆ
ರಾಳ-ದಲ್ಲಿದೆ
ರಾಶಿ
ರಾಶಿ-ಗಳ
ರಾಶಿ-ಗ-ಳನ್ನು
ರಾಶಿಯ
ರಾಶಿ-ಯಂತೆ
ರಾಶಿ-ಯನ್ನು
ರಾಶಿ-ಯಲ್ಲಿ
ರಾಶಿ-ಯಿಂದ
ರಾಶಿ-ಯೊಂದಿಗೆ
ರಾಶಿ-ಯೊಂದೆ
ರಾಶಿ-ಯೊ-ಳಗೆ
ರಾಷ್ಟ್ರ-ಗಳು
ರಾಷ್ಟ್ರೀಯ
ರಾಸಾಯ-ನಿಕ
ರಾಸಾಯ-ನಿಕ-ರಿಂದ
ರಾಸಾಯ-ನಿಕರು
ರಿಂದ
ರಿಂದಲೂ
ರಿಂದಾಗುವ
ರಿಗೂ
ರಿಗೆ
ರಿದ್ದರು
ರಿಸಿ-ರುವ
ರೀತಿ
ರೀತಿ-ಗಳಲ್ಲಿ
ರೀತಿ-ನೀತಿ
ರೀತಿಯ
ರೀತಿ-ಯಂತೆ
ರೀತಿ-ಯನ್ನು
ರೀತಿ-ಯ-ಲಿಲ್ಲ
ರೀತಿ-ಯಲ್ಲ
ರೀತಿ-ಯಲ್ಲಿ
ರೀತಿ-ಯಲ್ಲಿದೆ
ರೀತಿ-ಯಲ್ಲಿಯೇ
ರೀತಿ-ಯಲ್ಲೆ
ರೀತಿ-ಯವು
ರೀತಿ-ಯಾಗಿ
ರೀತಿ-ಯಾದ
ರೀತಿ-ಯಿಂದ
ರೀತಿಯೆ
ರೀತಿಯೇ
ರೀತ್ಯಾ
ರುಗುತ್ತಾರೆ
ರುಗು-ವುವು
ರುಚಿ
ರುಚಿ-ಕರ-ವಾದ
ರುಚಿ-ನೋ-ಡುವ
ರುಚಿ-ಯಾದ
ರುಚಿ-ಸು-ವುದು
ರುಚಿಸ್ಪರ್ಶ
ರುಜಿನ-ಗಳು
ರುಜು-ವಾತು
ರುತ್ತದೆ
ರುತ್ತಿ-ರ-ಲಿಲ್ಲ
ರುಳಿ
ರುವ
ರುವನು
ರುವನೆ
ರುವನೋ
ರುವರು
ರುವ-ವರು
ರುವಾಗ
ರುವಾ-ಗಲೂ
ರುವಿರಿ
ರುವು-ದಕ್ಕಿಂತ
ರುವು-ದಕ್ಕೆ
ರುವು-ದ-ರಿಂದ
ರುವು-ದಿದ್ದರೆ
ರುವುದು
ರುವುದೇ
ರುವುವು
ರುವೆನು
ರುವೆನೊ
ರುವೆವೊ
ರೂಢಿ-ಗಳು
ರೂಢಿಗೆ
ರೂಢಿ-ಯಲ್ಲಿ
ರೂಢಿ-ಯಲ್ಲಿದೆ
ರೂಢಿ-ಯಲ್ಲಿ-ರುವ
ರೂಢಿ-ಯಲ್ಲಿ-ರು-ವಂತೆ
ರೂಢಿಯಾ-ಗಿ-ರುವ
ರೂಢಿಯಾ-ಗಿ-ರು-ವಂತೆ
ರೂಢಿಸ-ಬೇಕೆಂದು
ರೂಢಿಸಿ
ರೂಢಿಸಿ-ಕೊಳ್ಳ-ಬಹುದು
ರೂಢಿಸಿ-ಕೊಳ್ಳ-ಬೇಕು
ರೂಢಿ-ಸು-ವು-ದಕ್ಕೆ
ರೂಢಿ-ಸು-ವುದು
ರೂಪ
ರೂಪಕ
ರೂಪ-ಕ-ಗಳ
ರೂಪಕ್ಕೂ
ರೂಪಕ್ಕೆ
ರೂಪ-ಗ-ಳನ್ನು
ರೂಪ-ಗಳಲ್ಲಿ
ರೂಪ-ಗಳಲ್ಲಿಯೂ
ರೂಪ-ಗಳಷ್ಟೆ
ರೂಪ-ಗಳಿಗೆ
ರೂಪ-ಗಳು
ರೂಪ-ಗಳೇ
ರೂಪ-ತಾಳಿ
ರೂಪತೆ
ರೂಪದ
ರೂಪ-ದಲ್ಲಿ
ರೂಪ-ದಲ್ಲಿದೆ
ರೂಪ-ದಲ್ಲಿ-ರುವ
ರೂಪ-ದಲ್ಲಿ-ರುವುದು
ರೂಪ-ದಲ್ಲಿ-ರುವು-ದು-ಮನುಷ್ಯನು
ರೂಪ-ದಲ್ಲಿ-ರುವು-ದೆಲ್ಲ
ರೂಪ-ದಲ್ಲಿ-ರು-ವುವು
ರೂಪ-ದಲ್ಲಿವೆ
ರೂಪ-ದಾಳಿ-ದೆಯೇ
ರೂಪ-ದಿಂದ
ರೂಪದ್ದೇ
ರೂಪ-ಲಾವಣ್ಯ-ಬಲ-ವಜ್ರಸಂಹನನತ್ವಾನಿ
ರೂಪ-ವನ್ನು
ರೂಪ-ವಾಗಿ
ರೂಪ-ವಾ-ಗು-ವುದು
ರೂಪ-ವಾದುದೆ
ರೂಪ-ವಿ-ರುವ
ರೂಪವು
ರೂಪ-ವೆಂದು
ರೂಪ-ವೊಂದೇ
ರೂಪಾಂತರ
ರೂಪಾಂತರ-ಗೊಂಡರೆ
ರೂಪಾಂತರ-ವನ್ನು
ರೂಪಾಂತರ-ವೆನ್ನು-ವುದು
ರೂಪಾಯಿ
ರೂಪಿ-ನಲ್ಲಿ
ರೂಪಿ-ಸಲು
ರೂಪಿಸಿ-ಕೊಳ್ಳಿ-ನಿಮ್ಮ
ರೂಪಿಸಿ-ಕೊಳ್ಳು-ವರೊ
ರೆಂದು
ರೆಂಬೆ
ರೆಂಬೆಗೆ
ರೆಂಬೆಯ
ರೆಂಬೆ-ಯಲ್ಲಿದೆ
ರೆಂಬೆ-ಯಿಂದ
ರೆಕ್ಕೆ-ಗ-ಳನ್ನು
ರೆಕ್ಕೆಯ
ರೆಲ್ಲರೂ
ರೇಕನ್ನು
ರೇಕು-ಗ-ಳನ್ನು
ರೇಖೆಯ
ರೇಖೆ-ಯಲ್ಲಿ-ರ-ಬೇಕು
ರೇಚಕ
ರೇತಸ್ಸಿನ
ರೇನು
ರೇಶ್ಮೆಯ
ರೇಷ್ಮೆ
ರೇಷ್ಮೆಯ
ರೈತನು
ರೈಲಿ
ರೈಲು-ಗಾಡಿ-ಯಲ್ಲ
ರೈಲುಬಂಡಿ
ರೈಲ್ವೆ
ರೈಲ್ವೆ-ಬಂಡಿ-ಯನ್ನು
ರೊಂದಿಗೆ
ರೊಂದು
ರೊಬ್ಬನಿಗೇ
ರೋಗ
ರೋಗ-ಗಳಾ-ಗಲಿ
ರೋಗ-ಗಳು
ರೋಗದ
ರೋಗ-ದಂತೆ
ರೋಗ-ದಲ್ಲಿ
ರೋಗ-ದಿಂದ
ರೋಗ-ನಿ-ವಾರ-ಣೆ-ಯೆಂದು
ರೋಗ-ರುಜಿನ-ಗಳಿಲ್ಲ
ರೋಗ-ರುಜಿನ-ಗಳಿಲ್ಲದ
ರೋಗ-ವನ್ನು
ರೋಗ-ಶೋಕ-ಗಳಿಲ್ಲ
ರೋಗಿ-ಗ-ಳನ್ನು
ರೋಗಿ-ಗಳಾಗಿ
ರೋಗಿ-ಗಳೂ
ರೋಗಿಗೆ
ರೋಗಿಯ
ರೋಗಿ-ಯನ್ನು
ರೋಗಿ-ಯಲ್ಲಿ
ರೋಗಿಯು
ರೋಗಿ-ಯೊಬ್ಬ
ರೋಗಿಷ್ಟನು
ರೋಗ್ಯ
ರೋಮನ್
ರೋಮಿನ
ರ್ವರಿಗೂ
ಲಂಡನ್
ಲಂಡನ್ನಿ-ನಲ್ಲಿ
ಲಕ್ಷ
ಲಕ್ಷಣ
ಲಕ್ಷ-ಣ-ಗ-ಳನ್ನು
ಲಕ್ಷಣ-ಗಳಿಂದ
ಲಕ್ಷಣ-ಗಳು
ಲಕ್ಷಣ-ಗಳುಳ್ಳ
ಲಕ್ಷಣ-ದಿಂದ
ಲಕ್ಷ-ಣ-ವಿದೆ
ಲಕ್ಷಣವೇ
ಲಕ್ಷಾಂತರ
ಲಕ್ಷಿಸು-ವಂತೆ
ಲಕ್ಷ್ಯ
ಲಕ್ಷ್ಯ-ವಿರಿಸಿ
ಲಕ್ಷ್ಯ-ವಿಲ್ಲದೆ
ಲಗಾ-ಮನ್ನು
ಲಗಾಮಿನ
ಲಗಾಮು
ಲತೆಗೆ
ಲನೆ
ಲಭಿ-ಸದು
ಲಭಿಸ-ಲಾರದು
ಲಭಿ-ಸಿದ
ಲಭಿ-ಸುತ್ತದೆ
ಲಭಿ-ಸು-ವು-ದಿಲ್ಲ
ಲಭಿ-ಸು-ವುದು
ಲಭಿ-ಸು-ವುದು-ಅ-ವ-ನಿಗೆ
ಲಭಿ-ಸು-ವುವು
ಲಯ
ಲಯ-ಗೊಳಿ-ಸಿ-ದಾಗ
ಲಯ-ಗೊಳಿ-ಸು-ವುದರ
ಲಯ-ಬದ್ಧ
ಲಯ-ಬದ್ಧ-ವಾಗಿ
ಲಯ-ಬದ್ಧ-ವಾಗುತ್ತದೆ
ಲಯ-ಬದ್ಧ-ವಾದ
ಲಯ-ವಾಗ-ಬೇಕು
ಲಯ-ವಾಗಿ
ಲಯ-ವಾಗುತ್ತಿದ್ದರೆ
ಲಯ-ವಾ-ದಾಗ
ಲಯ-ಹೊಂದು
ಲಲಿತ
ಲವಲೇಶವೂ
ಲಾಗದ
ಲಾಗದು
ಲಾಗು-ವು-ದಿಲ್ಲ
ಲಾದ
ಲಾದರೆ
ಲಾದ-ರೊಮ್ಮೆ
ಲಾಭ
ಲಾಭ-ಕಾರಿ
ಲಾಭ-ದೃಷ್ಟಿ
ಲಾಭ-ವನ್ನು
ಲಾಭ-ವಾ-ಗು-ವುದು
ಲಾಭವೆ
ಲಾಯಿಸುತ್ತಿ-ರುವ
ಲಾಯಿಸುತ್ತಿ-ರುವುದು
ಲಾರ
ಲಾರ-ದ-ವರು
ಲಾರದು
ಲಾರ-ದು-ಎಲ್ಲಾ
ಲಾರದೆ
ಲಾರರು
ಲಾರವೊ
ಲಾರಿರಿ
ಲಾರೆವು
ಲಾವಣೆ
ಲಾವಣೆ-ಗಳೆಲ್ಲ
ಲಾವಣ್ಯ
ಲಿಂಗ
ಲಿಂಗ-ಭೇದ
ಲಿಂಗ-ಭೇದ-ವಾ-ಗಲೀ
ಲಿಂಗ-ಭೇದ-ವಿಲ್ಲ
ಲಿಂಗ-ವಿಲ್ಲ
ಲಿಂಗಾ-ತೀತ
ಲಿಂಗಾರ್ಚಕ-ರಿಗೆ
ಲಿಂಗಾರ್ಚನೆ
ಲಿಖಿತ
ಲಿಲ್ಲ
ಲೀನ-ವಾಗ-ಬೇಕು
ಲೀನ-ವಾಗಿ-ದೆಯೊ
ಲೀನ-ವಾ-ಗಿ-ರುವುದು
ಲೀನವಾ-ಗುತ್ತಾರೆ
ಲೀನ-ವಾಗು-ವು-ದನ್ನು
ಲೀನ-ವಾ-ಗು-ವುದು
ಲೀನ-ವಾ-ಗು-ವುವು
ಲೀಲಾ-ಭೂಮಿ-ಯಾ-ಗು-ವುದು
ಲೀಲೆ
ಲುಬಕ್ನಂತಹ
ಲೆಂದು
ಲೆಕ್ಕ-ದಲ್ಲಿ
ಲೆಕ್ಕ-ವಿಲ್ಲ
ಲೆಕ್ಕ-ವಿಲ್ಲ-ದಷ್ಟು
ಲೆಕ್ಕಿ-ಸದ
ಲೆಕ್ಕಿ-ಸದಿ-ರ-ಬಹುದು
ಲೆಕ್ಕಿ-ಸದೆ
ಲೆಕ್ಕಿಸ-ಬೇಕಾ-ಗಿಲ್ಲ
ಲೆಕ್ಕಿಸ-ಬೇಡಿ
ಲೆಕ್ಕಿಸ-ಲಿಲ್ಲ
ಲೆಕ್ಕಿ-ಸು-ವು-ದಿಲ್ಲ
ಲೆಕ್ಕಿ-ಸು-ವು-ದಿಲ್ಲ-ವೆಂಬು-ದನ್ನು
ಲೆಕ್ಕಿ-ಸು-ವು-ದಿಲ್ಲವೊ
ಲೆಕ್ಕಿಸು-ವುದೇ
ಲೇಪಿ-ಸುತ್ತಿರು
ಲೇಸು
ಲೈಂಗಿಕ
ಲೈಂಗಿಕ-ವೆಂದು
ಲೋಕ
ಲೋಕಕ್ಕೆ
ಲೋಕ-ಗಳಲ್ಲಿ
ಲೋಕ-ಗಳಲ್ಲೆಲ್ಲಾ
ಲೋಕ-ಗಳಾಗಿ
ಲೋಕ-ಗಳಿಗೂ
ಲೋಕ-ಗಳಿಗೆ
ಲೋಕ-ಗಳಿವೆ
ಲೋಕ-ಗಳು
ಲೋಕ-ಗಳೆಲ್ಲ
ಲೋಕದ
ಲೋಕ-ದಲ್ಲಿ
ಲೋಕ-ದಲ್ಲೊ
ಲೋಕ-ಪಿತ
ಲೋಕ-ವನ್ನು
ಲೋಕವು
ಲೋಕಾಯತ
ಲೋಚನೆಯ
ಲೋಟ
ಲೋಟ-ವಾಗಿದೆ
ಲೋಟವು
ಲೋಭ
ಲೋಭಕ್ರೋಧ-ಮೋಹ
ಲೋಹ
ಲೋಹ-ವನ್ನು
ಲೌಕಿಕ
ಲೌಕಿಕ-ತೆಯೇ
ಲೌಕಿ-ಕರೇ
ಲ್ಪಟ್ಟಿತ್ತು
ಲ್ಪಟ್ಟಿರುತ್ತವೆ
ಲ್ಲದ
ಲ್ಲದೆ
ಲ್ಲವೋ
ಲ್ಲಿಂದ
ಲ್ಲಿದೆ
ಲ್ಲಿಯೂ
ಲ್ಲಿಯೇ
ಲ್ಲಿರುವ
ಲ್ಲಿವೆ
ಲ್ಲೆಲ್ಲಾ
ಳಗೇ
ಳನ್ನೂ
ಳುತ್ತವೆ
ವಂಚಕ-ರಿಂದ
ವಂಚನೆ-ಯಿಂದ
ವಂಚಿಸು-ವರು
ವಂತ-ನ-ವರೆಗೆ
ವಂತ-ನಾ-ಗದೆ
ವಂತ-ನಾ-ಗಿ-ರುವನು
ವಂತರ
ವಂತಹ
ವಂತೆ
ವಂದನೆ
ವಂದಿಸಿ-ದನು
ವಂಶ
ವಂಶ-ಜರು
ವಂಶದ
ವಂಶ-ದಲ್ಲಿ
ವಂಶ-ದ-ವರು
ವಕ್ರತೆ
ವಜಾ-ಮಾಡಿ-ರುವುದು
ವಜ್ರದ
ವಜ್ರ-ದಂತೆ
ವಜ್ರಮನಸ್ಸಾ-ದರೂ
ವಜ್ರಮುಷ್ಟಿ-ಯಲ್ಲಿ
ವಜ್ರೋಪಮ
ವಜ್ರೋಪಮ-ವಾದ
ವಟಿಕೆ
ವಣಿಗೆ
ವಣೆ
ವಣೆ-ಯಾಗು-ವುದೆಂಬು-ದನ್ನು
ವದನ
ವದ-ನ-ಗಳ
ವಧಿ-ಯಾ-ಗು-ವುದು
ವನ
ವನನ್ನಾಗಿ
ವನಿಗೆ
ವನು
ವನೆಂಬು-ದನ್ನು
ವನೊ
ವನ್ನಾ-ದರೂ
ವನ್ನು
ವನ್ನೂ
ವನ್ನೆಲ್ಲಾ
ವನ್ನೇ
ವನ್ನೋ
ವಯಸ್ಸಾಗಿ-ರ-ಬಹುದು
ವಯಸ್ಸಾದ
ವಯಸ್ಸಾ-ದಂತೆಲ್ಲ
ವಯಸ್ಸಾ-ದಾಗ
ವಯಸ್ಸು
ವರ
ವರಕ್ಕಾಗಿ
ವರ-ಗ-ಳನ್ನು
ವರ-ಣಕ್ಕೆ
ವರ-ಣ-ಭೇದಸ್ತು
ವರ-ತವೂ
ವರ-ದಿಂದ
ವರ-ದಿ-ಯಾ-ಗಿದೆ
ವರ-ವನ್ನಾ-ದರೂ
ವರ-ವನ್ನು
ವರ-ವನ್ನೂ
ವರ-ವಾಗಿ
ವರವೆ
ವರಿಗೆ
ವರಿದ
ವರಿ-ದಿ-ರು-ವಿರಿ
ವರಿ-ಯಲಿ
ವರಿ-ಯುತ್ತಾರೆಯೋ
ವರಿ-ಯುತ್ತಿತ್ತು
ವರಿ-ಯುತ್ತಿ-ರುವುದು
ವರಿ-ಯು-ವುದು
ವರು
ವರುಣ
ವರು-ಣ-ನಿಲ್ಲಿ-ರುವನು
ವರುಷ
ವರುಷಕ್ಕೆ
ವರುಷ-ಗಟ್ಟಲೆ
ವರುಷ-ಗಳ
ವರುಷ-ಗಳಲ್ಲಿಯೋ
ವರುಷ-ಗಳ-ವರೆಗೆ
ವರುಷ-ಗಳಷ್ಟು
ವರುಷ-ಗಳಾದ
ವರುಷ-ಗಳಾ-ದುವು
ವರುಷ-ಗಳಿಂದ
ವರುಷ-ಗಳಿಂದಲೂ
ವರುಷ-ಗಳು
ವರುಷ-ದಿಂದ
ವರುಷ-ವಾದರೂ
ವರುಷವೂ
ವರೆಗೂ
ವರೆಗೆ
ವರೆಲ್ಲ-ರನ್ನೂ
ವರೆ-ವಿಗೂ
ವರೆ-ವಿಗೆ
ವರೊ
ವರ್ಗ
ವರ್ಗಕ್ಕೆ
ವರ್ಗ-ಗಳ
ವರ್ಗ-ಗಳಲ್ಲಿ
ವರ್ಗ-ಗಳಿ-ರು-ವುವು
ವರ್ಗದ
ವರ್ಗ-ದಲ್ಲಿ
ವರ್ಗದ್ದು
ವರ್ಗ-ವನ್ನು
ವರ್ಗವು
ವರ್ಗವೂ
ವರ್ಗಾ
ವರ್ಗಾ-ಯಿಸ-ಬಹುದು
ವರ್ಗಾ-ಯಿ-ಸಲು
ವರ್ಗಾ-ವಣೆ
ವರ್ಗಾ-ವಣೆಯ
ವರ್ಗೀ-ಕರಣ
ವರ್ಗೀ-ಕರ-ಣ-ದಲ್ಲಿದೆ
ವರ್ಗೀ-ಕರ-ಣ-ವಾಗಿ
ವರ್ಗೀ-ಕರಿಸ-ಬಹುದು
ವರ್ಗೀಕ-ರಿಸಿ-ದರು
ವರ್ಗೀಕ-ರಿಸು
ವರ್ಜಿ-ತನು
ವರ್ಣ-ಗಳು
ವರ್ಣದ
ವರ್ಣನೆ
ವರ್ಣ-ನೆಯೂ
ವರ್ಣಿಸ-ಬಲ್ಲದು
ವರ್ಣಿಸಿ-ರು-ವರು
ವರ್ಣಿಸುತ್ತಾರೆ
ವರ್ತ
ವರ್ತ-ಮಾನ
ವರ್ತ-ಮಾನ-ಕಾಲ
ವರ್ತ-ಮಾನ-ಗಳ
ವರ್ತ-ಮಾನ-ಗಳೆ-ರಡೂ
ವರ್ತ-ಮಾನ-ಗಳೆಲ್ಲಾ
ವರ್ತ-ಮಾನ-ದಲ್ಲಿ
ವರ್ತ-ಮಾನ-ದಲ್ಲಿ-ರು-ವೆನು
ವರ್ತ-ಮಾನ-ವನ್ನು
ವರ್ತ-ಮಾನ-ವಾಗುತ್ತವೆ
ವರ್ತಿ-ಸಿದ
ವರ್ತಿ-ಸಿದ್ದಾರೆ
ವರ್ತಿ-ಸುತ್ತೀರಿ
ವರ್ತಿ-ಸು-ವರು
ವರ್ಷ
ವರ್ಷಕ್ಕೆ
ವರ್ಷಕ್ಕೇ
ವರ್ಷ-ಗಟ್ಟಲೆ
ವರ್ಷ-ಗಳ
ವರ್ಷ-ಗಳಲ್ಲಿ
ವರ್ಷ-ಗಳ-ವರೆಗೆ
ವರ್ಷ-ಗ-ಳಾದರೂ
ವರ್ಷ-ಗ-ಳಾದವು
ವರ್ಷ-ಗಳಿಂದ
ವರ್ಷ-ಗಳಿಂದಲೂ
ವರ್ಷ-ಗಳು
ವರ್ಷವೂ
ವರ್ಷಿ-ಸು-ವನು
ವಲಯ-ದೊಳ-ಗಿನ
ವಲ್ಲ
ವಲ್ಲದ
ವಳಿ-ಗಳ
ವಳಿ-ಗಳಿಂದ
ವವನು
ವವ-ರನ್ನು
ವವ-ರಿಗೂ
ವವರು
ವವರೆಗೂ
ವವರೆಗೆ
ವವರೆ-ವಿಗೂ
ವವ-ರೊಂದಿಗೆ
ವವೂ
ವಶಕ್ಕೆ
ವಶ-ದಲ್ಲಿ-ರುತ್ತವೆ-ಇಲ್ಲಿ-ಯ-ವರೆ-ವಿಗೂ
ವಶ-ದಲ್ಲಿ-ರು-ವು-ದ-ರಿಂದ
ವಶ-ಮಾಡಿ-ಕೊಂಡಂತೆ
ವಶ-ಮಾಡಿ-ಕೊಳ್ಳುವ
ವಶ-ರಾಗು-ವರು
ವಶ-ರಾಗು-ವು-ದ-ರಿಂದ
ವಶ-ರಾ-ದರೆ
ವಶ-ವರ್ತಿ-ಯಾಗುವ
ವಶ-ವಾಗ-ದಂತೆ
ವಶ-ವಾಗಿದೆ
ವಶ-ವಾ-ಗು-ವುದು
ವಶ-ವಾ-ಗು-ವುವು
ವಶ-ವಾದ
ವಶ-ವಾ-ದರೆ
ವಶೀ-ಕಾರಃ
ವಶೀ-ಕಾರ-ಸಂಜ್ಞಾ
ವಶ್ಯಕ-ವಾಗಿದೆ
ವಶ್ಯಕ-ವೆಂದೂ
ವಷ್ಟು
ವಸಂತಋತು
ವಸನಭೂಷಣ-ಗಳಲ್ಲಿ-ರುವನು
ವಸ್ತು
ವಸ್ತು-ಆ-ಕಾಶವೇ
ವಸ್ತು-ಕಣ-ಗಳ
ವಸ್ತು-ಗಳ
ವಸ್ತು-ಗ-ಳನ್ನು
ವಸ್ತು-ಗ-ಳನ್ನೂ
ವಸ್ತು-ಗಳಲ್ಲ
ವಸ್ತು-ಗಳಲ್ಲಿ
ವಸ್ತು-ಗಳಲ್ಲಿ-ರುವ
ವಸ್ತು-ಗಳಲ್ಲೆಲ್ಲಾ
ವಸ್ತು-ಗಳಾ
ವಸ್ತು-ಗ-ಳಾದ
ವಸ್ತು-ಗಳಿಂದ
ವಸ್ತು-ಗಳಿ-ಗಿಂತಲೂ
ವಸ್ತು-ಗಳಿಗೂ
ವಸ್ತು-ಗಳಿಗೆ
ವಸ್ತು-ಗಳಿರ
ವಸ್ತು-ಗಳಿಲ್ಲ
ವಸ್ತು-ಗಳಿವೆ
ವಸ್ತು-ಗಳು
ವಸ್ತು-ಗಳೂ
ವಸ್ತು-ಗ-ಳೆಂದು
ವಸ್ತು-ಗಳೆಲ್ಲ
ವಸ್ತು-ಗಳೆಲ್ಲವೂ
ವಸ್ತು-ಗಳೊಂದಿಗೆ
ವಸ್ತುಜ್ಞಾತಾಜ್ಞಾ-ತಮ್
ವಸ್ತುತಃ
ವಸ್ತುತ್ವ
ವಸ್ತುಪ್ರಜ್ಞೆ-ಯಾ-ಗು-ವುದು
ವಸ್ತುಪ್ರಜ್ಞೆ-ವಸ್ತು-ವಾ-ಗು-ವುದು
ವಸ್ತುಪ್ರ-ದರ್ಶನ
ವಸ್ತು-ಭೇದ
ವಸ್ತು-ರ-ಚನೆ-ಯನ್ನು
ವಸ್ತು-ರಾಶಿ-ಯಲ್ಲಿ
ವಸ್ತು-ವನ್ನಾ-ಗಲಿ
ವಸ್ತು-ವನ್ನಾಗಿ
ವಸ್ತು-ವನ್ನು
ವಸ್ತು-ವನ್ನೂ
ವಸ್ತು-ವನ್ನೇ
ವಸ್ತು-ವಲ್ಲ
ವಸ್ತು-ವಾಗಿ
ವಸ್ತು-ವಾ-ದಾಗ
ವಸ್ತು-ವಿ-ಗಾಗಿ
ವಸ್ತು-ವಿಗೂ
ವಸ್ತು-ವಿಗೆ
ವಸ್ತು-ವಿದೆ
ವಸ್ತು-ವಿದೆಯೋ
ವಸ್ತು-ವಿದ್ದರೆ
ವಸ್ತು-ವಿನ
ವಸ್ತು-ವಿನ-ಮೇಲೆ
ವಸ್ತು-ವಿ-ನಲ್ಲಿ
ವಸ್ತು-ವಿನಲ್ಲಿ-ರುವ
ವಸ್ತು-ವಿನ-ವರೆಗೂ
ವಸ್ತು-ವಿನ-ವರೆಗೆ
ವಸ್ತು-ವಿ-ನಿಂದ
ವಸ್ತು-ವಿ-ನಿಂದಲೂ
ವಸ್ತು-ವಿ-ನಿಂದಲೇ
ವಸ್ತು-ವಿ-ನೆ-ಡೆಗೆ
ವಸ್ತು-ವಿ-ನೊಂದಿಗೆ
ವಸ್ತು-ವಿಲ್ಲ
ವಸ್ತು-ವಿವೇ-ಕದ
ವಸ್ತುವು
ವಸ್ತುವೂ
ವಸ್ತು-ವೆಂದಾಗಲಿ
ವಸ್ತು-ವೆಂದು
ವಸ್ತು-ವೆಂದೂ
ವಸ್ತು-ವೆಂಬ
ವಸ್ತು-ವೆಲ್ಲ
ವಸ್ತುವೇ
ವಸ್ತು-ವೊಂದಿದೆ
ವಸ್ತು-ವೊಂದೇ
ವಸ್ತು-ಶೂನ್ಯೋ
ವಸ್ತು-ಸತ್ತಾ
ವಸ್ತು-ಸತ್ತಾ-ವಾದವೂ
ವಸ್ತು-ಸತ್ತಾ-ವಾದಿ
ವಸ್ತು-ಸತ್ತಾ-ವಾದಿ-ಗಳ
ವಸ್ತು-ಸತ್ತಾ-ವಾದಿಯು
ವಸ್ತು-ಸಾಮ್ಯೇ
ವಸ್ತುಸ್ಥಿತಿ-ಯೇ-ನೆಂದರೆ
ವಸ್ತು-ಹೀನ
ವಸ್ತು-ಹೀನ-ವಾಗಿ
ವಸ್ಥೆ-ಯನ್ನು
ವಸ್ಥೆ-ಯಲ್ಲಿ-ರುವಾಗ
ವಹಿ-ಸ-ಬೇಕು
ವಹಿಸ-ಲಾರದು
ವಹಿಸಿ
ವಹಿಸಿ-ಕೊಂಡಿ
ವಹಿಸಿ-ಕೊಂಡಿದ್ದ
ವಹಿಸಿ-ಕೊಂಡಿ-ರು-ವುದು
ವಹಿಸುತ್ತಾರೆ
ವಹಿಸುತ್ತೀರಿ
ವಹಿ-ಸುತ್ತೇವೆ
ವಹಿ-ಸು-ವುವು
ವಾ
ವಾಕ್ಝರೀ
ವಾಕ್ಯ-ಗ-ಳನ್ನು
ವಾಕ್ಯ-ಗಳು
ವಾಗ
ವಾಗ-ತೀತವೂ
ವಾಗ-ಬೇಕಾ-ಗಿದೆ
ವಾಗ-ಬೇಕಾ-ದರೆ
ವಾಗ-ಲಾರದು
ವಾಗಲಿ
ವಾಗ-ಲಿಲ್ಲ
ವಾಗಲೀ
ವಾಗಿ
ವಾಗಿಟ್ಟಿ-ರ-ಬೇಕು
ವಾಗಿಟ್ಟು
ವಾಗಿತ್ತು
ವಾಗಿದೆ
ವಾಗಿದ್ದರೆ
ವಾಗಿದ್ದು
ವಾಗಿಯೇ
ವಾಗಿ-ರ-ಬಹುದು
ವಾಗಿ-ರ-ಬೇಕು
ವಾಗಿ-ರಲಿ
ವಾಗಿ-ರ-ಲಿಲ್ಲ
ವಾಗಿರಿ
ವಾಗಿ-ರುತ್ತವೆ
ವಾಗಿ-ರುವ
ವಾಗಿ-ರು-ವಂತೆ
ವಾಗಿ-ರುವನು
ವಾಗಿ-ರು-ವು-ದ-ರಿಂದ
ವಾಗಿ-ರು-ವು-ದ-ರಿಂದಲೂ
ವಾಗಿ-ರುವುದು
ವಾಗಿ-ರುವುದೋ
ವಾಗಿ-ರು-ವುವು
ವಾಗಿ-ರು-ವೆನು
ವಾಗಿ-ರು-ವೆವು
ವಾಗಿಲ್ಲ
ವಾಗಿಲ್ಲದೇ
ವಾಗಿವೆ
ವಾಗಿ-ವೆಯೆ
ವಾಗುತ್ತ
ವಾಗುತ್ತದೆ
ವಾಗುತ್ತ-ದೆ-ಯಾ-ದರೂ
ವಾಗುತ್ತಿ-ರ-ಲಿಲ್ಲ
ವಾಗುವ
ವಾಗು-ವನು
ವಾಗು-ವ-ವರೆಗೂ
ವಾಗು-ವು-ದಕ್ಕೆ
ವಾಗು-ವು-ದಿಲ್ಲ
ವಾಗು-ವುದು
ವಾಗು-ವುದು-ದಿ-ನಕ್ಕೆ
ವಾಗು-ವುದೆಂದರೆ
ವಾಗು-ವುದೇ
ವಾಗು-ವುವು
ವಾಗ್ಮಿ
ವಾಗ್ವಾದದ
ವಾಗ್ವಾದ-ಪೂರಿತ-ವಾದು-ದಲ್ಲ
ವಾಗ್ವೈಖರೀ
ವಾಚಕಃ
ವಾಡು-ವುದು
ವಾಣಿ
ವಾಣಿ-ಗಳು
ವಾಣಿಯ
ವಾಣಿ-ಯಂತೂ
ವಾಣಿ-ಯಂತೆ
ವಾಣಿ-ಯನ್ನು
ವಾಣಿ-ಯನ್ನೇ
ವಾಣಿ-ಯಲ್ಲಿ
ವಾಣಿಯೆ
ವಾಣಿ-ಯೆ-ಡೆಗೆ
ವಾತ-ರೋಗ-ದಂತೆ
ವಾತಾ
ವಾತಾ-ವರಣ
ವಾತಾ-ವರಣಕ್ಕೆ
ವಾತಾ-ವರಣದ
ವಾತಾ-ವರಣ-ದಲ್ಲಿ
ವಾತಾ-ವರಣ-ದಿಂದ
ವಾತಾ-ವರಣ-ವನ್ನು
ವಾತಾ-ವರಣವು
ವಾದ
ವಾದಈ
ವಾದಕ್ಕೂ
ವಾದಕ್ಕೆ
ವಾದಕ್ಕೆಲ್ಲಾ
ವಾದ-ಗಳಿ-ವೆಯೆ
ವಾದ-ಗಳು
ವಾದ-ಗಳೂ
ವಾದದ
ವಾದ-ದಂತೆ
ವಾದ-ದಲ್ಲಿ
ವಾದ-ದಿಂದ
ವಾದದ್ದು
ವಾದ-ಮಾಡು-ವು-ದಿಲ್ಲ
ವಾದರೂ
ವಾದರೆ
ವಾದ-ವನ್ನು
ವಾದ-ವಾದ
ವಾದವು
ವಾದವೂ
ವಾದವೇ
ವಾದಷ್ಟೂ
ವಾದ-ಸರಣಿ
ವಾದ-ಸರಣಿಗೆ
ವಾದ-ಸರಣಿ-ಯನ್ನು
ವಾದ-ಸರಣಿ-ಯಲ್ಲಿ
ವಾದಾಗ
ವಾದಿ
ವಾದಿ-ಗಳ
ವಾದಿ-ಗಳು
ವಾದಿ-ಗಳೂ
ವಾದಿಗೆ
ವಾದಿ-ಸ-ಬಹುದು
ವಾದಿ-ಸ-ಬಾ-ರದು
ವಾದಿ-ಸ-ಬೇಡಿ
ವಾದಿ-ಸಿ-ದರು
ವಾದಿ-ಸಿ-ದರೆ
ವಾದಿಸು
ವಾದಿ-ಸುತ್ತಿದ್ದರು
ವಾದಿ-ಸು-ವು-ದಾದರೆ
ವಾದಿ-ಸು-ವುದೇ
ವಾದು-ದಕ್ಕೆ
ವಾದು-ದನ್ನು
ವಾದುದು
ವಾದು-ದೆಂದು
ವಾದುದೇ
ವಾದು-ದೇನೂ
ವಾದುವು
ವಾದ್ಯ-ಗಳು
ವಾನ್
ವಾಯಿತು
ವಾಯು
ವಾಯು-ಗುಣದ
ವಾಯು-ತ-ರಂಗ-ಗಳು
ವಾಯು-ವನ್ನು
ವಾರ
ವಾಸ
ವಾಸನ
ವಾಸನೆ
ವಾಸ-ನೆ-ಯನ್ನು
ವಾಸ-ಮಾಡುವ
ವಾಸ-ಮಾಡು-ವುದು
ವಾಸ-ವಾಗಿದ್ದ
ವಾಸಸ್ಥಾನ-ವಾದ
ವಾಸಿ-ಗಳೂ
ವಾಸಿ-ಸುತ್ತಾನೆ
ವಾಸಿ-ಸುತ್ತಾರೆ
ವಾಸಿ-ಸುತ್ತಿದ್ದರು
ವಾಸಿ-ಸುತ್ತಿ-ರು-ವೆನು
ವಾಸಿ-ಸುವ
ವಾಸಿ-ಸುವನು
ವಾಸಿ-ಸುವನೊ
ವಾಸಿ-ಸುವರು
ವಾಸಿ-ಸುವ-ವರೂ
ವಾಸಿ-ಸುವಾಗ
ವಾಸಿ-ಸು-ವು-ದನ್ನು
ವಾಸಿ-ಸು-ವುದು
ವಾಸ್ತವ
ವಾಸ್ತವಾಂಶ-ಗಳ
ವಾಸ್ತವಾಂಶ-ಗ-ಳನ್ನು
ವಾಸ್ತವಾಂಶ-ಗಳನ್ನೆ
ವಾಸ್ತವಾಂಶ-ಗಳು
ವಾಸ್ತ-ವಿಕ
ವಾಸ್ತ-ವಿಕ-ರಾಗಿ-ರು-ವರು
ವಾಸ್ತ-ವಿಕ-ವಾಗಿ
ವಾಹಕ
ವಿಂಗಡಿಸ-ಬಹುದು
ವಿಂಗಡಿ-ಸಿದೆ
ವಿಂಗಡಿ-ಸುವ
ವಿಕ-ರಣ-ಭಾವಃ
ವಿಕರ್ಷಣ
ವಿಕರ್ಷಣ-ದಂತೆ
ವಿಕರ್ಷಿತ-ರಾಗು-ವೆವು
ವಿಕರ್ಷಿ-ಸು-ವೆವು
ವಿಕಲ್ಪ
ವಿಕಲ್ಪಃ
ವಿಕಲ್ಪ-ಗಳಿಂದ
ವಿಕಲ್ಪ-ವೆಂದು
ವಿಕ-ಸನ
ವಿಕ-ಸಿತ
ವಿಕಾರ
ವಿಕಾರ-ಗ-ಳನ್ನು
ವಿಕಾರ-ಗಳು
ವಿಕಾರ-ವನ್ನು
ವಿಕಾರ-ವಸ್ತು-ವಿ-ನಿಂದ
ವಿಕಾರ-ವಾಗದ
ವಿಕಾರ-ವಾಗದೆ
ವಿಕಾರ-ವಾಗ-ಬಲ್ಲವು
ವಿಕಾರ-ವಾ-ಗಿಲ್ಲ
ವಿಕಾರ-ವಾಗುತ್ತಿದೆ
ವಿಕಾರ-ವಾಗುತ್ತಿ-ರುವ
ವಿಕಾರ-ವಾಗುತ್ತಿ-ರುವು-ದನ್ನು
ವಿಕಾರ-ವಾಗುವ
ವಿಕಾರ-ವಾಗು-ವುದೇ
ವಿಕಾರ-ವಾದ
ವಿಕಾರ-ವಾ-ದಂತೆ
ವಿಕಾರ-ವಾ-ದಕ್ಕೆ
ವಿಕಾರ-ವಾದು-ದನ್ನು
ವಿಕಾರ-ವಾದು-ದು-ಮತ್ತೊಂದು
ವಿಕಾರ-ವೆಲ್ಲ
ವಿಕಾರಸ್ಕೆ
ವಿಕಾರ-ಹೊಂದದ
ವಿಕಾರ-ಹೊಂದಿ
ವಿಕಾರ-ಹೊಂದುತ್ತದೆ
ವಿಕಾರಿ-ಯಾ-ದುದು
ವಿಕಾಸ
ವಿಕಾಸಕ್ಕಾಗಿ
ವಿಕಾಸಕ್ಕೂ
ವಿಕಾಸಕ್ಕೆ
ವಿಕಾಸ-ಗಳಿಗೆ
ವಿಕಾಸ-ಗಳು
ವಿಕಾಸ-ಗೊಳಿ
ವಿಕಾಸ-ಗೊಳ್ಳು-ವುದು
ವಿಕಾ-ಸದ
ವಿಕಾಸ-ದಲ್ಲೆಲ್ಲಾ
ವಿಕಾಸ-ದಿಂದ
ವಿಕಾಸ-ದೆ-ಡೆಗೆ
ವಿಕಾಸ-ವನ್ನು
ವಿಕಾಸ-ವನ್ನೂ
ವಿಕಾಸ-ವನ್ನೆ
ವಿಕಾಸ-ವಲ್ಲ
ವಿಕಾಸ-ವಾಗ-ಬೇಕಾ-ದರೆ
ವಿಕಾಸ-ವಾಗ-ಬೇಕು
ವಿಕಾಸ-ವಾಗ-ಲಾರದು
ವಿಕಾಸ-ವಾ-ಗಲು
ವಿಕಾಸ-ವಾಗಿ
ವಿಕಾಸ-ವಾಗಿ-ರುವನು
ವಿಕಾಸ-ವಾ-ಗಿ-ರುವುದು
ವಿಕಾಸ-ವಾಗು
ವಿಕಾಸ-ವಾಗುತ್ತ
ವಿಕಾಸ-ವಾಗುತ್ತದೆ
ವಿಕಾಸ-ವಾಗುತ್ತಾ
ವಿಕಾಸ-ವಾಗುತ್ತಿದೆ
ವಿಕಾಸ-ವಾಗುತ್ತಿ-ರುವ
ವಿಕಾಸ-ವಾಗುತ್ತಿ-ರುವನು
ವಿಕಾಸ-ವಾಗುತ್ತಿ-ರುವುದು
ವಿಕಾಸ-ವಾಗುತ್ತಿವೆ
ವಿಕಾಸ-ವಾಗುವ
ವಿಕಾಸ-ವಾಗು-ವಂತೆ
ವಿಕಾಸ-ವಾಗು-ವನು
ವಿಕಾಸ-ವಾಗು-ವು-ದಕ್ಕೆ
ವಿಕಾಸ-ವಾಗು-ವು-ದಿಲ್ಲ
ವಿಕಾಸ-ವಾಗು-ವುದು
ವಿಕಾಸ-ವಾಗು-ವುದೊ
ವಿಕಾಸ-ವಾಗು-ವುವು
ವಿಕಾಸ-ವಾದ
ವಿಕಾಸ-ವಾ-ದಂತೆ
ವಿಕಾಸ-ವಾ-ದಕ್ಕೆ
ವಿಕಾಸ-ವಾದದ
ವಿಕಾಸ-ವಾ-ದರೂ
ವಿಕಾಸ-ವಾದ-ವನ್ನು
ವಿಕಾಸ-ವಾದ-ವೆಂದ-ರೇನು
ವಿಕಾಸ-ವಾದಿ
ವಿಕಾಸ-ವಾದಿ-ಗಳ
ವಿಕಾಸ-ವಾದಿ-ಗಳು
ವಿಕಾಸ-ವಾದಿ-ಗಳೊಂದಿಗೆ
ವಿಕಾಸ-ವಾದಿಯ
ವಿಕಾಸ-ವಾದಿ-ಯಾದ
ವಿಕಾಸ-ವಾ-ಯಿತು
ವಿಕಾಸವು
ವಿಕಾಸವೂ
ವಿಕಾಸ-ವೆಂದು
ವಿಕಾಸ-ವೆಂಬ
ವಿಕಾಸ-ವೆಂಬುದು
ವಿಕಾಸ-ವೆನ್ನು-ವುದು
ವಿಕಾಸ-ವೆಲ್ಲ
ವಿಕಾಸವೇ
ವಿಕಾಸಸ್ಥಿತಿ
ವಿಕಾಸ-ಹೊಂದಿ-ರು-ವಿರಿ
ವಿಕಾಸ-ಹೊಂದು-ವನು
ವಿಕೃತ-ಗೊಳಿ-ಸದೆ
ವಿಕೃತ-ಗೊಳಿ-ಸಿ-ರ-ಬಹುದು
ವಿಕೃತಿ
ವಿಕೃ-ತಿಗೆ
ವಿಕ್ಷಿಪ್ತ
ವಿಕ್ಷಿಪ್ತಾ-ವಸ್ಥೆ
ವಿಕ್ಷೇಪಸಹಭುವಃ
ವಿಗಿಂತ
ವಿಗೂ
ವಿಗ್ರಹ
ವಿಗ್ರಹ-ಇವು
ವಿಗ್ರಹ-ಗಳಿಗೆ
ವಿಗ್ರಹದ
ವಿಗ್ರಹ-ದಲ್ಲಿ
ವಿಗ್ರಹ-ವನ್ನು
ವಿಗ್ರಹ-ವನ್ನೋ
ವಿಗ್ರಹಾದಿ
ವಿಘಟಿಸಿ
ವಿಚಲಿತ
ವಿಚಲಿತ-ವಾ-ದಾಗ
ವಿಚಾರ
ವಿಚಾ-ರಕ್ಕೆ
ವಿಚಾರ-ಗಳ
ವಿಚಾರ-ಗ-ಳನ್ನು
ವಿಚಾರ-ಗಳಾ
ವಿಚಾರ-ಗಳು
ವಿಚಾರದ
ವಿಚಾರ-ದಲ್ಲಿ
ವಿಚಾರ-ದಿಂದ
ವಿಚಾರ-ದೃಷ್ಟಿಗೆ
ವಿಚಾರ-ಪರ
ವಿಚಾರ-ಪರ-ನಾದ
ವಿಚಾರ-ಪರನೂ
ವಿಚಾರ-ಪರ-ರನ್ನು
ವಿಚಾರ-ಪರ-ರಾ-ಗಲಿ
ವಿಚಾರ-ಪರ-ರಾಗಿ
ವಿಚಾರ-ಪರರು
ವಿಚಾರ-ಮಾಡ-ಬೇಕಾ-ಗಿದೆ
ವಿಚಾರ-ಮಾಡಿ
ವಿಚಾರ-ಮಾಡು
ವಿಚಾರ-ಮಾಡು-ವು-ದಕ್ಕೆ
ವಿಚಾರ-ಮಾಡು-ವುದಿ-ರಲಿ
ವಿಚಾರ-ಮಾಡು-ವುದು
ವಿಚಾರ-ವಂತ-ರನ್ನು
ವಿಚಾರ-ವನ್ನು
ವಿಚಾರ-ವಾಗಿ
ವಿಚಾರ-ವಾದಿ
ವಿಚಾರ-ವಿದೆ
ವಿಚಾರವೆ
ವಿಚಾರ-ವೆಂಬ
ವಿಚಾರ-ಶಕ್ತಿ
ವಿಚಾರ-ಶೀಲ-ನಾದ-ವ-ನಿಗೆ
ವಿಚಾರ-ಹೀನ-ರಾದ
ವಿಚಾರ-ಹೀನ-ರೆಂದು
ವಿಚಾರಿ-ಸದೆ
ವಿಚಾರಿಸ-ಬಹು-ದಾದ
ವಿಚಾರಿಸ-ಬೇಕಾ-ಗಿದೆ
ವಿಚಾರಿಸಿ-ದಾಗ
ವಿಚಾ-ರಿಸಿ-ರು-ವರು
ವಿಚಾರಿ-ಸುತ್ತಾ
ವಿಚಾರಿ-ಸೋಣ
ವಿಚಿತ್ರ
ವಿಚಿತ್ರ-ವಾಗಿ
ವಿಚಿತ್ರ-ವಾದ
ವಿಚಿತ್ರ-ವಾದುದು
ವಿಚಿತ್ರ-ವೇನೆಂದರೆ
ವಿಜಯಿ-ಯಾಗು-ವನು
ವಿಜ್ಞಾ
ವಿಜ್ಞಾನ
ವಿಜ್ಞಾನಕ್ಕಿಂತ
ವಿಜ್ಞಾನಕ್ಕೂ
ವಿಜ್ಞಾನಕ್ಕೆ
ವಿಜ್ಞಾನ-ಗಳು
ವಿಜ್ಞಾ-ನದ
ವಿಜ್ಞಾನ-ದಲ್ಲಿ
ವಿಜ್ಞಾನ-ವನ್ನಾಗಿ
ವಿಜ್ಞಾನ-ವಾದಕ್ಕೆ
ವಿಜ್ಞಾನವು
ವಿಜ್ಞಾನವೂ
ವಿಜ್ಞಾನ-ವೆನ್ನುವೆವೊ
ವಿಜ್ಞಾನ-ಶಾಸ್ತ್ರ
ವಿಜ್ಞಾನ-ಶಾಸ್ತ್ರ-ವನ್ನು
ವಿಜ್ಞಾನಿ
ವಿಜ್ಞಾನಿ-ಗಳ
ವಿಜ್ಞಾನಿ-ಗಳಂತೆ
ವಿಜ್ಞಾನಿ-ಗಳನ್ನು
ವಿಜ್ಞಾನಿ-ಗಳಿಗೂ
ವಿಜ್ಞಾನಿ-ಗಳು
ವಿಜ್ಞಾನಿಗೆ
ವಿಜ್ಞಾನಿಯ
ವಿಜ್ಞಾನಿ-ಯಂತೆ
ವಿಜ್ಞಾನಿ-ಯಾಗು-ವುದು
ವಿಜ್ಞಾನಿಯು
ವಿಜ್ಞಾನಿ-ಯೊಬ್ಬ
ವಿತ-ತಸ್ಯ
ವಿತರ್ಕ
ವಿತರ್ಕ-ಬಾಧನೇ
ವಿತರ್ಕವಿ-ಚಾರಾನನ್ದಾಸ್ಮಿ-ತಾನು-ಗ-ಮಾತ್
ವಿತರ್ಕಾ
ವಿತೃಷ್ಣಸ್ಯ
ವಿತ್ತು
ವಿದಿತ್ವಾ
ವಿದು-ಷೋಽಪಿ
ವಿದೆ
ವಿದೇಹ
ವಿದೇಹಾ-ವಸ್ಥೆ-ಯನ್ನು
ವಿದ್ದಂತೆ
ವಿದ್ಯಾಭ್ಯಾಸ
ವಿದ್ಯಾರ್ಥಿ
ವಿದ್ಯಾರ್ಥಿ-ಗ-ಳನ್ನು
ವಿದ್ಯಾರ್ಥಿ-ಗಳಿಗೆ
ವಿದ್ಯಾರ್ಥಿ-ಯಂತೆ
ವಿದ್ಯಾ-ವಂತನೂ
ವಿದ್ಯಾ-ವಂತರ
ವಿದ್ಯಾ-ವಂತ-ರಲ್ಲದ-ವ-ರಿಗೆ
ವಿದ್ಯಾ-ವಂತ-ರಿಗೆ
ವಿದ್ಯಾ-ವಂತರು
ವಿದ್ಯು
ವಿದ್ಯುಚ್ಛಕ್ತಿ
ವಿದ್ಯುಚ್ಛಕ್ತಿಯ
ವಿದ್ಯುಚ್ಛಕ್ತಿ-ಯನ್ನು
ವಿದ್ಯುತ್
ವಿದ್ಯುತ್ತು
ವಿದ್ಯುತ್ಲೋ-ಕಕ್ಕೆ
ವಿದ್ಯುತ್ಶಕ್ತಿ
ವಿದ್ಯುತ್ಶಕ್ತಿಗೂ
ವಿದ್ಯುತ್ಶಕ್ತಿಯ
ವಿದ್ಯುತ್ಶಕ್ತಿಯು
ವಿದ್ಯೆ
ವಿದ್ಯೆ-ಗ-ಳನ್ನು
ವಿದ್ಯೆಗೂ
ವಿದ್ಯೆಗೆ
ವಿದ್ಯೆಯ
ವಿದ್ಯೆ-ಯನ್ನು
ವಿದ್ಯೆ-ಯಷ್ಟು
ವಿದ್ಯೆ-ಯಿಂದ
ವಿದ್ಯೆ-ಯೊಂದೆ
ವಿದ್ವಾಂಸ-ನಿಗೂ
ವಿದ್ವಾಂಸರು
ವಿಧ
ವಿಧ-ಗ-ಳನ್ನು
ವಿಧ-ಗಳಲ್ಲಿ
ವಿಧ-ಗಳಲ್ಲಿಯೂ
ವಿಧ-ಗಳಿಲ್ಲ
ವಿಧ-ಗಳಿವೆ
ವಿಧದ
ವಿಧ-ದಲ್ಲಿ
ವಿಧ-ದಲ್ಲಿಯೂ
ವಿಧ-ದಲ್ಲೂ
ವಿಧ-ದಿಂದ
ವಿಧ-ದಿಂದ-ಲಾ-ದರೂ
ವಿಧಧ
ವಿಧ-ವಾಗಿ
ವಿಧ-ವಾ-ಗಿ-ರುವುದು
ವಿಧ-ವಾದ
ವಿಧ-ವಿಧ-ವಾದ
ವಿಧ-ವೆಯ
ವಿಧ-ವೆ-ಯರು
ವಿಧಾ-ತನು
ವಿಧಾನ
ವಿಧಾನ-ಗಳು
ವಿಧಾ-ನದ
ವಿಧಾನ-ವನ್ನು
ವಿಧಾನ-ವನ್ನೇ
ವಿಧಾ-ನವು
ವಿಧಾ-ನವೇ
ವಿಧಿ
ವಿಧಿ-ಯಿಲ್ಲ
ವಿಧಿಯೇ
ವಿಧೇಯ-ಳಾದ
ವಿನಃ
ವಿನಲ್ಲಿ
ವಿನಾ
ವಿನಿಂದ
ವಿನಿ-ಯೋಗಃ
ವಿನಿ-ಯೋಗಿ-ಸು-ವುದು
ವಿನೊಂದಿಗೆ
ವಿಪರೀತ
ವಿಪರ್ಯಯ
ವಿಪರ್ಯಯವು
ವಿಪರ್ಯಯೋ
ವಿಫಲ-ವಾಗಿದೆ
ವಿಫಲ-ವಾಗಿ-ರ-ಬಹುದು
ವಿಫಲ-ವಾ-ಗಿ-ರು-ವಂತೆಯೇ
ವಿಭ-ಜನೆ
ವಿಭಾಗ
ವಿಭಾಗ-ಗ-ಳನ್ನು
ವಿಭಾಗ-ವಾಗಿ
ವಿಭಾಗ-ವಾ-ಗಿ-ರು-ವುವು
ವಿಭಾಗ-ವಾ-ಗು-ವುದು
ವಿಭಾಗ-ವಿದೆ
ವಿಭಾಗಿಸ-ಲಾರೆವು
ವಿಭಾಗಿಸ-ಲಾರೆವೊ
ವಿಭಾಗಿ-ಸಲು
ವಿಭಾಗಿಸಲ್ಪಟ್ಟಿ-ರು-ವುವು
ವಿಭಾ-ಗಿಸಿ
ವಿಭಾ-ಗಿಸಿ-ರು-ವರು
ವಿಭಾಗಿಸು-ವರು
ವಿಭಿನ್ನ
ವಿಭಿನ್ನ-ರಾದ
ವಿಭಿನ್ನ-ವಾದ
ವಿಭು
ವಿಮರ್ಶಿ-ಸದೆ
ವಿಮರ್ಶಿಸ-ಬೇಕಾ-ದರೆ
ವಿಮರ್ಶಿಸ-ಲಾರೆವೊ
ವಿಮರ್ಶಿ-ಸಲಿಚ್ಛಿ-ಸು-ವನು
ವಿಮರ್ಶಿ-ಸಲು
ವಿಮರ್ಶಿಸಿ-ದರೆ
ವಿಮರ್ಶಿ-ಸು-ವುದು
ವಿಮರ್ಶಿ-ಸೋಣ
ವಿಮರ್ಶೆ
ವಿಮರ್ಶೆಗೂ
ವಿಮರ್ಶೆ-ಮಾಡಿ
ವಿಮೋ-ಚನೆ
ವಿಯೋಗದ
ವಿಯೋಗ-ವಾ-ಯಿತು
ವಿರ-ಬೇಕು
ವಿರ-ಲಾರದು
ವಿರಳ
ವಿರ-ಳ-ವಾ-ಗು-ವುದು
ವಿರಸ
ವಿರಾಟ್
ವಿರಾಮ
ವಿರಾ-ಮ-ದಂತಿದೆ
ವಿರಾ-ಮ-ದಿಂದ
ವಿರಾ-ಮ-ವಿಲ್ಲದೆ
ವಿರಾ-ಮ-ವೆನ್ನು
ವಿರಿ
ವಿರುದ್ಧ
ವಿರುದ್ಧ-ವಾಗಿ
ವಿರುದ್ಧ-ವಾದ
ವಿರುದ್ಧ-ವಾದುದು
ವಿರುದ್ಧವೂ
ವಿರು-ವುದು
ವಿರೋಧ
ವಿರೋಧ-ಗಳ
ವಿರೋಧದ
ವಿರೋಧ-ಭಾ-ವನೆ
ವಿರೋಧ-ಭಾವನೆ-ಗಳಿಗೆ
ವಿರೋಧ-ವನ್ನು
ವಿರೋಧ-ವಾಗ-ದಂತೆ
ವಿರೋಧ-ವಾಗಿ
ವಿರೋಧ-ವಾಗಿದೆ
ವಿರೋಧ-ವಾಗಿ-ರ-ಬಹುದು
ವಿರೋಧ-ವಾಗಿ-ರುವ
ವಿರೋಧ-ವಾಗಿ-ರು-ವಂತೆ
ವಿರೋಧ-ವಾಗಿ-ರುವುದೇ
ವಿರೋಧ-ವಾ-ಗಿಲ್ಲ
ವಿರೋಧ-ವಾಗಿ-ವೆಯೆ
ವಿರೋಧ-ವಾದ
ವಿರೋಧ-ವಾದು-ವಲ್ಲ
ವಿರೋಧ-ವಿದೆ
ವಿರೋಧ-ವಿ-ರ-ಲಿಲ್ಲ
ವಿರೋಧವೆ
ವಿರೋಧ-ಶಕ್ತಿ-ಗಳ
ವಿರೋ-ಧಾಚ್ಚ
ವಿರೋ-ಧಾಭಾಸ-ದಂತೆ
ವಿರೋ-ಧಾಭಾಸ-ದಿಂದ
ವಿರೋಧಿ
ವಿರೋಧಿ-ಗಳಲ್ಲ
ವಿರೋಧಿ-ಯಲ್ಲ
ವಿರೋಧಿ-ಯಾದ
ವಿರೋಧಿ-ಯಾ-ದದ್ದು
ವಿರೋಧಿ-ಸ-ಕೂಡದು
ವಿರೋಧಿ-ಸದಿ-ರು-ವುದು
ವಿರೋಧಿ-ಸದೆ
ವಿರೋಧಿಸಿ
ವಿರೋಧಿ-ಸಿ-ದರೆ
ವಿರೋಧಿ-ಸಿಲ್ಲ
ವಿರೋಧಿಸು
ವಿರೋಧಿ-ಸುತ್ತವೆ
ವಿರೋಧಿ-ಸುತ್ತೇನೆ
ವಿರೋಧಿ-ಸುವ
ವಿರೋಧಿ-ಸು-ವ-ವರ
ವಿರೋಧಿ-ಸು-ವ-ವ-ರಲ್ಲಿ
ವಿರೋಧಿ-ಸು-ವು-ದಕ್ಕೆ
ವಿರೋಧಿ-ಸು-ವು-ದಿಲ್ಲ
ವಿರೋಧಿ-ಸು-ವು-ದಿಲ್ಲವೋ
ವಿರೋಧಿ-ಸು-ವುದು
ವಿರೋ-ಧೋಕ್ತಿ
ವಿಲ್ಲ
ವಿಲ್ಲದ
ವಿಲ್ಲದೆ
ವಿಲ್ಲ-ದೆಯೆ
ವಿಲ್ಲಿದೆ
ವಿವರ-ಗ-ಳನ್ನು
ವಿವರ-ಗಳನ್ನೆತ್ತಿ-ಕೊಂಡು
ವಿವರ-ಗಳು
ವಿವರಣೆ
ವಿವರ-ಣೆ-ಗ-ಳನ್ನು
ವಿವರ-ಣೆ-ಗಳು
ವಿವರ-ಣೆ-ಗಿಂತಲೂ
ವಿವರ-ಣೆಗೆ
ವಿವರ-ಣೆಯ
ವಿವರ-ಣೆ-ಯಂತೆ
ವಿವರ-ಣೆ-ಯನ್ನು
ವಿವರ-ಣೆ-ಯಲ್ಲ
ವಿವರ-ಣೆ-ಯಲ್ಲಿ
ವಿವರ-ಣೆ-ಯಿತ್ತು
ವಿವರ-ಣೆ-ಯಿಲ್ಲ
ವಿವರ-ಣೆಯು
ವಿವರ-ವಾಗಿ
ವಿವರಿಸ
ವಿವರಿ-ಸ-ಬಲ್ಲ
ವಿವರಿ-ಸ-ಬಲ್ಲದು
ವಿವರಿ-ಸ-ಬಲ್ಲದೋ
ವಿವರಿ-ಸ-ಬಲ್ಲುದು
ವಿವರಿ-ಸ-ಬಹುದು
ವಿವರಿ-ಸ-ಬೇಕಾ-ಗಿದೆ
ವಿವರಿ-ಸ-ಬೇಕಾ-ದರೆ
ವಿವರಿ-ಸ-ಬೇಕು
ವಿವರಿ-ಸ-ಲಾ-ಗದೆ
ವಿವರಿ-ಸ-ಲಾ-ಗಿದೆ
ವಿವರಿ-ಸ-ಲಾಗು-ವು-ದಿಲ್ಲ
ವಿವರಿ-ಸ-ಲಾರದು
ವಿವರಿ-ಸಲಿ
ವಿವರಿ-ಸ-ಲಿ-ಅವು-ಗಳ
ವಿವರಿ-ಸಲು
ವಿವ-ರಿಸಿ
ವಿವ-ರಿಸಿ-ದಂತಾಯಿತು
ವಿವ-ರಿಸಿ-ದಂತೆ
ವಿವ-ರಿಸಿ-ದರೆ
ವಿವ-ರಿಸಿ-ರುವನು
ವಿವ-ರಿಸಿ-ರು-ವರು
ವಿವ-ರಿಸು
ವಿವರಿ-ಸುತ್ತದೆ
ವಿವ-ರಿಸುತ್ತವೆ
ವಿವ-ರಿಸುತ್ತಾರೆ
ವಿವ-ರಿಸುತ್ತಾರೆ-ಒಂದು
ವಿವ-ರಿಸುತ್ತಿದ್ದ
ವಿವ-ರಿಸುತ್ತಿದ್ದರು
ವಿವ-ರಿಸುತ್ತೀರಿ
ವಿವರಿ-ಸುತ್ತೇವೆ
ವಿವ-ರಿಸುದು
ವಿವ-ರಿಸುವ
ವಿವ-ರಿಸು-ವನು
ವಿವ-ರಿಸು-ವರು
ವಿವ-ರಿಸು-ವ-ವರೆಗೆ
ವಿವ-ರಿಸು-ವಾಗ
ವಿವರಿ-ಸು-ವು-ದಕ್ಕೆ
ವಿವ-ರಿಸು-ವುದಕ್ಕೋಸ್ಕರ
ವಿವ-ರಿಸು-ವು-ದ-ರಲ್ಲಿ
ವಿವ-ರಿಸು-ವು-ದಾದರೆ
ವಿವರಿ-ಸು-ವು-ದಿಲ್ಲ
ವಿವ-ರಿಸು-ವುದು
ವಿವ-ರಿಸು-ವುವು
ವಿವಾದ-ಗಳಲ್ಲಿ
ವಿವಾಹ-ಗಳಿಗೆ
ವಿವಾಹ-ದಿಂದ
ವಿವಿಧ
ವಿವಿಧ-ತೆಯ
ವಿವಿಧ-ತೆ-ಯನ್ನು
ವಿವಿಧ-ತೆ-ಯಲ್ಲಿ
ವಿವಿಧ-ವಾಗಿ
ವಿವಿಧ-ವಾ-ಗಿ-ರುವುದು
ವಿವೇಕ
ವಿವೇಕಖ್ಯಾತಿರವಿಪ್ಲವಾ
ವಿವೇಕಖ್ಯಾತೇರ್ಧರ್ಮ-ಮೇಘಃ
ವಿವೇಕ-ಗಳಿಂದ
ವಿವೇಕ-ಚೂಡಾ-ಮಣಿ
ವಿವೇಕಜಂ
ವಿವೇಕಜ್ಞಾನ-ವನ್ನು
ವಿವೇ-ಕದ
ವಿವೇಕ-ದ-ವರೆ
ವಿವೇಕ-ದಿಂದ
ವಿವೇಕ-ನಿಮ್ನಂ
ವಿವೇಕ-ವಿಲ್ಲ-ದಿ-ರು-ವುದು
ವಿವೇ-ಕವು
ವಿವೇಕ-ವೆಂಬು-ದನ್ನು
ವಿವೇಕ-ಶಕ್ತಿ-ಯನ್ನು
ವಿವೇಕಾ-ನಂದರ
ವಿವೇ-ಕಿಗೆ
ವಿವೇ-ಕಿನಃ
ವಿವೇ-ಚನಾ
ವಿವೇ-ಚನಾ-ಪರರೂ
ವಿವೇ-ಚನೆ
ವಿಶದ
ವಿಶದ-ಪಡಿ-ಸು-ವು-ದಿಲ್ಲವೆ
ವಿಶದ-ವಾಗಿ
ವಿಶದ-ವಾದ
ವಿಶಾದಿ-ಗಳೂ
ವಿಶಾಲ
ವಿಶಾಲ-ದೃಷ್ಟಿ-ಯಿಂದ
ವಿಶಾಲ-ಮಾಡಿ
ವಿಶಾಲ-ವಾಗ
ವಿಶಾಲ-ವಾಗ-ಬೇಕು
ವಿಶಾಲ-ವಾಗಿ
ವಿಶಾಲ-ವಾಗಿ-ರ-ಬೇಕು
ವಿಶಾಲ-ವಾಗಿ-ರುವ
ವಿಶಾಲ-ವಾಗಿ-ರುವುದು
ವಿಶಾಲ-ವಾಗು
ವಿಶಾಲ-ವಾಗುತ್ತದೆ
ವಿಶಾಲ-ವಾಗುತ್ತಿದೆ
ವಿಶಾಲ-ವಾಗು-ವುದು
ವಿಶಾಲ-ವಾದ
ವಿಶಾಲ-ವಾ-ದಂತೆ
ವಿಶಾಲ-ವಾ-ದರೆ
ವಿಶಾಲ-ವಾದು-ದಲ್ಲ-ಈಗ
ವಿಶಿಷ್ಟ
ವಿಶಿಷ್ಟಾದ್ವೈತ
ವಿಶಿಷ್ಟಾದ್ವೈತದ
ವಿಶಿಷ್ಟಾದ್ವೈತಿ
ವಿಶಿಷ್ಟಾದ್ವೈತಿ-ಗಳ
ವಿಶಿಷ್ಟಾದ್ವೈತಿ-ಗಳಿಂದ
ವಿಶಿಷ್ಟಾದ್ವೈತಿ-ಗಳು
ವಿಶಿಷ್ಟಾದ್ವೈತಿ-ಗಳು-ಇಬ್ಬರೂ
ವಿಶುದ್ಧ
ವಿಶೇಷ
ವಿಶೇಷಃ
ವಿಶೇಷ-ಣ-ಗಳಿಂದ
ವಿಶೇಷ-ಣ-ರ-ಹಿತ
ವಿಶೇಷ-ಣ-ವಾದ
ವಿಶೇಷದ
ವಿಶೇಷ-ದರ್ಶಿನ
ವಿಶೇಷ-ದಿಂದ
ವಿಶೇಷ-ವನ್ನು
ವಿಶೇಷ-ವಾಗಿ
ವಿಶೇಷ-ವಾದ
ವಿಶೇಷ-ವಾದು-ದನ್ನು
ವಿಶೇಷ-ವಾದು-ದ-ರಿಂದ
ವಿಶೇಷ-ವಿ-ರು-ವುದು
ವಿಶೇಷವೆ
ವಿಶೇಷಾರ್ಥತ್ವಾತ್
ವಿಶೇಷಾ-ವಿಶೇಷ-ಲಿಂಗ-ಮಾತ್ರಾ-ಲಿಂಗಾನಿ
ವಿಶೋಕಾ
ವಿಶ್ರಮಿ-ಸದೆ
ವಿಶ್ರಾಂತಿ
ವಿಶ್ರಾಂತಿ-ಯನ್ನು
ವಿಶ್ರಾಂತಿ-ಯಲ್ಲ
ವಿಶ್ಲೇಷಣೆ
ವಿಶ್ಲೇಷಣೆ-ಮಾಡಿ
ವಿಶ್ಲೇಷ-ಣೆಯ
ವಿಶ್ಲೇಷಣೆ-ಯಿಂದ
ವಿಶ್ಲೇಷಿಸಿ
ವಿಶ್ಲೇಷಿಸು-ವಾಗ
ವಿಶ್ಲೇಷಿ-ಸು-ವುದು
ವಿಶ್ವ
ವಿಶ್ವಕ್ಕಾಗಿ
ವಿಶ್ವಕ್ಕಿಂತ
ವಿಶ್ವಕ್ಕೆ
ವಿಶ್ವ-ಗ-ಳನ್ನು
ವಿಶ್ವ-ಗಳಲ್ಲಿ
ವಿಶ್ವ-ಚಿತ್
ವಿಶ್ವ-ಚಿತ್ತನ್ನೇ
ವಿಶ್ವ-ಚಿತ್ತಿನ
ವಿಶ್ವ-ಚೇ-ತನ-ವೆಂದು
ವಿಶ್ವ-ಚೈ-ತನ್ಯ
ವಿಶ್ವ-ಚೈ-ತನ್ಯದ
ವಿಶ್ವ-ಚೈ-ತನ್ಯ-ವನ್ನೇ
ವಿಶ್ವ-ಚೈ-ತನ್ಯ-ವಾದ
ವಿಶ್ವ-ಚೈ-ತನ್ಯ-ವಿತ್ತು
ವಿಶ್ವ-ಚೈ-ತನ್ಯವೇ
ವಿಶ್ವ-ಜೀವನ
ವಿಶ್ವದ
ವಿಶ್ವ-ದಂತೆ
ವಿಶ್ವ-ದಲ್ಲಿ
ವಿಶ್ವ-ದಲ್ಲಿ-ರುವ
ವಿಶ್ವ-ದಲ್ಲಿ-ರುವುದು
ವಿಶ್ವ-ದಲ್ಲೆಲ್ಲ
ವಿಶ್ವ-ದಲ್ಲೆಲ್ಲಾ
ವಿಶ್ವ-ದಾಚೆಗೆ
ವಿಶ್ವ-ದಿಂದ
ವಿಶ್ವ-ದೃಷ್ಟಿ-ಯಿಂದ
ವಿಶ್ವ-ದೇ-ವ-ನಿಗೆ
ವಿಶ್ವ-ದೊಳಗೊಂದು
ವಿಶ್ವ-ಧರ್ಮ
ವಿಶ್ವ-ಧರ್ಮದ
ವಿಶ್ವ-ಧರ್ಮ-ವನ್ನು
ವಿಶ್ವ-ಧರ್ಮ-ವಿ-ರು-ವುದು
ವಿಶ್ವ-ಧರ್ಮವು
ವಿಶ್ವ-ಧರ್ಮವೂ
ವಿಶ್ವ-ಧರ್ಮ-ವೆಂದರೆ
ವಿಶ್ವಪ್ರಳಯ-ವಾ-ದಾಗ
ವಿಶ್ವಪ್ರ-ವರ್ತ-ಕ-ಶಕ್ತಿ
ವಿಶ್ವ-ಬಾಂಧವ್ಯ
ವಿಶ್ವ-ಬಾಂಧವ್ಯದ
ವಿಶ್ವ-ಬಾಂಧವ್ಯ-ವನ್ನು
ವಿಶ್ವ-ಬಾಂಧವ್ಯ-ವಿದೆ
ವಿಶ್ವ-ಭಾ-ವನೆ
ವಿಶ್ವಭ್ರಾತೃತ್ವ
ವಿಶ್ವ-ಮನಸ್ಸಿನ
ವಿಶ್ವ-ರಹಸ್ಯ
ವಿಶ್ವ-ರಹಸ್ಯದ
ವಿಶ್ವ-ರಹಸ್ಯ-ವನ್ನು
ವಿಶ್ವ-ವನ್ನಾ-ಳುವ
ವಿಶ್ವ-ವನ್ನು
ವಿಶ್ವ-ವನ್ನೆ
ವಿಶ್ವ-ವನ್ನೆಲ್ಲ
ವಿಶ್ವ-ವನ್ನೆಲ್ಲಾ
ವಿಶ್ವ-ವನ್ನೇ
ವಿಶ್ವ-ವಾ-ಗಲಿ
ವಿಶ್ವ-ವಾಗಿ
ವಿಶ್ವ-ವಾಗಿದೆ
ವಿಶ್ವ-ವಾ-ಗಿ-ರುವುದು
ವಿಶ್ವ-ವಾ-ಗು-ವುದು
ವಿಶ್ವ-ವಿಲ್ಲಿದೆ
ವಿಶ್ವವು
ವಿಶ್ವವೂ
ವಿಶ್ವವೆ
ವಿಶ್ವ-ವೆನ್ನಿ
ವಿಶ್ವ-ವೆನ್ನುವೆವೊ
ವಿಶ್ವ-ವೆಲ್ಲ
ವಿಶ್ವ-ವೆಲ್ಲ-ದರ
ವಿಶ್ವ-ವೆಲ್ಲವು
ವಿಶ್ವ-ವೆಲ್ಲವೂ
ವಿಶ್ವ-ವೆಲ್ಲಾ
ವಿಶ್ವವೇ
ವಿಶ್ವವ್ಯಾಪಿ
ವಿಶ್ವವ್ಯಾಪಿ-ಯಾದ
ವಿಶ್ವ-ಶಕ್ತಿ
ವಿಶ್ವ-ಶಕ್ತಿಯ
ವಿಶ್ವ-ಸತ್ಯದ
ವಿಶ್ವ-ಸಮಸ್ಯೆ-ಯನ್ನು
ವಿಶ್ವ-ಸಹೋ-ದರತ್ವದ
ವಿಶ್ವ-ಸಿದ್ಧಾಂತ-ವನ್ನು
ವಿಶ್ವಾತ್ಮ
ವಿಶ್ವಾತ್ಮ-ನಲ್ಲಿ
ವಿಶ್ವಾತ್ಮ-ನೆಂದು
ವಿಶ್ವಾದ್ಯಂತವೂ
ವಿಶ್ವಾಧಿ-ಪತಿ-ಯಾದ
ವಿಶ್ವಾನು-ಕಂಪ
ವಿಶ್ವಾಸ
ವಿಶ್ವಾಸ-ದಿಂದ
ವಿಶ್ವಾಸ-ವನ್ನು
ವಿಶ್ವಾಸ-ವಿಡ
ವಿಶ್ವಾಸ-ವಿದೆಯೆ
ವಿಶ್ವಾಸವೇ
ವಿಶ್ವೇಶ್ವರ
ವಿಶ್ವೇಷಣೆ
ವಿಶ್ವ್ಯಾಪಿ-ಯಾಗ-ಬೇಕು
ವಿಷ
ವಿಷ-ದಿಂದ
ವಿಷಯ
ವಿಷಯ-ಕ-ವಾಗಿ
ವಿಷಯ-ಕ-ವಾದ
ವಿಷಯಕ್ಕೆ
ವಿಷಯ-ಗಲು
ವಿಷಯ-ಗಳ
ವಿಷಯ-ಗಳನ್ನಾಲೋ-ಚಿ-ಸಲು
ವಿಷಯ-ಗಳನ್ನು
ವಿಷಯ-ಗಳನ್ನೂ
ವಿಷಯ-ಗಳನ್ನೆಲ್ಲ
ವಿಷಯ-ಗಳನ್ನೊಳ-ಗೊಂಡ
ವಿಷಯ-ಗಳಲ್ಲಿ
ವಿಷಯ-ಗಳಾದ
ವಿಷಯ-ಗಳಿಗೆ
ವಿಷಯ-ಗಳಿವೆ
ವಿಷಯ-ಗಳು
ವಿಷಯ-ಗಳೆಲ್ಲ
ವಿಷಯ-ಜಗತ್ತು
ವಿಷಯದ
ವಿಷಯ-ದಲ್ಲಿ
ವಿಷಯ-ದಲ್ಲಿಯ
ವಿಷಯ-ದಲ್ಲಿಯೂ
ವಿಷಯ-ದಲ್ಲಿ-ರುವ
ವಿಷಯ-ದಲ್ಲೂ
ವಿಷಯ-ದಲ್ಲೆ
ವಿಷಯ-ದಲ್ಲೇ
ವಿಷಯ-ಪ್ರಲೋ-ಭನ-ಗಳಿಗೆ
ವಿಷಯ-ಭೋಗ-ಗಳಿಂದ
ವಿಷಯ-ವತೀ
ವಿಷಯ-ವನ್ನು
ವಿಷಯ-ವನ್ನೊಪ್ಪಿ-ಕೊಂಡರೆ
ವಿಷಯ-ವಸ್ತು-ಗಳನ್ನು
ವಿಷಯ-ವಾಗಿ
ವಿಷಯ-ವಾಗಿದೆ
ವಿಷಯ-ವಾದ
ವಿಷಯ-ವಿದೆ
ವಿಷಯವು
ವಿಷಯವೂ
ವಿಷಯ-ವೆಂದರೆ
ವಿಷಯ-ವೆಲ್ಲ
ವಿಷಯ-ವೆಲ್ಲವೂ
ವಿಷಯವೇ
ವಿಷಯಾ-ಪೇಕ್ಷೀ
ವಿಷಯಿ
ವಿಷಯಿ-ಯನ್ನು
ವಿಷಯಿಯು
ವಿಷಯೀ
ವಿಷ-ಯೀ-ಕರ-ಣವೇ
ವಿಷ-ಯೀ-ಕರಿಸ-ಲಾರೆವು
ವಿಷ-ಯೀ-ಕರಿ-ಸು-ವಿರಿ
ವಿಷ-ವನ್ನು
ವಿಷವು
ವಿಸ್ತರ
ವಿಸ್ತ-ರಿಸಿ
ವಿಸ್ತ-ರಿಸು-ವಂತೆ
ವಿಸ್ತರಿ-ಸು-ವುದು
ವಿಸ್ತಾರ
ವಿಸ್ತಾರ-ವಾಗಿ
ವಿಸ್ತಾರ-ವಾಗಿದೆ
ವಿಸ್ತಾರ-ವಾಗುತ್ತ
ವಿಸ್ತಾರ-ವಾಗುತ್ತಾ
ವಿಸ್ತಾರ-ವಾಗು-ವು-ದಕ್ಕೆ
ವಿಸ್ತಾರ-ವಾ-ಗು-ವುದು
ವಿಸ್ತಾರ-ವಾದ
ವಿಹ-ರಿಸು-ವುದೋ
ವಿಹಾ-ರಕ್ಕೆ
ವೀಕ್ಷಣಾಲ-ಯಕ್ಕೆ
ವೀಕ್ಷಣೆ
ವೀತ-ರಾಗ-ವಿಷಯಂ
ವೀರರು
ವೀರ-ರೆಲ್ಲಾ
ವೀರೋಧಿ-ಸುವ
ವೀರ್ಯ
ವೀರ್ಯ-ಕಣ-ದಲ್ಲಿ
ವೀರ್ಯ-ಲಾಭಃ
ವು
ವುಗಳೆಲ್ಲ
ವುತ್ತಿ-ರು-ವೆವು
ವುದಕ್ಕಾ-ಗಲೀ
ವುದಕ್ಕಿಂತ
ವುದಕ್ಕಿಂತಲೂ
ವುದಕ್ಕೆ
ವುದನ್ನು
ವುದನ್ನೂ
ವುದ-ರಲ್ಲಿ
ವುದ-ರಿಂದ
ವುದ-ರಿಂದಲೂ
ವುದಲ್ಲ
ವುದಿಲ್ಲ
ವುದು
ವುದು-ಇದೇ
ವುದು-ಇ-ವೆಲ್ಲ
ವುದು-ಯಾವ
ವುದೂ
ವುದೆ
ವುದೆಂದರೆ
ವುದೆಂದ-ರೇ-ನೆಂಬುದು
ವುದೆಂದು
ವುದೆಂಬು-ದನ್ನು
ವುದೆಲ್ಲಿಗೆ
ವುದೇ
ವುದೇನೊ
ವುದೊ
ವುದೋ
ವುಳ್ಳದ್ದು
ವುವು
ವುವೂ
ವೃಕ್ಷ
ವೃಕ್ಷ-ಗಳೆಂಬ
ವೃಕ್ಷದ
ವೃತ್ತ
ವೃತ್ತದ
ವೃತ್ತ-ದಂತೆ
ವೃತ್ತ-ದಲ್ಲಿ
ವೃತ್ತ-ದೊ-ಳಗೆ
ವೃತ್ತ-ದೊಳಗೊಂದು
ವೃತ್ತಯಃ
ವೃತ್ತ-ವನ್ನು
ವೃತ್ತ-ವಾ-ಗು-ವುದು
ವೃತ್ತ-ವಿದ್ದ
ವೃತ್ತ-ವಿಲ್ಲ
ವೃತ್ತಾ-ಕಾರ-ದಲ್ಲಿ
ವೃತ್ತಾ-ಕಾರ-ವಾಗಿ-ರುತ್ತದೆ
ವೃತ್ತಿ
ವೃತ್ತಿ-ಗಳ
ವೃತ್ತಿ-ಗ-ಳನ್ನು
ವೃತ್ತಿ-ಗಳಿ-ರು-ವುವು
ವೃತ್ತಿ-ಗಳು
ವೃತ್ತಿ-ಗಳೇ
ವೃತ್ತಿ-ಗಳೇ-ಳು-ವಂತೆ
ವೃತ್ತಿಗೆ
ವೃತ್ತಿಯ
ವೃತ್ತಿ-ಯನ್ನಾಗಿ
ವೃತ್ತಿ-ಯನ್ನು
ವೃತ್ತಿ-ಯಾಗಿ
ವೃತ್ತಿ-ಯಾ-ಗು-ವುದು
ವೃತ್ತಿಯೇ
ವೃತ್ತಿರ್ನಿದ್ರಾ
ವೃತ್ತಿರ್ಮಹಾ-ವಿ-ದೇಹಾ
ವೃತ್ತಿ-ಸಾ-ರೂಪ್ಯ-ಮಿತ-ರತ್ರ
ವೃಥಾ
ವೃದ್ಧ
ವೃದ್ಧ-ನಾಗಿ
ವೃದ್ಧ-ನಾ-ಗು-ವುದು
ವೃದ್ಧ-ನಾ-ದರೆ
ವೃದ್ಧನು
ವೃದ್ಧ-ನೊಬ್ಬನು
ವೃದ್ಧರ
ವೃದ್ಧ-ರಲ್ಲಿ
ವೃದ್ಧ-ರಾಗ-ಬಹುದು
ವೃದ್ಧ-ರಿಗೆ
ವೃದ್ಧಾಪ್ಯ
ವೃದ್ಧಾಪ್ಯ-ದಲ್ಲಿ
ವೃದ್ಧಾಪ್ಯ-ದ-ವರೆಗೆ
ವೃದ್ಧಿ
ವೃದ್ಧಿಕ್ಷಯ-ಗಳಿಗೆ
ವೃದ್ಧಿಕ್ಷಯ-ಗಳಿ-ರು-ವಂತೆ
ವೃದ್ಧಿಗೆ
ವೃದ್ಧಿ-ಗೊಳಿಸಿ
ವೃದ್ಧಿ-ಗೊಳಿ-ಸುವ
ವೃದ್ಧಿ-ಯಾಗ-ಲೆಂಬುದೆ
ವೃದ್ಧಿ-ಯಾಗ-ಲೆಂಬುದೇ
ವೃದ್ಧಿ-ಯಾಗಿ
ವೃದ್ಧಿ-ಯಾ-ಗಿಲ್ಲ
ವೃದ್ಧಿ-ಯಾ-ಗುತ್ತದೆ
ವೃದ್ಧಿ-ಯಾಗುತ್ತಾ
ವೃದ್ಧಿ-ಯಾಗುತ್ತಿದೆ
ವೃದ್ಧಿ-ಯಾಗುತ್ತಿ-ರು-ವರು
ವೃದ್ಧಿ-ಯಾಗು-ವು-ದಿಲ್ಲ
ವೃದ್ಧಿ-ಯಾಗು-ವುದೆ
ವೃದ್ಧಿ-ಯಾ-ದರೆ
ವೃದ್ಧಿ-ಯಾ-ದವು
ವೆಂತಲೂ
ವೆಂದರೆ
ವೆಂದರೇ-ನೆಂಬು-ದನ್ನು
ವೆಂದು
ವೆಂದೂ
ವೆಂಬ
ವೆಂಬು-ದನ್ನು
ವೆಂಬು-ದಿಲ್ಲ
ವೆಂಬುದು
ವೆಂಬು-ದೊಂದು
ವೆದ್ದಿ-ರು-ವಾಗ
ವೆನು
ವೆನೆಂದು
ವೆನ್ನ-ಲಾಗು-ವು-ದಿಲ್ಲ
ವೆನ್ನು-ವುದು
ವೆನ್ನು-ವು-ದೆಲ್ಲ
ವೆಲ್ಲ
ವೆಲ್ಲವೂ
ವೆಲ್ಲಿ-ರುವುದೋ
ವೆವು
ವೇಕೆ
ವೇಗ
ವೇಗ-ದಲ್ಲಿ
ವೇಗ-ದಿಂದ
ವೇಗ-ವನ್ನು
ವೇಗ-ವಾಗಿ
ವೇಗ-ವಾಗಿಯೊ
ವೇಗ-ವಾಗಿ-ರು-ವು-ದ-ರಿಂದ
ವೇಗ-ವಾದ
ವೇಗ-ವಿದೆ
ವೇಗವು
ವೇದ
ವೇದ-ಋಷಿ-ಗಳು
ವೇದ-ಗಳ
ವೇದ-ಗ-ಳನ್ನು
ವೇದ-ಗಳನ್ನೆಲ್ಲ
ವೇದ-ಗಳಲ್ಲಿ
ವೇದ-ಗಳಲ್ಲಿದೆ
ವೇದ-ಗಳಿಂದ
ವೇದ-ಗಳಿಂದಲೂ
ವೇದ-ಗಳು
ವೇದ-ಗಳೂ
ವೇದದ
ವೇದ-ದಲ್ಲಿ
ವೇದ-ದಲ್ಲಿದೆ
ವೇದ-ದಲ್ಲಿ-ರುವ
ವೇದ-ದಲ್ಲಿ-ರು-ವಂತೆ
ವೇದ-ದಲ್ಲೆಲ್ಲೂ
ವೇದ-ದಿಂದ
ವೇದ-ನೆ-ಗ-ಳನ್ನು
ವೇದ-ನೆ-ಗ-ಳನ್ನೂ
ವೇದ-ಭಾಗ-ಗ-ಳನ್ನು
ವೇದ-ವನ್ನು
ವೇದವು
ವೇದ-ಸಾ-ಹಿತ್ಯ-ದಲ್ಲಿ
ವೇದಾಂತ
ವೇದಾಂತಕ್ಕೂ
ವೇದಾಂತ-ಗ-ಳಾದ
ವೇದಾಂತ-ಗಳೆ-ರಡೂ
ವೇದಾಂತ-ತತ್ತ್ವವು
ವೇದಾಂತತ್ತ್ವವು
ವೇದಾಂತದ
ವೇದಾಂತ-ದಲ್ಲಿ
ವೇದಾಂತ-ದಲ್ಲಿಟ್ಟಿ-ರು-ವರು
ವೇದಾಂತ-ದಲ್ಲೆಲ್ಲಾ
ವೇದಾಂತ-ದಿಂದ
ವೇದಾಂತ-ವನ್ನು
ವೇದಾಂತ-ವಾ-ದರೋ
ವೇದಾಂತವು
ವೇದಾಂತವೂ
ವೇದಾಂತವೆ
ವೇದಾಂತ-ಸಿದ್ಧಾಂತವು
ವೇದಾಂತಿ
ವೇದಾಂತಿ-ಗಳ
ವೇದಾಂತಿ-ಗಳಲ್ಲಿ
ವೇದಾಂತಿ-ಗಳಿಗೂ
ವೇದಾಂತಿ-ಗಳಿಗೆ
ವೇದಾಂತಿ-ಗಳಿ-ರು-ವರು
ವೇದಾಂತಿ-ಗಳು
ವೇದಾಂತಿ-ಗಳೆಲ್ಲ
ವೇದಾಂತಿ-ಗಳೊ
ವೇದಾಂತಿಗೆ
ವೇದಾಂತಿಯು
ವೇದಾಧ್ಯಯ-ನಕ್ಕೆ
ವೇದಾಧ್ಯಯನ-ದಿಂದ
ವೇದಿಕೆ
ವೇದಿ-ಕೆಗೆ
ವೇದಿ-ಕೆಯೂ
ವೇದ್ಯ
ವೇದ್ಯ-ತೆಯು
ವೇದ್ಯ-ವಾಗ-ಲಾರದು
ವೇದ್ಯ-ವಾಗಿ
ವೇದ್ಯ-ವಾಗಿದೆ
ವೇದ್ಯ-ವಾಗು
ವೇದ್ಯ-ವಾಗುತ್ತದೆ
ವೇದ್ಯ-ವಾಗು-ವುದು
ವೇದ್ಯ-ವಾದ
ವೇನು
ವೇನೆಂದರೆ
ವೇನೆಂಬು-ದನ್ನು
ವೇನೊ
ವೇಳೆ
ವೇಳೆಗೆ
ವೇಳೆ-ಯನ್ನು
ವೇಳೆ-ಯಲ್ಲಿ
ವೇಳೆಯೋ
ವೇಷ-ದಲ್ಲಿ
ವೈಕುಂಠಕ್ಕೆ
ವೈಕುಂಠದ
ವೈಚಾರಿಕ
ವೈಚಾರಿಕ-ವಾಗಿ
ವೈಜ್ಞಾನಿಕ
ವೈಜ್ಞಾನಿ-ಕರ
ವೈಜ್ಞಾನಿ-ಕ-ರಿಗೂ
ವೈಜ್ಞಾನಿ-ಕ-ರಿಗೆ
ವೈಜ್ಞಾನಿ-ಕರು
ವೈಜ್ಞಾನಿ-ಕರೂ
ವೈಜ್ಞಾನಿ-ಕ-ವಾಗಿ
ವೈಜ್ಞಾನಿ-ಕ-ವಾದು-ದಲ್ಲ
ವೈಜ್ಞಾನಿ-ಕ-ವಾ-ದುದು
ವೈದಿಕ
ವೈದ್ಯನಿ-ಗಿಂತ
ವೈದ್ಯನು
ವೈದ್ಯರು
ವೈಧವ್ಯ-ವನ್ನು
ವೈಭವ
ವೈಭವ-ದಲ್ಲಿ
ವೈಭವ-ಯುಕ್ತ-ವಾದು-ದೆಲ್ಲ
ವೈಮನಸ್ಯದ
ವೈಯಕ್ತಿ-ಕತೆ
ವೈಯಕ್ತಿ-ಕತೆ-ಯನ್ನು
ವೈರತ್ಯಾಗಃ
ವೈರಾಗ್ಯ
ವೈರಾಗ್ಯ-ಗಳಿಂದ
ವೈರಾಗ್ಯದ
ವೈರಾಗ್ಯಮ್
ವೈರಾಗ್ಯ-ವಿದೆ
ವೈರಾಗ್ಯ-ವೆಂದು
ವೈರಿ-ಗ-ಳನ್ನು
ವೈರಿ-ಗಳಿಂದ
ವೈರಿ-ಗಳು
ವೈರಿಯೂ
ವೈವಿಧ್ಯ
ವೈವಿಧ್ಯಕ್ಕೆ
ವೈವಿಧ್ಯ-ಗಳ
ವೈವಿಧ್ಯ-ಗ-ಳನ್ನು
ವೈವಿಧ್ಯ-ಗಳೆಲ್ಲ
ವೈವಿಧ್ಯದ
ವೈವಿಧ್ಯ-ದಲ್ಲಿ
ವೈವಿಧ್ಯ-ದಲ್ಲೆಲ್ಲ
ವೈವಿಧ್ಯ-ದಿಂದ
ವೈವಿಧ್ಯ-ಪೂರ್ಣ
ವೈವಿಧ್ಯ-ಮಯ
ವೈವಿಧ್ಯ-ವನ್ನು
ವೈವಿಧ್ಯ-ವನ್ನೂ
ವೈವಿಧ್ಯ-ವನ್ನೆಲ್ಲ
ವೈವಿಧ್ಯ-ವೆಲ್ಲ-ವನ್ನು
ವೈವಿಧ್ಯವೇ
ವೈಶಾಲ್ಯ-ವನ್ನು
ವೈಶಿಷ್ಟ್ಯ-ಗಳೂ
ವೈಶಿಷ್ಟ್ಯದ
ವೈಶಿಷ್ಟ್ಯ-ವನ್ನು
ವೈಶಿಷ್ಟ್ಯ-ವನ್ನೂ
ವೈಶಿಷ್ಟ್ಯ-ವಿದೆ
ವೈಶಿಷ್ಟ್ಯ-ವೆಂದರೆ
ವೊಂದಿದೆ
ವೊಂದೆ
ವೊಂದೇ
ವ್ಯಕಿತ್ತ್ವ-ವಾಗಿ
ವ್ಯಕ್ತ
ವ್ಯಕ್ತಕ್ಕಿಂತ
ವ್ಯಕ್ತ-ಗೊಳಿ-ಸ-ಬೇಕಾ-ದರೆ
ವ್ಯಕ್ತ-ಗೊಳಿ-ಸ-ಬೇಕು
ವ್ಯಕ್ತ-ಗೊಳಿ-ಸ-ಲಾ-ಗದ
ವ್ಯಕ್ತ-ಗೊಳಿ-ಸ-ಲಾರಿರಿ
ವ್ಯಕ್ತ-ಗೊಳಿ-ಸ-ಲಾರೆ
ವ್ಯಕ್ತ-ಗೊಳಿ-ಸ-ಲಾರೆವು
ವ್ಯಕ್ತ-ಗೊಳಿ-ಸಲು
ವ್ಯಕ್ತ-ಗೊಳಿಸಿ
ವ್ಯಕ್ತ-ಗೊಳಿ-ಸಿ-ದನೊ
ವ್ಯಕ್ತ-ಗೊಳಿ-ಸಿ-ದರೆ
ವ್ಯಕ್ತ-ಗೊಳಿ-ಸಿದೆ
ವ್ಯಕ್ತ-ಗೊಳಿ-ಸಿ-ರಲಿ
ವ್ಯಕ್ತ-ಗೊಳಿ-ಸಿ-ರು-ವರು
ವ್ಯಕ್ತ-ಗೊಳಿಸು
ವ್ಯಕ್ತ-ಗೊಳಿ-ಸುತ್ತದೆ
ವ್ಯಕ್ತ-ಗೊಳಿ-ಸುತ್ತಾ
ವ್ಯಕ್ತ-ಗೊಳಿ-ಸುತ್ತಿದೆ
ವ್ಯಕ್ತ-ಗೊಳಿ-ಸುತ್ತೇವೆ
ವ್ಯಕ್ತ-ಗೊಳಿ-ಸುವ
ವ್ಯಕ್ತ-ಗೊಳಿ-ಸು-ವನು
ವ್ಯಕ್ತ-ಗೊಳಿ-ಸು-ವು-ದಕ್ಕಾಗಿ
ವ್ಯಕ್ತ-ಗೊಳಿ-ಸು-ವು-ದಕ್ಕೆ
ವ್ಯಕ್ತ-ಗೊಳಿ-ಸು-ವುದು
ವ್ಯಕ್ತ-ಗೊಳಿ-ಸು-ವುವು
ವ್ಯಕ್ತ-ಗೊಳ್ಳುತ್ತಿ-ರುವ
ವ್ಯಕ್ತ-ಗೊಳ್ಳುತ್ತಿ-ರುವನು
ವ್ಯಕ್ತ-ಗೊಳ್ಳು-ವು-ದಕ್ಕೆ
ವ್ಯಕ್ತದ
ವ್ಯಕ್ತ-ದಾಚೆ
ವ್ಯಕ್ತ-ನಾಗುತ್ತಾನೆ
ವ್ಯಕ್ತ-ಪಡಸ-ಬಲ್ಲದು
ವ್ಯಕ್ತ-ಪಡಿಸ
ವ್ಯಕ್ತ-ಪಡಿ-ಸ-ಲಾರ-ದೆ-ಹೋ-ದುದು
ವ್ಯಕ್ತ-ಪಡಿ-ಸ-ಲಾರವು
ವ್ಯಕ್ತ-ಪಡಿ-ಸಲು
ವ್ಯಕ್ತ-ಪಡಿಸಿ
ವ್ಯಕ್ತ-ಪಡಿ-ಸಿ-ಕೊಂಡಂತೆ
ವ್ಯಕ್ತ-ಪಡಿಸು
ವ್ಯಕ್ತ-ಪಡಿ-ಸುತ್ತೇವೆ
ವ್ಯಕ್ತ-ಪಡಿ-ಸುವ
ವ್ಯಕ್ತ-ಪಡಿ-ಸು-ವರು
ವ್ಯಕ್ತ-ಪಡಿ-ಸು-ವುದು
ವ್ಯಕ್ತ-ಪಡಿ-ಸು-ವುದೇ
ವ್ಯಕ್ತ-ಮಾಡ-ಬಹು
ವ್ಯಕ್ತ-ಮಾ-ಡಲು
ವ್ಯಕ್ತ-ರೂಪ-ಗಳು
ವ್ಯಕ್ತ-ರೂಪ-ಧಾರಣೆ
ವ್ಯಕ್ತ-ವನ್ನು
ವ್ಯಕ್ತ-ವನ್ನೇ
ವ್ಯಕ್ತ-ವಾಗದ
ವ್ಯಕ್ತ-ವಾಗ-ದಂತೆ
ವ್ಯಕ್ತ-ವಾಗ-ಬೇಕಾ-ದರೆ
ವ್ಯಕ್ತ-ವಾಗ-ಲಾರದು
ವ್ಯಕ್ತ-ವಾ-ಗಲು
ವ್ಯಕ್ತ-ವಾಗ-ಲೇ-ಬೇಕು
ವ್ಯಕ್ತ-ವಾಗಿ
ವ್ಯಕ್ತ-ವಾಗಿದೆ
ವ್ಯಕ್ತ-ವಾಗಿ-ರ-ಬಹುದು
ವ್ಯಕ್ತ-ವಾಗಿ-ರುವ
ವ್ಯಕ್ತ-ವಾಗಿ-ರುವುದು
ವ್ಯಕ್ತ-ವಾಗಿ-ರು-ವುವು
ವ್ಯಕ್ತ-ವಾಗಿವೆ
ವ್ಯಕ್ತ-ವಾಗಿ-ವೆಯೊ
ವ್ಯಕ್ತ-ವಾಗಿ-ವೆಯೋ
ವ್ಯಕ್ತ-ವಾಗುತ್ತ
ವ್ಯಕ್ತ-ವಾಗುತ್ತದೆ
ವ್ಯಕ್ತ-ವಾಗುತ್ತಾ
ವ್ಯಕ್ತ-ವಾಗುತ್ತಿತ್ತೊ
ವ್ಯಕ್ತ-ವಾಗುತ್ತಿದೆ
ವ್ಯಕ್ತ-ವಾಗುತ್ತಿ-ರುವ
ವ್ಯಕ್ತ-ವಾಗುತ್ತಿ-ರು-ವಂತೆ
ವ್ಯಕ್ತ-ವಾಗುತ್ತಿ-ರುವುದು
ವ್ಯಕ್ತ-ವಾಗುತ್ತಿ-ರುವುದೂ
ವ್ಯಕ್ತ-ವಾಗುತ್ತಿ-ರುವುದೆ
ವ್ಯಕ್ತ-ವಾಗುವ
ವ್ಯಕ್ತ-ವಾಗು-ವಂತೆ
ವ್ಯಕ್ತ-ವಾಗು-ವನು
ವ್ಯಕ್ತ-ವಾಗು-ವು-ದಕ್ಕೆ
ವ್ಯಕ್ತ-ವಾಗು-ವುದು
ವ್ಯಕ್ತ-ವಾಗು-ವುದೆಂಬು-ದನ್ನು
ವ್ಯಕ್ತ-ವಾಗು-ವುವು
ವ್ಯಕ್ತ-ವಾದ
ವ್ಯಕ್ತ-ವಾ-ದರೂ
ವ್ಯಕ್ತ-ವಾದುದು
ವ್ಯಕ್ತ-ವಾದು-ವೆಲ್ಲ
ವ್ಯಕ್ತ-ಸೂಕ್ಷ್ಮಾ
ವ್ಯಕ್ತಸ್ಥಿತಿಯೇ
ವ್ಯಕ್ತಸ್ವ-ರೂಪ
ವ್ಯಕ್ತಾ-ವಸ್ಥೆ
ವ್ಯಕ್ತಾ-ವಸ್ಥೆಗೆ
ವ್ಯಕ್ತಾ-ವಸ್ಥೆಯ
ವ್ಯಕ್ತಾ-ವಸ್ಥೆಯು
ವ್ಯಕ್ತಾ-ವಸ್ಥೆಯೇ
ವ್ಯಕ್ತಿ
ವ್ಯಕ್ತಿ-ಗತ
ವ್ಯಕ್ತಿ-ಗತ-ವಾದ
ವ್ಯಕ್ತಿ-ಗಳ
ವ್ಯಕ್ತಿ-ಗಳನ್ನಾಗಿ
ವ್ಯಕ್ತಿ-ಗ-ಳನ್ನು
ವ್ಯಕ್ತಿ-ಗಳಲ್ಲ
ವ್ಯಕ್ತಿ-ಗಳಲ್ಲಿಯೂ
ವ್ಯಕ್ತಿ-ಗಳಾ-ಗಿಲ್ಲ
ವ್ಯಕ್ತಿ-ಗಳಿಂದ
ವ್ಯಕ್ತಿ-ಗಳಿ-ಗಾದ
ವ್ಯಕ್ತಿ-ಗಳು
ವ್ಯಕ್ತಿ-ಗ-ಳೆಂದು
ವ್ಯಕ್ತಿ-ಗಳೆಲ್ಲ
ವ್ಯಕ್ತಿಗೂ
ವ್ಯಕ್ತಿಗೆ
ವ್ಯಕ್ತಿ-ಜೀವ-ನಕ್ಕೆ
ವ್ಯಕ್ತಿತ್ವ-ವನ್ನು
ವ್ಯಕ್ತಿತ್ವ
ವ್ಯಕ್ತಿತ್ವಕ್ಕಾಗಿ
ವ್ಯಕ್ತಿತ್ವಕ್ಕೆ
ವ್ಯಕ್ತಿತ್ವ-ಗಳು
ವ್ಯಕ್ತಿತ್ವದ
ವ್ಯಕ್ತಿತ್ವ-ಭಾವ
ವ್ಯಕ್ತಿತ್ವ-ವನ್ನು
ವ್ಯಕ್ತಿತ್ವ-ವಿದೆ
ವ್ಯಕ್ತಿತ್ವ-ವಿಲ್ಲ
ವ್ಯಕ್ತಿತ್ವವು
ವ್ಯಕ್ತಿತ್ವವೂ
ವ್ಯಕ್ತಿತ್ವ-ವೆಂದಾಗಲಿ
ವ್ಯಕ್ತಿತ್ವವೇ
ವ್ಯಕ್ತಿ-ದೃಷ್ಟಿ-ಯಿಂದ
ವ್ಯಕ್ತಿಯ
ವ್ಯಕ್ತಿ-ಯನ್ನು
ವ್ಯಕ್ತಿ-ಯಲ್ಲ
ವ್ಯಕ್ತಿ-ಯಲ್ಲಿ
ವ್ಯಕ್ತಿ-ಯಲ್ಲಿಯೂ
ವ್ಯಕ್ತಿ-ಯ-ವರೆಗೆ
ವ್ಯಕ್ತಿ-ಯಾಗಿ-ರ-ಬೇಕು
ವ್ಯಕ್ತಿ-ಯಾಗಿ-ರುವನು
ವ್ಯಕ್ತಿ-ಯಾ-ದನು
ವ್ಯಕ್ತಿಯು
ವ್ಯಕ್ತಿಯೂ
ವ್ಯಕ್ತಿಯೆ
ವ್ಯಕ್ತಿಯೇ
ವ್ಯಕ್ತಿ-ಯೊಬ್ಬನ
ವ್ಯಕ್ತೀ-ಕರ-ಣ-ಗೊಂಡ-ವನು
ವ್ಯತ್ಯಾಸ
ವ್ಯತ್ಯಾಸಕ್ಕೆ
ವ್ಯತ್ಯಾಸಕ್ಕೆಲ್ಲ
ವ್ಯತ್ಯಾಸಕ್ಕೆಲ್ಲಾ
ವ್ಯತ್ಯಾಸ-ಗಳಿ-ಗೆಲ್ಲಾ
ವ್ಯತ್ಯಾಸ-ಗಳಿವೆ
ವ್ಯತ್ಯಾಸ-ಗಳು
ವ್ಯತ್ಯಾಸ-ಗಳೆಲ್ಲವೂ
ವ್ಯತ್ಯಾಸ-ದಲ್ಲಿ
ವ್ಯತ್ಯಾಸ-ದೊಂದಿಗೆ
ವ್ಯತ್ಯಾಸ-ವನ್ನು
ವ್ಯತ್ಯಾಸ-ವನ್ನೂ
ವ್ಯತ್ಯಾಸ-ವಾಗ-ಬಹುದು
ವ್ಯತ್ಯಾಸ-ವಾಗಿ
ವ್ಯತ್ಯಾಸ-ವಾಗಿದೆ
ವ್ಯತ್ಯಾಸ-ವಾಗಿ-ರು-ವು-ದ-ರಿಂದ
ವ್ಯತ್ಯಾಸ-ವಾ-ಗು-ವುದು
ವ್ಯತ್ಯಾಸ-ವಾದ
ವ್ಯತ್ಯಾಸ-ವಾ-ದರೂ
ವ್ಯತ್ಯಾಸ-ವಿದೆ
ವ್ಯತ್ಯಾಸ-ವಿರು
ವ್ಯತ್ಯಾಸ-ವಿರು-ವಂತೆ
ವ್ಯತ್ಯಾಸ-ವಿರು-ವುದು
ವ್ಯತ್ಯಾಸ-ವಿಲ್ಲ
ವ್ಯತ್ಯಾಸ-ವಿಲ್ಲ-ದಂತೆ
ವ್ಯತ್ಯಾಸ-ವಿಷ್ಟೆ
ವ್ಯತ್ಯಾಸವು
ವ್ಯತ್ಯಾಸವೂ
ವ್ಯತ್ಯಾಸವೆ
ವ್ಯತ್ಯಾಸ-ವೆಲ್ಲ
ವ್ಯತ್ಯಾಸವೇ
ವ್ಯತ್ಯಾಸ-ವೇನೂ
ವ್ಯಥೆ
ವ್ಯಥೆ-ಗಳು
ವ್ಯಥೆ-ಗೀಡಾಗುತ್ತೇವೆ
ವ್ಯಥೆ-ಗೀಡಾಗು-ವು-ದನ್ನು
ವ್ಯಥೆ-ಗೀಡಾ-ದರೆ
ವ್ಯಥೆ-ಪಡ-ಬೇಕು
ವ್ಯಥೆ-ಪಡ-ಬೇಡಿ
ವ್ಯಥೆ-ಪಡಿ
ವ್ಯಥೆ-ಪಡುತ್ತಿರು
ವ್ಯಥೆ-ಪಡುತ್ತಿರು-ವರು
ವ್ಯಥೆ-ಪಡು-ವಿರಿ
ವ್ಯಥೆ-ಪಡು-ವು-ದಿಲ್ಲ
ವ್ಯಥೆ-ಪ-ಡು-ವುದು
ವ್ಯಥೆಯ
ವ್ಯಥೆ-ಯನ್ನು
ವ್ಯಥೆ-ಯಾಗುತ್ತಿದೆ
ವ್ಯಥೆ-ಯಾ-ಗು-ವುದು
ವ್ಯಥೆ-ಯೆಲ್ಲ
ವ್ಯಭಿ-ಚಾರವೇ
ವ್ಯಯ-ಮಾಡಿ
ವ್ಯಯ-ಮಾಡುತ್ತಿದೆ
ವ್ಯಯ-ವಾ-ಗಿಲ್ಲ
ವ್ಯಯ-ವಾಗುತ್ತದೆ
ವ್ಯಯ-ವಾಗು-ವ-ವ-ರಿಗೆ
ವ್ಯಯ-ವಾ-ಗು-ವುದು
ವ್ಯರ್ಥ
ವ್ಯರ್ಥ-ಮಾಡಿ-ದರೆ
ವ್ಯರ್ಥ-ಮಾಡು-ವರು
ವ್ಯರ್ಥ-ವಾಗಿದೆ
ವ್ಯರ್ಥ-ವಾ-ಗಿಲ್ಲ-ವೆಂಬುದು
ವ್ಯರ್ಥ-ವಾಗಿವೆ
ವ್ಯರ್ಥ-ವಾ-ಗು-ವುದು
ವ್ಯರ್ಥ-ವಾಗು-ವುದೂ
ವ್ಯರ್ಥ-ವಾದುವು
ವ್ಯರ್ಥ-ವಾ-ಯಿತು
ವ್ಯರ್ಥ-ವೆಂದರೂ
ವ್ಯರ್ಥ-ವೆನ್ನು-ವುದು
ವ್ಯವಸ್ಥಿತ
ವ್ಯವಸ್ಥೆ
ವ್ಯವಸ್ಥೆ-ಗೊಳಿ-ಸುವ
ವ್ಯವ-ಹಾರ
ವ್ಯವ-ಹಾರ-ಕುಶಲಿ
ವ್ಯವ-ಹಾ-ರಕ್ಕೆ
ವ್ಯವ-ಹಾರಕ್ಕೇ
ವ್ಯವ-ಹಾರ-ಗ-ಳನ್ನು
ವ್ಯವ-ಹಾರ-ಗಳಿಗೆ
ವ್ಯವ-ಹಾರ-ಗಳು
ವ್ಯವ-ಹಾರ-ಚತುರ
ವ್ಯವ-ಹಾರ-ಚತುರ-ರಲ್ಲ-ವೆಂದೂ
ವ್ಯವ-ಹಾರ-ಚತುರರು
ವ್ಯವ-ಹಾರ-ಚತುರರೆ
ವ್ಯವ-ಹಾರ-ಚತುರ-ರೆಂಬ
ವ್ಯವ-ಹಾರ-ದಲ್ಲಿ
ವ್ಯವ-ಹಾರ-ದೃಷ್ಟಿಯ
ವ್ಯವ-ಹಾರ-ದೃಷ್ಟಿ-ಯಲ್ಲಿ
ವ್ಯವ-ಹಾರ-ವೆಂದೂ
ವ್ಯವ-ಹಾರವೇ
ವ್ಯವ-ಹಿತಾ-ನಾಮಪ್ಯಾ-ನಂತರ್ಯಂ
ವ್ಯಷ್ಟಿ
ವ್ಯಷ್ಟಿಗೂ
ವ್ಯಸನ
ವ್ಯಸನ-ವಾ-ಗು-ವುದು
ವ್ಯಸ-ನ-ವಿಲ್ಲ
ವ್ಯಾಖ್ಯಾತಾ
ವ್ಯಾಖ್ಯಾತಾಃ
ವ್ಯಾಖ್ಯಾನ
ವ್ಯಾಖ್ಯಾನ-ಗ-ಳನ್ನು
ವ್ಯಾಘ್ರ
ವ್ಯಾಘ್ರ-ಗಳ
ವ್ಯಾಘ್ರ-ಗಳಂತೆ
ವ್ಯಾಘ್ರದ
ವ್ಯಾಘ್ರ-ವನ್ನು
ವ್ಯಾಜ್ಯ-ದಲ್ಲಿ
ವ್ಯಾಧಿ
ವ್ಯಾಧಿತ್ಯಾನ-ಸಂಶಯಪ್ರಮಾದಾಲಸ್ಯಾ-ವಿರ-ತಿಭ್ರಾಂತಿ
ವ್ಯಾಪಾರ-ವೆಲ್ಲ-ವನ್ನು
ವ್ಯಾಪಾರ-ವೊಂದೇ
ವ್ಯಾಪಾರಸ್ಥನಾ-ದರೆ
ವ್ಯಾಪಿ
ವ್ಯಾಪಿ-ತ-ಯುಳ್ಳ
ವ್ಯಾಪಿ-ಯಲ್ಲ
ವ್ಯಾಪಿ-ಯಾಗಿ-ರ-ಬೇಕು
ವ್ಯಾಪಿ-ಯಾದ
ವ್ಯಾಪಿಯೂ
ವ್ಯಾಪಿ-ಸದೇ
ವ್ಯಾಪಿ-ಸ-ಬೇಕು
ವ್ಯಾಪಿ-ಸಿತು
ವ್ಯಾಪಿ-ಸಿ-ರುವ
ವ್ಯಾಪಿ-ಸಿ-ರುವನು
ವ್ಯಾಪಿ-ಸಿಲ್ಲ
ವ್ಯಾಪಿ-ಸು-ವ-ವರೆಗೆ
ವ್ಯಾಪಿ-ಸು-ವುದು
ವ್ಯಾಪ್ತಿ
ವ್ಯಾಮೋಹ-ದಲ್ಲಿ
ವ್ಯಾವ
ವ್ಯಾವ-ಹಾರಿಕ
ವ್ಯಾವ-ಹಾರಿ-ಕಕ್ಕೆ
ವ್ಯಾವ-ಹಾರಿ-ಕ-ವಾದ
ವ್ಯಾವ-ಹಾರಿ-ಕವೇ
ವ್ಯಾಸಂಗ
ವ್ಯಾಸಂಗದ
ವ್ಯಾಸಾರ್ಧ-ರೇಖೆ-ಗಳು
ವ್ಯಾಸಾರ್ಧ-ರೇಖೆ-ಗಳೆಲ್ಲ
ವ್ಯಾಸಾರ್ಧ-ರೇಖೆಯ
ವ್ಯುತ್ಥಾನ-ನಿರೋಧ-ಸಂಸ್ಕಾರ-ಯೋರಭಿ-ಭವಪ್ರಾ-ದುರ್ಭಾವೌ
ವ್ಯುತ್ಥಾನೇ
ವ್ಯೂಹ-ಗಳೆಲ್ಲ
ವ್ಯೆದ್ಯ-ಕೀಯ
ವ್ರತ-ಗಳು
ವ್ರತ-ದಲ್ಲಿ
ವ್ರತ-ನಿಯ-ಮಾದಿ-ಗ-ಳನ್ನೂ
ವ್ರತ-ಭಂಗ-ಮಾಡುತ್ತಿರು-ವು-ದನ್ನು
ವ್ರತ-ವನ್ನು
ವ್ರತಾದಿ-ಗ-ಳನ್ನು
ವ್ರಯ-ವಾಗಿ
ಶಂಕರಾ
ಶಂಕರಾ-ಚಾರ್ಯರ
ಶಂಕರಾ-ಚಾರ್ಯರಂತಹ
ಶಂಕರಾ-ಚಾರ್ಯರು
ಶಕ್ತ-ನಾಗಿದ್ದರೆ
ಶಕ್ತನು
ಶಕ್ತ-ರಾ-ಗಿ-ರು-ವರೋ
ಶಕ್ತ-ರಾ-ದಷ್ಟೂ
ಶಕ್ತಿ
ಶಕ್ತಿ-ಗಳ
ಶಕ್ತಿ-ಗ-ಳನ್ನು
ಶಕ್ತಿ-ಗ-ಳನ್ನೂ
ಶಕ್ತಿ-ಗಳನ್ನೆಲ್ಲ
ಶಕ್ತಿ-ಗಳಲ್ಲಿ
ಶಕ್ತಿ-ಗಳಾಗಿ
ಶಕ್ತಿ-ಗಳಿಗೂ
ಶಕ್ತಿ-ಗಳಿಗೆ
ಶಕ್ತಿ-ಗಳಿವು
ಶಕ್ತಿ-ಗಳಿವೆ
ಶಕ್ತಿ-ಗಳು
ಶಕ್ತಿ-ಗಳೂ
ಶಕ್ತಿ-ಗಳೆ-ರಡೂ
ಶಕ್ತಿ-ಗಳೆಲ್ಲ
ಶಕ್ತಿ-ಗಳೆಲ್ಲಾ
ಶಕ್ತಿ-ಗಳೇನೂ
ಶಕ್ತಿ-ಗಿಂತ
ಶಕ್ತಿ-ಗಿಂತಲೂ
ಶಕ್ತಿ-ಗುಂದು-ವ-ವರೆಗೂ
ಶಕ್ತಿಗೂ
ಶಕ್ತಿಗೆ
ಶಕ್ತಿ-ತ-ರಂಗ-ಗ-ಳನ್ನು
ಶಕ್ತಿ-ತ-ರಂಗ-ವನ್ನು
ಶಕ್ತಿತ್ವ
ಶಕ್ತಿಪ್ರ-ಭಾವ-ವಿದ್ದರೆ
ಶಕ್ತಿಪ್ರವಾಹ-ವನ್ನು
ಶಕ್ತಿಪ್ರವಾ-ಹವೇ
ಶಕ್ತಿ-ಮೀರಿ
ಶಕ್ತಿಯ
ಶಕ್ತಿ-ಯಂತೆ
ಶಕ್ತಿ-ಯನ್ನಾ-ಗಲಿ
ಶಕ್ತಿ-ಯನ್ನಾ-ಗಲೀ
ಶಕ್ತಿ-ಯನ್ನು
ಶಕ್ತಿ-ಯನ್ನೂ
ಶಕ್ತಿ-ಯನ್ನೆಲ್ಲ
ಶಕ್ತಿ-ಯನ್ನೆಲ್ಲಾ
ಶಕ್ತಿ-ಯನ್ನೇ
ಶಕ್ತಿ-ಯನ್ನೇನೋ
ಶಕ್ತಿ-ಯಲ್ಲ
ಶಕ್ತಿ-ಯಲ್ಲಿ
ಶಕ್ತಿ-ಯಲ್ಲೆಲ್ಲ
ಶಕ್ತಿಯಾ
ಶಕ್ತಿ-ಯಾ-ಗಲೀ
ಶಕ್ತಿ-ಯಾಗಿ
ಶಕ್ತಿ-ಯಾ-ಗಿ-ರು-ವು-ದ-ರಿಂದ
ಶಕ್ತಿ-ಯಾ-ಗು-ವುದು
ಶಕ್ತಿ-ಯಿಂದ
ಶಕ್ತಿ-ಯಿದೆ
ಶಕ್ತಿ-ಯಿಲ್ಲದ
ಶಕ್ತಿಯು
ಶಕ್ತಿ-ಯು-ತ-ವಾಗಿ
ಶಕ್ತಿ-ಯು-ತ-ವಾ-ದದ್ದು
ಶಕ್ತಿ-ಯುಳ್ಳದ್ದಾಗಿ
ಶಕ್ತಿಯೂ
ಶಕ್ತಿಯೆ
ಶಕ್ತಿ-ಯೆಂದು
ಶಕ್ತಿ-ಯೆಂದೂ
ಶಕ್ತಿ-ಯೆಂಬ
ಶಕ್ತಿ-ಯೆಲ್ಲ
ಶಕ್ತಿ-ಯೆಲ್ಲ-ವನ್ನೂ
ಶಕ್ತಿಯೇ
ಶಕ್ತಿ-ಯೊಂದಿಗೆ
ಶಕ್ತಿ-ಯೊ-ಡನೆ
ಶಕ್ತಿ-ರೂಪದ
ಶಕ್ತಿ-ರೂಪ-ಧಾರಣೆ
ಶಕ್ತಿ-ರೂಪ-ವಾದ
ಶಕ್ತಿ-ವರ್ಧಕ-ವಾ-ಗಿ-ರುವುದು
ಶಕ್ತಿ-ಸ-ಮೂಲ-ವೆಲ್ಲದರ
ಶಕ್ತಿ-ಸಾ-ಗರ
ಶಕ್ತಿ-ಸಾ-ಗರ-ದೊಂದಿಗೆ
ಶಕ್ತಿ-ಸಾಧ್ಯ-ತೆ-ಯನ್ನು
ಶಕ್ಯ-ವಿದ್ದ-ವ-ರಿಗೆ
ಶತ-ಮಾನದ
ಶತ-ಮಾನ-ದಲ್ಲಿ
ಶತ-ಮಾನ-ದ-ವರು
ಶತಶತ-ಮಾನ-ಗಳು
ಶತಾಬ್ದಿದ
ಶತ್ರು-ವನ್ನು
ಶತ್ರು-ವನ್ನೂ
ಶಪಥ
ಶಪಥ-ಗ-ಳನ್ನು
ಶಪಥ-ವಿ-ರಲಿ
ಶಬ್ದ
ಶಬ್ದಕ್ಕೆ
ಶಬ್ದ-ಗ-ಳನ್ನು
ಶಬ್ದ-ಗಳಿಗೂ
ಶಬ್ದ-ಗಳು
ಶಬ್ದ-ಜಾಲ-ದಿಂದ
ಶಬ್ದ-ಜಾಲ-ವನ್ನು
ಶಬ್ದಜ್ಞಾನ-ದಿಂದ
ಶಬ್ದಜ್ಞಾನಾ-ನು-ಪಾತೀ
ಶಬ್ದ-ಝರೀ
ಶಬ್ದದ
ಶಬ್ದ-ದಲ್ಲಿ
ಶಬ್ದ-ದೊಂದಿಗೆ
ಶಬ್ದ-ಮೂಲ
ಶಬ್ದ-ವನ್ನು
ಶಬ್ದ-ವನ್ನೆ
ಶಬ್ದ-ವಿದೆ
ಶಬ್ದ-ವಿದೆಯೆ
ಶಬ್ದ-ವಿ-ರುತ್ತದೆ
ಶಬ್ದ-ವಿಲ್ಲದೆ
ಶಬ್ದ-ವಿಲ್ಲದೇ
ಶಬ್ದವು
ಶಬ್ದವೂ
ಶಬ್ದ-ವೆಂದರೆ
ಶಬ್ದ-ವೆಂದು
ಶಬ್ದ-ವೆಂಬ
ಶಬ್ದ-ವೆನ್ನು-ವುದು
ಶಬ್ದೋಚ್ಚಾರಣ
ಶಬ್ಧಾದ್ಯನ್ತರ್ಧಾನ-ಮುಕ್ತಮ್
ಶಬ್ಧಾರ್ಥಜ್ಞಾನ-ವಿ-ಕಲ್ಪೈಃ
ಶಬ್ಧಾರ್ಥಪ್ರತ್ಯಯಾ-ನಾ-ಮಿತ-ರೇ-ತರಾಧ್ಯಾಸಾತ್
ಶಮನ-ವಾಗು-ವು-ದಿಲ್ಲ
ಶಮನ-ವಾ-ಗು-ವುದು
ಶಯರೆ
ಶರಣಾ-ಗತ-ನಾಗು-ವು-ದ-ರಿಂದ
ಶರಣಾ-ಗತಿ
ಶರಣಾ-ಗಿ-ರು-ವರೋ
ಶರಣಾ-ನಾಗುತ್ತೇನೆ
ಶರೀರ
ಶರೀ-ರಕ್ಕೆ
ಶರೀರ-ಗಳ
ಶರೀರ-ಗಳಿವೆ
ಶರೀರದ
ಶರೀರ-ದಲ್ಲಿ
ಶರೀರ-ದಲ್ಲಿ-ರುವ
ಶರೀರ-ದಲ್ಲಿವೆ
ಶರೀರ-ದಲ್ಲೆ
ಶರೀರ-ವನ್ನು
ಶರೀರ-ವಿಜ್ಞಾನಿ
ಶರೀರ-ವಿದೆ
ಶರೀರವು
ಶರೀರವೂ
ಶರೀರವೇ
ಶರೀರ-ಶಾಸ್ತ್ರಜ್ಞರು
ಶರೀರ-ಶಾಸ್ತ್ರದ
ಶವಕ್ಕೆ
ಶವ-ದಂತೆ
ಶವ-ವನ್ನು
ಶಸ್ತ್ರಚಿಕಿತ್ಸಕ-ನನ್ನಾಗಿ
ಶಸ್ತ್ರಾಸ್ತ್ರ-ಗಳಲ್ಲದೆ
ಶಾಂತ
ಶಾಂತ-ಗೊಳಿಸಿ
ಶಾಂತ-ಜೀವಿ-ಗಳು
ಶಾಂತ-ನಾಗುತ್ತಾನೆ
ಶಾಂತ-ನಾದ
ಶಾಂತ-ಮನಸ್ಕ-ನಾಗಿ
ಶಾಂತ-ರಾಗು-ವಿರಿ
ಶಾಂತರೋ
ಶಾಂತ-ವಾಗ-ಬೇಕು
ಶಾಂತ-ವಾಗಿ
ಶಾಂತ-ವಾಗಿದ್ದಷ್ಟೂ
ಶಾಂತ-ವಾಗಿ-ರಲಿ
ಶಾಂತ-ವಾ-ಗಿ-ರುವುದು
ಶಾಂತ-ವಾಗು-ವ-ವರೆಗೆ
ಶಾಂತ-ವಾ-ಗು-ವುದು
ಶಾಂತ-ವಾ-ಗು-ವುವು
ಶಾಂತ-ವಾ-ದರೂ
ಶಾಂತ-ವಾ-ದವು
ಶಾಂತ-ವಾ-ದಾಗ
ಶಾಂತ-ವಾ-ದೊ-ಡನೆ
ಶಾಂತವೂ
ಶಾಂತ-ಸಾ-ಗರದ
ಶಾಂತಸ್ವಭಾ-ವ-ದ-ವನು
ಶಾಂತಸ್ವಾ-ಭಾವ-ದ-ವನು
ಶಾಂತಿ
ಶಾಂತಿ-ಗಾಗಿ
ಶಾಂತಿಗೆ
ಶಾಂತಿಯ
ಶಾಂತಿ-ಯನ್ನು
ಶಾಂತಿ-ಯಿಂದ
ಶಾಂತಿಯೂ
ಶಾಂತಿಯೆ
ಶಾಂತಿಯೇ
ಶಾಂತಿ-ಯೊಂದಿಗೆ
ಶಾಕಾ-ಹಾರಿ-ಗಳು
ಶಾಕೆ-ಯಿಂದ
ಶಾಖ
ಶಾಖಕ್ಕೆ
ಶಾಖದ
ಶಾಖ-ವಿದೆ
ಶಾಖ-ವಿಲ್ಲ-ದಂತೆಯೆ
ಶಾಖೆ
ಶಾಖೆ-ಗಳು
ಶಾಖೆಗೆ
ಶಾಖೆಯ
ಶಾಖೆ-ಯ-ವ-ರೆಲ್ಲ
ಶಾಖೋಪ
ಶಾಖೋಪ-ಶಾಖೆ-ಗ-ಳನ್ನು
ಶಾನ್ತೋದಿತಾವ್ಯಪ-ದೇಶ್ಯ-ಧರ್ಮಾನು-ಪಾತೀ
ಶಾನ್ತೋದಿತೌ
ಶಾಪ
ಶಾಪ-ಕೊಟ್ಟು
ಶಾಪ-ವನ್ನು
ಶಾರೀರಿಕ
ಶಾರೀರಿಕ-ವಾ-ದುದು
ಶಾರೀರಿ-ಕವೇ
ಶಾಲಿ
ಶಾಲೆಗೆ
ಶಾಲೆಯ
ಶಾಲೆ-ಯಲ್ಲಿ
ಶಾಶ್ವತ
ಶಾಶ್ವತಧ್ವನಿ
ಶಾಶ್ವತ-ವನ್ನು
ಶಾಶ್ವತ-ವಲ್ಲ
ಶಾಶ್ವತ-ವಾಗಿ
ಶಾಶ್ವತ-ವಾಗಿಯೂ
ಶಾಶ್ವತ-ವಾಗಿರ
ಶಾಶ್ವತ-ವಾಗಿ-ರ-ಲಾರದು
ಶಾಶ್ವತ-ವಾಗಿ-ರು-ವುದು
ಶಾಶ್ವತ-ವಾದ
ಶಾಶ್ವತ-ವಾದು-ದಾ-ವುದೂ
ಶಾಶ್ವತ-ವಾ-ದೊಂದು
ಶಾಸ್ತ್ರ
ಶಾಸ್ತ್ರ-ಕಾರರು
ಶಾಸ್ತ್ರಕ್ಕೂ
ಶಾಸ್ತ್ರಕ್ಕೆ
ಶಾಸ್ತ್ರ-ಗಳ
ಶಾಸ್ತ್ರ-ಗ-ಳನ್ನು
ಶಾಸ್ತ್ರ-ಗಳನ್ನೆಲ್ಲ
ಶಾಸ್ತ್ರ-ಗಳಲ್ಲಿ
ಶಾಸ್ತ್ರ-ಗಳಲ್ಲಿಯೂ
ಶಾಸ್ತ್ರ-ಗಳಿಂದ
ಶಾಸ್ತ್ರ-ಗಳಿಗೆ
ಶಾಸ್ತ್ರ-ಗಳು
ಶಾಸ್ತ್ರ-ಗಳೂ
ಶಾಸ್ತ್ರ-ಗಳೆಲ್ಲ
ಶಾಸ್ತ್ರಗ್ರಂಥ-ಗಳಿ-ಗಿಂತ
ಶಾಸ್ತ್ರಜ್ಞ
ಶಾಸ್ತ್ರಜ್ಞ-ನಾದರೆ
ಶಾಸ್ತ್ರಜ್ಞನು
ಶಾಸ್ತ್ರಜ್ಞರ
ಶಾಸ್ತ್ರಜ್ಞ-ರಾಗು-ವಿರಿ
ಶಾಸ್ತ್ರಜ್ಞರು
ಶಾಸ್ತ್ರದ
ಶಾಸ್ತ್ರ-ದಲ್ಲಿ
ಶಾಸ್ತ್ರ-ದಲ್ಲಿಯೂ
ಶಾಸ್ತ್ರ-ದಲ್ಲಿಲ್ಲ
ಶಾಸ್ತ್ರ-ಪುರಾಣ-ಗಳಿಂದಲೂ
ಶಾಸ್ತ್ರ-ಪ್ರಕಾರ
ಶಾಸ್ತ್ರ-ಪ್ರಮಾಣಕ್ಕೆ
ಶಾಸ್ತ್ರ-ವನ್ನು
ಶಾಸ್ತ್ರ-ವಲ್ಲ
ಶಾಸ್ತ್ರ-ವಿರ-ಬೇಕು
ಶಾಸ್ತ್ರವು
ಶಾಸ್ತ್ರವೂ
ಶಾಸ್ತ್ರವೇ
ಶಾಸ್ತ್ರೀಯ-ವಾಗಿ
ಶಾಸ್ವತ
ಶಿಕ್ಷಕ-ರಲ್ಲಿ
ಶಿಕ್ಷಣ
ಶಿಕ್ಷಣಕ್ರಮ
ಶಿಕ್ಷಿಸು-ವುದಕ್ಕೆ
ಶಿಕ್ಷಿಸು-ವುದಿಲ್ಲ
ಶಿಕ್ಷೆ
ಶಿಕ್ಷೆಗೆ
ಶಿಕ್ಷೆಯ
ಶಿಕ್ಷೆ-ಯನ್ನು
ಶಿಖ-ರಕ್ಕೆ
ಶಿಖ-ರದ
ಶಿಖರ-ದಲ್ಲಿ-ರಲಿ
ಶಿಖರ-ವನ್ನು
ಶಿಥಿಲ-ಗೊಳಿ-ಸು-ವುವು
ಶಿರಸಾ-ವಹಿಸಿ
ಶಿರಸ್ಸಿನ
ಶಿರಸ್ಸು
ಶಿಲುಬೆಗೆ
ಶಿಲುಬೆಯ
ಶಿಲೆ
ಶಿಲೆಯ
ಶಿವ
ಶಿವ-ಮಯ-ವಾಗಿ-ರುವು-ದುಆ
ಶಿವ-ಲಿಂಗದ
ಶಿವೋಹಂ
ಶಿವೋಽಹಂ
ಶಿಶು
ಶಿಶು-ಗಳು
ಶಿಶು-ವಾಗು-ವುದು
ಶಿಶು-ವಿನ
ಶಿಶುವು
ಶಿಷ್ಯ
ಶಿಷ್ಯನ
ಶಿಷ್ಯ-ನನ್ನು
ಶಿಷ್ಯ-ನಾದ
ಶಿಷ್ಯ-ನಿಗೆ
ಶಿಷ್ಯನು
ಶಿಷ್ಯನೂ
ಶಿಷ್ಯ-ರಿಗೆ
ಶಿಷ್ಯರು
ಶಿಸ್ತಿನ
ಶೀಘ್ರ-ವಾಗಿ
ಶೀತ
ಶೀತ-ವೇರುತ್ತ
ಶೀಲ
ಶೀಲಕ್ಕೆ
ಶೀಲದ
ಶೀಲ-ವಂತ-ರಾಗು-ವುದು
ಶೀಲ-ವನ್ನು
ಶೀಲವು
ಶೀಲವೆ
ಶೀಲ-ವೆಂದರೆ
ಶೀಲ-ವೆಂದು
ಶೀಲ-ವೆಂಬ
ಶೀಲ-ವೆಲ್ಲ
ಶೀಲ-ವೆಲ್ಲಾ
ಶೀಲ-ವೇನೂ
ಶುಕ್ಲ-ಪಕ್ಷಕ್ಕೆ
ಶುಚಿ
ಶುಚಿ-ಗಿಂತ
ಶುಚಿ-ಯಾಗಿಟ್ಟಿರು-ವುದೇ
ಶುಚಿ-ಯಿಂದ
ಶುಚಿ-ಯಿಲ್ಲದೆ
ಶುಚಿಯೂ
ಶುದ್ದಿ
ಶುದ್ಧ
ಶುದ್ಧತೆ
ಶುದ್ಧ-ನಾದ
ಶುದ್ಧ-ನಾದರೂ
ಶುದ್ಧನು
ಶುದ್ಧನೂ
ಶುದ್ಧ-ಮಾಡಿ
ಶುದ್ಧ-ಮಾಡಿದ್ದರೆ
ಶುದ್ಧ-ರಾಗಿ
ಶುದ್ಧರು
ಶುದ್ಧ-ವಾಗಿ
ಶುದ್ಧ-ವಾಗಿ-ದೆಯೋ
ಶುದ್ಧ-ವಾಗಿರ-ಬೇಕು
ಶುದ್ಧ-ವಾಗು-ವುದು
ಶುದ್ಧ-ವಾದ
ಶುದ್ಧ-ವಾದುದು
ಶುದ್ಧವೂ
ಶುದ್ಧ-ವೆಂದೂ
ಶುದ್ಧಾತ್ಮನ
ಶುದ್ಧಾತ್ಮನು
ಶುದ್ಧಾ-ವಾದಾಗ
ಶುದ್ಧಿ
ಶುದ್ಧಿ-ಮಾಡು-ವುದು
ಶುದ್ಧಿಯ
ಶುದ್ಧಿ-ಯಾಗು-ವುದು
ಶುದ್ಧಿ-ಯೆಂಬ
ಶುದ್ಧಿಯೇ
ಶುದ್ಧಿ-ಸಾಮ್ಯ-ವಿರು-ವುದ-ರಿಂದ
ಶುದ್ಧೋಪಿ
ಶುಭದ
ಶುಭ-ವನ್ನು
ಶುಭ-ವಾಗಲಿ
ಶುಭ-ವಾಗು-ವುದು
ಶುಭ-ವಾದು-ದನ್ನೆಲ್ಲಾ
ಶುಭ್ರ
ಶುಭ್ರ-ವಾಗು-ವುದು
ಶುಶ್ರೂಷೆಗೆ
ಶುಷ್ಕ
ಶುಷ್ಕ-ರಾಗಿರು-ವರೊ
ಶುಷ್ಕ-ವಾಗಿ
ಶುಷ್ಕ-ವಾದ
ಶುಷ್ಕ-ವೆಂದರೆ
ಶುಷ್ಕ-ವೆನ್ನು-ವರು
ಶೂನ್ಯ
ಶೂನ್ಯಕ್ಕೆ
ಶೂನ್ಯಕ್ಕೇ
ಶೂನ್ಯ-ತೆಯ
ಶೂನ್ಯತೆ-ಯನ್ನು
ಶೂನ್ಯತೆ-ಯಿಂದ
ಶೂನ್ಯ-ದಿಂದ
ಶೂನ್ಯ-ಮಾಡಲು
ಶೂನ್ಯ-ವಾಗಿ
ಶೂನ್ಯ-ವಾದಿ
ಶೂನ್ಯ-ವಾದಿ-ಗಳ
ಶೂನ್ಯ-ವಾದಿ-ಯಾಗ-ಬಲ್ಲವ-ನನ್ನು
ಶೂನ್ಯ-ವೆಂದು
ಶೃಂಖಲೆ-ಗಳಿಂದ
ಶೃಂಖಲೆ-ಯನ್ನು
ಶೇಕಡ
ಶೇಖರ-ವಾಗಿದೆ
ಶೇಖರಿಸ
ಶೇಖರಿಸಲ್ಪಟ್ಟ
ಶೇಖರಿಸಲ್ಪಡು-ವುವು
ಶೇಷವೂ
ಶೈಲಿ-ಯಲ್ಲಿ-ಡಲು
ಶೋಕ-ದೂರ-ವಾದ
ಶೋಚನೀಯ
ಶೋಧನೆ
ಶೋಭಾಯ-ಮಾನ-ವಾದ
ಶೋಷಿಸಿ
ಶೋಷಿಸು-ವುದು
ಶೌಚ
ಶೌಚದ
ಶೌಚ-ದಲ್ಲಿ
ಶೌಚ-ವನ್ನು
ಶೌಚ-ವಿದ್ದರೆ
ಶೌಚ-ಸಂತೋಷ-ತಪಃಸ್ವಾಧ್ಯಾಯೇಶ್ವರಪ್ರಣಿ-ಧಾನಾನಿ
ಶೌಚಾತ್ಸ್ವಾಂಗಜುಗುಪ್ಸಾ
ಶ್ಮಶಾನ-ದಲ್ಲಿ
ಶ್ಯಾಸ್ತ್ರೀ-ಯ-ವಾಗಿ
ಶ್ರದ್ಧಾ
ಶ್ರದ್ಧಾ-ಚಿಕಿತ್ಸೆಗೆ
ಶ್ರದ್ಧಾ-ಪೂರ್ವಕ-ವಾದ
ಶ್ರದ್ಧಾ-ವೀರ್ಯಸ್ಮೃತಿ-ಸಮಾ-ಧಿಪ್ರಜ್ಞಾ-ಪೂರ್ವಕ
ಶ್ರದ್ಧಾ-ವೈದ್ಯರು
ಶ್ರದ್ಧೆ
ಶ್ರದ್ಧೆಯ
ಶ್ರದ್ಧೆ-ಯನ್ನು
ಶ್ರದ್ಧೆ-ಯನ್ನೂ
ಶ್ರದ್ಧೆ-ಯಲ್ಲ
ಶ್ರದ್ಧೆ-ಯಿಂದ
ಶ್ರದ್ಧೆಯೂ
ಶ್ರಮ
ಶ್ರಮದ
ಶ್ರಮ-ದಿಂದ
ಶ್ರಮ-ಸಾಧನೆ-ಯಾದ
ಶ್ರಮಿಸುತ್ತಾರೆ
ಶ್ರಮಿ-ಸು-ವುದು
ಶ್ರವಣ-ವಲ್ಲ
ಶ್ರಿಮಾನ್
ಶ್ರೀ
ಶ್ರೀಕೃಷ್ಣ
ಶ್ರೀಕೃಷ್ಣ-ನನ್ನು
ಶ್ರೀಕೃಷ್ಣನು
ಶ್ರೀಮಂತನ
ಶ್ರೀಮಂತ-ನಂತೆ
ಶ್ರೀಮಂತ-ನಾಗು
ಶ್ರೀಮಂತ-ನಿಂದ
ಶ್ರೀಮಂತರ
ಶ್ರೀಮಂತ-ರಿಗೆ
ಶ್ರೀಮತಿ
ಶ್ರೀರಾಮ-ಕೃಷ್ಣ
ಶ್ರುತಾನು-ಮಾನಪ್ರಜ್ಞಾಭ್ಯಾಮನ್ಯ-ವಿಷಯಾ
ಶ್ರುತಿ
ಶ್ರುತಿ-ಯನ್ನು
ಶ್ರುತಿಯು
ಶ್ರುತಿ-ವಾಕ್ಯದ
ಶ್ರೇಣಿಗೆ
ಶ್ರೇಣಿಯನ್ನೆಲ್ಲಾ
ಶ್ರೇಣಿ-ಯೆಲ್ಲ
ಶ್ರೇಯಸ್ಕರ-ವಾಗಿದೆ-ಆದರೆ
ಶ್ರೇಯಸ್ಸನ್ನು
ಶ್ರೇಯಸ್ಸಿ-ಗಾಗಿ
ಶ್ರೇಯಸ್ಸು
ಶ್ರೇಯೋಭಿ-ವೃದ್ಧಿಗೆ
ಶ್ರೇಷ್ಠ
ಶ್ರೇಷ್ಠ-ಕಾವ್ಯ-ದಂತಿ-ರುವ
ಶ್ರೇಷ್ಠ-ಗುರು-ಗಳೆಲ್ಲಾ
ಶ್ರೇಷ್ಠ-ತಮ
ಶ್ರೇಷ್ಠ-ತಮ-ವಾದುದು
ಶ್ರೇಷ್ಠ-ತರ
ಶ್ರೇಷ್ಠ-ತರ-ವಾದುದು
ಶ್ರೇಷ್ಠ-ಪದ
ಶ್ರೇಷ್ಠ-ಭಾವನೆ
ಶ್ರೇಷ್ಠ-ಭಾವ-ನೆಯೇ
ಶ್ರೇಷ್ಠ-ಲೋಕ
ಶ್ರೇಷ್ಠ-ವಾದ
ಶ್ರೇಷ್ಠ-ವಾದುದು
ಶ್ರೇಷ್ಠ-ವಾದು-ದೆಲ್ಲ
ಶ್ರೇಷ್ಠ-ವಾದುವು-ಗಳೆಲ್ಲ
ಶ್ರೇಷ್ಠ-ವೆನ್ನುವ
ಶ್ರೇಷ್ಠ-ವೆನ್ನು-ವೆವು
ಶ್ರೇಷ್ಠ-ಶೀಲ
ಶ್ರೇಷ್ಠ-ಸತ್ಯ-ಗಳ
ಶ್ರೇಷ್ಠಸ್ಥಿತಿ-ಯಿಂದ
ಶ್ರೋತ್ರಮ್
ಶ್ರೋತ್ರಾ-ಕಾಶಯೋಃ
ಶ್ರೋತ್ರೇಂದ್ರಿಯ-ದಿಂದ
ಶ್ಲಾಘಿಸುತ್ತವೆ
ಶ್ಲೋಕ
ಶ್ಲೋಕ-ವಿದೆ
ಶ್ಲೋಕವು
ಶ್ವಾಸ-ಕೋಶ
ಶ್ವಾಸ-ಕೋಶ-ಗಳ
ಶ್ವಾಸ-ಕೋಶ-ಗಳನ್ನು
ಶ್ವಾಸ-ಕೋಶ-ಗಳಿಗೆ
ಶ್ವಾಸ-ಕೋಶ-ಗಳು
ಶ್ವಾಸ-ಕೋಶ-ವನ್ನು
ಶ್ವಾಸಪ್ರಶ್ವಾಸ-ಯೋರ್ಗತಿ-ವಿಚ್ಛೇದಃ
ಶ್ವಾಸ-ವನ್ನು
ಶ್ವಾಸೋ
ಶ್ವಾಸೋಛ್ವಾಸ-ಗಳ
ಶ್ವಾಸೋಛ್ವಾಸ-ಗಳನ್ನು
ಶ್ವೇತಕೇತು
ಶ್ವೇತಕೇತು-ವಿಗೆ
ಶ್ವೇತಕೇತುವು
ಶ್ವೇತಾಶ್ವ-ತರ
ಷೋಪನೇರನು
ಷೋಪನೇರ್
ಷೋಫನೇರನು
ಷ್ಠಾನಕ್ಕೆ
ಷ್ಯನು
ಸ
ಸಂಕಟ
ಸಂಕಟ-ಗಳ
ಸಂಕಟ-ಗಳಾ-ಗಲಿ
ಸಂಕಟ-ಗಳಿ-ಗೀಡಾಗಿ
ಸಂಕಟ-ಗಳು-ಇವೆಲ್ಲ
ಸಂಕಟ-ಗಳೆಲ್ಲ-ದರಿಂದ
ಸಂಕಟ-ದಲ್ಲಿ
ಸಂಕರ-ವಾಗು-ವುದು
ಸಂಕರಸ್ತತ್ಪ್ರ-ವಿಭಾಗ-ಸಂಯಮಾತ್
ಸಂಕಲ್ಪ
ಸಂಕಷ್ಟ-ಗಳ
ಸಂಕೀರ್ಣಾ
ಸಂಕು-ಚಿತ
ಸಂಕು-ಚಿತ-ಭಾವ
ಸಂಕು-ಚಿತ-ಭಾವನೆ
ಸಂಕು-ಚಿತ-ವಾಗಿದೆ
ಸಂಕು-ಚಿತ-ವಾಗುತ್ತ
ಸಂಕು-ಚಿತ-ವಾಗು-ವುದು
ಸಂಕೇತ
ಸಂಕೇತಕ್ಕೂ
ಸಂಕೇತ-ವನ್ನು
ಸಂಕೇತ-ವಾಗಿ-ದೆಯೋ
ಸಂಕೇತ-ವಾಗಿರ-ಲಿಲ್ಲ
ಸಂಕೇತವು
ಸಂಕೋಚ-ದಿಂದ
ಸಂಕೋಚನ
ಸಂಕೋಚನ-ಗೊಂಡದ್ದು
ಸಂಕೋಚನ-ಗೊಂಡು
ಸಂಕೋಚ-ನದ
ಸಂಕೋಚ-ಮಾಡು-ವುದು
ಸಂಕೋಚ-ವಾಗಿ
ಸಂಕೋಚ-ವಾಗು-ವರು
ಸಂಕೋಚವಾ-ದಂತೆ
ಸಂಕ್ಷಿಪ್ತ-ವಾಗಿ
ಸಂಕ್ಷೇಪ
ಸಂಕ್ಷೇಪ-ವಾಗಿ
ಸಂಕ್ಷೇಪ-ವಾಗಿ-ರು-ವುದು
ಸಂಕ್ಷೇಪಾರ್ಥ-ವಿದು-ಅ-ದನ್ನು
ಸಂಖ್ಯೆ
ಸಂಖ್ಯೆ-ಯನ್ನು
ಸಂಖ್ಯೆ-ಯಲ್ಲಿ
ಸಂಖ್ಯೆ-ಯಲ್ಲಿ-ರು-ವರು
ಸಂಖ್ಯೆಯೇ
ಸಂಗ
ಸಂಗತಿ
ಸಂಗತಿ-ಯನ್ನಷ್ಟೆ
ಸಂಗತಿ-ಯನ್ನು
ಸಂಗತಿಯು
ಸಂಗದ
ಸಂಗದಿಂದ
ಸಂಗಮ
ಸಂಗಮ-ವಾಗ-ಬೇಕು
ಸಂಗಮವು
ಸಂಗವನ್ನು
ಸಂಗವು
ಸಂಗವೇ
ಸಂಗಸ್ಮಯಾ-ಕರಣಂ
ಸಂಗೀತ-ವನ್ನು
ಸಂಗೃಹೀ-ತತ್ವಾದೇಷಾಮ-ಭಾವೇ
ಸಂಗ್ರ-ಹಕ್ಕೆ
ಸಂಗ್ರಹ-ವಾಗಿದೆ
ಸಂಗ್ರಹ-ವಾಗಿರು-ವುದು-ಎಷ್ಟು
ಸಂಗ್ರಹಿಸ-ಲಾರೆವೋ
ಸಂಗ್ರಹಿಸಿ
ಸಂಗ್ರಹಿಸಿ-ಡ-ಬಹುದು
ಸಂಗ್ರಹಿಸಿದ
ಸಂಗ್ರಹಿಸು-ತ್ತವೆ
ಸಂಗ್ರಹಿಸು-ತ್ತೇವೆ
ಸಂಗ್ರಹಿಸುವ
ಸಂಗ್ರಹಿಸು-ವು-ದಕ್ಕೂ
ಸಂಘ
ಸಂಘ-ಟಿತ
ಸಂಘರ್ಷಣೆ
ಸಂಘ-ವನ್ನು
ಸಂಘಾ-ತಕ್ಕೆ
ಸಂಘಾ-ತದ
ಸಂಘಾತ-ದಿಂದ
ಸಂಘಾತ-ವಾದ
ಸಂಘಾ-ತವು
ಸಂಚ-ಕಾರ
ಸಂಚರಿಸ-ಬಲ್ಲದು
ಸಂಚರಿಸ-ಬಾರದು
ಸಂಚರಿಸ-ಲಾರದು
ಸಂಚ-ರಿಸಿ
ಸಂಚ-ರಿಸಿ-ದಂತೆ
ಸಂಚ-ರಿಸಿ-ದರೆ
ಸಂಚರಿ-ಸುತ್ತಿದೆ
ಸಂಚರಿ-ಸುತ್ತಿದ್ದರು
ಸಂಚರಿ-ಸುತ್ತಿದ್ದರೂ
ಸಂಚರಿ-ಸುತ್ತಿದ್ದೆ
ಸಂಚರಿ-ಸುತ್ತಿ-ರ-ಬಾ-ರದು
ಸಂಚರಿ-ಸುತ್ತಿ-ರಲಿ
ಸಂಚರಿ-ಸುತ್ತಿ-ರುವನು
ಸಂಚ-ರಿಸುತ್ತಿ-ರು-ವೆನು
ಸಂಚ-ರಿಸುತ್ತಿ-ರು-ವೆವು
ಸಂಚ-ರಿಸುತ್ತೇನೆ
ಸಂಚರಿ-ಸುವ
ಸಂಚರಿ-ಸು-ವಂತೆ
ಸಂಚರಿ-ಸು-ವುದು
ಸಂಚಾರ
ಸಂಚಾರದ
ಸಂಚಾರ-ಮಾಡಿದ
ಸಂಚಾರ-ವನ್ನು
ಸಂಜೆ
ಸಂಜೆಯ
ಸಂಜ್ಞ-ಯಾಗಿ
ಸಂಜ್ಞ-ಯಾಗಿರ-ಬೇಕು
ಸಂಜ್ಞೆ-ಗಳ
ಸಂಜ್ಞೆ-ಗಳಿಂದ
ಸಂತತಿ-ಯನ್ನು
ಸಂತ-ತಿಯ-ವ-ರಾದು-ದ-ರಿಂದ
ಸಂತ-ತಿಯ-ವರು
ಸಂತ-ನಿಗೆ
ಸಂತ-ರಿಗೂ-ಪಾಪಿ-ಗಳಿಗೂ
ಸಂತೃಪ್ತಿ-ಯಿಂದ
ಸಂತೋಷ
ಸಂತೋಷ-ಇ-ವು-ಗಳೆಲ್ಲಾ
ಸಂತೋಷ-ಗಳಂತೆ
ಸಂತೋಷ-ಗ-ಳನ್ನು
ಸಂತೋಷ-ದಲ್ಲಿ-ರು-ವಂತೆ
ಸಂತೋಷ-ದಾಯಕ
ಸಂತೋಷ-ದಾಯ-ಕ-ವಾಗಿದೆ
ಸಂತೋಷ-ದಾಯ-ಕ-ವಾಗಿ-ರಲಿ
ಸಂತೋಷ-ದಾಯ-ಕ-ವಾ-ಗಿ-ರುವುದೇ
ಸಂತೋಷ-ದಾಯ-ಕ-ವಾದು-ದೆಲ್ಲ
ಸಂತೋಷ-ದಿಂದ
ಸಂತೋಷ-ದಿಂದಿ
ಸಂತೋಷ-ದಿಂದಿ-ರ-ಬೇಕು
ಸಂತೋಷ-ಪಡ-ಬೇಕು
ಸಂತೋಷ-ಪಡು-ವವ-ರನ್ನು
ಸಂತೋಷ-ವನ್ನು
ಸಂತೋಷ-ವನ್ನುಂಟು-ಮಾಡುವ
ಸಂತೋಷ-ವಾಗಿ
ಸಂತೋಷ-ವಾಗಿ-ರ-ಬೇಕಾ-ದರೆ
ಸಂತೋಷ-ವಾಗಿರಿ
ಸಂತೋಷ-ವಾಗಿ-ರುತ್ತಾನೆ
ಸಂತೋಷ-ವಾ-ಗಿ-ರು-ವೆನು
ಸಂತೋಷ-ವಾ-ಗು-ವುದು
ಸಂತೋಷ-ವಾಗು-ವು-ದೆಂದು
ಸಂತೋಷ-ವಾ-ಯಿತು
ಸಂತೋಷವು
ಸಂತೋಷವೆ
ಸಂತೋಷ-ವೆನ್ನು-ವು-ದಿಲ್ಲ
ಸಂತೋಷ-ವೆನ್ನು-ವುದು
ಸಂತೋಷವೇ
ಸಂತೋಷವೋ
ಸಂತೋಷ-ಸಾಧನ-ಗಳು
ಸಂತೋಷಾ-ದನುತ್ತಮಃ
ಸಂತೋಷಿಸ-ಬಲ್ಲರು
ಸಂತೋಷಿ-ಸುತ್ತೇವೆ
ಸಂದರ್ಭ-ಗಳಲ್ಲಿ
ಸಂದರ್ಭ-ಗಳಲ್ಲೂ
ಸಂದರ್ಭ-ಗಳಿಗೂ
ಸಂದರ್ಭ-ಗಳು
ಸಂದರ್ಭ-ದಲ್ಲಿ
ಸಂದರ್ಶಿಸಿ
ಸಂದರ್ಶಿ-ಸಿ-ರು-ವೆವು
ಸಂದರ್ಶಿಸು
ಸಂದರ್ಶಿ-ಸು-ವನು
ಸಂದರ್ಶಿ-ಸು-ವರು
ಸಂದಿಗೊಂದಿ-ಗಳಿಂದ
ಸಂದಿಗೊಂದಿ-ಗಳು
ಸಂದೇಶ
ಸಂದೇಶ-ವನ್ನು
ಸಂದೇಹ
ಸಂದೇಹ-ಗಳು
ಸಂದೇಹ-ವಾದಿ-ಗಳ
ಸಂದೇಹ-ವಿದೆ
ಸಂದೇಹ-ವಿದ್ದರೆ
ಸಂದೇಹ-ವಿಲ್ಲ
ಸಂದೇ-ಹವೇ
ಸಂದೇಹ-ವೇನೂ
ಸಂಧಿಸ-ಬೇಕಾ-ಗಿದೆ
ಸಂಧಿ-ಸಲಿ
ಸಂಧಿಸಿ
ಸಂಧಿಸಿ-ದಾಗ
ಸಂಧಿ-ಸುವ
ಸಂಧಿ-ಸುವರು
ಸಂಧಿ-ಸು-ವರೊ
ಸಂಧಿ-ಸು-ವಲ್ಲಿ
ಸಂಧಿ-ಸು-ವುದು
ಸಂಧಿ-ಸು-ವುವು
ಸಂಧಿಸ್ಥಳ-ದಿಂದ
ಸಂನ್ಯಾಸಿ-ಗಳ
ಸಂನ್ಯಾಸಿ-ಗಳಾ-ಗು-ವರು
ಸಂನ್ಯಾಸಿ-ಗಳಿಗೆ
ಸಂಪತ್ತು
ಸಂಪತ್ತೂ
ಸಂಪರ್ಕ
ಸಂಪರ್ಕಕ್ಕೆ
ಸಂಪರ್ಕ-ದಿಂದ
ಸಂಪರ್ಕ-ವನ್ನು
ಸಂಪರ್ಕ-ವನ್ನೂ
ಸಂಪರ್ಕಿಸಿ-ದರೆ
ಸಂಪಾದನೆ
ಸಂಪಾದಿ-ಸ-ಬಲ್ಲ
ಸಂಪಾದಿ-ಸ-ಬಹುದು
ಸಂಪಾದಿ-ಸ-ಬೇಕು
ಸಂಪಾದಿ-ಸ-ಲೋಸುಗ
ಸಂಪಾದಿ-ಸಿದ
ಸಂಪಾದಿ-ಸಿದ್ದರು
ಸಂಪಾದಿ-ಸುವ
ಸಂಪಾದಿ-ಸು-ವು-ದಕ್ಕೆ
ಸಂಪುಟ
ಸಂಪೂರ್ಣ
ಸಂಪೂರ್ಣ-ಗೊಳಿ-ಸು-ವುದು
ಸಂಪೂರ್ಣ-ವಾಗಿ
ಸಂಪೂರ್ಣ-ವಾದ
ಸಂಪೂರ್ಣಾ
ಸಂಪ್ರ
ಸಂಪ್ರಜ್ಞಾತ
ಸಂಪ್ರಜ್ಞಾತಃ
ಸಂಪ್ರ-ದಾಯ
ಸಂಪ್ರ-ದಾಯ-ಗಳಲ್ಲಿ
ಸಂಪ್ರ-ದಾಯ-ಗಳಿವೆ
ಸಂಪ್ರ-ದಾಯ-ಗಳು
ಸಂಪ್ರ-ದಾಯ-ಗಳೆ
ಸಂಪ್ರ-ದಾಯದ
ಸಂಪ್ರ-ದಾಯ-ದ-ವರು
ಸಂಪ್ರ-ದಾಯವು
ಸಂಪ್ರ-ದಾಯ-ಶೀಲರ
ಸಂಬಂಧ
ಸಂಬಂಧಕ್ಕೆ
ಸಂಬಂಧ-ಗಳಿಂದಲೂ
ಸಂಬಂಧ-ಗಳು
ಸಂಬಂಧದ
ಸಂಬಂಧ-ದಿಂದ
ಸಂಬಂಧ-ದೊಂದಿಗೆ
ಸಂಬಂಧ-ನಾಗಿ
ಸಂಬಂಧ-ಪಟ್ಟ
ಸಂಬಂಧ-ಪಟ್ಟದ್ದು
ಸಂಬಂಧ-ಪಟ್ಟವು
ಸಂಬಂಧ-ಪಟ್ಟಿ
ಸಂಬಂಧ-ಪಟ್ಟಿ-ರು-ವುದು
ಸಂಬಂಧ-ಪಟ್ಟು-ದಲ್ಲ
ಸಂಬಂಧ-ವನ್ನು
ಸಂಬಂಧ-ವನ್ನೂ
ಸಂಬಂಧ-ವಿದೆ
ಸಂಬಂಧ-ವಿದ್ದರೆ
ಸಂಬಂಧ-ವಿ-ರ-ಬೇಕು
ಸಂಬಂಧ-ವಿರು
ಸಂಬಂಧ-ವಿರು-ವುದು
ಸಂಬಂಧ-ವಿಲ್ಲ
ಸಂಬಂಧ-ವಿಲ್ಲದ
ಸಂಬಂಧ-ವಿಲ್ಲದೆ
ಸಂಬಂಧವು
ಸಂಬಂಧವೂ
ಸಂಬಂಧ-ವೆನ್ನುವೆವೊ
ಸಂಬಂಧ-ವೆಲ್ಲ
ಸಂಬಂಧ-ಸಂಯಮಾದ್ದಿವ್ಯಂ
ಸಂಬಂಧ-ಸಂಯಮಾಲ್ಲಘುತೂಲ
ಸಂಬಂಧಿ-ಸದೇ
ಸಂಬಂಧಿ-ಸ-ಬಹುದು
ಸಂಬಂಧಿ-ಸ-ಬೇಕೊ
ಸಂಬಂಧಿಸಿ
ಸಂಬಂಧಿ-ಸಿದ
ಸಂಬಂಧಿ-ಸಿ-ದಂತೆ
ಸಂಬಂಧಿ-ಸಿ-ದಷ್ಟು
ಸಂಬಂಧಿ-ಸಿ-ದುದು
ಸಂಬಂಧಿ-ಸಿ-ದುದೊ
ಸಂಬಂಧಿ-ಸಿದೆ
ಸಂಬಂಧಿ-ಸಿದ್ದಲ್ಲ
ಸಂಬಂಧಿ-ಸಿದ್ದು
ಸಂಬಂಧಿ-ಸಿದ್ದೇ
ಸಂಬಂಧಿ-ಸಿ-ರುವು-ದ-ರಿಂದ
ಸಂಬೋಧಃ
ಸಂಭವ
ಸಂಭವ-ವಿದೆ
ಸಂಭವ-ವಿಲ್ಲ
ಸಂಭೋಗ-ಶಕ್ತಿ-ಯನ್ನೆಲ್ಲ
ಸಂಯಮ
ಸಂಯಮಃ
ಸಂಯಮ-ಗ-ಳನ್ನೂ
ಸಂಯಮದ
ಸಂಯಮ-ದಲ್ಲಿ
ಸಂಯಮ-ಮಾಡಿ-ದರೆ
ಸಂಯಮ-ವನ್ನು
ಸಂಯಮವು
ಸಂಯಮ-ವೆಂದು
ಸಂಯಮಾತ್
ಸಂಯಮಾದ-ಪರಾನ್ತಜ್ಞಾನ-ಮರಿಷ್ಟೇಭ್ಯೋ
ಸಂಯಮಾ-ದಿಂದ್ರಿಯ-ಜಯಃ
ಸಂಯಮಾದ್ವಿವೇಕಜಂ
ಸಂಯುಕ್ತ
ಸಂಯೋ
ಸಂಯೋಗ
ಸಂಯೋಗಃ
ಸಂಯೋ-ಗಕ್ಕೆ
ಸಂಯೋಗ-ಗಳಾಗಿ
ಸಂಯೋಗ-ಗಳಿಗೂ
ಸಂಯೋಗ-ಗಳು
ಸಂಯೋ-ಗದ
ಸಂಯೋಗ-ದಿಂದ
ಸಂಯೋಗ-ದಿಂದಲೂ
ಸಂಯೋಗ-ದಿಂದಾ-ಗು-ವುದು
ಸಂಯೋಗ-ದಿಂದಾದ
ಸಂಯೋಗ-ವನ್ನು
ಸಂಯೋ-ಗ-ವನ್ನೂ
ಸಂಯೋ-ಗ-ವಲ್ಲ
ಸಂಯೋ-ಗ-ವಲ್ಲ-ದಿ-ರು-ವು-ದ-ರಿಂದ
ಸಂಯೋ-ಗ-ವಾಗ-ಲಾರದು
ಸಂಯೋ-ಗ-ವಾಗಿ
ಸಂಯೋ-ಗ-ವಾ-ಗಿ-ರುವುದು
ಸಂಯೋ-ಗ-ವಾಗುವ
ಸಂಯೋಗವು
ಸಂಯೋ-ಗವೂ
ಸಂಯೋ-ಗಾ-ಭಾವೋ
ಸಂಯೋಗೋ
ಸಂಯೋ-ಜನೆ
ಸಂಯೋ-ಜನೆ-ಗಳೆಲ್ಲ
ಸಂರಕ್ಷಕ
ಸಂರಕ್ಷಣಾರ್ಥ-ವಾಗಿ
ಸಂರಕ್ಷಣೆಗೆ
ಸಂರಕ್ಷಿ-ಸ-ಬೇಕು
ಸಂರಕ್ಷಿ-ಸಲು
ಸಂರಕ್ಷಿ-ಸುವು-ದಕ್ಕಾಗಿ
ಸಂವತ್ಸರ
ಸಂವಾದ
ಸಂವೇದನ
ಸಂವೇದನಾ
ಸಂವೇದನೆ
ಸಂವೇದ-ನೆ-ಗಳ
ಸಂವೇದ-ನೆ-ಗ-ಳನ್ನು
ಸಂವೇದ-ನೆ-ಗಳನ್ನೆಲ್ಲ
ಸಂವೇದ-ನೆ-ಗಳಿಗೂ
ಸಂವೇದ-ನೆ-ಗಳಿವೆ
ಸಂವೇದ-ನೆ-ಗಳು
ಸಂವೇದ-ನೆ-ಗಳೂ
ಸಂವೇದ-ನೆ-ಗಳೆಲ್ಲ
ಸಂವೇದ-ನೆಯ
ಸಂವೇದ-ನೆ-ಯನ್ನು
ಸಂವೇದ-ನೆ-ಯಿಂದ
ಸಂವೇದ-ನೆಯು
ಸಂವೇದ-ಮನ್
ಸಂಶಯ
ಸಂಶಯ-ಗಳು
ಸಂಶಯ-ಗಳೆಲ್ಲ
ಸಂಶಯ-ವಿಲ್ಲ
ಸಂಶೋಧ-ಕರು
ಸಂಶೋಧನಾಕ್ರಮ-ವನ್ನು
ಸಂಶೋಧನೆ
ಸಂಶೋಧನೆಯ
ಸಂಶೋಧನೆ-ಯಿಂದ
ಸಂಶೋಧನೆಯು
ಸಂಶೋಧನೆ-ಯೆಲ್ಲ
ಸಂಸರ್ಗಕ್ಕೆ
ಸಂಸರ್ಗ-ಗಳಿಂದ
ಸಂಸಾರ
ಸಂಸಾ-ರಕ್ಕೆ
ಸಂಸಾರದ
ಸಂಸಾರ-ದಲ್ಲಿ
ಸಂಸಾರ-ದಾಚೆ
ಸಂಸಾರ-ದಿಂದ
ಸಂಸಾರ-ವನ್ನು
ಸಂಸಾರ-ಸಾ-ಗರದ
ಸಂಸಾರ-ಸಾ-ಗರ-ದಿಂದ
ಸಂಸ್ಕಾರ
ಸಂಸ್ಕಾರ-ಇವು-ಗಳ
ಸಂಸ್ಕಾ-ರಕ್ಕೆ
ಸಂಸ್ಕಾರ-ಗಳ
ಸಂಸ್ಕಾರ-ಗ-ಳನ್ನು
ಸಂಸ್ಕಾರ-ಗ-ಳನ್ನೂ
ಸಂಸ್ಕಾರ-ಗಳನ್ನೆಲ್ಲ
ಸಂಸ್ಕಾರ-ಗಳಾ
ಸಂಸ್ಕಾರ-ಗಳಾಗಿ
ಸಂಸ್ಕಾರ-ಗಳಾ-ಗುತ್ತದೆ
ಸಂಸ್ಕಾರ-ಗಳಿಂದ
ಸಂಸ್ಕಾರ-ಗಳಿಗೆ
ಸಂಸ್ಕಾರ-ಗಳಿ-ಗೆಲ್ಲ
ಸಂಸ್ಕಾರ-ಗಳು
ಸಂಸ್ಕಾರ-ಗಳೂ
ಸಂಸ್ಕಾರ-ಗಳೆಂದ-ರೇನು
ಸಂಸ್ಕಾರ-ಗಳೆಲ್ಲ
ಸಂಸ್ಕಾರ-ಗಳೆಲ್ಲವೂ
ಸಂಸ್ಕಾರ-ಗಳೇ
ಸಂಸ್ಕಾರದ
ಸಂಸ್ಕಾರ-ದಿಂದ
ಸಂಸ್ಕಾರ-ದೊ-ಡನೆ
ಸಂಸ್ಕಾರ-ವನ್ನು
ಸಂಸ್ಕಾರ-ವನ್ನೂ
ಸಂಸ್ಕಾರ-ವನ್ನೆಲ್ಲಾ
ಸಂಸ್ಕಾರ-ವಾಗಿದೆ
ಸಂಸ್ಕಾರ-ವಾ-ಗಿ-ರುವ
ಸಂಸ್ಕಾರವು
ಸಂಸ್ಕಾರವೂ
ಸಂಸ್ಕಾರ-ಶೇಷೋನ್ಯ
ಸಂಸ್ಕಾರ-ಸಾಕ್ಷಾತ್ಕರ-ಣಾತ್
ಸಂಸ್ಕಾರಾತ್
ಸಂಸ್ಕಾರೇಭ್ಯಃ
ಸಂಸ್ಕಾರೋಽನ್ಯ-ಸಂಸ್ಕಾರಪ್ರತಿ-ಬನ್ಧೀ
ಸಂಸ್ಕೃತ
ಸಂಸ್ಕೃತ-ದಲ್ಲಿ
ಸಂಸ್ಕೃತಿ
ಸಂಸ್ಕೃತಿ-ಇ-ವೆಲ್ಲಾ
ಸಂಸ್ಕೃತಿಗೂ
ಸಂಸ್ಕೃತಿಯು
ಸಂಸ್ಥಾನದ
ಸಂಸ್ಥಾನ-ದಲ್ಲಾ-ದರೋ
ಸಂಸ್ಥಾ-ಬದ್ಧ
ಸಂಸ್ಥೆ
ಸಂಸ್ಥೆ-ಗಳ
ಸಂಸ್ಥೆ-ಗಳಲ್ಲಿ
ಸಂಸ್ಥೆ-ಗಳಿಗೂ
ಸಂಸ್ಥೆ-ಗಳು
ಸಂಸ್ಥೆ-ಗಳೂ
ಸಂಸ್ಥೆಗೆ
ಸಂಸ್ಥೆಯ
ಸಂಹತ್ಯ-ಕಾರಿತ್ವಾತ್
ಸಂಹಿತಾ-ಭಾಗ-ದಲ್ಲಿ
ಸಂಹಿತೆ
ಸಂಹಿತೆ-ಗಳಲ್ಲಿ
ಸಂಹಿತೆ-ಗಳಿಗೆ
ಸಂಹಿತೆ-ಯಲ್ಲಿ
ಸಕಲ
ಸಕಲ-ದ-ರಲ್ಲಿಯೂ
ಸಕಲವೂ
ಸಕಾ-ರಣ-ವಾಗಿ
ಸಕ್ತಿ-ಯನ್ನು
ಸಖ
ಸಖನೆ
ಸಖರಾ-ಗು-ವುವು
ಸಗುಣ
ಸಗುಣಕ್ಕಿಂತ
ಸಗುಣ-ನಾದ
ಸಗುಣನು
ಸಗುಣಬ್ರಹ್ಮ
ಸಗುಣ-ವನ್ನು
ಸಗುಣವೂ
ಸಚೇ-ತನ
ಸಚೇ-ತನ-ವಾದ
ಸಚ್ಚಿ-ದಾನಂದ
ಸಚ್ಚಿ-ದಾನಂದದ
ಸಚ್ಚಿ-ದಾ-ನಂದವು
ಸಚ್ಚಿ-ದಾನಂದಸ್ವ-ರೂಪನು
ಸಚ್ಚಿ-ದಾನಂದಸ್ವ-ರೂಪವೇ
ಸಜ್ಜನ-ನಾಗಿ-ರ-ಬೇಕು
ಸಜ್ಜನ-ನಾ-ಗು-ವುದು
ಸಜ್ಜನನೆ
ಸಡಿಲ
ಸಡಿಲ-ಬಿಡಿ
ಸಡಿಲ-ವಾಗುತ್ತಿವೆ
ಸಡಿಲ-ವಾದ
ಸಡಿಲಾ-ಯಿತು
ಸಡಿಲಿಸಿ
ಸಣ್ಣ
ಸಣ್ಣ-ದಲ್ಲ
ಸಣ್ಣ-ದಾಗ-ಬಲ್ಲ
ಸಣ್ಣ-ದಾದ
ಸಣ್ಣದು
ಸಣ್ಣ-ದೊಂದು
ಸಣ್ಣ-ಸಣ್ಣ
ಸತ
ಸತತ
ಸತತ-ವಾಗಿ
ಸತತವೂ
ಸತಿ
ಸತಿ-ಪತಿ-ಯರ
ಸತಿಯ
ಸತಿ-ಯನ್ನು
ಸತಿ-ಯಲ್ಲಿ
ಸತ್
ಸತ್ಕ-ದರ್ಶನ-ವಾಗಿ-ದೆಯೊ
ಸತ್ಕರ್ಮ
ಸತ್ಕರ್ಮ-ಗ-ಳನ್ನು
ಸತ್ಕರ್ಮ-ಗಳಿಂದ
ಸತ್ಕರ್ಮ-ಗಳು
ಸತ್ಕರ್ಮದ
ಸತ್ಕರ್ಮ-ದಿಂದ
ಸತ್ಕರ್ಮ-ಫಲ
ಸತ್ಕರ್ಮ-ವನ್ನು
ಸತ್ಕರ್ಮವೂ
ಸತ್ಕರ್ಮಿ-ಗಳೊ
ಸತ್ಕಾರ್ಯ-ವನ್ನು
ಸತ್ಚಿತ್ಆ-ನಂದ
ಸತ್ಚಿತ್ಆ-ನಂದವೇ
ಸತ್ತ
ಸತ್ತ-ಮೇಲೆ
ಸತ್ತರೂ
ಸತ್ತರೆ
ಸತ್ತ-ರೇನು
ಸತ್ತಾಗ
ಸತ್ತಾ-ವಾದಿ-ಗಳಿ-ರು-ವರು
ಸತ್ತಿತು
ಸತ್ತು
ಸತ್ತು-ಹೋದ
ಸತ್ತೆ
ಸತ್ತೆಯು
ಸತ್ತ್ವ
ಸತ್ತ್ವ-ಇವು
ಸತ್ತ್ವ-ಕಣ-ಗಳಿಂದ
ಸತ್ತ್ವಕ್ರಿಯೆ-ಗಳೆಲ್ಲ
ಸತ್ತ್ವ-ಗುಣ
ಸತ್ತ್ವ-ಗುಣ-ವನ್ನು
ಸತ್ತ್ವ-ಗುಣವು
ಸತ್ತ್ವ-ಗುಣಾ-ಧಿಕ-ರಾದ
ಸತ್ತ್ವ-ದಿಂದಲೇ
ಸತ್ತ್ವ-ಪುರುಷಯೋಃ
ಸತ್ತ್ವ-ಪುರುಷ-ಯೋರತ್ಯಂತಾಸಂಕೀರ್ಣಯೋಃ
ಸತ್ತ್ವ-ಪುರುಷಾನ್ಯತಾಖ್ಯಾತಿ-ಮಾತ್ರಸ್ಯ
ಸತ್ತ್ವ-ವನ್ನು
ಸತ್ತ್ವವು
ಸತ್ತ್ವವೇ
ಸತ್ತ್ವ-ಶುದ್ಧಿ
ಸತ್ತ್ವ-ಶುದ್ಧಿ-ಯಾ-ಗುತ್ತದೆ
ಸತ್ತ್ವ-ಶುದ್ಧಿ-ಯಾಗು-ವಂತಹ
ಸತ್ತ್ವ-ಶುದ್ಧಿ-ಸೌಮನಸ್ಯೈಕಾಗ್ರ್ಯೇಂದ್ರಿಯ-ಜಯಾತ್ಮ-ದರ್ಶನ-ಯೋಗ್ಯತ್ವಾನಿ
ಸತ್ಪುರಷ-ರಿಗೆ
ಸತ್ಪುರುಷ-ರೆಲ್ಲಾ
ಸತ್ಯ
ಸತ್ಯ-ಈಗ
ಸತ್ಯ-ಕಾಮ
ಸತ್ಯ-ಕಾಮನ
ಸತ್ಯ-ಕಾಮ-ನಿಗೆ
ಸತ್ಯ-ಕಾಮನು
ಸತ್ಯಕ್ಕಿಂತ
ಸತ್ಯಕ್ಕೂ
ಸತ್ಯಕ್ಕೆ
ಸತ್ಯಕ್ಕೇ
ಸತ್ಯ-ಗ-ಳನ್ನು
ಸತ್ಯ-ಗಳನ್ನೆ
ಸತ್ಯ-ಗಳಿಗೆ
ಸತ್ಯ-ಗಳಿವೆ
ಸತ್ಯ-ಗಳಿ-ವೆ-ಯೇನೊ
ಸತ್ಯ-ಗಳು
ಸತ್ಯ-ಗ-ಳೆಂದು
ಸತ್ಯ-ಗಳೇ
ಸತ್ಯ-ಜೀವ-ನದ
ಸತ್ಯ-ಜೀವ-ನವು
ಸತ್ಯಜ್ಯೋತಿ
ಸತ್ಯ-ತೆಯು
ಸತ್ಯದ
ಸತ್ಯ-ದಲ್ಲಿ
ಸತ್ಯ-ದಲ್ಲಿ-ರು-ವೆವು
ಸತ್ಯ-ದಲ್ಲೆ
ಸತ್ಯ-ದಷ್ಟೆ
ಸತ್ಯ-ದಿಂದ
ಸತ್ಯ-ದೆ-ಡೆಗೆ
ಸತ್ಯಪ್ರತಿಷ್ಠಾಯಾಂ
ಸತ್ಯ-ಯುಗದ
ಸತ್ಯ-ರತ್ನ-ಗಳಿವೆ
ಸತ್ಯರ್ಮ-ಗ-ಳನ್ನು
ಸತ್ಯ-ವಚನ-ಗಳಿವು
ಸತ್ಯ-ವನು
ಸತ್ಯ-ವನ್ನರಿತು
ಸತ್ಯ-ವನ್ನರಿ-ಯಲು
ಸತ್ಯ-ವನ್ನರಿ-ಸಲು
ಸತ್ಯ-ವನ್ನು
ಸತ್ಯ-ವನ್ನೆಲ್ಲ
ಸತ್ಯ-ವನ್ನೆಲ್ಲಾ
ಸತ್ಯ-ವನ್ನೇ
ಸತ್ಯ-ವಲ್ಲ
ಸತ್ಯ-ವಲ್ಲದೆ
ಸತ್ಯ-ವಾಗ-ಬಲ್ಲವು
ಸತ್ಯ-ವಾಗ-ಬಾ-ರದು
ಸತ್ಯ-ವಾಗ-ಬೇಕು
ಸತ್ಯ-ವಾಗ-ಲಾರದು
ಸತ್ಯ-ವಾಗಿ
ಸತ್ಯ-ವಾಗಿ-ದೆಯೋ
ಸತ್ಯ-ವಾಗಿದ್ದರೂ
ಸತ್ಯ-ವಾಗಿದ್ದರೆ
ಸತ್ಯ-ವಾಗಿದ್ದು
ಸತ್ಯ-ವಾಗಿಯೂ
ಸತ್ಯ-ವಾಗಿ-ರ-ಬೇಕು
ಸತ್ಯ-ವಾಗಿ-ರ-ಬೇಕೆಂದು
ಸತ್ಯ-ವಾಗಿ-ರುವ
ಸತ್ಯ-ವಾಗಿ-ರು-ವಿರಿ
ಸತ್ಯ-ವಾಗಿ-ರುವು-ದನ್ನು
ಸತ್ಯ-ವಾಗಿ-ರುವುದು
ಸತ್ಯ-ವಾಗಿ-ರು-ವುವೋ
ಸತ್ಯ-ವಾ-ಗು-ವುದು
ಸತ್ಯ-ವಾಗು-ವುದೆ
ಸತ್ಯ-ವಾಣಿ-ಗಳು
ಸತ್ಯ-ವಾದ
ಸತ್ಯ-ವಾ-ದದ್ದು
ಸತ್ಯ-ವಾ-ದರೂ
ಸತ್ಯ-ವಾ-ದರೆ
ಸತ್ಯ-ವಾದು-ದನ್ನು
ಸತ್ಯ-ವಾದು-ದಲ್ಲ
ಸತ್ಯ-ವಾದುದು
ಸತ್ಯ-ವಾದುವು
ಸತ್ಯ-ವಿದೆ
ಸತ್ಯ-ವಿದೆಯೇ
ಸತ್ಯ-ವಿದ್ದರೆ
ಸತ್ಯ-ವಿ-ರ-ಬಹುದು
ಸತ್ಯ-ವಿಲ್ಲ
ಸತ್ಯ-ವಿಲ್ಲದ
ಸತ್ಯವು
ಸತ್ಯವೂ
ಸತ್ಯವೆ
ಸತ್ಯ-ವೆಂದಿಗೂ
ಸತ್ಯ-ವೆಂದು
ಸತ್ಯ-ವೆಂದೂ
ಸತ್ಯ-ವೆಂಬ
ಸತ್ಯ-ವೆನ್ನು-ವುದು
ಸತ್ಯ-ವೆಲ್ಲ
ಸತ್ಯ-ವೆಲ್ಲಾ
ಸತ್ಯವೇ
ಸತ್ಯ-ವೇನು
ಸತ್ಯ-ವೇನೆಂದರೆ
ಸತ್ಯ-ವೇನೊ
ಸತ್ಯ-ವೊಂದು
ಸತ್ಯ-ವೊಂದೇ
ಸತ್ಯವೋ
ಸತ್ಯ-ಶಕ್ತಿ
ಸತ್ಯ-ಶೋಧನೆಯ
ಸತ್ಯ-ಸಂಗತಿ
ಸತ್ಯ-ಸಂಧ-ನಾಗಿ-ರ-ಲಿಲ್ಲ
ಸತ್ಯ-ಸಾಕ್ಷಾತ್ಕಾರ-ವಾಗಿ-ದೆಯೊ
ಸತ್ಯ-ಸಾಧಕ-ರಿಗೂ
ಸತ್ಯ-ಸಾರ-ವನ್ನು
ಸತ್ಯಸ್ಯ-ಸತ್ಯ-ದಂತೆ
ಸತ್ಯಾಂಶ
ಸತ್ಯಾಂಶ-ಗಳಿದ್ದುವು
ಸತ್ಯಾಂಶ-ವನ್ನು
ಸತ್ಯಾಂಶವೇ
ಸತ್ಯಾಂಶ-ವೇ-ನೆಂದರೆ
ಸತ್ಯಾಂಶ-ವೇನೊ
ಸತ್ಯಾ-ನು-ಭವ
ಸತ್ಯಾನ್ವೇಷಣೆ-ಯನ್ನು
ಸತ್ಯಾಭಿಲಾಷೆ-ಯಿಂದ
ಸತ್ಯಾ-ಸತ್ಯ-ಗಳ
ಸತ್ವ-ವುಳ್ಳ-ವು-ಗಳೇ
ಸತ್ವಾನ್ವೇಷಣೆ
ಸತ್ಸಂಗ
ಸತ್ಸಂಗಕ್ಕಿಂತ
ಸತ್ಸಂಗ-ವನ್ನು
ಸತ್ಸಂಪ್ರ-ದಾಯ
ಸದಸ್ಯ
ಸದಾ
ಸದಾ-ಕಾಲ-ದಲ್ಲಿಯೂ
ಸದಾ-ಕಾಲವೂ
ಸದೃಶ-ವಾದ
ಸದೆ
ಸದೆ-ಬಡಿ-ಯುತ್ತಾ
ಸದ್ಗುಣ
ಸದ್ಗುಣ-ಗಳು
ಸದ್ದು
ಸದ್ಯಕ್ಕೆ
ಸದ್ಯದ
ಸದ್ಯ-ದಲ್ಲಿ
ಸಧ್ಯಕ್ಕೆ
ಸಧ್ಯ-ದಲ್ಲಿ
ಸನತ್ಕುಮಾರನ
ಸನತ್ಕುಮಾರನು
ಸನಾ-ತನ
ಸನಾ-ತನ-ತತ್ತ್ವ
ಸನಾ-ತನ-ವಾ-ದುದು
ಸನಾತ-ನವೇ
ಸನ್ನಿ
ಸನ್ನಿ-ವೇಶ
ಸನ್ನಿ-ವೇಶ-ಗ-ಳನ್ನೂ
ಸನ್ನಿ-ವೇಶ-ಗಳಲ್ಲಿ
ಸನ್ನಿ-ವೇಶ-ಗಳಿವೆ
ಸನ್ನಿ-ವೇಶ-ಗಳು
ಸನ್ನಿ-ವೇಶ-ದಲ್ಲಿ
ಸನ್ನಿ-ಹಿತ-ವಾಗುತ್ತಿದೆ
ಸನ್ನಿ-ಹಿತ-ವಾ-ಗು-ವುದು
ಸನ್ನಿ-ಹಿತ-ವಾ-ದಂತೆ
ಸನ್ನಿ-ಹಿತ-ವಾ-ದಾಗ
ಸನ್ಮಾರ್ಗ-ದಲ್ಲಿಯೋ
ಸಪ್ತಧಾ
ಸಪ್ಪೆ
ಸಪ್ರಮಾಣ-ವಾಗಿ
ಸಫಲ-ವಾಗುವ
ಸಫಲ-ವಾದ
ಸಬಲ
ಸಬಲ-ನಾಗಿ-ರ-ಬಹುದು
ಸಬಲ-ರಿಗೆ
ಸಬೀಜಃ
ಸಬೇಕು
ಸಭಾ
ಸಭ್ಯ-ನಾದ
ಸಭ್ಯ-ಮನುಷ್ಯ-ನೊಬ್ಬ-ನನ್ನು
ಸಭ್ಯ-ರಾಗಿ
ಸಮ
ಸಮಂಜಸ-ವಾದ
ಸಮ-ಕಟ್ಟಿ-ನಲ್ಲಿ
ಸಮತೂಕ-ದಿಂದಿ-ರುತ್ತದೆ
ಸಮತ್ವ
ಸಮತ್ವಕ್ಕೆ
ಸಮತ್ವದ
ಸಮತ್ವ-ದಲ್ಲಿ
ಸಮತ್ವ-ದಲ್ಲಿ-ರುತ್ತವೆ
ಸಮತ್ವ-ಬುದ್ಧಿ-ಯುಳ್ಳ-ವರೋ
ಸಮತ್ವ-ವನ್ನು
ಸಮತ್ವವು
ಸಮ-ದೃಷ್ಟಿ-ಯಿಂದ
ಸಮ-ನಾಗಿ
ಸಮ-ನಾಗಿದ್ದರೆ
ಸಮ-ನಾಗಿದ್ದಾಗ
ಸಮ-ನಾಗಿ-ರ-ಬಹುದು
ಸಮ-ನಾ-ಗಿ-ರುವ
ಸಮ-ನಾ-ಗಿ-ರು-ವಂತೆ
ಸಮ-ನಾ-ಗಿ-ರುವನು
ಸಮ-ನಾ-ಗಿ-ರುವು-ದಿಲ್ಲ
ಸಮ-ನಾ-ಗಿ-ರುವುದು
ಸಮ-ನಾ-ಗಿ-ರು-ವುವು
ಸಮ-ನಾ-ಗಿ-ರು-ವೆವೆ
ಸಮ-ನಾದ
ಸಮನ್ವಯ
ಸಮನ್ವಯ-ಗೊಂಡ
ಸಮನ್ವಯ-ಗೊಳಿ-ಸುವ
ಸಮನ್ವಯ-ವನ್ನು
ಸಮಪ್ರಮಾಣ-ದಲ್ಲಿ
ಸಮ-ಬುದ್ಧಿಯ
ಸಮ-ಬುದ್ಧಿ-ಯುಳ್ಳ
ಸಮಯ
ಸಮಯ-ಗಳಲ್ಲಿ
ಸಮಯ-ದಲ್ಲಿ
ಸಮಯ-ವನ್ನು
ಸಮಯ-ವಿದೆ
ಸಮಯ-ವಿ-ರ-ಲಿಲ್ಲ
ಸಮಯ-ವಿಲ್ಲ
ಸಮ-ಯವು
ಸಮರದ
ಸಮರ-ದಲ್ಲಿ
ಸಮರ-ಭೂಮಿ-ಯಂತೆ
ಸಮರಾಂಗಣ-ದಲ್ಲಿ
ಸಮರ್ತಿ-ಸು-ವು-ದಕ್ಕೆ
ಸಮರ್ಥ
ಸಮರ್ಥ-ನಾಗು-ವುದೆ
ಸಮರ್ಥ-ನೆಗೆ
ಸಮರ್ಥ-ನೆ-ಯನ್ನು
ಸಮರ್ಥ-ನೆ-ಯಲ್ಲ
ಸಮರ್ಥರು
ಸಮರ್ಥಿ-ಸಲು
ಸಮರ್ಥಿಸಿ
ಸಮರ್ಥಿ-ಸುತ್ತದೆ
ಸಮರ್ಥಿ-ಸು-ವು-ದಕ್ಕೆ
ಸಮರ್ಪಕ-ವಾಗಿ
ಸಮರ್ಪಕ-ವಾಗಿ-ರ-ಲಿಲ್ಲ
ಸಮರ್ಪಕ-ವಾ-ಗಿಲ್ಲ
ಸಮರ್ಪಕ-ವಾಗಿಲ್ಲ-ದು-ದನ್ನು
ಸಮರ್ಪಕ-ವಾದ
ಸಮರ್ಪಕ-ವಾ-ದಂತೆ
ಸಮ-ವಾಗಿ
ಸಮಶ್ರುತಿ-ಗೊಲಿಸಿ
ಸಮಷ್ಟಿ
ಸಮಷ್ಟಿಗೂ
ಸಮಷ್ಟಿಯ
ಸಮಷ್ಟಿ-ಯನ್ನು
ಸಮಷ್ಟಿ-ಯಾಗಿ-ರ-ಬೇಕು
ಸಮಷ್ಟಿ-ಯಾದ
ಸಮಷ್ಟಿ-ವಿಶ್ವಕ್ಕೂ
ಸಮಸ್ತ
ಸಮಸ್ತ-ದಲ್ಲಿಯೂ
ಸಮಸ್ತ-ವನ್ನೂ
ಸಮಸ್ತವೂ
ಸಮಸ್ಯಾ-ಪರಿ-ಹಾ-ರಕ್ಕೆ
ಸಮಸ್ಯೆ
ಸಮಸ್ಯೆ-ಗಳ
ಸಮಸ್ಯೆ-ಗ-ಳನ್ನು
ಸಮಸ್ಯೆ-ಗಳು
ಸಮಸ್ಯೆಗೆ
ಸಮಸ್ಯೆಯ
ಸಮಸ್ಯೆ-ಯನ್ನು
ಸಮಸ್ಯೆ-ಯನ್ನೆಲ್ಲಾ
ಸಮಸ್ಯೆಯೂ
ಸಮಾ
ಸಮಾ-ಚಾರ-ವನ್ನು
ಸಮಾಜ
ಸಮಾಜಕ್ಕೂ
ಸಮಾಜಕ್ಕೆ
ಸಮಾಜ-ಗಳಿ-ಗಿಂತ
ಸಮಾಜದ
ಸಮಾಜ-ದಲ್ಲಿ
ಸಮಾಜ-ದಲ್ಲಿಯೂ
ಸಮಾಜ-ದ-ಷೆಟೇ
ಸಮಾಜ-ದಿಂದ
ಸಮಾಜ-ರಚನೆ
ಸಮಾಜ-ವನ್ನು
ಸಮಾಜ-ವನ್ನೂ
ಸಮಾಜ-ವಿರುವ
ಸಮಾಜ-ವಿಲ್ಲದೆ
ಸಮಾಜವು
ಸಮಾಜ-ಸುಧಾ-ರಕ-ರಿಂದ
ಸಮಾ-ದಾನ
ಸಮಾ-ಧಾನ
ಸಮಾ-ಧಾನ-ಕರ
ಸಮಾ-ಧಾನ-ಕರ-ವಾಗಿ-ರುತ್ತದೆ
ಸಮಾ-ಧಾನ-ಗ-ಳನ್ನು
ಸಮಾ-ಧಾನ-ದಾಯ-ಕ-ವಾದ
ಸಮಾ-ಧಾನ-ದಿಂದ
ಸಮಾ-ಧಾನ-ವನ್ನು
ಸಮಾ-ಧಾನ-ವಾಗಿ
ಸಮಾ-ಧಾನ-ವಿ-ರು-ವ-ವರು
ಸಮಾ-ಧಾವುಪ-ಸರ್ಗಾ
ಸಮಾಧಿ
ಸಮಾಧಿಃ
ಸಮಾ-ಧಿ-ಗ-ಳನ್ನು
ಸಮಾ-ಧಿ-ಗಳೆಂಬುದು
ಸಮಾ-ಧಿಗೆ
ಸಮಾ-ಧಿ-ಪರಿ-ಣಾಮಃ
ಸಮಾ-ಧಿ-ಭಾವ-ನಾರ್ಥಃ
ಸಮಾ-ಧಿಯ
ಸಮಾ-ಧಿ-ಯನ್ನು
ಸಮಾ-ಧಿ-ಯಲ್ಲಿ
ಸಮಾ-ಧಿ-ಯಲ್ಲಿ-ರು-ವ-ವನು
ಸಮಾ-ಧಿ-ಯಾ-ಗು-ವುದು
ಸಮಾ-ಧಿ-ಯಿಂದ
ಸಮಾ-ಧಿಯು
ಸಮಾ-ಧಿ-ಯುಂಟಾ-ಗುತ್ತದೆ
ಸಮಾ-ಧಿಯೇ
ಸಮಾ-ಧಿ-ಯೊಂದಿಗೆ
ಸಮಾ-ಧಿ-ಸಿದ್ಧಿ-ರೀಶ್ವರಪ್ರಣಿ-ಧಾನಾತ್
ಸಮಾನ
ಸಮಾನ-ಜಯಾತ್
ಸಮಾನತೆ
ಸಮಾನ-ತೆಯ
ಸಮಾನ-ತೆ-ಯನ್ನು
ಸಮಾನದ
ಸಮಾನ-ರಾದ
ಸಮಾನರು
ಸಮಾನ-ವಾಗಿ
ಸಮಾನ-ವಾದ
ಸಮಾನಾಂತರ
ಸಮಾನಾರ್ಥದ
ಸಮಾ-ಪತ್ತಿಃ
ಸಮಾ-ಪತ್ತೇಶ್ಚಾ-ಕಾಶ-ಗಮ-ನಮ್
ಸಮಾಪ್ತ-ವಾ-ಗು-ವುದು
ಸಮಾಪ್ತಿ-ಯನ್ನು
ಸಮಾವಸ್ಥೆ-ಯಾದ
ಸಮೀಪ
ಸಮೀಪಕ್ಕೂ
ಸಮೀಪಕ್ಕೆ
ಸಮೀಪ-ದಲ್ಲಿ
ಸಮೀಪ-ದಲ್ಲಿ-ರುವ
ಸಮೀಪ-ದಲ್ಲಿ-ರುವನು
ಸಮೀಪ-ದಲ್ಲಿ-ರುವುದು
ಸಮೀಪ-ದಲ್ಲೆ
ಸಮೀಪ-ವಾದ-ವರೊಬ್ಬರು
ಸಮೀಪ-ವಾ-ದಷ್ಟೂ
ಸಮೀಪ-ವಿದೆ
ಸಮೀಪಿಸ-ಬೇಕೆಂದು
ಸಮೀಪಿಸ-ಲಾರೆವು
ಸಮೀಪಿಸಿ-ದಂತೆ
ಸಮೀಪಿ-ಸಿದ-ವ-ರಿಗೆ
ಸಮೀಪಿಸಿ-ದಷ್ಟೂ
ಸಮೀಪಿ-ಸಿದೆ
ಸಮೀಪಿಸಿದ್ದರೂ
ಸಮೀಪಿ-ಸಿದ್ದಾರೆ
ಸಮೀಪಿಸಿ-ರು-ವರೊ
ಸಮೀಪಿ-ಸುತ್ತಲೆ
ಸಮೀಪಿ-ಸುತ್ತಿ-ರುವುದು
ಸಮೀಪಿಸುತ್ತಿ-ರು-ವೆವು
ಸಮೀಪಿ-ಸುವ
ಸಮೀಪಿ-ಸುವನು
ಸಮೀಪಿ-ಸುವು-ದೆಂದು
ಸಮುದ್ರ
ಸಮುದ್ರಕ್ಕೆ
ಸಮುದ್ರ-ತೀರ-ದಲ್ಲಿ-ರುವ
ಸಮುದ್ರದ
ಸಮುದ್ರ-ದಿಂದ
ಸಮುದ್ರವೇ
ಸಮೂಲಕ್ಕೆ
ಸಮೂಹಕ್ಕಿಂತ
ಸಮೂ-ಹದ
ಸಮೂಹ-ದಂತೆ
ಸಮೃದ್ಧಿಯಿ-ರುವ
ಸಮ್ಮತ-ವಾಗು-ವಂತಹುದು
ಸಮ್ಮತಿ
ಸಮ್ಮತಿ-ಯನ್ನು
ಸಮ್ಮತಿ-ಯಲ್ಲ
ಸಮ್ಮಿಶ್ರ-ಣ-ದಿಂದ
ಸಮ್ಮಿಶ್ರ-ಣವೆ
ಸಮ್ಮುಖ-ದಲ್ಲಿಯೂ
ಸಮ್ಮುಖವೇ
ಸರಣಿಯ
ಸರಣಿ-ಯಂತೆ
ಸರದಿ
ಸರದಿ-ಗಾಗಿ
ಸರಪಣಿ
ಸರಪಣಿ-ಗಳು
ಸರಪಣಿಗೆ
ಸರಪಣಿ-ಯಲ್ಲಿ-ರುವ
ಸರಪಳಿ
ಸರಪಳಿಗೆ
ಸರಪಳಿಯ
ಸರಪಳಿ-ಯಂತೆ
ಸರಪಳಿ-ಯನ್ನು
ಸರಪಳಿ-ಯನ್ನೆಲ್ಲಾ
ಸರಪಳಿ-ಯಲ್ಲಿ
ಸರಪಳಿ-ಯಲ್ಲಿ-ರುವ
ಸರಪಳಿ-ಯಾ-ಗು-ವುದು
ಸರಪಳಿ-ಯಾಗು-ವು-ದೆಂದು
ಸರಪಳಿಯೇ
ಸರಳ
ಸರಳ-ರೇಖೆ
ಸರಳ-ರೇಖೆ-ಯನ್ನು
ಸರಳ-ರೇಖೆ-ಯಲ್ಲಿ
ಸರಳ-ವಸ್ತು
ಸರಳ-ವಾಗಿದೆ
ಸರಳ-ವಾದ
ಸರಳಾನು-ವಾದ
ಸರಿ
ಸರಿ-ದಂತೆ
ಸರಿಪ್ರಮಾಣ-ದಲ್ಲಿ
ಸರಿ-ಮಾಡಿ-ದರೆ
ಸರಿ-ಯಲ್ಲ
ಸರಿ-ಯಲ್ಲ-ವೆಂದರೆ
ಸರಿ-ಯಲ್ಲ-ವೆನ್ನು-ವರು
ಸರಿ-ಯಾಗಿ
ಸರಿ-ಯಾ-ಗಿದೆ
ಸರಿ-ಯಾಗಿ-ದೆಯೋ
ಸರಿ-ಯಾಗಿದ್ದರೆ
ಸರಿ-ಯಾಗಿಯೂ
ಸರಿ-ಯಾಗಿ-ರ-ಲಿಲ್ಲ
ಸರಿ-ಯಾ-ಗಿಲ್ಲ
ಸರಿ-ಯಾಗು-ವು-ದೆಂದು
ಸರಿ-ಯಾದ
ಸರಿ-ಯಾದು
ಸರಿ-ಯಾದು-ದನ್ನು
ಸರಿ-ಯಾದು-ದಲ್ಲ
ಸರಿ-ಯಿಲ್ಲ-ವೆಂದು
ಸರಿ-ಯುತ್ತಿ-ರುವ
ಸರಿ-ಯುತ್ತಿ-ರುವುದು
ಸರಿ-ಯು-ವುದು
ಸರಿಯೆ
ಸರಿ-ಯೆಂದು
ಸರಿಯೋ
ಸರಿ-ಸಮ
ಸರಿ-ಸಮ-ನಾಗಿ
ಸರಿ-ಸಮ-ನಾಗಿ-ರು-ವು-ದ-ರಿಂದ
ಸರಿ-ಸಮ-ನಾದ
ಸರಿ-ಸಮ-ವಾಗಿ
ಸರಿಸು
ಸರಿ-ಸು-ವರು
ಸರಿ-ಹೊಂದಿ-ಸುವ
ಸರಿ-ಹೋಗದ
ಸರಿ-ಹೋಗ-ದೆಂದು
ಸರಿ-ಹೋ-ಗುತ್ತದೆ
ಸರಿ-ಹೋಗು-ವಂತಹ
ಸರಿ-ಹೋಗು-ವು-ದಿಲ್ಲ
ಸರಿ-ಹೋ-ಗು-ವುವು
ಸರಿ-ಹೋಲು-ವುದು
ಸರೂ
ಸರೋ-ವರ
ಸರೋವ-ರಕ್ಕೆ
ಸರೋ-ವರದ
ಸರೋ-ವರ-ದಂತೆ
ಸರೋ-ವರ-ದಲ್ಲಿ
ಸರೋ-ವರ-ದಿಂದ
ಸರೋ-ವರ-ವನ್ನೂ
ಸರೋ-ವರ-ವಲ್ಲ
ಸರೋ-ವರ-ವಿದೆ
ಸರ್
ಸರ್ಕಾರ
ಸರ್ಪ
ಸರ್ವ
ಸರ್ವಂ
ಸರ್ವ-ಕಾಲ-ದಲ್ಲಿಯೂ
ಸರ್ವಜ್ಞ
ಸರ್ವಜ್ಞತೆ
ಸರ್ವಜ್ಞ-ತೆ-ಯನ್ನು
ಸರ್ವಜ್ಞ-ತೆಯು
ಸರ್ವಜ್ಞತ್ವ
ಸರ್ವಜ್ಞತ್ವ-ಗಳು
ಸರ್ವಜ್ಞ-ನಾದ
ಸರ್ವಜ್ಞನು
ಸರ್ವಜ್ಞ-ಬೀಜಮ್
ಸರ್ವಜ್ಞ-ರಾಗಿ-ರ-ಬೇಕು
ಸರ್ವಜ್ಞರು
ಸರ್ವಜ್ಞ-ವೆಂದೂ
ಸರ್ವಜ್ಞಾತೃತ್ವಂ
ಸರ್ವತ್ರ
ಸರ್ವಥಾ
ಸರ್ವ-ಥಾ-ವಿಷ-ಯ-ಮಕ್ರಮಂ
ಸರ್ವದ
ಸರ್ವ-ದಲ್ಲಿಯೂ
ಸರ್ವದಾ
ಸರ್ವ-ನಾಶ-ವಾ-ಯಿತು
ಸರ್ವ-ನಾಶ-ವಾ-ಯಿತು-ಲಕ್ಷಾಂತರ
ಸರ್ವ-ನಿರೋ-ಧಾನ್ನಿರ್ಬೀಜಃ
ಸರ್ವ-ಭಾವಾಧಿಷ್ಠಾತೃತ್ವಂ
ಸರ್ವ-ಭೂತ-ರುತಜ್ಞಾನಮ್
ಸರ್ವಮ್
ಸರ್ವರ
ಸರ್ವ-ರತ್ನೋಪಸ್ಥಾನಮ್
ಸರ್ವ-ರಿಗೂ
ಸರ್ವ-ವನ್ನು
ಸರ್ವ-ವಸ್ತು-ಗಳ
ಸರ್ವ-ವಸ್ತು-ಗಳೂ
ಸರ್ವ-ವಿಷಯಂ
ಸರ್ವ-ವಿಷ-ಯ-ಗಳ
ಸರ್ವವೂ
ಸರ್ವವ್ಯಾಪಿ
ಸರ್ವವ್ಯಾಪಿ-ಗಳು
ಸರ್ವವ್ಯಾಪಿತ್ವ
ಸರ್ವವ್ಯಾಪಿತ್ವ-ವನ್ನು
ಸರ್ವವ್ಯಾಪಿಯ
ಸರ್ವವ್ಯಾಪಿ-ಯಲ್ಲ
ಸರ್ವವ್ಯಾಪಿ-ಯಾಗಿ
ಸರ್ವವ್ಯಾಪಿ-ಯಾಗಿದ್ದರೆ
ಸರ್ವವ್ಯಾಪಿ-ಯಾಗಿ-ರ-ಬೇಕು
ಸರ್ವವ್ಯಾಪಿ-ಯಾಗಿ-ರುವ
ಸರ್ವವ್ಯಾಪಿ-ಯಾಗಿ-ರುವುದು
ಸರ್ವವ್ಯಾಪಿ-ಯಾಗಿ-ರುವು-ದು-ಇದು
ಸರ್ವವ್ಯಾಪಿ-ಯಾದ
ಸರ್ವವ್ಯಾಪಿ-ಯಾ-ದುದು
ಸರ್ವವ್ಯಾಪಿ-ಯಾದು-ದೆಂದು
ಸರ್ವವ್ಯಾಪಿಯು
ಸರ್ವವ್ಯಾಪಿಯೂ
ಸರ್ವವ್ಯಾಪಿ-ಯೆಂದು
ಸರ್ವವ್ಯಾಪ್ತಿ-ಯಾದ
ಸರ್ವ-ಶಕ್ತ
ಸರ್ವ-ಶಕ್ತನ
ಸರ್ವ-ಶಕ್ತ-ನನ್ನು
ಸರ್ವ-ಶಕ್ತ-ನಾಗಿ
ಸರ್ವ-ಶಕ್ತ-ನಾದ
ಸರ್ವ-ಶಕ್ತನು
ಸರ್ವ-ಶಕ್ತನೆ
ಸರ್ವ-ಶಕ್ತ-ರಾಗಿ
ಸರ್ವ-ಶಕ್ತರು
ಸರ್ವ-ಶಕ್ತ-ವಾ-ದರೆ
ಸರ್ವ-ಶಕ್ತವೂ
ಸರ್ವ-ಶಕ್ತಿ
ಸರ್ವ-ಶಕ್ತಿತ್ವ
ಸರ್ವ-ಶಕ್ತಿತ್ವ-ವನ್ನು
ಸರ್ವ-ಶಕ್ತಿತ್ವವೂ
ಸರ್ವ-ಶಕ್ತಿ-ಮಾನ್
ಸರ್ವ-ಶಕ್ತಿ-ಯುಳ್ಳದ್ದು
ಸರ್ವ-ಶಕ್ತಿಸ್ವ-ರೂಪ-ನಾದ
ಸರ್ವಶ್ರೇಷ್ಠ
ಸರ್ವಶ್ರೇಷ್ಠ-ನಾದ
ಸರ್ವ-ಸಾ-ಧಾರಣ
ಸರ್ವ-ಸಾ-ಧಾರ-ಣ-ವಾ-ದುದು
ಸರ್ವ-ಸಾ-ಮಾನ್ಯ
ಸರ್ವ-ಸಾ-ಮಾನ್ಯಕ್ಕೆ
ಸರ್ವ-ಸಾ-ಮಾನ್ಯ-ತೆಯು
ಸರ್ವ-ಸಾ-ಮಾನ್ಯ-ವಾದ
ಸರ್ವ-ಸಾ-ಮಾನ್ಯ-ವಾದುದು
ಸರ್ವಸ್ವ
ಸರ್ವಸ್ವಲ್ಲ
ಸರ್ವಸ್ವ-ವನ್ನಾಗಿ
ಸರ್ವಸ್ವ-ವಾ-ಯಿತು
ಸರ್ವಸ್ವ-ವೆಂದು
ಸರ್ವಾಂತ-ರಾಳ-ದಲ್ಲಿಯೂ
ಸರ್ವಾಂತರ್
ಸರ್ವಾಂತರ್ಯಾಮಿ-ಯೆಂದು
ಸರ್ವಾರ್ಥ-ತೈಕಾಗ್ರತಯೋಃಕ್ಷಯೋದಯೌ-ಚಿತ್ತಸ್ಯ
ಸರ್ವಾರ್ಥಮ್
ಸರ್ವಾ-ವರ-ಣ-ಮಲಾಪೇ-ತಸ್ಯ
ಸರ್ವೇಶ್ವರ
ಸರ್ವೇಶ್ವರ-ನಾದ
ಸರ್ವೇಶ್ವರನು
ಸರ್ವೋತ್ಕೃಷ್ಟ-ವಾದ
ಸರ್ವೋತ್ತಮ-ವಾದ
ಸರ್ವೋತ್ತಮ-ವಾದುದೆ
ಸಲ
ಸಲ-ಕರ-ಣೆ-ಯನ್ನು
ಸಲಕ್ಕೆ
ಸಲ-ವಾ-ದರೂ
ಸಲವೂ
ಸಲ-ಹೆ-ಗಳ
ಸಲ-ಹೆ-ಗ-ಳನ್ನು
ಸಲ-ಹೆ-ಗಳಿಂದ
ಸಲ-ಹೆ-ಗಳಿಗೂ
ಸಲ-ಹೆ-ಗಳು
ಸಲ-ಹೆ-ಗಳೆಲ್ಲ
ಸಲ-ಹೆ-ಯಂತೆ
ಸಲ-ಹೆ-ಯನ್ನು
ಸಲ-ಹೆಯೇ
ಸಲು
ಸಲು-ವಾಗಿ
ಸಲೂ
ಸಲ್ಪಟ್ಟಿ-ರು-ವುದು
ಸಲ್ಲದು
ಸಲ್ಲಿ-ಸು-ವು-ದಿಲ್ಲ
ಸಲ್ಲಿಸು-ವುದೇ
ಸವರಿ
ಸವರಿ-ದಂತೆ
ಸವಾಲನ್ನೇ
ಸವಾಲು
ಸವಿ
ಸವಿ-ಚಾರ
ಸವಿ-ಚಾರಾ
ಸವಿ-ತರ್ಕ
ಸವಿ-ತರ್ಕ-ದಲ್ಲಿ
ಸವಿ-ತರ್ಕ-ವೆನ್ನುತ್ತಾನೆ
ಸವಿ-ತರ್ಕಾ
ಸವೆ-ಯಿಸಿ
ಸವೆಯಿ-ಸುತ್ತಿದ್ದರೆ
ಸವೆಯು-ವ-ವರೆಗೂ
ಸವೆಯು-ವ-ವರೆಗೆ
ಸವೆಸಿ
ಸಸಿ
ಸಸಿ-ಗಳು
ಸಸಿಯ
ಸಸಿ-ಯಾಗುವ
ಸಸಿ-ಯಾಗು-ವು-ದನ್ನು
ಸಸಿಯು
ಸಸ್ಯ
ಸಸ್ಯ-ಗಳ
ಸಸ್ಯ-ಗಳಾ-ಗಿವೆ
ಸಸ್ಯ-ಗಳು
ಸಸ್ಯ-ಜಾ-ತಿಯ
ಸಸ್ಯದ
ಸಸ್ಯ-ದಲ್ಲಿ-ರ-ಬಹು-ದು-ಇದು
ಸಸ್ಯಪ್ರಪಂಚ-ದಲ್ಲಿ-ರುವ
ಸಸ್ಯ-ರಾಶಿ-ಗಳು-ಇಡೀ
ಸಸ್ಯ-ವನ್ನು
ಸಸ್ಯ-ವರ್ಗ
ಸಸ್ಯ-ವರ್ಗಕ್ಕೆ
ಸಸ್ಯ-ವಾ-ಗು-ವುದು
ಸಸ್ಯವು
ಸಸ್ಯಾ-ಹಾರಿ-ಯಾಗದೇ
ಸಹ
ಸಹ-ಕರಿ-ಸಲಿ
ಸಹ-ಕಾರದ
ಸಹ-ಕಾರ-ವನ್ನು
ಸಹ-ಕಾರಿ
ಸಹ-ಕಾರಿ-ಯಾಗ-ಬೇಕಾ-ದರೆ
ಸಹ-ಕಾರಿ-ಯಾ-ಗು-ವುದು
ಸಹಜ
ಸಹಜ-ಗುಣ-ವಾಗು-ವಂತೆ
ಸಹಜ-ವಾಗಿಯೇ
ಸಹಜ-ವಾ-ಗಿ-ರುವ
ಸಹಜ-ವಾದ
ಸಹಜ-ವಾ-ದರೂ
ಸಹಜ-ವಾದುದು
ಸಹಜಸ್ವಭಾ-ವ-ದೊಂದಿಗೆ
ಸಹಜಸ್ವ-ಭಾವ-ವನ್ನು
ಸಹಜಾ-ವಸ್ಥೆ-ಯಲ್ಲಿ
ಸಹನಾ-ಗುಣ
ಸಹನೆ
ಸಹನೆ-ಯಿಂದ
ಸಹ-ವಾಸ-ಗಳನ್ನೆಲ್ಲ
ಸಹಸ್ರ
ಸಹಸ್ರ-ದಳ
ಸಹಸ್ರ-ದಳದ
ಸಹಸ್ರ-ಪಾಲು
ಸಹಸ್ರಾರ
ಸಹಸ್ರಾ-ರಕ್ಕೆ
ಸಹಸ್ರಾರು
ಸಹಾನು-ಭೂತಿ
ಸಹಾನು-ಭೂತಿ-ಯನ್ನು
ಸಹಾನು-ಭೂತಿ-ಯಿಂದ
ಸಹಾಯ
ಸಹಾಯಕ
ಸಹಾಯ-ಕ-ಳಾಗ-ಬಹು-ದೆಂದು
ಸಹಾಯ-ಕ-ವಾಗ-ಬಲ್ಲದು
ಸಹಾಯ-ಕ-ವಾ-ಗಿದ್ದವು
ಸಹಾಯ-ಕ-ವಾಗಿ-ರ-ಬೇಕು
ಸಹಾಯ-ಕ-ವಾ-ಗಿ-ರುವುದು
ಸಹಾಯ-ಕ-ವಾ-ಗಿ-ರು-ವುವು
ಸಹಾಯ-ಕ-ವಾ-ಗು-ವುದು
ಸಹಾಯಕ್ಕಲ್ಲ
ಸಹಾಯಕ್ಕಾಗಿ
ಸಹಾ-ಯಕ್ಕೆ
ಸಹಾಯಕ್ಕೋಸುಗ-ವಾಗಿ
ಸಹಾಯ-ಗ-ಳನ್ನು
ಸಹಾ-ಯದ
ಸಹಾಯ-ದಿಂದ
ಸಹಾಯ-ಮಾಡ-ದಿ-ರು-ವುದು
ಸಹಾಯ-ಮಾ-ಡಲು
ಸಹಾಯ-ಮಾಡ-ಲೂ-ಬಹುದು
ಸಹಾಯ-ಮಾಡಿ
ಸಹಾಯ-ಮಾಡಿದೆ
ಸಹಾಯ-ಮಾಡಿ-ದೆವು
ಸಹಾಯ-ಮಾಡುವ
ಸಹಾಯ-ಮಾಡು-ವುದು
ಸಹಾಯ-ಮಾಡು-ವುದೊ
ಸಹಾಯ-ವನ್ನು
ಸಹಾಯ-ವಾಗ-ಲಾರದು
ಸಹಾಯ-ವಾಗು
ಸಹಾಯ-ವಾಗುವ
ಸಹಾಯ-ವಾದೀ-ತೆಂದು
ಸಹಾಯ-ವಾ-ಯಿತು
ಸಹಾಯ-ವಿಲ್ಲದೆ
ಸಹಾ-ಯವು
ಸಹಾ-ಯವೂ
ಸಹಾಯ-ವೆಲ್ಲ
ಸಹಾಯ-ವೆಲ್ಲವೂ
ಸಹಾ-ಯವೇ
ಸಹಿತ
ಸಹಿತ-ವಾಗಿ
ಸಹಿಷ್ಣುತೆ
ಸಹಿಸ
ಸಹಿ-ಸದೆ
ಸಹಿ-ಸ-ಬೇಕು
ಸಹಿಸ-ಬೇಕೆಂದೂ
ಸಹಿಸ-ಲಾರ
ಸಹಿಸ-ಲಾರದ
ಸಹಿಸ-ಲಾರರು
ಸಹಿ-ಸಲು
ಸಹಿಸಿ
ಸಹಿಸು
ಸಹಿಸು-ವರೊ
ಸಹಿ-ಸು-ವು-ದಕ್ಕೆ
ಸಹೋ
ಸಹೋ-ದರ
ಸಹೋ-ದರತ್ವದ
ಸಹೋ-ದರತ್ವ-ವನ್ನು
ಸಹೋ-ದರ-ನನ್ನು
ಸಹೋ-ದರ-ನನ್ನೇ
ಸಹೋ-ದರನೆ
ಸಹೋ-ದರನೇ
ಸಹೋ-ದರ-ಭಾ-ವನೆ
ಸಹೋ-ದರ-ರಲ್ಲಿ
ಸಹೋ-ದರ-ರಿಲ್ಲ
ಸಹೋ-ದರರು
ಸಹೋ-ದರ-ರೆಂಬ
ಸಹೋ-ದರಿ-ಯರು
ಸಹೋದ್ಯೋಗಿ-ಗಳಿಗೆ
ಸಾಂಕೇತಿಕ
ಸಾಂಕೇತಿಕ-ವಾಗಿ
ಸಾಂಕ್ರಾಮಿಕ
ಸಾಂಕ್ರಾಮಿಕ-ವಾ-ದುದು
ಸಾಂಖ್ಯ
ಸಾಂಖ್ಯ-ತತ್ತ್ವ
ಸಾಂಖ್ಯ-ತತ್ತ್ವದ
ಸಾಂಖ್ಯ-ತತ್ತ್ವ-ದಲ್ಲಿ
ಸಾಂಖ್ಯ-ತತ್ತ್ವ-ವನ್ನು
ಸಾಂಖ್ಯ-ದರ್ಶನ-ದಲ್ಲಿ
ಸಾಂಖ್ಯ-ದರ್ಶನ-ವನ್ನು
ಸಾಂಖ್ಯ-ಯೋಗವು
ಸಾಂಖ್ಯರ
ಸಾಂಖ್ಯ-ರಲ್ಲಿ
ಸಾಂಖ್ಯರು
ಸಾಂಖ್ಯ-ಸಿದ್ಧಾಂತದ
ಸಾಂತ
ಸಾಂತ-ಅವು
ಸಾಂತದ
ಸಾಂತ-ದಲ್ಲಿ
ಸಾಂತ-ದಲ್ಲಿ-ಡಲು
ಸಾಂತ-ದಿಂದ
ಸಾಂತ-ವಸ್ತು-ಗಳಿಂದ
ಸಾಂತ-ವಾಗ
ಸಾಂತ-ವಾಗಿ
ಸಾಂತ-ವಾಗಿದೆ
ಸಾಂತ-ವಾಗಿ-ರ-ಲಾರದು
ಸಾಂತ-ವಾಗಿ-ರಲೇ-ಬೇಕು
ಸಾಂತ-ವಾಗಿ-ರುವ
ಸಾಂತ-ವಾಗಿ-ರುವನು
ಸಾಂತ-ವಾಗಿ-ರುವುದು
ಸಾಂತ-ವಾ-ಗು-ವುದು
ಸಾಂತ-ವಾಗು-ವೆವು
ಸಾಂತ-ವಾದ
ಸಾಂತ-ವಾ-ದದ್ದು
ಸಾಂತ-ವಾದು
ಸಾಂತ-ವಾದುದು
ಸಾಂತವು
ಸಾಂತ-ವೆಂದೇ
ಸಾಂತವೋ
ಸಾಂದ್ರ-ತೆಯು
ಸಾಂದ್ರ-ವಾಗಿದೆ
ಸಾಂಪ್ರ-ದಾಯಿಕ
ಸಾಂಬಂಧಿಕ
ಸಾಕಷ್ಟು
ಸಾಕಾ-ಗಿದೆ
ಸಾಕಾಗು
ಸಾಕಾಗು-ವಷ್ಟು
ಸಾಕಾ-ದಷ್ಟು
ಸಾಕಾರ
ಸಾಕಾರ-ಗಳನ್ನು
ಸಾಕಾರ-ಗಳೆಲ್ಲ
ಸಾಕಾರ-ದಂತೆಯೊ
ಸಾಕಾರ-ದೇವ-ನೊಬ್ಬ-ನಿರು-ವನು
ಸಾಕಾರ-ದೇವ-ರನ್ನು
ಸಾಕಾರ-ದೇವ-ರಿಂದ
ಸಾಕಾರ-ದೇವರಿ-ಗಿಂತ
ಸಾಕಾರ-ದೇವರು
ಸಾಕಾರನು
ಸಾಕಾರ-ವನ್ನು
ಸಾಕಾರವೂ
ಸಾಕಾರವ್ಯಕ್ತಿ
ಸಾಕು
ಸಾಕು-ಇದು
ಸಾಕೋ
ಸಾಕ್ಷಾ
ಸಾಕ್ಷಾತ್
ಸಾಕ್ಷಾತ್ಕಾರ
ಸಾಕ್ಷಾತ್ಕಾರಕ್ಕೂ
ಸಾಕ್ಷಾತ್ಕಾ-ರಕ್ಕೆ
ಸಾಕ್ಷಾತ್ಕಾರದ
ಸಾಕ್ಷಾತ್ಕಾರ-ದಿಂದ
ಸಾಕ್ಷಾತ್ಕಾರ-ಮಾಡಿ-ಕೊಳ್ಳಿ
ಸಾಕ್ಷಾತ್ಕಾರ-ಮಾಡಿ-ಕೊಳ್ಳು-ವರು
ಸಾಕ್ಷಾತ್ಕಾರ-ವನ್ನು
ಸಾಕ್ಷಾತ್ಕಾರ-ವಾಗಿ-ದೆಯೋ
ಸಾಕ್ಷಾತ್ಕಾರ-ವಾದ
ಸಾಕ್ಷಾತ್ಕಾರ-ವಾ-ದಾಗ
ಸಾಕ್ಷಾತ್ಕಾರ-ವಾದುವು
ಸಾಕ್ಷಾತ್ಕಾರ-ವಿಲ್ಲದೆ
ಸಾಕ್ಷಾತ್ಕಾರವೇ
ಸಾಕ್ಷಿ
ಸಾಕ್ಷಿ-ಯಾಗ
ಸಾಕ್ಷಿ-ಯಾಗಿ
ಸಾಗ-ಬಹುದು
ಸಾಗರ
ಸಾಗರಕ್ಕೂ
ಸಾಗ-ರಕ್ಕೆ
ಸಾಗರದ
ಸಾಗರ-ದಂತೆ
ಸಾಗರ-ದಲ್ಲಿ
ಸಾಗರ-ದಲ್ಲಿದ್ದಾಗ
ಸಾಗರ-ದಿಂದ
ಸಾಗರ-ದೆಡೆ
ಸಾಗರ-ದೆ-ಡೆಗೆ
ಸಾಗರ-ದೊಂದಿಗೆ
ಸಾಗರ-ವನ್ನು
ಸಾಗರ-ವನ್ನೆ
ಸಾಗರ-ವಾ-ಗು-ವುದು
ಸಾಗರ-ವಿದೆ
ಸಾಗರವೂ
ಸಾಗರವೇ
ಸಾಗರಾಭಿ-ಮುಖ-ವಾಗಿ
ಸಾಗಿ
ಸಾಗಿ-ಹೋಗ
ಸಾಗಿ-ಹೋಗ-ಬೇಕಾ-ಗಿದೆ
ಸಾಗಿ-ಹೋಗ-ಬೇಕಾದ
ಸಾಗುತ್ತಿದೆ
ಸಾಗುತ್ತಿರು
ಸಾಗುತ್ತಿರುವ
ಸಾಗುತ್ತಿರು-ವು-ದನ್ನು
ಸಾಗುತ್ತಿರು-ವುದು
ಸಾಗುತ್ತಿವೆ
ಸಾಗು-ವನು
ಸಾಗುವು-ದಕ್ಕಾಗಿ
ಸಾಗು-ವುದು
ಸಾತ್ತ್ವಿಕ
ಸಾತ್ತ್ವಿಕ-ಗುಣ
ಸಾತ್ತ್ವಿಕ-ತೆಯ
ಸಾತ್ತ್ವಿಕ-ನಿಗೆ
ಸಾತ್ತ್ವಿಕಾ
ಸಾತ್ತ್ವಿಕಾ-ವಸ್ಥೆ
ಸಾಧಕ-ನಲ್ಲಿ
ಸಾಧ-ಕನು
ಸಾಧಕ-ವಾಗಿದೆ
ಸಾಧನ
ಸಾಧನ-ಗಳು
ಸಾಧ-ನದ
ಸಾಧನ-ವಿ-ರ-ಬೇಕು
ಸಾಧನ-ವೆಂದು
ಸಾಧ-ನವೇ
ಸಾಧನೆ
ಸಾಧನೆ-ಗ-ಳನ್ನು
ಸಾಧನೆ-ಗಳಲ್ಲಿ
ಸಾಧನೆ-ಗಳಿ-ರು-ವುದು
ಸಾಧನೆ-ಗಳೂ
ಸಾಧನೆಗೆ
ಸಾಧನೆ-ಮಾಡ-ಬಲ್ಲ
ಸಾಧನೆ-ಮಾಡಿ
ಸಾಧನೆಯ
ಸಾಧನೆ-ಯನ್ನು
ಸಾಧನೆ-ಯಲ್ಲಿ
ಸಾಧನೆ-ಯಾದ
ಸಾಧನೆ-ಯಿಂದ
ಸಾಧನೆಯೇ
ಸಾಧಾ
ಸಾಧಾರಣ
ಸಾಧಾರಣ-ವಾಗಿ
ಸಾಧಾರಣ-ವಾಗಿ-ರುವು
ಸಾಧಾರಣ-ವಾದ
ಸಾಧಾರಣ-ವಾದು-ದಕ್ಕೆ
ಸಾಧಾರಣ-ವಾದು-ದನ್ನು
ಸಾಧಾರಣ-ವಾದುವು
ಸಾಧಿತ-ವಾಗು-ತ್ತದೆ
ಸಾಧಿಸ
ಸಾಧಿಸ-ಬಲ್ಲರು
ಸಾಧಿಸ-ಬಲ್ಲರೊ
ಸಾಧಿಸ-ಬಲ್ಲುದು
ಸಾಧಿಸ-ಬಹುದು
ಸಾಧಿಸ-ಬೇಕಾ
ಸಾಧಿಸ-ಬೇಕಾ-ದರೆ
ಸಾಧಿಸ-ಬೇಕು
ಸಾಧಿಸ-ಬೇಕೆಂದು
ಸಾಧಿಸ-ಬೇಕೆಂಬು-ದನ್ನು
ಸಾಧಿಸ-ಲಾಗುವು-ದಿಲ್ಲ
ಸಾಧಿಸ-ಲಾರರು
ಸಾಧಿಸ-ಲಾರೆವು
ಸಾಧಿ-ಸಲು
ಸಾಧಿಸಲ್ಪಡುತ್ತದೆ
ಸಾಧಿಸಿ
ಸಾಧಿ-ಸಿದ
ಸಾಧಿಸಿ-ದಂತೆ
ಸಾಧಿಸಿ-ದರೆ
ಸಾಧಿಸಿ-ದಾಗ
ಸಾಧಿಸು
ಸಾಧಿಸುತ್ತಾರೆ
ಸಾಧಿ-ಸುವ
ಸಾಧಿಸು-ವಂತಿರ-ಬೇಕು
ಸಾಧಿಸು-ವನು
ಸಾಧಿಸು-ವರು
ಸಾಧಿಸು-ವವ-ರಿಗೆ
ಸಾಧಿಸು-ವಾಗ
ಸಾಧಿಸು-ವುದಕ್ಕೆ
ಸಾಧಿಸು-ವುದರ
ಸಾಧಿಸು-ವುದ-ರಿಂದ
ಸಾಧಿಸು-ವುದು
ಸಾಧಿಸು-ವುದೇ
ಸಾಧಿಸು-ವುವು
ಸಾಧು
ಸಾಧು-ಪುರುಷರು
ಸಾಧು-ವನ್ನು
ಸಾಧು-ವಾಗಿಯೆ
ಸಾಧುವಿ-ನಲ್ಲಿ
ಸಾಧುವು
ಸಾಧು-ಸಂತರು
ಸಾಧ್ಯ
ಸಾಧ್ಯ-ಆತ್ಮನ
ಸಾಧ್ಯತೆ-ಗ-ಳನ್ನು
ಸಾಧ್ಯತೆ-ಗಳು
ಸಾಧ್ಯತೆ-ಗಳೆಲ್ಲ
ಸಾಧ್ಯತೆ-ಯಿದೆ
ಸಾಧ್ಯ-ವಲ್ಲ
ಸಾಧ್ಯ-ವಾಗದೆ
ಸಾಧ್ಯ-ವಾಗ-ಬಲ್ಲವೋ
ಸಾಧ್ಯ-ವಾಗ-ಬಾ-ರದು
ಸಾಧ್ಯ-ವಾಗ-ಬೇಕು
ಸಾಧ್ಯ-ವಾಗ-ಲಾರದು
ಸಾಧ್ಯ-ವಾಗಲಿ
ಸಾಧ್ಯ-ವಾಗ-ಲಿಲ್ಲ
ಸಾಧ್ಯ-ವಾಗಿತ್ತು
ಸಾಧ್ಯ-ವಾಗಿದ್ದಿದ್ದರೆ
ಸಾಧ್ಯ-ವಾಗುತ್ತದೆ
ಸಾಧ್ಯ-ವಾಗುತ್ತಿತ್ತು
ಸಾಧ್ಯ-ವಾಗು-ತ್ತಿರ-ಲಿಲ್ಲ
ಸಾಧ್ಯ-ವಾಗುತ್ತಿಲ್ಲ
ಸಾಧ್ಯ-ವಾಗುವ
ಸಾಧ್ಯ-ವಾಗು-ವಂತೆ
ಸಾಧ್ಯ-ವಾಗು-ವುದಕ್ಕೆ
ಸಾಧ್ಯ-ವಾಗು-ವುದಿಲ್ಲ
ಸಾಧ್ಯ-ವಾಗು-ವುದು
ಸಾಧ್ಯ-ವಾದ
ಸಾಧ್ಯ-ವಾದರೂ
ಸಾಧ್ಯ-ವಾದರೆ
ಸಾಧ್ಯ-ವಾದಷ್ಟು
ಸಾಧ್ಯ-ವಾದಷ್ಟೂ
ಸಾಧ್ಯ-ವಾದಾಗ
ಸಾಧ್ಯ-ವಾದೊಡ-ನೆಯೇ
ಸಾಧ್ಯ-ವಾಯಿತು
ಸಾಧ್ಯ-ವಾಯಿ-ತೆಂದು
ಸಾಧ್ಯ-ವಿದೆ
ಸಾಧ್ಯ-ವಿದ್ದರೆ
ಸಾಧ್ಯ-ವಿದ್ದಿತೋ
ಸಾಧ್ಯ-ವಿರು-ವಂತೆ
ಸಾಧ್ಯ-ವಿಲ್ಲ
ಸಾಧ್ಯ-ವಿಲ್ಲದ
ಸಾಧ್ಯ-ವಿಲ್ಲ-ದಂತೆ
ಸಾಧ್ಯ-ವಿಲ್ಲ-ದಷ್ಟು
ಸಾಧ್ಯ-ವಿಲ್ಲ-ದಿದ್ದರೆ
ಸಾಧ್ಯ-ವಿಲ್ಲದೆ
ಸಾಧ್ಯ-ವಿಲ್ಲದೇ
ಸಾಧ್ಯ-ವಿಲ್ಲ-ವೆಂದು
ಸಾಧ್ಯ-ವಿಲ್ಲ-ವೆಂಬು-ದನ್ನು
ಸಾಧ್ಯ-ವಿಲ್ಲ-ವೆಂಬುದು
ಸಾಧ್ಯ-ವಿಲ್ಲ-ವೆನ್ನುವ
ಸಾಧ್ಯ-ವಿಲ್ಲವೇ
ಸಾಧ್ಯ-ವಿಲ್ಲವೊ
ಸಾಧ್ಯ-ವಿಲ್ಲವೋ
ಸಾಧ್ಯವೂ
ಸಾಧ್ಯವೆ
ಸಾಧ್ಯ-ವೆಂದು
ಸಾಧ್ಯ-ವೆನ್ನುತ್ತಾನೆ
ಸಾಧ್ಯ-ವೆನ್ನುತ್ತಾರೆ
ಸಾಧ್ಯ-ವೆನ್ನು-ವುದು
ಸಾಧ್ಯವೇ
ಸಾಧ್ಯ-ವೇನೊ
ಸಾಧ್ಯವೊ
ಸಾಧ್ಯವೋ
ಸಾನಂದ
ಸಾನ್ನಿಧ್ಯ
ಸಾನ್ನಿಧ್ಯ-ದಲ್ಲಿದ್ದು
ಸಾಪೇಕ್ಷ
ಸಾಪೇಕ್ಷ-ವನ್ನು
ಸಾಪೇಕ್ಷ-ವಾಗಿ
ಸಾಪೇಕ್ಷ-ವಾ-ಗು-ವುದು
ಸಾಪೇಕ್ಷ-ವಾ-ದದ್ದು
ಸಾಪೇಕ್ಷ-ವಾ-ದುದು
ಸಾಮಗ್ರಿ
ಸಾಮರಸ್ಯ
ಸಾಮರಸ್ಯದ
ಸಾಮರಸ್ಯ-ದಲ್ಲಿ
ಸಾಮರಸ್ಯ-ದಿಂದ
ಸಾಮರಸ್ಯ-ದಿಂದಿ-ರು-ವುದು
ಸಾಮರಸ್ಯ-ವನ್ನು
ಸಾಮರಸ್ಯ-ವಿದೆ
ಸಾಮರಸ್ಯವು
ಸಾಮರ್ಥ್ಯ
ಸಾಮರ್ಥ್ಯ-ವನ್ನು
ಸಾಮರ್ಥ್ಯ-ವಿದ್ದಷ್ಟು
ಸಾಮಾಜಿಕ
ಸಾಮಾಜಿಕ-ವಾ-ದದ್ದು
ಸಾಮಾನು-ಗಳೆಲ್ಲ-ವನ್ನೂ
ಸಾಮಾನ್ಯ
ಸಾಮಾನ್ಯಕ್ಕೆ
ಸಾಮಾನ್ಯತೆ
ಸಾಮಾನ್ಯ-ದಿಂದ
ಸಾಮಾನ್ಯರ
ಸಾಮಾನ್ಯ-ರಲ್ಲಿ
ಸಾಮಾನ್ಯ-ರೂಪಕ್ಕೆ
ಸಾಮಾನ್ಯ-ವಾಗಿ
ಸಾಮಾನ್ಯ-ವಾಗಿದೆ
ಸಾಮಾನ್ಯ-ವಾಗಿ-ರುವ
ಸಾಮಾನ್ಯ-ವಾದ
ಸಾಮಾನ್ಯ-ವಾದು-ದನ್ನು
ಸಾಮಾನ್ಯ-ವಾದುದು
ಸಾಮಾನ್ಯ-ವೆಂದು
ಸಾಮಾನ್ಯೀ-ಕರಣ
ಸಾಮಾನ್ಯೀ-ಕರ-ಣದ
ಸಾಮಾನ್ಯೀ-ಕರ-ಣ-ವಲ್ಲ
ಸಾಮಾನ್ಯೀಕ-ರಿಸಿ-ರು-ವರು
ಸಾಮೂಹಿಕ
ಸಾಮ್ಯ-ವನ್ನೂ
ಸಾಮ್ಯ-ವಿದೆಯೆ
ಸಾಮ್ಯೇ
ಸಾಯ
ಸಾಯಂಕಾಲ
ಸಾಯ-ದಂತೆ
ಸಾಯ-ಬಲ್ಲ
ಸಾಯ-ಬೇಕಾ-ಗಿಲ್ಲ
ಸಾಯ-ಬೇಕಾದ
ಸಾಯ-ಬೇಕು
ಸಾಯ-ಲಾರದು
ಸಾಯ-ಲಾರದೊ
ಸಾಯ-ಲಿಚ್ಛಿಸು-ವೆನು
ಸಾಯು
ಸಾಯುಜ್ಯ
ಸಾಯುತ್ತಾನೆ
ಸಾಯುತ್ತಾರೆ
ಸಾಯುತ್ತಿದೆ
ಸಾಯುತ್ತಿದ್ದಂತೆ
ಸಾಯುತ್ತಿದ್ದೇವೆ
ಸಾಯುತ್ತಿ-ರು-ವರು
ಸಾಯುತ್ತಿ-ರುವ-ವರು
ಸಾಯುತ್ತೇನೆ
ಸಾಯುತ್ತೇವೆ
ಸಾಯುವ
ಸಾಯು-ವನು
ಸಾಯು-ವ-ನೆಂಬು-ದನ್ನು
ಸಾಯು-ವರು
ಸಾಯು-ವರೊ
ಸಾಯು-ವ-ವ-ನನ್ನು
ಸಾಯು-ವಾಗ
ಸಾಯು-ವು-ದಕ್ಕೆ
ಸಾಯು-ವು-ದನ್ನು
ಸಾಯು-ವು-ದ-ರಲ್ಲಿಯೇ
ಸಾಯು-ವು-ದ-ರಲ್ಲಿ-ರು-ವೆವು
ಸಾಯು-ವು-ದ-ರಿಂದ
ಸಾಯು-ವು-ದಿಲ್ಲ
ಸಾಯು-ವು-ದಿಲ್ಲ-ವೆಂದು
ಸಾಯು-ವುದು
ಸಾಯು-ವುದೂ
ಸಾಯು-ವೆವು
ಸಾಯು-ವೆವೆ
ಸಾರ
ಸಾರಥಿ
ಸಾರ-ಥಿಯ
ಸಾರದ
ಸಾರ-ದಂತೆ
ಸಾರ-ದಂತೆಯೊ
ಸಾರನು
ಸಾರಲು
ಸಾರ-ವನ್ನು
ಸಾರ-ವನ್ನೇ
ಸಾರ-ವಲ್ಲ
ಸಾರ-ವಸ್ತು-ವಿಗೆ
ಸಾರ-ವಸ್ತು-ವೆಂದು
ಸಾರ-ವಾಗಿದೆ
ಸಾರ-ವಾ-ಗಿ-ರುವ
ಸಾರ-ವಾದ
ಸಾರವು
ಸಾರವೂ
ಸಾರವೆ
ಸಾರವೇ
ಸಾರವೊ
ಸಾರವೋ
ಸಾರ-ಸರ್ವಸ್ವ
ಸಾರ-ಸರ್ವಸ್ವವೆ
ಸಾರಸ್ವ-ರೂಪ-ವಾದ
ಸಾರಾಂಶ-ವನ್ನು
ಸಾರಾಂಶ-ವಿದು
ಸಾರಾಂಶ-ವಿಷ್ಟೆ
ಸಾರಿ
ಸಾರಿದ
ಸಾರಿ-ದರು
ಸಾರಿ-ದು-ದ-ರಿಂದ
ಸಾರಿದೆ
ಸಾರಿ-ಯಾ-ದರೂ
ಸಾರಿ-ರುವನು
ಸಾರಿ-ರು-ವು-ದನ್ನು
ಸಾರಿವೆ
ಸಾರು
ಸಾರುತ್ತದೆ
ಸಾರುತ್ತವೆ
ಸಾರುತ್ತಾ
ಸಾರುತ್ತಾನೆ
ಸಾರುತ್ತಾರೆ
ಸಾರುವ
ಸಾರು-ವರು
ಸಾರುವು
ಸಾರು-ವು-ದಕ್ಕೆ
ಸಾರು-ವು-ದಲ್ಲ
ಸಾರು-ವು-ದಿಲ್ಲ
ಸಾರು-ವುದು
ಸಾರು-ವು-ದೇನು
ಸಾರು-ವುವು
ಸಾರು-ವುವೊ
ಸಾರೂಪ್ಯ
ಸಾರ್ಥಕ-ವಾದ
ಸಾರ್ಥಕ-ವಾ-ಯಿತು
ಸಾರ್ವ
ಸಾರ್ವತ್ರಿಕ
ಸಾರ್ವತ್ರಿಕ-ತೆ-ಯಲ್ಲಿದೆ
ಸಾರ್ವ-ಭೌ-ಮತ್ವ
ಸಾರ್ವ-ಭೌಮ-ರೊಂದಿಗೆ
ಸಾರ್ವ-ಭೌಮಾ
ಸಾಲ
ಸಾಲಂಬನಂ
ಸಾಲ-ದಾ-ಯಿತು
ಸಾಲದು
ಸಾಲ-ದು-ನಾನು
ಸಾಲದೆ
ಸಾಲ-ದೆಂಬಂತೆ
ಸಾಲ-ದೆಂಬುದು
ಸಾಲ-ಮಾಡಿಯಾ
ಸಾಲವು
ಸಾವ-ಧಾನ-ದಿಂದ
ಸಾವ-ಧಾನ-ವಾಗಿ
ಸಾವ-ಧಾನ-ವಾದ
ಸಾವನ್ನು
ಸಾವಿ
ಸಾವಿಗೆ
ಸಾವಿನ
ಸಾವಿ-ನಲ್ಲಿ
ಸಾವಿ-ನಿಂದ
ಸಾವಿರ
ಸಾವಿ-ರಕ್ಕೆ
ಸಾವಿ-ರದ
ಸಾವಿ-ರ-ಪಾಲು
ಸಾವಿ-ರ-ವಾ-ಗಿ-ರು-ವೆವು
ಸಾವಿ-ರಾರು
ಸಾವಿಲ್ಲ
ಸಾವು
ಸಾವು-ಗಳಿಲ್ಲ-ವೆನ್ನು-ವುದು
ಸಾವು-ಗಳು
ಸಾವೆಂದು
ಸಾಸಿವೆ-ಕಾಳಿ-ನಷ್ಟು
ಸಾಸುವೆ
ಸಾಸ್ಮಿತ
ಸಾಹ-ಚರ್ಯ
ಸಾಹಸ
ಸಾಹಸಕ್ಕೆ
ಸಾಹ-ಸದ
ಸಾಹಸಪ್ರಿಯರೂ
ಸಾಹಸ-ವೆಂದು
ಸಾಹಸಿಗ-ರಿಗೆ
ಸಾಹಸಿ-ಗಳಾ-ಗಿದ್ದರು
ಸಾಹಸಿ-ಗಳು
ಸಾಹಿತ್ಯ-ದಲ್ಲಿ
ಸಿ
ಸಿಂಧು
ಸಿಂಧು-ವಿ-ನಲ್ಲಿ
ಸಿಂಹ
ಸಿಂಹಕ್ಕೆ
ಸಿಂಹ-ಗಳ
ಸಿಂಹ-ಗಳು
ಸಿಂಹದ
ಸಿಂಹ-ದಂತೆ
ಸಿಂಹ-ವನ್ನು
ಸಿಂಹ-ವನ್ನೂ
ಸಿಂಹ-ವಾಗಿ
ಸಿಂಹವು
ಸಿಂಹವೂ
ಸಿಂಹಾಸ-ನದ
ಸಿಕ್ಕದ
ಸಿಕ್ಕದೆ
ಸಿಕ್ಕದೇ
ಸಿಕ್ಕ-ಬೇಕಾ-ದರೆ
ಸಿಕ್ಕ-ಲಿಲ್ಲ
ಸಿಕ್ಕ-ಲಿಲ್ಲ-ವೆಂದು
ಸಿಕ್ಕಿ
ಸಿಕ್ಕಿ-ಕೊಂಡ
ಸಿಕ್ಕಿ-ಕೊಳ್ಳು-ವುದು
ಸಿಕ್ಕಿತು
ಸಿಕ್ಕಿದ
ಸಿಕ್ಕಿ-ದಂತೆ
ಸಿಕ್ಕಿ-ದರೆ
ಸಿಕ್ಕಿ-ದಾಗ
ಸಿಕ್ಕಿದೆ
ಸಿಕ್ಕಿ-ದೊ-ಡ-ನೆಯೆ
ಸಿಕ್ಕಿ-ಬೀಳು-ವುದು
ಸಿಕ್ಕಿ-ರುವ
ಸಿಕ್ಕಿ-ರುವ-ವರು
ಸಿಕ್ಕಿಲ್ಲ
ಸಿಕ್ಕಿವೆ
ಸಿಕ್ಕಿ-ಸಿದ
ಸಿಕ್ಕಿ-ಸಿದಾಗ
ಸಿಕ್ಕುತ್ತದೆ
ಸಿಕ್ಕುತ್ತದೆ-ಅ-ದಕ್ಕೆ
ಸಿಕ್ಕುತ್ತಾರೆ
ಸಿಕ್ಕುತ್ತಿತ್ತು
ಸಿಕ್ಕುವ
ಸಿಕ್ಕು-ವರು
ಸಿಕ್ಕು-ವಾಗ
ಸಿಕ್ಕು-ವು-ದಿಲ್ಲ
ಸಿಕ್ಕು-ವು-ದಿಲ್ಲ-ವೆಂದು
ಸಿಕ್ಕು-ವುದು
ಸಿಕ್ಕು-ವುದೆ
ಸಿಕ್ಕುವು-ದೆಂದು
ಸಿಗರೇಟಿನ
ಸಿಗರೇಟು
ಸಿಗಿದು
ಸಿಗುತ್ತದೆ
ಸಿಗು-ವುದೆ
ಸಿಡಿಗುಂಡನ್ನು
ಸಿಡಿ-ಲನ್ನು
ಸಿಡಿ-ಲಿನ
ಸಿಡಿಲು
ಸಿತ್ತು
ಸಿದ್ಥಾಂತ-ವಿರು-ವುದು
ಸಿದ್ದಾಂತ
ಸಿದ್ದಿಸು
ಸಿದ್ಧ
ಸಿದ್ಧತೆ
ಸಿದ್ಧತೆ-ಗಳ
ಸಿದ್ಧತೆ-ಗಳು
ಸಿದ್ಧ-ದರ್ಶನಮ್
ಸಿದ್ಧ-ದರ್ಶನ-ವಾ-ಗು-ವುದು
ಸಿದ್ಧ-ನಾಗ-ಬೇಕಾ-ದರೆ
ಸಿದ್ಧ-ನಾಗಿದ್ದ
ಸಿದ್ಧ-ನಾಗಿದ್ದನು
ಸಿದ್ಧ-ನಾಗಿರು-ವಾಗ
ಸಿದ್ಧ-ನಾಗಿರು-ವೆನು
ಸಿದ್ಧ-ನಾದ
ಸಿದ್ಧ-ಪಡಿ-ಸಲು
ಸಿದ್ಧ-ಪಡಿ-ಸುವ-ತನಕ
ಸಿದ್ಧ-ಪುರುಷ
ಸಿದ್ಧ-ಪುರುಷ-ನ-ವರೆಗೆ
ಸಿದ್ಧ-ಪುರುಷರು
ಸಿದ್ಧ-ಮಾಡಿ
ಸಿದ್ಧ-ಮಾನವ
ಸಿದ್ಧ-ಮಾನ-ವನ
ಸಿದ್ಧಯಃ
ಸಿದ್ಧ-ರನ್ನು
ಸಿದ್ಧ-ರಾಗಲು
ಸಿದ್ಧ-ರಾಗಿದ್ದರು
ಸಿದ್ಧ-ರಾಗಿದ್ದರೋ
ಸಿದ್ಧ-ರಾಗಿರ-ಬೇಕು
ಸಿದ್ಧ-ರಾಗಿರಿ
ಸಿದ್ಧ-ರಾಗಿ-ರುವ
ಸಿದ್ಧ-ರಾಗಿ-ರುವರು
ಸಿದ್ಧ-ರಾಗಿ-ರುವಿರಾ
ಸಿದ್ಧ-ರಾದರೆ
ಸಿದ್ಧರು
ಸಿದ್ಧರೊ
ಸಿದ್ಧ-ವಾಗಿದ್ದನು
ಸಿದ್ಧ-ವಾಗಿ-ರ-ಬೇಕು
ಸಿದ್ಧ-ವಾಗಿ-ರುತ್ತವೆ
ಸಿದ್ಧ-ವಾಗಿ-ರುವ
ಸಿದ್ಧ-ವಾಗಿರು-ವಂತಹ
ಸಿದ್ಧ-ವಾಗಿರು-ವರು
ಸಿದ್ಧ-ವಾಗಿರು-ವುದು
ಸಿದ್ಧ-ವಾಗಿ-ರುವು-ದುಈ
ಸಿದ್ಧ-ವಾಗು-ತ್ತದೆ
ಸಿದ್ಧ-ವಾಗು-ವುದು
ಸಿದ್ಧ-ವಾದ
ಸಿದ್ಧಾಂತ
ಸಿದ್ಧಾಂತಕ್ಕೆ
ಸಿದ್ಧಾಂತ-ಗಳ
ಸಿದ್ಧಾಂತ-ಗಳಂತೆಯೇ
ಸಿದ್ಧಾಂತ-ಗಳನ್ನು
ಸಿದ್ಧಾಂತ-ಗಳನ್ನೂ
ಸಿದ್ಧಾಂತ-ಗಳನ್ನೆಲ್ಲಾ
ಸಿದ್ಧಾಂತ-ಗಳ-ಲೆಲ್ಲಾ
ಸಿದ್ಧಾಂತ-ಗಳಲ್ಲಿ
ಸಿದ್ಧಾಂತ-ಗಳಲ್ಲಿಯೂ
ಸಿದ್ಧಾಂತ-ಗಳಾ
ಸಿದ್ಧಾಂತ-ಗಳಿಗೆ
ಸಿದ್ಧಾಂತ-ಗಳಿವೆ
ಸಿದ್ಧಾಂತ-ಗಳು
ಸಿದ್ಧಾಂತ-ಗಳೂ
ಸಿದ್ಧಾಂತ-ಗಳೆ
ಸಿದ್ಧಾಂತ-ಗಳೆಲ್ಲ
ಸಿದ್ಧಾಂತ-ಗಳೆಲ್ಲ-ವನ್ನೂ
ಸಿದ್ಧಾಂತ-ಗಳೆಲ್ಲಾ
ಸಿದ್ಧಾಂತದ
ಸಿದ್ಧಾಂತ-ದಲ್ಲಿಯೂ
ಸಿದ್ಧಾಂತ-ದಿಂದ
ಸಿದ್ಧಾಂತ-ದಿಂದಲೂ
ಸಿದ್ಧಾಂತ-ವನು
ಸಿದ್ಧಾಂತ-ವನ್ನು
ಸಿದ್ಧಾಂತ-ವನ್ನೂ
ಸಿದ್ಧಾಂತ-ವನ್ನೇ
ಸಿದ್ಧಾಂತ-ವಲ್ಲ
ಸಿದ್ಧಾಂತ-ವಲ್ಲದೆ
ಸಿದ್ಧಾಂತ-ವಾಗಿ
ಸಿದ್ಧಾಂತ-ವಾಗಿದೆ
ಸಿದ್ಧಾಂತ-ವಾಗುತ್ತದೆ
ಸಿದ್ಧಾಂತ-ವಾ-ಗು-ವುದು
ಸಿದ್ಧಾಂತ-ವಿದು
ಸಿದ್ಧಾಂತ-ವಿದೆ
ಸಿದ್ಧಾಂತವು
ಸಿದ್ಧಾಂತವೂ
ಸಿದ್ಧಾಂತ-ವೆಂದರೆ
ಸಿದ್ಧಾಂತ-ವೆಂಬ
ಸಿದ್ಧಾಂತ-ವೆನ್ನು-ವುದು
ಸಿದ್ಧಾಂತವೇ
ಸಿದ್ಧಾಂತ-ವೇನಾ
ಸಿದ್ಧಾಂತ-ವೇನೂ
ಸಿದ್ಧಾಂತ-ವೇ-ನೆಂದರೆ
ಸಿದ್ಧಾಂತ-ವೇನೋ
ಸಿದ್ಧಾಂತ-ವೊಂದು
ಸಿದ್ಧಾತ-ಗಳೂ
ಸಿದ್ಧಾತ್ಮ-ನಾಗುವ-ವರೆಗೆ
ಸಿದ್ಧಿ
ಸಿದ್ಧಿ-ಗಳನ್ನು
ಸಿದ್ಧಿ-ಗಳನ್ನೂ
ಸಿದ್ಧಿ-ಗಳನ್ನೆಲ್ಲ
ಸಿದ್ಧಿ-ಗಳಿಂದ
ಸಿದ್ಧಿ-ಗಳು
ಸಿದ್ಧಿ-ಗಳೆಲ್ಲ
ಸಿದ್ಧಿ-ಗಾಗಿ
ಸಿದ್ಧಿಯ
ಸಿದ್ಧಿ-ಯನ್ನು
ಸಿದ್ಧಿ-ಯಾಗಿದ್ದರೆ
ಸಿದ್ಧಿ-ಯೊಂದಿಗೆ
ಸಿದ್ಧಿ-ಸದ
ಸಿದ್ಧಿ-ಸದೆ
ಸಿದ್ಧಿ-ಸ-ಲಿಲ್ಲ
ಸಿದ್ಧಿ-ಸಿದ
ಸಿದ್ಧಿ-ಸುತ್ತದೆ
ಸಿದ್ಧಿ-ಸುವ
ಸಿದ್ಧಿ-ಸುವ-ವರೆಗೂ
ಸಿದ್ಧಿ-ಸುವ-ವರೆಗೆ
ಸಿದ್ಧಿ-ಸು-ವು-ದಿಲ್ಲ
ಸಿದ್ಧಿ-ಸು-ವುದು
ಸಿಪಾಯಿ-ದಂಗೆಯ
ಸಿಲು-ಕಿದ
ಸಿಲುಕಿ-ದರೆ
ಸಿಲುಕಿ-ರುವ
ಸಿಲುಕಿ-ರುವು-ದೆಲ್ಲ
ಸಿಲುಕು-ವಂತೆ
ಸಿವೆ
ಸಿಹಿ
ಸಿಹಿ-ಕಹಿ
ಸಿಹಿ-ಯಾ-ಗಲಿ
ಸೀಮಿತ-ವಾ-ಗಿ-ರುವುದು
ಸೀಮಿತ-ವಾದ
ಸೀಮಿತ-ವಾ-ದಾಗ
ಸೀಮಿತ-ವಾ-ಯಿತು
ಸೀಮೆ
ಸೀಯು-ವುದು
ಸೀಳಿ-ದಾಗ
ಸೀಳು-ವಂತೆ
ಸೀಳು-ವುದು
ಸುಂಟರ-ಗಾಳಿ
ಸುಂಟರ-ಗಾಳಿ-ಯಂತೆ
ಸುಂಟರ-ಗಾಳಿ-ಯಿಂದ
ಸುಂದರ
ಸುಂದರ-ಕು-ರೂಪಿ-ಗಳಲ್ಲಿ
ಸುಂದರ-ನೆಂದು
ಸುಂದರ-ವಾಗಿ
ಸುಂದರ-ವಾಗಿ-ಡ-ಬೇಕು
ಸುಂದರ-ವಾಗಿದೆ
ಸುಂದರ-ವಾಗಿ-ರ-ಬಹುದು
ಸುಂದರ-ವಾಗಿ-ರು-ವು-ದನ್ನು
ಸುಂದರ-ವಾಗಿವೆ
ಸುಂದರ-ವಾ-ಗು-ವುದು
ಸುಂದರ-ವಾದ
ಸುಂದರಿ-ಯಾದ
ಸುಕರ್ಮಿ-ಗಳೊ
ಸುಖ
ಸುಖ-ಇವು-ಗಳ
ಸುಖ-ಇವು-ಗಳು
ಸುಖ-ಕರ-ವಾಗಿ-ರು-ವು-ದಿಲ್ಲ
ಸುಖಕ್ಕಾಗಿ
ಸುಖಕ್ಕಿಂತ
ಸುಖಕ್ಕೆ
ಸುಖಕ್ಕೋಸ್ಕರ-ವಾಗಿಯೇ
ಸುಖಕ್ಷೇತ್ರ
ಸುಖ-ಗ-ಳನ್ನು
ಸುಖ-ಗಳಲ್ಲಿ
ಸುಖ-ಗಳಿಗೆ
ಸುಖ-ಗಳು
ಸುಖದ
ಸುಖ-ದಲ್ಲಿ
ಸುಖ-ದಷ್ಟೇ
ಸುಖ-ದಾಯಕ
ಸುಖ-ದಾಯ-ಕವೊ
ಸುಖ-ದಿಂದ
ಸುಖ-ದುಃಖ
ಸುಖ-ದುಃಖ-ಗಳ
ಸುಖ-ದುಃಖ-ಗ-ಳನ್ನು
ಸುಖ-ದುಃಖ-ಗಳಲ್ಲಿ
ಸುಖ-ದುಃಖ-ಗಳಾಚೆ
ಸುಖ-ದುಃಖ-ಗಳಿಗೆ
ಸುಖ-ದುಃಖ-ಗಳು
ಸುಖ-ದುಃಖ-ಗಳೆಲ್ಲ
ಸುಖ-ದುಃಖ-ದಲ್ಲಿ
ಸುಖ-ದುಃಖ-ವೆನ್ನು-ವು-ದನ್ನೆಲ್ಲ
ಸುಖ-ದೆ-ಡೆಗೆ
ಸುಖ-ಪಡು
ಸುಖ-ಭೋಗ-ಗ-ಳನ್ನು
ಸುಖ-ಭೋಗ-ಗಳಿಂದ
ಸುಖ-ಮಯ-ವಾದ
ಸುಖ-ಮ-ಯವೂ
ಸುಖ-ಲಾಭಃ
ಸುಖ-ಲಾಭ-ವಾ-ಗು-ವುದು
ಸುಖ-ಲಾಲಸೆ-ಗ-ಳನ್ನು
ಸುಖ-ವನ್ನು
ಸುಖ-ವನ್ನೂ
ಸುಖ-ವಾ-ಗಲಿ
ಸುಖ-ವಾಗಿ
ಸುಖ-ವಾಗಿದ್ದರೆ
ಸುಖ-ವಾಗಿದ್ದು
ಸುಖ-ವಾಗಿದ್ದೇನೆ
ಸುಖ-ವಾಗಿ-ರ-ಬೇಕು
ಸುಖ-ವಾಗಿ-ರ-ಬೇಕೆಂಬ
ಸುಖ-ವಾಗಿ-ರಲಿ
ಸುಖ-ವಾಗಿ-ರು-ವರು
ಸುಖ-ವಾಗಿ-ರು-ವು-ದಕ್ಕೆ
ಸುಖ-ವಾ-ಗಿ-ರು-ವೆನು
ಸುಖ-ವಾಗಿ-ರು-ವೆ-ನೆಂದು
ಸುಖ-ವಾ-ಗು-ವುದು
ಸುಖ-ವಾದ
ಸುಖ-ವಿ-ರುವ
ಸುಖವು
ಸುಖವೂ
ಸುಖ-ವೆಂದ-ರೇನು
ಸುಖ-ವೆಂದು
ಸುಖ-ವೆಂಬ
ಸುಖ-ವೆಂಬು-ದಿಲ್ಲ
ಸುಖ-ವೆಲ್ಲ
ಸುಖ-ವೆಲ್ಲಾ
ಸುಖವೇ
ಸುಖ-ಸಂಪತ್ತು-ಗಳಲ್ಲಿ
ಸುಖ-ಸಂಸ್ಕಾರ-ಗಳು
ಸುಖಾನುಶಯೀ
ಸುಖಾಭಿಲಾಷೆ-ಗಳಿಗೆ
ಸುಖಿ
ಸುಖಿ-ಗಳನ್ನಾಗಿ
ಸುಖಿ-ಗಳನ್ನಾಗಿಯೂ
ಸುಖಿ-ಗಳಾ-ಗಿ-ರು-ವು-ದಕ್ಕೆ
ಸುಖಿ-ಗಳು
ಸುಖಿ-ದುಃಖಿ-ಗಳಲ್ಲಿ
ಸುಖಿ-ಯಾಗಿರು
ಸುಖಿ-ಯಾಗಿ-ರು-ವೆನು
ಸುಗಣ
ಸುಗುಣ
ಸುಗುಣ-ಗಳನ್ನೆಲ್ಲ
ಸುಜಾಗ್ರತ-ನಾಗಿ
ಸುಟ್ಟರೂ
ಸುಡಲೂ-ಬಹುದು
ಸುಡುತ್ತಿದ್ದರು
ಸುಡುವ
ಸುತರು
ಸುತ್ತ
ಸುತ್ತದೆ
ಸುತ್ತನ್ನು
ಸುತ್ತ-ಮುತ್ತಲೂ
ಸುತ್ತ-ಲಿ-ರುವ
ಸುತ್ತ-ಲಿ-ರುವು-ದೆಲ್ಲ
ಸುತ್ತಲೂ
ಸುತ್ತಾ-ಡ-ಬಹುದು
ಸುತ್ತಾಡಿ
ಸುತ್ತಾ-ಡುತ್ತಿ-ರ-ಬೇಕು
ಸುತ್ತಿ
ಸುತ್ತಿದೆ
ಸುತ್ತಿದ್ದರು
ಸುತ್ತಿ-ರುವನು
ಸುತ್ತಿ-ರುವುದು
ಸುತ್ತು
ಸುತ್ತುತ್ತಿದ್ದಾನೆ
ಸುತ್ತುತ್ತಿ-ರುವ
ಸುತ್ತುತ್ತಿ-ರುವನು
ಸುತ್ತುತ್ತಿ-ರುವು-ದ-ರಿಂದ
ಸುತ್ತುತ್ತಿ-ರುವುದು
ಸುತ್ತುತ್ತಿ-ರು-ವುವು
ಸುತ್ತು-ವರಿ-ದಿದ್ದರೂ
ಸುತ್ತು-ವರಿ-ಯಲ್ಪಟ್ಟಿದೆ
ಸುತ್ತು-ವಾಗ
ಸುತ್ತೇವೆ
ಸುದ್ದಿ
ಸುದ್ದಿಯ
ಸುದ್ದಿ-ಯನ್ನು
ಸುದ್ಧಿಯು
ಸುಧಾರಕ
ಸುಧಾರ-ಕನು
ಸುಧಾರ-ಕರು
ಸುಧಾರ-ಣೆ-ಗಳು
ಸುನಿಶ್ಚಿತಾರ್ಥ-ವಾದ
ಸುಪ್ತ
ಸುಪ್ತ-ವಾಗಿ
ಸುಪ್ತ-ವಾಗಿ-ದೆಯೋ
ಸುಪ್ತ-ವಾಗಿದ್ದ-ವೆಂದೂ
ಸುಪ್ತ-ವಾಗಿ-ರುವ
ಸುಪ್ತ-ವಾಗಿ-ರುವುದು
ಸುಪ್ತ-ವಾಗಿ-ರು-ವುವು
ಸುಪ್ತ-ವಾದ
ಸುಪ್ತ-ವಾ-ದಂತೆ
ಸುಪ್ತಾ-ವಸ್ಥೆ
ಸುಪ್ತಾ-ವಸ್ಥೆಗೆ
ಸುಪ್ತಾ-ವಸ್ಥೆ-ಯಲ್ಲಿದ್ದು
ಸುಪ್ತಾ-ವಸ್ಥೆ-ಯಿಂದ
ಸುಪ್ತಿ
ಸುಪ್ತ್ಯಾವಾ-ಹಕ
ಸುಪ್ತ್ಯಾವಾ-ಹ-ಕನು
ಸುಪ್ತ್ಯಾವಾ-ಹ-ಕರು
ಸುಪ್ತ್ಯಾವಾ-ಹ-ನೆಗೆ
ಸುಪ್ತ್ಯಾವಾ-ಹ-ನೆಯ
ಸುಮಾರು
ಸುಮ್ಮನಿ-ರಿಸು-ವು-ದಲ್ಲ
ಸುಮ್ಮನಿರು
ಸುಮ್ಮನೆ
ಸುರಕ್ಷಿತ
ಸುರಕ್ಷಿತ-ವಾಗಿ
ಸುರಕ್ಷಿತ-ವಾಗಿ-ರುತ್ತಿತ್ತೊ
ಸುರಕ್ಷಿತ-ವಾಗಿ-ರು-ವುವೊ
ಸುರಿಯು-ವುದು
ಸುರೆ-ಯನ್ನು
ಸುಲಭ
ಸುಲಭ-ಫ-ವಾಗಿ
ಸುಲಭ-ವಲ್ಲ
ಸುಲಭ-ವಾ-ಗಲಿ
ಸುಲಭ-ವಾಗ-ಲೆಂದು
ಸುಲಭ-ವಾಗಿ
ಸುಲಭ-ವಾ-ಗಿತ್ತು
ಸುಲಭ-ವಾಗು-ವುದೆ
ಸುಲಭ-ವಾದ
ಸುಲಭ-ವಾದುದು
ಸುಲಭ-ವಾ-ಯಿತು
ಸುಲಭ-ವೇನೂ
ಸುಲ-ಭವೋ
ಸುಲ್ತಾ-ನನು
ಸುಳಿ
ಸುಳಿ-ಗ-ಳಾದುವು
ಸುಳಿ-ಗಳಿವೆ
ಸುಳಿ-ಗಳು
ಸುಳಿಗೆ
ಸುಳಿದು
ಸುಳಿ-ಯಂತಿದೆ
ಸುಳಿ-ಯಂತೆ
ಸುಳಿ-ಯನ್ನು
ಸುಳಿ-ಯಲ್ಲಿ
ಸುಳಿ-ಯಲ್ಲಿಯೇ
ಸುಳಿಯೂ
ಸುಳಿವು
ಸುಳಿವೇ
ಸುಳ್ಳನ್ನು
ಸುಳ್ಳಲ್ಲವೆ
ಸುಳ್ಳಾ-ದರೂ
ಸುಳ್ಳಿಗೆ
ಸುಳ್ಳಿನ
ಸುಳ್ಳು
ಸುಳ್ಳು-ಮೋಸ-ಗಳ
ಸುಳ್ಳೆ
ಸುವ
ಸುವನು
ಸುವಾಸ-ನೆಯ
ಸುವಿರಿ
ಸುವು-ದಕ್ಕೆ
ಸುವು-ದಿಲ್ಲ
ಸುವು-ದಿಲ್ಲವೋ
ಸುವುದು
ಸುವುವು
ಸುವೆವು
ಸುವ್ಯವಸ್ಥಿತ
ಸುವ್ಯವಸ್ಥಿತ-ವಾಗಿದೆ
ಸುಷಮ್ನಾ
ಸುಷಮ್ನಾ-ಕಾಲುವೆ
ಸುಷುಮ್ನ
ಸುಷುಮ್ನಾ
ಸುಸಂಸ್ಕೃತ
ಸುಸಂಸ್ಕೃತರು
ಸುಸಂಸ್ಕೃತ-ವಾಗಿದೆ
ಸುಸಂಸ್ಕೃತ-ವಾಗಿ-ರಲಿ
ಸುಸ್ಪಷ್ಟ-ಗೊಳಿ-ಸ-ಬೇಕಾ-ಗಿದೆ
ಸೂಕ್ತ
ಸೂಕ್ತ-ವಾದ
ಸೂಕ್ತ-ವಾದು-ದನ್ನು
ಸೂಕ್ತ-ವೆಂದು
ಸೂಕ್ಷಾ-ಕಾರ
ಸೂಕ್ಷಾ-ವಸ್ಥೆ-ಯಲ್ಲಿ-ರುವ
ಸೂಕ್ಷಾ-ವಸ್ಥೆ-ಯಲ್ಲಿ-ರುವುದು
ಸೂಕ್ಷ್ಮ
ಸೂಕ್ಷ್ಮ-ಕಣ-ಗಳ
ಸೂಕ್ಷ್ಮ-ಕಣ-ಗಳಿಂದಾಗಿ
ಸೂಕ್ಷ್ಮ-ಕಣದ
ಸೂಕ್ಷ್ಮ-ಕಾರ-ಣಕ್ಕೆ
ಸೂಕ್ಷ್ಮ-ಕಾರ-ಣವೊ
ಸೂಕ್ಷ್ಮಕ್ಕೆ
ಸೂಕ್ಷ್ಮಗ್ರಹಣ
ಸೂಕ್ಷ್ಮ-ತತ್ತ್ವದ
ಸೂಕ್ಷ್ಮ-ತತ್ತ್ವ-ವಾದ
ಸೂಕ್ಷ್ಮ-ತಮ
ಸೂಕ್ಷ್ಮ-ತರ
ಸೂಕ್ಷ್ಮ-ತರ-ವಾದ
ಸೂಕ್ಷ್ಮ-ತರ್ಕದ
ಸೂಕ್ಷ್ಮದ
ಸೂಕ್ಷ್ಮ-ದಲ್ಲಿಯೇ
ಸೂಕ್ಷ್ಮ-ದೃಷ್ಟಿ-ಯನ್ನು
ಸೂಕ್ಷ್ಮ-ದೇಹದ
ಸೂಕ್ಷ್ಮ-ದೇಹ-ದೊಂದಿಗೆ
ಸೂಕ್ಷ್ಮ-ದೇಹ-ವಾ-ಗಿ-ರುವನು
ಸೂಕ್ಷ್ಮ-ದೇಹ-ವಿ-ರು-ವುದು
ಸೂಕ್ಷ್ಮ-ದೇಹ-ವೆಂಬ
ಸೂಕ್ಷ್ಮ-ದೇಹಿ
ಸೂಕ್ಷ್ಮದ್ರವ್ಯ-ವಾ-ಗು-ವುವು
ಸೂಕ್ಷ್ಮ-ನರ-ಗಳು
ಸೂಕ್ಷ್ಮಪ್ರಪಂಚದ
ಸೂಕ್ಷ್ಮ-ಭಾವ-ದಲ್ಲಿ
ಸೂಕ್ಷ್ಮ-ಭಾವ-ನೆ-ಗ-ಳನ್ನು
ಸೂಕ್ಷ್ಮ-ಭೂತ-ಗಳು
ಸೂಕ್ಷ್ಮ-ಮ-ತಿಯೂ
ಸೂಕ್ಷ್ಮ-ರೂಪ-ದಲ್ಲಿ
ಸೂಕ್ಷ್ಮ-ವನ್ನು
ಸೂಕ್ಷ್ಮ-ವಸ್ತು-ಗಳ
ಸೂಕ್ಷ್ಮ-ವಸ್ತು-ಗ-ಳನ್ನು
ಸೂಕ್ಷ್ಮ-ವಸ್ತು-ವಿನ
ಸೂಕ್ಷ್ಮ-ವಸ್ತು-ವಿ-ನಿಂದ
ಸೂಕ್ಷ್ಮ-ವಾಗಿ
ಸೂಕ್ಷ್ಮ-ವಾಗಿದೆ
ಸೂಕ್ಷ್ಮ-ವಾಗಿ-ದೆಯೊ
ಸೂಕ್ಷ್ಮ-ವಾಗಿದ್ದರೂ
ಸೂಕ್ಷ್ಮ-ವಾಗಿ-ರುವ
ಸೂಕ್ಷ್ಮ-ವಾಗಿ-ರುವಾಗ
ಸೂಕ್ಷ್ಮ-ವಾಗಿ-ರುವು-ದ-ರಿಂದ
ಸೂಕ್ಷ್ಮ-ವಾಗಿ-ರುವುದು
ಸೂಕ್ಷ್ಮ-ವಾಗಿ-ರು-ವುವು
ಸೂಕ್ಷ್ಮ-ವಾಗುತ್ತ
ಸೂಕ್ಷ್ಮ-ವಾಗುತ್ತದೆ
ಸೂಕ್ಷ್ಮ-ವಾ-ಗು-ವುದು
ಸೂಕ್ಷ್ಮ-ವಾ-ಗು-ವುವು
ಸೂಕ್ಷ್ಮ-ವಾದ
ಸೂಕ್ಷ್ಮ-ವಾದ-ವು-ಗಳು
ಸೂಕ್ಷ್ಮ-ವಾ-ದಷ್ಟೂ
ಸೂಕ್ಷ್ಮ-ವಾದುದು
ಸೂಕ್ಷ್ಮ-ವಿಶ್ವ-ದಿಂದ
ಸೂಕ್ಷ್ಮ-ವಿಷ-ಯತ್ವಂ
ಸೂಕ್ಷ್ಮ-ವಿಷಯಾ
ಸೂಕ್ಷ್ಮವು
ಸೂಕ್ಷ್ಮವೂ
ಸೂಕ್ಷ್ಮವೇ
ಸೂಕ್ಷ್ಮವ್ಯವ-ಹಿತ-ವಿಪ್ರ-ಕೃಷ್ಣಜ್ಞಾನಮ್
ಸೂಕ್ಷ್ಮ-ಶಕ್ತಿ-ಗ-ಳನ್ನು
ಸೂಕ್ಷ್ಮ-ಶಕ್ತಿ-ಗಳು
ಸೂಕ್ಷ್ಮ-ಶಕ್ತಿಯ
ಸೂಕ್ಷ್ಮ-ಶರೀರ
ಸೂಕ್ಷ್ಮ-ಶರೀರವು
ಸೂಕ್ಷ್ಮ-ಸೂಕ್ಷ್ಮ-ವಾಗಿ
ಸೂಕ್ಷ್ಮಸ್ಥಿತಿ
ಸೂಕ್ಷ್ಮಸ್ಥಿತಿಯ
ಸೂಕ್ಷ್ಮಸ್ಥಿತಿ-ಯಲ್ಲಿ
ಸೂಕ್ಷ್ಮಸ್ಥಿತಿ-ಯಿಂದ
ಸೂಕ್ಷ್ಮಸ್ಪಂದನ-ವನ್ನು
ಸೂಕ್ಷ್ಮಸ್ಪಂದನ-ವೊಂದೇ
ಸೂಕ್ಷ್ಮಾ
ಸೂಕ್ಷ್ಮಾಃ
ಸೂಕ್ಷ್ಮಾ-ತಿ-ಸೂಕ್ಷ್ಮ
ಸೂಕ್ಷ್ಮಾ-ವಸ್ಥೆ
ಸೂಕ್ಷ್ಮಾ-ವಸ್ಥೆಗೆ
ಸೂಕ್ಷ್ಮಾ-ವಸ್ಥೆ-ಯಲ್ಲಿ-ರ-ಬೇಕು
ಸೂಕ್ಷ್ಮಾ-ವಸ್ಥೆ-ಯಲ್ಲಿ-ರು-ವಾ-ಗಲೇ
ಸೂಚನೆ
ಸೂಚ-ನೆಗೆ
ಸೂಚನೆ-ಯನ್ನು
ಸೂಚನೆ-ಯೊಂದೇ
ಸೂಚಿತ-ವಾ-ಗಿ-ರುವ
ಸೂಚಿಸಿದ್ದರು
ಸೂಚಿ-ಸುತ್ತದೆ
ಸೂಚಿಸುತ್ತವೆ
ಸೂಚಿ-ಸುವ
ಸೂಚಿ-ಸು-ವಂತೆ
ಸೂಚಿ-ಸು-ವುದು
ಸೂಚಿಸು-ವುದೆಂಬು-ದನ್ನು
ಸೂಚಿ-ಸು-ವುವೊ
ಸೂಜಿ
ಸೂಜಿ-ಯನ್ನು
ಸೂತ್ರ
ಸೂತ್ರ-ಗಳಲ್ಲಿ
ಸೂತ್ರ-ಗಳು
ಸೂತ್ರತೆ
ಸೂತ್ರದ
ಸೂತ್ರ-ದಲ್ಲಿ
ಸೂತ್ರ-ಭಾಷೆ
ಸೂತ್ರ-ವನ್ನು
ಸೂತ್ರವು
ಸೂರೆ
ಸೂರೆ-ಗೊಳ್ಳಲು
ಸೂರೆ-ಮಾಡು-ವು-ದಕ್ಕೆ
ಸೂರ್ಯ
ಸೂರ್ಯ-ಚಂದ್ರ
ಸೂರ್ಯ-ಚಂದ್ರ-ನಕ್ಷತ್ರ-ಗಳಲ್ಲಿ
ಸೂರ್ಯ-ಚಂದ್ರ-ನಕ್ಷತ್ರ-ಗಳೆಲ್ಲ
ಸೂರ್ಯ-ಚಂದ್ರರ
ಸೂರ್ಯ-ಚಂದ್ರ-ರಿ-ಗಿಂತ
ಸೂರ್ಯ-ಚಂದ್ರರು
ಸೂರ್ಯ-ಚಂದ್ರಾ-ವಳಿ-ಗಳಿಂದ
ಸೂರ್ಯನ
ಸೂರ್ಯ-ನನ್ನು
ಸೂರ್ಯ-ನಲ್ಲಿ
ಸೂರ್ಯ-ನಲ್ಲಿದ್ದಿತೋ
ಸೂರ್ಯ-ನಲ್ಲಿ-ರು-ವಷ್ಟೇ
ಸೂರ್ಯ-ನ-ವರೆಗೆ
ಸೂರ್ಯ-ನಾ-ಗು-ವುದು
ಸೂರ್ಯ-ನಿಂದ
ಸೂರ್ಯ-ನಿಗೂ
ಸೂರ್ಯ-ನಿಗೆ
ಸೂರ್ಯನು
ಸೂರ್ಯನೆ
ಸೂರ್ಯ-ನೆ-ಡೆಗೆ
ಸೂರ್ಯನೇ
ಸೂರ್ಯರ
ಸೂರ್ಯ-ಲೋ-ಕಕ್ಕೆ
ಸೂರ್ಯ-ವಂಶದ
ಸೂರ್ಯೇ
ಸೂರ್ಯೋದಯ
ಸೂರ್ಯೋದಯ-ವನ್ನು
ಸೃಜಿಸಿ-ದರು
ಸೃಷ್ಟಿ
ಸೃಷ್ಟಿ-ಕರ್ತ
ಸೃಷ್ಟಿ-ಕರ್ತನ
ಸೃಷ್ಟಿ-ಕರ್ತ-ನನ್ನು
ಸೃಷ್ಟಿ-ಕರ್ತ-ನಲ್ಲ
ಸೃಷ್ಟಿ-ಕರ್ತ-ನಾದ
ಸೃಷ್ಟಿ-ಕರ್ತನು
ಸೃಷ್ಟಿಗೆ
ಸೃಷ್ಟಿ-ತತ್ವ
ಸೃಷ್ಟಿ-ಪಾಲಕ
ಸೃಷ್ಟಿ-ಪಾಲಕ-ನಾದ
ಸೃಷ್ಟಿ-ಪಾಲಕ-ನಿಲ್ಲ
ಸೃಷ್ಟಿ-ಮಾಡಿದ್ದರೆ
ಸೃಷ್ಟಿಯ
ಸೃಷ್ಟಿ-ಯನ್ನು
ಸೃಷ್ಟಿ-ಯನ್ನೆಲ್ಲಾ
ಸೃಷ್ಟಿ-ಯಲ್ಲಿ
ಸೃಷ್ಟಿ-ಯಲ್ಲಿ-ರುವ
ಸೃಷ್ಟಿ-ಯಲ್ಲೆಲ್ಲ
ಸೃಷ್ಟಿ-ಯಾ-ಗಿದೆ
ಸೃಷ್ಟಿ-ಯಾ-ಗುತ್ತದೆ
ಸೃಷ್ಟಿ-ಯಾಗು-ವು-ದನ್ನು
ಸೃಷ್ಟಿ-ಯಾ-ಗು-ವುದು
ಸೃಷ್ಟಿ-ಯಾದ
ಸೃಷ್ಟಿ-ಯಾ-ದವು
ಸೃಷ್ಟಿಯು
ಸೃಷ್ಟಿಯೂ
ಸೃಷ್ಟಿ-ಯೆಲ್ಲ
ಸೃಷ್ಟಿ-ಯೆಲ್ಲಾ
ಸೃಷ್ಟಿ-ರಕ್ಷ-ಕನ
ಸೃಷ್ಟಿ-ವಾದ-ವನ್ನು
ಸೃಷ್ಟಿವ್ಯೂಹವೂ
ಸೃಷ್ಟಿ-ಶಕ್ತಿಯು
ಸೃಷ್ಟಿ-ಸ-ಬಲ್ಲನು
ಸೃಷ್ಟಿ-ಸಬೇ-ಕಿತ್ತು
ಸೃಷ್ಟಿ-ಸ-ಲಾರ
ಸೃಷ್ಟಿ-ಸ-ಲಾರದು
ಸೃಷ್ಟಿ-ಸ-ಲಾರೆ-ವೆಂಬು-ದನ್ನು
ಸೃಷ್ಟಿ-ಸಲಿಕ್ಕೆ
ಸೃಷ್ಟಿ-ಸ-ಲಿಲ್ಲ
ಸೃಷ್ಟಿ-ಸಲು
ಸೃಷ್ಟಿ-ಸಲ್ಪಟ್ಟದ್ದಲ್ಲ
ಸೃಷ್ಟಿಸಿ
ಸೃಷ್ಟಿ-ಸಿ-ಕೊಳ್ಳ-ಬಹುದು
ಸೃಷ್ಟಿ-ಸಿ-ಕೊಳ್ಳುತ್ತೇವೆ
ಸೃಷ್ಟಿ-ಸಿದ
ಸೃಷ್ಟಿ-ಸಿ-ದನು
ಸೃಷ್ಟಿ-ಸಿ-ದರು
ಸೃಷ್ಟಿ-ಸಿ-ದ-ವನೇ
ಸೃಷ್ಟಿ-ಸಿ-ದಾ-ತನ
ಸೃಷ್ಟಿ-ಸಿ-ದು-ದನ್ನು
ಸೃಷ್ಟಿ-ಸಿದೆ
ಸೃಷ್ಟಿ-ಸಿದ್ದಾರೆ
ಸೃಷ್ಟಿ-ಸಿ-ರುವನು
ಸೃಷ್ಟಿ-ಸಿ-ರು-ವರು
ಸೃಷ್ಟಿ-ಸಿಲ್ಲ
ಸೃಷ್ಟಿ-ಸಿವೆ
ಸೃಷ್ಟಿಸು
ಸೃಷ್ಟಿ-ಸುತ್ತಾನೆ
ಸೃಷ್ಟಿ-ಸುತ್ತಿ-ರುವನು
ಸೃಷ್ಟಿ-ಸುತ್ತಿ-ರು-ವರು
ಸೃಷ್ಟಿ-ಸುತ್ತಿವೆ
ಸೃಷ್ಟಿ-ಸುವ
ಸೃಷ್ಟಿ-ಸು-ವನು
ಸೃಷ್ಟಿ-ಸು-ವರು
ಸೃಷ್ಟಿ-ಸು-ವ-ವ-ನಿಲ್ಲ
ಸೃಷ್ಟಿ-ಸು-ವ-ವನು
ಸೃಷ್ಟಿ-ಸು-ವು-ದಕ್ಕೆ
ಸೃಷ್ಟಿ-ಸು-ವುದು
ಸೆರೆ
ಸೆರೆ-ಮನೆ
ಸೆರೆ-ಮನೆ-ಯಾಗದೆ
ಸೆರೆ-ಮ-ನೆಯೆ
ಸೆರೆ-ಯನ್ನು
ಸೆರೆ-ಯಿಂದ
ಸೆರೆ-ಹಿಡಿದು
ಸೆಳೆ-ತಕ್ಕೆ
ಸೆಳೆ-ದರೆ
ಸೆಳೆದು
ಸೆಳೆದು-ಕೊಂಡಂತೆ
ಸೆಳೆದು-ಕೊಂಡು
ಸೆಳೆದು-ಕೊಳ್ಳ-ಬೇಕು
ಸೆಳೆದು-ಕೊಳ್ಳಿ
ಸೆಳೆದು-ಕೊಳ್ಳಿ-ಅ-ನಂತರ
ಸೆಳೆದು-ಕೊಳ್ಳುವ
ಸೆಳೆದು-ಕೊಳ್ಳು-ವಾಗ
ಸೆಳೆದು-ಕೊಳ್ಳು-ವು-ದ-ರಿಂದ
ಸೆಳೆದು-ಕೊಳ್ಳು-ವುದು
ಸೆಳೆ-ಯದೆ
ಸೆಳೆಯ-ಬೇಕಾ-ದರೆ
ಸೆಳೆಯಲೆತ್ನಿಸು
ಸೆಳೆಯಿ-ತೆಂದು
ಸೆಳೆಯುತ್ತದೆ
ಸೆಳೆಯುತ್ತವೆ
ಸೆಳೆ-ಯುತ್ತಿದ್ದರೆ
ಸೆಳೆಯುತ್ತಿ-ರು-ವುವೋ
ಸೆಳೆ-ಯುವ
ಸೆಳೆ-ಯುವುದು
ಸೆಳೆ-ಯು-ವುವು
ಸೇದು-ವಾಗ
ಸೇನೆಗೆ
ಸೇಬಿನ
ಸೇಬು
ಸೇಬು-ಗಳೂ
ಸೇರ
ಸೇರದ
ಸೇರ-ದ-ವರೊ-ಡನೆ
ಸೇರ-ಬಹುದು
ಸೇರ-ಬಾ-ರದುಆ
ಸೇರ-ಬೇಕು
ಸೇರ-ಬೇಕೆಂದಿ-ರುವ
ಸೇರ-ಬೇಕೆಂದು
ಸೇರ-ಬೇಕೆಂದೂ
ಸೇರ-ಲಾರ
ಸೇರ-ಲಾರರು
ಸೇರ-ಲಾರವು
ಸೇರಲು
ಸೇರಲೇ
ಸೇರ-ಲೇ-ಬೇಕಾ-ಗಿದೆ
ಸೇರಿ
ಸೇರಿ-ಕೊಂಡಿದ್ದುವು
ಸೇರಿ-ಕೊಂಡಿ-ರು-ವುವೊ
ಸೇರಿ-ಕೊಂಡಿ-ರು-ವೆವು
ಸೇರಿತು
ಸೇರಿದ
ಸೇರಿ-ದ-ಮೇಲೆ
ಸೇರಿ-ದ-ವನು
ಸೇರಿ-ದ-ವ-ನೆಂದು
ಸೇರಿ-ದ-ವ-ರಿಗೆ
ಸೇರಿ-ದ-ವರು
ಸೇರಿ-ದವು
ಸೇರಿ-ದ-ವು-ಗಳು
ಸೇರಿ-ದವೆ
ಸೇರಿ-ದಾಗ
ಸೇರಿ-ದುದು
ಸೇರಿ-ದು-ದೆಂದು
ಸೇರಿ-ದುವು
ಸೇರಿದೆ
ಸೇರಿ-ದೆವು
ಸೇರಿದ್ದರೂ
ಸೇರಿದ್ದರೆ
ಸೇರಿದ್ದಲ್ಲ
ಸೇರಿದ್ದು
ಸೇರಿದ್ದುವು
ಸೇರಿಯೂ
ಸೇರಿಯೇ
ಸೇರಿ-ರ-ಬೇಕು
ಸೇರಿ-ರಲಿ
ಸೇರಿ-ರುತ್ತದೆ
ಸೇರಿ-ರುವ
ಸೇರಿ-ರುವನು
ಸೇರಿ-ರು-ವರು
ಸೇರಿ-ರುವುದು
ಸೇರಿ-ರುವುದೇ
ಸೇರಿ-ರು-ವುವೋ
ಸೇರಿ-ರು-ವೆವು
ಸೇರಿಲ್ಲ
ಸೇರಿಲ್ಲ-ವೆಂದು
ಸೇರಿವೆ
ಸೇರಿ-ಸ-ಬಹುದು
ಸೇರಿ-ಸ-ಲಾರಿರಿ
ಸೇರಿ-ಸಲು
ಸೇರಿಸಿ
ಸೇರಿ-ಸಿ-ಕೊಳ್ಳಲು
ಸೇರಿ-ಸಿ-ಕೊಳ್ಳು-ವಷ್ಟು
ಸೇರಿ-ಸಿ-ಕೊಳ್ಳು-ವು-ದಿಲ್ಲ
ಸೇರಿ-ಸಿ-ಕೊಳ್ಳು-ವುದು
ಸೇರಿ-ಸಿ-ದಂತಿ-ರುತ್ತದೆ
ಸೇರಿ-ಸಿ-ದಂತೆ
ಸೇರಿ-ಸಿ-ದರು
ಸೇರಿ-ಸಿ-ದರೂ
ಸೇರಿ-ಸಿ-ದರೆ
ಸೇರಿಸಿ-ರುವ
ಸೇರುತ್ತವೆ
ಸೇರುತ್ತವೆಯೋ
ಸೇರುತ್ತಾನೆ
ಸೇರುತ್ತಾರೆ
ಸೇರುತ್ತೀರಿ
ಸೇರುವ
ಸೇರುವ-ತನಕ
ಸೇರುವನು
ಸೇರು-ವರು
ಸೇರು-ವ-ವರೆಗೂ
ಸೇರು-ವ-ವರೆಗೆ
ಸೇರು-ವಾಗ
ಸೇರು-ವಿರಿ
ಸೇರು-ವು-ದಕ್ಕೆ
ಸೇರು-ವು-ದನ್ನು
ಸೇರು-ವು-ದ-ರಲ್ಲಿ
ಸೇರು-ವು-ದಲ್ಲ
ಸೇರು-ವು-ದಿಲ್ಲ
ಸೇರು-ವುದು
ಸೇರುವುದೆ
ಸೇರು-ವುವು
ಸೇರು-ವೆವು
ಸೇವಕ-ರನ್ನಾಗಿ
ಸೇವ-ಕರು
ಸೇವನೆ
ಸೇವನೆ-ಯಿಂದ
ಸೇವಿ-ಸ-ಬೇಕು
ಸೇವಿ-ಸಲು
ಸೇವಿಸುತ್ತಿ-ರುವ-ವನು
ಸೇವಿ-ಸುತ್ತೇವೆ
ಸೇವಿ-ಸುತ್ತೇವೆಯೋ
ಸೇವಿ-ಸುವ
ಸೇವೆ
ಸೈತಾನ-ನನ್ನು
ಸೈತಾನನಾ-ದನು
ಸೈತಾನ-ನಿಲ್ಲ
ಸೈತಾನನಿಲ್ಲದೆ
ಸೈತಾನನಿಲ್ಲವೆ
ಸೈತಾ-ನನ್ನು
ಸೈತಾನ್
ಸೈದ್ಧಾಂತಿಕ-ವಾಗಿಯೂ
ಸೈನ್ಯ
ಸೈನ್ಯದ
ಸೈರಿ-ಸು-ವು-ದಿಲ್ಲ
ಸೊಗಸಾದ
ಸೊನ್ನೆ-ಗ-ಳನ್ನು
ಸೊಳ್ಳೆ
ಸೊಳ್ಳೆಯ
ಸೊಳ್ಳೆ-ಯೊಂದು
ಸೋಣ
ಸೋತರೂ
ಸೋತೆ-ನೆಂದು
ಸೋತೆವೊ
ಸೋದರತ್ವ
ಸೋದರ-ನಂತೆ
ಸೋದರಿಕೆ-ಯಲ್ಲಿ
ಸೋಪಕ್ರಮಂ
ಸೋಪಾನ
ಸೋಮ-ರಾಜನು
ಸೋಮಾರಿ
ಸೋಮಾರಿ-ಗಳಾಗಿ
ಸೋಮಾರಿ-ಗಳಾ-ಗಿ-ರ-ಬೇಕು
ಸೋಮಾರಿ-ಗಳೋ
ಸೋಮಾರಿ-ತನ
ಸೋಮಾರಿ-ತನ-ವನ್ನು
ಸೋಮಾರಿ-ಯನ್ನಾಗಿ
ಸೋಮಾರಿ-ಯಲ್ಲ
ಸೋಮಾರಿ-ಯಾ-ದಾಗ
ಸೋಲನ್ನು
ಸೋಲನ್ನೇ
ಸೋಲು
ಸೋಲುತ್ತದೆ
ಸೋಲು-ವು-ದಿಲ್ಲ
ಸೋಲು-ವುದು
ಸೋಲು-ವುದೊ
ಸೋಲೇ
ಸೌಂದರ್ಯ
ಸೌಂದರ್ಯ-ಮಯನು
ಸೌಕರ್ಯಕ್ಕಾಗಿ
ಸೌಖ್ಯ
ಸೌಖ್ಯ-ದಲ್ಲಿ-ರುವ
ಸೌಖ್ಯವೇ
ಸೌಖ್ಯ-ವೇನೋ
ಸೌದೆ
ಸೌದೆ-ಯಲ್ಲಿ
ಸೌಭಾಗ್ಯ-ವಿದು
ಸೌಮ್ಯ
ಸೌಮ್ಯ-ವಾಗಿ
ಸೌಮ್ಯ-ವಾ-ದುದು
ಸೌಲಭ್ಯ
ಸೌಲಭ್ಯ-ಗ-ಳನ್ನು
ಸೌಲಭ್ಯ-ಗಳಿ-ವೆಯೋ
ಸೌಲಭ್ಯ-ಗಳು
ಸೌಲಭ್ಯ-ದಿಂದ
ಸೌಹಾರ್ದ
ಸೌಹಾರ್ದ-ಗಳ
ಸೌಹಾರ್ದ-ತೆಯ
ಸೌಹಾರ್ದ-ದಿಂದ
ಸೌಹಾರ್ದ-ದಿಂದಿ-ರು-ವುದು
ಸೌಹಾರ್ದ-ಭಾವ-ನೆ-ಯನ್ನು
ಸೌಹಾರ್ದ-ವನ್ನು
ಸ್ಕರ-ವಾಗಿಯೇ
ಸ್ಕಾಂಡಿ
ಸ್ಟುಯರ್ಟ್
ಸ್ತಂಭವೇ
ಸ್ತಂಭೀ-ಭೂತ-ರಾಗಿ
ಸ್ತಬ್ದ-ವಾ-ಗು-ವುದು
ಸ್ತಬ್ಧ
ಸ್ತಬ್ಧ-ಗೊಳಿ-ಸುವ
ಸ್ತಬ್ಧ-ನಾಗಿ
ಸ್ತಬ್ಧ-ವಾ-ಗು-ವುದು
ಸ್ತರ
ಸ್ತರಕ್ಕೆ
ಸ್ತರ-ಗ-ಳನ್ನೂ
ಸ್ತರ-ಗಳಲ್ಲಿ
ಸ್ತರ-ಗಳಿಗೆ
ಸ್ತರ-ಗಳಿವೆ
ಸ್ತರ-ದಲ್ಲಿ
ಸ್ತರ-ದಲ್ಲಿ-ರುವ
ಸ್ತರ-ವಾ-ಗು-ವುದು
ಸ್ತರವೂ
ಸ್ತಾಪ
ಸ್ತುತಿ-ನಿಂದೆ-ಗ-ಳನ್ನು
ಸ್ತುತ್ಯಾರ್ಹ-ವಾ-ಗು-ವುವು
ಸ್ತೋತ್ರ-ರೂಪ-ವಾದ
ಸ್ತೋಮವೆ
ಸ್ತ್ರೀ
ಸ್ತ್ರೀಪುರುಷ
ಸ್ತ್ರೀಪುರುಷ-ಬಾಲ-ಕ-ರನ್ನೂ
ಸ್ತ್ರೀಪುರುಷ-ಬಾಲರ
ಸ್ತ್ರೀಪುರುಷರ
ಸ್ತ್ರೀಪುರುಷ-ರಂತೆ
ಸ್ತ್ರೀಪುರುಷ-ರನ್ನು
ಸ್ತ್ರೀಪುರುಷ-ರಲ್ಲಿ
ಸ್ತ್ರೀಪುರುಷ-ರಿಗೂ
ಸ್ತ್ರೀಪುರುಷರು
ಸ್ತ್ರೀಪುರುಷರೂ
ಸ್ತ್ರೀಯ
ಸ್ತ್ರೀಯರ
ಸ್ತ್ರೀಯ-ರನ್ನು
ಸ್ತ್ರೀಯ-ರಲ್ಲಿ
ಸ್ತ್ರೀಯ-ರಿಗೂ
ಸ್ತ್ರೀಯರು
ಸ್ತ್ರೀಯು
ಸ್ತ್ರೀಯೂ
ಸ್ತ್ರೀಯೊ
ಸ್ಥಳ
ಸ್ಥಳಕ್ಕೆ
ಸ್ಥಳಕ್ಕೇ
ಸ್ಥಳ-ಗಳಿಂದ
ಸ್ಥಳದ
ಸ್ಥಳ-ದಂತೆ
ಸ್ಥಳ-ದಲ್ಲಾ-ದರೂ
ಸ್ಥಳ-ದಲ್ಲಿ
ಸ್ಥಳ-ದಿಂದ
ಸ್ಥಳ-ವನ್ನು
ಸ್ಥಳ-ವಾಗಿದೆ
ಸ್ಥಳ-ವಿದೆ
ಸ್ಥಳ-ವಿಲ್ಲ
ಸ್ಥಳವು
ಸ್ಥಳವೂ
ಸ್ಥಳ-ವೆಲ್ಲಿ-ರುವುದು
ಸ್ಥಳವೇ
ಸ್ಥಳವ್ಯತ್ಯಾಸ-ದಿಂದ
ಸ್ಥಾನ-ಗಳು
ಸ್ಥಾನಚ್ಯು-ತಿಯ
ಸ್ಥಾನ-ದಲ್ಲಿ
ಸ್ಥಾನ-ದಿಂದ
ಸ್ಥಾನ-ವನ್ನು
ಸ್ಥಾನ-ವನ್ನು-ಕೊಡು-ವರು
ಸ್ಥಾನ-ವಾದ
ಸ್ಥಾನ-ವಿದೆ
ಸ್ಥಾನ-ವಿಲ್ಲ
ಸ್ಥಾನ್ಯುಪನಿ-ಮಂತ್ರಣೇ
ಸ್ಥಾಪಕ
ಸ್ಥಾಪನೆ
ಸ್ಥಾಪಿತ-ವಾಗಿ-ರ-ಲಿಲ್ಲ
ಸ್ಥಾಪಿತ-ವಾದ
ಸ್ಥಾಪಿಸ
ಸ್ಥಾಪಿಸ-ಬಲ್ಲರು
ಸ್ಥಾಪಿಸ-ಬೇಕು
ಸ್ಥಾಪಿಸ-ಲಾರೆವು
ಸ್ಥಾಪಿ-ಸಲು
ಸ್ಥಾಪಿ-ಸಿದ
ಸ್ಥಾಪಿಸಿ-ರು-ವರು
ಸ್ಥಾಪಿ-ಸು-ವು-ದಕ್ಕೆ
ಸ್ಥಾಪಿಸು-ವುದು
ಸ್ಥಾಯಿತ್ವ
ಸ್ಥಿತಿ
ಸ್ಥಿತಿ-ಗತಿ-ಗ-ಳನ್ನು
ಸ್ಥಿತಿ-ಗಳ
ಸ್ಥಿತಿ-ಗ-ಳನ್ನು
ಸ್ಥಿತಿ-ಗಳಲ್ಲಿ
ಸ್ಥಿತಿ-ಗಳಿವೆ
ಸ್ಥಿತಿ-ಗಳು
ಸ್ಥಿತಿ-ಗಳೆ
ಸ್ಥಿತಿ-ಗಾಗಿ
ಸ್ಥಿತಿಗೂ
ಸ್ಥಿತಿಗೆ
ಸ್ಥಿತಿ-ಗೆಲ್ಲ
ಸ್ಥಿತಿ-ಗೇ-ರ-ಬೇಕು
ಸ್ಥಿತಿ-ಪರಿ-ಣಾಮ
ಸ್ಥಿತಿ-ಬಂಧಿನೀ
ಸ್ಥಿತಿಯ
ಸ್ಥಿತಿ-ಯನ್ನು
ಸ್ಥಿತಿ-ಯನ್ನೂ
ಸ್ಥಿತಿ-ಯನ್ನೆ
ಸ್ಥಿತಿ-ಯಲ್ಲವೆ
ಸ್ಥಿತಿ-ಯಲ್ಲಿ
ಸ್ಥಿತಿ-ಯಲ್ಲಿಟ್ಟಿ-ರು-ವುದು
ಸ್ಥಿತಿ-ಯಲ್ಲಿ-ಡ-ಬೇಕು
ಸ್ಥಿತಿ-ಯಲ್ಲಿದೆ
ಸ್ಥಿತಿ-ಯಲ್ಲಿದ್ದನು
ಸ್ಥಿತಿ-ಯಲ್ಲಿದ್ದರೂ
ಸ್ಥಿತಿ-ಯಲ್ಲಿದ್ದು
ಸ್ಥಿತಿ-ಯಲ್ಲಿಯೂ
ಸ್ಥಿತಿ-ಯಲ್ಲಿ-ರುತ್ತವೆ
ಸ್ಥಿತಿ-ಯಲ್ಲಿ-ರುವ
ಸ್ಥಿತಿ-ಯಲ್ಲಿ-ರು-ವರು
ಸ್ಥಿತಿ-ಯಲ್ಲಿ-ರು-ವರೊ
ಸ್ಥಿತಿ-ಯಲ್ಲಿ-ರುವ-ವರು
ಸ್ಥಿತಿ-ಯಲ್ಲಿ-ರುವಾಗ
ಸ್ಥಿತಿ-ಯಲ್ಲಿ-ರುವು-ದನ್ನು
ಸ್ಥಿತಿ-ಯಲ್ಲಿ-ರುವುದು
ಸ್ಥಿತಿ-ಯಲ್ಲಿ-ರುವುದೊ
ಸ್ಥಿತಿ-ಯಲ್ಲೂ
ಸ್ಥಿತಿ-ಯಲ್ಲೇ
ಸ್ಥಿತಿ-ಯ-ವರೆ-ವಿಗೂ
ಸ್ಥಿತಿ-ಯಾಗಿ-ರ-ಬೇಕು
ಸ್ಥಿತಿ-ಯಾಗಿ-ರ-ಲಾರದು
ಸ್ಥಿತಿ-ಯಾ-ದರೆ
ಸ್ಥಿತಿ-ಯಿಂದ
ಸ್ಥಿತಿಯು
ಸ್ಥಿತಿಯೆ
ಸ್ಥಿತಿಯೇ
ಸ್ಥಿತಿ-ಯೊಂದು
ಸ್ಥಿತೌ
ಸ್ಥಿರ
ಸ್ಥಿರ-ಗೊಳಿ-ಸ-ಬಹುದು
ಸ್ಥಿರತೆ
ಸ್ಥಿರ-ತೆ-ಯನ್ನು
ಸ್ಥಿರ-ನಾ-ಗಿ-ರುವನು
ಸ್ಥಿರ-ನಾದ
ಸ್ಥಿರ-ವಲ್ಲ
ಸ್ಥಿರ-ವಾಗಿ
ಸ್ಥಿರ-ವಾಗಿ-ರ-ಬೇಕು
ಸ್ಥಿರ-ವಾಗಿ-ರು-ವಿರಿ
ಸ್ಥಿರ-ವಾ-ಗಿಲ್ಲ
ಸ್ಥಿರ-ವಾಗಿಲ್ಲದೆ
ಸ್ಥಿರ-ವಾ-ಗು-ವುದು
ಸ್ಥಿರ-ವಾದ
ಸ್ಥಿರ-ವಾದ-ಮೇಲೆ
ಸ್ಥಿರ-ವಾದ-ವನ
ಸ್ಥಿರ-ವಾದು-ದೆಂದು
ಸ್ಥಿರವೂ
ಸ್ಥಿರ-ಸುಖ-ಮಾಸ-ನಮ್
ಸ್ಥಿರಾಸನ-ವೆಂದರೆ
ಸ್ಥೂಲ
ಸ್ಥೂಲಗ್ರಹಣ
ಸ್ಥೂಲ-ತಮ
ಸ್ಥೂಲ-ತರ
ಸ್ಥೂಲದ
ಸ್ಥೂಲ-ದಿಂದ
ಸ್ಥೂಲ-ದಿಂದಾಗಿ-ರುವನು
ಸ್ಥೂಲ-ದೃಷ್ಟಿ-ಯಿಂದ
ಸ್ಥೂಲ-ದೇಹದ
ಸ್ಥೂಲ-ದೇಹ-ದಲ್ಲಿ
ಸ್ಥೂಲ-ಭಾವ-ನೆ-ಗ-ಳನ್ನು
ಸ್ಥೂಲ-ಭಾ-ವವೇ
ಸ್ಥೂಲ-ರೂಪ
ಸ್ಥೂಲ-ರೂಪದ
ಸ್ಥೂಲ-ರೂಪ-ದಲ್ಲಿ
ಸ್ಥೂಲ-ರೂಪ-ವನ್ನು
ಸ್ಥೂಲ-ರೂಪ-ವಾ-ಗಿ-ರು-ವಂತೆ
ಸ್ಥೂಲ-ವನ್ನು
ಸ್ಥೂಲ-ವಸ್ತು
ಸ್ಥೂಲ-ವಸ್ತು-ಗ-ಳನ್ನು
ಸ್ಥೂಲ-ವಸ್ತು-ಗಳಿಗೆ
ಸ್ಥೂಲ-ವಸ್ತು-ಗಳು
ಸ್ಥೂಲ-ವಸ್ತು-ವಿನ
ಸ್ಥೂಲ-ವಸ್ತುವೇ
ಸ್ಥೂಲ-ವಾಗಿ
ಸ್ಥೂಲ-ವಾಗಿದೆ
ಸ್ಥೂಲ-ವಾಗಿ-ರುವ
ಸ್ಥೂಲ-ವಾಗಿ-ರುವುದು
ಸ್ಥೂಲ-ವಾಗಿಲ್ಲದ
ಸ್ಥೂಲ-ವಾಗುತ್ತ
ಸ್ಥೂಲ-ವಾಗುತ್ತಾ
ಸ್ಥೂಲ-ವಾಗು-ವು-ದಕ್ಕೆ
ಸ್ಥೂಲ-ವಾಗು-ವುದನ್ನೇ
ಸ್ಥೂಲ-ವಾ-ಗು-ವುದು
ಸ್ಥೂಲ-ವಾದ
ಸ್ಥೂಲ-ವಾದುವು
ಸ್ಥೂಲವು
ಸ್ಥೂಲಸ್ಥಿತಿ
ಸ್ಥೂಲಸ್ವ-ರೂಪ-ಸಕ್ಷ್ಮಾನ್ವ-ಯಾರ್ಥವತ್ತ್ವ-ಸಂಯಮಾದ್
ಸ್ಥೈರ್ಯ-ಗಳು
ಸ್ಥೈರ್ಯರ್ಮ
ಸ್ನಾನ
ಸ್ನಾಯು-ಕೇಂದ್ರ
ಸ್ನಾಯು-ಕೇಂದ್ರವು
ಸ್ನಾಯು-ಗತಿ
ಸ್ನಾಯುಜಾಲ
ಸ್ನಾಯುಜಾಲಕ್ಕೆ
ಸ್ನೇಹ
ಸ್ನೇಹಿತ
ಸ್ನೇಹಿತ-ನಂತೆ
ಸ್ನೇಹಿತ-ನಾದ
ಸ್ನೇಹಿತನೆ
ಸ್ನೇಹಿತ-ನೆಂದು
ಸ್ನೇಹಿತರ
ಸ್ನೇಹಿತರು
ಸ್ನೇಹಿತರೊ
ಸ್ನೇಹಿತ-ರೊಂದಿಗೆ
ಸ್ಪಂದನ
ಸ್ಪಂದನಕ್ಕೆ
ಸ್ಪಂದನ-ಗ-ಳನ್ನು
ಸ್ಪಂದನ-ಗಳಾಚೆ
ಸ್ಪಂದನ-ಗಳು
ಸ್ಪಂದನ-ಗಳುಳ್ಳ
ಸ್ಪಂದನದ
ಸ್ಪಂದನ-ದಲ್ಲಿದೆ
ಸ್ಪಂದನ-ದಲ್ಲಿ-ರುವ
ಸ್ಪಂದನ-ದಿಂದ
ಸ್ಪಂದನ-ದೊಂದಿಗೆ
ಸ್ಪಂದನ-ವನ್ನು
ಸ್ಪಂದನ-ವಾಗಿ-ರ-ಬಹುದು
ಸ್ಪಂದನ-ವಾ-ಗು-ವುವು
ಸ್ಪಂದನ-ವಿದೆ
ಸ್ಪಂದನವು
ಸ್ಪಂದನವೂ
ಸ್ಪಂದನ-ವೇನೋ
ಸ್ಪಂದನಾ
ಸ್ಪಂದನೆ
ಸ್ಪಂದನೆ-ಯನ್ನು
ಸ್ಪಂದಿ-ಸದೆ
ಸ್ಪಂದಿ-ಸುತ್ತೀ-ರೇನು
ಸ್ಪಂದಿ-ಸುವ
ಸ್ಪಂದಿ-ಸುವನು
ಸ್ಪಂದಿ-ಸು-ವುದು
ಸ್ಪಂದಿ-ಸು-ವುವು
ಸ್ಪರ್ಧಿ-ಸುವ
ಸ್ಪರ್ಧೆ
ಸ್ಪರ್ಧೆ-ಗಳೆಲ್ಲ
ಸ್ಪರ್ಧೆಯ
ಸ್ಪರ್ಧೆ-ಯನ್ನು
ಸ್ಪರ್ಧೆಯು
ಸ್ಪರ್ಶ
ಸ್ಪರ್ಶ-ದಿಂದ
ಸ್ಪಷ್ಟ
ಸ್ಪಷ್ಟ-ಪಡಿ-ಸು-ವುದು
ಸ್ಪಷ್ಟ-ಮಾಡಿ-ರು-ವರು
ಸ್ಪಷ್ಟ-ವಾಗಿ
ಸ್ಪಷ್ಟ-ವಾಗಿದೆ
ಸ್ಪಷ್ಟ-ವಾಗಿ-ರ-ಬೇಕು
ಸ್ಪಷ್ಟ-ವಾ-ಗಿಲ್ಲ
ಸ್ಪಷ್ಟ-ವಾಗುತ್ತ
ಸ್ಪಷ್ಟ-ವಾಗುತ್ತದೆ
ಸ್ಪಷ್ಟ-ವಾ-ಗು-ವುದು
ಸ್ಪಷ್ಟ-ವಾದ
ಸ್ಪಷ್ಟ-ವಾ-ದರೆ
ಸ್ಪೆನ್ಸರ್
ಸ್ಪೆನ್ಸ್ರ್
ಸ್ಫಟಿಕ
ಸ್ಫಟಿಕ-ಮಣಿ
ಸ್ಫಟಿಕ-ಮಣಿಯ
ಸ್ಫಟಿಕ-ಮಣಿ-ಯಂತೆ
ಸ್ಫಟಿಕ-ಮಣಿಯೂ
ಸ್ಫರ್ಧೆ-ಗಳು
ಸ್ಫೂರ್ತಿ
ಸ್ಫೂರ್ತಿ-ದಾಯ-ಕ-ವಾಗಿದೆ
ಸ್ಫೂರ್ತಿಯ
ಸ್ಫೂರ್ತಿ-ಯನ್ನು
ಸ್ಫೂರ್ತಿ-ಯನ್ನೂ
ಸ್ಫೂರ್ತಿ-ಯನ್ನೆಲ್ಲಾ
ಸ್ಫೂರ್ತಿ-ಯಾ-ಗು-ವುದು
ಸ್ಫೂರ್ತಿ-ಯಿಂದ
ಸ್ಫೂರ್ತಿಯು
ಸ್ಮರಿ-ಸು-ವನು
ಸ್ಮಶಾನ
ಸ್ಮಾತ್ತಾಗಿ
ಸ್ಮೃತಿ
ಸ್ಮೃತಿಃ
ಸ್ಮೃತಿ-ಪರಿ-ಶುದ್ಧೌಸ್ವ-ರೂಪ-ಶೂನ್ಯೇ-ವಾರ್ಥ-ಮಾತ್ರ
ಸ್ಮೃತಿಯ
ಸ್ಮೃತಿ-ಯನ್ನು
ಸ್ಮೃತಿಯು
ಸ್ಮೃತಿ-ಸಂಕ-ರವೇ
ಸ್ಮೃತಿ-ಸಂಕರಶ್ಚ
ಸ್ಮೃತಿ-ಸಂಸ್ಕಾರ-ಯೋರೇಕ-ರೂಪತ್ವಾತ್
ಸ್ಯಾಮು-ಯಲ್ಲನು
ಸ್ರವಿ-ಸು-ವುದು
ಸ್ರಾರು
ಸ್ವಂತ
ಸ್ವಂತ-ವಾಗಿ
ಸ್ವಕೇಂದ್ರೀ
ಸ್ವಚ್ಛಂದ
ಸ್ವಚ್ಛಂದ-ವಾಗಿ
ಸ್ವಚ್ಛ-ವಾದ
ಸ್ವತಂತ್ರ
ಸ್ವತಂತ್ರ-ನಾಗಿ-ರಲಿ
ಸ್ವತಂತ್ರ-ನಾ-ಗಿ-ರುವನು
ಸ್ವತಂತ್ರ-ನಾದ-ವನು
ಸ್ವತಂತ್ರನು
ಸ್ವತಂತ್ರನೂ
ಸ್ವತಂತ್ರ-ರಲ್ಲ-ವೆಂದು
ಸ್ವತಂತ್ರ-ರಾ-ಗಲು
ಸ್ವತಂತ್ರ-ರಾಗಿ
ಸ್ವತಂತ್ರ-ರಾಗಿದ್ದರೆ
ಸ್ವತಂತ್ರ-ರಾಗಿದ್ದೀರಿ
ಸ್ವತಂತ್ರ-ರಾಗಿದ್ದು
ಸ್ವತಂತ್ರ-ರಾಗಿ-ರ-ಲಿಲ್ಲ-ವೆಂದು
ಸ್ವತಂತ್ರ-ರಾಗಿ-ರು-ವರು
ಸ್ವತಂತ್ರ-ರಾಗು-ವಿರಿ
ಸ್ವತಂತ್ರ-ರಾದ
ಸ್ವತಂತ್ರರು
ಸ್ವತಂತ್ರ-ರೆಂದು
ಸ್ವತಂತ್ರ-ರೆಂಬ
ಸ್ವತಂತ್ರ-ವಲ್ಲ
ಸ್ವತಂತ್ರ-ವಲ್ಲ-ವೆಂಬುದು
ಸ್ವತಂತ್ರ-ವಾಗ-ಲಾರದು
ಸ್ವತಂತ್ರ-ವಾಗಿದೆ
ಸ್ವತಂತ್ರ-ವಾಗಿ-ರದು
ಸ್ವತಂತ್ರ-ವಾ-ಗಿ-ರುವುದು
ಸ್ವತಂತ್ರ-ವಾದ
ಸ್ವತಂತ್ರ-ವಾದು-ದಕ್ಕೆ
ಸ್ವತಂತ್ರ-ವಾದುದು
ಸ್ವತಂತ್ರ್ಯ-ರೆಂದು
ಸ್ವತಂವಾಗಿ
ಸ್ವತಃ
ಸ್ವತಃಸಿದ್ಧ-ವಾ-ಗಿ-ರುವುದೊ
ಸ್ವತ್ತಾ-ಗಿದೆ
ಸ್ವಪ್ನ
ಸ್ವಪ್ನದ
ಸ್ವಪ್ನ-ದಂತೆ
ಸ್ವಪ್ನ-ದಲ್ಲಿ
ಸ್ವಪ್ನ-ದಲ್ಲಿ-ರುವ
ಸ್ವಪ್ನ-ದಲ್ಲಿ-ರುವಾಗ
ಸ್ವಪ್ನ-ದಿಂದ
ಸ್ವಪ್ನ-ವನ್ನು
ಸ್ವಪ್ನ-ವಲ್ಲ
ಸ್ವಪ್ನ-ವಲ್ಲದ
ಸ್ವಪ್ನ-ವಾಗದೆ
ಸ್ವಪ್ನ-ವಿದ್ರಾಜ್ಞಾನಾ-ಲಂಬನಂ
ಸ್ವಪ್ನವು
ಸ್ವಪ್ನ-ವೆಂದು
ಸ್ವಪ್ನ-ವೆಂಬ
ಸ್ವಪ್ನ-ವೆನ್ನು-ವುದು
ಸ್ವಪ್ನ-ವೆಲ್ಲ
ಸ್ವಪ್ನವೇ
ಸ್ವಪ್ನಾ-ವಸ್ಥೆ
ಸ್ವಪ್ರ-ಕಾಶವೂ
ಸ್ವಪ್ರಜ್ಞೆ-ಯಿಂದ
ಸ್ವಪ್ರ-ಯತ್ನ-ದಿಂದ
ಸ್ವಬುದ್ಧಿ
ಸ್ವಭಾ
ಸ್ವಭಾವ
ಸ್ವಭಾ-ವ-ಅ-ವನು
ಸ್ವಭಾವಕ್ಕೂ
ಸ್ವಭಾ-ವಕ್ಕೆ
ಸ್ವಭಾ-ವ-ಗಳಿವೆ
ಸ್ವಭಾ-ವತಃ
ಸ್ವಭಾ-ವದ
ಸ್ವಭಾ-ವ-ದಂತೆ
ಸ್ವಭಾ-ವ-ದಲ್ಲಲ್ಲ
ಸ್ವಭಾ-ವ-ದಲ್ಲಿ
ಸ್ವಭಾ-ವ-ದಲ್ಲೆ
ಸ್ವಭಾ-ವ-ದ-ವನಿ-ರುವನು
ಸ್ವಭಾ-ವ-ದ-ವರ
ಸ್ವಭಾ-ವ-ದ-ವರಿ-ರು-ವರು
ಸ್ವಭಾ-ವ-ದಿಂದ
ಸ್ವಭಾ-ವ-ದಿಂದಲೇ
ಸ್ವಭಾ-ವನೆ
ಸ್ವಭಾ-ವ-ವ-ದಲ್ಲಿ
ಸ್ವಭಾವ-ವನ್ನು
ಸ್ವಭಾ-ವ-ವನ್ನೂ
ಸ್ವಭಾ-ವ-ವನ್ನೆಲ್ಲ
ಸ್ವಭಾ-ವ-ವನ್ನೆಲ್ಲಾ
ಸ್ವಭಾವ-ವಲ್ಲದೆ
ಸ್ವಭಾ-ವ-ವಾಗಿ
ಸ್ವಭಾ-ವ-ವಾ-ಗು-ವುದು
ಸ್ವಭಾ-ವ-ವಾದ
ಸ್ವಭಾ-ವ-ವಿ-ರು-ವುದು
ಸ್ವಭಾ-ವವು
ಸ್ವಭಾ-ವ-ವುಳ್ಳ
ಸ್ವಭಾ-ವ-ವುಳ್ಳದ್ದು
ಸ್ವಭಾ-ವವೂ
ಸ್ವಭಾ-ವವೆ
ಸ್ವಭಾ-ವ-ವೆಂದರೆ
ಸ್ವಭಾ-ವ-ವೆಲ್ಲ
ಸ್ವಭಾ-ವವೇ
ಸ್ವಭಾ-ವ-ವೇನು
ಸ್ವಭಾ-ವ-ವೇ-ನೆಂದರೆ
ಸ್ವಯಂ
ಸ್ವಯಂಕೃತ
ಸ್ವಯಂಚೈ-ತನ್ಯ-ನಾ-ಗಿ-ರುವನು
ಸ್ವಯಂಜ್ಯೋತಿ-ಯನ್ನು
ಸ್ವಯಂಜ್ಯೋತಿ-ರೂಪ-ನಾದ
ಸ್ವಯಂಪ್ರ-ಕಾಶ
ಸ್ವಯಂಪ್ರ-ಕಾಶತ್ವ-ವಿಲ್ಲ
ಸ್ವಯಂಪ್ರ-ಕಾಶ-ಮಾನ
ಸ್ವಯಂಪ್ರ-ಕಾಶ-ಮಾನರು
ಸ್ವಯಂಪ್ರ-ಕಾಶ-ಮಾನ-ವಾಗಿ
ಸ್ವಯಂಪ್ರ-ಕಾಶ-ಮಾನ-ವಾ-ಗಿ-ರುವು-ದು-ಅದು
ಸ್ವಯಂಪ್ರ-ಕಾಶ-ಮಾನ-ವಾದ
ಸ್ವಯಂಪ್ರ-ಕಾಶಿತ-ವಾ-ಗಿ-ರುವುದೋ
ಸ್ವಯಂಪ್ರಭೆ
ಸ್ವಯಂಪ್ರಭೆ-ಯಿಂದ
ಸ್ವಯಂಭು
ಸ್ವಯಂಭುವು
ಸ್ವಯಂವೇದ್ಯ
ಸ್ವಯಂವೇದ್ಯ-ವಲ್ಲ
ಸ್ವರ
ಸ್ವರ-ಸವಾಹೀ
ಸ್ವರಾಜ್ಯ-ವಿತ್ತು
ಸ್ವರೂಪ
ಸ್ವರೂಪಕ್ಕೆ
ಸ್ವರೂಪ-ತೋಸ್ತ್ಯಧ್ವ-ಭೇ-ದಾದ್ಧರ್ಮಾ-ಣಾಮ್
ಸ್ವರೂಪದ
ಸ್ವರೂಪ-ದಲ್ಲಿ
ಸ್ವರೂಪ-ದಲ್ಲಿದೆ
ಸ್ವರೂಪ-ದಿಂದಲೇ
ಸ್ವರೂಪನು
ಸ್ವರೂಪನೂ
ಸ್ವರೂಪಪ್ರತಿಷ್ಠಾ
ಸ್ವರೂಪರು
ಸ್ವರೂಪ-ವನ್ನು
ಸ್ವರೂಪ-ವೆಂದು
ಸ್ವರೂಪ-ವೆಲ್ಲ
ಸ್ವರೂಪವೇ
ಸ್ವರೂಪ-ಶೂನ್ಯ-ಮಿವ
ಸ್ವರೂಪೇವಸ್ಥಾನಮ್
ಸ್ವರ್ಗ
ಸ್ವರ್ಗಕ್ಕಾಗಿ-ಯಾ-ಗಲಿ
ಸ್ವರ್ಗಕ್ಕಿಂತ
ಸ್ವರ್ಗಕ್ಕೂ
ಸ್ವರ್ಗಕ್ಕೆ
ಸ್ವರ್ಗ-ಗಳ
ಸ್ವರ್ಗ-ಗಳಲ್ಲಿ
ಸ್ವರ್ಗ-ಗಳಿ-ಗಿಂತ
ಸ್ವರ್ಗ-ಗಳಿವೆ
ಸ್ವರ್ಗ-ಗಳು
ಸ್ವರ್ಗ-ಗಳೂ
ಸ್ವರ್ಗ-ಜೀವ-ನ-ದಲ್ಲಿಯೂ
ಸ್ವರ್ಗದ
ಸ್ವರ್ಗ-ದಂತೆ
ಸ್ವರ್ಗ-ದಂತೆಯೇ
ಸ್ವರ್ಗ-ದಲ್ಲಿ
ಸ್ವರ್ಗ-ದಲ್ಲಿಯೂ
ಸ್ವರ್ಗ-ದಲ್ಲಿ-ರುವ
ಸ್ವರ್ಗ-ದಲ್ಲಿ-ರುವನು
ಸ್ವರ್ಗ-ದಲ್ಲಿರು-ವರೋ
ಸ್ವರ್ಗ-ದಿಂದ
ಸ್ವರ್ಗ-ಧಾಮ
ಸ್ವರ್ಗ-ಭಾವನೆ
ಸ್ವರ್ಗ-ಭಾವನೆ-ಯನ್ನು
ಸ್ವರ್ಗ-ಮರ್ತ್ಯ
ಸ್ವರ್ಗ-ಲೋಕ
ಸ್ವರ್ಗ-ಲೋಕ-ಗಳಿ-ಗಿಂತ
ಸ್ವರ್ಗ-ಲೋಕದ
ಸ್ವರ್ಗ-ಲೋಕ-ವೆಂಬ
ಸ್ವರ್ಗ-ವನ್ನು
ಸ್ವರ್ಗ-ವನ್ನೇ
ಸ್ವರ್ಗ-ವಾಗ-ಬಲ್ಲದೊ
ಸ್ವರ್ಗ-ವಾಗಲಿ
ಸ್ವರ್ಗ-ವಾಗಿ
ಸ್ವರ್ಗ-ವಾಗು-ವುದು
ಸ್ವರ್ಗ-ವಾಗು-ವು-ದೆನ್ನು-ವರು
ಸ್ವರ್ಗ-ವಿದೆ
ಸ್ವರ್ಗ-ವಿದೆಯೋ
ಸ್ವರ್ಗ-ವಿದ್ದರೆ
ಸ್ವರ್ಗ-ವಿರ-ಲಾರದೋ
ಸ್ವರ್ಗವು
ಸ್ವರ್ಗವೂ
ಸ್ವರ್ಗ-ವೆಂದ-ರೇನು
ಸ್ವರ್ಗ-ವೆಂದು
ಸ್ವರ್ಗ-ವೆಂಬು-ದೇನೂ
ಸ್ವರ್ಗ-ವೆನ್ನುತ್ತೇವೆ
ಸ್ವರ್ಗ-ವೆಲ್ಲ
ಸ್ವರ್ಗ-ವೆಲ್ಲಿದೆ
ಸ್ವರ್ಗವೇ
ಸ್ವರ್ಗ-ಸದೃಶ-ವಾಗು-ವುದು
ಸ್ವರ್ಗ-ಸುಖ
ಸ್ವರ್ಗ-ಸುಖ-ವನ್ನು
ಸ್ವರ್ಣ
ಸ್ವರ್ಣ-ತಂತು
ಸ್ವರ್ಣ-ತಂತು-ವನ್ನು
ಸ್ವರ್ಣ-ಯುಗ
ಸ್ವರ್ಣ-ವಸ್ತ್ರ-ದಿಂದ
ಸ್ವರ್ಣ-ವಸ್ತ್ರ-ವನ್ನು
ಸ್ವರ್ಣಸ್ವಪ್ನ
ಸ್ವರ್ಣಸ್ವಪ್ನ-ಗ-ಳನ್ನು
ಸ್ವರ್ಣಸ್ವಪ್ನ-ದಂತೆ
ಸ್ವಲ್ಪ
ಸ್ವಲ್ಪ-ಕಾಲದ
ಸ್ವಲ್ಪ-ಭಾಗ
ಸ್ವಲ್ಪ-ಮಟ್ಟಿಗೆ
ಸ್ವಲ್ಪ-ವಾಗಿ
ಸ್ವಲ್ಪವೂ
ಸ್ವಲ್ಪಸ್ವಲ್ಪ-ವಾಗಿ
ಸ್ವಲ್ಪಾಂಶ-ವನ್ನು
ಸ್ವಷ್ಪ
ಸ್ವಸಮ್ಮೋಹಿನೀ
ಸ್ವಸ್ಯಾಮಿ-ಶಕ್ತ್ಯೋಃಸ್ವ-ರೂಪೋಪಲಬ್ಧಿ-ಹೇತುಃ
ಸ್ವಸ್ವ-ವಿಷ-ಯಾಸಂಪ್ರ-ಯೋಗೇ
ಸ್ವಾಗ-ತಿ-ಸ-ಬೇಕು
ಸ್ವಾಗ-ತಿಸಿ
ಸ್ವಾಗ-ತಿ-ಸುವ
ಸ್ವಾಗ-ತಿ-ಸುವು-ದಕ್ಕಾಗಿ
ಸ್ವಾತಂತ್ರ
ಸ್ವಾತಂತ್ರ-ರಲ್ಲ
ಸ್ವಾತಂತ್ರ-ವೆಂಬು-ದಿಲ್ಲದೆ
ಸ್ವಾತಂತ್ರ್ಯ
ಸ್ವಾತಂತ್ರ್ಯಕ್ಕೆ
ಸ್ವಾತಂತ್ರ್ಯದ
ಸ್ವಾತಂತ್ರ್ಯ-ದಂತೆ
ಸ್ವಾತಂತ್ರ್ಯ-ದಲ್ಲಿ
ಸ್ವಾತಂತ್ರ್ಯ-ದಲ್ಲಿ-ರು-ವುದು
ಸ್ವಾತಂತ್ರ್ಯ-ದಿಂದ
ಸ್ವಾತಂತ್ರ್ಯ-ದೆ-ಡೆಗೆ
ಸ್ವಾತಂತ್ರ್ಯ-ವನ್ನು
ಸ್ವಾತಂತ್ರ್ಯ-ವನ್ನೇ
ಸ್ವಾತಂತ್ರ್ಯ-ವಿತ್ತು
ಸ್ವಾತಂತ್ರ್ಯ-ವಿದೆ
ಸ್ವಾತಂತ್ರ್ಯ-ವಿಲ್ಲ
ಸ್ವಾತಂತ್ರ್ಯವು
ಸ್ವಾತಂತ್ರ್ಯ-ವೆಂದರೆ
ಸ್ವಾತಂತ್ರ್ಯ-ವೆಂಬ
ಸ್ವಾತಂತ್ರ್ಯ-ವೆಂಬುದು
ಸ್ವಾತಂತ್ರ್ಯ-ವೆನ್ನು-ವುದು
ಸ್ವಾತಂತ್ರ್ಯವೇ
ಸ್ವಾತಂತ್ರ್ಯ-ಶಕ್ತಿಯ
ಸ್ವಾತಂತ್ರ್ಯ-ಹೀನ-ರೆಂದು
ಸ್ವಾತಂತ್ರ್ಯಾವ-ಕಾಶ
ಸ್ವಾತೀ
ಸ್ವಾಧಿಷ್ಠಾನ
ಸ್ವಾಧೀ
ಸ್ವಾಧೀನ
ಸ್ವಾಧೀ-ನಕ್ಕೆ
ಸ್ವಾಧೀ-ನತೆ
ಸ್ವಾಧೀ-ನ-ತೆಗೆ
ಸ್ವಾಧೀ-ನ-ತೆ-ಯನ್ನು
ಸ್ವಾಧೀ-ನ-ತೆ-ಯಿತ್ತು
ಸ್ವಾಧೀ-ನ-ತೆಯೂ
ಸ್ವಾಧೀ-ನ-ತೆ-ಯೆಲ್ಲ
ಸ್ವಾಧೀ-ನ-ದಲ್ಲಿ
ಸ್ವಾಧೀ-ನ-ದಲ್ಲಿಟ್ಟು-ಕೊಂಡಿ-ರು-ವುದು
ಸ್ವಾಧೀ-ನ-ದಲ್ಲಿ-ಡ-ಲಾಗು-ವು-ದಿಲ್ಲ
ಸ್ವಾಧೀ-ನ-ದಲ್ಲಿ-ಡುವು-ದಕ್ಕಾಗಿ
ಸ್ವಾಧೀ-ನ-ದಲ್ಲಿಯೂ
ಸ್ವಾಧೀ-ನ-ದಲ್ಲಿ-ರುತ್ತವೆ
ಸ್ವಾಧೀ-ನ-ದಿಂದ
ಸ್ವಾಧೀ-ನ-ಪಡಿಸಿ
ಸ್ವಾಧೀ-ನ-ವಾಗುತ್ತದೆ
ಸ್ವಾಧೀ-ನ-ವಾ-ಗು-ವುದು
ಸ್ವಾಧೀ-ನ-ವಾದ
ಸ್ವಾಧೀ-ನ-ವಿ-ದೆಯೊ
ಸ್ವಾಧೀ-ನ-ವಿಲ್ಲದೆ
ಸ್ವಾಧ್ಯಾಯ
ಸ್ವಾಧ್ಯಾಯ-ವೆಂದರೆ
ಸ್ವಾಧ್ಯಾಯ-ವೆಂದು
ಸ್ವಾಧ್ಯಾ-ಯಾದಿಷ್ಟ-ದೇವ-ತಾ-ಸಂಪ್ರ-ಯೋಗಃ
ಸ್ವಾಭಾ-ವಿಕ
ಸ್ವಾಭಾವಿ-ಕ-ವಲ್ಲ
ಸ್ವಾಭಾವಿ-ಕ-ವಾಗಿ
ಸ್ವಾಭಾವಿ-ಕ-ವಾಗಿಯೆ
ಸ್ವಾಭಾವಿ-ಕ-ವಾಗಿಯೇ
ಸ್ವಾಭಾವಿ-ಕ-ವಾದ
ಸ್ವಾಭಾವಿ-ಕ-ವಾದದು
ಸ್ವಾಭಾಸಂ
ಸ್ವಾಮಿ
ಸ್ವಾಮಿ-ಇ-ವರ
ಸ್ವಾಮಿ-ಗಳ
ಸ್ವಾಮಿ-ಗಳಿದ್ದರು
ಸ್ವಾಮಿ-ಗಳು
ಸ್ವಾಮಿಯ
ಸ್ವಾಮಿ-ಯಾಗ-ಬೇಕೆಂಬುದು
ಸ್ವಾರಸ್ಯ-ವಾದ
ಸ್ವಾರಸ್ಯ-ವಿಲ್ಲ-ದಿ-ರು-ವುದು
ಸ್ವಾರಾಜ್ಯ-ವನ್ನು
ಸ್ವಾರಾಜ್ಯ-ಸಿದ್ಧಿ
ಸ್ವಾರ್ಥ
ಸ್ವಾರ್ಥಕ್ಕೆ
ಸ್ವಾರ್ಥತೆ
ಸ್ವಾರ್ಥ-ತೆ-ಯಿಂದ
ಸ್ವಾರ್ಥತ್ಯಾಗ
ಸ್ವಾರ್ಥದ
ಸ್ವಾರ್ಥ-ದಿಂದ
ಸ್ವಾರ್ಥ-ದೊಂದಿಗೆ
ಸ್ವಾರ್ಥ-ನಾಶವೇ
ಸ್ವಾರ್ಥ-ಪರ-ತೆಯ
ಸ್ವಾರ್ಥ-ಪರ-ನಾಗ-ಬಾ-ರದು
ಸ್ವಾರ್ಥ-ಪರ-ನಾಗು-ವುದೇ
ಸ್ವಾರ್ಥ-ಪರ-ನಾ-ದಷ್ಟೂ
ಸ್ವಾರ್ಥ-ಪರ-ವಾದದ್ದೇ
ಸ್ವಾರ್ಥ-ವನ್ನು
ಸ್ವಾರ್ಥ-ವನ್ನೂ
ಸ್ವಾರ್ಥ-ವಿಲ್ಲವೊ
ಸ್ವಾರ್ಥ-ಸಂಯಮಾತ್
ಸ್ವಾರ್ಥ-ಹೀನ-ವಾದ
ಸ್ವಾರ್ಥಿ-ಯಾಗು
ಸ್ವಾರ್ಥಿ-ಯಾದ
ಸ್ವಾಸ್ಥ್ಯ-ವನ್ನು
ಸ್ವಿಕ-ರಿಸ-ಬಹುದು
ಸ್ವೀಕರಿ
ಸ್ವೀಕರಿಸ
ಸ್ವೀಕರಿ-ಸದೆ
ಸ್ವೀಕರಿ-ಸ-ಬಹುದು
ಸ್ವೀಕರಿ-ಸ-ಬಾ-ರದು
ಸ್ವೀಕರಿ-ಸ-ಬೇಕು
ಸ್ವೀಕರಿ-ಸ-ಬೇಕೆಂದು
ಸ್ವೀಕರಿ-ಸ-ಬೇಡಿ
ಸ್ವೀಕರಿ-ಸ-ಲಾರೆವು
ಸ್ವೀಕರಿ-ಸಲಿ
ಸ್ವೀಕರಿ-ಸಲು
ಸ್ವೀಕ-ರಿಸಿ
ಸ್ವೀಕರಿ-ಸಿದ
ಸ್ವೀಕರಿ-ಸಿ-ದರೆ
ಸ್ವೀಕರಿ-ಸಿ-ರು-ವರು
ಸ್ವೀಕ-ರಿಸು
ಸ್ವೀಕರಿ-ಸುತ್ತದೆ
ಸ್ವೀಕರಿ-ಸುತ್ತಾನೆ
ಸ್ವೀಕರಿ-ಸುತ್ತಾರೆ
ಸ್ವೀಕರಿ-ಸುತ್ತಾರೆಯೊ
ಸ್ವೀಕರಿ-ಸುತ್ತಿ-ರುವ-ವರೆಗೂ
ಸ್ವೀಕರಿ-ಸುತ್ತೇನೆ
ಸ್ವೀಕರಿ-ಸುತ್ತೇವೆ
ಸ್ವೀಕರಿ-ಸುವ
ಸ್ವೀಕರಿ-ಸು-ವನು
ಸ್ವೀಕರಿ-ಸು-ವರು
ಸ್ವೀಕರಿ-ಸು-ವ-ವನು
ಸ್ವೀಕರಿ-ಸು-ವಿರಿ
ಸ್ವೀಕರಿ-ಸು-ವು-ದಕ್ಕೆ
ಸ್ವೀಕರಿ-ಸು-ವು-ದ-ರಲ್ಲಿ
ಸ್ವೀಕರಿ-ಸು-ವು-ದ-ರಿಂದ
ಸ್ವೀಕರಿ-ಸು-ವು-ದಿಲ್ಲ
ಸ್ವೀಕರಿ-ಸು-ವುದು
ಸ್ವೀಕರಿ-ಸು-ವುವು
ಸ್ವೀಕರಿ-ಸು-ವೆನು
ಸ್ವೀಕರಿ-ಸು-ವೆವು
ಸ್ವೀಕರಿ-ಸೋಣ
ಸ್ವೀಕಾರ
ಸ್ವೀಕಾರಕ್ಕೂ
ಸ್ವೀಕಾರ-ವನ್ನು
ಸ್ವೇಚ್ಛಾವಿ-ಹಾರಿ-ಗಳಾ-ಗು-ವೆವು
ಸ್ವೇಚ್ಛಾ-ಸಂಸ್ಕಾರ-ಗಳಾ-ಗುತ್ತವೆ
ಹಂತ
ಹಂತಕ್ಕೆ
ಹಂತ-ಗ-ಳನ್ನು
ಹಂತ-ಗಳಲ್ಲಿ
ಹಂತ-ಗಳಿವೆ
ಹಂತ-ಗಳೂ
ಹಂತ-ದಲ್ಲಿ
ಹಂತ-ದಲ್ಲಿಯೂ
ಹಂತ-ದಿಂದ
ಹಂತ-ವನ್ನೂ
ಹಂತ-ವಾಗಿ
ಹಂತ-ವಾಗಿ-ರ-ಬಹುದು
ಹಂತವೆ
ಹಂತ-ಹಂತ-ವಾಗಿ
ಹಂದಿ
ಹಂದಿ-ಗಳನ್ನೆಲ್ಲ
ಹಂದಿ-ಮಾಂಸ
ಹಂದಿಯ
ಹಂದಿ-ಯನ್ನು
ಹಂದಿ-ಯಲ್ಲ
ಹಂದಿ-ಯಾಗ-ಬೇಕೆಂದು
ಹಂದಿ-ಯಾಗಿ
ಹಂಫ್ರಿ
ಹಂಬಲವು
ಹಂಸ
ಹಕ್ಕನ್ನು
ಹಕ್ಕಾಗಿದೆ
ಹಕ್ಕಾದ
ಹಕ್ಕಿ
ಹಕ್ಕಿ-ಗಳಿಗೆ
ಹಕ್ಕಿ-ಗಳಿವೆ
ಹಕ್ಕಿ-ಗಳು
ಹಕ್ಕಿದೆ
ಹಕ್ಕಿಯ
ಹಕ್ಕಿ-ಯಂತೆ
ಹಕ್ಕಿ-ಯನ್ನು
ಹಕ್ಕಿ-ಯಾಗಿದ್ದೆ
ಹಕ್ಕಿ-ಯಾಗು-ವು-ದ-ರಿಂದ
ಹಕ್ಕಿಯು
ಹಕ್ಕಿಯೇ
ಹಕ್ಕಿಲ್ಲ
ಹಕ್ಕು
ಹಕ್ಕು-ಗ-ಳನ್ನು
ಹಕ್ಕು-ಗಳಿವೆ
ಹಕ್ಸ್ಲೆ
ಹಕ್ಸ್ಲೇ
ಹಗಲ
ಹಗ-ಲಾಗು
ಹಗ-ಲಿಗೆ
ಹಗಲಿ-ನಷ್ಟು
ಹಗಲಿ-ನಿಂದ
ಹಗಲಿ-ರುಳು
ಹಗಲು
ಹಗಲೂ
ಹಗ-ಲೆಲ್ಲ
ಹಗಲೇ
ಹಗುರ-ವಾಗ-ಬಲ್ಲ
ಹಗುರ-ವಾಗಿ
ಹಗುರ-ವಾಗು-ವನು
ಹಗ್ಗ
ಹಗ್ಗದ
ಹಗ್ಗ-ವನ್ನು
ಹಗ್ಗವು
ಹಗ್ಗ-ವೆಂದು
ಹಚ್ಚಿ-ಕೊಳ್ಳ-ಬಾ-ರದು
ಹಚ್ಚಿ-ಕೊಳ್ಳು-ವು-ದಿಲ್ಲ
ಹಠ-ಯೋಗದ
ಹಠ-ಯೋಗ-ವನ್ನು
ಹಠ-ಯೋಗವು
ಹಠ-ಯೋಗಿ
ಹಠ-ಯೋಗಿಯು
ಹಡ-ಗಿನ
ಹಡ-ಗಿನಲ್ಲಿದ್ದ
ಹಡಗು-ಗಳ
ಹಡಗು-ಗಳೂ
ಹಡಗೊಂದು
ಹಣ
ಹಣಕ್ಕಾಗಿ
ಹಣದ
ಹಣ-ದಲ್ಲಿ
ಹಣ-ವಿಲ್ಲದೆ
ಹಣೆಯ
ಹಣ್ಣನ್ನು
ಹಣ್ಣನ್ನೂ
ಹಣ್ಣು
ಹಣ್ಣು-ಗ-ಳನ್ನು
ಹಣ್ಣು-ಗಳೂ
ಹತ-ಭಾಗ್ಯನಾ-ದಾಗ
ಹತಾಶರೂ
ಹತ್ತ-ರಷ್ಟು
ಹತ್ತಿ
ಹತ್ತಿ-ಯಂತೆ
ಹತ್ತಿರ
ಹತ್ತಿ-ರಕ್ಕಿಂತ
ಹತ್ತಿ-ರಕ್ಕೆ
ಹತ್ತಿ-ರ-ದಕ್ಕಿಂತ
ಹತ್ತಿ-ರ-ದಲ್ಲಿ
ಹತ್ತಿ-ರ-ದಲ್ಲಿದೆ
ಹತ್ತಿ-ರ-ದಲ್ಲಿ-ರುವ
ಹತ್ತಿ-ರ-ದಲ್ಲಿ-ರುವುದು
ಹತ್ತಿ-ರ-ದಲ್ಲಿ-ರುವುದೇ
ಹತ್ತಿ-ರ-ದ-ವನು
ಹತ್ತಿ-ರ-ವಾ-ಗಿಲ್ಲ
ಹತ್ತಿ-ರವೂ
ಹತ್ತಿ-ಸ-ಬಲ್ಲ-ವರು-ಇಂಥ-ವ-ರಿಗೆ
ಹತ್ತಿ-ಹೋಗ-ಬೇಕಾ-ಗಿ-ರುವ
ಹತ್ತು
ಹತ್ತು-ವರು
ಹತ್ತು-ವುವು
ಹತ್ತೊಂಬತ್ತನೇ
ಹದ-ಗೆಡೆಸಿ-ದಂತೆ
ಹದಿನಾರು
ಹದಿನೈದು
ಹದ್ದಿಗೆ
ಹದ್ದು
ಹನಿ
ಹನಿಗೆ
ಹನಿ-ಯಂತೆ
ಹನಿ-ಯಾಗಿ
ಹನಿಯು
ಹನಿ-ಯೊಂದು
ಹನ್ನೆ-ರಡು
ಹಬ್ಬಿ-ರುವ
ಹಬ್ಬುತ್ತಾ
ಹರಕು
ಹರ-ಟಲಿ
ಹರಟೆ
ಹರ-ಡ-ಬೇಕಾ-ಗಿತ್ತು
ಹರ-ಡಲು
ಹರಡಿ
ಹರ-ಡಿ-ಕೊಂಡಿವೆ
ಹರ-ಡಿ-ದನು
ಹರ-ಡಿ-ದರು
ಹರ-ಡಿ-ದಾಗ
ಹರ-ಡಿದ್ದರೂ
ಹರ-ಡಿದ್ದರೆ
ಹರ-ಡಿ-ರು-ವ-ವನು
ಹರ-ಡಿಲ್ಲವೊ
ಹರ-ಡಿ-ವೆಯೊ
ಹರಡು
ಹರಡುತ್ತ
ಹರಡುತ್ತ-ವೆಆ-ವರೂ
ಹರಡುತ್ತಿ-ರು-ವರು
ಹರಡುತ್ತಿವೆ
ಹರ-ಡುವ
ಹರಡು-ವು-ದಕ್ಕೆ
ಹರ-ಣೆಗೆ
ಹರ-ವಾದ
ಹರ-ಸ-ಬಹುದು
ಹರಸಿ
ಹರಾಕ್
ಹರಾಜಿನ
ಹರಿ-ದಾಡುವ
ಹರಿ-ದಿದೆ
ಹರಿದು
ಹರಿದು-ಬರು-ವುದಕ್ಕೆ
ಹರಿದು-ಹಾಕಿ-ದಂತೆ
ಹರಿದು-ಹೋಗುತ್ತಿದೆಯೋ
ಹರಿದು-ಹೋಗುತ್ತಿವೆ
ಹರಿದು-ಹೋಗುವ
ಹರಿದು-ಹೋಗು-ವಂತೆ
ಹರಿದು-ಹೋದ
ಹರಿ-ಯಲಿ
ಹರಿ-ಯಲು
ಹರಿ-ಯಿತು
ಹರಿಯು
ಹರಿ-ಯುತ್ತದೆ
ಹರಿ-ಯುತ್ತಿದೆ
ಹರಿ-ಯುತ್ತಿ-ರುತ್ತದೆ
ಹರಿ-ಯುತ್ತಿ-ರುವ
ಹರಿ-ಯುತ್ತಿ-ರುವಾಗ
ಹರಿ-ಯುತ್ತಿ-ರುವುದು
ಹರಿ-ಯುತ್ತಿವೆ-ಯೆಂದೂ
ಹರಿ-ಯುವ
ಹರಿ-ಯು-ವಂತೆ
ಹರಿ-ಯು-ವು-ದ-ರಿಂದ
ಹರಿ-ಯು-ವುದು
ಹರಿ-ಯು-ವುದೋ
ಹರಿಸ
ಹರಿಸ-ದಿರು-ವಾಗ
ಹರಿಸ-ಬೇಕು
ಹರಿಸಿ
ಹರಿಸಿದ
ಹರಿಸಿ-ದಾಗ
ಹರಿ-ಸಿಲ್ಲ
ಹರಿಸುವ
ಹರಿಸು-ವು-ದ-ರಿಂದ
ಹರಿಸು-ವುದು
ಹರ್ಬರ್ಟ್
ಹಲ-ವನ್ನು
ಹಲ-ವರ
ಹಲವ-ರನ್ನು
ಹಲವ-ರಲ್ಲಿ
ಹಲ-ವರಿಗೆ
ಹಲ-ವರು
ಹಲ-ವಾಗಿ
ಹಲ-ವಾರು
ಹಲವು
ಹಲ-ವೆಂಬಂತೆ
ಹಲ್ಲಿಲ್ಲದೆ
ಹಳದಿ-ಯಾಗಿ
ಹಳಿಯು-ವರು
ಹಳೆಯ
ಹಳೆಯ-ದರಿಂದ
ಹಳೆಯ-ದಿದು-ಮಾನ-ವನ
ಹಳೆ-ಯದು
ಹಳೆಯ-ದೆಲ್ಲ
ಹಳ್ಳಕ್ಕೆ
ಹಳ್ಳಿ
ಹಳ್ಳಿ-ಗಳು
ಹಳ್ಳಿಯ
ಹಳ್ಳಿ-ಯನ್ನು
ಹವಣಿಸುತ್ತಿ-ರುವ
ಹವಣಿ-ಸು-ವುದು
ಹವಣಿ-ಸು-ವೆವು
ಹವ್ಯಾಸಕ್ಕೆ
ಹವ್ಯಾಸ-ಗಳನ್ನು
ಹಸಿದ-ವನ
ಹಸಿ-ವಾಗಿದ್ದರೆ
ಹಸಿ-ವಾಗು-ವು-ದಿಲ್ಲ
ಹಸಿ-ವಾಗು-ವುದು
ಹಸಿ-ವಿನ
ಹಸಿವಿ-ನಿಂದ
ಹಸಿವು
ಹಸು
ಹಸು-ಗಳನ್ನು
ಹಸು-ವನ್ನು
ಹಸು-ವಿಗೆ
ಹಸು-ವಿದೆ
ಹಸು-ವಿನ
ಹಸುವಿ-ನಲ್ಲೂ
ಹಸುವಿ-ನಿಂದ
ಹಸುವು
ಹಸು-ವೆಂಬ
ಹಸ್ತಲಾಘವ-ವನ್ನು
ಹಸ್ತಾಮಲಕ-ದಂತೆ
ಹಸ್ತಿಬ-ಲಾದೀನಿ
ಹಾಕದೆ
ಹಾಕ-ಬಲ್ಲ
ಹಾಕ-ಬೇಕು
ಹಾಕ-ಲಾರೆವು
ಹಾಕಲು
ಹಾಕಿ
ಹಾಕಿ-ಕೊಂಡು
ಹಾಕಿ-ಕೊಳ್ಳ-ಬಹುದು
ಹಾಕಿ-ಕೊಳ್ಳ-ಬೇಡಿ
ಹಾಕಿ-ಕೊಳ್ಳುತ್ತಿದ್ದರೊ
ಹಾಕಿದ
ಹಾಕಿ-ದಂತೆ
ಹಾಕಿ-ದನು
ಹಾಕಿ-ದರು
ಹಾಕಿ-ದರೆ
ಹಾಕಿದ್ದರೂ
ಹಾಕಿರು-ವಿರೋ
ಹಾಕಿರು-ವುದನ್ನು
ಹಾಕು
ಹಾಕುತ್ತ
ಹಾಕು-ತ್ತಾನೆ
ಹಾಕು-ತ್ತಾರೆ
ಹಾಕು-ತ್ತಿದ್ದ
ಹಾಕು-ತ್ತಿದ್ದೇನೆ
ಹಾಕುವ
ಹಾಕು-ವರು
ಹಾಕುವು
ಹಾಕು-ವು-ದ-ರಿಂದ
ಹಾಕು-ವುದು
ಹಾಗಲ್ಲ
ಹಾಗಲ್ಲದೆ
ಹಾಗಲ್ಲದೇ
ಹಾಗಾ-ಗಿಲ್ಲ
ಹಾಗಾ-ದರೆ
ಹಾಗಾ-ಯಿತು
ಹಾಗಿದ್ದರೆ
ಹಾಗಿದ್ದಿದ್ದರೆ
ಹಾಗಿರ-ಲಾರದು
ಹಾಗಿರು-ವು-ದೆಂದು
ಹಾಗಿಲ್ಲ
ಹಾಗೂ
ಹಾಗೆ
ಹಾಗೆಂದರೆ
ಹಾಗೆಂದ-ರೇನು
ಹಾಗೆಂದಿಗೂ
ಹಾಗೆ-ತಟಸ್ಥ-ವಾ-ಗು-ವುವು
ಹಾಗೆಯ
ಹಾಗೆಯೆ
ಹಾಗೆಯೇ
ಹಾಗೇ
ಹಾಡನ್ನು
ಹಾಡಿನ
ಹಾಡು
ಹಾಡುತ್ತ
ಹಾಡು-ವರು
ಹಾದಿ
ಹಾದಿಗೆ
ಹಾದಿಯ
ಹಾದಿ-ಯನ್ನು
ಹಾದಿ-ಯಲ್ಲಿ
ಹಾದು
ಹಾದು-ಹೋಗ-ಬೇಕಾ-ಗಿದೆ
ಹಾದು-ಹೋಗುತ್ತಿದೆ
ಹಾನಂ
ಹಾನ-ವೇಶಾಂ
ಹಾನಿ
ಹಾನಿ-ಕರ
ಹಾನಿ-ಕರ-ವಾದ
ಹಾನಿ-ಯಿಲ್ಲ
ಹಾನೋಪಾಯಃ
ಹಾಯಿಸಿ-ದರೆ
ಹಾರ-ಬಲ್ಲನು
ಹಾರ-ಲಾರ
ಹಾರಾಡುತ್ತಿ-ರ-ಬೇಕು
ಹಾರಾ-ಡು-ವುದು
ಹಾರಿ
ಹಾರಿಕ
ಹಾರಿ-ಹೋಗ-ಬೇಕೆಂದು
ಹಾರುತ್ತ
ಹಾರುತ್ತಿರು-ವಂತೆ
ಹಾರುವ
ಹಾರೈಕೆ
ಹಾರೈಸು-ವೆನು
ಹಾಲನ್ನು
ಹಾಲಿನ
ಹಾಲು
ಹಾಳಾ-ದಾಗ
ಹಾಳು
ಹಾಳು-ಮಾಡಿ
ಹಾಳು-ಮಾಡಿ-ಕೊಳ್ಳು-ವುದು
ಹಾಳು-ಮಾಡಿ-ದಾಗ
ಹಾಳು-ಮಾಡುತ್ತದೆ
ಹಾಳು-ಮಾಡು-ವುದು
ಹಾಳು-ಮಾಡು-ವುವು
ಹಾಳೆ
ಹಾಳೆ-ಗಳ
ಹಾವಿ-ನಂತೆ
ಹಾವು
ಹಾವೆಂದು
ಹಾಸಿಗೆ-ಯಲ್ಲಿ
ಹಾಸುಹೊಕ್ಕಾಗಿ
ಹಾಸುಹೊಕ್ಕಾಗಿ-ರುವನು
ಹಾಸ್ಯಾಸ್ಪದ-ವಾ-ಗು-ವುದು
ಹಿಂಗೆ
ಹಿಂಜರಿ
ಹಿಂಜರಿ-ಯದ
ಹಿಂಜರಿ-ಯು-ವರೋ
ಹಿಂಜರಿ-ಯು-ವು-ದಿಲ್ಲ
ಹಿಂಡ-ಬೇಕಾ-ಗಿದೆ
ಹಿಂಡು-ಗಳು
ಹಿಂತಿ
ಹಿಂತಿ-ರುಗ-ಬೇಕಾ-ಗು-ವುದು
ಹಿಂತಿ-ರುಗ-ಬೇಕು
ಹಿಂತಿ-ರು-ಗಲು
ಹಿಂತಿರುಗಿ
ಹಿಂತಿರುಗಿ-ಕೊಡು-ವು-ದನ್ನು
ಹಿಂತಿರುಗಿತು
ಹಿಂತಿರುಗಿದ
ಹಿಂತಿರುಗಿ-ದನು
ಹಿಂತಿರುಗಿ-ದರು
ಹಿಂತಿರುಗಿ-ಸು-ವುದು
ಹಿಂತಿ-ರುಗು
ಹಿಂತಿರು-ಗುತ್ತದೆ
ಹಿಂತಿ-ರುಗುತ್ತಾನೆ
ಹಿಂತಿ-ರುಗು-ವನು
ಹಿಂತಿ-ರುಗು-ವರು
ಹಿಂತಿ-ರುಗು-ವ-ವರೆಗೆ
ಹಿಂತಿ-ರುಗು-ವಾಗ
ಹಿಂತಿ-ರು-ಗುವು-ದಕ್ಕಾಗಿಯೆ
ಹಿಂತಿ-ರುಗು-ವು-ದಕ್ಕೆ
ಹಿಂತಿ-ರುಗು-ವು-ದಿಲ್ಲ
ಹಿಂತಿರು-ಗು-ವುದು
ಹಿಂತಿರು-ಗು-ವುದು-ಆತ್ಮವು
ಹಿಂತಿ-ರುಗು-ವುವು
ಹಿಂತೆಗೆ-ದು-ಕೊಳ್ಳು-ವುದು
ಹಿಂತೆರಳು-ವುದು
ಹಿಂದಕ್ಕೆ
ಹಿಂದನ್ನು
ಹಿಂದಿ-ಗಿಂತ
ಹಿಂದಿನ
ಹಿಂದಿ-ನಂತೆ
ಹಿಂದಿ-ನಂತೆಯೇ
ಹಿಂದಿನ-ಕಾಲದ
ಹಿಂದಿನ-ದಕ್ಕಿಂತಲೂ
ಹಿಂದಿನ-ದಕ್ಕೆ
ಹಿಂದಿನ-ದನ್ನು
ಹಿಂದಿನ-ದನ್ನೆಲ್ಲ
ಹಿಂದಿನ-ದರ
ಹಿಂದಿನದು
ಹಿಂದಿನದೂ
ಹಿಂದಿನ-ದೆಲ್ಲ
ಹಿಂದಿನ-ವ-ರನ್ನು
ಹಿಂದಿನ-ವರಿ-ಗಿಂತ
ಹಿಂದಿನ-ವ-ರಿಗೆ
ಹಿಂದಿನ-ವರು
ಹಿಂದಿನವು
ಹಿಂದಿನ-ವು-ಗಳಿ-ಗಿಂತ
ಹಿಂದಿ-ನಷ್ಟು
ಹಿಂದಿ-ನಿಂದ
ಹಿಂದಿ-ನಿಂದಲೂ
ಹಿಂದಿರುಗಿ
ಹಿಂದಿರು-ಗು-ವಂತೆ
ಹಿಂದಿ-ರುವ
ಹಿಂದು
ಹಿಂದು-ಗಳು
ಹಿಂದೂ
ಹಿಂದೂ-ಗಳ
ಹಿಂದೂ-ಗಳಂತೆ
ಹಿಂದೂ-ಗಳನ್ನು
ಹಿಂದೂ-ಗಳಲ್ಲಿ
ಹಿಂದೂ-ಗಳಲ್ಲಿದೆ
ಹಿಂದೂ-ಗಳಿಗೆ
ಹಿಂದೂ-ಗಳು
ಹಿಂದೂ-ಗಳೂ
ಹಿಂದೂ-ಗಳೊಂದಿಗೆ
ಹಿಂದೂ-ದರ್ಶನ-ಗಳಲ್ಲಿಯೂ
ಹಿಂದೂ-ದರ್ಶನ-ಗಳೂ
ಹಿಂದೂ-ದೇಶ
ಹಿಂದೂ-ವಿಗೆ
ಹಿಂದೂ-ವಿ-ನಲ್ಲಿ-ರುವ
ಹಿಂದೆ
ಹಿಂದೆಂದೂ
ಹಿಂದೆಯೂ
ಹಿಂದೆಯೆ
ಹಿಂದೆಯೇ
ಹಿಂದೆಲ್ಲ
ಹಿಂಧೂ-ಧರ್ಮ-ದಲ್ಲಿ
ಹಿಂಬಾಲಿ-ಸ-ಬೇಕು
ಹಿಂಬಾಲಿಸ-ಲಾರದ
ಹಿಂಬಾಲಿಸಿ
ಹಿಂಬಾಲಿಸು
ಹಿಂಬಾಲಿಸುತ್ತಿ-ರು-ವೆವು
ಹಿಂಬಾಲಿಸು-ವು-ದ-ರಲ್ಲಿ
ಹಿಂಬಾಲಿಸು-ವುದು
ಹಿಂಸಾದಯಃ
ಹಿಂಸಿಸ-ಕೂಡದು
ಹಿಂಸಿಸ-ಬಾ-ರದು
ಹಿಂಸಿಸಿ-ದಷ್ಟೂ
ಹಿಂಸೆ
ಹಿಂಸೆಗೆ
ಹಿಂಸೆ-ಯನ್ನು
ಹಿಡಿ
ಹಿಡಿತ
ಹಿಡಿ-ತಕ್ಕೆ
ಹಿಡಿ-ತ-ವಿ-ರು-ವುದು
ಹಿಡಿ-ದಂತೆ
ಹಿಡಿದಿ
ಹಿಡಿ-ದಿರಿ
ಹಿಡಿ-ದಿ-ರು-ವರು
ಹಿಡಿ-ದಿ-ರು-ವ-ವರು
ಹಿಡಿ-ದಿ-ರು-ವಿರಿ
ಹಿಡಿದು
ಹಿಡಿ-ದು-ಕೊಂಡ
ಹಿಡಿ-ದು-ಕೊಂಡಳೊ
ಹಿಡಿ-ದು-ಕೊಂಡಿದ್ದರೆ
ಹಿಡಿ-ದು-ಕೊಂಡಿರಿ
ಹಿಡಿ-ದು-ಕೊಂಡಿ-ರು-ವುದು
ಹಿಡಿ-ದು-ಕೊಳ್ಳಿ
ಹಿಡಿದೇ
ಹಿಡಿ-ಯ-ಬೇಕಾ-ಗುತ್ತದೆ
ಹಿಡಿ-ಯ-ಬೇಕು
ಹಿಡಿ-ಯ-ಲಾರಿರಿ
ಹಿಡಿ-ಯಲು
ಹಿಡಿ-ಯುತ್ತದೆ
ಹಿಡಿ-ಯುತ್ತವೆ
ಹಿಡಿ-ಯು-ವಂತೆ
ಹಿಡಿ-ಯು-ವನೋ
ಹಿಡಿ-ಯುವ-ವ-ರಲ್ಲ
ಹಿಡಿ-ಯುವು-ದಕ್ಕೆ
ಹಿಡಿ-ಯು-ವು-ದ-ರಲ್ಲಿ
ಹಿಡಿ-ಯು-ವುದು
ಹಿಡಿ-ಯು-ವುದೇ
ಹಿಡಿ-ಸು-ವುದು
ಹಿತ
ಹಿತ-ಕರ
ಹಿತ-ಕರ-ವಾಗಿದೆ
ಹಿತ-ಕರ-ವಾಗಿಯೂ
ಹಿತ-ಕರ-ವಾ-ದರೂ
ಹಿತ-ಕಾರಿ
ಹಿತ-ಕಾರಿ-ಯಾ-ಗಿದೆ
ಹಿತಕ್ಕಾಗಿ
ಹಿತಕ್ಕೆ
ಹಿತಕ್ಕೋಸುಗ
ಹಿತ-ಚಿಂತನೆಯೇ
ಹಿತ-ದೃಷ್ಟಿ
ಹಿತ-ರಕ್ಷಣೆ-ಯನ್ನು
ಹಿತ-ವನ್ನು
ಹಿತ-ವಾಗು-ವುದೆ
ಹಿನ್ನೆಲೆ
ಹಿನ್ನೆಲೆ-ಯಂತೆ
ಹಿನ್ನೆಲೆ-ಯನ್ನು
ಹಿನ್ನೆಲೆ-ಯಲ್ಲಿ
ಹಿನ್ನೆಲೆ-ಯಾ-ಗಿದೆ
ಹಿನ್ನೆಲೆ-ಯಾಗಿ-ರು-ವು-ದನ್ನು
ಹಿನ್ನೆಲೆ-ಯಾ-ಗು-ವುದು
ಹಿನ್ನೆಲೆ-ಯಿಂದ
ಹಿನ್ನೆಲೆಯೂ
ಹಿಬ್ರು-ಗಳಿಗೆ
ಹಿಬ್ರು-ಗಳೂ
ಹಿಮ
ಹಿಮ-ದಲ್ಲಿ
ಹಿಮ-ಮಣಿ-ಗಳ
ಹಿರಿದ
ಹಿರಿಮೆ
ಹಿರಿಮೆ-ಯನ್ನು
ಹಿರಿ-ಯರು
ಹಿಸುಕಿ-ದಂತೆ
ಹಿಸು-ವು-ದನ್ನು
ಹೀಂದೆ
ಹೀಗಲ್ಲ
ಹೀಗಾಗಿ
ಹೀಗಿದೆಯೆ
ಹೀಗಿದ್ದರೆ
ಹೀಗಿ-ರುತ್ತದೆ
ಹೀಗಿ-ರು-ವಾಗ
ಹೀಗಿ-ರುವು
ಹೀಗಿ-ರುವುದು
ಹೀಗಿ-ರು-ವೆವು
ಹೀಗೆ
ಹೀಗೆಂದನು
ಹೀಗೆಂದರೆ
ಹೀಗೆಂದ-ರೇನು
ಹೀಗೆಂದು
ಹೀಗೆನ್ನುತ್ತಾನೆ
ಹೀಗೆನ್ನು-ವುದು
ಹೀಗೆಯೆ
ಹೀಗೆಯೇ
ಹೀಗೇ
ಹೀತ-ಕೃತ್ಯ-ಗಳನ್ನು
ಹೀನ
ಹೀನ-ಕರ್ಮ-ಗಳನ್ನು
ಹೀನ-ಕರ್ಮ-ಗಳೂ
ಹೀನ-ನೆಂದೂ
ಹೀನ-ರನ್ನಾಗಿ
ಹೀನರು
ಹೀನ-ವಂಶ-ದಲ್ಲಿ
ಹೀನ-ವಾಗಿ
ಹೀನ-ವಾದ
ಹೀನ-ವಾ-ದದ್ದು
ಹೀನ-ವಾದುದು
ಹೀನ-ವಾದು-ದೆಂದರೆ
ಹೀನಸ್ಥಿತಿಗೆ
ಹೀಬ್ರೂ
ಹೀರಿ
ಹೀರಿ-ಕೊಂಡು
ಹೀರಿ-ಕೊಳ್ಳುತ್ತದೆ
ಹೀರಿ-ಕೊಳ್ಳು-ವು-ದೆಂದು
ಹೀರು-ವು-ದಕ್ಕೆ
ಹೀರು-ವುದೇ
ಹುಚ್ಚ
ಹುಚ್ಚ-ನಾಗಿ
ಹುಚ್ಚನು
ಹುಚ್ಚ-ನೆಂದು
ಹುಚ್ಚ-ನೊಬ್ಬ
ಹುಚ್ಚರ
ಹುಚ್ಚ-ರಂತೆ
ಹುಚ್ಚ-ರಾಗುವ
ಹುಚ್ಚರು
ಹುಚ್ಚಾಸ್ಪತ್ರೆಯ
ಹುಚ್ಚಿನ
ಹುಚ್ಚು
ಹುಟ್ಟದ
ಹುಟ್ಟ-ಬೇಕಾ-ಗುತ್ತದೆ
ಹುಟ್ಟ-ಬೇಕಾದ
ಹುಟ್ಟ-ಲಾರದು
ಹುಟ್ಟಲಿ
ಹುಟ್ಟ-ಲಿಲ್ಲ
ಹುಟ್ಟಿ
ಹುಟ್ಟಿತು
ಹುಟ್ಟಿದ
ಹುಟ್ಟಿ-ದಂದಿ-ನಿಂದಲೂ
ಹುಟ್ಟಿ-ದ-ನೆಂದು
ಹುಟ್ಟಿ-ದ-ವನು
ಹುಟ್ಟಿ-ದಾಗಿ-ನಿಂದ
ಹುಟ್ಟಿ-ದಾಗಿ-ನಿಂದಲೂ
ಹುಟ್ಟಿ-ದು-ದ-ರಿಂದ
ಹುಟ್ಟಿ-ನಿಂದಲೇ
ಹುಟ್ಟಿ-ಬಂದ
ಹುಟ್ಟಿಯೂ
ಹುಟ್ಟಿ-ರು-ವಂತೆ
ಹುಟ್ಟಿ-ರು-ವರು
ಹುಟ್ಟಿ-ರುವು-ದೆಲ್ಲ
ಹುಟ್ಟಿ-ರು-ವೆವು
ಹುಟ್ಟಿಲ್ಲ
ಹುಟ್ಟಿ-ಸ-ಬಹುದು
ಹುಟ್ಟಿ-ಸಿದೆ
ಹುಟ್ಟಿ-ಸುತ್ತದೆ
ಹುಟ್ಟಿ-ಸುತ್ತವೆ
ಹುಟ್ಟಿ-ಸುವ
ಹುಟ್ಟಿ-ಸುವು-ದಿಲ್ಲ
ಹುಟ್ಟಿ-ಸು-ವುದು
ಹುಟ್ಟು
ಹುಟ್ಟು-ಗುಣ
ಹುಟ್ಟು-ಗುಣಕ್ಕೆ
ಹುಟ್ಟು-ಗುಣ-ಗಳು
ಹುಟ್ಟು-ಗುಣ-ಗ-ಳೆಂದು
ಹುಟ್ಟು-ಗುಣದ
ಹುಟ್ಟು-ಗುಣ-ದಿಂದ
ಹುಟ್ಟು-ಗುಣ-ವನ್ನೇ
ಹುಟ್ಟು-ಗುಣ-ವಾಗಿ
ಹುಟ್ಟು-ಗುಣ-ವಾಗಿ-ದೆಯೊ
ಹುಟ್ಟು-ಗುಣ-ವಾಗುತ್ತದೆ
ಹುಟ್ಟು-ಗುಣ-ವಾಗು-ವುದು
ಹುಟ್ಟು-ಗುಣ-ವಿರುವ
ಹುಟ್ಟು-ಗುಣ-ವೆಂದರೆ
ಹುಟ್ಟು-ಗುಣ-ವೆಂದ-ರೇನು
ಹುಟ್ಟು-ಗುಣ-ವೆಂಬ
ಹುಟ್ಟು-ಗುಣ-ವೆನ್ನು-ತ್ತೇವೆಯೊ
ಹುಟ್ಟು-ಗುಣ-ವೆನ್ನುವ
ಹುಟ್ಟು-ಗುಣವೇ
ಹುಟ್ಟು-ತ್ತಾರೆ
ಹುಟ್ಟು-ತ್ತಿರು-ವುವು
ಹುಟ್ಟು-ತ್ತಿವೆ
ಹುಟ್ಟು-ತ್ತೇನೆ
ಹುಟ್ಟುವ
ಹುಟ್ಟು-ವಂತೆ
ಹುಟ್ಟು-ವರು
ಹುಟ್ಟು-ವಾಗ
ಹುಟ್ಟು-ವು-ದಕ್ಕೆ
ಹುಟ್ಟು-ವು-ದಿಲ್ಲ
ಹುಟ್ಟು-ವುದು
ಹುಟ್ಟು-ವುದೂ
ಹುಟ್ಟು-ವುದೇ
ಹುಟ್ಟು-ವುವು
ಹುಟ್ಟೂ
ಹುಟ್ಟೇ
ಹುಡು
ಹುಡುಕ-ಬಹುದು
ಹುಡುಕ-ಬೇಕು
ಹುಡುಕ-ಬೇಕೆಂದು
ಹುಡುಕಲು
ಹುಡುಕಾಟ-ಗಳನ್ನೆಲ್ಲ
ಹುಡುಕಾಟ-ವನ್ನು
ಹುಡು-ಕಾ-ಡಲು
ಹುಡು-ಕಾಡಿ
ಹುಡು-ಕಾಡಿದ
ಹುಡು-ಕಾಡು
ಹುಡುಕಿ
ಹುಡುಕಿ-ಕೊಂಡು
ಹುಡುಕಿ-ಕೊಡು-ವುದು
ಹುಡುಕಿ-ಕೊಳ್ಳಲು
ಹುಡುಕಿದ
ಹುಡುಕಿ-ಹುಡುಕಿ
ಹುಡುಕು
ಹುಡುಕು-ತ್ತಿದೆ
ಹುಡುಕು-ತ್ತಿದ್ದನೊ
ಹುಡುಕು-ತ್ತಿದ್ದರೆ
ಹುಡುಕು-ತ್ತಿದ್ದಾರೆ
ಹುಡುಕು-ತ್ತಿದ್ದಿರೊ
ಹುಡುಕು-ತ್ತಿದ್ದೆವೊ
ಹುಡುಕು-ತ್ತಿರು-ವನು
ಹುಡುಕು-ತ್ತಿರು-ವಿರೋ
ಹುಡುಕು-ತ್ತಿರು-ವುದು
ಹುಡುಕು-ವರು
ಹುಡುಕು-ವು-ದಕ್ಕೆ
ಹುಡುಕು-ವುದು
ಹುಡುಕು-ವು-ದೆಲ್ಲ
ಹುಡುಕು-ವುದೇ
ಹುಡುಗ
ಹುಡುಗನ
ಹುಡುಗ-ನನ್ನು
ಹುಡುಗ-ನಾಗಿದ್ದಾಗ
ಹುಡುಗ-ನಾ-ದರೋ
ಹುಡುಗ-ನಿಗೆ
ಹುಡುಗನು
ಹುಡು-ಗರ
ಹುಡುಗ-ರಂತೆ
ಹುಡುಗರು
ಹುಡು-ಗಾಟ-ವೆಂದು
ಹುಡುಗಿ
ಹುಡುಗಿಯ
ಹುಡುಗಿ-ಯೊಂದಿಗೆ
ಹುಡುಗಿ-ಯೊಡನೆ
ಹುಣಸೇ-ಮರ-ವನ್ನು
ಹುಣ್ಣಾಗು-ವುದಕ್ಕೆ
ಹುಣ್ಣು-ಗಳನ್ನೆಲ್ಲಾ
ಹುತ್ತ
ಹುತ್ತ-ವಿ-ರುವ
ಹುದುಗಿ-ಕೊಂಡಿ-ವೆಯೋ
ಹುದು-ಗಿದೆ
ಹುದುಗಿದ್ದರು
ಹುದು-ಗಿ-ರುವ
ಹುದು-ಗಿ-ರುವನು
ಹುದು-ಗಿ-ರುವುದು
ಹುರಿದ
ಹುರಿ-ದಾಗ
ಹುರಿದುಂಬಿ-ಸುತ್ತಿದ್ದ
ಹುರಿಯ-ಲಾರವು
ಹುರಿ-ಯಲ್ಲಿ
ಹುರುಳಿಲ್ಲ
ಹುರುಳಿಲ್ಲದ
ಹುರುಳಿಲ್ಲದ್ದು
ಹುಲಿ
ಹುಲಿ-ಗಳ
ಹುಲಿಯ
ಹುಲಿ-ಯಂತೆ
ಹುಲಿಯು
ಹುಲು
ಹುಲ್ಲನ್ನು
ಹುಲ್ಲಿಗೂ
ಹುಲ್ಲು
ಹುಳು
ಹುಳುವಿ-ನಂತೆ
ಹುಳುವಿ-ನಿಂದ
ಹುವು-ಗಳಿಂದ
ಹೂಡಿದ
ಹೂಡಿ-ರು-ವರು
ಹೂಡು-ವು-ದಕ್ಕೆ
ಹೂಳಿಸಿ-ಕೊಂಡು
ಹೂವನ್ನು
ಹೂವಿನ
ಹೂವಿ-ನಿಂದ
ಹೂವು
ಹೂವು-ಗಳ
ಹೂವು-ಗ-ಳನ್ನು
ಹೂವು-ಗಳು
ಹೃತ್ಪುರ್ವಕ-ವಾಗಿ
ಹೃತ್ಪೂರ್ವ-ಕ-ವಾಗಿ
ಹೃದಯ
ಹೃದ-ಯಕ್ಕೂ
ಹೃದ-ಯಕ್ಕೆ
ಹೃದ-ಯದ
ಹೃದಯ-ದಲ್ಲಿ
ಹೃದಯ-ದಲ್ಲಿದ್ದ
ಹೃದಯ-ದಲ್ಲಿಯೂ
ಹೃದಯ-ದಲ್ಲಿಯೇ
ಹೃದಯ-ದಲ್ಲಿರ
ಹೃದಯ-ದಲ್ಲಿ-ರುವ
ಹೃದಯ-ದಲ್ಲೂ
ಹೃದಯ-ದ-ವರಾ-ಗಿ-ರು-ವರೊ
ಹೃದಯ-ದಿಂದ
ಹೃದಯ-ವಂತಿಕೆ
ಹೃದಯ-ವಂತಿಕೆಯು
ಹೃದಯ-ವನ್ನು
ಹೃದಯ-ವಿತ್ತು
ಹೃದಯ-ವಿ-ದೆಯೋ
ಹೃದಯ-ವಿದ್ರಾವಕ
ಹೃದಯ-ವಿರು-ವುದು
ಹೃದ-ಯವು
ಹೃದ-ಯವೂ
ಹೃದ-ಯವೇ
ಹೃದಯ-ವೇನೊ
ಹೃದಯಾಂತ
ಹೃದಯಾಂತ-ರಾಳ-ದಲ್ಲಿ
ಹೃದಯಾಂತ-ರಾಳ-ದಲ್ಲಿಯೂ
ಹೃದಯಾಂತ-ರಾಳ-ದಲ್ಲಿ-ರುವ
ಹೃದಯಾಂತ-ರಾಳ-ದಲ್ಲೂ
ಹೃದಯಾಂತ-ರಾಳ-ದಲ್ಲೆ
ಹೃದಯಾಂತ-ರಾಳ-ದಲ್ಲೇ
ಹೃದಯಾಂತ-ರಾಳ-ದಿಂದ
ಹೃದ-ಯಾ-ಕಾಶ
ಹೃದಯಿ
ಹೃದಯೇ
ಹೃದ್ಗತ-ವಾಗಿ
ಹೆಂಗ
ಹೆಂಗ-ಸರು
ಹೆಂಗ-ಸಾಗಲೀ
ಹೆಂಗ-ಸಿಗೆ
ಹೆಂಗಸು
ಹೆಂಡ
ಹೆಂಡತಿ
ಹೆಂಡತಿಗೆ
ಹೆಂಡತಿ-ಯನ್ನು
ಹೆಂಡತಿ-ಯಲ್ಲಿ
ಹೆಂಡತಿ-ಯಿದ್ದಳು
ಹೆಂಡತಿಯೂ
ಹೆಂಡತಿ-ಯೊ-ಡನೆ
ಹೆಂಡಿ-ರನ್ನು
ಹೆಗಲ
ಹೆಚ್ಚದೆ
ಹೆಚ್ಚಲ್ಲ
ಹೆಚ್ಚಾಗ-ಬೇಕು
ಹೆಚ್ಚಾಗಿ
ಹೆಚ್ಚಾಗಿದ್ದು-ದ-ರಿಂದ
ಹೆಚ್ಚಾಗಿಯೇ
ಹೆಚ್ಚಾಗಿ-ರುವ
ಹೆಚ್ಚಾಗಿ-ರುವು-ದ-ರಿಂದ
ಹೆಚ್ಚಾಗಿ-ರು-ವುವು
ಹೆಚ್ಚಾಗಿಲ್ಲ
ಹೆಚ್ಚಾಗು
ಹೆಚ್ಚಾಗುತ್ತಿದೆ
ಹೆಚ್ಚಾಗು-ವುದು
ಹೆಚ್ಚಾಗು-ವುದು-ಅದ-ರಲ್ಲಿ
ಹೆಚ್ಚಾದ
ಹೆಚ್ಚಾ-ದರೆ
ಹೆಚ್ಚಾದು-ದನ್ನು
ಹೆಚ್ಚಿಗೆ
ಹೆಚ್ಚಿದಂತಾ-ಗುತ್ತದೆ
ಹೆಚ್ಚಿ-ದಂತೆ
ಹೆಚ್ಚಿ-ದರೆ
ಹೆಚ್ಚಿ-ದಷ್ಟೂ
ಹೆಚ್ಚಿ-ದುವು
ಹೆಚ್ಚಿದೆ
ಹೆಚ್ಚಿನ
ಹೆಚ್ಚಿನ-ದನ್ನು
ಹೆಚ್ಚಿನ-ದಲ್ಲ
ಹೆಚ್ಚಿ-ನದು
ಹೆಚ್ಚಿನ-ದೇನೂ
ಹೆಚ್ಚಿ-ನವು
ಹೆಚ್ಚಿ-ರುತ್ತದೆ
ಹೆಚ್ಚಿವು-ದವು
ಹೆಚ್ಚಿ-ಸ-ಬೇಕು-ನಾಲ್ಕು
ಹೆಚ್ಚಿಸ-ಬೇಡಿ
ಹೆಚ್ಚಿ-ಸಲೇ
ಹೆಚ್ಚಿಸಿ
ಹೆಚ್ಚಿಸಿ-ಕೊಳ್ಳ-ಬೇಕಾ-ಗುತ್ತದೆ
ಹೆಚ್ಚಿಸಿ-ಕೊಳ್ಳು-ತ್ತವೆ
ಹೆಚ್ಚಿಸಿ-ದರೆ
ಹೆಚ್ಚಿಸಿ-ದಷ್ಟೂ
ಹೆಚ್ಚಿಸುತ್ತವೆ
ಹೆಚ್ಚಿ-ಸು-ವುದು
ಹೆಚ್ಚು
ಹೆಚ್ಚುತ್ತಾ
ಹೆಚ್ಚುತ್ತಿದೆ
ಹೆಚ್ಚುತ್ತಿವೆ
ಹೆಚ್ಚು-ಪಾಲು
ಹೆಚ್ಚು-ವುದರ
ಹೆಚ್ಚು-ವುದು
ಹೆಚ್ಚು-ವು-ದೆಂದು
ಹೆಚ್ಚು-ವು-ದೆಂಬುದು
ಹೆಚ್ಚು-ಹೆಚ್ಚಾಗಿ
ಹೆಚ್ಚು-ಹೆಚ್ಚು
ಹೆಚ್ಚೂ
ಹೆಚ್ಚೇನೂ
ಹೆಜ್ಜೆ
ಹೆಜ್ಜೆ-ಗಳಷ್ಟು
ಹೆಜ್ಜೆಗೂ
ಹೆಜ್ಜೆ-ಯನ್ನು
ಹೆಜ್ಜೆ-ಯನ್ನೂ
ಹೆಜ್ಜೆಯೂ
ಹೆಜ್ಜೆ-ಹೆಜ್ಜೆಗೂ
ಹೆಣ
ಹೆಣ-ವನ್ನು
ಹೆಣೆದು-ಕೊಂಡಿ-ರು-ವೆವು
ಹೆಣೆಯು-ವೆವು
ಹೆಣ್ಣು
ಹೆಣ್ಣು-ಮಕ್ಕಳು
ಹೆಬ್ಬೆಟ್ಟನ್ನು
ಹೆಬ್ಬೆಟ್ಟಿನ
ಹೆಬ್ಬೆಟ್ಟಿ-ನಿಂದ
ಹೆಮ್ಮೆ
ಹೆಸ-ರನ್ನು
ಹೆಸ-ರನ್ನೇ
ಹೆಸ-ರಾದ-ವರು
ಹೆಸ-ರಾಯಿತು
ಹೆಸ-ರಿಂದ-ಲಾ-ದರೂ
ಹೆಸರಿದು
ಹೆಸರಿದೆ
ಹೆಸರಿ-ನಲ್ಲಿ
ಹೆಸರಿ-ನಿಂದ
ಹೆಸರಿ-ನಿಂದ-ಲಾ-ದರೂ
ಹೆಸರಿ-ನಿಂದಲೇ
ಹೆಸರು
ಹೆಸರು-ಅಂದರೆ
ಹೆಸರು-ಇ-ದನ್ನು
ಹೆಸರು-ಇಲ್ಲಿ
ಹೆಸರು-ಗಳ
ಹೆಸರು-ಗ-ಳನ್ನು
ಹೆಸರು-ಗಳಲ್ಲ
ಹೆಸರು-ಗಳಿಂದ
ಹೆಸರು-ಗಳು
ಹೆಸರು-ಗಳೆಲ್ಲ
ಹೆಸರು-ದೇಹ-ದಲ್ಲಿ
ಹೆಸರೇ
ಹೇ
ಹೇಗ-ಬೇಕೆಂದು
ಹೇಗಾ-ಗುತ್ತದೆ
ಹೇಗಿದೆ
ಹೇಗಿದೆಯೊ
ಹೇಗಿದೆಯೋ
ಹೇಗಿ-ರು-ವುದು
ಹೇಗೆ
ಹೇಗೆಂದರೆ
ಹೇಗೆ-ಎಂಬು-ದನ್ನು
ಹೇಗೆ-ನಿಸಿ-ತೆಂದು
ಹೇಗೊ
ಹೇಗೋ
ಹೇಡಿ-ಗಳಾಗಿ
ಹೇತು
ಹೇತುಃ
ಹೇತು-ಫಲಾಶ್ರಯಾಲಂಬನೈಃ
ಹೇತು-ರ-ವಿದ್ಯಾ
ಹೇತು-ವಲ್ಲ
ಹೇತು-ವಾ-ಗು-ವುದು
ಹೇಯಂ
ಹೇಯ-ವಾಗಿ
ಹೇಯ-ವಾಗಿ-ರ-ಲಿಲ್ಲ
ಹೇಯ-ವಾ-ದುದು
ಹೇಯ-ವಾದು-ದೆಂದು
ಹೇಯ-ವೆಂದು
ಹೇರಲು
ಹೇರಿದ್ದೀರಿ
ಹೇಳ
ಹೇಳ-ತೊಡಗಿ-ದನು
ಹೇಳದೆ
ಹೇಳನು
ಹೇಳ-ಬಲ್ಲ
ಹೇಳ-ಬಲ್ಲೆ
ಹೇಳ-ಬಹು-ದಾದರೆ
ಹೇಳ-ಬಹುದು
ಹೇಳ-ಬಹು-ದೆಂಬು-ದನ್ನು
ಹೇಳ-ಬಾ-ರದು
ಹೇಳ-ಬಾರ-ದೆಂದು
ಹೇಳ-ಬೇಕಾ
ಹೇಳ-ಬೇಕಾ-ಗಿದೆ
ಹೇಳ-ಬೇಕಾ-ಗಿಯೇ
ಹೇಳ-ಬೇಕಾ-ಗಿಲ್ಲ
ಹೇಳ-ಬೇಕಾ-ಗುತ್ತದೆ
ಹೇಳ-ಬೇಕಾ-ದರೆ
ಹೇಳ-ಬೇಕಾ-ಯಿತು
ಹೇಳ-ಬೇಕು
ಹೇಳ-ಬೇಕು-ಇ-ದನ್ನು-ರಹಸ್ಯ-ವನ್ನಾಗಿ
ಹೇಳ-ಬೇಕೆಂದು
ಹೇಳ-ಬೇಡ
ಹೇಳ-ಬೇಡಿ
ಹೇಳ-ಲಾ-ಗಿದೆ
ಹೇಳ-ಲಾ-ಗಿದೆ-ಕಾರ-ಣ-ವಿದ್ದಂತೆ
ಹೇಳ-ಲಾಗು-ವು-ದಿಲ್ಲ
ಹೇಳ-ಲಾಗು-ವು-ದಿಲ್ಲ-ಯಾವು-ದ-ರೊಂದಿಗೆ
ಹೇಳ-ಲಾರ-ಅ-ವನು
ಹೇಳ-ಲಾರೆ
ಹೇಳ-ಲಾರೆವು
ಹೇಳ-ಲಿಚ್ಛಿಸು-ವಂತೆ
ಹೇಳ-ಲಿಲ್ಲ
ಹೇಳಲು
ಹೇಳಲೂ
ಹೇಳ-ಹೇತುಃ
ಹೇಳಿ
ಹೇಳಿಕೆ
ಹೇಳಿ-ಕೆ-ಗಳಿಂದ
ಹೇಳಿ-ಕೆಗೆ
ಹೇಳಿ-ಕೆಯ
ಹೇಳಿ-ಕೆಯು
ಹೇಳಿ-ಕೆಯೂ
ಹೇಳಿ-ಕೊಂಡೆ-ಯಲ್ಲ
ಹೇಳಿ-ಕೊಡ-ಬೇಕೆಂದು
ಹೇಳಿ-ಕೊಳ್ಳ
ಹೇಳಿ-ಕೊಳ್ಳ-ಬಹುದು
ಹೇಳಿ-ಕೊಳ್ಳ-ಬೇಕು
ಹೇಳಿ-ಕೊಳ್ಳಿ
ಹೇಳಿ-ಕೊಳ್ಳುತ್ತಿತ್ತು
ಹೇಳಿ-ಕೊಳ್ಳುತ್ತಿ-ರು-ವರು
ಹೇಳಿ-ಕೊಳ್ಳುವ
ಹೇಳಿ-ಕೊಳ್ಳು-ವರು
ಹೇಳಿ-ಕೊಳ್ಳು-ವ-ವ-ನಲ್ಲಿ
ಹೇಳಿ-ಕೊಳ್ಳು-ವುದಲ್ಲದೆ
ಹೇಳಿ-ಕೊಳ್ಳು-ವುದು
ಹೇಳಿತು
ಹೇಳಿದ
ಹೇಳಿ-ದಂತೆ
ಹೇಳಿ-ದನು
ಹೇಳಿ-ದ-ನೆನ್ನು-ವರು
ಹೇಳಿ-ದರು
ಹೇಳಿ-ದರೂ
ಹೇಳಿ-ದರೆ
ಹೇಳಿ-ದ-ವನ
ಹೇಳಿ-ದಾಗ
ಹೇಳಿ-ದಾಗಲೂ
ಹೇಳಿ-ದಿರಾ
ಹೇಳಿ-ದಿ-ರೇನು
ಹೇಳಿ-ದು-ದನ್ನು
ಹೇಳಿದೆ
ಹೇಳಿ-ದೊ-ಡನೆ
ಹೇಳಿದ್ದನು
ಹೇಳಿದ್ದನ್ನು
ಹೇಳಿದ್ದರ
ಹೇಳಿದ್ದಾ-ಯಿತು
ಹೇಳಿದ್ದಾರೆ
ಹೇಳಿದ್ದು
ಹೇಳಿದ್ದೆನೊ
ಹೇಳಿದ್ದೇವೆ
ಹೇಳಿರು
ಹೇಳಿ-ರುತ್ತೇನೆ
ಹೇಳಿ-ರುವ
ಹೇಳಿ-ರು-ವಂತೆ
ಹೇಳಿ-ರು-ವನು
ಹೇಳಿ-ರು-ವರು
ಹೇಳಿ-ರು-ವುದರ
ಹೇಳಿ-ರು-ವು-ದ-ರಿಂದ
ಹೇಳಿ-ರು-ವುದು
ಹೇಳಿ-ರು-ವೆನು
ಹೇಳಿ-ರು-ವೆನೊ
ಹೇಳಿ-ರು-ವೆವು
ಹೇಳಿಲ್ಲ
ಹೇಳಿಸ-ಬೇಕೆಂದು
ಹೇಳಿಸ-ಬೇಕೆಂಬುದು
ಹೇಳು
ಹೇಳುತ್ತದೆ
ಹೇಳುತ್ತವೆ
ಹೇಳುತ್ತಾ
ಹೇಳುತ್ತಾನೆ
ಹೇಳುತ್ತಾ-ನೆಯೋ
ಹೇಳುತ್ತಾರೆ
ಹೇಳುತ್ತಾ-ರೆಈ
ಹೇಳುತ್ತಾ-ರೆಯೋ
ಹೇಳುತ್ತಿ
ಹೇಳುತ್ತಿದ್ದರು
ಹೇಳುತ್ತಿದ್ದು-ದನ್ನು
ಹೇಳುತ್ತಿದ್ದುದು
ಹೇಳುತ್ತಿದ್ದೆ
ಹೇಳುತ್ತಿದ್ದೆವು
ಹೇಳುತ್ತಿ-ರುವ
ಹೇಳುತ್ತಿ-ರು-ವರು
ಹೇಳುತ್ತಿ-ರು-ವುದನ್ನೇ
ಹೇಳುತ್ತಿ-ರುವುದು
ಹೇಳುತ್ತಿಲ್ಲ
ಹೇಳುತ್ತೀ-ರಲ್ಲ
ಹೇಳುತ್ತೀರಾ
ಹೇಳುತ್ತೀರಿ
ಹೇಳುತ್ತೀರೋ
ಹೇಳುತ್ತೇನೆ
ಹೇಳುತ್ತೇನೆಯೋ
ಹೇಳುತ್ತೇವೆ
ಹೇಳುವ
ಹೇಳು-ವಂತೆ
ಹೇಳು-ವನು
ಹೇಳು-ವರು
ಹೇಳು-ವರೋ
ಹೇಳು-ವ-ವನ
ಹೇಳು-ವ-ವನು
ಹೇಳು-ವವ-ರನ್ನು
ಹೇಳು-ವ-ವ-ರಿಗೆ
ಹೇಳು-ವ-ವರಿದ್ದಾರೆ
ಹೇಳು-ವ-ವರು
ಹೇಳು-ವ-ವರೆ
ಹೇಳು-ವಷ್ಟು
ಹೇಳು-ವಾಗಲೂ
ಹೇಳು-ವಿರಲ್ಲ
ಹೇಳುವು
ಹೇಳು-ವುದ-ಕ್ಕಿಂತ
ಹೇಳು-ವುದಕ್ಕೆ
ಹೇಳು-ವುದನ್ನು
ಹೇಳು-ವುದರ
ಹೇಳು-ವುದ-ರಲ್ಲಿ
ಹೇಳು-ವುದ-ರಿಂದ
ಹೇಳು-ವುದ-ರೊಂದಿಗೆ
ಹೇಳು-ವು-ದಾಗಲಿ
ಹೇಳು-ವು-ದಾದರೆ
ಹೇಳು-ವು-ದಿಲ್ಲ
ಹೇಳು-ವುದು
ಹೇಳು-ವುದು-ನಿರಪೇಕ್ಷ-ವಾದುದು
ಹೇಳು-ವುದೂ
ಹೇಳು-ವು-ದೆಂದು
ಹೇಳು-ವು-ದೆಂಬು-ದನ್ನು
ಹೇಳು-ವುದೇ
ಹೇಳು-ವು-ದೇನು
ಹೇಳು-ವು-ದೇ-ನೆಂದರೆ
ಹೇಳು-ವು-ದೇನೋ
ಹೇಳು-ವು-ದೊಂದು
ಹೇಳು-ವುದೋ
ಹೇಳು-ವುವು
ಹೇಳು-ವೆನು
ಹೇಳು-ವೆವು
ಹೇಳೋಣ
ಹೊಂಗರಿ-ಗಳ
ಹೊಂಚು
ಹೊಂದದೆ
ಹೊಂದ-ಬಹುದು
ಹೊಂದ-ಬಹು-ದೆಂಬು-ದನ್ನು
ಹೊಂದ-ಬೇಕಾ-ಗಿಲ್ಲ
ಹೊಂದ-ಬೇಕಾ-ದರೆ
ಹೊಂದ-ಬೇಕು
ಹೊಂದಲು
ಹೊಂದಿ
ಹೊಂದಿ-ಕೊಂಡಾಗ
ಹೊಂದಿ-ಕೊಂಡಿ
ಹೊಂದಿ-ಕೊಂಡಿ-ದ್ದರೆ
ಹೊಂದಿ-ಕೊಂಡು
ಹೊಂದಿ-ಕೊಳ್ಳ
ಹೊಂದಿ-ಕೊಳ್ಳದೇ
ಹೊಂದಿ-ಕೊಳ್ಳ-ಬೇಕು
ಹೊಂದಿ-ಕೊಳ್ಳು-ವುದಕ್ಕೆ
ಹೊಂದಿ-ಕೊಳ್ಳು-ವುದಲ್ಲ
ಹೊಂದಿ-ಕೊಳ್ಳು-ವುದು
ಹೊಂದಿತು
ಹೊಂದಿದ
ಹೊಂದಿ-ದಂತೆ
ಹೊಂದಿ-ದರೆ
ಹೊಂದಿ-ದವು
ಹೊಂದಿ-ದಾಗ-ಲೆಲ್ಲ
ಹೊಂದಿದೆ
ಹೊಂದಿ-ದೆಯೋ
ಹೊಂದಿದ್ದ
ಹೊಂದಿದ್ದರೆ
ಹೊಂದಿ-ರದ
ಹೊಂದಿರ-ಬೇಕು
ಹೊಂದಿರು-ವುದು
ಹೊಂದಿಲ್ಲವೋ
ಹೊಂದಿ-ಸಲು
ಹೊಂದು
ಹೊಂದು-ತ್ತದೆ
ಹೊಂದು-ತ್ತವೆ
ಹೊಂದು-ತ್ತವೆಯೇ
ಹೊಂದು-ತ್ತಿರುವ
ಹೊಂದು-ತ್ತಿರು-ವುದು
ಹೊಂದುತ್ತೇನೆ
ಹೊಂದುವ
ಹೊಂದು-ವುದ-ಕ್ಕಾಗಿ-ರುವ
ಹೊಂದು-ವುದ-ರಲ್ಲಿ
ಹೊಂದು-ವುದ-ರಿಂದ
ಹೊಂದು-ವುದಾ-ಗಲೀ
ಹೊಂದು-ವುದಿಲ್ಲ
ಹೊಂದು-ವುದು
ಹೊಂದು-ವುದೇ
ಹೊಂದು-ವುವು
ಹೊಂದು-ವೆನು
ಹೊಕ್ಕ-ಮೇಲೆ
ಹೊಕ್ಕುಳಿನ
ಹೊಗ-ಲಾರದ
ಹೊಗಳ-ಬೇಕಾ-ಗಿಲ್ಲ
ಹೊಗಳ-ಬೇಕು
ಹೊಗಳಿ
ಹೊಗಳಿಕೆ-ಗಳಾ-ವು-ದನ್ನೂ
ಹೊಗಳು-ತ್ತಿರ-ಬೇಕೆಂದು
ಹೊಗಳು-ವುದಲ್ಲದೆ
ಹೊಗಳು-ವುದು
ಹೊಗೆ-ಯಂತೆ
ಹೊಟ್ಟೆಕಿಚ್ಚನ್ನೂ
ಹೊಟ್ಟೆ-ಯಲ್ಲಿ
ಹೊಡೆ-ದಂತೆ
ಹೊಡೆ-ದರೆ
ಹೊಡೆ-ದಾಗ
ಹೊಡೆದು
ಹೊಡೆಯು-ತ್ತಿರು-ವಂತೆ
ಹೊಡೆಯು-ವಂತೆ
ಹೊಡೆಯು-ವನು
ಹೊಡೆಯು-ವರು
ಹೊಡೆಯು-ವವ-ನಾದರೆ
ಹೊಡೆಯು-ವು-ದನ್ನು
ಹೊಡೆಯು-ವುದು
ಹೊಡೆಯು-ವುದೇ
ಹೊಣೆ
ಹೊಣೆ-ಗಾರ-ನಲ್ಲ
ಹೊಣೆ-ಗಾರ-ರನ್ನಾಗಿ
ಹೊಣೆ-ಗಾರ-ರಾಗಿದ್ದು-ದನ್ನು
ಹೊಣೆ-ಗಾರ-ರಾಗು-ವೆವು
ಹೊಣೆ-ಗಾರ-ರಾಗು-ವೆವೊ
ಹೊಣೆ-ಗಾರ-ರಾದಾಗ
ಹೊಣೆ-ಗಾರರು
ಹೊಣೆ-ದೇವರು
ಹೊತ್ತಿಗೆ
ಹೊತ್ತಿ-ರು-ವಿರೋ
ಹೊತ್ತು
ಹೊತ್ತು-ಕೊಂಡು
ಹೊದಿಕೆ
ಹೊಮ್ಮಿ
ಹೊಮ್ಮುತ್ತದೆ
ಹೊಮ್ಮು-ವುದಕ್ಕೆ
ಹೊರ
ಹೊರಕ್ಕೆ
ಹೊರ-ಗಡೆ
ಹೊರ-ಗಡೆಯೋ
ಹೊರ-ಗಾದರೂ
ಹೊರಗಿ
ಹೊರ-ಗಿಡು-ವುದಲ್ಲ
ಹೊರ-ಗಿನ
ಹೊರ-ಗಿನದು
ಹೊರ-ಗಿನವು
ಹೊರ-ಗಿನಿಂದ
ಹೊರ-ಗಿರುವ
ಹೊರಗೂ
ಹೊರಗೆ
ಹೊರ-ಗೆಡಹು
ಹೊರ-ಗೆಡಹು-ತ್ತದೆ
ಹೊರ-ಗೆಡಹು-ತ್ತೇವೆ
ಹೊರ-ಗೆಡಹು-ವನು
ಹೊರ-ಗೆಸೆಯು-ವುದ
ಹೊರಟ
ಹೊರ-ಟನು
ಹೊರ-ಟವೊ
ಹೊರ-ಟಿತು
ಹೊರಟಿ-ರುವೆ
ಹೊರಟು
ಹೊರಟು-ಹೋಗು-ತ್ತದೆ
ಹೊರಟೆ
ಹೊರ-ತಾದ
ಹೊರ-ತಾದ-ವರು
ಹೊರತು
ಹೊರತು-ಪಡಿಸಿ
ಹೊರ-ದೋರ-ಬೇಕಾ-ಗಿತ್ತೊ
ಹೊರ-ಪಡಿಸು-ವುದ-ರಿಂದ
ಹೊರ-ಪದರ
ಹೊರ-ಬಂದವು
ಹೊರ-ಬಂದಿದೆ
ಹೊರ-ಬಂದು
ಹೊರ-ಬರು-ವಂತೆ
ಹೊರ-ಬರು-ವರು
ಹೊರ-ಬರು-ವುವು
ಹೊರ-ಬಾಗಿ-ಲಿನ
ಹೊರ-ಭಾಗ
ಹೊರ-ಭಾಗ-ವನ್ನು
ಹೊರ-ಹೊಮ್ಮಿ-ದಂತೆ
ಹೊರ-ಹೊಮ್ಮಿ-ರು-ವುದು
ಹೊರ-ಹೊಮ್ಮು-ವುದು
ಹೊರ-ಹೋಗಿ
ಹೊರಿಸು-ವರು
ಹೊರುವ-ವ-ರೆಲ್ಲ
ಹೊರುವು-ದಕ್ಕೆ
ಹೊರೆ
ಹೊರೆ-ಯನ್ನು
ಹೊರೆ-ಯಾಗಿ
ಹೊರೆ-ಯಿಂದ
ಹೊಲ
ಹೊಲಿ-ಸುವು
ಹೊಳೆದ
ಹೊಳೆಯ-ಬೇಕು
ಹೊಳೆಯಿತು
ಹೊಳೆಯಿತೋ
ಹೊಳೆಯು-ತ್ತದೆ
ಹೊಳೆಯು-ತ್ತಿದೆ
ಹೊಳೆಯು-ತ್ತಿರ-ಬಹುದು
ಹೊಳೆಯು-ವು-ದನ್ನು
ಹೊಳೆಯು-ವು-ದಿಲ್ಲ
ಹೊಳೆಯು-ವುದು
ಹೊಳೆಯು-ವುದು-ಆಗ
ಹೊಳೆಯು-ವುದೇ
ಹೊಳೆಯು-ವುವು
ಹೊಳ್ಳೆ-ಗಳನ್ನೂ
ಹೊಳ್ಳೆ-ಯಿಂದ
ಹೊಸ
ಹೊಸ-ತಾ-ಗಿದೆ
ಹೊಸ-ದನ್ನು
ಹೊಸ-ದಾಗಿ
ಹೊಸ-ದಾಗಿ-ರುವ
ಹೊಸ-ದಾಗಿ-ರುವುದು
ಹೊಸ-ದಾ-ವುದೂ
ಹೊಸ-ದಿಕ್ಕಿ-ನಲ್ಲಿ
ಹೊಸ-ದಿರ-ಲಾರದು
ಹೊಸ-ದಿಲ್ಲ
ಹೊಸ-ದೊಂದನ್ನು
ಹೊಸ-ದೊಂದು
ಹೊಸ-ಬರು
ಹೊಸವು
ಹೊಸ-ಹೊಸ
ಹೋಗ
ಹೋಗ-ಕೂಡದು
ಹೋಗ-ದಂತೆ
ಹೋಗದೆ
ಹೋಗ-ಬಲ್ಲ
ಹೋಗ-ಬಲ್ಲದು
ಹೋಗ-ಬಲ್ಲನು
ಹೋಗ-ಬಲ್ಲುದು
ಹೋಗ-ಬಹುದಾ-ದಷ್ಟು
ಹೋಗ-ಬಹುದು
ಹೋಗ-ಬಾರದು
ಹೋಗ-ಬೇಕಾ-ಗಿದೆ
ಹೋಗ-ಬೇಕಾ-ಗಿಲ್ಲ
ಹೋಗ-ಬೇಕಾ-ಗು-ವುದು
ಹೋಗ-ಬೇಕಾ-ದರೆ
ಹೋಗ-ಬೇಕು
ಹೋಗ-ಬೇಕೆಂದು
ಹೋಗ-ಬೇಕೆಂಬು-ದನ್ನು
ಹೋಗ-ಬೇಕೆಂಬುದು
ಹೋಗ-ಬೇಕೆನ್ನು-ವರು
ಹೋಗ-ಬೇಕೋ
ಹೋಗ-ಬೇಡ
ಹೋಗ-ಬೇಡಿ
ಹೋಗ-ಲಾಗ-ದಂತೆ
ಹೋಗ-ಲಾಗ-ದ್ದಿದರೂ
ಹೋಗ-ಲಾಡಿಸು-ವುದಕ್ಕೆ
ಹೋಗ-ಲಾಡಿಸು-ವುದು
ಹೋಗ-ಲಾರ
ಹೋಗ-ಲಾರದು
ಹೋಗ-ಲಾರರು
ಹೋಗ-ಲಾರರೊ
ಹೋಗ-ಲಾರವು
ಹೋಗ-ಲಾರಿರಿ
ಹೋಗ-ಲಾರೆವು
ಹೋಗಲಿ
ಹೋಗ-ಲಿಚ್ಛಿ-ಸು-ವನು
ಹೋಗ-ಲಿಲ್ಲ
ಹೋಗಲು
ಹೋಗ-ಲೆತ್ನಿ-ಸು-ವನು
ಹೋಗ-ಲೆತ್ನಿ-ಸು-ವುದು
ಹೋಗಲೇ
ಹೋಗಲೇ-ಬೇಕಾ-ಗಿದೆ
ಹೋಗಲೇ-ಬೇಕು
ಹೋಗಿ
ಹೋಗಿದೆ
ಹೋಗಿದ್ದ-ನೆಂದು
ಹೋಗಿ-ಬೇಕೆಂದು
ಹೋಗಿ-ರುತ್ತದೆ
ಹೋಗಿ-ರುವ
ಹೋಗಿ-ರುವನು
ಹೋಗಿ-ರುವನೊ
ಹೋಗಿ-ರು-ವರು
ಹೋಗಿ-ರುವುದು
ಹೋಗಿಲ್ಲ
ಹೋಗಿವೆ
ಹೋಗು
ಹೋಗು-ತ್ತದೆ
ಹೋಗು-ತ್ತಲೂ
ಹೋಗು-ತ್ತಲೇ
ಹೋಗು-ತ್ತವೆ
ಹೋಗು-ತ್ತಾನೆ
ಹೋಗು-ತ್ತಾರೆ
ಹೋಗು-ತ್ತಿತ್ತು
ಹೋಗು-ತ್ತಿದೆ
ಹೋಗು-ತ್ತಿದ್ದ
ಹೋಗು-ತ್ತಿದ್ದರು
ಹೋಗು-ತ್ತಿದ್ದರೆ
ಹೋಗು-ತ್ತಿದ್ದು
ಹೋಗು-ತ್ತಿರು-ತ್ತೇನೆ
ಹೋಗು-ತ್ತಿರುವ
ಹೋಗು-ತ್ತಿರು-ವನು
ಹೋಗು-ತ್ತಿರು-ವನೊ
ಹೋಗು-ತ್ತಿರು-ವರು
ಹೋಗು-ತ್ತಿರು-ವಾಗ
ಹೋಗು-ತ್ತಿರು-ವಿರಿ
ಹೋಗು-ತ್ತಿರು-ವು-ದನ್ನು
ಹೋಗು-ತ್ತಿರು-ವುದು
ಹೋಗು-ತ್ತಿರು-ವುವು
ಹೋಗು-ತ್ತಿರುವೆ
ಹೋಗು-ತ್ತಿರು-ವೆನು
ಹೋಗು-ತ್ತಿರು-ವೆವು
ಹೋಗು-ತ್ತಿಲ್ಲ
ಹೋಗು-ತ್ತಿವೆ
ಹೋಗು-ತ್ತೇನೆ
ಹೋಗು-ತ್ತೇವೆ
ಹೋಗುವ
ಹೋಗು-ವಂತೆ
ಹೋಗು-ವಂಥದು
ಹೋಗು-ವನು
ಹೋಗು-ವನೋ
ಹೋಗು-ವರು
ಹೋಗು-ವ-ವನೂ
ಹೋಗು-ವ-ವ-ರಾರು
ಹೋಗು-ವಾಗ
ಹೋಗುವು
ಹೋಗು-ವು-ದಕ್ಕಾಗಿಯೇ
ಹೋಗು-ವು-ದಕ್ಕೆ
ಹೋಗು-ವು-ದಕ್ಕೇ
ಹೋಗು-ವು-ದನ್ನು
ಹೋಗು-ವು-ದನ್ನೂ
ಹೋಗು-ವು-ದನ್ನೆ
ಹೋಗು-ವು-ದರ
ಹೋಗು-ವು-ದ-ರಲ್ಲಿ
ಹೋಗು-ವು-ದ-ರಿಂದ
ಹೋಗು-ವು-ದರೊ-ಳ-ಗಾಗಿ
ಹೋಗು-ವು-ದ-ರೊ-ಳಗೆ
ಹೋಗು-ವು-ದಾಗಲಿ
ಹೋಗು-ವು-ದಿಲ್ಲ
ಹೋಗು-ವು-ದಿಲ್ಲವೋ
ಹೋಗು-ವುದು
ಹೋಗು-ವುದೂ
ಹೋಗು-ವುದೆ
ಹೋಗು-ವು-ದೆಂದರೆ
ಹೋಗು-ವು-ದೆಂದ-ರೇನು
ಹೋಗು-ವುದೆಂಬು-ದನ್ನು
ಹೋಗು-ವು-ದೆಲ್ಲಿಗೆ
ಹೋಗು-ವುದೇ
ಹೋಗು-ವುರು
ಹೋಗು-ವುವು
ಹೋಗುವೆ
ಹೋಗು-ವೆನು
ಹೋಗು-ವೆವು
ಹೋಗು-ವೆವೊ
ಹೋಟೆಲಿ-ನಲ್ಲಿ
ಹೋದ
ಹೋದಂತೆ
ಹೋದಂತೆಯೇ
ಹೋದಂತೆಲ್ಲ
ಹೋದನು
ಹೋದನು-ದೇವತೆ-ಗಳು
ಹೋದ-ನೆಂದು
ಹೋದನೋ
ಹೋದ-ಮೇಲೆ
ಹೋದರು
ಹೋದರೂ
ಹೋದರೆ
ಹೋದವ-ನಲ್ಲ
ಹೋದವ-ರಿಗೆ
ಹೋದವು
ಹೋದಷ್ಟೂ
ಹೋದಾ
ಹೋದಾಗ
ಹೋದಿರಿ
ಹೋದೆ
ಹೋಮ
ಹೋಮಿಯೋ-ಪತಿ
ಹೋಯಿತು
ಹೋಯಿ-ತೆಂದು
ಹೋರಾ
ಹೋರಾಟ
ಹೋರಾಟ-ಗ-ಳನ್ನು
ಹೋರಾಟ-ಗಳು
ಹೋರಾಟ-ಗಳು-ಇವು-ಗಳಲ್ಲಿ
ಹೋರಾಟ-ಗಳೂ
ಹೋರಾಟದ
ಹೋರಾಟ-ದಂತೆ
ಹೋರಾಟ-ದಲ್ಲಿ
ಹೋರಾಟ-ವನ್ನು
ಹೋರಾಟ-ವಲ್ಲದೆ
ಹೋರಾಟ-ವಾಗು-ವುದು
ಹೋರಾಟವೂ
ಹೋರಾಟ-ವೆಲ್ಲ
ಹೋರಾಟವೇ
ಹೋರಾಡ
ಹೋರಾಡ-ಬಲ್ಲರೋ
ಹೋರಾಡ-ಬಲ್ಲೆವು
ಹೋರಾಡ-ಬಹುದು
ಹೋರಾಡ-ಬೇಕು
ಹೋರಾಡ-ಲಾರೆ
ಹೋರಾಡಲು
ಹೋರಾಡಲೇ-ಬೇಕು
ಹೋರಾಡಿ
ಹೋರಾಡಿ-ದರೆ
ಹೋರಾಡಿ-ರು-ವರೋ
ಹೋರಾಡು
ಹೋರಾಡು-ತ್ತಿದೆ
ಹೋರಾಡು-ತ್ತಿರುವ
ಹೋರಾಡು-ತ್ತಿರು-ವನೋ
ಹೋರಾಡು-ತ್ತಿರು-ವುದಕ್ಕೆ
ಹೋರಾಡು-ತ್ತಿರು-ವುದು
ಹೋರಾಡು-ತ್ತಿರು-ವುವು
ಹೋರಾಡು-ತ್ತಿರು-ವುವೊ
ಹೋರಾಡು-ತ್ತಿರು-ವೆವು
ಹೋರಾಡು-ತ್ತೇವೆ
ಹೋರಾಡುವ
ಹೋರಾಡು-ವನು
ಹೋರಾಡು-ವರು
ಹೋರಾಡು-ವವ-ರಾರು
ಹೋರಾಡು-ವು-ದಕ್ಕೆ
ಹೋರಾಡು-ವುದು
ಹೋರಾಡು-ವೆನು
ಹೋರಾಡು-ವೆವು
ಹೋರಾಡು-ವೆವೊ
ಹೋಲಿ-ಕೆಯ
ಹೋಲಿ-ದರೂ
ಹೋಲಿಸ-ಬೇಕು
ಹೋಲಿ-ಸಲು
ಹೋಲಿಸಿ
ಹೋಲಿಸಿ-ದರೆ
ಹೋಲಿಸಿ-ದಲ್ಲದೆ
ಹೋಲಿಸಿ-ದಾಗ
ಹೋಲಿಸಿ-ರು-ವುದು
ಹೋಲಿಸು-ವುದಕ್ಕೆ
ಹೋಲಿಸು-ವುದರ
ಹೋಲಿಸು-ವೆನು
ಹೋಲು-ತ್ತದೆ
ಹೋಲುವ
ಹೋಲು-ವು-ದಿಲ್ಲ
ಹೌದಾದರೆ
ಹೌದು
ಹ್ಯಾರಿಸ್
ಹ್ಯೇಷಾ
ಹ್ಲಾದ-ಪರಿ-ತಾಪ-ಫಲಾಃ
}
