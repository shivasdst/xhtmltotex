
\chapter{ವಿವೇಕವಾಣಿ}

\section{ಭಾರತೀಯರಿಗೆ}

ಆರ್ಯಮಾತೆಯ ಅಮೃತಪುತ್ರರಿರಾ, ಮಹರ್ಷಿಕುಲಸಂಜಾತ ಧನ್ಯಾತ್ಮ ರಿರಾ, ಮರೆಯದಿರಿ–ನಿಮ್ಮ ಆದರ್ಶ ಮಾತೆಯರೆಂದರೆ ಸೀತಾ, ಸಾವಿತ್ರಿ, ದಮಯಂತಿಯರು! ಮರೆಯದಿರಿ–ನಿಮ್ಮ ಆರಾಧನೆಯ ಮಹೇಶ್ವರನು ತ್ಯಾಗಿ-ಕುಲಚೂಡಾಮಣಿ, ವೈರಾಗ್ಯನಿಧಿ, ಉಮಾನಾಥ ಶಂಕರ! ನಿಮ್ಮ ವಿವಾಹ ಸಂಪತ್ತು ಬರಿಯ ಇಂದ್ರಿಯ ಭೋಗಕ್ಕಲ್ಲ, ನಿಮ್ಮ ಸ್ವಾರ್ಥಸುಖ ಕ್ಕಲ್ಲ. ಮರೆಯದಿರಿ! ಮಾತೆಯ ಬಲಿ ಪೀಠದ ಮೇಲೆ ಯಜ್ಞಶಿಶುಗಳಾಗಿ ಜನ್ಮವೆತ್ತಿರುವಿರಿ, ಮರೆಯದಿರಿ–ಶೂದ್ರರು, ಅಂತ್ಯಜರು, ದರಿದ್ರರು, ಅಜ್ಞಾನಿಗಳು, ಚಂಡಾಲ, ಚಮ್ಮಾರರೆಲ್ಲರೂ ನಿಮ್ಮ ರಕ್ತಬಂಧುಗಳಾದ ಸೋದರರು! ವೀರಾತ್ಮರೆ, ನೆಚ್ಚುಗೆಡದಿರಿ; ಧೀರರಾಗಿ. ಭಾರತೀಯರು ನಾವೆಂಬ ಹೆಮ್ಮೆಯಿಂದ ಸಾರಿ ಹೇಳಿ. ಅಜ್ಞಾನಿ ಭಾರತೀಯರೆಮ್ಮ ಸಹೋದರರು; ದರಿದ್ರ ದೀನ ಅನಾಥ ಭಾರತೀಯರು ನಮ್ಮ ಸಹೋದರರು; ಬ್ರಾಹ್ಮಣ ಭಾರತೀಯರೆಮ್ಮ ಸಹೋದರರು, ಛಿದ್ರಮಲಿನ ವಸನಧಾರಿ ಯಾದರೂ, ಅಭಿಮಾನಪೂರ್ಣವಾದ, ಮೇಘ ಗಂಭೀರವಾಣಿಯಿಂದ ಘೋಷಿಸಿರಿ–ಆರ್ಯರೆಮ್ಮ ಬಂಧುಗಳು ಆರ್ಯರೆಮ್ಮ ಪ್ರಾಣ. ಆರ್ಯದೇವದೇವತೆಗಳೆಲ್ಲರೂ ನಮ್ಮ ದೇವರು. ಆರ್ಯರ ಸಮಾಜ ನಮ್ಮ ಶೈಶವದ ತೊಟ್ಟಿಲು, ತಾರುಣ್ಯದ ಉಯ್ಯಾಲೆ, ಯೌವನದ ಉದ್ಯಾನ, ವೃದ್ಧಾ ಪ್ಯದ ವಾರಣಾಸಿ. ಸಹೋದರರೆ, ಇಂತೆಂದು ಗಾನ ಮಾಡಿರಿ: ಆರ್ಯ ಭೂಮಿಯೇ ನಮ್ಮ ಸ್ವರ್ಗ, ಆರ್ಯ ಭೂಮಿಯೇ ನಮಗೆ ಪರಂಧಾಮ, ಆರ್ಯಮಾತೆಯ ಶುಭವೇ ನಮ್ಮ ಶುಭ, ಆಕೆಯ ಸುಖವೇ ನಮ್ಮ ಸುಖ. ದಿವಾರಾತ್ರೆಯೂ ಇದು ನಿಮ್ಮ ಪ್ರಾರ್ಥನೆಯಾಗಿರಲಿ: ಹೇ ಗೌರಿನಾಥ, ಹೇ ಜಗನ್ಮಾತೆ, ಪೌರುಷವನ್ನು ನಮಗೆ ದಯಪಾಲಿಸು! ಹೇ ಸರ್ವಶಕ್ತಿಶಾಲಿನಿ, ನಮ್ಮ ದೌರ್ಬಲ್ಯವನ್ನು ದಹಿಸು. ಕ್ಲೈಬ್ಯವನ್ನು ಪರಿಹರಿಸು. ನಮ್ಮನ್ನು ಪೌರುಷವಂತರನ್ನಾಗಿ ಮಾಡು. ಪುರುಷಸಿಂಹರನ್ನಾಗಿ ಮಾಡು.

ಹೇ ಭಾರತ ಭೂಮಿ, ಇದೇ ನಿನ್ನ ಮಹೋತ್ಪಾತ. ಪಾಶ್ಚಾತ್ಯರನ್ನು ಅನುಕರಿಸಬೇಕೆಂಬ ಮೋಹ ನಿಮ್ಮನ್ನು ಪ್ರಬಲವಾಗಿ ಆಕ್ರಮಿಸುತ್ತಿದೆ. ಯಾವುದು ಒಳ್ಳೆಯದು, ಯಾವುದು ಕೆಟ್ಟದ್ದು ಎಂಬುದನ್ನು ಯುಕ್ತ ಅಯುಕ್ತ ವಿಮರ್ಶೆ ಅಥವಾ ಶಾಸ್ತ್ರಜ್ಞಾನದಿಂದ ನಿರ್ಧರಿಸುತ್ತಿಲ್ಲ. ಯಾವ ಭಾವ ಅಥವಾ ಆಚಾರವನ್ನು ಬಿಳಿಯ ಜನರು ಹೊಗಳುತ್ತಾರೆಯೋ ಅವು ಒಳ್ಳೆಯವು. ಯಾವುದನ್ನು ಅವರು ನಿಂದಿಸುವರೋ ಅದು ಕೆಟ್ಟದ್ದು! ಅಯ್ಯೋ! ಇದಕ್ಕಿಂತ ಹೆಚ್ಚು ನಿಮ್ಮ ಮೌಢ್ಯತೆಗೆ ಕುರುಹು ಏನು ಬೇಕಾಗಿರುವುದು. 

ನಮ್ಮ ದೇಶದ ಸಂಘ ಸಂಸ್ಥೆಗಳು ಘನ ಆದರ್ಶವನ್ನು ಮೀರಿರುವ ದೇಶ ಪ್ರಪಂಚದಲ್ಲಿ ಮತ್ತೆಲ್ಲಿಯೂ ಇಲ್ಲವೆಂಬುದನ್ನು ನೆನಪಿನಲ್ಲಿಡಿ. ಎಲ್ಲಾ ದೇಶದಲ್ಲಿಯೂ ನಾನಾ ವರ್ಣಗಳನ್ನು ನೋಡಿರುವೆನು. ಆದರೆ ನಮ್ಮಲ್ಲಿರುವ ಮಹದಾಲೋಚನೆ, ಘನಯೋಜನೆ, ಉದ್ದೇಶ ಮತ್ತೆಲ್ಲಿಯೂ ಇಲ್ಲ. ವರ್ಣ ದಿಂದ ಪಾರಾಗಲು ಅಶಕ್ತರಾದರೆ, ಡಾಲರಿನ ಮೇಲೆ ನಿಂತ ವರ್ಣಕ್ಕಿಂತ ಪವಿತ್ರತೆ, ಸಂಸ್ಕೃತಿ, ತ್ಯಾಗದ ಮೇಲೆನಿಂತ ವರ್ಣವಿರಲಿ. ಆದಕಾರಣ, ಶಪಿ ಸುವ ಪದಗಳು ನಿಮ್ಮ ಬಾಯಿಂದ ಬೀಳದಿರಲಿ. ನಿಮ್ಮ ಬಾಯನ್ನು ಮುಚ್ಚಿ, ಹೃದಯ ತೆರೆಯಲಿ. ಈ ಜನರ ಮತ್ತು ಪ್ರಪಂಚದ ವಿಮೋಚನೆಗೆ ಪ್ರಯತ್ನಿಸಿ. ಈ ಭಾರವೆಲ್ಲ ನಿಮ್ಮ ಮೇಲೆ ಬಿದ್ದಿದೆ ಎಂದು ಪ್ರತಿಯೊಬ್ಬರೂ ಭಾವಿಸಿ. ವೇದಾಂತದ ಜೀವನ ಮತ್ತು ಸಂದೇಶವನ್ನು ಪ್ರತಿಯೊಂದು ಮನೆಗೂ ನೀಡಿ. ಪ್ರತಿಯೊಂದು ಜೀವಿಯಲ್ಲಿಯೂ ಸುಪ್ತವಾಗಿರುವ ಪವಿತ್ರತೆಯನ್ನು ಜಾಗ್ರತ ಗೊಳಿಸಿ. ಇದರಿಂದ ಬರುವ ಪ್ರತಿಫಲ ಕಡಿಮೆಯಾದರೂ, ಮಹಾಕಾರ್ಯ ಕ್ಕಾಗಿ ಬಾಳಿ. ದುಡಿದು ಮಡಿದೆ ಎಂಬ ತೃಪ್ತಿಯಾದರೂ ಬರುವುದು. ಈ ಉದ್ದೇಶ ಸಫಲವಾದರೆ, ಇದರ ಮೇಲೆ ಮಾನವಕೋಟಿಯ ಇಂದಿನ ಮತ್ತು ಮುಂದಿನ ಕಲ್ಯಾಣ ನಿಂತಿದೆ. ಅದು ಯಾರಿಂದ ಅದರೇನಂತೆ.

ರಾಜಕೀಯ ಮತ್ತು ಸಮಾಜಸಾಮ್ಯವಾದದ ಭಾವನೆಯ ಬೀಜವನ್ನು ಬಿತ್ತುವುದಕ್ಕೆ ಮುಂಚೆ, ಆಧ್ಯಾತ್ಮಿಕ ಭಾವನೆಯನ್ನು ದೇಶದ ಮೇಲೆ ಹೊನ ಲಾಗಿ ಹರಿಸಬೇಕು. ಇದೇ ನಾವು ಮಾಡಬೇಕಾದ ಪ್ರಥಮ ಕರ್ತವ್ಯ. ನಮ್ಮ ಉಪನಿಷತ್ತಿನಲ್ಲಿ, ಶಾಸ್ತ್ರದಲ್ಲಿ, ಪುರಾಣದಲ್ಲಿ ಇರುವ ಅದ್ಭುತ ಸತ್ಯಗಳನ್ನು ಗ್ರಂಥಗಳಿಂದ ಮಠಗಳಿಂದ, ವನಕುಟೀರಗಳಿಂದ, ಈ ವಿದ್ಯೆ ನಮಗೇ ಮೀಸ ಲಾಗಿದೆ ಎಂದು ತಿಳಿವ ಪಂಗಡಗಳಿಂದ ಹೊರಗೆ ತರಬೇಕು. ದೇಶದಲ್ಲೆಲ್ಲಾ ಇದು ಹರಡಬೇಕು. ಉತ್ತರದಿಂದ ದಕ್ಷಿಣಕ್ಕೆ, ಪೂರ್ವದಿಂದ ಪಶ್ಚಿಮಕ್ಕೆ, ಹಿಮಾಲಯದಿಂದ ಕನ್ಯಾಕುಮಾರಿಯವರೆಗೆ, ಸಿಂಧುವಿನಿಂದ ಬ್ರಹ್ಮಪುತ್ರದ ವರೆಗೆ ಬೆಂಕಿಯ ಹೊಳೆ ಹರಿಯುವಂತೆ ಮಾಡಬೇಕು.

ಭರತಖಂಡ ನಾಶವಾಗುವುದೇ? ಆಗ ಪ್ರಪಂಚದಿಂದ ಆಧ್ಯಾತ್ಮಿಕತೆಯೆಲ್ಲ ಮಾಯವಾಗುವುದು: ನೈತಿಕೋನ್ನತಿಯೆಲ್ಲ ನಿರ್ನಾಮವಾಗುವುದು; ಧಾರ್ಮಿಕ ಭಾವಗಳಿಗೆ ತೋರುವ ಹೃತ್ಪೂರ್ವಕ ಸಹಾನುಭೂತಿ ಮಾಯವಾಗುವುದು; ಎಲ್ಲಾ ಭವ್ಯ ಉದ್ದೇಶಗಳೂ ಮಾಯವಾಗುವುವು. ಅದರ ಸ್ಥಳದಲ್ಲಿ ಕಾಮ ಮತ್ತು ಭೋಗ ಎಂಬ ಗಂಡು ಮತ್ತು ಹೆಣ್ಣು ದೇವರು ಆಳುವರು; ಹಣವೇ ಪುರೋಹಿತರಲ್ಲಿ; ಮೋಸ, ಬಲತ್ಕಾರ, ಸ್ಪರ್ಧೆಯೇ ಆಚಾರಗಳಲ್ಲಿ; ಜೀವಾತ್ಮನೇ ಯಜ್ಞಪಶುವಲ್ಲಿ; ಇಂತಹ ಪ್ರಸಂಗವೆಂದಿಗೂ ಬರಲಾರದು. ಕೆಲಸಮಾಡುವ ಶಕ್ತಿಗಿಂತ, ಅನುಭವಿಸುವ ಶಕ್ತಿ ಹೆಚ್ಚು; ಕೋಟಿಪಾಲು ಹೆಚ್ಚು. ದ್ವೇಷಕ್ಕಿಂತ ಪ್ರೀತಿ ಶಕ್ತಿ ಅದ್ಭುತ ಪರಿಣಾಮಕಾರಿ.

ಓ ನನ್ನ ಭರತಮಾತೆ, ಓ ನನ್ನ ಭಾರತೀಯ ಭ್ರಾತೃಗಳಿರಾ, ನಿದ್ರೆಯಿಂ ದೆದ್ದೇಳಿ, ನವೋದಯವಾಗುತ್ತಿದೆ. ಕಣ್ದೆರೆದು ನೋಡಿ. ಅದೋ ಸುದೀರ್ಘ ರಾತ್ರಿ ಕಡೆಗಿಂದು ಕೊನೆಗಾಣುತ್ತಿದೆ. ಬಹುಕಾಲದ ಶೋಕತಾಪಗಳು ಕಡೆ ಗಿಂದು ಮಾಯವಾಗುತ್ತಲಿವೆ. ಇದುವರೆಗೆ ಶವದಂತೆ ಬಿದ್ದ ಶರೀರ ಇಂದು ಸಚೇತನವಾಗುತ್ತಿದೆ. ಅದೋ ಕಿವಿಗೊಡಿ, ವಾಣಿಯೊಂದು ಕೇಳಿಬರುತ್ತಿದೆ. ಬಹು ಪುರಾತನಕಾಲದ ಇತಿಹಾಸಕಾಲದ ಗರ್ಭದಿಂದ ಹೊಮ್ಮಿ, ಪರ್ವತ ಶಿಖರಗಳಿಂದ ಮರುದನಿಯಾಗಿ ಚಿಮ್ಮಿ, ಅರಣ್ಯಾರಣ್ಯ ಕಂದ ಕಂದರಗಳಲ್ಲಿ ಸಂಚರಿಸಿ, ಬರಬರುತ್ತಾ ಪ್ರಬಲವಾಗಿ ಬಂದಂತೆಲ್ಲ ಅಪ್ರತಿಹತವಾಗಿ, ನಮ್ಮೀ ಪುಣ್ಯಭೂಮಿಯನ್ನು ನಿದ್ದೆಯಿಂದ ಹೊಡೆದು ಎಬ್ಬಿಸಿ. ಜ್ಞಾನ-ಭಕ್ತಿ- ವೈರಾಗ್ಯ-ಸೇವಾ ತತ್ತ್ವಗಳನ್ನು ಉಚ್ಚಕಂಠದಿಂದ ಸಾರುವ ತೂರ್ಯವಾಣಿ ಯೊಂದು ಕೇಳಿಬರುತ್ತಿದೆ; ಹಿಮಾಲಯದಿಂದ ಬೀಸಿಬರುವ ಪುಣ್ಯ ಸಮೀರ ದಂತೆ ನಿರ್ಜೀವದಂತಿದ್ದ ಅಸ್ಥಿಮಾಂಸಗಳಿಗೆ ಜೀವದಾನಮಾಡುತ್ತಿದೆ. ಜಡ ನಿದ್ರೆಯನ್ನು ಪರಿಹರಿಸುತ್ತಿದೆ. ಕಾರ್ಯೋತ್ಸಾಹ, ಸ್ಥೈರ್ಯ-ಧೈರ್ಯಗಳನ್ನು ಉದ್ರೇಕಿಸುತ್ತಿದೆ. ಕುರುಡರಿಗೆ ಕಾಣದು. ಮೂರ್ಖರಿಗೆ ತಿಳಿಯದು. ನಮ್ಮೀ ಭಾರತಭೂಮಿ ಯುಗಯುಗಗಳ ನಿದ್ರೆಯಿಂದ ಮೇಲೇಳುತ್ತಿದೆ. ಆಕೆಯನ್ನು ಇನ್ನು ಯಾರೂ ತಡೆಯಬಲ್ಲವರಿಲ್ಲ. ಇನ್ನಾಕೆ ನಿದ್ದೆ ಮಾಡುವುದಿಲ್ಲ. ಯಾವ ಶಕ್ತಿಯೂ ಆಕೆಯನ್ನು ಬಗ್ಗಿಸಲಾರದು ಏಕೆಂದರೆ, ಅದೋ ನೋಡಿ! ಮಹಾ ಕಾಳಿ ಮತ್ತೊಮ್ಮೆ ಎಚ್ಚೆತ್ತು ಮೈಕೊಡವಿ, ಉಸಿರೆಳೆದು ನಿಲ್ಲುತ್ತಿರುವಳು.

ಪುರಾತನ ಆರ್ಯಮಹರ್ಷಿ ಕುಲಸಂಭೂತರೆಂದು ನೀವೆಂತು ಹಿಗ್ಗಿದರೂ, ಪುರಾತನ ಆರ್ಯಾವರ್ತದ ಮಹಿಮೆ ಗೌರವಗಳನ್ನು ಹಗಲಿರುಳು ನೀವೆನಿತು ಕೀರ್ತಿಸಿದರೂ, ಉಚ್ಚಕುಲ ಪ್ರಸೂತರೆಂದು ನೀವೆನಿತು ಉಬ್ಬಿದರೂ –ಆರ್ಯಾವರ್ತದಲ್ಲಿ ಉತ್ತಮ ವರ್ಗದವರು ಎಂದೆನಿಸಿಕೊಂಡಿರುವ ನೀವೆ ಲ್ಲರೂ–ಸಪ್ರಾಣರೆಂದು ತಿಳಿದಿರುವಿರೇನು? ನೀವೆಲ್ಲ ಹೆಣಗಳಾಗಿದ್ದೀರಿ– ಶತಶತಮಾನಗಳ ಮಮ್ಮಿಗಳಾಗಿದ್ದೀರಿ! ಯಾರನ್ನು ನಿಮ್ಮ ಪೂರ್ವಿಕರು “ಚಲಮಾನ ಸ್ಮಶಾನ”ಗಳೆಂದು ತಿರಸ್ಕರಿಸಿದರೋ ಅವರಲ್ಲಿ ಮಾತ್ರವೇ ಭರತವರ್ಷದ ಪ್ರಾಣವಾಯು ಸ್ವಲ್ಪವಾದರೂ ಇನ್ನೂ ಸಂಚರಿಸುತ್ತಿದೆ. ನೀವು ಮಾತ್ರ ನಿಜವಾಗಿಯೂ “ಜೀವಶವ”ಗಳಾಗಿದ್ದೀರಿ. ನಿಮ್ಮ ನಿವಾಸವನ್ನು, ಅಲ್ಲಿರುವ ಸಾಮಾನನ್ನು ನೋಡಿದರೆ, ಪ್ರಾಕ್ತನ ವಸ್ತುಸಂಗ್ರಹ ಶಾಲೆಯ ನೆನಪಾಗುತ್ತದೆ–ಜೀವವಿಲ್ಲ, ಸತ್ತ್ವವಿಲ್ಲ, ನಿಮ್ಮ ರೀತಿ ನೀತಿ ಆಚಾರ ವ್ಯವ ಹಾರಗಳಂತೂ ನೋಡಿದವರಿಗೆ ಅಡಗೂಲಜ್ಜಿಯ ಕಥೆಯ ಪ್ರಪಂಚವನ್ನು ಮನಸ್ಸಿಗೆ ತರುತ್ತದೆ! ಮುಖತಃ ನಿಮ್ಮ ಪರಿಚಯ ಮಾಡಿಕೊಂಡು ಹಿಂದಿರುಗಿ ದವನಿಗೆ ಯಾವುದೋ ಬಹು ಪುರಾತನದ ಚಿತ್ರಶಾಲೆಯೊಂದನ್ನು ಸಂದರ್ಶಿಸಿ ಬಂದಂತೆ ಭಾಸವಾಗುತ್ತದೆ. ಈ ಮಾಯಾ ಪ್ರಪಂಚದಲ್ಲಿ ಮಾಯೆ ಎಂದರೆ ನೀವೇ! ಉಚ್ಚವರ್ಗದವರೆಂದು ಕೂಗಿಕೊಳ್ಳುವ ಭಾರತೀಯರಿರಾ, ನೀವೆ ಲ್ಲರೂ ಛಾಯೆಗಳು, ಮರೀಚಿಕೆಗಳು! ನಿಮಗಿಂತಲೂ ಹೆಚ್ಚಾದ ಛಾಯೆ ಯಿಲ್ಲ. ಮರುಮರೀಚಕೆ ಇಲ್ಲ. ನೀವೆಲ್ಲ ಭೂತಕಾಲದ ಭೂತಗಳಾಗಿದ್ದೀರಿ; ನಿಮ್ಮ ವರ್ತಮಾನತೆಯು ಒಂದು ಹುಸಿಗನಸು; ಬರಿಯ ಭ್ರಾಂತಿ ಪಿತ್ತದಿಂದ ಉತ್ಪನ್ನವಾಗುವ ಚಿತ್ತಗ್ಲಾನಿಗೆ ಉದಾಹರಣೆ. ನೀವು ಸೊನ್ನೆಗಳು–ಭವಿತ ವ್ಯದ ದೃಷ್ಟಿಗೆ ನೀವೆಲ್ಲ ಶೂನ್ಯ! ಸ್ವಪ್ನಲೋಕದ ಛಾಯಾಮೂರ್ತಿಗಳಿರಾ! ಏಕಿನ್ನೂ ಉಳಿದುಕೊಂಡಿದ್ದೀರಿ? ಮೃತಗತಭಾರತದ ರಕ್ತಮಾಂಸವಿಹೀನ ಅಸ್ಥಿಪಂಜರಗಳಿರಾ, ನೀವೇಕೆ ಆದಷ್ಟು ಬೇಗನೆ ಹುಡಿಯಲ್ಲಿ ಹುಡಿಯಾಗಿ, ಗಾಳಿಯಲ್ಲಿ ಗಾಳಿಯಾಗಿ ಹೋಗಬಾರದು? ಹೌದು, ನಿಮ್ಮ ಎಲುಬಿನ ಬೆರಳು ಗಳಲ್ಲಿ ನಿಮ್ಮ ಪೂರ್ವಿಕರು ತೊಡಿಸಿದ ಅತ್ಯಮೂಲ್ಯ ರತ್ನದುಂಗುರಗಳು ಇನ್ನೂ ಬಿದ್ದುಹೋಗದೆ ಸಿಕ್ಕಿಕೊಂಡಿವೆ. ಕೊಳೆತು ಹೋಗುತ್ತಿರುವ ನಿಮ್ಮ ಹೆಣಗಳ ಅಸಹ್ಯ ಆಲಿಂಗನೆಯಲ್ಲಿ ಪೂರ್ವೀಕರಿತ್ತ ಐಶ್ವರ್ಯಮಂಜೂಷೆಗಳು ಕೆಲವಿನ್ನೂ ಅವಿನಾಶವಾಗಿವೆ. ಇದುವರೆವಿಗೂ ಅವುಗಳನ್ನು ದಾನಮಾಡುವ ಪುಣ್ಯಾವಕಾಶ ನಿಮಗೆ ಲಭಿಸಿರಲಿಲ್ಲ. ಇಂದು ಉದಾರ ವಿದ್ಯಾಪ್ರಚಾರವಾಗು ತ್ತಿರುವಾಗ, ನಿಮಗೆ ಶುಭಾವಕಾಶ ದೊರೆತಿದೆ. ನಿಮ್ಮ ಸಂತಾನರಿಗೆ ಅವು ಗಳನ್ನು ದಾನಮಾಡಿ. ಆದಷ್ಟು ಬೇಗನೆ ದಾನಮಾಡಿ, ಶೂನ್ಯದಲ್ಲಿ ಐಕ್ಯವಾಗಿ ಕಾಣದೆ ತೊಲಗಿಹೋಗಿ. ನಿಮ್ಮ ಸ್ಥಾನದಲ್ಲಿ ನವೀನ ಭಾರತವೇಳಲಿ, ಕೃಷಿಕನ ದಾರಿದ್ರ್ಯ ನಿವಾಸದಿಂದ ಹಲಹಸ್ತೆಯಾಗಿ ನವೀನ ಭಾರತಾಂಬೆ ಮೈದೋರಲಿ. ಬೆಸ್ತನ ಜೋಪಡಿಯಿಂದಾಕೆ ಮೂಡಲಿ. ಚಮ್ಮಾರ ಜಾಡಮಾಲಿಗಳ ಬಡ ಗುಡಿಸಲುಗಳಿಂದ ಆಕೆ ಹೊರಹೊಮ್ಮಲಿ, ಮಳಿಗೆಗಳಿಂದ, ಕಾರ್ಖಾನೆಗಳಿಂದ, ಅಂಗಡಿಗಳಿಂದ, ಸಂತೆಗಳಿಂದ, ಆಕೆ ಪ್ರತ್ಯಕ್ಷವಾಗಲಿ, ಪರ್ವತಕಾನನಗಳಿಂದ ಕಂದರ ವನಗಳಿಂದ ಆಕೆಯ ಮೂರ್ತಿ ರೂಪುಗೊಳ್ಳಲಿ. ಸಾಮಾನ್ಯ ಜನರು, ಸಹಸ್ರಾರು ವರ್ಷಗಳಿಂದ ಪದದಲಿತರಾಗಿರುವರು. ಉಚ್ಚವರ್ಗದವರ ಕ್ರೂರ ಪದಾಘಾತವನ್ನು ಗೊಣಗಾಡದೆ ಸಹಿಸಿ, ಸಹಿಸಿ, ಅವರಲ್ಲಿ ಒಂದು ಅದ್ಭುತ ಸಹಿಷ್ಣುತೆ ತಲೆದೋರಿದೆ. ಅನಂತ ದುಃಖಭಾಜನರಾದುದರಿಂದ ಅಮಿತ ಶಕ್ತಿಶಾಲಿಗಳಾಗಿರುವರು. ಬೊಗಸೆಗಂಜಿ ಕುಡಿದು ಜೀವಿಸುತ್ತಿದ್ದರೂ ಅವರು ಲೋಕವನ್ನೇ ಅಲುಗಿಸಲು ಸಮರ್ಥರಾಗಿರುವರು. ಇನ್ನೊಂದು ಮುಷ್ಟಿ ಅನ್ನ ವನ್ನು ಅವರಿಗೆ ನೀಡಿ; ಅವರ ಶಕ್ತಿಯನ್ನೊಳಗೊಳ್ಳಲು ಜಗತ್ತು ಸಾಲದಾಗು ತ್ತದೆ. ರಕ್ತ ಬೀಜನಿಗೆ ಇದ್ದಂಥಾ ಅಕ್ಷಯವೀರ್ಯ ಅವರಲ್ಲಿದೆ. ಅದೂ ಅಲ್ಲದೆ ನಿರ್ಮಲಜೀವನದಿಂದ ಮಾತ್ರ ಸಾಧ್ಯವಾಗುವ ಅದ್ಭುತ ತೇಜಸ್ಸು ಅವರಲ್ಲಿದೆ. ಅಂತಹ ತೇಜಸ್ಸು ಪ್ರಪಂಚದ ಬೇರಾವ ಭಾಗದಲ್ಲಿಯೂ ಅಲಭ್ಯ. ಅಂತಹ ಶಾಂತಿ, ಅಂತಹ ತೃಪ್ತಿ, ಅಂತಹ ಪ್ರೀತಿ, ಅಂತಹ ಮೌನ, ನಿರಂತರ ಕಾರ್ಯದಕ್ಷತೆ. ಕಾರ್ಯನಿರ್ವಾಹ ಸಮಯಗಳಲ್ಲಿ ಅವಶ್ಯಕವಾದ, ಸಿಂಹಸದೃಶವಾದ ಸಬಲತೆ ಇವನ್ನೆಲ್ಲ ಇನ್ನೆಲ್ಲಿ ಕಾಣುವಿರಿ? ಗತಕಾಲದ ಅಸ್ಥಿಪಂಜರಗಳಿರಾ, ನಿಮ್ಮ ಮುಂದೆ ನಿಂತಿರುವರು ನಿಮ್ಮ ಸಂತಾನರು; ಆವಿರ್ಭವಿಸಲಿರುವ ಭಾರತವರ್ಷದೋಪಾದಿಯಲ್ಲಿ! ನಿಮ್ಮ ಐಶ್ವರ್ಯ ಮಂಜೂಷೆಗಳನ್ನೂ ನಿಮ್ಮ ರತ್ನದುಂಗುರಗಳನ್ನೂ ಆದಷ್ಟು ಬೇಗನೆ ಅವರಿಗೆ ದಾನಮಾಡಿ, ತರುವಾಯ ನೀವು ಗಾಳಿಯಲ್ಲಿ ಗಾಳಿಯಾಗಿ ಮಾಯವಾಗಿ; ಕಿವಿಗೊಟ್ಟು ಮಾತ್ರ ಆಲಿಸುತ್ತಿರಿ. ನೀವು ಮಾಯವಾದ ಕೂಡಲೆ ಸದ್ಯಃ ಪ್ರಸೂತ ನವಭಾರತದ ಮಹಾಪ್ರಾರಂಭಗಾನ ಮೇಘ ಗಂಭೀರ ಧ್ವನಿಯಂತೆ ವಿಶ್ವದ ದಿಕ್ತಟಗಳಿಂದ ಅನುರಣಿತವಾಗುವುದು.

ರಾಜಕೀಯ ವಿದ್ಯೆಯನ್ನು ತುತೂರಿಯೂದಿ ಸಾರಬಹುದು. ಆರ್ಥಿಕ ಮತ್ತು ಸಾಮಾಜಿಕ ವಿಷಯಗಳನ್ನು ಕತ್ತಿ ಹಿಡಿದು, ಬೆಂಕಿ ಹಚ್ಚಿ ಬೋಧಿಸ ಬಹುದು. ಆದರೆ ಧರ್ಮಬೋಧೆ ನೀರವವಾಗಿ ನಡೆಯಬೇಕು, ಹೊತ್ತಾರೆ ಮಲಗಿರುವ ಗುಲಾಬಿ ಮೊಗ್ಗುಗಳ ಮೇಲೆ ಬೀಳುವ ಇಬ್ಬನಿಯಂತೆ. ಜಗತ್ತಿಗೆ ಧರ್ಮವೇ ಭರತಖಂಡ ನೀಡುವ ಕಾಣಿಕೆ.

ನಮಗೆ ಈಗ ಪುರುಷವೀರರು ಬೇಕು. ಉಳಿದುದೆಲ್ಲ ತಾನಾಗಿಯೇ ಸಿದ್ಧವಾಗುತ್ತದೆ. ಆದರೆ ಬಲಿಷ್ಠ, ದೃಢಿಷ್ಠ, ಆಶಿಷ್ಟ ಮೇಧಾವಿಗಳಾದ ತರುಣ ಸಿಂಹರು ಬೇಕು. ಅಂತಹ ಒಂದುನೂರು ಜನ ಮುಂದೆ ಬಂದರೆ, ಅವರಿಂದ ಪ್ರಪಂಚವನ್ನೇ ಬದಲಾಯಿಸಬಹುದು.... ಶಕ್ತಿದಾಯಕವಾದ ನಿಮ್ಮ ಧರ್ಮದ ಪ್ರಚಂಡ ಸತ್ಯಗಳನ್ನು ಸರ್ವರಿಗೂ ಬೋಧಿಸಿ, ಆ ಅಮೃತಪಾನಕ್ಕಾಗಿ ಜಗತ್ತು ಗಂಟಲೊಣಗಿ ಕಾಯುತ್ತಿದೆ. ಜನರಿಗೆ ಶತಮಾನಗಳಿಂದಲೂ ನೈಜ್ಯ ಭಾವಗಳನ್ನು ಬೋಧಿಸಿದ್ದಾರೆ. ಅವರು ಕೈಲಾಗದ ಕ್ರಿಮಿಗಳೆಂದು ತಿಳಿಸಿದ್ದಾರೆ. ಸಾಮಾನ್ಯ ಜನರಿಗೆ ಅವರು ಮಾನವರೇ ಅಲ್ಲವೆಂದು ಉಪದೇಶಿಸಿದ್ದಾರೆ. ಅದುವರೆವಿಗೂ ಅವರ ಕಿವಿಗೆ ಅವರಾತ್ಮದ ಮಹಿಮೆಯ ಗಾನ ಕೇಳಿಸಿಯೇ ಇಲ್ಲ. ಅವರಿಗೂ ಕೊಂಚ ಗೊತ್ತಾಗಲಿ. ತಾವು ನಿತ್ಯ ಶುದ್ಧಬುದ್ಧ ಮುಕ್ತಾತ್ಮ ರೆಂದು. ನಮ್ಮ ನರಗಳಿಗೆ ವಜ್ರಸಾಣೆಯಾಗಲಿ. ನಮಗೀಗ ಬೇಕಾಗಿರುವುದು ಕಬ್ಬಿಣದ ಮಾಂಸ ಖಂಡಗಳು, ಉಕ್ಕಿನ ನರಗಳು. ಬಹುಕಾಲದಿಂದಲೂ ನಾವು ಕಣ್ಣೀರು ಕರೆದಿದ್ದೇವೆ. ಇನ್ನು ಸಾಕು. ಅಳು ಬೇಡ. ಮೇಲೆದ್ದು ನಿಂತು ಪುರುಷರಾಗಿ. ನಮಗೆ ಬೇಕು ವೀರಧರ್ಮ. ವೀರನಿರ್ಮಾಪಕ ಧರ್ಮ. ಸತ್ಯಕ್ಕೆ ಇದೇ ಒರೆಗಲ್ಲು. ದೇಹವನ್ನಾಗಲಿ, ಮನಸ್ಸನ್ನಾಗಲಿ, ಆತ್ಮವನ್ನಾಗಲಿ ದುರ್ಬಲ ಗೊಳಿಸುವ ಪ್ರತಿಯೊಂದನ್ನೂ ವಿಷವೆಂದು ತಿಳಿದು ತಿರಸ್ಕರಿಸಿರಿ. ಸತ್ಯ ಬಲ ಕಾರಿ; ಉಪನಿಷತ್ತಿನ ಸತ್ಯ ನಿಮ್ಮ ಮುಂದಿದೆ. ಅದನ್ನು ಅನುಷ್ಠಾನ ಮಾಡಿ. ಭರತಖಂಡದ ಮೋಕ್ಷ ಸಮೀಪಿಸುತ್ತದೆ.

ಮಹತ್ಕಾರ್ಯ ಸಾಧನೆಗೆ ಮೂರು ಗುಣಗಳು ಅತ್ಯಾವಶ್ಯಕ. ಮೊದಲ ನೆಯದೇ ಹೃದಯವೇದೆ. ಎಂದರೆ ಎದೆ ಮರುಕ. ಮೃಗಗಳಂತೆ ಬಾಳುತ್ತಿರುವ ಕೋಟ್ಯಂತರ ಸೋದರರಿಗಾಗಿ ನಿಮ್ಮೆದೆ ಮರುಗುತ್ತಿದೆಯೇ? ನಿಮಗೆ ಊಟ ಬೇಡವಾಗಿದೆಯೇ? ನಿಮಗೆ ನಿದ್ದೆ ಬರುವುದಿಲ್ಲವೇ? ಅದಕ್ಕಾಗಿ ಹಗಲಿರಳೂ ಮರುಗುತ್ತಿರುವಿರಾ?.... ಹಾಗಿದ್ದರೆ ನೀವು ಸುಧಾರಕ ಪಟ್ಟಕ್ಕೆ ಅರ್ಹರಾ ಗುವ ಮೊದಲನೆ ಮೆಟ್ಟಲನ್ನು ಹತ್ತಿರುವಿರಿ. ಎರಡನೆಯದೇ ಬರಿದೇ ಮರುಗಿ ದರೆ ಸಾಲದು. ಮರುಕ ಕಾರ್ಯಕಾರಿಯಾಗಬೇಕು. ಮೂರನೆಯದು, ಎಡರು ಗಳನ್ನು ಜಯಸಿ ಮುಂದುವರಿಯುವ “ಇಚ್ಛಾ ಶಕ್ತಿ” ಇರಬೇಕು. ಜಗತ್ತೆಲ್ಲ ನಿಮ್ಮ ಮೇಲೆ ಕತ್ತಿ ಕಟ್ಟಿದರೂ ಹೆದರದೆ ಸತ್ಯಕ್ಕಾಗಿ ಹೋರಾಡುತ್ತೀರಾ? ಹೆಂಡಿರು ಮಕ್ಕಳೇ ನಿಮಗೆ ಎದುರುಬಿದ್ದರೆ, ನಿಮ್ಮ ಹಣವೆಲ್ಲ ಕೈಬಿಟ್ಟು ಹೋಗಿ ಭಿಕಾರಿಗಳಾದರೆ, ಅಪಕೀರ್ತಿ ಬಂದರೆ, ಆಗಲೂ ಬಿಡದೆ ಮುಂಬರಿ ಯುತ್ತೀರೇನು? ಹಾಗಿದ್ದರೆ ನೀವೂ ಸುಧಾರಕರಾಗಲು ಸರ್ವೋತ್ಕೃಷ್ಟರಾದ ಅಧಿಕಾರಿಗಳಾಗುತ್ತೀರಿ.


\section{ಭಕ್ತಿ}

ಭಕ್ತಿಯೋಗವೆಂದರೆ ಸತ್ಯವಾದ ನಿಷ್ಕಪಟ ಭಗವಂತನ ಅನ್ವೇಷಣೆ. ಈ ಅನ್ವೇಷಣೆ ಪ್ರೇಮದಲ್ಲಿ ಮೊದಲಾಗಿ, ಪ್ರೇಮದಲ್ಲಿ ಸಾಗಿ, ಪ್ರೇಮದಲ್ಲಿ ಕೊನೆಗಾಣುವುದು. ಒಂದು ಕ್ಷಣ ಪರಮಾತ್ಮನ ಮೇಲೆ ಇಡುವ ಅತ್ಯುತ್ಕಟ ಪ್ರೇಮವೇ ನಮಗೆ ಚಿರ ಸಾಯುಜ್ಯ ಪದವಿಯನ್ನು ನೀಡುವುದು.

ಭಗವಂತನ ಮೇಲೆ ನಮಗೆ ಇರುವ ಪರಮಪ್ರೇಮವೇ ಭಕ್ತಿ. ಅದು ಸಿದ್ಧಿಸಿದರೆ ಅವನು ಎಲ್ಲರನ್ನೂ ಪ್ರೀತಿಸುವನು. ಯಾರನ್ನೂ ದ್ವೇಷಿಸುವುದಿಲ್ಲ. ನಿತ್ಯತೃಪ್ತನಾಗುವನು. ಈ ಪ್ರೇಮವನ್ನು ನಾವು ವಿಷಯವಸ್ತುಗಳ ಲಾಭಕ್ಕೆ ಇಳಿಸಲಾಗುವುದಿಲ್ಲ. ಏಕೆಂದರೆ ಎಲ್ಲಿಯವರೆಗೂ ಪ್ರಾಪಂಚಿಕ ಆಸೆ ಆಕಾಂಕ್ಷೆ ಗಳು ನಮ್ಮ ಹೃದಯಲ್ಲಿ ಕುದಿಯುತ್ತಿರುವುವೋ ಅಲ್ಲಿಯವರೆಗೆ ಈ ಪರಮ ಪ್ರೇಮ ನಮ್ಮಲ್ಲಿ ಉದಿಸಲಾರದು.

ಭಗವಂತನನ್ನು ಪಡೆಯುವುದಕ್ಕೆ ಸ್ವಾಭಾವಿಕ ಮತ್ತು ಸುಲಭವಾದ ಮಾರ್ಗವೇ ಭಕ್ತಿಯೋಗ. ಇದೇ ಅದರ ಅನುಕೂಲ. ಇದರ ಒಂದು ದೊಡ್ಡ ಕೊರತೆಯೇ ಪ್ರಾರಂಭದಲ್ಲಿ ಅನೇಕ ವೇಳೆ ಭಕ್ತಿ ಭಯಾನಕ ಮತಭ್ರಾಂತಿಯ ವಿಕಾರರೂಪವನ್ನು ಧರಿಸುವ ಅಂಜಿಕೆ ಇದೆ. ಭಕ್ತಿಯ ಈ ಕೆಳದರ್ಜೆಯಲ್ಲಿ ಆರಾಧಿಸುತ್ತಿರುವವರಿಂದಲೇ ಹಿಂದೂ ಮಹಮ್ಮದೀಯ ಕ್ರೈಸ್ತಪಂಗಡದ ಮತಭ್ರಾಂತರು ಬಂದಿರುವುದು.

ನಿಮ್ಮ ಅಂತರಂಗದಲ್ಲಿ ಭಕ್ತಿ ಇದೆ. ಆದರೆ ಕಾಮಕಾಂಚನ ಅದನ್ನು ಆವರಿಸಿರುವುದು, ಅದನ್ನು ನಿವಾರಿಸಿದೊಡನೆ ಭಕ್ತಿ ತಾನೆ ವ್ಯಕ್ತವಾಗುವುದು.

ಭಗವಂತ ಮತ್ತು ಆತನ ಅತ್ಯುತ್ತಮ ಮಕ್ಕಳ ಕೃಪೆಯನ್ನು ಪಡೆಯಿರಿ. ಈಶ್ವರಲಾಭಕ್ಕೆ ಎರಡೇ ಮುಖ್ಯ ಸಾಧನ. ಇಂತಹ ಭಗವಂತನ ಮಕ್ಕಳ ಸಂಗ ಲಭಿಸುವುದು ಅತ್ಯಂತ ದುರ್ಲಭ. ಅವರ ಐದು ನಿಮಿಷದ ಸಹವಾಸ ನಮ್ಮ ಇಡೀ ಜೀವನವನ್ನೇ ಬದಲಾಯಿಸಬಲ್ಲದು. ನಿಮಗೆ ನಿಜವಾಗಿಯೂ ಅಂತಹ ವರು ಯಾರಾದರೂ ಬೇಕಾಗಿದ್ದರೆ ನಿಮಗೆ ಒಬ್ಬರು ಸಿಕ್ಕುವರು. ಭಗವದ್ಭಕ್ತರು ಇದ್ದ ಸ್ಥಳ ಪವಿತ್ರವಾಗುವುದು. “ಭಗವದ್ಭಕ್ತರ ಮಹಿಮೆ ಇದು.” ಭಕ್ತರೇ ದೇವರು, ಅವರ ನುಡಿಯೇ ಶಾಸ್ತ್ರ. ಅವರಿದ್ದ ಸ್ಥಳ ಪವಿತ್ರ ವಾತಾವರಣ ಗಳಿಂದ ತುಂಬಿ ತುಳಕಾಡುವುದು. ಯಾರು ಅಲ್ಲಿಗೆ ಹೋಗುವರೋ ಅವರಿಗೆ ಇದು ಭಾಸವಾಗುವುದು. ಅವರೂ ಪವಿತ್ರರಾಗುವ ಸಂಭವವಿದೆ.

“ತ್ಯಜಿಸಿ” ಎಂದು ಭಕ್ತಿಯೋಗ ಸಾರುವುದಿಲ್ಲ. “ಪ್ರೀತಿಸಿ, ಅತ್ಯುತ್ತಮ ವನ್ನು ಪ್ರೀತಿಸಿ” ಎನ್ನುವುದು. ಯಾರ ಪ್ರೇಮದ ಇಷ್ಟಮೂರ್ತಿ ಈ ಪರಮ ಪವಿತ್ರ ವಸ್ತುವಾಗಿರುವುದೋ ಅವರ ಮನಸ್ಸಿನಿಂದ ಹೀನಭಾವನೆಗಳೆಲ್ಲ ಸ್ವಾಭಾವಿಕವಾಗಿಯೇ ಮಾಯವಾಗುವುವು.

ದೇವರು ನಮಗೆ ಸತ್ಯವಾಗಿಯೂ ತೀವ್ರವಾಗಿಯೂ ಆವಶ್ಯಕ ಎನ್ನುವುದೇ ಭಕ್ತಿಯೋಗದಲ್ಲಿ ಅತಿಮುಖ್ಯವಾಗಿ ಬೇಕಾಗಿರುವುದು. ನಮಗೆ ದೇವರಲ್ಲದೆ ಎಲ್ಲ ಬೇಕು. ಹೊರಗಿನ ಪ್ರಪಂಚದಿಂದ ನಮ್ಮ ಸಾಧಾರಣ ಬಯಕೆಗಳೆಲ್ಲ ಈಡೇರುತ್ತವೆ. ಎಲ್ಲಿಯವರೆವಿಗೂ ನಮ್ಮ ಕೋರಿಕೆಗಳು ಬಾಹ್ಯ ಜಡಜಗತ್ತಿನ ಮಿತಿಯೊಳಗೆ ಇರುವುವೋ ಅಲ್ಲಿಯವರೆವಿಗೂ ದೇವರ ಆವಶ್ಯಕತೆ ನಮಗೆ ತೋರುವುದಿಲ್ಲ. ಈ ಪ್ರಪಂಚದಲ್ಲಿ ನಮಗೆ ಪೆಟ್ಟುಗಳು ಬಿದ್ದು ನಿರಾಶ ರಾದಾಗ ಮಾತ್ರ ಏನಾದರೂ ಮೇಲಿರುವುದನ್ನು ಅರಸಬಯಸುವೆವು. ಆಗ ಮಾತ್ರ ದೇವರನ್ನು ಹುಡುಕುವೆವು.

ನಿಜವಾದ ಪ್ರೇಮದಲ್ಲಿ ಅಂಜಿಕೆ ಇಲ್ಲ. ಅಂಜಿಕೆಯ ಸುಳಿ ಇರುವವರೆಗೂ ಭಕ್ತಿ ಹುಟ್ಟುವುದು ಕೂಡ ಇಲ್ಲ. ಭಕ್ತಿಯಲ್ಲಿ ದೇವರ ಹತ್ತಿರ ಬೇಡುವು ದಾಗಲೀ ವ್ಯಾಪಾರವಾಗಲೀ ಇಲ್ಲ. ಭಗವಂತನ ಹತ್ತಿರ ಏನನ್ನಾದರೂ ಕೇಳಬೇಕೆಂಬ ಯೋಚನೆಯೇ ದೈವದ್ರೋಹ ಮಾಡಿದಂತೆ. ಅವನು ಆರೋಗ್ಯ ವನ್ನಾಗಲಿ, ಐಶ್ವರ್ಯವನ್ನಾಗಲೀ, ಸ್ವರ್ಗಸುಖವನ್ನಾಗಲೀ ಇಚ್ಛಿಸುವುದಿಲ್ಲ.

ಮನುಷ್ಯನಿಗೆ ಒಂದು ದಿನ ಆಹಾರವಿಲ್ಲದೆ ಇದ್ದರೆ ತೊಂದರೆ ಪಡುವನು. ಅವನ ಮಗ ಸತ್ತರೆ ಎಷ್ಟು ಪ್ರಾಣ ಸಂಕಟ ಅವನಿಗೆ! ನಿಜವಾದ ಭಕ್ತ ಭಗವಂತನನ್ನು ಆಶಿಸುವಾಗ ಇದೇ ದಾರುಣಯಾತನೆಯನ್ನು ಪಡುವನು, ಭಕ್ತಿಯ ಒಂದು ಮಹಾಗುಣವೇ ನಮ್ಮ ಮನಸ್ಸು ಶುದ್ಧಿಮಾಡುವುದು. ಈಶ್ವರನ ಮೇಲೆ ಇಡುವ ಅಚಲಪ್ರೇಮವೇ ನಮ್ಮನ್ನು ಪರಿಶುದ್ಧರನ್ನಾಗಿ ಮಾಡುವುದಕ್ಕೆ ಸಾಕು.

ಭಕ್ತೋತ್ತಮನು, ದೇವರನ್ನು ನೋಡುವುದಕ್ಕೆ ದೇವಸ್ಥಾನಕ್ಕೆ ಚರ್ಚಿಗೆ ಹೋಗುವುದಿಲ್ಲ. ಜಗತ್ತಿನಲ್ಲಿ ಅವನಿಲ್ಲದಿರುವ ಸ್ಥಳವೇ ಅವನಿಗೆ ಗೋಚರಿ ಸುವುದಿಲ್ಲ. ದೇವಸ್ಥಾನದಲ್ಲಿ ಅವನನ್ನು ಕಾಣುವನು. ಹಾಗೆಯೇ ಹೊರಗೂ ಅವನನ್ನು ಕಾಣುವನು. ಮಹಾತ್ಮನ ಮಹಾತ್ಮ್ಯೆಯಲ್ಲಿ, ದುರ್ಜನನ ದೌರ್ಜನ್ಯ ದಲ್ಲಿ ಅವನನ್ನು ಕಾಣುವನು. ಏಕೆಂದರೆ ಅವನ ಹೃದಯಮಂದಿರದಲ್ಲಿ ಸರ್ವಶಕ್ತನಾದ, ಅನವರತ ಪ್ರಕಾಶಿಸುತ್ತಿರುವ, ನಂದಿಸಲಾಗದ ಪ್ರೇಮ ದೀವಿಗೆ ನೆಲಸಿರುವುದು.

ಪ್ರಾಪಂಚಿಕರು ಹುಚ್ಚನೆಂದು ಕರೆಯುತ್ತಿದ್ದ ಒಬ್ಬರ ನೆನಪು ನನಗೆ ಇದೆ. ಇದಕ್ಕೆ ಅವರ ಉತ್ತರವಿದು–“ನನ್ನ ಸ್ನೇಹಿತರೆ, ಪ್ರಪಂಚವೇ ಒಂದು ದೊಡ್ಡ ಹುಚ್ಚರ ಆಸ್ಪತ್ರೆ. ಕೆಲವರಿಗೆ ಪ್ರಾಪಂಚಿಕವಸ್ತುಗಳಿಗಾಗಿ ಹುಚ್ಚು. ಕೆಲವರಿಗೆ ಹೆಸರಿನ ಮೇಲೆ. ಕೆಲವರಿಗೆ ಕೀರ್ತಿಯ ಮೇಲೆ ಹುಚ್ಚು. ಕೆಲವರಿಗೆ ದ್ರವ್ಯದ ಮೇಲೆ ಹುಚ್ಚು. ಮುಕ್ತಿ ಗಳಿಸಬೇಕು. ಸ್ವರ್ಗಕ್ಕೆ ಹೋಗಬೇಕು ಎಂದು ಕೆಲವರಿಗೆ ಹುಚ್ಚು. ನನಗೆ ದೇವರ ಮೇಲೆ ಹುಚ್ಚು. ನೀವೂ ಹುಚ್ಚರು. ನಾನೂ ಹುಚ್ಚ. ನನ್ನ ಹುಚ್ಚು ಎಲ್ಲರ ಹುಚ್ಚಿಗಿಂತ ಮೇಲಾದುದೆಂದು ತಿಳಿಯುತ್ತೇನೆ.” ನಿಜವಾದ ಭಕ್ತನ ಪ್ರೀತಿ ಹೀಗೆ ದಹಿಸುತ್ತಿರುವ ಹುಚ್ಚಿನಂತೆ. ಅವನಿಗೆ ಇದರ ಎದುರಿಗೆ ಎಲ್ಲಾ ಮಾಯವಾಗುವುದು. ಅವನಿಗೆ ಪ್ರಪಂಚವೆಲ್ಲ ಪ್ರೇಮ ಮಯ. ಪ್ರೇಮಿಗೆ ಪ್ರಪಂಚ ಕಾಣುವುದು ಹೀಗೆ. ಯಾರಲ್ಲಿ ಈ ಪ್ರೇಮ ಇದೆಯೋ ಅವನು ನಿತ್ಯಪವಿತ್ರನಾಗುವನು. ನಿತ್ಯಾನಂದಲ್ಲಿರುವನು. ಈ ಪರಮಪ್ರೇಮವೆಂಬ ದಿವ್ಯೋನ್ಮಾದವೇ ನಮ್ಮ ಭವರೋಗವನ್ನು ನಾಶ ಮಾಡಬಲ್ಲದು.


\section{ಶಕ್ತಿ}

ನನ್ನ ನೆಚ್ಚಿನ ಧೀರ ಸುತರೆ! ನೀವೆಲ್ಲ ಮಹಾಕಾರ್ಯವನ್ನು ಮಾಡಲು ಜನ್ಮವೆತ್ತಿರುವುದೆಂದು ದೃಢವಾಗಿ ನಂಬಿ. ನಾಯಿಮರಿಗಳ ಕೂಗಾಟ ನಿಮ್ಮನ್ನು ಚಕಿತರನ್ನಾಗಿ ಮಾಡದಿರಲಿ. ಇಲ್ಲ, ಕಾರ್ಮುಗಿಲಿನಿಂದ ಬೀಳುವ ಸಿಡಿಲು ಕೂಡ ನಿಮ್ಮನ್ನು ಅಂಜಿಸಕೂಡದು, ಸ್ಥಿರವಾಗಿ ನಿಲ್ಲಿ. ಕಾರ್ಯೋ ನ್ಮುಖರಾಗಿ.

ನಮ್ಮ ದೇಶಕ್ಕೆ ಪುರುಷಸಿಂಹರು ಬೇಕಾಗಿದ್ದಾರೆ. ಪುರುಷಸಿಂಹರಾಗಿ! ಬಂಡೆಯಂತೆ ಅಚಲರಾಗಿ ನಿಲ್ಲಿ. ಸತ್ಯವೇ ಅನವರತ ಗೆಲ್ಲುವುದು. ಭರತ ಖಂಡಕ್ಕೆ ಇಂದು ಬೇಕಾಗಿರುವುದು ರಾಷ್ಟ್ರದ ನಾಡಿಯಲ್ಲಿ ನವಚೇತನವನ್ನು ಜಾಗ್ರತಗೊಳಿಸಬಲ್ಲ ವಿದ್ಯುತ್ ಅಗ್ನಿ. ಧೀರರಾಗಿ ವೀರರಾಗಿ. ಮನುಷ್ಯ ಸಾಯುವುದು ಒಂದೇ ಬಾರಿ. ನನ್ನ ಶಿಷ್ಯರು ಹೇಡಿಗಳಾಗಕೂಡದು. ಹೇಡಿ ತನವನ್ನು ದ್ವೇಶಿಸುವೆನು ನಾನು. ಸಮಾಧಾನಚಿತ್ತರಾಗಿರಿ. ಕೆಲಸಕ್ಕೆ ಬಾರದ ಮನುಜರು ನಿಮ್ಮ ಮೇಲೆ ಮಾಡುವ ದೋಷಾರೋಪಣೆಗೆ ಸ್ವಲ್ಪವೂ ಗಮನ ಕೊಡಬೇಡಿ. ನಿರ್ಲಕ್ಷ್ಯ! ನಿರ್ಲಕ್ಷ್ಯ! ನಿರ್ಲಕ್ಷ್ಯ! ಇದನ್ನು ಗಮನದಲ್ಲಿಡಿ. ಕಣ್ಣುಗಳು ಇರುವುದು ಎರಡು. ಕಿವಿಗಳೆರಡು. ಆದರೆ ಬಾಯಿ ಒಂದೇ. ಎಲ್ಲಾ ಮಹತ್ಕಾರ್ಯಗಳೂ ಜಯಪ್ರದವಾಗುವುದು, ಅಗಾಧ ಆತಂಕಗಳ ಪರಂಪರೆ ಯನ್ನು ದಾಟಿಹೋದ ಮೇಲೆ. ನಿಮ್ಮ ಪೌರುಷ ಪ್ರಯೋಗಿಸಿ, ಕಾಮಕಾಂಚನ ವಶರಾದ ಹೀನಮಾನವರನನ್ನು ತಾತ್ಸಾರದಿಂದ ಕಾಣಬೇಕಾಗಿದೆ.

ನನ್ನ ಗೆಳೆಯನೆ! ಯಾವುದು ನಿನ್ನನ್ನು ಅಳುವಂತೆ ಮಾಡುವುದು? ಅನಂತಶಕ್ತಿ ನಿನ್ನಲ್ಲಿ ಸುಪ್ತವಾಗಿರುವುದು. ಹೇ ಬಲಾಢ್ಯನೇ! ಸರ್ವವನ್ನೂ ಗೆಲ್ಲುವ ನಿನ್ನ ಅನಂತಶಕ್ತಿಯನ್ನು ಪ್ರಯೋಗಿಸು. ಈ ಸೃಷ್ಟಿ ನಿನ್ನ ಪಾದತಳ ದಲ್ಲಿ ಬೀಳುವುದು. ಚೇತನವೊಂದೇ ಪ್ರಧಾನ,ಜಡವಲ್ಲ. ತಾವೇ ದೇಹವೆಂದು ಭಾವಿಸಿರುವ ಮೂಢರು “ಅಯ್ಯೋ! ನಾವು ದುರ್ಬಲರು, ದುರ್ಬಲರು” ಎಂದು ಗೋಳಿಡುವರು. ದೇಶಕ್ಕೆ ಇಂದು ಬೇಕಾಗಿರುವುದು ಧೈರ್ಯ ಮತ್ತು ವೈಜ್ಞಾನಿಕ ಪ್ರತಿಭೆ. ನಮಗೆ ಮಹಾ ಧೈರ್ಯ ಬೇಕು. ಅಗಾಧ ಶಕ್ತಿ ಬೇಕು. ಮೇರೆ ಇಲ್ಲದ ಉತ್ಸಾಹ ಬೇಕು. ಹೇಡಿತನ ಕೂಡದು. ಐಶ್ವರ್ಯದ ಅಧಿ ದೇವತೆ ಲಕ್ಷ್ಮಿ ಕರ್ಮವೀರನನ್ನು ಸಿಂಹ ಸದೃಶ ಕೆಚ್ಚೆದೆಯವನನ್ನು ಒಲಿಯು ವಳು. ಹಿಂದೆ ನೋಡಬೇಕಾದ ಆವಶ್ಯಕತೆ ಇಲ್ಲ. ಮುಂದೆ! ಮುಂದೆ! ನಮಗೆ ಅನಂತ ಶಕ್ತಿ, ಅನಂತ ಉತ್ಸಾಹ, ಅನಂತ ಧೈರ್ಯ, ಅನಂತ ತಾಳ್ಮೆ ಬೇಕು. ಆಗ ಮಾತ್ರ ಮಹಾಕಾರ್ಯ ಸಫಲವಾಗುವುದು.

ವೇದಾಂತ ಎಂದಿಗೂ ಪಾಪವನ್ನು ಒಪ್ಪಿಕೊಳ್ಳುವುದಿಲ್ಲ; ತಪ್ಪನ್ನು ಮಾತ್ರ ಒಪ್ಪಿಕೊಳ್ಳುವುದು. ಮಹಾಪಾಪವೇ ನೀನು ದುರ್ಬಲ, ಪಾಪಿ, ಕೆಲಸಕ್ಕೆ ಬಾರದವನು, ಶಕ್ತಿಯಿಲ್ಲದವನು, ನಿನ್ನ ಕೈಯಿಂದ ಏನೂ ಸಿಗುವುದಿಲ್ಲವೆಂದು ಭ್ರಾಂತಿ ಪಡುವುದು.

ಹಿಂದಿನ ಧರ್ಮ, ದೇವರಲ್ಲಿ ನಂಬಿಕೆ ಇಲ್ಲದವನನ್ನು ನಾಸ್ತಿಕನೆಂದು ಸಾರಿದವು. ಇಂದಿನ ಧರ್ಮ ಯಾರಿಗೆ ತನ್ನಲ್ಲಿ ನಂಬಿಕೆ ಇಲ್ಲವೋ ಅವನನ್ನು ನಾಸ್ತಿಕನೆಂದು ಸಾರುವುದು.

ಶಕ್ತಿಯೇ ಜೀವನ; ದುರ್ಬಲತೆಯೇ ಮರಣ. ಶಕ್ತಿಯೇ ಪರಮಾನಂದ, ಅನಂತಜೀವನ, ಅಮೃತತ್ತ್ವ. ದುರ್ಬಲತೆಯೇ ಅನವರತ ದುಡಿತ, ದುಃಖ ಮಯ, ಮರಣ. ನಿರ್ದಿಷ್ಟವಾದ ಶಕ್ತಿದಾಯಕ, ಉತ್ತೇಜನಕಾರಿ ಮಹದಾ ಲೋಚನೆಗಳು, ಬಾಲ್ಯದಿಂದಲೂ ನಿಮ್ಮ ಮೆದುಳನ್ನು ಪ್ರವೇಶಿಸಲಿ.

ದುರ್ಬಲತೆಯೇ ದುಃಖಕ್ಕೆ ಮೂಲಕಾರಣ. ನಾವು ದುರ್ಬಲರಾದುದರಿಂದ ದುಃಖಭಾಗಿಗಳಾಗುವೆವು. ನಾವು ದುರ್ಬಲರಾದುದರಿಂದ ಸುಳ್ಳು ಹೇಳುವೆವು, ಕದಿಯುವೆವು, ಕೊಲ್ಲುವೆವು, ಉಳಿದ ಹೀನ ಕೆಲಸವನ್ನು ಮಾಡುವೆವು. ನಾವು ದುರ್ಬಲರಾದುದರಿಂದ ಕಷ್ಟಪಡುವೆವು. ನಾವು ದುರ್ಬಲರಾದುದರಿಂದ ಸಾಯುವೆವು. ನಮ್ಮನ್ನು ಯಾವುದೂ ದುರ್ಬಲರನ್ನಾಗಿ ಮಾಡದೇ ಇದ್ದರೆ ಸಾವಿಲ್ಲ, ಸಂಕಟವಿಲ್ಲ.

ನಮಗೆ ಇಂದು ಆವಶ್ಯಕವಾಗಿ ಬೇಕಾಗಿರುವುದು ಶಕ್ತಿ. ಶಕ್ತಿಯೇ ಪ್ರಪಂಚದ ಮಹಾವ್ಯಾಧಿಗೆ ದಿವ್ಯೌಷಧಿ. ಶ್ರೀಮಂತರ ದಬ್ಬಾಳಿಕೆಗೆ ತುತ್ತಾ ದಾಗ ದೀನರಿಗೆ ಬೇಕಾಗಿರುವುದೇ ಶಕ್ತಿಸಂಜೀವಿನಿ. ಜ್ಞಾನಿಗಳ ಉಪಟಳಕ್ಕೆ ತುತ್ತಾದಾಗ ಅಜ್ಞಾನಿಗಳಿಗೆ ಬೇಕಾಗಿರುವುದು ಶಕ್ತಿಸಂಜೀವಿನಿ. ಪಾಪಿಗಳು ಇತರ ಪಾಪಿಗಳ ಕೋಟಲೆಗೆ ತುತ್ತಾದಾಗ ಬೇಕಾಗಿರುವುದು ಶಕ್ತಿಸಂಜೀವಿನಿ.

ಎದ್ದು ನಿಲ್ಲಿ! ಧೀರರಾಗಿ. ಬಲಾಢ್ಯರಾಗಿ, ಜವಾಬ್ದಾರಿಯನ್ನೆಲ್ಲ ನೀವೇ ವಹಿಸಿಕೊಳ್ಳಿ. ನಿಮ್ಮ ಭವಿಷ್ಯನಿರ್ಮಾಪಕರು ನೀವು ಎಂಬುದನ್ನು ಅರಿಯಿರಿ. ನಿಮಗೆ ಬೇಕಾದ ಸಹಾಯವೆಲ್ಲ ಆಗಲೇ ನಿಮ್ಮಲ್ಲಿದೆ. ಆದಕಾರಣ ನಿಮ್ಮ ಭವಿಷ್ಯವನ್ನು ನೀವೇ ನಿರ್ಧರಿಸಿ.

ರೋಗಿಗಳು ನಾವು ಎಂದು ಯಾವಾಗಲೂ ಆಲೋಚಿಸುತ್ತಿದ್ದರೆ ಗುಣ ಮುಖರಾಗುವುದಿಲ್ಲ. ಔಷಧಿ ಬೇಕು. ದುರ್ಬಲತೆಯನ್ನೇ ನೆನೆಯುತ್ತಿದ್ದರೆ ಹೆಚ್ಚು ಸಹಾಯವಾಗದು. ಶಕ್ತಿಯನ್ನು ನೀಡಿ. ದುರ್ಬಲತೆಯ ಆಲೋಚನೆ ಯನ್ನು ಮೆಲ್ಲುತ್ತಿದ್ದರೆ ಶಕ್ತಿ ಬರುವುದಿಲ್ಲ. ದುರ್ಬಲತೆಗೆ ಚಿಕಿತ್ಸೆ ದುರ್ಬಲತೆ ಯನ್ನೇ ಕುರಿತು ಚಿಂತಿಸುವುದಲ್ಲ. ಶಕ್ತಿಯನ್ನೇ ಕುರಿತು ಚಿಂತಿಸುವುದು.

ವ್ಯಾವಹಾರಿಕ ಪ್ರಪಂಚದಲ್ಲಾಗಲಿ, ಪಾರಮಾರ್ಥಿಕ ಪ್ರಪಂಚದಲ್ಲಾಗಲಿ, ಅವನತಿಗೆ ಮತ್ತು ಪಾಪಕ್ಕೆ ಮೂಲ ಅಂಜಿಕೆ ಎಂಬುದು ಸತ್ಯ. ದುಃಖವನ್ನು ತರುವುದು ಅಂಜಿಕೆ. ಸಾವನ್ನು ಕರೆವುದು ಅಂಜಿಕೆ. ಪಾಪವನ್ನು ಸೃಷ್ಟಿಸುವುದು ಅಂಜಿಕೆ. ಅಂಜಿಕೆಗೆ ಕಾರಣ ಯಾವುದು? ನಮ್ಮ ನೈಜ ಸ್ವಭಾವದ ಅಜ್ಞಾನ. ನಮ್ಮಲ್ಲಿ ಪ್ರತಿಯೊಬ್ಬರೂ ಸಾರ್ವಭೌಮನ ಪೀಠಕ್ಕೆ ಹಕ್ಕುದಾರರು.

ಎಲ್ಲಾ ಪಾಪವನ್ನೂ, ದೌರ್ಜನ್ಯವನ್ನೂ ಒಂದು ಪದದಲ್ಲಿ ಹುದುಗಿಸ ಬಹುದು–ಅದೇ ದುರ್ಬಲತೆ. ಎಲ್ಲಾ ದೌರ್ಜನ್ಯದ ಹಿಂದೆ, ಅದನ್ನು ಪ್ರಚೋ ದಿಸುವ ಶಕ್ತಿ ದುರ್ಬಲತೆ. ಮತ್ತೊಬ್ಬರನ್ನು ಹಿಂಸೆಗೆ ಒಳಪಡಿಸುವಂತೆ ಪ್ರೇರೇಪಿಸುವುದು ದುರ್ಬಲತೆ. ನಮ್ಮ ನೈಜತ್ವವನ್ನು ಮರೆಮಾಚುವಂತೆ ಮಾಡುವುದು ದುರ್ಬಲತೆ.

ಎಲ್ಲಿಯವರೆವಿಗೂ ಒಬ್ಬನು ಪ್ರಕೃತಿಯನ್ನು ಮೀರಲು ಅದರೊಂದಿಗೆ ಹೋರಾಡುವನೋ ಅಲ್ಲಿಯವರೆವಿಗೂ ಅವನು ಮನುಜ. ಈ ಪ್ರಕೃತಿ ಆಂತರಿಕ ಮತ್ತು ಬಾಹ್ಯರೂಪದಲ್ಲಿರುವುದು, ಬಾಹ್ಯ ಪ್ರಕೃತಿಯನ್ನು ಗೆಲ್ಲು ವುದು ಒಳ್ಳೆಯದು. ಇದೊಂದು ಮಹಾಸಾಹಸ. ಆದರೆ ಆಂತರಿಕ ಪ್ರಕೃತಿ ಯನ್ನು ಗೆಲ್ಲುವುದು ಅದನ್ನು ಮೀರಿದ ಸಾಹಸ. ಅನಂತ ತಾರೆ ಗ್ರಹಗಳ ಚಲನವಲನಗಳ ನಿಯಮವನ್ನು ತಿಳಿದುಕೊಳ್ಳುವುದು ಮೇಲು. ಚಿತ್ತಾಕರ್ಷ ಣವೆ ಸರಿ ಇದು. ಆದರೆ ಇದಕ್ಕಿಂತಲೂ ವೈಭವಯುತವಾದುದು, ಅದ್ಭುತ ವಾದುದು, ಮಾನವಕೋಟಿಯ ಆಸೆ ಆಕಾಂಕ್ಷೆ ಭಾವನೆಗಳ ಹಿಂದಿರುವ ನಿಯಮವನ್ನು ತಿಳಿದುಕೊಳ್ಳುವುದು.

ಮಾನವನೇ ಪ್ರಾಣಿಗಳೆಲ್ಲಕ್ಕಿಂತ ಶ್ರೇಷ್ಠ, ದೇವತೆಗಳಿಗಿಂತಲೂ ಮಿಗಿಲು. ಮಾನವನನ್ನು ಮೀರಿರುವರಾರೂ ಇಲ್ಲ. ದೇವತೆಗಳು ಕೂಡ ಇಲ್ಲಿಗೆ ಬಂದು ಮಾನವಜನ್ಮವನ್ನು ತಾಳಿ ಮುಕ್ತಿಯನ್ನು ಗಳಿಸಬೇಕಾಗಿದೆ. ಮಾನವ ಮಾತ್ರ ಪರಿಪೂರ್ಣತೆಯನ್ನು ಗಳಿಸಬಲ್ಲ. ದೇವತೆಗಳಿಗೂ ಅಸಾಧ್ಯವಿದು.

ನಮ್ಮ ದೇಶಕ್ಕೆ ಇಂದು ಬೇಕಾಗಿರುವುದು ಕಬ್ಬಿಣದಂತಹ ಮಾಂಸಖಂಡ, ಉಕ್ಕಿನಂತಹ ನರ, ಎದುರಿಸುವುದಕ್ಕೆ ಅಸದಳವಾದ ವಿಶ್ವದ ರಹಸ್ಯದಂತ ರಾಳವನ್ನು ಭೇದಿಸಿ, ಸಮಯ ಬಂದರೆ, ಕಡಲಾಳಕ್ಕೆ ನುಗ್ಗಿ ಮೃತ್ಯುವನ್ನು ಎದುರಿಸಿ, ತಮ್ಮ ಇಚ್ಛೆಯನ್ನು ಜಯಪ್ರದವಾಗಿ ಈಡೇರಿಸಿಕೊಳ್ಳಬಲ್ಲ ಪ್ರಚಂಡ ಇಚ್ಛಾಶಕ್ತಿ.

ನಾವು ಬಹಳ ಕಾಲ ಅತ್ತಿರುವೆವು. ಇನ್ನು ಹೆಚ್ಚು ಅಳಕೂಡದು. ನಿಮ್ಮ ಕಾಲಮೇಲೆ ನಿಂತು ಮನುಷ್ಯರಾಗಿ. ನಮಗಿಂದು ಪುರುಷಸಿಂಹರನ್ನಾಗಿ ಮಾಡುವ ಧರ್ಮ ಬೇಕು, ಪುರುಷಸಿಂಹರನ್ನಾಗಿ ಮಾಡುವ ಸಿದ್ಧಾಂತ ಬೇಕು, ಪುರುಷಸಿಂಹರನ್ನಾಗಿ ಮಾಡುವ ಸರ್ವತೋಮುಖಿಗಳನ್ನಾಗಿ ಮಾಡುವ ವಿದ್ಯಾಭ್ಯಾಸ ಬೇಕು. ಸತ್ಯದ ಪರೀಕ್ಷೆ ಇಲ್ಲಿರುವುದು. ಯಾವುದು ನಿಮ್ಮನ್ನು ದೈಹಿಕವಾಗಾಗಲಿ, ಮಾನಸಿಕವಾಗಾಗಲಿ, ಆಧ್ಯಾತ್ಮಿಕವಾಗಾಗಲಿ, ದುರ್ಬಲ ರನ್ನಾಗಿ ಮಾಡುವುದೋ ಅದನ್ನು ವಿಷದಂತೆ ತ್ಯಜಿಸಿ. ಅದರಲ್ಲಿ ಚೇತನವಿಲ್ಲ. ಅದು ಸತ್ಯವಾಗಿರಲಾರದು. ಸತ್ಯ ಶಕ್ತಿವರ್ಧಕ. ಸತ್ಯ ಪರಿಶುದ್ಧವಾದುದು. ಸತ್ಯವೇ ಅನಂತಜ್ಞಾನ. ಸತ್ಯ ಶಕ್ತಿಯನ್ನು ಕೊಡಬೇಕು, ಬೆಳಕನ್ನು ತರಬೇಕು, ಹೊಸಚೇತನವನ್ನು ತುಂಬಬೇಕು.

ನಾವು ಅರಗಿಳಿಯಂತೆ ಎಷ್ಟೋ ವಿಷಯಗಳನ್ನು ಕುರಿತು ಮಾತನಾಡು ವೆವು. ಆದರೆ ಅದನ್ನು ಎಂದಿಗೂ ಮಾಡುವುದಿಲ್ಲ. ಬರಿ ಮಾತು. ಕೆಲಸವಿಲ್ಲ. ಇದು ನಮ್ಮ ಬಾಳಚಾಳಿಯಾಗಿದೆ. ಇದಕ್ಕೆ ಕಾರಣವೇನು? ದೈಹಿಕ ದುರ್ಬಲತೆ. ದುರ್ಬಲ ಮೆದುಳು ಏನನ್ನೂ ಸಾಧಿಸಲಾರದು. ಅದನ್ನು ನಾವು ಬಲಗೊಳಿಸ ಬೇಕು. ಮೊದಲು ನಮ್ಮ ಯುವಕರು ಬಲಿಷ್ಠರಾಗಬೇಕು. ಧರ್ಮ ನಂತರ ಬರುವುದು. ಗೀತಾಧ್ಯಯನಕ್ಕಿಂತ ಚಂಡಿನಾಟದಿಂದ ನೀವು ಸ್ವರ್ಗದ ಸಮೀಪಕ್ಕೆ ಸಾಗುವಿರಿ. ಭುಜಬಲ ಹೆಚ್ಚಾಗಿ, ಮಾಂಸಖಂಡಗಳು ಬಲವಾ ದಾಗ, ನೀವು ಗೀತೆಯನ್ನು ಚೆನ್ನಾಗಿ ಅರ್ಥಮಾಡಿಕೊಳ್ಳಬಲ್ಲಿರಿ. ನಿಮ್ಮಲ್ಲಿ ಸ್ವಲ್ಪ ಹೆಚ್ಚು ಶಕ್ತಿವರ್ಧಕ ರಕ್ತವಿದ್ದರೆ ಶ್ರೀಕೃಷ್ಣನ ಅದ್ಭುತಶಕ್ತಿ ಮತ್ತು ಸಾಮರ್ಥ್ಯವನ್ನು ಹೆಚ್ಚು ಗ್ರಹಿಸಬಲ್ಲಿರಿ. ನೀವು ನಿಜವಾಗಿ ಮನುಷ್ಯರಾಗಿ, ನಡುಗದೆ ನಿಮ್ಮ ಕಾಲಮೇಲೆ ನಿಂತಾಗ ಉಪನಿಷತ್ತನ್ನು ಚೆನ್ನಾಗಿ ತಿಳಿದು ಕೊಳ್ಳಬಲ್ಲಿರಿ. ಆತ್ಮನ ಮಹಿಮೆಯನ್ನು ಚೆನ್ನಾಗಿ ತಿಳಿದುಕೊಳ್ಳಬಲ್ಲಿರಿ.

ವಿಶಾಲವಾಗುವುದಕ್ಕೆ ಪ್ರಯತ್ನಿಸಿ. ಚಟುವಟಿಕೆ ಮತ್ತು ಬೆಳವಣಿಗೆಯೇ ಚೇತನದ ಚಿಹ್ನೆ ಎಂಬುದನ್ನು ಜ್ಞಾಪಕದಲ್ಲಿಡಿ.

ನೈತಿಕರಾಗಿ. ಧೀರರಾಗಿ. ನಿಮ್ಮ ಹೃದಯವನ್ನೆಲ್ಲ ಇದಕ್ಕೆ ಸಮರ್ಪಿಸಿ. ಸ್ವಲ್ಪವೂ ಅಚ್ಚಳಿಯದ ಚಾರಿತ್ರ್ಯವಿರಲಿ. ಪ್ರಾಣಹೋದರೂ ಧೀರರಾಗಿ. ಧಾರ್ಮಿಕ ಸಿದ್ಧಾಂತಗಳನ್ನು ಕುರಿತು ನಿಮ್ಮ ತಲೆಯನ್ನು ಕೆಡಿಸಿಕೊಳ್ಳಬೇಡಿ. ಹೇಡಿಗಳು ಪಾಪ ಮಾಡುವರು. ಧೀರರೆಂದಿಗೂ ಅಲ್ಲ. ಎಲ್ಲರನ್ನೂ ಪ್ರೀತಿ ಸಲು ಯತ್ನಿಸಿ.

ಕಾಪಟ್ಯದಿಂದ ಯಾವ ಮಹಾಕಾರ್ಯವನ್ನೂ ಮಾಡಲು ಆಗುವುದಿಲ್ಲ. ಪ್ರೀತಿಯಿಂದ, ಸತ್ಯನಿಷ್ಠೆಯಿಂದ, ಅತ್ಯದ್ಭುತ ಶಕ್ತಿಯಿಂದ ಮಾತ್ರ ಎಲ್ಲಾ ಉದ್ಯಮಗಳೂ ಜಯಪ್ರದವಾಗುವುವು. ಆದಕಾರಣ ನಿಮ್ಮ ಪೌರುಷವನ್ನು ವ್ಯಕ್ತಪಡಿಸಿ, ನಮಗೆ ಸದ್ಯಕ್ಕೆ ಅತಿ ತ್ವರಿತವಾಗಿ ಬೇಕಾಗಿರುವುದು ರಾಜಸಿಕ ಸ್ವಭಾವ. ಈಗ ನೀವು ಸಾತ್ತ್ವಿಕರೆಂದು ತಿಳಿದುಕೊಂಡಿರುವವರಲ್ಲಿ ಶೇಕಡ ತೊಂಬತ್ತು ಮಂದಿ ತಾಮಸ ಸ್ವಭಾವದಲ್ಲಿ ಮಗ್ನರಾಗಿರುವರು. ನಮಗೆ ಇಂದು ಅದ್ಭುತ ರಾಜಸಿಕ ಶಕ್ತಿ ಬೇಕಾಗಿದೆ. ದೇಶವೆಲ್ಲ ಶುದ್ಧ ತಮಸ್ಸಿನಲ್ಲಿ ಆವೃತವಾಗಿರುವುದು. ಈ ದೇಶದ ಜನರಿಗೆ ಆಹಾರ ಕೊಡಬೇಕು, ವಸನ ಕೊಡಬೇಕು, ಅವರನ್ನು ಜಾಗ್ರತರನ್ನಾಗಿ ಮಾಡಬೇಕು, ಹೆಚ್ಚು ಚಟುವಟಿಕೆ ಯುಳ್ಳವರನ್ನಾಗಿ ಮಾಡಬೇಕು. ಇಲ್ಲದೇ ಇದ್ದರೆ ಅವರು ಗಿಡಮರಗಳಂತೆ, ಕಲ್ಲು ಬಂಡೆಯಂತೆ ಚೇತನಹೀನರಾಗುವರು.

ಆದರ್ಶವುಳ್ಳ ಮನುಷ್ಯ ಸಾವಿರ ತಪ್ಪನ್ನು ಮಾಡಿದರೆ, ಆದರ್ಶವಿಲ್ಲದವನು ಐವತ್ತು ಸಾವಿರ ತಪ್ಪನ್ನು ಮಾಡುವನೆಂಬುದಕ್ಕೆ ಸಂದೇಹವಿಲ್ಲ. ಆದಕಾರಣ ಒಂದು ಆದರ್ಶ ಇರುವುದು ಒಳ್ಳೆಯದು.

ಓ! ನಿಮ್ಮ ಸ್ವಭಾವ ನಿಮಗೆ ಗೊತ್ತಿಲ್ಲವೆ! ನೀವು ಆತ್ಮ. ನೀವು ದೇವರು. ನಾನು ನಿಮ್ಮನ್ನು ಮನುಷ್ಯರೆಂದು ಕರೆದಾಗ ಮಹಾಪಾತಕವನ್ನು ಮಾಡುತ್ತಿರುವೆನು.

ನಾನು ಸತ್ಯಕ್ಕೆ ನಿಲ್ಲುವೆನು. ಸತ್ಯವೆಂದಿಗೂ ಮಿಥ್ಯದೊಂದಿಗೆ ರಾಜಿ ಮಾಡಿ ಕೊಳ್ಳಲಾರದು. ಪ್ರಪಂಚವೇ ತನ್ನನ್ನು ಎದುರಿಸಿದರೂ ಕೊನೆಗೆ ಸತ್ಯವೇ ಜಯಿಸಬೇಕು. ಹೇಡಿಗಳಿಗಲ್ಲ ಈ ಪ್ರಪಂಚ. ಓಡಿಹೋಗುವುದಕ್ಕೆ ಯತ್ನಿಸ ಬೇಡಿ. ಜಯಾಪಜಯಗಳನ್ನು ಲೆಕ್ಕಿಸಬೇಡಿ.

ನಾನು ಸೇಡು ತೀರಿಸಿಕೊಳ್ಳುವ ಮಾತನ್ನೇ ಎಂದಿಗೂ ಆಡಲಿಲ್ಲ. ನಾನು ಯಾವಾಗಲೂ ಸತ್ಯವನ್ನು ಒತ್ತಿ ಹೇಳಿರುವೆನು. ಸಮುದ್ರದ ನೊರೆಯ ಮೇಲೆ ಸೇಡನ್ನು ತೀರಿಸಿಕೊಳ್ಳುವೆವು ಎಂದು ಕನಸಿನಲ್ಲಾದರೂ ಎಂದಾದರೂ ಭಾವಿ ಸುವೆವೆ? ಆದರೆ ಕ್ಷುದ್ರಕೀಟಕ್ಕೆ ಇದೇ ಒಂದು ಮಹಾಸಾಹಸಕಾರ್ಯ.

ಜಾಗ್ರತರಾಗಿ. ಚಕ್ರಕ್ಕೆ ನಿಮ್ಮಹೆಗಲನ್ನು ಕೊಡಿ. ಈ ಜೀವನವಿರುವುದು ಏತಕ್ಕೆ? ನೀವು ಈ ಪ್ರಪಂಚಕ್ಕೆ ಬಂದ ಮೇಲೆ ನಿಮ್ಮ ಚಿಹ್ನೆಯನ್ನೇನಾದರೂ ಅಲ್ಲಿ ಬಿಡಿ. ಇಲ್ಲದೆ ಇದ್ದರೆ ನಿಮಗೂ, ಗಿಡಮರಗಳಿಗೂ, ಕಲ್ಲಿಗೂ ಏನು ವ್ಯತ್ಯಾಸ. ಅವೂ ಪ್ರಪಂಚಕ್ಕೆ ಬರುವುವು. ಕ್ಷಯಿಸುವುವು. ಸಾಯುವುವು.

ಧೀರರಾಗಿ. ಎಲ್ಲರಿಗಿಂತ ನನ್ನ ಮಕ್ಕಳು ಧೈರ್ಯಶಾಲಿಗಳಾಗಬೇಕು. ಯಾವ ಸಮಯದಲ್ಲಿಯೂ ಯಾವುದರೊಂದಿಗೂ ರಾಜಿಮಾಡಿಕೊಳ್ಳ ಕೂಡದು. ಉದಾತ್ತ ಸತ್ಯವನ್ನು ಬೋಧಿಸಿ, ಹರಡಿ. ನಿಮ್ಮ ಆತ್ಮಗೌರವಕ್ಕೆ ಚ್ಯುತಿಬರುವುದೆಂದು, ಅನಾವಶ್ಯಕವಾಗಿ ಮನಸ್ತಾಪ ಬರುವುದೆಂದು ಎಣಿಸ ಬೇಡಿ. ಸತ್ಯವನ್ನು ತ್ಯಜಿಸೆಂದು ಪ್ರೇರೇಪಿಸುವ ಪ್ರಲೋಭನೆಯನ್ನು ಮೀರಿ. ಯಾರು ಸತ್ಯನಿಷ್ಠರಾಗಿರುವರೋ ಅವರಿಗೆ ಒಂದು ಸ್ವರ್ಗೀಯಶಕ್ತಿ ಪ್ರಾಪ್ತ ವಾಗುವುದು. ಅವರೆದುರಿಗೆ ನೀನು ಅಸತ್ಯವೆಂದು ತಿಳಿದುಕೊಂಡಿರುವುದನ್ನು ಜನರು ವ್ಯಕ್ತಪಡಿಸಲು ಅಂಜುವರು. ನಿರಂತರವಾಗಿ ಹದಿನಾಲ್ಕು ವರ್ಷ ನೀನು ಸತ್ಯವನ್ನು ಉಪಾಸನೆ ಮಾಡಿದರೆ ನೀನು ಜನರಿಗೆ ಹೇಳುವುದೆಲ್ಲವನ್ನೂ ಅವರು ನಂಬುವರು.

ಮಗು, ನನಗೆ ಬೇಕಾಗಿರುವುದು ಕಬ್ಬಿಣದಂತಹ ಮಾಂಸಖಂಡ, ಉಕ್ಕಿ ನಂತಹ ನರ. ಒಳಗೆ ಸಿಡಿಮದ್ದಿನಂತಿರುವ ಮನಸ್ಸು.

ಸತ್ಯ ಅಸತ್ಯಕ್ಕಿಂತಲೂ ಅಪಾರ ಪರಿಣಾಮಕಾರಿ, ಅದರಂತೆಯೇ ಒಳ್ಳೆ ಯದು ಕೂಡ. ಇವರೆಡೂ ನಿನ್ನಲ್ಲಿದ್ದರೆ ಅವೇ ತಮ್ಮ ಸ್ವಂತ ಶಕ್ತಿಯಿಂದ ದಾರಿ ಬಿಡಿಸುವುವು.

ಪರ್ವತೋಪಮ ಅಡ್ಡಿ ಆತಂಕಗಳಿಂದ ಪಾರಾಗಬಲ್ಲ ಇಚ್ಛಾಶಕ್ತಿ ನಿನ್ನ ಲ್ಲಿದೆಯೇ? ಪ್ರಪಂಚವೇ ಹಿರಿದ ಕತ್ತಿಯಿಂದ ನಿನ್ನನ್ನು ಪ್ರತಿಭಟಿಸುತ್ತಿದ್ದರೂ ನಿನಗೆ ಯಾವುದು ಧರ್ಮವೆಂದು ತೋರುವುದೋ ಅದನ್ನು ಮಾಡಲು ಸಾಹಸ ಇದೆಯೆ? ನಿನ್ನ ಹೆಂಡತಿ ಮಕ್ಕಳು ವಿರೋಧಿಸಿದರೂ, ನಿನ್ನ ಹಣವೆಲ್ಲ ವ್ಯಯ ವಾದರೂ ಕೀರ್ತಿ ಹೋದರೂ, ಐಶ್ವರ್ಯ ಹೋದರೂ, ನೀನು ಅದನ್ನು ಬಿಡುವುದಿಲ್ಲವೆ? ಆದರೂ ಅದನ್ನು ನೀನು ಅನುಸರಿಸಿ ನಿನ್ನ ಜೀವನದ ಗುರಿ ಎಡೆಗೆ ಸತತವೂ ಮುಂದುವರಿಯಬಲ್ಲೆಯೋ? ಸತ್ಯ ನಿನ್ನನ್ನು ಎಲ್ಲಿ ಗೊಯ್ದರೂ ಹೋಗು. ಭಾವವನ್ನು ಆಮೂಲಾಗ್ರವಾಗಿ ಅನುಷ್ಠಾನಕ್ಕೆ ತೆಗೆದು ಕೊಂಡು ಬಾ. ಹೇಡಿಯಾಗಬೇಡ, ಕಪಟಿಯಾಗಬೇಡ.

ಸತ್ಯ, ಅಸತ್ಯವನ್ನು ಮೆಟ್ಟಿ ನಿಲ್ಲುವುದೆಂಬ ಆಶಾಭಾವದಿಂದ ಪ್ರಯತ್ನಿಸಿ. ನಾವು ಸುಮ್ಮನೆ ಕುಳಿತುಕೊಂಡು ನಮ್ಮ ದೇಹ ಅಥವಾ ಮನಸ್ಸಿನ ಕುಂದು ಕೊರತೆಗಳನ್ನೇ ಕುರಿತು ಆಲೋಚಿಸುತ್ತಿದ್ದರೆ ಪ್ರಯೋಜನವಿಲ್ಲ. ನಮಗೆ ವಿರೋಧವಾದ ಪ್ರಸಂಗದಲ್ಲಿ ಅವುಗಳನ್ನು ಮೆಟ್ಟಿ ನಿಲ್ಲಬಲ್ಲೆ ಎಂಬ ಧೀರ ಯತ್ನವೊಂದೇ ನಮ್ಮನ್ನು ಮುಂದೆ ಕರೆದೊಯ್ಯುವುದು.

ಯಾಗ, ಯಜ್ಞಗಳು, ನಮಸ್ಕಾರ, ಅರಗಿಳಿಯಂತೆ ಮಂತ್ರೋಚ್ಚಾರಣೆ ಇವೇ ಧರ್ಮವಲ್ಲ. ಘನವಾದ ಸಾಹಸ ಕಾರ್ಯಗಳಿಗೆ ಅದು ನಮ್ಮನ್ನು ಪ್ರಚೋಸಿದರೆ ಭಗವಂತನ ಪರಿಪೂರ್ಣತೆಯ ಅರಿವನ್ನು ಕೊಟ್ಟರೆ ಮಾತ್ರ ಅವು ಹಿತಕಾರಿ.

ಮೊದಲು ನಾವು ದೇವರಾಗೋಣ. ನಂತರ ದೇವರಂತೆ ಆಗುವುದಕ್ಕೆ ಇತರರಿಗೆ ಸಹಾಯ ಮಾಡೋಣ. ನೀವು ದೇವರಾಗಿ, ಮತ್ತೊಬ್ಬರನ್ನೂ ದೇವರನ್ನಾಗಿ ಮಾಡಿ. ಇದೇ ನಮ್ಮ ಆದರ್ಶದ ಪಲ್ಲವಿಯಾಗಲಿ.

ನಮ್ಮ ಶಾಸ್ತ್ರ ಮಾತ್ರ ದೇವರಿಗೆ “ಅಭೀಃ ಅಭೀಃ” ಎಂದು ಕರೆಯುವುದು. ನೀವು “ಅಭೀಃ”, ನಿರ್ಭಯರಾಗಬೇಕು. ಆಗ ನಮ್ಮ ಕೆಲಸ ಪೂರೈಸಿದಂತೆ.

ನಿಮ್ಮ ಆಂತರ್ಯದಲ್ಲಿರುವ ದೇವತ್ವವನ್ನು ಪ್ರಕಾಶಿಸಿ, ಉಳಿದುದೆಲ್ಲ ಅದರ ಸುತ್ತಲೂ ಅಂದವಾಗಿ ತಾವೇ ಅಣಿಯಾಗುವುವು.


\section{ಸೇವೆ}

ಪ್ರತಿಯೊಬ್ಬ ನರನಾರಿಯರನ್ನೂ ದೇವರೆಂದು ನೋಡು. ನೀನು ಯಾರಿಗೂ ಸಹಾಯ ಮಾಡಲಾರೆ. ಭಗವಂತನ ಮಕ್ಕಳಿಗೆ ಸೇವೆ ಮಾಡಬಲ್ಲೆ. ಅದೃಷ್ಟವಿದ್ದರೆ ಭಗವಂತನಿಗೇ ಸೇವೆ ಮಾಡಬಲ್ಲೆ. ಭಗವಂತ ತನ್ನ ಯಾವ ಮಗುವಿಗಾದರೂ ಸೇವೆ ಮಾಡೆಂದು ನಿನಗೆ ಆಜ್ಞಾಪಿಸಿದರೆ ನೀನೇ ಭಾಗ್ಯವಂತ. ನೀನೇ ಬಹಳ ಮುಖ್ಯವೆಂದು ಯೋಚಿಸಬೇಡ. ಮತ್ತೊಬ್ಬನಿಗೆ ಈ ಅದೃಷ್ಟ ವಿಲ್ಲದೇ ಇರುವಾಗ ನಿನಗೆ ಅದು ದೊರಕಿದುದರಿಂದ ನೀನೇ ಭಾಗ್ಯಶಾಲಿ. ಸೇವೆಯನ್ನು ಪೂಜೆಯಂತೆ ಮಾಡು. ದೀನರು ದರಿದ್ರರು ನಮ್ಮ ಉದ್ಧಾರ ಕ್ಕಾಗಿ ಇರುವರು. ರೋಗಿಯಂತೆ, ಹುಚ್ಚನಂತೆ, ಕುಷ್ಠನಂತೆ, ಪಾಪಿಯಂತೆ ಬರುವ ಭಗವಂತನಿಗೆ ಸೇವೆ ಮಾಡಲು ನಮಗೆ ಅವಕಾಶ ಕಲ್ಪಿಸುವನು.

ಮಾನವ ದೇಹದಲ್ಲಿರುವ ಅವನ ಆತ್ಮವೇ ಆರಾಧನೆಯ ಏಕಮಾತ್ರ ಮೂರ್ತಿ. ಪ್ರಾಣಿಗಳೆಲ್ಲ ಅವನ ಮಂದಿರವೇನೋ ನಿಜ. ಆದರೆ ಮಾನವ ತಾಜಮಹಲ್​ನಂತೆ, ದೇಗುಲಗಳಲ್ಲೆಲ್ಲ ಸರ್ವಶ್ರೇಷ್ಠವಾದ ದೇಗುಲದಂತೆ. ನಾನು ಅಲ್ಲಿ ಆತನನ್ನು ಆರಾಧಿಸದೆ ಇದ್ದರೆ, ಮತ್ತಾವ ದೇಗುಲದಿಂದಲೂ ಏನು ಪ್ರಯೋಜನ?

ಪ್ರತಿದಿನವೂ ಅಧಃಪಾತಾಳಕ್ಕೆ ಹೋಗುತ್ತಿರುವ ಮೂವತ್ತುಕೋಟಿ ಮಾನ ವರ ಸೇವೆಗಾಗಿ ಇಡೀ ಜನ್ಮವನ್ನು ಅರ್ಪಿಸುವೆನೆಂದು ಬದ್ಧ ಕಂಕಣನಾಗು. ಯಾರ ಹೃದಯ ದೀನರಿಗೆ ಮರುಗುವುದೋ ಅವರನ್ನು ಮಹಾತ್ಮರೆಂದು ಕರೆಯುತ್ತೇನೆ. ಇಲ್ಲದೇ ಇದ್ದರೆ ಅವನೊಬ್ಬ ದುರಾತ್ಮ. 

ಎಲ್ಲಿಯವರೆವಿಗೂ ಕೋಟ್ಯಂತರ ಜನ ಅಜ್ಞಾನದಲ್ಲಿ, ಉಪವಾಸದಲ್ಲಿ ನರಳುತ್ತಿರುವರೋ, ಅಲ್ಲಿಯವರೆವಿಗೂ ಇತರರ ದುಡಿತದಿಂದ ಕೃತವಿದ್ಯ ನಾಗಿ, ಕಡೆಗೆ, ಅವರನ್ನೇ ನಿರ್ಲಕ್ಷ್ಯದಿಂದ ನೋಡುವ ಎಲ್ಲಾ ಮಾನವರನ್ನೂ ಕುಲಘಾತುಕರೆಂದು ಕರೆಯುತ್ತೇವೆ.

ನಿಮ್ಮ ಜನರನ್ನು ನೀವು ಪ್ರೀತಿಸುವಿರೇನು? ನೀವು ದೇವರನ್ನು ಹುಡುಕಿ ಕೊಂಡು ಎಲ್ಲಿಗೆ ಹೋಗುವಿರಿ? ಎಲ್ಲಾ ದೀನರೂ, ದುಃಖಿಗಳೂ, ದುರ್ಬಲರೂ ದೇವರಲ್ಲವೆ? ಮೊದಲು ಅವರನ್ನು ಏಕೆ ಪೂಜಿಸಬಾರದು? ಗಂಗಾನದಿಯ ತೀರದಲ್ಲಿ ಬಾವಿಯನ್ನು ಏತಕ್ಕೆ ತೋಡುವುದಕ್ಕೆ ಹೋಗು ತ್ತೀರಿ? ಪ್ರೇಮದ ಅನಂತಶಕ್ತಿಯನ್ನು ನಂಬಿ. ಹುರುಳಿಲ್ಲದ ಬರಿ ನೊರೆಯ ಕೀರ್ತಿಯನ್ನು ಯಾರು ಲೆಕ್ಕಿಸುವರು? ನಿಮಗೆ ಪ್ರೀತಿ ಇದೆಯೆ?–ಇದ್ದರೆ ಸರ್ವಶಕ್ತರು ನೀವು. ನೀವು ಸಂಪೂರ್ಣ ನಿಃಸ್ವಾರ್ಥಿಗಳೆ? ಇದ್ದರೆ ನಿಮ್ಮನ್ನು ಎದುರಿಸುವವರು ಯಾರೂ ಇಲ್ಲ. ಎಲ್ಲಾ ಕಡೆ ಆದರಣೆ ಶೀಲಕ್ಕೆ ಮಾತ್ರ. ದ್ವೇಷ, ಅಸೂಯೆಯನ್ನು ತ್ಯಜಿಸಿ. ಇತರರಿಗೆ ಸಾಮೂಹಿಕವಾಗಿ ದುಡಿಯು ವುದನ್ನು ಅಭ್ಯಾಸ ಮಾಡಿ. ನಮ್ಮ ದೇಶಕ್ಕೆ ಅತ್ಯಾವಶ್ಯಕವಾಗಿ ಬೇಕಾಗಿರುವುದು ಇದು. ತಾಳ್ಮೆ ಇರಲಿ, ಆಮರಣಾಂತ ನಿಷ್ಠಾವಂತರಾಗಿ. ನಿಮ್ಮಲ್ಲೇ ಜಗಳ ಕಾಯಬೇಡಿ. ಹಣದ ವಿಷಯದಲ್ಲಿ ನಿಮ್ಮ ನಡತೆ ಪರಿಶುದ್ಧವಾಗಿರಲಿ. ನಿಮ್ಮಲ್ಲಿ ಎಲ್ಲಿಯವರೆವಿಗೂ ಸತ್ಯಸಂಧತೆ, ನಿಷ್ಠೆ, ದೈವಭಕ್ತಿ ಇರುವುದೋ, ಅಲ್ಲಿಯವರೆವಿಗೂ ಎಲ್ಲಾ ಅಭಿವೃದ್ಧಿಯಾಗುವುದು. ಭಗವಂತನ ದಯೆ ಯಿಂದ ನಿಮ್ಮಲ್ಲಿ ಎಲ್ಲಿಯವರೆವಿಗೂ ಅನೈಕ್ಯತೆ ಬರುವುದಿಲ್ಲವೋ, ಅಲ್ಲಿಯ ವರೆವಿಗೂ ನಿಮಗೆ ಯಾವ ಅಪಾಯವೂ ಪ್ರಾಪ್ತವಾಗಲಾರದೆಂದು ಭರವಸೆ ಕೊಡುತ್ತೇನೆ. ನಿಜವಾಗಿಯೂ ಮತ್ತೊಬ್ಬರಿಗೆ ಸಹಾಯವಾಗುವುದೆಂದು ನಿಸ್ಸಂದೇಹವಾಗಿ ನಿಮಗೆ ಗೊತ್ತಾಗುವ ತನಕ ಮನಸ್ಸಿನಲ್ಲಿರುವುದನ್ನು ಪ್ರದರ್ಶಿಸಬೇಡಿ. ಪರಮ ವೈರಿಗೂ ಹಿತಕರವಾದ ಒಳ್ಳೆಯ ಭಾಷೆಯನ್ನು ಉಪಯೋಗಿಸಿ.

ವಿಕಾಸವೇ ಜೀವನ. ಸಂಕೋಚವೇ ಮರಣ. ಎಲ್ಲಾ ಬಗೆಯ ಪ್ರೀತಿಯೂ ವಿಕಾಸ. ಸ್ವಾರ್ಥತೆಯೇ ಸಂಕೋಚ, ಪ್ರೀತಿಯೇ ಜೀವನದ ಏಕಮಾತ್ರ ನಿಯಮ. ಯಾರು ಪ್ರೀತಿಸುವರೋ ಅವರು ಬದುಕಿರುವರು. ಯಾರು ಸ್ವಾರ್ಥಿಗಳೊ ಅವರು ಸಾಯುತ್ತಿರುವರು. ಪ್ರೀತಿಗಾಗಿ ಪ್ರೀತಿಸಿ. ಅದು ಒಂದೇ ಜೀವನದ ನಿಯಮ.

ನಮ್ಮ ಗುರಿ ಜಗತ್ತಿಗೆ ಒಳ್ಳೆಯದನ್ನು ಮಾಡುವುದು. ನಮ್ಮ ಹೆಸರನ್ನು ಕೊಚ್ಚಿಕೊಳ್ಳುವುದಲ್ಲ.

ನಿಮ್ಮೆಲ್ಲರಿಂದ ನಾನು ಇದನ್ನು ಬಯಸುವೆನು. ಸ್ವಪ್ರತಿಷ್ಠೆ, ವೈಮನಸ್ಯ ವನ್ನು ಹರಡುವುದು, ಅಸೂಯೆ ಇವುಗಳು ತ್ಯಾಜ್ಯ. ನೀವೆಲ್ಲ ಭೂದೇವಿಯಂತೆ ಸಹನಶೀಲರಾಗಬೇಕು. ನೀವು ಇದನ್ನು ಸಾಧಿಸಿದರೆ ಪ್ರಪಂಚವೇ ನಿಮ್ಮ ಪದದಡಿ ಬಾಗುವುದು.

ಯಾರು ಆಣತಿಯನ್ನು ಪಾಲಿಸಬಲ್ಲರೋ ಅವರು ಆಜ್ಞೆಯನ್ನು ಮಾಡ ಬಲ್ಲರು. ಮೊದಲು ವಿಧೇಯತೆಯನ್ನು ಅಭ್ಯಾಸಮಾಡಿ. ನಮಗೆ ವ್ಯವಸ್ಥಿತ ಸಂಸ್ಥೆ ಬೇಕು. ಅದೇ ಶಕ್ತಿ. ಅದರ ಮೂಲ ವಿಧೇಯತೆಯಲ್ಲಿದೆ.

ಯಾರು ಮಾನವಕೋಟಿಗೆ ಸಹಾಯ ಮಾಡಬೇಕೆಂದು ಬಯಸುವರೋ ಅವರು ತಮ್ಮ ಸ್ವಂತಸುಖ, ದುಃಖ, ಹೆಸರು, ಕೀರ್ತಿ ಇವುಗಳನ್ನು ಮೂಟೆಕಟ್ಟಿ, ಕಡಲಿಗೆ ಬಿಸುಟು ದೇವರ ಸಮೀಪಕ್ಕೆ ಬರಬೇಕು. ಇದನ್ನೇ ಜಗದ ಮಹಾತ್ಮರೆಲ್ಲ ಬೋಧಿಸಿದ್ದು ಮತ್ತು ಅನುಸರಿಸಿದ್ದು.

ಪಕ್ಷಪಾತವೇ ಎಲ್ಲಾ ಪಾಪಕ್ಕೂ ಮೂಲವೆಂಬುದನ್ನು ತಿಳಿ. ನನ್ನ ಮಾನವ ಬಂಧುಗಳನ್ನು ಸೇವಿಸಬೇಕು. ನಾನು ಅದೊಂದನ್ನೇ ಆಶಿಸುವುದು.

ನಿಮಗೆ ಏನಾದರೂ ಒಳ್ಳೆಯದಾಗಬೇಕಾದರೆ ನಿಮ್ಮ ಬಾಹ್ಯ ಕರ್ಮಾಚರಣೆ ಯನ್ನು ಆಚೆಗೆ ಒಗೆದು, ಮಾನವರೂಪನ್ನು ಧರಿಸಿರುವ, ವಿರಾಡ್ರೂಪಿಯಾಗಿ ಮತ್ತು ವ್ಯಕ್ತಿರೂಪಿಯಾಗಿರುವ ಸಜೀವ ಮಾನವ ದೇವನನ್ನು ಪೂಜಿಸಿ.

ದಯೆಯಿಂದ ಪ್ರೇರೇಪಿತರಾಗಿ ಮತ್ತೊಬ್ಬರಿಗೆ ಹಿತಕರವಾದುದನ್ನು ಮಾಡುವುದು ಒಳ್ಳೆಯದು. ಇವರೆಲ್ಲರೂ ಭಗವಂತನ ಸ್ವರೂಪವೆಂದು ಭಾವಿಸಿ ಸೇವಿಸುವುದು ಅದಕ್ಕಿಂತ ಶ್ರೇಯಸ್ಕರ.

ಶಾಂತವಾಗಿ, ಮೌನದಿಂದ ಸತತ ಕರ್ಮದಲ್ಲಿ ನಿರತನಾಗು. ವೃತ್ತಪತ್ರಿಕಾ ಕೀರ್ತಿಯಲ್ಲ. ಇದನ್ನು ನೆನಪಿನಲ್ಲಿಡಿ.

ಹೇ ಕುಲಘಾತಕರೆ! ಒಂದು ಕಡೆ ಗುಲಾಮಗಿರಿ. ಅದರ ಮತ್ತೊಂದು ಕಡೆಯೆ ದಬ್ಬಾಳಿಕೆ ಎಂಬುದು ನಿಮಗೆ ತಿಳಿಯದು. ಕ್ರೌರ್ಯ ಮತ್ತು ಗುಲಾಮಗಿರಿ ಪರ್ಯಾಯ ಪದ. ನೀವೆಲ್ಲ, ಎಲ್ಲಿ ಜನರು ಪ್ಲೇಗು, ಬರಗಾಲ, ಅಥವಾ ಮತ್ತಾವುದಾದರೂ ಉಪದ್ರವಕ್ಕೆ ಸಿಕ್ಕಿ ನರಳುತ್ತಿರುವರೋ ಅಲ್ಲಿಗೆ ಹೋಗಿ ಅವರ ಕಷ್ಟವನ್ನು ತಗ್ಗಿಸಲು ಯತ್ನಿಸಿ. ಮಿತಿಮೀರಿತು ಎಂದರೆ ನೀವು ಪ್ರಯತ್ನದಲ್ಲಿ ಸಾಯಬಹುದು. ಆದರೇನು? ನಿಮ್ಮಂತಹವರು ಎಷ್ಟು ಪ್ರತಿ ದಿನವೂ ಕ್ರಿಮಿಗಳಂತೆ ಹುಟ್ಟಿ ಸಾಯುವರು. ಇದರಿಂದ ವಿಶಾಲ ಪ್ರಪಂಚಕ್ಕೆ ಏನು ಕಡಿಮೆ? ನೀವು ಸಾಯಲೇಬೇಕಾಗಿರುವಾಗ ಪ್ರಾಣ ಒಪ್ಪಿಸುವುದಕ್ಕೆ ಒಂದು ಘನ ಆದರ್ಶವಾದರೂ ಇರಲಿ. ಜೀವನದಲ್ಲಿ ಒಂದು ಘನ ಉದ್ದೇಶಕ್ಕೆ ಸಾಯುವುದು ಮೇಲು. ಈ ಸಂದೇಶವನ್ನು ಮನೆಯಿಂದ ಮನೆಗೆ ಹರಡಿ. ಇದರಿಂದ ನಿಮಗೂ ಮೇಲು. ದೇಶಕ್ಕೂ ಮೇಲು. ನಿಮ್ಮ ಮೇಲೆ ನಮ್ಮ ದೇಶದ ಭವಿಷ್ಯ ನಿಂತಿದೆ. ನೀವು ಶುದ್ಧ ಸೋಮಾರಿಗಳಾಗಿ ಬಾಳುತ್ತಿರು ವುದನ್ನು ನೋಡಿ ನನಗೆ ವ್ಯಥೆಯಾಗುವುದು. ಕಾರ್ಯಕ್ಕೆ ಕೈಹಾಕಿ. ಕಾಲವಿಳಂಬ ಮಾಡಬೇಡಿ. ದಿನಕಳೆದಂತೆ ಮೃತ್ಯು ಸಮೀಪಿಸುತ್ತಿರುವುದು. ಎಲ್ಲಾ ನಂತರ ಸಕಾಲದಲ್ಲಿ ಆಗುವುದೆಂದು ಯೋಚಿಸುತ್ತಾ ಸುಮ್ಮನೆ ಕುಳಿತುಕೊಳ್ಳಬೇಡಿ. ಯಾವುದನ್ನೂ ಹೀಗೆ ಸಾಧಿಸುವುದಕ್ಕೆ ಆಗುವುದಿಲ್ಲ. ಇದನ್ನು ನೆನಪಿನಲ್ಲಿಡಿ.

ಶಕ್ತಿ ಮುಂತಾದುವೆಲ್ಲಾ ತಮಗೆ ತಾವೇ ಬರುವುವು. ಕೆಲಸಕ್ಕೆ ಕೈಹಾಕಿ. ಅಗಾಧಶಕ್ತಿ ಬರುವುದು. ಅದನ್ನು ಸಹಿಸುವುದೇ ಕಷ್ಟವಾಗುವುದು. ಇತರ ರಿಗಾಗಿ ಮಾಡಿದ ಒಂದು ಚೂರು ಕೆಲಸವೂ ಆಂತರಿಕ ಶಕ್ತಿಯನ್ನು ಜಾಗ್ರತ ಗೊಳಿಸುವುದು. ಇತರರಿಗೆ ಮಾಡಿದ ಅತ್ಯಲ್ಪ ಹಿತಚಿಂತನೆ ಕೂಡ ಸಿಂಹಸದೃಶ ಶಕ್ತಿಯನ್ನು ನಿಮ್ಮೆದೆಗೆ ತುಂಬುವುದು. ನಾನು ನಿಮ್ಮನ್ನೆಲ್ಲ ಎಷ್ಟೋ ಪ್ರೀತಿ ಸುತ್ತೇನೆ. ಆದರೆ ಮತ್ತೊಬ್ಬರ ಸೇವೆಯಲ್ಲಿ ನಿಮಗೆ ಮರಣ ಪ್ರಾಪ್ತವಾಗಲಿ ಎಂದು ಬಯಸುತ್ತೇನೆ–ನೀವು ಹಾಗೆ ಮಾಡಿದರೆ ನನಗೆ ಸಂತೋಷ!

ನೀವು ದರಿದ್ರರು ಎಂಬ ಭಾವನೆಯನ್ನು ಗಣನೆಗೆ ತರಬೇಡಿ. ನೀವು ಯಾವ ರೀತಿಯಲ್ಲಿ ಬಡವರು? ನಿಮಗೆ ಕುದುರೆಗಾಡಿ, ಅಪ್ಪಣೆ ಮಾಡಲು ಆಳುಕಾಳು, ಸಿಬ್ಬಂದಿ ಇಲ್ಲವೆಂದು ವೃಥೆಪಡುವಿರಾ? ಇಲ್ಲದೇ ಇದ್ದರೇನಂತೆ? ನೀವು ಹಗಲಿರಳೂ ಹೃತ್ಪೂರ್ವಕ ಮತ್ತೊಬ್ಬರಿಗೆ ದುಡಿದರೆ ಪ್ರಪಂಚಕ್ಕೆ ನಷ್ಟವೇನೂ ಇಲ್ಲ ಎಂಬುದು ನಿಮಗೆ ತಿಳಿಯದು.

ಎಲ್ಲಿಯವರೆಗೂ “ಮುಟ್ಟಬೇಡ” ಎನ್ನುವುದು ನಿಮ್ಮ ಮತವಾಗಿದೆಯೋ, “ಅಡಿಗೆಯ ಪಾತ್ರೆ” ನಿಮ್ಮ ದೇವರಾಗಿರುವುದೋ, ಅಲ್ಲಿಯವರೆಗೆ ನೀವು ಆಧ್ಯಾತ್ಮಿಕ ಜೀವನದಲ್ಲಿ ಮುಂದುವರಿಯಲಾರಿರಿ. ಪ್ರತಿಯೊಂದು ಕರ್ಮ ಫಲವೂ ಒಳ್ಳೆಯದು ಕೆಟ್ಟದ್ದರೊಂದಿಗೆ ಮಿಶ್ರವಾಗಿರುವುದು. ಸೋಂಕಿಲ್ಲದ ಒಳ್ಳೆಯ ಕೆಲಸವೇ ಇಲ್ಲ. ಬೆಂಕಿಯ ಸುತ್ತಲಿರುವ ಹೊಗೆಯಂತೆ ಸ್ವಲ್ಪ ದೋಷ ಯಾವಾಗಲೂ ಕರ್ಮವನ್ನು ಆವರಿಸಿರುವುದು. ಯಾವ ಕೆಲಸದಿಂದ ಹೆಚ್ಚು ಒಳ್ಳೆಯದಾಗಿ ಅತಿ ಕಡಿಮೆ ಕೆಟ್ಟದ್ದು ಆಗುವುದೋ ಆ ಕರ್ಮದಲ್ಲಿ ನಿರತರಾಗಬೇಕು.

ಏನು ಒಂದು ಇರುವೆಯಾದರೂ ನಿಮ್ಮ ಸಹಾಯವಿಲ್ಲದೆ ಸಾಯುವು ದೆಂದು ಭಾವಿಸುವಿರಾ? ಇದೊಂದು ಘೋರ ದೇವನಿಂದೆ! ಪ್ರಪಂಚಕ್ಕೆ ನಿಮ್ಮ ಅವಶ್ಯವೇ ಇಲ್ಲ. ಆತನಿಗಾಗಿ ದುಡಿಯುವ ಸುಯೋಗ ಒದಗಿದುದರಿಂದ ನೀವೇ ಧನ್ಯರು. ನಿಮ್ಮ ಮನಸ್ಸಿನಿಂದ ಉಪಕಾರವೆಂಬುದನ್ನೇ ತೊಡೆದುಹಾಕಿ. ನೀವು ಮತ್ತೊಬ್ಬರಿಗೆ ಉಪಕಾರಮಾಡಲಾರಿರಿ. ಇದೊಂದು ಘೋರ ದೇವ ನಿಂದೆ. ನೀವು ಪೂಜೆ ಮಾಡಬಲ್ಲಿರಿ. ವಿಶ್ವದೆದುರು ಪೂಜ್ಯಭಾವದಿಂದ ನಿಲ್ಲಿ. ಆಗ ಮಾತ್ರ ಸಂಪೂರ್ಣ ಅನಾಸಕ್ತಿ ಬರುವುದು.

ನಾನು ಹುಡುಗನಾಗಿದ್ದಾಗ, ಮತಭ್ರಾಂತಿ ಕಾರ್ಯಕ್ಕೆ ಬಹಳ ಸಹಾಯ ಮಾಡುವುದೆಂದು ಭಾವಿಸಿದ್ದೆ. ಆದರೆ ಈಗ ನನಗೆ ವಯಸ್ಸಾದಂತೆ ಅದು ಸತ್ಯವಲ್ಲವೆನ್ನುವುದು ಗೊತ್ತಾಯಿತು.

ಯಾವ ಬಗೆಯ ಕರ್ತವ್ಯವನ್ನಾಗಲಿ ಅಸಡ್ಡೆಯಿಂದ ಕಾಣಕೂಡದು. ಒಬ್ಬ ನನ್ನು ಅವನು ಮಾಡುವ ಕೆಲಸದ ಮೂಲಕ ಪರೀಕ್ಷಿಸುವುದಕ್ಕೆ ಆಗುವುದಿಲ್ಲ. ಯಾವ ರೀತಿ ಆ ಕೆಲಸವನ್ನು ಮಾಡುತ್ತಾನೆ ಅದನ್ನು ಗಮನಿಸಬೇಕು. ತನ್ನ ಕಸುಬಿನ ದೃಷ್ಟಿಯಿಂದ ಬಹಳ ಕಡಿಮೆ ಕಾಲದಲ್ಲಿ ಬಾಳಿಕೆ ಬರುವ ಅಂದವಾದ ಒಂದು ಜೊತೆ ಜೋಡನ್ನು ತಯಾರುಮಾಡುವ ಮೋಚಿ, ವರುಷವೆಲ್ಲ ಕೆಲಸಕ್ಕೆ ಬಾರದುದನ್ನೇ ಮಾತನಾಡುವ ಪ್ರೊಫೆಸರಿಗಿಂತ ಮೇಲು. ಪ್ರತಿ ಯೊಂದು ಕರ್ತವ್ಯವೂ ಪವಿತ್ರ. ಕರ್ತವ್ಯಪರಾಯಣತೆಯೆ ಭಗವಂತನಿಗೆ ಮಾಡುವ ಶ್ರೇಷ್ಠ ಆರಾಧನೆ.

ಮರಣ ಪರ್ಯಂತವೂ ಕಾರ್ಯತತ್ಪರರಾಗಿ. ನಾನು ನಿಮ್ಮೊಂದಿಗೆ ಇರು ವೆನು. ನಾನು ಕಾಲವಾದ ಮೇಲೆ ನನ್ನಾತ್ಮ ನಿಮ್ಮೊಂದಿಗೆ ಸಹಕರಿಸುವುದು. ಐಶ್ವರ್ಯ ಕೀರ್ತಿ ಭೋಗವೆಲ್ಲ ಕೆಲವು ದಿನ ಬಾಳುವೆಯದು. ಸತ್ಯವನ್ನು ಬೋಧಿಸುತ್ತ ಕರ್ತವ್ಯ ಕ್ಷೇತ್ರದಲ್ಲಿ ಪ್ರಾಣಬಿಡುವುದು ಪ್ರಾಪಂಚಿಕ ಕೀಟದಂತೆ ಸಾಯುವುದಕ್ಕಿಂತ ಉತ್ತಮ. ಎಷ್ಟೋ ಉತ್ತಮ.

ಅಸೂಯೆ ಮತ್ತು ಅಹಂಕಾರವನ್ನು ತೊರೆಯಿರಿ. ಇತರರಿಗಾಗಿ ಸಾಮೂ ಹಿಕವಾಗಿ ಕೆಲಸಮಾಡುವುದನ್ನು ಕಲಿಯಿರಿ. ನಮ್ಮ ದೇಶಕ್ಕೆ ಅತ್ಯಾವಶ್ಯಕವಾಗಿ ಬೇಕಾಗಿರುವುದು ಇದು.

ಮಗು! ಯಾವ ದೇಶವಾಗಲಿ, ವ್ಯಕ್ತಿಯಾಗಲಿ ಮತ್ತೊಬ್ಬನನ್ನು ದ್ವೇಷಿಸಿ ಬಾಳಲಾರದು. ಮ್ಲೇಚ್ಛರೆಂಬ ಪದವನ್ನು ಕಂಡುಹಿಡಿದು ಹೊರಗಿನವರೊಂ ದಿಗೆ ಸಂಪರ್ಕವನ್ನು ನಿಲ್ಲಿಸಿದ ದಿನದಿಂದಲೇ ಭಾರತ ಭಾಗ್ಯ ಅವನತಿಗೆ ಇಳಿಯಿತು.

ಯುವಕರೆ! ದೀನರಿಗೆ, ದಲಿತರಿಗೆ, ಅಜ್ಞಾನಿಗಳಿಗೆ ಅನುಕಂಪ, ಅವರ ಮೇಲ್ಮೆಗಾಗಿ ದುಡಿತ, ಇದೇ ನಾನು ನಿಮಗೆ ಬಿಡುವ ಆಸ್ತಿ. ಈ ಕ್ಷಣವೇ ಪಾರ್ಥಸಾರಥಿ ದೇವಾಲಯಕ್ಕೆ ಹೋಗಿ, ಯಾರು ಗೋಕುಲದ ದೀನಗೋಪಾಲ ರಿಗೆ ಸ್ನೇಹಿತನಾಗಿದ್ದನೋ, ರಾಮಾವತಾರದಲ್ಲಿ ಚಂಡಾಲ ಗುಹನನ್ನು ಆಲಂಗಿಸಿಕೊಳ್ಳುವುದಕ್ಕೆ ಸ್ವಲ್ಪವೂ ಅನುಮಾನಪಡಲಿಲ್ಲವೋ, ಯಾರು ಬುದ್ಧಾವತಾರದಲ್ಲಿ ಶ್ರೀಮಂತರನ್ನು ನಿರಾಕರಿಸಿ ವೇಶ್ಯಾಂಗನೆಯ ಆತಿಥ್ಯ ವನ್ನು ಸ್ವೀಕರಿಸಿದನೋ ಅವನೆದುರಿಗೆ ಬಾಗಿ, ಮಹಾತ್ಯಾಗವನ್ನು ಮಾಡಿ. ಯಾರಿಗೋಸುಗ ಭಗವಂತ ಕಾಲಕಾಲಕ್ಕೆ ಬರುವನೋ, ಯಾವ ದೀನರು, ದರಿದ್ರರು ದಲಿತರನ್ನು ಎಲ್ಲರಿಗಿಂತ ಹೆಚ್ಚು ಪ್ರೀತಿಸುವನೋ, ಅವರಿಗಾಗಿ ನಿಮ್ಮ ಇಡೀ ಜೀವನವನ್ನು ಮುಡುಪಾಗಿಡಿ.

ನಾವು ಪ್ರಪಂಚಕ್ಕೆ ಒಳ್ಳೆಯದನ್ನು ಮಾಡಬಲ್ಲವೆ? ನಿರಪೇಕ್ಷೆ ದೃಷ್ಟಿಯಿಂದ ಇಲ್ಲ. ಸಾಪೇಕ್ಷ ದೃಷ್ಟಿಯಿಂದ ಹೌದು.

ನನ್ನ ಸಹೋದರರೆ, ಎಲ್ಲಾ ಕಷ್ಟಪಟ್ಟು ಕೆಲಸಮಾಡುವ. ನಿದ್ರಿಸುವ ಸಮಯವಲ್ಲ ಇದು. ನಮ್ಮ ಇಂದಿನ ಶ್ರಮದ ಮೇಲೆ ಭವಿಷ್ಯ ಭಾರತ ನಿಂತಿರುವುದು. ನಮಗಾಗಿ ಆಕೆ ಕಾಯುತ್ತಿರುವಳು. ಆಕೆ ಕೇವಲ ನಿದ್ರಿಸು ತ್ತಿರುವಳು. ಜಾಗ್ರತರಾಗಿ! ಏಳಿ. ನಮ್ಮ ಮಾತೃಭೂಮಿ ತನ್ನ ಸನಾತನ ಸಿಂಹಾಸನದ ಮೇಲೆ ಎಂದಿಗಿಂತಲೂ ಹೆಚ್ಚು ಕಾಂತಿಯುತಳಾಗಿ ಜೀವಕಳೆ ಯಿಂದ ಮಂಡಿಸಿರುವುದನ್ನು ನೋಡಿ.

ಯಾರು ಶಿವನನ್ನು ಸೇವಿಸಬೇಕೆಂದು ಬಯಸುವರೋ ಅವರು ಶಿವನ ಮಕ್ಕಳನ್ನು ಸೇವಿಸಬೇಕು. ಮೊದಲು ಪ್ರಪಂಚದ ಎಲ್ಲಾ ಜೀವಿಗಳನ್ನೂ ಸೇವಿಸಬೇಕು. ಯಾರು ಭಗವಂತನ ಸೇವಕರನ್ನು ಸೇವಿಸುವರೋ ಅವರೇ ಆತನ ಅತಿ ಶ್ರೇಷ್ಠ ಸೇವಕರೆಂದು ಶಾಸ್ತ್ರ ಸಾರುವುದು. ನಿಃಸ್ವಾರ್ಥತೆಯೇ ಧರ್ಮದ ಪರೀಕ್ಷೆ. ಯಾರಲ್ಲಿ ಈ ನಿಃಸ್ವಾರ್ಥತೆ ಹೆಚ್ಚು ಇದೆಯೋ ಅವನು ಹೆಚ್ಚು ಧರ್ಮಿಷ್ಠ, ಭಗವಂತನ ಸಮೀಪದಲ್ಲಿರುವನು. ಅವನು ಸ್ವಾರ್ಥಿ ಯಾಗಿದ್ದರೆ, ಅವನು ಎಲ್ಲಾ ದೇವಸ್ಥಾನಗಳಿಗೂ ಹೋಗಿದ್ದರೂ ಎಲ್ಲಾ ಯಾತ್ರೆಗಳನ್ನು ಮಾಡಿದ್ದರೂ, ದೇಹದ ಮೇಲೆಲ್ಲ ಚಿರತೆಯಂತೆ ವಿಭೂತಿ ಯನ್ನು ಬಳಿದುಕೊಂಡಿದ್ದರೂ ಅವನು ಶಿವನಿಂದ ಇನ್ನೂ ಬಹಳ ದೂರದ ಲ್ಲಿರುವನು.

ಎಲ್ಲಾ ಪೂಜೆಯ ಸಾರವಿದು–ಪರಿಶುದ್ಧವಾಗಿರುವುದು. ಇತರರಿಗೆ ಶುಭ ವನ್ನು ಮಾಡುವುದು. ಯಾರು ಶಿವನನ್ನು ದರಿದ್ರರಲ್ಲಿ, ದುರ್ಬಲರಲ್ಲಿ, ರೋಗಿಗಳಲ್ಲಿ ನೋಡುವರೋ ಅವರು ನಿಜವಾಗಿಯೂ ಶಿವನನ್ನು ಪೂಜಿಸು ವರು. ಅವನು ಕೇವಲ ವಿಗ್ರಹದಲ್ಲಿ ಮಾತ್ರ ಶಿವನನ್ನು ಕಂಡರೆ ಪೂಜೆಗೆ ಇನ್ನೂ ಸನ್ನಾಹವಾಗುತ್ತಿದೆ, ಅಷ್ಟೆ. ಯಾರು ಒಬ್ಬ ದೀನನನ್ನು ಕಂಡು, ಅವನ ಜಾತಿ, ಕುಲ, ಗೋತ್ರ ಯಾವುದನ್ನೂ ವಿಚಾರಿಸದೆ ಅವನನ್ನು ಸೇವಿಸಿ ಸಹಾಯ ಮಾಡಿರುವನೋ, ಅಂತಹವರ ಮೇಲೆ ಶಿವನ ಒಲುಮೆ ಹೆಚ್ಚು. ತನ್ನನ್ನು ಕೇವಲ ದೇವಸ್ಥಾನದಲ್ಲಿ ನೋಡಿರುವವನ ಮೇಲೆ ಅಲ್ಲ.

ಯಾರ ನಂಬಿಕೆಯನ್ನೂ ಕೆಡಿಸುವುದಕ್ಕೆ ಯತ್ನಿಸಬೇಡಿ. ಆತನಿಗೆ ಈಗ ಇರುವುದಕ್ಕಿಂತ ಉತ್ತಮವಾಗಿರುವುದನ್ನು ಕೊಡುವುದಕ್ಕೆ ಸಾಧ್ಯವಾದರೆ, ಅವ ನಿರುವ ಸ್ಥಿತಿಗಿಂತ ಮೇಲೆ ಕರೆದೊಯ್ಯಲು ಸಾಧ್ಯವಾದರೆ ಹಾಗೆ ಮಾಡಿ. ಆದರೆ ಆಗಲೆ ಅವನಿರುವ ಸ್ಥಿತಿಗೆ ಭಂಗ ತರಬೇಡಿ.

ನಾವು ಪ್ರಾರ್ಥಿಸುವಾಗ ದೇವರನ್ನು ತಂದೆಯೆಂದು ಒಪ್ಪಿಕೊಂಡು, ನಮ್ಮ ನಿತ್ಯಜೀವನದಲ್ಲಿ ಮಾನವರನ್ನು ನಮ್ಮ ಸಹೋದರಂತೆ ಕಾಣದೆ ಇದ್ದರೆ ಪ್ರಯೋಜನವೇನು?

ಪ್ರಕೃತಿ ಗುಲಾಮರ ಹಣೆಯ ಮೇಲಿಡುವ ಅಸೂಯೆ ಎಂಬ ಹೀನ ಚಿಹ್ನೆಯನ್ನು ಮೊದಲು ನಿವಾರಿಸುವ. ಯಾರೊಂದಿಗೂ ಅಸೂಯೆಪಡಬೇಡಿ. ಪ್ರತಿಯೊಬ್ಬ ಒಳ್ಳೆಯ ಕೆಲಸವನ್ನು ಮಾಡುವವನಿಗೂ ಸಹಾಯಮಾಡಿ. ತ್ರಿಭುವನ ವಿಹಾರಿಗಳಿಗೆಲ್ಲ ಒಳ್ಳೆಯ ಆಲೋಚನೆಯನ್ನು ಕಳುಹಿಸಿ. ಸಾಗರದ ಕಡೆ ನೋಡಿ, ಅಲೆಯ ಕಡೆ ನೋಡಬೇಡಿ. ಇರುವೆಗೂ ದೇವತೆಗೂ ಯಾವ ಭಿನ್ನತೆಯನ್ನೂ ನೋಡಬೇಡಿ. ಪ್ರತಿಯೊಂದು ಕೀಟವೂ ಏಸುಕ್ರಿಸ್ತನ ಸಹೋದರ. ಹೀಗಿರುವಾಗ ಒಬ್ಬ ಮೇಲು, ಒಬ್ಬ ಕೀಳು ಹೇಗೆ? ಪ್ರತಿಯೊಂದು ತನ್ನ ಸ್ಥಾನದಲ್ಲಿ ದೊಡ್ಡದೇ.

ಆ ಮಹಾವೀರನ ಶೀಲ ನಿಮ್ಮ ಆದರ್ಶವಾಗಲಿ. ರಾಮಚಂದ್ರನ ಆಜ್ಞೆ ಯಂತೆ ಹೇಗೆ ಅವನು ಸಾಗರವನ್ನು ಲಂಘಿಸಿದನು! ಅವನಿಗೆ ಜೀವನ ಮರಣಗಳ ಯೋಚನೆಯೇ ಇರಲಿಲ್ಲ. ಅವನು ಜಿತೇಂದ್ರಿಯನು, ಅತಿ ಜಾಣ. ಗುರುಸೇವೆ ಎಂಬ ಮಹಾ ಆದರ್ಶದಲ್ಲಿ ನಿಮ್ಮ ಜೀವನವನ್ನು ರೂಪುಗೊಳಿಸಿ. ಇದರಿಂದ ಉಳಿದುದೆಲ್ಲ ಆದರ್ಶಗಳೂ ಕ್ರಮೇಣ ವಿಕಾಸಕ್ಕೆ ಬರುವುವು. ಮರುಮಾತಿಲ್ಲದೆ ಗುರುವಿಗೆ ವಿಧೇಯತೆ ಮತ್ತು ಜಿತೇಂದ್ರಿಯತೆ–ಇವೇ ಜಯದ ರಹಸ್ಯ. ಹನುಮಂತ ಒಂದು ಕಡೆ ಸೇವಾತತ್ಪರತೆಯನ್ನು ಪ್ರಕಟಿ ಸಿದರೆ ಮತ್ತೊಂದು ಕಡೆ ಪ್ರಪಂಚವನ್ನು ಬೆರಗುಗೊಳಿಸುವ ಮಹಾ ಪೌರುಷ ವನ್ನೂ ಬೀರುವನು.


\section{ಮನೋನಿಗ್ರಹ}

ರಹಸ್ಯದಲ್ಲಿ ಮತ್ತೊಬ್ಬನನ್ನು ದೂರುವುದು ಪಾಪ. ಇದನ್ನು ನೀವು ಸಂಪೂರ್ಣ ತ್ಯಜಿಸಬೇಕು. ಮನಸ್ಸಿಗೆ ಎಷ್ಟೋ ವಿಷಯಗಳು ಹೊಳೆಯ ಬಹುದು. ಆದರೆ ಅದನ್ನೆಲ್ಲ ವ್ಯಕ್ತಪಡಿಸುವುದುಕ್ಕೆ ಪ್ರಯತ್ನಿಸಿದರೆ ಹುತ್ತ ಒಂದು ಬೆಟ್ಟವಾಗುವುದು. ನೀವು ಕ್ಷಮಿಸಿ ಮರೆತುಬಿಟ್ಟರೆ ಎಲ್ಲಾ ಅಲ್ಲಿಗೇ ಕೊನೆಗಾಣುವುದು.

ಸುಮ್ಮನೆ ಯಾರಾದರೂ ವಾದಕ್ಕೆ ನಿಮ್ಮ ಹತ್ತಿರ ಬಂದರೆ ಗಂಭೀರವಾಗಿ ಹಿಂದೆ ಸರಿಯಿರಿ. ಎಲ್ಲಾ ಪಂಗಡದ ದೀನರಿಗೂ ನಿಮ್ಮ ಸಹಾನುಭೂತಿಯನ್ನು ತೋರಬೇಕು. ಎಂದು ಈ ಎರಡು ಗುಣ ನಿಮ್ಮಲ್ಲಿ ವ್ಯಕ್ತವಾಗುವುದೋ, ಆಗ ನೀವು ಹೆಚ್ಚು ಶಕ್ತಿಯಿಂದ ಕೆಲಸಮಾಡಬಹುದು.

ಎಳೆ ಹಸುಳೆಯ ಸರಳತೆಯೊಂದಿಗೆ ಗಂಭೀರ ಭಾವವಿರಲಿ. ಎಲ್ಲರೊಂದಿಗೆ ಸೌಹಾರ್ದದಿಂದ ಬಾಳಿ. ಎಲ್ಲಾ ವಿಧದ ಅಹಂಕಾರಭಾವನೆಯನ್ನೂ ತೊರೆ ಯಿರಿ. ಯಾವ ವಿಧದ ಸಂಕುಚಿತ ಧಾರ್ಮಿಕತೆಗೆ ಎಡಗೊಡಬೇಡಿ. ವೃಥಾ ಜಗಳವಾಡುವುದು ಮಹಾಪಾಪ.

ನಿರತ್ಸಾಹ ಧರ್ಮವಲ್ಲ. ಅದು ಮತ್ತೆ ಏನಾದರೂ ಆಗಿರಬಹುದು. ಯಾವಾ ಗಲೂ ಮಂದಹಾಸದಿಂದ ಸಂತೋಷವಾಗಿದ್ದರೆ ಅದು ಎಲ್ಲಾ ಪ್ರಾರ್ಥನೆ ಗಳಿಗಿಂತ ನಮ್ಮನ್ನು ಹೆಚ್ಚು ಭಗವಂತನ ಸಮೀಪಕ್ಕೆ ಕರೆದೊಯ್ಯುವುದು.

ಜೀವಾತ್ಮ ವಾಸಮಾಡುವ ಈ ನಮ್ಮ ದೇಹವೇ ನಮ್ಮ ಕರ್ಮಕ್ಕೆ ಮುಖ್ಯ ಸಾಧನ. ಯಾರು ಇದನ್ನು ಪಾಪಕೂಪದಂತೆ ಮಾಡುವರೋ ಅವರು ತಪ್ಪಿ ತಸ್ಥರು. ಯಾರು ಇದನ್ನು ಗಣನೆಗೆ ತರುವುದಿಲ್ಲವೋ ಅವರು ಕೂಡ ನಿಂದಾರ್ಹರು.

ಯಾವುದು ಪರಿಶುದ್ಧವಲ್ಲವೋ ಅದನ್ನು ನಿಮ್ಮ ಅಂಗುಷ್ಠದಿಂದಲೂ ಮುಟ್ಟಬೇಡಿ. ಅದನ್ನು ಅರಸಬೇಡಿ.

ಯಾರಾದರೂ ಅವನ ಸಹೋದರರನ್ನು ನಿಮ್ಮ ಹತ್ತಿರ ದೂರಲು ಬಂದರೆ ಅವನು ಹೇಳುವುದಾವುದನ್ನೂ ಕೇಳಬೇಡಿ. ಕೇಳುವುದೂ ಒಂದು ಮಹಾ ಪಾಪ. ಇಲ್ಲಿಯೇ ಮುಂದಿನ ದುರಂತದ ಅಂಕುರವಿರುವುದು. ಎಲ್ಲರ ನ್ಯೂನತೆಯನ್ನೂ ತಾಳ್ಮೆಯಿಂದ ಸಹಿಸಿ, ಕೋಟ್ಯಂತರ ಅಪರಾಧಗಳನ್ನು ಕ್ಷಮಿಸಿ.

ನನಗೆ ಆಧ್ಯಾತ್ಮ ವಿಷಯದಲ್ಲಿ ಸಂತೋಷ ಸಿಕ್ಕದೇ ಇದ್ದರೆ ಇಂದ್ರಿಯ ಜೀವನದಲ್ಲಿ ತೃಪ್ತಿಯನ್ನು ಪಡೆದುಕೊಳ್ಳಲೇ? ನನಗೆ ಅಮೃತ ಸಿಕ್ಕದೇ ಇದ್ದರೆ ಬಚ್ಚಲು ನೀರನ್ನು ಕುಡಿಯಲೇ?

ಸುಖ, ದುಃಖದ ಕಿರೀಟವನ್ನು ಧರಿಸಿ ಮನುಷ್ಯನಿಗೆ ಗೋಚರಿಸುವುದು ಯಾರು ಸುಖವನ್ನು ಸ್ವಾಗತಿಸುವರೋ ಅವರು ದುಃಖವನ್ನೂ ಸ್ವಾಗತಿಸಬೇಕು.

ಒಬ್ಬ ಸಾಮಾಜಿಕ ಮತ್ತು ರಾಜಕೀಯ ಸ್ವಾತಂತ್ರ್ಯವನ್ನು ಪಡೆಯಬಹುದು. ಆದರೆ ಅವನು ತನ್ನ ಆಸೆ ಆಕಾಂಕ್ಷೆಗಳಿಗೆ ಆಳಾಗಿದ್ದರೆ ನಿಜವಾದ ಸ್ವಾತಂತ್ರ್ಯದ ಸವಿಯನ್ನು ಅನುಭವಿಸಲಾರ.

ಎಂತಹ ದಡ್ಡನಾದರೂ ತನ್ನ ಇಚ್ಛೆಗೆ ಅನುಸಾರವಾದ ಕೆಲಸವಾದರೆ ಅದನ್ನು ನೆರವೇರಿಸಬಲ್ಲ. ಆದರೆ ಯಾವನು ನಿಜವಾಗಿ ಬುದ್ಧಿವಂತನೋ ಅವನು ಪ್ರತಿಯೊಂದು ಕೆಲಸವನ್ನೂ ತನ್ನ ಮನಸ್ಸಿಗೆ ಹಿತವಾದ ಕೆಲಸವನ್ನಾಗಿ ಮಾಡಿಕೊಳ್ಳಬಲ್ಲ. ಯಾವ ಕೆಲಸವೂ ಕೀಳಲ್ಲ.

ಪ್ರಕೃತಿಯೊಂದಿಗೆ ಹೋರಾಡುವುದು; ಅದರೊಂದಿಗೆ ರಾಜಿ ಮಾಡಿಕೊಳ್ಳು ವುದಲ್ಲ ಮನುಷ್ಯನ ಲಕ್ಷಣ.

ಮತ್ತೊಬ್ಬರ ತಪ್ಪನ್ನು ಕುರಿತು ಮಾತನಾಡಬೇಡ. ಅವು ಎಷ್ಟೇ ದೊಡ್ಡದಾ ಗಿದ್ದರೂ ಚಿಂತೆಯಿಲ್ಲ. ಇದರಿಂದ ಏನೂ ಪ್ರಯೋಜನವಿಲ್ಲ. ಮತ್ತೊಬ್ಬನ ತಪ್ಪನ್ನು ಅವನಿಗೆ ತೋರಿದರೆ ಅವನಿಗೆ ಸಹಾಯವೇನೊ ಆಗುವುದಿಲ್ಲ. ಅವನಿಗೆ ವೃಥಾ ವ್ಯಥೆಯಾಗುವುದು. ನಿನಗೂ ತೊಂದರೆಯಾಗುವುದು.

ತಂಬಾಕಿನ ಚುಂಗಾಣೆಯನ್ನು ಯಾರು ವಿಚಕ್ಷಣೆಯಿಂದ ತಯಾರಿಸ ಬಲ್ಲರೋ ಅವರು ಚೆನ್ನಾಗಿ ಧ್ಯಾನವನ್ನು ಮಾಡಬಲ್ಲರು.

ಯಾರು ತಮ್ಮ ಇಂದ್ರಿಯನಿಗ್ರಹವನ್ನು ಚೆನ್ನಾಗಿ ಮಾಡಿರುವರೋ ಅವರು ಯಾರ ಹೊರಗಿನ ಪ್ರಭಾವಕ್ಕೂ ಬೀಳುವುದಿಲ್ಲ. ಆತ್ಮನಿಗೆ ಇನ್ನು ಬಂಧನ ವಿಲ್ಲ. ಅವನ ಮನಸ್ಸು ಸ್ವತಂತ್ರವಾಗಿರುವುದು. ಅಂತಹವನೇ ಪ್ರಪಂಚದಲ್ಲಿ ಚೆನ್ನಾಗಿ ಬಾಳಲು ಅರ್ಹನು.

ನಾವು ಎಷ್ಟು ಶಾಂತಚಿತ್ತರಾಗಿದ್ದು ಉದ್ವೇಗವಶರಾಗುವುದಿಲ್ಲವೋ ಅಷ್ಟೂ ಹೆಚ್ಚು ಪ್ರೀತಿಸಬಲ್ಲೆವು. ನಮ್ಮ ಕರ್ಮವೂ ಅಷ್ಟೂ ಉತ್ತಮ ವಾಗುವುದು.

ನೀವು ಸ್ವಾಮಿಯಂತೆ ಕೆಲಸಮಾಡಬೇಕು. ಆಳಂತೆ ಅಲ್ಲ. ಎಡೆಬಿಡದೆ ಕೆಲಸಮಾಡಿ. ಆದರೆ ಆಳಿನ ಕೆಲಸ ಮಾಡಬೇಡಿ. ನಿಸ್ಸಂಗನಾಗಿರಿ; ಕೆಲಸ ವಾಗುತ್ತಿರಲಿ. ಮೆದಳು ಆಲೋಚಿಸುತ್ತಿರಲಿ. ಎಡೆಬಿಡದೆ ಕೆಲಸ ಮಾಡಿ. ಆದರೆ ಒಂದು ಆಲೋಚನೆ ಅಲೆಯೂ ನಿಮ್ಮ ಮನಸ್ಸನ್ನು ವಶಪಡಿಸಿಕೊಳ್ಳದಿರಲಿ. ಈ ಪ್ರಪಂಚಕ್ಕೆ ನೀವು ಅಪರಿಚಿತರು, ನೀವೊಬ್ಬ ದಾರಿಗ ಎಂದು ತಿಳಿದು ಕೆಲಸ ಮಾಡಿ. ಅನವರತವೂ ಕೆಲಸ ಮಾಡಿ. ಆದರೆ ಬದ್ಧರಾಗಬೇಡಿ. ಬಂಧನ ಭಯಂಕರ.

ಸೋಮಾರಿತನವನ್ನು ಎಲ್ಲಾ ವಿಧದಿಂದಲೂ ತ್ಯಜಿಸಬೇಕು. ಚಟುವಟಿಕೆ ಎಂದರೆ ಯಾವಾಗಲೂ ನಿಗ್ರಹ. ಎಲ್ಲಾ ದೈಹಿಕ ಮತ್ತು ಮಾನಸಿಕ ಪಾಪವನ್ನು ನಿಗ್ರಹಿಸಿ. ನಿಗ್ರಹ ಸಿದ್ಧಿಸಿದರೆ ಶಾಂತಿ ಪ್ರಾಪ್ತಿ.

ಯಾವ ವ್ಯಕ್ತಿಯಲ್ಲಿ ಆತ್ಮನಿಂದೆ ಮೊದಲಾಗಿರುವುದೋ ಅವನ ಅವ ನತಿಯು ಪ್ರಾರಂಭವಾದಂತೆ. ಅದರಂತೆ ದೇಶವು ಕೂಡ. ನಮ್ಮ ಮೊದಲ ಕರ್ತವ್ಯ ಆತ್ಮನಿಂದೆಯನ್ನು ಮಾಡಿಕೊಳ್ಳದೆ ಇರುವುದು. ನಾವು ಮುಂದುವರಿಯಬೇಕಾದರೆ ಮೊದಲು ನಮ್ಮಲ್ಲಿ ನಮಗೆ ಶ್ರದ್ಧೆ ಇರಬೇಕು. ನಂತರ ದೇವರಲ್ಲಿ.

ಭರತಖಂಡದಲ್ಲಿ ಈಗ ಹೆಚ್ಚು ಕೇಳಿಸುತ್ತಿರುವ ಪಾಪವೆಂದರೆ ದಾಸ್ಯ. ಪ್ರತಿಯೊಬ್ಬನೂ ಅಪ್ಪಣೆ ಕೊಡಲು ತವಕ ಪಡುವನು. ಯಾರೂ ಅದನ್ನು ಪಾಲಿಸಲು ಇಚ್ಛಿಸುವುದಿಲ್ಲ. ಇದಕ್ಕೆ ಕಾರಣ, ಹಿಂದಿನ ಕಾಲದ ಅತಿ ಮುಖ್ಯ ವಾದ ಬ್ರಹ್ಮಚರ್ಯಶೀಲವಿಲ್ಲದೆ ಇರುವುದು. ಮೊದಲು ವಿಧೇಯತೆ ಕಲಿ. ನಂತರ ಆಜ್ಞೆ ಮಾಡುವುದು ತಾನೆ ಬರುವುದು. ಯಾವಾಗಲೂ ಮೊದಲು ಭೃತ್ಯರಾಗುವುದನ್ನು ಕಲಿ. ನಂತರ ನೀನು ಸ್ವಾಮಿಯಾಗಲು ಯತ್ನಿಸು.

ಆತ್ಮನ ಶಕ್ತಿ ಮತ್ತು ಪವಿತ್ರತೆಯನ್ನು ಸಂಕುಚಿತಮಾಡುವ ಆಲೋಚನೆ ಮತ್ತು ಕಾರ್ಯವೆಲ್ಲ ಪಾಪ. ಆತ್ಮನಲ್ಲಿ ಸುಪ್ತವಾಗಿರುವ ಚೇತನ ವಿಕಸಿತ ವಾಗುವಂತೆ ಮಾಡುವ ಆಲೋಚನೆ ಮತ್ತು ಕಾರ್ಯವೆಲ್ಲ ಪುಣ್ಯ.

ನಿಗ್ರಹಿಸದೆ ಇರುವ ಮನಸ್ಸು ಎಂದೆಂದಿಗೂ ನಮ್ಮನ್ನು ಅಧೋಗತಿಗೆ ಒಯ್ಯುವುದು, ನಮ್ಮನ್ನು ಕೊಲ್ಲುವುದು. ನಿಗ್ರಹಿಸಿದ ಮನಸ್ಸು ನಮ್ಮನ್ನು ರಕ್ಷಿಸುವುದು, ಮುಕ್ತರನ್ನಾಗಿ ಮಾಡುವುದು.

ಪ್ರಕೃತಿಯೊಂದಿಗೆ ಹೊಂದಿಕೆ ಎಂದರೆ ತಾಟಸ್ಥ್ಯ, ಮೃತ್ಯು. ಮನುಷ್ಯ ಈ ಮನೆಯನ್ನು ಹೇಗೆ ಕಟ್ಟಿದ? ಪ್ರಕೃತಿಯೊಂದಿಗೆ ಹೊಂದಿಕೊಂಡೆ? ಇಲ್ಲ. ಪ್ರಕೃತಿಯೊಂದಿಗೆ ಹೋರಾಡಿ. ಪ್ರಕೃತಿಯೊಂದಿಗೆ ನಡೆಸಿದ ಎಡೆಬಿಡದ ಹೋರಾಟವೇ ಮಾನವ ಪುರೋಗಮನ. ಅದಕ್ಕೆ ಅಡಿಯಾಳಾಗುವುದಲ್ಲ.

ಇತರರಲ್ಲಿರುವ ದುರ್ಗುಣಗಳನ್ನು ಒಪ್ಪಿಕೊಳ್ಳಬೇಡ. ದುರ್ಗುಣವೇ ಅಜ್ಞಾನ, ದುರ್ಬಲತೆ.

ಮುಂದುವರಿಯುವುದನ್ನು ಯಾವುದು ತಡೆಯುವುದೋ ಅಥವಾ ಕೆಳಗೆ ಬೀಳಲು ಯಾವುದು ಸಹಾಯ ಮಾಡುವುದೋ ಅದೇ ಪಾಪ. ಯಾವುದು ಮೇಲೆ ಬರುವುದಕ್ಕೆ ಸಹಾಯ ಮಾಡುವುದೋ, ಸೌಹಾರ್ದತೆಯನ್ನು ಕೊಡು ವುದೋ ಅದೇ ಪುಣ್ಯ.

ಹಣದಿಂದ ಪ್ರಯೋಜನವಿಲ್ಲ, ಹೆಸರಿನಿಂದಲೂ ಇಲ್ಲ. ಕೀರ್ತಿಯಿಂದ ಪ್ರಯೋಜನವಿಲ್ಲ, ಪಾಂಡಿತ್ಯದಿಂದಲೂ ಇಲ್ಲ. ಪ್ರೇಮವೇ ಪರೋಪಕಾರಿ. ಚಾರಿತ್ರ್ಯಶುದ್ಧಿಯೇ ಅಭೇದ್ಯ ಕಡುಕಷ್ಟಗಳ ಪರಂಪರೆಯಲ್ಲಿಯೂ ದಾರಿ ಮಾಡಿಕೊಂಡು ಹೋಗುವುದು.

ಜೀವನದ ಕರ್ತವ್ಯಗಳು ಯಾತನಾಮಯವೇ ಸರಿ. ಅದರಿಂದ ಬರುವ ಸುಖ ಕ್ಷಣಿಕ; ನಿಷ್ಪ್ರಯೋಜನ. ಸೇರುವ ಗುರಿ ಛಾಯಾಮಯವಾಗಿ ಮೊಬ್ಬಾ ಗಿದೆ. ಆದರೂ ಧೀರನೆ, ನಿನ್ನ ಶಕ್ತಿ, ಸಾಮರ್ಥ್ಯವನ್ನೆಲ್ಲ ಪ್ರಯೋಗಿಸು. ಈ ಅಂಧಕಾರದಲ್ಲಿ ವಿರಾಮವಿಲ್ಲದೆ ಮುಂದುವರಿ.

ನಮ್ಮ ಅಹಂಕಾರದ ಭಾವನೆ ಇಲ್ಲದೆ ಇರುವಾಗ ನಮ್ಮ ಅತ್ಯುತ್ತಮ ಕಾರ್ಯ ಸಫಲವಾಗುವುದು, ನಮ್ಮ ಅತ್ಯುತ್ತಮ ಪ್ರಭಾವ ಪರಿಣಾಮಕಾರಿ ಯಾಗುವುದು. ಸಂಪೂರ್ಣ ಶರಣಾಗತನಾಗು. ಅನಾಸಕ್ತನಾಗು. ಆಗ ಮಾತ್ರ ನೀನು ಯಾವುದಾದರೂ ಒಳ್ಳೆಯ ಕೆಲಸವನ್ನು ಮಾಡಬಹುದು.

ಹೀನ ಆಲೋಚನೆ ಹೀನ ಕಾರ್ಯದಷ್ಟೇ ಕೆಟ್ಟದ್ದು. ನಿಗ್ರಹಿಸಿದ ಆಸೆಯೇ ಅತ್ಯುತ್ತಮ ಫಲದಾಯಕ. 

ನಿಮ್ಮ ಕೆಲಸ ಒಳ್ಳೆಯದೋ ಕೆಟ್ಟದೋ ಅದನ್ನು ಉಗುಳುವಂತೆ ಮಾಡಿ. ಅದನ್ನು ಕುರಿತು ಪುನಃ ಆಲೋಚಿಸಬೇಡಿ. ಮಾಡಿದ್ದು ಆಗಿಹೋಯಿತು. ಮೂಢನಂಬಿಕೆಯನ್ನು ಆಚೆಗೆ ಎಸೆಯಿರಿ, ಮೃತ್ಯು ಸಮ್ಮುಖದಲ್ಲಿಯೂ ಅಧೀರರಾಗಬೇಡಿ. ಪಶ್ಚಾತ್ತಾಪಪಡಬೇಡಿ. ನಿಮ್ಮ ಹಿಂದಿನ ಕೆಲಸಗಳನ್ನೇ ಕುರಿತು ಮನನಮಾಡಬೇಡಿ; ನೀವು ಮಾಡಿದ ಒಳ್ಳೆಯ ಕೆಲಸಗಳನ್ನು ಜ್ಞಾಪಿಸಿ ಕೊಳ್ಳಬೇಡಿ; ಮುಕ್ತರಾಗಿ.

ಪ್ರಪಂಚಕ್ಕೆ ಬೇಕಾಗಿರುವುದು ಚಾರಿತ್ರಶುದ್ಧಿ. ಯಾವ ಪ್ರೇಮ ನಿಃಸ್ವಾರ್ಥ ನಂದಾದೀವಿಗೆಯಂತೆ ಉರಿಯುತ್ತಿದೆಯೋ ಅದು ಜಗತ್ತಿಗೆ ಬೇಕಾಗಿದೆ. ಅಂತಹ ಪ್ರೇಮದಿಂದ ಪ್ರೇರಿತವಾದ ಪ್ರತಿಯೊಂದು ಪದವೂ ವಜ್ರಾಘಾತ ದಂತೆ ಪರಿಣಾಮಕಾರಿಯಾಗುವುದು. ಮಹಾಪುರುಷರೇ! ಏಳಿ, ಜಾಗ್ರತರಾಗಿ! ದುಃಖಾಗ್ನಿಯಲ್ಲಿ ಬೆಂದು ನೋಯುತ್ತಿದೆ ಪ್ರಪಂಚ. ನೀವು ನಿದ್ರಿಸಬಲ್ಲಿರಾ?

ನನಗೆ ವಯಸ್ಸಾದಂತೆಲ್ಲ ಸಣ್ಣ ಕೆಲಸದಲ್ಲಿ ಮಹಾಗುಣವನ್ನು ನೋಡ ಬಯಸುವೆನು. ಒಬ್ಬ ಮಹಾಪುರುಷ ಏನು ಉಣ್ಣುತ್ತಾನೆ, ಏನು ಹೊದೆಯು ತ್ತಾನೆ. ತನ್ನ ಆಳಿನವರೊಂದಿಗೆ ಹೇಗೆ ಮಾತನಾಡುತ್ತಾನೆ, ಎಂಬುದನ್ನು ನೋಡಲು ಇಚ್ಛಿಸುವೆನು. ನಾನು ಸರ್ ಫಿಲಿಪ್ ಸಿಡ್ನಿಯ ಮಹಿಮೆಯನ್ನು ನೋಡಬಯಸುವೆನು. ತನ್ನ ಮರಣಕಾಲದಲ್ಲಿಯೂ ಮತ್ತೊಬ್ಬರ ದಾಹವನ್ನು ಜ್ಞಾಪಿಸಿಕೊಳ್ಳುವವರು ಬಹಳ ಅಪರೂಪ. ದೊಡ್ಡ ಸ್ಥಾನದಲ್ಲಿ ಯಾರು ಬೇಕಾದರೂ ದೊಡ್ಡವರಾಗಿರಬಹುದು. ಕಣ್ಣನ್ನು ಕೋರೈಸುವ ಕೀರ್ತಿಕಾಂತಿ ಎದುರಿಗೆ ಎಂತಹ ಹೇಡಿಯಾದರೂ ಧೀರನಾಗುವನು. ಪ್ರಪಂಚವೇ ಅವನನ್ನು ನೋಡುತ್ತಿರುವುದು. ಆಗ ಯಾರ ಎದೆ ಸಾಹಸಕ್ಕೆ ತವಕಪಡುವುದಿಲ್ಲ? ತಮ್ಮ ಅತ್ಯುತ್ತಮವನ್ನು ಮಾಡುವ ಪರಿಯಂತವೂ ಯಾರು ಸುಮ್ಮನೆ ಇರಬಲ್ಲರು? ನಿಜವಾದ ಮಹಿಮೆ ನನಗೆ ಹೆಚ್ಚು ಹೆಚ್ಚು ಕಾಣುವುದು, ಮೌನವಾಗಿ, ನಿರಂತರವಾಗಿ, ಪ್ರತಿ ಕ್ಷಣವೂ ಪ್ರತಿ ಗಂಟೆಯೂ ತನ್ನ ಕರ್ತವ್ಯವನ್ನು ಮಾಡುತ್ತಿರುವ ಕ್ರಿಮಿಯಲ್ಲಿ.


\section{ತ್ಯಾಗ}

ಮಹಾಕಾರ್ಯಗಳು ಮಹಾತ್ಯಾಗದಿಂದ ಮಾತ್ರ ಸಾಧ್ಯ. ಸಾರ್ವತ್ರಿಕತೆ ಎಂಬ ಒಂದು ಭಾವಕ್ಕೆ, ಆವಶ್ಯಕವಾದರೆ ಎಲ್ಲವನ್ನೂ ಸಮರ್ಪಿಸಬೇಕು.

ಇತರರ ಮುಕ್ತಿ ಸಂಪಾದನೆಗೆ ಬೇಕಾದರೆ ನೀನು ನರಕಕ್ಕೆ ಬೇಕಾದರೆ ಹೋಗು. ನನ್ನದು ಎಂದು ಹೇಳಿಕೊಳ್ಳುವ ಮುಕ್ತಿ ಪ್ರಪಂಚದಲ್ಲಿ ಇಲ್ಲ.

ಯಾರು ತಮ್ಮ ಸರ್ವಸ್ವವನ್ನು ಇತರರಿಗೆ ತೆರುವರೋ ಅವರಿಗೆ ಮಾತ್ರ ಮುಕ್ತಿ. “ನನ್ನ ಮುಕ್ತಿ, ನನ್ನ ಮುಕ್ತಿ” ಎಂದು ಹಗಲಿರುಳು ಯಾರು ಆಲೋಚಿಸುತ್ತಿರುವರೋ ಅವರು ತಮ್ಮ ಇಹಪರ ಸುಖಗಳೆರಡನ್ನೂ ಕಳೆದು ಕೊಂಡು ಅಲೆದಾಡುವರು. ಇದನ್ನು ನಾನು ಹಲವು ವೇಳೆ ಪ್ರತ್ಯಕ್ಷ ನನ್ನ ಕಣ್ಣುಗಳಿಂದಲೇ ನೋಡಿರುವೆನು.

ಏನನ್ನೂ ಕೇಳಬೇಡಿ. ಹಿಂದಿರುಗಿ ಏನನ್ನೂ ಆಶಿಸಬೇಡಿ. ಕೇವಲ ಕೊಡುವು ದಕ್ಕಾಗಿ ನಿಮ್ಮಲ್ಲಿ ಏನಿರುವುದೋ ಅದನ್ನು ಕೊಡಿ. ಅದು ನಿಮಗೆ ಹಿಂತಿರುಗಿ ಬರುವುದು. ಆದರೆ ಅದನ್ನು ಕುರಿತು ಈಗ ಆಲೋಚಿಸಬೇಡಿ. ಅದು ಸಾವಿರ ಪಾಲು ಹೆಚ್ಚಾಗಿ ನಿಮಗೆ ಬರುವುದು. ಆದರೆ ನಮ್ಮ ಗಮನ ಅದರ ಕಡೆ ಇರಕೂಡದು. ಕೊಡುವುದಕ್ಕೆ ನಿಮಗೆ ಶಕ್ತಿ ಇದೆ, ಕೊಡಿ. ಅಲ್ಲಿಗೆ ಕೊನೆಗಾಣಲಿ.

ದಾನಕ್ಕಿಂತ ಮಿಗಿಲು ಒಳ್ಳೆಯ ಗುಣವಿಲ್ಲ. ಸ್ವೀಕರಿಸುವುದಕ್ಕೆ ಯಾರು ಕೈನೀಡುವರೋ ಅವರು ಹೀನಮನುಜರು. ಯಾರು ಕೊಡಲು ಕೈಎತ್ತುವರೋ ಅವರು ಮಹಾಮಹಿಮರು. ಯಾವಾಗಲೂ ನೀಡುವುದಕ್ಕೆ ಕೈಗಳಿರುವುದು. ನೀವು ಉಪವಾಸವಿದ್ದರೂ ನಿಮ್ಮ ಪಾಲಿನ ಅನ್ನವನ್ನೆಲ್ಲ ಮತ್ತೊಬ್ಬರಿಗೆ ಕೊಡಿ. ನೀವು ದಾನಮಾಡಿ ಉಪವಾಸದಿಂದ ಸತ್ತರೆ ಕ್ಷಣದಲ್ಲಿ ಮುಕ್ತರಾಗು ವಿರಿ. ತಕ್ಷಣವೇ ನೀವು ಪರಿಶುದ್ಧರಾಗುವಿರಿ, ದೇವರಾಗುವಿರಿ.

ಹೆಸರು ಯಾರಿಗೆ ಬೇಕು? ಅದು ತೊಲಗಲಿ. ಉಪವಾಸದಿಂದ ನರಳುತ್ತಿರು ವವರಿಗೆ ಅನ್ನ ಒದಗಿಸುತ್ತಿರುವ ಪ್ರಯತ್ನದಲ್ಲಿ ನಿನ್ನ ಕೀರ್ತಿ ಐಶ್ವರ್ಯವೆಲ್ಲ ನಾಶವಾದರೂ ನೀನು ಧನ್ಯ. ಹೃದಯ, ಹೃದಯವನ್ನು ಗೆಲ್ಲುವುದು. ಮೆದು ಳಲ್ಲ. ಪಾಂಡಿತ್ಯ, ಬುದ್ಧಿವಂತಿಕೆ, ಯೋಗಾಭ್ಯಾಸ, ಜ್ಞಾನ ಇವೆಲ್ಲ ಪ್ರೀತಿ ಯೊಂದಿಗೆ ಹೋಲಿಸಿದಾಗ ಧೂಳಿಗೆ ಸಮಾನ.

ಭರತಖಂಡಕ್ಕೆ ತನ್ನ ಐದು ಸಾವಿರ ಯುವಕರ ಬಲಿಯಾದರೂ ಬೇಕು. ಮನುಷ್ಯರು, ಮಾನವ ಪಶುಗಳಲ್ಲ. ಇದನ್ನು ನೆನಪಿನಲ್ಲಿಡಿ.

ಶ್ರೀಮಂತರ ಉಪಾಸನೆ ಎಂದು ಧಾರ್ಮಿಕ ಪಂಗಡಕ್ಕೆ ಪ್ರವೇಶಿಸು ವುದೋ, ಅಂದು ಆ ಧರ್ಮದ ಅವನತಿ ಆರಂಭವಾಯಿತು.

ಅತ್ಯಾವಶ್ಯಕವಾಗಿ ಬೇಕಾಗಿರುವುದು ತ್ಯಾಗ. ತ್ಯಾಗ ಇಲ್ಲದೆ ಮತ್ತೊಬ್ಬರ ಸೇವೆಗೆ ತಮ್ಮ ಇಡೀ ಹೃದಯವನ್ನು ಮುಡಿಪಾಗಿಡಲಾರರು. ತ್ಯಾಗಿ ಎಲ್ಲ ರನ್ನೂ ಸಮದೃಷ್ಟಿಯಿಂದ ನೋಡುವನು. ಹೀಗಿರುವಾಗ ಹೆಂಡತಿ ಮಕ್ಕಳನ್ನು ಇತರರಿಗಿಂತ ಹೆಚ್ಚು ತನ್ನವರು ಎಂದು ಏಕೆ ಭಾವಿಸುವೆ? ನಿನ್ನ ಮುಂದೆ ಶ್ರೀಮನ್ನಾರಾಯಣನೇ ದೀನ ಭಿಕ್ಷುಕನಂತೆ ಉಪವಾಸದಿಂದ ನರಳುತ್ತಿರು ವನು. ಅವರಿಗೆ ಏನಾದರೂ ಕೊಡುವ ಬದಲು ನಿನ್ನ ಹೆಂಡತಿ ಮಕ್ಕಳ ಹಸಿವನ್ನು ಭಕ್ಷ್ಯಭೋಜನಾದಿಗಳಿಂದ ತೃಪ್ತಿಪಡಿಸುತ್ತಿರುವೆಯೇನು? ಅದು ಮೃಗೀಯ ರೀತಿ!

ಯಾವುದು ಸ್ವಾರ್ಥವೋ ಅದು ಪಾಪ. ಯಾವುದು ನಿಃಸ್ವಾರ್ಥವೋ ಅದು ಪುಣ್ಯ.

ನಾವೆಲ್ಲ ಪ್ರಪಂಚಕ್ಕೆ ಪುಣಿಗಳು. ಪ್ರಪಂಚ ನಮಗೆ ಏನನ್ನೂ ಕೊಡ ಬೇಕಾಗಿಲ್ಲ ಎಂಬುದನ್ನು ನೆನಪಿನಲ್ಲಿಡಬೇಕು. ಪ್ರಪಂಚಕ್ಕೆ ಏನನ್ನಾದರೂ ಮಾಡುವುದು ನಮಗೊಂದು ಭಾಗ್ಯ. ಪ್ರಪಂಚಕ್ಕೆ ಸಹಾಯಮಾಡಿದರೆ ನಿಜ ವಾಗಿ ನಾವೇ ಉದ್ಧಾರವಾಗುವುದು.

ನೀನೊಂದು ದೊಡ್ಡ ವೇದಿಕೆಯ ಮೇಲೆ ನಿಂತು ಕೈಯಲ್ಲಿ ಐದು ಕಾಸು ಗಳನ್ನು ಹಿಡಿದು, “ತೆಗೆದುಕೊ ನನ್ನ ನಿರ್ಭಾಗ್ಯನೆ” ಎಂದು ಹೇಳಬೇಡ. ಆದರೆ ಆ ಭಿಕ್ಷುಕನಿಗೆ ಕೃತಜ್ಞತೆಯನ್ನು ತೋರು. ಆತನಿಗೆ ದಾನಮಾಡುವುದರಿಂದ ಕೊನೆಗೆ ನೀನೆ ಉದ್ಧಾರವಾಗುವುದು. ಸ್ವೀಕರಿಸುವವನಲ್ಲ ಧನ್ಯ. ಕೊಡುವವನು ಧನ್ಯನಾಗುವನು.

ನಿಃಸ್ವಾರ್ಥತೆ ಹೆಚ್ಚು ಲಾಭದಾಯಕ; ಆದರೆ ಅದನ್ನು ಅಭ್ಯಾಸ ಮಾಡುವು ದಕ್ಕೆ ಜನರಿಗೆ ತಾಳ್ಮೆ ಇಲ್ಲ. 

ಮಹಾಪುರುಷರಾಗಿ. ತ್ಯಾಗವಿಲ್ಲದೆ ಯಾವ ಮಹಾಕಾರ್ಯವನ್ನೂ ಸಾಧಿಸ ಲಾಗುವುದಿಲ್ಲ. ಪುರುಷನೇ ಪ್ರಪಂಚ ಸೃಷ್ಟಿಗೆ ತನ್ನನ್ನೇ ತ್ಯಾಗಮಾಡಿಕೊಂಡನು. ನಿಮ್ಮ ಅನುಕೂಲ, ಸುಖ, ಹೆಸರು, ಕೀರ್ತಿ, ಅಧಿಕಾರ, ಇಲ್ಲ ನಿಮ್ಮ ಜೀವನವನ್ನು ಅರ್ಪಿಸಿ. ಒಂದು ಮಾನವ ಸೇತುವೆಯನ್ನು ನಿರ್ಮಿಸಿ. ಇದರ ಮೇಲೆ ಕೋಟ್ಯಂತರ ಜನ ಭವಸಾಗರವನ್ನು ದಾಟುವರು. ಒಳ್ಳೆಯ ಶಕ್ತಿ ಯನ್ನೆಲ್ಲ ಒಟ್ಟುಗೂಡಿಸಿ, ಯಾವ ಹೆಸರಿನಲ್ಲಿ ನೀವು ಮುಂದುವರಿಯು ತ್ತೀರೋ ಅದನ್ನು ಲಕ್ಷಿಸಬೇಡಿ. ನಿಮ್ಮ ಬಣ್ಣ ಕೆಂಪೆ. ಕಂದೆ, ನೀಲಿಯೇ ಅದನ್ನು ಗಮನಿಸಬೇಡಿ. ಎಲ್ಲ ಬಣ್ಣವನ್ನೂ ಒಟ್ಟುಗೂಡಿಸಿ, ಪ್ರೀತಿ ಎಂಬ ಕೋರೈಸುವ ಶ್ವೇತಶಾಂತಿಯನ್ನು ಸೃಷ್ಟಿಸಿ. ನಮ್ಮ ಕರ್ತವ್ಯ ಕೆಲಸಮಾಡು ವುದು. ಫಲ ತನ್ನನ್ನು ತಾನು ನೋಡಿಕೊಳ್ಳುವುದು.

ಪ್ರಪಂಚದಲ್ಲಿ ಯಾವಾಗಲೂ ದಾನಿಯ ಸ್ಥಾನವನ್ನು ತೆಗೆದುಕೊಳ್ಳಿ. ಸರ್ವವನ್ನೂ ತ್ಯಾಗಮಾಡಿ, ಪ್ರತಿಫಲವನ್ನು ನಿರೀಕ್ಷಿಸಬೇಡಿ. ಪ್ರೀತಿಕೊಡಿ, ಸಹಾಯಮಾಡಿ, ಸೇವೆಮಾಡಿ. ನಿಮ್ಮಲ್ಲಿರುವ ಎಂತಹ ಸಣ್ಣದನ್ನಾದರೂ ಕೊಡಿ. ಆದರೆ ವ್ಯಾಪಾರ ಮಾಡಬೇಡಿ. ಯಾವ ಷರತ್ತನ್ನೂ ಹಾಕಬೇಡಿ. ಯಾವುದನ್ನೂ ಬಲಾತ್ಕರಿಸಬೇಡಿ. ಭಗವಂತ ಹೇಗೆ ನಮಗೆ ನೀಡುವನೋ ಹಾಗೆ ಔದಾರ್ಯತೆಯಿಂದ ನಾವು ಪರರಿಗೆ ನೀಡುವ.


\section{ಶ್ರದ್ಧೆ}

ಶ್ರೀಮಂತರೆಂದು ಕರೆಸಿಕೊಳ್ಳವವರನ್ನು ನೆಚ್ಚಬೇಡಿ. ಅವರು ಬದುಕುವು ದಕ್ಕಿಂತ ಹೆಚ್ಚು ಮೃತರಂತಿರುವರು. ಭರವಸೆ ಇರುವುದು ಯಾರು ದೀನರೋ, ವಿನೀತರೋ, ಆದರೆ ಶ್ರದ್ಧಾಳುಗಳೋ ಅವರ ಮೇಲೆ. ದೇವರಲ್ಲಿ ನಂಬಿ. ನನಗೆ ಒಬ್ಬ ನಿಃಸ್ಪೃಹ ಮನುಷ್ಯನನ್ನು ಕೊಡಿ. ಒಂದು ಜನಮಂದೆ ಬೇಡ. ಯಾರ ಸಹಾಯವನ್ನೂ ನೀವು ನಿರೀಕ್ಷಿಸಬೇಡಿ. ಎಲ್ಲ ಮಾನವ ಸಹಾಯಕ್ಕಿಂತ ಭಗವಂತ ಅತಿ ಮೇಲಲ್ಲವೇ? ಪರಿಶುದ್ಧರಾಗಿ, ದೇವರಲ್ಲಿ ನಂಬಿ. ಆತನನ್ನೆ ಅನುಗಾಲವೂ ಆಶ್ರಯಿಸಿ. ಆಗ ನೀವು ಸರಿಯಾದ ದಾರಿಯಲ್ಲಿರುವಿರಿ. ಯಾವುದೂ ನಿಮ್ಮನ್ನು ವಿರೋಧಿಸಲಾರದು.

ವಿಧೇಯತೆ ಎಂಬ ಸದ್ಗುಣವನ್ನು ಅಭ್ಯಾಸ ಮಾಡಿ. ಆದರೆ ನಿಮ್ಮ ಭಕ್ತಿಯನ್ನು ಅದಕ್ಕೆ ಬಲಿಕೊಡಬೇಕಾಗಿಲ್ಲ. ಹಿರಿಯರಿಗೆ ವಿಧೇಯತೆ ಇಲ್ಲದೇ ಇದ್ದರೆ ಯಾವ ಕೇಂದ್ರೀಕರಣವೂ ಅಸಾಧ್ಯ. ವ್ಯಕ್ತಿಗಳ ಸಾಮರ್ಥ್ಯವನ್ನೆಲ್ಲ ಕೇಂದ್ರೀಕರಿಸುವವರೆಗೆ ಯಾವ ಮಹಾ ಕಾರ್ಯವನ್ನೂ ಮಾಡಲಾಗುವುದಿಲ್ಲ.

ನಿಮ್ಮಲ್ಲಿ ಪ್ರತಿಯೊಬ್ಬರೂ ಮಹಾವ್ಯಕ್ತಿಗಳಾಗಲೇಬೇಕು. ಇದೇ ನನ್ನ ಹಾರೈಕೆ. ಒಂದು ಆದರ್ಶಕ್ಕೆ ವಿಧೇಯತೆ, ಪ್ರೀತಿ, ಏನು ಬೇಕಾದರೂ ಮಾಡಲು ಸಿದ್ಧತೆ, ಇವು ಮೂರೂ ನಿಮ್ಮಲ್ಲಿದ್ದರೆ ಯಾವುದೂ ನಿಮ್ಮನ್ನು ತಡೆಯಲಾರದು.

ಕಾದಿರುವಾಗ ಕಬ್ಬಿಣವನ್ನು ಹೊಡೆಯಿರಿ. ಸೋಮಾರಿತನದಿಂದ ಪ್ರಯೋ ಜನವಿಲ್ಲ. ಒಂದೇ ಸಲ ಅಸೂಯೆ, ಅಹಂಕಾರವನ್ನೆಲ್ಲ ಕಿತ್ತೊಗೆಯಿರಿ. ಅದ್ಭುತ ಸಾಮರ್ಥ್ಯದಿಂದ ಹೃತ್ಪೂರ್ವಕವಾಗಿ ಕೆಲಸಮಾಡಲು ಕಾರ್ಯ ರಂಗಕ್ಕೆ ಇಳಿಯಿರಿ. ಉಳಿದುದನ್ನು ದೇವರೇ ತೋರುವನು.

ಅವಸರದಿಂದ ಏನನ್ನೂ ಮಾಡಬೇಡಿ. ಪರಿಶುದ್ಧತೆ, ತಾಳ್ಮೆ, ಸತತ ಪ್ರಯತ್ನ, ಇವು ಮೂರು ಜಯಕ್ಕೆ ಅತ್ಯಾವಶ್ಯಕ. ಎಲ್ಲಕ್ಕಿಂತ ಪ್ರೀತಿ ಇರಬೇಕು. ಕಾಲವೆಲ್ಲ ನಿಮ್ಮಲ್ಲಿದೆ. ಅಯುಕ್ತ ಅವಸರ ಬೇಡ. 

ಪ್ರತಿ ಜನಾಂಗದ ಇತಿಹಾಸದಲ್ಲೂ ಯಾರು ತಮ್ಮಲ್ಲಿ ನಂಬಿದ್ದರೋ ಅವರು ಮಾತ್ರ ಪ್ರಖ್ಯಾತರಾಗಿ ಬಲಾಢ್ಯರಾದವರು ಎಂಬುದನ್ನು ನೀವು ಕಾಣುವಿರಿ.

ಎಲ್ಲರೂ ಒಂದು ದಿನ, ಒಂದು ನಿಮಿಷ, “ಯಾರೂ ನಿಮ್ಮ ಇಚ್ಛಾಮಾತ್ರ ದಿಂದಲೇ ಪ್ರಖ್ಯಾತರಾಗಲಾರರು. ಭಗವಂತ ಯಾರನ್ನು ಮೇಲೆತ್ತುವನೋ, ಅವನು ಮೇಲೇಳುವನು. ಯಾರನ್ನು ಅವನು ಕೆಳಗೆ ತರುವನೋ, ಅವನು ತೇಲುವನು” ಎಂಬುದನ್ನು ತಿಳಿದರೆ ಎಲ್ಲಾ ತೊಂದರೆಯೂ ಕೊನೆಗಾಣು ವುದು. ಆದರೆ ಅಹಂಕಾರವಿದೆಯೆಲ್ಲ! ಏನೂ ಹುರುಳಿಲ್ಲ, ಒಂದು ಬೆರಳನ್ನು ಆಡಿಸುವುದಕ್ಕೇ ಶಕ್ತಿಯಿಲ್ಲ; ಆದರೂ “ನಾನು ಮೇಲೇಳಲು ಯಾರಿಗೂ ಬಿಡುವುದಿಲ್ಲ” ಎಂದು ಹೇಳುವುದು ಎಷ್ಟು ನಾಚಿಕೆಗೇಡು. ಅಸೂಯೆ, ಎಲ್ಲರೂ ಕಲೆತು ಒಮ್ಮತದಿಂದ ಕೆಲಸ ಮಾಡುವ ಅಭಾವ, ಇವು ದಾಸ್ಯದಲ್ಲಿ ನರಳುವ ಎಲ್ಲಾ ಜನಾಂಗದ ಸ್ವಭಾವ. ಅದರಿಂದ ನಾವು ಪಾರಾಗಲು ಯತ್ನಿಸಬೇಕು.

ಸಣ್ಣ ಕೆಲಸಗಳ ಮಹತ್ವ–ಗೀತೆ ಇದನ್ನು ಬೋಧಿಸುವುದು. ಆ ಪುರಾತನ ಶಾಸ್ತ್ರಕ್ಕೆ ಅನಂತ ವಂದನೆ.

ಮೃತ್ಯುವಶರಾದವರು ಮರಳಿ ಬಾರರು. ಕಳೆದ ಇರುಳು ಮರಳಿ ಬಾರದು. ಇಳಿದು ಮಾಯವಾದಕಡಲಿನ ಉಬ್ಬರ ಪುನಃ ಮೇಲೇಳಲಾರದು. ಅಥವಾ ಜೀವ ಪುನಃ ತನ್ನ ದೇಹವನ್ನೇ ಎರಡನೇ ಬಾರಿ ಪ್ರವೇಶಿಸದು. ಆದರಿಂದ ಹೇ ಮಾನವ! ಅತೀತದ ಪೂಜೆಯಿಂದ ಪ್ರತ್ಯಕ್ಷದ ಪೂಜೆಗೆ ನಿನಗಿದೋ ಆಹ್ವಾನ. ಹೋದ ದುಃಖದಿಂದ ಬರುವ ಸುಖಕ್ಕೆ ನಿನಗೆ ಇದೋ ಆಹ್ವಾನ. ಗತಾನು ಶೋಚನೆಯಿಂದ ಆಧುನಿಕ ಪ್ರಯತ್ನಕ್ಕೆ ನಿನಗಿದೊ ಆಹ್ವಾನ. ಲುಪ್ತಪಂಥದ ಪುನರುದ್ಧಾರದ ವೃಥಾ ಶಕ್ತಿಕ್ಷಯದಿಂದ ಸದ್ಯೋನಿರ್ಮಿತ ವಿಶಾಲ ಸನ್ನಿಕಟ ಪಥಕ್ಕೆ ನಿನಗಿದೋ ಆಹ್ವಾನ! ಓ ಬುದ್ದಿಮಾನ್ ಮಾನವ; ಇದನ್ನು ತಿಳಿ; ಸಿದ್ಧನಾಗು, ಪ್ರಬುದ್ಧನಾಗು.

ನಿಜವಾದ ಶ್ರದ್ಧೆಯ ಅರ್ಥವನ್ನು ನಾವು ಪುನಃ ತಿಳಿಯಬೇಕು. ಆತ್ಮ ಶ್ರದ್ಧೆಯನ್ನು ಮತ್ತೊಮ್ಮೆ ಜಾಗ್ರತಗೊಳಿಸಬೇಕು. ಆಗ ಮಾತ್ರ ನಮ್ಮ ದೇಶವನ್ನು ಆವರಿಸಿರುವ ಹಲವಾರು ಸಮಸ್ಯೆಗಳನ್ನು ನಾವೇ ನಿವಾರಿಸಬಲ್ಲೆವು.

ನೀನು ಸ್ವಲ್ಪ ನಾಜೂಕಾಗಿ ಮಾತನಾಡುವುದರಿಂದ ಬೀದಿಹೋಕನಿಗಿಂತ ಮೇಲೆಂದು ಭಾವಿಸುವೆಯೇನು? ಎಲ್ಲಕ್ಕಿಂತ ಹೆಚ್ಚಾಗಿ ಆಧ್ಯಾತ್ಮಿಕ ಅಹಂ ಕಾರಕ್ಕೆ ನೀನು ತುತ್ತಾದರೆ ಭಗವಂತನೆ ನಿನ್ನನ್ನು ರಕ್ಷಿಸಬೇಕು. ಇದು ಇರುವ ಬಂಧನಗಳಲ್ಲೆಲ್ಲ ಅತಿ ಭಯಂಕರ ಬಂಧನ.

ಕಠೋಪನಿಷತ್ತಿನಲ್ಲಿ ಬರುವ ಶ್ರದ್ಧೆ ಅಥವಾ ಅತ್ಯದ್ಭುತವಾದ ನಂಬಿಕೆ ಯೆಂಬ ಮಹಾಪದವನ್ನು ಜ್ಞಾಪಿಸಿಕೊಳ್ಳುತ್ತೇನೆ. ಶ್ರದ್ಧಾ ಸಿದ್ಧಾಂತವನ್ನು ಬೋಧಿಸುವುದೇ ನನ್ನ ಜೀವನದ ಸಂದೇಶ. ಈ ಶ್ರದ್ಧೆಯೇ ಮಾನವಕೋಟಿಯ ಮತ್ತು ಎಲ್ಲಾ ಧರ್ಮದ ಶಕ್ತಿಮೂಲ ಎಂಬುದನ್ನು ಒತ್ತಿ ಹೇಳುತ್ತೇನೆ. ಮೊದಲು ನಿಮ್ಮಲ್ಲಿ ಶ್ರದ್ಧೆ ಇರಲಿ. ಶ್ರೀಮಂತರು ಮತ್ತು ಗೌರವಸ್ಥರಿಗೆ ಕಾಯಬೇಡಿ. ಬಡವರು ಜಗದ ಮಹತ್ಕಾರ್ಯಗಳನ್ನು ಅದ್ಭುತ ಕಾರ್ಯ ಗಳನ್ನು ಸಾಧಿಸಿದರು. ಚಂಚಲಚಿತ್ತರಾಗಬೇಡಿ. ಎಲ್ಲಕ್ಕಿಂತ ಹೆಚ್ಚಾಗಿ ಪರಿ ಶುದ್ಧರಾಗಿ, ಹೃತ್ಪೂರ್ವಕ ನಿಷ್ಕಪಟಿಗಳಾಗಿ. ನಿಮ್ಮ ಅದೃಷ್ಟದಲ್ಲಿ ನಂಬಿಕೆ ಇರಲಿ.

ನಮ್ಮ ಜೀವನಾಡಿಯೊಳಗೆ ಪ್ರವೇಶಿಸುತ್ತಿರುವ ಎಲ್ಲದನ್ನೂ, ತಾತ್ಸಾರ ದಿಂದ ಕಾಣುವ ಶ್ರದ್ಧಾಹೀನ ಸ್ವಭಾವದ ಕ್ರೂರ ರೋಗದಿಂದ ಪಾರಾಗಿ. ಅದನ್ನು ತ್ಯಜಿಸಿ. ಪೌರುಷವಂತರಾಗಿ, ಶ್ರದ್ಧಾವಂತರಾಗಿ, ಉಳಿದುವೆಲ್ಲ ಇದನ್ನು ಅನುಸರಿಸಿಯೇ ತೀರಬೇಕು.

ಅಯ್ಯೋ, ನಾವು ದರಿದ್ರರು, ಸ್ನೇಹಿತರಿಲ್ಲವೆಂದು ಯೋಚಿಸಬೇಡಿ. ಹಣ ಮನುಷ್ಯನನ್ನು ನಿರ್ಮಾಣಮಾಡುವುದನ್ನು ಯಾರಾದರೂ ನೋಡಿರುವಿರಾ? ಮನುಷ್ಯ ಯಾವಾಗಲೂ ಹಣವನ್ನು ಸೃಷ್ಟಿಸುವುದು. ಮಾನವ ತನ್ನ ಶಕ್ತಿ ಯಿಂದ, ಸಾಮರ್ಥ್ಯದಿಂದ, ಉತ್ಸಾಹದಿಂದ, ಶ್ರದ್ಧಾಬಲದಿಂದ ಪ್ರಪಂಚ ವನ್ನು ವಿರಚಿಸುವನು.

ನನಗೆ ಮತಭ್ರಾಂತನ ತೀವ್ರತೆ, ಜೊತೆಗೆ ಜಢವಾದಿಯ ವೈಶಾಲ್ಯತೆ ಬೇಕು. ಸಾಗರದಷ್ಟು ಆಳವಾದ, ಆಗಸದಷ್ಟು ವಿಶಾಲವಾದ ಹೃದಯ ಬೇಕು.

ಸ್ವಭಾವತಃ ಅರ್ಥವಿಲ್ಲದ ಅಸಂಬದ್ಧ ಪ್ರಲಾಪವನ್ನು ಕೊನೆಗಾಣಿಸಿ. ಕಳೆದ ಆರೇಳು ಶತಮಾನಗಳ ಅವನತಿಯನ್ನು ಕುರಿತು ಯೋಚಿಸಿ ನೋಡಿ. ನೂರಾರು ಜನ ಘನ ವಿದ್ವಾಂಸರು ಹಲವು ವರ್ಷಗಳು ಬಹಳ ಆಸಕ್ತಿಯಿಂದ ಚರ್ಚಿಸು ತ್ತಿರುವ ಸಮಸ್ಯೆಯು ಇದು: ನಾವು ಒಂದು ಬಟ್ಟಲು ನೀರನ್ನು ಎಡಗೈಯಿಂದ ಕುಡಿಯುವುದೇ ಅಥವಾ ಬಲಗೈಯಿಂದ ಕುಡಿಯುವುದೇ, ಕೈಗಳನ್ನು ಮೂರು ಸಲ ತೊಳೆಯುವುದೇ ಅಥವಾ ನಾಲ್ಕು ಸಲ ತೊಳೆಯುವುದೇ, ನಾವು ಬಾಯಿಯನ್ನು ಐದು ಸಲವೋ ಅಥವಾ ಆರು ಸಲವೋ ಮುಕ್ಕಳಿಸುವುದು, ಇಂತಹ ಮಹಾಪ್ರಶ್ನೆಗಳನ್ನು ಚರ್ಚಿಸುತ್ತಿರುವ ಮತ್ತು ಅವುಗಳ ಮೇಲೆ ದೊಡ್ಡ ತತ್ತ್ವಗ್ರಂಥವನ್ನೇ ಬರೆಯುವುದರಲ್ಲಿ ತಮ್ಮ ಕಾಲವನ್ನು ವ್ಯಯ ಮಾಡುತ್ತಿರುವ ಜನರಿಂದ ನೀವು ಏನನ್ನು ನಿರೀಕ್ಷಿಸಬಲ್ಲಿರಿ? ನಮ್ಮ ಧರ್ಮ ಪಾಕಶಾಲೆಗೆ ಪ್ರವೇಶಿಸುವ ಅಪಾಯವಿದೆ. ನಮ್ಮಲ್ಲಿ ಹಲವರು ವೇದಾಂತಿ ಗಳೂ ಅಲ್ಲ, ತಾಂತ್ರಿಕರೂ ಅಲ್ಲ, ನಾವೆಲ್ಲ “ನಮ್ಮನ್ನು ಮುಟ್ಟಬೇಡಿ” ಎನ್ನುವವರು. ನಮ್ಮ ಧರ್ಮ ಅಡಿಗೆಮನೆಯಲ್ಲಿದೆ. ನಮ್ಮ ದೇವರೇ ಅಡಿಗೆ ಮಾಡುವ ಪಾತ್ರೆ. ನಮ್ಮ ಧರ್ಮವೇ “ನನ್ನನ್ನು ಮುಟ್ಟಬೇಡಿ, ನಾನು ಪವಿತ್ರ” ಎನ್ನುವುದು. ಹೀಗೆ ಮತ್ತೊಂದು ಶತಮಾನ ಕಳೆದರೆ ಪ್ರತಿಯೊಬ್ಬರೂ ಹುಚ್ಚರ ಆಸ್ಪತ್ರೆಯಲ್ಲಿರಬೇಕಾಗುವುದು.

ನಿನ್ನ ಆದರ್ಶದ ಮೇಲೆ ಭಕ್ತಿ ಇರಬೇಕು. ಕ್ಷಣಿಕ ಭಕ್ತಿಯಲ್ಲ. ಗುಡುಗಾಡಿ ಮಿಂಚುತ್ತಿರುವಾಗಲೂ ಮತ್ತಾವ ನೀರನ್ನೂ ಕುಡಿಯದೆ ಆಕಾಶದ ಕಡೆ ಮೋಡವನ್ನು ನೋಡುತ್ತಿರುವ ಚಾತಕದಂತೆ. ಶಾಂತವಾದ ಸೋಲನ್ನು ಒಪ್ಪಿ ಕೊಳ್ಳದ ನಿರಂತರ ಭಕ್ತಿ ಬೇಕು. ಪವಿತ್ರನಾಗಬೇಕೆಂಬ ಸಾಧನೆಯಲ್ಲಿ ಮಡಿ. ಸಾವಿರ ಸಲವಾದರೂ ಮೃತ್ಯುವನ್ನು ಎದುರಿಸು. ಎಂದಿಗೂ ನಿರಾಶನಾಗದಿರು. ಅಮೃತ ಸಿಕ್ಕದೇ ಇದ್ದರೆ ವಿಷವನ್ನು ಪಾನಮಾಡಬೇಕಾಗಿಲ್ಲ.

ಧರ್ಮದ ವಿಷಯದಲ್ಲಿ ಜಗಳವಾಡಬೇಡಿ. ಧರ್ಮಕ್ಕೆ ಸಂಬಂಧಪಟ್ಟ ಎಲ್ಲಾ ವ್ಯಾಜ್ಯ ಮತ್ತು ವಾದ ನಮ್ಮ ಅಧ್ಯಾತ್ಮಿಕತೆ ಅಭಾವವನ್ನು ತೋರು ವುದು. ಮತಕಲಹ ಯಾವಾಗಲೂ ಹೊಟ್ಟಿನ ಮೇಲೆ. ಎಂದು ಪವಿತ್ರತೆ ಮತ್ತು ಆಧ್ಯಾತ್ಮಿಕತೆ ನಮ್ಮಲ್ಲಿ ಇಲ್ಲವೋ, ಹೃದಯ ಶುಷ್ಕವಾಗುವುದೊ, ಆಗ ವ್ಯಾಜ್ಯ ತಲೆದೋರುವುದು.

ಸಿದ್ಧಾಂತ ಶಾಸ್ತ್ರಾಂಧತೆ, ಪಂಗಡ, ಚರ್ಚು, ದೇವಸ್ಥಾನ ಇವನ್ನು ಪ್ರತಿ ಯೊಬ್ಬ ಮಾನವ ಜೀವನದ ಸಾರವಾದ ಆಧ್ಯಾತ್ಮಿಕದೊಂದಿಗೆ ಹೋಲಿಸಿದಾಗ ಇವಕ್ಕೇನೂ ಬೆಲೆಯಿಲ್ಲ. ಈ ಆಧ್ಯಾತ್ಮಿಕತೆ ಮನುಷ್ಯನಲ್ಲಿ ಹೆಚ್ಚು ವಿಕಾಸ ವಾದಷ್ಟೂ ಅವನು ಒಳ್ಳೆಯ ಕೆಲಸಕ್ಕೆ ಹೆಚ್ಚು ಶಕ್ತಿ ಪಡೆಯುವನು. ಮೊದಲು ಅದನ್ನು ಪಡೆಯಿರಿ; ಸಂಪಾದಿಸಿ. ಯಾರನ್ನೂ ದೂರಬೇಡಿ, ಎಲ್ಲಾ ಸಿದ್ಧಾಂತ ದಲ್ಲೂ ಮತ್ತು ಮತದಲ್ಲೂ ಕೆಲವು ಒಳ್ಳೆಯ ಅಂಶಗಳಿವೆ. ಧರ್ಮವೆಂದರೆ ಮಾತಲ್ಲ, ಹೆಸರಲ್ಲ, ಪಂಗಡವಲ್ಲ. ಅದು ಆತ್ಮಸಾಕ್ಷಾತ್ಕಾರ ಎಂಬುದನ್ನು ನಿಮ್ಮ ಜೀವನದಲ್ಲಿ ತೋರಿ.

ನಿಷ್ಕಾಪಟ್ಯ, ನಂಬಿಕೆ, ಪರಿಶುದ್ಧ ಉದ್ದೇಶ ಇವೇ ಕೊನೆಗೆ ಜಯಿಸುವುವು. ಇವುಗಳಿಂದ ಸನ್ನದ್ಧರಾದವರು ಅಲ್ಪಪಕ್ಷದವರಾದರೂ ಎಲ್ಲರನ್ನೂ ವಿರೋ ಧಿಸಿ ನಿಲ್ಲುವ ಸಾಮರ್ಥ್ಯ ಇದೆ.

ಸತ್ಯಸಂಧನೆ, ಪವಿತ್ರತೆ, ನಿಃಸ್ವಾರ್ಥತೆ ಎಲ್ಲಿ ಈ ಮೂರು ಗುಣಗಳು ನೆಲಸಿರುವವೋ ಅಲ್ಲಿ ಇವುಗಳೊಡೆಯನನ್ನು ನಾಶಮಾಡುವ ಶಕ್ತಿ ಈ ಪ್ರಪಂಚದಲ್ಲಿ ಅಥವಾ ಇದರಿಂದ ಹೊರಗೆ ಎಲ್ಲಿಯೂ ಇಲ್ಲ. ಈ ಶೀಲ ಗಳಿಂದ ಸುಸಜ್ಜಿತವಾದ ವ್ಯಕ್ತಿ ಇಡೀ ಜಗತ್ತನ್ನೇ ಎದುರಿಸಬಲ್ಲ. ಏಳಿ, ಜಾಗ್ರತರಾಗಿ, ಗುರಿ ಸೇರುವವರೆಗೂ ನಿಲ್ಲಬೇಡಿ.


\section{ಶಿಕ್ಷಣ}

ಶಿಕ್ಷಣವೆಂದರೆ ಸುಪ್ತವಾದ ಪರಿಪೂರ್ಣತೆಯನ್ನು ವ್ಯಕ್ತಗೊಳಿಸುವುದು. 

ವಿದ್ಯಾಭ್ಯಾಸವೆಂದರೆ ಏನು? ಇದು ಪುಸ್ತಕ ಪಾಂಡಿತ್ಯವೆ? ಅಲ್ಲ. ಬಗೆ ಬಗೆಯ ವಿಷಯಸಂಗ್ರಹವೆ? ಇದೂ ಅಲ್ಲ. ಯಾವ ತರಬೇತಿನಿಂದ ಮಾನವ ಇಚ್ಛಾಪ್ರವಾಹ ಮತ್ತು ಅದು ಆವಿರ್ಭವಿಸುವ ರೀತಿ ನಮ್ಮ ಸ್ವಾಧೀನಕ್ಕೆ ಬಂದು ಫಲಕಾರಿಯಾಗುವುದೋ ಅವೇ ವಿದ್ಯಾಭ್ಯಾಸ. ಇಚ್ಛಾಪ್ರವಾಹ ತಲೆ ತಲಾಂತರಗಳಿಂದ ಅಡಚಣೆಗೆ ತುತ್ತಾಗಿ ಇನ್ನೇನು ನಿರ್ನಾಮವಾಗುವ ಸ್ಥಿತಿ ಯಲ್ಲಿರುವುದು ವಿದ್ಯಾಭ್ಯಾಸವೆ? ಇದನ್ನು ಆಲೋಚಿಸಿ ನೋಡಿ. ಈ ಅಡ ಚಣೆಯ ಪ್ರತಿಫಲವಾಗಿ ಹೊಸ ಭಾವನೆಗಳೆಲ್ಲ, ಜೊತೆಗೆ ಹಳೆಯ ಭಾವನೆ ಗಳೆಲ್ಲ, ಒಂದಾದ ಮೇಲೊಂದು ಕ್ರಮೇಣ ಮಾಯವಾಗುತ್ತಿವೆ. ಯಾವುದು ಕ್ರಮೇಣ ಮನುಷ್ಯನನ್ನು ಯಂತ್ರವಾಗಿ ಮಾಡುತ್ತಿರುವುದೋ ಅದು ವಿದ್ಯಾ ಭ್ಯಾಸವೆ?

ನನಗೆ ವಿದ್ಯಾಭ್ಯಾಸದ ಸಾರವೆ ಮಾನಸಿಕ ಏಕಾಗ್ರತೆ, ವಿಷಯಸಂಗ್ರಹವಲ್ಲ. ನಾನು ಪುನಃ ವಿದ್ಯಾಭ್ಯಾಸವನ್ನು ಮಾಡಬೇಕಾದರೆ ಅದರಲ್ಲಿ ನನಗೆ ಸ್ವಾತಂತ್ರ್ಯ ವಿದ್ದರೆ, ನಾನು ಪಾಠಗಳನ್ನು ಓದುವುದೇ ಇಲ್ಲ. ಮೊದಲು ಏಕಾಗ್ರತೆ ಮತ್ತು ಅನಾಸಕ್ತಿಯ ಶಕ್ತಿಯನ್ನು ಉತ್ಪತ್ತಿಮಾಡಿಕೊಂಡು ನಂತರ ಒಂದು ಪರಿಶುದ್ಧ ವಾದ ಮಾನಸಿಕ ಯಂತ್ರದ ಮೂಲಕ ನನ್ನ ಇಚ್ಛಾನುಸಾರ ವಿಷಯಸಂಗ್ರಹ ಮಾಡುವೆನು. ಮಗುವಿನಲ್ಲಿ ಒಟ್ಟಿಗೆ ಏಕಾಗ್ರತೆ ಮತ್ತುಅನಾಸಕ್ತಿಯನ್ನು ರೂಪಿ ಸಬೇಕು.

ಕೆಲವು ಪರೀಕ್ಷೆಗಳಲ್ಲಿ ಮನುಷ್ಯ ಉತ್ತೀರ್ಣನಾಗಿ ಚೆನ್ನಾಗಿ ಮಾತನಾಡಲು ಬಂದರೆ ಅವನನ್ನು ವಿದ್ಯಾವಂತನೆಂದು ಕರೆಯುವಿರೇನು? ಯಾವ ವಿದ್ಯೆ ಜನಸಾಮಾನ್ಯರ ಜೀವನೋಪಾಯಕ್ಕೆ ಸಹಾಯ ಮಾಡಲಾರದೋ, ಯಾವುದು ಶುದ್ಧ ಚಾರಿತ್ರ, ಅನುಕಂಪ ಹೃದಯ, ಸಿಂಹಸದೃಶ ಧೈರ್ಯವನ್ನು ನೀಡಲಾ ರದೋ, ಅದಕ್ಕೆ ಗೌರವವಿದೆಯೆ? ತನ್ನ ಸ್ವಂತ ಕಾಲಮೇಲೆ ನಿಂತುಕೊಳ್ಳು ವುದಕ್ಕೆ ಸಹಾಯ ಮಾಡುವುದೇ ನಿಜವಾದ ವಿದ್ಯೆ. ಈಗ ನಮಗೆ ಶಾಲೆ ಕಾಲೇಜುಗಳಲ್ಲಿ ಸಿಕ್ಕುವ ವಿದ್ಯೆ ನಿಮ್ಮನ್ನು ಚಿರಕಾಲ ಅಜೀರ್ಣ ವ್ಯಾಧಿಯಿಂದ ನರಳುತ್ತಿರುವ ಜನಾಂಗವಾಗಿ ಮಾಡಿದೆ. ನೀವು ಯಂತ್ರದಂತೆ ಕೆಲಸ ಮಾಡು ತ್ತಿರುವಿರಿ. ಚಲನಾಹೀನ ಲೋಳೆಮೀನನಂತೆ \eng{(Jelly fish)} ಬಾಳುತ್ತಿರುವಿರಿ.

ವಿದ್ಯಾಭ್ಯಾಸವೆಂದರೆ ನಮ್ಮ ಮೆದುಳಿನೊಳಗೆ ಸಂಗ್ರಹಿಸಿದ ವಿಷಯ ಸಂಗ್ರಹವಲ್ಲ. ಅಲ್ಲಿ ಅವು ಕೊಳೆತು ನಾರಿ ಇಡೀ ಜೀವನದಲ್ಲೇ ಜೀರ್ಣ ವಾಗದೆ ಇರುವುದು. ವಿದ್ಯೆ ನಮ್ಮ ಜೀವನಾಭ್ಯುದಯಕ್ಕೆ ಸಹಾಯಮಾಡ ಬೇಕು. ನಾವು ವೀರ ಪುರುಷಸಿಂಹರಾಗಬೇಕು. ಚಾರಿತ್ರಶುದ್ಧರಾಗಬೇಕು. ಭಾವನೆಯನ್ನು ರಕ್ತಗತ ಮಾಡಿಕೊಳ್ಳಬೇಕು. ವಿದ್ಯಾಭ್ಯಾಸ ಇದಕ್ಕೆ ಸಹಾಯ ವಾಗಬೇಕು. ನೀನು ಐದು ಭಾವನೆಗಳನ್ನು ಜೀವನದ ಉಸಿರಾಗಿ ಮಾಡಿಕೊಂಡಿ ದ್ದರೆ, ಪುಸ್ತಕಾಲಯ ಕಂಠಪಾಠಮಾಡಿಕೊಂಡ ಘನಪಂಡಿತನಿಗಿಂತ ಹೆಚ್ಚು ವಿದ್ಯಾವಂತ. ವಿದ್ಯಾಭ್ಯಾಸ ಕೇವಲ ವಿಷಯಸಂಗ್ರಹವಾದರೆ, ಗ್ರಂಥ ಭಂಡಾರಗಳೇ ಪ್ರಪಂಚದಲ್ಲಿ ಘನವಿದ್ವಾಂಸರು, ವಿಶ್ವಕೋಶಗಳೇ \eng{(Encyclopedia)} ಪುಷಿಗಳಾಗುವುವು.

ಭಾಷೆಯಲ್ಲಿ ಸಾಹಿತ್ಯದಲ್ಲಿ, ಕಾವ್ಯದಲ್ಲಿ ಮತ್ತು ಲಲಿತ ಕಲೆಯಲ್ಲಿ ಮತ್ತು ಎಲ್ಲಾ ಕಾರ್ಯಕ್ಷೇತ್ರದಲ್ಲಿಯೂ ಜನರು ಆಲೋಚನೆಯಲ್ಲಿ ಮತ್ತು ಕಾರ್ಯ ದಲ್ಲಿ ಮಾಡುತ್ತಿರುವ ತಪ್ಪನ್ನು ಎತ್ತಿತೋರಕೂಡದು. ಆದರೆ ಕ್ರಮೇಣ ಅದನ್ನು ಹೇಗೆ ಉತ್ತಮಗೊಳಿಸಬೇಕು ಎಂಬುದನ್ನು ತೋರಬೇಕು. ಸುಮ್ಮನೆ ತಪ್ಪನ್ನು ತೋರುವುದು ಜನರನ್ನು ವ್ಯಥೆಗೆ ಈಡುಮಾಡುವುದು. ನಾವು ಯಾರನ್ನು ಶುದ್ಧ ಅವಿವೇಕಿಗಳು, ಕೆಲಸಕ್ಕೆ ಬಾರದವರೆಂದು ಪರಿಗಣಿಸಿ ದ್ದೆವೋ, ಅವರನ್ನು ಶ್ರೀರಾಮಕೃಷ್ಣರು ಪ್ರೋತ್ಸಾಹಿಸಿ ಅವರ ಜೀವನವನ್ನೇ ಬದಲಾಯಿಸಿರುವುದನ್ನು ನಾವು ನೋಡಿರುವೆವು. ಅವರ ಬೋಧನಾರೀತಿಯ ಒಂದು ಅದ್ಭುತ ಚಮತ್ಕಾರವಿದು.


\section{ಬಿಡಿ ನುಡಿಗಳು}

ಪಾಶ್ಚಾತ್ಯರ ಜೀವನದ ಮೇಲುಗಡೆ ಅಟ್ಟಹಾಸವಿದ್ದರೂ ಒಳಗಡೆ ಗೋಳಿದೆ. ಅದು ಗೋಳಿನಲ್ಲಿಯೇ ಕೊನೆಗಾಣುವುದು. ನಗೆ-ನಲಿದಾಟಗಳೆಲ್ಲ ಮೇಲೆಮೇಲೆ. ನಿಜವಾಗಿ ಅದರೊಳಗಡೆ ರುದ್ರಮುಖದ ಬಾಳು ಕಾವಾಗಿ ಸಿಡಿಯುತ್ತಿದೆ...ಇಂಡಿಯಾದೇಶದಲ್ಲಿ ಬಾಳು ಮೇಲೆಮೇಲೆ ನೋಡುವುದಕ್ಕೆ ವಿಷಣ್ಣವಾಗಿದ್ದರೂ, ಅಂತರಂಗದಲ್ಲಿ ಪ್ರಸನ್ನವಾಗಿದೆ.

ಅಖಿಲ ಯುರೋಪುಖಂಡವೂ ಜ್ವಾಲಾಮುಖಿಯ ಮೇಲೆ ಕುಳಿತಂತಿದೆ. ಧರ್ಮಧಾರಾಪ್ರವಾಹದಿಂದ ಆ ಬೆಂಕಿಯನ್ನು ಆರಿಸದೆ ಇದ್ದರೆ ಅದು ಆಸ್ಫೋಟಿಸದೆ ಇರುವುದಿಲ್ಲ.

ಬ್ರಾಹ್ಮಶಕ್ತಿ, ಕ್ಷಾತ್ರಶಕ್ತಿಗಳ ಆಳ್ವಿಕೆ ಪೂರೈಸಿ ಈಗ ವೈಶ್ಯಶಕ್ತಿ ಆಳುತ್ತಿದೆ. ನಾಲ್ಕನೆಯ ಘಟ್ಟ ಶೂದ್ರಶಕ್ತಿಯಾಗುತ್ತದೆ. 

ನಾಗರಿಕತೆ ಎಂಬ ರೋಗವಿರುವ ತನಕ ದಾರಿದ್ರ್ಯವಿದ್ದೇ ಇರಬೇಕು. ಅದರ ಉಪಶಮನದ ಆವಶ್ಯಕತೆಯೂ ಇರಬೇಕು.

ಕಲ್ಲೆದೆಯ ಸಾಲಿಗರ ಭಾರದಲ್ಲಿ ಪಾಶ್ಚಾತ್ಯ ನರಳುತ್ತಿದೆ. ಪುರೋಹಿತರ ಉಪಟಳದಲ್ಲಿ ಪೌರಾತ್ಯ ನರಳುತ್ತಿದೆ.

ಸ್ತ್ರೀಯರ ಸ್ಥಿತಿ ಮೇಲಾಗುವ ತನಕ ಪ್ರಪಂಚ ಉದ್ಧಾರವಾಗುವುದಕ್ಕೆ ಮಾರ್ಗವಿಲ್ಲ. ಒಂದೇ ರೆಕ್ಕೆಯ ಸಹಾಯದಿಂದ ಹಾರಲು ಹಕ್ಕಿಗೆ ಅಸಾಧ್ಯ.

ಸ್ತ್ರೀಯರು ತಮ್ಮ ಸಮಸ್ಯೆಗಳನ್ನು ತಾವೇ ಬಗೆಹರಿಸಿಕೊಳ್ಳುವ ಸ್ಥಿತಿಗೆ ಬರಬೇಕು. ಉಳಿದವರಿಗೆ ಇದು ಸಾಧ್ಯವಿಲ್ಲ. ಅವರು ಇದನ್ನು ಸಾಧಿಸುವುದಕ್ಕೆ ಪ್ರಪಂಚದಲ್ಲಿ ಇತರರಿಗೆ ಇರುವಷ್ಟೇ ಸಾಮರ್ಥ್ಯವಿದೆ.

ಯಾವ ಜನಾಂಗ ಸೀತೆಯ ಶೀಲವನ್ನು ಸೃಷ್ಟಿಸಿತೋ ಅದು ಆ ಶೀಲವನ್ನು ಕೇವಲ ಕನಸು ಕಂಡಿದ್ದರೂ, ಪ್ರಪಂಚದಲ್ಲಿ ಮತ್ತೆಲ್ಲೂಇಲ್ಲದ ನಾರೀಗೌರವ ಅದರಲ್ಲಿರುವುದು.

ಯಾರು ಇತರರನ್ನು ಆಶ್ರಯಿಸುವರೋ ಅವರು ಸತ್ಯನಾರಾಯಣನನ್ನು ಸೇವಿಸಲಾರರು.

ಜೀವನವೊಂದು ಅವಿರಳ ಹೋರಾಟ. ನಮ್ಮ ಸುಂದರ ಭ್ರಾಂತಿ ಕನಸು ಗಳ ಸಿಡಿದಾಟ. ಜೀವನದ ರಹಸ್ಯ, ಭೋಗವಲ್ಲ; ಅನುಭವದ ಮೂಲಕ ಬುದ್ಧಿ ಕಲಿಯುವುದು. ಆದರೆ ಅಯ್ಯೋ! ನಾವು ನಿಜವಾಗಿ ಕಲಿಯುವುದಕ್ಕೆ ಪ್ರಾರಂ ಭಿಸಿದಾಗ ಮೃತ್ಯು ಕರೆ ಬರುವುದು.

ಪ್ರಪಂಚದಲ್ಲಿ ಶುಭದ ದಾರಿ ಕಡಿದು, ಕಾಡುಮೇಡುಗಳಿಂದ ತುಂಬಿದೆ. ಇಷ್ಟು ಜನ ಗುರಿ ಸೇರುವುದೇ ಆಶ್ಚರ್ಯ; ಹಿಂದೆ ಬೀಳುವುದಲ್ಲ, ಸಾವಿರಾರು ವೇಳೆ ಎಡವಿ ನಮ್ಮ ಶೀಲವನ್ನು ತಿದ್ದಬೇಕಾಗಿದೆ.

ನಾವು ಕಲಿಯಬೇಕಾದ ಮೊದಲನೆಯ ಬುದ್ಧಿವಾದವೆ ಇದು: ಇದನ್ನು ಶಪಥಮಾಡಿ. ಹೊರಗಿರುವುದನ್ನು ದೂರಬೇಡಿ, ಇತರರನ್ನು ನಿಂದಿಸಬೇಡಿ. ಧೀರರಾಗಿ ಎದ್ದು ನಿಲ್ಲಿ. ಜವಾಬ್ದಾರಿಯನ್ನು ನೀವೇ ವಹಿಸಿಕೊಳ್ಳಿ. ನಿತ್ಯ ಸತ್ಯ ಆಗ ವ್ಯಕ್ತವಾಗುವುದು. ಮೊದಲು ನಿಮ್ಮನ್ನು ಸ್ವಾಧೀನಕ್ಕೆ ತೆಗೆದುಕೊಳ್ಳಿ.

ಈ ಜೀವನದ ಮಹಾ ಉಪಕಾರವೇ ಹೋರಾಟ. ನಾವು ಇದರ ಮೂಲಕ ಸಾಗಿ ಹೋಗಬೇಕಾಗಿದೆ. ಸ್ವರ್ಗಕ್ಕೆ ಏನಾದರೂ ಹಾದಿ ಇದ್ದರೆ ಅದು ನರಕದ ಮೂಲಕ. ನರಕದಿಂದ ಸ್ವರ್ಗಕ್ಕೆ ದಾರಿ ಯಾವಾಗಲೂ. ಜೀವನದ ದಾರಿಯಲ್ಲಿ ಹಲವಾರು ಘಟನೆಗಳೊಂದಿಗೆ ಹೋರಾಡಿ. ಸಾವಿರಾರು ವೇಳೆ ಮೃತ್ಯುವನ್ನು ಎದುರಿಸಿ, ಆದರೂ ಎದೆಗೆಡದೆ ಮುಂದೆ ಮುಂದಕ್ಕೆ ಹೋಗಲು ಪುನಃ ಪುನಃ ಹೋರಾಡಿದಾಗ, ಆತ್ಮ ಮಹಾ ಪರಾಕ್ರಮಶಾಲಿಯಾಗಿ ಹೊರಗೆ ಬಂದು ತಾನು ಹೋರಾಡುತ್ತಿರುವ ಆದರ್ಶವನ್ನು ನೋಡಿ ನಗುವುದು. ಆಗ ಆದರ್ಶ ಕ್ಕಿಂತ ತಾನು ಎಷ್ಟು ಮಹಾಮಹಿಮನೆಂಬುದು ಅರಿವಾಗುವುದು. ನನ್ನಾತ್ಮನೇ ಗುರಿ ಮತ್ತಾವುದೂ ಆಲ್ಲ. ತನ್ನ ಆತ್ಮನೊಂದಿಗೆ ಹೋಲಿಸುವುದಕ್ಕೆ ಮತ್ತಾ ವುದು ಇರುವುದು?

