\sethyphenation{kannada}{
ಅಂಕು-ರ-ವಿ-ರು-ವುದು
ಅಂಗ-ಡಿ-ಗಳಿಂದ
ಅಂಗು-ಷ್ಠ-ದಿಂ-ದಲೂ
ಅಂಜಿಕೆ
ಅಂಜಿ-ಕೆಗೆ
ಅಂಜಿ-ಕೆಯ
ಅಂಜಿ-ಸ-ಕೂ-ಡದು
ಅಂಜು-ವರು
ಅಂತ-ರಂ-ಗ-ದಲ್ಲಿ
ಅಂತಹ
ಅಂತ-ಹ-ವನೇ
ಅಂತ-ಹ-ವರ
ಅಂತ್ಯ-ಜರು
ಅಂದ-ವಾಗಿ
ಅಂದ-ವಾದ
ಅಂದು
ಅಂಧ-ಕಾ-ರ-ದಲ್ಲಿ
ಅಂಶ-ಗ-ಳಿವೆ
ಅಕ್ಷ-ಯ-ವೀರ್ಯ
ಅಖಿಲ
ಅಗಾಧ
ಅಗಾ-ಧ-ಶಕ್ತಿ
ಅಗ್ನಿ
ಅಚ-ಲ-ಪ್ರೇ-ಮವೇ
ಅಚ-ಲ-ರಾಗಿ
ಅಚ್ಚ-ಳಿ-ಯದ
ಅಜೀರ್ಣ
ಅಜ್ಞಾನ
ಅಜ್ಞಾನ-ದಲ್ಲಿ
ಅಜ್ಞಾನಿ
ಅಜ್ಞಾನಿ-ಗ-ಳಿಗೆ
ಅಜ್ಞಾನಿ-ಗಳು
ಅಟ್ಟ-ಹಾ-ಸ-ವಿ-ದ್ದರೂ
ಅಡ
ಅಡ-ಗೂ-ಲ-ಜ್ಜಿಯ
ಅಡ-ಚ-ಣೆಗೆ
ಅಡಿಗೆ
ಅಡಿ-ಗೆ-ಮ-ನೆ-ಯ-ಲ್ಲಿದೆ
ಅಡಿ-ಗೆಯ
ಅಡಿ-ಯಾ-ಳಾ-ಗು-ವು-ದಲ್ಲ
ಅಡ್ಡಿ
ಅಣಿ-ಯಾ-ಗು-ವುವು
ಅತಿ
ಅತಿ-ಮು-ಖ್ಯ-ವಾಗಿ
ಅತೀ-ತದ
ಅತ್ತಿ-ರು-ವೆವು
ಅತ್ಯಂತ
ಅತ್ಯ-ದ್ಭುತ
ಅತ್ಯ-ದ್ಭು-ತ-ವಾದ
ಅತ್ಯ-ಮೂಲ್ಯ
ಅತ್ಯಲ್ಪ
ಅತ್ಯಾ-ವ-ಶ್ಯಕ
ಅತ್ಯಾ-ವ-ಶ್ಯ-ಕ-ವಾಗಿ
ಅತ್ಯು-ತ್ಕಟ
ಅತ್ಯು-ತ್ತಮ
ಅತ್ಯು-ತ್ತ-ಮ-ವನ್ನು
ಅಥವಾ
ಅದ-ಕ್ಕಾಗಿ
ಅದ-ಕ್ಕಿಂತ
ಅದಕ್ಕೆ
ಅದನ್ನು
ಅದ-ನ್ನೆಲ್ಲ
ಅದರ
ಅದ-ರಂತೆ
ಅದ-ರಂ-ತೆಯೇ
ಅದ-ರಲ್ಲಿ
ಅದ-ರ-ಲ್ಲಿ-ರು-ವುದು
ಅದ-ರಿಂದ
ಅದ-ರೇ-ನಂತೆ
ಅದ-ರೊಂ-ದಿಗೆ
ಅದ-ರೊ-ಳ-ಗಡೆ
ಅದು
ಅದು-ವರೆ-ವಿಗೂ
ಅದೂ
ಅದೃಷ್ಟ
ಅದೃ-ಷ್ಟ-ದಲ್ಲಿ
ಅದೃ-ಷ್ಟ-ವಿ-ದ್ದರೆ
ಅದೇ
ಅದೊಂ-ದನ್ನೇ
ಅದೋ
ಅದ್ಭುತ
ಅದ್ಭು-ತ-ಶಕ್ತಿ
ಅಧಃ-ಪಾ-ತಾ-ಳಕ್ಕೆ
ಅಧಿ
ಅಧಿ-ಕಾರ
ಅಧಿ-ಕಾ-ರಿ-ಗ-ಳಾ-ಗು-ತ್ತೀರಿ
ಅಧೀ-ರ-ರಾ-ಗ-ಬೇಡಿ
ಅಧೋ-ಗ-ತಿಗೆ
ಅಧ್ಯ-ಕ್ಷರು
ಅಧ್ಯಾ-ತ್ಮಿ-ಕತೆ
ಅನಂತ
ಅನಂ-ತ-ಜೀ-ವನ
ಅನಂ-ತ-ಜ್ಞಾನ
ಅನಂ-ತ-ಶಕ್ತಿ
ಅನಂ-ತ-ಶ-ಕ್ತಿ-ಯನ್ನು
ಅನ-ವ-ರತ
ಅನ-ವ-ರ-ತವೂ
ಅನಾಥ
ಅನಾ-ವ-ಶ್ಯ-ಕ-ವಾಗಿ
ಅನಾ-ಸ-ಕ್ತ-ನಾಗು
ಅನಾ-ಸಕ್ತಿ
ಅನಾ-ಸ-ಕ್ತಿಯ
ಅನು-ಕಂಪ
ಅನು-ಕ-ರಿ-ಸ-ಬೇ-ಕೆಂಬ
ಅನು-ಕೂಲ
ಅನು-ಗಾ-ಲವೂ
ಅನು-ಭ-ವದ
ಅನು-ಭ-ವಿ-ಸ-ಲಾರ
ಅನು-ಭ-ವಿ-ಸುವ
ಅನು-ಮಾ-ನ-ಪ-ಡ-ಲಿ-ಲ್ಲವೋ
ಅನು-ರ-ಣಿ-ತ-ವಾ-ಗು-ವುದು
ಅನು-ಷ್ಠಾನ
ಅನು-ಷ್ಠಾ-ನಕ್ಕೆ
ಅನು-ಸ-ರಿಸಿ
ಅನು-ಸ-ರಿ-ಸಿದ್ದು
ಅನು-ಸ-ರಿ-ಸಿಯೇ
ಅನು-ಸಾ-ರ-ವಾದ
ಅನೇಕ
ಅನೈ-ಕ್ಯತೆ
ಅನ್ನ
ಅನ್ನ-ವ-ನ್ನೆಲ್ಲ
ಅನ್ವೇ-ಷಣೆ
ಅಪ-ಕೀರ್ತಿ
ಅಪ-ರಾ-ಧ-ಗಳನ್ನು
ಅಪ-ರಿ-ಚಿ-ತರು
ಅಪ-ರೂಪ
ಅಪಾ-ಯ-ವಿದೆ
ಅಪಾ-ಯವೂ
ಅಪಾರ
ಅಪ್ಪಣೆ
ಅಪ್ರ-ತಿ-ಹ-ತ-ವಾಗಿ
ಅಭಾವ
ಅಭಾ-ವ-ವನ್ನು
ಅಭಿ-ಮಾ-ನ-ಪೂ-ರ್ಣ-ವಾದ
ಅಭಿ-ವೃ-ದ್ಧಿ-ಯಾ-ಗು-ವುದು
ಅಭೀಃ
ಅಭೇದ್ಯ
ಅಭ್ಯಾಸ
ಅಭ್ಯಾ-ಸ-ಮಾಡಿ
ಅಮಿತ
ಅಮೃತ
ಅಮೃ-ತತ್ತ್ವ
ಅಮೃ-ತ-ಪಾ-ನ-ಕ್ಕಾಗಿ
ಅಮೃ-ತ-ಪು-ತ್ರ-ರಿರಾ
ಅಯುಕ್ತ
ಅಯ್ಯೋ
ಅರ-ಗಿಳಿ-ಯಂತೆ
ಅರ-ಣ್ಯಾ-ರಣ್ಯ
ಅರ-ಸ-ಬ-ಯ-ಸು-ವೆವು
ಅರ-ಸ-ಬೇಡಿ
ಅರಿ-ಯಿರಿ
ಅರಿ-ವನ್ನು
ಅರಿ-ವಾ-ಗು-ವುದು
ಅರ್ಥ-ಮಾ-ಡಿ-ಕೊ-ಳ್ಳ-ಬ-ಲ್ಲಿರಿ
ಅರ್ಥ-ವನ್ನು
ಅರ್ಥ-ವಿ-ಲ್ಲದ
ಅರ್ಪಿಸಿ
ಅರ್ಪಿ-ಸು-ವೆ-ನೆಂದು
ಅರ್ಹನು
ಅರ್ಹರಾ
ಅಲಭ್ಯ
ಅಲು-ಗಿ-ಸಲು
ಅಲೆ-ದಾ-ಡು-ವರು
ಅಲೆಯ
ಅಲೆಯೂ
ಅಲ್ಪ-ಪ-ಕ್ಷ-ದ-ವ-ರಾ-ದರೂ
ಅಲ್ಲ
ಅಲ್ಲದೆ
ಅಲ್ಲ-ವೆಂದು
ಅಲ್ಲಿ
ಅಲ್ಲಿಗೆ
ಅಲ್ಲಿಗೇ
ಅಲ್ಲಿಯ
ಅಲ್ಲಿ-ಯ-ವ-ರೆಗೆ
ಅಲ್ಲಿ-ಯ-ವರೆ-ವಿಗೂ
ಅಲ್ಲಿ-ರುವ
ಅಳ-ಕೂ-ಡದು
ಅಳು
ಅಳು-ವಂತೆ
ಅವ
ಅವ-ಕಾಶ
ಅವನ
ಅವ-ನತಿ
ಅವ-ನ-ತಿಗೆ
ಅವ-ನ-ತಿ-ಯನ್ನು
ಅವ-ನನ್ನು
ಅವ-ನಿಗೆ
ಅವ-ನಿ-ರುವ
ಅವ-ನಿ-ಲ್ಲ-ದಿ-ರುವ
ಅವನು
ಅವ-ನೆ-ದು-ರಿಗೆ
ಅವ-ನೊಬ್ಬ
ಅವರ
ಅವ-ರನ್ನು
ಅವ-ರನ್ನೇ
ಅವ-ರಲ್ಲಿ
ಅವ-ರ-ಲ್ಲಿದೆ
ಅವ-ರಾ-ತ್ಮದ
ಅವ-ರಿಂದ
ಅವ-ರಿ-ಗಾಗಿ
ಅವ-ರಿಗೂ
ಅವ-ರಿಗೆ
ಅವ-ರಿದ್ದ
ಅವರು
ಅವರೂ
ಅವರೆ-ದು-ರಿಗೆ
ಅವರೇ
ಅವ-ಶ್ಯ-ಕ-ವಾದ
ಅವ-ಶ್ಯವೇ
ಅವ-ಸರ
ಅವ-ಸ-ರ-ದಿಂದ
ಅವಿ-ನಾ-ಶ-ವಾ-ಗಿವೆ
ಅವಿ-ರಳ
ಅವಿ-ವೇ-ಕಿ-ಗಳು
ಅವು
ಅವು-ಗಳ
ಅವು-ಗಳನ್ನು
ಅವೂ
ಅವೇ
ಅಶ-ಕ್ತ-ರಾ-ದರೆ
ಅಷ್ಟೂ
ಅಷ್ಟೆ
ಅಸಂ-ಬದ್ಧ
ಅಸ-ಡ್ಡೆ-ಯಿಂದ
ಅಸ-ತ್ಯ-ಕ್ಕಿಂ-ತಲೂ
ಅಸ-ತ್ಯ-ವನ್ನು
ಅಸ-ತ್ಯ-ವೆಂದು
ಅಸ-ದ-ಳ-ವಾದ
ಅಸಹ್ಯ
ಅಸಾಧ್ಯ
ಅಸಾ-ಧ್ಯ-ವಿದು
ಅಸೂಯೆ
ಅಸೂ-ಯೆ-ಪ-ಡ-ಬೇಡಿ
ಅಸೂ-ಯೆ-ಯನ್ನು
ಅಸ್ಥಿ-ಪಂ-ಜ-ರ-ಗ-ಳಿರಾ
ಅಸ್ಥಿ-ಮಾಂ-ಸ-ಗ-ಳಿಗೆ
ಅಹಂ
ಅಹಂ-ಕಾ-ರದ
ಅಹಂ-ಕಾ-ರ-ಭಾ-ವ-ನೆ-ಯನ್ನೂ
ಅಹಂ-ಕಾ-ರ-ವನ್ನು
ಅಹಂ-ಕಾ-ರ-ವ-ನ್ನೆಲ್ಲ
ಅಹಂ-ಕಾ-ರ-ವಿ-ದೆ-ಯೆಲ್ಲ
ಆ
ಆಂತ-ರಿಕ
ಆಂತ-ರ್ಯ-ದ-ಲ್ಲಿ-ರುವ
ಆಕಾಂಕ್ಷೆ
ಆಕಾಂ-ಕ್ಷೆ-ಗ-ಳಿಗೆ
ಆಕಾ-ಶದ
ಆಕೆ
ಆಕೆಯ
ಆಕೆ-ಯನ್ನು
ಆಕ್ರ-ಮಿ-ಸು-ತ್ತಿದೆ
ಆಗ
ಆಗಲೂ
ಆಗಲೆ
ಆಗಲೇ
ಆಗ-ಸ-ದಷ್ಟು
ಆಗಿ-ರ-ಬ-ಹುದು
ಆಗಿ-ಹೋ-ಯಿತು
ಆಗು-ವು-ದಕ್ಕೆ
ಆಗು-ವು-ದಿಲ್ಲ
ಆಗು-ವು-ದೆಂದು
ಆಗು-ವುದೋ
ಆಚಾರ
ಆಚಾ-ರ-ಗಳಲ್ಲಿ
ಆಚಾ-ರ-ವನ್ನು
ಆಚೆಗೆ
ಆಜ್ಞಾ-ಪಿ-ಸಿ-ದರೆ
ಆಜ್ಞೆ
ಆಜ್ಞೆ-ಯನ್ನು
ಆಡ-ಲಿಲ್ಲ
ಆಡಿ-ಸು-ವು-ದಕ್ಕೇ
ಆಣ-ತಿ-ಯನ್ನು
ಆತಂ-ಕ-ಗಳ
ಆತಂ-ಕ-ಗಳಿಂದ
ಆತನ
ಆತ-ನನ್ನು
ಆತ-ನನ್ನೆ
ಆತ-ನಿ-ಗಾಗಿ
ಆತ-ನಿಗೆ
ಆತಿಥ್ಯ
ಆತ್ಮ
ಆತ್ಮ-ಗೌ-ರ-ವಕ್ಕೆ
ಆತ್ಮನ
ಆತ್ಮ-ನಲ್ಲಿ
ಆತ್ಮ-ನಿಂದೆ
ಆತ್ಮ-ನಿಂ-ದೆ-ಯನ್ನು
ಆತ್ಮ-ನಿಗೆ
ಆತ್ಮ-ನೊಂ-ದಿಗೆ
ಆತ್ಮ-ವ-ನ್ನಾ-ಗಲಿ
ಆತ್ಮವೇ
ಆತ್ಮ-ಸಾ-ಕ್ಷಾ-ತ್ಕಾರ
ಆದ-ಕಾ-ರಣ
ಆದ-ರಣೆ
ಆದ-ರಿಂದ
ಆದರೂ
ಆದರೆ
ಆದ-ರೇನು
ಆದರ್ಶ
ಆದ-ರ್ಶಕ್ಕೆ
ಆದ-ರ್ಶ-ಗಳೂ
ಆದ-ರ್ಶದ
ಆದ-ರ್ಶ-ದಲ್ಲಿ
ಆದ-ರ್ಶ-ವನ್ನು
ಆದ-ರ್ಶ-ವಾ-ಗಲಿ
ಆದ-ರ್ಶ-ವಾ-ದರೂ
ಆದ-ರ್ಶ-ವಿ-ಲ್ಲ-ದ-ವನು
ಆದ-ರ್ಶ-ವುಳ್ಳ
ಆದಷ್ಟು
ಆಧು-ನಿಕ
ಆಧ್ಯಾತ್ಮ
ಆಧ್ಯಾ-ತ್ಮಿಕ
ಆಧ್ಯಾ-ತ್ಮಿ-ಕತೆ
ಆಧ್ಯಾ-ತ್ಮಿ-ಕ-ತೆ-ಯೆಲ್ಲ
ಆಧ್ಯಾ-ತ್ಮಿ-ಕ-ದೊಂ-ದಿಗೆ
ಆಧ್ಯಾ-ತ್ಮಿ-ಕ-ವಾ-ಗಾ-ಗಲಿ
ಆಮ-ರ-ಣಾಂತ
ಆಮೂ-ಲಾ-ಗ್ರ-ವಾಗಿ
ಆರಂ-ಭ-ವಾ-ಯಿತು
ಆರಾ-ಧನೆ
ಆರಾ-ಧ-ನೆಯ
ಆರಾ-ಧಿ-ಸದೆ
ಆರಾ-ಧಿ-ಸು-ತ್ತಿ-ರು-ವ-ವ-ರಿಂ-ದಲೇ
ಆರಿ-ಸದೆ
ಆರು
ಆರೇಳು
ಆರೋಗ್ಯ
ಆರ್ಥಿಕ
ಆರ್ಯ
ಆರ್ಯ-ದೇ-ವ-ದೇ-ವ-ತೆ-ಗ-ಳೆ-ಲ್ಲರೂ
ಆರ್ಯ-ಮ-ಹರ್ಷಿ
ಆರ್ಯ-ಮಾ-ತೆಯ
ಆರ್ಯರ
ಆರ್ಯ-ರೆಮ್ಮ
ಆರ್ಯಾ-ವ-ರ್ತದ
ಆರ್ಯಾ-ವ-ರ್ತ-ದಲ್ಲಿ
ಆಲಂ-ಗಿ-ಸಿ-ಕೊ-ಳ್ಳು-ವು-ದಕ್ಕೆ
ಆಲಿಂ-ಗ-ನೆ-ಯಲ್ಲಿ
ಆಲಿ-ಸು-ತ್ತಿರಿ
ಆಲೋ-ಚನೆ
ಆಲೋ-ಚ-ನೆ-ಯನ್ನು
ಆಲೋ-ಚ-ನೆ-ಯಲ್ಲಿ
ಆಲೋ-ಚಿ-ಸ-ಬೇಡಿ
ಆಲೋ-ಚಿಸಿ
ಆಲೋ-ಚಿ-ಸು-ತ್ತಿ-ದ್ದರೆ
ಆಲೋ-ಚಿ-ಸು-ತ್ತಿ-ರಲಿ
ಆಲೋ-ಚಿ-ಸು-ತ್ತಿ-ರು-ವರೋ
ಆಲ್ಲ
ಆಳಂತೆ
ಆಳ-ವಾದ
ಆಳಾ-ಗಿ-ದ್ದರೆ
ಆಳಿನ
ಆಳಿ-ನ-ವ-ರೊಂ-ದಿಗೆ
ಆಳು-ಕಾಳು
ಆಳು-ತ್ತಿದೆ
ಆಳು-ವರು
ಆಳ್ವಿಕೆ
ಆವ-ರಿ-ಸಿ-ರುವ
ಆವ-ರಿ-ಸಿ-ರು-ವುದು
ಆವ-ಶ್ಯಕ
ಆವ-ಶ್ಯ-ಕತೆ
ಆವ-ಶ್ಯ-ಕ-ತೆಯೂ
ಆವ-ಶ್ಯ-ಕ-ವಾಗಿ
ಆವ-ಶ್ಯ-ಕ-ವಾ-ದರೆ
ಆವಿ-ರ್ಭ-ವಿ-ಸ-ಲಿ-ರುವ
ಆವಿ-ರ್ಭ-ವಿ-ಸುವ
ಆವೃ-ತ-ವಾ-ಗಿ-ರು-ವುದು
ಆಶಾ-ಭಾ-ವ-ದಿಂದ
ಆಶಿಷ್ಟ
ಆಶಿ-ಸ-ಬೇಡಿ
ಆಶಿ-ಸು-ವಾಗ
ಆಶಿ-ಸು-ವುದು
ಆಶ್ಚರ್ಯ
ಆಶ್ರಮ
ಆಶ್ರ-ಯಿಸಿ
ಆಶ್ರ-ಯಿ-ಸು-ವರೋ
ಆಸ-ಕ್ತಿ-ಯಿಂದ
ಆಸೆ
ಆಸೆಯೇ
ಆಸ್ತಿ
ಆಸ್ಪತ್ರೆ
ಆಸ್ಪ-ತ್ರೆ-ಯ-ಲ್ಲಿ-ರ-ಬೇ-ಕಾ-ಗು-ವುದು
ಆಸ್ಫೋ-ಟಿ-ಸದೆ
ಆಹಾರ
ಆಹಾ-ರ-ವಿ-ಲ್ಲದೆ
ಆಹ್ವಾನ
ಇಂತಹ
ಇಂತೆಂದು
ಇಂದಿನ
ಇಂದು
ಇಂದ್ರಿಯ
ಇಂದ್ರಿ-ಯ-ನಿ-ಗ್ರ-ಹ-ವನ್ನು
ಇಚ್ಛಾ
ಇಚ್ಛಾ-ನು-ಸಾರ
ಇಚ್ಛಾ-ಪ್ರ-ವಾಹ
ಇಚ್ಛಾ-ಮಾತ್ರ
ಇಚ್ಛಾ-ಶಕ್ತಿ
ಇಚ್ಛಿ-ಸು-ವು-ದಿಲ್ಲ
ಇಚ್ಛಿ-ಸು-ವೆನು
ಇಚ್ಛೆಗೆ
ಇಚ್ಛೆ-ಯನ್ನು
ಇಡೀ
ಇಡುವ
ಇತರ
ಇತ-ರರ
ಇತ-ರ-ರನ್ನು
ಇತ-ರ-ರ-ಲ್ಲಿ-ರುವ
ಇತ-ರ-ರಿ-ಗಾಗಿ
ಇತ-ರ-ರಿ-ಗಿಂತ
ಇತ-ರ-ರಿಗೆ
ಇತಿ-ಹಾ-ಸ-ಕಾ-ಲದ
ಇತಿ-ಹಾ-ಸ-ದಲ್ಲೂ
ಇದ-ಕ್ಕಿಂತ
ಇದ-ಕ್ಕಿಂ-ತಲೂ
ಇದಕ್ಕೆ
ಇದನ್ನು
ಇದನ್ನೇ
ಇದರ
ಇದ-ರಿಂದ
ಇದು
ಇದು-ವ-ರೆಗೆ
ಇದು-ವರೆ-ವಿಗೂ
ಇದೂ
ಇದೆ
ಇದೆಯೆ
ಇದೆ-ಯೆ-ಇ-ದ್ದರೆ
ಇದೆಯೋ
ಇದೇ
ಇದೊಂದು
ಇದೋ
ಇದ್ದ
ಇದ್ದಂಥಾ
ಇದ್ದರೆ
ಇದ್ದ-ರೇ-ನಂತೆ
ಇನ್ನಾಕೆ
ಇನ್ನು
ಇನ್ನೂ
ಇನ್ನೆಲ್ಲಿ
ಇನ್ನೇನು
ಇನ್ನೊಂದು
ಇಬ್ಬ-ನಿ-ಯಂತೆ
ಇರ-ಕೂ-ಡದು
ಇರ-ಬ-ಲ್ಲರು
ಇರ-ಬೇಕು
ಇರಲಿ
ಇರ-ಲಿಲ್ಲ
ಇರು
ಇರುಳು
ಇರುವ
ಇರು-ವರು
ಇರು-ವ-ವ-ರೆಗೂ
ಇರು-ವಷ್ಟೇ
ಇರು-ವಾಗ
ಇರು-ವು-ದ-ಕ್ಕಿಂತ
ಇರು-ವು-ದಿಲ್ಲ
ಇರು-ವುದು
ಇರು-ವುದೋ
ಇರು-ವುವೋ
ಇರು-ವೆಗೂ
ಇರು-ವೆ-ಯಾ-ದರೂ
ಇಲ್ಲ
ಇಲ್ಲದ
ಇಲ್ಲ-ದ-ವ-ನನ್ನು
ಇಲ್ಲದೆ
ಇಲ್ಲದೇ
ಇಲ್ಲ-ವೆಂದು
ಇಲ್ಲ-ವೆಂ-ಬು-ದನ್ನು
ಇಲ್ಲವೋ
ಇಲ್ಲಿಗೆ
ಇಲ್ಲಿಯೇ
ಇಲ್ಲಿ-ರು-ವುದು
ಇಳಿದು
ಇಳಿ-ಯಿತು
ಇಳಿ-ಯಿರಿ
ಇಳಿ-ಸ-ಲಾ-ಗು-ವು-ದಿಲ್ಲ
ಇವ-ಕ್ಕೇನೂ
ಇವನ್ನು
ಇವ-ನ್ನೆಲ್ಲ
ಇವ-ರೆಡೂ
ಇವರೆ-ಲ್ಲರೂ
ಇವು
ಇವು-ಗಳನ್ನು
ಇವು-ಗಳಿಂದ
ಇವು-ಗಳು
ಇವು-ಗ-ಳೊ-ಡೆ-ಯ-ನನ್ನು
ಇವೆಲ್ಲ
ಇವೇ
ಇಷ್ಟ-ಮೂರ್ತಿ
ಇಷ್ಟು
ಇಹ-ಪರ
ಈ
ಈಗ
ಈಡು-ಮಾ-ಡು-ವುದು
ಈಡೇ-ರಿ-ಸಿ-ಕೊ-ಳ್ಳ-ಬಲ್ಲ
ಈಡೇ-ರು-ತ್ತವೆ
ಈಶ್ವ-ರನ
ಈಶ್ವ-ರ-ಲಾ-ಭಕ್ಕೆ
ಉಕ್ಕಿ
ಉಕ್ಕಿನ
ಉಕ್ಕಿ-ನಂ-ತಹ
ಉಗು-ಳು-ವಂತೆ
ಉಚ್ಚ-ಕಂ-ಠ-ದಿಂದ
ಉಚ್ಚ-ಕುಲ
ಉಚ್ಚ-ವ-ರ್ಗ-ದ-ವರ
ಉಚ್ಚ-ವ-ರ್ಗ-ದ-ವ-ರೆಂದು
ಉಣ್ಣು-ತ್ತಾನೆ
ಉತ್ತಮ
ಉತ್ತ-ಮ-ಗೊ-ಳಿ-ಸ-ಬೇಕು
ಉತ್ತ-ಮ-ವಾ-ಗಿ-ರು-ವುದನ್ನು
ಉತ್ತ-ರ-ದಿಂದ
ಉತ್ತ-ರ-ವಿ-ದು-ನನ್ನ
ಉತ್ತೀ-ರ್ಣ-ನಾಗಿ
ಉತ್ತೇ-ಜ-ನ-ಕಾರಿ
ಉತ್ಪ-ತ್ತಿ-ಮಾ-ಡಿ-ಕೊಂಡು
ಉತ್ಪ-ನ್ನ-ವಾ-ಗುವ
ಉತ್ಸಾಹ
ಉತ್ಸಾ-ಹ-ದಿಂದ
ಉದಾತ್ತ
ಉದಾರ
ಉದಾ-ಹ-ರಣೆ
ಉದಿ-ಸ-ಲಾ-ರದು
ಉದ್ದೇಶ
ಉದ್ದೇ-ಶಕ್ಕೆ
ಉದ್ದೇ-ಶ-ಗಳೂ
ಉದ್ಧಾರ
ಉದ್ಧಾ-ರ-ವಾ-ಗು-ವು-ದಕ್ಕೆ
ಉದ್ಧಾ-ರ-ವಾ-ಗು-ವುದು
ಉದ್ಯ-ಮ-ಗಳೂ
ಉದ್ಯಾನ
ಉದ್ರೇ-ಕಿ-ಸು-ತ್ತಿದೆ
ಉದ್ವೇ-ಗ-ವ-ಶ-ರಾ-ಗು-ವು-ದಿ-ಲ್ಲವೋ
ಉಪ-ಕಾ-ರ-ಮಾ-ಡ-ಲಾ-ರಿರಿ
ಉಪ-ಕಾ-ರ-ವೆಂ-ಬು-ದನ್ನೇ
ಉಪ-ಕಾ-ರವೇ
ಉಪ-ಟ-ಳಕ್ಕೆ
ಉಪ-ಟ-ಳ-ದಲ್ಲಿ
ಉಪ-ದೇ-ಶಿ-ಸಿ-ದ್ದಾರೆ
ಉಪ-ದ್ರ-ವಕ್ಕೆ
ಉಪ-ನಿ-ಷ-ತ್ತನ್ನು
ಉಪ-ನಿ-ಷ-ತ್ತಿನ
ಉಪ-ನಿ-ಷ-ತ್ತಿ-ನಲ್ಲಿ
ಉಪ-ಯೋ-ಗಿಸಿ
ಉಪ-ವಾ-ಸ-ದಲ್ಲಿ
ಉಪ-ವಾ-ಸ-ದಿಂದ
ಉಪ-ವಾ-ಸ-ವಿ-ದ್ದರೂ
ಉಪ-ಶ-ಮ-ನದ
ಉಪಾ-ಸನೆ
ಉಬ್ಬರ
ಉಬ್ಬಿ-ದರೂ
ಉಮಾ-ನಾಥ
ಉಯ್ಯಾಲೆ
ಉರಿ-ಯು-ತ್ತಿ-ದೆಯೋ
ಉಳಿದ
ಉಳಿ-ದ-ವ-ರಿಗೆ
ಉಳಿ-ದು-ಕೊಂ-ಡಿ-ದ್ದೀರಿ
ಉಳಿ-ದು-ದನ್ನು
ಉಳಿ-ದು-ದೆಲ್ಲ
ಉಳಿ-ದು-ವೆಲ್ಲ
ಉಸಿ-ರಾಗಿ
ಉಸಿ-ರೆ-ಳೆದು
ಊಟ
ಎಂತಹ
ಎಂದರೆ
ಎಂದಾ-ದರೂ
ಎಂದಿ-ಗಿಂ-ತಲೂ
ಎಂದಿಗೂ
ಎಂದು
ಎಂದೆಂ-ದಿಗೂ
ಎಂದೆ-ನಿ-ಸಿ-ಕೊಂ-ಡಿ-ರುವ
ಎಂಬ
ಎಂಬು-ದನ್ನು
ಎಂಬುದು
ಎಚ್ಚೆತ್ತು
ಎಡ-ಗೈ-ಯಿಂದ
ಎಡ-ಗೊ-ಡ-ಬೇಡಿ
ಎಡರು
ಎಡವಿ
ಎಡೆಗೆ
ಎಡೆ-ಬಿ-ಡದ
ಎಡೆ-ಬಿ-ಡದೆ
ಎಣಿಸ
ಎತ್ತಿ-ತೋ-ರ-ಕೂ-ಡದು
ಎದು-ರಿಗೆ
ಎದು-ರಿ-ಸ-ಬಲ್ಲ
ಎದು-ರಿಸಿ
ಎದು-ರಿ-ಸಿ-ದರೂ
ಎದು-ರಿಸು
ಎದು-ರಿ-ಸು-ವ-ವರು
ಎದು-ರಿ-ಸು-ವು-ದಕ್ಕೆ
ಎದು-ರು-ಬಿ-ದ್ದರೆ
ಎದೆ
ಎದೆ-ಗೆ-ಡದೆ
ಎದ್ದು
ಎನ್ನು-ವ-ವರು
ಎನ್ನು-ವುದು
ಎನ್ನು-ವುದೇ
ಎಬ್ಬಿಸಿ
ಎರ-ಡ-ನೆ-ಯದೇ
ಎರ-ಡನೇ
ಎರಡು
ಎರಡೇ
ಎಲು-ಬಿನ
ಎಲ್ಲ
ಎಲ್ಲ-ಕ್ಕಿಂತ
ಎಲ್ಲ-ದನ್ನೂ
ಎಲ್ಲರ
ಎಲ್ಲ-ರನ್ನೂ
ಎಲ್ಲ-ರಿ-ಗಿಂತ
ಎಲ್ಲರೂ
ಎಲ್ಲ-ರೊಂ-ದಿಗೆ
ಎಲ್ಲ-ವನ್ನೂ
ಎಲ್ಲಾ
ಎಲ್ಲಿ
ಎಲ್ಲಿಗೆ
ಎಲ್ಲಿ-ಯ-ವ-ರೆಗೂ
ಎಲ್ಲಿ-ಯ-ವರೆ-ವಿಗೂ
ಎಲ್ಲಿಯೂ
ಎಳೆ
ಎಷ್ಟು
ಎಷ್ಟೇ
ಎಷ್ಟೋ
ಎಸೆ-ಯಿರಿ
ಏಕ-ಮಾತ್ರ
ಏಕಾ-ಗ್ರತೆ
ಏಕಿನ್ನೂ
ಏಕೆ
ಏಕೆಂ-ದರೆ
ಏತಕ್ಕೆ
ಏನ-ನ್ನಾ-ದರೂ
ಏನನ್ನು
ಏನನ್ನೂ
ಏನಾ-ದರೂ
ಏನಿ-ರು-ವುದೋ
ಏನು
ಏನೂ
ಏಳಿ
ಏಸು-ಕ್ರಿ-ಸ್ತನ
ಐಕ್ಯ-ವಾಗಿ
ಐದು
ಐವತ್ತು
ಐಶ್ವರ್ಯ
ಐಶ್ವ-ರ್ಯದ
ಐಶ್ವ-ರ್ಯ-ಮಂ-ಜೂ-ಷೆ-ಗಳು
ಐಶ್ವ-ರ್ಯ-ವ-ನ್ನಾ-ಗಲೀ
ಐಶ್ವ-ರ್ಯ-ವೆಲ್ಲ
ಒಂದಾದ
ಒಂದು
ಒಂದು-ನೂರು
ಒಂದೇ
ಒಗೆದು
ಒಟ್ಟಿಗೆ
ಒಟ್ಟು-ಗೂ-ಡಿಸಿ
ಒತ್ತಿ
ಒದ-ಗಿ-ದು-ದ-ರಿಂದ
ಒದ-ಗಿ-ಸು-ತ್ತಿ-ರುವ
ಒಪ್ಪಿ
ಒಪ್ಪಿ-ಕೊಂಡು
ಒಪ್ಪಿ-ಕೊ-ಳ್ಳ-ಬೇಡ
ಒಪ್ಪಿ-ಕೊ-ಳ್ಳು-ವು-ದಿಲ್ಲ
ಒಪ್ಪಿ-ಕೊ-ಳ್ಳು-ವುದು
ಒಪ್ಪಿ-ಸು-ವು-ದಕ್ಕೆ
ಒಬ್ಬ
ಒಬ್ಬನು
ಒಬ್ಬರ
ಒಬ್ಬರು
ಒಮ್ಮ-ತ-ದಿಂದ
ಒಯ್ಯು-ವುದು
ಒರೆ-ಗಲ್ಲು
ಒಲಿಯು
ಒಲುಮೆ
ಒಳ-ಗಡೆ
ಒಳಗೆ
ಒಳ-ಪ-ಡಿ-ಸು-ವಂತೆ
ಒಳ್ಳೆ
ಒಳ್ಳೆಯ
ಒಳ್ಳೆ-ಯ-ದನ್ನು
ಒಳ್ಳೆ-ಯ-ದಾ-ಗ-ಬೇ-ಕಾ-ದರೆ
ಒಳ್ಳೆ-ಯ-ದಾಗಿ
ಒಳ್ಳೆ-ಯದು
ಒಳ್ಳೆ-ಯದೋ
ಒಳ್ಳೆ-ಯವು
ಓ
ಓಡಿ-ಹೋ-ಗು-ವು-ದಕ್ಕೆ
ಓದು-ವುದೇ
ಔದಾ-ರ್ಯ-ತೆ-ಯಿಂದ
ಔಷಧಿ
ಕಂಕ-ಣ-ನಾಗು
ಕಂಠ-ಪಾ-ಠ-ಮಾ-ಡಿ-ಕೊಂಡ
ಕಂಡರೆ
ಕಂಡಿ-ದ್ದರೂ
ಕಂಡು
ಕಂಡು-ಹಿ-ಡಿದು
ಕಂದ
ಕಂದರ
ಕಂದ-ರ-ಗಳಲ್ಲಿ
ಕಂದೆ
ಕಟ್ಟಿದ
ಕಟ್ಟಿ-ದರೂ
ಕಠೋ-ಪ-ನಿ-ಷ-ತ್ತಿ-ನಲ್ಲಿ
ಕಡ-ಲಾ-ಳಕ್ಕೆ
ಕಡ-ಲಿಗೆ
ಕಡಿದು
ಕಡಿಮೆ
ಕಡಿ-ಮೆ-ಯಾ-ದರೂ
ಕಡು-ಕ-ಷ್ಟ-ಗಳ
ಕಡೆ
ಕಡೆ-ಗಿಂದು
ಕಡೆಗೆ
ಕಡೆಯೆ
ಕಣ್ಣನ್ನು
ಕಣ್ಣೀರು
ಕಣ್ಣು-ಗ-ಳಿಂ-ದಲೇ
ಕಣ್ಣು-ಗಳು
ಕಣ್ದೆ-ರೆದು
ಕತ್ತಿ
ಕತ್ತಿ-ಯಿಂದ
ಕಥೆಯ
ಕದಿ-ಯು-ವೆವು
ಕನ-ಸಿ-ನ-ಲ್ಲಾ-ದರೂ
ಕನಸು
ಕನ್ಯಾ-ಕು-ಮಾ-ರಿ-ಯ-ವ-ರೆಗೆ
ಕಪ-ಟಿ-ಯಾ-ಗ-ಬೇಡ
ಕಬ್ಬಿ-ಣದ
ಕಬ್ಬಿ-ಣ-ದಂ-ತಹ
ಕಬ್ಬಿ-ಣ-ವನ್ನು
ಕರೆ
ಕರೆ-ದಾಗ
ಕರೆ-ದಿ-ದ್ದೇವೆ
ಕರೆ-ದೊ-ಯ್ಯಲು
ಕರೆ-ದೊ-ಯ್ಯು-ವುದು
ಕರೆ-ಯು-ತ್ತಿದ್ದ
ಕರೆ-ಯು-ತ್ತೇನೆ
ಕರೆ-ಯು-ತ್ತೇವೆ
ಕರೆ-ಯು-ವಿ-ರೇನು
ಕರೆ-ಯು-ವುದು
ಕರೆ-ವುದು
ಕರೆ-ಸಿ-ಕೊ-ಳ್ಳ-ವ-ವ-ರನ್ನು
ಕರ್ತವ್ಯ
ಕರ್ತ-ವ್ಯ-ಗಳು
ಕರ್ತ-ವ್ಯ-ಪ-ರಾ-ಯ-ಣ-ತೆಯೆ
ಕರ್ತ-ವ್ಯ-ವ-ನ್ನಾ-ಗಲಿ
ಕರ್ತ-ವ್ಯ-ವನ್ನು
ಕರ್ತ-ವ್ಯವೂ
ಕರ್ಮ
ಕರ್ಮಕ್ಕೆ
ಕರ್ಮ-ದಲ್ಲಿ
ಕರ್ಮ-ವನ್ನು
ಕರ್ಮ-ವೀ-ರ-ನನ್ನು
ಕರ್ಮವೂ
ಕರ್ಮಾ-ಚ-ರಣೆ
ಕಲಿ
ಕಲಿ-ಯ-ಬೇ-ಕಾದ
ಕಲಿ-ಯಿರಿ
ಕಲಿ-ಯು-ವು-ದಕ್ಕೆ
ಕಲಿ-ಯು-ವುದು
ಕಲೆತು
ಕಲೆ-ಯಲ್ಲಿ
ಕಲ್ಪಿ-ಸು-ವನು
ಕಲ್ಯಾಣ
ಕಲ್ಲಿಗೂ
ಕಲ್ಲು
ಕಲ್ಲೆ-ದೆಯ
ಕಳು-ಹಿಸಿ
ಕಳೆದ
ಕಳೆ-ದರೆ
ಕಳೆದು
ಕಷ್ಟ-ಪಟ್ಟು
ಕಷ್ಟ-ಪ-ಡು-ವೆವು
ಕಷ್ಟ-ವನ್ನು
ಕಷ್ಟ-ವಾ-ಗು-ವುದು
ಕಸು-ಬಿನ
ಕಾಂತಿ-ಯು-ತ-ಳಾಗಿ
ಕಾಡು-ಮೇ-ಡು-ಗಳಿಂದ
ಕಾಣ-ಕೂ-ಡದು
ಕಾಣದು
ಕಾಣದೆ
ಕಾಣ-ಬೇ-ಕಾ-ಗಿದೆ
ಕಾಣಿಕೆ
ಕಾಣುವ
ಕಾಣು-ವನು
ಕಾಣು-ವಿರಿ
ಕಾಣು-ವುದು
ಕಾದಿ-ರು-ವಾಗ
ಕಾಪ-ಟ್ಯ-ದಿಂದ
ಕಾಮ
ಕಾಮ-ಕಾಂ-ಚನ
ಕಾಯ-ಬೇಡಿ
ಕಾಯು-ತ್ತಿದೆ
ಕಾಯು-ತ್ತಿ-ರು-ವಳು
ಕಾರಕ್ಕೆ
ಕಾರಣ
ಕಾರ-ಣ-ವೇನು
ಕಾರಿ
ಕಾರ್ಖಾ-ನೆ-ಗಳಿಂದ
ಕಾರ್ಮು-ಗಿ-ಲಿ-ನಿಂದ
ಕಾರ್ಯ
ಕಾರ್ಯ-ಕಾ-ರಿ-ಯಾ-ಗ-ಬೇಕು
ಕಾರ್ಯಕ್ಕೆ
ಕಾರ್ಯ-ಕ್ಷೇ-ತ್ರ-ದ-ಲ್ಲಿಯೂ
ಕಾರ್ಯ-ಗ-ಳಿಗೆ
ಕಾರ್ಯ-ತ-ತ್ಪ-ರ-ರಾಗಿ
ಕಾರ್ಯ-ದ-ಕ್ಷತೆ
ಕಾರ್ಯ-ದಷ್ಟೇ
ಕಾರ್ಯ-ನಿ-ರ್ವಾಹ
ಕಾರ್ಯ-ವನ್ನೂ
ಕಾರ್ಯ-ವೆಲ್ಲ
ಕಾರ್ಯೋ
ಕಾರ್ಯೋ-ತ್ಸಾಹ
ಕಾಲ
ಕಾಲ-ಕಾ-ಲಕ್ಕೆ
ಕಾಲದ
ಕಾಲ-ದಲ್ಲಿ
ಕಾಲ-ಮೇಲೆ
ಕಾಲ-ವನ್ನು
ಕಾಲ-ವಾದ
ಕಾಲ-ವಿ-ಳಂಬ
ಕಾಲ-ವೆಲ್ಲ
ಕಾಲೇ-ಜು-ಗಳಲ್ಲಿ
ಕಾಳಿ
ಕಾವಾಗಿ
ಕಾವ್ಯ-ದಲ್ಲಿ
ಕಾಸು
ಕಿತ್ತೊ-ಗೆ-ಯಿರಿ
ಕಿರೀ-ಟ-ವನ್ನು
ಕಿವಿ-ಗ-ಳೆ-ರಡು
ಕಿವಿಗೆ
ಕಿವಿ-ಗೊಟ್ಟು
ಕಿವಿ-ಗೊಡಿ
ಕೀಟ-ದಂತೆ
ಕೀಟವೂ
ಕೀರ್ತಿ
ಕೀರ್ತಿ-ಕಾಂತಿ
ಕೀರ್ತಿಯ
ಕೀರ್ತಿ-ಯನ್ನು
ಕೀರ್ತಿ-ಯಲ್ಲ
ಕೀರ್ತಿ-ಯಿಂದ
ಕೀರ್ತಿ-ಸಿ-ದರೂ
ಕೀಳಲ್ಲ
ಕೀಳು
ಕುಂದು
ಕುಡಿದು
ಕುಡಿ-ಯದೆ
ಕುಡಿ-ಯಲೇ
ಕುಡಿ-ಯು-ವುದೇ
ಕುದಿ-ಯು-ತ್ತಿ-ರು-ವುವೋ
ಕುದು-ರೆ-ಗಾಡಿ
ಕುರಿತು
ಕುರು-ಡ-ರಿಗೆ
ಕುರುಹು
ಕುಲ
ಕುಲ-ಘಾ-ತ-ಕರೆ
ಕುಲ-ಘಾ-ತು-ಕ-ರೆಂದು
ಕುಲ-ಸಂ-ಭೂ-ತ-ರೆಂದು
ಕುಳಿ-ತಂ-ತಿದೆ
ಕುಳಿ-ತು-ಕೊಂಡು
ಕುಳಿ-ತು-ಕೊ-ಳ್ಳ-ಬೇಡಿ
ಕುಷ್ಠ-ನಂತೆ
ಕೂಗಾಟ
ಕೂಗಿ-ಕೊ-ಳ್ಳುವ
ಕೂಡ
ಕೂಡದು
ಕೂಡಲೆ
ಕೃತ-ಜ್ಞ-ತೆ-ಯನ್ನು
ಕೃತ-ವಿದ್ಯ
ಕೃಪೆ-ಯನ್ನು
ಕೃಷಿ-ಕನ
ಕೆಂಪೆ
ಕೆಚ್ಚೆ-ದೆ-ಯ-ವ-ನನ್ನು
ಕೆಟ್ಟದೋ
ಕೆಟ್ಟ-ದ್ದ-ರೊಂ-ದಿಗೆ
ಕೆಟ್ಟದ್ದು
ಕೆಡಿ-ಸಿ-ಕೊ-ಳ್ಳ-ಬೇಡಿ
ಕೆಡಿ-ಸು-ವು-ದಕ್ಕೆ
ಕೆಲ-ವ-ರಿಗೆ
ಕೆಲ-ವಿನ್ನೂ
ಕೆಲವು
ಕೆಲಸ
ಕೆಲ-ಸಕ್ಕೆ
ಕೆಲ-ಸ-ಗಳ
ಕೆಲ-ಸ-ಗಳನ್ನು
ಕೆಲ-ಸ-ಗ-ಳನ್ನೇ
ಕೆಲ-ಸದ
ಕೆಲ-ಸ-ದಲ್ಲಿ
ಕೆಲ-ಸ-ದಿಂದ
ಕೆಲ-ಸ-ಮಾ-ಡ-ಬ-ಹುದು
ಕೆಲ-ಸ-ಮಾ-ಡ-ಬೇಕು
ಕೆಲ-ಸ-ಮಾ-ಡಲು
ಕೆಲ-ಸ-ಮಾಡಿ
ಕೆಲ-ಸ-ಮಾಡು
ಕೆಲ-ಸ-ಮಾ-ಡುವ
ಕೆಲ-ಸ-ಮಾ-ಡು-ವುದನ್ನು
ಕೆಲ-ಸ-ವ-ನ್ನಾಗಿ
ಕೆಲ-ಸ-ವನ್ನು
ಕೆಲ-ಸ-ವನ್ನೂ
ಕೆಲ-ಸ-ವಾ-ದರೆ
ಕೆಲ-ಸ-ವಿಲ್ಲ
ಕೆಲ-ಸವೂ
ಕೆಲ-ಸವೇ
ಕೆಳಗೆ
ಕೆಳ-ದ-ರ್ಜೆ-ಯಲ್ಲಿ
ಕೇಂದ್ರೀ-ಕ-ರ-ಣವೂ
ಕೇಂದ್ರೀ-ಕ-ರಿ-ಸು-ವ-ವ-ರೆಗೆ
ಕೇಳ-ಬೇ-ಕೆಂಬ
ಕೇಳ-ಬೇಡಿ
ಕೇಳಿ-ಬ-ರು-ತ್ತಿದೆ
ಕೇಳಿ-ಸಿಯೇ
ಕೇಳಿ-ಸು-ತ್ತಿ-ರುವ
ಕೇಳು-ವುದೂ
ಕೇವಲ
ಕೈಎ-ತ್ತು-ವರೋ
ಕೈಗಳನ್ನು
ಕೈಗ-ಳಿ-ರು-ವುದು
ಕೈನೀ-ಡು-ವರೋ
ಕೈಬಿಟ್ಟು
ಕೈಯಲ್ಲಿ
ಕೈಯಿಂದ
ಕೈಲಾ-ಗದ
ಕೈಹಾಕಿ
ಕೊಂಚ
ಕೊಂಡು
ಕೊಚ್ಚಿ-ಕೊ-ಳ್ಳು-ವು-ದಲ್ಲ
ಕೊಟ್ಟರೆ
ಕೊಡ
ಕೊಡ-ಬೇಕು
ಕೊಡ-ಬೇಡಿ
ಕೊಡಲು
ಕೊಡಿ
ಕೊಡು
ಕೊಡು-ತ್ತೇನೆ
ಕೊಡುವ
ಕೊಡು-ವ-ವನು
ಕೊಡುವು
ಕೊಡು-ವು-ದಕ್ಕೆ
ಕೊನೆ-ಗಾ-ಣಲಿ
ಕೊನೆ-ಗಾ-ಣಿಸಿ
ಕೊನೆ-ಗಾಣು
ಕೊನೆ-ಗಾ-ಣು-ತ್ತಿದೆ
ಕೊನೆ-ಗಾ-ಣು-ವುದು
ಕೊನೆಗೆ
ಕೊರ-ತೆ-ಗ-ಳನ್ನೇ
ಕೊರ-ತೆಯೇ
ಕೊಲ್ಲು-ವುದು
ಕೊಲ್ಲು-ವೆವು
ಕೊಳೆತು
ಕೊಳ್ಳದ
ಕೊಳ್ಳ-ಬ-ಲ್ಲಿರಿ
ಕೊಳ್ಳ-ಬೇಡಿ
ಕೊಳ್ಳ-ಲಾ-ರದು
ಕೋಟ-ಲೆಗೆ
ಕೋಟಿ-ಪಾಲು
ಕೋಟ್ಯಂ-ತರ
ಕೋರಿ-ಕೆ-ಗಳು
ಕೋರೈ-ಸುವ
ಕ್ಕಲ್ಲ
ಕ್ಕಾಗಿ
ಕ್ಕಿಂತ
ಕ್ರಮೇಣ
ಕ್ರಿಮಿ-ಗ-ಳಂತೆ
ಕ್ರಿಮಿ-ಗ-ಳೆಂದು
ಕ್ರಿಮಿ-ಯಲ್ಲಿ
ಕ್ರೂರ
ಕ್ರೈಸ್ತ-ಪಂ-ಗ-ಡದ
ಕ್ರೌರ್ಯ
ಕ್ಲೈಬ್ಯ-ವನ್ನು
ಕ್ಷಣ
ಕ್ಷಣ-ದಲ್ಲಿ
ಕ್ಷಣವೂ
ಕ್ಷಣವೇ
ಕ್ಷಣಿಕ
ಕ್ಷಮಿಸಿ
ಕ್ಷಯಿ-ಸು-ವುವು
ಕ್ಷಾತ್ರ-ಶ-ಕ್ತಿ-ಗಳ
ಕ್ಷುದ್ರ-ಕೀ-ಟಕ್ಕೆ
ಕ್ಷೇತ್ರ-ದಲ್ಲಿ
ಖಂಡಕ್ಕೆ
ಖಂಡ-ಗಳು
ಗಂಗಾ-ನ-ದಿಯ
ಗಂಟ-ಲೊ-ಣಗಿ
ಗಂಟೆಯೂ
ಗಂಡು
ಗಂಭೀರ
ಗಂಭೀ-ರ-ವಾಗಿ
ಗಂಭೀ-ರ-ವಾ-ಣಿ-ಯಿಂದ
ಗಣ-ನೆಗೆ
ಗತ-ಕಾ-ಲದ
ಗತಾನು
ಗಮನ
ಗಮ-ನ-ದ-ಲ್ಲಿಡಿ
ಗಮ-ನಿ-ಸ-ಬೇಕು
ಗಮ-ನಿ-ಸ-ಬೇಡಿ
ಗರ್ಭ-ದಿಂದ
ಗಲೂ
ಗಳ
ಗಳನ್ನು
ಗಳಲ್ಲಿ
ಗಳಿಂದ
ಗಳಿ-ಗಿಂತ
ಗಳಿ-ಸ-ಬಲ್ಲ
ಗಳಿ-ಸ-ಬೇ-ಕಾ-ಗಿದೆ
ಗಳಿ-ಸ-ಬೇಕು
ಗಳು
ಗಳೂ
ಗಳೆಲ್ಲ
ಗಾನ
ಗಾಳಿ-ಯಲ್ಲಿ
ಗಾಳಿ-ಯಾಗಿ
ಗಿಂದು
ಗಿಡ-ಮ-ರ-ಗ-ಳಂತೆ
ಗಿಡ-ಮ-ರ-ಗ-ಳಿಗೂ
ಗಿದೆ
ಗಿದ್ದರೂ
ಗೀತಾ-ಧ್ಯ-ಯ-ನ-ಕ್ಕಿಂತ
ಗೀತೆ-ಯನ್ನು
ಗುಡಿ-ಸ-ಲು-ಗಳಿಂದ
ಗುಡು-ಗಾಡಿ
ಗುಣ
ಗುಣ-ಗಳು
ಗುಣ-ವಿಲ್ಲ
ಗುರಿ
ಗುರು-ವಿಗೆ
ಗುರು-ಸೇವೆ
ಗುಲಾಬಿ
ಗುಲಾ-ಮ-ಗಿರಿ
ಗುಲಾ-ಮರ
ಗುವ
ಗುಹ-ನನ್ನು
ಗೆಲ್ಲು
ಗೆಲ್ಲುವ
ಗೆಲ್ಲು-ವುದು
ಗೆಳೆ-ಯನೆ
ಗೊಣ-ಗಾ-ಡದೆ
ಗೊತ್ತಾ-ಗಲಿ
ಗೊತ್ತಾ-ಗುವ
ಗೊತ್ತಾ-ಯಿತು
ಗೊತ್ತಿ-ಲ್ಲವೆ
ಗೊಯ್ದರೂ
ಗೊಳಿಸಿ
ಗೊಳಿ-ಸುವ
ಗೊಳಿ-ಸು-ವುದು
ಗೋಕು-ಲದ
ಗೋಚರಿ
ಗೋಚ-ರಿ-ಸು-ವುದು
ಗೋತ್ರ
ಗೋಳಿ-ಡು-ವರು
ಗೋಳಿದೆ
ಗೋಳಿ-ನ-ಲ್ಲಿಯೇ
ಗೌರ-ವ-ಗಳನ್ನು
ಗೌರ-ವ-ವಿ-ದೆಯೆ
ಗೌರ-ವ-ಸ್ಥ-ರಿಗೆ
ಗೌರಿ-ನಾಥ
ಗ್ರಂಥ
ಗ್ರಂಥ-ಗಳಿಂದ
ಗ್ರಹ-ಗಳ
ಗ್ರಹಿ-ಸ-ಬ-ಲ್ಲಿರಿ
ಘಟ-ನೆ-ಗ-ಳೊಂ-ದಿಗೆ
ಘಟ್ಟ
ಘನ
ಘನ-ಪಂ-ಡಿ-ತ-ನಿ-ಗಿಂತ
ಘನ-ಯೋ-ಜನೆ
ಘನ-ವಾದ
ಘನ-ವಿ-ದ್ವಾಂ-ಸರು
ಘೋರ
ಘೋಷಿ-ಸಿ-ರಿ-ಆ-ರ್ಯ-ರೆಮ್ಮ
ಚಂಚ-ಲ-ಚಿ-ತ್ತ-ರಾ-ಗ-ಬೇಡಿ
ಚಂಡಾಲ
ಚಂಡಿ-ನಾ-ಟ-ದಿಂದ
ಚಕಿ-ತ-ರ-ನ್ನಾಗಿ
ಚಕ್ರಕ್ಕೆ
ಚಟು-ವ-ಟಿಕೆ
ಚಣೆಯ
ಚಮ-ತ್ಕಾ-ರ-ವಿದು
ಚಮ್ಮಾರ
ಚಮ್ಮಾ-ರ-ರೆ-ಲ್ಲರೂ
ಚರ್ಚಿಗೆ
ಚರ್ಚಿಸು
ಚರ್ಚಿ-ಸು-ತ್ತಿ-ರುವ
ಚರ್ಚು
ಚಲ-ನ-ವ-ಲ-ನ-ಗಳ
ಚಲ-ನಾ-ಹೀನ
ಚಲ-ಮಾನ
ಚಾತ-ಕ-ದಂತೆ
ಚಾರಿತ್ರ
ಚಾರಿ-ತ್ರ-ಶು-ದ್ಧ-ರಾ-ಗ-ಬೇಕು
ಚಾರಿ-ತ್ರ-ಶುದ್ಧಿ
ಚಾರಿ-ತ್ರ್ಯ-ವಿ-ರಲಿ
ಚಾರಿ-ತ್ರ್ಯ-ಶು-ದ್ಧಿಯೇ
ಚಿಂತಿ-ಸು-ವು-ದಲ್ಲ
ಚಿಂತಿ-ಸು-ವುದು
ಚಿಂತೆ-ಯಿಲ್ಲ
ಚಿಕಿತ್ಸೆ
ಚಿತ್ತ-ಗ್ಲಾ-ನಿಗೆ
ಚಿತ್ತಾ-ಕರ್ಷ
ಚಿತ್ರ-ಶಾ-ಲೆ-ಯೊಂ-ದನ್ನು
ಚಿಮ್ಮಿ
ಚಿರ
ಚಿರ-ಕಾಲ
ಚಿರ-ತೆ-ಯಂತೆ
ಚಿಹ್ನೆ
ಚಿಹ್ನೆ-ಯನ್ನು
ಚಿಹ್ನೆ-ಯ-ನ್ನೇ-ನಾ-ದರೂ
ಚುಂಗಾ-ಣೆ-ಯನ್ನು
ಚೂರು
ಚೆನ್ನಾಗಿ
ಚೇತನ
ಚೇತ-ನದ
ಚೇತ-ನ-ವಿಲ್ಲ
ಚೇತ-ನ-ವೊಂದೇ
ಚೇತ-ನ-ಹೀ-ನ-ರಾ-ಗು-ವರು
ಚ್ಯುತಿ-ಬ-ರು-ವು-ದೆಂದು
ಛಾಯಾ-ಮ-ಯ-ವಾಗಿ
ಛಾಯಾ-ಮೂ-ರ್ತಿ-ಗ-ಳಿರಾ
ಛಾಯೆ
ಛಾಯೆ-ಗಳು
ಛಿದ್ರ-ಮ-ಲಿನ
ಜಗ-ತ್ತನ್ನೇ
ಜಗ-ತ್ತಿಗೆ
ಜಗ-ತ್ತಿ-ನಲ್ಲಿ
ಜಗತ್ತು
ಜಗ-ತ್ತೆಲ್ಲ
ಜಗದ
ಜಗ-ನ್ಮಾತೆ
ಜಗಳ
ಜಗ-ಳ-ವಾ-ಡ-ಬೇಡಿ
ಜಗ-ಳ-ವಾ-ಡು-ವುದು
ಜಡ
ಜಡ-ಜ-ಗ-ತ್ತಿನ
ಜಢ-ವಾ-ದಿಯ
ಜನ
ಜನ-ಮಂದೆ
ಜನರ
ಜನ-ರನ್ನು
ಜನ-ರಿಂದ
ಜನ-ರಿಗೆ
ಜನರು
ಜನ-ವಿಲ್ಲ
ಜನ-ಸಾ-ಮಾ-ನ್ಯರ
ಜನಾಂಗ
ಜನಾಂ-ಗದ
ಜನಾಂ-ಗ-ವಾಗಿ
ಜನ್ಮ-ವನ್ನು
ಜನ್ಮ-ವೆ-ತ್ತಿ-ರು-ವಿರಿ
ಜನ್ಮ-ವೆ-ತ್ತಿ-ರು-ವು-ದೆಂದು
ಜಯಕ್ಕೆ
ಜಯದ
ಜಯ-ಪ್ರ-ದ-ವಾಗಿ
ಜಯ-ಪ್ರ-ದ-ವಾ-ಗು-ವುದು
ಜಯ-ಪ್ರ-ದ-ವಾ-ಗು-ವುವು
ಜಯಸಿ
ಜಯಾ-ಪ-ಜ-ಯ-ಗಳನ್ನು
ಜಯಿ-ಸ-ಬೇಕು
ಜಯಿ-ಸು-ವುವು
ಜವಾ-ಬ್ದಾ-ರಿ-ಯನ್ನು
ಜವಾ-ಬ್ದಾ-ರಿ-ಯ-ನ್ನೆಲ್ಲ
ಜಾಗ್ರತ
ಜಾಗ್ರ-ತ-ಗೊ-ಳಿ-ಸ-ಬಲ್ಲ
ಜಾಗ್ರ-ತ-ಗೊ-ಳಿ-ಸ-ಬೇಕು
ಜಾಗ್ರ-ತ-ರ-ನ್ನಾಗಿ
ಜಾಗ್ರ-ತ-ರಾಗಿ
ಜಾಡ-ಮಾ-ಲಿ-ಗಳ
ಜಾಣ
ಜಾತಿ
ಜಿತೇಂ-ದ್ರಿ-ಯ-ತೆ-ಇವೇ
ಜಿತೇಂ-ದ್ರಿ-ಯನು
ಜೀರ್ಣ
ಜೀವ
ಜೀವ-ಕಳೆ
ಜೀವ-ದಾ-ನ-ಮಾ-ಡು-ತ್ತಿದೆ
ಜೀವನ
ಜೀವ-ನದ
ಜೀವ-ನ-ದಲ್ಲಿ
ಜೀವ-ನ-ದಲ್ಲೇ
ಜೀವ-ನ-ವನ್ನು
ಜೀವ-ನ-ವನ್ನೇ
ಜೀವ-ನ-ವಿ-ರು-ವುದು
ಜೀವ-ನ-ವೊಂದು
ಜೀವ-ನಾ-ಡಿ-ಯೊ-ಳಗೆ
ಜೀವ-ನಾ-ಭ್ಯು-ದ-ಯಕ್ಕೆ
ಜೀವ-ನೋ-ಪಾ-ಯಕ್ಕೆ
ಜೀವ-ಶ-ವ-ಗ-ಳಾ-ಗಿ-ದ್ದೀರಿ
ಜೀವಾತ್ಮ
ಜೀವಾ-ತ್ಮನೇ
ಜೀವಿ-ಗಳನ್ನೂ
ಜೀವಿ-ಯ-ಲ್ಲಿಯೂ
ಜೀವಿ-ಸು-ತ್ತಿ-ದ್ದರೂ
ಜೊತೆ
ಜೊತೆಗೆ
ಜೋಡನ್ನು
ಜೋಪ-ಡಿ-ಯಿಂ-ದಾಕೆ
ಜ್ಞಾನ
ಜ್ಞಾನ-ಭ-ಕ್ತಿ
ಜ್ಞಾನಿ-ಗಳ
ಜ್ಞಾಪ-ಕ-ದ-ಲ್ಲಿಡಿ
ಜ್ಞಾಪಿಸಿ
ಜ್ಞಾಪಿ-ಸಿ-ಕೊ-ಳ್ಳು-ತ್ತೇನೆ
ಜ್ಞಾಪಿ-ಸಿ-ಕೊ-ಳ್ಳು-ವ-ವರು
ಜ್ವಾಲಾ-ಮು-ಖಿಯ
ಡಾಲ-ರಿನ
ಣವೆ
ತಂದೆ-ಯೆಂದು
ತಂಬಾ-ಕಿನ
ತಕ್ಷ-ಣವೇ
ತಗ್ಗಿ-ಸಲು
ತಡೆ-ಯ-ಬ-ಲ್ಲ-ವ-ರಿಲ್ಲ
ತಡೆ-ಯ-ಲಾ-ರದು
ತಡೆ-ಯು-ವುದೋ
ತತ್ತ್ವ-ಗಳನ್ನು
ತತ್ತ್ವ-ಗ್ರಂ-ಥ-ವನ್ನೇ
ತನಕ
ತನ-ವನ್ನು
ತನ್ನ
ತನ್ನನ್ನು
ತನ್ನನ್ನೇ
ತನ್ನಲ್ಲಿ
ತನ್ನ-ವರು
ತಪ್ಪನ್ನು
ತಪ್ಪಿ
ತಮಗೆ
ತಮ-ಸ್ಸಿ-ನಲ್ಲಿ
ತಮ್ಮ
ತಮ್ಮಲ್ಲಿ
ತಯಾ-ರಿಸ
ತಯಾ-ರು-ಮಾ-ಡುವ
ತರ-ಬೇಕು
ತರ-ಬೇಡಿ
ತರ-ಬೇ-ತಿ-ನಿಂದ
ತರುಣ
ತರು-ತ್ತದೆ
ತರು-ವನೋ
ತರು-ವಾಯ
ತರು-ವು-ದಿ-ಲ್ಲವೋ
ತರು-ವುದು
ತಲಾಂ-ತ-ರ-ಗಳಿಂದ
ತಲೆ
ತಲೆ-ದೋ-ರಿದೆ
ತಲೆ-ದೋ-ರು-ವುದು
ತಲೆ-ಯನ್ನು
ತವಕ
ತವ-ಕ-ಪ-ಡು-ವು-ದಿಲ್ಲ
ತಸ್ಥರು
ತಾಂತ್ರಿ-ಕರೂ
ತಾಜ-ಮ-ಹ-ಲ್ನಂತೆ
ತಾಟಸ್ಥ್ಯ
ತಾತ್ಸಾರ
ತಾತ್ಸಾ-ರ-ದಿಂದ
ತಾನಾ-ಗಿಯೇ
ತಾನು
ತಾನೆ
ತಾಮಸ
ತಾರು-ಣ್ಯದ
ತಾರೆ
ತಾಳಿ
ತಾಳ್ಮೆ
ತಾಳ್ಮೆ-ಯಿಂದ
ತಾವು
ತಾವೇ
ತಿದ್ದ-ಬೇ-ಕಾ-ಗಿದೆ
ತಿರ-ಸ್ಕ-ರಿ-ಸಿ-ದರೋ
ತಿರ-ಸ್ಕ-ರಿ-ಸಿರಿ
ತಿಳಿ
ತಿಳಿ-ದರೆ
ತಿಳಿ-ದಿ-ರು-ವಿ-ರೇನು
ತಿಳಿದು
ತಿಳಿ-ದು-ಕೊಂ-ಡಿ-ರು-ವ-ವ-ರಲ್ಲಿ
ತಿಳಿ-ದು-ಕೊಂ-ಡಿ-ರು-ವುದನ್ನು
ತಿಳಿ-ದು-ಕೊ-ಳ್ಳ-ಬ-ಲ್ಲಿರಿ
ತಿಳಿ-ದು-ಕೊ-ಳ್ಳು-ವುದು
ತಿಳಿ-ಯದು
ತಿಳಿ-ಯ-ಬೇಕು
ತಿಳಿ-ಯು-ತ್ತೇನೆ
ತಿಳಿವ
ತಿಳಿ-ಸಿ-ದ್ದಾರೆ
ತೀರ-ದಲ್ಲಿ
ತೀರ-ಬೇಕು
ತೀರಿ-ಸಿ-ಕೊ-ಳ್ಳುವ
ತೀರಿ-ಸಿ-ಕೊ-ಳ್ಳು-ವೆವು
ತೀವ್ರತೆ
ತೀವ್ರ-ವಾ-ಗಿಯೂ
ತುಂಬ-ಬೇಕು
ತುಂಬಿ
ತುಂಬಿದೆ
ತುಂಬು-ವುದು
ತುತೂ-ರಿ-ಯೂದಿ
ತುತ್ತಾ
ತುತ್ತಾಗಿ
ತುತ್ತಾ-ದರೆ
ತುತ್ತಾ-ದಾಗ
ತುಳ-ಕಾ-ಡು-ವುದು
ತೂರ್ಯ-ವಾಣಿ
ತೃಪ್ತಿ
ತೃಪ್ತಿ-ಪ-ಡಿ-ಸು-ತ್ತಿ-ರು-ವೆ-ಯೇನು
ತೃಪ್ತಿ-ಯನ್ನು
ತೃಪ್ತಿ-ಯಾ-ದರೂ
ತೆಗೆದು
ತೆಗೆ-ದುಕೊ
ತೆಗೆ-ದು-ಕೊಳ್ಳಿ
ತೆರು-ವರೋ
ತೆರೆ-ಯಲಿ
ತೇಜಸ್ಸು
ತೇಲು-ವನು
ತೊಂದರೆ
ತೊಂದ-ರೆ-ಯಾ-ಗು-ವುದು
ತೊಂದ-ರೆಯೂ
ತೊಂಬತ್ತು
ತೊಟ್ಟಿಲು
ತೊಡಿ-ಸಿದ
ತೊಡೆ-ದು-ಹಾಕಿ
ತೊರೆ
ತೊರೆ-ಯಿರಿ
ತೊಲ-ಗಲಿ
ತೊಲ-ಗಿ-ಹೋಗಿ
ತೊಳೆ-ಯು-ವುದೇ
ತೋಡು-ವು-ದಕ್ಕೆ
ತೋರ-ಬೇಕು
ತೋರಿ
ತೋರಿ-ದರೆ
ತೋರು
ತೋರುವ
ತೋರು-ವನು
ತೋರು-ವು-ದಿಲ್ಲ
ತೋರು-ವುದು
ತೋರು-ವುದೋ
ತ್ತದೆ
ತ್ತಾನೆ
ತ್ತಿರುವ
ತ್ತಿರು-ವಳು
ತ್ತಿರು-ವಾಗ
ತ್ತಿರು-ವಿರಿ
ತ್ತೀರಿ
ತ್ತೀರೋ
ತ್ಯಜಿ-ಸ-ಬೇಕು
ತ್ಯಜಿಸಿ
ತ್ಯಜಿ-ಸೆಂದು
ತ್ಯಾಗ
ತ್ಯಾಗದ
ತ್ಯಾಗ-ಮಾಡಿ
ತ್ಯಾಗ-ಮಾ-ಡಿ-ಕೊಂ-ಡನು
ತ್ಯಾಗ-ವಿ-ಲ್ಲದೆ
ತ್ಯಾಗಿ
ತ್ಯಾಗಿ-ಕು-ಲ-ಚೂ-ಡಾ-ಮಣಿ
ತ್ಯಾಜ್ಯ
ತ್ರಿಭು-ವನ
ತ್ವರಿ-ತ-ವಾಗಿ
ದಂತೆ
ದಕ್ಕಾಗಿ
ದಕ್ಕಿಂತ
ದಕ್ಕೆ
ದಕ್ಷಿ-ಣಕ್ಕೆ
ದಡ್ಡ-ನಾ-ದರೂ
ದಬ್ಬಾ-ಳಿಕೆ
ದಬ್ಬಾ-ಳಿ-ಕೆಗೆ
ದಮ-ಯಂ-ತಿ-ಯರು
ದಯ-ಪಾ-ಲಿಸು
ದಯೆ
ದಯೆ-ಯಿಂದ
ದರಿದ್ರ
ದರಿ-ದ್ರ-ರಲ್ಲಿ
ದರಿ-ದ್ರರು
ದರೆ
ದಲಿ-ತ-ರನ್ನು
ದಲಿ-ತ-ರಿಗೆ
ದಲ್ಲಿ
ದಲ್ಲೂ
ದವ-ನಿಗೆ
ದಹಿಸು
ದಹಿ-ಸು-ತ್ತಿ-ರುವ
ದಾಗ
ದಾಗಲೀ
ದಾಟಿ-ಹೋದ
ದಾಟು-ವರು
ದಾನ-ಕ್ಕಿಂತ
ದಾನ-ಮಾಡಿ
ದಾನ-ಮಾ-ಡುವ
ದಾನ-ಮಾ-ಡು-ವು-ದ-ರಿಂದ
ದಾನಿಯ
ದಾರಿ
ದಾರಿಗ
ದಾರಿದ್ರ್ಯ
ದಾರಿ-ದ್ರ್ಯ-ವಿದ್ದೇ
ದಾರಿ-ಯಲ್ಲಿ
ದಾರಿ-ಯ-ಲ್ಲಿ-ರು-ವಿರಿ
ದಾರು-ಣ-ಯಾ-ತ-ನೆ-ಯನ್ನು
ದಾಸ್ಯ
ದಾಸ್ಯ-ದಲ್ಲಿ
ದಾಹ-ವನ್ನು
ದಿಂದ
ದಿಂದಲೇ
ದಿಕ್ತ-ಟ-ಗಳಿಂದ
ದಿಗೆ
ದಿನ
ದಿನ-ಕ-ಳೆ-ದಂತೆ
ದಿನ-ದಿಂ-ದಲೇ
ದಿನವೂ
ದಿವಾ-ರಾ-ತ್ರೆಯೂ
ದಿವ್ಯೋ-ನ್ಮಾ-ದವೇ
ದಿವ್ಯೌ-ಷಧಿ
ದಿಸುವ
ದೀನ
ದೀನ-ಗೋ-ಪಾಲ
ದೀನ-ನನ್ನು
ದೀನ-ರಿಗೂ
ದೀನ-ರಿಗೆ
ದೀನರು
ದೀನರೂ
ದೀನರೋ
ದೀವಿಗೆ
ದುಃಖ
ದುಃಖಕ್ಕೆ
ದುಃಖದ
ದುಃಖ-ದಿಂದ
ದುಃಖ-ಭಾ-ಗಿ-ಗ-ಳಾ-ಗು-ವೆವು
ದುಃಖ-ಭಾ-ಜ-ನ-ರಾ-ದು-ದ-ರಿಂದ
ದುಃಖ-ವನ್ನು
ದುಃಖ-ವನ್ನೂ
ದುಃಖಾ-ಗ್ನಿ-ಯಲ್ಲಿ
ದುಃಖಿ-ಗಳೂ
ದುಡಿತ
ದುಡಿ-ತ-ದಿಂದ
ದುಡಿ-ದರೆ
ದುಡಿದು
ದುಡಿಯು
ದುಡಿ-ಯುವ
ದುರಂ-ತದ
ದುರಾತ್ಮ
ದುರ್ಗು-ಣ-ಗಳನ್ನು
ದುರ್ಗು-ಣವೇ
ದುರ್ಜ-ನನ
ದುರ್ಬಲ
ದುರ್ಬ-ಲತೆ
ದುರ್ಬ-ಲ-ತೆಗೆ
ದುರ್ಬ-ಲ-ತೆಯ
ದುರ್ಬ-ಲ-ತೆ-ಯನ್ನೇ
ದುರ್ಬ-ಲ-ತೆಯೇ
ದುರ್ಬ-ಲ-ರ-ನ್ನಾಗಿ
ದುರ್ಬ-ಲ-ರಲ್ಲಿ
ದುರ್ಬ-ಲ-ರಾ-ದು-ದ-ರಿಂದ
ದುರ್ಬ-ಲರು
ದುರ್ಬ-ಲರೂ
ದುರ್ಲಭ
ದೂರದ
ದೂರ-ಬೇಡಿ
ದೂರಲು
ದೂರು-ವುದು
ದೃಢ-ವಾಗಿ
ದೃಢಿಷ್ಠ
ದೃಷ್ಟಿಗೆ
ದೃಷ್ಟಿ-ಯಿಂದ
ದೆಂದು
ದೆದ್ದೇಳಿ
ದೇಗು-ಲ-ಗ-ಳ-ಲ್ಲೆಲ್ಲ
ದೇಗು-ಲ-ದಂತೆ
ದೇಗು-ಲ-ದಿಂ-ದಲೂ
ದೇವ
ದೇವತೆ
ದೇವ-ತೆ-ಗ-ಳಿ-ಗಿಂ-ತಲೂ
ದೇವ-ತೆ-ಗ-ಳಿಗೂ
ದೇವ-ತೆ-ಗಳು
ದೇವ-ತೆಗೂ
ದೇವ-ತ್ವ-ವನ್ನು
ದೇವ-ನನ್ನು
ದೇವ-ನಿಂದೆ
ದೇವರ
ದೇವ-ರಂತೆ
ದೇವ-ರ-ನ್ನಾಗಿ
ದೇವ-ರನ್ನು
ದೇವ-ರ-ಲ್ಲದೆ
ದೇವ-ರ-ಲ್ಲವೆ
ದೇವ-ರಲ್ಲಿ
ದೇವ-ರಾಗಿ
ದೇವ-ರಾ-ಗಿ-ರು-ವುದೋ
ದೇವ-ರಾ-ಗು-ವಿರಿ
ದೇವ-ರಾ-ಗೋಣ
ದೇವ-ರಿಗೆ
ದೇವರು
ದೇವ-ರೆಂದು
ದೇವರೇ
ದೇವ-ಸ್ಥಾನ
ದೇವ-ಸ್ಥಾ-ನಕ್ಕೆ
ದೇವ-ಸ್ಥಾ-ನ-ಗ-ಳಿಗೂ
ದೇವ-ಸ್ಥಾ-ನ-ದಲ್ಲಿ
ದೇವಾ-ಲ-ಯಕ್ಕೆ
ದೇಶ
ದೇಶಕ್ಕೂ
ದೇಶಕ್ಕೆ
ದೇಶದ
ದೇಶ-ದ-ಲ್ಲಿಯೂ
ದೇಶ-ದಲ್ಲೆಲ್ಲಾ
ದೇಶ-ವನ್ನು
ದೇಶ-ವಾ-ಗಲಿ
ದೇಶವು
ದೇಶ-ವೆಲ್ಲ
ದೇಹ
ದೇಹದ
ದೇಹ-ದ-ಲ್ಲಿ-ರುವ
ದೇಹ-ವ-ನ್ನಾ-ಗಲಿ
ದೇಹ-ವನ್ನೇ
ದೇಹ-ವೆಂದು
ದೇಹವೇ
ದೈವ-ದ್ರೋಹ
ದೈವ-ಭಕ್ತಿ
ದೈಹಿಕ
ದೈಹಿ-ಕ-ವಾ-ಗಾ-ಗಲಿ
ದೊಡ್ಡ
ದೊಡ್ಡದಾ
ದೊಡ್ಡದೇ
ದೊಡ್ಡ-ವ-ರಾ-ಗಿ-ರ-ಬ-ಹುದು
ದೊರ-ಕಿ-ದು-ದ-ರಿಂದ
ದೊರೆ-ತಿದೆ
ದೋಷ
ದೋಷಾ-ರೋ-ಪ-ಣೆಗೆ
ದೌರ್ಜನ್ಯ
ದೌರ್ಜ-ನ್ಯದ
ದೌರ್ಜ-ನ್ಯ-ವನ್ನೂ
ದೌರ್ಬ-ಲ್ಯ-ವನ್ನು
ದ್ದರೆ
ದ್ದೆವೋ
ದ್ರವ್ಯದ
ದ್ವೇಶಿ-ಸು-ವೆನು
ದ್ವೇಷ
ದ್ವೇಷ-ಕ್ಕಿಂತ
ದ್ವೇಷಿಸಿ
ದ್ವೇಷಿ-ಸು-ವು-ದಿಲ್ಲ
ಧನ್ಯ
ಧನ್ಯ-ನಾ-ಗು-ವನು
ಧನ್ಯರು
ಧನ್ಯಾತ್ಮ
ಧರಿಸಿ
ಧರಿ-ಸಿ-ರುವ
ಧರಿ-ಸುವ
ಧರ್ಮ
ಧರ್ಮಕ್ಕೆ
ಧರ್ಮದ
ಧರ್ಮ-ಧಾ-ರಾ-ಪ್ರ-ವಾ-ಹ-ದಿಂದ
ಧರ್ಮ-ಬೋಧೆ
ಧರ್ಮ-ವಲ್ಲ
ಧರ್ಮ-ವೆಂದರೆ
ಧರ್ಮ-ವೆಂದು
ಧರ್ಮವೇ
ಧರ್ಮಿಷ್ಠ
ಧಾರ್ಮಿಕ
ಧಾರ್ಮಿ-ಕ-ತೆಗೆ
ಧಿಸಿ
ಧೀರ
ಧೀರ-ನಾ-ಗು-ವನು
ಧೀರನೆ
ಧೀರ-ರಾಗಿ
ಧೀರ-ರೆಂ-ದಿಗೂ
ಧೂಳಿಗೆ
ಧೈರ್ಯ
ಧೈರ್ಯ-ವನ್ನು
ಧೈರ್ಯ-ಶಾ-ಲಿ-ಗ-ಳಾ-ಗ-ಬೇಕು
ಧ್ಯಾನ-ವನ್ನು
ಧ್ವನಿ-ಯಂತೆ
ನಂತರ
ನಂತಹ
ನಂದಾ-ದೀ-ವಿ-ಗೆ-ಯಂತೆ
ನಂದಿ-ಸ-ಲಾ-ಗದ
ನಂಬಿ
ನಂಬಿಕೆ
ನಂಬಿ-ಕೆ-ಯನ್ನೂ
ನಂಬಿ-ದ್ದರೋ
ನಂಬು-ವರು
ನಗು-ವುದು
ನಗೆ-ನ-ಲಿ-ದಾ-ಟ-ಗ-ಳೆಲ್ಲ
ನಡತೆ
ನಡು-ಗದೆ
ನಡೆ-ಯ-ಬೇಕು
ನಡೆ-ಸಿದ
ನತಿಯು
ನನಗೆ
ನನ್ನ
ನನ್ನದು
ನನ್ನನ್ನು
ನನ್ನಾತ್ಮ
ನನ್ನಾ-ತ್ಮನೇ
ನನ್ನು
ನಮ-ಗಾಗಿ
ನಮ-ಗಿಂದು
ನಮ-ಗೀಗ
ನಮಗೆ
ನಮಗೇ
ನಮ-ಗೊಂದು
ನಮ-ಸ್ಕಾರ
ನಮ್ಮ
ನಮ್ಮನ್ನು
ನಮ್ಮಲ್ಲಿ
ನಮ್ಮ-ಲ್ಲಿ-ರುವ
ನಮ್ಮೀ
ನರ
ನರ-ಕಕ್ಕೆ
ನರ-ಕದ
ನರ-ಕ-ದಿಂದ
ನರ-ಗ-ಳಿಗೆ
ನರ-ಗಳು
ನರ-ನಾ-ರಿ-ಯ-ರನ್ನೂ
ನರ-ಳು-ತ್ತಿದೆ
ನರ-ಳು-ತ್ತಿರು
ನರ-ಳು-ತ್ತಿ-ರುವ
ನರ-ಳು-ತ್ತಿ-ರು-ವರೋ
ನರ-ಳುವ
ನವ-ಚೇ-ತ-ನ-ವನ್ನು
ನವ-ಭಾ-ರ-ತದ
ನವೀನ
ನವೋ-ದ-ಯ-ವಾ-ಗು-ತ್ತಿದೆ
ನಷ್ಟ-ವೇನೂ
ನಾಗ-ರಿ-ಕತೆ
ನಾಗಿ
ನಾಚಿ-ಕೆ-ಗೇಡು
ನಾಜೂ-ಕಾಗಿ
ನಾಡಿ-ಯಲ್ಲಿ
ನಾನಾ
ನಾನು
ನಾನೂ
ನಾಯಿ-ಮ-ರಿ-ಗಳ
ನಾರಿ
ನಾರೀ-ಗೌ-ರವ
ನಾಲ್ಕ-ನೆಯ
ನಾಲ್ಕು
ನಾವು
ನಾವೆಂಬ
ನಾವೆಲ್ಲ
ನಾವೇ
ನಾಶ
ನಾಶ-ಮಾ-ಡುವ
ನಾಶ-ವಾ-ಗು-ವುದೇ
ನಾಶ-ವಾ-ದರೂ
ನಾಸ್ತಿ-ಕ-ನೆಂದು
ನಿಂತ
ನಿಂತಾಗ
ನಿಂತಿದೆ
ನಿಂತಿ-ರು-ವರು
ನಿಂತಿ-ರು-ವುದು
ನಿಂತು
ನಿಂತು-ಕೊಳ್ಳು
ನಿಂದಾ-ರ್ಹರು
ನಿಂದಿ-ಸ-ಬೇಡಿ
ನಿಂದಿ-ಸು-ವರೋ
ನಿಂದೆ
ನಿಃಸ್ಪೃಹ
ನಿಃಸ್ವಾರ್ಥ
ನಿಃಸ್ವಾ-ರ್ಥತೆ
ನಿಃಸ್ವಾ-ರ್ಥ-ತೆಯೇ
ನಿಃಸ್ವಾ-ರ್ಥವೋ
ನಿಃಸ್ವಾ-ರ್ಥಿ-ಗಳೆ
ನಿಗ್ರಹ
ನಿಗ್ರ-ಹಿ-ಸದೆ
ನಿಗ್ರ-ಹಿಸಿ
ನಿಗ್ರ-ಹಿ-ಸಿದ
ನಿಜ
ನಿಜ-ವಾಗಿ
ನಿಜ-ವಾ-ಗಿಯೂ
ನಿಜ-ವಾದ
ನಿತ್ಯ
ನಿತ್ಯ-ಜೀ-ವ-ನ-ದಲ್ಲಿ
ನಿತ್ಯ-ತೃ-ಪ್ತ-ನಾ-ಗು-ವನು
ನಿತ್ಯ-ಪ-ವಿ-ತ್ರ-ನಾ-ಗು-ವನು
ನಿತ್ಯಾ-ನಂ-ದ-ಲ್ಲಿ-ರು-ವನು
ನಿದ್ದೆ
ನಿದ್ದೆ-ಯಿಂದ
ನಿದ್ರಿ-ಸ-ಬ-ಲ್ಲಿರಾ
ನಿದ್ರಿಸು
ನಿದ್ರಿ-ಸುವ
ನಿದ್ರೆ-ಯನ್ನು
ನಿದ್ರೆ-ಯಿಂ
ನಿದ್ರೆ-ಯಿಂದ
ನಿನ-ಗಿದೊ
ನಿನ-ಗಿದೋ
ನಿನಗೂ
ನಿನಗೆ
ನಿನ್ನ
ನಿನ್ನನ್ನು
ನಿನ್ನಲ್ಲಿ
ನಿನ್ನ-ಲ್ಲಿ-ದ್ದರೆ
ನಿಮ-ಗಿಂ-ತಲೂ
ನಿಮಗೂ
ನಿಮಗೆ
ನಿಮಿಷ
ನಿಮಿ-ಷದ
ನಿಮ್ಮ
ನಿಮ್ಮಂ-ತ-ಹ-ವರು
ನಿಮ್ಮನ್ನು
ನಿಮ್ಮ-ನ್ನೆಲ್ಲ
ನಿಮ್ಮಲ್ಲಿ
ನಿಮ್ಮ-ಲ್ಲಿದೆ
ನಿಮ್ಮ-ಲ್ಲಿ-ದ್ದರೆ
ನಿಮ್ಮ-ಲ್ಲಿ-ರುವ
ನಿಮ್ಮಲ್ಲೇ
ನಿಮ್ಮ-ಹೆ-ಗ-ಲನ್ನು
ನಿಮ್ಮೆದೆ
ನಿಮ್ಮೆ-ದೆಗೆ
ನಿಮ್ಮೆ-ಲ್ಲ-ರಿಂದ
ನಿಮ್ಮೊಂ-ದಿಗೆ
ನಿಯಮ
ನಿಯ-ಮ-ವನ್ನು
ನಿರಂ-ತರ
ನಿರಂ-ತ-ರ-ವಾಗಿ
ನಿರ-ತ-ನಾಗು
ನಿರ-ತ-ರಾ-ಗ-ಬೇಕು
ನಿರ-ತ್ಸಾಹ
ನಿರ-ಪೇಕ್ಷೆ
ನಿರಾ-ಕ-ರಿಸಿ
ನಿರಾಶ
ನಿರಾ-ಶ-ನಾ-ಗ-ದಿರು
ನಿರೀ-ಕ್ಷಿ-ಸ-ಬ-ಲ್ಲಿರಿ
ನಿರೀ-ಕ್ಷಿ-ಸ-ಬೇಡಿ
ನಿರುವ
ನಿರ್ಜೀ-ವ-ದಂ-ತಿದ್ದ
ನಿರ್ದಿ-ಷ್ಟ-ವಾದ
ನಿರ್ಧ-ರಿಸಿ
ನಿರ್ಧ-ರಿ-ಸು-ತ್ತಿಲ್ಲ
ನಿರ್ನಾ-ಮ-ವಾ-ಗುವ
ನಿರ್ನಾ-ಮ-ವಾ-ಗು-ವುದು
ನಿರ್ಭ-ಯ-ರಾ-ಗ-ಬೇಕು
ನಿರ್ಭಾ-ಗ್ಯನೆ
ನಿರ್ಮ-ಲ-ಜೀ-ವ-ನ-ದಿಂದ
ನಿರ್ಮಾ-ಣ-ಮಾ-ಡು-ವುದನ್ನು
ನಿರ್ಮಿಸಿ
ನಿರ್ಲಕ್ಷ್ಯ
ನಿರ್ಲ-ಕ್ಷ್ಯ-ದಿಂದ
ನಿಲ್ಲ-ಬಲ್ಲೆ
ನಿಲ್ಲ-ಬೇಡಿ
ನಿಲ್ಲಿ
ನಿಲ್ಲಿ-ಸಿದ
ನಿಲ್ಲು-ತ್ತಿ-ರು-ವಳು
ನಿಲ್ಲುವ
ನಿಲ್ಲು-ವು-ದೆಂಬ
ನಿಲ್ಲು-ವೆನು
ನಿವಾ-ರಿ-ಸ-ಬ-ಲ್ಲೆವು
ನಿವಾ-ರಿ-ಸಿ-ದೊ-ಡನೆ
ನಿವಾ-ರಿ-ಸುವ
ನಿವಾ-ಸ-ದಿಂದ
ನಿವಾ-ಸ-ವನ್ನು
ನಿಷ್ಕ-ಪಟ
ನಿಷ್ಕ-ಪ-ಟಿ-ಗ-ಳಾಗಿ
ನಿಷ್ಕಾ-ಪಟ್ಯ
ನಿಷ್ಠಾ-ವಂ-ತ-ರಾಗಿ
ನಿಷ್ಠೆ
ನಿಷ್ಪ್ರ-ಯೋ-ಜನ
ನಿಸ್ಸಂ-ಗ-ನಾ-ಗಿರಿ
ನಿಸ್ಸಂ-ದೇ-ಹ-ವಾಗಿ
ನೀಡಲಾ
ನೀಡಿ
ನೀಡುವ
ನೀಡು-ವನೋ
ನೀಡು-ವು-ದಕ್ಕೆ
ನೀಡು-ವುದು
ನೀತಿ
ನೀನು
ನೀನೆ
ನೀನೇ
ನೀನೊಂದು
ನೀರನ್ನು
ನೀರನ್ನೂ
ನೀರ-ವ-ವಾಗಿ
ನೀಲಿಯೇ
ನೀವು
ನೀವೂ
ನೀವೆ
ನೀವೆಂತು
ನೀವೆ-ನಿತು
ನೀವೆಲ್ಲ
ನೀವೇ
ನೀವೇಕೆ
ನೀವೊಬ್ಬ
ನುಗ್ಗಿ
ನುಡಿ-ಗಳು
ನುಡಿಯೇ
ನೂರಾರು
ನೆಚ್ಚ-ಬೇಡಿ
ನೆಚ್ಚಿನ
ನೆಚ್ಚು-ಗೆ-ಡ-ದಿರಿ
ನೆನ-ಪಾ-ಗು-ತ್ತ-ದೆ-ಜೀ-ವ-ವಿಲ್ಲ
ನೆನ-ಪಿ-ನ-ಲ್ಲಿ-ಡ-ಬೇಕು
ನೆನ-ಪಿ-ನ-ಲ್ಲಿಡಿ
ನೆನಪು
ನೆನೆ-ಯು-ತ್ತಿ-ದ್ದರೆ
ನೆಯದೇ
ನೆರ-ವೇ-ರಿ-ಸ-ಬಲ್ಲ
ನೆಲ-ಸಿ-ರು-ವವೋ
ನೆಲ-ಸಿ-ರು-ವುದು
ನೈಜ
ನೈಜ-ತ್ವ-ವನ್ನು
ನೈಜ್ಯ
ನೈತಿ-ಕ-ರಾಗಿ
ನೈತಿ-ಕೋ-ನ್ನ-ತಿ-ಯೆಲ್ಲ
ನೊರೆಯ
ನೋಡ
ನೋಡ-ಬ-ಯ-ಸು-ವೆನು
ನೋಡ-ಬೇ-ಕಾದ
ನೋಡ-ಬೇಡಿ
ನೋಡಲು
ನೋಡಿ
ನೋಡಿ-ಕೊ-ಳ್ಳು-ವುದು
ನೋಡಿ-ದರೆ
ನೋಡಿ-ದ-ವ-ರಿಗೆ
ನೋಡಿ-ರು-ವ-ವನ
ನೋಡಿ-ರು-ವಿರಾ
ನೋಡಿ-ರು-ವೆನು
ನೋಡಿ-ರು-ವೆವು
ನೋಡು
ನೋಡು-ತ್ತಿ-ರುವ
ನೋಡು-ತ್ತಿ-ರು-ವುದು
ನೋಡುವ
ನೋಡು-ವನು
ನೋಡು-ವರೋ
ನೋಡು-ವು-ದಕ್ಕೆ
ನೋಯು-ತ್ತಿದೆ
ನ್ಮುಖ-ರಾಗಿ
ನ್ಯೂನ-ತೆ-ಯನ್ನೂ
ಪಂಗಡ
ಪಂಗ-ಡಕ್ಕೆ
ಪಂಗ-ಡ-ಗಳಿಂದ
ಪಂಗ-ಡದ
ಪಂಗ-ಡ-ವಲ್ಲ
ಪಕ್ಷ-ಪಾ-ತವೇ
ಪಟ್ಟಕ್ಕೆ
ಪಡು-ವನು
ಪಡು-ವುದು
ಪಡೆ-ದು-ಕೊ-ಳ್ಳಲೇ
ಪಡೆ-ಯ-ಬ-ಹುದು
ಪಡೆ-ಯಿರಿ
ಪಡೆ-ಯು-ವನು
ಪಡೆ-ಯು-ವು-ದಕ್ಕೆ
ಪಥಕ್ಕೆ
ಪದ
ಪದ-ಗಳು
ಪದ-ದಡಿ
ಪದ-ದ-ಲಿ-ತ-ರಾ-ಗಿ-ರು-ವರು
ಪದ-ದಲ್ಲಿ
ಪದ-ವನ್ನು
ಪದ-ವಿ-ಯನ್ನು
ಪದವೂ
ಪದಾ-ಘಾ-ತ-ವನ್ನು
ಪರಂ-ಧಾಮ
ಪರಂ-ಪರೆ
ಪರಂ-ಪ-ರೆ-ಯ-ಲ್ಲಿಯೂ
ಪರಮ
ಪರ-ಮ-ಪ್ರೇ-ಮ-ವೆಂಬ
ಪರ-ಮ-ಪ್ರೇ-ಮವೇ
ಪರ-ಮಾ-ತ್ಮನ
ಪರ-ಮಾ-ನಂದ
ಪರ-ರಿಗೆ
ಪರಾ-ಕ್ರ-ಮ-ಶಾ-ಲಿ-ಯಾಗಿ
ಪರಿ
ಪರಿ-ಗ-ಣಿಸಿ
ಪರಿ-ಚಯ
ಪರಿ-ಣಾ-ಮ-ಕಾರಿ
ಪರಿ-ಣಾ-ಮ-ಕಾ-ರಿ-ಯಾ-ಗು-ವುದು
ಪರಿ-ಪೂ-ರ್ಣ-ತೆಯ
ಪರಿ-ಪೂ-ರ್ಣ-ತೆ-ಯನ್ನು
ಪರಿ-ಯಂ-ತವೂ
ಪರಿ-ಶುದ್ಧ
ಪರಿ-ಶು-ದ್ಧತೆ
ಪರಿ-ಶು-ದ್ಧ-ರ-ನ್ನಾಗಿ
ಪರಿ-ಶು-ದ್ಧ-ರಾಗಿ
ಪರಿ-ಶು-ದ್ಧ-ರಾ-ಗು-ವಿರಿ
ಪರಿ-ಶು-ದ್ಧ-ವ-ಲ್ಲವೋ
ಪರಿ-ಶು-ದ್ಧ-ವಾ-ಗಿ-ರಲಿ
ಪರಿ-ಶು-ದ್ಧ-ವಾ-ದುದು
ಪರಿ-ಹ-ರಿಸು
ಪರಿ-ಹ-ರಿ-ಸು-ತ್ತಿದೆ
ಪರೀ-ಕ್ಷಿ-ಸು-ವು-ದಕ್ಕೆ
ಪರೀಕ್ಷೆ
ಪರೀ-ಕ್ಷೆ-ಗಳಲ್ಲಿ
ಪರೋ-ಪ-ಕಾರಿ
ಪರ್ಯಂ-ತವೂ
ಪರ್ಯಾಯ
ಪರ್ವತ
ಪರ್ವ-ತ-ಕಾ-ನ-ನ-ಗಳಿಂದ
ಪರ್ವ-ತೋ-ಪಮ
ಪಲ್ಲ-ವಿ-ಯಾ-ಗಲಿ
ಪವಿತ್ರ
ಪವಿ-ತ್ರತೆ
ಪವಿ-ತ್ರ-ತೆ-ಯನ್ನು
ಪವಿ-ತ್ರ-ನಾ-ಗ-ಬೇ-ಕೆಂಬ
ಪವಿ-ತ್ರ-ರಾ-ಗುವ
ಪವಿ-ತ್ರ-ವಾ-ಗು-ವುದು
ಪಶು-ಗ-ಳಲ್ಲ
ಪಶ್ಚಾ-ತ್ತಾ-ಪ-ಪ-ಡ-ಬೇಡಿ
ಪಶ್ಚಿ-ಮಕ್ಕೆ
ಪಾಂಡಿತ್ಯ
ಪಾಂಡಿ-ತ್ಯ-ದಿಂ-ದಲೂ
ಪಾಂಡಿ-ತ್ಯವೆ
ಪಾಕ-ಶಾ-ಲೆಗೆ
ಪಾಠ-ಗಳನ್ನು
ಪಾತ್ರೆ
ಪಾದ-ತಳ
ಪಾನ-ಮಾ-ಡ-ಬೇ-ಕಾ-ಗಿಲ್ಲ
ಪಾಪ
ಪಾಪ-ಕೂ-ಪ-ದಂತೆ
ಪಾಪಕ್ಕೂ
ಪಾಪಕ್ಕೆ
ಪಾಪ-ವನ್ನು
ಪಾಪ-ವನ್ನೂ
ಪಾಪ-ವೆಂದರೆ
ಪಾಪಿ
ಪಾಪಿ-ಗಳ
ಪಾಪಿ-ಗಳು
ಪಾಪಿ-ಯಂತೆ
ಪಾರ-ಮಾ-ರ್ಥಿಕ
ಪಾರಾ-ಗ-ಬಲ್ಲ
ಪಾರಾ-ಗಲು
ಪಾರಾಗಿ
ಪಾರ್ಥ-ಸಾ-ರಥಿ
ಪಾಲಿನ
ಪಾಲಿ-ಸ-ಬ-ಲ್ಲರೋ
ಪಾಲಿ-ಸಲು
ಪಾಲು
ಪಾಶ್ಚಾತ್ಯ
ಪಾಶ್ಚಾ-ತ್ಯರ
ಪಾಶ್ಚಾ-ತ್ಯ-ರನ್ನು
ಪಿತ್ತ-ದಿಂದ
ಪೀಠಕ್ಕೆ
ಪೀಠದ
ಪುಣಿ-ಗಳು
ಪುಣ್ಯ
ಪುಣ್ಯ-ಭೂ-ಮಿ-ಯನ್ನು
ಪುಣ್ಯಾ-ವ-ಕಾಶ
ಪುನಃ
ಪುನ-ರು-ದ್ಧಾ-ರದ
ಪುರಾ-ಣ-ದಲ್ಲಿ
ಪುರಾ-ತನ
ಪುರಾ-ತ-ನ-ಕಾ-ಲದ
ಪುರಾ-ತ-ನದ
ಪುರು-ಷನೇ
ಪುರು-ಷ-ರಾಗಿ
ಪುರು-ಷ-ವೀ-ರರು
ಪುರು-ಷ-ಸಿಂ-ಹ-ರ-ನ್ನಾಗಿ
ಪುರು-ಷ-ಸಿಂ-ಹ-ರಾ-ಗ-ಬೇಕು
ಪುರು-ಷ-ಸಿಂ-ಹ-ರಾಗಿ
ಪುರು-ಷ-ಸಿಂ-ಹರು
ಪುರೋ-ಗ-ಮನ
ಪುರೋ-ಹಿ-ತರ
ಪುರೋ-ಹಿ-ತ-ರಲ್ಲಿ
ಪುಷಿ-ಗ-ಳಾ-ಗು-ವುವು
ಪುಸ್ತಕ
ಪುಸ್ತ-ಕಾ-ಲಯ
ಪೂಜಿ-ಸ-ಬಾ-ರದು
ಪೂಜಿಸಿ
ಪೂಜಿಸು
ಪೂಜೆ
ಪೂಜೆಗೆ
ಪೂಜೆಯ
ಪೂಜೆ-ಯಂತೆ
ಪೂಜೆ-ಯಿಂದ
ಪೂಜ್ಯ-ಭಾ-ವ-ದಿಂದ
ಪೂರೈಸಿ
ಪೂರೈ-ಸಿ-ದಂತೆ
ಪೂರ್ವ-ದಿಂದ
ಪೂರ್ವಿ-ಕರು
ಪೂರ್ವೀ-ಕ-ರಿತ್ತ
ಪೆಟ್ಟು-ಗಳು
ಪೌರಾತ್ಯ
ಪೌರುಷ
ಪೌರು-ಷ-ವಂ-ತ-ರ-ನ್ನಾಗಿ
ಪೌರು-ಷ-ವಂ-ತ-ರಾಗಿ
ಪೌರು-ಷ-ವನ್ನು
ಪ್ಯದ
ಪ್ರಕಟಿ
ಪ್ರಕಾ-ಶ-ಕರು
ಪ್ರಕಾ-ಶಿಸಿ
ಪ್ರಕಾ-ಶಿ-ಸು-ತ್ತಿ-ರುವ
ಪ್ರಕೃತಿ
ಪ್ರಕೃ-ತಿ-ಯನ್ನು
ಪ್ರಕೃ-ತಿ-ಯೊಂ-ದಿಗೆ
ಪ್ರಖ್ಯಾ-ತ-ರಾ-ಗ-ಲಾ-ರರು
ಪ್ರಖ್ಯಾ-ತ-ರಾಗಿ
ಪ್ರಚಂಡ
ಪ್ರಚೋ
ಪ್ರಚೋ-ಸಿ-ದರೆ
ಪ್ರತಿ
ಪ್ರತಿ-ದಿ-ನವೂ
ಪ್ರತಿ-ಫಲ
ಪ್ರತಿ-ಫ-ಲ-ವನ್ನು
ಪ್ರತಿ-ಫ-ಲ-ವಾಗಿ
ಪ್ರತಿ-ಭ-ಟಿ-ಸು-ತ್ತಿ-ದ್ದರೂ
ಪ್ರತಿಭೆ
ಪ್ರತಿ-ಯೊಂ-ದನ್ನೂ
ಪ್ರತಿ-ಯೊಂದು
ಪ್ರತಿ-ಯೊಬ್ಬ
ಪ್ರತಿ-ಯೊ-ಬ್ಬನೂ
ಪ್ರತಿ-ಯೊ-ಬ್ಬರೂ
ಪ್ರತ್ಯಕ್ಷ
ಪ್ರತ್ಯ-ಕ್ಷದ
ಪ್ರತ್ಯ-ಕ್ಷ-ವಾ-ಗಲಿ
ಪ್ರಥಮ
ಪ್ರದ-ರ್ಶಿ-ಸ-ಬೇಡಿ
ಪ್ರಧಾ-ನ-ಜ-ಡ-ವಲ್ಲ
ಪ್ರಪಂಚ
ಪ್ರಪಂ-ಚಕ್ಕೆ
ಪ್ರಪಂ-ಚದ
ಪ್ರಪಂ-ಚ-ದ-ಲ್ಲಾ-ಗಲಿ
ಪ್ರಪಂ-ಚ-ದಲ್ಲಿ
ಪ್ರಪಂ-ಚ-ದಿಂದ
ಪ್ರಪಂ-ಚ-ವನ್ನು
ಪ್ರಪಂ-ಚ-ವನ್ನೇ
ಪ್ರಪಂ-ಚ-ವೆಲ್ಲ
ಪ್ರಪಂ-ಚವೇ
ಪ್ರಬ-ಲ-ವಾಗಿ
ಪ್ರಬು-ದ್ಧ-ನಾಗು
ಪ್ರಭಾವ
ಪ್ರಭಾ-ವಕ್ಕೂ
ಪ್ರಯತ್ನ
ಪ್ರಯ-ತ್ನಕ್ಕೆ
ಪ್ರಯ-ತ್ನ-ದಲ್ಲಿ
ಪ್ರಯ-ತ್ನಿಸಿ
ಪ್ರಯ-ತ್ನಿ-ಸಿ-ದರೆ
ಪ್ರಯೋ
ಪ್ರಯೋ-ಗಿಸಿ
ಪ್ರಯೋ-ಗಿಸು
ಪ್ರಯೋ-ಜನ
ಪ್ರಯೋ-ಜ-ನ-ವಿಲ್ಲ
ಪ್ರಯೋ-ಜ-ನ-ವೇನು
ಪ್ರಲಾ-ಪ-ವನ್ನು
ಪ್ರಲೋ-ಭ-ನೆ-ಯನ್ನು
ಪ್ರವೇ-ಶಿ-ಸದು
ಪ್ರವೇ-ಶಿ-ಸಲಿ
ಪ್ರವೇ-ಶಿಸು
ಪ್ರವೇ-ಶಿ-ಸು-ತ್ತಿ-ರುವ
ಪ್ರವೇ-ಶಿ-ಸುವ
ಪ್ರಸಂ-ಗ-ದಲ್ಲಿ
ಪ್ರಸಂ-ಗ-ವೆಂ-ದಿಗೂ
ಪ್ರಸ-ನ್ನ-ವಾ-ಗಿದೆ
ಪ್ರಸೂತ
ಪ್ರಸೂ-ತ-ರೆಂದು
ಪ್ರಾಕ್ತನ
ಪ್ರಾಣ
ಪ್ರಾಣ-ಬಿ-ಡು-ವುದು
ಪ್ರಾಣ-ವಾಯು
ಪ್ರಾಣ-ಹೋ-ದರೂ
ಪ್ರಾಣಿ-ಗ-ಳೆಲ್ಲ
ಪ್ರಾಣಿ-ಗ-ಳೆ-ಲ್ಲ-ಕ್ಕಿಂತ
ಪ್ರಾಪಂ-ಚಿಕ
ಪ್ರಾಪಂ-ಚಿ-ಕರು
ಪ್ರಾಪಂ-ಚಿ-ಕ-ವ-ಸ್ತು-ಗ-ಳಿ-ಗಾಗಿ
ಪ್ರಾಪ್ತ
ಪ್ರಾಪ್ತ-ವಾ-ಗ-ಲಾ-ರ-ದೆಂದು
ಪ್ರಾಪ್ತ-ವಾ-ಗಲಿ
ಪ್ರಾಪ್ತಿ
ಪ್ರಾರಂ
ಪ್ರಾರಂ-ಭ-ದಲ್ಲಿ
ಪ್ರಾರಂ-ಭ-ವಾ-ದಂತೆ
ಪ್ರಾರ್ಥನೆ
ಪ್ರಾರ್ಥ-ನೆ-ಯಾ-ಗಿ-ರಲಿ
ಪ್ರಾರ್ಥಿ-ಸು-ವಾಗ
ಪ್ರೀತಿ
ಪ್ರೀತಿ-ಕೊಡಿ
ಪ್ರೀತಿ-ಗಾಗಿ
ಪ್ರೀತಿ-ಯಿಂದ
ಪ್ರೀತಿಯೂ
ಪ್ರೀತಿಯೇ
ಪ್ರೀತಿ-ಸ-ಬ-ಲ್ಲೆವು
ಪ್ರೀತಿಸಿ
ಪ್ರೀತಿ-ಸು-ವನು
ಪ್ರೀತಿ-ಸು-ವನೋ
ಪ್ರೀತಿ-ಸು-ವರೋ
ಪ್ರೀತಿ-ಸು-ವಿ-ರೇನು
ಪ್ರೇಮ
ಪ್ರೇಮದ
ಪ್ರೇಮ-ದಲ್ಲಿ
ಪ್ರೇಮ-ದಿಂದ
ಪ್ರೇಮ-ವನ್ನು
ಪ್ರೇಮವೇ
ಪ್ರೇಮಿಗೆ
ಪ್ರೇರಿ-ತ-ವಾದ
ಪ್ರೇರೇ-ಪಿ-ತ-ರಾಗಿ
ಪ್ರೇರೇ-ಪಿ-ಸುವ
ಪ್ರೇರೇ-ಪಿ-ಸು-ವುದು
ಪ್ರೊಫೆ-ಸ-ರಿ-ಗಿಂತ
ಪ್ರೋತ್ಸಾ-ಹಿಸಿ
ಪ್ಲೇಗು
ಫಲ
ಫಲ-ಕಾ-ರಿ-ಯಾ-ಗು-ವುದೋ
ಫಲ-ದಾ-ಯಕ
ಫಲವೂ
ಫಿಲಿಪ್
ಬಂಡೆ-ಯಂತೆ
ಬಂದ
ಬಂದಂತೆ
ಬಂದಂ-ತೆಲ್ಲ
ಬಂದರೆ
ಬಂದಿ-ರು-ವುದು
ಬಂದು
ಬಂಧನ
ಬಂಧ-ನ-ಗ-ಳ-ಲ್ಲೆಲ್ಲ
ಬಂಧು-ಗಳನ್ನು
ಬಂಧು-ಗಳು
ಬಗೆ
ಬಗೆಯ
ಬಗೆ-ಹ-ರಿ-ಸಿ-ಕೊ-ಳ್ಳುವ
ಬಗ್ಗಿ-ಸ-ಲಾ-ರದು
ಬಚ್ಚಲು
ಬಟ್ಟಲು
ಬಡ
ಬಡ-ವರು
ಬಣ್ಣ
ಬಣ್ಣ-ವನ್ನೂ
ಬದ-ಲಾ-ಯಿ-ಸ-ಬ-ಲ್ಲದು
ಬದ-ಲಾ-ಯಿ-ಸ-ಬ-ಹುದು
ಬದ-ಲಾ-ಯಿ-ಸಿ-ರು-ವುದನ್ನು
ಬದಲು
ಬದು-ಕಿ-ರು-ವರು
ಬದು-ಕುವು
ಬದ್ಧ
ಬದ್ಧ-ರಾ-ಗ-ಬೇಡಿ
ಬಯ-ಕೆ-ಗ-ಳೆಲ್ಲ
ಬಯ-ಸು-ತ್ತೇ-ನೆ-ನೀವು
ಬಯ-ಸು-ವರೋ
ಬಯ-ಸು-ವೆನು
ಬರ-ಗಾಲ
ಬರ-ಬ-ರುತ್ತಾ
ಬರ-ಬೇಕು
ಬರ-ಲಾ-ರದು
ಬರಿ
ಬರಿದೇ
ಬರಿಯ
ಬರುವ
ಬರು-ವನೋ
ಬರು-ವು-ದಕ್ಕೆ
ಬರು-ವು-ದಿಲ್ಲ
ಬರು-ವು-ದಿ-ಲ್ಲವೇ
ಬರು-ವು-ದಿ-ಲ್ಲವೋ
ಬರು-ವುದು
ಬರು-ವು-ದೆಂದು
ಬರು-ವುವು
ಬರೆ-ಯು-ವು-ದ-ರಲ್ಲಿ
ಬಲ
ಬಲ-ಗೈ-ಯಿಂದ
ಬಲ-ಗೊ-ಳಿಸ
ಬಲ-ತ್ಕಾರ
ಬಲವಾ
ಬಲಾ-ಢ್ಯನೇ
ಬಲಾ-ಢ್ಯ-ರಾಗಿ
ಬಲಾ-ಢ್ಯ-ರಾ-ದ-ವರು
ಬಲಾ-ತ್ಕ-ರಿ-ಸ-ಬೇಡಿ
ಬಲಿ
ಬಲಿ-ಕೊ-ಡ-ಬೇ-ಕಾ-ಗಿಲ್ಲ
ಬಲಿ-ಯಾ-ದರೂ
ಬಲಿಷ್ಠ
ಬಲಿ-ಷ್ಠ-ರಾ-ಗ-ಬೇಕು
ಬಲ್ಲರು
ಬಲ್ಲರೋ
ಬಳಿ-ದು-ಕೊಂ-ಡಿ-ದ್ದರೂ
ಬಹಳ
ಬಹು
ಬಹು-ಕಾ-ಲದ
ಬಹು-ಕಾ-ಲ-ದಿಂ-ದಲೂ
ಬಹುದು
ಬಹು-ದು-ಅದೇ
ಬಾ
ಬಾಗಿ
ಬಾಗು-ವುದು
ಬಾಯನ್ನು
ಬಾಯಿ
ಬಾಯಿಂದ
ಬಾಯಿ-ಯನ್ನು
ಬಾರದ
ಬಾರ-ದ-ವನು
ಬಾರ-ದ-ವ-ರೆಂದು
ಬಾರದು
ಬಾರ-ದು-ದನ್ನೇ
ಬಾರರು
ಬಾರಿ
ಬಾಲ್ಯ-ದಿಂ-ದಲೂ
ಬಾಳ-ಚಾ-ಳಿ-ಯಾ-ಗಿದೆ
ಬಾಳ-ಲಾ-ರದು
ಬಾಳಲು
ಬಾಳಿ
ಬಾಳಿಕೆ
ಬಾಳು
ಬಾಳು-ತ್ತಿರು
ಬಾಳು-ತ್ತಿ-ರುವ
ಬಾಳು-ತ್ತಿ-ರು-ವಿರಿ
ಬಾಳು-ವೆ-ಯದು
ಬಾವಿ-ಯನ್ನು
ಬಾಹ್ಯ
ಬಾಹ್ಯ-ರೂ-ಪ-ದ-ಲ್ಲಿ-ರು-ವುದು
ಬಿಡದೆ
ಬಿಡಿ
ಬಿಡಿ-ಸು-ವುವು
ಬಿಡುವ
ಬಿಡು-ವು-ದಿಲ್ಲ
ಬಿಡು-ವು-ದಿ-ಲ್ಲವೆ
ಬಿತ್ತು-ವು-ದಕ್ಕೆ
ಬಿದ್ದ
ಬಿದ್ದಿದೆ
ಬಿದ್ದು
ಬಿದ್ದು-ಹೋ-ಗದೆ
ಬಿಳಿಯ
ಬಿಸುಟು
ಬೀಜ-ನಿಗೆ
ಬೀಜ-ವನ್ನು
ಬೀದಿ-ಹೋ-ಕ-ನಿ-ಗಿಂತ
ಬೀರು-ವನು
ಬೀಳ-ದಿ-ರಲಿ
ಬೀಳಲು
ಬೀಳುವ
ಬೀಳು-ವು-ದಲ್ಲ
ಬೀಳು-ವು-ದಿಲ್ಲ
ಬೀಳು-ವುದು
ಬೀಸಿ-ಬ-ರುವ
ಬುದ್ದಿ-ಮಾನ್
ಬುದ್ಧಾ-ವ-ತಾ-ರ-ದಲ್ಲಿ
ಬುದ್ಧಿ
ಬುದ್ಧಿ-ವಂ-ತನೋ
ಬುದ್ಧಿ-ವಂ-ತಿಕೆ
ಬುದ್ಧಿ-ವಾ-ದವೆ
ಬೆಂಕಿ
ಬೆಂಕಿಯ
ಬೆಂಕಿ-ಯನ್ನು
ಬೆಂದು
ಬೆಟ್ಟ-ವಾ-ಗು-ವುದು
ಬೆರ-ಗು-ಗೊ-ಳಿ-ಸುವ
ಬೆರ-ಳನ್ನು
ಬೆರಳು
ಬೆಲೆ-ಯಿಲ್ಲ
ಬೆಳ-ಕನ್ನು
ಬೆಳ-ವ-ಣಿ-ಗೆಯೇ
ಬೆಸ್ತನ
ಬೇಕಾ-ಗಿದೆ
ಬೇಕಾ-ಗಿ-ದ್ದರೆ
ಬೇಕಾ-ಗಿ-ದ್ದಾರೆ
ಬೇಕಾ-ಗಿ-ರು-ವುದು
ಬೇಕಾ-ಗಿ-ರು-ವುದೇ
ಬೇಕಾ-ಗಿಲ್ಲ
ಬೇಕಾದ
ಬೇಕಾ-ದರೂ
ಬೇಕಾ-ದರೆ
ಬೇಕು
ಬೇಗನೆ
ಬೇಡ
ಬೇಡ-ವಾ-ಗಿ-ದೆಯೇ
ಬೇಡಿ
ಬೇಡುವು
ಬೇರಾವ
ಬೊಗ-ಸೆ-ಗಂಜಿ
ಬೋಧ-ನಾ-ರೀ-ತಿಯ
ಬೋಧಿಸ
ಬೋಧಿಸಿ
ಬೋಧಿ-ಸಿ-ದ್ದಾರೆ
ಬೋಧಿ-ಸಿದ್ದು
ಬೋಧಿ-ಸುತ್ತ
ಬೋಧಿ-ಸು-ವುದು
ಬೋಧಿ-ಸು-ವುದೇ
ಬ್ರಹ್ಮ-ಚ-ರ್ಯ-ಶೀ-ಲ-ವಿ-ಲ್ಲದೆ
ಬ್ರಹ್ಮ-ಪು-ತ್ರದ
ಬ್ರಾಹ್ಮಣ
ಬ್ರಾಹ್ಮ-ಶಕ್ತಿ
ಭಂಗ
ಭಂಡಾ-ರ-ಗಳೇ
ಭಕ್ತ
ಭಕ್ತನ
ಭಕ್ತರೇ
ಭಕ್ತಿ
ಭಕ್ತಿಯ
ಭಕ್ತಿ-ಯನ್ನು
ಭಕ್ತಿ-ಯಲ್ಲ
ಭಕ್ತಿ-ಯಲ್ಲಿ
ಭಕ್ತಿ-ಯೋಗ
ಭಕ್ತಿ-ಯೋ-ಗ-ದಲ್ಲಿ
ಭಕ್ತಿ-ಯೋ-ಗ-ವೆಂದರೆ
ಭಕ್ತೋ-ತ್ತ-ಮನು
ಭಕ್ಷ್ಯ-ಭೋ-ಜ-ನಾ-ದಿ-ಗಳಿಂದ
ಭಗ-ವಂತ
ಭಗ-ವಂ-ತನ
ಭಗ-ವಂ-ತ-ನನ್ನು
ಭಗ-ವಂ-ತ-ನಿಗೆ
ಭಗ-ವಂ-ತ-ನಿಗೇ
ಭಗ-ವಂ-ತನೆ
ಭಗ-ವ-ದ್ಭ-ಕ್ತರ
ಭಗ-ವ-ದ್ಭ-ಕ್ತರು
ಭಯಂ-ಕರ
ಭಯಾ-ನಕ
ಭರತ
ಭರ-ತ-ಖಂಡ
ಭರ-ತ-ಖಂ-ಡಕ್ಕೆ
ಭರ-ತ-ಖಂ-ಡದ
ಭರ-ತ-ಖಂ-ಡ-ದಲ್ಲಿ
ಭರ-ತ-ಮಾತೆ
ಭರ-ತ-ವ-ರ್ಷದ
ಭರ-ವಸೆ
ಭವ-ರೋ-ಗ-ವನ್ನು
ಭವ-ಸಾ-ಗ-ರ-ವನ್ನು
ಭವಿಷ್ಯ
ಭವಿ-ಷ್ಯ-ನಿ-ರ್ಮಾ-ಪ-ಕರು
ಭವಿ-ಷ್ಯ-ವನ್ನು
ಭವ್ಯ
ಭಾಗ-ದ-ಲ್ಲಿಯೂ
ಭಾಗ್ಯ
ಭಾಗ್ಯ-ವಂತ
ಭಾಗ್ಯ-ಶಾಲಿ
ಭಾರತ
ಭಾರ-ತ-ಭೂಮಿ
ಭಾರ-ತ-ವ-ರ್ಷ-ದೋ-ಪಾ-ದಿ-ಯಲ್ಲಿ
ಭಾರ-ತ-ವೇ-ಳಲಿ
ಭಾರ-ತಾಂಬೆ
ಭಾರ-ತೀಯ
ಭಾರ-ತೀ-ಯ-ರಿಗೆ
ಭಾರ-ತೀ-ಯ-ರಿರಾ
ಭಾರ-ತೀ-ಯರು
ಭಾರ-ತೀ-ಯ-ರೆಮ್ಮ
ಭಾರ-ದಲ್ಲಿ
ಭಾರ-ವೆಲ್ಲ
ಭಾವ
ಭಾವಕ್ಕೆ
ಭಾವ-ಗಳನ್ನು
ಭಾವ-ಗ-ಳಿಗೆ
ಭಾವನೆ
ಭಾವ-ನೆ-ಗಳ
ಭಾವ-ನೆ-ಗಳನ್ನು
ಭಾವ-ನೆ-ಗ-ಳೆಲ್ಲ
ಭಾವ-ನೆಯ
ಭಾವ-ನೆ-ಯನ್ನು
ಭಾವ-ವನ್ನು
ಭಾವ-ವಿ-ರಲಿ
ಭಾವಿ
ಭಾವಿಸಿ
ಭಾವಿ-ಸಿದ್ದೆ
ಭಾವಿ-ಸಿ-ರುವ
ಭಾವಿ-ಸು-ವಿರಾ
ಭಾವಿ-ಸುವೆ
ಭಾವಿ-ಸು-ವೆ-ಯೇನು
ಭಾಷೆ-ಯನ್ನು
ಭಾಷೆ-ಯಲ್ಲಿ
ಭಾಸ-ವಾ-ಗು-ತ್ತದೆ
ಭಾಸ-ವಾ-ಗು-ವುದು
ಭಿಕಾ-ರಿ-ಗ-ಳಾ-ದರೆ
ಭಿಕ್ಷು-ಕ-ನಂತೆ
ಭಿಕ್ಷು-ಕ-ನಿಗೆ
ಭಿನ್ನ-ತೆ-ಯನ್ನೂ
ಭಿಸಿ-ದಾಗ
ಭುಜ-ಬಲ
ಭೂತ-ಕಾ-ಲದ
ಭೂತ-ಗ-ಳಾ-ಗಿ-ದ್ದೀರಿ
ಭೂದೇ-ವಿ-ಯಂತೆ
ಭೂಮಿ
ಭೂಮಿಯೇ
ಭೃತ್ಯ-ರಾ-ಗು-ವುದನ್ನು
ಭೇದಿಸಿ
ಭೋಗ
ಭೋಗ-ಕ್ಕಲ್ಲ
ಭೋಗ-ವಲ್ಲ
ಭೋಗ-ವೆಲ್ಲ
ಭ್ಯಾಸವೆ
ಭ್ರಾಂತಿ
ಭ್ರಾತೃ-ಗ-ಳಿರಾ
ಮಂಜೂ-ಷೆ-ಗಳನ್ನೂ
ಮಂಡಿ-ಸಿ-ರು-ವುದನ್ನು
ಮಂತ್ರೋ-ಚ್ಚಾ-ರಣೆ
ಮಂದ-ಹಾ-ಸ-ದಿಂದ
ಮಂದಿ
ಮಂದಿ-ರ-ವೇನೋ
ಮಕ್ಕಳ
ಮಕ್ಕ-ಳನ್ನು
ಮಕ್ಕ-ಳಿಗೆ
ಮಕ್ಕಳು
ಮಕ್ಕಳೇ
ಮಗ
ಮಗು
ಮಗು-ವಿ-ಗಾ-ದರೂ
ಮಗು-ವಿ-ನಲ್ಲಿ
ಮಗ್ನ-ರಾ-ಗಿ-ರು-ವರು
ಮಠ-ಗಳಿಂದ
ಮಡಿ
ಮಡಿದೆ
ಮತ-ಕ-ಲಹ
ಮತ-ದಲ್ಲೂ
ಮತ-ಭ್ರಾಂ-ತನ
ಮತ-ಭ್ರಾಂ-ತರು
ಮತ-ಭ್ರಾಂತಿ
ಮತ-ಭ್ರಾಂ-ತಿಯ
ಮತ-ವಾ-ಗಿ-ದೆಯೋ
ಮತ್ತಾ
ಮತ್ತಾವ
ಮತ್ತಾ-ವು-ದಾ-ದರೂ
ಮತ್ತಾ-ವುದೂ
ಮತ್ತು
ಮತ್ತು-ಅ-ನಾ-ಸ-ಕ್ತಿ-ಯನ್ನು
ಮತ್ತೆ
ಮತ್ತೆ-ಲ್ಲಿಯೂ
ಮತ್ತೆ-ಲ್ಲೂ-ಇ-ಲ್ಲದ
ಮತ್ತೊಂದು
ಮತ್ತೊ-ಬ್ಬನ
ಮತ್ತೊ-ಬ್ಬ-ನನ್ನು
ಮತ್ತೊ-ಬ್ಬ-ನಿಗೆ
ಮತ್ತೊ-ಬ್ಬರ
ಮತ್ತೊ-ಬ್ಬ-ರನ್ನು
ಮತ್ತೊ-ಬ್ಬ-ರನ್ನೂ
ಮತ್ತೊ-ಬ್ಬ-ರಿಗೆ
ಮತ್ತೊಮ್ಮೆ
ಮನ-ನ-ಮಾ-ಡ-ಬೇಡಿ
ಮನ-ಸ್ತಾಪ
ಮನ-ಸ್ಸ-ನ್ನಾ-ಗಲಿ
ಮನ-ಸ್ಸನ್ನು
ಮನ-ಸ್ಸಿಗೆ
ಮನ-ಸ್ಸಿನ
ಮನ-ಸ್ಸಿ-ನ-ಲ್ಲಿ-ರು-ವುದನ್ನು
ಮನ-ಸ್ಸಿ-ನಿಂದ
ಮನಸ್ಸು
ಮನುಜ
ಮನು-ಜರು
ಮನುಷ್ಯ
ಮನು-ಷ್ಯನ
ಮನು-ಷ್ಯ-ನನ್ನು
ಮನು-ಷ್ಯ-ನಲ್ಲಿ
ಮನು-ಷ್ಯ-ನಿಗೆ
ಮನು-ಷ್ಯ-ರಾಗಿ
ಮನು-ಷ್ಯರು
ಮನು-ಷ್ಯ-ರೆಂದು
ಮನೆಗೂ
ಮನೆಗೆ
ಮನೆ-ಯನ್ನು
ಮನೆ-ಯಿಂದ
ಮನೋ-ನಿ-ಗ್ರಹ
ಮಮ್ಮಿ-ಗ-ಳಾ-ಗಿ-ದ್ದೀರಿ
ಮಯ
ಮರಣ
ಮರ-ಣ-ಕಾ-ಲ-ದ-ಲ್ಲಿಯೂ
ಮರ-ಣ-ಗಳ
ಮರಳಿ
ಮರೀ-ಚಿ-ಕೆ-ಗಳು
ಮರುಕ
ಮರುಗಿ
ಮರು-ಗು-ತ್ತಿ-ದೆಯೇ
ಮರು-ಗು-ತ್ತಿ-ರು-ವಿರಾ
ಮರು-ಗು-ವುದೋ
ಮರು-ದ-ನಿ-ಯಾಗಿ
ಮರು-ಮ-ರೀ-ಚಕೆ
ಮರು-ಮಾ-ತಿ-ಲ್ಲದೆ
ಮರೆ-ತು-ಬಿ-ಟ್ಟರೆ
ಮರೆ-ಮಾ-ಚು-ವಂತೆ
ಮರೆ-ಯ-ದಿರಿ
ಮರೆ-ಯ-ದಿ-ರಿ-ನಿಮ್ಮ
ಮರೆ-ಯ-ದಿ-ರಿ-ಶೂ-ದ್ರರು
ಮಲ-ಗಿ-ರುವ
ಮಳಿ-ಗೆ-ಗಳಿಂದ
ಮಹ-ತ್ಕಾರ್ಯ
ಮಹ-ತ್ಕಾ-ರ್ಯ-ಗಳನ್ನು
ಮಹ-ತ್ಕಾ-ರ್ಯ-ಗಳೂ
ಮಹ-ತ್ವ-ಗೀತೆ
ಮಹದಾ
ಮಹ-ದಾ-ಲೋ-ಚನೆ
ಮಹ-ಮ್ಮ-ದೀಯ
ಮಹ-ರ್ಷಿ-ಕು-ಲ-ಸಂ-ಜಾತ
ಮಹಾ
ಮಹಾ-ಕಾರ್ಯ
ಮಹಾ-ಕಾ-ರ್ಯ-ಗಳು
ಮಹಾ-ಕಾ-ರ್ಯ-ವನ್ನು
ಮಹಾ-ಕಾ-ರ್ಯ-ವನ್ನೂ
ಮಹಾ-ಗು-ಣ-ವನ್ನು
ಮಹಾ-ಗು-ಣವೇ
ಮಹಾ-ತ್ಮನ
ಮಹಾ-ತ್ಮ-ರೆಂದು
ಮಹಾ-ತ್ಮ-ರೆಲ್ಲ
ಮಹಾ-ತ್ಮ್ಯೆ-ಯಲ್ಲಿ
ಮಹಾ-ತ್ಯಾ-ಗ-ದಿಂದ
ಮಹಾ-ತ್ಯಾ-ಗ-ವನ್ನು
ಮಹಾ-ಪ-ದ-ವನ್ನು
ಮಹಾ-ಪಾ-ತ-ಕ-ವನ್ನು
ಮಹಾ-ಪಾಪ
ಮಹಾ-ಪಾ-ಪವೇ
ಮಹಾ-ಪು-ರುಷ
ಮಹಾ-ಪು-ರು-ಷ-ರಾಗಿ
ಮಹಾ-ಪು-ರು-ಷರೇ
ಮಹಾ-ಪ್ರ-ಶ್ನೆ-ಗಳನ್ನು
ಮಹಾ-ಪ್ರಾ-ರಂ-ಭ-ಗಾನ
ಮಹಾ-ಮ-ಹಿ-ಮ-ನೆಂ-ಬುದು
ಮಹಾ-ಮ-ಹಿ-ಮರು
ಮಹಾ-ವೀ-ರನ
ಮಹಾ-ವ್ಯ-ಕ್ತಿ-ಗ-ಳಾ-ಗ-ಲೇ-ಬೇಕು
ಮಹಾ-ವ್ಯಾ-ಧಿಗೆ
ಮಹಾ-ಸಾ-ಹಸ
ಮಹಾ-ಸಾ-ಹ-ಸ-ಕಾರ್ಯ
ಮಹಿಮೆ
ಮಹಿ-ಮೆಯ
ಮಹಿ-ಮೆ-ಯನ್ನು
ಮಹೇ-ಶ್ವ-ರನು
ಮಹೋ-ತ್ಪಾತ
ಮಾಂಸ
ಮಾಂಸ-ಖಂಡ
ಮಾಂಸ-ಖಂ-ಡ-ಗಳು
ಮಾಡ
ಮಾಡ-ದಿ-ರಲಿ
ಮಾಡದೇ
ಮಾಡ-ಬ-ಲ್ಲದು
ಮಾಡ-ಬ-ಲ್ಲರು
ಮಾಡ-ಬ-ಲ್ಲವೆ
ಮಾಡ-ಬ-ಲ್ಲಿರಿ
ಮಾಡ-ಬಲ್ಲೆ
ಮಾಡ-ಬ-ಹುದು
ಮಾಡ-ಬೇ-ಕಾದ
ಮಾಡ-ಬೇ-ಕಾ-ದರೆ
ಮಾಡ-ಬೇಕು
ಮಾಡ-ಬೇ-ಕೆಂದು
ಮಾಡ-ಬೇಡಿ
ಮಾಡ-ಲಾ-ಗು-ವು-ದಿಲ್ಲ
ಮಾಡ-ಲಾ-ರದೋ
ಮಾಡ-ಲಾರೆ
ಮಾಡಲು
ಮಾಡಿ
ಮಾಡಿ-ಕೊಂಡಿ
ಮಾಡಿ-ಕೊಂಡು
ಮಾಡಿ-ಕೊ-ಳ್ಳದೆ
ಮಾಡಿ-ಕೊ-ಳ್ಳ-ಬಲ್ಲ
ಮಾಡಿ-ಕೊ-ಳ್ಳ-ಬೇಕು
ಮಾಡಿ-ಕೊಳ್ಳು
ಮಾಡಿದ
ಮಾಡಿ-ದಂತೆ
ಮಾಡಿ-ದರೆ
ಮಾಡಿದೆ
ಮಾಡಿ-ದ್ದರೂ
ಮಾಡಿದ್ದು
ಮಾಡಿರಿ
ಮಾಡಿ-ರು-ವನೋ
ಮಾಡಿ-ರು-ವರೋ
ಮಾಡು
ಮಾಡು-ತ್ತಾನೆ
ಮಾಡು-ತ್ತಿ-ರುವ
ಮಾಡು-ತ್ತಿ-ರು-ವುದೋ
ಮಾಡು-ತ್ತಿ-ರು-ವೆನು
ಮಾಡುವ
ಮಾಡು-ವ-ನೆಂ-ಬು-ದಕ್ಕೆ
ಮಾಡು-ವರು
ಮಾಡು-ವರೋ
ಮಾಡು-ವ-ವ-ನಿಗೂ
ಮಾಡುವು
ಮಾಡು-ವು-ದಕ್ಕೆ
ಮಾಡು-ವು-ದಿಲ್ಲ
ಮಾಡು-ವುದು
ಮಾಡು-ವು-ದೆಂದು
ಮಾಡು-ವುದೇ
ಮಾಡು-ವುದೋ
ಮಾಡು-ವೆನು
ಮಾಡು-ವೆವು
ಮಾಡೆಂದು
ಮಾಡೋಣ
ಮಾತ-ನಾ-ಡ-ಬೇಡ
ಮಾತ-ನಾ-ಡಲು
ಮಾತ-ನಾಡು
ಮಾತ-ನಾ-ಡು-ತ್ತಾನೆ
ಮಾತ-ನಾ-ಡುವ
ಮಾತ-ನಾ-ಡು-ವು-ದ-ರಿಂದ
ಮಾತನ್ನೇ
ಮಾತಲ್ಲ
ಮಾತು
ಮಾತೃ-ಭೂಮಿ
ಮಾತೆಯ
ಮಾತೆ-ಯ-ರೆಂ-ದರೆ
ಮಾತ್ರ
ಮಾತ್ರವೇ
ಮಾನ
ಮಾನವ
ಮಾನ-ವ-ಕೋ-ಟಿಗೆ
ಮಾನ-ವ-ಕೋ-ಟಿಯ
ಮಾನ-ವ-ಜ-ನ್ಮ-ವನ್ನು
ಮಾನ-ವ-ನನ್ನು
ಮಾನ-ವನೇ
ಮಾನ-ವ-ರನ್ನು
ಮಾನ-ವ-ರನ್ನೂ
ಮಾನ-ವ-ರೂ-ಪನ್ನು
ಮಾನ-ವರೇ
ಮಾನ-ಸಿಕ
ಮಾನ-ಸಿ-ಕ-ವಾ-ಗಾ-ಗಲಿ
ಮಾಯ-ವಾಗಿ
ಮಾಯ-ವಾ-ಗು-ತ್ತ-ಲಿವೆ
ಮಾಯ-ವಾ-ಗು-ತ್ತಿವೆ
ಮಾಯ-ವಾ-ಗು-ವುದು
ಮಾಯ-ವಾ-ಗು-ವುವು
ಮಾಯ-ವಾದ
ಮಾಯ-ವಾ-ದ-ಕ-ಡ-ಲಿನ
ಮಾಯಾ
ಮಾಯೆ
ಮಾರ್ಗ-ವಿಲ್ಲ
ಮಾರ್ಗವೇ
ಮಿಂಚು-ತ್ತಿ-ರು-ವಾ-ಗಲೂ
ಮಿಗಿಲು
ಮಿತಿ-ಮೀ-ರಿತು
ಮಿತಿ-ಯೊ-ಳಗೆ
ಮಿಥ್ಯ-ದೊಂ-ದಿಗೆ
ಮಿಶ್ರ-ವಾ-ಗಿ-ರು-ವುದು
ಮೀರಲು
ಮೀರಿ
ಮೀರಿದ
ಮೀರಿ-ರುವ
ಮೀರಿ-ರು-ವ-ರಾರೂ
ಮೀಸ
ಮುಂಚೆ
ಮುಂತಾ-ದು-ವೆಲ್ಲಾ
ಮುಂದಕ್ಕೆ
ಮುಂದಿದೆ
ಮುಂದಿನ
ಮುಂದು-ವರಿ
ಮುಂದು-ವ-ರಿ-ಯ-ಬ-ಲ್ಲೆಯೋ
ಮುಂದು-ವ-ರಿ-ಯ-ಬೇ-ಕಾ-ದರೆ
ಮುಂದು-ವ-ರಿ-ಯ-ಲಾ-ರಿರಿ
ಮುಂದು-ವ-ರಿಯು
ಮುಂದು-ವ-ರಿ-ಯುವ
ಮುಂದು-ವ-ರಿ-ಯು-ವುದನ್ನು
ಮುಂದೆ
ಮುಂಬರಿ
ಮುಕ್ಕ-ಳಿ-ಸು-ವುದು
ಮುಕ್ತ-ರ-ನ್ನಾಗಿ
ಮುಕ್ತ-ರಾಗಿ
ಮುಕ್ತ-ರಾಗು
ಮುಕ್ತಾತ್ಮ
ಮುಕ್ತಿ
ಮುಕ್ತಿ-ಯನ್ನು
ಮುಖತಃ
ಮುಖ-ರಾ-ಗು-ವು-ದಿಲ್ಲ
ಮುಖ್ಯ
ಮುಖ್ಯ-ವೆಂದು
ಮುಚ್ಚಿ
ಮುಟ್ಟ-ಬೇಡ
ಮುಟ್ಟ-ಬೇಡಿ
ಮುಡಿ-ಪಾ-ಗಿ-ಡ-ಲಾ-ರರು
ಮುಡು-ಪಾ-ಗಿಡಿ
ಮುಷ್ಟಿ
ಮೂಟೆ-ಕಟ್ಟಿ
ಮೂಡಲಿ
ಮೂಢ-ನಂ-ಬಿ-ಕೆ-ಯನ್ನು
ಮೂಢರು
ಮೂರ-ನೆ-ಯದು
ಮೂರು
ಮೂರೂ
ಮೂರ್ಖ-ರಿಗೆ
ಮೂರ್ತಿ
ಮೂಲ
ಮೂಲಕ
ಮೂಲ-ಕಾ-ರಣ
ಮೂಲ-ವೆಂ-ಬು-ದನ್ನು
ಮೂವ-ತ್ತು-ಕೋಟಿ
ಮೃಗ-ಗ-ಳಂತೆ
ಮೃಗೀಯ
ಮೃತ-ಗ-ತ-ಭಾ-ರ-ತದ
ಮೃತ-ರಂ-ತಿ-ರು-ವರು
ಮೃತ್ಯು
ಮೃತ್ಯು-ವನ್ನು
ಮೃತ್ಯು-ವ-ಶ-ರಾ-ದ-ವರು
ಮೆಟ್ಟ-ಲನ್ನು
ಮೆಟ್ಟಿ
ಮೆದಳು
ಮೆದು
ಮೆದು-ಳನ್ನು
ಮೆದು-ಳಿ-ನೊ-ಳಗೆ
ಮೆದುಳು
ಮೆಲ್ಲು-ತ್ತಿ-ದ್ದರೆ
ಮೇಘ
ಮೇಧಾ-ವಿ-ಗ-ಳಾದ
ಮೇರೆ
ಮೇಲ-ಲ್ಲವೇ
ಮೇಲಾ-ಗುವ
ಮೇಲಾ-ದು-ದೆಂದು
ಮೇಲಿ-ಡುವ
ಮೇಲಿ-ರು-ವುದನ್ನು
ಮೇಲು
ಮೇಲು-ಗಡೆ
ಮೇಲೆ
ಮೇಲೆಂದು
ಮೇಲೆ-ತ್ತು-ವನೋ
ಮೇಲೆದ್ದು
ಮೇಲೆ-ನಿಂತ
ಮೇಲೆ-ಮೇಲೆ
ಮೇಲೆಲ್ಲ
ಮೇಲೇ-ಳ-ಲಾ-ರದು
ಮೇಲೇ-ಳಲು
ಮೇಲೇ-ಳು-ತ್ತಿದೆ
ಮೇಲೇ-ಳು-ವನು
ಮೇಲೊಂದು
ಮೇಲ್ಮೆ-ಗಾಗಿ
ಮೈಕೊ-ಡವಿ
ಮೈದೋ-ರಲಿ
ಮೈಸೂರು
ಮೊಗ್ಗು-ಗಳ
ಮೊದಲ
ಮೊದ-ಲನೆ
ಮೊದ-ಲ-ನೆಯ
ಮೊದ-ಲಾಗಿ
ಮೊದ-ಲಾ-ಗಿ-ರು-ವುದೋ
ಮೊದಲು
ಮೊಬ್ಬಾ
ಮೋಕ್ಷ
ಮೋಚಿ
ಮೋಡ-ವನ್ನು
ಮೋಸ
ಮೋಹ
ಮೌಢ್ಯ-ತೆಗೆ
ಮೌನ
ಮೌನ-ದಿಂದ
ಮೌನ-ವಾಗಿ
ಮ್ಲೇಚ್ಛ-ರೆಂಬ
ಯಂತೆ
ಯಂತ್ರದ
ಯಂತ್ರ-ದಂತೆ
ಯಂತ್ರ-ವಾಗಿ
ಯಜ್ಞ-ಗಳು
ಯಜ್ಞ-ಪ-ಶು-ವಲ್ಲಿ
ಯಜ್ಞ-ಶಿ-ಶು-ಗ-ಳಾಗಿ
ಯತ್ನ-ವೊಂದೇ
ಯತ್ನಿಸ
ಯತ್ನಿ-ಸ-ಬೇಕು
ಯತ್ನಿ-ಸ-ಬೇಡಿ
ಯತ್ನಿಸಿ
ಯತ್ನಿಸು
ಯದು
ಯನ್ನು
ಯನ್ನೆಲ್ಲ
ಯನ್ನೇ
ಯಲ್ಲಿ-ರು-ವುದು
ಯಾಗ
ಯಾಗಿ-ದ್ದರೆ
ಯಾಗು-ವುದು
ಯಾತ-ನಾ-ಮ-ಯವೇ
ಯಾತ್ರೆ-ಗಳನ್ನು
ಯಾದರೂ
ಯಾದ-ವ-ಗಿರಿ
ಯಾರ
ಯಾರನ್ನು
ಯಾರನ್ನೂ
ಯಾರಲ್ಲಿ
ಯಾರಾ-ದರೂ
ಯಾರಿಂದ
ಯಾರಿಗೂ
ಯಾರಿಗೆ
ಯಾರಿ-ಗೋ-ಸುಗ
ಯಾರು
ಯಾರೂ
ಯಾರೊಂ-ದಿಗೂ
ಯಾವ
ಯಾವನು
ಯಾವಾ
ಯಾವಾ-ಗಲೂ
ಯಾವುದನ್ನು
ಯಾವು-ದನ್ನೂ
ಯಾವು-ದ-ರೊಂ-ದಿಗೂ
ಯಾವು-ದಾ-ದರೂ
ಯಾವುದು
ಯಾವುದೂ
ಯಾವುದೋ
ಯಿಂದ
ಯಿರಿ
ಯಿಲ್ಲ
ಯುಕ್ತ
ಯುಗ-ಯು-ಗ-ಗಳ
ಯುತ್ತೀ-ರೇನು
ಯುರೋ-ಪು-ಖಂ-ಡವೂ
ಯುಳ್ಳ-ವ-ರ-ನ್ನಾಗಿ
ಯುವ-ಕರ
ಯುವ-ಕರು
ಯುವ-ಕರೆ
ಯೆಂಬ
ಯೊಂದಿಗೆ
ಯೊಂದು
ಯೊಬ್ಬ
ಯೋಗಾ-ಭ್ಯಾಸ
ಯೋಚ-ನೆಯೇ
ಯೋಚಿ-ಸ-ಬೇಡ
ಯೋಚಿ-ಸ-ಬೇಡಿ
ಯೋಚಿಸಿ
ಯೋಚಿ-ಸುತ್ತಾ
ಯೌವ-ನದ
ರಂಗಕ್ಕೆ
ರಕ್ತ
ರಕ್ತ-ಗತ
ರಕ್ತ-ಬಂ-ಧು-ಗ-ಳಾದ
ರಕ್ತ-ಮಾಂ-ಸ-ವಿ-ಹೀನ
ರಕ್ತ-ವಿ-ದ್ದರೆ
ರಕ್ಷಿ-ಸ-ಬೇಕು
ರಕ್ಷಿ-ಸು-ವುದು
ರತ್ನ-ದುಂ-ಗು-ರ-ಗಳನ್ನೂ
ರತ್ನ-ದುಂ-ಗು-ರ-ಗಳು
ರದೋ
ರನ್ನಾಗಿ
ರನ್ನೂ
ರಹಸ್ಯ
ರಹ-ಸ್ಯ-ದಂತ
ರಹ-ಸ್ಯ-ದಲ್ಲಿ
ರಾಜ-ಕೀಯ
ರಾಜ-ಸಿಕ
ರಾಜಿ
ರಾಜಿ-ಮಾ-ಡಿ-ಕೊಳ್ಳ
ರಾತ್ರಿ
ರಾದಾಗ
ರಾಮ-ಚಂ-ದ್ರನ
ರಾಮಾ-ವ-ತಾ-ರ-ದಲ್ಲಿ
ರಾಳ-ವನ್ನು
ರಾಷ್ಟ್ರದ
ರಿಗಾಗಿ
ರಿಗೆ
ರಿರಾ
ರೀತಿ
ರೀತಿ-ಯಲ್ಲಿ
ರುದ್ರ-ಮು-ಖದ
ರೂಪಿ
ರೂಪು-ಗೊ-ಳಿಸಿ
ರೂಪು-ಗೊ-ಳ್ಳಲಿ
ರೆಂದು
ರೆಕ್ಕೆಯ
ರೋಗ-ದಿಂದ
ರೋಗ-ವಿ-ರುವ
ರೋಗಿ-ಗಳಲ್ಲಿ
ರೋಗಿ-ಗಳು
ರೋಗಿ-ಯಂತೆ
ಲಂಘಿ-ಸಿ-ದನು
ಲಕ್ಷಣ
ಲಕ್ಷಿ-ಸ-ಬೇಡಿ
ಲಕ್ಷ್ಮಿ
ಲಭಿ-ಸಿ-ರ-ಲಿಲ್ಲ
ಲಭಿ-ಸು-ವುದು
ಲಲಿತ
ಲಾಗಿ
ಲಾಗಿದೆ
ಲಾಗು-ವು-ದಿಲ್ಲ
ಲಾಭಕ್ಕೆ
ಲಾಭ-ದಾ-ಯಕ
ಲುಪ್ತ-ಪಂ-ಥದ
ಲೆಕ್ಕಿ-ಸ-ಬೇಡಿ
ಲೆಕ್ಕಿ-ಸು-ವರು
ಲೋಕ-ವನ್ನೇ
ಲೋಚ-ನೆ-ಗಳು
ಲೋಳೆ-ಮೀ-ನ-ನಂತೆ
ಲ್ಲರೂ
ಲ್ಲರೂ-ಸ-ಪ್ರಾ-ಣ-ರೆಂದು
ಲ್ಲಿದೆಯೇ
ಲ್ಲಿರು-ವನು
ಳಲ್ಲ
ವಂದನೆ
ವಜ್ರ-ಸಾ-ಣೆ-ಯಾ-ಗಲಿ
ವಜ್ರಾ-ಘಾತ
ವನ-ಕು-ಟೀ-ರ-ಗಳಿಂದ
ವನ-ಗಳಿಂದ
ವನು
ವನ್ನಾ-ಗಲಿ
ವನ್ನು
ವನ್ನೂ
ವಯ-ಸ್ಸಾ-ದಂತೆ
ವಯ-ಸ್ಸಾ-ದಂ-ತೆಲ್ಲ
ವರ
ವರು
ವರು-ಷ-ವೆಲ್ಲ
ವರೆಗೆ
ವರೆ-ವಿಗೂ
ವರ್ಗ-ದ-ವರು
ವರ್ಣ
ವರ್ಣ-ಕ್ಕಿಂತ
ವರ್ಣ-ಗಳನ್ನು
ವರ್ಣ-ವಿ-ರಲಿ
ವರ್ತ-ಮಾ-ನ-ತೆಯು
ವರ್ಷ
ವರ್ಷ-ಗಳಿಂದ
ವರ್ಷ-ಗಳು
ವಳು
ವವ-ರಿಗೆ
ವಶ-ಪ-ಡಿ-ಸಿ-ಕೊ-ಳ್ಳ-ದಿ-ರಲಿ
ವಶ-ರಾದ
ವಸನ
ವಸ-ನ-ಧಾರಿ
ವಸ್ತು-ವಾ-ಗಿ-ರು-ವುದೋ
ವಸ್ತು-ಸಂ-ಗ್ರಹ
ವಹಿ-ಸಿ-ಕೊಳ್ಳಿ
ವಾಗದೆ
ವಾಗ-ಬೇಕು
ವಾಗಿ
ವಾಗು-ತ್ತಿ-ರಲಿ
ವಾಗು-ವಂತೆ
ವಾಗು-ವುದು
ವಾಣಿ-ಯೊಂದು
ವಾತಾ-ವ-ರಣ
ವಾದ
ವಾದಕ್ಕೆ
ವಾದರೂ
ವಾದಷ್ಟೂ
ವಾದುದು
ವಾರ-ಣಾಸಿ
ವಾಸ-ಮಾ-ಡುವ
ವಿಕ-ಸಿತ
ವಿಕಾ-ರ-ರೂ-ಪ-ವನ್ನು
ವಿಕಾಸ
ವಿಕಾ-ಸಕ್ಕೆ
ವಿಕಾ-ಸವೇ
ವಿಗ್ರ-ಹ-ದಲ್ಲಿ
ವಿಚ-ಕ್ಷ-ಣೆ-ಯಿಂದ
ವಿಚಾ-ರಿ-ಸದೆ
ವಿದ್ದರೆ
ವಿದ್ಯಾ
ವಿದ್ಯಾ-ಪ್ರ-ಚಾ-ರ-ವಾಗು
ವಿದ್ಯಾ-ಭ್ಯಾಸ
ವಿದ್ಯಾ-ಭ್ಯಾ-ಸದ
ವಿದ್ಯಾ-ಭ್ಯಾ-ಸ-ವನ್ನು
ವಿದ್ಯಾ-ಭ್ಯಾ-ಸವೆ
ವಿದ್ಯಾ-ಭ್ಯಾ-ಸ-ವೆಂದರೆ
ವಿದ್ಯಾ-ವಂತ
ವಿದ್ಯಾ-ವಂ-ತ-ನೆಂದು
ವಿದ್ಯುತ್
ವಿದ್ಯೆ
ವಿದ್ಯೆ-ಯನ್ನು
ವಿದ್ವಾಂ-ಸರು
ವಿಧದ
ವಿಧ-ದಿಂ-ದಲೂ
ವಿಧೇ-ಯತೆ
ವಿಧೇ-ಯ-ತೆ-ಯನ್ನು
ವಿಧೇ-ಯ-ತೆ-ಯ-ಲ್ಲಿದೆ
ವಿನೀ-ತರೋ
ವಿಭೂತಿ
ವಿಮರ್ಶೆ
ವಿಮೋ-ಚ-ನೆಗೆ
ವಿರ-ಚಿ-ಸು-ವನು
ವಿರಾ-ಡ್ರೂ-ಪಿ-ಯಾಗಿ
ವಿರಾ-ಮ-ವಿ-ಲ್ಲದೆ
ವಿರಿ
ವಿರೋ
ವಿರೋ-ಧ-ವಾದ
ವಿರೋ-ಧಿ-ಸ-ಲಾ-ರದು
ವಿರೋ-ಧಿ-ಸಿ-ದರೂ
ವಿಲ್ಲ
ವಿಲ್ಲದೇ
ವಿವಾಹ
ವಿವೇ-ಕ-ವಾಣಿ
ವಿವೇಕಾನಂದ
ವಿಶಾಲ
ವಿಶಾ-ಲ-ವಾ-ಗು-ವು-ದಕ್ಕೆ
ವಿಶಾ-ಲ-ವಾದ
ವಿಶ್ವ-ಕೋ-ಶ-ಗಳೇ
ವಿಶ್ವದ
ವಿಶ್ವ-ದೆ-ದುರು
ವಿಷ-ಣ್ಣ-ವಾ-ಗಿ-ದ್ದರೂ
ವಿಷ-ದಂತೆ
ವಿಷಯ
ವಿಷ-ಯ-ಗಳನ್ನು
ವಿಷ-ಯ-ಗಳು
ವಿಷ-ಯ-ದಲ್ಲಿ
ವಿಷ-ಯ-ವ-ಸ್ತು-ಗಳ
ವಿಷ-ಯ-ಸಂ-ಗ್ರಹ
ವಿಷ-ಯ-ಸಂ-ಗ್ರ-ಹ-ವಲ್ಲ
ವಿಷ-ಯ-ಸಂ-ಗ್ರ-ಹ-ವಾ-ದರೆ
ವಿಷ-ಯ-ಸಂ-ಗ್ರ-ಹವೆ
ವಿಷ-ವನ್ನು
ವಿಷ-ವೆಂದು
ವಿಹಾ-ರಿ-ಗ-ಳಿ-ಗೆಲ್ಲ
ವೀರ
ವೀರ-ಧರ್ಮ
ವೀರ-ನಿ-ರ್ಮಾ-ಪಕ
ವೀರ-ರಾಗಿ
ವೀರಾ-ತ್ಮರೆ
ವುದಕ್ಕೆ
ವುದನ್ನು
ವುದಲ್ಲ
ವುದು
ವುದೋ
ವೃತ್ತ-ಪ-ತ್ರಿಕಾ
ವೃಥಾ
ವೃಥೆ-ಪ-ಡು-ವಿರಾ
ವೃದ್ಧಾ
ವೆನು
ವೆವು
ವೇದಾಂತ
ವೇದಾಂ-ತದ
ವೇದಾಂತಿ
ವೇದಿ-ಕೆಯ
ವೇಳೆ
ವೇಶ್ಯಾಂ-ಗ-ನೆಯ
ವೈಜ್ಞಾ-ನಿಕ
ವೈಭ-ವ-ಯು-ತ-ವಾ-ದುದು
ವೈಮ-ನಸ್ಯ
ವೈರಾ-ಗ್ಯ-ಸೇವಾ
ವೈರಾ-ಗ್ಯ-ನಿಧಿ
ವೈರಿಗೂ
ವೈಶಾ-ಲ್ಯತೆ
ವೈಶ್ಯ-ಶಕ್ತಿ
ವ್ಯಕ್ತ-ಗೊ-ಳಿ-ಸು-ವುದು
ವ್ಯಕ್ತ-ಪ-ಡಿ-ಸಲು
ವ್ಯಕ್ತ-ಪ-ಡಿಸಿ
ವ್ಯಕ್ತ-ಪ-ಡಿ-ಸು-ವು-ದುಕ್ಕೆ
ವ್ಯಕ್ತ-ವಾ-ಗು-ವುದು
ವ್ಯಕ್ತ-ವಾ-ಗು-ವುದೋ
ವ್ಯಕ್ತಿ
ವ್ಯಕ್ತಿ-ಗಳ
ವ್ಯಕ್ತಿ-ಯಲ್ಲಿ
ವ್ಯಕ್ತಿ-ಯಾ-ಗಲಿ
ವ್ಯಕ್ತಿ-ರೂ-ಪಿ-ಯಾ-ಗಿ-ರುವ
ವ್ಯತ್ಯಾಸ
ವ್ಯಥೆಗೆ
ವ್ಯಥೆ-ಯಾ-ಗು-ವುದು
ವ್ಯದ
ವ್ಯಯ
ವ್ಯವ
ವ್ಯವ-ಸ್ಥಿತ
ವ್ಯಾಜ್ಯ
ವ್ಯಾಧಿ-ಯಿಂದ
ವ್ಯಾಪಾರ
ವ್ಯಾಪಾ-ರ-ವಾ-ಗಲೀ
ವ್ಯಾವ-ಹಾ-ರಿಕ
ಶಂಕರ
ಶಕ್ತಿ
ಶಕ್ತಿ-ಕ್ಷ-ಯ-ದಿಂದ
ಶಕ್ತಿ-ಗಿಂತ
ಶಕ್ತಿ-ದಾ-ಯಕ
ಶಕ್ತಿ-ದಾ-ಯ-ಕ-ವಾದ
ಶಕ್ತಿ-ಮೂಲ
ಶಕ್ತಿ-ಯನ್ನು
ಶಕ್ತಿ-ಯನ್ನೇ
ಶಕ್ತಿ-ಯ-ನ್ನೊ-ಳ-ಗೊ-ಳ್ಳಲು
ಶಕ್ತಿ-ಯಿಂದ
ಶಕ್ತಿ-ಯಿಲ್ಲ
ಶಕ್ತಿ-ಯಿ-ಲ್ಲ-ದ-ವನು
ಶಕ್ತಿಯೂ
ಶಕ್ತಿಯೇ
ಶಕ್ತಿ-ವ-ರ್ಧಕ
ಶಕ್ತಿ-ಶಾ-ಲಿ-ಗ-ಳಾ-ಗಿ-ರು-ವರು
ಶಕ್ತಿ-ಸಂ-ಜೀ-ವಿನಿ
ಶತ-ಮಾನ
ಶತ-ಮಾ-ನ-ಗಳ
ಶತ-ಮಾ-ನ-ಗ-ಳಿಂ-ದಲೂ
ಶತ-ಶ-ತ-ಮಾ-ನ-ಗಳ
ಶಪ-ಥ-ಮಾಡಿ
ಶಪಿ
ಶರ-ಣಾ-ಗ-ತ-ನಾಗು
ಶರೀರ
ಶವ-ದಂತೆ
ಶಾಂತ-ಚಿ-ತ್ತ-ರಾ-ಗಿದ್ದು
ಶಾಂತ-ವಾಗಿ
ಶಾಂತ-ವಾದ
ಶಾಂತಿ
ಶಾಲೆ
ಶಾಲೆಯ
ಶಾಸ್ತ್ರ
ಶಾಸ್ತ್ರಕ್ಕೆ
ಶಾಸ್ತ್ರ-ಜ್ಞಾ-ನ-ದಿಂದ
ಶಾಸ್ತ್ರ-ದಲ್ಲಿ
ಶಾಸ್ತ್ರಾಂ-ಧತೆ
ಶಿಕ್ಷಣ
ಶಿಕ್ಷ-ಣ-ವೆಂದರೆ
ಶಿಖ-ರ-ಗಳಿಂದ
ಶಿವನ
ಶಿವ-ನನ್ನು
ಶಿವ-ನಿಂದ
ಶಿಷ್ಯರು
ಶೀಲ
ಶೀಲಕ್ಕೆ
ಶೀಲ-ವನ್ನು
ಶುದ್ಧ
ಶುದ್ಧ-ಬುದ್ಧ
ಶುದ್ಧ-ರಾಗಿ
ಶುದ್ಧಿ-ಮಾ-ಡು-ವುದು
ಶುಭ
ಶುಭದ
ಶುಭವೇ
ಶುಭಾ-ವ-ಕಾಶ
ಶುಷ್ಕ-ವಾ-ಗು-ವುದೊ
ಶೂದ್ರ-ಶ-ಕ್ತಿ-ಯಾ-ಗು-ತ್ತದೆ
ಶೂನ್ಯ
ಶೂನ್ಯ-ದಲ್ಲಿ
ಶೇಕಡ
ಶೈಶ-ವದ
ಶೋಕ-ತಾ-ಪ-ಗಳು
ಶೋಚ-ನೆ-ಯಿಂದ
ಶ್ರದ್ಧಾ
ಶ್ರದ್ಧಾ-ಬ-ಲ-ದಿಂದ
ಶ್ರದ್ಧಾ-ಳು-ಗಳೋ
ಶ್ರದ್ಧಾ-ವಂ-ತ-ರಾಗಿ
ಶ್ರದ್ಧಾ-ಹೀನ
ಶ್ರದ್ಧೆ
ಶ್ರದ್ಧೆಯ
ಶ್ರದ್ಧೆ-ಯನ್ನು
ಶ್ರದ್ಧೆಯೇ
ಶ್ರಮದ
ಶ್ರೀಕೃ-ಷ್ಣನ
ಶ್ರೀಮಂ-ತರ
ಶ್ರೀಮಂ-ತ-ರನ್ನು
ಶ್ರೀಮಂ-ತರು
ಶ್ರೀಮಂ-ತ-ರೆಂದು
ಶ್ರೀಮ-ನ್ನಾ-ರಾ-ಯ-ಣನೇ
ಶ್ರೀರಾ-ಮ-ಕೃಷ್ಣ
ಶ್ರೀರಾ-ಮ-ಕೃ-ಷ್ಣರು
ಶ್ರೇಯ-ಸ್ಕರ
ಶ್ರೇಷ್ಠ
ಶ್ವೇತ-ಶಾಂ-ತಿ-ಯನ್ನು
ಷರ-ತ್ತನ್ನೂ
ಸಂಕಟ
ಸಂಕ-ಟ-ವಿಲ್ಲ
ಸಂಕು-ಚಿತ
ಸಂಕು-ಚಿ-ತ-ಮಾ-ಡುವ
ಸಂಕೋಚ
ಸಂಕೋ-ಚವೇ
ಸಂಗ
ಸಂಗ್ರ-ಹ-ವಲ್ಲ
ಸಂಗ್ರ-ಹಿ-ಸಿದ
ಸಂಘ
ಸಂಚ-ರಿಸಿ
ಸಂಚ-ರಿ-ಸು-ತ್ತಿದೆ
ಸಂತಾ-ನ-ರಿಗೆ
ಸಂತಾ-ನರು
ಸಂತೆ-ಗಳಿಂದ
ಸಂತೋಷ
ಸಂತೋ-ಷ-ವಾ-ಗಿ-ದ್ದರೆ
ಸಂದ-ರ್ಶಿಸಿ
ಸಂದೇಶ
ಸಂದೇ-ಶ-ವನ್ನು
ಸಂದೇ-ಹ-ವಿಲ್ಲ
ಸಂಪತ್ತು
ಸಂಪ-ರ್ಕ-ವನ್ನು
ಸಂಪಾ-ದ-ನೆಗೆ
ಸಂಪಾ-ದಿಸಿ
ಸಂಪೂರ್ಣ
ಸಂಬಂ-ಧ-ಪಟ್ಟ
ಸಂಭ-ವ-ವಿದೆ
ಸಂಸ್ಕೃತಿ
ಸಂಸ್ಥೆ
ಸಂಸ್ಥೆ-ಗಳು
ಸಕಾ-ಲ-ದಲ್ಲಿ
ಸಚೇ-ತ-ನ-ವಾ-ಗು-ತ್ತಿದೆ
ಸಜೀವ
ಸಣ್ಣ
ಸಣ್ಣ-ದ-ನ್ನಾ-ದರೂ
ಸತತ
ಸತ-ತವೂ
ಸತ್ತರೆ
ಸತ್ತ್ವ-ವಿಲ್ಲ
ಸತ್ಯ
ಸತ್ಯ-ಕ್ಕಾಗಿ
ಸತ್ಯಕ್ಕೆ
ಸತ್ಯ-ಗಳನ್ನು
ಸತ್ಯದ
ಸತ್ಯ-ನಾ-ರಾ-ಯ-ಣ-ನನ್ನು
ಸತ್ಯ-ನಿ-ಷ್ಠ-ರಾ-ಗಿ-ರು-ವರೋ
ಸತ್ಯ-ನಿ-ಷ್ಠೆ-ಯಿಂದ
ಸತ್ಯ-ವನ್ನು
ಸತ್ಯ-ವ-ಲ್ಲ-ವೆ-ನ್ನು-ವುದು
ಸತ್ಯ-ವಾ-ಗಿಯೂ
ಸತ್ಯ-ವಾ-ಗಿ-ರ-ಲಾ-ರದು
ಸತ್ಯ-ವಾದ
ಸತ್ಯ-ವೆಂ-ದಿಗೂ
ಸತ್ಯವೇ
ಸತ್ಯ-ಸಂ-ಧತೆ
ಸತ್ಯ-ಸಂ-ಧನೆ
ಸದೃಶ
ಸದ್ಗು-ಣ-ವನ್ನು
ಸದ್ಯಃ
ಸದ್ಯಕ್ಕೆ
ಸದ್ಯೋ-ನಿ-ರ್ಮಿತ
ಸನಾ-ತನ
ಸನ್ನ-ದ್ಧ-ರಾ-ದ-ವರು
ಸನ್ನಾ-ಹ-ವಾ-ಗು-ತ್ತಿದೆ
ಸನ್ನಿ-ಕಟ
ಸಫ-ಲ-ವಾ-ಗು-ವುದು
ಸಫ-ಲ-ವಾ-ದರೆ
ಸಬ-ಲತೆ
ಸಬೇಕು
ಸಮ-ದೃ-ಷ್ಟಿ-ಯಿಂದ
ಸಮಯ
ಸಮ-ಯ-ಗಳಲ್ಲಿ
ಸಮ-ಯ-ದ-ಲ್ಲಿಯೂ
ಸಮ-ಯ-ವಲ್ಲ
ಸಮ-ರ್ಥ-ರಾ-ಗಿ-ರು-ವರು
ಸಮ-ರ್ಪಿ-ಸ-ಬೇಕು
ಸಮ-ರ್ಪಿಸಿ
ಸಮ-ಸ್ಯೆ-ಗಳನ್ನು
ಸಮ-ಸ್ಯೆಯು
ಸಮಾಜ
ಸಮಾ-ಜ-ಸಾ-ಮ್ಯ-ವಾ-ದದ
ಸಮಾ-ಧಾ-ನ-ಚಿ-ತ್ತ-ರಾ-ಗಿರಿ
ಸಮಾನ
ಸಮೀ-ಪಕ್ಕೆ
ಸಮೀ-ಪ-ದ-ಲ್ಲಿ-ರು-ವನು
ಸಮೀ-ಪಿ-ಸು-ತ್ತದೆ
ಸಮೀ-ಪಿ-ಸು-ತ್ತಿ-ರು-ವುದು
ಸಮೀರ
ಸಮು-ದ್ರದ
ಸಮ್ಮು-ಖ-ದ-ಲ್ಲಿಯೂ
ಸರ-ಳ-ತೆ-ಯೊಂ-ದಿಗೆ
ಸರಿ
ಸರಿ-ಯಾದ
ಸರಿ-ಯಿರಿ
ಸರ್
ಸರ್ವ-ತೋ-ಮು-ಖಿ-ಗ-ಳ-ನ್ನಾಗಿ
ಸರ್ವ-ರಿಗೂ
ಸರ್ವ-ವನ್ನೂ
ಸರ್ವ-ಶ-ಕ್ತ-ನಾದ
ಸರ್ವ-ಶ-ಕ್ತರು
ಸರ್ವ-ಶ-ಕ್ತಿ-ಶಾ-ಲಿನಿ
ಸರ್ವ-ಶ್ರೇ-ಷ್ಠ-ವಾದ
ಸರ್ವ-ಸ್ವ-ವನ್ನು
ಸರ್ವೋ-ತ್ಕೃ-ಷ್ಟ-ರಾದ
ಸಲ
ಸಲ-ವಾ-ದರೂ
ಸಲವೋ
ಸಲು
ಸವಿ-ಯನ್ನು
ಸಹ-ಕ-ರಿ-ಸು-ವುದು
ಸಹ-ನ-ಶೀ-ಲ-ರಾ-ಗ-ಬೇಕು
ಸಹ-ವಾಸ
ಸಹ-ಸ್ರಾರು
ಸಹಾ-ನು-ಭೂತಿ
ಸಹಾ-ನು-ಭೂ-ತಿ-ಯನ್ನು
ಸಹಾಯ
ಸಹಾ-ಯ-ಕ್ಕಿಂತ
ಸಹಾ-ಯ-ದಿಂದ
ಸಹಾ-ಯ-ಮಾಡ
ಸಹಾ-ಯ-ಮಾಡಿ
ಸಹಾ-ಯ-ಮಾ-ಡಿ-ದರೆ
ಸಹಾ-ಯ-ವನ್ನೂ
ಸಹಾ-ಯ-ವಾ-ಗದು
ಸಹಾ-ಯ-ವಾ-ಗು-ವು-ದೆಂದು
ಸಹಾ-ಯ-ವಿ-ಲ್ಲದೆ
ಸಹಾ-ಯ-ವೆಲ್ಲ
ಸಹಾ-ಯ-ವೇನೊ
ಸಹಿ-ಷ್ಣುತೆ
ಸಹಿಸಿ
ಸಹಿ-ಸು-ವುದೇ
ಸಹೋ-ದರ
ಸಹೋ-ದ-ರಂತೆ
ಸಹೋ-ದ-ರ-ರನ್ನು
ಸಹೋ-ದ-ರರು
ಸಹೋ-ದ-ರರೆ
ಸಾಕು
ಸಾಗ-ರದ
ಸಾಗ-ರ-ದಷ್ಟು
ಸಾಗ-ರ-ವನ್ನು
ಸಾಗಿ
ಸಾಗು-ವಿರಿ
ಸಾತ್ತ್ವಿ-ಕ-ರೆಂದು
ಸಾಧನ
ಸಾಧ-ನೆಗೆ
ಸಾಧ-ನೆ-ಯಲ್ಲಿ
ಸಾಧಾ-ರಣ
ಸಾಧಿಸ
ಸಾಧಿ-ಸ-ಲಾ-ರದು
ಸಾಧಿ-ಸಿ-ದರು
ಸಾಧಿ-ಸಿ-ದರೆ
ಸಾಧಿ-ಸು-ವು-ದಕ್ಕೆ
ಸಾಧ್ಯ
ಸಾಧ್ಯ-ವಾ-ಗುವ
ಸಾಧ್ಯ-ವಾ-ದರೆ
ಸಾಧ್ಯ-ವಿಲ್ಲ
ಸಾಪೇಕ್ಷ
ಸಾಮರ್ಥ್ಯ
ಸಾಮ-ರ್ಥ್ಯ-ದಿಂದ
ಸಾಮ-ರ್ಥ್ಯ-ವನ್ನು
ಸಾಮ-ರ್ಥ್ಯ-ವ-ನ್ನೆಲ್ಲ
ಸಾಮ-ರ್ಥ್ಯ-ವಿದೆ
ಸಾಮಾ-ಜಿಕ
ಸಾಮಾ-ನನ್ನು
ಸಾಮಾನ್ಯ
ಸಾಮೂ
ಸಾಮೂ-ಹಿ-ಕ-ವಾಗಿ
ಸಾಯ-ಬ-ಹುದು
ಸಾಯ-ಲೇ-ಬೇ-ಕಾ-ಗಿ-ರು-ವಾಗ
ಸಾಯುಜ್ಯ
ಸಾಯು-ತ್ತಿ-ರು-ವರು
ಸಾಯು-ವರು
ಸಾಯುವು
ಸಾಯು-ವು-ದ-ಕ್ಕಿಂತ
ಸಾಯು-ವುದು
ಸಾಯು-ವುವು
ಸಾಯು-ವೆವು
ಸಾರ-ಬ-ಹುದು
ಸಾರ-ವಾದ
ಸಾರ-ವಿ-ದು-ಪ-ರಿ-ಶು-ದ್ಧ-ವಾ-ಗಿ-ರು-ವುದು
ಸಾರವೆ
ಸಾರಿ
ಸಾರಿ-ದವು
ಸಾರುವ
ಸಾರು-ವು-ದಿಲ್ಲ
ಸಾರು-ವುದು
ಸಾರ್ವ-ತ್ರಿ-ಕತೆ
ಸಾರ್ವ-ಭೌ-ಮನ
ಸಾಲ-ದಾಗು
ಸಾಲದು
ಸಾಲಿ-ಗರ
ಸಾವನ್ನು
ಸಾವಿತ್ರಿ
ಸಾವಿರ
ಸಾವಿ-ರಾರು
ಸಾವಿಲ್ಲ
ಸಾಹಸ
ಸಾಹ-ಸಕ್ಕೆ
ಸಾಹಿ-ತ್ಯ-ದಲ್ಲಿ
ಸಿಂಧು-ವಿ-ನಿಂದ
ಸಿಂಹ
ಸಿಂಹರು
ಸಿಂಹ-ಸ-ದೃಶ
ಸಿಂಹ-ಸ-ದೃ-ಶ-ವಾದ
ಸಿಂಹಾ-ಸ-ನದ
ಸಿಕ್ಕದೇ
ಸಿಕ್ಕಿ
ಸಿಕ್ಕಿ-ಕೊಂ-ಡಿವೆ
ಸಿಕ್ಕುವ
ಸಿಕ್ಕು-ವರು
ಸಿಗು-ವು-ದಿ-ಲ್ಲ-ವೆಂದು
ಸಿಡಿ-ದಾಟ
ಸಿಡಿ-ಮ-ದ್ದಿ-ನಂ-ತಿ-ರುವ
ಸಿಡಿ-ಯು-ತ್ತಿ-ದೆ-ಇಂ-ಡಿ-ಯಾ-ದೇ-ಶ-ದಲ್ಲಿ
ಸಿಡಿಲು
ಸಿಡ್ನಿಯ
ಸಿದರೆ
ಸಿದ್ಧತೆ
ಸಿದ್ಧ-ನಾಗು
ಸಿದ್ಧ-ವಾ-ಗು-ತ್ತದೆ
ಸಿದ್ಧಾಂತ
ಸಿದ್ಧಾಂ-ತ-ಗಳನ್ನು
ಸಿದ್ಧಾಂ-ತ-ವನ್ನು
ಸಿದ್ಧಿ-ಸಿ-ದರೆ
ಸಿಬ್ಬಂದಿ
ಸೀತಾ
ಸೀತೆಯ
ಸುಂದರ
ಸುಖ
ಸುಖಕ್ಕೆ
ಸುಖ-ಗ-ಳೆ-ರ-ಡನ್ನೂ
ಸುಖ-ವನ್ನು
ಸುಖವೇ
ಸುತರೆ
ಸುತ್ತ-ಲಿ-ರುವ
ಸುತ್ತಲೂ
ಸುತ್ತೇನೆ
ಸುದೀರ್ಘ
ಸುಧಾ-ರಕ
ಸುಧಾ-ರ-ಕ-ರಾ-ಗಲು
ಸುಪ್ತ-ವಾ-ಗಿ-ರುವ
ಸುಪ್ತ-ವಾ-ಗಿ-ರು-ವುದು
ಸುಪ್ತ-ವಾದ
ಸುಮ್ಮನೆ
ಸುಯೋಗ
ಸುಲ-ಭ-ವಾದ
ಸುಳಿ
ಸುಳ್ಳು
ಸುವ
ಸುವು-ದಿಲ್ಲ
ಸುವೆವೆ
ಸುಸ-ಜ್ಜಿ-ತ-ವಾದ
ಸೃಷ್ಟಿ
ಸೃಷ್ಟಿಗೆ
ಸೃಷ್ಟಿಸಿ
ಸೃಷ್ಟಿ-ಸಿತೋ
ಸೃಷ್ಟಿ-ಸು-ವುದು
ಸೇಡನ್ನು
ಸೇಡು
ಸೇತು-ವೆ-ಯನ್ನು
ಸೇರುವ
ಸೇರು-ವ-ವ-ರೆಗೂ
ಸೇರು-ವುದೇ
ಸೇವ-ಕ-ರನ್ನು
ಸೇವ-ಕ-ರೆಂದು
ಸೇವಾ-ತ-ತ್ಪ-ರ-ತೆ-ಯನ್ನು
ಸೇವಿ-ಸ-ಬೇಕು
ಸೇವಿ-ಸ-ಬೇ-ಕೆಂದು
ಸೇವಿ-ಸ-ಲಾ-ರರು
ಸೇವಿಸಿ
ಸೇವಿ-ಸು-ವರೋ
ಸೇವಿ-ಸು-ವುದು
ಸೇವೆ
ಸೇವೆ-ಗಾಗಿ
ಸೇವೆಗೆ
ಸೇವೆ-ಮಾಡಿ
ಸೇವೆ-ಯನ್ನು
ಸೇವೆ-ಯಲ್ಲಿ
ಸೊನ್ನೆ-ಗ-ಳು-ಭ-ವಿತ
ಸೋಂಕಿ-ಲ್ಲದ
ಸೋದ-ರ-ರಿ-ಗಾಗಿ
ಸೋದ-ರರು
ಸೋಮಾ-ರಿ-ಗ-ಳಾಗಿ
ಸೋಮಾ-ರಿ-ತ-ನ-ದಿಂದ
ಸೋಮಾ-ರಿ-ತ-ನ-ವನ್ನು
ಸೋಲನ್ನು
ಸೌಹಾ-ರ್ದ-ತೆ-ಯನ್ನು
ಸೌಹಾ-ರ್ದ-ದಿಂದ
ಸ್ತ್ರೀಯರ
ಸ್ತ್ರೀಯರು
ಸ್ಥಳ
ಸ್ಥಳ-ದಲ್ಲಿ
ಸ್ಥಳವೇ
ಸ್ಥಾನ-ದಲ್ಲಿ
ಸ್ಥಾನ-ವನ್ನು
ಸ್ಥಿತಿ
ಸ್ಥಿತಿ-ಗಿಂತ
ಸ್ಥಿತಿಗೆ
ಸ್ಥಿರ-ವಾಗಿ
ಸ್ಥೈರ್ಯ-ಧೈ-ರ್ಯ-ಗಳನ್ನು
ಸ್ನೇಹಿ-ತ-ನಾ-ಗಿ-ದ್ದನೋ
ಸ್ನೇಹಿ-ತ-ರಿ-ಲ್ಲ-ವೆಂದು
ಸ್ನೇಹಿ-ತರೆ
ಸ್ಪರ್ಧೆಯೇ
ಸ್ಮಶಾ-ನ-ಗ-ಳೆಂದು
ಸ್ವಂತ
ಸ್ವಂತ-ಸುಖ
ಸ್ವತಂ-ತ್ರ-ವಾ-ಗಿ-ರು-ವುದು
ಸ್ವಪ್ನ-ಲೋ-ಕದ
ಸ್ವಪ್ರ-ತಿಷ್ಠೆ
ಸ್ವಭಾವ
ಸ್ವಭಾ-ವತಃ
ಸ್ವಭಾ-ವದ
ಸ್ವಭಾ-ವ-ದಲ್ಲಿ
ಸ್ವರೂ-ಪ-ವೆಂದು
ಸ್ವರ್ಗ
ಸ್ವರ್ಗಕ್ಕೆ
ಸ್ವರ್ಗದ
ಸ್ವರ್ಗ-ಸು-ಖ-ವ-ನ್ನಾ-ಗಲೀ
ಸ್ವರ್ಗೀ-ಯ-ಶಕ್ತಿ
ಸ್ವಲ್ಪ
ಸ್ವಲ್ಪ-ವಾ-ದರೂ
ಸ್ವಲ್ಪವೂ
ಸ್ವಾಗ-ತಿ-ಸ-ಬೇಕು
ಸ್ವಾಗ-ತಿ-ಸು-ವರೋ
ಸ್ವಾತಂತ್ರ್ಯ
ಸ್ವಾತಂ-ತ್ರ್ಯದ
ಸ್ವಾತಂ-ತ್ರ್ಯ-ವನ್ನು
ಸ್ವಾಧೀ-ನಕ್ಕೆ
ಸ್ವಾಭಾ-ವಿಕ
ಸ್ವಾಭಾ-ವಿ-ಕ-ವಾ-ಗಿಯೇ
ಸ್ವಾಮಿ
ಸ್ವಾಮಿ-ಯಂತೆ
ಸ್ವಾಮಿ-ಯಾ-ಗಲು
ಸ್ವಾರ್ಥ-ತೆಯೇ
ಸ್ವಾರ್ಥವೋ
ಸ್ವಾರ್ಥ-ಸುಖ
ಸ್ವಾರ್ಥಿ
ಸ್ವಾರ್ಥಿ-ಗಳೊ
ಸ್ವೀಕ-ರಿ-ಸಿ-ದನೋ
ಸ್ವೀಕ-ರಿ-ಸು-ವ-ವ-ನಲ್ಲ
ಸ್ವೀಕ-ರಿ-ಸು-ವು-ದಕ್ಕೆ
ಹಕ್ಕಿಗೆ
ಹಕ್ಕು-ದಾ-ರರು
ಹಗ-ಲಿ-ರಳೂ
ಹಗ-ಲಿ-ರುಳು
ಹಚ್ಚಿ
ಹಣ
ಹಣದ
ಹಣ-ದಿಂದ
ಹಣ-ವನ್ನು
ಹಣ-ವೆಲ್ಲ
ಹಣವೇ
ಹಣೆಯ
ಹತ್ತಿರ
ಹತ್ತಿ-ರು-ವಿರಿ
ಹದಿ-ನಾಲ್ಕು
ಹನು-ಮಂತ
ಹರ-ಡ-ಬೇಕು
ಹರಡಿ
ಹರ-ಡು-ವುದು
ಹರಿ-ಯು-ವಂತೆ
ಹರಿ-ಸ-ಬೇಕು
ಹಲ-ವರು
ಹಲ-ವಾರು
ಹಲವು
ಹಲ-ಹ-ಸ್ತೆ-ಯಾಗಿ
ಹಳೆಯ
ಹಸಿ-ವನ್ನು
ಹಸು-ಳೆಯ
ಹಾಕ-ಬೇಡಿ
ಹಾಗಿ-ದ್ದರೆ
ಹಾಗೆ
ಹಾಗೆಯೇ
ಹಾದಿ
ಹಾರ-ಗ-ಳಂತೂ
ಹಾರಲು
ಹಾರೈಕೆ
ಹಿಂತಿ-ರುಗಿ
ಹಿಂದಿನ
ಹಿಂದಿ-ರುಗಿ
ಹಿಂದಿ-ರುವ
ಹಿಂದೂ
ಹಿಂದೆ
ಹಿಂಸೆಗೆ
ಹಿಕ-ವಾಗಿ
ಹಿಗ್ಗಿ-ದರೂ
ಹಿಡಿದು
ಹಿತ-ಕ-ರ-ವಾದ
ಹಿತ-ಕ-ರ-ವಾ-ದು-ದನ್ನು
ಹಿತ-ಕಾರಿ
ಹಿತ-ಚಿಂ-ತನೆ
ಹಿತ-ವಾದ
ಹಿಮಾ-ಲ-ಯ-ದಿಂದ
ಹಿರಿದ
ಹಿರಿ-ಯ-ರಿಗೆ
ಹೀಗಿ-ರು-ವಾಗ
ಹೀಗೆ
ಹೀನ
ಹೀನ-ಭಾ-ವ-ನೆ-ಗ-ಳೆಲ್ಲ
ಹೀನ-ಮ-ನು-ಜರು
ಹೀನ-ಮಾ-ನ-ವ-ರ-ನನ್ನು
ಹುಚ್ಚ
ಹುಚ್ಚ-ನಂತೆ
ಹುಚ್ಚ-ನೆಂದು
ಹುಚ್ಚರ
ಹುಚ್ಚರು
ಹುಚ್ಚಿ-ಗಿಂತ
ಹುಚ್ಚಿ-ನಂತೆ
ಹುಚ್ಚು
ಹುಟ್ಟಿ
ಹುಟ್ಟು-ವುದು
ಹುಡಿ-ಯಲ್ಲಿ
ಹುಡಿ-ಯಾಗಿ
ಹುಡುಕಿ
ಹುಡು-ಕು-ವೆವು
ಹುಡು-ಗ-ನಾ-ಗಿ-ದ್ದಾಗ
ಹುತ್ತ
ಹುದು-ಗಿಸ
ಹುರು-ಳಿಲ್ಲ
ಹುರು-ಳಿ-ಲ್ಲದ
ಹುಸಿ-ಗ-ನಸು
ಹೃತ್ಪೂ-ರ್ವಕ
ಹೃತ್ಪೂ-ರ್ವ-ಕ-ವಾಗಿ
ಹೃದಯ
ಹೃದ-ಯ-ಮಂ-ದಿ-ರ-ದಲ್ಲಿ
ಹೃದ-ಯಲ್ಲಿ
ಹೃದ-ಯ-ವನ್ನು
ಹೃದ-ಯ-ವ-ನ್ನೆಲ್ಲ
ಹೃದ-ಯ-ವೇದೆ
ಹೆಂಡತಿ
ಹೆಂಡಿರು
ಹೆಚ್ಚಾಗಿ
ಹೆಚ್ಚಾದ
ಹೆಚ್ಚು
ಹೆಣ-ಗಳ
ಹೆಣ-ಗ-ಳಾ-ಗಿ-ದ್ದೀರಿ
ಹೆಣ್ಣು
ಹೆದ-ರದೆ
ಹೆಮ್ಮೆ-ಯಿಂದ
ಹೆಸ-ರನ್ನು
ಹೆಸ-ರಲ್ಲ
ಹೆಸ-ರಿನ
ಹೆಸ-ರಿ-ನಲ್ಲಿ
ಹೆಸ-ರಿ-ನಿಂ-ದಲೂ
ಹೆಸರು
ಹೇ
ಹೇಗೆ
ಹೇಡಿ
ಹೇಡಿ-ಗ-ಳಾ-ಗ-ಕೂ-ಡದು
ಹೇಡಿ-ಗ-ಳಿ-ಗಲ್ಲ
ಹೇಡಿ-ಗಳು
ಹೇಡಿ-ತನ
ಹೇಡಿ-ಯಾ-ಗ-ಬೇಡ
ಹೇಡಿ-ಯಾ-ದರೂ
ಹೇಳ-ಬೇಡ
ಹೇಳಿ
ಹೇಳಿ-ಕೊ-ಳ್ಳುವ
ಹೇಳಿ-ರು-ವೆನು
ಹೇಳು-ತ್ತೇನೆ
ಹೇಳು-ವು-ದಾ-ವು-ದನ್ನೂ
ಹೇಳು-ವುದು
ಹೇಳು-ವು-ದೆ-ಲ್ಲ-ವನ್ನೂ
ಹೇಳು-ವೆವು
ಹೊಂದಿಕೆ
ಹೊಂದಿ-ಕೊಂಡೆ
ಹೊಗ-ಳು-ತ್ತಾ-ರೆಯೋ
ಹೊಗೆ-ಯಂತೆ
ಹೊಟ್ಟಿನ
ಹೊಡೆದು
ಹೊಡೆ-ಯಿರಿ
ಹೊತ್ತಾರೆ
ಹೊದೆಯು
ಹೊನ
ಹೊಮ್ಮಿ
ಹೊರ-ಗಿನ
ಹೊರ-ಗಿ-ನ-ವ-ರೊಂ
ಹೊರ-ಗಿ-ರು-ವುದನ್ನು
ಹೊರಗೂ
ಹೊರಗೆ
ಹೊರ-ಹೊ-ಮ್ಮಲಿ
ಹೊಳೆ
ಹೊಳೆಯ
ಹೊಸ
ಹೊಸ-ಚೇ-ತ-ನ-ವನ್ನು
ಹೋಗ-ಬಾ-ರದು
ಹೋಗ-ಬೇ-ಕಾ-ಗಿದೆ
ಹೋಗ-ಬೇಕು
ಹೋಗಲು
ಹೋಗಿ
ಹೋಗಿ-ದ್ದರೂ
ಹೋಗು
ಹೋಗು-ತ್ತಿ-ರುವ
ಹೋಗು-ವರೋ
ಹೋಗು-ವಿರಿ
ಹೋಗು-ವು-ದಿಲ್ಲ
ಹೋಗು-ವುದು
ಹೋದ
ಹೋದರೂ
ಹೋರಾಟ
ಹೋರಾ-ಟವೇ
ಹೋರಾಡಿ
ಹೋರಾ-ಡಿ-ದಾಗ
ಹೋರಾ-ಡು-ತ್ತಿ-ರುವ
ಹೋರಾ-ಡು-ತ್ತೀರಾ
ಹೋರಾ-ಡು-ವನೋ
ಹೋರಾ-ಡು-ವುದು
ಹೋಲಿ-ಸಿ-ದಾಗ
ಹೋಲಿ-ಸು-ವು-ದಕ್ಕೆ
ಹೌದು
}
