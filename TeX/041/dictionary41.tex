\sethyphenation{kannada}{
गुरुराजो
विजयते
श्री
श्री-ः
ಅ
ಅಂಗ
ಅಂಗ-ವಾಗಿ
ಅಂಗ-ಸಂಗ
ಅಂಗ-ಸುಖ-ವನ್ನು
ಅಂಗೀಕೃತ್ಯ
ಅಂಡಂಬ್ರಹ್ಮಾಂಡ-ವನ್ನು
ಅಂಡೇಬ್ರಹ್ಮಾಂಡ-ದಲ್ಲಿ-ರುವ
ಅಂತಃಒಳಗಿ-ರುವುದೋ
ಅಂತಃಕರ-ಣಕ್ಕೆ
ಅಂತಃಪ್ರವಿಷ್ಟಾಬ್ರಹ್ಮಾಂಡದ
ಅಂತರಾ-ಒಳಗೆ
ಅಂತರಾತ್ಮ-ನಾ-ಗಿದ್ದಾನೆ
ಅಂತರಾತ್ಮ-ನೆಂಬ
ಅಂತರಾತ್ಮ-ರಾದ
ಅಂತರಾತ್ಮ-ರೂಪ-ದಿಂದ
ಅಂತರಾಯ
ಅಂತರ್ಬಹಿಶ್ಚ
ಅಂತರ್ಯಾಮಿ-ಯಾದ
ಅಂತಹ
ಅಂತಹ-ವ-ನಿಗೆ
ಅಂತಹ-ವನು
ಅಂತಹ-ವ-ರಿಗೆ
ಅಂತಿಮ
ಅಂತಿಮ-ವಾದ
ಅಂತಿಮಾ-ವರ-ಣವ್ಯೋಮ್ನೋರಂತರೇ
ಅಂತ್ಯ
ಅಂದರೆ
ಅಂಧಂ
ಅಂಧಂತಮಃ
ಅಂಧಂತಮಸ್ಸನ್ನು
ಅಂಧಂತಮಸ್ಸಿಗೆ
ಅಂಧಂತಮಸ್ಸಿಗೇ
ಅಂಧಂತಮಸ್ಸಿ-ನಲ್ಲಿ
ಅಂಧಂತಮಸ್ಸಿನಲ್ಲಿಯೂ
ಅಂಧಂತಮಸ್ಸು
ಅಂಧಂತಮಸ್ಸೆಂಬ
ಅಂಧಂತಮಸ್ಸೇ
ಅಂಧ-ಕಾರ-ಗಳಿಗೆ
ಅಂಧೇ
ಅಂಫ್ರಿ
ಅಂಫ್ರಿ-ಪಾನಃಹೆಬ್ಬೆರಳನ್ನು
ಅಂಭಸ್ವ-ರೂಪಿಣೀ
ಅಂಶಃಜೀವನು
ಅಂಶಕ್ಕೆ
ಅಂಶ-ಗಳಲ್ಲ
ಅಂಶ-ಗಳಿಂದ
ಅಂಶ-ಗಳು
ಅಂಶ-ದಂತೆ
ಅಂಶ-ದಿಂದ
ಅಂಶನಲ್ಲ-ವೆಂದು
ಅಂಶನು
ಅಂಶನೆಂದರ್ಥ
ಅಂಶ-ನೆಂದು
ಅಂಶನೇ
ಅಂಶ-ಭೂತನು
ಅಂಶವು
ಅಂಶಾಂಶೇನ-ಅನೇಕ
ಅಂಶೋ
ಅಃಎಲ್ಲಕ್ಕಿಂತಲೂ
ಅಕಾಮ್ಯ
ಅಕಾರ
ಅಕಾರಃ
ಅಕಾರಃಅ
ಅಕಾರ-ಅ-ಭಾವ
ಅಕಾರಕ್ಕೂ
ಅಕಾರ-ಗಳ
ಅಕಾರದ
ಅಕಾರ-ದಿಂದ
ಅಕಾರ-ರೂಪ-ವರ್ಣ
ಅಕಾರವು
ಅಕಾರಶ್ಚ
ಅಕಾರಶ್ಚಅ
ಅಕಾರ-ಸಮಾ-ನಾರ್ಥಕ
ಅಕಾರಸ್ಕಾಧಿ-ಕಾರ್ಥತಃ
ಅಕಾರಸ್ಯ
ಅಕಾರಸ್ಯಅ
ಅಕಾರಸ್ಯಾಪಿ
ಅಕಾರಸ್ಯಾಪಿಅ
ಅಕಾರೇ
ಅಕಾರ್ಯ-ವನ್ನು
ಅಕ್ಷರಕ್ಕೆ
ಅಕ್ಷರ-ಗಳ
ಅಕ್ಷರ-ಗಳನ್ನು
ಅಕ್ಷರ-ಗಳಿಂದ
ಅಕ್ಷರದ
ಅಕ್ಷರದ್ದೂ
ಅಕ್ಷರವು
ಅಕ್ಷರ-ಸಮನ್ವ-ಯೇನ
ಅಕ್ಷೇಪಕ್ಕೆ
ಅಗಮ್ಯತ್ವಾತ್
ಅಗಲುವಿಕೆಯ
ಅಗಿ
ಅಗ್ನಿ
ಅಗ್ನಿಯು
ಅಗ್ನಿ-ಯುಕ್ತಃಸೂರ್ಯ
ಅಗ್ನಿರ್ಮೂರ್ಧಾ
ಅಗ್ರತಃಮುಂದೆ
ಅಗ್ರತೋ
ಅಗ್ರೇಸರಃ
ಅಗ್ರೇಸರಃಮುಂದೆ
ಅಗ್ರೇಸರಾನ್
ಅಗ್ರೇಸರಾನ್ಮುಂದೆ
ಅಚೇತನ
ಅಚೇತನ-ವಾದ
ಅಚೇತನ-ಶುದ್ದಸಾತ್ವಾತ್ಮಕ-ಗಳಾ-ಗಿದ್ದರೂ
ಅಚೇತನೈವ
ಅಚ್ಚಳಿಯದೇ
ಅಜಂತ-ವೆಂದು
ಅಜಂತ-ವೆಂಬ
ಅಜಸ್ಯ
ಅಜಸ್ರಂನಿತ್ಯ-ದಲ್ಲಿಯೂ
ಅಜಾಂ
ಅಜಾತ್ಶ್ರೀ-ಹರಿ-ಯಿಂದ
ಅಜಾನತಾ
ಅಜ್
ಅಜ್ಅಲ್ಲಿ
ಅಜ್ಞಾನ
ಅಜ್ಞಾನ-ರೂಪ-ವಾದ
ಅಜ್ಞಾನ-ವೆಂಬ
ಅಜ್ಞಾನಾಂ
ಅಜ್ಞಾನಿ-ಗಳಿಗೆ
ಅಣು-ಭಾಷ್ಯ
ಅಣು-ಭಾಷ್ಯ-ದಲ್ಲಿ
ಅಣ್
ಅತ
ಅತಃ
ಅತಃಆ
ಅತಃಆದು-ದ-ರಿಂದ
ಅತಃಪ್ರಾರಬ್ಧ-ಕರ್ಮ-ದಿಂದ
ಅತಶ್ಚಾಂಶತ್ವಮುದ್ದಿ
ಅತಿ
ಅತಿ-ರೋಹಿತಜ್ಞಾನತ್ವಾತ್
ಅತಿ-ರೋಹಿತವಿಜ್ಞಾನಾದ್ವಾಯು-ರಷ್ಯಮೃತಃ
ಅತೋ
ಅತೋ-ಽನ್ಯದಪೀತ್ಯೆಕೇಷಾಮುಭಯೋಃ
ಅತ್ಯಂತ
ಅತ್ಯಂತ-ದೂರ
ಅತ್ಯಂತ-ಭಕ್ತಿ-ಭರಿ-ತಸ್ಯ
ಅತ್ಯರ್ಥಪ್ರಸಾ-ದಿಂದ
ಅತ್ಯುತ್ತಮ-ವಾಗಿ
ಅತ್ರ
ಅತ್ರ-ಪಾಪ
ಅಥ
ಅಥವ
ಅಥವಾ
ಅದಕ್ಕಿಂತ
ಅದಕ್ಕೆ
ಅದನ್ನು
ಅದರ
ಅದರಂತೆ
ಅದ-ರಲ್ಲಿ
ಅದ-ರಲ್ಲಿ-ರುವ
ಅದ-ರಿಂದಲೇ
ಅದರೂ
ಅದು
ಅದೃಶ್ಯನಾಗುತ್ತಾ-ನೆಂದು
ಅದೃಶ್ಯನಾಗು-ವುದು
ಅದೇ
ಅದೈತಮತ-ವನ್ನು
ಅದ್ಯಜಂತಃದೋಷ-ವೆಂದು
ಅದ್ಯೋ
ಅಧರ್ಮ
ಅಧಿಕಗೊಳಿ-ಸಿದರು
ಅಧಿಕತ್ವಂ
ಅಧಿಕತ್ವಂಆಧಿಕ್ಯ
ಅಧಿಕತ್ವತಃಎಲ್ಲಕ್ಕಿಂತಲೂ
ಅಧಿಕ-ರೆಂದೂ
ಅಧಿಕ-ವಾದ
ಅಧಿ-ಕಾರ
ಅಧಿ-ಕಾರ-ವಿದೆ
ಅಧಿ-ಕಾರಿ-ಗಳಿಂದ
ಅಧಿ-ಕಾರಿಯು
ಅಧಿಗಮೇ-ಸತಿ-ಸಾಕ್ಷಾತ್ಕಾರ-ವಾಗುತ್ತಿ-ರಲು
ಅಧಿಪತಿಃ
ಅಧಿಷ್ಠಾನ-ಗಳು
ಅಧಿಷ್ಠಾಯ
ಅಧೀನ-ದಲ್ಲಿ-ರುವಿಕೆ
ಅಧೀನ-ನಾಗಿ
ಅಧೀ-ನರು
ಅಧೀ-ಯತೇ-ಪರಬ್ರಹ್ಮ-ನಲ್ಲಿ
ಅಧ್ಯ-ಯನ
ಅಧ್ಯ-ಯನ-ಮಾಡುತ್ತಾರೆ
ಅಧ್ಯಸಂಸ್ಥಂದಹ-ಸೂಕ್ಷ್ಮ-ನಾದ
ಅಧ್ಯಾಯ
ಅಧ್ಯಾಯ-ಗಳಲ್ಲಿ
ಅಧ್ಯಾಯ-ದಲ್ಲಿ
ಅಧ್ಯಾಯೇ-ವಾಯು-ಪುರಾ-ಣದ
ಅಧ್ಯಾಸ್ತೇ
ಅಧ್ಯಾಸ್ತೇ-ತನ್ನ
ಅನಂತ-ನಿಗೆ
ಅನಂತಮಡಿ-ಯಾಗಿ
ಅನಂತರ-ದಲ್ಲಿ
ಅನಂತ-ವಾದ
ಅನಂತಾನ-ವದ್ಯಕಲ್ಯಾಣ-ಗುಣತ್ವ
ಅನಂತಾನಿ
ಅನಂತಾಸನ
ಅನಂತಾಸನ-ವೈಕುಂಠ
ಅನಂತಾಸನ-ವೈಕುಂಠ-ನಾರಾ-ಯ-ಣ-ಪುರೇಷು
ಅನಂತಾಸನ-ವೈಕುಂಠಾಶ್ವೇತದ್ವೀಪ
ಅನಕಾಮಮಾರ
ಅನಭೀಷ್ಟ
ಅನಭೀಷ್ಟಂ
ಅನಯ
ಅನವೇಕ್ಷ್ಯ
ಅನಸ್ಯ
ಅನಾದಿ
ಅನಾದಿ-ಕಾಲ-ದಿಂದ
ಅನಾದಿ-ನಿತ್ಯ-ನಾದ
ಅನಾದಿ-ನಿತ್ಯ-ವಾ-ದುದು
ಅನಾದಿಪ್ರಕೃತಿ-ಬಂಧ-ನದ
ಅನಾದಿ-ಲಿಂಗ
ಅನಾದಿ-ಲಿಂಗ-ಶರೀರ-ರೂಪಪ್ರಕೃತಿ-ಬಂಧ
ಅನಾದ್ಯನಂತ
ಅನಾದ್ಯನಂತ-ಕಾಲ-ದಲ್ಲಿ
ಅನಾದ್ಯನಂತೇ
ಅನಾರಬ್ದಂ
ಅನಾರಬ್ದಂಅಪ್ರಾರಬ್ದ
ಅನಾಶಾತ್ನಾಶ-ರಹಿತ-ನಾದು-ದ-ರಿಂದ
ಅನಿತ್ಯತ್ವ
ಅನಿತ್ಯತ್ವ-ವಿ-ಕಾರಿತ್ವ
ಅನಿತ್ಯತ್ವಾದಿ-ದೋಷ-ಶೂನ್ಯಂ
ಅನಿತ್ಯಾ-ದಿ-ದೋಷ-ರಹಿತ
ಅನಿರುದ್ದ
ಅನಿರುದ್ದಂಅ-ನಿರುದ್ಧ-ನನ್ನು
ಅನಿರುದ್ದ-ಗುರ್ವಾದಿ
ಅನಿರುದ್ದ-ಗುರ್ವಾದಿದ್ವಾರಾ-ಽ-ಯನಮ್
ಅನಿರುದ್ಧ
ಅನಿರುದ್ಧ-ರೂಪ-ಗಳನ್ನು
ಅನಿರುದ್ಧ-ರೂಪ-ಗಳಿಂದಿ-ರುವ
ಅನಿಲ-ದೇವ-ರೆಂದು
ಅನಿಶ್ಚಿತ
ಅನಿಷ್ಟ
ಅನಿಷ್ಟ-ಕಾಮ್ಯ
ಅನಿಷ್ಟ-ಗಳೆಂದು
ಅನಿಷ್ಟ-ಪುಣ್ಯ
ಅನು-ಗುಣ-ವಾಗಿ
ಅನು-ಗುಣ-ವಾದ
ಅನುಗ್ರಹ
ಅನುಗ್ರಹಕ್ಕೆ
ಅನುಗ್ರಹ-ದಿಂದ
ಅನುಗ್ರಹ-ದಿಂದಲೇ
ಅನುಗ್ರಹ-ಪೂರ್ವ-ಕ-ವಾಗಿ
ಅನುಗ್ರಹ-ಮಾಡಿ-ದ-ರೆಂಬ
ಅನುಗ್ರಹ-ವಿಲ್ಲದೇ
ಅನುಗ್ರಹಿ-ಸಿದನು
ಅನುಗ್ರಹಿ-ಸುತ್ತಾನೆ
ಅನುಭ-ವಕ್ಕೆ
ಅನುಭವ-ವನ್ನು
ಅನುಭವವೂ
ಅನುಭ-ವವೇ
ಅನುಭವಿಸದೇ
ಅನುಭವಿಸಬೇಕಾ-ದರೆ
ಅನುಭವಿಸ-ಬೇಕು
ಅನುಭವಿ-ಸಲು
ಅನುಭವಿಸಲೇ
ಅನುಭವಿಸಿ
ಅನುಭವಿ-ಸಿದ
ಅನುಭವಿ-ಸಿಯೇ
ಅನುಭವಿಸುತ್ತಾ
ಅನುಭವಿಸುತ್ತಾ-ನಲ್ಲದೆ
ಅನುಭವಿ-ಸುತ್ತಾನೆ
ಅನುಭವಿಸುತ್ತಾ-ನೆ-ಯಾದು-ದ-ರಿಂದ
ಅನುಭವಿಸುತ್ತಾರೆ
ಅನುಭವಿಸುವ
ಅನುಭವಿಸುವ-ವರು
ಅನುಭವಿಸುವಾಗ
ಅನುಭವಿಸು-ವುದಿಲ್ಲ
ಅನುಭವಿಸುವುದೇ
ಅನು-ಭೂಯ
ಅನು-ಮಾನ-ಗಳು
ಅನು-ವಾದ
ಅನು-ವಾದ-ಗೊಂಡಿದ್ದು
ಅನು-ವಾದ-ಗೊಂಡು
ಅನು-ವಾದವು
ಅನುವಿಷ್ಟಅಂತಃಪ್ರವಿಷ್ಟ-ನಾಗಿ
ಅನುವ್ಯಾಖ್ಯಾನ
ಅನುವ್ಯಾಖ್ಯಾನೇ-ಅನುವ್ಯಾಖ್ಯಾನ
ಅನುವ್ಯಾಖ್ಯಾನೇ-ಅನುವ್ಯಾಖ್ಯಾನ-ದಲ್ಲಿ
ಅನುವ್ರಜಂತಿ
ಅನುವ್ರಜತಿ-ಭಕ್ತ-ರನ್ನು
ಅನುವ್ರಜಾಮಿ-ಅನು-ಸ-ರಿಸಿ
ಅನುವ್ರಜಾಮ್ಯಹಂ
ಅನು-ಸ-ರಿಸಿ
ಅನು-ಸರಿ-ಸಿಯೇ
ಅನು-ಸರಿ-ಸುತ್ತವೆ
ಅನು-ಸರಿ-ಸುತ್ತಾ-ನಾದು-ದ-ರಿಂದ
ಅನು-ಸಾರ-ವಾಗಿ
ಅನೇಕ
ಅನೇಕರು
ಅನೇನ
ಅನೇನ-ಇವ-ನಿಂದ
ಅನೇನೇತಿ
ಅನ್ನ-ರೂಪ-ದಿಂದ
ಅನ್ನಾದಿ
ಅನ್ನಾದಿ-ಜಾತಂಅನ್ನವೇ
ಅನ್ಯತ್ಬೇರೆ-ಯಾದ
ಅನ್ಯತ್ರ
ಅನ್ಯತ್ವ
ಅನ್ಯತ್ವಂ
ಅನ್ಯತ್ವ-ವನ್ನು
ಅನ್ಯತ್ವ-ವಾಚೀ-ತತ್ರ-ಅಲ್ಲಿ
ಅನ್ಯತ್ವ-ವಾಚ್ಯ-ಕ-ಎಲ್ಲ-ದ-ರಿಂದಲೂ
ಅನ್ಯತ್ವಾತ್
ಅನ್ಯತ್ವಾತ್ಸರ್ವ-ಜೀವೇಭ್ಯೋ
ಅನ್ಯಥಾ
ಅನ್ಯರಲ್ಲ
ಅನ್ಯೈಃ
ಅನ್ವೇಷ್ಟವ್ಯಂತಸ್ಮಿನ್ಆ
ಅನ್ವೇಷ್ಟವ್ಯಂಹುಡುಕಲ್ಪಡ-ಬೇಕು
ಅಪ-ಗತಿ
ಅಪ-ಗತಿಃ
ಅಪ-ಗತಿಃಹೋಗು-ವುದು
ಅಪ-ಮಾನ-ಕರ-ವಾಗಿ
ಅಪರ
ಅಪರ-ನಾಮಿಕಾ-ಗಂಗೆಯೇ
ಅಪರಾಧಕ್ಕೆ
ಅಪರೋಕ್ಷ
ಅಪರೋಕ್ಷಜ್ಞಾನಕ್ಕೆ
ಅಪರೋಕ್ಷಜ್ಞಾನದ
ಅಪರೋಕ್ಷಜ್ಞಾನ-ದಿಂದ
ಅಪರೋಕ್ಷಜ್ಞಾನ-ಯುಕ್ತ-ರಾದ-ವ-ರನ್ನು
ಅಪರೋಕ್ಷಜ್ಞಾನ-ವಾಗುವ
ಅಪರೋಕ್ಷಜ್ಞಾನ-ವಾದ
ಅಪರೋಕ್ಷಜ್ಞಾನ-ವೆಂಬ
ಅಪರೋಕ್ಷಜ್ಞಾನಾ-ನಂತರ
ಅಪರೋಕ್ಷಜ್ಞಾನಾ-ನಂತರದ
ಅಪರೋಕ್ಷಜ್ಞಾನಾ-ನಂತರವೂ
ಅಪರೋಕ್ಷಜ್ಞಾನಿ-ಗಳ
ಅಪರೋಕ್ಷಜ್ಞಾನಿ-ಗಳಿಗೆ
ಅಪರೋಕ್ಷಜ್ಞಾನಿ-ಗಳೆಂಬುದು
ಅಪರೋಕ್ಷಜ್ಞಾನಿಯ
ಅಪರೋಕ್ಷೀ-ಕೃತಾತ್ಸಾಕ್ಷಾದ್ದರ್ಶನದ
ಅಪವಿತ್ರ-ವಾಗು-ವುದಿಲ್ಲ
ಅಪಹಾರ
ಅಪಾರ-ವಾದ
ಅಪಿ
ಅಪಿ-ಎಂದೆಂದಿಗೂ
ಅಪಿ-ದೇಹ-ದ-ಮೇಲಿನ
ಅಪಿ-ನೀ-ರನ್ನೂ
ಅಪಿ-ಪುಣ್ಯವೂ
ಅಪಿ-ಬೇರೆ
ಅಪಿ-ಹತ್ತಿರ-ದಲ್ಲಿಯೇ
ಅಪೂರ್ಣ-ಗುಣತಾ
ಅಪೂರ್ಣ-ನೆಂದು
ಅಪೇಕ್ಷಣೀಯಂಕಾರ-ಣ-ವಾಗಬೇಕೆಂಬ
ಅಪೇಕ್ಷಣೀಯ-ಮಿತಿ
ಅಪೇಕ್ಷಿಸಿ
ಅಪೇಕ್ಷಿಸುವವ-ರಿಗೆ
ಅಪೇಕ್ಷೆ
ಅಪ್ರಕೃತ
ಅಪ್ರಮಾಣ-ವೆಂದು
ಅಪ್ರಾಕೃತ
ಅಪ್ರಾರಬ್ದ
ಅಪ್ರಾರಬ್ದಂಫಲ-ಕೊಡಲು
ಅಪ್ರಾರಬ್ಧ
ಅಪ್ರಾರಬ್ಧ-ಕಾಮ್ಯ-ದಲ್ಲಿ
ಅಪ್ರಾರಬ್ಧ-ಪಾಪ
ಅಪ್ರಾರಬ್ಧ-ಮನಭೀಷ್ಟಂ
ಅಭಾವ
ಅಭಾವ-ವಾಚ-ಕತ್ವ
ಅಭಾವ-ವಾಚಿ-ಅ-ಭಾವ
ಅಭಾವ-ವಾಚೀ
ಅಭಾವ-ವಾಚ್ಯ-ಕಾರಃ
ಅಭಾವ-ವಾಚ್ಯ-ಕಾರಸ್ಯ
ಅಭಾವವು
ಅಭಾವಾರ್ಥತ್ವಾತ್ಅ-ಭಾವ
ಅಭಿಧಾ-ನ-ದಿಂದ
ಅಭಿಪ್ರಾಯ
ಅಭಿಪ್ರಾಯ-ದಿಂದ
ಅಭಿಪ್ರಾಯ-ದಿಂದಲೇ
ಅಭಿಮತ-ವಾಗಿದೆ
ಅಭಿ-ಮಾನ-ದಿಂದ
ಅಭಿ-ಮಾನಿ
ಅಭಿ-ಮಾನಿ-ಗ-ಳಾದ
ಅಭಿ-ಮಾನಿ-ದೇವ-ತೆ-ಗಳೂ
ಅಭಿ-ಮಾನಿ-ಯಾಗಿ-ರುವು-ದ-ರಿಂದಲೂ
ಅಭಿ-ಮಾನಿಯು
ಅಭಿ-ಮಾನಿವ್ಯಪ-ದೇಶಸ್ತು
ಅಭಿ-ಮಾನ್ಯಧಿ-ಕರ-ಣನ್ಯಾ-ಯೇನ-ಅಭಿ-ಮಾನಿವ್ಯವಸ್ಥಾ
ಅಭಿ-ವದ್ಯರ್ಥಕ-ತಯಾ
ಅಭಿ-ವಿಧ್ಯರ್ಥಕ-ತಯಾ
ಅಭಿವ್ಯಕ್ತಿಗೊಳಿಸು
ಅಭಿಷೇಕ
ಅಭಿಹಿತಂ
ಅಭೀಷ್ಟ-ಗಳನ್ನು
ಅಭೀಷ್ಟ-ವಲ್ಲದ
ಅಭೂತ್ಹುಟ್ಟಿತು
ಅಭೇದ-ಗಳಲ್ಲ
ಅಭೇದ-ದಿಂದಲೂ
ಅಭೇದೇನ
ಅಮಂಗಳ-ವಾದ
ಅಮಂದನಿಷ್ಕಲ್ಮಷ-ವಾದ
ಅಮಂದ-ಮಾನಂದ-ಮಜಸ್ರಮೇವ
ಅಮ-ರತ್ವಪ್ರ-ಸಿದ್ದೇಃ
ಅಮ-ರರೆಂದರೆ
ಅಮ-ರವಾಣಿಯೇ
ಅಮರಾಃ
ಅಮ-ರಾಃನ
ಅಮಾನೋನಾ
ಅಮಾನೋನಾಃ
ಅಮಿ-ತಪ್ರ-ಮತಿಂ
ಅಮುಕ್ತ-ರಿಂದಲೂ
ಅಮುಕ್ತ-ರಿ-ಗಿಂತಲೂ
ಅಮೂಲ್ಯ
ಅಮೃತ
ಅಮೃತನು
ಅಮೃತ-ಮುಕ್ತ
ಅಮೃತೋ
ಅಮೆಯ
ಅಯ
ಅಯಂ
ಅಯಂಈ
ಅಯಃ
ಅಯಃಆ
ಅಯಜ್ಞಾನಂಜ್ಞಾನವು
ಅಯತಿ
ಅಯತಿ-ಗಮ-ಯತಿಜ್ಞಾನದ್ವಾರಾ
ಅಯತೆ-ಗಮ್ಯ-ತೇ-ಹೊಂದಲ್ಪಡ-ತಕ್ಕ-ವನು
ಅಯತೆ-ಗಮ್ಯ-ತೇ-ಹೊಂದಲ್ಪಡುತ್ತಾನೆ
ಅಯತೇ
ಅಯಥಾರ್ಥಜ್ಞಾನ-ದಿಂದ
ಅಯನ
ಅಯನಂ
ಅಯನಂಆಶ್ರಯಂಆಶ್ರಯ-ನಾದ-ವನು
ಅಯನಂಆಶ್ರಯಃ
ಅಯನಂಆಶ್ರಯಃಆಶ್ರಯ-ವಾಗಿ-ದೆಯೋ
ಅಯನಂಆಶ್ರಯಃಆಹಾರದ
ಅಯನಂಆಶ್ರಯ-ಆಶ್ರಯ-ನಾಗಿ-ರುವು-ದ-ರಿಂದ
ಅಯನಂಆಶ್ರಯ-ನಾದ-ವನು
ಅಯನಂಆಶ್ರಯ-ವಾಗಿ-ದೆಯೋ
ಅಯನಂಆಶ್ರಯ-ವಾಯಿತು
ಅಯನಂಆಶ್ರಯವು
ಅಯನಂಕರ್ಮ-ಜ-ಶುಭ-ವಿಷಯ-ಭೋಗಾಶ್ರಯೋ
ಅಯನಂಗತಿ-ಸಾ-ಧನಂಗಮನ-ಸಾ-ಧನ-ವೈರಿನಿರಸನ-ಸಾ-ಧನ-ಮಿತಿ
ಅಯನಂಗಮನಾ-ಗಮನಂ
ಅಯನಂಗಮನಾ-ಗಮನಂಹೋಗು-ವುದು
ಅಯನಂನಾರಾ-ಯಣ
ಅಯನಂಪಂಚೇಂದ್ರಿಯ-ಗಳಿಂದ
ಅಯನಂಬ್ರಹ್ಮಾಂಡಾಂತರಾ-ಗಮನಂ
ಅಯನಂಮಾರ್ಗಃ
ಅಯನ-ಆ-ವಾಸಸ್ಥಾನಂವಾಸ-ಮಾಡುವ
ಅಯ-ನತ್ವಾತ್
ಅಯ-ನತ್ವಾತ್ಆಶ್ರಯ-ನಾದು-ದ-ರಿಂದ
ಅಯನನೂ
ಅಯನ-ಮಾಶ್ರ-ಯಶ್ಚ
ಅಯನ-ಮಾಶ್ರಯೋ
ಅಯನ-ಮಿತಿ
ಅಯನಾನಿ
ಅಯನಾನಿ-ಧಾ-ಮಾನಿ
ಅಯಮೇವ
ಅಯೋ
ಅಯೋಗ್ಯ-ಕರ್ಮ-ಮಾಡಿ-ದ-ವರೆಲ್ಲ
ಅಯ್
ಅರ
ಅರ-ಋ-ಕಾರವು
ಅರ-ತಿರಹಿ-ತತ್ವಾತ್ಕ್ರೀಡಾದಿ
ಅರ-ಮಣಂ
ಅರ-ವಃಇಂತಹ
ಅರ-ವಃದೋಷ-ಯುಕ್ತ-ರಾದ-ವರು
ಅರ-ಶಬ್ದೋ
ಅರಶ್ಚಾಸೌ
ಅರಾ
ಅರಾಃ
ಅರಾಃಜಗನ್ನಿಥ್ಯಾದಿ
ಅರಾಃದೋಷ-ಗಳು
ಅರಾಃದೋಷಾಃದೋಷ-ಗಳು
ಅರಾಃನಾರಾಃ
ಅರಾನ್
ಅರಾನ್ಇ-ರುವು-ದ-ರಿಂದ
ಅರಾನ್ದೋಷಾನ್ಪಾಪ-ಗಳನ್ನು
ಅರಾಯ
ಅರಾಯಃ
ಅರಾ-ಯಃಅರಾ-ಯನು
ಅರಾ-ಯಃಹೀಗಿ-ರುವು-ದ-ರಿಂದ
ಅರಾ-ಯಣ
ಅರಾ-ಯ-ಣಃಹೊಂದಲ್ಪಡುವ-ವನು
ಅರಾ-ಯ-ಣನು
ಅರಾ-ಯಣೋ
ಅರಾ-ಯೇತ್ಯತ್ರ
ಅರಾಶ್ರಯಃ
ಅರಾಶ್ರಯೋ
ಅರಿಯದೆ
ಅರುಣ
ಅರೇತಿ
ಅರೇತ್ಯತ್ರ
ಅರೇತ್ಯತ್ರನ
ಅರೇತ್ಯುಚ್ಯತೇಆ
ಅರೈಃ
ಅರ್ಚ-ಕರೂ
ಅರ್ಜಿ-ತತ್ತೇನ-ಸಂಪಾದಿಸಲ್ಪಟ್ಟ
ಅರ್ಜೀ-ತತ್ವೇನ
ಅರ್ಜುನ
ಅರ್ಜುನ-ಅರ್ಜುನನೇ
ಅರ್ಜುನನ
ಅರ್ಜುನ-ನಿಗೆ
ಅರ್ಜು-ನನು
ಅರ್ಜುನನೇ
ಅರ್ಜುನ-ನೊಡನೆ
ಅರ್ಜು-ನಾದಿ
ಅರ್ಜು-ನಾದಿ-ರೂಪೇಣ
ಅರ್ಥ
ಅರ್ಥಃ
ಅರ್ಥಃಅರ್ಥವು
ಅರ್ಥ-ಉಳ್ಳದ್ದು
ಅರ್ಥಕ್ಕೆ
ಅರ್ಥ-ಗಳನ್ನು
ಅರ್ಥ-ಗಳನ್ನೇ
ಅರ್ಥ-ಗಳಲ್ಲಿ
ಅರ್ಥ-ಗಳಿಂದ
ಅರ್ಥ-ಗಳಿಂದಲೂ
ಅರ್ಥ-ಗಳು
ಅರ್ಥದ
ಅರ್ಥ-ದಲ್ಲಿ
ಅರ್ಥದ್ವಯೇ-ಎರಡು
ಅರ್ಥ-ಮಾಡಿ-ದರೆ
ಅರ್ಥ-ಮಾಡಿ-ರುವುದಕ್ಕೆ
ಅರ್ಥ-ವನ್ನು
ಅರ್ಥ-ವನ್ನುಳ್ಳ
ಅರ್ಥ-ವನ್ನೂ
ಅರ್ಥ-ವನ್ನೇ
ಅರ್ಥ-ವಾಗಿದೆ
ಅರ್ಥ-ವಾ-ದರೂ
ಅರ್ಥ-ವಿ-ರುವ
ಅರ್ಥ-ವಿ-ರುವು-ದ-ರಿಂದಲೂ
ಅರ್ಥ-ವಿ-ವರ-ಣೆ-ಯನ್ನು
ಅರ್ಥ-ವಿ-ಶೇಷ-ವನ್ನು
ಅರ್ಥವು
ಅರ್ಥ-ವುಳ್ಳ
ಅರ್ಥ-ವುಳ್ಳದ್ದು
ಅರ್ಥ-ವುಳ್ಳ-ವು-ಗಳು
ಅರ್ಥ-ವೆಂದೂ
ಅರ್ಥವೇ
ಅರ್ಥ-ವೈ-ವಿಧ್ಯತೆ-ಯಿಂದಲೂ
ಅರ್ಥಾಃ
ಅರ್ಪಿ-ಸಿಕೊಳ್ಳುತ್ತಾನೆ
ಅರ್ಪಿಸುತ್ತೇನೆ
ಅರ್ಭಕಾಃಬಾಲ-ಕರೇ
ಅರ್ಭಕಾಣಾಂ
ಅರ್ಶ
ಅರ್ಶಾದ್ಯಜಂತಃಅರ
ಅರ್ಹತೆ-ಯನ್ನುಳ್ಳ
ಅರ್ಹನಲ್ಲ
ಅರ್ಹ-ನಾದ-ವನೇ
ಅಲಂಕಾರ-ವಸ್ತು-ಗಳೆಲ್ಲವೂ
ಅಲ್ಪಾಂಶ
ಅಲ್ಲ
ಅಲ್ಲ-ದಿ-ರುವು-ದ-ರಿಂದ
ಅಲ್ಲದೆ
ಅಲ್ಲದೇ
ಅಲ್ಲವೇ
ಅಲ್ಲಿ
ಅಲ್ಲಿಂದ
ಅಲ್ಲಿಗೆ
ಅಲ್ಲಿಯೂ
ಅಲ್ಲಿಯೇ
ಅವ-ಕಾಶ
ಅವ-ಕಾಶಃಸಾಕಷ್ಟುಸ್ಥಳ
ಅವ-ಕಾಶಪ್ರದಃ
ಅವ-ಕಾಶಪ್ರದ-ನಾದು-ದ-ರಿಂದ
ಅವ-ಕಾಶಪ್ರದೋ
ಅವ-ಕಾಶವೇ
ಅವಗಂತವ್ಯಂ
ಅವಗಂತವ್ಯಂತಿಳಿಯ-ಬೇಕು
ಅವಗಂತವ್ಯಂಹೀಗೆ
ಅವ-ಗತಿ
ಅವ-ಗತ್ಯರ್ಥತ್ವಾತ್ಗತಿ-ಯನ್ನು
ಅವಗತ್ಯಾರ್ಥಾ-ಗತಿ
ಅವತ-ರಿಸಿ
ಅವತಾರ
ಅವತಾರ-ಗಳ
ಅವತಾರ-ಗಳಲ್ಲಿ
ಅವತಾರ-ಗಳಲ್ಲಿಯೂ
ಅವತಾರ-ಗಳಿಂದ
ಅವತಾರ-ಗಳು
ಅವತಾರತ್ರಯ
ಅವತಾರ-ಭೂತ-ರಾದ
ಅವತಾರ-ಭೂತ-ರಾದು-ದ-ರಿಂದ
ಅವತಾರ-ಭೂತ-ರೆಂದು
ಅವತಾರ-ಮಾಡಿ-ದಾಗ
ಅವತಾರ-ಮಾಡಿದ್ದಾಗ
ಅವತಾರ-ಮಾಡಿ-ರುವ
ಅವತಾರ-ಮಾಡಿ-ರುವು-ದ-ರಿಂದ
ಅವತಾರ-ರೂಪ-ಗಳನ್ನು
ಅವತಾರ-ರೂಪ-ಗಳಿಗೆ
ಅವತಾರ-ರೂಪ-ಗಳು
ಅವತಾರೂಪ-ಗಳಿಗೂ
ಅವತೀರ್ಣತ್ವೇನ
ಅವಧ್ಯ-ರಾದ
ಅವನ
ಅವ-ನನ್ನು
ಅವನವೇ
ಅವ-ನಿಂದಲೇ
ಅವ-ನಿಂದ-ಲೇ-ಅನ್ಯ-ರಿಂದಲ್ಲ
ಅವ-ನಿಗೆ
ಅವನು
ಅವನೇ
ಅವ-ಯವ-ಗಳಲ್ಲಿ
ಅವರ
ಅವ-ರನ್ನು
ಅವ-ರಲ್ಲಿ
ಅವರ-ವರ
ಅವ-ರಿಂದ
ಅವ-ರಿಗೆ
ಅವರು
ಅವರೇ
ಅವರೊಂದಿಗೆ
ಅವ-ರೊಡನೆಯೇ
ಅವರೋಹೌ-ನರ-ಕ-ದಿಂದ
ಅವಲಂಬಿಸಿ-ರುತ್ತದೆ
ಅವಲಂಬಿಸಿ-ರುವ-ವ-ರಿಂದ
ಅವಶ್ಯಕತೆ
ಅವಶ್ಯಕತೆ-ಯನ್ನು
ಅವಸ್ಥಾ
ಅವಸ್ಥೆ-ಗಳ
ಅವಸ್ಥೆ-ಗಳನ್ನು
ಅವಸ್ಥೆ-ಗಳಲ್ಲಿಯೂ
ಅವಸ್ಥೆ-ಗಳಿಗೆ
ಅವಸ್ಥೆ-ಗಳು
ಅವಾಂತರ
ಅವಾಸಸ್ಥಾನಂ
ಅವಾಸಸ್ಥಾನಂಅ-ಯನಂವಾಸಸ್ಥಳ-ವಾಗಿ-ದೆಯೋ
ಅವಿಚ್ಛಿನ್ನ
ಅವಿದ್ಯಾ
ಅವಿದ್ಯೆ-ಯನ್ನು
ಅವಿರೋಧಃಪರಮಾತನು
ಅವಿರೋಧಾಧ್ಯಾಯ
ಅವಿಲುಪ್ತಮನೋವೃತ್ತಿ
ಅವಿಷ್ಟ-ನಾಗಿ-ರುವ
ಅವಿಹಾಯ-ಅವ-ನನ್ನು
ಅವು
ಅವು-ಗಳ
ಅವು-ಗಳನ್ನು
ಅವು-ಗಳಲ್ಲಿ
ಅವು-ಗಳಿಗೆ
ಅವು-ಗಳು
ಅವು-ಗಳೇ
ಅವ್ಯಕ್ತ-ನಾದ
ಅವ್ಯಯಂನಾಶ-ರಹಿತ-ವಾದ
ಅವ್ಯಾ-ಕೃತಾ-ಕಾಶ-ಗಳ
ಅವ್ಯಾ-ಕೃತಾ-ಕಾಶದ
ಅಶಿವಾಂ
ಅಶುಚಿತ್ವಾ-ದಿಕಂ
ಅಶುದ್ಧ-ವಸ್ತ್ರ
ಅಶುಭ
ಅಶುಭ-ಫಲ-ಗಳ
ಅಶುಭ-ವಾಗಲಿ
ಅಶೆ-ಗಳ
ಅಶೇಷ
ಅಶೇಷ-ದೋಷಾಣಾಂಯಾಪ-ನಾತ್
ಅಶೇಷಾಮ್ನಾಯಾಃ
ಅಶೋತ್ತರ-ಗಳನ್ನು
ಅಶ್ಚಾಸೌ
ಅಶ್ಲೇಷ-ಗಳು
ಅಶ್ಲೇಷ-ವಿನಾ-ಶೌ-ನಂತರ
ಅಶ್ವಮೇಧ
ಅಶ್ವಾಸೌ
ಅಶ್ವಿನಿ
ಅಷ್ಟ-ಕರ್ತೃತ್ವ
ಅಷ್ಟ-ಕರ್ತೃತ್ವ-ಕಾರ-ಣ-ದಿಂದ
ಅಷ್ಟ-ಕರ್ತೃತ್ವ-ವನ್ನು
ಅಷ್ಟಕ್ಕೂ
ಅಷ್ಟೋತ್ತರಶತ-ನಾಮಾ-ವಳಿ-ಯಲ್ಲಿ-ನರ-ಶಬ್ದಸ್ಯ
ಅಷ್ಟೌ
ಅಷ್ಟೌಗ್ರಾವಾಣ-ಎಂಟು
ಅಷ್ಟೌಗ್ರಾವ್ಣಃ
ಅಸಂಖ್ಯಾತ
ಅಸಜ್ಜ-ನರ
ಅಸಜ್ಜನ-ರಿಗೆ
ಅಸಜ್ಜನಾನ್
ಅಸತ್ಅವನೇ
ಅಸತ್ಆಗು-ವನು
ಅಸದುಪಾಸನಯಾಮಿಥ್ಯಾ
ಅಸದುಪಾಸನಯಾಽತ್ಮಹಾ-ನಃಇತಿ
ಅಸದೃಶ-ವಾದ
ಅಸದೃಶ-ವಾದ-ಭಕ್ತಿ-ಭರಿತ-ರಾದ
ಅಸಾಧಾರಣ
ಅಸಾಧಾರಣ-ವಾದ
ಅಸಾಧ್ಯ-ನಾದು-ದ-ರಿಂದ
ಅಸಾರ್ವತ್ರಿಕಮೀಶೇ
ಅಸು
ಅಸುರ
ಅಸು-ರ-ಯೋನಿ-ಯಲ್ಲಿ
ಅಸುರಾ
ಅಸೌ
ಅಸೌ-ಅಂತಹ-ವನು
ಅಸೌಈ
ಅಸೌ-ದೇವ-ತೆ-ಗಳ
ಅಸೌ-ಸಮ್ಯಕ್ಪೂರ್ಣ-ವಾದ
ಅಸ್ಕಿನ್ಈ
ಅಸ್ಮಿನ್
ಅಸ್ಯ-ಇವನ
ಅಸ್ಯ-ಜೀವ-ನಿಗೆ
ಅಸ್ವತಂತ್ರರು
ಅಸ್ಸಾತ್
ಅಹಂ
ಅಹಂನಾನು
ಅಹಮೇವಂ
ಅಹಿತಶತ್ರು-ವಾದ
ಅಹೋಆಶ್ಚರ್ಯ-ಹೀಗೆಂದು
ಆ
ಆಂದೋಲಿಕಾ
ಆಅರಾ
ಆಕಾರ
ಆಕಾಶ
ಆಕಾಶಃಸೂಕ್ಷಾ-ಕಾಶವು
ಆಕಾಶಕ್ಕಿಂತಲೂ
ಆಕ್ರಮಿ-ಸಿದ
ಆಗ
ಆಗ-ಬಹು-ದಾದ
ಆಗ-ಬೇಕು
ಆಗಮೋಕ್ತವೃತ್ತಿ
ಆಗಲಿ
ಆಗಲೇ
ಆಗಾಮಿ
ಆಗಿ
ಆಗಿದೆ
ಆಗಿ-ದೆಯೋ
ಆಗಿ-ರುತ್ತದೆ
ಆಗಿ-ರುತ್ತಾನೆಯೋ
ಆಗಿ-ರುತ್ತಾರೆ
ಆಗಿ-ರುತ್ತಾಳೆ
ಆಗಿ-ರುತ್ತೀಯೆ
ಆಗಿ-ರುವ
ಆಗಿ-ರುವನೋ
ಆಗಿ-ರುವಳೋ
ಆಗಿ-ರುವ-ವನು
ಆಗಿ-ರುವು-ದ-ರಿಂದ
ಆಗಿ-ರುವು-ದ-ರಿಂದಲೂ
ಆಗಿ-ರುವುವೋ
ಆಗುತ್ತದೆ
ಆಗುತ್ತದೆಯೋ
ಆಗುತ್ತವೆ
ಆಗುತ್ತವೆಯೋ
ಆಗುತ್ತಾನೆ
ಆಗುತ್ತಾನೆಯೋ
ಆಗುವ
ಆಗು-ವನು
ಆಗುವ-ವನಲ್ಲ
ಆಗು-ವುದಿಲ್ಲ
ಆಗು-ವುದಿಲ್ಲ-ವಾದು-ದ-ರಿಂದ
ಆಗು-ವುದಿಲ್ಲ-ವೆಂದರ್ಥ
ಆಗು-ವುದಿಲ್ಲ-ವೋ-ಪಾಪಾತ್ಮ-ರಿಗೆ
ಆಚತುರ್ದಶಮಾದ್ವರ್ಷಾತ್ಕರ್ಮಾಣಿ
ಆಚರಣೆ
ಆಚರಣೆ-ಯಲ್ಲಿ
ಆಚಾರ್ಯರ
ಆಚಾರ್ಯರು
ಆಚಾರ್ಯ-ವಾನ್
ಆಜ್ಞೆ-ಯಿಂದ
ಆಟಪಾಟ-ಗಳಲ್ಲಿಯೇ
ಆಡನ್ನು
ಆತನು
ಆತ್ಮ
ಆತ್ಮಕ್ರೀಡ
ಆತ್ಮಕ್ರೀಡಃ
ಆತ್ಮನಃ
ಆತ್ಮ-ನಃತನ್ನ
ಆತ್ಮ-ನಿ-ಪರಮಾತ್ಮನು
ಆತ್ಮ-ಮಿಥುನ
ಆತ್ಮ-ಮಿಥುನಃ
ಆತ್ಮ-ಹನಃತಮ್ಮ
ಆತ್ಮಾಂತರಾತ್ಮ-ರೂಪೀ
ಆತ್ಮಾಂತರಾತ್ಮ-ರೂಪೀ-ಆತ್ಮ
ಆತ್ಮಾಂತರಾತ್ಮೇತಿ
ಆತ್ಮಾಂತರಾತ್ಮೇತಿ-ಆತ್ಮ
ಆತ್ಮಾನೌ
ಆತ್ಮಾನೌ-ಆತ್ಮ-ಶಬ್ದ-ವಾಚ್ಯ-ರಾದ
ಆತ್ಮೀಯಮ್
ಆದ
ಆದರೂ
ಆದರೆ
ಆದಾವಂತೇ
ಆದು-ದ-ರಿಂದ
ಆದೇಶ-ದಿಂದ
ಆದೇಶ-ರೂಪಾಂತರ-ವಾಗಿದೆ
ಆದ್ದ-ರಿಂದ
ಆದ್ಯಂ
ಆಧಾರ
ಆಧಾರಕ್ಕಾಗಿ
ಆಧಾರ-ತಯಾ
ಆಧಾರ-ದಾತ-ನಾದ
ಆಧಾರ-ನಾಗಿ
ಆಧಾರ-ನಾಗಿ-ರುವು-ದ-ರಿಂದ
ಆಧಾರ-ನಾದ
ಆಧಾರ-ವಿಲ್ಲದೇ
ಆಧಿಕ್ಯ
ಆಧಿಕ್ಯ-ಬೋಧಕ
ಆಧಿಕ್ಯ-ಮುಕ್ತಾ-ಮುಕ್ತ-ರಿಂದ
ಆಧಿಕ್ಯ-ವಕ್ತೃತಾ
ಆಧಿಕ್ಯವು
ಆನಂತ್ಯಾತ್ಕೊನೆಯೇ
ಆನಂದ
ಆನಂದಂ
ಆನಂದಂಆನಂದ-ವನ್ನು
ಆನಂದಂಸುಖವು
ಆನಂದಃ
ಆನಂದ-ತೀರ್ಥ-ರಾಗಿ
ಆನಂದ-ತೀರ್ಥ-ರಿಗೆ
ಆನಂದದ
ಆನಂದ-ದಶ್ಚ
ಆನಂದ-ದಾತನೂ
ಆನಂದ-ದಾಯಕ-ವಾದ
ಆನಂದ-ಪಡುತ್ತಾನೆ
ಆನಂದ-ಪಡುತ್ತಿದ್ದಾನೆ
ಆನಂದಪ್ರದ-ನೆಂದು
ಆನಂದ-ರೂಪತ್ವಾತ್
ಆನಂದ-ರೂಪತ್ವಾತ್ಆನಂದ-ರೂಪ-ವನ್ನು
ಆನಂದ-ರೂಪ-ನಾದ
ಆನಂದ-ರೂಪಿ
ಆನಂದ-ರೂಪಿ-ಯಾಗಿ-ರುವ
ಆನಂದ-ರೂಪೀ
ಆನಂದ-ವನ್ನು
ಆನಂದ-ವಾಚ-ಕ-ವಾ-ದುದು
ಆನಂದ-ವಿ-ಶೇಷ-ವನ್ನು
ಆನಂದವು
ಆನಂದ-ವುಳ್ಳ-ವನು
ಆನಂದವೇ
ಆನಂದ-ಶಬ್ದ-ವಾಚ್ಯ-ನಾದ
ಆನಂದಶ್ಯ
ಆನಂದಸ್ಯ
ಆನಂದಸ್ವ-ರೂಪ-ನಾದ
ಆನಂದಸ್ವ-ರೂಪಿ-ಯಾದ
ಆನಂದಾತಿಶ-ಯಕ್ಕೆ
ಆನಂದಾದಿ
ಆನಂದೀ
ಆನಯ-ಕರೆದು-ಕೊಂಡು
ಆನ-ಯತಿ
ಆನ-ಯತಿ-ಕರೆ-ಸಿಕೊಳ್ಳುತ್ತಾನೆ
ಆಪಃಅವ-ನಿಂದ
ಆಪಃಜಲವು
ಆಪಃನೀರು
ಆಪ-ತತ್
ಆಪಾದನೆ
ಆಪಾದ-ನೆ-ಯನ್ನು
ಆಪೋ
ಆಪ್ಪ-ಗುರು-ವಿ-ನೊಡನೆ
ಆಪ್ಯ
ಆಪ್ಯ-ಬೃಹಸ್ಪತಿ-ಯನ್ನು
ಆಮರ್ಯಾದ-ಯಾ-ಅನು-ಸಾರ-ವಾಗಿ
ಆಮೆಯ
ಆಯ
ಆಯಃ
ಆಯಃಆಯ
ಆಯನ
ಆಯನಂ
ಆಯನಃ
ಆಯನಃಆಯಾನ್ಸಮಸ್ತ
ಆಯನಶ್ಚೇತಿ
ಆಯಶ್ಚಾಸೌ
ಆಯಾಂತಂ
ಆಯಾನ್
ಆಯಿತು
ಆಯುಧ
ಆಯುಧ-ಗಳನ್ನು
ಆಯುಧ-ಗಳು
ಆಯುಧ-ಸಮೂಹವೇ
ಆಯುಷ್ಯವು
ಆಯ್
ಆಯ್ಯಾ
ಆರ
ಆರಂಭವಾಗದೇ
ಆರಃಆರ
ಆರಃಯಸ್ಯ
ಆರನೂ
ಆರಶ್ಚಾಸೌ
ಆರು
ಆರು-ಣಯಃಸೂಕ್ಷ್ಮಜ್ಞಾನಿ-ಗಳಾದ
ಆರೋಹ
ಆರ್ಜಿ-ತತ್ವೇನ-ಸಂಪಾದಿಸಿದ
ಆಲಂಗಿ-ಸಿಕೊಳ್ಳುತ್ತಾನೆ
ಆಲದ
ಆಲದೆಲೆ-ಯಾಗಿ
ಆಲಿಂಗ-ನದ
ಆಲಿಂಗ-ನಾದಿ-ಗಳನ್ನು
ಆಲೋ-ಚನಾ
ಆಲೋ-ಚನಾಚ್ಚ
ಆಲೋ-ಚನೆ
ಆಲೋ-ಚನೆ-ಯಿಂದ
ಆವರ-ಣ-ದಲ್ಲಿ-ರುವ
ಆವಹ
ಆವಾಸಸ್ಥಾನ-ವಾದ
ಆವಾಹನ
ಆವಿರ್ಭೂತ-ವಾದುದೇ
ಆವಿಷ್ಟೋ
ಆವಿಷ್ಟೋ-ಸನ್ನಿಧಾನ-ದಿಂದ
ಆವೃತ-ವಾದ
ಆವೃತ್ತಿಃನಾರ
ಆವೇಶ-ವನ್ನು
ಆವೇಶ-ವನ್ನುಂಟು-ಮಾಡಿ
ಆಶುಚ್ಯಾದಿ
ಆಶ್ಚರ್ಯ-ವೇನು
ಆಶ್ರಯ
ಆಶ್ರಯಃ
ಆಶ್ರಯಃಆಶ್ರಯ-ದಾತನು
ಆಶ್ರಯಃಆಶ್ರಯನು
ಆಶ್ರಯಃಆಶ್ರಯವು
ಆಶ್ರಯ-ಕೊಟ್ಟು
ಆಶ್ರಯತ್ವ
ಆಶ್ರಯತ್ವಾತ್
ಆಶ್ರಯ-ದಲ್ಲಿ
ಆಶ್ರಯ-ದಲ್ಲಿದೆ
ಆಶ್ರಯ-ದಲ್ಲಿಲ್ಲ
ಆಶ್ರಯ-ದಾತ
ಆಶ್ರಯ-ದಾತ-ನಂತೆ
ಆಶ್ರಯ-ದಾತ-ನಾದ
ಆಶ್ರಯ-ದಾತ-ನಾದು-ದ-ರಿಂದ
ಆಶ್ರಯ-ದಾತನು
ಆಶ್ರಯ-ದಿಂದಲೇ
ಆಶ್ರಯ-ನಾಗಿ
ಆಶ್ರಯ-ನಾಗಿದ್ದನು
ಆಶ್ರಯ-ನಾಗಿದ್ದ-ನೆಂಬ
ಆಶ್ರಯ-ನಾಗಿ-ರುತ್ತಾನೆ
ಆಶ್ರಯ-ನಾಗಿ-ರುತ್ತಾನೆಂದರ್ಥ
ಆಶ್ರಯ-ನಾಗಿ-ರುವ
ಆಶ್ರಯ-ನಾಗಿ-ರುವ-ವನು
ಆಶ್ರಯ-ನಾಗಿ-ರುವು-ದ-ರಿಂದ
ಆಶ್ರಯ-ನಾಗಿ-ರುವು-ದ-ರಿಂದ-ನಾರಂ
ಆಶ್ರಯ-ನಾದ
ಆಶ್ರಯ-ನಾದ-ವನು
ಆಶ್ರಯ-ನಾದು-ದ-ರಿಂದ
ಆಶ್ರಯನು
ಆಶ್ರಯನೂ
ಆಶ್ರಯ-ನೆಂದು
ಆಶ್ರಯ-ಪಡೆ-ದಿ-ರುವ-ವ-ನನ್ನು
ಆಶ್ರಯಪ್ರದ-ನಾದು-ದ-ರಿಂದ
ಆಶ್ರಯ-ಭೂತ-ನಾಗಿ-ರುವ-ವನು
ಆಶ್ರಯ-ಭೂತ-ನಾಗಿ-ರುವು-ದ-ರಿಂದ
ಆಶ್ರಯ-ಭೂತ-ನಾದುದ-ರಲ್ಲಿ
ಆಶ್ರಯ-ಭೂತ-ವಾದ
ಆಶ್ರಯ-ರಾಗಲು
ಆಶ್ರಯ-ರಾದರು
ಆಶ್ರಯಳು
ಆಶ್ರಯ-ವನ್ನು
ಆಶ್ರಯ-ವಾಗಿ
ಆಶ್ರಯ-ವಾಗಿ-ದೆಯೋ
ಆಶ್ರಯ-ವಾದ
ಆಶ್ರಯ-ವಾಯಿತು
ಆಶ್ರಯವು
ಆಶ್ರಯವೂ
ಆಶ್ರಯವೋ
ಆಶ್ರಯಸ್ಥಾನವು
ಆಶ್ರಯಿಸಿ-ಕೊಂಡಿವೆ
ಆಶ್ರಯೋ
ಆಸಕ್ತ-ನಾಗಿ
ಆಸಕ್ತಿ
ಆಸಮಂತಾತ್ಸರ್ವಸ್ಮಿನ್
ಆಸಮಸ್ತ-ವಾದ
ಆಸಮ್ಯಕ್ಪೂರ್ಣ-ವಾದ
ಆಸುರೀಂ
ಆಸುರೇಣೈವ
ಆಸ್ತಿ
ಆಸ್ತಿಕ
ಆಹಾರ
ಆಹಾರ-ಗಳ
ಆಹಾರ-ಪದಾರ್ಥ-ಗಳಲ್ಲಿ
ಆಹಾರ-ವನ್ನು
ಆಹಾರವು
ಆಹೂತ
ಇಂಗ್ಲಿಷ್
ಇಂತಹ
ಇಂಥಹ
ಇಂದಿಗೂ
ಇಂದು
ಇಂದ್ರ
ಇಂದ್ರ-ನನ್ನು
ಇಂದ್ರನು
ಇಂದ್ರನೇ
ಇಂದ್ರಾಂಶ
ಇಂದ್ರಾಂಶರೇ
ಇಂದ್ರಾದಿ
ಇಂದ್ರಿಯ-ಗಳ
ಇಂದ್ರಿಯ-ಗಳಲ್ಲಿ
ಇಂದ್ರಿಯ-ಗಳಿಂದ
ಇಂದ್ರಿಯ-ಗಳಿಗೆ
ಇಂದ್ರಿಯ-ಗಳು
ಇಂದ್ರಿಯ-ಗಳೂ
ಇಂದ್ರಿಯಪಂಚಕಂ
ಇಂದ್ರೋ
ಇಕಾರಸ್ಯಇ
ಇಚ್ಚಿಸುವುದೇ
ಇಚ್ಛಯಾ
ಇಚ್ಛಾ
ಇಚ್ಛಾ-ನು-ಸಾರ
ಇಚ್ಛಾ-ನು-ಸಾರ-ವಾಗಿ
ಇಚ್ಛೆಗೆ
ಇಚ್ಛೆ-ಯಂತೆ
ಇಚ್ಛೆ-ಯಿಂದ
ಇಚ್ಛೆ-ಯಿಂದಲೇ
ಇಚ್ಛೆ-ಯಿಂದ-ಲೇ-ಜಾಗೃದವಸ್ಟಾಪ್ರ-ವರ್ತ-ಕನೂ
ಇಚ್ಛೆಯು
ಇಚ್ಛೆಯುಳ್ಳ
ಇಚ್ಛೆಯುಳ್ಳ-ವ-ರಿಗೆ
ಇಟ್ಟಿ-ರುತ್ತಾನೆ
ಇಟ್ಟಿ-ರುವ
ಇಟ್ಟು
ಇಡಲು
ಇಡೀ
ಇಡುತ್ತಾನೆ
ಇಡುವ
ಇಣ್
ಇತರ
ಇತರರ
ಇತರ-ರಿಗೆ
ಇತರರು
ಇತ-ರೇಷಾಂ
ಇತ-ರೇಷಾಂಇತರ-ರಿಗೆ
ಇತಿ
ಇತಿಅ
ಇತಿ-ಅ-ಭಾವ-ವಾಚಿ-ಅ-ಭಾವ-ವೆಂದು
ಇತಿ-ಇ-ದ-ರಿಂದ
ಇತಿಈ
ಇತಿ-ಕೆಲ-ವರ
ಇತಿ-ತಂದೊದಗಿಸುತ್ತಾನೆ
ಇತಿನ
ಇತಿ-ನಾರಾ-ಯಣ
ಇತಿ-ನಿತ್ಯಂನಿರಂತರ-ವಾಗಿ
ಇತಿ-ಪವಿತ್ರ-ನನ್ನಾಗಿ
ಇತಿಪ್ರಮಾಣ-ವೆಂದು
ಇತಿಪ್ರಾಗುಕ್ತ
ಇತಿ-ಬಲ
ಇತಿ-ಯಾವು-ದ-ರಿಂದ
ಇತಿ-ವತ್
ಇತಿ-ವತ್ಇದರಂತೆ
ಇತಿ-ಸಮಸ್ತ-ರಿಗೂ
ಇತಿ-ಹಾಸ-ಗಳು
ಇತಿ-ಹೀಗಿ-ರುವು-ದ-ರಿಂದ
ಇತಿ-ಹೀಗಿ-ರು-ವುದು
ಇತಿ-ಹೀಗೆಂದು
ಇತಿ-ಹೀಗೆಂಬ
ಇತಿ-ಹೇಳ-ಲಾಗಿದೆ
ಇತ್ಯಕ್ಷರ
ಇತ್ಯಕ್ಷರ-ವಿ-ಭಾಗಃನ
ಇತ್ಯತ್ರ
ಇತ್ಯತ್ರನೃ
ಇತ್ಯರಾಯಃಅರೈಃದೋಷ-ವದ್ಭಿಃದೋಷ
ಇತ್ಯಾದಿ
ಇತ್ಯಾದಿ-ಗಳನ್ನು
ಇತ್ಯಾದಿ-ಯಾಗಿ
ಇತ್ಯಾದಿಸ
ಇತ್ಯಾದಿಹೇ
ಇತ್ಯಾದ್ಯಾಃ
ಇತ್ಯಾಶ-ಯೇನ
ಇತ್ಯುಕೇಃಅ-ಕಾರಕ್ಕೆ
ಇತ್ಯುಚ್ಯತೇ-ಅಯ್ಪಯ್
ಇತ್ಯುಚ್ಯತೇ-ಎಂದು
ಇತ್ಯುಚ್ಯತೇ-ನರಾ-ಣಾಂಶಿಷ್ಯಾಣಾಂಶಿಷ್ಯ-ರಿಗೆ
ಇತ್ಯುಚ್ಯತೇ-ಯ-ಕಾರ-ವಾಯಿತು
ಇತ್ಯುನುವ್ರಜಾಮಿ
ಇದಂ
ಇದಂಯಾ-ವುದು
ಇದಂಸಂಬಂಧ-ಪಟ್ಟಿದ್ದು
ಇದಂಸಂಬಂಧಿ-ಸಿ-ದುದು
ಇದನ್ನು
ಇದನ್ನೇ
ಇದರ
ಇದ-ರಲ್ಲಿ
ಇದ-ರಿಂದ
ಇದು
ಇದೆ
ಇದೆಯೋ
ಇದೇ
ಇದ್ದ
ಇದ್ದಂತೆ
ಇದ್ದರೂ
ಇದ್ದರೆ
ಇದ್ದಾನೆ
ಇದ್ದಿದ್ದರೆ
ಇದ್ದು
ಇದ್ದು-ದ-ರಿಂದ
ಇದ್ದೇ
ಇನ್ನೂ
ಇನ್ನೊಂದು
ಇಬ್ಬರು
ಇಬ್ಬರೂ
ಇಮಂ
ಇಮಾಂ
ಇಮಾನಿ
ಇಮೇ
ಇಮೇ-ಚೇತನ-ವರ್ಗಕ್ಕೆ
ಇರಬೇಕಾಗಿತ್ತು
ಇರಲಾರದು
ಇರಲು
ಇರುತಿಹನು
ಇರುತ್ತದೆ
ಇರುತ್ತದೆಯೋ
ಇರುತ್ತವೆ
ಇರುತ್ತಾನೆ
ಇರುತ್ತಾನೆಂದು
ಇರುತ್ತಾನೆಯೋ
ಇರುತ್ತಾರೆ
ಇರುತ್ತಾಳೆಯೋ
ಇರುತ್ತಿದ್ದರೆ
ಇರುವ
ಇರುವನು
ಇರುವ-ನೆಂದು
ಇರುವನೋ
ಇರುವ-ರೆಂದು
ಇರುವಳು
ಇರುವ-ವ-ನಾದು-ದ-ರಿಂದ
ಇರುವ-ವನು
ಇರುವ-ವ-ರಾಗಿಯೇ
ಇರುವ-ವರು
ಇರುವ-ವ-ಳಾದು-ದ-ರಿಂದ
ಇರುವಾಗ
ಇರುವುದಕ್ಕಿಂತಲೂ
ಇರುವು-ದ-ರಿಂದ
ಇರುವು-ದ-ರಿಂದಲೂ
ಇರು-ವುದಿಲ್ಲ
ಇರು-ವುದಿಲ್ಲವೋ
ಇರು-ವುದು
ಇರುವುದೆಂಬ
ಇರುವುದೇ
ಇಲ್ಲ
ಇಲ್ಲದ
ಇಲ್ಲ-ದಂತೆ
ಇಲ್ಲ-ದ-ವನೂ
ಇಲ್ಲ-ದಿದ್ದರೆ
ಇಲ್ಲ-ದಿ-ರುವ
ಇಲ್ಲ-ದಿ-ರುವಿಕೆಯು
ಇಲ್ಲ-ದಿ-ರುವು-ದ-ರಿಂದ
ಇಲ್ಲ-ದಿ-ರು-ವುದು
ಇಲ್ಲದೆ
ಇಲ್ಲದೇ
ಇಲ್ಲ-ವೆಂದು
ಇಲ್ಲ-ವೆಂಬುದು
ಇಲ್ಲ-ವೆನ್ನು-ವುದು
ಇಲ್ಲವೇ
ಇಲ್ಲವೋ
ಇಲ್ಲ-ಹೀಗೆಂದು
ಇಲ್ಲಿ
ಇಲ್ಲಿಯ
ಇಳಿದು
ಇವ
ಇವನ
ಇವ-ನಿಂದ
ಇವನು
ಇವರ
ಇವ-ರನ್ನು
ಇವ-ರಿಂದ
ಇವ-ರಿಗೆ
ಇವರು
ಇವು
ಇವು-ಗಳ
ಇವು-ಗಳಲ್ಲಿದ್ದು
ಇವು-ಗಳಿಂದ
ಇವು-ಗಳಿಗೆ
ಇವು-ಗಳೇ
ಇವೆ
ಇವೆಯೋ
ಇವೆ-ರಡೂ
ಇವೇ
ಇಷ್ಟ
ಇಷ್ಟಂತನ್ನ
ಇಷ್ಟಾನಿಷ್ಟ
ಇಷ್ಟು
ಈ
ಈಗ
ಈಗಲೂ
ಈಡೇ-ರಿಸಿದ
ಈಶ
ಈಶಃಪ್ರಭು-ವಾದ
ಈಶಾವಾಸ್ಯಭಾಷ್
ಈಶೋಪನಿಷತ್ತು
ಉಂಟಾಗ-ದಂತೆ
ಉಂಟಾಗಲೆಂದು
ಉಂಟಾಗುತ್ತದೆ
ಉಂಟು-ಮಾಡಲು
ಉಂಟು-ಮಾಡುತ್ತಾನೆ
ಉಂಟು-ಮಾಡುವ-ವನು
ಉಕಂ
ಉಕ-ರೀತ್ಯಾ
ಉಕ-ರೀತ್ಯಾ-ಮೇಲೆ
ಉಕ್ಕ
ಉಕ್ಕಂ
ಉಕ್ತ
ಉಕ್ತಂ
ಉಕ್ತಂಚ
ಉಕ್ತ-ರೀತ್ಯಾ
ಉಕ್ತ-ರೀತ್ಯಾ-ಆಗಲೇ
ಉಕ್ತ-ವಾದ
ಉಕ್ತಾ-ನಾರಾ
ಉಕ್ತ್ವಾ
ಉಗುರಿ-ನಿಂದ
ಉಚ್ಚ-ರಿಸಿ
ಉಚ್ಯತೇ
ಉಚ್ಯತೇ-ನಾರ
ಉಚ್ಯತೇ-ಹೇಳಲ್ಪಡುತ್ತಾನೆ
ಉಜ್ವಲಕಾಂತಿ-ಯಿಂದ
ಉತ್ಕೃಮಣೇ
ಉತ್ಕ್ರಮಣೇ
ಉತ್ಕ್ರಾಂತೋ
ಉತ್ತಮ
ಉತ್ತಮಂ
ಉತ್ತಮತ್ವ
ಉತ್ತಮತ್ವ-ವನ್ನು
ಉತ್ತಮತ್ವವು
ಉತ್ತಮ-ರಾ-ದ-ವರು
ಉತ್ತರ
ಉತ್ತರತ್ರ
ಉತ್ತರತ್ರ-ಮುಂದೆ
ಉತ್ತರತ್ರಾಪಿ
ಉತ್ತರತ್ರಾಪಿ-ಇನ್ನು
ಉತ್ತರ-ಪೂರ್ವಾಘಯೋಃ
ಉತ್ತರ-ಪೂರ್ವಾಘಯೋರಶ್ಲೇಷ-ವಿನಾಶೌ
ಉತ್ತರೀತ್ಯಾ
ಉತ್ತಿಷ್ಠಂತೇಪಿ
ಉತ್ಥತಿಃ
ಉತ್ಥಾನಂ
ಉತ್ಪನ್ನ-ವಾದ
ಉತ್ಪನ್ನಾನಾಂ
ಉತ್ಪನ್ನಾನಾಂಅ-ವತಾರ
ಉತ್ಪಾದ-ಕನು
ಉತ್ಪಾದಿಸಿ
ಉತ್ಪಾದಿಸುವ-ವನು
ಉತ್ಸತ್ತಿ-ರಹಿತ
ಉದಕಂ
ಉದಕ-ದಲ್ಲಿ
ಉದಕೇ
ಉದಕೇಆ
ಉದರಂ
ಉದರಃಈ
ಉದರ-ದಲ್ಲಿ
ಉದರ-ನಾಮಕ
ಉದರಮುಪಾಸತೇ
ಉದರ-ವನ್ನು
ಉದರೇ-ನಮ್ಮ
ಉದ-ಹರಿ-ಸಿ-ರುತ್ತಾರೆ
ಉದಾ
ಉದಾ-ಹರಣೆ-ಯಂತೆ
ಉದಾ-ಹರಣೆ-ಯನ್ನು
ಉದಾ-ಹೃತ
ಉದೀರ್ಣಂ
ಉದ್ದಿಶ್ಯ
ಉದ್ದೇಶ
ಉದ್ದೇಶ-ದಿಂದ
ಉದ್ದೇಶಿಸಿ
ಉದ್ಯುಕ್ತ-ವಾಗಿ-ರುವ
ಉದ್ಯುಕ್ತ-ವಾಗಿವೆ
ಉದ್ಯೋತಿದ-ವಾದ
ಉದ್ಯೋಷಿ-ಸಲು
ಉನ್ನತ-ವಾದುವು
ಉಪ-ಕಾರ-ಕ-ತಯಾ
ಉಪ-ಕಾರ-ತಯಾ
ಉಪ-ಕಾರ-ತಯಾ-ಉಪ-ಕಾರ
ಉಪ-ಕಾರ-ಮಾಡುವುದರಿಂದ
ಉಪ-ಕಾರಿತ್ವೇನಉಪ-ಕಾರಿ-ಯಾಗಿ-ರುವು-ದ-ರಿಂದ
ಉಪ-ಜೀವಕ
ಉಪ-ಜೀವ್ಯ
ಉಪ-ದೇಶ
ಉಪ-ದೇಶಕ
ಉಪ-ದೇಶ-ಕ-ನಾದ
ಉಪ-ದೇಶಕ್ಕಾಗಿ
ಉಪ-ದೇಶ-ಗಳ
ಉಪ-ದೇಶದ
ಉಪ-ದೇಶ-ದಿಂದಲೇ
ಉಪ-ದೇಶ-ಮಾಡಲು
ಉಪ-ದೇಶ-ಮಾಡಿ
ಉಪ-ದೇಶ-ಮಾಡಿ-ರು-ವುದು
ಉಪ-ದೇಶ-ವನ್ನು
ಉಪ-ದೇಶವು
ಉಪದೇ-ಷೃ-ತಯಾ
ಉಪದೇ-ಷೃ-ತಯಾ-ಉಪ-ದೇಶ-ನಿಮಿತ್ತ-ಕಾರ-ಣ-ದಿಂದ
ಉಪದ್ರವಸಂಕಟ-ಗಳ
ಉಪ-ಪಾದಿಸುತ್ತಾರೆ
ಉಪಯೋಗಮಾಡಲ್ಪಡುತ್ತದೆ
ಉಪಯೋಗಿಸ-ಬೇಕು
ಉಪಯೋಗಿಸ-ಲಾಗಿದೆ
ಉಪಯೋಗಿ-ಸಲ್ಪಟ್ಟು
ಉಪಯೋಗಿಸಿ
ಉಪಯೋಗಿಸಿ-ದರೆ
ಉಪಯೋಗಿಸುವು-ದ-ರಿಂದ
ಉಪಲಕ್ಷಣ-ದಿಂದ
ಉಪಲಕ್ಷಿತ-ವಾಗಿದೆ
ಉಪ-ಸಾಧಕನು
ಉಪ-ಸಾಧಕೋ
ಉಪಸೇ-ವತೇ-ಅನುಭವಿ-ಸುತ್ತಾನೆ
ಉಪಸ್ಥಿತನಿದ್ದು
ಉಪಾ-ಗತಃ
ಉಪಾ-ಗತಃಹೊಂದಿತು
ಉಪಾಸತೇ
ಉಪಾಸತೇ-ಉಪಾಸನೆ
ಉಪಾಸನೆ
ಉಪ್ತನ್ನ-ವಾದ
ಉಭ-ಯವೂ
ಉಭಯೋಃಜ್ಞಾನೋತ್ತರಪೂರ್ವ-ಕಾಲೀನ-ಗ-ಳಾದ
ಉರ್ಧ್ವಾಂಡಕಟಾಹ-ಮೇಲಿನ
ಉಲ್ಲೇಖಿಸಿ-ರುವ
ಉಳಿದ
ಉಳ್ಳ
ಉಳ್ಳದ್ದು
ಉಳ್ಳದ್ದೇ
ಉಳ್ಳ-ವ-ನಾದ
ಉಳ್ಳ-ವ-ನಾದ-ನುಈ
ಉಳ್ಳ-ವ-ನಾದು-ದ-ರಿಂದ
ಉಳ್ಳ-ವನು
ಉಳ್ಳ-ವನುಈ
ಉಳ್ಳ-ವರಾ-ಗಿದ್ದರು
ಉಳ್ಳ-ವ-ರಿಂದ
ಉಳ್ಳ-ವ-ರಿಗೆ
ಉಳ್ಳ-ವಳೇ
ಉಳ್ಳ-ವು-ಗಳು
ಊಟ-ಮಾಡಿ-ದರೆ
ಊಟ-ಮಾಡುತ್ತಾನೆ
ಊಹಿಸಲೂ
ಋ
ಋಕಾರಕ್ಕೆ
ಋಕಾರ-ವಾದರೋ
ಋಕಾರವು
ಋಜುಗ-ಣಕ್ಕೆ
ಋಜು-ಮಾರ್ಗ-ದಲ್ಲಿ
ಋಭು-ಗಳು
ಋವರ್ಣಸ್ಯ
ಋಷಿ-ಗಳು
ಋಷಿಯಲ್ಲ
ಎಂಟು
ಎಂಟೂ
ಎಂದರೆ
ಎಂದರ್ಥ
ಎಂದಲ್ಲ
ಎಂದಾಗ-ಬೇಕಿತ್ತು
ಎಂದಾಗ-ಬೇಕೇ
ಎಂದಾಗಿದೆ
ಎಂದಾಗುತ್ತದೆ
ಎಂದಾಗು-ವುದಿಲ್ಲ
ಎಂದಾಯಿತು
ಎಂದಿಗೂ
ಎಂದಿದೆ
ಎಂದಿ-ರುವು-ದ-ರಿಂದ
ಎಂದು
ಎಂದೂ
ಎಂದೆಂದಿಗೂ
ಎಂದೆನಿಸಿಕೊಳ್ಳುತ್ತದೆ
ಎಂದೆನಿಸಿಕೊಳ್ಳುತ್ತವೆ
ಎಂದೆನಿ-ಸುತ್ತದೆ
ಎಂದೆನಿ-ಸುತ್ತವೆ
ಎಂದೆನಿಸುವ
ಎಂದೆನಿಸು-ವನು
ಎಂದೆನಿಸುವ-ನು-ಅಂದರೆ
ಎಂದೆನಿಸು-ವರು
ಎಂದೆನಿ-ಸು-ವುದು
ಎಂದೇ
ಎಂಬ
ಎಂಬಂತೆ
ಎಂಬರ್ಥ
ಎಂಬರ್ಥದ
ಎಂಬರ್ಥ-ದಲ್ಲಿ
ಎಂಬರ್ಥ-ದಿಂದ
ಎಂಬರ್ಥ-ವನ್ನೂ
ಎಂಬರ್ಥ-ವನ್ನೇ
ಎಂಬರ್ಥ-ವಲ್ಲ
ಎಂಬರ್ಥ-ವಾಗುತ್ತದೆ
ಎಂಬರ್ಥ-ವಿದೆ
ಎಂಬರ್ಥ-ವಿ-ರುವು-ದ-ರಿಂದಲೂ
ಎಂಬರ್ಥ-ವುಳ್ಳದ್ದು
ಎಂಬರ್ಥ-ವುಳ್ಳದ್ದೂ
ಎಂಬರ್ಥವೂ
ಎಂಬಲ್ಲಿ
ಎಂಬಲ್ಲಿನ
ಎಂಬುದಕ್ಕೆ
ಎಂಬುದನ್ನು
ಎಂಬು-ದಾಗಿ
ಎಂಬು-ದಾಗಿಯೂ
ಎಂಬು-ದಾಗಿಯೇ
ಎಂಬುದು
ಎಂಬುದೇ
ಎಂಬುವುದೇ
ಎಡಗಾಲಿನ
ಎಣ್ಣೆ-ಗಳ
ಎತ್ತಲು
ಎತ್ತಿನ
ಎತ್ತುವ-ವರೂ
ಎತ್ತೇ
ಎದುರಿಗೇ
ಎದ್ದು
ಎನ್ನುವ
ಎನ್ನುವುದಕ್ಕಿಂತ
ಎರಡನೇ
ಎರಡು
ಎರಡೂ
ಎಲೆ
ಎಲೆಯ
ಎಲೆ-ಯಲ್ಲಿ
ಎಲ್ಲ
ಎಲ್ಲಕ್ಕಿಂತಲೂ
ಎಲ್ಲ-ದ-ರಿಂದ
ಎಲ್ಲ-ರಿಂದ
ಎಲ್ಲರೂ
ಎಲ್ಲ-ವನ್ನೂ
ಎಲ್ಲವೂ
ಎಲ್ಲಾ
ಎಲ್ಲೆ-ಯನ್ನು
ಎಷ್ಟರಮಟ್ಟಿಗೆ
ಎಷ್ಟು
ಎಸೆಯಲ್ಪಟ್ಟಿ-ರುವು-ದ-ರಿಂದಲೂ
ಏಕ
ಏಕಃ
ಏಕಃಸರ್ವಶ್ರೇಷ್ಠ-ನಾದ
ಏಕ-ದೇಶ-ಈಗ
ಏಕಪ್ರ-ಕಾರ-ವಾದ
ಏಕ-ಮಾಡಿ
ಏಕಾದಶೀ
ಏಕಾರ್ಥತ್ಯಾತ್ಒಂದೇ
ಏಕಾರ್ಥತ್ವ
ಏಕಾರ್ಥತ್ವಾತ್
ಏಕೀಭವನಾಧಿ-ಕರಣಂ
ಏಕೆಂದರೆ
ಏಕೇ-ಕೆಲ-ವರು
ಏಕೇಷಾಂಮತ್ತೆ
ಏಕೋ
ಏತಂ
ಏತತ್
ಏತದಾ-ಸೀತ್
ಏತದಿತ್ಯಾಹುರಥ
ಏತನ್ನಾನಾ-ವತಾರಾಣಾಂ
ಏತಸ್ಮಾಜ್ಜಾ-ಯತೇ
ಏತಸ್ಯಈ
ಏತೇಖಿಲಾ
ಏತೇನ-ಅ-ವ-ರಿಂದ
ಏತೇನೈವ
ಏತೇಽ-ಶುಭ-ಕೃತಃ
ಏನ
ಏನಾಗಬೇಕಾಗಿದೆ
ಏನು
ಏನೂ
ಏನೇತಿ
ಏಳು
ಏಳುತ್ತದೆ
ಏಳು-ವುದಿಲ್ಲ
ಏವ
ಏವಂ
ಏವಂಆ
ಏವಂವಿದಿ-ಅ-ಪರೋಕ್ಷ
ಏವಅ
ಏವ-ಅಂತಹ
ಏವಆ
ಏವ-ಆದು-ದ-ರಿಂದಲೇ
ಏವ-ಇಂದ್ರಿಯ-ಗಳಲ್ಲಿದ್ದು
ಏವ-ಒಬ್ಬನೇ
ಏವ-ಜಗತ್ತಿಗೆ
ಏವಣ
ಏವ-ತಮ್ಮ
ಏವ-ನೀರೇ
ಏವ-ಫಲ-ವನ್ನು
ಏವ-ಮುಖ್ಯ-ನಾದ
ಏವಮೇವ
ಏವಶ್ವೇತದ್ವೀಪ
ಏವ-ಹಾಗೆಯೇ
ಏವಾಥ
ಏವಾನುವಿಷ್ಟಃ
ಏವಾ-ಯನಂ
ಏವಾರ್ಥಃಎಂಬರ್ಥ
ಏವಾರ್ಥಃಯಾಪನಂ
ಏವೈಕೋ
ಐಕಾರ್ಥ್ಯಾದ್ಗತ್ಯ-ಭಾವ-ಯೋಃಗತಿ
ಐಕಾರ್ಥ್ಯಾದ್ಗತ್ವ-ಭಾವಯೋಃ
ಐಕ್ಯ-ವಾಗುವ
ಐಕ್ಯ-ವಾಗುವುದಕ್ಕೆ
ಐಕ್ಯ-ವಾಗುವು-ದ-ರಿಂದ
ಐತರೇಯ
ಐತರೇಯ-ಭಾಷ್ಯ-ದಲ್ಲಿ
ಐತರೇಯ-ಭಾಷ್ಯೇ-ಐತರೇಯ
ಐತರೇಯ-ಭಾಷ್ಯೋಕ್ತ-ರೀತ್ಯಾ
ಐತಿಹ್ಯ-ವಿದೆ
ಐದು
ಐಹಿಕ
ಒಂದಂಶ-ವನ್ನು
ಒಂದಕ್ಕೊಂದು
ಒಂದು
ಒಂದು-ಗೂಡಿಸಿ
ಒಂದೇ
ಒಂದೊಂದು
ಒಂದೊಂದೂ
ಒಂಭತ್ತು
ಒಟ್ಟಿ-ನಲ್ಲಿ
ಒಡನಾಡಿ-ಗಳೂ
ಒಡೆಯ-ನಾದು-ದ-ರಿಂದ
ಒದಗಿ-ಸಲು
ಒದಗಿಸುತ್ತಾ-ನೆಂಬ
ಒದಗಿಸುವ
ಒದಗಿಸುವ-ವನು
ಒಬ್ಬ
ಒಬ್ಬನೇ
ಒಬ್ಬರೇ
ಒಯ್ದುದು
ಒಳಗಿರು-ವನು
ಒಳಗೂ
ಒಳಗೆ
ಒಳ-ಹೊರ-ಗೆ-ಯವ
ಓಂ
ಓದುಗರ
ಓದುಗ-ರಿಂದ
ಓದುಗ-ರಿಗೆ
ಕಂಠ-ದಿಂದ
ಕಂಡಂತೆ
ಕಂಡು
ಕಚ್ಚಲ್ಪಟ್ಟ
ಕಡಗ
ಕಡಿಮೆಯಾಗುತ್ತದೆಯೋ
ಕಡೆ-ಯ-ದಾದ
ಕಡೆ-ಯಲ್ಲಿ
ಕಡೆ-ಯಲ್ಲಿಯೂ
ಕಣ-ಗಳಿಂದ
ಕಣ್ಣನ್ನೂ
ಕಣ್ಣು-ಗಳಿಂದಲೂ
ಕಣ್ಣು-ಗಳು
ಕಥಂ
ಕಥಂಚನ
ಕಥಿತಾಃ
ಕದಾಪಿ
ಕನ್ನಡ-ದಲ್ಲಿ
ಕನ್ಯಾಯಾ
ಕನ್ಯೆ-ಯನ್ನು
ಕಪಟ
ಕಪಟಿ-ಗಳಿಂದ
ಕಪಿಧ್ವಜದ
ಕಪಿಲಂ
ಕಪಿಲ-ರೂಪಿ
ಕಪ್ಪು-ಬಟ್ಟೆ
ಕಮಲ-ದಲ್ಲಿ
ಕಮಲ-ಪುಷ್ಪ
ಕಮಲ-ಪುಷ್ಪದ
ಕಮಲ-ವೆಂಬ
ಕಮಲಾಪತೌ
ಕರ-ಕಮಲ-ಸಂಜಾತ-ರಾದ
ಕರಣ
ಕರ-ಸಂಸ್ಪರ್ಶಂ
ಕರ-ಸಂಸ್ಪರ್ಶಂಹಸ್ತಲಾಘವ-ವನ್ನು
ಕರಾರ್ಚಿತ-ವಾದು-ದ-ರಿಂದ
ಕರುಣಾ-ಕರಃ
ಕರುಷಣ
ಕರೆದು-ಕೊಂಡು
ಕರೆದುಕೊಳ್ಳುವ
ಕರೆಯಲ್ಪಡುತ್ತದೆ
ಕರೆಯಲ್ಪಡುತ್ತಾನೆ
ಕರೆಯಲ್ಪಡುತ್ತಾರೆ
ಕರೆಯಲ್ಪಡುತ್ತಾಳೆ
ಕರೆಯಲ್ಪಡುವ
ಕರೆಯಲ್ಪಡು-ವರು
ಕರೆಯಲ್ಪಡುವಳು
ಕರೆಸಿ-ಕೊಂಡಿದೆ
ಕರೆಸಿಕೊಳ್ಳಲಡುತವೆ
ಕರೆಸಿಕೊಳ್ಳಲಡುತ್ತದೆ
ಕರೆಸಿಕೊಳ್ಳಲಡುವ-ವನು
ಕರೆಸಿಕೊಳ್ಳಲ್ಪಡುತ್ತದೆ
ಕರೆ-ಸಿಕೊಳ್ಳಲ್ಪಡುತ್ತಾನೆ
ಕರೆಸಿಕೊಳ್ಳಲ್ಪಡುವ
ಕರೆ-ಸಿಕೊಳ್ಳುತ್ತಾನೆ
ಕರೆ-ಸಿಕೊಳ್ಳುವ
ಕರೆಸು-ವನು
ಕರೋತಿ
ಕರೋತ್ಯಯಂ
ಕರ್ತೃತ್ವಾತ್
ಕರ್ತೃ-ವಾದ
ಕರ್ಮ
ಕರ್ಮ-ಅ-ಪರೋಕ್ಷ
ಕರ್ಮ-ಕರ್ಮ-ಗಳು
ಕರ್ಮ-ಕೃತಾಂ
ಕರ್ಮಕ್ಕೆ
ಕರ್ಮಕ್ಷಯ-ವಾಗುವುದು
ಕರ್ಮಕ್ಷಯಾನ್ಮುಕ್ತಿಃ
ಕರ್ಮ-ಗಳ
ಕರ್ಮ-ಗಳನ್ನು
ಕರ್ಮ-ಗಳನ್ನೇ
ಕರ್ಮ-ಗಳಿಂದ
ಕರ್ಮ-ಗಳಿಗನು-ಸಾರ-ವಾಗಿ
ಕರ್ಮ-ಗಳು
ಕರ್ಮಜ
ಕರ್ಮಜಂ
ಕರ್ಮ-ಜನ್ಯ-ವಾದ
ಕರ್ಮಣಾ
ಕರ್ಮಣಾಂ
ಕರ್ಮ-ಣಾಂಕರ್ಮ-ಗಳ
ಕರ್ಮಣಿ
ಕರ್ಮ-ದಿಂದ
ಕರ್ಮ-ದಿಂದಲೂ
ಕರ್ಮ-ನಾಶ
ಕರ್ಮ-ನಾಶ-ಮಾಡಿ-ಕೊಳ್ಳಲು
ಕರ್ಮ-ಪದ-ಗಳ
ಕರ್ಮ-ಪದ-ಗಳಿವೆ
ಕರ್ಮ-ಪದ-ಗಳು
ಕರ್ಮ-ಪದ-ಗಳೂ
ಕರ್ಮ-ಪೂರ್ತಿಗೂ
ಕರ್ಮ-ಮಾಡಿದ
ಕರ್ಮ-ರಾಶಿ-ಯನ್ನು
ಕರ್ಮ-ರಾಶಿ-ಯೊಳ-ಗಿನ
ಕರ್ಮ-ವನ್ನು
ಕರ್ಮ-ವನ್ನೂ
ಕರ್ಮ-ವಾಗುತ್ತವೆ
ಕರ್ಮವೇ
ಕರ್ಮ-ಸಂಘಾತಂ
ಕರ್ಮ-ಸಂಘಾತಂಸಂಚಿತ
ಕರ್ಮಸ್ವ-ರೂಪ
ಕರ್ಮಸ್ವ-ರೂಪ-ತಾರತಮ್ಯೇನೇತಿ
ಕರ್ಮಸ್ವ-ರೂಪದ
ಕರ್ಮಾಚರಣೆಗೆ
ಕರ್ಮಾಣಿ
ಕರ್ಮಾಣಿ-ಅನಂತ-ವಾದ
ಕರ್ಮಾಣ್ಯನಂತಾನಿ
ಕರ್ಮಾಣ್ಯನೇಕಶಃ
ಕರ್ಮಾನು-ಸಾರ-ವಾಗಿ
ಕರ್ಮಾಪೇಕ್ಷಯಾ
ಕರ್ಹಿಚಿತ್
ಕಲಾಕ್ಷಯೇನ
ಕಲಾದೀನ್
ಕಲಿ
ಕಲಿ-ಕಾಲ-ಕಲುಷಿ-ತಾಂತಃಕರ-ಣಾನಾಂ
ಕಲಿ-ಗಾಲದ
ಕಲಿ-ಯನ್ನು
ಕಲಿಯು
ಕಲಿ-ಯು-ಗ-ದಲ್ಲಿ
ಕಲಿಯೇ
ಕಲಿ-ರೇಕಲಃ
ಕಲ್ಯಾದ-ಯೋಽಖಿಲಾಃ
ಕಲ್ಯಾದಿ
ಕಲ್ಯಾದೀನ್
ಕಳೆದು-ಕೊಂಡ
ಕಳೆದು-ಕೊಂಡ-ವರು
ಕಳೆದು-ಕೊಂಡು
ಕಳೆದುಕೊಳ್ಳಬೇಕಲ್ಲದೆ
ಕಳೆದು-ಕೊಳ್ಳಲು
ಕಳೆದುಕೊಳ್ಳು-ವುದು
ಕಳೆಯು
ಕಳ್ಳರು
ಕಷ್ಟ-ಗಳ
ಕಸ್ಮಿನ್ನ್ವಹಮುತ್ಕ್ರಾಂತೋ
ಕಸ್ಮಿನ್ವಾಪ್ರತಿಷ್ಠಿತೇ
ಕಸ್ಯಚಿನ್ನ
ಕಾಣದೆ
ಕಾಣದೇ
ಕಾಣ-ಬಹುದು
ಕಾಣಲಾರರು
ಕಾಣಿ-ಸಿಕೊಳ್ಳುವುದು
ಕಾಮಂ
ಕಾಮಮಸೌ
ಕಾಮ-ರೂಪತ್ವ-ಇಚ್ಛೆ-ಯಿಂದ
ಕಾಮ-ರೂಪತ್ವ-ಮುಕ್ತ-ಮಿತ್ಯವ-ಗಂತವ್ಯಂಇತಿ
ಕಾಮ-ರೂಪೀತಿ
ಕಾಮ-ರೂಪೀ-ತಿ-ಕಾಮುಕ-ನಾ-ದ-ವನು
ಕಾಮಾನ್ನೀ
ಕಾಮ್ಯ
ಕಾರ
ಕಾರ-ಕ-ನಾದು-ದ-ರಿಂದಲೂ
ಕಾರಕ್ಕೂ
ಕಾರಕ್ಕೆ
ಕಾರಣ
ಕಾರಣಂ
ಕಾರ-ಣಕ್ಕೆ
ಕಾರ-ಣ-ಗಳಿಂದ
ಕಾರ-ಣ-ದಿಂದ
ಕಾರ-ಣ-ದಿಂದ-ಲಾಗಲೀ
ಕಾರ-ಣ-ದಿಂದಲೂ
ಕಾರ-ಣ-ದಿಂದಲೇ
ಕಾರ-ಣ-ನಾಗಿ
ಕಾರ-ಣ-ನಾಗಿ-ರುವು-ದ-ರಿಂದ-ನಾರಾ-ಯಣ
ಕಾರ-ಣ-ನಾಗಿ-ರು-ವುದು
ಕಾರ-ಣ-ವನ್ನು
ಕಾರ-ಣ-ವಾಗಿ
ಕಾರ-ಣ-ವಾಗುತ್ತದೆ
ಕಾರ-ಣ-ವಾಗುತ್ತವೆ
ಕಾರ-ಣ-ವಾಗುವ
ಕಾರ-ಣ-ವಾದ
ಕಾರ-ಣಾನಿ
ಕಾರದ
ಕಾರ-ದಿಂದ
ಕಾರಪ್ರಕೃತಿ
ಕಾರ-ಬೀಜಃಅ
ಕಾರ-ಯತಿ
ಕಾರ-ಯೋಃನ
ಕಾರ-ರೇಫೋತ್ತರಯೋರ-ಕಾರ-ಯೋರರ್ಥ
ಕಾರ-ಲೋಪಃಈ
ಕಾರ-ವನ್ನು
ಕಾರ-ವಾಗಿದೆ
ಕಾರ-ವಾಯಿತು
ಕಾರವು
ಕಾರವೂ
ಕಾರವೇ
ಕಾರಸ್ವಾರ್ಥ-ಕಡೆ-ಯಲ್ಲಿ-ರುವ
ಕಾರಾರ್ಥವೇ
ಕಾರಾರ್ಥಸು
ಕಾರಾರ್ಥಸ್ತು
ಕಾರಾರ್ಥೋ
ಕಾರೇ
ಕಾರ್ಯ
ಕಾರ್ಯ-ಗಳನ್ನು
ಕಾರ್ಯ-ಗಳನ್ನೂ
ಕಾರ್ಯ-ಗಳಿಂದ
ಕಾರ್ಯ-ಗಳು
ಕಾರ್ಯದ
ಕಾರ್ಯ-ದರ್ಶಿ-ಯವರ
ಕಾರ್ಯಪ್ರಚೋದ-ಕನೂ
ಕಾರ್ಯಪ್ರವೃತ್ತ-ವಾದ
ಕಾರ್ಯ-ವನ್ನು
ಕಾರ್ಯ-ವನ್ನೂ
ಕಾರ್ಯ-ಶೂನ್ಯರಲ್ಲ-ವಾದು-ದ-ರಿಂದ
ಕಾರ್ಯಾಂತೇ
ಕಾಲ
ಕಾಲಂ
ಕಾಲಕ್ಕೂ
ಕಾಲ-ಗಳ
ಕಾಲತಃ
ಕಾಲ-ದಲ್ಲಿ
ಕಾಲ-ದಲ್ಲಿಯೂ
ಕಾಲ-ದಲ್ಲಿಯೇ
ಕಾಲ-ದಿಂದ
ಕಾಲ-ನೇಮಿ
ಕಾಲ-ವನ್ನು
ಕಾಲವೂ
ಕಾಲಾ-ವರ್ಣಾಂಡ
ಕಾಲೇ
ಕಾಲೇ-ಎಲ್ಲ
ಕಾಲೇನ
ಕಾವೇರೀ
ಕಾವೇರೀಶ್ರೀ-ರಂಗ-ದಲ್ಲಿ-ರುವ
ಕಾಶ
ಕಿಂಚಿತ್
ಕಿಂತು
ಕಿತವೈಃಬೇರೆ
ಕಿತವೈಸ್ತು
ಕಿಮತ್ರ
ಕಿಮಲಭ್ಯಂ
ಕಿಮು
ಕಿಮ್
ಕಿರೀಟ
ಕಿರೀಟ-ಕಟಕಾದ್ಯಾ
ಕಿವಿ-ಗಳು
ಕೀರ್ತನಂ
ಕೀರ್ತಿ
ಕುಂಠಿತವಾಗದ
ಕುಟಿಲಾತೃನಾಂ
ಕುಟಿಲಾತ್ಮನಾಂ
ಕುಟುಂಬ
ಕುತ
ಕುಯೋಗಗಾಮಿನಃ
ಕುರಿತದ್ದೇ
ಕುರಿತು
ಕುರುತೇಬೂದಿ-ಯಂತೆ
ಕುರ್ವಂತಿ
ಕುರ್ವಂತಿ-ಭಕ್ತಿ-ಯಿಂದ
ಕುರ್ವನ್
ಕುರ್ವನ್ನೇವಂ
ಕುಲ
ಕುಳಿತು-ಕೊಂಡು
ಕೂಡ
ಕೂಡಿ
ಕೂಡಿದ
ಕೂಡಿ-ದ-ವ-ರಿಂದ
ಕೂಡಿ-ದ-ವರು
ಕೂಡಿದ್ದು
ಕೂಡಿ-ರುವ
ಕೂಡಿ-ರುವು-ದಾಗಿ
ಕೂರ್ಮಃ
ಕೂರ್ಮಃಕೂರ್ಮ-ರೂಪ-ದಿಂದಿ-ರುವ
ಕೂರ್ಮ-ಪುರಾಣ
ಕೂರ್ಮ-ರೂಪದ
ಕೂರ್ಮ-ರೂಪ-ದಲ್ಲಿ-ರುವ
ಕೂರ್ಮ-ರೂಪ-ದಿಂದ
ಕೂರ್ಮ-ರೂಪ-ವನ್ನು
ಕೂರ್ಮ-ರೂಪಸ್ಯ
ಕೂರ್ಮ-ರೂಪೇಣ
ಕೂರ್ಮ-ರೂಪೋ
ಕೂರ್ಮಸ್ತ್ವಂಡೇ
ಕೂರ್ಮೊ
ಕೃತಃ
ಕೃತಃರಚಿಸಲ್ಪಟ್ಟ
ಕೃತಜ್ಞ-ತೆ-ಗಳನ್ನು
ಕೃತಸ್ನಾನಾ
ಕೃತಾ
ಕೃತಾರ್ಥರು
ಕೃತಿ-ಗಳಲ್ಲಿ
ಕೃತ್ಯ-ಗಳನ್ನು
ಕೃತ್ಯ-ಗಳಿಂದ
ಕೃತ್ಯ-ಗಳು
ಕೃತ್ಯಲುಟೋ
ಕೃತ್ಯ-ವನ್ನು
ಕೃಪಯಾ
ಕೃಪೆ-ಮಾಡಿ
ಕೃಪೆ-ಯಿಂದ
ಕೃಷ್ಣ
ಕೃಷ್ಣನ
ಕೃಷ್ಣ-ನಿಗೆ
ಕೃಷ್ಣನೇ
ಕೃಷ್ಣ-ಪಕ್ಷ-ದಲ್ಲಿ
ಕೃಷ್ಣ-ಪಕ್ಷೇ
ಕೃಷ್ಣ-ರೂಪ-ದಿಂದ
ಕೃಷ್ಣ-ರೂಪೀ
ಕೃಷ್ಣ-ರೂಪೀ-ಕೃಷ್ಣನ
ಕೃಷ್ಣ-ರೂಪೀ-ಕೃಷ್ಣ-ರೂಪ-ದಿಂದ
ಕೃಷ್ಣ-ರೂಪೀ-ಕೃಷ್ಣ-ರೂಪ-ದಿಂದಿ-ರುವ
ಕೃಷ್ಣಾದಿ
ಕೃಷ್ಣಾಮೃತ-ಮಹಾರ್ಣವ
ಕೆಡಿಸಿ-ದ-ವನು
ಕೆಲ-ವಕ್ಕೆ
ಕೆಲ-ವರ
ಕೆಲ-ವ-ರಿಗೆ
ಕೆಲ-ವರು
ಕೆಲವು
ಕೆಲಸ
ಕೆಳ-ಗಿನ
ಕೆಳಗೆ
ಕೇಳಿ
ಕೇಳಿರಿ
ಕೇವಲ
ಕೇವಲೇನ
ಕೇಶವಂ
ಕೇಶವಃ
ಕೇಶವಃಶ್ರೀ-ಹರಿಯು
ಕೈಗೂಡುವಂತೆ
ಕೈಯನ್ನು
ಕೈಯಲ್ಲಿ
ಕೈಯಿಟ್ಟು
ಕೊಟ್ಟ
ಕೊಟ್ಟರೆ
ಕೊಟ್ಟಿದ್ದರು
ಕೊಟ್ಟಿ-ರುತ್ತಾನೆ
ಕೊಟ್ಟಿ-ರುತ್ತಾರೆ
ಕೊಟ್ಟು
ಕೊಟ್ಟು-ಬಿಟ್ಟ-ರೆಂದು
ಕೊಡ-ದಿ-ರುವ-ವನು
ಕೊಡ-ಬಲ್ಲರು
ಕೊಡಬೇಕೆಂಬ
ಕೊಡಲು
ಕೊಡಲ್ಪಟ್ಟ
ಕೊಡುತ್ತವೆ
ಕೊಡುತ್ತಾನೆ
ಕೊಡುತ್ತಿ-ರುವು-ದ-ರಿಂದ
ಕೊಡುವ
ಕೊಡುವಲ್ಲಿ
ಕೊಡುವ-ವ-ನಾದು-ದ-ರಿಂದ
ಕೊಡುವ-ವನು
ಕೊಡುವ-ವನೂ
ಕೊಡುವುದರ
ಕೊಡು-ವುದಿಲ್ಲ
ಕೊನೆ-ಯಲ್ಲಿ
ಕೋಶಾತ್
ಕೋಽಪಿ
ಕೌಂಠರವ್ಯ
ಕೌಂಠರವ್ಯಶ್ರುತಿ
ಕೌಂತೇಯ
ಕೌಷಾರವ
ಕ್ಕೆ
ಕ್ರಮ-ವನ್ನು
ಕ್ರಮ-ವಾಗಿ
ಕ್ರಮಾತ್
ಕ್ರಮಾದ್ವೃದ್ದಿರುದೀರಿತಾ
ಕ್ರಮೇಣ
ಕ್ರಿಯತೇ
ಕ್ರಿಯಮಾ-ಣಾನಿ
ಕ್ರಿಯಾ-ಗಳು
ಕ್ರಿಯಾಪದ
ಕ್ರಿಯಾಪದಕ್ಕೆ
ಕ್ರಿಯಾಪದದ
ಕ್ರಿಯಾಪದ-ವಿ-ರುವಾಗ
ಕ್ರಿಯೆ-ಯನ್ನು
ಕ್ರೀಡತೇ
ಕ್ರೀಡ-ತೇತಿ
ಕ್ರೀಡ-ತೇ-ರಮಿ-ಸುತ್ತಾನೆ
ಕ್ರೀಡಾಂಧಾ
ಕ್ರೀಡಾಂಧಾಃಆಟಪಾಟ-ಗಳಲ್ಲಿಯೇ
ಕ್ರೀಡಾದಿ-ಗಳನ್ನು
ಕ್ರೀಡಾದಿ-ಗಳು
ಕ್ರೀಡಾದಿಗಾಗಿ
ಕ್ರೀಡಾದಿ-ಗುಣ-ಗಳಿಂದ
ಕ್ರೀಡಾಮಿ
ಕ್ರೀಡಿ-ಸುತ್ತವೆಯೋ
ಕ್ರೀಡಿ-ಸುತ್ತಾನೆ
ಕ್ರೀಡಿ-ಸುತ್ತಾನೆ-ಹೀಗೆಂದು
ಕ್ರೀಡೆಯು
ಕ್ರೌರ್ಯ
ಕ್ವಚಿತ್
ಕ್ವಚಿತ್ಸ್ವಲ್ಪವೂ
ಕ್ವಚಿದ್ದ-ದಾತಿ
ಕ್ವಚಿನ್ನೋ
ಕ್ಷಣ
ಕ್ಷಣ-ದಲ್ಲಿಯೇ
ಕ್ಷಮಿಸು
ಕ್ಷಯಂ
ಕ್ಷಯ-ರಹಿ-ತತ್ವಾತ್ನಾಶ-ವಿಲ್ಲ-ದ-ವ-ರಾದು-ದ-ರಿಂದ
ಕ್ಷಯೇ
ಕ್ಷಯೋ
ಕ್ಷೀಣತೆಯೂ
ಕ್ಷೀಯಂತೇ
ಕ್ಷೀಯಂತೇ-ಅಂದರೆ
ಕ್ಷೀಯಂತೇ-ನಾಶ-ರಹಿತ-ರಾ-ದ-ವರು
ಕ್ಷೀಯತೆ-ನಾಶ-ಹೊಂದು-ವುದಿಲ್ಲ-ವೆಂಬ
ಕ್ಷೀಯತೇ
ಕ್ಷೇತ್ರಮೂರ್ತಿ-ಗಳು
ಕ್ಷೇಪ್ತುಂ
ಖರ್ಪರ-ದಲ್ಲಿ
ಗಂಗಾ
ಗಂಗಾದಿ
ಗಂಗೆ
ಗಂಗೆಯೇ
ಗಂಗೈವಾಪರ-ನಾಮಿಕಾ
ಗಂಡಸುತನ
ಗಂತಾ
ಗಂಧವೂ
ಗಗನಂ
ಗಚ್ಛಂತಿ
ಗಜೇಂದ್ರನ
ಗಣನೆಗೆ
ಗತಃ
ಗತಃಣ
ಗತಾಂ
ಗತಾವಿತಿ
ಗತಿ
ಗತಿಂ
ಗತಿಃ
ಗತಿಃಅಪ-ಗತಿಃಹೋಗು-ವುದು
ಗತಿ-ಗಳಿಗೆ
ಗತಿಪ್ರದ-ನಾಗಿ-ರುವು-ದ-ರಿಂದ
ಗತಿ-ಯನ್ನು
ಗತಿಯು
ಗತಿಯೇ
ಗತಿ-ಯೇನು
ಗತಿ-ರಿತಿ
ಗತಿ-ಸಾ-ಧನಂ
ಗತೋ
ಗತೌ
ಗತೌಋ
ಗತೌ-ಹೋಗು-ವುದು
ಗತ್ಯ-ಭಾವೌ
ಗತ್ಯರ್ಥ
ಗತ್ಯರ್ಥತ್ವಾತ್ಪರಮಾತ್ಮನನ್ನು
ಗತ್ಯರ್ಥ-ವುಳ್ಳ-ವು-ಗಳಿಗೆ
ಗತ್ಯರ್ಥಾನಾಂ
ಗತ್ಯರ್ಥಾಶ್ಚ
ಗತ್-ಅಯ್
ಗದಾಪ್ರಹಾರ-ದಿಂದ
ಗಮನಂ
ಗಮನಂಇತಿ
ಗಮನಂನಾಶವು
ಗಮನಂಪ್ರವೇಶವು
ಗಮನಂಹೊರಗೆ
ಗಮನ-ಸಾ-ಧನಂ
ಗಮನಾಗಮನ
ಗಮನಾನ್ನಿರಸನಾ-ದಿತಿ
ಗಮನಿಸಿ
ಗಮನಿಸಿ-ದಲ್ಲಿ
ಗಮನೇ
ಗಮ-ಯತಿ
ಗಮ್ಯ
ಗಮ್ಯಂ
ಗಮ್ಯತೇ
ಗಮ್ಯತ್ವ
ಗರುಡ
ಗರುಡಃಅದರಂತೆ
ಗರುಡ-ನಿಗೆ
ಗರುಡನೇ
ಗರುಡ-ಪತ್ತಿ-ಯನ್ನು
ಗರುಡ-ಮಾರ್ಗ
ಗರುಡರ
ಗರುಡರು
ಗರುಡ-ವಾಹ-ನತ್ವಾನ್ನಾರಾ-ಯ-ಣಃಆ
ಗರುಡ-ವಾಹನ-ನಾಗಿ-ರುವು-ದ-ರಿಂದ
ಗರುಡಶ್ಚ
ಗರುಡಸ್ಯ
ಗರ್ಭೋದಕ-ದಲ್ಲಿದ್ದು
ಗಳಲ್ಲಿ
ಗಾಢನಿದ್ರಾವಸ್ಥೆಯು
ಗಾಢನಿದ್ರೆ-ಯನ್ನು
ಗಾಢನಿದ್ರೆಯು
ಗಾಯತ್ರಿ
ಗಾಳಿಬೀಸುವು-ದ-ರಿಂದ
ಗಾಳಿ-ಯಿಂದ
ಗೀತಾ
ಗೀತಾ-ತಾತ್ಪರ್ಯ
ಗೀತೋಪ-ದೇಶ
ಗೀತೋಪ-ದೇಶವು
ಗೀಯತೇ
ಗುಂಪಿಗೆ
ಗುಣ-ಕೀರ್ತನೆ-ಗಳಲ್ಲಿಯೂ
ಗುಣ-ಗಳ
ಗುಣ-ಗಳನ್ನು
ಗುಣ-ಗಳಿಂದ
ಗುಣ-ಗಳಿಂದಲೂ
ಗುಣ-ಗಳಿಗೂ
ಗುಣ-ಗಳಿಗೆ
ಗುಣ-ಗಳು
ಗುಣತಃ
ಗುಣ-ಪರಿ-ಪೂರ್ಣ-ನಾದ
ಗುಣ-ಪೂರ್ಣತ್ವ
ಗುಣ-ಪೂರ್ಣನು
ಗುಣ-ಪೂರ್ಣನೂ
ಗುಣ-ಪೂರ್ಣಸ್ವ-ರೂಪತ್ವಂಸಮಸ್ತ-ದೋಷ-ವಿರುದ್ಧ
ಗುಣ-ವಿದ್ದರೂ
ಗುಣವೂ
ಗುಣಸ್ವ-ರೂಪ-ನಾಗಿ-ರುವ
ಗುಣಾ
ಗುಣಾಃ
ಗುಣಾಃನಾರಾಃಗು-ಣ-ಗಳು
ಗುಣಾತ್ಮಾನಃ
ಗುಣಾ-ಧಿಕಃ
ಗುಣೇ
ಗುಣೈಃ
ಗುಣೈಸ್ಸರ್ವೈಃ
ಗುರುಂ
ಗುರುಃ
ಗುರು-ಗಳ
ಗುರು-ಗಳನ್ನೇ
ಗುರು-ಗಳಲ್ಲಿ
ಗುರು-ಗಳಲ್ಲಿಯೂ
ಗುರು-ಗಳಾದ
ಗುರು-ಗಳಿಂದ
ಗುರು-ಗಳು
ಗುರು-ತಯಾ-ಸಜ್ಜನ-ರಿಗೆ
ಗುರುತ್ತೇನ
ಗುರುತ್ವನ
ಗುರುತ್ವೇನ-ಯಥಾರ್ಥಜ್ಞಾನ-ವನ್ನು
ಗುರು-ದತ್ತಂ
ಗುರುದ್ವಿಷಃ
ಗುರುದ್ವೇಷಿ-ಗಳೂ
ಗುರುಪ್ರಸಾದದ
ಗುರುಪ್ರಸಾದವೇ
ಗುರು-ಮಾಪ್ಯ
ಗುರು-ಮೇವಾಭಿಗಚ್ಛೇತ್
ಗುರುವು
ಗುರುವೇ
ಗುರೂಂಶ್ಚಾಪಿ
ಗುರೌ
ಗುಹಾಂ
ಗುಹಾಂಹೃದಯ
ಗುಹೆ-ಯನ್ನು
ಗೃಹೀತಸ್ತ್ರೀ-ರೂಪೇಣ
ಗೋವಿಂದನಲ್ಲಿಯೂ
ಗೋವಿಂದೇ
ಗೌ
ಗ್ರಂಥ-ಗಳ
ಗ್ರಂಥ-ಗಳನ್ನೂ
ಗ್ರಂಥ-ಗಳಿಂದ
ಗ್ರಂಥ-ಗಳಿಗೆ
ಗ್ರಂಥ-ಗಳು
ಗ್ರಂಥದ
ಗ್ರಂಥ-ದಲ್ಲಿ
ಗ್ರಂಥವು
ಗ್ರಹಿಸ-ಬೇಕು
ಗ್ರಹಿಸುವ
ಗ್ರಾಮಂ
ಗ್ರಾಮಕ್ಕೆ
ಗ್ರಾಮ-ವನ್ನು
ಗ್ರಾಮಾಣಃನರಾ-ಣಾಂಸಜ್ಜ-ನಾನಾಂಸಜ್ಜ-ನರ
ಗ್ರಾವ್ಣೋ
ಘಟ
ಘಟಾದಿ
ಘರ್ಮಾ-ಸಮಂತಾತ್
ಘೋರಾಂ
ಘೋಷಿ-ಸುತ್ತಿವೆ
ಘ್ರಾಣಂಮೂಗನ್ನೂ
ಘ್ರಾಣಮೇವ
ಘ್ರಾಣೇಂದ್ರಿಯ
ಚ
ಚಂಡಾಲಪರ್ಯಂತ
ಚಂಡಾಲರೇ
ಚಂದ್ರ
ಚಂದ್ರನ
ಚಂದ್ರನು
ಚಂದ್ರನೇ
ಚಂದ್ರ-ಮಂಡಲಸ್ಯ
ಚಂದ್ರ-ಸೂರ್ಯರು
ಚಂದ್ರ-ಸೂರ್ಯೌ
ಚಆದು-ದ-ರಿಂದ
ಚಆಶ್ರಯ-ಭೂತ-ನಾಗಿ-ರುವು-ದ-ರಿಂದ
ಚಈ
ಚಕ್ರಂ
ಚಕ್ರವೇ
ಚಕ್ರಾದಿ
ಚಕ್ರಾದ್ಯಾಯುಧ-ಜಾತಂಸು-ದರ್ಶನ-ಚಕ್ರವೇ
ಚಕ್ಷು
ಚಕ್ಷುಃ
ಚಕ್ಷು-ರಾದಿ
ಚಕ್ಷು-ರಿಂದ್ರಿಯ
ಚಕ್ಷುರ್ಮನೋ-ಹೃದಯ-ರೂಪಸ್ಥಾನತ್ರಯಂ
ಚಕ್ಷುಷೀ
ಚತುರಾನನಂ
ಚತುರ್ಥಕಂ
ಚತುರ್ದಶಭುವನ
ಚತುರ್ದಶಸಂಖ್ಯೋಪೇತಂ
ಚತುರ್ಮುಖ
ಚತುರ್ಮುಖಃ
ಚತುರ್ಮುಖಃಮೇಲೆ
ಚತುರ್ಮುಖನ
ಚತುರ್ಮುಖ-ನನ್ನು
ಚತುರ್ಮುಖಬ್ರಹ್ಮ-ದೇವರ
ಚತುರ್ಮುಖಬ್ರಹ್ಮ-ದೇವರು
ಚತುರ್ಮುಖ-ರೂಪ
ಚತುರ್ಮುಖ-ಶರೀರೇ
ಚತುರ್ಮುಖ-ಶರೀರೇಬ್ರಹ್ಮನ
ಚತುರ್ಮುಖಾದಿ
ಚದುಃಖಾನುಭವ
ಚಪರಿಹಾರ-ಮಾಡುವುದರಿಂದಲೂ
ಚಪುಣ್ಯವೂ
ಚಮತ್ತು
ಚಮೋಕ್ಷ-ವನ್ನು
ಚಯಾವ
ಚರಮಃ
ಚವರುಣ
ಚಹಾಗಾ-ದರೆ
ಚಹಾಗೆಯೇ
ಚಹೀಗಿ-ರುವು-ದ-ರಿಂದ
ಚಾಂಡಾಲ
ಚಾಂಡಾಲ-ಪತಿತೋದಕ್ಕಾ
ಚಾಂದೇ
ಚಾನ್ಯಥಾ
ಚಾಪಿ
ಚಾಪ್ಪುದಕಂ
ಚಾವಿಶತ್
ಚಾಹಂ
ಚಿತ್ತ-ಮಿತಿ
ಚಿತ್ತ-ವಿಷಯ-ಮನಸ್ಸಿಗೆ
ಚಿತ್ತ-ವಿಷಯೋ
ಚಿತ್ರಭಾನು
ಚಿನ್ನದ
ಚೀಪುತ್ತಾ
ಚೆನ್ನಾಗಿಯೂ
ಚೇತಃನರರ
ಚೇತಃಹೃದಯ
ಚೇತನ
ಚೇತನಕೆ
ಚೇತನದ
ಚೇತನರ
ಚೇತನ-ರಲ್ಲಿ
ಚೇತನ-ರಲ್ಲಿದ್ದು
ಚೇತನ-ರಿಂದ
ಚೇತನ-ರಿಗೂ
ಚೇತನ-ರಿಗೆ
ಚೇತನರು
ಚೇತನ-ರು-ಗಳ
ಚೇತನರೂ
ಚೇತನ-ವರ್ಗ-ದಿಂದ
ಚೇತನ-ಸ-ಮುದಾ-ಯಕ್ಕೆ
ಚೇತನ-ಸ-ಮುದಾ-ಯ-ದಿಂದ
ಚೇತನ-ಸಮೂದಾಯ-ವನ್ನೂ
ಚೇತನಾ-ಚೇತನಾತ್ಮಕ-ವಾದ
ಚೇತಸಿ-ಕರೆಸಿ-ಕೊಂಡ-ವ-ನಂತೆ
ಚೇತಿ
ಚೇತ್
ಚೇತ್ಹೀಗೆಂದು
ಚೈಕ
ಚೈತತ್ಸಪ್ತಮಸ್ಕಂಧೆಈ
ಚೈವ
ಚೈವಂ
ಚೋರ
ಚೋರ-ಚಂಡಾಲ-ಪತಿತಶ್ವೋದಕ್ಕಾದಿ
ಛಾಂದಸಃಸಾ-ಮಾನ್ಯ-ರೀತಿ-ಯಲ್ಲಿ
ಛಾಂದೋಗ್ಯ
ಛಾಂದೋಗ್ಯ-ಭಾಷ್ಯ-ದಲ್ಲಿ
ಛಾಂದೋಗ್ಯ-ಭಾಷ್ಯೇ
ಛಾಂದೋಗ್ಯಶ್ರುತಿ-ಯಲ್ಲಿ
ಛಾಂದೋಗ್ಯಶ್ರುತಿಯು
ಛಿದ್ರ
ಛೇದಿಸಿ
ಜಗತಃ
ಜಗತ್
ಜಗತ್ಈ
ಜಗತ್ತಿ-ಗಿಂತ
ಜಗತ್ತಿಗೆ
ಜಗತ್ತಿನ
ಜಗತ್ತಿ-ನಲ್ಲಿ
ಜಗತ್ತಿನಲ್ಲಿ-ರುವ
ಜಗತ್ತಿ-ನಿಂದ
ಜಗತ್ತು
ಜಗತ್ಪ್ರಭುವು
ಜಗತ್ಪ್ರಲಯಃಜಗತ್ತಿನ
ಜಗತ್ಪ್ರಲಯಃನ-ಕಾರಸ್ಯನ
ಜಗತ್ಸದಾ
ಜಗತ್ಸರ್ವಂ
ಜಗದ-ಭಾವಃ
ಜಗದ-ಭಾವಃಪ್ರಲಯಃಜಗತ್ತು
ಜಗದವ್ಯಕಮೂರ್ತಿನಾ
ಜಗದೀಶಸ್ತತೋ
ಜಗದೇತಚ್ಚರಾಚರಂ
ಜಗದ್ವಿಖ್ಯಾತ-ವಾದ
ಜಗದ್ವಿಲಕ್ಷಣ
ಜಗದ್ವಿಲಕ್ಷಣನು
ಜಗದ್ವಿಷಯ-ಕಜ್ಞಾನವು
ಜಗದ್ವ್ಯಾವೃತ್ತಿಮಂತ
ಜಗನ್ನಾಥ-ದಾಸರು
ಜಗನ್ನಿಥ್ಯಾ-ವಾದಿ-ಗ-ಳಾದ
ಜಗನ್ನಿ-ಷೇಧಃಜಗತ್ತಿನ
ಜಗನ್ನಿ-ಷೇಧೋ
ಜಠರ-ದಲ್ಲಿ
ಜಠರೇ
ಜಠರೇಣ
ಜಡ
ಜಡ-ಗಳಿಂದ
ಜಡ-ಪದಾರ್ಥ-ಗಳಲ್ಲ-ವೆಂದೂ
ಜಡೇಭ್ಯಃ
ಜಡೇಭ್ಯಶ್ಚ
ಜಡೇಭ್ಯೋ
ಜತೆ-ಯಲ್ಲಿ
ಜನಕಂ
ಜನ-ಕನು
ಜನ-ಗಳಿಗೆ
ಜನಜನ-ಗಳಲ್ಲಿ
ಜನನ-ರಹಿತ-ನಾದ
ಜನನಿ
ಜನನೀ
ಜನರ
ಜನ-ರನ್ನು
ಜನ-ರಲ್ಲಿ
ಜನ-ರಿಂದ
ಜನ-ರಿಂದಲೂ
ಜನರು
ಜನಾರ್ದನ
ಜನಾರ್ದನಂ
ಜನಾರ್ದನಃ
ಜನಿತಾ
ಜನಿಸುವ
ಜನೈಃ
ಜನ್ಮ
ಜನ್ಮ-ಗಳಲ್ಲಿ
ಜನ್ಮ-ಗಳಿಗೆ
ಜನ್ಮನಿ
ಜನ್ಮ-ಶತಾಬ್ದಿ
ಜನ್ಮ-ಶತಾಬ್ಲಿ
ಜನ್ಮಾಂತರ-ಗಳಲ್ಲಿ
ಜನ್ಮಾದ್ಯಸ್ಯ
ಜನ್ಯ-ವಾದ
ಜಯ-ತೀರ್ಥರು
ಜಯತ್ಯಮಿತಸುಜ್ಞಾನ-ಸುಖ-ಶಕ್ತಿಪಯೋನಿಧಿ
ಜಯ-ವನ್ನು
ಜಯಾತನೂಜೋ
ಜಯಾ-ರೂಪೀ
ಜರುಗಿ-ದಾಗ
ಜಲ
ಜಲ-ದಲ್ಲಿ
ಜಲ-ದಲ್ಲಿಯೂ
ಜಲಧಾರಾಬ್ರಹ್ಮಾಂಡದ
ಜಲವು
ಜಲವೇ
ಜಾಂಬ-ವತಿಯೇ
ಜಾಂಬ-ವತ್ಯಾದ್ಯಾ
ಜಾಗರಿತೇ
ಜಾಗ್ರತ್
ಜಾಗ್ರತ್ಎಚ್ಚರಿಕೆ-ಯಿಂದ
ಜಾಗ್ರದಪಿ
ಜಾಗ್ರದವಸ್ಥೆ
ಜಾಗ್ರದವಸ್ಥೆ-ಯನ್ನು
ಜಾಗ್ರದವಸ್ಥೆಯೂ
ಜಾಗ್ರ-ದಾದಿ
ಜಾಗ್ರ-ದಾದ್ಯವಸ್ಥಾತ್ರಯಂ
ಜಾಗ್ರ-ದಾದ್ಯವಸ್ಥಾಪ್ರ-ವರ್ತನೆ-ಜಾಗ್ರತ್
ಜಾತಂ
ಜಾತಂಪುತ್ರ
ಜಾತಾನಿ
ಜಾತಿ
ಜಾತಿ-ವಿವೇಕೋಸ್ತಿ
ಜಾತೇ
ಜಾತೇ-ಇಲ್ಲ
ಜಾಯಂತೇ
ಜಾಯತೇ
ಜಿಜ್ಞಾಸೆಮಾಡು
ಜಿಜ್ಞಾಸ್ಯ
ಜಿಹ್ವೇಂದ್ರಿಯ
ಜೀವ
ಜೀವಂತಿ
ಜೀವ-ಗಳ
ಜೀವ-ಜಡ
ಜೀವ-ಜಡ-ಗಳಿಂದ
ಜೀವ-ಜಡಾತ್ಮಕ
ಜೀವ-ಜಡಾತ್ಮಕ-ವಾದ
ಜೀವನ
ಜೀವ-ನನ್ನು
ಜೀವ-ನಿಗೆ
ಜೀವನು
ಜೀವನೂ
ಜೀವ-ಭೂತಃ
ಜೀವ-ಮಾತ್ರ-ಪರಃ
ಜೀವ-ಮಾತ್ರ-ಪರಃನರ
ಜೀವ-ಮಾದಾಯ
ಜೀವರ
ಜೀವ-ರನ್ನು
ಜೀವ-ರನ್ನೂ
ಜೀವ-ರಲ್ಲಿ
ಜೀವ-ರಾಶಿ-ಗಳು
ಜೀವ-ರಿಂದ
ಜೀವ-ರಿಗೂ
ಜೀವ-ರಿಗೆ
ಜೀವರು
ಜೀವ-ರು-ಗಳ
ಜೀವ-ರು-ಗಳನ್ನು
ಜೀವ-ರು-ಗಳಿಂದಲೂ
ಜೀವ-ರು-ಗಳಿಗೆ
ಜೀವ-ರು-ಗಳು
ಜೀವ-ಲೋಕೇ
ಜೀವ-ಸ-ಮುದಾ-ಯ-ದಲ್ಲಿ
ಜೀವಾತ್ಮದಾ-ಪರೋಕ್ಷ-ವರ್ಜಿ-ತಾನ್
ಜೀವಾನ್
ಜೀವಾಭಿ-ಮಾನಿ
ಜೀವಾ-ಭೇದೋ
ಜೀವೋಂಧತಾಮಿಸ್ರಂ
ಜೀವೋಂಶಸ್ತಸ್ಯ
ಜೀವೋತ್ತಮನು
ಜೀವೋತ್ತಮ-ರಾದ
ಜೀವೋತ್ತ-ಮರು
ಜೇಷ್ಠ
ಜೊತೆಗೂಡಿ
ಜೊತೆ-ಯಲ್ಲಿ
ಜ್ಞಾತುಂ
ಜ್ಞಾತ್ವಾ
ಜ್ಞಾನ
ಜ್ಞಾನಂ
ಜ್ಞಾನಂಜ್ಞಾನವು
ಜ್ಞಾನಕ್ಕೆ
ಜ್ಞಾನತ್ವಾತ್
ಜ್ಞಾನದ
ಜ್ಞಾನ-ದಿಂದ
ಜ್ಞಾನ-ದಿಂದಲೂ
ಜ್ಞಾನದೋ
ಜ್ಞಾನದ್ವಾರಾ
ಜ್ಞಾನದ್ವಾರಾ-ಯಥಾರ್ಥಜ್ಞಾನ-ವನ್ನು
ಜ್ಞಾನ-ನರ
ಜ್ಞಾನ-ಪೂರ್ಣನು
ಜ್ಞಾನ-ಪೂರ್ವಕಃ
ಜ್ಞಾನ-ಪೂರ್ವಕ-ವಾದ
ಜ್ಞಾನ-ಭಕ್ತಿ-ವೈರಾಗ್ಯ-ಗಳಿಂದ
ಜ್ಞಾನ-ಭಕ್ತ್ಯಾ-ದಿ-ಗುಣ-ಗಳಿಂದ
ಜ್ಞಾನ-ಭಾವಃ
ಜ್ಞಾನ-ಮಾತ್ರಾಶ್ಚ
ಜ್ಞಾನ-ಯಜ್ಞ
ಜ್ಞಾನ-ಯಜ್ಞದ
ಜ್ಞಾನ-ಯುಕ್ತ-ರಾದ-ವ-ರಿಗೆ
ಜ್ಞಾನ-ರೂಪ
ಜ್ಞಾನ-ರೂಪತ್ವಾತ್
ಜ್ಞಾನ-ರೂಪ-ನಾದು-ದ-ರಿಂದ
ಜ್ಞಾನ-ರೂಪಾನ್
ಜ್ಞಾನ-ರೂಪಾನ್ಯಥಾರ್ಥ
ಜ್ಞಾನ-ರೂಪಿ
ಜ್ಞಾನ-ರೂಪಿ-ಯಾಗಿ-ರುವು-ದ-ರಿಂದ
ಜ್ಞಾನ-ರೂಪಿಯೂ
ಜ್ಞಾನ-ವನ್ನು
ಜ್ಞಾನ-ವಿಲ್ಲ
ಜ್ಞಾನ-ವಿ-ಶೇಷ-ದಿಂದಲೂ
ಜ್ಞಾನ-ವಿಷಯತ್ವಾತ್
ಜ್ಞಾನವು
ಜ್ಞಾನ-ವು-ಕೇವಲ
ಜ್ಞಾನ-ವುಳ್ಳ-ವನು
ಜ್ಞಾನ-ವು-ಸಕಲ
ಜ್ಞಾನ-ವೆಂದರ್ಥ
ಜ್ಞಾನ-ಸಾಧಕ
ಜ್ಞಾನ-ಸಾ-ಧನ
ಜ್ಞಾನ-ಸಾ-ಧನಕ್ಕೆ
ಜ್ಞಾನ-ಸಾ-ಧನತ್ವಾತ್
ಜ್ಞಾನ-ಸಾ-ಧನತ್ವಾತ್ಮುಖ್ಯ-ವಾಯು-ವಿಗೆ
ಜ್ಞಾನಾಗ್ನಿಃ
ಜ್ಞಾನಾದಿ
ಜ್ಞಾನಾದಿಸ್ವ-ರೂಪನೇ
ಜ್ಞಾನಾ-ದೇವ
ಜ್ಞಾನಾ-ನಂತರ
ಜ್ಞಾನಾನಂದಮಯ
ಜ್ಞಾನಾನಂದಾದಿ
ಜ್ಞಾನಿ-ಗಳ
ಜ್ಞಾನಿ-ಗಳನ್ನು
ಜ್ಞಾನಿ-ಗಳಾದ
ಜ್ಞಾನಿ-ಗಳಿಗೆ
ಜ್ಞಾನಿ-ಗಳು
ಜ್ಞಾನಿ-ಗಳೆಂದರೆ
ಜ್ಞಾನಿನಾಂ
ಜ್ಞಾನಿನಾಂಅ-ಪರೋಕ್ಷ
ಜ್ಞಾನಿ-ನಾಯಿತಿ
ಜ್ಞಾನಿ-ಪರಃಹಿಂದೆ
ಜ್ಞಾನಿ-ಮಾತ್ರ-ಪರಃ
ಜ್ಞಾನಿಯಲ್ಲಿ
ಜ್ಞಾನಿಸಂಘಾನ್ಜ್ಞಾನಿ-ಗಳ
ಜ್ಞಾನೇನ
ಜ್ಞಾನೇನೈವ
ಜ್ಞಾನೋತ್ತರಂ
ಜ್ಞಾನೋ-ಪ-ದೇಶ
ಜ್ಞಾನೋ-ಪ-ದೇಶದ
ಜ್ಞಾಪಕ-ಮಾಡಿ-ಕೊಂಡ
ಜ್ಞೇಯ
ಜ್ಞೇಯಂ
ಜ್ಞೇಯತ್ವ
ಜ್ಞೇಯತ್ವಾತ್
ಜ್ಞೇಯತ್ವಾತ್ಯಥಾ
ಜ್ಞೇಯನು
ಜ್ಯೋತಿ
ಜ್ಯೋತಿಷ್ಟೋಮ
ಜ್ಯೋತಿಷ್ಟೋಮ-ಅಶ್ವಮೇಧ-ರಾಜಸೂ-ಯಾದಿ
ಜ್ಯೋತಿಷ್ಟೋಮವೇ
ಜ್ಯೋತಿಷ್ಟೋಮಾಶ್ವಮೇಧ-ರಾಜಸೂ-ಯಾದಿ-ಕರ್ಮ-ಕೃತಾಂ
ಟೀಕಾ-ಗಳನ್ನು
ಣ
ಣಂ
ಣಃ
ಣಃಣ
ಣಃಬಲ
ಣಃಬಲ-ವನ್ನು
ಣಕಾರಾರ್ಥಶ್ಯ
ಣಕಾರೋ
ಣತ್ಯಂಣ
ಣತ್ವಂ
ಣಬಲ
ಣಶ್ಚ
ಣಶ್ಚೇತಿ
ತಂ
ತಂಅವ-ನನ್ನು
ತಂತ್ರ-ಸಾರ
ತಂದು
ತಂದು-ಕೊಟ್ಟದ್ದೂ
ತಂದು-ಕೊಳ್ಳ-ಬಹುದು
ತಂದೆ-ತಾಯಿಯ-ರನ್ನು
ತಂದೆ-ಯಾದ
ತಂದೆಯು
ತಂದೆಯೆಂದು
ತಂದೆಯೆನಿ-ಸಿಕೊಳ್ಳುತ್ತಾನೆ-ಅಂದರೆ
ತಕ್ಕ
ತಕ್ಷಣವೇ
ತಚ್ಛಕ್ತಿ-ದನು-ಯೆನಿಸಿ
ತಚ್ಛರೀರಂ
ತಚ್ಛರೀರ-ತಯಾ
ತಚ್ಛ್ರುತೇರಾತ್ಮನಿ
ತಜ್ಜನ್ಯ-ತಯಾ
ತಡೆ-ಮಾಡುವ
ತಡೆ-ಮಾಡುವುದು
ತತ
ತತಃ
ತತಃಪರಮಾತ್ಮನ-ದೆಸೆ-ಯಿಂದ
ತತ-ಮಿದಂ
ತತೋ
ತತೋ-ಽ-ವರುಹ್ಯ
ತತೋ-ಽವ್ಯಯಂ
ತತ್
ತತ್ಅದು
ತತ್ಅ-ಪರೋಕ್ಷಕ್ಕೆ
ತತ್ಆ
ತತ್ಕರ್ತೃತ್ವಾದ್ವಾ
ತತ್ತಿರೋಧಾನ-ಕರ್ತಾ
ತತ್ತೋಪ-ದೇಶದ್ರಷೃ-ತಯಾ
ತತ್ತೋಪ-ದೇಶದ್ರಷ್ಟೃ-ತಯಾ
ತತ್ತೋ-ಪದೇಷ್ಟೃ-ತಯಾ-ಯಥಾರ್ಥಜ್ಞಾನೋ-ಪ-ದೇಶ-ವನ್ನು
ತತ್ತ್ವತಃ
ತತ್ತ್ವಪ್ರತ್ಯೇನ
ತತ್ಪತ್ನೀ-ತೇನ
ತತ್ಪ್ರತಿ
ತತ್ಪ್ರಮಾಣಸ್ಯ
ತತ್ಫಲಂ
ತತ್ಫೂಜಿ-ತತ್ವೇನ
ತತ್ರ
ತತ್ರ-ಅಲ್ಲಿ
ತತ್ರತ್ಯಜನ
ತತ್ರತ್ಯಜನಪ್ರೇರಣಾಯ
ತತ್ರತ್ಯಜನಪ್ರೇರಣಾಯ-ಅಲ್ಲಿ-ರುವ-ವ-ರಿಗೆ
ತತ್ರಾದ್ಯಃ
ತತ್ರಾಪಿ
ತತ್ರಾಪಿಆ
ತತ್ರಾಪಿ-ತೈತ್ತಿರೀಯ
ತತ್ರೈವ
ತತ್ವಂಬಂಧಿ-ತಯಾ
ತತ್ವ-ಗಳಿಗೂ
ತತ್ವತೋ
ತತ್ವತ್ತಾದಿ
ತತ್ವಾಭಿ-ಮಾನಿ
ತತ್ವಾಭಿ-ಮಾನಿ-ದೇವ-ತೆ-ಗಳಿಗೆ
ತತ್ವೇನ
ತತ್ವೋ-ಪದೇಷ್ಟೃ-ತಯಾ
ತತ್ಸಂಬಂಧಿ
ತತ್ಸಂಬಂಧಿ-ಅವನ
ತತ್ಸಂಬಂಧಿತ್ವಾತ್
ತತ್ಸಂಬಂಧಿತ್ವಾತ್ಅದರ
ತತ್ಸಂಬಂಧಿತ್ವಾತ್ಅರ್ಜುನ-ನೊಡನೆ
ತತ್ಸಂಬಂಧಿತ್ವಾತ್ಅರ್ಜು-ನಾದಿ-ರೂಪ-ಗಳಲ್ಲಿದ್ದವ-ರಿಗೆ
ತತ್ಸಂಬಂಧಿತ್ವಾತ್ಆ
ತತ್ಸಂಬಂಧಿತ್ವಾತ್ಚೇತನರ
ತತ್ಸಂಬಂಧಿತ್ವಾತ್ವಾಯು-ದೇವ-ರಿಗೆ
ತತ್ಸಂಬಂಧಿತ್ವಾದ್ವಾ-ಅವ-ರಿಗೆ
ತತ್ಸಂಬಂಧಿತ್ವಾದ್ವಾ-ನರಾ-ಣಾಂತದ್ರೂಪಾಂತರತ್ವೇನಆ
ತತ್ಸಂಬಂಧಿನಃ
ತತ್ಸಂಬಂಧಿ-ನರ
ತತ್ಸಂಬಂಧಿ-ನೀ-ಬೆ-ವರಿನ
ತತ್ಸಂಬಂಧಿ-ರೂಪಾಂತರೇಣ
ತತ್ಸಂಬಂಧೀ
ತತ್ಸಂಬಂಧೀ-ನರ
ತತ್ಸರ್ವಂ
ತತ್ಸೃಷ್ಟತ್ವೇನ
ತತ್ಸ್ವಾನೇನ
ತತ್-ಕರ್ತೃತ್ವಾತ್
ತಥಾ
ತಥಾಚ
ತಥಾ-ಚ-ಆದು-ದ-ರಿಂದ
ತಥಾ-ಹಾಗೆಯೇ
ತಥಾ-ಹಿ-ಹೇಗೆಂದರೆ
ತಥೈವ
ತಥೈವಾ-ನಾದಿ
ತಥೈವಾ-ನಾದಿ-ನಿತ್ಯಕಂ
ತಥೈವೈಕಾದಶೀವ್ರತೇ
ತಥೋಕ್ಕಂಪ್ರಮಾಣವು
ತದಂಗಸ್ವೇದ-ಜನಿ-ತತ್ವೇನ-ಅವನ
ತದಂಧಂತಮೇ
ತದಧಿಗಮ
ತದ-ನಂತರಂ
ತದನ-ಯತ್ವಾತ್ಅವು-ಗಳ
ತದನುಜ್ಞಯಾ
ತದನು-ಭೂಯೇಯಂ
ತದನು-ಸಾರಿ-ಯಾದ
ತದನ್ಯತ್ಎಲ್ಲ-ದ-ರಿಂದಲೂ
ತದನ್ಯೇಷಾಂ
ತದ-ಭಾವಃ
ತದ-ಭಾವಾಃ
ತದ-ಭಾವೋ
ತದಯ-ನತ್ವಾತ್
ತದಯ-ನತ್ವಾತ್ಆ
ತದಯ-ನತ್ವಾತ್ಉಂಟು-ಮಾಡುವ-ವ-ನಾದು-ದ-ರಿಂದ
ತದಯ-ನತ್ವಾನ್ನಾರಾ-ಯಣಃ
ತದಯ-ನತ್ವಾನ್ನಾರಾ-ಯ-ಣಃಈ
ತದರ್ಥಃ
ತದರ್ಥಃಆದರಂತೆ
ತದವಮಂತಾರಃ
ತದಾ
ತದಾತ್ವ-ನಾರಬ್ಧಂ
ತದಾ-ಧಾರಾರ್ಥಮಸೃಜದ್ವೈರಾಜಂ
ತದಾಶ್ನಾತಿ
ತದಾಶ್ರ-ಯತ್ವಾತ್
ತದಾಶ್ರ-ಯತ್ವಾತ್ತತ್ನಾರಾ
ತದಿಚ್ಛಯಾ
ತದೀಯಂ
ತದೀಯ-ಪಶ್ಚಾದ್ಘಾಗೇ
ತದುಕ್ತಂ
ತದುತ್ತರಸ್ಯ
ತದುಪಾಸಿತವ್ಯಂ
ತದುಪಾಸಿತವ್ಯಂಬ್ರಹ್ಮ-ವೆಂಬ
ತದುಭಯ
ತದೇವ
ತದೇವಆ
ತದೇವಾಂತಿಕಮಂತಿಕಾತ್
ತದೇವಾ-ಯನಂ
ತದೇವಾ-ಯನ-ಮಾಶ್ರಯೋ
ತದೈವ
ತದ್ಗತಿ-ದರ್ಶನಾತ್
ತದ್ದತಿ-ದರ್ಶನಾತ್ಈ
ತದ್ದರ್ಶನಾತ್
ತದ್ದರ್ಶನಾತ್ಶ್ರುತಿ-ಗಳೂ
ತದ್ಧತ
ತದ್ಧೃವಂ
ತದ್ಭಕ್ತದ್ವೇಷ
ತದ್ಭಹೇತಿ
ತದ್ಯಥಾ
ತದ್ಯಥಾ-ಅದು
ತದ್ರೂಪದ
ತದ್ರೂಪಾಂತರತ್ತೇನ
ತದ್ರೂಪಾಂತರತ್ಯೇನ
ತದ್ವಂದನ-ತಯಾ
ತದ್ವಾಕ್ಯಸಮಂ
ತದ್ವಾನ್ನಾರಾ-ಯ-ಣಸ್ಕೃತಃ
ತದ್ವಾರಾ
ತದ್ವಾರಾ-ಪಾದದ
ತದ್ವಿಜಿಜ್ಞಾಸಸ್ವ
ತದ್ವಿಜ್ಞಾನಾರ್ಥಂ
ತದ್ವಿರುದ್ದ-ವಿರೋಧದ್ಯೋತಕ-ವಾ-ದುದು
ತದ್ವಿರುದ್ದಾಃ
ತದ್ವಿರುದ್ದಾಃಅರ
ತದ್ವಿರುದ್ಧ-ದ-ಭಾವಾಃ
ತದ್ವಿಷಯ
ತದ್ವೇದಆ
ತದ್ವ್ಯಪ-ದೇಶಾತ್
ತದ್ವ್ಯಪ-ದೇಶಾತ್ಹೀಗೆ
ತದ್ವ್ಯಾಪಾರಾತ್ದುಃಖಪ್ರೇರಕ
ತದ್ವ್ಯಾಪಾ-ರಾದ-ವಿರೋಧಃ
ತನ
ತನಗಾಗಿ
ತನಗಿ-ರುವ
ತನಗಿಲ್ಲ
ತನಗೂ
ತನಗೆ
ತನ್ನ
ತನ್ನದೇ
ತನ್ನನ್ನು
ತನ್ನನ್ನೇ
ತನ್ನಲ್ಲಿ
ತನ್ನಿಂದಲೇ
ತನ್ಮಾರ್ಗ-ಗತಿ-ತೋಽಥವಾ
ತಪಸ್ಸು
ತಪ್ಪದೇ
ತಬ್ಬಿ-ಕೊಂಡರು
ತಮಃಅಂತಹ
ತಮ-ಗಿಂತ
ತಮಗೆ
ತಮಗೆಲ್ಲಾ
ತಮಸಃ
ತಮಸಿ
ತಮಸಿ-ಅವನೇ
ತಮಸ್ಯಂಧೇ
ತಮಸ್ಸಿನ
ತಮಿಮಂ
ತಮೋ
ತಮೋಂಧಂ
ತಮೋ-ಯೋಗ್ಯ
ತಮೋ-ಯೋಗ್ಯ-ಚೇತನರು
ತಮೋ-ಯೋಗ್ಯರ
ತಮೋ-ಯೋಗ್ಯರು
ತಮ್ಮ
ತಮ್ಮನ್ನೇ
ತಯಾ
ತಯಾಆ
ತಯಾ-ರಿಸಿ
ತರಲ್ಪಟ್ಟ
ತರಲ್ಪಟ್ಟು
ತರಿಸುವ
ತರುವಾಯ
ತರ್ಪಕ-ನಾಗಿ
ತಲುಪಿಸು-ವಲ್ಲಿ
ತಲ್ಲಿಂಗ-ಭಂಜನಂ
ತವಾಶ್ರ-ಯತ್ವತೋ
ತವೇದಂ
ತಸ್ಕಾಶ್ರಯ-ತಯಾ
ತಸ್ಮಾತ್
ತಸ್ಮಾತ್ಈ
ತಸ್ಮಾತ್ದೋಷ-ಯುಕ್ತ
ತಸ್ಮಾದ-ಕಾರ-ಬೀಜೋಽಯಂ
ತಸ್ಮಾದ್ಬಲ-ವತ್ತರಮ್
ತಸ್ಮಾದ್ಭೋಗಾದಿನಾ
ತಸ್ಮಾನ್ನ
ತಸ್ಮಾನ್ಮಾಂ
ತಸ್ಮಾ-ಯನ-ಮಾಶ್ರಯಃಆ
ತಸ್ಮಿನ್
ತಸ್ಮಿನ್ಅದ-ರಲ್ಲಿ
ತಸ್ಮಿನ್ಗುರು-ವಿ-ನಲ್ಲಿ
ತಸ್ಯ
ತಸ್ಯಆ
ತಸ್ಯಈ
ತಸ್ಯ-ಪರಮಾತ್ಮನಿಗೆ
ತಸ್ಯ-ಮೇಲೆ
ತಸ್ಯಾಂ
ತಸ್ಯಾಂಆ
ತಸ್ಯಾಂಫ್ರಿ
ತಸ್ಯಾಂಫ್ರಿ-ರೇಣುಭಿರ್ವಾತನೀತ್ಸೆಃ
ತಸ್ಯಾಃ
ತಸ್ಯಾಪ್ಯ-ಯನ-ಮಾಶ್ರಯಃ
ತಸ್ಯಾಯಂ
ತಸ್ಯಾ-ಯನಂ
ತಸ್ಯಾ-ಯನ-ಮಾಶ್ರಯಃ
ತಸ್ಯಾ-ಯನ-ಮಾಶ್ರಯೋ
ತಸ್ಯೇದಂ
ತಸ್ಯೈತೇ
ತಸ್ಯೈವ
ತಾ
ತಾಂ
ತಾಃ
ತಾಃಆ
ತಾತ್ಪರ್ಯ
ತಾತ್ಪರ್ಯ-ನಿರ್ಣಯ-ದಲ್ಲಿ
ತಾತ್ಪರ್ಯಪ್ರಾರಬ್ಧ
ತಾತ್ವಿಕ
ತಾತ್ವಿಕ-ದೈತ್ಯ-ರಿಗೂ
ತಾತ್ಸರ್ಯವೂ
ತಾನಿ
ತಾನು
ತಾನೂ
ತಾನೇ
ತಾನೊಬ್ಬನೇ
ತಾನ್
ತಾನ್ಯೇವ
ತಾಮಸದುರ್ಗತಿಃ
ತಾಮಸರು-ಇ-ವರ
ತಾಯಿಯ
ತಾಯಿ-ಯಾದ
ತಾರತಮ್ಯ
ತಾರತಮ್ಯಕ್ಕೆ
ತಾರತಮ್ಯ-ವನ್ನು
ತಾರತಮ್ಯೇನ-ಅ-ವರ-ವರ
ತಾವತ್
ತಾವದ-ಕಾರಾರ್ಥಾಃ
ತಾವದ-ಕಾರಾರ್ಥಾಃಅ
ತಾವಾಗಿಯೇ
ತಾವು
ತಾಸಾಂ
ತಿಂದಿ-ರುವ
ತಿನ್ನಲ್ಪಡುವ
ತಿನ್ನುವ
ತಿರಸ್ಕರ್ತಾ
ತಿರುಗಿ
ತಿರುಪತಿ
ತಿರುಪತಿಯ
ತಿರೋಧಾನ
ತಿರೋಧಾನ-ಕರ್ತನೂ
ತಿರೋಧಾನ-ವಿಲ್ಲದ
ತಿರೋಹಿತಂ
ತಿರೋಹಿತಂಸ್ವಪ್ನ-ನಾಶವೂ
ತಿರ್ಯಕ್ಮೂಕಜಂತು-ಗಳು
ತಿಳಿ-ದ-ವನು
ತಿಳಿ-ದಿ-ರುವುದನ್ನೂ
ತಿಳಿದು
ತಿಳಿದು-ಬ-ರುತ್ತದೆ
ತಿಳಿದು-ಬ-ರುತ್ತವೆ
ತಿಳಿಯ-ತಕ್ಕದ್ದು
ತಿಳಿಯ-ತಕ್ಕದ್ದು-ಅಂದರೆ
ತಿಳಿಯದ
ತಿಳಿಯ-ಬಹುದು
ತಿಳಿಯ-ಬೇಕು
ತಿಳಿಯಲು
ತಿಳಿಯಲ್ಪಡ-ತಕ್ಕ-ವ-ನಾಗಿ-ರುವ
ತಿಳಿಯಲ್ಪಡ-ತಕ್ಕ-ವನು
ತಿಳಿಯಲ್ಪಡ-ತಕ್ಕ-ವನೂ
ತಿಳಿಯಲ್ಪಡುತಾರೆ
ತಿಳಿಯಲ್ಪಡುತ್ತಾನೆ-ಯಾದು-ದ-ರಿಂದ
ತಿಳಿಯಲ್ಪಡುತ್ತಾರೆ
ತಿಳಿಯಲ್ಪಡುವ
ತಿಳಿಯಲ್ಪಡುವ-ವನು
ತಿಳಿಯುತ್ತದೆ
ತಿಳಿಯುತ್ತಾನೆ
ತಿಳಿಯುತ್ತಾನೆಯೋ
ತಿಳಿಯುತ್ತಾರೆಯೋ
ತಿಳಿಯುತ್ತೇನೆ
ತಿಳಿಯು-ವುದಿಲ್ಲ
ತಿಳಿಯು-ವುದು
ತಿಳಿವಳಿಕೆ
ತಿಳಿ-ಸಲು
ತಿಳಿ-ಸುತ್ತದೆ
ತಿಳಿಸುವ
ತಿಳಿಸುವು-ದ-ರಿಂದ
ತಿಷ್ಠನ್
ತೀಕ್ಷ್ಣ-ವಾದ
ತೀರೇಸ್ವಾಮಿ-ಪುಷ್ಕರಣೀ
ತೀರ್ತ್ವಾ
ತೀರ್ಥ
ತೀರ್ಥ-ವಾಗು-ವುದಿಲ್ಲ
ತೀರ್ಥ-ವೆನಿಸು-ವುದಿಲ್ಲ
ತೀರ್ಥಾಭಿ-ಮಾನಿ
ತು
ತುಂಬಿದ
ತುಅನುಭವಿ-ಸಿಯೇ
ತುಆ
ತುಏನಾಗಬೇಕಾಗಿದೆ
ತುತೋಷ
ತುರಾಷಾಟ್
ತುವ್ಯರ್ಥವೇ
ತುಶ್ರೀ-ರಂಗ-ದಲ್ಲಿನ
ತೃಣ-ಜೀವಿಪರ್ಯಂತ
ತೃಪ್ತ-ನಾಗಿ
ತೃಪ್ತ-ರಾಗುತ್ತಾರೆ
ತೃಪ್ತಿ-ಯನೀವ
ತೆಗೆ-ದು-ಕೊಂಡ
ತೆಗೆ-ದು-ಕೊಂಡು
ತೆಗೆ-ದುಕೊಳ್ಳುವ
ತೆರ-ನಾಗಿ
ತೆರೆದ
ತೇ
ತೇಅಂತಹ-ವರು
ತೇಅ-ವರು
ತೇಜಸ್ವಿ-ಗ-ಳಾದ
ತೇಜಸ್ಸಿಗೆ
ತೇಜಸ್ಸಿ-ನಿಂದ
ತೇಜಸ್ಸು
ತೇನ
ತೇನಆ
ತೇನಈ
ತೇನಾನ್ಯೈಃ
ತೇಭಂಗ-ಮಾಡಲ್ಪಟ್ಟ
ತೇಭ್ಯೋ-ಽಪಿ
ತೇಮೇಲೆ
ತೇಷಾಂ
ತೇಷಾಂಅಂತಹ-ವರ
ತೇಷಾಂಆ
ತೇಷಾಂಚೇತನ-ರಿಗೆ
ತೇಷ್ಟವಸ್ಥಿತಃ
ತೈಃ
ತೈಃಜೀವ
ತೈಜಸ
ತೈತ್ತಿರೀಯ
ತೈತ್ತಿರೀಯಶ್ರುತೌ
ತೈತ್ತಿರೀಯಶ್ರುತೌ-ತೈತ್ತಿರೀಯಶ್ರುತಿ-ಯಲ್ಲಿ
ತೊಟ್ಟಿಲು
ತೋತಾದ್ರಿ
ತೋತಾದ್ರಿಂ
ತೋರಿ
ತೋರಿ-ಸಿ-ರುತ್ತಾರೆ
ತೋರಿ-ಸುವುದಕ್ಕಾಗಿಯೇ
ತೋಷಾತ್
ತ್ತೇನ-ಅದನ್ನು
ತ್ಯಕ್ತ-ಲಿಂಗಾಃ
ತ್ಯಕ್ತ್ವಾ
ತ್ಯಕ್ತ್ವಾ-ಬಿಟ್ಟು
ತ್ಯಜಿಸಿ
ತ್ರಿಂಶದುದ್ಯತ-ನರೈಃ
ತ್ರಿಧಾಮಾ
ತ್ರಿಪಾದಸ್ಯಾಮೃತಂ
ತ್ರಿಭಿರಂಶೈಃ
ತ್ರಿವಿಕ್ರಮ
ತ್ರಿವಿಕ್ರಮ-ದೇವರ
ತ್ರಿವಿಕ್ರಮ-ರೂಪ-ದಿಂದ
ತ್ರಿವಿಕ್ರಮ-ರೂಪ-ವನ್ನು
ತ್ರಿವಿಕ್ರಮ-ರೂಪೀ
ತ್ರಿವೃತಂ
ತ್ರಿವೃತಂಮೂರು-ವಿಧ-ವಾದ
ತ್ವಂ
ತ್ವಂನೀನು
ತ್ವಗಿಂದ್ರಿಯ
ತ್ವನನ್ಯಯಾ
ತ್ವನು-ಭೂಯ
ತ್ವನು-ಭೂಯೇತ-ರೇಷಾ-ಮಾರೋಹಾ-ವರೋಹೌ
ತ್ವರೆ-ಯಿಂದ
ತ್ವಾತ್ನರ-ನೆಂದು
ತ್ವಾನ್ನಾರಾಃ
ದಂದಗ್ಧರಿಸೈನ್ಯಮಾಶು
ದಂದದ್ಧಿ
ದಡ-ದಲ್ಲಿ
ದತ್ತಾ-ಕೊಟ್ಟು
ದತ್ವಾ
ದದತೇ
ದದಾತಿ
ದದಾತಿ-ಕೆಲ-ವಕ್ಕೆ
ದದೌ
ದದೌ-ಕೊಟ್ಟರು
ದಬ್ಬುತ್ತಾರೆ
ದಯಪಾಲಿ-ಸಿದರು
ದಯಪಾಲಿಸುವ-ವನೂ
ದಯಾರಾಹಿತ್ಯ
ದಯಾಸಮುದ್ರ-ನಾದ
ದರ್ಭ-ವನ್ನು
ದರ್ಶನ
ದರ್ಶನಂ
ದರ್ಶನ-ಕೊಡುತ್ತಾನೆ
ದರ್ಶನ-ಕೊಡುವುದೂ
ದರ್ಶನ-ವಂದ-ನಾದಿನಾ
ದರ್ಶನ-ವನ್ನು
ದರ್ಶ-ಯತಿ
ದರ್ಶ-ಯತೋ
ದಲ್ಲಿ
ದಶಾ-ವರಾಣಾಂ
ದಹಂ
ದಹರ
ದಹರಂ
ದಹರಂಸೂಕ್ಷ್ಮ-ವಾದ
ದಹರೋ
ದಹ್ರಂ
ದಾಟಿ
ದಾತಾ
ದಾತೃ-ತಯಾ
ದಾತೃತ್ವ
ದಾರ-ದಲ್ಲಿ
ದಾರಿ
ದಾರಿ-ಮಾಡಿ-ಕೊಟ್ಟ
ದಾರಿ-ಯಂತೆ
ದಾರಿ-ಯಲ್ಲಿ
ದಾರಿ-ಯಾಗಿ
ದಾರಿ-ಯಿಂದ
ದಾರುಣ-ವಾದ
ದಾಶಕಿತ-ವಾದಿತ್ವ
ದಾಶಕಿತ-ವಾದಿತ್ವಂಬೆಸ್ತ
ದಿ
ದಿಕ್ಕು-ಗಳು
ದಿನ-ಗಳು
ದಿನ-ದಲ್ಲಿಯೂ
ದಿನದಿ-ನಕ್ಕೂ
ದಿನೇ
ದಿನೇ-ದಿನೇಪ್ರತಿ-ನಿತ್ಯವೂ
ದಿವಂಗತರ
ದಿವಂಗತ-ರಿಂದ
ದಿವಂಗತ-ರಿಗೆ
ದಿವಂಗತರು
ದಿವಸ-ದಲ್ಲಿ
ದಿವಿಅಂತರಿಕ್ಷ-ದಲ್ಲಿ
ದಿವ್ಯಜ್ಞಾನೋ-ಪ-ದೇಶಾಯಮುಕಿಪ್ರದ-ವಾದ
ದಿವ್ಯರತ್ತೋಜ್ವಲಚ್ಚಿತ್ರಭೂಷಣೇ
ದಿವ್ಯ-ವರ್ಷ-ಸಹಸ್ರಕಂ
ದಿಶಃ
ದೀಪತೈಲಸ್ಯ
ದೀಪದ
ದೀರ್ಘ-ಮಾಡಿ-ದಾಗ
ದುಃಖ
ದುಃಖ-ಗಳನ್ನು
ದುಃಖ-ಗಳೂ
ದುಃಖ-ಪಡುತ್ತಿ-ರುವ
ದುಃಖ-ಪೂರಿತ-ವಾದ
ದುಃಖಪ್ರದ-ವಾದ
ದುಃಖಪ್ರಹಾಣಾಯ
ದುಃಖ-ಭುಕ್
ದುಃಖ-ಭುಗೀಶ್ವರಃ
ದುಃಖ-ಭುಗುಚ್ಯತೇ
ದುಃಖ-ಭೋಗ-ವಿಲ್ಲ
ದುಃಖ-ಭೋಗವು
ದುಃಖ-ಭೋಗಿಯಲ್ಲ
ದುಃಖ-ವನ್ನು
ದುಃಖ-ವನ್ನೇ
ದುಃಖವು
ದುಃಖಸ್ಥಳ-ಗಳಲ್ಲಿ
ದುಃಖಸ್ಥಾನ-ಗಳಲ್ಲಿದ್ದರೂ
ದುರ್ಜನೈಃ
ದುಷ್ಕರ್ಮ-ಗಳ
ದುಷ್ಟಚೇತಸಾಂಪಾಪಿ-ಗಳಿಗೆ
ದುಷ್ಟಚೇತಸಾಮ್
ದುಷ್ಟ-ರಿಗೂ
ದೂರ-ದಲ್ಲಿಯೇ
ದೂರ-ನಾ-ದ-ವನು
ದೂರಸ್ಥಃಅ-ವರ
ದೂರಸ್ಥೋ
ದೂರಾದ್ದೂರ-ತರಂ
ದೂಷಕ
ದೂಷತೋ
ದೃಢ-ವಾದ
ದೃಷ್ಟಿ-ಯಲ್ಲಿ
ದೃಷ್ಟಿ-ಯಿಂದ
ದೆಸೆ-ಯಿಂದ
ದೇವ
ದೇವತಾ
ದೇವ-ತಾ-ಗ-ಣ-ಗಳು
ದೇವ-ತಾ-ವಿಪ್ರವಿ-ತೀರ್ಥಕ್ಷೇತ್ರಾದಿ-ಕಸ್ಯ
ದೇವ-ತಾ-ಸಮೂಹಕ್ಕೆ
ದೇವ-ತಾಸು
ದೇವ-ತಿರ್ಯಕ್-ನ-ರಾದಯಃ
ದೇವ-ತೆ-ಗಳ
ದೇವ-ತೆ-ಗಳನ್ನು
ದೇವ-ತೆ-ಗಳಿಂದ
ದೇವ-ತೆ-ಗಳಿಗೂ
ದೇವ-ತೆ-ಗಳಿಗೆ
ದೇವ-ತೆ-ಗಳು
ದೇವ-ತೆ-ಗಳೂ
ದೇವ-ದತ್ತಾದಿ
ದೇವ-ದತ್ತಾದಿ-ಶಬ್ದ-ವತ್
ದೇವ-ದೇವ-ತೆ-ಗಳು
ದೇವ-ಭಾಗ-ಗತಾತ್ಪಾಪಾತ್ತೇಭ್ಯೋಂಶಂ
ದೇವ-ಮಾನ-ದಿಂದ
ದೇವ-ರ-ಪೂಜೆಯು
ದೇವ-ರಿಗೆ
ದೇವರು
ದೇವ-ಲೋಕ-ದಲ್ಲಿ
ದೇವಸ್ಥಾನವು
ದೇವ-ಹೂತಿ-ಯಿಂದ
ದೇವ-ಹೂತ್ಯಾ
ದೇವಾಂಶಕಾ-ನಪಿ
ದೇವಾಂಶ-ರನ್ನೂ
ದೇವಾಃ
ದೇವಾನಾಂ
ದೇವಾಲ-ಯವೇ
ದೇವಿಯು
ದೇವಿಯೇ
ದೇವೀ
ದೇವೀತಿ
ದೇವೇ
ದೇವೋ
ದೇಶ
ದೇಶತಃ
ದೇಹ
ದೇಹ-ಗಳಲ್ಲಿ
ದೇಹ-ಗಳೂ
ದೇಹದ
ದೇಹ-ದಲ್ಲಿ
ದೇಹ-ದಲ್ಲಿಯೇ
ದೇಹ-ದಿಂದ
ದೇಹ-ಮಾತ್ಮನಃ
ದೇಹ-ಯೋಗಾತ್
ದೇಹ-ಯೋಗಾದ್ವಾಸೋ-ಽಪಿ
ದೇಹ-ವನ್ನು
ದೇಹಾತ್
ದೇಹಾತ್ಶರೀರ-ಗಳಿಂದ
ದೇಹಾನಾಂ
ದೇಹಾಪ-ಗಮನೇಪ್ರಾಕೃತ-ದೇಹವು
ದೇಹಾಭಿ-ಮಾನ-ದಿಂದ
ದೇಹಾಸ್ತಸ್ಯ
ದೇಹೇ
ದೈತ್ಯರ
ದೈತ್ಯ-ರನ್ನು
ದೈತ್ಯ-ರಿಗೂ
ದೈತ್ಯ-ರಿಗೆ
ದೈತ್ಯರು
ದೈತ್ಯರೂ
ದೈತ್ಯಾನಾಂ
ದೈತ್ಯಾನಾಂದೈತ್ಯನ
ದೈತ್ಯಾನಾಂದೈತ್ಯ-ರಿಗೆ
ದೊರಕಿ-ರುವು-ದ-ರಿಂದ
ದೊರಕಿಸಿ-ಕೊಡುವ
ದೊರಕಿಸುವ
ದೊರಕಿಸುವ-ವನೂ
ದೊರೆಯದ
ದೊರೆಯುತ್ತದೆ
ದೊರೆಯು-ವುದಿಲ್ಲ
ದೋಷ
ದೋಷ-ಗಂಧ-ವಿಧುರ-ತಯಾ
ದೋಷ-ಗಳ
ದೋಷ-ಗಳನ್ನು
ದೋಷ-ಗಳಿಂದ
ದೋಷ-ಗಳಿಗೆ
ದೋಷ-ಗಳಿಲ್ಲ
ದೋಷ-ಗಳು
ದೋಷ-ಜ-ನಾನಾಂ
ದೋಷದ
ದೋಷ-ದಿಂದ
ದೋಷ-ದೂರತ್ವ
ದೋಷ-ಪೂರಿತರು
ದೋಷ-ಪೂರಿತ-ವಾದ
ದೋಷ-ಯುಕ್ತ
ದೋಷ-ಯುಕ್ತ-ರಾದ
ದೋಷ-ರಹಿತ
ದೋಷ-ರಹಿತ-ನಾದು-ದ-ರಿಂದ
ದೋಷ-ರಹಿತ-ರಾದ
ದೋಷ-ರಹಿತ-ರಾದ-ವರು
ದೋಷ-ರಹಿತ-ವಾದ
ದೋಷ-ವದ್ಭಿಃ
ದೋಷ-ವದ್ಭಿಃದೋಷ-ಗಳಿಂದ
ದೋಷ-ವರ್ಜಿತಂ
ದೋಷ-ವರ್ಜಿ-ತತ್ವ
ದೋಷ-ವರ್ಜಿತ-ನಾಗಿ-ರುವು-ದ-ರಿಂದ
ದೋಷ-ವಾಚ
ದೋಷ-ವಿರುದ್ದ
ದೋಷ-ವಿರುದ್ಧ-ವಾದ
ದೋಷ-ವಿಲ್ಲ
ದೋಷವು
ದೋಷವೂ
ದೋಷಾಃ
ದೋಷಾಃಅರ
ದೋಷಾಃಅರಾಃ
ದೋಷಾಣಾಂ
ದೋಷಾನ್
ದೋಷಾನ್ದೋಷ-ಗಳನ್ನು
ದೋಷಾ-ಭಾವ
ದೋಷಾರ-ಛಿದ್ರ
ದೋಷಾರ-ಶಬ್ದಯೋಃ
ದೋಷಿ-ಗಳಿಂದ
ದೋಷಿಜನಪರ-ದೋಷ-ದಿಂದ
ದೋಷಿಣಾಂ
ದೋಷಿ-ಭಗವಂತ-ನಲ್ಲಿ
ದೋಷ್ಯಹಿತಃ
ದ್ದೆರಡು
ದ್ಯೋತಕ-ವಾಗಿದೆ
ದ್ರವ್ಯ
ದ್ರಷ್ಟವ್ಯಃ
ದ್ರಷ್ಟವ್ಯಃಣ
ದ್ರಷ್ಟುಂ
ದ್ರುತಂ
ದ್ರೋಹ-ಮಾಡುವುದೇ
ದ್ರೋಹಾದಿ
ದ್ರೋಹೀ
ದ್ವಾದ-ಶಸ್ತೋತ್ರ
ದ್ವಾರಾ-ಅ-ನಿರುದ್ಧ
ದ್ವಿಕರ್ಮಕಃ
ದ್ವಿಕರ್ಮ-ಕಃಎರಡು
ದ್ವಿತಿ-ಯಾವಿ-ಭಕ್ತಿ-ಯಲ್ಲಿದ್ದರೂ
ದ್ವಿತೀಯ
ದ್ವಿತೀಯಃ
ದ್ವಿತೀಯೋ
ದ್ವಿಧಾ
ದ್ವಿಧಾ-ಎರಡು
ದ್ವಿವಿಧ-ಶುಭ
ದ್ವಿಷಂತಃ
ದ್ವಿಷಂತಿ
ದ್ವಿಷಂತ್ಯತ್ರಾಪಿ
ದ್ವೇಷ
ದ್ವೇಷದ
ದ್ವೇಷ-ದಲ್ಲಿಯೂ
ದ್ವೇಷ-ಮಾಡದ
ದ್ವೇಷ-ಮಾಡಿ-ದಂತೆಯೇ
ದ್ವೇಷ-ವಿದೆ
ದ್ವೇಷಾ
ದ್ವೇಷಿ-ಗಳಿಂದ
ದ್ವೇಷಿ-ಗಳಿಗೂ
ದ್ವೇಷಿ-ಗಳಿಗೆ
ದ್ವೇಷಿ-ಸುತ್ತಾರೆ
ದ್ವೇಷಿ-ಸುವ-ವನುಈ
ದ್ವೇಷಿ-ಸುವ-ವರು
ದ್ವೇಷಿ-ಸು-ವುದು
ಧನ
ಧನಂಜಯ
ಧನಾಭಿ-ಜನ-ರೂಪ-ತಪಃ
ಧನ್ಯ-ನನ್ನಾಗಿ
ಧರ-ನಾದ
ಧರಿಸಿ
ಧರಿಸಿದ
ಧರಿಸಿ-ರುತ್ತಾನೆ
ಧರಿಸಿ-ರುತ್ತಾರೆ
ಧರಿಸಿ-ರುವ
ಧರಿಸಿ-ರುವನೋ
ಧರಿ-ಸುತ್ತಾನೆ
ಧರಿಸುವ
ಧರ್ಮ-ತಯಾ
ಧರ್ಮರಾಯ
ಧರ್ಮವೂ
ಧಾತು
ಧಾತು-ವಿಗೆ
ಧಾತು-ವಿನ
ಧಾತೋ
ಧಾತೋಃ
ಧಾತೋಃಇಣ್
ಧಾತೋಃಎಂಬ
ಧಾತೋಃಯಾ
ಧಾತೋಃರೀ
ಧಾತೋ-ರಿ-ಕಾರಸ್ಯ
ಧಾನ್ಯ
ಧಾಮಾನಿ
ಧಾರ-ಕನು
ಧೂಳಿನ
ಧೂಳಿ-ನಿಂದ
ಧೂಳು
ಧೃತಿ
ಧೈಯ
ಧೈಯ-ವಾಗಿಟ್ಟು-ಕೊಂಡು
ಧ್ಯಾನ
ಧ್ಯಾನಕ್ಕೆ
ಧ್ಯಾನ-ಮಾಡಲ್ಪಡುವವ-ನಾಗಿ
ಧ್ಯಾನ-ಮಾಡಲ್ಪಡುವ-ವನು-ಎಂಬು-ದಾಗಿ
ಧ್ಯಾನ-ಮಾಡಿ
ಧ್ಯಾನ-ವಿಷಯ-ನಾಗು-ವುದಿಲ್ಲ
ಧ್ಯಾನಿ-ಸು-ವುದಿಲ್ಲ
ಧ್ರಿಯಮಾಣ-ಶರೀರಕಂ
ನ
ನಂತರ
ನಂತರದ
ನಂತರವೂ
ನಃ
ನಃಅವ-ನಿಗೆ
ನಃನ
ನಅದು
ನಅನಾ
ನಆ-ಅರ-ಆಯ
ನಆ-ಅರ-ಆ-ಯಣ
ನಆ-ಅರ-ಆಯ್ಣ
ನಇಲ್ಲ
ನಇಶ್ಯ
ನಕಾರ
ನಕಾರಃ
ನಕಾರಃನ
ನಕಾರಃವಿರೋಧಾರ್ಥ-ವನ್ನು
ನಕಾರಕ್ಕೆ
ನಕಾರತ್ವಂ
ನಕಾರತ್ವನ
ನಕಾರ-ದ-ಮೇಲೆ
ನಕಾರಸ್ಯ
ನಕಾರಸ್ಯಾರ್ಥಃಹಾಗೆಯೇ
ನಕಾರಾ-ದುತ್ತರ
ನಕುಲ
ನಞಶ್ಚನ
ನಡೆದು
ನಡೆಯುತ್ತದೆ
ನಡೆಯುತ್ತಲೇ
ನಡೆಯುವು-ದ-ರಿಂದ
ನಡೆಸಬೇಕೆಂಬ
ನತ್ವಾ
ನದಿಯ
ನದಿ-ಯನ್ನು
ನದಿಯೇ
ನದೀ
ನದೀಂ
ನನ-ಗಿಂತಲೂ
ನನಿಂದ
ನನ್ನ
ನನ್ನದೇ
ನನ್ನನ್ನು
ನನ್ನನ್ನೇ
ನನ್ನಲ್ಲಿ
ನನ್ನಿಂದ
ನನ್ನಿಂದಲೇ
ನನ್ವೇವಂ
ನಭಃ
ನಭಃಸ್ಥಿತಃ
ನಭಸ್ಥ-ನಾಗಿ
ನಮಗೆ
ನಮಗೆಲ್ಲ
ನಮಸ್ಕರಿಸಲ್ಪಡ-ತಕ್ಕ-ವನು
ನಮಸ್ಕ-ರಿಸಿ
ನಮಸ್ಕರಿಸುತ್ತೇನೆ
ನಮಸ್ಕಾರ-ಮಾಡಿ
ನಮಸ್ಕಾರಾ-ದಿ-ಗಳನ್ನು
ನಮಸ್ಕೃತ್ಯ
ನಮ್ಮ
ನಮ್ಮನ್ನು
ನಮ್ಮೊಂದಿಗೆ
ನಮ್ಮೊಡನೆ
ನಯತಿ
ನಯತಿ-ಕರೆದು-ಕೊಂಡು
ನರ
ನರಂ
ನರಃ
ನರಃನರ
ನರಃನರ-ನೆಂದು
ನರಕ
ನರಕಂ
ನರ-ಕಕ್ಕೆ
ನರ-ಕ-ದಲ್ಲಿ
ನರ-ಕ-ದಲ್ಲಿಯೂ
ನರ-ಕ-ದಲ್ಲಿಯೇ
ನರ-ಕ-ದಿಂದ
ನರ-ಕ-ದುಃಖ-ವನ್ನು
ನರ-ಕಪ್ರ-ದೇಶ-ದಲ್ಲಿಯೂ
ನರ-ಕಪ್ರ-ದೇಶವೇ
ನರ-ಕವು
ನರ-ಕಸ್ಥಾನವು
ನರ-ಕಾದಿ
ನರಕೇ
ನರ-ಕೇ-ಅವನೇ
ನರ-ಕೇಪಿ
ನರ-ನಾರಾ-ಯಣ
ನರ-ನಾರಾ-ಯ-ಣಾಶ್ರಮಂ
ನರ-ನಾರಾ-ಯ-ಣಾಶ್ರಮಈ
ನರ-ನಿಗೆ
ನರ-ನೆಂದು
ನರ-ರಿಗೆ
ನರ-ಶಬ್ದೋ
ನರ-ಶಬ್ದೋ-ಪಲಕ್ಷಿತ
ನರಶ್ಛೋಕ್ತಃ
ನರ-ಸಂಬಂಧಿತ್ವಾತ್
ನರ-ಸಂಬಂಧಿತ್ವಾತ್ಚೇತನರ
ನರ-ಸಂಬಂಧಿತ್ವಾದ್ವಾ
ನರಸಿಂಹ
ನರಸಿಂಹ-ರೂಪ-ದಿಂದ
ನರಸಿಂಹೋಖಲಾಜ್ಞಾನ-ಮತಾಂತದಿವಾ-ಕರಃ
ನರ-ಸೂನವಃ
ನರ-ಸೂನವಃಹರಿಯ
ನರಸ್ಯೇಯಂ
ನರಾಂಶಸಂಪತ್ಯಾ
ನರಾಃ
ನರಾಃನರ-ರೆಂದು
ನರಾಃನರಾ
ನರಾಃನರಾಃ
ನರಾಃಯೇಷಾಂಯಾ-ರಿಗೆ
ನರಾಣಾಂ
ನರಾ-ಣಾಂಚೇತನ-ಸ-ಮುದಾ-ಯಕ್ಕೆ
ನರಾ-ಣಾಂಜನ-ರಿಗೆ
ನರಾ-ಣಾಂಪುಂಸಾಂಪುರುಷ
ನರಾ-ಣಾಂಮನುಷ್ಟ-ರೂಪ-ದಿಂದಿದ್ದ
ನರಾ-ಣಾ-ಮಿದಂ
ನರಾದಯಃ
ನರಾನಾಂ
ನರಾನ್
ನರಾನ್ಸಜ್ಜನ-ರನ್ನು
ನರಾಯ
ನರಾ-ಯಣ
ನರಾ-ಯ-ಣೇತಿ
ನರೈಃ
ನರೈಃಚೇತನ-ರಿಂದ
ನರೈಃಚೇತನ-ರು-ಗಳಿಂದ
ನರೈರಾರ್ಜಿ-ತತ್ವೇನ
ನರೋ
ನರೋತ್ತಮಂ
ನರೋ-ಽ-ನಾಶಾತ್ಪರೋ
ನರೋ-ಽರ್ಜುನಃಅರ್ಜುನ-ನಿಗೆ
ನರೋ-ಽಹಂ
ನಶ್ಯತಿ
ನಶ್ಯತಿ-ನಾಶ
ನಶ್ಯಾಸೌ
ನಶ್ವಾಸೌ
ನಶ್ವೇತಿ
ನಷ್ಟದ
ನಷ್ಟ-ವಾಗುತ್ತದೆ
ನಷ್ಟ-ವಾಗುತ್ತವೆ
ನಷ್ಟ-ವಾಗು-ವುದಿಲ್ಲ
ನಸ್ಯ
ನಾ
ನಾಗಿ-ರುವು-ದ-ರಿಂದಲೂ
ನಾಗು-ವುದಿಲ್ಲ
ನಾಡಿ-ಗಳಲ್ಲಿ
ನಾಡಿಸ್ಥೇ
ನಾಡೀಷು
ನಾಡೀಷು-ನಾಡಿ-ಗಳಲ್ಲಿ
ನಾಡೀಸ್ಥ
ನಾಣ್ಯ-ಗಳನ್ನೂ
ನಾತ್ಮನಃ
ನಾತ್ರ
ನಾಥ
ನಾಥಸ್ವಾಮಿಯೇ
ನಾನಾ
ನಾನಾ-ಭೇದ-ಗಳನ್ನುಳ್ಳ
ನಾನಾ-ಭೇದ-ಸಮನ್ವಿತಾ
ನಾನಾ-ಭೋಗ್ಯ-ವಸ್ತು-ರೂಪ-ಗಳನ್ನು
ನಾನಾ-ವತಾರಾಣಾಂಈ
ನಾನಾವ್ಯಪ-ದೇಶಾತ್ಜೀವನ
ನಾನಾವ್ಯಪ-ದೇಶಾದನ್ಯಥಾ
ನಾನಾ-ಸು-ವರ್ಣ-ಕಾನ್
ನಾನಾ-ಸು-ವರ್ಣ-ಕಾನ್ಅನೇಕ
ನಾನು
ನಾನೇ
ನಾನ್ಯತ್ಕಿಂಚಿದಸ್ತಿ
ನಾನ್ಯ-ದನು-ದಾತುಮಮುಷ್ಯ
ನಾಪಿ
ನಾಪೈತಿ
ನಾಭಾವಿತಿ
ನಾಭಿ-ಯಲ್ಲಿಟ್ಟು-ಕೊಂಡು
ನಾಭಿ-ಯಿಂದ
ನಾಭೇಃಹೊಕ್ಕಳಿನ
ನಾಭೌ
ನಾಭೌ-ರಭೂಚ್ಛ್ರುತೇಃ
ನಾಮ-ಕ-ನಾದ
ನಾಮ-ಗಳ
ನಾಮ-ಗಳು
ನಾಮ-ಜಾತಂ
ನಾಮದ
ನಾಮದಿ
ನಾಮಧಾ
ನಾಮ-ಧಾಃಹೆಸರು-ಗಳನ್ನು
ನಾಯಿ
ನಾರ
ನಾರಂ
ನಾರಂಅಂತಹ
ನಾರಂಅದರ
ನಾರಂಅ-ವರ
ನಾರಂಅ-ವರಿಬ್ಬರ
ನಾರಂಚೇತನ-ಸಂಬಂಧಿ
ನಾರಂತೇಷಾಂ
ನಾರಂನರ-ಕವೇ
ನಾರಂನರ-ನಿಗೆ
ನಾರಂನರ-ರಿಗೆ
ನಾರಂನರಾ-ಣಾಂಚೇತನರ
ನಾರಂನರಾ-ಣಾಂಚೇತನ-ಸ-ಮುದಾ-ಯಕ್ಕೆ
ನಾರಂನಾರ
ನಾರಂನಾರಂ
ನಾರಂಬ್ರಹ್ಮನ
ನಾರಂಬ್ರಹ್ಮಾಂಡ
ನಾರಂಯಸ್ಯ-ಯಾ-ರಿಗೆ
ನಾರಂಸೃಷ್ಟಿ
ನಾರಃ
ನಾರಃಉಕ್ತ-ರೀತ್ಯಾ-ಆಗಲೇ
ನಾರಃಚೇತನರ
ನಾರಃನರಾಣಾಂ
ನಾರಃನಾರ
ನಾರಃಪುತ್ರ
ನಾರಃಸಖಿತ್ವಾದಿ-ನಾ-ಅರ್ಜುನ-ನೊಡನೆ
ನಾರಃಸ್ವಾರ್ಥ-ದಲ್ಲಿ
ನಾರ-ದೀಯೇ-ನಾರ-ದ-ಪುರಾ-ಣ-ದಲ್ಲಿ
ನಾರ-ದೇನ-ನಾರ-ದ-ರಿಂದ
ನಾರ-ನಾರ
ನಾರ-ನಾರಂ
ನಾರ-ಮಿತ್ಯುಚ್ಯತೇ
ನಾರ-ಮಿತ್ಯುಚ್ಯತೇ-ಯತ್ರ-ಯಾವ
ನಾರ-ರಿಗೆ
ನಾರ-ಶಬ್ದನ
ನಾರ-ಶಬ್ದೇನೋಚ್ಯತೇ-ನಾರ
ನಾರಶ್ಚಾಸೌ
ನಾರಶ್ಚಾಸ್
ನಾರಾ
ನಾರಾಃ
ನಾರಾಃನ
ನಾರಾಃನಾರಾ
ನಾರಾ-ಧನಾಯ
ನಾರಾ-ನಾಗಿ
ನಾರಾ-ನಾರಾ
ನಾರಾಯ
ನಾರಾಯಃ
ನಾರಾ-ಯಃಅಥವ
ನಾರಾ-ಯಃನಾರಾಯ
ನಾರಾ-ಯಃಪ್ರಾಪ-ಣ-ತಂದೊದಗಿಸು
ನಾರಾ-ಯಃಹೀಗಿ-ರುವು-ದ-ರಿಂದ
ನಾರಾ-ಯಣ
ನಾರಾ-ಯಣಂ
ನಾರಾ-ಯಣಃ
ನಾರಾ-ಯ-ಣಃಅಂತಹ
ನಾರಾ-ಯ-ಣಃಅಂದರೆ
ನಾರಾ-ಯ-ಣಃಅ-ನಿತ್ಯತ್ವಾದಿ-ದೋಷ-ಶೂನ್ಯಂಅ-ನಿತ್ಯತ್ವವೇ
ನಾರಾ-ಯ-ಣಃಅ-ಯನಂತದೀಯ
ನಾರಾ-ಯ-ಣಃಆ
ನಾರಾ-ಯ-ಣಃಆದು-ದ-ರಿಂದ
ನಾರಾ-ಯ-ಣಃಜಗತ್ತೇ
ನಾರಾ-ಯ-ಣಃಜ್ಞಾನಿ-ಗಳ
ನಾರಾ-ಯ-ಣಃತತ್
ನಾರಾ-ಯ-ಣಃತತ್ರತ್ಯಜನಪ್ರೇರಣಾಯ-ನರ-ಕಾದಿ
ನಾರಾ-ಯ-ಣಃತ-ದೇವ-ನಾರಂ
ನಾರಾ-ಯ-ಣಃತದ್ವಂದನ
ನಾರಾ-ಯ-ಣಃತಸ್ಯಆ
ನಾರಾ-ಯ-ಣಃತಾನಿ
ನಾರಾ-ಯ-ಣಃತಾಸಾಂಆ
ನಾರಾ-ಯ-ಣಃತೇಷಾ
ನಾರಾ-ಯ-ಣಃದೇವ-ತೆ-ಗಳ
ನಾರಾ-ಯ-ಣಃನ
ನಾರಾ-ಯ-ಣಃನಾರ
ನಾರಾ-ಯ-ಣಃನಾರಾಃ
ನಾರಾ-ಯ-ಣಃನಾರಾ-ಯಣ
ನಾರಾ-ಯ-ಣಃನಾರಾ-ಯ-ಣನು
ನಾರಾ-ಯ-ಣಃಮುಖ್ಯ
ನಾರಾ-ಯ-ಣಃಯಸ್ಯ
ನಾರಾ-ಯ-ಣಃಷಡ್ಗುಣೈಶ್ಚರ್ಯ-ಪೂರ್ಣ-ನಾದ
ನಾರಾ-ಯ-ಣಃಸ
ನಾರಾ-ಯ-ಣಃಸಃ
ನಾರಾ-ಯ-ಣಃಸರ್ವತ್ರವ್ಯಾಪ್ತ-ನಾದ
ನಾರಾ-ಯ-ಣಃಸರ್ವಸ್ಯ-ಎಲ್ಲ
ನಾರಾ-ಯ-ಣ-ಕಾರಾರ್ಥೋ
ನಾರಾ-ಯ-ಣತಃ
ನಾರಾ-ಯ-ಣ-ದೇವ-ರಿಂದ
ನಾರಾ-ಯ-ಣನ
ನಾರಾ-ಯ-ಣ-ನದೇ
ನಾರಾ-ಯ-ಣ-ನಿಗೇ
ನಾರಾ-ಯ-ಣ-ನಿರು
ನಾರಾ-ಯ-ಣನು
ನಾರಾ-ಯ-ಣನೇ
ನಾರಾ-ಯ-ಣ-ಪದಸ್ಯಾರ್ಥಃ
ನಾರಾ-ಯ-ಣ-ಪದಸ್ಯಾರ್ಥಃನಾರಾ-ಯಣ
ನಾರಾ-ಯ-ಣ-ಪದಸ್ಯಾರ್ಥೇ
ನಾರಾ-ಯ-ಣ-ಪುರ-ಗಳಲ್ಲಿ-ರುವ
ನಾರಾ-ಯ-ಣಪ್ರಸಾದಮೃತೇ
ನಾರಾ-ಯ-ಣ-ಭಿದಾ-ನಾರಾ-ಯಣ
ನಾರಾ-ಯ-ಣ-ಶಬ್ದಾರ್ಥ
ನಾರಾ-ಯ-ಣ-ಶಬ್ದಾರ್ಥವು
ನಾರಾ-ಯ-ಣ-ಶಯ್ಯೋ
ನಾರಾ-ಯ-ಣಶ್ವಾಸೌ
ನಾರಾ-ಯ-ಣಾನಿ-ರುದ್ಧ-ರೂಪಸ್ಯ
ನಾರಾ-ಯ-ಣಾನಿ-ರುದ್ಧ-ರೂಪಸ್ಯ-ನಾರಾ-ಯಣ
ನಾರಾ-ಯ-ಣಾನಿ-ರುದ್ಧೌ
ನಾರಾ-ಯಣೋ
ನಾರಾ-ಯನ
ನಾರಾ-ಯನೂ
ನಾರಾ-ಯಶ್ಚಾಸೌ
ನಾರಾ-ಯಶ್ವಾಸೌ
ನಾರಾಯಾಃ
ನಾರಾ-ಯಾಃನಾರಾ
ನಾರಾ-ಶಬ್ಧೇನೋಚ್ಯತೇ
ನಾರಾಶ್ಚಾಸೌ
ನಾರಾಶ್ರೀಃ
ನಾರೀ
ನಾರೇತಿ
ನಾರೇತ್ಯಸ್ಯ
ನಾರೇತ್ಯುಕ್ತಾ
ನಾರೈಃ
ನಾರೈಃನಿರ್ದೋಷೈಃ
ನಾರೋ
ನಾಲ್ಕನೇ
ನಾಲ್ಕು
ನಾವು
ನಾಶ
ನಾಶದ
ನಾಶ-ದೊಂದಿಗೆ
ನಾಶ-ಮಾಡಕೊಳ್ಳುವ
ನಾಶ-ಮಾಡುತ್ತಾನೆ
ನಾಶ-ಮಾಡುತ್ತಾನೆಂಬ
ನಾಶ-ಮಾಡುತ್ತಾನೆ-ಯಾದು-ದ-ರಿಂದ
ನಾಶ-ಮಾಡುವ
ನಾಶ-ಯತಿ
ನಾಶ-ಯತಿ-ಅ-ಪರೋಕ್ಷ
ನಾಶ-ರಹಿತ
ನಾಶ-ರಹಿತ-ವಾಗಿ-ರುವುದು
ನಾಶ-ರಹಿತ-ವಾದ
ನಾಶ-ರಹಿತ-ವಾದುವು-ಗಳು
ನಾಶ-ರಾಹಿತ್ಯವು
ನಾಶ-ವಾಗುತ್ತವೆ
ನಾಶ-ವಾಗು-ವುದಿಲ್ಲ
ನಾಶ-ವಾಗು-ವುದಿಲ್ಲ-ವಾದು-ದ-ರಿಂದ
ನಾಶ-ವಾಗುವುದು
ನಾಶ-ವಿಷಯ-ದಲ್ಲಿ
ನಾಶವು
ನಾಶ-ವೆಂಬ
ನಾಶ-ಹೊಂದು
ನಾಶ-ಹೊಂದುತ್ತದೆ
ನಾಶ-ಹೊಂದುತ್ತವೆ
ನಾಶ-ಹೊಂದು-ವಿಕೆ
ನಾಶ-ಹೊಂದು-ವುದಿಲ್ಲ
ನಾಶಾ-ಭಾವಪ್ರ-ಸಿದ್ದೇಃ
ನಾಶುಭಂ
ನಾಸೌ
ನಾಸ್ತಿಕ
ನಾಸ್ತಿಕೋ
ನಾಽಕೃತಂ
ನಾಽಸೌ
ನಿಂತು
ನಿಂದಕಃ
ನಿಂದಾ
ನಿಂದಿ-ಸು-ವುದು-ಇವೆಲ್ಲ
ನಿಃಶೇಷ-ಸುಖ-ವರ್ಜಿತೇ
ನಿಗೆ
ನಿಗ್ರಹಿಸುವ-ವನು
ನಿಘಂಟಿ-ನಲ್ಲಿ
ನಿತ್ಯ
ನಿತ್ಯಂ
ನಿತ್ಯಂಯಾವಾಗಲೂ
ನಿತ್ಯಂಸರ್ವದಾ
ನಿತ್ಯಕಂ
ನಿತ್ಯಜ್ಞಾನಾತ್ಮಕತ್ವತಃ
ನಿತ್ಯಜ್ಞಾನೋ
ನಿತ್ಯತ್ವ-ವಿಲ್ಲ-ದಿ-ರು-ವುದು
ನಿತ್ಯ-ದಲ್ಲಿಯೂ
ನಿತ್ಯ-ನರ-ಕಕ್ಕೆ
ನಿತ್ಯ-ನರ-ಕ-ದಲ್ಲಿ
ನಿತ್ಯ-ನಾದ
ನಿತ್ಯಪ್ರಳಯ-ಕಾಲ-ದಲ್ಲಿಯೂ
ನಿತ್ಯ-ಬಲಃ
ನಿತ್ಯ-ಮುಕ್ಕಳು
ನಿತ್ಯ-ವಾಗಿವೆ
ನಿತ್ಯ-ವಾದ
ನಿತ್ಯ-ವಾದುವು
ನಿತ್ಯವೂ
ನಿತ್ಯವೇ
ನಿತ್ಯ-ಸಮಃಮುಖ್ಯ-ವಾಯು-ದೇವರ
ನಿತ್ಯಾ
ನಿತ್ಯಾಃ
ನಿತ್ಯಾ-ನಂದ-ವನ್ನು
ನಿತ್ಯಾ-ನಂದೋ
ನಿತ್ಯಾ-ನಾದಯ
ನಿತ್ಯಾ-ವಿಯೋ-ಗಿನಿ-ಯಾಗಿ
ನಿತ್ಯಾಸ್ತೇಽ-ಚೇತನಾ
ನಿತ್ಯೈವ
ನಿದಧೇ
ನಿದರ್ಶನ-ಗಳಿಂದಲೂ
ನಿಧಾನಂ
ನಿಧಾನ-ಆಶ್ರಯಸ್ಥಾನವು
ನಿಧಾನ-ಮೇಕೀ
ನಿನಗೆ
ನಿನ್ನ
ನಿಯಂತ್ರಿ-ಸುತ್ತಾನೆ
ನಿಯಮನ
ನಿಯಮ-ವನ್ನು
ನಿಯಮ-ವಿಲ್ಲ
ನಿಯಮವು
ನಿಯಮಿಸುತ್ತ
ನಿಯಮೇನ
ನಿಯಾಮಕಃ
ನಿಯಾಮಕ-ನಾಗಿ-ರುವ-ವನು
ನಿಯಾಮ-ಕನೂ
ನಿರಂತರ-ವಾಗಿ
ನಿರಂತರ-ವಾಗಿ-ದೆಯೋ
ನಿರಪೇಕ್ಷಂ
ನಿರರ್ಥಕ
ನಿರವಧಿಕ
ನಿರಸ-ನಾತ್ತನ್ನ
ನಿರಾಸನಾಚ್ಚ
ನಿರಾಸ-ನಾತ್
ನಿರೀಕ್ಷಣೈಃ
ನಿರುಕ್ತೇನ
ನಿರುಕ್ತೇನ-ರ-ಚನೆಯ
ನಿರುತ
ನಿರು-ಪ-ಮಾನಂದಾತ್ಮ
ನಿರೂಪಿ-ತ-ವಾಗಿ-ರುವ
ನಿರೂಪಿ-ಸ-ಬಹುದು
ನಿರೂಪಿ-ಸ-ಲಾಗಿದೆ
ನಿರೂಪಿ-ಸಲ್ಪಟ್ಟಿದೆ
ನಿರೂಪಿಸಿ
ನಿರೂಪಿ-ಸಿ-ರುತ್ತಾರೆ
ನಿರೂಪಿ-ಸುತ್ತದೆ
ನಿರೂಪ್ಯತೇ
ನಿರ್ಗಮನೇ
ನಿರ್ಗು-ಣತ್ವಂ
ನಿರ್ಣಯ
ನಿರ್ಣಯ-ದಲ್ಲಿ
ನಿರ್ಣಯಿಸಲ್ಪಟ್ಟಿದೆ
ನಿರ್ದೆಶ
ನಿರ್ದೆಶಿ-ಸುತ್ತವೆ
ನಿರ್ದೇಶಿ-ಸಲು
ನಿರ್ದೋಷ
ನಿರ್ದೋಷತ್ವಾತ್
ನಿರ್ದೋಷತ್ವಾತ್ದೋಷ
ನಿರ್ದೋಷ-ನಾದ
ನಿರ್ದೋಷ-ನಾದು-ದ-ರಿಂದ
ನಿರ್ದೋಷನು
ನಿರ್ದೋಷ-ವಾದ
ನಿರ್ದೋಷೈಃ
ನಿರ್ಬಂಧ-ಕ-ನಾದು-ದ-ರಿಂದಲೂ
ನಿರ್ಭಿನ್ನ
ನಿರ್ಮಿತಂ
ನಿರ್ಮಿ-ಸಿದನು
ನಿರ್ಲೇಪೋ
ನಿರ್ವಚ-ನತ್ವಾತ್
ನಿರ್ವಹಿ-ಸುತ್ತಾನೆ
ನಿರ್ವಹಿಸುತ್ತಾರೆಯೋ
ನಿರ್ವಹಿಸುವ
ನಿರ್ವಹಿಸುವ-ವನು
ನಿರ್ವಿ-ಕಾರಶ್ಚ
ನಿರ್ವೃತಿ
ನಿರ್ವೃತಿಃ
ನಿರ್ವೈರಂ
ನಿರ್ವೈರಂಯಾ-ರಲ್ಲಿಯೂ
ನಿಲ್ಲಿಸಬೇಕಾಗಿ-ದೆಯೋ
ನಿಲ್ಲಿ-ಸಿದೆ
ನಿಲ್ಲಿಸುತ್ತೇನೆ
ನಿಲ್ಲುವಂತೆ
ನಿವಾರಣೆ-ಮಾಡುವುದರ
ನಿವಿಷ್ಟಃಇದ್ದು
ನಿವಿಷ್ಟೋ
ನಿವೇದಿಸಿಕೊಳ್ಳುತ್ತಿದ್ದೇವೆ
ನಿವೇಶಿತಾಃ
ನಿಶಮ್ಯ
ನಿಶ್ಚಯ
ನಿಶ್ಚ-ಯ-ವಾಗಿ
ನಿಷೇಧ
ನಿಷೇಧಾರ್ಥ-ಕತ್ವೇನ
ನಿಷೇಧಾರ್ಥ-ಕತ್ವೇನ-ಇಲ್ಲ
ನಿಷ್ಕಾಮ
ನೀಚಸ್ತ್ರೀ
ನೀಡಿ
ನೀಡುವುದರ-ಮೂಲಕ
ನೀತಿ-ಗಳ
ನೀನು
ನೀರಿಗೆ
ನೀರಿನ
ನೀರು
ನೀರೇ
ನೀಲ-ಕಾರ್ಪಾಸ-ವಸ್ತ್ರಸ್ಯ
ನು
ನುಡಿ
ನುಡಿಯು
ನೂ
ನೂರು
ನೃ
ನೃಇತ್ಯತ್ರನೃ
ನೃಣಾಂ
ನೃಣಾಂಸಜ್ಜ-ನರ
ನೃಣಾಂಸಜ್ಜ-ನಾನಾಂಸಜ್ಜ-ನರ
ನೃಸಿಂಹ-ರೂಪೀ
ನೄ
ನೄಅರಆ
ನೄಣಾಂ
ನೄಣಾಂಸಜ್ಜ-ನಾನಾಂಸಜ್ಜನ-ರಿಗೆ
ನೄನ್
ನೆಂದು
ನೆಂಬ
ನೆನಪಿಗೆ
ನೆನ-ಸಿದರೆ
ನೆರವೇರಿತೆಂಬುದನ್ನು
ನೆರವೇರಿ-ಸುವ-ವನು
ನೆಲಸಿ-ರುವ
ನೆಲಸಿರು-ವನು
ನೇ
ನೇತ್ರ
ನೇಮಿ
ನೇರ-ವಾಗಿ
ನೇರ-ವಾದ
ನೈಕಸ್ಥಿನ್
ನೈಕೋಪಿ
ನೈಮಿಷಂ
ನೈಮಿಷಾರಣ್ಯ
ನೈವ
ನೈವಂ
ನೈವಾ-ಯಾತಿ
ನೈವೈತ
ನೋ
ನೋಈ
ನೋಡ-ಬೇಕು
ನೋಡಲು
ನೋಡಿಕೊಳ್ಳ-ತಕ್ಕದ್ದು
ನೋಡಿತು
ನೋಡಿರಿ
ನೋಡಿ-ರುತ್ತಾರೆಂಬ
ನೋಡುತ್ತಾನೆ
ನೋಡುವ
ನೋಣಃ
ನ್ಯಾಯ-ದಂತೆ
ಪಂಚಕಂತ-ದೇವ
ಪಂಚಕಷ್ಟೇತಿದಾರುಣೇ
ಪಂಚ-ರೂಪ-ದಲಿ
ಪಂಡಿತಾಚಾರ್ಯರು
ಪಚನ-ಕರ್ತೃ
ಪಚನಕ್ಕೆ
ಪಚನ-ವಾಗಲು
ಪಚನೇನಜೀರ್ಣ-ವಾಗುವಂತೆ
ಪಟಶ್ಚ
ಪಠ-ತಾಂವ್ಯಾ-ಸಂಗ
ಪಠತಾಮರ್ಭಕಾಣಾಮಯಂ
ಪಡೆ-ದರು
ಪಡೆದವ-ರಿಗೆ
ಪಡೆದಿರಬೇಕೆಂದು
ಪಡೆ-ದಿ-ರುವ
ಪಡೆ-ದಿ-ರುವು-ದ-ರಿಂದ
ಪಡೆದು
ಪಡೆದು-ಕೊಳ್ಳಲಾ-ಗಿದ್ದು
ಪಡೆಯಲು
ಪಡೆಯುತ್ತಾನೆ
ಪಡೆಯು-ವರು
ಪಡೆಯುವ-ವರು
ಪತಂತಿ
ಪತಂತ್ಯಂಧೇ
ಪತಂತ್ಯೇವ
ಪತಿತ
ಪತಿತಾನಾಂ
ಪತ್ತೀ-ಸಂಬಂಧ
ಪತ್ನಿ-ಪುತ್ರರು
ಪತ್ನಿ-ಯಾಗಿ-ರುವು-ದ-ರಿಂದ
ಪತ್ನೀತ್ವಾನ್ನಾರಾ-ಽಸಿ
ಪತ್ರ
ಪದ
ಪದಂ
ಪದಂಎದುರಿಗಿ-ರುವ
ಪದಂತಜ್ಜನ್ಯ-ತಯಾ-ಅವ-ನಿಂದ
ಪದಕ್ಕೆ
ಪದ-ಗಳ
ಪದ-ಗಳಿಗೆ
ಪದ-ಗಳು
ಪದದ
ಪದ-ದಲ್ಲಿ
ಪದ-ದಲ್ಲಿ-ರುವ
ಪದ-ದಿಂದ
ಪದ-ದಿಂದಲೇ
ಪದಬಿಡಿ-ಸು-ವುದು
ಪದ-ರ-ಚನೆ-ಯಿಂದ
ಪದ-ವನ್ನು
ಪದ-ವಾಗಿ
ಪದ-ವಿ-ಭಾಗಃನಾರಾಯ
ಪದವು
ಪದವೇ
ಪದಾಗ್ರ
ಪದಾಗ್ರ-ಪರ-ಪರ್ಯಾಯ
ಪದಾಧಿ-ಕಾರಿ-ಗಳನ್ನು
ಪದಾಧಿ-ಕಾರಿ-ಗಳೆಲ್ಲರ
ಪದ್ಭ್ಯಾಂ
ಪದ್ಮ
ಪದ್ಮಕ್ಕೆ
ಪದ್ಮ-ನಾಭ
ಪದ್ಮ-ನಾಭಃ
ಪದ್ಮ-ನಾಭ-ರೂಪೀ
ಪದ್ಮ-ನಾಭೋ
ಪದ್ಮ-ಪುರಾ-ಣ-ದಲ್ಲಿ
ಪದ್ಮ-ವನ್ನು
ಪದ್ಮ-ಶಂಖಾಂಬ-ರಾದಿ
ಪದ್ಮಾ-ಸಹಿತಃಲಕ್ಷ್ಮೀ-ಸಮೇತ-ನಾಗಿ
ಪದ್ಮಾ-ಸಹಿತೋ
ಪದ್ಯ-ಪದ್ಮವು
ಪಯ್
ಪರಂ
ಪರಂತಪ
ಪರಂಪರಾನುಗತ-ವಾಗಿ
ಪರತಃ
ಪರತರಂ
ಪರದಾರರತಾಸ್ಪರ್ಶಾತ್
ಪರದ್ರವ್ಯ-ವನ್ನು
ಪರದ್ರವ್ಯಾಪಹಾರಿಣಃ
ಪರದ್ವಿಷೋ
ಪರಬ್ರಹ್ಮಜ್ಞಾನಿನಃ
ಪರಬ್ರಹ್ಮನ
ಪರಬ್ರಹ್ಮ-ನನ್ನು
ಪರಬ್ರಹ್ಮ-ನಲ್ಲಿಯೇ
ಪರಬ್ರಹ್ಮ-ನಿಂದಲೇ
ಪರಬ್ರಹ್ಮ-ನಿಗೆ
ಪರಬ್ರಹ್ಮನು
ಪರಬ್ರಹ್ಮಾ-ಭಿದಾಯಕಃಭಾಷ್ಯ
ಪರಮ
ಪರಮಂ
ಪರಮ-ಪವಿತ್ರ-ವಾದ
ಪರಮ-ಪುರುಷಾತ್
ಪರಮಪ್ರೇಮಾಸ್ಪದ-ನಾದ
ಪರಮ-ಮುಖ್ಯ
ಪರಮ-ಮುಖ್ಯ-ತಯಾ
ಪರಮ-ಮುಖ್ಯ-ವಾಗಿ
ಪರಮಾ-ಗತಿಃ
ಪರಮಾಣು-ವತ್
ಪರಮಾಣುವಿ-ನಂತೆ
ಪರಮಾತ್ಮನ
ಪರಮಾತ್ಮನನು
ಪರಮಾತ್ಮನನ್ನು
ಪರಮಾತ್ಮನಲ್ಲಿ
ಪರಮಾತ್ಮನಿ
ಪರಮಾತ್ಮನಿಂದ
ಪರಮಾತ್ಮನಿಂದಲೇ
ಪರಮಾತ್ಮನಿಗೆ
ಪರಮಾತ್ಮನು
ಪರಮಾತ್ಮನೇ
ಪರಮಾತ್ಮನೊಬ್ಬನೇ
ಪರಮಾತ್ಮ-ರಿಗೆ
ಪರಮಾತ್ಮಾ-ಸರ್ವೋತ್ತಮ-ನಾದ
ಪರ-ಮಾನಂದ
ಪರ-ಮಾನಂದಂ
ಪರ-ಮಾನಂದಃ
ಪರ-ಮಾನಂದ-ರೂಪಿ-ಣಂಶ್ರೇಷ್ಠ-ವಾದ
ಪರ-ಮಾನಂದ-ರೂಪಿ-ಣಮ್
ಪರಮಾಭಿತುಷ್ಟಃ
ಪರಮೇಶಿತುಃ
ಪರಮೇಶ್ಮ-ಭೂತಂ
ಪರಮೋಜ್ವಲೇ
ಪರವಶನು
ಪರವೇಶ್ಮ-ಭೂತಂಶ್ರೇಷ್ಠ-ವಾದ
ಪರಸ್ತತೋ
ಪರಸ್ತ್ರೀರತ
ಪರಸ್ಯ
ಪರಾಂತೇ
ಪರಾತ್ಮನಃ
ಪರಾನಷ್ಟೌ
ಪರಾನ್ಪವಿತ್ರ-ವಾದ
ಪರಾಪರ-ವಸ್ತು-ಗಳ
ಪರಾ-ಭಕ್ತಿಃ
ಪರಾಭಿಧ್ಯಾ-ನಾತ್
ಪರಾಭಿಧ್ಯಾ-ನಾತ್ತು
ಪರಾಭಿಧ್ಯಾ-ನಾತ್ತುಆ
ಪರಾ-ವರಃ
ಪರಿ-ಕೀರ್ತಿತಾ
ಪರಿಕ್ಷಯಃ
ಪರಿಗೃಹ್ಯ
ಪರಿತ್ಯಜಿಸಿ
ಪರಿ-ಪೂರ್ಣ
ಪರಿ-ಪೂರ್ಣನು
ಪರಿ-ಪೂರ್ಣ-ವಾದ
ಪರಿಫುಲ್ಲಲೋ-ಚನಃ
ಪರಿರೇಭೇ
ಪರಿಶುದ್ದ-ವಾದ
ಪರಿಸರಪದ್ದತಿಂ
ಪರಿಸರಪದ್ಧತಿಂವರ್ತ್ಮ-ಯನಿ-ಮುಖ್ಯಪ್ರಾಣ-ಮಾರ್ಗ-ದಿಂದ
ಪರಿಸ್ಥಿತಿ-ಯಲ್ಲಿ
ಪರಿ-ಹರಿ-ಸಲು
ಪರಿ-ಹ-ರಿಸಿ
ಪರಿ-ಹರಿ-ಸುವ-ವನು
ಪರಿಹಾರ
ಪರಿಹಾರ-ಕತ್ವೇನ
ಪರಿಹಾರ-ಕ-ನಾದ
ಪರಿಹಾರ-ಗಳಿಗೆ
ಪರಿಹಾರದ
ಪರಿಹಾರ-ಮಾಡಿ
ಪರಿಹಾರ-ಮಾಡುವುದರಿಂದ
ಪರಿಹಾ-ರಾದಿ-ನಾ-ಪರಿಹಾರ-ಮಾಡುವುದೇ
ಪರಿಹಾರಾಯ
ಪರಿಹಾರಾಯದ್ವೇಷದ
ಪರ್ಯಾಯ
ಪರ್ಯಾಯತ್ವಂ
ಪರ್ಯಾಯತ್ವಾತ್
ಪರ್ಯಾಯೌ-ಗತಿ
ಪವ-ನಾತ್ಕಜಸ್ಯ
ಪವಿತ್ರ-ನಾಗುತ್ತಾನೆ
ಪವಿತ್ರ-ರನ್ನಾಗಿ
ಪಶು
ಪಶು-ಕೀರ್ತ್ಯಾದ್ಯಭೀಷ್ಟ
ಪಶ್ಚಾದ್ಭಾಗೇ
ಪಶ್ಯತ-ನೋಡಿರಿ
ಪಶ್ಯತಿ
ಪಾಕಸ್ಪರ್ಶ-ನಾದಿನಾ
ಪಾಠಾತ್
ಪಾತಾಲಂ
ಪಾದ
ಪಾದ-ಕಮಲ-ಗಳಲ್ಲಿ
ಪಾದ-ಗಳ
ಪಾದ-ಗಳಲ್ಲಿ
ಪಾದದ
ಪಾದ-ಧೂಳು
ಪಾದ-ಪಂಕಜೇ
ಪಾದ-ಪದ್ಮಾರಾಧಕ-ರಾದ
ಪಾದ-ಮೂಲಂ
ಪಾದ-ಮೂಲಂಪಾದದ
ಪಾದ-ರಜಸ್ಸಿ-ನಿಂದ
ಪಾದ-ವನ್ನು
ಪಾದವು
ಪಾದಸ್ಪರ್ಶ-ವಾದ
ಪಾದಸ್ಪೃಷ್ಟರಜೋ
ಪಾದಸ್ಸಷ್ಟರಜಃಪಾದಕ್ಕೆ
ಪಾದಾಂಗುಷ್ಟನಖ-ಎಡಗಾಲಿನ
ಪಾದ್ಮೇ
ಪಾಪ
ಪಾಪಂ
ಪಾಪಂಪಾಪ-ಲೇಪ-ವಿಲ್ಲ-ವೆಂದು
ಪಾಪ-ಕರ್ಮ
ಪಾಪ-ಕರ್ಮಕ್ಕನು-ಸಾರ-ವಾಗಿ
ಪಾಪ-ಕರ್ಮ-ಗಳಿಗೆ
ಪಾಪ-ಕರ್ಮ-ಜಾತಂಪಾಪ-ಕರ್ಮದ
ಪಾಪ-ಕರ್ಮದ
ಪಾಪ-ಕರ್ಮ-ಸಹಾಯಾ
ಪಾಪ-ಕರ್ಮಾಚರಣೆ-ಯಲ್ಲಿ
ಪಾಪ-ಕರ್ಮಾಣಿ
ಪಾಪ-ಗಳ
ಪಾಪ-ಗಳನ್ನು
ಪಾಪ-ಗಳನ್ನೂ
ಪಾಪ-ಗಳು
ಪಾಪ-ಗಳೂ
ಪಾಪತ್ಕ-ರಿಗೆ
ಪಾಪದ
ಪಾಪ-ದಲ್ಲಿ
ಪಾಪ-ಪರಿಹಾರ-ಕತ್ವೇನ-ಪಾಪ-ಗಳನ್ನು
ಪಾಪ-ಪರಿಹಾರ-ಕ-ನಾದ
ಪಾಪ-ಪುಣ್ಯ
ಪಾಪ-ಪುಣ್ಯ-ಗಳ
ಪಾಪ-ಪುಣ್ಯ-ಗಳು
ಪಾಪ-ಫಲ-ವಾದ
ಪಾಪ-ಫಲವು
ಪಾಪ-ಮಾಡಲು
ಪಾಪ-ಮಾಡಿ-ದ-ವರು
ಪಾಪ-ಮಾಪ್ನುಯುಃ
ಪಾಪ-ಯುಕ್ತ-ರಾದ
ಪಾಪ-ರಹಿತ-ರನ್ನಾಗಿ
ಪಾಪ-ರಾಶಿ-ಯನ್ನು
ಪಾಪ-ರೂಪಿಣಃ
ಪಾಪ-ವಾಗಲಿ
ಪಾಪ-ವಾಗಲೀ
ಪಾಪವು
ಪಾಪಾತ್ಮ-ರಿಂದ
ಪಾಪಾತ್ಮರು
ಪಾಪಾತ್ಮಾನಸ್ತು
ಪಾಪಾನಾಂ
ಪಾಪಿ-ಗಳ
ಪಾಪಿ-ಗಳು
ಪಾಪಿ-ಗಳೆಲ್ಲರೂ
ಪಾಪಿನಃ
ಪಾಪಿಷ್ಟ-ರಿಗೂ
ಪಾಪಿಷ್ಠರ
ಪಾಪೇನ
ಪಾರತಂತ್ರ್ಯಾದ-ಯಃಸ್ವತಂತ್ರ-ನಲ್ಲದೇ
ಪಾರತಂತ್ರ್ಯಾದಯೋ
ಪಾರೋಕ್ಷ್ಯೇಣ
ಪಾರ್ಥ-ನಾಗಿ
ಪಾರ್ಥನೂ
ಪಾರ್ಥೋಪೀಷತ್ತದಾತ್ಮಕಃ
ಪಾಲು
ಪಾವ-ನಾತ್
ಪಾವಿತ್ರ್ಯ-ಕರ್ತೃತ್ವೇನ
ಪಾಶ-ದಿಂದ
ಪಿತಾ
ಪಿತುಃ
ಪಿತುಷ್ಟಿತಾ
ಪಿತುಷ್ಟಿತಾ-ಜಗತ್ತಿಗೆ
ಪಿಬತಿ
ಪಿಬತಿಸ್ವೀ-ಕರಿ-ಸುತ್ತಾನೆ
ಪಿಬತ್ಯಸೌ
ಪಿಬೇತ್
ಪೀಠಾಧಿಪತಿ-ಗಳಿಂದ
ಪೀತಾಂಬರ
ಪುಂಡರೀಕಂ
ಪುಂಡರೀಕ-ಪುರಂಅಧ್ಯಸಂಸ್ಥಂ
ಪುಂಸಾಂ
ಪುಂಸೋ
ಪುಂಸ್ತ್ವ
ಪುಂಸ್ತ್ವಂಪುಂಸ್ತ್ವವು
ಪುಂಸ್ತ್ವದ
ಪುಂಸ್ತ್ವಾಂ
ಪುಚ್ಚಂ
ಪುಚ್ಛಂಕೂರ್ಮ-ರೂಪದ
ಪುಣ್ಯ
ಪುಣ್ಯಂ
ಪುಣ್ಯ-ಕರ್ಮಕ್ಕನು-ಸಾರ-ವಾಗಿ
ಪುಣ್ಯ-ಗಳ
ಪುಣ್ಯ-ಗಳಿಗೆ
ಪುಣ್ಯ-ಗಳೂ
ಪುಣ್ಯ-ಗಳೆ-ರಡೂ
ಪುಣ್ಯದ
ಪುಣ್ಯ-ದಲ್ಲಿ
ಪುಣ್ಯ-ಪಾಪ-ಗಳ
ಪುಣ್ಯ-ಮನಿಷ್ಟಕಂ
ಪುಣ್ಯ-ಮಷ್ಯಸ್ಯ
ಪುಣ್ಯ-ರಹಿತ-ರಿಗೂ
ಪುಣ್ಯ-ಲೋಕಂ
ಪುಣ್ಯ-ಲೋಕ-ಗಳಿಗೆ
ಪುಣ್ಯ-ವಾಗಲಿ
ಪುಣ್ಯ-ವಾಗಲೀ
ಪುಣ್ಯವು
ಪುಣ್ಯವೂ
ಪುಣ್ಯೇನ
ಪುತ್ಥಳಿ-ಯನ್ನೂ
ಪುತ್ರ
ಪುತ್ರ-ಧನ
ಪುತ್ರ-ಭಾತೃ-ಸಖಿತ್ವೇನ
ಪುತ್ರಭ್ರಾತೃ-ಸಖಿತ್ವಾದಿನಾ
ಪುತ್ರಭ್ರಾತೃ-ಸಖಿತ್ವೇನ
ಪುತ್ರಾದಿ
ಪುನಃ
ಪುನರ-ವ-ಕಾಶಃನರಾಣಾಂ
ಪುನರಾ-ವರ್ತಿ-ವರ್ಜಿತೇ
ಪುನರಾ-ವೃತ್ತರಾಹಿತ್ಯೇನ
ಪುನರಾ-ವೃತ್ತಿ
ಪುನಸ್ತೋಷಾನ್ಮಧ್ವಃ
ಪುನ್ನಾಮಾ
ಪುಮಾನಿತಿ
ಪುಮಾನ್
ಪುರಂ
ಪುರಾ
ಪುರಾ-ಣ-ಗಳು
ಪುರಾ-ಣ-ದಲ್ಲಿ
ಪುರಾ-ಣಪ್ರ-ಸಿದ್ದ
ಪುರಾಣೇ
ಪುರಾ-ಣೋಕ್ತ
ಪುರಾ-ಹಿಂದೆ
ಪುರುಷಂ
ಪುರುಷಂಈ
ಪುರುಷ-ನಾಮ-ಕ-ರೂಪ-ವನ್ನು
ಪುರುಷಾತ್ಯಾವ
ಪುರುಷಾರ್ಥಕ್ಕೆ
ಪುರುಷಾರ್ಥಹೇತು-ವಾದ
ಪುರುಷೋ
ಪುರುಷೋತ್ತಮಃ
ಪುರುಷೋತ್ತಮ-ನಾದ
ಪುರುಷೋತ್ತಮ-ನೆಂದು
ಪುರೇ-ಹೃದಯ-ಗುಹೆ-ಯಲ್ಲಿ
ಪುರೋನೀತೋ
ಪುಲೋನ್ಮುಖಾನಿ
ಪುಷ್ಕರ
ಪುಷ್ಕರಂ
ಪುಷ್ಕರಂಪದ್ಮವು
ಪುಷ್ಕರ-ಪಲಾಶ
ಪುಷ್ಕರ-ಪಲಾಶೇ-ಕಮಲದ
ಪೂಜಕ
ಪೂಜನಂ
ಪೂಜ-ನಾದಿನಾ
ಪೂಜಯೇತ್ತಾಂ
ಪೂಜಾ
ಪೂಜಾ-ಭಾ-ವಾದ್ದಶಾಹಂ
ಪೂಜಾ-ಮಾಡಲ್ಪಟ್ಟ
ಪೂಜಾ-ವಿಚ್ಛೇದನೇ
ಪೂಜಿತಃಪೂಜಿಸಲ್ಪಟ್ಟ
ಪೂಜಿ-ತತ್ವೇನ
ಪೂಜಿತಸ್ತತ್ರ
ಪೂಜಿಸ-ತಕ್ಕದ್ದು
ಪೂಜಿಸ-ಬಹುದು
ಪೂಜಿಸ-ಬೇಕು
ಪೂಜಿ-ಸಲು
ಪೂಜಿಸಲ್ಪಟ್ಟ
ಪೂಜಿಸಲ್ಪಟ್ಟಿ-ರುವು-ದ-ರಿಂದ
ಪೂಜಿಸಲ್ಪಡುತ್ತಿವೆ
ಪೂಜಿ-ಸಿದ-ರೆಂಬುದು
ಪೂಜಿಸುವ
ಪೂಜಿ-ಸು-ವುದು
ಪೂಜೆ
ಪೂಜೆ-ಯನ್ನು
ಪೂಜೆಯು
ಪೂಜೆಯೇ
ಪೂಜ್ಯ
ಪೂಜ್ಯ-ಪಾದ-ಪದ್ಮಾರಾಧಕ
ಪೂಯೇಯ
ಪೂಯೇಯೇತ್ಪನುವ್ರಜತಿ
ಪೂಯೇಯೇತ್ಯಂಫ್ರಿರೇಣುಭಿಃ
ಪೂರೈಸುವ
ಪೂರ್ಣ
ಪೂರ್ಣಂ
ಪೂರ್ಣ-ಗುಣಾಃ
ಪೂರ್ಣ-ಗೊಳಿ-ಸುತ್ತಿದ್ದರು
ಪೂರ್ಣ-ನಾದ
ಪೂರ್ಣ-ನಾದ-ವನು
ಪೂರ್ಣ-ಪರೋಕ್ಷ
ಪೂರ್ಣಪ್ರಜ್ಞಸ್ಟೃತೀಯಸ್ತು
ಪೂರ್ಣ-ರಾದ
ಪೂರ್ಣ-ರಾದು-ದ-ರಿಂದ
ಪೂರ್ಣ-ವಾಗಿ
ಪೂರ್ಣ-ವಾಗಿ-ದೆಯೆಂದು
ಪೂರ್ಣ-ವಾದ
ಪೂರ್ಣ-ವಾ-ದರೂ
ಪೂರ್ಣವೇ
ಪೂರ್ಣಸ್ವ-ರೂಪ-ತಃಸಮಸ್ತ-ದೋಷ-ಗಳಿಂದ
ಪೂರ್ವಂ
ಪೂರ್ವಂಪ್ರಳಯ-ಕಾಲ-ದಲ್ಲಿ
ಪೂರ್ವಕಂ
ಪೂರ್ವದ
ಪೂರ್ವ-ದಲ್ಲಿ
ಪೂರ್ವಮ-ಯನಂ
ಪೂರ್ವ-ವತ್
ಪೂರ್ವ-ವತ್ಣ
ಪೂರ್ವ-ವತ್ಹಿಂದಿ-ನಂತೆಯೇ
ಪೂರ್ವವ-ದೇವ
ಪೂರ್ವವ-ದೇವಾವಗಂತವ್ಯಃಣ
ಪೂರ್ವ-ವದ್ವಿಗ್ರಹಃ
ಪೂರ್ವವನ್ನರ-ಶಬ್ದೋ
ಪೂರ್ವಾಣಿ
ಪೃಕ್ಷೌದ-ರಾದಿ-ವತ್
ಪೃಥಿವೀ
ಪೃಥಿವೀ-ಮಿಮಾಂ
ಪೃಥ್ವಿ-ಯನ್ನು
ಪೃಥ್ವಿಯು
ಪೃಥ್ವೀಂಈ
ಪೈಂಗಿ
ಪೋಣಿ-ಸಿದ
ಪೋಷ-ಕನು
ಪೌತ್ರಾ-ಯಣ
ಪೌರುಷಂ
ಪ್ರಕಟಗೊಳ್ಳುತ್ತಾನೆ-ಹೀಗೆಂದು
ಪ್ರಕಟ-ನಾಗಿ-ರುವಿ
ಪ್ರಕಟ-ನಾದ
ಪ್ರಕರ-ಣ-ದಲ್ಲಿ
ಪ್ರಕಾರ
ಪ್ರಕಾರದ
ಪ್ರಕಾರ-ವಾಗಿ
ಪ್ರಕಾರ-ವಾಗಿ-ರುವುವು
ಪ್ರಕಾಶಂತೇ
ಪ್ರಕಾಶ-ಕರ
ಪ್ರಕಾಶ-ಕರೆಲ್ಲ-ರನ್ನೂ
ಪ್ರಕಾಶ-ನಗೊಳಿಸಲ್ಪಟ್ಟಿದ್ದು
ಪ್ರಕಾಶ-ನದ
ಪ್ರಕಾಶಿಸಲ್ಪಟ್ಟ
ಪ್ರಕೃತ
ಪ್ರಕೃತಿ
ಪ್ರಕೃತಿಃ
ಪ್ರಕೃತಿ-ಬಂಧ
ಪ್ರಕೃತಿ-ಬಂಧಃಚೇತನರ
ಪ್ರಕೃತಿ-ಬಂಧ-ನ-ವನ್ನು
ಪ್ರಕೃತಿಯು
ಪ್ರಕೃತಿ-ರೇವ
ಪ್ರಕೃತೇಃ
ಪ್ರಣಮ್ಯ
ಪ್ರಣಯಾಮೃತ-ಪೂರ್ಣ-ಮಾನಸಃ
ಪ್ರತಿ
ಪ್ರತಿಕ್ರಿಯೆಯು
ಪ್ರತಿ-ಜಗಾಮ
ಪ್ರತಿ-ಜಗಾಮ-ಹಿಂತಿರುಗಿ-ದರು
ಪ್ರತಿ-ಜನ್ಯ-ದಲ್ಲಿಯೂ
ಪ್ರತಿಜ್ಞೆ-ಗಳನ್ನು
ಪ್ರತಿ-ದಾತುಂ
ಪ್ರತಿ-ಪತ್ತವ್ಯಂಅತ
ಪ್ರತಿ-ಪತ್ತವ್ಯಂಗಮನಿಸ-ತಕ್ಕದ್ದು
ಪ್ರತಿ-ಪತ್ತವ್ಯಂತಿಳಿಯ-ಬೇಕು
ಪ್ರತಿ-ಪತ್ತವ್ಯಂರೇಫಾಪೇಕ್ಷಯಾರ
ಪ್ರತಿ-ಪತ್ತವ್ಯಃನಾರಾ-ಯಣ
ಪ್ರತಿ-ಪಾದನೆ
ಪ್ರತಿ-ಪಾದ-ಯಿತುಂ
ಪ್ರತಿ-ಪಾದಿ-ಸುತ್ತವೆ
ಪ್ರತಿ-ಪಾದಿಸುವ
ಪ್ರತಿ-ಪಾದ್ಯ
ಪ್ರತಿ-ಪಾದ್ಯ-ತಯಾ
ಪ್ರತಿ-ಪಾದ್ಯ-ತಯಾ-ತದಯ-ನತ್ವಾತ್ಈ
ಪ್ರತಿ-ಪಾದ್ಯ-ನಾ-ದ-ವನು
ಪ್ರತಿ-ಬಿಂಬ
ಪ್ರತಿ-ಬಿಂಬಕ್ರಿಯಾಯಾಃ
ಪ್ರತಿ-ಬಿಂಬನು
ಪ್ರತಿ-ಬಿಂಬನೂ
ಪ್ರತಿ-ಬಿಂಬರ
ಪ್ರತಿ-ಭಾಗ-ವತ-ದಲ್ಲಿ
ಪ್ರತಿ-ಭಾತಿ
ಪ್ರತಿ-ಮಾಂತರ್ಗತ-ನಾದ
ಪ್ರತಿ-ಮಾಃಶ್ರೀ-ರಂಗ-ದಲ್ಲಿ-ರುವ
ಪ್ರತಿ-ಮಾ-ಗಳಿಗೆ
ಪ್ರತಿ-ಮಾದಿ-ಗಳ
ಪ್ರತಿ-ಮಾ-ರೂಪ-ದಲ್ಲಿ-ರುವ
ಪ್ರತಿ-ಮೆ-ಗಳನ್ನು
ಪ್ರತಿ-ಮೆ-ಗಳಲ್ಲಿ-ರುವ
ಪ್ರತಿ-ಮೆ-ಗಳು
ಪ್ರತಿ-ಮೆ-ಯನ್ನು
ಪ್ರತಿ-ಮೆ-ಯಲ್ಲಿ
ಪ್ರತಿ-ಮೆ-ಯಲ್ಲಿ-ರುವ
ಪ್ರತಿ-ಮೆಯೇ
ಪ್ರತಿ-ಷೇಧೇ
ಪ್ರತಿ-ಷೇಧೋ
ಪ್ರತಿಷ್ಠಾಪನೆ
ಪ್ರತಿಷ್ಠಾಪೇಕ್ಷಾಸ್ತಿ
ಪ್ರತಿಷ್ಠಾಮಾಡಬೇಕಾದ
ಪ್ರತಿಷ್ಠಾಸ್ವಾಮಿ
ಪ್ರತಿಷ್ಠಿತಂಇಡಲ್ಪಟ್ಟಿದೆ-ಹೀಗೆಂದು
ಪ್ರತಿಷ್ಠಿತಮ್
ಪ್ರತಿಷ್ಠಿತೌ
ಪ್ರತಿಷ್ಠೆ
ಪ್ರತಿಷ್ಠೆ-ಮಾಡಿದ
ಪ್ರತಿಷ್ಠೆ-ಮಾಡಿ-ದ-ವರ
ಪ್ರತೀಕ-ಗಳಲ್ಲಿ
ಪ್ರತೀಕ-ದಲ್ಲಿ-ಯಾದರೋ
ಪ್ರತೀಕ-ದಲ್ಲಿಯೂ
ಪ್ರತ್ಯಕ್ಷಂ
ಪ್ರತ್ಯಕ್ಷ-ವಾಗಿ
ಪ್ರತ್ಯಕ್ಷ-ವೈಕುಂಠ
ಪ್ರತ್ಯಯಂ
ಪ್ರತ್ಯಯ-ಗಳು
ಪ್ರತ್ಯ-ಯನ್ನು
ಪ್ರತ್ಯಯ-ವನ್ನು
ಪ್ರತ್ಯ-ಯವು
ಪ್ರತ್ಯ-ಯಶ್ಚ
ಪ್ರತ್ಯ-ಯಶ್ಚ-ಕರ-ಣೇಲ್ಯುಟ್
ಪ್ರತ್ಯಯೇ
ಪ್ರತ್ಯುಪ-ಕಾರ-ಕ-ವಾದ
ಪ್ರತ್ಯೇಕಂ
ಪ್ರಥಮಜ್ಞಾನೀ
ಪ್ರಥಮಜ್ಞಾನೀಜ್ಞಾನೋತ್ಪಾದ-ಕ-ನಾದ
ಪ್ರಥಮೋ
ಪ್ರಥಾ-ನಾನು
ಪ್ರದರ್ಶಿಸಿ
ಪ್ರದ-ವಾದ
ಪ್ರದಾನವ-ದೇವ
ಪ್ರದಾನವ-ದೇವ-ಅನುಗ್ರಹ-ಮಾಡುವ
ಪ್ರದಾನಾಯ
ಪ್ರದಾನಾಯ-ಮುಕ್ತಿ-ಯನ್ನು
ಪ್ರದೇಶ
ಪ್ರದೇಶ-ಗಳಲ್ಲಿ-ರುವ
ಪ್ರದೇಶ-ದಲ್ಲಿ
ಪ್ರದೇಶವು
ಪ್ರದೇಶವೋ
ಪ್ರದ್ಯುಮ್ನ
ಪ್ರದ್ಯುಮ್ನ-ರೂಪ-ಗಳಿಂದ
ಪ್ರದ್ಯುಮ್ನಸ್ತತ್ರ
ಪ್ರಧಾನ
ಪ್ರಧಾನ-ಪರಮವ್ಯೋಮ್ನಃಪ್ರಧಾನ-ವೆಂದು
ಪ್ರಧಾನ-ಪರಮವ್ಯೋಮ್ನೋರಂತರಾ
ಪ್ರಧಾನ-ರೂಪ-ಗಳಿಂದ
ಪ್ರಧಾನ-ವಾಯು-ದೇವ-ರಿಗೆ
ಪ್ರಪತಂತಿ
ಪ್ರಪದ
ಪ್ರಪದ-ಪ-ದಾಗ್ರ
ಪ್ರಪದಯೋಃ
ಪ್ರಪದಾಭ್ಯಾಂ
ಪ್ರಪದೋಪಲಕ್ಷಣ-ಮೇ-ತತ್ಇಲ್ಲಿ-ರುವ
ಪ್ರಬೋಧಕಃ
ಪ್ರಭವೋ
ಪ್ರಭಾವ
ಪ್ರಭಾವ-ದಿಂದ
ಪ್ರಭಾವ-ಬಲ-ಪೌರುಷಬುದ್ದಿಯೋಗಾಃ
ಪ್ರಭು
ಪ್ರಭುಃ
ಪ್ರಭು-ವಾತ್ಸರ್ವಂ
ಪ್ರಭು-ವಾದ
ಪ್ರಮಾಣ
ಪ್ರಮಾಣಂ
ಪ್ರಮಾ-ಣಕ್ಕೆ
ಪ್ರಮಾಣ-ಗಳಂತೆ
ಪ್ರಮಾಣ-ಗಳನ್ನೂ
ಪ್ರಮಾಣ-ಗಳಲ್ಲಿ
ಪ್ರಮಾಣ-ಗಳಲ್ಲಿ-ರುವಂತೆ
ಪ್ರಮಾಣ-ಗಳಿಂದ
ಪ್ರಮಾಣ-ಗಳಿಂದಲೂ
ಪ್ರಮಾಣ-ಗಳು
ಪ್ರಮಾಣ-ಗಳುಂಟು
ಪ್ರಮಾಣ-ದಂತೆ
ಪ್ರಮಾಣ-ದಲ್ಲಿ
ಪ್ರಮಾಣ-ವನ್ನು
ಪ್ರಮಾಣ-ವಾಕ್ಯ-ಗಳ
ಪ್ರಮಾಣ-ವಿದೆ
ಪ್ರಮಾಣವು
ಪ್ರಮಾಣೇ
ಪ್ರಮಾದ-ದಿಂದಾಗಲೀ
ಪ್ರಮಾ-ದಾತ್ಪ್ರಣ-ಯೇನ
ಪ್ರಮಾದ್ಯಥ
ಪ್ರಮುಖ
ಪ್ರಮೇಯ
ಪ್ರಮೇಯ-ವನ್ನು
ಪ್ರಮೇಯ-ವನ್ನೇ
ಪ್ರಮೇ-ಯವು
ಪ್ರಮೋದತೇ
ಪ್ರಯಾಣ-ಮಾಡಲು
ಪ್ರಯಾಣ-ಮಾಡುತ್ತಿ-ರುವಾಗ
ಪ್ರಯಾತಿ
ಪ್ರಯೋಗ
ಪ್ರಯೋಗ-ದಲ್ಲಿ
ಪ್ರಯೋಗ-ದಿಂದ
ಪ್ರಯೋಗಿಸಲ್ಪಟ್ಟಿದೆ
ಪ್ರಯೋಗಿಸಲ್ಪಡ-ತಕ್ಕದ್ದಲ್ಲ
ಪ್ರಯೋಗಿಸಲ್ಪಡುತ್ತದೆ
ಪ್ರಲಯ
ಪ್ರಲಯಃ
ಪ್ರಲಯ-ಕಾಲ-ದಲ್ಲಿ
ಪ್ರಲಯ-ಕಾಲೀನ
ಪ್ರಲಯ-ಕಾಲೇಪಿ
ಪ್ರಲಯ-ಕಾಲೇ-ಽಪಿ
ಪ್ರಲಯೇ
ಪ್ರಳಯ
ಪ್ರಳಯ-ಕಾರ-ಕನು
ಪ್ರಳಯ-ಕಾರ-ಕನೂ
ಪ್ರಳಯ-ಕಾಲ
ಪ್ರಳಯ-ಕಾಲ-ದಲ್ಲಿಯೂ
ಪ್ರಳಯ-ದಲ್ಲಿ
ಪ್ರಳಯೋದಕ-ದಲ್ಲಿ
ಪ್ರವ-ಚನ
ಪ್ರವರ್ತಂತೇ
ಪ್ರವರ್ತಕಃ
ಪ್ರವರ್ತಕಃಹಾಗೆಯೇ
ಪ್ರವರ್ತಕ-ತಯಾಈ
ಪ್ರವರ್ತಕ-ನಾಗಿ-ರುವು-ದ-ರಿಂದ
ಪ್ರವರ್ತ-ಕನೂ
ಪ್ರವರ್ತತೇ
ಪ್ರವರ್ತನೆ-ಮಾಡುವುದರಿಂದ
ಪ್ರವರ್ತಿ-ಸುತ್ತವೆ
ಪ್ರವರ್ಧ-ಮಾನಂ
ಪ್ರವಹಿ-ಸುತ್ತದೆ
ಪ್ರವಾಹ-ರೂಪ-ದಿಂದ
ಪ್ರವಾಹ-ರೂಪ-ವಾ-ದುದು
ಪ್ರವಾಹವು
ಪ್ರವಾಹೋ-ಯಮೇವಮೇ-ವಾದಿ-ಕಾಲತಃ
ಪ್ರವಿಶಂತಿ
ಪ್ರವಿಶ್ಯಾನುಗ್ರಹಾದ್ದರೇಃ
ಪ್ರವಿಷ್ಟಾಯಾ
ಪ್ರವಿಷ್ಟೌ
ಪ್ರವಿಷ್ಟೌಪ್ರವೇಶ-ಮಾಡಿ-ರುವ
ಪ್ರವೃತ್ತಿಕೃತ್ಪ್ರ-ವರ್ತಕ-ನಾ-ಗಿದ್ದಾನೆ
ಪ್ರವೃತ್ತಿಕೃದೇಕ
ಪ್ರವೇಶ
ಪ್ರವೇಶಕ್ಕೆ
ಪ್ರವೇಶನೇ
ಪ್ರವೇಶ-ಮಾಡಿ
ಪ್ರವೇಶ-ಮಾಡಿದ
ಪ್ರವೇಶ-ಮಾಡಿ-ರುವ
ಪ್ರವೇಶ-ಮಾಡುತ್ತದೆ
ಪ್ರವೇಶ-ಮಾಡುತ್ತಾನೆಯೋ
ಪ್ರವೇಶಸ್ಥಾನ
ಪ್ರವೇಶಸ್ಥಾನಂಪ್ರವೇಶ-ಮಾಡುವ
ಪ್ರವೇಶಾಧಿ-ಕರಣಂ
ಪ್ರವೇಶಾಧಿ-ಕರ-ಣಂಅ-ಯನಂಅರ್ಹತೆ-ಯನ್ನು
ಪ್ರವೇಶಿ-ಸಲು
ಪ್ರವೇಶಿಸಿ
ಪ್ರವೇಶಿಸಿ-ದನು
ಪ್ರವೇಶಿ-ಸುತ್ತಾನೆ
ಪ್ರವೇಶಿಸುತ್ತಾರೆ
ಪ್ರವೇಶಿಸುತ್ತಾಳೆ
ಪ್ರವೇಶಿ-ಸುತ್ತಿ-ರಲು
ಪ್ರವೇಶಿಸುವಂತೆ
ಪ್ರವೇಶೇ
ಪ್ರವೇಶೇ-ಜೀವ-ರು-ಗಳನ್ನು
ಪ್ರವೇಷ್ಟುಂ
ಪ್ರಶ್ನ-ಉಪನಿಷತ್ತಿ-ನಲ್ಲಿ
ಪ್ರಶ್ನೆ
ಪ್ರಶ್ನೋಪನಿಷದಿ-ಷಟ್
ಪ್ರಸನ್ನಃ
ಪ್ರಸನ್ನತೆ-ಯನ್ನು
ಪ್ರಸನ್ನತೆ-ಯಿಂದ
ಪ್ರಸನ್ನ-ನಾದ
ಪ್ರಸನ್ನ-ನಾದರೆ
ಪ್ರಸನ್ನಾತ್
ಪ್ರಸನ್ನಾತ್ಪರಮಾತ್ಮನ
ಪ್ರಸನ್ನೇ
ಪ್ರಸನ್ನೋ
ಪ್ರಸಭಂ
ಪ್ರಸಾದಕ್ಕೆ
ಪ್ರಸಾದ-ದಿಂದಲೇ
ಪ್ರಸಿದ್ದ
ಪ್ರಸಿದ್ದಂ
ಪ್ರಸಿದ್ದಂಮೃತ್ಯು-ವಿಗೆ
ಪ್ರಸಿದ್ದ-ರಾದ
ಪ್ರಸಿದ್ದ-ವಾಗಿದೆ
ಪ್ರಸಿದ್ದ-ವಾಗಿ-ರುವ
ಪ್ರಸಿದ್ದ-ವಿದೆ
ಪ್ರಸಿದ್ದಾಃಎಂಬುದು
ಪ್ರಸಿದ್ಧ
ಪ್ರಸಿದ್ಧಂ
ಪ್ರಸಿದ್ಧ-ನಾ-ಗಿದ್ದಾನೆ
ಪ್ರಸಿದ್ಧ-ರಾಗುತ್ತಾರೆ
ಪ್ರಸಿದ್ಧ-ಳಾದ-ಳೆಂಬ
ಪ್ರಸಿದ್ಧ-ವಾಗಿದೆ
ಪ್ರಸಿದ್ಧ-ವಾದ
ಪ್ರಸಿದ್ಧ-ವಾ-ದುದು
ಪ್ರಸಿದ್ಧವು
ಪ್ರಹ್ಲಾದ
ಪ್ರಹ್ಲಾದಃ
ಪ್ರಹ್ಲಾದನ
ಪ್ರಹ್ಲಾದ-ನಿಗೆ
ಪ್ರಹ್ಲಾದನು
ಪ್ರಹ್ಲಾದೋ
ಪ್ರಾಕೃತ-ದೇಹವು
ಪ್ರಾಕೃತೇಂದ್ರಿಯ-ಗಳಿಂದ
ಪ್ರಾಗುಕ್ತ-ಬಂಧಾತ್
ಪ್ರಾಜ್ಞ
ಪ್ರಾಣ
ಪ್ರಾಣ-ನನ್ನು
ಪ್ರಾಣಪ್ರತಿಷ್ಠೆ-ಗೋಸ್ಕರ
ಪ್ರಾಣ-ಮಸೃಜತ-ಅವನು
ಪ್ರಾಣ-ಮಾತುಃಮುಖ್ಯಪ್ರಾಣ-ದೇವರ
ಪ್ರಾಣ-ವಿ-ರುವುದಕ್ಕೆ
ಪ್ರಾಣವು
ಪ್ರಾಣ-ಸಖಾ
ಪ್ರಾಣಸ್ಯ
ಪ್ರಾಣಸ್ಯ-ಮುಖ್ಯಪ್ರಾ-ಣನ
ಪ್ರಾಣಾಧಿಪೇ
ಪ್ರಾಣಿ-ಗಳಿಗೆ
ಪ್ರಾಣೋ
ಪ್ರಾದುರ್ಭಾವ-ವಿಪರ್ಯಾಸಃ
ಪ್ರಾಪ
ಪ್ರಾಪಣ
ಪ್ರಾಪ-ಣಾರ್ಥ
ಪ್ರಾಪ-ಣಾರ್ಥಾನಾಂ
ಪ್ರಾಪಣೆ
ಪ್ರಾಪದ್ಯತ
ಪ್ರಾಪದ್ಯತ-ಹೊಂದಿ-ದನು
ಪ್ರಾಪ-ಪಡೆ-ದರು
ಪ್ರಾಪ-ಯತಿ
ಪ್ರಾಪ-ಹೊಂದಿ-ದನು
ಪ್ರಾಪ್ತ-ನಾದ
ಪ್ರಾಪ್ತ-ವಾಗುತ್ತದೆ
ಪ್ರಾಪ್ತ-ವಾಗುವು-ದ-ರಿಂದ
ಪ್ರಾಪ್ತಿ
ಪ್ರಾಪ್ತಿ-ಯಾಗುವಂತೆ
ಪ್ರಾಪ್ತೇ
ಪ್ರಾಪ್ಯ
ಪ್ರಾಪ್ಯತೇ
ಪ್ರಾಪ್ಯತ್ವತೋ-ಽಥವೇತಿಜ್ಞಾನವೇ
ಪ್ರಾಪ್ಯತ್ವಾತ್
ಪ್ರಾಪ್ಯ-ನಾದು-ದ-ರಿಂದ
ಪ್ರಾಮಾಣಿಕವೂ
ಪ್ರಾಯಃ
ಪ್ರಾರಂಭ
ಪ್ರಾರಂಭಿಸಿ-ರು-ವುದು
ಪ್ರಾರಬ್ದ
ಪ್ರಾರಬ್ದಂ
ಪ್ರಾರಬ್ದ-ಪಾಪ
ಪ್ರಾರಬ್ದ-ವೆಂದರೆ
ಪ್ರಾರಬ್ದ-ವೆಂಬ
ಪ್ರಾರಬ್ಧ
ಪ್ರಾರಬ್ಧ-ಕರ್ಮ
ಪ್ರಾರಬ್ಧ-ಕರ್ಮ-ಗಳು
ಪ್ರಾರಬ್ಧ-ಕರ್ಮ-ಣೋಽನ್ಯಸ್ಯ
ಪ್ರಾರಬ್ಧ-ಕರ್ಮ-ಭೋಗಿ-ಸಿದ
ಪ್ರಾರಬ್ಧ-ಕರ್ಮ-ವನ್ನು
ಪ್ರಾರಬ್ಧ-ಕರ್ಮ-ಶೇಷಂ
ಪ್ರಾರಬ್ಧ-ಕರ್ಮ-ಶೇಷ-ವನ್ನೂ
ಪ್ರಾರಬ್ಧ-ಕಾನಿ
ಪ್ರಾರಬ್ಧ-ಪುಣ್ಯಾನಿ
ಪ್ರಾರಬ್ಧ-ವಾಗಿ
ಪ್ರಿಯ-ವಾದ
ಪ್ರೀತಿ-ಗಿಂತ
ಪ್ರೀತಿ-ಯಿಂದ
ಪ್ರೀತ್ಯಾ-ದರ-ಗಳನ್ನು
ಪ್ರೀತ್ಯಾಸ್ಪದ-ವಾದ
ಪ್ರೇಮ
ಪ್ರೇಮ-ದಿಂದ
ಪ್ರೇಮಪ್ರವಾಹವೇ
ಪ್ರೇಮಪ್ರವಾಹೋ
ಪ್ರೇಮಾಸ್ಪದ
ಪ್ರೇಮಾಸ್ಪದ-ತಯಾ
ಪ್ರೇರಕ
ಪ್ರೇರಕತ್ವ
ಪ್ರೇರಕ-ನಾಗಿ
ಪ್ರೇರಕ-ನಾಗಿ-ರುವ
ಪ್ರೇರಣಾಯಾ-ವಾಸಸ್ಥಾನಂ
ಪ್ರೇರಣೆ-ಮಾಡಲು
ಪ್ರೇರಣೆ-ಮಾಡುವ
ಪ್ರೋಕಃಹೇಳಲ್ಪಟ್ಟಿದ್ದಾನೆ
ಪ್ರೋಕ್ಕಾ
ಪ್ರೋಕ್ತಃ
ಪ್ರೋಕ್ತಾಃ
ಪ್ರೋಕ್ತಾಃನಾರಾ
ಪ್ರೋಕ್ತೋ
ಪ್ರೋತಂ
ಪ್ರೌಢ
ಪ್ಲೇಯ
ಫಲ
ಫಲಂ
ಫಲ-ಕಾರಿಯಾಗು-ವುದಿಲ್ಲ
ಫಲ-ಕೊಡಲು
ಫಲ-ಕೊಡುತ್ತಾನೆ
ಫಲ-ಕೊಡುವ
ಫಲ-ಕೊಡುವಾಗ
ಫಲ-ಕೊಡುವುದಕ್ಕೆ
ಫಲ-ಕೊಡುವು-ದ-ರಿಂದ
ಫಲ-ಗಳನ್ನು
ಫಲ-ಗಳು
ಫಲದಂ
ಫಲ-ದಲ್ಲಿ
ಫಲ-ದಾತೃತ್ವಾತ್
ಫಲದಾನೇ
ಫಲದಾನೇ-ಜೀವ-ರು-ಗಳಿಗೆ
ಫಲದಾಯಕ-ವಾಗುತ್ತವೆ
ಫಲಪ್ರದಃ
ಫಲ-ಭೋಕ್ತೃ-ವಾಗಿ
ಫಲಮಿಚ್ಛಯೈವ
ಫಲ-ವನ್ನು
ಫಲವು
ಫಲ-ಸಿದ್ಧಿಗೂ
ಫಲಸ್ಯ
ಫಲಾಧ್ಯಾಯ
ಫಲಾಪೇಕ್ಷೆ-ಯಿಂದ
ಫಲಿತಾರ್ಥ
ಫ್ರಾಣಮೇವ
ಬಂದ
ಬಂದ-ನಂತರ
ಬಂದರೂ
ಬಂದಿ-ರುವ
ಬಂದು
ಬಂಧ
ಬಂಧಕೋ
ಬಂಧ-ನದ
ಬಂಧ-ನ-ದಿಂದ
ಬಂಧ-ನವು
ಬಂಧ-ನಾಶವು
ಬಂಧ-ಮೋಕ್ಷ-ಗಳು
ಬಂಧ-ಮೋಕ್ಷ-ಗಳೂ
ಬಂಧ-ಮೋಕ್ಷೌ
ಬಂಧ-ವಿಪರ್ಯಯೌ
ಬಂಧಾತ್ಹಿಂದೆ
ಬಂಧಿ-ಸಲ್ಪಟ್ಟಿ-ರು-ವುದು
ಬಂಧಿಸಿ
ಬಂಧಿ-ಸುವ-ವನೂ
ಬಂಧು
ಬಗೆ
ಬಗೆ-ಗಳಲ್ಲಿ
ಬಗೆ-ಬಗೆ-ಯಾಗಿ
ಬಗೆಯ
ಬಗೆ-ಯಿಂದಲೂ
ಬಗ್ಗೆ
ಬಗ್ಯೆನ್ನು
ಬಟ್ಟೆ
ಬದರಿ-ಕಾಶ್ರಮಕ್ಕೆ
ಬದಲಾಗಿ
ಬದು-ಕುತ್ತಿವೆಯೋ
ಬಧ್ವಾ
ಬಭೂವ
ಬರ-ದಂತೆ
ಬರದೇ
ಬರ-ಬಹು-ದಾದ
ಬರ-ಬಹು-ದಾದಂತಹ
ಬರ-ಬೇಕು
ಬರ-ಲಾಗಿ
ಬರ-ಲಾರದ
ಬರಲು
ಬರೀ
ಬರುತ್ತದೆ
ಬರುತ್ತಾರೆ-ಹೀಗೆಂದು
ಬರುತ್ತಿ-ರುವಾಗ
ಬರುವ
ಬರುವ-ವ-ನನ್ನು
ಬರು-ವುದಿಲ್ಲ
ಬರು-ವುದಿಲ್ಲ-ವಾದ್ದ-ರಿಂದ
ಬರು-ವುದು
ಬರುವುದೂ
ಬರುವುದೇ
ಬರೆದಿ-ರುತ್ತಾರೆ
ಬರೆಯುವಾಗ
ಬರೋಣವು
ಬರೋಣ-ವೆಂಬುದೇ
ಬಲ
ಬಲಂ
ಬಲ-ಆನಂದ
ಬಲ-ಮಿತಿ
ಬಲ-ಮೇಧಾ-ಧೃತಿಸ್ಥೈರ್ಯಧ್ಯಾನ-ವೈರಾಗ್ಯ-ಮಾನ-ವಾನ್
ಬಲ-ರೂಪತ್ವಾತ್
ಬಲ-ರೂಪ-ನಾದ
ಬಲ-ವಾನ್
ಬಲ-ವುಳ್ಳ-ವನು
ಬಲ-ಸೂಚಕ-ವಾದ
ಬಲಾದಿ
ಬಲಾನಂದ
ಬಲಾನಂದ-ರೂಪತ್ವಾತ್
ಬಲಾನಂದ-ರೂಪ-ನಾದು-ದ-ರಿಂದಲೂ
ಬಲಿಚಕ್ರ-ವರ್ತಿಯು
ಬಲಿಷ್ಠ-ವಾ-ದುದು
ಬಲ್ಲ
ಬಳಿ
ಬಳಿಗೆ
ಬಹಿರಾ-ವರಣ
ಬಹಿರಾ-ವರ-ಣ-ಗಳಲ್ಲಿ
ಬಹಿರಾ-ವರ-ಣ-ದಲ್ಲಿದ್ದ
ಬಹಿರಾವ್ರಜೇತ್
ಬಹಿರಾಶ್ರ-ಯತ್ವಾನ್ನಾರಾ-ಯ-ಣಃಆ
ಬಹಿಷ್ಠಸ್ತ್ರೀ
ಬಹಿಷ್ಠೆ-ಯಾದ
ಬಹಿಸ್ತಸ್ಮಾ-ದೇವಂ
ಬಹು
ಬಹುದು
ಬಹುಧಾ
ಬಹು-ನೋಕ್ತೇನ
ಬಹು-ಮಾನ-ಪುರಸ್ಪರಃ
ಬಹುಲ
ಬಹು-ಲಗ್ರಹಣಾತ್
ಬಹು-ಲ-ಮಿತಿ
ಬಹು-ವತ್
ಬಾ
ಬಾಧಾದಿತೊಂದರೆಯೇ
ಬಾರದಂತಹ
ಬಾಲಕ-ರಿಗೆ
ಬಾಲ-ಕರೆ
ಬಾಹ್ಯ
ಬಾಹ್ಯ-ಜಲಧಾರಾ
ಬಿ
ಬಿಂಬ
ಬಿಂಬಕ್ರಿಯಾನುಸ್ಯೂತತಾ
ಬಿಂಬ-ನಾದ
ಬಿಂಬನು
ಬಿಟ್ಟು
ಬಿಟ್ಟು-ಹೋಗುತ್ತಿ-ರಲು
ಬಿಡದೇ
ಬಿಡಲ್ಪಟ್ಟ
ಬಿಡಿಸಿ
ಬಿಡಿಸಿ-ದುದು
ಬಿಡಿ-ಸುವ-ವನು
ಬಿಡುಗಡೆ
ಬಿಡುಗಡೆ-ಗಳು
ಬಿಡುವುದೂ
ಬಿದರಹಳ್ಳಿ
ಬಿದಿಗೆಯಂದು
ಬಿದ್ದವ-ರಿಗೆ
ಬಿದ್ದು
ಬಿಭರ್ತಿ
ಬಿಭರ್ತಿ-ಧರಿ-ಸುತ್ತಾನೆ
ಬಿಭರ್ತೀದಂ
ಬಿಭರ್ತ್ಯಂಡಂ
ಬೀಜಂ
ಬೀಜಮವ್ಯಯಂ
ಬೀಳದೇ
ಬೀಳುತ್ತದೆ
ಬೀಳುತ್ತಾರೆ
ಬೀಳು-ವರು
ಬೀಳುವುದೂ
ಬುದ್ಧಿಶಕ್ತಾನು-ಸಾರ-ವಾಗಿ
ಬುದ್ಧ್ಯಸ್ಥವಾಗಲು
ಬೂದಿ-ಯಲ್ಲಿ
ಬೃಹದಾರಣ್ಯ-ಭಾಷ್ಯ
ಬೃಹದ್ಭಾಷ್ಯ
ಬೃಹದ್ಭಾಷ್ಯ-ದಲ್ಲಿ
ಬೃಹದ್ಭಾಷ್ಯ-ದಲ್ಲಿಯೂ
ಬೃಹದ್ಭಾಷ್ಯೇ
ಬೃಹಸ್ಪತಿ
ಬೃಹಸ್ಪತಿಯೇ
ಬೆಂಕಿ
ಬೆಂಕಿ-ಯಂತೆ
ಬೆನ್ನನ್ನು
ಬೆನ್ನಿನ
ಬೆವರಿನ
ಬೆವರಿನಿಂದ
ಬೆವರು
ಬೇಕಾ-ದರೂ
ಬೇಕು
ಬೇಡ
ಬೇಡಿಕೆಯಿದ್ದ
ಬೇರೆ
ಬೇರೆ-ಯವರ
ಬೇರೆ-ಯಾದದ್ದು
ಬೇರೆ-ಯಾ-ದುದು
ಬೇರೊಂದು
ಬೇರೊಬ್ಬರು
ಬೋಧಿಸುವ
ಬ್ರಹ
ಬ್ರಹೃದ್ವಿಷಃ
ಬ್ರಹೇಂದ್ರಾದ್ಯಭಿವಂದಿತಮ್
ಬ್ರಹೇಮಂ
ಬ್ರಹ್ಮ
ಬ್ರಹ್ಮಜ್ಞಾನಿನೋ
ಬ್ರಹ್ಮಣಾ
ಬ್ರಹ್ಮ-ಣಾ-ಚತುರ್ಮುಖ
ಬ್ರಹ್ಮ-ಣೋಪಿ
ಬ್ರಹ್ಮ-ದೃಷ್ಟಿರ್ಭ-ವತಿ
ಬ್ರಹ್ಮ-ದೇವರ
ಬ್ರಹ್ಮ-ದೇವ-ರನ್ನು
ಬ್ರಹ್ಮ-ದೇವ-ರಿಂದ
ಬ್ರಹ್ಮ-ದೇವ-ರಿಗೆ
ಬ್ರಹ್ಮ-ದೇವರು
ಬ್ರಹ್ಮ-ದೇವರೂ
ಬ್ರಹ್ಮ-ದೇವರೇ
ಬ್ರಹ್ಮ-ದೇವ-ರೊಡನೆ
ಬ್ರಹ್ಮನ
ಬ್ರಹ್ಮ-ನಲ್ಲಿ
ಬ್ರಹ್ಮ-ನಿಂದ
ಬ್ರಹ್ಮ-ನಿಗೆ
ಬ್ರಹ್ಮ-ನಿಷ್ಠಂಭಾಷ್ಯ
ಬ್ರಹ್ಮನು
ಬ್ರಹ್ಮನೇ
ಬ್ರಹ್ಮನ್
ಬ್ರಹ್ಮ-ಪರಬ್ರಹ್ಮನು
ಬ್ರಹ್ಮ-ಪುರಿ
ಬ್ರಹ್ಮ-ಪುರೇ
ಬ್ರಹ್ಮ-ಮೀ-ಮಾಂಸಾ
ಬ್ರಹ್ಮ-ರುದ್ರ
ಬ್ರಹ್ಮ-ಶಬ್ದ-ವಾಚ್ಯನು
ಬ್ರಹ್ಮ-ಸೂತ್ರ
ಬ್ರಹ್ಮ-ಸೂತ್ರಕ್ಕೆ
ಬ್ರಹ್ಮ-ಸೂತ್ರ-ಗಳ
ಬ್ರಹ್ಮ-ಸೂತ್ರ-ಗಳಿಗೆ
ಬ್ರಹ್ಮ-ಸೂತ್ರ-ಗಳು
ಬ್ರಹ್ಮಾ
ಬ್ರಹ್ಮಾಂಡ
ಬ್ರಹ್ಮಾಂಡಂ
ಬ್ರಹ್ಮಾಂಡಕ್ಕೆ
ಬ್ರಹ್ಮಾಂಡಕ್ಕೇ
ಬ್ರಹ್ಮಾಂಡ-ಖರ್ಪರ-ವನ್ನು
ಬ್ರಹ್ಮಾಂಡ-ಖರ್ಪರವು
ಬ್ರಹ್ಮಾಂಡ-ಛೇದನದ್ವಾರಾಬ್ರಹ್ಮಾಂಡ-ಖರ್ಪರ-ವನ್ನು
ಬ್ರಹ್ಮಾಂಡದ
ಬ್ರಹ್ಮಾಂಡ-ದಲ್ಲಿ
ಬ್ರಹ್ಮಾಂಡ-ದಲ್ಲಿಯೂ
ಬ್ರಹ್ಮಾಂಡ-ಭೇದ-ನದ್ವಾರಾ-ಽ-ಯನಂ
ಬ್ರಹ್ಮಾಂಡ-ವನ್ನು
ಬ್ರಹ್ಮಾಂಡ-ವನ್ನೂ
ಬ್ರಹ್ಮಾಂಡ-ವಿಗ್ರಹಂ
ಬ್ರಹ್ಮಾಂಡವು
ಬ್ರಹ್ಮಾಂಡವೇ
ಬ್ರಹ್ಮಾಂಡ-ವೋಢಾ-ಕೂರ್ಮ-ರೂಪ-ದಿಂದ
ಬ್ರಹ್ಮಾಂಡಾಂತರಾ-ಗಮನಂ
ಬ್ರಹ್ಮಾಂಡಾಂತರ್ವಾಸೋಪ್ಪು
ಬ್ರಹ್ಮಾ-ದಯಃ
ಬ್ರಹ್ಮಾ-ದಯಃಚತುರ್ಮುಖ
ಬ್ರಹ್ಮಾದಿ
ಬ್ರಹ್ಮಾ-ದಿ-ಗಳಿಂದ
ಬ್ರಹ್ಮಾ-ದಿಭಿಃ
ಬ್ರಹ್ಮಾ-ದಿ-ಭಿಃಬ್ರಹ್ಮ-ದೇವರೇ
ಬ್ರಹ್ಮಾಪಿ
ಬ್ರಹ್ಮಾ-ಮೇಲೆ
ಬ್ರಹ್ಮೈವ
ಬ್ರಾಹ್ಮಣ-ತೀರ್ಥಕ್ಷೇತ್ರ-ಗಳ
ಬ್ರಾಹ್ಮಣದ್ವೇಷಿ-ಗಳು
ಬ್ರಾಹ್ಮಣ-ರಿಂದ
ಬ್ರಾಹ್ಮಣಾದ್ಯಾ
ಭಂಗ-ವಾಗುತ್ತದೆ
ಭಂಗ-ವಾದ-ನಂತರ
ಭಕ್ತಜನ-ರಿಗೆ
ಭಕ್ತನ
ಭಕ್ತ-ನನ್ನು
ಭಕ್ತ-ನಿಗೆ
ಭಕ್ತನು
ಭಕ್ತರ
ಭಕ್ತ-ರನ್ನು
ಭಕ್ತರಿಂದ
ಭಕ್ತರಿಗೂ
ಭಕ್ತರು
ಭಕ್ತಾದಿ
ಭಕ್ತಾದಿ-ಗಳ
ಭಕ್ತಾದಿ-ಗಳನ್ನು
ಭಕ್ತಿ
ಭಕ್ತಿಃ
ಭಕ್ತಿಗೆ
ಭಕ್ತಿಯ
ಭಕ್ತಿ-ಯನ್ನು
ಭಕ್ತಿ-ಯಲ್ಲಿ
ಭಕ್ತಿ-ಯಿಂದ
ಭಕ್ತಿ-ಯಿಂದಲೇ
ಭಕ್ತಿಯು
ಭಕ್ತಿಯೇ
ಭಕ್ತಿ-ರಹಿತ-ನಾಗಿ-ರುವ-ವ-ನಿಗೆ
ಭಕ್ತಿ-ರಿತಿ
ಭಕ್ತಿ-ರಿತ್ಯುಚ್ಯತೇ
ಭಕ್ತಿ-ರೇವ
ಭಕ್ತಿ-ರೇವೈನಂ
ಭಕ್ತಿರ್ಭ-ವತಿ
ಭಕ್ತಿ-ವಶಃ
ಭಕ್ತ್ಯಾ
ಭಗವಂತನ
ಭಗವಂತನು
ಭಗ-ವತಃ
ಭಗ-ವತಃಯಾವ
ಭಗ-ವತಃಷಡ್ಗುಣೈಶ್ವರ್ಯ-ಪೂರ್ಣ-ನಾದ
ಭಗ-ವತಿ
ಭಗ-ವತಿ-ಷಡ್ಗುಣೈಶ್ವರ್ಯ
ಭಗ-ವತ್ಕಾರ್ಯ-ಸಾಧಕಃ
ಭಗ-ವತ್ಪಾದೈಃ
ಭಗ-ವತ್ಪಾದೈಃಶ್ರೀ-ಮದಾಚಾರ್ಯ-ರಿಂದ
ಭಗ-ವತ್ಪಾದೈಃಶ್ರೀ-ಮದಾನಂದ-ತೀರ್ಥ-ರಿಂದ
ಭಗ-ವತ್ಪ್ರ-ಯುಕ್ತಂ
ಭಗ-ವತ್ಸನ್ನಿಧಾನವು
ಭಗ-ವದ್ಗೀತೆಯಲ್ಲಂತೂ
ಭಗ-ವದ್ರೂಪ-ಗಳು
ಭಗ-ವದ್ರೂಪ-ಗಳೂ
ಭಗ-ವದ್ರೂಪ-ಗಳೇ
ಭಗವದ್ವೇಷಿ-ಗಳನ್ನು
ಭಗವದ್ವೇಷಿ-ಗ-ಳಾದ
ಭಗವನ್ನಾರಾ-ಯಃಆ
ಭಗವನ್ನಾರಾ-ಯಣಃ
ಭಗವನ್ನಾರಾ-ಯ-ಣಃಷಡ್ಗುಣೈಶ್ವರ್ಯ-ಪೂರ್ಣ-ನಾದ
ಭಗವನ್ನಾರಾ-ಯ-ಣಃಹೀಗಿ-ರುವು-ದ-ರಿಂದ
ಭಗವನ್ನಾರಾ-ಯ-ಣ-ನ-ರಸಿಂಹ
ಭಗವಾಂಸ್ತತ್ರ
ಭಗ-ವಾನ್
ಭಗ-ವಾನ್ಅಂತಹ
ಭಗ-ವಾನ್ನನು
ಭಗ-ವಾನ್ನಾರಾಯಃ
ಭಗ-ವಾನ್ನಾರಾ-ಯಃಆ-ಲೋಕಾಪೇಕ್ಷಯಾ
ಭಗ-ವಾನ್ನಾರಾ-ಯಣಃ
ಭಗ-ವಾನ್ನಾರಾ-ಯ-ಣಃತಸ್ಯ
ಭಗ-ವಾನ್ನಾರಾ-ಯ-ಣಃಷಡ್ಗುಣೈಶ್ವರ್ಯ-ಪೂರ್ಣ-ನಾದ
ಭಗ-ವಾನ್ನಾರಾ-ಯ-ಣಃಹೀಗೆ
ಭಗ-ವಾನ್ಪದ್ಮ-ನಾಭ-ರೂಪ-ದಿಂದಿ-ರುವ
ಭಗ-ವಾನ್ಪರಮಾತ್ತನು
ಭಗ-ವಾನ್ಮೇಲೆ
ಭಗ-ವಾನ್ವಿಷ್ಣುಃ
ಭಗ-ವಾನ್ಶ್ರೀ-ರಾಮ-ರೂಪ-ದಿಂದ
ಭಗ-ವಾನ್ಷಡ್ಗುಣೈಶ್ವರ-ಪೂರ್ಣ-ನಾದ
ಭಗ-ವಾನ್ಷಡ್ಗುಣೈಶ್ವರ್ಯ
ಭಗ-ವಾನ್ಷಡ್ಗುಣೈಶ್ವರ್ಯ-ಪೂರ್ಣ-ನಾದ
ಭಗ-ವಾನ್ಷಡ್ಗುಣೈಶ್ವರ್ಯ-ಪೂರ್ಣ-ನಾದ-ವನು
ಭಗ-ವಾನ್ಷಡ್ಗುಣೈಶ್ವರ್ಯ-ಪೂರ್ಣನು
ಭರಣೋತ್ತುಕಃ
ಭರಿತ-ನಾದ
ಭವಂತಿ
ಭವಚ್ಚ
ಭವತಃ
ಭವತಿ
ಭವತಿ-ಯಾ-ರಿಂದ
ಭವತೀತಿ
ಭವತೀತ್ಯರ್ಥಃದೋಷಿಣಾಂ
ಭವತ್ಯಸೌ
ಭವನಾಧಿ-ಕರ-ಣ-ಮಿತಿ
ಭವಪಾಶೇನ
ಭವಿಷ್ಯ
ಭವಿಷ್ಯತಃ
ಭವಿಷ್ಯತಿ
ಭವಿಷ್ಯತ್
ಭವಿಷ್ಯಾಮಿ
ಭವೇತ್
ಭವೇತ್ಆಗುವ-ಳು-ಹೀಗೆ
ಭವೇತ್ಆದಾನು
ಭವೇತ್ಶುದ್ದ-ನಾಗಿ
ಭವೇತ್ಹುಟ್ಟುತ್ತದೆ
ಭವೇ-ದಿತಿ
ಭವ್ಯಂ
ಭಸ್ಮನಿ
ಭಸ್ಮ-ವಾಗುತ್ತದೆ
ಭಸ್ಮ-ಸಾತ್
ಭಸ್ಮ-ಸಾತ್ಕುರುತೇರ್ಜುನ
ಭಾಗ
ಭಾಗಃನಾಽಸೌ
ಭಾಗ-ಗಳು
ಭಾಗದ
ಭಾಗ-ದಿಂದ
ಭಾಗ-ವತ
ಭಾಗ-ವತ-ದಲ್ಲಿ
ಭಾಗ-ವತೇ
ಭಾಗ-ವತೇ-ಭಾಗ-ವತ-ದಲ್ಲಿ
ಭಾಗ-ವನ್ನು
ಭಾಗವು
ಭಾಗವೇ
ಭಾರ-ತದಾದ್ಯಂತ
ಭಾರತೇ
ಭಾವ
ಭಾವ-ನಾಯೇತಿ-ಯಾವತ್ನಾರಂ
ಭಾವನೆ
ಭಾವೇ
ಭಾವೇ-ಇ-ರುವಿಕೆ
ಭಾವೇ-ಭಾವ
ಭಾವೇ-ಭಾವಾರ್ಥ-ದಲ್ಲಿ
ಭಾಷೆಯ
ಭಾಷೆ-ಯಲ್ಲಿ
ಭಾಷ್ಠೆ-ಹೀಗೆ
ಭಾಷ್ಯ
ಭಾಷ್ಯಃ
ಭಾಷ್ಯ-ಛಾಂದೋಗ್ಯ
ಭಾಷ್ಯ-ಜಾಗ್ರತ್ಸ್ವಪ್ನಾ-ಭಾವಃ
ಭಾಷ್ಯದ
ಭಾಷ್ಯ-ದಲ್ಲಿ
ಭಾಷ್ಯ-ದಲ್ಲಿಯೂ
ಭಾಷ್ಯ-ದಿಂದ
ಭಾಷ್ಯ-ದೇಹ-ಯೋಗೇನ
ಭಾಷ್ಯ-ಯಾವ
ಭಾಷ್ಯ-ಯೋಃಭಾಷ್ಯ-ಗಳಲ್ಲಿಯೂ
ಭಾಷ್ಯ-ವನ್ನು
ಭಾಷ್ಯಸ್ಕಾಂದ
ಭಾಷ್ಯಾದಿ
ಭಾಷ್ಯೇ-ಐತರೇಯ
ಭಾಸತೇ
ಭಾಸ-ವಾಗುತ್ತವೆ
ಭಿನತ್ತಿ
ಭಿನ್ನ
ಭಿನ್ನತೆ
ಭಿನ್ನ-ನಾಗಿ-ರುವು-ದ-ರಿಂದ
ಭಿನ್ನನೂ
ಭಿನ್ನ-ನೆಂತಲೂ
ಭಿನ್ನ-ಲಿಂಗ
ಭಿನ್ನ-ಲಿಂಗ-ಶರೀರಕಾಃ
ಭಿನ್ನ-ಲಿಂಗಾಃ
ಭಿನ್ನ-ವಾಗಿಯೇ
ಭಿನ್ನ-ವಾಗಿ-ರುವ
ಭೀಮ
ಭೀಮ-ರಾಯರ
ಭೀಮ-ರಾಯ-ರಿಂದ
ಭೀಮ-ರಾಯರು
ಭೀಮ-ಸೇನ
ಭೀಮ-ಸೇನಃ
ಭೀಮ-ಸೇನಃತದ್ರೂಪಾಂತರತ್ವೇನ-ಬೇರೆ
ಭೀಮ-ಸೇನ-ದೇವ-ರಿಗೆ
ಭೀಮ-ಸೇನ-ದೇವರು
ಭುಂಕ್ತೇ
ಭುಂಜಂತಂಅನುಭವಿಸುವ-ರಾಗಿ
ಭುಂಜಂತಮಾತ್ಮೀಯಮಜಾತ್
ಭುಕ್ತಾನ್ನ
ಭುಕ್ತಾನ್ನ-ಪ-ಚನೇನೋದರೇ
ಭುಕ್ತ್ವಾ
ಭುವನಂಹದಿ-ನಾಲ್ಕು
ಭುವನಾನಿ
ಭೂತಂ
ಭೂತ-ಗಳು
ಭೂತಾ
ಭೂತಾಃಬ್ರಹ್ಮಾಂಡದ
ಭೂತಾನಿ
ಭೂತ್ವಾ
ಭೂತ್ವಾ-ಧ-ರಿಸಿ
ಭೂದೇವಿ
ಭೂದೇ-ವಿಗೆ
ಭೂದೇವಿ-ಯನ್ನು
ಭೂಮಿ-ಯನ್ನೆಲ್ಲ
ಭೂಮಿ-ಯಲ್ಲಿಯೂ
ಭೂಮೋಪಾಸನ
ಭೂಮೋಪಾಸನಕ್ಕೆ
ಭೂಮೌ
ಭೂಮೌ-ಅವನೇ
ಭೂಯಸೀ
ಭೂಲೋಕಕ್ಕೆ
ಭೇದ
ಭೇದಃ
ಭೇದ-ಗಳಿಂದ
ಭೇದ-ದಿಂದಲೂ
ಭೇದ-ವಿಲ್ಲ-ವೆಂಬ
ಭೇದಾ-ಭೇದೌ
ಭೇದಿಸಲ್ಪಟ್ಟ
ಭೇದೇನ
ಭೋಕೃಣಾಂಭೋಜನ-ಮಾಡುವ
ಭೋಕ್ತುಂ
ಭೋಕ್ತೃ
ಭೋಕ್ತೃ-ಗಳಲ್ಲಿ
ಭೋಕ್ತೃ-ಗಳೊಳಗೆ
ಭೋಕ್ತೃಣಾಂ
ಭೋಕ್ತೃ-ಭೋಗ್ಯ-ಗಳೆಂಬ
ಭೋಕ್ತೃ-ವಾಗಿ
ಭೋಕ್ತೃಷು
ಭೋಗ
ಭೋಗ-ಜಾತಂ
ಭೋಗ-ನಾಶಃ
ಭೋಗ-ನಾಶಃಅವು-ಗಳನ್ನು
ಭೋಗ-ವನ್ನು
ಭೋಗವು
ಭೋಗ-ಸ-ಮುದಾ-ಯ-ವನ್ನು
ಭೋಗ-ಸಮೂಹ
ಭೋಗಾಯ
ಭೋಗಿಸಲ್ಪಡುತ್ತಾನೆ
ಭೋಗಿಸುತ್ತಾ-ನಲ್ಲದೆ
ಭೋಗಿಸುವ
ಭೋಗ್ಯ-ನಾಗಿ-ರುತ್ತಾನೆ
ಭೋಗ್ಯ-ರೂಪಃ
ಭೋಜನ-ಮಾಡುವ
ಭೋಜನ-ಮಾಡುವಾಗ
ಭೋಜ್ಯಂ
ಭೋಜ್ಯಪದಾರ್ಥ-ಗಳಲ್ಲಿ
ಭೋಜ್ಯಮನ್ನಾದಿ-ಜಾತಂ
ಭೋಜ್ಯ-ವಸ್ತುಗ
ಭೋಜ್ಯ-ವಸ್ತುಷು
ಭ್ರಮತ್ಸಮಂತಾತ್
ಭ್ರಾತೃ
ಭ್ರಾತೃ-ಸಂಬಂಧ
ಭ್ರಾತೃ-ಸಖಿತ್ವೇನ
ಭ್ರಾಮ-ಯತೇ
ಮಂಗಳ-ಗಳನ್ನೂ
ಮಂಗಳಾಚರಣ
ಮಂಗಳಾಚರಣೆಯ
ಮಂಡೂ-ಕರಂತೆ
ಮಂತ್ರ-ವಾಕ್ಯ-ವನ್ನು
ಮಂದಹಾಸ-ದಿಂದಲೂ
ಮಂದಿರ-ದಲ್ಲಿ
ಮಕ್ಕಳಂತೆ
ಮಗ-ನನ್ನಾಗಿ
ಮಗ-ನಾಗಿ
ಮಗು-ವಿನ
ಮಗ್ತಾನಾಂ
ಮಗ್ನ-ರಾಗಿ
ಮಠಮ್
ಮಠವೇ
ಮಣಿಗಣಾ
ಮಣಿ-ಗಳಂತೆ
ಮಣ್ಣು
ಮತಃ
ಮತಾ
ಮತಾಃ
ಮತು
ಮತ್ತಃ
ಮತ್ತು
ಮತ್ತೆ
ಮತ್ತೊಂದು
ಮತ್ವರ್ಥ-ದಲ್ಲಿ
ಮತ್ವರ್ಥೇ
ಮತ್ವಾ
ಮತ್ಸ್ಥಾನಿ
ಮದ್ಯಪಾನ
ಮದ್ಯಪಾನ-ಮಾಡುವ-ವರು
ಮದ್ಯಪೋ
ಮದ್ರಾಸ್ನ
ಮಧೀ-ಯತೇ
ಮಧೈ
ಮಧ್ಯ
ಮಧ್ಯ-ದಲ್ಲಿ
ಮಧ್ಯ-ದಲ್ಲಿ-ರುವ
ಮಧ್ಯಾವ-ತರಣಖಂಡ-ದಲ್ಲಿ
ಮಧ್ಯೆ
ಮಧ್ವ
ಮಧ್ವಃಶ್ರೀ-ಮಧ್ವಾಚಾರ್ಯರು
ಮಧ್ವರ
ಮಧ್ವ-ಸಮೋ
ಮಧ್ವಾಚಾರ್ಯರು
ಮಧ್ವಾ-ವತ-ರಣಖಂಡೇ
ಮಧ್ವಾ-ವತಾರ-ಗಳು
ಮಧ್ವಾ-ವತಾರ-ದಲ್ಲಿ
ಮನಃ
ಮನಃಮನಸ್ಸನ್ನೂ
ಮನ-ದಲ್ಲಿ
ಮನನಶೀಲ-ರಾದ
ಮನ-ನಾದಿ-ಗಳಿಂದ
ಮನ-ಬಂದಂತೆ
ಮನಶ್ಚಾಯಂ
ಮನಸ್ಸನ್ನು
ಮನಸ್ಸಿಗೆ
ಮನಸ್ಸಿನ
ಮನಸ್ಸಿ-ನಲ್ಲಿ
ಮನಸ್ಸಿ-ನಿಂದ
ಮನಸ್ಸು
ಮನುಷ್ಯ-ರಿಂದ
ಮನುಷ್ಯರು
ಮನುಷ್ಯರೇ
ಮನೆ-ಯಾಗಿ
ಮನೆಯಾಗುಳ್ಳ-ವನು
ಮನೆಯು
ಮನೋವೃತಿಜ್ಞಾನವು
ಮನ್ಯೇ
ಮಮ
ಮಮೈವಾಂಶೋ
ಮಯಾ
ಮಯಿ
ಮರಣ
ಮರಣ-ರಹಿತರು
ಮರಣ-ಸಾ-ಧನೀ-ಭೂತ-ಸರ್ಪಾದಿವಿಷಹರಣೇನ
ಮರು
ಮರು-ಮುದ್ರಣ-ವಾಗಿಲ್ಲ-ವೆಂಬುದನ್ನು
ಮರೆ
ಮರ್ಯಾದಯಾ
ಮರ್ಯಾದಾಭಿ-ವಿಧೋ-ರಿತಿ
ಮಲಗಿರುತ್ತಾ-ನೆಂಬುದು
ಮಲಗಿರುತ್ತೀಯೇ
ಮಲಗಿ-ರುವ
ಮಲಗಿ-ರುವು-ದ-ರಿಂದ
ಮಲಿನವಾಸಸಃ
ಮಹತ್
ಮಹತ್ಕಾರ್ಯ-ವನ್ನು
ಮಹತ್ತತ್ವವೇ
ಮಹತ್ಬಹಳ
ಮಹ-ದಾದ್ಯಖಿಲೈಸ್ತತ್ವೈಃ
ಮಹಾ
ಮಹಾ-ತಮಸಿ
ಮಹಾ-ತಮಸ್ಸಿ-ನಲ್ಲಿ
ಮಹಾತ್ಮನಃ
ಮಹಾ-ನು-ಭಾವರ
ಮಹಾ-ಪಾತಕಸಂಸರ್ಗೀ
ಮಹಾ-ಪು-ಮಾನ್
ಮಹಾ-ಬಲಃ
ಮಹಾ-ಭಾರತ-ತಾತ್ಪರ್ಯ-ನಿರ್ಣಯ
ಮಹಾ-ಭಾರತದ
ಮಹಾ-ಭಾರತ-ದಲ್ಲಿ
ಮಹಾ-ಮಣಿಹೇಮಮಯ್ಯಖಿಲ
ಮಹಾ-ಮಹಿ-ಮೆ-ಗಳನ್ನು
ಮಹಾ-ಲಕ್ಷ್ಮಿಗೆ
ಮಹಾ-ಲಕ್ಷ್ಮಿಯೇ
ಮಹಾ-ಲಕ್ಷ್ಮೀ
ಮಹಾ-ಲಕ್ಷ್ಮೀಃಮಹಾ-ಲಕ್ಷ್ಮೀ-ದೇವಿಯು
ಮಹಾ-ಲಕ್ಷ್ಮೀ-ದೇವಿಯ
ಮಹಾ-ಲಕ್ಷ್ಮೀ-ದೇವಿಯರ
ಮಹಾ-ಲಕ್ಷ್ಮೀ-ದೇವಿಯ-ರಿಗೆ
ಮಹಾ-ಲಕ್ಷ್ಮೀ-ದೇವಿಯ-ರೊಡನೆ
ಮಹಾ-ಶೃಂಗಾರ
ಮಹಿಮಾ
ಮಹಿ-ಮಾನಂ
ಮಹಿಮೆ-ಯನ್ನು
ಮಹೀತಲೇ
ಮಹೇಶ್ವರ-ಪುರಸ್ಪರಃ
ಮಾ
ಮಾಂ
ಮಾಂಡೂಕ್ಯೋಪನಿಷತ್
ಮಾಂಡೂಕ್ಯೋಪನಿಷ-ದೇವ-ಮಾಂಡೂಕ್ಯೋಪನಿಷತ್ತೇ
ಮಾಂಸಭಕ್ಷ-ಕನು
ಮಾಡ-ತಕ್ಕ-ವು-ಗಳೇ
ಮಾಡ-ದ-ವನು
ಮಾಡ-ದಿದ್ದರೆ
ಮಾಡದೇ
ಮಾಡ-ಬಹು-ದಾದ
ಮಾಡ-ಬಹುದು
ಮಾಡ-ಬೇಕು
ಮಾಡಬೇಕೆಂದು
ಮಾಡ-ಬೇಡಿರಿ
ಮಾಡಲಾರದು
ಮಾಡಲಾ-ರನು
ಮಾಡಲಾರರು
ಮಾಡಲು
ಮಾಡಲ್ಪಟ್ಟ
ಮಾಡಲ್ಪಟ್ಟಿದೆ
ಮಾಡಲ್ಪಟ್ಟಿದೆ-ಮೋಕ್ಷದಃ
ಮಾಡಲ್ಪಡ-ತಕ್ಕದ್ದು
ಮಾಡಲ್ಪಡದಿದ್ದರೂ
ಮಾಡಿ
ಮಾಡಿ-ಕೊಂಡಿರು-ವನು
ಮಾಡಿ-ಕೊಂಡು
ಮಾಡಿ-ಕೊಟ್ಟ
ಮಾಡಿ-ಕೊಡುತ್ತದೆ
ಮಾಡಿ-ಕೊಡುತ್ತಾನೆ
ಮಾಡಿ-ಕೊಳ್ಳ-ತಕ್ಕದ್ದು
ಮಾಡಿ-ಕೊಳ್ಳುತ್ತಾ-ನೆಂದು
ಮಾಡಿ-ತಲ್ಲದೇ
ಮಾಡಿದ
ಮಾಡಿ-ದನು
ಮಾಡಿ-ದರೂ
ಮಾಡಿ-ದರೆ
ಮಾಡಿ-ದ-ರೆಂದು
ಮಾಡಿ-ದ-ವನು
ಮಾಡಿ-ದ-ವ-ರಿಂದ
ಮಾಡಿ-ದ-ವ-ರಿಗೆ
ಮಾಡಿ-ದ-ವರೂ
ಮಾಡಿ-ದಾಗ
ಮಾಡಿ-ದುದು
ಮಾಡಿದ್ದರು
ಮಾಡಿದ್ದಾರೆಂಬುದು
ಮಾಡಿದ್ದೇ
ಮಾಡಿ-ರುವ
ಮಾಡಿ-ಸಿದ
ಮಾಡಿ-ಸಿದರೆ
ಮಾಡಿ-ಸುತ್ತದೆ
ಮಾಡಿ-ಸುತ್ತಾನೆ
ಮಾಡಿ-ಸುವಾಗಲೂ
ಮಾಡುತ್ತದೆ
ಮಾಡುತ್ತಾನೆ
ಮಾಡುತ್ತಾರೆ
ಮಾಡುತ್ತಾರೆಯೋ
ಮಾಡುತ್ತಿ-ರುವ
ಮಾಡುತ್ತೇನೆ
ಮಾಡುವ
ಮಾಡುವಂತಹದದಲ್ಲ
ಮಾಡುವ-ನಾದು-ದ-ರಿಂದ
ಮಾಡುವ-ನೆಂದೂ
ಮಾಡುವ-ವನೂ
ಮಾಡುವ-ವ-ರನ್ನು
ಮಾಡುವ-ವ-ರಿಗೆ
ಮಾಡುವ-ವರು
ಮಾಡುವಿಕೆ-ಗಳನ್ನು
ಮಾಡುವುದರಿಂದ
ಮಾಡುವುದರಿಂದಲೂ
ಮಾಡುವುದರಿಂದಲೇ
ಮಾಡು-ವುದಿಲ್ಲ-ವೆಂದು
ಮಾಡುವುದು
ಮಾಡುವುದುಈ
ಮಾತನಾಡಿತು
ಮಾತನಾಡುತ್ತಾನೆ
ಮಾತನಾಡುತ್ತಾನೆಯೋ
ಮಾತನ್ನು
ಮಾತಾಪಿತ್ರೋಶ್ಚ
ಮಾತೃತ್ವೇನ
ಮಾತೃಯೋನಿ-ಯಿಂದ
ಮಾತೆಯು
ಮಾತ್ರ
ಮಾತ್ರ-ದಿಂದ
ಮಾತ್ರವೇ
ಮಾಧವ-ಸಮೋ
ಮಾಧ್ವ
ಮಾನ
ಮಾನ-ವ-ನಂತೆ
ಮಾಮಪ್ರಾಪ್ಯೈವ
ಮಾಮಭಿಜಾನಾತಿ
ಮಾಯಾಂ
ಮಾಯಾವೀ
ಮಾಯಾವೀ-ಎಲ್ಲ-ರನ್ನೂ
ಮಾಯಾಶಿಶುಃಮೋಹಗೊಳಿಸುವ
ಮಾಯಾಶಿಶುರಂಫ್ರಿಪಾನಃ
ಮಾರ್ಗ
ಮಾರ್ಗಃ
ಮಾರ್ಗ-ಗಳಲ್ಲಿ
ಮಾರ್ಗ-ಗಳಿಂದ
ಮಾರ್ಗ-ದಲ್ಲಿ
ಮಾರ್ಗ-ದಿಂದಲ್ಲ
ಮಾರ್ಗವು
ಮಾರ್ಗಾಧ್ವ
ಮಾರ್ಗೋ
ಮಾಳ್ಪನು
ಮಾಹಾತ್ಮ್ಯಜ್ಞಾನ-ಪೂರ್ವಕ-ವಾದ
ಮಾಹಾತ್ಮ್ಯಜ್ಞಾನ-ಪೂರ್ವಸ್ತು
ಮಾಹಾತ್ಮ್ಯಜ್ಞಾನ-ಸಹಿತ-ವಾದ
ಮಿತ್ರ
ಮಿತ್ರ-ನಾದ
ಮಿತ್ರನೇ
ಮಿಥುನೀ
ಮಿಥುನೀ-ಭವೇತ್ವಿಷ್ಣು-ವಿನ
ಮಿಥುನೀ-ಭೂಯ
ಮಿಥ್ಯಾಜ್ಞಾನ
ಮಿಥ್ಯಾಜ್ಞಾನ-ದಿಂದ
ಮಿಥ್ಯಾಜ್ಞಾನಿ-ಗಳಿಗೂ
ಮಿಥ್ಯಾಜ್ಞಾನೇನ
ಮುಂತಾಗಿ
ಮುಂತಾದ
ಮುಂತಾದ-ವ-ರಿಂದ
ಮುಂದಿನ
ಮುಂದಿ-ರುವ
ಮುಂದೆ
ಮುಂದೆಯೂ
ಮುಂಭಾಗ
ಮುಂಭಾಗದ
ಮುಂಭಾಗ-ದಿಂದ
ಮುಂಭಾಗ-ವೆಂದರೆ
ಮುಕರು
ಮುಕಿಪ್ರದಾನಾಯ
ಮುಕ್ಕಾಃನ
ಮುಕ್ಕೋ
ಮುಕ್ತ
ಮುಕ್ತ-ನಾಗುತ್ತಾನೆ
ಮುಕ್ತನು
ಮುಕ್ತಪ್ರಾಪಕ
ಮುಕ್ತಪ್ರಾಪ್ಯನೂ
ಮುಕ್ತಪ್ರಾಯರೇ
ಮುಕ್ತಬ್ರಹನ
ಮುಕ್ತಬ್ರಹ್ಮಣಃ
ಮುಕ್ತಬ್ರಹ್ಮ-ಣಃಮುಕ್ತಬ್ರಹ್ಮನ
ಮುಕ್ತಬ್ರಹ್ಮಣಾ
ಮುಕ್ತಬ್ರಹ್ಮ-ದೇವರ
ಮುಕ್ತಬ್ರಹ್ಮನ
ಮುಕ್ತಬ್ರಹ್ಮ-ನಿಗೆ
ಮುಕ್ತರ
ಮುಕ್ತ-ರಾದ
ಮುಕ್ತ-ರಿಂದ
ಮುಕ್ತ-ರಿ-ಗಿಂತಲೂ
ಮುಕ್ತ-ರಿಗೆ
ಮುಕ್ತರು
ಮುಕ್ತ-ರು-ಅರ-ವಿಧುರತ್ವಾತ್ದೋಷ
ಮುಕ್ತ-ಸಜ್ಜನೈಃ
ಮುಕ್ತಾಃಮುಕ್ತರು
ಮುಕ್ತಾನಾಂ
ಮುಕ್ತಾ-ಮುಕ್ತ-ರಿಂದ
ಮುಕ್ತಾಶ್ರಯತಾತ್ಮುಕ್ತ-ರಿಗೆ
ಮುಕ್ತಿ
ಮುಕ್ತಿಃ
ಮುಕ್ತಿಗೆ
ಮುಕ್ತಿ-ದಃಉಪಾಸನೆ-ಮಾಡುವ-ವ-ರಿಗೆ
ಮುಕ್ತಿ-ದಾಯ-ಕನು
ಮುಕ್ತಿದೋ
ಮುಕ್ತಿಪ್ರದನೂ
ಮುಕ್ತಿ-ಯನ್ನು
ಮುಕ್ತಿ-ಯಲ್ಲಿ
ಮುಕ್ತಿ-ಯೋಗ್ಯ
ಮುಕ್ತಿ-ಯೋಗ್ಯ-ರಿಗೆ
ಮುಕ್ತಿ-ಯೋಗ್ಯರು
ಮುಕ್ತೇಭ್ಯೋಽಧಿಕತ್ವಮೇವಾರ್ಥಃತದುತ್ತರಸ್ಯ-ಅದರ
ಮುಕ್ತೌ
ಮುಕ್ತೌ-ಮುಕ್ತಿ-ಯಲ್ಲಿ
ಮುಖ
ಮುಖ್ಯ
ಮುಖ್ಯತಃ
ಮುಖ್ಯಪ್ರಾಣ-ದೇವ-ರಿಗೆ
ಮುಖ್ಯಪ್ರಾಣ-ನಲ್ಲಿ
ಮುಖ್ಯಪ್ರಾಣ-ನಿಗೆ
ಮುಖ್ಯಪ್ರಾ-ಣನು
ಮುಖ್ಯಪ್ರಾಣ-ಮಾತುಃ
ಮುಖ್ಯಪ್ರಾಣ-ರೂಪ-ಪುತ್ರಸ್ತೋತೃತ್ತಿಕಾಮಃ
ಮುಖ್ಯ-ವಾಗಿ
ಮುಖ್ಯ-ವಾಗುಳ್ಳ
ಮುಖ್ಯ-ವಾಯು
ಮುಖ್ಯ-ವಾಯುಃ
ಮುಖ್ಯ-ವಾಯುಃಚೇತನ-ವರ್ಗಕ್ಕೆ
ಮುಖ್ಯ-ವಾಯುಃಜೀವೋತ್ತಮ-ರಾದ
ಮುಖ್ಯ-ವಾಯುಃತಥಾ
ಮುಖ್ಯ-ವಾಯುಃನ
ಮುಖ್ಯ-ವಾಯುಃಮುಖ್ಯ-ವಾಯು-ದೇವ-ರಿಗೆ
ಮುಖ್ಯ-ವಾಯು-ದೇವ-ರದ್ವಾರಾ
ಮುಖ್ಯ-ವಾಯು-ಪರೋ
ಮುಖ್ಯ-ವಾಯು-ವನ್ನೇ
ಮುಖ್ಯ-ವಾಯು-ವಿಗೆ
ಮುಖ್ಯ-ವಾಯು-ವಿನ
ಮುಖ್ಯ-ವಾಯುವು
ಮುಖ್ಯ-ವಾಯುವೇ
ಮುಖ್ಯಾರ್ಥ-ವಾದ
ಮುಖ್ಯಾಶ್ರಯ-ವಾದ
ಮುಗಿದ-ನಂತರ
ಮುಗಿ-ದೊಡನೆ
ಮುಟ್ಟಿ-ದರೂ
ಮುಟ್ಟುತ್ತಾ-ರೆಂಬ
ಮುಟ್ಟುವು-ದ-ರಿಂದಲೂ
ಮುಟ್ಟು-ವುದು
ಮುತ್ತುರತ್ನ-ಗಳು
ಮುದಾ
ಮುದ್ರಣ-ಕಾರ್ಯ
ಮುದ್ರಣಗೊಳಿಸಬೇಕೆಂಬ
ಮುನಿಂ
ಮುನ್ನುಡಿ
ಮುಮುಕ್ಷುಭಿಃ
ಮುರ್ಖರು
ಮೂಢ-ರಾದ
ಮೂಢಾ
ಮೂಢಾನಾಂ
ಮೂರು
ಮೂರ್ಖ-ರಿಗೂ
ಮೂರ್ಖರು
ಮೂರ್ಖಾ
ಮೂರ್ಛೆ
ಮೂರ್ತಯಸ್ಸಂತಿ
ಮೂಲ
ಮೂಲಕ
ಮೂಲ-ಕ-ವಲ್ಲದೆ
ಮೂಲ-ದಿಂದ
ಮೂಲ-ರೂಪ
ಮೂಲ-ರೂಪಕ್ಕೂ
ಮೂಲ-ರೂಪದ
ಮೂಲ-ರೂಪ-ದಲ್ಲಿ
ಮೂಲ-ರೂಪ-ದೊಂದಿಗೆ
ಮೂಲ-ರೂಪ-ನಾದ
ಮೂಲ-ರೂಪ-ವಾದ
ಮೂಲ-ರೂಪವು
ಮೂಲವೇ
ಮೂವತ್ತು
ಮೃತಾನ್
ಮೃತಾನ್ಸತ್ತ-ವ-ರನ್ನು
ಮೃತ್ಯು-ದೇವ-ತೆ-ಗಳು
ಮೇ
ಮೇಧಾ
ಮೇನನ್ನ
ಮೇಲಕ್ಕೆ
ಮೇಲಿದ್ದ
ಮೇಲೆ
ಮೈತ್ರಿಯೇ
ಮೈವಾ-ಎಂದಿಗೂ
ಮೈವಾರುಣೋ
ಮೈವಾರುಣ್ಯಃ
ಮೈವಾರುಣ್ಯೋ
ಮೈವಾ-ರುವಣ್ಯಃ
ಮೊದಲನೇ
ಮೊದಲಾಗಿ
ಮೊದಲಾದ
ಮೊದಲಾದ-ವರ
ಮೊದಲಾದ-ವ-ರಿಂದ
ಮೊದಲಾದ-ವ-ರಿಗೆ
ಮೊದಲಾದ-ವು-ಗಳನ್ನು
ಮೊದಲಾದುದನ್ನು
ಮೊದಲಿನ
ಮೊದಲು
ಮೊದಲು-ಗೊಂಡು
ಮೋಕ್ಷ
ಮೋಕ್ಷಃ
ಮೋಕ್ಷಈ
ಮೋಕ್ಷಕ್ಕೆ
ಮೋಕ್ಷದಃ
ಮೋಕ್ಷ-ದಲ್ಲಿ
ಮೋಕ್ಷ-ದಲ್ಲಿಯೂ
ಮೋಕ್ಷ-ದಶ್ಚ
ಮೋಕ್ಷ-ದಾತ
ಮೋಕ್ಷ-ದಾತ-ನಾದ
ಮೋಕ್ಷ-ದಾತೃತ್ವಾವಗಮಾತ್ಬ್ರಹ್ಮ-ದೇವ-ರಿಗೂ
ಮೋಕ್ಷಪ್ರದ
ಮೋಕ್ಷ-ವನ್ನು
ಮೋಕ್ಷ-ವಿಲ್ಲವೋ
ಮೋಕ್ಷವು
ಮೋಕ್ಷಸ್ಯ
ಮೋಕ್ಷಾಃ
ಮೋಕ್ಷಾಪೇಕ್ಷಿ-ಯಾದ
ಮೋಚಕಶ್ಚ
ಮೋದತೇ
ಮೋಸ
ಮೋಹಗೊಳಿಸುವ
ಮೌದ್ಗಲ್ಯ
ಮ್ರಿಯತೇ
ಯ
ಯಂ
ಯಂಜ್ಞಾನಂಜ್ಞಾನವು
ಯಃ
ಯಃಯಾವ
ಯಕಾರತ್ವಂ
ಯಕಾರತ್ವಂಗತಃ
ಯಕಾರಸ್ವಾರ್ಥಃಯ
ಯಕಾರಾ-ದುತ್ತರೋ-ಽಪಿ
ಯಕಾರಾ-ದೇಶಃ
ಯಕಾರಾ-ದೇಶಃಯ
ಯಕಾರೋ
ಯಕಾರೋ-ಪಿಯ
ಯಕಾರೋಪ್ಯ-ಭಾವ-ವಾಚ್ಯ-ಕಾರ
ಯಜ್ಞ-ಯಾಗಾದಿ-ಗಳನ್ನು
ಯಜ್ಞಾನೇ
ಯಣಾ-ದೇಶೇ
ಯಣಾ-ದೇಶೇ-ಯಣ್
ಯತಃ
ಯತಸ್ತತಃ
ಯತೋ
ಯತ್
ಯತ್ತು
ಯತ್ನಯಂತ್ಯಭಿಸಂವಿಶಂತಿ
ಯತ್ಪಾಪಂ
ಯತ್ಯಾವ
ಯತ್ರ
ಯತ್ರಾಸ್ತೇ-ಎಲ್ಲಿ
ಯಥಾ
ಯಥಾ-ಪರಾಧಂ
ಯಥಾ-ಮತಿ
ಯಥಾ-ಯೋಗ್ಯ-ವಾಗಿ
ಯಥಾ-ಯೋಗ್ಯ-ವಾದ
ಯಥಾರ್ಥಜ್ಞಾನ-ದಿಂದಲೇ
ಯಥಾರ್ಥಜ್ಞಾನ-ವನ್ನು
ಯಥಾರ್ಥಜ್ಞಾನವೂ
ಯಥಾರ್ಥ-ರೂಪ-ವನ್ನೂ
ಯಥಾ-ವತಿ
ಯಥಾಸ್ಥಾನಂ
ಯಥೇಷ್ಟಮೀಶಃ
ಯಥೋಕಂ
ಯಥೋಕ್ಕಂಹೀಗೆ
ಯಥೋಕ್ತಂ
ಯಥೋಕ್ತಂಅಜಸ್ಯ
ಯಥೋಕ್ತಂತತ್ರ
ಯಥೋಕ್ತಂಪಾತಾಲಮೇ-ತಸ್ಯ
ಯಥೋಕ್ತಂಹೀಗೆ
ಯಥೋಕ್ತಂಹೇ
ಯಥೋಕ್ತಮೈತರೇಯ
ಯಥೋಕ್ತಮೈತರೇಯ-ಭಾಷ್ಯೇ-ಐತರೇಯ
ಯದಂತಃ
ಯದಂತಃಆ
ಯದಸ್ಮೈ
ಯದಾ
ಯದಾಶ್ನಾತಿ
ಯದಿದಸ್ಮಿನ್
ಯದು-ಕುಲೋದ್ಭವ
ಯದುಕ್ತಂ
ಯದ್ವಾ
ಯದ್ವೃದ್ಧಿಹ್ರಾಸಾವಪಿ
ಯನ್ನು
ಯಮೇವೈಷ
ಯರ್ಹಿ
ಯರ್ಹಿ-ಯಸ್ಸಾತ್ಯಾವ
ಯಶ್ಚ
ಯಶ್ಚಾಸ್ಮಿ
ಯಶ್ವಾಸೌ
ಯಸ್ಕಾಸೌ
ಯಸ್ತದ್ವೇದ
ಯಸ್ತಸ್ಮಿನ್
ಯಸ್ತಾ
ಯಸ್ಮಾತ್
ಯಸ್ಮಾತ್ಯಾ-ರಿಂದ
ಯಸ್ಮಾದ-ಪರೋಕ್ಷೀ-ಕೃತಾತ್
ಯಸ್ಯ
ಯಸ್ಯತ್ರೀಣ್ಯುದಿ-ತಾನಿ
ಯಸ್ಯಪ್ರದ್ಯುಮ್ನ-ಸಂಕರ್ಷ-ರೂಪಿಣಃ
ಯಸ್ಯ-ಯಾರ
ಯಸ್ಯ-ಯಾ-ರಿಗೆ
ಯಸ್ಯ-ಯಾವ
ಯಸ್ಯಾಂಶಾಂಶೇನ
ಯಸ್ಯಾಸಾವಿತಿ
ಯಸ್ಯಾಸೌ
ಯಸ್ಯಾಽಸೌ
ಯಸ್ಯೇತಿ
ಯಸ್ಯೋದರ
ಯಾ
ಯಾಂತಿ
ಯಾಂತು
ಯಾಂತು-ಉಂಟುಮಾಡಲಿ
ಯಾಂತ್ಯಧಮಾಂ
ಯಾಕೆಂದರೆ
ಯಾಗಮಾಡುತ್ತಿದ್ದ
ಯಾಗಾಂತದ
ಯಾಗಿ-ರುವು-ದ-ರಿಂದ
ಯಾತಿ
ಯಾದವ
ಯಾದ-ವಾಚಾರ್ಯಗುರೂಣಾಂ
ಯಾದ-ವಾರ್ಯ
ಯಾಧಾತೋಃಯಾ
ಯಾನಿ
ಯಾನೀಹ
ಯಾಪನಂ
ಯಾಪನಂಹೋಗಿ-ಸು-ವುದು
ಯಾಪ-ನಾತ್ಗಮ-ನಾತ್ನಿರಸ-ನಾತ್
ಯಾಪ-ಯತಿ
ಯಾಪ-ಯತಿಪ್ರಾಪ-ಯತಿ-ತಂದೊದಗಿಸುತ್ತಾನೆ
ಯಾಪ-ಯತಿಪ್ರಾಪ-ಯತಿ-ದ-ದಾತಿ-ವಿಚ್ಛತ್ತಿ
ಯಾಪಯತೀತಿ
ಯಾಯಾನ್ಮುಮುಕ್ಷೌ
ಯಾಯಾವ
ಯಾರ
ಯಾರನ್ನು
ಯಾರಿಂದ
ಯಾರಿಂದಲೇ
ಯಾರಿಗೂ
ಯಾರಿಗೆ
ಯಾರಿಗೋ
ಯಾರು
ಯಾರೂ
ಯಾರೋ
ಯಾವ
ಯಾವಚ್ಚ
ಯಾವತ್
ಯಾವತ್ಅಭಿವ್ಯಕ್ತ-ವಾಗುವುದು
ಯಾವತ್ಅ-ವರ
ಯಾವತ್ಅ-ಸಮ್ಯಕ್ಪೂರ್ತಿ-ಯಾದ
ಯಾವತ್ಎಲ್ಲ
ಯಾವತ್ತ-ದೇವ-ಅದೇ
ಯಾವತ್ನಾಶ-ಕಾರ-ಕ-ವಾ-ದುದು
ಯಾವಪ್ರತಿ-ಷೇಧಃ
ಯಾವಪ್ರಲಯ-ಕಾಲ-ದಲ್ಲಿ
ಯಾವತ್ಭದ್ರ-ವಾದ
ಯಾವತ್ವಾಸಸ್ಥಳ-ಗಳು
ಯಾವತ್ವಿಷಯ-ಸಂಬಂಧ-ವಾ-ದುದು
ಯಾವತ್ಶತ್ರು-ಗಳ
ಯಾವತ್ಹೀಗೆಂದು
ಯಾವತ್ಹೀಗೆಂಬ
ಯಾವತ್
ಯಾವತ್-ಉದ್ದಿಶ್ಯ-ಲಕ್ಷ್ಮ-ದಲ್ಲಿಟ್ಟು
ಯಾವ-ದಿತಿ
ಯಾವ-ರೀತಿ
ಯಾವಾಗ
ಯಾವಾಗಲೂ
ಯಾವಾನ್
ಯಾವುದನ್ನೂ
ಯಾವುದೂ
ಯಾಸು
ಯಾಸು-ಯಾವ
ಯುಕ್ತಂನಾರಾ
ಯುಕ್ತ-ಮಿತ್ಯವ-ಗಂತವ್ಯಂ
ಯುಕ್ತ-ವಾದ
ಯುಕ್ತವು
ಯುಗದ
ಯುಗಾಂತಾನಲತಿಗ್ಮ-ನೇಮಿ
ಯುಗಾಂತೇ
ಯುತಃಇ-ವ-ರಿಂದ
ಯುದ್ಧ-ದಲ್ಲಿ
ಯೂಥಪಾಯ
ಯೇ
ಯೇನ
ಯೇನ್ಯೇ
ಯೇಪಿ
ಯೇಯಾರು
ಯೇವಿದ್ಯಾಮುಪಾಸತೇ
ಯೇಷಾಂ
ಯೇಷ್ಯಲಂಕಾರಾ
ಯೋ
ಯೋಗ್ಯ
ಯೋಗ್ಯಂ
ಯೋಗ್ಯ-ತಯಾ
ಯೋಗ್ಯ-ತಯಾ-ಇಂತಹ
ಯೋಗ್ಯ-ತಯಾ-ಯೋಗ್ಯ-ವಾದ
ಯೋಗ್ಯ-ತೆಗೆ
ಯೋಗ್ಯ-ನಾಗಿ
ಯೋಗ್ಯ-ನಾದ
ಯೋಗ್ಯ-ನಾದ-ವ-ನಾದರೋ
ಯೋಗ್ಯ-ವಾಗಿ
ಯೋಗ್ಯ-ವಾದ
ಯೋಗ್ಯಸ್ತು
ಯೋನಿ-ಗಳಲ್ಲಿ
ಯೋನಿಮಾಪನ್ನಾ
ರ
ರಂ
ರಂಗ-ನಾಥ
ರಂಗ-ನಾಥನ
ರಂಗ-ನಾಥ-ನಂತೆಯೇ
ರಂಗ-ನಾಥ-ನನ್ನೂ
ರಂಗ-ನಾಥನು
ರಂಗಮಂದಿರಂ
ರಂಗಸಂಜ್ಞಕಂ
ರಂಗೇಶಃ
ರಂಗೇಶಃಅಲ್ಲಿ
ರಂರಮಣಂ
ರಂರಮಣಂಕ್ರೀಡಿಸುವ
ರಂರಮಣಂಕ್ರೀಡೆಯು
ರಂರಮಣಂಸುಖ-ಸುಖವು
ರಕ್ತಪ-ಟಃಕೆಂಪು
ರಕ್ತಶ್ಚಾಸೌ
ರಕ್ಷ-ಕನು
ರಕ್ಷಣೆ-ಯಲ್ಲಿಯೇ
ರಕ್ಷಾಭ್ಯಾಂ
ರಕ್ಷಿ-ಸಿ-ದುದು
ರಚನಾಕ್ರಮ-ದಿಂದ
ರಚನೆ
ರಚನೆ-ಯನ್ನು
ರಚ-ಯತೇ-ಽಖಿಲಂ
ರಚಿತ-ವಾಗಿದೆ
ರಚಿತ-ವಾದ
ರಚಿತ-ವಾ-ದುದು
ರಚಿ-ಸಿದ-ರೆಂಬುದು
ರಚಿ-ಸಿ-ರುತ್ತಾರೆ
ರತ್ನಸು-ವರ್ಣ-ರೂಪಳೂ
ರತ್ನಾಭ-ರಣಭೂಷಿತ-ನಾಗಿ
ರಥದ
ರಮಂತ್ಯಹೋ
ರಮಣ
ರಮಣಂ
ರಮಣಂಸುಖ-ದಿಂದ
ರಮತೇ
ರಮಯಂತಿಸ್ವರ್ಗಾದಿ
ರಮಯಾ
ರಮಾಂ
ರಮಾ-ದೇವಿಯ
ರಮಾ-ದೇವಿಯರು
ರಮಾ-ದೇವಿಯ-ವರ
ರಮಾಪತೇ
ರಮಾಬ್ರಹ್ಮಾದಿ
ರಮಾಬ್ರಹ್ಮಾ-ದಿಕಂ
ರಮಿಸುತ್ತಾ-ನೆಂದು
ರಮಿಸುವ
ರಲ್ಲಿ
ರವತಿಈ
ರಶ್ಯಾಸೌ
ರಷಾಭ್ಯಾಂ
ರಷಾಭ್ಯಾಂನೋಣಃ
ರಸಂ
ರಸಂಕರ್ಮ-ದಿಂದ
ರಸನಂ
ರಸನಂನಾಲಿಗೆ-ಯನ್ನೂ
ರಹಸ್ಯ
ರಹಿತ-ನಾಗಿ-ರುವ
ರಹಿತ-ನೆಂದರ್ಥ
ರಹಿತ-ರಾಗಿ
ರಹಿತ-ರಾದ-ವ-ರಾದು-ದ-ರಿಂದ
ರಾ
ರಾಕ್ಷಸ-ಬಾಧಾದ್ಯುಪದ್ರವ-ಪರಿಹಾ-ರಾದಿ-ನೋಪ-ಕಾರ-ಕ-ತಯಾ-ರಾಕ್ಷಸ
ರಾಕ್ಷಸರ
ರಾಕ್ಷಸ-ರನ್ನು
ರಾಜಸೂಯ
ರಾಮ
ರಾಮ-ಕೃಷ್ಣಾದಿ
ರಾಮ-ದೇವರ
ರಾಮಾದಿ
ರಾಮಾದ್ಯವ-ತಾರಾಃ
ರಾಮಾ-ಯಣ-ದಲ್ಲಿ
ರಾಮಾ-ಯಣೇ
ರಾಮೋಪಿ
ರಾಶಿ
ರಾಹಿತ್ಯೇನ-ಹೊರಗೆ
ರಿಗೆ
ರೀಂಕ್ಷಯೇ
ರೀಙ್
ರೀತಿ
ರೀತಿ-ಯಂತೆಯೇ
ರೀತಿ-ಯಲ್ಲಿ
ರೀತಿ-ಯಲ್ಲಿಯೂ
ರೀತಿ-ಯಲ್ಲಿಯೇ
ರೀತಿ-ಯಾಗಿ
ರೀತಿ-ಯಿಂದ
ರೀತಿ-ಯಿಂದಲೂ
ರೀಯಂತೆ
ರೀಯಂತೆ-ನಾಶ
ರೀಯಂತೇ
ರೀಯಂತೇನ
ರೀಯತೆ-ನಾಶ-ರಹಿತಳು
ರೀಯತೆ-ನಾಶ-ಹೊಂದು-ವುದಿಲ್ಲ
ರೀಯತೇ
ರುದ್ರ
ರುದ್ರ-ವರ್ತ್ಮನಿ
ರೂಢ-ಮಾತ್ರ
ರೂಢಿ-ಬಲ-ದ-ಮೇಲೆ
ರೂಪ
ರೂಪಂ
ರೂಪಂಹನುಮಂತ-ನೆಂಬ
ರೂಪ-ಗಳ
ರೂಪ-ಗಳನ್ನು
ರೂಪ-ಗಳನ್ನುಳ್ಳ
ರೂಪ-ಗಳಿಂದ
ರೂಪ-ಗಳಿವೆ
ರೂಪ-ಗಳು
ರೂಪ-ಗಳೂ
ರೂಪ-ಗಳೇ
ರೂಪತ್ವವು
ರೂಪತ್ವಾತ್
ರೂಪ-ದಲ್ಲಿ
ರೂಪ-ದಲ್ಲಿ-ರುವ
ರೂಪದಿ
ರೂಪ-ದಿಂದ
ರೂಪ-ದಿಂದಿದ್ದು
ರೂಪ-ದಿಂದಿ-ರುವ
ರೂಪ-ಧರ-ನಾದ
ರೂಪ-ವನ್ನು
ರೂಪ-ವರ್ಣ
ರೂಪ-ವಾಗಿ
ರೂಪ-ವಾಗಿ-ರುವು-ದ-ರಿಂದ
ರೂಪ-ವಾದ
ರೂಪ-ವಾದು-ದ-ರಿಂದ
ರೂಪವು
ರೂಪ-ವು-ಎಂದು
ರೂಪವೇ
ರೂಪ-ಹೊಂದು-ವ-ವನು
ರೂಪಾಂತರ
ರೂಪಾನ್ನೄನ್ಅ-ಸಜ್ಜನಾನ್ಸಜ್ಜನ-ರಲ್ಲದ-ವ-ರನ್ನು
ರೂಪಾನ್ಸತ್ಪುರುಷ-ರಿಗೆ
ರೂಪಿಣಃ
ರೂಪೀ
ರೂಪೇಣ
ರೇಣುಭಿಃಅವನ
ರೇಣುಭಿಃನನ್ನ
ರೇಫಃ
ರೇಫಃರ-ಕಾರದ
ರೇಫಸ್ಕೋತ್ತರತ್ವತಃಈ
ರೇಫಸ್ಯಾರ್ಥಃತಥಾ
ರೇಫಸ್ಯಾರ್ಥಃರ
ರೇಫಾಪೇಕ್ಷಯಾ
ರೇಫೋತ್ತರಯೋಃ
ರೇಫೋಪಿ
ರೇಫೋಪಿರ
ರೋಮಕೂಪಗಂ
ರೋಮಕೂಪ-ಗಳಲ್ಲಿ
ಲಕ್ಷಣ-ಗಳನ್ನು
ಲಕ್ಷಣ-ವನ್ನು
ಲಕ್ಷಿಸಿ
ಲಕ್ಷ್ಮೀ
ಲಕ್ಷ್ಮೀಃನೀರಿನ
ಲಕ್ಷ್ಮೀ-ದೇ-ವಿಗೆ
ಲಕ್ಷ್ಮೀ-ದೇವಿಯ
ಲಕ್ಷ್ಮೀ-ದೇವಿಯೇ
ಲಕ್ಷ್ಮೀ-ನಾರಾ-ಯಣಃ
ಲಕ್ಷ್ಮೀ-ರಂಭಸ್ವ-ರೂಪಿಣೀ
ಲಕ್ಷ್ಮೀ-ಸಮೇತ-ನಾಗಿ
ಲಕ್ಷ್ಮೀಸ್ವ-ರೂಪ-ವಾದ
ಲಕ್ಷ್ಮ್ಯಾತ್ಮ-ಕತ್ವೇನ
ಲಕ್ಷ್ಮ್ಯಾತ್ಮಕ-ವಾಗಿ-ರುವು-ದ-ರಿಂದ
ಲಕ್ಷ್ಮ್ಯಾತ್ಮಕ-ವಾದ
ಲಕ್ಷ್ಮ್ಯಾತ್ಮಕ-ವಾ-ದುದು
ಲಕ್ಷ್ಯ-ದಲ್ಲಿಟ್ಟು-ಕೊಂಡು
ಲಭಿ-ಸುತ್ತದೆಯೋ
ಲಭ್ಯಃ
ಲಭ್ಯವಾಗುತ್ತಿತ್ತೆಂಬುದನ್ನು
ಲಯ
ಲಯ-ಕರ್ತೃತ್ವಾತ್ಜಗತ್ತಿನ
ಲಯ-ಕರ್ತೃತ್ವಾತ್ತಥಾ
ಲಯ-ಕಾಲೇ
ಲಯ-ಕಾಲೇಪ್ರಳಯ-ಕಾಲವು
ಲಯ-ಮಾರ್ಗ-ಗಳನ್ನು
ಲಯ-ವನ್ನು
ಲಯ-ಹೊಂದುವ
ಲಾಭಃತತ್ಅದು
ಲಾಭಃಫಲಿತಾರ್ಥ-ವಾಗುತ್ತದೆ
ಲಾಭಾಚ್ಯ
ಲಿಂಗ-ದೇಹ
ಲಿಂಗ-ದೇಹ-ವನ್ನು
ಲಿಂಗ-ದೇಹ-ವನ್ನೂ
ಲಿಂಗ-ದೇಹವು
ಲಿಂಗ-ದೇಹಾ-ದಿಕಂ
ಲಿಂಗ-ಮಾತ್ರ
ಲಿಂಗ-ಶರೀರ
ಲಿಂಗ-ಶರೀರ-ಗಳನ್ನು
ಲಿಂಗ-ಶರೀರದ
ಲಿಂಗ-ಶರೀರ-ರೂಪ-ದಿಂದ
ಲಿಂಗ-ಶರೀರ-ವನ್ನು
ಲೀಲಯಾ
ಲೀಲಯಾ-ನಯದಿಮಾಮಸೌ
ಲೀಲೆ-ಯಿಂದ
ಲೇಪವಾಗ-ದಿ-ರುವಿಕೆಯು
ಲೇಪಿಸು-ವುದಿಲ್ಲ
ಲೋಕ
ಲೋಕಂ
ಲೋಕ-ಗಳ
ಲೋಕ-ಗಳನ್ನೂ
ಲೋಕ-ಗಳಲ್ಲಿ
ಲೋಕ-ಗಳಲ್ಲಿಯೂ
ಲೋಕ-ಗಳಿಗೂ
ಲೋಕ-ಗಳಿಗೆ
ಲೋಕ-ಗಳು
ಲೋಕ-ಗಳೂ
ಲೋಕದ
ಲೋಕ-ದಲ್ಲಿ
ಲೋಕ-ರೂಪೇಣ
ಲೋಕ-ಸಾರಂ
ಲೋಕಾನ್ನೋ
ಲೋಕಾಪೇಕ್ಷಯಾ
ಲೋಕೇ
ಲೋಪ-ವಾಗು-ವುದಿಲ್ಲ
ಲೋಪ-ಹೊಂದಿತು
ಲ್ಕುಟ್
ಲ್ಯುಟ್
ಲ್ಯುಟ್ಕರ್ಮಣಿ
ಲ್ಯುಟ್ಭಾವಾರ್ಥ-ದಲ್ಲಿ
ಲ್ಯುಟ್ಲ್ಯುಟ್
ವಂದಿತ-ವಾದ
ವಂದೇ
ವಂದ್ಯ
ವಂದ್ಯ-ನಾದ
ವಂದ್ಯ-ನಾದು-ದ-ರಿಂದಲೂ
ವಃ
ವಃನಮಗೆಲ್ಲ
ವಕ್ತಿ
ವಕ್ಷ-ದಲ್ಲಿ
ವಚಃಮಾ-ತನ್ನು
ವಚನ-ಗಳಿವೆ
ವಚನ-ದಂತೆ
ವಚನಭ್ರಷ್ಟನಾಗ-ದಂತೆ
ವಚ-ನಾತ್
ವಚ-ನಾತ್ಆ
ವಚ-ನಾತ್ಇವೇ
ವಚ-ನಾತ್ಈ
ವಚೋ
ವಟ-ಪತ್ರ
ವಟ-ಪತ್ರದ
ವಟ-ಪತ್ರ-ಮಿತಿ
ವಟ-ಪತ್ರವೇ
ವಟ-ಪತ್ರಾತ್ತಿಕಾ
ವಟ-ಪತ್ರಾತ್ಮಿಕಾ-ಆಲದ
ವಣ್ಯಃಧ್ಯಾನಕ್ಕೆ
ವಣ್ಯಃಧ್ಯಾನಿ-ಸಲ್ಪಡಲು
ವದನ್
ವದಕ್ಷಯ
ವಯಸ್ಯಾನ್ಪಶ್ಯತ
ವಯಸ್ಯಾನ್ಸಂಸಾರ-ದಲ್ಲಿ
ವರ-ಬಲ-ದಿಂದ
ವರಾನನೇ
ವರಾಹ
ವರಾಹ-ದೇವ-ರಿಂದ
ವರಾಹ-ದೇವ-ರಿಗೆ
ವರಾಹ-ನರ
ವರಾಹ-ನಿಗೆ
ವರಾಹ-ಪುರಾಣ
ವರಾಹ-ಪುರಾ-ಣೇ-ವರಾಹ-ಪುರಾ-ಣ-ದಲ್ಲಿ
ವರಾಹಾಷ್ಟೋತ್ತರಶತ-ನಾಮ-ಸು-ವರಾಹ
ವರುಣ
ವರ್ಗಕ್ಕೆ
ವರ್ಜಿತ-ನಾದು-ದ-ರಿಂದ
ವರ್ಜಿತ-ರಾಗಿ-ರುವು-ದ-ರಿಂದ
ವರ್ಣಕ್ಕೂ
ವರ್ಣಕ್ಕೆ
ವರ್ಣ-ಗಳ
ವರ್ಣ-ಗಳನ್ನು
ವರ್ಣ-ಗಳು
ವರ್ಣದ
ವರ್ಣ-ದಲ್ಲಿ
ವರ್ಣ-ವನ್ನು
ವರ್ಣವು
ವರ್ಣವೂ
ವರ್ಣಸ್ಯಋ
ವರ್ಣಿತುಂ
ವರ್ಣಿತುಂವರ್ಣಿ-ಸಲು
ವರ್ಣಿಸ-ಲಾಗಿದೆ
ವರ್ಣಿ-ಸಲು
ವರ್ಣಿಸಲ್ಪಟ್ಟಿದೆ
ವರ್ಣಿಸಲ್ಪಡ-ತಕ್ಕ-ವನಲ್ಲ
ವರ್ತತೇ
ವರ್ತ-ಮಾನ
ವರ್ತ್ಮನಿ
ವರ್ತ್ಯ
ವರ್ಷ
ವರ್ಷ-ಗಳು
ವರ್ಷ-ವಯಸ್ಸಿ-ನಿಂದ
ವರ್ಷಾಚರಣೆಯ
ವರ್ಷಾಚರಣೆಯು
ವವ್ರ
ವವ್ರಂ
ವವ್ರೇಽಸ್ಮಿನ್
ವಶ-ದಲ್ಲಿಯೂ
ವಶೇ
ವಸನ್
ವಸ್ತು-ಗಳ
ವಸ್ತು-ಗಳಲ್ಲಿ
ವಸ್ತು-ಗಳಿಂದ
ವಸ್ತು-ಗಳಿಗೂ
ವಸ್ತು-ಗಳಿಗೆ
ವಸ್ತು-ಗಳು
ವಸ್ತು-ನಿರ್ದೆಶ
ವಸ್ತುಭ್ಯೋಽನೇಕ
ವಸ್ತು-ಮಯೀ
ವಸ್ತು-ರೂಪ-ದಿಂದ
ವಸ್ತು-ವನ್ನು
ವಸ್ತು-ವಿಷಯ-ಕ-ವಾದ
ವಸ್ತುವು
ವಸ್ತುವೂ
ವಾ
ವಾಂಛಂತಿ
ವಾಂಛಂತಿ-ಅಪೇಕ್ಷಿಸುತ್ತಾರೆ
ವಾಅಂತಹ
ವಾಅಥವಾ
ವಾಅದಕ್ಕೆ
ವಾಕ್ಯ-ಗಳಲ್ಲಿ
ವಾಕ್ಯ-ಗಳು
ವಾಕ್ಯ-ಗಳೂ
ವಾಕ್ಯ-ವನ್ನು
ವಾಕ್ಯಾನಿ
ವಾಕ್ಯಾನಿ-ಮಾತು-ಗಳನ್ನು
ವಾಗಿ-ರುವುವು
ವಾಗ್ವಿವೃತಾಶ್ಚ
ವಾಚಕ
ವಾಚಕಃ
ವಾಚೀ
ವಾಚೀ-ಬೇರೆ-ಯಾ-ದುದು
ವಾಚ್ಯ-ಕಾರ
ವಾಚ್ಯತ್ವ
ವಾಚ್ಯತ್ವದ
ವಾಚ್ಯ-ರಾದ
ವಾತನೀ-ತೈಃಗಾಳಿ-ಯಿಂದ
ವಾದ
ವಾಪಸ್ಸು
ವಾಪಿ
ವಾಮ
ವಾಮ-ನ-ದೇವರು
ವಾಮ-ಪಾದಾಂಗುಷ್ಠನಖಾತ್ತನ್ನ
ವಾಮ-ಪಾದಾಂಗುಷ್ಠನಖಾದ್
ವಾಯು
ವಾಯುಃ
ವಾಯುಃವಾಯು-ದೇವ-ರಿಗೆ
ವಾಯು-ಕುಮಾರ-ರಾದ
ವಾಯು-ಕೂರ್ಮ-ರೂಪ-ವನ್ನು
ವಾಯು-ಗುರು-ಗಳ
ವಾಯು-ದೂತರು
ವಾಯು-ದೂತೈಃ
ವಾಯು-ದೇವ
ವಾಯು-ದೇವನೇ
ವಾಯು-ದೇವರ
ವಾಯು-ದೇವ-ರಲ್ಲಿ
ವಾಯು-ದೇವ-ರಾಗು-ವುದಿಲ್ಲ
ವಾಯು-ದೇವ-ರಿಗೂ
ವಾಯು-ದೇವ-ರಿಗೆ
ವಾಯು-ದೇವರು
ವಾಯು-ದೇವರೂ
ವಾಯು-ದೇವರೇ
ವಾಯು-ಪುರಾ-ಣ-ದಲ್ಲಿ
ವಾಯು-ಪುರಾ-ಣ-ದಿಂದಲೂ
ವಾಯು-ಪುರಾಣೇ
ವಾಯು-ರನಿಲಮಮೃತಂ
ವಾಯು-ರಮೃತಃ
ವಾಯು-ರುದ್ದಿಷ್ಟೋ
ವಾಯುರ್ನರಃಆ
ವಾಯುರ್ನರೋತ್ತಮೋ-ನಾಮ
ವಾಯುರ್ಯ
ವಾಯುರ್ಹಿ
ವಾಯು-ವಿಗೆ
ವಾಯು-ವಿನ
ವಾಯುವು
ವಾಯುಸ್ತುತಿಯ
ವಾಯುಸ್ತುತಿಯಲ್ಲಿಯೂ
ವಾಯೋ
ವಾಯೋಃ
ವಾಯೋರ್ಗದಾಪ್ರಹಾರೇಣ
ವಾಯೌ
ವಾರೀಶ
ವಾರೀಶ-ಯುತೋ-ನಿರುದ್ದಂ
ವಾರೀಶ-ವರು-ಣನು
ವಾರುಣಿ-ಯಾದರೋ
ವಾರುಣೀಂ
ವಾರುಣೀಂವಾರುಣಿ-ಯನ್ನು
ವಾಸಃ
ವಾಸ-ಮಾಡಿ-ಕೊಂಡಿ-ರುವ-ವನು
ವಾಸ-ಮಾಡುತ್ತಿ-ರುವು-ದ-ರಿಂದ
ವಾಸ-ಮಾಡುವ
ವಾಸ-ವನ್ನು
ವಾಸಸ್ಥಳ
ವಾಸಸ್ಥಳ-ವಾಗಿದೆ
ವಾಸಸ್ಥಳ-ವಾಗಿ-ದೆಯೋ
ವಾಸಸ್ಥಾನ
ವಾಸಸ್ಥಾನ-ಗಳಿವೆ
ವಾಸಸ್ಥಾನ-ವನ್ನಾಗಿ
ವಾಸಿ-ಸುತ್ತಿದ್ದಾ-ನೆಂದು
ವಾಸುಖಾನುಭವ-ವನ್ನು
ವಾಸು-ದೇವ
ವಾಸು-ದೇವಂ
ವಾಸು-ದೇವಂವಾಸು-ದೇವ
ವಾಸು-ದೇವಃವಾಸು-ದೇವ-ನೆಂದು
ವಾಸು-ದೇವಃವಾಸು-ದೇವ-ರೂಪ-ದಿಂದಿ-ರುವ
ವಾಸು-ದೇವಃಸಾಕ್ಷಾತ್
ವಾಸು-ದೇ-ವನು
ವಾಸು-ದೇವ-ನೆಂದು
ವಾಸು-ದೇವ-ನೆಂಬ
ವಾಸು-ದೇವ-ರೂಪ-ದಿಂದ
ವಾಸು-ದೇವ-ರೂಪೀ
ವಾಸು-ದೇವ-ರೂಪೀ-ವಾಸು-ದೇವ-ರೂಪ-ಉಳ್ಳ
ವಾಸು-ದೇವಾಖ್ಯಂ
ವಾಸು-ದೇವಾಭಿದಂ
ವಾಸು-ದೇವೋ
ವಾಸೋ
ವಾಹನ-ನಾಗಿ
ವಾಹನ-ಮಿತಿ
ವಾಽಯಃ
ವಿಕಾರ
ವಿಕಾರ-ರಹಿತನು
ವಿಕಾರ-ವಿಲ್ಲದೇ
ವಿಕಾರವೇ
ವಿಕಾರಸ್ವ-ರೂಪವೇ
ವಿಕಾರಾಂಗೀ-ಕಾರಾತ್ಅ
ವಿಕಾರಾಂಗೀ-ಕಾರಾತ್ನೃ
ವಿಕಾರಿತ್ವ
ವಿಕ್ರ-ಮತಃಮಹಾ-ಸಾಮರ್ಥ್ಯ-ದಿಂದ
ವಿಕ್ರ-ಮತಃವಾಮ-ಪಾದಾಂಗುಷ್ಠನಖ-ನಿರ್ಭಿನ್ನೋರ್ಧ್ವಾಂಡಕಟಾಹವಿ-ವರೇಣಾಂತಃ
ವಿಗಾಹ್ಯ
ವಿಗ್ರಹಃವಿಗ್ರಹವು
ವಿಗ್ರಹಃವಿಗ್ರಹವೂ
ವಿಗ್ರಹ-ಗಳನ್ನು
ವಿಗ್ರಹ-ಗಳು
ವಿಗ್ರಹವು
ವಿಗ್ರಹಶ್ಚ
ವಿಗ್ರಹೋ
ವಿಘ್ನ-ಗಳು
ವಿಚಾರ-ಗಳು
ವಿಚಾರವು
ವಿಚ್ಛತ್ತಿ
ವಿಜಾ-ನಾತ್
ವಿಜಾ-ನಾತ್ಸ-ಪಿತುಷ್ಟಿತಾ
ವಿಜ್ಞೇಯಂ
ವಿದಿ
ವಿದ್ಯಂತೇ
ವಿದ್ಯಂತೇ-ಇರು-ವುದಿಲ್ಲವೋ
ವಿದ್ಯಂತೇ-ಇಲ್ಲವೋ
ವಿದ್ಯತೇ
ವಿದ್ಯತೇ-ಇರು-ವುದಿಲ್ಲವೋ
ವಿದ್ಯತೇ-ಇಲ್ಲ
ವಿದ್ಯತೇ-ಇಲ್ಲವೋ
ವಿದ್ಯದೇಪಿ
ವಿದ್ಯಮಾ-ನಸ್ಯ
ವಿದ್ಯಮಾ-ನಸ್ಯ-ಇ-ರುವ
ವಿದ್ಯೆ
ವಿದ್ವಾಂಸಃಜ್ಞಾನಿ-ಗಳಿಗೆ
ವಿದ್ವಾಂಸೋ
ವಿಧ
ವಿಧ-ವಾದ
ವಿಧಾತಾ
ವಿಧಾಯ
ವಿಧಿಂ
ವಿಧಿಯ
ವಿಧೇಃಬ್ರಹ್ಮ-ದೇವರ
ವಿಧೇರ್ದೇಹಾಪ-ಗಮನೇ
ವಿಧೋಽರ್ಜುನ
ವಿನಃ
ವಿನಶ್ಯತಿ
ವಿನಾ
ವಿನಾಆ
ವಿನಾಶ್ಯ
ವಿನಿಶ್ಚ-ಯಾತ್
ವಿಪರೀ-ತೇನ
ವಿಪಾಪಂ
ವಿಪಾಪಂದೋಷ-ವರ್ಜಿತ-ನಾದ
ವಿಭಜತ್ಯಸೌ
ವಿಭಜಿಸ
ವಿಭಜಿಸ-ಲಾಗಿದೆ
ವಿಭಜಿಸಿ
ವಿಭಾಗ
ವಿಭಾಗಃನಾರಾಯ
ವಿಭಾಗಿಸ-ಬಹುದು
ವಿಭಾಗಿಸಲ್ಪಡುತ್ತದೆ
ವಿಮರ್ಶೆ-ಯಿಂದ
ವಿಮಲಾಕೃತಿಃ
ವಿಮಾನಂ
ವಿರಂಚ
ವಿರಂಚ-ನಾಮಾ
ವಿರಚಿತ
ವಿರಜಾ
ವಿರಜಾ-ತೋಯಂ
ವಿರಜಾ-ನದಿ
ವಿರಜಾ-ನದಿ-ಯಲ್ಲಿ
ವಿರಜಾ-ನದಿಸ್ನಾ-ನದ
ವಿರಜಾ-ನದೀ
ವಿರಜಾ-ನದೀ-ಎಂದರೆ
ವಿರಜಾ-ನದೀ-ವಿರಜಾ-ನದೀ
ವಿರಜಾಯಾಂ
ವಿರಾಜಾಯಾಂ
ವಿರಾಜಿಸುತ್ತಾರೆ
ವಿರಾಟ್ರೂಪೀ
ವಿರಾಡ್ರೂಪ-ದಿಂದ
ವಿರಾಡ್ರೂಪೀ
ವಿರಾಡ್ರೂಪೀ-ವಿರಾಟ್ರೂಪ
ವಿರುದ್ದ
ವಿರುದ್ದ-ವಾ-ದುದು
ವಿರುದ್ದಾರ್ಥಕ
ವಿರುದ್ದಾರ್ಥ-ವಾಚ-ಕಃಅ
ವಿರುದ್ಧ-ವಾದ
ವಿರುದ್ಧ-ಸರ್ವ-ದೋಷ-ವಿರುದ್ಧ
ವಿರುದ್ಧಾರ್ಥ-ವನ್ನು
ವಿರುದ್ಧಾರ್ಥ-ವಾಚಿತ್ವಾತ್
ವಿರುದ್ಧಾರ್ಥ-ವಾಚ್ಯ-ಕಾರ-ಸಮಾ-ನಾರ್ಥಕ
ವಿರೋಧ-ವಾದ
ವಿರೋಧ-ವಿಲ್ಲ
ವಿಲಕ್ಷ್ಯ
ವಿವರ-ಣೆ-ಯಲ್ಲಿ
ವಿವರಿಸ-ಲಾಗಿದೆ
ವಿವರಿಸಲ್ಪಡುತ್ತದೆ
ವಿವ-ರಿಸಿ
ವಿವರೇಣಉಂಟಾದ
ವಿವಿಕ್ಷಿತ-ವಾಗಿದೆ
ವಿವಿಧ
ವಿವಿಧಾರ್ಥ-ಗಳನ್ನು
ವಿವೇಕ
ವಿವೇಶ
ವಿಶಂತಿ
ವಿಶಂತಿಪ್ರವೇಶಿಸುತ್ತಾರೆ
ವಿಶಂತ್ಯೇತೇ
ವಿಶತಿಕಾಮ-ನಲ್ಲಿ
ವಿಶತಿಪ್ರವೇಶ-ಮಾಡುತ್ತದೆ
ವಿಶತೇ
ವಿಶತ್ಯಸಸೌ
ವಿಶಿಷ್ಟ-ಗುಣಾ-ನಾಚಷ್ಟೇ-ಆ-ದರೆ
ವಿಶೇಷ
ವಿಶೇಷಜ್ಞಾನ-ವುಳ್ಳವ-ರಾದು-ದ-ರಿಂದ
ವಿಶೇಷ-ವಾಗಿ
ವಿಶೇಷ-ವಾದ
ವಿಶೇಷವು
ವಿಶೇಷ-ಸಮಸ್ತ-ವಾದ
ವಿಶೇಷಾಂಶ
ವಿಶೇಷಾಂಶಃ
ವಿಶೇಷಾನು-ಗತಿಭ್ಯಾಮ್
ವಿಶೇಷಾರ್ಥ-ಗಳೂ
ವಿಶೇಷೇಣಾ-ಪರೋಷ್ಯಾಚ್ಚ
ವಿಶೋಕಃ
ವಿಶ್ವ
ವಿಶ್ವಂ
ವಿಶ್ವಂಜಗತ್ತನ್ನು
ವಿಶ್ವ-ತೋ-ಮುಖಂ
ವಿಶ್ವ-ಮಸ್ಯ
ವಿಶ್ವ-ರೂಪ-ದರ್ಶನ
ವಿಶ್ವ-ರೂಪ-ದಿಂದ
ವಿಶ್ವ-ವನ್ನು
ವಿಶ್ವ-ವನ್ನೇ
ವಿಶ್ವವು
ವಿಶ್ವವ್ಯಾಪಕ-ವಾ-ದುದು
ವಿಶ್ವಾ
ವಿಶ್ವಾ-ದಿ-ರೂಪ-ಗಳಿಂದ
ವಿಶ್ವಾ-ದಿ-ರೂಪೀ
ವಿಶ್ವಾ-ದಿ-ರೂಪೀ-ವಿಶ್ವ
ವಿಷಜಂತು-ಗಳ
ವಿಷಯ
ವಿಷಯಕ
ವಿಷಯ-ಕ-ವಾದ
ವಿಷ-ಯಕ್ಕೆ
ವಿಷಯ-ಗಳನ್ನು
ವಿಷಯ-ಗಳೂ
ವಿಷಯ-ತಯಾ
ವಿಷಯ-ದಲ್ಲಿ
ವಿಷಯ-ನಾಗಿ
ವಿಷಯ-ನಾಗಿ-ರುವ
ವಿಷಯ-ನಾಗಿ-ರುವು-ದ-ರಿಂದ
ವಿಷಯ-ನಾಗು-ವುದಿಲ್ಲ-ವೆಂದೇ
ವಿಷಯ-ನಾಗು-ವುದಿಲ್ಲ-ವೆಂಬ
ವಿಷಯನು
ವಿಷಯ-ವನ್ನು
ವಿಷಯ-ವನ್ನೇ
ವಿಷಯವು
ವಿಷಯಾನುಪಸೇ-ವತೇ
ವಿಷಯಾನ್ಶುಭ-ಫಲ-ಗಳನ್ನು
ವಿಷ-ವನ್ನು
ವಿಷ್ಣು
ವಿಷ್ಣುಃ
ವಿಷ್ಣುಃನಾಶ-ರಹಿತ-ನಾ-ದ-ವನು
ವಿಷ್ಣುದ್ವೇಷಿ-ಗಳಿಗೆ
ವಿಷ್ಣುನಾ
ವಿಷ್ಣು-ಪುರಾಣ
ವಿಷ್ಣು-ಪುರಾ-ಣ-ದಲ್ಲಿ
ವಿಷ್ಣು-ರಹಸ್ಯ
ವಿಷ್ಣು-ರಿತಿ
ವಿಷ್ಣುರ್ಹಿ
ವಿಷ್ಣು-ಲೋಕ-ಗಳು
ವಿಷ್ಣು-ವನ್ನು
ವಿಷ್ಣು-ವಿನ
ವಿಷ್ಣುವು
ವಿಷ್ಣುವೇ
ವಿಷ್ಣೋ
ವಿಷ್ಣೋಃ
ವಿಷ್ಣೋಃವಿಷ್ಣು-ವಿನ
ವಿಷ್ಣೋರ್ನಾರಾ-ಯ-ಣಾಭಿದಾ
ವಿಷ್ಣೋರ್ವಾಯೋ-ರನಂತಸ್ಯ
ವಿಷ್ಣೋ-ವಿಷ್ಣು-ವಿನ
ವಿಷ್ಣೋಸ್ತಿಷ್ಠತಿ
ವಿಷ್ಣ್ವ-ನಂತರಂ
ವಿಹರಿ-ಸುತ್ತೇನೆ
ವಿಹಾರ
ವೃಣುತೇ
ವೃಥಾ
ವೃದ್ಧಿ-ಗಳಿಗೆ
ವೃದ್ಧಿಯುಂಟು
ವೃದ್ಧಿ-ಹೊಂದು-ವು-ದೆಂದು
ವೆಂಕಟಾಚಲ
ವೆಂಕಟಾದ್ರಿ-ಯಲ್ಲಿ-ರುವ
ವೆಂಕಟೇಶನ
ವೆಂಕಟೇಶ-ಮಹಾತ್ಮೆ
ವೆಂಕಟೇಶಮಾಹಾತ್ಮ್ಯೇ
ವೆಂಕಟೇಶಸ್ಯ
ವೇ
ವೇಂಕಟಾದ್ರಿಂ
ವೇಣುಪಲ್ಲಿ
ವೇದ
ವೇದ-ಗಳ
ವೇದ-ಗಳನ್ನು
ವೇದ-ಗಳಲ್ಲಿ
ವೇದ-ಗಳಿಂದ
ವೇದ-ಗಳು
ವೇದ-ಗಳೂ
ವೇದ-ದಲ್ಲಿ
ವೇದ-ದಿಂದ
ವೇದದ್ವೇಷಿ
ವೇದ-ನರ-ಕ-ದಲ್ಲಿ-ರುವ-ವ-ರಿಗೆ
ವೇದ-ನಿಂದಕಃ
ವೇದ-ನಿಂದ-ಕರೂ
ವೇದಪ್ರತಿ-ಪಾದ್ಯ-ನೆಂಬ
ವೇದ-ಮಂತ್ರ-ಗಳನ್ನು
ವೇದ-ವ-ಚನೇ
ವೇದವ್ಯಾ-ಸ-ರಿಂದ
ವೇದವ್ಯಾಸರು
ವೇದವ್ಯಾಸ-ರೆಂದು
ವೇದಾಃ
ವೇದಾಧಿ-ಕಾರಿ-ಗಳು
ವೇದಾಧ್ಯ-ಯನ
ವೇದಾಧ್ಯಯಯಹೀನಶ್ಚ
ವೇದಾರ್ಥ-ಗಳನ್ನು
ವೇದೇ
ವೇದೇಷ್ಟಧಿ-ಕೃತಾಸ್ತೈಸ್ತೈ-ರೇವಾಭಿ-ಪೂಜ್ಯತೇ
ವೇದೈಃ
ವೇದೈಃಯತಃಯಾವ
ವೇದೈಃವೇದ-ಗಳಿಂದ
ವೇದೋತ್ಪನ್ನತ್ವಾದ್ವಾ
ವೇದ್ಯ-ವಾಗುತ್ತದೆ
ವೇಶ್ಮ-ಕಮಲ-ವೆಂಬ
ವೇಶ್ಮ-ದಹರೋಸ್ಕಿನ್ನಂತರಾ-ಕಾಶಃ
ವೈ
ವೈಕುಂಠ
ವೈಕುಂಠಂ
ವೈಕುಂಠಂವೈಕುಂಠ-ಲೋಕ
ವೈಕುಂಠ-ಗಳು
ವೈಕುಂಠದ
ವೈಕುಂಠ-ದಲ್ಲಿ
ವೈಕುಂಠ-ದಲ್ಲಿ-ರುವ
ವೈಕುಂಠ-ನಗರ-ವನ್ನು
ವೈಕುಂಠ-ಪರಿಘಾಕೃತ್ಯಾಂ
ವೈಕುಂಠ-ಮುತ್ತಮಂ
ವೈಕುಂಠ-ಲೋಕ
ವೈಕುಂಠ-ಲೋಕ-ದಲ್ಲಿ
ವೈಕುಂಠ-ಲೋಕ-ವನ್ನು
ವೈಕುಂಠಶ್ರೇಷ್ಠ-ವಾದ
ವೈದಿಕ-ಶಬ್ದ-ವಾದು-ದ-ರಿಂದ
ವೈನೀರು
ವೈರಾಗ್ಯ
ವೈರಾಗ್ಯ-ಗಳೆಂಬ
ವೈರಾಗ್ಯಾದಿ-ಗಳಿಂದ
ವೈರಾಜ
ವೈರಾಜ-ದೇಹ-ವನ್ನು
ವೈರಾಜ-ನೆಂದು
ವೈರಿನಿರಸನ-ಸಾ-ಧನ-ಮಿತಿ
ವೈಶ್ವಾ-ನರ
ವೈಷಮ್ಯ
ವೈಷಮ್ಯ-ನೈರ್ಘೃಣ್ಯೇ
ವೈಷಮ್ಯಾದಿ
ವೈಷಮ್ಯಾದಿ-ದೋಷ-ಗಳು
ವೈಷ್ಣವಾ
ವ್ಯಕ್ತ
ವ್ಯಕ್ತ-ಮಿತಿ
ವ್ಯಕ್ತ-ಮಿದಂ
ವ್ಯಕ್ತಾ
ವ್ಯಕ್ತಿ-ಗಳು
ವ್ಯಕ್ತಿಯೇ
ವ್ಯಕ್ತ್ಯಾ-ದಿಕಂಸ್ವ-ರೂಪ-ಭೂತ
ವ್ಯರ್ಥ-ಮಾಂಸ-ಭುಕ್
ವ್ಯರ್ಥ-ಮಾಡ-ಬೇಡಿರಿ
ವ್ಯವಸ್ಥಿತಾಃ
ವ್ಯಾಕರಣ
ವ್ಯಾಕರ-ಣ-ಸೂತ್ರದ
ವ್ಯಾಕರ-ಣ-ಸೂತ್ರಾನು-ಸಾರಿ-ಯಾಗಿ
ವ್ಯಾಖ್ಯಾತಂ
ವ್ಯಾಖ್ಯಾನ
ವ್ಯಾಖ್ಯಾನದ
ವ್ಯಾಖ್ಯಾನ-ಮಾಡಿ-ರುವ
ವ್ಯಾಖ್ಯಾನ-ಮಾಡುವ
ವ್ಯಾಖ್ಯಾನ-ರೂಪ-ವಾದ
ವ್ಯಾಖ್ಯಾನ-ರೂಪವೇ
ವ್ಯಾಪಕತ್ವಕ್ಕೆ
ವ್ಯಾಪತುಃಘರ್ಮಾ
ವ್ಯಾಪತುಃವ್ಯಾಪಿ-ಸಿ-ಕೊಂಡಿವೆ
ವ್ಯಾಪಾರ-ಗಳನ್ನು
ವ್ಯಾಪಾರ-ಗಳು
ವ್ಯಾಪಾರ-ವನ್ನೂ
ವ್ಯಾಪಾರವೂ
ವ್ಯಾಪಿ-ಸಲ್ಪಟ್ಟಿ-ರುತ್ತದೆ
ವ್ಯಾಪ್ತತ್ವಕ್ಕೆ
ವ್ಯಾಪ್ತ-ನಾಗಿ
ವ್ಯಾಪ್ತ-ನಾದ
ವ್ಯಾಪ್ತಿ-ಯನ್ನೂ
ವ್ಯಾಪ್ಯ
ವ್ಯಾಸಃ
ವ್ಯಾಸಮುಷ್ಟಿ-ಗಳು
ವ್ಯಾಸಮುಷ್ಟಿ-ಗಳೆಂಬ
ವ್ಯುತ್ಪತ್ತಿಃನಾರನೂ
ವ್ಯುತ್ಪತ್ತಿ-ಗಳಿಂದ
ವ್ಯುತ್ಪತ್ತ್ಯಾ
ವ್ಯುತ್ಪತ್ತ್ಯಾ-ನರೋ
ವ್ಯುತ್ಪತ್ತ್ಯಾ-ಶಬ್ದ-ರ-ಚನೆ-ಯಿಂದ
ವ್ಯುತ್ಪತ್ತ್ಯಾಶ್ರೀ-ನಿವಾಸತ್ವಾತ್
ವ್ರಜತ್ಯೇಕಃ
ವ್ರತಾಚರಣೆ-ಯಲ್ಲಿಯೂ
ಶಂಖ
ಶಂಖ-ದಿಂದ
ಶಕ್ಕೋ
ಶಕ್ತನಿದ್ದರೂ
ಶಕ್ತರಲ್ಲ
ಶಕ್ಯ
ಶಕ್ಯಃಸಾಧ್ಯವೇ
ಶಕ್ಯತೇ
ಶಕ್ಯನು
ಶಕ್ರಂ
ಶಠರು
ಶಠಾ
ಶತ್ರು-ನಾಶ-ಕರ-ವಾದ
ಶತ್ರುಸೈನ್ಯ-ವನ್ನು
ಶಬ್ದ
ಶಬ್ದಃ
ಶಬ್ದಃನಾರಾ-ಯಣ
ಶಬ್ದಕ್ಕೆ
ಶಬ್ದ-ಗಳ
ಶಬ್ದ-ಗಳಲ್ಲಿ
ಶಬ್ದ-ಗಳಿಗೆ
ಶಬ್ದ-ಗಳು
ಶಬ್ದದ
ಶಬ್ದ-ದಂತೆ
ಶಬ್ದ-ದಿಂದ
ಶಬ್ದ-ದೊಡನೆ
ಶಬ್ದ-ರ-ಚನಾ
ಶಬ್ದ-ರ-ಚನಾಪ್ರ-ಕಾರ-ವಾಗಿ
ಶಬ್ದ-ರ-ಚನೆ-ಯಿಂದ
ಶಬ್ದ-ರ-ಚನೆಯು
ಶಬ್ದ-ವತ್
ಶಬ್ದ-ವತ್ದೇವ-ದತ್ತ
ಶಬ್ದ-ವನ್ನು
ಶಬ್ದ-ವಾಗಿದೆ
ಶಬ್ದ-ವಾಚ್ಯ
ಶಬ್ದ-ವಾಚ್ಯ-ನಾದ
ಶಬ್ದ-ವಾಚ್ಯ-ರಿಗೆ
ಶಬ್ದವು
ಶಬ್ದಾತ್ಶ್ರುತಿಯು
ಶಬ್ದಾನಾಂ
ಶಬ್ದಾರ್ಥಃ
ಶಬ್ದಾರ್ಥದ
ಶಬ್ದಾರ್ಥವು
ಶಬ್ದೋ
ಶಬ್ದೋ-ನರ
ಶಯನಂ
ಶಯನಂಆಶ್ರಯಃಹಾಸಿಗೆ
ಶಯನ-ಮಾಡಿ-ರುತ್ತಾನೆ
ಶಯನ-ಮಾಡುವ
ಶರಣಂ
ಶರಣುಹೊಂದ-ಬೇಕು
ಶರೀರ
ಶರೀರಂ
ಶರೀರಂಶರೀರವು
ಶರೀರಕಾಃ
ಶರೀರಕ್ಕೆ
ಶರೀರ-ಗಳಲ್ಲಿ
ಶರೀರ-ಗಳಲ್ಲಿಯೂ
ಶರೀರ-ಗಳೇ
ಶರೀರದ
ಶರೀರ-ದಲ್ಲಿ
ಶರೀರ-ರೂಪ
ಶರೀರ-ವನ್ನು
ಶರೀರ-ವಾಗಿ
ಶರೀರ-ವಾಗಿದೆ
ಶರೀರ-ವಾಗಿ-ರುವು-ದ-ರಿಂದ
ಶರೀರವು
ಶರೀರ-ವೆನಿಸುವ
ಶರೀರವೇ
ಶರೀರ-ಸಂಬಂಧ
ಶರೀರಿಣಃ
ಶರೀರೇ
ಶವ
ಶವಾದ್ಯುಪಹತತೌ
ಶಶ್ವದೇಕಪ್ರ-ಕಾರ-ವಾ-ದುದು
ಶಾಂಡಿಲ್ಯ-ತತ್ವ-ವಾಯು-ಪುರಾಣ
ಶಾಂತಂ
ಶಾಂತಂಶಾಂತ
ಶಾಲಗ್ರಾಮ
ಶಾಲಗ್ರಾಮಕ್ಕೆ
ಶಾಲಗ್ರಾಮ-ವನ್ನು
ಶಾಲಗ್ರಾಮ-ಶಿಲೆಯಲ್ಲಿ
ಶಾಲಿಗ್ರಾಮ-ದಲ್ಲಿ
ಶಾಶ್ವತಾಶ್ಚ
ಶಾಸ್ತಂ
ಶಾಸ್ತ್ರ
ಶಾಸ್ತ್ರಂ
ಶಾಸ್ತ್ರಗ್ರಂಥ-ಗಳಿಗೆ
ಶಾಸ್ತ್ರದ
ಶಾಸ್ತ್ರ-ದಲ್ಲಿ
ಶಾಸ್ತ್ರ-ವಿಧಿಯನ್ನರಿತು
ಶಾಸ್ತ್ರ-ವೆಂದು
ಶಿಕ್ಷೆ-ಯನ್ನು
ಶಿರಸ್ಸಿ-ನಲ್ಲಿ
ಶಿರಸ್ಸು
ಶಿಲಾ
ಶಿಲಾಂ
ಶಿಲಾಪ್ರತಿ-ಮೆ-ಗಳನ್ನು
ಶಿಲಾಪ್ರತಿ-ಮೆಯಂತಿ-ರುವ
ಶಿಲಾಯಾಂ
ಶಿಲಾ-ರೂಪ-ದವಿಗ್ರಹ-ಗಳಲ್ಲಿ
ಶಿವಂ
ಶಿವಮಂಗಳ-ಕರ-ವಾದ
ಶಿಶು-ರೂಪ-ದಿಂದ
ಶಿಶು-ವಾಗಿ
ಶಿಷ್ಯ-ರಾದ
ಶಿಷ್ಯ-ರಿಗೆ
ಶಿಷ್ಯ-ರೊಡನೆ
ಶಿಷ್ಯಾಣಾಂ
ಶೀಘ್ರಂ
ಶೀಘ್ರ-ವಾಗಿ
ಶೀಘ್ರ-ವಾಗಿಯೂ
ಶುಚಿಃ
ಶುದ್ಧ
ಶುದ್ಧ-ಶಿಲಾ-ತಪ್ರತಿಮಾಃ
ಶುದ್ಧ-ಶಿಲಾತ್ಮಪ್ರತಿ-ಮಾಃಪವಿತ್ರ-ವಾದ
ಶುದ್ಧ-ಸತ್ವಸ್ಥಾಃ
ಶುದ್ಧ-ಸತ್ವಸ್ವ-ರೂಪ-ದವು
ಶುಭ
ಶುಭಂ
ಶುಭ-ಗುಣ-ಗಳಿಂದ
ಶುಭ-ಫಲ-ಗಳನ್ನು
ಶುಭ-ಫಲ-ವನ್ನು
ಶುಭ-ವಾಗಲಿ
ಶುಭ-ವಿಷಯ-ಗಳನ್ನು
ಶುಭ-ವಿಷಯ-ಭೋಗಾಶ್ರಯಃ
ಶುಭಾನಿ
ಶುಭಾ-ಶುಭಪ್ರದ-ವಾದ
ಶುಭಾ-ಶುಭಾಭ್ಯಾಂ
ಶೂದ್ರ
ಶೂದ್ರ-ಸಂಸ್ಪರ್ಶ-ನಾತ್ಸದ್ಯಃ
ಶೃಣುತ
ಶೃಣುತ-ಕೇಳಿರಿ
ಶೇತೇ
ಶೇಷ
ಶೇಷಃತು-ಶೇಷ-ದೇವ-ರಾದರೋ
ಶೇಷಃಮೇಲೆ
ಶೇಷ-ಗರುಡ-ಶರೀರಪ್ರವೇಶ-ವನ್ನು
ಶೇಷ-ದೇವಂ
ಶೇಷ-ದೇವಂಶೇಷ-ದೇವ-ರನ್ನು
ಶೇಷ-ದೇವರ
ಶೇಷ-ದೇವ-ರದೇ
ಶೇಷ-ದೇವ-ರಿಗೆ
ಶೇಷ-ದೇವರು
ಶೇಷ-ಪತ್ನಿ-ಯನ್ನು
ಶೇಷ-ಭೋಗ-ಶಾಯಿತ್ವಾನ್ನಾರಾ-ಯಣಃ
ಶೇಷ-ಮಾರ್ಗ
ಶೇಷಸ್ತು
ಶೇಷಸ್ಯ
ಶೇಷೋ
ಶೋಕ-ರಹಿತ-ನಾದ
ಶೋತೇಂದ್ರಿ-ಯವೇ
ಶೋತ್ರೇ
ಶೋಧಯಾಮಿ
ಶೋಧಯಾಮಿ-ನಿರ್ದೋಷ-ರಾಗು-ವರಂತೆ
ಶೋಧಯಾಮಿ-ಶುದ್ಧ-ನನ್ನಾಗಿ
ಶೋಭಿಸುವ
ಶೌರಿ
ಶೌರಿಃನಾರಾ-ಯಣ
ಶ್ಚ
ಶ್ಚಾಸೌ
ಶ್ಚೇತಿ
ಶ್ರವಣ
ಶ್ರವಣ-ಮಾಡಿ
ಶ್ರವಣ-ಮಾತ್ರೇಣ
ಶ್ರವಣಾದಿ
ಶ್ರವಣಾದಿ-ಗಳಿರ-ಬೇಕು
ಶ್ರಿಯಃ
ಶ್ರೀ
ಶ್ರೀಃಮಹಾ-ಲಕ್ಷ್ಮೀ-ದೇವಿಯರು
ಶ್ರೀಕೃಷ್ಣ
ಶ್ರೀಕೃಷ್ಣನ
ಶ್ರೀಕೃಷ್ಣ-ನಿಂದ
ಶ್ರೀಕೃಷ್ಣ-ನಿಗೂ
ಶ್ರೀಕೃಷ್ಣ-ನಿಗೆ
ಶ್ರೀಕೃಷ್ಣನು
ಶ್ರೀಕೃಷ್ಣನೇ
ಶ್ರೀಜಯ-ತೀರ್ಥ-ರೆಂಬ
ಶ್ರೀತೌಜ
ಶ್ರೀದೇವಿಯು
ಶ್ರೀದೇವೀ
ಶ್ರೀದೇವೀ-ಮಹಾ-ಲಕ್ಷ್ಮೀ
ಶ್ರೀದೇವ್ಯಾಂಮಾಯಾ
ಶ್ರೀದೇವ್ಯಾಃ
ಶ್ರೀನಾರಾ-ಯ-ಣನ
ಶ್ರೀನಾರಾ-ಯ-ಣ-ನನ್ನೇ
ಶ್ರೀನಾರಾ-ಯ-ಣ-ನಿಗೆ
ಶ್ರೀನಾರಾ-ಯ-ಣ-ಶಬ್ದಾರ್ಥಃ
ಶ್ರೀನಾರಾ-ಯ-ಣ-ಶಬ್ದಾರ್ಥ-ವೆಂಬ
ಶ್ರೀನಿಕೇತನೇ
ಶ್ರೀನಿವಾಸ
ಶ್ರೀನಿವಾಸ-ತೀರ್ಥ
ಶ್ರೀನಿವಾಸ-ತೀರ್ಥ-ರಚಿತಃ
ಶ್ರೀನಿವಾಸ-ತೀರ್ಥ-ರಿಂದ
ಶ್ರೀನಿವಾಸತ್ವಾತ್ಶ್ರೀ-ದೇ-ವಿಗೆ
ಶ್ರೀನಿವಾಸನ
ಶ್ರೀನಿವಾಸ-ನನ್ನೂ
ಶ್ರೀನಿವಾಸ-ನಲ್ಲಿಯೂ
ಶ್ರೀನಿವಾ-ಸೇನ
ಶ್ರೀನಿವಾ-ಸೇನಶ್ರೀ-ನಿವಾಸ-ತೀರ್ಥ-ರಿಂದ
ಶ್ರೀಪಾದಂಗಳ-ವರ
ಶ್ರೀಭಾಗವು
ಶ್ರೀಭಾಗಾಃ
ಶ್ರೀಭೀಮ-ಸೇನ-ದೇವ-ರನ್ನು
ಶ್ರೀಮದಾಚಾರ್ಯ-ರಿಂದ
ಶ್ರೀಮದಾಚಾರ್ಯ-ರಿಗೆ
ಶ್ರೀಮದಾಚಾರ್ಯರು
ಶ್ರೀಮದಾಚಾರ್ಯರೇ
ಶ್ರೀಮದಾನಂದ-ತೀರ್ಥಃಶ್ರೀ-ಮದಾನಂದ-ತೀರ್ಥರು
ಶ್ರೀಮದಾನಂದ-ತೀರ್ಥ-ರಿಗೆ
ಶ್ರೀಮದಾನಂದ-ತೀರ್ಥರು
ಶ್ರೀಮದುತ್ತ-ರಾದಿ
ಶ್ರೀಮದ್ಯಾದ-ವರ್ಯಾಂತೇವಾಸಿ
ಶ್ರೀಮದ್ಯಾದ-ವಾರ್ಯರ
ಶ್ರೀಮದ್ರಘೋತ್ತಮ-ತೀರ್ಥ
ಶ್ರೀಮದ್ರಘೋತ್ತಮ-ತೀರ್ಥ-ಕರ-ಕಮಲ-ಸಂಜಾತ
ಶ್ರೀಮದ್ವಾಚಾರ್ಯರ
ಶ್ರೀಮದ್ವೇದೇ-ಶ-ತೀರ್ಥ
ಶ್ರೀಮಧ್ವಾಚಾರ್ಯ-ರನ್ನು
ಶ್ರೀಮಧ್ವೇಶಾರ್ಪಣಮಸ್ತು
ಶ್ರೀಮನ್ನಾರಾ-ಯ-ಣನ
ಶ್ರೀಮನ್ನಾರಾ-ಯ-ಣ-ನನ್ನು
ಶ್ರೀಮನ್ನಾರಾ-ಯ-ಣನು
ಶ್ರೀಮನ್ನಾರಾ-ಯ-ಣನೇ
ಶ್ರೀಮನ್ಮಹಾ-ಭಾರತ
ಶ್ರೀಮನ್ಮಹಾ-ಭಾರತ-ತಾತ್ಪರ್ಯ
ಶ್ರೀಮನ್ಮಹಾ-ಭಾರತ-ತಾತ್ಪರ್ಯ-ನಿರ್ಣಯ-ದಲ್ಲಿ
ಶ್ರೀಮಹಾ-ಲಕ್ಷ್ಮಿಯ
ಶ್ರೀಮಹಾ-ಲಕ್ಷ್ಮೀ
ಶ್ರೀಮುಖ್ಯಪ್ರಾಣ-ನಲ್ಲಿ
ಶ್ರೀಮುಖ್ಯಪ್ರಾಣ-ನಿಗೆ
ಶ್ರೀಮುಖ್ಯಪ್ರಾ-ಣನು
ಶ್ರೀಮುಷ್ಠಂ
ಶ್ರೀಮುಷ್ಣ
ಶ್ರೀಯುತರ
ಶ್ರೀರಂಗ
ಶ್ರೀರಂಗ-ದಲ್ಲಿ-ರುವ
ಶ್ರೀರಂಗ-ನಾಥ
ಶ್ರೀರಂಗ-ನಾಥನ
ಶ್ರೀರಂಗ-ನಾಥ-ನಲ್ಲಿಯೂ
ಶ್ರೀರಂಗಪ್ರತಿ-ಮಾಯಾಂ
ಶ್ರೀರಂಗ-ಮಾಹಾತ್ಮ್ಯೇಶ್ರೀ-ರಂಗ-ಮಹಾತ್ಮೆ
ಶ್ರೀರಂಗ-ಮಾಹಾತ್ಮ್ಯೇಶ್ರೀ-ರಂಗ-ಮಹಾತ್ಮೆ-ಯೆಂಬ
ಶ್ರೀರಂಗ-ವೆಂಕಟೇಶಾದಿ
ಶ್ರೀರಂಗ-ವೆಂಕಟೇಶಾದಿಪ್ರತಿ-ಮಾ-ತತ್ಪೂಜಿ-ತತ್ವೇನಆ
ಶ್ರೀರಾಮ-ಚಂದ್ರ-ದೇವರು
ಶ್ರೀರಾಮನ
ಶ್ರೀರಾಮನು
ಶ್ರೀರುದೀರಿತಾ
ಶ್ರೀರೇವ
ಶ್ರೀವಾಯು-ದೇವರ
ಶ್ರೀವಿಷ್ಣು-ವನ್ನು
ಶ್ರೀವಿಷ್ಣುವು
ಶ್ರೀವಿಷ್ಣುವೇ
ಶ್ರೀವೇದವ್ಯಾಸ-ದೇವ-ರಿಂದ
ಶ್ರೀವೇದವ್ಯಾಸ-ದೇವರು
ಶ್ರೀವೇದವ್ಯಾಸರು
ಶ್ರೀವೇದವ್ಯಾಸ-ರೂಪ-ದಿಂದ
ಶ್ರೀವೇದವ್ಯಾಸ-ರೂಪ-ದಿಂದಿ-ರುವ
ಶ್ರೀವೇದವ್ಯಾಸ-ರೂಪೀ
ಶ್ರೀಶೋ
ಶ್ರೀಸರ್ವಜ್ಞಾಚಾರ್ಯ
ಶ್ರೀಹರಿ
ಶ್ರೀಹ-ರಿಗೆ
ಶ್ರೀಹರಿಯ
ಶ್ರೀಹರಿ-ಯನ್ನು
ಶ್ರೀಹರಿ-ಯಲ್ಲಿ
ಶ್ರೀಹರಿ-ಯಲ್ಲಿಯೇ
ಶ್ರೀಹರಿ-ಯಿಂದ
ಶ್ರೀಹರಿ-ಯಿಂದಲೇ
ಶ್ರೀಹರಿಯು
ಶ್ರೀಹರಿಯೂ
ಶ್ರೀಹರಿಯೇ
ಶ್ರೀಹರಿ-ಯೊಡನೆ
ಶ್ರುತಿ
ಶ್ರುತಿಃ
ಶ್ರುತಿ-ಗಳನ್ನೂ
ಶ್ರುತಿ-ಗಳು
ಶ್ರುತಿ-ಗೀತಾ-ಯಾಂಶ್ರುತಿ-ಗೀತೆ-ಯಲ್ಲಿ
ಶ್ರುತಿ-ಗೀತಾ-ವಚ-ನಾತ್
ಶ್ರುತಿ-ಗೀತೆ-ಯಲ್ಲಿ
ಶ್ರುತಿಗೆ
ಶ್ರುತಿದ್ವಿಷಃ
ಶ್ರುತಿ-ಭಯಂಕರ-ವಾದ
ಶ್ರುತಿಯ
ಶ್ರುತಿ-ಯಲ್ಲಿ
ಶ್ರುತಿಯು
ಶ್ರುತಿಯೂ
ಶ್ರುತಿ-ರೇಷಾ
ಶ್ರುತೀಶ್ವರಃ
ಶ್ರುತೇಃ
ಶ್ರುತೇಃವೇದದ
ಶ್ರುತೇಃಶ್ರುತಿ-ರೀತಿ-ಯಾಗಿ
ಶ್ರುತೌಶ್ರುತಿ-ಯಲ್ಲಿ
ಶ್ರುತ್ಯಾ
ಶ್ರೇಷ್ಠ
ಶ್ರೇಷ್ಠ-ಗುಣ-ಗಳಿಂದ
ಶ್ರೇಷ್ಠ-ಗುಣ-ಪೂರ್ಣ
ಶ್ರೇಷ್ಠತೆ
ಶ್ರೇಷ್ಠ-ತೆ-ಯನ್ನು
ಶ್ರೇಷ್ಠ-ನಾಗಿ-ರುವ
ಶ್ರೇಷ್ಠ-ನಾಗಿ-ರುವ-ವನು
ಶ್ರೇಷ್ಠ-ನಾಗಿ-ರುವು-ದ-ರಿಂದ
ಶ್ರೇಷ್ಠನೂ
ಶ್ರೇಷ್ಠ-ನೆಂದೂ
ಶ್ರೇಷ್ಠ-ವಾದ
ಶ್ರೇಷ್ಠ-ವಾ-ದುದು
ಶ್ರೇಷ್ಠಸ್ಥಳ-ವಾದ
ಶ್ರೋತ್ರಂ
ಶ್ರೋತ್ರಕಿವಿ-ಯನ್ನೂ
ಶ್ರೋತ್ರಾದಿ
ಶ್ರೋತ್ರಾದೀಂದ್ರಿಯ
ಶ್ರೋತ್ರಿಯಂ
ಶ್ರೋತ್ರೇಂದ್ರಿಯ
ಶ್ಲಿಷ್ಯಂತೇ
ಶ್ಲಿಷ್ಯಂತೇ-ಅಂಟಿಕೊಳ್ಳು-ವುದಿಲ್ಲವೋ
ಶ್ಲಿಷ್ಯತೇ
ಶ್ಲಿಷ್ಯತೇ-ಲೇಪ-ವಾಗು-ವುದಿಲ್ಲ
ಶ್ಲೇಷಂ
ಶ್ಲೋಕದ
ಶ್ಲೋಕ-ದಲ್ಲಿನ
ಶ್ಲೋಕ-ದಿಂದ
ಶ್ವನಾರಾ-ಯ-ಣಃನಾರಾಯ
ಶ್ವಪಾಕಾಂತಾಸ್ತತ್ಪೂಜಾ-ಧಿ-ಕೃತಾ
ಶ್ವೇತಕೇತುವೇ
ಶ್ವೇತದ್ವೀಪ
ಶ್ವೇತಿ
ಷಟ್
ಷಟ್ಪ್ರಶ್ನ
ಷಡ್ಗುಣೈಶ್ಚರ್ಯ-ಪೂರ್ಣ-ನಾದ
ಷಡ್ಗುಣೈಶ್ವರ-ಪೂರ್ಣ-ನಾದ
ಷಡ್ಗುಣೈಶ್ವರ್ಯ
ಷಡ್ಗುಣೈಶ್ವರ್ಯ-ಪೂರ್ಣ-ನಾದ
ಷಡ್ಗುಣೈಶ್ವರ್ಯ-ಪೂರ್ಣ-ನಾ-ದನು
ಷಡ್ಗುಣೈಶ್ವರ್ಯ-ಪೂರ್ಣ-ನಾದ-ವನು
ಷ್ಟಂ
ಸ
ಸಂ
ಸಂಕರ್ಷಣ
ಸಂಕರ್ಷಣ-ರೂಪ-ಗಳಿಂದ
ಸಂಕರ್ಷಣ-ರೂಪೀ
ಸಂಕರ್ಷಣಶ್ಚ
ಸಂಕರ್ಷಣಾಚ್ಚಾಪಿ
ಸಂಕಲ್ಪ
ಸಂಕಲ್ಪದ
ಸಂಕಲ್ಪ-ದ-ಡಿ-ಯಲ್ಲಿ
ಸಂಕಲ್ಪಿಸಲಾ-ಗಿದ್ದರ
ಸಂಕ್ಷೇಪ-ವಾಗಿಯೂ
ಸಂಗ-ಮಾಡಿ
ಸಂಗ್ರಹ-ವಾಗಿ
ಸಂಘದ
ಸಂಘ-ದಿಂದ
ಸಂಘವು
ಸಂಘಸಂಸ್ಥೆ-ಗಳಿಂದ
ಸಂಘಸಂಸ್ಥೆ-ಗಳಿಗೆ
ಸಂಚಿತ
ಸಂಚಿತ-ಕರ್ಮದ
ಸಂಚಿತ-ಕರ್ಮವು
ಸಂಚಿತದ
ಸಂಚಿತ-ವಾಗಿ
ಸಂಚಿತ-ವೆಂದರೆ
ಸಂಚಿತಾನಿ
ಸಂತೋಷಂ
ಸಂತೋಷಂಆನಂದ-ವನ್ನು
ಸಂತೋಷ-ದಿಂದ
ಸಂತೋಷಪಡಲಿ
ಸಂದರ್ಭ-ದಲ್ಲಿ
ಸಂದೇಶ-ವನು
ಸಂದೇಹವೇ
ಸಂದೋಹ
ಸಂಧಿ-ಕಾಲ-ದಲ್ಲಿ
ಸಂಧಿ-ಯಿಂದ
ಸಂಪದ್ಯ-ಹೊಂದಿ
ಸಂಪದ್ಯಾವಿಹಾಯ
ಸಂಪನ್ನೋ
ಸಂಪರ್ಕ-ವಿಲ್ಲ
ಸಂಪರ್ಕ-ವಿಲ್ಲ-ದ-ವನು
ಸಂಪರ್ಕ-ಹೊಂದಿ-ದ-ವನು
ಸಂಪರ್ಕಿಸಿ
ಸಂಪಾದ್ಯ
ಸಂಪಾದ್ಯ-ಸಂಪಾದಿಸುವಂತೆ
ಸಂಪುಟ-ಗಳಲ್ಲಿ
ಸಂಪೂರ್ಣ
ಸಂಪೂರ್ಣಃ
ಸಂಪೂರ್ಣಜ್ಞಾನ
ಸಂಪೂರ್ಣಜ್ಞಾನ-ವುಳ್ಳ-ವರು
ಸಂಪೂರ್ಣ-ವಾಯಿತು
ಸಂಪೂರ್ಣೇ
ಸಂಪ್ರಾಪ್ತೇ
ಸಂಬಂಧ
ಸಂಬಂಧಃ
ಸಂಬಂಧ-ಉಳ್ಳದ್ದ-ರಿಂದ
ಸಂಬಂಧ-ಉಳ್ಳದ್ದು
ಸಂಬಂಧದ
ಸಂಬಂಧ-ದಿಂದ
ಸಂಬಂಧ-ಪಟ್ಟ
ಸಂಬಂಧ-ಪಟ್ಟದ್ದು
ಸಂಬಂಧ-ಪಟ್ಟ-ವು-ಗ-ಳಾದ
ಸಂಬಂಧ-ಪಟ್ಟ-ವು-ಗಳು
ಸಂಬಂಧ-ಪಟ್ಟಿ-ದುದು
ಸಂಬಂಧ-ಪಟ್ಟಿದ್ದು
ಸಂಬಂಧ-ಪಟ್ಟಿ-ರುವಂತಹುದು
ಸಂಬಂಧ-ಪಟ್ಟಿ-ರುವು-ದ-ರಿಂದಲೂ
ಸಂಬಂಧ-ಪಟ್ಟಿ-ರು-ವುದು
ಸಂಬಂಧ-ವನ್ನು
ಸಂಬಂಧ-ವಾ-ದುದು
ಸಂಬಂಧ-ವಿ-ರುವು-ದ-ರಿಂದ
ಸಂಬಂಧವು
ಸಂಬಂಧ-ವುಳ್ಳ
ಸಂಬಂಧ-ವೆಂದರೆ
ಸಂಬಂಧ-ಹೊಂದಿದ್ದ-ರಿಂದ
ಸಂಬಂಧ-ಹೊಂದಿ-ರುವ
ಸಂಬಂಧ-ಹೊಂದಿ-ರುವು-ದ-ರಿಂದ
ಸಂಬಂಧಿ
ಸಂಬಂಧಿ-ಅ-ವರ
ಸಂಬಂಧಿಆ
ಸಂಬಂಧಿ-ಚೇತನ
ಸಂಬಂಧಿ-ಚೇತನರ
ಸಂಬಂಧಿನೀ
ಸಂಬಂಧಿ-ಸಿದ
ಸಂಬಂಧಿ-ಸಿ-ದುದು
ಸಂಬಂಧೀ
ಸಂಭೃತಃಹುಟ್ಟಿ-ರುವಿ
ಸಂಭೃತೋ
ಸಂಯಮನೇ
ಸಂಯಮನೇ-ನರ-ಕ-ಗಳಲ್ಲಿ
ಸಂಯೋಜನೆ-ಯಿಂದ
ಸಂವತ್ಸರ-ಗಳ
ಸಂವತ್ಸರದ
ಸಂಶಯಃ
ಸಂಶಯಃಸಂದೇಹವೇ
ಸಂಸಾರ
ಸಂಸಾರಕ್ಕೆ
ಸಂಸಾರ-ದಲ್ಲಿ
ಸಂಸಾರ-ಪಾಶ-ದಿಂದ
ಸಂಸಾರ-ವೆಂಬ
ಸಂಸಾರಾಖ್ಯ
ಸಂಸ್ಕೃತ
ಸಂಸ್ಥಾಪಕ-ರಾದ
ಸಂಸ್ಥಿತೌ
ಸಂಹರಿ-ಸಿ-ದುದು
ಸಂಹಾರ
ಸಂಹಾರ-ರೂಪ-ಕ-ವಾಗಿ
ಸಃ
ಸಃಅಂತಹ-ವನು
ಸಃಅವನು
ಸಃಆ
ಸಅವನೇ
ಸಕರ್ಮಕ
ಸಕಲ
ಸಕಲ-ಗುಣಾ-ಕರ-ತಯಾ
ಸಕಲ-ವಸ್ತು-ಗಳಲ್ಲಿ
ಸಕಾಶಾತ್ತನ್ನ
ಸಖ
ಸಖಿತ್ವಾದಿನಾ
ಸಖೇ
ಸಖೇತಿ
ಸಚ್ಛಾಸ್ತ್ರ-ಗಳ
ಸಚ್ಛಾಸ್ತ್ರ-ಗಳನ್ನೂ
ಸಜ್ಜನ
ಸಜ್ಜ-ನರ
ಸಜ್ಜನ-ರಿಂದ
ಸಜ್ಜನ-ರಿಗೆ
ಸಜ್ಜನ-ರೆಂದೂ
ಸಜ್ಜ-ನಾನಾಂ
ಸತತ್ವರತ್ನಮಾಲಾ
ಸತಾ
ಸತಿ
ಸತಿಅ
ಸತಿ-ವಿಚಾರವು
ಸತಿ-ಹೋಗುತ್ತಿ-ರಲು
ಸತ್
ಸತ್ಕರ್ಮ-ಮಾಡಿ-ದ-ವ-ರಿಗೆ
ಸತ್ತತ್ವರತ್ನಮಾಲಾ
ಸತ್ತಾ-ಇ-ರುವಿಕೆ
ಸತ್ತಾದಿ-ಗಳನ್ನು
ಸತ್ತ್ವರ-ಜಸ್ತಮೋ
ಸತ್ಯಃ
ಸತ್ಯ-ಭೂತ-ವಾದ
ಸತ್ಯ-ಭೂತ-ವಾದುದೆಂದೂ
ಸತ್ಯ-ಶಬ್ದ-ವಾಚ್ಯ-ನಾದ
ಸತ್ವ-ಗುಣಾ-ವರಣ
ಸತ್ವ-ಭೂತ-ವಾದ
ಸತ್ವ-ರೂಪಾಃ
ಸತ್ವಸ್ಥ-ವಾಗಿವೆ
ಸದಾ
ಸದಾ-ಚಾ-ರಸಂಪನ್ನ-ರಾದ
ಸದಾ-ನಿತ್ಯ-ದಲ್ಲಿಯೂ
ಸದಾ-ಯಾವಾಗಲೂ
ಸದೃಶ
ಸದೃಶನೇ
ಸದೃಶ-ನೇ-ಅಂಶ-ದಂತೆ
ಸದೈವ
ಸದ್ಗುಣ-ಗಳಿಗೆ
ಸದ್ಧತಿ-ಯನ್ನೇ
ಸದ್ವತ
ಸನಾತನಃ
ಸನಾತನಃಗೀತಾ
ಸನ್
ಸನ್ನಪಿ
ಸನ್ನಾಶ-ರಹಿತ-ನಾದ
ಸನ್ನಿಧಾನ
ಸನ್ನಿಧಾನಂ
ಸನ್ನಿಧಾನತ್ವಾತ್
ಸನ್ನಿಧಾನತ್ವಾತ್ಸನ್ನಿಧಾನದ
ಸನ್ನಿಧಾನ-ದಿಂದ
ಸನ್ನಿಧಾನ-ದಿಂದಿ-ರುವ
ಸನ್ನಿಧಾನ-ಪದತಾಂ
ಸನ್ನಿಧಾನ-ವನ್ನಿಡಲು
ಸನ್ನಿಧಾನ-ವನ್ನು
ಸನ್ನಿಧಾನ-ವಿದೆ
ಸನ್ನಿಧಾನ-ವಿ-ರುವ
ಸನ್ನಿಧಾನವು
ಸನ್ನಿಧಾನಸ್ಥಾನಂ
ಸನ್ನಿಧಾನಸ್ಥಾನಂತನ್ನ
ಸನ್ನಿಧಾನಸ್ಥಾನಂಸನ್ನಿಧಾನ-ವಿ-ರುವ
ಸನ್ನಿಧಿಃ
ಸನ್ನಿಧಿಯು
ಸನ್ನಿವೇಶ್ಯ
ಸನ್ನಿಹಿತಃ
ಸನ್ನಿಹಿತಃತನ್ನ
ಸನ್ನಿಹಿತಃಸನ್ನಿಧಾನ
ಸನ್ನಿಹಿತ-ನಾಗಿ-ರುತ್ತಾನೆ
ಸನ್ನಿಹಿತ-ನಾಗಿ-ರುತ್ತಾನೆಂದು
ಸನ್ನಿಹಿತ-ನಾಗಿ-ರುವ
ಸನ್ನಿಹಿತೋ
ಸನ್ಮುಖ್ಯಪ್ರಾಣ-ದೇವ-ರನ್ನು
ಸಪ್ತಮಸ್ಕಂಧ-ದಲ್ಲಿ
ಸಪ್ತಮಾಧ್ಯಾಯೇಏಳನೆಯ
ಸಪ್ತ-ಲೋಕ-ಗಳನ್ನು
ಸಪ್ತಸ್ಕಂಧ-ಗತೋ
ಸಮಂಜಸವೂ
ಸಮಂತತಃ
ಸಮಂತತಃಪೂರ್ಣ-ವಾಗಿ
ಸಮಂತಾತ್
ಸಮಂತಾತ್ಪೂರ್ಣ-ವಾಗಿ
ಸಮಂತಾತ್ಸರ್ವಸ್ಮಿನ್
ಸಮಂತಾದಿತ್ಯರ್ಥಕಃತತ್ರಾದ್ಯಃ
ಸಮಂತಾದಿತ್ಯರ್ಥಕಃಪೂರ್ಣ-ವಾಗಿ
ಸಮಃ
ಸಮಗ್ರ-ವಾಗಿ
ಸಮದರ್ಶಿನಂ
ಸಮದರ್ಶಿನಂನಿರ್ದೋಷ-ನಾದ
ಸಮನೆ
ಸಮನ್ವಯ
ಸಮನ್ವಯಾಧ್ಯಾಯ-ದಲ್ಲಿ
ಸಮ-ರಹಿತನು
ಸಮರೆಂದೋ
ಸಮರ್ಥನೇ
ಸಮರ್ಥಿ-ಸುತ್ತದೆ
ಸಮರ್ಥಿ-ಸುತ್ತವೆ
ಸಮರ್ಪ-ಯತಿ
ಸಮವಯಸ್ಕ-ನೆಂದು
ಸಮಸ್ತ
ಸಮಸ್ತ-ಗುಣ
ಸಮಸ್ತ-ಗುಣ-ಪೂರ್ಣ
ಸಮಸ್ತ-ಜಗತ್ತಿಗೆ
ಸಮಸ್ತ-ತತ್ತ್ವ-ಗಳಿಂದ
ಸಮಸ್ತ-ದೇವ-ತೆ-ಗಳ
ಸಮಸ್ತ-ದೋಷ-ಗಳಿಂದ
ಸಮಸ್ತ-ದೋಷ-ರಹಿತ
ಸಮಸ್ತ-ರಿಂದ
ಸಮಸ್ತ-ಲೋಕವೂ
ಸಮಸ್ತ-ವನ್ನೂ
ಸಮಸ್ತ-ವಸ್ತು
ಸಮಸ್ತ-ವಸ್ತು-ಗಳಿ-ಗಿಂತಲೂ
ಸಮಸ್ತ-ವಸ್ತು-ಗಳಿಗೂ
ಸಮಸ್ತ-ವಾದ
ಸಮಸ್ತವು
ಸಮಸ್ತವೂ
ಸಮಸ್ತಶಃ
ಸಮಸ್ತಾನಿ
ಸಮಾಜ
ಸಮಾಧಿ
ಸಮಾಧೇಯಂಏವಮೇವ-ಹೀಗೆಯೇ
ಸಮಾಧೇಯಂಸಮಾಧಾನ
ಸಮಾನ-ಪದೇ
ಸಮಾ-ನಾರ್ಥ
ಸಮಾ-ನಾರ್ಥಃ
ಸಮಾ-ನಾರ್ಥಃಅ-ಭಾವ-ವನ್ನು
ಸಮಾ-ನಾರ್ಥಃರೀಂಕ್ಷಯೇ
ಸಮಾ-ನಾರ್ಥಕ
ಸಮಾ-ನಾರ್ಥ-ವಾಗಿ-ರುವು-ದ-ರಿಂದ
ಸಮಾ-ನಾರ್ಥ-ವುಳ್ಳ
ಸಮಾಶ್ರಿತಃ
ಸಮಾಶ್ರಿತಃಆಶ್ರಯ
ಸಮಾಶ್ರಿತ್ಯ
ಸಮಾಶ್ರಿತ್ಯ-ಆಧಾರ-ವನ್ನಾಗಿ
ಸಮಾಶ್ಲಿಷದಮುಂ
ಸಮಾಸತೇ
ಸಮಾಸತೇ-ವೈಕುಂಠ
ಸಮಾಸ-ದಿಂದ
ಸಮಿತ್ಪಾಣಿಃ
ಸಮೀಪ
ಸಮೀಪಕ್ಕೆ
ಸಮೀಪ-ದಲ್ಲಿಯೇ
ಸಮೀಪಸ್ಥಃ
ಸಮೀಪಸ್ಥೋ-ಽಪಿ
ಸಮುದಾಯ
ಸಮುದಾ-ಯಕ್ಕೆ
ಸಮುದಾ-ಯ-ದಿಂದ
ಸಮುದಾ-ಯ-ವನ್ನು
ಸಮೂಹ
ಸಮೂಹಃ
ಸಮೂಹಃಗುಂಪು
ಸಮೂಹಃಸ-ಮುದಾ-ಯವು
ಸಮೂಹಕ್ಕೆ
ಸಮೂಹ-ಗಳು
ಸಮೂಹವು
ಸಮೂಹವೇ
ಸಮೂಹೋ
ಸಮೇತ
ಸಮೇತ-ನಾದ
ಸಮ್ಯಕ್
ಸಮ್ಯಕ್ಒಂದೇ
ಸಮ್ಯಕ್ತ್ವಾರ್ಥಕಃಎರಡನೆಯ
ಸಮ್ಯಕ್ಪೂರ್ತಿ-ಯಾಗಿ
ಸಮ್ಯಕ್ಸಮಸ್ತ-ರಾದ
ಸಮ್ಯಗಾನಂದವ್ಯಕ್ತ್ಯಾ-ದಿಕಂ
ಸಮ್ಯಗ್ನಿರ್ದುಃಖ-ವಾದ
ಸಮ್ಯಗ್ಭಸ್ಮೀ-ಭವೇತ್ಸರ್ವಂ
ಸರಮಾ
ಸರಿ
ಸರಿ-ಯಷ್ಟೆ
ಸರಿ-ಯಾಗಿ
ಸರಿ-ಯಾಯಿತು
ಸರೋತ್ತಮ-ನಾದ
ಸರ್ಪವೇ
ಸರ್ಪಾದಿ-ಗಳಿಂದ
ಸರ್ವ
ಸರ್ವಂ
ಸರ್ವ-ಕರ್ತಾರ
ಸರ್ವ-ಕರ್ತೃತ್ವವೂ
ಸರ್ವ-ಕರ್ಮಾಣಿ
ಸರ್ವ-ಕರ್ಮಾಣಿ-ಸಮಸ್ತ
ಸರ್ವ-ಕಾಲ
ಸರ್ವ-ಜೀವ
ಸರ್ವ-ಜೀವ-ಜಡೇಭ್ಯೋ-ಽನ್ನತ್ತಾತ್
ಸರ್ವ-ಜೀವ-ರು-ಗಳಿಗೂ
ಸರ್ವ-ಜೀವಾ-ನಾಮ-ಭಿ-ಮಾನೀ
ಸರ್ವ-ಜೀವಾಭಿ-ಮಾನಿತ್ವಾತ್ನರ
ಸರ್ವ-ಜೀವಾಭಿ-ಮಾನಿಯೂ
ಸರ್ವ-ಜೀವೇಭ್ಯೋ-ಸಮಸ್ತ
ಸರ್ವಜ್ಞಾಚಾರ್ಯ
ಸರ್ವಜ್ಞಾನಿಜನ-ಪರಃನರ
ಸರ್ವತಃ
ಸರ್ವ-ತೋ-ಧಿಕಃ
ಸರ್ವ-ತೋಪ್ಯಧಿಕತ್ವತಃ
ಸರ್ವ-ತೋಪ್ಯಧಿಕತ್ವತಃಸಮಸ್ತ-ರಿಂದಲೂ
ಸರ್ವ-ತೋಪ್ಯಧಿಕತ್ವ-ವಾಚೀ
ಸರ್ವ-ತೋಪ್ಯಧಿಕತ್ವ-ವಾಚೀ-ನ-ಕಾರದ
ಸರ್ವತ್ರ
ಸರ್ವದಾ
ಸರ್ವ-ದಾ-ನಿತ್ಯವೂ
ಸರ್ವ-ದೇವ-ತಾ-ನಾಮ-ವಾಚ್ಯತ್ವಪ್ರ-ಸಿದ್ಧೇಃಯಃಯಾರು
ಸರ್ವ-ದೇಶ-ಗಳಲ್ಲಿ
ಸರ್ವ-ದೋಷ-ರಹಿತ
ಸರ್ವ-ದೋಷ-ವರ್ಜಿತನೂ
ಸರ್ವ-ದೋಷ-ವಿರುದ್ದ-ಗ-ಳಾದ
ಸರ್ವ-ದೋಷ-ವಿರುದ್ಧ-ವಾದ
ಸರ್ವ-ದೋಷ-ವಿರುದ್ಧೋರು
ಸರ್ವ-ದೋಷ-ವಿರುದ್ಧೋರು-ಗುಣ
ಸರ್ವ-ದೋಷ-ವಿರುದ್ಧೋರು-ಗುಣ-ಪೂರ್ಣಸ್ವ-ರೂಪತಃ
ಸರ್ವ-ದೋಷ-ವಿರುದ್ಧೋರು-ಗುಣ-ಪೂರ್ಣಸ್ವ-ರೂಪತ್ವಾತ್
ಸರ್ವ-ದೋಷ-ವಿ-ವರ್ಜಿತೇ
ಸರ್ವ-ದೋಷ-ವಿಹೀನತ್ವಂ
ಸರ್ವ-ದೋಷೋಜ್ಜಿ-ತತ್ವಾಚ್ಚ
ಸರ್ವ-ದೋಷೋಜ್ಜಿ-ತತ್ವಾತ್
ಸರ್ವ-ಭೂತಾಂತರಾತ್ಮಾ
ಸರ್ವ-ಭೂ-ತಾನಿ
ಸರ್ವ-ಮಿದಂ
ಸರ್ವ-ಮುಕ-ರಿ-ಗಿಂತಲೂ
ಸರ್ವ-ಮುಕ್ತೇಭ್ಯಃ
ಸರ್ವ-ಮುಕ್ತೇಭ್ಯಃಸಮಸ್ತ-ರಾದ
ಸರ್ವ-ಮುಕ್ತೇಭ್ಯೋಽಧಿಕತ್ವಂ
ಸರ್ವ-ಮುಕ್ತೇಭ್ಯೋಽಧಿಕತ್ವಾತ್ಸಮಸ್ತ-ಮುಕ್ತ-ರಿ-ಗಿಂತಲೂ
ಸರ್ವ-ರಿಂದಲೂ
ಸರ್ವ-ರಿಗೂ
ಸರ್ವ-ವನ್ನೂ
ಸರ್ವ-ವಾಗಾತ್ಮಾ
ಸರ್ವ-ವಿಲಕ್ಷಣ
ಸರ್ವವ್ಯಾಪಿ
ಸರ್ವಶಃ
ಸರ್ವಶ್ರೇಷ್ಠ-ನಾದು-ದ-ರಿಂದ
ಸರ್ವ-ಸಂಮಂತಂ
ಸರ್ವ-ಸದ್ಗುಣ-ಮೂರ್ತಿ-ಗಳೇ
ಸರ್ವ-ಸಮರ್ಥನು
ಸರ್ವಸ್ಟಿನ್
ಸರ್ವಸ್ಮಿನ್
ಸರ್ವಸ್ಯ
ಸರ್ವಸ್ಯಾ-ಯನಂ
ಸರ್ವಸ್ವ
ಸರ್ವಸ್ವಾಮಿ-ಯಾದು-ದ-ರಿಂದ
ಸರ್ವಸ್ವೇತಿ
ಸರ್ವಾಂತರಾತ್ಮಕಃ
ಸರ್ವಾಂತರಾತ್ಮಕಃಸಮಸ್ತ-ರಲ್ಲಿಯೂ
ಸರ್ವಾಧಿಕತ್ವಾತ್
ಸರ್ವಾನ್
ಸರ್ವಾಭೀಷ್ಟ-ಗಳನ್ನೂ
ಸರ್ವಾಭೀಷ್ಟಮವಾಪ್ನು-ಯಾತ್
ಸರ್ವಾವಸ್ಥಾ
ಸರ್ವೇ
ಸರ್ವೇಂದ್ರಿಯಾಣಿ
ಸರ್ವೈಃ
ಸರ್ವೋತ್ತಮ
ಸರ್ವೋತ್ತಮತ್ವ
ಸರ್ವೋತ್ತಮನ
ಸರ್ವೋತ್ತಮನೂ
ಸರ್ಶ-ದೋಷಃಸಾಲಿಗ್ರಾಮ-ವನ್ನು
ಸಲ
ಸಲು-ವಾಗಿ
ಸಲ್ಲಿಸಿ
ಸಲ್ಲಿಸಿ-ದ-ವರು
ಸಲ್ಲುತವೆಈ
ಸವರ್ಣ-ದೀರ್ಘ
ಸವರ್ಣ-ದೀರ್ಘೇ
ಸವರ್ಣ-ದೀರ್ಘೇ-ಸ-ವರ್ಣ-ದೀರ್ಘ-ಸಂಧಿ-ಯಿಂದ
ಸವಾಸವಾಃ
ಸವಾಸವಾಃಇಂದ್ರ-ನಿಂದ
ಸವಿತೃ-ಮಂಡಲ-ಮಧ್ಯ-ವರ್ತೀ
ಸವಿ-ಶೇಷಂ
ಸಹ
ಸಹ-ಕಾರಿಣಃ
ಸಹಜಮೂರ್ತಿತ್ವಾಚ್ಛ್ವ
ಸಹಜಮೂರ್ತಿ-ಯಾದು-ದ-ರಿಂದ
ಸಹಜ-ವಾಗಿಯೇ
ಸಹ-ದೇವ
ಸಹಪಾಠಿ-ಗಳೆಂದು
ಸಹ-ಭೋಗ-ಮನನ್ಯಲಭಭ್ಯಮಸ್ಮೈ
ಸಹ-ಭೋಗ-ವನ್ನೇ
ಸಹ-ಭೋಗ-ವೆಂಬ
ಸಹ-ಲಕ್ಷ್ಮೀ-ದೇವಿಯಿಂದ
ಸಹಸ್ರ
ಸಹಸ್ರಾರು
ಸಹಸ್ರೇಣಾಷ್ಯಪ್ರತಿ-ಬದ್ದಃ
ಸಹಾಯಕ-ನಾಗಿ
ಸಹಾಯಕ-ನಾಗಿದ್ದಾನೆಯೋ
ಸಹಾಯಕ-ನಾಗಿ-ರುತ್ತಾನೆ
ಸಹಾಯಕ-ರಾಗಿ-ರುವರೋ
ಸಹಾಯಕ-ವಾಗಿ-ರುವು-ದ-ರಿಂದ
ಸಹಾಯ-ದಿಂದ
ಸಹಾಯ-ಮಾಡಿದ
ಸಹಾಯ-ಮಾಡುವ
ಸಹಾಯ-ವಿಲ್ಲದೇ
ಸಹಾಯಾಃ
ಸಹಿತ
ಸಹಿತ-ನಾಗಿ
ಸಹಿತ-ನಾದಗಿ
ಸಹಿತ-ರಾಗಿ
ಸಹಿತ-ರಾದ
ಸಹಿತ-ವಾದ
ಸಹೈ-ತೇನ
ಸಹೋದರ
ಸಾ
ಸಾಂಕೇತಿಕಃ
ಸಾಕಂ
ಸಾಕಾಯಿತು
ಸಾಕ್ಷಾತ್
ಸಾಕ್ಷಾತ್ಕಾರ-ವಾಗುವ-ವರೆಗೆ
ಸಾಕ್ಷಾತ್ಕಾರ-ವಾಗು-ವುದಿಲ್ಲ
ಸಾಕ್ಷಾತ್ಕಾರ-ವಾದ
ಸಾಕ್ಷಾತ್ನಿಸ್ಸಂದೇಹ-ವಾಗಿ
ಸಾಕ್ಷಾದ್ದರ್ಶನ-ವಾದ
ಸಾಕ್ಷಾದ್ಬಲ-ಸಂವಿ-ದಾತ್ಮಾ
ಸಾಕ್ಷಾದ್ಯಜ್ಞ-ಲಿಂಗಸ್ಯ
ಸಾಕ್ಷಾದ್ಯಜ್ಞ-ಲಿಂಗಸ್ಯ-ಯಜ್ಞದ
ಸಾಗರೇ
ಸಾಧನ
ಸಾಧನಪ್ರಹ್ಲಾದನ
ಸಾಧನ-ವಾಗಿ-ದೆಯೋ
ಸಾಧನ-ವಾಗಿ-ವೆಯೋ
ಸಾಧನವು
ಸಾಧನಾಧ್ಯಾಯ
ಸಾಧನೆ-ಯಲ್ಲಿ
ಸಾಧನೆಯಾಗುತ್ತದೆ
ಸಾಧ್ಯವಾಗಲು
ಸಾಧ್ಯ-ವಿಲ್ಲ
ಸಾಧ್ಯ-ವಿಲ್ಲದೇ
ಸಾಧ್ಯ-ವಿಲ್ಲ-ವೆಂದರ್ಥ
ಸಾಧ್ಯ-ವಿಲ್ಲ-ವೆಂದು
ಸಾಧ್ಯವೇ
ಸಾಧ್ಯವೋ
ಸಾನ್ನಿಧ್ಯಂ
ಸಾನ್ನಿಧ್ಯ-ವನ್ನು
ಸಾನ್ನಿಧ್ಯವು
ಸಾಪೇಕ್ಷತ್ವಾತ್
ಸಾಮರ್ಥ್ಯ
ಸಾಮರ್ಥ್ಯ-ವನ್ನು
ಸಾಮರ್ಥ್ಯವು
ಸಾಮರ್ಥ್ಯ-ವುಳ್ಳದ್ದು
ಸಾಮಾನ್ಯ
ಸಾಮ್ಯಾಧಿಕ್ಕೇ
ಸಾಯುಜ್ಯ-ಮೋಕ್ಷ-ವನ್ನು
ಸಾಯುವ-ವ-ರನ್ನು
ಸಾಯು-ವುದಿಲ್ಲ
ಸಾರ
ಸಾರ-ಥಿ-ಯಾಗಿ
ಸಾರ-ಭೂತ-ವಾಗಿದೆ
ಸಾರ್ಥಕ
ಸಾರ್ಥಕ್ಯದ
ಸಾಲಗ್ರಾಮಂ
ಸಾಲಿಗ್ರಾಮ
ಸಾಲಿಗ್ರಾಮ-ಗಳನ್ನು
ಸಾಲಿಗ್ರಾಮ-ಗಳಲ್ಲಿ
ಸಾಲಿಗ್ರಾಮ-ಗಳಿಗೆ
ಸಾಲಿಗ್ರಾಮ-ಗಳೇ
ಸಾಲಿಗ್ರಾಮ-ದಲ್ಲಿ
ಸಾಲಿಗ್ರಾಮ-ವನ್ನು
ಸಾಲಿಗ್ರಾಮ-ವಿಲ್ಲದೇ
ಸಾಲಿಗ್ರಾಮವು
ಸಾಲಿಗ್ರಾಮವೇ
ಸಾಲಿಗ್ರಾಮ-ಶಿಲಾ
ಸಾಲಿಗ್ರಾಮ-ಶಿಲಾಂಸಾಲಿಗ್ರಾಮ-ಶಿಲೆ-ಯನ್ನು
ಸಾಲಿಗ್ರಾಮ-ಶಿಲಾ-ನರಾ-ಣಾಂಸಜ್ಜನ-ರಿಗೆ
ಸಾಲಿಗ್ರಾಮ-ಶಿಲಾಸ್ಪರ್ಶಂ
ಸಾಲಿಗ್ರಾಮ-ಶಿಲೆಯು
ಸಾಲಿಗ್ರಾಮಸ್ಪರ್ಶ
ಸಿದ್ದ-ವಾಗುತ್ತದೆ
ಸೀತಾ-ದೇವಿಯ-ವ-ರಿಗೆ
ಸೀಳಿ
ಸುಕೃತ
ಸುಖ
ಸುಖಂ
ಸುಖಂಆನಂದ-ವನ್ನು
ಸುಖಕ್ಕೆ
ಸುಖ-ದುಃಖ-ದಿಂದಲೇ
ಸುಖ-ನಿವಾರಣಾತ್
ಸುಖ-ನಿವಾರಣಾತ್ಸುಖವು
ಸುಖ-ನಿ-ಷೇಧ
ಸುಖ-ನಿ-ಷೇಧ-ಕರ್ತೃತ್ವಾತ್
ಸುಖ-ಪೂರ್ಣ-ನಾಗಿಯೇ
ಸುಖಪ್ರತಿ-ಷೇಧ
ಸುಖಪ್ರತಿ-ಷೇಧ-ಕರ್ತೃತ್ವಾತ್ಸುಖ
ಸುಖಪ್ರದ-ನೆಂಬ
ಸುಖಪ್ರಾಪ್ತವಾಗ-ದಂತೆ
ಸುಖ-ಬಾರ-ದಂತೆ
ಸುಖ-ಲೇಶವೂ
ಸುಖ-ವನ್ನು
ಸುಖವು
ಸುಖಸ್ಥಾನ-ಗಳೆಂದು
ಸುಖಾನುಭವ-ಮಾಡುವ-ವ-ರಿಗೆ
ಸುಖಾನುಭವ-ವನ್ನು
ಸುಖಿ-ಸುತ್ತಾನೆ
ಸುಡುತ್ತದೆ
ಸುಡುವ
ಸುತ್ತಿ
ಸುತ್ತಿ-ಸುತ್ತದೆ
ಸುತ್ತು-ವರಿ-ದಿ-ರುವ
ಸುತ್ವಾಶ್ರೀ-ವೇದವ್ಯಾಸ-ರನ್ನು
ಸುದರ್ಶನ
ಸುದರ್ಶನ-ಚಕ್ರಾದಿ
ಸುದರ್ಶನ-ಚಕ್ರಾದ್ಯಾಯುಧ-ಜಾತಂ
ಸುದರ್ಶನವು
ಸುದುರ್ಧರಾಂ
ಸುದೃಢಃ
ಸುಧಾ
ಸುಧಾ-ಯಾಂಈ
ಸುಪರ್ಣಂ
ಸುಪರ್ಣಂಗರುಡ-ನನ್ನೂ
ಸುಪರ್ಣಪತ್ನೀಂ
ಸುಪ್ತಿಃ
ಸುಪ್ತೋ
ಸುಮಧ್ವ-ವಿಜಯ
ಸುಮಧ್ವ-ವಿಜಯಈ
ಸುಮಧ್ವ-ವಿಜಯ-ದಲ್ಲಿ
ಸುಮಧ್ವ-ವಿಜಯೇ-ಸು-ಮಧ್ವ-ವಿಜಯ-ದಲ್ಲಿ
ಸುರಾಪಾಃ
ಸುಷುಪ್ತಿ
ಸುಷುಪ್ತಿ-ಗಳನ್ನು
ಸುಷುಪ್ತಿ-ಗಾಢನಿದ್ರೆ
ಸುಷುಮ್ನಾ
ಸುಹೃತ್
ಸುಹೃದೋ
ಸೂಕ್ಷಾ-ಕಾಶ-ದಲ್ಲಿ
ಸೂಕ್ಷ್ಮಧಿಯಃ
ಸೂಕ್ಷ್ಮ-ನಾದ
ಸೂಕ್ಷ್ಮ-ವಿ-ಶೇಷ-ದಿಂದ
ಸೂಕ್ಷ್ಮಾ-ಕಾಶ-ದಲ್ಲಿ
ಸೂಕ್ಷ್ಮಾ-ಕಾಶವು
ಸೂಚಕ-ವಾದ
ಸೂಚಕ-ವಾಯಿತು
ಸೂಚಿಸಿ
ಸೂಚಿ-ಸುತ್ತದೆ
ಸೂಚಿಸುತ್ತಾ
ಸೂಚಿಸುವ
ಸೂಚಿಸುವಂತಹುದು
ಸೂತಕ-ದಲ್ಲಿ-ರುವ-ವರು
ಸೂತಿಕಾದಿ
ಸೂತ್ರ
ಸೂತ್ರ-ಕೃತ-ಸೂತ್ರ-ಕಾರ-ರಿಂದ
ಸೂತ್ರಕ್ಕೆ
ಸೂತ್ರದ
ಸೂತ್ರ-ದಲ್ಲಿ
ಸೂತ್ರ-ದಿಂದ
ಸೂತ್ರ-ವನ್ನು
ಸೂತ್ರಾರ್ಥ
ಸೂತ್ರಾರ್ಥ-ವನ್ನು
ಸೂತ್ರೇಣ
ಸೂರ್ಯ
ಸೂರ್ಯನ
ಸೂರ್ಯ-ಮಂಡಲ
ಸೂರ್ಯ-ಮಂಡಲಂಕ್ಷೀಣತೆ-ಯನ್ನು
ಸೂರ್ಯ-ಮಂಡಲ-ಮಧ್ಯ-ದಲ್ಲಿ
ಸೂರ್ಯ-ಮಂಡಲ-ಮಧ್ಯ-ದಲ್ಲಿ-ರುವ
ಸೂರ್ಯ-ಮಂಡಲ-ಮಧ್ಯ-ವರ್ತಿತ್ವಾತ್ನಾರಾ-ಯಣಃ
ಸೂರ್ಯ-ಮಂಡಲವೇ
ಸೂರ್ಯ-ಮಂಡಲಸ್ಯೇತಿ
ಸೂರ್ಯೋಽಗ್ನಿ-ಯುಕ್ತೋ
ಸೃಜ್ಯಂತೇ
ಸೃಜ್ಯಂತೇ-ಸೃಷ್ಟಿ-ಸಲ್ಪಡುತ್ತಾರೆಯೋ
ಸೃಷ್ಟ
ಸೃಷ್ಟಂ
ಸೃಷ್ಟ-ತಯಾ
ಸೃಷ್ಟಾದಿ
ಸೃಷ್ಟಿ
ಸೃಷ್ಟಿ-ಕಾಲ
ಸೃಷ್ಟಿಗೆ
ಸೃಷ್ಟಿ-ಮಾಡಿ-ರುವ
ಸೃಷ್ಟಿ-ಯಾ-ದುದು
ಸೃಷ್ಟಿ-ಸಲ್ಪಟ್ಟಿತೋ
ಸೃಷ್ಟಿ-ಸಲ್ಪಟ್ಟು
ಸೃಷ್ಟಿ-ಸಿದನು
ಸೃಷ್ಟ್ಯಾದಿ
ಸೇತಿ-ಕರ್ತವ್ಯೇನ
ಸೇತುಂ
ಸೇತುನಾ
ಸೇರಬೇಕಾದ
ಸೇರಿ
ಸೇರಿದ
ಸೇರಿ-ದುದು
ಸೇರಿ-ಸ-ಬೇಕು
ಸೇರಿಸಿ
ಸೇರಿ-ಸಿ-ದರೆ
ಸೇರಿ-ಸಿ-ದಾಗ
ಸೇರುತ್ತದೆ
ಸೇರುತ್ತವೆ
ಸೇರುವ
ಸೇರು-ವುದಿಲ್ಲ
ಸೇವಾ
ಸೇವಿಸಿ
ಸೇವೆ
ಸೇವೆ-ಯನ್ನು
ಸೇವೆ-ಯಿಂದ
ಸೈವ
ಸೈವಾ-ಯನಂ
ಸೈವಾ-ಯನ-ಮಾಶ್ರಯಃ
ಸೋಮಂ
ಸೋಮಪಾನಕ್ಕೆ
ಸೋಮಸ್ತು
ಸೋಮಾದೀನಾಂ
ಸೋಮ್ಯ
ಸೋಽಂಧೀ
ಸೋಽಕ್ಷರಃ
ಸೌಂದರ್ಯಸಾಗರ-ವಾಗಿ-ರುವ
ಸೌಖ್ಯ-ವನ್ನು
ಸೌಜನ್ಯ-ತೆಗೆ
ಸೌಪರ್ಣಿ-ಯನ್ನು
ಸೌಭಾಗ್ಯ-ವನ್ನು
ಸ್ಕಾಂದ
ಸ್ಕಾಂದ-ಪುರಾ-ಣವಾಕ್ಯ-ವನ್ನೂ
ಸ್ಟೇಚ್ಛೆ-ಯಿಂದ
ಸ್ಟೇವಂ
ಸ್ಟೈರ್ಯ
ಸ್ತುತಿಸಲ್ಪಡುತ್ತಾನೆ
ಸ್ತುತ್ವಾ
ಸ್ತೇಜಃ
ಸ್ತೇನಾಃ
ಸ್ತೋತ್ರಂ
ಸ್ತೋತ್ರ-ಮಾಡಿ
ಸ್ತ್ರೀಯ-ರಲ್ಲಿಯೂ
ಸ್ತ್ರೀಯೂ
ಸ್ತ್ರೀರೂಪಃಸ್ತ್ರೀ-ರೂಪ-ದಿಂದಿ-ರುವ
ಸ್ತ್ರೀರೂಪಕೇ
ಸ್ತ್ರೀರೂಪದ
ಸ್ತ್ರೀರೂಪ-ದಲ್ಲಿ
ಸ್ತ್ರೀರೂಪ-ದಲ್ಲಿ-ರುವ
ಸ್ತ್ರೀರೂಪ-ದಿಂದ
ಸ್ತ್ರೀರೂಪ-ದಿಂದಿ-ರುವ
ಸ್ತ್ರೀರೂಪ-ಮುಕ್ತಬ್ರಹ್ಮಣಃ
ಸ್ತ್ರೀರೂಪ-ಮುಕ್ತಬ್ರಹ್ಮ-ಣಃಸ್ತ್ರೀ-ರೂಪ-ದಿಂದ
ಸ್ತ್ರೀರೂಪ-ವನ್ನು
ಸ್ತ್ರೀರೂಪೋ
ಸ್ತ್ರೀಸಂಗ
ಸ್ಥಳ
ಸ್ಥಳ-ಗಳಲ್ಲಿ
ಸ್ಥಳ-ಗಳೂ
ಸ್ಥಳ-ದಲ್ಲಿ
ಸ್ಥಳ-ವನ್ನು
ಸ್ಥಳ-ವಾಗಿ-ದೆಯೋ
ಸ್ಥಳವು
ಸ್ಥಳಾವ-ಕಾಶ
ಸ್ಥಾನಂ
ಸ್ಥಾನಂತನ್ನ
ಸ್ಥಾನಂವಾಸಸ್ಥಳ-ವಾದ
ಸ್ಥಾನ-ಗಳು
ಸ್ಥಾನತ್ರಯಂರೂಪ-ಗಳಿಂದ
ಸ್ಥಾನ-ದಲ್ಲಿ
ಸ್ಥಾನ-ವನ್ನು
ಸ್ಥಾನವು
ಸ್ಥಾಪನೆ-ಯಲ್ಲಿ
ಸ್ಥಾಪ-ಯತಿ
ಸ್ಥಾಪಿ-ಸಿದ
ಸ್ಥಾಪಿ-ಸಿದನು
ಸ್ಥಿತಃ
ಸ್ಥಿತಃಈ
ಸ್ಥಿತಃಜಠರಾ-ಕಾಶಾದಿ-ಗಳಲ್ಲಿದ್ದು
ಸ್ಥಿತಃಶರೀರದ
ಸ್ಥಿತ-ನಾಗಿ
ಸ್ಥಿತಿ
ಸ್ಥಿತಿಯು
ಸ್ಥಿರಪಡಿ-ಸುತ್ತವೆ
ಸ್ಥಿರ-ವಾಗಿ
ಸ್ಥೂಲ-ವಾಗಿಯೂ
ಸ್ನಾನದಿಂದ
ಸ್ನಾನ-ಮಾಡಿ
ಸ್ನಾನ-ಮಾಡುವುದರಿಂದ
ಸ್ನೇಚ್ಛೆ-ಯಾಗಿ
ಸ್ನೇಹಃ
ಸ್ನೇಹಕ್ಕೆ
ಸ್ನೇಹ-ದಿಂದಾಗಲೀ
ಸ್ನೇಹವೇ
ಸ್ನೇಹಾನು-ಬಂಧೋ
ಸ್ನೇಹಿತ
ಸ್ಪರ್ಶ
ಸ್ಪರ್ಶಂಸಾಲಿಗ್ರಾಮ-ಶಿಲೆಯ
ಸ್ಪರ್ಶಃನಂತ್ವಗಿಂದ್ರಿಯ-ವನ್ನೂ
ಸ್ಪರ್ಶ-ದಿಂದ
ಸ್ಪರ್ಶ-ದೋಷಃಇತರರ
ಸ್ಪರ್ಶ-ದೋಷೋ
ಸ್ಪರ್ಶನ
ಸ್ಪರ್ಶನಂ
ಸ್ಪರ್ಶ-ನ-ಪೂಜ-ನಾದಿನಾ
ಸ್ಪರ್ಶ-ನ-ಮಾತ್ರೇಣ
ಸ್ಪರ್ಶ-ನಾದ್ಭಗ-ವತೋ-ಽತಿ
ಸ್ಪರ್ಶ-ಮಾಡಿ-ದ-ಮಾತ್ರ-ದಿಂದಲೇ
ಸ್ಪರ್ಶ-ಮಾಡಿ-ದರೆ
ಸ್ಪರ್ಶ-ವನ್ನು
ಸ್ಪರ್ಶ-ವಾದ
ಸ್ಪರ್ಶಾಚ್ಚ
ಸ್ಪರ್ಶಾತ್
ಸ್ಪರ್ಶಿತ-ವಾ-ದರೂ
ಸ್ಪಷ್ಟ
ಸ್ಪಷ್ಟ-ಪಡಿಸ-ಲಾಗಿದೆ
ಸ್ಪಷ್ಟ-ವಾಗುತ್ತದೆ
ಸ್ಪಷ್ಟ-ವಾದ
ಸ್ಪಷ್ಟೋಕ್ತಿ-ಗಳು
ಸ್ಮ
ಸ್ಮಆಲದ
ಸ್ಮರಂತಿ
ಸ್ಮರಣಂ
ಸ್ಮರಣಾತ್
ಸ್ಮರಣಾತ್ಅಮ
ಸ್ಮರಣೆ
ಸ್ಮಿತವಕ್ತ್ರಃ
ಸ್ಮೃತಃ
ಸ್ಮೃತಃಆಪಃನೀರು
ಸ್ಮೃತಃನಾರಾ-ಯಣ
ಸ್ಮೃತಃಪರಮಾತ್ಮನು
ಸ್ಮೃತಿ
ಸ್ಮೃತಿಃಗುರುಪ್ರಸಾದೋ
ಸ್ಮೃತಿ-ಗಳಲ್ಲಿ
ಸ್ಮೃತಿ-ಗಳೂ
ಸ್ಮೃತಿ-ಸಿದ್ಧ-ವಾ-ದುದು
ಸ್ಯನ
ಸ್ಯಾತೇಷಾಂ
ಸ್ಯಾತ್
ಸ್ಯಾತ್ನರಾ-ಯಣ
ಸ್ಯಾದಾ-ಮುಕ್ತಸ್ತು
ಸ್ಯಾದಿತಿ
ಸ್ಯಾದೇವಂ
ಸ್ಯಾದ್ರಜಸ್ವಲಾ
ಸ್ಯುಃ
ಸ್ಯುರ-ಶುಭಾನಿ
ಸ್ವಚ್ಛ-ಮಾಡಲು
ಸ್ವಚ್ಛ-ಮಾಡುತ್ತೇನೆ
ಸ್ವಚ್ಛೆ-ಯಿಂದ
ಸ್ವತ
ಸ್ವತಂತ್ರ
ಸ್ವತಂತ್ರ-ವಾಗಿ
ಸ್ವಪರಗತ-ತನ್ನ
ಸ್ವಪರಗತಾ-ಶೇಷ-ವಿ-ಶೇಷ-ವಿಷಯ-ಕೇತಿ
ಸ್ವಪ್ನ
ಸ್ವಪ್ನದ
ಸ್ವಪ್ನ-ದಲ್ಲಿ
ಸ್ವಪ್ನ-ಸುಷುಪ್ತ್ಯಾದಿ
ಸ್ವಪ್ನಾದಿ-ಗಳ
ಸ್ವಪ್ನಾದಿಬುದ್ಧಿ-ಕರ್ತಾ
ಸ್ವಪ್ನಾವಸ್ಥೆ
ಸ್ವಪ್ನಾವಸ್ಥೆ-ಗಳ
ಸ್ವಪ್ನೇ
ಸ್ವಪ್ಪಕನಸುಕಾಣು-ವುದು
ಸ್ವಭಕ್ತಂತನ್ನ
ಸ್ವಭಕ್ತರ
ಸ್ವಭಕ್ತೇಭ್ಯಃ
ಸ್ವಭಕ್ತೇಭ್ಯಃತನ್ನ
ಸ್ವಭಕ್ತೇಭ್ಯೋ
ಸ್ವಭಾಕ್ತಂ
ಸ್ವಭಾ-ವಕ್ಕೆ
ಸ್ವಭಾವ-ಗಳು
ಸ್ವಭಾವ-ಗುಣದ
ಸ್ವಭಾವ-ದಿಂದ
ಸ್ವಭಾವ-ಧರ್ಮ-ಗಳು
ಸ್ವಭಾವ-ವುಳ್ಳ
ಸ್ವಭಾ-ವೇನ
ಸ್ವಭೀಷ್ಟ-ಕರ್ತ್ರೇ
ಸ್ವಮೇವಂ
ಸ್ವಯಂ
ಸ್ವಯಂವ್ಯಕ್ತ
ಸ್ವಯಂವ್ಯಕ್ತ-ವಲ್ಲದ
ಸ್ವಯಮೇಕಂ
ಸ್ವಯೋಗ್ಯ
ಸ್ವರ-ಣಾತ್
ಸ್ವರೂಪ
ಸ್ವರೂ-ಪತ್ವವು
ಸ್ವರೂಪದ
ಸ್ವರೂ-ಪ-ದಿಂದ
ಸ್ವರೂ-ಪ-ದೇಹ
ಸ್ವರೂ-ಪ-ದೇಹ-ವುಳ್ಳ-ವನು
ಸ್ವರೂಪನೇ
ಸ್ವರೂ-ಪ-ಭೂತ
ಸ್ವರೂ-ಪ-ಭೂತ-ವಾದ
ಸ್ವರೂ-ಪ-ಳಾದ
ಸ್ವರೂಪಳೂ
ಸ್ವರೂ-ಪ-ವನ್ನೇ
ಸ್ವರೂಪವು
ಸ್ವರೂಪವೇ
ಸ್ವರೂ-ಪ-ಸುಖವು
ಸ್ವರೂ-ಪಾ-ನಂದ-ವನ್ನು
ಸ್ವರೂಪಿಣ್ಯಾಂ
ಸ್ವರೂಪೇಣ
ಸ್ವರ್ಗ
ಸ್ವರ್ಗಃಸ್ವರ್ಗ-ಲೋಕವು
ಸ್ವರ್ಗ-ದಲ್ಲಿ
ಸ್ವರ್ಗ-ದಲ್ಲಿಯೂ
ಸ್ವರ್ಗ-ವೆಂಬುದನ್ನು
ಸ್ವರ್ಗವೇ
ಸ್ವರ್ಗಾದಿ
ಸ್ವರ್ಗಾದಿ-ಗಳಲ್ಲಿ
ಸ್ವರ್ಗೇ
ಸ್ವರ್ಗೇಆ
ಸ್ವರ್ಗೇಸ್ವರ್ಗ-ಲೋಕ-ದಲ್ಲಿ
ಸ್ವರ್ಗೋ
ಸ್ವರ್ಶ-ವಾ-ದರೆ
ಸ್ವಸಮೀಪಂ
ಸ್ವಸಮೀಪಂತನ್ನ
ಸ್ವಸಹಾ-ಯೇನ
ಸ್ವಸಹಾ-ಯೇನ-ತನ್ನ
ಸ್ವಸ್ತ್ರೀ-ರೂಪಂ
ಸ್ವಸ್ತ್ರೀ-ರೂಪ-ಸಹಿತ
ಸ್ವಸ್ಯ
ಸ್ವಸ್ವ-ರೂಪದ
ಸ್ವಸ್ವಾತ್ಮೀಯ
ಸ್ವಾಂಘ್ರಿರೇಣುಭಿಃ
ಸ್ವಾಂಘ್ರಿರೇಣುಭಿಃತನ್ನ
ಸ್ವಾಂಫ್ರಿರೇಣುಭಿಃವಾತನೀ-ತೈಃಗಾಳಿ-ಯಿಂದ
ಸ್ವಾಂಫ್ರಿರೇಣುಭಿರ್ವಾತನೀತೈಃ
ಸ್ವಾಗತಿ-ಸುತ್ತಾ-ರೆಂದು
ಸ್ವಾತಂತ್ರ್ಯಾಂಶದಷ್ಟು
ಸ್ವಾತಪ್ರದಾನಮಧಿಕಂ
ಸ್ವಾತ್ಮಾನಂ
ಸ್ವಾತ್ಮಾನ-ಮೇವೈಷ
ಸ್ವಾತ್ಹೇಗೆ
ಸ್ವಾಧಿಷ್ಟಾನಾನ್
ಸ್ವಾಧಿಷ್ಠಾನ್
ಸ್ವಾಮಿ
ಸ್ವಾಮಿತ್ವೇನ
ಸ್ವಾಮಿ-ಪುಷ್ಕರಣಿ
ಸ್ವಾಮಿ-ಪುಷ್ಕರಿಣೀತೀರೇ
ಸ್ವಾರ್ಥ-ತನಗಾಗಿಯೇ
ಸ್ವಾರ್ಥೇ
ಸ್ವೀಕರಿ-ಸಲು
ಸ್ವೀಕರಿಸಲ್ಪಟ್ಟರೆ
ಸ್ವೀಕ-ರಿಸಿದ
ಸ್ವೀಕರಿ-ಸುತ್ತಾನೆ
ಸ್ವೀಕರಿ-ಸುತ್ತಾನೆಂದು
ಸ್ವೀಕರಿ-ಸುತ್ತಾನೆಯೋ
ಸ್ವೀಕರಿಸುವ
ಸ್ವೀಕರಿಸು-ವುದಿಲ್ಲ
ಸ್ವೇಚ್ಛಯಾ
ಸ್ವೇನ-ತನ್ನದೇ
ಸ್ವೇನ-ಶಬ್ದಾತ್
ಹ
ಹಂಚಿ
ಹಂತ-ಅಯ್ಯೋ
ಹಂತಾರ್ಭಕಾಃ
ಹಂಬಲ-ವುಳ್ಳ-ವನು
ಹಇದು
ಹಈ
ಹಕ್ಕನ್ನು
ಹತ್ತಿರ-ದಿಂದ
ಹತ್ತು
ಹದಿ-ನಾಲ್ಕು
ಹನುಮ
ಹನುಮಂತ
ಹನುಮಂತ-ದೇವರ
ಹನುಮಂತ-ದೇವ-ರಾಗಿ
ಹನುಮಂತ-ದೇವ-ರಿಗೆ
ಹನುಮ-ದಾಖ್ಯಂ
ಹನುಮದ್ರೂಪ-ವತೇ
ಹನು-ಮಾನ್ನಾಮ
ಹಮ್ಮಿಕೊಳ್ಳಲಾದ
ಹಯಗ್ರೀವ-ದೇವರು
ಹರಸಿ
ಹರಿ
ಹರಿಃ
ಹರಿಃನಾರಾ-ಯ-ಣನು
ಹರಿಃಪರಃಸರ್ವೋತ್ತಮ-ನಾದ
ಹರಿಃಪರಮಾತ್ಮನು
ಹರಿಃವಿಷ್ಣುವು
ಹರಿಃಷಡ್ಗುಣೈಶ್ಚರ್ಯ-ಪೂರ್ಣ-ನಾದ
ಹರಿಃಹರಿಯು
ಹರಿ-ಕಥಾಮೃತ-ಸಾರ-ದಲ್ಲಿ
ಹರಿ-ಗುರು
ಹರಿಗೆ
ಹರಿದ್ವೇಷ-ವಲ್ಲದ
ಹರಿದ್ವೇಷ-ವೆಂದರೆ
ಹರಿದ್ವೇಷಿ-ಗಳೂ
ಹರಿ-ಪಾಪ
ಹರಿ-ಪಾಪ-ಪರಿಹಾರ-ಕ-ನಾದ
ಹರಿ-ಭಕ್ತರ
ಹರಿ-ಭಕ್ತ-ರನ್ನು
ಹರಿಯ
ಹರಿ-ಯನ್ನು
ಹರಿಯು
ಹರಿ-ಯು-ವುದು
ಹರಿ-ರೈಧ-ಯತ್ತಮೇವ
ಹರಿರ್ನೋ
ಹರಿ-ವಿಷ್ಣುವು
ಹರಿಸ್ತಸ್ಮಿನ್
ಹರೇ
ಹರೇಃ
ಹರೇಃಪರಮಾತ್ಮನ
ಹರೇಃಯಾವ
ಹರೇ-ರನುಗ್ರಹೇಣೈವ
ಹರೇ-ರಸೌ
ಹರೇ-ರಾಂದೋಲಿಕಾ
ಹರ್ಯಂಗಸ್ನೇದ-ಜನಿತಾಶ್ರೀ-ಹರಿಯ
ಹರ್ಯಂಗಸ್ವೇದ-ಜನಿತಾ
ಹರ್ಯತ್ಯರ್ಥಪ್ರಸಾದಸ್ಯ
ಹರ್ಷಿಸುತ್ತೇವೆ
ಹಲವಾರು
ಹಸ್ತಲಾಘವ-ವನ್ನು
ಹಾಗಾಗಬಾರದು
ಹಾಗಿದ್ದಲ್ಲಿ
ಹಾಗಿಲ್ಲ-ದಿದ್ದರೆ
ಹಾಗೂ
ಹಾಗೆ
ಹಾಗೆಯೇ
ಹಾನಿ
ಹಿ
ಹಿಂದಿನ
ಹಿಂದಿ-ನಂತೆ
ಹಿಂದಿ-ನಂತೆಯೇ
ಹಿಂದಿರುಗು-ವುದು
ಹಿಂದುಗಡೆ
ಹಿಂದೆ
ಹಿಂಬಾಲಿ-ಸುತ್ತಾನೆ
ಹಿಂಸಾಪರಾ-ಯಣಃ
ಹಿಂಸಾಪರಾಯ-ಣನು
ಹಿಂಸೆ-ಯನ್ನು
ಹಿಅ
ಹಿಇದು
ಹಿಇಲ್ಲವೇ
ಹಿಜಡ-ವಸ್ತು-ಗಳಿಂದಲೂ
ಹಿಡಿದು-ಕೊಂಡು
ಹಿಪಾತಾಳ-ಲೋಕ-ದಲ್ಲಿಯೇ
ಹಿಶ್ರುತಿ-ಗಳಲ್ಲಿ
ಹಿಹೇಗೆಂದರೆ-ತದನ್ಯ
ಹೀಗಾಗುತ್ತದೆ
ಹೀಗಿದ್ದರೂ
ಹೀಗಿ-ರಲು
ಹೀಗಿ-ರುವ
ಹೀಗಿ-ರುವಾಗ
ಹೀಗಿ-ರುವು-ದ-ರಿಂದ
ಹೀಗೂ
ಹೀಗೆ
ಹೀಗೆಂದು
ಹೀಗೆಯೇ
ಹುಟ್ಟಿದ
ಹುಟ್ಟಿ-ದರು
ಹುಟ್ಟಿ-ದವು
ಹುಟ್ಟಿದ್ದಲ್ಲ
ಹುಟ್ಟಿ-ರುವ
ಹುಟ್ಟಿ-ಸಲು
ಹುಟ್ಟಿ-ಸುವ-ವನು
ಹುಟ್ಟುತವೆಯೋ
ಹುಟ್ಟುತ್ತಾ
ಹುಟ್ಟು-ವುದಿಲ್ಲ
ಹುಟ್ಟು-ವುದು
ಹುಟ್ಟುವುವು
ಹುಟ್ಟುಹಾಕುವ
ಹುಡುಕ-ಬೇಕು
ಹೃತ್ಪದ್ಮಂಜೀವ-ರು-ಗಳಿಗೆ
ಹೃತ್ಪದ್ಮ-ಗತ್ವಾನ್ನಾರಾ-ಯ-ಣಃಜೀವ-ರು-ಗಳ
ಹೃದಬ್ಜಗಃಅಕ್ಷರಃ
ಹೃದಬ್ಜಗ-ಹೃದಯ-ವೆಂಬ
ಹೃದಯ
ಹೃದಯಂ
ಹೃದಯಂಹೃದಯ-ನಾಮಕ
ಹೃದಯ-ಕಮ-ದಲ್ಲಿ-ರುವ
ಹೃದಯ-ಕಮಲ
ಹೃದಯ-ಕಮಲ-ದಲ್ಲಿ
ಹೃದಯ-ಕಮಲವೇ
ಹೃದಯ-ದಲ್ಲಿ
ಹೃದಯ-ದಲ್ಲಿದ್ದರೂ
ಹೃದಯ-ದಲ್ಲಿ-ರುವ
ಹೃದಯ-ಮಾರುಣಯೋ
ಹೃದಯ-ರೂಪಸ್ಥಾನ-ಗಳಲ್ಲಿ
ಹೃದ-ಯಲ್ಲಿ
ಹೃದಯ-ವೆಂಬ
ಹೃದಯ-ಹೃದಯ
ಹೃದಯಾ-ಕಾಶ-ದಲ್ಲಿ-ರುವ
ಹೃದಯೇ
ಹೃದಯೇ-ಹೃದಯ-ದಲ್ಲಿ
ಹೃರ್ಥೋ
ಹೆಚ್ಚಾದ
ಹೆಚ್ಚು
ಹೆಚ್ಚು-ಗೊಳಿ-ಸು-ವುದು
ಹೆಜ್ಜೆ-ಯಾಗಿ
ಹೆಬ್ಬೆಟ್ಟಿನ
ಹೆಬ್ಬೆರಳಿನ
ಹೆಸ-ರನ್ನು
ಹೆಸರಿನ
ಹೆಸರಿ-ನಿಂದ
ಹೆಸರು
ಹೆಸರು-ಗಳನ್ನು
ಹೆಸರು-ಗಳು
ಹೆಸರು-ಗಳೂ
ಹೆಸರುಳ್ಳ
ಹೇ
ಹೇಗಿ-ರುವೆನೋ
ಹೇಗೆ
ಹೇಗೆಂದರೆ
ಹೇತುತ್ವ
ಹೇರಳ-ವಾಗಿವೆ
ಹೇಳ-ತಕ್ಕದ್ದು
ಹೇಳ-ಬಹುದು
ಹೇಳಬೇಕಾದ
ಹೇಳಬೇಕಾದುದೇ
ಹೇಳ-ಲಾಗಿದೆ
ಹೇಳ-ಲಾಗಿದೆಜ್ಞಾನ-ರೂಪತ್ವತೋ
ಹೇಳ-ಲಾಗಿದೆಜ್ಞಾನಿಯು
ಹೇಳಲಾಗುತದೆ
ಹೇಳಲಾಗುತ್ತದೆ
ಹೇಳಲಾಗುವ
ಹೇಳಲು
ಹೇಳಲ್ಪಟ್ಟ
ಹೇಳಲ್ಪಟ್ಟಿದೆ
ಹೇಳಲ್ಪಟ್ಟಿದ್ದಾರೆ
ಹೇಳಲ್ಪಟ್ಟಿ-ರುತ್ತಾನೆ
ಹೇಳಲ್ಪಟ್ಟಿ-ರುತ್ತಾರೆ-ಹೀಗೆಂಬ
ಹೇಳಲ್ಪಟ್ಟಿ-ರುತ್ತಾಳೆ
ಹೇಳಲ್ಪಟ್ಟಿ-ರುವು-ದ-ರಿಂದ
ಹೇಳಲ್ಪಡ-ತಕ್ಕದ್ದು
ಹೇಳಲ್ಪಡದ
ಹೇಳಲ್ಪಡುತದೆ
ಹೇಳಲ್ಪಡುತ್ತದೆ
ಹೇಳಲ್ಪಡುತ್ತವೆ
ಹೇಳಲ್ಪಡುತ್ತಾನೆ
ಹೇಳಲ್ಪಡುತ್ತಾರೆ
ಹೇಳಲ್ಪಡುವ
ಹೇಳವ
ಹೇಳಿ
ಹೇಳಿದ
ಹೇಳಿ-ದಂತಾಯಿತು
ಹೇಳಿ-ದಂತೆ
ಹೇಳಿ-ದರೆ
ಹೇಳಿ-ದುದು
ಹೇಳಿದೆ
ಹೇಳಿದ್ದಾರೆ
ಹೇಳಿ-ರುತ್ತವೆ
ಹೇಳಿ-ರುತ್ತಾರೆ
ಹೇಳಿ-ರುವ
ಹೇಳಿ-ರುವಂತೆ
ಹೇಳಿ-ರುವು-ದ-ರಿಂದ
ಹೇಳಿ-ರು-ವುದು
ಹೇಳಿ-ಸಿಕೊಳ್ಳಲ್ಪಡುತ್ತಾನೆ
ಹೇಳಿ-ಸಿಕೊಳ್ಳಲ್ಪಡುತ್ತಾರೆ
ಹೇಳಿ-ಸಿಕೊಳ್ಳುತ್ತಾನೆ
ಹೇಳಿ-ಸಿಕೊಳ್ಳುತ್ತಾನೆಯೋ
ಹೇಳಿ-ಸಿಕೊಳ್ಳುವ
ಹೇಳುತ್ತದೆ
ಹೇಳುತ್ತವೆ
ಹೇಳುತ್ತಾ
ಹೇಳುತ್ತಾನೆ
ಹೇಳುತ್ತಾರೆ
ಹೇಳುತ್ತಾ-ರೆನೃ
ಹೇಳುತ್ತೇವೆ
ಹೇಳುವ
ಹೇಳುವಂತಹುದು
ಹೇಳುವಂತೆ
ಹೇಳು-ವಲ್ಲಿ
ಹೇಳುವಾಗ
ಹೇಳುವುದನ್ನು
ಹೇಳುವು-ದ-ರಿಂದ
ಹೇಳುವು-ದಾಗಿದೆ
ಹೇಳು-ವುದು
ಹೇಳುವುದೇನು
ಹೈಪುಣ್ಯ-ವತಾಂ
ಹೊಂದದ
ಹೊಂದದೆಯೇ
ಹೊಂದ-ಬಹು-ದಾ-ಗಿದ್ದರೆ
ಹೊಂದ-ಬಹು-ದೆಂದು
ಹೊಂದಬೇಕೆಂಬ
ಹೊಂದಲು
ಹೊಂದಲ್ಪಡ-ತಕ್ಕ-ವನಲ್ಲ-ವಾದ
ಹೊಂದಲ್ಪಡ-ತಕ್ಕ-ವನಲ್ಲ-ವೆಂಬ
ಹೊಂದಲ್ಪಡ-ತಕ್ಕ-ವನು
ಹೊಂದಲ್ಪಡ-ತಕ್ಕ-ವನೂ
ಹೊಂದಲ್ಪಡಲು
ಹೊಂದಲ್ಪಡುವ
ಹೊಂದಲ್ಪಡು-ವನು
ಹೊಂದಲ್ಪಡುವ-ವ-ನಾದು-ದ-ರಿಂದ
ಹೊಂದಿ
ಹೊಂದಿತು
ಹೊಂದಿ-ದ-ವನು
ಹೊಂದಿ-ರುವ
ಹೊಂದಿ-ರುವುದಕ್ಕೆ
ಹೊಂದಿ-ರುವು-ದ-ರಿಂದ
ಹೊಂದಿ-ರುವು-ದ-ರಿಂದಲೂ
ಹೊಂದಿ-ರು-ವುದು
ಹೊಂದಿ-ಸುತ್ತಾನೆ
ಹೊಂದಿ-ಸುತ್ತಾನೆ-ಹೀಗೆ
ಹೊಂದಿ-ಸುವ-ವನು
ಹೊಂದಿ-ಸು-ವುದು
ಹೊಂದುತ್ತದೆ
ಹೊಂದುತ್ತಾನೆ
ಹೊಂದುತ್ತಾರೆ
ಹೊಂದುವ
ಹೊಂದು-ವರು
ಹೊಂದು-ವಿಕೆ
ಹೊಂದು-ವಿಕೆ-ಯನ್ನು
ಹೊಂದು-ವುದಿಲ್ಲ
ಹೊಂದು-ವುದಿಲ್ಲ-ವಾದು-ದ-ರಿಂದ
ಹೊಂದು-ವುದು
ಹೊಂದು-ವುದೇ
ಹೊಕ್ಕಳಿ-ನಲ್ಲಿ
ಹೊಟ್ಟೆ-ಯಲ್ಲಿ
ಹೊಟ್ಟೆ-ಯಿಂದ
ಹೊಟ್ಟೆಯು
ಹೊತ್ತಿ-ರುತ್ತಾನೆ
ಹೊರ
ಹೊರ-ಗಡೆ-ಯಿಂದ
ಹೊರ-ಗಿದ್ದ
ಹೊರ-ಗಿನ
ಹೊರ-ಗಿ-ನಿಂದ
ಹೊರಗೂ
ಹೊರಗೆ
ಹೊರ-ಟಿವೆ
ಹೊರ-ಡಿಸಿ-ಕೊಂಡು
ಹೊರ-ಡಿ-ಸುವಾಗಲೂ
ಹೊರತು
ಹೊಳೆಯುತ್ತವೆ
ಹೋಗಬೇಕಾಗಿ-ದೆಯೋ
ಹೋಗಲು
ಹೋಗಿದ್ದಾಗ
ಹೋಗಿ-ರು-ವುದು
ಹೋಗುತ್ತದೆ
ಹೋಗುತ್ತಾನೆ
ಹೋಗುತ್ತಾನೆಯೋ
ಹೋಗುತ್ತಾರಲ್ಲದೆ
ಹೋಗುತ್ತಾರೆ
ಹೋಗುತ್ತಿ-ರುವ
ಹೋಗುತ್ತಿ-ರುವ-ವ-ರನ್ನು
ಹೋಗುತ್ತೇನೆ
ಹೋಗುವ
ಹೋಗುವಂತೆ
ಹೋಗು-ವರು
ಹೋಗುವ-ವ-ನನ್ನು
ಹೋಗುವ-ವನು-ಅ-ಯನಂ
ಹೋಗುವ-ವ-ರಿಗೆ
ಹೋಗುವ-ವರು
ಹೋಗುವಾಗ
ಹೋಗುವಿಕೆಯು
ಹೋಗು-ವುದು
ಹೋದ
ಹೋದ-ವರು
ಹೋದುದು
ಹೋಮ-ದಂತೆ
ಹೋಮ-ವತ್
ಹೋಮ-ವತ್ಬೂದಿ-ಯಲ್ಲಿ
ಹ್ಯಕಾರಾರ್ಥೋ-ಽಯಮೇವ
ಹ್ಯಕಾರೋ-ಲಽಸೌ
ಹ್ಯಧರ್ಮೇಣ
ಹ್ಯಪುಣ್ಯ-ವತಾಂ
ಹ್ಯಸ್ಯ
ಹ್ಯಾನಂತ್ಯಾತ್ಕರ್ಮಣಾಂ
ಹ್ಯುಕ್ತಂ
ಹ್ಯೇಷ
ಹ್ರಾಸ
ೠ
}
