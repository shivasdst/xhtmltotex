
\chapter*{ಅನುವಾದಕನ ವಿಜ್ಞಾಪನೆ}

 ಶ‍್ರೀ ಬಿದರಹಳ್ಳಿ ಶ‍್ರೀನಿವಾಸತೀರ್ಥರು ಗ್ರಹಸ್ಥಾಶ್ರಮದಲ್ಲಿದ್ದರೂ, ಶ‍್ರೀಮಂತ್ರಾಲಯ ಪ್ರಭು\-ಗಳಿಂದ "ತೀರ್ಥ'' ಎಂಬ ಬಿರುದನ್ನು ಪಡೆದ ಮಹಾಪಂಡಿತರು. ಮಹಾ ತಪಸ್ವಿ\-ಗಳೂ, ಪ್ರಕಾಂಡ ಪಂಡಿತರೂ ಆದ ಶ‍್ರೀ ಯಾದವಾರ್ಯರಲ್ಲಿ ಶ‍್ರೀಮನ್ಮಧ್ವಶಾಸ್ತ್ರವನ್ನು ಅಭ್ಯಾಸ\-ಮಾಡಿ ಶ‍್ರೀಮದಾಚಾರ್ಯರ ಗ್ರಂಥಗಳಿಗೆ ಶ‍್ರೀಮಜ್ಜಯತೀರ್ಥರಿಂದ ರಚಿತವಾದ ಟೀಕಾ\-ಗಳ ಮೇಲೆ ಟಿಪ್ಪಣಿಗಳನ್ನು ರಚಿಸಿ ಪ್ರಖ್ಯಾತ ಟಿಪ್ಪಣಿಕಾರರೆಂದು ಪರಿಗಣಿಸಲ್ಪಟ್ಟಿದ್ದಾರೆ. ಶ‍್ರೀ ಯಾದವಾರ್ಯರ ಅಪ್ಪಣೆಯ ಮೇಲೆ "ನಾರಾಯಣಶಬ್ದಾರ್ಥ"ವೆಂಬ ಗ್ರಂಥವನ್ನು ರಚಿಸಿ, ಅದರಲ್ಲಿ ಎಪ್ಪತ್ತೆರಡು ಶಾಸ್ತ್ರಸಮ್ಮತವಾದ ಅರ್ಥಗಳನ್ನು ನಿರೂಪಿಸಿ ಆ\break ಸರ್ವೇಶ್ವರನ ಮಹಾಮಹಿಮೆಗಳನ್ನು ಪ್ರಕಾಶಪಡಿಸಿರುತ್ತಾರೆ. ಒಂದೊಂದು ಅರ್ಥವೂ ದ್ವೈತ\-ಸಿದ್ದಾಂತದ ಪ್ರಮೇಯವನ್ನು ಪ್ರತಿಪಾದಿಸುತ್ತದೆ.

ಈ ಮಹಿಮೋಪೇತವಾದ ಸಂಸ್ಕೃತಗ್ರಂಥವನ್ನು ಕನ್ನಡದಲ್ಲಿ ಅನುವಾದ ಮಾಡಲು ಪ್ರಯತ್ನ\-ಮಾಡಿರುತ್ತೇನೆ. ಸಂಸ್ಕೃತಭಾಷೆಯ ಜ್ಞಾನವು ಸಾಕಷ್ಟು ಇಲ್ಲದಿದ್ದರೂ, ಶಾಸ್ತ್ರಗಳ ಪರಿ\-ಚಯವು ನನಗೆ ದೂರವಾಗಿದ್ದರೂ ಶ‍್ರೀಹರಿವಾಯುಗಳ ಸೇವೆ ಎಂಬ ದೃಷ್ಟಿಯಿಂದ ಈ ಅನುವಾದದ ಯತ್ನ ಮಾಡಿದ್ದೇನೆ. ಇದರಲ್ಲಿ ಅನೇಕ ದೋಷಗಳು ಇರಬಹುದು. ಸಜ್ಜನರು ಅಂತಹ ತಪ್ಪುಗಳನ್ನು ನನಗೆ ತೋರಿಸಿಕೊಟ್ಟರೆ ತುಂಬ ಕೃತಜ್ಞನಾಗಿರುತ್ತೇನೆ.

ದ್ವೈತಸಿದ್ಧಾಂತ ಪ್ರಮೇಯ ಭಾಗವನ್ನು ವಿವರಿಸಬೇಕಾದ ಸ್ಥಳಗಳಲ್ಲಿ "ವಿಶೇಷಾಂಶ" ಎಂಬ ಶಿರೋನಾಮೆಯ ಕೆಳಗೆ ನನ್ನ ಯೋಗ್ಯತೆಗೆ ತಕ್ಕಂತೆ ವಿವರಿಸಿರುತ್ತೇನೆ. ಸಾಧ್ಯವಾದ ಕಡೆ\-ಗಳಲ್ಲಿ ಪ್ರಮಾಣಗಳನ್ನೂ ಉದಹರಿಸಿರುತ್ತೇನೆ.

\newpage

ಈ ಅನುವಾದ ಕಾರ್ಯದಲ್ಲಿ ನನಗೆ ಸೂಕ್ತ ಮಾರ್ಗದರ್ಶನವನ್ನು ಆಗಾಗ್ಯೆ ಮಾಡಿ ಉಪಯುಕ್ತ ಸಲಹೆಗಳನ್ನು ನೀಡಿದ ನನ್ನ ಮಾನ್ಯ ಮಿತ್ರರಾದ ಪ್ರೊ~।। ಆರ್. ರಾಮರಾಯರಿಗೆ ಕೃತ\-ಜ್ಞತೆಗಳನ್ನು ಸಲ್ಲಿಸುತ್ತೇನೆ. ಈ ಅನುವಾದವನ್ನು ಪ್ರಕಟಪಡಿಸಿರುವ ಶ‍್ರೀ ಮಧ್ವಸಮಾಜ\-ದವರಿಗೆ ನನ್ನ ವಂದನೆಗಳು.

\vskip 1cm

\begin{flushleft}
\hfill ಸಜ್ಜನ ವಿಧೇಯ,\\ಮದ್ರಾಸ್–33\hfill ಏರೀ ಸುಬ್ಬಣ್ಣಾಚಾರ್ಯ
\end{flushleft}

\begin{center}
***
\end{center}

\newpage

\phantom{}

\vfill

\begin{center}
।। ಶ‍್ರೀಃ~।।\\ ದ್ವೈತಸಿದ್ಧಾಂತ ಪ್ರಚಾರಕ್ಕಾಗಿಯೇ ತಮ್ಮ ಇಡೀ ಜೀವನವನ್ನೇ\\ ಮುಡುಪಾಗಿಟ್ಟು ತಮ್ಮ ಕೊನೆಯ ಉಸಿರಿನ ತನಕ\\ ಗ್ರಂಥಗಳನ್ನು ರಚಿಸಿ ಶ‍್ರೀ ಹರಿವಾಯುಗಳ\\ ಸೇವೆಯನ್ನು ಅತ್ಯುತ್ತಮ ರೀತಿಯಲ್ಲಿ\\ ಸಲ್ಲಿಸಿದ ಪ್ರಕಾಂಡಪಂಡಿತರಾದ\\\textbf{ಮಾಧ್ವಭೂಷಣ, ದಾವಣಗೆರೆ ಬಿ. ಭೀಮರಾಯರ}\\ ಪಾದಕಮಲಗಳಲ್ಲಿ\\ ಭಕ್ತಿಪುರಸ್ಸರವಾಗಿ ಅರ್ಪಿಸಿದ ಅಲ್ಪ ಕಾಣಿಕೆ.
\end{center}

\begin{flushright}
ಏರೀ ಸುಬ್ಬಣ್ಣಾಚಾರ್ಯ
\end{flushright}

