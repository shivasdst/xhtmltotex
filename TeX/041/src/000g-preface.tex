

\begin{center}
\phantom{\dev{।। ॐ नमो नारायणाय~।।}}
\end{center}

\chapter*{ಗ್ರಂಥ ಪ್ರಸ್ತಾವನೆ}

\vskip -10pt

ನಿತ್ಯನಿರ್ದುಃಖಾನಂದಮಯವಾದ ಪರಮಪರ (ಮೋಕ್ಷ) ಪ್ರಾಪ್ತಿಯು ಸಕಲ ಸುಜೀವರ\break ಮುಖ್ಯಗುರಿ. ಇಂತಹ ಫಲಪ್ರಾಪ್ತಿಯು ಭಗವಂತನ ಅತ್ಯರ್ಥ ಪ್ರಸಾದದಿಂದ ಮಾತ್ರ ಸಾಧ್ಯ.\break ಈ ಅತ್ಯರ್ಥ ಪ್ರಸಾದರೂಪ ಅನುಗ್ರಹವು ಕೇವಲ ಆತನ ಜ್ಞಾನದಿಂದಷ್ಟೇ ಲಭ್ಯ. ಇಂತಹ ಜ್ಞಾನವು ಸದ್ಗುರು ಸೇವಾರೂಪ ಶ್ರದ್ಧಾಭಕ್ತಿಯಿಂದ ಮಾತ್ರವೇ ದೊರಕುವಂತಹದು. ಅದಕ್ಕಾಗಿ ಹಿಂದಿನಿಂದಲೂ ಪರಂಪರಾಗತವಾಗಿ ಜ್ಞಾನ - ಪ್ರಸಾರ - ಜ್ಞಾನಾರ್ಜನೆಯ ಪ್ರಚಾರವು ಸಾಗು\-ತ್ತಲೇ ಬಂದಿದೆ.

\vskip 4pt

ಪ್ರಸ್ತುತ ಶ‍್ರೀ ಬಿದರಹಳ್ಳಿ ಶ‍್ರೀನಿವಾಸ ತೀರ್ಥರು ಶ‍್ರೀರಾಘವೇಂದ್ರ ಸ್ವಾಮಿಗಳವರ ವಿಶೇಷ ಅನುಗ್ರಹಕ್ಕೆ ಪಾತ್ರರಾಗಿ ಅವರಿಂದ "ತೀರ್ಥ'' ಎಂಬ ಬಿರುದನ್ನು ತಾವು ಗೃಹಸ್ಥಾ\-ಶ್ರಮದಲ್ಲಿದ್ದರೂ ಪಡೆದ ಮಹಾಪಂಡಿತರು. ಮಹಾಜ್ಞಾನಿಗಳಾದ ಶ‍್ರೀಯಾದವಾರ್ಯರಿಂದ ಮಧ್ವಸಿದ್ಧಾಂತವನ್ನು ವಿಶೇಷವಾಗಿ ಅಭ್ಯಾಸ ಮಾಡಿ ನಂತರ ಶ‍್ರೀಮನ್ಮಧ್ವಾಚಾರ್ಯರ ಗ್ರಂಥಗಳಿಗೆ ಅತ್ಯಂತ ಶ್ರೇಷ್ಠವಾದ ಟೀಕಾ ರಚಿಸಿದ ಶ‍್ರೀಜಯತೀರ್ಥರ ಟೀಕಾ ಗ್ರಂಥಗಳ ಮೇಲೆ ಅತ್ಯದ್ಭುತ ಟಿಪ್ಪಣಿಗಳನ್ನು ರಚಿಸುವುದರ ಮೂಲಕ ಟಿಪ್ಪಣಿಕಾರರೆಂದೇ ಪ್ರಸಿದ್ಧರಾಗಿದ್ದಾರೆ. ಮುಂದೆ ಇವರೇ ಶ‍್ರೀಯಾದವಾರ್ಯರ ಆಜ್ಞಾನುಸಾರ "ನಾರಾಯಣ ಶಬ್ದಾರ್ಥಃ" ಎಂಬ ಶ್ರೇಷ್ಠವಾದ ಗ್ರಂಥವನ್ನು ರಚಿಸಿ, ಅಲ್ಲಿ "ನಾರಾಯಣ" ಶಬ್ದಕ್ಕೆ ಶಾಸ್ತ್ರಸಮ್ಮತವಾದ ಎಪ್ಪತ್ತೆರಡು (೭೨) ಅರ್ಥಗಳನ್ನು ಮನೋಜ್ಞವಾಗಿ ಶ‍್ರೀಹರಿಯ ಸರ್ವೋತ್ತಮತ್ವವನ್ನು ದ್ವೈತ ಸಿದ್ಧಾಂತ\break ಪ್ರಮೇಯಗಳ ಸಹಿತವಾಗಿ ಅರ್ಥೈಸಿರುವುದು, ಅವರ ಅಗಾಧವಾದ ಪಾಂಡಿತ್ಯಕ್ಕೆ ಸಾಕ್ಷಿ\-ಯಾಗಿದೆ.

\vskip 4pt

ಇಂತಹ ಗ್ರಂಥರತ್ನವನ್ನು ಸಂಸ್ಕೃತ ಭಾಷಾ ಜ್ಞಾನದ ಕೊರತೆಯಿರುವವರಿಗಾಗಿ ಪಂಡಿತರಾದ ಶ‍್ರೀ ಏರೀ ಸುಬ್ಬಣ್ಣಾಚಾರ್ಯರು ಕನ್ನಡದಲ್ಲಿ ಸರಳವಾಗಿ ನಿರೂಪಿಸಿದ್ದಾರೆ. ಅವಶ್ಯವಿದ್ದಲ್ಲೆಲ್ಲಾ ದ್ವೈತ ಸಿದ್ಧಾಂತದ ಪ್ರಮೇಯಗಳನ್ನು ಪ್ರಮಾಣ ಸಹಿತವಾಗಿ ಉದಹರಿಸಿರುವುದು ಈ ಗ್ರಂಥದಲ್ಲಿ ಕಾಣುವ ಮತ್ತೊಂದು ವಿಶೇಷ.

ಈಗ ಇಂತಹ ಪರಮಶ್ರೇಷ್ಠವಾದ ಈ ಹೊತ್ತಿಗೆಯನ್ನು ಪುನಃ ಮುದ್ರಣಮಾಡಿಸುವ ಕಾರ್ಯವನ್ನು ದಾವಣಗೆರೆಯ ಶ‍್ರೀ ಸರ್ವಜ್ಞಾಚಾರ್ಯ ಸೇವಾ ಸಂಘದವರು ಕೈಗೆತ್ತಿಕೊಂಡಿರುವುದು ನಿಜಕ್ಕೂ ಸ್ತುತ್ಯವಾದ ಕಾರ್ಯ. ಮೊದಲಿನಿಂದಲೂ ಗ್ರಂಥ ಮುದ್ರಣಾ ಕಾರ್ಯದಲ್ಲಿ ವಿಶೇಷ ಆಸಕ್ತಿ ತೋರುತ್ತಿರುವ ಈ ಸಂಘವು ಇಂತಹ ಇನ್ನೂ ಅನೇಕ ಉದ್ಗ್ರಂಥಗಳ ಮುದ್ರಣ, ಪುನರ್ಮುದ್ರಣ ಕಾರ್ಯವನ್ನು ಮಾಡಿಸುವ ಮೂಲಕ ಮಧ್ವಸಿದ್ಧಾಂತ ಪ್ರಚಾರರೂಪ ಸೇವಾ ನಡೆಸಲಿ ಎಂದು ಹಾರೈಸುವ,

\begin{flushright}
ಇತಿ ಸಜ್ಜನ ವಿಧೇಯ\\\textbf{ಶ‍್ರೀ ಮಣ್ಣೂರು ಗೋಪಾಲಾಚಾರ್ಯ}
\end{flushright}

\noindent
ಸ್ವಭಾನು ಸಂವತ್ಸರ ಚೈತ್ರಶುದ್ಧ ಅಷ್ಟಮಿ\\\textbf{ಶ‍್ರೀ ಶ‍್ರೀ ಸತ್ಯಧ್ಯಾನತೀರ್ಥರ ಆರಾಧನಾ ದಿನ}\\ ದಾವಣಗೆರೆ - ೨\\ ೧೦-೪-೨೦೦೩

