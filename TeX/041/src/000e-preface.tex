
\chapter*{PUBLISHER'S NOTE}

Sri Madhwa Samaj was started in West Mambalam, Madras, on 8-10-1967 with the lofty ideal of helping Madhwas learn the philosophy of Acharya Sri Madhwa as also Sadacharas. This is being achieved by holding religious classes and lectures every Sunday and Tuesday; Bhakti Sudha is being tasted in the weekly Bhajans every Saturday. Some of the sastras on which classes were held so far include Sri Harikathamrutasara, Sri Sumadhwa Vijaya, Brahmsutra Bhasya with the aid of Bhashya Deepika, Sri Bhagavad Geetha, Vishnu Rahasya, Magha Masa Mahatmya, Tatwa Sankhyana, Sangraha Ramayana, Sarasabharati Vilasa, Rukminisha Vijaya.

In 1971, in connection with the Tri-Centenary of Sri Raghavendra Swamiji, a booklet was published by the Samaj containing among others Sri Ramacharitra Manjari and Sri Krishnacharitra Manjari with the financial assistance of Sri Ganga Bai Charities' trustee, Sri K. Balakrishna Rao. It was distributed free of cost.

We are now publishing Narayana Sabdartha of Sri Srinivasa Teertha, a deep scholar and a disciple of Sri Yadavaryaru, and one who was blessed by Sri Raghavendra Swamiji. The Sanskrit text is translated into Kannada with commentary. Our thanks are due to Sri A. R. Subbannachar, Retd. Gazetted Post-Master, for the translation and commentary as well as for handling classes on Bhagavad Geetha based on the Vivruti of Sri Raghavendra Swamiji.

We are indebted to the Tirumala-Tirupati Devasthanams authorities for their financial aid to bring out this work.

The Samaja is grateful to the following for having rendered monetary aid:

\begin{enumerate}
\item 
 The Secretary (Sri S.A.N. Ranganathachar)

 Sri Madhwa Siddanthonnahini Sabha, Tiruchanur

 \item Sri M. R. Gopalakrishnan, Chitradurga

 \item Sri Krishna Dwaipayanacharya, Chitradurga

 \item Sri K. Sama Rao, Bombay

 \item Sri K. Seshagiri Rao, Vridhachalam

\end{enumerate}

We thank Sri D. S. Krishnachar, M.Sc., Proprietor of Prabha Printing House, Bangalore, for having got the book printed neatly, punctually and faultlessly.

We hope that this publication will enthuse Madhwas in particular to come forth in larger numbers to learn our philosophy and thus give us a fillip in our noble efforts.

\noindent
Madras 600 033\\ 30th September 1983

\begin{flushright}
\textbf{C. Nagaraja Vittal Rao}\\\textit{Vice-President}
\end{flushright}

