
\chapter{A Study of Vedic References in the Sixth \textit{Tirumuṟai} of Appar’s \textit{Tēvāram}}\label{intro}

\Authorline{R. Saraswati Sainath}


\section*{Abstract}

Tirumuṟai-s are set of hymns in praise of Lord Śiva in Tamiḻ. These Tirumuṟai-s were composed by many saints, popularly called the Nāyaṉār-s. Tirunāvukkaracar or Appar, as he is known, is an important Nāyanār. He forms part of the group of four important saints called \textit{nālvar} along with Tiruñānacampantar, Cuntarar and Māṇikkavācakar. Appar’s hymns occupy the fourth, fifth and sixth books of the Tirumuṟai-s. His hymns are called as \textit{patikam-}s which are usually a set of ten hymns. Some \textit{patikam-}s, however, consist of more than ten hymns. The fourth \textit{Tirumuṟai} consists of 113 \textit{patikam-}s, the fifth 100, and the sixth 99. In this paper, though, I will confine myself to the sixth. This paper studies the Vedic references and culture that are represented by Appar in these hymns. The sixth \textit{Tirumuṟai} begins with the description of Śiva as one who is present in the hearts of the Brahmins. Likewise, while praising the Śiva of Tiruviṭaimarutūr, Appar makes specific reference to the Aṣṭasahasram group of Tamiḻ \textit{smārta} brahmins. It is noteworthy that when Appar was tied to a stone and cast into the ocean to drown, Appar addresses Śiva as one who recites the Veda-s. This shows his reverence towards the Veda-s even in the most dangerous situation in his life. Thus, it is evident that a study of the Vedic references in the sixth \textit{Tirumuṟai} will enable us to get a good picture of the importance of Vedic culture in the Tamiḻ speaking South at the times of the Nāyaṉār-s and will also enable us to study the development of Śaivism as a Vedic religion in South India.

\textbf{The Nāyanmār:} The sixty-three Nāyaṉar-s were responsible for propagating \textit{Śiva-bhakti} in Tamiḻ Nāṭu. Of these sixty-three saints, twenty-seven have composed hymns praising Śiva. These are called\break Tirumuṟai-s (“the sacred books”). Nampi Āṇṭār Nampi, a Śaiva scholar (860-907 C.E), has classified these hymns in eleven sections (Vaittiya\-nathan 1995:12). The content of these can be described as follows: the first, second and third Tirumuṟai-s contain the hymns of Tiruñānacampantar; the fourth, fifth and sixth, those of Appar or Tirunāvu\-kkaracar; the seventh, those of Cuntarar; (index Sundarar) the eighth, those of Māṇikkavācakar; the ninth, those of Cēṇtanār, Karuvūrttēvar, PūnturuttiNampi, Kaṇṭarātittar, Vēṇāṭṭaṭikaḷ, Tiruvāliyamutaṉār,\break Puruṭōttama Nampi, Cētirāyar and Cēntanār; the tenth, those of\break Tirumūlar, which are called the \textit{Tirumantiram}; and the eleventh, those of TiruālavāyUṭayār, KāraikkālAmmaiyār, IyaṭikaḷKāṭavarkōṉ,\break CēramāṉPerumāḷ, Nakkīratēvar, Kallāṭatēvar, Paraṇatēvar,\break IḷamperumāṉAṭikaḷ, Atirāvaṭikaḷ, PaṭṭiṇattupPiḷḷaiyār and Nampi\break Āṇṭār Nampi. The twelfth section is Cēkkiḻar’s hagiographies of the Nāyaṉār-s, a text that is called \textit{Tiruttoṇṭarpurāṇam}, or \textit{Periyapurāṇam} (Vaittiyanathan 1995:36).These hymns total about 18,497 verses in all (Vaittiyanathan 1995: 7-8).

The hymns of Tiruñānacampantar are called as \textit{Tirukkaṭaikkāppu}, those of Appar, as \textit{Tēvāram} and those of Cuntarar as \textit{Tiruppāṭṭu }(Vaittiya\-nathan 1995:14).However, the hymns of the \textit{mūvar} (Tiruñānacampantar, Appar and Cuntarar), are popularly referred to as \textit{Tēvāram}. This reveals the greatness of Appar.

The Nāyaṉār-s visited various Śiva temples and composed hymns on Śiva, describing his exploits and physical beauty. They set these hymns to musicaltunes called \textit{paṇ}, equivalent to \textit{raga}-s of South Indian classical music (Vaittiyanathan 1995:18-20). The templeswhere these saints worshipped, especially \textit{mūvar}, came to be called \textit{pāṭalpeṟṟastalaṅkaḷ} (Sanskrit \textit{sthala}). There are 276 of these \textit{sthala}-s.

\textbf{Appar:} Appar was the earliest of the \textit{mūvar}. Scholars have established that he lived from the second half of the sixth century C.E to the middle of the seventh. Of the \textit{mūvar, }Appar may be considered the most important, as, of the three, only Appar lacked a divine element.Tiruñānacampantar owed his poetic skill to the milk of Goddess Pārvatī, which he drank at the age of three. Cuntarar (index Sundarar) was Śiva’s banished attendant from Kailāsa. Appar, therefore, could be the perfect model of devotion and service for his fellow human beings.


\section*{Life of Appar}

\textbf{Appar’s Birth and Early Life:} According to Cēkkiḻār’s hagiography in the \textit{Periyapurāņam}, Appar was born in Tiruvāmūr, in the Śaivavēḷāḷa—an agricultural caste—the son of Pukaḻaṉār and Mātiṉi. They belonged to the family of Kurukkaiyar-s. Appar’s parents named him Maruṇīkkiyār. Appar had an elder sister by name Tilakavati (Mudaliyar 1975: 1277, 1280-83). At the age of twelve, Tilakavati was betrothed to Kalippakayār, a member of the king’s army. Maruṇīkkiyār’s parents passed away. In the meanwhile, Kalippakayār was killed in a battle. Hearing the news of Kalippakayār’s death, Tilakavati decided to commit \textit{sati}, but consented to live for the sake of Maruṇīkkiyār, and led an ascetic life.As Maruṇīkkiyār grew up, he devoted himself to acts of charity such as donating food and money to the needy, erecting water sheds for the travellers and rendering hospitality to guests who visited his house. Realizing the impermanence of life, Maruṇīkkiyār left for Pāṭaliputra (Tiruppātalipuliyūr near modern Cuddalore) to become a Jain. He became well versed in Jain scriptures and took a new name Tarumacēṉar (Sanskrit “Dharmasena”).In the \textit{Periyapurāņam, }Cēkkiḻār writes that in those times, Śaivism was in great danger especially from Jainism and in order to safeguard the faith and restore its importance, Appar approached the Jain academies to understand their philosophy (Vanmikanathan 1985: 275). 

\textbf{Appar’s return to Śaivism and its consequences:} Tilakavati who was staying at TiruvatikaiVīraṭṭāṉam was rendering services to the Śiva temple, such as cleaning the temple premises andmaking flower garlands for the deity. Upset by her brother’s conversion, Tilakavati prayed to Śiva for his return to Śaivism. Śiva appeared in her dream and promised to grace her brother (\textit{āṭkoḷḷutal}). Accordingly, Śiva gave Tarumacēṉar a stomach-ache. Finding no effective remedy from the Jains, Maruṇīkkiyār returned to Tilakavati. She took him to the Śiva temple, gave him the sacred ash, and chanted the five-syllable \textit{mantra}. Śiva cured him and gave him a new name: Tirunāvukkaracar (Tiru “respectful” nāvukkaracar “lord of speech”). He became a great devotee of Śiva. He wore the symbols of Śaivism such as the sacred ash and \textit{rudrākṣa} beads and rendered service to the Śiva temple by cleaning the temple premises of grass and weeds with a hoe. His expressed his intense devotion to Śiva by composing many Śaiva hymns.

Unable to withstand his reconversion to Śaivism, the Jains with the support of the king, tortured Tirunāvukkaracar in many ways. These included shutting him in a lime kiln, poisoning him, making the royal elephant trample him, and binding him to a rock and drowning him in the sea. Tirunāvukkaracar escaped all these challenges by the grace of Śiva which he acquired by composing hymns appropriate to the situation. Realizing his greatness, the king apologized for his mistake and converted to Śaivism. He destroyed the Jain monasteries and built a temple for Śiva at Guṇaparavīccaram (Mudaliyar 1975: 1411).

\textbf{Tirunāvukkaracar’s pilgrimage:} Tirunāvukkaracar then started on a pilgrimage to many Śiva temples. These include Tiruveṇṇainallūr, Tiruāmattūr, Tirukkōvalūr, Tiruppeṇṇākaṭam and so on. In all these places he continued his services to Śiva. At Tiruppeṇṇākaṭam, he requested Śiva to mark the emblems of trident and bull on his shoulders to show his surrender to Śiva, and also to cleanse himself of the defect that arose due to the association with the Jains. A goblin of Śiva unnoticed by anyone fulfilled the request of Tirunāvukkaracar. Overjoyed by the appreciation, Tirunāvukkaracar continued his pilgrimage to other Śiva temples. At Tiṅkaḷūr he visited Appūti Aṭikaḷ, his devotee, and accepted his invitation for lunch. Appūti Aṭikaḷ’s son died of a snake bite and Appar restored him back to life by singing a hymn on Śiva.

\textbf{Tirunāvukkaracar and Tiruñānacampantar:} Tirunāvukkaracar joined Tiruñānacampantar at Cīrkāḻi. On another occasion, while Tirunāvukkaracar showed his respect to the young Campantar by being one of his palanquin bearers, Campantar in turn showed his respect for Tirunāvukkaracar by respectfully calling him Appar (Appaṉ “father” with the honorific suffix “r”).Together, they visited many temples and sang hymns in those places. They also conducted miracles. Two of those miracles—related to Tirvīḻimiḻalai and Tirumaṛaikkāṭu—require mention. At Tirvīḻimiḻalai these two devotees eradicated the famine by getting gold coins from Śiva. It is interesting to note that while the gold coins given to Appar were of higher quality, those of Campantar’s (index Sambandhar) were of slightly inferior in quality. At Tirumaṛaikkāṭu both the saints sang hymns to open and close the temple door which was locked for many years.

\textbf{Appar’s journey and his liberation:} Parting from Campantar,Appar continued his pilgrimage. Despite his ill health, Appar was determined to continue his journey to Kailāsa.Appreciating his determination, Śiva gave him a fresh body and miraculously transported him to Tiruvaiyāru, where Appar had a vision of Śiva as he is in Kailāsa. To reveal the greatness of Appar to the world, Śiva presented before him precious gems and heavenly nymphs. But Appar, the ascetic, overcame all these temptations. Finally, at Tiruppukalūr Appar attained union with Śiva (Ramachandran 1995: 16-25).

\textbf{Hymns of Appar:} Appar’s hymns are called \textit{patikam-}s. Most have ten stanzas. Sometimes, however, a \textit{patikam} has more than ten stanzas. His fourth \textit{Tirumuṟai }contains 113 \textit{patikam}-s, the fifth 100 \textit{patikam}-s, and the sixth 99. Some \textit{patikam}-s praise particular temples; others praise all of them. This paper, however, shall confine itself to the sixth \textit{Tirumuṟai} which is set in \textit{tāṇṭaka} meter known for its exceptional musical quality, lucidity and simplicity. Appar is praised as the king of \textit{tāṇṭaka }composers (Nagaswamy 1989: 37).

\textbf{Vedic References:} References to Vedic terms are found throughout the hymns of Appar. This paper scans hundreds of verses in the sixth \textit{Tirumuṟai. }It is observed that almost every \textit{patikam} of Appar has at least one reference to the Veda-s. Appar describes Śiva as the very embodiment of Veda-s. It is Śiva who appears as the Veda-s consisting of the six auxiliaries of \textit{śikṣā}, \textit{vyākaraṇa}, \textit{kalpa}, \textit{jyotiṣa}, \textit{chandas} and \textit{nirukta}. It is Śiva who appears as \textit{mantra} and the meaning of the Veda-s. Śiva is of the nature of \textit{Oṃkāra }that is expounded in the Veda-s. Śiva himself is the brahmin and performs the Vedic sacrifice. Wearing the sacred thread is one of the characteristic features of Śiva. Śiva is worshipped by the sacred Aṣṭasahasram sect of Tamiḻ \textit{smārta} brahmins (\textit{ennāyirattār}).

Several instances of Vedic references in the sixth \textit{Tirumuṟai} of Appar’s \textit{Tēvāram} are provided in the table below. The continuous numbering for the \textit{Tirumuṟai }has been followed and the \textit{patikam-}wise numbering for the sixth \textit{Tirumuṟai }are given within brackets.

\begin{longtable}{|m{2.7cm}|m{2.7cm}|>{\raggedright}m{3cm}|}
\hline
\multicolumn{1}{|m{2.5cm}}{\centering \textit{Patikam} No.} & \multicolumn{1}{|m{2.5cm}}{\centering \textit{Tirumuṟai} Reference} & \multicolumn{1}{|m{2.5cm}|}{\centering Text} \\
\hline
6.1 \textit{KōilPeriyaTiruttāṇṭakam} & 6244 (6.1.1) & \textit{antaṇartamcintaiyānai}  (Śiva is thought by the brahmins). \tabularnewline
\hline
 &  & \textit{arumaṟaiyinakattānai} (He is the meaning of the Veda-s). \tabularnewline
\hline
 & 6249 (6.1.6) & \textit{arumaṟaiyōṭāraṅkamā\-yinānai}(He became the Veda-s with the six limbs). \tabularnewline
\hline
6.2 \textit{Kōil Pukka Tiruttāṇṭakam} & 6255 (6.2.2) & \textit{vētamumvēlvippukai\-yumōvāvirinīrmiḻalai\-eḻunāḷtaṅki} (Śiva resides in Virinīrmiḻalai which is filled with Vedic chants and sacrifices). \tabularnewline
\hline
 & 6256 (6.2.3) & \textit{mantiramumtantira\-mumtāmēpōlum araṅkāṭṭiyantaṇarkaṉṟālanīḻalaṟamaruḷicceytaaraṉār} (Hara became the both \textit{mantra }and \textit{tantra} and he preached \textit{dharma} to the brahmins under the shade of the banyan tree). \tabularnewline
\hline
 & 6262 (6.2.9) & \textit{veṇnūṉmārpar} (He wears the white sacred thread on his chest). \tabularnewline
\hline
 & 6264 (6.2.11) & \textit{veṇṇūluṇṭēōtuvatum\-vētamē} (He wears the sacred thread and he chants the Veda-s). \tabularnewline
\hline
6.3 \textit{Tiruatikaivīraṭṭāṉam-ēḻaittiruttāṇṭakam} & 6268 (6.3.4) & \textit{mantiramummaṟai\-poruḷumāṉāṉtaṉṉai} (He became the \textit{mantra}-s and the meaning of the Veda-s). \tabularnewline
\hline
 & 6273 (6.3.9) & \textit{maṟaiyāṉai} (Śiva is of the nature of Veda-s). \tabularnewline
\hline
6.4 \textit{Tiruatikaivīraṭṭā\-ṉam-aṭaiyāḷatti\-ruttāṇṭakam} & 6276 (6.4.1) & \textit{cāmavētakantaruvam\-virumpumē} (He likes the \textit{Sāma Veda} and music). \tabularnewline
\hline
 & 6280 (6.4.5) & \textit{pāṭumēoḻiyāmēnālvēta\-mum} (He ceaselessly sings all the four Veda-s). \tabularnewline
\hline
 & 6284 (6.4.9) & \textit{nāṉmaṟaikaḷtoḻaniṉṟāṉē} (He is worshipped by the four Veda-s). \tabularnewline
\hline
6.5 \textit{Tiruatikaivīra\-ṭṭāṉam-pōṟṟitti\-ruttāṇṭakam} & 6294 (6.5.8) & \textit{nāṉmaṟaiyōṭāraṅkamā\-naipōṟṟi}(Praise to Him who became the four Veda-s with the six limbs). \tabularnewline
\hline
6.6 \textit{Tiruatikaivīraṭṭāṉam-Tiruvaṭittiruttāṇ\-ṭakam} & 6297 (6.6.1) & \textit{arumaṟaiyāṉceṉṉikka\-ṇiyāmaṭi}(Śiva’s feet adorns the head of Brahmā who chants the four Veda-s). \tabularnewline
\hline
 & 6634 (6.6.8) & \textit{mantiramumtantira\-mumāyavaṭi} (The feet of Śiva  became the \textit{mantra}-s and tantra-s). \tabularnewline
\hline
6.11. \textit{Tiruppaṇkūrum\-Tirunīṭūrum\-Tiruttāṇṭakam} & 6355 (6.11.6) & \textit{pūṇalāppūṇāṉai}( He wears the serpent as the sacred thread which nobody wears). \tabularnewline
\hline
6.12 \textit{Tirukkaḻippālai- Tiruttāṇṭakam} & 6364 (6.12.5) & \textit{Vētattāi} (He became the Veda-s). \tabularnewline
\hline
6.16 \textit{Tiruviṭaimarutūr-Tiruttāṇṭakam} & 6402 (6.16.1) & \textit{mantiramumtantira\-mumāṉārpōlum}  (He became the \textit{mantras }and the Āgama-s which are called as Tantra-s). \tabularnewline
\hline
 & 6404 (6.16.3) & \textit{vētaṅkaḷvēḷvipayantār\-pōḷum} (He authored the Veda-s and the sacrifices prescribed in the Vedas). \tabularnewline
\hline
 & 6405 (6.16.4) & \textit{eṇkuṇattāreṇṇāyiravar\-pōlum} (Śiva is worshipped by the group of Brahmins consisting of 8000 who are popularly called as Aṣṭasahasram; He is infinite). \tabularnewline
\hline
 & 6407 (6.16.6) & \textit{aṟumūṉṟumānārpōlum} (He became the eighteen branches of knowledge: four Veda-s , sixVedāṅgas, Purāṇa-s, Nyāya, Mīmāṃsā, Smṛti, Āyurveda, Dhanurveda, Gāndharvaveda and Arthaśāstra). \tabularnewline
\hline
6.17 \textit{Tiruviṭaimarutūr-Tiruttāṇṭakam} & 6416 (6.17.5) & \textit{tūyamaṟaimoḻiyar} (Śiva is the speaker of the pure Vedic language). \tabularnewline
\hline
 & 6417 (6.17.6) & \textit{pulittōlarpoṅkaravar\-pūṇanūlaraṭiyārkuṭi\-yāvarantaṇāḷarākuti\-yinmantirattār} (He wears the tiger skin and wears the serpent as the sacred thread and  he is the \textit{mantra} and the sacrificial offering offered by the brahmins). \tabularnewline
\hline
6.19 \textit{Tiruvālavāi- Tiruttāṇṭakam} & 6437 (6.19.5) & \textit{arumaṟaiyālnāṉmuka\-ṉummālumpōṟṟumcī\-rāṉai} (Brahmā and Viṣṇu praise him through Veda-s). \tabularnewline
\hline
6.23 \textit{Tirumaṟaikkāṭu- Tiruttāṇṭakam} & 6479 (6.23.5) & \textit{āriyaṉkaṇṭāytamiḻaṉ\-kaṇṭay} (Śiva is Ārya and Tamiḻ). \tabularnewline
\hline
 & 6481 (6.23.7) & \textit{pālnaycērānañcumāṭi\-kaṇṭāy} (Śiva delights in the \textit{abhiṣeka}of \textit{pañcagavya}consisting of milk and ghee). \tabularnewline
\hline
 & 6483 (6.23.9) & \textit{muttamiḻumnāṉmaṟai\-yumāṉāṉkaṇṭāy} (Śiva became the Tamiḻin the form of literature, music and drama and also the four Veda-s). \textit{āliṉkīḻnālvarkkaṟantāṉkaṇṭāy} (Śiva instructed \textit{dharma} to the four under the banyan tree). \tabularnewline
\hline
6.25 \textit{Tiruvārūr- Tiruttāṇṭakam} & 6502 (6.25.8) & \textit{aḷaviṟkuṉṟāaviyaṭuvā\-raṟumaṟyōraṟintēṉu\-ṉṉai} (The brahmins offer the right amount of sacrificial offering to Śiva). \tabularnewline
\hline
6.26 \textit{Tiruvārūr- Tiruttāṇṭakam} & 6507 (6.26.2) & \textit{ōtātēvetamuṇarntāṉta\-ṉṉai} (Śiva has mastered the Veda-s without learning them). \tabularnewline
\hline
 & 6510 (6.26.5) & \textit{periyavētattuṇṭattiltu\-ṇiporulai} (Śiva is mentioned  as the meaning of the meters of the Veda-s). \tabularnewline
\hline
6.28\textit{Tiruvārūr- Tiruttāṇṭakam} & 6527 (6.28.6) & \textit{collākicoṟporuḷāiniṉṟār\-pōlum} (He became the word and the meaning of the word). \textit{nāmanaiyumvētattārtāmēpōlum} (He is the personification of the Veda-s that are recited by the tongue). \tabularnewline
\hline
 & 6529 (6.28.8) & \textit{mantirattaimaṉattuḷḷē\-vaittārpōlum} (He keeps the \textit{Namaśivāya mantra} in the minds of his devotees). \tabularnewline
\hline
6.33 \textit{Tiruvārūr- Tiruttāṇṭakam} & 6575 (6.33.3) & \textit{vētiyaṉaitaṉṉaṭiyār\-kkeliyāṉtaṉṉai}(Śiva is the brahmin who has mastered the Veda-s and one who is easily accessible to his devotees). \tabularnewline
\hline
 & 6576 (6.33.4) & \textit{antaṇanaiaṟaneriyila\-ppaṉtaṉṉai}(Śiva is the brahmin who is the father of \textit{dharma}). \tabularnewline
\hline
6.37 \textit{Tiruvārūr- Tiruttāṇṭakam} & 6615 (6.37.3) & \textit{nāvalarkaḷnāṉmaṟaiyē} (He is the Veda-s of the learned). \tabularnewline
\hline
6.43 \textit{Tiruppūnturutti- Tiruttāṇṭakam} & 6675 (6.43.5) & \textit{nakkāṉainālmaṟaikaḷ\-pāṭiṉānai} (He chanted the four Veda-s). \tabularnewline
\hline
6.44 \textit{Tirucōṟṟuttuṟai- Tiruttāṇṭakam} & 6683 (6.44.3) & \textit{ōtumvētaṅkaṟṟānē} (He learned the Veda-s). \tabularnewline
\hline
 & 6684 (6.44.4) & \textit{vētanāivētamvirittiṭṭānē} (He arranged the Veda-s). \tabularnewline
\hline
 & 6685 (6.44.5) & \textit{nampanēnāṉmaṟaika\-ḷāyinānē} (He became the four Veda-s). \tabularnewline
\hline
6.45 \textit{Tiruvoṟṟiyūr- Tiruttāṇṭakam} & 6692 (6.45.2) & \textit{ōmattālnāṉmaṟaikaḷō\-talōvā}( He lives in the city where the four Veda-s are chanted at the time of the sacred fire). \tabularnewline
\hline
6.46 \textit{Tiruāvaṭutuṟai- Tiruttāṇṭakam} & 6701(6.46.1) & \textit{nampaṉainālvētaṅ\-kaṟaikaṇṭāṉai}( I hail him who is an expert in the four Veda-s). \tabularnewline
\hline
 & 6703 (6.46.3) & \textit{colluvārcoṟporuḷintōṟṟa\-māki }(He became the meaning of the words). \tabularnewline
\hline
 & 6710 (6.46.10) & \textit{centamiḻōṭāriyanaiccīri\-yānaittirumārpilpuri\-veṇnūltikaḻappūṇṭaan\-taṇaṉai}( I hail the one who became the  classical Tamiḻ and Sanskrit and the brahmin who wears the sacred thread on his chest). \tabularnewline
\hline
6.48 \textit{Tiruvalivalam- Tiruttāṇṭakam} & 6722 (6.48.1) & \textit{nallāṉkāṇnāṉmaṟai\-kaḷāyiṉāṉkāṇ} (Behold the one who became the four Veda-s). \tabularnewline
\hline
 & 6730 (6.48.9) & \textit{nītiyaṉkāṇvētiyaṉkāṇ} (Behold the lord of justice and the expert in Veda-s). \tabularnewline
\hline
6.50 \textit{Tiruvīḻimiḻalai- Tiruttāṇṭakam} & 6745 (6.50.4) & \textit{cantōkacāmamōtumvā\-yāṉai}( I bow down to the one who chants the \textit{ChandogaSāma Veda}) \textit{mantarippārmaṉattulānai}( He resides on the minds of those who chant the \textit{mantra}-s). \tabularnewline
\hline
6.51 \textit{Tiruvīḻimiḻalai- Tiruttāṇṭakam} & 6753 (6.51.2) & \textit{pūṇanūlār} (He wears the sacred thread). \textit{vētikuṭiyuḷḷār}( He lives in Vētikuṭi). \tabularnewline
\hline
 & 6754 (6.51.3) & \textit{antaṇarkaḷmāṭakkōil}( He is worshipped in the mind temple of brahmins). \tabularnewline
\hline
 & 6755 (6.51.4) & \textit{vētanāvār} (He chants the Veda-s). \tabularnewline
\hline
 & 6756 (6.51.5) & \textit{puṭaicūḻntapūtaṅkaḷ\-vētampāṭa} (He is surrounded by the goblins who sing the Veda-s). \tabularnewline
\hline
6.53\textit{ Tiruvīḻimiḻalai- Tiruttāṇṭakam} & 6776 (6.53.4) & \textit{aivēḷviāṟaṅkamāṉār\-pōlum} (He became of the form of five- fold sacrifices and the six auxiliaries of the Veda-s). \tabularnewline
\hline
 & 6778 (6.53.6) & \textit{nālāyamaṟaikkiṟaiva\-rāṉārpōlum}( He became the lord of the four Veda-s). \tabularnewline
\hline
6.54 \textit{Tiruppuḷḷirukkuvēlūr- Tiruttāṇṭakam} & 6787 (6.54.4) & \textit{āraṅkamnālvētatta\-ppālniṉṟaporuḷānai}( He is beyond the four Veda-s and the six auxiliaries). \tabularnewline
\hline
 & 6790 (6.54.7) & \textit{vētiyanaivētavittai} (The one who is the brahmin and the seer of the Veda-s). \tabularnewline
\hline
 & 6791 (6.54.8) & \textit{mantiramumtantira\-mummaruntumāki} (He became the \textit{mantra}-s, Tantra-s and medicine). \tabularnewline
\hline
 & 6792 (6.54.9) & \textit{nāṉmaṟaiyiṉnaṟporuḷai} (He became the meaning of the four Veda-s). \tabularnewline
\hline
6.56 \textit{Tirukkayilāyam-Pōṟṟittiruttāṇṭakam} & 6805 (6.56.1) & \textit{maṟaiyuṭayavētamviri\-ttāypōṟṟi} (Praise to you who explained the Veda-sthat have secret meaning). \tabularnewline
\hline
6.57 \textit{Tirukayilāyam- Pōṟṟittiruttāṇṭakam} & 6819 (6.57.5) & \textit{mantiramumtantira\-mumāṉāypōṟṟi} (Praise to you who became the \textit{mantra-s} and Tantra\textit{-s}). \tabularnewline
\hline
6.60 \textit{Tirukkaṟkuṭi- Tiruttāṇṭakam} & 6849 (6.60.6) & \textit{vētiyanaivētattiṉkītam\-pāṭumpaṇṇavaṉai} (Behold the one who is the brahmin and chants the tune of the Veda-s ). \tabularnewline
\hline
6.62 \textit{Tiruvāṉaikkā- Tiruttāṇṭakam} & 6867 (6.62.5) & \textit{vemmāṉamatakariyi\-ṉurivaippōrttavētiyaṉē} (The brahmin who is wearing the skin of the elephant). \tabularnewline
\hline
 & 6871 (6.62.9) & \textit{viṇṇārumpuṉalpoti\-ceñcaṭaiyāyvētaneṟi\-yānē} (Śiva follows the path laid down in the Veda-s and has divine river in his matted hair). \tabularnewline
\hline
6.63 \textit{Tiruvāṉaikkā- Tiruttāṇṭakam} & 6875 (6.63.3) & \textit{ōmkārattuṭporuḷai} (Śiva is the inner meaning of   Oṃkāra). \tabularnewline
\hline
 & 6876 (6.63.4) & \textit{teivanāṉmaṟaikaḷpūṇ\-ṭatērāṉait}(He rode the chariot yoked by the four Veda-s). \tabularnewline
\hline
 & 6879 (6.63.7) & \textit{malarkkoṉṟaittolṉūl\-pūṇṭavētiyaṉai} (behold the brahmin who is wearing the \textit{koṉṟai} flowers on his head and the sacred thread on his chest). \tabularnewline
\hline
 & 6881`(6.63.9) & \textit{nacaiyāṉainālvētatta\-ppālāṉai} (He is beyond the four Veda-s). \tabularnewline
\hline
6.64 \textit{Tiruvēkampam - Tiruttāṇṭakam} & 6892 (6.64.10) & \textit{vētattiṉporulāṉkāṇ} (Behold the one who is the meaning of the four Veda-s). \tabularnewline
\hline
6.68 \textit{Tirumutukuṉ\-ṟam-Tiruttāṇ\-ṭakam} & 6929 (6.68.6) & \textit{viḻavuoliyumviṇṇoli\-yumāṉāṉtaṉṉai} (He became the festive sound and sound in the space). \tabularnewline
\hline
6.69 \textit{Tiruppaḷḷiyiṇmu\-kkūṭal-Tiruttāṇ\-ṭakam} & 6935 (6.69.2) & \textit{vētiyaṉaiveṇtiṅkaḷcū\-ṭuñceṉṉiccaṭaiyāṉāi\-cāmampōlkaṇṭattāṉāi} (I bow down to the brahmin who wears the white moon on his matted hair  and who chants the \textit{Sāma Veda}). \tabularnewline
\hline
6.74 \textit{Tirunāraiyūr- Tiruttāṇṭakam} & 6985 (6.74.9) & \textit{mēlāyavētiyarkkuvēl\-viyākivēlviyiṉiṉpaya\-ṉāyavimalaṉtaṉṉai\-nālāyamaṟaikiṟaiva\-ṉāyiṉāṉai. nāraiyūrnaṉṉakariṟkaṇṭēṉnāṉē} (Ibehold the pure one who became the Vedic sacrifices, the result of those Vedic sacrifices and the lord of the Veda-s in the holy city of Nāraiyūr). \tabularnewline
\hline
6.75 \textit{Tirukkuṭantai-kīḻkōṭṭam- Tiruttāṇ\-ṭakam} & 6987 (6.75.1) & \textit{coṉmalintamaṟaināṉ\-kāṟaṅkamākiccoṟpo\-ruṅkaṭantacuṭarccōti\-pōlum} (He became the four Veda-s and six auxiliaries, and he is of the nature of effulgent light that is beyond word and meaning). \tabularnewline
\hline
 & 6990 (6.75.4) & \textit{mikkatiṟalmaṟaiyava\-rālviḷaṅkuvēḷvi mikupukaipōyviṇpoḻi\-yakkaḻaṉiyellāṅ}  (The smoke arising the sacrifices performed by the brahmins touch the sky and spread over all the fields). \tabularnewline
\hline
6.81 \textit{Tirukkōṭikā- Tiruttāṇṭakam} & 7055 (6.81.8) & \textit{nāṉmaṟaiyiṉporuḷkaṇ\-ṭāynātaṇkaṇṭāy} (Behold  the Lord, the meaning of the four Veda-s). \tabularnewline
\hline
6.85 \textit{Tirumuṇṭīccaram- Tiruttāṇṭakam} & 7090 (6.85.5) & \textit{maṟaināṉkumāyiṉāṉ\-kāṇ} (Behold the one who became the four Veda-s). \tabularnewline
\hline
6.86 \textit{Tiruvalampoḻil- Tiruttāṇṭakam} & 7097 (6.86.3) & \textit{ōmkārameypporuḷai}\break  (I salute the meaning of the Oṃkāra). \tabularnewline
\hline
 & 7100 (6.86.6) & \textit{vētavittai}(I salute the one who is the seed of the Veda-s). \tabularnewline
\hline
6.87 \textit{Tiruccivapuram- Tiruttāṇṭakam} & 7104 (6.87.1) & \textit{vaṭamoḻiyumteṉtami\-ḻummaṟaikaḷnāṉku\-māṉavaṉkāṇ} (Śivabecame Sanskrit, Tamiḻ and the four Veda-s). \tabularnewline
\hline
 & 7111 (6.87.8) & \textit{kalayārunūlaṅkamāyi\-ṉāṉkāṇ} (Behold the one who became the Veda-s with arts and the six auxiliaries). \tabularnewline
\hline
6.89 \textit{Tiruviṉṉampar- Tiruttāṇṭakam} & 7121 (6.89.1) & \textit{collumarumaṟaikaḷ\-tāmēpōlum}(He is the personification of the Veda-s). \tabularnewline
\hline
 & 7126 (6.89.6) & \textit{vētapporuḷāyviḷaivār\-pōlum}(He is the meaning of the Veda-s). \tabularnewline
\hline
6.92 \textit{Tirukkaḻukuṉṟam- Tiruttāṇṭakam} & 7152 (6.92.2) & \textit{collōṭuporulaṉaitttu\-māṉāṉtaṉṉai…ālinkīḻi\-runtāṉai} (He became the words and the meanings … he was under the shade of the banyan tree). \tabularnewline
\hline
6.94 \textit{Niṉṟatiruttāṇṭakam} & 7165 (6.94.3) & \textit{collākicollukkōrporu\-ḷumāki} (He became the words and the meanings of the words). \tabularnewline
\hline
 & 7168 (6.94.6) & \textit{aṅkamāyātiyāyvēta\-māki} (He became the six auxiliaries, four Veda-s and the rare \textit{mantra}-s). \tabularnewline
\hline
 & 7170 (6.94.8) & \textit{nātaṉāyvētattiṉuḷḷōṉāki }( He is the lord of sound and resides in Veda-s). \tabularnewline
\hline
6.95 \textit{Taṉittruttāṇṭakam} & 7176 (6.95.4) & \textit{nalañcuṭarēnālvētatta\-ppālniṉṟa} (Śiva is the effulgent one and one who is beyond the four Veda-s). \tabularnewline
\hline
 & 7179 (6.95.7) & \textit{maṟaināṉkumāṉāyā\-ṟaṅkamāṉāy} (Śiva became the four Veda-s and six auxiliaries). \tabularnewline
\hline
6.96 \textit{Taṉittiruttāṇ\-ṭakam} & 7184 (6.96.2) & \textit{muppurinūlvaraimār\-piṉmuḻankakkoṇṭār} (He wears the three stranded sacred thread on his chest). \tabularnewline
\hline
 & 7187 (6.96.5) & \textit{arumaṟaiyaitērkuti\-raiākkikkoṇṭār} (He has the four Veda-s as the horses of his chariot). \tabularnewline
\hline
6.99 \textit{Tiruppukalūr- Tiruttāṇṭakam\-Tiruttāṇṭakam} & 7218 (6.99.4) & \textit{ātiyāyvētamāki} (Śiva became the primordial one and he became the Veda-s). \tabularnewline
\hline
 & 7220 (6.99.6) & \textit{viricaṭayāyvētiyanē\-vētakītā} (Śiva is the brahmin who has a wide matted hair and one who sings the Veda-s). \tabularnewline
\hline
\end{longtable}

The above cited Vedic references clearly mention the importance given by Appar to the Vedic aspect of Śaivism in his hymns. It can be clearly seen that almost every \textit{patikam} refers to Śiva as the follower of Vedic path. The image of Śiva that emerges from these \textit{patikam-s} is that Śiva is thought by the brahmins; he is the meaning of the Veda-s; he became the Veda-s with the six limbs; he wears the white sacred thread on his chest;Hara became the both \textit{mantra} and Tantra and he preached \textit{dharma} to the brahmins under the shade of the banyan tree; Śiva is of the nature of Veda-s; he likes the \textit{Sāma Veda} and music; he sings all the four Veda-s; he is worshipped by the four Veda-s; Śiva is worshipped by the sect of Brahmins consisting of 8000 who are popularly called as Aṣṭasahasram; he became the eighteen branches of knowledge: four Veda-s, six Vedāṅga-s, Purāṇa-s, Nyāya, Mīmāṃsā, Smṛti, Āyurveda, Dhanurveda, Gāndharvaveda and Arthaśāstra; Śiva is the speaker of the pure Vedic language; he wears the tiger skin and wears the serpent as the sacred thread and he is the \textit{mantra} and the sacrificial offering offered by the brahmins; he has the divine qualities mentioned in the four Veda-s which he has mastered; the brahmins offer the right amount of sacrificial offering to Śiva; Brahmā and Viṣṇu worship him through Vedic chants; Śiva transcends the four Veda-s and he is of the nature of \textit{Oṃkāra}.

From the above cited references, we get a clear picture that Śiva as addressed by Appar is a Vedic deity. When speaking of the Vedic nature of Śaivism one cannot forget the importance of the \textit{Śatarudrīya} hymn.It is undisputable that the \textit{Śatarudrīya} hymn in the \textit{Yajur Veda} occupies a very important place in the development of Śaivism. Śiva worship is never complete without the chanting of the \textit{Śatarudrīya}. \textit{Śatarudrīya} is chanted when \textit{abhiṣeka }is performed over the Ś\textit{ivaliṅga}. Scholars who study the hymns of Appar point out that the \textit{namaccivāyapatikam}, which is recorded in the Fourth \textit{Tirumuṟai} is the essence of the \textit{Śatarudrīy}a. It may be mentioned that this \textit{namaccivāyapatikam }was sung by Appar when he was tied to a stone and cast to die in the sea. From this it is clear that even in the most dangerous situations to his life, Appar envisioned Śiva as a Vedic deity. G. V. Narayana Iyer writes regarding \textit{Śatarudrīya,} that the fact that the Gods themselves are described as wearing the sacred thread of \textit{yajña (yajñopavīta}) indicates the importance that sacrificial rituals were given by that society.

\begin{myquote}
“From the Śatarudrīya hymn, we can get a glimpse of the country in the period of Yajur Veda. Verse 17 addresses Rudra as the wearer of the “sacrificial cord” (Upavītine) which shows that the formalities of the sacrificial rituals had assumed importance. Sacrifices are usually performed by mortals for gaining their desires. The \textit{mantras} uttered during sacrifices are addressed to deities who grant the desires of the sacrifice. The cord is to be woven by the sacrificer, but when Rudra himself was the god, whose favour was sought, what necessity could there be for the god to wear the cord? He was not a sacrificer, but only a deity to whom the sacrifice had to be offered. How then is the epithet \textit{Upavītine} to be explained? The only answer that seems possible is that since, in the eyes of the worshipper, the man most worthy of the respect was the person who performed the largest number of sacrifices, he gave the form of the sacrificer to Rudra also.………….. This is reflected in the verses of the hymn where the king’s attributes are applied to Rudra. Rudra is called “the leader of the armies” “the lord of regions” “the inoxious charioteer.” (Narayana Iyer 1974: 5,6)
\end{myquote}

Similarly, R. Nagaswamy in his book \textit{Śivabhakti} while mentioning the Vedic Nature of Appar’s Śaivism mentions the following:

\begin{myquote}
“Such references to Siva as the embodiment of \textit{Vedas}, \textit{Vedānta} and \textit{Vedavelvi} and that the Vedic chants were used in Śiva worship, clearly indicate that Appar followed the Vaidika Śaiva system, also known as \textit{Siddhāntamārga}.
\end{myquote}

\begin{myquote}
Mention has been made of the fact that by the time of Appar (7th Century), Śiva was adored as the embodiment of the four \textit{Vedas} and the six \textit{ańgas}. It is also said that the Vedas praise Śiva’s real nature. Naturally Appar’s Tevārams reflect the ideas enshrined in the Vedas. Among the four Vedas the ‘\textit{Satarudrīya}’ also called \textit{Rudram}, a part of the \textit{Krishnayajurveda} is held to be the hymn par excellence, on Siva. “The \textit{Satarudrīya} is in eleven \textit{anuvākas}. In the eighth \textit{anuvāka}, occurs the words \textit{Namassivāya ca Sivatarāya ca}. This is considered the most significant and almost the one mystic line in the whole of \textit{Yajurveda}…In fact most of the concepts connected with Saivite faith can be traced to the \textit{Satarudrīya}. Among the 63 Saivite Saints, Rudrapasupati Nāyanār, as his name itself indicates, was the one who recited constantly the ‘Rudra’(another name for ‘\textit{Satarudrīya}’ and attained liberation.
\end{myquote}

\begin{myquote}
Some of Appar’s references could undoubtedly be traced to \textit{Rudra}. For example the \textit{Satarudrīya} refers to Śiva as a child with limbs still undeveloped, ‘\textit{apagalbhāya}’ interpreted ‘\textit{aprarūdhendryabālah}’. This is a rare expression. Appar describes Śiva, exactly with the same epithet ‘\textit{bālanāyvaḷarntilāpānmaiyāne}’. Similarly, another interesting epithet, applied to Śiva, in the ‘\textit{Rudra’} is that he is present in the swift current of streams \textit{Sībyāya ca} (p.67). Appar echoes the same concept when he sings \textit{Sulāvāhi Sulāvukkōr Sūḻalāhi}. SimilarlyAppar’s reference to Śiva as “The lake full of water- \textit{Eri niraintanaya Selvan} is clearly the term \textit{Sarasyāya ca} of the \textit{Satarudrīya}.” (Nagaswamy 1989: 70-71).
\end{myquote}

However, as we look through the origin of Śaivism, scholars point out diversified development of Śaivism over the centuries. The Rudra of the Brāhmaṇa-s emerges as the Śiva of the Upaniṣad-s. There are numerous schools of Śaivism and not all of them agree with the Veda-s and some schools of Śaivism are non-Vedic. In this connection\break K. C. Pandey writes that Vedic philosophy reached a happy blend with Saivism practised in most parts of the country.

\begin{myquote}
“Whatever may have been the Brahmanic antagonism towards Śaivism in the early Vedic period, as some hold on the basis of reference to its followers as “Phallus worshippers”, etc., this antagonism died out with the passage of time; and Brahmanism and Śaivism became more and more reconciled, as testified by the inclusion of the hundred names of Śiva in the Śukla and the Kṛṣṇa Yajurveda, numerous references to him in the Atharvaveda and change in the conception of the god from “terrific”under the name “Rudra” to “the protector of the cattle” under the name “Paśupati”. Towards the end of the Vedic period, in the tenth book of the Taittirīya Āraṇyaka, we find the five Mantras, on which the Lakulīśa Pāśupata system is based…a careful study of the works on the various Śaiva systems shows that the attitude of the Śaiva Philosophy as a whole towards the Veda was not that of condemnation…nor that of opposition. It was rather like that of a step-daughter, whose agreements and differences with the father are those which the mother has with him.
\end{myquote}

\begin{myquote}
Thus Śaivism owes its allegiance to, acknowledges the authority of the Veda only in so as the Veda agrees with the Śaivāgamas, some of which assert that the Śaivāgama is the essence of the Veda (Vedasāraḥ Śivāgamaḥ). It may, however, be noted here that some systems of the Śaiva Philosophy agree with the Veda more than the others.” (Pandey 1986: 5-6)
\end{myquote}

\vskip 1pt

Such a dualistic approach of Śaivism towards the Veda-s is explicitly evident in the case of Pāśupata which is classified into Vaidika Pāśupata and Tāntrika Pāśupata (Nagaswamy 2006: 24). In fact, while commenting on the \textit{Brahmasūtra Patyurasamañjasyāt} (2.2.7.37), Ādi Śaṅkarācārya mentions that there are many branches of Śaivism and some of them are against the Veda-s.

\vskip 1pt

Given this, the development of Śaivism in South India as a Vedic religion owes its credit to the Nāyanmār-s especially Tiruñānacampantar and Appar. In this connection it is worth mentioning that before starting the hagiography of Campantar, Cēkkiḻār mentions that the saint was born to establish the \textit{Vaidika dharma }and also for the supremacy of Śaivism (\textit{vētaneṟitaḻaittōṅkamikucaivattuṟaiviḷaṅka}) (Mudaliyar 1975: 1899).While Campatar’s inclination towards Vedic \textit{dharma} is not surprising owing to the fact that he was a brahmin, Appar’s contribution to the development of Śaivism as a Vedic religion draws scholarly attention as Appar was an agriculturist (\textit{veḷāḷa}) by birth. And there is no doubt that the popularity and strength of Śaivism as seen even today is mostly due to its representation as a Vedic religion.

\vskip 1pt

The study of development of Śaivism as a Vedic religion not only shows the extent of pervasion of Vedic religion to all the branches of Hindu society at the times of Nāyaṉmār-s, but also throws light on re-visiting and studying some of the important problems in Indian philosophy such the date of Ādi Śaṇkarācārya. It is because scholars who have dealt with the problem of the date of Ādi Śaṇkarācārya have not approached the problem by placing it in the background of development of religion, especially, Śaivism in South India. In the view of Śaivism being developed as a Vedic religion by Appar, it is worth contemplating an alternative date to Ādi Śaṇkarācārya, a period in which the Vedic religion was facing serious problems instead of being accepted by all sects of the society.


\section*{Bibliography}

\begin{thebibliography}{99}
\bibitem{chap3-key01} Civa. Tiruciṟṟampalam (2006) \textit{Periyapurāṇam of Cēkkiḻār}. Kankai Puttaka Nilayam. Chennai.

 \bibitem{chap3-key01} Cuvāmikal, Caṇmukatēcikañānacampantaparamācārya(Ed.)(1997) \textit{Tirunāvukkaracucuvamikaḷarulicceytatēvārattiruppatikaṅkaḷ}. Mayilāṭutuṟai, TarumaiĀtīṉam. (accessed online through Tamil Virtual University, \url{http://www.tamilvu.org/library/libindex.htm}, on November 15, 2017).

 \bibitem{chap3-key02} Gambhirānanda(Tr.) (1983) \textit{Brahmasūtrabhāṣyaof ŚrīŚaṅkarācārya}. Advaita Ashrama. Calcutta.

 \bibitem{chap3-key03} Mudaliyar, C.K. Cuppiramaniya (Ed.) (1975). \textit{Periyapurāṇam of Cekkilār. }Tamil Virtual University. \url{ http://www.tamilvu.org/library/libindex.htm} accessed on January 21, 2019.

 \bibitem{chap3-key04} Nagaswamy, R.(1989) \textit{Śiva Bhakti}. Navrang. New Delhi.

 \bibitem{chap3-key05} Nagaswamy, R. (2006) \textit{Art and Religion of the Bhairavas: Illumined By Two Rare Sanskrit Texts, Sarvasiddhāntaviveka and Jñānasiddhi}. Tamil Arts Academy. Chennai.

 \bibitem{chap3-key06} Narayana Ayyar, G.V. (1974) \textit{Origin and Early History of Saivism in South India.} University of Madras. Chennai.

 \bibitem{chap3-key07} Pandey, Kanti Chandra (1986) \textit{An Outline of History of Śaiva Philosophy}. Motilal Banarsidass. Delhi. pp. 5-6.

 \bibitem{chap3-key08} \textit{Periya Purāņam}. See Mudaliyar.

 \bibitem{chap3-key09} Ramachandran, T.N. (Tr.) (1995) \textit{Tirumurai The Sixth}. DharmapuramAadheenam. Dharmapuram.

 \bibitem{chap3-key10} Subrahmanyam, S.V. (Ed.) (2013) \textit{PaṉṉiruTirumuṟai}. ManivacakarPatippakam. Chennai.

 \bibitem{chap3-key11} Vaittiyanathan, Ku. (1995) \textit{Caivacamayavaralāṟumpaṉṉirutirumuṟaivaralāṟum.}Thiruvavaduthuraiadina caivacittantanermukappayircimaiyam.Thiruvidaimarudur. chapter 2.

 \bibitem{chap3-key12} Vanmikanathan, G. (1985) \textit{Periya Purāņam by Sekkizhaar}. Sri Ramakrishna Math. Chennai.

 \end{thebibliography}

