
\chapter{Mangala-sūtra and Wedding Attire in the Sangam Age}\label{chap02}

\Authorline{K. Chitra Rao}


\section*{Abstract}

There are dichotomous views as to whether there was the practice of tying mangala-sūtra (tāli\index{tali@tāli}) for women or not in the Sangam period. Tamil / Dravidian scholars hold that the custom of mangala-sūtra\index{mangala-sutra@mangala-sūtra} (tāli) existed in Tamil Nadu since ancient times, whereas others say that it was not there. Those who moot the Aryan-Dravidian divide hold that the custom came in only after the so-called Aryan influences and the description of eight forms of marriage as mentioned in the Dharmaśāstra-s. To understand this, we have to investigate what kind of relationships existed between men and women as recorded in the ancient Tamil literature. 

Tolkāppiyam speaks of two types of relationships known as Kalaviyal\index{Kalaviyal} and Karpiyal\index{Karpiyal}. The first was clandestine, while Karpiyal was the love-relationship known to the society. Since clandestine wooing and illicit relationships led to many occasions where the man claimed that he did not know who the woman was and this resulted in the woman being deceived and jilted, certain customs were introduced, keeping Karpu as a high virtue. 

\newpage

Karpu\index{Karpu} was solemnized by the parents of the young couple and the wedding celebrated with festivities that included many groups of people. This type of marriage was held in high esteem in society, with alliances formalized between families of similar social and economic background. Sometimes, it was also the practice for young people to woo and fall in love and subsequently have their marriage solemnized by parents and elders in society, respecting their sentiments and relationship.

In general, whatever was relevant to antanar\index{Antanar}-s, arasar\index{arasar}-s and vanigar\index{vanigar}-s were applicable to other classes in society. Sometimes, the man in a relationship, whether tiring of the companion or hesitant to take on the responsibility, tried to abandon his partner after sharing physical intimacy. The ayyar-s (elderly persons/ priests) in society brought in measures to control this, to safeguard the interests of the concerned women. 

Marriage festivities, ceremonies, bridal ornaments, attire and other arrangements are described vividly in Akanānūru\index{Akananuru@\textit{Akanānūru}}, Silappadikāram\index{Silappadikaram@\textit{Silappadikāram}}, Perungathai and other texts. Did married women dress differently to unmarried or widowed women? Was their standing in society different according to their marital status? Is the tāli\index{tali@\textit{tāli}} an external identity of the married state? Many texts describe the various ornaments along with the tāli worn by married ladies, especially as part of the wedding ceremonies in Pattupāṭṭu\index{Pattupattu@\textit{Pattupāṭṭu}}, Tolkāppiyam\index{Tolkappiyam@\textit{Tolkāppiyam}}, Tirukkural\index{Tirukkural}, Aingurunuru\index{Aingurunuru} etc.

A debate whether Tāli existed in Sangam\index{Sangam} period was raised in 19th and 20th centuries CE. Based on the Aryan-Dravidian race theory, Tamil Nationalists\index{Tamil Nationalists} had maintained that such practices did not exist. However, Tamil scholars point out that there existed ornaments such as īkaiyariya iḻai and that these ornaments were removed when their husbands died and they observed the rite of widowhood called kaimmai nōṉpu. In this research paper this problem is analyzed on the basis of literary and archaeological evidences. It is proved that tāli / Mangala-sūtra existed and the practice of tying the tāli during marriage also existed.

\newpage

\begin{center}
\textbf{\tamil{“சங்க” இலக்கியத்தில் மங்கள அணி அல்லது தாலி இருந்ததா, இல்லையா – சரித்திர நோக்கில் ஒரு நுண்ணிய ஆய்வு}}\\\tamil{கே. சித்ரா ராவ்,}\\\tamil{ஆய்வாளர், பாரதிய இதிகாச சங்கலன சமிதி, சென்னை.}
\end{center}

\tamil{\textbf{முன்னுரை:}} \tamil{ராபர்ட் கால்டுவெல்லின் “மொழியியல் திராவிடம்”, இனரீதியல் திரிபுவாதத்திற்கு உட்பட்டப் பிறகு, தமிழர், “திராவிடர்,” என்றே வழங்கலாயிற்று. “தனித்தமிழ் இயக்கம்” திராவிடர்களை, ஆரியர்களிடமிருந்து வேறுபடுத்திக் காட்டும் முயற்சிகளும் நடந்தன. மறைமலை அடிகள் இவ்விசயத்தில், தீவிரமாக செயல்பட்டு, “தமிழர் திருமணாமுறை” தனித்தது போன்ற கருத்துருவாக்கத்தை உண்டாக்கினார்}\endnote{1. In 1939, Adigal was invited to preside at a conference devoted to examining and establishing a consensus on Tamil marriage rites and rituals entitled Tamilar Tirumana Manadu(Tami1ian Marriage Conference). Adigal had published a few years earlier his English pamphlet entitled The Tamilian and Aman Forms of Marriage.}. \tamil{ஆனால், தேவநேயன் போன்றோர், தாலி கட்டும் முறையெல்லாம் தமிழருடையதே என்று எடுத்துக் காட்டினர்}\endnote{2. \tamil{ஞா. தேவநேயன், \textbf{தமிழர் திருமணம்}, திருவள்ளுவர் தவச்சாலை, திருச்சி, திருச்சிராப்பள்ளி}, 1956, 82-83}. \tamil{சங்க இலக்கியத்தில் தாலி கட்டும் பழக்கம் இருந்தது பற்றி, இருந்தது மற்றும் இல்லை என்று இரு மாறுபட்ட கருத்துகள், இதனால் உண்டாயின. ஆனால், இவ்விரு கருத்துகளும் சங்க இலக்கியத்தில் உள்ள சான்றுகளை வைத்து, அதற்கான பிற்காலத்தில் எழுதப்பட்ட உரை மற்றும் விளக்கங்களை வைத்துக் கொண்டு தான் வாதங்கள் இருந்து வந்துள்ளன. தாலி கட்டும் பழக்கம் இல்லை என்பவர், “ஆரிய-திராவிட” இனவாத சர்ச்சையை வைத்து, முதலில் திராவிடர்களிடம் அத்தகைய பழக்கம் இல்லை, ஆனால், பிறகு ஆரியர்கள் வந்தேறி, எட்டுவகையான திருமணம், சடங்குகள், சம்பிரதாயங்கள் முதலியவற்றை திணித்தபோது ஏற்பட்டது என்று வாதிடுவர்}\endnote{3. \tamil{\enginline{“}திருமணச் சடங்குகளில், தலைமையானதாகத் தாலியணிவதும், மஞ்சள் தோய்த்த நாணைக் கழுத்திற் கட்டுவதும் போன்ற சில அனாரிய வழக்கங்கள், தெற்கத்துப் பிராமணப் பெண்டிர்க்கு அருமை யாக உள்ளன"}, P. T. Sriniuvasa Iyengar, \textbf{History of the Tamils}, AEA reprint, p.57.

\tamil{\enginline{“}இது (தாலியணிவது) கிருகிய சூத்திரங்களிலேயே சொல்லப்படாத ஒரு தூய தமிழ வழக்கம். கையைப் பிடிக்கும் பாணிக் கிரகணத்தையும் ஏழடியிடும் சப்தபதியையுமே, திருமணச் சடங்கின் உயிர்நாடிப் பகுதிகளாகக் கிருகிய சூத்திரங்கள் கொள்கின்றன" மேற்படி, அடிக்குறிப்பு.}}. \tamil{அடுத்த வாதமும், “ஆரிய-திராவிட” இனவாத கோட்பாட்டை ஒப்புக் கொண்டாலும், தமிழரிடையே தாலி கட்டும் பழக்கம் இருந்தது என்று இலக்கிய சான்றுகள் மூலம் மெய்ப்பிப்பர்}\endnote{4. \tamil{ ஞா.தேவநேயன், \textbf{தமிழர் திருமணம்}, மேலே குறிப்பிட்டது.}}. \tamil{இந்நிலையில், இப்பிரச்சினை அறிய, பழந்தமிழர் ஆண்-பெண் உறவுகள் எப்படியிருந்தன என்பதனை கூர்மையாக ஆராய வேண்டியுள்ளது.}

\tamil{\textbf{களவியல் மற்றும் கற்பியல்:} கிடைத்துள்ள மிகத்தொன்மையான தமிழ் இலக்கண நூல் என்று கருதப் படக் கூடிய தொல்காப்பியம், களவியல் மற்றும் கற்பியல் என்ற இருவகை ஆண்-பெண் உறவுமுறைகளை எடுத்துக் காட்டுகிறது\supskpt{\endnote{5. \tamil{தொல்காப்பியத்தின் காலம் பாரம்பரிய தமிழ் பண்டிதர்கள்} c.1000 BCE \tamil{வரையில் என்று கொண்டனர். சங்ககாலத்திற்கு} [c.500/300 BCE to 100 CE] \tamil{முன்பு என்றும் கணித்தனர். ஆனால், இப்பொழுது, இடைசெருகல்களை ஆய்ந்து அதன் காலத்தை} 7-9 \tamil{ம் நூற்றாண்டு} CE \tamil{வரை வைக்க முயல்கின்றனர்.}}}. களவியல் என்பது ஆண்-பெண் விருப்பப்பட்டு, ஆனால், மற்றவர்களுக்குத் தெரியாமல் உறவு வைத்துக் கொள்ளும் முறை. கற்பியல் என்பது ஆண்-பெண் விருப்பப்பட்டு, ஆனால், மற்றவர்களுக்குத் தெரிந்து உறவு வைத்துக் கொள்ளும் முறை. முந்தைய முறையில் ஆண் ஏமாற்றிய நிலை உண்டானதால், “பொய்யும் வழுவும் தோன்றிய பின்னர் ஐயர் யாத்தனர் கரணம்” என்றபடி, கரணம், அதாவது சடங்குகள் அறிமுகப்படுத்தப்பட்டன. தமிழரிடையே கற்பு உயர்வான நிலையில் வைக்கப் பட்டு, அது சிறந்த குணமாக, மேன்மை தன்மையாக மற்றும் சீரிய பெண்மையாகக் கொள்ளப்பட்டது. ஆனால், தேவநேயன் போன்றோர், “பொது மனைவியர்” இருந்தனர் என்ற கருத்தையும் பதிவு செய்துள்ளனர்\supskpt{\endnote{6. \tamil{முதற்காலத்தில், மக்கள் குடும்பப்பிரிவின்றிக் குலங்குலமாய் அல்லது தொகுதிதொகுதியாய் வாழ்ந்துவந்த நிலையில், ஒரு குலத்துப் பெண்டிர்,பெரும்பாலும் குலத்தலைவன் மனைவியர் நீங்கலாக, அக் குலத்து ஆடவர் அனைவர்க்கும் பொது மனைவியராகவே இருந்து வந்தனர். பின்பு நாகரிகந் தோன்றித் தனிமனைவியர் ஏற்பட்டபின், ஒருவன் தன் மனைவியை அல்லது மனைவியரை வேறாகப் பிரித்து வைத்தற்கு, தாலிகட்டும் வழக்கம் ஏற்பட்டது. ஞா.தேவநேயன், \textbf{தமிழர் திருமணம்}, திருவள்ளுவர் தவச்சாலை, திருச்சி,} 1997, \tamil{பக்கம்.}14}}.}

\tamil{அகத்திணை என்ற முறையில், கைக்கிளை, ஐந்திணை மற்றும் பெருந்திணை என்ற பிரிவுகள் எடுத்துக் காட்டப்படுகின்றன. கைக்கிளை அல்லது ஒருதலை காமம் என்ற முறையில் மூன்று பிரிவுகள் உள்ளன. ஐந்திணை என்பது, குறிஞ்சி, முல்லை, மருதம், நெய்தல் மற்றும் பாலை என்ற ஐந்து நிலப்பிரிவுகளுக்கு உரியதாகக் கொள்ளப்பட்டன. பெருந்திணை என்பது சரிசமம் இல்லாத, ஒவ்வாத, சேர்க்கைக்குரியதல்லாத முறைகள் கொண்டதாகவும், அவற்றால் தீய விளைவுகள் உண்டாகும் என்றும் விளக்கப்பட்டது. பெருந்திணையில், மடலேறுதல் (பனையோலையால் பின்னப்பட்ட குதிரையின் மீது உட்கார்ந்து சவாரி செய்து வருத்திக் கொள்வது) மற்றும் வரைப்பாய்தல் (உயிரை மாய்த்துக் கொள்வது) போன்ற தீவிரமான வகைகளும் விவரிக்கப் பட்டன.}

\tamil{கற்பியல் என்ற ஆண்-பெண் இணையும் முறை பெற்றோர்களால் நிச்சயிக்கப் பட்டு, ஏற்றுக்கொள்ளப்பட்டு, சடங்குகளுடன் செய்யப்படும் இணைப்பு – திருமண முறையாகும். களவியலில் ஏற்பட்ட தகாத உறவுகளைத் தடுக்க, தீமைகளை மாற்றி, ஆண்-பெண் இணைப்பை முறைப்படுத்த, கற்பியல் ஏற்படுத்தப் பட்டது என்பது தெரிகிறது. சடங்குகளுடன் கூடிய மணம் சமூகப் பிணைப்பு மற்றும் சேர்ந்து வாழும் முறையை மேன்மைப் படுத்தியாக அறியப்படுகிறது. “பொய்யும் வழுவும் தோன்றிய பின்னர் ஐயர் யாத்தனர் கரணம்” என்ற தொல்காப்பிய சூத்திரத்திலிருந்து இது உறுதியாகின்றது.}

\textbf{\tamil{கற்பியல், கற்பு, கரணம், கற்பு மணம் என்பன}}\endnote{7. \tamil{தொல்காப்பியம்: கற்பியல்.முதல் நான்கு சூத்திரங்கள். }

\tamil{‘கொண்டானிற் சிறந்த தெய்வம் இன்றெனவும், அவனை இன்னவாறே வழிபடுகவெனவும் இருமுது குரவர் கற்பித்தலானும் ‘அந்தணர் திறத்துஞ் சான்றோர்தே எத்தும்’, ‘ஐயர் பாங்கினும் அமரர்ச்சுட்டியும் ஒழுகும் ஒழுக்கம் தலைமகன் கற்பித்தலானும் கற்பாயிற்று’ என்றும் ‘கற்றல்’ என்ற பொருளிலேயே கற்பு என்ற சொல் நச்சினார்க்கினியர் உரையில் பயன்படுத்தப்பட்டுள்ளது.
 }}: \tamil{தொல்காப்பியத்தில் இவை தெளிவாக விளக்கப்பட்டுள்ளன.}

\tamil{\textbf{‘கற்பெனப் படுவது கரணமொடு புணரக்}\\ கொளற்குரிய மரபிற் கிழவன் கிழத்தியைக்\\ கொடைக்குரி மரபினோர் கொடுப்பக்கொள்வதுமே' }

\tamil{என்று தொல்காப்பியர் விளக்கியுள்ளார். பெற்றோர் தமது பெண்ணைக் கொடுக்ககூடிய மரபுள்ள பெற்றோரின் மகனுக்கு, அவனுக்கு ஏற்றுக் கொள்ளும் தகுதியுடன், விவரங்கள் அறிந்து, கரணம் முடிய சடங்குகளுடன் திருமணம் செய்து கொடுப்பதே கற்பு எனப்படும் என்றாகிறது.}

\textbf{\tamil{கொடுப்போர் இன்றியும் கரணம் உண்டே\\ புணர்ந்து போகிய காலை யான.}}

\tamil{ஒருவேளை பெற்றோர் அவ்வாறு முறையாக விசாரித்து, கொடுக்க முடியாத நேரத்திலும்-சந்தர்ப்பத்திலும், பெண் ஒரு ஆணோடு சென்றுவிட்டால், அறிந்த பின்னர் முறைப்படி சடங்குகளுடன் திருமணம் செய்து கொடுக்கும் வழக்கமும் இருந்தது.}

\textbf{\tamil{மேலோர் மூவர்க்கும் புணர்ந்த கரணம்\\ கீழோர்க்கு ஆகிய காலமுன் உண்டே.}}

\tamil{அந்தணர், அரசர், வணிகர் முதலியோருக்கு இருந்த சடங்குமுறைகள், அடுத்தவருக்கும் உண்டு என்ற நிலை ஏற்படுத்தப் பட்ட காலமும் உண்டு. அதாவது சடங்கு முறைகள் எல்லோருக்கும் இருந்தன, ஆனால், மேலோர்-கீழோர் என்ற பாகுபாடு ஏற்பட்டபோது, அவர்கள் மறுத்ததால், பிறகு, அவர்களுக்கும் ஏற்புடையதாக மாற்றப்பட்டது.}

\textbf{\tamil{பொய்யும் வழுவும் தோன்றிய பின்னர் \\ ஐயர் யாத்தனர் கரணம் என்ப,}}

\tamil{ஆண்-பெண் காதலால் ஒன்றாக பழகி விட்டு, ஏதோ காரணங்களினால், அவளை எனக்குத் தெரியாது என்று ஆண் மகன் பொய் கூறிய காலத்தில், பழகி உடலுறவு கொண்டு, வழுவாகிய நிலையில், அவளை யானறியேன் என்று விலகும் காலத்திலும், அத்தகைய மீறல்களைக் கட்டுப்படுத்த, பந்தங்களை ஒழுங்கப் படுத்த, ஐயர் - சமூகத்தின் மேலோர், சான்றோர், பெரியோர் கரணம் என்ற சடங்கு முறைகளை அறிமுகப் படுத்தி சீரமைத்தனர். இனி கரணம், கரணத்தின் சடங்குகள் முதலியவற்றில் என்ன இருந்தன, எப்படி நடத்தப் பட்டன, முதலியவை மற்ற சங்கப்பாடல்கள்ல் வரும் விவரங்களிலிருந்து அறியப்படுகின்றன.}

\tamil{\textbf{சங்க இலக்கியம் விவரிக்கும் வதுவை நன்மணம் விவரங்கள்:} திருமணம் அல்லது கல்யாணம் என்ற சொல், கடி, வதுவை, மன்றல் மற்றும் வரை என்ற வார்த்தைப் பிரயோகங்களில் சங்க இலக்கியத்தில் காணப்படுகின்றது\supskpt{\endnote{8. \tamil{கடி \enginline{–} அகநானூறு.} 136; \tamil{வதுவை \enginline{–} அகநானூறு.}166; \tamil{வதுவை நன்மணம்-} 86.17; \tamil{கடிமகள்- அகநானூறு.}244.5; \tamil{மன்றல்- தொல்காப்பியம் \enginline{–} களவியல்.}120.}}. அகநானூறு பாடல்கள்} 86 \tamil{மற்றும்}136 \tamil{கீழ்கண்ட விவரங்களைக் கொடுக்கின்றன. திருமணம் செய்து கொண்ட வாழ்க்கை நிலையை கற்பு என்று அகத்திணையில் சொல்லப்படுகிறது. மணப்பென் மற்றும் மணமகன் பெற்றோர்கள் திருமணத்திற்கு ஒப்புக்கொள்கின்றனர். ரோஹினி நட்சத்திரம் சந்திரனும் கூடிய நன்னாளில், நல்லநேரத்தில் திருமணம் நடத்தப்படும். நேரம் விடியற்காலையாக இருக்கும். வீட்டிற்கு முன்னர், வெண்ணிற மணல் பரப்பப்பட்டு, பந்தல் அமைக்கப்படும். அப்பந்தல் பூமாலைகளால் அலங்கரிக்கப்பட்டிருக்கும். மேளங்கள் முழங்க, விளக்குகள் ஏற்ற்ப்பட்டு, தெய்வங்கள் வணங்கப்படும். மங்கள மகளிரால் நீராட்டப்பட்டு, அலங்கரிக்கப்பட்டு, மணப்பெண் அழைத்து வரப்படுவாள். திருமணத்திற்கு முன்பாக, சிலம்புகழி நோன்பும் கடைப்பிடிக்கப்படும். ஆனால், மணமான பெண்களும் சிலம்பு அணிந்திருந்தது கண்ணகி, கோப்பெருந்தேவி போன்ற உதாரணங்களிலிருந்து அறியலாம். நான்கு வயதான, புதல்வர்களைப் பெற்ற மங்கள மகளிரால் தங்கள் தலைமீது, புனிதநீர் கொண்டுவரப்படும். மணப்பெண்ணின் தலையில் நெல், பூக்கள் முதலியவை சொரியப்பட்டு, நீராட்டப்படுவாள். இந்த நீராட்டு வைபபம் “வதுவை” என்றழைக்கப்படும். அச்சமயம், “கற்பிற்சிறந்த வாழ்க்கைத் துணையாகி, நின் கணவனை விரும்பும் பெண்ணாக” என்று மங்கல வார்த்தை சொல்லி சொரிவர்\supskpt{\endnote{9. \tamil{உ.வே.சாமிநாத ஐயர், \textbf{அகநானூறு \enginline{–} களிற்றியானைநிரை \enginline{–} மூலமும்}, உரையும், மகாமகோபாத்தியாய டாக்டர் உ.வே.சாமிநாதையர் நூல் நிலையம்,} 1990, \tamil{சென்னை, பக்கம்.}264.}}. பெற்றோர்களும் வாழ்த்துவர், வந்தவர்களுக்கு விருந்துணவு அளிக்கப்படும். பிறகு, மணமக்கள் தனியான அறைக்கு அனுப்பப்படுவர். பெண் முகம் மூடிய நிலையில், இருக்கும் போது, மணமகன், முகத்திரையை விலக்க, அவள் பெருமூச்சு விடுவாள். அவ்வுடையை “முருங்காக் கலிங்கம்” அதாவது, வளையாத புதிய ஆடை என்றும் விவரிக்கப்பட்டுள்ளது\supskpt{\endnote{10. \tamil{அக்காலத்திலேயே பசை போட்டு, அத்தகைய புத்தாடை சுருங்காமல் நீண்டிருக்க வழிமுறைகள் இருந்தது போலும்.}}}. மேலும், இரவு கழிந்த பின்னர், அவ்வாடை சுருங்கி விடும், களைந்து விடும் என்பதும் தொக்கிச் சொல்லப்பட்டது.}

\tamil{\textbf{சங்ககாலத்திற்குப் பிறகு விவரிக்கப்பட்ட திருமணம்:} சிலப்பதிகாரத்தில் கோவலன் கண்ணகி திருமணம் விவரிக்கப்பட்டுள்ளது. வணிககுலத்தைச் சேர்ந்தவர்களது திருமணம் சிறப்பாக கோவலனுக்கு வயது பதினாறு மற்றும் கண்ணகிக்கு பன்னிரெண்டு என்ற போது விமரசியாக பெற்றோர் நிச்சயத்தபடி நன்னாளில் நடத்தப்பட்டது. அணியிழை அணிந்த மங்கல மங்களிர் (அணியிழையார்) யானையின் மீது உட்காரவைத்து வரவழைக்கப்பட்டனர். மங்கள அணியும் யானைமீது ஊர்வலமாக எடுத்துவரப்பட்டது\supskpt{\endnote{11. \tamil{மங்கள அணி \enginline{–} சிலப்பதிகாரம் \enginline{–} மதுரை காண்டம்-} 21:46 \tamil{மற்றும்} 4:20.}}. ஊர்வலம் வாத்தியங்கள் ஒலிக்க, சங்கநாதம் முழங்க, கொடைகள் சகிதம் அரச ராஜிய ஊர்வலம் போல நடத்தப்பட்டது. மண்டபம் பூமாலைகள், பட்டுத் துணிகள், முத்துகள் என்று அலங்காரம் செய்யப்பட்டிருந்தது. சந்திரன் ரோகிணியுடன் கூடிய நன்னாளில், சுபமுகூர்த்தத்தில், கண்ணகி அருந்ததியைப் போன்று சிறப்பாக இருந்தாள். வயது முதிர்ந்த அந்தணன் வேதநூல் முறைப்படி திருமணச் சடங்குகளை [மாமுது பர்ப்பான் மறைவழி காட்டிடத் தீவலம் செய்வது] செய்வித்தான். மணமக்கள் தீவலம் வந்தனர். நான்காம் காதையில், “மங்கள அணி” தவிர வேறெந்த அணிகலனையும் கண்ணகி விரும்பவில்லை என்றுள்ளது\supskpt{\endnote{12. \tamil{மறுவில் மங்கல வணியே அன்றியும் பிறிதணி அணியப் பெற்றதை எவன்கொல்!} 2. \tamil{மனையறம் படுத்த காதை} - 64-65}}. “மங்கல வணியிற் பிறிதணி மகிழாள்” என்று பிறகும் சொல்லப்பட்டுள்ளது\supskpt{\endnote{13. \tamil{ அந்திமாலைச் சிறாப்புச்செய் காதை.}50}}. கண்ணகி} 21 \tamil{ம் காதையில், “விளங்கிழையாள்” அதாவது, விலையுயர்ந்த, மிகவுயர்ந்த, சிறப்பான இழையணிந்தவள் என்று வர்\break ணிக்கப்படுகிறாள்\supskpt{\endnote{14. \tamil{விளங்கிழையாள் - சிலப்பதிகாரம் \enginline{–} மதுரை காண்டம்-} 21:46.}}. அதாவது, திருமணத்திற்கு முன்பாக, மங்கள அணி யானைமீது ஊர்வலமாக எடுத்துவரப்பட்டது, பிறகு, “மங்கள அணி” தவிர வேறெந்த அணிகலனையும் கண்ணகி விரும்பவில்லை என்றாகி, அதற்கும் பிறகு, “விளங்கிழையாள்” என்றதால், அது அவளது கழுத்தில் அணியப்பட்டது என்றாகிறது. ஆக, அந்த விளங்கிழை, மங்கள அணியாக இருந்த பட்சத்தில், அதை தாலி என்று கொள்ளலாம். மேலும், வேதநூல் முறைப்படி திருமணச் சடங்குகள் நடந்து, தீயை வலம் வந்தனர் என்ற பட்சத்தில், அது அவளது கழுத்தில் அணிவிக்கப்பட்டது என்றாகிறது. அது கட்டாமல், அவளது கழுத்தில் ஏறியிருக்காது. அதை அவளாகவே அணிந்து கொண்டால் என்றால், அதனை மங்கள அணி அல்லது வியங்கிழை என்றும் சொல்லவேண்டிய அவசியம் இல்லை. பெருங்கதையிலும், அந்தணன் தீவளர்க்க, மணமகன், மணமகளின் கைப்பற்றி வலம் வந்தான் என்றுள்ளது\supskpt{\endnote{15. \tamil{பெருங்கதை} \enginline{–} 2.3; 9-14; 108-119.}}. ஆகையால், இந்த குறிப்புகளிலிருந்து, சங்ககாலத்திற்குப் பிறகு தாலி கட்டும் பழக்கம் இருந்தது என்று தெரிகிறது.}

\tamil{\textbf{திருமணமாகாதவர் மற்றும் திருமணமானவர் என்று இருபிரிவினர்களுக்கும் தனித்தனியான அணிகலன்கள் இருந்தனவா?:} பழங்கால-சங்ககால பெண்டிர் சங்கு, கல், தீயினால் சுட்ட மண், கண்ணாடி, வெள்ளி, தங்கம் முதலியவற்றால் தயாரிக்கப் பட்ட அணிகலன்கள், கழுத்தணிகள், மாலைகள் முதலியவற்றை அணிந்தனர் என்று தெரிகிறது. இதைத்தவிர (கழுத்தணிகள்) வலையல்கள், கம்மல், மூக்குத்தி, நெற்றிச்சுட்டி, காலணி, இடுப்பணி என்று பல அணிகலன்களை அணிந்திருந்தனர். “புலிப்பல்தாலி” என்பது கழுத்தைச் சுற்றி, மார்பை தொடும் அளவிற்கு இருந்தது என்றுள்ளது\supskpt{\endnote{16. P. T. Srinivasa Iyengar, \textbf{History of Tamils}, Madras, 1929, p.225.}}. வாலிழை, ஆயிழை, அணியிழை, ஒளியிழை, மணியிழை, இலங்கிழை, சேயிழை, பாசிழை, விரலிழை, தெரியிழை, நேரிழை, திருந்திழை, புனையிழை, மின்னிழை, வீங்கிழை, புலையிழை, அவிரிழை, வயங்கிழை, சுடரிழை, நுணங்கிழை என்று பற்பல அணிகலன்களை அணிந்துள்ளதாக விவரிக்கப்படுகின்றது. இங்கு “இழை” என்ற சொல் மற்ற உரிச்சொற்களுடன் சேர்த்து, சொற்றோடர்களாக உபயோகப்படுத்தப் பட்டுள்ளன. சிலம்பைப் பொறுத்த வரையில், “சிலம்பு கழி நோன்பு” என்ற சடங்குள்ளதாகத் தெரிகிறது. குழந்தை பருவத்தில் அணிவிக்கப்பட்டு, வயது வந்ததும் அல்லது திருமணத்தின் போது, அது அவிழ்க்கப்படுகிறது. ஆனால், கண்ணகி, பெருங்கோதை போன்ற திருமணமான பெண்களும் சிலம்பு அணிந்துள்ளதாக இருப்பதனால், திருமணமாகாதவர் மற்றும் திருமணமானவர் என்று இருபிரிவினர்களுக்கும் தனித்தனியான சிலம்புகள் இருந்துள்ளனவா என்று கவனிக்க வேண்டும். அதே போல “இழை” என்ற சொல்லின் முக்கியத்துவத்தையும் அலசவேண்டியுள்ளது.}

\tamil{\textbf{இழை பற்றிய அலசல்:} இழை என்ற சொல் சங்க இலக்கியங்களில் எங்கெல்லாம் மற்ற சொற்களுடன் சேர்ந்து, சொற்றோடராக, ஒரு குறிப்பிட்ட அணிகலனைக் குறிப்பத்ஹால், அவை, கீழ்கண்டவாறு அட்டவணையில் குறிக்கப் படுகின்றன:}

\begin{longtable}{|l|m{2.5cm}|m{2.5cm}|m{2.5cm}|}
\hline
\tamil{எண்} & \tamil{இழை-பிரயோகம்} & \tamil{பொருள் / விளக்கம்} & \tamil{குறிப்பு} \\
\hline
1 & \tamil{வாலிழை} & \tamil{இளமையான, தூய்மையான, வெண்மையான அணிகலன்} & \tamil{பதிற்றுப்பத்து.} 5:15, \tamil{குறுந்.} 386.3;. 45:2; \tamil{கலித்.} 119:14. \\
\hline
2 & \tamil{அணியிழை} & \tamil{பல இழை/அடுக்குக் கொண்ட, அழகான, பெருமையான } & \tamil{ஐங்.} 359:3 \\
\hline
3 & \tamil{ஆயிழை} & \tamil{முக்கியமான, தேர்ந்தெடுத்த அணிகலன்; பெண்} & \tamil{நற்றிணை.}75.8; \tamil{புறம்.} 34. \\
\hline
4 & \tamil{ஒள்ளிழை} & \tamil{பிரகாசமான, நல்ல, சிறப்பான} & \tamil{கலி.} 122;16,17 \\
\hline
5 & \tamil{மணியிழை} & \tamil{உயர்ந்த, மேன்மையான, சிறப்பான} & \tamil{புறம்.}78:8-12 \\
\hline
6 & \tamil{இளங்கிழை} & \tamil{இளமையான, நீண்ட} &  \\
\hline
7 & \tamil{சேயிழை} & \tamil{அணிகலன்களை அணிந்த பெண் } & \tamil{ஏழாம் பத்து} 65.9-10 \\
\hline
8 & \tamil{பாசிழை} & \tamil{பச்சைநிற அணிகலனை அணிந்தவள்} & \tamil{பதிற்றுப்பத்து.}  73:5,10. \tamil{மதுரைக்காஞ்சி.}580 \\
\hline
9 & \tamil{விரலிழை} & \tamil{மிகயுயர்ந்த ஆபரணம்} & \tamil{ஐங்கு.}235.3;  \\
\hline
10 & \tamil{தெரியிழை} & \tamil{அணிகலன்களை அணிந்த பெண்} & \tamil{கலி.}14.53; 49.1 \\
\hline
11 & \tamil{நேரிழை} & \tamil{அணிகலன்களை அணிந்த பெண்} & \tamil{நற்றிணை} 40.3-4 \\
\hline
12 & \tamil{திருந்திழை} & \tamil{மிகயுயர்ந்த, மேன்மையான} & \tamil{கலி.}131:1-2. \\
\hline
13 & \tamil{புனையிழை} & \tamil{அலங்கரிக்கப் பட்ட ஆபரணம்; அணிகலன்களை அணிந்த பெண்} & \tamil{குறுந்தொகை.}21:2 \\
\hline
14 & \tamil{மின்னிழை} & \tamil{மின்னுகின்றன, ஒளிர்கின்ற அணிகலன்} & \tamil{ஐங்.}455.3; \tamil{பரி.வையை.}135 \\
\hline
15 & \tamil{வீங்கிழை} & \tamil{அடர்ந்த, அடர்த்தியான, } & \tamil{கலி.}139.11; \tamil{அகம்.}251.3. \\
\hline
16 & \tamil{புலையிழை} & \tamil{மெல்லிய} & \tamil{ஐங்.}382.5 \\
\hline
17 & \tamil{அவிரிழை} & \tamil{ஒளிர்கின்ற} & \tamil{மதுரை காஞ்சி.}443-446. \\
\hline
18 & \tamil{வயங்கிழை} & \tamil{ஒளிர்கின்ற அணிகலன்; சிலம்பு} & \tamil{இரண்டாம் பத்து.} 12.22-24  \\
\hline
19 & \tamil{சுடரிழை} & \tamil{சுடர் போன்ற ஒளிர்கின்ற அணிகலன்; மின்னலை  உமிழ்ந்தாற் போன்ற ஒளிதிகழும் அணிகளையுடைய}\supskpt{\endnote{17. \tamil{ சுடரிழையினையு முடைய ஆயமகளிர் எனச் சிறப்பித்தவர், அரச மாதேவி அத்தகைய பூ வொன்றும் அணிந்து கொள்ளாது பிரிவுத் துயருற்றிருந்தமை தோன்ற, ஒன்றும் கூறாராயினார். மகி்ழந்தென்பதற்கு, \enginline{“}விரும்பிச் சூடியென்றவா\enginline{”} றென்றும், \enginline{“}மின்னுமிழ்ந்தன்ன சுடரிழை யென்றது, மேகம் மின்களை உமிழ்ந்தாற் போன்ற சுடர்களையுடைய இழையென்றவா\enginline{”} றென்றும் பழையவுரைகாரர் கூறுவர்.}}} & \tamil{கலி.}138.10 \\
\hline
20 & \tamil{நுணங்கிழை} & \tamil{மெல்லிய, நுண்ணிய, மிக்க வேலைப்பாடு கொண்ட; நுண்ணிய வேலைப்பாடமைந்த இழை. } & \tamil{ஐங்குறுநூறு :} 258.3; \tamil{பரிபாடல்:} 228.38. \\
\hline
\end{longtable}

\tamil{\textbf{இழை என்பது என்ன – எதைக் குறிக்கும்?:} “இழை” என்ற சொல் மற்றும் அதன் பொருள் என்ன என்று மேலே பட்டியல் இடபட்டது. இனி அவற்றின் பொருள் என்ன என்பதனை, உபயோகப் படுத்தப் பட்டுள்ள பாடல், வரி முதலியவற்றிலுள்ள “இடம், பொருள், ஏவல்” ரீதியில் ஆராய்ந்து அர்த்தம் கொள்ளப்படுகிறது.}

\tamil{\textbf{நூலாக் கலிங்கம் வால் அரைக் கொளீஇ;\\ வணர் இருங் கதுப்பின், வாங்கு அமை மென் தோள்,\\ வசை இல் மகளிர் வயங்குஇழை அணிய;} இரண்டாம் பத்து.} 12.22-24

\tamil{குற்றமற்ற பாண்மகளிர் எல்லோரும் ஒளி விளங்கும் அணிகலங்களையும் அணிந்திருப்பார் ஆயினர்.}

\tamil{ஒள் இழை மகளிரொடு மள்ளர் மேன; இரண்டாம் பத்து.} 13.21

\tamil{\textbf{வான் உறை மகளிர், நலன், இகல் கொள்ளும்;\\ வயங்கு இழை கரந்த, வண்டு படு கதுப்பின்;\\ ஒடுங்கு ஈர் ஓதிக் கொடுங்குழை கணவ!} இரண்டாம் பத்து.} 14. 13-15

\tamil{ஒளி பொறுந்திய அணிகலங்களை }

\tamil{\textbf{நாடுடன் விளங்கும் நாடா நல்லிசைத்\\ திருந்திய இயல் மொழித் திருந்திழை கணவ!} மூன்றாம் பத்து.} 24.11

\tamil{\textbf{காமர் கடவுளும் ஆளும் கற்பின்,\\ சேண் நாறு நறு நுதல், சேயிழை கணவ!} ஏழாம் பத்து} 65.9-10

\tamil{\textbf{வால் இழை கழித்த நறும் பல் பெண்டிர்} பதிற்றுப்பத்து.} 5:15,

\tamil{தங்களது கணவரை இழந்தபோது, வாலிழையை (இளமையான, தூய்மையான, வெண்மையான இழையை) அகற்றினர் என்றுள்ளது.}

\tamil{\textbf{பூ வாட் கோவலர் பூவுடன் உதிரக்\\ கொய்து கட்டுஅழித்த வேங்கையின்\\ மெல்லியல் மகளிரும் இழை களைந்தனரே}} (\tamil{புறம்}.224: 15-17). 

\tamil{பூக்கள் உதிரும் வண்ணம், மென்மையான இயல்புடைய மகளிர் இழை களைந்தனர்.}

\tamil{\textbf{ஈகை அரிய இழையணி மகளிரொடு}} (\tamil{புறம்.}127:6) = \tamil{பிறருக்கு அளிக்க இயலாத/ கொடுக்க முடியாத இழையணி - மங்கல அணிகலன்கள் மட்டுமே அணிந்த மகளிர். அதாவது அத்தகைய இழையணி, தாலி போன்றது தான் என்று தெரிகிறது.}

\tamil{\textbf{கொய்ம்மழித் தலையொடு கைம்மையுறக் கலங்கிய\\ கழிகலம் மகடூஉப் போல\\ புல்என் றனையால், பல்அணி இழந்தே.}} (261. 17-20)

\tamil{விளக்கம்: கணவன் [காரியாதி] இறந்ததால், மனைவி கைம்மைக் கோலத்துடன் மயிர் மழித்த மொட்டைத்தலையோடு, அணிகலன்கள் அனைத்தையும் களைந்துவிட்ட கோலத்தில் புல்லென்று அற்பமானது போல, காணப்பட்டாள். இங்கு “கைம்மை” மற்றும் “கழிகலமகடூஉ” நிச்சயமாக விதவைத் தன்மையினையும், தாலி போன்ற அணிகலனை அகற்றிய நிலையையும் எடுத்துக் காட்டுகிறது.}

\tamil{பழங்கன்று கறித்த பயம்பமல் அறுகைத்\\ தழங்குகுரல் வானின் தலைப்பெயற்கு ஈன்ற\\ மண்ணுமணி அன்ன மாஇதழ்ப் பாவைத்\\ தண்நறு முகையொடு வெந்நூல் சூட்டித்,\\ தூஉடைப் பொலிந்து மேவரத் துவன்றி,} 15\\\tamil{மழைபட் டன்ன மணன்மலி பந்தர்,\\......................................மணன்மலி பந்தர்,\\ இழைஅணி சிறப்பின் பெயர்வியர்ப்பு ஆற்றித்\\ தமர்நமக்கு ஈத்த தலைநாள் இரவின்,......................... (அகநானூறு. 136: 11-18).}

\tamil{திருமணம் ஆன இரவன்று பணப்பெண், புத்தாடையுடுத்தி,...இரவில் தலைவினிடம் செல்கிறாள். இது முன்னமே விளக்கப்பட்டது.}

\tamil{ பெரும்பாண் காவல் பூண்டென, ஒரு சார்,\\ திருந்துஇழை மகளிர் விரிச்சி நிற்ப,} (\tamil{நற்றிணை} 40.3-4)

\tamil{வால் இழைக் குறுமகள்} (\tamil{நற்றிணை} 76.5) \tamil{தூய கலன்களை யணிந்த இளமடந்தை}

\tamil{வேய் மருள் பணைத் தோள் விறல் இழை நெகிழவும்,....} (\tamil{நற்றிணை} 85.2-3)

\tamil{மூங்கிலை யொத்த பருத்த தோளிலணிந்த ஏனைய கலன்களை வெற்றி கொள்ளும் வளை நெகிழ்ந்து விழவும்......}

\tamil{காவலுக்கு நிற்கும் மகளிர் திருந்திய அணிகளை அணிந்திருந்தனர். மகனை ஈன்றேடுத்த தாயைக் காக்கும் பெண்டிர் திருந்திய கலனணிந்திருந்தனர்.}

\tamil{ஒள் இழை மாதர் மகளிரோடு அமைந்து,} [\tamil{கலித்தொகை} 122.16]

\tamil{பெரும் கடல் தெய்வம் நீர் நோக்கித் தெளித்து, என் \\ திருந்திழை மென் தோள் மணந்தவன் செய்த} [\tamil{கலித்தொகை} 131.1-2] 

\tamil{செயிர்தீர் கற்பின் சேயிழை கணவ!} [\tamil{புறநானூறு} 3.6] 

\tamil{மாண்இழை மகளிர் கருச்சிதைத் தோர்க்கும்,} [\tamil{புறநானூறு} 34.2]\\...........................\\\tamil{அறம் பாடின்றே ஆயிழை கணவ!} [\tamil{புறநானூறு} 34.6]

\tamil{புல்லென் கண்ணர்; புறத்திற் பெயர,\\ ஈண்டுஅவர் அடுதலும் ஒல்லான்; ஆண்டுஅவர்\\ மாண்இழை மகளிர் நாணினர் கழியத் \\ தந்தை தம்மூர் ஆங்கண்,\\ தெண்கிணை கறங்கச்சென்று, ஆண்டு அட்டனனே.} [\tamil{புறநானூறு} 78.8-12]

\tamil{\textbf{“அருவி தாழ்ந்த பெருவரை போல\\ ஆரமொடு பொலிந்த மார்பின் தண்டாக்,\\ கடவுள் சான்ற, கற்பின் சேயிழை\\ மடவோள் பயந்த மணிமருள் அவ்வாய்க்\\ கிண்கிணிப் புதல்வர் பொலிக!”} என்று ஏத்தி} (\tamil{புறநானூறு.}198: 1-5)

\tamil{கற்பின் சேயிழை மடவோள் என்பது கற்புடன் கொண்ட சிறந்த அணிகலனை அணிந்த பெண் என்பதால், அவ்வணி முக்கியமானது, மங்களகரமானது, “ஈகையரிய இழை” பொன்றது என்றாகிறது.}

\tamil{நாள்மகிழ் இருக்கை காண்மார் பூணொடு\\ தெள்ளரிப் பொற்சிலம் பொலிப்ப வொள்ளழல்\\ தாவற விளங்கிய வாய்பொன் னவிரிழை\\ அணங்குவீழ் வன்ன பூந்தொடி மகளிர். மதுரை காஞ்சி.}443-446.

\tamil{திருமணம், இழை மற்றும் திருக்குறள்: திருக்குறள் சங்கம் மருவிய காலம் என்று கருதப்பட்டு, இப்பொழுது} 7-9 \tamil{நூற்றாண்டுகளை சேர்ந்தது என்றெல்லாம் சில ஆராய்ச்சியாளர்கள் எடுத்துக் காட்டுகின்றனர். திருக்குறள் ஒருவனுக்கு-ஒருத்தி என்ற சமுதாய ஏற்புடமை மற்றும் கற்பின் மேன்மை முதலியவற்றைப் போற்றுகிறது. காமத்துப் பால் என்ற இறுதிப் பிரிவில், தொல்காப்பியத்தின் களவு-கற்பு நிலைகளை எற்றுக்கொண்டதும் புலப்படுகிறது. வள்ளுவர் எல்லாவற்றையும், ஒரே இடத்தில் குறிப்பிடவில்லை என்றாலும், அவற்றை மற்ற இடங்களில் உட்பொதிந்த வகையில் வெளிப்படுத்தியுள்ளார். அதேபோல, ஒரே கருத்தை, திரும்ப-திரும்ப சொல்வது போல காணப்பட்டாலும், அவர் அதையே திரும்பச் சொல்லவில்லை, ஆனால், வேறொரு பொருள் பொதிந்த கருத்தை உணர்த்துகிறார். இல்வாழ்க்கை, பிறனில் விழையாமை, பெண்வழிசேரல், வரைவின் மகளிர் போன்ற அத்தியாயங்களில் களவு மற்றும் கற்பு நிலைகளை விளக்கியுள்ளார். மணமான பெண்ணை “வரையில்”, ஆணை “மணந்தர்”, மற்றும் மணநாளை “மணந்த நாள்” என்று குறிப்பிடுகின்றார். இழை என்ற சொல்லை கீழ்கண்டவாறு பிரயோகித்துள்ளார்.}

\begin{longtable}{|l|m{2.5cm}|m{5cm}|}
\hline
\tamil{எண்} & \tamil{சொற்றொடர்} & \tamil{பொருள் / விளக்கம்} \\
\hline
1 & \tamil{மணியிழை} & \tamil{மிகமேன்மைப் பொறுந்திய, மங்களகரமான அணிகலனை அணிந்த பெண்} \\
\hline
2 & \tamil{அணியிழை} & \tamil{அணியைணணிகலனை, நகையை அணிந்த பெண்} \\
\hline
3 & \tamil{சேயிழை} & \tamil{உயர்ந்த, சிறந்த, போற்றக்க்கூடிய அணிகலனை அணிந்த பெண்} \\
\hline
4 & \tamil{ஆயிழை} & \tamil{ஒலிப்பொறுந்திய, பிரகாசமான அணிகலனை அணிந்த பெண்} \\
\hline
5 & \tamil{ஒலியிழை} & \tamil{தேர்ந்தெடுத்த, முக்கியமான அணிகலனை அணிந்த பெண்} \\
\hline
\end{longtable}

\tamil{வள்ளுவர், “வரைவின் மகளிரையும்” அதாவது திருமண பந்தத்தில் வரமுடியாத பரத்தையரையும், “மணியிழையாள்” என்று குறிப்பிடுகிறார். அதாவது, அத்தகைய மகளிரும், அணிந்தனர் என்றால், பாதுகாப்புக் கருதி அல்லது சமூகத்தில் தானும், மணமானவள் என்று காட்டிக் கொள்ள அணிந்தாள் என்றாகிறது. இருவகை மகளிருக்கும் உள்ள வேறுபாட்டை, “நலம்புனைந்துரைத்தல்” என்ற அத்தியாயத்தில் எடுத்துக் காட்டுகிறார். இங்கும் “மணியிழை” அணிந்தவள் இருக்கிறாள், ஆனால், “கற்பியல்” கீழ் வரும் பெண்ணாக இருக்கிறாள். அம்மகளிரோ “வரைவின் மகளிர்” வகையறாவில் வருகிறாள். மணமகன், மணமகளை சேர்ந்த பின்னர், “சேயிழை” என்றும், அவளது மிகச்சிறந்த காதலை போற்றும்போது “ஆயிழை” என்றும், மணமகள் தனது சேர்ந்த பந்தத்தை தனது தோழியிடம் அறிவிக்கும் போது, “ஒலியிழை” என்றும் வள்ளுவர் குறிப்பிடுகின்றார். ஆகவே, வள்ளுவர் “இழை” என்பதை மணமான பெண்களுக்கு, மங்கள அணி போன்றே குறிப்பிட்டுள்ளார் என்றாகிறது.}

\tamil{\textbf{“ஐயர்” யார்- பிராமணரா, ஆரியரா?:} “பொய்யும் வழுவும் தோன்றிய பின்னர் ஐயர் யாத்தனர் கரணம்” என்ற தொல்காப்பிய சூத்திரத்தில் குறிப்பிடப் பட்ட “ஐயர்” யார் என்று அடையாளம் கொள்வதில் தான் தமிழ் பண்டிதர்கள், ஆய்வாளர்கள் 20ம் நூற்றாண்டில் அதிக அளவில் வேறுபட்டனர். “ஆரிய-திராவிட” கருதுகோள் மற்றும் கோட்பாடுகளின் தாக்கங்களில் மூழ்கியிருந்தபோது, அவற்றை உறுதிபடுத்துவதாகவே இச்சொற்பிரயோகம் இருந்ததாக கொண்டனர். ஆனால், மற்ற குழு தந்தை, தமையர், பெரியவர், உயர்ந்தவர் என்பதைக் குறிப்பதாகக் கொண்டனர். சில உதாரணங்கள், கீழே அட்டவணையில் கொடுக்கப் படுகிறது: }

\begin{longtable}{|m{3cm}|m{2.7cm}|m{2.7cm}|}
\hline
\tamil{புலவர் / ஆய்வாளர்} & \tamil{காலம் / குறிப்பிட்ட புத்தகம் முதலியன} & \tamil{ஐயர் என்ற சொல்லுக்கான அர்த்தம், விளக்கம்} \\
\hline
\tamil{நச்சினார்க்கு இனியர்} & 14 / 15 \tamil{ம் நூற்றாண்டு} & \tamil{இருடிகள், ரிஷிக்கள், அதனால் ஆரிய ரிஷிக்கள்} \\
\hline
\tamil{மு. ராகவ ஐயங்கார்} & Tamil Studies & \tamil{ஆரிய பிராமணர்} \\
\hline
\tamil{பி.எஸ்.சுப்பிரமணிய சாஸ்திரி} & An Enquiry into the Relationship of Sanskrit and Tamil, University of Travancore, 1946. & \tamil{ஆரிய பிராமணர் “தத்பவம்” ரீதியில் பிராமணர், சத்தியர், வைசியர் என்று குறித்தாலும், பிராமாரைக் குறிக்கிறது.} \\
\hline
\tamil{கே. ஏ.நீலகண்ட சாஸ்திரி} & A History of South India, Oxford University Press, 1996, p.130. & \tamil{ஆரியர்} \\
\hline
\tamil{மயிலை சீனி. வேங்கடசாமி} & \tamil{ஆய்வுக் களஞ்சியம் -} 4, \tamil{பண்டைத் தமிழகம் வணிகம் - நகரங்கள் மற்றும் பண்பாடு, பதிப்பு வீ. அரசு, இளங்கணி பதிப்பகம், சென்னை} - 600 017. & \tamil{சத்திரியன் (ஐய்யன் ஆரிதனார், ஐயடிகள் காடவர்கோன்\supskpt{\endnote{18. \tamil{ ஐயடிகள் என்பது ஐயனடிகள் என்பதன் மரூவாகும்.} Travancore Archaeological Series, Vol.II, p.61.}}, ஐயன் என்கின்ற தம்பி உதயன்\supskpt{\endnote{19. Mahavamsa. XXII, Vol.II, p.82; Epigraphica Zeylanica, Vol.VII.}})} \\
\hline
\tamil{கே. ஏ.நீலகண்ட சாஸ்திரி \supskpt{\endnote{20. K. A. Nilakanta Sastri has interpreted ‘Ayamani’ as “aryadeva’, Journal of Oriental Research, X-13, 96, ff.}}} & \tamil{இலங்கை கல்வெட்டுகள்} & \tamil{பௌத்த பிக்குகள் (மஹா அய/ஐய, திஸ்ஸா அய/ஐய)} \\
\hline
\tamil{கல்வெட்டுகள்} & \tamil{ஜுனார் மயிடவோலு மட்டபாட்} & \tamil{அய, ஐயா, அய்யர், ஐயர், அஜ, அஜம, அயஶ்ரீ முதலியன\supskpt{\endnote{21. Archaeological Survey of Western Indkia, Vol.IV, No.18, p.103.}}. } \\
\hline
\end{longtable}

\tamil{ஆக, “ஐயர்” என்பது பிராமணரை ஆரியரைக் குறிக்கும் சொல் என்று உறுதியாக சொல்லமுடியாது. மேலும், இடைக்காலத்து இலக்கியம் மற்றும் நிகண்டுகளை வைத்துப் பார்க்கும் போது\supskpt{\endnote{22. \tamil{திவாகரம், சூடாமணி, ஆசிரியர் (தொகுப்பு) முதலிய நிகண்டுகள். சங்கத்தமிழ் அகராதி, தமிழ் பல்கலைக்கழகம், தஞ்சாவூர்.}}}, பார்ப்பர், அறுதொழிலாளர், எரி வளர்ப்போர், வேள்வியர், மேற்குலத்தோர், வேதப்பாரகர், முத்தீமரபினர், தருப்பையர், ஆய்ந்தோர், உயந்தோர், அறவோர், அந்தணர், ஆதிவருணர், மறையோர், முற்குலத்தோர், முப்பிரிநூலோர். கமலத்தாரினர், முஞ்சியர், மெய்யர், பூசுரர், இரு பிறாப்பாளர், வேதியர், விப்பிரர், ஐயர், தொழுகுலர், வேதக்கொடியோர், சமித்தினர், முனிவர், சிகைவர் என்று பற்பல பெயர்கள், சொல்லாடல்கள் முதலியவை உபயோகப்படுத்தப் பட்டு வந்துள்ளன. இவையெல்லாமே “பிராமணரா” என்ற கேள்வியும் எழுகின்றது. ஏனெனில், சங்க இலக்கியத்தில் பிராமணர், திராவிடர் போன்ற சொற்கள் இல்லை. ஆனால், ஆரிய, ஆரியர் என்ற சொற்பிரயோகம் உண்டு. ஐயர் என்ற வார்த்தை கீழ்கண்ட இடங்களில் காணப்படுகின்றன:}

\begin{longtable}{|m{4.5cm}|m{4.5cm}|}
\hline
\tamil{தொல்காப்பியம்} (III-143-2, 144-29) & \tamil{பரிபாடல் திரட்டு} (2-63) \\
\hline
\tamil{திருமுருகாற்றுப்படை} (107) & \tamil{கலித்தொகை} (39-17, 21, 130-9) \\
\hline
\tamil{குறிஞ்சிப்பாட்டு} (17) & \tamil{அகநானூறு} (226-7, 259-16, 302-9) \\
\hline
\tamil{நற்றிணை} (122-1, 127-5) & \tamil{புறநானூறு} (337-16, 350-8) \\
\hline
\tamil{குறுந்தொகை} (272-4) & \tamil{சிலப்பதிகாரம்} (7-17-1, 7-19-1, 10-160, 12-14-1, 12-15-1, 12-10-1) \\
\hline
\tamil{ஐங்குறுநூறு} (365-1) & \tamil{தினைமாலை நூற்றைம்பது} (22-2) \\
\hline
\tamil{பதிற்றுப்பத்து} (70-19) & \tamil{களவியல்} (28-1) \\
\hline
\end{longtable}

\tamil{\textbf{தாலி - வெளிப்புற அடையாள சின்னமாகுமா?:} எட்டுத்தொகை-பத்துப்பாட்டு (சங்க இலக்கியம்), ஐம்பெருங்காப்பியங்கள், தொல்காப்பியம், திருக்குறள் முதலியவற்றை, கோர்வையாக படிக்கும் போது, மணம், திருமணத்துடனான சடங்குகள் மற்றும் சம்பந்தப்பட்ட சின்னங்கள்-அணிகலன்கள் முதலியவை, எவ்வாறு படிப்படியாக பரிமணித்தன, வளர்ந்தன, ஏற்புடையதாகின என்று அறிந்து கொள்ளலாம். எட்டுத்தொகை-பத்துப்பாட்டு நூல்கள் பல காலங்களில், பல புலவர்கள் பாடப்பட்டவற்றை தொகுத்ததால், அக்காலத்தை} 500 BCE – 100 CE, 500 BCE – 300 CE, 300 BCE – 300 CE, 300 BCE – 100 CE \tamil{என்று பலவாறு குறிக்கப்படுகின்றன. இது இந்திய சரித்திர காலத்துடன் ஒத்துப் போகிறது. “பொய்யும் வழுவும் தோன்றிய பின்னர் ஐயர் யாத்தனர் கரணம்” என்ற நிலையில், சடங்குகளுடன், ஈகையரிய இழை போன்ற அணிகள் அணிவிக்கப் பட்டது, அப்பெண் மணமானவள் என்பதைக் காட்டக் கூடிய புறச்சின்னனாக இருந்தது என்று கொள்ளலாம். சிலம்பு அணிவிக்கப்படும் சடங்கு இருந்தது எனும்போது, அது புறச்சின்னமாகவே இருந்தது. அதுபோல, தாலி என்ற பெயர் சங்க இலக்கியத்தில் குறிப்பிடப்படாமல் இருந்தாலும், அத்தகைய அணி அணிவிக்கப்பட்டது என்று சிலப்பதிகாரத்தில் குறிப்பிடப்படும் “மங்கல அணி” எடுத்துக் காட்டுகிறது. எனவே, மங்கள அணி என்பது அந்நிலையில்} 100 BCE – 100 CE \tamil{ புறச்சின்னமாக ஏற்றுக் கொள்ளப்பட்டது என்றாகிறது. “ஈகையரிய இழை”, கயிறு, தாலிக்கொடியாடி என்றிருக்கும் போது, அத்தகைய பந்தத்தை உண்டாக்க புறச்சின்னம் சங்க காலத்திலேயே} \enginline{c.500/300 BCE – 100 CE} \tamil{இருந்தது என்றாகிறது.}

\tamil{\textbf{அகழ்வாய்வு ஆதாரங்கள்:} தொல்லியல் அகழ்வாய்வுகளில் உலோகம், இயற்கை கற்கள் மற்றும் சுட்டமண் பொருட்கள் என்று பலவகைகளில் இழை, இழைகள் (கோர்க்கப் பட்டவை) கிடைத்துள்ளன. அவை, இழை என்ற வழக்கமான சொற்பிரயோகத்துடன் ஒத்துப் போகின்றன. இழை என்பதற்கு பல அர்த்தங்கள் வழங்கி வருகின்றன:}

\begin{enumerate}
\item \tamil{இழை – நூல், அணிகலன் (நேரிழை, முற்றிழை), இழைவாங்கி (ஊசி), மூன்றிழை, நாலிழை (நெசவு)}

 \item \tamil{இழை (வினைச்சொல்) – தேய், உடலுறவு கொள் (பாம்பு போல), சேர்ந்திரு, பிணை, மூச்சிறுகு, மனம் பொருந்து.}

 \item \tamil{இழை – பதி, சீவு, கல, உரை, முடை, சூழ் (கல்லிழை, நூலிழை, இழைப்புளி, இழைக்கூடு, தீங்கிழை)}

\end{enumerate}

\tamil{ஆக, இழை அல்ணிகலனாக, மனம் பொருந்த, ஆண்-பெண் சேரும் அடையாளமான, ஆனால், கரணங்கள் கூடிய நடத்தப்பட்ட கட்டாயமான சடங்குடன் ஏற்படுத்தப்பட்டது என்று தெரிகிறது. ஆகவே, தாலி, மாங்கல்யம், தாலிக்கயிறு பொட்டு என்று எவ்வாறு வழங்கப் பெற்றாலும், அது இருந்தது புலனாகிறது. இலங்கையில் சிகிரியா என்ற கற்கோட்டையில் உள்ள சுவர் சித்திரங்களில் மணமான பெண்கள் நெற்றியில் குங்கும், கழுத்தில் தாலி முதலியவற்றுடன் இருப்பதை எடுத்துக் காட்டப்பட்டுள்ளது}\endnote{23. In his paper on the Sigiriya Frescoes, M.D. Raghavan says: \textit{“It is remarkable that almost every figure has first a necklet, a string on which is strung three beads-One central bead escorted by one rather smaller bead on either side. This is the typical form of the tali, the marriage symbol of all Hindu women. The tali is single beaded, the shape of which varies according to the caste and the religion. The commonest form is the circular oval bead with two smaller beads, one on either side. Certain Hindu castes\index{caste} in Tinnevelly, use a flattened fonn with star shaped golden bead on either side, the whole necklet being tenned. chirakum taliyum or the wings and the tali. With some castes the tali takes a leaf pattern. The tali is worn on a cotton strand sometimes on a double strand, as we notice in some of the Sigiriya figures. In only two of the Sigiriya figures including the figure of the dark coloured maid, we do find this necklet. The tali is the unfailing symbol of a Sumangali, living with her husband. If the beads were meant to serve a purely ornamental purpose, a full string of beads would obviously have graced the neck, as seen in the Ajanta frescoes, and not first three beads or a single one or double. The presence of what looks unmistakably like a tali, would raise a number of interesting side issues, such as the survival of the tali, in Ceylon, a distinctly Hindu trait. The presence of the tali\index{tali@tāli} as a marriage symbol in Ceylon, would indeed seem to be strongly indicated by the custom in the wedding ceremonies of tile Sinhalese, which goes by the name· of tali pili andavina or the traditional custom of the bridegroom tying the tali on the necklace round the neck of the bride. followed by the exchange of pressure wedding apparel. The simple tali would thus seem to have evolved in course of time in to tile more showy necklace.}

\textit{The big gem set pendant in the necklace. and the gem set crown, are well matched by the solid broad bmcelets set with big stones, which embellish the wrist in all the figures Bracelets. As a decoration of the ladies symNolised the married status of the women and indicative of a summangali.”} M.D. Raghavan, \textbf{Srigiri Frescoes}, quoted by C. J. Jayadev, see below.}. \tamil{தமிழகத்தில் அத்தகைய ஆய்வுகளை மேற்கொள்ளவில்லை என்றே தோன்றுகிறது. ஏனெனில், சங்க காலத்தில் தாலி இருந்தது என்றால், திராவிட அரசு அல்லது திராவிட ஆட்சியாளர்கள் ஏற்றுக்கொள்ள மாட்டார்கள். அதனால் தான் போலும்}, 2007\tamil{ல் சென்னை அருங்காட்சியகம், “தாலி சம்பந்தப்பட்ட தென்னிந்திய சடங்குமுறைகள்” என்ற அறிக்கை வெளியிட்டபோது, தாலி போன்றவை கிடைத்தன என்றாலும், சங்ககாலத்தைப் பற்றிய விவரங்கள் விவரங்கள் கொடுக்கப்படவில்லை}\endnote{24. C. J. Jayadev, \textbf{\textit{The Tali in Relation to South Indian Initation Rites}}, Bulletin of Madras Government Museum, New Series – General Section – Vol.XIII, No.2, Madras, 2007.}. \tamil{இருப்பினும் இழை மற்ற இழை வகைகள் தாலியாக இருக்கக் கூடும் என்று எடுத்துக் காட்டப்பட்டுள்ளது}\endnote{25. K. V. Ramakrishna Rao, \textit{\textbf{Tali System in Sangam Age}, Proceedings of the Indian History Congress}, Vol. 52 (1991), p. 192.}.

\tamil{சப்தபதிக்கு பதிலாக தாலிகட்டுவதை ஏற்றுக்கொண்டது} (1967): \tamil{சுயமரியாதை திருமணங்கள் சட்டரீதியாக செல்லாத என்ற நிலை வந்தபோது,} 1967 \tamil{ல் முதலமைச்சர் அண்ணாதுரை முதல்வராக இருந்த காலத்தில் தமிழக அரசு இந்து திருமணங்கள் சட்டத்தில் திருத்தத்தை எடுத்து வந்தது. அது “இந்துத் திருமணங்கள் (சென்னைத் திருத்தம்) சட்டம்} 1967” \tamil{மசோதா ஆகும். இந்தச் சட்டம் இந்து திருமணச் சட்டத்தின்} 7\tamil{-வது பிரிவிற்கு பின்} 7\tamil{-ஏ என்ற புதிய பிரிவைப் புகுத்தியது}\endnote{26. Tamil Nadu Act No.21 of 1967 – The Hindu Marriage (Tamil Nadu Amendment) Act, 1967, dated 28.11.1967.

(1) After section 7, insert the following section, namely:—

(a) by each party to the marriage declaring in any language understood by the parties that each takes the other to be his wife or, as the case may be, her husband; or

(b) by each party to the marriage garlanding the other or putting a ring upon any finger of the other; or

(c) by the tying of the thali.

(2) (a) Notwithstanding anything contained in section 7, but subject to the other provisions of this Act, all marriages to which the section applies solemnised after the commencement of the Hindu Marriage (Madras Amendment) Act, 1967, shall be good and valid in law.

Notwithstanding anything contained in section 7 or in any text, rule or interpretation of Hindu law or any custom or usage as part of that law in force immediately before the commencement of the Hindu Marriage (Madras Amendment) Act, 1967, or in any other law in force immediately before such commencement or in any judgment, decree or order of any court, but subject to sub-section (3) all marriages to which this section applies solemnised at any time, before such commencement shall be deemed to have been, with effect on and from the date of the solemnisation of each such marriage, respectively, good and valid in law.

(3) Nothing contained in this section shall be deemed to—

(a) render valid any marriage referred to in clause (b) of sub-section (2), if before the commencement of the Hindu Marriage (Madras Amendment) Act, 1967,—

..........

(4) Any child of the parties to a marriage referred to in clause (b) of sub-section (2) born of such marriage shall be deemed to be their legitimate child:

Provided that in a case falling under sub-clause (i) or sub-clause (ii) of clause (a) of sub-section (3), such child was begotten before the date of the dissolution of the marriage or, as the case may be, before the date of the second of the marriages referred to in the said sub-clause (ii).” [Vide Tamil Nadu Act 21 of 1967, sec. 2 (w.e.f. 20-1-1968).]}. \tamil{இதன்படி சுயமரியாதை அல்லது சீர்த்திருத்த திருமணத்தில் புரோகிதர் மந்திரம் ஓதுதல், மணமக்கள் ஹோமத்தை ஏழு அடி எடுத்து, நடந்து தீயை வைத்து வலம் வரல் ஆகியவை நடைபெற வேண்டும் என்பதற்கு, பதிலாக, மணமக்கள் ஒருவரை ஒருவர் பார்த்து மற்றவர்களுக்குப் புரியும் மொழியில் கணவனாக அல்லது மனைவியாக மற்றவரை ஏற்றுக் கொள்வதாகச் சொல்ல வேண்டும். மணமக்கள் மாலை அல்லது மோதிரம் மாற்றிக் கொள்வதோ அல்லது தாலி கட்டும் சடங்கைச் செய்து கொள்வதோ மேற்கொள்ளலாம். அது சட்டரீதியாகியது. இது தமிழ்நாட்டிற்கு மட்டுமே உரியது. அதாவது, அப்படி தடாலடியாக செய்து வைத்த திருமணங்கள் எல்லாம் செல்லாது, ……….என்றெல்லாம் நீதிமன்றங்களில் தீர்ப்புகள் வந்தபோது அதிர்ந்து விட்டனர் பகுத்தறிவி ஜீவிகள்! அதாவது இந்து திருமண சட்டத்தில் தான்}[2] \tamil{அந்த “சுய மரியாதை” அடங்கிவிடுகிறது! அனால், இன்றும், இப்படி பொய்களை பேசியே வாழ்க்கையை நடத்துகின்றனர். இந்துமதத்தை ஆபாசமாக வர்ணித்த பகுத்தறிவு பகலவன் பாதையில் திருமணம் செய்து கொண்டவர்கள், திராவிடர்கள் “இந்துக்களாகி” தமது மானத்தைக் காப்பாற்றிக் கொண்டனர். இதில் முக்கியமான பகுதி என்னவென்றால் ‘சப்தபதி’ என்ற சடங்கிற்கு பதிலாக தாலிகட்டுவதை ஒப்புக்கொண்டது ஆகும். அதாவது எரிவளர்க்கும் தமிழர் பாரம்பரியத்தை மறைத்தாலும், மறுத்தாலும், தாலியை, கட்டுவது திராவிட அரசு, சட்டரீதியாக ஏற்றுக் கொண்டது.}

\tamil{\textbf{முடிவுரை:} மேற்கண்ட விவரங்கள், ஆதாரங்கள், விலக்கங்கள் முதலியவற்ரை வைத்து, கீழ்காணும் முடிவுகள் பெறப்படுகின்றன:}

\begin{enumerate}
\item \textbf{\tamil{சங்க காலத்தில் தாலி இருந்ததா, இல்லையா}?:} \tamil{சங்க இலக்கியத்தில் தாலி இருந்ததா, இல்லையா என்ற சர்ச்சை,} 19-20 \tamil{நூற்றாண்டுகளில் எழுப்பப்பட்டது. ஆரிய-திராவிட இனசித்தாந்த ரீதியில், அத்தகைய வழக்கம் இல்லை என்று பொதுவாக தமிழ் தேசியவாதிகள் மறுத்து வந்தனர். சமீபத்தில் தொ. பரமசிவன் என்பவர், கீழ்காணும் பிரச்சினைகளை எழுப்பியிருந்தார்}\endnote{27. \tamil{தொ. பரமசிவன், தாலியின் சரித்திரம், உண்மை, மார்ச்} 16-31, 2014;\hfill \break \enginline{http://www.unmaionline.com/index.php/2014-magazine/93-\break \%E0\%AE\%AE\%E0\%AE\%BE\%E0\%AE\%B0\%E0\%AF\%8D\%E0\%AE\%9A\break \%E0\%AF\%8D-16-31/1939-\%E0\%AE\%A4\%E0\%AE\%BE\%E0\%AE\%B2\break \%E0\%AE\%BF\%E0\%AE\%AF\%E0\%AE\%BF\%E0\%AE\%A9\%E0\%AF\%8D-\break \%E0\%AE\%9A\%E0\%AE\%B0\%E0\%AE\%BF\%E0\%AE\%A4\%E0\%AF\%8D\break \%E0\%AE\%A4\%E0\%AE\%BF\%E0\%AE\%B0\%E0\%AE\%AE\%E0\%AF\%8D-\break \%E0\%AE\%AA\%E0\%AF\%87\%E0\%AE\%B0\%E0\%AE\%BE\%E0\%AE\%9A\break \%E0\%AE\%BF\%E0\%AE\%B0\%E0\%AE\%BF\%E0\%AE\%AF\%E0\%AE\%B0\break \%E0\%AF\%8D-\%E0\%AE\%AE\%E0\%AF\%81\%E0\%AE\%A9\%E0\%AF\%88\break \%E0\%AE\%B5\%E0\%AE\%B0\%E0\%AF\%8D-\%E0\%AE\%A4\%E0\%AF\%8A-\break \%E0\%AE\%AA\%E0\%AE\%B0\%E0\%AE\%AE\%E0\%AE\%9A\%E0\%AE\%BF\break \%E0\%AE\%B5\%E0\%AE\%A9\%E0\%AF\%8D.html}}. \tamil{இருப்பினும். “ஈகையரிய இழை” போன்ற அணிகலன்கள் இருந்தன, அவை “கைம்மை” நோன்பு கடைபிடுக்கும் போது, களையப்பட்டன என்ற குறிப்புகளை தமிழ் பண்டிதர்கள் எடுத்துக் காட்டியுள்ளனர். அவற்றிற்கு பதிலாக, மேற் குறிப்பிடப்பட்ட விவரங்களிலிருந்து பதில் கொடுக்கப்படுகிறது. இனி எழுப்பப் பட்ட பிரச்சினைகளுக்கு, மேற்கொண்ட ஆராய்ச்சி மூலம் பதில் அளிக்க படுகிறது:}

 \item \tamil{\textbf{தாலி – என்ற சொல்லின் வேர்ச்சொல்லை இனங்காண முடியவில்லை:} “தாலி” என்ற பிரயோகம், காலத்தில் இருந்தது. ‘ஐம்படைத் தாலி’ என்ற பெயரில் சிறுவருக்குப் பெற்றோர் அணிவிக்கும் பயன்பாடு புறநானூறு} 77\tamil{ம் பாட்டின்} 7 \tamil{ஆம் வரியிலும் [புலிப் பற்றாலிப் புற்றலைச் சிறா அர்...], அகநானூறு} 54 \tamil{ஆம் பாட்டின்} 18 \tamil{ஆம் வரியிலும் [பொன்னுடைத் தாலி என்மகன் ஒற்றி], திணைமாலை நூற்றியைம்பதின்} 66 \tamil{ஆம் பாட்டில்} 3\tamil{வது வரியிலும், மணிமேகலையின் மூன்றாம் காதையில்} 138 \tamil{ஆம் வரியிலும், கலிங்கத்துப் பரணியின்} 240 \tamil{ஆம் பாட்டிலும் கூறப்பட்டிருக்கிறது. சிறுமிகளுக்கு சிலம்பு கழீ நோன்பு இருந்திருக்கிறது. நாளடைவில் அவை தாலி, மெட்டி என்று மாறியிருக்கலாம். இழை என்பதற்கான அர்த்தங்கள் கொடுக்கப்பட்டன. தாலி என்பது தொங்கும் அணிகலன் என்றால், இழையும், அது போன்றதே. தாலி எவ்வுருவில் இருந்தாலும், கழுத்தில் அணியப்பட்டன, தொங்கின என்பது தெரிகிறது.}

 \item \tamil{\textbf{நமக்கு கிடைக்கும் தொல்லிலக்கியச் சான்றுகளிலிருந்து (சங்க இலக்கியம், சிலப்பதிகாரம்) அக்காலத்தில் தாலி கட்டும் பழக்கம் இருந்ததில்லை என்றே தோன்றுகிறது:} சங்க இலக்கிய சான்றுகளிலிருந்து, இழை தான் தாலி என்பது பல இலக்கிய சான்றுகளுடன் மேலே எடுத்துக் காட்டப்பட்டது}\endnote{28. In his paper on the Sigiriya Frescoes, M.D. Raghavan says: \textit{“It is remarkable that almost every figure has first a necklet, a string on which is strung three beads-One central bead escorted by one rather smaller bead on either side. This is the typical form of the tali, the marriage symbol of all Hindu women. The tali is single beaded, the shape of which varies according to the caste and the religion. The commonest form is the circular oval bead with two smaller beads, one on either side. Certain Hindu castes in Tinnevelly, use a flattened fonn with star shaped golden bead on either side, the whole necklet being tenned. chirakum taliyum or the wings and the tali. With some castes the tali takes a leaf pattern. The tali is worn on a cotton strand sometimes on a double strand, as we notice in some of the Sigiriya figures. In only two of the Sigiriya figures including the figure of the dark coloured maid, we do find this necklet. The tali is the unfailing symbol of a Sumangali, living with her husband. If the beads were meant to serve a purely ornamental purpose, a full string of beads would obviously have graced the neck, as seen in the Ajanta frescoes, and not first three beads or a single one or double. The presence of what looks unmistakably like a tali, would raise a number of interesting side issues, such as the survival of the tali, in Ceylon, a distinctly Hindu trait. The presence of the tali\index{tali@tāli} as a marriage symbol in Ceylon, would indeed seem to be strongly indicated by the custom in the wedding ceremonies of tile Sinhalese, which goes by the name· of tali pili andavina or the traditional custom of the bridegroom tying the tali on the necklace round the neck of the bride. followed by the exchange of pressure wedding apparel. The simple tali would thus seem to have evolved in course of time in to tile more showy necklace.}

\textit{The big gem set pendant in the necklace. and the gem set crown, are well matched by the solid broad bmcelets set with big stones, which embellish the wrist in all the figures Bracelets. As a decoration of the ladies symNolised the married status of the women and indicative of a summangali.”} M.D. Raghavan, \textbf{\textit{Srigiri Frescoes, quoted}} by C. J. Jayadev, see below.}. \tamil{அவை சங்க காலத்திலிருந்தே} (c.500/300 BCE to 100 CE) \tamil{இருந்தது என்பதற்காக அகழ்வாய்வு ஆதாரங்களும் கொடுக்கப் பட்டன. ஆயிரக் கணக்கான மணிகள், பல அகழ்வாய்வுகளில் கிடைத்துள்ளன. அவையெல்லாம், இழையணியாக உபயோகப் படுத்தியருக்கலாம்.. இலங்கையில் சிகிரியா என்ற கற்கோட்டையில் உள்ள சுவர் சித்திரங்களில் மணமான பெண்கள் நெற்றியில் குங்கும், கழுத்தில் தாலி முதலியவற்றுடன் இருப்பதை மேலே எடுத்துக் காட்டப்பட்டது. தமிழகத்தில் இதுவரை கிடைக்கவில்லை. கீழடி போன்று ஆய்வுகள் மேற்கொள்ளும் போது கிடைக்கலாம்.}

 \item 
 \tamil{\textbf{சிலப்பதிகாரமும், மங்க அணியும், தாலியும்:} தமிழர் திருமணத்தில் தாலி உண்டா இல்லையா என்று தமிழறிஞர்களுக்கு மத்தியில் 1954-ல் ஒரு பெரிய விவாதமே நடந்தது. இதைத் தொடங்கி வைத்தவர் கண்ணதாசன். தாலி தமிழர்களின் தொல் அடையாளம்தான் என வாதிட்ட ஒரே ஒருவர் ம.பொ.சி மட்டுமே! மபொசி அப்பொழுதே, சிலப்பதிகார உதாரணங்களைக் குறிப்பிட்டு விளக்கினார். }

 \tamil{மறுஇல் மங்கல அணியே அன்றியும் \\ பிறிதுஅணி அணியப் பெற்றதை எவன்கொல்? }

 \tamil{சாலியொருமீன் தகையாளைக் கோவலன்\\ மாமுது பார்ப்பான் மறைவழிக் காட்டிடத்\\ தீவலம் செய்வது காண்பார்.....}

 \tamil{ஆனால், திராவிடத்துவவாதிகள் ஒப்புக்கொள்ளவில்லை. மாறாக சிலப்பதிகாரத்தின் காலத்தையும் பின்னோக்கி தள்ள முயற்சிகள் ஆரம்பிக்கப்பட்டன}\supskpt{\endnote{29. \tamil{கோவலனும் கண்னகியும் காவிரிப்பூம்பட்டினத்தை விட்டுப் புறப்பட்ட காலம் பற்றியும், மதுரை எரியுண்ட காலம் பற்றியும் வரும் சோதிடக் குறிப்புகள் கொண்டு நோக்கின்} 756 CE [c.8th cent.CE] \tamil{ஆகிய ஓராண்டே நன்கு பொருந்தும் என்பது} L. D. \tamil{சுவாமிக்கண்ணுப்பிள்ளை அவர்கள் கருத்து} (An Indian Ephemeris, VoL-1,pt-1, app.iii).}}.

 \item \textbf{10 \tamil{ஆம் நூற்றாண்டு} CE \tamil{வரை தமிழ்நாட்டில் தாலிப் பேச்சே கிடையாது:}} ‘10 \tamil{ஆம் நூற்றாண்டு} CE \tamil{வரை தமிழ்நாட்டில் தாலிப் பேச்சே கிடையாது’ – கே. அப்பாத்துரையார். தேவநேயன் தாலி கட்டும் வழக்கம் இருந்தது என்றது மேலே சுட்டிக் காட்டப்பட்டது. அவர்களது பாரபட்சம் வெளிப்படுகிறது. ‘பழந்தமிழர்களிடத்தில் தாலி வழக்கு இல்லவே இல்லை’ என்று மா. இராசமாணிக்கனார் வாதிட்டார். தமிழாய்வாளர்கள் முழுமையான ஆராய்ச்சி செய்வதில்லை என்பது தெரிகிறது, அதாவது, “ஆரிய-திராவிட” சித்தாந்தங்களில் கட்டுப்பட்டு கிடப்பதினால், அவர்கள், தொடர்ந்த என்ன ஆராய்ச்சிகள் நடக்கின்றன, அவற்றின் விளைவு, முடிவு, தாக்கம் என்ன என்பதனை ஆய்ந்து பார்ப்பதில்லை.} 7\tamil{-ம் நூற்றாண்டில்} CE \tamil{திருமண சடங்குகளை ஒவ்வொன்றாகப் பாடுகின்ற ஆண்டாளின் பாடல்களில் தாலி பேச்சே கிடையாது. ஆனால், சங்க இலக்கியத்தில் ஈகையறிய இழை...முதலியவை தெரியாதது வியப்பாக இருக்கிறது. }

 \item \tamil{\textbf{தாலிக்கான அகழ்வாய்வு ஆதாரங்கள்:} தமிழ்நாட்டில் பல்வேறு இடங்களில் தோண்டி எடுக்கப்பட்ட புதைபொருள்களில் இதுவரை தாலி எதுவும் கிடைக்கவில்லை என்ற பிரச்சினை ஒரு கருதுகோள் ரீதியில் எழுப்பப்பட்டுள்ளது.. அகழ்வாய்வு ஆதாரங்கள் - இழை, அவை பற்றிய நிலை எடுத்துக் காட்டப்பட்டன.} 10\tamil{ம் நூற்றாண்டிற்கு} CE \tamil{பிறகே தமிழகத்தில் பெண்ணின் கழுத்துத்தாலி புனிதப் பொருளாகக் கருதப்பட்டு வந்துள்ளதாக கொள்ளலாம் என்று வாதிடும் போது, சங்காலத்து, “ஈகையரிய இழை.......,” முதலியவை என்னவாயிற்று என்று ஓசிக்க வேண்டும். சிலப்பதிகார மகல அணி மறைந்து விட்டதா அல்லது தொடர்ந்ததா என்று அலசப்பட வேண்டும். ஆகவே, சிலப்பதிகாரம் (இலக்கியம்), சிகிரியா (சுவர் சித்திரங்கள்) மற்றும் இழை (அகழ்வாய்வு) முதலியவை இதை பொய்யாக்குகிறது. அரிக்கமேடு, கீழடி ஆய்வுகளின் ஆதாரங்கள், முடிவுகள் முதலியவற்றையும், எடுத்துக் கொண்டு ஆராய வேண்டும்.} 11\tamil{-ஆம் நூற்றாண்டில் கச்சியப்பரால் இயற்றப்பட்ட கந்தபுராணத்தில் தான் திருமணத்தின்போது தாலி கட்டப்பட்டதாகக் கூறப்பட்டுள்ளது. சங்க இலக்கியம்} (c.500/300 BCE), \tamil{சிகிரியா சுவர் சித்திரங்கள்} (6th-7th cent CE) \tamil{மற்றும் இழை} (\tamil{அகழ்வாய்வு,} c.500-300 BCE) \tamil{முதலியவை இதை பொய்யாக்குகிறது.}

 \item \textbf{\tamil{சுயமரியாதை திருமணம்} 1969 \tamil{லிருந்து தான், இந்து திருமண சட்டத்தில் இடம் பெற்று மரியாதை பெற்றது:}} \tamil{இந்திய சிந்தனையாளர்களில் பெரியார்தான் முதன்முதலில் தாலியை நிராகரித்துப் பேசவும், எழுதவும் துவங்கினார். அவரது தலைமையில் தாலி இல்லாத் திருமணங்கள் நடைபெறத் தொடங்கின, என்றெல்லாம் ஒரு பக்கம், திராவிட சித்தாந்திகள் வாதிக்கலாம். ஆனால், அவை சட்டவிரோதமாகி, கணவன் மனைவி பந்தங்கள், பிறந்த குழந்தைகள் சட்டப் படி நடக்கவில்லை என்று பிரச்சினையாகியது. நீதிமன்றங்களுக்குச் சென்றபோது தான், உண்மை அவர்களுக்கு விளங்கியது. தாலி கட்டினாலும்-கட்டா விட்டாலும், தீயைச் சுற் றினாலும்-சுற்றாவிட்டாலும், சில சடங்குகள் நடக்க வேண்டும் என்ற கட்டாயம் ஏற்பட்டது. ஆகவே, அத்தகைய விருப்பங்களுடன், தாலி கட்டும் திருமணம் தான் சட்டரீதியாக்க, சட்டத் திருத்தம் செய்யப் பட்டதும், பெரியார் அப்பொழுது உயிரோடு இருந்ததும் தெரிந்த விசயமே. அதாவது, இந்து திருமண சட்டத்தில் தம் தாலியில்லா திருமணத்தை நுழைக்காதே என்று சொல்லவில்லை! மேலே விளக்கப்பட்டுள்ளது. பின்னர்,} 1969\tamil{-ல் அண்ணா காலத்தில் நிறைவேற்றப்பட்ட சுயமரியாதைத் திருமணச் சட்டம் தாலி இல்லா திருமணத்தைச் சட்டபூர்வமாக அங்கீகரித்தது. அதிலும், இந்து திருமண சட்டத்தில் தான் புகுந்து, கௌரவமாக்கப் பட்டது என்பது கவனிக்கத் தக்கது.}

\end{enumerate}

\tamil{ஆகவே, தாலி இருந்தது, அது இழையாக இருந்தது. அதனால், “இழை” என்ற வார்த்தை “தாலி” என்ற பொருளில் உபயோகப்படுத்தப் பட்டது என்பதை, வாலிழை, அணியிழை, ஆயிழை, ஒள்ளிழை, மணியிழை, இளங்கிழை, சேயிழை, பாசிழை, விரலிழை, தெரியிழை, நேரிழை, திருந்திழை, புனையிழை, மின்னிழை, வீங்கிழை, புலையிழை, அவிரிழை, வயங்கிழை, சுடரிழை, நுணங்கிழை போன்ற சொல்லாடல்களில் உள்ளன என்பதை எடுத்துக் காட்டப் பட்டது. அகழ்வாய்வுளில் அத்தகைய அணிகலன்கள் கண்டுபிடிக்கப் பட்டுள்ளன, ஆனால், அவ்வாறே அடையாளங் காணப்படவில்லை. அவை இழைகள் தாம், இழைகள் உலோகமாக இல்லாமல் இருந்தால், மணிகள் முதலியன தனியாக சிதறியிருக்கலாம். அரிக்கமேடு, கீழடி முதலிய இடங்களில் கிடைத்தவை அவ்வாறே கொண்டால், அவை அத்டாட்சியாகின்றன. இந்த ஆய்வுக்கட்டுரையில், இலக்கியம் மற்றும் அகழ்வாய்வு ஆதாரங்களுடன், இப்பிரச்சினை அலசப்பட்டு, “தாலி” இருந்தது மற்றும் திருமணத்தில் கட்டப்பட்ட பழக்கம் இருந்தது என்று எடுத்துக் காட்டப் பட்டது.}

\theendnotes

