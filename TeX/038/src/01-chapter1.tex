
\chapter{Comparative Study in Aṣṭādhyāyī\index{Astadhyayi@Aṣṭādhyāyī} and Tolkāppiyam\index{Tolkappiyam@Tolkāppiyam}}\label{chap01}

\Authorline{B. Sankareswari}


\section*{Synopsis}

\enginline{Tolkāppiyar\index{Tolkappiyar@Tolkāppiyar} and Pāņini\index{Panini@Pāņini} in their respective grammars described the articulatory process of various sounds and speak of eight physical organs which are involved in the production of speech sounds. Each school excels in its own way in giving even the minute phonetic details which can only be explained with the advent of scientific knowledge and technology. We conclude that the two schools are fundamentally different.}

\enginline{Phonetics is the scientific study of speech sounds. Among the Indian languages Tamil\index{Tamil} and Sanskrit\index{Sanskrit} are ancient and have the earliest extant grammars, dealing with the phonetic aspects not only in detail but also in the most sophisticated way. Indian phoneticians talked about the phonetics\index{phonetics} of the respective language for which they wrote the grammars. So we can only reconstruct theory of general phonetics from the description for the sounds of individual languages.}


\section*{Comparison}

\begin{itemize}
\item \enginline{Tolkāppiyar gives separate description for vowels\index{vowels} and consonants\index{consonants} whereas Pāņini gives the same description for the vowels and consonants which are homorganic.}

 \item \enginline{Tolkāppiyar makes a distinction between active and passive articulators. From his statement one can conclude that the articulators make different kinds of configurations in the orator cavity. But from Pāņini’s statement we can know only the place of articulation\index{articulation}.}

 \item \enginline{Tolkāppiyar’s description for the production of later as sounds which are produced by the swelling of the side edge of the tongue allowing the breath air to pass only through the sides is considered to be a very astute observation particularly when we consider the ancient time period of the description.}

 \item \enginline{Pāņini talks about the secondary articulation and dependent sounds. The secondary articulation due to one of the five factors is responsible for discriminating speech sounds. Pitch and secondary features are not clearly described by Tolkāppiyar. We may conclude that the two schools of phonetics, that of Tolkāppiyar and Pāņini, are different. }

 \item \enginline{The language structure of Tamil and Sanskrit being basically different, the grammar rules are also arranged accordingly, in every topic, such as nouns, verbs, prefixes and suffixes, as shown in the tables. }

 \item \enginline{While both grammars address their respective languages in a well-developed, mature form not expected to change from the classic structure, the purpose of writing the two treatises appears to be quite different. }

 \item \enginline{The Tolkāppiyam appears to be composed primarily for non-Tamil (presumably Sanskrit) speakers to easily learn Tamil by presenting the phonetics in a form similar to the Aṣṭādhyāyī\index{Astadhyayi@Aṣṭādhyāyī} for more familiarity, according to experts in the field.}

 \item \enginline{There is some debate among scholars as to whether the Tolkāppiyam was composed by a single author or several.}

\end{itemize}


\section*{தொல்காப்பியமும்பாணினீயமும்}

\begin{flushright}
பா.சங்கரேஸ்வரி,
\end{flushright}

உலகின்செவ்வியல்மொழிகளாகவும்தொன்மைமொழிகளாகவும்கிரீக், இலத்தீன், சீனம், எபிரேயம், தமிழ், சமஸ்கிருதம்ஆகியனகருதப்படுகின்றன. இவற்றுள்தமிழும் - சமஸ்கிருதமும்இந்தியத்துணைக்கண்டத்தில்செழித்துவளர்ந்தவை.

இவ்விருமொழிகளும்இலக்கணவளம்பெற்றவையாகத்திகழ்வதுடன்இந்தியத்துணைக்கண்டத்தின்பல்வேறுமொழிகளுக்குமூலமொழிகளாகவும்விளங்குகின்றன. 

இருதொன்மையானமொழிகளும்ஒன்றுக்கொன்றுதொடர்புகொண்டிருந்தாலும்இவர்களின்இலக்கணமரபுஎன்பதுவேறாகத்தான்இருக்கிறது. 

ஒருமொழியில்இலக்கணம்தோன்றுவதற்குப்பலவிதமானகாரணங்கள்கூறப்படுகின்றன. 

சமூகஅரசியல்தளங்களில்செவ்வாக்குப்பெற்றுத்திகழும்ஒருமொழியின்இலக்கணமரபுமற்றொருமொழியின்இலக்கணமரபிற்குமாதிரியாகவிளங்குகின்றனவா? 

தமிழ்மொழிக்குமுதலில்கிடைத்தநூல்தொல்காப்பியம்.சமஸ்கிருதமொழியில்முதலில்தோன்றியதுஅஷ்டாத்தியாயிஇந்தஇருநூல்களும்இலக்கணத்தைஎவ்வாறுஅமைத்துவிதிகளைஉருவாக்கியுள்ளார்கள்என்பதைஇனிக்காண்போம்.

\subsection{பாணினீயின்காலம்:}

பாணினீஎன்பவர்ஸாலாதூரஎன்றஇடத்தில்பிறந்தவர். இவருடையதந்தையின்பெயர்தாஷீஎனவேஇவரைதாஷீஎன்றுஅடைத்தனர். இவருடையகாலம்கி.மு 3ம்நூற்றாண்டாகஇருக்கலாம்எனக்கூறப்படுகிறது.

 இந்தோஆரியமொழியின்தொடக்ககாலநிலையில்அமைந்திருந்தமொழியின்இலக்கணமேஅஷ்டாத்டாயி. இம்மொழியைபாணினிபாஷாஎன்றுகுறிப்பிடுகிறார். 

ஸம்ஸ்கிருதம்என்றசொல்லைபாணினிஎந்தஇடத்திலும்குறிப்பிடவில்லை. 

முதன்முதலாகஸம்ஸ்கிருதம்என்றசொல்இராமாயணத்தில்காணப்படுகின்றது. 

அஷ்டாத்யாயிக்கும்முன்னும்பலஇலக்கணநூல்கள்இருந்ததுஎன்பதைபாணினிஅந்தந்தஆசிரியார்களைமேற்கோள்காட்டுகிறார். 

ஆபிஸலி (\enginline{6.1.91})கஸ்யப(\enginline{1.2.15: 8.4.67}) கார்க்ய(\enginline{7.3.99}) காலவ(\enginline{6.3.61:,7.1.74}) பாரத்வாஜ(\enginline{7.2.63}) ஸாகடாயன(\enginline{3.4.111: 8.3:4:50}) ஸாகல்ய(\enginline{1.1016:6.1.27)} இவர்களுடையநூல்கள்ஒன்றுகூடதற்போதுகிடைக்கவில்லை.


\subsection{அஷ்டாத்யாயியின்அமைப்பு:}

\enginline{4,000} சூத்திரங்களைஉள்ளடக்கியஎட்டுஅத்தியாயங்களைக்கொண்டது.

\begin{itemize}
\item அத்தியாயம்ஒன்று: ஸம்க்ஞாவிதி, (கலைச்சொல், (உதாத்தம், அனுதாத்தம், ஆத்மனேபதம், பரஸ்மைபதம், ஸ்வரிதம்)

 \item அத்தியாயம்இரண்டு: காரகங்கள்பற்றியவிதிகள்தொகை, வேற்றுமைஉருபு, தாதுக்கள், எண், பால்

 \item அத்தியாயம்மூன்று: வினைகள்தொடர்பானவிதிகள்

 \item அத்தியாயம்நான்கு, ஐந்து: வேற்றுமையுருபுகள். பெண்பால்ஒட்டுக்கள்.,

 \item அத்தியாயம்ஆறு, ஏழு: இரட்டித்தல், ஸந்தி, ஒலியனியல்தொடர்பானவிதிகள், ஒட்டுக்கள்ஆகமங்களில்செயற்பாங்குபற்றியவிதிகள்.

 \item அத்தியாயம்எட்டு: சொற்களின்இரட்டிப்பு, சொல்தொடர்பானவிதிகள்.

\end{itemize}

பாணினிசூத்திரங்களைப்புரிந்துகொள்ள \enginline{7} துணைநூல்களும்உள்ளன. இதுபாணினியால்எழுதப்பட்டதா? அல்லதுவேறுயாரேனும்எழுதினார்களா? என்றகருத்துநிலவுகிறது.

\begin{enumerate}
\item சிவசூத்திரம்: \enginline{14} மாகேஸ்வரசூத்திரம், பிரத்யாஹாரம்என்றுபெயர். (\enginline{44})சிவசூத்திரங்களைப்பயன்படுத்திபிரத்யாஹாரசூத்திரங்களைப்படைத்துள்ளார்.

 \item தாதுபாடம்: \enginline{1970} தாதுக்களைக்குறிப்பிடுகிறார்

 \item கணபாடம்:பெயர்ச்சொல்

 \item உணாதிசூத்திரம்:தாதுக்களுடன்ஒட்டுக்களைச்சேர்ப்பது

 \item பிட்சூத்திரம்:பெயரடிச்சொற்களின்ஒலியன்வடிவம்

 \item லிங்காணுசாசனம்:பும்லிங்கம், ஸ்திரீலிங்கம், நபும்ஸலிங்கம்.

 \item சிக்ஷா:ஒலியனியல்

\end{enumerate}

இவைஏழும்பாணினிசூத்திரங்களைப்புரிந்துகொள்வதற்குப்பயன்படுகிறது. பாணினியின்இலக்கணக்கொள்கைகள்:

\begin{enumerate}
\item அ, இ, உ (ண்)

 \item ரு,ல், ரு, (ங்)

 \item ஏ, ஓ (ங்)

 \item ஐ, ஓள (ச்)

 \item ஹ, ய. வ, ர, (ட்) 

 \item ல (ண்)

 \item ஞ, ம, ங, ண, ந, (ம்)

 \item ஜ, ப, (ஞ்)

 \item கடத (ஷ்)

 \item ஜ, ப, க, ட, த (ஸ்)

 \item க, ப, ச,ட, த, ச, ட, த (வ்) 

 \item கப (ய்)

 \item ஸ, ஷ, ஸ (ர்)

 \item ஹ (ல்) (\enginline{41} பிரத்யாஹாரம்) (எ. கா)

\end{enumerate}

அல் - எல்லாஎழுத்தும்\\ அச் - உயிரெழுத்துக்கள்\\ ஹ - மெய்யெழுத்துக்கள்

பாணினிபொதுவானவிதிகளைஅமைத்துஅவைகளைச்சார்ந்தசிலபுறனடைவிதிகளை (அபவாதம்) அமைத்துள்ளார். உத்ஸர்கவிதி (பொதுவிதி) விரிவானசெயல்எல்லைகொண்டதாகக்காணப்படுகிறது.

இரண்டையுமேஒருசூழ்நிலையில்பயன்படுத்தநேர்ந்தால்அபவாதவிதிதான்பலம்பொருந்தியது.

மேலும்ஒலியழுத்தத்தைப்பற்றியும்கூறுகிறார். தாதுக்கள், ஒட்டுக்கள், பதங்கள்ஆகியவைகளின்சுரங்களைகுறிப்பிட்டுவிட்டு, எந்தெந்தசூழல்களில்மாற்றம்ஏற்படுகிறதுஎன்பதையும்கூறுகிறார்.

பதம் - அனுதாத்தம்\\ தாது - இறுதியசையில்உதாத்தம்\\ ஒட்டு - உதாத்தசுரம்

இவைத்தவிரதொகைச்சொற்களின்இறுதியிலும்ஒலியசைகளைக்குறிப்பிடுகிறார்.


\subsection{வேற்றுமையுருபுகள்:}

\begin{longtable}{|r|l|l|l|}
\hline
 & ஒருமை & இருமை & பன்மை \\
\hline
\enginline{1.} & ஸீ (ராம): & ஓள (ராமௌ) & ஜஸ் (ராமா) \\
\hline
\enginline{2.} & அம் (ராமம்) & ஓளட் (ராமான்) & ஸஸ் \\
\hline
\enginline{3.} & டா (ராமேணே) & ப்யாம் (ராமாப்யாம்) & பிஸ்  (ராமை) \\
\hline
\enginline{4.} & ஙே (ராமாய) & ப்யாம் (ராமாப்யாம்) & ப்யஸ் (ராமேப்ய:) \\
\hline
\enginline{5.} & நுஸி (ராமாத்) & ப்யாம் (ராமாப்யாம்) & ப்யஸ் (ராமேப்ய:) \\
\hline
\enginline{6.} & ஙஸ் (ராமஸ்ய) & ஒஸ் (ராமயோ) & ஆம் (ராமாணாம்) \\
\hline
\enginline{7.} & ஙி (ரமே) & ஒஸ் (ராமயோ:) & ஸீப் (ராமேஷ்) \\
\hline
\end{longtable}


\subsection{மூவிடப்பெயரொட்டுக்கள்}

\begin{longtable}{|r|l|l|l|l|l|l|}
\hline
இடம் & பரஸ்மைபதம் & ஆத்மனேபதம் \\
\hline
 & ஒருமை & இருமை & பன்மை & ஒருமை & இருமை & பன்மை \\
\hline
படர்க்கை & திப் & தஸ் & ஜி & து & ஆதாம் & ஜ \\
\hline
முன்னிலை & ஸிப் & தஸ் & த & தாஸ் & ஆதாம் & த்வம் \\
\hline
தன்மை & மிப் & வஸ் & மஸ் & இட் & வஹி & மஹிங் \\
\hline
\end{longtable}

பாணினியின்அஷ்டாத்யாயீபேச்சுமொழியையும், வேதமொழியையும்உள்வைத்துஎழுதப்பட்டது.

மேலும்இந்தோஆரியமொழிபிராக்கிருதமொழிக்கூறுகள்எதுவுமின்றிசுத்தமாகஇருக்கவேண்டும்என்பதற்காகஉருவாக்கப்பட்டதுஅஷ்டாத்யாயீ.


\subsection{தொல்காப்பியஅமைப்பு:}

தொல்காப்பியஇலக்கணம்பேச்சுமொழி, செய்யுள்மொழிஇரண்டிற்கும்உரியவை. தொல்காப்பியம்மூன்றுஅதிகாரங்களையும்ஒன்பதுஇயல்களையும்கொண்டுஅமைந்துள்ளன. தொல்காப்பியம்அன்றையவழக்குமொழிக்கும்செய்யுள்மொழிக்கும்எனஉருவாக்கப்பட்டஇலக்கணம்ஆகும்.

தமிழ்எழுத்துக்களின்வடிவத்தைப்பற்றிக்கூறும்போதுபுள்ளிஎந்தஇடத்தில்நிற்கவேண்டுமென்றுகூறுகிறார். மெய்யெழுத்துக்கள்புள்ளியோடுதான்வரும் (15) 

உயிர்மெய்யெழுத்திற்குப்புள்ளிகிடையாது. அதனால்அந்தஎழுத்து ‘அ’வுடன்சேர்ந்துஒலிக்கும் (17)

மெய்யைஈறாகக்கொண்டவைகள்புள்ளியொடுமுடியும் (16) 

எழுத்துக்களின்வைப்புமுறையிலும்இரண்டுமொழிக்கும்பொதுவானஎழுத்துக்களைமுதலில்கூறிவிட்டு, அந்தமொழியிலில்லாத, தமிழின்சிறப்பெழுத்துக்களானழ, ள, ற, னஇவைகளைஇறுதியில்வைக்கிறார்.

வேற்றுமொழிபேசும்மக்களுடையமொழியில்எந்தெந்தஎழுத்துக்கள்இருக்கின்றனவோ, அவைகளோடுஉருவத்திலோஉச்சரிப்பிலோஒருபுதியமொழியின்எழுத்துக்களைஒப்பிட்டுக்காட்டிமொழியைபயிற்றுவித்தால், அம்மொழியைப்பயில்வதுமற்றவர்களுக்குஎளிது. இதைநன்றாகஉணர்ந்துகொண்டதொல்காப்பியர்.

இம்முறையில்தொல்காப்பியத்தைஉருவாக்கியிருக்கிறார். வரலாற்றுப்பின்னணியில்இந்நூலைஆராயும்போதுவேற்றுமொழியாளர்களுக்காகத்தான்இந்நூல்எழுதப்பட்டிருக்கும்என்றுமீனாட்சிகுறிப்பிடுகிறார்.

ஒருமொழியின்இலக்கணம்என்பதுபேச்சுமொழிக்கும்கவிதைமொழிக்கும்உரியஒருஇலக்கணமாகஇருத்தல்வேண்டும்.

 என்பதுதொல்காப்பியரின்ஒருங்கிணைந்தஇலக்கணக்கொள்கையாகும்.வளரும்மொழிக்காகஆக்கப்படாமல்அப்படியேஇருக்கவேண்டும்என்றஒருநிலையில்ஆக்கப்பட்டதுஇவ்விலக்கணம்.

தொல்காப்பியம்பல்வேறுமாற்றங்களைப்புதிதாகப்பெற்றுவளரும்மொழியாகஇருத்தல்வேண்டும்என்றநிலையில்தான்பல்வேறுபுறநடைச்சூத்திரங்களையும்உருவாக்குகின்றது. தொல்காப்பியஎழுத்ததிகாரம் (\enginline{482,483}) சொல்லதிகாரம் (\enginline{946}) காணப்படும்சூத்திரங்கள்மேலேசொன்னசெய்திகளைவலியுறுத்துகின்றன.

\begin{center}
“செய்யுள்மருங்கினும்வழக்கியல்மருங்கினும்”
\end{center}

போன்றசூத்திரங்கள்அவரதுமொழிபற்றியகொள்கையைசுட்டிக்காட்டும.; செய்யுள்என்றசொல்லும்தொல்காப்பியர்எண்ணத்தின்படிவெறும்கவிதைஎன்றபொருளில்அமையவில்லை.

\begin{center}
செய்யுள்
\end{center}

\begin{longtable}{|l|l|l|}
\hline
அடிவரையறையுள்ளது &  & அடிவரையறையில்லாதது \\
\hline
பாட்டு & நாட்டுப்புறவியல்சார்ந்தவை & மற்றவை \\
\hline
1.பிசி & 1. உரை &  \\
\hline
2.முதுசொல் & 2. நூல் \\
\hline
3. அங்கதம் &  &  \\
\hline
4. வாய்மொழி &  &  \\
\hline
\end{longtable}

\begin{longtable}{|m{3.5cm}|m{3.5cm}|m{3.5cm}|}
\hline
வரிசைஎண் & தொல்காப்பியம் & பாணனீ \\
\hline
\enginline{1.} & எழுத்துக்கள்உயிர்– \enginline{12}\\ 
							மெய்எழுத்துக்கள் - \enginline{18}\\
							சார்பெழுத்துக்கள்– \enginline{03}\\
							சார்பெழுத்துக்கள்– \enginline{3} & உயிர்எழுத்துக்கள்ஸ்ரீ \enginline{9}\\
								மெய்எழுத்துக்கள்ஸ்ரீ \enginline{33}\\
								--------\\\enginline{42}\\
								-------\\\enginline{5}\\
								உயிர்ஓசைஅயோகவாகம்அனுஸ்வார, விசர்க, \\
							ஜிஹ்வமூலிய, உபத்மானீய, யமா: \\
\hline
\enginline{2.} & ஒருமை, பன்மை & ஒருமை, இருமை, பன்மை \\
\hline
\enginline{3.} & உயர்திணைஅஃறிணைபாகுபாடுஉண்டு\\ ஆண்பால், பெண்பால் (பொருள்) & பாகுபாடுஇல்லை.ஆண்பால், பெண்பால்,\\ அலிப்பால். (சொல்வகைளைக்குறித்தது) \\
\hline
\enginline{4.} & வேற்றுமைஉருபுகள்உயர்திணைக்கும்அ\\ ஃறிணைக்கும்ஒன்றே & வேற்றுமைஉருபுமாறுபடுகின்றது. ஒருமை,\\ இருமை, பன்மை, ராம: ராமௌராமா: \\
\hline
\enginline{5.} & பிறப்புறுப்புகள்  -  \enginline{8} & சிரஷ், கண்டாஹ், உரஹ், நாசிகா, தந்தா,\\ ஜிஜ்வாமூலம், தந்தா, ஸ்தௌசாடலு \\
\hline
\enginline{6.} & ஆய்தம் & ஆஹ - என்றஉச்சரிப்பில்பயன்படுத்துகின்றனர் \\
\hline
\enginline{7.} & ஐந்திரம்நிறைந்ததொல்காப்பியம் & இந்திரனால்செய்யப்பட்டவியாகரணம் \\
\hline
\enginline{8.} & ‘தொல்காப்பியன்எனத்தன்பெயர்தோற்றி’ & தோன்றி –தன்வினைதோன்றி–பிறவினை.\\ இதுவடமொழியில்மொழிஎனப்படும் \\
\hline
\enginline{9.} & கண்இமை …… மாத்திரை & மாத்ராஎன்றசொல்\\ ‘மாத்திரை’என்றுதற்பவமாக்கிப்பயன்படுத்தப்பட்டுள்ளது \\
\hline
\enginline{10.} & ல, ளிஃகான், ணனகான்என்பது	 & இரண்டெழுத்திற்குஒருகாரப்பிரத்தியயம்கொடு\\ த்து ‘தலகாரம்’என்பர் \\
\hline
\enginline{11.} & குன்றிசைமொழிவயின் (தொ. எ. \enginline{411})	 & புலுதச்சந்தியைஒத்துவந்துள்ளது \\
\hline
\enginline{12.} & அகரஇகரம்ஐகாரம் (தொ. எ. \enginline{54}) & சந்தியஷரம்–உயிர்ப்புணர்ச்சி \\
\hline
\enginline{13.} & ‘உப்பகாரம்ஒன்றுஎனமொழிப’.தபுஎன்றசொல்கெடுஎனவும், கெடுவிஎனவும்வரும்.	 & அந்தர்பாவிதணிச் \\
\hline
\enginline{14.} & சுட்டுச்சினைநீடியமென்தொடர்மொழியும் \\(தொ. எ. \enginline{159})\\  ஆங்குக்கொண்டான் & சுட்டுச்சினைஅவ்வியயத்திதன்எனவழங்கப்படும் \\
\hline
\enginline{15.} & ழுகரம்உகரம்நீடிடன்உடைத்தேகரம்வருதல்ஆவயினான.\\ (தொல்எ. \enginline{261}) (எ.கா) பழூஉப்பல் & வடநூலார்குறில்நின்றஇடத்தில்புலுதம்வரும்என்பர். \\
\hline
\enginline{16.} & ‘மன்னம்சின்னும் (தொல். எ. \enginline{333})இடைச்சொல் & அவ்வியயதத்திதன் \\
\hline
\enginline{17.} & மூன்றன்உருபாகிய ‘ஒடுசொல்லை\\ ‘ஆசிரியனொடுமாணவன் வந்தான்எனத்தொல்காப்பியர்கூறுகிறார் & பாணினீஒடுஎன்னும்உருபுஏற்றசொல்லைஅப்பிரதானம்என்றும் , \\வந்தான்என்னும்வினையொடுமுடிந்தசொல்லைப்பிரதானம்என்றும்கூறுவர். \\
\hline
\end{longtable}

தமிழ்மற்றும்சமஸ்கிருதமொழிகளுக்குப் (தொல்காப்பியத்திற்கும், அஷ்டாத்தியாயிக்கும்) பொதுவான, இலக்கணக்கூறுகளைஇரண்டுநூல்களும்எவ்வாறுவிளக்குகின்றனஎன்பதைஇதுகாறும்விளக்கினோம். 

இப்பொழுதுஇருநூல்களும்இலக்கணத்தைஎவ்வாறுஅமைத்துவிதிகளைஉருவாக்கியுள்ளார்கள்என்பதனைஒவ்வொருஇலக்கணநூலாகச்சுருக்கமாகக்காணலாம்.


\subsection{தொல்காப்பியம்-விதிஅமைப்புமுறை:}

\begin{enumerate}
\item தொல்காப்பியர்தலைப்புவாரியாகப்பொருளைஅடுக்கிவைத்துள்ளார். 

 \item தொல்காப்பியவிதிகள்சூத்திரநடையில்இல்லாதுநூற்பாஎன்றவகையில்மிகநீண்டதாகஅமைகின்றன.

 \item முழுமையானஎண், விளக்கமானவழக்குபற்றியகுறிப்பு, பிறப்புஎனத்தமிழ்ஒலிகளைவிளக்குகின்றன. எழுத்துஅதிகாரம்ஒருமுழுப்பகுதியாகஉள்ளது.

 \item தொல்காப்பியர்பல்வேறுவகையானவாக்கியஅமைப்புக்கள், வாக்கியவகைகள், வாக்கியங்களின்வழாநிலை,வழுவமைதிஆகியவற்றைவிரிவாகவிளக்குகிறார். இத்துடன்சிலவழக்குகள்செய்யுளுக்கேஉரியவைஎன்றும்அவைஅன்றாடவழக்கிற்குஉடையவையல்லஎன்றும்விளக்குகிறது.

 \item இலக்கியத்திற்கேஉரியஇடம்பெறும்சொற்கள்எனஇயற்சொல், திசைச்சொல், எனச்சொற்களைகுறிப்பிடுகிறது.

 \item தொல்காப்பியசெய்யுளின் (கரு) பாடுபொருள், யாப்பு, அணிஎன்றுஒருதனிப்பகுதியைசெய்யுளுக்காகஇலக்கணத்தில்கொண்டுவிளக்;குகிறது.

 \item தொல்காப்பியத்தில்விதிகள்பொதுவானவை, புறனடைஎனப்பகுக்கப்பட்டுள்ளன. அஷ்டாத்தியாயியில்தொல்காப்பியத்தைப்போலபொதுவாகவிதிகள்ஒன்றோடுஒன்றுஇணைக்கப்படவில்லை. சிலவிதிகள்மட்டுமேமாற்றேற்றுவிதிகளாகஉள்ளன. 

\end{enumerate}


\subsection{அஷ்டாத்யாயி -;-விதிஅமைப்புமுறை:}

\begin{enumerate}
\item பாணினியின்அமைப்புமுறையில்பாடுபொருள்கள்தலைப்புவாரியாகஅமைக்கப்படவில்லை. அவர்விதிகளைசிலஒட்டுக்களின்அடிப்படையில் n அமைத்துள்ளார்.

 \item பாணினியின்இலக்கணம்பல்வேறுவகையானவிதிகளைக்கொண்டது. பெரும்பாலும்அவைஒன்றோடுஒன்றுஉறவுடையவை. இவ்வாறாகவிதிகள்உத்ஸர்கா, \enginline{apavāda, antarańga-- bahirańga, pratiṣedha} எனஒன்றோடுஒன்றுஉறவுடையதால்இவற்றைஒருகுறிப்பிட்டவைப்புமுறையில்தான்பயன்படுத்தப்படவேண்டும். பலபரிபாஸாவிதிகள்என்றகருத்துஎதுவரவேண்டும்என்றுவிளக்கும். ஆகையால்விதிகளைவைப்புமுறையைபின்பற்றியேசரியானவற்றைஉருவாக்கவேண்டும்.

 \item பாணினியின்விதிகள்சூத்திரநடையில்உருவாக்கப்பட்டுள்ளன. அவைசுருக்கமாகவும்செறிவாகவும்அல்ஐப்ராசூத்திரங்களைப்போலஇருக்கும். இந்தநடைஆசிரியரைவிதிகளைச்சுருக்கமாகஆக்கத்துணைபுரிந்தாலும், அவற்றைப்புரிந்துகொள்ளவும், விளக்கவும்சிலஉதவிகள்தேவைப்படும். அவையும்அந்தசூத்திரங்களில்இடம்பெறுமாறுபலஉபாயங்களைஉள்ளேவைத்துப்படைத்துள்ளார்.

 \item பெரும்பான்மையானபாணினியின்விதிகள்சரியானசொற்களைஆக்கிக்கொள்வதுதொடர்பானவையாகவேஉள்ளன. அதற்கேற்பஅவருடையவிளக்கம்சொற்களைஆக்குவதற்குரியபல்வேறுஉறுப்புக்களைகுறித்ததாகவேஉள்ளன. பாணினிபல்வேறுஅலகுகளையும்அவற்றைஇணைத்துச்சரியானவார்த்தைகளைஉருவாக்குவதும்குறித்தும், அப்போதுஅவைஅடையும்மாற்றங்கள்குறித்தும், அப்போதுசந்திவிதிகள்குறித்தும்பேசுகிறார். பாணினிசிலசொற்கள்எப்படிஆக்கப்பட்டனஎன்றுவிளக்கமுடியாதவடிவங்களை (\enginline{ńīp} யவயயௌ) ஆங்காங்கேவிளக்குகிறார்.

\end{enumerate}


\subsection{அமைப்புவேறுபாட்டிற்க்கானகாரணங்கள்}

இந்தஇரண்டுஇலக்கணநூல்களையும்ஒட்டுமொத்தமாகஒப்பிடும்போதுஒற்றுமைகளைவிடவேறுபாடுகளேநிறையஉள்ளனஎன்பதுதெரியும். 

ஆகையால்வேறுப்பாட்டிற்க்கானகாரணங்களும்ஒன்றல்லபலஎன்பதுபுரியும்.

\begin{enumerate}
\item இந்தஇரண்டுமொழிகளுக்கும்இடiயில்உள்ளஅமைப்புவேறுபாடேஅவைவெவ்வேறுஅணுகுமுறையில்எழுதகாரணமாகஅமைந்தன.

 \item இரண்டுஇலக்கணஆசிரியர்களும்அவர்கள்இலக்கணம்எழுதக்காரணமாயிருந்ததாக்கமும்காரணமும்வேறுவேறாகஇருந்ததும்ஒருகாரணம்ஆகும்.

 \item தொல்காப்பியத்தைஎழுதியதுயார்என்பதுகுறித்துவேறுபட்டகருத்துநிலவுகிறது.

\end{enumerate}

\begin{itemize}
\item தொல்காப்பியம்ஒருவராலேஎழுதமுடியாது. அதாவதுதொல்காப்பியர்மூன்றுஅதிகாரங்களும்ஒருவரேஎழுதியவைஎன்பது.

 \item முதல்இரண்டுஅதிகாரங்களும்எழுதியவர்ஒருவர். மூன்றாவதுஅதிகாரத்தைஎழுதியவர் (பொருள்) வேறுஒருவர்என்றும், பின்னர்எழுதிச்சேர்க்கப்பட்டது. இன்னொருகருத்துபொருளதிகாரத்தில்இடைச்செருகல்கள்உள்ளனஎன்பதாகும். (\enginline{zvlabil (1973): T.P. Meenakshi Sundaram (1974) etr}. தொல்காப்பியம்பலரால்எழுதப்பட்டதுஎன்பதைஏற்றுக்கொண்டால்எழுத்ததிகாரத்தைஎழுதுவதற்குஒருநோக்கமும்அதாவதுதொடக்கநிலையில்உள்ளவேற்றுமொழியாளர்கள்கற்பதற்குஎழுதப்பட்டது. அந்தஇயலில்எழுத்துக்களின்எண்ணிக்கை, எழுத்துக்களைஎப்படிஉச்சரிக்கவேண்டும்என்பதும், பேச்சொலிகளின்வழக்காறுகள், மற்றுஎளியவிளக்கமானசந்திவிதிகள்இயங்கும்முறைகள்எல்லாம்இடம்பெற்றுள்ளன.

\end{itemize}

தொல்காப்பியம்சொல்லதிகாரத்தில்பெரும்பான்மையானஇயல்கள் (கிளவியாக்கம்தவிர) அதையேகாட்டுகிறது. கிளவியாக்கம்வாக்கியங்களின்வழாநிலையையும், வழுவையும், வழுவமைதியையும்பேசுகின்றன. இதுமொழியில்முன்னறிவுபெற்றிருக்கவேண்டும்என்பதைஉணர்த்துகிறது. அதிகாரத்தினமுதல்இயலாகஅமைந்திருப்பதேஇதற்குக்காரணமாகும். எழுத்ததிகாரம்படித்தப்பின்அமைவதால்எளிமையாகபுரிவதற்குவழிவகுக்கிறது.

தொல்காப்பியம்பொருளதிகாரம்தொடக்கநிலையில் (புதியவர்கள்) உள்ளவர்களுக்குஅல்ல. அந்தஇயல்செய்யுளைஎப்படிஎழுதுவதுஎன்றுகூறுகிறது.

ஒன்றுக்குமேற்பட்டவர்கள்எழுதியதுதொல்காப்பியம்என்றுகூறுவோர்முதல்இரண்டுஅதிகாரங்கள்காலத்தைகி.மு\enginline{3.} என்றும், மூன்றாவதுஅதிகாரம்பிற்காலத்தில்எழுதப்பட்டதுஎன்பர்.

தொல்காப்பியத்தைஎழுதியவர்ஒருவரேஎன்பவர்கள்ஒரேகாலத்தையேகூறுகிறார்கள். இருந்தாலும்கூறும்காலத்தில்வேறுபாடுகள்உள்ளன.

தொல்காப்பியம் - சங்கஇலக்கியங்களின்காலம்

தொல்காப்பியம்சங்கஇலக்கியங்களுக்குமுந்தியதாஅல்லதுபிந்தியதாஎன்பதேஇங்குஎழும்முதல்வினாஆகும். இதுமற்றொருவினாவுக்குவழிவகுக்கும். தொல்காப்பியம்முதலில்தோன்றியதமிழ்நூலாயிருந்தால்அதுசங்கஇலக்கியத்துக்குமுந்தியதாகும். முதல்தமிழ்இலக்கணமாயிருப்பின்அதுசங்கஇலக்கியங்களுக்குப்பிந்தியதாகும். 

சங்கஇலக்கியம், தொல்காப்பியம்இரண்டின்காலத்தைஒப்பிடுவதில்அறிஞர்கள்வேறுபடுகின்றனர்.

\begin{enumerate}
\item அகத்தியம்என்பதேபழையஇலக்கணம்கி.பி.\enginline{9} ஆம்நூற்றாண்டுஎன்றுகருதப்பட்டது. அண்மையில்தொல்காப்பியமேபழமையானஇலக்கணம்அகத்தியம்அல்லஎன்றுநிறுவப்பட்டுள்ளது. (சிவத்தம்பி \enginline{1986: 35})

 \item தொல்காப்பியம்பழையதமிழ்இலக்கணம்மட்டும்அல்ல. இதுசங்கஇலக்கியங்களுக்குமுந்தியது.

\end{enumerate}

சமஸ்கிருதஇலக்கணத்தின்தோற்றம்வேதங்களின்மொழிபுரியாததாகப்போனநிலையில்தான்தொடங்கியதுஎனப்பொதுவாகநம்பப்படுகிறது. 

வேதச்சடங்குகளைவேதங்கள்புரிந்துகொண்டுயாகங்களில்பலியிடுதல்என்றவாய்பாடுகளைபுரிந்துகொள்ள, வேதமொழியைவிளக்குதல்ஒருதேவைஇருந்தது. 

இதுவே \enginline{pratisakhya} போன்றஇலக்கியங்களைஎழுதுவதற்குக்காரணமாய்அமைந்தது. இதுவேபாணினிஅவருடையஇலக்கணமானஅஷ்டாத்யாயியைஎழுதுவதற்குநோக்கமாய்அமைந்தது. அகச்சான்றுகள்இதைக்காட்டவில்லை. 

அஷ்டாத்யாயியின்பெரும்பான்மையானவிதிகள் \enginline{bhāṣā} யைவிளக்குவதாகவேஉள்ளன. \enginline{bhāṣā} என்பது, இந்தியாவின்வடமேற்குப்பகுதியில்வாழ்ந்தமனிதர்கள்பேசும்மொழியாகும். மேலும், அஷ்டாத்யாயியில்உள்ளநாலாயிரம்சூத்திரங்களில்ஐந்நூறுசூத்திரங்களேவேதங்கள்குறித்தவை. இவையும்கூட \enginline{bhāṣā} வின்ஒலித்திரிபுகள்பற்றியேபேசுகின்றன. ஆகையால்பாணினிவேதமொழியைக்காக்கும்நோக்கத்தில்இலக்கணத்தைஎழுதினார்என்பதுசரியாகத்தோன்றவில்லை. ஏனெனில் 

\enginline{1}) அவருடையமொழிபற்றியவிளக்கம் \enginline{prātiśākhyā-s} போலில்லாமல், பேச்சுஒலிகளின்உச்சரிப்ப, அவற்றைவிளக்குதல்என்றுஅமையவில்லை. 

\enginline{2}) அவர் \enginline{Yāska} செய்ததைப்போலசொற்களின்சொற்பிறப்பபை; பற்றிவிளக்கமுடியவில்லை.

பாணினிஅவருடையஅவருடையதனிமனிதபேச்சுவழக்கைவிளக்குவதற்குமுயன்றதாகத;தோன்றுகிறது.இந்தமுறையைஅவர்மேற்கொண்டதற்குஒருகாரணம்பிறமொழிக்குடும்பங்பகளின்கூறுகள்உட்புகுந்துவிடும்என்பதனாலேயேயாம்.இதற்குச்சிலசான்றுகள்உள்ளன.இந்தோஇரானியின்மொழிக்குடும்பத்தின்தாக்கம்சமஸ்;கிருத்தில்உள்ளன.இதனைப்பாணினியின்விதியொன்றில்காணலாம். (\enginline{1983}).

 பிராஹ\_யிஎன்றவடதிராவிடமொழிபிராக்ரித்முதலியவற்றின்தாக்கம்உள்ளது.இருந்தாலும்பாணினியின்காலத்துப்பெரியமனிதர்களின்இந்தோ-ஆரியமொழியில்இத்தகையதாக்கம்எந்தநிலையிலும்இருந்ததாகக்கூறமுடியாது. 

பாணினிமனிதர்களின்பேச்சுமொழியின்தூய்மையைகாக்கவிரும்பினார்.அவருடையசூத்திரநடைஅந்தக்காலத்துஅறிஞர்களால்ஏற்றுகொள்ளப்பட்டநடைஅந்தக்காலத்துஎளிதில்புரிந்துகொள்ளமுடியாததாகவும்அதற்குபாஸ்யாவ்ருத்திபோன்றவைதேவைபடுவதாகஇருந்தன.

 உயர்ந்ததொழில்நுட்பம்வாய்ந்தாகஅமைந்தபாணினியின்இலக்கணம்சரியாகப்புரிந்துகொள்வதுஅறிவுசால்பெரியோர்களுக்குமட்டுமேஇயலும்என்றநிலையில்அவைஎழுதும்முறையைஅறிமுகப்படுத்துவதற்குக்ஒருகாரணமாகஅமைந்தது.

காந்தாரப்பகுதிகளில்பாணினியின்காலத்தில்தோன்றியகரோடிஎழுத்துஅக்காலத்தில்எழுதும்முறைபரவலாகியிருந்ததைக்காட்டுகிறது. 

இவ்வாறுஅ;ஷ்டாத்தியாயியைஎழுதுவதற்கானகாரணம்மற்றமொழிகளின்தாக்கம்இந்தோஆரியமொழிகளில்ஏற்பட்டதேஆகும்.

அதன்தூய்மையைக்காக்கும்நோக்கத்துடனேபாணினிஇலக்கணம்எழுதினார்.

இவ்வாறுஇரண்டுஇலக்கணநூல்களும்வேறுசூழலில்வேறுவேறுநோக்கத்தோடுஎழுதப்பட்டவைஎன்பதுசுட்டிக்காட்டப்பட்டன.

அதனால்இரண்டுஇலக்கணங்களுக்கும்இடையில்ஒற்றுமையைவிடவேற்றுமையேஅதிகம்இருக்கும். முன்னரேகுறிப்பட்டவாறு(\enginline{Burnell 1875}) தொல்காப்பியத்திற்கும்சமஸ்கிருதங்களான \enginline{prātiśākhyā-s} கச்சாயனாரின்பாலிஇலக்கணம்இவற்றிற்கிடையேசிலஒற்றுமைகளைக்கண்டார்.

தமிழும்சமஸ்கிருதமும்மொழிக்குடும்பஅமைப்பில்தொடர்பேஇல்லாதமொழிகளாம். அவைதம்தம்மொழிக்குடும்பத்திற்கேஉரியசிலகுறிப்பிட்டகூறுகளைக்கொண்டுவிளக்குகின்றன. அவர்கள்அவரவருடையமொழியின்அமைப்பைஅப்படியேஏற்றுக்கொண்டுஅவரவர்இலக்கணங்களைப்படைத்துள்ளனர். 

எனவேநாம்தமிழ்சமஸ்கிருதம்ஆகியத்தம்கூறுகளையும்பலதனித்தகூறுகளையும்அடையாளம்காணலாம். 

மேலேகொடுக்கப்பட்டதொகுப்புரைஇவ்விருஇலக்கணக்களுக்கும்இடையில்சிலஒற்றுமைகளையும்சிலவேறுபாடுகளையும்காட்டுகின்றன.


\section*{Bibliography}

\begin{thebibliography}{99}
\bibitem{chap1-key01} \enginline{Ilakkuvanar,S. Tolkappiyam in English with critical studies(1963). Madurai Kural Neri publishing House.}

 \bibitem{chap1-key02} \enginline{Keith A.B.A History of Sanskrit Literature (1920) Reprint London, Oxford University Press. }

 \bibitem{chap1-key03} \enginline{Winternitz, M. – History of Indian Literature vol. III. Translated from German in to English by Subhadra Jha. Delhi. Motilal Banarasidas. 1963.}

 \bibitem{chap1-key04} \enginline{Anavarathavinayagam pillai, S. (1923) ‘The Sanskrit element in the vocabularies of the Dravidian languages’, Dravidian Studies, 3: 29 – 53.}

 \bibitem{chap1-key05} \enginline{Caldwell, R. (1956) A Comparative Grammar of the Dravidian or South Indian Family of Languages, (Reprinted), Madras University.}

 \bibitem{chap1-key06} \enginline{Kanapathy pillai, K.(193) ‘A study of the language of Tamil Inscriptions from 7th century and 8th century A.D.’ ( ph.D., Dissertation, university of London).}

 \bibitem{chap1-key07} \enginline{Meenakshi,K(1997)TOLKAPPIYAM AND ASTADHYAYI, International Institute of Tamil studies,Chennai-6000113.}

 \bibitem{chap1-key08} \enginline{Meenakshisundram, T.P. \& Shanmugam pillai, M (1951) ‘The portugese influence revealed by Tamil words’ Journal of the Annamalai University, vol. 16.}

 \bibitem{chap1-key09} \enginline{Murugaiyan, K. (1970) ‘Tolkappiar’s concept of phonetics’, Seminar on Tolkappiyam (memeo), Annamalai University.}

 \bibitem{chap1-key10} \enginline{Subrahmanya Sastri, P.S. (1924) History of Grammatical Theories in Tamil, Madras.}

 \bibitem{chap1-key11} \enginline{Vaidhianathan, S. (191) Indo – Aryan Loan words in Old Tamil. Madras.}

 \bibitem{chap1-key12} \enginline{G+uzre;jpud; f.,e;jpankhopfs; Xh; mwpkfk;\textgreater  epNtjpjh ghjpg;gfk;\textgreater  jpUr;rp\textgreater  2004.}

 \bibitem{chap1-key13} \enginline{Ifehj uhIh\textgreater  K.F.,e;jpa nkhopfspy; xg;gpyf;fpak; eh;kjh gjpg;gfk;\textgreater  nrd;id – 1989.}

 \bibitem{chap1-key14} \enginline{Ayyappa Panikar, k., studies in Comparative Litrature, Blackie \& Sons Publishers Pvt. Ltd., Madras, 1985.}

 \bibitem{chap1-key15} \enginline{Burnell, A.C. The Aindra School of Sanskrit Grammarians. Mangalore: Basel Mission Book of Tract Depossitory, 1875}

 \end{thebibliography}

