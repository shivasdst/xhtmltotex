
\chapter{Comparative Study in Aṣṭādhyāyī\index{Astadhyayi@Aṣṭādhyāyī} and Tolkāppiyam\index{Tolkappiyam@Tolkāppiyam}}\label{chap01}

\Authorline{B. Sankareswari}


\section*{Synopsis}

\enginline{Tolkāppiyar\index{Tolkappiyar@Tolkāppiyar} and Pāņini\index{Panini@Pāņini} in their respective grammars described the articulatory process of various sounds and speak of eight physical organs which are involved in the production of speech sounds. Each school excels in its own way in giving even the minute phonetic details which can only be explained with the advent of scientific knowledge and technology. We conclude that the two schools are fundamentally different.}

\enginline{Phonetics is the scientific study of speech sounds. Among the Indian languages Tamil\index{Tamil@Tamiḻ} and Sanskrit\index{Sanskrit} are ancient and have the earliest extant grammars, dealing with the phonetic aspects not only in detail but also in the most sophisticated way. Indian phoneticians talked about the phonetics\index{phonetics} of the respective language for which they wrote the grammars. So we can only reconstruct theory of general phonetics from the description for the sounds of individual languages.}

\newpage

\section*{Comparison}

\begin{itemize}
\item \enginline{Tolkāppiyar gives separate description for vowels\index{vowels} and consonants\index{consonants} whereas Pāņini gives the same description for the vowels and consonants which are homorganic.}

 \item \enginline{Tolkāppiyar makes a distinction between active and passive articulators. From his statement one can conclude that the articulators make different kinds of configurations in the orator cavity. But from Pāņini’s statement we can know only the place of articulation\index{articulation}.}

 \item \enginline{Tolkāppiyar’s description for the production of later as sounds which are produced by the swelling of the side edge of the tongue allowing the breath air to pass only through the sides is considered to be a very astute observation particularly when we consider the ancient time period of the description.}

 \item \enginline{Pāņini talks about the secondary articulation and dependent sounds. The secondary articulation due to one of the five factors is responsible for discriminating speech sounds. Pitch and secondary features are not clearly described by Tolkāppiyar. We may conclude that the two schools of phonetics, that of Tolkāppiyar and Pāņini, are different. }

 \item \enginline{The language structure of Tamil and Sanskrit being basically different, the grammar rules are also arranged accordingly, in every topic, such as nouns, verbs, prefixes and suffixes, as shown in the tables. }

 \item \enginline{While both grammars address their respective languages in a well-developed, mature form not expected to change from the classic structure, the purpose of writing the two treatises appears to be quite different. }

 \item \enginline{The Tolkāppiyam appears to be composed primarily for non-Tamil (presumably Sanskrit) speakers to easily learn Tamil by presenting the phonetics in a form similar to the Aṣṭādhyāyī\index{Astadhyayi@Aṣṭādhyāyī} for more familiarity, according to experts in the field.}

 \item \enginline{There is some debate among scholars as to whether the Tolkāppiyam was composed by a single author or several.}

\end{itemize}


\section*{\tamil{தொல்காப்பியமும் பாணினீயமும்}}

\begin{flushright}
\tamil{பா. சங்கரேஸ்வரி,}
\end{flushright}

\tamil{உலகின் செவ்வியல் மொழிகளாகவும் தொன்மை மொழிகளாகவும் கிரீக், இலத்தீன், சீனம், எபிரேயம், தமிழ், சமஸ்கிருதம் ஆகியன கருதப்படுகின்றன. இவற்றுள் தமிழும் – சமஸ்கிருதமும் இந்தியத்துணைக்கண்டத்தில் செழித்துவளர்ந்தவை. இவ்விருமொழிகளும் இலக்கண வளம் பெற்றவையாகத் திகழ்வதுடன் இந்தியத் துணைக்கண்டத்தின் பல்வேறுமொழிகளுக்கு மூலமொழிகளாகவும் விளங்குகின்றன. இரு தொன்மையான மொழிகளும் ஒன்றுக்கொன்று தொடர்புகொண்டிருந்தாலும் இவர்களின் இலக்கண மரபு என்பது வேறாகத்தான் இருக்கிறது. ஒருமொழியில் இலக்கணம் தோன்றுவதற்குப் பலவிதமான காரணங்கள் கூறப்படுகின்றன. }

\tamil{சமூக அரசியல் தளங்களில் செவ்வாக்குப் பெற்றுத்திகழும் ஒருமொழியின் இலக்கணமரபு மற்றொருமொழியின் இலக்கணமரபிற்கு மாதிரியாக விளங்குகின்றனவா? தமிழ் மொழிக்கு முதலில் கிடைத்த நூல் தொல்காப்பியம். சமஸ்கிருத மொழியில் முதலில் தோன்றியது அஷ்டாத்தியாயி இந்த இரு நூல்களும் இலக்கணத்தை எவ்வாறு அமைத்து விதிகளை உருவாக்கியுள்ளார்கள் என்பதை இனிக்காண்போம்.}

\subsection{\tamil{பாணினீயின் காலம்:}}

\tamil{பாணினீ என்பவர் ஸாலாதூர என்ற இடத்தில் பிறந்தவர். இவருடைய தந்தையின் பெயர் தாக்ஷீ புத்திரன். எனவே இவரை தாக்ஷீ என்று அடைத்தனர். இவருடைய காலம் கி.மு 3ம் நூற்றாண்டாக இருக்கலாம் எனக்கூறப்படுகிறது. இந்தோ ஆரிய மொழியின் தொடக்ககால நிலையில் அமைந்திருந்த மொழியின் இலக்கணமே அஷ்டாத்டாயி. இம்மொழியை பாணினி பாஷா என்று குறிப்பிடுகிறார். ஸம்ஸ்கிருதம் என்ற சொல்லை பாணினி எந்த இடத்திலும் குறிப்பிடவில்லை. முதன் முதலாக ஸம்ஸ்கிருதம் என்ற சொல் இராமாயணத்தில் காணப்படுகின்றது. அஷ்டாத்யாயிக்கும் முன்னும் பல இலக்கண நூல்கள் இருந்தது என்பதை பாணினி அந்தந்த ஆசிரியார்களை மேற்கோள்காட்டுகிறார்.}

\tamil{ஆபிஸலி} (\enginline{6.1.91})\tamil{கஸ்யப}(\enginline{1.2.15: 8.4.67}) \tamil{கார்க்ய}(\enginline{7.3.99}) \tamil{காலவ}\break (\enginline{6.3.61:,7.1.74}) \tamil{பாரத்வாஜ}(\enginline{7.2.63}) \tamil{ஸாகடாயன}(\enginline{3.4.111: 8.3:4:50}) \tamil{ஸாகல்ய}(\enginline{1.1016:6.1.27)} \tamil{இவர்களுடைய நூல்கள் ஒன்றுகூட தற்போது கிடைக்கவில்லை.}


\subsection{\tamil{அஷ்டாத்யாயியின் அமைப்பு:}}

\enginline{4,000} \tamil{சூத்திரங்களை உள்ளடக்கிய எட்டு அத்தியாயங்களைக் கொண்டது.}

\begin{itemize}
\item \tamil{அத்தியாயம் ஒன்று: ஸம்க்ஞாவிதி, (கலைச்சொல், (உதாத்தம், அனுதாத்தம், ஆத்மனேபதம், பரஸ்மைபதம், ஸ்வரிதம்)}

 \item \tamil{அத்தியாயம் இரண்டு: காரகங்கள் பற்றிய விதிகள் தொகை, வேற்றுமை உருபு, தாதுக்கள், எண், பால்}

 \item \tamil{அத்தியாயம் மூன்று: வினைகள் தொடர்பான விதிகள்}

 \item \tamil{அத்தியாயம் நான்கு, ஐந்து: வேற்றுமையுருபுகள், பெண்பால் ஒட்டுக்கள்}

 \item \tamil{அத்தியாயம் ஆறு, ஏழு: இரட்டித்தல், ஸந்தி, ஒலியனியல் தொடர்பான விதிகள், ஒட்டுக்கள் ஆகமங்களில் செயற்பாங்கு பற்றிய விதிகள்.}

 \item \tamil{அத்தியாயம் எட்டு: சொற்களின் இரட்டிப்பு, சொல் தொடர்பான விதிகள்.}

\end{itemize}

\tamil{பாணினி சூத்திரங்களைப் புரிந்துகொள்ள} \enginline{7} \tamil{துணை நூல்களும் உள்ளன. இதுபாணினியால் எழுதப்பட்டதா}? \tamil{அல்லது வேறுயாரேனும் எழுதினார்களா}? \tamil{என்ற கருத்துநிலவுகிறது.}

\begin{enumerate}
\item \tamil{சிவசூத்திரம்:} \enginline{14} \tamil{மாகேஸ்வரசூத்திரம், பிரத்யாஹாரம் என்று பெயர்.}. (\enginline{44})\tamil{சிவசூத்திரங்களைப் பயன்படுத்தி பிரத்யாஹார சூத்திரங்களைப் படைத்துள்ளார்}.

 \item \tamil{தாது பாடம்:} \enginline{1970} \tamil{ தாதுக்களைக் குறிப்பிடுகிறார்}

 \item \tamil{கணபாடம்: பெயர்ச்சொல்}

 \item \tamil{உணாதி சூத்திரம்: தாதுக்களுடன் ஒட்டுக்களைச் சேர்ப்பது}

 \item \tamil{பிட்சூத்திரம்: பெயரடிச் சொற்களின் ஒலியன்வடிவம்}

 \item \tamil{லிங்காணு சாசனம்: பும்லிங்கம், ஸ்திரீலிங்கம், நபும்ஸலிங்கம்.}.

 \item \tamil{சிக்ஷா: ஒலியனியல்}

\end{enumerate}

\tamil{இவை ஏழும்பாணினி சூத்திரங்களைப் புரிந்து கொள்வதற்குப் பயன்படுகிறது. பாணினியின் இலக்கணக் கொள்கைகள்:}:

\begin{enumerate}
\item \tamil{அ, இ, உ (ண்)}

 \item \tamil{ரு, ல், ரு, (ங்)}

 \item \tamil{ஏ, ஓ (ங்)}

 \item \tamil{ஐ, ஓள (ச்)}

 \item \tamil{ஹ, ய. வ, ர, (ட்)} 

 \item \tamil{ல (ண்)}

 \item \tamil{ஞ, ம, ங, ண, ந, (ம்)}

 \item \tamil{ஜ, ப, (ஞ்)}

 \item \tamil{கடத (ஷ்)}

 \item \tamil{ஜ, ப, க, ட, த (ஸ்)}

 \item \tamil{க, ப, ச, ட, த, ச, ட, த (வ்)} 

 \item \tamil{கப (ய்)}

 \item \tamil{ஸ, ஷ, ஸ (ர்)}

 \item \tamil{ஹ (ல்)} (\enginline{41} \tamil{பிரத்யாஹாரம்})

\end{enumerate}

\tamil{(எ. கா)}

\tamil{அல் - எல்லாஎழுத்தும்}\\\tamil{அச் - உயிரெழுத்துக்கள்}\\\tamil{ஹ - மெய்யெழுத்துக்கள்}

\tamil{பாணினி பொதுவான விதிகளை அமைத்து அவைகளைச் சார்ந்த சிலபுறனடை விதிகளை (அபவாதம்) அமைத்துள்ளார். உத்ஸர்கவிதி (பொதுவிதி) விரிவான செயல் எல்லை கொண்டதாகக் காணப்படுகிறது. இரண்டையுமே ஒரு சூழ்நிலையில் பயன்படுத்த நேர்ந்தால் அபவாதவி திதான்பலம் பொருந்தியது.}

\tamil{மேலும் ஒலியழுத்தத்தைப் பற்றியும் கூறுகிறார். தாதுக்கள், ஒட்டுக்கள், பதங்கள் ஆகியவைகளின் சுரங்களை குறிப்பிட்டுவிட்டு, எந்தெந்த சூழல்களில் மாற்றம் ஏற்படுகிறது என்பதையும் கூறுகிறார்.}

\tamil{பதம் - அனுதாத்தம்}\\\tamil{தாது - இறுதியசையில் உதாத்தம்}\\\tamil{ஒட்டு - உதாத்தசுரம்}

\tamil{இவைத் தவிர தொகைச் சொற்களின் இறுதியிலும் ஒலியசைகளைக் குறிப்பிடுகிறார்.}


\subsection{\tamil{வேற்றுமையுருபுகள்}:}

\begin{longtable}{|r|l|l|l|}
\hline
 & \tamil{ஒருமை} & \tamil{இருமை} & \tamil{பன்மை} \\
\hline
\enginline{1.} & \tamil{ஸீ (ராம)}: & \tamil{ஓள (ராமௌ)} & \tamil{ஜஸ் (ராமா)} \\
\hline
\enginline{2.} & \tamil{அம் (ராமம்)} & \tamil{ஓளட் (ராமான்)} & \tamil{ஸஸ்} \\
\hline
\enginline{3.} & \tamil{டா (ராமேணே)} & \tamil{ப்யாம் (ராமாப்யாம்)} & \tamil{பிஸ்  (ராமை)} \\
\hline
\enginline{4.} & \tamil{ஙே (ராமாய)} & \tamil{ப்யாம் (ராமாப்யாம்)} & \tamil{ப்யஸ் (ராமேப்ய:)} \\
\hline
\enginline{5.} & \tamil{நுஸி (ராமாத்)} & \tamil{ப்யாம் (ராமாப்யாம்)} & \tamil{ப்யஸ் (ராமேப்ய:)} \\
\hline
\enginline{6.} & \tamil{ஙஸ் (ராமஸ்ய)} & \tamil{ஒஸ் (ராமயோ)} & \tamil{ஆம் (ராமாணாம்)} \\
\hline
\enginline{7.} & \tamil{ஙி (ரமே)} & \tamil{ஒஸ் (ராமயோ:)} & \tamil{ஸீப் (ராமேஷ்)} \\
\hline
\end{longtable}


\subsection{\tamil{மூவிடப்பெயரொட்டுக்கள்}}

\begin{longtable}{|r|l|l|l|l|l|l|}
\hline
\tamil{இடம்} & \tamil{பரஸ்மைபதம்} & \tamil{ஆத்மனேபதம்} \\
\hline
 & \tamil{ஒருமை} & \tamil{இருமை} & \tamil{பன்மை} & \tamil{ஒருமை} & \tamil{இருமை} & \tamil{பன்மை} \\
\hline
\tamil{படர்க்கை} & \tamil{திப்} & \tamil{தஸ்} & \tamil{ஜி} & \tamil{து} & \tamil{ஆதாம்} & \tamil{ஜ} \\
\hline
\tamil{முன்னிலை} & \tamil{ஸிப்} & \tamil{தஸ்} & \tamil{த} & \tamil{தாஸ்} & \tamil{ஆதாம்} & \tamil{த்வம்} \\
\hline
\tamil{தன்மை} & \tamil{மிப்} & \tamil{வஸ்} & \tamil{மஸ்} & \tamil{இட்} & \tamil{வஹி} & \tamil{மஹிங்} \\
\hline
\end{longtable}

\tamil{பாணினியின் அஷ்டாத்யாயீ பேச்சுமொழியையும், வேத மொழியையும் உள்வைத்து எழுதப்பட்டது. மேலும் இந்தோ ஆரிய மொழி பிராக்கிருத மொழிக்கூறுகள் எதுவுமின்றி சுத்தமாக இருக்கவேண்டும் என்பதற்காக உருவாக்கப்பட்டது அஷ்டாத்யாயீ.}


\subsection{\tamil{தொல்காப்பிய அமைப்பு}:}

\tamil{தொல்காப்பிய இலக்கணம் பேச்சுமொழி, செய்யுள்மொழி இரண்டிற்கும் உரியவை. தொல்காப்பியம் மூன்று அதிகாரங்களையும் ஒன்பது இயல்களையும் கொண்டு அமைந்துள்ளன. தொல்காப்பியம் அன்றையவழக்கு மொழிக்கும் செய்யுள் மொழிக்கும் என உருவாக்கப்பட்ட இலக்கணம்ஆகும்.}

\tamil{தமிழ் எழுத்துக்களின் வடிவத்தைப் பற்றிக்கூறும் போதுபுள்ளி எந்த இடத்தில் நிற்க வேண்டுமென்று கூறுகிறார். மெய்யெழுத்துக்கள் புள்ளியோடுதான் வரும்} \enginline{(15)} \tamil{உயிர்மெய்யெழுத்திற்குப் புள்ளி கிடையாது. அதனால் அந்த எழுத்து ‘அ’வுடன் சேர்ந்து ஒலிக்கும்} \enginline{(17)} \tamil{மெய்யை ஈறாகக் கொண்டவைகள் புள்ளியொடு முடியும்} \enginline{(16)} \tamil{எழுத்துக்களின் வைப்புமுறையிலும் இரண்டு மொழிக்கும் பொதுவான எழுத்துக்களை முதலில் கூறிவிட்டு, அந்த மொழியிலில்லாத, தமிழின் சிறப்பெழுத்துக்களான ழ, ள, ற, ன இவைகளை இறுதியில் வைக்கிறார்.}

\tamil{வேற்றுமொழி பேசும் மக்களுடைய மொழியில் எந்தெந்த எழுத்துக்கள் இருக்கின்றனவோ, அவைகளோடு உருவத்திலோ உச்சரிப்பிலோ ஒருபுதிய மொழியின் எழுத்துக்களை ஒப்பிட்டுக்காட்டி மொழியை பயிற்றுவித்தால், அம்மொழியைப் பயில்வது மற்றவர்களுக்கு எளிது. இதைநன்றாக உணர்ந்து கொண்ட தொல்காப்பியர். இம்முறையில் தொல்காப்பியத்தை உருவாக்கியிருக்கிறார். வரலாற்றுப் பின்னணியில் இந்நூலை ஆராயும் போது வேற்று மொழியாளர்களுக்காகத்தான் இந்நூல் எழுதப்பட்டிருக்கும் என்றுமீனாட்சி குறிப்பிடுகிறார்.}

\tamil{ஒரு மொழியின் இலக்கணம் என்பது பேச்சு மொழிக்கும் கவிதை மொழிக்கும் உரிய ஒரு இலக்கணமாக இருத்தல் வேண்டும். என்பது தொல்காப்பியரின் ஒருங்கிணைந்த இலக்கணக் கொள்கையாகும். வளரும் மொழிக்காக ஆக்கப்படாமல் அப்படியே இருக்கவேண்டும் என்ற ஒருநிலையில் ஆக்கப்பட்டது இவ்விலக்கணம்.}

\tamil{தொல்காப்பியம் பல்வேறு மாற்றங்களைப் புதிதாகப் பெற்றுவளரும் மொழியாக இருத்தல் வேண்டும் என்ற நிலையில் தான் பல்வேறு புறநடைச் சூத்திரங்களையும் உருவாக்குகின்றது. தொல்காப்பிய எழுத்ததிகாரம்} (\enginline{482,483}) \tamil{சொல்லதிகாரம்} (\enginline{946}) \tamil{காணப்படும் சூத்திரங்கள் மேலே சொன்ன செய்திகளை வலியுறுத்துகின்றன.}

\begin{center}
\tamil{“செய்யுள் மருங்கினும் வழக்கியல் மருங்கினும்”}
\end{center}

\tamil{போன்ற சூத்திரங்கள் அவரது மொழி பற்றிய கொள்கையை சுட்டிக்காட்டும.; செய்யுள் என்ற சொல்லும் தொல்காப்பியர் எண்ணத்தின் படி வெறும் கவிதை என்ற பொருளில் அமையவில்லை.}

\begin{center}
\tamil{செய்யுள்}
\end{center}

\begin{longtable}{|p{3cm}|p{3cm}|p{3cm}|}
\hline
\tamil{அடிவரையறையுள்ளது} &  & \tamil{அடிவரையறையில்லாதது} \\
\hline
\tamil{பாட்டு} & \tamil{நாட்டுப்புறவியல்சார்ந்தவை} & \tamil{மற்றவை} \\
\hline
1. \tamil{பிசி} & 1. \tamil{உரை} &  \\
\hline
2. \tamil{முதுசொல்} & 2. \tamil{நூல்} &  \\
\hline
3. \tamil{அங்கதம்} &  &  \\
\hline
4. \tamil{வாய்மொழி} &  &  \\
\hline
\end{longtable}

\begin{longtable}{|m{3.5cm}|m{3.5cm}|m{3.5cm}|}
\hline
\tamil{வரிசைஎண்} & \tamil{தொல்காப்பியம்} & \tamil{பாணனீ} \\
\hline
\enginline{1.} & \tamil{எழுத்துக்கள்உயிர்} - \enginline{12}\\\tamil{மெய்எழுத்துக்கள்} - \enginline{18}\\\tamil{சார்பெழுத்துக்கள்} - \enginline{03}\\\tamil{சார்பெழுத்துக்கள்} - \enginline{3} & \tamil{உயிர்எழுத்துக்கள்} - \enginline{9}\\\tamil{மெய்எழுத்துக்கள்} - \enginline{33}\\
								--------\\\enginline{42}\\
								-------\\\enginline{5}\\\tamil{உயிர் ஓசை அயோகவாகம் அனுஸ்வார,\break விசர்க, ஜிஹ்வமூலிய, உபத்மானீய, யமா:} \\
\hline
\enginline{2.} & \tamil{ஒருமை, பன்மை} & \tamil{ஒருமை, இருமை, பன்மை} \\
\hline
\enginline{3.} & \tamil{உயர்திணை  அஃறிணை}\\\tamil{பாகுபாடு  உண்டு  ஆண்பால்,}\\\tamil{பெண்பால் (பொருள்)} & \tamil{பாகுபாடு இல்லை.  ஆண்பால், பெண்பால்,},\\\tamil{அலிப்பால். (சொல்வகைளைக் குறித்தது)} \\
\hline
\enginline{4.} & \tamil{வேற்றுமை  உருபுகள்}\\\tamil{உயர்திணைக்கும்}\\\tamil{அஃறிணைக்கும் ஒன்றே} & \tamil{வேற்றுமை  உருபு  மாறுபடுகின்றது.}\\\tamil{ஒருமை, இருமை, பன்மை,} \\\tamil{ராம: ராமௌராமா:} \\
\hline
\enginline{5.} & \tamil{பிறப்புறுப்புகள்}  -  \enginline{8} & \tamil{சிரஷ்,  கண்டாஹ்,  உரஹ்,  நாசிகா, தந்தா,}\\\tamil{ஜிஜ்வா  மூலம்,  தந்தா, ஸ்தௌசாடலு} \\
\hline
\enginline{6.} & \tamil{ஆய்தம்} & \tamil{ஆஹ – என்ற உச்சரிப்பில்பயன்படுத்துகின்றனர்} \\
\hline
\enginline{7.} & \tamil{ஐந்திரம்நி றைந்த தொல்காப்பியம்} & \tamil{இந்திரனால்  செய்யப்பட்ட வியாகரணம்} \\
\hline
\enginline{8.} & \tamil{‘தொல்காப்பியன் எனத்தன் பெயர் தோற்றி’} & \tamil{தோன்றி –தன்வினை தோன்றி–பிறவினை.}\\\tamil{இது வடமொழியில் மொழி  எனப்படும்} \\
\hline
\enginline{9.} & \tamil{கண் இமை …… மாத்திரை} & \tamil{மாத்ரா  என்ற  சொல் ‘மாத்திரை’என்று}\\\tamil{தற்பவமாக்கிப் பயன் படுத்தப்பட்டுள்ளது} \\
\hline
\enginline{10.} & \tamil{ல, ளிஃகான், ணனகான் என்பது}	 & \tamil{இரண்டெழுத்திற்கு ஒரு காரப்பிரத்தியயம்}\\\tamil{கொடுத்து ‘தலகாரம்’என்பர்} \\
\hline
\enginline{11.} & \tamil{குன்றிசை  மொழிவயின் (தொ. எ.} \enginline{411}) & \tamil{புலுதச்சந்தியை  ஒத்துவந்துள்ளது} \\
\hline
\enginline{12.} & \tamil{அகர இகரம்  ஐகாரம் (தொ. எ. }\enginline{54}) & \tamil{சந்தியஷரம் – உயிர்ப்புணர்ச்சி} \\
\hline
\enginline{13.} & \tamil{‘உப்பகாரம் ஒன்று என மொழிப’.தபு என்ற சொல்கெடு எனவும், கெடுவி எனவும்வரும்.} & \tamil{அந்தர்  பாவிதணிச்} \\
\hline
\enginline{14.} & \tamil{சுட்டுச்சினை நீடியமென் தொடர் மொழியும்} \\\tamil{(தொ. எ.} \enginline{159})\\\tamil{ஆங்குக் கொண்டான்} & \tamil{சுட்டுச்சினை அவ்வியயத்திதன் என வழங்கப்படும்} \\
\hline
\enginline{15.} & \tamil{ழுகரம் உகரம் நீடிடன் உடைத்தே கரம் வருதல் ஆவயினான.}\\ (\tamil{தொல்எ.}. \enginline{261}) \tamil{(எ.கா) பழூஉப்பல்} & \tamil{வடநூலார் குறில் நின்ற இடத்தில் புலுதம்வரும் என்பர்.} \\
\hline
\enginline{16.} & \tamil{‘மன்னம் சின்னும் (தொல். எ.} \enginline{333})\tamil{ இடைச்சொல்} & \tamil{அவ்வியயதத்திதன்} \\
\hline
\enginline{17.} & \tamil{மூன்றன் உருபாகிய ‘ஒடு சொல்லை}\\\tamil{‘ஆசிரியனொடுமாணவன் வந்தான் எனத் தொல்காப்பியர் கூறுகிறார்} & \tamil{பாணினீ ஒடு என்னும் உருபு ஏற்றசொல்லை அப்பிரதானம் என்றும், வந்தான் என்னும்}\\\tamil{வினையொடு முடிந்த சொல்லைப் பிரதானம் என்றும்கூறுவர்.} \\
\hline
\end{longtable}

\tamil{தமிழ் மற்றும் சமஸ்கிருத மொழிகளுக்குப் (தொல்காப்பியத்திற்கும், அஷ்டாத்தியாயிக்கும்) பொதுவான, இலக்கணக் கூறுகளை இரண்டு நூல்களும் எவ்வாறு விளக்குகின்றன என்பதை இதுகாறும் விளக்கினோம். இப்பொழுது இரு நூல்களும் இலக்கணத்தை எவ்வாறு அமைத்து விதிகளை உருவாக்கியுள்ளார்கள் என்பதனை ஒவ்வொரு இலக்கண நூலாகச் சுருக்கமாகக் காணலாம்.}


\subsection{\tamil{தொல்காப்பியம்-விதிஅமைப்புமுறை}:}

\begin{enumerate}
\item \tamil{தொல்காப்பியர் தலைப்புவாரியாகப் பொருளை அடுக்கிவைத்துள்ளார்.}

 \item \tamil{தொல்காப்பியவிதிகள் சூத்திர நடையில் இல்லாது நூற்பா என்ற வகையில் மிக நீண்டதாக அமைகின்றன.}

 \item \tamil{முழுமையான எண், விளக்கமான வழக்கு பற்றிய குறிப்பு, பிறப்பு எனத் தமிழ் ஒலிகளை விளக்குகின்றன. எழுத்து அதிகாரம் ஒரு முழுப் பகுதியாக உள்ளது.}

 \item \tamil{தொல்காப்பியர் பல்வேறு வகையான வாக்கிய அமைப்புக்கள், வாக்கிய வகைகள், வாக்கியங்களின் வழாநிலை, வழுவமைதி ஆகியவற்றை விரிவாக விளக்குகிறார். இத்துடன் சிலவழக்குகள் செய்யுளுக்கே உரியவை என்றும் அவை அன்றாட வழக்கிற்கு உடையவையல்ல என்றும் விளக்குகிறது.}

 \item \tamil{இலக்கியத்திற்கே உரிய இடம்பெறும் சொற்கள் என இயற்சொல், திசைச்சொல், எனச் சொற்களை குறிப்பிடுகிறது.}

 \item \tamil{தொல்காப்பிய செய்யுளின் (கரு) பாடுபொருள், யாப்பு, அணி என்று ஒரு தனிப்பகுதியை செய்யுளுக்காக இலக்கணத்தில் கொண்டு விளக்குகிறது.}

 \item \tamil{தொல்காப்பியத்தில் விதிகள் பொதுவானவை, புறனடை எனப் பகுக்கப்பட்டுள்ளன. அஷ்டாத்தியாயியில் தொல்காப்பியத்தைப் போல பொதுவாக விதிகள் ஒன்றோடு ஒன்று இணைக்கப் படவில்லை. சில விதிகள் மட்டுமே மாற்றேற்று விதிகளாக உள்ளன.}

\end{enumerate}


\subsection{\tamil{அஷ்டாத்யாயி -;-விதி அமைப்பு முறை}:}

\begin{enumerate}
\item \tamil{பாணினியின் அமைப்பு முறையில் பாடுபொருள்கள் தலைப்பு வாரியாக அமைக்கப்படவில்லை. அவர்விதிகளை சில ஒட்டுக்களின் அடிப்படையில் n அமைத்துள்ளார்.}

 \item \tamil{பாணினியின் இலக்கணம் பல்வேறு வகையான விதிகளைக் கொண்டது. பெரும்பாலும் அவை ஒன்றோடு ஒன்று உறவுடையவை. இவ்வாறாகவிதிகள் உத்ஸர்கா,}, \enginline{apavāda, antarańga-- bahirańga, pratiṣedha} \tamil{என ஒன்றோடு ஒன்று உறவுடையதால் இவற்றை ஒரு குறிப்பிட்ட வைப்பு முறையில்தான் பயன்படுத்தப் படவேண்டும். பல பரிபாஸா விதிகள் என்ற கருத்து எதுவர வேண்டும் என்று விளக்கும். ஆகையால் விதிகளை வைப்பு முறையை பின்பற்றியே சரியானவற்றை உருவாக்கவேண்டும்.}

 \item \tamil{பாணினியின் விதிகள் சூத்திரநடையில் உருவாக்கப்பட்டுள்ளன. அவை சுருக்கமாகவும் செறிவாகவும் அல் ஐ Pப்ராசூ த்திரங்களைப் போல இருக்கும். இந்த நடை ஆசிரியரை விதிகளைச் சுருக்கமாக ஆக்கத்துணை புரிந்தாலும், அவற்றைப் புரிந்து கொள்ளவும், விளக்கவும் சில உதவிகள் தேவைப்படும். அவையும் அந்த சூத்திரங்களில் இடம்பெறுமாறு பல உபாயங்களை உள்ளே வைத்துப் படைத்துள்ளார்.}

 \item \tamil{பெரும்பான்மையான பாணினியின் விதிகள் சரியான சொற்களை ஆக்கிக்கொள்வது தொடர்பானவையாகவே உள்ளன. அதற்கேற்ப அவருடைய விளக்கம் சொற்களை ஆக்குவதற்குரிய பல்வேறு உறுப்புக்களை குறித்ததாகவே உள்ளன. பாணினி பல்வேறு அலகுகளையும் அவற்றை இணைத்துச் சரியான வார்த்தைகளை உருவாக்குவதும் குறித்தும், அப்போது அவை அடையும் மாற்றங்கள் குறித்தும், அப்போது சந்தி விதிகள் குறித்தும் பேசுகிறார். பாணினி சில சொற்கள் எப்படி ஆக்கப்பட்டன என்று விளக்க முடியாத வடிவங்களை} (\enginline{ńīp} \tamil{யவயயௌ) ஆங்காங்கே விளக்குகிறார்.}

\end{enumerate}


\subsection{\tamil{அமைப்பு வேறு பாட்டிற்க்கான காரணங்கள்}}

\tamil{இந்த இரண்டு இலக்கண நூல்களையும் ஒட்டுமொத்தமாக ஒப்பிடும்போது ஒற்றுமைகளைவிட வேறுபாடுகளே நிறைய உள்ளன என்பது தெரியும். ஆகையால் வேறுப்பாட்டிற்க்கான காரணங்களும் ஒன்றல்ல பல என்பது புரியும்.}

\begin{enumerate}
\item \tamil{இந்த இரண்டு மொழிகளுக்கும் இடiயில் உள்ள அமைப்பு வேறுபாடே அவை வெவ்வேறு அணுகுமுறையில் எழுதகாரணமாக அமைந்தன}

 \item \tamil{இரண்டு இலக்கண ஆசிரியர்களும் அவர்கள் இலக்கணம் எழுதக் காரணமாயிருந்த தாக்கமும் காரணமும் வேறு வேறாக இருந்ததும் ஒரு காரணம்ஆகும்.}

 \item \tamil{தொல்காப்பியத்தை எழுதியதுயார் என்பது குறித்து வேறுபட்ட கருத்து நிலவுகிறது.}

\end{enumerate}

\begin{itemize}
\item \tamil{தொல்காப்பியம் ஒருவராலே எழுதமுடியாது. அதாவது தொல்காப்பியர் மூன்று அதிகாரங்களும் ஒருவரே எழுதியவை என்பது.}

 \item \tamil{முதல் இரண்டு அதிகாரங்களும் எழுதியவர் ஒருவர். மூன்றாவது அதிகாரத்தை எழுதியவர் (பொருள்) வேறு ஒருவர் என்றும், பின்னர் எழுதிச் சேர்க்கப்பட்டது. இன்னொரு கருத்து பொருளதிகாரத்தில் இடைச் செருகல்கள் உள்ளன என்பதாகும்.} (\enginline{zvlabil (1973): T.P. Meenakshi Sundaram (1974) etr}. \tamil{தொல்காப்பியம் பலரால் எழுதப்பட்டது என்பதை ஏற்றுக்கொண்டால் எழுத்ததிகாரத்தை எழுதுவதற்கு ஒருநோக்கமும் அதாவது தொடக்க நிலையில் உள்ள வேற்றுமொழியாளர்கள் கற்பதற்கு எழுதப்பட்டது. அந்த இயலில் எழுத்துக்களின் எண்ணிக்கை, எழுத்துக்களை எப்படி உச்சரிக்க வேண்டும் என்பதும், பேச்சொலிகளின் வழக்காறுகள், மற்று எளிய விளக்கமான சந்தி விதிகள் இயங்கும் முறைகள் எல்லாம் இடம்பெற்றுள்ளன.}

\end{itemize}

\tamil{தொல்காப்பியம் சொல்லதிகாரத்தில் பெரும்பான்மையான இயல்கள் (கிளவியாக்கம்தவிர) அதையே காட்டுகிறது. கிளவியாக்கம் வாக்கியங்களின் வழா நிலையையும், வழுவையும், வழுவமைதியையும் பேசுகின்றன. இது மொழியில் முன்னறிவு பெற்றிருக்க வேண்டும் என்பதை உணர்த்துகிறது. அதிகாரத்தின முதல் இயலாக அமைந்திருப்பதே இதற்குக் காரணமாகும். எழுத்ததிகாரம் படித்தப்பின் அமைவதால் எளிமையாக புரிவதற்கு வழிவகுக்கிறது.}

\tamil{தொல்காப்பியம் பொருளதிகாரம் தொடக்க நிலையில் (புதியவர்கள்) உள்ளவர்களுக்கு அல்ல. அந்த இயல்செய்யுளை எப்படி எழுதுவது என்று கூறுகிறது.}

\tamil{ஒன்றுக்கு மேற்பட்டவர்கள் எழுதியது தொல்காப்பியம் என்று கூறுவோர் முதல் இரண்டு அதிகாரங்கள் காலத்தை கி.மு} \enginline{3.} \tamil{என்றும், மூன்றாவது அதிகாரம் பிற்காலத்தில் எழுதப்பட்டது என்பர்.}

\tamil{தொல்காப்பியத்தை எழுதியவர் ஒருவரே என்பவர்கள் ஒரே காலத்தையே கூறுகிறார்கள். இருந்தாலும் கூறும் காலத்தில் வேறுபாடுகள் உள்ளன.}

\tamil{தொல்காப்பியம் – சங்க இலக்கியங்களின் காலம்}

\tamil{தொல்காப்பியம் சங்க இலக்கியங்களுக்கு முந்தியதா அல்லது பிந்தியதா என்பதே இங்கு எழும் முதல்வினா ஆகும். இது மற்றொரு வினாவுக்கு வழிவகுக்கும். தொல்காப்பியம் முதலில் தோன்றிய தமிழ் நூலாயிருந்தால் அதுசங்க இலக்கியத்துக்கு முந்தியதாகும். முதல் தமிழ் இலக்கணமாயிருப்பின் அது சங்க இலக்கியங்களுக்குப் பிந்தியதாகும்.}

\tamil{சங்க இலக்கியம், தொல்காப்பியம் இரண்டின் காலத்தை ஒப்பிடுவதில் அறிஞர்கள் வேறுபடுகின்றனர்.}

\begin{enumerate}
\item \tamil{அகத்தியம் என்பதே பழைய இலக்கணம் கி.பி.}\enginline{9} \tamil{ஆம் நூற்றாண்டு என்று கருதப்பட்டது. அண்மையில் தொல்காப்பியமே பழமையான இலக்கணம் அகத்தியம் அல்ல என்று நிறுவப்பட்டுள்ளது. (சிவத்தம்பி} \enginline{1986: 35})

 \item \tamil{தொல்காப்பியம் பழைய தமிழ் இலக்கணம் மட்டும் அல்ல. இதுசங்க இலக்கியங்களுக்கு முந்தியது.}

\end{enumerate}

\tamil{சமஸ்கிருத இலக்கணத்தின் தோற்றம் வேதங்களின் மொழி புரியாததாகப் போன நிலையில்தான் தொடங்கியது எனப் பொதுவாக நம்பப்படுகிறது. வேதச்சடங்குகளை வேதங்கள் புரிந்து கொண்டு யாகங்களில் பலியிடுதல் என்ற வாய்பாடுகளை புரிந்துகொள்ள, வேதமொழியை விளக்குதல் ஒரு தேவை இருந்தது. இதுவே} \enginline{prātisākhya} \tamil{போன்ற இலக்கியங்களை எழுதுவதற்குக் காரணமாய் அமைந்தது. இதுவே பாணினி அவருடைய இலக்கணமான அஷ்டாத்யாயியை எழுதுவதற்கு நோக்கமாய் அமைந்தது. அகச்சான்றுகள் இதைக் காட்டவில்லை. அஷ்டாத்யாயியின் பெரும்பான்மையான விதிகள்} \enginline{bhāṣā} \tamil{யை விளக்குவதாகவே உள்ளன.} \enginline{bhāṣā} \tamil{என்பது, இந்தியாவின் வடமேற்குப் பகுதியில் வாழ்ந்த மனிதர்கள் பேசும் மொழியாகும். மேலும், அஷ்டாத்யாயியில் உள்ள நாலாயிரம் சூத்திரங்களில் ஐந்நூறு சூத்திரங்களே வேதங்கள் குறித்தவை. இவையும் கூட} \enginline{bhāṣā} \tamil{வின் ஒலித்திரிபுகள் பற்றியே பேசுகின்றன. ஆகையால் பாணினி வேதமொழியைக் காக்கும் நோக்கத்தில் இலக்கணத்தை எழுதினார் என்பது சரியாகத் தோன்றவில்லை. ஏனெனில்}

\enginline{1}) \tamil{அவருடைய மொழி பற்றிய விளக்கம்} \enginline{prātiśākhyā-s} \tamil{போலில்லாமல், பேச்சு ஒலிகளின் உச்சரிப்ப, அவற்றை விளக்குதல் என்று அமையவில்லை.} 

\enginline{2}) \tamil{அவர்} \enginline{Yāska\index{Yaska@Yāska}} \tamil{செய்ததைப் போல சொற்களின் சொற்பிறப்பபை; பற்றி விளக்க முடியவில்லை.}.

\tamil{பாணினி அவருடைய அவருடைய தனிமனித பேச்சு வழக்கை விளக்குவதற்கு முயன்றதாகத; தோன்றுகிறது. இந்தமுறையை அவர் மேற்கொண்டதற்கு ஒரு காரணம் பிறமொழிக் குடும்பங்பகளின் கூறுகள் உட்புகுந்து விடும் என்பதனாலேயேயாம். இதற்குச் சில சான்றுகள் உள்ளன. இந்தோ இரானியின் மொழிக் குடும்பத்தின் தாக்கம் சமஸ்;கிருத்தில் உள்ளன. இதனைப் பாணினியின் விதியொன்றில் காணலாம்.} (\enginline{1983}). \tamil{பிராஹ\_யி என்ற வட திராவிட மொழி பிராக்ரித் முதலியவற்றின் தாக்கம் உள்ளது. இருந்தாலும் பாணினியின் காலத்துப் பெரியமனிதர்களின் இந்தோ-ஆரிய மொழியில் இத்தகைய தாக்கம் எந்த நிலையிலும் இருந்ததாகக் கூற முடியாது. பாணினி மனிதர்களின் பேச்சு மொழியின் தூய்மையை காக்க விரும்பினார். அவருடைய சூத்திரநடை அந்தக்காலத்து அறிஞர்களால் ஏற்றுகொள்ளப்பட்ட நடை அந்தக்காலத்து எளிதில் புரிந்துகொள்ள முடியாததாகவும் அதற்கு பாஸ்யாவ்ருத்தி போன்றவை தேவைபடுவதாக இருந்தன. உயர்ந்த தொழில்நுட்பம் வாய்ந்தாக அமைந்த பாணினியின் இலக்கணம் சரியாகப் புரிந்துகொள்வது அறிவுசால் பெரியோர்களுக்கு மட்டுமே இயலும் என்ற நிலையில் அவை எழுதும் முறையை அறிமுகப்படுத்துவதற்கு ஒருகாரணமாக அமைந்தது. காந்தாரப் பகுதிகளில் பாணினியின் காலத்தில் தோன்றிய கரோடி எழுத்து அக்காலத்தில் எழுதும் முறை பரவலாகியிருந்ததைக் காட்டுகிறது.}

\tamil{இவ்வாறு அஷ்டாத்தியாயியை எழுதுவதற்கான காரணம் மற்ற மொழிகளின் தாக்கம் இந்தோ ஆரிய மொழிகளில் ஏற்பட்டதே ஆகும். அதன் தூய்மையைக் காக்கும் நோக்கத்துடனே பாணினி இலக்கணம் எழுதினார்.}

\tamil{இவ்வாறு இரண்டு இலக்கண நூல்களும் வேறு சூழலில் வேறுவேறு நோக்கத்தோடு எழுதப்பட்டவை என்பது சுட்டிக் காட்டப்பட்டன. அதனால் இரண்டு இலக்கணங்களுக்கும் இடையில் ஒற்றுமையை விட வேற்றுமையே அதிகம் இருக்கும். முன்னரே குறிப்பட்டவாறு} (\enginline{Burnell 1875}) \tamil{தொல்காப்பியத்திற்கும் சமஸ்கிருதங்களான} \enginline{prātiśākhyās} \tamil{கச்சாயனாரின் பாலி இலக்கணம் இவற்றிற்கிடையே சில ஒற்றுமைகளைக் கண்டார்.}

\tamil{தமிழும் சமஸ்கிருதமும் மொழிக் குடும்ப அமைப்பில் தொடர்பே இல்லாத மொழிகளாம். அவை தம் தம் மொழிக்குடும்பத்திற்கே உரிய சில குறிப்பிட்ட கூறுகளைக் கொண்டு விளக்குகின்றன. அவர்கள் அவரவருடைய மொழியின் அமைப்பை அப்படியே ஏற்றுக்கொண்டு அவரவர் இலக்கணங்களைப் படைத்துள்ளனர். எனவே நாம் தமிழ் சமஸ்கிருதம் ஆகியத்தம் கூறுகளையும் பலதனித்த கூறுகளையும் அடையாளம் காணலாம். மேலே கொடுக்கப்பட்ட தொகுப்புரை இவ்விரு இலக்கணக்களுக்கும் இடையில் சில ஒற்றுமைகளையும் சில வேறுபாடுகளையும் காட்டுகின்றன.}


\section*{Bibliography}

\begin{thebibliography}{99}
\bibitem{chap1-key01} \enginline{Anavarathavinayagam Pillai, S. (1923) “The Sanskrit element in the vocabularies of the Dravidian languages”. \textit{Dravidian Studies.} 3: 29 – 53.}

 \bibitem{chap1-key02} \enginline{Ayyappa Panikar, K. (1985) \textit{Studies in Comparative Literature.} Madras: Blackie \& Sons Publishers Pvt. Ltd.}

 \bibitem{chap1-key03} \enginline{Burnell, A.C. (1875) \textit{The Aindra School of Sanskrit Grammarians.} Mangalore: Basel Mission Book of Tract Depository.}

 \bibitem{chap1-key04} \enginline{Caldwell, R. (1956) \textit{A Comparative Grammar of the Dravidian or South Indian Family of Languages (Reprinted).} Madras: Madras University.}

 \bibitem{chap1-key05} \enginline{Ilakkuvanar, S. (1963) \textit{Tolkappiyam in English with Critical Studies.} Madurai: Kural Neri Publishing House.}

 \bibitem{chap1-key06} \enginline{Keith, A.B. (1920) \textit{A History of Sanskrit Literature} (Reprint). London: Oxford University Press.}

 \bibitem{chap1-key07} \enginline{Meenakshi, K. (1997) \textit{Tolkappiyam and Astadhyayi.} Madras: International Institute of Tamil studies.}

 \bibitem{chap1-key08} \enginline{\textbf{Meenakshisundram, T.P. \& Shanmugam Pillai, M. (1951) “The Portuguese influence revealed by Tamil words”. \textit{Journal of the Annamalai University}, vol. 16.}}

 \bibitem{chap1-key09} \enginline{Subrahmanya Sastri, P.S. (1924) \textit{History of Grammatical Theories in Tamil}. Madras: The Kuppuswami Sastri Research Institute.}

 \bibitem{chap1-key10} \enginline{Vaidhianathan, S. (1991) \textit{Indo – Aryan Loan words in Old Tamil.} Madras.}

 \bibitem{chap1-key11} \enginline{Winternitz, M. (1963) \textit{History of Indian Literature vol. III. Translated from German in to English by Subhadra Jha.} Delhi: Motilal Banarasidas.}

 \end{thebibliography}

