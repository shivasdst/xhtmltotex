
\chapter{Introduction -\break No Racism in Vedic Tradition}\label{intro}

\Authorline{Sharda Narayanan}


\section*{1 Migrations in the Ancient World}

Since the dawn of civilization it appears that there were many pockets of human habitation, interspersed with vast tracts of dense forest in India. In the days when people were still nomadic, it is easy to imagine that different groups moving in different directions would cover great distances within a few generations so that by the time they found advantage in a more settled life with cultivated fields and livestock, settlements may have come up quite independently in many places at around the same time. Human migration\index{migrations} in and out of the Indian subcontinent seems to go far back to prehistoric times and was a gradual but continuous process that did not merit special historical observation in those times.

On the western side of the subcontinent, the current understanding accepted is that the Harappan civilization may be called the cradle of culture on the Saraswati river\index{Sarasvati river} basin (Danino 2010). In the Mohenjo-daro\index{Mohenjo-daro} and Harappan\index{Harappan} settlements we have found well-developed urban organization, coins, statues and pictures on clay pots of Paśupati and Nandī corresponding to the Ṛgvedic civilization. In the archeological evidence we have, fire altars are also said to be found. Considering that most fire altars for Vedic yajña were freshly constructed according to various measurements required for specific ritual, they were more temporary in nature and it is no wonder that large permanent structures have not been found.

\newpage

The excavations at Adichanallur\index{Adichanallur} in Tamil Nadu date from 1000 BC to 500 BC, far earlier to the Sangam era. Items similar to those at Adichanallur such as Vel, gold pieces and head bands used in Muruga worship have been found at Cypress, Palestine and Gaza, which have been assigned a date of 1200 to 2000 BCE. So, as K. K. Pillay\index{Pillay, K. K.} writes, we do know that trade relations existed between Tamil Nadu and Asia Minor even in those early times. In fact, tools found at Attirampakkam (near Chennai), a site in Tamil Nadu, have been luminescence dated to be 385,000 years old. (\url{ https://www.nature.com/articles/nature25444})

\vskip 2pt

Existing models of Ancient migration out of Africa and into India are questionable to say the least. Modern findings like those at Atirampakkam suggest alternate directions for migration and the spread of humans.\endnote{This finding is actually totally uncontested and disproves any migration story into South India - \url{https://www.google.com/search?q=attirampakkam&rlz=1C1AOHY_enIN722IN728&oq=at9rampakkam&aqs=chrome.1.69i57j0l5.3871j0j7&sourceid=chrome&ie=UTF-8}}

\vskip 2pt

On the eastern side, it is well-known that India and South East Asia had close ties in racial, linguistic and cultural aspects since prehistoric times. These associations and meeting of different cultures must have occurred through migration and transportation of people over land or sea routes. In fact, the sea was more an accessible highway than a barrier to Andamanese islanders who had canoes for dugong fishing (Mahapatra 2003:1).

\vskip 2pt

\begin{myquote}
Further, Coedes seeks to differentiate the colonization of America by the Europeans from the influx of Indians in Southeast Asia, as “in this part of the world the newcomers (Indians) were not strangers, discovering new lands. At some time... the sporadic influx of traders and immigrants became a steady flow that resulted in the founding of Indian kingdoms practicing the arts, customs and religions of India and using Sanskrit as their sacred language.” Coedes conceives of this process of Indianization of Further India as “the continuation, overseas of a “Brahmanization” that had its earliest focus in Northwestern India... and in fact, the most ancient Sanskrit inscriptions of Further India are not much later than the first Sanskrit inscriptions of India itself.” (Mahapatra 2003: 4)
\end{myquote}


\section*{2 Demography of Ancient India}

We can see that in its earliest history, India has seen an active and continuous mixing of different groups of peoples. We do not have any record of when these migrations took place in Tamil Nadu. None of the Sangam\index{Sangam} works indicate when the Aryan or Vedic people entered Tamiḻaham or who formed the first groups.

\begin{myquote}
“Tolkappiyam\index{Tolkappiyam}, the ancient Tamil Grammar, precedes the Ettuttogai and the Pattuppattu. It contains no traces of Jainism and Buddhism and hence it might have been composed either in the 4th century B.C. or prior to it.” (Pillay 1979: 191)
\end{myquote}

The earliest Tamil literary works are from the Sangam period; the word “Sangam” itself derives from the Sanskrit word referring to the period of sangam-s or vihāra-s, i.e. the scholarly academies. Pāņini\index{Panini@Pāņini}, who lived in the 6th century BCE does not mention the kingdoms of the south; Kalinga is the farthest that he mentions. On the other hand, Kātyāyana\index{Katyayana@Kātyāyana} specifies the Chola\index{Chola} kingdom. This is the earliest reference to the extreme south. K.K.Pillay writes that from the Brāhmī inscriptions from all over the country we understand that in all probability the Buddhists and Jains\index{Jain} came into South India in the 4th century BCE (Pillay 1979: 175). We also know from history that both Mahāvīra and Buddha were born in aristocratic families of the kṣatriya class of Vedic society.

It is an undeniable fact that Vedic thought has been absorbed into the Tamil region from the Sangam period itself. It forms a harmonious component of the local ethos and can in no way be viewed as a foreign import or even from North India. The Vedic and Purāņic deities both find prominence in the Sangam works. Indra, festivals held in his honour and Mount Meru are all mentioned. Mahāviṣņu is mentioned as the supreme god in Paripāḍal and Tolkāppiyam. All the other gods, the sun and moon, the \textit{asura}-s and the fire elements are said to arise from Him and He is said to recline on Ādiśeṣa. Even the \textit{avatāra}-s\break are mentioned in the Sangam classics, as also the Ramāyaņa\index{Ramayana@Ramāyaņa} and Mahābhārata\index{Mahabharata@Mahābhārata}.

There is much reason to believe that a more or less similar culture existed all over India even in Vedic times, allowing for regional variations. Modern writers, lead by Western scholarship, tend to view “Vedic” and “Dravidian” cultures in opposition, but in reality, India had many systems of thought developing in parallel: Sāńkhya, Vaiśeṣika, Bauddha, Jaina, Śaiva, Śākta etc are only a few noteworthy ones among the myriad schools of thought that have survived down to our times. A large number of smaller schools of thought merged into the major ones as time wore on but evidence of their mutual influences is undeniable. Study of philosophical texts such as Nyāyamañjarī or Ślokavārtika can amply illustrate this fact. So even accepting that “Vedic” indicates a specific school of thought or culture, the non-Vedic cultures were many; therefore the word “Dravidian” to indicate the non-Vedic is a misnomer used by many writers.

\begin{myquote}
“Those Vedic Gods, the etymology of whose names is not patent and who have no analogies in other Indo-Germanic dialects, must have been originally Dravidian deities.” (Pillay\index{Pillay, K.K.} 1979: 191)
\end{myquote}

Under the influence of Jainism and Buddhism, animal sacrifice in temples and yajña became less prevalent in Tamil Nadu. Although these religions were very widely followed, they were still so only in the urban centres; the peasant populations continued with their traditional, local forms of worship. Buddhism and Jainism held sway for many centuries, Kanchipuram\index{Kanchipuram} being a great centre of learning, producing such stalwarts as Dinnāga and Dharmapāla. But both faiths declined after some time for several reasons. Jainism is said to have very stringent and austere standards that were difficult for the common man to follow and the Jainas were also not fanatical about converting others into their faith. Buddhism is said to have declined due to excessive corruption, immoral behavior and elaborate ritualism in Mahāyāna practices that earned them disfavor among the masses. Also, when the Brāhmaņical religion lost popularity, it underwent reformation so that the people, who did not reject Vedic thought but only its ritualism, came back into its fold (Chaurasia\index{Chaurasia, Radhe Shyam} 2008: 65).

\vskip 2pt

\begin{myquote}
“.. It was during the time of the Pallavas that the revival of Sanskrit and Hinduism took place. Many of the Pallava inscriptions are bilingual, in Sanskrit and Tamil……the Pallava King, Mahendravarman was himself a Sanskrit scholar and poet.” (Sundaram\index{Sundaram, C.S.} 1999:1)
\end{myquote}

\vskip 2pt

\begin{myquote}
“It can be said in conclusion that Buddhism was not a new religion but just a reform movement within Hinduism. According to Dr. V.A. Smith: “Buddha may have justly been regarded as having been originally a Hindu reformer. He did not give any holy scripture to his disciples nor did he condemn any fundamental principles of Brahmanism. He himself was a kshatriya prince and inspired by the Upanishads. He only gave his teachings in the common language of the people.”
\end{myquote}

\vskip 2pt

\begin{myquote}
“\textbf{Reforms in Hindu religion}. The people left Brahmanism and accepted Buddhism not because they had lost faith in the basic principles of Brahmanism but because they condemned outward rituals and ceremonies. When the Brahmanas openly saw revolt against them they reformed their shortcomings with the result that the people came back into their fold again.” (Chaurasia 2008: 63-70)
\end{myquote}

\vskip 2pt

\begin{myquote}
“....Buddhism flourished in Tamil Nadu with unabated vigour for several centuries and as a result this land has produced important and valuable treatises in Sanskrit and Pali. Kancipuram, the great cultural centre of Tamil Nadu had long association with Buddhism.” (Sundaram 1999: 1)
\end{myquote}

\vskip 2pt

Southern India was considered a region of great scholarship right from very ancient times. Kātyāyana, the Vārtikakāra who wrote addendums on Pāņini’s grammar, was well-known as a Southerner. Patanjali, who wrote the Yoga sūtra-s was also a South Indian, associated with worship of Lord Naṭarāja\index{Nataraja@Naṭarāja} at Tillai (Cidambaram\index{Cidambaram}) (Natarajan 1994: 145). There is an interesting account of the history of Sanskrit grammar that forms the concluding portion of the Vākyakāņḍa of Bhartrhari’s\index{Bhartrhari} Vākyapadīya\index{Vakyapadiya@Vākyapadīya} (V.P. II. 496), dating from about 450 CE It says that Vyākaraņa (grammar) studies suffered near obliteration, even texts were lost, and found in book form only in South India, was thereafter resurrected. This indicates that in the dim period of history in the early centuries, Sanskrit scholarship had greatly advanced in the South and in fact helped preserve some aspects of the tradition when it suffered decline in the North.

\vskip 2pt


\section*{3 Varņa\index{varna@varņa} \& Caste\index{caste}}

It is generally given to believe by modern Indology, that the Vedic culture was racist, upholding the superior lineage of Brahmins, rigidly enforcing the caste\index{caste} divide in society. Under the influence of the Aryan Invasion Theory\index{Aryan Invasion Theory}, the so-called “upper castes”, corresponding to Brahmins, are blamed for propagating social hierarchy based on birth, which is cited as the sole reason for all of India’s woes. (But even these votaries of the Vedas’ foreign origins do not say that the caste system was a foreign idea!) We shall examine if careful analysis does indeed point to racism in the texts.

We now look at how social amalgamations took place in Vedic period. One was through migrations which accelerated after the Saraswati River\index{Sarasvati river} began to dry up. The other was through battles for material gain and plunder. Shortly after people began living in well-established villages it appears that battles between tribes broke out for material gain and territorial rights. It appears that people of interior regions had cattle and gold while the Ṛgvedic people had chariots and horses. 

When these groups amalgamated either through war or through peaceful means, we would expect that socially similar strata would combine. That is, the priests or higher social categories would join the ranks of the priestly class in the new hierarchy and labourers would join the bottom level. But this was clearly not the case. Often the priests and warriors of the Vedic people were reduced to labourers and the priests of the so-called Dravidian tribes actually retained their status as the Brahmans. There seems to be complete absorption of the conqueror and the conquered in these Vedic battles. Some of the conquered chiefs obtained positions of power in the new society, participating in religious ceremonies and giving gifts to the Brahmin priests. They were well-regarded as eminent donors in society and some were said to join Indra\index{Indra}, the King of the Vedic gods (Sharma 1958). These cultural confluences have been conceived and promoted as racial rivalry by colonial historians; unfortunately majority of the public, especially among Indians, believe these false theories that have no evidence to support them, to be true.

Another frequent accusation on the Brahmin is that the Veda-s censure those who do not give gifts freely to the priests, but these seem to be directed more towards individuals within their own society, so as to prevent gross disparity in wealth, which would lead to jealousy and hatred. The wealthy man was goaded to share his riches through feasts and gifts to the community – the priests constituted only a small fraction of the beneficiaries.

\begin{myquote}
“Of the passages which describe the Panis as niggards and condemn illiberal people in general, some may have been inspired by greedy priests eager for gifts, but on the whole they seem to reflect the tendency among certain Aryans to accumulate wealth at the cost of their fellow tribesman, who naturally expected some share in the acquisitions through sacrifices made to Indra and other gods, thus providing frequent occasions for the common feasts of the community. Failure to check the process was bound to give rise to economic and social inequalities.” (Sharma\index{Sharma, Ram Sharan} 1958: 20)
\end{myquote}

\begin{myquote}
“In essence, the Rigvedic society was characterized by the absence of sharp class divisions amongst its members, a feature which is found usually in early tribal societies. It was possible to have different ranks but not social classes.” (Sharma 1958: 10)
\end{myquote}

Western Indologists use the references to the Aryan’s conquest of Dasyus, extolled in the Vedic literature, as evidence of a racial strife, but that is based on misinterpretation. Manogna Sastry and Megh Kalyanasundaram have made a detailed analysis of this problem from the academic, literary and linguistic standpoints (Sastry and Kalyanasundaram 2019). While the word “Ārya”\index{Arya@Ārya} refers to a society that aspires to a Vedic way of life, “Dasyu”\index{dasyu} refers to those individuals who cause anarchy, disruption and are retrograde to social and moral progress. The word derives from “\textit{das}” or “\textit{tas}” \textit{dhātu} which in Panini’s grammar refers to “cause harm.” If there were a racial connotation, it would have surely been explained. It does not even refer to a clan or family group but only those with evil characteristics; the word is \textit{guņa-vācaka}, not \textit{jāti-vācaka}, as Sāyaņācārya\index{Sayanacarya@Sāyaņācārya} clearly explains in his notes on the Ṛg Veda.

The verb root “\textit{tas}” and “\textit{das upakṣaye}” from \textit{divādi sūtra} are derived according to Pāņini\index{Panini@Pāņini} grammar as \textit{tasyati} and \textit{dasyati,} to mean “to harass” or “to harm.” “Dasyu” is “one who causes harm in society.” Consider the following examples from the Rgveda.

From the Vaiśvānara Sūkta: \textit{Vaiśvānaro dasyuṃ Agniḥ jaghanvān | I. 59. 6} On this Sayanacharya comments, \textit{“Dasyu = Rasānāṃ karmaņāṃ vā upakṣayitāraṃ; jaghanvān = hatavān.”}

\textit{Rākṣasānāṃ vikroṣantyaḥ yasya rāṣṭrāt hriyante dasyubhiḥprajāḥ | II. 12. 10} Here also, \textit{dasyu} is explained as one who causes harm to others in civic society.

The Dharmaśāstra-s\index{Dharma Sastra@\textit{Dharma Śāstra}} may be considered the authority in interpreting these issues of law and criminality. Here too, we find \textit{dasyu} explained as those with criminal misdemeanor. E.g. \textit{Dasyor hantā sajanāsa Indraḥ}|(Manu Smrti VII.143). The explanation given is \textit{Upakṣayituḥ śatroḥ hantā ghātakaḥ} | Evidently there is no racial connotation in the instances where the word “\textit{dasyu}” is used in Sanskrit texts and is misinterpreted as such by modern or western Indologists either deliberately or by a failure to understand correctly.

From Purāņic accounts of the ancient history of the people of India in different parts, it is clear that there was no racial connotation. We have no account of any exotic group bringing the Vedas into this land. A society that followed a cultured administrative structure, with due regard for \textit{dharma}, was considered “Āryan” and literature endorses a constant endeavor of society to attain a more refined way of life by keeping the baser tendencies of human nature under check.

All the inhabitants of this ancient land are said to be native to the country as far back as human memory goes. There is no whiff of any migration from outside or a “superior” race of people having founded the Vedic civilization. On the contrary, we find in Purāņic accounts that clans descended from lowly people actually became noble aristocratic rulers by their superior conduct whereas some tribes descended from so-called “upper caste\index{caste}” such as Brahmans were actually relegated to the position of asura-s and rākṣasa-s by their base behavior.

David\index{Frawley, David} Frawley makes careful study of the Purāņic lore that gives an account of the five main clans from which the ancient Indian kingdoms sprang, to understand where the ideas on caste could have begun and concludes that the varņa system had no basis in race.

\begin{myquote}
“Hence three of the original five Vedic peoples had Asuric blood in them through their mother. Puru, whose group ultimately predominated, had Asuric blood, whereas the Yadus, who were most criticized in Vedic and Puranic literature, had no Asuric blood but rather that of the Brahmins. In this story we see that both groups of people – thought by the Aryan invasion theory to be the invading Aryans and the indigenous people – had the same religion and ancestry.
\end{myquote}

\begin{myquote}
These five peoples were styled either Arya or Dasyu, which means something like good or bad, holy or unholy according to their behavior. These designations can shift quickly. The descendant of an Aryan king can be called Dasyu\index{dasyu} or its equivalent (Rakshasa, Dasa, Asura, etc.) if their behavior changes.” (David Frawley (2001): \textit{The Myth of Aryan Invasion of India}, p 21)
\end{myquote}

We have much evidence that inter-marriage was common. There is no mention of Brahmin-s or Kṣatriya-s being racially different to others in society, in the Purāņa-s\index{Purana@Purāṇa}, Veda-s, Śāstra-s or poetic literature. King Harsha’s biographer Bāņa Bhaṭṭa\index{Bana Bhatta@Bāņa Bhaṭṭa} in the 8th century CE speaks of a large number of groups differing in habits and persuasion living harmoniously in society, supported and protected equally by the ruler (Kane 1918: 28).

Many societies around the world had a class hierarchy where some were born privileged and others less so; it is unreasonable to regard it as a peculiar Vedic problem. Ancient Indian society should be studied in relation to other ancient cultures in a corresponding epoch and not by modern societal norms. It appears that the Indian system was perhaps more assertive of the privileges and welfare of the weaker sections who were better off than most of their counterparts in other parts of the world. Megasthenes is said to have reported that there were no slaves in India (Sharma 1958) which, perhaps an exaggeration, still indicates that slaves were treated humanely. The Dharmaśāstra-s\index{Dharma Sastra@\textit{Dharma Śāstra}}\break speak of the circumstances wherein a citizen (from any of the four varņa-s\index{varna@varņa}) would enter slavery, the obligations of the master towards his welfare, and the terms under which he may gain freedom. Slavery\index{slavery} appears to result more from economic duress of the individual, not capture of foreign populations after war, unlike in the western world from Afghanistan to Turkey to Europe.

\begin{myquote}
“…In other words, wealthy people might be considered as good kṣatriyas and brahmaņas. If enterprising individuals from the lower classes rose to the throne on a wave of reaction against the ruling class, or on account of their growing wealth, the brāhmaņical ideologues were prudent enough assimilate ……………Much has been made of the Roman virtue of maintaining the basic social structure by admitting into the fold of the ruling class the leading members from the underprivileged classes and keeping out the rest. The virtue, it would seem, was cultivated in no small measure by the ruling class of ancient India. (Sharma\index{Sharma, Ram Sharan} 1959: 237)\endnote{“From post-Mauryan times onwards the way for the exaltation of the rich foreign rulers or the wealthy members of the lower castes to higher social states may easily have been paved by the importance of wealth existing in the consciousness of the people. It was stated in the Pañcatantra3 that it is wealth which makes a person powerful or learned. In other words, wealthy people might be considered as good kṣatriyas and brahmaņas. If enterprising individuals from the lower classes rose to the throne on a wave of reaction against the ruling class, or on account of their growing wealth, the brāhmaņical ideologues were prudent enough assimilate them to the kṣatriya caste by recasting the genealogical legends and thus causing the least dislocation in the existing social system. The process is going on even in recent times4. Much has been made of the Roman virtue of maintaining the basic social structure by admitting into the fold of the ruling class the leading members from the underprivileged classes and keeping out the rest. The virtue, it would seem, was cultivated in no small measure by the ruling class of ancient India. (Sharma 1959: 237)}
\end{myquote}

In India we know that the Islamic community has many divisions according to the origin of the groups. Perhaps not much is known about the importance given to ethnicity and racial stock by the rulers and trend-setting aristocracy of India after the Islamic invasions, which may have influenced Hindu society as a whole.

\begin{myquote}
Though the division among Muslims appear to be too simplistic to outsiders, more so due to the religious overtone of equalitarianism, the actual situation reveals a most complex set-up neither resembling the Hindu caste system nor the western class pattern, yet drawing clear boundaries through diversified ethnicity and differing religious ideologies (Rizvi and Roy 1984).\endnote{Class system in medieval Islamic Persia

Egalitarian and meritocratic ideas became, during the first centuries of Islam, the ideology of protest movements among underprivileged groups contending for political power, especially the aristocratic Persian Šo`ūbīya, the democratic Kharijites, and the radical Shi`ite (ḡolāt) movements. Later, however, egalitarian ideas lost much of their revolutionary potential and were incorporated gradually into the ethics of the established order, according to which egalitarianism, asceticism, and consideration for the poor were admired at the same time that hierarchy and inequality were endorsed in society.

Proponents of inequality in medieval Islam:

Advocates of inequality viewed social differentiation as inevitable or even desirable because they considered it necessary for the survival of society as a whole (functional model), because of the unequal distribution of inherited intelligence (biohereditary perspective), or because of inherited or acquired wealth.

The biohereditary perspective. The most common explanation of social inequality in medieval Islam was that nobility was inherited. In pre-Islamic Persia, members of dominant strata, no matter what social functions they performed, were considered men of rank with noble status; they boasted about the value of their hereditary rank and lineage and were forbidden to associate with commoners (Nāma-ye Tansar, pp. 57-59, 64-65, 69-70). Among the Arabs “knowledge of the common descent of certain groups [and] the glory of a tribe,” as Ignaz Goldziher noted (I, P. 45), stood “at the center of Arab social consciousness.” The term ḥasab connoted the accumulation of the famous deeds of one’s ancestors, the ancestors themselves, and one’s own contributions. These ideas and practices were common during the Islamic era…

\url{http://www.iranicaonline.org/articles/class-system-iv Encyclopaedia Iranica}}
\end{myquote}

This aspect of hierarchy in society by accident of birth originated in ancient times in most parts of the old world, as attested by \textit{Encyclopaedia Iranica} (online) which speaks of ancient Persian society, a cradle of civilization in the Middle East.

\begin{myquote}
Class system in medieval Islamic Persia-\index{Persia} Proponents of inequality in medieval Islam:\endnote{Though the division among Muslims appear to be too simplistic to outsiders, more so due to the religious overtone of equalitarianism, the actual situation reveals a most complex set-up neither resembling the Hindu caste\index{caste} system nor the western class pattern, yet drawing clear boundaries through diversified ethnicity and differing religious ideologies (Rizvi and Roy 1984). The Muslims in India do not belong to a single ethnic or cultural group. The Arabs, Afghans, Persians and Turks came to this country one after another, and those Muslims who claim foreign descent trace their origin from these ethnic groups.

http://shodhganga.inflibnet.ac.in/bitstream/10603/84507/5/05\_chapter\%202.pdf}
\end{myquote}

\begin{myquote}
Advocates of inequality viewed social differentiation as inevitable or even desirable because they considered it necessary for the survival of society as a whole (functional model), because of the unequal distribution of inherited intelligence (biohereditary perspective), or because of inherited or acquired wealth.
\end{myquote}

\vskip 2pt

While the original Tamil society did not have the four varņa-s\index{varna@varņa}\break that Vedic culture speaks of, works of the Sangam\index{Sangam} period do mention around ten different groups of people based on occupation and region within South India where they hailed from. Groups called Kuruvar, Vellalar, Maravar, Pradavar etc are mentioned in Puranānuru. Brahmins and their high position is also mentioned. Gradually these groups became endogamous.

\vskip 2pt

\begin{myquote}
“In this connection, it is well worth noticing the occurrence of terms like ‘Melor’, ‘Uyarndor’ and ‘Arivar’ which occur in Tolkappiyam\index{Tolkappiyam}, the celebrated grammar. The term ‘Melor” seems to have specified all persons of high character……Purananuru(183) show that ‘Melor’ or men of character could be members of the higher castes.” (Sharma 1958: 177)\endnote{“In this connection, it is well worth noticing the occurrence of terms like ‘Melor’, ‘Uyarndor’ and ‘Arivar’ which occur in Tolkappiyam, the celebrated grammar. The term ‘Melor” seems to have specified all persons of high character. From Karpiyal 3, Tolkappiyam, it would appear that it included the first three classes under this designation. There is a slight difference in the denotation of the term as interpreted by the commentators of Tolkappiyam. Ilampuranar interpreted ‘Melor’ as the devas or celestial beings, Nachchinarkkaniyar provides a very wide interpretation to the term. He states that the norms of conduct prescribed for Vanigar or traders in respect of earning wealth isapplicable to Brahmins (antanar\index{antanar}), arasar (kings) and all those comprised under Velalar. According to him, therefore ‘Melor’ denoted those members who followed a high standard of conduct. If that were so, it is a notably democratic conception. References in Tolkappiyam (Tol Karpiyal 3) and Purananuru (183) show that ‘Melor’ or men of character could be members of the higher castes.” (Sharma 1958: 177)}
\end{myquote}

\vskip 2pt

\begin{myquote}
“As regards the ‘Arivar’ the interpretations suggested are illuminating. Tolkappiyar does not identify Parppar (Brahmin) exclusively with Arivar.......Thus ‘Arivar’ in the orginal sense used by Tolkappiyar, applied to learned men among the people. No exclusive reference to class or community is implied by the term. This suggests that a certain measure of fluidity existed in the in the caste\index{caste} system in respect of the Arivar.” (Sharma: 178)
\end{myquote}

\vskip 2pt

\begin{myquote}
“Caste is the development of thousands of years, from the association of many different social and other groups in a single cultural system….early literature paid scant attention to it, but it is certain that caste did not originate from the four varnas” (Basham 1963: 145)
\end{myquote}

\vskip 2pt

The caste issue is a much-maligned issue – let us look at what made it continue for so long. In earlier times, perhaps till recently, the economy being unstable and unpredictable, the caste provided social security\endnote{Even today Jāti is the primary means of social capital. Research of economic clusters in India over past few decades is indicative of this. \url{https://www.epw.in/journal/2014/10/perspectives/caste-social-capital.html } Ground breaking work by Prof. Kanagasabapathy ( a keynote speaker at the SI 3 conference), R. Vaidyanathan and their team reiterates the “knowledge on the street” about socio-economic factors. It s the social groups and dynamic fludiity in availibility of capital and structures of trust that are critical in India’s unique style of capitalism. \url{https://www.amazon.in/Indian-Models-Economy-Business-Management/dp/812033888Xhttps://www.amazon.in/India-Uninc-Prof-R-Vaidyanathan/dp/9383260564/https://www.amazon.in/Caste-as-Social-Capital-Vaidyanathan/dp/9388689119/}} after the joint family, helping destitute members and providing for widows and orphans (Basham 1963: 149). It was also the first umbrella outside the immediate family for activities connected with the occupation and livelihood – training in skills, procuring economic opportunities, production of goods, marketing and investment in capital. Caste was more significant as community, defined by life-style which was governed by occupation. In fact, endogamy was preferred as way of keeping property and wealth within the community. With constant migration and entry of new groups in every occupation, there was much competition for economic activity and livelihood.

\vskip 2pt

To this day, caste\index{caste} politics revolve around economic rivalry and the Brahmin, who is much removed from all this, is conveniently blamed! In this regard, Tamil Nadu seems to have been even more caste-conscious than Vedic society. The four varņa-s and the different castes have never been truly harmonized in ideology. Somehow after British rule began in India, British scholars and administrators have deftly laid the blame on Brahmin and Vedic outlook.


\section*{4 Brāhmaņical Oppression in Modern Writing}

\vskip 2pt

The trend of modern studies by westerners and even many Indians, is to articulate rather inordinately, on the oppression wreaked by Brahmins on the rest of society. We examine if this is justified.

\vskip 2pt

In medieval Europe, we know that Kings were generally illiterate and signed their orders with seals and signet rings. It is well known that the Roman Catholic Church\index{Church} was very powerful in Europe, with the Pope interfering in all political and economic issues as well. Only after the invention of printing press did common people became aware of events at various levels and this increase of information ultimately lead to the Reformation followed by Counter-Reformation of the Church. We know that England turned to Protestant religion coming out of the clutches of the Pope when King Henry VIII wanted to divorce or behead several of his wives. Perhaps all this background influences the English-speaking scholar into reading too much ambition and greed in the Brahmin priesthood.

\vskip 2pt

In our country on the other hand, from the earliest times, the kings and administrators were expected to be well-educated. The priests and rulers knew what their expectations and limits were. The priesthood constituted a minority in a population consisting of farmers, soldiers, craftsmen, artisans, traders, administrators, musicians, performers and rulers. Brahmin priests do not go out to conquer new lands and convert the people, but rather follow other groups in a natural search for livelihood. We do not come across a single war in ancient India waged to propagate religious belief; they were always fought for military supremacy and material gain or for defending one’s faith.

The Vedic society speaks of varṇa based on occupation. The Nāṭya"sāstra may be considered as having come down to us from the early centuries BCE (Subrahmanyam 1997: 35) and echoes the Vedic social order. In a society where the community functioned as training institutute, workshop and trade market and occupations were most efficiently transmitted in a hereditary manner, the performing arts have from their inception been explicitly open to all people. This stems from the understanding that creativity and artistic ability are available to all people. With greater imagination of the practical realities of ancient society, modern researchers can understand the policies of ancient India better and need not always read discrimination and oppression into every aspect.

Traditionally, Brahmins always represented the teachers of society, preserving ancient wisdom for future generations. That they were held in high esteem, regarded as the intellectual capital of the country and spared the atrocities of war only indicate the high position Indian tradition placed on the Veda-s and learning in general. Modern writing that censure the role of Brahmins does much harm, throwing the baby out with the bath water. Westerners and foreigners will not easily tell us this, for they have little to gain from it. On the other hand, they have much to gain by our internal bickering. Within the country also, our leaders practice vote-bank politics to the hilt with little integrity guiding their ideology. Their only view seems to be towards what they have to gain, at the expense of the people and the country. As we can see, caste politics play a huge role, even leaving the Brahmins completely out of the picture. To distract attention from the real issues and avoid punishment for those who are responsible for these social and economic crimes, the Brahmins are blamed for their so-called ideology by politicians and academicians who are influenced by those in power. Unfortunately, with weak governance and internal bickering, we as a society become victims to foreign manipulation, losing our own identity and glorious heritage.


\section*{5 The Holistic Appeal of Vedic thought}

Modern criticism of the spread of Brāhmaņical (Vedic) religion seem to ignore its merits and fail to grasp the deep appeal it has to people. In its essence the Upaniṣadic philosophy teaches against superstition and speaks of every individual being’s connection to the cosmos. Its philosophical or religious hierarchy does not centre about any particular race, geographical location or individual prophet. It teaches the equality of all living creatures and advocates respect for God’s creation. Its ethics and morals are based on universal principles such as honesty, justice, selflessness and detachment from vices. It advocates the importance of ethical behaviour as the basis of the \textit{Karma}\index{karma@\textit{karma}} theory.

At a practical level it advocates sacrifice to the gods as an expression of prayer and thanksgiving. Such religious rituals were also social and cultural in nature, as is normal in any wholesome society. Over the centuries, beginning in Vedic times, devotion or \textit{bhakti} and temple worship have also found a firm place alongside Vedic rituals in the practice of religion. The colourful stories of the Purāņa-s\index{Purana@Purāṇa} serve to illustrate many Vedic and Āgamic ideals in an interesting manner to the common people. The Indian ideal of simple living and high thinking is very attractive to those of refined thinking. 

What is stated by L.K. Mahapatra\index{Mahapatra, L.K.} in the context of spread of Vedic culture to Indonesia may be considered relevant within the country too –

\begin{myquote}
“(Coedes 1968: 24) The Indians brought the native chiefs not only a complete administration but an administrative technique capable of being adapted to new conditions in foreign countries”. He credits the Brahmana scholars and priests with the crucial role in adapting the Indian traditions and norms to the local indigenous cultural base.” (Mahapatra 2003: 4)
\end{myquote}

It is no exaggeration to observe that for any system of thought to stand the test of time, its tenets have to be brought in synchronization with Vedic thought. In the history of India this has happened time and again. In today's world, the appropriation of traditional Indian concepts such as Yoga, cosmology and monism by Western faiths and philosophy is the digestion that Rajiv Malhotra speaks about (Malhotra 2011: 223). Evangelism in India leaves no stone unturned in adopting Hindu practices and Vedic concepts under garb of cultural assimilation.


\section*{6 Swadeshi Indology}

While the Western Indologists may require methods based on a framework familiar to themselves in order to understand a foreign culture, we need to understand traditional methods in order to gain Swadeshi perspective which can lead to deeper insights. In today’s scenario in Western academia, Philology has reached such high levels of sophistication that modern studies are far removed from relevant interpretation of Indian classical texts, rather exposing the shortcomings and unfamiliarity of the research scholar with Indian culture. It has become essential for honest scholarship to examine and rectify the methods of modern Indological studies.

Rajiv Malhotra\index{Malhotra, Rajiv} explains why Swadeshi Indology is important and speaks of 5 waves of Indology in his lecture at IGNCA, New Delhi (video available online). They are as follow:

\begin{enumerate}[{\rm 1)}]
\itemsep=0pt
\item Marxism which is all about Leftist ideas and how they should be adopted at the expense of all else.

 \item Post-colonialism is a technique to understand the mind of the colonizer according to the colonizer himself to understand events in history.

 \item Subalternism – where to champion the cause of the underclass, the downtrodden and the so-called oppressed people is fashionable; history is viewed from their perspective so that everybody looks like a loser. 

 \item Post-modernism - every aspect of Indian life is deconstructed and labeled with disdain using western cultural constructs. The original meanings are deliberately distorted and mangled to suit the Western hegemonic agenda and reduced to political oppressor – oppressed, master – slave and similar extreme barriers. India is studied as an oppressor country.

 \item Neo – Orientalism uses our own traditional source books to malign us.

\end{enumerate}

The papers presented in this volume display careful analysis of the issues relevant to Tamil Nadu today, as, meticulous academic research should lead public discourse in a progressive society. In India, the public at large and the youth in particular, are often unaware of the facts of many controversial issues; much conflict and wasted energy ensues in public discourse, leading to further disharmony and dissatisfaction, expressed through the media and in private conversation. The importance of liberal-minded and progressive people equipping themselves with the right information cannot be over-emphasized.

Papers in Tamil

It is evident that Tamil culture cannot be viewed as separate from Indian culture although it has distinct features that give it its uniqueness. To regard it as discrete, in isolation is unreal perception. Discretization and polarization of culture are modern trends driven by agenda, which harm both cultures, individually and collectively and bring the citizens no good. Dr Sankaresvari’s paper makes a comparative study of Tolkāppiyam and Aṣṭādhyāyī, the two landmark grammars of Tamil and Sanskrit respectively. While Tolkāppiyam retains the unique features of Tamil influences, it also imbibes some of the features of Aṣṭādhyāyī that are worth emulating. This picture is fairly representative of Tamil culture in the backdrop of pan-Indian history.

Chitra Rao’s paper explores the descriptions of marriage ceremonies, bridal jewellery and attire for married women in texts ranging from Tolkāppiyam, Akanānūru, Pattupāṭṭu and other works of the Sangam era to later works such as Silappadikāram and Ainkural to understand the popular South Indian tradition of married women wearing the mańgalasūtra (tāli), the sacred thread around the neck that is a symbol of marriage. Some people hold that this tradition, followed in several states in India, did not exist in Tamil Nadu until the Aryan influences came, while others hold that it is indeed an ancient (perhaps original) Tamil tradition.

We find that a sacred thread being tied around the neck as part of sacred ceremonies sanctifying marriage are found in many parts of India, ranging from Kashmir in the North to Kerala and Tamil Nadu in the South; in many cases these are used for a brief duration during the days of ceremonies and later discarded. But in the Southern states of Maharashtra, Karnataka, Tamil Nadu, Andhra Pradesh, Telangana and Kerala, the practice seems to have stronger presence, where the cotton thread is replaced by an ornament of gold and beads to form a more permanent ornament. Contrary to popular understanding, men also wear external marks of their married status, originating in the wedding ceremony itself but this is not so apparent in Indian society today where modern dress is adopted in normal life more by men than by women.

Papers in English

Tirunāvukkarasar\index{Tirunavukkaracar@Tirunāvukkaracar} (Appar\index{Appar}) sings that Lord Śiva became both Sanskrit and Tamil – \textit{āriyankaņṭāy tamiḻaņkanṭāy}, 6.46.10; He became the three varieties of Tamil and four Veda-s -- \textit{muttamiḻum nānmaraiyumanankaņ\break ṭāy,} 6.23.9. Tirumūlar also says that Lord Śiva imparted Sanskrit and Tamil at once. \textit{āriyamum tami}ḻum uṭane colli (Verse 65). Saraswati Sainath establishes the harmony between Śaiva Siddhānta\index{Saiva Siddhanta@Śaiva Siddhānta} and Vedic thought by showcasing the many references to Lord Śiva as a Vedic god and upholding Vedic traditions in the Tevāram-s of Appar. There is no doubt that the Tevāram\index{Tevaram@Tevāram} reflect the faith of the Tamil people from ancient times. A keen scholar of classical Tamil and Sanskrit, she has examined hundreds of verses to reveal how much is in harmony with the Veda-s.

Proselytization takes many routes; in addition to emotional, social and monetary exploitation, a problem most people are not aware of is cultural appropriation of India’s ancient traditions. Our laws on these issues are neither adequate nor clear perhaps because our founding fathers never envisaged a problem of this magnitude; most people assume that any protest is a result of narrow-mindedness or partisan outlook. It is essential that the facts of the matter are understood, as these are usually glossed over by the media, perhaps due to larger problems taking precedence. While India remains as secular, tolerant, friendly and peace-loving as ever, in today’s scenario there is a real possibility of the country’s demography being altered due to foreign investments on this front. The fact that India does not have a Uniform Civil Code in its Constitution for all its citizens in this modern age should cause great concern among liberal-minded people.

A.V. Gopalakrishnan’s paper explains systematically how the bedrock of Tamil culture, Tirukkural\index{Tirukkural}, has been portrayed variously as a Christian or Jaina\index{Jain} work in order to put down indigenous culture and make conversion attractive to the common people, mostly unlettered, ostensibly by misleading them to believe that they are not breaking away from the faith of their forefathers.

The next two papers discuss linguistics and the parallel development of Sanskrit and Tamil. Even when Pāņini\index{Panini@Pāņini} composed his grammar in the 5th or 6th century BCE, from the many references to earlier grammarians, regional variations and preferences in verbs and optional forms for many words, it is evident that classical Sanskrit was a well-established language for at least several hundred years prior to it. Language is the true window into the working of the mind – the structure of language is the structure of thought process, the very consciousness, which Bhartrhari\index{Bhartrhari} aptly termed, “\textit{śabdatattva.}”

V. Yamuna Devi studies the structural similarities and differences between Sanskrit and Tamil and makes a comparison of the two works on grammar, \textit{Aṣṭādhyāyī}\index{Astadhyayi@\textit{Aṣṭādhyāyī}} and \textit{Tolkāppiya}m.\index{Tolkappiya@\textit{Tolkāppiya}} T.N. Sudarshan and T.N. Madhusudan present a comprehensive account of Comparative Linguistics, developed by Western methodology to study Sanskrit in the backdrop of its relation to other languages. They explain the biases and fallacies of Western linguistics, which ignore the holistic analysis of language in the Indic tradition. Why is it, they ask, that Pāņini, Patañjali and other Indian grammarians do not even find mention in the World Atlas of Language Structure?

The Aryan Invasion Theory\index{Aryan Invasion Theory} and the Aryan Migration Theory are fairly well-known and yet not so well understood. Most people such as politicians, journalists and even academicians often regard these dubious theories as well accepted facts while the truth is that they are now obsolete notions. The next three papers explain the Aryan-Dravidian issue lucidly to the reader, from three different perspectives. Shivshankar Sastry presents the compulsions for propagating such a theory on the part of European scholars, to suit their requirements and agenda. Nilesh Oak presents the evidence summoned for the AIT/AMT from historical, archeological, linguistic and geological standpoint and shows that it is scant, proven wrong and therefore it is high time the notion was discarded. Subhodeep Mukhopadhyaya presents a lucid response to the issue by three saints – Swami Vivekananda\index{Vivekananda, Swami}, Sri Aurobindo\index{Aurobindo, Sri} and Sri Sri Ravi Shankar\index{Ravi Shankar, Sri Sri} who counter the invasion theory with the wisdom of our traditional learning and scriptural evidence.

In today’s media of news networks and published literature, many conflicting views regarding interpretation of ancient and sacred texts are apparent, leaving the average reader quite dismayed. Sensationalism and novelty in interpretation are highly regarded, but novelty does not always imply artistic creativity – true genius is in originality, not distortion of acclaimed works of artistic endeavor, be it in painting, sculpture or literature. In this confusing scenario, Parthasarathy Desikan presents a systematic analysis of modern interpretations of the Rāmāyaņa.

In most nations of the world, history and cultural studies of the country form a part of the mainstream education of its citizens in schools and colleges while India continues for the most part to follow the curriculum pattern set by the erstwhile British rulers. And yet, traditional learning in Sanskrit studies, classical dance and music continue in the living tradition of guru-śiṣya-paramparā that the formal education system may not know of. India’s performing arts traditions are witnessing a huge revival, with more and more experts undertaking research and resurrecting waning art forms.

Much work is yet to be done, as the arts are not fully understood in their full glory in different epochs as yet. For the past few hundred years, Sanskrit scholars were not usually acquainted with the fine arts and artistes were not familiar with Sanskrit, causing a rift between theory and practice. Dancer and writer Dr. Padmaja Venkatesh says that one of the effects of Sanskrit receding to the background was that the dancers lost in-depth understanding even though the art was rooted in the Āgama-s\index{Agama@Āgama} (Venkatesh 2018). Now the gap is being bridged, with more and more dancers and musicians delving into the treatises and interpreting the theory as only practitioners of the arts can.

As part of modernization, classical music and dance are often used for purposes other than what they were designed for, sometimes inadvertently and sometimes deliberately by vested interests. Senior dancer and scholar Prakruti Prativadi who has been teaching Bharatanāṭyam\index{Bharatanatyam@Bharatanāṭyaṃ} for many years in the U.S., explains this issue clearly with the authority of the original treatise on the subject, the \textit{Nāṭyaśāstra}\index{Natyasastra@\textit{Nāṭyaśāstra}}.

A brief table of commonly used words with different form in Tamil and Sanskrit with English equivalent is given below.

\newpage

\begin{longtable}{|p{2.5cm}|p{2.5cm}|p{3cm}|}
\hline
Tamil & Sanskrit & English Equivalents Commonly Used \\
\hline
Akattiyar, Agattiyar & Agastya & Agasthya \\
\hline
Akanānūru &  & Ahananuru, Agananuru \\
\hline
Āḻvār &  & Alvar, Azhvar \\
\hline
āriyar & ārya & arya \\
\hline
Cambandar, Campantar & Sambandhar & Sambandar \\
\hline
Cańkam & Sańgam & Sangam \\
\hline
Cāttira & Śāstra & Sastra \\
\hline
Cekkiḻār &  & Sekkilar, Sekkizhar \\
\hline
Cilappatikāram & Śilapadhikāram & Silapadikaram \\
\hline
Civan & Śiva & Siva, Shiva \\
\hline
Coḻa & Cola & Cola, Chola, Cozha \\
\hline
Cuntarar & Sundara & Sundarar \\
\hline
Cuppiramaņiya & Subrahmaņya & Subrahmanya \\
\hline
Cuvāmināta & Swāminātha & Swaminatha \\
\hline
drāviḍa & drāviḍa & dravida, dravidian \\
\hline
Kandan & Skanda & Skanda \\
\hline
Kaḻagam &  & kazhagam \\
\hline
Kaņņaki &  & Kannagi \\
\hline
Kāppiyam & Kāvya &  \\
\hline
Mantiram & mantra & mantra \\
\hline
Māņikkavācagar & Māņikyavācaka & Manikkavasagar, Manickavacagar \\
\hline
Māyoņ & Viṣņu, Kŗṣņa & Vishnu, Krishna \\
\hline
Māl, Tirumāl & Viṣņu, Kŗṣņa & Vishnu, Krishna \\
\hline
Marai & Veda &  \\
\hline
Murukan & (Kumara) & Muruga, Murugan \\
\hline
Nānmarai & Four Vedas &  \\
\hline
Parapakka & Pūrvapakṣa\index{Purvapaksa@Pūrvapakṣa} & Purvapaksha \\
\hline
Pārpannar & Brāhmaņa & Brahmin \\
\hline
Pūņūl, Pūņal & yajñopavīta & sacred thread \\
\hline
Siththāntham & Siddhānta & Siddhanta \\
\hline
Silappatikāram &  & Silapadikaram \\
\hline
Sūttiram & Sūtra & Sutra \\
\hline
Tamiḻ & Tamil & Tamizh, Tamil \\
\hline
Tamiḻakam &  & Tamizhagam, Tamilaham \\
\hline
Tēvāram & Deva-aram & Tevaram, Devaram \\
\hline
Tiru, Thiru & Śrī & Tiru, Thiru \\
\hline
Tirumoḻi &  & Tirumozhi, Thirumoli \\
\hline
Tiruñānacampan\-tar & Tirujñānasam\-bandhar & Tirugnanasambandar, Gnanasambandar \\
\hline
Tiruppāvai &  & Tirupavai, Thirupavai \\
\hline
Tīṭcitar & Dīkṣita & Dikshita, Dixita \\
\hline
Tottiram & Stotra & Stotra \\
\hline
Vaittiyanathan & Vaidyanāthan &  \\
\hline
Vaņigar & Vaņij & merchant \\
\hline
Veda-cāram & Veda-sāram  &  \\
\hline
Velvi & yajña &  \\
\hline
\end{longtable}


\section*{Bibliography}

\begin{thebibliography}{99}
\bibitem{00e-key01} Basham, A.L.(1963) \textit{The Wonder that was India}. Bombay: Orient Longmans Limited.

 \bibitem{00e-key02} Chaurasia, Radhe Shyam (2008) \textit{History of Ancient India: Earliest Times to 1200 A.D}. New Delhi: Atlantic Publishers and Distributors (P) Ltd.

 \bibitem{00e-key03} Danino, Michel (2010) \textit{The Lost River: On the Trails of Sarasvati}. Penguin.

 \bibitem{00e-key04} Kane, P.V.(Ed.)(1918) \textit{The Harshacarita of Bāņa Bhaṭṭa}. Delhi: Motilal Banarsidass.

 \bibitem{00e-key05} Mahapatra, L.K.(2003) \textit{State, Society and Religion}. Chennai: Emerald Publishers.

 \bibitem{00e-key06} Malhotra, Rajiv (2011) \textit{Being Different}. Noida: Harper Collins.

 \bibitem{00e-key07} Natarajan, B. (1994) \textit{Tillai and Nataraja}. Madras: Mudgala Trust.

 \bibitem{00e-key08} Pillay, K.K.\index{Pillay, K.K.} (1979) \textit{Historical Heritage of the Tamils}. Chennai: MJP Publishers.

 \bibitem{00e-key09} Sastry, Manogna \& Kalyanasundaram, Megh (2019) “The A of ABC of Indian chronology: Dimensions of the Aryan problem revisited in 2017”.In Tilak, Shrinivas \& Narayanan, Sharda (Ed.) \textit{Studies in Tamil Civilization: Land of Dharma}. Chennai: Infinity Foundation India.

 \bibitem{00e-key10} Sharma, Ram Sharan (1958) \textit{Sudras in Ancient India}. Delhi: Motilal Banarsidass.

 \bibitem{00e-key11} Subrahmanyam, Padma (1997) \textit{Nāya”śāsṭra and National Unity}. Tripunithura: Sri Ramavarma Sanskrit College.

 \bibitem{00e-key12} Sharma, Ram Sharan(1959)\textit{Aspects of Political Ideas and Institutions in Ancient India}. Delhi: Motilal Banarsidass.

 \bibitem{00e-key13} Sundaram, C.S.\index{Sundaram, C.S.} (1999) \textit{Contribution of Tamil Nadu to Sanskrit}. Chennai: Institute of Asian Studies.

 \bibitem{00e-key14} Frawley, David\index{Frawley, David} (2001) Myth of Aryan Invasion in India. \url{https://www.scribd.com/document/8489882/Myth-of-Aryan-Invasion-in-India-David-Frawley} accessed on Dec 31, 2018.

 \bibitem{00e-key15} \url{https://en.wikipedia.org/wiki/Sarasvati_River} accessed on Dec 31, 2018.

 \bibitem{00e-key16} \url{http://www.academia.edu/23253475/The_Sarasvati_River_-_an_educational_module
 } accessed on Dec 31, 2018.

 \bibitem{00e-key17} \url{https://www.pgurus.com/rajiv-malhotras-lecture-on-swadeshi-indology-at-ignca-new-delhi-part-1/
 }

 \bibitem{00e-key18} \url{https://www.pgurus.com/rajiv-malhotras-lecture-on-swadeshi-indology-at-ignca-new-delhi-part-2-colonization/}

 \bibitem{00e-key19} \url{https://www.speakingtree.in/.blog/why-swadeshi-indology-is-required-part-2/m-lite}

 \bibitem{00e-key20} Venkatesh, Padmaja(2018) “Dance evolves on a regular basis.” The HINDU. April 14, 2018. Friday Review.

 \bibitem{00e-key21} \url{https://www.thehindu.com/entertainment/dance/dance-evolves-on-a-regular-basis/article23681405.ecec}

 \bibitem{00e-key22} \url{https://www.indusscrolls.com/its-high-time-we-restore-the -original-stature-to-the-temple-dance-forms/}

 \bibitem{00e-key23} \url{http://www.iranicaonline.org/articles/class-system-iv Encyclopaedia Iranica} accessed on Feb 27, 2019.

 \bibitem{00e-key24} \url{http://shodhganga.inflibnet.ac.in/bitstream/10603/84507/5/05_chapter%202.pdf} accessed on Feb 27, 2019.

 \bibitem{00e-key25} \url{https://www.nature.com/articles/nature25444}

 \end{thebibliography}

\vskip 5pt

\theendnotes

