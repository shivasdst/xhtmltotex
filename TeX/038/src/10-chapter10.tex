
\chapter{\textit{Rāmāyaņa} and \textit{Mahābhārata:} Interaction with Tamil Literary and Cultural Traditions through the Ages}\label{intro}

\Authorline{Parthasarathy Desikan}


\section*{Abstract}

The geography of Bhārata took care long ago to see that several Prakrits should appear and develop among \textit{Bhāratīya-s} north of the Vindhyas while a few Dravidian languages should similarly take root in the south. Sanskrit, refined from its earlier Vedic form, was eagerly learnt all over Bhārata\textit{,} and its grammar-basics helped to create grammatical rules for major languages in the regions, whenever they were ready. Tamil, one of the most ancient languages of the land, refers to Vedic practices in its oldest literature. The introduction of the two \textit{itihāsa}-s well-known in Sanskrit helped in the healthy growth of \textit{Viṣṇu bhakti} in the Tamil region, and with Tamil participation in the creation of healthy adaptations in both Tamil and Sanskrit, also in the rest of Bhārat. These adaptations have been time tested. Recognition of Rāma and Kṛṣṇa as \textit{avatāra-s} of \textit{Viṣṇu} became an essential prerequisite. The universal love that the two \textit{avatāra-s} could inspire in people of their time, and in all of us as \textit{mūrtis} in temples and as \textit{itihāsic} persona, their availability as guiding force on earth and as ultimate refuge, define them for the average \textit{Bhāratīya}. This paper considers this phenomenon as a test for defining adaptations of the epics as stories of Rāma and Kṛṣṇa. A few examples of such adaptations and commentaries from western Indologists and similar-minded \textit{Bhāratīya-s} are examined in some detail to show which of them fail to meet the specification and in what manner. It is hoped that the analysis would help both in discouraging the future generation from such writing and protecting \textit{Bhāratīya-s} from the negative fallouts from what is already available.


\section*{Introduction}

Some western Indologists and similar thinkers in India have a tendency to access literary spaces which we consider sacred and interpret them inappropriately. This was efficiently and critically examined in the previous meeting of this conference (Conference Report, Kannan 2017) and the prolific and not too responsible writings of one western Indologist were duly taken apart by more than one participant. The availability of recent casual retellings of our \textit{itihāsa-s} and \textit{purāņa-s} on the one hand, and the highly motivated agenda of western Indology on the other, which lead to such misinterpretations, need not at all surprise us. The work of Rajiv Malhotra (\textit{The Battle for Sanskrit}), the Swadeshi Indology deliberations and efforts of other ‘intellectual \textit{kṣatriyas’} must continue for quite a while to keep the invasion in check and weaken it. This paper first looks at a related but agreeably different phenomenon - our sacred spaces that has been working for at least two millennia, starting with the telling effects of the introduction of the two \textit{itihāsa-s,} namely \textit{Rāmāyaṇa} and \textit{Mahābhārata} in ancient Tamilnadu, all other regions of Bhārata and later beyond its frontiers too. The epics in their own form and their Tamil adaptations that did not compromise the original purpose have indeed been received as blissful gifts to the life and psyche of the Tamilians and in turn, all \textit{Bhāratīya-s. Bhakti} in general and specifically \textit{Viṣṇu Bhakti} had blossomed throughout \textit{Bhārata}, helped along by the simultaneous use of Sanskrit, Prakrits and Tamil over a very long time. High proficiency in Sanskrit among Tamil scholars right through the centuries, dwindling only in recent times, was responsible for the appearance of good Tamil adaptations, which brought out the best present in the originals, while also adding some regional value. Excellent adaptations were also made in Sanskrit which emphasized the divinity in the characters of Rāma and Kṛṣṇa. The stories of \textit{Rāmāvatāra and Kṛṣṇāvatāra} spread facilely throughout the Tamil land and were perpetuated in the \textit{sthalapurāṇa-s} of the magnificent temples that dot its landscape. The main \textit{mūrti-s} of many of these temples are Rāma or Kṛṣṇa, and festivities built round episodes of the epics have kept the devotees happy. Śaivism and Vaiṣṇavism took some time to learn to grow in stature together rather than in competition. The appearance of Sanskrit and Tamil works of local origin, including the \textit{Bhāgavatapurāṇa} and the \textit{Adhyātma Rāmāyaņa} helped the process along. A \textit{Sanātanic} specification for any adaptation of either \textit{itihāsa}, which would seek to be recognized as another \textit{Rāmāyaṇa} or \textit{Kṛṣṇacarita} thus came into being. As against this, other adaptations of the epics as well as commentaries have appeared more recently, some of which could be appreciated as interesting cultural and literary variations and read or studied for pleasure and research. Many however, which did not reinforce the recognition of the two divine \textit{avatāra-s} are unlikely to interest lay \textit{Bhāratīya-s} including Tamils, who always remember the devotion of Hanumān to Rāma, the surrender of Sugrīva and Vibhīṣaņa to Rāma, the love that the simple \textit{Gopī-s} felt for Kṛṣṇa and the inspiring messages of Kṛṣṇa’s \textit{Gīta} through Arjuna to all the world.


\section*{The Pūrvapakṣa}

All Rāma stories are \textit{Rāmāyaṇa}. All Kṛṣṇa stories are to be regarded as equal to Kṛṣṇa stories in \textit{Mahābhārata or the Bhāgavatapurāṇa.}

This assertion is examined in the light of the eager assimilation of the \textit{itihāsa-s} in Tamil regions well before the Common Era, and the adaptation of their basic stories through Tamil and Sanskrit works carrying the same message to the people.

\subsection*{How the region got ready}

\begin{itemize}
\item The region originally known as \textit{Tamizhagam} has shrunk in size in recent centuries ago due to the flowering of the related Malayalam language between the 12th and 16th centuries in the \textit{Cera} region west of the Western Ghats (Santhosh 2014) and more recently when some revision of state borders with neighbours was negotiated in free India. The northern boundary of the Tamil region in ancient India was the \textit{Venkaṭa} hill, according to the Sangam era Tamil Grammar treatise \textit{Tolkāppiyam} (Danino 2009). \textit{Panambāranār} in his prefatory verse to this treatise, refers to Tamilnadu as “the fine land lying between Venkaṭa in the north and Kumari in the South, where Tamil is spoken.” (Raghavan 2016: 231). The hill temple is mentioned as part of Tamil land in \textit{Śilappadikāram} (Dikshitar 1939: 173; Desikan 2007: 94-95). Bounded by hills and seas, the Tamil land could preserve Tamil language from undergoing significant changes over the years. The Eastern and Western Ghats created different Dravidian language cells, while Prākritic languages somewhat resembling one another developed north of the Vindhyas. However, pilgrimage traditions and thirst for education kept \textit{Bhāratīya-s} travelling \textit{Ā-Setu-Himācala} (from Rameswaram to the Himalayas) frequently, resulting in regular interchange of literature and culture. ‘Sanskrit and Tamil, two of the oldest languages of India’ had indeed ‘grown together since the dawn of historical times (Raghavan 2016: 231).’ Archaeological evidence including recent finds at Adichanallur (Subramanian: 2005) and in Kizhaḍi (The Hindu Net Desk 2017) indicates that Brāhmī script had arrived among the Tamils well before the Common Era. Prof V. Raghavan (index Raghavan, V.) also states that the earliest inscriptions of Tamilnadu are in Brāhmī script with an admixture of Prakrit words, the Devanāgari script for Sanskrit having come in later. The pattern of epigraphy of Tamilnadu, the scholar noted, ‘became settled into bilingual texts comprising a poetic historical part in Sanskrit and a prose Tamil part giving the details of the endowments.’ The use of \textit{maṇipravāla} (mixture of Sanskrit and Tamil expressions) (Govindaraju and Srirangaraj 2009: 126) has continued in Tamil Vaiṣṇavite ritualistic writing until now, with the \textit{granthalipi} serving along with Tamil (\textit{vattezhuthu} and modern\textit{) lipi} to write both languages.

\end{itemize}


\subsection*{Effect on Sangam literature}

\begin{itemize}
\item When did the \textit{itihāsa-s} arrive among the Tamils? We cannot say, because the earliest literary works in Tamil available to us are only those of the Sangam era (2 to 3 centuries before and after the start of the Common Era).We learn from the \textit{paripāḍal} verses in \textit{puranānūru} and the epic \textit{Śilappadikāram} of that era that recitation of Vedas and study of Sanskrit \textit{itihāsa-s} and \textit{purāņa-s}\break were already prevalent (index Danino, Michel) (Danino 2009). Prof. R. Nagaswamy (index Nagaswamy, R.) (Nagaswamy 1980: 23) noted that there are more references in Sangam literature to Vedic practices and \textit{itihāsa-purāṇa-s} than to temples. Prof. V. Raghavan(index Raghavan, V.) observed that \textit{Puranānūru} has an invocation song from “\textit{Perumdevanār}, who sang the \textit{Bhārata} in Tamil” and mentions a second \textit{Perumdevanār}, who also wrote a \textit{Bhārata} version (Raghavan 2016: 243).

\end{itemize}


\subsection*{The devotional works and the pilgrimage of the Āļvār-s}

\begin{itemize}
\item The same knowledge of the \textit{itihāsa-s} and \textit{purāņas} also inspired the \textit{Ā}ļv\textit{ār-s}, the foremost devotees in the Tamil Vaishnavite tradition, (between the 6th and 9th centuries CE) to travel throughout the country and compose some 4000 hymns to \textit{Viṣṇu} enshrined in 106 temples, which got named \textit{divyadeśa-s} in consequence (Srivatsan, 1984, Part 2). Āļvār\textit{-s} added the celestial \textit{Śrivaikuntha} and \textit{Kśīrasāgara} mentioned in the \textit{purāṇa-s} to the 106 \textit{kṣetra-s} on earth, mostly in South India. They included a select few in the north, such as Ayodhya, Dwaraka, Mathura, Brindāvan, and Badrināth, nearly all associated with \textit{Rāmāvatāra} and \textit{Kṛṣṇāvatāra}. The poem-collection is known as \textit{Divyaprabandham} (Srivatsan 1984, Parts 1\&2). In some Southern shrines, the Viṣṇu icon was named after Rāma\textit{ or} Kṛṣṇa. Thus, you have \textit{Valvili Rāma} in Pullambūdanguḍi, \textit{Kolavilli Rāma} in Vellianguḍi\textit{, Ayodhyā Rāma} posing as one of the eleven Rudra-s at Tiruvaṇpurushottamam, and Rāma reclining on \textit{darbha} grass at Tiruppullāni. Rāmeswaram where Rāma and Vānara\textit{-s} are believed to have offered worship to Śiva (\textit{Rāmanāthaswāmi),} is nearby. You also have Vijayarāghava at Tirupputkuzhi and \textit{Vīrarāghava} at Tiruevvulūr, popularly known as Tiruvallūr, apart from Rāma icons in the famous temples at Vaḍuvūr and Madurāntakam. Kṛṣṇa is encountered as Govindarāja both in Tirupati and in Chidambaram, as Rājagopala at Mannārgudi and Kṛṣṇa in other \textit{Gopala} forms in the five \textit{Kṛṣṇa-kṣetra-s}\break at Tirukkannanguḍi, Tirukkannapuram, Tirukkannamangai, Kapisthalam and Tirukkovalūr. Kṛṣṇa as \textit{Gītācārya} with the names Venkaṭakṛṣṇa\textit{ and} Pārthasārathi graciously offers \textit{sevā} to us at Tiruvallikkēṇi. The Āļvār-\textit{s} often chose to sing about Rāma and Kṛṣṇa in the Viṣṇu shrines they visited\textit{.} An Āļvār for instance becomes Rāma’s mother and sings a lullaby for him and another becomes a Gopi complaining to Yashoda about the pranks of the child Kṛṣṇa. Yet others become Kṛṣṇa’s lovelorn \textit{Gopī} playmates and pine for his company. The lone lady among the \textit{Āļvār-s}, Āṇḍāḷ\textit{,} falls in love with \textit{Bhagav}ān at Srivilliputtūr and merges with him at the Śriraṅgam shrine. Her \textit{Tiruppāvai} verses are recited by the devout early in the mornings of the days of the \textit{Mārgazhi} month year after year. In these sweet verses, some \textit{Gopī-s} who practise a special \textit{vrata} go around Gokula and wake up others in their group and finally meet Kṛṣṇa\textit{.} They forget their original intention of borrowing a musical instrument from him for their ritual and instead surrender before him and plead that he should be with them in all their future births as well. Assuming \textit{Nāyakībhāva,} the male Āļvār Tirumangai Āļvār too pines for the Lord in several verses.

 \item Vaishnavite pilgrims make it a point to visit as many \textit{divyadeśa-s} as possible. In the past two decades, the \textit{Kinchitkāram} trust (\url{https://www.kinchit.org/}) of Velukkuḍi Krishnan,a great exponent of discourses on Vaishnavite subjects has been arranging Rāma\textit{-}specific and Kṛṣṇa\textit{-} specific \textit{yātra-}s, and tours also to places visited by \textit{Ācārya Rāmānuja.}

\end{itemize}


\subsection*{The effect on the rulers and the Ācāryas}

\begin{itemize}
\item The \textit{Divyaprabandham} poetry of the Āļvār\textit{-s} inspired the Tamil Kings who ruled over the area (Coļa-s, Cera-s, Pāndya-s, Pallava-\textit{s} and others who followed them) to provide the old shrines with larger and better equipped structures of great artistry, functionality, and beauty, endow them with lands, and encourage scholars to study \textit{sthalapurāna-s}, create \textit{āgamas} for worship and provide for daily and seasonal rituals and festivals with public participation. Endowments are also known to have been made during and after the reign of the \textit{Pallava-s} for the recital and exposition of the epics as well as purāņas in temples. All Tamil Vaishnavite \textit{Ācārya-s} were \textit{ubhayavedānti-s} in that they were well versed in Vedic \textit{samskāra-s} and religious Sanskrit literature apart from knowing Tamil \textit{Ā}ļv\textit{ār} literature thoroughly. Starting from Nāthamuni, Yāmunācārya and Rāmānuja, all of them wrote in Sanskrit and Tamil, and their writing included verses on Rāma, Kṛṣṇa and Viṣṇu’s other manifestations.

\end{itemize}


\subsection*{The Tamil adaptations of the epics}

\begin{itemize}
\item \textit{Kavi Cakravarti Kamban} is the famed author of \textit{Kambarāmāyaṇam}, (called by him \textit{Rāmāvatāram}, spelt in Tamil with a default I in front), containing well over 10,000 beautifully crafted verses in the \textit{virutthappā} style of metre. He was a court poet during the reign of Kulottunga Coļa the first, which was in the late 11th century and the beginning of the twelfth. Kamban’s preface (\textit{Rāmāvatāram, pāyiram}) acknowledges the Sanskrit original to \textit{Vālmīki (tavan}) and to three rishis including Vālmīki (moovar) (Gnanasambandar and Gnanasundaram (2002). He has depended also on references in \textit{Divyaprabandham} (the collected works of \textit{Āļvār-s,} which contain over 4,000 verses) and some of his own embellishments to advantage. \textit{Saṭhakoparant}ā\textit{di}, in praise of the famous Āļvār, Nammāļvār is reputed to be \textit{Kamban}’s work. Kamban’s conviction that Rāma was an \textit{avatāra} of \textit{Viṣṇu} can be inferred from his title for his epic poem. This paper will not speak about the great literary merit of the \textit{Kamba Rāmāyaņam}. This Tamil epic is of course avidly researched by many Tamil scholars in India and abroad. But we are concerned with the fact that the story conveyed by it has firmly reinforced the lay Tamilian’s devotion to Rāma as an incarnation of Lord \textit{Viṣṇu}. The Wikipedia entry on \textit{Rāmāvatāram} notes, “This epic is read by many Hindus during prayers. In some households, the entire epic is read once during the Tamil calendar's month of \textit{Āḑi.} It is also read in Hindu Temples and other religious associations. On many occasions, Kamban talks about surrendering to Rāma, who is a manifestation of Viṣṇu himself.

 \item Villiputtūrār, who had the privilege of writing the most famous Tamil adaptation of \textit{Mahābhārata} wrote only 10 out of the 18 \textit{parva-}s of the epic, feeling satisfied with ending his book with the end of the \textit{Kurukṣetra} war and the \textit{Pāṇḍava} brothers getting back to \textit{Hastināpura}. This excellent rendering, also having over 10,000 verses, seems to have been made in the 14th century. Other than providing elaborate narration of the epic \textit{Kurukṣetra} war and the destiny of the \textit{Pāṇḍava-s} and the \textit{Kaurava-s}, this Tamil version of the \textit{Mahābhārata} comprises of much devotional and philosophical content and also poetic beauty. An elegant poem in \textit{Udyoga Parva} for instance describes the fluttering of the flags on the ramparts of Kṛṣṇa’s palace, signalling a big ‘No’ to Duryodhana visiting Dwaraka to seek Kṛṣṇa’s help during the war, as if to make it clear that there is no hope for him. Villi Bh\textit{ā}ratam has also inspired the composition, by the great modern Tamil poet Subrahmanya Bhārati\textit{,} of a long poem on Draupadi’s vow, which truly maps the admiration that Tamils feel for the heroine of the epic (\textit{Bhārati} Poems 1997). Villages all over Tamilnadu have shrines for Draupadi\textit{,} and annual festivals are held in honour of this brave heroine of the Pāṇḍava princes. Bhārati too, like the Āļvār\textit{-s} makes us fall in love with Kaṅṅan (Kṛṣṇa\textit{)} in other poems, where we see Kṛṣṇa in many roles, as a male lover, then as a beloved girl and again as a lovable servant.

 \item There have been other scholarly adaptations and retellings, both large and small, of both the \textit{itihāsa-s} in Tamil by several authors including India’s first and only Indian Governor General, Shri C. Rajagopalachari. Literal translations in Tamil prose meant for \textit{pārāyaṇa} purposes are also available. Books have been written about any given episode, \textit{upakathā} or portion of the two epics too. For instance, there are books on \textit{Sītā Svayamvaram, Rukminī Kalyāṇam} or \textit{Naḷacaritra}, \textit{Rāmapādukāprabhāva}, or \textit{Sabarī’s} devotion, \textit{Hariscandra}, \textit{Subhadrā Kalyānam}, or \textit{Aṅgulīyapradānam}.

\end{itemize}


\subsection*{Bhāṣya-s of the Bhagavadgītā}

\begin{itemize}
\item Following the traditions of Sanātanic schools of thought, both Śankara and Rāmānuja, who founded \textit{advaita} and \textit{viśi}ṣ\textit{tādvaita} philosophies, wrote \textit{bhāṣya-s} in Sanskrit and Tamil, of the \textit{Bhagavadgītā}, which is considered the very essence of the Vedas, and is revered as a scripture by all Hindus. The two \textit{bhāṣya-s} are the sourcebooks for religious and philosophical discourses all over Tamilnadu. \textit{Śankara} in his philosophical poem, \textit{Bhaja Govindam}, calls upon all mortals to meditate on \textit{Govinda}, and recite the \textit{Gīta} and the \textit{Viṣṇu Sahasran}ā\textit{ma}. All Hindus remember \textit{Mahābhārata} more for these two philosophical and devotional gems occurring in it just before and after the account of the Kurukṣetra war, than for the entire description of the war.

\end{itemize}


\subsection*{How the stories reach the masses}

\begin{itemize}
\item The \textit{Bhagavadgītā} is discoursed upon every day in Tamilnadu. Not a day passes in Tamil cities like Chennai, Madurai or Tiruchi without religious discourses, Harikatha or Villuppattu, and the subject is more often \textit{Rāma} or \textit{Kṛṣṇa} than any other. Versions of \textit{Rāmāyaṇa} and \textit{Mahābhārata} as now available in print are taken up for holy recitations (\textit{pārāyaṇa}-s) and religious discourses (\textit{Harikathā-s and kālakshepa-s}) with enjoyment by the people of the land. The \textit{Dvaitasampradāya} of Madhwācārya also has a good base in Tamilnadu. Udupi Kṛṣṇa and Mantr\textit{āl}aya Rāma are favourite deities for many Tamils. The Hare \textit{Rāma} Hare \textit{Kṛṣṇa} based \textit{mahāmantra} from the \textit{Kalisantaranopaniṣad} which has caught the imagination of the \textit{Hare Kṛṣṇa} Movement has been used in \textit{bhajans} in Tamil Nadu from very distant times. The ISKCON temples in Tamil Nadu are well attended by locals. A parallel to the ISKCON homage paid to \textit{Bhagavān} Kṛṣṇa’s beloved devotee-consort Rādhā is the much older Tamil practice of \textit{Rādhākalyāṇa Mahotsavam}, a celebration of the marriage of Rādhā with Kṛṣṇa (which did not take place in the \textit{purāņa}-s) during which middle aged and old men take roles of Rādhā and other Gopis and dance along with a devotee taking the role of Kṛṣṇa! The Tamils enjoy \textit{bhajans} and have taken to \textit{Satsang-s} organized by worshippers of Shirdi Sai and Puttaparthi Sathya Sai with great enthusiasm. Translations of the Marathi \textit{Mahābhaktavijaya} are avidly read by Tamil devotees of \textit{Pāṇduraṅga Viṭhala} in whom they see Kṛṣṇa. The Tamils also continue to name their children with the many names given to Rāma and Kṛṣṇa in the \textit{itihāsa-s} and \textit{purāṇa-s.}

\end{itemize}

\newpage

\subsection*{What the Tamil ethos did to the \textit{Rāmāyaṇa} story}

\begin{itemize}
\item Tamilians were happy that \textit{Kamban} skipped the \textit{Uttara K}āņḍ\textit{a} where the story of banishment of Sītā is described by Vālmīki. The \textit{Viśiṣtādvaitic Ācārya Vedānta Deśika’s Raghuvīragadya} (written in Sanskrit) goes through the \textit{Uttara K}ā\textit{nda} rapidly indicating the poet’s personal agony in writing about the separation of the divine couple. The \textit{Adhyātma Rāmāyaṇa}, also in Sanskrit, in which the divinity of Sītā and Rāma is made explicit, unlike in \textit{Vālmīki Rāmāyaṇa}, is a part of or add-on to\textit{ Brahmāņḍa Purāna}. There is a reference in it to Rameswaram, a Tamil temple site where Śiva is worshipped by Rāma. The author makes Śiva tell the entire story to Pārvati, emphasizing that Rāma is the ultimate \textit{Paraṃ brahma.}He also introduces \textit{Advaitic} philosophical messages to distract himself and the readers from the grief of the banishment episode. These devices could suggest a South Indian author, or at the least a scholar trained by a Tamil \textit{Advaitin.} Though the actual identity of the author of \textit{Adhyātma Rāmāyaṇa} has not yet been determined, I go along with the view held by some experts that the author is \textit{Swāmi Rāmānanda}. This \textit{sant}’s tendency in reconciling the \textit{nirguņa} nature of \textit{paraṃ brahman} with the \textit{saguņa} nature of \textit{Rāma} is well known, as also his knowledge of Tamil language. \textit{Swāmi Tapasyānanda} (2001), who has translated the work into English for Sri Ramakrishna Math at Mylapore, Chennai, says in his Introduction, ‘We have translated only (these) six \textit{K}āņḍ\textit{a}-s, excluding the seventh chapter called \textit{Uttara K}ā\textit{nda}, as it seems to be extraneous in \textit{Rāmāyaṇa} proper’. The \textit{Swāmi} and the Chennai branch of Ramakrishna Mission must have felt the author’s own reservations about the \textit{Uttara K}āņḍ\textit{a}. \textit{Adhyātma Rāmāyaṇa} is believed in turn to have inspired \textit{Tulsi D}ā\textit{s} to write his famous \textit{Rām Charit Mānas}. The \textit{Tulsi} book is brief in its allusion to the banishment of \textit{Sītā} in its 7th\textit{Kāņḍ}a. The \textit{Sant}’s sensitivities can be understood when we learn that his Acārya, Narahari Dās had been one of the ten famous \textit{śiṣya}-s, disciples of \textit{Swāmi Rāmānanda}.

\end{itemize}


\subsection*{Temple practices and a \textit{Sthalapurāṇa}}

\begin{itemize}
\item Rāma telling his \textit{Vānara} counsellors (Valmiki Ramayana 6.18.33), that he offers total protection to anyone who surrenders before him and begs refuge even once, from all living sources of danger and Kṛṣṇa’s promise in the \textit{Bhagavadgīta} (BG 18.66) of ridding a devotee totally of all his sins when he gives up all possessiveness of his \textit{karma} and surrenders to His will, have made deep impressions on Srivaishnavites in Tamilnadu. That Bharata became a regent of Rāma’s sandals for fourteen years and ruled the kingdom on their behalf was given a heavier import through \textit{Vedā}n\textit{ta Desika’s kāvya} of 1008 verses, \textit{Sri Pādukāsahasram}, (Desikan 2010) extolling the merit of Rāma’s \textit{Pādukā-s}, identified by him as \textit{Pādukā-s} worn by \textit{arcāmūrtis} of Viṣṇu everywhere, and especially by \textit{Ranganātha mūrti} at Srirangam. This icon, along with a \textit{vimāna} shaped like the \textit{praṇava} letter \textit{Om}, is believed to have been placed in Srirangam by \textit{Vibhīṣaṇa} during his return to Lanka after Rāma’s coronation. This \textit{mūrti} that had earlier been worshipped by all kings of the \textit{Ikṣvāku} clan had been presented to \textit{Vibhīṣaṇa} by \textit{Rāma}, according to the \textit{sthalapurāṇa} of Srirangam. The reference in the epic to \textit{kuladhanam} obtained by \textit{Vibhīṣaṇa} before he proceeded to Lanka (Valmiki Ramayana - 6.128.90) makes it the primary (if not clear) source for the story. All Viṣṇu temples have a likeness of the \textit{utsavamūrti’s} sandals fixed to a noble metal crown at the top, which goes by the name of ś\textit{aṭhakopam} or ś\textit{aṭhāri}. (the enemy of meanness).This is placed by the \textit{arcaka} either doing \textit{ārādhanā} to the \textit{mūrti} in the shrine, or accompanying the \textit{mūrti} outside the temple during festivals, on the bowed head of every devotee who approaches for Sevā.

\end{itemize}

We shall now look at the nature of some other adaptations, mostly of the Rāma story. It can easily be seen that most of them have been unlike \textit{Rāmāvat}ā\textit{ram}, \textit{Bhāgavatam} or Rām Charit Mānas, in that they do not consider Rāma and Kṛṣṇa as incarnations of Viṣṇu.


\subsection*{\textit{Many Ramāyaņas}}

\begin{itemize}
\item The book, \textit{Many Rāmāyaṇas} edited by Paula Richman (index Richman, Paula) discusses some of them. In its Introduction, which is also the first chapter, Richman reports her finding that the television (Doordarshan) serial on \textit{Rāmāyaṇa} which was aired every Sunday morning from January 1987 in India was extremely popular with a lot of people all over the country watching it with great involvement. Their devotion for the iconic Rāma in temples seemed to have transferred itself to the Rāma on the screens of the individual and shared television sets. “They bathed before watching, garlanded the set like a shrine, and considered the viewing of Rāma to be a religious experience. The size, response, and nature of the television \textit{Rāmāyaṇa}'s audience led Philip Lutgendorf (index Lutgendorf, Philip) a scholar of Hindi \textit{Rāmāyaṇa} traditions, to comment:

\end{itemize}

\begin{myquote}
‘The \textit{Rāmāyaṇa} serial had become the most popular program ever shown on Indian television—and something more: an event, a phenomenon of such proportions that intellectuals and policy makers struggled to come to terms with its significance and long-range import. Never had such a large percentage of South Asia's population been united in a single activity; never before had a single message instantaneously reached so enormous a regional audience.” (Lutgendorf 1995: 127)
\end{myquote}

It would seem therefore, that \textit{Rāmānand Sāgar’s} TV serial can indeed be considered a \textit{Rāmāyaṇa} in the genre of the original \textit{itihāsa} or Kambarāmāyaņa. It had to be, because Sāgar used episodes from Vālmīki, Kamban, Tulsidās and others, whose Rāma stories showed directly or indirectly that Rāma was an \textit{avatāra} of Viṣṇu. The historian Romila Thapar (index Thapar, Romila) was among the few in India, who were not happy with the broadcast and its huge response. She felt that it reflected the concerns only of “the middle class and other aspirants to the same status”. (Thapar 2014) She felt that “the homogenization of any narrative tradition results in cultural loss; other tellings of the \textit{Rāmāyaṇa} story might be irretrievably submerged or marginalized.” Richman, however had decided to consider the televised \textit{Rāmāyaṇa} “not as heralding the demise of other tellings, but as affirming the creation of yet another rendition of the \textit{Rāmāyaṇa}, the latest product of an ongoing process of telling and retelling the story of Rāma.”

\begin{itemize}
\item The second chapter of Richman’s book (index Richman, Paula) is ‘Three hundred \textit{Rāmāyaṇa}-s’ by A K Ramanujan, of retellings in various regions of India and in Southeast Asia. However, after mentioning that there are probably three hundred different \textit{Rāmāyaṇa } retellings in India and abroad, Ramanujan chooses only five different \textit{Rāmāyaṇa-s} for a close look: Vālmīki's Sanskrit poem; Kamban's (I)rāmāvatāram, a Tamil literary account that incorporates characteristically South Indian material; Jain tellings, which provide a non-Hindu perspective on familiar events; a Kannada folktale that reflects preoccupations with sexuality and childbearing; and the \textit{Rāmākien}, produced for a Thai rather than an Indian audience. Ramanujan finds them to be “the expression of an extraordinarily rich set of resources existing, throughout history, both within India and wherever Indian culture took root. Like the set of landscape conventions of classical Tamil poetry, the elements of the \textit{Rāmāyaṇa} tradition have been a source on which poets can draw to produce a potentially infinite series of varied and sometimes contradictory tellings.” Ramanujan likens the \textit{Rāmāyaṇa} tradition “to a pool of signifiers that includes plot, characters, names, geography, incidents, and relations” seeing in each \textit{Rāmāyaṇa} a ‘crystallization’. He expects both constraints and fluidity in the creation of \textit{Rāmāyaṇa}-s and expects one of the factors to be “the beliefs of individual religious communities.”

 \item In another chapter, Reynolds holds that the Buddhist telling is older than Vālmīki’s and that both Hindu traditions and Buddhist values are reflected in the way the Thai \textit{Rāmākien} has been created.

 \item Richman thinks (index Richman, Paula) that she really began her book, owing to her puzzlement about the reaction of the Tamil people to the writings of the founder of \textit{Drāvidakazhagam}, E V Ramasamy, as she says in the preface of her book. For years, she had heard people refer to Ramasamy’s interpretations of the \textit{Rāmāyaṇa} in a mocking and dismissive way. When she analysed his story of Rāma, however, she found much of it strikingly compelling and coherent if viewed in the light of his anti-North Indian ideology. She writes

\end{itemize}

\begin{myquote}
“E.V.R. singled out the \textit{Rāmāyaṇa} to censure. For E.V.R., the \textit{Rāmāyaṇa} story was a thinly disguised historical account of how North Indians, led by Rāma, subjugated South Indians, ruled by Rāvaṇa.” (Richman 1991)
\end{myquote}

Although his ideas were comparatively radical—and potentially disorienting—to a population of devout Hindus, she thinks many people responded enthusiastically. According to her, “his ‘North vs. South’ interpretation of the \textit{Rāmāyaṇa} matched the political context in which E.V.R. was operating.”


\subsection*{Wendy Doniger (index Doniger, Wendy)}

\begin{itemize}
\item In her contribution entitled ‘The enduring \textit{Rāmāyaṇa}’, for a feature examining the cultural geographies of the \textit{Rāmāyaṇa} in an issue of the Geo magazine six years ago (2011), Doniger considers it a shame that Ramanujan’s essay in the Richman book about versions of \textit{Rāmāyaṇa} “in which, Sita, Rama and Lakshmana act in ways that violate the squeaky-clean stereotypes of their characters”, became a subject for legal action from ‘the saffron brigade’, when it was included in the readings at the Delhi University. Earlier in the same contribution, Doniger says, “The living tradition is constantly enriched by written and oral performances, by storytellers from different historical periods and different parts of India, and from different castes and backgrounds and language groups,- from Valmiki to Nina Paley, (index Paley, Nina) via Bhavabhuti, Kamban, Krittibasa, Michael Dutta, and all the village and Bollywood and diaspora storytellers (as Paula Richman has richly documented in Many Rāmāyaņa-s and Questioning Rāmāyaņa-s), --”.

 \item Wendy Doniger gives(index Doniger, Wendy) us a good indication of the kind of Rāmāyaņa that she herself would have written, by her references to \textit{Vālmīki Rāmāyaņa} characters in her well-publicised book, \textit{The Hindus, an} \textit{alternate History}, in which she makes irreverent remarks about several of the main characters. She interprets a remark of Lakṣmaṇa to Rāma to mean that Lakṣmaṇa found his father Daśaratha very lustful. She remarks that Rāma spent a week or more in great sexual indulgence with Sītā\textit{,}just before banishing her to the forest (Doniger 2009: 153).

\end{itemize}


\subsection*{Sheldon Pollock and his School}

\begin{itemize}
\item Pollock’s \textit{Rāmāyaņa}, Book 2, Ayodhya, published in 1986 and 2005 is highly acclaimed. He had published an earlier work in 1984, also based on \textit{Vālmīki Rāmāyaņa}. Through these and other smaller articles, he framed \textit{Rāmāyaņa} as a socially irresponsible work. He considers Rajyābhiṣeka or the rites of crowning a King in ancient India, as described in \textit{Rāmāyaņa}, as a process of divinization of an otherwise normal human king, carried out by Vedic Brahmins to infuse divine qualities of invincibility in him. The Brahmins simultaneously demonize some others whom they consider their enemy, so that the king is expected to challenge them and kill them. It is Pollock’s contention that the same process of branding which decided that Rāvaṇa and other \textit{asura-s} should be killed by Rāma, has been at work with the divinization of Hindu rulers subsequently as well. The Turk invaders of the 12th century and Muslims in general nowadays are considered the other by Hindus according to him, fit to be demonized and then attacked. The kings or the rulers of the Hindu state thus have absolute powers through projection from the \textit{Rāmarājya} concept and are not bound by any constraints in this special \textit{Rājdharma}. He even finds a \textit{śloka} in \textit{Mahābhārata} conferring on the king the absolute rights of dictatorship over the subjects and connects it back to \textit{Vālmīki}, as Rajiv Malhotra points out in his book, “The Battle for Sanskrit”. Pollock also holds that Hindu \textit{dharmaśāstra-s} had all been made in changeless forms and none of them were later contextual additions in the religious life of the Hindus. He insists that all of them had their origin and authority right in the epics. Malhotra (2016) quotes him paraphrasing the \textit{Bhagavadgītā}, 16.23-24 in the following words: “Whoever abandons the injunctive rules of śāstra and proceeds according to his own, will never achieve success, or happiness, or final beatitude. Therefore, let \textit{śāstra} be your guide in deciding what to do and what not to do. Once you determine what \textit{śāstraic} regulation pronounces, you may proceed to action.” After citing many such examples, Pollock concludes that knowledge of every variety from the transcendental to the social (by his inference including even historic) is “by and large viewed as permanently fixed in its dimensions; knowledge, along with the practices that depend on it, does not change or grow, but is frozen for all time in a given set of texts that are continually made available to human beings in whole or in part during the ever repeated cycles of cosmic creation,” He points out that the dominant position in the West is just the opposite.

 \item Chicago Professors Lloyd and Susan Rudolph wrote in 1993, “How did it happen that the Bharatiya Janata Party was able to hijack Hinduism, replacing its diversity, multivocality and generativity with a monotheist Ram cult?” “In time, Ram stories became consolidated”. “In January 1987, an eighteen-month-long serial of the Ramayana based on the (\textit{Rām Carita Mānas}) began airing at 9-30 am prime time, on state run TV”. “Ten months after the \textit{Rāmāyaņa} megaseries, the Vishwa Hindu Parishad (World Hindu Council) called on Hindus throughout India to make holy bricks, inscribed with \textit{Rāma}’s name for use at Ayodhya. There, at the site of \textit{Rāma}’s birth, and on the place of the Babri Masjid, they would build a temple to \textit{Rāma}.” Pollock and his school, Malhotra notes, thus “consider that the \textit{Rāmāyaņa} intrinsically encodes and supports violence.” (Malhotra 2016: 321)

\end{itemize}


\subsection*{The Uttarapakṣa}

The epics reinforced the loving devotion with which ancient Tamils worshipped manifestations of the infinite, which were traditionally pointed out to them by their ancestors. In the \textit{Sangam} epic, \textit{Śilappadikāram} (2007), we notice references to only a few temples of Viṣṇu, none of them yet dedicated to Rāma or Kṛṣṇa. The three main temples referred to are the Venkata hill temple and the temples at Srirangam and Thiruvananthapuram. Anecdotes from the \textit{itihāsa-s, Rāmāyaṇa} and \textit{Mahābhārata}, however, are presented in detail in the Tamil epic, Rāma and Kṛṣṇa being already understood to be \textit{avatāra-s} of Viṣṇu. Within a few centuries however, the temples for Viṣṇu had multiplied and the Āļvār\textit{-s} sang about the \textit{Viṣṇumūrti} enshrined in over a hundred of them. Some of them had Rāma and Kṛṣṇa names, but the Āļvār\textit{-s} saw Rāma and Kṛṣṇa in most of them. For the devotees used to the earlier \textit{avatāra-s} such as \textit{Vāmana-Trivikrama, Varāha and Narasimha}, the two human manifestations of the divine who walked among the people of Bharat all along the Bharat became great favourites. Born in Ayodhya, Rāma crossed the Vindhyas and passed through Tamil land to Lanka. Kṛṣṇa shifted his capital from Mathura to Dwaraka and rode his chariot to Assam to fight \textit{Naraka}. It is possible that the extended South Indian lap in Rāma\textit{’}s sylvan travails was an interpolation. It is similarly possible that most of the well-known Kṛṣṇa stories are based on \textit{Bhāgavata Purāṇa,} which has a South Indian origin, rather than on \textit{Mahābhārata}. But the effect of the closeness of the Infinite in human form depicted in these ancient stories had been magically endearing to the Tamil devotees and their brethren all over \textit{Bhārata}. Therefore, we should recognize only such adaptations and retellings of \textit{Rāmāyaṇa} or \textit{Mahābhārata} or the \textit{purāṇa-s}, or commentaries on them that produce the same or similar effect on followers of our \textit{Dharma},as adaptations of \textit{Rāmāyaṇa} and \textit{Kṛṣṇa-carita}, not any other. The other narratives will of course be read for other purposes.


\subsection*{Many Rāmāyaņa-s}

\begin{itemize}
\item The enthusiasm with which the mega TV serial \textit{Rāmāyaṇa} produced by Ramanand Sagar was watched all over the country in 1987, to which Richman (index Richman, Paula) and Lutgendorf (index Lutgendorf, Philip) refer with appropriate wonder, flowed from the devotion to the person and the story of Rāma, deeply entrenched for centuries in the hearts of all \textit{Bhāratiya-s}, including of course the Tamils. The viewers included some men and women who were following at least some worship-rituals which had been passed on through centuries. It also included a larger majority whom the requirements of modern life had kept busy otherwise, but in whom there was latent devotion to \textit{Bhagavān} Viṣṇu, happily identified with Rāma, thanks to stories heard from grandparents or read and viewed in \textit{citrakathā}-s. The TV serial rekindled the devotion, and the few hours spent weekly approximated to a satisfying worship ritual. Whatever the level of its merit in the eyes of scholars, this serial helped to unite people across the length and breadth of \textit{Bhārata}.

 \item It is not surprising that a historian with leftist leanings did not expect such spontaneity across all sections of the Hindus in welcoming the serial. Looking for a harmful effect in the phenomenon, Thapar (2014) (index Thapar, Romila) finds that this wave could work against public support for versions of the Rāma story that do not see Rāma as an \textit{avatāra}. Richman, who just enjoys great variety in the Rāma story tellings, without traditional or leftist bias, welcomes the TV serial as just another telling.

 \item A. K. Ramanujan’s essay(index Ramanujan, A.K.) (1991) finds a prime spot in Richman’s Many \textit{Rāmāyaṇa}-s with his \textit{Three hundred} \textit{Rāmāyaṇa}s for obvious reasons. Ramanujan selects five tellings and enjoys the variety in them. As a highly knowledgeable linguist, he feels fulfilled also with the variety introduced by different languages, and his essay reflects his enjoyment. It is a pity that there was a temporary ban on its use by the History department of the Delhi University. The reference to Sītā being born to Rāvaṇa, especially through Rāvaṇa becoming pregnant and delivering the child, and subsequently desiring her to marry him seems to have offended the sensitivities of some members of the Academic Council. They did not see that Ramanujan was not the author of that specific narrative, but was abstracting it from a Kannada folk tale claiming to be a \textit{Rāmāyaṇa}. Similar bad taste must have been felt in the reference to a small Buddhist story about a Rāma, prince of Vārāṇasī and son of a king named Daśaratha, who marries his own sibling, named Sītā. The Council members who felt offended could have comforted themselves that this story perhaps followed some foreign traditions, though it had some names common with Vālmīki\textit{’}s epic-personae. They did not. For us, it suffices to note, that the Kannada folk telling and the Buddhist Rāma story were neither talking about divine \textit{avatāra-s}, nor could they be understood as aids to spread \textit{Viṣṇu bhakti}.

 \item Frank E. Reynolds could be right about both the Buddhist Rāma story and \textit{Vālmīki’s} epic having influenced the Thai telling, called \textit{Rāmākien} (Reynolds 1991). The names of the characters in \textit{Rāmākien} sound rather Tamilized, suggesting that \textit{Rāmākien} could have been created based on Kamban’s Tamil narrative, as conveyed by South Indian traders and travellers from their own earlier reading experience. The story recognizes the divinity of Rāma and Sītā, and the storyline is close to the one by Vālmīki. The settings are Thai and their own \textit{Ayutthaya} kingdom features in place of India’s Ayodhya. The Thai audience who came heavily under the influence of Buddhism may not relate to Rāma as God, however, and there is probably no temple for Rāma in \textit{Ayutthaya}. \textit{Rāmākien} seems to be a beautifully created Thai story based on some Indian versions of \textit{Rāmāyaṇa}, which will not harm the practice of Vaishnavism in India by devotees of Rāma. The brief Indian Buddhist story of Rāma, Prince of \textit{Vārāṇasī} too could have been a part source for the Thai \textit{Rāmākien}, but its dating as having preceded the Vālmīki story is purely a Western academic construct. It is unlikely to have been a source for the Vālmīki epic, as Reynolds contends. Most of the essential components of the \textit{Rāmāyaṇa} story are missing in it, such as the all-important Rāma-Rāvaṇa war. There is, in addition, no hint of a divine attribution to the Rāma character in it.

 \item Richman (1991) finds E V Ramasamy’s version of \textit{Rāmāyaņa} compelling and interesting when viewed under the lens of his anti-North Indian ideology. This ideology took him to several anti-stances such as against Sanskrit, Hindi, what he called Brahminism, practice of \textit{sanātana dharma} in any form, Indian \textit{itihāsa-s} and \textit{purāṇa-s,} Hindu sacred literature of any kind and Central rule over the Tamil region. Rāma and the \textit{Rāmāyaṇa} came in for his harshest censure and insults. The DMK and the AIADMK which were politically successful offshoots of his party, the Dravida Kazhagam, have always paid lip service to him and honoured statues and institutions installed in his memory, but kept some distance respectfully from him, while he was alive. The DMK President, M. Karunanidhi (index Karunanidhi, M.)was very recently involved in the production of a popular teleserial on the life of Rāmānuja (2016). Not so his ‘mentor’, E.V.Ramasamy. At different stages in his political career, his own concept of Buddhism, Islam, and Protestant Christianity got the seal of his approval, but \textit{sanātana dharma}, not in any form. We cannot, naturally consider Ramasamy’s version of the Rāma story as a \textit{Rāmāyaṇa}.

 \item The present President of the Dravida Kazhagam Dr Veeramani and an active American chapter of Periyar’s Self Respect Movement have been coordinating their activities for about two decades. Towards the end of July 2017 in Germany, the Cologne University’s Department of South Asian and Tamil studies hosted a conference on Periyar’s Self Respect Movement (2017) and invited Veeramani over. Shortly after, on Yajur Upakarma day, when Brahmin men in Tamil Nadu were changing their sacred threads and resuming Vedic studies during an annual ritual, some eight or so Dravida kazhagam enthusiasts in Chennai seem to have taken out a procession with some piglets wearing ‘sacred’ thread, and to have been quickly seized by the police.

\end{itemize}


\subsection*{The Western Nay Science}

\begin{itemize}
\item Bagchee and Adluri, authors of‘The Nay Science (2017)’, were interviewed by Eram Agha (index Agha, Eram) for News18.com on Aug 16, 2017.They admitted to having found German Indology being far from secular. Most German Indologists were theologically trained Protestants, they say. Several were pronounced anti-Semites as well, some of them being complicit in the cover-up of their colleagues’ anti-Semitism. Often in these Indologists’ writings, “Brahmans” was a code for “Jews.”

 \item They told the interviewer, “Almost every leading German Indologist of the past two centuries authored an anti-Brahmanic polemic. Besides making explicit comparisons between “Brahmanic” Hinduism and Jewish tradition, the Indologists advocated a program of reform, entailing breaking the Brahmans’ social status, taking away their authority, and transferring custody of Sanskrit texts to the new priesthood-professoriate. A prejudice against traditional hermeneutics and textual transmission was inscribed at every level of the method. Consequently, students graduating from their Indology programs emerged as critics of Brahmanism.”

\end{itemize}


\subsection*{Wendy Doniger (index Doniger, Wendy)}

\begin{itemize}
\item Quite a few of us would go along with Wendy Doniger when she rued the ban on Ramanujan’s essay on the ‘three hundred’ \textit{Rāmāyaņas} by the Delhi University. But we need not agree with her blanket labelling of every story on Rāma and/or Sītā as \textit{Rāmāyaṇa}. Michael Madhusudan Dutta’s (index Dutta, Michael Madhususdan) Meghnādbadh ballad talks about the might and repute of Rāvaṇa\textit{'s} son Meghanāda as a great warrior and how Lakṣmana had to kill him when he was unarmed and engaged in performing a \textit{homa}, using inside information provided by Vibhīṣaṇa. As Vālmīki’s treatment of the story too is similar, this fine Bengali literary effort can be taken as a \textit{Rāmāyaṇa} episode. Killing Meghanāda was a necessary preliminary to defeating Rāvaṇa.

 \item Nina Paley’s 2D (index Paley, Nina) animation film, ‘Sita sings the Blues’ won many awards for technical excellence as well as for its theme. It helped the author to create a greater appreciation of her bad luck in having married badly and being discarded by her husband. While being generally confident of her narration, Nina Paley did admit to showing Rāma in a bad light, in her interview with India West. She was looking after promotion of her interests and did not bother about hurting devotees of Rāma. There would be similar less acclaimed accounts of the story of Sītā by other Indian and foreign authors. Those in which the divinity of Rāma and Sītā is not obvious cannot be considered \textit{Rāmāyaṇa}.

 \item As to Doniger’s own comments about the main characters in \textit{Vālmīki Rāmāyaṇa} it is possible to agree with Vishal Agarwal (2014) who finds this author of an alleged alternate history for Hinduism to have done scant preparation for the serious business, committed many factual errors and exhibited her biases without any restraint.

\end{itemize}


\subsection*{Sheldon Pollock and his School (index Pollock, Sheldon)}

\begin{itemize}
\item This paper began with a reference to the misinterpretation of our sacred spaces by some western Indologists, and on the attention paid by the previous meeting of the Swadeshi Conference on one author. In his report on the Conference, Prof. K.S. Kannan (index Kannan, K.S.)has this to say on that author: “The focus on Pollock (as with SI-1) was on grounds of his being the most influential scholar in academic circles (also through his students who are well placed in celebrated academia) firstly, and secondly, in his reach on the general public, especially Indian (by virtue of his editorship in leading publishing houses); but even more importantly on grounds of the complex and complicated intellectual maze he has conjured, aimed at confounding and bamboozling the reading public by an ostentation of scholarship: in width, depth, or impact, or even the novelty of ideas or interpretations – in short, in terms of the \textbf{Neo-orientalism} he has sought to bring about in a copiously camouflaged and convoluted language – he has added new negative dimensions to Indology (and hence totally incomparable to some of the earlier Orientalist Indologists, the damage caused by whom pales into the background in quality and quantity).”

 \item Bagchee and Adluri said more to News18.com. They see evidence of the work of German Indologists becoming anachronistic within the German university itself. They find their programs declining. From twenty-two and a half chairs in 1997, only sixteen survive, more closures being anticipated. They consider American Indology to be merely a stepchild of German Indology. Almost every leading American program at some point imported German expertise, in the form of either German professors, German-trained returnees or German models and ideals of study (almost every Sanskrit doctoral program in the US). Many principles of American Indology (a suspicion of traditional hermeneutics, criticism of the Brahmans, restricting works’\break meaning to their sociological context, historicism and a so-called critical philology) are borrowed from German Indology. Do these features ring a bell?

 \item Malhotra’s “Battle for Sanskrit” (2016) takes on the entire Pollock challenge almost singlehandedly. I shall quote here just one thought from the book. He finds that “Pollock’s involvement in the Ramayana has left a lasting imprint on the way the epic is being seen. His hegemonic discourse is visible in the way the Ramayana has become tainted in several English-speaking circles. Over the past 25 years, academic literature and journalistic writings have been filled with assertions resembling Pollock’s spin, not only of the Ramayana, but also in the portrayal of Sanskrit as a carrier of social toxicity.” This cannot be allowed unchecked.

 \item In an e mail communication, Shastry M points out that Pollock has a “proclivity to base ideas on a rather small subset of data but build upon them sweeping generalisations that address the largest of questions. Culture in his hands becomes equivalent to the set Language and to a narrower subset Literature – \textit{kāvya}. He briefly mentions Power in context of \textit{rājya}, only once at the very beginning of his magnum opus, ‘The Language of Gods in the World of Men’. Afterwards, he does not look even once at \textit{rājya} and Power through the lens of traditional paradigms, but keeps exploring the relationship through anachronistic socio–political models of legitimation, socialisation and communication.” She sees this manipulative procedure repeated in all his work. Also, obvious to her are his refusal to value \textit{Pāramārthika}, and his blind assertion that \textit{kāvya} in Sanskrit could have developed only after the writing tradition was ‘introduced’ into the oral Sanskrit tradition.

 \item Simply on a question of hermeneutics, we could dismiss the Pollock (index Pollock, Sheldon) adaptation of/reflections on \textit{Rāmāyaṇa} as non-\textit{Rāmāyaṇa}.

\end{itemize}



\section*{Conclusion}

Some Western Indologists and their Indian followers have been translating/ adaptingas well as commenting on Rāma and Kṛṣņa stories from our \textit{itihāsa}-s inappropriately. This paper sets up a standard for any work of adaptation of our epics, based on the excellent performance of such work both in Tamil and in Sanskrit since the beginning of the Common Era in Tamil Nadu and checks out some of the provocative as well as neutral literature that has appeared in recent times on our \textit{itihāsa}-s.


\section*{Bibliography}

\begin{thebibliography}{99}
\bibitem{chap10-key01} Agarwal, Vishal. “A Critique of Wendy Doniger’s “The Hindus, an Alternative History””. Feb 20th, 2014 \url{http://hindureview.com/2014/02/20/critique-wendy-donigers-hindus-alternative-history/ } Accessed July 17, 2017.

 \bibitem{chap10-key01} Agha, Eram in News18.com “Historicism has collapsed, Mahabharata the way forward: Bagchee and Vishwa Adluri” updated 16th August 2017, Accessed August 2017.

 \bibitem{chap10-key02} Danino, Michel. “Vedic Roots of Early Tamil Culture.” \url{micheldanino.bharatvani.org}/\url{tamilculture.html} accessed June 2016

 \bibitem{chap10-key03} Desikan, Parthasarathy. (2007)“\textit{Silappadikaaram}”. Margabandhu.com, pp 260-1

 \bibitem{chap10-key04} Desikan, Parthasarathy. (2010) “\textit{Sri Padukasahasram}” (English translation) Malola Granthamala. Hyderabad.

 \bibitem{chap10-key05} Desikan, Parthasarathy (2007) \textit{Silappadikaaram.} Margabandhu.com.

 \bibitem{chap10-key06} Dikshitar, V R Ramachandra.(1939) “\textit{The Silappadikaram}”. Book Source: \url{Digital Library of India Item 2015.201802}, (accessioned: 2015-07-09T12:40:20Z, available from 2015-07-09T12:40:20Z, digital publication date: 2005-02-09), page 173, accessed July 2017

 \bibitem{chap10-key07} Doniger, Wendy. \textit{Enduring Ramayana.} Geo (Indian Edition), October 2011, p 36.

 \bibitem{chap10-key08} Doniger, Wendy (2009). \textit{Hindus - an alternate History.} Penguin India. Delhi.

 \bibitem{chap10-key09} Gnanasambandan A.S; Gnanasundaram D.,(2002) Editors,” KambaRamayanam,” Part I, in Tamil, payiram (preface), verses 5\&10.

 \bibitem{chap10-key10} Govindaraju, Venu \& Srirangaraj, Setlur (2009) \textit{Guide to OCR for Indic Scripts:} \textit{Document Recognition and Retrieval}. Springer. New York.

 \bibitem{chap10-key11} Kannan, K.S. (2016) \url{https://rajivmalhotra.com/swadeshi-indology-2-conference-a-report}

 \bibitem{chap10-key12} Lutgendorf, Philip.(1990) “Ramayan, the Video”. The Drama Review, 34.2. pp 127-176.

 \bibitem{chap10-key13} Malhotra, Rajiv (2016). \textit{The Battle for Sanskrit.} HarperCollins Publishers. Noida.

 \bibitem{chap10-key14} Meyyappan, S. (Ed.) (1997) \textit{BharatiyarKavithaigal(Tamil)}, Manivasagar Padippagam,. Chennai. p.413

 \bibitem{chap10-key15} Nagaswamy, R. (1980) \textit{Art and Culture of Tamil Nadu.} Sundeep Prakashan. New Delhi.

 \bibitem{chap10-key16} Pollock, Sheldon. (1986) \textit{Ramayana of Valmiki, Ayodhyakanda. (Translation) Vol.II.} University of California Press.Berkeley.

 \bibitem{chap10-key17} Pollock, Sheldon.(2005) \textit{Ramayana, Book Two, Ayodhya, by Valmiki}. New York University Press and JJC Foundation. New York.

 \bibitem{chap10-key18} Pollock, Sheldon. “The Divine King in the Indian Epic”. The Journal of Oriental Society, 104(3). Pp. 505-28

 \bibitem{chap10-key19} Pollock, Sheldon (1985).“The Theory of Practice and the Practice of Theory in Indian intellectual History.” Journal of the American Oriental Society.105(3). Pp 499-519.

 \bibitem{chap10-key20} Raghavan, V. (2016) “The Northern Boundary of Tamilnadu.” In Raghavan, V. (Ed.) \textit{Madras and Tamilnadu - An Anthology}. Dr. V. Raghavan Centre for Performing Arts. Chennai.

 \bibitem{chap10-key21} Raghavan, V. (2016) “Sanskrit and Tamilnadu” In Raghavan, V. (Ed.) \textit{Madras and Tamilnadu - An} \textit{Anthology.} Dr. V. Raghavan Centre for Performing Arts. Chennai.

 \bibitem{chap10-key22} Ramanujan, A.K. (1991) “Three Hundred Ramayanas: Five Examples and three Thoughts on Translation\textit{”.} In Richman, Paula\textit{. Many Ramayanas.} University of California Press. Berkeley.

 \bibitem{chap10-key23} Reynolds, Frank, E. (1991) “Ramayana, Rama Jataka, and Ramakien: A Comparative Study of Hindu and Buddhist Traditions\textit{.”} In Richman, Paula.\textit{ Many Ramayanas}. University of California Press. Berkeley.

 \bibitem{chap10-key24} Richman, Paula. (1991). \textit{Many Ramayanas}. University of California Press. Berkeley.

 \bibitem{chap10-key25} Richman, Paula. (1991) “E. V. Ramasami's Reading of the Ramayana” In \textit{Many}

 \bibitem{chap10-key26} \textit{Ramayanas}. University of California Press. Berkeley.

 \bibitem{chap10-key27} Rudolph, Lloyd \& Rudolph, Susanne. (1993) “Modern Hate: How Ancient Animosities get Invented”, “The New Republic\textit{”}, 22 March 1993, Pp. 24-9, quoted in Malhotra, Rajiv. (2016) \textit{The Battle for Sanskrit}. HarperCollins Publishers. Noida

 \bibitem{chap10-key28} Santhosh, K.(2014) “When Malayalam found its Feet” (updated on July 17, 2014), (1939 \url{www.thehindu.com.news/national/kerala/when-malayalam--found-its-feet/article6217620.ece}, accessed June 2016.

 \bibitem{chap10-key29} Srivatsan (ed.)(1984) \textit{NalayiraDivyaprabandham Parts 1 \& 2 (Tamil)}. The Little Flower Company. Chennai.

 \bibitem{chap10-key30} Subramanian, T. S. “‘Rudimentary Tamil-Brahmi script' unearthed at Adichanallur” (February 17, 2005), \url{http://www.thehindu.com/2005/02/17/stories/2005021704471300.htm}, accessed June 2016.

 \bibitem{chap10-key31} Tapasyananda (2001) \textit{Adhyātma Ramāyaņa}. Sri Ramakrishna Math, Chennai.

 \bibitem{chap10-key32} Thapar, Romila (2014) \textit{“The Ramayana Syndrome” in “Past as Present - Forging} \textit{Contemporary Identities through History”, Book 3}. Aleph Book Company. Delhi.

 \bibitem{chap10-key33} Yamunan, Sruthisagar. “TTD Boost to Karunanidhi’s ‘Ramanujar’. ”(January 08, 2016,Updated: September 22, 2016,\url{,http://www.thehindu.com/news/cities/chennai/TTD-boost-toKarunanidhi%E2%80%99s%E2%80%98Ramanujar%E2%80%99/article13988346.ece}, accessed July 2017

 \bibitem{chap10-key34} The Hindu Net Desk, “Keezhadi excavation leads to ancient civilisation on the banks of Vaigai “(May 16, 2017 Updated: July 28, 2017) \url{http://www.thehindu.com/news/national/tamil-nadu/keezhadi-excavation-the-story-till-now/article18464920.ece}, accessed July 2017

 \bibitem{chap10-key35} Express News Service, “Delegation of 50 Tamil scholars takes Periyar’s ideology to Germany” 27th July 2017 Last Updated: 27th July 2017, Accessed, July 2017

 \end{thebibliography}

