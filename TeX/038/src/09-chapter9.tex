
\chapter{Yogic Perceptions of Aryan-Dravidian Controversy}\label{chap09}

\Authorline{Subhodeep Mukhopadhyay}


\section*{Abstract}

The epistemology\index{epistemology} of \textit{Nyāya}\index{Nyaya@\textit{Nyāya}} accepts four \textit{pramāṇa}-s\index{pramana@\textit{pramāṇa}} or valid means of knowledge, \textit{pratyakṣa} (perception), \textit{anumāna}\index{anumana@\textit{anumāna}} (inference), \textit{upamāna}\index{upamana@\textit{upamāna}} (comparison and analogy) and \textit{śabda}\index{sabda@\textit{śabda}} (testimony of an authority figure). Traditionally in India, the testimony of an authority figure or āpta\index{apta@āpta} has been held in high esteem. In this paper we apply a \textit{nyāya} framework to the vexed issue of the Aryan-Dravidian controversy using the \textit{śabdapramāṇa} in relation to the utterances of three authority figures, Swami Vivekananda\index{Vivekananda, Swami}, Sri Aurobindo\index{Aurobindo, Sri}, and Sri Sri Ravi Shankar\index{Ravi Shankar, Sri Sri}. Each of these three people have been very influential individuals, having directly or indirectly impacted the lives of numerous people and have considerable pan-India as well as international following and may be considered\textit{ āpta} based on the traditional pre-conditions, of right knowledge, desire to share knowledge and the ability to communicate effectively. This paper examines the key ideological foundations of the Dravidian movement and the Aryan-Dravidian divide. It is then demonstrated that all these three \textit{āpta-s} have been highly critical of the racial connotations of the Aryan-Dravidian divide, its social and cultural ramifications and have categorically rejected the Aryan invasion/ migration theory. This paper therefore critically examines the doctrinal basis of the Aryan-Dravidian controversy from the standpoint of dharma using a \textit{nyāya} framework and Indic categories, and concludes that colonial racist theories are invalid and have to be discarded.


\section*{Introduction}

Prior to the advent of European Indology, the words \textit{Ārya} and \textit{Draviḍa} traditionally had cultural and geographical connotations only. The racial and linguistic interpretation of the terms were provided by Max Müller\index{Muller, Max} and Bishop Caldwell,\index{Caldwell, Robert} and this shift in meaning led to the rise of Indology as a discipline and the so called Aryan-Dravidian divide (Danino\index{Danino, Michel} 2007:2). There are two parts to this issue. The first is the popularization of the Aryan Invasion theory which posits that around 3500-4000 years ago, barbarians from outside Indian subcontinent, the Aryans, invaded India and overran the indigenous Dravidians. Sanskrit, the language of the invading Aryans thus became a rival of Tamil. According to Colonial Indologists, these invading Aryans who shared common ancestry with Europeans brought civilization to India. Under the influence of 20th century German identity-need and with the popularization of European race science,what started out as a linguistic hypothesis, morphedinto the notion of an Aryan race(Malhotra\index{Malhotra, Rajiv} and Neelakandan\index{Neelakandan, Aravindan} 2011:14). The second part of this issue is strengthening of the victimhood narrative of the supposed autochthonous Dravidians who were allegedly forced to migrate from northwest to south India in order to escape the marauding Aryan invaders. In this narrative north Indians and other upper caste Indians became the descendants of the invading Aryans who had overran the ancestors of the native Dravidian Tamils (Danino 2001:148). Over and above that, British colonial administrators and the Anglican Church\index{Church}, popularized the linguistic separation of Tamil from other Indian languages, the notion of an original non-Aryan aboriginal “Tamulian” race and the alleged role of “cunning Aryan Brahmins” in subduing native Dravidians(Malhotra and Neelakandan 2011:62). The conflation of the four streams, traditional, cultural, racial and linguistic gave rise to tragic consequences like a north-south divide, rise of Dravidian nationalism in the 1920s, perception of Brahmin domination and anti-Brahmin protests and attacks, anti-Hindi agitations and Tamil identity politics. E.V. Ramasamy\index{E.V. Ramasamy}, who founded the Draviḍar Kazhagam\index{Draviḍar Kazhagam}, believed that Tamilians were the ruler of the land prior to the advent of nomadic barbarians. The movement was based on the premise that Tamilians could re-emerge as a superior race if they give up their Indian and Hindu identity. Moreover since Sanskrit was associated with Aryans and Hindus, hence an anti-Sanskrit attitude was a key component of their movement.

\begin{myquote}
“Aryans were nomads in different places and picked up different dialects. And what they call today their Sanskrit language is actually a combination of these dialects and languages spoken at different places in different ages. The Sanskrit language has nothing noble in it and the Brahmins spoke high about Sanskrit only to make themselves superior and to humiliate other languages.” (From the collection “The Great Falsehood”, Viduthalai, 31-July-2014).(Neelakandan 2015)
\end{myquote}

There is not much confusion with regards to the traditional cultural and geographical connotations of the terms \textit{Ārya} and \textit{Drāviḍa}, and it is also accepted that the Indo-Aryan language family is distinct from the Dravidian language family. However the conflation of tradition, race, culture and linguistics and the ideological basis of the Dravidian-Aryan divide have been challenged by a large number of scholars. Shrikant Talageri in (Talageri\index{Talageri, Srikant} 2000) and (Talageri 2008) has systematically challenged the Aryan invasion scenario on various ground and has demonstrated that an indigenous Aryan model better explains the Indo-European spread linguistic problem and points to migrations out of India rather than invasions of India. On the Dravidian question, numerous scholars such as Ramachandran Nagaswamy\index{Nagaswamy, R.} in (Nagaswamy 2016) or Michel Danino in (Danino 2001) have demonstrated with evidence, affinity of Tamil culture with the north rather than perceived animosity. Rajiv Malhotra\index{Malhotra, Rajiv} and Aravindan Neelakandan\index{Neelakandan, Aravindan} in their book Breaking India focus on the intervention by foreign churches, academics, think-tanks, foundations, government and human rights groups in engendering Dravidian and Dalit separatism, encouraging Dravidian Christianity\index{Christianity} movement and exploiting various social and political fault lines(Malhotra and Neelakandan 2011). Outside the field of academics, many prominent gurus and leaders from different parts of India over the last century, have questioned the Aryan-Dravidian issue and rejected the ideological foundations of the perceived north-south or invader-victim narrative\index{invader-victim narrative}.

\newpage

This paper is neither an examination nor a rebuttal of the political Dravidian movement which was started by E.V. Ramasamy\index{Ramasamy, E.V.}. In Indic traditions, a great deal of importance is given to the words of an \textit{Āpta}, a term which can be roughly translated as a trustworthy authority. Spiritual gurus and leaders, provided they adhere to certain moral conditions are considered as \textit{Āpta}-s (Phillips 2011).Therefore in this paper we are focused on understanding the utterances of a few prominent gurus and thinkers (\textit{Āpta}-s\index{apta@\textit{āpta}}) like Swami Vivekananda, Sri Aurobindo and Sri Sri Ravi Shankar, on the ideological foundations of the Aryan-Dravidian divide. We have examined primary first hand sources like their speeches, transcripts of \textit{satsangs}, discourses and writings, and additionally also examined secondary sources wherever available.


\section*{Nyāya epistemology\index{epistemology} and śabdapramāṇa}

All schools of Indian thought regard ignorance as the root cause of all sufferings and true knowledge is required to overcome sufferings. Epistemology is the study of knowledge and deals with two fundamental questions, the nature of knowledge and the extent of knowledge. It tries to address issues like determining what is known and what is not, the limits of what can be known and the various valid means through which such knowledge can gained. Indian knowledge systems rely on a variety of sources of valid means of knowledge. \textit{Nyāya}\index{Nyaya@\textit{Nyāya}} is one of the six āstika schools of India philosophy and provides a methodology to systematically conduct investigations of objects and subjects of human knowledge. Vatsyayana defines \textit{nyāya} as “a critical examination of the objects of knowledge by means of the canons of logical proof” (Bernard 2003:35). \textit{Nyāya} is closely related to \textit{hetu vidyā} (science of causes), \textit{Ānvīkṣikī} (science of inquiry), \textit{pramāṇa śāstra} (epistemology), \textit{tattvaśāstra}\index{tattvasastra@\textit{tattvaśāstra}} (science of categories) and \textit{tarka vidyā} (science of reasoning) (Vidyabhushana 1913:i).The \textit{nyāya} system relies on four valid means of knowledge i.e., \textit{pratyakṣa}(perception), \textit{anumāna} (inference), \textit{upamāna} (comparison and analogy) and \textit{Śabda} (testimony of authority figures).

\textit{Śabda}\index{sabda@\textit{śabda}} or word is defined as the instructive assertion of a reliable person. Vidhyabhushana gives the following illustration:

\newpage

\begin{myquote}
“Suppose a young man coming to the side of a river cannot ascertain whether the river is fordable or not, and immediately an old experienced man of the locality, who has no enmity against him, comes and tells him that the river is easily fordable: the word of the old man is to be accepted as a means of right knowledge called verbal testimony.”(Vidyabhushana 1913:4)
\end{myquote}

\textit{Śabda} pertains to two kinds of assertions. One is about matters which can be seen and actually verified. The traditional example is a physician’s assertion that one gains physical strength by consuming butter. The other pertains that which is not seen, and that which we cannot verify but must ascertain by means of inference. The traditional example is the assertion of a religious teacher that a person obtains heaven by performing horse-\textit{yajña} (Ibid.)

\textit{Śabda} is accepted as a valid means of knowledge by most of the Indian philosophical schools except by the Jains, Buddhists, \textit{Cārvāka} and \textit{Vaiśeṣika} schools. The \textit{Vaiśeṣika}-s however accept verbal testimony as inference provided it is the utterance of an infallible person. Discussions on \textit{śabda}\index{sabda@\textit{śabda}} were an important component of historical philosophical debates in old India. In fact \textit{śabda} as a \textit{pramāṇa} is a key feature of the \textit{Mīmāṃsā}\index{Mimamsa@\textit{Mīmāṃsā}} School.

\begin{myquote}
“Śabda (word) as a pramāṇa means the knowledge that we get about things (not within the purview of our perception) from relevant sentences by understanding the meaning of the words of which they are made up. These sentences may be of two kinds, viz. those uttered by men and those which belong to the Vedas. The first becomes a valid means of knowledge when it is not uttered by untrustworthy persons and the second is valid in itself.”(Dasgupta 2012: 395)
\end{myquote}


\section*{Āpta and Āpta Vacana}

According to the \textit{nyāya} system, there are two kinds of testimony, viz. \textit{vaidika} and \textit{laukika}. \textit{Vaidika} testimony is associated with the transcendental realm while \textit{laukika} is associated with the real world. As far as \textit{laukika} testimony is concerned, only that which comes from a trustworthy person is valid.

\begin{myquote}
\textit{“vākyaṃ dvividham|vaidikaṃ laukikaṃ ca |vaidikamīśvaroktatvātsarvameva pramāṇam |laukikaṃ tvāptoktaṃ pramāṇam |anyadapramāṇam |”}
\end{myquote}

\begin{center}
Tarkasaṅgrahaḥ śabdanirūpaṇam 4
\end{center}

In the \textit{nyāya} system, \textit{śabda} is defined as correct knowledge derived from the utterances of one who is reliable and truthful, and such a person is known as an\textit{Āpta}.

\begin{myquote}
\textit{“āptopadeśaḥ śabdaḥ”} Gautama’s Nyāya sūtra 1.1.7
\end{myquote}

A reliable person may be a \textit{ṛṣi} (seer), an \textit{ārya}(countryman) or even a \textit{mleccha}(foreigner or barbarian) provided that as an expert on a certain matter he is willing to share the same with others(Vidyabhushana 1913:4). Such a person is characterized by three qualities, viz. (1) he has direct and right knowledge of a given subject, (2) he is compassionate and wants to share the knowledge with others and (3) he has the ability to communicate such knowledge effectively.

\begin{myquote}
\textit{“āptaḥ khalu sākṣātkṛtadharmā yathādṛṣṭasyārthasya. cikhyāpayiṣayā prayukta upadeṣṭā”} Vātsyāyana-bhāṣhya on Gautama’s Nyāya sūtra
\end{myquote}

An \textit{Āpta} who is a \textit{ṛṣi} or seer is able to “see” or cognize higher truths, and this has been corroborated repeatedly in the Vedic corpus. A technical term commonly used is \textit{mantradraṣṭā} or seer of the mantra. Swami Vivekananda\index{Vivekananda, Swami} says in this regard:

\begin{myquote}
“Rishi-state is not limited by time or place, by sex or race. Vâtsyâyana boldly declares that this Rishihood is the common property of the descendants of the sage, of the Aryan, of the non-Aryan, of even the Mlechchha.”(Swami Vivekananda 2008)
\end{myquote}

Such higher states of consciousness are a natural outcome of Yogic practices and there are numerous accounts of spiritual masters performing supernormal activities or attaining siddhis. It should be noted that the biographies of spiritual gurus have mostly been recorded by their spiritual disciples or by people who are in some way related to their spiritual movement. Most academicians in order to highlight the apparent “subjective” component of such accounts, prefer to use the term hagiography rather than biography. Dorthe Refslund Christensen says:

\begin{myquote}
“Hagiographies\index{hagiography} are not “objective” historical accounts put forward in a narrative style meant to reproduce all the highlights of the person‘s life. On the contrary, hagiographies are social and textual constructions produced with the particular aim of informing the recipient about specific paradigmatic events and actions connected to the founder or originator of a religion.”(Christensen 2005:233)
\end{myquote}

\newpage

As far as gurus are concerned, an average Hindu sees no dichotomy between biography and hagiography, and would tend to treat them as the same. In fact in India traditions, the holiness of a guru is assessed based on many signs and symbolisms, like time of birth, planetary alignments, special powers and rare capabilities exhibited from very early on. The lens used to view the stories of our gurus is a lens of śraddhā or trust and reverence. In their interpretation of Mahābhārata\index{Mahabharata@Mahābhārata}, Adluri and Bagchee refer to this as the hermeneutics of respect which more than establishing historicity aims to assess the underlying philosophical implications of such texts and understanding through the stories about ourselves in relation to the cosmos(Banerjee 2017). In a similar way lives of saints and gurus are usually studied by Hindus using a hermeneutics of respect as opposed to a hermeneutics\index{hermeneutics} of suspicion.

Having said that, there are studies which suggest that these higher “Rishi States” or so-called “mystical powers” are a natural progression of humans beyond adulthood, and are not momentary or unverifiable, but rather “a developmental level of subtlety and comprehensiveness that goes beyond the level which can be readily appreciated within the boundaries of ordinary adult thought.(Alexander,, and Alexander 1987). Another study has shown positive correlation between Transcendental Meditation program of Maharishi Mahesh Yogi and higher states of consciousness during sleep (Mason et al. 1997). Hence it may not be unreasonable to say that an \textit{Āpta} may make use of higher cognitive powers to garner insights which a non-\textit{Āpta} may not readily perceive either through direct perception, or inference or comparison. Swami Vivekananda, Sri Aurobindo and Sri Sri Ravi Shankar are all \textit{Āpta-}s, given their proven track record of being knowledgeable, their willingness to share such knowledge and their ability to communicate such knowledge in a form that non-experts can comprehend without much difficulty.


\section*{Swami Vivekananda}

Swami Vivekananda\index{Vivekananda, Swami} (1863 – 1902) was an Indian Hindu monk and mystic who brought Hinduism to the status of a major world religion during the late 19th century. He was the founder of Ramakrishna Math and the Ramakrishna Mission. As a student he excelled in music, gymnastics and studies, and had a strong understanding of both Western and Indian philosophy and history. He was born with a yogic temperament, and would practice meditation even from his boyhood. While he is considered a great scholar and an extraordinary intellectual, he was also a mystic, and a number of accounts of his mystical or higher powers are available from his disciples both within and outside India. (Nandy 2014) based on accounts of a French opera singer, an American national and Indian disciples and devotees, touches upon Swami Vivekananda’s advanced yogic capabilities.

Swami Vivekananda accepted the Aryan and Dravidian language families as distinct language families but viewed entire India as one people and did not accept the view that Aryans invaded India and subjugated the native Dravidians. He understood the serious implications of the Aryan-Dravidian divide for the unity of India and especially for Tamil Nadu, and opposed it in various public platforms. In his “Lectures from Colombo to Almora” in 1897 he says:

\begin{myquote}
“…I want to discuss one question which it has a particular bearing with regard to Madras. There is a theory that there was a race of mankind in Southern India called Dravidians, entirely differing from another race in Northern India called the Aryans, and that the Southern India Brâhmins are the only Aryans that came from the North, the other men of Southern India belong to an entirely different caste and race to those of Southern India Brahmins. Now I beg your pardon, Mr. Philologist, this is entirely unfounded. The only proof of it is that there is a difference of language between the North and the South. I do not see any other difference.”(Swami Vivekananda 2008: 221)
\end{myquote}

He rejected the “Aryan theory and all its vicious corollaries” and insisted that all Indians, especially those of the south needed a “gentle yet clear brushing off of the cobwebs” of the theory. He accepted the traditional definition of the term \textit{Ārya} and insists that the whole of India is Aryan. He says in this regard:

\begin{myquote}
“We stick, in spite of Western theories, to that definition of the word "Arya" which we find in our sacred books, and which includes only the multitude we now call Hindus. This Aryan race, itself a mixture of two great races, Sanskrit-speaking and Tamil-speaking, applies to all Hindus alike.”(Swami Vivekananda\index{Vivekananda, Swami} 2009)
\end{myquote}

Many Indologists of his times ascribed the term dasyu\index{dasyu} to native aborigines of India, and equated them with a Dravidians race, a notion which Swami Vivekananda categorically rejects. He says:

\begin{myquote}
“Whatever may be the import of the philological terms "Aryan" and "Tamilian", even taking for granted that both these grand sub-divisions of Indian humanity came from outside the Western frontier, the dividing line had been, from the most ancient times, one of language and not of blood. Not one of the epithets expressive of contempt for the ugly physical features of the Dasyus of the Vedas would apply to the great Tamilian race; in fact if there be a toss for good looks between the Aryans and Tamilians, no sensible man would dare prognosticate the result.”(Swami Vivekananda 2009)
\end{myquote}

He not only saw a unifying theme in India, but also firmly believed that for the nation to progress, the Aryan-Dravidian divide and Brahmin-non-Brahmin divide had to be dissolved, and had given a warning that if not resolved at the earliest, these disagreements could potentially transform into schisms.

\begin{myquote}
“And the more you go on fighting and quarrelling about all trivialities such as "Dravidian" and "Aryan", and the question of Brahmins and non-Brahmins and all that, the further you are off from that accumulation of energy and power which is going to make the future India. For mark you, the future India depends entirely upon that.”(Swami Vivekananda 2008:230)
\end{myquote}


\section*{Sri Aurobindo}

Sri Aurobindo(1872 – 1950)\index{Aurobindo, Sri} was a philosopher, yogi, mystic and spiritual guru who developed the philosophy of Integral Yoga and human evolution. As a polyglot he was fluent in Sanskrit, Hindi, Bengali, Tamil, English, French, German, Greek and Latin. Having studied and grown up in England prior to coming back to India, he was conversant with both Western philosophies and the Hindu philosophy of Veda-s, Upaniṣad-s\index{Upanisad@Upaniṣad}, Tantra\index{tantra@\textit{tantra}} and Bhagavad Gītā, and his own philosophical position was solidly grounded in the Vedic tradition. After his return to India in 1893, he had a number of number of intense spiritual experiences such as being engulfed in a vast calm for many months, vision of Godhead springing forth to save his life, experience of a living presence of Goddess Kali in a temple on the banks of Narmada River and vision of the Infinite in an ancient Shankaracharya Temple in Srinagar in Kashmir (Sri Aurobindo 2006:110). M. P. Pandit says:

\newpage

\begin{myquote}
“Sri Aurobindo does not think or build in vacuum. His philosophy…is based on the solid foundations of his spiritual experience ripening into realization. He takes care to verify this experience with reference to past spiritual realizations and see how far they corroborate his line of experience and the vision that he sees, \textit{darshana}. …Sri Aurobindo is at one with the perception of the Rishis of the Upanishad that man is essentially divine…. It is in this sense that his philosophy is called realistic \textit{advaita}.”(Pandit 2014)
\end{myquote}

Sri Aurobindo understood the Aryan-Dravidian identity issue and all its nuances.He was extremely critical of the Aryan Invasion theory where “barbarous Aryan invaders” supposedly invaded, conquered and subjugated the “civilized Dravidians” after coming through the Punjab and he wondered if the whole theory was not simply a “myth of the philologists\index{philologists}. (Sri Aurobindo 2012:6)

\begin{myquote}
“…the indications in the Veda on which this theory of a recent Aryan invasion is built, are very scanty in quantity and uncertain in their significance. There is no actual mention of any such invasion. The distinction between Aryan and un-Aryan on which so much has been built, seems on the mass of the evidence to indicate a cultural rather than a racial difference.”(Sri Aurobindo 2012:26)
\end{myquote}

In his study of the Vedas, based on which European Indologists had posited the invasion scenario, he concluded that not only was there no mention of any invasion, but also the differences between Aryans and others were of a cultural nature rather than being racial. He completely rejected the racial Indological thesis of fair-skinned barbarian Aryans overwhelming and subduing the original black-skinned dasyus\index{dasyu} who inhabited a civilized Dravidian peninsula.

\begin{myquote}
“It is urged that the Dasyus are described as black of skin and noseless in opposition to the fair and high-nosed Aryans. But the former distinction is certainly applied to the Aryan Gods and the Dasa Powers in the sense of light and darkness, and the word anāsah does not mean noseless. Even if it did, it would be wholly inapplicable to the Dravidian races; for the southern nose can give as good an account of itself as any “Aryan” proboscis in the North.”(Sri Aurobindo 2012:26)
\end{myquote}

Sri Aurobindo\index{Aurobindo, Sri } during his stay in the south, while admitting a “general impression of a southern type”, found it difficult to pinpoint what it was that separated the supposedly Aryan northern Indians from the Dravidian south Indians and instead observed that behind all variations there remained a “a unity of physical as well as of cultural type throughout India” and discarded the “sharp distinction between Aryan and Dravidian races created by the philologists”(Sri Aurobindo 2012:37-38). Based on his deep knowledge of numerous languages like Sanskrit, Greek, Latin and Tamil, he came to the conclusion that at some point in history the Indo-Aryan and Dravidian languages were linked and originated from a common source. He says:

\begin{myquote}
“For on examining the vocables of the Tamil language, in appearance so foreign to the Sanskritic form and character, I yet found myself continually guided by words or by families of words supposed to be pure Tamil in establishing new relations between Sanskrit and its distant sister, Latin, and occasionally, between the Greek and the Sanskrit. Sometimes the Tamil vocable not only suggested the connection, but proved the missing link in a family of connected words. And it was through this Dravidian language that I came first to perceive what seems to me now the true law, origins and, as it were, the embryology of the Aryan tongues. I was unable to pursue my examination far enough to establish any definite conclusion, but it certainly seems to me that the original connection between the Dravidian and Aryan tongues was far closer and more extensive than is usually supposed and the possibility suggests itself that they may even have been two divergent families derived from one lost primitive tongue.”(Sri Aurobindo 2012:38)
\end{myquote}


\section*{Sri Sri Ravi Shankar}

Sri Sri Ravi Shankar\index{Ravi Shankar, Sri Sri} is a modern day guru and spiritual leader belonging to the sampradāya of Swami Brahmananda Saraswati, Shankaracharya of Jyotirmath and is a disciple of Maharishi Mahesh Yogi. He is the founder of the Art of Living Foundation\index{Art of Living Foundation}, a nonprofit educational and humanitarian organization. Sri Sri Ravi Shankar was born on May 13th 1956 in Papanasam in Tamil Nadu on the day of Shankara Jayanti and he was given his birth name on the 11th day corresponding to Ramanuja Jayanti, both dates being considered very auspicious in Hindu tradition. As a child he was gifted, and by the age of four he was able to recite the Bhagavad Gītā\index{Bhagavad Gita@Bhagavad Gītā} and was adept in meditation techniques. According to his biography compiled and collated from his devotees, the Shankaracharya-s of Sringeri and Kanchi, as well, as the Shankaracharya of Shivaganga recognized the divinity within him. Avdeeff points out that these are signs of a young saint’s lives, something which Marine Carrin refers to as “exceptional maturity” displayed by saints in their childhood(Avdeeff 2004:322). Another legend is that when he was an infant the hanging cradle on which he was rocking, crashed and the supporting metal chains instead of crushing him, fell outwards miraculously. Inga Bardsen Tollefsen says that there “is a plethora of stories from various sources concerning Shankar’s childhood, and in a hagiographical\index{hagiography} fashion most of them point to the guru he would become” (Tøllefsen 2012:39).In 1982, during a 10-day period of withdrawal and silence, he is said to have received the Sudarṣan Kriyā\index{Sudarsan Kriya@Sudarṣan Kriyā} technique, a central tenet of his teachings. Over the next few decades he and his students reintroduced this technique and trained numerous people globally (Avdeeff 2004:324-326). As per the official website today Art of Living is spread across 155 countries and has “touched the lives of over 370 million people” by teaching them “stress-elimination programs which include breathing techniques, meditation and yoga” which help “overcome stress, depression and violent tendencies.” (The Art of Living 2017)

Sri Sri Ravi Shankar says that “Aryans” did not come into India from somewhere outside and that Indian civilization has been there for thousands of years. He says that the idea of “Aryans” was developed without “taking into account available records of history, astrology and the books that are available” (Sri Sri Ravi Shankar 2008). He refers to the Aryan invasion theory as “complete falsehood” and laments the fact that textbooks still teach the theory (Sri Sri Ravi Shankar 2012). He insists that the term Ārya refers to individuals who are refined, cultured and educated and does not have any racial connotations. He sees the Aryan issue as part of a larger question of the origin of people. The prevalent theory is that people are born in one place and from that focal point they migrate to different places. According to Sri Sri Ravi Shankar\index{Ravi Shankar, Sri Sri} this sort of linear thinking is erroneous and there is no reason why there should be a single origin, and the Purāņa-s prefer an independent multiple origins model (Sri Sri Ravi Shankar 2011). In other words there is no requirement for an Aryans invasion theory where Aryans come from central Asia and displace the native Dravidians. According to him, scholars have attempted to highlight differences and create a north south divide in India using the Aryan-Dravidian framework.

\begin{myquote}
“They made lot of distinction and tried to create a north south divide in India. They pumped up the Tamil by saying, you have been invaded by Aryans and they did complete injustice to you, so they raked the issue and the Dravidian movement started in India which was opposed to north. They opposed Hindi, Sanskrit and anything in the north.” (Sri Sri Ravi Shankar 2008)
\end{myquote}


\section*{Conclusion}

It is evident that all the three authorities we have considered have categorically rejected the racial basis of Aryan-Dravidian divide and spoken against the ideological foundations of the Dravidian program. Moreover, many Hindu spiritual gurus and their sampradāya-s reject the Aryan Invasion or migration theory, and firmly believe that Vedic teachings are Indian in origin. David Frawley\index{Frawley, David} points out that organizations like Self-Realization Fellowship (SRF), Ramanashram (of Ramana Maharshi’s\index{Maharshi, Ramana} lineage), and many members of Maharishi Mahesh Yogi’s\index{Mahesh Yogi, Maharishi} Transcendental Meditation program do not accept the racial Aryan invasion/migration theory (Frawley 2014). Social reformer and independent India’s first Law Minister, Babasaheb Ambedkar\index{Ambedkar, B.R.} (1891 – 1956) rejected the Aryan invasion theory and the racial connotations of the terms “\textit{Ārya}” and “\textit{Draviḍa}”. He considered \textit{Ārya} to be a cultural term and not a race and agreed to Ketkar’s assertion that all the princes in India, whether Aryan or Dravidians were Aryas and that \textbf{“whether a tribe or a family was racially Aryan or Dravidian was a question which never troubled the people of India, until foreign scholars came in and began to draw the line.”} (Ambedkar 2014:21) (emphasis ours)

\begin{myquote}
“The Aryans were not a race. The Aryans were a collection of people.The cement that held them together was their interest in the maintenance of a type of culture called Aryan culture. Anyone who accepted the Aryan culture was an Aryan. Not being a race there was no fixed type of colour and physiognomy which could be called Aryan.”(Ambedkar 2014:419)
\end{myquote}

Using \textit{Nyāya}\index{Nyaya@\textit{Nyāya}} epistemology and \textit{śabda\index{sabda@\textit{śabda}} pramāṇa} or verbal testimony as a means of valid knowledge, we have thus tried to examine the ideological foundations of Dravidian-Aryan divide from a truly indigenous Dhārmic perspective using Indic categories. Despite the small sample set, we can argue that based on the śabda of āpta-s\index{apta@āpta}, the colonial racist theories pertaining to Dravidian-Aryan schism are invalid and have to be discarded.


\section*{Bibliography}

\begin{thebibliography}{99}
\bibitem{chap9-key01} Alexander, Charles N. and Alexander, Victoria K. (1987) “Higher states of consciousness in the Vedic Psychology of Maharishi Mahesh Yogi: A theoretical introduction and research review.” \textit{Modern science and Vedic science} 1(1):89-126.

 \bibitem{chap9-key02} Ambedkar, Babasaheb (2014) “Castes\index{caste} in India.” Pp. 5-22 in \textit{Dr. Babasaheb Ambedkar Writings and Speeches}, edited by Vasant Moon. Mumbai: Dr. Ambedkar Foundation.

 \bibitem{chap9-key03} Ambedkar, Babasaheb (2014) “Shudras and the Counter-Revolution.” Pp. 416-428 in \textit{Dr. Babasaheb Ambedkar: Writings and Speeches}, edited by Vasant Moon and Hari Narake. Mumbai: Dr. Ambedkar Foundation.

 \bibitem{chap9-key04} Avdeeff, Alexis (2004) “Sri Sri Ravi Shankar and the Art of Spreading Awareness over the World.” \textit{Journal of Dharma} XXIX(3):321-335.

 \bibitem{chap9-key05} Banerjee, Aditi (2017) “Rechurning Turbulent Waters of Mahabharata Studies: Removing Poison, Revealing Nectar.” IndiaFacts. Retrieved August 8, 2017 (\url{http://indiafacts.org/rechurning-turbulent-waters-mahabharata-studies-removing-poison-revealing-nectar2/}).

 \bibitem{chap9-key06} Bernard, Theos (2003) \textit{Hindu Philosophy}. Mumbai, Maharashtra: Jaico Publishing House.

 \bibitem{chap9-key07} Christensen, Dorthe R. (2005) “Inventing L. Ron Hubbard: On the Construction and Maintenance of the Hagiographic Mythology of Scientology's Founder.” Pp. 227-258 in \textit{Controversial New Religions}, edited by James R Lewis and Jesper Aagaard Petersen. New York: Oxford University Press.

 \bibitem{chap9-key08} Danino, Michel (2001) “Vedic Roots of Early Tamil Culture.” \textit{Vivekananda Kendra Patrika} 40 No. 1(79):147-159.

 \bibitem{chap9-key09} Danino, Michel (2007) “A Dravido-Harappan\index{Harappan} Connection? The Issue of Methodology.” Paper presented at the International Symposium on Indus Civilization and Tamil Language, 2007, Department of Ancient History and Archaeology, University of Madras, Chennai.

 \bibitem{chap9-key10} Dasgupta, Surendranath (2012) \textit{A History of Indian Philosophy}. Delhi, Delhi: Motilal Banarasidas.

 \bibitem{chap9-key11} Frawley, David\index{Frawley, David} (2014) \textit{Vedic Yoga: The Path of the Rishi}. Twin Lakes: Lotus Press.

 \bibitem{chap9-key12} Malhotra, Rajiv and Aravindan Neelakandan (2011) \textit{Breaking India: Western Interventions in Dravidian and Dalit Faultlines}. Delhi, Delhi: Amaryllis.

 \bibitem{chap9-key13} Mason, Lynne I., C. N. Alexander, F. T. Travis, Marsh G., D. W. Orme-Johnson, J. Gackenbach, D. C. Mason, M Rainforth, and K. G. Walton. (1997) “Electrophysiological Correlates of Higher States of Consciousness During Sleep in Long-Term: Practitioners of the Transcendental Meditation Program.” \textit{Sleep} 20(2):102-110.

 \bibitem{chap9-key14} Nagaswamy, R. (2016) \textit{Tamil Nadu – The Land of Vedas}. Chennai: Tamil Arts Academy.

 \bibitem{chap9-key15} Nandy, Rajarshi (2014) “The Mysticism Of Swami Vivekananda.” Swarajya. Retrieved August 31, 2017 (\url{https://swarajyamag.com/culture/the-mysticism-of-swami-vivekananda}).

 \bibitem{chap9-key16} Neelakandan, Aravindan (2015) “10 Reasons Why Ambedkar Would Not Get Along Very Well With 'Periyar'.” Swarajya. Retrieved Augisy 30, 2017 (\url{https://swarajyamag.com/politics/10-reasons-why-ambedkar-would-not-get-along-very-well-with-periyar}).

 \bibitem{chap9-key17} Pandit, M. P. (2014) \textit{Sri Aurobindo and His Yoga}. Twin Lakes: Lotus Prress.

 \bibitem{chap9-key18} Phillips, Stephen (2011) “Epistemology in Classical Indian Philosophy.” Stanford Encyclopedia of Philosophy. Retrieved August 30, 2017 (\url{https://plato.stanford.edu/entries/epistemology-india/}).

 \bibitem{chap9-key19} Sri Aurobindo (2006) \textit{Autobiographical Notes and Other Writings of Historical Inetrest}. Pondicherry, Pondicherry: Sri Aurobindo Ashram Publication Department.

 \bibitem{chap9-key20} Sri Aurobindo (2012) \textit{The Secret of the Veda}. Sri Aurobindo Ashram Publication Department.

 \bibitem{chap9-key21} Sri Sri Ravi Shankar (2008) “Western Scholars on Indian History.” Guru Kripa - Master's Grace. Retrieved August 29, 2017 (\url{https://gurukripa.wordpress.com/2008/01/23/aryan-theory-by-guruji/}).

 \bibitem{chap9-key22} Sri Sri Ravi Shankar (2011) “July 10 2011 QnA 11.” The Art of Living. Retrieved August 29, 2017 (\url{https://www.artofliving.org/july-10-2011-qna-11}).

 \bibitem{chap9-key23} Sri Sri Ravi Shankar (2012) “10 May 2012 - QA 1.” The Art of Living. Retrieved August 29, 2017 (\url{https://www.artofliving.org/wisdom-q-a-10-may-2012-qa-1}).

 \bibitem{chap9-key24} Talageri, Shrikant (2000) \textit{The Rigveda: A Historical Analysis}. New Delhi: Aditya Prakashan.

 \bibitem{chap9-key25} Talageri, Shrikant (2008) \textit{The Rigveda and the Avesta: the final evidence}. New Delhi: Aditya Prakashan.

 \bibitem{chap9-key26} The Art of Living. (2017) “About Us.” The Art of Living. Retrieved August 29, 2017 (\url{https://www.artofliving.org/in-en/about-us}).

 \bibitem{chap9-key27} Tøllefsen, Inga B. (2012) “A study of the Art of Living Foundation.” MS thesis, Universitetet i Tromsø.

 \bibitem{chap9-key28} Vidyabhushana, Satish C. 1913. \textit{The Nyaya Sutras of Gotama}. Allahabad: The Panini Office.

 \bibitem{chap9-key29} Vivekananda, Swami (2008) \textit{Lectures from Colombo to Almora}. Mayavati, Uttarakhand: Advaita Ashrama.

 \bibitem{chap9-key30} Vivekananda, Swami (2009) \textit{The Complete Works of Swami Vivekananda}. Mayavati, Uttarakhand: Advaita Ashrama.

 \end{thebibliography}

