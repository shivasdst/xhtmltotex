
\chapter{Unique Aspects of Tamil Grammar in Comparison with Sanskrit Grammar – A Bird’s Eye View}\label{intro}

\Authorline{V. Yamuna Devi}


\section*{Abstract}

Effective communication is the prime aim of any language. Tamil and Sanskrit being ancient languages have special features which attract attention of philologists. A comparative study throws much light on the influences of the two languages on each other. For example words like \textit{mīna, mukha} etc, are found in both languages by their mutual borrowing and enrichment. Verbs of Tamil not only indicate the tense and person but also the gender as in \textit{paḍittān} – he read\textit{, paḍittāḻ} – she read. With regard to nouns too Tamil case suffixes are distinct and clear where in some of the Sanskrit nouns are ambiguous as in \textit{harau-} which is nominative dual of \textit{hara} as well as locative singular of \textit{hari} which is understood only in relation to syntax. Sanskrit language has special aspects such as dual number, extensive compounds and \textit{taddhita} derivations.

A study of such differences and parallels in both these languages reveal their contribution to philology and their unique beauty.


\section*{FULL PAPER}

\subsection*{INTRODUCTION}

Effective communication is the primary aim of any language. Language is the autobiography of the human race and the words employed in them are its character; hence it is rightly called 'fossil poetry’ and a linguistic study, reveals the cultural and social aspects of a society.

Every language, ancient or modern, has its own charm and beauty with some special features quite different from those of others. Each language is special; but ascribing superiority to one over the other is ignorance, due to lack of knowledge of other languages.

Tamil is a language spoken by about eleven crore people of the world population and of which six crore are Indians alone as recent survey claims. (Wikipedia - \url{https://g.co/kgs/RvMeMy}) This language has a rich cultural heritage and is very ancient. Sanskrit, once a spoken language of a large population in ancient times is now a language spoken by a much smallernumber. Yet Sanskrit language is still growing in its literary worldwith many modern writers and poets adding to its treasure. In ancient times Sanskrit was probably a common official language in many parts of India.

The uniqueness of Tamil is that it continues to be a spoken language and is also a source for thousands of people to write poems and prose even now. A language exists as long as it is used by the people in the literary world andany language is kept intact by its grammar.

The south Indian languages have been termed as Dravidian languages of which Tamil, Telugu, Kannada and Malayalam are the major ones. Dravidian theories have been proposed by scholars with various views. All the theories and studies led to the conclusion that the Dravidians are native to this land.

Both theselanguages Sanskrit and Tamil, existing in the same land has influenced one another. Not only do these languages have mutual influences but Tamil has influenced all languages in south India and Sanskrit is regarded as the mother of the languages of north India.

A comparitive study of these mother languages help in understanding them better and to appreciate their unique features. This further aids in understanding the languages and culture of other parts of India.Their special aspects attract the attention of philologists who can evolve the overall overlap among these languages. In understanding a language, grammatical traditions play a vital role.


\subsection*{Tamil Grammatical Tradition}

A brief survey of the grammatical texts in Tamil are mentioned below\endnote{\textit{History of Grammatical theories in Tamil and their relation to the grammatical literature in Sanskrit}, The Kuppuswami Sastri Research Institute, Chennai, 1997, p.1-15.}:

\begin{enumerate}[{\rm 1)}]
\itemsep=0pt
\item Tradition holds that Agastya formulated the first grammatical text of Tamil and imparted it to his 12 disciples:(Subrahmanya Sastri 1997: 1-15)(index Subrahmanya Sastri, P.S.)

\begin{longtable}{ll}
1. Tolkāppiyaṉār \& 7. Paṉampāraṉ \\
2. Ataṅkoṭṭācāṉ \& 8. Kaḻārampaṉ \\
3. Turāliṅkaṉ \& 9. Avinayaṉ \\
4. Cempūṭcey \& 10. Kākkaipāṭiṉiyaṉ \\
5. Vaiyāpikaṉ \& 11. Naṟṟattaṉ \\
6. Vāyppiyaṉ \& 12. Vāmaṉaṉ 
\end{longtable}

 \item Iḻampūraṇar also known as Uraiyāciriyaris the earliest known commentator on \textit{Tolkāppiyam} ( 10th C.A.D.)

 \item Puttamittiraṉar author of \textit{Vīracoḻiyam} (11. C.A.D.)

 \item Peruntevaṉār commentator on \textit{Vīracoḻiyam}(11th or 12 C.A.D.)

 \item Kuṇavīrapaṇṭitar wrote \textit{Nīminātam} (13th C.A.D.)

 \item Pavaṇanti composed \textit{Naṉṉūl} (13th C.A.D.)

 \item Mayilainātar earliest commentator on \textit{Naṉṉūl} (14th C.A.D.)

 \item Ceṉāvaraiyar, Teyavaccilaiyār and Nacciṉārkkiṉiyar – commentators on \textit{Tolkāppiyam}

 \item Cuppiramaṇiya – tīṭcitar author of \textit{Pirayokaviekam} (17th C.A.D.)

 \item Vaittianāta – tecikar wrote \textit{Ilakkaṇaviḻakkam} - (17th C.A.D.)

 \item Cuvāmināta – tecikar author of \textit{Ilakkaṇakottu} - (17th C.A.D.)

 \item Caìkaranamaccivāyappulavar commentator on \textit{Naṉṉūl} - (17th C.A.D.)

 \item Rev. C.J.Beschi author of \textit{Toṉṉūlviḻakkam} - (18th C.A.D.)

 \item Civañāṉamuṉivar wrote \textit{Tolkāppiyamutaṛ}-\textit{Cūttiravirutti} and \textit{Ilakkaṇa\-viḻakka}- c- \textit{cūṟāvḻi}.

 \item A few other commentators on \textit{Tolkāppiyam}and \textit{Naṉṉūl}.

\end{enumerate}


\subsection*{A brief note on the Tamil grammatical works}

\begin{enumerate}[{\rm 1)}]
\itemsep=0pt
\item \textit{Tolkāppiyam} contains 1600 \textit{sūtra-s} in three \textit{atikāram-s}. \textit{Eḻuttatikāram} deals with phonology in nine chapters called \textit{iyals.} First three deal with sounds and last six with \textit{sandhi}. The second is the \textit{Collatikāram} pertaining to the syntax and morphology of Tamil language compiled in nine chapters. The third \textit{Poruḻatikāram} deals with science of poetics in nine chapters.

 \item The author of \textit{Vīracoḻiyam} improves upon \textit{Tolkāppiyam} citing a few references from literature of his times which were not in the time of \textit{Tolkāppiyam}.

 \item Incorporating all the concepts contained in the above two (\textit{Tolkāppiyam} and \textit{Vīracoḻiyam}) the author of Naṉṉūl adds a few more in a concise manner. But he differs from Tolkāppiyam in not following the general principle that one \textit{sūtra} should have only one \textit{vidheya} or logical predicate with reference to one \textit{uddeśya} or logical subject.

 \item \textit{Ilakkaṇaviḻakkam} is a critical study of \textit{Tolkāppiyam} and \textit{Naṉṉūl} and hence is more like a commentary than an independent work.

 \item \textit{Prayokavivekam} is a treatise modelled on Sanskrit grammar.

 \item \textit{Ilakkaṇakkottu} acts as a compendium of Tamil grammatical works.

\end{enumerate}


\subsection*{Sanskrit grammatical tradition}

The \textit{Prātiśākhya-s} and \textit{Nirukta} of Yāska who predated Pāṇini, have the grammatical rules and etymology for Vedic words. The first extant systemic grammatical treatise in Sanskrit is the \textit{Aṣṭādhyāyī} of Pāṇini. In eight \textit{adhyāya-s} consisting of about 4000 \textit{sūtra-s,} the whole of grammar is dealt with. It belongs to about 6th to 4th century BCE. These \textit{sūtra-s} were supplemented, modified and corrected by Kātyāyana with his \textit{Vārttika-s} to suit the language of his times by about 5th century BCE. Patañjali wrote the commentary called the \textit{Mahābhāṣya} for the \textit{sūtra-s} of Pāṇini by the 2nd century BCE.

Other schools of grammar in Sanskrit also developed such as the Kātantra, Cāndra, Aindra, and so on. Yet the Pāṇinian system is the most popular and widely followed school of Sanskrit grammar.

It is opined by many scholars and philologists that Tolkāppiyaṉār composed the \textit{Tolkāppiyam} on the model of \textit{Aṣṭādhyāyī} of Pāṇini. Another school claims \textit{Tolkāppiyam} to be independent, devoid of any influence from other texts. Yet a comparative study of the two grammars reveals aspects of similarities and dissimilarities. Some of these are presented here in brief.


\section*{PHONOLOGY}

\subsection*{The Alphabets:}

There are thirty primary sounds in Tamil. The vowel is aptly called \textit{uyireḻuttu}- a life letter and the consonant without a vowel is body without soul and hence termed as \textit{meyeḻuttu.} In Sanskrit the vowels are called \textit{svara-s} – existing independently and \textit{vyañjana-s} – those dependent. Thus the Tamil Grammarians have shown their deep understanding in naming of the vowels and consonants as \textit{uyir} and \textit{mey,} unique to Tamil Language.

The Tamil terminologies for the letters are more expressive than that of Sanskrit.

According to Tolkāppiyaṉār\endnote{\textit{Eḻuttena-p-paṭupapa}

\textit{Akaramutal}

\textit{ṉakara-v-iṟuvāymuppa:. teṉpa}

\textit{Cārntu-varaṉ marapiṉ mūṉṟalań kaṭaiye} \textit{Tol}. E. 1

\textit{Aukāra-v-iṟuvāy-p}

\textit{paṉṉi r-eḻuttu m-uyir-ena moḻipa} \textit{Tol}. E. 8

\textit{ṉakara v-iṟuvāy p}

\textit{patiṉeņ-e ḻuttu mey-yena moḻipa} \textit{Tol}. E. 9} the primary sounds are thirty in number consisting of 12 vowels and 18 consonants and he excludes \textit{kuṟṟiyalikaram} and\textit{ kuṟṟiyalukaram}. Except author of \textit{Vīraccoḻiyam} who adds \textit{āytam} between vowels and consonants and takes the primary sounds to thirty one, others follow Tolkāppiyaṉār.

\begin{longtable}{|p{2.8cm}|p{2.8cm}|p{2.8cm}|}
\hline
 & TAMIL  - 30 & SANSKRIT \\
\hline
Simple vowels & Five  - \textit{a i u e o} & Five - \textit{a i u ṛḻ} \tabularnewline
\hline
Long Simple Vowel sounds & Five - \textit{āīū ê ô} & Eight - \textit{āīūṛ e ai o au} \tabularnewline
\hline
 & Ordinary Diphthong sounds \textit{ai and au} & \textit{ayogavāhau– am} and \textit{au} \tabularnewline
\hline
Consonants & 18 \textit{K ṅ c ñṭṇ t  n  p m y r l v ḻḻ ṟ and ṉ} & 33 \textit{K kh g ghṅ c ch j jhñṭṭhḍḍhṇ t th d dh n p ph b bh m y r l v śṣ s h} \tabularnewline
\hline
\end{longtable}

The secondary classification of the letters are more elaborate in both the languages with unique feature of classification.

Later grammarians divided the speech sounds in Tamil as \textit{mutaleḻuttu}- primary and \textit{cārpeḻuttu} secondary. \textit{Nannūl}. 59-61 –

\begin{verse}
\textit{uyirum uṭampumā muppattu mutale}.\\\textit{Uyir mey āytam uyiraḻapu oṟṟalapu}\\\textit{A:. kiya i-u ai-au ma:.kāṉ}\\\textit{Taṉi-nilai pattuñ cārpeḻut t-ākum} 
\end{verse}

Thus till the 12th century CE the Tamil sounds existed as they were. Before Naṉṉūlār it seems that the pronunciation of the secondary sounds of \textit{i} shortened \textit{u} and.: had changed; for Tolkāppiyaṉār states that their pronunciation varies according to the nature of their neighboring sounds and hence their organs of production are not the same. But on the other hand, Naṉṉūlār has mentioned that the organ of articulation of \textit{i} is that of ‘\textit{i}’, that of shortened \textit{u} is \textit{u} and that of.: is head.

The \textit{āytam} which was pronounced in six ways as a guttural, palatal, alveolar, dental, labial and cerebral sound began to have only one sound – the guttural in later Tamil.

\textbf{Note:}

\begin{enumerate}[{\rm 1)}]
\itemsep=0pt
\item The comparative study of the two languages also reveal that the Tamil pronunciation of ‘a’ may have influenced the South Indians to pronounce the Sanskrit ’a ’ as open vowel at the end of words such as – naya, vada, paṭha, etc., though it is a close one according to \textit{Aṣṭādhyāyī} (VIII.4.68) and pronounced by the north Indians in that form.

 \item 
 It should be noted at this point that the letters e and o are diphthongs in Sanskrit while they are always simple sounds in Tamil. For in Tamil a + i = a-v-ias in a-v-viṭam

 \textit{ā} +\textit{i} = \textit{ā-y-i}as in ā-y-irutiṇai

 \textit{a} + \textit{u} = \textit{a-v-u} as in pala-v-uṇṭu

 \item Tolkāppiyaṉār says that ai may be split into a and i (Tol. E. 54) – akaraikara m-aikāram- ākum.

\end{enumerate}

He also adds that ay may be used instead of ai (Tol. E. 56) –

\begin{verse}
Akara-t- t-imparyakara-p puḻḻiyum\\ Ai-y-e ṉeṭuñ-ciṉaimey-peṟa-t toṉṟum
\end{verse}

Considering the above P.S. SubrahmanyaSastri\endnote{\textit{History of Grammatical theories in Tamil and their relation to the grammatical literature in Sanskrit}, p.31.} opines (index Subrahmanya Sastri,P.S.) that the way in which Tamil aiwas, and is\break pronounced, may have influenced the pronunciation of Sanskrit ai in the pre - Christian era.

After the time of Naṉṉūlār more sounds were added among Tamil consonants –

\begin{myquote}
\textit{k} – began to be pronounced as\textit{k, g} and \textit{h} in different places as in \textit{kaṭṭu, taṅgai} and \textit{akam}.
\end{myquote}

\begin{myquote}
\textit{c} – was pronounced as \textit{c, j} and \textit{ś} viz. \textit{taccantañjai}and śaṭṭi
\end{myquote}

\begin{myquote}
\textit{ṭ –} began to be pronounced as \textit{ṭ}and \textit{ḍ} - \textit{taṭṭān, paṇḍam} and aḍai
\end{myquote}

\begin{myquote}
\textit{t –} was to be pronounced as \textit{t} and \textit{d} as in \textit{attai, tandai}and pudai.
\end{myquote}

\begin{myquote}
\textit{p –} began to be pronounced as \textit{p} and \textit{b} like appam, pambaram.
\end{myquote}

It is clear from the above that though seven sounds were newly introduced no new symbol was incorporated to represent them independently. This made the Tamil script un-phonetic.

By the influence of Sanskrit on Tamil a few of the following letters are also incorporated from Grantha script to Tamil. They are – \textit{jaśañasaha} \tamil{ ஜ், ஷ், ஶ், ஸ், ஹ்,}. But some Tamil scholars do not acknowledge the use of such letters.


\subsection*{Dissimilarities between Tamil and Sanskrit phonology -}

\begin{enumerate}[{\rm 1)}]
\itemsep=0pt
\item As already stated the e and o are simple sonants in Tamil while they are diphthongs in Sanskrit.

 \item In Tamil language, \textit{sandhi} is effected between two words only if the meaning is known which is not necessary in Sanskrit. The sandhi in Tamil language is divided into case related sandhi and non-case related sandhi. But with regard to Sandhi of words in Sanskrit, several aspects of internal sandhi (sandhi within a word) differ from external sandhi (sandhi between words). This plays a vital role in proving that the two languages are fundamentally different.

 \item Normally, in Tamil language words do not begin with conjunct consonants. Example

\end{enumerate}

\begin{longtable}{p{2.5cm}p{2.5cm}}
\textbf{Sanskrit} & \textbf{Tamil} \\
\textit{Klṛpta} & \textit{kilupta} \\
\textit{Prathamā} & \textit{Piratamai} \\
\textit{Kriyā} & \textit{Kiriyai} \\
\end{longtable}


\section*{MORPHOLOGY}

The word or \textit{Col}\endnote{\textit{Ellā collum poru ḻ kurit tanave.} (\textit{Tol. Col}. 155)} in Tamil Language is defined by grammarians\endnote{Except the author of \textit{Vīraccoliyam} who follows Pāņini in defining the word as \textit{suptińantam padam,} where \textit{sup-anta} is noun and \textit{tiń-anta -} a verb.} as that which conveys sense unlike Sanskrit in which a word is defined as that which has a suffix \textit{suptińantam padam} (Pā.I.IV.14).

More importantly the gender in Tamil language is in accordance with the natural gender where as in Sanskrit the grammatical gender is recognized as a category distinct from natural gender.

Broadly \textit{Col} is \textit{peyar} and \textit{vinai} and secondarily \textit{iṭai} and \textit{uri-c-col}. This is similar to Yāska’s (\textit{NiruktaVIII}.8) \textit{nāma-ākhyāta-upasarga-nipāta}.

The genders of words in Tamil are simple. The masculine nouns are gods and men; feminine gender for goddesses and women and all inanimate in neuter gender.

Tamil has only the singular and the plural.

\newpage

The case suffixes are thus 16 in all (8 cases in singular and plural) and do not undergo much transformation. The philologists have identified Tamil as an agglutinating language in which any number of affixes are glued on to the root element in order to express the idea.

\begin{longtable}{lll}
I Case & - & \textit{makaṉ} \\
II Case  & \textit{ai} & \textit{makaṉai} \\
III Case & \textit{oṭu, āl, ān} & \textit{makaṉoṭu etc.} \\
IV Case & \textit{ku} & \textit{makaṉukku} \\
V Case & \textit{iṉ,il} & \textit{makaṉil} \\
VI Case & \textit{atu} & \textit{makaṉatu} \\
VII Case & \textit{il, etc.} & \textit{makaṉil} \\
\end{longtable}

While Sanskrit is inflectional language where the root itself and affixes may be modified in sound and shape. Hence in Sanskrit every word ending differently is declined variously. \textit{Hari}, \textit{pati} though both are in masculine gender and ending with \textit{i} have different forms in various cases. In many instances verbs are also completely unidentifiable. Also the word \textit{harau} in Sanskrit exists in the locative singular of the word \textit{hari} and in the nominative and Accusative of the word \textit{hara}. These are identified in syntax alone or with the context. Thus Tamil is simpler than Sanskrit.

Morphologically both nouns and verbs in Tamil are classified into rational and irrational. Rational words are those which denote rational beings and irrational words are those which denote all other than rational beings. Tamil names them as \textit{Uyar-tiṇai}.

\begin{verse}
\textit{Uyartiṇai Y-eṉmaṉārmakkaṭ-cuṭṭe}\\\textit{ A:.riṇai y-enmanār-avaralapirave}\\\textit{ā-y-irutiṇaiyi-n--icaikkumanacolle} ( \textit{Tol}.col -1).
\end{verse}

Rational words are classified under three heads –

That denoting a single male – \textit{āṭūvaṛicol} or \textit{āṇpāl},

Those that denote a single female – \textit{makaṭūvaṛicol} or \textit{penpāl}

 Those that denote a male and female or males or females or both - \textit{palloraṛiyuñcol} or \textit{palarpāl}.

\begin{verse}
\textit{A:.riṇai} is divided into two \textit{pals} – \textit{oṇrāṇpāl} and \textit{palaviṇpāl.}
\end{verse}

Irrational words are classified under two heads those that denote one irrational being - \textit{oṇraṛicol}or \textit{oṇrānpāl} and those that denote more than one irrational beings- \textit{palavaricol} or \textit{palavinpāl}.

\vskip 2pt

Hence in Tamil language gender and number are not taken as separate entities and the \textit{pāl} or the gender and number are determined more from their meaning than their ending. Example – \textit{peṇmakan} – is feminine singular, \textit{makkaḻ} – epicene plural and so on. Hence the oblique cases are formed by adding the case- suffixes to the forms \textit{alavan} and \textit{makkaḻ}. But, in Sanskrit, nouns which do not end in \textit{a,} consonant or \textit{‘i’} generally denote their grammatical gender only through their suffix.

\vskip 2pt

In pronouns like \textit{avaṇai}, \textit{avaḻai}, \textit{avarai} gender, number and case are \textit{NOT} denoted by \textit{3} separate suffixes but only by\textit{ 2} suffixes of which one denotes both gender and number, the other - case.

\vskip 2pt

Sanskrit resembles Tamil in having two suffixes denoting the three - gender, number and case in the oblique cases of nouns and differs from it in having one suffix denoting gender alone and another denoting both number and case. Eg. D\textit{evasya} ‘\textit{a’} of Deva denotes masculine gender and ‘\textit{sya’} denotes singular number and genetive case. In Tamil the word \textit{Tevanai},\textit{ n} denotes masculine singular and \textit{ai} denotes accusative case. \textit{Tevarai}, \textit{r} denotes epicene plural and \textit{ai}denotes accusative case.On this basis P.S.S Sastri (index Subrahmanya Sastri,P.S.) opiṇes\endnote{\textit{Comparative grammar of the Tamil Language}, Tiruvadi, Tanjore, 1947, p.13.} that it is safer to inculde the Dravidian languages among the amalgamating inflectional languages.

\vskip 2pt

Old Tamil may have the same from in \textit{a}:.\textit{riṇai} nouns both in singular and plural: for instance the word \textit{māṭu} may mean both ox and oxen. In the sentence \textit{māṭuvantatu}, it is a singular because it takes a singular verb and in the sentence \textit{māṭuvantana}, it is plural since it takesa plural verb. Similarly it is singular if it is qualified by the word \textit{oru}, as in \textit{orumāṭu} and it is plural if it qualified by words which mean more than one, as in \textit{iraṇṭumāṭu} or \textit{irumāṭu} etc.

The use of \textit{cāriyai -} flexion increments between the stem and the case-suffix which is found in Tamil is not easily recognizable in Sanskrit, except the \textit{n} which is inserted between the stem ending vowel and the genitive plural suffix (\textit{rāmāṇām}) and the instrumental singular suffix (\textit{rāmeṇa}), through similarity with the corresponding cases of stems ending in \textit{n} (\textit{guṇinām} and \textit{guṇinā}).

\subsection*{Verb morphology in Tamil differs from that of Sanskrit}

The terminations \textit{n, ḻ, r, tū} and \textit{a} invariably denote masculine singular, feminine singular, epicene plural, neuter singular and neuter plural; for instance the Tamil words \textit{ceytān}, \textit{ceytāḻ} and \textit{ceykiratu} respectively mean, “he did, she did”, and “it did” respectively and \textit{ceytār} and \textit{Ceytana} respectively mean ‘they humans and non-humans did.’ But in Sanskrit a single termination denotes the masculine singular, feminine singular and neuter singular and another denotes masculine plural, feminine plural and neuter plural; for instance the Sanskrit word \textit{akorot} means ‘he did’ ‘she did’ and ‘it does’ and the word ‘\textit{akurvan’} means ‘they did’.

The element \textit{t} in the Tamil word \textit{ceytān} is considered to denote the past tense, while the initial \textit{a} in \textit{akorot} is considered to denote the same.

The Tamil verbal system is much simpler than the Sanskrit verbal system.

In Tamil language there are only three moods -

\begin{longtable}{|m{2.4cm}|m{6.7cm}|}
\hline
i) Indicative & Has different forms in the three persons, two numbers and three tenses (past, present and future) \\
\hline
ii) Optative & 
							
\begin{enumerate}\item Same form both in singular and plural 
								\item Used only in the III person in Old Tamil
							
\end{enumerate}
Extended to all persons  in later Tamil

						 \tabularnewline
\hline
iii) Infinitive & 
							\textbf{\textit{PeyarEccam}}

							
\begin{enumerate}\item Qualifies noun
								\item Same form in all persons and numbers
								\item In old Tamil same form was used both in the present and the future tenses and different from in past tense. But in later Tamil the old form was restricted in denote the future and a new form is introduced to denote present.
							
\end{enumerate}
\textbf{\textit{Vinai-y-eccam}}

							
\begin{enumerate}\item Governs a verb and hence termed so.
								\item Has same from in all tenses, persons and numbers through it has different forms to denote different kinds of infinitive action.
							
\end{enumerate}
 \tabularnewline
\hline
\end{longtable}

These are illustrated below:

\textbf{Indicative mood – Past tense}

\begin{longtable}{|c|c|c|}
\hline
 & \textbf{Singular} & \textbf{Plural} \\
\hline
I person & \textit{ceytēṉ} (I did) & \textit{ceytōm} (we did) \tabularnewline
\hline
II person & \textit{ceytāy}  (you did) & \textit{ceytīr} (you did) \tabularnewline
\hline
III person & \textit{ceytān}  (he did) & \textit{ceytār} (the human beings did \tabularnewline
\hline
 & \textit{ceytāḻ}  (She did) &  \tabularnewline
\hline
 & \textit{ceytatū} (it did) & \textit{ceytaṉa} (the non human beings did) \tabularnewline
\hline
\end{longtable}

\newpage

\textbf{Present tense – old Tamil}

\begin{longtable}{|l|l|p{3.9cm}|}
\hline
 & \textbf{Singular} & \textbf{Plural} \\
\hline
III person & \textit{avaṉceyyum} (he does) &  \tabularnewline
\hline
 & \textit{avaḻceyyum} (she does) &  \tabularnewline
\hline
 & \textit{atuceyyum} (it does) & \textit{avaiceyyum}\supskpt{\endnote{\textbf{\textit{c}} is pronounced as \textbf{\textit{c}} in Tinnevelly region and as \textbf{\textit{ç}} in Tanjavur and Trichy districts.}} (they – non – human beings- do) \tabularnewline
\hline
\end{longtable}

\textbf{Modern Tamil}

\begin{longtable}{|l|l|p{4.2cm}|}
\hline
I person & \textit{ceykiṛēṉ} (I do) & \textit{ceykiṛōm} (we do) \\
\hline
II person & \textit{ceykiṛāy} (you do) & \textit{ceykiṛīr} (you do) \tabularnewline
\hline
III person  & \textit{ceykiṛāṉ} (he does) & \textit{ceykiṛār} (They –human beings do \tabularnewline
\hline
 & \textit{ceykiṛāḻ} (She does) &  \tabularnewline
\hline
 & \textit{ceykiṛatu} (it does) & \textit{ceykiṉraṉa} (they non- human   beings do) \\
\hline
\end{longtable}

\textbf{Future tense – Modern Tamil }

\begin{longtable}{|l|l|p{4cm}|}
\hline
I person & \textit{ceyvēṉ} (I shall do) & \textit{cevōm} (we shall do) \\
\hline
II person & \textit{ceyvāy} ( you will do) & \textit{ceyvīr} (you will do) \tabularnewline
\hline
III person & \textit{ceycāṉ} (he will do) & \textit{ceyvār} - (they- human beings will do) \tabularnewline
\hline
 & \textit{ceyvāḻ} (she will do) &  \tabularnewline
\hline
 & \textit{ceyyum} (it will do) & \textit{ceyyum} - they non human beings – will do \tabularnewline
\hline
\end{longtable}

\textbf{Optative mood – old Tamil}

\begin{longtable}{|l|l|p{3.8cm}|}
\hline
III person & \textit{avaṉceyka} (let him do) & \textit{avarceyka} – (let them - human beings do \\
\hline
 & \textit{avaḻceyka} (let her do) &  \tabularnewline
\hline
 & \textit{atuceyka} (let it do) & \textit{avaiceyka}  (let them-non human beings –do \tabularnewline
\hline
\end{longtable}

\textbf{Modern Tamil}

\begin{longtable}{|l|l|l|}
\hline
I Person & \textit{nāṉceyka} & \textit{nāmceyka} \\
\hline
II Person & \textit{nīceyka} & \textit{nīrceyka} \tabularnewline
\hline
III Person & \textit{avaṉceyka} & \textit{avarceyka} \tabularnewline
\hline
 & \textit{avaḻceyka} &  \tabularnewline
\hline
 & \textit{atuceyka} & \textit{avaiceyka} \tabularnewline
\hline
\end{longtable}

\textbf{Infinitive mood – Peyar – eccam or relative participle}

\vskip 3pt

\textbf{Old Tamil}

Present and future relative participle – \textit{ceyyum}\\ Past relative participle- \textit{ceyta}

\vskip 3pt

\textbf{Modern Tamil}

Past relative Participle – \textit{ceyta}\\ Present relative participle - \textit{ceykiṉra}\\ Future relative participle - \textit{ceyyum}

\vskip 3pt

\textbf{\textit{Viṉai-y-eccam}}

Old Tamil – ceytu, Ceypu, ceyiṉ, ceya, ceyaṛku etc., ceyteṉa, ceyyiya

Modern Tamil– \textit{ceytu, ceyyā, ceytāl, ceya etc}.

The negative forms of verbs are interesting in Tamil. Though in Sanskrit the negative \textit{na} 'not' is added before the verb, an inflexive verb is exclusive to Tamil language as –

\begin{myquote}
\textit{paṭiyen, paṭyāy, paṭiyān, paṭiyāḻ, paṭi}yātu\\\textit{paṭiyom, paṭyīr, paṭiyīrgaḻ, paṭiyār, paṭiyārgaḻ, paṭyā}\\\textit{naṭaven, naṭavāy, naṭavān, naṭavāḻ, naṭavātu}\\\textit{naṭavom, naṭavīr, naṭavīrgaḻ, naṭavār, naṭavārgaḻ, naṭavā}
\end{myquote}

These negative forms are pure negative tense as above, negative present, negative past tense, negative Imperative and negative verbal participle, negative relative participle and negative participial nouns.

The verbal system in Sanskrit is very complicated with more moods – indicative, imperative, subjunctive, benedictive, injunctive etc.; Tenses are three, with further details as perfect, imperfect and aorist in past tense; and first and second futures. The verbs have two terminations as \textit{parasmaipadi} and \textit{ātmanepadi}. Passive voice is the specialty of Sanskrit language.

The \textit{peyar-eccam} in Tamil is in Sanskrit, declinable like nouns which take different forms according to the gender, number and case of the nouns which they qualify.

There is no negative voice in Sanskrit. 

Regarding the causal verbs Tamil language has the capacity to form verbs in the second move, third move etc. as \textit{kāṭṭinān, kāṭṭiviṭṭān}, \textit{kāṭṭivippitān, kāṭṭivippivittān} while in Sanskrit language there is only one move causal form.

This double and triple causal forms are also later developments in Tamil Language such as –

\begin{verse}
\textit{Uḻapikkum - Uḻapikkuñ cūtu} (\textit{Kuraḻ}. 938)
\end{verse}

\begin{verse}
\textit{āṭṭuvittāl – āṭṭuvittālār oruvar āṭātāre} (\textit{Tevāram}.1229 stanza.3)
\end{verse}

\textit{Vīraccoḻiyam} (\textit{V}. Tāt.6) says that \textit{āṭṭu ārru} are causals or \textit{kāritam} and if \textit{vi} and\textit{ pi} are added twice after them as in \textit{āṭṭuvippān} it would be triple causal or \textit{kāritak-kāritak-kāritam}.

\textbf{SYNTAX}

Connecting two or more finite verbs by copulative particle \textit{ca} is one aspect of Sanskrit language as - \textit{rāmaḥ vanam agamat tatra uvāsa ca} such a usage is not permitted in Tamil language.

Tamil has a special usage wherein the verb is in plural when the gender of person is unknown as \textit{oruvarvantār, avanoavaḻovantār.} Such usage is ungrammatical in Sanskrit as the singular verb has same form for masculine and feminine genders.

With regard to Absolutes, Tamil has usages as –\textit{katiravan tonra nām cenrom} (Sun rising, we went) while such usage is absent in Sanskrit which otherwise has genitive and locative absolutes as -

\begin{verse}
\textit{Paśyataḥ guroḥ māṇavakāḥ śabdam kurvanti |}\\\textit{Rāme vanam gate daśarathaḥ pañcatvam avāpa|}
\end{verse}

The declension of nouns in Tamil is much easier than in Sanskrit.

Tamil case suffixes are distinct and clear where as some of the Sanskrit nouns are ambiguous as in \textit{harau-} which is nominative dual of \textit{hara} as well as locative singular of \textit{hari} which is understood only in relation to syntax.

Third case in the sense of association is used in Sanskrit with a special purpose of denoting the noun of lesser importance. \textit{Rāmaḥ lakṣmaṇena vanam gataḥ.} Here Rāma is the principal noun and Lakṣmaṇa the secondary hence the latter is in third case. This was the reverse in old tamil (Tol. Col. 91) – \textit{oruvanai –y-oṭu-c-col –uyar-pin vaḻitte}. But in later Tamil this was indiscriminately used.

\textbf{Vocabulary}

In the scientific study of a language, phonology, accidence, syntax and vocabulary are like the four legs of a chair. Words are the current coins of a language. Just as the value of a coin differs from time to time and also goes out of circulation, words also undergo change in structure, sense and sometimes even become obsolete.

The \textit{Tolkāppiyam} in the Poruḻatikāram which deals with science of poetics and classifies words into four categories - \textit{iyarcol} indigenous words, \textit{tiricol} – indigenous words which have changed their form having been used in poetry or by passing of time, \textit{ticai-c-col} – words borrowed from Kannada, Telugu, Malayalam and other languages spoken in the neighbouring states and \textit{vaṭacol –} borrowed from Sanskrit.

The \textit{tiricol} words give an idea as to how the words have changed their meaning over a period of time. In old Tamil \textit{peṇḍāṭṭi} denoted a woman while in later days it denotes a wife. \textit{Kombu} originally denoted the branch of a tree while later it meant a musical instrument made of tree branch. \textit{Olai} referred to a palm or coconut leaf in early times while later it meant an ear ornament.

Thus a diachronic study of semantics also throws much light on how the modes of expression and in turn the culture of the society has been changing.

Also the grammarians had seen that the close association of other languages had compelled people and poets to adopt words from other languages. While adopting them to Tamil certain linguistic transformation in words was also observed like the \textit{tatsama}or \textit{tadbhava}. For example words like \textit{jala} (water) from Sanskrit were used as it is \textit{tatsama} by just adding \textit{m} – \textit{jalam} and also as \textit{tadbhava} or Tamilised as \textit{calam}.

The \textit{Vīracoliyam} (Tat.8.) mentions the following rule for the Tamilisation of Sanskrit words –

\begin{verse}
\textit{Kūṭṭeḻuttin- pinya-ra la-k-kaṭonriṭirkūṭṭiṭaiye.}\\\textit{Oṭṭeḻuttāka-p-per}um-or-ikāram ….
\end{verse}

Some examples are

\begin{longtable}{|l|l|}
\hline
\textbf{Sanskrit} & \textbf{Tamil} \\
\hline
\textit{pathya} & \textit{pattiya} \tabularnewline
\hline
\textit{lakṣya} & \textit{ilakkiya} \tabularnewline
\hline
\textit{śukla} & \textit{sukkila} \tabularnewline
\hline
\textit{pattra} & \textit{pattira} \tabularnewline
\hline
\end{longtable}

It may also be observed that in the process of \textit{tatsama} and \textit{tadbhava} the meaning is sometimes retained as in words like \textit{mīna, mukha} etc,.While many a times the sense also changes as in the following –

\begin{longtable}{|l|l|}
\hline
\textbf{Sanskrit} & \textbf{Tamil} \\
\hline
\textit{nija} (own) & \textit{nijam} (Truth) \tabularnewline
\hline
\textit{avasara} (opportunity) & \textit{avasara} (haste) \tabularnewline
\hline
\textit{pramāda} (mistake or haste) & \textit{pramāda} (excellence) \tabularnewline
\hline
\textit{lābha} (acquisition) & \textit{ilābha} (profit) \tabularnewline
\hline
\textit{balavanta} (strong person) & \textit{balavantam} ( by force/pressure) \tabularnewline
\hline
\textit{lakṣmaṇa} & \textit{ilakkumanan} \tabularnewline
\hline
\end{longtable}

A comparative study of these languages throws much light on the influences of both languages on each other. Both the languages by their mutual borrowing have been enriched. Verbs of Tamil not only indicate the tense and person but also the gender as in \textit{paḍittān} – he read\textit{, paḍittāḻ} – she read. Sanskrit language also has some special aspects of dual number, extensive compounds and \textit{taddhita} derivations.

As seen above in many aspects Tamil is much simpler. There are minimum number of alphabets. Agglutinative nature of Tamil language makes it easier than Sanskrit language. In Sanskrit language, the identification of number and gender becomes terse or impossible in cases like \textit{te} which is a plural of masculine, dual of feminine and dual of neuter too of the root word \textit{tad}. Similar are words like \textit{Harau} and so on. The ten forms of verbs with different tenses and moods and two voices in Sanskrit makes the verb forms too complicated.

With simple cases, tenses and moods Tamil has a felicity of expression that is easier than Sanskrit. Also Tamil has influenced other Dravidian languages like Telugu, Kannada and Malayalam. The languages of north India have also been greatly drawn from Sanskrit. Thus these two source languages have mutually influenced the languages of our country and in turn form the base of Indian culture.

The differences that existed between old Tamil and ancient Sanskrit shows the independent development of these languges. It is also clear that Tamil was not allied to Indo-European language but independent.

A study of such differences and parallels in both these languages reveal their contribution to philology and their unique beauty. Each has enriched the other by adding to and also adopting certain aspects from the other language. They have mutually contributed to the literary and linguistic growth.

As Daṇḍin rightly remarks\endnote{\textit{ikṣukṣīraguḍādīnāṃ mādhuryasyāntaraṃ mahat |}\\\textit{tathāpi tadākhyātuṃ sarasvatyāpi na śakyate || (Kāvyādarśa I. 102)}}- there is vast difference in the sweetness of sugarcane, milk, jaggery and others and that even goddess Sarasvatī would be short of words to express it. This holds good in case of languages too– their beauty, specialities and socio-cultural contributions are unique, distinct, each evolving as a mode of expression based on the evolution of social mileu. A study of the evolution of languages cannot be restricted to a particular time-frame as language is dynamic and undergoes continuous change to adapt itself to the ever-changing world that includes social,cultural, economic, scientific, political and philosophical change.


\section*{Bibliography}

\begin{thebibliography}{99}
\bibitem{005-key01} Agesthialingom S. and Subrahmanyam,P.S.(Ed.)(1976) \textit{DravidianLinguistics-V}. Annamalai University. AnnamalaiNagar.

 \bibitem{005-key02} Caldwell, Robert (1976) \textit{A comparative grammar of the Dravidian or South Indian Family of Languages}. University of Madras. Chennai.

 \bibitem{005-key03} Kumar, Shashiprabha (Ed.) (2007) \textit{Sanskrit and other Indian Languages.} D.K. Printworld. New Delhi.

 \bibitem{005-key04} Neelamalur, M. (1994) \textit{Tolkāppiyam - with Eng translation and critical Studies.} Educational Publishers. Madras, 

 \bibitem{005-key05} Pillai, R.P. Sethu (1953) \textit{Words and their significance – A study in Tamil Linguistics.} University of Madras. Chennai.

 \bibitem{005-key06} Subrahmanya Sastri, P. S.( 1947) \textit{Comparative grammar of the Tamil Language.} Tiruvadi. Tanjore.

 \bibitem{005-key07} Subrahmanya Sastri, P. S. (1967) \textit{Historical Tamil Reader.} Tiruvaiyaru.

 \bibitem{005-key08} Subramanian, S.V. and Irulappan, K.M. (Ed.) (1980). \textit{Heritage of the Tamils – Language and Grammar.} International Institute of Tamil Studies. Madras.

 \bibitem{005-key09} Subrahmanya Sastri, P. S. (1997) \textit{History of Grammatical Theories in Tamil.} The Kuppuswami Sastri Research Institute. Chennai.

 \bibitem{005-key10} Taraporewala, Irach Jehangir Sorabji (1962). \textit{Elements of the science of Language.} Calcutta University. Calcutta.

 \end{thebibliography}

\theendnotes

