
\chapter*{Volume Editorial}\label{volumeeditorial}

Swadeshi Indology (SI) Conferences were started to consider the various attacks that are being made on Sanātana Dharma in the name of “Indology” by various scholars who are trained in the dual streams of modern fields of scholarship as well as traditional Indian \textit{śāstra}–s. The Fifth Swadeshi Indology Conference titled \textbf{“Karnāṭaka Śāstrīya Saṅgīta: Past, Present and Future”} shares some commonalities with the earlier SI Conferences in that it was conceived of in order to shield the \textit{sanātana dhārmic} aspects of the Classical music tradition of India from the attack by some modern scholars and musicians of skewed thinking; and differs from them in that it the Conference focussed not just on intervention but also on presenting the traditional point of view with regard to the various aspects of the musical tradition – such as the philosophy embedded in the literature, the contributions of some of the great \textit{vāggeyakāra}–s (composers of music especially of South India) of the tradition; and alongside, discussing what constitutes legitimate creativity while still honouring the tradition.

The present volume contains eight papers which were presented in the Conference. Of these some deal with the constructive part and others with the critical part. The following paragraphs give a brief overview of what is to be expected in the compilation of the papers presented.

\delimiter

The paper by \textbf{Gayathri Girish (Ch.1)} is on the topic of the contribution of Muttusvāmi Dīkṣita, entitled “The Music of Muttusvāmi Dīkṣita – a Window into His Personality”. The author brings out some of the aspects of the contributions of the great \textit{vāggeyakāra} as well as his personality.

After giving a brief biographical sketch of Śrī Dīkṣita, the author points out briefly, how Śrī Dīkṣita is different from his two contemporaries – Śrī Tyāgarāja and Śrī Śyāma Śāstri – in that he brings in, apart from his scholarship in Saṁskṛta language, his expertise of various \textit{śāstra}–s\break to inform his devotional and musically scholarly compositions. This makes each \textit{kṛti} a treasure trove of information on culture and philosophy as well. 

The next part of the paper discusses five aspects of the compositions: music, language, other śāstraic expertise, philosophical base, and broadness of vision.

Musically, Śrī Dīkṣita’s \textit{kṛti}–s brought many unique contributions, be it the changed structure of \textit{kṛti}–s with the \textit{pallavi–caraṇa} format or usage of \textit{rāga–mudrā} in several \textit{kṛti}–s. His creations included sets of \textit{kṛti}–s on a specific theme such as the Navāvaraṇa \textit{kṛti}–s\break and Navagraha \textit{kṛti}–s. The author has given a good number of examples for each and has highlighted the scholarly language that has been used in many of the \textit{kṛti}–s. Linguistic analysis of the \textit{sāhitya} brings out the nuances of word–compounding, usage of similes and metaphors, alliteration and so on. The expertise that Śrī Dīkṣita had in other \textit{śāstra}–s such as \textit{jyautiṣa, mantra–śāstra} and \textit{tantra–śāstra} have been well highlighted.

Giving an overview of the experimentation that Śrī Dīkṣita did with what had been, until that point, predominantly Hindustāni \textit{rāga}–s, and of course, the much–discussed Noṭṭusvara–s which were influenced by Western music. The author makes a nuanced argument to counter the one being put up for giving Noṭṭusvara–s as a justification for the evangelists plagiarizing Karnatic \textit{kṛti}–s. The final section of the paper outlines how Śrī Dīkṣita was a very generous \textit{guru} who taught interested and deserving pupils from a cross–section of the society, which included Tambiappan, Kamalam and the Tanjavur Quartet.

The paper \textbf{(Ch.2)} by \textbf{Vrinda Acharya} is titled “The Non–translatables of (South) Indian Music”. Every language has a number of words that cannot be translated because of the culture that the language is inextricably linked to; hence, attempting to use an “equivalent” involves many problems. The paper introduces the concept of non–translatables in the context of Saṁskṛta terms which deal with the core concepts of the different aspects of our culture, be it the knowledge systems or the arts.

Rajiv Malhotra’s \textit{Being Different} (2013)\supskpt{\endnote{ Malhotra, Rajiv. (2013). \textit{Being Different}. Noida: HarperCollins Publishers India.}} discusses the concept to point out how the knowledge that is embodied in the Saṁskṛta language is compromised or even eliminated when we replace with an equivalent from a language with a different cultural root (for instance, translation from Saṁskṛta to Kannada would be fine whereas from Saṁskṛta to English would dilute or misrepresent the concept). Whether it is done due to genuine ignorance (or superficial understanding) or with an ulterior motive of diluting the concepts, it can finally result in digestion (which is essentially stripping off the core inconvertible idea – another idea that has been discussed in Malhotra 2013). This will lead to the coloniser culture gradually imposing its own concepts on the colonised through language – for instance, ‘\textit{dharma}’ being translated as ‘religion’.

Going on the terms which are non–translatables in Karnatic music, the author discusses how the music culture has developed from deep roots in \textit{sanātana dharma}. Apart from the names of the \textit{rāga}–s and \textit{tāla}–s, the technical terms which are commonly used in music are non–translatables. The present author has selected twelve such words and has discussed the common translations that are used for those technical words.

Some of the common translations are: \textit{nāda} as ‘sound’, \textit{śruti} as ‘semitone’, \textit{svara} as ‘note’, \textit{laya} as ‘rhythm’ and so on. For instance, the word ‘\textit{rāga}’ has often been translated as ‘melody’. Melody has been defined, amongst other things, as a coherent succession of pitches and as successive tones (not simultaneous ones) while, \textit{rāga} is a concept which is much wider and deeper than this. Not only is it a combination and a specific arrangement of \textit{svara}–s with a definite relationship with the \textit{ādhāra ṣaḍja} (“fundamental note” approximately), but a \textit{rāga} goes much beyond the \textit{svara}–s that provide the framework. There are several distinctive phrases for each \textit{rāga} which give it personality and individuality, and this is the result of an evolution over a long period of time. Both Hindustani and Karnatic classical music systems have at their core the elaborate system of \textit{rāga}–s. The author cites from ancient texts to define \textit{rāga,} and from other authorities in musicology to bring out the nuances of the topic.

Sometimes the original term in Saṁskṛta does not have an equivalent at all in the modern, Western parlance. Some of the terms which have been considered in this article are \textit{mela}, \textit{saṅgati} and \textit{gamaka}. \textit{Gamaka}, for instance, is very central to Karnatic music. It is what makes a \textit{rāga} come alive – a movement, turn, or stress given to \textit{svara}(–s) to give it its personality. However, the effects that are used in Western music like \textit{vibrato} and \textit{staccato} not only have very limited usage (due to the very nature of the system of music where it is used) but they cannot be compared to the intricacy that is found in the \textit{gamaka} system.

The author has also provided several more non–translatables in the appendix. The entire import of the article is bringing out the importance of recognising and retaining the original nomenclature in order to prevent distortion and digestion of the original concepts.

In this paper \textbf{(Ch.3)} entitled “Tyāgarāja’s Philosophy and Rebuttal of Allegations”, \textbf{Korada Subrahmanyam} writes about the deep, Vedic roots of the philosophy that we see in the \textit{kṛti}–s of Śrī Tyāgarāja as well as discusses and rebuts certain allegations that have been made by T. M. Krishna against the saint.

Karnatic music, as a rule, abounds in compositions that are steeped in \textit{bhakti}. The compositions of Śrī Tyāgarāja are no exception. Hence, in a paper that is clearly demarcated into two parts, the author first goes into the roots of the important concept of \textit{bhakti}. Starting from the universal appeal of the concept, which transcends all barriers as we can see in the variety of origins of the greatest of \textit{bhakta}–s through times immemorial, the author details about the influence sages such as Nārada have had on Śrī Tyāgarāja. Touching upon several interesting and key concepts in the śāstraic tradition such as \textit{saguṇopāsanā}, \textit{kutsita–sevā–nindana} and \textit{dhyāna}, it is shown clearly how these are reflected through and through in the various \textit{kṛti}–s of Śrī Tyāgarāja. The author discourses in greater detail on the concept of \textit{nādotpatti} or the “creation of \textit{nāda}”. The śāstraic concepts are amply substantiated with quotes from very many traditional texts such as the \textit{Bhagavadgīta}, the Upaniṣad–s and the \textit{Vākyapadīya}.

The latter part of the article takes up the various allegations that have been made against Śrī Tyāgarāja. To name but one of them: T. M. Krishna accuses the saint of misogyny citing some of the \textit{kṛti}–s like \textit{duḍukugala} and \textit{ênta muddo}. The author of this paper addresses the issue by considering in detail the full text of each of the \textit{kṛti} cited in support of the accusation and places it in its appropriate context – cultural and otherwise. He provides evidence to the contrary (for the allegation) citing the \textit{sāhitya} of other \textit{kṛti}–s of Śrī Tyāgarāja.

The author cleverly exposes the contradiction in the argument of Krishna regarding the \textit{sāhitya} being mere pegs for music, and his claim elsewhere that the \textit{sāhitya} composed by Śrī Tyāgarāja is violent. Finally, the term ‘brahminisation’ is tackled briefly, with the author juxtaposing the essential meaning of ‘\textit{brāhmaṇa}’ and the perversion that is brought in by the comparatively newly coined term of ‘brahminisation’.

In sum, the author has brought out the spirituality that is the source of the \textit{kṛti}–s composed by Śrī Tyāgarāja, reiterating their inextricable link to \textit{sanātana dharma} even as he has countered the arguments against the saint point–by–point.

The paper \textbf{(Ch.4)} entitled “Experimentation in Karnatic Music – How Far is Too Far?” has been presented by \textbf{Radha Bhaskar}. In the article, the author considers the tradition of Karnatic musical concerts from the aspects of the structure of the concert along with the structures of the composition. The art form which has creativity and \textit{manodharma} at its core, also has a deep and long tradition of \textit{bhakti}–laden ideology which is reflected in the \textit{kṛti}–s of the highly respected \textit{vāggeyakāra}–s. This said, the author voices the concern that for taking the art form to the masses, many musicians of late have been diluting the \textit{kṛti}–s and have begun using them as mere pegs for their \textit{manodharma}.

In the first part of the article, the concert format in Karnatic music has been taken up and analysed, with focus on the role played by the compositions of \textit{vāggeyakāra}–s there. It might be mentioned here that the current format of the concert owes its structure to Ariyakuḍi Rāmānuja Ayyaṅgār and is only about a century old. It has clearly withstood the test of time since it has not really undergone drastic changes in popular circles. The author then compares this to the popularity of the concerts which have experiments with regard to structure (like RTP concerts or those where Mṛdaṅga takes the centre stage) or the composition (such as taking \textit{varṇa} as the main piece or singing \textit{ālāpana} of a \textit{rāga} followed by a \textit{kṛti} of a different \textit{rāga}). The failure of such experiments to gain popular approval or consumption underscores that experiment for the sake of experimentation is \textit{inutilis}.

In the next part of the article, the author considers experimentation in creation – of \textit{rāga}–s, of approaching different \textit{rāga}–s, of compositions etc. She addresses certain questions raised by some musicians with regard to the creation of \textit{rāga}–s, discussing what kind of experimentation will really bear fruits that will be relished by generations to come. Moving on to the matter of \textit{sāhitya}, given the very nature of Karnatic music, it is expectable that the artform abounds in various kinds of\break traditional \textit{kṛti}–s by the great \textit{vāggeyakāra}–s, predominantly addressing divinities. The author very rightly points out that the \textit{bhakti}–centric \textit{sāhitya} such as \textit{Divya–prabandham, Tevāram, Tiruppugaḻ, Taraṅgam, }and\textit{ Aṣṭapadī} which used to be the primary compositions in the pre–Trinity era, have paved way to the rich and lofty content during the Trinity period and the post–Trinity period as well. It is to be recognised that \textit{sāhitya} and \textit{saṅgīta} are the two wheels on which the chariot of the concert cruises and they are to complement each other well, and no compromises should be done with either, if one has to deliver a high–quality musical concert.

In the final part of the article, she considers other kinds of experimentations such as Jugalbandi, disconnecting the Hindu nature of Karnatic music by composing and presenting on Christ, ‘Carnatic Rock Bands’ and the like. She also briefly addresses adapting non–traditional/modern instruments for this genre of music, concluding on the note that the balance between exploring new avenues and being faithful to the tradition is a must in order to get the best out of this art form.

The paper by \textbf{Jataayu (Ch.5)} is entitled “Christian Attempts to Appropriate Karnatic Music: A Historical Overview”. The paper introduces the topic of organised missionary activity in South India from the 16th century and the various methods that are employed to achieve their objectives – of conversion and deculturation. This includes appropriation of language, customs, religious symbols and – here comes the crux – of art forms which obviously includes Karnatic classical music. After giving a brief introduction to the essential “Hinduness” of the traditional art form (which overturns the false claims of art being “beyond religion”), he introduces the concept of “Christian Keerthanam” which started in the 18th century Tamilnadu. When showing open contempt for Hindu practices did not work, the missionaries came up with the cunning plan of appropriation – in order to pull the crowds which were naturally drawn to the Hindu customs, symbols, and art–forms. Starting from composers such as Ziegenbalg (1682–1719), a Lutheran missionary who pioneered this attempt, and prolific composers like Beschi (1680–1742), a Jesuit missionary who reinvented himself as “Veeramunivarar”, to the really very influential Bishop Caldwell (1814–1891) (the father of the “Dravidian race” hypothesis), and influential converts like H A Krishna Pillai (1827–1900) – we find here several samples of compositions. The common trait amongst these is the natural employment of \textit{sanātana dhārmic} concepts, terminologies, and symbolisms – such as \textit{veda, bodha, sat, niṣkala,} and \textit{ānanda –} to name but a few.

A complete section of the paper is devoted to considering the work of Vedanayaga Sastriyar (1774–1864) whose enormous output (100+ poetic works) continues to fuel the missionary stoves. The notable point about this composer is that he had royal patronage by Serfoji Maharaj, the Maratha ruler of Tanjore. The evangelical machinations are manifest in the very name of the said composer who is called “Vedanāyaka” (but who has a visceral hatred for Hindus), and who is (ironically) given the title of “Śāstri” (which is generally a title given to learned scholars). Samples of his open mockery of Hindu customs have been given in this section. His masterstroke, as it can be called, was the appropriation of the “Kuravañji” concept which is a dance–drama which can be located between a Bhāgavata Mela Nāṭaka and a folk dance–drama, with its underlying devotional aspect (See Sambamoorthy 1960:93–94\supskpt{\endnote{ Sambamoorthy, P. (1960). \textit{History of Indian Music}. Madras: The Indian Music Publishing House.}}). The famous story of a gypsy girl and her devotion towards Lord Śiva in Tiru Kuṟāla Kuravañji is appropriated – in order to present the story of Jesus in (Vedanayaga Sastriar’s) \textit{Bethlehem Kuravanji}. The author then goes on to discuss whether this is “collaboration”, appropriation or plain plagiarism, and makes critical comments on various contemporary scholars – who, either through their ignorance or complacence or malicious intent, call for “artistic freedom”, and are only keenly aware of the socio–cultural impact of subverting the indigenous theological framework.

Another section of the paper focuses on the work of Abraham Panditar, whose 1400–page musical treatise, \textit{Karuṇāmirta Sāgaram}, received commendations from stalwarts like Muttaiah Bhagavatar, and great Tamil scholars like U. Ve. Swaminatha Iyer and R. Raghavaiyangar, who did not realise that it was in fact a suspect stew of Biblical stories, discredited theories (such as Aryan–Dravidian divide, Lemuria, and Kumari Kandam), Tamil supremacist arguments and a mishmash of ideas on “South Indian Music”.

The paper also draws our attention to the blatant plagiarism of the much–loved and much–recognised Karnatic music compositions like the Piḷḷāri gīta–s of Śrī Purandaradāsa, and \textit{varṇa}–s like Ninnukori and Jalajākṣa into their “Christian” versions, published in a book, a so–called Christian Tamil Music Primer (\textit{Kristava Tamiḻ Isai Bodini}). The paper concludes by discussing how simple or straightforward, or rather not, it is, to just claim artistic freedom to sing Christian Keertanams, and by arguing for seeing these issues, not in isolation, but in the civilizational context.

The next paper \textbf{(Ch.6)}, which is by \textbf{V. Ramanathan}, takes up the critique of the book \textit{A Southern Music: The Karnatik story}, authored by T. M. Krishna. Right off the bat, the author points out that, in this book, Krishna is not objectively talking about Karnatic music but bringing in his own takes and opinions on its various aspects, presenting them as “narration of the story” of Karnatic music. The whole exercise is one of “deconstruction” of the framework of the art–form. The author of the paper deals with this by posing certain questions and the discussion on those topics bring out the fallacies in the book.

The first question the author poses is with regard to \textit{bhakti} – whether it is indeed the be–all and end–all of the art–form. Krishna claims in his book that ‘god’ and ‘religion’ have nothing to do with music, with a long–winded argument bringing in etymology; and our author shrewdly catches the fallacies of false equivalence (of dhārmic and Abrahamic) and oversimplification of concepts. Krishna’s claims – that the meaning of \textit{sāhitya} of a \textit{kṛti} is not the focus, that the composers’ genius lay in the musical aspect of the composition rather than the \textit{sāhitya}, that the \textit{utsava–sampradāya}\textit{kṛti}–s have no place in Karnatic music genre etc. – have been well taken apart, and the counter–arguments have been adequately provided. The author rightly points out the issues with separating the art–form from the \textit{bhakti} aspect, and argues against such a dissection – by citing from an original source like \textit{Nāṭya–śāstra} (which differentiates \textit{gāndharva} and \textit{gāna}), culled from the writings of a \textit{vīṇā vidvān}, Ranga Ramanuja Iyengar (who writes about the link with \textit{bhakti} of Karnatic music), and from the writings of Western scholars like Barrel (who talks of the link of philosophies with Indian music).

Krishna questions the need for placing \textit{sāhitya}–oriented musical experience within the Karnatic music itself. The present author tackles the issue in a multi–disciplinary manner, bringing in evidence from acoustic studies, which substantiate that the lyrical and melodic are mutually supportive in enhancing the experience – well–juxtaposed with the famous verse of Kālidāsa (\textit{vāgarthāviva sampṛktau}…), together with the \textit{bīja mantra}–s that form the essence of Muttusvāmi Dīkṣita’s Navāvaraṇa \textit{kṛti}–s. He also brings out the inherent contradictions in the points made by Krishna – such as the importance of lyrics being absent for a \textit{kṛti} like Kāmākṣi, and their becoming all important for \textit{jāvaḷi}–s.

The further section deals with the usual rant typically seen in leftist literature about brahmins – that brahmins dominated the Karnatic music arena. While Krishna liberally makes use of the subaltern theories of the contemporary ethnomusicologists, the author points out, the conclusions he draws are totally erroneous. In a detailed discussion, the author delves into those areas which are wilfully omitted by Krishna while looking at the history – such as, the role played by the Anglo–Indian laws, the impact of the “Social Purity Movement” in India, and of course, the ubiquitous missionary influence on the Tamil society. Citing extensively from a paper by Kunal Parker (1998), the author brings out the judicial activism that led to \textit{devadāsī}–s being branded as ‘prostitutes’ while they were actually nothing of the sort. As \textit{devadāsī}–s were dedicated to music and dance, their getting convicted as criminal was nothing short of a bolt from the blue for the art–forms. The author argues that the brahmin men and women who stepped in at this crucial juncture proved to be the emancipators of the art–forms.

The paper concludes by summing up how the works of Krishna and the like, which evidently are agenda–driven, present a veneer of intellectualism while catalysing the evangelisation of our traditional art–forms.

This paper \textbf{(Ch.7)} by \textbf{Arathi V. B.} is entitled “Is Karnatic Music a Bastion of Brahminical Patriarchy?”. Expectably, the paper concerns itself with the allegations that have been made by T. M. Krishna specifically regarding the field.

The author argues that the meaning of word ‘\textit{brāhmaṇa}’ itself has been twisted – as have been the case with many other words, through the stranglehold on the narrative the leftists have exercised – and discusses the meaning of the word within the framework of \textit{sanātana dharma}. Hence, when artists like T. M. Krishna accuse the art space of being ‘brahmanical’, it comes from the skewed narrative that has dominated all these years. The author argues for the knowledge of Saṁskṛta, an understanding of lyrics and an understanding that the \textit{rasa} experience is at the core. She contests the claim – that there have been attempts to ‘brahminise’ Karnatic music – by the counter that the general conventions that are followed in a \textit{sabhā} have been \textbf{labelled} brahminical. She attacks Krishna’s claims regarding ‘brahminisation’ by pointing out that unique \textit{sampradāya}–s abound in each group of people, whether brahmins or otherwise; and that the line of argument which treats non–brahmins as inherently uncivilised or amateurish even while they form the majority in their representation in many key fields, is essentially a leftist (and a faulty) take on the issue. She brings out the struggles that the brahmins underwent to preserve this heritage, while detailing about how \textit{devadāsī}–s were hounded in this colonised country because of which those struggles had become necessary.

To counter Krishna’s argument about the ‘domination’ of the domain by brahmins, the author lists the various non–brahmin students taught by great musicians through the times, as well as the very many non–brahmin musicians who have made a great name for themselves in the field. This includes many contemporary musicians as well as those legends of yore. We might also note that non–brahmins have constituted a large percentage of musicians from Kerala for a long time(in addition to the many names listed by the author who are from Karnataka, Tamilnadu and Andhra/Telangana regions).

Towards the end of the article, the author takes up the allegations regarding the compositions of \textit{devadāsī}–s getting a low representation as well as those against Śrī Tyāgarāja about his “brahminical prejudice”. In conclusion, while not gainsaying that some prejudice does colour all societies, the author warns against outside interventions which can be inimical to the very existence of the society, and suggests some practical steps to counter this.

The final article \textbf{(Ch.8)} is by \textbf{Aravind Brahmakal}, entitled “Role of Patronage in Karnatic Music – Past, Present and the Future”. This is an exception in this set of articles, in that, it is not (and is not meant to be) an “academic” paper. This one encapsulates the views from a patron regarding the role patronage plays in the way an art form flourishes in society.

\newpage

Ananda Coomaraswamy (1956:65)\supskpt{\endnote{ Coomaraswamy, Ananda K. (1956). \textit{Christian and Oriental philosophy of art}. New York: Dover Publications Inc.}} argues “...but we are either indifferent to its \textit{raison d'etre} and final cause, or find this ultimate reason and justification for the very existence of the work in the pleasure to be derived from its beauty by the patron.” In other words, it is the patron that dictates and holds the artist responsible for the quality of the art. Any fall in the quality in the sensibilities, and we find the quality of the art form degenerating. Hence the responsibility of a patron is rather huge. A good example of what it means to be a discerning patron can be understood through an anecdote of the Mysore king, Nālvaḍi Krishnaraja Wodeyar (1884–1940). He had commissioned the “translation” into Kannada of some popular \textit{kṛti}–s\break of Karnatic music that were in Telugu. Even while some scholars engaged themselves in what was asked of them, the king himself thought this over and then decided, it is said, that this project of translation will take away significantly from the uniqueness of \textit{bhāva} and the flavour of the \textit{kṛti}–s, and asked the project to be suspended.

This paper puts forth an overview of patronage \textit{vis–à–vis} Karnatic classical music, in three parts – (1) how it was during the times of royal patronage, (2) how the scene is in the modern times, \textit{sans} royal courts, and (3) what can be done in order to make things better for young artists to take this up as a profession.

The view into the past considers the contribution made by various royalty in the Southern Indian kingdoms such as the Tanjavur Maratha rulers and the Mysore Wodeyars, and the strides Karnatic classical music took during that period.

An analysis of the present–day scenario delineates the various sources of support for professional classical musicians, such as governmental organisations and \textit{sabhā}–s run by \textit{rasika}–s who offer patronage in lieu of the royal patronage of earlier times. It also considers the challenges of patronage in the current world. The author presents the views of organisers and artists regarding the issue.

The final part of the write–up is with regard to who can contribute in what way to ensure that the artists receive adequate patronage – initiative of the artists themselves, the \textit{rasika}–s, the \textit{sabhā}–s, the government and the corporates.

\delimiter

\textbf{Note on spellings: }

It needs to be clarified here that a scheme for an accurate representation of Indian languages in Roman script has been used throughout this volume (and the SI Series) and the transliteration scheme (which is mainly IAST but incorporates elements of ISO 15919 for ease of reading) has been provided. There are some exceptions since some terms are used with their popular spelling rather than with the more accurate standard diacritics. In all the papers, contemporary locations have been used with their popular spellings. Names of people who have published only in Indian languages (and hence we do not have a record of the spelling of their names in Roman) have been given with diacritics, and not others.

Needless to say, the authors take the responsibility for the arguments and facts they place on record.

\begin{flushright}
Dr. H. R. Meera\\ Editor
\end{flushright}

Chennai

\theendnotes

