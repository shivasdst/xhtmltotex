
\chapter{Christian Attempts to Appropriate\index{appropriate@appropriate} Karnatic\index{Karnatic@Karnatic} Music: A Historical Overview}\label{chapter5}

\Authorline{Jataayu\footnote{ pp. 131--155. In: Meera, H. R. (Ed.) (2020). \textit{Karnāṭaka Śāstrīya Saṅgīta - Its Past, Present, and Future.} Chennai: Infinity Foundation India.}}

\vspace{-.3cm}

\lhead[\small\thepage\quad Jataayu]{}

\begin{flushright}
\textit{(jataayu.b@gmail.com)}
\end{flushright}


\section*{Abstract}

From the very start, Christian missionary\index{missionary@missionary} activity in South India employed various devious methods of influencing the native Hindus towards conversion to Christianity, which included attempts to appropriate their language, literature and idioms, religious symbols, cultural practices and art forms. This practice, institutionalized as inculturation or indigenization in Christian theology is an issue of bitter controversy in societies and cultures on which it is unleashed. South Indian classical music also known as Karnatic music, an Indic art form totally integrated with Hindu religious tradition also did not escape these appropriation attempts which were centered in the Thanjavur region that played a key role in the growth and evolution of this music in the last three centuries. This paper aims to give a historical overview of such attempts, starting from the 17th century till the present times. The real character, aesthetics and themes of Tamil Christian Keerthanams\index{Christian Keerthanam@Christian Keerthanam}\endnote{ Popular spelling retained here.} by composers like Vedanāyagam Sāstriyār\index{Sastriyar, Vedanayaga@Sastriyar, Vedanayaga} will be evaluated based on established Tamil literary standards and generally accepted norms within the domain of Karnatic\index{Karnatic@Karnatic} music compositions, corroborated in the backdrop of their life and mission. The questionable religious and historical narratives regarding Indian classical music in the renowned musical treatise \textit{Karunamirtha Sāgaram}\index{Karunamirtha Sagaram@\textit{Karunamirtha Sāgaram}} of Abraham Pandithar\index{Abraham Pandithar@Abraham Pandithar} will be analyzed in detail based on the primary source material in Tamil. The impact and consequences of such appropriation\index{appropriation@appropriation} attempts by Christian institutions as well as individuals will be discussed, considering the overall religious, cultural and socio-political factors in the larger context of Indian civilization.

\vspace{-.3cm}

\section*{Introduction}

Christian missionary\index{missionary@missionary} activity in South India started with the arrival of Portuguese colonizers in the 16th century and got consolidated in a major way in many pockets with the establishment of British rule. During the initial phase, the Christian propaganda proclaimed that Hindu culture, like the Hindu religion was a creation of the devil. It had to be scrapped and the stage swept clean for the culture of Christianity to take over. But over time, with failures and very slow progress in gaining converts, the missionaries came up with the Theology of Inculturation or Indigenization.

\begin{myquote}
“… In the new language, Hindu culture was credited with great creations in philosophy, literature, art, architecture, music, painting and the rest. There was reservation only at one point. This culture, it was said, had stopped short of reaching the crest because its spiritual perceptions were deficient, even defective. It could surge forward on its aborted journey only by becoming a willing vehicle for ‘Christian truths’. That was the Theology of Inculturation or Indigenization. It created another lot of literature. The missions did not stop at the theoretical proposition. They demonstrated practically how Hindu culture should serve Jesus Christ. A number of Christian missionaries started masquerading as Hindu sannyasins, wearing the ochre robe, eating vegetarian food, sleeping on the floor and worshipping with the accoutrements of Hindu puja..”\index{puja@\textit{puja}} 

~\hfill (Goel 1986: Ch.18)
\end{myquote}

It is well documented as to how Christian missionaries in South India, from 17th century onwards, attempted to appropriate language, literature and idioms, religious symbols, cultural practices and art forms of the local native Hindus in so many ways and this process is still ongoing. Karnatic\index{Karnatic@Karnatic} music also did not escape these appropriation attempts. The aim of this paper is to give a historical overview of such attempts, while also analyzing the merit of the arguments that justify such attempts in the name of artistic freedom.

\vspace{-.3cm}

\section*{The Essential Hindu Character of Karnatic Music}

The South Indian classical music known by the name Karnatic Music has a long and rich history and is a vibrant and living art tradition, both in its traditional, classical form, and as a base, inspiration and major influence in many semi-classical and popular musical forms like film music, folk music, devotional songs etc. It is an unambiguous fact that from its very roots, Karnatic music has been an integral part of Hindu religion and culture in an inseparable way and the same continues till today, even after its modern renaissance in the 20th century. The argument of Indian classical music, and specifically Karnatic music being a ‘pure art’ not associated with any religion or culture or social context is a mere notion without substance. It does not hold true under scrutiny, given the fact that the very birth, growth and evolution of the art form are intertwined with these factors. In fact, this is true with every classical art form that has a long history.

The fact that the current performers include people of non-Hindu religions and atheists, or the lyrics\index{sahitya@\textit{sāhitya}} part of some compositions revolves around seemingly non-religious, “secular” themes – such things do not take away the essential Hindu character of Karnatic\index{Karnatic@Karnatic} music. It is beyond doubt that this essential Hinduness is derived because the compositions are addressed to the Hindu deities and the performances take place in Hindu temples and religious centers. In addition, it also equally or rather more significant that the fundamental ideas that give uniqueness and separate identity to this music – viz. \textit{śruti},\index{sruti@\textit{śruti}} \textit{laya},\index{laya@\textit{laya}} \textit{bhāva},\index{bhava@\textit{bhāva}} \textit{rasa}, \textit{nāda}\index{nada@\textit{nāda}} etc. are tightly coupled to the Hindu philosophical truths and have been developed along with the Hindu spiritual tenets. This is how it was viewed all along, as Tyāgarāja crisply elucidates in his composition \textit{śobhillu saptasvara}.

\begin{myquote}
“Worship the beautiful goddesses presiding over the seven svaras,\index{svara@\textit{svara}} which shine through navel, heart, throat, tongue, nose etc., and in and through Rik and Sāma Vedas,\index{Samaveda@\textit{Sāmaveda}}\index{Rgveda@\textit{Ṛg Veda}} the heart of the Gāyatri Mantra and the minds of gods and holy men and Tyāgarāja” 

~\hfill (Ramanujachari 1958: 596) (\textit{spellings as in the original})
\end{myquote}

Not just the \textit{Bhakti}\index{Bhakti@\textit{Bhakti}} or philosophical hymns, but even the musical compositions with strong erotic themes are very much inside the Hindu aesthetic tradition under the \textit{Śṛṅgāra Rasa}, as seen in the texts like the \textit{Gīta Govinda} or \textit{Rādhikā Sāntvanamu}. But, to put such a divine art form at the “service of Jesus” with a clear devious motive and a cunning and criminal strategy to undermine and eradicate the very spiritual fountain that gave birth to it and sustains it – is nothing but a rank perversion of the very art form.

\vspace{-.3cm}

\section*{The Origin and Growth of\hfill \break “Christian Keerthanam”\index{Christian Keerthanam@Christian Keerthanam}}

It is well known that the Kaveri delta region, comprising the towns of Thanjavur, Kumbhakonam and Mayiladuthurai (formerly called Mayavaram) and the numerous villages nestled around them was the cradle of Karnatic\index{Karnatic@Karnatic} music from 16th century till very recent times until Chennai became its unofficial capital. This is the region that produced the Karnatic Trinity\index{The, Trinity@The, Trinity} and numerous great composers and illustrious Vidwans. Ironically, this region was also the center of hectic Christian missionary\index{missionary@missionary} activity during the very same period.

During the period from 1732 to 1859, Christian missionary stations were setup in many towns of this region viz. Thanjavur, Thiruchirappalli, Nagappattinam, Mayavaram, Mannargudi, Thiruvarur. The missions involved were British sponsored Royal Danish Mission (RDM), Society for Promotion of Christian Knowledge (SPCK), Society for Propagation of the Gospel (SPG), London Missionary Society (LMS), Church Missionary Society (CMS) and the Methodists. So, much so that, “By the beginning of the ninth decade of the nineteenth century, the whole region had become honeycombed with mission stations” (Goel 1986: Ch. 12).

The initial Christian propaganda mostly centered around abusing Hindu practices openly and expressing contempt against them in public.

\begin{myquote}
“... According to a missionary report of 1821, “an immense population lies enslaved in the grossest darkness.” Hindu temple worship was “calculated to corrupt the heart, to sensualize the mind and to lead to every description of vice.” James Mowat, a Methodist missionary wrote in a letter to his headquarters at London, “I never had so plain a demonstration of depravity heathenism binds upon its votaries in the shape of religion. The principal pagoda abounds with the most obscene and polluting representations, and decidedly proves, if proof be necessary, how greatly this people need the hallowing light of Christianity.” 

~\hfill (Manickam 1981: 82)
\end{myquote}

But over time, this gave way to singing and music, shamelessly copying the “obscene and polluting representations of the Hindu pagodas” as that proved to be the major attraction to pull the crowds, as the report records further.

\begin{myquote}
“.. They visited Choultries, market places and bathing tanks, preached in the city streets, under the shadow of idol cars, under the canopy of country-trees, in the open square of the villages and other places where they could meet a good number of people... The singing of Christian lyrics\index{sahitya@\textit{sāhitya}}and hymns accompanied by musical instruments soon attracted a congregation.” 

~\hfill (Manickam 1981: 83)
\end{myquote}

So, this is how the “Christian Keerthanam”\index{Christian Keerthanam@Christian Keerthanam} came into existence, as an aid and tool in the missionary\index{missionary@missionary} propaganda exercise. It was not motivated by artistic or musical inclinations of any kind. Till today, it retains this character.

A comprehensive Tamil work \textit{kiṟistava kīrttaṉai kaviñarkaḷ} (poets of Christian Keerthanai) compiled by Y. Gnana Chandra Johnson chronicles about 70 such composers over the ages (Johnson 2012). The compiler is a professor in Madras Christian College, Tambaram, Chennai.

It starts right from the pioneer Lutheran missionary Ziegenbalg\index{Ziegenbalg@Ziegenbalg} (1682-1719), who translated Christian hymns from German to colloquial Tamil and set them to Western music. The next major composer was Jesuit missionary Constantine Joseph Beschi\index{Beschi@Beschi} (1680-1742) who rechristened himself as Veeramamunivar (brave, great Muni) and dressed in ochre robes to deceive native Hindus. A Tamil literary work \textit{Tēmpāvaṇi }written in the typical \textit{kāvya}\index{kavya@\textit{kāvya}} style narrating the life of St. Joseph is generally attributed to him, though there is broad scholarly opinion that the Tamil verses were written by his teacher Supradeepa Kavirayar,\index{Supradeepa Kavirayar@Supradeepa Kavirayar} who was an accomplished poet. The literary work is filled with so many Hindu similes and metaphors all hanging along with Christian dogma couched in Tamil words. Here is a part of a Keerthanam composed by Veeramamunivar set in Dhanyāsi \textit{Rāga}:\index{raga@\textit{rāga}}

\begin{myquote}
hspace{2cm}\textit{jagaṉṉātā gurupara nātā - tiru \\hspace{2cm} aruḷ nātā yesu pirasāda nātā }
\end{myquote}

\begin{myquote}
hspace{2cm}\textit{tigaḻuṟum tātā pugaḻuṟum pādā\\hspace{2cm} tītaṟum veda bodā}
\end{myquote}

\begin{myquote}
hspace{2cm}\tamil{ஜகன்னாதா குருபர நாதா \enginline{-} திரு\\hspace{2cm} அருள் நாதா ஏசு பிரசாத நாதா \\hspace{2cm} திகழரும் தாதா புகழரும் பாதா \\hspace{2cm} தீதறும் வேத போதா}
\end{myquote}

As even those not familiar with Tamil can discern, the lyrics\index{sahitya@\textit{sāhitya}} are replete with Hindu phrases like \textit{jagannātha}, \textit{gurupara}, \textit{veda}, \textit{bodha}, \textit{nātha} etc.

Bishop Robert Caldwell\index{Caldwell, Bishop@Caldwell, Bishop} (1814-1891) was the British missionary who played a major role in inventing ‘Dravidian Race’ by transforming Linguistics into ethnology and spinning nasty conspiracy theories involving Aryans (Malhotra\index{Malhotra, Rajiv@Malhotra, Rajiv} and Neelakandan\index{Neelakandan, Aravindan@Neelakandan, Aravindan} 2011: 64-67). He also penned some Keerthanams. A sample line:

\begin{myquote}
hspace{2cm}\textit{ecaiyā piḷanta āti malaiye\\hspace{2cm} mocanāḷil uṉṉil ôḷippeṉe }
\end{myquote}

\begin{myquote}
hspace{2cm}\tamil{ஏசையாபிளந்தஆதிமலையே\\hspace{2cm} மோசநாளில்உன்னில்ஒளிப்பேனே}
\end{myquote}

\begin{myquote}
hspace{2cm}Oh, primeval mountain pierced by Jesus\\hspace{2cm} I will hide behind you on the day of distress
\end{myquote}

John Balmer (1812-1883), the author of \textit{Christhayanam} is another important composer. A very popular Keerthanam by him goes like this:

\begin{myquote}
\textit{têṉ iṉimaiyilum ecuvin nāmam divya maduram āme – atai\\ teḍ iyê nāḍi oḍiye varuvāy diṉamum nī maname}
\end{myquote}

\begin{myquote}
\tamil{தேன் இனிமையிலும் ஏசுவின் நாமம் திவ்ய மதுரம் ஆமே\enginline{ - } அதை\\ தேடியே நாடி ஓடியே வருவாய் தினமும் நீ மனமே }
\end{myquote}

The name (\textit{nāmam}) of Jesus is of divine sweetness (\textit{divya madhuram}) – this phrase is nothing but an echo of thousands of such hymns extolling the divine names of Rama and Krishna and ordering the mind (\textit{maname}) to go after the divine name is also an established convention in Hindu \textit{Bhakti}\index{Bhakti@\textit{Bhakti}} poetry.

\newpage

H. A. Krishna Pillai\index{Pillai, H A Krishna@Pillai, H A Krishna} (1827-1900) a Christian convert from an orthodox Vaishnava family was an important poet and composer. A famous Keerthanam by him set in Śaṅkarābharaṇa goes like this:

\begin{myquote}
hspace{2cm}\textit{sattāy niṣkaḷamāy - ôru\\hspace{2cm} sāmiyamum ilatāy \\hspace{2cm} cittāy āṉaṉdamāy\\ tikaḻkiṉṟa tirittuvame\\hspace{2cm} êttāy nāyaṭiyeṉ\\ kaṭaitteṟuvaṉ bavam tīrṉtu\\hspace{2cm} attāy uṉṉai allāl - êṉakku\\hspace{2cm} ār tuṇai ār uṟave.}
\end{myquote}

\begin{myquote}
hspace{2cm}\tamil{சத்தாய் நிஷ்களமாய் \enginline{-} ஒரு\\hspace{2cm} சாமியமும் இலதாய்\\hspace{2cm} சித்தாய் ஆனந்தமாய்\\hspace{2cm} திகழ்கின்ற திரித்துவமே\\hspace{2cm} எத்தாய் நாயடியேன்\\hspace{2cm} கடைத்தேறுவன் பவம் தீர்ந்து\\hspace{2cm} அத்தா உன்னை அல்லால் \enginline{-} எனக்கு\\hspace{2cm} ஆர் துணை ஆர் உறவே.}
\end{myquote}

Oh Trinity (\textit{tirittuvam}), you are \textit{Sat} (Existence), \textit{Niṣkala} (without blemish), unparalleled, \textit{Cit} (conscious) and \textit{Ānanda} (Bliss). This is the meaning of the first four lines (\textit{translation mine}). The Vedantic attributes of the Absolute (\textit{Brahman})\index{brahman@\textit{Brahman}} have nothing to do with Christian ideas of God or Trinity. But, still the composer has just flicked them from the Hindu hymns where they are commonly used. The fifth, sixth and the eighth line in this verse are taken from different places in \textit{Tiruvāsagam}, the famed Śaivite scripture and the seventh line is straight from the very popular \textit{Tevāram}\index{Tevaram@Tevāram} hymn of Sundarar\index{Sundarar@Sundarar} (\textit{pittā piṟaicūṭi}).

According to Johnson, there must have been about thousand Christian Keerthanams\index{Christian Keerthanams@Christian Keerthanams} written by more than hundred composers till date, out of which about two hundred are in vogue and are sung popularly across Churches in Tamil Nadu. The compilation \textit{Jñāna Kīrtanaikaḷ} (2003) published by Kanyakumari Diocese is a standard reference for these compositions (Johnson 2014).

The next section deals with Vedanāyagam Sāstriyār,\index{Sastriyar, Vedanayaga@Sastriyar, Vedanayaga} the pioneer and perhaps the most prolific of Tamil Christian poets and Keerthanam composers.


\section*{Vedanāyagam Sāstriyār\index{Sastriyar, Vedanayaga@Sastriyar, Vedanayaga}}

Vedanāyagam Sāstriyār (1774-1864) was hailed as “Suviseda Kavirāyar” (King among Evangelical Poets) and is credited with 120 poetic works big and small, that include musical compositions. Unless otherwise specified, all the biographical details and the translations of verses given in this section are from the book, Appasamy (1995).

He was a second-generation Christian convert who came under the influence of Rev. Christian Frederick Schwartz\index{Schwartz, Rev. Christian Frederick@Schwartz, Rev. Christian Frederick} of Lutheran mission station at Thanjavur at a very young age of 12 years and eventually became his godson.

It is well documented that Rev. Schwartz had extreme hatred towards Hindus, their religion and culture. This is what he wrote when war broke out in the Tanjore region and there was all around devastation everywhere.

\begin{myquote}
“.. Let us observe even in this affair the footsteps of Providence; how things will end, and what will be the effects of them. For nothing, God could never have permitted it. Idolatry in the Tanjore country is very deeply rooted; and to overthrow it gradually, who knows but God may use the present affliction? We pray, and will pray, ‘Thy kingdom come,’ to us, to all, to Tanjore. Amen.” 

~\hfill (Schwartz 1835: 274)
\end{myquote}

When the entire Thanjavur region was under attack during the Karnatic\index{Karnatic@Karnatic} wars due to the plunder and pillage by the armies of Hyder Ali, Schwartz termed it as the retribution of God on “Idol worshipping Hindoos” for not accepting Christianity.

\begin{myquote}
“It is true Coromandel has been visited by the Lord; the inhabitants of it have had time, and places to be instructed; the book of God, and other useful treatises, have been freely offered to them; nay, they have been pressed to accept of these spiritual treasures; but they have neglected, not to say despised, the gracious counsel of God, preferring the friendship and things of the world before the blessings of God. Now the Lord God begins to visit them in a different manner. Their idols, on which they leaned, are taken away; their houses burnt, their cattle driven away, and, what afflicts many thousand parents unspeakably more, is, that Hyder sends their best children away.” 

~\hfill (Schwartz 1835:381)
\end{myquote}

\newpage

Given this, it is not surprising that Vedanāyagam\index{Sastriyar,Vedanayaga@Sastriyar,Vedanayaga} imbibed similar mindsets, even when he was well versed in classical Tamil and its literature and was knowledgeable in Sanskrit and Telugu. This is how he sang the glory of his godfather Rev. Schwartz\index{Schwartz, Rev. Christian Frederick@Schwartz, Rev. Christian Frederick} later (\textit{emphasis as in the original}):

\begin{myquote}
“\textbf{How did the light down on the darkness of India}\\ And Christian society be formed.\\ The Christians from Europe arrived\\ Rev. Schwartz gathered the Disciples of Christ in Thanjavur\\ The Siva baktan from Tirunelveli became the disciple of Jesus Christ...” 

~\hfill (Jebamalai I: 26 verse II)
\end{myquote}

Tulajāji Mahārāja, the then Maratha ruler\index{Tanjore rulers@Tanjore rulers} of Thanjavur had entrusted his son Serfoji’s\index{Serfoji Maharaj@Serfoji Maharaj} education also to Rev. Schwartz. So, the bond of friendship that formed between the two continued well over time, earning Vedanāyagam a prestigious place in the court of ruler Serfoji, and the appellation ‘Sāstriyār’, generally adopted by highly learned Brahmins or scholars of arts and sciences. Even while enjoying the royal patronage\index{patronage@patronage} and the friendship of the Hindu king who treated him with affection and respect and using it to the hilt in his Christian evangelism activities, Vedanāyagam openly expressed his venomous hatred for Hindu religion and culture. He authored a work \textit{Sastra Kummi} in 1814 wherein he mocks and ridicules Hindu customs and traditions in abusive terms, in the guise of admonishing Christian coverts who are still stuck with their old ways. This book is now out of print, after the protests when it was published in 1990 by Thanjavur Saraswathi Mahal Library (Sivasubramanian 2011). Here is a verse from this work quoted in the same article.

\begin{myquote}
\textit{māṭṭu mūttirattai kuḍitte – anda māṭṭu cāṇiyai pūcik kôṇḍu\\ māṭṭai tāṉe kumbiṭṭu niṉṟa uṉ – māṭṭu buddiyo ñāṉappêṇṇe}
\end{myquote}

\begin{myquote}
\tamil{மாட்டு மூத்திரத்தை குடித்தே \enginline{-} அந்த மாட்டு சாணியை பூசிக் கொண்டு\\ மாட்டை தானே கும்பிட்டு நின்ற உன் \enginline{-} மாட்டு புத்தியோ ஞானப் பெண்ணே }
\end{myquote}

\begin{myquote}
Drinking cow urine, and smearing yourself with cow dung,\\ Worshipping cow; Isn’t that your bovine brain, O wise women? 

~\hfill (Sivasubramanian 2011) (\textit{translation ours})
\end{myquote}

\newpage

Such exhortations are to be considered as pure hate speech and cannot be bracketed under ‘condemning the superstitions and social evils’ category, because the author was not a rationalist or reformer of any sort, but a sworn Christian fundamentalist. As per the above article, he was not free from caste prejudices and vehemently opposed attempts to allow “lower caste” people into Church worship. This verse is exactly the same as some of the anti-Hindu statements by Pakistani government leaders supporting Islamic terrorism in the aftermath of IAF air strikes in February 2019, for which they had to apologize (Swati and Madhur 2019). But Vedanāyagam\index{Sastriyar, Vedanayaga@Sastriyar, Vedanayaga} had enshrined it for posterity in a supposedly “musical composition”. It is an irony of the highest order when the “progressive and egalitarian” vocalist T. M. Krishna\index{Krishna, T.M.@Krishna, T.M.} chose to sing a Christian Keerthanam\index{Christian Keerthanam@Christian Keerthanam} by such an obscurantist hate-monger in his Karnatic music concert to “rebuff” his critics (Ramanujam 2018).

\textit{Bethlehem Kuravañji}\endnote{ Spellings retained as in popular circulation.} is often claimed as a “masterpiece” of Vedanāya\-gam Sāstriyār. According to the author, it was written based on the inspiration from \textit{Tiru Kuṟṟāla Kuravañji} of Thirikooda Rasappa Kavirayar, a folk opera dedicated to Lord Shiva dwelling in the famed ancient shrine in the Kutralam hills, which is hailed as a great piece of literature by many Tamil scholars. The Kuravañji opera, as a literary form centers around the life and pride of mountain dwelling communities and the devotional love of the maiden (Nāyikā) towards her Lord, themes that are well established in the Hindu poetic tradition. But in \textit{Bethlehem Kuravañji}, the maiden is a character that represents the Church according to the author, which is the Bride of Lord Jesus. And, in place of the lovely Malaya mountain (Western Ghats), it sings the glory of the mountains of Canaan, a Semitic-speaking desert region in the ancient Near East mentioned in the Biblical narratives. Discerning readers of Tamil literature would immediately recognize that the work by Vedanāyagam Sāstriyār comes out not just as a cheap imitation of the original work devoid of any beauty and originality, but the one that has totally perverted the very aesthetics of the literary form by associating it with alien cultural and geographic domains, just to fit it into the Christian framework. \textit{Tiru Kuṟṟāla Kuravañji }remains much loved and much staged piece in \textit{Bharatanāṭyam} dance performances till today whereas \textit{Bethlehem Kuravañji }due to its poor artistic value, is performed by comparatively lesser number of artists, be it \textit{Bharatanāṭyam}or Karnatic Music\endnote{ \url{https://www.youtube.com/watch?v=PhTvCoH_gbI}}.

\newpage

Analyzing the Keerthanams of Vedanāyagam Sāstriyār undoubtedly establishes his overt and excessive usage of words and phrases replete with Hindu spiritual connotation. This is not just because of “linguistic” and “literary” reasons, Tamil being a language endowed with Indic (Hindu, Buddhist, Jain) classical literature. This is done deliberately to obfuscate the boundary between Christianity and the popular Hindu Bhakti traditions, even while actively engaging in the insult and condemnation of those very same Hindu Bhakti traditions as false and vulgar. A few examples from some of his popular Keerthanams are given.

\begin{myquote}
\textit{aṉanta ñāṉa svarūpā - nama om\\ aṉanta ñāṉa svarūpā}
\end{myquote}

\begin{myquote}
\textit{gaṉamkôḷ mahimaiyiṉ karttāve - gāttira nettira barttāve \\ kāṇka vantāre - nama om kāṇka vantāre\\ karuṇākara devā (aṉanta) }
\end{myquote}

\begin{myquote}
\tamil{அனந்த ஞான ஸ்வரூபா \enginline{-} நம ஓம்\\ அனந்த ஞான ஸ்வரூபா }
\end{myquote}

\begin{myquote}
\tamil{கனம் கொள் மகிமையின் கர்த்தாவே \enginline{-} காத்திர நேத்திர பர்த்தாவே \\ காண்க வந்தாரே நம ஓம் \enginline{-} காண்க வந்தாரே \\ கருணாகர தேவா (அனந்த) }
\end{myquote}

Apart from the usage of the Tamil forms of the Sanskrit words \textit{ananta}, \textit{jñāna}, \textit{svarūpa}, \textit{mahima}, \textit{kartā}, \textit{gātra}, \textit{netra}, \textit{bhartā}, \textit{deva} etc. the Hindu chant ‘\textit{nama om}’ has been used unabashedly in this song.

\begin{myquote}
\textit{devā irakkam illaiyo - ecu devā irakkam illaiyo}
\end{myquote}

\begin{myquote}
\textit{jīva parabrahma yehovā tirittuvattiṉ \\ mūvāḷ ôṉṟāka vanta tāvītiṉ maintaṉ ôre (devā)}
\end{myquote}

\begin{myquote}
\tamil{தேவா இரக்கம் இல்லையோ \enginline{-} ஏசு தேவா இரக்கம் இல்லையோ }
\end{myquote}

\begin{myquote}
\tamil{ஜீவ பரப்ரஹ்ம யேஹோவா திரித்துவத்தின்\\ மூவாள் ஒன்றாக வந்த தாவீதின் மைந்தன் ஒரே (தேவா)}
\end{myquote}

Here, the composer takes the established Vedantic philosophical terms \textit{Jīva} and \textit{Parabrahman} and just mixes them with Jehovah and calls this threesome as Trinity (\textit{tirittuvam}) which came as David’s Son. This, apart from being very comical, also smacks of clear intent at appropriation.

\begin{myquote}
hspace{.3cm}\textit{ādiyum antamum illāy caraṇam\\hspace{.3cm} atumuṉ atumuṉ atumuṉ caraṇam\\hspace{.3cm} ālfā ômegāve caraṇam\\hspace{.3cm} adiyeṉ naduve mudive caraṇam}
\end{myquote}

\begin{myquote}
hspace{.3cm}\tamil{ஆதியும் அந்தமும் இல்லாய் சரணம்\\hspace{.3cm} அதுமுன் அதுமுன் அதுமுன் சரணம் \\hspace{.3cm} ஆல்பா ஒமேகாவே சரணம்\\hspace{.3cm} அடியேன் நடுவே முடிவே சரணம்}
\end{myquote}

This song runs to several stanzas, all ending with the popular Hindu chant “\textit{saraṇam}” signifying surrender, combining \textit{ādi} (beginning), \textit{anta} (end), Alpha, Omega all in a heady mix.

As per the official website \url{http://www.sastriars.org/} of the descendants of Vedanāyagam Sāstriyār, they are still involved in performing ‘\textit{Kathākālakṣepa}-s’ (a generic name for various styles of Hindu musical discourses of \textit{Itihāsa} and \textit{Purāṇa}) based on his works and some of them even included the appellation “Bhāgavatar” to their names.


\section*{Plagiarism, Collaboration or Appropriation?}

A few examples from Christian Keerthanam were given in the above sections. A complete survey of the corpus of Christian Keerthanam would not change our view of the general style, tone and tenor of these compositions. It is evident that most of the “composers” of these Keerthanams were neither good original musicians, nor good original poets. The only force that propelled them was the religious zeal to “create” Tamil Christian hymns set to Karnatic music tunes. Naturally, one cannot expect much of originality or genius in such a pursuit. They unabashedly plagiarized and pilfered both the music (\textit{saṅgīta}) and much of the lyrics (\textit{sāhitya}) from the prevalent Hindu Keerthanam compositions and tweaked them here and there with Christian names and themes.

It is important to note that this is not just plagiarism of words, phrases or lines in the literary or artistic sense, indulged in by individual poets or composers. That can be pardoned in the spirit of artistic freedom. This is clearly part of the long-term strategy backed by a well-established theological concept, having serious consequences in the socio-cultural developments and conflicts taking place in the Indian society. Even a renowned artist like Chitravina Ravikiran discusses this in isolation in a very casual manner in his article “Don’t crucify the artists” (Ravikiran 2018). He creates totally false and baseless equivalences of Aruṇagirinātar using a word ‘\textit{salām}’ in one place out of thousands of Tiruppugaḷ verses (because that Arabic word had sneaked into Tamil colloquial vocabulary by that time) and Dīkṣitar composing songs on Hindu deities in “Western Notes” tunes (this would form not even 0.5\% of his otherwise rich and magnificent musical repertoire) to the massive and near-total plagiarism of Christian Keerthanam composers shamelessly stealing everything from Hindu \textit{Kīrtana}-s and Hindu religious texts. More importantly, he completely misses the point about the motive of Christian composers, who were not really artists in the true sense of the word but were all evangelists and proselytizers without exception.

The music historian V. Sriram, in his article “A chronicle of collaboration” (Sriram 2018) goes to the extent of citing a “long standing Carnatic tradition in the Church”. He writes -

\begin{myquote}
“The composer’s (Thyagaraja) contemporary, Vedanayagam Sastriar, created songs and operas in the Carnatic style. Some of the tunes are very closely modelled on Tyagaraja’s songs. ‘Sujana Jivana’ (Harikamboji) has a parallel in ‘Parama Pavana’. At this point in time it is impossible to state who borrowed whose tune and made it his own”. 

~\hfill (\textit{spellings as in the original})
\end{myquote}

It would take either extreme sense of ignorance or perversion to even speculate that the sagely Tyāgarāja, the unparalleled musical genius would have “borrowed” an ordinary musical tune from a copycat composer like Vedanāyagam Sāstriyār whose musical prowess is not even remotely comparable. In all the examples that he gives throughout this article, what is common is the fact that the Hindu Karnatic \textit{vidvān}-s, out of their magnanimity and large heart, engaging and entertaining Christians who show interest in Karnatic music, sometimes helping them to compose songs, sometimes even signing one or two Christian hymns, whereas the Christian aspirants are very clear in their sole aim, of putting Karnatic music at the “service” of Christian faith and evangelism. To term this as “collaboration” calls for extreme sense of imagination. What the Hindu Karnatic \textit{vidvān}-s\break displayed was a behavior described as \textit{sadguṇa vikṛti} (perversion of a noble trait). Such patronizing behavior of accommodation and forgiving, even towards the sworn enemies of their \textit{dharma} and society has been the bane of Hindus historically, with many examples from the pages of history like the king Prithviraj Chauhan who pardoned the murderous Islamic invader Mohammed of Ghor, despite winning him in war multiple times.

\vspace{-.2cm}

\section*{Abraham Pandithar and His Musical Treatise}

\subsection*{1. Antecedents of Pandithar}

The life story of Abraham Pandithar (1859-1919) is quite interesting and flamboyant. He was born in the village Sambavar Vadakari near Kutralam in a Nadar family that had quasi-religious affiliation to Christianity, in the sense that they continued with many Hindu traditional practices inherited from their ancestors while also indulging in Christian worship and commune. When he was well settled in his career as an English teacher in the Norman English school in Dindigul in his thirties, he was drawn to Karuṇānandar, a \textit{siddha puruṣa} living in the nearby Suruli hills. He became an ardent devotee and disciple of Karuṅānandhar and got transformed into an adept \textit{vaidya} (traditional physician) in the Siddha medicine. Around 1883, he moved to Thanjavur and established a flourishing Siddha medical center that brought him a lot of fame and wealth. The British government awarded him with the title “Rao Bahadur” in 1909 for his services in the medical profession. He had also developed keen interest in Karnatic music and continued learning under competent teachers in Dindigul and Thanjavur and became adept at playing many musical instruments like harmonium, fiddle and \textit{vīṇā}. He established Thanjavur Sangita Vidya Mahajana Sangam that held six music conferences during the years 1912-1916. Being a devout Christian, Pandithar also authored about 100 Christian Keerthanams, all falling under the same genre of the ones discussed above in this paper, apart from his famed treatise \textit{Karunamirtha Sagaram. }(Jeyamohan 2001) (TVU thanjai-1)

\vspace{-.2cm}

\subsection*{2. Musicological Aspects in the Treatise}

\vspace{-.2cm}

\textit{Karunamirtha Sagaram}, the voluminous 1400-page musical treatise in Tamil written by Abraham Pandithar remains a much-celebrated work. The name has dual connotation, one meaning music being the nectar to ears (\textit{karṇa} + \textit{amṛta}) and the other denoting the name of his Guru Karuṇānandar. The author claimed it was the fruit of his labor of love and fifteen years of research on the subject. When published in 1917, it received wide range of testimonials and accolades from a variety of savants and experts, which included the pontiffs of Śaiva Adhīnams, renowned Tamil scholars like U.Ve. Swaminatha Iyer and R. Raghavaiyangar, doyens of music like Harikeśanallur Muthaiah Bhagavathar and renowned music patrons like Ettayapuram Zamindar (Pandithar 1917: Ch. \textit{paayiram})

The musicological aspects of this work and the axioms and the theories it proposed have been the topic of much discussion among the experts in the field ever since it appeared. These are dealt in Parts 2,3 and 4 of the treatise. Part 2 titled “On the Shrutis” is dedicated to the detailed analysis of the \textit{śruti} and \textit{rāga} system as propounded by Śārṅgadeva, the author of \textit{Saṅgīta Ratnākara}, a seminal text in Karnatic music tradition. Part 3 titled “Tamil musical Shrutis in vogue in South India” elaborately describes the \textit{śruti} and \textit{rāga} system practiced in Tamil music based on Sangam literary works and \textit{Silappatikāram}, the renowned Tamil epic. Book 4 deals with assorted topics like Yāḻ (ancient \textit{vīṇā}- like string instrument), science of \textit{śruti}, survival and sustenance of Tamil music and the author’s general opinions about Indian music. It is neither the purpose nor in the scope of this paper to get into these topics.

\vspace{-.2cm}

\subsection*{3. Questionable Religious and Historical Narratives in the Treatise}

The purpose of this paper regarding this treatise is to give a glimpse of all the chapters of Part 1 titled “The summary of the history of Indian music” (280 pages). This is to illustrate how the author systematically attempts to build a specific kind of dubious quasi-historical narrative about Indian classical music by combining Biblical legends, Lemurian theory and Dravidian racist theories in a heady mix. At the end of this, he postulates Tamil to be mother of all languages of India and Tamil music to be the mother of all classical music traditions in India.

Chapter 1 titled “Glory of Music and Its Origin” starts with paying obeisance to Kartan (the creator) in an elaborate manner, and then continues with the Bible quote “In the beginning was the Word (\textit{nādam}), and the Word was with God, and the Word was God… (John 1:1)”. After equating the Word of the Bible with the concept of \textit{nāda} in Indian music tradition, the work continues to explain how all creation started from \textit{nāda}. There is a brief mention of \textit{Sāma Veda}, Śiva, Sarasvatī etc. after which it elucidates how divine and lofty music is. Chapter 2 is titled “The Evidences in the True Scripture (\textit{sathiya vedam}) for the Antiquity of Music and the Musical Instruments of that age” and runs to 20 pages. The author emphatically puts his complete belief in the history as narrated in the True Scripture, which means the \textit{Bible}. “As per the Genesis given by Moses (\textit{mosē munivar}) 3400 years ago...”, the text goes quoting several verses from the Bible. The subsequent sub-headings in this chapter go like this - “The song of Moses and dance of Miriam”, “The song of Deborah”, “The music and devotion of King David and the instruments of his time”, “The music of King Solomon and appointment of 288 singers in the temple”, “The Golden Statue established by Nebuchadnezzar in Babylon and the instruments played in front of it”, “Playing of instruments in Babylon palace morning and evening”, “Nimrod building Babylon, Nineveh and other towns”, “The glory of Babylon and its destruction”, “The glory of Nineveh and its destruction”, “The penance of the noble Enoch” and “Noah before the Great Deluge”. In these pages, the author goes on and on with narration of these Biblical legends in Tamil, with the reader exasperatedly wondering what has all this got to do with South Indian music, really.

Then starts Chapter 3 titled “The Tamil Nations and the Arts destroyed due to Great Deluge” (40 pp) with the opening paras saying that it is very difficult to assess any history of this period, as all this is told only in “old stories”. It goes on to narrate the story of the “Dravidian King Satyavrata” from \textit{Śrīmad Bhāgavata} 11.30 \textit{verbatim}, followed by direct quotations of pages and pages from the Manual of the administration of the Madras Presidency (the Manual faithfully documents the Aryan Dravidian racist narratives peddled by the Colonial historians of that period as the authentic history of the region). Then it postulates the thoroughly scientifically discredited Lemurian and ‘Kumari Kandam’ theories with many citations from colonial writings including those of Robert Caldwell.

Chapter 4 titled “The Antiquity of Tamil language” (60 pp) presents the typical Tamil Supremacist and Dravidianist arguments about how Tamil is older than Sanskrit, resorting to pseudo-linguistic propositions and strawman arguments like making classical Sanskrit based on Pāṇini grammar as an entirely different language discontinuous with Vedic Prakrit language etc.

Chapter 5 titled “Various opinions about Indian music” (35 pp) gives a survey of then prevalent ideas on the topic.

Chapter 6 titled “Some Notes about the Practitioners of South Indian Music” (100 pp) gives historical information on how classical music thrived in Tamil Nadu, right from Sangam age through Chola, Pandya dynasties and all the way to 19th century.

Chapter 7 (45 pp) is about the charter of Thanjavur Sangita Vidya Mahajana Sangham and its deliberations.


\subsection*{4. The Legacy of Abraham Pandithar}

When the treatise appeared and got its initial acclaim, all the discussions about it were concerning the debate on \textit{śruti }system and on the unique musical ideas present in ancient Tamil literary works that were not present in any ancient Sanskrit works like \textit{Nāṭyaśāstra} of Bharata. It appears that the historiography filled with Biblical and Aryan-Dravidianist racist narratives and Lemuria theories was totally neglected and overlooked by those who read and praised the book. But, in the later decades, when Abraham Pandithar is presented as an eminent South Indian musicologist of the 20th century and a “Christian icon” of Karnatic music, his dubious historiography also gets credence and validity, along with whatever original scholarly contributions that he might have made regarding some aspects of ancient Tamil music.

For those familiar with the long history of nexus between the colonial racist theories, Dravidian movement and Christian evangelism in Tamil Nadu, it is not difficult to speculate the logical consequences of Pandithar’s legacy. His treatise is positioned as a powerful tool in furthering the attempts to appropriate and even digest Karnatic music partly or fully into Christianity.

How a similar nexus operates in the domain of religion is well explained in the two chapters of the book \textit{Breaking India} – viz. Digesting Hinduism into ‘Dravidian’ Christianity and Propagation of ‘Dravidian’ Christianity (Malhotra and Neelakandan 2011: 88-153).

\newpage

\section*{“Christian Tamil Music Primer”}

A Christian Karnatic music Primer for learners has been published recently (Lakshmi and Renuka 2014), on the lines of the renowned and repeatedly republished \textit{Gānāmirta Bodhini: Saṅgīta Bāla Pāṭam} by A. S. Panchapakesa Ayyar (first edition 1954).

Nothing so blatantly exemplifies as to how a typical Christian mindset deals with a classical Hindu art like Karnatic music like this book. It is also to be noted that the book is titled as the primer of “Tamil music” mainstreaming the theoretical framework of Abraham Pandithar that was discussed in the previous section. The striking feature of this primer is that it has Christian lyrics in Tamil for all the traditional \textit{gīta} and \textit{varṇa} songs learned by Karnatic music students world over. There are Christian lyrics set to the notes of famous Piḷḷāri \textit{gīta} compositions like \textit{Śrī Gaṇanātha}, \textit{Kêraya nīranu} and \textit{Paduma nābha }(attributed to Purandara Dāsa), Sañcāri \textit{gīta} compositions like \textit{Varavīṇā}, \textit{Kamalajādaḷa} etc. and to the notes of well-known \textit{varṇa} compositions like \textit{Ninnukori}, \textit{Jalajākṣa}, \textit{Vanajākṣi}, \textit{Sarasuḍa}, \textit{Erā nāpai} etc.

There have been numerous non-Hindu students and artists of Karnatic music over many decades of the past, including Westerners and people from other foreign countries, not to mention Indian Muslims and Christians. Till today, there is no record of instances where any of them had any aversion to singing these simple yet sublime traditional lyrics. The very fact that they came to learn this divine music meant that they respect the heritage and the tradition of this art form. For a lover and practitioner of this music, these \textit{gīta}-s and \textit{varṇa}-s are not just learning aids or “combinations of musical notes”, but they are a tribute in the honor of the divinities that embody the music and the great \textit{guru}-s and masters who grew and nurtured it over centuries. To separate these lofty traditional lyrics from the notes and replacing them with Christian ones is nothing but a grave insult to the very art, its soul and its masters. It is very clear as to what motivated the above-mentioned duo in this deplorable pursuit. They wanted to create a “Christian” version of Karnatic music, right from the very learning stages, by simply copying everything from the existing tradition and shamelessly replacing Christian stuff over it. If this is not a grotesque attempt at appropriation, what else is? Those paragons of harmony who keep trumpeting “music has no boundaries”: what is their reaction to such attempts?


\section*{Conclusion}

It is evident that the attempts outlined in this article are an ongoing process under the expansionist agenda of different Christian denominations and Churches, though it may not have any centralized authority or coordinated strategy. This is because, once the theological import of it is well understood and assimilated, Christian evangelists or mission groups or even individuals with the religious zeal do not need any further directive or guidance and they go ahead on their own to put it into action. This paper already presented the example of \textit{Christian Tamil Music Primer} regarding which no evidence of “official” Christian connection can be explicitly proved. But it would be \textit{naïve} on part of those Hindus concerned with preserving the civilizational bond of Karnatic music intact to dismiss such developments as one-off events or fancies of fringe elements.

We already have a good parallel case in point regarding Tamil classical literature, where the attempts of systematically “De-Hinduizing” it has gained momentum over the past decades. A threshold point has already been reached wherein making outrageous Christian association claims or denying the very Hindu character of Tamil literary works and idioms are considered very normal and acceptable (Jataayu 2017).

Given this, it is very important that Karnatic music insiders and \textit{rasika}-s are aware of this phenomenon and are sensitive to it. Whenever there are debates or controversies about some seemingly “simple” and “straightforward” questions like the appropriateness of Karnatic music \textit{vidvān}-s singing Christian Keerthanams or someone advocating alienation of Hindu ethos from music performances or an academic presenting unsubstantiated things about the history of Karnatic music, those issues must be approached and analyzed not in isolation, but with the awareness about the overall civilizational context, the Indian socio-cultural realities and the clash of religions and ideologies playing out.


\section*{Bibliography}

\begin{thebibliography}{99}
\itemsep=0pt
\bibitem{chap5-key01} Appasamy, Grace Parimala. (1995). \textit{Vedanayaga Sastriar: a Biography of the Suviseda Kavirayar of Thanjavur}. Thanjavur: Thanjai Vedanayaga Sastriar Peravai.

 \bibitem{chap5-key02} Goel, Sita Ram.(1986).\textit{ History of Hindu-Christian Encounters: AD 304 to 1996. }New Delhi: Voice of India. (Online version).

 \bibitem{chap5-key03} Jataayu. (2017). “DeHinduization of Tamil Literature” panel talk in Swadeshi Indology – 3 Conference \url{⟨https://youtu.be/e8OTfay7a2I?t=41m33s⟩}. Accessed on 10 Mar 2019.

 \bibitem{chap5-key04} Jeyamohan. (2001). “Tanjai Abraham Pandithar” \url{⟨https://www.jeyamohan.in/369/⟩}. Accessed on 10 Mar 2019.

 \bibitem{chap5-key05} Johnson, Y. Gnana Chandra.(2012).\textit{ Kristava Kīrtanai Kaviñargaḷ, }Chennai:F2, Bethel Enclave, 5, Bethel Puram, East Tambaram,.

 \bibitem{chap5-key06} - (2014). “\textit{Varalāŕŕu nokkil kristava kīrtanaikaḷ}” (11 August 2014) \url{⟨http://johnson11mcc.blogspot.com/2014/08/blog-post_9.html⟩}. Accessed on 10 Mar 2019.

 \bibitem{chap5-key07} \textit{Karunamirtha Sagaram. }See Pandithar (1917).

 \bibitem{chap5-key08} Lakshmi Bai, I. K. and Suresh, Renuka. (2014). “\textit{Kristava Thamizhisai Bōdini”} \url{⟨http://myjesus.in/?p=content/christhava%20thamizisai%20pOthini.pdf⟩}. Accessed on 10 Mar 2019.

 \bibitem{chap5-key09} Malhotra, Rajiv and Neelakandan, Aravindan. (2011). \textit{Breaking India: Western Intervention in Dravidian and Dalit Faultlines}. New Delhi: Amaryllis.

 \bibitem{chap5-key10} Manickam, S. (1981). “Hindu Reaction to Missionary Activities in the Negapatam and Trichinopoly District of the Methodists, 1870-1924”. \textit{Indian Church History Review}, December 1981.

 \bibitem{chap5-key11} Pandithar, Abraham. (1917). \textit{Karunamirta Sāgaram. }Revised Edition. Chennai: Tamil Virtual University. (Online version). \url{⟨http://www.tamilvu.org/ta/library-l9800-html-l9800ind-147220⟩}. Accessed on 10 Mar 2019.

 \bibitem{chap5-key13} Ramanujachari, C. (Tr.) (2001, 19571). \textit{The Spiritual Heritage of Tyagaraja. }Madras: The Ramakrishna Mission Students’ Home.

 \bibitem{chap5-key14} Ramanujam, Srinivasa. (2018). “TM Krishna’s kutcheri at church”. \textit{The Hindu}. \url{⟨https://www.thehindu.com/entertainment/music/tm-krishnas-kutcheri-at-church/article22868515.ece⟩}. Accessed on 10 Mar 2019.

 \bibitem{chap5-key15} Ravikiran, Chitravina. (2018). “Don’t crucify the artists: Chitravina Ravikiran debunks the myths surrounding the OS Arun fracas” (\textit{Inmathi}, Aug 11, 2018) \url{⟨https://inmathi.com/2018/08/11/10685/⟩}. Accessed on 10 Mar 2019.

 \bibitem{chap5-key16} Sivasubramanian, A. (2011). \textit{“Vedanāyaga sāstiriyin vedasāstirak kummi”} \url{⟨https://www.keetru.com/index.php/2010-06-24-04-31-11/2011-sp-204473665/16372-2011-08-30-03-51-22⟩}. Accessed on 10 Mar, 2019.

 \bibitem{chap5-key17} Sharma, Swati Goel. and Sharma, Madhur. (2019). “Peace Gesture Or ‘Ghazwa-E-Hind’? Imran Khan’s Ministers Indulge In Rabid Anti-Hindu Rhetoric Post Pulwama”. \textit{Swarajya.} \url{⟨https://swarajyamag.com/world/peace-gesture-or-ghazwa-e-hind-imran-khans-ministers-indulge-in-rabid-anti-hindu-rhetoric-post-pulwama⟩}. Accessed on 10 Mar 2019.

 \bibitem{chap5-key18} Sriram, V. (2018). \textit{“}A Chronicle of Collaboration\textit{”}. \textit{The Hindu}. \url{⟨https://www.thehindu.com/opinion/op-ed/a-chronicle-of-collaboration/article24732325.ece⟩}. Accessed on 10 Mar 2019.

 \bibitem{chap5-key19} Swartz, Christian Frederick. and Pearson, Hugh. (1835). \textit{Memoirs of the life and correspondence of the Rev. Christian Frederick Swartz: to which is prefixed a sketch of the history of Christianity in India: Vol I}. London: J. Hatchard \& Son.

 \bibitem{chap5-key20} TVU thanjai-1. “\textit{thanjai mu. Abraham pandithar}”, Tamil Virtual University. \url{⟨http://www.tamilvu.org/courses/degree/d051/d0513/html/d0513661.htm⟩} Accessed on 10 Mar 2019.

 \end{thebibliography}

\theendnotes

