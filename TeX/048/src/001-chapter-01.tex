
\chapter{The Music of Muttusvāmi Dīkṣita –\\ a Window into His Personality}\label{chapter1}

\Authorline{Gayathri Girish}

\begin{flushright}
\textit{(gayathrigirish@gmail.com)}
\end{flushright}


\section*{Abstract}

It is undeniable that Karnatic Music is what it is, due to its great composers, particularly the Trinity. This paper proposes to examine the pioneering contributions of one of the Trinity composers of Karnatic Music, Śrī Muttusvāmi Dīkṣita which in many ways, are universal and eternal. Considering the time when he lived, some of his broad–minded thoughts inspire current musicians and music lovers alike. Through the medium of music, Dīkṣita has conveyed various aspects of our religion, philosophy and culture, for posterity.

Keeping in mind the current propensities of demeaning the great composers of the past, this paper will highlight some of the special features of Dīkṣita’s music and his personality, which firmly establish that he was a divine saint composer. He was indeed a true \textit{‘Nādopāsaka’}. Like all other great composers, \textit{bhakti} and glorification of God form the foundation of his compositions too. In addition to this, his deep knowledge in other subjects like Vedānta, temple architecture, temple iconography, \textit{mantra–śāstra}, astrology etc. is ample proof that he was a true genius. 

Some aspects that will be examined in this paper are:


\section*{Contribution To Music}

Dīkṣita’s music has stood the test of time; it is not just relevant but enchanting and awe–inspiring even today. His contribution to the \textit{rāga} system, use of \textit{Madhyamakāla sāhitya, rāga mudrā, prāsa, yati, rāgamālikā, samudāya} \textit{kṛti}–s, \textit{vibhakti kṛti}–s, etc will be analyzed, which ultimately establish that his music will always be one of the indispensable aspects of Karnatic Music.


\section*{Basic Tenets of Hinduism}

Dīkṣita has composed \textit{kṛti}–s on several deities and several forms of each deity to indicate \textit{Saguṇa} worship. During the initial stages of one’s spiritual journey, one requires personal Gods based on varied interests of seekers. At the same time in almost all \textit{kṛti}–s he introduces the concepts of Advaita Vedānta to clearly establish the path towards realization of the non–dual \textit{Nirguṇa} Brahman. This speaks volumes about the personality of the composer and gives a very clear response to the Western criticism of \textit{mūrti–pūjā} and multiple Gods.


\section*{Composing In Hindustani Rāga–s}

Hindustani music came to be a separate genre of music only after Muslim invasion into our country. Before that, it was all one – \textit{Bhāratīya Saṅgīta}, with only \textit{Mārga} and \textit{Deśī} variations. Dīkṣita established the oneness of Indian music by composing in Hindustani \textit{rāga}–s. He also undertook extensive pilgrimages and composed on \textit{kṣetra}–s like Varanasi, Badrinath and Nepal in the North and many \textit{kṣetra}–s in the states of Tamilnadu, Andhra Pradesh, Kerala, thereby promoting national integration.


\section*{Disciples}

Dīkṣita taught music without considering the caste, creed and gender of his disciples. The Tanjore Quartet were not Brahmins and Kamalam was his lady disciple. In this context, some of the misconceptions of today relating to alleged casteism in music will also be discussed in the paper.


\section*{\textit{Noṭṭusvara}–S}

Though he composed Noṭṭusvara–s, inspired by Western notes, it only shows he was open to recognize the good from other cultures and religions, and this in no way amounts to any appropriation. Also, the Noṭṭusvara–s are only a small aspect of his compositions. He was a musical genius par excellence which is evident in his masterly compositions. For the kind of divine composer he was, he didn’t have any need to “copy” a Western tune.

Thus, this paper hopes to prove that the music of Dīkṣita reveals to us his divine personality and he was by no means an ordinary composer.


\section*{1. Introduction}

It is undeniable that Karnatic Music is what it is, due to its great composers, particularly the Trinity. The period of the Trinity can be considered as the golden age of Karnatic Music. Śrī Tyāgarāja, Śrī Muttusvāmi Dīkṣita and Śrī Śyāma Śāstrī, the Karnatic Music Trinity composers, were the trend setters in transforming Karnatic Music to what we see it today. Appropriately the period when they lived, could be called the “Golden Period of Karnatic Music”. 

This paper proposes to examine the pioneering contributions of Śrī Muttusvāmi Dīkṣita which are universal and eternal. Considering the time when he lived, some of his broad–minded thoughts inspire current musicians and music lovers alike. Through the medium of music, Dīkṣita has conveyed various aspects of our religion, philosophy and culture, for posterity. 

Origin of music can be traced from Vedic and Purāṇic periods. Our music tradition cherishes its origin in Sāma–veda. Our music treatises declare \textit{sāmavedād idam gītaṁ sañjagrāha pitāmahaḥ}

Śrī Tyāgarāja says in his \textit{kṛti Sāmaja–vara–gamana} in \textit{rāga} Hindoḷa:

\begin{verse}
\textit{sāma–nigamaja–sudhāmaya–gāna}
\end{verse}

There were exponents of music and dance during the Purāṇic period like Nandi, Arjuna, Hanuman etc and works attributed to them have been referred to in the \textit{kṛti, Hāṭakeśvara saṁrakṣatu mām} of Śri Muttusvāmi Dīkṣita in \textit{rāga} Bilahari:

\begin{verse}
\textit{māruti–nandy–arjunādi–bharatācāryair avedita–nartana–sphūrte}
\end{verse}

Music was considered as \textit{mokṣa–sādhana} and the art of music was referred to as Gāndharva–veda and was considered as Upa–veda.

\textit{Yājñavalkya Smṛti} (3.4.115) says

\begin{verse}
\textit{vīṇāvādanatattvajñaḥ śrutijātiviśāradaḥ} \dev{।}\\\textit{tālajñaścāprayāsena mokṣamārgaṁ niyacchati} \dev{।।}
\end{verse}

“When one knows the science (\textit{tattva}) of playing (\textit{vādana}) the vīṇā and is an expert in the classification of quarter tones (\textit{śruti}) and when he also knows rhythm (\textit{tāla}), he controls (\textit{niyacchati}) the path of liberation (\textit{mokṣa}) without any other effort (\textit{aprayāsa})”.

\begin{myquote}
Since music indirectly helps the ātman, it is respectfully called Upa–veda. Except Artha–śāstra all the other three – Āyurveda, Dhanur–veda and Gāndharva–veda have the term “veda” in their names. 

~\hfill (Ganapathi 1976:435) (translation my own)
\end{myquote}

There are several references to music in Tamil Sangam literature, \textit{Silappadikāram} etc. Later, between 5th century CE and 10th century CE, during the \textit{Bhakti} movement in Tamilnadu, saints like Tirunāvukkarasar, Sundarar, Sambandar and Māṇickavāsakar used music as an effective vehicle to spread \textit{bhakti} by composing several Tevāram hymns – the focus was on \textit{bhakti} and not the art form of music.

Only during the 17th century during the period of Karnatic Music Trinity, well–structured art–form was effectively combined with intense \textit{bhakti} to transform music into \textit{nādopāsanā} which guides us till date. Divine compositions of the Trinity soaked in \textit{bhakti} help the singers and listeners to understand basic tenets of \textit{Sanātana Dharma} and guide them towards the path of spiritual growth. Also, innovations that they have brought into the art form of music are path–breaking and help us to continue the improvements in our music which will make our music eternal.

In the backdrop of the current propensities of demeaning the great composers of the past, this paper will highlight some of the special features of Dīkṣita’s music and his personality, which establish beyond doubt that he was a divine saint composer. ‘Composer’ as a term would be inadequate to describe him; he was a true \textit{‘Nādopāsaka’}. While \textit{bhakti} and glorification of God is the foundation of his compositions like those of other great composers, his deep knowledge in other subjects like Vedānta, temple architecture, temple iconography, \textit{mantra–śāstra}, astrology etc. makes him an unbelievable genius.


\section*{2.1. Śrī Muttusvāmi Dīkṣita – A Brief Profile}

Muttusvāmi Dīkṣita was born in Tiruvarur (a town near Tanjavur, Tamilnadu). His father was Rāmasvāmi Dīkṣita, a composer, musician and musicologist himself. Muttusvāmi Dīkṣita had undergone intense education of Sanskrit language and all Vedic scriptures at an early age and learnt music from his father. He also became well–versed in playing the \textit{vīṇā}.

At a very young age, Dīkṣita attained proficiency in the Veda–s, \textit{āgama, kāvya, nāṭaka, alaṅkāra, jautiṣa} and \textit{śāstra}–s. Under the tutelage of his father, he mastered Veṅkaṭamakhin’s treatise, the \textit{Caturdanḍī–prakāśikā}. He thus acquired knowledge, wisdom and great piousness.

His family had traditionally followed the teachings of Advaita Vedānta. At a very young age he came under the tutelage of Cidambaranātha Yogin and later under Śrī Upaniṣad Brahmendra Yogin, a great Advaita Ācharya, who gave him a strong foundation in Advaita philosophy. He started composing in the Tiruttaṇi Murugan temple by Divine grace and used “Guruguha” as his signature. He then undertook pilgrimages to several shrines in Tamilnadu, Andhra and Kerala and composed wonderful \textit{kṛti}–s on all deities. He reached the Lotus Feet of the Lord in the year 1835 in Eṭṭayapuram in Tamilnadu, leaving behind several gems of \textit{kṛti}–s in Sanskrit. Around 472 of these are available to us today.

Muttusvāmi Dīkṣita, stands very different from the other two, Śrī Tyāgarāja and Śrī Śyāma Śāstri, in his approach. In addition to devotion to the Lord and excellence in musical structure, he brings in several new dimensions – scholarship in Sanskrit language, expertise in several \textit{śāstra}–s like \textit{mantra–śāstra}, astrology, astronomy, scriptures, temple architecture, iconography etc. His compositions, hence, provide a rich treasure of information on various aspects of our culture and provide valuable seeds for continuing research on several subjects. A very distinct message of Advaita Philosophy is embedded in most of his \textit{kṛti}–s.


\section*{2.2. Śrī Muttusvāmi Dīkṣita – Contributions and Versatility}

This paper will discuss Muttusvāmi Dīkṣita’s contributions and versatility with specific reference to the following:

\begin{itemize}
\item Music

 \item Language

 \item Expertise in other sciences

 \item Basic Tenets of Hinduism – Religion and Philosophy

 \item Universal Approach and Broad Outlook

\end{itemize}


\subsection*{2.2.1. Music}

The splendour of the music of Muttusvāmi Dīkṣita’s compositions can be analyzed under the following heads.

\subsubsection*{2.2.1.1. Rāga Mudrā}

In addition to his \textit{mudrā} “Guruguha”, Muttusvāmi Dīkṣita, in his compositions, introduced \textit{rāga mudrā} very intelligently and seamlessly integrated it with the \textit{sāhitya}. For example:

\begin{itemize}
\item \textit{Śrī–\textbf{kurañji}ta–kāma} in “\textit{Śrī–Veṇugopāla}” in the \textit{rāga} Kurañjī (\textit{Śrī} = Lakṣmī, \textit{ku} = Mother Earth).

 \item \textit{samā\textbf{na–varauj}ase mahase}) in “\textit{Hastivadanāyanamas tubhyam}” in the \textit{rāga} Navroj (\textit{samāna–vara–ojase} means “Gaṇapati’s lustre is like Guruguha, the son of Lord Śiva, husband of Umā).

 \item \textit{cintayāmy a\textbf{tanu–kīrti}m} in “\textit{Cidambara–naṭarājamūrtim}” in the \textit{rāga} Tanukīrti (\textit{atanu} = great, \textit{atanu–kīrtim} refers to Lord Śiva whose glory is great and unmatched).

 \item \textit{Vīṇ\textbf{ābherī} veṇu} in “\textit{Vīṇābherī}” in the \textit{rāga} Ābherī.

 \item \textit{cid–bim\textbf{bau lī}lā–vigrahau} in the \textit{kṛti} “\textit{Srī–pārvati–parameśvarau}” (meaning that Pārvati and Parameśvara are reflections of consciousness and have assumed forms) in the \textit{rāga} Bauḷī.

 \item \textit{sadgati–dāya\textbf{kāmbhoja}–caraṇena} in the \textit{kṛti} “\textit{Kailasanāthena}” in the \textit{rāga} Kāmbhojī, Dīkṣita describes the Lord’s feet (–\textit{ambhoja–caraṇena}) as capable of bestowing salvation (\textit{sadgati–dāyaka}–).

 \item \textit{pātāla\textbf{bila–hari}hayādy–amara–nuta)} in the \textit{kṛti} “\textit{Hāṭakeśvara}”, one of the \textit{pañcaliṅga}–\textit{kṛti–s} of Tiruvarur, Dīkṣita incorporates both the local mythology of this \textit{liṅga} (namely, the description of the Lord as existing in the cave of Pātāla (\textit{pātāla–bila–}), the nether–world and being worshipped by celestials (–\textit{amara–nuta}) Hari and others (–\textit{hari–hara–ādi–})) and the name of the \textit{rāga} Bilahari in the phrases.

 \item \textit{sadyojātādi–pañca\textbf{mukhāri–}ṣaḍ–varga–rahita–hṛt–sañcāra} in the\break \textit{caraṇa} of the Mukhārī\textit{ rāgakṛti “Pāhimām–Ratnācala}”. Sadyojāta, Vāmadeva, Aghora, Tatpuruṣaand Īśāna are the five faces of Lord Śiva. This phrase explains that the five–faced Lord (\textit{pañca–mukha}) resides in the hearts of (\textit{hṛt–sañcāra}) of people who have won over the six evils (\textit{ari–ṣaḍ–varga}), viz., \textit{kāma, krodha, lobha, moha, mada, mātsarya}. The name of the \textit{rāga} is found beautifully embedded in this phrase.

 \item \textit{ati–samī\textbf{parju}–mārga–darśitam} in the \textit{kṛti} in \textit{rāga} Paras, “\textit{Cintaye–mahāliṅgamūrtim”}. Dīkṣita refers to Lord Śiva directing his devotees to the straightforward path (–\textit{ṛju–mārga–}) to liberation. The \textit{rāga–mudrā} has been very cleverly incorporated in this passage.

\end{itemize}


\subsubsection*{2.2.1.2. Structure of Kṛti–s}

His \textit{kṛti–s} are well–structured, close–knit and written in graceful Sanskrit. Dīkṣita’s \textit{kṛti–s} do not usually have more than one \textit{caraṇa}; and as many as 157 of his creations are compositions carrying no \textit{anupallavi} or the \textit{anupallavi} acting as \textit{caraṇa}.

Dīkṣita’s \textit{kṛti–s} with the \textit{pallavi–caraṇa} format have enriched the variety of musical forms in Karnatic Music. \textit{Kṛti–s} composed in \textit{madhyamakāla} are highly popular, for example, “\textit{Śrī–Sarasvati”} (Ārabhī); “\textit{Pārvatī–patim”} (Haṁsadhvani); “\textit{Sarasvatī Vidhi–yuvatī}” (Hindoḷa); “\textit{Śrī–Raṅganāthāya”} (Dhanyāsī).

Since he did not compose multiple \textit{caraṇa–s}, his single \textit{caraṇa–s} tend to be quite lengthy as compared to the \textit{kṛti–s} composed in \textit{Pallavi–anupallavi–caraṇa} format. Such long \textit{caraṇa–s}, however, helped Dīkṣita to provide detailed information about various deities, shrines, \textit{Śrīvidyā} etc. The \textit{Madhyama–kāla–sāhitya} that he employed for such \textit{kṛti–s} helped in introducing variation in such long \textit{caraṇa–s}. Perhaps his only multiple–\textit{caraṇa} creations are his \textit{kṛti–s} ‘\textit{Māye tvaṁ}’ (Taraṅgiṇī) and his four \textit{rāgamālikā–s}.

Most compositions of Dīkṣita are set in \textit{Viḷamba–kāla} (slow tempo). This gives an opportunity to listeners to understand the meaning of \textit{sāhitya} and enjoy the music and mental peace that music can give.

Dr V Raghavan observes in his Śrī Muttusvāmi Dīkṣitacaritam, Pariśiṣṭam:

\begin{verse}
“\textit{yāvaddakṣiṇa deśyam gāndharvam bhuvi tiṣṭhet} \dev{।}\\\textit{tāvaddīkṣitanāma stheyam syād dṛṣadopi} \dev{।।}\\\textit{saṅgītasya guṇoyam nādāsvādarasena} \dev{।}\\\textit{lokāveśalayād yad dhatte citta samādhim} \dev{।।}
\end{verse}

\begin{myquote}
– As long as Karnatic Music exists, Dīkṣita’s name will remain as a firm stone. Music through its enchanting, enjoyable, rhythmic sounds engulfs the listeners’ minds and keeps them in an absorbed state. 

~\hfill Raghavan Dr. V. (1980:195) (Translation my own)
\end{myquote}


\subsubsection*{2.2.1.3. Svarākṣara}

\textit{Svarākṣara} is the concept in which the \textit{svara} coincides with the \textit{sāhitya–akṣara–s}. Dīkṣita has used these in plenty in his compositions. A few examples are the Bhūpāla \textit{kṛti} “\textit{Sadācaleśvaraṁ bhāvaye’ham}”, the Śaṅkarābharaṇam \textit{kṛti} “\textit{Sadāśivam upāsmahe}” and the Māhurī \textit{kṛti} “\textit{Māmava raghuvīra}”, the Sarasvati Manohari \textit{kṛti} “\textit{Sarasvati Manohari”.}


\subsubsection*{2.2.1.4. Group Kṛti–s}

Dīkṣita had developed a fascination for composing a series of \textit{kṛti–s} on common themes like the same deity, \textit{kṣetra} etc, perhaps wanting to explore the various dimensions of the subject. In some of these, he employed all the eight \textit{vibhakti–s}, the various cases that delineate a noun. He also composed a series of \textit{kṛti–s} in a set of \textit{rāga–s}, all ending with the same suffix (for example, “–gauḷa”). No other composer has attempted so many group \textit{kṛti–s} in such a planned and orderly manner.

The following are some of the important \textit{kṛti} collections:

\begin{itemize}
\item \textit{Guruguha–vibhaktikṛti–s}

 \item \textit{Kamalāmbā–Navāvaraṇakṛti–s}

 \item \textit{Navagraha kṛti–s}

 \item \textit{Nīlotpalāmbā–vibhaktikṛti}–s

 \item \textit{Pañcabhūta–kṣetrakṛti–s}

 \item \textit{Rāma–vibhaktikṛti–s}

 \item Tiruvarur\textit{–pañcaliṅgakṛti–s}

 \item \textit{Tyāgarāja–vibhaktikṛti–s}

 \item \textit{Abhayāmbā–vibhaktikṛti–s}

 \item \textit{Madhurāmbā–vibhaktikṛti–s}

\end{itemize}



\subsection*{2.2.2. Language and Choice of Words}

He had a good command over Sanskrit; and learnt to use it to express his aspirations in his compositions. He had a fascination for \textit{śabdālaṅkāra}, beautiful phrases and wordplay. Dīkṣita’s \textit{kṛti–s} are therefore adorned with poetic imagery and majesty steeped in devotion.

Except for one \textit{varṇa} in Toḍiand a\textit{ kṛti (daru)} in Telugu and three \textit{maṇipravāḷa– kṛti–s}(Sanskrit+Telugu+Tamil), all his other compositions are in Sanskrit. The choice of right words for his \textit{kṛti–s} is admirable. Some examples are:

\begin{itemize}
\item In the \textit{kṛti} “ \textit{ŚrīMātṛbhūtam}”, in Kannaḍa \textit{rāga,} the \textit{anupallavi} has “\textit{somamśirodhṛtasūrya gaṅgam}” where \textit{somam} is “sa + umam meaning “with Umā”, \textit{ śirodhṛtagaṅgam}refers to “gaṅgā adorned on his head”. The word \textit{sūrya} in this line does not refer to the Sun, since Lord Śiva does not have the Sun on his head. \textit{Sūrya} here refers to \textit{arka} (\textit{erukkam}) plant. Lord Śiva bears this plant on his head.

 \item Kāmākṣī – \textit{kā} refers to Sarasvatī, \textit{mā} to Lakṣmī and Kāmākṣī refers to the one who has Sarasvatī and Lakṣmī as her eyes. Dīkṣita has composed a lot of \textit{kṛti–s} on Kāmākṣī and has also brought out this meaning. For example, he uses “\textit{śāradā–ramā–nayane}” in the Hindoḷa\textit{ rāgakṛti} \textbf{“}\textit{nīrajākṣi kāmākṣī}\textbf{”.}

 \item He calls Kṛṣṇa “\textit{Rauhiṇeyānuja}” in the \textit{kṛti} “\textit{Siddhivināyakam}” – this refers to the one who is worshipped by Kṛṣṇa, who is the younger brother (\textit{anuja}) of Balarāma (Rauhiṇeya– son of Rohiṇī). He has used the term Rohiṇī to refer to Balarāma, to indicate the context of Kṛṣṇa worshipping Gaṇeśa. Rohiṇī brings to our mind Candra (husband of another Rohiṇī) and the \textit{Mahābhārata} incident on Śyamantakāmaṇi when Kṛṣṇa worships Gaṇeśa.

\end{itemize}


\subsection*{2.2.2.1. Usage of Compound Words}

Examples:

\begin{itemize}
\item In the Maṇiraṅgu \textit{kṛti} “\textit{Māmava paṭṭābhirāma}”, he says “\textit{paṅkaja–mitra–vamśa–sudhāmbudhi–candra medinīpāla rāmacandra”}.\textit{paṅkaja–mitra–vamśa}refers to \textit{sūryavaṁśa}, \textit{ambudhi} refers to the ocean/sea and \textit{ candra}refers to the moon. Dīkṣita says that Rāma, to the \textit{sūryavaṁśa} is like the moon to the ocean.

 \item In the Maṅgaḷakaiśikī \textit{kṛti} “\textit{Śrībhārgavībhadram}”, he says “\textit{pada–nayanānana–kara–naliṇīparamapuruṣaharipraṇayinī vadana–kamala–guruguha–dharaṇīvara–nuta–raṅganātha–ramaṇī}” which refers to Devī (Mahālakṣmī) as the wife of Lord Raṅganātha praised by the lotus–faced Guruguha and kings.

\end{itemize}


\subsection*{2.2.2.2. Upamā}

\textit{Upamā} is the art of creative or comparative imagery. Several examples of these are found in Dīkṣita’s \textit{kṛti–s}.

\begin{itemize}
\item The purity of the full moon is compared with the heart of a pious person – “\textit{candram bhaja mānasa sādhu–hṛdaya–sadṛśam}” (in “\textit{Candram–bhaja”})

 \item The radiance of Lord Śiva is compared with crores of suns – “\textit{bhānu–koṭi–koṭi–saṅkāśam}” (in “\textit{Ānanda–naṭana–prakāśam”})

 \item The heart of Lord Śiva is compared with melting butter – “\textit{navanīta–hṛdaya}” (in “\textit{Akṣayaliṅga–vibho”})

\end{itemize}

\begin{flushright}
Srivatsa, V.V. (2000:8)
\end{flushright}


\subsubsection*{2.2.2.3. Prāsa}

\paragraph*{2.2.2.3.1. Prathamākṣara–prāsa}

In his first composition “\textit{Śrī–Nāthādi–Guruguho}” in Māyāmāḷavagauḷa, the \textit{caraṇa} of the \textit{kṛti} has sixteen passages, all starting with the \textit{akṣara} “\textit{ma}”.


\paragraph*{2.2.2.3.2. Dvitīyākṣara–prāsa and anuprāsa}

There are numerous instances of this type of \textit{prāsa} seen in Dīkṣita’s compositions. A few examples are cited below:

The \textit{akṣara} “\textit{kṣa}” in the \textit{kṛti} “\textit{Akṣayaliṅga–vibho}”:

\begin{myquote}
\textit{dakṣaśikṣaṇa–}\\\textit{dakṣatara sura–}\\\textit{lakṣaṇavidhivi–}\\\textit{lakṣaṇa lakṣya–}\\\textit{lakṣaṇa}–\\\textit{bahuvi–}\\\textit{cakṣaṇa sudhā–}\\\textit{bhakṣaṇa guruka–}\\\textit{ṭākṣa vīkṣaṇa}
\end{myquote}


\paragraph*{2.2.2.3.3. Antya–anuprāsa}

The \textit{akṣara} “\textit{ṅgam}” in the \textit{kṛti} “\textit{Aruṇācalanātham”}:

\begin{myquote}
\textit{Aprākṛta–tejomaya–liṅgam atyadbhuta–karadhṛta–sāraṅgam}\\\textit{aprameyamaparṇābja–bhṛṅgam ārūḍhottuṅga– vṛṣaturaṅgam}\\\textit{viprottama–viśeṣāntaraṅgam vīraguruguhatāraprasaṅgam}\\\textit{svapradīpamaulividhṛtagaṅgam svaprakāśajita–somāgnipataṅgam}
\end{myquote}



\subsubsection*{2.2.2.4. Yati}

Dīkṣita often structured his lyrics in geometric patterns. He has often employed \textit{yati–s} such as \textit{gopuccha} (tapering like the tail of a cow) and the \textit{śrotovahā} (broadening like the flow of a river) for structuring his lyrics. For instance, in the following examples of \textit{kṛti–s}, he has used the tapering pattern of \textit{gopuccha}.

In the \textit{kṛti} “\textit{Śrī–varalakṣmi}” (\textit{rāga} Śrī):

\begin{myquote}
\textit{sārasapade}\\\textit{rasapade} \textit{sapade}\\\textit{pade}
\end{myquote}

In the \textit{kṛti“Māye tvaṁ yāhi”} (\textit{rāga} Taraṅgiṇī):

\begin{myquote}
\textit{sarasakāye}\\\textit{rasakāye}\\\textit{sakāye}\\\textit{āye}
\end{myquote}

In his \textit{kṛti} “\textit{Tyāgarāja–yoga–vaibhavam”, rāga} Ānandabhairavī, Dīkṣita uses both the \textit{yati–s}: \textit{gopuccha–yati} and \textit{śrotovahā –yati.}

The phrases are: (\textit{gopuccha–yati} – like a cow’s tail)

\begin{myquote}
\textit{Tyāgarāja–yoga–vaibhavam}\\\textit{aga–rāja–yoga–vaibhavam}\\\textit{rāja–yoga–vaibhavam}\\\textit{yoga–vaibhavam}\\\textit{vaibhavam}\\\textit{bhavam}\\\textit{avam}
\end{myquote}

and (\textit{śrotovahā–yati} – flowing or expanding like a river)

\begin{myquote}
\textit{śam}\\\textit{prakāśam}\\\textit{svarūpaprakāśam}\\\textit{tattvasvarūpaprakāśam}\\\textit{sakala–tattva–svarūpaprakāśam}\\\textit{śiva–śaktyādi–sakala–tattva–svarūpaprakāśam}
\end{myquote}


\subsubsection*{2.2.2.5. Rāga–s used byDīkṣita}

Dīkṣita followed the \textit{meḷa–paddhati} (a method or system of classifying \textit{rāga–s}) founded by Veṅkaṭamakhin. Since Dīkṣita belonged to this school, while handling \textit{vivādi meḷa–s}, Dīkṣita followed Veṅkaṭamakhin and avoided certain \textit{prayoga–s}. Kharaharapriyā did not belong to Veṅkaṭamakhin’s scheme, perhaps that is the reason why there is no known composition of Dīkṣita in that \textit{rāga}. The twenty–second \textit{meḷa} in that scheme was Śrīrāga. Veṅkaṭamakhin’s tradition treated Bhairavī and Ānandabhairavī as \textit{upāṅga–rāga–s} and Dīkṣita did the same.

Scholars opine that Dīkṣita’s major service to Karnatic music is that he gave expression to nearly 200 \textit{rāga–s} of Veṅkaṭamakhin. He also revived many of ancient \textit{rāga–s} that were fading away. Several ancient \textit{rāga–s} found a new lease of life through Dīkṣita’s \textit{kṛti–s}. Examples are \textit{rāga–s} like Maṅgaḷakaiśikī, Ghaṇṭā, Gopikāvasanta, Nārāyaṇagauḷa, Śūlinī, Sāmanta, Mārgadeśīand Mohananāṭa\textit{. }Even today their \textit{lakṣaṇa–s} are illustrated mainly through Dīkṣita’s creations.


\subsection*{2.2.3. Expertise in Other śāstra}

Dīkṣita had acquired great knowledge of \textit{jautiṣa, mantra–śāstra}, iconography and of temple architecture. He was a pilgrim all his life. He visited many shrines and sang about them and the deities enshrined there. About 74 of such temples are featured in his \textit{kṛti–s}. Maximum number of his \textit{kṛti–s} (176) are in praise of Devī, the Divine Mother, followed by (132) \textit{kṛti–s} on Śiva.Dīkṣita was the only major composer who sang in praise of Brahmā\textbf{.}He was intensely devotional and composed songs in praise of many Gods and Goddesses.

\subsubsection*{Astronomy and Jautiṣa}

One of the major contributions of Muttusvāmi Dīkṣita is the \textit{Navagrahakṛti}–\textit{s}, the compositions on the nine planets that he composed. Dīkṣita has made the texts of these \textit{kṛti–s} rich in symbolism, using a wide range of terminology drawn from Indian Philosophy, \textit{mantra–śāstra}, astronomy/astrology, mythological and Puranic allusions and iconographical descriptions of the planetary deities.

\textit{Navagraha} worship has been indicated in an enormous amount of literature like \textit{smṛti–s}, Dharma–śāstra–s, Upaniṣad–s, Purāṇa–s, \textit{jautiṣa}, \textit{āgama–s} etc. The varied texts inform us of the planets directly influencing our earth and the remote constellations, their interactions, their nature and function described symbolically through the character of the presiding deities. The \textit{Smṛti–s} and Purāṇa–s deal with the\textit{śānti} rites for the nine \textit{graha}–s. Yājñavalkya (I.294) says:

\begin{myquote}
“One desirous of prosperity, of removing evil or calamities, of rainfall (for crops), long life, bodily health and one desirous of performing magic rites against enemies and others should perform a sacrifice to planets”.

~\hfill (Janaki S.S 2012:180)
\end{myquote}

The \textit{Sūryatāpinyupaniṣad} contains many \textit{mantra}–s related to Sun. In its third \textit{paṭala} (p.57) it gives the eight–syllabled mantra “\textit{om ghṛṇiḥ sūryam āditya}” and says:

\begin{myquote}
\textit{Atra sauramanūni pravakṣyāmi nigamoditāni \dev{।} ghṛṇiriti dve akṣare \dev{।} sūryam iti trīṇi \dev{।} āditya iti trīṇi \dev{।} etadvai sāvitrasyāṣtākṣaram padam śriyābhishiktam \dev{।}} 

~\hfill (\textit{Sūryatāpinyupaniṣad} 1933: 57)
\end{myquote}

\vspace{-.1cm}

\begin{myquote}

~\hfill (Janaki S.S 2012:189)
\end{myquote}

The nine \textit{kṛti–s} are very special in terms of their content. Not only do they reveal several astronomical details about the planets, but also glorify the planets and presiding deities. These have been composed in a variety of \textit{tāḷa–s}. Information on \textit{mantra–s} related to \textit{Graha–s} have been referred to. It is believed that these compositions are as effective as chanting \textit{mantra–s}. According to tradition, the songs were composed by Dīkṣita to relieve the acute stomach–pain of one of his disciples, Tambiappan, who used to play on the Śuddha Maddaḷam in the Tyāgarāja temple.

Dr V.Raghavan, observes in his \textit{Śri Muttusvāmi–Dīkṣita}–\textit{Caritam Maha\-kavyam}:

\vspace{-.2cm}

\begin{verse}
mantrairupāsanamanuṣya navagrahāṇām\\ no śakyamityayamupāyamavaikṣataivam \dev{।}\\ śāstrārthameduranavagrahakīrtanāni\\ gānena tasya sukhasiddhikarāṇi cakre \dev{।।} 

~\hfill Raghavan Dr. V (1980:141)
\end{verse}

After composing these songs that richly glorify the nine planetary deities, Dīkṣita prayed to them that they confer their blessings on Tambiappan. The teacher also asked his pupil to sing the songs with sincere devotion and prayerful attitude, completely surrendering himself to Lord Tyāgarāja.


\subsubsection*{Mantra–śāstra and Tantra–śāstra}

Dīkṣita was an ardent \textit{Śrīvidyā–upāsaka} and an intense devotee of Devī. He was a master of \textit{tantra} and of \textit{yantra–pūja.} There are built in \textit{mantra–s} in many of his \textit{kṛti–s}.

The magnificent \textit{Kamalāmbā–Navāvaraṇakṛti–s} are jewels of Karnatic music. These compositions, intellectually sublime and steeped in deep devotion, are a testimony to Dīkṣita’s musical genius, his mastery over the Sanskrit language and his thorough knowledge of and intense dedication to \textit{Śrīvidyā}, \textit{Śrīcakra} and the worship of its \textit{āvaraṇa–s}. Through its graceful lyrics, majestic sweep of \textit{rāga}–s and descriptive details rich in mystical symbolism of \textit{tantra}, \textit{mantra}, \textit{yoga}, \textit{Śrīvidyā} and Advaita, Dīkṣita has virtually thrown open the doors to the secret world of \textit{Śrīvidyā}, to all those eager to approach the Divine Mother through devotion and music. It is amazing how he builds into each of his crisp and well–knit structure of lyrics, the references to the name of the \textit{cakra,} the names of its presiding deity, \textit{yoginī–s}, \textit{mudrā–s}, \textit{siddhi–s} and the Guru–s of the \textit{Kādi} tradition of \textit{Śrīvidyā,} and to the seed(\textit{bīja}) \textit{mantra–s}.


\subsubsection*{2.2.3.1. Dīkṣita and Temples}

Dīkṣita’s compositions serve as a guide with respect to temples – the aspects to see in a \textit{kṣetra}, the significance of the \textit{kṣetra} etc. They also describe temple festivals in detail. An example is the Śrīrāga\textit{ kṛti} “\textit{Tyāgarāja–mahadhvajāroha}” which describes in detail, the \textit{vasantotsava} in the Tiruvarur Tyāgarāja temple.


\subsubsection*{2.2.3.2. References to Temple Architecture}

In the \textit{kṛti} “\textit{Bhaktavatsalam}” in the \textit{rāga} Vaṁśavatī, Dīkṣita directly and indirectly alludes to the seven attributes of a \textit{saptāṁṛta} or \textit{sapta–puṇya–kṣetra} which are:

\begin{itemize}
\item The \textit{vimāna} or the canopy above the \textit{sanctum sanctorum}

 \item The \textit{maṇṭapa} in the temple complex

 \item The forest(\textit{vana}) found here

 \item The river that flows through the \textit{kṣetra}

 \item The location of the temple

 \item The city or \textit{nagarī}

 \item The \textit{puṣkariṇī} or tank

\end{itemize}

It may be interesting to add that unique features of temples have been captured like \textit{ĀsīnaBhairava} (\textit{Bhairava} in seated posture) in Śrīvāñcyam and Navanandi in Tyāgarāja Svāmi temple (Nandi in standing posture). Reference to Purāṇic anecdotes associated with the \textit{kṣetra–s} make these very interesting.


\subsubsection*{2.2.3.3. Dhyāna–Śloka–s}

In Hindu meditation, in \textit{japa} and \textit{dhyāna}, the form of personal God is evoked in the mind with a preliminary verse describing that form. This is called the \textit{dhyāna–śloka}. The \textit{mantra–śāstra}–s prescribe the forms of the deities to be contemplated, the posture(\textit{āsana}), the expression on the face, \textit{mudrā} (like the gesture of assurance–\textit{abhaya}) and the various kinds of weapons (\textit{astra}) held in the hands, the accompanying Goddess(es), attendants etc.

Dīkṣita was adept in \textit{mantra–śāstra}–s and the \textit{sāhitya} in many of his\break \textit{kṛti–s} describe the exact details of the deity as seen in the corresponding \textit{dhyāna–śloka}. The iconography set forth in his \textit{kṛti–s} is remarkable for its accuracy and conformity to \textit{mantra–śāstra} and \textit{śilpa–śāstra.}

\begin{itemize}
\item For example, the Malahari song “\textit{Pañcamātaṅga–mukha–gaṇapa\-tinā}” is a small song on Gaṇeśa who has five faces and eight hands and is found in the Tyāgarāja shrine at Tiruvarur. In the \textit{madhyamakāla–sāhitya} here, the various things that are held in the eight hands are mentioned – “\textit{varadābhaya–pāśa–śṛṇi–kapāla–danta–modaka–mudgara–akṣamālā–kareṇa”}. That is, one of the eight hands grant freedom from fear (\textit{abhaya}). The others carry a rope(\textit{–pāśa–}), goad(\textit{–sṛṇi–}), skull(\textit{–kapāla–}), broken tusk (\textit{–danta–}), sweetmeat(\textit{–modaka–}), hammer(\textit{–mudgara–}) and bead–garland(\textit{–akṣamālā–}). These iconographical details are the same as those in \textit{dhyāna–śloka} of Heramba Gaṇapati with five faces.

\end{itemize}



\subsection*{2.2.4. Basic Tenets of Hinduism – Religion and Philosophy}

\textit{Sanātana Dharma} is the \textit{Bhāratīya} way of life of which the following are the foundations:

\begin{itemize}
\item Steadfast faith that the Veda–s are primary \textit{pramāṇa} and enlighten us on issues which cannot be obtained from any other sources of knowledge.

 \item Faith in the concept of four goals of life (\textit{puruṣārtha–s}) and faith that \textit{mokṣa} or liberation is the goal to be sought after.

 \item Faith in Rebirth and law of \textit{Karman.}

 \item Simple life with contentment.

 \item Strive to achieve purity of one’s mind, which is devoid of six internal enemies, namely, \textit{kāma}, \textit{krodha}, \textit{lobha}, \textit{moha}, \textit{mada} and \textit{mātsarya.}

 \item Live in harmony with nature, fellow human beings and other entities of creation (animals, plants etc).

\end{itemize}

The above are seamlessly woven into one’s life–style according to \textit{Sanātana Dharma}.

Śri Muttusvāmi Dīkṣita has woven Vedic ideas, especially, concepts of Upaniṣad–s into almost all his \textit{kṛti–s}. Purity of mind is stressed as a qualification for Divine grace, in compositions like “\textit{Pāhi māṁ Ratnācala–nāyaka}” (\textit{kāmāri–ṣaḍvarga–rahita}) etc.

Noble men in our tradition have shunned materialistic wealth. Śri Muttusvāmi Dīkṣita was one such divine soul who refused Rājaśraya, as he didn’t want to do any kind of \textit{narastuti}. This has been stressed in \textit{kṛti–s} like “\textit{Hiraṇmayīṁ lakṣmīm}” (phrases like \textit{Hīna–mānavāśrayaṁ tyajāmi} and \textit{cira–tara–sampat–pradām)}. Hence Dīkṣita’s compositions clearly establish the superiority of the way of life propounded in \textit{Sanātana Dharma,} which holds that inner spiritual experience is much greater than material wealth.

\begin{myquote}
“Idol worship has been extensively criticized by people from diverse background. The rationalists and modern–day liberals dismiss it as superstition…. Max Muller considered idol worship as a sign that Hindus are still in the state of savagery and has proposed that Hindus should be civilized through European and Christian influence. Dr Ambedkar had questioned the rationale of the practice of \textit{Prāṇa–pratiṣṭhāpana}.....Kancha Ilaiah, Dalit activist and writer, has linked idol worship with rigid caste system and has claimed that caste system will become irrelevant only when idol worship becomes irrelevant. But, the staunchest criticism of idol worship has come from within the tradition. Swami Dayananda of Arya Samaj who had given the clarion call for returning to Vedas, has also criticised idol worship in the very severest of words.” 

~\hfill (Nithin Sridhar)
\end{myquote}

In Hinduism, religion and philosophy are complementary.

\textbf{It is commonly said that religion without philosophy is incomplete (since the purpose of religion is to help in spiritual growth) and philosophy without religion is impossible}.

Religious way of life gives a seeker, the basic qualifications required to understand and internalize the philosophical truth. Our tradition recognizes that God is only one, common to all living beings in this universe. It believes in a single formless Lord of the universe, Absolute Reality (\textit{Brahman}):

\textit{sadeva saumya idamagra āsīt ekameva advitīyam \dev{।} (Chāndogya Upaniṣad 6.2.1)}

\begin{myquote}
– In the beginning, dear boy, there was \textit{sat} (existence) alone, one only” (Translation by Svami Paramarthananda)
\end{myquote}

\newpage

\textit{satyamjñānamanantam brahma \dev{।} (Taittirīya Upaniṣad 2.1)}

\begin{myquote}
– \textit{Brahman} is Real (existence) (\textit{sat}), Consciousness (\textit{cit}), Infinite” (Translation by Svami Paramarthananda)
\end{myquote}

\begin{verse}
ānando brahmeti vyajānāt \dev{।} (\textit{Taittirīya Upaniṣad} 3.6)
\end{verse}

\begin{myquote}
– Pure Bliss(\textit{ānanda}) is \textit{Brahman}” (Translation by Svami Paramarthananda)
\end{myquote}

\begin{verse}
yattadadreśyam agrāhyam agotram avarṇam acakṣuḥśrotram tadapāṇi pādam \dev{।}\\ nityam vibhum sarvagatam susūkṣmam tadavyayam yadbhūtayonim paripaśyanti dhīrāḥ \dev{।।} 

~\hfill (Muṇḍaka Upaniṣad 1.6)
\end{verse}

\begin{myquote}
– That which is invisible, inconceivable, without lineage, without \textit{varṇa}, without eyes and ears, without hands and feet, and that which is eternal, all–pervasive, omnipresent, extremely subtle and non–decaying – that is what the wise behold as the source of all beings.” 

~\hfill (Translation by Svami Paramarthananda)
\end{myquote}

However, we have a plethora of godheads being worshipped all the time. Since the attribute–less God cannot be accessed by us even in thought, it is necessary to consider a tangible form for God.

Human life is a continuous mixture of pleasures and pains driven by \textit{karman–s} done in several births. Everyone requires a permanent hold to share experiences– both positive and negative and to get the confidence to face the uncertain future. It is very easy for human beings to relate to another human being for sharing emotions. Since all relations with fellow human beings are short–lived due to temporary or permanent separation, we need an imperishable entity to relate to, all the time.

God in a tangible form, becomes the natural choice for the entity to relate to. To be effective, the relationship should be very strong. Since human nature and aptitudes are widely varied, a single form of God will not be adequate to provide for the psychological needs of varied human beings. Thus, the concept of multiple forms of God evolved, so that a form chosen in line with the aptitude of an individual will help him or her to establish a strong, lasting relationship.

Most of the compositions of the Trinity are on various forms of deities to help worship by devotees to choose their \textit{iṣṭa–devatā}. Śrī Tyāgarāja has composed most of his \textit{kṛti–s} on his \textit{iṣṭa–devatā} Śrī Rāma and composed a few \textit{kṛti–s} on Lord Kṛṣṇa, Śiva, Devī etc. Śyāma Śāstri’s compositions are on Devī Kāmākṣi of Kancipuram and Tanjavur and a few \textit{kṛti–s} on Devī in Madurai etc.

Muttusvāmi Dīkṣita has composed on various forms of God as given below:

\begin{longtable}{|l|l|}
\hline
Gaṇeśa & 27 \\
\hline
Subramaṇya & 36 \\
\hline
Śiva & 132 \\
\hline
Devī & 175 \\
\hline
Mahāviṣṇu & 75 (includes Rāma and Kṛṣṇa) \\
\hline
Lakṣmī & 9 \\
\hline
Sarasvatī & 11 \\
\hline
Brahmā & 1 \\
\hline
\end{longtable}

It is seen that Muttusvāmi Dīkṣita has provided a wide canvas of \textit{kṛti–s} on a variety of forms of God to cater to a variety of devotees to select their chosen form of God to worship.

As Saxena observes, there is a distinct difference in the way Hinduism and other religions visualize God. While other religions consider God as isolated from Nature, created nature and ordains all those created, to worship the isolated God, Hinduism believes in a single, formless God who manifests in the entire creation. God manifests into several \textit{deva–s} for different roles and to provide an easy method to relate to one’s own personal God. Treating the entire creation as a manifestation of God helps a seeker to love everything in creation and reduce \textit{ahaṁkāra} and \textit{mamakāra.} This difference in opinion has resulted in confused criticism of Hinduism as a polytheistic religion or religion with multiplicity of Gods.

Our scriptures clearly establish one formless \textit{Brahman} as the Absolute truth and \textit{Saguṇaupāsanā} of various deities is only a stepping–stone to realize the Absolute reality. It is necessary in the beginning stages of spiritual growth. Once a seeker gets the required focusing capability and strong devotion, he should move to \textit{nirguṇa upāsanā}. Our scriptures prescribe an intermediate step of \textit{Viśvarūpa upāsanā} when the seeker sees the entire creation as the Lord. Hinduism not only tells us our destination but also outlines the means to reach the same. In line with this approach, \textit{kṛti–s} of Muttusvāmi Dīkṣita which primarily glorify the \textit{Saguṇa} deities also have references to concepts of \textit{Nirguṇa Brahman}, to lead the seeker in the right direction.

A few examples have been listed below:

\begin{longtable}{|p{4.7cm}|l|}
\hline
\textbf{PHRASE} & \textbf{KṚTI} \\
\hline
\textit{akhaṇḍa saccidānandam} & \textit{cintaya mākanda mūla kandam} \\
\hline
\textit{saccidānanda mūrte} & \textit{namaste para devate} \\
\hline
\textit{svayam prakāśakam} & \textit{śrī valmīka liṅgam} \\
\hline
\textit{akhaṇḍaika rasa pūrṇo\hfill \break  saccidānanda rūpiṇo} & \textit{akhilāndeśvaro rakṣatu} \\
\hline
\textit{śiva saccidānanda rūpeṇa} & \textit{ānandeśvareṇa} \\
\hline
\textit{nitya śuddha satva guṇam} & \textit{kāyārohaṇeśam} \\
\hline
\textit{nitya śuddha buddha muktāya} & \textit{kumbheśvarāya namaste} \\
\hline
\textit{abheda nitya śuddha buddha mukta}\hfill \break  \textit{saccidānandamaya paramādvaita sphūrteḥ} & \textit{śrī kamalāmbikāyāḥ param} \\
\hline
\end{longtable}

The above references indicate that Muttusvāmi Dīkṣita was truly an Advaitin. His family background also indicates a strong foundation in Advaita Vedānta. The time he spent with Upaniṣad Brahmendra Yogin in Kancipuram also strengthened the foundation of his knowledge in Advaita. Additionally, he received \textit{Śrīvidyā–dīkśā} from his Guru Cidambaranātha Yogin in Varanasi and appears to have practised \textit{Śrīvidyā–upāsanā}.

Thus, we see that through his divine compositions, Dīkṣita has seamlesslyintegrated a religious way of life through \textit{saguṇaupāsanā} based on Advaitic philosophy. He has thus addressed the criticism on idol worship as well as multiplicity of Godheads in Hinduism very effectively.

\newpage

\subsection*{2.2.5. Universal Approach and Broad Outlook}

Muttusvāmi Dīkṣita belonged to an orthodox \textit{Smārta} Brahmin family. He had his formal training in Veda–s and \textit{śāstra–s} at an early age. He was brought up in accordance with traditional disciplines and rigors. Also, he spent seven years with his Guru in Varanasi where he received additional teachings on our scriptures and was initiated into \textit{Śrīvidyā–upāsanā} which has its own disciplines and procedures. Someone with this background was expected to be conservative, restricted to the concepts and thoughts he was accustomed to. Interestingly, however, Dīkṣita was very different in his outlook. He believed in

\begin{verse}
\textit{ā no bhadrāḥkratavo yantu visvataḥ (Ṛg Veda} 1.89.1)
\end{verse}

“Let noble thoughts come to me from all directions”

Muttusvāmi Dīkṣita was way ahead of his time in thinking out–of–the–box and accepting good ideas from wherever they came from.

This could be illustrated as following:

\subsubsection*{2.2.5.1. Pilgrimage}

Muttusvāmi Dīkṣita lived in Tiruvarur for several years worshipping Somāskanda Tyāgarāja and Kamalāmbā. Unlike other composers, he chose to travel widely on pilgrimage to many places and compose on deities in those temples. His pilgrimage was very extensive in Tamilnadu and included Andhra Pradesh and Kerala. His stay in Varanasi made a deep impression on him and inspired by deities in Varanasi, he composed songs on Kālabhairava, Ḍhunḍhi Ganeśa, Viśvanātha and Viśālākṣī in temples in South India, built in accordance with temples in Varanasi on these deities. He has also composed songs on the deities at Badrinath (the \textit{kṛtiŚrī satyanārāyaṇam)} and at Nepal (the\textit{ kṛti paśupatīśvaram}).

This gives an opportunity for people in one location to learn about the greatness of other places and temples, which in turn generates desire to undertake pilgrimages to such places. This was made possible since his compositions not only glorified the deities but also described the specialties of temples and incorporated references to \textit{purāṇic} episodes related to the temples.


\subsubsection*{2.2.5.2. Hindustani Rāga–s}

Dīkṣita spent seven years at Varanasi, in the prime of his youth. He was captivated by the grandeur, the spaciousness and the purity of the ancient Dhrupad School of Hindustani Music. He learnt Dhrupad diligently; and that left a lasting impression on his works.

The Dhrupad way of elaboration appears to have captured his imagination while handling compositions in Karnatic \textit{rāga–s} too, by way of elaborate beginning of the composition, the \textit{tempo} of his songs being mostly in \textit{Viḷambakāla} (slow, measured and majestic), rich in \textit{gamaka,}just as the sliding \textit{meenḍ–s} in Hindustani music, as is seen, for instance, in the grandeur and slow paced majesty of\textit{Akṣayaliṅga Vibho}(in \textit{rāga}Śaṅkarābharaṇam)in contemplation of Lord Śiva, or in \textit{Bālagopāla}(in \textit{rāga} Bhairavi)\textit{\textbf{, }}portraying the delight and beauty of the divine child Kṛṣṇa, the \textit{kṛtiNīrajākṣi Kāmākṣi}\textbf{(}in \textit{rāga}Hindoḷa)and so on.

\textit{Kṛti–s} in Hindustani \textit{rāga–s}, like \textit{Jambūpate} in \textit{rāga} Yamunā, \textit{CetaŚri} in \textit{rāga} Dvijāvanti, \textit{Saundararājam} and \textit{Raṅgapura Vihāra} in Bṛndāvana Sāraṅga, \textit{Parimaḷa Raṅganātham} in Hamīrkalyāṇi are classic examples that bear testimony to his virtuosity.

In fact, it is well–known that Hindustani music came to be a separate genre of music only after Persian and Muslim invasion into our country. Before that it was all one – \textit{Bhāratīya Saṅgīta}, with only \textit{Mārga} and \textit{Deśī} variations. Dīkṣita established the oneness of Indian music by composing in Hindustani \textit{rāga–s}. He thus adopted and assimilated the Hindustani \textit{rāga–s} to the Southern style thus making them sound very indigenous and not alien.


\subsubsection*{2.2.5.3. Noṭṭusvara–s}

It is said that when Dīkṣita was in Madras, he had the opportunity to listen to the Western band in Fort St.George and impressed bythe simplicityof the notes, Dīkṣita composed the \textit{Noṭṭusvara–sāhitya–s}. Dīkṣita's remarkable versatility is exhibited in these \textit{Noṭṭusvara–sāhitya–s},\break which are a completely different genre from the rest of his work. The \textit{Noṭṭusvara–sāhitya–s} are thirty–nine in number. They are simple, short compositions sung at one stretch without the \textit{pallavi}, \textit{anupallavi, caraṇa} divisions. Dīkṣita composed Sanskrit \textit{sāhitya} on various deities set to Western tunes, some of whichare said to closely resemble English band music. These simple pieces are based on the scale of the \textit{Rāga} Dhīra\-śaṅkarābharaṇam and set to different \textit{tāḷa–s} such as \textit{tiśra–eka}, \textit{catuśra–eka, rūpaka} etc. Some famous \textit{Noṭṭusvara–sāhitya–s} are “\textit{Śakti–sahita–gaṇapatim”}, \textit{“Śyāmaḷe–Mīnākṣī}”. Even in these short compositions,\break Dīkṣita's impeccable command over Sanskrit and his adherence to the rules of prosody shines through. These compositions provide an ideal method of introducing Karnatic music to children and at the same time introduce to them as prayer to our deities.

It is hence observed that, Dīkṣita, an ardent Hindu devotee composer has appreciated other forms of music and adapted them in his \textit{kṛti–s}.\break This only shows that he practiced acceptance and not rigid conservatism.

Though it is commonly believed that the \textit{noṭṭusvara–sāhitya–s} are largely of Irish, Scottish origin and are entirely Western in terms of melodic content and approach, it is noteworthy that the concept of pure notes was no alien to Indian Music, considering that the ancient Indians had made significant advancement in development of scales through modal shift of tonic called ‘\textit{Grahabheda}’. Also, the evolution of the most refined and sophisticated system of \textit{gamaka}–s that is the essence of Karnatic Music could not have been possible without knowledge and employment of pure or flat notes. Also, devotional literature in the form of \textit{Stotra sāhitya} set to simple melodies based on pure notes have always existed in our musical tradition.

As a justification of the recent attempts of plagiarizing \textit{kṛti}–s of our composers with lyrics in praise of Jesus, Dīkṣita’s \textit{noṭṭusvara–sāhitya}–s are being given as a counter example\supskpt{\endnote{ \url{https://www.youtube.com/watch?v=FeD6iUOOgds}}}. However, his composition of \textit{noṭṭusvara–}s only shows he was open to recognize the good from other cultures and religions, and his intention was neither to appropriate or digest the Western tunes, nor to convert anybody. Adaptation is different from deliberate agenda–driven appropriation. Also, the \textit{noṭṭusvara}–s are only a small aspect of his compositions. He was a musical genius par excellence which is evident in his masterly compositions. He obviously didn’t have any need to ‘copy’ a Western tune. Also, it is not clear as to what were the circumstances or constraints under which he composed the \textit{noṭṭusvara}–s.



\section*{2.3. Disciples}

The main line of Dīkṣita’s \textit{śiṣya–s} was represented by his own family. After Bālusvāmi, there was Subbarāma Dīkṣita, Bālusvāmi’s daughter’s son whom he adopted as his son. Subbarāma Dīkṣita’s son was Ambi Dīkṣita who succeeded him as a court musician at Eṭṭayapuram.

There were many enthusiasts desirous of undergoing tutelage in music under Muttusvāmi Dīkṣita at Tiruvarur. The musicians and dancers attached to the Tyāgarāja temple were naturally attracted by his music. It was thus that slowly, a few \textit{nādasvara} Vidvāns and Devadāsī–s,\break people belonging to the \textit{Paraśaiva} community and the \textit{nāṭyācārya–s} belonging to Tiruvarur and adjoining villages decided to sit at Dīkṣita's feet and imbibe his art.

One of the earlier disciples was Śuddha Maddaḷam Tambiappan who belonged to the \textit{Paraśaiva} community and whose severe stomach ailment Dīkṣita cured miraculously by composing the \textit{navagraha kṛ}ti\textit{ ‘Bṛhaspate’} in praise of Jupiter. Nādasvara Vidvān Kūraināḍu Rāmasvāmi Piḷḷai, another eminent \textit{nādasvaravidvān} of those times, Terazundūr Bilvavanam, Tiruvārūr Ayyāsāmi Naṭṭuvanār who was a \textit{Nāṭyācārya} were some of the early disciples. Ammaṇi of Vaḷḷalar Koil and Kamalam of Tiruvarur, both of whom belonged to the Devadāsi community were the two prominent lady disciples. Tevur Subramaṇia Iyer, Tirukkaḍayur Bhārati, a Tamil scholar and the then eminent \textit{vīṇāvidvān} Avuḍayarkoil Veṅkaṭarāmayyar were a few of his disciples who were well–known musicians. Tambiappan had many disciples, some of whom became prominent musicians and composers. Many of these, in turn, taught Dīkṣita \textit{kṛti–s} to many people.

It was for the \textit{arangeṭram} of his lady disciple Kamalam that Dīkṣita wrote the Telugu \textit{varṇa} in Toḍi, “\textit{Rūpamu jūci”} in praise of Lord Tyāgarāja and the \textit{daru} in Śrīrañjanī, “\textit{Nī sarisāṭi”}.

Fascinated by these two exquisite pieces, one Gaṅgamuthu Oduvār requested Dīkṣita to come and live in Tanjavur for a few years. Thus, the four sons of Subbarāya Naṭṭuvanār (son of Gaṅgamuthu Oduvār and a veteran \textit{Nāṭyācārya}) namely Ponnayyā, Cinnayyā, Śivānandam and Vaḍivelu (popularly known as the Tañjāvūr quartet) were fortunate enough to come under the tutelage of Muttusvāmi Dīkṣita.

\newpage

There was another great revolution that was brewing as a result of Dīkṣita’s tutelage of the four brothers. The great composer’s music inspired the brothers to formulate the \textit{mārgam} of \textit{Bharatanāṭyam}. Today’s \textit{Bharatanāṭyamārgam} consisting of \textit{puṣpāñjali}, \textit{alārippu}, \textit{jatisvara}, \textit{śabdam}, \textit{varṇa}, \textit{pada}, \textit{jāvaḷi} and \textit{tillānā} was constructed by the Tanjavur Quartet. They also composed many pieces in all the above genres. It will be no exaggeration to declare that today’s \textit{Bharatanāṭyam} owes its high status to the seed sown by Dīkṣita.

Thus, it is significant to note that Dīkṣita's disciples not only came from all communities and belonged to both genders, but also that they comprised not only vocalists but instrumentalists too. Thus, all this proves beyond doubt that caste or gender was never an issue. Indeed, our tradition has always valued only merit and intrinsic worth in a person and not his caste, creed, gender or linguistic background. \textit{Bhāratīya saṁskṛti} believed in only \textit{varṇa} system. Even in the history of Indian music, there were many composers from different castes among the Nāyanmār–s, Āḻvār–s, Haridāsa–s, Vacanakāra–s and so on. Dīkṣita also belonged to this glorious tradition which never considered caste or creed as a criterion for accepting students. He was totally universal in outlook and with a mind open both in receiving and giving. This should be adequate to demolish some of the misconceptions and false narratives of today relating to casteism in music.

\vspace{-.3cm}

\section*{3. Conclusion}

Thus, it is evident that Śrī Muttusvāmi Dīkṣita was an enlightened master, a divine saint composer with mystic qualities, a versatile genius and a true \textit{nādopāsaka} – a realized soul who had the visualization of the Almighty through his music. His spontaneous outpourings of \textit{kṛti–}s which are timeless masterpieces are a result of \textit{tapas}.; each one of them brimming with devotion and lyrical excellence stand testimony to the fact that he was no ordinary composer.

In line with the way of life of \textit{Sanātana Dharma}, he ingeniously and seamlesslyintegrated \textit{bhakti} with disciplines and thoughts related to Vedānta in his compositions, which are indeed universal and eternal. His ‘vision’ and contributions to Karnatic music have made a deep impact on musicians as well as \textit{rasika}–s even today and there is no doubt that they will remain eternal. Notable too is the influence he wielded on later composers like Mysore Vāsudevācārya and Jayachamarajendra Wodeyar.

This paper, thus, has highlighted not only the greatness of the music of this divine composer, but has also shed light on the divine personality that he was. It is impossible to imagine Karnatic music without compositions of saint seer composers like him. We have to be ever grateful for this long rich legacy that we have inherited from the past which is the essential foundation for our present and future.

\vspace{-.3cm}

\begin{verse}
\textit{Śrīpuro vijayatām tyāgarāṭ vijayatām} \dev{।}\\\textit{dīkṣito vijayatām tatkṛtir vijayatām} \dev{।।}
\end{verse}

\vspace{-.3cm}

–Victorious be Tiruvarur; victorious be Lord Tyāgarāja; victorious be Dīkṣita; and victorious be his \textit{kṛti–}s. (Raghavan. V. Dr, ŚrīMuttusvāmiDīkṣitacaritam – \textit{granthasamarpaṇam, śloka} 11)

\vspace{-.3cm}

\section*{Bibliography}

\begin{thebibliography}{99}
\bibitem{chap1–key01} Aiyar, T.L. Venkatarama. (1968). \textit{Muthusvāmi Dikshitar}. National Book Trust, New Delhi.

 \bibitem{chap1–key02} Asha.R. (2013). \textit{Concepts, Contexts and Conflations in the Kṛti–s of Śrī Muttusvāmi–Dīkṣita}, Chennai: Author.

 \bibitem{chap1–key03} Ashutoshananda Svami. (2002). \textit{Muṇḍaka Upaniṣad}, (Tr.), Ramakrishna Math.

 \bibitem{chap1–key04} —. (2002). \textit{Taittirīya Upaniṣad}, (Tr.), Ramakrishna Math.

 \bibitem{chap1–key05} Ganapathi. Ra (1976). Deivaththin Kural, Vol 1. Chennai: Vanathi Padhippagam.

 \bibitem{chap1–key06} Govinda Rao, T.K. (1997). Compositions of Muttusvāmi–Dīkṣita, Chennai: Ganamandir Publications.

 \bibitem{chap1–key07} Janaki S.S. (2012). \textit{Saṁskṛita and Saṅgīta}, Chennai: The Kuppuswami Sastri Research Institute.

 \bibitem{chap1–key08} Kashyap, R. L. (Ed.) (2003). Rig Veda Mantra Samhitā. Bangalore: Sri Aurobindo Kapali Sastry Institute of Vedic Culture.

 \bibitem{chap1–key09} Pansikar, W. L. (Ed.) (1926). Yājñvalkyasmrti with Mitākṣara of Vijñaāneśvara. Bombay: Nirnaya Sagar Press.

 \bibitem{chap1–key11} Paramarthananda Svami. (2000). \textit{Muṇḍaka Upaniṣad} – (Tr.). Chennai: Yogamalika.

 \bibitem{chap1–key12} —. (2005). \textit{Introduction to Vedānta}. Chennai: Yogamalika.

 \bibitem{chap1–key13} —. (2005). \textit{Taittirīya Upaniṣad} – (Tr.). Chennai: Yogamalika.

 \bibitem{chap1–key14} Raghavan.V.(1975). \textit{Muttusvāmi Dikshitar.} Bombay: National Center for the Performing Arts.

 \bibitem{chap1–key15} —. (1980). \textit{Śri Muttusvāmi–Dīkṣita}–\textit{Caritam Mahakavyam}. Madras: Punarvasu.

 \bibitem{chap1–key16} —. (201). \textit{Collected Writings on Indian Music}. Chennai: Dr. Raghavan Centre For Performing Arts.

 \bibitem{chap1–key17} \textit{Ṛg Veda.} See Kashyap.

 \bibitem{chap1–key18} Saxena, Prashant. (2011). “Idol Worshipping, Western Terminologies and the Modern view”. \url{https://www.chakranews.com/idol–worshipping–western–terminologies–and–the–modern–view/1131}. Accessed on 30th Oct 2019.

 \bibitem{chap1–key19} Sridhar, Nithin. (2015). “Fallacies in the criticism of Idol worship”. \url{https://www.academia.edu/25108638/Fallacies_in_the_Criticism_of_Idol_Worship}. Accessed on 30th Oct 2019.

 \bibitem{chap1–key20} Srivatsa, V.V. (2000). \textit{Bhāva–Rāga–Tāla–Modini}. Chennai: Guruguhanjali.

 \bibitem{chap1–key21} Subbarāma–Dīkṣita. (1968). \textit{Saṅgīta Sampradāya Pradarśini}, (Ed) Rajam Ayyar.B., Ramanathan.S. Madras: The Music Academy.

 \bibitem{chap1–key21} \textit{Sūryatāpinyupaniṣad, Unpublished Upaniṣads,} (1933), Theosophical Society, Adyar

 \bibitem{chap1–key21} Svami Swaahaananda(2007).\textit{ Chāndogya Upaniṣad,} Chennai: Rama\-krishna Math.

 \bibitem{chap1–key21} \textit{Taittirīya Upaniṣad} – (with \textit{Sāṅkara Bhaṣyam}). (2012). Gorakhpur: Gītā Press.

 \bibitem{chap1–key21} \textit{Yājñavalkyasmṛti.} See Pansikar.

 \end{thebibliography}

\theendnotes

