
\chapter{A Critique of ‘A Southern Music: The Karnatik story’}\label{chapter6}

\Authorline{Ramanathan}

\begin{flushright}
\textit{vraman.iitk@gmail.com}
\end{flushright}


\section*{Abstract}

Whereas Karnatic music has been extolled to the pinnacle of \textit{nāda yoga} or \textit{nāda yajña} whereby the \textit{sādhaka} experiences the sublime and arrests time for the listeners, through this book and his non–musical egregious engagements, T. M. Krishna dons upon himself the garb of the sole liberator of this very music itself. Liberate from what? Liberate it from the elements of devotion in it, and simultaneously presenting a cocktail of subaltern theories, prejudiced handling of the music by the upper caste artistes and marginalization of certain elements of music, as presented in this book by Krishna. Such arguments are nothing new in the discourse of Karnatic music which witnessed fierce exchanges between the proponents of \textit{Tamil Isai} movement and the then group of Karnatic musicians, their patrons and the Music Academy in the middle of the last century. Creating a public narrative concerning music along the fault lines of caste is another dimension, yet, not at all strange in South India, its music as well as Karnatic music and once again the \textit{Tamil Isai} movement is a glaring testimony in this front as well. So, in this book by Krishna we get to see a rehash of all these arguments, exaggerated with contemporary atrocity literature along the fault lines of caste hierarchy and Krishna altogether taking this divide to a new dimension of calling for a Karnatic music bereft of the aspects of devotion. Some of the crucial points presented by Krishna in his book like the purpose of music and whether \textit{bhakti} is the \textit{summum bonum} of Karnatic music; the allegations on Brahmins for their domination and their usurping the professions of certain community of people who were traditionally associated with music; the role of lyrics in Karnatic music and corroborations from the \textit{Tamil Isai} movement will be critiqued in this paper in the light of the conspicuously missing discussion on certain documented history.

\begin{verse}
\textit{sa–ri–ga–ma–pa–dha–ni–ratāṁ tāṁ vīṇā–saṅkrānta–kānta–hastāntām~।}\\\textit{śāntāṁ mṛdula–svāntāṁ kuca–bharatāntāṁ namāmi śiva–kāntām}
\end{verse}

\begin{flushright}
Kālidāsa in Devi \textit{Navaratnamālā}
\end{flushright}

\begin{verse}
\textit{jagrāha pāṭhyam ṛgvedāt sāmabhyo gītam eva ca~।}\\\textit{yajurvedād abhinayān rasān ātharvaṇād api~।।}
\end{verse}

\begin{flushright}
\textit{Nāṭyaśāstra} 1.17
\end{flushright}

\begin{verse}
\textit{nāda–tanum aniśam śaṅkaraṁ namāmi me manasā śirasā}\\\textit{modakara–nigamottama–sāmaveda–sāram vāram vāram}\\\textit{sadyojātādi–pañcavaktraja–sarigamapadhani–vara–sapta–svara–}\\\textit{–vidyā–lolaṁ vidalita–kālaṁ vimala–hṛdaya–tyāgarāja–pālam}
\end{verse}

\begin{flushright}
Śrī Tyāgarāja (Rao 1999:260)
\end{flushright}


\section*{Introduction}

It is indeed befitting to commence this paper with the invocation by Sri Śaṅkarācārya who has righty visualized Mother Goddess as the very form of music. Furthermore, in the earliest available literature on music, namely the \textit{Nāṭyaśāstra}, Bharatamuni states that the music has its genesis in the \textit{Sāma veda}. Finally, when we come to a relatively recent period of Sri Tyāgarāja he upholds the longstanding concept on the divine nature of music as he, in the song cited above, sees the \textit{sapta svara} emanating from Lord Śivā’s face. And now we come to this book by T. M. Krishna (henceforth Krishna) which in some sense is antagonistic to all the sentiments espoused above. He appears to treat the idea of \textit{bhakti} in Karnatic music in a manner different from the convention and the following sentence from the book is merely a sample to buttress this point:

\begin{myquote}
“Abstract music generates bhava in the most profound sense, without having to refer to any direct associations to god. In thrusting this experience into the box of religiosity, we have only reduced the intensity of the experience that art music can offer.” 

~\hfill Krishna (2013:304)
\end{myquote}

It is more than obvious, through the choice of words and style of articulation; Krishna verily showcases his views on Karnatic music’s divine connection and thereby taking a stance that is diametrically opposite to the tradition as espoused through samples of lofty thoughts quoted above by eminent personalities at various instances in history.

The title of this book, “A Southern Music: The Karnatik Story” is partly deceiving because it is not an objective, impersonal discussion or collation on the Karnatic music, rather it is an exposition of Krishna’s moorings. He has nonchalantly injected his opinion while claiming to narrate the ‘story’ of Karnatic music. The book is internally arranged in three smaller books. Per design or by default, Krishna’s highly opinionated assertions on the socio–political underpinnings as well as repercussions of Karnatic music appears sandwiched between the general description of various aspects of music and its history. Whereas a few sections of the book are verbiage laden with sophistry, a few on the other hand are a rehash of older arguments that have been advanced and analyzed by academics from universities on foreign soil. Krishna deconstructs the entire edifice of Karnatic music through a clever series of winding arguments. He asserts that god has no business in Karnatic music, and that Brahmin musicians in the past as well as in the present are the main culprits for whatever vile exists in Karnatic music today. In the 550+ pages, he has put forth only these two assertions, implicitly in a few places and overtly in others. In this paper I will address the following questions both in light of this book as well as the generally accepted notion and thereby highlight not only the contrarian viewpoint of the book but also show that this contrarian viewpoint is not well grounded:

\item Is \textit{bhakti} the \textit{summum bonum} of Karnatic music?

 \item What is the role of lyrics in Karnatic music? (The question considered from the point of view of the impact that \textit{Tamil Isai} movement had on Karnatic music.)

 \item Were the Brahmins in Karnatic music in past villains or emancipators? (I take up historically documented events regarding which Krishna maintains silence.)

Let us look at each of these questions in some details now.


\section*{Is Bhakti the Summum Bonum of Karnatic Music?}

One of the strong ideas that Krishna has woven through the entire fabric of the book is that God is not and should not be the focus of Karnatic music, that the music transcends the religious realm and dwells in a much superior realm of pure art. Let us begin by considering the way Krishna perceives and introduces ‘god’, ‘divine’ and ‘religion’ in this discourse. In page 295, he writes on ‘god’,

\begin{myquote}
“A word that could evoke trust in its most intense form is ‘god’. The origin of this word has been widely debated, but one of the hypotheses is that it comes from the German ‘guth’, and that is about as much as we know. Most dictionaries describe it to mean a superhuman force, an object of worship, the Supreme Being in monotheist faiths, the Creator or even just an idol.” 

~\hfill Krishna (2013:295)
\end{myquote}

Then on ‘divine’, he has the following:

\begin{myquote}
“Connected to this idea is a closely related one – divine. This is derived from the Latin ‘divinus’, which is traceable to the ‘divus’, both meaning godlike.” 

~\hfill Krishna (2013:295)
\end{myquote}

Finally, on ‘religion’ Krishna writes:

\begin{myquote}
“Derived from the Latin word ‘religio’, meaning binding or obligation, ‘religion’ is closely linked with the Latin ‘lagire’ (the root of ligature), which again means something that binds. In these descriptions of bond and binding, there is an indication of a fastening, a joining, but no suggestion that this is forced nor of any tension in this bond. This nuance is important.” 

~\hfill Krishna (2013:295)
\end{myquote}

It is indeed obvious that this ‘god’, ‘divine’, and ‘religion’ has nothing to do with the Indian concepts and everything to do with the Western/Abrahamic ‘god’, ‘divine’, and ‘religion.’ But very quickly Krishna makes a sweeping statement in the very next page when he writes,

\begin{myquote}
“There is no gainsaying that even within the Indian ethos words like god, divine and religion have very varied meanings and spiritual or philosophical connotations. But in their most universal application within the Indian context, the core emotion and associations around ‘god’, ‘divine’ and ‘religion’ \textit{are not at any fundamental variance from those of other cultures}.” 

~\hfill Krishna (2013:296) (\textit{italics ours})
\end{myquote}

This blatant false equivalence that equates Hinduism to the Abrahamic faiths is illogical and naïve. Over–simplistic treatment of these concepts within Hinduism is not only frivolous but also unwarranted. It does not augur well to impose the Abrahamic religious framework on Indian ethos in order to argue later that such ‘binding’ or ‘tense bond’ is forced upon the Karnatic music and hence one must relieve the music from such a religious clutch in the interest of music. It is under such an erroneous premise and by the use of the lens of Abrahamic religious does Krishna critique the Indian concept of \textit{bhakti} and \textit{śaraṇāgati }as espoused through various compositions in Karnatic music. It is left to the reader to discern the veracity of his arguments thence. Notwithstanding this shaky premise it is noteworthy to highlight his obsession with agnostic (bordering atheistic) approach to Karnatic music because he writes:

\begin{myquote}
“The power of religion is so strong that once we place our experience under its umbrella, everything else becomes submissive to it. If one is able to move into a space without a religious identity, it is quite possible that all the other elements that constitute the experience come to life in a way that never existed before.” 

~\hfill Krishna (2013:296)
\end{myquote}

It is indeed highly befitting to contrast these utterances with the lofty and lucid exposition of the saint composer Sri Tyāgarāja. In his Dhanyāsi\textit{ kṛti}, “\textit{saṅgīta jñānamu bhakti vinā sanmārgamu galade manasā}” the saint composer categorically rebukes the very idea of musical knowledge without \textit{bhakti} towards god. If this is the identity and status of the Karnatic music, why should one obsess to shun the ‘religious burden’ from Karnatic music, as asserted by Krishna?

Well, he is indeed completely entitled to his view (atheistic or agnostic be that as it may) but when he psychoanalyses the saint composers through the lens of reductionism and Abrahamic religion, he oversteps from merely exercising his freedom of expression to belittling the contribution of these saint composers besides attempting to create a negative impression regarding them. For instance, he writes:

\begin{myquote}
“If his [\textit{Tyāgarāja}] main objective was to convey devotional meaning through the lyrics, he would have probably used repetitive melodic themes for all the compositions in the raga to make the composition more accessible to people”
\end{myquote}

\begin{myquote}
“All this indicates a detail that can only be oriented towards an art music experience. The interplay he [\textit{Muttuswami Dikshitar}] created using syllabo–melodic identities shows that this was a composer who was doing more than describing the divine.”
\end{myquote}

\begin{myquote}
“This presents us a musician’s clear musicological and historical perspective. At the same time, we know that Shyama Shastri created a new raga, chintamani. These are other indicators tell us that these compositions were far more than bhakti creations. They were serious art music pieces.” 

~\hfill Krishna (2013:282–283) (\textit{italics ours})
\end{myquote}

He goes on to say,

\begin{myquote}
“Any vaggeyakara would necessarily provide sahitya that expressed his impressions, aspirations, disappointments and inclinations. This religiously charged environment of the period and their own households naturally led the trinity to express their religious philosophical and moral opinions. This does not mean that they intended to make any massive religious impact on society through these compositions or initiate new religious trends. If that were the case, they would have created simple compositions in familiar ragas, melodies and talas that everyone could repeat. They were geniuses who created musical masterpieces, which contained their beliefs and used sahitya as a musical idea. The meaning of the sahitya was not the focus of the compositions. It was merely the natural input from the experience of each unique human being. These observations on the trinity have been made after an inter–textual analysis of the older pathantaras (rendering variations) of their compositions.” 

~\hfill Krishna (2013:284) (\textit{no diacritics in the original})
\end{myquote}

All these excerpts are more than enough to give us a glimpse of what Krishna is aiming at. In his world view of Karnatic music sans any \textit{bhakti}, he has imagined the supremacy of the abstract art and the \textit{bhakti} is indeed an (avoidable) accident that has only crippled both the musicians and the listeners from accessing that supreme realm where art dwells. Krishna alleges that the real intent of the Trinity’s music stands misinterpreted due to its religious overtones. Thereby, Krishna completely trivializes both the \textit{bhakti} of the Trinity and the \textit{bhakti} elements that they have suffused in their compositions.

In other words, any composition by the Trinity can be looked at as a highly convoluted signal. Upon deconvolution we get a number of principal components (\textit{rāga} phrases, lyrics, rhythmic structure, intent, emotiveness etc.) and the weight of these individual components will vary. For example, if a teacher is teaching a composition to students, components like the correct lyrics, \textit{rāga} phrases etc. will be stressed upon whereas when one is performing on the stage, the emotiveness and other factors may be weighed more (it is not that the other components like the correct lyrics, \textit{rāga} phrases etc. are any less important in a performance on stage. A performer would have internalized all these factors, to the extent of making it automatic). Now, all that Krishna is trying to say is that upon deconvolution, the art component of a composition has a weight that is heads and shoulders above the \textit{bhakti} factor or that the \textit{bhakti} factor is insignificant.

This clearly stems from intellectual arrogance and poor understanding of the composers, musicians and society at large. This is because none of these composers sang in isolation and the fact that their compositions and creations were received extremely well, right from their time and even now, shows a common societal perception and acceptance of these composers’ intent. It is not as though the society, the music community or these composers felt or expressed any ‘corruption’ in music by alleging to \textit{bhakti} to their gods.

Another subtle nuance is the way Krishna creates a binary between \textit{bhakti} towards god and art (page 283). This is his creative genius to pit one against the other and when he claims art to be prominent, obviously \textit{bhakti} towards god gets trivialized and hence gets amenable to be reduced to redundancy or to be considered as an exception – both of which lead to the notion that \textit{bhakti} towards god is an unnecessary burden in the realm of Karnatic music, eventually. He invokes the \textit{gāndharva} and \textit{gāna} classification all the way from \textit{Nāṭyaśāstra} to surreptitiously legitimize his claim. This, however, is erroneous. \textit{Ācārya} Abhinavagupta in his \textit{Abhinavabhārati}, an elaborate commentary on \textit{Nāṭyaśāstra}, categorically states the following

\begin{myquote}
“\textit{kiṁ ca antarāla–niyamaḥ antaḥ pramāṇa–sthāna–svara–kālāṁśa–vadhānna(śāda)sāratayā gāndharve avaśya–saṁvedyaḥ~। na tvevaṁ gāne”}
\end{myquote}

\begin{myquote}
“\textit{lopo’pi niyata–gāndharve darśito grāma–dvaya–bhedena ca jātyaṁśa–bhedena darśitaḥ~। gāne tu raktyanusareṇa pravente(vṛtte) rasāvaniyataḥ}” 

~\hfill Kavi and Pade (1964:394)
\end{myquote}

\begin{myquote}
Meaning: The rule regarding \textit{śruti}–intervals has to be strictly observed in \textit{gāndharva} music, not so in \textit{gāna}.
\end{myquote}

\begin{myquote}
Dropping of notes in the two \textit{grāma}–s and on the basis of \textit{aṁśa} notes in each \textit{jāti} was governed by definite rules in \textit{gāndharva} ……but in \textit{gāna}, any note could be dropped in order to bring about a particular aesthetic effect. 

~\hfill Singh (1984:18)
\end{myquote}

We see that \textit{Nāṭyaśāstra} classifies music into \textit{gāndharva} and \textit{gāna.} Whereas in the \textit{gāndharva} the rules pertaining to \textit{śruti, tāla} and \textit{svara–sthāna} are strictly adhered to, no requirement of such strict adherence exists in the case of the \textit{gāna} music. Given the dynamic nature of our music system, both these forms constantly evolved to newer styles. But, according to Krishna it is only the \textit{gāndharva} that constituted the “art music.” Where \textit{Nāṭyaśāstra} is silent about art music and non–art music, delineating only on the applicability of rules, Krishna juxtaposes the modern concept of art music on to \textit{Nāṭyaśāstra} thereby inventing a filter to sieve through the music right from the days of \textit{Nāṭyaśāstra} all the way until now. This invented filter serves handy to Krishna with which he dissects the compositions of the Trinity and concludes that his assertions are correct. This is a fitting illustration for circular reasoning by Krishna.

Krishna is indeed very careful in his choice of words as well as style, while trivializing the \textit{bhakti} aspects of the musical Trinity. On the one hand, Krishna squarely blames them for indoctrinating people through their (unwanted) allegiance to \textit{bhakti} and on the other, he creates another class of divinity where \textit{bhakti} exists albeit not towards any god, but towards Art. With the creation of this new class of divinity and \textit{bhakti}, Krishna praises the Trinity by saying:

\begin{myquote}
“I would say this: irrespective of the votive divinity being sung to, each of the three musical geniuses have in common another ‘divinity’ – one that is the essence of art music, which for want of a more effective phrase I will describe as the fulfillment of a quest for the purest musical abstraction.” 

~\hfill Krishna (2013:283)
\end{myquote}

The last sentence about “these observations … have been made” (Krishna 2013:284) in the excerpt cited above helps in highlighting the error in Krishna’s deconstruction of the Trinity’s compositions into art and \textit{bhakti} (to god). First, he appears to give logic and legitimacy to his conclusion by saying that he has arrived at this conclusion only through research. And what is the nature of this research? He states that he has researched the variations in rendering of the composition by the Trinity. Interestingly, this completely falsifies the premise on art superseding \textit{bhakti} (to god) in the Trinity’s compositions because of the wrong design of experiment. Merely taking cognizance of variations in rendering neither indicates nor proves what Krishna has concluded and wants his readers to believe. Second, through this sentence, he is covertly highlighting the ‘argument from authority’ fallacy that exists in the music community today. For instance, when he says that he has arrived at his contrarian conclusion after ‘research’, he is indirectly urging the readers to infer that the extant approach to these compositions from \textit{bhakti} (to god) point of view is not a product of research and thereby stemming from authority which indeed is a fallacy that is hence rightly quashed by his conclusion!

Right after this creative and skillful deception, Krishna unequivocally announces his intent in page 285–286 when he writes,

\begin{myquote}
“But having placed them [utsava sampradaya kṛti–s] beside art music compositions, we begin to look for ‘meaning’ in the bhakti sense in the latter as well. This blurs any critical differentiation between the two. I believe that the \textit{utsava sampradaya compositions do not have a place within Karnatik art music}………I strongly believe that the art music compositions of the trinity, or indeed any other composer, must be treated as art music \textit{and nothing else}. At the same time, compositions that do not possess the quality to abstract as an art music piece \textit{should not be used within Karnatik music}.” 

~\hfill Krishna (2013:285–286) (\textit{italics ours})
\end{myquote}

Let us now look at how people other than Krishna looked at art and non–art music. It is indeed well documented that from the days of \textit{Nāṭyaśāstra} until now, musicians and musicologists have maintained a clear distinction between art music and non–art music. For instance, let us look at the definition of \textit{gāndharva} music as given in \textit{Nāṭyaśāstra} itself:

\begin{verse}
\textit{atyartham iṣṭaṁ devānāṁ tathā prītikaraṁ punaḥ~।}\\\textit{gandharvāṇāṁ ca yasmāddhi tasmād gāndharvam ucyate~।।}
\end{verse}

\begin{flushright}
(\textit{Nāṭyaśāstra} 28.9)
\end{flushright}

Meaning: Because it is extremely desired by the gods and gives great pleasure to the \textit{gāndharva}–s, therefore, is it called \textit{gāndharva}.

\begin{flushright}
Singh (1984:20)
\end{flushright}

Glossing on this, Abhinavagupta says the following in his \textit{Abhinavabhārati},

\begin{myquote}
“…\textit{prītivardhanam iti evam anāditvāt dṛṣṭādṛṣṭaphalatvāt ca pradhānaṁ gāṅdharvaṁ … gānaṁ hi kevalaṁ prītikārye vartate~। tena tādātmyaṁ tāvadayuktam~।} 

~\hfill Kavi and Pade (1964:6)
\end{myquote}

\begin{myquote}
Meaning: "Gandharva has been used from time immemorial. It has both drishta and adrishta phala i.e. it is both pleasant (which is evident) and lays in store merit for the future according to which one earns liberation or is given a place in heaven which is not evident (adrishta). But Gana is used only for its pleasant or aesthetic effect. It is, therefore, unjust to identify the two." 

~\hfill Singh (1984:20) (\textit{spellings as in the original})
\end{myquote}

As said earlier, the two forms of \textit{Nāṭyaśāstra} namely the \textit{gāndharva} and the \textit{gāna} have evolved to different forms over a period of time. Clear distinctions between them have been passed on even in their evolved forms.

In the more recent past Sri Ranga Ramanuja Iyengar, one of the \textit{vīṇā vidvān}–s, a disciple of \textit{vīṇā} Dhanammāl and a great musicologist, observed the following:

\begin{myquote}
“Carnatic music was essentially spiritual. It was woven into the texture of religion and yoga sadhana. Its great exponents, composers and grammarians were men and women of saintly character, mystics, siddhas and bhaktas. That is why we find an undercurrent of philosophy and moral teaching even in ordinary simple folk songs like those of the snake charmer, the boat man, and the cart driver. What is the import of all this? 

~\hfill Iyengar (1958:12)(\textit{spellings as in the original})
\end{myquote}

\begin{myquote}
“Many must have noted with a bleeding heart how popular taste is being corrupted especially by things like the incoherent orchestral preludes to the song–hits of stars and of pitch, range, volume, melody and flexibility of voice. Came the Brindaganam and the variety entertainment, and vandalism reached its limits.” 

~\hfill Iyengar (1958:36–38) (\textit{spellings as in the original})
\end{myquote}

Let us look at how the westerners have handled this theme when they encountered and described Karnatic music. In the book titled \textit{Music and Musical Thoughts in early India} by Lewis Barrel, the author tries to answer the question, “what was music like in early India?”. Interestingly while responding to this question the author arrives at ‘music’ only after two chapters in some 50 pages. In these two chapters the author introduces Indian philosophy viz. the six philosophies and only in the third chapter delves into the aspects of music (See Rowell 1992; Tingey 1994).

This form of presentation clearly indicates that the author has made an attempt to familiarize the readers with adequate pre–requisites so as to contextualize the divine nature of music in early India. Interestingly, while Krishna’s target audience for his book appears to be largely ill–informed urbanites or westerners (as he uses words like “death day” for “samadhi”; subscribing to Western etymology of Sanskrit words) he absolutely refrains from laying out the broader philosophical ecosystem in which the Indian music actually had its genesis and subsequent evolution.

Explicit call for separation of \textit{bhakti} to god from the art aspects of Karnatic music has never appeared in the works of ethnomusicologists or historians of music even from the West barring a few, one of them being Baltzell, who has the following to say in his book \textit{A Complete History Of Music} in the beginning of the 20th century,

\begin{myquote}
“The main reason why Hindoo music did not develop in the past centuries doubtless lies in the fact that, as in Egypt, the ruling power was vested in the priesthood, which controlled all the arts and sciences. Music was so interwoven with their religious rites and observances, and so hedged around with irrevocable and sacred laws that the slightest alteration was considered a sacrilege.” 

~\hfill Baltzell (1908:33)
\end{myquote}

It must also be mentioned that this author subscribes to the then existing theory of Aryan invasion of India as he writes, “The Hindoos belong to the Aryan race, (from which we also sprang), and had their home originally in Central Asia, probably north of the Hindoo Koosh range” (Baltzell 1908:31) thereby rendering his diagnosis on the ‘lack of development’ in the Indian music in the past, infructuous. Interestingly, it appears as though Krishna is informed and influenced by this book as he has taken cues from it (apparently).

In conclusion of this section, it would indeed be apt to draw parallels between Krishna and Prof. Sheldon Pollock. One of the ill–grounded observations of Prof. Pollock was on the non–commonality of \textit{rasa} across different art forms (Gopinath 2018:90); interestingly somewhat akin to Krishna’s denial of the commonality of \textit{bhakti} to god as the \textit{summum bonum} of Karnatic music. What is \textit{rasa} to Pollock, \textit{bhakti} towards god is to Krishna. If Pollock got it all wrong with respect to \textit{rasa}, as shown by Gopinath through \textit{Citrasūtra, }who elaborates on the relationship between and amongst the different disciplines like sculpture, painting, dance and music (Gopinath 2018:105), the aforementioned discussion suffices to prove Krishna wrong with regard to \textit{bhakti} towards god. Finally, it is worth reiterating that Krishna is entitled to his opinion on the lines of atheism or agnosticism but it is indeed completely wrong to project his personal indulgence with god, or the lack thereof, on the wider canvas of Karnatic music. Interestingly, Krishna does not refer to or cite Abhinavagupta’s work anywhere in this book. So yes, \textit{bhakti} is the \textit{summum bonum} of Karnatic music. In the parlance of Sri Rajiv Malhotra, \textit{bhakti} is the poison pill of Karnatic music and anyone who wants Karnatic music has to take it with the poison pill. Krishna is attempting to dilute or destroy this poison pill (Malhotra 2014:269).


\section*{The Role of Lyrics in Karnatic Music }

While dismissing the \textit{bhakti} to god as the primary motif of Karnatic music, as evidenced from the discussion above, the excerpts also throw light on Krishna’s ideas on the role, impact and necessity of lyrics in the domain of what he classifies as the art form of the Karnatic music. In this section we will look at this in greater detail to expose the inconsistencies in Krishna’s arguments. The \textit{Tamil Isai} movement will also be discussed in this section and here again we will see how Krishna has misrepresented historical facts through information bias.

Krishna writes,

\begin{myquote}
“……and the musician expresses every melodic motif as an extension of an appendage to the lyrical meaning and not as a distinct melodic import. And that is when the music, regretfully, ceases to be art music. The shift in focus from melodic to lyrical meaning deeply affects the compositional experience for the listener. In my world of meaning, ‘Kamakshi’ is a melodic and not a lyrical concept.” 

~\hfill Krishna (2013:278)
\end{myquote}

After having made his world–view clear that the melodic aspect enjoys the top priority and that lyrical aspect has no place, Krishna goes further to add,

\begin{myquote}
“I am not devaluing the experience that sahitya–oriented music can generate, but I am definitely questioning the necessity for such experience within Karnatik music.” 

~\hfill Krishna (2013:279)
\end{myquote}

Relationship between the tonal and linguistic aspects has been a subject of serious studies world across. In one of the recent publications in the Journal of Acoustic Society of America (Wang \textit{et al} 2016:2432), while reviewing the earlier literature, Wang \textit{et al} state that cross domain pitch processing for speech and music pointing to bidirectional influences of pitch–related proficiency between speech and music as a function of experience. Furthermore, it is stated in the paper that such bidirectional influences appear to be associated with enhancement of both lower level acoustic sensitivity and higher–level cognitive resources in speech and music. As a conclusion of their own findings, Wang \textit{et al} conclude that

\begin{myquote}
“The bi–directional transfer of tonal categorization supports our theoretical prediction that domain–specific experience with categories may enhance domain–general category learning mechanisms which in turn facilitate cross–domain categorization. These patterns suggest that experience which leads to increased efficiency of sound category learning could benefit the categorization in both music and speech under domain–general processes.” 

~\hfill Wang \textit{et al} (2016:2445)
\end{myquote}

In other words, both the lyrical and the melodic (tonal) aspects of sound have a mutual influence on each other, together enhancing certain desirable attributes in the individual. Although there exists a paucity of such extensive research on the relation between the lyrical and tonal aspects of sound in the Indian musical context, discussions regarding the relationship between word (\textit{vāk}) and meaning (\textit{artha}) has been in vogue from distant past and the verdict has also been given in the past by the master poet Kālidāsa in his verse:

\begin{verse}
\textit{vāgarthāviva saṁpṛktau vāgartha–pratipattaye~। }\\\textit{jagataḥ pitarau vande pārvatī–parameśvarau~।।}
\end{verse}

\begin{flushright}
(\textit{Raghuvaṁśa} 1.1)
\end{flushright}

When we revisit the idea of Krishna on the insignificance of lyrical aspect in Karnatic (art) music with the backdrop of the above cited research findings, it is obvious that Krishna merely asserts his opinion without any supporting evidence.

It is indeed a well–known fact that the compositions of Muttusvāmi Dīkṣita, particularly the \textit{Kamalāmbā Navāvaraṇa kṛti–}s, are pregnant with \textit{bīja mantra}–s. If the lyrical aspect is really to be downplayed in (art) music as Krishna asserts it is indeed a matter of great mystery as to why a composer like Dīkṣita should take such great pains to compose the lyrics for his \textit{kṛti}–s. At this juncture we can conclude that Krishna’s assertion is not well–grounded although he is entitled to hold on it his opinions.

Further on the lyrical aspect, let us now look at some contradictions in Krishna’s arguments. Whereas the lyrical aspect of, say \textit{Kāmākṣī}, is of no significance to Krishna, interestingly when it comes to the \textit{pada–}s or \textit{jāvaḷi–}s, the lyrical aspect finds utmost prominence all of a sudden. Had Krishna exercised similar preference in these compositional forms (\textit{pada–}s and \textit{jāvaḷi–}s) as he did for the \textit{kṛti–}s, the approach could have been called objective, even though he classifies \textit{pada}–s and \textit{jāvaḷi}–s as art objects. He mocks both the listeners as well as the musicians by stating that they show selective obsession with the lyrical aspects of the composition. Whereas they completely ignore it in \textit{pada–}s and \textit{jāvaḷi–}s, they go to extreme details in other compositions, notes Krishna. He further questions the near paucity in serious discussions and lecture demonstrations on the \textit{pada–}s and \textit{jāvaḷi–}s. Having raised the question, he also offers the answer by stating that the overtly sexual content in \textit{pada–}s and \textit{jāvaḷi–}s makes us uncomfortable (Krishna 2013:287).

While discussing about \textit{pada–}s and \textit{jāvaḷi–}s, Krishna puts them in the same group here but elsewhere he rightly combines \textit{pada–}s with \textit{kīrtana} and separates out \textit{jāvaḷi–}s. For several elements, \textit{pada–}s for instance, Krishna mentions the dates in order to contextualize that feature in a historical perspective. Interestingly, he chooses to remain silent on the date or the historical evolution of \textit{jāvaḷi–}s. It can be gleaned that \textit{jāvaḷi–}s made its appearance in the then music of south India from the latter part of the 18th century (Seetha 1968:341). Both the date as well as the context surrounding the appearance of \textit{jāvaḷi–}s is an interesting topic needing a distinct treatment which will be taken up in later section while discussing the socio–political ecosystem in late 19th century.

\textit{Pada–}s are laden with \textit{śṛṅgāra rasa} which may at the best be translated in English as divine romance. But literature is replete, not excluding Krishna, with the term ‘erotic’ to refer to the \textit{śṛṅgāra rasa} in \textit{pada–}s. ‘Erotic’ – much like the term ‘gay’ of Wordsworth in his \textit{Daffodils} (“A poet could not be but gay, In such a jocund company”) – has a different connotation. Erotic dwells only in the realm of pure art. But the intent of the \textit{śṛṅgāra}–laden \textit{pada–}s is also devotion where the ‘\textit{nāyaka–nāyikā’} relation is extolled to its utmost glory, with God being the \textit{nāyaka} always. A major shift happens in the \textit{jāvaḷi–}s where mortals replace god as the \textit{nāyaka}.

Now we need to go back to our first question and reflect on the established fact that \textit{bhakti} to god is the heart of Karnatic music through different points in time. With the advent of \textit{jāvaḷi–}s, does the aspect of \textit{bhakti} to god get shaky? No. This can easily be established by statistical argument and analogy from Darwinian evolution. \textit{Jāvaḷi}–s are few, though they are in the group of art music. Hence, they cannot be seen to have been 'selected' (in the Darwinian sense). There are other aspects of \textit{jāvaḷi–}s which will be taken up later.


\section*{Genesis and Impact of Tamil Isai Movement}

It is verily in this lyrical context that the \textit{Tamil Isai} gets discussed by Krishna. It is beyond the scope of this paper to dwell into all the details of this movement. Very briefly, this movement started in 1940s that mandated the use \textit{Tamil} compositions in a performance. But the very genesis of this movement calls forth for a relook at the discussion on the lyrical aspects of music. It had its origin in the early 1940s taking formal shape through the establishment of \textit{Tamil IsaiSangam} in 1943. Raja Annamalai Chettiar was highly instrumental right from its inception and early growth. Needless to say, there were fierce verbal exchanges between the proponents of this movement and its detractors drawn from the members and office bearers of Madras Music Academy. Despite several big–wigs of the day throwing their weights on either side, with the collective maturity and conscience of the society back then, amicability was soon reached and \textit{Tamil} songs started gaining popularity. It is indeed not true to say that there were no Tamil compositions prior to the formation of \textit{Tamil IsaiSangam}. It was indeed the immense popularity and appeal for the Trinity’s music that has kept their music continuously in vogue but after the \textit{Tamil IsaiSangam’s} formation, newer Tamil compositions gained currency. Papanasam Sivan’s popularity went beyond cinema only after the formation of this \textit{Sangam}.

Coming to Krishna’s analysis of \textit{Tamil Isai}, he expresses his surprise that people who were not even remotely connected with music were the torch bearers in this movement. In summary, Krishna welcomes the \textit{Tamil Isai} movement as the need of the hour then, due to its political underpinnings rather than from the language point of view. Although \textit{Sangam}’s contention began with the language, it soon got gave way into politics and landed along the caste fault lines of Tamil Nadu. Even though the \textit{Tamil IsaiSangam} originated with troubles, it soon stabilized as most of the leading artistes of the day took an active part in spreading the Tamil music. Many performers made sure that they devoted considerable time of their performance for Tamil songs. Over a period of time, however, people who hated Brahmins saw that the \textit{Tamil IsaiSangam} was taken over by Brahmins. They were frustrated to see that the \textit{Tamil IsaiSangam} was patronizing Brahmins when, according to them, it should have exclusively patronized non–brahmins. This gave rise to splinter groups formation and one such is the \textit{Tandai PeriyārTamil IsaiMandram} (TPTIM), started by Mr. N. Arunachalam in 1993, an industrialist from Chennai. It is a documented fact that Mr. Arunachalam has strong links with LTTE and considers Prabhakaran as his ideal (Terada 2008a:216). Interestingly, the mandate of this TPTIM was to patronize only non–Brahmin musicians and promote only those Tamil songs that align with the atheistic ideology. So, god has no business in this organization. It is very clear that this group’s main intent was to spread Brahmin hatred and atheism through music. This is where we need to check honesty and intent of Krishna. Whereas in the earlier parts of the book, he extensively psychoanalysed the Trinity and trivialized the \textit{bhakti}–to–god aspects of it, Krishna remains completely silent on splinter groups like TPTIM. If his approach was objective and the intent was truly to narrate the story of Karnatic music, he should have mentioned TPTIM, analysed its mandate, analysed its members and their thought processes and finally should have preached to them as he had done to the religious group. Alas, he is silent. Why?

One might surmise that he perhaps is not aware of this splinter group. However, this is not possible as he has cited the works of Terada. Incidentally it is this author and this author alone who has written about TPTIM. It is hence clear that Krishna has indulged in suppressing this information.

In summary, Krishna’s has made very clear his stance the on lyrical (in)significance of our traditional musical compositions. By underplaying the lyrics, he has subtly legitimized the act of replacing the words of these compositions with any individual’s choice and thereby legitimizing the Christianization of our music. Once he convinces the reader that the lyrics is not that important and that it is only the music and its art appeal that is important, it naturally (and importantly) follows that it does not and should not matter whether the composition is on Jesus or Allah or the song is penned by Perumal Murugan or by a Tamil Sufi (Parakala 2017\textbf{; }Rao 2017). Thus, he is acting as an agent for the breaking–India–forces and paving way for their onslaught. It is indeed noteworthy to mention that there have been a few composers like Vedanāyagam Pillai and Abraham Pandithar in the past who have composed many Christian \textit{kṛti–}s which were all in praise of Jesus. There have been few instances in past when top Karnatic musicians have even included a couple of these songs in their performance. Such instances can indeed only be construed as sporadic, as the practice has neither gained wide popularity nor has the community witnessed enormous growth in such compositions until now. It is noteworthy to highlight Krishna’s admiration for non–Hindu musicians and thereby glean his intent as well as double standard:

\begin{myquote}
“You will find their homes adorned with pictures of Hindu deities and their immense respect for Hindu gods and goddesses even when their religious practices are Islamic. This is a credit to their ability to straddle two worlds. But they cannot display apathy for Hinduism and be accepted as musicians by the Karnatic world” 

~\hfill Krishna (2013:312)
\end{myquote}

First, the admiration for these musicians straddling the two worlds is highly misplaced. Krishna has hinted to the Hindus amongst his peers that since they do not straddle the two worlds, they gain no admiration from him. Furthermore, just as these Islamic musicians do not show apathy to Hinduism in order to be accepted in the Karnatic fold, his Hindu peers ought not to show any apathy towards other religions lest they be forced to relinquish their coveted position in the Karnatic world. In all this, silence of Krishna on how these musicians came to Islamic fold is glaringly conspicuous. Let us look at the way Krishna has handled ‘Islamic musicians’ and ‘Brahmin musicians.’ Music is common in both the groups. But ‘Islam’ in the ‘Islamic musicians’ gets admiration and ‘Brahminism’ in the ‘Brahmin musicians’ gets vitriolic rebuke from Krishna (see below). Hence it gets vivid beyond any doubt that all his deconstruction tactics are to legitimize, aid, and normalize the invasion of the Karnatic system by breaking–India–forces!


\section*{Brahmins in Karnatic Music: Villains or Emancipators?}

Let us now come to the most crucial part of Krishna’s narration which is carrying out the hit–job on Brahmins – a theme and agenda that underpins his entire book as well as discourses elsewhere. In the guise of discussing Karnatic music from a sociological view point, Krishna has in fact misrepresented facts and merely rehashed the modern and post–modern ethnomusicologist’s subaltern theories (Weidman 2003:194; Weidman 2005:485; Terada 2008b:108; Sinha 1996:477). The starting point for all these researchers, as it is for Krishna, is that Karnatic music was and is dominated by Brahmins. With copious atrocity literature generated in the last 120–150 years, Brahmins are the usual suspects in any and every evil in the society (Google search with “Brahmins dominated Carnatic music” as key words returns 50k+ results!). In the context of Karnatic music, it is stated by these researchers (and is rehashed by Krishna) in the following words:

\begin{myquote}
“A mediocre Brahmin Karnatikmusician could find his voice in the modern narrative but an average nagasvara vidvan had no place.” 

~\hfill Krishna (2013:340)
\end{myquote}

\begin{myquote}
“When many non–brahmin musicians nurtured brahmin students, there were very few non–brahmins in the next generation of musicians, a further monopolization of the music by the Brahmins.” 

~\hfill Krishna (2013:344)
\end{myquote}

\begin{myquote}
“With Brahmins controlling the violin and the mrdanga, their domination of Karnatik music was complete.” 

~\hfill Krishna (2013:346)
\end{myquote}

\begin{myquote}
“Brahmin driven sanctification has played a largely negative role in the past.” 

~\hfill Krishna (2013:357)
\end{myquote}

With these and many more, the view of Krishna (a rehash of the writings of other researchers) on Brahmins is quite clear. One is reminded of the story of an experiment by a mad scientist who cuts the legs of a frog in succession, commands it to jump at the end of every cut and observes that the frog jumps. After cutting all the four legs, the scientist again commands the frog to jump but the frog remains still. The scientist then concludes from this experiment that when all the four legs of the frog are cut, the frog loses the ability to hear. Similar to these are the conclusions by those researchers Krishna has quoted on the role of Brahmins in Karnatic music. In other words, the observations are all perfectly correct but the inference drawn is flawed. Whereas such a faulty conclusion in the case of a fictitious scientist is hilarious, the conclusions drawn by Krishna are disgraceful.

Although Krishna has delved into the history of a few aspects in Karnatic music in the book, he remains completely silent on the role of Anglo–Indian law extant in the 19th and early 20th century, the impact of British ‘Social Purity Movement’ in India, and the role of Christian missionaries in the Tamil society. If one intends to give a socio–political context to Karnatic music, one cannot ignore the dynamics at different layers of the society back then. For a comprehensive history of the ‘\textit{pāṇar–s}’, ‘\textit{devadāsī}’ and ‘\textit{rājādasī}’ the readers are requested to consult literature on these (Varagunapandian 1952:9; Naidu 1986:89) and only a brief sketch is presented here. \textit{Pāṇar–s} were a community of people associated with music and the playing of \textit{yāḻ} right from the distant past. Hagiographic accounts of these \textit{pāṇar–s} are contrarian in nature. They have been described as the untouchables in some literature but Sekkizhār, the chronicler of the 63 Nāyanmār (Śaiva devotees of yore) offers a completely different picture. The episode of Nīlakaṇṭha Yāḻpāṇar’s (7th century CE) experience at the Madurai temple where he was offered a golden pedestal by the divine command to sing for the lord is documented history (Ghose 1996:412; Sivananda 1997:61). Another instance of social reformation and emancipation is also presented by Sekkiḻār while he describes Nīlakaṇṭha Yāḻpāṇar\textit{’s} visit to another Brahmin saint by the name Tirunīlanakka. It is reported (Sivananda 1997:26) that the Brahmin saint was caste conscious and it was only upon the insistence of Sambandar who had accompanied Yāḻpāṇar, did the Brahmin saint agree to accommodate Yāḻpāṇar and offered him a place to sleep next to the \textit{homa kuṇḍa} (fire altar). Sekkiḻār records that the \textit{homa} fire grew huge shone brightly, an omen to testify Yāḻpāṇar’s devotion.

It calls for deep research to reveal whether the ‘\textit{melakkārar}’ comprising ‘\textit{devadāsī}, \textit{rājādasī}, \textit{naṭṭuvanār} and \textit{nāgasvara} players’ were the \textit{pāṇar}–s of the yesteryears or not. For, in the discourse of Brahmins usurping the music from other communities, it is only these ‘\textit{melakkārar}’ who are often discussed and we find absolutely no mention of \textit{pāṇar}–s in this discourse. By the way, this alleged usurpation supposedly starts in the middle of 19th century, peaks in the 1880–1900s, culminating in the early 20th century, by when the Brahmins were branded as having the monopoly over Karnatic music. Given that we still have few contemporary singers suffixing their name with \textit{pāṇar} names is testimony to the fact that \textit{pāṇar}–s did exist during the period of alleged usurping of control over music by Brahmins. It is indeed mysterious as to why these \textit{pāṇar}–s do not find their presence in the history of Karnatic music, given the presence of copious atrocity literature created for this period.

Let us look at the ‘\textit{melakkārar}’ discussion. It is absolutely certain, without any shred of doubt, that these groups of people were custodians of music and dance, and yet they ended up at the receiving end of the stick wielded by the colonial masters, Christian missionaries and definitely not by the Brahmins (see below), contrary to what Krishna would want his readers to believe (recollect the frog experiment). In the chapter of Unequal Music, Krishna begins thus, “In this essay, I will look at the position of caste within the Karnatic community and use history as a reference towards understanding the present.” (Krishna 2013:335)

On the contrary, there is hardly any real allegiance to history and only the present prejudices are bemoaned in whole of the chapter. Whether his silence on the real history is per chance or by design is left to speculations. So, let us fill that gap by looking at the history, not at those presented by the modern and post–modern ethnomusicologist whom Krishna refers to, but from an apparently distant yet connected domain of Anglo–Indian judiciary and the evolution of legal instruments. This is succinctly captured in the 1998 paper by Kunal Parker (Parker 1998:559), currently a law professor at Miami Law School, titled, “‘A Corporation of Superior Prostitutes’ Anglo–Indian Legal Conceptions of Temple Dancing Girls, 1800–1914.” In this paper, Parker gives an elaborate account of how the dancing girls (same were referred to as \textit{melakkārar–}s comprising of \textit{devadāsī}, \textit{rājādasī}, \textit{nāgasvara} players, \textit{naṭṭuvanār –}s etc. earlier in this paper) and the whole community got ostracized and driven to penury by the Anglo–Indian legal mechanism. Let us look at the material presented by Parker.

Parker notes in the introduction of his work that in the imagination of the reformist activity related to women in the colonial India, judicial reformist activity gets largely disregarded. To this front, Parker remarks that this is not a new phenomenon and quotes Judge West of Bombay High Court from \textit{Indu Prakash} (14 March 1887) who says, “judicial decisions have silently promoted the cause of female emancipation and progress.” Parker further observes that dramatic changes were produced through the incremental efforts of the Anglo–Indian Judiciary and that these efforts from the judiciary enjoyed unfettered discretion in shaping legal conceptions of temple dancing girls. It is significant to highlight the existence of two broad legal frameworks within the same judiciary. One offered the legal solicitude to the individual and this gets firmly located within the British 'public law.’ The second focuses on legal recognition of the demands of community firmly located within the Indian Law.

There existed a continuous tussle between the two contrarian legal frameworks in the Anglo–Indian judiciary and that shaped the society in the past. The incremental efforts from this judiciary was in fact not in the direction of bridging this gap between the two laws, rooted in two distinct and different cultures, but rather to increasingly view the Hindu private law through the Imperialist’s lens. This had severe repercussions and its consequences are glaringly visible as well as active even in today’s judiciary.

Parker notes that conviction of the temple dancing girls and temple servants (\textit{devadāsī}) under the Indian Penal Code (IPC) began in late 1860s under the provisions of articles 372 and 373 in IPC. Judicial activism gets rightly spotted by Parker here as he says that the legislative history of these provisions revealed no intention on the part of the framers of the IPC of targeting the temple dancers. Parker says, “The 'crime' of 'dedicating girls to a life of temple–harlotry' was, therefore, a pure judicial invention.” So, it becomes clear that the judiciary, in its silent approach as activist to emancipate women and bring about a reform in the society, had to criminalize this act of dedicating girls to temple as temple dancers (read \textit{devadāsī–}s). This incrimination needed the \textit{devadāsī–}s to be declared as ‘prostitutes’ and this is when the community starts facing insurmountable troubles. Parker writes that,

\begin{myquote}
“The legal representation of temple dancing girls as 'prostitutes' rested upon patriarchal Hindu legal norms surrounding the sexual activity of women.” 

~\hfill Kunal Parker (1998:568)
\end{myquote}

Here we need to note that there existed a clear and informed distinction between extra–marital sexual affairs of sundry women vis–a–vis the activities (not restricted to sexual alone) of \textit{devadāsī}. Stemming from poor understanding of the Hindu customs of the then individuals and of the society as a whole, the judiciary did not succeed by merely declaring the \textit{devadāsī–}s as prostitutes. To criminalize, indict or convict these \textit{devadāsī–}s, the ‘private law’ required these \textit{devadāsī–}s to belong to a particular ‘caste’ (\textit{jāti}).

Furthermore, Parker alludes to one Bombay Sudder Dewanee Adawlut who described Section 26 of Regulation IV of 1827 as, “requiring 'that cases should be decided according to the custom of the caste, and established usage’” (Kunal Parker 1998:564).

From this instance and few others (mentioned in the paper) it is more than clear that the Imperialists consolidated their power in India through adjudicating the caste cases and now for them to intervene and adjudicate the cases pertaining to \textit{devadāsī–}s, their affiliation to a caste was a legal imperative.

Hitherto we aver that there were prostitutes, distinct from \textit{devadāsī–}s, both were under the ambit of Hindu law and the Hindu law had clear legal pronouncements for prostitutes. The public law wanted to criminalize the \textit{devadāsī–}s but could not do so because from their lens, they could not fit the \textit{devadāsī–}s in one caste bracket to eventually criminalize them. One thing that we need to understand in all this legal nitty gritty is that there were issues on property entitlement and inheritance. As the Hindu society and Hindu law had treated \textit{devadāsī–}s in a completely different manner, the matrilineal succession, inheritance, right to adoption and other rights were naturally bestowed upon them under the Hindu legal framework (which too underwent a lot of changes due to the impact of Victorian morals being imposed). Yet, the Hindu legal framework looked down upon the sexual activity of women outside their marriages (who were either referred to as degraded women or prostitutes) as well as those women who live outside the institution of marriage and had dealt differently on the entitlement, succession and inheritance for these classes of women.

It is in the latter half of the 19th century that the Anglo–Indian judiciary initiates a death blow to the \textit{devadāsī–}s by blurring the distinction between these degraded Hindu women (as looked from the Hindu private law point of view) and the \textit{devadāsī–}s.

Parker proceeds further to give minutest details on how this criminalization happened all the way until 1914 by looking at the appellate opinions, statutes and legislative documents pertaining to the \textit{devadāsī–}s.

Now let us understand the rationale behind all these. It all started with the East India Company trying to consolidate its colony in India. That means it needed to have a complete control on both the wealth production as well as its dissemination. Temples in India back then were the epicenter for not just religion but also for culture as well as economy. It was imperative to curtail people’s allegiance to these temples. People’s allegiance to temple was multi–dimensional yet coherent and cogent. \textit{Devadāsī–}s were part and parcel of temples and hence subsisted on the indulgence of the kings, feuds and society at large. It must be noted here that these \textit{devadāsī–}s were not living in the sympathy of the society; rather they were accorded highest respects, enjoyed high privileges. For instance, whenever a \textit{devadāsī }died, the temple mourned for her death and as a mark of respect, the temple remained closed for that whole day besides sending cloths from the temple to wrap around the body of the \textit{devadāsī.}

Society at large gave lands, properties and gifts to the temples and these were managed by the \textit{devadāsī–}s. Indirectly the society was patronizing the art and thereby upholding the tradition. This peaceful, productive, creative and practical nexus was the stumbling block that the Imperialists wanted to shatter, which they rightly did through the Anglo–Indian judiciary. Having declared them as “prostitutes” and hence “criminals”, \textit{devadāsī–}s were stigmatized and ostracized. Adding to the judicial agony that the \textit{devadāsī–}s faced, India at that time saw the impact of the ‘social purity movement’ of Britain which came upon very heavily on not just the \textit{devadāsī–}s in south India but also their counterparts elsewhere in the country (Courtney 1998). Under this movement, biblical morality was sought to be thrust upon the society. In India, the judicial framework as well as the Christian missionary worked hand in glove to see to it that the \textit{devadāsī} system got abolished, resources to temples stopped pouring in and thereby prepared grounds for evangelizing the society \textit{en masse}. Sadly, this process is active to this day. With the onset of the first World War, the baton was passed on to and was well–received by the locals whose influence, affluence, talent and political prowess were put to best use for uprooting the \textit{devadāsī} system, with which their music got exterminated. That is how we have Dr. Muthulakshmi Reddy, initiating the anti–nautch bill when she became the first woman to enter the Madras Legislative Council in 1927. This bill became The Madras \textit{Devadāsī–}s (Prevention of Dedication) Act (also called the \textit{Tamil} Nadu \textit{Devadāsī–}s (Prevention of Dedication) Act or the Madras \textit{Devadāsī} Act) right after India’s independence in 1947.

Significant to note here is yet another nuance from this period that went on to become the new norm in the reformists’, activists’ and revivalists’ narrative on the evolution of \textit{bharatanāṭyam}. The narrative has invented a binary between \textit{sadir} and \textit{bharatham/baradam }(short for\textit{ bharatanāṭyam}) where the latter, we are told, is a sanitized version of the former. Krishna alludes to this theory in his book, albeit without any accurate historical information. It is at this juncture we need to recall the episodic encounter in the 17th century between the saint Kumaraguruparar, of the Tamil Śaivite tradition and Dara Shikoh, the Mughal prince of Varanasi. History tells that the saint wanted to build a temple in Varanasi for which he needed to meet the Mughal prince. Prior to this meeting, he is said to have prayed to goddess Sarasvatī and in this prayer we come across this following verse:

\begin{verse}
\textit{paṇṇum baradamum kalviyum tīncôl punavalum yān,}\\\textit{êṇṇum pôḻudu yêlidu edhanalkāy eḻutha maraiyum}\\\textit{viṇṇum puviyum punalum kanalum vengālum anbār}\\\textit{kaṇṇum karuthum nirainthāy sakalakalā valliye}
\end{verse}

\begin{verse}
\tamil{\textit{பண்ணும், பரதமும், கல்வியும் தீஞ்சொல்பனுவலும், யான்\\ எண்ணும்பொழுதுஎளிதுஎய்தநல்காய்; எழுதாமறையும்,\\ விண்ணும், புவியும், புனலும், கனலும், வெங்காலும் அன்பர்\\ கண்ணும் கருத்தும் நிறைந்தாய் சகலகலா வல்லியே}!}
\end{verse}

Attention must be paid to the second word in the very first line. The saint is praising the goddess as the one who is adept in “\textit{Paṇṇum baradamum}” where \textit{Paṇṇum} refers to music and the saint uses the term \textit{baradam} for dance. Going by the reformist’s narrative, there must have not been any word called \textit{baradam} back in 17th century because according to these theories, it was only \textit{sadir} that existed before and it was once again the evil Brahmins who imposed their Aryan sense of morality, sanitized \textit{sadir} by clipping off its sensuous parts and thus was born \textit{bharatanāṭyam}. Well, \textit{bharatanāṭyam} is neither the sanitized version of \textit{sadir} nor is it an ‘Aryan’ import. It has been in the Tamil soil since ages. Then what about \textit{sadir}?

It is conjectured that perhaps this term is a derivative of the \textit{Tamil} word ‘\textit{sadukkam}’ which means ‘square’ as in ‘city square.’ So, the dance for public entertainment was perhaps referred to as ‘\textit{sadir}.’ But the temple dancers, the \textit{devadāsī–}s, were altogether a different group of people and their dance was different from these ‘street dancers.’ Even Parker notes that the incremental judicial activism that initially blurred the difference between temple dance girls and prostitutes eventually did not differentiate between ‘temple dance girls’ and ‘dance girls.’ This indeed gives an indication that dance girls existed back then who danced not in the temples but elsewhere and this indirectly gives credence to the conjecture stated above on the etymology of \textit{sadir}.

It is also significant to note that such dynamic social changes brought about by the judiciary were well–captured in the transformation of the word \textit{devaraḍiyā} (for \textit{devadāsī} in Tamil) to a cuss word ‘\textit{thevidiyā}.’ This linguistic transformation institutionalized the stigma on these \textit{devadāsī–}s so much so that this cuss word elicits extreme violent reaction on the utterer when used in public even until this day.

Interestingly the genre of \textit{Jāvaḷi} with its overt sexual composition appears during this point of time in the Karnatic history. So, it could be conjectured that this was a subtle, silent activism to distance people from the dancing girls. No wonder, the \textit{jāvaḷi–}s did not gain much currency, as explained above.

With the \textit{devadāsī–}s getting convicted as criminals, the fate of music and dance was in extreme peril. It was during such troubled circumstances that the Brahmins (incidentally, while occupying positions of power or otherwise) took upon themselves, acted responsibly and came forward to help them as well as to sustain this art form.

I would like to draw a corollary at this juncture. Provision 377 under the same IPC was in vogue until it got decriminalized in 2016. Hence, between 1864 and 2016 when same–sex relation was seen as a criminal activity, people obviously exercised utmost caution in not indulging in same–sex act (not to be construed that if this provision was not there, people naturally engaged in such relationships). So, when the Naz foundation fought for its abrogation and succeeded, the foundation got praised for its ‘bravery’, progressive thinking and liberal approach. Now contrast this with the Brahmin girls who took to learning dance and music and those teachers who took to teaching almost a 100 years ago, when (even until recently) Sections 372 and 373 of IPC were in vogue, coupled with the social stigma associated. Were these Brahmin women any less emancipators or progressive thinkers? They must be venerated for their courage, determination and dedication due to which we have the art alive till this day. On the contrary, Krishna and the likes of him continue to harp around the \textit{devādāsī }background of M.S. Subbulakshmi and M.L. Vasanthakumari when the society at large has only appreciated them for their music without dwelling into their personal backgrounds. The manner in which Krishna discusses the musical performance journey of MSS and D.K. Pattammal through the lens of \textit{devādāsī }and brahminism is abominable.

This crucial intervention by the bold and progressive Brahmins was the turning point in the history of Karnatic music, something that is not recognized at all. On the contrary, Krishna squarely blame the Brahmins for every evil without caring to substantiate with accurate historical evidence. How is it justified?


\section*{Concluding Remarks}

It is indeed pertinent to mention the episode of Vellai in Srirangam. This is an orally transmitted legend which finds its mention in the document pertaining to the temple history namely, “Kovil Olugu” (Parthasarathy 1954:22 (Note: this document mentions “devout devadasi” without mentioning the name explicitly); The name gets mentioned in this news article \url{https://www.thehindu.com/news/cities/Tiruchirapalli/the–legend–of–vellayi/article2774700.ece}) Vellai was a \textit{devadāsi} attached to the Srirangam temple. When the temple was attacked by the Muslim commander, Vellai played a key role in helping the temple retain its glory. Vellai danced in front of the commander and thereby distracted him while in the meantime the temple \textit{arcaka}–s transported the \textit{vigraha} to a safe place. After executing her plan successfully, Vellai killed herself from jumping off the \textit{gopuram} for having danced in front of a mortal. She has been immortalized by naming that \textit{gopuram} after her.

Such was the devotion of \textit{devadāsi–}s to the Lord and loyalty to the temple. Alas, clandestine activism by the Anglo–Indian judiciary in the guise of emancipating women decimated these temple dancers. These temple dancers were pivotal in sustaining and enriching the art. This art, along with them was at the brink of peril. Yet their art form survived and has lived until this day much to the agony of the anti–Hindu parties. This is the reason we are witnessing the Christianization of both \textit{bharatanāṭyam} and Karnatic music. And people like Krishna, per design or otherwise, are catalyzing this process of evangelization. Krishna and his likes are merely peddling the anti–Hindu propaganda of political outfits (\textit{Drāviḍa} parties and their partners), Christian Evangelical organizations and those self–declared progressive writers like Perumal Murugan who all have one thing in common – decimation of Hinduism. So whatever Krishna utters, other than Karnatic classical songs by traditional composers, is merely polemic pandering to anti–Hindu groups for their interest. Throughout the book he has chosen to belittle the Hindu sentiments in the garb of objectively narrating the story of Karnatic music. He has problems if Karnatic music is flourishing in the North America and yet grumbles that Karnatic has no international appeal or acceptance like the Hindustani (he cites Pt. Ravi Shankar as an example, yet does not utter a word on Annapurna Devi or her music; (Surti 2000)). On the lack of international acceptance of Karnatic, once again Krishna puts the blame squarely on the highly ritualistic Brahmins and their strong hold of the Karnatic ecosystem that has come in the way of this music form getting international appeal. In Krishna’s parlance, those Karnatic musicians who made their niche at the international level, have dabbled in fusion and have strayed away from Karnatic; while he is single–handedly enriching this art form through his innovative projects, ‘\textit{kaṭṭaikūttu}’ for instance. In sum, Krishna has a clear agenda which is working all the way through, under the garb of intellectualism. It is up to us and the society at large to take appropriate and adequate measures. Such a conference through which we can show up the inconsistencies of Krishna’s arguments is an apt illustration for a measure. And yes, by all accounts this book is indeed not merely an objective narration of the Karnatic story.


\section*{Acknowledgement}

I extend my sincere thanks to Sri Rajiv Malhotra ji and his Swadeshi Indology team for giving me this opportunity. I also thank Dr. Nirmalya Guha (Department of Humanistics, IIT(BHU) Varanasi), Sri Aravindan Neelakantan, Sri Veejay Sai, a colleague and two performing Karnatic musician (all three have requested anonymity) who gave their valuable time and inputs during my discussion with them at various instances for this work. Finally, I thank the reviewer for valuable inputs.


\section*{Bibliography}

\begin{thebibliography}{99}
\bibitem{chap6–key01} Baltzell, W. J. (1908). \textit{A Complete History of Music}. Philadelphia: Theodore Presser Co.

 \bibitem{chap6–key02} Courtney, David. (1998). “The Tawaif, The Anti – Nautch Movement, And The Development Of North Indian Classical Music” \url{https://chandrakantha.com/articles/tawaif/} Accessed on March 05, 2019.

 \bibitem{chap6–key03} Dvivedi, Pārasanātha. (Ed.) (2004). \textit{Nāṭyaśāstra} of Bharatamuni With the Commentary \textit{Abhinavabhāratī }Part I. (Gaṅgnātha Jhā – Graṅthamālā Series). Varanasi: Sampurnanand Sanskrit University.

 \bibitem{chap6–key04} Ghose, Rajeshwari. (1996). \textit{The Tyāgarāja Cult in Tamilnāḍu: A Study in Conflict and Accommodation}. New Delhi: Motilal Banarsidass Publication.

 \bibitem{chap6–key05} Gopinath, K. (2018). “Towards a Computational Theory of Rasa”. In Kannan, K. S. (2018). pp 89–177.

 \bibitem{chap6–key06} Iyengar, R. Ramanuja. (1958). \textit{Carnatic Music.} Rishikesh: The Yoga–Vedanta Forest Academy.

 \bibitem{chap6–key07} Kannan, K. S. (Ed.) (2018). \textit{Western Indology on Rasa – A Pūrvapakṣa}. (Proceedings of the Swadeshi Indology Conference Series). Chennai: Infinity Foundation India.

 \bibitem{chap6–key08} Kavi, M. Ramakrishna and Pade, J. S. (Ed.s) (1972). \textit{Nāṭyaśāstra} of Bharatamuni With the Commentary \textit{Abhinavabhāratī. }Vol IV. Baroda: Gaekwad’s Oriental Series.

 \bibitem{chap6–key09} Kavyatirtha, Narayan Ram Acharya. (Ed.) (2002). \textit{Raghuvaṁśa of Kālidāsa.} Varanasi: Chaukhambha Publishers.

 \bibitem{chap6–key10} Krishna, T.M. (2013). \textit{A Southern Music: The Karnatik Story. }Noida: Harper Collins Publishers.

 \bibitem{chap6–key11} Malhotra Rajiv (2014). \textit{Indra’s Net Defending Hinduism’s Philosophical Unity }Noida: Harper Collins Publishers.

 \bibitem{chap6–key12} Naidu, V. R. (1986). \textit{Our Glorious Heritage –A Dravidian Experience. }Isipingi: Tate Publications.

 \bibitem{chap6–key13} \textbf{\textit{Nāṭyaśāstra}} of Bharatamuni with the Commentary \textbf{\textit{Abhinavabhāratī.}} See Kavi and Pade (1964). See Dvivedi (2004).

 \bibitem{chap6–key14} \textbf{Parakala}, \textbf{Vangmayi. (}2017\textbf{). }“From ‘sabhas’ to streets, TMK style” \url{https://www.livemint.com/Leisure/ToL0FPCgmpXxbIU2Q6d00N/From–sabhas–to–streets–TMK–style.html}. Accessed on March 05, 2019.

 \bibitem{chap6–key15} Parker, M. Kunal. (1998). “'A Corporation of Superior Prostitutes' Anglo–Indian Legal Conceptions of Temple Dancing Girls, 1800–1914”. \textit{Modern Asian Studies}. Vol. 32, No. 3. pp 559–633.

 \bibitem{chap6–key16} Parthasarathy T. S. (1954). \textit{The Kovil Olugu: History of the Sri Rangam Temple}. Tirupati: Tirumalai Tirupati Devasthanams.

 \bibitem{chap6–key17} \textit{\textbf{Raghuvaṁśa}} of Kālidāsa. See Kavyatirtha (2002).

 \bibitem{chap6–key18} Rao, T. K. Govinda (Ed.) (1999). \textit{Compositions of Tyāgarāja. }Chennai: Gānamandir Publications.

 \bibitem{chap6–key19} Rao, Manasa. (2017). “Music beyond religion: TM Krishna sings \textit{Tamil} Sufi song at Mumbai church”. \url{https://www.thenewsminute.com/article/music–beyond–religion–tm–krishna–sings–Tamil–sufi–songs–mumbai–church–73535}. Accessed on March 05, 2019.

 \bibitem{chap6–key20} Rowell, Lewis. (1992). \textit{Music and Musical Thought in Early India– Chicago Studies in Ethnomusicology.} Chicago and London: University of Chicago Press.

 \bibitem{chap6–key21} Scott, Joan. (Ed.) (1996).\textit{Feminism and History. }New York: Oxford University Press.

 \bibitem{chap6–key22} Seetha, S. (1968). \textit{“Tanjore As A Seat of Music During The 17th, 18th And 19th Centuries” }(Ph.D. Thesis). Madras: Madras University.

 \bibitem{chap6–key23} Singh, Jaideva. (1984). “Abhinavagupta's Contribution to Musicology”. \textit{National Centre for the Performing Arts, }13(4). pp13–22.

 \bibitem{chap6–key24} Sinha, Mrinalini. (1996). “Gender in the Critiques of Colonialism and Nationalism: Locating the "Indian Woman". In Scott (1996). pp. 477–504.

 \bibitem{chap6–key25} Surti, Alif. (2000). “Annapurna Devi: The Tragedy And Triumph Of Ravi Shankar’s First Wife” \url{https://www.mansworldindia.com/people/annapurna–devi–the–tragedy–and–triumph–of–ravi–shankars–first–wife/} Accessed on March 06, 2019.

 \bibitem{chap6–key26} Sivananda, Swami. (1999). \textit{Sixty–three Nayanar Saints} (4th Ed.). Shivanandanagar: The Divine Life Society.

 \bibitem{chap6–key27} Terada, Yoshitaka. (2008a). “\textit{Tamil Isai} as a Challenge to Brahmanical Music Culture in South India”. \textit{Senri Ethnological Studies}. 71. pp203–226.

 \bibitem{chap6–key28} —. (2008b). “Temple Music Traditions in Hindu South India: Periya Melam and Its Performance Practice”. \textit{Asian Music}. Vol. 39, No. 2, pp108–151.

 \bibitem{chap6–key29} Tingey, Carol. (1994). “Review of Music and Musical Thought in Early India”. \textit{Journal of the Royal Asiatic Society} Third Series, Vol. 4, No. 1. pp 130–132.

 \bibitem{chap6–key30} Varagunapandyan, A. (1952). \textit{Paanar Kaivali (Panaar's Art)}. Chennai The South India Saiva Siddhanta Works Publishing Society, pp 9–11.

 \bibitem{chap6–key31} Chang, Daniel., Hedberg, Nancy., and Wang, Yue. (2016). “Effects of musical and linguistic experience on categorization of lyrical and melodic tones”. \textit{Journal of the Acoustical Society of America} 139 (5). pp 2432–2447.

 \bibitem{chap6–key32} Weidman, Amanda. (2003). “Gender and the Politics of Voice: Colonial Modernity and Classical Music in South India”. \textit{Cultural Anthropology}. Vol. 18, No. 2. pp194–232.

 \bibitem{chap6–key33} —. (2005). “Can the subaltern sing? Music, language, and the politics of voice in early twentieth–century south India”. \textit{Indian Economic Social History Review}. 42. pp 485–511.

 \bibitem{chap6–key34} Wordsworth, William. \textit{Poems in Two Volumes} (1807). \url{https://rpo.library.utoronto.ca/poems/i–wandered–lonely–cloud}.
 
 \end{thebibliography}

