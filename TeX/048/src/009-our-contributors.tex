
\chapter*{Our Contributors \namesinorder{(in alphabetical order of last names)}}\label{contributors}

\section*{Vrinda Acharya}

Smt. Vrinda Acharya is a well-known Karnatic Classical Vocalist from Bangalore. Being a prime disciple of Sangeetha Kalacharya Vidushi Neela Ramgopal, she has carved a niche for herself through her traditional concerts, thematic programs, lecture demonstrations and music workshops. She is a graded artist of AIR, recipient of several awards including Ananya Yuva Puraskara and Aryabhata Award, has performed across India and abroad, given programs on many television channels and has released many albums.

Vrinda is an M.Com. degree holder (Gold Medalist - 1st rank for Bangalore University), an M.A. in Sanskrit and a rank holder in Vidwat in Karnatic Music. A multi-faceted personality, she is also an academic scholar, researcher and writer in her own right. She has presented papers at several national and international conferences on music and Vedic heritage. A recipient of Research Fellowship from Karnataka State Government for her research on ‘Pre-trinity Sanskrit Compositions in Karnatic Music’, she has given lectures on Indian Music at many universities in the US. She has worked earlier for several years as a commerce faculty in reputed colleges and business schools in Bangalore. She is also well-versed with the Sampradaya Bhajana Paddhati and an amateur Veena artist. She is the Vice-President and Managing trustee of ‘Anubhooti’ – an organisation co-founded by her for the promotion and propagation of Indian traditional arts and cultural heritage.


\section*{V. B. Arathi}

Dr. V. B. Arathi is Chairperson, Vibhu Academy, Bengaluru. She has been teaching Sanskrit and Indology to students in India and abroad, from about 18 years, through online courses and workshops. She has been associated with Samskrita Bharati and other NGOs. She is a resource person in the Kannada print and electronic media and has presented her views on many platforms across inside and outside India.

Dr. Arathi integrates the wisdom of India into her Training programmes and youth mentoring. Through Vibhu Academy, she works towards empowering the \textit{desi} linguistics, Social leadership, Spirituality and Art appreciation in youth. Her clientele includes Corporates, Educators, Bankers and miscellaneous groups. She also conducts leadership programmes based on the epics, \textit{purāṇa}-s, Arthaśāstra and Indian history.

Apart from this, she is a Faculty, Academy For Creative Teaching, Bangalore; Trustee, Youth For Seva organization; Former Professor of Sanskrit, VVS College, Bengaluru; Guest Faculty, P G Courses, JU, Bengaluru.

She has published books, albums and articles and has been felicitated by Management, Spiritual, Educational and Cultural institutions in the state.

Her other pursuits include translations, Karnatic classical music, poetry and painting; writing as a columnist for Kannada and English newspapers.


\section*{Radha Bhaskar}

Dr.Radha Bhaskar is a unique combination of a vocal musician, musicologist, teacher, journalist and cultural organizer. She is a disciple of Padma Bhushan Sri. P.S. Narayanaswamy and has performed many vocal concerts all over India and abroad. In recognition of her yeoman service to music, she has received several titles like Kalaa Seva Bharathi, Yuvakalaa Bharathi, Sangeetha Kalasevak, Sangeetha Kala Bharathy, Acharya Award, Sathya Patrika Sundaram and Outstanding Best Musicologist Award.

Radha holds a doctorate degree in music for her thesis - “Karnatic Music Concerts – an analytical study” and received the Junior Research Fellowship from UGC. She was also awarded the Junior \& Senior Fellowship and a Production Grant from the Ministry of Culture, Govt. of India for research projects. She has given many educative lecture- demonstrations and also participated in several prestigious panel discussions all over India. For the past 15 years, Radha is the Editor of the reputed art magazine \textit{Samudhra}. She is Treasurer of the unique 25 year old cultural organization Mudhra, noted for its novel and purposeful programmes.

Dr.Radha has been sharing her music knowledge and experience by conducting special Music Appreciation Programmes to enlighten \textit{rasika}-s about the nuances of Karnatic music. She had the honour of being invited by Sangeet Natak Akademi, New Delhi to conduct a two days workshop on “Appreciating Karnatic Music.”

Dr. Radha has served as an Expert Member in the Ministry of Culture, Government of India and also been a Member of the Board of Studies of Indian Music at Madras University and Annamalai University.

\vspace{-.3cm}

\section*{Arvind Brahmakal}

Sri Arvind Brahmakal is a keen arts enthusiast and believes Indian art forms are an integral part of life. He established Ranjani Fine Arts in 2012, a \textit{sabhā} at Bengaluru, along with other enthusiasts with a framework to enable, educate and provide access to high quality performances in the local community. With over 150 programs, this has truly become a distinct community initiative. To consolidate fine arts celebration in Karnataka, he pioneered the formation of Karnataka Fine Arts Council - a first and only one of its kind which is a registered federation of 10 prominent \textit{sabhā}-s and serves as the Hon. Secretary. Kalavanta, an international level Karnatic Music concert competition for youth has recently completed its 5th edition. "Purandara Darshana", a unique program to highlight the contribution of the "Sangeetha Pitamaha", was conceptualised and successfully executed. With a vision to enrich people's lives through the arts medium, he established ArtsforLife Foundation as a digital platform. He volunteers his time for all the above 3 charities. His articles on contemporary issues and challenges facing Karnatic Music have been published in reputed magazines.

Trained as a Chartered Accountant and a Cost Accountant, he held senior leadership positions in IBM and Britannia prior to co-founding a CFO advisory firm, Goldklix Business Services.


\section*{Gayathri Girish}

“Kalaimamani” Smt. Gayathri Girish is one of the leading musicians in the field of Karnatic Music. She started her music lessons from Vidwan Sri Vaigal S. Gnanaskandan and later on came under the tutelage of Sangeetha Kalanidhi Sri. Madurai T.N. Seshagopalan. She has given concerts in all major \textit{sabhā}-s in Chennai and has travelled widely throughout India and many other countries. She represented our country and performed in Russia for “The Year of India in Russia-2009” in May 2009. She is the recipient of several prestigious titles like “Kalaimamani”, “Isai Peroli”, “Sahithyapriya”, “Sangeetha Bhaskara”, “Gaana Rathna” from the Department of Cultural Affairs, Colombo, “Yuva Kala Bharathi” and many more. She also received the “Ustad Bismillah Khan Yuva Puraskar” instituted by Sangeet Natak Akademi, New Delhi, in August 2014. She is an “A-grade” artiste of All India Radio, Chennai and has performed for Doordarshan and other private television channels. Under the Production Grant Scheme of The Ministry Of Culture, New Delhi, she did a thematic multimedia production in 2013-14 on the topic “Myriad Forms of Lord Shiva”. She is now currently doing research under the Senior Fellowship scheme awarded by the Ministry Of Culture for 2018-20. She has also given several lecture-concerts and lecture-demonstrations on various topics in music. Smt Gayathri holds a postgraduate degree in Computer Science.


\section*{Jataayu}

Jataayu has been writing on a range of topics centered around Hindutva and Indic social and cultural themes since 2005, both in Tamil and English. He is well read on Hindu philosophy, history, culture and arts. He is a scholar of Tamil literature, both classical and modern, especially Kamba Ramayanam and Bhakti poetry and gives discourses on these subjects. A collection of his Tamil essays has been published as a book \textit{Kaalamthorum Narasingam} (2015). His Tamil writings have also appeared in the anthology \textit{Panpaattai Pesuthal} (2009). He is on the editorial board of the popular website \textit{tamilhindu.com} and the Tamil monthly magazine \textit{Valam}. An Electronics Technology professional by vocation, he is currently a resident of Bengaluru. His real name is Sri Sankara Narayanan and his pseudonym is inspired by the supreme devotion, valour and sacrifice of the legendary bird from \textit{Rāmāyaṇa} whom he holds as a great ideal.


\section*{V. Ramanathan}

Dr. V. Ramanathan is an Assistant Professor in the Department of Chemistry at IIT(BHU) Varanasi. Earlier he worked at SASTRA, a Deemed to be University in Thanjavur, Tamil Nadu. Prior to that, he carried out his post-doctoral research works in University of Stuttgart in Germany, University of Basel in Switzerland and Seoul National University in South Korea. He obtained his PhD in physical chemistry from IIT Kanpur, India.

His areas of academic research are Raman imaging and spectroscopy, laser spectroscopy and computational chemistry. His teaching assignments include handling courses in physical chemistry subjects like quantum chemistry, group theory and spectroscopy, chemical kinetics etc. at both the undergraduate and graduate level. He has around 25 research publications in peer reviewed international journals pertaining to his areas of academic research. His other research interest lies in studying the scientific and mathematical heritage of India, Indian history, philosophy, Indian traditional medicine, Indian classical music and Indian languages (comfortable in 7 Indian languages).

He is a Fulbright scholar and a member of Indian National Young Academy of Science (a body of Indian National Science Academy (INSA)) within which he is one of the seven core committee members.


\section*{Korada Subrahmanyam }

Dr. Korada Subrahmanyam is a Professor of Sanskrit, Centre for Applied Linguistics and Translation Studies, University of Hyderabad, Hyderabad. He has teaching experience of over 36 years and his mentorship has helped to produce 20 M.Phils and 8 Ph.Ds. His specializations are Pāṇinian Grammar, Philosophy of Language, Vedas, Vedāṅga-s, Darśana-s and Upaveda-s (17 Vidyasthanams). He learnt Kṛṣṇayajurveda, Upaniṣads, Vedāńta, Pūrvamīmāmsa, Nyāya, Vaiśeṣika, Vyākaraṇa, and Jyautiṣa in a \textit{gurukula} and teaches \textit{śāstra}-s online in the \textit{gurukula system}. He has presented papers at 60 conferences and seminars and published 50 articles among which 20 are for the Jigyasa Foundation USA. He has also created 17 modules for the ePathshala, UGC. His publications include \textit{Mahāvākyavicāraḥ, Vākyapadīyam} (Brahmakāṇḍa) English Translation, \textit{Four Vṛtti-s in Pāṇini, Theories of Language: Oriental and Occidental, Pramāṇas in Indian Philosophy, Vedāṅgas and Darśanas} (MP3 CD in Telugu). He has to his name 30 Panditasammanams and has been conferred the title “Mahāmahopādhyāya”.

