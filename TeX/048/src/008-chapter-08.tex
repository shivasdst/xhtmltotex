
\chapter{Role of Patronage in Karnatic Music – Past, Present and the Future}\label{chapter8}

\Authorline{Aravind Brahmakal\footnote{pp. 211--226. In: Meera, H. R. (Ed.) (2020). \textit{Karnāṭaka Śāstrīya Saṅgīta - Its Past, Present, and Future.} Chennai: Infinity Foundation India.}}

\lhead[\small\thepage\quad Aravind Brahmakal]{}

\vspace{-.3cm}

\begin{flushright}
\textit{(aravind arvind.brahmakal@gmail.com)} 
\end{flushright}


\section*{Abstract}

Karnatic music lives through the artistes. Promotion of artistes leads to a sustainable celebration of the art form over time. As such, ensuring that artistes are able to eke out a good living by pursuing Karnatic music is of paramount importance.

In the documented past, musicians were provided patronage by royalty, big businessmen and by the general public through temple festivals, etc. Teaching Karnatic music also earned a livelihood for artistes. Hamlets used to be the place where people largely lived in and hence, the people to people connect was fairly direct which lent itself to the above patronage model.

With the small villages growing to mega urban landscapes, there arose a need for bringing musicians and listeners together to a common platform. This led to the creation of \textit{sabhā-}s. Typically, not-for-profit organisations, each \textit{sabhā} would become relevant to\textit{ rasika-}s in a specific geographical area. The patronage model significantly changed from the past to what it is today. Patrons donate to the \textit{sabhā-}s who, in turn, pay out the artistes. Another significant change has been the establishment of the digital media. Government organisations like Doordarshan, All India Radio and several music production houses have provided patronage to the artistes. With the advent of social media, there are monetization opportunities for artistes though Youtube, Spotify, Gaana, and many other applications. With larger audiences attending performances, Corporates are also showing some interest in sponsoring programs. Government has grants that are given out on an annual basis for festivals organised by \textit{sabhā-}s in addition to awards with prize money. Teaching students in person as well as through digital media like Skype, Lync, WhatsApp, etc provides income to the artistes. Workshops organised by \textit{sabhā-}s\break and the foreign tours have added to the source of income of the artistes. Many artistes have creatively expanded their incomes by having cruise concerts, etc. Overall, if we look at the present, there are many more avenues than the past to monetize the time and expertise of the artistes.

A peek into the future will make us even more optimistic of how the size of this pie can be grown manifold. The fact that the art form on digital media is just about to take off gives enormous hope. “Direct to consumer” of content through digital will define and also eliminate the intermediaries. Monetization opportunities on digital through Corporate sponsorships are bound to grow given larger number of audiences. Live performances will progressively move towards paid programs. The model is expected to move towards “user pays” model vs not-for-profit \textit{sabhā-}s. The big growth in terms of patronage for the art form will also be because of the better standard of living for humankind in the years to come. As Karnatic music will be placed alongside yoga, meditation etc., – something that provides peace of mind – this avenue will be patronized like never before as people will have higher disposable incomes and are in need of peace of mind. Government is also expected to invest a lot more on our culture as this is the soft power India can exercise.

The golden period for patronage in Karnatic Music has just begun and will last for a long time into the future.


\section*{Introduction}

A foundational question that needs to be asked is whether Karnatic music is an art form or is it something beyond? We have heard the commonly used adage: “Music is life and Life is Music”. This comes from a very deep sense of linkage one could draw between music and life itself. As with life where there is sound and rhythm\textit{,} so with music – \textit{nāda} and \textit{laya}. With this kind of bond, how should we look at music distinct of life? An easy answer can be found in that, this kind of relation is at a metaphysical level and hence one needs to attain a particular state of being to realise such a connection.

Let us then move to the understanding of music as an art – a performing art. Practice of music as an art is a necessary step towards the ultimate realisation. Among the very many art forms, Karnatic music is one of the most evolved ones. Over several millennia, this art form has grown in terms of form and substance. A combination of \textit{saṅgīta, sāhitya} and \textit{adhyātma} has contributed to this growth. There has been substantial progress in \textit{saṅgīta} from the \textit{Sāma Veda} period to the formulation of the the most comprehensive 72 \textit{melakarta rāga} system. Additionally, Karnatic music has developed with deeply devotional lyrics over time that has ensured its relevance at any point in time. The icing on the cake is the liberal sprinkling of spirituality – \textit{“adhyātma”.} With all of these elements generously present, this art form has something really substantive. One can trace the reason for its sustenance and growth from millennia to the depth it has acquired. The tradition has grown – with additions.

“Culture eats strategy for breakfast” – says Peter F Drucker, the globally acclaimed marketing genius, or at least, attributed to him. A deep appreciation of culture and adequate patronage is a \textit{sine qua non} for promotion of the artistes – thereby, the art.

\vspace{-.4cm}

\section*{Role of Patronage – the Past}

\vspace{-.2cm}

The patronage model for Karnatic music, in the past, was fairly simple with very few dimensions. Kings used to appoint \textit{āsthāna vidvān-}s. The needs of the musicians were looked after by royalty, so they could steadfastly devote time to the art form. Wealthy businessman patronized artists and so did temples. Teaching students too would bring in its share of sustenance to the musicians. Overall, the means of monetization were limited. A comprehensive analysis of the past is presented below of how the patronage worked in the past few centuries.

Indian Classical music has its origins attributed to Vedic times and also celestial beings like Nārada, but the seeds of the form familiar today can be traced back to the period between the 14th and the 17th centuries, thanks to the contributions of Purandaradāsa, the Karnāṭaka Saṅgīta Pitāmaha and also his contemporaries like Annamācārya, Bhadrācala Rāmadāsa, and Kṣetrayya. This music flourished during the reign of the deeply dhārmic Vijayanagar kings who were also connoisseurs of art. Then came the Tanjavur Nayaks and the Maratha kings of Tamilnadu whogenerously patronized all forms of art and literature. The temple, as always, served as the hub for music, dance and literary activities such as festivals and annual competitions that attracted very many scholars, composers and artistes from across the region. We will briefly look at the patronage granted by some of these rulers.

Raghunātha Nāyaka (1600 -1634 CE) who was the most prominent of the Nāyaka rulers was not only a discerning patron but himself an artist and a composer. In addition to composing Yakṣagāna-s, he was a good \textit{vaiṇika} and also a poet of merit in Telugu and Sanskrit. His liberal patronage was extended to poets, poetesses, scholars, composers, \textit{yati}-s\break and temples.

His successor, Vijayarāghava Nāyaka (1634 CE-1673 CE), matched his predecessor in all respects, being a patron and a scholar par excellence himself. Amongst others in his court, were the great Veṅkaṭamakhin (the author of \textit{Caturdaṇḍi Prakāśikā}), his brother Yajñanārāyaṇa Dīkṣita, and Chengalvakala Kavi (the author of \textit{Rājagopālavilāsamu}).

The period of Shahaji Bhonsle (1684 -1712 CE), the Maratha ruler of Tanjavur, can be identified with a remarkable literary and musical output. His successor Saraboji I (1712 – 1728 CE) too followed suit. \textit{Agrahāra}-s of Tiruvenkadu and Tirukkadiyur were exclusively established to provide the necessary ambience for culture to flourish. Amongst those who adorned his court was Girirāja Kavi (who was Śrī Tyāgarāja’s grandfather). Following him was Tulajaji I (1728 – 1736 CE) who is also credited with the authorship of \textit{Saṅgīta Sārāmṛta}. He promoted \textit{Sadir} and \textit{Bharatanāṭyam} styles of dance while composing a few Yakṣagāna-s. During his time, the Tanjavur \textit{vīnā} came to be known as Tulaja \textit{vīṇā}.

The reign of the next significant patron king in the lineage, Tulajaji II (1763–1787 CE), was a landmark period in the history of Karnatic music, in that it provided the necessary background for the phenomenon which was to be known as Karnatik Music Trinity. Tyāgarāja’s \textit{guru}, Sôṇṭi Veṅkaṭarāmaṇayya, and Śyāma Śāstri’s \textit{guru}, Paccimiriyam Ādiyappayya were \textit{āsthāna-vidvān}-s in his court. Śyāma Śāstri’s father deciding to settle in the region was mainly because of the Bangāru Kāmākṣī temple built by this king. While Tyāgarāja’s father, Rāmabrahma, was employed by the king to take care of the administration of some \textit{agrahāra}-s, Muttusvāmi Dīkṣita’s father, Rāmasvāmi Dīkṣita, was entrusted with composing and formalizing songs for the \textit{devadāsī}-s of the Tiruvarur temple. Apart from these, the father of the Tanjavur Quartet, Subbarāya Oduvar, too served in the court of this king. It was during this time that the Tanjavur Quartet and the Dīkṣitar brothers, Muttusvāmi and Bālusvāmi came into prominence. The Quartet - Cinnayya, Pônnayya, Śivānandam and Vaḍivelu – who fine-tuned the art of \textit{Bharatanāṭyam} and also composed a number of \textit{varṇa}-s and \textit{kṛti}-s, later moved to the court of Mahārāja Svāti Tirunāḷ (who happened to be a great friend of Serfoji, the last Maratha ruler of Tanjavur).

In Mysore kingdom, music in particular and arts in general received much encouragement. The Golden Age for Karnatic music however was for more than 150 years during the reigns of Mummaḍi Krishnaraja Wodeyar (1794–1868), Chamaraja Wodeyar (1862–1894), Nālvaḍi Krishnaraja Wodeyar (1884–1940) and Jayachamaraja Wodeyar (1919–1974). These kings themselves were noted composers and were proficient in playing musical instruments. This was the time when Mysore in modern Karnataka emerged as a prominent centre of Karnatic music, just as Tanjavur did in modern Tamilnadu. The Mysore court not only had native musicians like Vīṇe Śeṣaṇṇa, Vīṇe Subbaṇṇa, Mysore Vāsudevācārya, Biḍāram Kṛṣṇappa and Cikkarāmarāyaru but also artists and composers from other parts of South India like Muttayya Bhāgavatar and Tiger Varadachariar\endnote{

 \textbf{Editor’s Note}: This part of the article was compiled by collecting information from the following sources:

\begin{myquote}
Subramanian, Lakshmi. (2011). \textit{From the Tanjore Court to the Madras Music Academy: a Social History of Music in South India.} New Delhi: Oxford University Press.
\end{myquote}

\begin{myquote}
Sampatkumaracharya, V. S., and Ramarathnam, V. (2012). \textit{Karnāṭaka Saṅgīta Dīpike} (in Kannada). Mysuru: D. V. K. Murthy Prakashana.
\end{myquote}

\begin{myquote}
Iyengar, Rangaramanuja. (1993$^{2}$). \textit{History of South Indian (Carnatic) Music.} Bombay: Vipanchi Cultural Trust.
\end{myquote}}.

Over time, human settlements starting getting larger in a particular place. With the end of British rule and India gaining independence, the erstwhile provinces were re-organised into linguistic states. The era of kings ended and democratic governments started administering the affairs of the people. Technological developments started gaining pace like radio, TV, internet, etc. With these developments, Karnatic music patronage model had to re-invent itself.

\vspace{-.4cm}

\section*{Role of Patronage - Current}

\vspace{-.2cm}

Governments had to invest lot of resources in building basic infrastructure of the people. As a consequence, the grants for arts and culture were limited and could not be a sufficient stream of earnings for artists. The setting up of All India Radio, and eventually Doordarshan, provided a great opportunity for artists to be employed in their area of expertise. In addition, programs on those platforms also provided good income to free-lance artists. Many government agencies like Indian Council for Cultural Relations (ICCR), Central Ministry of Culture, Sangeet Natak Akademi, and State Departments of Culture provide great support to the arts through their grants, awards, fellowships and scholarships.

However, for live performances, there arose a need for an intermediary to facilitate music programs. This led to the setting up of \textit{sabhā-}s by various noble connoisseurs. Largely being not-for-profit organisations, the volunteers strived to raise money from philanthropists and would pay out the artists and the surrounding ecosystem. Such noble deeds started spreading across the globe and now, these organisations provide a significant source of earnings for artists. With the internet and global connectivity, artists can monetize content creatively on the net through multiple applications. Thanks to the global outreach digitally, musicians can teach students located anywhere in the world. This development has significantly improved the earnings of teachers.

Despite several such avenues that have opened up, the challenges in today’s patronage model are very many. In a recent comparative study conducted by Mckinsey \& Co. on philanthropy in the USA and India, the stark differences are disturbing. In a fairly recent civilization as the USA 10\% of philanthropy goes towards art and culture while in India, itis less than a percentage for the entire arts and culture. And, for Karnatic music, it is practically so miniscule. The size of the overall philanthropy purse is much larger in the USA and hence the absolute amount that goes to promotion of arts and culture is significantly high. As a society, we need to start recognizing the relevance and importance of contributing to our arts and culture.

The ramifications of poor patronage have a direct adverse impact on the quality of art. Youth do not get the confidence to take up this as a full-time profession. Resultantly, as a part time musician, the required investment of time on the art is diminished leading to sub-optimal delivery of a performance. Hence, Karnatic music is still a “laggard profession”.

The role of \textit{rasika}-s in supporting the art form has to be looked into. Often, we do not think twice to shell out hard earned money on a pizza, burger or a movie in a multiplex, justifying it as ‘weekend relaxation'. However, the same people resist and complain if a Karnatic music performance is ticketed. With many, there is a perceived sense of entitlement that Karnatic music performances should be free of charge.

The average age group of people attending concerts regularly has never come down. It is the aged people who are mostly spotted in Karnatic music concerts – as a meaningful post retirement engagement. However, there is a dire need to get a mix of all age groups – as this enhances the chances of improving the patronage for the art form. How this can be done without diluting the “quality”, is a challenge artists need to explore.

In today’s context of Karnatic music, the ways of attaining \textit{artha} (monetary gain to satisfy the needs of a decent livelihood) and \textit{kāma} (desire to reach the top and be well acclaimed) becomes very important. Following the \textit{dharma} \textit{mārga} (righteous way) to claim \textit{artha} and \textit{kāma} should be the only way ideally. However, with a deficit on \textit{artha} and \textit{kāma}, there are shortcuts that artists have started taking. Many artists have been trapped in the allurements by the expansionist faiths like Christianity and Islam. Substantial sums of money being offered to sing a composition hailing Jesus or Allah have been heard from many quarters. In this context, I am cited from an article that I had written for \textit{Sruti Magazine} called “Native and Immigrant Sounds” below:

\begin{myquote}
“One dimension of the recent debate is whether songs on Jesus and Allah … should form part of the Carnatic Music idiom. Every faith has some native sounds and the people practicing them are conditioned to listening to them, over several centuries. This is very much like symbols, structures, culture, etc. Compositions on Hindu Gods are in Carnatic Music idiom as these sounds are native to this faith. Hence, the audience you find in any Carnatic Music concert is largely ones who practice Hindu faith. Feeding these audiences with songs on other faiths is like thrusting other faiths on Hinduism. The question to then ask is towards what end is this experiment aimed at? …
\end{myquote}

\begin{myquote}
Freedom of expression, subject to reasonable restrictions, has to be protected for every artiste. Artistes being creative people will want to explore different dimensions to expand the acceptability of their art. Let the experiment be to promote social harmony in the true sense of the term by not mixing everything up but by respecting and honouring each for what it is. Within these confines, surely there is plenty to be done as music is universal and can provide peace, happiness and relaxation to everyone on this planet. Artistic freedom is good but artistic adventurism can be lead to a more hateful world, which goes exactly opposite to the purpose of practicing the art.
\end{myquote}

\begin{myquote}
…Every innovative artistic attempt without a sound rationale and a purpose may be viewed as a back door support for inter-faith conversions. Music collapses...”
\end{myquote}

There are several organisers and musicians who were interviewed as part of this research. The actual comments are reproduced below:


\section*{Views of Organisers}

\begin{enumerate}
\itemsep=0pt

 \item Sri Anantaramiah, \textit{BTM Cultural Academy}
 
\begin{enumerate}
\itemsep=0pt

 \item A strong investment of money is necessary to encourage more youngsters and give sustainable bright future to music. This far, no further - is the sad expression on our musician’s face. They are in need of more opportunities and better remuneration.

 \item Call for a more liberal government funding.

 \item Many organisations are coming up today out of immense love and reverence towards classical music. They spend their own money to run the organisation. Such organisations have a small shelf life and their future is uncertain post the founder’s life term. Organisations that conduct concerts all through the year cannot sustain for a long time due to growing financial demands in the market.

 \item Karnatic music also needs the volunteering by youth/young \textit{rasika}-s who are proactive and ready to lend their helping hand to attend and organise concerts in any way that is conveniently possible to them.

 \item The organiser also has a role to play in discerning and directing the show of an authentic Karnatic music concert. It is his/her duty to condemn and shun presentation of diverted music forms for want of instant popularity, if any, to keep the traditional authentic music intact.
\end{enumerate}


 \item Sri Ravishankar, \textit{Bharatiya Samagana Sabhā}
 
\begin{enumerate}
\itemsep=0pt

 \item Indian classical music has a very limited number of connoisseurs as not everyone can understand the nuances and subtle intricacies of this art form. It amounts to approximately 4-5\\% of the total number of fans of music in the world music platform. Out of this, Karnatic music rasika-s amount to only a third of the whole set. That is 1\\% fan base/connoisseurs/patrons is what is available to Karnatic music. In such a situation, where the fan and patron count is low, the opportunities for an artist to grow and flourish becomes limited, owing to growing number of people in the music scene.

 \item The need of the hour is to encourage and produce more full time Karnatic music professionals. This should begin with mobilisation and investment of more money and mind into the industry.

 \item A solution would be to adopt musicians by institutions such as banks, Railways businesses or individuals, as it is done in the sports world where sportsmen become brand ambassadors and get paid huge revenues no matter how many number of matches they play or how they play in an year. This model, if followed in Karnatic music scene, is very promising and ensures musicians with a secured life with constant and sustainable financial support, which gives them enough freedom and time to be immersed in the \textit{sādhanā} and thus produce high-quality music over the years.

 \item The corporates have a big role to play in today’s music world. Through Corporate Social Responsibility (CSR), they can very conveniently sponsor big events and cater to a large number of budding musicians and assure them a secured future. All these efforts result directly in an upsurge of quality and quantity of authentic music that we can offer to the world.
\end{enumerate}


 \newpage

 \item Dr. R. Raghavendra, \textit{Ananya}
 
\begin{enumerate}
\itemsep=0pt

 \item Organisers have a great responsibility in identifying authentic music and encouraging such music, instead of succumbing to external pressures or going with the trend to get instant fame.

 \item The future of many organisations whose torch-bearers are single handedly working for the cause of music is bleak. What after them? The government can identify such organisations and lend liberal support.

 \item “Rich \textit{sabhā}-s” in Bengaluru should collaborate with \textit{sabhā}-s in smaller towns.
\end{enumerate}


 \item 
 Dr. Deepthi Navaratna, Executive Director, \textit{IGNCA}, MoC, Govt of India.

 While the commitment to fostering arts and culture through central and state cultural agencies is clear music presenters have to think more creatively at diversifying their revenue streams. In today’s world it is more than a necessity.

\end{enumerate}


\section*{Views of Artistes}

\begin{enumerate}
\itemsep=0pt

 \item Sunaad Anoor (Khanjira artist)
 
\begin{enumerate}
\itemsep=0pt

 \item Patronage given to music today is a lot better and promising than previous generation and yet, a lot can still be done. Each artiste needs to grow and evolve with one’s music.

 \item The \textit{sabhā}-s, corporate agencies and individuals are showing generosity and encouraging musicians who are deserving and talented, thus making more young artists confident about pursuing Karnatic music full time.

 \item Karnatic music can be taken as a profession if one is totally sincere to the art and to oneself and believes in music. It is the responsibility of the artist to be worthy of good patronage.

 \item Another responsibility on the artist’s shoulder is to relate to the audience – educated as well as lay – and modify one’s music without diluting the core values.

 \item Unity among artists is of utmost importance for Karnatic music to flourish and be in safe hands. We have to work together and support one another. Professional jealousy needs to be kept at bay as much as possible, so as to give an opportunity to all musicians according to their capacities.
\end{enumerate}


 \item Anjana P. Rao (Vocalist)
 
\begin{enumerate}
\itemsep=0pt

 \item The current patronage trend is encouraging and improving every year with a growing population of upcoming\break artists in the Karnatic music scene.

 \item To take music as a full-time profession, it requires a lot of courage and thinking ahead to have a sustainable career as a musician.

 \item Musicians who have succumbed to various attempts of missionaries to propagate Christianity at the cost of compromising one’s identity are not only money-hungry but also ignorant about the consequences of the act. They have to be more responsible and educated about online media and its reach.

 \item Mobilisation of money in the \textit{sabhā-s} by organisers is very important to keep the show running. \textit{Rasika}-s and organisers need to recognize the years of devotion, hard work and perseverance of the artist and financially compensate suitably.
\end{enumerate}


 \item Shilpa Shashidhar (Vocalist)
 
\begin{enumerate}
\itemsep=0pt

 \item Need of the hour is to have more systematic and organised methods of arranging concerts. It is the knack of the organiser to mobilise funds - which is important. The organiser also has to be well-informed and know where to tap the resources to have a good continuous flow of funds and never to misuse the resources.

 \item Research in Karnatic music needs improvement and a lot of funding. \textit{Sabhā-s} can give incentives to encourage performers who are researchers.

 \item Organistions can plan and organise concerts in a suitable ambience that satisfies acoustic requirements of a Karnatic concert presentation to improve the listening experience. There are some \textit{sabhā-s} that are taking care of this exceptionally well and serve as a model to most others. Construction of acoustically treated halls and state-of-the-art sound systems would require funds to be raised and organisers need to work in this direction.
\end{enumerate}


\end{enumerate}


\section*{Role of Patronage - Future}

With the right kind of corrective measures taken, there is a bright future for patronage for Karnatic music. It is not that there is a shortfall of money in philanthropy. However, it is about how we are able to get a “share of the wallet”. Whether it is the listener, the governments, philanthropists, corporates – there is a need to re-package to ensure continued and enhanced patronage.

\begin{itemize}
\itemsep=0pt

 \item Artists can spearhead this movement by exposing the beauty of Karnatic music to global audiences by tapping the power of internet. Getting as many videos up there of performances and explainers will let the beauty of the music draw more people towards the art form. When there are more people, the chances of getting funding to move in this direction are much brighter.

 \item There is a dire need to revert to “popular Karnatic music” and keep it distinct from “art music”. While the latter is for a limited audience, popular Karnatic music is something that blends into life and thereby attracts more people.

 \item \textit{Rasika-s} have a responsibility. As they draw personal happiness from this art form, they should be willing to open their purse strings consistently. How about 2\% of their annual income to be donated to any credible organization?

 \item One important aspect to accomplish is to invite youth to take charge of running \textit{sabhā-s.} The kind of energy they can bring in will certainly attract more youth to be present in performances. Such volunteering activity also helps such youth in their academic pursuits for admissions and scholarships.

 \item 
 It is wonderful to have more and more \textit{sabhā-s.} However, at all points in time, the demand v/s supply will be more skewed towards excess of supply. Artists need opportunities to exhibit their talents and at regular frequencies. A suggestion is to expand the concept of chamber concerts and celebration concerts. Can each \textit{rasika} host a concert at his/her home at least once a year, and look for opportunities to celebrate and invite artistes to perform? In this context, I am sharing my article “Nurturing talent” published in \textit{Ananya Kalasinchana} magazine:

\begin{myquote}
“It is wonderful to see any new talent emerge in the field of arts. The freshness this brings energises the entire eco system. This is required to continue the ongoing art tradition. It is an opportunity to witness new feelings, new thoughts, new style, new dimensions, new vistas, etc. The artiste emerging also has so many dreams and expectations of showcasing the talent to as many rasikas in the shortest possible time. This energy is truly infectious.
\end{myquote}

\begin{myquote}
The number of platforms available for young artistes is limited. All the organised sabhas have their structured programming. Although they find some place for the youth, it is well below what is needed to nurture talent. There are certain sabhas that dedicate themselves to promoting new talent out of sheer passion. Similarly, many temples promote new talent. With all of these avenues available, new talent is still in search of more opportunities, better opportunities.
\end{myquote}

\begin{myquote}
The concept of nurturing talent is all about how many opportunities does the new talent get during the learning phase and at what frequency. This becomes a big learning ground and a medium through which the art of performance can be fine-tuned. Each of these performances can explore a different aspect, thereby enlarging the repertoire.
\end{myquote}

\begin{myquote}
How then can the eco system provide so many opportunities for so many new artistes? The existing organisations can barely meet the demands of so many youngsters. Is the answer for these sabhas to scale up the number of events? Even if possible, it may not be practical and sustainable. The answer is certainly not about creating many new organisations. Anything that gets formalised is bound by structures and commercials. This becomes a hindrance for nurturing new talent at free will.
\end{myquote}

\begin{myquote}
A concept that was in existence from a very long time that seems to have reduced significantly is that of chamber programs. This is not a very formal organisation. Interested rasikas can host a program every month or every other month in their own homes. The invitees would be their neighbours, friends and relatives. A personalised environment that also helps satiate the desire of the rasikas without having to travel distances. This is also not a very expensive proposition as the new talent is looking for opportunities to perform and not necessarily to earn a livelihood. These performances could also be to celebrate some occasion in their homes – festivals, birthdays, graduation, anniversaries, etc
\end{myquote}

\begin{myquote}
Teachers are another big source of encouraging emerging talent. Once a quarter, in their home or wherever they teach the art form, such teachers could host a performance of someone other than their own student. This will inspire their students and also provide a good platform for a promising talent. Imagine the number of performance opportunities this would open up!
\end{myquote}

\begin{myquote}
Those youngsters who are technologically savvy can create a dynamic portal where a list of such emerging talent can register themselves with their bios and a link to any of their performances. This will provide a ready database for the rasikas and teachers to invite artistes. Constructive feedback possibilities can also be created on this portal which will then go to build the personal brand of the artistes.
\end{myquote}

\begin{myquote}
While the above-stated may not be a new thought, the newness can be brought about when this is done on a large scale. Teachers and rasikas can step forward in big numbers to enable such a framework. The impact of this becomes visible if we create additional 100x opportunities every month for youngsters through these media. In a very brief period, we can witness a vibrant youth circuit that will take shape …..
\end{myquote}

\begin{myquote}
This model will reduce the dependence of new talent on organised sabhas, endlessly seeking opportunities. The possibility and frequency are so few and far between that the artistes start losing their edge. Those artistes performing well in the youth circuit will automatically push the organised sabhas to sit up and take note. A constant supply of artistes is only good news for any organiser.”
\end{myquote}

 Well, is this practical? It is upto each of us, as \textit{rasika}-s and teachers, to embrace this proposition in its spirit. It is about what kind of value we attach to our art forms and our passion to hand over this legacy to the next generation. Our people living outside India are conscious about this and do enough and more to preserve Bharatīyatā. Living here, can we?

 \item Government grants and schemes need to be significantly increased to promote Arts and Culture. In addition to increasing the size of the pie, there is a dire need to support every genuine organisation with ample grants. The methodology and screening process needs to be tightened and all well-meaning \textit{sabhā-s} need to be supported. There is a need for a complete overhaul in this process.

 \item As per the Corporate Social Responsibility (CSR) guidelines, a Company needs to spend 2\% of its net profit towards CSR. While this is a very good move, I wish to appeal to the Government to regulate a minimum of 10\% of the CSR amount to be spent on Arts and Culture.

 \item Corporates have a big role to play in sustainable patronage. A huge fillip can be provided to Karnatic music if Corporates take on themselves the responsibility of building high quality auditoria in the cities. These auditoria will become the nerve centres of cultural activities. Additionally, they can adopt worthy musicians who show enormous promise.

 \item What corporates need to do is to shed the veil of “pseudo-secularism”. It is a travesty that some corporates refrain from supporting \textit{sabhā-s} as Karnatic music is a “Hindu” art form. Earning profits from the land and not supporting the art and culture of the land is not acceptable. This is a call for all corporates to be truly secular – support every art form for what it is!

 \item Three areas of funding that can be explored at an individual level would be from the Non-Resident Indian (NRI) diaspora, High Net-worth Individuals (HNI), and crowd-funding.

\end{itemize}


\section*{Conclusion}

Role of patronage – past, present and future is a progressing timeline. The opportunities that exist to promote Karnatic music globally are phenomenal. There are challenges, of course. However, these are not insurmountable. With constant focus and dedication, there is a mighty lot that can be done in a relatively short period of time – thanks to digitalization.

In the past few years, Yoga and Āyurveda have become India’s contribution to humankind. I can visualize that the next big thing India will take to the globe will be our Classical Music and Classical Dance forms.

\theendnotes

