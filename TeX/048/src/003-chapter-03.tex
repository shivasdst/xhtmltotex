
\chapter{Tyāgarāja’s Philosophy and Rebuttal of Allegations}\label{chapter3}

\Authorline{Korada Subrahmanyam}

\begin{flushright}
\textit{korada11@gmail.com}
\end{flushright}


\section*{Abstract}

Indian culture stands on four pillars called \textit{puruṣārtha}–s (the purposes of human life), viz. \textit{dharma, artha}, \textit{kāma} and \textit{mokṣa}. \textit{Mokṣa} is arresting the cycle of births and deaths and merging the \textit{jīvātman} with \textit{paramātman}/\textit{parabrahman}. There are three ways of achieving \textit{mokṣa} – \textit{bhakti}, \textit{karma} and \textit{jñāna}. While \textit{karma} and \textit{jñāna} are not within the reach of common people, \textit{bhakti} is. \textit{Bhakti} can be traced to Vedic literature, and in India there are devotional musical compositions in almost all regional languages apart from Sanskrit. In fact, \textit{Gāndharva–veda} (\textit{saṅgīta}/Indian Music) is one of the 18 \textit{vidyā–sthāna}–s enumerated in \textit{smṛti}–s. It is an \textit{upaveda} of \textit{Sāmaveda}. In Karnatic \textit{saṅgīta} there were three great composers of \textit{kṛti}–s: Tyāgarāja, Śyāmaśāstri and Muttusvāmidīkṣita. This trio is called by the name – \textit{Trimūrti} or Vāggeyakāra–traya. Among them Tyāgarāja had had Nārada as his \textit{mārgadarśin} and his \textit{kṛti}–s reflect the essence of \textit{Nārada–bhakti–sūtra}–s. Tyāgarāja was a scholar in Veda, Vedāṅga, \textit{darśana}, \textit{purāṇa} and \textit{itihāsa}. He composed his \textit{kṛti}–s in Sanskrit and Telugu, which run into hundreds. Tyāgarāja composed in a language that is lucid – called \textit{drākṣā–pāka}. 

According to \textit{Nārada–bhakti–sūtra}–s, there is no bar on gender, caste etc for \textit{bhakti}.  \textit{Nāda–brahman} (\textit{Śabda–brahman}) is worshipped through \textit{Śāstrīya–saṅgīta} and that is an indirect path to \textit{mokṣa}. 

Recently some musicians have initiated a vicious campaign against Karnatic Music in general and Tyāgarāja in particular, without even a minimum knowledge of the philosophy of \textit{saṅgīta} and the nuances thereof. “Tyagaraja was a social commentator…compositions in which misogyny is obvious” (Krishna 2017) “I hold the view that lyrics in Carnatic music are abstract entities of sound” (Krishna 2018) “ …Tyagaraja that I find disturbing on social and ethical terms” (Krishna 2018) – are just some of the venomous comments that are made on Social Media and the Fourth Estate. Such expressions are highly objectionable, offending and provoking. This is simply to tarnish the image of a great \textit{vāggeyakāra}, who, without any desire for money/fame/mundane comforts, tried his best to cleanse the minds of common people expecting nothing in return. Very few people do we come across, who led such a simple and ascetic life. 

This Paper has two parts – the first part looks into the essential philosophy of Tyāgarāja that is embedded in his \textit{kṛti}–s whereas the second part deals with the rebuttal of allegations against the sage. All translations are my own.  


\section*{Introduction}

Indian culture stands on four pillars, viz. \textit{dharma, artha, kāma and mokṣa}. \textit{Dharma} is a phenomenon by which the universe is upheld. The term is pregnant with meaning and does not tolerate translation into any language. It is to be adapted and explained. \textit{Dharma} is the base of Indian culture or \textit{sanātana dharma} (an immutable \textit{dharma}). 

For the pursuit of \textit{mokṣa}, while educated and intellectuals embrace the path of \textit{karma} and \textit{jñāna}, common people prefer \textit{bhakti}. Between \textit{saguṇopāsanā} and \textit{niruguṇopāsanā} advocated by \textit{Veda}, it is the former that is connected with the path of \textit{bhakti} which is via worshipping \textit{brahman} in a form like Rāma, Kṛṣṇa, Śiva, Lakṣmī etc..

\textit{Saṅgīta} has a special place in Indian culture. It has always been given the exalted status of \textit{Gāndharvaveda} (an offshoot of \textit{Sāmaveda}) and is considered as one of the four \textit{upaveda}–s. \textit{Ṛgveda} combined with \textit{saṅgīta} became \textit{Sāmaveda} and even during \textit{yāga} it is the only Veda that is to be recited with \textit{sāmagāna} (rest of the Veda–s are to be recited with \textit{ekaśruti}\supskpt{\endnote{[ 1 ] Pāṇini rules (\textit{yajñakarmaṇyajapanyūṅkhasāmasu}, \textit{Aṣṭādhyāyī} 1–2–34) – except in the case of \textit{japa}, \textit{nyūṅkha}, and \textit{Sāmaveda}, all the \textit{mantra}–s in a \textit{yajñākarma} have to be recited in \textit{ekaśruti} (not \textit{traisvaryam} – \textit{udāttānudāttasvaritāḥ}).}}).

An old saying goes – even a child, an animal and a cobra enjoy \textit{saṅgīta}. Compared with poetry, \textit{saṅgīta} is much more appealing and the resonance of the sound would certainly have a soothing effect. And there is a philosophical side which is more important – the \textit{bhakti–saṅgīta} combination is employed to achieve \textit{mokṣa}. \textit{Nāda–brahman} (\textit{śabda–brahman}) is worshipped through \textit{śāstrīya–saṅgīta} that is considered as an indirect path to \textit{mokṣa}. Hence, since time immemorial, \textit{bhakti} and \textit{saṅgīta} are intertwined. 

In the Indian musical tradition, all composers propagated \textit{bhakti} and Karnatic musicis no exception to this\textit{.} The three great composers of \textit{kṛti–s} in Karnatic music popular as Trimūrti–s or \textit{Vāggeyakāra–traya}, are Tyāgarāja, Śyāmaśātriand Muttusvāmidīkṣita and all the three hail from Tanjavūr district of Tamilnāḍu. 

This paper discusses the personality of Tyāgarāja and the philosophy behind his \textit{kṛti}–s as well as the criticisms of the saint and his \textit{kṛti}–s by some present day ‘liberal’ musicians. Also intended is strong refutation of the allegations against the saint with proof / authority.

Tyāgarāja was a scholar in Veda, Vedāṅga, \textit{darśana}, \textit{purāṇa} and \textit{itihāsa}. He composed \textit{kṛti}–s in Sanskrit and Telugu in a lucid style called \textit{drākṣā–pāka}. He had had Nārada as his \textit{guru}/\textit{mārgadarśī} and his \textit{kṛti}–s reflect the essence of \textit{Nārada–bhakti–sūtra}–s.

\subsection*{Bhakti}

What is \textit{bhakti}? – this question is addressed by several \textit{ācārya}–s – 

“\textit{athāto bhaktiṁ vyākhyāsyāmaḥ}” (\textit{Nārada–bhakti–sūtra–1}).

Śaṅkarācārya (\textit{Viveka–cūḍāmaṇi}, 32 \& 33) explained \textit{bhakti} thus –

\begin{verse}
“\textit{mokṣa–sādhana–sāmagryāṁ}\\\textit{bhaktir eva garīyasī~।}\\\textit{sva–svarūpānusandhānaṁ}\\\textit{bhaktirityapare jaguḥ~।”}
\end{verse}

Among different things that are instrumental in achieving \textit{mokṣa}, \textit{bhakti} is the best. Uninterrupted meditation of \textit{parabrahman} is called \textit{bhakti}. Others said that identifying oneself with \textit{brahman} (\textit{aham brahmāsmi, brahmaivāham}) is called \textit{bhakti.}

Here in this context \textit{bhakti} is the one supported by \textit{śama} (control on mind), \textit{dama} (control on sense–organs) etc.

\textit{Nārada–bhakti–sūtra} (16)cites the opinion of Vedavyāsa that love in worship etc. is \textit{Bhakti} –

\begin{verse}
“\textit{pūjādiṣvanurāga iti pārāśaryaḥ}”
\end{verse}


\subsection*{No Bar:}

 \textit{Nārada–bhakti–sūtra} (72) clearly says that for embracing the path of \textit{bhakti}, there is no  restriction in terms of \textit{jāti} (\textit{strījāti / puruṣajāti}), \textit{vidyā} (educated and uneducated), \textit{rūpa} (beautiful or ugly), \textit{kula} (\textit{brāhmaṇa/kṣatriya/vaiśya/śūdra}), \textit{dhana} (wealthy or poor), \textit{kriyā} (one who performed \textit{yāga} etc. and the one who did nothing) etc. –

\begin{verse}
“\textit{nāsti teṣu jāti–vidyā–rūpa–kula–dhana–kriyādi–bhedaḥ}”
\end{verse}

Here is a list of \textit{bhakta}–s collected from \textit{Purāṇa}–s –

\begin{tabular}{|r|l|l|}
\hline
1. & Bhṛgu etc. & - \textit{brāhmaṇa}-s \\
\hline
2. & Viśvāmitra etc. & - \textit{kṣatriya}-s \\
\hline
3. & Tulādhāra etc. & - \textit{vaiśya}-s \\
\hline
4. & Sūta, Uddhava, Vidura etc. & - \textit{śūdra}-s \\
\hline
5. & Vālmīki etc. & - \textit{kirāta}-s \\
\hline
6. & Maitreyī , Caṇḍālā, Kātyāyanī, & - Ladies \\
\hline
 & Sulabhā, Madālasa, Śabarī, Gārgī etc. &  \\
\hline
\end{tabular}

\textit{Bhagavadgītā} (9–32) correlates the above concept of no–bar in \textit{bhakti} –

\begin{verse}
“\textit{māṁ hi pārtha vyapāśritya}\\\textit{ye’pi syuḥ papa–yonayaḥ~।}\\\textit{striyo vaiśyās tathā śūdrāḥ} \\\textit{te’pi yānti parāṁ gatim~।।}”
\end{verse}

Arjuna! those, who find me as a resort, the sinners, ladies, \textit{vaiśya}–s, \textit{śūdra}–s etc. also, will get \textit{mokṣa}.


\subsection*{Influence of Nārada and others on Tyāgarāja:}

As has already been stated, \textit{saṅgīta} is the best device to reach out to people, more so common people, whose knowledge of language and literature is very limited. Sadguru Tyāgarāja, therefore, chose to club \textit{bhakti} with \textit{saṅgīta} and drive the people towards the path of \textit{mokṣa}. While doing so he also put in a lot of effort in cleansing the minds of people in the society, i.e. tried that everyone in the society would get rid of the mental blemishes like lust, anger, hatred, jealousy, selfishness, lust for money, wealth, name and fame etc. 

If we take a close look at Tyāgarāja’s life, we certainly come to a conclusion that he was not only a great philosopher but also a proven social reformer, who could influence huge masses and has had great following even after hundreds of years. His journey was peaceful, inspiring, dedicated for a cause and his conduct was impeccable. Tyāgarāja did not exhibit any dislike against the fair sex or other castes and saw \textit{brahman} in every ‘thing’ as per the Upaniṣadic saying – “\textit{sarvaṁ khalv idaṁ brahma}” (certainly every ‘thing’ that is visible is \textit{brahman}).

In order to understand the real personality of the great saint Tyāgarāja, let us take some of his \textit{kṛti}–s and discuss the tenor and spirit of the same with authority.

The available literature regarding the birth and life of Tyāgarāja tells us that the Saint, when advised by his eldest brother, Pañcanadabrahmam, to approach the King, exhibit his talent in \textit{saṅgīta}, just like other scholars, receive his patronage and enjoy a luxury life, had rejected it outright – records say that it is at this point that Tyāgarāja composed the popular Kalyāṇi \textit{kṛti} –

\begin{verse}
\textit{“nidhi cāla sukhamā rāmuni san–}\\\textit{–nidhi sevā sukhamā nijamuga balku manasā}”
\end{verse}

Is it the treasure of wealth that is comfortable or is it the service in the \textit{sannidhi} of Lord Rāma?

Tyāgarāja used to live through \textit{bhikṣā} (begging) by going from house to house to feed his family. He thought that through \textit{bhikṣāṭana} one can kill his ego easily. Tyāgarāja composed a number of \textit{kṛti}–s, sang and became a \textit{kulapati}\supskpt{\endnote{[ 2 ] The term is currently used as a rough translation for a Chancellor or a Vice–Chancellor but used to be the term used for the head of a \textit{gurukula}.}} of a large group of disciples, who lived in his \textit{gurukula} free of cost. Such an uninterrupted \textit{guru–śiṣya–paramparā} continues even today. 

Let us take up some \textit{kṛti}–s of Tyāgarāja and fathom the cryptic message and the deep meaning: 

Tyāgarāja was a \textit{jīvan–mukta} and as such did not possess any vices. However, in order to inculcate a kind of introspection in the minds of people he authored a beautiful \textit{kṛti}. It is believed that Tyāgarāja, while on a pilgrimage, visited many holy places. When he visited Tirumala Tirupati for the \textit{darśana} of Śrīveṅkaṭeśvara and entered the \textit{garbhālaya}, it was curtained. Then this particular \textit{kṛti} had emanated from his mouth spontaneously (\textit{āśu–kavitā}) –

Veṅkaṭaramaṇa! why do you not draw (remove) the curtain called \textit{mātsarya} (jealousy) as it (in fact the \textit{ari–ṣaḍ–varga}) is driving away the four \textit{puruṣārtha}–s called \textit{dharma, artha, kāma} and \textit{mokṣa} –  it is disturbing like a fly at the time of having food happily – it is like the mind going to an unholy place at the time of \textit{hari–dhyāna} –  like a hungry fish getting entangled in a hook;  blocking a splendid light of a lamp;  a herd of deer getting entangled in a net due to ignorance –  my \textit{mada} and \textit{mātsarya} are, like curtains, blocking your \textit{darśana} –  

\begin{verse}
\textit{têra tīyagarādā (Gauḷipantu–Ādi)}\\\textit{“têra tīyagarādā loni}\\\textit{tirupati veṅkaṭaramaṇa maccaramanu}\\\textit{parama–puruṣa dharmādi mokṣamula}\\\textit{pāradolucunnadi nāloni}
\end{verse}

\begin{verse}
\textit{vāguramani têliyaga mṛgagaṇamula}\\\textit{vaccitagulu rīti nunnadi}\\\textit{vegamê nīmatamunanusariñcina}\\\textit{tyāgarājanuta madamatsaramanu}
\end{verse}

Just ten days before Tyāgarāja’s death, LordRāma appeared before him and assured ‘I will save you’ (give you \textit{kaivalya – mokṣa}). He unmistakably visualised Rāma installed on the hill as members of his retinue were vying with each other in serving him with flowers and fans in their hands.  Tyāgarāja was thrilled by this divine vision, he struggled for words to express his feelings and there were tears in his eyes due to joy. That is the intensity of \textit{bhakti} that Tyāgarāja had in Śrīrāma.

The following \textit{kṛti} proves in no uncertain terms Tyāgarāja’s divinity. He rendered this \textit{kṛti}, which emanated spontaneously, with Karuṇa \textit{rasa} in Śahāna \textit{rāga} –

\begin{verse}
\textit{giripai nêlakonna rāmuni} 
\end{verse}

\begin{flushright}
(Śahāna \textit{rāga,} Ādi)
\end{flushright}

\begin{verse}
\textit{“giripai nêlakônna rāmuni guritappaka kaṇṭi} \\\textit{parivārulu viri suraṭulace nilabaḍi visarucu gôsarucu sevimpaga}\\\textit{pulakānkituḍai yānandāśruvula nimpucu māṭalāḍavalenani}\\\textit{kaluvarinca gani padipūṭalapai gācedananu tyāgarāja vinutuni”}
\end{verse}


\subsection*{\textit{Brahmajijñasā:–}}

\textit{‘Athāto brahmajijñāsā (Brahmasūtra}, 1–1–1) is the first \textit{sūtra} of Vedāntadarśana – one should do \textit{brahma–jijñāsā} after being qualified – there are four qualifications (\textit{Śāṅkara–bhāṣya} on \textit{Brahmasūtra} 1.1.1) –

\begin{enumerate}
\item \textit{Nityānitya–vastu–viveka} – having the wisdom of the things that are eternal and perishable;
 
\item \textit{Ihāmutrārtha–phala–bhoga–virāga} – detachment in the comforts of this and the other worlds;
 
\item \textit{Śama–damādi–sādhana–sampat} – ‘\textit{śānto dānta uparatas titikṣuḥ samāhito bhūtvātmanye–vātmānaṁ paśyati’ (Bṛhadāraṇyakopaniṣad}  4.4.23) – control upon mind and upon sense organs, keeping the mind away from mundane matters, sustaining difficulties, being firm on \textit{ātma–tattva} and confidence in \textit{guru, śāstra} etc. – one should possess all these instrumental characteristics;
 
\item \textit{Mumukṣutva} – utmost desire in attaining \textit{mokṣa.}
 
\end{enumerate}

The above four qualifications are prescribed by Śaṅkarācārya for one who wants to start \textit{brahma–jijñāsā} (a desire to know \textit{brahman}).

Let us take up some \textit{kṛti}–s of Tyāgarāja and discuss as to how he tried to deliver the deep Vedāntic message to common people –

\textit{Jñānamôsagarādā} (Pūrvikalyāṇi – Rūpaka) –

\begin{verse}
\textit{jñānamôsagarādā garuḍagamana vādā} \\\textit{nī nāmamuce nāmadi nirmalamainadi} \\\textit{paramātmuḍu jīvātmuḍu padunālugu lokamulu}\\\textit{nara–kinnara–kimpuruṣulu nāradādi–munulu}\\\textit{paripūrṇa niṣkaḷaṅka niravadhi–sukha–dāyaka}\\\textit{vara tyāgarājārcita vāramu tānane}
\end{verse}

In this \textit{kṛti,} Tyāgarāja expresses his desire for \textit{mokṣa} (\textit{mumukṣutva}) – O! Viṣṇu! my mind is free of any blemishes by meditating your name– why do you not provide me with \textit{jñāna}?

Śrī Kṛṣṇa in \textit{Bhagavadgītā} (9.22) clearly says that he would take care of those people, who without any other avocation, keep on meditating him –

\begin{verse}
“\textit{ananyāś cintayanto māṁ}\\\textit{ye janāḥ paryupāsate~।}\\\textit{teṣāṁ nityābhiyuktānāṁ}\\\textit{yoga–kṣemaṁ vahāmy aham}~।।”
\end{verse}

“\textit{tasyāḥ jñānam eva sādhanam ity eke}” – is the \textit{Nārada–bhakti–sūtra} (28). Then Tyāgarāja says (in the aforementioned \textit{kṛti}) do not argue with me and why do you not provide me with \textit{jñāna}? A \textit{yogin} should not get involved in argument with regard to \textit{Bhagavān} or his \textit{bhakta}–s as such a \textit{vāda} is likely to escalate and also as there is no limit to such a \textit{vāda} –

\begin{verse}
\textit{“vādo nāvalambyaḥ”}\\\textit{“bāhulyāvakāśatvāt aniyatatvācca”}
\end{verse}

\begin{flushright}
(\textit{Nārada–bhakti–sūtra} 74, 75)
\end{flushright}

The above attitude falls under “\textit{śama–damādi–sampat}” as well as “\textit{mumukṣutva}”.

Since he declares that he has a mind that is cleansed we can believe that Tyāgarāja became a \textit{yogin} as per Yogānuśāsana of Patañjali (1.1) –

\begin{verse}
“\textit{yogaḥ citta–vṛtti–nirodhaḥ}”
\end{verse}

\textit{Paramātmuḍu jīvātmuḍu … vāramu tānane} etc. is nothing but the essence of \textit{Upaniṣad}–s –

\begin{verse}
“\textit{sadeva somya idam agra āsīd ekamevādvitīyaṁ brahma}”
\end{verse}

\begin{flushright}
(\textit{Chāndogyopaniṣad} 6.2.1)
\end{flushright}

\begin{verse}
\textit{ātmā vā idameka evāgra āsīt, nānyat kiñcana miṣat, sa aikṣata lokānnu}\\\textit{sṛjā iti, sa imān lokān asṛjata} (\textit{Aitareyopaniṣad} 1.1.1)\\\textit{idam sarvam yad ayam ātmā} (\textit{Bṛhadāraṇyakopaniṣad} 2.4.6)\\\textit{aitadātmyam idaṁ sarvaṁ, sa ātmā} (\textit{Chāndogyopaniṣad} 6.14.3)\\\textit{jīvenātmanānupraviśya nāma–rūpe vyākaravāṇi} (\textit{Chāndogyopaniṣad} 6.3.2)\\\textit{ātmata evedam sarvam} (\textit{Chāndogyopaniṣad} 7.26.1)
\end{verse}


\subsection*{\textit{Mahāvākyārtha:–}}

Tyāgarāja, in another \textit{kṛti}, directly refers to the famous \textit{mahāvākya}, i.e. “\textit{tattvamasi}”, and says that the \textit{vākyārtha} of “\textit{tattvamasi} is Rāma”, which is the essence of all Veda–s and Śāstra–s, is very difficult to understand – it is possible if and only if one gets the \textit{sattva–guṇa} by killing \textit{rajo–guṇa} and \textit{tamo–guṇa}, which keep on generating problems –

\begin{verse}
“\textit{tattva mêruga taramā (Garuḍadhvani – Rūpāka)}
\end{verse}

\begin{verse}
“\textit{tattvamêruga taramā para}\\\textit{tattvamasi yanu vākyārthamu}\\\textit{rāmā nīvanu para}\\\textit{tāmasa–rājasa–guṇamula tannukôḷḷa bodayā}\\\textit{rāmabhaktuḍaina tyāgarājavinuta vedaśāstra}”
\end{verse}

In sixth \textit{adhyāya} of \textit{Chāndogyopaniṣad} (6.14.3) the following sentence is found –

\begin{verse}
“\textit{aitadātmyam idam sarvam sa ātmā tattvam asi śvetaketo}”
\end{verse}

Whatever is perceived is \textit{brahman} only, O! Śvetaketu! You are \textit{brahman} only.

“\textit{Tattvamasi}” is repeated in that chapter and it is called a \textit{mahāvākya}, as it renders the purport of that chapter.

This \textit{mahāvākya} proposed the concept of Advaita – non–duality of \textit{jīvātman} and \textit{paramātman}.

Tyāgarāja, in the above \textit{kṛti} asserts that – in fact, \textit{tattvamasi}, the \textit{mahāvākya} or the term \textit{para–tattva}, both mean Rāmaonly. If one wants to know that, he should kill both \textit{tāmasa} and \textit{rājasa}, the \textit{guṇa}–s   which cause perennial worries to a person and should fill his mind with \textit{sattva–guṇa} – “\textit{sattvāt saṁjāyate jñānam}” (\textit{Bhagavadgītā} 14.17).


\subsection*{\textit{Dvaita and Advaita:–}}

In a similar vein Tyāgarāja rakes up a question as to which, i.e. Dvaita or Advaita, is comfortable –

\begin{verse}
\textit{dvaitamu sukhamā} (Rītigauḷa – Ādi)\\ “\textit{dvaitamu sukhamā, advaitamu sukhamā}\\\textit{caitanyamā vinu sarvasākṣī, vi–}\\\textit{stāramugānu dêlpumu nāto}\\\textit{gagana–pavana–tapana–bhuvanādyavanilo nagadharāja–śivendrādi–surulalo}\\\textit{bhagavadbhakta–varāgresarulalo bāga ramince tyāgarājārcita}”
\end{verse}

Dichotomy or mutual difference (\textit{bheda}) between two \textit{jīva}–s or between \textit{jīva} and \textit{īśvara} is called Dvaita. Advaita is non–dualism or non–difference between \textit{jīvātman} and \textit{paramātman}. 

In the first line, having put the question as to whether it is Dvaita or Advaita that is comfortable, Tyāgarāja says ‘\textit{caitanyamā vinu sarvasākśī}’ – Dvaita and Advaita denote \textit{bheda} and \textit{abheda} in the form of \textit{śuddha–caitanya – ātman} is in the form of \textit{śuddha–caitanya}. It is also called \textit{sarva–sākṣin}. Here is \textit{Śvetāśvataropaniṣad} (6.11) –

\begin{verse}
“\textit{eko devaḥ sarvabhūteṣu gūḍhaḥ sarvavyāpī}\\\textit{sarvabhūtāntarātmā karmādhyakṣaḥ sarva–}\\\textit{bhūtādhivāsaḥ sākṣī cetā kevalo nirguṇaśca}”
\end{verse}


\subsection*{\textit{Saguṇopāsanā}:–}

The \textit{Pāṇini–sūtra} “\textit{devāttal}” (\textit{Aṣṭādhyāyī} 5.4.27) – “\textit{tal–pratyaya}” – rules it as “\textit{svārthika}” (in the sense of \textit{prakṛti only}) – so \textit{deva} is  \textit{devatā} – whether it is masculine, i.e. Śiva, Rāma etc. or feminine, i.e. Sītā\textit{,} Pārvatī etc. the term \textit{devatā} can be used. 

\textit{Ṛgvedasaṁhitā} (1.164.46) says that it is a single “\textit{sat}” (\textit{brahman}) that is described differently by different scholars –

\begin{verse}
“\textit{indram mitram varuṇam agnim āhuḥ}\\\textit{atho divyaḥ sa suparṇo garutmān}~।\\\textit{ekam sad viprā bahudhā vadanti}\\\textit{agnim yamam mātariśvānamāhuḥ}~।।”
\end{verse}

Tyāgarāja, while keeping \textit{saguṇopāsanā} in mind asks Rāma – which is your place of abode? Where all are you manifest? You are not seen even when searched carefully – are you there in \textit{strīdevatā}–s like Sītā\textit{,} Gaurī\textit{,} Sarasvatī etc. or in the five elements namely earth, water, fire, air and ether, or in the different worlds? O Rāma, who is worshipped by Tyāgarāja, are you manifest in Śiva, Mādhava and Brahmā?  – this also refers to “\textit{sarvaṁ khalv idaṁ brahma}” (\textit{Chāndogyopaniṣad} 3.14.1) – 

\textit{Etāvunarā (}Kalyāṇi– Ādi\textit{)}

\begin{verse}
\textit{etāvunarā nilakaḍa nīku}\\\textit{ênci jūḍa nagapaḍavu}\\\textit{sītāgaurivāgīśvari yanu strīrūpamulandā govinda}\\\textit{bhūkamalārkānila nabhamulanda lokakoṭulanda}\\\textit{śrīkaruḍagu tyāgarājakarārcita}\\\textit{śivamādhavabrahmādula yandā}
\end{verse}

This concept is borrowed from \textit{Śvetāśvataropaniṣad} (4.3) –

\begin{verse}
\textit{tvaṁ strītvaṁ pumān asi tvaṁ kumāra uta vā kumārī~।}\\\textit{tvaṁ jīrṇo daṇḍena vañcasi tvaṁ jāto bhavasi viśvatomukhaḥ}~।।
\end{verse}


\subsection*{\textit{Kutsitasevā:–}}

Tyāgarāja, just like Potana (the Telugu poet of 15th century, who translated \textit{Bhāgavata}), refused to flatter the wicked people, who were full of ego due to wealth, because, Tyāgarāja says “it is you (Rāma\textit{/daiva}), who provides wealth, grains etc. and you are \textit{dharma} personified. O Rāma, praised by Tyāgarāja, the poets who dedicate their literary attainments to the wicked and unworthy in a \textit{sabhā} are mean, and I shall not admire them”. 

\textit{durmārgacarādhamulanu (}Rañjani – Rūpaka\textit{)}

\begin{verse}
\textit{“durmārgacarādhamulanu dôra nīvanajālarā}\\\textit{dharmātmaka dhana–dhānyamu daivamu nīvai yunḍaga}\\\textit{palukuboṭini sabhalona patitamānavulakôsage}\\\textit{khalulanêccaṭa bôgaḍani śrīkara tyāgarājavinuta”}
\end{verse}

This \textit{kṛti} explains the purport of the \textit{Nārada–bhakti–sūtra} (43) – “\textit{dussaṅgaḥ sarvathaiva tyājyaḥ”.}


\subsection*{\textit{Śānti} and \textit{Dānti}:–}

If one has control over the mind, he does not require \textit{mantra} and \textit{tantra}; if one thinks that body is not \textit{ātman,} he need not perform \textit{tapas}; if one thinks that everything is Rāma, he does not have the difference of \textit{āśrama}–s such as \textit{brahmacarya}, \textit{gārhasthya} etc. If one considers the entire world as nothing but \textit{māyā}, he is unaffected by the charm of women and if one, throughout his life, is not involved in any vices, there is no fear of the cycle of birth and death – says Tyāgarāja in this \textit{kṛti}:

\textit{Manasu svādhīnamaina} (Śaṅkarābharaṇa– Miśra chāpu)

\begin{verse}
\textit{“manasu svādhīnamaina yā ghanuniki}\\\textit{mari mantratantramulela}\\\textit{tanuvu tānugādani yêncuvāniki}\\\textit{tapasu ceyanêla daśarathabāla}\\\textit{anni nīvanucu yêncinavāniki yāśramabhedamulela}\\\textit{kannugaṭṭu māyalani yêncuvāniki kāntala bhramalela daśarathabāla}\\\textit{ājanmamu durviṣayarahituniki gatāgatamika yela}\\\textit{rājarājeśa niranjana nirupama rājavadana tyāgarājavinuta “}
\end{verse}

It is a great art that Tyāgarāja does possess: however profound be the philosophy, he encodes it in common man’s language. The above \textit{kṛti} is one such example. 

In \textit{Bhagavadgītā} (2–55, 56) Śrīkṛṣṇa calls such a person a \textit{sthitaprajña} –

\begin{verse}
“\textit{prajahāti yadā kāmān sarvān pārtha manogatān~।}\\\textit{ātmany evātmanā tuṣṭas sthita–prajñas tadocyate}~।।
\end{verse}

\begin{verse}
“\textit{duḥkheṣv anudvigna–manāḥ sukheṣu vigata–spṛhaḥ~।}\\\textit{vītarāga–bhaya–krodhas sthitā–dhīr munir ucyate}”  ।।
\end{verse}

“\textit{yogaś citta–vṛtti–nirodhaḥ}” (\textit{Yogānuśāsana} 1.1) is also very much applicable here. In the \textit{anupallavi}, Tyāgarāja says – “\textit{tanuvu tānu gādani yencuvāniki tapasu ceyanêla}” – this line gives the essence of the \textit{mantra} of \textit{Kaṭhopaniṣad} (2.5.6–7) –

\begin{verse}
“\textit{hanta ta imaṁ pravakṣyāmi}\\\textit{guhyaṁ brahma sanātanam~।}\\\textit{yathā ca maraṇaṁ prāpya}\\\textit{ātmā bhavati gautama~।।}\\\textit{yonim anye prapadyante}\\\textit{śarīratvāya dehinaḥ~।}\\\textit{sthāṇumanye’nusaṁyanti}\\\textit{yathākarma yathāśrutam}~।।”
\end{verse}

The above message is given in \textit{Bhagavadgītā} (2.22) also –

\begin{verse}
\textit{“vāsāmsi jīrṇāni yathā vihāya}\\\textit{navāni gṛhṇāti naro’parāṇi~।}\\\textit{tathā śarīrāṇi vihāya jīrṇāny–}\\\textit{anyāni saṁyāti navāni dehī  ।।”}
\end{verse}

Then the \textit{caraṇa} “\textit{anni nīvanucu êncinavāniki yāśramabhedamulela}” reflects the same idea as that of several Upaniṣadic messages (here \textit{nīvu} means \textit{brahman} in the form of Rāma) –

\begin{verse}
“\textit{sarvam khalv idam brahma}” (\textit{Chāndogyopaniṣad} 3.14.1)\\ “\textit{yatra tvasya sarvamātmaivābhūt kena kam paśyet}” (\textit{Bṛhadāraṇyakopaniṣad} 4.5.15)\\ “\textit{omkāra evedam sarvam}” (\textit{Chāndogyopaniṣad} 2.23.3)\\ “\textit{idam sarvam yadayamātmā}” (\textit{Bṛhadāraṇyakopaniṣad} 2.4.6)
\end{verse}


\subsection*{\textit{Dhyāna}:–}

\textit{Bṛhadāraṇyakopaniṣad} (2.4.5) prescribes the following –

\begin{verse}
“\textit{ātmā vā are draṣṭavyaḥ śrotavyaḥ mantavyaḥ nididhyāsitavyaḥ}”
\end{verse}

\textit{Ātman} should be perceived, to be heard, to be pondered over and to be meditated upon.

 \textit{Nididhyāsana} of Vedānta is \textit{dhyāna} in Yoga.

\textit{Yoga} consists of eight \textit{aṅga}–s (limbs) –

\begin{myquote}
“\textit{yama–niyamāsana–prāṇāyāma–pratyāhāra–dhāraṇā–dhyāna–samādhayo’ṣṭāv aṅgāni}”
\end{myquote}

\begin{flushright}
(\textit{Yogānuśāsana} 2.29)
\end{flushright}

Among them \textit{dhyāna} is the penultimate device –

\begin{verse}
“\textit{tatra pratyayaikatānatā dhyānam}” (\textit{Yogānuśāsana}, 3–2)
\end{verse}

While having \textit{dhāraṇā} on a point, if one, while arresting other things, concentrates on the same with the stream of similar cognitions it is called \textit{dhyāna}.

The earlier six \textit{aṅga}–s help in attaining \textit{dhyāna}, whereas the latter is the cause of the last \textit{aṅga}, i.e. \textit{samādhi} –

\begin{verse}
“\textit{tad–rūpa–pratyayaikāgrya–santatiś cānya–nispṛhā~।}\\\textit{taddhyānam prathamair aṅgaiḥ ṣaḍbhir niṣpadyate nṛpa}~।।”
\end{verse}

\begin{flushright}
(\textit{Viṣṇupurāṇa}, 6.7.89)
\end{flushright}

\textit{Bhagavadgītā} (6.11, 12) clearly explains the concept of \textit{dhyāna} –

\begin{verse}
“\textit{śucau deśe pratiṣṭhāpya}\\\textit{sthiram āsanam ātmanaḥ~।}\\\textit{nātyucchritam nātinīcam}\\\textit{cailājina–kuśottaram~।।}
\end{verse}

\begin{verse}
\textit{tatraikāgram manaḥ kṛtvā}\\\textit{yata–cittendriya–kriyaḥ~।}\\\textit{upaviśyā same yuñjyāt}\\\textit{yogam ātma–viśuddhaye}~।।”
\end{verse}

\textit{Dhyāna} is the most difficult of all the \textit{aṅga}–s as is explained by Arjuna (\textit{Bhagavadgītā} 6.34) – \textit{vāyu}  (air) cannot be controlled and mind is just the same – it is worrisome, strong, hard and fickle –

\begin{verse}
“\textit{cañcalam hi manaḥ kṛṣṇa}\\\textit{pramāthi balavad dṛḍham~।}\\\textit{tasyāham nigrahaṁ manye}\\\textit{vāyor iva suduṣkaram}~।।
\end{verse}

Tyāgarāja, who is a \textit{yogin}, asserts that \textit{dhyāna} itself is \textit{Gaṅgāsnāna} –

\begin{verse}
\textit{Dhyāname varamaina} (Dhanyāsi – Ādi)\\ “\textit{dhyāname varamaina gaṅgā–snāname manasā}\\\textit{vānanīṭa munuga munuga loni} \\\textit{vañcana drohamanu kara bonā}\\\textit{paradhana–nārīmaṇulanu dūri paranindala parahiṁsala mīri}\\\textit{dharanu vêlayu śrīrāmuni gori tyāgarāja dêlusukônna rāma}”
\end{verse}

Meditating on Rāma (\textit{Rāma–dhyāna}) is itself \textit{Gaṅgā–snāna}. By bathing, time and again, in the rainy water, one cannot get rid of vices such as deceit and treachery. 

Tyāgarāja also preaches that one should stop desiring others’ wealth as well as other women and desist from blaming and harming others; instead one should long for the grace of Rāma and do \textit{Rāma–dhyāna}.


\subsection*{\textit{Nādotpatti}:–}

Further, Śikṣā, Vyākarana, Pūrvamīmāṁsā and Gāndharvaveda\textit{,} in line with the philosophy expounded above, clearly explain the process of speech generation that is undoubtedly laced with philosophy: it is \textit{brahman} itself which emanates in the form of acoustic sounds; it is the \textit{Nādabrahman}. \textit{Saṅgītaratnākara} (1.3.3 and 1.3.4) has expounded the concept –

\begin{verse}
\textit{“ātmā vivakṣamāno’yam}\\\textit{manaḥ prerayate manaḥ}~।\\\textit{dehastham vahnim āhanti}\\\textit{sa prerayati mārutam}~।।\\\textit{brahma–granthi–sthitaḥ so’tha}\\\textit{kramād ūrdhva–pathe caran}~।\\\textit{nābhi–hṛt–kaṇṭha–mūrdhāsye–}\\\textit{ṣvāvirbhāvayati dhvanim}~।।”
\end{verse}

In fact, according to Vyākaraṇadarśana (\textit{Vākyapadīya} 1.44) there are two kinds of \textit{śabda}–s accepted as the cause of understanding – \textit{vaikharī} and \textit{madhyamā}. The latter is popular as \textit{sphoṭa}. The former is the acoustic sound that emanates from the mouth and it is the cause of \textit{sphoṭa–śabda}, which is in mind and helps in understanding. Both are identical.

\begin{verse}
“\textit{dvāv upādāna–śabdeṣu}\\\textit{śabdau śabda–vido viduḥ}।\\\textit{eko nimittam śabdānām}\\\textit{aparo’rthe prayujyate}।।”
\end{verse}

\textit{Śabda} is of four types: \textit{parā} (that is there on \textit{Mūlacakra}), \textit{paśyantī} (at the navel), \textit{madhyamā} or \textit{sphoṭa} (in the mind) and \textit{vaikhar}ī (that is the acoustic sound emanating from the body) –

\begin{verse}
“\textit{parā vāṅ–mūla–cakra–sthā}\\\textit{paśyantī nābhi–saṁsthitā~।}\\\textit{hṛdi–sthā madhyamā jñeyā}\\\textit{vaikharī kaṇṭha–deśagā}~।।” 
\end{verse}

\textit{Parā} and \textit{paśyantī} are available to \textit{yogin}–s only. \textit{Sphoṭa} and \textit{vaikharī} are for ordinary humans. The production of \textit{śabda} is clearly stated in \textit{Pāṇinīyaśikṣā} (6, 7, 9) –

\begin{verse}
“\textit{ātmā buddhyā sametyārthān mano yuṅkte vivakṣayā~।}\\\textit{manaḥ kāyāgnim āhanti sa prerayati mārutam~।।}\\\textit{mārutastūrasi caran mandraṁ janayati svaram~।}\\\textit{so’dīrṇo mūrdhnyabhihato vaktram āpadya mārutaḥ~।}\\\textit{varṇān janayate}...~।।”
\end{verse}

The \textit{antaraṅga} gets the things associated with \textit{buddhi} and puts \textit{manas} on the job of a desire to speak. The mind in turn would hit the \textit{jaṭharāgni} (digestive fire) and the latter pushes the air. The air, while moving in the chest generates a feeble sound which goes up, hits the palate and touches the speech organs to produce \textit{varṇa}–s/\textit{akṣara}–s.

Pāṇini (\textit{Pāṇinīyaśikṣā} 11, 12; also \textit{Nāradīyaśikṣā} 8) explains the birth of \textit{Niṣāda} etc. –

\begin{verse}
“\textit{udāttaś cānudāttaś ca svaritaś ca svarās trayaḥ~।}\\\textit{udātte niṣāda–gāndhārāv anudātta ṛṣabha–dhaivatau~।}\\\textit{svaritaprabhavā hyete ṣaḍja–madhyama–pañcamāḥ}~।।
\end{verse}

There are three main \textit{svara}–s (accents) called \textit{udātta}, \textit{anudātta} and \textit{svarita}. \textit{Niṣāda} and \textit{gāndhāra} are born out of \textit{udātta}. \textit{Ṛṣabha} and \textit{dhaivata} are from \textit{anudātta} while \textit{svarita} is the base of \textit{ṣaḍja}, \textit{madhyama} and \textit{pañcama}.

In Indian tradition \textit{akṣara} is considered as \textit{brahman}. \textit{Brahman} itself emanates from the body in the form of an \textit{akṣara}:

\begin{verse}
“\textit{akṣaram na kṣaram vidyāt”}\\\textit{“varṇam vāhuḥ pūrvasūtre”}\\\textit{“varṇa–jñānam vāg–viṣayo yatra ca brahma vartate}”
\end{verse}

\begin{flushright}
(\textit{Vārtika}–s, \textit{Pratyāhārāhnika, Mahābhāṣya})
\end{flushright}

\begin{myquote}
\textit{“na kṣīyate na kṣaratīti vā akṣaram… so’yamakṣarasamāmnāyaḥ vāksamāmnāyaḥ puṣpitaḥ phalitaḥ candratārakavat pratimaṇḍitaḥ veditavyo brahmarāśiḥ sarvaveda– puṇyaphalāvāptiśca asya jñāne bhavati mātāpitarau cāsya svarge loke mahīyete”}
\end{myquote}

\begin{flushright}
(\textit{Pratyāhārāhnika, Mahābhāṣya})
\end{flushright}

The purport is that \textit{vāk/śabda} is \textit{brahman,} and one who worships \textit{vāk} would attain the \textit{puṇya} of the study of all the Veda–s.

\textit{Muṇḍakopaniṣad} says that the universe emerges from \textit{akṣara} (1.1.7) –

\begin{verse}
“\textit{akṣarāt sambhavatīha viśvam”}
\end{verse}

Bhartṛhari at the outset of \textit{Brahmakāṇḍa} of \textit{Vākyapadīya} (1.1) declares that the universe is emerged from \textit{śabda–brahman} –

\begin{verse}
“\textit{anādi–nidhanam brahma śabda–tattvam yad–akṣaram~।}\\\textit{vivartate’rtha–bhāvena prakriyā jagato yataḥ} ”।।
\end{verse}

Bhartṛhari also clarifies (\textit{Vākyapadīya, Brahmakāṇḍa}, 1.119) that the seven popular \textit{svara}–s, viz. \textit{niṣāda, ṛṣabha, gāndhāra, ṣaḍja, madhyama, dhaivata} and \textit{pañcama} are to be identified through \textit{śabda} only –

\begin{verse}
“\textit{ṣaḍjādi–bhedaḥ śabdena vyākhyāto rūpyate yataḥ~।}\\\textit{tasmād artha–vidhāḥ sarvāḥ śabda–mātrāsu niśritāḥ”}~।।
\end{verse}

According to Veda the universe is the transformation of \textit{śabda} and in the beginning it is from Veda only that this world got the transition – explains Bhartṛhari (\textit{Vākyapadīya, Brahmakāṇḍa}, 1.120) –

\begin{verse}
“\textit{śabdasya pariṇāmo’yam ity āmnāya–vido viduḥ~।}\\\textit{chandobhya eva prathamam etad viśvaṁ vyavartata”}~।।
\end{verse}

Keeping all the above concepts in mind Tyāgarājacomposed the following \textit{kṛti} –

\begin{verse}
\textit{mokṣamu galadā bhuvilo} (Sāramati–Ādi)
\end{verse}

\begin{verse}
\textit{mokṣamu galadā bhuvilo jīvanmuktulu gāni vāralaku}\\\textit{sākṣātkāra nī sadbhakti sangīta jñāna–vihīnulaku}\\\textit{prāṇānala–saṁyogamuvalla praṇava–nādamu sapta–svaramulai baraga}\\\textit{vīṇā–vādana–loluḍau śiva–manovidhamêrugaru tyāgarāja–vinuta}”
\end{verse}

Oh Lord praised by Tyāgarāja! In this world, is \textit{mokṣa} attainable for those who have not become \textit{jīvanmukta}–s and who are bereft of knowledge of music combined with true devotion towards you? The \textit{Praṇava–nāda} is produced through the combination of \textit{prāṇa–vāyu} (air) and \textit{anala} (fire) and the same manifests as \textit{sapta–svara}–s (Cf. \textit{śobhillu sapta svara}). People do not understand the blissful state of Lord Śiva, who is a great connoisseur of \textit{vīṇā–vādana}.


\section*{Criticism of Tyāgarāja and its Refutation:}

A reputed musician Sri TM Krishna (2017) accuses on many accounts the great saint Tyāgarāja, who was leading nothing but a purely ascetic life – “... there are compositions in which misogyny is quite obvious.... In compositions like ‘\textit{Menu joochi Mosa Bokave}’, ‘\textit{Dudukugala}’, ‘\textit{Enta muddo}’ and ‘\textit{Entha nerchina}’ he follows the classical norm of objectifying women, implying the sexual vulnerability of men. She is always the vice, the seductress who will enslave the man, therefore he needs to be ever watchful. Women and their own self–worth, rarely of any consequence… Tyagaraja’s position on women and caste was undoubtedly shaped by his times….”.

This reputed musician, it seems, has not digested the \textit{āśrama–vyavasthā} nor has he tried to understand the nuances of language (\textit{dhvani}). In order to attain \textit{mokṣa}, one can directly take \textit{Saṁnyāsa} after \textit{brahmacarya} or one may embrace \textit{gārhasthya}, then onto \textit{vānaprastha} and finally \textit{Saṁnyāsa}. Tyāgarāja followed the latter, i.e. got married, had a female child and got her married. Actually, the saint lost his first wife and got married for the second time.

\textit{Taittirīyasaṁhitā} (6.1.8.5) says –

\begin{verse}
“\textit{ardho vā eṣa ātmano yatpatnī}”
\end{verse}

\textit{Patnī} (wife) is half of the husband, i.e. ‘\textit{ardhāṅgī}’.

\textit{Śatapathabrāhmaṇa (}5.2.1.10; 8.7.2.3\textit{)} also rules the same \textit{–}

\begin{verse}
\textit{}“\textit{ardho vā eṣa ātmano yajjāyā}”
\end{verse}

The scholar saint, Tyāgarāja did not only follow the Vedic dictum \textit{in toto} besides treating his mother as \textit{devatā,} as can be seen in his \textit{kṛti}–s like ‘\textit{Sītamma māyamma}’ (\textit{mātṛdevo bhava – Taittirīyopaniṣad} – \textit{Śikṣā}). This is not possible had the saint’s mind was full of misogyny.

Tyāgarāja indeed had great respect for women – so much so that, in one \textit{kṛti,} he tells Rāma that it is due to ‘our’ Jānakī that he attained name, fame etc.:  “Rāma! You became an emperor after Sītā became your wife; you achieved the fame as the killer of Rāvaṇa.  Having gone to forest along with you, while following your orders, (when Rāvaṇa kidnapped her), she kept her \textit{māyārūpa} with him while keeping her \textit{nijarūpa} with Agni. Having gone to Laṅkā with Rāvaṇa and staying there, under Aśoka tree, she became angry due to Rāvaṇa’s words. Although she had the capacity to kill Rāvaṇa with a single look she desisted in order to see that you should get the fame” –  

\begin{verse}
\textit{mā jānaki cêṭṭa baṭṭaga maharājavaitivi} 
\end{verse}

\begin{flushright}
(Kambhoji–Deśādi)
\end{flushright}

\begin{verse}
\textit{“mā jānaki cêṭṭa baṭṭaga maharāja vaitivi}\\\textit{rāja rājavara! rājīvākṣa vinu rāvaṇāriyani rājillu kīrtiyu}\\\textit{kānakegiyājña mīraka māyākāramunici śikhi cêntane yunḍi}\\\textit{dānavuni vêṇṭane cani yaśokatarumūlanuṇḍi}\\\textit{vāni māṭalaku kopagiñci kaṇṭa vadhiyiṇcakane yuṇḍi} \\\textit{śrīnāyaka yaśamu nīke kalga jeyaledā tyāgarāja paripāla}
\end{verse}

The expression \textit{‘mā jānaki’} / ‘our Sīta’ is enough to show the reverence and the sentiments of the saint for women. In the above \textit{kṛti,} Tyāgarāja worshipped Sītādevī whereas in other \textit{kṛti}–s he is allegedly misogynistic! 

The jibe that ‘he follows the classical norm of objectifying women’ is highly objectionable. No sane person who has gone through Tyagaraja’s literature would pass such a sweeping remark. 

Another example which have seen earlier is – “\textit{etāvunarā nilakaḍa nīku ….. sītāgaurīvāgīśvari yanu strīrūpamulamdā} ” – if Tyāgarāja was against women or wanted to objectify women, he would have not used this kind of respectful words.

Organized prostitution had been there in that society and some youth used to frequent to prostitutes’ houses. In order to see that they pay attention to their \textit{dharma}, i.e. be content with wife, Tyāgarāja composed ‘\textit{menu jūci mosabokave}’, ‘\textit{ênta muddo}’, ‘\textit{ênta nercina}’ etc. These \textit{kṛti}–s should be seen through the prism of \textit{sanātana dharma} and social reform, rather than through cheap mundane classical binoculars. Of course, ‘sexual vulnerability of men’ (which even a great sage like Viśvāmitra faced) has been there since time immemorial. It is very difficult to control the sense organs, and more specially the mind (\textit{jahi śatruṁ mahābāho kāmarūpam durāsadam, Bhagavadgītā,} 3.43).

Even Śaṅkarācārya, Bhartṛhari and others preached the same: not to get seduced. What one should understand here is that Tyāgarāja is addressing the weakness of mind of youth and not propagating any hatred against women.

Let us understand the real purport of the \textit{kṛti}–s that Krishna quotes in support of his allegations.

\begin{enumerate}
\item 
 \textit{1. menu jūci mosabokave manasā} (Sarasāṅgi – Deśādi)

\begin{verse}
\textit{(strīla) “menu jūci mosabokave manasā}\\\textit{lonijāḍa līlāgu gādā}\\\textit{hīnamaina mala–mūtra–raktamula}\\\textit{kiravaṁcu māyāmayamaina cāna”}
\end{verse}

 Tyāgarāja, through this \textit{kṛti,} wants to impress upon the youth not to get attracted towards the young ladies by their physical appearance because just like any other ‘body’, theirs too is filled with disposable wastage. This message is simply to maintain the essential moral values and conjugal relations in the society, not to preach hatred against women. Verses with similar import are plenty in other books, such as \textit{Vairāgyaśataka} of Bhartṛhari. Such an \textit{upadeśa} is aimed at generating detachment in the minds of people that would culminate in attaining \textit{jñāna} and \textit{mokṣa} through that. Also, this aspect is an \textit{upalakṣaṇa} (example) of the other so–called mundane comforts. Krishna, it seems, did not take the purport (\textit{tātparya}) and went by the primary meaning (\textit{vācyārtha}).

 I fail to understand as to how a great Vedāntīn like Tyāgarāja can preach hatred against women while being well aware of the \textit{Upaniṣadvākya} (\textit{Bṛhadāraṇyakopaniṣat} 2.4.6) – \textit{idaṁ sarvaṁ yadayam ātmā} – all this is \textit{brahman} only.

\begin{verse}
\textit{Aṭu kārādani balka} (Manoranjani–Ādi)\\\textit{“aṭu kārādani balka nā abhimānamu lekapoyênā} \\\textit{êṭulortunu o! daya jūḍavayya e vêlpu seyu calamo têlisi}\\\textit{vedaśāstropaniṣad viduḍainanijadārini} \\\textit{baṭṭi dāsuḍaina nādupai nêpamêñcite tyāgarājanuta.}” –
\end{verse}

 This is a preemptive strike by Tyāgarāja – “I am a scholar of Veda, Śāstra, Upaniṣad etc and on my own I became a \textit{dāsa} (of Rāma) and if someone throws an allegation at me, how can I tolerate?” We have discussed \textit{śama–damādi–sampat} and how Tyāgarāja proved through his life how he possessed that treasure. From that one can easily see that the allegations are absolutely unjustified.

 \item 
 In a mood of introspection, Tyāgarāja enumerates a number of things that should not be done as well as those that encumber one from praying to God. Though he does not possess any one of the vices listed, it is for people who may possess the same, so that they would desist from getting involved in such things any longer –  

\begin{myquote}
\textit{duḍukugala nanne dôrakôḍuku} (Gauḷa – Ādi)\\\textit{“…têliyani naṭa–viṭa–ṣūdrulu vanitalu svavaśamauṭa kupadeśiṁci}\\\textit{santasilli svaralayambu lêruṅgakanu śilātmulai subhaktulaku samānamanu..”}
\end{myquote}

 “Śrīrāma! Which God would save a wicked person like me? \textit{.....} I was just like a stone, ignorant of  \textit{svara} and \textit{laya}, equalling myself to great \textit{bhakta}–s, preaching \textit{naṭa}–s, \textit{viṭa}–s and other uneducated people as well as ladies simply to enslave them.” Here Tyāgarāja wants to convey the strong message that many people in the society are cheating the less educated/uneducated people by posing as great musicians.

 In the same \textit{kṛti}, he says “\textit{…môdaṭikulajuḍagucu bhuvini śūdrula panulu salpucunu yuṇṭini gāka..”} “On this earth, though I have been born in the highest Brahmin class, I have been performing functions, which are very unbecoming of my class.”

 Here he is targeting those \textit{brāhmaṇa}–s who are not practicing the duties prescribed for them and getting involved in vices, and this in no way amounts to demeaning of other castes. The terms \textit{môdaṭikulajuḍu} and \textit{śūdra} denote the secondary meaning (\textit{lakṣyārtha}) rather than the primary meaning (\textit{vācyārtha}). The term “\textit{gaṅgāyām}” in the sentence “\textit{gaṅgāyām ghoṣah}” (the cowshed is in Ganges) means “on the bank of river Ganges” (\textit{lakṣaṇā}). This aspect is discussed in all major systems of Indian philosophy and is universal – “London is on the Thames” means “London is on the banks of the Thames”.

 This \textit{kṛti} clearly explains Tyāgarāja’s \textit{tapasyā}. Although he did not commit anything wrong, he imposes upon himself those vices, only to give a message to the people. Also, by this \textit{kṛti} Tyāgarāja expects everyone to be more spiritual, nurture virtues and give up the vices that are commonly found in people.

 \item 
 “Who can adequately describe the beauty and charm of Lord Śrīrāma? Men (being blind to this), however great they may be, are constantly engaged in the thought of women. They put on the garb of real \textit{Bhāgavata}–s (devotees), but become slaves to the charm of women. This is just like the milk container that cannot enjoy the delicious taste of its own content” –

\begin{verse}
\textit{ênta muddo! ênta sôgaso!} (Bindumālini –  Ādi)\\ “\textit{ênta muddo! ênta sôgaso! êvarivalla varṇimpa tagune}\\\textit{ênta vāralainagāni kāntacintākrāntulaināru}\\\textit{attamīda kanulāsaku dāsulai sattabhāgavatavesulairi}\\\textit{duttapālaruci têliyu sāmyame dhurīṇuḍau tyāgarājanutuḍu}”
\end{verse}

 This \textit{kṛti} is a perfect example to exhibit the clean mind and flawless behavior of Tyāgarāja – one should emulate him.

 \item 
 “\textit{êntanercina ênta jūcina} (Udayaravicandrika – Deśādi)

\begin{verse}
“\textit{êntanercina êntajūcina êntavāralaina kāntadāsule}\\\textit{santatambu śrīkāntasvānta siddhāntamaina mārgacintalenivā(ru)}\\\textit{para–hiṁsa parabhāmānyadhana paramānavāpavāda} \\\textit{parajīvanādulakanṛtame bhāṣiñcêrayya tyāgarājanuta}”
\end{verse}

 “One may agree or disagree, but the essence of human behavior is this:  however great be the amount of knowledge one has acquired, however great be the amount of experience one has gained or however great might one be, all are slaves to women. Those who do not ponder over (\textit{nididhyāsana /dhyāna}) the right path that leads to the philosophy of life, those who do not contemplate on the Almighty, those who hurt others, those who speak untruth only desiring for wealth and women belonging to others and those who slander others for personal gains [are all slaves to women despite their knowledge or experience or greatness].”

 This \textit{kṛti} reflects Tyāgarāja’s pure and perfect personality. Itonly brings out the natural human tendency to slip into vices, which the saint is warning us against. Where is the question of any gender bias or casteism here?

\end{enumerate}

“She is always the vice” is a statement that is nebulous and ambiguous. The term ‘she’ refers to a \textit{jāti} (class) which means all \textit{vyakti}–s (individuals) irrespective of age, color, form etc. Some ‘scholars’ commit such blunders due to lack of deep Indian linguistic knowledge. The term ‘cow’ in ‘do not kill a cow’ means \textit{jāti} (all cows) whereas the same in ‘fetch a cow’ means \textit{vyakti} (a single cow) and both are inseparable (\textit{nāntarīyaka}). 

If one wants to decide the meaning of a word or sentence (\textit{padārtha} or \textit{vākyārtha}) he should proceed to a \textit{gurukula} and learn at least Vyākaraṇa, Nyāya and Mīmāmsā, which deal with the nuances of language. 

A similar mistake again – “(she is) … the seductress” (Krishna 2017). Seductress is a woman who seduces someone. How can the whole gamut of women (\textit{strījāti}) irrespective of age, color etc. seduce men and what for? 

Tyāgarāja’s position on women and caste was not at all shaped by his times. Actually if you take a serious look at his \textit{kṛti}–s, it is clear that he focused on \textit{bhakti} throughout. One has to take into account the \textit{tātparya} (import) of his gamut of literature rather than pick up a couple of words or sentences. This aspect is discussed in Mīmāṁsādarśana (\textit{ekavākyatā}). 

Moreover, for over two centuries not a single layman or scholar raised a finger against Tyāgarāja, much less, women or people belonging to other castes. Rather Tyāgarāja’s \textit{kṛti}–s are learnt with devotion, recited (they are on par with Veda, so recited – otherwise rendered) and propagated by thousands and this also helped in maintaining harmony in the society. So Tyāgarājais vindicated and his critics are outflanked. 

One should bear in mind that ours is called \textit{sanātana dharma}, i.e. a \textit{dharma} that is immutable/eternal, and it does not undergo any change even centuries later. So, one cannot apply phrases like ‘social commentator’, ‘\textit{smārta network}’ etc. to Tyāgarāja, who rejected all kinds of worldly comforts (Ref. \textit{nidhi cāla sukhamā} ), contacts etc. – he was a \textit{Yogin} who mentally renounced the world. Such concepts can be digested by only those who are trained in Vedānta.

Vidvān Krishna (2018) says the following –

\begin{myquote}
“…however considering that his words are even today sacrosanct – unquestionable pearls of wisdom that no one within the Carnatic universe would dare challenge – I am forced to debate his opinions”.
\end{myquote}

How sarcastic is Krishna! He rather thinks that scholars as well as the common people are so blind that they are not able to understand the mind of Tyāgarāja through his words and challenge. People do react or get provoked if there is something that hurts their feelings or offends the tenets of social wellbeing. Tyāgarāja did not intend to cause any pain to anyone. He was a real \textit{yogin} and has his mind cleansed through the \textit{adhyayana} of Veda–s, Vedāṅga–s, \textit{darśana}–s etc. Moreover, it is simply incredible that so many people, both educated and uneducated, for so many years did not understand Tyāgaraja’s words and all of them were suffering from inferiority complex. The \textit{yogin} was just following the guidelines of \textit{sanātana dharma;} and is that a crime? In fact, it is Krishna who is provoking people against a great composer of lyrics which has earned him bread, name and fame. A critic must think beyond \textit{rāga} and \textit{dveṣa}. Such writings insinuate that the person in question is neither honest nor speaks the truth. 

There is another aspect. Krishna writes about the \textit{kṛti}–s of Tyāgarāja that they are “even today sacrosanct”. What does it mean? Does Krishna think that Tyāgarāja’s \textit{kṛti}–s are just like a novel, which is thrown after a single reading? They are the gist of Veda, Vedānta etc. Since, as per the tradition, Veda is \textit{nitya (}eternal); so also are the \textit{kṛti}–s. Otherwise they would have not been alive for hundreds of years. 

Here is another jibe by Krishna (2018) –

\begin{myquote}
“I hold the view that lyrics in Carnatic music are abstract entities of sound, language itself is a creative sonic body. In other words, the meaning of the lyrics does not matter as much as the sound of every syllable, accent, enunciation, extension and aspiration…. I realised that my own music – Carnatic music – could be violent to others”.
\end{myquote}

What does Krishna mean by this? Is this not self–contradictory? If meanings of lyrics do not matter as he holds, that what is his problem if the lyrics are violent? Moreover, he is living on a music that is violent to others. What kind of \textit{dharma} is this? Why is he even singing?

Patañjali (\textit{Mahābhāṣya,} ‘\textit{bhāve}’, 3.3.18, ‘\textit{tasyāpatyam}’, 4.1.92) says that there are people who strictly follow a theory or behavior but denounce the same –

\begin{verse}
“\textit{tatkārī ca bhavān taddveṣī ca}”
\end{verse}

A perfect example for the above norm is unequivocally the behavior of Krishna, who lives on Gāndharva Veda, propagates and eulogizes the same but at the same time denounces the \textit{kṛti}–s and blames their composers. 

\textit{Saṅgītaratnākara} (1.1.30) asserts that \textit{saṅgīta} is an important instrument in achieving all the four \textit{puruṣārtha}–s, viz. \textit{dharma, artha, kāma} and \textit{mokṣa}.

\begin{verse}
“\textit{tasya gītasya māhātmyaṁ ke praśaṁsitum īśate}~।\\\textit{dharmārtha–kāma–mokṣāṇām idam evaika–sādhanam}।।”
\end{verse}

Elsewhere in Vyākaraṇa\textit{, Mahābhārata} (Śāntiparvan) etc. \textit{nāda–brahman} is considered as \textit{para–brahman}, which is the very cause of the universe. \textit{Saṅgītaratnākara, (}1.2.2; 1.2.167) declares that speech in different forms had emerged from \textit{nāda} and \textit{āhatanāda}, in the form of \textit{śruti} etc. turns into \textit{geya} and the same is useful in not only entertaining people but also in achieving \textit{mokṣa} –

\begin{verse}
“\textit{nādena vyajyate varṇaḥ padaṁ varṇāt padāt vacaḥ}~।\\\textit{vacaso vyavahāro’yaṁ nādādhīnam ato jagat}~।। (1.2.2)\\\textit{tasmād āhata–nādasya śrutyādidvārato’khilam}~।\\\textit{geyaṁ vitanvato loka–rañjanaṁ bhava–bhañjanam}~।। (1.2.167)”
\end{verse}

Now let a scholar, nay, a layman compare this with the statement of Krishna (2018 cited above) and draw his own conclusion.

Krishna (2015) had also made the following remarks –  

\begin{myquote}
“We brush aside the obvious brahminisation they had to undergo for acceptance”. “... It is only from these investigations that we will realise that at the core of it Carnatic music is very upper caste”.
\end{myquote}

In the first place, I do not understand as to what Krishna means by the term ‘brahminisation’. Since time immemorial different groups of people in the society embraced different arts to eke out their livelihood, so also \textit{brāhmaṇa}–s. The fact is that a \textit{brāhmaṇa} should not strive for riches. He has to serve the society by disseminating knowledge, act as a \textit{purohita,} as a \textit{jyautiṣika,} should keep himself away from vices, try to generate spiritual outlook in the minds of people and so on. There is not a single tenet which restricts one, on the grounds of caste, gender, birth etc., from learning \textit{saṅgīta} including Karnatic music. Smt. M. S. Subbulakshmi is a good example. Anyone is welcome to follow the tenets prescribed for \textit{brāhmaṇa}–s. \textit{Satya, ahiṁsā, bhūtadayā, kṣamā} etc. are the virtues to  be followed by one and all. 

Strictly speaking, the term “brahminisation” is quite unheard of. It seems that the term is coined by Krishnasimply to provoke people against a particular section of the society without any basis. As far as the scriptures are concerned, there is not any ceremony or procedure to that effect.

In the same interview, he says “Music and specifically Carnatic music has given me a gift and that is a window to experiencing life beyond myself” (Krishna 2015). At the same time, however, he wishes to tarnish the image of the same that is centuries old. If he is indeed right, how come nobody had raised a finger against either Tyāgarāja or the music he propagated with all his zeal for a period of centuries but today someone finds something amiss? 

The comments are uncalled for. His remarks show that he is not only dishonest but also did not bother to digest the sublime philosophy of Tyāgarāja’s \textit{kṛti}–s. 

It is upto the general public to decide as to whether Tyāgarājadid hurt anyone at all by his \textit{kṛti}–s on the basis of caste and gender or simply dedicated his life to disseminate Karnatic music and the great Indian culture laced with philosophy to common people.

Therefore, in conclusion, it can be reiterated that Tyāgarāja’s \textit{kṛti}–s are universal, useful in maintaining harmony in the society through spiritual and moral education, which is essential especially in the modern times, and of course, they lead one to the final goal of \textit{sanātana dharma}, viz \textit{mokṣa}.


\section*{Bibliography}

\begin{thebibliography}{99}
\bibitem{chap3–key01} \textit{Bhagavadgītā}. (1968).  Madras: Vavilla Press.

 \bibitem{chap3–key02} \textit{Bhāgavatam}. (1927).  Madras: Vavilla Press.

 \bibitem{chap3–key03} Bodas, Rajaram Sastri (Ed.) (1917). \textit{Pātañjalayogasūtrāṇi} of Patañjali. Bombay: Sanskrit and Prakrit Series.

 \bibitem{chap3–key04} \textit{Bṛhadāraṇyakopaniṣad}. (1940). Bombay: Gangavishnu Sri Krishna Das.

 \bibitem{chap3–key05} \textit{Chāndogyopaniṣat}. See \textit{Complete Works of Śaṅkarācārya}.

 \bibitem{chap3–key06} \textit{Complete Works of Śaṅkarācārya}. (2012). Kalady, Kerala: Sree Śaṅkarācārya University of Sanskrit.

 \bibitem{chap3–key07} Apte,Harinarayana (Ed.) (1820). \textit{Aitareyāraṇyakam}. Puna: Anandasrama Press.

 \bibitem{chap3–key08} \textit{Kaṭhopaniṣat}. See \textit{Complete Works of Śaṅkarācārya}.

 \bibitem{chap3–key09} Krishna, T. M. (2015). “There is a distinct caste – elitism in Carnatic sabha culture in Chennai, says musician TM Krishna”. \textit{DNA}. September 12, 2015. \textless  https://www.dnaindia.com/lifestyle/report–there–s–a–distinct–caste–elitism–in–carnatic–sabha–culture–in–chennai–says–musician–tm–krishna–2124546\textgreater . Accessed on 15 Feb 2019.

 \bibitem{chap3–key10} —. (2017). “A Case of Aesthetic Extravagance”. \textit{The Hindu}, May 04, 2017;

 \bibitem{chap3–key11} Updated May 04, 2017. \textless https://www.thehindu.com/entertainment/music/tyagarajas–musical–span–and–insight–reiterates–his–genius/article18384186.ece\textgreater  Accessed on 15 Feb 2019.

 \bibitem{chap3–key12} —. (2018). “Boycotts and bans are not enough, we must problematize the work of Artists who disturb us”. \textit{First Post}. 06–Aug–2018.  \textless https://www.firstpost.com/living/boycotts–and–bans–are–not–enough–we–must–problematise–the–work–of–artists–who–disturb–us–4857281.html?fbclid=IwAR2dH6NjxyxGrP0WMaoYOeE80aptWPo0zFOsLyhCIDqvkNdteYWdq51JrVI\textgreater . Accessed on 15 Feb 2019.

 \bibitem{chap3–key13} \textit{Mahābhārata}. (1907). Bombay: Nirnayasagar Press. 

 \bibitem{chap3–key14} \textit{Mahābhāṣya.} See Sharma (1988).

 \bibitem{chap3–key15} \textit{Naradabhaktisūtramulu}. (1989). of Nārada. Erpedu: Vyāsāśrama.

 \bibitem{chap3–key16} \textit{Pātañjalayogasūtras}. See Bodas (1917).

 \bibitem{chap3–key17} Ramanujachariar, C and Raghavan, V. (1957). \textit{Spiritual Heritage of Tyagaraja}. Mylapur, Madras: The Ramakrishna Mission Students’ Home.

 \bibitem{chap3–key18} \textit{Śābarabhāṣya.} See Sastri (2004).

 \bibitem{chap3–key19} \textit{Saṅgītaratnākara.} See Sastri (1944).

 \bibitem{chap3–key20} Sarma, Uppala Srinivasa. (2009). \textit{Tyāgarāju Kīrtanalu, Nāradabhakti sūtramulu: Tulanātmaka Adhyayanam}.  Hyderabad: Bharadwāja Pracuraṇalu.

 \bibitem{chap3–key21} Sastri, Gajanana. (Ed.) (2004). \textit{Śābarabhāṣyam} of Śabarasvāmī. Varanasi: Chowkhamba.

 \bibitem{chap3–key22} Sastri, Peri Subrahmanya (Ed.) (1983).  \textit{Viveka–cūḍāmaṇi} of Śaṅkarācārya. Mylapore, Madras: Sri Ramakrishna Math.

 \bibitem{chap3–key23} Sastri, S Subrahmanya (Ed.) (1944). \textit{Saṅgītaratnākaram} of Sāraṅgadeva. Madras: Adyar Library. 

 \bibitem{chap3–key24} \textit{Śatapathabrāhmaṇa}. (1940). Bombay: Gangavishnu Sri Krishna Das.

 \bibitem{chap3–key25} Satyanarayana, N. C. (2002). \textit{TyāgarājaSāraswataSarvasvam}. Secunderabad: Sundarakrupa. 

 \bibitem{chap3–key26} Sharma, Shivdatta (Ed.) (1988). \textit{Mahābhāṣyam} of Patañjali. Varanasi: Chowkhamba.

 \bibitem{chap3–key27} \textit{Śikṣāsaṅgraha.} SeeTripathi (1989).

 \bibitem{chap3–key28} \textit{Śvetāśvataropaniṣat} (1962). Gorakhapur: Gita Press.

 \bibitem{chap3–key29} \textit{Taittirīya Saṁhitā}. (1947). Mysore: Coronation Press.

 \bibitem{chap3–key30} \textit{Taittirīyāraṇyakopaniṣat}. (1980). Mysore: Coronation Press. 

 \bibitem{chap3–key31} Tripathi, Ramaprasad (Ed.) (1989). \textit{Śikṣāsaṅgraha}. Varanasi: Sampurnananda Sanskrit University. 

 \bibitem{chap3–key32} \textit{Vākyapadīyam}. (2016). Varanasi: Sampurnananda Sanskrit University.

 \bibitem{chap3–key33} \textit{Viveka–cūḍāmaṇi}. See Sastri (1983).

 \bibitem{chap3–key34} \textit{Viṣṇupurāṇam}. (2006).  Of Vyāsa. Delhi: Nag Publishers. \\

 \end{thebibliography}

\theendnotes

