\sethyphenation{kannada}{
ಂ
ಅಂಗ-ಗ-ಳೆಂ-ಬು-ದನ್ನು
ಅಂಗಡಿ
ಅಂಗ-ಡಿಗೆ
ಅಂಗ-ಡಿ-ಯನ್ನು
ಅಂಗ-ಡಿ-ಯಲ್ಲಿ
ಅಂಗ-ಲಾ-ಚಿ-ದರು
ಅಂಗಿಯ
ಅಂಚಿಗೆ
ಅಂಚಿನ
ಅಂಚಿ-ನಲ್ಲಿ
ಅಂಜ
ಅಂಜ-ಬೇ-ಕಾ-ಗಿಲ್ಲ
ಅಂಜಿ
ಅಂಜಿಕೆ
ಅಂಜಿ-ಕೆ-ಯಾ-ಗಿದೆ
ಅಂಜಿ-ಕೆ-ಯಾ-ಗು-ವುದು
ಅಂಜಿ-ಕೆ-ಯುಂ-ಟಾಗಿ
ಅಂಜಿ-ಕೊಂಡು
ಅಂಜಿ-ದರು
ಅಂಜಿ-ಸಲು
ಅಂಜು-ತ್ತಿ-ದ್ದರು
ಅಂಜು-ವು-ದಿಲ್ಲ
ಅಂಜು-ವುದು
ಅಂಟಿ-ಕೊಂ-ಡಿ-ರು-ವು-ದಿಲ್ಲ
ಅಂಡ-ಕೋಶ
ಅಂತ
ಅಂತಃ-ಪು-ರಕ್ಕೆ
ಅಂತ-ರಿ-ಕ್ಷ-ದಲ್ಲಿ
ಅಂತ-ರ್ಚ-ಕ್ಷು-ವಿ-ನಿಂದ
ಅಂತ-ರ್ದೃಷ್ಟಿ
ಅಂತ-ರ್ಮು-ಖ-ವಾ-ಗು-ವುದು
ಅಂತ-ರ್ಯಾಮಿ
ಅಂತಲೂ
ಅಂತಹ
ಅಂತ-ಹ-ವರ
ಅಂತೆಯೇ
ಅಂತ್ಯ
ಅಂದರೆ
ಅಂದಳು
ಅಂದ-ವಾಗಿ
ಅಂದಿ
ಅಂದಿಗೆ
ಅಂದಿದ್ದು
ಅಂದಿನ
ಅಂದಿ-ನಿಂದ
ಅಂದು
ಅಂದು-ಕೊಂ-ಡರು
ಅಂದು-ಕೊಂ-ಡಳು
ಅಂದೇ
ಅಂಶ
ಅಕ-ಸ್ಮಾ-ತ್ತಾಗಿ
ಅಕಾ-ಲ-ದಲ್ಲಿ
ಅಕ್ಕರೆ
ಅಕ್ಕ-ಸಾ-ಲಿಗ
ಅಕ್ಕ-ಸಾ-ಲಿ-ಗನ
ಅಕ್ಟೋ-ಬರ್
ಅಕ್ಬರ್
ಅಕ್ಷ-ರಶಃ
ಅಖಂಡ
ಅಗತ್ಯ
ಅಗಲ
ಅಗ-ಲು-ವುದು
ಅಗ-ಸ-ರಂತೆ
ಅಗ-ಸ-ರವ
ಅಗ-ಸರು
ಅಗಾ-ಧ-ವಾದ
ಅಗೆ
ಅಗೆದ
ಅಗೆ-ದಿ-ದ್ದನು
ಅಗೆ-ದಿ-ದ್ದರೆ
ಅಗೆದು
ಅಗೆ-ಯಲು
ಅಗೆ-ಯು-ತ್ತಿದ್ದ
ಅಗೆ-ಯು-ವು-ದಕ್ಕೆ
ಅಗೆ-ಯು-ವುದನ್ನು
ಅಗೆ-ಯು-ವು-ದ-ರೊ-ಳಗೆ
ಅಗೋ-ಚರ
ಅಗ್ರ-ಸ್ಥಾನ
ಅಘ-ಟಿ-ತ-ಘ-ಟನಾ
ಅಚಲ
ಅಜ್ಜ
ಅಜ್ಞಾನ
ಅಜ್ಞಾನಕ್ಕೆ
ಅಜ್ಞಾನ-ಗ-ಳಿಗೆ
ಅಜ್ಞಾನದ
ಅಜ್ಞಾನ-ವಿದೆ
ಅಜ್ಞಾನವೂ
ಅಜ್ಞಾನವೇ
ಅಜ್ಞಾನಿ
ಅಜ್ಞಾನಿ-ಗ-ಳಿ-ಗಾಗಿ
ಅಟ್ಟಿ-ಸಿ-ಕೊಂಡು
ಅಡಗಿ-ಸಿ-ಕೊಂ-ಡಿ-ರು-ವಳು
ಅಡಗಿ-ಸಿ-ಟ್ಟಿದ್ದ
ಅಡ-ಚಣೆ
ಅಡಿ
ಅಡಿ-ಗ-ಳಷ್ಟು
ಅಡಿಗೆ
ಅಡಿ-ದಾ-ವರೆ-ಗ-ಳಿಗೆ
ಅಡಿ-ಯಲ್ಲೆ
ಅಡಿ-ಯ-ವ-ರೆಗೆ
ಅಡ್ಡ-ಬಿದ್ದು
ಅಡ್ಡ-ಹೆ-ಸರು
ಅಡ್ಡಾ-ಡು-ತ್ತಿದ್ದೆ
ಅಣ-ತಿ-ಯಂತೆ
ಅಣಿ
ಅಣಿ-ಯಾ-ಗಿ-ರು-ವುದನ್ನು
ಅಣಿ-ಯಾಗು
ಅಣಿ-ಯಾ-ಗು-ತ್ತಿದ್ದ
ಅಣಿ-ಯಾ-ಗು-ತ್ತಿ-ರು-ವನು
ಅಣಿ-ಯಾದ
ಅಣ್ಣ
ಅಣ್ಣ-ತ-ಮ್ಮಂ-ದಿರು
ಅಣ್ಣಯ್ಯ
ಅತಿ
ಅತಿ-ಯಾಗಿ
ಅತಿ-ಯಾ-ಸೆಯ
ಅತೀಂ-ದ್ರಿಯ
ಅತೀತ
ಅತೀ-ತ-ನಾ-ಗ-ಬೇಕು
ಅತೀ-ತ-ನಾಗು
ಅತೀ-ತನು
ಅತೃ-ಪ್ತ-ನಾಗಿ
ಅತೃಪ್ತಿ
ಅತೃ-ಪ್ತಿ-ಕ-ರ-ವಾದ
ಅತ್ತ
ಅತ್ತ-ಕ-ಡೆಗೆ
ಅತ್ತು
ಅತ್ತೆ-ಗಾಗಿ
ಅತ್ಯಂತ
ಅತ್ಯಲ್ಪ
ಅತ್ಯು-ತ್ಕೃಷ್ಟ
ಅಥವಾ
ಅದ-ಕ್ಕಾಗಿ
ಅದ-ಕ್ಕಾ-ಗಿಯೇ
ಅದ-ಕ್ಕಿಂತ
ಅದಕ್ಕೆ
ಅದ-ಕ್ಕೆಲ್ಲ
ಅದಕ್ಕೇ
ಅದ-ಕ್ಕೇ-ನಾದ್ರೂ
ಅದ-ಕ್ಕೇನು
ಅದ-ಕ್ಕೊಂದು
ಅದನ್ನು
ಅದನ್ನೂ
ಅದ-ನ್ನೆಲ್ಲ
ಅದನ್ನೇ
ಅದರ
ಅದ-ರಂತೆ
ಅದ-ರಂ-ತೆಯೇ
ಅದ-ರಲ್ಲಿ
ಅದ-ರಲ್ಲೇ
ಅದ-ರಿಂದ
ಅದ-ರಿಂ-ದಲೂ
ಅದ-ರಿಂ-ದಲೇ
ಅದ-ರೊ-ಳ-ಗಿ-ರುವ
ಅದ-ರೊ-ಳಗೆ
ಅದ-ಲ್ಲದೆ
ಅದಾ-ದ-ಮೇಲೆ
ಅದು
ಅದೂ
ಅದೃಷ್ಟ
ಅದೃ-ಷ್ಟ-ವನ್ನು
ಅದೆಲ್ಲ
ಅದೇ
ಅದೇ-ನಾ-ದರೂ
ಅದೇನು
ಅದೇನೇ
ಅದೇನೊ
ಅದೊಂದು
ಅದೊಂದೇ
ಅದ್ದಿ
ಅದ್ಭುತ
ಅದ್ಭು-ತ-ವನ್ನು
ಅದ್ಭು-ತ-ವಾದ
ಅದ್ಭು-ತ-ವಾ-ದುದು
ಅದ್ವಿ-ತೀಯ
ಅದ್ವೈತ
ಅದ್ವೈ-ತ-ದಲ್ಲಿ
ಅದ್ವೈ-ತ-ವನ್ನು
ಅದ್ವೈ-ತಿ-ಗಳು
ಅದ್ಹೇ-ಗಮ್ಮ
ಅಧರ್ಮ
ಅಧ-ರ್ಮ-ವಿದೆ
ಅಧ-ರ್ಮಿ-ಯಲ್ಲಿ
ಅಧಿಕ
ಅಧಿ-ಕ-ವಾಗಿ
ಅಧಿ-ಕ-ವಾ-ಗು-ವುದು
ಅಧಿ-ಕಾರ
ಅಧಿ-ಕಾ-ರ-ಗಳನ್ನು
ಅಧಿ-ಕಾ-ರದ
ಅಧಿ-ಕಾ-ರ-ಮುದ್ರೆ
ಅಧಿ-ಕಾ-ರ-ವನ್ನು
ಅಧಿ-ಕಾ-ರಿ-ಗ-ಳಿಗೆ
ಅಧಿ-ಕಾ-ರಿ-ಗಳು
ಅಧಿ-ಕಾ-ರಿಗೆ
ಅಧಿ-ಕಾ-ರಿಯು
ಅಧಿ-ಕಾ-ರಿ-ಯೊಬ್ಬ
ಅಧಿ-ದೇ-ವತೆ
ಅಧಿ-ದೇ-ವ-ತೆ-ಯಾಗಿ
ಅಧಿ-ಪತಿ
ಅಧಿ-ಪ-ತಿ-ಯಾದ
ಅಧೀನ
ಅಧೀ-ನ-ದ-ಲ್ಲಿ-ರು-ವರು
ಅಧೀ-ನರು
ಅಧ್ಯ-ಕ್ಷರು
ಅಧ್ಯ-ಯನ
ಅಧ್ಯಾತ್ಮ
ಅಧ್ಯಾ-ತ್ಮ-ದಲ್ಲಿ
ಅನಂತ
ಅನಂ-ತರ
ಅನಂ-ತ-ರವೂ
ಅನಂ-ತ-ವ-ನ್ನಾಗಿ
ಅನಂ-ತ-ವಾ-ಗಲಿ
ಅನಂ-ತ-ವಾ-ದದ್ದು
ಅನಂ-ತ-ವಾ-ದುದು
ಅನನ್ಯ
ಅನಾಥ
ಅನಾದಿ
ಅನಾ-ಸಕ್ತ
ಅನಾ-ಸ-ಕ್ತ-ನಾದ
ಅನಾ-ಸ-ಕ್ತ-ನಾ-ದ್ದ-ರಿಂದ
ಅನಾ-ಸ-ಕ್ತ-ರಾದ
ಅನಾ-ಹು-ತ-ವನ್ನು
ಅನಿ-ವಾ-ರ್ಯ-ಗಳು
ಅನಿಷ್ಟ
ಅನು
ಅನು-ಕ-ರಣ
ಅನು-ಕ-ರಣೆ
ಅನು-ಕ-ರಿ-ಸಿ-ದರೆ
ಅನು-ಕ-ರಿ-ಸು-ತ್ತಿದ್ದ
ಅನು-ಕೂ-ಲ-ಗಳು
ಅನು-ಗ್ರ-ಹ-ದಿಂದ
ಅನು-ಗ್ರ-ಹ-ವಿ-ದ್ದರೆ
ಅನು-ಭವ
ಅನು-ಭ-ವ-ಗಳನ್ನು
ಅನು-ಭ-ವ-ದಿಂದ
ಅನು-ಭ-ವ-ವನ್ನು
ಅನು-ಭ-ವ-ವಿ-ದೆಯೋ
ಅನು-ಭ-ವಿ-ಸ-ಬೇ-ಕಾ-ಗಿತ್ತು
ಅನು-ಭ-ವಿ-ಸ-ಬೇ-ಕೆಂಬ
ಅನು-ಭ-ವಿ-ಸ-ಲಾ-ಗ-ಲಿಲ್ಲ
ಅನು-ಭ-ವಿ-ಸಲೇ
ಅನು-ಭ-ವಿ-ಸ-ಲೇ-ಬೇಕು
ಅನು-ಭ-ವಿಸಿ
ಅನು-ಭ-ವಿ-ಸಿದ
ಅನು-ಭ-ವಿ-ಸಿ-ದಂ-ತಾ-ಯಿತು
ಅನು-ಭ-ವಿ-ಸಿದೆ
ಅನು-ಭ-ವಿ-ಸಿ-ರು-ವನೋ
ಅನು-ಭ-ವಿಸು
ಅನು-ಭ-ವಿ-ಸು-ತ್ತಾನೊ
ಅನು-ಭ-ವಿ-ಸು-ತ್ತಿ-ರು-ವನು
ಅನು-ಭ-ವಿ-ಸು-ವನು
ಅನು-ಭ-ವಿ-ಸು-ವು-ದಕ್ಕೂ
ಅನು-ಭ-ವಿ-ಸು-ವುದು
ಅನು-ಮ-ತಿ-ಯನ್ನು
ಅನು-ಮಾ-ನ-ಪ-ಡು-ವ-ವನು
ಅನು-ಮಾ-ನಿ-ಸ-ಲಿ-ಕ್ಕಿಲ್ಲ
ಅನು-ಯಾ-ಯಿ-ಗಳು
ಅನು-ಯಾ-ಯಿ-ಗ-ಳೆಲ್ಲ
ಅನು-ಷ್ಠಾನ
ಅನು-ಸ-ರಿ-ಸಲಿ
ಅನು-ಸ-ರಿಸಿ
ಅನು-ಸ-ರಿ-ಸಿ-ಕೊಂಡು
ಅನು-ಸ-ರಿ-ಸಿದ
ಅನು-ಸ-ರಿ-ಸಿ-ದರೆ
ಅನು-ಸ-ರಿ-ಸಿ-ದವು
ಅನು-ಸ-ರಿಸು
ಅನು-ಸ-ರಿ-ಸು-ತ್ತಿದ್ದ
ಅನು-ಸ-ರಿ-ಸು-ತ್ತಿ-ದ್ದನು
ಅನು-ಸ-ರಿ-ಸು-ತ್ತಿ-ರುವೆ
ಅನು-ಸ-ರಿ-ಸು-ತ್ತೇನೆ
ಅನು-ಸ-ರಿ-ಸು-ವ-ವರು
ಅನು-ಸ-ರಿ-ಸು-ವೆನು
ಅನೇಕ
ಅನೇ-ಕರ
ಅನೇ-ಕ-ರಿಗೆ
ಅನೇ-ಕರು
ಅನೇಕೆ
ಅನ್ನ-ಛ-ತ್ರ-ದಲ್ಲಿ
ಅನ್ನ-ಪ್ರಾ-ಶನ
ಅನ್ನ-ವನ್ನು
ಅನ್ನು-ವ-ನಲ್ಲ
ಅನ್ಯ
ಅನ್ಯಾಯ
ಅನ್ವ-ಯಿ-ಸು-ವುದು
ಅಪ-ಕಾ-ರ-ವನ್ನು
ಅಪ-ಘಾ-ತ-ದಿಂದ
ಅಪ-ರಿ-ಚಿತ
ಅಪ-ರಿ-ಚಿ-ತನು
ಅಪ-ರೂಪ
ಅಪಾಯ
ಅಪಾ-ಯ-ಕರ
ಅಪಾ-ಯ-ದಲ್ಲಿ
ಅಪಾ-ಯ-ದ-ಲ್ಲಿ-ರು-ವನು
ಅಪಾ-ಯ-ವೇನೂ
ಅಪಾ-ವಿ-ತ್ರ್ಯ-ವಿದೆ
ಅಪೂರ್ವ
ಅಪೇ-ಕ್ಷೆ-ಯಿಂದ
ಅಪ್ಪ
ಅಪ್ಪಟ
ಅಪ್ಪಣೆ
ಅಪ್ಪ-ಣೆ-ಯಂತೆ
ಅಪ್ಪ-ಣೆ-ಯನ್ನು
ಅಪ್ಪಳಿ
ಅಪ್ಪ-ಳಿ-ಸಿದ
ಅಪ್ಪಿ-ಕೊ-ಳ್ಳಲು
ಅಫೀ-ಮನ್ನು
ಅಫೀ-ಮಿನ
ಅಭಾ-ವ-ದಿಂದ
ಅಭಿ-ಪ್ರಾಯ
ಅಭಿ-ಪ್ರಾ-ಯ-ವನ್ನು
ಅಭ್ಯಂ-ತ-ರ-ವೇನೂ
ಅಭ್ಯಾಸ
ಅಭ್ಯಾ-ಸ-ಗಳನ್ನು
ಅಮಲು
ಅಮೂಲ್ಯ
ಅಮೃ-ತದ
ಅಮೃ-ತ-ನಾ-ಗುವೆ
ಅಮೃ-ತ-ಸಾ-ಗರ
ಅಮೇ-ಧ್ಯಕ್ಕೆ
ಅಮ್ಮ
ಅಮ್ಮನ
ಅಮ್ಮ-ನ-ವ-ರಿಗೆ
ಅಯೋಗ್ಯ
ಅಯೋ-ಧ್ಯಾ-ನ-ಗ-ರ-ದ-ಲ್ಲಿ-ರುವ
ಅಯ್ಯ
ಅಯ್ಯೋ
ಅರ-ಚಿ-ಕೊಂ-ಡರೆ
ಅರ-ಚಿ-ಕೊಂ-ಡಳು
ಅರ-ಚಿ-ಕೊ-ಳ್ಳು-ತ್ತಿತ್ತು
ಅರ-ಚಿ-ಕೊ-ಳ್ಳು-ತ್ತಿದ್ದ
ಅರ-ಚುತ್ತ
ಅರ-ಚು-ತ್ತಿ-ದ್ದು-ದನ್ನು
ಅರ-ಣ್ಯದ
ಅರ-ಣ್ಯ-ದೊ-ಳಗೆ
ಅರ-ಣ್ಯ-ವಿತ್ತು
ಅರ-ಮ-ನೆಗೆ
ಅರ-ಮ-ನೆಯ
ಅರ-ಮ-ನೆ-ಯಲ್ಲಿ
ಅರ-ಸ-ನಾಗು
ಅರ-ಸಿ-ಕೊಂಡು
ಅರ-ಸು-ತ್ತಿದ್ದ
ಅರ-ಸು-ತ್ತಿರು
ಅರಿತ
ಅರಿ-ತರು
ಅರಿ-ತರೆ
ಅರಿ-ತಿ-ರುವೆ
ಅರಿತು
ಅರಿ-ತೊ-ಡ-ನೆಯೆ
ಅರಿಯ
ಅರಿ-ಯ-ದ-ವನು
ಅರಿ-ಯದು
ಅರಿ-ಯ-ಬ-ಲ್ಲನೆ
ಅರಿ-ಯ-ಬ-ಲ್ಲರು
ಅರಿ-ಯ-ಬ-ಲ್ಲೆವು
ಅರಿ-ಯರು
ಅರಿ-ಯ-ಲಿ-ಚ್ಛಿ-ಸು-ವನು
ಅರಿ-ಯವು
ಅರಿ-ಯು-ವನು
ಅರಿ-ಯು-ವೆವು
ಅರಿ-ವಾಗಿ
ಅರಿ-ವಾ-ಗು-ತ್ತದೆ
ಅರಿ-ವಾ-ಯಿತೊ
ಅರಿ-ವಿದೆ
ಅರಿ-ವಿ-ರು-ವುದು
ಅರಿ-ವಿ-ಲ್ಲದೆ
ಅರಿವು
ಅರಿವೂ
ಅರಿವೇ
ಅರ್ಜುನ
ಅರ್ಜು-ನ-ನಿಗೆ
ಅರ್ಜು-ನ-ನೊಂ-ದಿಗೆ
ಅರ್ಜು-ನನ್ನು
ಅರ್ಥ
ಅರ್ಥ-ಮಾ-ಡಿ-ಕೊಳ್ಳಿ
ಅರ್ಥ-ವನ್ನು
ಅರ್ಥ-ವಾ-ಗದ
ಅರ್ಥ-ವಾ-ಗ-ಬ-ಲ್ಲದೆ
ಅರ್ಥ-ವಾ-ಗ-ಲಿಲ್ಲ
ಅರ್ಥ-ವಾ-ಗು-ತ್ತಿ-ರ-ಲಿಲ್ಲ
ಅರ್ಥ-ವಾ-ಯಿ-ತ-ವ-ರಿಗೆ
ಅರ್ಥ-ವಿಲ್ಲ
ಅರ್ಥ-ವಿ-ಲ್ಲದ
ಅರ್ಥ-ವೇ-ನಂ-ದರೆ
ಅರ್ಥ-ವೇನು
ಅರ್ಥ-ವೇನೆಂದರೆ
ಅರ್ಧ
ಅರ್ಧ-ದ-ಷ್ಟನ್ನು
ಅರ್ಧ-ದಷ್ಟು
ಅಲಂ-ಕ-ರಿಸಿ
ಅಲಂ-ಕಾ-ರ-ಗಳು
ಅಲು-ಗಾ-ಡಲು
ಅಲೆಕ್
ಅಲೆ-ದಾ-ಡುತ್ತ
ಅಲೆದು
ಅಲೆ-ಯ-ತೊ-ಡ-ಗಿ-ದರು
ಅಲೆ-ಯು-ತ್ತಿ-ರುವೆ
ಅಲೆ-ಯು-ವನು
ಅಲೆ-ಯು-ವುದು
ಅಲ್ಪ
ಅಲ್ಪ-ಜ್ಞಾ-ನ-ದಿಂದ
ಅಲ್ಪ-ಮ-ನು-ಷ್ಯರ
ಅಲ್ಲ
ಅಲ್ಲ-ಗಳೆ
ಅಲ್ಲ-ಗ-ಳೆ-ಯಲು
ಅಲ್ಲ-ವೇ-ಎಂ-ಬು-ದನ್ನು
ಅಲ್ಲಾ
ಅಲ್ಲಾ-ಡಿ-ಸಿ-ದರೆ
ಅಲ್ಲಾ-ಡಿ-ಸಿ-ದಾಗ
ಅಲ್ಲಾ-ಡು-ತ್ತಿ-ದ್ದವು
ಅಲ್ಲಾನ
ಅಲ್ಲಾ-ನನ್ನು
ಅಲ್ಲಿ
ಅಲ್ಲಿಂದ
ಅಲ್ಲಿಗೂ
ಅಲ್ಲಿಗೆ
ಅಲ್ಲಿಗೇ
ಅಲ್ಲಿಟ್ಟು
ಅಲ್ಲಿದೆ
ಅಲ್ಲಿದ್ದು
ಅಲ್ಲಿನ
ಅಲ್ಲಿ-ಯ-ವ-ರೆಗೆ
ಅಲ್ಲಿಯೂ
ಅಲ್ಲಿಯೇ
ಅಲ್ಲಿ-ರುವ
ಅಲ್ಲಿ-ರು-ವನು
ಅಲ್ಲಿಲ್ಲ
ಅಲ್ಲೆ
ಅಲ್ಲೇ
ಅಳ-ತೊ-ಡ-ಗಿದ
ಅಳ-ಲಾ-ರಂ-ಭಿ-ಸಿ-ದರು
ಅಳ-ಲಿಲ್ಲ
ಅಳಲೆ
ಅಳಿ-ದು-ಳಿದ
ಅಳಿಯ
ಅಳಿ-ಸಿ-ಹೋ-ಗಲಿ
ಅಳಿ-ಸಿ-ಹೋ-ಗು-ವುದು
ಅಳು
ಅಳುತ್ತ
ಅಳು-ತ್ತಿದ್ದ
ಅಳು-ತ್ತಿ-ದ್ದರು
ಅಳು-ತ್ತಿ-ದ್ದಳು
ಅಳು-ತ್ತಿ-ರು-ವ-ರಲ್ಲ
ಅಳು-ತ್ತಿ-ರು-ವರು
ಅಳು-ತ್ತಿ-ರು-ವಿರಿ
ಅಳು-ವನು
ಅಳು-ವು-ದಕ್ಕೆ
ಅಳು-ವೆಲ್ಲ
ಅಳೆ
ಅಳೆದ
ಅವ
ಅವ-ಕಾಶ
ಅವ-ಕಾ-ಶ-ವನ್ನೇ
ಅವ-ತ-ರಿ-ಸಿದ
ಅವ-ತಾರ
ಅವ-ತಾ-ರಕ್ಕೆ
ಅವ-ತಾ-ರ-ಗಳು
ಅವ-ತಾ-ರ-ವನ್ನು
ಅವ-ತಾ-ರ-ವೆಂದು
ಅವ-ಧೂತ
ಅವ-ಧೂ-ತ-ನಿಗೆ
ಅವನ
ಅವ-ನಂತೆ
ಅವ-ನ-ತಿ-ಗಳು
ಅವ-ನ-ತಿಯ
ಅವ-ನನ್ನು
ಅವ-ನನ್ನೇ
ಅವ-ನಲ್ಲಿ
ಅವ-ನ-ಲ್ಲಿ-ದ್ದು-ದ-ನ್ನೆಲ್ಲ
ಅವ-ನಿಂದ
ಅವ-ನಿ-ಗಾಗಿ
ಅವ-ನಿ-ಗುಂ-ಟಾ-ಗಿತ್ತು
ಅವ-ನಿಗೂ
ಅವ-ನಿಗೆ
ಅವ-ನಿ-ಗೊಂದು
ಅವ-ನಿಚ್ಛೆ
ಅವನು
ಅವನೂ
ಅವ-ನೆ-ದು-ರಿಗೆ
ಅವನೇ
ಅವ-ನೇನು
ಅವ-ನೊಂದು
ಅವ-ನೊ-ಡನೆ
ಅವ-ನೊಬ್ಬ
ಅವ-ನೊ-ಳಗೆ
ಅವನ್ನು
ಅವರ
ಅವ-ರನ್ನು
ಅವ-ರಲ್ಲಿ
ಅವ-ರಿಂದ
ಅವ-ರಿಗೂ
ಅವ-ರಿಗೆ
ಅವ-ರಿ-ಗೇನೂ
ಅವ-ರಿ-ಬ್ಬರೂ
ಅವರು
ಅವರೂ
ಅವರೆ-ದು-ರಿಗೆ
ಅವರೆ-ದುರು
ಅವರೆಲ್ಲ
ಅವರೇ
ಅವ-ರೊಂ-ದಿಗೆ
ಅವ-ರೊ-ಡನೆ
ಅವ-ರೊ-ಳ-ಗೆಲ್ಲ
ಅವ-ರ್ಣ-ನೀಯ
ಅವ-ಲಕ್ಕಿ
ಅವಳ
ಅವ-ಳದು
ಅವ-ಳನ್ನು
ಅವ-ಳಲ್ಲ
ಅವ-ಳಲ್ಲಿ
ಅವ-ಳಿಂದ
ಅವ-ಳಿ-ಗಾಗಿ
ಅವ-ಳಿಗೆ
ಅವಳು
ಅವಸ್ಥೆ
ಅವ-ಸ್ಥೆ-ಯನ್ನು
ಅವ-ಸ್ಥೆ-ಯಲ್ಲಿ
ಅವ-ಸ್ಥೆ-ಯ-ಲ್ಲಿಯೇ
ಅವಿ-ಭಾಜ್ಯ
ಅವು
ಅವು-ಗಳನ್ನು
ಅವು-ಗಳನ್ನೆಲ್ಲ
ಅವು-ಗಳಲ್ಲಿ
ಅವು-ಗ-ಳ-ಲ್ಲಿ-ರುವ
ಅವು-ಗಳಿಂದ
ಅವು-ಗ-ಳೊ-ಡನೆ
ಅವೆಲ್ಲ
ಅವೇ
ಅವೇಳೆ
ಅವೇ-ಳೆ-ಯಾ-ಗಿತ್ತು
ಅಶ-ರೀ-ರ-ವಾಣಿ
ಅಶುಚಿ
ಅಶ್ವತ್ಥ
ಅಷ್ಟಕ್ಕೆ
ಅಷ್ಟ-ರಲ್ಲಿ
ಅಷ್ಟ-ರಲ್ಲೇ
ಅಷ್ಟ-ವ-ಕ್ರ-ನಂತೆ
ಅಷ್ಟ-ವ-ಸು-ಗಳಲ್ಲಿ
ಅಷ್ಟು
ಅಷ್ಟು-ಹೊ-ತ್ತಿಗೆ
ಅಷ್ಟೆ
ಅಷ್ಟೇ
ಅಷ್ಟೊಂದು
ಅಸಂ-ಬದ್ಧ
ಅಸಂ-ಬ-ದ್ಧ-ತೆ-ಯನ್ನು
ಅಸಂ-ಬ-ದ್ಧ-ವನ್ನು
ಅಸತ್ಯ
ಅಸ-ತ್ಯದ
ಅಸ-ದಳ
ಅಸ-ಮಾ-ಧಾ-ನ-ವಾ-ಗು-ವುದು
ಅಸ-ಹ್ಯ-ವಾ-ಯಿತು
ಅಸಾ-ಧಾ-ರಣ
ಅಸಾ-ಧಾ-ರ-ಣ-ವಾದ
ಅಸಾಧ್ಯ
ಅಸಾ-ಧ್ಯ-ವಲ್ಲ
ಅಸಾ-ಧ್ಯ-ವಾ-ಗಿ-ರು-ವುದು
ಅಸಾ-ಧ್ಯ-ವಾ-ದುದು
ಅಸಾ-ಧ್ಯವೂ
ಅಸಾ-ಮಿ-ಯಲ್ಲ
ಅಸೀ-ಮ-ವಾದ
ಅಸೂಯೆ
ಅಸ್ತಿತ್ವ
ಅಸ್ಥಿ-ಪಂ-ಜರ
ಅಸ್ಥಿ-ಪಂ-ಜ-ರ-ದಂತೆ
ಅಸ್ಪೃಶ್ಯ
ಅಸ್ವಾ-ಭಾ-ವಿಕ
ಅಹಂ
ಅಹಂ-ಕಾರ
ಅಹಂ-ಕಾ-ರಕ್ಕೆ
ಅಹಂ-ಕಾ-ರದ
ಅಹಂ-ಕಾ-ರ-ದಿಂದ
ಅಹಂ-ಕಾ-ರ-ಪಟ್ಟ
ಅಹಂ-ಕಾ-ರ-ಪ-ಟ್ಟನು
ಅಹಂ-ಕಾ-ರ-ಪ-ಡಲು
ಅಹಂ-ಕಾ-ರ-ವನ್ನು
ಅಹಂ-ಕಾ-ರ-ವಿ-ದೆಯೆ
ಅಹಂ-ಕಾ-ರ-ವಿ-ದ್ದರೂ
ಅಹಂ-ಕಾ-ರವು
ಅಹಂ-ಕಾ-ರವೂ
ಅಹಂ-ಕಾ-ರ-ವೆಂ-ಬುದು
ಅಹಂ-ಕಾ-ರವೇ
ಅಹಲ್ಯೆ
ಅಹ-ಲ್ಯೆಗೆ
ಅಹ-ಲ್ಯೆ-ಯನ್ನು
ಆ
ಆಂಗ್ಲ-ಭಾ-ಷೆಯ
ಆಂತ-ರ್ಯ-ವನ್ನು
ಆಕ-ರ್ಷ-ಣೀಯ
ಆಕ-ರ್ಷಣೆ
ಆಕ-ಳಿತ್ತು
ಆಕ-ಸ್ಮಿಕ
ಆಕಾಂಕ್ಷೆ
ಆಕಾಂ-ಕ್ಷೆ-ಗಳು
ಆಕಾ-ರ-ಗಳಲ್ಲಿ
ಆಕಾ-ರ-ದಲ್ಲಿ
ಆಕಾ-ರ-ವನ್ನು
ಆಕಾ-ರ-ವಿ-ದೆಯೆ
ಆಕಾ-ರ-ವಿ-ದ್ದರೆ
ಆಕಾ-ರವೂ
ಆಕಾಶ
ಆಕಾ-ಶದ
ಆಕಾ-ಶ-ದಲ್ಲಿ
ಆಕೆ
ಆಕೆಗೆ
ಆಕೆಯ
ಆಕೆ-ಯನ್ನು
ಆಗ
ಆಗದು
ಆಗಲಿ
ಆಗ-ಲಿಲ್ಲ
ಆಗ-ಲಿ-ಲ್ಲವೇ
ಆಗಲೂ
ಆಗಲೆ
ಆಗ-ಲೆಲ್ಲ
ಆಗಲೇ
ಆಗ-ಸಾಧು
ಆಗ-ಹ-ಣ-ವನ್ನು
ಆಗಾಗ
ಆಗಿ
ಆಗಿತ್ತು
ಆಗಿದೆ
ಆಗಿ-ದೆಯೊ
ಆಗಿ-ದ್ದಳು
ಆಗಿ-ದ್ದಾನೆ
ಆಗಿ-ದ್ದಾಳೆ
ಆಗಿ-ರುವ
ಆಗಿ-ರು-ವನು
ಆಗಿ-ರು-ವರು
ಆಗಿ-ರುವೆ
ಆಗಿ-ಲ್ಲ-ವೇನೋ
ಆಗು-ತ್ತಾನೆ
ಆಗು-ತ್ತಿದೆ
ಆಗು-ತ್ತಿ-ದ್ದೀ-ಯಲ್ಲ
ಆಗು-ತ್ತಿರ
ಆಗು-ತ್ತಿ-ರ-ಲಿಲ್ಲ
ಆಗುವ
ಆಗು-ವು-ದಿಲ್ಲ
ಆಗು-ವುದು
ಆಘ್ರಾಣಿ
ಆಚ-ಮನ
ಆಚ-ರಿ-ಸು-ತ್ತಾರೆ
ಆಚ-ರಿ-ಸು-ತ್ತಿದ್ದ
ಆಚಾ-ರ್ಯ-ನನ್ನು
ಆಚಾ-ರ್ಯ-ನಾ-ಗಿ-ರು-ವನೋ
ಆಚಾ-ರ್ಯ-ನಾದ
ಆಚಾ-ರ್ಯರು
ಆಚೆ
ಆಚೆಗೆ
ಆಚೆಯ
ಆಜ್ಞೆ
ಆಜ್ಞೆ-ಯನ್ನು
ಆಟ
ಆಟ-ಗ-ಳೆಲ್ಲ
ಆಟ-ದಲ್ಲಿ
ಆಟ-ವನ್ನು
ಆಟ-ವಾ-ಡು-ತ್ತಿತ್ತು
ಆಟ-ವಾ-ಡು-ತ್ತಿ-ದ್ದಾಗ
ಆಟಿ-ಕೆ-ಗಳು
ಆಡದೆ
ಆಡ-ಬಾ-ರದು
ಆಡ-ಲಿಲ್ಲ
ಆಡಿ-ಕೊ-ಳ್ಳ-ತೊ-ಡ-ಗಿ-ದರು
ಆಡಿ-ಕೊ-ಳ್ಳು-ತ್ತಾರೆ
ಆಡಿ-ನ-ಮ-ರಿ-ಯೊಂದು
ಆಡಿಸಿ
ಆಡಿ-ಸು-ತ್ತಿದ್ದ
ಆಡು
ಆಡು-ತ್ತಿ-ದ್ದನು
ಆಡು-ತ್ತಿ-ದ್ದೆವು
ಆಡು-ವ-ವರು
ಆಡು-ವು-ದಾ-ದರೆ
ಆಣ-ತಿ-ಯಂತೆ
ಆಣ-ತಿ-ಯನ್ನು
ಆಣೆ
ಆತ
ಆತಂಕ
ಆತಂ-ಕ-ಗಳು
ಆತಂ-ಕದ
ಆತಂ-ಕ-ವಾ-ಗಿದೆ
ಆತನ
ಆತ-ನನ್ನು
ಆತ-ನಿಗೆ
ಆತುರ
ಆತ್ಮ
ಆತ್ಮ-ಜ್ಞಾ-ನ-ವನ್ನು
ಆತ್ಮ-ನಿ-ಗಿಂತ
ಆತ್ಮನು
ಆತ್ಮ-ಶ್ರದ್ಧೆ
ಆತ್ಮ-ಹತ್ಯೆ
ಆದ
ಆದ-ಕಾ-ರಣ
ಆದ-ಕಾರಣವೆ
ಆದ-ಕಾರಣವೇ
ಆದ-ಮೇಲೆ
ಆದರ
ಆದ-ರ-ದಿಂದ
ಆದ-ರಿಸಿ
ಆದರೂ
ಆದರೆ
ಆದ-ರೇನು
ಆದರ್ಶ
ಆದವು
ಆದಾಯ
ಆದಿ
ಆದಿ-ಬ್ರ-ಹ್ಮ-ಸ-ಮಾ-ಜದ
ಆದು-ದ-ರಿಂದ
ಆದೊ-ಡ-ನೆಯೆ
ಆದ್ದ-ರಿಂದ
ಆದ್ದ-ರಿಂ-ದಲೇ
ಆಧಾ-ರ-ಗ್ರಂ-ಥ-ವಾ-ಗಿ-ರಿ-ಸಿ-ಕೊ-ಳ್ಳ-ಲಾಗಿದೆ
ಆಧಾ-ರದ
ಆಧು-ನಿಕ
ಆಧ್ಯಾ-ತ್ಮಿಕ
ಆಧ್ಯಾ-ತ್ಮಿ-ಕತೆ
ಆನಂದ
ಆನಂ-ದಕ್ಕೆ
ಆನಂ-ದದ
ಆನಂ-ದ-ದಿಂದ
ಆನಂ-ದ-ಬಾ-ಷ್ಪ-ವನ್ನು
ಆನಂ-ದ-ವನ್ನು
ಆನಂ-ದ-ವಾಗಿ
ಆನಂ-ದ-ವಾ-ಗಿ-ದೆಯೊ
ಆನಂ-ದಾಶ್ರು
ಆನೆ
ಆನೆ-ಗಳ
ಆನೆ-ಗಳನ್ನು
ಆನೆಗೆ
ಆನೆಯ
ಆನೆ-ಯನ್ನು
ಆನೆ-ಯಲ್ಲಿ
ಆನೆ-ಯ-ಲ್ಲಿಯೂ
ಆನೆ-ಯಾ-ಗಿದೆ
ಆನೆಯು
ಆನೆ-ಯೊಂದು
ಆಪ-ತ್ತು-ಗಳಿಂದ
ಆಫೀ-ಸರು
ಆಫೀ-ಸಿಗೆ
ಆಫೀ-ಸಿ-ನಲ್ಲಿ
ಆಫೀಸ್
ಆಭ-ರಣ
ಆಮೇಲೆ
ಆಯಾಯ
ಆಯಾಸ
ಆಯಾ-ಸ-ದಿಂದ
ಆಯಿತು
ಆಯಿತೊ
ಆಯಿ-ತೊ-ಯಾರು
ಆಯು-ಧ-ಗಳೂ
ಆಯ್ತು
ಆರಂ-ಭಿ-ಸಿದ
ಆರಾಣೆ
ಆರಾ-ಧ್ಯ-ವಸ್ತು
ಆರಿ-ಸಿ-ಕೊಂಡು
ಆರಿ-ಹೋ-ಗು-ತ್ತಿದೆ
ಆರು
ಆರೈಕೆ
ಆರ್ಭಟ
ಆಲಿಂ-ಗನ
ಆಲಿಂ-ಗ-ನಕ್ಕೆ
ಆಲಿಸಿ
ಆಲೂ-ಗ-ಡ್ಡೆಗೆ
ಆಲೋ-ಚನೆ
ಆಲೋ-ಚ-ನೆ-ಮಾಡಿ
ಆಲೋ-ಚ-ನೆ-ಯಂತೆ
ಆಲೋ-ಚ-ನೆ-ಯನ್ನು
ಆಲೋಚಿ
ಆಲೋ-ಚಿಸಿ
ಆಲೋ-ಚಿ-ಸಿದ
ಆಲೋ-ಚಿ-ಸಿ-ದಂತೆ
ಆಳಕ್ಕೆ
ಆಳ-ದ-ವ-ರೆಗೂ
ಆಳನ್ನು
ಆಳ-ವನ್ನು
ಆಳ-ವಾ-ಗಿದೆ
ಆಳ-ವಾದ
ಆಳಾ-ಗು-ತ್ತಾನೆ
ಆಳಿಗೆ
ಆಳಿನ
ಆಳಿ-ನಂತೆ
ಆಳು
ಆಳು-ಕಾಳು-ಗಳು
ಆಳು-ಗಳನ್ನು
ಆಳು-ಗ-ಳಿಗೆ
ಆಳು-ಗಳು
ಆಳು-ತ್ತಿದ್ದೆ
ಆವ-ರಿಸಿ
ಆವ-ರಿ-ಸಿತು
ಆವ-ಶ್ಯಕ
ಆವ-ಶ್ಯ-ಕತೆ
ಆವ-ಶ್ಯ-ಕ-ತೆಯೂ
ಆವ-ಶ್ಯ-ಕ-ತೆಯೇ
ಆವಿ-ರ್ಭಾ-ವ-ಗಳು
ಆವಿ-ರ್ಭಾ-ವ-ವನ್ನು
ಆಶಿ-ಸಿ-ದನು
ಆಶಿ-ಸಿ-ದರೆ
ಆಶಿ-ಸಿ-ದಳು
ಆಶಿಸು
ಆಶಿ-ಸು-ತ್ತಿತ್ತು
ಆಶಿ-ಸು-ತ್ತಿದ್ದ
ಆಶಿ-ಸು-ವನು
ಆಶ್ಚ-ರ್ಯ-ಚ-ಕಿ-ತ-ರಾ-ದರು
ಆಶ್ಚ-ರ್ಯ-ದಿಂದ
ಆಶ್ಚ-ರ್ಯ-ಪ-ಟ್ಟರು
ಆಶ್ಚ-ರ್ಯ-ಭ-ರಿತ
ಆಶ್ಚ-ರ್ಯ-ವಾಗಿ
ಆಶ್ಚ-ರ್ಯ-ವಾ-ಗಿದೆ
ಆಶ್ಚ-ರ್ಯ-ವಾ-ಯಿತು
ಆಶ್ಚ-ರ್ಯ-ವಾಯ್ತು
ಆಶ್ಚ-ರ್ಯ-ವಿಲ್ಲ
ಆಶ್ರಮ
ಆಶ್ರ-ಮಕ್ಕೆ
ಆಶ್ರ-ಮದ
ಆಶ್ರ-ಮ-ದ-ಲ್ಲಿದ್ದ
ಆಶ್ರ-ಯ-ವನ್ನು
ಆಸಕ್ತಿ
ಆಸ-ಕ್ತಿ-ಯಿಂದ
ಆಸ-ಕ್ತಿಯೂ
ಆಸ-ನ-ಗಳೂ
ಆಸ-ನ-ವನ್ನು
ಆಸೀ-ನ-ನಾ-ಗಿದ್ದ
ಆಸೆ
ಆಸೆ-ಗಳನ್ನೂ
ಆಸೆ-ಗ-ಳಿಗೆ
ಆಸೆ-ಗ-ಳಿ-ದ್ದವು
ಆಸೆಯ
ಆಸೆ-ಯಾಗಿ
ಆಸೆ-ಯಿಂದ
ಆಸೆ-ಯೇನೋ
ಆಸ್ತಿ
ಆಸ್ತಿ-ಯಾ-ಗಿತ್ತು
ಆಸ್ಥಾನ
ಆಸ್ಥಾ-ನಕ್ಕೆ
ಆಸ್ಪ-ದ-ವಿಲ್ಲ
ಆಸ್ವಾ-ದಿಸಿ
ಆಹಾ
ಆಹಾ-ರಕ್ಕೆ
ಆಹಾ-ರದ
ಆಹಾ-ರ-ವನ್ನು
ಆಹಾ-ರ-ವ-ಸ್ತು-ವಾ-ಗಿದ್ದೆ
ಆಹಾ-ರ-ವಾಗಿ
ಆಹ್ನಿಕ
ಆಹ್ನಿ-ಕ-ಗಳನ್ನು
ಆಹ್ನಿ-ಕ-ಗಳನ್ನೆಲ್ಲ
ಇಂಗಿ-ತಾರ್ಥ
ಇಂಗಿ-ಹೋದ
ಇಂಗ್ಲಿಷ್
ಇಂಗ್ಲೀಷ್
ಇಂತಹ
ಇಂತ-ಹುದೇ
ಇಂಥ
ಇಂದಿ
ಇಂದಿನ
ಇಂದಿ-ನಿಂದ
ಇಂದ್ರ
ಇಂದ್ರ-ಧ-ನು-ಸ್ಸನ್ನು
ಇಂದ್ರ-ನಿಗೆ
ಇಂದ್ರ-ನಿ-ಗೇಕೆ
ಇಂದ್ರಿಯ
ಇಂದ್ರಿ-ಯ-ಗಳು
ಇಂದ್ರಿ-ಯ-ಜಿ-ತನೂ
ಇಚ್ಛಾ-ಶಕ್ತಿ
ಇಚ್ಛಿ-ಸಿ-ದರು
ಇಚ್ಛಿ-ಸಿ-ದರೆ
ಇಚ್ಛಿ-ಸು-ವರು
ಇಚ್ಛಿ-ಸು-ವು-ದಿ-ಲ್ಲವೆ
ಇಚ್ಛೆ
ಇಚ್ಛೆಗೆ
ಇಚ್ಛೆ-ಪ-ಟ್ಟರೂ
ಇಚ್ಛೆ-ಪ-ಟ್ಟರೆ
ಇಚ್ಛೆಯ
ಇಚ್ಛೆ-ಯಂತೆ
ಇಚ್ಛೆ-ಯನ್ನು
ಇಚ್ಛೆ-ಯಿಂದ
ಇಚ್ಛೆ-ಯಿಂ-ದಲೇ
ಇಟ್ಟ
ಇಟ್ಟಳು
ಇಟ್ಟಾಗ
ಇಟ್ಟಿ-ಗೆ-ಯಿಂದ
ಇಟ್ಟಿದ್ದ
ಇಟ್ಟಿ-ದ್ದನೊ
ಇಟ್ಟಿ-ದ್ದರು
ಇಟ್ಟಿ-ದ್ದರೆ
ಇಟ್ಟಿ-ದ್ದಳು
ಇಟ್ಟಿ-ರ-ಬೇಕೋ
ಇಟ್ಟಿ-ರುವ
ಇಟ್ಟಿ-ರು-ವನು
ಇಟ್ಟಿ-ರು-ವೆನು
ಇಟ್ಟು
ಇಟ್ಟು-ಕೊಂ-ಡಳು
ಇಟ್ಟು-ಕೊಂ-ಡಿದೆ
ಇಟ್ಟು-ಕೊಂ-ಡಿದ್ದ
ಇಟ್ಟು-ಕೊಂ-ಡಿ-ದ್ದಾಳೆ
ಇಟ್ಟು-ಕೊ-ಳ್ಳ-ಕೂ-ಡದು
ಇಟ್ಟು-ಕೊ-ಳ್ಳ-ಬಾ-ರದು
ಇಡ-ಬೇ-ಕಾ-ಗಿಲ್ಲ
ಇಡೀ
ಇಡುತ್ತ
ಇಡು-ತ್ತದೆ
ಇಡು-ತ್ತಾನೆ
ಇಡು-ವೆನು
ಇಣಿಕಿ
ಇಣುಕು
ಇತರ
ಇತ-ರರ
ಇತ-ರ-ರಂತೆ
ಇತ-ರ-ರಿಗೆ
ಇತ-ರರು
ಇತ-ರರೂ
ಇತ-ರ-ರೊ-ಡನೆ
ಇತ್ತ
ಇತ್ತಿ-ರ-ಲಿಲ್ಲ
ಇತ್ತೀ-ಚೆಗೆ
ಇತ್ತು
ಇತ್ತು-ಶ್ರೀ-ಕೃ-ಷ್ಣನು
ಇತ್ಯಾ-ದಿ-ಗಳು
ಇದ
ಇದ-ಕ್ಕಾಗಿ
ಇದಕ್ಕೆ
ಇದ-ಕ್ಕೆಲ್ಲ
ಇದನ್ನು
ಇದ-ನ್ನೆಂ-ದಿಗೂ
ಇದ-ನ್ನೆಲ್ಲ
ಇದರ
ಇದ-ರಂತೆ
ಇದ-ರಂ-ತೆಯೆ
ಇದ-ರಂ-ತೆಯೇ
ಇದ-ರಲ್ಲಿ
ಇದ-ರ-ಲ್ಲೇನೂ
ಇದ-ರಿಂದ
ಇದಲ್ಲ
ಇದಾದ
ಇದು
ಇದು-ವ-ರೆಗೆ
ಇದೆ
ಇದೆಂ-ತಹ
ಇದೆ-ಯಂತೆ
ಇದೆ-ಯಲ್ಲ
ಇದೆ-ಯ-ಲ್ಲಯ್ಯ
ಇದೆಯೆ
ಇದೆಯೇ
ಇದೆಯೊ
ಇದೆಯೋ
ಇದೆಲ್ಲ
ಇದೇ
ಇದೇನು
ಇದೇನೂ
ಇದೇನೆ
ಇದೊಂದು
ಇದೊಂದೆ
ಇದ್ದ
ಇದ್ದಂತೆ
ಇದ್ದಂ-ತೆಯೇ
ಇದ್ದಕ್ಕಿ
ಇದ್ದ-ಕ್ಕಿ-ದ್ದಂತೆ
ಇದ್ದ-ದ್ದ-ನ್ನೆಲ್ಲ
ಇದ್ದನು
ಇದ್ದರು
ಇದ್ದರೂ
ಇದ್ದರೆ
ಇದ್ದಳು
ಇದ್ದ-ವರು
ಇದ್ದವು
ಇದ್ದಾಗ
ಇದ್ದಾನೆ
ಇದ್ದಾರೆ
ಇದ್ದಿತು
ಇದ್ದು
ಇದ್ದು-ದನ್ನು
ಇದ್ದು-ದ-ರಿಂದ
ಇದ್ದೆ
ಇದ್ರೆ
ಇನ್ನಷ್ಟು
ಇನ್ನಾ-ರದೋ
ಇನ್ನು
ಇನ್ನು-ಮೇಲೆ
ಇನ್ನೂ
ಇನ್ನೂರು
ಇನ್ನೇನು
ಇನ್ನೇನೂ
ಇನ್ನೊ
ಇನ್ನೊಂದು
ಇನ್ನೊಂ-ದೆಡೆ
ಇನ್ನೊಬ್ಬ
ಇನ್ನೊ-ಬ್ಬ-ನಿಗೆ
ಇನ್ನೊ-ಬ್ಬ-ರನ್ನು
ಇನ್ನೊ-ಬ್ಬ-ರಿಗೆ
ಇನ್ನೊ-ಬ್ಬರು
ಇನ್ನೊಮ್ಮೆ
ಇಪ್ಪ-ತ್ತ-ನಾಲ್ಕು
ಇಪ್ಪ-ತ್ತು-ನಾಲ್ಕು
ಇಪ್ಪ-ತ್ತೈದು
ಇಬ್ಬ-ರನ್ನೂ
ಇಬ್ಬ-ರಾಗಿ
ಇಬ್ಬರು
ಇಬ್ಬರೂ
ಇಬ್ಭಾಗ
ಇಬ್ಭಾ-ಗ-ವಾಗಿ
ಇರ-ಕೂ-ಡದು
ಇರ-ಬ-ಹುದು
ಇರ-ಬೇಕು
ಇರ-ಲಾ-ರವು
ಇರ-ಲಾರೆ
ಇರಲಿ
ಇರ-ಲಿಲ್ಲ
ಇರಲು
ಇರು
ಇರು-ತ್ತದೆ
ಇರು-ತ್ತ-ದೆಂ-ದರೆ
ಇರು-ತ್ತ-ದೆಯೇ
ಇರು-ತ್ತಾನೆ
ಇರು-ತ್ತಾರೆ
ಇರು-ತ್ತಿತ್ತು
ಇರು-ತ್ತಿದ್ದ
ಇರು-ತ್ತಿದ್ದೆ
ಇರು-ತ್ತಿ-ರ-ಲಿಲ್ಲ
ಇರುವ
ಇರು-ವನು
ಇರು-ವನೊ
ಇರು-ವರು
ಇರು-ವ-ವ-ರನ್ನು
ಇರುವು
ಇರು-ವು-ದಕ್ಕೆ
ಇರು-ವು-ದ-ರಿಂದ
ಇರು-ವು-ದಿಲ್ಲ
ಇರು-ವುದು
ಇರು-ವು-ದೆಲ್ಲ
ಇರುವೆ
ಇರು-ವೆ-ಗ-ಳಂತೆ
ಇರು-ವೆ-ನಲ್ಲ
ಇರು-ವೆನು
ಇರು-ವೆಯೊ
ಇರು-ವೆ-ಯೊಂದು
ಇರು-ವೆವೊ
ಇಲಿ-ಗಳು
ಇಲಿಯ
ಇಲ್ಲ
ಇಲ್ಲದ
ಇಲ್ಲ-ದಿ-ದ್ದರೆ
ಇಲ್ಲದೆ
ಇಲ್ಲ-ವಲ್ಲ
ಇಲ್ಲವೆ
ಇಲ್ಲ-ವೆ-ನ್ನ-ಬ-ಹುದು
ಇಲ್ಲವೇ
ಇಲ್ಲವೊ
ಇಲ್ಲಿ
ಇಲ್ಲಿಂದ
ಇಲ್ಲಿಗೆ
ಇಲ್ಲಿದೆ
ಇಲ್ಲಿ-ದ್ದೇನೆ
ಇಲ್ಲಿ-ಯ-ವ-ರೆಗೆ
ಇಲ್ಲಿ-ರುವ
ಇಲ್ಲಿವೆ
ಇಲ್ಲೇ
ಇಳಿದ
ಇಳಿ-ದರು
ಇಳಿ-ದರೆ
ಇಳಿದಾ
ಇಳಿ-ದಾಗ
ಇಳಿದು
ಇಳಿ-ಯು-ತ್ತಿ-ದ್ದಂತೆ
ಇಳಿ-ವ-ಯ-ಸ್ಸಿ-ನಲ್ಲಿ
ಇಳಿ-ಸಿದ
ಇವ-ತ್ತಿನ
ಇವ-ತ್ತಿ-ನಿಂ-ದಲೇ
ಇವತ್ತು
ಇವತ್ತೇ
ಇವನ
ಇವ-ನನ್ನು
ಇವ-ನಿಂದ
ಇವ-ನಿಗೆ
ಇವನು
ಇವನೆ
ಇವನೇ
ಇವರ
ಇವ-ರನ್ನು
ಇವ-ರಲ್ಲ
ಇವ-ರಲ್ಲಿ
ಇವ-ರಿಗೆ
ಇವ-ರಿ-ಬ್ಬ-ರಲ್ಲಿ
ಇವರು
ಇವರೆಲ್ಲ
ಇವರೇ
ಇವಳ
ಇವ-ಳನ್ನು
ಇವು
ಇವು-ಗಳ
ಇವು-ಗಳನ್ನು
ಇವು-ಗಳನ್ನೆಲ್ಲ
ಇವು-ಗಳಲ್ಲಿ
ಇವು-ಗ-ಳಾ-ವುದೂ
ಇವು-ಗಳು
ಇವು-ಗ-ಳೆಲ್ಲ
ಇವೆ
ಇವೆಯೋ
ಇವೆ-ರ-ಡನ್ನೂ
ಇವೆಲ್ಲ
ಇಷ್ಟಕ್ಕೇ
ಇಷ್ಟ-ದ-ಮೇಲೆ
ಇಷ್ಟ-ದೇ-ವತೆ
ಇಷ್ಟ-ದೇ-ವರ
ಇಷ್ಟ-ದೇ-ವ-ರಲ್ಲಿ
ಇಷ್ಟ-ರಿಂದ
ಇಷ್ಟ-ವಾ-ದರೆ
ಇಷ್ಟು
ಇಷ್ಟೆ
ಇಷ್ಟೇ
ಇಷ್ಟೇನೆ
ಇಷ್ಟೊಂದು
ಈ
ಈಕೆ
ಈಗ
ಈಗಲೂ
ಈಗಲೆ
ಈಗಲೇ
ಈಜಿ-ಕೊಂಡು
ಈಜು
ಈಡಾದ
ಈಡು-ಮಾಡು
ಈಡೇ-ರಿ-ಸಿ-ಕೊಡು
ಈಡೇ-ರು-ವುದು
ಈತ
ಈತನ
ಉಂಟಾ-ಗು-ವುದು
ಉಂಟು-ಮಾ-ಡು-ತ್ತದೆ
ಉಂಟು-ಮಾ-ಡು-ತ್ತಾರೆ
ಉಂಡಿದ್ದ
ಉಂಡೆ-ಯನ್ನು
ಉಕ್ಕು-ತ್ತಿತ್ತು
ಉಗು-ರಿನ
ಉಗು-ರಿ-ನಿಂದ
ಉಗ್ರ-ವಾ-ದ-ವು-ಗಳು
ಉಚ್ಚ-ರಿ-ಸ-ತೊ-ಡ-ಗಿ-ದರು
ಉಚ್ಚ-ರಿಸಿ
ಉಚ್ಚ-ರಿ-ಸಿದ
ಉಚ್ಚ-ರಿ-ಸಿ-ದರೆ
ಉಚ್ಚ-ರಿ-ಸಿದೆ
ಉಚ್ಚ-ರಿಸು
ಉಚ್ಚ-ರಿ-ಸು-ತ್ತಾರೆ
ಉಚ್ಚ-ರಿ-ಸು-ತ್ತಿ-ದ್ದು-ದ-ರಿಂದ
ಉಚ್ಚ-ರಿ-ಸು-ವು-ದಕ್ಕೆ
ಉಚ್ಚಾರ
ಉಣ-ಬ-ಡಿ-ಸ-ಲಾಗಿದೆ
ಉಣ-ಬ-ಡಿ-ಸಿ-ದಳು
ಉಣಿ-ಸಿ-ದನು
ಉಣಿ-ಸು-ವುದನ್ನು
ಉತ್ಕಟ
ಉತ್ಕೃಷ್ಟ
ಉತ್ತ-ಮ-ನಾ-ಗು-ತ್ತಾನೆ
ಉತ್ತ-ಮ-ವಾದ
ಉತ್ತರ
ಉತ್ತ-ರ-ಕೊ-ಟ್ಟಳು
ಉತ್ತ-ರಕ್ಕೆ
ಉತ್ತ-ರದ
ಉತ್ತ-ರ-ವನ್ನು
ಉತ್ತ-ರವೂ
ಉತ್ತ-ರಿ-ಸಿ-ದನು
ಉದಾರ
ಉದಾ-ಹ-ರಣೆ
ಉದಾ-ಹ-ರ-ಣೆ-ಯನ್ನು
ಉದಾ-ಹ-ರಿಸಿ
ಉದು-ರಿ-ತೋ-ಯಾರು
ಉದ್ಗ
ಉದ್ಗ-ರಿ-ಸಿದ
ಉದ್ಗ-ರಿ-ಸಿ-ದನು
ಉದ್ದೇಶ
ಉದ್ದೇ-ಶವೇ
ಉದ್ದೇ-ಶ-ವೇನು
ಉದ್ದೇ-ಶಿಸಿ
ಉದ್ಧಾ-ರ-ವಾ-ಗು-ವುದು
ಉದ್ಭ-ವಿ-ಸಿತು
ಉದ್ಯಾನ
ಉನ್ನತ
ಉನ್ನ-ತ-ವಾ-ಗಿ-ದ್ದಲ್ಲಿ
ಉನ್ನತಿ
ಉನ್ಮ-ತ್ತ-ರಂತೆ
ಉಪ
ಉಪ-ಕಾರ
ಉಪ-ಕಾ-ರ-ವನ್ನು
ಉಪ-ಕ್ರ-ಮಿ-ಸಿದ
ಉಪ-ಗು-ರು-ಗ-ಳಾ-ಗು-ತ್ತಾರೆ
ಉಪ-ಗು-ರು-ಗಳು
ಉಪ-ಚ-ರಿ-ಸಿ-ದಳು
ಉಪ-ಚಾ-ರ-ವೆಲ್ಲ
ಉಪ-ದೇಶ
ಉಪ-ದೇ-ಶ-ಗಳಲ್ಲಿ
ಉಪ-ದೇ-ಶವು
ಉಪ-ದೇ-ಶಾ-ಮೃತ
ಉಪ-ನ-ಯನ
ಉಪ-ನಿ-ಷತ್
ಉಪ-ನ್ಯಾಸ
ಉಪ-ನ್ಯಾ-ಸಕ
ಉಪ-ನ್ಯಾ-ಸಕ್ಕೆ
ಉಪ-ನ್ಯಾ-ಸ-ಗಳನ್ನು
ಉಪ-ನ್ಯಾ-ಸ-ದಿಂದ
ಉಪ-ಪತಿ
ಉಪ-ಪತ್ನಿ
ಉಪ-ಯೋಗಿ
ಉಪ-ಯೋ-ಗಿ-ಸದೆ
ಉಪ-ಯೋ-ಗಿ-ಸು-ತ್ತಾರೆ
ಉಪ-ಯೋ-ಗಿ-ಸುವ
ಉಪ-ವಾಸ
ಉಪ-ವಾ-ಸ-ದಿಂದ
ಉಪ-ಹಾರ
ಉಪಾಧಿ
ಉಪಾ-ಧಿ-ಗಳನ್ನು
ಉಪಾ-ಧಿ-ಗಳಲ್ಲಿ
ಉಪಾ-ಧಿ-ಗಳು
ಉಪಾಯ
ಉಪಾ-ಯ-ವನ್ನು
ಉಪಾ-ಯ-ವಿದೆ
ಉಪ್ಪಿನ
ಉಪ್ಪಿ-ನ-ಗೊಂಬೆ
ಉಬ್ಬ-ರದ
ಉಮಾ
ಉರಿ-ಯು-ತ್ತಿ-ರುವ
ಉಳಿ-ತಾಯ
ಉಳಿದ
ಉಳಿ-ದರೆ
ಉಳಿ-ದಿ-ರು-ವುದನ್ನು
ಉಳಿ-ದಿ-ರು-ವು-ದೆಲ್ಲ
ಉಳಿದು
ಉಳಿ-ದು-ದನ್ನೂ
ಉಳಿ-ದು-ದೆಲ್ಲ
ಉಳಿ-ದೆ-ಲ್ಲವೂ
ಉಳಿ-ಯು-ವುದು
ಉಳಿ-ಸಪ್ಪ
ಉಳಿ-ಸ-ಬ-ಹು-ದಿತ್ತು
ಉಳಿ-ಸಲು
ಉಳಿಸಿ
ಉಳಿ-ಸಿ-ಕೊ-ಳ್ಳು-ವು-ದಕ್ಕೆ
ಉಳಿಸು
ಉಳು-ತ್ತಾರೆ
ಉಳುವ
ಉಳು-ವು-ದಕ್ಕೆ
ಉಸಿ-ರಾ-ದಾಗ
ಊಟ
ಊಟಕ್ಕೆ
ಊಟದ
ಊಟ-ದಿಂದ
ಊಟ-ಮಾಡಿ
ಊಟ-ಮಾ-ಡಿದೆ
ಊಟ-ವನ್ನು
ಊಟ-ವನ್ನೂ
ಊಟ-ವಾದ
ಊದಿನ
ಊದು-ತ್ತಿ-ದ್ದನು
ಊದು-ತ್ತಿ-ರುವೆ
ಊರನ್ನು
ಊರ-ಲ್ಲೆಲ್ಲ
ಊರಿಗೆ
ಊರಿನ
ಊರಿ-ನಲ್ಲಿ
ಊರಿ-ನ-ಲ್ಲೆಲ್ಲ
ಊರಿ-ನಿಂದ
ಊಸ-ರ-ವಳ್ಳಿ
ಊಸ-ರ-ವ-ಳ್ಳಿಯು
ಊಹಿ
ಊಹಿಸಿ
ಊಹಿ-ಸಿ-ದ್ದರು
ಊಹಿ-ಸಿಯೆ
ಎಂಜ-ಲನ್ನೇ
ಎಂಜಲು
ಎಂಟು
ಎಂತಲೂ
ಎಂತಹ
ಎಂತ-ಹ-ವರು
ಎಂಥ
ಎಂಥದು
ಎಂಥ-ವನು
ಎಂದ
ಎಂದನು
ಎಂದ-ರ-ವರು
ಎಂದ-ರಿತು
ಎಂದ-ರಿ-ಯು-ವುದು
ಎಂದರು
ಎಂದರೆ
ಎಂದಳು
ಎಂದ-ವನು
ಎಂದಾಗ
ಎಂದಾ-ದರೂ
ಎಂದಿ-ಗಿಂತ
ಎಂದಿಗೂ
ಎಂದಿ-ಟ್ಟು-ಕೊಳ್ಳಿ
ಎಂದಿತು
ಎಂದಿ-ದ್ದರು
ಎಂದಿ-ದ್ದಾರೆ
ಎಂದಿದ್ದು
ಎಂದಿ-ನಂತೆ
ಎಂದಿ-ರಲ್ಲ
ಎಂದಿ-ರು-ವ-ರು-ಎಂದು
ಎಂದಿ-ರು-ವಾಗ
ಎಂದು
ಎಂದು-ಕೊಂ-ಡಿದ್ದ
ಎಂದು-ಕೊಂಡು
ಎಂದು-ಶ್ರೀ-ಕೃ-ಷ್ಣನು
ಎಂದೂ
ಎಂದೆ
ಎಂದೆಂ-ದಿಗೂ
ಎಂದೆನು
ಎಂದೆನ್ನು
ಎಂದೆ-ನ್ನು-ತ್ತಿ-ದ್ದನು
ಎಂಬ
ಎಂಬಂತೆ
ಎಂಬು-ದ-ನ್ನ-ರಿ-ಯಲು
ಎಂಬು-ದನ್ನು
ಎಂಬು-ದನ್ನೂ
ಎಂಬುದು
ಎಂಬು-ವರು
ಎಕ್ಕಡ
ಎಚ್ಚ-ರ-ವಾ-ಯಿತು
ಎಚ್ಚ-ರಿ-ಕೆ-ಯಿಂ-ದಿ-ರ-ಬೇಕು
ಎಚ್ಚ-ರಿ-ಸಿದ
ಎಡ-ಗ-ಡೆಗೆ
ಎಡ-ಗ-ಡೆ-ಯಿಂದ
ಎಡ-ವು-ತ್ತಾರೆ
ಎಡೆಗೆ
ಎಡೆಯೇ
ಎಣಿ-ಸ-ಬೇಡಿ
ಎಣಿ-ಸು-ವು-ದಕ್ಕೆ
ಎಣಿ-ಸು-ವು-ದ-ರಲ್ಲಿ
ಎಣ್ಣೆ
ಎಣ್ಣೆಯ
ಎಣ್ಣೆ-ಯನ್ನು
ಎಣ್ಣೆಯೂ
ಎತ್ತನ್ನು
ಎತ್ತ-ರ-ದ-ಲ್ಲಿ-ರು-ತ್ತದೆ
ಎತ್ತ-ರ-ವಾ-ಗಿತ್ತು
ಎತ್ತ-ರ-ವಾ-ಗಿ-ರುವ
ಎತ್ತ-ರ-ವಾದ
ಎತ್ತ-ಲಿಲ್ಲ
ಎತ್ತಿ
ಎತ್ತಿ-ಕೊಂಡು
ಎತ್ತಿದ
ಎತ್ತಿನ
ಎತ್ತು
ಎತ್ತು-ಗಳು
ಎತ್ತು-ತ್ತಿ-ದ್ದರು
ಎತ್ತು-ತ್ತಿ-ದ್ದಾಗ
ಎತ್ತು-ತ್ತಿ-ರ-ಲಿಲ್ಲ
ಎತ್ತು-ತ್ತಿ-ಲ್ಲ-ದು-ದನ್ನು
ಎದು-ರಿ-ಗಿ-ರುವ
ಎದು-ರಿಗೆ
ಎದು-ರು-ಗ-ಡೆ-ಯಿಂದ
ಎದೆ
ಎದೆಯ
ಎದೆ-ಯನ್ನು
ಎದೆ-ಯೊ-ಳಗೆ
ಎದ್ದರು
ಎದ್ದರೆ
ಎದ್ದಿತು
ಎದ್ದಿ-ದೆಯೆ
ಎದ್ದು
ಎದ್ದು-ನಿಂತ
ಎದ್ದೊ-ಡ-ನೆಯೆ
ಎನಗೆ
ಎನ್ನ
ಎನ್ನ-ಬ-ಹು-ದಲ್ಲ
ಎನ್ನ-ಲಿಲ್ಲ
ಎನ್ನಿ-ಸ-ಲಿಲ್ಲ
ಎನ್ನಿ-ಸಿತು
ಎನ್ನು
ಎನ್ನುತ್ತ
ಎನ್ನುತ್ತಾ
ಎನ್ನು-ತ್ತಾನೆ
ಎನ್ನು-ತ್ತಾರೆ
ಎನ್ನುತ್ತಿ
ಎನ್ನು-ತ್ತಿದ್ದ
ಎನ್ನು-ತ್ತಿ-ದ್ದ-ನಲ್ಲ
ಎನ್ನು-ತ್ತಿ-ದ್ದನು
ಎನ್ನು-ತ್ತಿ-ದ್ದ-ವನ
ಎನ್ನು-ತ್ತಿ-ದ್ದ-ವನು
ಎನ್ನು-ತ್ತಿ-ದ್ದೀಯ
ಎನ್ನು-ತ್ತೀಯೆ
ಎನ್ನುವ
ಎನ್ನು-ವನು
ಎನ್ನು-ವರು
ಎನ್ನು-ವಲ್ಲಿ
ಎನ್ನು-ವ-ವರು
ಎನ್ನು-ವಾಗ
ಎನ್ನು-ವಿರೊ
ಎನ್ನು-ವುದನ್ನು
ಎನ್ನು-ವು-ದರ
ಎನ್ನು-ವು-ದ-ರಲ್ಲಿ
ಎನ್ನು-ವುದು
ಎನ್ನು-ವುದೂ
ಎಪ್ಪ-ತ್ತೆ-ರಡು
ಎಪ್ಪ-ತ್ತೈದು
ಎಬ್ಬಿ-ಸಿ-ದ-ವನು
ಎಬ್ಬಿ-ಸಿ-ದಿರಿ
ಎಬ್ಬಿ-ಸಿ-ದು-ದ-ರಿಂದ
ಎಬ್ಬಿ-ಸಿ-ದೆ-ಯಲ್ಲ
ಎಬ್ಬಿ-ಸು-ವಂ-ತಹ
ಎರ-ಗಿತು
ಎರ-ಚಿದೆ
ಎರ-ಡನೆ
ಎರ-ಡ-ನೆ-ಯದು
ಎರ-ಡ-ನೆ-ಯ-ವನು
ಎರ-ಡನ್ನು
ಎರ-ಡನ್ನೂ
ಎರ-ಡ-ರಂತೆ
ಎರ-ಡ-ರಷ್ಟು
ಎರ-ಡಾಣೆ
ಎರ-ಡಾ-ಣೆ-ಯನ್ನು
ಎರಡು
ಎರಡೂ
ಎರೆ-ಹು-ಳ-ದಂತೆ
ಎರೆ-ಹು-ಳು-ಗ-ಳಂತೆ
ಎಲೆ
ಎಲೆ-ಗಳನ್ನು
ಎಲೆ-ಗಳು
ಎಲೆಯ
ಎಲೈ
ಎಲ್ಲ
ಎಲ್ಲಕ್ಕೂ
ಎಲ್ಲ-ದಕ್ಕೂ
ಎಲ್ಲ-ದರ
ಎಲ್ಲ-ರಂತೆ
ಎಲ್ಲ-ರ-ಲ್ಲಿಯೂ
ಎಲ್ಲ-ರಿಗೂ
ಎಲ್ಲರೂ
ಎಲ್ಲ-ರೆ-ದು-ರಿಗೆ
ಎಲ್ಲ-ವನ್ನು
ಎಲ್ಲ-ವನ್ನೂ
ಎಲ್ಲವೂ
ಎಲ್ಲಾ
ಎಲ್ಲಿ
ಎಲ್ಲಿಂದ
ಎಲ್ಲಿಂ-ದ-ಲಾ-ದರೂ
ಎಲ್ಲಿಗೆ
ಎಲ್ಲಿದೆ
ಎಲ್ಲಿ-ಯ-ವ-ರೆಗೆ
ಎಲ್ಲಿ-ಯಾ-ದರೂ
ಎಲ್ಲಿಯೂ
ಎಲ್ಲಿ-ರು-ವನು
ಎಲ್ಲಿ-ರು-ವನೋ
ಎಲ್ಲೂ
ಎಲ್ಲೆ-ಲ್ಲಿಯೂ
ಎಲ್ಲೆಲ್ಲೋ
ಎಲ್ಲೊ
ಎಲ್ಲೋ
ಎಳೆ
ಎಳೆದು
ಎಳೆ-ದು-ಕೊಂಡು
ಎಳೆ-ದು-ಕೊಂ-ಡು-ಹೋಗಿ
ಎಳೆಯ
ಎಳೆ-ಯಲು
ಎಳ್ಳಷ್ಟೂ
ಎಷ್ಟನ್ನು
ಎಷ್ಟಾ-ದರೂ
ಎಷ್ಟು
ಎಷ್ಟೇ
ಎಷ್ಟೊಂದು
ಎಷ್ಟೋ
ಎಸೆದ
ಎಸೆ-ದನು
ಎಸೆ-ದರು
ಎಸೆ-ದಿ-ರು-ವೆನು
ಎಸೆದು
ಎಸೆ-ಯ-ಬೇಕು
ಎಸೆ-ಯು-ವರು
ಏ
ಏಕ-ಕಾ-ಲ-ದಲ್ಲೇ
ಏಕ-ನಿ-ಷ್ಠಾ-ಭಕ್ತಿ
ಏಕ-ನಿ-ಷ್ಠೆ-ಯಿಂದ
ಏಕ-ಮಾತ್ರ
ಏಕ-ಮೇವ
ಏಕಾ
ಏಕಾಗ್ರ
ಏಕಾ-ಗ್ರತೆ
ಏಕಾ-ಗ್ರ-ವಾ-ಗಲಿ
ಏಕೆ
ಏಕೆಂ-ದರೆ
ಏತಕ್ಕೆ
ಏತಕ್ಕೋ
ಏನ-ನ್ನಾ-ದರೂ
ಏನನ್ನು
ಏನನ್ನೂ
ಏನನ್ನೊ
ಏನನ್ನೋ
ಏನಯ್ಯ
ಏನಾ-ಗ-ಬೇಕು
ಏನಾ-ಗಿದೆ
ಏನಾ-ಗಿ-ರು-ವನೊ
ಏನಾ-ಗು-ತ್ತಿದೆ
ಏನಾ-ಗು-ವುದೊ
ಏನಾ-ದನೋ
ಏನಾ-ದರೂ
ಏನಿದೆ
ಏನಿ-ದೆಯೊ
ಏನಿ-ರ-ಬ-ಹುದು
ಏನು
ಏನೂ
ಏನೆಂ-ದರೆ
ಏನೆಂ-ಬು-ದನ್ನು
ಏನೆಂ-ಬುದು
ಏನೇ
ಏನೇನು
ಏನೇನೂ
ಏನೇನೊ
ಏನೇನೋ
ಏನೊ
ಏನೋ
ಏಯ್
ಏಳನೆ
ಏಳನೇ
ಏಳ-ಬ-ಹುದು
ಏಳು
ಏಶಿ-ಯಾ-ಟಿಕ್
ಐಕ್ಯ-ತೆಯ
ಐದಾರು
ಐದು
ಐಶ್ವರ್ಯ
ಐಶ್ವ-ರ್ಯಕ್ಕೆ
ಐಶ್ವ-ರ್ಯ-ದೇ-ವ-ತೆಗೆ
ಐಶ್ವ-ರ್ಯ-ವಂ-ತ-ನಾದ
ಒಂಟೆ
ಒಂದನ್ನು
ಒಂದನ್ನೇ
ಒಂದರ
ಒಂದಾ-ಗು-ವುದನ್ನು
ಒಂದಾ-ದಾಗ
ಒಂದಾ-ನೊಂದು
ಒಂದು
ಒಂದು-ದಿನ
ಒಂದೂ-ರಿಗೆ
ಒಂದೂ-ರಿ-ನಲ್ಲಿ
ಒಂದೇ
ಒಂದೊಂದು
ಒಂಬ-ತ್ತು-ನೂರು
ಒಂಬೈ-ನೂರು
ಒಂಭತ್ತು
ಒಗೆ-ಯಿರಿ
ಒಟ್ಟಿಗೆ
ಒಟ್ಟು
ಒಡನೆ
ಒಡ-ನೆಯೆ
ಒಡ-ವೆ-ಗಳನ್ನೆಲ್ಲ
ಒಡ-ವೆ-ಗ-ಳಿ-ಗಾಗಿ
ಒಡ-ವೆ-ಯನ್ನು
ಒಡೆ
ಒಡೆದು
ಒಡೆ-ದು-ಹಾಕಿ
ಒಡೆಯ
ಒಡೆ-ಯದೆ
ಒಡೆ-ಯನ
ಒಡೆ-ಯರು
ಒಡೆ-ಯುವ
ಒಡೆ-ಯು-ವನು
ಒಡೆ-ಯು-ವವ
ಒಡೆ-ಯು-ವ-ವ-ನಾ-ಗಿ-ದ್ದುದು
ಒಡೆ-ಯು-ವ-ವ-ನಿಗೆ
ಒಡೆ-ಯು-ವ-ವನು
ಒಡೆ-ಯು-ವುದು
ಒಡೆ-ಯು-ವು-ದೆಂದು
ಒಡ್ಡ
ಒಣ
ಒಣ-ಗಿದ
ಒಣ-ಗಿ-ಸಲು
ಒಣ-ಗಿ-ಹೋ-ಗಿ-ದ್ದರೂ
ಒಣ-ಗಿ-ಹೋ-ಗು-ತ್ತಿದೆ
ಒತ್ತ-ತೊ-ಡ-ಗಿದ
ಒತ್ತ-ತೊ-ಡ-ಗಿ-ದಳು
ಒತ್ತಿ-ಹಿ-ಡಿ-ದು-ಕೊಂಡು
ಒತ್ತೆ
ಒದಗಿ
ಒದೆಯು
ಒದ್ದು-ಬಿ-ಟ್ಟೆ-ಯಲ್ಲ
ಒದ್ದೆ-ಯಾ-ಗ-ದಿ-ರಲಿ
ಒದ್ದೆ-ಯಾ-ಗಿ-ದ್ದವು
ಒಪ್ಪ-ಬೇ-ಕಾ-ಗಿದೆ
ಒಪ್ಪ-ಲಿಲ್ಲ
ಒಪ್ಪ-ಲೇ-ಬೇಕು
ಒಪ್ಪಿ
ಒಪ್ಪಿ-ಕೊಳ್ಳಿ
ಒಪ್ಪಿಗೆ
ಒಬ್ಬ
ಒಬ್ಬನ
ಒಬ್ಬ-ನನ್ನು
ಒಬ್ಬ-ನ-ಲ್ಲಿ-ದ್ದರೆ
ಒಬ್ಬ-ನಿಗೆ
ಒಬ್ಬ-ನಿದ್ದ
ಒಬ್ಬನು
ಒಬ್ಬನೂ
ಒಬ್ಬನೇ
ಒಬ್ಬ-ರಾದ
ಒಬ್ಬರು
ಒಬ್ಬಳು
ಒಬ್ಬೊಬ್ಬ
ಒಬ್ಬೊ-ಬ್ಬ-ರ-ನ್ನಾಗಿ
ಒಮ್ಮೆ
ಒಯ್ದರು
ಒಯ್ಯು-ತ್ತಿ-ರು-ವರು
ಒರ-ಸಿ-ಕೊಂಡು
ಒರೆ-ಯ-ಲ್ಲಿ-ರು-ವಂತೆ
ಒಲಿ-ದೆ-ಯಲ್ಲ
ಒಲಿ-ಸಿ-ಕೊ-ಳ್ಳು-ವು-ದಕ್ಕೆ
ಒಳ-ಒ-ಳಗೆ
ಒಳ-ಗಾ-ಗಿ-ರುವೆ
ಒಳ-ಗಾ-ಗು-ವುದು
ಒಳ-ಗಿ-ನಿಂದ
ಒಳ-ಗಿ-ರು-ವುದನ್ನು
ಒಳಗೆ
ಒಳಿತು
ಒಳ್ಳೆಯ
ಒಳ್ಳೆ-ಯ-ದನ್ನು
ಒಳ್ಳೆ-ಯ-ದಲ್ಲ
ಒಳ್ಳೆ-ಯ-ದಾ-ಗಿದೆ
ಒಳ್ಳೆ-ಯದೆ
ಒಳ್ಳೆ-ಯ-ವ-ನಲ್ಲಿ
ಒಳ್ಳೆ-ಯ-ವನು
ಒಳ್ಳೆ-ಯ-ವ-ರೊ-ಡನೆ
ಒಳ್ಳೆ-ಯವೂ
ಒಳ್ಳೇ
ಓ
ಓಟ-ಕಿತ್ತ
ಓಡಿ
ಓಡಿ-ಬಂದು
ಓಡಿ-ಹೋ-ಗ-ಬೇಕು
ಓಡಿ-ಹೋ-ಗು-ತ್ತಿ-ರು-ವಳು
ಓಡಿ-ಹೋ-ಗು-ತ್ತೇನೆ
ಓಡಿ-ಹೋ-ಗು-ವಂತೆ
ಓಡಿ-ಹೋ-ಗು-ವೆನು
ಓಡಿ-ಹೋದ
ಓಡಿ-ಹೋ-ದರು
ಓಡು
ಓದಲು
ಓದಿ
ಓದಿದ
ಓದಿ-ದರೂ
ಓದಿ-ದರೆ
ಓದಿ-ದ-ವರು
ಓದಿ-ರುವೆ
ಓದಿ-ರು-ವೆನು
ಓದಿಲ್ಲ
ಓದಿ-ಲ್ಲವೆ
ಓದು
ಓದು-ಗರ
ಓದು-ಗ-ರಿಗೆ
ಓದು-ತ್ತಿ-ರು-ವಂತೆ
ಓರ್ವ
ಓಹೋ
ಔಷ-ಧ-ಗಳ
ಔಷ-ಧದ
ಔಷ-ಧ-ದಿಂದ
ಔಷಧಿ
ಔಷ-ಧಿ-ಯನ್ನು
ಕಂಕಣ
ಕಂಕು-ಳಲ್ಲಿ
ಕಂಕು-ಳಿಂದ
ಕಂಗೆ-ಟ್ಟಿದ್ದ
ಕಂಠದ
ಕಂಡ
ಕಂಡ-ದ್ದ-ನ್ನೆಲ್ಲ
ಕಂಡನು
ಕಂಡರು
ಕಂಡರೆ
ಕಂಡಳು
ಕಂಡಾಗ
ಕಂಡಿತು
ಕಂಡಿ-ದ್ದೇನೆ
ಕಂಡಿಲ್ಲ
ಕಂಡಿ-ಲ್ಲದ
ಕಂಡು
ಕಂಡು-ಹಿಡಿ
ಕಂಡು-ಹಿ-ಡಿದ
ಕಂಡು-ಹಿ-ಡಿ-ದರು
ಕಂಡು-ಹಿ-ಡಿ-ಯಲು
ಕಂಡೆ
ಕಂಡೊ-ಡ-ನೆಯೆ
ಕಂತೆ
ಕಂತೆ-ಯ-ನ್ನೇನೊ
ಕಂದು
ಕಂಪಿ-ಸು-ವಂತೆ
ಕಂಬ-ಗಳ
ಕಂಬ-ದಂತೆ
ಕಂಬ-ದಲ್ಲಿ
ಕಂಬನಿ
ಕಂಬ-ಳಿ-ಹು-ಳು-ಗಳು
ಕಂಬ-ವನ್ನು
ಕಂಬಿ
ಕಕ್ಕಲೂ
ಕಚ್ಚ-ಬ-ಹುದು
ಕಚ್ಚ-ಬೇಡ
ಕಚ್ಚಲು
ಕಚ್ಚಿ-ಕೊಂ-ಡಿತು
ಕಚ್ಚಿ-ಕೊಂ-ಡಿದ್ದ
ಕಚ್ಚಿ-ಕೊಂಡು
ಕಚ್ಚಿತು
ಕಚ್ಚಿದ್ದು
ಕಚ್ಚು-ತ್ತಿ-ರುವೆ
ಕಚ್ಚು-ತ್ತಿಲ್ಲ
ಕಚ್ಚು-ವು-ದಿಲ್ಲ
ಕಟುಕ
ಕಟು-ಕ-ನಂತೆ
ಕಟು-ಕ-ನಿಗೆ
ಕಟು-ಕನು
ಕಟ್ಟ-ಬೇ-ಕಾ-ಯಿತು
ಕಟ್ಟಿ
ಕಟ್ಟಿ-ಕೊಂಡ
ಕಟ್ಟಿ-ಕೊಂಡಿ
ಕಟ್ಟಿ-ಕೊಂ-ಡಿದ್ದ
ಕಟ್ಟಿಗೆ
ಕಟ್ಟಿ-ಗೆ-ಯನ್ನು
ಕಟ್ಟಿದ
ಕಟ್ಟಿ-ದರು
ಕಟ್ಟಿದೆ
ಕಟ್ಟಿದ್ದ
ಕಟ್ಟಿ-ರುವೆ
ಕಟ್ಟಿ-ಸಿ-ಕೊಂಡ
ಕಟ್ಟಿ-ಸು-ವು-ದಕ್ಕಾ
ಕಟ್ಟಿ-ಹಾ-ಕಿದ
ಕಟ್ಟಿ-ಹಾ-ಕು-ವುದು
ಕಟ್ಟು-ತ್ತಾರೆ
ಕಟ್ಟು-ತ್ತಿದ್ದ
ಕಟ್ಟುವ
ಕಟ್ಟು-ವಂತೆ
ಕಟ್ಟು-ವು-ದಕ್ಕೆ
ಕಟ್ಟೆ
ಕಟ್ಟೆ-ಯನ್ನು
ಕಟ್ಟೆ-ಯೆಲ್ಲ
ಕಠಿಣ
ಕಠಿ-ಣ-ವಾದ
ಕಡಿ-ದು-ಹಾ-ಕಿ-ದನು
ಕಡಿಮೆ
ಕಡಿ-ಮೆ-ಮಾ-ಡು-ತ್ತಿ-ರು-ವನು
ಕಡಿ-ಮೆ-ಯಾ-ಗ-ಲಿಲ್ಲ
ಕಡಿ-ಮೆ-ಯಾ-ಗುತ್ತಾ
ಕಡಿ-ಮೆ-ಯಾ-ಗು-ವು-ದಿಲ್ಲ
ಕಡಿ-ಮೆ-ಯಾ-ದರೆ
ಕಡಿ-ಮೆ-ಯಾ-ದೊ-ಡನೆ
ಕಡಿ-ಯಲು
ಕಡೆ
ಕಡೆಗೂ
ಕಡೆಗೆ
ಕಡೆ-ಯ-ಲ್ಲಿಯೂ
ಕಡೆ-ಯ-ವರು
ಕಡೆ-ಯಿಂದ
ಕಡ್ಡಿ
ಕಡ್ಡಿಯ
ಕಡ್ಡಿ-ಯನ್ನು
ಕಣ-ಜ-ವನ್ನು
ಕಣ್ಣನ್ನು
ಕಣ್ಣಿ-ನಲ್ಲಿ
ಕಣ್ಣಿ-ನಿಂದ
ಕಣ್ಣಿ-ಲ್ಲವೆ
ಕಣ್ಣೀ-ರನ್ನು
ಕಣ್ಣೀರು
ಕಣ್ಣು
ಕಣ್ಣು-ಗಳಿಂದ
ಕಣ್ಣು-ಗಳು
ಕತೆ-ಯನ್ನು
ಕತ್ತನ್ನು
ಕತ್ತ-ರಿಸಿ
ಕತ್ತ-ರಿ-ಸಿ-ದರೆ
ಕತ್ತ-ರಿ-ಸುವು
ಕತ್ತ-ಲೆಯ
ಕತ್ತಿ
ಕತ್ತಿನ
ಕತ್ತಿ-ಯಿಂದ
ಕಥಾ-ಮೃತ
ಕಥೆ
ಕಥೆ-ಗಳನ್ನು
ಕಥೆ-ಗ-ಳ-ಲ್ಲಿ-ರುವ
ಕಥೆ-ಗ-ಳು-ಮ-ಕ್ಕಳ
ಕಥೆ-ಮ-ಕ್ಕಳ
ಕಥೆ-ಯನ್ನು
ಕಥೆ-ಯಿದೆ
ಕದ-ಲ-ಲಿಲ್ಲ
ಕದ-ಲಿ-ಸಿತು
ಕದಿ-ಯ-ಬೇ-ಕೆಂದು
ಕದಿ-ಯು-ವ-ವ-ನಲ್ಲ
ಕದಿ-ಯು-ವು-ದಕ್ಕೆ
ಕದ್ದ
ಕದ್ದ-ಮಾ-ಲನ್ನು
ಕದ್ದಿ
ಕನ-ಸಾ-ಯಿತು
ಕನಸಿ
ಕನ-ಸಿ-ನಂತೆ
ಕನ-ಸಿ-ನಲ್ಲಿ
ಕನ-ಸಿ-ನ-ಲ್ಲಿಯೂ
ಕನಸು
ಕನ್ನ-ಡಾ-ನು-ವಾದ
ಕನ್ನಡಿ
ಕನ್ನ-ಡಿ-ಯಲ್ಲಿ
ಕನ್ನ-ಹಾಕಿ
ಕಪಾ-ಲ-ದೊ-ಳಗೆ
ಕಪೋ-ಲದ
ಕಪ್ಪು
ಕಪ್ಪೆ
ಕಪ್ಪೆಗೆ
ಕಪ್ಪೆಯ
ಕಪ್ಪೆ-ಯನ್ನು
ಕಬ್ಬಿಣ
ಕಮ-ಲವು
ಕಮ್ಮಾ-ರನ
ಕರ-ಗಿ-ಹೋ-ಯಿತು
ಕರ-ಟ-ದಿಂದ
ಕರು
ಕರು-ಣಾ-ಮ-ಯ-ನಾದ
ಕರು-ಣೆ-ಯನ್ನು
ಕರು-ಳನ್ನು
ಕರು-ಳಿ-ನಿಂದ
ಕರು-ವನ್ನು
ಕರೆ
ಕರೆ-ತ-ರಲು
ಕರೆದ
ಕರೆ-ದನು
ಕರೆ-ದರೆ
ಕರೆ-ದಳು
ಕರೆ-ದಾ-ಗ-ಲೆಲ್ಲ
ಕರೆದು
ಕರೆ-ದು-ಕೊಂಡು
ಕರೆ-ದು-ಕೊಂ-ಡು-ಹೋ-ದಳು
ಕರೆ-ದೊ-ಯ್ಯು-ವು-ದಕ್ಕೆ
ಕರೆಯ
ಕರೆ-ಯನ್ನು
ಕರೆ-ಯ-ಬೇಕು
ಕರೆಯು
ಕರೆ-ಯು-ತ್ತಿ-ದ್ದರು
ಕರೆ-ಯು-ವು-ದಕ್ಕೆ
ಕರೆ-ಯು-ವುದನ್ನು
ಕರೆ-ಯು-ವೆಯೋ
ಕರ್ಮ-ಗಳ
ಕರ್ಮ-ದಿಂದ
ಕರ್ಮ-ಫ-ಲ-ವನ್ನು
ಕರ್ಮ-ಫ-ಲ-ವಿ-ರ-ಬೇಕು
ಕರ್ಮವೇ
ಕರ್ಮೇಂ-ದ್ರಿ-ಯ-ಗಳು
ಕಲ-ಕಿ-ದವು
ಕಲಿ-ತರೂ
ಕಲಿ-ತ-ವನು
ಕಲಿ-ತಿರು
ಕಲಿ-ತಿ-ರು-ವು-ದೆ-ಲ್ಲ-ವಿದ್ಯೆ
ಕಲಿ-ತು-ಕೊಂಡು
ಕಲಿ-ತು-ಕೊಳ್ಳಿ
ಕಲಿ-ಯ-ಬೇ-ಕೆಂದು
ಕಲಿ-ಯ-ಬೇ-ಕೆಂ-ದು-ಕೊಂ-ಡ-ವನು
ಕಲಿ-ಯು-ತ್ತಿ-ದ್ದರು
ಕಲಿ-ಯು-ವುದು
ಕಲಿ-ಸ-ಬೇ-ಕೆಂದು
ಕಲಿ-ಸಿ-ದರು
ಕಲಿ-ಸಿದೆ
ಕಲು-ವೀ-ರ-ನನ್ನು
ಕಲ್ಕ-ತ್ತ-ದಲ್ಲಿ
ಕಲ್ಕ-ತ್ತೆಗೆ
ಕಲ್ಕ-ತ್ತೆ-ಯೆಲ್ಲಿ
ಕಲ್ಪ-ತರು
ಕಲ್ಪ-ತ-ರು-ವಿನ
ಕಲ್ಪನೆ
ಕಲ್ಪ-ವೃಕ್ಷ
ಕಲ್ಪ-ವೃ-ಕ್ಷದ
ಕಲ್ಲನ್ನು
ಕಲ್ಲಾ-ಗಿದ್ದ
ಕಲ್ಲಿನ
ಕಲ್ಲು-ರಾ-ಶಿ-ಯನ್ನು
ಕಲ್ಲೂ
ಕಳ-ಕಳಿ
ಕಳ-ಚಿ-ಕೊಂ-ಡರೆ
ಕಳ-ವ-ಳಕ್ಕೆ
ಕಳ-ವ-ಳ-ವಾ-ಯಿತು
ಕಳು
ಕಳು-ಹಿ-ಸಿ-ಕೊಟ್ಟ
ಕಳು-ಹಿ-ಸಿದ
ಕಳು-ಹಿ-ಸಿ-ದನು
ಕಳು-ಹಿ-ಸಿ-ದರು
ಕಳು-ಹಿ-ಸಿ-ದಳು
ಕಳು-ಹಿಸು
ಕಳು-ಹಿ-ಸು-ತ್ತೇನೆ
ಕಳೆದ
ಕಳೆ-ದ-ಸಾರಿ
ಕಳೆದು
ಕಳೆ-ದು-ಕೊಂ-ಡರು
ಕಳೆ-ದು-ಕೊಂಡು
ಕಳೆ-ದು-ಕೊ-ಳ್ಳು-ತ್ತಾನೆ
ಕಳೆ-ದು-ಕೊ-ಳ್ಳು-ವನು
ಕಳೆ-ದು-ಕೊ-ಳ್ಳು-ವಳು
ಕಳೆದೆ
ಕಳೆ-ಯ-ಬೇ-ಕಾ-ಯಿತು
ಕಳೆ-ಯಲು
ಕಳೆ-ಯಿತು
ಕಳೆ-ಯು-ತ್ತಿತ್ತು
ಕಳೆ-ಯು-ತ್ತಿದ್ದ
ಕಳೆ-ಯು-ವನು
ಕಳ್ಳ
ಕಳ್ಳ-ಕಾ-ಕರ
ಕಳ್ಳ-ಕಾ-ಕ-ರನ್ನು
ಕಳ್ಳ-ನಿ-ರ-ಬೇಕು
ಕಳ್ಳ-ನೊ-ಬ್ಬನು
ಕಳ್ಳ-ರಿಗೆ
ಕಳ್ಳರು
ಕವಡೆ
ಕವಿ
ಕವಿ-ದಿತ್ತು
ಕವಿ-ಯು-ತ್ತದೆ
ಕಶ್ಮಲ
ಕಷ್ಟ
ಕಷ್ಟ-ಕ-ರ-ವಾ-ದ-ವು-ಗಳು
ಕಷ್ಟಕ್ಕೆ
ಕಷ್ಟ-ಗ-ಳಿಗೆ
ಕಷ್ಟ-ದಲ್ಲಿ
ಕಷ್ಟ-ದ-ಲ್ಲಿ-ರು-ವನು
ಕಷ್ಟ-ದಿಂದ
ಕಷ್ಟ-ಪ-ಟ್ಟಿ-ದ್ದಾನೆ
ಕಷ್ಟ-ಪಟ್ಟು
ಕಷ್ಟ-ಪ-ಡು-ತ್ತೀರಿ
ಕಷ್ಟವೂ
ಕಸ
ಕಸ-ಕ-ಡ್ಡಿಯೆ
ಕಸಕ್ಕೂ
ಕಸ-ದಂತೆ
ಕಸಾಯಿ
ಕಸಾ-ಯಿ-ಖಾ-ನೆಗೆ
ಕಸಿ-ದು-ಕೊಂಡ
ಕಸಿ-ದು-ಕೊಂ-ಡರು
ಕಹಿ
ಕಾ
ಕಾಂಚನ
ಕಾಂಚ-ನಕ್ಕೆ
ಕಾಂಚ-ನದ
ಕಾಂಚ-ನ-ದೊಂ-ದಿಗೆ
ಕಾಂಚ-ನ-ವನ್ನು
ಕಾಂಚ-ನವೇ
ಕಾಂತಿ-ಯನ್ನು
ಕಾಗದ
ಕಾಗ-ದ-ಕ್ಕಾಗಿ
ಕಾಗ-ದದ
ಕಾಗ-ದ-ದಲ್ಲಿ
ಕಾಗ-ದ-ವನ್ನು
ಕಾಗ-ದವೇ
ಕಾಗೆ
ಕಾಗೆ-ಗಳು
ಕಾಗೆಯ
ಕಾಗೆಯು
ಕಾಟ-ದಿಂದ
ಕಾಟ-ದಿಂ-ದಲೂ
ಕಾಟ-ವನ್ನು
ಕಾಡಿಗೆ
ಕಾಡಿನ
ಕಾಡಿ-ನಲ್ಲಿ
ಕಾಡಿ-ನಿಂದ
ಕಾಡು
ಕಾಡು-ಗಿ-ಚ್ಚಿ-ನಂತೆ
ಕಾಡು-ತ್ತಿ-ರುವ
ಕಾಡು-ಹುಲಿ
ಕಾಣದೆ
ಕಾಣ-ಬ-ರು-ತ್ತಿವೆ
ಕಾಣ-ಲಿಲ್ಲ
ಕಾಣಲು
ಕಾಣಿ
ಕಾಣಿಕೆ
ಕಾಣಿ-ಕೆ-ಗಳನ್ನು
ಕಾಣಿಸಿ
ಕಾಣಿ-ಸಿ-ಕೊಂ-ಡನು
ಕಾಣಿ-ಸಿ-ಕೊಂ-ಡಿತು
ಕಾಣಿ-ಸಿ-ಕೊಂಡು
ಕಾಣಿ-ಸಿ-ಕೊ-ಳ್ಳು-ತ್ತಾನೆ
ಕಾಣಿ-ಸಿ-ಕೊ-ಳ್ಳು-ವು-ದಕ್ಕೆ
ಕಾಣಿ-ಸಿ-ಕೊ-ಳ್ಳು-ವುದು
ಕಾಣಿ-ಸಿದ
ಕಾಣಿ-ಸು-ತ್ತಿತ್ತು
ಕಾಣು-ತ್ತದೆ
ಕಾಣು-ತ್ತಾನೆ
ಕಾಣು-ತ್ತಿದ್ದ
ಕಾಣು-ತ್ತಿ-ರು-ವು-ದ-ರಿಂದ
ಕಾಣು-ವು-ದಕ್ಕೆ
ಕಾಣು-ವು-ದಿಲ್ಲ
ಕಾಣು-ವು-ದಿ-ಲ್ಲವೆ
ಕಾಣು-ವುದು
ಕಾಣುವೆ
ಕಾತ-ರತೆ
ಕಾತ-ರ-ಳಾ-ಗಿ-ರು-ವೆಯಾ
ಕಾದಾ-ಡು-ವ-ರಲ್ಲ
ಕಾದಿ-ದೆಯೊ
ಕಾದಿ-ರ-ಲಾರೆ
ಕಾನ-ನ-ದಲ್ಲಿ
ಕಾಪಾಡ
ಕಾಪಾ-ಡ-ಬೇಕು
ಕಾಮ
ಕಾಮ-ಕಾಂ-ಚನ
ಕಾಮ-ಕಾಂ-ಚ-ನ-ದಲ್ಲಿ
ಕಾಮಕ್ಕೆ
ಕಾಮ-ರ್ಹಾ-ಟಿ-ಯಲ್ಲಿ
ಕಾಮ-ವನ್ನು
ಕಾಮ-ವೆಂಬ
ಕಾಮಾರ
ಕಾಮಾ-ರ-ಪು-ಕುರ
ಕಾಮಾ-ರ-ಪು-ಕು-ರಕ್ಕೆ
ಕಾಮಾ-ರ-ಪು-ಕು-ರದ
ಕಾಮಾ-ರ-ಪು-ಕು-ರ-ದಲ್ಲಿ
ಕಾಮಾ-ರ-ಪು-ಕು-ರ-ದ-ಲ್ಲಿ-ದ್ದಾಗ
ಕಾಮಾ-ರ-ಪು-ಕು-ರ-ದ-ಲ್ಲಿದ್ದೆ
ಕಾಮಿನಿ
ಕಾಮಿ-ನಿ-ಕಾಂ-ಚ-ನ-ದಿಂದ
ಕಾಮಿ-ನಿ-ಕಾಂ-ಚ-ನವೇ
ಕಾಮು-ಕನ
ಕಾಮ್ಯ-ಕ-ವ-ನಕ್ಕೆ
ಕಾಯ-ಬೇ-ಕಾ-ಗಿಲ್ಲ
ಕಾಯ-ಬೇ-ಕಾ-ಗು-ತ್ತದೆ
ಕಾಯಿ
ಕಾಯಿ-ಯನ್ನು
ಕಾಯಿಲೆ
ಕಾಯಿ-ಲೆಯ
ಕಾಯು-ತ್ತಿ-ದ್ದರು
ಕಾಯು-ವಂತೆ
ಕಾಯು-ವರು
ಕಾಯು-ವ-ವರು
ಕಾಯು-ವುದು
ಕಾರಂ-ತರ
ಕಾರಣ
ಕಾರ-ಣ-ದಿಂದ
ಕಾರ-ಣ-ನಾದ
ಕಾರ-ಣ-ವನ್ನು
ಕಾರ-ಣ-ವಾ-ಗ-ಬ-ಲ್ಲದು
ಕಾರ-ಣ-ವಾ-ಗಿದೆ
ಕಾರ-ಣ-ವಿದೆ
ಕಾರ-ಣ-ವೇನು
ಕಾರ-ಬೇ-ಕಾ-ಗಿಲ್ಲ
ಕಾರ-ವನ್ನು
ಕಾರ್ತಿ
ಕಾರ್ತಿ-ಕೇಯ
ಕಾರ್ತಿ-ಕೇ-ಯ-ನಂತೆ
ಕಾರ್ಯ
ಕಾರ್ಯ-ನಿ-ಮಿತ್ತ
ಕಾಲ
ಕಾಲಕ್ಕೆ
ಕಾಲ-ಕ್ರ-ಮೇಣ
ಕಾಲದ
ಕಾಲ-ದ-ಮೇಲೆ
ಕಾಲ-ದಲ್ಲಿ
ಕಾಲ-ದ-ವ-ರೆಗೆ
ಕಾಲ-ದಿಂದ
ಕಾಲನ್ನು
ಕಾಲ-ನ್ನೆತ್ತಿ
ಕಾಲರಾ
ಕಾಲ-ವಾದ
ಕಾಲ-ವಾ-ದರೆ
ಕಾಲ-ವಾ-ಯಿತು
ಕಾಲ-ವಿದ್ದ
ಕಾಲ-ವಿದ್ದು
ಕಾಲಾ-ನಂ-ತರ
ಕಾಲಿಗೆ
ಕಾಲಿನ
ಕಾಲಿ-ನಿಂದ
ಕಾಲು
ಕಾಲು-ಗಳನ್ನು
ಕಾಲು-ಗ-ಳಿಗೆ
ಕಾಲು-ಗಳು
ಕಾಲು-ವೆ-ಯನ್ನು
ಕಾಲ್ಕೆ-ಳಗೆ
ಕಾಳಿ
ಕಾಳಿಕಾ
ಕಾಳಿ-ಕಾ-ಮಾತೆ
ಕಾಳಿ-ಕಾ-ಮಾ-ತೆ-ಯನ್ನು
ಕಾಳಿ-ದೇ-ವ-ಸ್ಥಾ-ನದ
ಕಾಳಿ-ನಷ್ಟು
ಕಾಳೀ-ಕೃಷ್ಣ
ಕಾವಿ
ಕಾಶಿಗೆ
ಕಾಷ್ಠೆ-ಯನ್ನು
ಕಾಸನ್ನೂ
ಕಾಸಿ-ಲ್ಲದೆ
ಕಾಸು-ಗಳನ್ನೂ
ಕಿಟಕಿ
ಕಿತ್ತು
ಕಿತ್ತು-ಕೊ-ಳ್ಳು-ವು-ದ-ಕ್ಕಾಗಿ
ಕಿರಣ
ಕಿರಿಚಾ
ಕಿರಿಚಿ
ಕಿರಿ-ಚು-ತ್ತಿದ್ದ
ಕಿರಿ-ಚು-ತ್ತಿ-ರು-ವವು
ಕಿರಿ-ದಾದ
ಕಿರಿಯ
ಕಿರು-ಚ-ತೊ-ಡ-ಗಿತು
ಕಿರುಚಿ
ಕಿರು-ಚಿ-ಕೊಂ-ಡಿತು
ಕಿವಿ
ಕಿವಿ-ಗ-ಳಿಗೆ
ಕಿವಿ-ಗಳು
ಕಿವಿ-ಗೊಡ
ಕಿವಿ-ಯನ್ನು
ಕಿವಿ-ಯಲ್ಲಿ
ಕೀರಿ-ಟ-ವನ್ನು
ಕೀರ್ತಿ
ಕೀರ್ತಿ-ಗ-ಳೆಲ್ಲ
ಕೀರ್ತಿಗೆ
ಕೀರ್ತಿ-ಯನ್ನು
ಕೀರ್ತಿ-ಯೆಲ್ಲ
ಕೀಳಾಗಿ
ಕೀಳಾ-ಗಿ-ರು-ವುದನ್ನು
ಕೀಳಾ-ಗಿ-ರು-ವುದು
ಕೀಳಾದ
ಕೀಳು
ಕೀಳು-ವುದು
ಕುಂಬಳ
ಕುಂಬ-ಳ-ಕಾಯಿ
ಕುಂಬ-ಳ-ಕಾ-ಯಿ-ಯನ್ನು
ಕುಂಭಕ
ಕುಕ್ಕಿ
ಕುಟ್ಟು-ವುದನ್ನು
ಕುಡಿ
ಕುಡಿಕೆ
ಕುಡಿ-ಕೆ-ಯನ್ನು
ಕುಡಿದ
ಕುಡಿ-ದನು
ಕುಡಿ-ದರು
ಕುಡಿ-ದಿ-ದ್ದರೆ
ಕುಡಿ-ದಿ-ರುವ
ಕುಡಿ-ದಿ-ರು-ವೆನು
ಕುಡಿದು
ಕುಡಿ-ಯನ್ನು
ಕುಡಿ-ಯ-ಬ-ಹುದು
ಕುಡಿ-ಯ-ಬೇಕು
ಕುಡಿ-ಯ-ಲಾ-ರಂ-ಭಿ-ಸಿತು
ಕುಡಿ-ಯಲು
ಕುಡಿ-ಯು-ತ್ತಿ-ರ-ಲಿಲ್ಲ
ಕುಡಿ-ಯು-ತ್ತಿಲ್ಲ
ಕುಡಿ-ಯು-ತ್ತೇನೆ
ಕುಡಿ-ಯು-ವಾಗ
ಕುಡಿ-ಯು-ವುದನ್ನು
ಕುಡಿ-ಯುವೆ
ಕುಡಿ-ಸಿ-ದಲ್ಲಿ
ಕುಡುಕ
ಕುಣಿ
ಕುಣಿದು
ಕುಣಿ-ಯುತ್ತ
ಕುಣಿ-ಯು-ತ್ತಿ-ದ್ದಳು
ಕುಣಿ-ಯು-ತ್ತಿರು
ಕುಣಿ-ಯು-ತ್ತಿ-ರುವೆ
ಕುಣಿ-ಯು-ವಾಗ
ಕುಣಿ-ಯು-ವು-ದಕ್ಕೆ
ಕುಣಿ-ಯು-ವು-ದರ
ಕುಣಿ-ಯೋಣ
ಕುತೂ-ಹಲ
ಕುತೂ-ಹ-ಲ-ಪ-ಟ್ಟರು
ಕುದಿ-ಯು-ತ್ತಿ-ರುವ
ಕುದುರೆ
ಕುದು-ರೆ-ಗಳು
ಕುದು-ರೆ-ಗಳೇ
ಕುದು-ರೆಯ
ಕುಪ್ಪ-ಳಿ-ಸು-ತ್ತಿದ್ದ
ಕುಪ್ಪ-ಳಿ-ಸು-ತ್ತಿ-ರುವೆ
ಕುರಿ
ಕುರಿ-ಗಳು
ಕುರಿತು
ಕುರಿ-ಮಂ-ದೆಯ
ಕುರಿ-ಮಂ-ದೆ-ಯ-ಲ್ಲಿದ್ದ
ಕುರಿ-ಮಂ-ದೆ-ಯ-ಲ್ಲಿಯೇ
ಕುರಿಯ
ಕುರಿ-ಯಂ-ತೆಯೇ
ಕುರಿ-ಯನ್ನು
ಕುರಿ-ಯಲ್ಲಿ
ಕುರು
ಕುರು-ಕ್ಷೇ-ತ್ರದ
ಕುರು-ಡ-ರಿಗೆ
ಕುರು-ಡರು
ಕುರು-ಡು-ತ-ನ-ವನ್ನು
ಕುರುವೂ
ಕುರ್ಚಿ
ಕುಲ-ದೇ-ವ-ರನ್ನು
ಕುಲುಮೆ
ಕುಳಿ-ತರು
ಕುಳಿ-ತಿತ್ತು
ಕುಳಿ-ತಿದ್ದ
ಕುಳಿ-ತಿ-ದ್ದರು
ಕುಳಿ-ತಿದ್ದೆ
ಕುಳಿ-ತಿ-ರು-ವನು
ಕುಳಿ-ತಿ-ರು-ವಾಗ
ಕುಳಿ-ತಿ-ರು-ವುದನ್ನು
ಕುಳಿತು
ಕುಳಿ-ತು-ಕೊಂಡ
ಕುಳಿ-ತು-ಕೊಂ-ಡಿತು
ಕುಳಿ-ತು-ಕೊಂ-ಡಿದ್ದ
ಕುಳಿ-ತು-ಕೊಂ-ಡಿ-ರುವೆ
ಕುಳಿ-ತು-ಕೊಂಡು
ಕುಳಿ-ತು-ಕೊ-ಳ್ಳುವ
ಕುಳಿ-ತು-ಕೊ-ಳ್ಳು-ವು-ದಕ್ಕೆ
ಕುಳಿ-ತು-ಕೊ-ಳ್ಳೋಣ
ಕುಳ್ಳಿ-ರಿ-ಸ-ಬಾ-ರದು
ಕುಳ್ಳಿ-ರಿಸಿ
ಕುಳ್ಳಿ-ರಿ-ಸಿ-ಕೊಂ-ಡಿದ್ದೆ
ಕುಳ್ಳಿ-ರಿ-ಸಿದ
ಕುಸ್ತಿ
ಕುಸ್ತಿ-ಪ-ಟು-ಗಳು
ಕುಸ್ತಿಯ
ಕುಸ್ತಿ-ಯ-ವನು
ಕುಸ್ತಿ-ಯಾ-ಡು-ತ್ತಿ-ದ್ದರು
ಕೂಗಾಟ
ಕೂಗಿ
ಕೂಗಿದ
ಕೂಗಿದೆ
ಕೂಗು
ಕೂಗುತ್ತ
ಕೂಗು-ತ್ತಾನೆ
ಕೂಗು-ತ್ತಿ-ದ್ದನೋ
ಕೂಗು-ವುದು
ಕೂಡ
ಕೂಡದು
ಕೂಡಲೆ
ಕೂಡಲೇ
ಕೂಡಿದ
ಕೂಡಿ-ರು-ತ್ತಾನೆ
ಕೂಡಿ-ಹಾ-ಕ-ಬೇ-ಕೆಂಬ
ಕೂದ-ಲನ್ನು
ಕೂದ-ಲಿಗೆ
ಕೂದಲು
ಕೂರಿಸಿ
ಕೂಲಿಯ
ಕೃತ-ಜ್ಞ-ತೆ-ಯಿಂದ
ಕೃತ-ಜ್ಞ-ನಾ-ಗು-ವೆನು
ಕೃತ್ಯ-ವನ್ನು
ಕೃಪಾ-ಕ-ಟಾ-ಕ್ಷ-ವನ್ನು
ಕೃಪೆ
ಕೃಪೆಗೆ
ಕೃಪೆ-ಯಿಂದ
ಕೃಷ್ಣ
ಕೃಷ್ಣನ
ಕೃಷ್ಣ-ನಿಗೆ
ಕೃಷ್ಣನು
ಕೃಷ್ಣ-ರಾ-ಗಿ-ದ್ದರು
ಕೃಷ್ಣರು
ಕೆಂಪಾಗಿ
ಕೆಂಪಾ-ಯಿತು
ಕೆಂಪಿತ್ತು
ಕೆಂಪು
ಕೆಟ್ಟ
ಕೆಟ್ಟ-ದನ್ನು
ಕೆಟ್ಟ-ದನ್ನೂ
ಕೆಟ್ಟ-ದ್ದನ್ನು
ಕೆಟ್ಟ-ದ್ದಿ-ರ-ಲಿ-ಕ್ಕಿಲ್ಲ
ಕೆಟ್ಟದ್ದು
ಕೆಟ್ಟದ್ದೂ
ಕೆಟ್ಟ-ವರ
ಕೆಟ್ಟ-ವ-ರಿಂದ
ಕೆಟ್ಟವೂ
ಕೆಡ-ವ-ಬೇಡಿ
ಕೆಡ-ವು-ತ್ತಿದ್ದ
ಕೆಡಿ-ಸು-ತ್ತಿ-ದ್ದರು
ಕೆಡಿ-ಸು-ವುದನ್ನು
ಕೆಡಿ-ಸು-ವು-ದ-ರಲ್ಲಿ
ಕೆಣ-ಕಲು
ಕೆನೆ
ಕೆನೆ-ಯನ್ನು
ಕೆನ್ನೆಯ
ಕೆರ-ಳಿತು
ಕೆರೆಗೆ
ಕೆರೆಯ
ಕೆರೆಯು
ಕೆಲ
ಕೆಲ-ಕ್ಷಣ
ಕೆಲವ
ಕೆಲ-ವನ್ನು
ಕೆಲ-ವರ
ಕೆಲ-ವರು
ಕೆಲವು
ಕೆಲಸ
ಕೆಲ-ಸಕ್ಕೂ
ಕೆಲ-ಸಕ್ಕೆ
ಕೆಲ-ಸ-ಗಳನ್ನು
ಕೆಲ-ಸ-ಗಾರ
ಕೆಲ-ಸದ
ಕೆಲ-ಸ-ದಲ್ಲಿ
ಕೆಲ-ಸ-ವನ್ನು
ಕೆಲ-ಸ-ವ-ನ್ನೆಲ್ಲ
ಕೆಲ-ಸ-ವನ್ನೇ
ಕೆಳಕ್ಕೆ
ಕೆಳ-ಗಡೆ
ಕೆಳ-ಗಿಟ್ಟ
ಕೆಳಗೆ
ಕೆಳ-ಭಾಗ
ಕೆಸ-ರಿನ
ಕೆಸ-ರಿ-ನಲ್ಲಿ
ಕೇಯ
ಕೇರುವ
ಕೇರೆ
ಕೇರೆ-ಹಾ-ವಿನ
ಕೇಳದೆ
ಕೇಳ-ಬಲ್ಲ
ಕೇಳ-ಬ-ಹುದು
ಕೇಳ-ಬಾ-ರದು
ಕೇಳ-ಬೇ-ಕಾ-ಗಿಲ್ಲ
ಕೇಳ-ಬೇ-ಕಾ-ಯಿತು
ಕೇಳ-ಬೇಕೆ
ಕೇಳ-ಬೇ-ಕೆಂದು
ಕೇಳ-ಬೇ-ಕೆಂಬ
ಕೇಳ-ಬೇಡ
ಕೇಳ-ಲಿಲ್ಲ
ಕೇಳಲು
ಕೇಳಲೆ
ಕೇಳಿ
ಕೇಳಿ-ಕೊಂಡ
ಕೇಳಿ-ಕೊಂ-ಡನು
ಕೇಳಿ-ಕೊಂ-ಡರು
ಕೇಳಿ-ಕೊಳ್ಳು
ಕೇಳಿ-ಕೊ-ಳ್ಳು-ತ್ತಿ-ರು-ವರು
ಕೇಳಿತು
ಕೇಳಿದ
ಕೇಳಿ-ದನು
ಕೇಳಿ-ದರು
ಕೇಳಿ-ದರೂ
ಕೇಳಿ-ದರೆ
ಕೇಳಿ-ದ-ರೇನೇ
ಕೇಳಿ-ದಳು
ಕೇಳಿ-ದ-ವ-ರಿಗೆ
ಕೇಳಿ-ದಾಗ
ಕೇಳಿ-ದಾ-ಗಲೂ
ಕೇಳಿದೆ
ಕೇಳಿ-ದೆನು
ಕೇಳಿ-ದೊ-ಡ-ನೆಯೆ
ಕೇಳಿ-ದ್ದಕ್ಕೆ
ಕೇಳಿ-ದ್ದರೆ
ಕೇಳಿ-ದ್ದೇವೆ
ಕೇಳಿ-ಬಂತು
ಕೇಳಿ-ರ-ಬೇಕು
ಕೇಳಿ-ಸ-ದಿ-ರಲಿ
ಕೇಳಿ-ಸ-ಲಿಲ್ಲ
ಕೇಳಿ-ಸಿತು
ಕೇಳು
ಕೇಳು-ಗರ
ಕೇಳು-ತ್ತಾನೆ
ಕೇಳು-ತ್ತಾರೆ
ಕೇಳು-ತ್ತಿದ್ದ
ಕೇಳು-ತ್ತಿ-ದ್ದನು
ಕೇಳು-ತ್ತಿ-ದ್ದರು
ಕೇಳು-ತ್ತಿ-ದ್ದಳು
ಕೇಳು-ತ್ತಿ-ದ್ದ-ವ-ನನ್ನು
ಕೇಳು-ತ್ತಿದ್ದಿ
ಕೇಳು-ತ್ತಿದ್ದೆ
ಕೇಳು-ತ್ತಿ-ರ-ಲಿಲ್ಲ
ಕೇಳು-ತ್ತಿ-ರು-ವನು
ಕೇಳು-ತ್ತೇನೆ
ಕೇಳು-ವಂತೆ
ಕೇಳು-ವ-ನೇನು
ಕೇಳು-ವರು
ಕೇಳು-ವು-ದಕ್ಕೆ
ಕೇಳು-ವು-ದ-ರ-ಲ್ಲಿದ್ದೆ
ಕೇಳು-ವು-ದ-ರಿಂದ
ಕೇಳು-ವು-ದಿಲ್ಲ
ಕೇಳು-ವುದು
ಕೇಳುವೆ
ಕೇಳೋಣ
ಕೇವಲ
ಕೇಶಬ್
ಕೇಶವ
ಕೇಶ-ವ-ಚಂದ್ರ
ಕೇಶ-ವ-ಚಂ-ದ್ರನ
ಕೇಶ-ವ-ಚಂ-ದ್ರ-ಸೇನ
ಕೇಶ-ವ-ಸೇ-ನನ
ಕೈ
ಕೈಕಾ-ಲು-ಗಳನ್ನು
ಕೈಕೇಯಿ
ಕೈಗಳ
ಕೈಗಳನ್ನು
ಕೈಗ-ಳ-ನ್ನುಳ್ಳ
ಕೈಗಳಿಂದ
ಕೈಗ-ಳಿಗೆ
ಕೈಗಳು
ಕೈಬಿಟ್ಟ
ಕೈಮು-ಗಿ-ಯುತ್ತ
ಕೈಮೇಲೆ
ಕೈಯನ್ನು
ಕೈಯಲ್ಲಿ
ಕೈಯ-ಲ್ಲಿ-ರುವ
ಕೈಯ-ಲ್ಲಿ-ರು-ವುದು
ಕೈಯಿಂದ
ಕೈಯೊಂದು
ಕೈಲಾ-ದು-ದನ್ನು
ಕೈಲಾ-ಸ-ದಲ್ಲಿ
ಕೈಸೇರಿ
ಕೈಹಾ-ಕಿ-ದರೆ
ಕೊಂಡ
ಕೊಂಡರು
ಕೊಂಡಳು
ಕೊಂಡಾ-ಡಲು
ಕೊಂಡಾ-ಡು-ತ್ತಿದ್ದ
ಕೊಂಡಾ-ಡು-ವುದು
ಕೊಂಡಿತು
ಕೊಂಡಿದ್ದ
ಕೊಂಡಿ-ರು-ವ-ನಂತೆ
ಕೊಂಡು
ಕೊಂಡೊ-ಯ್ದರು
ಕೊಂದ
ಕೊಂದಾದ
ಕೊಂದು
ಕೊಂದು-ಬಿಟ್ಟೆ
ಕೊಂಬೆ-ಯನ್ನು
ಕೊಂಬೆ-ಯಲ್ಲಿ
ಕೊಕ್ಕರೆ
ಕೊಕ್ಕ-ರೆಗೆ
ಕೊಕ್ಕ-ರೆ-ಯನ್ನು
ಕೊಕ್ಕಿ-ನಲ್ಲಿ
ಕೊಕ್ಕಿ-ನ-ಲ್ಲಿದ್ದ
ಕೊಕ್ಕಿ-ನಿಂದ
ಕೊಚ್ಚಿ-ಕೊಂಡು
ಕೊಚ್ಚಿ-ಕೊಳ್ಳು
ಕೊಚ್ಚಿ-ಕೊ-ಳ್ಳುವ
ಕೊಚ್ಚಿ-ಕೊ-ಳ್ಳು-ವನೇ
ಕೊಚ್ಚಿ-ಕೊ-ಳ್ಳು-ವು-ದಿಲ್ಲ
ಕೊಟ್ಟ
ಕೊಟ್ಟನು
ಕೊಟ್ಟನೆ
ಕೊಟ್ಟರು
ಕೊಟ್ಟರೂ
ಕೊಟ್ಟರೆ
ಕೊಟ್ಟಳು
ಕೊಟ್ಟಾಗ
ಕೊಟ್ಟಾರು
ಕೊಟ್ಟಿ-ದ್ದನ್ನು
ಕೊಟ್ಟಿ-ದ್ದ-ನ್ನೆಲ್ಲ
ಕೊಟ್ಟಿ-ದ್ದಾನೆ
ಕೊಟ್ಟಿ-ದ್ದೆಯೋ
ಕೊಟ್ಟು
ಕೊಟ್ಟು-ಬಿಟ್ಟ
ಕೊಟ್ಟು-ಬಿಡು
ಕೊಡ
ಕೊಡ-ದಿ-ದ್ದರೆ
ಕೊಡದೆ
ಕೊಡ-ಬಲ್ಲ
ಕೊಡ-ಬ-ಹುದು
ಕೊಡ-ಬಾ-ರದು
ಕೊಡ-ಬೇ-ಕಾ-ಗಿದೆ
ಕೊಡ-ಬೇ-ಕಾ-ಗಿದ್ದ
ಕೊಡ-ಬೇ-ಕಾ-ಗಿ-ರುವ
ಕೊಡ-ಬೇ-ಕಾ-ಗಿಲ್ಲ
ಕೊಡ-ಬೇಕು
ಕೊಡ-ಬೇ-ಕೆಂ-ದಿದ್ದ
ಕೊಡ-ಬೇ-ಕೆಂದು
ಕೊಡ-ಬೇಡ
ಕೊಡಮ್ಮ
ಕೊಡ-ಲಾ-ಗು-ವು-ದಿಲ್ಲ
ಕೊಡ-ಲಾ-ರದು
ಕೊಡ-ಲಾರೆ
ಕೊಡ-ಲಿ-ಯಿಂದ
ಕೊಡ-ಲಿಲ್ಲ
ಕೊಡಲು
ಕೊಡ-ವನ್ನು
ಕೊಡಿ
ಕೊಡಿ-ಸು-ವೆನು
ಕೊಡು
ಕೊಡು-ತ್ತಾನೆ
ಕೊಡು-ತ್ತಾನೋ
ಕೊಡು-ತ್ತಾರೆ
ಕೊಡು-ತ್ತಾರೊ
ಕೊಡು-ತ್ತಾಳೆ
ಕೊಡು-ತ್ತಿದ್ದ
ಕೊಡು-ತ್ತಿ-ದ್ದನು
ಕೊಡು-ತ್ತಿ-ದ್ದಳು
ಕೊಡು-ತ್ತಿದ್ದೆ
ಕೊಡು-ತ್ತಿ-ರು-ವನು
ಕೊಡು-ತ್ತಿ-ರು-ವರು
ಕೊಡು-ತ್ತೇನೆ
ಕೊಡುವ
ಕೊಡು-ವನು
ಕೊಡು-ವರು
ಕೊಡು-ವಳು
ಕೊಡು-ವು-ದ-ಕ್ಕಾ-ಗಲೀ
ಕೊಡು-ವು-ದಕ್ಕೆ
ಕೊಡು-ವುದನ್ನು
ಕೊಡು-ವು-ದ-ರಲ್ಲಿ
ಕೊಡು-ವು-ದಿಲ್ಲ
ಕೊಡು-ವುದು
ಕೊಡುವೆ
ಕೊಡೋಣ
ಕೊನೆಗೆ
ಕೊನೆಯ
ಕೊನೆ-ಯಲ್ಲಿ
ಕೊನೆ-ಯ-ವನು
ಕೊನೆ-ಯಿ-ರ-ಲಿಲ್ಲ
ಕೊನೆ-ಯಿ-ಲ್ಲ-ವಲ್ಲ
ಕೊನ್ನ-ಗ-ರಕ್ಕೆ
ಕೊಪ್ಪ-ರಿಗೆ
ಕೊಪ್ಪ-ರಿ-ಗೆಗೆ
ಕೊಪ್ಪ-ರಿ-ಗೆಯ
ಕೊಪ್ಪ-ರಿ-ಗೆ-ಯನ್ನು
ಕೊಪ್ಪ-ರಿ-ಗೆ-ಯಲ್ಲಿ
ಕೊಬ್ಬರಿ
ಕೊಬ್ಬಿ-ದ್ದಾಳೆ
ಕೊರತೆ
ಕೊರ-ತೆ-ಗ-ಳಿವೆ
ಕೊರ-ತೆ-ಯಿಲ್ಲ
ಕೊರ-ತೆಯೂ
ಕೊರಳ
ಕೊರ-ಳಲ್ಲಿ
ಕೊರಳಿ
ಕೊಲ್ಲ-ಬ-ಹುದು
ಕೊಲ್ಲ-ಬೇಡಿ
ಕೊಲ್ಲಲು
ಕೊಲ್ಲಿ
ಕೊಲ್ಲು-ತ್ತಿರು
ಕೊಲ್ಲು-ತ್ತೇನೆ
ಕೊಲ್ಲು-ವು-ದ-ಕ್ಕಾಗಿ
ಕೊಲ್ಲು-ವು-ದಕ್ಕೆ
ಕೊಳ
ಕೊಳ-ಕಾ-ಗಿ-ದ್ದು-ದನ್ನು
ಕೊಳ-ಗಳು
ಕೊಳದ
ಕೊಳ-ದೊ-ಳಗೆ
ಕೊಳ-ವೆಯ
ಕೊಳೆ
ಕೊಳೆ-ಯಾ-ಗಿತ್ತು
ಕೊಳ್ಳ-ಬಾ-ರದು
ಕೊಳ್ಳ-ಬೇ-ಕಾ-ಗಿ-ರುವ
ಕೊಳ್ಳ-ಬೇ-ಕಾ-ದರೆ
ಕೊಳ್ಳ-ಬೇ-ಕಾ-ಯಿತು
ಕೊಳ್ಳ-ಬೇಕು
ಕೊಳ್ಳಲು
ಕೊಳ್ಳಿ
ಕೊಳ್ಳುತ್ತ
ಕೊಳ್ಳು-ತ್ತಲೇ
ಕೊಳ್ಳು-ತ್ತಿ-ರುವೆ
ಕೊಳ್ಳು-ತ್ತೇನೆ
ಕೊಳ್ಳು-ವು-ದಕ್ಕೆ
ಕೊಳ್ಳು-ವೆಯಾ
ಕೊಳ್ಳೆ-ಹೊ-ಡೆ-ಯು-ವು-ದಕ್ಕೆ
ಕೋಟಿ
ಕೋಣೆ
ಕೋಣೆಗೆ
ಕೋಣೆಯ
ಕೋಣೆ-ಯನ್ನು
ಕೋಣೆ-ಯಲ್ಲಿ
ಕೋಣೆ-ಯಲ್ಲೇ
ಕೋಣೆ-ಯಿಂದ
ಕೋತಿ
ಕೋಪ
ಕೋಪ-ದಿಂದ
ಕೋಪ-ವನ್ನು
ಕೋಪ-ವುಂ-ಟಾ-ಯಿತು
ಕೋಪವೂ
ಕೋಪ-ವೆಲ್ಲ
ಕೋಪಾ-ವಿ-ಷ್ಟ-ಳಾಗಿ
ಕೋಪಿ-ಸಿ-ಕೊ-ಳ್ಳ-ಲಾರ
ಕೋರನೆ
ಕೋರು-ತ್ತಿ-ದ್ದರು
ಕೋರೈ-ಸುವ
ಕೋರ್ಟಿಗೆ
ಕೋಲಿ-ನಿಂದ
ಕೌಪೀನ
ಕೌಪೀ-ನ-ಕ್ಕಾಗಿ
ಕೌಪೀ-ನ-ವ-ನ್ನಾಗಿ
ಕೌಪೀ-ನ-ವನ್ನು
ಕ್ಕಾಗಿ
ಕ್ಕಿಂತ
ಕ್ರಮ
ಕ್ರಮೇಣ
ಕ್ರಾಂತ-ರಾ-ದರು
ಕ್ರೂರಿ
ಕ್ರೋಧಾ-ದಿ-ಗಳ
ಕ್ಷಣ
ಕ್ಷಣ-ಗಳ
ಕ್ಷಣ-ಗಳಲ್ಲಿ
ಕ್ಷಣಾ-ರ್ಧ-ದಲ್ಲಿ
ಕ್ಷಮಿಸಿ
ಕ್ಷಾಮ
ಕ್ಷುದ್ರ
ಕ್ಷುಬ್ಧ-ಳಾಗಿ
ಕ್ಷುಲ್ಲಕ
ಕ್ಷೌರ
ಕ್ಷೌರದ
ಕ್ಷೌರಿಕ
ಕ್ಷೌರಿ-ಕನ
ಕ್ಷೌರಿ-ಕ-ನಿಗೆ
ಕ್ಷೌರಿ-ಕನು
ಖಂಡಿತ
ಖಡ್ಗ-ವನ್ನು
ಖರೀ-ದಿ-ಸೋಣ
ಖರ್ಚಿಗೆ
ಖರ್ಚು
ಖರ್ಚು-ಮಾ-ಡಲೋ
ಖರ್ಚು-ಮಾ-ಡು-ವರೋ
ಖಾನೆಗೆ
ಖಾನೆ-ಯಲ್ಲಿ
ಖಾಲಿ
ಖುಲಾ-ಯಿ-ಸಿದೆ
ಖುಲಾಸೆ
ಖುಲ್ಲನಾ
ಖೂನಿ
ಗಂಗಾ
ಗಂಗಾ-ನದಿ
ಗಂಗಾ-ನ-ದಿಗೆ
ಗಂಗಾ-ನ-ದಿಯ
ಗಂಗಾ-ನ-ದಿ-ಯನ್ನು
ಗಂಗಾ-ನ-ದಿ-ಯಲ್ಲಿ
ಗಂಗೆ-ಯಲ್ಲಿ
ಗಂಟಲು
ಗಂಟು-ಮೂ-ಟೆ-ಗಳನ್ನೆಲ್ಲ
ಗಂಟೆ-ಗಳ
ಗಂಟೆಗೆ
ಗಂಟೆ-ಯನ್ನು
ಗಂಡ
ಗಂಡಂ-ದಿ-ರಿ-ರು-ವುದನ್ನು
ಗಂಡನ
ಗಂಡ-ನನ್ನು
ಗಂಡ-ನಾದ
ಗಂಡ-ನಾರು
ಗಂಡ-ನಿಗೆ
ಗಂಡ-ನೇನು
ಗಂಡ-ಸರು
ಗಂಡ-ಸಿಗೂ
ಗಂಡಾಂ-ತ-ರ-ಗಳ
ಗಂಡು
ಗಂಧದ
ಗಜ-ಗಳು
ಗಟ್ಟಿ-ಮು-ಟ್ಟಾ-ಗಿದ್ದ
ಗಟ್ಟಿ-ಯಾಗಿ
ಗಟ್ಟಿ-ಯಾ-ಗಿ-ದ್ದರೆ
ಗಡ-ಗಡ
ಗಡಿ-ಗೆ-ಗಳು
ಗಡೆಗೆ
ಗಣ-ನೆಗೆ
ಗಣ-ಪತಿ
ಗಣ-ಪ-ತಿಗೆ
ಗಣ-ಪ-ತಿಯ
ಗಣಿ
ಗಣಿ-ಯನ್ನು
ಗಣಿ-ಯಿಂದ
ಗಣೇಶ
ಗಣೇ-ಶನ
ಗತಿ
ಗತಿಗೆ
ಗದ-ರಿ-ಸಿ-ದನು
ಗದ-ರಿ-ಸಿ-ದರು
ಗದ-ರಿ-ಸಿದೆ
ಗದ-ರಿ-ಸುತ್ತ
ಗದೆ-ಯಂತೆ
ಗದ್ದಲ
ಗದ್ದ-ಲ-ವೆ-ದ್ದಿತು
ಗದ್ದೆಗೆ
ಗದ್ದೆಯ
ಗಮ-ನ-ಕೊ-ಡದೆ
ಗಮ-ನಕ್ಕೆ
ಗಮ-ನ-ವೀ-ಯದೆ
ಗಮ-ನಿಸ
ಗಮ-ನಿ-ಸಿ-ದಳು
ಗಮ-ನಿ-ಸಿದ್ದ
ಗಮ-ನಿಸು
ಗಮ-ನಿ-ಸು-ವು-ದಿಲ್ಲ
ಗರ-ಗರ
ಗರ್ಜಿ-ಸಿ-ದನು
ಗರ್ಭಿ-ಣಿ-ಯಾದ
ಗಲಾಟೆ
ಗಲಿ-ಬಿ-ಲಿ-ಯೆಲ್ಲ
ಗಲೀಜು
ಗಲೂ
ಗಲ್ಲದ
ಗಳ
ಗಳ-ಗಳ
ಗಳನ್ನು
ಗಳ-ನ್ನೆಲ್ಲ
ಗಳಲ್ಲಿ
ಗಳಾ-ಗಿ-ದ್ದರು
ಗಳಾದ
ಗಳಿ-ಗಾಗಿ
ಗಳಿಗೆ
ಗಳಿಲ್ಲ
ಗಳು
ಗಳೂ
ಗಳೊ-ಡನೆ
ಗಾಗಿ
ಗಾಜಿನ
ಗಾಡಿ-ಗ-ಟ್ಟಳೆ
ಗಾಡಿ-ಯಲ್ಲಿ
ಗಾಢ
ಗಾಢ-ಸ-ಮಾ-ಧಿ-ಯಲ್ಲಿ
ಗಾಬರಿ
ಗಾಬ-ರಿ-ಯಿಂದ
ಗಾಯ
ಗಾಯ-ವಾಗಿ
ಗಾಯ-ವುಂ-ಟಾ-ದುದು
ಗಾರೆ-ಕಾ-ರರು
ಗಾಳಕ್ಕೆ
ಗಾಳದ
ಗಾಳ-ವನ್ನು
ಗಾಳ-ವನ್ನೇ
ಗಾಳಿ
ಗಾಳಿ-ಗಾಗಿ
ಗಾಳಿಯ
ಗಾಳಿ-ಸೇ-ವನೆ
ಗಿಂತ
ಗಿಟಕು
ಗಿಡ
ಗಿಡ-ಗಳನ್ನು
ಗಿಡ-ಗಳನ್ನೂ
ಗಿಡದ
ಗಿಡ-ಮ-ರ-ಗಳು
ಗಿತ್ತು
ಗಿಯೋ
ಗಿರ-ಲಿಲ್ಲ
ಗಿರಾಕಿ
ಗಿರಾ-ಕಿ-ಗ-ಳಿಗೆ
ಗಿರಾ-ಕಿ-ಗಳು
ಗಿರಾ-ಕಿಗೆ
ಗಿರಿ-ರಾಜ
ಗಿರುವ
ಗಿರು-ವನು
ಗೀತೆಯ
ಗೀತೆ-ಯನ್ನು
ಗೀತೆ-ಯಲ್ಲಿ
ಗೀತೋ-ಪ-ದೇ-ಶ-ವನ್ನು
ಗುಂಪಿಗೆ
ಗುಂಪಿ-ನಲ್ಲಿ
ಗುಟ್ಟಾಗಿ
ಗುಟ್ಟುತ್ತ
ಗುಡಾಣ
ಗುಡಾ-ಣ-ಗ-ಳಿ-ದ್ದವು
ಗುಡಾ-ಣ-ವನ್ನು
ಗುಡಿ-ಗು-ಡಿ-ಯನ್ನು
ಗುಡಿ-ಗು-ಡಿ-ಸಿ-ದನು
ಗುಡಿ-ಯ-ಲ್ಲೆಲ್ಲ
ಗುಡಿಸ
ಗುಡಿ-ಸ-ಲನ್ನು
ಗುಡಿ-ಸ-ಲಿಗೆ
ಗುಡಿ-ಸ-ಲಿನ
ಗುಡಿ-ಸ-ಲಿ-ನಿಂದ
ಗುಡಿ-ಸಲು
ಗುಡಿ-ಸಿ-ರ-ಲಿಲ್ಲ
ಗುಡಿ-ಸಿ-ಲಿನ
ಗುಡಿ-ಸು-ವ-ವಳೇ
ಗುಡುಗು
ಗುಡು-ಗುಡಿ
ಗುಡ್ಡೆ-ಯಿಂದ
ಗುಣ
ಗುಣ-ಗ-ಳಿಂ-ದಲೂ
ಗುಣ-ದಿಂದ
ಗುಣ-ಮಾಡು
ಗುಣ-ವಾ-ದ-ನೆ-ಹೇ-ಳು-ವುದು
ಗುಣವೂ
ಗುತ್ತಿ-ಗೆ-ದಾ-ರರು
ಗುದ್ದ-ಲಿ-ಯನ್ನು
ಗುರಾಣಿ
ಗುರಿ
ಗುರಿ-ಯನ್ನು
ಗುರಿ-ಯಾ-ಗು-ತ್ತಾರೆ
ಗುರಿ-ಯಿಂದ
ಗುರಿ-ಯಿಟ್ಟು
ಗುರು
ಗುರು-ಶಿಷ್ಯ
ಗುರು-ಗಳ
ಗುರು-ಗ-ಳಂತೆ
ಗುರು-ಗಳನ್ನು
ಗುರು-ಗ-ಳಾ-ಗ-ಬೇ-ಕಾ-ಗಿ-ದೆಯೋ
ಗುರು-ಗ-ಳಾ-ಗಲು
ಗುರು-ಗ-ಳಾದ
ಗುರು-ಗಳಿಂದ
ಗುರು-ಗ-ಳಿಗೂ
ಗುರು-ಗ-ಳಿಗೆ
ಗುರು-ಗ-ಳಿ-ದ್ದರು
ಗುರು-ಗಳು
ಗುರು-ಗಳೆ
ಗುರು-ಗ-ಳೆ-ಡೆಗೆ
ಗುರು-ಗಳೇ
ಗುರು-ಗ-ಳೊ-ಬ್ಬರ
ಗುರು-ತನ್ನು
ಗುರು-ತಿಸ
ಗುರು-ತಿ-ಸಿದ
ಗುರು-ತಿ-ಸಿ-ಬಿ-ಟ್ಟರು
ಗುರು-ದ-ಕ್ಷಿ-ಣೆ-ಯನ್ನು
ಗುರು-ದೇವ
ಗುರು-ದೇ-ವ-ಸ್ವಾಮಿ
ಗುರು-ವನ್ನು
ಗುರು-ವಾಗಿ
ಗುರು-ವಿ-ಗಿಂತ
ಗುರು-ವಿಗೆ
ಗುರು-ವಿನ
ಗುರು-ವಿ-ನಂತೆ
ಗುರು-ವಿ-ನಲ್ಲಿ
ಗುರುವು
ಗುಲಾ-ಬಳ
ಗುಲಾ-ಬಳು
ಗುಲಾ-ಮ-ನಾ-ಗು-ತ್ತೀಯ
ಗುಳಿಗೆ
ಗುಳಿ-ಗೆ-ಯನ್ನು
ಗುಳ್ಳೆ-ಗಳು
ಗುವುದನ್ನು
ಗುವುದು
ಗುಹ-ನಾಗಿ
ಗೂಟ
ಗೂಟಕ್ಕೆ
ಗೂಟ-ವನ್ನು
ಗೂಡಿನ
ಗೂಡಿ-ನಲ್ಲಿ
ಗೂಢಾರ್ಥ
ಗೃಹ-ಕೃ-ತ್ಯ-ಗಳನ್ನು
ಗೃಹ-ಕೃ-ತ್ಯ-ವನ್ನು
ಗೃಹಸ್ಥ
ಗೃಹ-ಸ್ಥ-ನಾದ
ಗೃಹ-ಸ್ಥರ
ಗೃಹ-ಸ್ಥ-ರಂತೆ
ಗೃಹ-ಸ್ಥರು
ಗೃಹ-ಸ್ಥಾ-ಶ್ರ-ಮಿ-ಗ-ಳಾ-ದರು
ಗೃಹಿ-ಣಿಯೂ
ಗೆದ್ದ
ಗೆದ್ದ-ವನು
ಗೆರೆಯೆ
ಗೆಲ್ಲ-ಬೇ-ಕಾ-ದರೆ
ಗೆಲ್ಲು-ವನು
ಗೆಲ್ಲು-ವರು
ಗೆಳತಿ
ಗೆಳೆಯ
ಗೆಳೆ-ಯ-ನನ್ನು
ಗೆಳೆ-ಯ-ನಿಗೆ
ಗೇಡು
ಗೈರು
ಗೈವ
ಗೊಂತಿನ
ಗೊಂತಿ-ನ-ಲ್ಲಿ-ರು-ತ್ತ-ವೆಯೆ
ಗೊಂದ-ಲ-ದಿಂದ
ಗೊಂದ-ಲ-ವನ್ನು
ಗೊಂದ-ಲ-ವೆ-ದ್ದಿತು
ಗೊಂದು
ಗೊಂಬೆ
ಗೊಂಬೆಯ
ಗೊಂಬೆ-ಯೊಂ-ದನ್ನು
ಗೊಣ-ಗಾ-ಡು-ವು-ದಿಲ್ಲ
ಗೊತ್ತಾ
ಗೊತ್ತಾಗ
ಗೊತ್ತಾ-ಗ-ದಿ-ರಲಿ
ಗೊತ್ತಾ-ಗ-ಬೇಕು
ಗೊತ್ತಾ-ಗ-ಲಿಲ್ಲ
ಗೊತ್ತಾಗಿ
ಗೊತ್ತಾ-ಗಿದೆ
ಗೊತ್ತಾ-ಗು-ತ್ತಿದೆ
ಗೊತ್ತಾ-ಗುವು
ಗೊತ್ತಾ-ಗು-ವುದು
ಗೊತ್ತಾಯಿ
ಗೊತ್ತಾ-ಯಿತು
ಗೊತ್ತಾ-ಯಿತೆ
ಗೊತ್ತಾ-ಯಿತೊ
ಗೊತ್ತಿದೆ
ಗೊತ್ತಿ-ದೆಯೆ
ಗೊತ್ತಿ-ದೆಯೊ
ಗೊತ್ತಿ-ದ್ದ-ವ-ರಿಗೆ
ಗೊತ್ತಿ-ದ್ದುವು
ಗೊತ್ತಿರ
ಗೊತ್ತಿ-ರ-ಲಿಲ್ಲ
ಗೊತ್ತಿರು
ಗೊತ್ತಿ-ರು-ತ್ತದೆ
ಗೊತ್ತಿ-ರುವ
ಗೊತ್ತಿ-ರು-ವುದು
ಗೊತ್ತಿಲ್ಲ
ಗೊತ್ತಿ-ಲ್ಲದ
ಗೊತ್ತಿ-ಲ್ಲ-ದ-ವನು
ಗೊತ್ತಿ-ಲ್ಲದೆ
ಗೊತ್ತಿ-ಲ್ಲವೆ
ಗೊತ್ತಿವೆ
ಗೊತ್ತು
ಗೊತ್ತು-ಮಾ-ಡಿದ
ಗೊತ್ತು-ಹ-ಚ್ಚು-ವರು
ಗೊತ್ತೆ
ಗೊರಕೆ
ಗೊಲ್ಲ
ಗೊಲ್ಲರು
ಗೊಳೋ
ಗೋಚ-ರಿ-ಸಿ-ದವು
ಗೋಚ-ರಿ-ಸು-ವನು
ಗೋಡೆ
ಗೋಡೆಯ
ಗೋಡೆ-ಯಿ-ರು-ವುದನ್ನು
ಗೋಪಾಲ
ಗೋಪಾ-ಲನ
ಗೋಪಾ-ಲ-ನನ್ನು
ಗೋಪಿ-ಯರು
ಗೋಪಿ-ಯರೂ
ಗೋಲೂಕ್
ಗೋಳನ್ನು
ಗೋಳಾ-ಡು-ತ್ತಿ-ದ್ದಳು
ಗೋಳು
ಗೋಳೋ
ಗೋವನ್ನು
ಗೋವಿಂದ
ಗೋವಿಂ-ದಜಿ
ಗೋವಿಂ-ದನೇ
ಗೋವಿನ
ಗೋಶಾ-ಲೆ-ಯಲ್ಲಿ
ಗೋಸ್ವಾ-ಮಿಯ
ಗೋಹತ್ಯಾ
ಗೌತಮ
ಗೌತಮ್
ಗೌರವ
ಗೌರ-ವಕ್ಕೆ
ಗೌರ-ವ-ದಿಂದ
ಗೌರ-ವ-ವನ್ನು
ಗೌರ-ವ-ಸ್ಥರ
ಗೌರ-ವಿಸು
ಗೌರ-ವಿ-ಸು-ತ್ತಿದ್ದ
ಗೌರ-ವಿ-ಸು-ತ್ತಿ-ದ್ದರು
ಗೌರಾಂಗ
ಗೌರಾಂ-ಗ-ದೇವ
ಗೌರಾಂ-ಗ-ದೇ-ವನ
ಗೌರಾಂ-ಗನ
ಗೌರ್
ಗ್ರಂಥ
ಗ್ರಂಥ-ಗಳನ್ನು
ಗ್ರತೆ-ಯನ್ನು
ಗ್ರಹ
ಗ್ರಹ-ಚಾ-ರ-ವನ್ನು
ಗ್ರಹ-ವಿ-ಲ್ಲದೆ
ಗ್ರಹಿಸಿ
ಗ್ರಹಿ-ಸಿ-ದರು
ಗ್ರಾಮ-ದಲ್ಲಿ
ಘಂಟಾ-ಕರ್ಣ
ಘಂಟಾ-ಕ-ರ್ಣ-ನಂತೆ
ಘಟ-ನೆ-ಗಳನ್ನು
ಘಟ-ನೆ-ಯನ್ನು
ಘಟ-ಸರ್ಪ
ಘನೀ-ಭೂ-ತ-ಳಾ-ಗಿದ್ದ
ಘೋರ-ಪಾ-ಪಕ್ಕೆ
ಚಂಗ್
ಚಂಡಾ-ಲ-ನಾಗು
ಚಂಡಿ-ಯಲ್ಲಿ
ಚಂದ-ನದ
ಚಂದಾ
ಚಕ್ರ
ಚಕ್ರ-ವರ್ತಿ
ಚಕ್ರ-ವ-ರ್ತಿ-ಯನ್ನು
ಚತು-ರ್ಮು-ಖಾಂ
ಚದು-ರಿ-ಸಿತು
ಚಪ್ಪಾಳೆ
ಚಮ್ಮಾ-ರ-ನನ್ನು
ಚಯ-ವಾ-ಗು-ವುದು
ಚರಂಡಿ
ಚರಂ-ಡಿ-ಯಲ್ಲಿ
ಚರ್ಚಿ-ಸುತ್ತಿ
ಚರ್ಚೆ
ಚರ್ಮ-ದಿಂದ
ಚಲ-ನೆಯ
ಚಲಿ-ಸ-ತೊ-ಡ-ಗಿತು
ಚಲಿ-ಸ-ಲಿಲ್ಲ
ಚಲಿಸಿ
ಚಲಿ-ಸುತ್ತ
ಚಲಿ-ಸು-ವು-ದಕ್ಕೇ
ಚಾಕರಿ
ಚಾಕ-ರಿ-ಗಾಗಿ
ಚಾಚಿ
ಚಾಚಿ-ದಾಗ
ಚಾರ
ಚಾರ-ವನ್ನು
ಚಾರ್ಯನ
ಚಿಂತನೆ
ಚಿಂತಾ
ಚಿಂತಾ-ಕ್ರಾಂ-ತಳಾ
ಚಿಂತಾ-ಕ್ರಾಂ-ತ-ಳಾ-ಗು-ವುದು
ಚಿಂತಿ
ಚಿಂತಿಸಿ
ಚಿಂತಿ-ಸಿ-ದ-ರೇನೆ
ಚಿಂತಿ-ಸಿ-ದಾಗ
ಚಿಂತಿಸು
ಚಿಂತಿ-ಸುತ್ತ
ಚಿಂತಿ-ಸು-ತ್ತಿ-ದ್ದರೆ
ಚಿಂತಿ-ಸು-ತ್ತಿ-ರು-ವನು
ಚಿಂತಿ-ಸು-ತ್ತಿ-ರು-ವನೊ
ಚಿಂತಿ-ಸು-ತ್ತಿ-ರು-ವಾಗ
ಚಿಂತಿ-ಸು-ವಂತೆ
ಚಿಂತಿ-ಸು-ವಾಗ
ಚಿಂತಿ-ಸು-ವಾ-ಗಲೂ
ಚಿಂತಿ-ಸು-ವು-ದಕ್ಕೆ
ಚಿಂತಿ-ಸು-ವು-ದಿಲ್ಲ
ಚಿಂತೆ-ಯಿಲ್ಲ
ಚಿಂದಿ-ಗಾಗಿ
ಚಿಕಿತ್ಸೆ
ಚಿಕ್ಕಂ-ದಿ-ನಿಂ-ದಲೇ
ಚಿಟಿಕೆ
ಚಿಟ್ಟೆ-ಗಳನ್ನು
ಚಿಟ್ಟೆ-ಯನ್ನು
ಚಿತ್ಶ-ಕ್ತಿ-ಯಂತೆ
ಚಿನ್ನಕ್ಕೆ
ಚಿನ್ನದ
ಚಿನ್ನ-ವನ್ನು
ಚಿನ್ನ-ವಿತ್ತು
ಚಿನ್ಮಯ
ಚಿನ್ಮ-ಯ-ವಾ-ಗಿ-ರು-ವುದನ್ನು
ಚಿಮು-ಕಿಸಿ
ಚಿವುಟಿ
ಚಿಹ್ನೆ
ಚಿಹ್ನೆಯು
ಚಿಹ್ನೆಯೂ
ಚಿಹ್ನೆಯೇ
ಚುಚ್ಚಿ-ಕೊಂ-ಡಿ-ರು-ವಾಗ
ಚುಚ್ಚು-ವು-ದಿಲ್ಲ
ಚುಪ್ಚುಪ್
ಚೂಪಾದ
ಚೆನ್ನಾಗಿ
ಚೆನ್ನಾ-ಗಿತ್ತು
ಚೆನ್ನಾ-ಗಿದೆ
ಚೆನ್ನಾ-ಗಿ-ದ್ದರೆ
ಚೆನ್ನಾ-ಗಿ-ದ್ದಾಳೆ
ಚೆನ್ನಾ-ಗಿದ್ದೆ
ಚೆನ್ನಾ-ಗಿ-ದ್ದೇನೆ
ಚೆನ್ನಾ-ಗಿಯೇ
ಚೆನ್ನಾ-ಗಿ-ರುವ
ಚೆನ್ನಾ-ಗಿ-ರು-ವುದು
ಚೆನ್ನಾ-ಗಿ-ರು-ವೆನು
ಚೆನ್ನಾ-ಗಿ-ರೋಣ
ಚೆನ್ನಾದ
ಚೆಲ್ಲ-ಬಾ-ರದು
ಚೆಲ್ಲಿ-ಬಿ-ಟ್ಟರು
ಚೆಲ್ಲಿ-ಹೋಗು
ಚೈತ-ನ್ಯ-ದೇವ
ಚೈತ-ನ್ಯ-ದೇ-ವರು
ಚೈತ-ನ್ಯ-ಮ-ಯ-ವಾಗಿ
ಚೈತ-ನ್ಯರು
ಚೈತ್ರ-ಮಾ-ಸ-ದಲ್ಲಿ
ಚೌಕಾಶಿ
ಚೌಧುರಿ
ಚ್ಚರಿಕೆ
ಛತ್ರಕ್ಕೆ
ಛೇ
ಜಂಬ
ಜಂಭ
ಜಂಭ-ಕೊ-ಚ್ಚಿ-ಕೊ-ಳ್ಳು-ತ್ತಿದ್ದ
ಜಗ
ಜಗದ
ಜಗ-ದಂಬ
ಜಗ-ದಂಬಾ
ಜಗ-ನ್ನಾಥ
ಜಗ-ನ್ನಾ-ಥ-ನನ್ನು
ಜಗ-ನ್ಮಾತೆ
ಜಗ-ನ್ಮಾ-ತೆಗೆ
ಜಗ-ನ್ಮಾ-ತೆ-ಯಾದ
ಜಗಳ
ಜಗ-ಳ-ವಾ-ಡು-ತ್ತಿ-ರು-ವಿರಿ
ಜಟಿಲ
ಜಟಿ-ಲ-ವಾದ
ಜಡ-ವಾಗಿ
ಜತೆ
ಜತೆ-ಗಾ-ರ-ನಾ-ಗು-ವನು
ಜತೆಗೆ
ಜತೆ-ಯಲ್ಲಿ
ಜದು-ಮ-ಲ್ಲಿಕ
ಜನ
ಜನಕ
ಜನ-ಕನ
ಜನ-ಕರು
ಜನನ
ಜನರ
ಜನ-ರನ್ನು
ಜನ-ರಿಂದ
ಜನ-ರಿಗೆ
ಜನ-ರಿ-ಗೆಲ್ಲ
ಜನರು
ಜನರೂ
ಜನ-ರೆಲ್ಲ
ಜನ-ವನ್ನು
ಜನ-ಸಾ-ಮಾ-ನ್ಯ-ರಲ್ಲಿ
ಜನಿಸಿ
ಜನಿ-ಸಿದ
ಜನ್ಮ
ಜನ್ಮ-ಗಳ
ಜನ್ಮ-ಗಳಿಂದ
ಜನ್ಮ-ಗಳು
ಜನ್ಮ-ತಾ-ಳಿದ
ಜನ್ಮದ
ಜನ್ಮ-ದಲ್ಲಿ
ಜನ್ಮ-ವನ್ನು
ಜನ್ಮ-ವೆ-ತ್ತ-ಬೇ-ಕಾ-ದರೂ
ಜನ್ಮ-ವೆ-ತ್ತಿ-ದಳು
ಜನ್ಮ-ವೆ-ತ್ತಿದ್ದ
ಜಪ
ಜಪ-ಮಾಲೆ
ಜಪ-ವನ್ನು
ಜಪ-ಸರ
ಜಪಿ-ಸು-ತ್ತಿ-ದ್ದರೆ
ಜಪಿ-ಸು-ತ್ತಿ-ರು-ವುದು
ಜಮಾ
ಜಮೀಂ-ದಾರ
ಜಮೀ-ನನ್ನು
ಜಮೀ-ನಿಗೆ
ಜಮೀನು
ಜಮೀ-ನ್ದಾರ
ಜಮೀ-ನ್ದಾ-ರ-ನಾದ
ಜಮೀ-ನ್ದಾ-ರ-ನಿಗೆ
ಜಮೀ-ನ್ದಾ-ರನು
ಜಯ-ಪು-ರದ
ಜಯ-ವಾ-ಗಲಿ
ಜಯ-ಶೀ-ಲ-ನಾದ
ಜರರ
ಜಾಗ-ದಲ್ಲಿ
ಜಾಗ-ರೂ-ಕ-ತೆ-ಯಿಂದ
ಜಾಗ-ರೂ-ಕ-ನಾ-ಗಿ-ರ-ಬೇಕು
ಜಾಗೃ-ತ-ಗೊ-ಳಿ-ಸು-ತ್ತಿದೆ
ಜಾಗೃ-ತ-ನಾ-ಗಿ-ರ-ಬೇಕು
ಜಾಗೃ-ತ-ನಾ-ಗಿರು
ಜಾಗೃ-ತ-ವೆಂದರೆ
ಜಾಗೃ-ತಾ-ವ-ಸ್ಥೆಯೂ
ಜಾಡ-ಮಾಲಿ
ಜಾಡಿ
ಜಾಡ್ಯ-ವನ್ನು
ಜಾತಿಗೆ
ಜಾತ್ರೆ-ಯನ್ನು
ಜಾರಿದ
ಜಾಸ್ತಿ
ಜಾಸ್ತಿ-ಕೊ-ಟ್ಟರೆ
ಜಾಸ್ತಿ-ಕೊ-ಡು-ವು-ದಿ-ಲ್ಲ-ವಂತೆ
ಜಾಸ್ತಿ-ಯಾ-ಗುತ್ತ
ಜಿಗಿ-ದರೆ
ಜಿಗಿ-ಯಿತು
ಜಿಪುಣ
ಜೀವಂ-ತ-ಳಾ-ಗಿ-ರು-ವು-ದ-ರಿಂದ
ಜೀವನ
ಜೀವ-ನದ
ಜೀವ-ನ-ದಲ್ಲಿ
ಜೀವ-ನ-ದಿಂದ
ಜೀವ-ನ-ವನ್ನು
ಜೀವ-ನ-ವೆಂ-ಬುದು
ಜೀವ-ನೋ-ಪಾಯ
ಜೀವ-ನೋ-ಪಾ-ಯ-ವನ್ನು
ಜೀವ-ರಿಗೆ
ಜೀವ-ವನ್ನು
ಜೀವ-ವಿ-ರು-ವು-ದ-ರಿಂದ
ಜೀವಾ-ವಧಿ
ಜೀವಿ
ಜೀವಿ-ಗ-ಳಿ-ದ್ದಂತೆ
ಜೀವಿ-ಗಳು
ಜೀವಿ-ಸುತ್ತ
ಜೀವಿ-ಸು-ತ್ತಿದ್ದ
ಜೀವಿ-ಸು-ತ್ತಿ-ರುವೆ
ಜುಮ್
ಜೂನ್
ಜೇಡಿ-ಮ-ಣ್ಣಿನ
ಜೇನು
ಜೇನು-ಗೂ-ಡನ್ನು
ಜೇನು-ತು-ಪ್ಪ-ವನ್ನು
ಜೇನು-ನೊಣ
ಜೇನು-ನೊ-ಣಕ್ಕೆ
ಜೇನು-ನೊ-ಣ-ವನ್ನು
ಜೇನು-ನೊ-ಣವೇ
ಜೈಲಿಗೆ
ಜೊಂಪೆ-ಗಳು
ಜೊತೆ-ಗಾರ
ಜೊತೆಗೆ
ಜೊತೆ-ಯಲ್ಲಿ
ಜೊಳ್ಳು
ಜೋಪಾನ
ಜೋಪಾ-ನ-ವಾಗಿ
ಜೋಪಾ-ನ-ವಾ-ಗಿರು
ಜೋರಾಗಿ
ಜೋಲಾ-ಡು-ತ್ತಿ-ರು-ವರು
ಜ್ಜನ-ನಿಯ
ಜ್ಞಾನ
ಜ್ಞಾನ-ಅ-ಜ್ಞಾ-ನ-ಗ-ಳಿಗೆ
ಜ್ಞಾನ-ಕ್ಕಿಂತ
ಜ್ಞಾನದ
ಜ್ಞಾನ-ಪ್ರಾ-ಪ್ತಿ-ಯಾ-ಗಿರ
ಜ್ಞಾನ-ಪ್ರಾ-ಪ್ತಿ-ಯಾದ
ಜ್ಞಾನ-ಭಂ-ಡಾ-ರಕ್ಕೆ
ಜ್ಞಾನ-ವನ್ನು
ಜ್ಞಾನ-ವಿದೆ
ಜ್ಞಾನ-ವಿ-ದೆಯೋ
ಜ್ಞಾನವೂ
ಜ್ಞಾನವೇ
ಜ್ಞಾನಿ
ಜ್ಞಾನಿ-ಗ-ಳಾದ
ಜ್ಞಾನಿಗೂ
ಜ್ಞಾನಿಯ
ಜ್ಞಾನೇಂ-ದ್ರಿ-ಯ-ಗಳು
ಜ್ಞಾನೋ-ದ-ಯ-ವಾ-ದಾ-ಗ-ಬ್ರ-ಹ್ಮ-ಸಾ-ಕ್ಷಾ-ತ್ಕಾ-ರ-ವಾ-ಗು-ವುದು
ಜ್ಞಾಪ-ಕಕ್ಕೆ
ಜ್ಞಾಪ-ಕ-ದ-ಲ್ಲಿ-ಟ್ಟಿ-ರು-ವೆನು
ಜ್ಞಾಪ-ಕವೇ
ಜ್ಞಾಪಿಸಿ
ಝಾಡ-ಮಾಲಿ
ಟನು
ಟವ-ಲನ್ನು
ಟವೆಲ್
ಠಣ್
ಠಾಕೂ-ರರೇ
ಠಾಕೂ-ರ-ರೊಂ-ದಿಗೆ
ಠೀವಿ-ಯಿಂದ
ಡಂಗುರ
ಡಕಾ
ಡಕಾ-ಯಿತ
ಡವ-ಡವ
ಡುವೆ
ಡೆಪ್ಯುಟಿ
ಡೆಪ್ಯೂಟಿ
ಡೆಲ್ಲಿ-ಯನ್ನು
ಡ್ಯಾಮ್
ಣಿಕನು
ತಂಗಿ
ತಂಟೆಗೆ
ತಂಡ
ತಂತಿ-ಯನ್ನು
ತಂತಿ-ಯಿಂದ
ತಂದ
ತಂದರು
ತಂದಳು
ತಂದಿ-ದ್ದಳು
ತಂದಿ-ದ್ದೀಯೆ
ತಂದು
ತಂದು-ಕೊ-ಟ್ಟಳು
ತಂದು-ಕೊ-ಡು-ವೆನು
ತಂದೆ
ತಂದೆಗೆ
ತಂದೆ-ತಾ-ಯಿ-ಗಳು
ತಂದೆ-ತಾ-ಯಿ-ಯರ
ತಂದೆಯ
ತಂದೆ-ಯನ್ನು
ತಂದೆ-ಯಲ್ಲ
ತಂದೆಯು
ತಂಬಾ-ಕನ್ನು
ತಕ್ಕ
ತಕ್ಕಂತೆ
ತಕ್ಷಣ
ತಕ್ಷ-ಣವೆ
ತಕ್ಷ-ಣವೇ
ತಗ-ಲಿದೆ
ತಗ್ಗಿ
ತಗ್ಗಿತು
ತಟ್ಟಿತು
ತಟ್ಟಿದ
ತಟ್ಟೆಗೆ
ತಟ್ಟೆ-ಯಲ್ಲಿ
ತಡ
ತಡವಾ
ತಡೆ-ಯ-ಲಾಗ
ತಡೆ-ಯ-ಲಾ-ಗದೆ
ತಡೆ-ಯಿರಿ
ತತ್ಕ್ಷ-ಣವೆ
ತತ್ಕ್ಷ-ಣವೇ
ತತ್ತ್ವ
ತತ್ತ್ವ-ಗಳನ್ನು
ತತ್ತ್ವ-ಗಳೂ
ತತ್ವ-ಶಾ-ಸ್ತ್ರ-ಗಳನ್ನೂ
ತನಕ
ತನ-ಗಿಂತ
ತನಗೆ
ತನ-ಗೇನೂ
ತನ್ನ
ತನ್ನಂ-ತಹ
ತನ್ನನ್ನು
ತನ್ನಲ್ಲಿ
ತನ್ನ-ಲ್ಲಿದ್ದ
ತನ್ನ-ಲ್ಲಿ-ರುವ
ತನ್ನಾ-ತ್ಮ-ವನ್ನು
ತನ್ನಿ
ತನ್ನಿಂದ
ತನ್ನೊ-ಡನೆ
ತನ್ಮ-ಯ-ನಾಗು
ತನ್ಮ-ಯ-ನಾ-ದಾಗ
ತನ್ಮ-ಯ-ರಾಗಿ
ತಪ-ಶ್ಶಕ್ತಿ-ಯಿಂದ
ತಪ-ಸ್ಸನ್ನೂ
ತಪ-ಸ್ಸಿನ
ತಪಸ್ಸು
ತಪ-ಸ್ಸೆಲ್ಲ
ತಪೋ-ಬ-ಲ-ದಿಂದ
ತಪೋ-ಶ-ಕ್ತಿ-ಯೆಲ್ಲ
ತಪ್ಪನ್ನು
ತಪ್ಪಾಗಿ
ತಪ್ಪಿ
ತಪ್ಪಿಗೆ
ತಪ್ಪಿತು
ತಪ್ಪಿದ
ತಪ್ಪಿ-ದ್ದಲ್ಲ
ತಪ್ಪಿಯೂ
ತಪ್ಪಿ-ಸಿ-ಕೊಂ-ಡರು
ತಪ್ಪಿ-ಸಿ-ಕೊಂಡು
ತಪ್ಪಿ-ಸಿ-ಕೊ-ಳ್ಳಲು
ತಪ್ಪು
ತಪ್ಪು-ತ್ತದೆ
ತಪ್ಪೇನು
ತಬ್ಬಿ-ಕೊಂಡು
ತಮಗೆ
ತಮಸ್ಸು
ತಮಾ-ಷೆ-ಯನ್ನು
ತಮ್ಮ
ತಮ್ಮಟೆ
ತಮ್ಮ-ಟೆ-ಗ-ಳೊಂ-ದಿಗೆ
ತಮ್ಮ-ದಾ-ಗಿ-ಸಿ-ಕೊಂ-ಡರು
ತಮ್ಮದೇ
ತಮ್ಮನ
ತಮ್ಮ-ನನ್ನು
ತಮ್ಮನೇ
ತಮ್ಮಲ್ಲಿ
ತಮ್ಮ-ಲ್ಲಿದ್ದ
ತರ-ಬ-ಹುದು
ತರ-ಲಾ-ಗ-ಲಿಲ್ಲ
ತರ-ಲಿಲ್ಲ
ತರಲು
ತರುಣ
ತರುಣಿ
ತರು-ಣಿಯ
ತರು-ಣಿ-ಯ-ನ್ನ-ಲ್ಲಿಟ್ಟು
ತರು-ತ್ತಿ-ದ್ದಳು
ತರು-ತ್ತಿ-ರು-ವೆ-ಯಲ್ಲಾ
ತರು-ವಾಯ
ತರು-ವು-ದಕ್ಕೆ
ತರು-ವುದು
ತರುವೆ
ತರ್ಕ
ತಲು-ಪಲು
ತಲೆ
ತಲೆ-ತ-ಗ್ಗಿಸಿ
ತಲೆ-ತ-ಗ್ಗಿ-ಸಿ-ದಳು
ತಲೆ-ಮಾ-ರಿನ
ತಲೆ-ಯನ್ನು
ತಲೆ-ಹಾ-ಕು-ವು-ದಿಲ್ಲ
ತಲ್ಲೀನ
ತಲ್ಲೀ-ನ-ನಾ-ಗಿ-ದ್ದ-ವನು
ತಲ್ಲೀ-ನ-ನಾದ
ತಲ್ಲೀ-ನ-ರಾ-ಗಿ-ರು-ತ್ತಿ-ದ್ದರು
ತಲ್ಲೀ-ನ-ವಾಗಿ
ತಲ್ಲೀ-ನ-ವಾ-ಗಿ-ರಲಿ
ತಳ-ಕೆ-ಳ-ಗಾ-ಯಿತು
ತಳ್ಳ-ಬೇಕು
ತವ-ಕಿ-ಸು-ತ್ತಿತ್ತು
ತಷ್ಟೆ
ತಾಕಿತು
ತಾಕಿತ್ತು
ತಾಗಿತು
ತಾತ್ತ್ವಿಕ
ತಾನು
ತಾನುಂಡ
ತಾನೆ
ತಾನೆಲ್ಲಿ
ತಾನೇ
ತಾನೊಂದು
ತಾನೊ-ಬ್ಬನೆ
ತಾಪ-ತ್ರಯ
ತಾಮ್ರದ
ತಾಯಿ
ತಾಯಿಗೆ
ತಾಯಿತು
ತಾಯಿಯ
ತಾಯಿ-ಯಂತೆ
ತಾಯಿ-ಯತ್ತ
ತಾಯಿ-ಯಾದ
ತಾಯಿ-ಯಾ-ದಳು
ತಾಯಿ-ಯಿ-ರು-ವೆ-ಡೆಗೆ
ತಾಯಿ-ಹಕ್ಕಿ
ತಾಯಿ-ಹ-ಕ್ಕಿ-ಯನ್ನು
ತಾರ-ತ-ಮ್ಯ-ಭಾವ
ತಾರಾ-ಡು-ವಂತೆ
ತಾಳ-ಮೇ-ಳ-ದೊಂ-ದಿಗೆ
ತಾಳ-ಲಾ-ರದೆ
ತಾಳಿ-ರು-ವನು
ತಾಳು
ತಾಳು-ವುದು
ತಾಳೆಯ
ತಾವು
ತಾವೇ
ತಿಂಗಳ
ತಿಂಗಳಲ್ಲಿ
ತಿಂಗ-ಳಿಗೆ
ತಿಂಗಳು
ತಿಂಗ-ಳು-ಗಳು
ತಿಂಡಿ
ತಿಂಡಿ-ತೀ-ರ್ಥ-ಗಳು
ತಿಂದರು
ತಿಂದರೂ
ತಿಂದರೆ
ತಿಂದಾದ
ತಿಂದಿ-ರ-ದಿ-ದ್ದರೆ
ತಿಂದು
ತಿಂದು-ಕೊಂ-ಡಿದ್ದು
ತಿಂದು-ಬಿಟ್ಟ
ತಿಂದು-ಹಾ-ಕು-ವಂತೆ
ತಿಂದೊ-ಡ-ನೆಯೆ
ತಿದ್ದಿ-ಕೊಂಡು
ತಿನ್ನ-ತೊ-ಡ-ಗಿತು
ತಿನ್ನನು
ತಿನ್ನ-ಬ-ಹುದು
ತಿನ್ನ-ಬಾ-ರದು
ತಿನ್ನ-ಬೇ-ಕೆಂ-ದಿ-ರುವೆ
ತಿನ್ನ-ಬೇಡ
ತಿನ್ನ-ಲಾ-ರದೆ
ತಿನ್ನಲು
ತಿನ್ನಲೂ
ತಿನ್ನಿ
ತಿನ್ನು
ತಿನ್ನು-ತ್ತಲೂ
ತಿನ್ನು-ತ್ತಿತ್ತು
ತಿನ್ನು-ತ್ತಿದ್ದ
ತಿನ್ನು-ತ್ತಿ-ದ್ದು-ದನ್ನು
ತಿನ್ನು-ತ್ತಿ-ರು-ವಿರಿ
ತಿನ್ನು-ತ್ತಿ-ರು-ವೆ-ಯಲ್ಲ
ತಿನ್ನು-ತ್ತಿವೆ
ತಿನ್ನು-ತ್ತೇನೆ
ತಿನ್ನು-ವರೋ
ತಿನ್ನು-ವು-ದ-ರ-ಲ್ಲಿದೆ
ತಿಯ
ತಿರ-ಸ್ಕಾರ
ತಿರು-ಗಾ-ಡು-ತ್ತಿ-ದ್ದನು
ತಿರು-ಗಾ-ಡು-ತ್ತಿ-ದ್ದಾಗ
ತಿರು-ಗಿ-ಸಲು
ತಿರು-ಗಿಸಿ
ತಿರು-ಗಿ-ಸು-ತ್ತದೆ
ತಿರುಪೆ
ತಿಳಿ
ತಿಳಿ-ಗೇಡಿ
ತಿಳಿದ
ತಿಳಿ-ದ-ನಂ-ತರ
ತಿಳಿ-ದಾದ
ತಿಳಿ-ದಿತ್ತು
ತಿಳಿ-ದಿ-ದ್ದೇನೆ
ತಿಳಿ-ದಿ-ರ-ಬೇಕು
ತಿಳಿ-ದಿ-ರು-ತ್ತದೆ
ತಿಳಿ-ದಿ-ರು-ವಂತೆ
ತಿಳಿದು
ತಿಳಿ-ದು-ಕೊಂಡ
ತಿಳಿ-ದು-ಕೊಂ-ಡರು
ತಿಳಿ-ದು-ಕೊಂ-ಡಿದ್ದ
ತಿಳಿ-ದು-ಕೊಂ-ಡಿ-ದ್ದರು
ತಿಳಿ-ದು-ಕೊಂ-ಡಿ-ರ-ಬೇಕು
ತಿಳಿ-ದು-ಕೊಂ-ಡಿ-ರು-ವಷ್ಟು
ತಿಳಿ-ದು-ಕೊಂ-ಡಿ-ರು-ವೆನು
ತಿಳಿ-ದು-ಕೊಂಡು
ತಿಳಿ-ದು-ಕೊ-ಳ್ಳ-ದಿ-ರು-ವಾಗ
ತಿಳಿ-ದು-ಕೊ-ಳ್ಳ-ಬಲ್ಲ
ತಿಳಿ-ದು-ಕೊ-ಳ್ಳ-ಬೇ-ಕೆಂಬ
ತಿಳಿ-ದು-ಕೊ-ಳ್ಳ-ಲಾ-ರರು
ತಿಳಿ-ದು-ಕೊ-ಳ್ಳ-ಲಿಲ್ಲ
ತಿಳಿ-ದು-ಕೊ-ಳ್ಳಲು
ತಿಳಿ-ದು-ಕೊಳ್ಳಿ
ತಿಳಿ-ದು-ಬಂತು
ತಿಳಿ-ಯದು
ತಿಳಿ-ಯದೆ
ತಿಳಿ-ಯ-ಬಾ-ರದು
ತಿಳಿ-ಯಲು
ತಿಳಿ-ಯ-ಲೆ-ತ್ನಿ-ಸಿದ
ತಿಳಿ-ಯಾ-ಗಿತ್ತು
ತಿಳಿ-ಯಿತು
ತಿಳಿ-ಯುತ್ತ
ತಿಳಿ-ಯು-ವರು
ತಿಳಿ-ಯು-ವು-ದು-ಕೆ-ಲವು
ತಿಳಿ-ಸ-ಲಾರ
ತಿಳಿಸಿ
ತಿಳಿ-ಸಿ-ಕೊ-ಡು-ವ-ವ-ರಾರು
ತಿಳಿ-ಸಿದೆ
ತಿಳಿ-ಸಿರಿ
ತಿಳಿಸು
ತಿಳಿ-ಹೇ-ಳಿ-ದ್ದಾರೆ
ತೀಕ್ಷ
ತೀರಕ್ಕೆ
ತೀರ-ದಲ್ಲಿ
ತೀರ-ದ-ಲ್ಲಿದ್ದ
ತೀರ-ದ-ಲ್ಲಿ-ರುವ
ತೀರಿ-ಕೊಂಡ
ತೀರಿ-ಕೊಂ-ಡನು
ತೀರಿದ
ತೀರಿ-ಹೋದ
ತೀರ್ಥ
ತೀವ್ರ
ತುಂಡು
ತುಂಬ
ತುಂಬದೆ
ತುಂಬಲಿ
ತುಂಬಾ
ತುಂಬಿ
ತುಂಬಿ-ಕೊಂ-ಡಿ-ರು-ವಳು
ತುಂಬಿತು
ತುಂಬಿತ್ತು
ತುಂಬಿದ
ತುಂಬಿದೆ
ತುಂಬಿ-ಸಿ-ದಾಗ
ತುಂಬಿ-ಸುವ
ತುಂಬಿ-ಹೋ-ಗಿವೆ
ತುಂಬು-ವುದು
ತುತ್ತನ್ನು
ತುತ್ತಾ-ಗು-ವರು
ತುದಿ-ಯಲ್ಲಿ
ತುಪ್ಪ-ವನ್ನು
ತುಳಿಸಿ
ತುಹು
ತೂಕ-ವನ್ನು
ತೂಗು-ವಾಗ
ತೂತಿ-ನ-ಮ-ಯ-ವಾ-ಗಿತ್ತು
ತೃಪ್ತ
ತೃಪ್ತಿ-ಪ-ಡಿ-ಸ-ಲಾ-ರರು
ತೃಪ್ತಿ-ಯಾ-ಗುವ
ತೃಪ್ತಿ-ಯಾ-ಗು-ವುದು
ತೃಪ್ತಿ-ಯಾ-ಯಿತು
ತೆಂಗಿ-ನ-ಕಾ-ಯಿ-ಯನ್ನು
ತೆಂಗಿ-ನ-ಕಾ-ಯಿ-ಯಲ್ಲಿ
ತೆಗೆದ
ತೆಗೆ-ದನು
ತೆಗೆ-ದರೂ
ತೆಗೆ-ದಾಗ
ತೆಗೆದು
ತೆಗೆ-ದುಕೊ
ತೆಗೆ-ದು-ಕೊಂಡ
ತೆಗೆ-ದು-ಕೊಂ-ಡರೆ
ತೆಗೆ-ದು-ಕೊಂ-ಡಳು
ತೆಗೆ-ದು-ಕೊಂ-ಡಿತು
ತೆಗೆ-ದು-ಕೊಂ-ಡಿ-ದ್ದಾಳೆ
ತೆಗೆ-ದು-ಕೊಂ-ಡಿ-ರು-ವಂ-ತಿದೆ
ತೆಗೆ-ದು-ಕೊಂ-ಡಿಲ್ಲ
ತೆಗೆ-ದು-ಕೊಂಡು
ತೆಗೆ-ದು-ಕೊಳ್ಳ
ತೆಗೆ-ದು-ಕೊ-ಳ್ಳ-ಬ-ಹುದು
ತೆಗೆ-ದು-ಕೊ-ಳ್ಳಲು
ತೆಗೆ-ದು-ಕೊಳ್ಳಿ
ತೆಗೆ-ದು-ಕೊಳ್ಳು
ತೆಗೆ-ದು-ಕೊ-ಳ್ಳು-ತ್ತಿ-ದ್ದರು
ತೆಗೆ-ದು-ಕೊ-ಳ್ಳು-ತ್ತೀ-ಯೇನು
ತೆಗೆ-ದು-ಕೊ-ಳ್ಳು-ವು-ದ-ಕ್ಕಾಗಿ
ತೆಗೆ-ದು-ಕೊ-ಳ್ಳು-ವು-ದು-ಇ-ವನ್ನು
ತೆಗೆ-ದು-ನೋಡಿ
ತೆಗೆ-ಯದೆ
ತೆಗೆ-ಯ-ಬೇಕು
ತೆಗೆ-ಯ-ಲಾ-ಯಿತು
ತೆಗೆ-ಯಲು
ತೆಗೆ-ಯು-ತ್ತಿ-ದ್ದಂತೆ
ತೆಪ್ಪಗೆ
ತೆರೆ-ದನು
ತೆರೆದು
ತೆರೆಯ
ತೆರೆ-ಯು-ವವು
ತೇಲಿ
ತೇಲುವ
ತೈಲದ
ತೈಲ-ವನ್ನು
ತೊಂದರೆ
ತೊಂದ-ರೆ-ಯನ್ನು
ತೊಟ್ಟು
ತೊಟ್ಟೂ
ತೊಡಗಿ
ತೊಡ-ಗಿತು
ತೊರೆದು
ತೋಚ-ಲಿಲ್ಲ
ತೋಚಿದ
ತೋಚು-ತ್ತಿಲ್ಲ
ತೋಟ
ತೋಟಕ್ಕೆ
ತೋಟ-ಗಾ-ರ-ನಿ-ರ-ಬೇಕು
ತೋಟದ
ತೋಟ-ದಲ್ಲಿ
ತೋಟ-ವನ್ನು
ತೋಡ-ತೊ-ಡ-ಗಿ-ದರು
ತೋಡಲು
ತೋಡು
ತೋಡುವ
ತೋತಾ-ಪುರಿ
ತೋಪಿನ
ತೋರದೆ
ತೋರ-ಲಿಲ್ಲ
ತೋರಲು
ತೋರಿ
ತೋರಿತು
ತೋರಿದ
ತೋರಿ-ದಂತೆ
ತೋರಿ-ದಳು
ತೋರಿ-ದು-ದ-ನ್ನೆಲ್ಲ
ತೋರಿದೆ
ತೋರಿ-ಸದೆ
ತೋರಿ-ಸ-ಬೇಡ
ತೋರಿ-ಸಲೇ
ತೋರಿಸಿ
ತೋರಿ-ಸಿ-ಕೊ-ಡು-ತ್ತಾನೆ
ತೋರಿ-ಸಿದ
ತೋರಿ-ಸಿ-ದನು
ತೋರಿ-ಸಿ-ದರು
ತೋರಿ-ಸಿ-ದಾಗ
ತೋರಿ-ಸಿದೆ
ತೋರಿಸು
ತೋರಿ-ಸುತ್ತಾ
ತೋರಿ-ಸು-ತ್ತಾನೆ
ತೋರಿ-ಸು-ತ್ತಿದ್ದ
ತೋರಿ-ಸು-ತ್ತೇನೆ
ತೋರಿ-ಸು-ವು-ದ-ಕ್ಕಾಗಿ
ತೋರಿ-ಸು-ವು-ದಕ್ಕೆ
ತೋರಿ-ಸು-ವುದು
ತೋರು
ತೋರು-ತ್ತಾನೆ
ತೋರು-ತ್ತಿದ್ದ
ತೋರು-ತ್ತಿ-ರು-ವ-ರಲ್ಲ
ತೋರು-ವರು
ತೋರು-ವು-ದ-ಕ್ಕಾಗಿ
ತೋರು-ವುದು
ತ್ಕಾರ-ವನ್ನು
ತ್ತದೆ
ತ್ತವೆ
ತ್ತಾನೆ
ತ್ತಾರೆ
ತ್ತಿತ್ತು
ತ್ತಿದ್ದ
ತ್ತಿದ್ದನು
ತ್ತಿದ್ದನೋ
ತ್ತಿದ್ದರು
ತ್ತಿದ್ದಳು
ತ್ತಿದ್ದಾಗ
ತ್ತಿದ್ದೀಯ
ತ್ತಿದ್ದು-ದನ್ನು
ತ್ತಿದ್ದು-ದ-ರಿಂದ
ತ್ತಿದ್ದೆ
ತ್ತಿರು-ವಾಗ
ತ್ತಿರು-ವಿರಿ
ತ್ತಿರು-ವು-ದರ
ತ್ತಿರೋ
ತ್ತಿವೆ
ತ್ತೀಯಾ
ತ್ತೀರಾ
ತ್ತೇನೆ
ತ್ತೇನೆ-ಎಂದು
ತ್ತೇವೆಯೋ
ತ್ಯಜಿ-ಸ-ಬೇ-ಕಾ-ಗಿಲ್ಲ
ತ್ಯಜಿ-ಸ-ಬೇಕು
ತ್ಯಜಿ-ಸ-ಲಾರ
ತ್ಯಜಿಸಿ
ತ್ಯಜಿ-ಸಿದ
ತ್ಯಜಿ-ಸಿ-ದರು
ತ್ಯಜಿ-ಸಿ-ದರೂ
ತ್ಯಜಿ-ಸಿ-ದರೆ
ತ್ಯಜಿ-ಸಿ-ರು-ವ-ವನು
ತ್ಯಜಿ-ಸು-ತ್ತಾನೆ
ತ್ಯಜಿ-ಸು-ವನೊ
ತ್ಯಜಿ-ಸು-ವನೋ
ತ್ಯಜಿ-ಸು-ವೆಯೊ
ತ್ಯಾಗ
ತ್ಯಾಗಕ್ಕೆ
ತ್ಯಾಗ-ದಲ್ಲಿ
ತ್ಯಾಗವೇ
ತ್ಯಾಗಿ
ತ್ಯಾಗಿ-ಗ-ಳಾಗಿ
ತ್ರಾಣ
ತ್ರಿಶೂ-ಲ-ದಿಂದ
ತ್ರೇತಾ-ಯು-ಗ-ದಲ್ಲಿ
ತ್ವರಿ-ತ-ದಿಂದ
ದಂಗೆ
ದಂಡಕ್ಕೆ
ದಂಡ-ವನ್ನು
ದಂಡಿ-ಸು-ವು-ದಕ್ಕೆ
ದಂಡು
ದಂಡೆಯ
ದಂಡೆ-ಯನ್ನು
ದಂತೆ
ದಕ್ಕೆ
ದಕ್ಷಿ-ಣಕ್ಕೆ
ದಕ್ಷಿ-ಣ-ದಿ-ಕ್ಕಿಗೆ
ದಕ್ಷಿ-ಣ-ಭಾ-ಗ-ದಲ್ಲಿ
ದಕ್ಷಿ-ಣೆ-ಯನ್ನು
ದಕ್ಷಿ-ಣೇ-ಶ್ವರ
ದಕ್ಷಿ-ಣೇ-ಶ್ವ-ರಕ್ಕೆ
ದಟ್ಟ-ವಾಗಿ
ದಡ
ದಡದ
ದಡ-ವನ್ನು
ದಡ್ಡ
ದಣಿಯ
ದಣಿ-ವಾ-ಯಿತು
ದನ
ದನ-ಕಾ-ಯುವ
ದನ-ಕಾ-ಯು-ವ-ವ-ರನ್ನು
ದನ-ಕಾ-ಯು-ವ-ವರು
ದನ-ದಂತೆ
ದನವು
ದನು
ದನ್ನು
ದಯ-ವಿಟ್ಟು
ದಯಾ-ಮಯ
ದಯೆ
ದಯೆ-ಯಿಂದ
ದರನ
ದರ-ಲ್ಲಿದ್ದ
ದರಿ
ದರು
ದರೂ
ದರೆ
ದರೋಡೆ
ದರೋ-ಡೆ-ಕೋರ
ದರೋ-ಡೆ-ಕೋ-ರ-ನ-ಲ್ಲಿ-ರು-ವಂ-ತಹ
ದರೋ-ಡೆ-ಕೋ-ರರ
ದರೋ-ಡೆ-ಕೋ-ರರು
ದರೋ-ಡೆ-ಕೋ-ರರೇ
ದರ್ಜೆಗೆ
ದರ್ಶನ
ದರ್ಶ-ನ-ದಿಂದ
ದರ್ಶ-ನ-ವನ್ನು
ದರ್ಶ-ನ-ವಾ-ಗಿತ್ತು
ದರ್ಶ-ನ-ವಾ-ಗು-ವಂತೆ
ದಲ್ಲದೆ
ದಲ್ಲಿ
ದಲ್ಲಿ-ರುವ
ದಲ್ಲೆ
ದಲ್ಲೇ
ದಳು
ದವ-ಸ-ಧಾ-ನ್ಯ-ಗಳನ್ನು
ದವು
ದವು-ಗ-ಳಿ-ಗಾ-ಗಿಯೋ
ದಶ-ಭು-ಜ-ಳಾದ
ದಶ-ರಥ
ದಸ್ತ-ಗಿರಿ
ದಾಗ
ದಾಟ
ದಾಟ-ಲಾ-ರೆಯಾ
ದಾಟಿ
ದಾಟು-ತ್ತಾರೆ
ದಾಟು-ತ್ತಿ-ದ್ದರು
ದಾಟು-ತ್ತಿ-ರುವೆ
ದಾಟುವ
ದಾಟು-ವು-ದ-ರ-ಲ್ಲಿದ್ದ
ದಾಟು-ವು-ದ-ರ-ಲ್ಲಿ-ದ್ದರು
ದಾಟು-ವುದು
ದಾತ
ದಾನ
ದಾನಿ-ಗ-ಳಾ-ಗಿ-ದ್ದೀರಿ
ದಾನಿಗೂ
ದಾಯಾದಿ
ದಾಯಿತು
ದಾರ
ದಾರ-ಗಳನ್ನು
ದಾರದ
ದಾರ-ದಲ್ಲಿ
ದಾರ-ವನ್ನು
ದಾರಿ
ದಾರಿಯ
ದಾರಿ-ಯನ್ನು
ದಾರಿ-ಯ-ನ್ನೇನೊ
ದಾರಿ-ಯಲ್ಲಿ
ದಾರು-ಣ-ವಾದ
ದಾಸ
ದಾಸ-ನಾ-ಗಿಲ್ಲ
ದಾಸಿ
ದಿಂದ
ದಿಂದಲೂ
ದಿಂಬು
ದಿಕ್ಕು-ಗ-ಳಿಗೆ
ದಿದೆಯೆ
ದಿನ
ದಿನ-ಗಳ
ದಿನ-ಗಳನ್ನು
ದಿನ-ಗ-ಳಾದ
ದಿನ-ಗ-ಳಾ-ದರೂ
ದಿನ-ಗ-ಳಾ-ದವು
ದಿನ-ಗಳಿಂದ
ದಿನ-ಗಳು
ದಿನದ
ದಿನ-ದಿಂದ
ದಿನ-ಪ-ತ್ರಿ-ಕೆ-ಯಲ್ಲಿ
ದಿನ-ವಾ-ಯಿತು
ದಿನವೂ
ದಿನ-ವೆಲ್ಲ
ದಿನ-ವೆಲ್ಲಾ
ದಿನವೇ
ದಿನಸಿ
ದಿಲ್ಲ
ದಿವ-ಸ-ಗ-ಳಾ-ದ-ಮೆಲೆ
ದಿವ್ಯ-ದ-ರ್ಶ-ನ-ದಲ್ಲಿ
ದಿವ್ಯ-ಪು-ತ್ರಿಯ
ದಿವ್ಯ-ಭಾ-ವ-ವನ್ನು
ದಿವ್ಯ-ಸ್ತೋತ್ರ
ದಿವ್ಯಾ-ನಂ-ದದ
ದೀನ
ದೀನ-ನಾದ
ದೀನನೂ
ದೀಪ
ದೀರ್ಘ
ದೀರ್ಘ-ಕಾಲ
ದುಂಬಿ
ದುಃಖ
ದುಃಖಕ್ಕೂ
ದುಃಖಕ್ಕೆ
ದುಃಖ-ಗಳಲ್ಲಿ
ದುಃಖ-ಗ-ಳಿ-ಗೆಲ್ಲ
ದುಃಖ-ದಿಂದ
ದುಃಖ-ಪ-ಟ್ಟರು
ದುಃಖಿಯೂ
ದುಃಖಿಸಿ
ದುಃಸ್ಥಿ-ತಿ-ಯನ್ನು
ದುಕೊ
ದುಡುಕು
ದುದ-ರಿಂದ
ದುರ
ದುರಂತ
ದುರಂ-ತಕ್ಕೆ
ದುರ-ದೃಷ್ಟ
ದುರ-ದೃ-ಷ್ಟ-ದಿಂದ
ದುರ-ದೃ-ಷ್ಟ-ವ-ಶಾತ್
ದುರ-ವಸ್ಥೆ
ದುರ-ವ-ಸ್ಥೆ-ಯ-ಲ್ಲಿ-ದ್ದು-ದನ್ನು
ದುರ-ಹಂ-ಕಾರ
ದುರ-ಹಂ-ಕಾ-ರದ
ದುರ-ಹಂ-ಕಾ-ರ-ದಿಂದ
ದುರಾ-ತ್ಮ-ನಾದ
ದುರು-ಳನು
ದುರ್ಗತಿ
ದುರ್ಗಾ-ಪೂಜೆ
ದುರ್ಗಾ-ಪೂ-ಜೆ-ಯನ್ನು
ದುರ್ಗೆಗೆ
ದುರ್ಜ-ನರು
ದುರ್ಬ-ಲವಾ
ದುರ್ಬ-ಲ-ವಾ-ಗಿ-ರು-ವುದ
ದುರ್ಯೋ-ಧನ
ದುರ್ವಾ-ಸನೆ
ದುರ್ವಾ-ಸ-ನೆ-ಯನ್ನು
ದುಷ್ಟ-ನನ್ನು
ದುಷ್ಟನೇ
ದುಷ್ಟರ
ದುಷ್ಟ-ರನ್ನು
ದುಷ್ಟ-ಸ್ವ-ಭಾ-ವ-ವನ್ನು
ದೂತ-ನಿಗೆ
ದೂತರು
ದೂರ
ದೂರದ
ದೂರ-ದಲ್ಲಿ
ದೂರ-ದ-ಲ್ಲಿ-ರುವ
ದೂರ-ದ-ಲ್ಲಿ-ರು-ವನು
ದೂರ-ದಲ್ಲೆ
ದೂರ-ದಿಂದ
ದೂರ-ಲಿಲ್ಲ
ದೂರ-ವಾ-ಗುತ್ತಾ
ದೂರ-ವಿದೆ
ದೂರ-ವಿ-ರ-ಬೇಕು
ದೂರ-ವಿ-ರು-ವುದೇ
ದೂರ-ಹೋಗು
ದೂರು
ದೂರ್ವಾ-ಸರ
ದೃಢ
ದೃಢ-ವಾ-ದಂ-ತೆಲ್ಲ
ದೃಶ್ಯ
ದೃಶ್ಯ-ವನ್ನು
ದೃಷ್ಟಾಂತ
ದೃಷ್ಟಿ
ದೃಷ್ಟಿ-ಯಲ್ಲಿ
ದೃಷ್ಟಿ-ಯಿಂದ
ದೃಷ್ಟಿ-ಯೆಲ್ಲ
ದೆನು
ದೆನೆ
ದೆಲ್ಲ
ದೆವ್ವದ
ದೆವ್ವ-ವಿ-ರುವ
ದೆಸೆ
ದೆಸೆ-ಯಿಂದ
ದೆಹ-ಲಿ-ಯಲ್ಲಿ
ದೇವ
ದೇವಕಿ
ದೇವ-ತೆ-ಗ-ಳಂತೆ
ದೇವ-ತೆ-ಗಳು
ದೇವ-ತೆಯ
ದೇವ-ದೇ-ವ-ತೆ-ಗಳನ್ನು
ದೇವ-ನೊ-ಬ್ಬನೇ
ದೇವರ
ದೇವ-ರಂ-ತೆಯೇ
ದೇವ-ರ-ಕೋಣೆ
ದೇವ-ರನ್ನು
ದೇವ-ರನ್ನೇ
ದೇವ-ರ-ಮ-ನೆ-ಯಲ್ಲಿ
ದೇವ-ರಿ-ಗಾಗಿ
ದೇವ-ರಿ-ಗಿಂತ
ದೇವ-ರಿಗೆ
ದೇವ-ರಿ-ರು-ವನು
ದೇವರು
ದೇವರೂ
ದೇವರೆ
ದೇವರೇ
ದೇವ-ರೊಂ-ದಿಗೆ
ದೇವ-ರೊಂದು
ದೇವ-ರೊಬ್ಬ
ದೇವ-ರೊ-ಬ್ಬನೆ
ದೇವ-ರೊ-ಬ್ಬನೇ
ದೇವರ್ಷಿ
ದೇವ-ಸ್ಥಾನ
ದೇವ-ಸ್ಥಾ-ನಕ್ಕೆ
ದೇವ-ಸ್ಥಾ-ನ-ಗ-ಳಿವೆ
ದೇವ-ಸ್ಥಾ-ನದ
ದೇವ-ಸ್ಥಾ-ನ-ದಲ್ಲಿ
ದೇವ-ಸ್ಥಾ-ನ-ದ-ಲ್ಲಿತ್ತು
ದೇವ-ಸ್ಥಾ-ನ-ದಿಂದ
ದೇವ-ಸ್ಥಾ-ನ-ವನ್ನು
ದೇವಾ-ಲ-ಯ-ವಾ-ಗಿತ್ತು
ದೇವಾ-ಲ-ಯವೇ
ದೇವಿ
ದೇವಿಯ
ದೇವಿಯು
ದೇವೇಂ-ದ್ರನು
ದೇಶಕ್ಕೆ
ದೇಶ-ವನ್ನು
ದೇಹ
ದೇಹದ
ದೇಹ-ವನ್ನು
ದೇಹ-ವಲ್ಲ
ದೇಹ-ವಿ-ರು-ವುದೋ
ದೈವ-ಭ-ಕ್ತ-ನಾ-ದು-ದ-ರಿಂದ
ದೈವಾಂ-ಶ-ದೊ-ಡನೆ
ದೊಂಬಿಯ
ದೊಡ-ನೆಯೆ
ದೊಡ್ಡ
ದೊಡ್ಡ-ದಾ-ಗಿ-ರ-ಲಾ-ರದು
ದೊಡ್ಡ-ದಾದ
ದೊಡ್ಡದು
ದೊಡ್ಡದೆ
ದೊಡ್ಡ-ವನೆ
ದೊರ-ಕಿತು
ದೊರೆ-ಗ-ಳಿಗೆ
ದೊರೆ-ತಿದೆ
ದೊರೆ-ಯಿಂದ
ದೊರೆ-ಯು-ವುದೆ
ದೊರೆ-ಯು-ವುವು
ದೋಣಿ
ದೋಣಿ-ಗಾಗಿ
ದೋಣಿಯ
ದೋಣಿ-ಯಲ್ಲಿ
ದೋಣಿ-ಯ-ಲ್ಲಿ-ದ್ದ-ವರೆಲ್ಲ
ದೋಣಿ-ಯಿಂದ
ದೋಷ-ವಿದೆ
ದ್ದಂತೆ
ದ್ದಂತೆಯೆ
ದ್ದನು
ದ್ದನೊ
ದ್ದನೋ
ದ್ದರೆ
ದ್ದವನು
ದ್ದವಳು
ದ್ದಾಗ
ದ್ದಾನೆ
ದ್ದಿದ್ದರೆ
ದ್ದಿರಿ
ದ್ದುದೂ
ದ್ಧಾತ್ರಿ
ದ್ರವ-ರೂ-ಪ-ದ-ಲ್ಲಿ-ರುವ
ದ್ರವ್ಯ
ದ್ರೌಪದಿ
ದ್ರೌಪ-ದಿ-ಯಲ್ಲಿ
ದ್ವಾರ-ಗ-ಳ-ಲ್ಲಿಯೂ
ದ್ವೇಷಿ-ಸ-ತೊ-ಡ-ಗಿದ
ದ್ವೇಷಿ-ಸು-ತ್ತಿದ್ದ
ದ್ವೇಷಿ-ಸು-ವೆಯೊ
ಧನು-ರ್ಬಾ-ಣ-ಗಳನ್ನು
ಧನು-ಸ್ಸನ್ನು
ಧನು-ಸ್ಸ-ನ್ನೆ-ತ್ತಿ-ದಾಗ
ಧನ್ಯ-ವಾ-ದ-ಗಳು
ಧರಿ-ಸ-ಬ-ಲ್ಲದು
ಧರಿ-ಸಿ-ದರು
ಧರಿ-ಸಿ-ದಳು
ಧರಿ-ಸಿ-ದ-ವ-ನನ್ನು
ಧರಿ-ಸಿ-ದಾಗ
ಧರಿ-ಸಿ-ದ್ದಾಳೆ
ಧರಿ-ಸಿದ್ದೆ
ಧರಿ-ಸಿ-ರು-ವ-ವನು
ಧರಿ-ಸಿ-ರು-ವೆನು
ಧರಿ-ಸು-ತ್ತಾನೆ
ಧರಿ-ಸು-ವನೋ
ಧರ್ಮ
ಧರ್ಮಕ್ಕೆ
ಧರ್ಮ-ಗಳನ್ನು
ಧರ್ಮದ
ಧರ್ಮ-ವನ್ನು
ಧರ್ಮ-ವಿದೆ
ಧರ್ಮಾಂ-ಧ-ನಾ-ಗಿ-ರ-ಬೇಡಿ
ಧರ್ಮಾ-ತ್ಮ-ನಲ್ಲಿ
ಧರ್ಮಾ-ತ್ಮನೇ
ಧಾತು-ವಿ-ನಿಂದ
ಧಾಮಕ್ಕೆ
ಧಾರ್ಮಿಕ
ಧಾರ್ಮಿ-ಕತೆ
ಧಾವಿ-ಸಿ-ದರು
ಧೂಳಿ-ಸ-ಮಾನ
ಧೂಳು
ಧೈರ್ಯ
ಧೈರ್ಯ-ದಿಂದ
ಧೈರ್ಯ-ಶಾಲಿ
ಧ್ಯಾನ
ಧ್ಯಾನಕ್ಕೆ
ಧ್ಯಾನ-ದಲ್ಲಿ
ಧ್ಯಾನ-ದ-ಲ್ಲಿ-ದ್ದ-ಳಾ-ದ್ದ-ರಿಂದ
ಧ್ಯಾನ-ದ-ಲ್ಲಿಯೂ
ಧ್ಯಾನಾ-ದಿ-ಗಳನ್ನೂ
ಧ್ಯಾನಾ-ಸ-ಕ್ತ-ನಾಗಿ
ಧ್ಯಾನಿ
ಧ್ಯಾನಿಸಿ
ಧ್ಯಾನಿ-ಸಿ-ದರೆ
ಧ್ಯಾನಿ-ಸು-ತ್ತಿದ್ದ
ಧ್ಯಾನಿ-ಸು-ತ್ತಿ-ರು-ವಾಗ
ಧ್ಯಾನಿ-ಸುವ
ಧ್ವನಿ-ಯನ್ನು
ನಂಟರು
ನಂತರ
ನಂತೆ
ನಂತೆಯೇ
ನಂದನ
ನಂದಿ
ನಂದಿಯ
ನಂಬ-ಬೇ-ಕಾ-ಗು-ತ್ತದೆ
ನಂಬ-ಲಿಲ್ಲ
ನಂಬಲು
ನಂಬಿ
ನಂಬಿಕೆ
ನಂಬಿ-ಕೆ-ಯಿಂದ
ನಂಬಿ-ಕೆ-ಯಿ-ರು-ತ್ತದೆ
ನಂಬಿದ
ನಂಬಿ-ದರೆ
ನಂಬಿದ್ದ
ನಂಬಿ-ದ್ದರು
ನಂಬಿ-ರ-ಲಿಲ್ಲ
ನಂಬು
ನಂಬು-ವನು
ನಂಬು-ವ-ನು-ನಿ-ರಾ-ಕಾರ
ನಂಬು-ವು-ದಿಲ್ಲ
ನಂಬು-ವುದು
ನಕಾ-ಶೆ-ಯನ್ನು
ನಕ್ಕ
ನಕ್ಕು
ನಕ್ಷತ್ರ
ನಗ
ನಗ-ನಾ-ಣ್ಯ-ಗಳನ್ನು
ನಗ-ರಿಗೆ
ನಗುತ್ತ
ನಗು-ತ್ತಾನೆ
ನಗು-ತ್ತಿ-ರುವೆ
ನಗು-ತ್ತೀರಿ
ನಗು-ನ-ಗುತ್ತ
ನಗು-ವನು
ನಗು-ವುದು
ನಗೆ-ಪಾ-ಟ-ಲಾ-ಗು-ವುದು
ನಟಿ-ಸ-ತೊ-ಡ-ಗಿ-ದನು
ನಟಿ-ಸಿ-ದನು
ನಟಿ-ಸಿ-ದರೆ
ನಟಿಸು
ನಟಿ-ಸುತ್ತ
ನಟಿ-ಸು-ತ್ತಿ-ದ್ದರು
ನಟಿ-ಸು-ತ್ತಿ-ದ್ದ-ವನ
ನಟಿ-ಸು-ವುದು
ನಡು-ಗ-ತೊ-ಡ-ಗಿ-ದನು
ನಡು-ಗುತ್ತ
ನಡು-ಗು-ತ್ತಿ-ದ್ದರು
ನಡು-ಗು-ತ್ತಿ-ದ್ದುವು
ನಡು-ನೀ-ರಿ-ನಲ್ಲಿ
ನಡು-ರಾ-ತ್ರಿ-ಯಲ್ಲಿ
ನಡೆದ
ನಡೆ-ದಿದೆ
ನಡೆದು
ನಡೆ-ದು-ಕೊಂಡು
ನಡೆ-ದು-ದ-ರಿಂದ
ನಡೆ-ಯ-ಲಾ-ರಂ-ಭಿ-ಸಿ-ದಳು
ನಡೆ-ಯಿತು
ನಡೆ-ಯು-ತ್ತಿತ್ತು
ನಡೆ-ಯು-ತ್ತಿದ್ದ
ನಡೆ-ಯು-ತ್ತಿ-ರುವೆ
ನಡೆ-ಯು-ತ್ತಿಲ್ಲ
ನಡೆ-ಯು-ವುದು
ನಡೆ-ಸಿದ
ನಡೆ-ಸು-ತ್ತಿದ್ದ
ನಡೆ-ಸು-ತ್ತಿ-ರು-ವನು
ನಡೆ-ಸುವ
ನಡೆ-ಸು-ವು-ದಕ್ಕೆ
ನಡೆ-ಸು-ವುದು
ನದಿ
ನದಿಯ
ನದಿ-ಯನ್ನು
ನದಿ-ಯಲ್ಲಿ
ನದಿ-ಯಿಂದ
ನದೀ
ನನ-ಗಾಗಿ
ನನ-ಗಿಂತ
ನನಗೂ
ನನಗೆ
ನನಗೇ
ನನ-ಗೇ-ನಾ-ದರೂ
ನನ-ಗೇನೂ
ನನ-ಗೊಂದು
ನನ್ನ
ನನ್ನಂತೆ
ನನ್ನಂ-ತೆಯೇ
ನನ್ನದು
ನನ್ನ-ದೊಂದು
ನನ್ನನ್ನು
ನನ್ನನ್ನೂ
ನನ್ನನ್ನೇ
ನನ್ನಲ್ಲಿ
ನನ್ನ-ಲ್ಲಿದ್ದ
ನನ್ನ-ಲ್ಲಿ-ರು-ವುದು
ನನ್ನ-ವ-ರೆಂದು
ನನ್ನವು
ನನ್ನಿಂದ
ನನ್ನು
ನನ್ನೊಂ-ದಿಗೆ
ನನ್ನೊ-ಡನೆ
ನಮ
ನಮಗೆ
ನಮ-ಗೇನೋ
ನಮ-ಸ್ಕ-ರಿಸಿ
ನಮ-ಸ್ಕ-ರಿ-ಸಿ-ದನು
ನಮ-ಸ್ಕ-ರಿ-ಸಿ-ದರು
ನಮ-ಸ್ಕಾರ
ನಮಾಜು
ನಮಿಸಿ
ನಮಿ-ಸಿ-ದರು
ನಮಿ-ಸು-ವು-ದಿಲ್ಲ
ನಮ್ಮ
ನಮ್ಮನ್ನು
ನಮ್ಮಲ್ಲಿ
ನಮ್ಮ-ಲ್ಲಿ-ರುವ
ನಮ್ಮ-ಲ್ಲಿ-ರು-ವುದ
ನಮ್ಮಿಂದ
ನಮ್ರತೆ
ನಮ್ರ-ತೆ-ಯನ್ನು
ನಮ್ರ-ನಾ-ದನು
ನರಕ
ನರ-ಕಕ್ಕೆ
ನರ-ಕ-ಗಳು
ನರ-ಕ-ದಲ್ಲಿ
ನರ-ಕ-ಪಾ-ಲಕ್ಕೆ
ನರ-ಕ-ಪಾ-ಲ-ದೊ-ಳಗೆ
ನರ-ಕ-ಪಾ-ಲ-ವನ್ನು
ನರ-ಕ-ಪಾ-ಲವೂ
ನರ-ಕ-ವನ್ನು
ನರ-ಕ-ವಾದಂ
ನರ-ಕ-ವಾಸ
ನರ-ಕ-ವಾ-ಸ-ವನ್ನು
ನರ-ಕ-ವೆ-ಲ್ಲಿದೆ
ನರ-ರೂಪಿ
ನರ-ಳ-ಬೇ-ಕಾ-ಗು-ವುದು
ನರ-ಳ-ಬೇ-ಕಾ-ಯಿತು
ನರ-ಳಾಡಿ
ನರಳಿ
ನರಳು
ನರ-ಳು-ತ್ತಿ-ರು-ವ-ವನು
ನರಿ
ನರಿ-ಗಳು
ನರಿಯೂ
ನರಿ-ಯೊಂದು
ನರೆ-ಹೊ-ರೆ-ಯ-ವರು
ನರೇಂದ್ರ
ನರೇಂ-ದ್ರ-ನಿಗೆ
ನರ್ತಿ-ಸು-ತ್ತಿ-ದ್ದಳು
ನಲ್ಲಿ
ನಲ್ಲಿಯೂ
ನವ-ತ-ರು-ಣಿ-ಯರ
ನವರು
ನವಿ-ಲಿಗೆ
ನವಿ-ಲಿನ
ನವಿಲು
ನವೆಂ-ಬರ್
ನವೆಂ-ಬ-ರ್ನಲ್ಲಿ
ನಶ್ವರ
ನಶ್ವ-ರ-ತೆ-ಯನ್ನು
ನಷ್ಟ
ನಸು-ನ-ಗುತ್ತ
ನಹ-ಬ-ತ್ಖಾ-ನೆಯ
ನಾ
ನಾಗ-ರ-ಹಾ-ವಿನ
ನಾಗ-ರ-ಹಾವು
ನಾಗಿ
ನಾಗಿದ್ದ
ನಾಗಿದ್ದೆ
ನಾಗು-ತ್ತಾನೆ
ನಾಚಿಕೆ
ನಾಚಿ-ಕೆ-ಯಾ-ಗು-ವು-ದಿ-ಲ್ಲವೆ
ನಾಚಿ-ಕೆ-ಯಿಂದ
ನಾಟಕ
ನಾಟ-ಕ-ವನ್ನು
ನಾಡ-ಲಾರೆ
ನಾಡಿ-ಯನ್ನು
ನಾಡು-ತ್ತಿ-ದ್ದರು
ನಾಣ್ಯ-ಗಳು
ನಾಣ್ಯ-ಗ-ಳೆಲ್ಲ
ನಾದ
ನಾನ-ದನ್ನು
ನಾನಲ್ಲ
ನಾನಾ-ಗಲಿ
ನಾನಿ-ರುವೆ
ನಾನಿ-ಲ್ಲದೆ
ನಾನು
ನಾನೂ
ನಾನೆ
ನಾನೆಷ್ಟು
ನಾನೇ
ನಾನೇನು
ನಾನೇನೋ
ನಾನೊಂದು
ನಾಮ
ನಾಮ-ಗಳನ್ನೂ
ನಾಮ-ಗಳು
ನಾಮ-ಜಪ
ನಾಮದ
ನಾಮ-ರೂ-ಪ-ಗ-ಳಾಚೆ
ನಾಮ-ವನ್ನು
ನಾಮ-ವ-ನ್ನು-ಚ್ಚ-ರಿಸಿ
ನಾಮ-ವನ್ನೂ
ನಾಮ-ಸ್ಮ-ರಣೆ
ನಾಯಿ
ನಾಯಿಗೆ
ನಾಯಿಯ
ನಾಯಿ-ಯನ್ನು
ನಾಯಿ-ಯೊಂ-ದಿಗೆ
ನಾರದ
ನಾರ-ದರ
ನಾರ-ದ-ರನ್ನು
ನಾರ-ದ-ರಿಗೆ
ನಾರ-ದರು
ನಾರ-ದರೆ
ನಾರ-ದರೇ
ನಾರ-ದ-ರೊಂ-ದಿಗೆ
ನಾರಾ-ಯಣ
ನಾರಾ-ಯ-ಣ-ದೇವ
ನಾರಾ-ಯ-ಣ-ನಿಂದ
ನಾಲಿಗೆ
ನಾಲಿ-ಗೆಯು
ನಾಲ್ಕ-ನೆ-ಯ-ವನು
ನಾಲ್ಕಾಣೆ
ನಾಲ್ಕು
ನಾಲ್ಕೈದು
ನಾಳೆ
ನಾಳೆಗೆ
ನಾಳೆಯೇ
ನಾವಿ-ಬ್ಬರೂ
ನಾವೀಗ
ನಾವು
ನಾವು-ಗ-ಳೆಲ್ಲ
ನಾವೊಬ್ಬ
ನಾಶ
ನಾಶ-ಮಾ-ಡಲು
ನಾಶ-ಮಾ-ಡು-ತ್ತದೆ
ನಾಶ-ಮಾ-ಡುವೆ
ನಾಶ-ವಾ-ಗು-ತ್ತಾನೆ
ನಾಶ-ವಾ-ಗುವ
ನಾಶ-ವಾ-ಗು-ವು-ದಿಲ್ಲ
ನಾಶ-ವಾ-ಗು-ವುದು
ನಾಶ-ವಾ-ಗು-ವುವು
ನಾಶ-ವಾ-ದೇವು
ನಾಶ-ವಾ-ಯಿತು
ನಾಸ್ತಿ-ಕ-ರಾ-ಗು-ತ್ತಾರೆ
ನಿಂತ
ನಿಂತನು
ನಿಂತಾಗ
ನಿಂತಿತು
ನಿಂತಿ-ರುವು
ನಿಂತು
ನಿಂತು-ಕೊಂಡು
ನಿಂತು-ಹೋ-ಗಿತ್ತು
ನಿಂತು-ಹೋ-ಗಿದ್ದು
ನಿಂತು-ಹೋ-ಗು-ವು-ದಲ್ಲ
ನಿಂತು-ಹೋ-ಗು-ವುದು
ನಿಂತು-ಹೋ-ದ-ದ್ದ-ರಿಂದ
ನಿಂತು-ಹೋ-ಯಿತು
ನಿಂದ
ನಿಂದಿ-ಸಿ-ಕೊ-ಳ್ಳುತ್ತ
ನಿಂದಿ-ಸು-ತ್ತಿ-ರು-ವನು
ನಿಃಸ್ವಾ-ರ್ಥ-ವಾದ
ನಿಕಟ
ನಿಕ-ಟ-ವಾಗಿ
ನಿಕಷೆ
ನಿಗಿಂತ
ನಿಗೆ
ನಿಗ್ರ-ಹ-ಮಾಡಿ
ನಿಜ
ನಿಜ-ವಲ್ಲ
ನಿಜ-ವಾಗಿ
ನಿಜ-ವಾ-ಗಿಯೂ
ನಿಜ-ವಾದ
ನಿಜವೋ
ನಿಜ-ಸ್ವ-ರೂ-ಪ-ವನ್ನು
ನಿಟ್ಟು-ಸಿರು
ನಿತಾ-ಯಿಯ
ನಿತೈ
ನಿತ್ತು
ನಿತ್ಯ
ನಿತ್ಯ-ಮು-ಕ್ತರ
ನಿತ್ಯ-ಮು-ಕ್ತ-ರಿ-ದ್ದಂತೆ
ನಿತ್ಯ-ಮು-ಕ್ತರು
ನಿತ್ಯ-ಸ್ಥಾ-ನಂದ
ನಿತ್ಯಾ-ನಂದ
ನಿತ್ಯಾ-ನಂ-ದರು
ನಿದ-ರ್ಶನ
ನಿದ್ರಾ-ಭಂ-ಗ-ಮಾಡು
ನಿದ್ರೆ
ನಿದ್ರೆ-ಗೆ-ಡು-ವ-ನೇನು
ನಿಧಾ-ನ-ವಾಗಿ
ನಿನ
ನಿನ-ಗಿಂತ
ನಿನಗೂ
ನಿನಗೆ
ನಿನ-ಗೆಷ್ಟು
ನಿನಗೇ
ನಿನ-ಗೇ-ನಾ-ಗಿದೆ
ನಿನ-ಗೇ-ನಾ-ದರೂ
ನಿನ-ಗೇನು
ನಿನ-ಗೇನೊ
ನಿನ-ಗೊಂದು
ನಿನ್ನ
ನಿನ್ನದು
ನಿನ್ನ-ನಾ-ಮ-ವನ್ನು
ನಿನ್ನನ್ನು
ನಿನ್ನಲ್ಲಿ
ನಿನ್ನ-ಲ್ಲಿಯೇ
ನಿನ್ನ-ಲ್ಲಿ-ರುವ
ನಿನ್ನೆ
ನಿಮ-ಗಿಂತ
ನಿಮಗೆ
ನಿಮಗೇ
ನಿಮಿತ್ತ
ನಿಮಿ-ಷದ
ನಿಮ್ಮ
ನಿಮ್ಮದು
ನಿಮ್ಮನ್ನು
ನಿಮ್ಮಪ್ಪ
ನಿಮ್ಮಲ್ಲಿ
ನಿಮ್ಮ-ವಲ್ಲ
ನಿಮ್ಮೊಂ-ದಿಗೆ
ನಿಮ್ಮೊಂ-ದಿಗೇ
ನಿಯ-ಮಕ್ಕೆ
ನಿಯ-ಮ-ಗಳು
ನಿಯ-ಮ-ದಿಂದ
ನಿಯ-ಮ-ವನ್ನು
ನಿರಂ-ತರ
ನಿರ-ತ-ನಾ-ಗಿದ್ದ
ನಿರ-ತ-ನಾ-ಗಿ-ದ್ದು-ದ-ರಿಂದ
ನಿರ-ತ-ನಾ-ಗಿ-ರು-ತ್ತಿದ್ದ
ನಿರ-ತ-ನಾ-ಗಿ-ರು-ವನು
ನಿರ-ತ-ನಾದ
ನಿರ-ತ-ವಾ-ಗಿತ್ತು
ನಿರಾ-ಕಾರ
ನಿರಾ-ಕಾ-ರ-ನಾ-ಗಿ-ದ್ದರೆ
ನಿರಾ-ಕಾ-ರ-ನಾ-ಗಿ-ಬಿ-ಡು-ವನು
ನಿರಾ-ಕಾ-ರ-ವಾ-ದರೂ
ನಿರಾ-ಶೆ-ಯಿಂದ
ನಿರಾ-ಸೆ-ಯಿಂದ
ನಿರು-ವನು
ನಿರ್ಜೀ-ವ-ವಾದ
ನಿರ್ಣ-ಯಕ್ಕೂ
ನಿರ್ಣ-ಯಕ್ಕೆ
ನಿರ್ಣ-ಯ-ವನ್ನು
ನಿರ್ಣ-ಯಿ-ಸು-ತ್ತಿ-ದ್ದರು
ನಿರ್ಣ-ಯಿ-ಸು-ವು-ದಕ್ಕೆ
ನಿರ್ದ-ಯ-ದಿಂದ
ನಿರ್ದಯಿ
ನಿರ್ದ-ಯಿ-ಯಾದ
ನಿರ್ಧ-ರಿ-ಸಲಿ
ನಿರ್ಧ-ರಿ-ಸಿದ
ನಿರ್ಧ-ರಿ-ಸಿ-ದರು
ನಿರ್ಧಾರ
ನಿರ್ಮಲ
ನಿರ್ಮಾ-ಣ-ವಾ-ಗು-ತ್ತದೆ
ನಿರ್ಮಿ-ತ-ವಾ-ಯಿತು
ನಿರ್ವಾ-ಹ-ವಿ-ಲ್ಲದೆ
ನಿಲ್ಲ-ಬ-ಹುದು
ನಿಲ್ಲಲಿ
ನಿಲ್ಲಿಸ
ನಿಲ್ಲಿ-ಸಿದ
ನಿಲ್ಲಿ-ಸಿ-ದ್ದಾರೆ
ನಿಲ್ಲಿ-ಸು-ತ್ತದೆ
ನಿಲ್ಲಿ-ಸು-ವುದು
ನಿಲ್ಲು
ನಿಲ್ಲು-ವನು
ನಿಲ್ಲು-ವು-ದಿಲ್ಲ
ನಿಲ್ಲು-ವುದೊ
ನಿವಾ-ರ-ಣೆ-ಯಾ-ಗು-ವುವು
ನಿಶ್ಚ-ಯ-ವಾಗಿ
ನಿಶ್ಚ-ಯಿ-ಸಿದ
ನಿಶ್ಚಿತ
ನಿಶ್ಚಿ-ತ-ವಾಗಿ
ನಿಷೇ-ಧ-ವಿದೆ
ನಿಷ್ಠಾ-ವಂತ
ನಿಸ್ಸಂ-ದೇ-ಹ-ವಾಗಿ
ನೀಚ
ನೀಚ-ನನ್ನು
ನೀಡಲು
ನೀಡಿ
ನೀಡಿದ
ನೀಡಿ-ದನು
ನೀಡು
ನೀಡುವ
ನೀಡೆಂದು
ನೀತಿ-ರ-ತ್ನ-ಗಳು
ನೀನ-ರ-ಸು-ವುದು
ನೀನಿ-ದನ್ನು
ನೀನು
ನೀನೂ
ನೀನೆ
ನೀನೆ-ಲ್ಲಿ-ರುವೆ
ನೀನೇ
ನೀನೇಕೆ
ನೀನೊಬ್ಬ
ನೀನೊ-ಬ್ಬನೇ
ನೀರನ್ನು
ನೀರಸ
ನೀರ-ಸ-ವಲ್ಲ
ನೀರಿ-ಗಾಗಿ
ನೀರಿನ
ನೀರಿ-ನಲ್ಲಿ
ನೀರಿ-ನ-ಲ್ಲಿದ್ದ
ನೀರಿ-ನಲ್ಲೇ
ನೀರಿ-ನಿಂದ
ನೀರಿ-ನೊ-ಳಗೆ
ನೀರಿ-ಲ್ಲದೆ
ನೀರು
ನೀರು-ಕ-ಟ್ಟಿ-ಕೊಂಡು
ನೀರೂ
ನೀರೆಲ್ಲ
ನೀರೆಲ್ಲಿ
ನೀರೊಂದೆ
ನೀಲಿ
ನೀವು
ನೀವು-ತ್ತಿ-ದ್ದಳು
ನೀವೂ
ನೀವೆ
ನೀವೆಲ್ಲ
ನೀವೇ
ನೀವೊಂದು
ನುಂಗಿ-ಬಿಟ್ಟ
ನುಂಗು
ನುಗ್ಗಲು
ನುಗ್ಗಿ
ನುಡಿದ
ನುಡಿ-ದನು
ನುಡಿ-ನ-ಮ-ನ-ಕು-ವೆಂಪು
ನುಡಿ-ಯಿತು
ನುರಿತ
ನೂಕು
ನೂಕು-ತ್ತಲೇ
ನೂಕು-ತ್ತಾನೆ
ನೂಕು-ತ್ತಿ-ರ-ಬೇಕು
ನೂರಕ್ಕೆ
ನೂರಾರು
ನೂರು
ನೂಲಿನ
ನೆಂಟ-ರಿ-ಷ್ಟ-ರಿಗೆ
ನೆಂಟ-ರಿ-ಷ್ಟ-ರೆ-ಲ್ಲರೂ
ನೆಂಟರು
ನೆಂಬುದೇ
ನೆಗೆ-ತ-ದಲ್ಲಿ
ನೆಗೆ-ದಾಡಿ
ನೆಗೆ-ಯಿತು
ನೆಚ್ಚದೆ
ನೆಚ್ಚ-ಬೇಡ
ನೆಟ್ಟ-ವನು
ನೆಟ್ಟಿ-ರುವೆ
ನೆನ-ಪಿಗೆ
ನೆನ-ಪಿ-ನ-ಲ್ಲಿ-ಡ-ಬೇಕು
ನೆನ-ಪಿವೆ
ನೆಪ
ನೆಪ-ಮಾ-ಡಿ-ಕೊಂಡು
ನೆಯ-ದನ್ನು
ನೆರೆ
ನೆರೆ-ದರು
ನೆರೆ-ಮನೆ
ನೆರೆ-ಮ-ನೆ-ಯ-ವ-ನನ್ನು
ನೆರೆ-ಮ-ನೆ-ಯ-ವರು
ನೆರೆ-ಯ-ವನು
ನೆಲದ
ನೆಲ-ದಲ್ಲಿ
ನೆಲ-ವನ್ನು
ನೆಲ-ವೆಲ್ಲ
ನೆಲೆ
ನೆಲೆಯೇ
ನೆಲೆ-ಸು-ವುದು
ನೇಗಿ-ಲಿಗೆ
ನೇಗಿ-ಲಿ-ನಿಂದ
ನೇಗಿ-ಲು-ಗಳಲ್ಲಿ
ನೇಗಿ-ಲು-ಗಳು
ನೇಡರು
ನೇಡಿ-ಯ-ರನ್ನು
ನೇಡಿ-ಯರು
ನೇತಾಡು
ನೇತಾ-ಡು-ತ್ತಿದ್ದ
ನೇತಾ-ರ-ನೊ-ಬ್ಬ-ನಾದ
ನೇತಿ
ನೇಮಕ
ನೇಯಿ-ಗೆ-ಯ-ವಳು
ನೇಯ್ಗೆ
ನೇಯ್ಗೆ-ಯ-ವನ
ನೇಯ್ಗೆ-ಯ-ವ-ನನ್ನು
ನೇಯ್ಗೆ-ಯ-ವ-ನಿಗೆ
ನೇಯ್ಗೆ-ಯ-ವ-ನಿದ್ದ
ನೇಯ್ಗೆ-ಯ-ವನು
ನೇಯ್ಗೆ-ಯ-ವಳು
ನೇಯ್ದ
ನೇರ-ಮಾ-ಡಲು
ನೇರ-ಮಾಡು
ನೇರಳೆ
ನೇರ-ವಾಗಿ
ನೈಜ
ನೈವೇ-ದ್ಯ-ದೊಂ-ದಿಗೆ
ನೈವೇ-ದ್ಯ-ವನ್ನು
ನೊಣ
ನೋಟಿನ
ನೋಟೀ-ಸನ್ನು
ನೋಟು-ಗಳನ್ನು
ನೋಡ
ನೋಡದೆ
ನೋಡ-ಬಲ್ಲೆ
ನೋಡ-ಬೇ-ಕಾ-ದರೆ
ನೋಡ-ಬೇ-ಕಾ-ದು-ದ-ನ್ನೆಲ್ಲ
ನೋಡ-ಬೇಕು
ನೋಡ-ಬೇ-ಕೆಂದು
ನೋಡ-ಬೇ-ಕೆಂ-ಬಾಸೆ
ನೋಡ-ಬೇಡ
ನೋಡ-ಬೇಡಿ
ನೋಡಯ್ಯ
ನೋಡ-ಲಾ-ಗದೆ
ನೋಡ-ಲಾ-ಗ-ಲಿಲ್ಲ
ನೋಡ-ಲಿ-ಚ್ಛಿ-ಸಿ-ದನು
ನೋಡ-ಲಿಲ್ಲ
ನೋಡಲು
ನೋಡಲೂ
ನೋಡಲ್ಲಿ
ನೋಡಿ
ನೋಡಿ-ಕೊಳ್ಳ
ನೋಡಿ-ಕೊ-ಳ್ಳಲಾ
ನೋಡಿ-ಕೊ-ಳ್ಳ-ಲೇ-ಬೇಕು
ನೋಡಿ-ಕೊಳ್ಳು
ನೋಡಿ-ಕೊ-ಳ್ಳು-ತ್ತಾನೆ
ನೋಡಿ-ಕೊ-ಳ್ಳು-ತ್ತಾರೆ
ನೋಡಿ-ಕೊ-ಳ್ಳು-ತ್ತಿದ್ದ
ನೋಡಿ-ಕೊ-ಳ್ಳು-ತ್ತಿ-ದ್ದಳು
ನೋಡಿ-ಕೊ-ಳ್ಳು-ವಂತೆ
ನೋಡಿ-ಕೊ-ಳ್ಳು-ವರು
ನೋಡಿ-ಕೊ-ಳ್ಳು-ವಳು
ನೋಡಿ-ಕೊ-ಳ್ಳು-ವ-ವ-ರು-ಯಾರೂ
ನೋಡಿ-ಕೊ-ಳ್ಳು-ವ-ವಳು
ನೋಡಿ-ಕೊ-ಳ್ಳು-ವು-ದ-ಕ್ಕಾಗಿ
ನೋಡಿ-ಕೊ-ಳ್ಳು-ವು-ದಕ್ಕೆ
ನೋಡಿತು
ನೋಡಿದ
ನೋಡಿ-ದನು
ನೋಡಿ-ದನೊ
ನೋಡಿ-ದರು
ನೋಡಿ-ದರೆ
ನೋಡಿ-ದಳು
ನೋಡಿ-ದಳೊ
ನೋಡಿ-ದ-ವನು
ನೋಡಿ-ದಾಗ
ನೋಡಿದೆ
ನೋಡಿ-ದೆನು
ನೋಡಿ-ದೆಯಾ
ನೋಡಿ-ದೆವು
ನೋಡಿದ್ದು
ನೋಡಿ-ದ್ದೇನೆ
ನೋಡಿ-ರ-ಲಿಲ್ಲ
ನೋಡಿ-ರು-ವೆನು
ನೋಡು
ನೋಡು-ತ್ತಾನೆ
ನೋಡು-ತ್ತಿದ್ದ
ನೋಡು-ತ್ತಿ-ದ್ದರು
ನೋಡು-ತ್ತಿ-ದ್ದಳು
ನೋಡು-ತ್ತಿ-ದ್ದುವು
ನೋಡು-ತ್ತಿದ್ದೆ
ನೋಡು-ತ್ತಿ-ರು-ವಾಗ
ನೋಡು-ತ್ತಿ-ರು-ವು-ದೆಲ್ಲ
ನೋಡು-ತ್ತಿ-ರುವೆ
ನೋಡು-ತ್ತೇನೆ
ನೋಡುವ
ನೋಡು-ವನೊ
ನೋಡು-ವ-ವ-ರೆಗೆ
ನೋಡುವು
ನೋಡು-ವು-ದಕ್ಕೆ
ನೋಡು-ವು-ದಿಲ್ಲ
ನೋಡು-ವುದು
ನೋಡುವೆ
ನೋಡೋಣ
ನೋವಾ-ಗಿ-ದೆಯೆ
ನೋವಿ-ನಿಂದ
ನೋವುಂ-ಟಾ-ಗ-ದಂತೆ
ನ್ನಾಥನ
ನ್ನೇರಿ
ನ್ಯಾಯ
ನ್ಯಾಯಾ-ಧಿ-ಪತಿ
ನ್ಯಾಯಾ-ಧಿ-ಪ-ತಿ-ಗಳು
ನ್ಯಾಯಾ-ಧಿ-ಪ-ತಿ-ಗಳೆ
ನ್ಯಾಯಾ-ಧಿ-ಪ-ತಿಗೆ
ನ್ಯಾಯಾ-ಧಿ-ಪ-ತಿಯ
ನ್ಯಾಸ-ವನ್ನು
ಪಂಚ-ಭೂ-ತ-ಗ-ಳಲ್ಲ
ಪಂಚ-ವ-ಟಿಯ
ಪಂಚಾಂ-ಗ-ವನ್ನು
ಪಂಚೆ
ಪಂಚೆ-ಯನ್ನು
ಪಂಜಾ-ಬಿನ
ಪಂಜಿ
ಪಂಜಿನ
ಪಂಡಿತ
ಪಂಡಿ-ತನ
ಪಂಡಿ-ತ-ನನ್ನು
ಪಂಡಿ-ತ-ನಾ-ಗ-ಲಾರ
ಪಂಡಿ-ತ-ನಾ-ಗಿ-ದ್ದರೂ
ಪಂಡಿ-ತ-ನಾ-ಗಿ-ರಲಿ
ಪಂಡಿ-ತ-ನಾ-ಗು-ತ್ತಾನೆ
ಪಂಡಿ-ತನು
ಪಂಡಿ-ತ-ನೊಬ್ಬ
ಪಂಡಿ-ತ-ನೊ-ಬ್ಬ-ನಿಂದ
ಪಂಡಿ-ತ-ನೊ-ಬ್ಬ-ನಿ-ಗಾಗಿ
ಪಂಡಿ-ತರು
ಪಂದ್ಯದ
ಪಂಪಾ
ಪಂಪಾ-ಸ-ರೋ-ವ-ರ-ದಲ್ಲಿ
ಪಕ್ಕ-ದಲ್ಲಿ
ಪಕ್ಕ-ದಲ್ಲೇ
ಪಕ್ಕ-ದ-ವನು
ಪಕ್ಕಾ
ಪಕ್ವ-ವಾಗಿ
ಪಕ್ಷಿ-ಗಳು
ಪಟ
ಪಟ-ಸ್ತಂ-ಭದ
ಪಟಿಂ-ಗ-ನನ್ನು
ಪಟ್ಟಳು
ಪಟ್ಟಿ-ರ-ಬೇಕು
ಪಟ್ಟಿ-ರುವೆ
ಪಡ-ಬೇ-ಕಾ-ಗಿದೆ
ಪಡ-ಬೇ-ಕಾ-ಯಿತು
ಪಡ-ಸಾ-ಲೆ-ಯಲ್ಲಿ
ಪಡಿ-ಸಲು
ಪಡಿ-ಸು-ತ್ತಾನೆ
ಪಡು-ವರು
ಪಡುವು
ಪಡೆ
ಪಡೆದ
ಪಡೆ-ದನು
ಪಡೆ-ದರೂ
ಪಡೆ-ದಳು
ಪಡೆ-ದಿದ್ದ
ಪಡೆ-ದಿ-ರು-ವರು
ಪಡೆ-ದಿ-ರು-ವರೋ
ಪಡೆದು
ಪಡೆ-ದು-ಕೊ-ಳ್ಳದೆ
ಪಡೆ-ದು-ಕೊ-ಳ್ಳ-ಬ-ಹುದು
ಪಡೆಯ
ಪಡೆ-ಯದೆ
ಪಡೆ-ಯ-ಬ-ಹುದು
ಪಡೆ-ಯ-ಲಾರ
ಪಡೆ-ಯಲು
ಪಡೆ-ಯು-ತಿ-ರುವೆ
ಪಡೆ-ಯು-ತ್ತದೆ
ಪಡೆ-ಯು-ತ್ತೇನೆ
ಪಡೆ-ಯುವ
ಪಡೆ-ಯು-ವರು
ಪಡೆ-ಯು-ವ-ವ-ರೆಗೂ
ಪಡೆ-ಯು-ವು-ದಕ್ಕೆ
ಪಡೆ-ಯುವೆ
ಪತಂ-ಗ-ಗಳು
ಪತನ
ಪತಿ-ತ-ರಾ-ದರು
ಪತಿ-ವ್ರ-ತ-ಳಾ-ಗಿದ್ದು
ಪತಿ-ವ್ರತೆ
ಪತಿ-ವ್ರ-ತೆ-ಯನ್ನು
ಪತ್ನಿ
ಪತ್ನಿಗೆ
ಪತ್ನಿ-ಯೊಂ-ದಿ-ಗಿ-ರ-ಬಾ-ರದು
ಪಥ್ಯದ
ಪದ-ಕ್ಕೇ-ನಾ-ದರೂ
ಪದ-ವಿ-ಯನ್ನು
ಪದೇ-ಪದೇ
ಪದ್ಮ
ಪದ್ಮ-ಲೋ-ಚನ
ಪದ್ಮ-ಲೋ-ಚ-ನ-ನೆಂ-ಬು-ವನು
ಪರಂ
ಪರ-ಚಿದ
ಪರ-ಚಿ-ದಂತೆ
ಪರ-ಚಿದೆ
ಪರ-ಚಿದ್ದು
ಪರ-ಬ್ರ-ಹ್ಮ-ನಿಂದ
ಪರ-ಬ್ರ-ಹ್ಮ-ಸ್ವ-ರೂ-ಪಿ-ಯಾದ
ಪರಮ
ಪರ-ಮ-ಜ್ಞಾ-ನದ
ಪರ-ಮ-ಜ್ಞಾ-ನ-ವನ್ನು
ಪರ-ಮ-ಜ್ಞಾನಿ
ಪರ-ಮ-ಜ್ಞಾ-ನಿಯೂ
ಪರ-ಮ-ನಿಷ್ಠೆ
ಪರ-ಮ-ಭಕ್ತ
ಪರ-ಮ-ಭಕ್ತಿ
ಪರ-ಮ-ಭ-ಕ್ತಿ-ಯನ್ನು
ಪರ-ಮ-ಹಂಸ
ಪರ-ಮ-ಹಂ-ಸ-ಕು-ವೆಂಪು
ಪರ-ಮ-ಹಂ-ಸನ
ಪರ-ಮಾ-ನಂ-ದ-ವಾ-ಗಿದೆ
ಪರ-ವಾ-ಗಿಲ್ಲ
ಪರ-ಸ್ಥ-ಳ-ದ-ವ-ನನ್ನು
ಪರ-ಸ್ಥ-ಳ-ದ-ವ-ನಿಗೆ
ಪರ-ಹಿ-ತ-ಕ್ಕಾಗಿ
ಪರಾ
ಪರಿ
ಪರಿ-ಚಯ
ಪರಿ-ಚ-ಯ-ವನ್ನೂ
ಪರಿ-ಚ-ಯ-ವಿದೆ
ಪರಿ-ಚ-ಯ-ವಿ-ರುವ
ಪರಿ-ಚ-ಯ-ವಿ-ರು-ವು-ದಿಲ್ಲ
ಪರಿ-ಚ-ಯವೇ
ಪರಿ-ಚಾ-ರಿ-ಕೆಗೂ
ಪರಿ-ಚಾ-ರಿ-ಕೆಯೇ
ಪರಿ-ಚಿ-ತ-ವಾದ
ಪರಿ-ಜ್ಞಾ-ನ-ವಿ-ಲ್ಲವೆ
ಪರಿ-ಜ್ಞಾ-ನವೇ
ಪರಿ-ಣ-ಮಿ-ಸು-ವುವು
ಪರಿ-ಣಾಮ
ಪರಿ-ಣಾ-ಮ-ವಾಗಿ
ಪರಿ-ತಾ-ಪ-ವನ್ನು
ಪರಿ-ತ್ಯಾಗಿ
ಪರಿ-ಪ-ರಿಯ
ಪರಿ-ಪಾ-ಲಿ-ಸು-ವೆವು
ಪರಿ-ಪೂ-ರ್ಣ-ತೆ-ಯನ್ನು
ಪರಿ-ಮ-ಳ-ದಿಂದ
ಪರಿ-ಮ-ಳ-ವನ್ನು
ಪರಿ-ವ-ರ್ತನೆ
ಪರಿ-ವಾ-ರ-ದ-ವರು
ಪರಿ-ಶುದ್ಧ
ಪರಿ-ಶು-ದ್ಧ-ವಾ-ಗಿತ್ತು
ಪರಿ-ಸ್ಥಿತಿ
ಪರಿ-ಸ್ಥಿ-ತಿ-ಯನ್ನು
ಪರಿ-ಸ್ಥಿ-ತಿ-ಯಿಂದ
ಪರಿ-ಹಾರ
ಪರಿ-ಹಾ-ರಕ್ಕೆ
ಪರೀ-ಕ್ಷಿ-ಸಲು
ಪರೀ-ಕ್ಷೆ-ಮಾಡಿ
ಪರ್ಣ
ಪರ್ಯ-ವ-ಸಾ-ನ-ಗೊಂಡು
ಪಲ್ಲ-ಕ್ಕಿ-ಯಲ್ಲಿ
ಪಳ-ಗಿದ
ಪವಾಡ
ಪವಾ-ಡಕ್ಕೆ
ಪವಾ-ಡ-ಗಳನ್ನು
ಪವಾ-ಡ-ವನ್ನು
ಪವಾ-ಡ-ಶ-ಕ್ತಿ-ಗಳನ್ನು
ಪವಾ-ಡ-ಶ-ಕ್ತಿ-ಗ-ಳಿ-ಗಾಗಿ
ಪವಿತ್ರ
ಪವಿ-ತ್ರ-ವಾದ
ಪಶ್ಚಾ-ತ್ತಾಪ
ಪಶ್ಚಾ-ತ್ತಾ-ಪ-ದಿಂದ
ಪಶ್ಚಾ-ತ್ತಾ-ಪ-ಪಡು
ಪಶ್ಚಿಮ
ಪಶ್ಚಿ-ಮಕ್ಕೆ
ಪಶ್ಚಿ-ಮದ
ಪಾಂಡ-ವರ
ಪಾಂಡ-ವ-ರನ್ನು
ಪಾಂಡ-ವರು
ಪಾಂಡಿ-ತ್ಯ-ವನ್ನು
ಪಾಂಡಿ-ತ್ಯ-ವಿಲ್ಲ
ಪಾಂಡಿ-ತ್ಯ-ವೆಲ್ಲ
ಪಾಡನ್ನು
ಪಾಡು
ಪಾತಂ-ಜಲ
ಪಾತ್ರ-ನಾ-ಗಿದ್ದ
ಪಾತ್ರ-ರಾ-ಗು-ವು-ದ-ಕ್ಕಾಗಿ
ಪಾತ್ರ-ವನ್ನು
ಪಾತ್ರೆ-ಗಳ
ಪಾತ್ರೆ-ಗ-ಳಿಗೆ
ಪಾತ್ರೆ-ಯನ್ನು
ಪಾದ
ಪಾದ-ಪ-ದ್ಮ-ಗ-ಳನ್ನೇ
ಪಾದ-ಪ-ದ್ಮ-ಗಳಲ್ಲಿ
ಪಾದ-ಪ-ದ್ಮ-ಗ-ಳಿಗೆ
ಪಾದ-ರಕ್ಷೆ
ಪಾದ-ರ-ಕ್ಷೆ-ಯನ್ನು
ಪಾದ-ಸ್ಪ-ರ್ಶ-ದಿಂದ
ಪಾನ-ಕದ
ಪಾನ-ಕ-ವನ್ನು
ಪಾಪ
ಪಾಪ-ಕೃ-ತ್ಯ-ಗಳನ್ನು
ಪಾಪ-ಕೃ-ತ್ಯ-ಗ-ಳಿಗೆ
ಪಾಪ-ಕೃ-ತ್ಯ-ದಿಂದ
ಪಾಪ-ಕೃ-ತ್ಯ-ವನ್ನು
ಪಾಪಕ್ಕೆ
ಪಾಪದ
ಪಾಪ-ದಿಂದ
ಪಾಪ-ಪ-ರಿ-ಹಾ-ರ-ಕ್ಕಾಗಿ
ಪಾಪ-ಭೀ-ತಿ-ಯಿಂದ
ಪಾಪ-ರಾ-ಶಿಗೆ
ಪಾಪ-ವಿದೆ
ಪಾಪ-ವೆಲ್ಲ
ಪಾಪಿಯೆ
ಪಾರಂ-ಗತ
ಪಾರಂ-ಗ-ತರು
ಪಾರ-ಮಾ-ರ್ಥಿಕ
ಪಾರ-ವಿಲ್ಲ
ಪಾರವೇ
ಪಾರಾ-ಗ-ಲಾರ
ಪಾರಾ-ಗ-ಲಾ-ರದೆ
ಪಾರಾ-ಗ-ಲಾ-ರರು
ಪಾರಾ-ಗ-ಲಾ-ರವು
ಪಾರಾ-ಗ-ಲಾರೆ
ಪಾರಾ-ಗಲು
ಪಾರಾಗಿ
ಪಾರಾ-ಗಿ-ರು-ತ್ತದೆ
ಪಾರಾ-ಗು-ತ್ತೀಯೆ
ಪಾರಾ-ಗು-ತ್ತೀರಿ
ಪಾರಾ-ಗು-ವಷ್ಟು
ಪಾರಾ-ಗು-ವು-ದಕ್ಕೆ
ಪಾರಾ-ಗು-ವುದು
ಪಾರಾ-ಗುವೆ
ಪಾರಾದ
ಪಾರಾ-ಯಣ
ಪಾರಿ-ವಾಳ
ಪಾರಿ-ವಾ-ಳ-ಗ-ಳಲ್ಲ
ಪಾರಿ-ವಾ-ಳ-ಗಳು
ಪಾರು
ಪಾರು-ಮಾ-ಡ-ಬೇಕು
ಪಾರು-ಮಾ-ಡಿತು
ಪಾರು-ಮಾ-ಡಿ-ಸು-ವ-ವ-ನು-ಎ-ನ್ನು-ವರು
ಪಾರು-ಮಾಡು
ಪಾರು-ಮಾ-ಡು-ವ-ವಳು
ಪಾರು-ಮಾ-ಡು-ವುದು
ಪಾರ್ವ
ಪಾರ್ವತಿ
ಪಾರ್ವ-ತಿಯ
ಪಾಲಾದ
ಪಾಲಾ-ದರು
ಪಾಲಿನ
ಪಾಲಿ-ಸದೆ
ಪಾಲಿ-ಸ-ಬೇ-ಕಾ-ಗಿದೆ
ಪಾಲು
ಪಾಳು
ಪಾಳು-ಬಿದ್ದ
ಪಾವಿ-ತ್ರ್ಯ-ವಿದೆ
ಪಿ
ಪಿಕಾ-ಸಿ-ಯಿಂದ
ಪಿಶಾ-ಚಿಗೆ
ಪಿಶಾ-ಚಿ-ಯಾಗಿ
ಪೀಠ
ಪೀಠದ
ಪೀಠ-ದಿಂದ
ಪೀಡಿ-ಸಿ-ದ್ದರ
ಪೀಡಿ-ಸು-ತ್ತಿ-ದ್ದವು
ಪೀಪಾ-ಯಿಗೆ
ಪೀಪಾ-ಯಿಯ
ಪೀಪಾ-ಯಿ-ಯಂ-ತಿದೆ
ಪೀಪಾ-ಯಿ-ಯಂತೆ
ಪೀಪಾ-ಯಿ-ಯನ್ನು
ಪೀಪಾ-ಯಿ-ಯಲ್ಲಿ
ಪುಕು-ರದ
ಪುಕು-ರ-ದಲ್ಲಿ
ಪುಟ್ಟ
ಪುಣಿ
ಪುಣ್ಯ
ಪುಣ್ಯ-ವನ್ನು
ಪುಣ್ಯ-ವಿದೆ
ಪುಣ್ಯಾ-ತ್ಮ-ರಿಲ್ಲ
ಪುನಃ
ಪುನ-ರ್ಜ-ನ್ಮ-ದಿಂದ
ಪುನೀ-ತ-ರಾ-ಗಿ-ದ್ದಾರೆ
ಪುರಾ-ಣ-ಗಳಿಂದ
ಪುರಾ-ಣ-ದಲ್ಲಿ
ಪುರಿ-ಯ-ಲ್ಲಿ-ರುವ
ಪುರು-ಷೋ-ತ್ತಮ
ಪುರು-ಸತ್ತು
ಪುರು-ಸತ್ತೇ
ಪುರೋ-ಹಿ-ತನ
ಪುರೋ-ಹಿ-ತ-ನನ್ನು
ಪುರೋ-ಹಿ-ತ-ರನ್ನು
ಪುರೋ-ಹಿ-ತರು
ಪುಷಿ
ಪುಷಿ-ಗಳ
ಪುಷಿ-ಗಳು
ಪುಷಿಯ
ಪುಷಿ-ಯೊಬ್ಬ
ಪುಷಿ-ವಾ-ಣಿಯ
ಪುಷ್ಕ-ಳ-ವಾಗಿ
ಪುಷ್ಪ-ಗಳನ್ನು
ಪುಷ್ಪ-ರಾಗ
ಪುಸ್ತಕ
ಪುಸ್ತ-ಕದ
ಪುಸ್ತ-ಕ-ದಲ್ಲಿ
ಪುಸ್ತ-ಕ-ವನ್ನು
ಪುಸ್ತ-ಕವು
ಪೂಜಾ-ಪಾ-ತ್ರೆ-ಗಳು
ಪೂಜಾರಿ
ಪೂಜಾ-ರಿ-ಗ-ಳಿಗೆ
ಪೂಜಾ-ರಿ-ಗಳು
ಪೂಜಾ-ರಿಗೆ
ಪೂಜಾ-ರಿ-ಯನ್ನು
ಪೂಜಾ-ರಿಯು
ಪೂಜಿ-ಸ-ಬಾ-ರದು
ಪೂಜಿ-ಸಲು
ಪೂಜಿ-ಸಿದ
ಪೂಜಿ-ಸಿದೆ
ಪೂಜಿಸು
ಪೂಜಿ-ಸುತ್ತಿ
ಪೂಜಿ-ಸು-ತ್ತಿದ್ದ
ಪೂಜಿ-ಸು-ತ್ತಿದ್ದೆ
ಪೂಜಿ-ಸು-ವಾಗ
ಪೂಜಿ-ಸು-ವು-ದ-ಕ್ಕಾ-ಗಲೀ
ಪೂಜಿ-ಸು-ವು-ದಕ್ಕೆ
ಪೂಜೆ
ಪೂಜೆ-ಗಾಗಿ
ಪೂಜೆಗೆ
ಪೂಜೆಯ
ಪೂಜ್ಯ
ಪೂಜ್ಯರೆ
ಪೂರಿ
ಪೂರಿಗೆ
ಪೂರಿ-ಪ್ರ-ಸಿದ್ಧ
ಪೂರೈ-ಸ-ಬ-ಹುದು
ಪೂರೈ-ಸ-ಬೇಕು
ಪೂರೈ-ಸಿ-ಕೊಂಡ
ಪೂರೈ-ಸಿ-ರ-ಬೇಕು
ಪೂರ್ಣ
ಪೂರ್ಣ-ಜ್ಞಾನ
ಪೂರ್ಣ-ತೆ-ಯನ್ನು
ಪೂರ್ಣ-ವಾಗಿ
ಪೂರ್ಣ-ವಾ-ಗಿ-ರು-ವನು
ಪೂರ್ಣ-ವಾ-ಗು-ವು-ದಿಲ್ಲ
ಪೂರ್ತಿ
ಪೂರ್ವ
ಪೂರ್ವ-ಜನ್ಮ
ಪೂರ್ವಿ-ಕ-ರಲ್ಲಿ
ಪೂರ್ವಿ-ಕ-ರೆಲ್ಲ
ಪೆಟ್ಟಿ-ಗೆ-ಯನ್ನು
ಪೆಟ್ಟಿ-ಗೆ-ಯ-ಲ್ಲಿತ್ತು
ಪೆಟ್ಟಿ-ಗೆ-ಯ-ಲ್ಲಿದೆ
ಪೇಟೆಗೆ
ಪೇಟೆ-ದರ
ಪೇಟೆ-ಧಾ-ರ-ಣೆ-ಗಿಂತ
ಪೇಪ-ರನ್ನು
ಪೇಪ-ರಿ-ನಲ್ಲಿ
ಪೈಲ್ವಾನ್
ಪೊದೆ
ಪೊಲೀ-ಸಿ-ನ-ವರು
ಪೋಣಿ-ಸ-ಲಾ-ಗು-ವು-ದಿಲ್ಲ
ಪೋಣಿಸಿ
ಪೋಲಿ-ಸರು
ಪೋಲಿಸಿ
ಪೋಲಿ-ಸಿ-ನ-ವನು
ಪೋಲಿ-ಸಿ-ನ-ವರು
ಪೋಲೀ-ಸಿ-ನ-ವರು
ಪ್ರಕ-ಟ-ಣೆ-ಗಳು
ಪ್ರಕಾರ
ಪ್ರಕಾ-ಶ-ಕರು
ಪ್ರಕೃತಿ
ಪ್ರಕೃ-ತಿಗೆ
ಪ್ರಖ್ಯಾ-ತಿಗೆ
ಪ್ರಗ-ತಿ-ಯನ್ನು
ಪ್ರಚಂ-ಡ-ರಾ-ಗಿ-ದ್ದರು
ಪ್ರಚೋ-ದಿ-ಸಿ-ದ-ವನು
ಪ್ರಜ್ಞೆ
ಪ್ರಜ್ಞೆ-ತ-ಪ್ಪು-ವ-ವ-ರೆಗೆ
ಪ್ರಜ್ಞೆ-ಯನ್ನು
ಪ್ರತಾಪ
ಪ್ರತಾ-ಪ-ಚಂದ್ರ
ಪ್ರತಿ
ಪ್ರತಿಜ್ಞೆ
ಪ್ರತಿ-ದಿನ
ಪ್ರತಿ-ದಿ-ನವೂ
ಪ್ರತಿ-ಫ-ಲವ
ಪ್ರತಿ-ಫ-ಲಾ-ಪೇಕ್ಷೆ
ಪ್ರತಿ-ಬಿಂಬ
ಪ್ರತಿ-ಬಿಂ-ಬ-ವನ್ನು
ಪ್ರತಿ-ಯೊಂದು
ಪ್ರತಿ-ಯೊಬ್ಬ
ಪ್ರತಿ-ಯೊ-ಬ್ಬನೂ
ಪ್ರತಿ-ಯೊ-ಬ್ಬರೂ
ಪ್ರತಿಷ್ಠೆ
ಪ್ರತೀ-ಕ-ವಾದ
ಪ್ರತ್ಯಕ್ಷ
ಪ್ರತ್ಯ-ಕ್ಷ-ನಾಗಿ
ಪ್ರತ್ಯ-ಕ್ಷ-ನಾದ
ಪ್ರತ್ಯ-ಕ್ಷ-ನಾ-ದನು
ಪ್ರತ್ಯ-ಕ್ಷ-ಳಾ-ದಳು
ಪ್ರತ್ಯ-ಕ್ಷ-ವಾಗಿ
ಪ್ರತ್ಯ-ಕ್ಷ-ವಾ-ದೆನು
ಪ್ರದ-ಕ್ಷಿಣೆ
ಪ್ರಪಂಚ
ಪ್ರಪಂ-ಚದ
ಪ್ರಪಂ-ಚ-ದಲ್ಲಿ
ಪ್ರಪಂ-ಚ-ವನ್ನು
ಪ್ರಪಂ-ಚ-ವೆಲ್ಲ
ಪ್ರಪಂ-ಚವೇ
ಪ್ರಬ-ಲ-ವಾದ್ದು
ಪ್ರಭಾವ
ಪ್ರಭಾ-ವ-ದಿಂದ
ಪ್ರಭಾ-ವ-ವನ್ನು
ಪ್ರಭಾ-ವ-ವಾ-ಗು-ವುದು
ಪ್ರಯತ್ನ
ಪ್ರಯ-ತ್ನಕ್ಕೆ
ಪ್ರಯ-ತ್ನ-ದಿಂದ
ಪ್ರಯ-ತ್ನ-ಪಟ್ಟ
ಪ್ರಯ-ತ್ನ-ಪ-ಟ್ಟಿತು
ಪ್ರಯ-ತ್ನ-ಪಟ್ಟೆ
ಪ್ರಯ-ತ್ನ-ವನ್ನೂ
ಪ್ರಯತ್ನಿ
ಪ್ರಯ-ತ್ನಿಸ
ಪ್ರಯ-ತ್ನಿ-ಸು-ವುದು
ಪ್ರಯಾ
ಪ್ರಯಾ-ಣಿಕ
ಪ್ರಯಾ-ಣಿ-ಕ-ನನ್ನು
ಪ್ರಯಾ-ಣಿ-ಕನು
ಪ್ರಯಾ-ಣಿ-ಕರು
ಪ್ರಯೋ
ಪ್ರಯೋ-ಜನ
ಪ್ರಯೋ-ಜ-ನ-ಕಾ-ರಿಯ
ಪ್ರಯೋ-ಜ-ನ-ವಾ-ಗ-ಲಿಲ್ಲ
ಪ್ರಯೋ-ಜ-ನ-ವಾ-ಗು-ವು-ದಿಲ್ಲ
ಪ್ರಯೋ-ಜ-ನ-ವಿಲ್ಲ
ಪ್ರಯೋ-ಜ-ನ-ವೇನು
ಪ್ರಲೋ
ಪ್ರಲೋ-ಭನೆ
ಪ್ರಲೋ-ಭ-ನೆಗೆ
ಪ್ರಲೋ-ಭ-ನೆ-ಯಿಂದ
ಪ್ರಳ-ಯ-ಗಳನ್ನು
ಪ್ರವ-ಚ-ನ-ಗಳನ್ನು
ಪ್ರವಾ-ಹದ
ಪ್ರವಾ-ಹ-ದಂತೆ
ಪ್ರವಾ-ಹ-ವೊಂದು
ಪ್ರವೇ-ಶಿ-ಸ-ಬೇ-ಕಾ-ಯಿತು
ಪ್ರವೇ-ಶಿ-ಸು-ವುದು
ಪ್ರಶಾಂತ
ಪ್ರಶಾಂ-ತಿ-ಯನ್ನು
ಪ್ರಶ್ನಿ-ಸಿ-ದ-ವನು
ಪ್ರಶ್ನೆ
ಪ್ರಶ್ನೆ-ಗಳನ್ನು
ಪ್ರಶ್ನೆ-ಯನ್ನೇ
ಪ್ರಸಂ-ಗ-ವನ್ನೇ
ಪ್ರಸಾ-ದನ
ಪ್ರಸಾ-ದ-ವನ್ನು
ಪ್ರಸಿದ್ಧ
ಪ್ರಹ್ಲಾದ
ಪ್ರಹ್ಲಾ-ದ-ನನ್ನು
ಪ್ರಾಂಗ-ಣ-ದಲ್ಲಿ
ಪ್ರಾಣ
ಪ್ರಾಣ-ದಂತೆ
ಪ್ರಾಣ-ವನ್ನು
ಪ್ರಾಣ-ಸಂ-ಕಟ
ಪ್ರಾಣಿ-ಗಳಲ್ಲಿ
ಪ್ರಾಣಿಯ
ಪ್ರಾಣಿ-ಯನ್ನು
ಪ್ರಾತಃ-ಕಾ-ಲದ
ಪ್ರಾಪಂ-ಚಿಕ
ಪ್ರಾಪಂ-ಚಿ-ಕ-ತೆಯ
ಪ್ರಾಪಂ-ಚಿ-ಕರ
ಪ್ರಾಪಂ-ಚಿ-ಕ-ರನ್ನು
ಪ್ರಾಪಂ-ಚಿ-ಕ-ರಿಗೆ
ಪ್ರಾಪಂ-ಚಿ-ಕರು
ಪ್ರಾಪಂ-ಚಿ-ಕ-ರೊ-ಡನೆ
ಪ್ರಾಪ್ತ
ಪ್ರಾಪ್ತ-ವಾ-ಗು-ವುದು
ಪ್ರಾಪ್ತ-ವಾ-ದವು
ಪ್ರಾಪ್ತ-ವಾ-ದಾಗ
ಪ್ರಾಮಾಣಿ
ಪ್ರಾಯ-ಶ್ಚಿತ್ತ
ಪ್ರಾಯ-ಶ್ಚಿ-ತ್ತದ
ಪ್ರಾಯ-ಶ್ಚಿ-ತ್ತ-ವನ್ನು
ಪ್ರಾರಂಭ
ಪ್ರಾರಂ-ಭ-ವಾ-ಯಿತು
ಪ್ರಾರಂ-ಭಿ-ಸಿದ
ಪ್ರಾರಂ-ಭಿ-ಸಿ-ದರು
ಪ್ರಾರಬ್ಧ
ಪ್ರಾರ-ಬ್ಧ-ಕ-ರ್ಮ-ಗಳು
ಪ್ರಾರ-ಬ್ಧ-ಕ-ರ್ಮ-ದಿಂದ
ಪ್ರಾರ-ಬ್ಧ-ಕ-ರ್ಮ-ವನ್ನು
ಪ್ರಾರ್ಥನೆ
ಪ್ರಾರ್ಥ-ನೆ-ಯನ್ನು
ಪ್ರಾರ್ಥ-ನೆ-ಯಲ್ಲಿ
ಪ್ರಾರ್ಥಿ-ಸ-ತೊ-ಡ-ಗಿದ
ಪ್ರಾರ್ಥಿ-ಸ-ಬೇಕು
ಪ್ರಾರ್ಥಿ-ಸ-ಲಿಲ್ಲ
ಪ್ರಾರ್ಥಿ-ಸಿದ
ಪ್ರಾರ್ಥಿ-ಸಿ-ದನು
ಪ್ರಾರ್ಥಿ-ಸಿ-ದರು
ಪ್ರಾರ್ಥಿ-ಸಿ-ದಳು
ಪ್ರಾರ್ಥಿಸು
ಪ್ರಾರ್ಥಿ-ಸು-ವಾಗ
ಪ್ರಾರ್ಥಿ-ಸೋಣ
ಪ್ರಿಯ
ಪ್ರಿಯ-ತ-ಮನ
ಪ್ರಿಯ-ತ-ಮನೆ
ಪ್ರಿಯ-ವಾ-ಗಿ-ದ್ದರೆ
ಪ್ರೀತಿ
ಪ್ರೀತಿಗೆ
ಪ್ರೀತಿಯ
ಪ್ರೀತಿ-ಯಿಂದ
ಪ್ರೀತಿಯೂ
ಪ್ರೀತಿಯೆ
ಪ್ರೀತಿ-ಸ-ಲಾರೆ
ಪ್ರೀತಿ-ಸಲಿ
ಪ್ರೀತಿ-ಸಿ-ದರೂ
ಪ್ರೀತಿ-ಸಿ-ದರೆ
ಪ್ರೀತಿಸು
ಪ್ರೀತಿ-ಸು-ತ್ತಾರೆ
ಪ್ರೀತಿ-ಸು-ತ್ತಾಳೆ
ಪ್ರೀತಿ-ಸು-ತ್ತಿದ್ದ
ಪ್ರೀತಿ-ಸು-ತ್ತಿ-ದ್ದರು
ಪ್ರೀತಿ-ಸು-ತ್ತಿದ್ದೆ
ಪ್ರೀತಿ-ಸು-ತ್ತೀಯ
ಪ್ರೀತಿ-ಸು-ತ್ತೇನೆ
ಪ್ರೀತಿ-ಸು-ವರೆ
ಪ್ರೀತಿ-ಸು-ವು-ದಿಲ್ಲ
ಪ್ರೀತಿ-ಸುವೆ
ಪ್ರೇಕ್ಷ-ಕ-ರಿಗೆ
ಪ್ರೇಮ
ಪ್ರೇಮ-ಗಳೇ
ಪ್ರೇಮ-ದಲ್ಲಿ
ಪ್ರೇಮಾಶ್ರು
ಫಕೀ-ರ-ರಂತೆ
ಫಲ
ಫಲ-ಕಾರಿ
ಫಲ-ಕಾ-ರಿ-ಯಾ-ಗ-ಲಿಲ್ಲ
ಫಲ-ಕಾ-ರಿ-ಯಾಗು
ಫಲ-ಗ-ಳನು
ಫಲ-ಗಳು
ಫಲ-ಬೇಕು
ಫಲ-ವಾಗಿ
ಫಲ-ವಿ-ರಲು
ಫಲಿ-ಸು-ವ-ವ-ರೆಗೆ
ಬಂಗ-ಲೆ-ಯಾ-ಗ-ಬೇಕು
ಬಂಗಾ-ರದ
ಬಂಗಾ-ರ-ದಿಂದ
ಬಂಗಾಳಿ
ಬಂತು
ಬಂತೋ
ಬಂದ
ಬಂದಂತೆ
ಬಂದದ್ದು
ಬಂದನು
ಬಂದ-ಮೇಲೆ
ಬಂದರು
ಬಂದರೂ
ಬಂದರೆ
ಬಂದಳು
ಬಂದ-ವನು
ಬಂದ-ವ-ರಿಂದ
ಬಂದ-ವರು
ಬಂದವು
ಬಂದವೋ
ಬಂದಾಗ
ಬಂದಾದ
ಬಂದಿತು
ಬಂದಿದೆ
ಬಂದಿ-ದೆಯೆ
ಬಂದಿ-ದ್ದರು
ಬಂದಿ-ದ್ದರೆ
ಬಂದಿ-ರ-ಲಿಲ್ಲ
ಬಂದಿ-ರಲ್ಲ
ಬಂದಿರಿ
ಬಂದಿ-ರು-ವನು
ಬಂದಿ-ರು-ವೆನು
ಬಂದಿ-ರು-ವೆವು
ಬಂದಿರೊ
ಬಂದಿಲ್ಲ
ಬಂದಿವೆ
ಬಂದು
ಬಂದು-ಬಿಡು
ಬಂದೆ
ಬಂದೊ-ಡನೆ
ಬಂಧನ
ಬಂಧ-ನಕ್ಕೆ
ಬಂಧ-ನ-ಗ-ಳಿವೆ
ಬಂಧ-ನ-ದಿಂದ
ಬಂಧು
ಬಂಧು-ಗಳ
ಬಂಧು-ಗ-ಳಿಗೆ
ಬಂಧು-ಗಳು
ಬಂಧ್
ಬಗೆ
ಬಗೆ-ದರೆ
ಬಗೆ-ಬ-ಗೆಯ
ಬಗೆಯ
ಬಗೆ-ಯು-ವನು
ಬಗ್ಗಿ
ಬಗ್ಗೆ
ಬಚ್ಚಿ-ಟ್ಟಿತ್ತು
ಬಚ್ಚಿ-ಟ್ಟಿ-ರು-ವಳು
ಬಚ್ಚಿ-ಟ್ಟಿ-ರು-ವೆನು
ಬಚ್ಚಿ-ಟ್ಟು-ಕೊಂ-ಡಳು
ಬಟ್ಟ-ಲನ್ನು
ಬಟ್ಟ-ಲಿನ
ಬಟ್ಟಲು
ಬಟ್ಟೆ
ಬಟ್ಟೆ-ಗಳೂ
ಬಟ್ಟೆಗೆ
ಬಟ್ಟೆಯ
ಬಟ್ಟೆ-ಯ-ನ್ನಿತ್ತು
ಬಟ್ಟೆ-ಯನ್ನು
ಬಟ್ಟೆ-ಯಲ್ಲಿ
ಬಟ್ಟೆ-ಯಿಂದ
ಬಟ್ಟೆಯೂ
ಬಡ
ಬಡ-ತ-ನ-ದಲ್ಲಿ
ಬಡ-ಬ್ರಾ-ಹ್ಮಣ
ಬಡವ
ಬಡ-ವಾ-ಗಿ-ರ-ಬ-ಹುದು
ಬಡಿ-ದಂತೆ
ಬಡಿ-ದು-ಕೊ-ಳ್ಳು-ವುದು
ಬಡಿ-ಯು-ತ್ತಿತ್ತು
ಬಡಿ-ಸಿ-ದನು
ಬಡಿ-ಸು-ವಾಗ
ಬಡ್ಡಿ
ಬಣ್ಣ
ಬಣ್ಣ-ಗಳನ್ನು
ಬಣ್ಣದ
ಬಣ್ಣದ್ದು
ಬಣ್ಣ-ವನ್ನು
ಬಣ್ಣವೂ
ಬದನೆ
ಬದ-ನೆ-ಕಾಯಿ
ಬದ-ನೆ-ಕಾ-ಯಿಗೆ
ಬದಲಾ
ಬದ-ಲಾ-ಯಿತು
ಬದ-ಲಾ-ಯಿಸಿ
ಬದ-ಲಾ-ಯಿ-ಸಿ-ಕೊಂಡು
ಬದ-ಲಾ-ಯಿ-ಸಿ-ರು-ವನು
ಬದ-ಲಾ-ವಣೆ
ಬದ-ಲಾ-ವ-ಣೆ-ಯನ್ನು
ಬದ-ಲಿ-ಸಿ-ಕೊಂಡು
ಬದಲು
ಬದಿ-ಗಿ-ರಿಸಿ
ಬದಿಗೆ
ಬದುಕಿ
ಬದು-ಕಿ-ಕೊ-ಳ್ಳು-ವು-ದ-ರಲ್ಲಿ
ಬದು-ಕಿ-ಸ-ಬ-ಲ್ಲೆಯಾ
ಬದು-ಕಿ-ಸಲು
ಬದು-ಕಿ-ಸಿದೆ
ಬದು-ಕಿ-ಸು-ತ್ತೇನೆ
ಬದು-ಕಿ-ಸು-ವಂತೆ
ಬದು-ಕುವ
ಬದು-ಕು-ವನು
ಬದ್ಧ
ಬದ್ಧ-ರಾ-ಗಿ-ರು-ವು-ದಿಲ್ಲ
ಬದ್ಧ-ರಾ-ದ-ವರು
ಬದ್ಧರೊ
ಬನ್ನಿ
ಬಯಕೆ
ಬಯಲ
ಬಯ-ಲಿನ
ಬಯ-ಲಿ-ನಲ್ಲಿ
ಬಯ-ಲು-ಗಾ-ವ-ಲಿಗೆ
ಬಯ-ಸಿದೆ
ಬಯ-ಸು-ವರೋ
ಬಯಸೆ
ಬಯ್ಯು-ತ್ತಿ-ದ್ದರು
ಬರ
ಬರ-ಗಾಲ
ಬರ-ಗಾ-ಲ-ವಿಲ್ಲ
ಬರ-ಗಾ-ಲವೂ
ಬರಡು
ಬರ-ದಂತೆ
ಬರದೆ
ಬರ-ಬ-ಹುದು
ಬರ-ಬಾ-ರದು
ಬರ-ಬೇಕಾ
ಬರ-ಬೇ-ಕಾ-ಗು-ವುದು
ಬರ-ಬೇ-ಕಾ-ದರೆ
ಬರ-ಬೇ-ಕಾ-ಯಿತು
ಬರ-ಬೇಕು
ಬರ-ಮಾಡಿ
ಬರ-ಮಾ-ಡಿ-ಕೊಂಡು
ಬರ-ಲಾ-ರಂ-ಭಿ-ಸಿ-ದರು
ಬರ-ಲಾ-ರ-ದಾಗ
ಬರಲಿ
ಬರ-ಲಿಲ್ಲ
ಬರಲು
ಬರಲೇ
ಬರ-ಲೇ-ಬೇಕು
ಬರೀ
ಬರು
ಬರು-ತ್ತದೆ
ಬರು-ತ್ತವೆ
ಬರು-ತ್ತಾನೆ
ಬರು-ತ್ತಾರೆ
ಬರು-ತ್ತಿತ್ತು
ಬರು-ತ್ತಿತ್ತೋ
ಬರು-ತ್ತಿದೆ
ಬರು-ತ್ತಿದ್ದ
ಬರು-ತ್ತಿ-ದ್ದರು
ಬರು-ತ್ತಿ-ದ್ದಾಗ
ಬರು-ತ್ತಿ-ದ್ದಾರೆ
ಬರು-ತ್ತಿರ
ಬರು-ತ್ತಿ-ರ-ಲಿಲ್ಲ
ಬರು-ತ್ತಿ-ರುವ
ಬರು-ತ್ತಿ-ರು-ವಾಗ
ಬರು-ತ್ತಿ-ರು-ವಿ-ರೇನು
ಬರು-ತ್ತಿ-ರು-ವುದನ್ನು
ಬರು-ತ್ತಿಲ್ಲ
ಬರು-ತ್ತೇನೆ
ಬರುವ
ಬರು-ವಂತೆ
ಬರು-ವನು
ಬರು-ವರೊ
ಬರು-ವ-ವರು
ಬರು-ವಾಗ
ಬರು-ವಾ-ಗಲೂ
ಬರುವು
ಬರು-ವು-ದ-ಕ್ಕಿಂತ
ಬರು-ವು-ದಕ್ಕೆ
ಬರು-ವುದನ್ನು
ಬರು-ವು-ದಿಲ್ಲ
ಬರು-ವುದು
ಬರು-ವುದೆ
ಬರು-ವು-ದೆಂದು
ಬರು-ವುದೊ
ಬರು-ವುವು
ಬರುವೆ
ಬರು-ವೆನು
ಬರು-ವೆಯಾ
ಬರೆದ
ಬರೆ-ದಿತ್ತು
ಬರೆ-ದಿ-ದ್ದರು
ಬರೆ-ದಿ-ರು-ವನೊ
ಬರೆ-ದಿ-ರು-ವುದನ್ನು
ಬರೆದು
ಬರೆ-ಯನ್ನು
ಬರ್ದ-ವಾನ್
ಬಲ
ಬಲ-ಗ-ಡೆ-ಯಿಂದ
ಬಲ-ದಿಂದ
ಬಲ-ವಾಗಿ
ಬಲ-ವಾ-ಗು-ವುದು
ಬಲ-ವಾ-ದವು
ಬಲ-ವಾ-ಯಿತು
ಬಲ-ಶಾಲಿ
ಬಲಾ-ತ್ಕಾ-ರ-ದಿಂದ
ಬಲಿ
ಬಲಿ-ಕೊ-ಡು-ತ್ತಾರೆ
ಬಲಿ-ಕೊ-ಡು-ತ್ತಿ-ದ್ದರು
ಬಲಿ-ಕೊ-ಡು-ವುದು
ಬಲಿಷ್ಠ
ಬಲಿ-ಷ್ಠ-ನಾ-ಗು-ತ್ತಾನೆ
ಬಲೆ
ಬಲೆಗೆ
ಬಲೆ-ಯನ್ನು
ಬಲೆ-ಯಲ್ಲಿ
ಬಲೆ-ಯಿಂದ
ಬಲೆ-ಯೊ-ಳಗೆ
ಬಲ್ಲ
ಬಲ್ಲನು
ಬಲ್ಲ-ವನು
ಬಲ್ಲೆ
ಬಲ್ಲೆಯಾ
ಬಳ-ಗ-ದ-ವರ
ಬಳ-ಲು-ತ್ತಿ-ರುವೆ
ಬಳ-ಸಿ-ಕೊಂಡು
ಬಳಿ
ಬಳಿಗೆ
ಬಳಿದು
ಬಳಿ-ದು-ಕೊಂ-ಡಿದ್ದ
ಬಳಿ-ದು-ಕೊಂ-ಡಿ-ದ್ದನ್ನು
ಬಳಿಯೇ
ಬಳೆಯ
ಬಳೆ-ಯಂತೆ
ಬಳೆ-ಯನ್ನು
ಬಳೆ-ಯಿಂದ
ಬಳ್ಳಿಯ
ಬವಣೆ
ಬಹಳ
ಬಹ-ಳ-ದಿ-ನ-ಗಳು
ಬಹು
ಬಹುದು
ಬಹು-ದೂರ
ಬಹು-ಭಾಗ
ಬಹು-ಮಾನ
ಬಹು-ಮಾ-ನ-ವನ್ನು
ಬಹು-ಮಾ-ನ-ವನ್ನೂ
ಬಹು-ಮಾ-ನ-ವಾಗಿ
ಬಹು-ರೂಪಿ
ಬಹುಶಃ
ಬಾ
ಬಾಕಿ-ಯನ್ನು
ಬಾಗ-ಬ-ಜಾರ್
ಬಾಗಿ-ರುವ
ಬಾಗಿ-ಲನ್ನು
ಬಾಗಿ-ಲಿಗೆ
ಬಾಗಿ-ಲಿನ
ಬಾಗಿಲು
ಬಾಗಿ-ಲು-ಗಳ
ಬಾಗಿ-ಲು-ಗಳೂ
ಬಾಡಿ-ಗೆಗೆ
ಬಾಣ
ಬಾಧಕ
ಬಾಧ-ಕ-ಗಳನ್ನು
ಬಾಧಿ-ಸು-ತ್ತಿತ್ತು
ಬಾಧಿ-ಸು-ವುದು
ಬಾಬು-ವಿನ
ಬಾಯಲ್ಲಿ
ಬಾಯ-ಲ್ಲಿದ್ದ
ಬಾಯಾ-ರಿಕೆ
ಬಾಯಾ-ರಿ-ಕೆ-ಯಾ-ಗಲಿ
ಬಾಯಾ-ರಿ-ಕೆ-ಯಾಗಿ
ಬಾಯಿಂದ
ಬಾಯಿಗೆ
ಬಾಯಿ-ಯಲ್ಲಿ
ಬಾಯಿ-ಯ-ಲ್ಲಿ-ಟ್ಟು-ಕೊಂಡು
ಬಾಯಿ-ಯ-ಲ್ಲಿದ್ದ
ಬಾಯಿ-ಯೊ-ಳಗೆ
ಬಾರದ
ಬಾರ-ದಂತೆ
ಬಾರ-ದ-ವ-ನಾ-ಗಿ-ರು-ತ್ತಾನೆ
ಬಾರ-ದ-ವನು
ಬಾರ-ದ-ವರು
ಬಾರದು
ಬಾರ-ದು-ದನ್ನು
ಬಾರದೆ
ಬಾರನು
ಬಾರಿ
ಬಾರಿಯೂ
ಬಾರಿ-ಸಾ-ಲ್ನಿಂದ
ಬಾರಿ-ಸು-ತ್ತಿದ್ದ
ಬಾಲ-ಕ-ನಂತೆ
ಬಾಲ-ಕನು
ಬಾಲಕ್ಕೆ
ಬಾಲ್ಯ-ದಲ್ಲೇ
ಬಾಳ
ಬಾಳನ್ನು
ಬಾಳನ್ನೇ
ಬಾಳ-ಬೇ-ಕೆಂಬ
ಬಾಳು
ಬಾವ-ಲಿ-ಗಳ
ಬಾವ-ಲಿ-ಗಳು
ಬಾವ-ಲಿ-ಗಳೇ
ಬಾವ-ಲಿಯ
ಬಾವಿ
ಬಾವಿ-ಗಿಂತ
ಬಾವಿಯ
ಬಾವಿ-ಯನ್ನು
ಬಾವಿ-ಯಲ್ಲಿ
ಬಾವಿ-ಯ-ಲ್ಲಿತ್ತು
ಬಾವಿ-ಯಷ್ಟು
ಬಾವಿ-ಯೊಂ-ದಿಗೆ
ಬಾಹ್ಯ
ಬಾಹ್ಯ-ದ-ಲ್ಲಿಯೂ
ಬಾಹ್ಯಾ-ಡಂ-ಬ-ರ-ವಾ-ಗಿತ್ತು
ಬಿಗಿ-ದಿರು
ಬಿಗಿ-ದಿ-ರು-ವುದು
ಬಿಗಿದು
ಬಿಗಿ-ಯಾಗಿ
ಬಿಗಿ-ಯು-ತ್ತಾರೆ
ಬಿಚ್ಚಿ
ಬಿಟ್ಟ
ಬಿಟ್ಟನು
ಬಿಟ್ಟ-ಬಿ-ಡುವೆ
ಬಿಟ್ಟ-ಮೇಲೆ
ಬಿಟ್ಟರೆ
ಬಿಟ್ಟಿ
ಬಿಟ್ಟಿತು
ಬಿಟ್ಟಿ-ರ-ಲಾ-ರಳು
ಬಿಟ್ಟಿರಿ
ಬಿಟ್ಟು
ಬಿಟ್ಟು-ಬಿಟ್ಟ
ಬಿಟ್ಟು-ಬಿ-ಟ್ಟಿತ್ತು
ಬಿಟ್ಟು-ಬಿಡಿ
ಬಿಟ್ಟೆ
ಬಿಡದೆ
ಬಿಡ-ಬ-ಹುದು
ಬಿಡ-ಬೇಕು
ಬಿಡ-ಬೇ-ಕೆಂ-ದರೂ
ಬಿಡ-ಬೇಡ
ಬಿಡ-ಲಾ-ರದು
ಬಿಡ-ಲಾ-ರಿರಿ
ಬಿಡ-ಲಾರೆ
ಬಿಡಲಿ
ಬಿಡ-ಲಿಲ್ಲ
ಬಿಡಿ-ಸ-ಲಾ-ಗ-ಲಿಲ್ಲ
ಬಿಡಿ-ಸಿದ
ಬಿಡಿಸು
ಬಿಡು
ಬಿಡು-ಗಡೆ
ಬಿಡು-ಗ-ಡೆ-ಮಾಡಿ
ಬಿಡು-ತ್ತದೆ
ಬಿಡು-ತ್ತಿದ್ದ
ಬಿಡು-ತ್ತಿ-ರು-ವನು
ಬಿಡು-ತ್ತಿ-ರು-ವನೆ
ಬಿಡು-ತ್ತೇನೆ
ಬಿಡುವ
ಬಿಡು-ವಾಗ
ಬಿಡುವು
ಬಿಡು-ವು-ದಕ್ಕೆ
ಬಿಡು-ವು-ದಿಲ್ಲ
ಬಿಡು-ವುದು
ಬಿಡುವೆ
ಬಿಡೋಣ
ಬಿತ್ತಿ-ದರೆ
ಬಿತ್ತಿ-ದ್ದೆಯೋ
ಬಿತ್ತು
ಬಿತ್ತೋ
ಬಿದ್ದ
ಬಿದ್ದದ್ದು
ಬಿದ್ದನು
ಬಿದ್ದ-ಮೇಲೆ
ಬಿದ್ದರೆ
ಬಿದ್ದ-ವರೆಲ್ಲ
ಬಿದ್ದವು
ಬಿದ್ದಿ
ಬಿದ್ದಿತು
ಬಿದ್ದಿದೆ
ಬಿದ್ದಿ-ರ-ಬೇಕು
ಬಿದ್ದಿ-ರು-ವುದನ್ನು
ಬಿದ್ದಿ-ರು-ವು-ದ-ರಿಂದ
ಬಿದ್ದಿ-ರುವೆ
ಬಿದ್ದಿ-ರು-ವೆನು
ಬಿದ್ದು
ಬಿದ್ದು-ದನ್ನು
ಬಿದ್ದು-ಹೋಗು
ಬಿದ್ದೆ
ಬಿರು-ಗಾಳಿ
ಬಿರು-ಗಾ-ಳಿ-ಯನ್ನು
ಬಿರುಸು
ಬಿಲಕ್ಕೆ
ಬಿಲದ
ಬಿಲ-ದಲ್ಲಿ
ಬಿಲ-ದಿಂದ
ಬಿಲ್ಲಿಗೆ
ಬಿಲ್ವ-ಮಂ-ಗಲ
ಬಿಲ್ವ-ಮಂ-ಗ-ಲನ
ಬಿಲ್ವ-ಮಂ-ಗ-ಲನು
ಬಿಲ್ವ-ಮಂ-ಗ-ಳನ
ಬಿಲ್ವ-ಮ-ರ-ವನ್ನು
ಬಿಸಾ-ಕಿದ್ದ
ಬಿಸಿ
ಬಿಸಿ-ಲಿ-ನಲ್ಲಿ
ಬಿಸಿ-ಲೆ-ನ್ನದೆ
ಬೀಗ
ಬೀಜ
ಬೀರಿ
ಬೀರು
ಬೀಳ-ತೊ-ಡಗು
ಬೀಳದ
ಬೀಳ-ದಂತೆ
ಬೀಳ-ಲಿಲ್ಲ
ಬೀಳಲು
ಬೀಳು-ತ್ತವೆ
ಬೀಳು-ತ್ತಿ-ರು-ವಾ-ಗಲೇ
ಬೀಳುವ
ಬೀಳು-ವು-ದರ
ಬೀಳು-ವು-ದ-ರ-ಲ್ಲಿತ್ತು
ಬೀಳು-ವು-ದಿಲ್ಲ
ಬೀಳು-ವುದು
ಬೀಳು-ವುದೇ
ಬೀಳ್ಕೊ-ಡು-ವಾಗ
ಬೀಸ
ಬೀಸಿ
ಬುಟ್ಟಿ-ಯನ್ನು
ಬುಡ-ದಲ್ಲಿ
ಬುದ್ಧಿ
ಬುದ್ಧಿ-ಮಂ-ಕರು
ಬುದ್ಧಿ-ಯಿ-ಲ್ಲದ
ಬುದ್ಧಿ-ವಂತ
ಬುದ್ಧಿ-ವಂ-ತ-ನಾ-ಗಿದ್ದ
ಬುದ್ಧಿ-ವಂ-ತ-ನಾದ
ಬುದ್ಧಿ-ವಾದ
ಬುದ್ಧಿ-ವಾ-ದ-ದಂತೆ
ಬುದ್ಧಿ-ವಾ-ದ-ದಿಂದ
ಬುದ್ಧಿ-ಶ-ಕ್ತಿ-ಯಿಂದ
ಬುದ್ಧಿ-ಹೀ-ನರು
ಬುಸು
ಬುಸು-ಗು-ಟ್ಟ-ಬೇಕು
ಬುಸು-ಗು-ಟ್ಟ-ಬೇಡ
ಬುಸು-ಗುಟ್ಟಿ
ಬುಸು-ಗುಟ್ಟು
ಬೂಟು-ಗಳನ್ನು
ಬೂದು
ಬೃಹತ್
ಬೆಂಕಿ
ಬೆಂಕಿ-ಗಾಗಿ
ಬೆಂಕಿ-ಪ-ಟ್ಟಿ-ಗೆ-ಯನ್ನು
ಬೆಂಡನ್ನು
ಬೆಂಡಿಗೆ
ಬೆಂಡು
ಬೆಕ್ಕನ್ನು
ಬೆಕ್ಕಿಗೂ
ಬೆಕ್ಕು
ಬೆಚ್ಚಗೆ
ಬೆಟ್ಟಕ್ಕೆ
ಬೆಟ್ಟದ
ಬೆಟ್ಟ-ವನ್ನೇ
ಬೆಣ್ಣೆ
ಬೆನ್ನಿನ
ಬೆರ-ಳನ್ನು
ಬೆರಳು
ಬೆರ-ಳು-ಗಳನ್ನು
ಬೆರೆ-ಯ-ಬ-ಹುದು
ಬೆರೆ-ಯ-ಬೇ-ಡಿ-ಜೋ-ಪಾನ
ಬೆಲೆ
ಬೆಲೆ-ಬಾ-ಳುವ
ಬೆಲೆ-ಯನ್ನು
ಬೆಲ್ಲ
ಬೆಲ್ಲದ
ಬೆಲ್ಲ-ವನ್ನು
ಬೆಳ-ಕಿಗೆ
ಬೆಳ-ಕಿನ
ಬೆಳ-ಕಿ-ನಿಂದ
ಬೆಳಕು
ಬೆಳ-ಗಿ-ನಿಂದ
ಬೆಳಗ್ಗೆ
ಬೆಳಿಗ್ಗೆ
ಬೆಳೆ
ಬೆಳೆದ
ಬೆಳೆ-ದಿ-ದ್ದವು
ಬೆಳೆ-ಯು-ವು-ದೆಂಬ
ಬೆಳೆ-ಯೆಲ್ಲ
ಬೆಳೆ-ಸ-ಬೇ-ಕಾ-ಗಿದೆ
ಬೆಳೆಸಿ
ಬೆಳ್ಳ-ಗಿತ್ತು
ಬೆಳ್ಳಿ
ಬೆಳ್ಳಿಯ
ಬೆವ-ರನ್ನು
ಬೆವೆ-ತಿತ್ತು
ಬೆಸ್ತ
ಬೆಸ್ತ-ನಿಗೆ
ಬೆಸ್ತ-ನೊಬ್ಬ
ಬೆಸ್ತರ
ಬೆಸ್ತ-ರ-ವನು
ಬೆಸ್ತ-ರ-ವ-ನೊಬ್ಬ
ಬೆಸ್ತರು
ಬೇಕಾ
ಬೇಕಾ-ಗಿತ್ತು
ಬೇಕಾ-ಗಿ-ರ-ಲಿಲ್ಲ
ಬೇಕಾ-ಗಿ-ರು-ವುದು
ಬೇಕಾ-ಗಿಲ್ಲ
ಬೇಕಾ-ಗು-ತ್ತದೆ
ಬೇಕಾ-ಗುವ
ಬೇಕಾದ
ಬೇಕಾ-ದರೂ
ಬೇಕಾ-ದರೆ
ಬೇಕಾ-ದಷ್ಟು
ಬೇಕಾ-ದು-ದನ್ನು
ಬೇಕಾ-ಯಿತು
ಬೇಕಿತ್ತು
ಬೇಕು
ಬೇಕೆಂ-ದಿತ್ತು
ಬೇಕೆಂದು
ಬೇಕೊ
ಬೇಕೋ
ಬೇಗ
ಬೇಗುದಿ
ಬೇಜಾ
ಬೇಜಾ-ರಾಗಿ
ಬೇಜಾ-ರಾ-ಗಿ-ಲ್ಲವೆ
ಬೇಜಾ-ರಾ-ಯಿತು
ಬೇಟೆ-ಗಾ-ರ-ನನ್ನು
ಬೇಟೆ-ಯ-ವನು
ಬೇಡ
ಬೇಡದ
ಬೇಡದೆ
ಬೇಡ-ಬೇಕಾ
ಬೇಡ-ಬೇ-ಕಾ-ಗಿ-ರ-ಲಿಲ್ಲ
ಬೇಡ-ಬೇ-ಕಾ-ದರೆ
ಬೇಡಲಿ
ಬೇಡಲು
ಬೇಡವೇ
ಬೇಡಿ
ಬೇಡಿ-ಕೊಂ-ಡನು
ಬೇಡಿ-ಕೊಂ-ಡಳು
ಬೇಡಿ-ಕೊಂಡು
ಬೇಡಿದ
ಬೇಡಿ-ದರೂ
ಬೇಡು-ತ್ತಿ-ದ್ದು-ದನ್ನು
ಬೇಡು-ತ್ತಿದ್ದೆ
ಬೇಡು-ತ್ತೇನೆ
ಬೇಡು-ವ-ವ-ನಲ್ಲ
ಬೇಡು-ವು-ದಕ್ಕೆ
ಬೇಡು-ವು-ದಿಲ್ಲ
ಬೇರೆ
ಬೇರೆ-ಯ-ವರ
ಬೇರೆ-ಯಾ-ಗು-ವುದು
ಬೇಸ-ತ್ತಿದ್ದ
ಬೇಸಾಯ
ಬೇಸಾ-ಯ-ಗಾರ
ಬೇಸಾ-ಯ-ಗಾ-ರ-ನೊಬ್ಬ
ಬೇಸಾ-ಯದ
ಬೇಸಾ-ಯ-ದಲ್ಲಿ
ಬೇಸಾ-ಯ-ದಿಂದ
ಬೈದು
ಬೋಧನೆ
ಬೋಧ-ನೆಗೆ
ಬೋಧ-ನೆ-ಯನ್ನು
ಬೋಧ-ನೆ-ಯನ್ನೂ
ಬೋಧಿ-ಸ-ಬಲ್ಲ
ಬೋಧಿ-ಸ-ಬಾ-ರದು
ಬೋಧಿ-ಸ-ಬೇ-ಕೆಂ-ದಿ-ದ್ದನು
ಬೋಧಿ-ಸಲು
ಬೋಧಿಸಿ
ಬೋಧಿ-ಸಿದ
ಬೋಧಿ-ಸಿ-ದರೆ
ಬೋಧಿಸು
ಬೋಧಿ-ಸು-ತ್ತೇನೆ
ಬೋಧಿ-ಸುವ
ಬ್ಬರ
ಬ್ಯಾನರ್ಜಿ
ಬ್ರಹ್ಮ
ಬ್ರಹ್ಮ-ಕಳೆ
ಬ್ರಹ್ಮ-ಚಾರಿ
ಬ್ರಹ್ಮ-ಚಾ-ರಿಗೆ
ಬ್ರಹ್ಮ-ಚಾ-ರಿಯ
ಬ್ರಹ್ಮ-ಚಾ-ರಿ-ಯನ್ನು
ಬ್ರಹ್ಮ-ಜ್ಞಾನ
ಬ್ರಹ್ಮ-ಜ್ಞಾ-ನ-ಕ್ಕಿಂತ
ಬ್ರಹ್ಮ-ಜ್ಞಾ-ನ-ವನ್ನು
ಬ್ರಹ್ಮ-ಜ್ಞಾ-ನ-ವಾ-ಗು-ವುದೊ
ಬ್ರಹ್ಮ-ಜ್ಞಾ-ನಿ-ಗ-ಳಾ-ಗಿ-ದ್ದರು
ಬ್ರಹ್ಮ-ಜ್ಞಾ-ನಿಗೆ
ಬ್ರಹ್ಮದ
ಬ್ರಹ್ಮನ
ಬ್ರಹ್ಮ-ನ-ನ್ನಾ-ಗಲಿ
ಬ್ರಹ್ಮ-ನನ್ನು
ಬ್ರಹ್ಮ-ನಿಂದ
ಬ್ರಹ್ಮನು
ಬ್ರಹ್ಮನೂ
ಬ್ರಹ್ಮ-ನೊ-ಬ್ಬನೆ
ಬ್ರಹ್ಮ-ಮಯ
ಬ್ರಹ್ಮ-ವನ್ನು
ಬ್ರಹ್ಮ-ವಿ-ದ್ಯೆ-ಯನ್ನು
ಬ್ರಹ್ಮವು
ಬ್ರಹ್ಮ-ಸ-ಮಾ-ಜಕ್ಕೆ
ಬ್ರಹ್ಮ-ಸ-ಮಾ-ಜದ
ಬ್ರಾಹ್ಮಣ
ಬ್ರಾಹ್ಮ-ಣನ
ಬ್ರಾಹ್ಮ-ಣ-ನನ್ನು
ಬ್ರಾಹ್ಮ-ಣ-ನ-ಲ್ಲಿದ್ದ
ಬ್ರಾಹ್ಮ-ಣ-ನಿಗೆ
ಬ್ರಾಹ್ಮ-ಣನು
ಬ್ರಾಹ್ಮ-ಣ-ನೊಬ್ಬ
ಭಂಗ
ಭಂಗಿ-ಯನ್ನು
ಭಂಡ
ಭಂಡಾರ
ಭಕ್ತ
ಭಕ್ತನ
ಭಕ್ತ-ನನ್ನು
ಭಕ್ತ-ನಲ್ಲಿ
ಭಕ್ತ-ನಾ-ಗಿದ್ದ
ಭಕ್ತ-ನಾದ
ಭಕ್ತ-ನಾ-ದರೋ
ಭಕ್ತ-ನಿಗೆ
ಭಕ್ತ-ನಿದ್ದ
ಭಕ್ತನು
ಭಕ್ತನೇ
ಭಕ್ತ-ನೊ-ಬ್ಬನ
ಭಕ್ತ-ಪ-ರಾ-ಧೀನ
ಭಕ್ತ-ಮಾ-ಲೆ-ಯಲ್ಲಿ
ಭಕ್ತ-ರಿ-ಗಾಗಿ
ಭಕ್ತರು
ಭಕ್ತರೂ
ಭಕ್ತಳು
ಭಕ್ತಿ
ಭಕ್ತಿ-ಗ-ಳಿ-ದ್ದರೆ
ಭಕ್ತಿ-ಗಾಗಿ
ಭಕ್ತಿಗೆ
ಭಕ್ತಿ-ಪ-ರ-ವ-ಶ-ನಾಗಿ
ಭಕ್ತಿ-ಬೀ-ಜ-ವನ್ನು
ಭಕ್ತಿ-ಭಾ-ವ-ದಿಂದ
ಭಕ್ತಿಯ
ಭಕ್ತಿ-ಯನ್ನು
ಭಕ್ತಿ-ಯಿಂದ
ಭಕ್ತಿ-ಯಿದ್ದ
ಭಕ್ತಿ-ಯಿ-ರ-ಬೇಕು
ಭಕ್ತಿ-ಯುಳ್ಳ
ಭಕ್ತೆ
ಭಗ
ಭಗ-ವಂತ
ಭಗ-ವಂ-ತನ
ಭಗ-ವಂ-ತ-ನದು
ಭಗ-ವಂ-ತ-ನನ್ನು
ಭಗ-ವಂ-ತ-ನನ್ನೇ
ಭಗ-ವಂ-ತ-ನಲ್ಲಿ
ಭಗ-ವಂ-ತ-ನಾದ
ಭಗ-ವಂ-ತ-ನಿಂದ
ಭಗ-ವಂ-ತ-ನಿ-ಗಾಗಿ
ಭಗ-ವಂ-ತ-ನಿಗೆ
ಭಗ-ವಂ-ತನು
ಭಗ-ವಂ-ತ-ನೆ-ಡೆಗೆ
ಭಗ-ವಂ-ತನೇ
ಭಗ-ವತಿ
ಭಗ-ವ-ತಿಗೆ
ಭಗ-ವ-ತಿಯ
ಭಗ-ವ-ತಿ-ಯನ್ನು
ಭಗ-ವ-ತಿಯು
ಭಗ-ವ-ತಿ-ಯೊ-ಬ್ಬಳೆ
ಭಗ-ವತ್
ಭಗ-ವ-ತ್ಪ್ರೇ-ಮ-ದಿಂದ
ಭಗ-ವ-ದ-ಧೀನ
ಭಗ-ವ-ದ-ನ್ವೇ-ಷ-ಣೆ-ಯಲ್ಲಿ
ಭಗ-ವ-ದಾ-ರಾ-ಧ-ನೆ-ಯಲ್ಲಿ
ಭಗ-ವ-ದಿಚ್ಛೆ
ಭಗ-ವ-ದ್ಭ-ಕ್ತ-ನಾ-ದಾನು
ಭಗ-ವ-ದ್ಭಕ್ತಿ
ಭಗ-ವ-ದ್ಭ-ಕ್ತಿ-ಯಲ್ಲಿ
ಭಗ-ವ-ದ್ಭಾ-ವ-ದಲ್ಲಿ
ಭಗ-ವ-ದ್ಭಾ-ವ-ನೆ-ಯಲ್ಲಿ
ಭಗ-ವ-ದ್ರಾ-ಜ್ಯ-ವನ್ನು
ಭಗ-ವ-ನ್ನಾ-ಮದ
ಭಗ-ವ-ನ್ನಾ-ಮೋ-ಚ್ಚಾ-ರಣೆ
ಭಗ-ವ-ನ್ಮ-ಯಿ-ಯಾದ
ಭಗ-ವಾನ್
ಭಜನೆ
ಭಜಿ-ಸು-ತ್ತಿದ್ದೆ
ಭಟರು
ಭತ್ತ
ಭತ್ತ-ವನ್ನು
ಭದ್ರ-ವಾಗಿ
ಭನೆ
ಭಯ
ಭಯಂ-ಕ-ರ-ವಾದ
ಭಯ-ದಿಂದ
ಭಯ-ಪ-ಡ-ಬೇ-ಕಾ-ಗಿಲ್ಲ
ಭಯ-ಪ-ಡ-ಬೇಡ
ಭಯ-ಭ-ಕ್ತಿ-ಯಿಂದ
ಭಯ-ವಾಗಿ
ಭಯ-ವಾ-ಗು-ತ್ತಿತ್ತು
ಭಯ-ವಾ-ಯಿತು
ಭಯವೂ
ಭಯಾ-ನ-ಕ-ವಾದ
ಭರ-ದಲ್ಲಿ
ಭರ-ವ-ಸೆ-ಯನ್ನು
ಭರ್ತಿ-ಯಾ-ಗುವ
ಭವವೂ
ಭವ-ಸಾ-ಗ-ರ-ದಿಂದ
ಭವ-ಸಾ-ಗ-ರ-ವನ್ನು
ಭವಿ-ಷ್ಯ-ವನ್ನು
ಭಾಗ
ಭಾಗಕ್ಕೆ
ಭಾಗ-ಗಳನ್ನು
ಭಾಗ-ಗಳು
ಭಾಗ-ಗ-ಳುಮ
ಭಾಗ-ವತ
ಭಾಗ-ವ-ತ-ದಲ್ಲಿ
ಭಾಗ-ವ-ತ-ದ-ಲ್ಲಿ-ರುವ
ಭಾಗ-ವ-ತ-ವನ್ನು
ಭಾಗ-ವ-ತ-ವನ್ನೇ
ಭಾಗ-ವನ್ನು
ಭಾಗಿ
ಭಾಗ್ಯ
ಭಾರ
ಭಾರ-ತ-ದೇ-ಶದ
ಭಾರದ
ಭಾರ-ದಿಂದ
ಭಾರ-ವನ್ನು
ಭಾವ
ಭಾವದ
ಭಾವ-ದಲ್ಲಿ
ಭಾವ-ದಿಂದ
ಭಾವ-ನೆ-ಗಳನ್ನು
ಭಾವ-ನೆ-ಗ-ಳಿ-ವೆ-ಯೇನು
ಭಾವ-ನೆ-ಗಳು
ಭಾವ-ಮು-ಖ-ದ-ಲ್ಲಿ-ರು-ತ್ತಿದ್ದ
ಭಾವ-ಸ-ಮಾ-ಧಿ-ಯಲ್ಲಿ
ಭಾವಾ
ಭಾವಿ-ಸ-ತೊ-ಡ-ಗಿದ
ಭಾವಿ-ಸ-ಬಾ-ರದು
ಭಾವಿ-ಸ-ಬೇಡಿ
ಭಾವಿಸಿ
ಭಾವಿ-ಸಿತು
ಭಾವಿ-ಸಿತ್ತು
ಭಾವಿ-ಸಿದ
ಭಾವಿ-ಸಿ-ದನು
ಭಾವಿ-ಸಿ-ದರು
ಭಾವಿ-ಸಿ-ದರೆ
ಭಾವಿ-ಸಿ-ದಾಗ
ಭಾವಿ-ಸಿದೆ
ಭಾವಿ-ಸಿ-ದ್ದರು
ಭಾವಿ-ಸಿ-ರ-ಲಿಲ್ಲ
ಭಾವಿ-ಸು-ತ್ತಾನೆ
ಭಾವಿ-ಸು-ತ್ತಾರೆ
ಭಾವಿ-ಸು-ತ್ತಿದ್ದ
ಭಾವಿ-ಸು-ತ್ತೇವೆ
ಭಾವಿ-ಸು-ವನು
ಭಾವಿ-ಸು-ವು-ದಿಲ್ಲ
ಭಾವಿ-ಸು-ವುದು
ಭಾವಿ-ಸು-ವುವು
ಭಾಷೆ-ಯನ್ನೇ
ಭಾಷೆ-ಯಲ್ಲಿ
ಭಿಕ್ಷು-ಕ-ನಾ-ಗಿ-ರು-ವುದನ್ನು
ಭಿಕ್ಷು-ಕ-ನಿಗೆ
ಭಿಕ್ಷು-ಕ-ರಲ್ಲಿ
ಭಿಕ್ಷೆ
ಭಿಕ್ಷೆಗೆ
ಭಿಕ್ಷೆ-ಯನ್ನು
ಭೀತಿ-ಯಿಂದ
ಭೀಷ್ಮ
ಭೀಷ್ಮ-ನಂ-ತ-ಹ-ವರೆ
ಭೀಷ್ಮ-ನನ್ನೇ
ಭೀಷ್ಮಾ
ಭೀಷ್ಮಾ-ಚಾ-ರ್ಯನು
ಭುಜದ
ಭುಸು-ಗು-ಟ್ಟ-ಬೇಕು
ಭೂತ
ಭೂತಕ್ಕೆ
ಭೂತ-ವನ್ನು
ಭೂತ-ವಾ-ಗು-ತ್ತಾನೆ
ಭೂತವು
ಭೂಪ-ಟ-ದಲ್ಲಿ
ಭೂಮಿ
ಭೂಮಿ-ಯನ್ನು
ಭೇಟಿ-ಕಾ-ರನು
ಭೇದ-ವನ್ನು
ಭೇದ-ವನ್ನೂ
ಭೇದ-ವಿರು
ಭೈಟಕ್
ಭೋಗೇಚ್ಛೆ
ಭೋಜನ
ಭ್ರಮೆ
ಭ್ರಷ್ಟ
ಭ್ರಷ್ಟ-ನಾ-ಗು-ತ್ತಾನೆ
ಭ್ರಷ್ಟ-ನಾದ
ಭ್ರಾಂತ-ರಾ-ದರು
ಭ್ರಾಂತಿ
ಭ್ರಾಂತಿಯೆ
ಭ್ರಾಂತಿಯೇ
ಮಂಗಳ
ಮಂಚ-ವನ್ನು
ಮಂಡ-ಲಿ-ಯ-ವರು
ಮಂಡಿ-ಸು-ತ್ತಾನೆ
ಮಂಡೋ
ಮಂಡೋ-ದರಿ
ಮಂತ
ಮಂತ್ರ
ಮಂತ್ರ-ಗಳನ್ನು
ಮಂತ್ರ-ಗಾರ
ಮಂತ್ರ-ತಂ-ತ್ರ-ಗಳೂ
ಮಂತ್ರ-ದಂ-ಡ-ವನ್ನು
ಮಂತ್ರ-ವನ್ನು
ಮಂತ್ರ-ವಾದಿ
ಮಂತ್ರ-ಶಕ್ತಿ
ಮಂತ್ರಿ-ಯೊ-ಡನೆ
ಮಂತ್ರಿಸಿ
ಮಂತ್ರಿ-ಸಿದ
ಮಂತ್ರೋ-ಪ-ದೇಶ
ಮಂತ್ರೋ-ಪ-ದೇ-ಶ-ವನ್ನು
ಮಂದ
ಮಂದ-ಹಾ-ಸ-ದಿಂದ
ಮಂದಿ-ರಕ್ಕೆ
ಮಕ-ರಂ-ದ-ವನ್ನು
ಮಕ್ಕ
ಮಕ್ಕಳ
ಮಕ್ಕ-ಳಂತೆ
ಮಕ್ಕ-ಳನ್ನು
ಮಕ್ಕ-ಳಾ-ದವು
ಮಕ್ಕ-ಳಿ-ಗಾಗಿ
ಮಕ್ಕ-ಳಿಗೆ
ಮಕ್ಕ-ಳಿ-ದ್ದರು
ಮಕ್ಕ-ಳಿ-ದ್ದಾರೆ
ಮಕ್ಕ-ಳಿಲ್ಲ
ಮಕ್ಕಳು
ಮಕ್ಕಳೆ
ಮಕ್ಕ-ಳೆಲ್ಲ
ಮಕ್ಕ-ಳೊಂ-ದಿಗೆ
ಮಗ
ಮಗನ
ಮಗ-ನಂತೆ
ಮಗ-ನನ್ನು
ಮಗ-ನಾದ
ಮಗ-ನಿಗೆ
ಮಗನು
ಮಗ-ನೊ-ಡನೆ
ಮಗಳ
ಮಗ-ಳಂತೆ
ಮಗಳನ್ನು
ಮಗ-ಳಾಗಿ
ಮಗ-ಳಿಗೂ
ಮಗಳು
ಮಗು
ಮಗು-ಆ-ಯಿತು
ಮಗು-ವನ್ನು
ಮಗು-ವಾ-ಗ-ಲಿದೆ
ಮಗು-ವಿ-ಗಾಗಿ
ಮಗು-ವಿಗೆ
ಮಗು-ವಿ-ನಂ-ತಹ
ಮಗು-ವಿ-ನಂತೆ
ಮಗು-ವಿ-ನಂಥ
ಮಗು-ವಿ-ನ-ಲ್ಲಿ-ರುವ
ಮಗು-ವಿ-ನ-ಲ್ಲಿ-ರು-ವಂ-ತಹ
ಮಗುವು
ಮಗುವೆ
ಮಗ್ನ-ನಾ-ಗಿದ್ದ
ಮಗ್ನ-ನಾ-ದನು
ಮಗ್ನ-ವಾಗು
ಮಜ
ಮಜುಂ-ದಾ-ರ-ನಿಗೆ
ಮಜುಂ-ದಾ-ರ-ಹೌದು
ಮಟ್ಟ-ದ-ಲ್ಲಿ-ದ್ದರು
ಮಟ್ಟ-ದ-ಲ್ಲಿ-ದ್ದ-ರೆಂ-ದರೆ
ಮಠ
ಮಠಕ್ಕೆ
ಮಡಚಿ
ಮಡಿ-ಸಿ-ಕೊಂಡೇ
ಮಡು
ಮಣ್ಣನ್ನು
ಮಣ್ಣಿಗೆ
ಮಣ್ಣಿನ
ಮಣ್ಣಿ-ನಿಂದ
ಮಣ್ಣು
ಮತ-ಪಂ-ಗ-ಡ-ಗಳ
ಮತಾಂ-ತ-ರಿ-ಸಿ-ದರು
ಮತಾಂ-ಧತೆ
ಮತಾಂ-ಧ-ತೆಗೆ
ಮತಾಂ-ಧ-ರಾ-ಗ-ಬೇಡಿ
ಮತಿ
ಮತ್ತಷ್ಟು
ಮತ್ತಾ-ರಿಗೂ
ಮತ್ತಾ-ರಿಗೋ
ಮತ್ತಾ-ರೊ-ಡನೆ
ಮತ್ತು
ಮತ್ತು-ಪಂ-ಜಾ-ಬಿನ
ಮತ್ತೂ
ಮತ್ತೆ
ಮತ್ತೆಲ್ಲೂ
ಮತ್ತೊಂ-ದನ್ನು
ಮತ್ತೊಂದು
ಮತ್ತೊಬ್ಬ
ಮತ್ತೊ-ಬ್ಬ-ನಂತೆ
ಮತ್ತೊ-ಬ್ಬ-ನಿಗೆ
ಮತ್ತೊ-ಬ್ಬನು
ಮತ್ತೊ-ಬ್ಬರ
ಮತ್ತೊ-ಬ್ಬ-ರಿಗೆ
ಮತ್ತೊಮ್ಮೆ
ಮತ್ಸ್ಯದ
ಮಥುರ
ಮಥು-ರನ
ಮಥು-ರ-ನಾ-ಥ-ನಿಗೆ
ಮಥು-ರ-ನಿಗೆ
ಮಥು-ರನು
ಮಥು-ರ-ಬಾಬು
ಮಥು-ರ-ಬಾ-ಬು-ವಿ-ನೊಂ-ದಿಗೆ
ಮದ
ಮದುವೆ
ಮದು-ವೆ-ಮಾ-ಡಿ-ಕೊ-ಳ್ಳು-ತ್ತಿ-ದ್ದರು
ಮದು-ವೆಯ
ಮದು-ವೆ-ಯಾಗಿ
ಮದು-ವೆ-ಯಾ-ಗಿದ್ದ
ಮದು-ವೆ-ಯಾ-ಗುವ
ಮದು-ವೆ-ಯಾ-ಗು-ವು-ದಿಲ್ಲ
ಮದು-ವೆ-ಯಾದ
ಮದು-ವೆ-ಯಾ-ದರು
ಮದು-ವೆ-ಯಾ-ದರೆ
ಮದ್ದಾನೆ
ಮದ್ದಿಲ್ಲ
ಮದ್ಯ
ಮದ್ಯ-ವನ್ನು
ಮಧು-ಸೂ-ದನ
ಮಧು-ಸೂ-ದ-ನ-ನನ್ನು
ಮಧ್ಯ
ಮಧ್ಯ-ದಲ್ಲಿ
ಮಧ್ಯಾಹ್ನ
ಮಧ್ಯಾ-ಹ್ನ-ವಾ-ಯಿತು
ಮಧ್ಯೆ
ಮಧ್ಯೆ-ಮಧ್ಯೆ
ಮನಃ
ಮನ-ಗಂಡು
ಮನ-ದ-ಣಿಯೆ
ಮನ-ಮು-ಟ್ಟು-ವಂತೆ
ಮನ-ರಂ-ಜನೆ
ಮನ-ಸೋ-ತರು
ಮನ-ಸ್ಸನ್ನು
ಮನ-ಸ್ಸನ್ನೂ
ಮನ-ಸ್ಸ-ನ್ನೆಲ್ಲ
ಮನ-ಸ್ಸಿಗೆ
ಮನ-ಸ್ಸಿನ
ಮನ-ಸ್ಸಿ-ನಲ್ಲಿ
ಮನ-ಸ್ಸಿ-ನ-ಲ್ಲಿದ್ದ
ಮನ-ಸ್ಸಿ-ನ-ಲ್ಲಿ-ರುವ
ಮನ-ಸ್ಸಿ-ನಲ್ಲೇ
ಮನಸ್ಸು
ಮನ-ಸ್ಸೆಲ್ಲ
ಮನಸ್ಸೇ
ಮನುಷ್ಯ
ಮನು-ಷ್ಯನ
ಮನು-ಷ್ಯ-ನಂತೆ
ಮನು-ಷ್ಯ-ನನ್ನು
ಮನು-ಷ್ಯ-ನಲ್ಲಿ
ಮನು-ಷ್ಯ-ನಾ-ಗಿದ್ದ
ಮನು-ಷ್ಯ-ನಿಂದ
ಮನು-ಷ್ಯ-ನಿಗೂ
ಮನು-ಷ್ಯ-ನಿಗೆ
ಮನು-ಷ್ಯನು
ಮನು-ಷ್ಯ-ರಂತೆ
ಮನು-ಷ್ಯ-ರಲ್ಲಿ
ಮನು-ಷ್ಯರು
ಮನು-ಷ್ಯ-ಳಾ-ಗು-ತ್ತೀಯೆ
ಮನು-ಷ್ಯ-ಸಾ-ಧ್ಯ-ವಲ್ಲ
ಮನು-ಷ್ಯಾ-ಕಾ-ರದ
ಮನು-ಷ್ಯಾ-ಕೃತಿ
ಮನೆ
ಮನೆ-ಗಳಲ್ಲಿ
ಮನೆ-ಗಳು
ಮನೆ-ಗ-ಳೆಲ್ಲ
ಮನೆಗೆ
ಮನೆ-ಬಿಟ್ಟು
ಮನೆಯ
ಮನೆ-ಯನ್ನು
ಮನೆ-ಯಲ್ಲಿ
ಮನೆ-ಯ-ಲ್ಲಿದ್ದ
ಮನೆ-ಯ-ಲ್ಲಿ-ದ್ದ-ವರೆಲ್ಲ
ಮನೆ-ಯ-ಲ್ಲಿ-ರ-ಲಿಲ್ಲ
ಮನೆ-ಯ-ಲ್ಲಿ-ರುವ
ಮನೆ-ಯ-ಲ್ಲಿ-ರು-ವ-ವರೆಲ್ಲ
ಮನೆ-ಯ-ವನು
ಮನೆ-ಯ-ವ-ನೊ-ಬ್ಬನು
ಮನೆ-ಯ-ವ-ರ-ನ್ನೆಲ್ಲ
ಮನೆ-ಯ-ವ-ರಿಗೆ
ಮನೆ-ಯ-ವರು
ಮನೆ-ಯಿಂದ
ಮನೆಯು
ಮನೆಯೂ
ಮನೆ-ಯೆ-ದು-ರಿಗೆ
ಮನೆ-ಯೊ-ಡೆಯ
ಮನೆ-ಯೊ-ಳಗೆ
ಮನೋ-ಭ್ರಾಂತಿ
ಮನ್ನ-ಣೆ-ಯನ್ನು
ಮಮತೆ
ಮರ
ಮರಕ್ಕೆ
ಮರ-ಗಳನ್ನೂ
ಮರ-ಗಳು
ಮರಣ
ಮರ-ಣ-ಗಳು
ಮರ-ಣ-ದಿಂದ
ಮರದ
ಮರ-ದಂತೆ
ಮರ-ದಡಿ
ಮರ-ದ-ಮೇಲೆ
ಮರ-ದಾಚೆ
ಮರ-ದಿಂದ
ಮರ-ಳಿದ
ಮರ-ವನ್ನು
ಮರ-ವ-ನ್ನೇರಿ
ಮರ-ವ-ನ್ನೇ-ರೋಣ
ಮರಿ
ಮರಿ-ಗ-ಳಿಗೆ
ಮರಿ-ಗ-ಳೊಂ-ದಿಗೆ
ಮರಿಗೆ
ಮರಿ-ಯನ್ನು
ಮರಿಯೂ
ಮರು-ಕ-ಳಿ-ಸು-ವುದು
ಮರು-ಕ್ಷ-ಣವೆ
ಮರು-ಕ್ಷ-ಣವೇ
ಮರೆ-ತಿ-ರು-ತ್ತಾರೆ
ಮರೆ-ತಿ-ರು-ವೆನು
ಮರೆತು
ಮರೆ-ತು-ಬಿ-ಟ್ಟಿತ್ತು
ಮರೆ-ತು-ಬಿ-ಟ್ಟಿ-ರುವೆ
ಮರೆ-ತು-ಬಿಟ್ಟೆ
ಮರೆ-ಮಾ-ಚಿ-ಕೊಂಡು
ಮರೆ-ಯ-ಬಾ-ರದು
ಮರೆ-ಯ-ಬೇಡ
ಮರೆ-ಯಾ-ದಾಗ
ಮರೆ-ಯು-ತಿದ್ದ
ಮರ್ಮ-ವಿ-ರ-ಬೇಕು
ಮಲ-ಗಿ-ಕೊಂಡ
ಮಲ-ಗಿ-ಕೊ-ಳ್ಳು-ತ್ತೇನೆ
ಮಲ-ಗಿ-ಕೊ-ಳ್ಳು-ವು-ದಕ್ಕೆ
ಮಲ-ಗಿದ
ಮಲ-ಗಿ-ದಾಗ
ಮಲ-ಗಿದ್ದ
ಮಲ-ಗಿ-ದ್ದರು
ಮಲ-ಗಿ-ದ್ದಾಗ
ಮಲ-ಗಿ-ರು-ವನು
ಮಲ-ಗಿ-ರು-ವೆನು
ಮಲ-ಗು-ತ್ತಿದ್ದ
ಮಲೇ-ರಿಯಾ
ಮಳೆ
ಮಳೆಯ
ಮಳೆ-ಯಿಲ್ಲ
ಮಹ-ನೀ-ಯರೆ
ಮಹ-ಮ್ಮ-ದೀಯ
ಮಹ-ಮ್ಮ-ದೀ-ಯ-ನಾದ
ಮಹ-ಮ್ಮ-ದೀ-ಯ-ನಾ-ದ-ವನು
ಮಹ-ಮ್ಮ-ದೀ-ಯರು
ಮಹಾ
ಮಹಾ-ತ್ಮ-ರಂ-ತಹ
ಮಹಾ-ತ್ಮೆ-ಯಿಂದ
ಮಹಾ-ಪಾ-ತಕಿ
ಮಹಾ-ಪಾಪಿ
ಮಹಾ-ಪು-ರು-ಷನ
ಮಹಾ-ಭ-ಕ್ತ-ನೊ-ಬ್ಬ-ನಿ-ರು-ವನು
ಮಹಾ-ಮಾಯೆ
ಮಹಾ-ಮಾ-ಯೆಯ
ಮಹಾ-ಮಾ-ಯೆ-ಯಂತೆ
ಮಹಾ-ಮಾ-ಯೆ-ಯನ್ನು
ಮಹಾ-ರಾ-ಜರು
ಮಹಾ-ರಾ-ಜರೆ
ಮಹಾ-ವೀರ
ಮಹಾ-ವೀ-ರನ
ಮಹಾ-ಶಯ
ಮಹಾ-ಶ-ಯರೆ
ಮಹಾ-ಶ-ಯರೇ
ಮಹಾ-ಸ್ವಾಮಿ
ಮಹಾ-ಸ್ವಾ-ಮಿ-ಗಳು
ಮಹಿಮೆ
ಮಹಿ-ಮೆಗೆ
ಮಹಿ-ಮೆ-ಯನ್ನು
ಮಹಿ-ಮೆ-ಯಿಂದ
ಮಹಿಳೆ
ಮಹಿ-ಳೆ-ಯ-ರೆಲ್ಲ
ಮಹೇಂದ್ರ
ಮಾಂತ್ರಿಕ
ಮಾಂಸ
ಮಾಂಸ-ವನ್ನು
ಮಾಂಸ-ವಿದೆ
ಮಾಗೂರ್
ಮಾಟ-ಗಾ-ರನ
ಮಾಡ
ಮಾಡ-ಕೂ-ಡದು
ಮಾಡ-ದಂತೆ
ಮಾಡ-ದಿ-ರಲಿ
ಮಾಡದೆ
ಮಾಡ-ಬಲ್ಲ
ಮಾಡ-ಬ-ಲ್ಲದು
ಮಾಡ-ಬ-ಹುದು
ಮಾಡ-ಬೇ-ಕಾ-ಗಿತ್ತು
ಮಾಡ-ಬೇ-ಕಾ-ಗಿದೆ
ಮಾಡ-ಬೇ-ಕಾ-ದರೆ
ಮಾಡ-ಬೇ-ಕಾ-ಯಿತು
ಮಾಡ-ಬೇಕು
ಮಾಡ-ಬೇ-ಕೆಂ-ದಿ-ರು-ವರೋ
ಮಾಡ-ಬೇ-ಕೆಂ-ದಿ-ರು-ವೆನು
ಮಾಡ-ಬೇ-ಕೆಂದು
ಮಾಡ-ಬೇಡ
ಮಾಡ-ಬೇಡಿ
ಮಾಡ-ಲಾ-ಗ-ಲಿಲ್ಲ
ಮಾಡ-ಲಾರ
ಮಾಡ-ಲಾ-ರಿರಿ
ಮಾಡ-ಲಾರೆ
ಮಾಡ-ಲಾ-ರೆಯ
ಮಾಡಲಿ
ಮಾಡ-ಲಿಲ್ಲ
ಮಾಡಲು
ಮಾಡ-ಲೆ-ತ್ನಿ-ಸಿದ
ಮಾಡ-ಲೇ-ಬೇ-ಕಾ-ಗಿ-ರು-ವುದು
ಮಾಡಿ
ಮಾಡಿ-ಕೊಂಡ
ಮಾಡಿ-ಕೊಂ-ಡನು
ಮಾಡಿ-ಕೊಂ-ಡರೂ
ಮಾಡಿ-ಕೊಂ-ಡರೆ
ಮಾಡಿ-ಕೊಂ-ಡಳು
ಮಾಡಿ-ಕೊಂ-ಡಿ-ದ್ದಾಳೆ
ಮಾಡಿ-ಕೊಂ-ಡಿ-ದ್ದೇನೆ
ಮಾಡಿ-ಕೊಂ-ಡಿಲ್ಲ
ಮಾಡಿ-ಕೊಂಡು
ಮಾಡಿ-ಕೊಳ್ಳ
ಮಾಡಿ-ಕೊ-ಳ್ಳ-ಬ-ಹುದು
ಮಾಡಿ-ಕೊ-ಳ್ಳ-ಬೇಕು
ಮಾಡಿ-ಕೊ-ಳ್ಳ-ಬೇಡ
ಮಾಡಿ-ಕೊ-ಳ್ಳ-ಲಿಲ್ಲ
ಮಾಡಿ-ಕೊಳ್ಳಿ
ಮಾಡಿ-ಕೊ-ಳ್ಳು-ತ್ತೇನೆ
ಮಾಡಿತು
ಮಾಡಿತೆ
ಮಾಡಿದ
ಮಾಡಿ-ದನು
ಮಾಡಿ-ದರು
ಮಾಡಿ-ದರೂ
ಮಾಡಿ-ದರೆ
ಮಾಡಿ-ದಳು
ಮಾಡಿ-ದಷ್ಟೂ
ಮಾಡಿ-ದಾಗ
ಮಾಡಿ-ದಿರಿ
ಮಾಡಿ-ದು-ದಾ-ದರೂ
ಮಾಡಿದೆ
ಮಾಡಿ-ದೊ-ಡನೆ
ಮಾಡಿದ್ದ
ಮಾಡಿ-ದ್ದ-ರಿಂದ
ಮಾಡಿ-ದ್ದರೆ
ಮಾಡಿ-ದ್ದಾನೆ
ಮಾಡಿ-ದ್ದೀಯೊ
ಮಾಡಿದ್ದು
ಮಾಡಿದ್ದೆ
ಮಾಡಿ-ಬಿ-ಡು-ತ್ತಿತ್ತು
ಮಾಡಿ-ಯಾದ
ಮಾಡಿ-ರ-ಬ-ಹುದು
ಮಾಡಿ-ರ-ಲಿಲ್ಲ
ಮಾಡಿ-ರು-ತ್ತಾರೆ
ಮಾಡಿ-ರುವ
ಮಾಡಿ-ರು-ವನು
ಮಾಡಿ-ರು-ವರೊ
ಮಾಡಿ-ರುವೆ
ಮಾಡಿಲ್ಲ
ಮಾಡಿ-ಸಲು
ಮಾಡಿಸಿ
ಮಾಡಿ-ಸಿ-ಕೊ-ಳ್ಳು-ತ್ತಿ-ದ್ದ-ವ-ನಿಗೆ
ಮಾಡಿ-ಸಿದ
ಮಾಡು
ಮಾಡುತ್ತ
ಮಾಡು-ತ್ತದೆ
ಮಾಡು-ತ್ತಲೇ
ಮಾಡು-ತ್ತಾನೆ
ಮಾಡು-ತ್ತಾನೊ
ಮಾಡು-ತ್ತಾರೆ
ಮಾಡು-ತ್ತಾ-ರೆಯೆ
ಮಾಡು-ತ್ತಾರೊ
ಮಾಡುತ್ತಿ
ಮಾಡು-ತ್ತಿದ್ದ
ಮಾಡು-ತ್ತಿ-ದ್ದ-ನಾದ್ದ
ಮಾಡು-ತ್ತಿ-ದ್ದನು
ಮಾಡು-ತ್ತಿ-ದ್ದರು
ಮಾಡು-ತ್ತಿ-ದ್ದಳು
ಮಾಡು-ತ್ತಿ-ದ್ದ-ವನೇ
ಮಾಡು-ತ್ತಿ-ದ್ದ-ವರ
ಮಾಡು-ತ್ತಿ-ದ್ದವು
ಮಾಡು-ತ್ತಿ-ದ್ದಾಗ
ಮಾಡು-ತ್ತಿ-ದ್ದೀರಿ
ಮಾಡು-ತ್ತಿ-ರ-ಬ-ಹುದು
ಮಾಡು-ತ್ತಿರು
ಮಾಡು-ತ್ತಿ-ರು-ತ್ತಾಳೆ
ಮಾಡು-ತ್ತಿ-ರುವ
ಮಾಡು-ತ್ತಿ-ರು-ವನು
ಮಾಡು-ತ್ತಿ-ರು-ವಾಗ
ಮಾಡು-ತ್ತಿ-ರು-ವಿರಿ
ಮಾಡು-ತ್ತಿ-ರು-ವುದನ್ನು
ಮಾಡು-ತ್ತಿ-ರು-ವುದು
ಮಾಡು-ತ್ತಿ-ರುವೆ
ಮಾಡು-ತ್ತೇನೆ
ಮಾಡುವ
ಮಾಡು-ವಂ-ತಿಲ್ಲ
ಮಾಡು-ವನು
ಮಾಡು-ವಲ್ಲಿ
ಮಾಡು-ವ-ವನು
ಮಾಡು-ವಾಗ
ಮಾಡು-ವಾ-ಗಲೂ
ಮಾಡು-ವಿರಿ
ಮಾಡು-ವು-ದಕ್ಕೆ
ಮಾಡು-ವುದನ್ನು
ಮಾಡು-ವು-ದ-ರಿಂದ
ಮಾಡು-ವು-ದಾ-ಗಿ-ದ್ದರೆ
ಮಾಡು-ವು-ದಿಲ್ಲ
ಮಾಡು-ವುದು
ಮಾಡುವೆ
ಮಾಡು-ವೆಯೊ
ಮಾಣಿ-ಕ್ಯ-ದಂತೆ
ಮಾತ
ಮಾತ-ನಾಡ
ಮಾತ-ನಾ-ಡ-ಬೇಡ
ಮಾತ-ನಾ-ಡಿ-ದರೆ
ಮಾತ-ನಾ-ಡಿದೆ
ಮಾತ-ನಾ-ಡಿ-ದೆವು
ಮಾತ-ನಾ-ಡಿ-ಸಿ-ದಳು
ಮಾತ-ನಾ-ಡು-ತ್ತಾರೆ
ಮಾತ-ನಾ-ಡು-ತ್ತಿದ್ದ
ಮಾತ-ನಾ-ಡು-ತ್ತಿ-ದ್ದರು
ಮಾತ-ನಾ-ಡು-ತ್ತಿ-ದ್ದಾಗ
ಮಾತ-ನಾ-ಡು-ತ್ತಿರ
ಮಾತ-ನಾ-ಡು-ತ್ತಿ-ರ-ಲಿಲ್ಲ
ಮಾತ-ನಾ-ಡು-ತ್ತಿ-ರು-ವಾಗ
ಮಾತ-ನಾ-ಡು-ವು-ದ-ಕ್ಕಿಂತ
ಮಾತನ್ನು
ಮಾತನ್ನೂ
ಮಾತಿ
ಮಾತಿಗೆ
ಮಾತಿನ
ಮಾತಿ-ನಲ್ಲಿ
ಮಾತಿ-ನಿಂದ
ಮಾತಿ-ಲ್ಲದೆ
ಮಾತು
ಮಾತು-ಕತೆ
ಮಾತು-ಗಳನ್ನೆಲ್ಲ
ಮಾತು-ಗಳು
ಮಾತೂ
ಮಾತೃ-ಸ್ವ-ರೂ-ಪರು
ಮಾತ್ರ
ಮಾತ್ರ-ವಲ್ಲ
ಮಾತ್ರೆ-ಯನ್ನು
ಮಾನ-ಮ-ರ್ಯಾದೆ
ಮಾನವ
ಮಾನ-ವ-ಜ-ನ್ಮದ
ಮಾನ-ವ-ನಿಗೆ
ಮಾನ-ವನು
ಮಾನ-ಸಿ-ಕ-ವಾಗಿ
ಮಾನ್ಯ
ಮಾಯ-ವಾ-ಗಿತ್ತು
ಮಾಯ-ವಾ-ಗು-ತ್ತದೆ
ಮಾಯ-ವಾ-ಗು-ತ್ತವೆ
ಮಾಯ-ವಾ-ಗು-ವುದು
ಮಾಯ-ವಾದ
ಮಾಯ-ವಾ-ದಳು
ಮಾಯ-ವಾ-ಯಿತು
ಮಾಯಾ
ಮಾಯಾ-ಮಂತ್ರ
ಮಾಯಾ-ಮಂ-ತ್ರ-ಗಳನ್ನು
ಮಾಯಾ-ಶ-ಕ್ತಿಗೂ
ಮಾಯಾ-ಶ-ಕ್ತಿ-ಯನ್ನು
ಮಾಯಾ-ಶ-ಕ್ತಿ-ಯಿಂದ
ಮಾಯೆ
ಮಾಯೆಗೆ
ಮಾಯೆಯ
ಮಾಯೆ-ಯಂತೆ
ಮಾಯೆ-ಯನ್ನು
ಮಾಯೆ-ಯಲ್ಲಿ
ಮಾಯೆ-ಯಿಂದ
ಮಾಯೆಯೇ
ಮಾರನೆ
ಮಾರ-ನೆಯ
ಮಾರನೇ
ಮಾರ-ಬೇ-ಕೆಂದು
ಮಾರ-ವಾಡಿ
ಮಾರಿ
ಮಾರಿ-ಕೊಂ-ಡಂತೆ
ಮಾರಿದ
ಮಾರಿದೆ
ಮಾರಿ-ಯಾದ
ಮಾರು
ಮಾರು-ತ್ತಿ-ರುವ
ಮಾರು-ತ್ತೇನೆ
ಮಾರು-ವ-ವನ
ಮಾರು-ವ-ವ-ನಿದ್ದ
ಮಾರು-ವ-ವನು
ಮಾರು-ವ-ವರು
ಮಾರು-ವ-ವಳು
ಮಾರ್ಗ
ಮಾರ್ಗದ
ಮಾರ್ಗ-ದಲ್ಲಿ
ಮಾರ್ಗ-ವಾಗಿ
ಮಾಲನ್ನು
ಮಾಲಿ-ಕನ
ಮಾವ
ಮಾವನ
ಮಾವಿನ
ಮಾಹುತ
ಮಾಹು-ತ-ದೇ-ವರ
ಮಾಹು-ತ-ನ-ಲ್ಲಿ-ರುವ
ಮಾಹು-ತ-ನಲ್ಲೂ
ಮಿಂಚನ್ನು
ಮಿಂಚು
ಮಿಕ್ಕಿ-ದೆ-ಲ್ಲವೂ
ಮಿಕ್ಕಿ-ರು-ವು-ದಕ್ಕೂ
ಮಿಠಾಯಿ
ಮಿತ್ರ-ನಾದ
ಮಿತ್ರ-ರಿ-ಬ್ಬ-ರನ್ನೂ
ಮಿಥ್ಯ
ಮಿಥ್ಯ-ವೆಂದು
ಮಿಥ್ಯೆ
ಮಿಶ್ರ-ಮಿ-ಸಿ-ಕೊ-ಳ್ಳು-ವು-ದಕ್ಕೆ
ಮೀನನ್ನು
ಮೀನ-ನ್ನೆಲ್ಲ
ಮೀನಿಗೆ
ಮೀನಿನ
ಮೀನು
ಮೀನು-ಗಳು
ಮೀನು-ಗ-ಳೆಲ್ಲ
ಮೀನೇ
ಮೀರಿ
ಮೀರಿದ
ಮೀರಿ-ಹೋ-ಗಲು
ಮೀಸೆ
ಮುಂಗುಸಿ
ಮುಂಗು-ಸಿಯ
ಮುಂಚೆ
ಮುಂಚೆಯೆ
ಮುಂಚೆಯೇ
ಮುಂತಾ
ಮುಂತಾಗಿ
ಮುಂತಾದ
ಮುಂತಾ-ದ-ವನ್ನು
ಮುಂತಾ-ದ-ವು-ಗಳು
ಮುಂತಾ-ದು-ವು-ಗಳ
ಮುಂದಕ್ಕೆ
ಮುಂದಾ
ಮುಂದು
ಮುಂದು-ಗಡೆ
ಮುಂದು-ವ-ರಿ-ದಂತೆ
ಮುಂದು-ವ-ರಿ-ದರೆ
ಮುಂದು-ವ-ರಿ-ದೆಯೆ
ಮುಂದೆ
ಮುಂಬಾ-ಗಿ-ಲಲ್ಲಿ
ಮುಕ್ಕಾಲು
ಮುಕ್ಕಾ-ಲು-ಪಾಲು
ಮುಕ್ತ
ಮುಕ್ತರು
ಮುಕ್ತ-ಳಾ-ಗ-ಲಿಲ್ಲ
ಮುಕ್ತ-ವಾ-ಗ-ಲಾ-ರವು
ಮುಕ್ತಿ-ಯನ್ನು
ಮುಖ
ಮುಖದ
ಮುಖ-ಪು-ಟದ
ಮುಖರ್ಜಿ
ಮುಖ-ವನ್ನು
ಮುಗಿದ
ಮುಗಿ-ದು-ಹೋ-ಗಿತ್ತು
ಮುಗಿ-ಯತು
ಮುಗಿ-ಯಿತು
ಮುಗಿ-ಸ-ಬ-ಹುದು
ಮುಗು-ಳ್ನಕ್ಕ
ಮುಗ್ಗ-ರಿಸಿ
ಮುಗ್ಧ
ಮುಚ್ಚ
ಮುಚ್ಚ-ಳ-ವನ್ನು
ಮುಚ್ಚಿ
ಮುಚ್ಚಿ-ಕೊಂಡು
ಮುಚ್ಚಿ-ಕೊ-ಳ್ಳು-ವರು
ಮುಚ್ಚಿ-ಟ್ಟು-ಕೊಂ-ಡಿ-ದ್ದನು
ಮುಚ್ಚಿ-ದಾ-ಗ-ಲೆಲ್ಲ
ಮುಚ್ಚಿ-ದೊ-ಡ-ನೆಯೆ
ಮುಚ್ಚಿ-ಬಿ-ಟ್ಟಿರಾ
ಮುಚ್ಚಿ-ರು-ತ್ತಾರೆ
ಮುಚ್ಚು-ವು-ದಕ್ಕೂ
ಮುಟ್ಟದ
ಮುಟ್ಟ-ಬೇಡ
ಮುಟ್ಟ-ಲಾರೆ
ಮುಟ್ಟಲು
ಮುಟ್ಟಲೂ
ಮುಟ್ಟಿ
ಮುಟ್ಟಿದ
ಮುಟ್ಟಿ-ದರೆ
ಮುಟ್ಟಿಲ್ಲ
ಮುಟ್ಟಿ-ಸಿ-ದರು
ಮುಟ್ಟು-ವಂತೆ
ಮುಟ್ಟು-ವು-ದಕ್ಕೆ
ಮುಟ್ಟುವೆ
ಮುತ್ತನ್ನು
ಮುತ್ತ-ಲಿ-ರುವ
ಮುತ್ತು
ಮುತ್ತು-ಗಳನ್ನು
ಮುತ್ತು-ರ-ತ್ನ-ಗಳನ್ನು
ಮುದುಕ
ಮುದು-ಕ-ನನ್ನು
ಮುದು-ಕಿ-ಯಾ-ಗಿ-ದ್ದಾಳೆ
ಮುದ್ದಾ-ಡಿ-ಸು-ವುದೇ
ಮುದ್ದಿ-ಸ-ಬೇಡ
ಮುದ್ದಿ-ಸಲಿ
ಮುದ್ದಿ-ಸು-ತ್ತಿದ್ದ
ಮುದ್ದಿ-ಸು-ತ್ತಿ-ದ್ದೆಯೋ
ಮುದ್ದಿ-ಸು-ವು-ದಿಲ್ಲ
ಮುದ್ದೆ
ಮುದ್ದೆ-ಯನ್ನು
ಮುದ್ದೆ-ಯ-ನ್ನೆಲ್ಲ
ಮುದ್ರೆ
ಮುದ್ರೆ-ಗಳನ್ನು
ಮುದ್ರೆ-ಗಳು
ಮುನ್ನ-ಡೆ-ಯಿರಿ
ಮುನ್ನ-ಡೆ-ಯುವೆ
ಮುನ್ನುಡಿ
ಮುನ್ನೂರು
ಮುನ್ನೆ
ಮುನ್ನೆ-ಚ್ಚ-ರಿ-ಕೆ-ಯನ್ನು
ಮುಮು-ಕ್ಷು-ಗಳು
ಮುರಿದು
ಮುರಿ-ಯುವೆ
ಮುರು-ಕಲು
ಮುಳು
ಮುಳು-ಗ-ಬೇಕು
ಮುಳು-ಗಲು
ಮುಳುಗಿ
ಮುಳು-ಗಿತು
ಮುಳು-ಗಿದ
ಮುಳು-ಗಿ-ದರೆ
ಮುಳು-ಗಿರು
ಮುಳು-ಗಿ-ರುವ
ಮುಳು-ಗಿ-ರು-ವನು
ಮುಳು-ಗಿ-ರು-ವರೊ
ಮುಳು-ಗಿ-ರು-ವಾ-ಗಲೂ
ಮುಳು-ಗಿಸಿ
ಮುಳುಗು
ಮುಳು-ಗುವ
ಮುಳು-ಗು-ವಂತೆ
ಮುಳು-ಗು-ವು-ದ-ರ-ಲ್ಲಿತ್ತು
ಮುಳು-ಗುವೆ
ಮುಳ್ಳನ್ನು
ಮುಳ್ಳಿ
ಮುಳ್ಳಿಲ್ಲ
ಮುಳ್ಳು
ಮುಳ್ಳೂ
ಮುಷ್ಟಿ-ಯನ್ನು
ಮುಸ-ಲ್ಮಾನ
ಮುಸ-ಲ್ಮಾ-ನನು
ಮುಸ-ಲ್ಮಾ-ನರು
ಮುಸು-ಕು-ಗಳು
ಮೂಕ
ಮೂಕ-ನಾಗಿ
ಮೂಕ-ನಾ-ಗು-ತ್ತಾನೆ
ಮೂಗಿಗೆ
ಮೂಗಿನ
ಮೂಗು
ಮೂಟೆ-ಯನ್ನೇ
ಮೂಡಿ-ಬಂ-ದಿದೆ
ಮೂಢ
ಮೂಢ-ಏ-ನನ್ನು
ಮೂಢ-ನಂ-ತಿರ
ಮೂದ-ಲಿ-ಸಿ-ದಳು
ಮೂರನೆ
ಮೂರ-ನೆಯ
ಮೂರ-ನೆ-ಯ-ವ-ನನ್ನು
ಮೂರ-ನೆ-ಯ-ವನು
ಮೂರನ್ನೂ
ಮೂರು
ಮೂರ್ಖ
ಮೂರ್ತಿಯ
ಮೂರ್ತಿ-ಯನ್ನು
ಮೂಲ
ಮೂಲಕ
ಮೂಲ-ಕಾ-ರಣ
ಮೂಲ-ದಲಿ
ಮೂಲ-ಶ್ರದ್ಧೆ
ಮೂಲಿ-ಕೆ-ಗಳನ್ನು
ಮೂಲೆ-ಯಲ್ಲಿ
ಮೂಳೆ-ಗ-ಳೆಲ್ಲ
ಮೂಳೆ-ಚ-ಕ್ಕಳ
ಮೂವತ್ತು
ಮೂವ-ತ್ತೈದು
ಮೃಗ-ವನ್ನು
ಮೃಣ್ಮಯಿ
ಮೃಣ್ಮ-ಯಿ-ಯನ್ನು
ಮೃತ್ಯು-ವ-ಶ-ಳಾ-ದಳು
ಮೃತ್ಯು-ವಿನ
ಮೃಷ್ಟಾನ್ನ
ಮೆಕ್ಕಲು
ಮೆಚ್ಚಿ-ಗೆ-ಯಾ-ಯಿತು
ಮೆಚ್ಚಿದ
ಮೆಚ್ಚಿ-ರು-ವೆನು
ಮೆತ್ತ-ಗಾಗಿ
ಮೆತ್ತ-ನೆಯ
ಮೆರ-ವ-ಣಿಗೆ
ಮೆರೆಯು
ಮೆರೆ-ಯು-ತ್ತಿದ್ದ
ಮೆಲುಕು
ಮೆಲು-ಕು-ತ್ತಿ-ದ್ದರೆ
ಮೆಲ್ಲು-ತ್ತಿ-ರು-ವನು
ಮೇಕೆ-ಗಳನ್ನು
ಮೇಧಾ-ವಿಯೂ
ಮೇಯಿತು
ಮೇಯು-ತ್ತಿದ್ದ
ಮೇಯು-ತ್ತಿ-ದ್ದಾಗ
ಮೇಲಕ್ಕೆ
ಮೇಲ-ಕ್ಕೆತ್ತಿ
ಮೇಲ-ಕ್ಕೆ-ತ್ತಿ-ದ್ದರು
ಮೇಲ-ಕ್ಕೆತ್ತು
ಮೇಲಿಂದ
ಮೇಲಿದೆ
ಮೇಲಿದ್ದ
ಮೇಲಿದ್ದು
ಮೇಲಿನ
ಮೇಲಿ-ನದು
ಮೇಲಿ-ರುವ
ಮೇಲು
ಮೇಲೂ
ಮೇಲೆ
ಮೇಲೆ-ತ್ತಿ-ಕೊಂಡು
ಮೇಲೆತ್ತು
ಮೇಲೆದ್ದು
ಮೇಲೆಯೇ
ಮೇಲೆ-ಯೇ-ಜ-ಗಳ
ಮೇಲೆಲ್ಲ
ಮೇಲ್ಕೋ-ಣೆ-ಯ-ಲ್ಲಿದ್ದ
ಮೇವಿ-ಗಾಗಿ
ಮೈ
ಮೈಗೆ
ಮೈಗೆಲ್ಲ
ಮೈಮು-ರಿಯು
ಮೈಯೆಲ್ಲ
ಮೈಯೆಲ್ಲಾ
ಮೈಲಿ-ಗಳು
ಮೈಸೂರು
ಮೊಗದ
ಮೊಟ್ಟೆ
ಮೊಟ್ಟೆ-ಯನ್ನು
ಮೊದ-ಮೊ-ದಲು
ಮೊದಲ
ಮೊದ-ಲನೆ
ಮೊದ-ಲ-ನೆಯ
ಮೊದ-ಲ-ನೆ-ಯ-ದಾಗಿ
ಮೊದ-ಲ-ನೆ-ಯ-ವನು
ಮೊದ-ಲ-ನೆ-ಯ-ವನೆ
ಮೊದ-ಲಿ-ಗಿಂತ
ಮೊದಲು
ಮೊರ
ಮೊರ-ದಂತೆ
ಮೊರೆ
ಮೊರೆ-ಯಿಟ್ಟ
ಮೊರೆ-ಯಿ-ಟ್ಟಳು
ಮೊಲೆ-ಗಳನ್ನು
ಮೊಸ-ರನ್ನು
ಮೊಸ-ರಿನ
ಮೊಸರು
ಮೊಹ-ರು-ಗಳನ್ನು
ಮೊಹ-ರು-ಗಳು
ಮೋಕ್ಷ-ದಾ-ಯಿ-ನಿ-ಯಾದ
ಮೋಕ್ಷ-ಫ-ಲ-ಗಳ
ಮೋಚಿ
ಮೋಚಿಯೇ
ಮೋಡ
ಮೋಡ-ವನ್ನು
ಮೋಸ
ಮೋಸ-ಮಾಡಿ
ಮೋಸ-ಮಾ-ಡು-ವು-ದಿಲ್ಲ
ಮೌನ
ಮೌನ-ವಾಗಿ
ಮೌನ-ವಾ-ಗಿದ್ದ
ಮೌನ-ವಿ-ರು-ವಲ್ಲಿ
ಮೌನಿ-ಯಾಗು
ಮೌಲ್ಯ-ಗಳ
ಮ್ಯಾಜಿ-ಸ್ಟ್ರೇಟ್
ಮ್ಯಾಜಿ-ಸ್ಟ್ರೇಟ್ಗೆ
ಮ್ಯಾನೇ
ಮ್ಯಾನೇ-ಜರ್
ಮ್ಯೂಸಿ-ಯಂ-ನ-ಲ್ಲಿ-ರುವ
ಯಂತೆ
ಯಂತೆಯೇ
ಯಕ್ಷ
ಯಕ್ಷ-ನಿಗೆ
ಯಕ್ಷ-ನಿ-ರುವ
ಯಕ್ಷನೇ
ಯಜ-ಮಾನ
ಯಜ-ಮಾ-ನನ
ಯಜ-ಮಾ-ನ-ನನ್ನು
ಯಜ-ಮಾ-ನ-ನಿಗೆ
ಯಜ-ಮಾ-ನ-ರಿಂದ
ಯಜ-ಮಾ-ನ-ರಿಗೆ
ಯಜ-ಮಾನಿ
ಯತ್ನಿಸಿ
ಯತ್ನಿ-ಸಿದ
ಯತ್ನಿ-ಸಿ-ದ-ಕಾ-ಲಿಟ್ಟ
ಯತ್ನಿ-ಸಿ-ದರು
ಯತ್ನಿ-ಸು-ತ್ತವೆ
ಯತ್ನಿ-ಸು-ತ್ತಾರೆ
ಯತ್ನಿ-ಸು-ತ್ತಿದ್ದೆ
ಯತ್ನಿ-ಸು-ತ್ತಿ-ರು-ವೆನು
ಯಥಾ-ಶಕ್ತಿ
ಯನ್ನು
ಯನ್ನೇ
ಯಮ
ಯಮನ
ಯಮ-ಪಾ-ಶ-ದಿಂದ
ಯಮ-ಲೋ-ಕಕ್ಕೆ
ಯಮುನಾ
ಯಮುನೆ
ಯಮು-ನೆ-ಯನ್ನು
ಯಲು
ಯಲ್ಲ
ಯಲ್ಲಿ
ಯಲ್ಲಿದ್ದ
ಯಲ್ಲಿಯೇ
ಯಲ್ಲಿ-ರುವ
ಯವನು
ಯವ-ರೊ-ಬ್ಬರು
ಯಶೋದೆ
ಯಶೋ-ದೆಯ
ಯಶೋ-ದೆ-ಯನ್ನು
ಯಾಗ-ವನ್ನು
ಯಾಗಿ
ಯಾಗಿದೆ
ಯಾಗಿದ್ದ
ಯಾಗಿದ್ದೆ
ಯಾಗುತ್ತ
ಯಾಗುವ
ಯಾಗು-ವುದು
ಯಾಗು-ವುದೋ
ಯಾಚಿ-ಸಿದೆ
ಯಾತ
ಯಾತಕ್ಕೆ
ಯಾತ-ದಿಂದ
ಯಾತ್ರೆಗೆ
ಯಾತ್ರೆಯ
ಯಾದಳು
ಯಾದ-ವ-ಗಿರಿ
ಯಾದಾನು
ಯಾನ
ಯಾಯಿತು
ಯಾರ
ಯಾರ-ದಯ್ಯ
ಯಾರದು
ಯಾರದೋ
ಯಾರನ್ನು
ಯಾರನ್ನೂ
ಯಾರಾ
ಯಾರಾ-ದರೂ
ಯಾರಿಂದ
ಯಾರಿ-ಗಾಗಿ
ಯಾರಿ-ಗಾ-ದರೂ
ಯಾರಿಗೂ
ಯಾರಿಗೆ
ಯಾರು
ಯಾರೂ
ಯಾರೊ
ಯಾರೊಂ-ದಿಗೂ
ಯಾರೋ
ಯಾವ
ಯಾವ-ಕ-ಡೆಗೆ
ಯಾವ-ತ್ತಾ-ದರೂ
ಯಾವಾ
ಯಾವಾಗ
ಯಾವಾ-ಗ-ಲಾ-ದರೂ
ಯಾವಾ-ಗ-ಲಾ-ದ-ರೊಮ್ಮೆ
ಯಾವಾ-ಗಲೂ
ಯಾವು-ದಕ್ಕೂ
ಯಾವು-ದಕ್ಕೆ
ಯಾವುದನ್ನು
ಯಾವು-ದನ್ನೂ
ಯಾವು-ದ-ರಿಂದ
ಯಾವು-ದ-ರಿಂ-ದಲೂ
ಯಾವು-ದಾ-ದರೂ
ಯಾವುದು
ಯಾವುದೂ
ಯಾವುದೇ
ಯಾವುದೋ
ಯಿಂದ
ಯಿಂದಲೇ
ಯಿಂದಿ-ಳಿದ
ಯಿತರ
ಯಿತು
ಯಿತೆ
ಯಿಲ್ಲದೆ
ಯುಕ್ತಾ-ಯುಕ್ತ
ಯುದ್ಧ-ದಲ್ಲಿ
ಯುಧಿ-ಷ್ಠಿ-ರ-ನಿಗೆ
ಯುವಕ
ಯುವ-ಕ-ನನ್ನು
ಯುವ-ಕರ
ಯುವತಿ
ಯುವ-ತಿಯ
ಯುವ-ತಿ-ಯ-ರಿಗೆ
ಯುವುದು
ಯುವೆ
ಯೊಂದು
ಯೊಳ-ಗಿ-ದ್ದಾಗ
ಯೋಗ
ಯೋಗಿ
ಯೋಗಿ-ಗಳ
ಯೋಗಿ-ಗಳು
ಯೋಗಿಗೆ
ಯೋಗಿ-ಯನ್ನು
ಯೋಗ್ಯ
ಯೋಗ್ಯ-ತೆಗೆ
ಯೋಗ್ಯ-ವಾ-ಗಿದ್ದೆ
ಯೋಚ-ನೆ-ಗ-ಳೆಲ್ಲ
ಯೋಚಿ-ಸದೆ
ಯೋಚಿ-ಸ-ಬೇಕು
ಯೋಚಿ-ಸಲೂ
ಯೋಚಿಸಿ
ಯೋಚಿ-ಸಿತು
ಯೋಚಿ-ಸಿದ
ಯೋಚಿ-ಸಿ-ದಳು
ಯೋಚಿಸು
ಯೋಚಿ-ಸು-ತ್ತಿ-ದ್ದನು
ಯೋಚಿ-ಸು-ತ್ತಿ-ರು-ವನು
ಯೋಣ
ಯೌವ-ನದ
ರಂಜಿ-ಸು-ತ್ತಿತ್ತು
ರಂಧ್ರದ
ರಂಧ್ರ-ದಲ್ಲಿ
ರಂಭಿ-ಸಿದ
ರಕ್ತ-ಕಾರಿ
ರಕ್ತದ
ರಕ್ತ-ಮ-ಯ-ವಾ-ಗಿತ್ತು
ರಕ್ತ-ವನ್ನು
ರಕ್ತ-ವ-ರ್ಣಾಂ
ರಕ್ಷಣೆ
ರಕ್ಷ-ಣೆ-ಗಾಗಿ
ರಕ್ಷಿ
ರಕ್ಷಿ-ಸ-ಬೇಕು
ರಕ್ಷಿ-ಸಲು
ರಕ್ಷಿಸಿ
ರಕ್ಷಿಸು
ರಕ್ಷಿ-ಸು-ವನು
ರಕ್ಷಿ-ಸು-ವಾಗ
ರಕ್ಷಿ-ಸು-ವು-ದ-ಕ್ಕಾಗಿ
ರಘು-ವೀ-ರನ
ರಚಿ-ಸಿದ
ರಚಿ-ಸಿ-ರು-ವೆನು
ರಜ-ಸ್ಸಿನ
ರಜಸ್ಸು
ರಣ
ರಣ-ಜಿತ್
ರಣ-ಜಿ-ತ್ರಾಯ್
ರಣ-ವಾದ
ರಥದ
ರಥ-ದಲ್ಲಿ
ರಥ-ವನ್ನು
ರನ್ನಾಗಿ
ರಬೇಕು
ರಭ-ಸ-ದಲ್ಲಿ
ರಲ್ಲ
ರಲ್ಲಿ
ರಸದ
ರಸ-ಭ-ರಿ-ತ-ವಾದ
ರಸ-ವನ್ನು
ರಸಿ-ಕನ
ರಸ್ತೆಯ
ರಸ್ತೆ-ಯಲ್ಲಿ
ರಹಸ್ಯ
ರಾಕ್ಷಸ
ರಾಕ್ಷ-ಸ-ನನ್ನು
ರಾಕ್ಷಸಿ
ರಾಖಾಲ
ರಾಖಾ-ಲನ
ರಾಖಾ-ಲ-ನನ್ನು
ರಾಜ
ರಾಜ-ಕು-ಮಾರ
ರಾಜ-ಕು-ಮಾರಿ
ರಾಜ-ಕು-ಮಾ-ರಿ-ಯನ್ನು
ರಾಜನ
ರಾಜ-ನಾಗಿ
ರಾಜ-ನಾ-ಗಿದ್ದ
ರಾಜ-ನಾ-ಗಿದ್ದೆ
ರಾಜ-ನಾಗು
ರಾಜ-ನಾದೆ
ರಾಜ-ನಿಗೆ
ರಾಜನು
ರಾಜನೇ
ರಾಜ-ನೊಬ್ಬ
ರಾಜ-ಮಾರ್ಗ
ರಾಜ-ಮಾ-ರ್ಗಕ್ಕೆ
ರಾಜರ
ರಾಜ-ರನ್ನು
ರಾಜರು
ರಾಜರೆ
ರಾಜ-ರೆ-ದು-ರಿಗೆ
ರಾಜರೇ
ರಾಜ-ಸೂಯ
ರಾಜ್ಯ
ರಾಜ್ಯದ
ರಾಜ್ಯ-ಭಾ-ರ-ವೇಕೆ
ರಾಜ್ಯ-ವನ್ನು
ರಾಜ್ಯ-ವಾ-ಳು-ತ್ತಿ-ದ್ದಾಗ
ರಾಣಿ
ರಾಣಿಗೂ
ರಾಣಿಗೆ
ರಾಣಿಯ
ರಾಣಿ-ಯಾದ
ರಾತ್ರಿ
ರಾತ್ರಿ-ಯೆಲ್ಲ
ರಾತ್ರಿಯೇ
ರಾಧೆ
ರಾಧೆ-ಯನ್ನು
ರಾಧೆಯು
ರಾಮ
ರಾಮ-ಲ-ಕ್ಷ್ಮ-ಣರು
ರಾಮ-ಕೃಷ್ಣ
ರಾಮ-ಕೃ-ಷ್ಣರ
ರಾಮ-ಚಂ-ದ್ರ-ನನ್ನು
ರಾಮನ
ರಾಮ-ನಂತೆ
ರಾಮ-ನನ್ನು
ರಾಮ-ನಾ-ಮ-ವನ್ನು
ರಾಮ-ನಾ-ಮ-ವಷ್ಟೇ
ರಾಮ-ನಿಂದ
ರಾಮ-ನಿಗೆ
ರಾಮನು
ರಾಮನೆ
ರಾಮ-ಮ-ಲ್ಲಿಕ
ರಾಮ-ಮ-ಲ್ಲಿ-ಕ-ನನ್ನು
ರಾಮ-ಲ-ಕ್ಷ್ಮ-ಣರು
ರಾಮ-ಲಾಲ
ರಾಮ-ಲಾ-ಲನ
ರಾಮ-ಲಾ-ಲ-ನಿಗೆ
ರಾಮಾ-ಯಣ
ರಾಮಾ-ಯ-ಣ-ದಲ್ಲಿ
ರಾಮ್
ರಾಯನ
ರಾಯಿತು
ರಾಯ್
ರಾವಣ
ರಾವ-ಣನ
ರಾವ-ಣ-ನಿಗೆ
ರಾವ-ಣ-ರಾ-ಮನ
ರಾಶಿ
ರಾಶಿ-ಯಲ್ಲಿ
ರಾಶಿಯೆ
ರಾಸ
ರಾಸ-ಮ-ಣಿಯ
ರಾಸ-ಲೀ-ಲೆಯ
ರಿಂದ
ರಿಗೂ
ರಿಗೆ
ರಿಜಿ-ಸ್ಟರ್
ರಿಜಿಸ್ಟ್ರಿ
ರಿಪೇ-ರಿ-ಮಾ-ಡು-ವು-ದಕ್ಕೆ
ರಿಸ-ತೊ-ಡ-ಗಿ-ದನು
ರಿಸಿದ
ರೀತಿ
ರೀತಿ-ಯನ್ನು
ರೀತಿ-ಯಲ್ಲಿ
ರುಗಿ
ರುಚಿ
ರುಚಿ-ಕ-ರ-ವಾದ
ರುಚಿಗೆ
ರುಚಿ-ರು-ಚಿ-ಯಾದ
ರುಚಿ-ಸು-ವಂತೆ
ರುದ್ರಾ-ವ-ತಾ-ರ-ವನ್ನು
ರುವ
ರೂಪ
ರೂಪ-ಗಳ
ರೂಪ-ಗ-ಳಂತೆ
ರೂಪ-ಗಳನ್ನು
ರೂಪ-ಗಳನ್ನೂ
ರೂಪ-ವನ್ನು
ರೂಪಾಯಿ
ರೂಪಾ-ಯಿ-ಗಳನ್ನು
ರೂಪಾ-ಯಿ-ಯ-ನ್ನಾ-ದರೂ
ರೂಪಾ-ಯಿ-ಯನ್ನು
ರೂಪಾ-ಯಿ-ಯನ್ನೂ
ರೂಪಿ-ಸಿ-ದ್ದಾರೆ
ರೆಲ್ಲ
ರೇಖಾ-ಚಿ-ತ್ರ-ಗಳ
ರೇಗಾ-ಡಿ-ದನು
ರೇಗಿ
ರೇಷ್ಮೆ
ರೈತ
ರೈತ-ನಿದ್ದ
ರೈತನು
ರೈತನೂ
ರೈತರು
ರೊಂದಿಗೆ
ರೊಡನೆ
ರೋಗ-ದಿಂದ
ರೋಗ-ವನ್ನು
ರೋಗಿ
ರೋಗಿಗೆ
ರೋಗಿಯ
ಲಂಕಾ
ಲಂಕಾ-ರಾ-ಜ್ಯದ
ಲಂಕೆಗೆ
ಲಕ್ಷ
ಲಕ್ಷಣ
ಲಕ್ಷ-ಣ-ಗಳು
ಲಕ್ಷ್ಮಣ
ಲಕ್ಷ್ಮ-ಣ-ನಿಗೆ
ಲಕ್ಷ್ಮಿ
ಲಕ್ಷ್ಮಿಯೇ
ಲಕ್ಷ್ಮೀ-ನಾ-ರಾ-ಯಣ
ಲಕ್ಷ್ಮೀ-ನಾ-ರಾ-ಯ-ಣರು
ಲನ್ನು
ಲಭಿ-ಸಿತ್ತು
ಲಭಿ-ಸಿ-ದರೆ
ಲಯ
ಲವ-ಕು-ಶ-ರಿಗೆ
ಲವ-ಕು-ಶರು
ಲಾಭ
ಲಾಭ-ವಾ-ಗು-ವುದು
ಲಾರ
ಲಾರಂ-ಭಿ-ಸಿದ
ಲಾರದು
ಲಾರೆವು
ಲಾಲ
ಲಾಲ್ಬಂಧ್
ಲಿಲ್ಲ
ಲಿಸಿದ
ಲೀಲಾ-ಪ್ರ-ಸಂಗ
ಲೀಲೆ
ಲೀಲೆಯ
ಲೀಲೆ-ಯನ್ನು
ಲೀಲೆ-ಯ-ನ್ನೆಲ್ಲ
ಲೂಟಿ
ಲೆಂದು
ಲೆಕ್ಕ
ಲೆಕ್ಕ-ಹಾ-ಕೋಣ
ಲೆಕ್ಕಾ-ಚಾರ
ಲೆಕ್ಕಾ-ಚಾ-ರ-ದಿಂದ
ಲೆಕ್ಕಾ-ಚಾ-ರ-ಮಾ-ಡು-ತ್ತಿ-ರು-ವಾಗ
ಲೇಖ-ಕನು
ಲೇಪಿಸಿ
ಲೋಕಕ್ಕೆ
ಲೋಚ-ನನ
ಲೋಚ-ನ-ನಿಗೆ
ಲೌಕಿಕ
ಲ್ಲದ
ಲ್ಲಿತ್ತು
ಲ್ಲಿದ್ದ
ಲ್ಲಿರುವ
ಳನ್ನು
ಳೊಡನೆ
ವಂಗ-ದೇ-ಶದ
ವಂತನ
ವಂತ-ನನ್ನು
ವಂತರು
ವಂತೆ
ವಂದಿ-ಸಿದ
ವಂದಿ-ಸಿ-ದನು
ವಂಶ-ದ-ವ-ರ-ನ್ನೆಲ್ಲ
ವಕ್ರ-ರೀತಿ
ವಕ್ರ-ವ-ಕ್ರ-ವಾಗಿ
ವಚ-ನ-ಗಳು
ವಚ-ನ-ವೇದ
ವಚ-ನ-ವೇ-ದ-ದಲ್ಲಿ
ವಜ್ರ
ವಜ್ರಕ್ಕೂ
ವಜ್ರದ
ವಜ್ರ-ಪಡಿ
ವಜ್ರ-ವನ್ನು
ವಟ-ಗುಟ್ಟು
ವಣೆಗೂ
ವತಿ
ವತಿಯ
ವದಂತಿ
ವಧೆ-ಯಾದ
ವನಲ್ಲಿ
ವನಿಗೆ
ವನು
ವನೆ
ವನೆಂದು
ವನೋ
ವನ್ನು
ವನ್ನೂ
ವನ್ನೇ
ವಯ-ಸ್ಸಾ-ಗಿತ್ತು
ವಯ-ಸ್ಸಾ-ಗುತ್ತಾ
ವಯ-ಸ್ಸಾ-ಗು-ವು-ದಿಲ್ಲ
ವಯ-ಸ್ಸಾದ
ವಯ-ಸ್ಸಿನ
ವರ
ವರ-ದಿ-ಯಾ-ಗಿರ
ವರ-ವ-ನ್ನಾ-ದರೂ
ವರ-ವನ್ನು
ವರಾಂ-ಡ-ದಲ್ಲಿ
ವರಿ-ಯಲು
ವರಿಸ
ವರು
ವರುಣ
ವರುಷ
ವರು-ಷ-ಗ-ಳಾದ
ವರು-ಷದ
ವರು-ಷ-ವಾ-ಗಿತ್ತು
ವರು-ಷ-ವೆಲ್ಲ
ವರ್ಣ-ಚಿ-ತ್ರ-ವನ್ನು
ವರ್ತಕ
ವರ್ತ-ಕನ
ವರ್ತ-ಕ-ನನ್ನು
ವರ್ತ-ಕ-ನಿಗೂ
ವರ್ತ-ಕನು
ವರ್ತಿಯು
ವರ್ಷ
ವರ್ಷ-ಗಳ
ವರ್ಷ-ಗ-ಳಾದ
ವಲ-ಯ-ದಲ್ಲಿ
ವವನ
ವವನು
ವವರು
ವವಳು
ವಶ-ದಿಂದ
ವಶ-ಮಾಡಿ
ವಶ-ಮಾ-ಡಿ-ಕೊಂ-ಡ-ವನು
ವಶ-ರಾಗಿ
ವಶ-ವಾ-ಗಿತ್ತು
ವಸಿ-ಷ್ಠರ
ವಸಿ-ಷ್ಠರು
ವಸ್ತು-ಗಳನ್ನು
ವಸ್ತು-ಗ-ಳಿಗೂ
ವಸ್ತು-ಗ-ಳಿಗೆ
ವಸ್ತು-ಗಳು
ವಸ್ತು-ಗ-ಳು-ಸಿ-ಕ್ಕಿ-ದವು
ವಸ್ತು-ವಾ-ಗಿ-ದ್ದರೂ
ವಸ್ತು-ವಿಗೆ
ವಸ್ತು-ವಿನ
ವಸ್ತು-ವಿ-ನಲ್ಲಿ
ವಸ್ತುವೇ
ವಸ್ತ್ರ
ವಸ್ತ್ರ-ಗಳನ್ನು
ವಸ್ತ್ರದ
ವಸ್ತ್ರಾ-ಭ-ರ-ಣ-ಗಳಿಂದ
ವಸ್ಥೆಗೆ
ವಾಗ
ವಾಗ-ಲಾರ
ವಾಗಿ
ವಾಗಿತ್ತೋ
ವಾಗಿ-ಬಿ-ಡು-ತ್ತದೆ
ವಾಗು
ವಾಗು-ತ್ತವೆ
ವಾಗು-ತ್ತಾನೆ
ವಾಗು-ವರು
ವಾಚಾ-ಳಿ-ಯಾ-ಗು-ತ್ತಾನೆ
ವಾಣಿ
ವಾದ
ವಾದನೆ
ವಾದ್ಯ-ಶಾ-ಲೆಯ
ವಾಯಿತು
ವಾಯು-ದೇ-ವನೆ
ವಾಯು-ದೇ-ವರ
ವಾಯು-ದೇ-ವ-ರಿಗೆ
ವಾಯು-ದೇ-ವರು
ವಾಯು-ದೇ-ವರೆ
ವಾಯು-ವಿನ
ವಾರ
ವಾರ-ದಿಂದ
ವಾಳದ
ವಾಸನೆ
ವಾಸ-ನೆ-ಯಿಂದ
ವಾಸ-ಮಾಡು
ವಾಸ-ಮಾ-ಡು-ತ್ತಿದ್ದ
ವಾಸ-ವಾ-ಗಿತ್ತು
ವಾಸ-ವಾ-ಗಿದ್ದ
ವಾಸ-ವಾ-ಗಿ-ದ್ದವು
ವಾಸಿ-ಸು-ತ್ತಿದ್ದ
ವಿಂಗ-ಡಿಸಿ
ವಿಗ್ರಹ
ವಿಗ್ರ-ಹದ
ವಿಗ್ರ-ಹ-ದಲ್ಲಿ
ವಿಗ್ರ-ಹ-ವನ್ನು
ವಿಗ್ರ-ಹ-ವನ್ನೂ
ವಿಚಾರ
ವಿಚಾ-ರ-ಗಳನ್ನೂ
ವಿಚಾ-ರ-ಣೆಗೆ
ವಿಚಾ-ರ-ದಲ್ಲಿ
ವಿಚಾ-ರ-ಧಾರೆ
ವಿಚಾ-ರ-ಮಾಡಿ
ವಿಚಾ-ರ-ವಂ-ತರು
ವಿಚಾ-ರಿ-ಸಿದ
ವಿಚಾ-ರಿ-ಸಿ-ದಳು
ವಿಚಾ-ರಿ-ಸಿ-ದಾಗ
ವಿಚಾ-ರಿಸು
ವಿಚಿತ್ರ
ವಿಚಿ-ತ್ರ-ವನ್ನು
ವಿಚಿ-ತ್ರ-ವಾ-ಗಿತ್ತು
ವಿಚಿ-ತ್ರ-ವಾದ
ವಿಚಿ-ತ್ರ-ವಾ-ದುದು
ವಿಜ್ಞಾನ
ವಿಜ್ಞಾ-ನ-ದಲ್ಲಿ
ವಿತ್ತಳು
ವಿತ್ತು
ವಿದುರ
ವಿದು-ರನ
ವಿದೆ
ವಿದ್ಯರು
ವಿದ್ಯಾ-ಭ್ಯಾಸ
ವಿದ್ಯಾ-ವಂತ
ವಿದ್ಯಾ-ವಂ-ತ-ನಿಲ್ಲ
ವಿದ್ಯೆ-ಗಳನ್ನು
ವಿದ್ಯೆಗೆ
ವಿದ್ಯೆ-ಯನ್ನು
ವಿಧ-ದಿಂ-ದಲೂ
ವಿಧವೆ
ವಿಧ-ವೆ-ಯಿಂದ
ವಿಧ-ವೆಯು
ವಿಧಿ
ವಿಧೇ-ಯ-ತೆ-ಯಿಂದ
ವಿನ-ಯ-ದಿಂದ
ವಿನಾ
ವಿಭ-ಜನೆ
ವಿಭೀ
ವಿಭೀ-ಷಣ
ವಿಭೀ-ಷ-ಣ-ನಿಗೆ
ವಿಭೂ-ತಿ-ಯನ್ನು
ವಿರ-ಬೇಕು
ವಿರ-ಮಿ-ಸ-ಲಿಲ್ಲ
ವಿರಾಮ
ವಿರಾ-ಮಕ್ಕೆ
ವಿರಾ-ಮವೇ
ವಿರಿ
ವಿರುವ
ವಿರೋ-ಧಾ-ಭಾ-ಸ-ವನ್ನು
ವಿರೋ-ಧಿ-ಸದೆ
ವಿರೋ-ಧಿ-ಸು-ವರೊ
ವಿಲ್ಲ
ವಿಲ್ಲ-ವೆಂದು
ವಿವ
ವಿವ-ರ-ಣೆ-ಯನ್ನು
ವಿವ-ರಿ-ಸ-ಲಾರ
ವಿವ-ರಿ-ಸ-ಲಾರೆ
ವಿವ-ರಿ-ಸಲು
ವಿವ-ರಿ-ಸಿದ
ವಿವ-ರಿಸು
ವಿವ-ರಿ-ಸು-ತ್ತವೆ
ವಿವ-ರಿ-ಸು-ತ್ತಾರೆ
ವಿವ-ರಿ-ಸು-ತ್ತೀರಿ
ವಿವ-ರಿ-ಸು-ವು-ದಕ್ಕೆ
ವಿವಾಹ
ವಿವಿಧ
ವಿವೇಕಾನಂದ
ವಿವೇ-ಚನೆ
ವಿಶಾಲ
ವಿಶೇಷ
ವಿಶ್ರ-ಮಿ-ಸಿದ
ವಿಶ್ವದ
ವಿಶ್ವ-ದಾ-ದ್ಯಂತ
ವಿಶ್ವ-ನ-ಕಾ-ಶೆ-ಯನ್ನು
ವಿಶ್ವಾ-ಸ-ಗಳು
ವಿಶ್ವಾ-ಸ-ವಾ-ಗು-ವು-ದಿಲ್ಲ
ವಿಶ್ವಾ-ಸ-ವಿತ್ತು
ವಿಶ್ವಾ-ಸ-ವಿ-ರ-ಬೇಕು
ವಿಷ
ವಿಷ-ದಲ್ಲಿ
ವಿಷಯ
ವಿಷ-ಯದ
ವಿಷ-ಯ-ದಲ್ಲಿ
ವಿಷ-ಯ-ಬೇರೆ
ವಿಷ-ಯ-ವನ್ನು
ವಿಷ-ಯ-ವ-ನ್ನೆಲ್ಲ
ವಿಷ-ಯ-ವ-ಸ್ತು-ಗಳಿಂದ
ವಿಷ-ಯ-ವಾಗಿ
ವಿಷ-ಯವೇ
ವಿಷ-ವನ್ನು
ವಿಷ-ವಿ-ರುವ
ವಿಷ-ಸ-ರ್ಪ-ವಿದೆ
ವಿಷ್ಣು
ವಿಷ್ಣು-ದೂ-ತರು
ವಿಷ್ಣು-ಪು-ರಕ್ಕೆ
ವಿಷ್ಣು-ಭ-ಕ್ತ-ನಾ-ದ್ದ-ರಿಂದ
ವಿಷ್ಣು-ಮಯ
ವಿಷ್ಣು-ವನ್ನು
ವಿಷ್ಣು-ವಿಗೆ
ವಿಷ್ಣು-ವಿದ್ದ
ವಿಷ್ಣು-ವಿನ
ವಿಷ್ಣು-ವಿ-ನೊ-ಡನೆ
ವಿಷ್ಣುವು
ವೀರ-ಭದ್ರ
ವೀರ-ಭ-ದ್ರನ
ವೀರ-ಭ-ದ್ರ-ನಿಗೆ
ವುದಕ್ಕೂ
ವುದಕ್ಕೆ
ವುದನ್ನು
ವುದರ
ವುದ-ರಲ್ಲಿ
ವುದ-ರ-ಲ್ಲಿದ್ದ
ವುದ-ರೊ-ಳಗೆ
ವುದಾ-ದರೆ
ವುದಿಲ್ಲ
ವುದು
ವುದೋ
ವೃತ್ತಿ
ವೃತ್ತಿ-ಯಂತೆ
ವೃತ್ತಿ-ಯಾ-ದು-ದ-ರಿಂದ
ವೃತ್ತಿ-ಯಿಂದ
ವೃದ್ಧ-ಬ್ರಾ-ಹ್ಮ-ಣ-ನಂತೆ
ವೃದ್ಧಿ
ವೃದ್ಧಿ-ಯಾಗಿ
ವೆಂದು
ವೆನು
ವೆಬ್ಬಿ-ಸಿ-ದವು
ವೆಯಲ್ಲ
ವೆಯಾ
ವೆಲ್ಲ
ವೆಲ್ಲಿ
ವೆವೋ
ವೇಗಕ್ಕೆ
ವೇಗ-ದಿಂದ
ವೇಗ-ವಾಗಿ
ವೇಣು-ನಾದ
ವೇದ-ಗಳಲ್ಲಿ
ವೇದ-ದಲ್ಲಿ
ವೇದ-ವೇ-ದಾಂತ
ವೇದಾಂತ
ವೇದಾಂ-ತದ
ವೇದಾಂ-ತ-ವನ್ನು
ವೇದಾಂತಿ
ವೇದಾಂ-ತಿ-ಯಾದ
ವೇಳೆ
ವೇಳೆಗೆ
ವೇಳೆ-ಯಾ-ಗಿದೆ
ವೇಶ-ದ-ಲ್ಲಿದ್ದ
ವೇಶ್ಯೆ
ವೇಶ್ಯೆ-ಗಾಗಿ
ವೇಶ್ಯೆಗೆ
ವೇಶ್ಯೆಯ
ವೇಶ್ಯೆ-ಯನ್ನು
ವೇಶ್ಯೆ-ಯರು
ವೇಶ್ಯೆ-ಯಾ-ದರೂ
ವೇಶ್ಯೆ-ಯಾ-ದರೋ
ವೇಷ
ವೇಷ-ಗಳನ್ನು
ವೇಷ-ಗಳಲ್ಲಿ
ವೇಷ-ದಲ್ಲಿ
ವೇಷ-ವನ್ನು
ವೈಕುಂಠ
ವೈಕುಂ-ಠಕ್ಕೆ
ವೈಕುಂ-ಠ-ದಲ್ಲಿ
ವೈಕುಂ-ಠ-ದ-ಲ್ಲಿ-ದ್ದರು
ವೈಕುಂ-ಠ-ದಿಂದ
ವೈಡೂರ್ಯ
ವೈದ್ಯ
ವೈದ್ಯನ
ವೈದ್ಯ-ನನ್ನು
ವೈದ್ಯನು
ವೈದ್ಯ-ನೋ-ರ್ವನು
ವೈದ್ಯ-ರಿಗೆ
ವೈಭವ
ವೈಭ-ವ-ದಿಂದ
ವೈಭ-ವ-ವನ್ನು
ವೈರಾಗ್ಯ
ವೈರಾ-ಗ್ಯದ
ವೈರಾ-ಗ್ಯ-ದಿಂದ
ವೈರಾ-ಗ್ಯವೂ
ವೈವಿ-ಧ್ಯ-ತೆ-ಗ-ಳೆಲ್ಲ
ವೈಷ್ಣ-ವ-ಚ-ರಣ
ವೈಷ್ಣ-ವ-ಚ-ರ-ಣ-ನಿಗೆ
ವೈಷ್ಣ-ವ-ಚ-ರ-ಣನು
ವೈಷ್ಣ-ವ-ಚ-ರ-ಣ-ನೆಂಬ
ವೈಷ್ಣ-ವ-ನಂತೆ
ವೈಷ್ಣ-ವರ
ವೈಷ್ಣ-ವರು
ವೈಷ್ಣ-ವ-ಶಾ-ಸ್ತ್ರ-ದಲ್ಲಿ
ವ್ಯಕ್ತ
ವ್ಯಕ್ತವಾ
ವ್ಯಕ್ತಿ
ವ್ಯಕ್ತಿ-ಗ-ಳಿ-ರು-ವರು
ವ್ಯಕ್ತಿ-ಯಂತೆ
ವ್ಯಕ್ತಿ-ಯನ್ನು
ವ್ಯಕ್ತಿ-ಯಲ್ಲಿ
ವ್ಯಕ್ತಿ-ಯಾ-ಗು-ತ್ತಾನೆ
ವ್ಯಕ್ತಿಯು
ವ್ಯತ್ಯಾ-ಸ-ವನ್ನು
ವ್ಯತ್ಯಾ-ಸ-ವನ್ನೂ
ವ್ಯತ್ಯಾ-ಸ-ವಿಲ್ಲ
ವ್ಯತ್ಯಾ-ಸವೂ
ವ್ಯತ್ಯಾ-ಸವೆ
ವ್ಯತ್ಯಾ-ಸವೇ
ವ್ಯಥೆ
ವ್ಯಥೆ-ಪಟ್ಟ
ವ್ಯಥೆ-ಪಡು
ವ್ಯಥೆ-ಪ-ಡು-ತ್ತಿದ್ದ
ವ್ಯಥೆ-ಪ-ಡು-ತ್ತಿ-ರುವೆ
ವ್ಯಥೆ-ಯಾ-ಗಿದೆ
ವ್ಯಥೆ-ಯಿಂದ
ವ್ಯಭಿ-ಚಾ-ರಿ-ಣಿಯ
ವ್ಯರ್ಥ
ವ್ಯರ್ಥ-ವಾ-ಯಿತು
ವ್ಯವ-ಸ್ಥೆ-ಯನ್ನೂ
ವ್ಯವ-ಹ-ರಿ-ಸು-ತ್ತಿ-ರು-ವುದು
ವ್ಯವ-ಹಾ-ರ-ಜ್ಞಾ-ನ-ವಿತ್ತು
ವ್ಯಸ-ನ-ಪಟ್ಟ
ವ್ಯಸ-ನ-ವಾ-ಯಿತು
ವ್ಯಾಕು-ಲತೆ
ವ್ಯಾಕು-ಲ-ತೆಯ
ವ್ಯಾಕು-ಲ-ತೆ-ಯಿಂದ
ವ್ಯಾಕು-ಲ-ತೆ-ಯಿ-ಲ್ಲದೆ
ವ್ಯಾಕು-ಲ-ನಾ-ಗ-ಬೇಕು
ವ್ಯಾಕು-ಲ-ನಾಗಿ
ವ್ಯಾಕು-ಲ-ನಾ-ಗಿದ್ದ
ವ್ಯಾಕು-ಲ-ನಾ-ದರೆ
ವ್ಯಾಖ್ಯಾನ
ವ್ಯಾಘ್ರ
ವ್ಯಾಘ್ರ-ನ-ಲ್ಲಿಯೂ
ವ್ಯಾಜ್ಯ-ದಲ್ಲಿ
ವ್ಯಾಪಾರ
ವ್ಯಾಪಾ-ರಕ್ಕೆ
ವ್ಯಾಪಾರಿ
ವ್ಯಾಪಾ-ರಿ-ಗ-ಳಿ-ದ್ದಾರೆ
ವ್ಯಾಪಾ-ರಿಗೆ
ವ್ಯಾಪಾ-ರಿಯ
ವ್ಯಾಪಾ-ರಿ-ಯೊಬ್ಬ
ವ್ಯಾಮೋ-ಹಕ್ಕೆ
ವ್ಯಾಮೋ-ಹವು
ವ್ಯಾಸ-ದೇ-ವರು
ವ್ಯಾಸರು
ವ್ರತ
ವ್ರತ-ವನ್ನು
ಶಂಕ-ರ-ರಿಗೆ
ಶಂಕ-ರಾ-ಚಾ-ರ್ಯ-ರಿಗೆ
ಶಂಕ-ರಾ-ಚಾ-ರ್ಯರು
ಶಂಖ-ಚ-ಕ್ರ-ಗ-ದಾ-ಧಾ-ರಿ-ಯಾದ
ಶಂಖದ
ಶಂಖ-ವನ್ನು
ಶಕ್ತಿ
ಶಕ್ತಿ-ಗಳು
ಶಕ್ತಿಯ
ಶಕ್ತಿ-ಯಂತೆ
ಶಕ್ತಿಯೇ
ಶತ್ರು
ಶನಿ-ವಾರ
ಶಪಥ
ಶಪಿ-ಸಿ-ದರು
ಶಬ್ದ
ಶಬ್ದ-ಗಳು
ಶಬ್ದ-ಮಾಡು
ಶಬ್ದ-ವನ್ನು
ಶಬ್ದ-ವಾ-ಗಿತ್ತು
ಶಬ್ದ-ವಾ-ಡುತ್ತ
ಶಬ್ದವೂ
ಶರಣಾ
ಶರ-ಣಾ-ಗ-ತ-ನೆಂ-ದಿಗೂ
ಶರ-ಣಾ-ಗತಿ
ಶರ-ಪಂ-ಜ-ರದ
ಶರೀ-ರಕ್ಕೆ
ಶವ
ಶವದ
ಶವ-ವನ್ನು
ಶವ-ಸಾ-ಧ-ನೆ-ಯಲ್ಲಿ
ಶಸ್ತ್ರ-ವನ್ನು
ಶಾಂತಿ
ಶಾಂತಿ-ಯಿಂದ
ಶಾಕ್ತ
ಶಾಕ್ತರು
ಶಾಪ
ಶಾಪ-ದಿಂದ
ಶಾರ-ದಾ-ದೇ-ವಿ-ಯ-ವ-ರಿ-ಗಾಗಿ
ಶಾರ-ದಾ-ನಂದ
ಶಾಲಿ-ಯಾ-ಗಿ-ದ್ದರೆ
ಶಾಲೆಗೆ
ಶಾಶ್ವ-ತ-ವಾದ
ಶಾಸ್ತ್ರ-ಗಳನ್ನು
ಶಾಸ್ತ್ರ-ಗಳನ್ನೂ
ಶಾಸ್ತ್ರ-ಗಳು
ಶಾಸ್ತ್ರಜ್ಞ
ಶಾಸ್ತ್ರ-ಜ್ಞಾ-ನದ
ಶಾಸ್ತ್ರದ
ಶಾಸ್ತ್ರ-ದಲ್ಲಿ
ಶಾಸ್ತ್ರ-ವಾಕ್ಯ
ಶಾಸ್ತ್ರಾ-ದಿ-ಗಳಿಂದ
ಶಾಸ್ತ್ರಾ-ದಿ-ಗಳು
ಶಿಕ್ಷೆ
ಶಿಕ್ಷೆಗೆ
ಶಿಥಿ-ಲ-ವಾದ
ಶಿರ-ಚ್ಛೇ-ದ-ನ-ವಾ-ಗು-ವು-ದ-ರ-ಲ್ಲಿತ್ತು
ಶಿವ
ಶಿವನ
ಶಿವ-ನಂತೆ
ಶಿವ-ನ-ನ್ನಾ-ಗಲೀ
ಶಿವ-ನನ್ನು
ಶಿವ-ನಿದ್ದ
ಶಿವನು
ಶಿವ-ರಾಮ
ಶಿವ-ರಾ-ಮ-ನಿಗೆ
ಶಿವ-ಹೇ-ಳಿದ
ಶಿವೋಽಹಂ
ಶಿಷ್ಯ
ಶಿಷ್ಯನ
ಶಿಷ್ಯ-ನ-ನ್ನಿ-ರಿಸಿ
ಶಿಷ್ಯ-ನನ್ನು
ಶಿಷ್ಯ-ನಾಗಿ
ಶಿಷ್ಯ-ನಾ-ಗು-ವೆನು
ಶಿಷ್ಯ-ನಿಗೂ
ಶಿಷ್ಯ-ನಿಗೆ
ಶಿಷ್ಯ-ನಿದ್ದ
ಶಿಷ್ಯನು
ಶಿಷ್ಯ-ನೊಬ್ಬ
ಶಿಷ್ಯ-ರನ್ನು
ಶಿಷ್ಯ-ರಿ-ದ್ದರು
ಶಿಷ್ಯ-ರಿ-ಬ್ಬರೂ
ಶಿಷ್ಯರು
ಶಿಷ್ಯರೂ
ಶಿಷ್ಯ-ರೊಂ-ದಿಗೆ
ಶಿಷ್ಯೆ
ಶಿಷ್ಯೆ-ಯಿ-ದ್ದಳು
ಶುಕ-ದೇವ
ಶುಚಿ
ಶುದ್ಧ
ಶುದ್ಧ-ಗೊ-ಳಿಸು
ಶುದ್ಧ-ನಾಗಿ
ಶುದ್ಧ-ಭ-ಕ್ತಿ-ಯನ್ನು
ಶುದ್ಧ-ಮಾ-ಡು-ವುದು
ಶುದ್ಧ-ವಾ-ದಾಗ
ಶುರು
ಶುರು-ಮಾ-ಡಿದ
ಶುರು-ಮಾ-ಡಿ-ದರು
ಶುರು-ಮಾ-ಡು-ತ್ತಾನೆ
ಶುರು-ವಾ-ಯಿತು
ಶೂದ್ರ-ಳೊ-ಬ್ಬ-ಳನ್ನು
ಶೈಲಿ-ಯಲ್ಲಿ
ಶ್ಯಾಮ
ಶ್ಯಾಮಳ
ಶ್ಯಾಮ-ಳನ್ನು
ಶ್ಯಾಮಳು
ಶ್ಯಾಮಾ
ಶ್ರದ್ಧೆ
ಶ್ರದ್ಧೆಗೆ
ಶ್ರದ್ಧೆ-ಭ-ಕ್ತಿ-ಯಿಂದ
ಶ್ರದ್ಧೆಯ
ಶ್ರದ್ಧೆ-ಯಿಂದ
ಶ್ರದ್ಧೆ-ಯಿತ್ತು
ಶ್ರದ್ಧೆ-ಯಿ-ಲ್ಲದ
ಶ್ರದ್ಧೆಯೇ
ಶ್ರಮ-ಪ-ಟ್ಟನು
ಶ್ರವಣ
ಶ್ರಾದ್ಧ-ವಿ-ದ್ದ-ದ್ದ-ರಿಂದ
ಶ್ರಾದ್ಧಾ-ದಿ-ಗಳು
ಶ್ರಾದ್ಧಾ-ನ್ನ-ವನ್ನು
ಶ್ರೀ
ಶ್ರೀಕೃಷ್ಣ
ಶ್ರೀಕೃ-ಷ್ಣನ
ಶ್ರೀಕೃ-ಷ್ಣ-ನನ್ನು
ಶ್ರೀಕೃ-ಷ್ಣ-ನಿಂದ
ಶ್ರೀಕೃ-ಷ್ಣ-ನಿಗೆ
ಶ್ರೀಕೃ-ಷ್ಣನು
ಶ್ರೀಕೃ-ಷ್ಣನೇ
ಶ್ರೀಚೈ-ತನ್ಯ
ಶ್ರೀಮಂತ
ಶ್ರೀಮಂ-ತನ
ಶ್ರೀಮಂ-ತ-ನಾದ
ಶ್ರೀಮಂ-ತನು
ಶ್ರೀಮ-ನ್ನಾ-ರಾ-ಯ-ಣನ
ಶ್ರೀಮಾತೆ
ಶ್ರೀರಾ-ಮ-ಕ-ಲ್ಪ-ತರು
ಶ್ರೀರಾ-ಮ-ಕೃಷ್ಣ
ಶ್ರೀರಾ-ಮ-ಕೃ-ಷ್ಣರ
ಶ್ರೀರಾ-ಮ-ಕೃ-ಷ್ಣ-ರನ್ನು
ಶ್ರೀರಾ-ಮ-ಕೃ-ಷ್ಣ-ರಿದ್ದ
ಶ್ರೀರಾ-ಮ-ಕೃ-ಷ್ಣರು
ಶ್ರೀರಾ-ಮ-ಕೃ-ಷ್ಣ-ರೊ-ಡನೆ
ಶ್ರೀರಾ-ಮ-ಕೃ-ಷ್ಣ-ಸ್ವಾಮಿ
ಶ್ರೀರಾ-ಮ-ಚಂದ್ರ
ಶ್ರೀರಾ-ಮ-ಚಂ-ದ್ರನ
ಶ್ರೀರಾ-ಮ-ಚಂ-ದ್ರ-ನಿಗೆ
ಶ್ರೀರಾ-ಮ-ತ-ರು-ವಿ-ಹುದು
ಶ್ರೀರಾ-ಮನ
ಶ್ರೀರಾ-ಮ-ನನ್ನು
ಶ್ರೀವಿಷ್ಣು
ಶ್ರೇಷ್ಠ
ಶ್ರೇಷ್ಠ-ಭಕ್ತ
ಶ್ರೇಷ್ಠ-ವಾದ
ಶ್ರೇಷ್ಠ-ವಾ-ದು-ದನ್ನು
ಷಟ್ಭು-ಜ-ಳಾ-ಗು-ತ್ತಾಳೆ
ಷಡ್ದ-ರ್ಶ-ನ-ಗ-ಳೆ-ಲ್ಲ-ವನ್ನೂ
ಷಣನು
ಷರ-ತ್ತು-ಗಳನ್ನೆಲ್ಲ
ಸಂಕಟ
ಸಂಕ-ಟಕ್ಕೆ
ಸಂಕ-ಟ-ಪ-ಡ-ಬೇಕಾ
ಸಂಕ-ಟ-ವಾ-ಯಿತು
ಸಂಕೋ-ಚ-ದಿಂದ
ಸಂಗ
ಸಂಗ-ಡಿ-ಗ-ರಿಗೆ
ಸಂಗ-ಡಿ-ಗರು
ಸಂಗ-ಡಿ-ಗ-ರೊಂ-ದಿಗೆ
ಸಂಗಾತಿ
ಸಂಗಿ
ಸಂಗ್ರಹ
ಸಂಗ್ರ-ಹಿ-ಸಿ-ಕೊಂಡು
ಸಂಗ್ರ-ಹಿ-ಸು-ತ್ತಿತ್ತು
ಸಂಚ-ರಿ-ಸಿ-ದರೆ
ಸಂಚ-ರಿ-ಸು-ತ್ತಾರೆ
ಸಂಚ-ರಿ-ಸು-ತ್ತಿ-ತ್ತು-ಹೀಗೆ
ಸಂಚ-ರಿ-ಸು-ತ್ತಿ-ರುವೆ
ಸಂಚಾರ
ಸಂಚಾ-ರಕ್ಕೆ
ಸಂಜೆ
ಸಂಜೆ-ಯಾದ
ಸಂಜ್ಞೆ
ಸಂತ-ನೆಂದು
ಸಂತ-ರಿ-ಗಾಗಿ
ಸಂತರು
ಸಂತೃ-ಪ್ತ-ನಾದ
ಸಂತೆಗೆ
ಸಂತೆ-ಯಲ್ಲಿ
ಸಂತೋಷ
ಸಂತೋ-ಷಕ್ಕೆ
ಸಂತೋ-ಷ-ದಿಂದ
ಸಂತೋ-ಷ-ದಿಂ-ದಿ-ರು-ವಾಗ
ಸಂತೋ-ಷ-ಪ-ಟ್ಟಳು
ಸಂತೋ-ಷ-ಭ-ರಿ-ತ-ರಾ-ಗಿ-ರು-ವುದನ್ನು
ಸಂತೋ-ಷ-ವಾಗಿ
ಸಂತೋ-ಷ-ವಾ-ಗು-ವುದು
ಸಂತೋ-ಷ-ವಾ-ಯಿತು
ಸಂತೋ-ಷ-ವುಂ-ಟಾಗ
ಸಂತೋ-ಷವೆ
ಸಂದರ್ಭ
ಸಂದ-ರ್ಭ-ಗಳಲ್ಲಿ
ಸಂದೇ-ಶದ
ಸಂದೇಹ
ಸಂದೇ-ಹ-ವಿಲ್ಲ
ಸಂನ್ಯಾಸಿ
ಸಂನ್ಯಾ-ಸಿ-ಗ-ಳಿಗೆ
ಸಂನ್ಯಾ-ಸಿನಿ
ಸಂನ್ಯಾ-ಸಿ-ನಿ-ಯರು
ಸಂನ್ಯಾ-ಸಿಯ
ಸಂನ್ಯಾ-ಸಿ-ಯನ್ನು
ಸಂನ್ಯಾ-ಸಿ-ಯಾ-ಗ-ಬೇ-ಕೆಂದು
ಸಂನ್ಯಾ-ಸಿ-ಯಾ-ಗ-ಲಾರೆ
ಸಂನ್ಯಾ-ಸಿ-ಯಾ-ಗಿ-ದ್ದ-ರಿಂದ
ಸಂನ್ಯಾ-ಸಿ-ಯಾ-ಗು-ವು-ದಿಲ್ಲ
ಸಂನ್ಯಾ-ಸಿ-ಯಾ-ದರೆ
ಸಂನ್ಯಾ-ಸಿಯೂ
ಸಂನ್ಯಾ-ಸಿ-ಯೊಬ್ಬ
ಸಂಪ-ದ್ಭ-ರಿ-ತ-ರ-ನ್ನಾಗಿ
ಸಂಪ-ರ್ಕ-ದಿಂದ
ಸಂಪ-ರ್ಕ-ವನ್ನು
ಸಂಪಾ-ದಿ-ಸಿ-ಕೊಳ್ಳಿ
ಸಂಪಾ-ದಿ-ಸಿದ
ಸಂಪೂರ್ಣ
ಸಂಪೂ-ರ್ಣ-ವಾಗಿ
ಸಂಪ್ರ-ದಾ-ಯ-ಗಳ
ಸಂಬಂಧ
ಸಂಬಂ-ಧ-ಪಟ್ಟ
ಸಂಬಂ-ಧ-ವನ್ನು
ಸಂಬಂ-ಧ-ವನ್ನೂ
ಸಂಬಂ-ಧ-ವಾಗಿ
ಸಂಬಂ-ಧ-ವಿಲ್ಲ
ಸಂಬಂ-ಧಿ-ಸಿದ
ಸಂಬಳ
ಸಂಬ-ಳ-ವನ್ನು
ಸಂಭ-ವ-ವಿದೆ
ಸಂಭಾ-ಷ-ಣೆ-ಯನ್ನು
ಸಂಭಾ-ಷ-ಣೆಯೂ
ಸಂಭ್ರ-ಮದ
ಸಂಭ್ರ-ಮ-ದಿಂದ
ಸಂರ-ಕ್ಷಿ-ಸು-ತ್ತಿ-ರು-ವರೊ
ಸಂಶಯ
ಸಂಶ-ಯ-ಗಳು
ಸಂಶ-ಯ-ಗ-ಳೆಲ್ಲ
ಸಂಶ-ಯ-ವಿ-ರು-ವು-ದಿಲ್ಲ
ಸಂಶ-ಯ-ವಿಲ್ಲ
ಸಂಸ-ರ್ಗ-ದಿಂದ
ಸಂಸಾರ
ಸಂಸಾ-ರ-ಕ್ಕಾಗಿ
ಸಂಸಾ-ರಕ್ಕೆ
ಸಂಸಾ-ರದ
ಸಂಸಾ-ರ-ದಲ್ಲಿ
ಸಂಸಾ-ರ-ದ-ಲ್ಲಿದ್ದು
ಸಂಸಾ-ರ-ದ-ಲ್ಲಿಯೂ
ಸಂಸಾ-ರ-ದಿಂದ
ಸಂಸಾ-ರ-ವನ್ನು
ಸಂಸಾ-ರಿ-ಕ-ರನ್ನು
ಸಂಸಾ-ರಿ-ಗ-ಳಿಗೆ
ಸಂಸ್ಕ-ರ-ಣಕ್ಕೆ
ಸಂಸ್ಕಾ-ರ-ಗಳ
ಸಂಸ್ಕಾ-ರ-ಗಳನ್ನು
ಸಂಸ್ಕಾ-ರ-ಗಳು
ಸಂಸ್ಕಾ-ರ-ವನ್ನು
ಸಂಸ್ಯಾಸಿ
ಸಕಾ-ಲ-ದಲ್ಲಿ
ಸಕ್ಕರೆ
ಸಖಿ
ಸಗಣಿ
ಸಗ-ಣಿಗೂ
ಸಗ-ಣಿಯ
ಸಗ-ಣಿ-ಯನ್ನು
ಸಚ್ಚಿ-ದಾ-ನಂದ
ಸಜಾ
ಸಡಿ-ಲ-ಬಿ-ಟ್ಟಿತು
ಸಡಿ-ಲ-ಬಿಡಿ
ಸಡಿ-ಲ-ಬಿಡು
ಸಣ್ಣ
ಸತ್ತ
ಸತ್ತಂತೆ
ಸತ್ತರು
ಸತ್ತರೆ
ಸತ್ತ-ವ-ನನ್ನು
ಸತ್ತ-ವನು
ಸತ್ತಾ-ಗಿದೆ
ಸತ್ತಿಲ್ಲ
ಸತ್ತು
ಸತ್ತು-ಬಿತ್ತು
ಸತ್ತು-ಹೋ-ಗು-ತ್ತೇನೆ
ಸತ್ತು-ಹೋದ
ಸತ್ತು-ಹೋ-ಯಿತು
ಸತ್ತ್ವ-ಗುಣ
ಸತ್ಯ
ಸತ್ಯ-ದೆ-ದು-ರಿಗೆ
ಸತ್ಯ-ನಿಷ್ಠ
ಸತ್ಯ-ನಿ-ಷ್ಠೆಗೆ
ಸತ್ಯ-ವಂ-ತ-ನಾ-ಗಿ-ರ-ಬೇ-ಕೆಂದು
ಸತ್ಯ-ವನ್ನು
ಸತ್ಯ-ವಾ-ಗಿತ್ತು
ಸತ್ಯ-ವಾ-ಗಿ-ದ್ದರೆ
ಸತ್ಯವೋ
ಸತ್ಯ-ವ್ರ-ತನೂ
ಸತ್ವ
ಸತ್ವ-ವಿ-ಲ್ಲ-ದಂತೆ
ಸತ್ವವು
ಸದಾ
ಸದೆ-ಬ-ಡಿ-ಯು-ವು-ದ-ರಲ್ಲಿ
ಸದ್ದು-ಕೇ-ಳಿ-ಸಿತು
ಸದ್ಯ
ಸದ್ಯಕ್ಕೆ
ಸದ್ವಿ-ಚಾ-ರ-ಗ-ಳೆಲ್ಲ
ಸನ್ನಿ-ಧಿ-ಯನ್ನು
ಸನ್ನಿ-ವೇ-ಶವೇ
ಸನ್ಮಾ-ನ್ಯರು
ಸನ್ಮಾ-ರ್ಗ-ವನ್ನು
ಸಪ್ಪ-ಳ-ವನ್ನು
ಸಫ-ಲ-ರಾ-ದು-ದಷ್ಟೇ
ಸಫ-ಲವ
ಸಭಿ-ಕರು
ಸಭ್ಯ-ವ್ಯಕ್ತಿ
ಸಮ
ಸಮ-ತ್ವ-ವನ್ನು
ಸಮ-ದೃ-ಷ್ಟಿಯೇ
ಸಮ-ನಾಗಿ
ಸಮಯ
ಸಮ-ಯಕ್ಕೆ
ಸಮ-ಯ-ದಲ್ಲಿ
ಸಮ-ಯ-ದ-ಲ್ಲಿಯೇ
ಸಮ-ರಾಂ-ಗ-ಣ-ದಲ್ಲಿ
ಸಮಸ್ಯೆ
ಸಮ-ಸ್ಯೆಯ
ಸಮಾ
ಸಮಾ-ಚಾರ
ಸಮಾ-ಚಾ-ರ-ವನ್ನು
ಸಮಾಜ
ಸಮಾ-ಜ-ವಾದಿ
ಸಮಾ-ಧಾನ
ಸಮಾ-ಧಾ-ನ-ವಾಗ
ಸಮಾ-ಧಾ-ನ-ವಾ-ಗ-ಲಿಲ್ಲ
ಸಮಾಧಿ
ಸಮಾ-ಧಿಗೆ
ಸಮಾ-ಧಿಯ
ಸಮಾ-ಧಿ-ಯಲ್ಲಿ
ಸಮಾ-ಧಿ-ಸ್ಥಿ-ತಿ-ಯ-ಲ್ಲಿದ್ದ
ಸಮಾ-ಧ್ಯಾ-ಯಿಯು
ಸಮಾನ
ಸಮಾ-ಲೋ-ಚನೆ
ಸಮಿ-ತ್ತನ್ನು
ಸಮೀ-ಪಕ್ಕೆ
ಸಮೀ-ಪ-ದಲ್ಲಿ
ಸಮೀ-ಪ-ದಿಂದ
ಸಮೀ-ಪಿ-ಸಿ-ದಂತೆ
ಸಮೀ-ಪಿ-ಸು-ತ್ತಿ-ದ್ದಂತೆ
ಸಮೀ-ಪಿ-ಸು-ವುದು
ಸಮುದ್ರ
ಸಮು-ದ್ರಕ್ಕೆ
ಸಮು-ದ್ರ-ತೀ-ರ-ದ-ಲ್ಲಿದ್ದ
ಸಮು-ದ್ರದ
ಸಮು-ದ್ರ-ದಲ್ಲಿ
ಸಮು-ದ್ರ-ದಿಂದ
ಸಮು-ದ್ರ-ವನ್ನು
ಸಮ್ಮತಿ
ಸರ-ಪ-ಣಿ-ಗಳು
ಸರ-ಪಳಿ
ಸರ-ಪ-ಳಿ-ಗಳಿಂದ
ಸರಳ
ಸರ-ಳ-ವಾಗಿ
ಸರ-ಳ-ವಾದ
ಸರ-ಸ್ವ-ತಿ-ಯಿಂದ
ಸರಿ-ಪ-ಡಿಸು
ಸರಿ-ಯಾಗಿ
ಸರಿ-ಯಾ-ಗಿ-ರು-ವುದನ್ನು
ಸರಿ-ಯಾ-ಗಿ-ರು-ವು-ದನ್ನೇ
ಸರಿ-ಯಾದ
ಸರಿ-ಯಾ-ಯಿತು
ಸರಿಯೆ
ಸರಿ-ಸ-ಮ-ರಿಲ್ಲ
ಸರೋ-ವರ
ಸರೋ-ವ-ರಕ್ಕೆ
ಸರೋ-ವ-ರದ
ಸರೋ-ವ-ರ-ದಲ್ಲಿ
ಸರ್ಪ
ಸರ್ಪಕ್ಕೆ
ಸರ್ಪ-ದಲ್ಲಿ
ಸರ್ಪವು
ಸರ್ವ-ಜ್ಞ-ನಾದ
ಸರ್ವ-ವನ್ನೂ
ಸರ್ವ-ಸಂಗ
ಸರ್ವೇ
ಸಲ
ಸಲವೂ
ಸಲ-ಹೆ-ಗಳು
ಸಲ-ಹೆ-ಯನ್ನು
ಸಲಿಗೆ
ಸಲು
ಸವರಿ
ಸವಾ-ರ-ನೊ-ಬ್ಬನೆ
ಸವಾ-ರಿ-ಮಾ-ಡು-ತ್ತಿ-ರು-ವ-ವ-ನನ್ನು
ಸವಿ
ಸವಿ-ಯನ್ನು
ಸವೆದು
ಸವೆ-ಸು-ತ್ತಿ-ರುವಿ
ಸಸಿ
ಸಸಿ-ಗಳನ್ನು
ಸಹಜ
ಸಹ-ವಾಸ
ಸಹ-ವಾ-ಸಕ್ಕೆ
ಸಹ-ವಾ-ಸ-ದಿಂದ
ಸಹ-ವಾ-ಸವೇ
ಸಹಾಯ
ಸಹಾ-ಯ-ದಿಂ-ದಲೇ
ಸಹಿ-ಯನ್ನು
ಸಹೋ
ಸಹೋ-ದ-ರನ
ಸಹೋ-ದ-ರ-ನಿಗೆ
ಸಹೋ-ದ-ರನೆ
ಸಾಂಖ್ಯ
ಸಾಂಸಾ-ರಿ-ಕತೆ
ಸಾಂಸಾ-ರಿ-ಕ-ನಿಗೂ
ಸಾಕಾಗಿ
ಸಾಕಾ-ದಷ್ಟು
ಸಾಕಾರ
ಸಾಕಾ-ರ-ನಾ-ಗಲು
ಸಾಕಾ-ರ-ನಾ-ದಲ್ಲಿ
ಸಾಕಿ
ಸಾಕಿ-ದನು
ಸಾಕಿದ್ದ
ಸಾಕು
ಸಾಕು-ವು-ದ-ಕ್ಕಾಗಿ
ಸಾಕ್ಷಾ
ಸಾಕ್ಷಾತ್
ಸಾಕ್ಷಾ-ತ್ಕ-ರಿ-ಸಿ-ಕೊಳ್ಳ
ಸಾಕ್ಷಾ-ತ್ಕಾರ
ಸಾಕ್ಷಾ-ತ್ಕಾ-ರ-ಕ್ಕಾಗಿ
ಸಾಕ್ಷಾ-ತ್ಕಾ-ರಕ್ಕೂ
ಸಾಕ್ಷಾ-ತ್ಕಾ-ರ-ವನ್ನು
ಸಾಕ್ಷಾ-ತ್ಕಾ-ರ-ವಾದ
ಸಾಕ್ಷಾ-ತ್ಕಾ-ರ-ವಾ-ದರೆ
ಸಾಕ್ಷಿ-ಯಾ-ಗಿದೆ
ಸಾಗ-ರದ
ಸಾಗ-ರ-ದಲ್ಲಿ
ಸಾಗ-ರವೇ
ಸಾಗಿ-ದವು
ಸಾಗಿ-ಸಿ-ಕೊಂಡು
ಸಾತ್ವಿಕ
ಸಾಧಕ
ಸಾಧ-ಕ-ರೆಲ್ಲಾ
ಸಾಧನಾ
ಸಾಧ-ನಾ-ವ-ಸ್ಥೆ-ಯಲ್ಲಿ
ಸಾಧನೆ
ಸಾಧ-ನೆಗೆ
ಸಾಧ-ನೆಯ
ಸಾಧ-ನೆ-ಯನ್ನೂ
ಸಾಧ-ನೆ-ಯಿಂದ
ಸಾಧ-ನೆಯೇ
ಸಾಧಾ
ಸಾಧಾ-ರಣ
ಸಾಧಾ-ರ-ಣ-ವಾಗಿ
ಸಾಧಿ-ಸ-ದಿ-ದ್ದರೆ
ಸಾಧಿ-ಸ-ಬ-ಹುದು
ಸಾಧಿ-ಸ-ಬೇ-ಕಾ-ದರೂ
ಸಾಧು
ಸಾಧು-ಗ-ಳಂತೆ
ಸಾಧು-ಗಳು
ಸಾಧು-ಗಳೆ
ಸಾಧು-ಗ-ಳೊಂ-ದಿಗೆ
ಸಾಧು-ಗ-ಳೊ-ಡನೆ
ಸಾಧು-ವನ್ನು
ಸಾಧು-ವಲ್ಲ
ಸಾಧು-ವಾದ
ಸಾಧು-ವಾ-ದರೆ
ಸಾಧುವಿ
ಸಾಧು-ವಿ-ಗ-ನಿ-ಸಿತು
ಸಾಧು-ವಿಗೆ
ಸಾಧು-ವಿನ
ಸಾಧು-ವಿ-ನಂತೆ
ಸಾಧು-ವೊ-ಬ್ಬ-ನನ್ನು
ಸಾಧು-ವೊ-ಬ್ಬ-ನಿಗೆ
ಸಾಧು-ವೊ-ಬ್ಬನು
ಸಾಧ್ಯ
ಸಾಧ್ಯ-ಎಲ್ಲಾ
ಸಾಧ್ಯ-ತೆ-ಗಳನ್ನು
ಸಾಧ್ಯ-ತೆ-ಗ-ಳಿವೆ
ಸಾಧ್ಯ-ತೆ-ಯಿಲ್ಲ
ಸಾಧ್ಯ-ವಾ-ಗ-ಬಾ-ರದು
ಸಾಧ್ಯ-ವಾ-ಗ-ಲಿಲ್ಲ
ಸಾಧ್ಯ-ವಾ-ಗು-ವುದು
ಸಾಧ್ಯ-ವಾ-ದರೆ
ಸಾಧ್ಯ-ವಾ-ಯಿತು
ಸಾಧ್ಯ-ವಿಲ್ಲ
ಸಾಧ್ಯವೆ
ಸಾಧ್ಯವೇ
ಸಾಧ್ಯವೊ
ಸಾಮರ್ಥ್ಯ
ಸಾಮಾ-ನನ್ನು
ಸಾಮಾನು
ಸಾಮಾ-ನು-ಗಳ
ಸಾಮಾ-ನು-ಗಳನ್ನು
ಸಾಮಾ-ನು-ಗಳನ್ನೆಲ್ಲ
ಸಾಮಾನ್ಯ
ಸಾಯಂ-ಕಾಲ
ಸಾಯಂ-ಕಾ-ಲದ
ಸಾಯ-ಲಾ-ರದು
ಸಾಯಿ-ಸಿದೆ
ಸಾಯಿ-ಸು-ವು-ದ-ರ-ಲ್ಲಿ-ದ್ದರು
ಸಾಯುಜ್ಯ
ಸಾಯುವ
ಸಾಯು-ವರು
ಸಾಯು-ವು-ದಿಲ್ಲ
ಸಾಯು-ವುದೂ
ಸಾಯು-ವೆನು
ಸಾಯು-ವೆನೋ
ಸಾರ
ಸಾರ-ಥಿಯ
ಸಾರ-ಥಿ-ಯಾ-ದರೂ
ಸಾರಿ
ಸಾರಿನ
ಸಾರಿ-ಸಿರ
ಸಾರು
ಸಾಲದು
ಸಾಲ-ವಿತ್ತು
ಸಾವ-ನ್ನ-ಪ್ಪು-ತ್ತವೆ
ಸಾವಿಗೆ
ಸಾವಿನ
ಸಾವಿರ
ಸಾವಿ-ರದ
ಸಾವಿ-ರ-ವಾ-ದರೂ
ಸಾವಿ-ರಾರು
ಸಾವು
ಸಾಸಿವೆ
ಸಿಂಗ್
ಸಿಂಗ್ನಾ-ದರೋ
ಸಿಂಥಿಯ
ಸಿಂಹ
ಸಿಂಹಳ
ಸಿಂಹ-ಳ-ದ್ವೀ-ಪಕ್ಕೆ
ಸಿಂಹಾ-ಸ-ನದ
ಸಿಕ್ಕದೆ
ಸಿಕ್ಕ-ಬ-ಹುದು
ಸಿಕ್ಕ-ಲಿಲ್ಲ
ಸಿಕ್ಕಾ-ಪಟ್ಟೆ
ಸಿಕ್ಕಿ
ಸಿಕ್ಕಿ-ಕೊಂ-ಡಿತು
ಸಿಕ್ಕಿ-ಕೊಂ-ಡಿ-ರುವ
ಸಿಕ್ಕಿತು
ಸಿಕ್ಕಿದ
ಸಿಕ್ಕಿ-ದನು
ಸಿಕ್ಕಿ-ದರೆ
ಸಿಕ್ಕಿ-ದ-ವ-ರನ್ನು
ಸಿಕ್ಕಿ-ದಾಗ
ಸಿಕ್ಕಿದೆ
ಸಿಕ್ಕಿ-ದೆಯೆ
ಸಿಕ್ಕಿದ್ದು
ಸಿಕ್ಕಿ-ಬಿದ್ದೆ
ಸಿಕ್ಕಿ-ರ-ಬೇಕು
ಸಿಕ್ಕೀತೆ
ಸಿಕ್ಕು-ತ್ತಿತ್ತು
ಸಿಕ್ಕುವ
ಸಿಕ್ಕು-ವರು
ಸಿಕ್ಕು-ವ-ವ-ನಲ್ಲ
ಸಿಕ್ಕು-ವು-ದಿಲ್ಲ
ಸಿಕ್ಕು-ವುದು
ಸಿಕ್ಕು-ವುವು
ಸಿಖ್ಖರು
ಸಿಗ-ದಿ-ರುವ
ಸಿಗದೆ
ಸಿಗು-ತ್ತಿ-ರುವ
ಸಿಗು-ವುದು
ಸಿಟ್ಟು
ಸಿಡಿಲು
ಸಿದ
ಸಿದರು
ಸಿದೆ
ಸಿದ್ಧ-ನಿಗೆ
ಸಿದ್ಧನು
ಸಿದ್ಧ-ಪು-ರುಷ
ಸಿದ್ಧ-ಪು-ರು-ಷರು
ಸಿದ್ಧ-ರಾ-ದರು
ಸಿದ್ಧಾಂತ
ಸಿದ್ಧಾಂ-ತ-ಗಳನ್ನು
ಸಿದ್ಧಿ-ಗಳನ್ನು
ಸಿದ್ಧಿ-ಗಳು
ಸಿದ್ಧಿ-ಸಿತು
ಸಿದ್ಧಿ-ಸು-ವುದು
ಸಿಹಿ-ತಿಂಡಿ
ಸೀತಾ-ಫ-ಲದ
ಸೀತೆ
ಸೀತೆ-ಯನ್ನು
ಸೀಮೆ-ಸು-ಣ್ಣ-ದಿಂದ
ಸೀಳಿ
ಸುಂದರ
ಸುಂದ-ರ-ವಾಗಿ
ಸುಂದ-ರ-ವಾದ
ಸುಖ
ಸುಖ-ಗಳೂ
ಸುಖ-ದುಃ-ಖ-ಗಳು
ಸುಖ-ವನ್ನು
ಸುಖ-ವಾ-ಗಿದ್ದ
ಸುಖ-ವಾ-ಯಿತು
ಸುಗಂಧ
ಸುಡು-ವು-ದ-ರ-ಲ್ಲಿದ್ದ
ಸುತ್ತ
ಸುತ್ತ-ಮುತ್ತ
ಸುತ್ತಲೂ
ಸುತ್ತಿ
ಸುತ್ತಿ-ಕೊಂಡು
ಸುತ್ತಿದ
ಸುತ್ತಿ-ದರೆ
ಸುತ್ತಿ-ದ್ದಾಗ
ಸುತ್ತಿ-ರು-ತ್ತಾರೆ
ಸುತ್ತು-ಗ-ಟ್ಟಿ-ದನು
ಸುತ್ತು-ವ-ರಿ-ದವು
ಸುತ್ತೇನೆ
ಸುದ್ದಿ
ಸುದ್ದಿ-ಯನ್ನು
ಸುದ್ದಿಯೂ
ಸುಪ್ಪ-ತ್ತಿಗೆ
ಸುಮೇ-ರು-ವಿ-ನಂತೆ
ಸುಮ್ಮ
ಸುಮ್ಮಗೆ
ಸುಮ್ಮ-ನಿ-ರಲು
ಸುಮ್ಮ-ನಿ-ರಿ-ಸಿದೆ
ಸುಮ್ಮನೆ
ಸುರ-ಕ್ಷಿ-ತ-ವಾ-ಗಿ-ದ್ದೇವೆ
ಸುರಿ-ಯು-ತ್ತಿತ್ತು
ಸುರಿ-ಸಿದ
ಸುರಿ-ಸು-ತ್ತಿದ್ದ
ಸುರಿ-ಸು-ತ್ತಿದ್ದೆ
ಸುರಿ-ಸುವ
ಸುರುಳಿ
ಸುಲ-ಭ-ವಲ್ಲ
ಸುಲ-ಭ-ವಾಗಿ
ಸುಲಿ-ಯೋ-ಣವೆ
ಸುಳಿವೇ
ಸುಳ್ಳನ್ನು
ಸುಳ್ಳಾ-ದರೆ
ಸುಳ್ಳಿ-ನಿಂದ
ಸುಳ್ಳು
ಸುಳ್ಳು-ಗಾರ
ಸುಳ್ಳೋ
ಸುವ
ಸುವಳು
ಸುವು-ದಕ್ಕೆ
ಸುಸ್ತಾಗಿ
ಸೂಕ್ಷ್ಮ-ವಾ-ದುದು
ಸೂಚನೆ
ಸೂಚಿ-ಸಿದ
ಸೂಜಿಗೆ
ಸೂಜಿಯ
ಸೂರ್ಯನ
ಸೂಳೆ-ಯಾ-ಗಿ-ದ್ದಾಳೆ
ಸೃಷ್ಟಿ
ಸೃಷ್ಟಿ-ಮಾ-ಡಿ-ರು-ವನು
ಸೃಷ್ಟಿ-ಯ-ನ್ನೆಲ್ಲ
ಸೃಷ್ಟಿ-ಯ-ಲ್ಲವೆ
ಸೃಷ್ಟಿ-ಯೆಲ್ಲ
ಸೃಷ್ಟಿ-ಸಿ-ಕೊ-ಳ್ಳು-ತ್ತಿ-ರು-ವೆವು
ಸೃಷ್ಟಿ-ಸಿ-ದ-ವ-ನೊಂ-ದಿಗೆ
ಸೃಷ್ಟಿ-ಸಿ-ರು-ವನೋ
ಸೆಪ್ಟೆಂ-ಬರ್
ಸೆರೆ-ಮ-ನೆ-ಯ-ಲ್ಲಿ-ದ್ದಾಗ
ಸೆರೆ-ಮ-ನೆ-ಯಿಂದ
ಸೆರೆ-ಯಿಂದ
ಸೆಳೆ-ತ-ದಲ್ಲಿ
ಸೆಳೆ-ದು-ಕೊಂ-ಡರು
ಸೆಳೆ-ಯು-ತ್ತಾನೆ
ಸೆಳೆ-ಯು-ವು-ದ-ರಲ್ಲಿ
ಸೇತುವೆ
ಸೇತು-ವೆಯ
ಸೇತು-ವೆ-ಯನ್ನು
ಸೇದ-ಬೇ-ಕೆಂದು
ಸೇದು
ಸೇದು-ತ್ತಿದ್ದ
ಸೇದು-ತ್ತಿರು
ಸೇದುವ
ಸೇದು-ವು-ದಕ್ಕೆ
ಸೇದು-ವುದು
ಸೇರನ್ನು
ಸೇರಿ
ಸೇರಿ-ಕೊಂ-ಡಂತೆ
ಸೇರಿದ
ಸೇರಿ-ದ-ವನು
ಸೇರಿ-ದ-ವ-ರಲ್ಲ
ಸೇರಿ-ದ-ವರು
ಸೇರಿದ್ದು
ಸೇರಿ-ರು-ವುದು
ಸೇರಿಸಿ
ಸೇರಿ-ಸಿದ್ದ
ಸೇರಿ-ಸಿ-ದ್ದನೊ
ಸೇರು
ಸೇರೇನು
ಸೇವಕ
ಸೇವ-ಕ-ರಾರೂ
ಸೇವ-ನೆ-ಯಿಂದ
ಸೇವಿ-ಸ-ಬೇಕು
ಸೇವಿ-ಸಲು
ಸೇವಿ-ಸು-ವರು
ಸೇವೆ
ಸೇವೆ-ಯಿಂದ
ಸೊಂಟ
ಸೊಂಟ-ದಲ್ಲಿ
ಸೊಂಟ-ದ-ಲ್ಲಿದೆ
ಸೊಂಟ-ದ-ಲ್ಲಿದ್ದ
ಸೊಂಡ-ಲಿ-ನಿಂದ
ಸೊಂಡಿ-ಲನ್ನು
ಸೊಂಡಿಲು
ಸೊಗ-ಸಾಗಿ
ಸೊಗ-ಸಾದ
ಸೊಸೈ-ಟಿ-ಯ-ಲ್ಲಿ-ರುವ
ಸೋಂಕ-ಕೂ-ಡದು
ಸೋಂಕಿ
ಸೋಂಕಿತು
ಸೋಂಕಿ-ಲ್ಲದ
ಸೋಕಿ-ದೊ-ಡ-ನೆಯೆ
ಸೋತು
ಸೋತು-ಹೋದ
ಸೋದ-ರನ
ಸೋದ-ರ-ಳಿಯ
ಸೋಫ
ಸೋಮ-ನಾ-ಥಾ-ನಂದ
ಸೋಮಾರಿ
ಸೋರ-ಲಿಲ್ಲ
ಸೋಲು-ವನು
ಸೋಲು-ವರು
ಸೌಂದ-ರ್ಯಕ್ಕೆ
ಸೌಂದ-ರ್ಯ-ವನ್ನು
ಸೌಕ-ರ್ಯ-ಗಳು
ಸೌಕ-ರ್ಯ-ವನ್ನು
ಸೌದೆ
ಸ್ಕರಿಸಿ
ಸ್ತಂಭಕ್ಕೆ
ಸ್ತಂಭದ
ಸ್ತಂಭ-ವನ್ನು
ಸ್ತುತಿ-ಸ-ಲಾ-ರಂ-ಭಿ-ಸಿದ
ಸ್ತೋತ್ರದ
ಸ್ತ್ರೀಭ-ಕ್ತರು
ಸ್ತ್ರೀಯರೂ
ಸ್ತ್ರೀಯೂ
ಸ್ಥಳ
ಸ್ಥಳಕ್ಕೆ
ಸ್ಥಳ-ದಲ್ಲಿ
ಸ್ಥಳ-ದ-ಲ್ಲಿ-ರುವೆ
ಸ್ಥಳ-ದಿಂದ
ಸ್ಥಳ-ವನ್ನು
ಸ್ಥಾನಕ್ಕೆ
ಸ್ಥಾನದ
ಸ್ಥಾನ-ದಲ್ಲಿ
ಸ್ಥಾಪನೆ
ಸ್ಥಿತಿ
ಸ್ಥಿತಿಗೆ
ಸ್ಥಿತಿ-ಯನ್ನು
ಸ್ಥಿತಿ-ಯ-ಲ್ಲಿತ್ತು
ಸ್ಥಿತಿ-ಯ-ಲ್ಲಿದ್ದ
ಸ್ಥಿತಿ-ಯ-ಲ್ಲಿ-ರು-ವರು
ಸ್ನಾನ
ಸ್ನಾನಕ್ಕೆ
ಸ್ನಾನದ
ಸ್ನಾನ-ಮಾ-ಡಿ-ಕೊಂಡು
ಸ್ನಾನ-ವನ್ನೂ
ಸ್ನಾನ-ವಾದ
ಸ್ನೇಹ
ಸ್ನೇಹ-ವನ್ನು
ಸ್ನೇಹಿತ
ಸ್ನೇಹಿ-ತ-ನಿಗೂ
ಸ್ನೇಹಿ-ತ-ನಿಗೆ
ಸ್ನೇಹಿ-ತನು
ಸ್ನೇಹಿ-ತನೆ
ಸ್ನೇಹಿ-ತನೇ
ಸ್ನೇಹಿ-ತ-ನೊ-ಡನೆ
ಸ್ನೇಹಿ-ತ-ರಂತೆ
ಸ್ನೇಹಿ-ತ-ರಾ-ಗಿ-ದ್ದೆವು
ಸ್ನೇಹಿ-ತ-ರಿಗೆ
ಸ್ನೇಹಿ-ತರು
ಸ್ನೇಹಿ-ತರೆ
ಸ್ನೇಹಿ-ತಳು
ಸ್ನೇಹಿ-ತಳೆ
ಸ್ನೇಹಿತೆ
ಸ್ನೇಹಿ-ತೆಯು
ಸ್ಪರ್ಶ
ಸ್ಪರ್ಶ-ವಾ-ಗ-ಲಿಲ್ಲ
ಸ್ಪಷ್ಟ-ವಾಗಿ
ಸ್ಫಟಿ-ಕದ
ಸ್ಫೂರ್ತಿ
ಸ್ಮರ-ಣೆಯ
ಸ್ಮರ-ಣೆ-ಯನ್ನು
ಸ್ಮರಿ-ಸಿ-ದಿರಿ
ಸ್ಮರಿ-ಸು-ತ್ತಿದ್ದೆ
ಸ್ಮಶಾ-ನ-ದಲ್ಲಿ
ಸ್ವ
ಸ್ವಪ್ನ
ಸ್ವಪ್ನಕ್ಕೆ
ಸ್ವಪ್ನ-ವನ್ನು
ಸ್ವಪ್ರ-ಯತ್ನ
ಸ್ವಪ್ರ-ಯ-ತ್ನದ
ಸ್ವಭಾವ
ಸ್ವಭಾ-ವಕ್ಕೆ
ಸ್ವಭಾ-ವ-ಗಳು
ಸ್ವಭಾ-ವತಃ
ಸ್ವಭಾ-ವ-ದ-ವ-ನಾ-ಗಿ-ರು-ವು-ದ-ರಿಂದ
ಸ್ವಭಾ-ವ-ದ-ವನು
ಸ್ವಭಾ-ವ-ದಿಂದ
ಸ್ವಭಾ-ವ-ವನ್ನು
ಸ್ವಭಾ-ವವೇ
ಸ್ವರೂಪ
ಸ್ವರೂ-ಪ-ವನ್ನು
ಸ್ವರೂ-ಪ-ವೇನು
ಸ್ವರ್ಗ
ಸ್ವರ್ಗಕ್ಕೆ
ಸ್ವರ್ಗ-ದ-ಲ್ಲಿದ್ದ
ಸ್ವರ್ಗವೂ
ಸ್ವಲ್ಪ
ಸ್ವಲ್ಪ-ಕಾಲ
ಸ್ವಲ್ಪ-ಕಾ-ಲದ
ಸ್ವಲ್ಪ-ದ-ರಲ್ಲೇ
ಸ್ವಲ್ಪ-ವನ್ನೂ
ಸ್ವಲ್ಪವೂ
ಸ್ವಲ್ಪವೇ
ಸ್ವಲ್ಪ-ಸ್ವ-ಲ್ಪ-ವಾಗಿ
ಸ್ವಸ್ಥಿ-ತಿಗೆ
ಸ್ವಾಗ-ತಿ-ಸಿ-ದರೆ
ಸ್ವಾತಂತ್ರ್ಯ
ಸ್ವಾತಂ-ತ್ರ್ಯ-ವನ್ನು
ಸ್ವಾತಂ-ತ್ರ್ಯ-ವಿದೆ
ಸ್ವಾತಿ
ಸ್ವಾಭಾ-ವಿ-ಕ-ವ-ಲ್ಲವೇ
ಸ್ವಾಭಾ-ವಿ-ಕ-ವಾಗಿ
ಸ್ವಾಭಾ-ವಿ-ಕವೇ
ಸ್ವಾಮಿ
ಸ್ವಾಮಿ-ಗಳೆ
ಸ್ವಾಮಿ-ಗಳೇ
ಸ್ವಾರ-ಸ್ಯ-ವಾದ
ಸ್ವೀಕ-ರಿ-ಸ-ಬಾ-ರದು
ಸ್ವೀಕ-ರಿ-ಸ-ಬೇ-ಕಾ-ಗಿಲ್ಲ
ಸ್ವೀಕ-ರಿ-ಸ-ಲಿಲ್ಲ
ಸ್ವೀಕ-ರಿಸಿ
ಸ್ವೀಕ-ರಿ-ಸಿದ
ಸ್ವೀಕ-ರಿ-ಸಿ-ದರೆ
ಸ್ವೀಕ-ರಿ-ಸಿದ್ದ
ಸ್ವೀಕ-ರಿಸು
ಸ್ವೀಕ-ರಿ-ಸು-ತ್ತೇನೆ
ಸ್ವೀಕ-ರಿ-ಸು-ವನು
ಸ್ವೀಕ-ರಿ-ಸು-ವ-ನೇನು
ಸ್ವೀಕ-ರಿ-ಸು-ವು-ದಾ-ದರೆ
ಸ್ವೀಕ-ರಿ-ಸು-ವೆನು
ಸ್ವೀಕಾರ
ಹಂಕಾರಿ
ಹಂಗಿ-ಸಿ-ದರು
ಹಂದಿ
ಹಂದಿಯ
ಹಂದಿ-ಯಾಗಿ
ಹಂಬ
ಹಂಬಾ
ಹಂಸರು
ಹಕ್ಕಿ
ಹಕ್ಕಿ-ಗಳು
ಹಕ್ಕಿಯ
ಹಕ್ಕಿ-ಯನ್ನು
ಹಗ-ಲಿ-ರುಳು
ಹಗಲು
ಹಗ್ಗ
ಹಜಾ-ರ-ದಲ್ಲಿ
ಹಟ
ಹಠ
ಹಠ-ಯೋ-ಗದ
ಹಠ-ಯೋ-ಗ-ವನ್ನು
ಹಠ-ಯೋಗಿ
ಹಡ-ಗಿನ
ಹಡಗು
ಹಡೆದು
ಹಣ
ಹಣ-ಸಂ-ಪ-ತ್ತು-ಗಳನ್ನು
ಹಣ-ಕೊಡಿ
ಹಣಕ್ಕೆ
ಹಣದ
ಹಣ-ದಿಂದ
ಹಣ-ವನ್ನು
ಹಣ-ವ-ನ್ನೆಲ್ಲ
ಹಣವೂ
ಹಣ-ವೆಲ್ಲ
ಹಣವೇ
ಹಣೆಯ
ಹಣ್ಣನ್ನು
ಹಣ್ಣು
ಹಣ್ಣು-ಗಳ
ಹಣ್ಣು-ಗಳನ್ನು
ಹಣ್ಣು-ಗ-ಳಲ್ಲ
ಹಣ್ಣು-ಗಳು
ಹತ-ನಾದ
ಹತ-ರಾ-ಗಿ-ದ್ದಾರೆ
ಹತ್ತಿ
ಹತ್ತಿದ
ಹತ್ತಿರ
ಹತ್ತಿ-ರದ
ಹತ್ತಿ-ರ-ದ-ಲ್ಲಿದ್ದ
ಹತ್ತಿ-ರ-ದ-ಲ್ಲಿಯೇ
ಹತ್ತಿ-ರ-ದ-ಲ್ಲಿ-ರುವ
ಹತ್ತಿ-ರ-ದಲ್ಲೇ
ಹತ್ತಿ-ರ-ವಿದ್ದ
ಹತ್ತಿ-ರವೂ
ಹತ್ತಿ-ರವೆ
ಹತ್ತಿ-ರವೇ
ಹತ್ತಿ-ಸು-ತ್ತಿ-ರು-ವರು
ಹತ್ತು
ಹತ್ಯೆ-ಮಾ-ಡಿ-ದ್ದಾನೆ
ಹತ್ಯೆಯ
ಹದಿ-ನಾರು
ಹದಿ-ನಾರೋ
ಹದಿ-ನಾಲ್ಕು
ಹದಿ-ನೇಳೋ
ಹದಿ-ನೈದು
ಹದ್ದನ್ನು
ಹದ್ದಿನ
ಹದ್ದು
ಹದ್ದು-ಗಳು
ಹನಿ
ಹನಿ-ಗಳು
ಹನಿಯ
ಹನಿ-ಯನ್ನು
ಹನಿ-ಹ-ನಿ-ಯಾಗಿ
ಹನು
ಹನು-ಮಂತ
ಹನು-ಮಂ-ತನ
ಹನು-ಮಂ-ತ-ನೆ-ದು-ರಿ-ಗಿ-ಟ್ಟ-ಳು-ಅ-ವನು
ಹನು-ಮಾನ್
ಹನ್ನೆ-ರಡು
ಹನ್ನೊಂ-ದ-ನೆಯ
ಹನ್ನೊಂದು
ಹಬ್ಬ-ವನ್ನು
ಹಬ್ಬಿತು
ಹರ
ಹರ-ಕಲು
ಹರ-ಟು-ತ್ತಿದ್ದ
ಹರಡಿ
ಹರ-ಡಿತು
ಹರ-ಡಿ-ದ್ದರು
ಹರಣ
ಹರ-ಳನ್ನು
ಹರ-ಳು-ಗಳನ್ನು
ಹರ-ಸಲು
ಹರಿ
ಹರಿ-ಹ-ರ-ನಂತೆ
ಹರಿ-ಜನ
ಹರಿದ
ಹರಿ-ದಾ-ಡ-ಬ-ಹುದು
ಹರಿ-ದಿ-ದ್ದವು
ಹರಿದು
ಹರಿ-ದು-ಕೊಂಡು
ಹರಿ-ನಾ-ಮದ
ಹರಿ-ನಾ-ಮ-ವನ್ನು
ಹರಿ-ಬೋಲ್
ಹರಿಯ
ಹರಿ-ಯಿತು
ಹರಿ-ಯು-ತ್ತಿತ್ತು
ಹರಿ-ಯು-ತ್ತಿ-ದ್ದರೂ
ಹರಿ-ಯು-ವುದನ್ನು
ಹರಿವ
ಹಲ
ಹಲ-ದಾರ
ಹಲ-ದಾ-ರ-ಪು-ಕು-ರ-ವೆಂಬ
ಹಲ-ಧಾರಿ
ಹಲ-ಧಾ-ರಿಯ
ಹಲ-ವನ್ನು
ಹಲ-ವ-ರಿ-ರ-ಬ-ಹುದು
ಹಲ-ವರು
ಹಲ-ವಾರು
ಹಲ-ವಿನ
ಹಲವು
ಹಲ್ಲನ್ನು
ಹಲ್ಲು-ಗಳನ್ನು
ಹಲ್ಲು-ಗಳು
ಹಲ್ಲು-ಗ-ಳೆಲ್ಲ
ಹಲ್ಲೂ
ಹಳದಿ
ಹಳೆಯ
ಹಳ್ಳಿಗೆ
ಹಳ್ಳಿ-ಯಲ್ಲಿ
ಹಳ್ಳಿ-ಯ-ಲ್ಲಿದ್ದ
ಹಳ್ಳಿ-ಯ-ವ-ರನ್ನು
ಹಳ್ಳಿ-ಯ-ವ-ರಿಗೆ
ಹಳ್ಳಿ-ಯ-ವರು
ಹವಿ-ಷ್ಯಾನ್ನ
ಹಸಿಯ
ಹಸಿ-ಯುತ್ತ
ಹಸಿ-ರೆ-ಲೆ-ಗಳನ್ನೆಲ್ಲ
ಹಸಿ-ವಾ-ಗಲಿ
ಹಸಿ-ವಾ-ಗಿದೆ
ಹಸಿ-ವಾ-ಯಿತು
ಹಸಿ-ವಿ-ನಿಂದ
ಹಸು
ಹಸು-ಗಳೂ
ಹಸು-ವನ್ನು
ಹಸು-ವಿನ
ಹಸು-ವೊಂದು
ಹಾಕ-ತೊ-ಡ-ಗಿದ
ಹಾಕಲಿ
ಹಾಕ-ಲಿ-ಲ್ಲ-ವಲ್ಲ
ಹಾಕಿ
ಹಾಕಿ-ಕೊಂ-ಡಾಗ
ಹಾಕಿ-ಕೊಂ-ಡಿ-ದ್ದನು
ಹಾಕಿ-ಕೊಂಡು
ಹಾಕಿ-ಕೊ-ಳ್ಳು-ವ-ವನು
ಹಾಕಿತು
ಹಾಕಿದ
ಹಾಕಿ-ದನು
ಹಾಕಿ-ದಳು
ಹಾಕಿ-ದಾಗ
ಹಾಕಿ-ದ್ದಳು
ಹಾಕಿ-ದ್ದವು
ಹಾಕಿಲ್ಲ
ಹಾಕಿ-ಸಿ-ಕೊ-ಳ್ಳಲು
ಹಾಕು
ಹಾಕು-ತ್ತದೆ
ಹಾಕು-ತ್ತಾರೆ
ಹಾಕು-ತ್ತಿ-ರು-ವಾಗ
ಹಾಕು-ವನು
ಹಾಕುವೆ
ಹಾಗಾ
ಹಾಗಾ-ದರೆ
ಹಾಗಿದೆ
ಹಾಗಿ-ರು-ವಾಗ
ಹಾಗೂ
ಹಾಗೆ
ಹಾಗೆಯೆ
ಹಾಗೆಯೇ
ಹಾಗೆ-ಯೇ-ಇತ್ತು
ಹಾಗೇ
ಹಾಗೇ-ನಿಲ್ಲ
ಹಾಗೇನು
ಹಾಜರಾ
ಹಾಜ-ರಾ-ಗಿತ್ತು
ಹಾಜ-ರಾ-ದರು
ಹಾಜ-ರಾ-ನಿಗೆ
ಹಾಜ-ರು-ಮಾ-ಡಿ-ದರು
ಹಾಜರ್
ಹಾಡನ್ನು
ಹಾಡಿ
ಹಾಡಿದ
ಹಾಡು
ಹಾಡು-ಗಳನ್ನು
ಹಾಡು-ಗಳು
ಹಾಡು-ವಾಗ
ಹಾತೊ-ರೆ-ಯು-ತ್ತಿದ್ದೆ
ಹಾದಿಯು
ಹಾನಿ-ಕ-ರ-ವಾ-ದುದು
ಹಾಯಾಗಿ
ಹಾಯಾ-ಗಿತ್ತು
ಹಾರ-ವನ್ನು
ಹಾರ-ವಿತ್ತು
ಹಾರಾಡು
ಹಾರಾ-ಡು-ತ್ತಿ-ರು-ವಾಗ
ಹಾರಾ-ಡು-ತ್ತಿವೆ
ಹಾರಿ
ಹಾರಿ-ಕೊಂಡು
ಹಾರಿತು
ಹಾರಿ-ದನು
ಹಾರಿ-ದರು
ಹಾರು
ಹಾರುಗೆ
ಹಾರು-ವಿ-ಗಾಗಿ
ಹಾರು-ವುದು
ಹಾರು-ವುವು
ಹಾರೈಕೆ
ಹಾಲನ್ನು
ಹಾಲ-ನ್ನೊ-ದ-ಗಿ-ಸಲು
ಹಾಲಿಗೆ
ಹಾಲಿನ
ಹಾಲಿ-ನ-ವ-ಳೊ-ಬ್ಬಳು
ಹಾಲು
ಹಾಲು-ಇ-ವನ್ನು
ಹಾಲು-ಣಿ-ಸಲು
ಹಾಳಾ-ಗಲಿ
ಹಾಳಾಗಿ
ಹಾಳು
ಹಾವನ್ನು
ಹಾವಿಗೂ
ಹಾವಿಗೆ
ಹಾವಿನ
ಹಾವು
ಹಾವೂ
ಹಾವೇ
ಹಾಸಿಗೆ
ಹಾಸಿ-ಗೆಯ
ಹಾಸಿ-ರುವ
ಹಾಸ್ಯ-ಮ-ಯ-ವಾಗಿ
ಹಾಸ್ಯಾ-ಸ್ಪದ
ಹಿಂಜ-ರಿ-ಯ-ಲಿಲ್ಲ
ಹಿಂತಿ
ಹಿಂತಿ-ರುಗಿ
ಹಿಂತಿ-ರು-ಗಿತು
ಹಿಂತಿ-ರು-ಗಿದ
ಹಿಂತಿ-ರು-ಗಿ-ದಾಗ
ಹಿಂತಿ-ರು-ಗು-ತ್ತಿ-ದ್ದಾಗ
ಹಿಂದಿನ
ಹಿಂದಿ-ನಂತೆ
ಹಿಂದಿ-ನಂ-ತೆಯೇ
ಹಿಂದಿ-ನ-ವನಿ
ಹಿಂದಿ-ನಿಂದ
ಹಿಂದಿ-ರುಗಿ
ಹಿಂದಿ-ರು-ಗಿದೆ
ಹಿಂದೂ
ಹಿಂದೂ-ವನ್ನು
ಹಿಂದೆ
ಹಿಂಬಾ
ಹಿಂಬಾ-ಲಿಸಿ
ಹಿಂಬಾ-ಲಿ-ಸಿ-ದವು
ಹಿಂಬಾ-ಲಿ-ಸು-ತ್ತಿತ್ತು
ಹಿಂಬಾ-ಲಿ-ಸು-ವು-ದಿಲ್ಲ
ಹಿಂಸಿ-ಸ-ಲಾರ
ಹಿಂಸಿ-ಸು-ವು-ದಿಲ್ಲ
ಹಿಂಸೆ
ಹಿಂಸೆಯ
ಹಿಂಸೆ-ಯನ್ನು
ಹಿಕ್ಕೆ-ಗಳು
ಹಿಡಿ
ಹಿಡಿ-ದನು
ಹಿಡಿ-ದಳು
ಹಿಡಿ-ದಿ-ರು-ತ್ತವೆ
ಹಿಡಿದು
ಹಿಡಿ-ದು-ಕೊಂ-ಡನು
ಹಿಡಿ-ದು-ಕೊಂ-ಡಿದ್ದ
ಹಿಡಿ-ದು-ಕೊಂಡು
ಹಿಡಿ-ದು-ಕೊ-ಳ್ಳ-ಬೇಕು
ಹಿಡಿ-ದು-ಕೊ-ಳ್ಳು-ವು-ದಕ್ಕೆ
ಹಿಡಿ-ಯಲು
ಹಿಡಿ-ಯಿತು
ಹಿಡಿಯು
ಹಿಡಿ-ಯು-ತ್ತದೆ
ಹಿಡಿ-ಯು-ತ್ತಿದ್ದ
ಹಿಡಿ-ಯು-ತ್ತಿ-ದ್ದಾಗ
ಹಿಡಿ-ಯು-ವ-ವ-ನಿಂದ
ಹಿಡಿ-ಯು-ವು-ದಕ್ಕೆ
ಹಿಡಿ-ಯು-ವು-ದ-ರಲ್ಲಿ
ಹಿಡಿಸ
ಹಿಡಿ-ಸದು
ಹಿಡಿ-ಸಿತು
ಹಿಮ-ದಂತೆ
ಹಿಮ-ರಾ-ಜನ
ಹಿರ-ಣ್ಯಾ-ಕ್ಷ-ನನ್ನು
ಹಿರಿ-ಯ-ನನ್ನು
ಹಿರಿ-ಯ-ನೊ-ಬ್ಬನು
ಹಿಸ-ಬೇ-ಕಾ-ಯಿತು
ಹಿಸಿದೆ
ಹಿಸು-ಕಿ-ದರೆ
ಹೀಗಿದೆ
ಹೀಗಿ-ದೆಯಾ
ಹೀಗೆ
ಹೀಗೆಂದ
ಹೀಗೆಯೆ
ಹೀಗೆಯೇ
ಹೀಗೇ
ಹೀನ-ಕೃ-ತ್ಯ-ವನ್ನು
ಹೀರು
ಹೀರು-ವು-ದ-ರಲ್ಲಿ
ಹುಚ್ಚ
ಹುಚ್ಚನ
ಹುಚ್ಚ-ನಂತೆ
ಹುಚ್ಚ-ನಾ-ಗುವೆ
ಹುಚ್ಚ-ರಂತೆ
ಹುಚ್ಚಿ
ಹುಚ್ಚು
ಹುಚ್ಚು-ತ-ನ-ವನ್ನು
ಹುಚ್ಚು-ಧೈರ್ಯ
ಹುಟ್ಟಿ-ದಂ-ದಿ-ನಿಂದ
ಹುಟ್ಟಿ-ನಿಂದ
ಹುಟ್ಟಿ-ನಿಂ-ದಲೇ
ಹುಟ್ಟಿ-ಬೆ-ಳೆದ
ಹುಟ್ಟು-ಕು-ರು-ಡ-ನಾ-ಗಿದ್ದ
ಹುಟ್ಟುವ
ಹುಡು-ಕಾಡಿ
ಹುಡು-ಕಾ-ಡಿದ
ಹುಡು-ಕಾ-ಡಿ-ದರು
ಹುಡುಕಿ
ಹುಡು-ಕಿ-ಕೊಂಡು
ಹುಡು-ಕು-ತ್ತಿ-ದ್ದರು
ಹುಡು-ಕು-ವ-ವನು
ಹುಡು-ಕು-ವು-ದಕ್ಕೆ
ಹುಡು-ಕು-ವೆವು
ಹುಡುಗ
ಹುಡು-ಗನ
ಹುಡು-ಗ-ನನ್ನು
ಹುಡು-ಗ-ನಾ-ಗಿ-ದ್ದಾಗ
ಹುಡು-ಗ-ನಿಗೆ
ಹುಡು-ಗನು
ಹುಡು-ಗರು
ಹುಡುಗಿ
ಹುಡು-ಗಿಯ
ಹುರಿ
ಹುಲಿ
ಹುಲಿ-ಗಿಂತ
ಹುಲಿ-ಮರಿ
ಹುಲಿ-ಮ-ರಿಯೂ
ಹುಲಿ-ಯನ್ನು
ಹುಲಿ-ಯಾ-ಯಿತು
ಹುಲಿ-ಯೇ-ರೋಗ
ಹುಲಿ-ಯೊಂದು
ಹುಲ್ಲನ್ನು
ಹುಲ್ಲನ್ನೂ
ಹುಲ್ಲನ್ನೇ
ಹುಲ್ಲಿ
ಹುಲ್ಲಿನ
ಹುಲ್ಲಿ-ನಿಂದ
ಹುಲ್ಲು-ಗಾ-ವ-ಲಿ-ನಲ್ಲಿ
ಹುಳು-ವಿ-ನಂತೆ
ಹೂ
ಹೂಗಳನ್ನು
ಹೂಳ-ಲಾಗಿದೆ
ಹೂಳಿ-ಟ್ಟರು
ಹೂಳಿದ್ದ
ಹೂವನ್ನು
ಹೂವಿನ
ಹೂವು
ಹೂವು-ಗಳ
ಹೂವು-ಗಳು
ಹೃ
ಹೃತ್ಪೂ-ರ್ವಕ
ಹೃತ್ಪೂ-ರ್ವ-ಕ-ವಾಗಿ
ಹೃದಯ
ಹೃದ-ಯ-ದಲ್ಲಿ
ಹೃದ-ಯನ
ಹೃದ-ಯ-ನನ್ನು
ಹೃದ-ಯ-ನಿಂದ
ಹೃದ-ಯ-ನಿಗೂ
ಹೃದ-ಯ-ನಿಗೆ
ಹೃದ-ಯನು
ಹೃದ-ಯ-ಮಂ-ದಿ-ರ-ದಲ್ಲಿ
ಹೃದ-ಯ-ಲ್ಲಿ-ರು-ವುದನ್ನು
ಹೃದ-ಯ-ವನ್ನು
ಹೃದ-ಯಾ-ಕಾಶ
ಹೃನ್ಮನ
ಹೆಂಗ-ಸನ್ನು
ಹೆಂಗ-ಸರ
ಹೆಂಗ-ಸ-ರಲ್ಲೂ
ಹೆಂಗ-ಸರು
ಹೆಂಗ-ಸಾಗಿ
ಹೆಂಗ-ಸಿ-ಗಾಗಿ
ಹೆಂಗ-ಸಿಗೂ
ಹೆಂಗ-ಸಿಗೆ
ಹೆಂಗ-ಸಿನ
ಹೆಂಗ-ಸಿ-ನಿಂ-ದಲೇ
ಹೆಂಗಸು
ಹೆಂಗಸೆ
ಹೆಂಡಂ-ದಿ-ರಿ-ದ್ದರು
ಹೆಂಡತಿ
ಹೆಂಡ-ತಿ-ಗಿಂತ
ಹೆಂಡ-ತಿಗೆ
ಹೆಂಡ-ತಿಯ
ಹೆಂಡ-ತಿ-ಯನ್ನು
ಹೆಂಡ-ತಿ-ಯ-ರಿ-ಬ್ಬರು
ಹೆಂಡ-ತಿ-ಯಾ-ಗಿ-ರು-ವಾಗ
ಹೆಂಡ-ತಿಯೂ
ಹೆಂಡ-ತಿ-ಯೊ-ಬ್ಬಳು
ಹೆಂಡ-ವನ್ನು
ಹೆಕ್ಕುವ
ಹೆಚ್ಚನ್ನು
ಹೆಚ್ಚನ್ನೇ
ಹೆಚ್ಚಾಗಿ
ಹೆಚ್ಚಾ-ಗು-ವುದು
ಹೆಚ್ಚಾ-ಯಿತು
ಹೆಚ್ಚಿ-ಸಿದ
ಹೆಚ್ಚಿ-ಸು-ತ್ತಿ-ರುವೆ
ಹೆಚ್ಚು
ಹೆಚ್ಚುತ್ತಾ
ಹೆಡೆ-ಯನ್ನು
ಹೆಣ-ವನ್ನು
ಹೆಣೆ-ಯು-ತ್ತಿ-ದ್ದಳು
ಹೆಣ್ಣು
ಹೆದ-ರುವೆ
ಹೆಮ್ಮೆ
ಹೆಮ್ಮೆ-ಯಿಂದ
ಹೆಸ-ರನ್ನು
ಹೆಸ-ರನ್ನೇ
ಹೆಸ-ರಿ-ಟ್ಟರು
ಹೆಸ-ರಿ-ನಲ್ಲಿ
ಹೆಸ-ರಿ-ನ-ಲ್ಲಿತ್ತು
ಹೆಸ-ರಿ-ನಿಂದ
ಹೆಸರು
ಹೆಸ-ರು-ಬಂತು
ಹೆಸರೇ
ಹೆಸ-ರೇನು
ಹೇ
ಹೇಗ-ನಿ-ಸಿತು
ಹೇಗಾ-ದರೂ
ಹೇಗಾ-ಯಿತು
ಹೇಗಿತ್ತು
ಹೇಗಿ-ತ್ತೆಂ-ದ-ರೆ-ನನ್ನ
ಹೇಗಿತ್ತೋ
ಹೇಗಿದೆ
ಹೇಗಿ-ದ್ದೀರಿ
ಹೇಗಿ-ರು-ತ್ತವೆ
ಹೇಗಿ-ರು-ತ್ತಾನೆ
ಹೇಗಿ-ರು-ವನೋ
ಹೇಗಿ-ರು-ವರು
ಹೇಗಿ-ರುವೆ
ಹೇಗೆ
ಹೇಗೋ
ಹೇರಿ-ರುವ
ಹೇಳ
ಹೇಳ-ಕೂ-ಡದೆ
ಹೇಳ-ತೀ-ರದು
ಹೇಳದೆ
ಹೇಳ-ಬ-ಲ್ಲರು
ಹೇಳ-ಬ-ಹುದು
ಹೇಳ-ಬೇ-ಕಾ-ಗು-ವುದು
ಹೇಳ-ಬೇಕು
ಹೇಳ-ಬೇಕೊ
ಹೇಳ-ಬೇಕೋ
ಹೇಳ-ಬೇಡ
ಹೇಳ-ಲಾಗ
ಹೇಳ-ಲಾ-ಗ-ಲಿಲ್ಲ
ಹೇಳ-ಲಿಲ್ಲ
ಹೇಳಲು
ಹೇಳಿ
ಹೇಳಿ-ಕ-ಳು-ಹಿ-ಸ-ಬೇ-ಕಾ-ಗಿ-ರ-ಲಿಲ್ಲ
ಹೇಳಿ-ಕ-ಳು-ಹಿ-ಸಿ-ದರು
ಹೇಳಿ-ಕೊಂಡ
ಹೇಳಿ-ಕೊಂ-ಡರು
ಹೇಳಿತು
ಹೇಳಿದ
ಹೇಳಿ-ದಂತೆ
ಹೇಳಿ-ದನು
ಹೇಳಿ-ದರು
ಹೇಳಿ-ದರೂ
ಹೇಳಿ-ದರೆ
ಹೇಳಿ-ದಳು
ಹೇಳಿ-ದ-ವನು
ಹೇಳಿ-ದಾಗ
ಹೇಳಿ-ದಿರಿ
ಹೇಳಿ-ದು-ದ-ನ್ನೆಲ್ಲ
ಹೇಳಿದೆ
ಹೇಳಿ-ದೆ-ಯಲ್ಲ
ಹೇಳಿ-ದೊ-ಡನೆ
ಹೇಳಿದ್ದ
ಹೇಳಿ-ದ್ದ-ರಿಂದ
ಹೇಳಿ-ದ್ದರು
ಹೇಳಿ-ದ್ದರೆ
ಹೇಳಿದ್ದು
ಹೇಳಿ-ದ್ದೆಲ್ಲ
ಹೇಳಿ-ಬಿ-ಡ-ಬ-ಹು-ದಾ-ಗಿ-ತ್ತಲ್ಲ
ಹೇಳಿರು
ಹೇಳಿ-ರು-ವಂತೆ
ಹೇಳಿ-ರು-ವನು
ಹೇಳಿ-ರುವೆ
ಹೇಳಿ-ರು-ವೆನು
ಹೇಳಿಲ್ಲ
ಹೇಳು
ಹೇಳುತ್ತ
ಹೇಳು-ತ್ತಾನೆ
ಹೇಳು-ತ್ತಾನೊ
ಹೇಳು-ತ್ತಾನೋ
ಹೇಳು-ತ್ತಾರೆ
ಹೇಳು-ತ್ತಾರೋ
ಹೇಳು-ತ್ತಾಳೆ
ಹೇಳುತ್ತಿ
ಹೇಳು-ತ್ತಿದ್ದ
ಹೇಳು-ತ್ತಿ-ದ್ದನು
ಹೇಳು-ತ್ತಿ-ದ್ದರು
ಹೇಳು-ತ್ತಿ-ದ್ದಾಗ
ಹೇಳು-ತ್ತಿರು
ಹೇಳು-ತ್ತಿ-ರು-ವ-ನಲ್ಲ
ಹೇಳು-ತ್ತಿ-ರು-ವನು
ಹೇಳು-ತ್ತಿ-ರು-ವುದನ್ನು
ಹೇಳು-ತ್ತಿ-ರು-ವು-ದ-ರಿಂದ
ಹೇಳು-ತ್ತಿ-ರುವೆ
ಹೇಳು-ತ್ತೀಯ
ಹೇಳು-ತ್ತೀರಾ
ಹೇಳು-ತ್ತೇನೆ
ಹೇಳು-ತ್ತೇ-ವೆಯೋ
ಹೇಳು-ವನು
ಹೇಳು-ವ-ವ-ರ-ಲ್ಲಿಯೂ
ಹೇಳು-ವ-ವ-ರಿಗೆ
ಹೇಳು-ವ-ವರು
ಹೇಳು-ವಾಗ
ಹೇಳು-ವಿರಾ
ಹೇಳು-ವುದ
ಹೇಳು-ವು-ದಕ್ಕೆ
ಹೇಳು-ವುದನ್ನು
ಹೇಳು-ವು-ದಿಲ್ಲ
ಹೇಳು-ವುದು
ಹೇಳು-ವೆನು
ಹೇಸಿಗೆ
ಹೇಸಿ-ಗೆ-ಯನ್ನು
ಹೇಸಿ-ಗೆ-ಯನ್ನೇ
ಹೇಸಿ-ಗೆ-ಯಿಂದ
ಹೊಂದಿ-ರುವ
ಹೊಂದಿ-ರು-ವನು
ಹೊಂಬ-ಣ್ಣ-ವಿರು
ಹೊಕ್ಕು
ಹೊಗ-ಳ-ಬೇಕು
ಹೊಗ-ಳಿದ
ಹೊಗ-ಳು-ತ್ತೀರಿ
ಹೊಗ-ಳು-ಭ-ಟ್ಟ-ರಿಗೂ
ಹೊಗ-ಳು-ವು-ದ-ಕ್ಕಾಗಿ
ಹೊಗೆ
ಹೊಟ್ಟೆ
ಹೊಟ್ಟೆಯ
ಹೊಟ್ಟೆ-ಯನ್ನು
ಹೊಡೆದ
ಹೊಡೆ-ದಂತೆ
ಹೊಡೆ-ದನು
ಹೊಡೆ-ದರೊ
ಹೊಡೆ-ದಳು
ಹೊಡೆ-ದಾರು
ಹೊಡೆ-ದೋ-ಡಿಸು
ಹೊಡೆ-ಯ-ಬೇ-ಕಾ-ಗು-ವುದು
ಹೊಡೆ-ಯ-ಬೇ-ಡ-ವೆಂದು
ಹೊಡೆ-ಯಲು
ಹೊಡೆ-ಯುತ್ತ
ಹೊಡೆ-ಯು-ತ್ತಾರೆ
ಹೊಡೆ-ಯು-ತ್ತಿ-ದ್ದು-ದನ್ನು
ಹೊಡೆ-ಯು-ವು-ದ-ರ-ಲ್ಲಿದ್ದ
ಹೊಡೆ-ಯು-ವು-ದ-ರ-ಲ್ಲಿ-ದ್ದರು
ಹೊಡೆ-ಯು-ವು-ದಿಲ್ಲ
ಹೊತ್ತಾಗಿ
ಹೊತ್ತಾ-ಗಿದೆ
ಹೊತ್ತಾದ
ಹೊತ್ತಾ-ದರೂ
ಹೊತ್ತಿಗೆ
ಹೊತ್ತಿಗೇ
ಹೊತ್ತಿನ
ಹೊತ್ತಿ-ನಲ್ಲಿ
ಹೊತ್ತು
ಹೊತ್ತು-ಕೊಂಡು
ಹೊನ್ನು
ಹೊರ
ಹೊರಕ್ಕೆ
ಹೊರ-ಗ-ಟ್ಟು-ತ್ತಿ-ರು-ವಳು
ಹೊರಗೆ
ಹೊರ-ಚಾ-ಚಿ-ದ್ದರೂ
ಹೊರ-ಜ-ಗ-ತ್ತಿನ
ಹೊರಟ
ಹೊರ-ಟನು
ಹೊರ-ಟರು
ಹೊರ-ಟ-ವನು
ಹೊರ-ಟಿತೋ
ಹೊರ-ಟಿದ್ದ
ಹೊರ-ಟಿರಿ
ಹೊರ-ಟು-ಹೋ-ಗು-ತ್ತಿ-ದ್ದೀರಿ
ಹೊರ-ಟು-ಹೋ-ಗು-ವುದು
ಹೊರ-ಟು-ಹೋದ
ಹೊರ-ಟು-ಹೋ-ದರು
ಹೊರ-ಟು-ಹೋ-ದಳು
ಹೊರ-ಟು-ಹೋ-ಯಿತು
ಹೊರ-ಟೇ-ಹೋದ
ಹೊರ-ಟೇ-ಹೋದೆ
ಹೊರ-ಡ-ಲಿಲ್ಲ
ಹೊರಡು
ಹೊರ-ಡು-ತ್ತೇನೆ
ಹೊರತು
ಹೊರ-ಬಂದ
ಹೊರ-ಬಂದು
ಹೊರ-ಳಾಡಿ
ಹೊರ-ಳಾ-ಡಿ-ದು-ದ-ರಿಂದ
ಹೊರ-ಳಾಡು
ಹೊರ-ಳಾ-ಡು-ತ್ತಿ-ದ್ದಳು
ಹೊರಿಸಿ
ಹೊರಿ-ಸಿ-ದರು
ಹೊರು-ತ್ತಿದ್ದ
ಹೊರು-ವಂತೆ
ಹೊಲಕ್ಕೆ
ಹೊಲದ
ಹೊಲ-ದಲ್ಲಿ
ಹೊಳೆದ
ಹೊಳೆ-ಯಿತು
ಹೊಸ
ಹೊಸ-ದಾಗಿ
ಹೊಸ-ದೊಂದು
ಹೋಗ
ಹೋಗ-ದಿರ
ಹೋಗ-ಬ-ಹುದು
ಹೋಗ-ಬಾ-ರದು
ಹೋಗ-ಬೇ-ಕಾಗಿ
ಹೋಗ-ಬೇ-ಕಾ-ಗಿದೆ
ಹೋಗ-ಬೇ-ಕಾಗು
ಹೋಗ-ಬೇ-ಕಾ-ಯಿತು
ಹೋಗ-ಬೇಕು
ಹೋಗ-ಬೇ-ಕೆಂದು
ಹೋಗ-ಬೇಡಿ
ಹೋಗ-ಲಾ-ಡಿ-ಸ-ಲಿಲ್ಲ
ಹೋಗ-ಲಿಲ್ಲ
ಹೋಗಲು
ಹೋಗಾಚೆ
ಹೋಗಿ
ಹೋಗಿತ್ತು
ಹೋಗಿದ್ದ
ಹೋಗಿ-ದ್ದಾಗ
ಹೋಗಿದ್ದೆ
ಹೋಗಿ-ಬಿ-ಟ್ಟಿರು
ಹೋಗಿರು
ಹೋಗಿ-ರುವೆ
ಹೋಗಿಲ್ಲ
ಹೋಗಿವೆ
ಹೋಗು
ಹೋಗು-ತ್ತವೆ
ಹೋಗು-ತ್ತಾರೆ
ಹೋಗು-ತ್ತಾರೋ
ಹೋಗು-ತ್ತಿತ್ತು
ಹೋಗು-ತ್ತಿದೆ
ಹೋಗು-ತ್ತಿದ್ದ
ಹೋಗು-ತ್ತಿ-ದ್ದನು
ಹೋಗು-ತ್ತಿ-ದ್ದರು
ಹೋಗು-ತ್ತಿ-ದ್ದಳು
ಹೋಗು-ತ್ತಿ-ದ್ದ-ವನು
ಹೋಗು-ತ್ತಿ-ದ್ದ-ವ-ನೊಬ್ಬ
ಹೋಗು-ತ್ತಿ-ದ್ದಾಗ
ಹೋಗು-ತ್ತಿ-ದ್ದಿರಿ
ಹೋಗು-ತ್ತಿ-ದ್ದೀ-ರಲ್ಲ
ಹೋಗು-ತ್ತಿ-ದ್ದು-ದನ್ನು
ಹೋಗು-ತ್ತಿದ್ದೆ
ಹೋಗು-ತ್ತಿ-ರ-ಲಿಲ್ಲ
ಹೋಗು-ತ್ತಿ-ರು-ವಳು
ಹೋಗು-ತ್ತಿ-ರು-ವಾಗ
ಹೋಗು-ತ್ತಿ-ರುವೆ
ಹೋಗು-ತ್ತಿ-ರು-ವೆನು
ಹೋಗು-ತ್ತೀ-ಯಲ್ಲ
ಹೋಗು-ತ್ತೇನೆ
ಹೋಗು-ತ್ತೇವೆ
ಹೋಗುವ
ಹೋಗು-ವನೊ
ಹೋಗು-ವಾಗ
ಹೋಗು-ವು-ದಕ್ಕೆ
ಹೋಗು-ವುದನ್ನು
ಹೋಗು-ವು-ದರ
ಹೋಗು-ವು-ದ-ರ-ಲ್ಲಿ-ದ್ದನು
ಹೋಗು-ವು-ದಿಲ್ಲ
ಹೋಗು-ವುದು
ಹೋಗುವೆ
ಹೋಗು-ವೆನು
ಹೋಗು-ವೆ-ಯೇನು
ಹೋಗೇ
ಹೋಗೋಣ
ಹೋದ
ಹೋದಂತೆ
ಹೋದಂ-ತೆಲ್ಲ
ಹೋದ-ನಂ-ತರ
ಹೋದನು
ಹೋದ-ಮೇಲೆ
ಹೋದರು
ಹೋದರೂ
ಹೋದರೆ
ಹೋದಲ್ಲೆಲ್ಲಾ
ಹೋದಳು
ಹೋದ-ವ-ನನ್ನು
ಹೋದವು
ಹೋದಾಗ
ಹೋದಿರಿ
ಹೋದೆ
ಹೋದೆಆ
ಹೋದೆವು
ಹೋದೊ-ಡ-ನೆಯೇ
ಹೋಮ
ಹೋಯಿತು
ಹೋಲಿಸು
ಹೌದು
ಹೌದೆ
ಹ್ಞೂಗು-ಟ್ಟ-ಲಿಲ್ಲ
ುಕರಿ-ಸು-ವು-ದಿಲ್ಲ
್ಣದೃ-ಷ್ಟಿಗೆ
}
