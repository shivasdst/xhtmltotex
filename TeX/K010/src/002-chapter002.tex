
\chapter{ಶ್ರದ್ಧೆ}

\section{\num{೬ಂ.} ಮಗುವಿನಂತೆ ಶ್ರದ್ಧೆ ಇರಬೇಕು}

ಜಟಿಲ ಎಂಬ ಹುಡುಗ ಒಂದು ಕಾಡಿನ ಮೂಲಕ ಶಾಲೆಗೆ ಹೋಗಿ ಬರು ತ್ತಿದ್ದ. ದಾರಿಯಲ್ಲಿ ಒಂದು ದೊಡ್ಡ ಅರಣ್ಯವಿತ್ತು. ಅಲ್ಲಿ ಹೋಗುವಾಗ ಅವನಿಗೆ ಭಯವಾಗುತ್ತಿತ್ತು. ಒಂದು ದಿನ ತನ್ನ ತಾಯಿಗೆ ಇದನ್ನು ಹೇಳಿದನು. “ನೀನು ಏತಕ್ಕೆ ಅಂಜುವುದು. ಮಧುಸೂದನನನ್ನು ಕರೆ” ಎಂದಳು ತಾಯಿ. “ಅಮ್ಮ ಮಧುಸೂದನ ಯಾರು?” ಎಂದು ಅವನು ಕೇಳಿದನು. “ಅವನು ನಿನ್ನ ಅಣ್ಣ” ಎಂದಳು ತಾಯಿ. ಇದಾದ ನಂತರ ಒಂದು ದಿನ ಹುಡುಗ ಶಾಲೆಗೆ ಹೋಗುತ್ತಿದ್ದಾಗ ಕಾಡಿನಲ್ಲಿ ಅವನಿಗೆ ಅಂಜಿಕೆಯುಂಟಾಗಿ\\“ಅಣ್ಣ, ಮಧುಸೂದನ” ಎಂದು ಕರೆದನು. ಆದರೆ ಅವನಿಗೆ\\ಯಾವ ಪ್ರತಿ ಉತ್ತರವೂ ಬರಲಿಲ್ಲ. ಹುಡುಗ ತುಂಬಾ ಅಳುತ್ತ,\\“ನನ್ನ ಅಣ್ಣ, ಮಧುಸೂದನ, ನೀನೆಲ್ಲಿರುವೆ? ನನ್ನ ಹತ್ತಿರ ಬಾ.\\ನನಗೆ ಅಂಜಿಕೆಯಾಗಿದೆ” ಎಂದನು. ದೇವರಿಗೆ ಇನ್ನು ಅವನಿಂದ ದೂರ ಇರಲು ಸಾಧ್ಯವಾಗಲಿಲ್ಲ. ಅವನು ಹುಡುಗನ ಎದುರಿಗೆ ಬಂದು “ನೋಡು ನಾನು ಇಲ್ಲಿದ್ದೇನೆ, ನೀನೇಕೆ ಹೆದರುವೆ?” ಎನ್ನುತ್ತಾ ಹುಡುಗನನ್ನು ಕಾಡಿನಿಂದ ಹೊರಗೆ ಕರೆದುಕೊಂಡು ಹೋಗಿ ದಾರಿಯನ್ನು ತೋರಿಸಿದನು. ಹುಡುಗನನ್ನು ಬೀಳ್ಕೊಡುವಾಗ ದೇವರು, “ನೀನು ಯಾವಾಗ ಕರೆಯುವೆಯೋ ಆಗಲೆಲ್ಲ ನಾನು ಬರುವೆನು. ಭಯಪಡಬೇಡ” ಎಂದನು.

ಮಗುವಿನಲ್ಲಿರುವ ಈ ಶ್ರದ್ಧೆ, ಈ ವ್ಯಾಕುಲತೆ ಇರಬೇಕು.


\section{\num{೬೧. } ಒಬ್ಬ ಹುಡುಗ ನಿಜವಾಗಿ ದೇವರಿಗೆ ಉಣಿಸಿದನು}

ಒಬ್ಬ ಬ್ರಾಹ್ಮಣ ತನ್ನ ಕುಲದೇವರನ್ನು ನೈವೇದ್ಯದೊಂದಿಗೆ ಪೂಜಿಸುತ್ತಿ ದ್ದನು. ಒಂದು ದಿನ ಅವನು ಕಾರ್ಯನಿಮಿತ್ತ ಹೊರಗೆ ಹೋಗಬೇಕಾಗಿ ಬಂತು. ಮನೆಯನ್ನು ಬಿಡುವಾಗ ತನ್ನ ಮಗನಿಗೆ “ನೀನು ದೇವರಿಗೆ ನೈವೇದ್ಯವನ್ನು ಕೊಡು. ದೇವರಿಗೆ ಉಣಿಸುವುದನ್ನು ಮರೆಯಬೇಡ” ಎಂದನು. ಹುಡುಗ ದೇವರಮನೆಯಲ್ಲಿ ದೇವರಿಗೆ ನೈವೇದ್ಯವನ್ನು ಕೊಟ್ಟನು. ಆದರೆ ವಿಗ್ರಹ ಸ್ವಲ್ಪವೂ ಚಲಿಸಲಿಲ್ಲ. ತನ್ನ ಪೀಠದ ಮೇಲೆ ಸುಮ್ಮನೆ ಕುಳಿತಿತ್ತು. ಬಹಳ ಕಾಲದವರೆಗೆ ಹುಡುಗ ಅಲ್ಲೇ ಇದ್ದ. ಆದರೂ ವಿಗ್ರಹ ಸ್ವಲ್ಪವೂ ಕದಲಲಿಲ್ಲ. ಹುಡುಗನು, ದೇವರು ಪೀಠದಿಂದ ಕೆಳಗೆ ಬಂದು ಆಹಾರವನ್ನು ಸ್ವೀಕರಿಸುವನು ಎಂದು ಪೂರ್ಣ ನಂಬಿದ್ದ. ಮತ್ತೆ ಮತ್ತೆ ಅವನು ಪ್ರಾರ್ಥಿಸಿದನು: “ದೇವರೇ, ಇಳಿದು ಬಂದು ಆಹಾರವನ್ನು ಸ್ವೀಕರಿಸು. ಆಗಲೆ ವೇಳೆ ಆಗಿದೆ. ನಾನು ಇನ್ನು ತುಂಬಾ ಹೊತ್ತು ಕಾದಿರಲಾರೆ.” ಆದರೆ ವಿಗ್ರಹ ಒಂದು ಮಾತನ್ನೂ ಆಡಲಿಲ್ಲ. ಹುಡುಗನು ಅಳತೊಡಗಿದ. “ದೇವರೆ, ನನ್ನ ತಂದೆ ನಿನಗೆ ಊಟ ಕೊಡು ಎಂದು ಹೇಳಿರುವನು. ನೀನು ಏತಕ್ಕೆ ಇಳಿದು ಬರಬಾರದು? ನನ್ನ ಕೈಯಿಂದ ಏತಕ್ಕೆ ಊಟ ಸ್ವೀಕರಿಸಬಾರದು?” ಎಂದು ಮೊರೆಯಿಟ್ಟ. ಬಹಳ ಕಳಕಳಿ ಯಿಂದ ಪ್ರಾರ್ಥಿಸಿದ. ಭಗವಂತ ಮಂದಹಾಸದಿಂದ ತನ್ನ ಪೀಠ ದಿಂದ ಇಳಿದು ಬಂದು, ಹುಡುಗ ಕೊಟ್ಟಿದ್ದನ್ನು ಸ್ವೀಕರಿಸಿದ. ದೇವರಿಗೆ ಊಟ ಮಾಡಿಸಿದ ಮೇಲೆ ಇವನು ದೇವರಕೋಣೆ ಯಿಂದ ಹೊರಬಂದ. ಇವನ ನೆಂಟರು, “ಈಗ ಪ್ರಸಾದವನ್ನು ತೆಗೆದುಕೊಂಡು ಬಾ” ಎಂದರು. “ಪೂಜೆ ಮುಗಿಯಿತು. ನಾನು ಕೊಟ್ಟಿದ್ದನ್ನೆಲ್ಲ ದೇವರು ತಿಂದುಬಿಟ್ಟ” ಎಂದನು. ಅವನ ನೆಂಟರು “ಇದು ಹೇಗೆ ಸಾಧ್ಯ?” ಎಂದು ಕೇಳಿದರು. ಹುಡುಗನು ಮುಗ್ಧ ಭಾವದಲ್ಲಿ, “ದೇವರು ಕೊಟ್ಟಿದ್ದನ್ನು ತಿಂದು ಬಿಟ್ಟ” ಎಂದನು. ನೆಂಟರು ದೇವರ ಮನೆಗೆ ಹೋಗಿ ನೋಡಿದಾಗ, ಆಶ್ಚರ್ಯಚಕಿತರಾದರು. ಹುಡುಗನು ಕೊಟ್ಟ ನೈವೇದ್ಯವನ್ನು ಭಗವಂತ ಸಂಪೂರ್ಣವಾಗಿ ಸ್ವೀಕರಿಸಿದ್ದ.


\section{\num{೬೨. } ಶಿಷ್ಯೆ ಮತ್ತು ಆಕೆಯ ಮೊಸರಿನ ಕುಡಿಕೆ}

ಒಂದು ಸಲ ಗುರುಗಳೊಬ್ಬರ ಮನೆಯಲ್ಲಿ ಅನ್ನಪ್ರಾಶನ ನಡೆಯಿತು. ಅವರ ಶಿಷ್ಯರು ಯಥಾಶಕ್ತಿ ಗುರುವಿಗೆ ಬೇಕಾದ ಕಾಣಿಕೆಗಳನ್ನು ಕೊಟ್ಟರು. ಅವರಿಗೆ ಒಬ್ಬ ಬಡ ಶಿಷ್ಯೆಯಿದ್ದಳು. ಅವಳು ವಿಧವೆ. ಅವಳ ಹತ್ತಿರ ಒಂದು ಆಕಳಿತ್ತು. ಅವಳು ಹಸುವಿನ ಹಾಲನ್ನು ಕರೆದು ಗುರುಗಳಿಗೆ ಒಂದು ಕುಡಿಕೆ ಹಾಲನ್ನು ಕೊಟ್ಟಳು. ಅಂದಿಗೆ ಬೇಕಾಗುವ ಹಾಲು ಮೊಸರು ಎರಡನ್ನೂ ಅವಳು ಒದಗಿ ಸುವಳು ಎಂದು ಗುರುಗಳು ಭಾವಿಸಿದ್ದರು. ಅವಳು ಕೊಟ್ಟ ಅಲ್ಪ ಹಾಲನ್ನು ನೋಡಿ ಅವರಿಗೆ ಸಿಟ್ಟು ಬಂದು ಆ ಹಾಲನ್ನು ಚೆಲ್ಲಿಬಿಟ್ಟರು. ಕೋಪದಿಂದ, “ನೀರಿನಲ್ಲಿ ಮುಳುಗಿ ಹೋಗು” ಎಂದು ಶಾಪ ಕೊಟ್ಟರು. ವಿಧವೆಯು ‘ಇದು ಗುರುವಿನ ಆಜ್ಞೆ’ ಎಂದು ಭಾವಿಸಿ ನೀರಿನಲ್ಲಿ ಮುಳುಗಲು ಹೋದಳು. ಆಕೆಯ ಸರಳ ಶ್ರದ್ಧೆಯಿಂದ ಸಂತೃಪ್ತನಾದ ದೇವರು ಅವಳಿಗೆ ಕಾಣಿಸಿಕೊಂಡು, “ಈ ಮೊಸರಿನ ಕುಡಿಕೆಯನ್ನು ತೆಗೆದುಕೊ. ಇದನ್ನೆಂದಿಗೂ ಖಾಲಿ ಮಾಡುವುದಕ್ಕೆ ಆಗುವುದಿಲ್ಲ. ನೀನು ಎಷ್ಟು ಮೊಸರನ್ನು ತೆಗೆದರೂ ಅಷ್ಟು ಹೊಸ ಮೊಸರು ಮತ್ತೆ ಬರುವುದು. ಇದರಿಂದ ನಿನ್ನ ಗುರುವಿಗೆ ತೃಪ್ತಿಯಾಗುವುದು,” ಎಂದು ಹೇಳಿ ಒಂದು ಮೊಸರಿನ ಪಾತ್ರೆಯನ್ನು ಆಕೆಗೆ ಕೊಟ್ಟನು. ಗುರುಗಳಿಗೆ ಮೊಸರಿನ ಪಾತ್ರೆಯನ್ನು ಕೊಟ್ಟಾಗ ಅವರಿಗೆ ಆಶ್ಚರ್ಯವಾಯಿತು. ವಿಧವೆಯಿಂದ ಈ ಘಟನೆಯನ್ನು ಕೇಳಿದ ಗುರುಗಳು “ನೀನು ದೇವರನ್ನು ತೋರಿಸದೆ ಇದ್ದರೆ ನಾನು ನೀರಿನಲ್ಲಿ ಬಿದ್ದು ಸಾಯುವೆನು” ಎಂದರು. ತಕ್ಷಣವೆ ದೇವರು ಪ್ರತ್ಯಕ್ಷನಾದನು. ಆದರೆ ಗುರುಗಳಿಗೆ ಅವನನ್ನು ನೋಡಲಾಗಲಿಲ್ಲ. ದೇವರಿಗೆ ವಿಧವೆ ಹೇಳಿದಳು: “ನನ್ನ ಗುರುಗಳು ನಿನ್ನನ್ನು ನೋಡಲಾಗದೆ ಆತ್ಮಹತ್ಯೆ ಮಾಡಿಕೊಂಡರೆ ನಾನೂ ಆತ್ಮಹತ್ಯೆ ಮಾಡಿಕೊಳ್ಳುತ್ತೇನೆ,” ಎಂದು. ಆಗ ದೇವರು ಗುರುಗಳಿಗೆ ಒಂದು ಬಾರಿ ಮಾತ್ರ ದರ್ಶನ ನೀಡಿದನು.


\section{\num{೬೩.} ಸರಳ ರಹಸ್ಯ}

ಮಗುವಿನಲ್ಲಿರುವಂತಹ ಶ್ರದ್ಧೆಭಕ್ತಿಯಿಂದ ಭಗವಂತನನ್ನು ನೋಡ ಬಹುದು. ಒಬ್ಬ ವ್ಯಕ್ತಿಯು ಸಾಧುವೊಬ್ಬನನ್ನು ಕಂಡು, ‘ನನಗೇನಾದರೂ ಬೋಧನೆ ಮಾಡಿ,’ ಎಂದು ಕೇಳಿದ. ಅದಕ್ಕೆ ಸಾಧು “ದೇವರನ್ನು ಹೃತ್ಪೂರ್ವಕ



\section{\num{೬೪.} ಮೂಲಶ್ರದ್ಧೆ}

ಯಾವುದಾದರೂ ಕೆಲಸವನ್ನು ಮಾಡುವುದಕ್ಕೆ ಮುಂಚೆ ಮೂಲಶ್ರದ್ಧೆ ಇರಬೇಕು. ಅದಲ್ಲದೆ ಅದನ್ನು ಕುರಿತು ಚಿಂತಿಸಿದಾಗ ಸಂತೋಷವುಂಟಾಗ ಬೇಕು. ಆಗ ಮಾತ್ರ ಯಾವುದೇ ಕೆಲಸ ಮಾಡಲು ಸಾಧ್ಯ. ಹೊಲದಲ್ಲಿ ಚಿನ್ನದ ಮೊಹರುಗಳನ್ನು ಹೂಳಲಾಗಿದೆ ಎಂದಿಟ್ಟುಕೊಳ್ಳಿ. ಮೊದಲನೆಯದಾಗಿ ಅಲ್ಲಿ ಚಿನ್ನದ ಮೊಹರುಗಳು ಇವೆ ಎನ್ನುವುದರಲ್ಲಿ ವಿಶ್ವಾಸವಿರಬೇಕು. ಚಿನ್ನದ ಮೊಹರುಗಳು ಇರುವ ಕೊಪ್ಪರಿಗೆ ಅಲ್ಲಿದೆ ಎಂದು ಭಾವಿಸಿದಾಗ ಅವನಿಗೆ ಸಂತೋಷವಾಗುವುದು. ಅವನು ಅಲ್ಲಿ ಅಗೆಯಲು ಶುರುಮಾಡುತ್ತಾನೆ. ಮೇಲೆ ಇರುವ ಮಣ್ಣನ್ನು ತೆಗೆಯುತ್ತಿದ್ದಂತೆ ಠಣ್ ಎಂಬ ಶಬ್ದವನ್ನು ಕೇಳುತ್ತಾನೆ. ಇದರಿಂದ ಅವನ ಸಂತೋಷ ಹೆಚ್ಚಾಗುವುದು. ಅನಂತರ\\ಕೊಪ್ಪರಿಗೆಯ ಒಂದು ಭಾಗವನ್ನು ನೋಡುತ್ತಾನೆ. ಆಗ ಅವನ\\ಸಂತೋಷ ಇನ್ನೂ ಅಧಿಕವಾಗುವುದು. ಅವನ ಸಂತೋಷ ವೃದ್ಧಿ\\ಯಾಗುತ್ತ ಬರುವುದು.

ಕಾಳಿದೇವಸ್ಥಾನದ ಮುಂಬಾಗಿಲಲ್ಲಿ ನಿಂತುಕೊಂಡು ಸಾಧುಗಳು ಸೇದು ವುದಕ್ಕೆ ಭಂಗಿಯನ್ನು ಮಾಡುವುದನ್ನು ನೋಡಿದ್ದೇನೆ. ಅದನ್ನು ಸೇದುವ ಅಪೇಕ್ಷೆಯಿಂದ ಸಂತೋಷಭರಿತರಾಗಿರುವುದನ್ನು ನೋಡಿರುವೆನು.


\section{\num{೬೫. } ನಿಜವಾದ ಭಕ್ತನ ಶ್ರದ್ಧೆ}

ಒಂದು ಸಲ ಕಾಮಾರಪುಕುರಕ್ಕೆ ಹೋಗುವಾಗ ಒಂದು ದೊಡ್ಡ ಬಿರುಗಾಳಿ ಎದ್ದಿತು. ಆಗ ನಾನು ಒಂದು ದೊಡ್ಡ ಹುಲ್ಲುಗಾವಲಿನಲ್ಲಿ ಇದ್ದೆ. ಆ ಸ್ಥಳದಲ್ಲಿ ಕಳ್ಳಕಾಕರ ಭಯ ಜಾಸ್ತಿ. ರಾಮ, ಕೃಷ್ಣ, ಭಗವತಿ ಎಂದು ಎಲ್ಲಾ ದೇವರ ನಾಮಗಳನ್ನೂ ಉಚ್ಚರಿಸಿದೆ. ಹನುಮಂತನ ನಾಮವನ್ನೂ ಉಚ್ಚರಿಸಿದೆ. ಎಲ್ಲ ದೇವರ ನಾಮಗಳನ್ನೂ ಉಚ್ಚರಿಸಿದೆ. ಇದರ ಅರ್ಥವನ್ನು ಹೇಳುತ್ತೇನೆ ಕೇಳಿ. ಆಳು ಕೊಳ್ಳಬೇಕಾಗಿರುವ ಸಾಮಾನನ್ನು ಲೆಕ್ಕಾಚಾರಮಾಡುತ್ತಿರುವಾಗ ಅವನು ಇದು ಆಲೂಗಡ್ಡೆಗೆ, ಇದು ಬದನೆಕಾಯಿಗೆ, ಇದು ಮೀನಿಗೆ ಎಂದು ಬೇರೆ ಬೇರೆ ಲೆಕ್ಕ ಹಾಕುವನು. ಲೆಕ್ಕ ಹಾಕಿದ ಮೇಲೆ ಆ ಹಣವನ್ನೆಲ್ಲ ಒಂದೇ ಕಡೆ ಇಡುತ್ತಾನೆ.

ಶ್ರದ್ಧೆಗೆ ಅಸಾಧ್ಯವಾಗಿರುವುದು ಯಾವುದು ಇದೆ? ನಿಜವಾದ ಭಕ್ತ ಎಲ್ಲ ವನ್ನೂ ನಂಬುವನು–ನಿರಾಕಾರ, ಸಾಕಾರ ಬ್ರಹ್ಮ ಮತ್ತು ರಾಮ, ಕೃಷ್ಣ, ಭಗವತಿ ಎಲ್ಲವನ್ನೂ ನಂಬುವನು.


\section{\num{೬೬.} ಅಪ್ಪಟ ಶ್ರದ್ಧೆ}

ಒಂದು ಸಲ ಒಬ್ಬ ತರುಣ ಸಂನ್ಯಾಸಿ ಭಿಕ್ಷೆ ಬೇಡುವುದಕ್ಕೆ ಒಂದು ಮನೆಗೆ ಹೋದ. ಅವನು ಚಿಕ್ಕಂದಿನಿಂದಲೇ ಸಂನ್ಯಾಸಿಯಾಗಿದ್ದರಿಂದ ಅವನಿಗೆ ಸಂಸಾರದ ವಿಷಯ ಅಷ್ಟು ಗೊತ್ತಿರಲಿಲ್ಲ. ಒಬ್ಬ ಯುವತಿ ಅವನಿಗೆ ಭಿಕ್ಷೆ ಯನ್ನು ಕೊಡುವುದಕ್ಕೆ ಬಂದಳು. ಅವಳ ಮೊಲೆಗಳನ್ನು ನೋಡಿ, “ಇವಳ ಎದೆಯ ಮೇಲೆ ಏನಾದರೂ ಕುರು ಎದ್ದಿದೆಯೆ” ಎಂದು ಕೇಳಿದ. ಅದಕ್ಕೆ ಆ ಯುವತಿಯ ತಾಯಿ ಉತ್ತರಕೊಟ್ಟಳು “ಇಲ್ಲ ಮಗು, ಅವಳ ಎದೆಯ ಮೇಲೆ ಯಾವ ಕುರುವೂ ಇಲ್ಲ. ಅವಳಿಗೆ ಸದ್ಯ ದಲ್ಲೇ ಮಗುವಾಗಲಿದೆ. ಆ ಮಗುವಿಗೆ ಹಾಲುಣಿಸಲು ದೇವರು ಈ ಮೊಲೆಗಳನ್ನು ಸೃಷ್ಟಿಮಾಡಿರುವನು.” ತರುಣ ಸಂಸ್ಯಾಸಿ ಇದನ್ನು ಕೇಳಿ ದೊಡನೆಯೆ, “ನಾನು ಇನ್ನು ಮೇಲೆ ಭಿಕ್ಷೆ ಬೇಡುವುದಿಲ್ಲ. ಯಾರು ನನ್ನನ್ನು ಸೃಷ್ಟಿಸಿರುವನೋ ಅವನೇ ಆಹಾರವನ್ನು ಕೊಡುವನು,” ಎಂದು ಉದ್ಗರಿಸಿದನು.


\section{\num{೬೭. } ಅಸಾಧಾರಣವಾದ ಶ್ರದ್ಧೆ}

ಒಂದು ಸಲ ಶ್ರೀಕೃಷ್ಣ ಅರ್ಜುನನೊಂದಿಗೆ ರಥದಲ್ಲಿ ಹೋಗುತ್ತಿದ್ದಾಗ ಆಕಾಶದ ಕಡೆ ನೋಡಿ “ನೋಡು, ಅಲ್ಲಿ ಎಷ್ಟೊಂದು ಚೆನ್ನಾಗಿರುವ ಪಾರಿವಾಳಗಳು

ಇದರ ಅರ್ಥವನ್ನು ತಿಳಿದುಕೊಳ್ಳಲು ಯತ್ನಿಸಿ. ಅರ್ಜುನ ಸತ್ಯನಿಷ್ಠೆಗೆ ಪ್ರಸಿದ್ಧ. ಶ್ರೀಕೃಷ್ಣನನ್ನು ಹೊಗಳುವುದಕ್ಕಾಗಿ ಎಲ್ಲದಕ್ಕೂ ಹ್ಞೂಗುಟ್ಟಲಿಲ್ಲ. ಆದರೆ ಶ್ರೀಕೃಷ್ಣನ ಮಾತಿನಲ್ಲಿ ಅಂತಹ ಶ್ರದ್ಧೆ ಇತ್ತು–ಶ್ರೀಕೃಷ್ಣನು ಏನೇ ಹೇಳಿದರೂ ಅದನ್ನು ಪ್ರತ್ಯಕ್ಷ ಕಾಣುತ್ತಿದ್ದ. ಶ್ರದ್ಧೆಯಿಲ್ಲದ ಜೀವನ ಬರಡು ಮರದಂತೆ.


\section{\num{೬೮. } ಅದ್ಭುತವಾದ ಶ್ರದ್ಧೆ}

ಒಬ್ಬ ಭಕ್ತ ನೂರಕ್ಕೆ ನೂರು ಪಾಲು ತನ್ನ ಇಷ್ಟದೇವತೆ ಭಗವಂತ ಎಂದು ನಂಬಿದರೆ ಆಗ ದೇವರನ್ನು ಕಾಣುತ್ತಾನೆ. ಹಿಂದಿನ ಕಾಲದಲ್ಲಿ ಜನರಿಗೆ ಅದ್ಭುತವಾದ ಶ್ರದ್ಧೆ ಇತ್ತು. ಹಲಧಾರಿಯ (ಶ್ರೀರಾಮಕೃಷ್ಣರ ದಾಯಾದಿ ಮತ್ತು ದಕ್ಷಿಣೇಶ್ವರ ದೇವಸ್ಥಾನದ ಪೂಜಾರಿ) ತಂದೆಗೆ ಎಂತಹ ಅದ್ಭುತವಾದ ಶ್ರದ್ಧೆಯಿತ್ತು! ಒಂದು ಸಲ ಅವನು ತನ್ನ ಮಗಳ ಮನೆಗೆ ಹೋಗುತ್ತಿದ್ದ. ಆಗ ದಾರಿಯಲ್ಲಿ ಚೆನ್ನಾದ ಹೂಗಳನ್ನು ಮತ್ತು ಬಿಲ್ವಮರವನ್ನು ನೋಡಿದ. ಅವುಗಳನ್ನು ಸಂಗ್ರಹಿಸಿಕೊಂಡು ತನ್ನ ಇಷ್ಟದೇವರ ಪೂಜೆಗಾಗಿ ಮನೆಗೆ ಐದಾರು ಮೈಲಿಗಳು ಹಿಂತಿರುಗಿ ಬಂದ.

ಒಂದು ಸಲ ಹಳ್ಳಿಯಲ್ಲಿ ಒಂದು ನಾಟಕ ಮಂಡಲಿಯವರು ಶ್ರೀರಾಮನ ಜೀವನವನ್ನು ನಟಿಸುತ್ತಿದ್ದರು. ಕೈಕೇಯಿ ಶ್ರೀರಾಮನನ್ನು ಕಾಡಿಗೆ ಹೋಗು ಎಂದಾಗ, ಈ ನಾಟಕವನ್ನು ನೋಡುತ್ತಿದ್ದ ಹಲಧಾರಿಯ ತಂದೆ ಮೇಲೆದ್ದು ಬಂದು, ಕೈಕೇಯಿ ಪಾತ್ರವನ್ನು ಧರಿಸಿದವನನ್ನು ಮಹಾಪಾತಕಿ ಎಂದು ಪಂಜಿ ನಿಂದ ಅವನ ಮುಖವನ್ನು ಸುಡುವುದರಲ್ಲಿದ್ದ.

ಅವನೊಬ್ಬ ಧಾರ್ಮಿಕ ವ್ಯಕ್ತಿ. ಅವನು ತನ್ನ ಸ್ನಾನ ಆಹ್ನಿಕ ಗಳನ್ನು ಮಾಡಿದ ಮೇಲೆ ನಡುನೀರಿನಲ್ಲಿ ನಿಂತುಕೊಂಡು ಭಗವತಿ ಯನ್ನು ಕುರಿತು ‘ರಕ್ತವರ್ಣಾಂ ಚತುರ್ಮುಖಾಂ’ ಎಂದು ಧ್ಯಾನಿ ಸುತ್ತಿದ್ದಾಗ ಅವನ ಕಣ್ಣಿನಿಂದ ಪ್ರೇಮಾಶ್ರು ಹರಿಯುತ್ತಿತ್ತು.


\section{\num{೬೯.} ಇದೇನೆ ಪ್ರಪಂಚ }

ನೀನು ಶ್ರದ್ಧೆಯ ಅದ್ಭುತವಾದ ಶಕ್ತಿಯ ವಿಷಯವನ್ನು ಕೇಳಿರಬೇಕು. ಪುರಾಣದಲ್ಲಿ ಸಾಕ್ಷಾತ್ ಭಗವಂತನಾದ, ಪರಬ್ರಹ್ಮಸ್ವರೂಪಿಯಾದ ರಾಮ ಸಿಂಹಳ ದೇಶಕ್ಕೆ ಹೋಗಲು ಸೇತುವೆಯನ್ನು ಕಟ್ಟಬೇಕಾಯಿತು. ಆದರೆ ಹನು ಮಂತ ಒಂದೇ ನೆಗೆತದಲ್ಲಿ ಅಲ್ಲಿಗೆ ಹೋದನು. ಅವನಿಗೆ ಸೇತುವೆಯನ್ನು ಕಟ್ಟುವ ಆವಶ್ಯಕತೆಯೇ ಇರಲಿಲ್ಲ.


\section{\num{೭೦.} ಹನುಮಾನ್ ಸಿಂಗ್ ಮತ್ತುಪಂಜಾಬಿನ ಕುಸ್ತಿಯವನು}

ಒಂದು ಸಲ ಇಬ್ಬರು ಕುಸ್ತಿಯಾಡುತ್ತಿದ್ದರು. ಅದರಲ್ಲಿ ಒಬ್ಬ ಹನುಮಾನ್ ಸಿಂಗ್ ಮತ್ತೊಬ್ಬ ಪಂಜಾಬಿನ ಮುಸಲ್ಮಾನ. ಮುಸಲ್ಮಾನ ಪೈಲ್​ವಾನ್ ಗಟ್ಟಿಮುಟ್ಟಾಗಿದ್ದ. ಅವನು ಹದಿನೈದು ದಿನಗಳಿಂದ ಬೆಣ್ಣೆ ಮತ್ತು ಮಾಂಸವನ್ನು



\section{\num{೭೧.} ಶ್ರದ್ಧೆಗೆ ಯಾವ ಮಂತ್ರತಂತ್ರಗಳೂ ಬೇಕಾಗಿಲ್ಲ}

ಒಂದು ಸಲ ಭಗವತ್ ಸಾಕ್ಷಾತ್ಕಾರಕ್ಕಾಗಿ ಇಬ್ಬರು ಯೋಗಿಗಳು ತಪಸ್ಸು ಮಾಡುತ್ತಿದ್ದರು. ಒಂದು ದಿನ ದೇವರ್ಷಿ ನಾರದರು ಅವರ ಆಶ್ರಮದ ಮಾರ್ಗದಲ್ಲಿ ಹೋಗುತ್ತಿದ್ದರು. ಅವರಲ್ಲಿ ಒಬ್ಬ “ನೀವು ವೈಕುಂಠದಿಂದ ಬರುತ್ತಿರುವಿರೇನು?” ಎಂದು ಕೇಳಿದನು. ನಾರದರು, “ಹೌದು” ಎಂದರು. “ದೇವರು ವೈಕುಂಠದಲ್ಲಿ ಏನು ಮಾಡುತ್ತಿದ್ದ? ಅದನ್ನು ಸ್ವಲ್ಪ ಹೇಳುವಿರಾ?” ಎಂದು ಆ ಯೋಗಿ ಕೇಳಿದ. ಅದಕ್ಕೆ ನಾರದರು “ಒಂಟೆ ಮತ್ತು ಆನೆಗಳನ್ನು ಸೂಜಿಯ ಕಣ್ಣಿನಲ್ಲಿ ಪೋಣಿಸಿ ಆಡುತ್ತಿದ್ದನು” ಎಂದರು. ಅದಕ್ಕೆ ಯೋಗಿ “ಇದರಲ್ಲೇನೂ ಆಶ್ಚರ್ಯವಿಲ್ಲ. ಭಗವಂತನಿಗೆ ಅಸಾಧ್ಯವಾದುದು ಯಾವುದೂ ಇಲ್ಲ” ಎಂದ. ಆದರೆ ಮತ್ತೊಬ್ಬ ಯೋಗಿ “ಅರ್ಥವಿಲ್ಲದ ಮಾತು ಇದು. ಇದು ಅಸಾಧ್ಯ. ನೀನು ವೈಕುಂಠಕ್ಕೆ ಹೋಗೇ ಇಲ್ಲ ಎಂದು ಗೊತ್ತಾಯಿ ತಷ್ಟೆ” ಎಂದನು.

ಮೊದಲನೆಯವನು ಭಕ್ತ, ಮಗುವಿನಂತಹ ಸರಳ ಶ್ರದ್ಧೆ ಇತ್ತು. ಭಗವಂತ ನಿಗೆ ಯಾವುದೂ ಅಸಾಧ್ಯವಲ್ಲ ಮತ್ತು ಅವನ ಸ್ವರೂಪವನ್ನು ಯಾರೂ ಪೂರ್ಣ ಅರಿಯರು. ಏನನ್ನು ಬೇಕಾದರೂ ಅವನು ಮಾಡಬಲ್ಲ.


\section{\num{೭೨. } ಶ್ರದ್ಧೆ ಪವಾಡವನ್ನು ಮಾಡಬಲ್ಲದು }

ಒಂದು ಸಲ ಒಬ್ಬ ರೋಗಿ ಸಾಯುವ ಸ್ಥಿತಿಯಲ್ಲಿದ್ದ. ಯಾರೂ ಅವನನ್ನು ಬದುಕಿಸುವಂತೆ ಕಾಣಲಿಲ್ಲ. ಆದರೆ ಒಬ್ಬ ಸಾಧು ರೋಗಿಯ ತಂದೆಗೆ ಈ ಮಾತನ್ನು ಹೇಳಿದ “ಅವನನ್ನು ಬದುಕಿ ಸುವುದಕ್ಕೆ ಒಂದು ಉಪಾಯವಿದೆ. ನೀನು ಒಂದು ನರಕಪಾಲವನ್ನು ಪಡೆದು, ಸ್ವಾತಿ ಮಳೆಯ ಹನಿಯನ್ನು ನಾಗರಹಾವಿನ ವಿಷದಲ್ಲಿ ಸೇರಿಸಿ ಕೊಟ್ಟರೆ ನಿನ್ನ ಹುಡುಗ ಬದುಕುವ ಸಂಭವವಿದೆ” ಎಂದ. ತಂದೆ ಪಂಚಾಂಗವನ್ನು ನೋಡಿದ. ಮಾರನೆ ದಿನ ಸ್ವಾತಿ ನಕ್ಷತ್ರ ಬರುವುದೆಂದು ತಿಳಿದುಕೊಂಡ. ಆಗ ಅವನು ದೇವರಿಗೆ ಪ್ರಾರ್ಥನೆ ಮಾಡಿದ: “ಭಗವಂತ, ಈ ಷರತ್ತುಗಳನ್ನೆಲ್ಲ ನೀನು ಪೂರೈಸಬೇಕು ಮತ್ತು ನನ್ನ ಮಗನ ಜೀವವನ್ನು ಕಾಪಾಡಬೇಕು.” ತುಂಬಾ ಶ್ರದ್ಧೆ ಮತ್ತು ಆಸೆಯಿಂದ ಮಾರನೆ ದಿನ ಸಾಯಂಕಾಲ ಒಂದು ನರಕಪಾಲಕ್ಕೆ ಸ್ಮಶಾನದಲ್ಲಿ ಹುಡುಕಾಡಿದ. ಕೊನೆಗೆ ಒಂದು ಮರದ ಕೆಳಗೆ ಅದು ದೊರಕಿತು. ಅದನ್ನು ಕೈಯಲ್ಲಿ ಹಿಡಿದು, ಸ್ವಾತಿ ಮಳೆಯ ನೀರಿಗಾಗಿ ಪ್ರಾರ್ಥಿಸಿದ. ಇದ್ದಕ್ಕಿ ದ್ದಂತೆ ಸ್ವಲ್ಪ ಮಳೆ ಬಿತ್ತು. ನರಕಪಾಲದೊಳಗೆ ಕೆಲವು ಸ್ವಾತಿ ಮಳೆ ಹನಿಗಳು ಬಿದ್ದವು. ಆಗ ಅವನು ‘ನನಗೆ ನರಕಪಾಲವೂ ಸಿಕ್ಕಿತು, ಸ್ವಾತಿ ಹನಿಯ ನೀರೂ ಸಿಕ್ಕಿತು’ ಎಂದುಕೊಂಡು ಮತ್ತೆ ಭಗವಂತನನ್ನು ಹೃತ್ಪೂರ್ವಕವಾಗಿ, “ದೇವರೆ, ಉಳಿದುದನ್ನೂ ಈಡೇರಿಸಿಕೊಡು” ಎಂದು ಪ್ರಾರ್ಥಿಸಿದ. ಸ್ವಲ್ಪ ದೂರದಲ್ಲೆ ಒಂದು ಕಪ್ಪೆಯನ್ನು ಕಂಡ. ಒಂದು ಹಾವು ಕಪ್ಪೆಯ ಮೇಲೆ ಬೀಳುವುದರ ಲ್ಲಿತ್ತು. ಅದರಿಂದ ತಪ್ಪಿಸಿಕೊಳ್ಳಲು ಕಪ್ಪೆ ಕಪಾಲದೊಳಗೆ ಜಿಗಿಯಿತು. ಹಾವೂ ಕೂಡ ಅದನ್ನು ತಿನ್ನಲು ಅದರ ಮೇಲೆ ನೆಗೆಯಿತು. ಆಗ ಒಂದು ಹನಿ ವಿಷ ಕಪಾಲದೊಳಗೆ ಬಿತ್ತು. ಭಗವಂತನ ಕೃಪಾಕಟಾಕ್ಷವನ್ನು ಪಡೆದು, ಅತ್ಯಂತ ಕೃತಜ್ಞತೆಯಿಂದ ವಂದಿಸಿದ. “ದೇವರೆ ನಿನ್ನ ಮಹಿಮೆಯಿಂದ ಅಸಾಧ್ಯವೂ ಸಾಧ್ಯವಾಗುವುದು. ಇನ್ನು ನನ್ನ ಮಗ ಬದುಕಿಕೊಳ್ಳುವುದರಲ್ಲಿ ಸಂದೇಹ ಇಲ್ಲ” ಎಂದ. ದೇವರ ಮೇಲೆ ಅನನ್ಯ ಶ್ರದ್ಧೆ ಭಕ್ತಿಗಳಿದ್ದರೆ ಅವನ ದಯೆಯಿಂದ ಎಲ್ಲವೂ ಸಾಧ್ಯ.


\section{\num{೭೩.} ಶ್ರದ್ಧೆಯೇ ಪವಾಡಕ್ಕೆ ಮಾರ್ಗ}

ನದಿಯ ಆಚೆ ಕಡೆ ಇರುವ ಬ್ರಾಹ್ಮಣ ಪೂಜಾರಿಗೆ ಹಾಲಿನವಳೊಬ್ಬಳು ಹಾಲನ್ನು ಕೊಡುತ್ತಿದ್ದಳು. ದೋಣಿ ಸಕಾಲದಲ್ಲಿ ಬರದೆ, ಕಾಲಕ್ಕೆ ಸರಿಯಾಗಿ ಹಾಲನ್ನೊದಗಿಸಲು ಅವಳಿಗೆ ಆಗುತ್ತಿರಲಿಲ್ಲ. ಒಂದು ದಿನ ಬಹಳ ಹೊತ್ತಾದ ಮೇಲೆ ಹಾಲನ್ನು ತಂದಳು. ಬ್ರಾಹ್ಮಣನು



\section{\num{೭೪.} ಭಗವನ್ನಾಮದ ಮಹಿಮೆ}

ಒಮ್ಮೆ ರಾಜನೊಬ್ಬ ಬ್ರಾಹ್ಮಣನ ಹತ್ಯೆಯ ಘೋರಪಾಪಕ್ಕೆ ಕಾರಣನಾದ. ಪಾಪಭೀತಿಯಿಂದ ವ್ಯಾಕುಲನಾಗಿ ಪಾಪಪರಿಹಾರಕ್ಕಾಗಿ ಪ್ರಾಯಶ್ಚಿತ್ತವನ್ನು ತಿಳಿಯಲು ಪುಷಿಗಳ ಆಶ್ರಮಕ್ಕೆ ಬಂದ. ಪುಷಿಗಳು ಮನೆಯಲ್ಲಿರಲಿಲ್ಲ. ಅವರ ಮಗ ಆಶ್ರಮದಲ್ಲಿದ್ದ. ಅವನು ದೊರೆಯಿಂದ ನಡೆದ ವಿಷಯವನ್ನು ಕೇಳಿ, “ಭಗವಂತನ ಹೆಸರನ್ನು ಮೂರು ಬಾರಿ ಉಚ್ಚರಿಸು. ನೀನು ಆಗ ಪಾಪದಿಂದ ಪಾರಾಗುವೆ” ಎಂದ. ಪುಷಿಗಳು ಹಿಂತಿರುಗಿ ಬಂದ ಮೇಲೆ ಮಗನು ಸೂಚಿಸಿದ ಪ್ರಾಯಶ್ಚಿತ್ತದ ವಿಷಯವನ್ನು ಕೇಳಿ ಮಗನಿಗೆ ಕೋಪದಿಂದ, “ಒಮ್ಮೆ ಭಗವಂತನ ನಾಮವನ್ನು ಉಚ್ಚರಿಸಿದರೆ ಸಾಕು, ಅನೇಕ ಕೋಟಿ ಜನ್ಮಗಳಿಂದ ಮಾಡಿದ ಪಾಪವೆಲ್ಲ ಅಳಿಸಿಹೋಗುವುದು. ದಡ್ಡ, ದೇವರ ಮೇಲೆ ಇರುವ ಶ್ರದ್ಧೆ ನಿನಗೆ ಇಷ್ಟೇ ಏನು? ಮೂರು ಬಾರಿ ಅವನ ನಾಮವನ್ನು ಉಚ್ಚಾರ ಮಾಡುವುದಕ್ಕೆ ಹೇಳಿದೆಯಲ್ಲ! ನಿನ್ನಲ್ಲಿರುವ ಶ್ರದ್ಧೆ ದುರ್ಬಲವಾಗಿರುವುದ ರಿಂದ ನೀನೊಬ್ಬ ಚಂಡಾಲನಾಗು” ಎಂದು ಶಪಿಸಿದರು. ಅವನೇ ರಾಮಾಯಣದಲ್ಲಿ ಗುಹನಾಗಿ ಜನ್ಮತಾಳಿದ.


\section{\num{೭೫.} ಅನುಮಾನಪಡುವವನು ನಾಶವಾಗುತ್ತಾನೆ}

ಒಂದು ಸಲ ಒಬ್ಬ ಸಮುದ್ರವನ್ನು ದಾಟುವುದರಲ್ಲಿದ್ದ. ವಿಭೀಷಣ ರಾಮನ ಹೆಸರನ್ನು ಒಂದು ಎಲೆಯ ಮೇಲೆ ಬರೆದು ಅವನ ಬಟ್ಟೆಯಲ್ಲಿ ಕಟ್ಟಿದ. “ನೀನು ಯಾವುದಕ್ಕೂ ಅಂಜಬೇಕಾಗಿಲ್ಲ. ಶ್ರದ್ಧೆಯಿಂದ ಸಮುದ್ರದ ಮೇಲೆ ನಡೆದು ಕೊಂಡು ಹೋಗು... ಆದರೆ, ನಿನ್ನಲ್ಲಿ ಶ್ರದ್ಧೆ ಕಡಿಮೆಯಾದೊಡನೆ ನೀನು ಮುಳುಗುವೆ” ಎಂದ. ಮನುಷ್ಯ ಸುಲಭವಾಗಿ ಸಮುದ್ರದ ಮೇಲೆ ಹೋಗು ತ್ತಿದ್ದನು. ಪಂಚೆ ತುದಿಯಲ್ಲಿ ವಿಭೀಷಣ ಏನನ್ನು ಬರೆದಿರುವನೊ ಅದನ್ನು ತಿಳಿದುಕೊಳ್ಳಬೇಕೆಂಬ ಕುತೂಹಲ ಕೆರಳಿತು. ಅದನ್ನು ಬಿಚ್ಚಿ ತೆಗೆದಾಗ ಸಾಧಾ ರಣವಾದ ಎಲೆಯ ಮೇಲೆ ಬರೀ ರಾಮನಾಮವನ್ನು ಬರೆದಿರುವುದನ್ನು ಕಂಡ. “ಇಷ್ಟೇನೆ, ಬರೀ ರಾಮನಾಮವಷ್ಟೇ!” ಎಂದು ಅವನ ಮನಸ್ಸಿನಲ್ಲಿ ಯಾವಾಗ ಸಂಶಯ ಬಂತೋ ಆ ಕೂಡಲೇ ಅವನು ಸಮುದ್ರದ ಪಾಲಾದ.

\chapter{ಭಕ್ತಿ}

\section{\num{೭೬.} ಭಗವಂತನಿಗೆ ನಾವು ನೀಡುವ ಶ್ರೇಷ್ಠ ವಸ್ತುವೇ ಪ್ರೀತಿ}

ಒಂದು ಸಲ ಒಬ್ಬ ಶ್ರೀಮಂತನ ಆಳು ಭಯಭಕ್ತಿಯಿಂದ ಬಂದು ಒಡೆಯನ ಮನೆಯ ಒಂದು ಮೂಲೆಯಲ್ಲಿ ನಿಂತ. ಅವನು ತನ್ನ ಕೈಯಲ್ಲಿ ಏನನ್ನೊ ಬಟ್ಟೆಯಿಂದ ಮುಚ್ಚಿಟ್ಟುಕೊಂಡಿದ್ದನು. “ನಿನ್ನ ಕೈಯಲ್ಲಿರುವುದು ಏನು?” ಎಂದು ಶ್ರೀಮಂತ ಕೇಳಿದ. ಬಟ್ಟೆ ಒಳಗಿನಿಂದ ಒಂದು ಸೀತಾಫಲದ ಹಣ್ಣನ್ನು ತೆಗೆದು ಯಜಮಾನನಿಗೆ ಬಹಳ ವಿಧೇಯತೆಯಿಂದ ಕೊಟ್ಟ. ಅದನ್ನು ಒಡೆಯ ಸ್ವೀಕರಿಸಿದರೆ ತಾನು ಕೃತಜ್ಞನಾಗುವೆನು ಎಂದು ಅವನ ಭಾವ. ಯಜಮಾನನಿಗೆ ಆಳಿನ ಭಕ್ತಿಯನ್ನು ನೋಡಿ ತುಂಬಾ ಸಂತೋಷವಾಯಿತು. ಅದು ಅತ್ಯಂತ ಅಲ್ಪ ವಸ್ತುವಾಗಿದ್ದರೂ ಅದನ್ನು ಪ್ರೀತಿಯಿಂದ ತೆಗೆದು ಕೊಂಡ. ಯಜಮಾನನಿಗೆ ತುಂಬಾ ಸಂತೋಷವಾಗಿ “ಎಂತಹ ಸೊಗಸಾದ ಹಣ್ಣನ್ನು ತಂದಿದ್ದೀಯೆ. ನಿನಗೆ ಇದು ಎಲ್ಲಿ ಸಿಕ್ಕಿತು?” ಎಂದು ಕೇಳಿದ. ಇದರಂತೆಯೆ ದೇವರು ಭಕ್ತನ ಹೃದಯವನ್ನು ನೋಡುತ್ತಾನೆ. ಅವನ ಮಹಿಮೆ ಅನಂತವಾದುದು. ಆದರೂ ಶ್ರದ್ಧೆ ಭಕ್ತಿಯಿಂದ ಭಕ್ತ ಏನನ್ನು ಕೊಟ್ಟರೂ ಅದನ್ನು ಸ್ವೀಕರಿಸುವನು.


\section{\num{೭೭. } ಪ್ರತಿಫಲಾಪೇಕ್ಷೆ ಇಲ್ಲದ ಪ್ರೀತಿಯೆ ಅತ್ಯುತ್ಕೃಷ್ಟ}

ಶ್ರೀಕೃಷ್ಣನ ಪರಮ ಮಿತ್ರನಾದ ಅರ್ಜುನನಿಗೆ ಒಮ್ಮೆ ಅಹಂಕಾರ ಬಂತು. ಭಗವದ್​ಭಕ್ತಿಯಲ್ಲಿ ತನಗೆ ಸರಿಸಮರಿಲ್ಲ ಎಂದು ಭಾವಿಸಿದ. ಸರ್ವಜ್ಞನಾದ ಶ್ರೀಕೃಷ್ಣನಿಗೆ ಇದು ಗೊತ್ತಾಗಿ ಒಂದು ಸಲ ಅರ್ಜುನ ನನ್ನು ತನ್ನೊಡನೆ ಹೊರಗೆ ಸಂಚಾರಕ್ಕೆ ಕರೆದುಕೊಂಡು ಹೋದ. ಅವರು ಸ್ವಲ್ಪ ದೂರ ಹೋಗು ವುದರೊಳಗೆ ಒಬ್ಬ ವಿಚಿತ್ರವಾದ ಬ್ರಾಹ್ಮಣ ಕಾಣಿಸಿದ. ಆ ಬ್ರಾಹ್ಮಣ ಒಣಗಿದ ಹುಲ್ಲನ್ನು ಆಹಾರವಾಗಿ ತಿನ್ನುತ್ತಿದ್ದ. ಆದರೂ ಅವನು ಸೊಂಟದಲ್ಲಿ ಕತ್ತಿ ಯನ್ನು ಕಟ್ಟಿಕೊಂಡಿದ್ದ. ಅರ್ಜುನನಿಗೆ ಈತ ಒಬ್ಬ ಸಾಧು ಮನುಷ್ಯ, ಪರಮ ಭಕ್ತ ಎಂದು ಗೊತ್ತಾಯಿತು. ಆತ ವಿಷ್ಣುಭಕ್ತನಾದ್ದರಿಂದ ಯಾರನ್ನೂ ಹಿಂಸಿಸಲಾರ. ಅದೇ ಅವನ ಪರಮ ಧರ್ಮ. ಹಸಿಯ ಹುಲ್ಲಿ\\ನಲ್ಲಿಯೂ ಜೀವವಿರುವುದರಿಂದ ಅವನು ನಿರ್ಜೀವವಾದ ಒಣ\\ಹುಲ್ಲನ್ನು ತಿನ್ನುತ್ತಿದ್ದ. ಆದರೂ ಅವನು ಸೊಂಟದಲ್ಲಿ ಕತ್ತಿ\\ಯನ್ನು ಹಿಡಿದುಕೊಂಡಿದ್ದ. ಅರ್ಜುನ ಈ ಅಸಂಬದ್ಧತೆಯನ್ನು\\ನೋಡಿ, ಆಶ್ಚರ್ಯದಿಂದ ಶ್ರೀಕೃಷ್ಣನನ್ನು ಕೇಳಿದ, “ಏತಕ್ಕೆ ಅವನು ಹೀಗೆ ಮಾಡುತ್ತಿರುವನು? ಯಾರನ್ನೂ, ಒಂದು ಎಳೆ ಹುಲ್ಲನ್ನೂ ಹಿಂಸಿಸುವುದಿಲ್ಲ ಎಂಬ ವ್ರತವನ್ನು ಅವನು ತೆಗೆದುಕೊಂಡಿರುವಂತಿದೆ. ಆದರೂ ಅವನ ಸೊಂಟ ದಲ್ಲಿ ಮೃತ್ಯುವಿನ ಮತ್ತು ಹಿಂಸೆಯ ಪ್ರತೀಕವಾದ ಖಡ್ಗವನ್ನು ಕಟ್ಟಿಕೊಂಡಿ ದ್ದಾನೆ.” ಆಗ ಶ್ರೀಕೃಷ್ಣನು “ನೀನೆ ಅವನನ್ನು ಕೇಳು” ಎಂದ. ಆಗ ಅರ್ಜುನ ಬ್ರಾಹ್ಮಣನ ಹತ್ತಿರ ಹೋಗಿ ವಿನಯದಿಂದ “ಸ್ವಾಮಿ, ನೀವು ಯಾರನ್ನೂ ಹಿಂಸಿಸುವುದಿಲ್ಲ. ಬರೀ ಒಣಗಿದ ಹುಲ್ಲನ್ನು ತಿನ್ನುತ್ತಿರುವಿರಿ. ಆದರೂ ಈ ಚೂಪಾದ ಕತ್ತಿ ಏಕೆ ನಿಮ್ಮ ಸೊಂಟದಲ್ಲಿದೆ?” ಎಂದು ಕೇಳಿದ.

ಬ್ರಾಹ್ಮಣ ಹೇಳಿದ, “ನನಗೇನಾದರೂ ಸಿಕ್ಕಿದರೆ ಆ ನಾಲ್ಕು ಜನರನ್ನು ದಂಡಿಸುವುದಕ್ಕೆ” ಎಂದ. ಅರ್ಜುನ “ಯಾರು ಅವರು?” ಎಂದು ಕೇಳಿದ. “ಮೊದಲನೆಯವನೆ ಆ ದುರಾತ್ಮನಾದ ನಾರದ” ಎಂದ. ಅರ್ಜುನ “ಅವನು ಏನು ತಪ್ಪು ಮಾಡಿದ?” ಎಂದು ಕೇಳಿದ. ಅದಕ್ಕೆ ಬ್ರಾಹ್ಮಣ ಹೇಳಿದ: “ಆ ದುರುಳನು ಯಾವಾಗಲೂ ಹರಿಯ ನಾಮಸ್ಮರಣೆ ಮಾಡುತ್ತ ದೇವರಿಗೆ ವಿರಾಮಕ್ಕೆ ಅವಕಾಶವನ್ನೇ ಕೊಡುವುದಿಲ್ಲ. ಭಗವಂತನ ಸೌಕರ್ಯವನ್ನು ಅವನು ಸ್ವಲ್ಪವೂ ಗಮನಿಸುವುದಿಲ್ಲ. ಅವನು ತನ್ನ ಪ್ರಾರ್ಥನೆ ಮಾಡುತ್ತ ಭಜನೆ ಯಿಂದ ಹಗಲು ರಾತ್ರಿ ಕಾಲ ಅಕಾಲದಲ್ಲಿ ಭಗವಂತನ ನಿದ್ರಾಭಂಗಮಾಡು ತ್ತಾನೆ.”

ಅರ್ಜುನ ಕೇಳಿದ, “ಎರಡನೆಯವನು ಯಾರು?” ಎಂದು. “ಆ ದುರ ಹಂಕಾರಿ ದ್ರೌಪದಿ.” “ಅವಳ ತಪ್ಪೇನು?” ಎಂದು ಅರ್ಜುನ ಕೇಳಿದ. ಬ್ರಾಹ್ಮಣ ಅದಕ್ಕೆ ಹೇಳಿದ: “ದ್ರೌಪದಿಯಲ್ಲಿ ಯುಕ್ತಾಯುಕ್ತ ಪರಿಜ್ಞಾನವೇ ಇಲ್ಲ. ಊಟ ಮಾಡುತ್ತಿದ್ದ ನನ್ನ ಭಗವಂತನನ್ನು ಎಬ್ಬಿಸುವಂತಹ ದುಡುಕು ಅವಳದು. ಅವನು ಊಟ ಬಿಟ್ಟು ಕಾಮ್ಯಕವನಕ್ಕೆ ದೂರ್ವಾಸರ ಶಾಪದಿಂದ ಪಾಂಡವರನ್ನು ರಕ್ಷಿಸುವುದಕ್ಕಾಗಿ ಹೋಗಬೇಕಾಯಿತು. ಅವಳು ಅತಿ ಕೊಬ್ಬಿದ್ದಾಳೆ. ತಾನುಂಡ ಎಂಜಲನ್ನೇ ಭಗವಂತನಿಗೆ ಉಣಬಡಿಸಿದಳು!”

ಮೂರನೆಯವನು ಯಾರು ಎಂದು ಅರ್ಜುನ ಕೇಳಿದ. ಆಗ ಬ್ರಾಹ್ಮಣ, “ಆ ನಿರ್ದಯಿಯಾದ ಪ್ರಹ್ಲಾದ. ಅವನು ಎಷ್ಟು ಕ್ರೂರಿ! ಪ್ರಹ್ಲಾದನನ್ನು ರಕ್ಷಿಸಲು ಭಗವಂತ ಕುದಿಯುತ್ತಿರುವ ಎಣ್ಣೆಯ ಕೊಪ್ಪರಿಗೆಯನ್ನು ಪ್ರವೇಶಿಸಬೇಕಾಯಿತು, ಆನೆಗಳ ಕಾಲ್ಕೆಳಗೆ ತುಳಿಸಿ ಕೊಳ್ಳಬೇಕಾಯಿತು, ಬೃಹತ್ ಕಂಬಗಳ ಮೂಲಕ ಸೀಳಿ ಬರಬೇಕಾಯಿತು, ನೋಡು! ಎಷ್ಟು ಕಷ್ಟ ನನ್ನ ಭಗವಂತನಿಗೆ!” ಎಂದ.

ಅರ್ಜುನ “ನಾಲ್ಕನೆಯವನು ಯಾರು?” ಎಂದಾಗ ಬ್ರಾಹ್ಮಣ, “ಆ ನೀಚ ಅರ್ಜುನ” ಎಂದ. “ಏಕೆ, ಅವನು ಏನು ತಪ್ಪನ್ನು ಮಾಡಿದ?” ಎಂದು ಅರ್ಜುನ ಕೇಳಿದ್ದಕ್ಕೆ ಬ್ರಾಹ್ಮಣ, “ಆ ನೀಚನನ್ನು ನೋಡಿ, ಕುರುಕ್ಷೇತ್ರದ ಯುದ್ಧದಲ್ಲಿ ನನ್ನ ಪ್ರಿಯ ಶ್ರೀಕೃಷ್ಣನಿಂದ ರಥವನ್ನು ನಡೆಸುವ ಸಾಮಾನ್ಯ ಸಾರಥಿಯ ಕೆಲಸವನ್ನು ಮಾಡಿಸಿದ” ಎಂದ.

ಅರ್ಜುನನಿಗೆ ಆ ಬಡ ಬ್ರಾಹ್ಮಣನಲ್ಲಿದ್ದ ಶ್ರದ್ಧೆ ಭಕ್ತಿಯನ್ನು ನೋಡಿ ಆಶ್ಚರ್ಯವಾಯ್ತು. ಅಂದಿನಿಂದ ಅವನ ದುರಹಂಕಾರ ಹೋಯಿತು. ತಾನು ಭಗವಂತನ ಶ್ರೇಷ್ಠಭಕ್ತ ಎನ್ನುವ ಅಹಂಕಾರವನ್ನು ತ್ಯಜಿಸಿದ.



\section{\num{೭೮} ಯಾರಿಗೆ ಬಹುಮಾನ ಸಿಕ್ಕುವುದು?}

ಕಾರ್ತಿಕೇಯ ಮತ್ತು ಗಣಪತಿ ಇಬ್ಬರೂ ಪಾರ್ವ ತಿಯ ಹತ್ತಿರ ಇದ್ದರು. ಭಗವತಿಯ ಕೊರಳಲ್ಲಿ ಒಂದು ವಜ್ರದ ಹಾರವಿತ್ತು. ಭಗವತಿ, “ಯಾರು ಈ ಪ್ರಪಂಚವನ್ನು ಮೊದಲು ಸುತ್ತಿಕೊಂಡು ಬರುವರೊ ಅವರಿಗೆ ಈ ಹಾರವನ್ನು ಬಹುಮಾನವಾಗಿ ಕೊಡುತ್ತೇನೆ” ಎಂದಳು. ಕೇಳಿದ ತಕ್ಷಣವೇ ಕಾರ್ತಿಕೇಯ ನವಿಲಿನ ಮೇಲೆ ಹೊರಟ. ಗಣೇಶ ನಿಧಾನವಾಗಿ ಭಗವತಿಯ ಸುತ್ತಲೂ ಪ್ರದಕ್ಷಿಣೆ ಮಾಡಿ ಬಂದ. ಅವನು ಭಗವತಿ ಸೃಷ್ಟಿಯನ್ನೆಲ್ಲ ತನ್ನಲ್ಲಿ ಅಡಗಿಸಿಕೊಂಡಿರುವಳು ಎಂಬುದನ್ನು ತಿಳಿದುಕೊಂಡಿದ್ದ. ಭಗವತಿ ಗಣೇಶನ ಭಕ್ತಿಯನ್ನು ನೋಡಿ ಸಂತೋಷ ದಿಂದ ತನ್ನ ಕೊರಳ ಹಾರವನ್ನು ಗಣಪತಿಗೆ ಹಾಕಿದಳು. ಬಹಳ ಕಾಲದ ಮೇಲೆ ಕಾರ್ತಿಕೇಯ ಹಿಂತಿರುಗಿ ಬಂದ. ಅಲ್ಲಿ ಆಗಲೆ ಪಾರ್ವತಿ ಗಣಪತಿಯ ಕೊರಳಿ ನಲ್ಲಿ ಹಾರವನ್ನು ಹಾಕಿದ್ದಳು. ಬರೀ ಸರಳವಾದ ಭಕ್ತಿಯಿಂದ ಸರ್ವವನ್ನೂ ಪಡೆದುಕೊಳ್ಳಬಹುದು. ಭಗವಂತನನ್ನು ಪ್ರೀತಿಸಿದರೆ ಯಾವ ಕೊರತೆಯೂ ಇರುವುದಿಲ್ಲ.


\section{\num{೭೯. } ಒಂದು ಕಾಗೆಯ ಪರಮಭಕ್ತಿ}

ರಾಮ-ಲಕ್ಷ್ಮಣರು ಪಂಪಾ ಸರೋವರಕ್ಕೆ ಬಂದರು. ಅಲ್ಲಿ ಒಂದು ಕಾಗೆ ನೀರನ್ನು ಕುಡಿಯಲು ಬಹಳ ತವಕಿಸುತ್ತಿತ್ತು. ಸರೋವರದ ಅಂಚಿಗೆ ಎಷ್ಟು ಸಲ ಹೋದರೂ ನೀರನ್ನು ಕುಡಿಯುತ್ತಿರಲಿಲ್ಲ. ಲಕ್ಷ್ಮಣ ರಾಮನಿಗೆ ಕಾರಣ ಕೇಳಿದ. ಅದಕ್ಕೆ ರಾಮ, “ಸಹೋದರನೆ, ಆ ಕಾಗೆ ಭಗವಂತನ ಪರಮ ಭಕ್ತ. ಹಗಲು ರಾತ್ರಿ ಅದು ಭಗವಂತನ ಹೆಸರನ್ನು ಜಪಿಸುತ್ತಿರುವುದು. ಅದರ ಗಂಟಲು ನೀರಿಲ್ಲದೆ ಒಣಗಿಹೋಗಿದ್ದರೂ ನೀರನ್ನು ಕುಡಿಯುತ್ತಿಲ್ಲ. ಏಕೆಂದರೆ ಕುಡಿಯುವಾಗ ಭಗವನ್ನಾಮೋಚ್ಚಾರಣೆ ನಿಂತುಹೋಗುವುದಲ್ಲ ಎಂದು.”


\section{\num{೮೦. } ಮೂರು ಜನ ಸ್ನೇಹಿತರು ಮತ್ತು ಹುಲಿ}

ಒಂದು ಸಲ ಮೂರು ಜನ ಸ್ನೇಹಿತರು ಒಂದು ಕಾಡಿನ ಮೂಲಕ ಹೋಗುತ್ತಿದ್ದರು. ಆಗ ಒಂದು ಹುಲಿ ಇದ್ದಕ್ಕಿದ್ದಂತೆ ಕಾಣಿಸಿಕೊಂಡಿತು. ಅವರಲ್ಲಿ ಮೊದಲನೆಯವನು, “ನಮ್ಮ ಗತಿ ಆಯಿತು!” ಎಂದ. ಎರಡನೆ ಯವನು, “ಏತಕ್ಕೆ ಹೀಗೆ ಹೇಳುತ್ತೀಯ? ನಾವು ಹೇಗೆ ನಾಶವಾದೇವು? ಭಗವಂತನನ್ನು ಪ್ರಾರ್ಥಿಸೋಣ” ಎಂದ. ಮೂರನೆಯವನು, “ದೇವರಿಗೆ ಏತಕ್ಕೆ ಕಷ್ಟ ಕೊಡುವುದು? ಹತ್ತಿರದ ಮರವನ್ನೇರಿ ಕುಳಿತುಕೊಳ್ಳೋಣ” ಎಂದನು.

‘ನಮ್ಮ ಗತಿ ಮುಗಿಯತು’ ಎಂದು ಹೇಳಿದವನು, ದೇವರೊಬ್ಬ ನಿರುವನು, ಅವನು ನಮ್ಮನ್ನು ರಕ್ಷಿಸುವನು ಎಂಬುದನ್ನು ಅರಿಯ. ‘ದೇವರನ್ನು ಕುರಿತು ಪ್ರಾರ್ಥಿಸೋಣ’ ಎಂದವನು ಜ್ಞಾನಿ. ದೇವರು ಸೃಷ್ಟಿ ಸ್ಥಿತಿ ಪ್ರಳಯಗಳನ್ನು ಮಾಡುವವನು ಎಂಬುದನ್ನು ಬಲ್ಲ. ‘ಇದಕ್ಕಾಗಿ ಏತಕ್ಕೆ ದೇವರಿಗೆ ಕಷ್ಟ ಕೊಡಬೇಕು? ಮರವನ್ನೇರೋಣ’ ಎಂದ ವನು ಭಗವಂತನ ಪರಮ ಭಕ್ತ. ಅಂತಹ ಪ್ರೀತಿ ಒಬ್ಬನಲ್ಲಿದ್ದರೆ ತಾನು ದೇವರಿಗಿಂತ ಬಲಿಷ್ಠ ಎಂದು ಭಾವಿಸುವನು. ತನ್ನ ದೇವರಿಗೆ ಯಾವ ಕಷ್ಟವೂ ಬರದಂತೆ ಅವನು ಸದಾ ನೋಡಿಕೊಳ್ಳುತ್ತಾನೆ. ಅವನ ಏಕಮಾತ್ರ ಉದ್ದೇಶ ತನ್ನ ಪ್ರಿಯತಮನ ಕಾಲುಗಳಿಗೆ ಒಂದು ಮುಳ್ಳೂ ಸೋಂಕಕೂಡದು ಎಂಬುದು.


\section{\num{೮೧. } ಇಷ್ಟದಮೇಲೆ ಪರಮನಿಷ್ಠೆ}

ಒಮ್ಮೆ ಪಾಂಡವರು ರಾಜಸೂಯ ಯಾಗವನ್ನು ಮಾಡಿದರು. ಎಲ್ಲರೂ ಯುಧಿಷ್ಠಿರನಿಗೆ ಅಗ್ರಸ್ಥಾನ ವನ್ನು ಕೊಟ್ಟು ಅವನಿಗೆ ಭಕ್ತಿಯಿಂದ ನಮಿಸಿದರು. ಆದರೆ ಲಂಕಾರಾಜ್ಯದ ಅಧಿಪತಿ ವಿಭೀಷಣ ನಾನು ಶ್ರೀಮನ್ನಾರಾಯಣನ ವಿನಾ ಮತ್ತಾರಿಗೂ ನಮಿಸುವುದಿಲ್ಲ ಎಂದ. ಆಗ ಯುಧಿಷ್ಠಿರನಿಗೆ ಶ್ರೀಕೃಷ್ಣನೇ ನಮಸ್ಕರಿಸಿದನು. ಆಗ ವಿಭೀ ಷಣನು ತನ್ನ ಕೀರಿಟವನ್ನು ತೆಗೆದು ಯುಧಿಷ್ಠಿರನಿಗೆ ನಮಿಸಿ ದನು.

ಇದೇ ಏಕನಿಷ್ಠಾಭಕ್ತಿ. ಅಲ್ಲಿ ಇಷ್ಟದೇವರ ವಿನಾ ಬೇರೆ ಯಾರನ್ನೂ ಗಮನಿಸುವುದಿಲ್ಲ. ಕಮಲವು ಸೂರ್ಯನ ಕಿರಣ ಗಳಿಗಾಗಿ ಏಕನಿಷ್ಠೆಯಿಂದ ಕಾಯುವಂತೆ, ಭಗವಂತನಲ್ಲಿ ಭಕ್ತಿಯಿರಬೇಕು.


\section{\num{೮೨. } ಸುಖ ದುಃಖಗಳಲ್ಲಿ ದೇವರೊಬ್ಬನೆ ಆರಾಧ್ಯವಸ್ತು}

ಒಂದೂರಿನಲ್ಲಿ ಒಬ್ಬ ನೇಯ್ಗೆಯವನಿದ್ದ. ಅವನು ಸಾತ್ವಿಕ ಸ್ವಭಾವದವನು. ಪ್ರತಿಯೊಬ್ಬರೂ ಅವನನ್ನು ಗೌರವಿಸುತ್ತಿದ್ದರು, ಪ್ರೀತಿಸುತ್ತಿದ್ದರು. ಅವನು ಸಂತೆಯಲ್ಲಿ ತಾನೇ ನೇಯ್ದ ಬಟ್ಟೆಯನ್ನು ವ್ಯಾಪಾರ ಮಾಡುತ್ತಿದ್ದನು. ನೇಯ್ಗೆಯವನನ್ನು ಯಾರಾದರೂ ವಸ್ತ್ರದ ಬೆಲೆಯನ್ನು ಕೇಳಿದರೆ ಅವನು “ಭಗವಂತನ ಇಚ್ಛೆಯಿಂದ, ನೂಲಿನ ಬೆಲೆ ಒಂದು ರೂಪಾಯಿ, ಕೂಲಿಯ ಬೆಲೆ ನಾಲ್ಕಾಣೆ, ಭಗವಂತನ ಇಚ್ಛೆಯಿಂದ ಅದರ ಲಾಭ ಎರಡಾಣೆ. ಬಟ್ಟೆಯ ಒಟ್ಟು ಬೆಲೆ ರಾಮನ ಇಚ್ಛೆಯಿಂದ ಒಂದು ರೂಪಾಯಿ ಆರಾಣೆ” ಎನ್ನುತ್ತಿದ್ದನು. ನೇಯ್ಗೆಯವನ ಮೇಲೆ ಜನರಿಗೆಲ್ಲ ಅಷ್ಟೊಂದು ನಂಬಿಕೆ. ಗಿರಾಕಿಗಳು ಆ ಹಣವನ್ನು ಕೊಟ್ಟು ಸಾಮಾನನ್ನು ತೆಗೆದುಕೊಳ್ಳುತ್ತಿದ್ದರು. ನೇಯ್ಗೆಯವನು ಭಗವಂತನ ಪರಮಭಕ್ತ. ಸಂಜೆ ಹೊತ್ತು ಊಟವಾದ ಮೇಲೆ ಬಹಳ ಕಾಲ ದೇವರನ್ನು ಕುರಿತು ಧ್ಯಾನಿಸುತ್ತಿದ್ದ, ಅವನ ಮಹಿಮೆಯನ್ನು ಕೊಂಡಾಡುತ್ತಿದ್ದ. ಒಂದು ದಿನ ಎಷ್ಟು ಹೊತ್ತಾದರೂ ಅವನಿಗೆ ನಿದ್ರೆ ಬರಲಿಲ್ಲ. ಅವನು ದೇವರ ಮನೆಯಲ್ಲಿ ಕುಳಿತುಕೊಂಡು ಗುಡಿಗುಡಿಯನ್ನು ಸೇದುತ್ತಿದ್ದ. ಆಗ ಒಂದು ದರೋಡೆಕೋರರ ದಂಡು ಅತ್ತ ಕಡೆ ಬಂತು. ದರೋಡೆಕೋರರು ನೇಯ್ಗೆಯವನ ಹತ್ತಿರ ಬಂದು “ನಮ್ಮ ಜೊತೆಯಲ್ಲಿ ಬಾ” ಎಂದು ಹೇಳಿದರು. ಹಾಗೆ ಹೇಳುತ್ತ ಅವನನ್ನು ಎಳೆದು ಕೊಂಡು ಹೋದರು. ಮನೆಯನ್ನು ದರೋಡೆ ಮಾಡಿದ ಮೇಲೆ ತಮ್ಮ ಕಳ್ಳ ಮಾಲನ್ನು ಹೊರುವಂತೆ ನೇಯ್ಗೆಯವನಿಗೆ ಹೇಳಿದರು. ಅಷ್ಟು ಹೊತ್ತಿಗೆ ಪೋಲಿಸಿನವರು ಬಂದರು. ದರೋಡೆಕೋರರು ಓಡಿಹೋದರು. ಆದರೆ ಸಾಮಾನನ್ನು ಹೊರುತ್ತಿದ್ದ ನೇಯ್ಗೆಯವನು ಸಿಕ್ಕಿ ಬಿದ್ದನು. ಅವನನ್ನು ಪೋಲಿಸಿ ನವರು ದಸ್ತಗಿರಿ ಮಾಡಿದರು. ಮಾರನೆ ದಿನ ವಿಚಾರಣೆಗೆ ನ್ಯಾಯಾಧಿಪತಿಯ ಮುಂದೆ ಅವನನ್ನು ಹಾಜರುಮಾಡಿದರು. ಹಳ್ಳಿಯವರಿಗೆ ಸಮಾಚಾರ ಗೊತ್ತಾಗಿ ಅವರೆಲ್ಲ ಕೋರ್ಟಿಗೆ ಬಂದರು. ಅವರು ನ್ಯಾಯಾಧಿಪತಿಗೆ “ಮಾನ್ಯ ನ್ಯಾಯಾಧಿಪತಿಗಳೆ, ಈ ಮನುಷ್ಯ ಎಂದಿಗೂ ಕದಿಯುವವನಲ್ಲ” ಎಂದರು. ಆಗ ನ್ಯಾಯಾಧಿಪತಿಗಳು ನೇಯ್ಗೆಯವನಿಗೆ “ಇದು ಹೇಗಾಯಿತು? ಹೇಳು” ಎಂದರು.

ನೇಯ್ಗೆಯವನು “ಮಾನ್ಯ ನ್ಯಾಯಾಧಿಪತಿಗಳೆ, ರಾಮನ ಇಚ್ಛೆ ಯಂತೆ ನಾನು ಊಟಮಾಡಿದೆ. ರಾಮನ ಇಚ್ಛೆಯಂತೆ ನಾನು ದೇವರ ಮನೆಯಲ್ಲಿ ಕುಳಿತಿದ್ದೆ. ರಾಮನ ಇಚ್ಛೆಯಂತೆ ಅಂದು ಅವೇಳೆಯಾಗಿತ್ತು. ಭಗವಂತನ ಇಚ್ಛೆಯಂತೆ ನಾನು ಭಗವಂತನ ಚಿಂತನೆ ಮಾಡಿ ಅವನನ್ನು ಭಜಿಸುತ್ತಿದ್ದೆ. ರಾಮನ ಇಚ್ಛೆಯಂತೆ ಒಂದು ಡಕಾ ಯಿತರ ತಂಡ ಅಲ್ಲಿಗೆ ಬಂತು. ರಾಮನ ಇಚ್ಛೆಯಂತೆ ಅವರು ನನ್ನನ್ನು ಎಳೆದುಕೊಂಡು ಹೋದರು. ರಾಮನ ಇಚ್ಛೆಯಂತೆ ಅವರು ದರೋಡೆ ಮಾಡಿ ದರು. ರಾಮನ ಇಚ್ಛೆಯಂತೆ ಅವರು ಕದ್ದಮಾಲನ್ನು ನನ್ನ ಮೇಲೆ ಹೊರಿಸಿ ದರು. ಅಷ್ಟು ಹೊತ್ತಿಗೆ ರಾಮನ ಇಚ್ಛೆಯಂತೆ ಪೋಲಿಸರು ಬಂದರು. ರಾಮನ ಇಚ್ಛೆಯಂತೆ ನನ್ನನ್ನು ದಸ್ತಗಿರಿ ಮಾಡಿದರು. ಆಗ ರಾಮನ ಇಚ್ಛೆಯಂತೆ ಪೋಲಿಸರು ನನ್ನನ್ನು ಜೈಲಿಗೆ ಕರೆದುಕೊಂಡು ಹೋದರು. ಇವತ್ತಿನ ಬೆಳಿಗ್ಗೆ ರಾಮನ ಇಚ್ಛೆಯಂತೆ ನನ್ನನ್ನು ನಿಮ್ಮ ಮುಂದೆ ನಿಲ್ಲಿಸಿದ್ದಾರೆ.” ನ್ಯಾಯಾಧಿಪತಿ ಗಳಿಗೆ ಗೊತ್ತಾಯಿತು ಈತ ಸತ್ಯನಿಷ್ಠ ಎಂದು. “ಅವನನ್ನು ಬಿಡುಗಡೆಮಾಡಿ” ಎಂದರು. ದಾರಿಯಲ್ಲಿ ಮನೆಗೆ ಹೋಗುತ್ತಿದ್ದಾಗ “ರಾಮನ ಆಣತಿಯಂತೆ ನನ್ನನ್ನು ಸೆರೆಮನೆಯಿಂದ ಖುಲಾಸೆ ಮಾಡಿದರು” ಎಂದನು. ನೀನು ಈ ಸಂಸಾರದಲ್ಲಿ ಇರುವೆಯೊ ಅಥವಾ ಅದನ್ನು ತ್ಯಜಿಸುವೆಯೊ ಎಲ್ಲಾ ರಾಮನ ಇಚ್ಛೆಯ ಮೇಲಿದೆ. ದೇವರ ಮೇಲೆ ಭಾರ ಹಾಕಿ ನಿನ್ನ ಪಾಲಿನ ಕೆಲಸವನ್ನು ಮಾಡು.


\section{\num{೮೩.} ರಾವಣ–ರಾಮನ ಪರಮಭಕ್ತ}

ಮಂಡೋದರಿ ತನ್ನ ಗಂಡನಾದ ರಾವಣನಿಗೆ, “ನೀನು ಸೀತೆಯನ್ನು ವರಿಸ ಬೇಕೆಂದು ಇಚ್ಛಿಸಿದರೆ ನೀನೇಕೆ ಮಾಯಾಶಕ್ತಿಯಿಂದ ರಾಮನ ವೇಷ ಹಾಕಿ ಕೊಳ್ಳಬಾರದು?” ಎಂದು ಕೇಳಿದಳು. “ನಿನಗೆ ಗೊತ್ತಿರುವುದು ಇಷ್ಟೆ! ನಾನು ರಾಮನಂತೆ ವೇಷ ಹಾಕಿದಾಗ ಇಂತಹ ಹೀನಕೃತ್ಯವನ್ನು ಮಾಡಲು ಸಾಧ್ಯವೆ? ಅವನನ್ನು ಕುರಿತು ಚಿಂತಿಸಿದರೇನೆ ನನ್ನ ಮನಸ್ಸು ಅವರ್ಣನೀಯ ಆನಂದ ದಿಂದ ತುಂಬುವುದು. ಶ್ರೇಷ್ಠವಾದ ಸ್ವರ್ಗವೂ ಕಸದಂತೆ ತೋರುವುದು.”


\section{\num{೮೪. } ಭಕ್ತಿಯಿಂದ ಎಲ್ಲಾ ಬಾಗಿಲುಗಳೂ ತೆರೆಯುವವು}

ಶ್ರೀಕೃಷ್ಣನಿಂದ ಯಾವ ಸುದ್ದಿಯೂ ಬಾರದೆ ಇದ್ದುದರಿಂದ ಒಮ್ಮೆ ಯಶೋದೆ ರಾಧೆ ಬಳಿಗೆ ಬಂದಳು. “ನನ್ನ ಗೋಪಾಲನ ವಿಚಾರ ನಿನಗೇನಾದರೂ ಗೊತ್ತೆ?” ಎಂದು ಕೇಳಿದಳು. ಈಗ ರಾಧೆ ಗಾಢ ಧ್ಯಾನದಲ್ಲಿದ್ದಳಾದ್ದರಿಂದ ಯಶೋದೆಯ ಮಾತುಗಳು ಕೇಳಿಸಲಿಲ್ಲ. ರಾಧೆ ಸಮಾಧಿ ಯಿಂದಿಳಿದ ನಂತರ ನಂದನ ರಾಣಿಯಾದ ಯಶೋದೆಯನ್ನು ಅವಳು ನೋಡಿದಳು. ತಕ್ಷಣವೇ ಅವಳಿಗೆ ನಮಸ್ಕಾರ ಮಾಡಿ ದಳು. ಏತಕ್ಕೆ ಬಂದಿರಿ ಎಂದು ರಾಧೆ ಯಶೋದೆಯನ್ನು ಕೇಳಿ ದಳು. ಕಾರಣವನ್ನು ತಿಳಿದಾದ ಮೇಲೆ ರಾಧೆ, “ತಾಯಿ, ನೀವು ಕಣ್ಣನ್ನು ಮುಚ್ಚಿಕೊಂಡು ಅವ ನನ್ನು ಧ್ಯಾನಿಸಿದರೆ ನಿಮಗೆ ಅವನು ಗೋಚರಿಸುವನು” ಎಂದಳು. ಯಶೋದೆ ಕಣ್ಣನ್ನು ಮುಚ್ಚಿದೊಡನೆಯೆ ಅಧ್ಯಾತ್ಮ ಭಾವದ ಘನೀಭೂತಳಾಗಿದ್ದ ರಾಧೆಯು ಯಶೋದೆಯ ಮೇಲೆ ತನ್ನ ಪ್ರಭಾವವನ್ನು ಬೀರಿ ಅವಳಿಗೆ ಗೋಪಾಲನ ದರ್ಶನವಾಗುವಂತೆ ಮಾಡಿದಳು. ಆಗ ಯಶೋದೆ ರಾಧೆಯನ್ನು “ನಾನು ಕಣ್ಣು ಮುಚ್ಚಿದಾಗಲೆಲ್ಲ ನನ್ನ ಪ್ರೀತಿಯ ಗೋಪಾಲನನ್ನು ನೋಡುವ ವರವನ್ನು ಕೊಡಮ್ಮ” ಎಂದು ಪ್ರಾರ್ಥಿಸಿದಳು.


\section{\num{೮೫. } ಯಾವುದನ್ನು ಪ್ರಾಪಂಚಿಕರು ಬಯಸುವರೋ,\\ಭಕ್ತ ಅದನ್ನು ತ್ಯಜಿಸುತ್ತಾನೆ}

ರಾವಣನ ವಧೆಯಾದ ನಂತರ ವಿಭೀಷಣ ಲಂಕಾ ರಾಜ್ಯದ ಅರಸನಾಗು ವುದಕ್ಕೆ ಒಪ್ಪಲಿಲ್ಲ. ವಿಭೀಷಣ “ಓ ರಾಮ, ನಿನ್ನನ್ನು ನಾನು ನೋಡಿದೆ. ನನಗೆ ಇನ್ನು ರಾಜ್ಯಭಾರವೇಕೆ” ಎಂದು ಹೇಳಿದ.

“ನೀನು ಅಜ್ಞಾನಿಗಳಿಗಾಗಿ ರಾಜನಾಗು. ರಾಮನ ಸೇವೆಯಿಂದ ನಿನಗೆ ಏನು ಬಂತು ಎಂದು ಕೇಳಿದರೆ, ಅದರಿಂದ ರಾಜನಾದೆ ಎಂದು ಹೇಳಬಹುದು. ಅವರಿಗೆ ಒಂದು ಸನ್ಮಾರ್ಗವನ್ನು ತೋರಿ ಸಲು ನೀನು ರಾಜನಾಗು” ಎಂದ ರಾಮ.


\section{\num{೮೬. } ಶ್ರೀಕೃಷ್ಣನಿಗೆ ಜಯವಾಗಲಿ}

ನಾನು ಒಂದು ಸಲ ಮಥುರಬಾಬುವಿನೊಂದಿಗೆ ಒಂದೂರಿಗೆ ಹೋಗಿದ್ದೆ. ನನ್ನೊಡನೆ ವಾದ ಮಾಡುವುದಕ್ಕೆ ಅನೇಕ ಜನ ಪಂಡಿತರು ಬಂದರು. ನಿಮಗೇ ಗೊತ್ತಿದೆ ನನಗೆ ಏನೂ ಪಾಂಡಿತ್ಯವಿಲ್ಲ ಎನ್ನುವುದು. ಪಂಡಿತರು ನನ್ನ ವಿಚಿತ್ರ ಸ್ವಭಾವವನ್ನು ಗ್ರಹಿಸಿದರು. ವಾದ ಮುಗಿದ ಮೇಲೆ ಅವರು ನನಗೆ ಹೇಳಿದರು: “ನಿಮ್ಮ ಸಂಭಾಷಣೆಯನ್ನು ಕೇಳಿದ ಮೇಲೆ, ನಾವು ಕಲಿತಿರುವುದೆಲ್ಲ–ವಿದ್ಯೆ, ಪಾಂಡಿತ್ಯವೆಲ್ಲ ಕೆಲಸಕ್ಕೆ ಬಾರದಂತೆ ಕಾಣುವುದು. ಈಗ ನಮಗೆ

ನನಗೆ ಏನೂ ಗೊತ್ತಿಲ್ಲ ಎಂಬುದು ನಿಮಗೇ ಗೊತ್ತಿದೆ. ಹಾಗಾದರೆ ಈ ಸದ್ವಿಚಾರಗಳೆಲ್ಲ ಯಾರಿಂದ ಬಂದವು? ಇದೆಲ್ಲ ಭಗವತಿಯ ಮಹಿಮೆ. ನಾವುಗಳೆಲ್ಲ ಅವಳಿಗೆ ನಿಮಿತ್ತ ಮಾತ್ರ. ಒಂದು ಸಲ ರಾಧೆ ತಾನು ಪತಿವ್ರತೆ ಎಂದು ತೋರಿಸುವುದಕ್ಕೆ ನೀರು ತುಂಬಿದ ರಂಧ್ರದ ಕೊಡವನ್ನು ತೆಗೆದು ಕೊಂಡು ಹೋದಳು. ಕೊಡ ತೂತಿನಮಯವಾಗಿತ್ತು. ಆದರೂ ಒಂದು ತೊಟ್ಟು ನೀರೂ ಸೋರಲಿಲ್ಲ. ಜನರೆಲ್ಲ ಇಂತಹ ಪತಿವ್ರತೆಯನ್ನು ಪ್ರಪಂಚ ಅರಿಯದು ಎಂದರು. ಆಗ ರಾಧೆ, “ನನ್ನನ್ನು ಏತಕ್ಕೆ ಹೊಗಳುತ್ತೀರಿ. ಇದೆಲ್ಲ ಕೃಷ್ಣನ ಮಹಿಮೆ; ಕೃಷ್ಣನಿಗೆ ಜಯವಾಗಲಿ! ನಾನು ಅವನ ದಾಸಿ ಅಷ್ಟೆ!” ಎಂದಳು.


\section{\num{೮೭. } ಭಗವಂತನ ಮೇಲೆ ಪರಿಶುದ್ಧ ಪ್ರೇಮ}

ಭಾರತದೇಶದ ದಕ್ಷಿಣಭಾಗದಲ್ಲಿ ತಿರುಗಾಡುತ್ತಿದ್ದಾಗ ಶ್ರೀ ಚೈತನ್ಯದೇವರು ಒಂದು ಅಪೂರ್ವ ದೃಶ್ಯವನ್ನು ಕಂಡರು. ಒಬ್ಬ ಪಂಡಿತ ಗೀತೆಯನ್ನು ವ್ಯಾಖ್ಯಾನ ಮಾಡುತ್ತಿದ್ದನು. ಅದನ್ನು ಕೇಳುತ್ತಿದ್ದ ಭಕ್ತನೊಬ್ಬನ ಕಣ್ಣುಗಳಿಂದ ನೀರು ಸುರಿಯುತ್ತಿತ್ತು. ಆ ಭಕ್ತನಿಗೆ ಗೀತೆಯ ಒಂದು ಶಬ್ದವೂ ಅರ್ಥವಾಗುತ್ತಿರಲಿಲ್ಲ. “ನೀನು ಏತಕ್ಕೆ ಕಣ್ಣೀರನ್ನು ಸುರಿಸುತ್ತಿದ್ದೆ” ಎಂದು ಕೇಳಿದಾಗ ಅವನು ಹೇಳಿದ, “ನನಗೆ ಗೀತೆಯಲ್ಲಿ ಯಾವ ಶಬ್ದವೂ ಅರ್ಥವಾಗಲಿಲ್ಲ. ಆದರೆ ಅದನ್ನು ಪಂಡಿತರು ಓದಿ ಹೇಳುತ್ತಿದ್ದಾಗ ನನ್ನ ಅಂತರ್​ಚಕ್ಷುವಿನಿಂದ ಭಗವಂತನ ಅಸಾಧಾರಣ ಸೌಂದರ್ಯವನ್ನು ನೋಡುತ್ತಿದ್ದೆ. ಅವನು ರಥದ ಮೇಲೆ ಕುಳಿತುಕೊಂಡಿದ್ದ. ಕುರುಕ್ಷೇತ್ರದ ಸಮರಾಂಗಣದಲ್ಲಿ ಅರ್ಜುನನ್ನು ಕುರಿತು ಗೀತೋಪದೇಶವನ್ನು ಮಾಡುತ್ತಿರುವುದನ್ನು ಕಂಡೆ. ನನ್ನ ಕಣ್ಣಿನಲ್ಲಿ ಆನಂದಾಶ್ರು ಬರುವುದಕ್ಕೆ ಇದೇ ಕಾರಣ” ಎಂದ. ಯಾವ ಜ್ಞಾನವೂ ಇಲ್ಲದ ಈ ಭಕ್ತನಿಗೆ ಶ್ರೇಷ್ಠವಾದ ಜ್ಞಾನ ಲಭಿಸಿತ್ತು. ಏಕೆಂದರೆ ಅವನಿಗೆ ಭಗವಂತನ ಮೇಲೆ ಅಷ್ಟೊಂದು ಪ್ರೀತಿ ವಿಶ್ವಾಸಗಳು ಇದ್ದವು. ಅದರಿಂದಲೇ ಅವನು ಸಾಕ್ಷಾತ್ಕಾರವನ್ನು ಪಡೆದ.


\section{\num{೮೮. } ಪುನಃ ಭಗವತಿಯ ಬಳಿಗೆ}

ಯಾವ ಭಕ್ತನಲ್ಲಿ ವಿಷ್ಣುವಿನ ಅಂಶ ಇದೆಯೋ ಅವನು ಭಕ್ತಿಯಿಂದ ಪಾರಾಗಲಾರ. ಒಂದು ಸಲ ನಾನು ಒಬ್ಬ ಜ್ಞಾನಿಯ ಬಲೆಗೆ ಸಿಕ್ಕಿಬಿದ್ದೆ. ಅವನು ಹನ್ನೊಂದು ತಿಂಗಳ ಕಾಲ ವೇದಾಂತವನ್ನು ಬೋಧಿಸಿದ. ಆದರೆ ನನ್ನಲ್ಲಿದ್ದ ಭಕ್ತಿಬೀಜವನ್ನು ಅವನು ನಾಶಮಾಡಲು ಆಗಲಿಲ್ಲ. ನನ್ನ ಮನಸ್ಸು ಎಲ್ಲಿ ಹೋದರೂ ಪುನಃ ಭಗವತಿಯ ಬಳಿಗೆ ಬರುತ್ತಿತ್ತು. ನಾನು ತೋತಾಪುರಿ ಎದುರಿಗೆ ಭಗವತಿಗೆ ಸಂಬಂಧಪಟ್ಟ ಹಾಡುಗಳನ್ನು ಹಾಡುವಾಗ ‘ಆಹಾ’ ಎಂದು ಭಕ್ತಿಪರವಶನಾಗಿ ಕಣ್ಣೀರು ಸುರಿಸುತ್ತಿದ್ದ. ಅವನು ಅಂತಹ ಜ್ಞಾನಿ ಯಾಗಿದ್ದ, ಆದರೂ ಭಕ್ತಿಯಿಂದ ಕಣ್ಣೀರು ಉಕ್ಕುತ್ತಿತ್ತು. ಯಾರಾದರೂ ಅಲೆಕ್ ಬಳ್ಳಿಯ ರಸವನ್ನು ಕುಡಿದಿದ್ದರೆ ಅವನೊಳಗೆ ಆ ಸಸಿ ಬೆಳೆಯುವುದೆಂಬ ವದಂತಿ ಇದೆ. ಒಂದು ಸಲ ಭಕ್ತಿಬೀಜವನ್ನು ಬಿತ್ತಿದರೆ ಪರಿಣಾಮ ಬರಲೇ ಬೇಕು. ನೀವು ಒಂದು ಸಾವಿರ ವೇಳೆ ವಿಚಾರಮಾಡಿ, ತರ್ಕ ಮಾಡಿ, ಆದರೆ ನಿಮ್ಮಲ್ಲಿ ಭಕ್ತಿ ಬೀಜ ಇದ್ದರೆ ನೀವು ಪುನಃ ಹರಿಯ ಬಳಿಗೆ ಬರಲೇಬೇಕು.


\section{\num{೮೯. } ಅಹಂಕಾರವು ಭಕ್ತಿಯ ಕುಡಿಯನ್ನು\\ಚಿವುಟಿ ಹಾಕುತ್ತದೆ}

ಒಮ್ಮೆ ನಾರದರಿಗೆ ತನ್ನಂತಹ ಭಕ್ತ ಯಾರೂ ಇಲ್ಲ ಎಂಬ ಅಹಂಕಾರ ಬಂತು. ಅವರ ಹೃದಯವನ್ನು ಅರಿತ ಭಗವಂತ ನಾರದರಿಗೆ ಹೇಳಿದ, “ನಾರದರೆ, ಇಂತಹ ಸ್ಥಳಕ್ಕೆ ಹೋಗಿ. ಅಲ್ಲಿ ನನ್ನ ಮಹಾಭಕ್ತನೊಬ್ಬನಿರುವನು. ಅವನ ಸ್ನೇಹವನ್ನು ಸಂಪಾದಿಸಿಕೊಳ್ಳಿ. ಅವನು ನನ್ನ ನಿಜವಾದ ಭಕ್ತ” ಎಂದು. ನಾರದರು ಆ ಸ್ಥಳಕ್ಕೆ ಹೋದರು. ಅಲ್ಲಿ ಬೇಸಾಯಗಾರನೊಬ್ಬ ಬೆಳಿಗ್ಗೆ ಅಷ್ಟು ಹೊತ್ತಿಗೇ ಎದ್ದು ಹರಿನಾಮವನ್ನು ಒಮ್ಮೆ ಉಚ್ಚರಿಸಿ, ಅನಂತರ ಉಳುವುದಕ್ಕೆ ಹೋಗುತ್ತಿದ್ದ. ದಿನವೆಲ್ಲ ಬೇಸಾಯದ ಕೆಲಸದಲ್ಲಿ ನಿರತನಾಗಿರುತ್ತಿದ್ದ. ರಾತ್ರಿ ಮತ್ತೊಂದು ಸಲ ಹರಿ



\section{\num{೯೦. } ದೇವರೊಬ್ಬನೇ ಎಲ್ಲರಿಗೂ ದಾತ}

ಅಕ್ಬರ್ ದೆಹಲಿಯಲ್ಲಿ ರಾಜ್ಯವಾಳುತ್ತಿದ್ದಾಗ ಒಂದು ಕಾಡಿನಲ್ಲಿ ಒಬ್ಬ ಸಾಧು ವಾಸಮಾಡುತ್ತಿದ್ದ. ಆ ಸಾಧುವನ್ನು ನೋಡುವುದಕ್ಕೆ ಹಲವರು ಹೋಗು ತ್ತಿದ್ದರು. ಒಂದು ಸಲ ಆ ಸಾಧುವಿಗನಿಸಿತು, ತನ್ನ ಹತ್ತಿರ ಬರುತ್ತಿರುವ ಜನರಿಗೆ ಏನಾದರೂ ಕೊಡಬೇಕೆಂದು. ಆದರೆ ಕೈಯಲ್ಲಿ ಕಾಸಿಲ್ಲದೆ ಅದನ್ನು ಹೇಗೆ ಮಾಡುವುದು? ಆದಕಾರಣ ಚಕ್ರವರ್ತಿಯನ್ನು ಕೇಳಬೇಕೆಂದು ಅವನ ಬಳಿ ಹೋದ. ಸಾಧು ಅಲ್ಲಿಗೆ ಹೋದಾಗ ಅಕ್ಬರ್ ನಮಾಜು ಮಾಡುತ್ತಿದ್ದನಾದ್ದ ರಿಂದ ಆಗ ಒಂದು ಮೂಲೆಯಲ್ಲಿ ಸಾಧು ಕುಳಿತುಕೊಂಡ. ಚಕ್ರ\\ವರ್ತಿಯು ದೇವರನ್ನು, “ದೇವರೇ ನನಗೆ ಹಣ-ಸಂಪತ್ತುಗಳನ್ನು\\ಕೊಡು” ಎಂದು ಬೇಡುತ್ತಿದ್ದುದನ್ನು ಸಾಧು ಕೇಳಿದ. ಇದನ್ನು ಕೇಳಿ\\ದಾಗ ಸಾಧು ಅಲ್ಲಿಂದ ಹೋಗುವುದರಲ್ಲಿದ್ದನು. ಆಗ ಚಕ್ರವರ್ತಿ\\“ನೀವು ನನ್ನನ್ನು ಏನನ್ನೊ ಕೇಳಲು ಬಂದಿರಿ. ಆದರೆ ಏನನ್ನೂ ಕೇಳದೆ ಹೋಗುತ್ತಿದ್ದೀರಲ್ಲ” ಎಂದು ಕೇಳಿದ. ಸಾಧು, “ನೀವು ಅದನ್ನು ಗಮನಿಸ ಬೇಕಾಗಿಲ್ಲ. ನಾನು ಈಗ ಹೋಗಬೇಕಾಗಿದೆ” ಎಂದನು. ಚಕ್ರವರ್ತಿ, “ನೀವು ಏತಕ್ಕೆ ಬಂದಿರೊ ಅದನ್ನು ಸ್ವಲ್ಪ ತಿಳಿಸಿ” ಎಂದು ಬೇಡಿಕೊಂಡನು. ಆಗ ಸಾಧು ಹೀಗೆಂದ, “ಅನೇಕ ಜನ ನನ್ನ ಗುಡಿಸಲಿಗೆ ಬರುತ್ತಾರೆ. ಅವರಿಗೆ ಏನನ್ನಾದರೂ ಕೊಡುವುದಕ್ಕೆ ನಿಮ್ಮ ಹತ್ತಿರ ಹಣ ಕೇಳಲು ಬಂದೆ.” “ಆದರೆ ಏನನ್ನೂ ಕೇಳದೆ ಏಕೆ ಹೊರಟುಹೋಗುತ್ತಿದ್ದೀರಿ?” ಅದಕ್ಕೆ ಸಾಧು, “ನೀನು ಕೂಡ ಒಬ್ಬ ಭಿಕ್ಷುಕನಾಗಿರುವುದನ್ನು ಕಂಡೆ. ನೀನು ಕೂಡ ದೇವರನ್ನು ಹಣ ಕೊಡು, ಐಶ್ವರ್ಯ ಕೊಡು ಎಂದು ಬೇಡುತ್ತಿದ್ದೆ. ನಾನು ಭಿಕ್ಷುಕರಲ್ಲಿ ಏತಕ್ಕೆ ತಿರುಪೆ ಬೇಡಲಿ. ನಾನು ಭಿಕ್ಷೆ ಬೇಡಬೇಕಾದರೆ ದೇವರನ್ನೇ ಕೇಳುತ್ತೇನೆ” ಎಂದನು.


\section{\num{೯೧. } ನಾನು ಸಾಧಾರಣ ವಸ್ತುಗಳನ್ನು ಬೇಡುವವನಲ್ಲ}

ಹನುಮಂತನ ಮನಸ್ಸು ಹೇಗಿತ್ತು ಎಂಬುದನ್ನು ನೋಡಿ. ಅವನಿಗೆ ಹಣ ಬೇಕಾಗಿರಲಿಲ್ಲ. ಕೀರ್ತಿ ಬೇಕಾಗಿರಲಿಲ್ಲ. ಸಂಸಾರದ ಯಾವ ಸುಖಗಳೂ ಬೇಕಾ ಗಿರಲಿಲ್ಲ. ಅವನಿಗೆ ದೇವರು ಮಾತ್ರ ಬೇಕಿತ್ತು. ಅವನು ಸ್ಫಟಿಕದ ಕಂಬದಲ್ಲಿ ಅಡಗಿಸಿಟ್ಟಿದ್ದ ಇಂದ್ರಧನುಸ್ಸನ್ನು ತೆಗೆದುಕೊಂಡು ಹೋಗುವಾಗ, ಮಂಡೋ ದರಿ ಹಲವು ಹಣ್ಣುಗಳನ್ನು ಹನುಮಂತನೆದುರಿಗಿಟ್ಟಳು–ಅವನು ಮರದಿಂದ ಕೆಳಗೆ ಬಂದು ಆ ಶಸ್ತ್ರವನ್ನು ಕೆಳಗೆ ಬಿಡಲಿ ಎಂದು. ಆದರೆ ಹನುಮಂತ ಅವಳ ಬಲೆಗೆ ಸುಲಭವಾಗಿ ಸಿಕ್ಕುವವನಲ್ಲ. ಅವಳು ಮತ್ತೆ ಮತ್ತೆ ಆಸೆ ತೋರಿಸಿದಾಗ ಅವನು ಈ ಹಾಡನ್ನು ಹಾಡಿದ:

\begin{verse}
ಯಾವ ಫಲಬೇಕು ಎನಗೆ\\ಜನ್ಮ ಸಫಲವ ಗೈವ ಫಲವಿರಲು ಸುಮ್ಮಗೆ ।\\ ಮೋಕ್ಷಫಲಗಳ ಬಿಡುವ ಶ್ರೀರಾಮತರುವಿಹುದು\\ಎನ್ನ ಎದೆಯೊಳಗೆ ॥
\end{verse}

\begin{verse}
ಶ್ರೀರಾಮಕಲ್ಪತರು ಮೂಲದಲಿ ನಾ ಕುಳಿತು\\ಬೇಕಾದ ಫಲಗಳನು ಪಡೆಯುತಿರುವೆ ॥\\ಪ್ರತಿಫಲವ ನಾ ಬಯಸೆ ಜಗದ ಕಹಿ ಫಲಗಳನು\\ನಿಮ್ಮೊಂದಿಗೇ ಬಿಟ್ಟು ಮುನ್ನಡೆಯುವೆ ॥
\end{verse}


\section{\num{೯೨. } ಬೇಡದ ಮುಳ್ಳು ಕಡಿಮೆ ಏನು ಚುಚ್ಚುವುದಿಲ್ಲ}

ಒಂದು ಸ್ವಪ್ರಯತ್ನ, ಮತ್ತೊಂದು ಭಗವಂತನಲ್ಲಿ ಶರಣಾಗತಿ. ಇವುಗಳ ಲ್ಲಿರುವ ವಿರೋಧಾಭಾಸವನ್ನು ನೋಡಿ! ಇಬ್ಬರು ಶಿಷ್ಯರು ಈ ಸಮಸ್ಯೆಯ ಪರಿಹಾರಕ್ಕೆ ಶ್ರೀರಾಮಕೃಷ್ಣರ ಬಳಿಗೆ ಹೋದರು. ಶ್ರೀರಾಮಕೃಷ್ಣರು, “ಸ್ವಪ್ರಯತ್ನದ ವಿಷಯದ ಮಾತನ್ನು ಏತಕ್ಕೆ ತರುವೆ? ಎಲ್ಲಾ ಭಗವದಧೀನ. ಹಸುವನ್ನು ಗೂಟಕ್ಕೆ ಕಟ್ಟುವಂತೆ ನಮ್ಮ ಇಚ್ಛೆಯನ್ನು ದೇವರಿಗೆ ಕಟ್ಟಿದೆ. ನಮಗೇನೋ ಸ್ವಲ್ಪ ಸ್ವಾತಂತ್ರ್ಯವಿದೆ, ಗೂಟಕ್ಕೆ ಕಟ್ಟಿಹಾಕಿದ ದನವು ತನ್ನ ಕಿರಿಯ ವಲಯದಲ್ಲಿ ತಾರಾಡುವಂತೆ. ಹಾಗೆಯೇ ಮಾನವನು ತನಗೆ ಸ್ವಾತಂತ್ರ್ಯವಿದೆ ಎಂದು ಭಾವಿಸುತ್ತಾನೆ. ಆದರೆ ಅವನ ಇಚ್ಛೆ ಭಗವಂತನ ಇಚ್ಛೆಗೆ ಅಧೀನ.”

ಶಿಷ್ಯ: “ಹಾಗಾದರೆ ನಾವು ಮಾಡುವ ತಪಸ್ಸು, ಧ್ಯಾನ, ಅಧ್ಯಯನ ಇವುಗಳ ಆವಶ್ಯಕತೆ ಏನಿದೆ? ಒಬ್ಬ ಸುಮ್ಮನೆ ಕುಳಿತುಕೊಂಡು, ಇದೆಲ್ಲ ಭಗವದಿಚ್ಛೆ, ಎಲ್ಲವೂ ಅವನ ಇಚ್ಛೆಯಂತೆ ಆಗುತ್ತಿದೆ ಎನ್ನಬಹುದಲ್ಲ?”

ಶ್ರೀರಾಮಕೃಷ್ಣ: “ಸುಮ್ಮನೆ ಬಾಯಿಯಲ್ಲಿ ಹೇಳಿದರೆ ಪ್ರಯೋಜನವಿಲ್ಲ. ನನ್ನ ಕಾಲಿಗೆ ಮುಳ್ಳು ಚುಚ್ಚಿಕೊಂಡಿರುವಾಗ ಮುಳ್ಳಿಲ್ಲ ಮುಳ್ಳಿಲ್ಲ ಎಂದು ಎಷ್ಟು ಬಾರಿ ಹೇಳಿದರೂ ನೋವಿನಿಂದ ಪಾರಾಗಲಾರೆ. ತಮ್ಮ ಇಚ್ಛೆ ಯಂತೆಯೇ ಎಲ್ಲರೂ ಸಾಧನೆ ಮಾಡುವುದಾಗಿದ್ದರೆ ಪ್ರತಿಯೊಬ್ಬನೂ ಹಾಗೆಯೇ ಮಾಡುತ್ತಿದ್ದ. ಇಲ್ಲ, ಎಲ್ಲರಿಗೂ ಇದು ಸಾಧ್ಯವಿಲ್ಲ. ಅದಕ್ಕೆ ಕಾರಣವೇನು? ಆದರೆ ಒಂದು ವಿಷಯ: ದೇವರು ಕೊಟ್ಟ ಸ್ವಾತಂತ್ರ್ಯವನ್ನು ನೀನು ಸರಿಯಾಗಿ ಉಪಯೋಗಿಸದೆ ಇದ್ದರೆ ಅವನು ಮತ್ತಷ್ಟು ಹೆಚ್ಚು ಸ್ವಾತಂತ್ರ್ಯವನ್ನು ಕೊಡುವುದಿಲ್ಲ. ಆದಕಾರಣವೆ ಸ್ವಪ್ರಯತ್ನದ ಆವಶ್ಯಕತೆ. ಭಗವಂತನ ಕೃಪೆಗೆ ಪಾತ್ರರಾಗುವುದಕ್ಕಾಗಿ ಸಾಧನೆ ಮಾಡಬೇಕಾಗಿದೆ. ಅಂತಹ ಪ್ರಯತ್ನದಿಂದ, ಅವನ ಕೃಪೆಯಿಂದ ಹಲವು ಜನ್ಮಗಳ ಕರ್ಮಫಲವನ್ನು ಒಂದೇ ಜನ್ಮದಲ್ಲಿ ಅನುಭವಿಸಿ ಮುಗಿಸಬಹುದು. ಆದರೆ ಸ್ವಲ್ಪ ಸ್ವಪ್ರಯತ್ನ ಆವಶ್ಯಕ. ನಾನೊಂದು ಕಥೆಯನ್ನು ನಿಮಗೆ ಹೇಳುತ್ತೇನೆ ಕೇಳಿ.

“ವೈಕುಂಠಕ್ಕೆ ಅಧಿಪತಿಯಾದ ವಿಷ್ಣು ಒಮ್ಮೆ ನಾರದರಿಗೆ ಶಾಪ ಕೊಟ್ಟ –ನೀನು ಕೆಲವು ಕಾಲ ನರಕವಾಸವನ್ನು ಅನುಭವಿಸು ಎಂದು. ಇದನ್ನು ಕೇಳಿ ನಾರದರಿಗೆ ತುಂಬಾ ಕಳವಳವಾಯಿತು. ಅವರು ಭಗವಂತನ ದಿವ್ಯಸ್ತೋತ್ರ ಗಳನ್ನು ಹಾಡಿ ಪ್ರಾರ್ಥಿಸಿದರು. ನರಕವೆಲ್ಲಿದೆ, ಹೇಗೆ ಅಲ್ಲಿಗೆ ಹೋಗುವುದು ತಿಳಿಸಿರಿ ಎಂದು ಅಂಗಲಾಚಿದರು. ವಿಷ್ಣು ವಿಶ್ವದ ನಕಾಶೆಯನ್ನು ಬರೆದು ಒಂದು ಸೀಮೆಸುಣ್ಣದಿಂದ ಸ್ವರ್ಗ ನರಕಗಳು ಇಲ್ಲಿವೆ ಎಂದು ತೋರಿಸಿದನು. ನಾರದರು ಭೂಪಟದಲ್ಲಿ ಇರುವ ನರಕವನ್ನು ತೋರಿಸುತ್ತಾ, “ಓಹೋ! ಹೀಗಿದೆಯಾ ನರಕ!” ಎಂದು ಅದರ ಮೇಲೆ ಹೊರಳಾಡಿ “ನಾನು ನರಕವನ್ನು ಅನುಭವಿಸಿದೆ” ಎಂದರು. ವಿಷ್ಣುವು ನಗುತ್ತ “ಅದು ಹೇಗೆ” ಎಂದ. ಆಗ ನಾರದರು “ಸ್ವರ್ಗ ನರಕಗಳು ನಿನ್ನ ಸೃಷ್ಟಿಯಲ್ಲವೆ? ನೀನೆ ವಿಶ್ವನಕಾಶೆಯನ್ನು ಬರೆದು ನರಕ ಇಲ್ಲಿದೆ ಎಂಬುದನ್ನು ತೋರಿಸಿದೆ. ಆಗ ಅದೇ ನರಕವಾದಂ ತಾಯಿತು. ನಾನು ಅದರ ಮೇಲೆ ಹೊರಳಾಡಿದುದರಿಂದ ನಾನು ನರಕವಾಸ ವನ್ನು ಅನುಭವಿಸಿದಂತಾಯಿತು,” ಎಂದು ಹೇಳಿದರು. ನಾರದರ ಪ್ರಾಮಾಣಿ ಕತೆಯನ್ನು ವಿಷ್ಣುವು ಮೆಚ್ಚಿದ.


\section{\num{೯೩. } ಭಕ್ತಿಯ ಮತ್ತೊಂದು ಹೆಸರೇ ಏಕಾಗ್ರತೆ}

ಒಬ್ಬ ಮನುಷ್ಯ ದಿನವೆಲ್ಲಾ ಮೀನು ಹಿಡಿಯುವುದರಲ್ಲಿ ನಿರತನಾಗಿದ್ದ. ಬಹಳ ಕಾಲದ ನಂತರ ಮೀನಿಗೆ ಇಟ್ಟಿದ್ದ ಬೆಂಡು ಚಲಿಸತೊಡಗಿತು. ಬೆಸ್ತ ಗಾಳವನ್ನು ಕೈಗಳಿಂದ ಬಿಗಿಯಾಗಿ ಹಿಡಿದುಕೊಂಡಿದ್ದ. ಅದನ್ನು ಮೇಲೆತ್ತು ವುದರಲ್ಲಿದ್ದ. ದಾರಿಯಲ್ಲಿ ಹೋಗುತ್ತಿದ್ದವನೊಬ್ಬ ನಿಂತು, “ಮಹಾಶಯರೆ, ಬ್ಯಾನರ್ಜಿ ಮನೆ ಎಲ್ಲಿದೆ ದಯವಿಟ್ಟು ಹೇಳುವಿರಾ” ಎಂದು ಕೇಳಿದ. ಮೀನು ಹಿಡಿಯುವವನಿಂದ ಯಾವ ಉತ್ತರವೂ ಬರಲಿಲ್ಲ. ಅವನು ಮೀನನ್ನು ಮೇಲಕ್ಕೆ ಸೆಳೆಯುವುದರಲ್ಲಿ ನಿರತನಾಗಿದ್ದ. ಪುನಃ ಪುನಃ ಅಪರಿಚಿತನು “ಬ್ಯಾನರ್ಜಿ ಮನೆ ಎಲ್ಲಿದೆ ಹೇಳುತ್ತೀರಾ” ಎಂದು ಕೇಳುತ್ತಿದ್ದ. ಆದರೆ ಬೆಸ್ತನಿಗೆ ಇದರ ಜ್ಞಾನವೇ ಇರಲಿಲ್ಲ. ಅವನ ಕೈಗಳು ನಡುಗುತ್ತಿದ್ದುವು. ಅವನ ಕಣ್ಣುಗಳು ಗಾಳಕ್ಕೆ ಕಟ್ಟಿದ ಬೆಂಡನ್ನು ನೋಡುತ್ತಿದ್ದುವು. ಪರಸ್ಥಳದವನಿಗೆ ಬೇಜಾರಾಗಿ ಅಲ್ಲಿಂದ ಮುಂದೆ ಹೋದ. ಅವನು ಸ್ವಲ್ಪ ದೂರ ಹೋದ ಮೇಲೆ, ಇತ್ತ ಮೀನಿನ ಬೆಂಡು ಮುಳುಗಿತು. ಆಗ ಒಂದೇ ಸೆಳೆತದಲ್ಲಿ ಮೀನನ್ನು ಮೇಲೆ ಎತ್ತಿದ. ಅವನು ತನ್ನ ಮೊಗದ ಮೇಲೆ ಇದ್ದ ಬೆವರನ್ನು ಒರಸಿಕೊಂಡು ಪರಸ್ಥಳದವನನ್ನು ಕುರಿತು, “ಇಲ್ಲಿ ಬಾ, ಕೇಳು” ಎಂದ. ಆದರೆ ಅವನು ಹಿಂತಿರುಗಿ ನೋಡಲಿಲ್ಲ. ಬಹಳ ಹೊತ್ತು ಕೂಗಿದ ಮೇಲೆ ಅವನು ಬಂದ. “ಏತಕ್ಕೆ ನನ್ನನ್ನು ಕೂಗಿದೆ?” ಎಂದು ಕೇಳಿದ. “ನೀನು ನನ್ನನ್ನು ಏನು ಕೇಳುತ್ತಿದ್ದಿ?” ಅಪರಿಚಿತ, “ನಾನು ನಿನ್ನನ್ನು ಎಷ್ಟೊಂದು ವೇಳೆ ಕೇಳಿದೆ! ಈಗ ಪುನಃ ಅದನ್ನು ಹೇಳು ಎನ್ನುತ್ತಿದ್ದೀಯ” ಎಂದ. ಬೆಸ್ತರವನು, “ನನ್ನ ಮೀನಿನ ಬೆಂಡು ನೀರಿನಲ್ಲಿ ಮುಳುಗುವುದರಲ್ಲಿತ್ತು. ಅದಕ್ಕೇ ನೀನು ಹೇಳಿದ್ದು ಏನೂ ಕೇಳಿಸಲಿಲ್ಲ” ಎಂದ.


\chapter{ವ್ಯಾಕುಲತೆ}

\section{\num{೯೪. } ದೇವರಿಗೆ ಒಬ್ಬ ವ್ಯಾಕುಲನಾದರೆ ಅವನನ್ನು ಪಡೆಯಬಹುದು}

ಒಬ್ಬನಿಗೆ ಒಬ್ಬಳು ಮಗಳು ಇದ್ದಳು. ಬಾಲ್ಯದಲ್ಲೇ ಅವಳು ವಿಧವೆ ಯಾದಳು. ಅವಳಿಗೆ ಗಂಡನ ಪರಿಚಯವೇ ಇರಲಿಲ್ಲ. ಇತರ ಯುವತಿಯರಿಗೆ ಗಂಡಂದಿರಿರುವುದನ್ನು ಅವಳು ನೋಡುತ್ತಿದ್ದಳು. ಒಂದು ದಿನ ತನ್ನ ತಂದೆ ಯನ್ನು, “ನನ್ನ ಗಂಡ ಎಲ್ಲಿರುವನು?” ಎಂದು ಕೇಳಿದಳು. ತಂದೆ “ಗೋವಿಂದನೇ ನಿನ್ನ ಗಂಡ. ನೀನು ಅವನನ್ನು ಕರೆದರೆ ಅವನು ಬರುವನು” ಎಂದನು. ಈ ಮಾತನ್ನು ಕೇಳಿ ಅವಳು ತನ್ನ ಕೋಣೆಗೆ ಹೋಗಿ ಬಾಗಿಲನ್ನು ಮುಚ್ಚಿ “ಗೋವಿಂದ, ನನ್ನ ಬಳಿಗೆ ಬಾ, ನಿನ್ನ ಮುಖವನ್ನು ತೋರು, ನೀನು ಏತಕ್ಕೆ ಬರುವುದಿಲ್ಲ,” ಎಂದು ವ್ಯಾಕುಲತೆಯಿಂದ ಅತ್ತು ಕರೆದಳು. ಭಗವಂತ ಅವಳ ವ್ಯಾಕುಲತೆಯ ಕರೆಯನ್ನು ಕೇಳಿ ಅವಳಿಗೆ ಪ್ರತ್ಯಕ್ಷನಾದನು.


\section{\num{೯೫. } ಭಗವತ್ ವ್ಯಾಕುಲತೆ}

ವ್ಯಾಕುಲತೆಯಿಲ್ಲದೆ ಭಗವಂತ ಬಾರನು. ಈ ವ್ಯಾಕುಲತೆ ಬರಬೇಕಾದರೆ ಪ್ರಾಪಂಚಿಕ ಅನುಭವಗಳನ್ನು ಪೂರೈಸಿರಬೇಕು. ಯಾರು ಕಾಮಕಾಂಚನದಲ್ಲಿ ಮುಳುಗಿರುವರೊ ಅವರಿಗೆ ವ್ಯಾಕುಲತೆ ಬರುವುದಿಲ್ಲ.

ನಾನು ಕಾಮಾರಪುಕುರದಲ್ಲಿದ್ದಾಗ ಹೃದಯನ ಮಗನಿಗೆ ನಾಲ್ಕೈದು ವರ್ಷ ವಯಸ್ಸಾಗಿತ್ತು. ಅವನು ದಿನವೆಲ್ಲ ನನ್ನೊಡನೆ ಇರುತ್ತಿದ್ದ. ಅವನು ಆಟದಲ್ಲಿ ಎಲ್ಲವನ್ನೂ ಮರೆಯುತಿದ್ದ. ಸಂಜೆಯಾದ ಒಡನೆ ನಾನು ತಾಯಿಯ ಬಳಿಗೆ ಹೋಗಬೇಕು ಎಂದು ಹಟ ಹಿಡಿಯುತ್ತಿದ್ದ. ನಾನು ಹೇಗೋ ಮಾಡಿ ಅವನ ಮನಸ್ಸನ್ನು ಬೇರೆ ಕಡೆ ತಿರುಗಿಸಲು ಯತ್ನಿಸುತ್ತಿದ್ದೆ. “ನೋಡು, ನಿನಗೊಂದು ಪಾರಿವಾಳ ಹಕ್ಕಿಯನ್ನು ತಂದುಕೊಡುವೆನು” ಎಂದು ಹೇಳಿದೆ. ಆ ಮಗುವಿಗೆ ಅದರಿಂದ ಸಮಾಧಾನವಾಗಲಿಲ್ಲ. ಅಳುತ್ತ ನಾನು ಅಮ್ಮನ ಬಳಿಗೆ ಹೋಗ ಬೇಕು ಎಂದು ಹಟ ಹಿಡಿಯುತ್ತಿದ್ದ. ಅವನಿಗೆ ಇನ್ನು ಮೇಲೆ ಆಟ ಹಿಡಿಸ ದಾಯಿತು. ಅವನ ಕಾತರತೆ ಯನ್ನು ನೋಡಿ ನನಗೇ ಕಣ್ಣಿನಲ್ಲಿ ನೀರು ಬರುತ್ತಿತ್ತು.

ಮಗುವಿನಂತೆ ದೇವರನ್ನು ಕರೆಯಬೇಕು. ಭಗವಂತನಿಗಾಗಿ ವ್ಯಾಕುಲತೆ ಎಂದರೆ ಇದೇನೆ. ಅವನಿಗೆ ಇನ್ನು ಮೇಲೆ ಆಟ ಊಟ ಯಾವುದೂ ಹಿಡಿಸದು. ಸಂಸಾರದ ಅನುಭವ ತೀರಿದ ಮೇಲೆ ಈ ವ್ಯಾಕುಲತೆ ಬರುವುದು. ಆಗ ಭಗವಂತನಿಗಾಗಿ ಮನುಷ್ಯ ಅಳುವನು.


\section{\num{೯೬. } ನಿನಗೆ ಉತ್ಕಟ ಆಕಾಂಕ್ಷೆ ಇದ್ದರೆ}

ಮಾನವನಿಗೆ ಸರಿಯಾದ ಮಾರ್ಗ ತೋರದೆ ಇರಬಹುದು. ಅವನಲ್ಲಿ ಭಕ್ತಿ ಇದ್ದರೆ, ಭಗವಂತನ ಬಳಿಗೆ ಹೋಗಬೇಕೆಂದು ಇಚ್ಛೆ ಇದ್ದರೆ ತೀವ್ರ ಭಕ್ತಿ ಯಿಂದಲೇ ಅವನನ್ನು ಪಡೆಯಬಹುದು. ಒಂದು ಸಲ ಒಬ್ಬ ಭಕ್ತ ಪೂರಿ ಯಲ್ಲಿರುವ ಜಗನ್ನಾಥನನ್ನು ನೋಡಲು ಹೊರಟ. ಅವನಿಗೆ ದಾರಿ ಗೊತ್ತಿರ ಲಿಲ್ಲ. ದಕ್ಷಿಣಕ್ಕೆ ಹೋಗುವ ಬದಲು ಅವನು ಪಶ್ಚಿಮದ ಕಡೆಗೆ ಹೊರಟನು. ಅವನು ದಾರಿಯನ್ನೇನೊ ತಪ್ಪಿದ. ದಾರಿಯಲ್ಲಿ ಸಿಕ್ಕಿದವರನ್ನು ಆಸಕ್ತಿಯಿಂದ ಕೇಳಿದ. ಅವರು ಅವನಿಗೆ ಸರಿಯಾದ ದಾರಿಯನ್ನು ತೋರಿಸಿದರು. ‘ಇದಲ್ಲ ದಾರಿ ಬೇರೆ ಕಡೆ ಹೋಗು’ ಎಂದರು. ಕೊನೆಗೆ ಭಕ್ತ ಪೂರಿಗೆ ಹೋಗಿ ಜಗ ನ್ನಾಥನ ದರ್ಶನ ಪಡೆದ. ನಿನಗೆ ಗೊತ್ತಿಲ್ಲದೆ ಇದ್ದರೂ ನಿನಗೆ ಆಸಕ್ತಿ ಇದ್ದರೆ ಯಾರಾದರೂ ದಾರಿಯನ್ನು ತೋರುವರು.


\section{\num{೯೭. } ಗುರು ಶಿಷ್ಯನಿಗೆ ಹೇಗೆ ದೇವರನ್ನು ತೋರಿದ}

ಒಬ್ಬ ಶಿಷ್ಯ ಗುರುವನ್ನು ಕುರಿತು, “ಸ್ವಾಮಿ, ನಾನು ದೇವರನ್ನು ಪಡೆಯುವ ಬಗೆ ಹೇಗೆ” ಎಂದು ಕೇಳಿದ. “ನನ್ನ ಜತೆಯಲ್ಲಿ ಬಾ. ನಾನು ನಿನಗೆ ತೋರು ವೆನು” ಎಂದನು ಗುರು. ಅವನು ಶಿಷ್ಯನನ್ನು ಒಂದು ಸರೋವರಕ್ಕೆ ಕರೆದು ಕೊಂಡು ಹೋದನು. ಇಬ್ಬರೂ ನೀರಿನಲ್ಲಿ ಇಳಿದರು. ಇದ್ದಕ್ಕಿದ್ದಂತೆ ಗುರುವು ಶಿಷ್ಯನನ್ನು ನೀರಿನಲ್ಲಿ ಮುಳುಗಿಸಿ ಹಿಡಿದುಕೊಂಡನು. ಕೆಲವು ಕ್ಷಣಗಳ ನಂತರ ಅವನನ್ನು ಮೇಲೆ ಬಿಟ್ಟನು. ಗುರು ಶಿಷ್ಯನನ್ನು ಕೇಳಿದನು: “ನೀನು ನೀರಿನಲ್ಲಿ ದ್ದಾಗ ಹೇಗನಿಸಿತು” ಎಂದು. ಶಿಷ್ಯ, “ನಾನು ಸತ್ತುಹೋಗುತ್ತೇನೆ ಎಂದು ಭಾವಿಸಿದೆ. ನಾನು ನೀರುಕಟ್ಟಿಕೊಂಡು ಗಾಳಿಗಾಗಿ ಹಾತೊರೆಯುತ್ತಿದ್ದೆ.” ಆಗ ಗುರುವು “ನಿನಗೆ ದೇವರ ವಿಷಯದಲ್ಲಿ ಅಂತಹ ವ್ಯಾಕುಲತೆ ಬಂದರೆ, ನೀನು ಅವನಿಗಾಗಿ ಬಹಳ ಕಾಲ ಕಾಯಬೇಕಾಗಿಲ್ಲ ಎಂಬುದು ಗೊತ್ತಾಗುವುದು”, ಎಂದನು.


\section{\num{೯೮. } ಸ್ವಪ್ರಯತ್ನ ಮತ್ತು ಶರಣಾಗತಿ}

ಒಬ್ಬ ವ್ಯಕ್ತಿ ತನ್ನ ಇಬ್ಬರು ಮಕ್ಕ ಳೊಡನೆ ಒಂದು ಬಯಲ ಮೂಲಕ ಹೋಗುತ್ತಿದ್ದ. ಒಬ್ಬ ಮಗನನ್ನು ತನ್ನ ಸೊಂಟದಲ್ಲಿ ಕೂರಿಸಿ ಕೊಂಡಿದ್ದ, ಮತ್ತೊಬ್ಬನು ತಂದೆಯ ಕೈ ಹಿಡಿದುಕೊಂಡು ಹೋಗುತ್ತಿದ್ದ. ಒಂದು ಹದ್ದು ಮೇಲೆ ಹಾರಾಡು ತ್ತಿತ್ತು. ಕೆಳಗೆ ನಡೆಯುತ್ತಿದ್ದ ಹುಡುಗ ಕೈ ಬಿಟ್ಟು ತಂದೆಗೆ “ಅಪ್ಪ, ನೋಡಲ್ಲಿ ಒಂದು ಹದ್ದು ಇದೆ” ಎಂದು ಆನಂದದಿಂದ ಚಪ್ಪಾಳೆ ಹಾಕತೊಡಗಿದ. ಆದರೆ ಕೂಡಲೆ ಮುಗ್ಗರಿಸಿ ಬಿದ್ದು ಕಾಲು ಗಾಯ ಮಾಡಿಕೊಂಡ. ಸೊಂಟದಲ್ಲಿದ್ದ ಮಗುವು ಕೂಡ ಹದ್ದನ್ನು ನೋಡಿ ಕೈಯಿಂದ ಚಪ್ಪಾಳೆ ತಟ್ಟಿತು. ಆದರೆ ಅದು ಕೆಳಗೆ ಬೀಳಲಿಲ್ಲ. ತಂದೆ ಅವನನ್ನು ಬಲವಾಗಿ ಹಿಡಿದುಕೊಂಡಿದ್ದ. ಮೊದಲನೆಯವನು ಸ್ವ ಪ್ರಯತ್ನಕ್ಕೆ ಉದಾಹರಣೆ. ಎರಡನೆಯವನು ಶರಣಾ ಗತಿಗೆ ಉದಾಹರಣೆ.


\section{\num{೯೯. } ನಾರಾಯಣದೇವ ಮತ್ತು ಸ್ವಪ್ರಯತ್ನ}

ಒಂದು ಸಲ ಲಕ್ಷ್ಮೀನಾರಾಯಣರು ವೈಕುಂಠದಲ್ಲಿದ್ದರು. ಲಕ್ಷ್ಮಿ ಅವನ ಕಾಲನ್ನು ನೀವುತ್ತಿದ್ದಳು. ನಾರಾಯಣ ಇದ್ದಕ್ಕಿದ್ದಂತೆ ಎದ್ದು ನಿಂತ. ಲಕ್ಷ್ಮಿ “ಸ್ವಾಮಿ, ನೀವು ಎಲ್ಲಿಗೆ ಹೊರಟಿರಿ?” ಎಂದು ಕೇಳಿದಳು. “ನನ್ನ ಒಬ್ಬ ಭಕ್ತ ತುಂಬಾ ಅಪಾಯದಲ್ಲಿರುವನು, ನಾನು ಅವನನ್ನು ರಕ್ಷಿಸಬೇಕು. ಅದಕ್ಕೆ ಹೋಗುತ್ತಿರುವೆನು” ಎಂದು ಹೇಳಿ ನಾರಾಯಣ ಹೊರ ಟನು. ಆದರೆ, ತತ್​ಕ್ಷಣವೇ ಹಿಂತಿರುಗಿದಾಗ, “ಸ್ವಾಮಿ, ಹಾಗಾ ದರೆ ನೀವು ಏತಕ್ಕೆ ಇಷ್ಟು ಬೇಗ ಬಂದದ್ದು?” ಎಂದು ಲಕ್ಷ್ಮಿ ಕೇಳಿದಳು. ನಾರಾಯಣ ನಗುತ್ತ ಹೇಳಿದ: “ನನ್ನಲ್ಲಿ ಭಕ್ತಿ ಭಾವದಿಂದ ತಲ್ಲೀನ ನಾಗಿ ಭಕ್ತನು ನಡೆದುಕೊಂಡು ಹೋಗುತ್ತಿದ್ದ. ಕೆಲವು ಅಗಸರು ಬಟ್ಟೆಯನ್ನು ಹರಡಿದ್ದರು. ಆ ಬಟ್ಟೆಯ ಮೇಲೆ ಭಕ್ತ ನಡೆದ. ಇದನ್ನು ನೋಡಿದ ಅಗಸರು ಭಕ್ತನನ್ನು ಕೋಲಿನಿಂದ ಹೊಡೆಯುವುದರಲ್ಲಿದ್ದರು. ಅದಕ್ಕೇ ಅವನನ್ನು ರಕ್ಷಿ ಸಲು ನಾನು ಹೋದೆ.” “ಆದರೆ ಏತಕ್ಕೆ ಇಷ್ಟು ಬೇಗ ಬಂದಿರಿ?” ಎಂದು ಲಕ್ಷ್ಮಿ ಕೇಳಿದಳು. ನಾರಾಯಣ ನಗುತ್ತ, “ಭಕ್ತನೇ ಇಟ್ಟಿಗೆಯಿಂದ ಅವರನ್ನು ಹೊಡೆಯಲು ಹೊರಟಿದ್ದ. ಅದಕ್ಕೇ ನಾನು ಹಿಂದಿರುಗಿದೆ” ಎಂದನು.


\section{\num{೧೦೦. } ಶರಣಾಗತನೆಂದಿಗೂ ಗೊಣಗಾಡುವುದಿಲ್ಲ}

ರಾಮಲಕ್ಷ್ಮಣರು ಪಂಪಾಸರೋವರದಲ್ಲಿ ಸ್ನಾನಕ್ಕೆ ಇಳಿದಾಗ, ತಮ್ಮ ಧನುಸ್ಸನ್ನು ನೆಲದಲ್ಲಿ ಹೂಳಿಟ್ಟರು. ನೀರಿನಿಂದ ಮೇಲೆ ಬಂದು ಲಕ್ಷ್ಮಣ ತನ್ನ ಧನುಸ್ಸನ್ನೆತ್ತಿದಾಗ ನೋಡಿದ, ಅದರ ಕೆಳಭಾಗ ರಕ್ತಮಯವಾಗಿತ್ತು. ರಾಮ ಲಕ್ಷ್ಮಣನಿಗೆ, “ತಮ್ಮ, ನಾವು ಯಾರಿಗಾದರೂ ಹಿಂಸೆ ಮಾಡಿರಬಹುದು. ನೋಡು” ಎಂದ. ಲಕ್ಷ್ಮಣ ನೆಲವನ್ನು ಅಗೆದು ನೋಡುತ್ತಾನೆ, ಅಲ್ಲಿ ಒಂದು ದೊಡ್ಡ ಕಪ್ಪೆ ಇತ್ತು. ಅದು ಸಾಯುವ ಸ್ಥಿತಿಯಲ್ಲಿತ್ತು. ರಾಮ ಕಪ್ಪೆಗೆ ದುಃಖ ದಿಂದ ಹೇಳಿದ “ನೀನು ಏತಕ್ಕೆ ಶಬ್ದ ಮಾಡಲಿಲ್ಲ? ಮಾಡಿದ್ದರೆ ನಿನ್ನನ್ನು ಉಳಿಸಬಹುದಿತ್ತು. ನೀನು ಹಾವಿನ ಬಾಯಿಗೆ ಬಿದ್ದರೆ ಬೇಕಾದಷ್ಟು ಕಿರಿಚಾ ಡುವೆ.” ಅದಕ್ಕೆ ಕಪ್ಪೆ ಹೇಳಿತು, “ಭಗವಂತ, ನಾನು ಹಾವಿನ ಬಾಯಿಗೆ ಸಿಕ್ಕಿದಾಗ ರಾಮನನ್ನು ‘ರಕ್ಷಿಸು’ ಎಂದು ಬೇಡುತ್ತೇನೆ. ಈ ಸಲ ರಾಮನೆ ಕೊಲ್ಲುತ್ತಿರು ವನು ಎಂಬುದನ್ನು ನೋಡಿದಾಗ ನಾನು ಸುಮ್ಮನೆ ಇದ್ದೆ.”

\chapter{ನಮ್ರತೆ}

\section{\num{೧೦೧. } ನಿಜವಾದ ನಮ್ರತೆಯನ್ನು ಕಲಿಯುವುದು ಬಹಳ ಕಷ್ಟ}

ಒಬ್ಬ ವ್ಯಕ್ತಿಯು ಸಾಧುವಿನ ಬಳಿಗೆ ಹೋಗಿ ನಮ್ರತೆಯನ್ನು ನಟಿಸುತ್ತ, “ನಾನು ತುಂಬಾ ಕೀಳು, ಸ್ವಾಮಿಗಳೆ, ನಾನು ಉದ್ಧಾರವಾಗುವುದು ಹೇಗೆ ಎಂಬುದನ್ನು ಹೇಳಿ” ಎಂದ. ಸಾಧುಗಳು ಅವನ ಆಂತರ್ಯವನ್ನು ಗ್ರಹಿಸಿ, “ಹೋಗಿ ನಿನಗಿಂತ ಕೀಳಾಗಿರುವುದನ್ನು ತೆಗೆದುಕೊಂಡು ಬಾ” ಎಂದರು. ಅವನು ಹೋಗಿ ಸುತ್ತ ನೋಡಿದ. ತನಗಿಂತ ಕೀಳಾಗಿ ಇರುವುದು ಯಾವುದೂ ಕಾಣಲಿಲ್ಲ. ಅವನು ತನ್ನ ಹೇಸಿಗೆಯನ್ನೇ ಕಂಡು “ಇದು ನಿಶ್ಚಯವಾಗಿ ನನಗಿಂತ ಕೀಳಾಗಿರುವುದು” ಎಂದು ಭಾವಿಸಿದನು. ಅದನ್ನು ಸಾಧುವಿಗೆ ಕೊಡಬೇಕೆಂದು ತೆಗೆದುಕೊಳ್ಳಲು ಕೈ ಚಾಚಿದಾಗ ಹೇಸಿಗೆಯಿಂದ ಒಂದು ವಾಣಿ ಬಂತು: “ಪಾಪಿಯೆ, ನನ್ನನ್ನು ಮುಟ್ಟಬೇಡ. ನಾನು ಮುಂಚೆ ಒಳ್ಳೆಯ ರುಚಿಕರವಾದ ಆಹಾರವಸ್ತುವಾಗಿದ್ದೆ, ಭಗವಂತನಿಗೆ ನೀಡಲು ಯೋಗ್ಯವಾಗಿದ್ದೆ. ನೋಡುವು ದಕ್ಕೆ ಎಲ್ಲರಿಗೂ ಚೆನ್ನಾಗಿದ್ದೆ. ನನ್ನ ದುರದೃಷ್ಟದಿಂದ ನನಗೆ ನಿನ್ನ ಸಹವಾಸ ಸಿಕ್ಕಿತು. ನಿನ್ನ ಸಹವಾಸದಿಂದ ಜನ ನನ್ನನ್ನು ನೋಡಿದರೆ ಮೂಗು ಮುಚ್ಚಿ ಕೊಳ್ಳಬೇಕು, ಅಂತಹ ಸ್ಥಿತಿಗೆ ಬಂದೆ. ಈಗ ಜನರು ಈ ದುರ್ವಾಸನೆಯನ್ನು ತಡೆಯಲಾಗದೆ ಬಟ್ಟೆಯಿಂದ ಮೂಗು ಮುಚ್ಚಿಕೊಳ್ಳುವರು. ನಾನು ಒಂದು ಸಲ ಮಾತ್ರ ನಿನ್ನ ಸಹವಾಸಕ್ಕೆ ಬಿದ್ದೆ. ಅದರಿಂದ ಎಂತಹ ದುರ್ಗತಿ ನನಗೆ ಬಂತು! ನೀನು ಮತ್ತೊಮ್ಮೆ ನನ್ನನ್ನು ಮುಟ್ಟಿದರೆ ಇನ್ನು ಎಂತಹ ಕೀಳಾದ ಅವಸ್ಥೆ ನನಗೆ ಕಾದಿದೆಯೊ ಏನೋ.” ಆ ವ್ಯಕ್ತಿಯು ನಿಜವಾದ ನಮ್ರತೆ ಏನು ಎಂಬುದನ್ನು ಇದರಿಂದ ಕಲಿತುಕೊಂಡು ಅತ್ಯಂತ ನಮ್ರನಾದನು. ಇದರಿಂದ ಅವನು ಪರಮ ಪದವಿಯನ್ನು ಪಡೆದನು.

\chapter{ತ್ಯಾಗ ಮತ್ತು ವೈರಾಗ್ಯ}

\section{\num{೧೦೨. } ಹೋಮ ಹಕ್ಕಿ}

ವೇದಗಳಲ್ಲಿ ಹೋಮ ಹಕ್ಕಿಯ ಕಥೆ ಇದೆ. ಅದು ಮೇಲೆ ಅಂತರಿಕ್ಷದಲ್ಲಿ ಬಹು ಎತ್ತರದಲ್ಲಿರುತ್ತದೆ. ಅಲ್ಲಿ ತಾಯಿಹಕ್ಕಿ ಮೊಟ್ಟೆಯನ್ನು ಇಡುತ್ತದೆ. ಅದು ಎಷ್ಟು ಮೇಲೆ ಇರುತ್ತದೆಂದರೆ ಆ ಮೊಟ್ಟೆ ಕೆಳಗೆ ಬೀಳಲು ಬಹಳದಿನಗಳು ಹಿಡಿಯುತ್ತದೆ. ಅದು ಬೀಳುತ್ತಿರುವಾಗಲೇ ಮೊಟ್ಟೆ ಒಡೆಯುವುದು. ಈಗ ಮರಿ ಹೊರಬಂದು ಕೆಳಗೆ ಬೀಳತೊಡಗು ತ್ತದೆ. ಅಷ್ಟರಲ್ಲೇ ಮರಿಗೆ ಕಣ್ಣು ಬಂದು, ಕಣ್ಣು ಬಿಡುತ್ತದೆ. ನೆಲದ ಹತ್ತಿರ ಬರುವಾಗ ನೆಲವನ್ನು ನೋಡು ವುದು. ತಾನು ಕೆಳಕ್ಕೆ ಬಿದ್ದರೆ ಸಾವು ನಿಶ್ಚಿತ ಎಂಬುದನ್ನು ಅರಿತು, ಜೋರಾಗಿ ಕೂಗಿ ಅನಂತರ ತನ್ನ ತಾಯಿಯಿರುವೆಡೆಗೆ ಮೇಲೆ ಹಾರು ವುದು. ತಾಯಿಹಕ್ಕಿ ಬಹಳ ಮೇಲೆ ಇರುವುದು. ಮರಿ ತಾಯಿಯತ್ತ ಹಾರುವುದು. ಮತ್ತೆಲ್ಲೂ ನೋಡುವು ದಿಲ್ಲ.

ದೈವಾಂಶದೊಡನೆ ಹುಟ್ಟುವ ಮಕ್ಕಳಿಗೆ ಪ್ರಪಂಚವನ್ನು ಮುಟ್ಟಿದರೆ ಅಪಾಯ ಎನ್ನುವುದರ ಅರಿವಿರುವುದು. ಹುಟ್ಟಿನಿಂದಲೇ ಅವರಿಗೆ, ಪ್ರಪಂಚ ಎಂದರೆ ಅಂಜಿಕೆ. ಅವರ ಒಂದು ಉದ್ದೇಶವೇ ತಾಯಿಹಕ್ಕಿಯನ್ನು, ಭಗವಂತನನ್ನು ಯಾವಾಗ ಸೇರೇನು ಎಂಬುದು.


\section{\num{೧೦೩. } ತನ್ನ ಭುಜದ ಮೇಲೆ ಟವಲನ್ನು ಹಾಕಿಕೊಂಡು ಹೊರಟುಹೋದ}

ಒಬ್ಬ ಮನುಷ್ಯ ಸ್ನಾನಕ್ಕೆ ಅಣಿಯಾಗುತ್ತಿದ್ದ. ಅವನ ಭುಜದ\\ಮೇಲೆ ಟವೆಲ್ ಇತ್ತು. ಅವನ ಹೆಂಡತಿ ಗಂಡನಿಗೆ, “ನೀವು\\ಕೆಲಸಕ್ಕೆ ಬಾರದವರು. ನಿಮಗೆ ವಯಸ್ಸಾಗುತ್ತಾ ಬರುತ್ತಿದೆ. ಆದರೂ ನಿಮ್ಮ ಹಳೆಯ ಅಭ್ಯಾಸಗಳನ್ನು ಬಿಡಲಾರಿರಿ. ನಾನಿಲ್ಲದೆ ಒಂದು ದಿನವೂ ಇರುವುದಕ್ಕೆ ಸಾಧ್ಯವಿಲ್ಲ. ಆ ನೆರೆಮನೆಯವನನ್ನು ನೋಡಿ, ಅವನು ಎಂತಹ ತ್ಯಾಗಿ!”

ಗಂಡ ಹೆಂಡತಿಯನ್ನು, ಕೇಳಿದ “ಅವನು ಏನು ಮಾಡಿದ?” ಎಂದು. ಆಗ ಹೆಂಡತಿ ಹೇಳಿದಳು, “ಅವನಿಗೆ ಹದಿನಾರು ಜನ ಹೆಂಡಂದಿರಿದ್ದರು. ಒಬ್ಬೊಬ್ಬ ರನ್ನಾಗಿ ಅವರನ್ನು ಬಿಡುತ್ತಿರುವನು. ನೀವು ಎಂದಿಗೂ ಅದನ್ನು ಮಾಡಲಾರಿರಿ” ಎಂದು ಮೂದಲಿಸಿದಳು.

ಅದಕ್ಕೆ ಗಂಡ ಹೇಳಿದ: “ಏನು ಒಬ್ಬೊಬ್ಬರನ್ನಾಗಿ ಅವನು ಬಿಡುತ್ತಿರುವನೆ! ನೀನು ಹುಚ್ಚಿ, ಅವನು ಅದನ್ನು ಮಾಡಲಾರ. ಸ್ವಲ್ಪಸ್ವಲ್ಪವಾಗಿ ಯಾರಾದರೂ ತ್ಯಾಗ ಮಾಡಲು ಸಾಧ್ಯವೇ?” “ಆದರೂ ಅವನು ನಿಮಗಿಂತ ಮೇಲು” ಎಂದಳು ಹೆಂಡತಿ ನಗುನಗುತ್ತ. ಗಂಡ ಹೆಂಡತಿಗೆ ಹೇಳಿದ: “ ನೀನು ಮಂದ ಮತಿ. ಅವನಿಂದ ತ್ಯಾಗ ಸಾಧ್ಯವಿಲ್ಲ. ಆದರೆ ಅದು ನನಗೆ ಸಾಧ್ಯ. ನೋಡು ನಾನು ಈಗಲೆ ಹೊರಟೇಹೋದೆ.”

ಅದೇನೇ ತೀವ್ರ ವೈರಾಗ್ಯ ಎಂಬುದು. ಅವನು ಮನಸ್ಸು ಮಾಡಿದೊಡನೆ ಅದನ್ನು ಮಾಡಿದ. ಅವನ ಭುಜದ ಮೇಲೆ ಟವೆಲ್ ಇತ್ತು. ಹಾಗೆಯೆ ಅವನು ಹೊರಟೇಹೋದ. ತನ್ನ ಸಂಸಾರಕ್ಕಾಗಿ ಯಾವ ವ್ಯವಸ್ಥೆಯನ್ನೂ ಮಾಡುವುದಕ್ಕೆ ಹೋಗಲಿಲ್ಲ. ಅವನು ಮನೆಯ ಕಡೆ ಹಿಂದಿರುಗಿ ನೋಡಲೂ ಇಲ್ಲ. ಯಾರು ತ್ಯಾಗ ಮಾಡಬೇಕೆಂದಿರುವರೋ ಅವರಿಗೆ ಅದ್ಭುತ ಇಚ್ಛಾಶಕ್ತಿ ಇರಬೇಕು. ದರೋಡೆಕೋರನಲ್ಲಿರುವಂತಹ ಹುಚ್ಚುಧೈರ್ಯ ಇರಬೇಕು. ಮನೆಯನ್ನು ಕೊಳ್ಳೆಹೊಡೆಯುವುದಕ್ಕೆ ಮುಂಚೆ ದರೋಡೆಕೋರರು ‘ಕೊಲ್ಲಿ; ಖೂನಿ ಮಾಡಿ, ಲೂಟಿ ಮಾಡಿ’ ಎನ್ನುವರು.


\section{\num{೧೦೪. } ವೈರಾಗ್ಯ ತೊಟ್ಟು ತೊಟ್ಟು ಬರುವುದಿಲ್ಲ; ಅದೊಂದು ಪ್ರವಾಹದಂತೆ ಬರುವುದು}

ಮನುಷ್ಯನಿಗೆ ವೈರಾಗ್ಯ ಹೇಗೆ ಬರುವುದು?

ಹೆಂಡತಿಯೊಬ್ಬಳು ಗಂಡನಿಗೆ ಹೇಳಿದಳು. “ನೋಡಿ ನನ್ನ ತಮ್ಮನನ್ನು ನೋಡಿದರೆ ನನಗೆ ಏನೋ ಅಂಜಿಕೆಯಾಗುವುದು. ಅವನು ಕಳೆದ ಒಂದು ವಾರದಿಂದ ಸಂನ್ಯಾಸಿಯಾಗಬೇಕೆಂದು ಯೋಚಿಸುತ್ತಿರುವನು. ಅಂತಹ ಜೀವನವನ್ನು ನಡೆಸುವುದಕ್ಕೆ ಅವನು ಅಣಿಯಾಗುತ್ತಿರುವನು. ಕ್ರಮ ವಾಗಿ ಎಲ್ಲ ಆಸೆಗಳನ್ನೂ ಕಡಿಮೆಮಾಡುತ್ತಿರುವನು.” ಗಂಡ ಹೇಳಿದ, “ನೋಡು, ನಿನ್ನ ತಮ್ಮನ ವಿಷಯದಲ್ಲಿ ನೀನು ಚಿಂತಾಕ್ರಾಂತಳಾಗುವುದು ಬೇಡ. ಅವನು ಎಂದಿಗೂ ಸಂನ್ಯಾಸಿಯಾಗುವುದಿಲ್ಲ.” “ಹಾಗಾದರೆ ಸಂನ್ಯಾಸಿ ಯಾಗುವ ಬಗೆ ಹೇಗೆ?” ಎಂದು ಹೆಂಡತಿ ಕೇಳಿದಳು. “ಅದು ಹೀಗೆ” ಎಂದು ತನ್ನ ಬಟ್ಟೆಯನ್ನು ಹರಿದು ಅದನ್ನು ಕೌಪೀನವನ್ನಾಗಿ ಮಾಡಿ, “ಇನ್ನುಮೇಲೆ ಹೆಂಡತಿ ಮತ್ತು ಇತರ ಮಹಿಳೆಯರೆಲ್ಲ ನನಗೆ ತಾಯಿ ಸಮಾನ” ಎಂದು ಮನೆ ಬಿಟ್ಟು ಹೊರಟವನು ಹಿಂತಿರುಗಿ ಬರಲೇ ಇಲ್ಲ.


\section{\num{೧೦೫. } ಭ್ರಮೆ ನಾಶವಾಗುವ ತನಕ}

ಒಬ್ಬ ಗುರು ತನ್ನ ಶಿಷ್ಯನಿಗೆ ಹೇಳಿದ, “ಪ್ರಪಂಚ ಒಂದು ಭ್ರಮೆ, ನನ್ನೊಡನೆ ಬಂದುಬಿಡು” ಎಂದು. “ಆದರೆ ಸ್ವಾಮಿ ನನ್ನ ಮನೆಯವರು, ತಂದೆ, ತಾಯಿ, ಹೆಂಡತಿ ಇವರೆಲ್ಲ ನನ್ನನ್ನು ಅಷ್ಟು ಪ್ರೀತಿಸುತ್ತಾರೆ. ನಾನು ಅವರನ್ನು ಅಗಲುವುದು ಹೇಗೆ?” ಎಂದು ಶಿಷ್ಯನು ಹೇಳಿದ. ಅದಕ್ಕೆ ಗುರು, “ಇದೆಲ್ಲ ನಿನ್ನ ಮನೋಭ್ರಾಂತಿ. ನಾನು ನಿನ ಗೊಂದು ಉಪಾಯವನ್ನು ತೋರು ತ್ತೇನೆ. ಅವರು ನಿನ್ನನ್ನು ನಿಜವಾಗಿ ಪ್ರೀತಿಸುವರೆ ಇಲ್ಲವೆ ಎಂಬುದು ಆಗ ಗೊತ್ತಾಗುವುದು”– ಹೀಗೆ ಹೇಳಿ ಗುರು ಶಿಷ್ಯ ನಿಗೆ ಒಂದು ಮಾತ್ರೆಯನ್ನು ಕೊಟ್ಟು ಹೇಳಿದ:“ಈ ಗುಳಿಗೆಯನ್ನು ಮನೆಯಲ್ಲಿ ನುಂಗು. ನೀನು ಸತ್ತಂತೆ ಕಾಣುವೆ. ಆದರೆ ನಿನಗೆ ಪ್ರಜ್ಞೆ ಇರುವುದು. ನೀನು ಎಲ್ಲವನ್ನೂ ನೋಡುವೆ. ಎಲ್ಲವನ್ನೂ ಕೇಳುವೆ. ಸ್ವಲ್ಪ ಕಾಲದಮೇಲೆ ನಾನು ನಿನ್ನ ಮನೆಗೆ ಬರುವೆ. ಅನಂತರ ನೀನು ಕ್ರಮೇಣ ಪ್ರಜ್ಞೆಯನ್ನು ಪಡೆಯುವೆ.” 

ಶಿಷ್ಯ ಗುರುಗಳು ಹೇಳಿದಂತೆ ಮಾಡಿದ. ತನ್ನ ಹಾಸಿಗೆಯ ಮೇಲೆ ಸತ್ತಂತೆ ಬಿದ್ದ. ಮನೆಯಲ್ಲಿದ್ದವರೆಲ್ಲ ಅಳುವುದಕ್ಕೆ ಶುರು ಮಾಡಿದರು. ಅವನ ತಾಯಿ, ಹೆಂಡತಿ ಮತ್ತು ಇತರರು ಶವದ ಸುತ್ತಲೂ ಗೋಳೋ ಎಂದು ಅಳುತ್ತಿದ್ದರು. ಆ ಸಮಯಕ್ಕೆ ಸರಿಯಾಗಿ ಒಬ್ಬ ಬ್ರಾಹ್ಮಣ ಮನೆಗೆ ಬಂದು, “ಏನು ಸಮಾ ಚಾರ?” ಎಂದು ಕೇಳಿದ. “ಈ ಹುಡುಗ ಸತ್ತುಹೋದ” ಎಂದರು. ಬ್ರಾಹ್ಮಣ ನಾಡಿಯನ್ನು ನೋಡಿ, “ಅದು ಹೇಗೆ? ಅವನು ಸತ್ತಿಲ್ಲ. ನನ್ನ ಹತ್ತಿರ ಒಂದು ಔಷಧಿ ಇದೆ. ಅವನನ್ನು ಬದುಕಿಸಲು ಸಾಧ್ಯ” ಎಂದ. ಇದನ್ನು ಕೇಳಿದ ಮೇಲೆ ಬಂಧುಗಳ ಸಂತೋಷಕ್ಕೆ ಪಾರವೇ ಇರಲಿಲ್ಲ. ತಮ್ಮ ಮನೆಗೆ ದೇವರೇ ಬಂದಂತೆ ಆಯ್ತು ಎಂದು ಭಾವಿಸಿದರು. ಆದರೆ ಬ್ರಾಹ್ಮಣ ಹೇಳಿದ: “ಒಂದು ವಿಷಯ–ಬೇರೆ ಯಾರಾದರೂ ಈ ಔಷಧಿಯನ್ನು ಮೊದಲು ಸೇವಿಸಬೇಕು. ಅನಂತರ ಉಳಿದಿರುವುದನ್ನು ರೋಗಿಗೆ ಕೊಡಬೇಕು. ಯಾರು ಮೊದಲು ತಿನ್ನುವರೋ ಅವರು ಸಾಯುವರು. ಆಗ ಮಾತ್ರ ಸತ್ತವನು ಬದುಕುವನು. ಇಲ್ಲಿ ಅವನ ಅನೇಕ ಬಂಧುಗಳು ಇರುವರು. ಅವರಲ್ಲಿ ಯಾರಾದರೂ ಮುಂದೆ ಬಂದು ಈ ಔಷಧಿಯನ್ನು ತೆಗೆದುಕೊಳ್ಳಬಹುದು. ರೋಗಿಯ ಹೆಂಡತಿ ಮತ್ತು ತಾಯಿ ತುಂಬಾ ಅಳುತ್ತಿರುವರು. ಅವರಲ್ಲಿ ಯಾರಾದರೂ ಇದನ್ನು ತೆಗೆದು ಕೊಳ್ಳುವುದಕ್ಕೆ ಖಂಡಿತ ಅನುಮಾನಿಸಲಿಕ್ಕಿಲ್ಲ” ಎಂದನು.

ತಕ್ಷಣವೆ ಅಳುವೆಲ್ಲ ನಿಂತಿತು. ಎಲ್ಲರೂ ಮೌನವಾಗಿ ಕುಳಿತರು. ತಾಯಿ ಹೇಳಿದಳು: “ಇದು ದೊಡ್ಡ ಸಂಸಾರ. ನಾನು ಸತ್ತರೆ ಅನಂತರ ಯಾರು ಅವರನ್ನು ನೋಡಿಕೊಳ್ಳುತ್ತಾರೆ” ಎಂದು. ತಾಯಿ ತುಂಬಾ ಚಿಂತಾಕ್ರಾಂತಳಾ ದಳು. ಒಂದು ನಿಮಿಷದ ಹಿಂದೆ ಅಳುತ್ತಿದ್ದ ಹೆಂಡತಿ ತನ್ನ ಗ್ರಹಚಾರವನ್ನು ನಿಂದಿಸಿಕೊಳ್ಳುತ್ತ, “ಎಲ್ಲರಂತೆ ಅವರೂ ಹೋದರು. ನೋಡಿ ಕೊಳ್ಳುವುದಕ್ಕೆ ಎರಡು ಮೂರು ಜನ ಮಕ್ಕಳಿದ್ದಾರೆ. ನಾನು ಸತ್ತರೆ ಮಕ್ಕಳನ್ನು ಯಾರು ನೋಡಿಕೊಳ್ಳುವರು” ಎಂದಳು.

ಶಿಷ್ಯ ಎಲ್ಲವನ್ನೂ ಕೇಳಿದ ಮತ್ತು ನೋಡಿದ. ಅವನು ತಕ್ಷಣ ಎದ್ದು ನಿಂತು “ಸ್ವಾಮಿಗಳೆ, ನಾವೀಗ ಹೋಗೋಣ. ನಾನು ನಿಮ್ಮನ್ನು ಅನುಸರಿಸುತ್ತೇನೆ” ಎಂದನು.


\section{\num{೧೦೬. } ಸತ್ತ ಮೇಲೆ ಯಾರೂ ನಿನ್ನನ್ನು ಹಿಂಬಾಲಿಸುವುದಿಲ್ಲ}

ಶಿಷ್ಯನೊಬ್ಬ ಗುರುಗಳಿಗೆ, “ನನ್ನ ಹೆಂಡತಿ ನನ್ನನ್ನು ತುಂಬಾ ಪ್ರೀತಿಸುತ್ತಾಳೆ. ಅದಕ್ಕಾಗಿ ನಾನು ಸಂನ್ಯಾಸಿಯಾಗಲಾರೆ” ಎಂದ. ಶಿಷ್ಯ ಹಠಯೋಗವನ್ನು ಅಭ್ಯಾಸ ಮಾಡುತ್ತಿದ್ದ. ಈ ಪ್ರಪಂಚದ ನಶ್ವರತೆಯನ್ನು ತೋರುವುದಕ್ಕಾಗಿ, ಗುರುಗಳು ಹಠಯೋಗದ ಕೆಲವು ಮುದ್ರೆಗಳನ್ನು ಕಲಿಸಿದರು. ಒಂದು ದಿನ ಇದ್ದಕ್ಕಿದ್ದಂತೆ ಶಿಷ್ಯನ ಮನೆಯಲ್ಲಿ ದೊಡ್ಡ ಗೋಳು ಕೇಳಿಸಿತು. ಅಳು, ತಲೆಯನ್ನು ಬಡಿದುಕೊಳ್ಳುವುದು ಇದೆಲ್ಲ ಶುರುವಾಯಿತು. ಸುತ್ತಮುತ್ತ ಇದ್ದವರು ಇವರ ಮನೆಗೆ ಧಾವಿಸಿದರು. ಹಾಸಿಗೆಯ ಮೇಲೆ ರೋಗಿ ವಕ್ರರೀತಿ ಯಲ್ಲಿ ಬಿದ್ದಿರುವುದನ್ನು ನೋಡಿದರು. ಅವರೆಲ್ಲ ಇವನು ಸತ್ತು ಹೋಗಿರು ವನು ಎಂದು ಭಾವಿಸಿದರು. ಶಿಷ್ಯನ ಹೆಂಡತಿ ಅಳುತ್ತಿದ್ದಳು. “ಅಯ್ಯೋ ನೀವು ಎಲ್ಲಿಗೆ ಹೋದಿರಿ. ಪ್ರಿಯತಮನೆ, ನಮ್ಮನ್ನು ಏತಕ್ಕೆ ಕೈ ಬಿಟ್ಟಿರಿ. ನಮಗೆ ಇಂತಹ ದುರಂತ ಪ್ರಾಪ್ತವಾಗುವುದು ಎಂದು ಊಹಿಸಿಯೆ ಇರಲಿಲ್ಲ,” ಎಂದು ಗೋಳಾಡುತ್ತಿದ್ದಳು. ಆ ಸಮಯದಲ್ಲಿ ಸತ್ತವನನ್ನು ಹೊತ್ತುಕೊಂಡು ಹೋಗಲು ಒಂದು ಮಂಚವನ್ನು ತಂದರು. ಹೆಣವನ್ನು ತೆಗೆದುಕೊಂಡು ಹೋಗುವ ಸಮಯದಲ್ಲಿ ಒಂದು ಸಮಸ್ಯೆ ಉದ್ಭವಿಸಿತು. ಅಷ್ಟವಕ್ರನಂತೆ ಮಲಗಿದ್ದ ಅವನ ದೇಹವನ್ನು ಬಾಗಿಲಿನ ಮೂಲಕ ತೆಗೆದುಕೊಂಡು ಹೋಗಲು ಸಾಧ್ಯವಾಗಲಿಲ್ಲ. ಇದರಿಂದ ಪಾರಾಗಲು ನೆರೆಮನೆಯವರು ಬಾಗಿಲನ್ನು ಒಡೆ ಯಲು ಸಿದ್ಧರಾದರು. ಇಲ್ಲಿಯವರೆಗೆ ದಾರುಣವಾದ ವ್ಯಥೆಯಿಂದ ನರಳು ತ್ತಿದ್ದ ಅವನ ಪತ್ನಿಗೆ ಕೊಡಲಿಯಿಂದ ಬಾಗಿಲನ್ನು ಕೆಡವುತ್ತಿದ್ದ ಸದ್ದುಕೇಳಿಸಿತು. “ನೀವು ಏನು ಮಾಡುತ್ತಿದ್ದೀರಿ” ಎಂದು ಹೆಂಡತಿ ಕೇಳಿದಳು. ನೆರೆಮನೆ ಯವರೊಬ್ಬರು “ಶವ ವಕ್ರವಕ್ರವಾಗಿ ಬಿದ್ದಿರುವುದರಿಂದ ಬಾಗಿಲನ್ನು ಒಡೆಯದೆ ಶವವನ್ನು ತೆಗೆದುಕೊಂಡು ಹೋಗಲು ಅಸಾಧ್ಯ” ಎಂದರು. ಹೆಂಡತಿ: “ಸದ್ಯಕ್ಕೆ ಅದನ್ನು ಮಾಡಬೇಡಿ. ನಾನು ಈಗ ವಿಧವೆ. ಈಗ ಮನೆಯನ್ನು ನೋಡಿಕೊಳ್ಳುವವರುಯಾರೂ ಇಲ್ಲ. ಈಗ ನನ್ನ ಅನಾಥ ಮಕ್ಕಳನ್ನು ಬೆಳೆಸಬೇಕಾಗಿದೆ. ನೀವು ಈಗ ಬಾಗಿಲನ್ನು ಕತ್ತರಿಸಿದರೆ ಅದನ್ನು ಪುನಃ ರಿಪೇರಿಮಾಡುವುದಕ್ಕೆ ಆಗುವುದಿಲ್ಲ. ನನ್ನ ಗಂಡ ಸತ್ತಾಗಿದೆ. ಇನ್ನೇನೂ ಮಾಡುವಂತಿಲ್ಲ. ಶವದ ಕೈಕಾಲುಗಳನ್ನು ಕತ್ತರಿಸಿ ಅದನ್ನು ತೆಗೆದುಕೊಂಡು ಹೋಗಿ” ಎಂದಳು. ಇದನ್ನು ಕೇಳಿದ ತಕ್ಷಣವೆ ಹಠಯೋಗಿ ಎದ್ದುನಿಂತ. ಔಷಧದ ಪರಿಣಾಮ ಅಷ್ಟು ಹೊತ್ತಿಗೆ ನಿಂತುಹೋಯಿತು. “ಎಲೈ ರಾಕ್ಷಸಿ, ನೀನು ನನ್ನ ಕೈಕಾಲುಗಳನ್ನು ಕತ್ತರಿಸುವು ದಕ್ಕೆ ಕಾತರಳಾಗಿರುವೆಯಾ!” ಹೀಗೆ ಹೇಳಿ ಮನೆಯನ್ನು ತೊರೆದು ಗುರುಗಳನ್ನು ಅನುಸರಿಸಿದ.


\section{\num{೧೦೭. } ಇಂದಿನ ಅನುಕರಣ ನಾಳೆಗೆ ಸ್ಫೂರ್ತಿ}

ಒಬ್ಬ ಕಳ್ಳ ನಡುರಾತ್ರಿಯಲ್ಲಿ ರಾಜನ ಅರಮನೆಗೆ ಹೋದ. ರಾಜ ರಾಣಿಗೆ, “ನಾಳೆ ನಾನು ನನ್ನ ಮಗಳನ್ನು ಪವಿತ್ರ ಗಂಗಾನದಿಯ ತೀರದಲ್ಲಿರುವ ಸಾಧುವೊಬ್ಬನಿಗೆ ಕೊಟ್ಟು ವಿವಾಹ ಮಾಡುತ್ತೇನೆ” ಎಂದು ಹೇಳುತ್ತಿರುವುದನ್ನು ಕಳ್ಳ ಕೇಳಿದ. ಕಳ್ಳ ಆಲೋಚನೆ ಮಾಡಿದ: “ನನಗೆ ಅದೃಷ್ಟ ಖುಲಾಯಿಸಿದೆ. ನಾಳೆ ವೇಷ ಮರೆಮಾಚಿಕೊಂಡು ಒಬ್ಬ ಸಾಧುವಿನಂತೆ ಇತರ ಸಾಧುಗಳೊಡನೆ ಕುಳಿತು ಕೊಳ್ಳುತ್ತೇನೆ. ಒಂದು ವೇಳೆ ರಾಜಕುಮಾರಿ ನನ್ನನ್ನು ಮದುವೆ ಮಾಡಿಕೊಂಡರೂ ಮಾಡಿಕೊಳ್ಳಬಹುದು.” ಮಾರನೇ ದಿನ ಅದರಂತೆಯೇ ಮಾಡಿದ. ರಾಜನ ಅಧಿಕಾರಿಗಳು ಸಂನ್ಯಾಸಿಗಳಿಗೆ ರಾಜನ ಮಗಳನ್ನು ಮದುವೆ ಮಾಡಿಕೊಳ್ಳಿ ಎಂದು ಕೇಳಿಕೊಂಡರು. ಯಾರೂ ಅದಕ್ಕೆ ಒಪ್ಪಲಿಲ್ಲ. ಕೊನೆಗೆ ಅವರು ಕಳ್ಳ ಸಾಧುವಿನ ಬಳಿಗೆ ಬಂದರು. ಅವನಿಗೂ ಈ ಸಲಹೆಯನ್ನು ಕೊಟ್ಟರು. ಕಳ್ಳ ಸಾಧು ಮೌನವಾಗಿದ್ದ. ಅಧಿಕಾರಿಗಳು ಹಿಂತಿರುಗಿ ಹೋಗಿ, “ಒಬ್ಬ ಯುವಕ ಸಂನ್ಯಾಸಿ ಇದ್ದಾನೆ. ಬಹುಶಃ ನಾವು ಪ್ರಯತ್ನ ಮಾಡಿದರೆ ಅವನು ಮದುವೆ ಮಾಡಿಕೊಳ್ಳಬಹುದು” ಎಂದರು. ಅನಂತರ ರಾಜನೇ ಈ ಸಾಧುವಿನ ಬಳಿಗೆ ಬಂದು ತನ್ನ ಮಗಳನ್ನು ಮದುವೆ ಮಾಡಿಕೊಳ್ಳಿ ಎಂದು ಕೇಳಿಕೊಂಡ. ಆದರೆ ರಾಜನೇ ಬಂದ ಮೇಲೆ ಆ ಕಳ್ಳ ಸಂನ್ಯಾಸಿಯ ಮನಸ್ಸು ಬದಲಾಯಿತು. “ನಾನು ಕಳ್ಳ ಸಾಧುವಿ ನಂತೆ ಬಂದೆ. ಇಷ್ಟಕ್ಕೇ ರಾಜರು ಮತ್ತು ಅನುಯಾಯಿಗಳೆಲ್ಲ ರಾಜಕುಮಾರಿಯನ್ನು ಮದುವೆ ಮಾಡಿಕೊಳ್ಳಿ ಎಂದು ಕೇಳಿಕೊಳ್ಳುತ್ತಿರುವರು. ನಾನು ನಿಜವಾದ ಸಂನ್ಯಾಸಿಯಾದರೆ ನನಗೆ ಯಾವ ಭಾಗ್ಯ ಬರುವುದೊ” ಎಂದು ಯೋಚಿಸಿದ. ಈ ಭಾವನೆಗಳು ಅವನ ಮನಸ್ಸನ್ನು ಕಲಕಿದವು. ಅವಳನ್ನು ಮದುವೆಯಾಗುವ ಬದಲು ತನ್ನ ಜೀವನವನ್ನು ತಿದ್ದಿಕೊಂಡು ಅವನು ಯೋಗ್ಯ ಸಾಧುವಾದ. ಅವನು ಮದುವೆ ಮಾಡಿಕೊಳ್ಳಲಿಲ್ಲ. ಅನಂತರ ಪ್ರಸಿದ್ಧ ಸಾಧುವಾದ. ಕೆಲವು ವೇಳೆ ಒಳ್ಳೆಯದನ್ನು ಅನುಕರಿಸಿದರೆ ಒಳ್ಳೆಯ ಫಲಗಳು ಸಿಕ್ಕುವುವು.


\section{\num{೧೦೮. } ಸಾಧುವನ್ನು ಅನುಸರಿಸಿದರೆ, ಅವನಂತೆ ಆಗುವ ಸಾಧ್ಯತೆಗಳಿವೆ}

ಒಂದು ರಾತ್ರಿ ಬೆಸ್ತರವನೊಬ್ಬ ಮೀನನ್ನು ಕದಿಯುವುದಕ್ಕೆ ಒಂದು ಕೊಳ ದಲ್ಲಿ ಬಲೆ ಹಾಕಿದ. ತೋಟದ ಯಜಮಾನನಿಗೆ ಇದು ಗೊತ್ತಾಗಿ ಇನ್ನೂ ಕೆಲವ ರೊಂದಿಗೆ ಅವನನ್ನು ಸುತ್ತುಗಟ್ಟಿದನು. ಉರಿಯುತ್ತಿರುವ ಪಂಜಿನ ಬೆಳಕಿನಿಂದ ಅವನನ್ನು ಹುಡುಕುತ್ತಿದ್ದರು. ಅಷ್ಟರಲ್ಲಿ ಆ ಬೆಸ್ತರವನು ತನ್ನ ಮೈಗೆಲ್ಲ ವಿಭೂತಿಯನ್ನು ಬಳಿದು ಒಂದು ಮರದಡಿ ಕುಳಿತುಕೊಂಡು ಒಬ್ಬ ಸಾಧುವಿ ನಂತೆ ನಟಿಸಿದನು. ತೋಟದ ಮಾಲಿಕನ ಕಡೆಯವರು ಬೇಕಾದಷ್ಟು ಹುಡುಕಾಡಿ ದರು. ಆದರೆ ಕಳ್ಳ ಸಿಕ್ಕಲಿಲ್ಲ. ಒಂದು ಮರದ ಕೆಳಗೆ ಮೈಗೆಲ್ಲ ವಿಭೂತಿಯನ್ನು ಬಳಿದುಕೊಂಡಿದ್ದ ಸಾಧುವನ್ನು ಮಾತ್ರ ಕಂಡರು. ಮಾರನೆಯ ದಿನದಿಂದ ಒಬ್ಬ ದೊಡ್ಡ ಸಾಧು ಅಲ್ಲಿರುವನು ಎಂಬ ವದಂತಿ ಹಬ್ಬಿತು. ಅನೇಕ ಜನ ಭಕ್ತರು ಅವನಿಗೆ ನಮಸ್ಕರಿಸಿ ಹೂವು ಹಣ್ಣು ಮಿಠಾಯಿ ಮುಂತಾದವನ್ನು ತಂದು ಕೊಟ್ಟರು. ಕೆಲವರು ಬೆಳ್ಳಿ, ತಾಮ್ರದ ಕಾಸುಗಳನ್ನೂ ಕೊಟ್ಟರು. “ಇದೊಂದು ವಿಚಿತ್ರ. ನಾನು ನಿಜವಾದ ಸಾಧುವಲ್ಲ. ಆದರೂ ಜನರು ಇಷ್ಟು ಗೌರವವನ್ನು ತೋರುತ್ತಿರುವರಲ್ಲ. ನಾನು ನಿಜವಾದ ಸಾಧುವಾದರೆ ಭಗವಂತನ ಸಾಕ್ಷಾ ತ್ಕಾರವನ್ನು ಪಡೆಯಬಹುದು” ಎಂದು ಅವನು ಭಾವಿಸಿದನು.


\section{\num{೧೦೯. } ತ್ಯಾಗದಲ್ಲಿ ಸಮದೃಷ್ಟಿಯೇ ಆದಿ ಮತ್ತು ಅಂತ್ಯ}

ಗಂಡ ಹೆಂಡತಿಯರಿಬ್ಬರು ಸಂಸಾರ ತೊರೆದು ತೀರ್ಥ\\ಯಾತ್ರೆಗೆ ಹೊರಟರು. ಒಂದು ಸಲ ಅವರು ರಸ್ತೆಯಲ್ಲಿ ಹೋಗು\\ತ್ತಿರುವಾಗ, ಹೆಂಡತಿಗಿಂತ ಮುಂದೆ ಹೋಗುತ್ತಿದ್ದ ಗಂಡ, ದಾರಿಯಲ್ಲಿ\\ಒಂದು ವಜ್ರವನ್ನು ನೋಡಿದನು. ತಕ್ಷಣವೇ ಮಣ್ಣಿನಿಂದ ವಜ್ರವನ್ನು ಮುಚ್ಚ ಲಾರಂಭಿಸಿದ, ಏನಾದರೂ ಅದನ್ನು ಹೆಂಡತಿ ನೋಡಿದರೆ ಅವಳಲ್ಲಿ ಪ್ರಲೋ ಭನೆ ಏಳಬಹುದು ಎಂದು. ಹಾಗೆ ಪ್ರಲೋಭನೆ ಎದ್ದರೆ ಅವಳು ತೀರ್ಥ ಯಾತ್ರೆಯ ಪುಣ್ಯವನ್ನು ಕಳೆದುಕೊಳ್ಳುವಳು. ಗಂಡ ನೆಲವನ್ನು ಕೆರೆಯು ತ್ತಿದ್ದಾಗ ಹೆಂಡತಿ ಬಂದು “ನೀವು ಏನು ಮಾಡುತ್ತಿರುವಿರಿ?” ಎಂದು ಕೇಳಿದಳು. ಏನೋ ಅತೃಪ್ತಿಕರವಾದ ವಿವರಣೆಯನ್ನು ಅವನು ಕೊಟ್ಟನು, ಅವಳಿಗೆ ಗೊತ್ತಾಗದಿರಲಿ ಎಂದು. ಹೆಂಡತಿ ವಜ್ರವನ್ನು ಕಂಡು ಗಂಡನ ಮನಸ್ಸಿನಲ್ಲಿದ್ದ ಆಲೋಚನೆಯನ್ನು ಗ್ರಹಿಸಿ, “ವಜ್ರಕ್ಕೂ ಕಸಕ್ಕೂ ವ್ಯತ್ಯಾಸವನ್ನು ನೀವು ಕಾಣುತ್ತಿರುವುದರಿಂದ ಸಂಸಾರವನ್ನು ಏಕೆ ತ್ಯಾಗ ಮಾಡಿದಿರಿ?” ಎಂದು ಕೇಳಿದಳು.


\section{\num{೧೧೦. } ಸಂನ್ಯಾಸಿಯ ವಿಧಿ ನಿಯಮಗಳು ಅತ್ಯಂತ ಕಠಿಣ}

ಸಂನ್ಯಾಸಿಯ ವಿಧಿ ನಿಯಮಗಳು ಬಹಳ ಕಷ್ಟಕರವಾದವುಗಳು. ಅವನು ಕಾಮಿನಿ ಕಾಂಚನದೊಂದಿಗೆ ಯಾವ ಸಂಬಂಧವನ್ನೂ ಇಟ್ಟುಕೊಳ್ಳಕೂಡದು. ಅವನು ತನ್ನ ಕೈಯಿಂದ ಯಾವ ಬಹುಮಾನವನ್ನೂ ಸ್ವೀಕರಿಸಬಾರದು. ಅವು ಅವನ ಹತ್ತಿರವೂ ಇರಕೂಡದು.

ಲಕ್ಷ್ಮೀನಾರಾಯಣ ಮಾರವಾಡಿ ಒಬ್ಬ ವೇದಾಂತಿ. ಅವನು ದಕ್ಷಿಣೇಶ್ವರಕ್ಕೆ ಕೆಲವು ವೇಳೆ ಬರುತ್ತಿದ್ದ. ಒಂದು ದಿನ ಅವನು ನನ್ನ ಹಾಸಿಗೆಯ ಮೇಲೆ ಹಾಸಿರುವ ಬಟ್ಟೆ ಕೊಳಕಾಗಿದ್ದುದನ್ನು ನೋಡಿ, “ನಾನು ನಿಮ್ಮ ಹೆಸರಿನಲ್ಲಿ ಹತ್ತು ಸಾವಿರ ರೂಪಾಯಿಗಳನ್ನು ಇಡುವೆನು. ಅದರಿಂದ ಬರುವ ಬಡ್ಡಿ ನಿಮ್ಮ ಖರ್ಚಿಗೆ ಆಗುವುದು” ಎಂದ. ಅವನು ಇದನ್ನು ಹೇಳಿದೊಡನೆ, ಯಾರೊ ನನ್ನನ್ನು ಹೊಡೆದಂತೆ ಆಗಿ ಪ್ರಜ್ಞೆ ತಪ್ಪಿ ಬಿದ್ದೆ. ನನಗೆ ಪ್ರಜ್ಞೆ ಬಂದಾದ ಮೇಲೆ “ನೀನು ಅಂತಹ ಮಾತನ್ನು ಆಡುವುದಾದರೆ ಇನ್ನೊಂದು ಸಲ ಇಲ್ಲಿಗೆ ಬರ ಬೇಡ. ನಾನು ಹಣವನ್ನು ಮುಟ್ಟಲಾರೆ,” ಎಂದು ಗದರಿಸಿದೆ. ಅವನು ಬಹಳ ಬುದ್ಧಿವಂತ. ಅವನು ಹೇಳಿದ: “ನಿಮ್ಮಲ್ಲಿ ಇನ್ನೂ ಸ್ವೀಕಾರ ತಿರಸ್ಕಾರ ಭಾವನೆಗಳಿವೆಯೇನು? ಹಾಗಾದರೆ ನೀವು ಇನ್ನೂ ಪೂರ್ಣತೆಯನ್ನು ಮುಟ್ಟಿಲ್ಲ.” “ಮಹಾಶಯರೆ, ನಾನು ಅಷ್ಟು ದೂರ ಇನ್ನೂ ಹೋಗಿಲ್ಲ,” ಎಂದೆನು. ಆಗ ಲಕ್ಷ್ಮೀನಾರಾಯಣ “ಹಣ ವನ್ನು ಹೃದಯನ ಕೈಯಲ್ಲಿ ಕೊಡುವೆ” ಎಂದ. ನಾನು ಹೇಳಿದೆ “ಅದನ್ನು ಮಾಡ ಕೂಡದು. ನೀನು ಹೃದಯನಿಗೆ ಹಣವನ್ನು ಕೊಟ್ಟರೆ, ಆಗ ನನಗೆ ತೋರಿದ ರೀತಿಯಲ್ಲಿ ಹಣವನ್ನು ಖರ್ಚು ಮಾಡು ಎಂದು ಅವನಿಗೆ ಹೇಳಬೇಕಾಗುವುದು. ನಾನು ಹೇಳಿದಂತೆ ಅವನು ಕೇಳದೆ ಇದ್ದರೆ ನನಗೆ ಕೋಪ ಬರ ಬಹುದು. ಹಣದ ಸಹವಾಸವೇ ಕೆಟ್ಟದ್ದು. ನೀವು ಹೃದಯನಿಗೂ ಅದನ್ನು ಕೊಡಬಾರದು. ಕನ್ನಡಿ ಎದುರಿಗೆ ಇಟ್ಟ ವಸ್ತುವಿನ ಪ್ರತಿಬಿಂಬ ಕನ್ನಡಿಯಲ್ಲಿ ಕಾಣದೆ ಇರುತ್ತದೆಯೇ?”



\section{\num{೧೧೧. } ಶಿವನ ವೇಷದಲ್ಲಿ ಬಹುರೂಪಿ}

ಒಬ್ಬ ಬಹುರೂಪಿ (ಹಲವು ವೇಷಗಳನ್ನು ಹಾಕಿಕೊಳ್ಳುವವನು) ಶಿವನಂತೆ ವೇಷ ಹಾಕಿಕೊಂಡು ಒಂದು ಮನೆಗೆ ಬಂದ. ಮನೆಯ ಯಜಮಾನ ಅವನಿಗೆ ಒಂದು ರೂಪಾಯಿ ಬಹುಮಾನವನ್ನು ಕೊಡಬೇಕೆಂದಿದ್ದ. ಆದರೆ ಬಹುರೂಪಿ ಅದನ್ನು ಸ್ವೀಕರಿಸಲಿಲ್ಲ. ಆಗ ಸಾಧು ತನ್ನ ಮನೆಗೆ ಹೋಗಿ ವೇಷವನ್ನು ಬದಲಿಸಿಕೊಂಡು ಬಂದು ಆ ಮನೆಯೊಡೆಯ ಕೊಡಬೇಕೆಂದಿದ್ದ ಒಂದು ರೂಪಾಯಿಯನ್ನು ಕೇಳಿದ. “ನೀನು ಮುಂಚೆ ಏತಕ್ಕೆ ಅದನ್ನು ತೆಗೆದುಕೊಳ್ಳ ಲಿಲ್ಲ” ಎಂದು ಕೇಳಿದ ಮನೆಯ ಯಜಮಾನ. ಆಗ ಬಹುರೂಪಿ, “ಆ ಸಮಯದಲ್ಲಿ ನಾನು ಶಿವನ ರೂಪವನ್ನು ಧರಿಸಿದ್ದೆ. ಆಗಹಣವನ್ನು ಮುಟ್ಟುವಂತೆ ಇರಲಿಲ್ಲ” ಎಂದನು.


\section{\num{೧೧೨. } ನೀನು ಗುದ್ದಲಿಯನ್ನು ಬಿಡಬೇಡ; ಭದ್ರವಾಗಿ ಹಿಡಿ}

ಒಂದು ಸಲ ಒಂದೂರಿನಲ್ಲಿ ಬರಗಾಲ ಬಂದಿತು. ರೈತರು ದೂರದಿಂದ ನೀರಿಗಾಗಿ ಕಾಲುವೆಯನ್ನು ತೋಡತೊಡಗಿದರು. ಒಬ್ಬ ಶಪಥ ಮಾಡಿದ, “ನದಿಯಿಂದ ಗದ್ದೆಗೆ ನೀರು ಬರುವ ತನಕ ಕೆಲಸವನ್ನು ಬಿಡುವುದಿಲ್ಲ” ಎಂದು. ಕೆಲಸ ಶುರು ಮಾಡಿದ. ಸ್ನಾನ ಮಾಡುವ ಸಮಯ ಬಂತು. ಅವನ ಹೆಂಡತಿ ಮಗಳ ಮೂಲಕ ಸ್ನಾನ ಮಾಡುವುದಕ್ಕೆ ಎಣ್ಣೆಯನ್ನು ಕಳುಹಿಸಿದಳು. ಮಗಳು ತಂದೆಗೆ, “ಅಪ್ಪ, ಆಗಲೇ ವೇಳೆಯಾಗಿದೆ. ಮೈಗೆ ಎಣ್ಣೆಯನ್ನು ಸವರಿ ಸ್ನಾನಕ್ಕೆ ಅಣಿಯಾಗು” ಎಂದಳು. “ಹೋಗಾಚೆ, ನನಗೆ ಮಾಡುವುದಕ್ಕೆ ಬಹಳ ಕೆಲಸ ವಿದೆ” ಎಂದು ಅವನು ಗರ್ಜಿಸಿದನು. ಮಧ್ಯಾಹ್ನವಾಯಿತು. ಆದರೂ ರೈತನು ಕೆಲಸವನ್ನು ಮಾಡುತ್ತಲೇ ಇದ್ದನು. ಅವನು ತನ್ನ ಸ್ನಾನದ ವಿಷಯವನ್ನು ಕುರಿತು ಯೋಚಿಸಲೂ ಇಲ್ಲ. ಅನಂತರ ಅವನ ಹೆಂಡತಿ ಬಂದಳು. “ಏತಕ್ಕೆ ನೀವು ಇನ್ನೂ ಸ್ನಾನ ಮಾಡಿಲ್ಲ. ಊಟ ಆರಿಹೋಗುತ್ತಿದೆ. ನಿಮ್ಮದು ಯಾವಾ ಗಲೂ ಒಂದು ಅತಿ. ನೀವು ಉಳಿದ ಕೆಲಸವನ್ನು ನಾಳೆ ಪೂರೈಸಬಹುದು, ಇಲ್ಲವೇ ಊಟವಾದ ಮೇಲೆ ಮಾಡಬಹುದು” ಎಂದಳು. ರೈತನು ಅವಳನ್ನು ಗದರಿಸುತ್ತ ಗುದ್ದಲಿಯನ್ನು ತೋರಿಸಿ, “ಏನು ನಿನಗೆ ಪರಿಜ್ಞಾನವಿಲ್ಲವೆ? ಮಳೆಯಿಲ್ಲ, ಬೆಳೆಯೆಲ್ಲ ಒಣಗಿಹೋಗುತ್ತಿದೆ. ಮಕ್ಕಳಿಗೆ ಊಟಕ್ಕೆ ಏನು ಗತಿ? ನೀವೆಲ್ಲ ಉಪವಾಸದಿಂದ ನರಳಬೇಕಾಗುವುದು. ನನ್ನ ಗದ್ದೆಗೆ ನೀರು ಬರುವ ತನಕ ನಾನು ಊಟವನ್ನೂ ಮಾಡುವುದಿಲ್ಲ, ಸ್ನಾನವನ್ನೂ ಮಾಡುವುದಿಲ್ಲ” ಎಂದು ಕೋಪದಿಂದ ಹೇಳಿದನು. ಗಂಡನ ರುದ್ರಾವತಾರವನ್ನು ನೋಡಿ ಅವಳು ಅಂಜಿಕೊಂಡು ಹೊರಟುಹೋದಳು. ಇಡೀ ದಿನವೆಲ್ಲ ಮೈಮುರಿಯು ವಂತೆ ಕೆಲಸ ಮಾಡಿ, ಸಂಜೆ ಹೊತ್ತಿಗೆ ನದಿಯಿಂದ ಗದ್ದೆಗೆ ನೀರು ಗಳಗಳ ಹರಿಯುವುದನ್ನು ನೋಡಿದ. ಆಗ ಅವನಿಗೆ ಶಾಂತಿ ಮತ್ತು ಸುಖವಾಯಿತು. ಅವನು ಮನೆಗೆ ಹೋಗಿ ಹೆಂಡತಿಯನ್ನು ಕರೆದು “ಈಗ ಸ್ನಾನಕ್ಕೆ ಎಣ್ಣೆ, ಸೇದುವುದಕ್ಕೆ ತಂಬಾಕನ್ನು ಅಣಿ ಮಾಡು” ಎಂದ. ಶಾಂತಿಯಿಂದ ಸ್ನಾನ ಮಾಡಿ ಊಟ ಮಾಡಿ ಮನದಣಿಯೆ ಗೊರಕೆ ಹೊಡೆಯುತ್ತ ನಿದ್ರೆ ಮಾಡಿದನು. ಅವನು ತೋರಿದ ದೃಢ ನಿರ್ಧಾರ ಉತ್ಕೃಷ್ಟ ವಾದ ತ್ಯಾಗಕ್ಕೆ ನಿದರ್ಶನ.

ಮತ್ತೊಬ್ಬ ರೈತನೂ ತನ್ನ ಗದ್ದೆಗೆ ನೀರನ್ನು ತರಲು ಕಾಲುವೆಯನ್ನು ತೋಡು ತ್ತಿದ್ದನು. ಅವನ ಹೆಂಡತಿಯೂ ಗದ್ದೆಗೆ ಬಂದು ಹೇಳಿದಳು: “ಈಗ ಆಗಲೇ ಬಹಳ ಹೊತ್ತಾಗಿದೆ. ಮನೆಗೆ ಹೋಗೋಣ. ಅತಿಯಾಗಿ ಕೆಲಸ ಮಾಡುವುದು ಒಳ್ಳೆಯದಲ್ಲ,” ಎಂದು. ರೈತ ವಿರೋಧಿಸದೆ ಗುದ್ದಲಿಯನ್ನು ಬದಿಗಿರಿಸಿ “ನೀನು ಹೇಳುತ್ತಿರುವುದರಿಂದ ನಾನು ಮನೆಗೆ ಬರುತ್ತೇನೆ” ಎಂದ. ಅವನು ತನ್ನ ಗದ್ದೆಗೆ ನದಿಯ ನೀರನ್ನು ತರಲಾಗಲಿಲ್ಲ. ಇದೇನೆ ಮಂದ ವೈರಾಗ್ಯದ ಚಿಹ್ನೆ.


\section{\num{೧೧೩. } ನೀನು ಮುಂದೆ ಹೋದಂತೆ ಗುರಿ ಸಮೀಪಿಸುವುದು}

ಒಬ್ಬ ಮುಸಲ್ಮಾನನು ದೇವರನ್ನು ಕುರಿತು ‘ಅಲ್ಲಾ ಅಲ್ಲಾ’ ಎಂದು ಜೋರಾಗಿ ಕಿರಿಚುತ್ತಿದ್ದ. ಮತ್ತೊಬ್ಬ “ನೀನು ಅಲ್ಲಾನನ್ನು ಕುರಿತು ಪ್ರಾರ್ಥಿಸು ತ್ತಿದ್ದೀಯ. ಅದೇನೊ ಒಳ್ಳೆಯದೆ. ಆದರೆ ನೀನು ಅಷ್ಟೊಂದು ಏತಕ್ಕೆ ಕಿರಿಚಿ ಕೊಳ್ಳುತ್ತಿರುವೆ. ಅವನು ಇರುವೆ ಕಾಲಿನ ಸಪ್ಪಳವನ್ನು ಕೂಡ ಕೇಳಬಲ್ಲ ಎಂಬುದು ನಿನಗೆ ಗೊತ್ತಿಲ್ಲವೆ?” ಎಂದನು.

ಮನಸ್ಸು ದೇವರೊಂದಿಗೆ ಒಂದಾದಾಗ ದೇವರು ನಮ್ಮ ಹೃದಯಾಕಾಶ ದಲ್ಲೆ ಇರುವನು ಎಂಬುದನ್ನು ಅರಿಯುವನು. ಆದರೆ ನೀನು ಒಂದು ವಿಷಯ ವನ್ನು ನೆನಪಿನಲ್ಲಿಡಬೇಕು. ಈ ಐಕ್ಯತೆಯ ಭಾವ ದೃಢವಾದಂತೆಲ್ಲ, ನಿನ್ನ ಮನಸ್ಸು ವಿಷಯವಸ್ತುಗಳಿಂದ ದೂರವಾಗುತ್ತಾ ಅಂತರ್ಮುಖವಾಗುವುದು. ಭಕ್ತಮಾಲೆಯಲ್ಲಿ ಬಿಲ್ವಮಂಗಳನ ಕಥೆ ಇದೆ. ಅವನು ವೇಶ್ಯೆಯ ಮನೆಗೆ ಹೋಗುತ್ತಿದ್ದ. ಒಂದು ದಿನ ಅವಳ ಮನೆಗೆ ಬಹಳ ಹೊತ್ತಾಗಿ ಹೋದ. ಅಂದು ಅವನ ಮನೆಯಲ್ಲಿ ಅವನ ತಂದೆತಾಯಿಯರ ಶ್ರಾದ್ಧವಿದ್ದದ್ದರಿಂದ ತಡವಾ ಗಿತ್ತು. ಅವನು ತನ್ನ ಕೈಯಲ್ಲಿ ವೇಶ್ಯೆಗಾಗಿ ಶ್ರಾದ್ಧಾನ್ನವನ್ನು ತೆಗೆದುಕೊಂಡು ಹೋಗುತ್ತಿದ್ದ. ಅವನ ಮನಸ್ಸೆಲ್ಲ ಅವಳ ಮೇಲೆ ಇದ್ದುದರಿಂದ ಅವನಿಗೆ ತನ್ನ ಚಲನೆಯ ಅರಿವೇ ಇರಲಿಲ್ಲ, ತಾನು ಹೇಗೆ ನಡೆಯುತ್ತಿರುವೆ ನೆಂಬುದೇ ಗೊತ್ತಿರಲಿಲ್ಲ. ದಾರಿಯಲ್ಲಿ ಒಬ್ಬ ಯೋಗಿ ಧ್ಯಾನ ಮಾಡುತ್ತಿದ್ದ. ಬಿಲ್ವಮಂಗಲನ ಕಾಲು ಅವನಿಗೆ ತಾಗಿತು. ಯೋಗಿಗೆ ತುಂಬಾ ಕೋಪ ಬಂದು, “ನಿನಗೇನು ಕಣ್ಣಿಲ್ಲವೆ? ನಾನು ದೇವರನ್ನು ಚಿಂತಿಸುತ್ತಿರುವಾಗ ನೀನು ಕಾಲಿನಿಂದ ನನ್ನನ್ನು ಒದ್ದುಬಿಟ್ಟೆಯಲ್ಲ” ಎಂದ. ಬಿಲ್ವಮಂಗಲ ಸಾಧುವಿಗೆ “ನಿಮ್ಮನ್ನು ಒಂದು ಮಾತನ್ನು ಕೇಳಲೆ? ನಾನು ವೇಶ್ಯೆಯನ್ನು ಕುರಿತು ಚಿಂತಿಸುತ್ತ ಎಲ್ಲವನ್ನೂ ಮರೆತಿರುವೆನು, ಆದರೆ ನಿಮಗೆ ಭಗವಂತನನ್ನು ಕುರಿತು ಚಿಂತಿಸುವಾಗಲೂ ಹೊರಜಗತ್ತಿನ ಅರಿವಿದೆ. ನಿಮ್ಮ ಧ್ಯಾನ ಎಂಥದು?” ಎಂದ. ಕೊನೆಯಲ್ಲಿ ಬಿಲ್ವಮಂಗಲನು ಸಂಸಾರ ವನ್ನು ತ್ಯಜಿಸಿ ಭಗವದಾರಾಧನೆಯಲ್ಲಿ ಮಗ್ನನಾದನು. ಅವನು ವೇಶ್ಯೆಗೆ ಹೇಳಿದ: “ನೀನೆ ನನ್ನ ಗುರು. ದೇವರಿಗಾಗಿ ಹೇಗೆ ವ್ಯಾಕುಲನಾಗಬೇಕು ಎನ್ನುವುದನ್ನು ತೋರಿದೆ.” ಬಿಲ್ವಮಂಗಲ ವೇಶ್ಯೆಯನ್ನು ತನ್ನ ತಾಯಿ ಎಂದು ಕರೆದು ಅವ ಳೊಡನೆ ಸಂಪರ್ಕವನ್ನು ಕಡಿದುಹಾಕಿದನು.


\section{\num{೧೧೪. } ರಾಜ ಮತ್ತು ಪಂಡಿತರು}

ಒಬ್ಬ ರಾಜನು ಪ್ರತಿದಿನವೂ ಪಂಡಿತನೊಬ್ಬನಿಂದ ಭಾಗವತ ಶ್ರವಣ ಮಾಡುತ್ತಿದ್ದ. ಪ್ರತಿದಿನ ಭಾಗವತವನ್ನು ಹೇಳಿದ ಮೇಲೆ ಪಂಡಿತನು ರಾಜರನ್ನು ಕುರಿತು “ನಿಮಗೆ ಚೆನ್ನಾಗಿ ಗೊತ್ತಾಯಿತೆ” ಎಂದು ಕೇಳುತ್ತಿದ್ದನು. ಪ್ರತಿದಿನವೂ ರಾಜನು, “ನೀವು ಅದನ್ನು ಮೊದಲು ಚೆನ್ನಾಗಿ ಅರ್ಥಮಾಡಿಕೊಳ್ಳಿ” ಎಂದು ಹೇಳುತ್ತಿದ್ದನು. ಪಂಡಿತ ಮನೆಗೆ ಬಂದು “ಪ್ರತಿದಿನ ರಾಜರು ಹೀಗೆ ಏತಕ್ಕೆ ಹೇಳುತ್ತಾರೆ. ನಾನು ಭಾಗವತವನ್ನು ಅಷ್ಟು ಸ್ಪಷ್ಟವಾಗಿ ಹೇಳುತ್ತೇನೆ, ಆದರೂ ಅವರು ನೀವೇ ಮೊದಲು ಅದನ್ನು ಚೆನ್ನಾಗಿ ತಿಳಿದುಕೊಳ್ಳಿ ಎಂದು ಹೇಳು ತ್ತಾರೆ. ಇದರ ಅರ್ಥವೇನು?” ಎಂದು ಯೋಚಿಸುತ್ತಿದ್ದನು. ಪಂಡಿತರು ಸಾಧನೆ ಯನ್ನು ಮಾಡುತ್ತಿದ್ದರು. ಕೆಲವು ಕಾಲದ ಮೇಲೆ ದೇವರೊಬ್ಬನೆ ಸತ್ಯ ಉಳಿದು ದೆಲ್ಲ ಮಿಥ್ಯವೆಂದು ಅರಿತರು. ಮನೆ, ಹೆಂಡತಿ, ಮಕ್ಕಳು, ಐಶ್ವರ್ಯ, ಸ್ನೇಹಿತರು, ಹೆಸರು, ಕೀರ್ತಿ ಎಲ್ಲ ಜೊಳ್ಳು ಎಂಬುದನ್ನು ಅರಿತರು. ಈ ಪ್ರಪಂಚದ ಮಿಥ್ಯೆ ಅರಿವಾಗಿ ಸಂಸಾರವನ್ನು ತ್ಯಜಿಸಿದರು. ಹೋಗುವಾಗ ದೊರೆಗಳಿಗೆ ಈ ಸಮಾ ಚಾರವನ್ನು ಕೊಟ್ಟರು: “ರಾಜರೆ, ನನಗೆ ಈಗ ಅದು ಗೊತ್ತಾಗಿದೆ.”


\section{\num{೧೧೫. } ನೀನು ಪ್ರಪಂಚವನ್ನು ಬಿಡಬೇಕೆಂದರೂ ಬಿಡಲಾರೆ}

ತಾನು ಇಚ್ಛೆಪಟ್ಟರೂ ಒಬ್ಬ ಸಂಸಾರವನ್ನು ತ್ಯಜಿಸಲಾರ. ಏಕೆಂದರೆ ಪ್ರಾರಬ್ಧಕರ್ಮಗಳು ಮತ್ತು ಸಂಸ್ಕಾರಗಳು ಅಡಚಣೆ ಯಾಗಿ ಪರಿಣಮಿಸುವುವು. ಒಬ್ಬ ಯೋಗಿ ರಾಜನಿಗೆ “ಇಲ್ಲಿ ಕುಳಿತುಕೊಂಡು ಧ್ಯಾನಿಸಿ” ಎಂದನು. ಅದಕ್ಕೆ ರಾಜ ಹೇಳಿದ: “ನಾನು ಹಾಗೆ ಮಾಡಲಾರೆ. ನಿಮ್ಮ ಬಳಿ ಇದ್ದರೂ ಪ್ರಪಂಚವನ್ನು ಅನುಭವಿಸಬೇಕೆಂಬ ಆಸೆ ನನ್ನಲ್ಲಿರುವುದು. ನಾನು ಕಾಡಿಗೆ ಹೋದರೂ, ಅಲ್ಲಿಯೇ ಒಂದು ರಾಜ್ಯ ನಿರ್ಮಾಣವಾಗುತ್ತದೆ. ಏಕೆಂದರೆ ಇನ್ನೂ ನನ್ನಲ್ಲಿ ಭೋಗೇಚ್ಛೆ ಇದೆ.”


\section{\num{೧೧೬. } ತ್ಯಾಗವೇ ಜೀವನದ ಉಸಿರಾದಾಗ}

ಕಾಮಿನಿ ಕಾಂಚನ ಬಿಡದೆ ಆಧ್ಯಾತ್ಮಿಕ ಉನ್ನತಿ ಸಾಧ್ಯವಿಲ್ಲ. ನಾನು ಹೆಣ್ಣು, ಹೊನ್ನು ಮತ್ತು ಮಣ್ಣು ಮೂರನ್ನೂ ಬಿಟ್ಟೆ. ನಾನು ಒಂದು ಭೂಮಿಯನ್ನು ರಿಜಿಸ್ಟರ್ ಮಾಡಿಸಲು ರಿಜಿಸ್ಟ್ರಿ ಆಫೀಸಿಗೆ ಹೋಗಿದ್ದೆ. ಅದು ರಘುವೀರನ ಹೆಸರಿನಲ್ಲಿತ್ತು. ಆಫೀಸರು ನನಗೆ, ನಿಮ್ಮ ಸಹಿಯನ್ನು ಮಾಡಿ ಎಂದರು. ಆದರೆ ನಾನು ಮಾಡಲಿಲ್ಲ, ಏಕೆಂದರೆ ಆ ಭೂಮಿ ನನ್ನದು ಎಂದು ಎನ್ನಿಸಲಿಲ್ಲ. ನಾನು ಕೇಶವಸೇನನ ಗುರು ಎಂದು ನನ್ನನ್ನು ಬಹಳ ಗೌರವದಿಂದ ಆದರಿಸಿ ಅವರು ನನಗೆ ಮಾವಿನ ಹಣ್ಣನ್ನು ಕೊಟ್ಟರು. ಆದರೆ ಅದನ್ನು ನಾನು ಮನೆಗೆ ತೆಗೆದುಕೊಂಡು ಹೋಗಲು ಆಗಲಿಲ್ಲ. ಸಂನ್ಯಾಸಿ ತನ್ನ ಹತ್ತಿರ ಏನೂ ಇಟ್ಟುಕೊಳ್ಳಬಾರದು.

ತ್ಯಾಗ ಇಲ್ಲದೆ ಇದ್ದರೆ ಹೇಗೆ ಭಗವಂತನ ಸಾಕ್ಷಾತ್ಕಾರವನ್ನು ಪಡೆಯ ಬಹುದು? ಒಂದು ವಸ್ತುವಿನ ಮೇಲೆ ಮತ್ತೊಂದನ್ನು ಇಟ್ಟಿದ್ದರೆ, ಮೊದಲ ನೆಯದನ್ನು ತೆಗೆಯದೆ ಎರಡನೆಯದು ಹೇಗೆ ಸಿಗುವುದು?


\section{\num{೧೧೭. } ಗೆಳೆಯನನ್ನು ಆಶಿಸುತ್ತಿದ್ದ ಭೂತ}

ಒಂದು ಭೂತವು ಒಬ್ಬ ಗೆಳೆಯನನ್ನು ಆಶಿಸುತ್ತಿತ್ತು. ಶನಿವಾರ, ಮಂಗಳ ವಾರ ಯಾರಾದರೂ ಕಾಲವಾದರೆ ಅವನು ಭೂತವಾಗುತ್ತಾನೆ ಎಂಬ ವದಂತಿ ಇದೆ. ಅದಕ್ಕಾಗಿ ಯಾರಾದರೂ ಮರದ ಮೇಲಿಂದ ಬಿದ್ದರೆ ಅಥವಾ ಯಾರಾದರೂ ದಾರಿಯಲ್ಲಿ ಹೋಗುತ್ತಿರುವಾಗ ಅಪಘಾತದಿಂದ ಬಿದ್ದರೆ, ಆಕಸ್ಮಿಕ ಮರಣದಿಂದ ಪಿಶಾಚಿಯಾಗಿ

ಈ ಪ್ರಪಂಚವನ್ನು ಸಂಪೂರ್ಣ ತ್ಯಜಿಸಿರುವವನು ಅಪರೂಪ.

