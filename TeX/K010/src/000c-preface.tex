
\chapter*{ಮುನ್ನುಡಿ}

ವಂಗದೇಶದ ಒಂದು ಸಣ್ಣ ಗ್ರಾಮದಲ್ಲಿ ಜನಿಸಿದ ಶ್ರೀರಾಮಕೃಷ್ಣ ಪರಮ ಹಂಸರು, ತಮ್ಮ ಪವಿತ್ರ ಜೀವನ ಮತ್ತು ಸಾಧನೆಯ ಬಲದಿಂದ ಆಧ್ಯಾತ್ಮಿಕ ಜೀವನದಲ್ಲಿ ಸಫಲರಾದುದಷ್ಟೇ ಅಲ್ಲ, ಎಲ್ಲ ಸಂಪ್ರದಾಯಗಳ, ಮತಪಂಗಡಗಳ ಪಾರಮಾರ್ಥಿಕ ಅನುಭವಗಳನ್ನು ತಮ್ಮದಾಗಿಸಿಕೊಂಡರು. ತಮ್ಮ ಸರಳ ಸಹಜ ಸ್ವಭಾವದಿಂದ ಉನ್ಮತ್ತರಂತೆ ಭಗವದನ್ವೇಷಣೆಯಲ್ಲಿ ತೊಡಗಿ, ಅದರ ಪರಾ ಕಾಷ್ಠೆಯನ್ನು ಪಡೆಯುವವರೆಗೂ ವಿರಮಿಸಲಿಲ್ಲ. ಅಂದಿನ ವಿಚಾರವಂತರು, ಪಂಡಿತರು, ಸನ್ಮಾನ್ಯರು ಹಾಗೂ ಸಾಧಕರೆಲ್ಲಾ ಅವರ ತುಂಬಿದ ಜೀವನ ಮತ್ತು ಸಂದೇಶದ ಸವಿಯನ್ನು ಆಸ್ವಾದಿಸಿ, ಪುನೀತರಾಗಿದ್ದಾರೆ. ವಿಶ್ವದಾದ್ಯಂತ ಅವರ ವಿಚಾರಧಾರೆ ಹರಡಿ ಜನಸಾಮಾನ್ಯರಲ್ಲಿ ಆತ್ಮಶ್ರದ್ಧೆ, ಭಗವದ್ಭಕ್ತಿ ಮತ್ತು ಧಾರ್ಮಿಕ ಪ್ರಜ್ಞೆಯನ್ನು ಜಾಗೃತಗೊಳಿಸುತ್ತಿದೆ.

ಶ್ರೀರಾಮಕೃಷ್ಣರ ವಚನಗಳು ನೀರಸವಲ್ಲ. ಪುರಾಣಗಳಿಂದ ಮತ್ತು ನಿತ್ಯ ಜೀವನದ ಸಾಮಾನ್ಯ ಘಟನೆಗಳನ್ನು ಬಳಸಿಕೊಂಡು ಅವರು ಜಟಿಲವಾದ ತತ್ತ್ವ ವಿಚಾರಗಳನ್ನೂ ಸರಳ ಮಾಡಿ ತಿಳಿಹೇಳಿದ್ದಾರೆ. ಅವರ ತೀಕ್ಷ ್ಣದೃಷ್ಟಿಗೆ ಬೀಳದ ಸನ್ನಿವೇಶವೇ ಇಲ್ಲ. ಹಾಸ್ಯಮಯವಾಗಿ ತಮ್ಮದೇ ಆದ ಶೈಲಿಯಲ್ಲಿ ಕೇಳುಗರ ಮನಮುಟ್ಟುವಂತೆ ಇದೆ ಅವರ ಕಥಾಮೃತ. ಶ್ರೀ ರಾಮಕೃಷ್ಣ ವಚನವೇದದಲ್ಲಿ ಬರುವ ಕಥೆಗಳನ್ನು ಅವುಗಳಲ್ಲಿರುವ ತಾತ್ತ್ವಿಕ ಮೌಲ್ಯಗಳ ಆಧಾರದ ಮೇಲೆ ವಿಂಗಡಿಸಿ ಓದುಗರಿಗೆ ರುಚಿಸುವಂತೆ ಈ ಪುಸ್ತಕದಲ್ಲಿ ಉಣಬಡಿಸಲಾಗಿದೆ. ಈ ಹೊಸ ಸಂಸ್ಕರಣಕ್ಕೆ \eng{‘Tales and Parables of Sri Ramakrishna’} ಎಂಬ ಆಂಗ್ಲಭಾಷೆಯ ಪುಸ್ತಕವನ್ನು ಆಧಾರಗ್ರಂಥವಾಗಿರಿಸಿಕೊಳ್ಳಲಾಗಿದೆ. ಮನರಂಜನೆ ಗಾಗಿ ಓದಿದರೂ ಈ ಕಥೆಗಳಲ್ಲಿರುವ ಅಮೂಲ್ಯ ನೀತಿರತ್ನಗಳು ನಮ್ಮನ್ನು ಸಂಪದ್ಭರಿತರನ್ನಾಗಿ ಮಾಡುವಲ್ಲಿ ಸಂದೇಹವಿಲ್ಲ. ಶ್ರೀ ಪಿ. ಪುರುಷೋತ್ತಮ ಕಾರಂತರ ಸುಂದರ ರೇಖಾಚಿತ್ರಗಳ ಮೂಲಕ ಪುಸ್ತಕವು ಇನ್ನಷ್ಟು ಸೊಗಸಾಗಿ ಮೂಡಿಬಂದಿದೆ. ಮುಖಪುಟದ ವರ್ಣಚಿತ್ರವನ್ನು ಶ್ರೀ ರಾಮ್ ಗೌತಮ್ ಸುಂದರವಾಗಿ ರೂಪಿಸಿದ್ದಾರೆ. ಶ್ರೀರಾಮಕೃಷ್ಣರ ಉಪದೇಶಗಳಲ್ಲಿ ವ್ಯಕ್ತವಾ ಗಿರುವ ಅನೇಕ ಭಾವನೆಗಳು ಈ ಪುಸ್ತಕದ ಮೂಲಕ ಓದುಗರ ಕೈಸೇರಿ ಹೃನ್ಮನ ತುಂಬಲಿ ಎಂಬುದು ನಮ್ಮ ಹಾರೈಕೆ.

\begin{flushright}
ಪ್ರಕಾಶಕರು
\end{flushright}

