
\chapter{ಬ್ರಹ್ಮ}

\section{\num{೧೧೮. } ಸಮುದ್ರದ ಆಳವನ್ನು ಅಳೆದ ಉಪ್ಪಿನ ಗೊಂಬೆ}

ಒಂದು ಉಪ್ಪಿನಗೊಂಬೆ ಸಮುದ್ರದ ಆಳವನ್ನು ಕಂಡುಹಿಡಿ ಯಲು ಹೋಯಿತು. ಇತರರಿಗೆ ಅದು ಎಷ್ಟು ಆಳವಾಗಿದೆ, ಎಂಬುದನ್ನು ಹೇಳ ಬೇಕೆಂದಿತ್ತು. ಆದರೆ ಅದಕ್ಕೆ ಇದು ಸಾಧ್ಯವಾಗಲಿಲ್ಲ. ಅದು ನೀರಿನೊಳಗೆ ಹೋದೊಡನೆಯೇ ಕರಗಿಹೋಯಿತು. ಇನ್ನು ಸಮುದ್ರದ ಆಳವನ್ನು ಇತರರಿಗೆ ತಿಳಿಸಿಕೊಡುವವರಾರು?

ಬ್ರಹ್ಮನ ಸ್ವರೂಪವನ್ನು ವಿವರಿಸಲು ಆಗುವುದಿಲ್ಲ. ಸಮಾಧಿಯಲ್ಲಿ ಜ್ಞಾನೋದಯವಾದಾಗ–ಬ್ರಹ್ಮಸಾಕ್ಷಾತ್ಕಾರವಾಗುವುದು. ಆಗ ವಿಚಾರ ನಿಂತು ಹೋಗುವುದು, ಮನುಷ್ಯ ಮೂಕನಾಗುತ್ತಾನೆ. ಬ್ರಹ್ಮನ ವಿಷಯವನ್ನು ಅವನು ತಿಳಿಸಲಾರ.


\section{\num{೧೧೯. } ನಾಲ್ಕು ಜನ ಹೊರಗೆ ನೋಡಿದರು}

ಒಂದು ಸಲ ನಾಲ್ಕು ಜನ ಸ್ನೇಹಿತರು ಹೋಗುತ್ತಿದ್ದಾಗ, ಒಂದು ಬಯಲಿನ ಸುತ್ತಲೂ ಗೋಡೆಯಿರುವುದನ್ನು ಕಂಡರು. ಗೋಡೆ ಬಹಳ ಎತ್ತರವಾಗಿತ್ತು. ಅದರೊಳಗೆ ಏನಿರಬಹುದು ಎಂಬುದನ್ನು ತಿಳಿಯಲು ಎಲ್ಲರೂ ಕುತೂಹಲಪಟ್ಟರು. ಅದರಲ್ಲಿ ಒಬ್ಬ ಗೋಡೆಯ ನ್ನೇರಿ ಒಳಗೆ ನೋಡಿದನು. ಒಳಗಿರುವುದನ್ನು ನೋಡಿದಾಗ ಆಶ್ಚರ್ಯದಿಂದ ಅವನು ಮೂಕನಾಗಿ ‘ಓ’ ಎಂದು ಹೇಳುತ್ತ ಒಳಗೆ ಹಾರಿದನು. ಅವನು ಏನನ್ನು ನೋಡಿದನೊ ಅದನ್ನು ಯಾರಿಗೂ ಹೇಳಲಾಗಲಿಲ್ಲ. ಇತರರೂ ಹಾಗೆ ಗೋಡೆ ಹತ್ತಿ ಒಳಗಿರುವುದನ್ನು ನೋಡಿ ಆಶ್ಚರ್ಯದಿಂದ ಒಳಗೆ ಹಾರಿದರು. ಈಗ ಒಳಗೆ ಏನಿದೆ ಎಂಬುದನ್ನು ಯಾರು ಹೇಳುವವರು!

ಬ್ರಹ್ಮನ ವಿಷಯವನ್ನು ಯಾರೂ ವಿವರಿಸುವುದಕ್ಕೆ ಆಗುವುದಿಲ್ಲ. ಯಾರು ಅದನ್ನು ಅನುಭವಿಸಿರುವನೋ ಅವನು ಕೂಡ ಅದನ್ನು ವಿವರಿಸಲಾರ.


\section{\num{೧೨೦. } ಮೌನವಿರುವಲ್ಲಿ ಮಾತು ನಿಂತು ಹೋಗುವುದು}

ಒಬ್ಬನಿಗೆ ಇಬ್ಬರು ಮಕ್ಕಳಿದ್ದರು. ತಂದೆ ಇಬ್ಬರನ್ನೂ ಗುರುಗಳ ಬಳಿಗೆ ಕಳುಹಿಸಿದನು. ಕೆಲವು ವರ್ಷಗಳಾದ ಮೇಲೆ ಗುರುಗಳ ಮನೆಯಿಂದ ಹಿಂತಿ ರುಗಿ ಬಂದು ತಂದೆಗೆ ನಮಸ್ಕರಿಸಿದರು. ಅವರ ಜ್ಞಾನದ ಆಳವನ್ನು ಅರಿಯ ಬೇಕೆಂದು, ಹಿರಿಯನನ್ನು “ಮಗು, ನೀನು ಎಲ್ಲ ಶಾಸ್ತ್ರಗಳನ್ನೂ ಓದಿರುವೆ. ಬ್ರಹ್ಮದ ಸ್ವರೂಪ ಏನೆಂಬುದನ್ನು ಹೇಳು,” ಎಂದನು. ಅವನು ಶಾಸ್ತ್ರವಾಕ್ಯ ಗಳನ್ನು ಉದಾಹರಿಸಿ ಬ್ರಹ್ಮವನ್ನು ವಿವ ರಿಸತೊಡಗಿದನು. ತಂದೆ ಏನೂ ಹೇಳದೆ ಕಿರಿಯ ಮಗನಿಗೆ ಅದೇ ಪ್ರಶ್ನೆ



\section{\num{೧೨೧. } ಹೌದು ಎಂತಲೂ ಇಲ್ಲ, ಇಲ್ಲ ಎಂತಲೂ ಇಲ್ಲ}

ಒಬ್ಬ ಹುಡುಗಿಯ ಗಂಡ ಮಾವನ ಮನೆಗೆ ಬಂದು ತನ್ನ ವಯಸ್ಸಿನ ಇತರರೊಡನೆ ಭೈಟಕ್ ಖಾನೆಯಲ್ಲಿ ಕುಳಿತಿದ್ದ. ಕಿಟಕಿ ಬಾಗಿಲಿನ ಮೂಲಕ ಆ ಹುಡುಗಿ ಮತ್ತು ಅವಳ ಸ್ನೇಹಿತರು ಇವರನ್ನು ನೋಡುತ್ತಿದ್ದರು. ಅವಳ ಸ್ನೇಹಿತರಿಗೆ ಇವಳ ಗಂಡನಾರು ಎಂಬುದು ತಿಳಿಯದು. ಒಬ್ಬ ಯುವಕನನ್ನು ತೋರಿಸಿ “ಅವನೇ ನಿನ್ನ ಗಂಡನೇನು?” ಎಂದು ಕೇಳಿದರು. ನಗುತ್ತ “ಅಲ್ಲ” ಅಂದಳು. ಮತ್ತೊಬ್ಬ ಯುವಕನನ್ನು ತೋರಿಸಿ “ಅವನೇ ನಿನ್ನ ಗಂಡ” ಎಂದು ಹೇಳಿದರು. ಆಗಲೂ “ಅಲ್ಲ” ಎಂದಳು. ಮೂರನೆಯವನನ್ನು ತೋರಿಸಿ ಕೇಳಿದಾಗಲೂ ಅದೇ ಉತ್ತರ. ಕೊನೆಗೆ ಅವಳ ಗಂಡನನ್ನು ತೋರಿಸಿ ಕೇಳಿದಾಗ ಅವಳು ಹೌದು ಎಂತಲೂ ಹೇಳಲಿಲ್ಲ, ಇಲ್ಲ ಅಂತಲೂ ಹೇಳಲಿಲ್ಲ. ಸುಮ್ಮನೆ ನಗುತ್ತ ತಲೆತಗ್ಗಿಸಿದಳು. ಅವಳ ಸ್ನೇಹಿತರಿಗೆ ಅವನೇ ಅವಳ ಗಂಡ ಎಂದು ಗೊತ್ತಾಯಿತು. ಬ್ರಹ್ಮನ ವಿಷಯವನ್ನು ಒಬ್ಬ ಅರಿತ ಮೇಲೆ ಮೌನಿಯಾಗು ವನು.


\section{\num{೧೨೨. } ರಾಜ ಮತ್ತು ಮಂತ್ರವಾದಿ}

ನೀವು ದೇವರನ್ನು ಸಮೀಪಿಸಿದಂತೆ ಅವನ ಉಪಾಧಿಗಳು ಕಡಿಮೆಯಾಗುತ್ತಾ ಬರುವುವು. ಭಕ್ತ ಮೊದಲು ದಶಭುಜಳಾದ ಭಗವತಿಯನ್ನು ನೋಡುತ್ತಾನೆ. ಅವಳನ್ನು ಸಮೀಪಿಸಿದಂತೆ ಅವಳು ಷಟ್​ಭುಜಳಾಗುತ್ತಾಳೆ. ಇನ್ನೂ ಮುಂದೆ ಹೋದರೆ ಎರಡು ಕೈಗಳ ಗೋಪಾಲನನ್ನು ನೋಡುತ್ತಾನೆ. ಭಗವಂತನನ್ನು ಸಮೀಪಿಸುತ್ತಿದ್ದಂತೆ ಅವನ ಉಪಾಧಿಗಳು ಕಡಿಮೆಯಾಗುತ್ತಾ\\ಬರುವುದು. ಭಗವಂತನು ಎದುರಿಗೆ ನಿಂತಾಗ ಯಾವ ಗುಣವೂ ಇಲ್ಲದ ಕಾಂತಿಯನ್ನು ನೋಡುತ್ತಾನೆ.

ವೇದಾಂತದ ಬೋಧನೆಯನ್ನು ಸ್ವಲ್ಪ ಕೇಳಿ. ಒಬ್ಬ ಮಾಂತ್ರಿಕ ತನಗೆ ಗೊತ್ತಿರುವ ಮಾಯಾಮಂತ್ರಗಳನ್ನು ತೋರಲು ರಾಜನ ಸಮೀಪಕ್ಕೆ ಹೋದ. ಮಾಂತ್ರಿಕ ಸ್ವಲ್ಪ ಮರೆಯಾದಾಗ ರಾಜನು ಒಂದು ಕುದುರೆಯ ಮೇಲೆ ಸವಾರಿಮಾಡುತ್ತಿರುವವನನ್ನು ಕಂಡ. ಅವನಿಗೆ ಕಣ್ಣು ಕೋರೈಸುವ ಬಟ್ಟೆಗಳೂ ಅವನ ಕೈಯಲ್ಲಿ ಹಲವು ಆಯುಧಗಳೂ ಇದ್ದವು. ರಾಜ ಮತ್ತು ಸಭಿಕರು ಇದರಲ್ಲಿ ಯಾವುದು ನಿಜ ಎಂದು ಭ್ರಾಂತರಾದರು. ಕುದುರೆ ನಿಜವಲ್ಲ, ಅವನ ಮೇಲಿರುವ ಬಟ್ಟೆ ನಿಜವಲ್ಲ. ಕೈಯಲ್ಲಿರುವ ಕತ್ತಿ ಗುರಾಣಿ ಗಳೂ ನಿಜವಲ್ಲ. ಅನಂತರ ಅವರಿಗೆ ತೋರಿತು. ಸವಾರನೊಬ್ಬನೆ ನಿಜ. ಇದರ ಅರ್ಥವೇನೆಂದರೆ ಬ್ರಹ್ಮನೊಬ್ಬನೆ ಸತ್ಯ. ಈ ಪ್ರಪಂಚ ಮಿಥ್ಯ. ಅವುಗಳನ್ನೆಲ್ಲ ವಿಭಜನೆ ಮಾಡಿದರೆ ಎಲ್ಲವೂ ಮಾಯವಾಗುತ್ತವೆ.


\section{\num{೧೨೩. } ಸತ್ಯದೆದುರಿಗೆ ನಿಂತಾಗ}

ನೇತಿ ನೇತಿ ಎಂಬುದನ್ನು ಅಭ್ಯಾಸ ಮಾಡುತ್ತ ಬಂದರೆ ಶಾಂತಿ ನೆಲೆಸುವುದು. ಅದೇ ಬ್ರಹ್ಮ. ರಾಜ ತನ್ನ ಅರಮನೆಯಲ್ಲಿ ಏಳು ಬಾಗಿಲುಗಳ ಒಳಗೆ ಇರು ವನು. ಭೇಟಿಕಾರನು ಮೊದಲನೆ ಬಾಗಿಲಿಗೆ ಬರುತ್ತಾನೆ. ಅಲ್ಲಿ ಬಹಳ ವೈಭವ ದಿಂದ ಇರುವ ವ್ಯಕ್ತಿಯನ್ನು ನೋಡುತ್ತಾನೆ. ಅವನ ಸುತ್ತಲೂ ಪರಿವಾರದವರು ಇದ್ದರು. ಭೇಟಿಕಾರನು “ಇವರೇ ಏನು ರಾಜರು?” ಎಂದು ಕೇಳುತ್ತಾನೆ. ಅವನ ಸ್ನೇಹಿತ ನಗುತ್ತ “ಇವರಲ್ಲ” ಎನ್ನುತ್ತಾನೆ.

ಇತರ ಎರಡನೆ ಮೂರನೆ ದ್ವಾರಗಳಲ್ಲಿಯೂ ಇದೇ ರೀತಿ ಕೇಳುತ್ತಾನೆ. ಅವನು ಅರಮನೆಯ ಒಳಒಳಗೆ ಹೋದಂತೆಲ್ಲ ಅದರ ವೈಭವ ಅಲಂಕಾರಗಳು ಹೆಚ್ಚುತ್ತಾ ಬರುತ್ತವೆ. ಅವನು ಏಳನೆ ಬಾಗಿಲಿಗೆ ಬಂದಾಗ ಇವನೆ ಏನು ರಾಜ ಎಂಬ ಪ್ರಶ್ನೆಯನ್ನೇ ಕೇಳುವುದಿಲ್ಲ. ರಾಜನ ಅನಂತ ವೈಭವವನ್ನು ನೋಡಿದಾಗ ಮಾತಿಲ್ಲದೆ ಮೂಕನಾಗುತ್ತಾನೆ. ಆಗ ರಾಜರೆದುರಿಗೆ ನಾನಿರುವೆ ಎಂದು ಗೊತ್ತಾಗುವುದು. ಆ ವಿಷಯದಲ್ಲಿ ಅವನಿಗೆ ಎಳ್ಳಷ್ಟೂ ಸಂಶಯವಿರುವುದಿಲ್ಲ.


\section{\num{೧೨೪. } ರಾಜರೆ ನೋಡಿ! ರಾಜರೆ ನೋಡಿ!}

ಒಂದು ಸಲ ಒಬ್ಬ ಯೋಗಿಯನ್ನು ರಾಜನು “ಒಂದು ಮಾತಿ ನಲ್ಲಿ ಬ್ರಹ್ಮಜ್ಞಾನವನ್ನು ಬೋಧಿಸಿ” ಎಂದು ಕೇಳಿದ. “ಆಗಲಿ, ಒಂದು ಮಾತಿನಲ್ಲಿ ಅದನ್ನು ಬೋಧಿಸುತ್ತೇನೆ” ಎಂದ ಯೋಗಿ. ಸ್ವಲ್ಪ ಕಾಲದ ಮೇಲೆ ಒಬ್ಬ ಮಂತ್ರವಾದಿ ರಾಜನ ಬಳಿಗೆ ಬಂದ. ಅವನು ತನ್ನ ಎರಡು ಬೆರಳುಗಳನ್ನು ಬಹಳ ವೇಗವಾಗಿ ಚಲಿಸುತ್ತ, “ನೋಡಿ ರಾಜರೆ, ನೋಡಿ ರಾಜರೆ” ಎಂದ. ರಾಜ ಅದನ್ನು ನೋಡುತ್ತಿರುವಾಗ ಎರಡು ಬೆರಳು ಗಳು ಒಂದಾಗುವುದನ್ನು ಕಂಡನು. ಮಂತ್ರವಾದಿ ತನ್ನ ಒಂದು ಬೆರಳನ್ನು ತ್ವರಿತದಿಂದ ಚಲಿಸಿ, “ನೋಡಿ ರಾಜರೆ ನೋಡಿ, ನೋಡಿ ರಾಜರೆ,” ಎಂದ. ಇದರ ಅರ್ಥವೇನಂದರೆ, ಬ್ರಹ್ಮ ಮತ್ತು ಮಾಯೆ ಮೊದಲು ಎರಡರಂತೆ ಕಾಣುವುದು. ಆದರೆ ಜ್ಞಾನವನ್ನು ಪಡೆದ ಮೇಲೆ ಎರಡನ್ನು ನೋಡುವುದಿಲ್ಲ. ಆಗ ವ್ಯತ್ಯಾಸವೆ ಕಾಣುವುದಿಲ್ಲ. ಅದು ಏಕಮೇವ ಅದ್ವಿತೀಯ.


\section{\num{೧೨೫. } ಒಂದು ಇರುವೆ ಸಕ್ಕರೆ ಬೆಟ್ಟಕ್ಕೆ ಹೋಯಿತು}

ಮನುಷ್ಯ ತಾನು ಬ್ರಹ್ಮನ ವಿಷಯವನ್ನೆಲ್ಲ ತಿಳಿದುಕೊಂಡಿರುವೆನು ಎಂದು ಭಾವಿಸುವನು. ಒಂದು ಸಲ ಇರುವೆಯೊಂದು ಸಕ್ಕರೆ ಬೆಟ್ಟಕ್ಕೆ ಹೋಯಿತು.



\section{\num{೧೨೬. } ಅವನು ತಿಂದರೂ ತಿನ್ನನು}

ಒಂದು ಸಲ ವ್ಯಾಸದೇವರು ಯಮುನಾ ನದಿಯನ್ನು ದಾಟುವುದರಲ್ಲಿದ್ದರು. ಜತೆಗೆ ಗೋಪಿಯರೂ ಇದ್ದರು. ಗೋಪಿಯರು ನದಿಯ ಅತ್ತಕಡೆಗೆ ಹೋಗಿ ಹಾಲು, ಮೊಸರು, ಕೆನೆ ಇವುಗಳನ್ನು ಮಾರಬೇಕೆಂದು ಇದ್ದರು. ಆ ಸಮಯ ದಲ್ಲಿ ದೋಣಿ ಸಿಕ್ಕಲಿಲ್ಲ. ಹೇಗೆ ನದಿಯನ್ನು ದಾಟುವುದು ಎಂದು ಚಿಂತಾ ಕ್ರಾಂತರಾದರು. ಆಗ ವ್ಯಾಸರು “ನನಗೆ ತುಂಬಾ ಹಸಿವಾಗಿದೆ” ಎಂದರು. ಹಾಲು ಮಾರುವವರು ತಮ್ಮಲ್ಲಿದ್ದ ಹಾಲು ಮತ್ತು ಕೆನೆಯನ್ನು ಕೊಟ್ಟರು. ವ್ಯಾಸರು ಅದನ್ನೆಲ್ಲ ತಿಂದರು. ಆಗ ವ್ಯಾಸರು ಎದುರಿಗಿರುವ ಯಮುನೆಯನ್ನು ಉದ್ದೇಶಿಸಿ, “ಯಮುನೆ, ನಾನು ಏನನ್ನೂ ತಿಂದಿರದಿದ್ದರೆ, ನೀನು ಇಬ್ಭಾಗ ವಾಗು. ನಾವು ನಡೆದುಕೊಂಡು ಹೋಗುತ್ತೇವೆ.” ನದಿ ಇಬ್ಭಾಗವಾಗಿ ಒಂದು ಹಾದಿಯು ನಿರ್ಮಿತವಾಯಿತು. ಎಲ್ಲರೂ ಅದರ ಮೂಲಕ ಆಚೆ ಕಡೆಗೆ ಹೋದರು.

ವ್ಯಾಸರು ನಾನು ಏನನ್ನೂ ತಿಂದಿರದಿದ್ದರೆ ಎಂದು ಹೇಳಿದ್ದರು. ಅಂದರೆ, ನಿಜವಾದ ಮನುಷ್ಯನು ಪರಿಶುದ್ಧ ಆತ್ಮನು. ಆತ್ಮ ಅನಾಸಕ್ತ, ಪ್ರಕೃತಿಗೆ ಅತೀತ. ಅವನಿಗೆ ಹಸಿವಾಗಲಿ ಬಾಯಾರಿಕೆಯಾಗಲಿ ಇಲ್ಲ. ಅವನಿಗೆ ಜನನ ಮರಣ ಗಳಿಲ್ಲ. ಅವನಿಗೆ ವಯಸ್ಸಾಗುವುದಿಲ್ಲ. ಅವನು ಸಾಯುವುದೂ ಇಲ್ಲ. ಅವನು ಸುಮೇರುವಿನಂತೆ ಅಚಲ.


\section{\num{೧೨೭. } ಎಲ್ಲ ಪರಿಶುದ್ಧ ಆತ್ಮ}

ದೇವರನ್ನು ನೋಡಿದ ಮೇಲೆ ಒಬ್ಬನ ಸಂಶಯಗಳೆಲ್ಲ ನಾಶವಾಗುವುವು. ದೇವರ ವಿಷಯವನ್ನು ಕೇಳುವುದು ಒಂದು, ಅವನ ಸನ್ನಿಧಿಯನ್ನು ಅನುಭವಿಸುವುದು ಮತ್ತೊಂದು. ಬರೀ ಕೇಳುವುದರಿಂದ ಮನುಷ್ಯನಿಗೆ ನೂರಕ್ಕೆ ನೂರು ಭಾಗ ವಿಶ್ವಾಸವಾಗುವುದಿಲ್ಲ. ಆದರೆ ಯಾವಾಗ ದೇವರನ್ನು ಪ್ರತ್ಯಕ್ಷ ಅನುಭವಿಸುತ್ತಾನೊ ಆಗ ಅವನ ಸಂಶಯಗಳೆಲ್ಲ ನಿವಾರಣೆಯಾಗುವುವು.

ಭಗವಂತನ ಸಾಕ್ಷಾತ್ಕಾರವಾದ ಮೇಲೆ ಬಾಹ್ಯ ಪೂಜೆ ನಿಂತುಹೋಗುವುದು. ದೇವಸ್ಥಾನದಲ್ಲಿ ನನ್ನ ಬಾಹ್ಯ ಪೂಜೆ ಹಾಗೆಯೇ ನಿಂತುಹೋಗಿದ್ದು, ನಾನು ಮುಂಚೆ ಕಾಳಿಕಾಮಾತೆಯನ್ನು ಪೂಜಿಸುತ್ತಿದ್ದೆ. ಒಂದು ದಿನ ಇದ್ದಕ್ಕಿದ್ದಂತೆ ಎಲ್ಲ ಚಿನ್ಮಯವಾಗಿರುವುದನ್ನು ಕಂಡೆ. ಪೂಜಾಪಾತ್ರೆಗಳು, ವಿಗ್ರಹ ಬಾಗಿಲು ಎಲ್ಲಾ ಚಿನ್ಮಯ! ಆಗ ನಾನು ಒಬ್ಬ ಹುಚ್ಚನಂತೆ ಹೂವನ್ನು ಎಲ್ಲಾ ಕಡೆ ಎರಚಿದೆ. ನಾನು ಕಂಡದ್ದನ್ನೆಲ್ಲ ಪೂಜಿಸಿದೆ.

\chapter{ಭಗವತಿಯ ಆವಿರ್ಭಾವಗಳು}

\section{\num{೧೨೮. } ಊಸರವಳ್ಳಿ}

ಒಂದು ಸಲ ಒಬ್ಬ ಕಾಡಿಗೆ ಹೋಗಿದ್ದಾಗ ಅಲ್ಲಿ ಮರದ ಮೇಲೆ\\ಒಂದು ಸಣ್ಣ ಪ್ರಾಣಿಯನ್ನು ಕಂಡ. ಅವನು ಹಿಂತಿರುಗಿ ಬಂದು ಮತ್ತೊಬ್ಬ ನಿಗೆ, “ನಾನೊಂದು ಮರದಮೇಲೆ ಸುಂದರವಾದ ಕೆಂಪು ಬಣ್ಣದ ಪ್ರಾಣಿಯನ್ನು ಕಂಡೆ” ಎಂದನು. ಎರಡನೆಯವನು, “ನಾನು ಕಾಡಿಗೆ ಹೋದಾಗ ಅದೇ ಪ್ರಾಣಿಯನ್ನು ಕಂಡೆ. ಅದನ್ನು ಕೆಂಪು ಎಂದು ಹೇಗೆ ಹೇಳುತ್ತೀಯ, ಅದು ಹಳದಿ ಬಣ್ಣದ್ದು” ಎಂದ. ಮತ್ತೊಬ್ಬ ಇದನ್ನು ನೋಡಿದವನು, “ಕಂದು ಬಣ್ಣದ್ದು” ಎಂದ. ಅನಂತರ ಮತ್ತೆ ಕೆಲವರು ಬಂದವರು, “ಅದು ಬೂದು ಬಣ್ಣದ್ದು” ಎಂದರು. ಇನ್ನೊಬ್ಬರು “ನೀಲಿ” ಎಂದರು. ಅನಂತರ ಅವರಲ್ಲಿ ದೊಡ್ಡ ಗೊಂದಲವೆದ್ದಿತು. ಇದನ್ನು ನಿರ್ಣಯಿಸುವುದಕ್ಕೆ ಎಲ್ಲರೂ ಮರದ ಕಡೆಗೆ ಹೋದರು. ಮರದ ಬುಡದಲ್ಲಿ ಒಬ್ಬನನ್ನು ಕಂಡರು. ಆ ಪ್ರಾಣಿಯ ಬಗ್ಗೆ ಅವನನ್ನು ವಿಚಾರಿಸಿದಾಗ ಅವನು “ನಾನು ಈ ಮರದ ಅಡಿಯಲ್ಲೆ ಇರುವೆನು. ಆ ಪ್ರಾಣಿಯನ್ನು ನಾನು ಚೆನ್ನಾಗಿ ಬಲ್ಲೆ. ನೀವು ಹೇಳಿದ್ದೆಲ್ಲ ಸರಿಯೆ. ಕೆಲವು ವೇಳೆ ಅದು ಕೆಂಪಾಗಿ ಕಾಣುವುದು. ಕೆಲವು ವೇಳೆ ಹೊಂಬಣ್ಣವಿರು ವುದು. ಕೆಲವು ವೇಳೆ ನೀಲಿ, ಕೆಲವು ವೇಳೆ ಕಂದು, ಹಾಗೆಯೇ ಹಲವು ರೂಪ ಗಳನ್ನು ಅದು ತಾಳುವುದು. ಕೆಲವು ವೇಳೆ ಅದಕ್ಕೆ ಯಾವ ಬಣ್ಣವೂ ಇರುವು ದಿಲ್ಲ,” ಎಂದನು.

ಇದರಂತೆಯೇ ಯಾರು ಯಾವಾಗಲೂ ದೇವರನ್ನು ಚಿಂತಿಸುತ್ತಿರುವನೊ ಅವನಿಗೆ ಭಗವಂತನ ಸ್ವಭಾವ ಗೊತ್ತಾಗುವುದು. ಅವನು ದೇವರು ಒಬ್ಬೊಬ್ಬ ರಿಗೆ ಒಂದೊಂದು ರೀತಿ ತೋರುತ್ತಾನೆ ಎಂಬುದನ್ನು ಬಲ್ಲನು. ದೇವರು ಕೆಲವು ವೇಳೆ ಗುಣದಿಂದ ಕೂಡಿರುತ್ತಾನೆ, ಕೆಲವು ವೇಳೆ ಯಾವ ಗುಣವೂ ಇರುವು ದಿಲ್ಲ. ಯಾರು ಮರದ ಬುಡದಲ್ಲಿ ಇರುವನೊ ಅವನಿಗೆ ಊಸರವಳ್ಳಿಯು ಹಲವು ಬಣ್ಣಗಳನ್ನು ಧರಿಸಬಲ್ಲದು, ಅದಕ್ಕೆ ಕೆಲವು ವೇಳೆ ಯಾವ ಬಣ್ಣವೂ ಇರುವುದಿಲ್ಲ ಎಂಬುದು ಗೊತ್ತಿರುತ್ತದೆ. ಇತರರು ಬರೀ ವ್ಯಾಜ್ಯದಲ್ಲಿ ಬವಣೆ ಪಡುವರು.


\section{\num{೧೨೯. } ಒಂದು ಬಣ್ಣದ ಪೀಪಾಯಿಯ ಮನುಷ್ಯ}

ಸಾಧಾರಣವಾಗಿ ಮನುಷ್ಯನಲ್ಲಿ ಈ ಸಂಶಯ ಬರುವುದು: ದೇವರು ನಿರಾಕಾರನಾಗಿದ್ದರೆ ಮತ್ತೆ ಅವನು ಸಾಕಾರನಾಗಲು ಹೇಗೆ ಸಾಧ್ಯ? ಅಷ್ಟೇ ಅಲ್ಲ, ಅವನು ಸಾಕಾರನಾದಲ್ಲಿ, ಪರಿಪರಿಯ ರೂಪಗಳನ್ನು ಏಕೆ ಹೊಂದಿರುವನು? ದೇವರನ್ನು ಪ್ರತ್ಯಕ್ಷ ನೋಡುವವರೆಗೆ ಈ ಸಂಶಯಗಳು ನಾಶವಾಗುವುದಿಲ್ಲ. ದೇವರು ಭಕ್ತರಿಗಾಗಿ ಹಲವು ರೂಪ ಗಳನ್ನು ಧರಿಸುತ್ತಾನೆ. ನೀರು ನಿರಾಕಾರವಾದರೂ ಪಾತ್ರೆಗಳಿಗೆ ತುಂಬಿಸಿದಾಗ ಆಯಾಯ ಪಾತ್ರೆಗಳ ಆಕಾರವನ್ನು ಪಡೆಯುತ್ತದೆ.

ಒಬ್ಬ ಮನುಷ್ಯ ಒಂದು ಬಣ್ಣದ ಪೀಪಾಯಿಯನ್ನು ಇಟ್ಟುಕೊಂಡಿದ್ದ. ಅನೇಕರು ತಮ್ಮ ಬಟ್ಟೆಗೆ ಬಣ್ಣ ಹಾಕಿಸಿಕೊಳ್ಳಲು ಅವನ ಹತ್ತಿರ ಬರುತ್ತಿದ್ದರು. ಅವನು ಗಿರಾಕಿಗೆ “ನಿಮ್ಮ ಬಟ್ಟೆಗೆ ಯಾವ ಬಣ್ಣ ಬೇಕು” ಎಂದು ಕೇಳುತ್ತಿದ್ದ. ಗಿರಾಕಿಗೆ ಕೆಂಪು ಬಣ್ಣ ಇಷ್ಟವಾದರೆ; ಅವನು ಪೀಪಾಯಿಗೆ ಅದ್ದಿ, “ತೆಗೆದು ಕೊಳ್ಳಿ, ನಿಮ್ಮ ಕೆಂಪು ಬಣ್ಣದ ಬಟ್ಟೆಯನ್ನು” ಎನ್ನುವನು. ಮತ್ತೊಬ್ಬ ಗಿರಾಕಿ ಹಳದಿ ಬಣ್ಣ ಕೇಳಿದರೆ, ಅವನು ಅದೇ ಪೀಪಾಯಿಗೆ ಅದ್ದಿ ಹಳದಿ ಬಣ್ಣದ ಬಟ್ಟೆಯನ್ನು ಕೊಡುತ್ತಿದ್ದ. ಮತ್ತೊಬ್ಬನಿಗೆ ನೀಲಿ ಬಣ್ಣ ಬೇಕಾದರೆ ಅವನು ಅದೇ ಪೀಪಾಯಿಗೆ ಅದ್ದಿ “ತೆಗೆದುಕೊಳ್ಳಿ, ನಿಮ್ಮ ನೀಲಿ ಬಣ್ಣದ ಬಟ್ಟೆಯನ್ನು” ಎನ್ನುತ್ತಿದ್ದ. ಗಿರಾಕಿಗಳಿಗೆ ಯಾವ ಬಣ್ಣ ಬೇಕಾದರೂ ಒಂದೇ ಪೀಪಾಯಿಯಲ್ಲಿ ಅದ್ದಿ ಆ ಬಣ್ಣವನ್ನು ಕೊಡುತ್ತಿದ್ದ. ಒಬ್ಬ ಗಿರಾಕಿಗೆ ಇದನ್ನು ನೋಡಿ ಆಶ್ಚರ್ಯವಾಯಿತು. “ಈಗ ನಿಮ್ಮ ಬಟ್ಟೆಗೆ ಯಾವ ಬಣ್ಣ ಹಾಕಲಿ” ಎಂದು ಕೇಳಿದ. ಅದಕ್ಕೆ ಗಿರಾಕಿ “ತಮ್ಮ ಪೀಪಾಯಿಯಲ್ಲಿ ಯಾವ ಬಣ್ಣ ಇದೆಯೊ ಅದನ್ನು ಹಾಕಿ” ಎಂದ.


\section{\num{೧೩೦. } ಭಗವತಿ ನನಗೆ ಏನು ತೋರಿದಳು}

ದೇವರು ನಿರಾಕಾರ ಎನ್ನುವವರು ಎಲ್ಲಿ ಎಡವುತ್ತಾರೆ ಗೊತ್ತೆ? ಅವನು ಬರೀ ನಿರಾಕಾರ ಮಾತ್ರ, ಯಾರು ಅದನ್ನು ವಿರೋಧಿಸುವರೊ ಅವರು ತಪ್ಪು ಎನ್ನುವಲ್ಲಿ. ಆದರೆ ದೇವರು ಸಾಕಾರ ಮತ್ತು ನಿರಾಕಾರ ಎರಡೂ ಆಗಿರುವನು ಎಂಬುದನ್ನು ನಾನು ತಿಳಿದಿದ್ದೇನೆ. ಅಷ್ಟೇ ಅಲ್ಲ, ಅವನು ಇನ್ನೂ ಅನೇಕ ಸಾಧ್ಯತೆಗಳನ್ನು ಹೊಂದಿರುವನು. ಎಲ್ಲವೂ ಅವನಿಗೆ ಸಾಧ್ಯ. ಮಹಾಮಾಯೆ ಚಿತ್​ಶಕ್ತಿಯಂತೆ ಇಪ್ಪತ್ತನಾಲ್ಕು ತತ್ತ್ವಗಳನ್ನು ಧರಿಸಿದ್ದಾಳೆ. ಒಂದು ಸಲ ನಾನು ಧ್ಯಾನ ಮಾಡುತ್ತಿದ್ದಾಗ ಮನಸ್ಸು ರಸಿಕನ ಮನೆಗೆ ಹೋಯಿತು. ಅವನು ಜಾಡಮಾಲಿ. ಆಗ ನನ್ನ ಮನಸ್ಸಿಗೆ, ‘ಎಲೆ ದುಷ್ಟನೇ, ನೀನು ಅಲ್ಲೇ ಇರು’ ಎಂದೆ. ಭಗವತಿ ನನಗೆ, ಆ ಮನೆಯಲ್ಲಿರುವವರೆಲ್ಲ ಬರೀ ಮುಸುಕುಗಳು; ಅವರೊಳಗೆಲ್ಲ ಭಗವತಿಯ ಶಕ್ತಿಯೇ ಇರುವುದು; ಎಂಬುದನ್ನು ತೋರಿಸಿ ದಳು.


\section{\num{೧೩೧. } ದೇವರ ವಿಷಯವನ್ನು ಒಬ್ಬ ಸಾಧು ತಿಳಿದುಕೊಂಡ ಬಗೆ}

ಒಬ್ಬ ಸಾಧು ಪುರಿಯಲ್ಲಿರುವ ಜಗನ್ನಾಥ ಮಂದಿರಕ್ಕೆ ಹೋದ. ದೇವರಿಗೆ ಆಕಾರವಿದೆಯೆ ಇಲ್ಲವೆ ಎಂಬ ಸಂಶಯ ಅವನನ್ನು ಬಾಧಿಸುತ್ತಿತ್ತು. ಪವಿತ್ರ ವಿಗ್ರಹವನ್ನು ನೋಡಿದಾಗ ಅದನ್ನು ಪರೀಕ್ಷೆಮಾಡಿ ಒಂದು ನಿರ್ಣಯಕ್ಕೆ ಬರಬೇಕು ಎಂದುಕೊಂಡಿದ್ದ.

ಅವನು ತನ್ನಲ್ಲಿದ್ದ ದಂಡವನ್ನು ವಿಗ್ರಹದ ಸುತ್ತ ಎಡಗಡೆಯಿಂದ ಬಲ ಗಡೆಗೆ ತೆಗೆದುಕೊಂಡು ಬಂದ. ಅವನ ದಂಡಕ್ಕೆ ಯಾವುದೂ ಸ್ಪರ್ಶವಾಗಲಿಲ್ಲ. ಆಗ ದೇವರು ನಿರಾಕಾರ ಎಂದು ನಿರ್ಧರಿಸಿದ. ಬಲಗಡೆಯಿಂದ ಎಡಗಡೆಗೆ ತೆಗೆದುಕೊಂಡು ಬರುವಾಗ ವಿಗ್ರಹ ವನ್ನು ತಾಕಿತು. ಆಗ ದೇವರು ನಿರಾಕಾರ ಮತ್ತು ಸಾಕಾರ ಎರಡೂ ಆಗಿರುವನು ಎಂದು ತಿಳಿದುಕೊಂಡ.


\section{\num{೧೩೨. } ಭಗವಂತನೇ ಎಲ್ಲವೂ ಆಗಿದ್ದಾನೆ}

ಒಂದು ಸಲ ರಾಮನಿಗೆ ತೀವ್ರ ವೈರಾಗ್ಯ ಬಂತು. ದಶರಥ ಇದರಿಂದ ತುಂಬಾ ಕಳವಳಕ್ಕೆ ಈಡಾದ. ವಸಿಷ್ಠರ ಬಳಿಗೆ ಹೋಗಿ, ರಾಮನು ಸಂಸಾರ ತ್ಯಾಗ ಮಾಡದಂತೆ ಮಾಡಿ ಎಂದು ಕೇಳಿಕೊಂಡ. ಗುರುಗಳು ರಾಮನ ಬಳಿಗೆ ಬಂದರು. ರಾಮ ತುಂಬಾ ವ್ಯಾಕುಲನಾಗಿದ್ದ. ರಾಮನ ಮನಸ್ಸು ಪರಮ ವೈರಾಗ್ಯದಿಂದ ತುಂಬಿತ್ತು. ವಸಿಷ್ಠರು ಹೇಳಿದರು, “ರಾಮ, ನೀನು ಏತಕ್ಕೆ ಪ್ರಪಂಚವನ್ನು ಬಿಡಬೇಕು ಎಂದು ಮನಸ್ಸು ಮಾಡಿರುವೆ? ಪ್ರಪಂಚ ದೇವ ರಿಂದ ಹೊರಗೆ ಇದೆಯೆ, ಇದನ್ನು ವಿಚಾರ ಮಾಡು” ಎಂದು. ಆಗ ರಾಮನಿಗೆ ಗೊತ್ತಾಯಿತು, ಈ ಪ್ರಪಂಚ ಪರಬ್ರಹ್ಮನಿಂದ ಬಂದಿದೆ ಎಂದು. ಆಮೇಲೆ ಏನನ್ನೂ ಹೇಳಲಿಲ್ಲ.


\section{\num{೧೩೩. } ನಾಮರೂಪಗಳಾಚೆ ನೋಡಿ}

ಒಂದು ಸಲ ಸಾಧುವೊಬ್ಬನು ಒಂದು ಸುಂದರವಾದ ತೋಟದಲ್ಲಿ ತನ್ನ ಶಿಷ್ಯನನ್ನಿರಿಸಿ ಅವನಿಗೆ ಆತ್ಮಜ್ಞಾನವನ್ನು ಬೋಧಿಸಬೇಕೆಂದಿದ್ದನು. ಕೆಲವು ದಿನ ಗಳ ಮೇಲೆ ಹಿಂತಿರುಗಿ, “ನಿನಗೆ ಏನಾದರೂ ಕೊರತೆ ಇದೆಯೆ?” ಎಂದು ಕೇಳಿದ. ಹೌದು ಕೆಲವು ಕೊರತೆಗಳಿವೆ ಎಂಬುದನ್ನು ಅರಿತು ಶ್ಯಾಮಾ ಎಂಬ ಸುಂದರವಾದ ತರುಣಿಯನ್ನಲ್ಲಿಟ್ಟು, “ಎಷ್ಟು ಬೇಕೋ ಅಷ್ಟು ಮೀನು ಮಾಂಸ ವನ್ನು ತಿನ್ನು” ಎಂದು ಹೇಳಿದ. ಕೆಲವು ದಿನಗಳಾದ ಮೇಲೆ ಸಾಧು ಪುನಃ ಬಂದು “ಏನಾದರೂ ಕೊರತೆ ಇದೆಯೇ?” ಎಂದು ಕೇಳಿದ. ಈಗ ಶಿಷ್ಯ, “ಇಲ್ಲ ನನಗೇನೂ ಕೊರತೆಯಿಲ್ಲ. ನಿಮಗೆ ಧನ್ಯವಾದಗಳು” ಎಂದ. ಸಾಧು, ಶಿಷ್ಯ ಮತ್ತು ಶ್ಯಾಮಳನ್ನು ಕರೆದು, ಅವಳ ಕೈಗಳನ್ನು ತೋರಿಸಿ, “ಇದು ಏನು ಎಂದು ಹೇಳ ಬಲ್ಲೆಯಾ” ಎಂದು ಶಿಷ್ಯನನ್ನು ಕೇಳಿದನು. “ಏತಕ್ಕೆ, ಇದು ಶ್ಯಾಮಳ ಕೈಗಳು” ಎಂದ. ಅವನು ಶಿಷ್ಯನಿಗೆ ಅದೇ ರೀತಿ ಪ್ರಶ್ನೆಗಳನ್ನು ಹಲವು ವೇಳೆ ಕೇಳಿ ಶ್ಯಾಮಳ ಕಣ್ಣುಗಳು, ಮೂಗು ಮತ್ತು ದೇಹದ ಇತರ ಭಾಗಗಳನ್ನು ತೋರಿಸಿದನು. ಅದಕ್ಕೆಲ್ಲ ಶಿಷ್ಯ ಸರಿಯಾದ ಉತ್ತರವನ್ನು ಕೊಟ್ಟನು. ಅಷ್ಟರಲ್ಲಿ ಶಿಷ್ಯನಿಗೆ ಹೊಳೆಯಿತು, “ನಾನು ಶ್ಯಾಮಳ ಅದು, ಶ್ಯಾಮಳ ಇದು ಎಂದು ಹೇಳುತ್ತಿರುವೆ. ಈ ಶ್ಯಾಮ ಯಾರು?” ಅವನು ಆಶ್ಚರ್ಯದಿಂದ ಗುರುವನ್ನು, “ಈ ಶ್ಯಾಮಳು ಯಾರು? ಈ ಕಣ್ಣು ಗಳು, ಕಿವಿಗಳು ಮತ್ತು ಇತರ ಭಾಗಗಳನ್ನು ಹೊಂದಿರುವ ಅವಳು ಯಾರು?” ಎಂದು ಕೇಳಿದ. “ಈ ಶ್ಯಾಮಳು ಯಾರು ಎಂಬುದನ್ನು ನೀನು ತಿಳಿದು ಕೊಳ್ಳಬೇಕಾದರೆ ನನ್ನೊಡನೆ ಬಾ, ನಾನು ನಿನಗೆ ಹೇಳುತ್ತೇನೆ” ಎಂದ ಗುರು. ಹೀಗೆ ಹೇಳಿ ಸತ್ಯವನ್ನು ಅವನಿಗೆ ವಿವರಿಸಿದ.


\section{\num{೧೩೪. } ಅಂತಹ ಜನರು ಅತ್ಯಲ್ಪ}

ಒಬ್ಬ ಶ್ರೀಮಂತ ತನ್ನ ಆಳಿಗೆ ಒಂದು ವಜ್ರವನ್ನು ಕೊಟ್ಟು, “ಸಂತೆಗೆ ಈ ವಜ್ರವನ್ನು ತೆಗೆದು ಕೊಂಡು ಹೋಗು. ಯಾರು ಯಾರು ಎಷ್ಟು ಬೆಲೆ ಕಟ್ಟುತ್ತಾರೆ ಎಂಬುದನ್ನು ವಿಚಾರಿಸು. ಮೊದಲು ಬದನೆಕಾಯಿ ಮಾರು ವವನ ಹತ್ತಿರ ಹೋಗು” ಎಂದನು. ಆಳು ಬದನೆಕಾಯಿ ಮಾರುವವನ ಹತ್ತಿರ ಹೋದ. ಅವನು ಅದನ್ನು ತನ್ನ ಕೈಯಲ್ಲಿ ತೆಗೆದುಕೊಂಡು ನೋಡಿ, “ಇದಕ್ಕೆ ನಾನು ಒಂಭತ್ತು ಸೇರು ಬದನೆ ಕಾಯಿಯನ್ನು ಕೊಡುತ್ತೇನೆ,

ಪ್ರತಿಯೊಬ್ಬನೂ ತನ್ನ ಯೋಗ್ಯತೆಗೆ ತಕ್ಕ ಬೆಲೆಯನ್ನು ಹೇಳುವನು. ಎಲ್ಲ ರಿಗೂ ಅಖಂಡ ಸಚ್ಚಿದಾನಂದ ಅರ್ಥವಾಗಬಲ್ಲದೆ? ಹನ್ನೆರಡು ಜನ ಪುಷಿ ಗಳು ಮಾತ್ರ ರಾಮಚಂದ್ರನನ್ನು ತಿಳಿದುಕೊಂಡರು. ಎಲ್ಲರೂ ಅವತಾರವನ್ನು ತಿಳಿದುಕೊಳ್ಳಲಾರರು. ಕೆಲವರು ಅವನು ಸಾಧಾರಣ ಮನುಷ್ಯ ಎಂದು ಭಾವಿಸಿದರು. ಮತ್ತೆ ಕೆಲವರು ಅವನು ಒಳ್ಳೆಯ ಸಾಧು ಎಂದು ಭಾವಿಸಿದರು. ಎಲ್ಲೋ ಕೆಲವರು ಅವನು ಅವತಾರವೆಂದು ಕಂಡುಹಿಡಿದರು.


\section{\num{೧೩೫. } ಅವಳು ಬಂದು ಹೋದಳು}

ಕಾಮಾರಪುಕುರದ ಹತ್ತಿರ ಇರುವ ದಾರಿಯಲ್ಲಿ ರಣಜಿತ್ ರಾಯನ ಸರೋವರ ಇದೆ. ಜಗನ್ಮಾತೆಯಾದ ಭಗವತಿಯು ಅವನ ಮಗಳಾಗಿ ಜನಿಸಿ ದಳು. ಈಗಲೂ ಕೂಡ ಚೈತ್ರಮಾಸದಲ್ಲಿ ಈ ದಿವ್ಯಪುತ್ರಿಯ ಹೆಸರಿನಲ್ಲಿ ದೊಡ್ಡ ಜಾತ್ರೆಯನ್ನು ಮಾಡುತ್ತಾರೆ.

ರಣಜಿತ್​ರಾಯ್ ಆ ಭಾಗಕ್ಕೆ ಸೇರಿದ ಜಮೀಂದಾರ. ತನ್ನ ತಪಶ್ಶಕ್ತಿಯಿಂದ ಭಗವತಿಯನ್ನು ತನ್ನ ಮಗಳಂತೆ ಪಡೆದ. ಅವಳನ್ನು ಕಂಡರೆ ಇವನಿಗೆ ತುಂಬಾ ಪ್ರೀತಿ. ಮಗಳಿಗೂ ತಂದೆಯನ್ನು ಕಂಡರೆ ಬಹಳ ಅಕ್ಕರೆ, ತಂದೆಯನ್ನು ಬಿಟ್ಟಿರಲಾರಳು. ಒಂದು ಸಲ ರಣಜಿತ್​ರಾಯ್ ಜಮೀನಿಗೆ ಸಂಬಂಧಪಟ್ಟ ವಿಚಾರದಲ್ಲಿ ಮಗ್ನನಾಗಿದ್ದ. ಅವನಿಗೆ ಪುರುಸತ್ತೇ ಇರಲಿಲ್ಲ. ಮಗಳು ತನ್ನ ಸ್ವಭಾವಕ್ಕೆ ತಕ್ಕಂತೆ ತಂದೆಯನ್ನು ಇದು ಏನು ಅದು ಏನು ಎಂದು ಕೇಳು ತ್ತಿದ್ದಳು. ರಣಜಿತ್​ರಾಯ್ ಒಳ್ಳೆಯ ಮಾತಿನಲ್ಲಿ “ನನ್ನ ಹತ್ತಿರ ಮಾತನಾಡ ಬೇಡ ಈಗ. ಮಗು, ನನಗೆ ಸ್ವಲ್ಪ ವಿರಾಮ ಕೊಡು. ನನಗೆ ಬೇಕಾದಷ್ಟು ಕೆಲಸ ಇದೆ.” ಎಂದನು. ಆದರೆ ಮಗು ಸುಮ್ಮನೆ ಇರಲಿಲ್ಲ. ಕೊನೆಗೆ ತಂದೆಯು ಯೋಚಿಸದೆ “ಇಲ್ಲಿಂದ ಬಿಟ್ಟು ಹೊರಡು” ಎಂದ. ಇದೇ ನೆಪಮಾಡಿಕೊಂಡು ಮಗಳು ಮನೆಬಿಟ್ಟು ಹೋದಳು. ದಾರಿಯಲ್ಲಿ ಕವಡೆ ಬಳೆಯ ವ್ಯಾಪಾರಿ ಒಬ್ಬ ಸಿಕ್ಕಿದ. ಅವನಿಂದ ಒಂದು ಜತೆ ಬಳೆಯನ್ನು ತೆಗೆದುಕೊಂಡಳು. ಹಣವನ್ನು ಕೇಳಿದಾಗ ಹಣ ತನ್ನ ಮನೆಯಲ್ಲಿರುವ ಪೆಟ್ಟಿಗೆಯಲ್ಲಿದೆ ಎಂದಳು. ಅನಂತರ ಅವಳು ಮಾಯವಾದಳು. ಅವಳು ಪುನಃ ಯಾರಿಗೂ ಕಾಣಲಿಲ್ಲ. ಅಷ್ಟು ಹೊತ್ತಿಗೆ ವ್ಯಾಪಾರಿ ತನ್ನ ಬಳೆಯ ಹಣವನ್ನು ಅವಳ ಮನೆಗೆ ಹೋಗಿ ಕೇಳಿದ. ಮಗು ಕಾಣದೆ ಇದ್ದಾಗ ನಂಟರು ಅವಳಿಗಾಗಿ ಹುಡುಕಾಡಿದರು. ರಣಜಿತ್ ರಾಯ್ ಎಲ್ಲಾ ಕಡೆಗೂ ಜನರನ್ನು ಕಳುಹಿಸಿದ, ಆ ಮಗುವನ್ನು ಹುಡುಕುವುದಕ್ಕೆ. ವ್ಯಾಪಾರಿಗೆ ಕೊಡಬೇಕಾಗಿದ್ದ ಹಣ ಆ ಮಗು ಹೇಳಿದಂತೆ ಅವಳ ಪೆಟ್ಟಿಗೆಯಲ್ಲಿತ್ತು. ರಣಜಿತ್​ರಾಯ್ ತುಂಬಾ ವ್ಯಸನಪಟ್ಟ. ಜನರು ಓಡಿಬಂದು, “ಸರೋವರದಲ್ಲಿ ಏನನ್ನೋ ನೋಡಿದೆವು” ಎಂದರು. ಎಲ್ಲರೂ ಅಲ್ಲಿಗೆ ಹೋಗಿ ನೋಡಿದಾಗ, ಕವಡೆ ಬಳೆಯಿಂದ ಕೂಡಿದ ಕೈಯೊಂದು ನೀರಿನ ಮೇಲೆ ಕಾಣಿಸುತ್ತಿತ್ತು. ಕೆಲವು ಕ್ಷಣಗಳಲ್ಲಿ ಅದೂ ಮಾಯವಾಯಿತು. ಈಗಲೂ ಕೂಡ ಈ ಸಮಯದಲ್ಲಿ ಭಗವತಿ ಹೆಸರಿನಲ್ಲಿ ಹಬ್ಬವನ್ನು ಆಚರಿಸುತ್ತಾರೆ.

ತಪೋಬಲದಿಂದ ಭಗವಂತನನ್ನು ಮಗನಂತೆ ಪಡೆಯಬಹುದು. ಭಗವಂತ ಹಲವು ರೀತಿ ಕಾಣಿಸಿಕೊಳ್ಳುತ್ತಾನೆ. ಕೆಲವು ವೇಳೆ ಮನುಷ್ಯನಂತೆ, ಬೇರೆ ಬೇರೆ ದೇವತೆಗಳಂತೆ, ಶಕ್ತಿಯಂತೆ ಕಾಣಿಸಿಕೊಳ್ಳುತ್ತಾನೆ.


\section{\num{೧೩೬. } ಅರ್ಜುನ ನೋಡಿದ್ದು ಹೀಗೆ}

ಜ್ಞಾನಿಯ ದೃಷ್ಟಿಯಿಂದ ಅವತಾರಕ್ಕೆ ಎಡೆಯೇ ಇಲ್ಲ. ಶ್ರೀಕೃಷ್ಣ ಅರ್ಜುನ ನಿಗೆ, “ನೀನು ನನ್ನನ್ನು ಅವತಾರ ಎನ್ನುತ್ತೀಯೆ. ನನ್ನೊಡನೆ ಬಾ. ನಾನು ನಿನಗೆ ಒಂದು ದೃಶ್ಯವನ್ನು ತೋರಿಸುತ್ತೇನೆ” ಎಂದ. ಅರ್ಜುನ ಶ್ರೀಕೃಷ್ಣನನ್ನು ಹಿಂಬಾ ಲಿಸಿದ. ಸ್ವಲ್ಪ ದೂರ ಹೋದನಂತರ ಶ್ರೀಕೃಷ್ಣ ಅರ್ಜುನನ್ನು ಕುರಿತು, “ನೀನು ಅಲ್ಲಿ ಏನನ್ನು ನೋಡುತ್ತಿರುವೆ” ಎಂದು ಕೇಳಿದ. ಅರ್ಜುನ ಹೇಳಿದ, “ಮರದ ತುಂಬ ನೇರಳೆ ಹಣ್ಣುಗಳ ಜೊಂಪೆಗಳು ಇವೆ.” ಆಗ ಶ್ರೀಕೃಷ್ಣ, “ಅವು ನೇರಳೆ ಹಣ್ಣುಗಳಲ್ಲ, ಇನ್ನೂ ಹತ್ತಿರ ಹೋಗಿ ನೋಡು” ಎಂದ. ಮುಂದೆ ಹೋಗಿ ಅರ್ಜುನ ನೋಡುತ್ತಾನೆ, ಅವರೆಲ್ಲ ಕೊಂಬೆಯನ್ನು ಹಿಡಿದುಕೊಂಡು ನೇತಾಡು ತ್ತಿದ್ದ ಕೃಷ್ಣರಾಗಿದ್ದರು. “ಈಗ ನಿನಗೆ ಗೊತ್ತಾಯಿತೊ, ನನ್ನಂತೆ ಎಷ್ಟೊಂದು ಜನ ಕೃಷ್ಣರು ಜೋಲಾಡುತ್ತಿರುವರು ಎಂದು”–ಶ್ರೀಕೃಷ್ಣನು ಹೇಳಿದ.


\section{\num{೧೩೭. } ಅವನಿಗೆ ಅಸಾಧ್ಯವಾದುದು ಯಾವುದೂ ಇಲ್ಲ}

ದೇವರ ಸಂಬಂಧವಾಗಿ ಮಾತನಾಡುತ್ತಿದ್ದಾಗ, ಮಥುರನು ಹೇಳಿದ: “ದೇವರೂ ಕೂಡ ತಾನು ಮಾಡಿರುವ ನಿಯಮಕ್ಕೆ ದಾಸ. ಅದನ್ನು ಅವನು ಮೀರಿಹೋಗಲು ಆಗುವುದಿಲ್ಲ.” “ಇದೆಂತಹ ಹಾಸ್ಯಾಸ್ಪದ! ಯಾರು ಒಂದು ನಿಯಮವನ್ನು ಮಾಡಿರುವರೊ ಅವನು ತನ್ನ ಮತ್ತೊಂದು ನಿಯಮದಿಂದ ಅದನ್ನು ಅಲ್ಲಗಳೆಯಲು ಸಾಧ್ಯ.ಎಲ್ಲಾ ಅವನ ಇಚ್ಛೆಯ ಮೇಲೆ ಇದೆ,” ಎಂದು ನಾನು ಹೇಳಿದೆ. ಮಥುರನಿಗೆ ಇದರಿಂದ ಸಮಾಧಾನವಾಗ ಲಿಲ್ಲ. “ಕೆಂಪು ಹೂವು ಬಿಡುವ ಗಿಡ ಬೇರೆ ಬಣ್ಣದ ಹೂವನ್ನು ಬಿಡಲಾರದು” ಎಂದನು. “ಅವನಿಚ್ಛೆ ಇದ್ದರೆ ಸಾಧ್ಯ” ಎಂದೆ ನಾನು. ಅವನು ಒಪ್ಪಲಿಲ್ಲ. ಮಾರನೆ ದಿನ ತೋಟದಲ್ಲಿ ನಾನು ಅಡ್ಡಾಡುತ್ತಿದ್ದೆ. ಅಲ್ಲಿ ಒಂದು ದಾಸ ವಾಳದ ಹೂವನ್ನು ನೋಡಿದೆ. ಅಲ್ಲಿ ಒಂದೇ ಕೊಂಬೆಯಲ್ಲಿ ಎರಡು ಹೂವುಗಳು ಇದ್ದವು. ಒಂದು ಕೆಂಪಿತ್ತು, ಮತ್ತೊಂದು ಹಿಮದಂತೆ ಬೆಳ್ಳಗಿತ್ತು. ಆ ಕೊಂಬೆಯನ್ನು ಮುರಿದು ಮಥುರನಿಗೆ ತೋರಿಸಿದೆ. ಅದನ್ನು ನೋಡಿ ಅವನಿಗೆ ಬಹಳ ಆಶ್ಚರ್ಯವಾಯಿತು. “ತಂದೆ, ನಾನು ಇನ್ನೊಮ್ಮೆ ನಿಮ್ಮೊಂದಿಗೆ ವಾದ ಮಾಡುವುದಿಲ್ಲ” ಎಂದ.



\section{\num{೧೩೮. } ಅವನಿಗೆ ಇದೆಲ್ಲ ಧೂಳಿಸಮಾನ}

ಒಂದು ಸಲ ಕಳ್ಳ ವಿಷ್ಣುವಿನ ದೇವಸ್ಥಾನವನ್ನು ಹೊಕ್ಕು ದೇವರ ಮೇಲಿದ್ದ ಒಡವೆಗಳನ್ನೆಲ್ಲ ಕದ್ದ. ಮಥುರಬಾಬು ಮತ್ತು ನಾನು ಇಬ್ಬರೂ ಏನು ನಡೆದಿದೆ ಎಂದು ನೋಡುವುದಕ್ಕೆ ಹೋದೆವು.

ವಿಗ್ರಹವನ್ನು ನೋಡಿ ಮಥುರ ತುಂಬಾ ಅತೃಪ್ತನಾಗಿ “ಎಂತಹ ನಾಚಿಕೆ ಗೇಡು! ದೇವರೆ, ನೀನು ಕೆಲಸಕ್ಕೆ ಬಾರದವನು. ಕಳ್ಳ ನಿನ್ನ ಮೇಲಿದ್ದ ಆಭರಣ ಗಳನ್ನೆಲ್ಲ ಕದ್ದ. ಆದರೂ ನಿನ್ನ ಕೈಯಲ್ಲಿ ಏನನ್ನೂ ಮಾಡಲಾಗಲಿಲ್ಲ!” ಆಗ ನಾನು ಮಥುರನಾಥನಿಗೆ ಹೇಳಿದೆ, “ನಿನಗೆ ನಾಚಿಕೆಯಾಗುವುದಿಲ್ಲವೆ? ನಿನ್ನ ಮಾತು ಎಷ್ಟು ಅಸಂಬದ್ಧ! ದೇವರಿಗೆ ನೀನು ಹೇರಿರುವ ನಗ ನಾಣ್ಯಗಳೆಲ್ಲ ಮಣ್ಣಿಗೆ ಸಮಾನ. ಐಶ್ವರ್ಯಕ್ಕೆ ಅಧಿದೇವತೆಯಾಗಿ ರುವ ಲಕ್ಷ್ಮಿಯೇ ಅವನ ಹೆಂಡತಿಯಾಗಿರುವಾಗ, ಯಾರೋ ಒಬ್ಬ ಕಳ್ಳ ಬಂದು ನಗನಾಣ್ಯಗಳನ್ನು ತೆಗೆದುಕೊಂಡು ಹೋಗಿಬಿಟ್ಟಿರು ವನೆಂದು ಅವನೇನು ನಿದ್ರೆಗೆಡುವನೇನು? ದೇವರಿಗೆ ಇದೆಲ್ಲ ಮಣ್ಣಿನ ಸಮಾನ. ನೀನು ಅಂತಹ ಮಾತನ್ನು ಆಡಬಾರದು.”


\section{\num{೧೩೯. } ಭಗವಂತನ ಸ್ವಭಾವ}

ಭಗವಂತನದು ಮಗುವಿನಂಥ ಸ್ವಭಾವ. ಮಗು ತನ್ನ ಬಟ್ಟೆಯಲ್ಲಿ ಮುತ್ತು ಗಳನ್ನು ಇಟ್ಟುಕೊಂಡಿದೆ. ಆ ಮಾರ್ಗದ ಮೂಲಕ ಹಲವರು ಹೋಗುತ್ತಾರೆ. ಅನೇಕರು ಆ ಮುತ್ತನ್ನು ಕೇಳುತ್ತಾರೆ. ಆದರೆ ತನ್ನ ಕೈಯಿಂದ ಅದನ್ನು ಮುಚ್ಚಿ ಕೊಂಡು ಅವರ ಕಡೆ ಕೂಡ ನೋಡುವುದಿಲ್ಲ. “ಇಲ್ಲ ನಾನು ಇವುಗಳನ್ನು ಕೊಡುವುದಿಲ್ಲ” ಎನ್ನುವುದು. ಮತ್ತೊಬ್ಬ ಬರುತ್ತಾನೆ. ಅವನು ಮುತ್ತನ್ನು ಕೇಳುವುದಿಲ್ಲ. ಆದರೂ ಮಗು ಅವನ ಹಿಂದೆ ಹೋಗಿ ಮುತ್ತುಗಳನ್ನು ಕೊಟ್ಟು “ದಯವಿಟ್ಟು ಇದನ್ನು ತೆಗೆದುಕೊಳ್ಳಿ” ಎನ್ನುವುದು.


\section{\num{೧೪೦. } ದೇವರು ಭಕ್ತಪರಾಧೀನ}

ಕಾಳಿ ದೇವಸ್ಥಾನದ ಪ್ರಾಂಗಣದಲ್ಲಿ ನನಗೆ ಕೆಲವು ಸಿಖ್ಖರು, “ಭಗವಂತ ದಯಾಮಯ” ಎಂದರು. ನಾನು, “ಅವನು ಯಾರಿಗೆ ದಯೆ ತೋರಿಸುತ್ತಾನೆ?” ಎಂದೆ. “ಮಹಾಶಯರೆ, ನಮ್ಮ ಮೇಲೆಲ್ಲ” ಎಂದರವರು. ಅದಕ್ಕೆ ನಾನು, “ನಾವು ಅವನ ಮಕ್ಕಳು. ತನ್ನ ಮಗುವಿಗೆ ಕರುಣೆಯನ್ನು ತೋರುವುದು ಏನು ಒಂದು ದೊಡ್ಡ ವಿಷಯವೇ? ತಂದೆ ತನ್ನ ಮಕ್ಕಳನ್ನು ನೋಡಿಕೊಳ್ಳಲೇಬೇಕು. ಇಲ್ಲದೆ ಇದ್ದರೆ ನರೆಹೊರೆಯವರು ಅದನ್ನು ಮಾಡುತ್ತಾರೆಯೆ? ದೇವರು ದಯಾಮಯ ಎಂದು ಹೇಳುವವರಿಗೆ, ನಾವು ಅವನ ಮಕ್ಕಳು, ಇನ್ನಾರದೋ ಅಲ್ಲ ಎನ್ನುವುದು ಗೊತ್ತಿಲ್ಲವೆ?”

ಹಾಗಾದರೆ ನಾವು ದೇವರನ್ನು ದಯಾಮಯ ಎಂದು ಹೇಳಕೂಡದೆ? ಎಲ್ಲಿಯವರೆಗೆ ಸಾಧನಾವಸ್ಥೆಯಲ್ಲಿ ಇರುವೆವೊ ಅಲ್ಲಿಯವರೆಗೆ ಹಾಗೆ ಹೇಳ ಬಹುದು. ಆದರೆ ಭಗವಂತನ ಸಾಕ್ಷಾತ್ಕಾರವಾದರೆ ದೇವರೇ ನಮ್ಮ ತಂದೆ ತಾಯಿ ಎಂಬುದನ್ನು ನಿಶ್ಚಿತವಾಗಿ ಅರಿಯುವೆವು. ನಾವು ಅವನನ್ನು ಇನ್ನೂ ತಿಳಿದುಕೊಳ್ಳದಿರುವಾಗ ಅವನು ನಮ್ಮಿಂದ ಬಹಳ ದೂರದಲ್ಲಿರುವನು, ನಾವು ಮತ್ತಾರಿಗೋ ಸೇರಿದವರು ಎಂದು ಭಾವಿಸುತ್ತೇವೆ.

ಸಾಧನಾ ಸಮಯದಲ್ಲಿ ಭಗವಂತನನ್ನು ಅವನ ಎಲ್ಲಾ ಗುಣಗಳಿಂದಲೂ ಹೊಗಳಬೇಕು. ಒಂದು ದಿನ ಹಾಜರಾ ನರೇಂದ್ರನಿಗೆ ಹೇಳಿದ, “ಭಗವಂತ ಅನಂತ. ಅವನ ಮಹಿಮೆ ಅನಂತವಾದದ್ದು. ನೀನು ಕೊಡುವ ಸಿಹಿತಿಂಡಿ, ತೀರ್ಥ, ಮಿಠಾಯಿ ಇವುಗಳನ್ನು ಅವನು ಸ್ವೀಕರಿಸುವನೇನು? ನಿನ್ನ ಹಾಡು ಕೇಳುವನೇನು? ಇವೆಲ್ಲ ನಿನ್ನ ತಪ್ಪು ಕಲ್ಪನೆ.” ಅದನ್ನು ಕೇಳಿದ ಮೇಲೆ ನರೇಂದ್ರ ಹತ್ತು ಮಾರು ಕೆಳಗೆ ಬಿದ್ದ. ಆಗ ನಾನು ಹಾಜರಾನಿಗೆ, “ಏ ತಿಳಿಗೇಡಿ, ನೀನು ಹೀಗೆ ಅವರೊಡನೆ ಮಾತನಾಡಿದರೆ ಯುವಕರ ಗತಿ ಏನು?” ಎಂದೆ. ಭಕ್ತಿ ಯಿಲ್ಲದೆ ಜೀವನ ಸಾಧ್ಯವೆ? ಭಗವಂತನ ಮಹಿಮೆಗೆ ಪಾರವಿಲ್ಲ. ಆದರೂ ಅವನು ಭಕ್ತಪರಾಧೀನ. ಶ್ರೀಮಂತನ ಒಬ್ಬ ಆಳು ಬಂದು ಪಡಸಾಲೆಯಲ್ಲಿ ಒಂದು ಮೂಲೆಯಲ್ಲಿ ನಿಲ್ಲುವನು. ಅವನು ಕೈಯಲ್ಲಿರುವ ಬಟ್ಟೆಯಲ್ಲಿ ಏನನ್ನೊ ಮುಚ್ಚಿಕೊಂಡು ಬಂದಿರುವನು. ಮನೆಯ ಯಜಮಾನ ಬಂದು, “ನಿನ್ನ ಕೈಯಲ್ಲಿರುವುದು ಏನು?” ಎಂದು ಕೇಳುತ್ತಾನೆ. ಆಳು ಬಹಳ ಸಂಕೋಚದಿಂದ ಬಟ್ಟೆಯಿಂದ ಒಂದು ಸೀತಾಫಲದ ಹಣ್ಣನ್ನು ಅವನಿಗೆ ನೀಡಿ, “ಮಹಾಸ್ವಾಮಿಗಳು, ತಾವು ಇದರ ರುಚಿ ನೋಡಬೇಕು” ಎಂದು ಕೇಳಿಕೊಳ್ಳು ತ್ತಾನೆ. ಮನೆಯ ಯಜಮಾನನಿಗೆ ಅವನ ಭಕ್ತಿಯನ್ನು ನೋಡಿ ತುಂಬಾ ಮೆಚ್ಚಿಗೆಯಾಯಿತು. ಅವನು ಪ್ರೀತಿಯಿಂದ ಅದನ್ನು ತೆಗೆದುಕೊಂಡು, “ಇದು ಬಹಳ ಒಳ್ಳೆಯ ಹಣ್ಣು. ನಿನಗೆ ಇದು ಎಲ್ಲಿ ಸಿಕ್ಕಿತು? ಅದನ್ನು ಹುಡುಕಿ ತರುವುದಕ್ಕೆ ನೀನು ತುಂಬಾ ಕಷ್ಟ ಪಟ್ಟಿರಬೇಕು” ಎಂದು ಕಳಕಳಿ ವ್ಯಕ್ತ ಪಡಿಸುತ್ತಾನೆ.

ಭಗವಂತ ಭಕ್ತಪರಾಧೀನ. ದುರ್ಯೋಧನ ಶ್ರೀಕೃಷ್ಣನನ್ನು ತುಂಬಾ ಗೌರವಿಸುತ್ತಿದ್ದ. “ದಯವಿಟ್ಟು ನಮ್ಮ ಮನೆಯಲ್ಲಿ ಊಟಮಾಡಿ” ಎಂದ. ಆದರೆ ದೇವರು ಬಡ ವಿದುರನ ಗುಡಿಸಲಿಗೆ ಹೋದ. ಅವನ ಪ್ರೀತಿ ಶ್ರೀ ಕೃಷ್ಣನಿಗೆ ತುಂಬಾ ಮೆಚ್ಚಿಗೆಯಾಯಿತು. ವಿದುರ ನೀಡಿದ ಸಾಧಾ ರಣ ಊಟವನ್ನು ಸ್ವೀಕರಿಸಿ ಸಂತೃಪ್ತನಾದ.


\section{\num{೧೪೧. } ಉಳಿದೆಲ್ಲವೂ ಮಿಥ್ಯ}

ದೇವನೊಬ್ಬನೇ ನಿಜ. ಉಳಿದಿರುವುದೆಲ್ಲ ಸುಳ್ಳು. ಮನೆ, ಮನುಷ್ಯರು, ಪ್ರಪಂಚ, ಮಕ್ಕಳು, ಇವುಗಳೆಲ್ಲ ಮಾಟಗಾರನ ಮಾಯೆಯಂತೆ. ಮಂತ್ರಗಾರ ತನ್ನ ಮಂತ್ರದಂಡವನ್ನು ಆಡಿಸಿ, “ಬಾ ಮಾಯೆ, ಬಾ ಭ್ರಾಂತಿಯೇ” ಎನ್ನು ತ್ತಾನೆ. ಅನಂತರ ಪ್ರೇಕ್ಷಕರಿಗೆ, “ಈ ಮುಚ್ಚಳವನ್ನು ತೆಗೆದುನೋಡಿ. ಅಲ್ಲಿಂದ ಹಕ್ಕಿಗಳು ಹಾರುವುವು” ಎನ್ನುತ್ತಾನೆ. ಆದರೆ ಮಂತ್ರವಾದಿ ಒಬ್ಬನೇ ಸತ್ಯ. ಅವನ ಮಾಯೆ ಸುಳ್ಳು. ಅಸತ್ಯದ ಅಸ್ತಿತ್ವ ಕೆಲವು ಕ್ಷಣ ಮಾತ್ರ. ಅನಂತರ ಮಾಯವಾಗುವುದು.

ಶಿವ ಕೈಲಾಸದಲ್ಲಿ ಆಸೀನನಾಗಿದ್ದ. ಅವನ ಜೊತೆಗಾರ ನಂದಿ ಬಳಿ ಯಲ್ಲಿಯೇ ಇದ್ದ. ಆಗ ಒಂದು ದೊಡ್ಡ ಶಬ್ದ ಕೇಳಿಬಂತು. “ಇದೇನು ಶಬ್ದ?” ಎಂದು ನಂದಿ ಕೇಳಿದ. ಶಿವ ಹೇಳಿದ, “ರಾವಣ ಜನ್ಮ ತಾಳಿರುವನು. ಅದಕ್ಕೆ ಈ ಶಬ್ದ.” ಕೆಲಕ್ಷಣ ನಂತರ ಮತ್ತೊಂದು ಆರ್ಭಟ ಕೇಳಿಸಿತು. “ಇದೇನು?” ಎಂದು ನಂದಿ ಕೇಳಿದ. ಶಿವ ನಸುನಗುತ್ತ, “ರಾವಣ ತೀರಿಹೋದ” ಎಂದ.

ಜನನ ಮರಣಗಳು ಮಾಯೆಯಂತೆ. ಒಂದು ಕ್ಷಣ ಮಾಯೆಯನ್ನು ನೋಡುವೆ. ಅನಂತರ ಅದು ಮಾಯವಾಗುತ್ತದೆ. ದೇವರೊಬ್ಬನೇ ಸತ್ಯ, ಉಳಿದುದೆಲ್ಲ ಮಿಥ್ಯ. ನೀರೊಂದೆ ಸತ್ಯ, ಅದರ ಮೇಲಿರುವ ಗುಳ್ಳೆಗಳು ಬರು ತ್ತವೆ ಹೋಗುತ್ತವೆ. ಅವು ಎಲ್ಲಿಂದ ಬಂದವೋ ಆ ನೀರಿನಲ್ಲೇ ಲಯ ವಾಗುತ್ತವೆ.


\section{\num{೧೪೨. } ಭಗವಂತನ ಲೀಲೆಯ ಆಕರ್ಷಣೆ}

ರಾಮನಿಂದ ರಾವಣ ಹತನಾದ ಮೇಲೆ ರಾವಣನ ತಾಯಿಯಾದ ನಿಕಷೆ ತನಗೆ ಏನಾಗುವುದೊ ಎಂಬ ಭೀತಿಯಿಂದ ಓಡಿ ಹೋಗುತ್ತಿದ್ದಳು. ಲಕ್ಷ್ಮಣ ರಾಮನನ್ನು ಕೇಳಿದ, “ಅಣ್ಣಯ್ಯ, ದಯವಿಟ್ಟು ನನಗೆ ಈ ವಿಚಿತ್ರವನ್ನು ವಿವರಿಸು. ಈಕೆ ಮುದುಕಿಯಾಗಿದ್ದಾಳೆ. ತನ್ನ ಮಕ್ಕಳೆಲ್ಲ ಹತರಾಗಿದ್ದಾರೆ. ಆದರೂ ತಾನೆಲ್ಲಿ ಸಾಯುವೆನೋ ಎಂದು ಪ್ರಾಣವನ್ನು ಉಳಿಸಿಕೊಳ್ಳುವುದಕ್ಕೆ ಓಡಿಹೋಗುತ್ತಿರುವಳು!” ರಾಮನು ಆಕೆಯನ್ನು ಹತ್ತಿರ ಕರೆದು, ಅವಳಿಗೆ ಭರವಸೆಯನ್ನು ನೀಡಿ, ಓಡು



\section{\num{೧೪೩. } ಅಫೀಮನ್ನು ರುಚಿ ನೋಡಿದ ನವಿಲು}

ಒಂದು ನವಿಲಿಗೆ ಒಬ್ಬ ಮಧ್ಯಾಹ್ನ ನಾಲ್ಕು ಗಂಟೆಗೆ ಒಂದು ಗುಳಿಗೆ ಅಫೀಮನ್ನು ಕೊಟ್ಟ. ಮಾರನೆ ದಿನ ಅದೇ ಹೊತ್ತಿಗೆ ನವಿಲು ಹಾಜರಾಗಿತ್ತು. ಅದಕ್ಕೆ ಅಫೀಮಿನ ರುಚಿ ತಾಕಿತ್ತು. ಮಾರನೆ ದಿನವೂ ಅಷ್ಟುಹೊತ್ತಿಗೆ ಸರಿಯಾಗಿ ಇನ್ನೊಮ್ಮೆ ಅಫೀಮನ್ನು ಸೇವಿಸಲು ಬಂತು. ಅದರಂತೆಯೇ, ಶ್ರೀರಾಮಕೃಷ್ಣರ ಮಾತನ್ನು ಒಮ್ಮೆ ಕೇಳಿದ ಮೇಲೆ ಪುನಃ ಅದನ್ನು ಕೇಳಬೇಕೆಂಬ ಆಸೆ ಬಾಧಿಸುವುದು.


\section{\num{೧೪೪. } ಕಾ, ಕಾ, ಕಾ!}

ತನ್ನ ಸಮಾನ ಇಲ್ಲ ಎಂದು ಮೆರೆಯುತ್ತಿದ್ದ ಪಂಡಿತನೊಬ್ಬ. ಅವನಿಗೆ ದೇವರ ಆಕಾರಗಳಲ್ಲಿ ನಂಬಿಕೆ ಇರಲಿಲ್ಲ. ಆದರೆ ಭಗವಂತನ ಲೀಲೆಯನ್ನು ಯಾರು ಅರಿಯಬಲ್ಲರು? ಭಗವತಿ ಅವನಿಗೆ ಮಹಾಮಾಯೆಯಂತೆ ಕಾಣಿಸಿ ಕೊಂಡಳು. ಈ ಅನುಭವದಿಂದ ಕೆಲವು ಕಾಲ ಅವನಿಗೆ ಪ್ರಜ್ಞೆ ತಪ್ಪಿತು. ಅವನಿಗೆ ಸ್ವಲ್ಪ ಪ್ರಜ್ಞೆ ಬಂದಮೇಲೆ, ಅವನು ಬರೀ ಕಾ, ಕಾ ಎಂದು ಉಚ್ಚರಿಸಿದ. ಅವನಿಗೆ ಕಾಳಿ ಎಂತಲೂ ಹೇಳಲಾಗಲಿಲ್ಲ.


\section{\num{೧೪೫. } ಭಗವಂತನ ನಿಯಮಗಳು ಅಗೋಚರ}

ನಮ್ಮಲ್ಲಿರುವ ಸ್ವಲ್ಪ ಬುದ್ಧಿಶಕ್ತಿಯಿಂದ ನಾವು ಭಗವಂತನನ್ನು ಹೇಗೆ ಅರಿಯಬಲ್ಲೆವು? ಭೀಷ್ಮಾಚಾರ್ಯನು ಶರಪಂಜರದ ಮೇಲೆ ಮಲಗಿದ್ದಾಗ, ಪಾಂಡವರು ಮತ್ತು ಕೃಷ್ಣ ಅವನ ಮುಂದೆ ಇದ್ದರು. ಮಹಾವೀರ ಭೀಷ್ಮಾ ಚಾರ್ಯನ ಕಣ್ಣಿನಲ್ಲಿ ನೀರು ಸುರಿಯುತ್ತಿತ್ತು. ಅರ್ಜುನ ಶ್ರೀಕೃಷ್ಣನನ್ನು ಕುರಿತು, “ಸ್ನೇಹಿತನೆ, ಇದೇನು ವಿಚಿತ್ರ! ಸತ್ಯವ್ರತನೂ ಇಂದ್ರಿಯಜಿತನೂ ಪರಮಜ್ಞಾನಿಯೂ ಅಷ್ಟವಸುಗಳಲ್ಲಿ ಒಬ್ಬನೂ ಆದ ಭೀಷ್ಮನಂತಹವರೆ ಮಾಯೆಗೆ ವಶರಾಗಿ ಸಾವಿನ ಭಯದಿಂದ ಅಳುತ್ತಿರುವರಲ್ಲ!” ಎಂದು ಕೇಳಿದ. ಶ್ರೀಕೃಷ್ಣ ಭೀಷ್ಮನನ್ನೇ ಇದರ ಬಗ್ಗೆ ವಿಚಾರಿಸಿದ. “ನಿನಗೆ ತಿಳಿದಿರುವಂತೆ ನನ್ನ ದುಃಖಕ್ಕೆ ಕಾರಣ ಇದಲ್ಲ, ದೇವರೇ ಪಾಂಡವರ ಸಾರಥಿಯಾದರೂ, ಅವರು ಕರೆದಾಗಲೆಲ್ಲ ಅವನು ಬಂದರೂ, ಅವರ ಕಷ್ಟಕ್ಕೆ ಕೊನೆಯಿಲ್ಲವಲ್ಲ! ಈ ಆಲೋಚನೆ ನನಗೆ ಬಂದಾಗ ಭಗವಂತನ ಮಹಿಮೆಯನ್ನು ನಾನು ಇನ್ನೂ ತಿಳಿದುಕೊಳ್ಳಲಿಲ್ಲ ಎಂದು ವ್ಯಥೆಪಡುತ್ತಿರುವೆ” ಎಂದ ಭೀಷ್ಮ.


\section{\num{೧೪೬. } ಒಂದು ರಸಭರಿತವಾದ ಸಂದರ್ಭ}

ಪದ್ಮಲೋಚನ ಮಹಾ ಜ್ಞಾನಿ. ಅವನಿಗೆ ನನ್ನ ಮೇಲೆ ತುಂಬ ಗೌರವ ಇತ್ತು. ಆಗ ನಾನು ಪದೇಪದೇ ಭಗವತಿಯ ನಾಮವನ್ನು ಸ್ಮರಿಸುತ್ತಿದ್ದೆ. ಅವನು ಬರ್ದವಾನ್ ಮಹಾ ರಾಜರ ಆಸ್ಥಾನ ಪಂಡಿತ. ಅವನು ಒಂದು ಸಲ ಕಲ್ಕತ್ತೆಗೆ ಬಂದು ಕಾಮರ್​ಹಾಟಿಯಲ್ಲಿ ಇರುವ ಉದ್ಯಾನ ಮನೆಯಲ್ಲಿದ್ದ. ನಾನು ಅವನನ್ನು ನೋಡಬೇಕೆಂದು ಬಯಸಿದೆ.

ಪದ್ಮಲೋಚನ ಒಂದು ಸ್ವಾರಸ್ಯವಾದ ಘಟನೆಯನ್ನು ಹೇಳಿದ. ಒಂದು ಸಲ ಶಿವ ದೊಡ್ಡವನೆ ಬ್ರಹ್ಮ ದೊಡ್ಡವನೆ ಎಂಬ ವಿಷಯದಲ್ಲಿ ದೊಡ್ಡ ಚರ್ಚೆ ನಡೆಯಿತು. ಯಾವ ಒಂದು ನಿರ್ಣಯಕ್ಕೂ ಬರಲಾರದಾಗ ಪಂಡಿತರು ಪದ್ಮ ಲೋಚನನ ಅಭಿಪ್ರಾಯವನ್ನು ಕೇಳಿದರು. ಅವನು ಅತ್ಯಂತ ಸರಳವಾಗಿ ಈ ಅಭಿಪ್ರಾಯವನ್ನು ಕೊಟ್ಟನು: “ಇದನ್ನು ನಾನು ಹೇಗೆ ನಿರ್ಧರಿಸಲಿ? ನಾನಾಗಲಿ ನನ್ನ ಪೂರ್ವಿಕರಲ್ಲಿ ಹದಿನಾಲ್ಕು ತಲೆಮಾರಿನ ತನಕ ಯಾರೂ ಶಿವನನ್ನಾಗಲೀ ಬ್ರಹ್ಮನನ್ನಾಗಲಿ ಕಂಡಿಲ್ಲ.” ಬಾಲಕನು ದೂರದ ಮೋಡವನ್ನು ಮುಟ್ಟುವೆ ಎಂದು ಜಿಗಿದರೆ, ಅದು ದೊರೆಯುವುದೆ? ಅಂತೆಯೇ ಭಗವಂತನನ್ನು ಅಳೆ ಯುವುದು ಮನುಷ್ಯಸಾಧ್ಯವಲ್ಲ.


\section{\num{೧೪೭. } ಮನುಷ್ಯನ ಮೂಲಕ ಏಕೆ ದೇವರನ್ನು ಪೂಜಿಸಬಾರದು?}

ದೇವರೇ ಮನುಷ್ಯನಂತೆ ವ್ಯವಹರಿಸುತ್ತಿರುವುದು. ಜೇಡಿಮಣ್ಣಿನ ಮೂರ್ತಿಯ ಮೂಲಕ ದೇವರನ್ನು ಪೂಜಿಸಲು ಸಾಧ್ಯವಾದರೆ ಮನುಷ್ಯನ ಮೂಲಕ ಏತಕ್ಕೆ ಸಾಧ್ಯವಾಗಬಾರದು? ಒಂದು ಸಲ ಒಬ್ಬ ವರ್ತಕ ಸಮುದ್ರಕ್ಕೆ ಬಿದ್ದನು. ಅವನು ಸಿಂಹಳದ್ವೀಪಕ್ಕೆ ಈಜಿಕೊಂಡು ಹೋದನು. ಅಲ್ಲಿ ವಿಭೀಷಣ ರಾಕ್ಷಸ ನಗರಿಗೆ ರಾಜನಾಗಿದ್ದ. ವಿಭೀಷಣ ತನ್ನ ಆಳುಗಳಿಗೆ ವರ್ತಕನನ್ನು ಕರೆತರಲು ಹೇಳಿದ. ಅವನನ್ನು ನೋಡಿ ವಿಭೀಷಣನಿಗೆ ಆನಂದವಾಗಿ, “ಇವನು ನರರೂಪಿ ಶ್ರೀರಾಮಚಂದ್ರ ನಂತೆ ಕಾಣುತ್ತಾನೆ!” ಎಂದು ಉದ್ಗ ರಿಸಿದ. ವರ್ತಕನನ್ನು ವಸ್ತ್ರಾಭರಣಗಳಿಂದ ಅಲಂಕರಿಸಿ ಪೂಜಿಸಿದ. ಮೊದಲನೆ ಬಾರಿ ನಾನು ಈ ಕಥೆಯನ್ನು ಕೇಳಿದಾಗ ನನಗೆ ಎಷ್ಟು ಸಂತೋಷವಾಯಿತು! ಅದನ್ನು ನಾನು ನಿಮಗೆ ವಿವರಿಸಲಾರೆ.


\section{\num{೧೨೭. } ದೇವರು ಯಾವಾಗ ನಗುವನು?}

ದೇವರು ಎರಡು ಸಂದರ್ಭಗಳಲ್ಲಿ ನಗುವನು. ವೈದ್ಯ ರೋಗಿಯ ತಾಯಿಗೆ, “ತಾಯಿ, ನೀವು ಸ್ವಲ್ಪವೂ ಭಯಪಡಬೇಕಾಗಿಲ್ಲ, ನಿಮ್ಮ ಹುಡುಗನನ್ನು ನಾನು ನಿಜವಾಗಿಯೂ ಬದುಕಿಸುತ್ತೇನೆ” ಎನ್ನುವಾಗ ಒಂದು ಸಲ. ದೇವರು, “ನಾನು ಈ ಹುಡುಗನನ್ನು ತೆಗೆದುಕೊಂಡು ಹೋಗಲು ಬಂದಿರುವೆನು. ಈ ಬಡ ವೈದ್ಯ ತಾನು ಗುಣ ಮಾಡುತ್ತೇನೆ ಅನ್ನುವನಲ್ಲ!” ಎಂದು ಭಾವಿಸಿ ನಗುತ್ತಾನೆ. ದೇವರು ಎರಡನೆ ಬಾರಿ ನಗುವುದು, ಇಬ್ಬರು ಅಣ್ಣತಮ್ಮಂದಿರು, ಹಗ್ಗ ಹಿಡಿದು ಭೂಮಿಯನ್ನು ಪಾಲು ಮಾಡಿಕೊಂಡು, ಈ ಕಡೆ ನನ್ನದು, ಆ ಕಡೆ ನಿನ್ನದು ಎಂದು ಹೇಳುವಾಗ. ಅವನು ನಗುತ್ತ ಭಾವಿಸುತ್ತಾನೆ, “ಈ ಪ್ರಪಂಚ ವೆಲ್ಲ ನನಗೆ ಸೇರಿದ್ದು. ಆದರೆ ಈ ಜನ ಇದು ನನಗೆ ಸೇರಿದ್ದು, ಅದು ಅವನಿಗೆ ಸೇರಿದ್ದು ಎಂದು ಕಾದಾಡುವರಲ್ಲ!”


\section{\num{೧೪೯. } ನೀವು ಇದನ್ನು ಹೇಗೆ ವಿವರಿಸುತ್ತೀರಿ?}

ವಿಗ್ರಹದಲ್ಲಿ ದೇವರ ಆವಿರ್ಭಾವವನ್ನು ನಂಬಬೇಕಾಗುತ್ತದೆ. ನಾನು ಒಂದು ಸಲ ವಿಷ್ಣುಪುರಕ್ಕೆ ಹೋದೆ. ಅಲ್ಲಿ ಹಲವು ಒಳ್ಳೆಯ ದೇವಸ್ಥಾನಗಳಿವೆ. ಒಂದು ದೇವಸ್ಥಾನದಲ್ಲಿ ದೇವತೆಯ ಹೆಸರು ಮೃಣ್ಮಯಿ ಎಂದು. ದೇವಸ್ಥಾನದ ಬಳಿ ಹಲವು ಕೊಳಗಳು ಇವೆ. ಅವುಗಳಲ್ಲಿ ಒಂದನ್ನು ಲಾಲ್​ಬಂಧ್ ಎನ್ನುತ್ತಾರೆ, ಮತ್ತೊಂದನ್ನು ಕೃಷ್ಣ ಬಂಧ್ ಎನ್ನುತ್ತಾರೆ. ಹೆಂಗಸರು ತಮ್ಮ ಕೂದಲಿಗೆ ಉಪಯೋಗಿಸುವ ಸುಗಂಧ ತೈಲದ ಪರಿಮಳವನ್ನು ಅಲ್ಲಿನ ಒಂದು ಕೊಳದ ನೀರಿನಲ್ಲಿ ಆಘ್ರಾಣಿ ಸಿದೆ. ಆಗ ನನಗೆ, ಮೃಣ್ಮಯಿಯನ್ನು ಪೂಜಿಸುವಾಗ ಸ್ತ್ರೀಭಕ್ತರು ಈ ತೈಲವನ್ನು ಉಪಯೋಗಿಸುತ್ತಾರೆ ಎಂಬುದು ಗೊತ್ತಿರಲಿಲ್ಲ. ಆ ಕೊಳದ ಸಮೀಪದಲ್ಲಿ ನಾನು ಸಮಾಧಿಗೆ ಹೋದೆ–ಆ ಭಾವದಲ್ಲಿ ದೇವಿಯು ನೀರಿನಿಂದ ಎದ್ದು ಬರುತ್ತಿರುವುದನ್ನು ಕಂಡೆ. ಆಗ ನಾನು ದೇವಿಯ ಮೂರ್ತಿಯನ್ನು ಇನ್ನೂ ನೋಡಿರಲಿಲ್ಲ.


\section{\num{೧೫೦. } ಯಾರಿಗೆ ಇದನ್ನು ಹೇಳಲು ಸಾಧ್ಯ?}

ಒಬ್ಬ ರೋಗಿಯ ವಿಷಯ ವನ್ನು ತೆಗೆದುಕೊಳ್ಳಿ. ಪ್ರಕೃತಿ ಅವ ನನ್ನು ಮುಕ್ಕಾಲುಪಾಲು ಗುಣ ಮಾಡಿದೆ. ಆಗ ವೈದ್ಯ ಹಲವು ಮೂಲಿಕೆಗಳನ್ನು ಕೊಟ್ಟು, ಅದರ ರಸವನ್ನು ಕುಡಿ ಎಂದು ಹೇಳು ತ್ತಾನೆ. ಅದನ್ನು ತೆಗೆದುಕೊಂಡ ಮೇಲೆ ಅವನು ಸಂಪೂರ್ಣ ಗುಣ ವಾಗುತ್ತಾನೆ. ಈಗ ಅವನು ವೈದ್ಯ ಕೊಟ್ಟ ಔಷಧದಿಂದ ಗುಣ ವಾದನೆ, ಅಥವಾ ತನ್ನಿಂದ ತಾನೇ ಗುಣವಾದನೆ–ಹೇಳುವುದು ಕಷ್ಟ.

ಲಕ್ಷ್ಮಣ ಲವಕುಶರಿಗೆ ಹೇಳಿದ, “ನೀವು ಇನ್ನೂ ಮಕ್ಕಳು. ರಾಮನ ಮಹಿಮೆ ನಿಮಗೆ ತಿಳಿಯದು. ಅವನ ಪಾದ ಸೋಕಿದೊಡನೆಯೆ ಕಲ್ಲಾಗಿದ್ದ ಅಹಲ್ಯೆ ಮನುಷ್ಯನ ರೂಪವನ್ನು ಧರಿಸಿದಳು.” ಆಗ ಲವಕುಶರು ಹೇಳಿದರು, “ನಾವು ಆ ಕತೆಯನ್ನು ಕೇಳಿದ್ದೇವೆ. ಕಲ್ಲಾಗಿದ್ದ ಅಹಲ್ಯೆ ಪುಷಿಯ ಮಹಿಮೆ ಯಿಂದ ಮಾನವ ರೂಪವನ್ನು ಪಡೆದಳು. ಗೌತಮ ಅಹಲ್ಯೆಗೆ, ‘ತ್ರೇತಾಯುಗದಲ್ಲಿ ರಾಮ ಈ ಸ್ಥಳಕ್ಕೆ ಬರುತ್ತಾನೆ. ಅವನ ಪಾದಸ್ಪರ್ಶದಿಂದ ನೀನು ಪುನಃ ಮನುಷ್ಯಳಾಗುತ್ತೀಯೆ’ ಎಂದು ಹೇಳಿದ್ದ.” ಈ ಪವಾಡ ಹೇಗೆ ಆಯಿತು ಎಂದು ಯಾರು ಹೇಳಲು ಸಾಧ್ಯ? ಪುಷಿವಾಣಿಯ ಪ್ರಭಾವದಿಂದ ಆಯಿತೊ, ಅಥವಾ ಅದು ರಾಮನ ಮಹಾತ್ಮೆಯಿಂದ ಆಯಿತೊ–ಯಾರು ಹೇಳಬಲ್ಲರು? ಕಾಗೆಯು ಕುಕ್ಕಿ ತಾಳೆಯ ಹಣ್ಣು ಬಿತ್ತೋ, ಅಥವಾ ಪಕ್ವವಾಗಿ ಉದುರಿತೋ–ಯಾರು ಹೇಳಬಲ್ಲರು?

\chapter{ಮನುಷ್ಯ ಭಗವದ್​ಭಾವದಲ್ಲಿ}

\section{\num{೧೫೧. } ದಿವ್ಯಾನಂದದ ಅಮಲು}

ಮಗನು ತಂದೆಗೆ, “ಅಪ್ಪ ನೀನು ಸ್ವಲ್ಪ ಮದ್ಯವನ್ನು ಕುಡಿ.\\ಅನಂತರ ನೀನು ನನಗೆ ಕುಡಿಯುವುದನ್ನು ಬಿಡು ಎಂದು ಹೇಳಿದರೆ, ಆಗ ಬಿಟ್ಟಬಿಡುವೆ” ಎಂದ. ಹೆಂಡವನ್ನು ಕುಡಿದ ಮೇಲೆ ತಂದೆ ಮಗನಿಗೆ, “ನೀನು ಬಿಟ್ಟರೆ ಬಿಡಬಹುದು. ಅದಕ್ಕೆ ನನ್ನ ಅಭ್ಯಂತರವೇನೂ ಇಲ್ಲ. ಆದರೆ ನಾನು ಬಿಡುವುದಿಲ್ಲ” ಎಂದ.


\section{\num{೧೫೨. } ಹಲವು ರೀತಿಯಲ್ಲಿ ಅವರು ಇರುವರು}

ಒಂದು ಸಲ ಭಗವದ್​ಭಾವನೆಯಲ್ಲಿ ತಲ್ಲೀನನಾದ ಸಾಧು ಕಾಳಿ ದೇವ ಸ್ಥಾನಕ್ಕೆ ಬಂದ. ಒಂದು ದಿನ ಅವನಿಗೆ ಯಾರೂ ಊಟ ಕೊಡಲಿಲ್ಲ. ತುಂಬ ಹೊಟ್ಟೆ ಹಸಿಯುತ್ತ ಇದ್ದರೂ ಅವನು ಯಾರನ್ನೂ ಕೇಳಲಿಲ್ಲ. ಎಂಜಲು ಬಿಸಾಕಿದ್ದ ಜಾಗದಲ್ಲಿ ಒಂದು ನಾಯಿ ಆಹಾರವನ್ನು ತಿನ್ನುತ್ತಿತ್ತು. ಅವನು ನಾಯಿಯ ಹತ್ತಿರ ಬಂದು, ನಾಯಿಯನ್ನು ತಬ್ಬಿಕೊಂಡು, “ಸಹೋದರನೆ, ನನಗೆ ಸ್ವಲ್ಪವನ್ನೂ ಕೊಡದೆ ಎಲ್ಲವನ್ನೂ ನೀನೊಬ್ಬನೇ ತಿನ್ನುತ್ತಿರುವೆಯಲ್ಲ” ಎಂದು ಕೇಳಿದ. ಹಾಗೆ ಹೇಳುತ್ತ, ನಾಯಿಯೊಂದಿಗೆ ಅವನೂ ಊಟ ಮಾಡಲು ಆರಂಭಿಸಿದ. ಇಂತಹ ಸ್ನೇಹಿತನೊಡನೆ ಅವನು ಊಟ ಮಾಡಿದ ಮೇಲೆ, ಕಾಳಿಕಾ ದೇವಸ್ಥಾನಕ್ಕೆ ಹೋಗಿ ಇಡೀ ದೇವಾಲಯವೇ ಕಂಪಿಸುವಂತೆ ಅತ್ಯಂತ ಭಕ್ತಿಭಾವದಿಂದ ಪ್ರಾರ್ಥಿಸತೊಡಗಿದ. ಅದಾದಮೇಲೆ ಅವನು ಹೋಗುವುದರ ಲ್ಲಿದ್ದ. ಆಗ ನಾನು ಹೃದಯನಿಗೆ, “ಅವನನ್ನೇ ಅನುಸರಿಸಿಕೊಂಡು ಹೋಗು. ಅವನು ಏನು ಹೇಳುತ್ತಾನೊ ಅದನ್ನು ನನಗೆ ತಿಳಿಸು” ಎಂದೆ. ಹೃದಯ ಕೆಲವು ಗಜಗಳು ಅವನ ಹಿಂದೆ ಹೋದ. ಆಗ ಆ ಸಾಧು ಹಿಂದಿರುಗಿ, “ಏತಕ್ಕೆ ನೀನು ನನ್ನನ್ನು ಅನುಸರಿಸುತ್ತಿರುವೆ?” ಎಂದ. “ಸ್ವಾಮಿಗಳೇ, ನನಗೆ ಏನಾದರೂ ಬೋಧನೆಯನ್ನು ನೀಡಿ” ಎಂದು ಹೃದಯ ಕೇಳಿದ. ಆಗ ಸಾಧು “ಚರಂಡಿ ಯಲ್ಲಿರುವ ನೀರು, ಹರಿವ ಗಂಗಾನದಿ ಒಂದೇ ಎಂದು ನಿನಗೆ ಕಂಡಾಗ, ಈ ವೇಣುನಾದ ಮತ್ತು ಆ ದೊಂಬಿಯ ಗದ್ದಲ ಇವುಗಳಲ್ಲಿ ಯಾವ ಭೇದವನ್ನೂ ನೀನು ಮಾಡದೆ ಇದ್ದರೆ, ಆಗ ನೀನು ಪರಮಜ್ಞಾನವನ್ನು ಪಡೆ ಯುವೆ” ಎಂದ. ಹಾಗೆ ಹೇಳಿ ಅವನು ಹೊರಟುಹೋದ.

ನಾನು ಹೃದಯನಿಂದ ಇದನ್ನು ಕೇಳಿದ ಮೇಲೆ, “ಆ ಮನುಷ್ಯ ಭಗವದ್​ಭಾವನೆಯಲ್ಲಿ ನಿರತನಾಗಿರುವನು, ಪರಮಜ್ಞಾನದ ನೆಲೆ ಅವನಿಗೆ ಸಿಕ್ಕಿದೆ” ಎಂದು ಹೇಳಿದೆ.

ಸಿದ್ಧಪುರುಷರು ಕೆಲವು ವೇಳೆ ಎಳೆಯ ಮಕ್ಕಳಂತೆ ಇರುತ್ತಾರೆ, ಕೆಲವು ವೇಳೆ ಫಕೀರರಂತೆ ಇರುತ್ತಾರೆ, ಕೆಲವು ವೇಳೆ ಹುಚ್ಚರಂತೆ ಇರುತ್ತಾರೆ. ಹಲವರು ಹಲವು ವೇಷಗಳಲ್ಲಿ ಸಂಚರಿಸುತ್ತಾರೆ.


\section{\num{೧೫೩. } ಎಲ್ಲಾ ವಿಷ್ಣುಮಯ}

ಒಬ್ಬ ಸಂನ್ಯಾಸಿ ಇದ್ದ. ಅವನು ಯಾವಾಗಲೂ ಭಾವಮುಖದಲ್ಲಿರುತ್ತಿದ್ದ. ಯಾರೊಂದಿಗೂ ಮಾತನಾಡುತ್ತಿರಲಿಲ್ಲ. ಅವನೊಬ್ಬ ಹುಚ್ಚ ಎಂದು ಹಲ ವರು ತಿಳಿದುಕೊಂಡಿದ್ದರು. ಒಂದು ದಿನ ಹಳ್ಳಿಯಲ್ಲಿ ಭಿಕ್ಷೆಯನ್ನು ಬೇಡಿ, ಒಂದು ನಾಯಿಯೊಂದಿಗೆ ಕುಳಿತುಕೊಂಡ. ಅನಂತರ ನಾಯಿಯೊಂದಿಗೆ ಊಟ ಮಾಡಿದ. ಆ ವಿಚಿತ್ರ ದೃಶ್ಯವನ್ನು ನೋಡಲು ಹಲವು ಜನ ನೆರೆದರು. ಸಾಧು


\begin{myquote}
ನೀವು ಏತಕ್ಕೆ ನಗುತ್ತೀರಿ?\\ವಿಷ್ಣು ವಿಷ್ಣುವಿನೊಡನೆ ಕುಳಿತಿರುವನು.\\ವಿಷ್ಣು ವಿಷ್ಣುವಿಗೆ ಊಟ ಕೊಡುತ್ತಿರುವನು.\\ ಏತಕ್ಕೆ ನೀನು ನಗುತ್ತಿರುವೆ, ಓ ವಿಷ್ಣು?\\ಇರುವುದೆಲ್ಲ ವಿಷ್ಣುಮಯ.
\end{myquote}


\section{\num{೧೫೪. } ಯಾರು ಎಲ್ಲೆಲ್ಲಿಯೂ ಒಂದೇ ಸತ್ಯವನ್ನು ನೋಡುವನೊ ಅವನಿಗೆ ದುಃಖ ಹೇಗೆ ಬರುವುದು?}

ಒಂದು ಸಲ ದಕ್ಷಿಣೇಶ್ವರಕ್ಕೆ ಒಬ್ಬ ವ್ಯಕ್ತಿಯು ತನ್ನ ಮಗನೊಡನೆ ಬಂದನು. ಮಗನಿಗೆ ಬ್ರಹ್ಮಜ್ಞಾನ ಲಭಿಸಿತ್ತು. ಆದರೆ ತಂದೆಗೆ ಇನ್ನೂ ಜ್ಞಾನಪ್ರಾಪ್ತಿಯಾಗಿರ ಲಿಲ್ಲ. ಇಬ್ಬರೂ ಶ್ರೀರಾಮಕೃಷ್ಣರಿದ್ದ ಕೋಣೆಯಲ್ಲಿ ಕುಳಿತುಕೊಂಡು ಮಾತ ನಾಡುತ್ತಿದ್ದರು. ಆ ಸಮಯದಲ್ಲಿ ಒಂದು ಬಿಲದಿಂದ ನಾಗರಹಾವು ಬಂದು ಮಗನನ್ನು ಕಚ್ಚಿತು. ತಂದೆಗೆ ಇದನ್ನು ನೋಡಿ ತುಂಬಾ ಅಂಜಿಕೆ ಆಯಿತು. ಸುತ್ತಲೂ ಇರುವವರನ್ನು ಕೂಗಿ ಕರೆದನು. ಆಗ ಮಗ ಸುಮ್ಮನೆ ಕುಳಿತಿದ್ದ. ಇದನ್ನು ನೋಡಿ ತಂದೆಗೆ ಮತ್ತೆ ಆಶ್ಚರ್ಯವಾಯಿತು. ಅವನು ಮಗನಿಗೆ, “ಏತಕ್ಕೆ ಸುಮ್ಮನೆ ಕುಳಿತುಕೊಂಡಿರುವೆ” ಎಂದು ಕೇಳಿದ. ಆಗ ಮಗ ನಗುತ್ತ, “ಹಾವು ಯಾವುದು, ಅದು ಕಚ್ಚಿದ್ದು ಯಾರನ್ನು?” ಎಂದ. ಅವನು ಸಮತ್ವವನ್ನು ಪಡೆದಿದ್ದ. ಆದ್ದರಿಂದ ಮನುಷ್ಯನಿಗೂ ಹಾವಿಗೂ ಯಾವ ವ್ಯತ್ಯಾಸವನ್ನೂ ಮಾಡಲಿಲ್ಲ.


\section{\num{೧೫೫. } ಶತ್ರು ಮಿತ್ರರಿಬ್ಬರನ್ನೂ ಸಂತರು ಪ್ರೀತಿಸುತ್ತಾರೆ}

ಒಂದು ಕಡೆ ಒಂದು ಮಠ ಇತ್ತು. ಅಲ್ಲಿ ವಾಸವಾಗಿದ್ದ ಸಾಧುಗಳು ಪ್ರತಿ ದಿನವೂ ಭಿಕ್ಷೆಗೆ ಹೋಗುತ್ತಿದ್ದರು. ಒಂದು ದಿನ ಒಬ್ಬ ಸಾಧು ಭಿಕ್ಷೆಗೆ ಹೋಗಿ ದ್ದಾಗ ಜಮೀನ್​ದಾರನು ಒಬ್ಬನನ್ನು ಕಟುಕನಂತೆ ಹೊಡೆಯುತ್ತಿದ್ದುದನ್ನು ಕಂಡನು. ಕರುಣಾಮಯನಾದ ಆ ಸಾಧು ಆತನನ್ನು ಹೊಡೆಯಬೇಡವೆಂದು ಜಮೀನ್ದಾರನಿಗೆ ಹೇಳಿದನು. ಈ ಸಾಧುವಿನ ಮೇಲೆ ಅವನ ಕೋಪವೆಲ್ಲ ಕೆರಳಿತು. ಪ್ರಜ್ಞೆತಪ್ಪುವವರೆಗೆ ಅವನನ್ನು ಹೊಡೆದ. ಯಾರೋ ಸುದ್ದಿಯನ್ನು ಮಠಕ್ಕೆ ಮುಟ್ಟಿಸಿದರು. ಉಳಿದ ಸಾಧುಗಳು ಬಂದು, ತಮ್ಮ ಸಹೋದರನಿಗೆ ಆಗಿರುವ ಅನಾಹುತವನ್ನು ಕಂಡರು. ನಾಲ್ಕೈದು ಜನ ಇವನನ್ನು ಎತ್ತಿಕೊಂಡು ಮಠಕ್ಕೆ ತಂದರು. ದುಃಖ ದಿಂದ ಅವನ ಬಳಿ ಕುಳಿತು, ಅವನ ಆರೈಕೆ ಮಾಡಿದರು. ಅವನಿಗೆ ಪ್ರಜ್ಞೆ ಇನ್ನೂ ಬಂದಿರಲಿಲ್ಲ. ಹಾಲು ಕುಡಿಸಿದಲ್ಲಿ ಪ್ರಜ್ಞೆ ಮರುಕಳಿಸುವುದು ಎಂದು ಭಾವಿಸಿ ದರು.

ಬಾಯಿಗೆ ಹಾಲನ್ನು ಹಾಕಿದಾಗ ಅವನು ಕಣ್ಣನ್ನು ಬಿಟ್ಟು ಸುತ್ತಲೂ ನೋಡಿ ದನು. ಅವನಿಗೆ ಪ್ರಜ್ಞೆ ಪೂರ್ಣವಾಗಿ ಬಂದಿದೆಯೆ, ತಮ್ಮ ಗುರುತನ್ನು ಹಿಡಿಯು ವನೆ ಎಂಬುದನ್ನು ಪರೀಕ್ಷಿಸಲು ಅವನ ಹೆಸರನ್ನು ಕೂಗುತ್ತ, “ಪೂಜ್ಯರೆ, ನಿಮಗೆ ಯಾರು ಹಾಲನ್ನು ಕೊಡುತ್ತಿರುವರು?” ಎಂದು ಕೇಳಿದರು. “ಯಾರು ನನ್ನನ್ನು ಹೊಡೆದರೊ ಅವನೇ ಹಾಲನ್ನು ಕೊಡುತ್ತಿರುವನು” ಎಂದನು ಆ ಸಾಧು.


\section{\num{೧೫೬. } ಎರಡೂ ಭ್ರಾಂತಿಯೆ}

ಒಬ್ಬ ಕಟ್ಟಿಗೆ ಮಾರುವವನಿದ್ದ. ಅವನು ಒಳ್ಳೆಯ ಆಧ್ಯಾತ್ಮಿಕ ಜೀವಿ. ಒಂದು ಸಲ ಅವನು ಒಂದು ಒಳ್ಳೆಯ ಸ್ವಪ್ನವನ್ನು ಕಾಣುತ್ತಿದ್ದ. ಯಾರೋ ಅವನನ್ನು ಅಕಸ್ಮಾತ್ತಾಗಿ ಎಬ್ಬಿಸಿದುದರಿಂದ ಅವನಿಗೆ ಕೋಪವುಂಟಾಯಿತು. “ನನ್ನನ್ನು ಏತಕ್ಕೆ ಎಬ್ಬಿಸಿದಿರಿ? ನಾನು ಕನಸಿನಲ್ಲಿ ರಾಜನಾಗಿದ್ದೆ. ಏಳು ಮಕ್ಕಳ ತಂದೆ ಯಾಗಿದ್ದೆ. ನನ್ನ ಮಕ್ಕಳು ಅನೇಕೆ ವಿದ್ಯೆಗಳನ್ನು ಕಲಿಯುತ್ತಿದ್ದರು. ನಾನು ಸಿಂಹಾಸನದ ಮೇಲೆ ಕುಳಿತು ರಾಜ್ಯವನ್ನು ಆಳುತ್ತಿದ್ದೆ. ಅಂತಹ ಸುಂದರ ಸ್ವಪ್ನಕ್ಕೆ ಭಂಗ ತಂದೆಯಲ್ಲ” ಎಂದು ಹೇಳಿದ. ಎಬ್ಬಿಸಿದವನು, “ಓ, ಅದು ಬರೀ ಕನಸು. ಅದರಿಂದ ಏನು ಪ್ರಯೋಜನ?” ಎಂದ. ಸೌದೆ ಮಾರುವವನು, “ನಾನು ಸೌದೆ ಒಡೆಯುವವನಾಗಿದ್ದುದು ಎಷ್ಟು ಸತ್ಯವೋ ಅಷ್ಟೇ ಸತ್ಯವಾಗಿತ್ತು ಈ ಕನಸು. ನಾನು ಸೌದೆ ಒಡೆಯುವುದು ಸತ್ಯವಾಗಿದ್ದರೆ, ನಾನು ರಾಜನಾಗಿ ದ್ದುದೂ ಅಷ್ಟೇ ಸತ್ಯ,” ಎಂದನು. ವೇದಾಂತದ ಪ್ರಕಾರ ಕನಸು ಎಷ್ಟು ಸತ್ಯವೋ ಜಾಗೃತಾವಸ್ಥೆಯೂ ಅಷ್ಟೇ ಸತ್ಯ.


\section{\num{೧೫೭. } ಪರಮಜ್ಞಾನಿ ಮಗುವಿನಂತೆ ಇರಬೇಕು}

ಒಂದು ಸಲ ಓರ್ವ ಸಂನ್ಯಾಸಿನಿ ಜನಕನ ಆಸ್ಥಾನಕ್ಕೆ ಬಂದಳು. ರಾಜ ತಲೆತಗ್ಗಿಸಿ ಅವಳ ಮುಖವನ್ನು ನೋಡದೆ ನಮಸ್ಕಾರ ಮಾಡಿದ. ಇದನ್ನು ನೋಡಿ ಸಂನ್ಯಾಸಿನಿ, “ಜನಕ, ಇದು ಎಷ್ಟು ವಿಚಿತ್ರವಾದುದು! ನಿನಗೆ ಹೆಂಗಸನ್ನು ಕಂಡರೆ ಇನ್ನೂ ಇಷ್ಟೊಂದು ಅಂಜಿಕೆ ಇದೆಯಲ್ಲ!” ಎಂದಳು. ಪೂರ್ಣಜ್ಞಾನ ಪ್ರಾಪ್ತವಾದಾಗ ಸ್ವಭಾವ ಸಣ್ಣ ಮಗುವಿನಂತೆ ಆಗುವುದು. ಗಂಡಸಿಗೂ ಹೆಂಗಸಿಗೂ ಯಾವ ವ್ಯತ್ಯಾಸವನ್ನೂ ನೋಡುವುದಿಲ್ಲ.


\section{\num{೧೫೮. } ಅವಳ ರೀತಿ ವಿಚಿತ್ರವಾಗಿತ್ತು}

ಒಬ್ಬ ಭಕ್ತಳು ಒಳ್ಳೆಯ ಗೃಹಿಣಿಯೂ ಆಗಿದ್ದಳು. ಮನೆಯಲ್ಲಿ ಗಂಡ ಮತ್ತು ಮಕ್ಕಳನ್ನು ಪ್ರೀತಿಯಿಂದ ನೋಡಿಕೊಳ್ಳುತ್ತಿದ್ದಳು. ಆದರೂ ತನ್ನ ಮನಸ್ಸನ್ನು ಭಗವಂತನ ಮೇಲೆ ಇಟ್ಟಿದ್ದಳು. ಗಂಡ ತೀರಿಹೋದ ಮೇಲೆ ಶ್ರಾದ್ಧಾದಿಗಳು ಆದೊಡನೆಯೆ ತನ್ನ ಗಾಜಿನ ಬಳೆಯನ್ನು ಒಡೆದುಹಾಕಿ ಎರಡು ಬಂಗಾರದ ಬಳೆಯನ್ನು ಧರಿಸಿದಳು. ಜನರು ಈ ಹೆಂಗಸಿನ ಅಸ್ವಾಭಾವಿಕ ರೀತಿಯನ್ನು ನೋಡಿ ಆಶ್ಚರ್ಯಪಟ್ಟರು. ಅವರಿಗೆ ಹೆಂಗಸು ಈ ರೀತಿ ಉತ್ತರ ವಿತ್ತಳು: “ಇದುವರೆಗೆ ನನ್ನ ಗಂಡನ ದೇಹ ಗಾಜಿನ ಬಳೆಯಂತೆ ದುರ್ಬಲವಾ ಗಿತ್ತು. ಯಾವ ಸಮಯದಲ್ಲಿ ಒಡೆಯುವುದೆಂದು ಹೇಳುವುದಕ್ಕೆ ಆಗುತ್ತಿರ ಲಿಲ್ಲ. ಈಗ ಆ ನಶ್ವರ ದೇಹ ಹೋಯಿತು. ಈಗ ಅವನು ಯಾವ ಬದಲಾ ವಣೆಗೂ ಸಿಗದೆ ಎಲ್ಲಾ ವಿಧದಿಂದಲೂ ಪೂರ್ಣವಾಗಿರುವನು. ಅದಕ್ಕೇ ಕ್ಷುದ್ರ ವಾದ ಗಾಜಿನ ಬಳೆಯನ್ನು ಬಿಟ್ಟು, ಶಾಶ್ವತವಾದ ಬಂಗಾರದ ಬಳೆಯನ್ನು ಧರಿಸಿರುವೆನು.”


\section{\num{೧೫೯. } ಸಾಧು ಸಂಗ}

ಭಗವನ್ಮಯಿಯಾದ ಉಮಾ ಹಿಮರಾಜನ ಮಗಳಾಗಿ ಜನ್ಮವೆತ್ತಿದಳು, ಎಂದು ಪುರಾಣದಲ್ಲಿ ಹೇಳಿದೆ. ಅವಳು ತಂದೆಗೆ ಭಗವತಿಯ ಹಲವು ರೂಪಗಳ ದರ್ಶನವನ್ನು ಕೊಟ್ಟಳು. ಆದರೆ ಗಿರಿರಾಜ ವೇದದಲ್ಲಿ ಬರುವ ಬ್ರಹ್ಮನನ್ನು ತೋರು ಎಂದಾಗ ಉಮಾ ಹೇಳಿದಳು: “ಅಪ್ಪ, ನೀನು ಬ್ರಹ್ಮನನ್ನು ನೋಡಬೇಕಾದರೆ ಸರ್ವಸಂಗ ಪರಿತ್ಯಾಗಿ ಗಳಾದ ಸಾಧುಗಳೊಂದಿಗೆ ಇರಬೇಕು.”


\section{\num{೧೬ಂ. } ಆದಿ}

ಎಲ್ಲಿಯವರೆಗೆ ದೇಹವಿರುವುದೋ ಅಲ್ಲಿಯವರೆಗೆ ಅದನ್ನು ನೋಡಿಕೊಳ್ಳ ಬೇಕು. ಆದರೆ ನನ್ನ ದೇಹ ಆತ್ಮನಿಗಿಂತ ಬೇರೆ ಎಂಬುದು ಗೊತ್ತಿದೆ. ಯಾವಾಗ ಒಬ್ಬನಿಗೆ ಕಾಮ ಕಾಂಚನದ ಮೇಲೆ ಮಮತೆ ಸಂಪೂರ್ಣ ನಿಲ್ಲುವುದೊ ಆಗ ದೇಹ ಬೇರೆ ಆತ್ಮ ಬೇರೆ ಎಂಬುದನ್ನು ಅರಿಯುವನು. ತೆಂಗಿನಕಾಯಿಯಲ್ಲಿ ನೀರೆಲ್ಲ ಇಂಗಿಹೋದ ಮೇಲೆ ಕೊಬ್ಬರಿ ಕರಟದಿಂದ ಬೇರೆಯಾಗುವುದು. ತೆಂಗಿನಕಾಯಿಯನ್ನು ಅಲ್ಲಾಡಿಸಿದರೆ ಅದರೊಳಗಿರುವ ಗಿಟಕು ಶಬ್ದ ಮಾಡು ವುದು. ಅದು ಕತ್ತಿ ತನ್ನ ಒರೆಯಲ್ಲಿರುವಂತೆ. ಆದಕಾರಣವೆ ಜಗನ್ಮಾತೆಗೆ ನನ್ನ



\section{\num{೧೬೧. } ಪರಮಹಂಸನ ಸ್ವಭಾವ}

ನಾನು ಒಂದು ಸಲ ಕಾಮಾರಪುಕುರದಲ್ಲಿದ್ದೆ. ಆಗ ಶಿವರಾಮನಿಗೆ (ಸೋದರಳಿಯ) ನಾಲ್ಕೈದು ವರುಷವಾಗಿತ್ತು. ಒಂದು ದಿನ ಅವನು ಕೆಲವು ಚಿಟ್ಟೆಗಳನ್ನು ಕೊಳದ ಸಮೀಪದಿಂದ ಹಿಡಿಯಲು ಯತ್ನಿಸಿದ. ಗಿಡಮರಗಳು ಅಲ್ಲಾಡುತ್ತಿದ್ದವು. “ಚುಪ್​ಚುಪ್, ನಾನು ಒಂದು ಚಿಟ್ಟೆಯನ್ನು ಹಿಡಿಯಲು ಯತ್ನಿಸುತ್ತಿರುವೆನು” ಎಂದ. ಮತ್ತೊಂದು ದಿನ ಶಿವರಾಮ ನನ್ನೊಡನೆ ಮನೆ ಯೊಳಗಿದ್ದಾಗ, ಹೊರಗೆ ದಟ್ಟವಾಗಿ ಮೋಡ ಕವಿದಿತ್ತು. ತುಂಬಾ ಜೋರಾಗಿ ಮಳೆ ಬಂತು. ಮಿಂಚು ಗುಡುಗು ಶಬ್ದ ಮಾಡುತ್ತಿದ್ದವು. ಹುಡುಗ ಹೊರಗೆ ಹೋಗಿ ಏನಾಗಿದೆ ಎಂಬುದನ್ನು ತಿಳಿಯಲೆತ್ನಿಸಿದ. ನಾನು ಬೈದು ಅವನನ್ನು ಸುಮ್ಮನಿರಿಸಿದೆ. ಆದರೂ ಅವನು ಕೆಲವು ವೇಳೆ ಹೊರಗೆ ಇಣಿಕಿ ಏನಾಗುತ್ತಿದೆ ಎಂಬುದನ್ನು ನೋಡಿದ. ಅವನು ಕೋರೈಸುವ ಮಿಂಚನ್ನು ನೋಡಿದಾಗ, “ಮಾವ, ಮಾವ, ಬೆಂಕಿಪಟ್ಟಿಗೆಯನ್ನು ಅವರು ಪುನಃ ಹತ್ತಿಸುತ್ತಿರುವರು” ಎಂದ. ಪರಮಹಂಸ ಐದು ವರುಷದ ಬಾಲಕನಂತೆ. ಅದಕ್ಕೇ ಎಲ್ಲವೂ ಚೈತನ್ಯಮಯವಾಗಿ ಅವನಿಗೆ ಕಾಣುತ್ತದೆ.


\section{\num{೧೬೨. } ಶಂಕರಾಚಾರ್ಯರು ಮತ್ತು ಕಟುಕ}

ಶಂಕರಾಚಾರ್ಯರು ನಿಜವಾಗಿ ಬ್ರಹ್ಮಜ್ಞಾನಿಗಳಾಗಿದ್ದರು. ಆದರೆ ಪ್ರಾರಂಭ ದಲ್ಲಿ ಅವರಲ್ಲಿ ಕೂಡ ತಾರತಮ್ಯಭಾವ ಇತ್ತು. ಈ ಪ್ರಪಂಚದಲ್ಲಿ ಎಲ್ಲವೂ ಬ್ರಹ್ಮಮಯ ಎಂಬುದನ್ನು ಸಂಪೂರ್ಣವಾಗಿ ನಂಬಿರಲಿಲ್ಲ. ಒಂದು ದಿನ ಅವರು ಸ್ನಾನಮಾಡಿಕೊಂಡು ಹೊರಗೆ ಬರುತ್ತಿರುವಾಗ ಒಬ್ಬ ಅಸ್ಪೃಶ್ಯ ಕಟುಕನು ಮಾಂಸ ಹೊತ್ತುಕೊಂಡು ಎದುರುಗಡೆಯಿಂದ ಬರುತ್ತಿದ್ದ ಮತ್ತು ಅವರನ್ನು ಅರಿವಿಲ್ಲದೆ ಸ್ಪರ್ಶ ಮಾಡಿದ. “ಅಲ್ಲೆ ನಿಲ್ಲು, ನನ್ನನ್ನು ಮುಟ್ಟುವುದಕ್ಕೆ ನಿನಗೆಷ್ಟು ಧೈರ್ಯ!” ಎಂದು ಆಚಾರ್ಯರು ಗದರಿಸಿದರು. ಅವರಿಗೆ ಅವನು ಹೇಳಿದ: “ಪೂಜ್ಯರೆ, ನಾನು ನಿಮ್ಮನ್ನು ಮುಟ್ಟಿಲ್ಲ, ನೀವು ನನ್ನನ್ನು ಮುಟ್ಟಿಲ್ಲ. ಪರಿಶುದ್ಧ ಆತ್ಮ ದೇಹವಲ್ಲ, ಪಂಚಭೂತಗಳಲ್ಲ, ಇಪ್ಪತ್ತುನಾಲ್ಕು ತತ್ತ್ವಗಳೂ ಅಲ್ಲ.” ಆಗ ಶಂಕರರಿಗೆ ಬುದ್ಧಿ ಬಂತು.

\chapter{ಗುರು}

\section{\num{೧೬೩. } ಬೆಲ್ಲದ ಗಡಿಗೆಗಳು ಮತ್ತು ವೈದ್ಯ}

ವೈದ್ಯನೋರ್ವನು ರೋಗಿಗೆ ಔಷಧಿಯನ್ನು ಕೊಟ್ಟು ಅವನಿಗೆ,\\“ಮತ್ತೊಂದು ದಿನ ಬಾ, ನಾನು ಪಥ್ಯದ ವಿಷಯವನ್ನು ಹೇಳುವೆನು” ಎಂದನು. ಅಂದಿನ ದಿನ ವೈದ್ಯನ ಕೋಣೆಯಲ್ಲಿ ಹಲವು ಗುಡಾಣ ಬೆಲ್ಲ ಇತ್ತು. ರೋಗಿ ವಾಸಿಸುತ್ತಿದ್ದ ಸ್ಥಳ ಬಹುದೂರ. ಕೆಲವು ಕಾಲದ ಮೇಲೆ ವೈದ್ಯನನ್ನು ನೋಡುವು ದಕ್ಕೆ ಹೋದ. ವೈದ್ಯ, “ನೀನು ಆಹಾರದ ವಿಷಯದಲ್ಲಿ ಜಾಗೃತನಾಗಿರು. ನೀನು ಬೆಲ್ಲವನ್ನು ತಿನ್ನಬಾರದು” ಎಂದನು. ರೋಗಿ ಹೋದ ಮೇಲೆ ಹತ್ತಿರದಲ್ಲೇ ಇದ್ದ ಮತ್ತೊಬ್ಬನು, “ಏತಕ್ಕೆ ಅವನನ್ನು ನಿನ್ನ ಬಳಿ ಮತ್ತೆ ಬರುವಂತೆ ಮಾಡಿದೆ? ಮೊದಲನೆಯ ದಿನವೇ ನೀನಿದನ್ನು ಹೇಳಿಬಿಡಬಹುದಾಗಿತ್ತಲ್ಲ,” ಎಂದು ಕೇಳಿದನು. ಆಗ ವೈದ್ಯನು ನಗುತ್ತ “ಅದಕ್ಕೊಂದು ಕಾರಣವಿದೆ. ಅಂದು ನನ್ನ ಕೋಣೆಯ ತುಂಬ ಬೆಲ್ಲದ ಗುಡಾಣಗಳಿದ್ದವು. ರೋಗಿಗೆ ಆಗ ಬೆಲ್ಲ ತಿನ್ನಬೇಡ ಎಂದು ಹೇಳಿದ್ದರೆ ನನ್ನ ಮಾತಿನಲ್ಲಿ ಅವನಿಗೆ ನಂಬಿಕೆ ಬರುತ್ತಿರ ಲಿಲ್ಲ. ‘ಅವನು ತನ್ನ ಕೋಣೆಯಲ್ಲೇ ಹಲವು ಜಾಡಿ ಬೆಲ್ಲವನ್ನು ಇಟ್ಟಿರುವನು. ಅವನು ಮಧ್ಯೆ ಮಧ್ಯೆ ಅದನ್ನು ತಿನ್ನುತ್ತಲೂ ಇರಬಹುದು. ಆದ್ದರಿಂದ ಬೆಲ್ಲ ಅಷ್ಟು ಕೆಟ್ಟದ್ದಿರಲಿಕ್ಕಿಲ್ಲ’ ಎಂದು ಭಾವಿಸುತ್ತಿದ್ದ. ಇವತ್ತು ನಾನು ಬೆಲ್ಲದ ಗುಡಾಣವನ್ನು ಬಚ್ಚಿಟ್ಟಿರುವೆನು. ಈಗ ನನ್ನ ಮಾತಿನಲ್ಲಿ ಅವನಿಗೆ ನಂಬಿಕೆಯಿರುತ್ತದೆ.”

ಯಾರು ಮತ್ತೊಬ್ಬರಿಗೆ ಗುರುಗಳಾಗಬೇಕಾಗಿದೆಯೋ ಅವರು ತ್ಯಾಗಿಗಳಾಗಿ ರಬೇಕು. ಯಾರು ಆಚಾರ್ಯನಾಗಿರುವನೋ ಅವನು ಕಾಮಿನಿ ಕಾಂಚನವನ್ನು ತ್ಯಜಿಸಬೇಕು. ಇಲ್ಲದೆ ಇದ್ದರೆ ಯಾರೂ ಅವನ ಮಾತನ್ನು ಕೇಳುವುದಿಲ್ಲ. ಮಾನಸಿಕವಾಗಿ ಮಾತ್ರ ಅದನ್ನು ತ್ಯಜಿಸಿದರೆ ಸಾಲದು. ಬಾಹ್ಯದಲ್ಲಿಯೂ ಅವನು ಅವುಗಳನ್ನು ಬಿಡಬೇಕು. ಆಗಲೇ ಅವನ ಬೋಧನೆ ಫಲಕಾರಿಯಾಗು ವುದು. ಇಲ್ಲದಿದ್ದರೆ ಜನ, ‘ಅವನು ಕಾಮಿನಿ ಕಾಂಚನವನ್ನು ತ್ಯಜಿಸಿ ಎಂದು ನಮಗೆ ಹೇಳಿದರೂ, ಗುಟ್ಟಾಗಿ ಅವನೇ ಇವುಗಳನ್ನೆಲ್ಲ ಅನುಭವಿಸುತ್ತಿರುವನು’ ಎಂದು ಭಾವಿಸುತ್ತಾರೆ.


\section{\num{೧೬೪. } ಅಧಿಕಾರದ ಮುದ್ರೆ}

ಕಾಮಾರಪುಕುರದಲ್ಲಿ ಹಲದಾರಪುಕುರವೆಂಬ ಸರೋವರ ಇದೆ. ಪ್ರತಿದಿನವೂ ಕೆಲವು ಜನರು ಹೇಸಿಗೆಯಿಂದ ಕೆರೆಯ ದಂಡೆಯನ್ನು ಕೆಡಿಸುತ್ತಿದ್ದರು. ಮಾರನೇ ದಿನ ಸ್ನಾನ ಮಾಡುವುದಕ್ಕೆ ಬಂದವರು ಅವರನ್ನು ಸಿಕ್ಕಾಪಟ್ಟೆ ಬಯ್ಯುತ್ತಿದ್ದರು. ಆದರೆ ಪುನಃ ಬೆಳಿಗ್ಗೆ ಅದೇ ರಾಮಾಯಣ. ಕೆರೆಯ ದಡವನ್ನು ಕೆಡಿಸುವುದನ್ನು ಜನ ನಿಲ್ಲಿಸ ಲಿಲ್ಲ. ಹಳ್ಳಿಯವರು ಕೊನೆಗೆ ಕೆರೆಗೆ ಸಂಬಂಧಪಟ್ಟ ಅಧಿಕಾರಿಗಳಿಗೆ ದೂರು ಕೊಟ್ಟರು. ಒಬ್ಬ ಪೋಲಿಸಿನವನು ಬಂದು ‘ಇಲ್ಲಿ ಹೇಸಿಗೆಯನ್ನು ಮಾಡು ವವರು ಶಿಕ್ಷೆಗೆ ಗುರಿಯಾಗುತ್ತಾರೆ’ ಎಂದು ಒಂದು ನೋಟೀಸನ್ನು ಹಾಕಿದ. ತಕ್ಷಣವೇ ಅದು ನಿಂತಿತು.

ಮತ್ತೊಬ್ಬರಿಗೆ ಬೋಧನೆ ಮಾಡಬೇಕಾದರೆ ಅಧಿಕಾರಮುದ್ರೆ ಇರಬೇಕು, ಇಲ್ಲದಿದ್ದರೆ ನಗೆಪಾಟಲಾಗುವುದು. ತನಗೇನೂ ಗೊತ್ತಿಲ್ಲದವನು ಇತರರಿಗೆ ಬೋಧನೆ ಮಾಡಿದರೆ, ಆಗ ಕುರುಡರು ಕುರುಡರಿಗೆ ದಾರಿಯನ್ನು ತೋರಿದಂತೆ ಆಗುವುದು! ಇತರರಿಗೆ ಉಪಕಾರವನ್ನು ಮಾಡುವ ಬದಲು ಅಪಕಾರವನ್ನು ಮಾಡುತ್ತಾರೆ. ಒಬ್ಬನಿಗೆ ಭಗವಂತನ ಸಾಕ್ಷಾತ್ಕಾರವಾದ ಮೇಲೆ ಅವನಿಗೆ ಅಂತರ್​ದೃಷ್ಟಿ ಉಂಟಾಗುವುದು. ಆಗ ಮಾತ್ರ ಆಧ್ಯಾತ್ಮಿಕ ಜೀವನದ ಸಾಧಕ ಬಾಧಕಗಳನ್ನು ಮನಗಂಡು ಇತರರಿಗೆ ಅವನು ಬೋಧಿಸಬಲ್ಲ.


\section{\num{೧೬೫. } ಭಗವಂತನಿಂದ ಅಪ್ಪಣೆ ಇಲ್ಲದೆ ಯಾರೂ ಇತರರಿಗೆ ಬೋಧಿಸಬಾರದು}

ಭಗವಂತನಿಂದ ಅಪ್ಪಣೆ ಬಂದಮೇಲೆ ಇನ್ನೊಬ್ಬನಿಗೆ ಬೋಧಿಸಿದರೆ ಬಾಧಕ ವಿಲ್ಲ. ಯಾರು ಭಗವಂತನಿಂದ ಅಧಿಕಾರವನ್ನು ಪಡೆದಿರುವರೋ ಅವರನ್ನು ಯಾರೂ ತಪ್ಪಾಗಿ ಭಾವಿಸುವುದಿಲ್ಲ. ಸರಸ್ವತಿಯಿಂದ ಒಂದು ಕಿರಣ ಜ್ಞಾನ ಪಡೆದರೂ, ವ್ಯಕ್ತಿ ಬಲಿಷ್ಠನಾಗುತ್ತಾನೆ. ಅವನೆದುರಿಗೆ ಇತರ ಪಂಡಿತರು ಎರೆಹುಳುಗಳಂತೆ ತೋರುವರು. ಭಗವಂತನಿಂದ ಅಧಿಕಾರ ಪಡೆಯದೆ ಬರೀ ಉಪನ್ಯಾಸಗಳನ್ನು ಮಾಡಿದರೆ ಬಂದ ಪ್ರಯೋಜನವೇನು? ಒಂದು ಸಲ ಒಬ್ಬ ಬ್ರಹ್ಮಸಮಾಜಕ್ಕೆ ಸೇರಿದವನು, “ಸ್ನೇಹಿತರೆ, ನಾನು ಹಿಂದೆ ಎಷ್ಟೊಂದು ಮದ್ಯವನ್ನು ಕುಡಿದಿರುವೆನು” ಎಂದು ಮುಂತಾಗಿ ಹೇಳಿದ. ಇದನ್ನು ಕೇಳಿ ಜನರು “ಮೂಢ–ಏನನ್ನು ಹೇಳುತ್ತಿರುವನು ಇವನು! ಮದ್ಯ ಕುಡಿದಿರುವ ನಂತೆ!” ಎಂದು ಆಡಿಕೊಳ್ಳತೊಡಗಿದರು. ಇದನ್ನು ಕೇಳಿದವರಿಗೆ ಅವನ ಮೇಲೆ ಒಳ್ಳೆಯ ಅಭಿಪ್ರಾಯ ಬರಲಿಲ್ಲ. ಉಪದೇಶವು ಒಳ್ಳೆಯ ಮನುಷ್ಯನಿಂದ ಬಂದರೆ ಮಾತ್ರ ನಮ್ಮ ಮೇಲೆ ಒಳ್ಳೆಯ ಪ್ರಭಾವವಾಗುವುದು.

ಬಾರಿಸಾಲ್​ನಿಂದ ಬಂದ ಉನ್ನತ ಅಧಿಕಾರಿಯೊಬ್ಬ, ‘ನೀವು ಉಪನ್ಯಾಸಕ್ಕೆ ಕೈಹಾಕಿದರೆ ನಾನು ಕೂಡ ನನ್ನ ಕೈಲಾದುದನ್ನು ಮಾಡುವೆ’ ಎಂದ. ಹಲದಾರ ಪುಕುರದ ದಂಡೆಯನ್ನು ಗಲೀಜು ಮಾಡುತ್ತಿದ್ದವರ ಕಥೆಯನ್ನು ಅವನಿಗೆ ಹೇಳಿ ದೆನು.

ಕೆಲಸಕ್ಕೆ ಬಾರದವನು ತನಗೆ ತೋರಿದುದನ್ನೆಲ್ಲ ಹೇಳಬಹುದು. ಅದರಿಂದ ಏನೂ ಪ್ರಯೋಜನವಾಗುವುದಿಲ್ಲ. ಭಗವಂತನಿಂದ ಅಪ್ಪಣೆ ಬಂದಿದ್ದರೆ ಜನ ಅವನ ಮಾತನ್ನು ಕೇಳುತ್ತಾರೆ. ಆಚಾರ್ಯನಾದ ವನಿಗೆ ಬೇಕಾದಷ್ಟು ಶಕ್ತಿ ಇರಬೇಕು. ಕಲ್ಕತ್ತದಲ್ಲಿ ಎಷ್ಟೋ ಜನ ಹನುಮಾನ್ ಪೂರಿ(ಪ್ರಸಿದ್ಧ ಕುಸ್ತಿಪಟು)ಗಳು ಇದ್ದಾರೆ. ಅವ ರೊಂದಿಗೆ ನೀವು ಕುಸ್ತಿ ಮಾಡಬೇಕಾಗಿದೆ.


\section{\num{೧೬೬. } ಅವಧೂತ ಮತ್ತು ಅವನ ಉಪಗುರುಗಳು}

ಗುರು ಒಬ್ಬನೇ. ಆದರೆ ಉಪಗುರುಗಳು ಹಲವರಿರಬಹುದು. ನಾವೊಬ್ಬ ರಿಂದ ಯಾವುದೇ ವಿಷಯವನ್ನು ಕಲಿತರೂ, ಅವರು ಉಪಗುರುಗಳಾಗುತ್ತಾರೆ. ಭಾಗವತದಲ್ಲಿ ಹೇಳಿದೆ ಅವಧೂತನಿಗೆ ಇಪ್ಪತ್ತುನಾಲ್ಕು ಗುರುಗಳಿದ್ದರು ಎಂದು.

೧. ಒಂದು ಸಲ ಅವಧೂತ ಬಯಲಿನಲ್ಲಿ ಹೋಗುತ್ತಿದ್ದಾಗ ಮದುವೆಯ ಮೆರವಣಿಗೆ ಬಾಣ ಬಿರುಸು ತಮ್ಮಟೆಗಳೊಂದಿಗೆ ಬರುತ್ತಿತ್ತು. ಹತ್ತಿರದಲ್ಲಿದ್ದ ಒಬ್ಬ ಬೇಡ ಒಂದು ಮೃಗವನ್ನು ಗುರಿಯಿಟ್ಟು ಹೊಡೆಯುವುದರಲ್ಲಿದ್ದ. ಸುತ್ತ ಮುತ್ತಲಿರುವ ಸಂಭ್ರಮದ ಕಡೆ ಸ್ವಲ್ಪವೂ ಗಮನವೀಯದೆ ತನ್ನ ಗುರಿ ಎಡೆಗೆ ನೋಡುತ್ತಿದ್ದ. ಅವಧೂತ ಬೇಟೆಗಾರನನ್ನು ಕಂಡು “ನನಗೆ ನೀವೇ ಗುರು. ನಾನು ಧ್ಯಾನಕ್ಕೆ ಕುಳಿತಿರುವಾಗ ಮನಸ್ಸು ಧ್ಯಾನಿಸುವ ವಸ್ತುವಿನಲ್ಲಿ ಹೀಗೇ ತಲ್ಲೀನವಾಗಿರಲಿ,” ಎಂದು ಅವನಿಗೆ ವಂದಿಸಿದನು.

೨. ಬೆಸ್ತನೊಬ್ಬ ಮೀನನ್ನು ಹಿಡಿಯುತ್ತಿದ್ದ. ಅವಧೂತ ಅವನ ಬಳಿಗೆ ಬಂದು “ಆ ಊರಿಗೆ ಹೋಗುವುದಕ್ಕೆ ದಾರಿ ಯಾವುದು?” ಎಂದು ಕೇಳಿದ. ಅವನು ನೀರಿನಲ್ಲಿದ್ದ ಗಾಳದ ಬೆಂಡಿಗೆ ಸೇರಿಸಿದ್ದ ಆಹಾರವನ್ನು ಮೀನು ಇನ್ನೇನು ತಿನ್ನುವುದರಲ್ಲಿದೆ ಎಂಬುದನ್ನು ನೋಡಿ, ಅವನು ಈತನ ಕಡೆ ಸ್ವಲ್ಪವೂ ಗಮನಕೊಡದೆ ತನ್ನ ಗಾಳವನ್ನೇ ನೋಡುತ್ತಿದ್ದ. ಅವನು ಮೊದಲು ಮೀನನ್ನು ಹಿಡಿದು ಹಿಂತಿರುಗಿ ನೋಡಿ, “ಸ್ವಾಮಿ, ನೀವು ಏನನ್ನು ಹೇಳುತ್ತಿ ದ್ದಿರಿ?” ಎಂದು ಕೇಳಿದ. ಅವಧೂತ ಅವನಿಗೆ ನಮಸ್ಕಾರ ಮಾಡಿ, “ಸ್ವಾಮಿ, ನೀವೇ ನನ್ನ ಗುರುಗಳು. ನಾನು ದೇವರನ್ನು ಕುರಿತು ಚಿಂತಿಸುವಾಗ ಈ ಉದಾಹರಣೆಯನ್ನು ಅನುಸರಿಸುವೆನು. ನೀವು ಮೀನನ್ನು ಹಿಡಿಯುತ್ತಿದ್ದಾಗ ಮನಸ್ಸು ಹೇಗೆ ಏಕಾಗ್ರ ವಾಗಿತ್ತೋ ಹಾಗೆ ನನ್ನ ಮನಸ್ಸು ನನ್ನ ಇಷ್ಟದೇವರಲ್ಲಿ ಏಕಾಗ್ರವಾಗಲಿ,” ಎಂದನು.

೩. ಒಂದು ಮೀನನ್ನು ಹದ್ದು ಕೊಕ್ಕಿನಲ್ಲಿ ಕಚ್ಚಿಕೊಂಡು ಹೋಗುತ್ತಿತ್ತು. ಹಲವು ಕಾಗೆಗಳು ಮತ್ತು ಹದ್ದುಗಳು ಅದನ್ನು ಸುತ್ತುವರಿದವು. ಬಾಯಿಯಲ್ಲಿದ್ದ ಮೀನನ್ನು ಕಿತ್ತುಕೊಳ್ಳುವುದಕ್ಕಾಗಿ ಇತರ ಪಕ್ಷಿಗಳು ಅದನ್ನು ಕೊಕ್ಕಿನಿಂದ ಪೀಡಿಸುತ್ತಿದ್ದವು. ಅದು ಯಾವ ಕಡೆಗೆ ಹೋದರೂ ಆ ಕಡೆಗೆ ಹೋಗಿ ಆ ಹದ್ದನ್ನು ಪೀಡಿಸುತ್ತಿದ್ದವು. ಕೊನೆಗೆ ಇತರ ಹಕ್ಕಿಗಳು ಹದ್ದನ್ನು ಪೀಡಿಸಿದ್ದರ ಫಲವಾಗಿ ಹದ್ದು ತನ್ನ ಬಾಯಲ್ಲಿದ್ದ ಮೀನನ್ನು ಸಡಿಲಬಿಟ್ಟಿತು. ಮತ್ತೊಂದು ಹದ್ದು ಆ ಮೀನನ್ನು ತೆಗೆದುಕೊಂಡಿತು. ತಕ್ಷಣವೇ ಇತರ ಹಕ್ಕಿಗಳು ಅದನ್ನು ಅನುಸರಿಸಿದವು. ಮೊದಲನೆ ಹದ್ದು ತನ್ನ ಕೊಕ್ಕಿನಲ್ಲಿದ್ದ ಮೀನನ್ನು ಬಿಟ್ಟಮೇಲೆ ಎಲ್ಲದರ ಕಾಟದಿಂದಲೂ ಪಾರಾಗಿ ಹಾಯಾಗಿ ಒಂದು ಮರದ ಮೇಲೆ ಕುಳಿತುಕೊಂಡಿತು. ಆ ಹದ್ದಿನ ಪ್ರಶಾಂತ ಸ್ಥಿತಿಯನ್ನು ನೋಡಿ ಅವಧೂತ ಅದಕ್ಕೆ ನಮಿಸಿ, “ನೀನೆ ನನ್ನ ಗುರು. ಯಾವಾಗ ಒಬ್ಬ ಉಪಾಧಿಗಳನ್ನು ತ್ಯಜಿಸುವನೊ ಆಗ ಅವನಿಗೆ ಶಾಂತಿ ಲಾಭವಾಗುವುದು ಎಂಬುದನ್ನು ಕಲಿಸಿದೆ” ಎಂದು ನುಡಿದನು.

೪. ಒಂದು ಕೊಕ್ಕರೆ ಒಂದು ಮೀನನ್ನು ಹಿಡಿಯಲು ಕೆಸರಿನಲ್ಲಿ ನಿಧಾನವಾಗಿ ನಡೆಯುತ್ತಿತ್ತು. ಒಬ್ಬ ಬೇಟೆಯವನು ಕೊಕ್ಕರೆಯನ್ನು ಹಿಡಿದುಕೊಳ್ಳುವುದಕ್ಕೆ ಅದನ್ನು ಅನುಸರಿಸುತ್ತಿದ್ದನು. ಆದರೆ ಆ ಕೊಕ್ಕರೆಗೆ ಇದು ಸ್ವಲ್ಪವೂ ಗೊತ್ತಾಗ ಲಿಲ್ಲ. ಅವಧೂತ ಕೊಕ್ಕರೆಗೆ ನಮಿಸಿ, “ನನ್ನ ಮನಸ್ಸು ಧ್ಯಾನಿಸುತ್ತಿರುವಾಗ ಈ ಕೊಕ್ಕರೆಯನ್ನು ಅನುಸರಿಸಲಿ” ಎಂದನು.

೫. ಅವಧೂತ ಜೇನುನೊಣವನ್ನು ನೋಡಿದಾಗ ಅವನಿಗೆ ಮತ್ತೊಬ್ಬ ಗುರು ಸಿಕ್ಕಿದನು. ಜೇನುನೊಣ ತುಂಬಾ ಕಷ್ಟಪಟ್ಟು ಹನಿಹನಿಯಾಗಿ ಜೇನುತುಪ್ಪವನ್ನು ತನ್ನ ಗೂಡಿನಲ್ಲಿ ಸಂಗ್ರಹಿಸುತ್ತಿತ್ತು. ಯಾರೊ ಒಬ್ಬ ಬಂದು ಜೇನುಗೂಡನ್ನು ಒಡೆದು ಅದರ ತುಪ್ಪವನ್ನು ಕುಡಿದನು. ಇಷ್ಟು ಕಾಲ ತಾನು ಸಂಪಾದಿಸಿದ ಜೇನುತುಪ್ಪವನ್ನು ಅದು ಅನುಭವಿಸಲಾಗಲಿಲ್ಲ. ಇದನ್ನು ನೋಡಿ ಅವಧೂತ ಜೇನುನೊಣಕ್ಕೆ, “ಜೇನುನೊಣವೇ, ನೀನೇ ನನಗೆ ಗುರು. ಯಾರು ಹಣವನ್ನು ಸಂರಕ್ಷಿಸುತ್ತಿರುವರೊ ಅವರ ಪಾಡು ಇದು ಎಂಬುದನ್ನು ನೀನು ತಿಳಿಸಿದೆ,” ಎಂದನು.


\section{\num{೧೬೭. } ಹುಲ್ಲನ್ನು ತಿನ್ನುತ್ತಿದ್ದ ಹುಲಿ}

ಒಂದು ಸಲ ಒಂದು ಗರ್ಭಿಣಿಯಾದ ಹುಲಿ ಕುರಿಮಂದೆಯ ಮೇಲೆ ಎರಗಿತು. ಹುಲಿ ಒಂದು ಕುರಿಯ ಮೇಲೆ ಹಾರಿ ಅದನ್ನು ಹಿಡಿದುಕೊಳ್ಳಬೇಕು ಎಂದಿರುವಾಗ, ಅದು ಮರಿಯನ್ನು ಹಡೆದು ಸತ್ತುಹೋಯಿತು. ಕುರಿಮಂದೆಯಲ್ಲಿಯೇ ಈ ಹುಲಿಮರಿ ವಾಸಮಾಡು ತ್ತಿತ್ತು. ಕುರಿಗಳು ಹುಲ್ಲನ್ನು ತಿಂದರೆ ಹುಲಿಮರಿಯೂ ಹುಲ್ಲನ್ನೇ ತಿನ್ನುತ್ತಿತ್ತು. ಕುರಿ ಅರಚಿಕೊಂಡರೆ ಹುಲಿ ಮರಿಯೂ ಕುರಿಯಂತೆಯೇ ಅರಚಿಕೊಳ್ಳುತ್ತಿತ್ತು. ಕಾಲಕ್ರಮೇಣ ಹುಲಿಮರಿ ದೊಡ್ಡ ಹುಲಿಯಾಯಿತು. ಮತ್ತೊಂದು ಸಲ ಹುಲಿ ಯೊಂದು ಕುರಿಮಂದೆಯ ಮೇಲೆ ಬಿತ್ತು. ಆಗ ಅದು ಒಂದು ಹುಲಿ ಹುಲ್ಲನ್ನು ತಿನ್ನುತ್ತಿದ್ದುದನ್ನು ನೋಡಿತು; ಅದೂ ಕುರಿಯಂತೆಯೇ ಅರಚುತ್ತಿದ್ದುದನ್ನು ನೋಡಿ ಅದಕ್ಕೆ ಆಶ್ಚರ್ಯವಾಯಿತು. ಕಾಡಿನ ಹುಲಿ ಅದನ್ನು ಅಟ್ಟಿಸಿಕೊಂಡು ಹೋಗಿ ಅದನ್ನು ಹಿಡಿಯಿತು. ಆಗ ಕುರಿಮಂದೆಯಲ್ಲಿದ್ದ ಹುಲಿ ಭಯದಿಂದ ಕುರಿಯಂತೆಯೇ ಕಿರುಚತೊಡಗಿತು. ಕಾಡಿನ ಹುಲಿ ಊರಿನ ಹುಲಿಯನ್ನು ನೀರಿನ ಹತ್ತಿರ ಎಳೆದುಕೊಂಡುಹೋಗಿ, “ನೀನು ನೀರಿನಲ್ಲಿ ನಿನ್ನ ಪ್ರತಿಬಿಂಬವನ್ನು ನೋಡು. ಅದು ನನ್ನಂತೆಯೇ ಇದೆ. ಇಲ್ಲಿ ಮಾಂಸವಿದೆ, ಇದನ್ನು ತಿಂದು ನೋಡು,” ಎಂದಿತು. ಆದರೆ ಹುಲ್ಲನ್ನು ಮೇಯುತ್ತಿದ್ದ ಹುಲಿ ಅದನ್ನು ತಿನ್ನಲಾರದೆ ಕುರಿಯಂತೆಯೇ ಕಿರುಚಿಕೊಂಡಿತು. ಕ್ರಮೇಣ ಅದಕ್ಕೆ ರಕ್ತದ ರುಚಿ ಕಂಡಿತು, ಅನಂತರ ಮಾಂಸವನ್ನು ತಿನ್ನತೊಡಗಿತು. ಈಗ ಕಾಡುಹುಲಿ, “ನೋಡಿದೆಯಾ, ನನಗೂ ನಿನಗೂ ಯಾವ ವ್ಯತ್ಯಾಸವೂ ಇಲ್ಲ. ನನ್ನ ಜತೆಯಲ್ಲಿ ಕಾಡಿಗೆ ಹೋಗೋಣ ಬಾ” ಎಂದು ಹೇಳಿ ಕರೆದುಕೊಂಡು ಹೋಯಿತು. ಗುರುವಿನ ಕೃಪೆ ನಿಮಗೆ ಲಭಿಸಿದರೆ ಇನ್ನು ಯಾವ ಭಯವೂ ಇರುವುದಿಲ್ಲ. ಅವನು, ನೀವು ಯಾರು, ನಿಮ್ಮ ನೈಜ ಸ್ವರೂಪವೇನು ಎಂಬುದನ್ನು ತೋರಿಸಿಕೊಡುತ್ತಾನೆ.


\section{\num{೧೬೮. } ಶ್ರೀ ಚೈತನ್ಯರು ಸಂಸಾರಿಕರನ್ನು ತಮ್ಮ ಬಳಿಗೆ ಹೇಗೆ ಸೆಳೆದುಕೊಂಡರು}

ಪ್ರಾಪಂಚಿಕರಿಗೆ ನೀವು ಪ್ರಪಂಚ ಬಿಟ್ಟುಬಿಡಿ, ಭಗವಂತನನ್ನೇ ಚಿಂತಿಸಿ ಎಂದು ಹೇಳಿದರೆ ಅವರು ನಿಮ್ಮ ಮಾತನ್ನು ಕೇಳುವುದಿಲ್ಲ. ಶ್ರೀಚೈತನ್ಯ ಮತ್ತು ಅವನ ಸಂಗಿ ನಿತೈ ಇಬ್ಬರೂ ಬಹಳ ಆಲೋಚಿಸಿ, ಈ ಪ್ರಾಪಂಚಿಕರನ್ನು ದೇವರ ಕಡೆ ಕರೆದೊಯ್ಯುವುದಕ್ಕೆ ಒಂದು ಉಪಾಯ ಮಾಡಿದರು. ಪ್ರಾಪಂಚಿಕರಿಗೆ, “ನೀವು ಹರಿನಾಮವನ್ನು ಹೇಳಿ, ನಿಮಗೆ ಮಾಗೂರ್ ಮತ್ಸ್ಯದ ರುಚಿಕರವಾದ ಮೀನು ಸಾರು, ಒಬ್ಬ ತರುಣಿಯ ಆಲಿಂಗನ ಸಿಕ್ಕುವುದು” ಎಂದರು. ಅನೇಕ ಜನ ಮೀನಿನ ಸಾರಿನ ರುಚಿಗೆ ಮತ್ತು ನವತರುಣಿಯರ ಆಲಿಂಗನಕ್ಕೆ ಮಾರು ಹೋಗಿ ಭಗವಂತನ ನಾಮವನ್ನು ಹೇಳಿದರು. ಭಗವಂತನ ಆನಂದದ ಸವಿ ಯನ್ನು ಸ್ವಲ್ಪ ಅನುಭವಿಸಿದ ಮೇಲೆ ಅವರಿಗೆ ತಕ್ಷಣವೇ ಗೊತ್ತಾಯಿತು, ಮೀನಿನ ಸಾರು ಎಂದರೆ ದೇವರ ಹೆಸರಿನಲ್ಲಿ ಸುರಿಸುವ ಕಣ್ಣೀರು ಮತ್ತು ನವತರುಣಿಯರ ಆಲಿಂಗನ ಎಂದರೆ ಭಾವದಲ್ಲಿ ನೆಲದ ಮೇಲೆ ಹೊರಳಾಡು ವುದು ಎಂದು.


\section{\num{೧೬೯. } ಗುರುವಿನಂತೆ ಶಿಷ್ಯ}

ನಾನು ಆದಿಬ್ರಹ್ಮಸಮಾಜದ ಆಚಾರ್ಯನನ್ನು ನೋಡಿದ್ದೇನೆ. ಅವನು ಎರಡನೆ ಸಲವೂ, ಮೂರನೆ ಬಾರಿಯೂ ಮದುವೆ ಮಾಡಿ ಕೊಂಡಿರುವನಂತೆ. ಅವನಿಗೆ ವಯಸ್ಸಾದ ಮಕ್ಕಳಿದ್ದಾರೆ. ಇಂತಹ ವರು ಗುರುಗಳು! ‘ದೇವರೊಬ್ಬನೆ ಸತ್ಯ, ಉಳಿದಿರುವುದೆಲ್ಲ ಮಿಥ್ಯ’ ಎಂದು ಅವರು ಹೇಳಿದರೆ ಯಾರು ಅವರನ್ನು ನಂಬು ತ್ತಾರೆ? ಅಂತಹವರ ಶಿಷ್ಯರು ಹೇಗಿರುವರು ಎಂಬುದನ್ನು ನೀವೇ ಊಹಿಸಿ ನೋಡಿ.

ಗುರುಗಳಂತೆ ಶಿಷ್ಯರೂ ಕೂಡ. ಸಂನ್ಯಾಸಿ ಮಾನಸಿಕವಾಗಿ ಕಾಮ ಕಾಂಚನವನ್ನು ತ್ಯಜಿಸಿದರೂ, ಬಾಹ್ಯ ಜೀವನದಲ್ಲಿ ಅವುಗಳೊಡನೆ ಇದ್ದರೆ, ಅವನು ಇತರರಿಗೆ ಗುರು ವಾಗಲಾರ. ಗುಟ್ಟಾಗಿ ಅವನು ಸಂಸಾರವನ್ನು ಮೆಲ್ಲುತ್ತಿರುವನು ಎಂದು ಭಾವಿಸುತ್ತಾರೆ. ಒಂದು ಸಲ ಸಿಂಥಿಯ ಮಹೇಂದ್ರ ಕವಿ ರಾಜ, ರಾಮಲಾಲನಿಗೆ ಐದು ರೂಪಾಯಿಯನ್ನು ಕೊಟ್ಟ. ನನಗೆ ಅದು ಗೊತ್ತಿರಲಿಲ್ಲ. ರಾಮಲಾಲ ಈ ವಿಷಯವನ್ನು ಹೇಳಿದಾಗ, ಅವನು ಯಾರಿಗಾಗಿ ಈ ಹಣವನ್ನು ಕೊಟ್ಟಿದ್ದಾನೆ ಎಂದು ಕೇಳಿದೆನು. ಆತ ಅದು ನನಗಾಗಿ ಎಂದ. ನಾನು ಮುಂಚೆ ಆ ಹಣವನ್ನು ಹಾಲಿನ ಖರ್ಚಿಗೆ ಕೊಡೋಣ ಎಂದು ಭಾವಿಸಿದೆ. ಆದರೆ ನೀವು ಇದನ್ನು ನಂಬು ತ್ತಿರೋ ಇಲ್ಲವೊ, ಒಂದು ಸ್ವಲ್ಪ ಹೊತ್ತು ಮಲಗಿದ ಮೇಲೆ, ನನಗೆ ಏನೋ ಸಂಕಟವಾಯಿತು. ಒಂದು ಬೆಕ್ಕು ಎದೆಯನ್ನು ಪರಚಿದಂತೆ ಆಯಿತು. ನಾನು ರಾಮಲಾಲನ ಬಳಿಗೆ ಹೋಗಿ, “ಹಣವನ್ನು ಅವನು ನಿನ್ನ ಅತ್ತೆಗಾಗಿ (ಶ್ರೀಮಾತೆ ಶಾರದಾದೇವಿಯವರಿಗಾಗಿ) ಕೊಟ್ಟನೆ?” ಎಂದು ಕೇಳಿದೆ. “ಇಲ್ಲ” ಎಂದ ರಾಮ ಲಾಲ. ಆಗ ನಾನು “ತಕ್ಷಣ ಹೋಗಿ ಅವನು ಕೊಟ್ಟ ಹಣವನ್ನು ಹಿಂತಿರುಗಿ ಕೊಟ್ಟುಬಿಡು” ಎಂದೆ. ರಾಮಲಾಲ ಮಾರನೇ ದಿನ ಹಣವನ್ನು ಕೊಟ್ಟುಬಿಟ್ಟ. ನೀರಿನಲ್ಲಿ ತೇಲುವ ಕಡ್ಡಿ ಗಟ್ಟಿಯಾಗಿದ್ದರೆ ಮಾತ್ರ ಹಕ್ಕಿ ಅದರ ಮೇಲೆ ನಿಲ್ಲಬಹುದು. ಅಂತೆಯೇ ಗುರುವಿನ ಜೀವನ ಉನ್ನತವಾಗಿದ್ದಲ್ಲಿ ಶಿಷ್ಯ ಉತ್ತಮನಾಗುತ್ತಾನೆ.


\section{\num{೧೭೦. } ವೈವಿಧ್ಯತೆಗಳೆಲ್ಲ ಹೋದಮೇಲೆ}

ಶುಕದೇವ ಜನಕನ ಬಳಿಗೆ ಹೋಗಿ ಬ್ರಹ್ಮವಿದ್ಯೆಯನ್ನು ನನಗೆ ಬೋಧಿಸಿ ಎಂದು ಕೇಳಿದ. ಜನಕ, “ಮೊದಲು ನನಗೆ ಗುರುದಕ್ಷಿಣೆಯನ್ನು ಕೊಟ್ಟುಬಿಡು. ನಿನಗೆ ಬ್ರಹ್ಮಜ್ಞಾನ ಬಂದಮೇಲೆ ನೀನು ನನಗೆ ದಕ್ಷಿಣೆಯನ್ನು ಕೊಡುವುದಿಲ್ಲ. ಏಕೆಂದರೆ ಬ್ರಹ್ಮಜ್ಞಾನಿಗೆ ಗುರು-ಶಿಷ್ಯ ವ್ಯತ್ಯಾಸವೇ ಇರುವುದಿಲ್ಲ” ಎಂದ.

