
\chapter{ಪ್ರಪಂಚ}

\section{\num{೧.} ಇದೇನೆ ಪ್ರಪಂಚ ಎಂದರೆ!}

ಒಂದು ಸಲ ಹೃದಯ (ಶ್ರೀ ರಾಮಕೃಷ್ಣರ ಸೋದರಳಿಯ) ಒಂದು ಸಣ್ಣ ಗಂಡು ಕರುವನ್ನು ದಕ್ಷಿಣೇಶ್ವರಕ್ಕೆ ತಂದ. ನಾನು (ಶ್ರೀರಾಮಕೃಷ್ಣರು) ಹೃದಯನನ್ನು ಒಂದು ದಿನ “ಏತಕ್ಕೆ ಕರುವನ್ನು ಅಲ್ಲಿ ಕಟ್ಟಿರುವೆ” ಎಂದು ಕೇಳಿದೆ. “ಮಾವ, ನಾನು ಕರುವನ್ನು ನಮ್ಮ ಹಳ್ಳಿಗೆ ಕಳುಹಿಸುತ್ತೇನೆ. ಅದು ದೊಡ್ಡದಾದ ಮೇಲೆ ಅದನ್ನು ನೇಗಿಲಿನಿಂದ ಉಳುವ ಕೆಲಸಕ್ಕೆ ಉಪಯೋಗಿ ಸುತ್ತೇನೆ” ಎಂದ. ಇದನ್ನು ಕೇಳಿದೊಡನೆಯೆ ನನಗೆ ಮೈ ಜುಮ್ ಎಂದಿತು. ಮಹಾಮಾಯೆಯ ಲೀಲೆ ಅದ್ಭುತವಾದುದು! ಕಲ್ಕತ್ತೆಯೆಲ್ಲಿ, ಕಾಮಾರಪುಕುರ ವೆಲ್ಲಿ! ಅದು ಎಷ್ಟು ದೂರವಿದೆ! ಈ ಬಡ ಕರು ಅಷ್ಟು ದೂರ ಇಲ್ಲಿಂದ ನಡೆದುಕೊಂಡು ಹೋಗಬೇಕು. ಅದು ಬೆಳೆದ ನಂತರ ಅದನ್ನು ನೇಗಿಲಿಗೆ ಹಾಕುತ್ತಾರೆ. ಇದೇ ಪ್ರಪಂಚದ ವಿಚಿತ್ರ. ಮಾಯೆ ಎಂದರೆ ಇದೆ! ನಾನು ಪ್ರಜ್ಞೆ ತಪ್ಪಿ ಬಿದ್ದೆ. ಕೆಲವು ಕಾಲದ ಮೇಲೆ ನನಗೆ ಪ್ರಜ್ಞೆ ಬಂತು.


\section{\num{೨.} ಸಂಸಾರದ ಕಾನನದಲ್ಲಿ}

ಒಂದು ಸಲ ಒಬ್ಬ ಕಾಡಿನ ಮೂಲಕ ನಡೆದುಕೊಂಡು ಹೋಗುತ್ತಿದ್ದ. ಆಗ ಮೂರು ಜನ ದರೋಡೆಕೋರರು ಅವನ ಮೇಲೆ ಬಿದ್ದು ಅವನಲ್ಲಿದ್ದುದನ್ನೆಲ್ಲ ಕಸಿದುಕೊಂಡರು. ಅವರಲ್ಲಿ ಮೊದಲನೆ ಕಳ್ಳ, “ಇವನನ್ನು ಉಳಿಸಿ ಏನು ಪ್ರಯೋಜನ,” ಎಂದು ಹೇಳುತ್ತ ಅವನನ್ನು ಕೊಲ್ಲಲು ಅಣಿಯಾದ. ಆಗ ಎರಡನೆ ದರೋಡೆಕೋರ, “ಅವನನ್ನು ಕೊಂದು ಏನು ಪ್ರಯೋಜನ? ಅವನ ಕೈಕಾಲುಗಳನ್ನು ಕಟ್ಟಿ ಇಲ್ಲೇ ಬಿಡೋಣ” ಎಂದ. ದರೋಡೆಕೋರರು ಅವನ ಕೈಕಾಲುಗಳನ್ನು ಕಟ್ಟಿ ಅಲ್ಲೇ ಬಿಟ್ಟು ಹೋದರು. ಸ್ವಲ್ಪಕಾಲದ ಮೇಲೆ ಮೂರನೆ ದರೋಡೆಕೋರ ಹಿಂತಿರುಗಿ ಬಂದು, “ಅಯ್ಯೋ, ನನಗೆ ತುಂಬ ವ್ಯಥೆಯಾಗಿದೆ. ನಿನಗೆ ತುಂಬಾ ನೋವಾಗಿದೆಯೆ? ನಾನು ನಿನ್ನನ್ನು ಬಂಧನದಿಂದ ಬಿಡಿಸು ತ್ತೇನೆ” ಎಂದ. ಅವನನ್ನು ಬಿಡಿಸಿದ ಮೇಲೆ, ದರೋಡೆಕೋರ, “ನನ್ನ ಜತೆಯಲ್ಲಿ ಬಾ, ನಾನು ನಿನ್ನನ್ನು ರಾಜಮಾರ್ಗಕ್ಕೆ ಕರೆದುಕೊಂಡು ಹೋಗಿ ಬಿಡುತ್ತೇನೆ” ಎಂದ. ಸ್ವಲ್ಪ ಕಾಲ ನಡೆದ ಮೇಲೆ ರಾಜಮಾರ್ಗ ದೊರಕಿತು. ಆಗ ಪ್ರಯಾ ಣಿಕನು “ನೀನು ನನಗೆ ತುಂಬ ಉಪಕಾರ ಮಾಡಿರುವೆ. ದಯವಿಟ್ಟು ನನ್ನ ಮನೆಗೆ ಬಾ” ಎಂದು ಕೇಳಿಕೊಂಡನು. ಆಗ ದರೋಡೆಕೋರ “ಹಾಗೆ ಮಾತ್ರ ನಾನು ಮಾಡಲಾರೆ. ಪೊಲೀಸಿನವರು ನನ್ನನ್ನು ಗೊತ್ತುಹಚ್ಚುವರು” ಎಂದ.

ಈ ಪ್ರಪಂಚವೇ ಒಂದು ಕಾಡು. ಇಲ್ಲಿ ಕಾಡುತ್ತಿರುವ ಮೂರು ಜನ ದರೋಡೆಕೋರರೇ ಸತ್ವ, ರಜಸ್ಸು ಮತ್ತು ತಮಸ್ಸು ಎಂಬುವರು. ಮನುಷ್ಯ ರೆಲ್ಲ ಈ ಕಾಡಿನಲ್ಲಿ ಸಿಕ್ಕಿಕೊಂಡಿರುವ ಪ್ರಯಾಣಿಕರು. ತಮಸ್ಸು ಪ್ರಯಾಣಿಕ ನನ್ನು ಕೊಲ್ಲಲು ಪ್ರಯತ್ನಿಸುವುದು. ರಜಸ್ಸು ಅವನನ್ನು ಕಟ್ಟಿಹಾಕುವುದು. ಸತ್ವ ಅವನನ್ನು ತಮಸ್ಸು ಮತ್ತು ರಜಸ್ಸಿನ ಬಂಧನದಿಂದ ಪಾರುಮಾಡುವುದು. ಸತ್ವ ರಕ್ಷಿಸುವಾಗ, ಕಾಮ ಕ್ರೋಧಾದಿಗಳ ಬಂಧನದಿಂದ ಪಾರು ಮಾಡುವುದು. ಆದರೆ ಸತ್ವ ಕೂಡ ಒಬ್ಬ ದರೋಡೆ ಕೋರನೆ. ಇದು ಮನುಷ್ಯನಿಗೆ ಪರಮ ಸಾಯುಜ್ಯ ಪದವಿಯನ್ನು ಕೊಡಲಾರದು. ಇದು ನಮ್ಮನ್ನು ಆ ದಾರಿಯ ಕಡೆ ಹೋಗುವುದಕ್ಕೆ ದಾರಿ ತೋರಿಸುವುದು. ಸತ್ವ ಪ್ರಯಾಣಿಕನನ್ನು ಬಿಡುಗಡೆ ಮಾಡಿ “ನೋಡು ಅಲ್ಲಿದೆ, ನಿನ್ನ ಮನೆ” ಎಂದು ತೋರುವುದು. ಸತ್ವವು ಕೂಡ ಬ್ರಹ್ಮಜ್ಞಾನಕ್ಕಿಂತ ಕೆಳಗಡೆ ಇರುವುದು.


\section{\num{೩.} ಪ್ರಪಂಚ ನಮಗೆ ಏನು ಮಾಡುವುದು?}

ನಾನು ಹುಡುಗನಾಗಿದ್ದಾಗ ರಾಮಮಲ್ಲಿಕನನ್ನು ತುಂಬಾ ಪ್ರೀತಿಸುತ್ತಿದ್ದೆ. ಅನಂತರ ಅವನನ್ನು ನೋಡಿದಾಗ ಅವನನ್ನು ಮುಟ್ಟಲು ಕೂಡ ಮನಸ್ಸು ಬರಲಿಲ್ಲ. ರಾಮಮಲ್ಲಿಕ ಮತ್ತು ನಾನು ಇಬ್ಬರೂ ಒಳ್ಳೆಯ ಸ್ನೇಹಿತರಾಗಿದ್ದೆವು. ಹಗಲು ರಾತ್ರಿ ಅವನೊಡನೆ ಇರುತ್ತಿದ್ದೆ. ಆಗ ನನಗೆ ಹದಿನಾರೋ ಹದಿನೇಳೋ ವರುಷವಾಗಿತ್ತು. ಆಗ ಜನರು “ಇವರಲ್ಲಿ ಒಬ್ಬ ಏನಾದರೂ ಹೆಂಗಸಾಗಿ ದ್ದಿದ್ದರೆ ಇಬ್ಬರೂ ಮದುವೆಮಾಡಿಕೊಳ್ಳುತ್ತಿದ್ದರು” ಎಂದು ಹೇಳುತ್ತಿದ್ದರು. ನಾವಿಬ್ಬರೂ ಅವನ ಮನೆಯಲ್ಲಿ ಆಡುತ್ತಿದ್ದೆವು. ಆ ದಿನಗಳು ನನಗೆ ಚೆನ್ನಾಗಿ ನೆನಪಿವೆ. ಅವನ ಬಂಧುಗಳು ಪಲ್ಲಕ್ಕಿಯಲ್ಲಿ ಬರುತ್ತಿದ್ದರು. ಈಗ ಅಲ್ಲಿ ಅವನಿಗೊಂದು ಅಂಗಡಿ ಇದೆ. ನಾನು ಅನೇಕ ವೇಳೆ ಅವನಿಗೆ ಹೇಳಿ ಕಳುಹಿಸು ತ್ತಿದ್ದೆ. ಕೆಲವು ದಿನಗಳ ಹಿಂದೆ ಅವನು ಬಂದು ನನ್ನೊಡನೆ ಎರಡು ದಿನಗಳನ್ನು ಕಳೆದ. ರಾಮ, “ನನಗೆ ಯಾರೂ ಮಕ್ಕಳಿಲ್ಲ” ಎಂದ. ಅವನು ತನ್ನ ಸಹೋ ದರನ ಮಗನನ್ನು ನೋಡಿಕೊಳ್ಳುತ್ತಿದ್ದ. ಆದರೆ ಆ ಹುಡುಗ ತೀರಿಹೋದ ಎಂದು ತುಂಬಾ ದುಃಖದಿಂದ ಹೇಳಿಕೊಂಡ. ಇದನ್ನು ಹೇಳುವಾಗ ನಿಟ್ಟುಸಿರು ಬಿಡುತ್ತಿದ್ದ. ದುಃಖದಿಂದ ಅವನ ಕಣ್ಣುಗಳು ಒದ್ದೆಯಾಗಿದ್ದವು. ಅವನು ತನ್ನ ಸೋದರನ ಮಕ್ಕಳಿಗಾಗಿ ವ್ಯಥೆಪಡುತ್ತಿದ್ದ. ತನಗೆ ಯಾರೂ ಮಕ್ಕಳು ಇಲ್ಲ ದುದರಿಂದ ಅವನ ಹೆಂಡತಿಯ ಪ್ರೀತಿಯೂ ಸಹೋದರನ ಮಗನ ಕಡೆ ಹರಿಯಿತು ಎಂದು ಅವನು ನನಗೆ ಹೇಳಿದನು. ಅವಳು ದುಃಖದಿಂದ ವ್ಯಥೆಪಡು ತ್ತಿದ್ದಳು. ರಾಮನು ಅವಳಿಗೆ, “ನೀನು ಹುಚ್ಚಿ, ಸುಮ್ಮನೆ ದುಃಖಿಸಿ ಏನು ಪ್ರಯೋಜನ? ನೀನು ಕಾಶಿಗೆ ಹೋಗುವೆಯೇನು?” ಎಂದು ಹೇಳಿದ. ನೋಡಿ, ಅವನು ತನ್ನ ಹೆಂಡತಿಯನ್ನು ಹುಚ್ಚಿ ಎಂದು ಕರೆಯುವುದನ್ನು! ಸತ್ತುಹೋದ ಹುಡುಗನ ದೆಸೆಯಿಂದ ಅವನು ಕಂಗೆಟ್ಟಿದ್ದ. ಅವನಲ್ಲಿ ಏನೂ ಸತ್ವವಿಲ್ಲದಂತೆ ಕಂಡಿತು. ನನಗೆ ಅವನನ್ನು ಮುಟ್ಟಲೂ ಆಗಲಿಲ್ಲ.

\chapter{ಪ್ರಾಪಂಚಿಕರು}

\section{\num{೪.} ಯಾವಾಗ ಅವರಿಗೆ ಅಸಮಾಧಾನವಾಗುವುದು?}

ನೋಡಿ, ಇಲ್ಲಿಗೆ ಬರುವವರು ಯಾರೂ ಕಾಣಿಕೆ ಕೊಡಬೇಕಾಗಿಲ್ಲ. ಜದುಮಲ್ಲಿಕ, “ಇತರ ಸಾಧುಗಳು ಯಾವಾಗಲೂ ಹಣ ಕೇಳುತ್ತಾರೆ. ಆದರೆ ನೀವು ಹಾಗೆ ಮಾಡುವುದಿಲ್ಲ” ಎನ್ನುತ್ತಾನೆ. ಪ್ರಾಪಂಚಿಕರಿಗೆ, ಯಾರಾ ದರೂ ಬಂದು ಅವರನ್ನು ಏನಾದರೂ ಕೇಳಿದರೆ ಅಸಮಾಧಾನವಾಗುವುದು.

ಒಂದೂರಿನಲ್ಲಿ ನಾಟಕ ನಡೆಯುತ್ತಿತ್ತು. ಒಬ್ಬ ಅಲ್ಲಿ ಕುಳಿತುಕೊಂಡು ಅದನ್ನು ನೋಡಲಿಚ್ಛಿಸಿದನು. ಅಲ್ಲಿಗೆ ಹೋಗಿ ನೋಡಿದಾಗ ಬಂದವರಿಂದ ಚಂದಾ ಎತ್ತುತ್ತಿದ್ದರು. ಅಲ್ಲಿಂದ ಹಾಗೇ ಜಾರಿದ. ಇನ್ನೊಂದು ಕಡೆ ಮತ್ತೊಂದು ನಾಟಕ ನಡೆಯುತ್ತಿತ್ತು. ಅಲ್ಲಿ ಹೋಗಿ ನೋಡಿದಾಗ ಚಂದಾ ಎತ್ತುತ್ತಿರಲಿಲ್ಲ ಎಂಬುದನ್ನು ಕಂಡ. ಅದನ್ನು ನೋಡಲು ಅನೇಕ ಜನರ ನೂಕು ನುಗ್ಗಲು. ಅವನು ಹೇಗೋ ದಾರಿಯನ್ನು ಮಾಡಿಕೊಂಡು ಮುಂದುಗಡೆ ಹೋಗಿ ನಿಂತನು. ಅಲ್ಲಿ ಒಂದು ಒಳ್ಳೇ ಸ್ಥಳವನ್ನು ಆರಿಸಿಕೊಂಡು ಮೀಸೆ ಹುರಿ ಮಾಡುತ್ತ ಕುಳಿತುಕೊಂಡು ನಾಟಕ ನೋಡಿದ.


\section{\num{೫.} ಹಲ್ಲುಗಳೆಲ್ಲ ಬಿದ್ದಮೇಲೆ}

ನಾನೊಂದು ಕಥೆಯನ್ನು ನಿಮಗೆ ಹೇಳುತ್ತೇನೆ. ಒಬ್ಬ ದುರ್ಗಾಪೂಜೆಯನ್ನು ವೈಭವದಿಂದ ಆಚರಿಸುತ್ತಿದ್ದ. ಮೇಕೆಗಳನ್ನು ಬೆಳಗಿನಿಂದ ಸಾಯಂಕಾಲದ ತನಕ ಬಲಿಕೊಡುತ್ತಿದ್ದರು. ಕೆಲವು ವರ್ಷಗಳ ತರುವಾಯ ಆ ದುರ್ಗಾಪೂಜೆ ಅಷ್ಟು ಸಂಭ್ರಮದಿಂದ ಆಗುತ್ತಿರಲಿಲ್ಲ. ಬಲಿಕೊಡುವುದು ನಿಂತುಹೋಗಿತ್ತು. ಯಾರೋ ಒಬ್ಬರು “ಮಹಾಶಯರೆ, ಇತ್ತೀಚೆಗೆ ನಿಮ್ಮ ದುರ್ಗಾಪೂಜೆ ಅಷ್ಟೊಂದು ಸಂಭ್ರಮದಿಂದ ಏತಕ್ಕೆ ನಡೆಯುತ್ತಿಲ್ಲ” ಎಂದು ಕೇಳಿದರು. ಆತ “ನೀವೇ ನೋಡಿ, ನನ್ನ ಹಲ್ಲುಗಳು ಬಿದ್ದು ಹೋಗಿವೆ” ಎಂದು ಉತ್ತರ ಕೊಟ್ಟ. ಪ್ರಾಪಂಚಿಕರ ವೈರಾಗ್ಯವೂ ಇಂತಹುದೇ.


\section{\num{೬.} ನಿಜವಾಗಿಯೂ ಇಂತಹ ಜನರೂ ಇದ್ದಾರೆ}

ವಿಜ್ಞಾನದಲ್ಲಿ ದೇವರು ಮನುಷ್ಯರಂತೆ ಬರುತ್ತಾನೆ ಎಂದು ಹೇಳಿಲ್ಲ. ಹಾಗಿರುವಾಗ ಅದನ್ನು ಹೇಗೆ ನಂಬುವುದು? ಅಂತಹ ಜನರೂ ಇದ್ದಾರೆ. ಒಂದು ಕಥೆಯನ್ನು ಕೇಳಿ. ದಡ ದಡ ಎಂದು ಒಂದು ಮನೆ ಬಿದ್ದುದನ್ನು ಕಂಡೆ, ಎಂದು ಒಬ್ಬ ತನ್ನ ಸ್ನೇಹಿತನಿಗೆ ಹೇಳಿದ. ಸ್ನೇಹಿತ ಇಂಗ್ಲಿಷ್ ಕಲಿತವನು. ಅವನು “ಸ್ವಲ್ಪ ತಾಳು, ದಿನಪತ್ರಿಕೆಯಲ್ಲಿ ಈ ಸುದ್ದಿ ಇದೆಯೇ ನೋಡೋಣ” ಎಂದ. ಅವನು ಪೇಪರನ್ನು ಓದಿದ. ಅದರಲ್ಲಿ ಮನೆ ಬಿದ್ದದ್ದು ವರದಿಯಾಗಿರ ಲಿಲ್ಲ. ಆಗ ಅವನು ತನ್ನ ಸ್ನೇಹಿತನಿಗೆ “ನಾನು ನಂಬುವುದಿಲ್ಲ. ಅದು ಪೇಪರಿನಲ್ಲಿ ಬಂದಿಲ್ಲ. ಆದಕಾರಣ ಅದು ಸುಳ್ಳು ಇರಬೇಕು” ಎಂದ.


\section{\num{೭.} ಎತ್ತನ್ನು ಬಿಡದೆ ಅನುಸರಿಸುತ್ತಿದ್ದ ನರಿ}

ಒಂದು ಸಲ ನರಿಯೊಂದು ಎತ್ತನ್ನು ನೋಡಿತು. ಅನಂತರ ಅದು ಯಾವಾಗಲೂ ಅದರ ಹತ್ತಿರವೇ ಇರುತ್ತಿತ್ತು. ಎತ್ತು ಮೇಯುತ್ತಿದ್ದಾಗ ನರಿಯೂ ಅದನ್ನು ಹಿಂಬಾಲಿಸುತ್ತಿತ್ತು. ಎತ್ತಿನ ನೇತಾಡುತ್ತಿದ್ದ ಅಂಡಕೋಶ ವನ್ನು ಗಮನಿಸಿದ್ದ ನರಿ ಭಾವಿಸಿತ್ತು “ಅಂಡಕೋಶ ಎಂದಾದರೂ ಬೀಳುವುದು. ಆಗ ಅದನ್ನು ನಾನು ತಿನ್ನುತ್ತೇನೆ” ಎಂದು. ಎತ್ತು ನೆಲದ ಮೇಲೆ ಮಲಗಿದಾಗ, ನರಿಯೂ ಅದರ ಹತ್ತಿರವೇ ಇರುತ್ತಿತ್ತು. ಎತ್ತು ಎದ್ದು ಸಂಚರಿಸಿದರೆ ನರಿಯೂ ಸಂಚರಿಸುತ್ತಿತ್ತು.ಹೀಗೆ ಎಷ್ಟು ದಿನಗಳಾದರೂ ಅಂಡಕೋಶ ಬೀಳಲಿಲ್ಲ. ನರಿ ನಿರಾಶೆಯಿಂದ ಹೊರಟುಹೋಯಿತು. ಇದು ಹೊಗಳುಭಟ್ಟರಿಗೂ ಅನ್ವಯಿಸುವುದು. ಶ್ರೀಮಂತನು ತನ್ನಲ್ಲಿರುವ ಹಣವನ್ನು ಕೊಡುತ್ತಾನೆ ಎಂದು ಅವರು ಆಶಿಸು ವರು. ಆದರೆ ಅವರಿಂದ ಏನೂ ಸಿಕ್ಕುವುದಿಲ್ಲ.


\section{\num{೮.} ಸಾಧುಗಳಂತೆ ತೋರುತ್ತಿದ್ದ ಕಳ್ಳರು}

ನಗನಾಣ್ಯಗಳನ್ನು ಮಾರುತ್ತಿರುವ ವ್ಯಾಪಾರಿ ಒಬ್ಬನಿದ್ದ. ಅವನು ಪಕ್ಕಾ ವೈಷ್ಣವನಂತೆ ಕಾಣುತ್ತಿದ್ದ. ಕತ್ತಿನ ಸುತ್ತಲೂ ಜಪಸರ, ಕೈಯಲ್ಲಿ ಜಪಮಾಲೆ, ಹಣೆಯ ಮೇಲೆಲ್ಲ ಮುದ್ರೆಗಳು ಇದ್ದವು. ಸ್ವಾಭಾವಿಕವಾಗಿ ಜನ ಇವನನ್ನು ನಂಬಿ ವ್ಯಾಪಾರಕ್ಕೆ ಇವನ ಅಂಗಡಿಗೆ ಬರುತ್ತಿದ್ದರು. ಅವನು ಇಷ್ಟು ಸಾಧು ಸ್ವಭಾವದವನಾಗಿರುವುದರಿಂದ ಅವನು ಯಾರಿಗೂ ಮೋಸಮಾಡುವುದಿಲ್ಲ ವೆಂದು ನಂಬಿದ್ದರು. ಗಿರಾಕಿಗಳು ಯಾರಾದರೂ ಅವನ ಅಂಗಡಿಗೆ ಬಂದರೆ ಒಬ್ಬ ಕೆಲಸಗಾರ ‘ಕೇಶವ ಕೇಶವ’ ಎನ್ನುತ್ತಿದ್ದ. ಸ್ವಲ್ಪ ಕಾಲದ ಮೇಲೆ ಮತ್ತೊಬ್ಬ ಕೆಲಸಗಾರ ‘ಗೋಪಾಲ ಗೋಪಾಲ’ ಎನ್ನುತ್ತಿದ್ದ. ಮೂರನೆಯ ವನು ‘ಹರಿ ಹರಿ’ ಎನ್ನುತ್ತಿದ್ದ. ಕೊನೆಯವನು ‘ಹರ ಹರ’ ಎನ್ನುತ್ತಿದ್ದ. ಇವುಗಳೆಲ್ಲ ಭಗವಂತನ ಹಲವು ನಾಮಗಳು. ದೇವರ ಹೆಸರನ್ನು ಇಷ್ಟೊಂದು ಉಚ್ಚರಿಸುತ್ತಿದ್ದುದರಿಂದ ಅಕ್ಕಸಾಲಿಗ ಸತ್ಯವಂತನಾಗಿರಬೇಕೆಂದು ಜನ ಭಾವಿಸಿದರು. ಆದರೆ ಅಕ್ಕಸಾಲಿಗನ ಉದ್ದೇಶವೇನು ಗೊತ್ತೆ? ‘ಕೇಶವ ಕೇಶವ’ ಎನ್ನುತ್ತಿದ್ದವನ ಇಂಗಿತಾರ್ಥ ಈ ಗಿರಾಕಿಗಳು ಎಂತಹವರು ಎಂಬುದು. [‘ಕೇಶವ’ ಶಬ್ದವನ್ನು ಬಂಗಾಳಿ ಭಾಷೆಯಲ್ಲಿ ಕೇಶಬ್ (ಅವರೆಲ್ಲ ಯಾರು) ಎಂದು ಉಚ್ಚರಿಸುತ್ತಾರೆ.] ಗೋಪಾಲ, ಗೋಪಾಲ ಎನ್ನುತ್ತಿದ್ದವನು ಅವರು ಬುದ್ಧಿಮಂಕರು ಎಂದು ಸೂಚನೆ ಕೊಡುತ್ತಿದ್ದನು. ಇವರೆಲ್ಲ ದನಕಾಯುವ ಗುಂಪಿಗೆ ಸೇರಿದವರು ಎಂದು ನಿರ್ಣಯಿಸುತ್ತಿದ್ದರು. ‘ಹರಿ ಹರಿ’ ಎನ್ನುತ್ತಿ ದ್ದವನು ‘ಇವರು ದನದಂತೆ ಬುದ್ಧಿಹೀನರು. ನಾವು ಇವರನ್ನು ಸುಲಿಯೋಣವೆ’ ಎಂದು ಕೇಳುತ್ತಿದ್ದ. ಯಾರು ‘ಹರ ಹರ’ ಎನ್ನು ತ್ತಿದ್ದನೋ ‘ಪರವಾಗಿಲ್ಲ. ಇವರಿಗೆ ಮೋಸಮಾಡಿ’ ಎಂದು ಒಪ್ಪಿಗೆ ಕೊಡುತ್ತಿದ್ದನು. [‘ಹರಿ’, ‘ಹರ’ ಎಂಬ ಶಬ್ದಗಳು ‘ಹೃ’ (ಹರಣ) ಎಂಬ ಧಾತುವಿನಿಂದ ಬಂದಿವೆ].


\section{\num{೯.} ಅವರು ಹೇಗೆ ಜಗಳ ಕಾಯುವರು!}

ನಾವು ಏನನ್ನು ಹೇಳುತ್ತೇವೆಯೋ ಅದೊಂದೇ ಸತ್ಯ, ಇತರರು ಹೇಳುವುದು ತಪ್ಪು ಎಂದು ನಾವು ತಿಳಿಯಬಾರದು. ನಾವು ದೇವರನ್ನು ನಿರಾಕಾರ ಎಂದು ಭಾವಿಸಿದರೆ, ಅವನು ನಿರಾಕಾರನಾಗಿಬಿಡುವನು, ಅವನಿಗೆ ಯಾವ ಆಕಾರವೂ ಇಲ್ಲ ಎಂದು ತಿಳಿಯಬಾರದು. ಅವನಿಗೆ ಆಕಾರವಿದ್ದರೆ ಅದೊಂದೇ ಸತ್ಯ, ನಿರಾಕಾರ ಸುಳ್ಳು ಎಂದು ತಿಳಿಯಬಾರದು. ದೇವರು ಹೇಗಿರುವನೋ ಅದನ್ನು ಮನುಷ್ಯ ಅರಿಯಬಲ್ಲನೆ?

ಶಾಕ್ತರು ಮತ್ತು ವೈಷ್ಣವರು ಈ ವಿಷಯದ ಮೇಲೆಯೇಜಗಳ ಕಾಯುವುದು. ವೈಷ್ಣವರು ‘ನಮ್ಮ ಕೇಶವ ಒಬ್ಬನೇ ಸಂಸಾರದಿಂದ ಪಾರುಮಾಡಿಸುವವನು’ಎನ್ನುವರು. ಶಾಕ್ತರು ನಮ್ಮ ಭಗವತಿಯೊಬ್ಬಳೆ ಪಾರುಮಾಡುವವಳು ಎಂದು ತಿಳಿಯುವರು. ನಾನು ಒಂದು ಸಲ ವೈಷ್ಣವಚರಣನೆಂಬ ಪಂಡಿತನನ್ನು, ಮಥುರಬಾಬು (ರಾಣಿ ರಾಸಮಣಿಯ ಅಳಿಯ) ಬಳಿಗೆ ಕರೆದುಕೊಂಡು ಹೋದೆ. ಮಥುರಬಾಬು ಅವನನ್ನು ಗೌರವದಿಂದ ಬರಮಾಡಿಕೊಂಡು ಬೆಳ್ಳಿಯ ತಟ್ಟೆಯಲ್ಲಿ ಉಪಹಾರ ವನ್ನು ಬಡಿಸಿದನು. ವೈಷ್ಣವಚರಣನು ವೈಷ್ಣವಶಾಸ್ತ್ರದಲ್ಲಿ ಪಾರಂಗತ ಮತ್ತು ಅವನಿಗೆ ವೈಷ್ಣವರ ಮೇಲೆ ಬಹಳ ಆದರ. ಆದರೆ ಮಥುರ ಭಗವತಿಯ ನಿಷ್ಠಾವಂತ ಭಕ್ತ. ಅವರಿಬ್ಬರೂ ಸ್ನೇಹಿತರಂತೆ ಮಾತನಾಡುತ್ತಿದ್ದರು. ಆಗ ವೈಷ್ಣವಚರಣ ‘ಕೇಶವ ಒಬ್ಬನೇ ನಮ್ಮನ್ನು ಭವಸಾಗರದಿಂದ ಪಾರುಮಾಡು ವವನು’ ಎಂದ. ಇದನ್ನು ಕೇಳಿದೊಡನೆಯೆ ಮಥುರನ ಮುಖ ಕೆಂಪಾಯಿತು, ‘ನೀನು ಎಂಥ ತಿಳಿಗೇಡಿ’ ಎಂದು ರೇಗಾಡಿದನು. ಅವನು ಶಾಕ್ತ, ಅವನು ಹಾಗೆ ಅಂದಿದ್ದು ಸ್ವಾಭಾವಿಕವಲ್ಲವೇ? ನಾನು ವೈಷ್ಣವಚರಣನಿಗೆ ಸುಮ್ಮನಿರಲು ಸಂಜ್ಞೆ ಮಾಡಿದೆ.


\section{\num{೧೦.} ಸಂಸಾರಿಗಳಿಗೆ ಶಾಸ್ತ್ರಜ್ಞಾನದ ಪರಿಚಯವಿರುವುದಿಲ್ಲ}

ಒಬ್ಬ ಭಾಗವತವನ್ನು ಕಲಿಯಬೇಕೆಂದು ಭಾಗವತ ಪಂಡಿತನೊಬ್ಬನಿಗಾಗಿ ಅರಸುತ್ತಿದ್ದ. ಆತನ ಸ್ನೇಹಿತ “ನನಗೆ ಒಬ್ಬ ಪ್ರಸಿದ್ಧ ಪಂಡಿತನ ಪರಿಚಯ ವಿದೆ. ಆದರೆ ಅವನಲ್ಲಿ ಒಂದು ದೋಷವಿದೆ. ಅವನು ಬೇಸಾಯದಲ್ಲಿ ಮುಳು ಗಿರುವನು. ಅವನಿಗೆ ನಾಲ್ಕು ನೇಗಿಲುಗಳು, ಎಂಟು ಎತ್ತುಗಳು ಇವೆ. ಅವನು ಯಾವಾಗಲೂ ಅದರಲ್ಲೇ ಮುಳುಗಿರುವನು. ಅವನಿಗೆ ವಿರಾಮವೇ ಇಲ್ಲ” ಎಂದನು. ಆಗ ಕಲಿಯಬೇಕೆಂದುಕೊಂಡವನು, “ಪುರುಸತ್ತು ಇಲ್ಲದ ಪಂಡಿತ ನನಗೆ ಬೇಕಾಗಿಲ್ಲ. ಎತ್ತು ನೇಗಿಲುಗಳಲ್ಲಿ ನಿರತನಾದ ಭಾಗವತ ಪಂಡಿತ ನನಗೆ ಬೇಕಾಗಿಲ್ಲ. ನಿಜವಾಗಿ ಭಾಗವತವನ್ನೇ ಬೋಧಿಸುವ ಪಂಡಿತ ನನಗೆ ಬೇಕು” ಎಂದ. ಸಾಂಸಾರಿಕತೆ ಮತ್ತು ಆಧ್ಯಾತ್ಮಿಕತೆ ಒಟ್ಟಿಗೆ ಇರಲಾರವು.


\section{\num{೧೧.} ಕುಂಬಳಕಾಯಿ ಒಡೆಯುವವ}

ಕೆಲವು ಮನೆಗಳಲ್ಲಿ ಹಿರಿಯನೊಬ್ಬನು ಇರುತ್ತಾನೆ. ಯಾವಾ ಗಲೂ ಅವನಿಗೆ ಮಕ್ಕಳನ್ನು ಮುದ್ದಾಡಿಸುವುದೇ ಕೆಲಸ. ಅವನು ಒಂದು ಹಜಾರದಲ್ಲಿ ಕುಳಿತುಕೊಂಡು ಗುಡುಗುಡಿ ಸೇದುತ್ತಿರು ತ್ತಾನೆ. ಅವನಿಗೆ ಏನೂ ಕೆಲಸ ಇಲ್ಲದೆ ಇರುವುದರಿಂದ ಶುದ್ಧ ಸೋಮಾರಿ ಯಾಗಿ ಕಾಲ ಕಳೆಯುವನು. ಯಾವಾಗಲಾದರೊಮ್ಮೆ ಅಡಿಗೆ ಮನೆಗೆ ಹೋಗಿ ಅಲ್ಲಿ ಕುಂಬಳಕಾಯಿಯನ್ನು ಒಡೆಯುವನು. ಏಕೆಂದರೆ ಹೆಂಗಸರು ಅದನ್ನು ಮಾಡಕೂಡದು ಎಂಬ ನಿಷೇಧವಿದೆ. ಅದಕ್ಕಾಗಿ ಈ ಮುದುಕನನ್ನು ಕರೆಯು ವರು. ಅವನು ಮಾಡುವ ಉಪಕಾರ ಇದೊಂದೆ. ಅದಕ್ಕಾಗಿ ಅವನಿಗೆ ಕುಂಬಳ ಕಾಯಿ ಒಡೆಯುವ ಮುದುಕ ಎಂದು ಅಡ್ಡಹೆಸರು ಬಂದಿದೆ. ಅವನು ಈ ಸಂಸಾರದಲ್ಲಿಯೂ ಯಾವ ಕೆಲಸಕ್ಕೂ ಬಾರದವನಾಗಿರುತ್ತಾನೆ ಮತ್ತು ಭಕ್ತಿಗೆ ಸಂಬಂಧಪಟ್ಟ ಜಪ ಧ್ಯಾನಾದಿಗಳನ್ನೂ ಮಾಡುವುದಿಲ್ಲ. ಈ ರೀತಿ ಇರುವುದು ಒಳ್ಳೆಯದಲ್ಲ.


\section{\num{೧೨.} ಜೀವನದಲ್ಲಿ ಎಲ್ಲವೂ ಬೇಕು}

ದುಷ್ಟರ ಆವಶ್ಯಕತೆಯೂ ಇದೆ. ಒಂದು ಸಲ ಗುತ್ತಿಗೆದಾರರು ದಂಗೆ ಎದ್ದರು. ಜಮೀನ್​ದಾರನು ಗೋಲೂಕ್ ಚೌಧುರಿ ಎಂಬ ಪಟಿಂಗನನ್ನು ಕಳು ಹಿಸಬೇಕಾಯಿತು. ಅವನು ದುಷ್ಟರನ್ನು ಸದೆಬಡಿಯುವುದರಲ್ಲಿ ಪಳಗಿದ ಕೈ. ಅವನ ಹೆಸರನ್ನು ಕೇಳಿದರೇನೇ ಆಳುಗಳು ನಡುಗುತ್ತಿದ್ದರು.

ಈ ಪ್ರಪಂಚದಲ್ಲಿ ಎಲ್ಲವೂ ಬೇಕು. ಒಂದು ದಿನ ಸೀತೆ ರಾಮನಿಗೆ ಹೇಳಿ ದಳು: “ರಾಮ, ಅಯೋಧ್ಯಾನಗರದಲ್ಲಿರುವ ಪ್ರತಿಯೊಂದು ಮನೆಯೂ ಒಂದು ಬಂಗಲೆಯಾಗಬೇಕು. ಹಲವು ಮನೆಗಳು ಹಾಳಾಗಿ ಬಿದ್ದುಹೋಗು ತ್ತಿವೆ.” ಆಗ ರಾಮ, “ಸೀತೆ, ಮನೆಗಳೆಲ್ಲ ಚೆನ್ನಾಗಿದ್ದರೆ ಗಾರೆಕಾರರು ಏನು ಮಾಡುವುದು?” ದೇವರು ಬಗೆಬಗೆಯ ವಸ್ತುಗಳನ್ನು ಮಾಡಿದ್ದಾನೆ. ಅವನು ಒಳ್ಳೆಯ ಮರಗಳನ್ನೂ, ವಿಷವಿರುವ ಮರಗಳನ್ನೂ, ಕೆಲಸಕ್ಕೆ ಬಾರದ ಮುಳ್ಳು ಗಿಡಗಳನ್ನೂ ಎಲ್ಲವನ್ನೂ ಮಾಡಿರುವನು. ಪ್ರಾಣಿಗಳಲ್ಲಿ ಒಳ್ಳೆಯವೂ ಕೆಟ್ಟವೂ ಇವೆ. ಸಿಂಹ, ಹುಲಿ, ಹಾವು ಇವುಗಳನ್ನೆಲ್ಲ ಮಾಡಿರುವನು.


\section{\num{೧೩.} ಮನುಷ್ಯರಲ್ಲಿ ಹಲವು ಬಗೆ}

ನಾಲ್ಕು ಬಗೆಯ ವ್ಯಕ್ತಿಗಳಿರುವರು: ಸಂಸಾರಕ್ಕೆ ಬದ್ಧರಾದವರು, ಮುಮುಕ್ಷುಗಳು, ಮುಕ್ತರು ಮತ್ತು ನಿತ್ಯಮುಕ್ತರು. ನಿತ್ಯಮುಕ್ತರ ಗುಂಪಿನಲ್ಲಿ ನಾರದರನ್ನು ತರಬಹುದು. ಅವರು ಪ್ರಪಂಚದಲ್ಲಿ ಪರಹಿತಕ್ಕಾಗಿ ಇದ್ದರು. ಅವರು ಜೀವರಿಗೆ ಆಧ್ಯಾತ್ಮಿಕ ಸತ್ಯ ಗಳನ್ನು ವಿವರಿಸುತ್ತಾರೆ. ಯಾರು ಬದ್ಧರೊ, ಅವರು ಸಂಸಾರದಲ್ಲಿ ಮುಳುಗಿರು ತ್ತಾರೆ. ದೇವರನ್ನು ಮರೆತಿರುತ್ತಾರೆ. ತಪ್ಪಿಯೂ ದೇವರನ್ನು ಚಿಂತಿಸುವುದಿಲ್ಲ. ಮುಮುಕ್ಷುಗಳು ಸಂಸಾರದಿಂದ ಪಾರಾಗಲು ಯತ್ನಿಸುತ್ತಾರೆ. ಅವರಲ್ಲಿ ಕೆಲ ವರು ಗೆಲ್ಲುವರು, ಕೆಲವರು ಸೋಲುವರು. ಸಾಧು ಮಹಾತ್ಮರಂತಹ ಮುಕ್ತ

ಮೀನನ್ನು ಹಿಡಿಯುವುದಕ್ಕೆ ಒಂದು ಬಲೆಯನ್ನು ಎಸೆಯುವರು. ಕೆಲವು ಮೀನುಗಳು ಎಷ್ಟು ಜಾಗೃತವೆಂದರೆ, ಅವು ಬಲೆಗೆ ಬೀಳುವುದೇ ಇಲ್ಲ. ಅವು ನಿತ್ಯಮುಕ್ತರಿದ್ದಂತೆ. ಅನೇಕ ಮೀನುಗಳು ಬಲೆಯೊಳಗೆ ಬೀಳುತ್ತವೆ. ಅವು ಗಳಲ್ಲಿ ಕೆಲವು ಬಲೆಯಿಂದ ಪಾರಾಗಲು ಯತ್ನಿಸುತ್ತವೆ. ಆದರೆ ಹಾಗೆ ಪ್ರಯತ್ನಿ ಸುವ ಮೀನುಗಳೆಲ್ಲ ಮುಕ್ತವಾಗಲಾರವು. ಎಲ್ಲೊ ಕೆಲವು ಚಂಗ್ ಎಂದು ಬಲೆಯಿಂದ ಹಾರಿಕೊಂಡು ಹೋಗುತ್ತವೆ. ಆಗ ಬೆಸ್ತ “ನೋಡು, ಒಂದು ದೊಡ್ಡ ಮೀನು ತಪ್ಪಿಸಿಕೊಂಡು ಹೋಗುತ್ತಿದೆ” ಎಂದು ಕೂಗುತ್ತಾನೆ. ಬಲೆ ಯಲ್ಲಿ ಬಿದ್ದ ಮುಕ್ಕಾಲು ಪಾಲು ಮೀನುಗಳು ಪಾರಾಗಲಾರವು. ಅವು ಅದಕ್ಕೆ ಪ್ರಯತ್ನವನ್ನೂ ಮಾಡುವುದಿಲ್ಲ. ಅದರ ಬದಲು ಕೆಸರಿನ ಆಳಕ್ಕೆ ಹೋಗಿ ಬಲೆಯನ್ನು ಬಾಯಲ್ಲಿ ಬಿಗಿಯಾಗಿ ಕಚ್ಚಿಕೊಂಡು, “ನಾವು ಇನ್ನು ಯಾವುದಕ್ಕೂ ಅಂಜಬೇಕಾಗಿಲ್ಲ. ಈಗ ಸುರಕ್ಷಿತವಾಗಿದ್ದೇವೆ” ಎಂದು ಭಾವಿಸುವುವು. ಆದರೆ ಬೆಸ್ತ ಅವುಗಳನ್ನು ಮೇಲಕ್ಕೆ ಸೆಳೆಯುತ್ತಾನೆ ಎಂಬುದನ್ನು ಅವು ಅರಿಯವು. ಅವೇ ಬದ್ಧ ಜೀವಿಗಳಿದ್ದಂತೆ.

\chapter{ಸಂಸಾರದ ಬೇಗುದಿ}

\section{\num{೧೪. } ನಮ್ಮ ದುಃಖಗಳಿಗೆಲ್ಲ ಮೂಲಕಾರಣ}

ಒಂದೂರಿನಲ್ಲಿ ಬೆಸ್ತನೊಬ್ಬ ಮೀನನ್ನು ಹಿಡಿಯುತ್ತಿದ್ದ. ಒಂದು ಹದ್ದು ಹಾರಿ ಬಂದು ಒಂದು ಮೀನನ್ನು ಕಚ್ಚಿಕೊಂಡಿತು. ಮೀನನ್ನು ಕಂಡ ಒಡನೆಯೆ ನೂರಾರು ಕಾಗೆಗಳು ಅದನ್ನು ತೆಗೆದುಕೊಳ್ಳುವುದಕ್ಕಾಗಿ ಕೂಗಾಟ



\section{\num{೧೫. } ಎಲ್ಲಾ ಒಂದು ಕೌಪೀನಕ್ಕಾಗಿ}

ಸಾಧುವೊಬ್ಬನು ತನ್ನ ಗುರುವಿನ ಅಣತಿಯಂತೆ ಜನರಿಂದ ಬಹುದೂರ ಒಂದು ಗುಡಿಸಲನ್ನು ಹಾಕಿಕೊಂಡು, ಅಲ್ಲಿ ಜಪ ಧ್ಯಾನ ಮಾಡುತ್ತಿದ್ದ. ಪ್ರತಿ ದಿನವೂ ಬೆಳಗ್ಗೆ ಸ್ನಾನವಾದ ಮೇಲೆ ತನ್ನ ಬಟ್ಟೆ ಮತ್ತು ಕೌಪೀನವನ್ನು ಒಣಗಿಸಲು ಗುಡಿಸಲಿನ ಹತ್ತಿರ ಇರುವ ಮರಕ್ಕೆ ಕಟ್ಟುತ್ತಿದ್ದ. ಒಂದು ದಿನ ಹಳ್ಳಿಯಲ್ಲಿ ಭಿಕ್ಷೆ ಬೇಡಿಕೊಂಡು ಹಿಂತಿರುಗಿದಾಗ, ಇಲಿಗಳು ಇವನ ಕೌಪೀನ ವನ್ನು ಹರಿದಿದ್ದವು. ಮಾರನೆಯ ದಿನ ಹೊಸ ಕೌಪೀನಕ್ಕಾಗಿ ಭಿಕ್ಷೆ ಬೇಡಬೇಕಾ ಯಿತು. ಕೆಲವು ದಿವಸಗಳಾದಮೆಲೆ ತನ್ನ ಬಟ್ಟೆಯನ್ನು ಗುಡಿಸಿಲಿನ ಮೇಲೆ ಇಟ್ಟು ಎಂದಿನಂತೆ ಭಿಕ್ಷೆಗೆ ಹೋದ. ಹಿಂತಿರುಗಿ ಬಂದಾಗ ಇವನ ಬಟ್ಟೆಯನ್ನು ಇಲಿಗಳು ಹರಿದು ಹಾಕಿದ್ದವು. ಆಗ ಅವನಿಗೆ ತುಂಬಾ ವ್ಯಸನವಾಯಿತು. ಈಗ ನಾನು ಯಾರನ್ನು ಭಿಕ್ಷೆ ಬೇಡಲಿ ಈ ಚಿಂದಿಗಾಗಿ ಎಂದು ಯೋಚಿಸಿದ. ಆದರೂ ಬಟ್ಟೆ ಹರಿದು ಹೋದ ವಿಷಯವನ್ನು ಹಳ್ಳಿಯವರಿಗೆ ಹೋಗಿ ಹೇಳಿದನು. ಇವನು ಹೇಳಿದುದನ್ನೆಲ್ಲ ಕೇಳಿದ ಮೇಲೆ “ಯಾರು ನಿಮಗೆ ಪ್ರತಿ ದಿನವೂ ಹೊಸ ಬಟ್ಟೆಯನ್ನು ಕೊಡುತ್ತಾರೆ. ನೀವು ಒಂದು ಕೆಲಸ ಮಾಡಿ. ನೀವೊಂದು ಬೆಕ್ಕನ್ನು ಸಾಕಿ. ಆಗ ಇಲಿಯ ಕಾಟದಿಂದ ಪಾರಾಗುತ್ತೀರಿ” ಎಂದರು. ಅಂದಿನಿಂದ ಇಲಿಯ ಕಾಟದಿಂದ ಪಾರಾಗಿ ಅವನು ಸುಖವಾಗಿದ್ದ. ಅವನ ಆನಂದಕ್ಕೆ ಪಾರವೇ ಇರಲಿಲ್ಲ. ಸಾಧು ಈಗ ಬೆಕ್ಕನ್ನು ಸಾಕುವುದಕ್ಕಾಗಿ ಹಾಲಿಗೆ ಭಿಕ್ಷೆ ಬೇಡಿದ. ಕೆಲವು ದಿನಗಳಾದ ಮೇಲೆ ಹಳ್ಳಿಯವರು “ಸಾಧುಗಳೆ, ನಿಮಗೆ ಪ್ರತಿ ದಿನವೂ ಹಾಲನ್ನು ಕೊಡಬೇಕಾಗಿದೆ. ಕೆಲವು ದಿನಗಳು ಬೇಕಾದರೆ ನಿಮಗೆ ಅದನ್ನು ಕೊಡಬಹುದು. ಯಾರು ತಾನೇ ವರುಷವೆಲ್ಲ ನಿಮಗೆ ಹಾಲು ಕೊಟ್ಟಾರು? ನೀವು ಒಂದು ಕೆಲಸ ಮಾಡಿ. ಒಂದು ಹಸುವನ್ನು ಸಾಕಿ. ಆಗ ನೀವೂ ಕುಡಿಯಬಹುದು. ಬೆಕ್ಕಿಗೂ ಕೊಡಬಹುದು.” ಕೆಲವು ದಿನಗಳ ನಂತರ ಹಸುವನ್ನು ಸಾಕಿದನು. ಇನ್ನು ಮೇಲೆ ಹಾಲಿಗೆ ಯಾರನ್ನೂ ಭಿಕ್ಷೆ ಬೇಡಬೇಕಾಗಿರಲಿಲ್ಲ. ಕಾಲಕ್ರಮೇಣ ಹಸುವಿನ ಮೇವಿಗಾಗಿ ಹುಲ್ಲನ್ನು ಭಿಕ್ಷೆ ಬೇಡಲು ಹೊರಟ. ಅದಕ್ಕಾಗಿ ಸುತ್ತಮುತ್ತ ಇರುವ ಹಳ್ಳಿಯವರನ್ನು ಕೇಳಬೇಕಾಯಿತು. ಹಳ್ಳಿಯವರು “ನಿಮ್ಮ ಗುಡಿಸಿಲಿನ ಬಳಿ ಖಾಲಿ ಜಮೀನು ಇದೆ. ಅದನ್ನು ಬೇಸಾಯ ಮಾಡಿದರೆ ನೀವು ಮೇವಿಗಾಗಿ ಯಾರನ್ನೂ ಕೇಳಬೇಕಾಗಿಲ್ಲ” ಎಂದರು. ಅವರ ಬುದ್ಧಿವಾದದಂತೆ ಸಾಧು ಬೇಸಾಯ ಪ್ರಾರಂಭಿಸಿದ. ಕ್ರಮೇಣ ಜಮೀನನ್ನು ನೋಡಿಕೊಳ್ಳುವುದಕ್ಕೆ ಕೆಲವು ಆಳುಗಳನ್ನು ಗೊತ್ತು ಮಾಡಿಕೊಂಡನು. ಆಮೇಲೆ ಬಂದ ದವಸಧಾನ್ಯಗಳನ್ನು ರಕ್ಷಣೆ ಮಾಡಲು ಒಂದು ಕಣಜವನ್ನು ಕಟ್ಟಿಕೊಂಡ. ಕಾಲಕ್ರಮೇಣ ಅವನೊಬ್ಬ ಜಮೀನ್​ದಾರನಾದ. ಕೊನೆಗೆ ಗೃಹಕೃತ್ಯಗಳನ್ನು ನೋಡಿಕೊಳ್ಳುವುದಕ್ಕಾಗಿ ಒಬ್ಬ ಳನ್ನು ಮದುವೆಯಾದ. ಈಗ ಇತರರಂತೆ ಅವನೂ ಒಬ್ಬ ಗೃಹಸ್ಥನಾದ.

ಕೆಲವು ದಿನಗಳಾದ ಮೇಲೆ ಇವನ ಗುರುಗಳು ಇವನನ್ನು ನೋಡಲು ಬಂದರು. ಸುತ್ತಲೂ ಆಳುಕಾಳುಗಳು ಇದ್ದರು. ಅವರಿಗೆ ಆಶ್ಚರ್ಯವಾಗಿ, “ಒಬ್ಬ ಸಂನ್ಯಾಸಿ ಇಲ್ಲಿ ವಾಸವಾಗಿದ್ದ, ಈಗ ಅವನು ಎಲ್ಲಿರುವನು ಹೇಳು ತ್ತೀರಾ?” ಎಂದು ಕೇಳಿದರು. ಆಳುಗಳಿಗೆ ಏನು ಹೇಳಬೇಕೋ ಗೊತ್ತಾಗಲಿಲ್ಲ. ಗುರುಗಳೇ ಧೈರ್ಯದಿಂದ ಮನೆಯೊಳಗೆ ಹೋಗಿ ನೋಡಿದರು. ಅಲ್ಲಿ ಶಿಷ್ಯ ಸಿಕ್ಕಿದ. ಗುರುಗಳು “ಏನು ಮಗು ಇದೆಲ್ಲ” ಎಂದು ಕೇಳಿದರು. ಶಿಷ್ಯ ತುಂಬಾ ನಾಚಿಕೆಯಿಂದ ತಲೆ ತಗ್ಗಿ ಗುರುವಿಗೆ ಅಡ್ಡಬಿದ್ದು “ಗುರುಗಳೇ ಇದೆಲ್ಲ ಒಂದು ಕೌಪೀನಕ್ಕಾಗಿ” ಎಂದನು. ಒಂದು ಅಲ್ಪ ಆಸಕ್ತಿಯೂ ದೊಡ್ಡ ಬಂಧನಕ್ಕೆ ಕಾರಣವಾಗಬಲ್ಲದು.


\section{\num{೧೬.} ನಮ್ಮ ಸಂಸಾರದ ಆನಂದದ ಹಿಂದೆ ಇರುವ ವ್ಯಾಘ್ರ}

ದೇವರು ನಾವು ಏನು ಬೇಡಿದರೂ ಅದನ್ನು ಕೊಡುವ ಕಲ್ಪವೃಕ್ಷ. ಒಬ್ಬನ ಮನಸ್ಸು ಸಾಧನೆಯಿಂದ ಶುದ್ಧವಾದಾಗ ಪ್ರಾಪಂಚಿಕ ಆಸೆಗಳಿಗೆ ಕೈ ಒಡ್ಡ ಬಾರದು.

ಈ ಒಂದು ಕಥೆಯನ್ನು ಕೇಳಿ. ಒಬ್ಬ ಪ್ರಯಾಣಿಕನು ವಿಶಾಲ ಬಯಲಿನಲ್ಲಿ ಹೋಗುತ್ತಿದ್ದನು. ಅವನು ಕೆಲವು ಗಂಟೆಗಳ ಕಾಲ ಬಿಸಿಲಿನಲ್ಲಿ ನಡೆದುದರಿಂದ ತುಂಬಾ ದಣಿವಾಯಿತು. ಮೈಯೆಲ್ಲಾ ಬೆವೆತಿತ್ತು. ಒಂದು ಮರದ ಕೆಳಗೆ ಮಿಶ್ರಮಿಸಿಕೊಳ್ಳುವುದಕ್ಕೆ ಕುಳಿತುಕೊಂಡ. ನನಗೆ ಮಲಗಿಕೊಳ್ಳುವುದಕ್ಕೆ ಒಂದು ಮೆತ್ತನೆಯ ಹಾಸಿಗೆ ಇದ್ದರೆ ಎಷ್ಟು ಚೆನ್ನಾಗಿತ್ತು ಎಂಬ ಆಲೋಚನೆ ಮಾಡಿದ. ತಕ್ಷಣ ಅದು ಬಂತು. ತಾನೊಂದು ಕಲ್ಪವೃಕ್ಷದ ಕೆಳಗೆ ಮಲಗಿರುವೆನು ಎಂದು ಅವನಿಗೆ ಗೊತ್ತಾಗ ಲಿಲ್ಲ. ಅವನಿಗೆ ತುಂಬಾ ಆಶ್ಚರ್ಯವಾಗಿ ಆ ಮರದ ಕೆಳಗೆ ವಿಶ್ರಮಿಸಿದ. ಒಬ್ಬ ತರುಣಿ ಬಂದು ನನ್ನ ಕಾಲನ್ನು ಹಿಸುಕಿದರೆ ಎಷ್ಟು ಚೆನ್ನಾಗಿರುವುದು ಎಂದು ಯೋಚಿಸಿದ. ಆಲೋಚನೆ ಮನಸ್ಸಿನ ಒಳಗೆ ಬಂದೊಡನೆ ಒಬ್ಬ ಸುಂದರ ತರುಣಿ ಬಂದು ಇವನ ಕಾಲನ್ನು ಒತ್ತತೊಡಗಿದಳು. ಆಗ ಅವನಿಗೆ ಹೊಟ್ಟೆ ಹಸಿವಾಯಿತು. “ನನಗೆ ಏನು ಬೇಕೊ ಅದೆಲ್ಲ ಸಿಕ್ಕಿದೆ. ನನಗೆ ಊಟಕ್ಕೆ ಏನಾದರೂ ಸಿಕ್ಕೀತೆ” ಎಂದು ಯೋಚಿಸಿದ. ಆಲೋಚನೆ ಹೊಳೆದ ತಕ್ಷಣವೆ ರುಚಿರುಚಿಯಾದ ತಿಂಡಿತೀರ್ಥಗಳು ಅವನ ಮುಂದೆ ಬಂದವು. ಅವನ್ನು ತಿನ್ನಲು ಶುರುಮಾಡಿದ. ತೃಪ್ತಿಯಾಗುವ ತನಕ ತಿಂದಾದ ಮೇಲೆ ಹಾಸಿಗೆಯ ಮೇಲೆ ಮಲಗಿಕೊಂಡ. ಬೆಳಗಿನಿಂದ ತನಗೆ ಆದ ಅನುಭವವನ್ನು ಮೆಲುಕು ಹಾಕುತ್ತಿರುವಾಗ, “ಒಂದು ಹುಲಿ ಬಂದು ನನ್ನ ಮೇಲೆ ಬಿದ್ದರೆ ಏನು ಗತಿ” ಎಂದು ಯೋಚಿಸಿದ. ತಕ್ಷಣವೆ ದೊಡ್ಡ ಹುಲಿಯೊಂದು ಅವನ ಮೇಲೆ ಬಿದ್ದು ಕತ್ತನ್ನು ಮುರಿದು ರಕ್ತವನ್ನು ಕುಡಿಯಲಾರಂಭಿಸಿತು. ಈ ಪ್ರಯಾಣಿಕ ಪಾಪ ಹೀಗೆ ಸತ್ತ.

ಸಾಧಾರಣ ಜನರ ಸ್ವಭಾವವೇ ಹೀಗೆ. ನೀವು ಧ್ಯಾನ ಮಾಡುತ್ತಿರುವಾಗ ದ್ರವ್ಯ, ಪ್ರಾಪಂಚಿಕ ವಸ್ತುಗಳು, ಕೀರ್ತಿ ಇವುಗಳನ್ನು ಆಶಿಸಿದರೆ ನಿಮ್ಮ ಆಸೆಯೇನೋ ಈಡೇರುವುದು. ಆದರೆ ಅದರ ಹಿಂದೆ ಭಯಾನಕವಾದ ಹುಲಿ ಇರುತ್ತದೆ. ಆ ಹುಲಿಯೇ–ರೋಗ, ಬಂಧು ಬಳಗದವರ ಸಾವು, ನಿಮ್ಮ ಮಾನಮರ್ಯಾದೆ, ಹಣ ಮುಂತಾದುವುಗಳ ನಷ್ಟ ಇತ್ಯಾದಿಗಳು. ಇವುಗಳು ಹುಲಿಗಿಂತ ಸಾವಿರ ಪಾಲು ಉಗ್ರವಾದವುಗಳು.


\section{\num{೧೭.} ಪ್ರಾಪಂಚಿಕತೆಯ ದುರ್ವಾಸನೆ}

ಒಮ್ಮೆ ಒಬ್ಬ ಬೆಸ್ತರ ಹೆಂಗಸು ಮೀನನ್ನೆಲ್ಲ ಮಾರಿಯಾದ ಮೇಲೆ, ಹೂವಿನ ತೋಟದಲ್ಲಿ ಒಂದು ರಾತ್ರಿ ಕಳೆಯಬೇಕಾಯಿತು. ಅವಳು ತನ್ನ ಖಾಲಿ ಮೀನಿನ ಬುಟ್ಟಿಯನ್ನು ತಂದಿದ್ದಳು. ಹೂವನ್ನು ಇಟ್ಟಿರುವ ಒಂದು ಕಡೆ ಆ ರಾತ್ರಿ ಕಳೆಯಲು ಅವಕಾಶ ಸಿಕ್ಕಿತು. ಈ ಹೂವುಗಳ ಪರಿಮಳದಿಂದ ಆಕೆಗೆ ಬಹಳ ಕಾಲದವರೆಗೆ ನಿದ್ರೆ ಬರಲಿಲ್ಲ. ಸುಮ್ಮನೆ ಹೊರಳಾಡುತ್ತಿದ್ದಳು. ಮನೆಯ ಯಜಮಾನಿ “ಏತಕ್ಕೆ ಇಷ್ಟೊಂದು



\section{\num{೧೮.} ಪ್ರಾಪಂಚಿಕ ವಸ್ತುಗಳು ಎಂದೆಂದಿಗೂ ನಿಮ್ಮವಲ್ಲ}

ಒಬ್ಬ ಶ್ರೀಮಂತ ತನ್ನ ಮನೆಯನ್ನು ನೋಡಿಕೊಳ್ಳುವಂತೆ ಒಬ್ಬ ಆಳನ್ನು ಗೊತ್ತುಮಾಡಿದ. ಯಾರಾದರೂ ಇವನನ್ನು ಇದೆಲ್ಲ ಯಾರದು ಎಂದು ಕೇಳಿದಾಗ, “ಮಹಾಶಯರೇ ಇದೆಲ್ಲ ನನ್ನದು. ಈ ಮನೆ, ಈ ತೋಟ ಇವುಗಳೆಲ್ಲ ನನ್ನವು” ಎನ್ನುತ್ತಿದ್ದ. ಹೀಗೆ ಹೇಳುತ್ತ ತನ್ನ ಸಮಾನ ಯಾರೂ ಇಲ್ಲ ಎಂದು ಜಂಭಕೊಚ್ಚಿಕೊಳ್ಳುತ್ತಿದ್ದ. ಒಂದು ದಿನ ಅವನು ತನ್ನ ಯಜಮಾನನ ಕೊಳದೊಳಗೆ ಇರುವ ಮೀನನ್ನು ಅಪ್ಪಣೆ ಇಲ್ಲದೆ ಕದ್ದ. ದುರದೃಷ್ಟವಶಾತ್ ಆ ಸಮಯದಲ್ಲಿಯೇ ಜಮೀನ್​ದಾರ ಅಲ್ಲಿಗೆ ಬಂದ. ತನ್ನ ಆಳು ಮಾಡುತ್ತಿರುವ ಮೋಸ ಗೊತ್ತಾಗಿ ಅವನು ಏನನ್ನು ಕದ್ದಿ ದ್ದನೊ ಅದನ್ನೆಲ್ಲ ಕಸಿದುಕೊಂಡ. ಆತ ತನ್ನ ಹರಕಲು ಮುರುಕಲು ಸಾಮಾನು ಇಟ್ಟಿದ್ದ ಹಳೆಯ ಪೆಟ್ಟಿಗೆಯನ್ನು ಕೂಡ ತೆಗೆದುಕೊಂಡು ಹೋಗಲು ಆಗಲಿಲ್ಲ. ಅದೊಂದೇ ಇವನ ಆಸ್ತಿಯಾಗಿತ್ತು. ದುರಹಂಕಾರದ ಪರಿಣಾಮ ಇದು.


\section{\num{೧೯.} ನಮ್ಮ ಆಸೆಯ ಕೊಪ್ಪರಿಗೆಯನ್ನು ಯಾರೂ ತೃಪ್ತಿಪಡಿಸಲಾರರು}

ದೆವ್ವವಿರುವ ಒಂದು ಮರದ ಕೆಳಗೆ ಒಬ್ಬ ಕ್ಷೌರಿಕ ಹೋಗುತ್ತಿದ್ದ. ಆಗ ಒಂದು ಅಶರೀರವಾಣಿ “ನೀನು ಏಳು ಕೊಪ್ಪರಿಗೆ ಚಿನ್ನವನ್ನು ತೆಗೆದುಕೊಳ್ಳು ವೆಯಾ?” ಎಂದು ಕೇಳಿಸಿತು. ಕ್ಷೌರಿಕ ಸುತ್ತಲೂ ನೋಡಿದ. ಅವನಿಗೆ ಯಾರೂ ಕಾಣಲಿಲ್ಲ. ಆದರೆ ಏಳು ಕೊಪ್ಪರಿಗೆ ಹಣದ ಆಸೆ ಅವನನ್ನು ವಶಮಾಡಿ ಕೊಂಡಿತು. “ಆಗಲಿ, ನಾನು ಅದನ್ನು ಸ್ವೀಕರಿಸುತ್ತೇನೆ” ಎಂದ. “ನೀನು ಮನೆಗೆ ಹೋಗು. ಅಲ್ಲಿ ನಾನು ಏಳು ಕೊಪ್ಪರಿಗೆ ಚಿನ್ನವನ್ನು ಇಟ್ಟಿರುವೆನು” ಎಂದಿತು. ಕ್ಷೌರಿಕ ಬಹಳ ವೇಗದಿಂದ ಇದೇನು ನಿಜವೋ ಸುಳ್ಳೋ ಎನ್ನುವುದನ್ನು ನೋಡು ವುದಕ್ಕೆ ಓಡಿ ಹೋದ. ಮನೆಗೆ ಹೋದಾಗ ಏಳು ಕೊಪ್ಪರಿಗೆ ನೋಡಿದ. ಅದನ್ನು ತೆರೆದು ನೋಡಿದಾಗ ಎಲ್ಲಾ ಬಂಗಾರದ ನಾಣ್ಯಗಳು. ಆದರೆ ಕೊನೆಯ ಕೊಪ್ಪರಿಗೆಯಲ್ಲಿ ಮಾತ್ರ ಅರ್ಧ ಚಿನ್ನವಿತ್ತು. ಕ್ಷೌರಿಕನಿಗೆ ಆ ಕೊಪ್ಪರಿಗೆಯನ್ನು ತುಂಬಿಸುವ ಆಸೆ ಬಲವಾಯಿತು. ಏಳನೇ ಕೊಪ್ಪರಿಗೆಯನ್ನು ಬಂಗಾರದಿಂದ ತುಂಬದೆ ಇದ್ದರೆ ಅವನ ಆಸೆ ಪೂರ್ಣವಾಗುವುದಿಲ್ಲ. ಅವನು ತನ್ನಲ್ಲಿದ್ದ ಒಡವೆಗಳನ್ನೆಲ್ಲ ಚಿನ್ನದ ಹಣಕ್ಕೆ ಪರಿವರ್ತನೆ ಮಾಡಿ ಕೊಪ್ಪರಿಗೆಗೆ ಹಾಕಿದ. ಆದರೆ ಆ ಮಾಯಾ ಕೊಪ್ಪರಿಗೆ ಹಿಂದೆ ಹೇಗಿತ್ತೋ ಹಾಗೆಯೇಇತ್ತು. ಇದರಿಂದ ಅವನು ವ್ಯಥೆಪಟ್ಟ. ತನ್ನ ಮತ್ತು ಹೆಂಡತಿ ಮಕ್ಕಳ ಹಣವನ್ನೆಲ್ಲ ಚಿನ್ನಕ್ಕೆ ಬದಲಾಯಿಸಿ ಕೊಪ್ಪರಿಗೆಗೆ ಹಾಕಿದ. ಆದರೆ ಕೊಪ್ಪರಿಗೆ ಮಾತ್ರ ಹಿಂದೆ ಇದ್ದಂತೆಯೇ ಇದ್ದಿತು. ಒಂದುದಿನ ರಾಜನಿಗೆ “ನಮ್ಮ ಸಂಸಾರಕ್ಕೆ ನೀವು ಕೊಡುವ ಹಣ ಸಾಲದು, ಅದನ್ನು ಜಾಸ್ತಿ ಮಾಡಬೇಕು” ಎಂದು ಕೇಳಿಕೊಂಡ. ಕ್ಷೌರಿಕನ ಮೇಲೆ ರಾಜನಿಗೆ ತುಂಬಾ ವಿಶ್ವಾಸವಿತ್ತು. ಕ್ಷೌರಿಕನ ಸಂಬಳವನ್ನು ಎರಡರಷ್ಟು ಹೆಚ್ಚಿಸಿದ. ಈ ಹಣವನ್ನೆಲ್ಲ ಉಳಿತಾಯ ಮಾಡಿ ಕೊಪ್ಪರಿಗೆ ಹಾಕಿದ. ಆದರೆ ಅತಿಯಾಸೆಯ ಕೊಪ್ಪರಿಗೆ ಭರ್ತಿಯಾಗುವ ಚಿಹ್ನೆಯೇ ಇಲ್ಲ. ಅವನು ಅನಂತರ ಭಿಕ್ಷೆ ಮಾಡಿ ಜೀವಿಸುತ್ತ ತನಗೆ ಯಾವ ಸಂಬಳ ಬರುತ್ತಿತ್ತೋ ಅದನ್ನೂ ಆ ಮಾಯಾ ಕೊಪ್ಪರಿಗೆಗೆ ಹಾಕಿದ. ಹಲವು ತಿಂಗಳು ಹೀಗೇ ಸಾಗಿದವು. ಆದರೆ ಆ ಜಿಪುಣ ಕ್ಷೌರಿಕನ ಸಂಕಟ ಜಾಸ್ತಿಯಾಗುತ್ತ ಬಂತು. ಅವನ ದುಃಸ್ಥಿತಿಯನ್ನು ನೋಡಿ ರಾಜ, “ಏನಯ್ಯ, ನಿನಗೆ ಸಂಬಳ ಅರ್ಧದಷ್ಟು ಇದ್ದಾಗ ಸಂತೋಷದಿಂದ ತೃಪ್ತ ನಾಗಿದ್ದೆ. ನಿನ್ನ ಸಂಬಳ ಎರಡರಷ್ಟು ಮಾಡಿದ ಮೇಲೆ ನೀನು ತುಂಬಾ ದುಃಖಿಯೂ ದೀನನೂ ಆಗುತ್ತಿದ್ದೀಯಲ್ಲ, ನಿನಗೇನಾಗಿದೆ? ನಿನಗೇನಾದರೂ ಏಳು ಕೊಪ್ಪರಿಗೆ ಹಣ ಸಿಕ್ಕಿದೆಯೆ?” ಎಂದು ಕೇಳಿದ. ಕ್ಷೌರಿಕನಿಗೆ ಇದನ್ನು ಕೇಳಿ ಆಶ್ಚರ್ಯವಾಯಿತು. “ರಾಜರೆ, ಯಾರು ನಿಮಗೆ ಇದನ್ನು ಹೇಳಿದರು?” ಎಂದು ಕೇಳಿದ. ಆಗ ರಾಜ ಹೇಳಿದ: “ಯಕ್ಷ ಯಾರಿಗೆ ಏಳು ಕೊಪ್ಪರಿಗೆ ಹಣವನ್ನು ಕೊಡುತ್ತಾನೋ ಅವನ ಲಕ್ಷಣ ಇದು. ನೀನು ತೆಗೆದುಕೊಳ್ಳುತ್ತೀಯೇನು ಎಂದು ಅವನು ನನ್ನನ್ನೂ ಕೇಳಿದ. ಆಗ ನಾನು ‘ಈ ಹಣವನ್ನು ನಾನು ಖರ್ಚುಮಾಡಲೋ ಅಥವಾ ಸುಮ್ಮನೆ ಇಟ್ಟಿರಬೇಕೋ?’ ಎಂದು ಕೇಳಿದೆ. ಇದನ್ನು ಕೇಳಿದ ತಕ್ಷಣ ಯಕ್ಷ ಯಾವ ಮಾತನ್ನೂ ಆಡದೆ ಓಡಿ ಹೋದ. ಯಾರೂ ಆ ಹಣವನ್ನು ಖರ್ಚು ಮಾಡುವುದಕ್ಕೆ ಆಗುವುದಿಲ್ಲ ಎಂಬುದನ್ನು ನೀನು ಅರ್ಥ ಮಾಡಿಕೊಂಡಿಲ್ಲ. ಅದನ್ನು ತೆಗೆದುಕೊಂಡರೆ ಕೂಡಿಹಾಕಬೇಕೆಂಬ ಆಸೆ ಬಲವಾಗುವುದು. ತಕ್ಷಣವೇ ಹೋಗಿ ಯಕ್ಷನಿಗೆ ಆ ಹಣವನ್ನು ಕೊಟ್ಟು ಬಿಡು” ಎಂದ. ಕ್ಷೌರಿಕನಿಗೆ ಇದರಿಂದ ಬುದ್ಧಿ ಬಂತು. ಅವನು ಯಕ್ಷನಿರುವ ಮರದ ಬಳಿಗೆ ಬಂದು “ಯಕ್ಷನೇ, ನೀನು ಯಾವ ಹಣವನ್ನು ಕೊಟ್ಟಿದ್ದೆಯೋ ಅದನ್ನು ತೆಗೆದುಕೊಂಡು ಬಿಡು” ಎಂದ. ಯಕ್ಷ “ಆಗಲಿ, ನಾನು ತೆಗೆದುಕೊಂಡು ಬಿಡುತ್ತೇನೆ” ಎಂದ. ಕ್ಷೌರಿಕ ಮನೆಗೆ ಬಂದು ನೋಡಿದ, ಏಳು ಕೊಪ್ಪರಿಗೆ ಹಣವೆಲ್ಲ ಮಾಯವಾಗಿತ್ತು. ಜೊತೆಗೆ ಕೊಪ್ಪರಿಗೆಗೆ ಇವನು ಏನನ್ನು ಸೇರಿಸಿದ್ದನೊ ಅದೂ ಹೋಯಿತು. ನಿಜವಾದ ಖರ್ಚು ಯಾವುದು, ನಿಜವಾದ ಜಮಾ ಯಾವುದು ಎಂಬುದನ್ನು ಅರಿಯದವನು ಇದ್ದದ್ದನ್ನೆಲ್ಲ ಕಳೆದುಕೊಳ್ಳುವನು.


\section{\num{೨೦.} ಯೋಗಿ ಏತಕ್ಕೆ ಭ್ರಷ್ಟನಾಗುತ್ತಾನೆ}

ಕಾಮಾರಪುಕುರದಲ್ಲಿ ಗೋಡೆಯ ಮೇಲಿನ ಒಂದು ರಂಧ್ರದಲ್ಲಿ ಒಂದು ಮುಂಗುಸಿ ಇತ್ತು. ಅದು ಅಲ್ಲಿ ಬೆಚ್ಚಗೆ ಹಾಯಾಗಿತ್ತು. ಕೆಲವು ವೇಳೆ ಮುಂಗುಸಿಯ ಬಾಲಕ್ಕೆ ತೂಕವನ್ನು ಕಟ್ಟುತ್ತಾರೆ. ಆಗ ಅದರ ಭಾರದಿಂದ ನೆಲದ ಮೇಲೆ ಬರಬೇಕಾ ಗುವುದು. ಪ್ರತಿಯೊಂದು ಸಲ ತನ್ನ ಬಿಲದಲ್ಲಿ ಚೆನ್ನಾಗಿರೋಣ ಎಂದರೆ ಭಾರದ ದೆಸೆಯಿಂದ ಹೊರಗೆ ಬರಬೇಕಾಗುವುದು.

ಪ್ರಾಪಂಚಿಕ ವಸ್ತುಗಳನ್ನು ಕುರಿತು ಚಿಂತಿಸು ವುದರ ಫಲವಾಗಿ ಯೋಗಿ ತನ್ನ ಗುರಿಯಿಂದ ಭ್ರಷ್ಟ ನಾಗುತ್ತಾನೆ.


\section{\num{೨೧.} ಕೆಲಸಕ್ಕೆ ಬಾರದ ವಸ್ತುಗಳು}

ದೇಹ ಮತ್ತು ಐಶ್ವರ್ಯ ನಶ್ವರ. ಅವುಗಳನ್ನು ಪಡೆಯಲು ಏತಕ್ಕೆ ಇಷ್ಟು ಕಷ್ಟಪಡುತ್ತೀರಿ? ಹಠ ಯೋಗಿಗಳ ಪಾಡನ್ನು ನೋಡಿ! ಅವರ ದೃಷ್ಟಿ ಒಂದರ ಮೇಲೆಯೇ ಇದೆ. ಅದೇ ಅವರ ದೇಹ. ಭಗವಂತನ ಸಾಕ್ಷಾತ್ಕಾರಕ್ಕಾಗಿ ಅವರು ಪ್ರಯತ್ನ ಪಡುವು ದಿಲ್ಲ. ಅವರು ತಮ್ಮ ಕರುಳನ್ನು ಶುದ್ಧಮಾಡುವುದು, ಒಂದು ಕೊಳವೆಯ ಮೂಲಕ ಹಾಲನ್ನು ತೆಗೆದುಕೊಳ್ಳುವುದು–ಇವನ್ನು ದೀರ್ಘಕಾಲ ಬಾಳ ಬೇಕೆಂದು ಮಾಡುತ್ತಾರೆ.

ಒಬ್ಬ ಅಕ್ಕಸಾಲಿಗ ಇದ್ದ. ಒಮ್ಮೆ ಅವನ ನಾಲಿಗೆ ಬಾಯಿಯೊಳಗೆ ಸಿಕ್ಕಿಕೊಂಡಿತು. ಅವನು ಸಮಾಧಿಯಲ್ಲಿ ಇದ್ದಂತೆ ಕಂಡಿತು. ಅವನು ಸಂಪೂರ್ಣ ಜಡವಾಗಿ ಆ ಅವಸ್ಥೆಯಲ್ಲಿ ಬಹಳ ಕಾಲವಿದ್ದ. ಜನ ಅವನನ್ನು ಪೂಜಿಸಲು ಶುರುಮಾಡಿದರು. ಕೆಲವು ವರುಷ ಗಳಾದ ಮೇಲೆ ಅವನ ನಾಲಿಗೆ ಸರಿಯಾಯಿತು. ಅವನಿಗೆ ಹಿಂದಿನಂತೆ ವಸ್ತುಗಳು ಗೋಚರಿಸಿದವು. ಅವನು ತನ್ನ ಹಿಂದಿನ ಕೆಲಸಕ್ಕೆ ಹೋದ. ಈ ಬಾಹ್ಯ ವಸ್ತುಗಳಿಗೂ ಭಗವಂತನ ಸಾಕ್ಷಾತ್ಕಾರಕ್ಕೂ ಏನೂ ಸಂಬಂಧವಿಲ್ಲ. ಒಬ್ಬನಿಗೆ ಎಪ್ಪತ್ತೆರಡು ಆಸನಗಳೂ ಗೊತ್ತಿದ್ದುವು. ಅವನು ಯೋಗ ಮತ್ತು ಸಮಾಧಿಯ ಮೇಲೆ ಬಹಳ ಮಾತನಾಡುತ್ತಿದ್ದ. ಆದರೆ ಅವನ ಮನಸ್ಸು ಕಾಮಿನಿಕಾಂಚನದಿಂದ ತುಂಬಿತ್ತು. ಒಂದು ಸಲ ಸಾವಿರಾರು ರೂಪಾಯಿ ನೋಟಿನ ಕಂತೆ ಸಿಕ್ಕಿತು. ಅವನು ಪ್ರಲೋಭನೆಯಿಂದ ಪಾರಾಗಲಾರದೆ ಆ ನೋಟುಗಳನ್ನು ನುಂಗಿಬಿಟ್ಟ. ಅನಂತರ ಅದನ್ನು ಯಾವುದಾದರೂ ಮಾರ್ಗದ ಮೂಲಕ ಪಡೆಯುತ್ತೇನೆ ಎಂದು ಭಾವಿಸಿದ. ನೋಟಿನ ಕಂತೆಯನ್ನೇನೊ ಹೊರಗೆ ತೆಗೆಯಲಾಯಿತು, ಆದರೆ ಅವನಿಗೆ ಮೂರು ವರುಷ ಸಜಾ ಆಯಿತು.

\chapter{ಕಾಮಕಾಂಚನ}

\section{\num{೨೨. } ನೀನು ಮದುವೆಯಾದರೆ ಗುಲಾಮನಾಗುತ್ತೀಯ}

ಕಾಮಿನಿ ಕಾಂಚನವೇ ಮನುಷ್ಯನನ್ನು ಬಂಧನಕ್ಕೆ ಈಡುಮಾಡು ವುದು. ಹೆಂಗಸಿನಿಂದಲೇ ಚಿನ್ನದ ಆವಶ್ಯಕತೆ, ಹೆಂಗಸಿಗಾಗಿ ಒಬ್ಬ ಮತ್ತೊಬ್ಬ ನಿಗೆ ಆಳಾಗುತ್ತಾನೆ, ತನ್ನ ಸ್ವಾತಂತ್ರ್ಯವನ್ನು ಕಳೆದುಕೊಳ್ಳುತ್ತಾನೆ.

ಜಯಪುರದ ಗೋವಿಂದಜಿ ದೇವಸ್ಥಾನದಲ್ಲಿ ಪೂಜಾರಿಗಳು ಬ್ರಹ್ಮಚಾರಿ ಗಳಾಗಿದ್ದರು. ಆಗ ಅವರು ಪ್ರಚಂಡರಾಗಿದ್ದರು. ಒಂದು ಸಲ ಜಯಪುರದ ಮಹಾರಾಜರು ಪೂಜಾರಿಗಳಿಗೆ ಹೇಳಿಕಳುಹಿಸಿದರು. ಆದರೆ ಅವರು ಬರಲಿಲ್ಲ. ಅವರು ದೂತನಿಗೆ “ರಾಜರೇ ಬೇಕಾದರೆ ಇಲ್ಲಿಗೆ ಬರಲಿ” ಎಂದರು. ರಾಜ, ಮಂತ್ರಿಯೊಡನೆ ಸಮಾಲೋಚನೆ ಮಾಡಿ ಪೂಜಾರಿಗಳಿಗೆ ಮದುವೆ ಮಾಡಿಸಿ ದನು. ಅಂದಿನಿಂದ ಪೂಜಾರಿಗಳಿಗೆ ರಾಜ ಹೇಳಿಕಳುಹಿಸಬೇಕಾಗಿರಲಿಲ್ಲ. ತಮಗೆ ತಾವೇ ಬರಲು ಪ್ರಾರಂಭಿಸಿದರು. “ಮಹಾರಾಜರೆ, ನಾವು ದೇವರ ಪ್ರಸಾದವನ್ನು ನಿಮಗೆ ಕೊಟ್ಟು ನಿಮ್ಮನ್ನು ಹರಸಲು ಬಂದಿರುವೆವು. ದಯವಿಟ್ಟು ಇದನ್ನು ಸ್ವೀಕರಿಸಿ,” ಎಂದು ಕೋರುತ್ತಿದ್ದರು. ಅವರು ಅರಮನೆಗೆ ಯಾವುದಾದರೂ ಕೆಲಸದ ನಿಮಿತ್ತ ಅಥವಾ ಒಂದು ಮನೆ ಕಟ್ಟಿಸುವುದಕ್ಕಾ ಗಿಯೋ, ಮಕ್ಕಳ ಅನ್ನಪ್ರಾಶನ ಅಥವಾ ವಿದ್ಯಾಭ್ಯಾಸ, ಉಪನಯನ ಮುಂತಾ ದವುಗಳಿಗಾಗಿಯೋ ಹಣ ಕೇಳಲು ಆಗಾಗ ಬರಲಾರಂಭಿಸಿದರು.


\section{\num{೨೩.} ಸಾವಿರದ ಇನ್ನೂರು ಜನರ ಪತನ}

ಸಾವಿರದ ಇನ್ನೂರು ನೇಡರು ಮತ್ತು ಸಾವಿರದ ಮುನ್ನೂರು ನೇಡಿಯರು\footnote{
\begin{verse}
೧. ‘ಬೋಳುತಲೆಯವರು’ ಎಂದು ಶಬ್ದಾರ್ಥ. ಸಂಪೂರ್ಣ ತ್ಯಾಗಿಗಳು ಎಂದರ್ಥ. 
\end{verse}} ಒಂದು ಕಥೆ ಇದೆ. ನಿತ್ಯಾನಂದ ಗೋಸ್ವಾಮಿಯ ಮಗ ವೀರಭದ್ರನಿಗೆ ಸಾವಿರದ ಮುನ್ನೂರು ಸಂನ್ಯಾಸಿ ಶಿಷ್ಯರು ಇದ್ದರು. ಅವರು ಅಧ್ಯಾತ್ಮದಲ್ಲಿ ಬಹಳ ಮೇಲಿನ ಮಟ್ಟದಲ್ಲಿದ್ದರು. ಗುರುವಿಗೆ ಇದರಿಂದ ಗಾಬರಿ ಆಯಿತು. “ನನ್ನ ಶಿಷ್ಯರು ಅದ್ಭುತವಾದ ಸಿದ್ಧಿಗಳನ್ನು ಪಡೆದಿರುವರು” ಎಂದು ವೀರಭದ್ರ ಆಲೋಚಿಸಿದ. “ಅವರು ಜನರಿಗೆ ಏನು ಹೇಳುತ್ತಾರೋ ಅದು ನಡೆಯುವುದು. ಹೋದಲ್ಲೆಲ್ಲಾ ಅವರು ಆತಂಕದ ಸ್ಥಿತಿಯನ್ನು ಉಂಟುಮಾಡುತ್ತಾರೆ, ಏಕೆಂದರೆ ಜನರು ತಿಳಿಯದೆ ಏನಾದರೂ ತಪ್ಪನ್ನು ಮಾಡಿ ಅದರಿಂದ ಸಂಕಟಕ್ಕೆ ತುತ್ತಾಗುವರು”, ಹೀಗೆ ಆಲೋಚನೆಮಾಡಿ ವೀರಭದ್ರ ಶಿಷ್ಯರನ್ನು ತನ್ನ ಬಳಿಗೆ ಕರೆದನು. “ನೀವು ಗಂಗೆಯಲ್ಲಿ ಸ್ನಾನ ಆಹ್ನಿಕಗಳನ್ನು ಮಾಡಿಯಾದ ಮೇಲೆ ನನ್ನನ್ನು ಬಂದು ಕಾಣಿ” ಎಂದು ಹೇಳಿದನು. ಶಿಷ್ಯರು ಆಧ್ಯಾತ್ಮಿಕ ಜೀವನದಲ್ಲಿ ಎಷ್ಟು ಉನ್ನತ ಮಟ್ಟದಲ್ಲಿದ್ದರೆಂದರೆ ಧ್ಯಾನ ಮಾಡುವಾಗ ಉಬ್ಬರದ ನೀರು ಅವರ ಮೇಲೆ ಹರಿಯುತ್ತಿದ್ದರೂ ಅವರಿಗೆ ಅರಿವೇ ಇರುತ್ತಿರಲಿಲ್ಲ. ನದಿ ನೀರು ಇಳಿದಾ ಗಲೂ ಅವರು ಧ್ಯಾನದಲ್ಲಿ ತಲ್ಲೀನರಾಗಿರುತ್ತಿದ್ದರು.

ಇವರಲ್ಲಿ ನೂರು ಜನ ಶಿಷ್ಯರು ಗುರುಗಳು ಏನನ್ನು ಹೇಳಬಹುದು ಎಂಬುದನ್ನು ಊಹಿಸಿದ್ದರು. ಗುರುವಿನ ಆಣತಿಯನ್ನು ಮೀರಿ ಹೋಗಬೇಕಾಗು ವುದು ಎಂದರಿತು, ಗುರುಗಳು ಅವರನ್ನು ಕರೆಯುವುದಕ್ಕೆ ಮುಂಚೆಯೇ ತಪ್ಪಿಸಿಕೊಂಡರು, ತಮ್ಮ ಗುರುಗಳಾದ ವೀರಭದ್ರನ ಬಳಿಗೆ ಹೋಗಲಿಲ್ಲ. ಉಳಿದ ಸಾವಿರದ ಇನ್ನೂರು ಜನ ಶಿಷ್ಯರು ಪ್ರಾತಃಕಾಲದ ಆಹ್ನಿಕಗಳನ್ನೆಲ್ಲ ಪೂರೈಸಿಕೊಂಡ ಮೇಲೆ ಗುರುಗಳ ಹತ್ತಿರ ಹಾಜರಾದರು. ವೀರಭದ್ರ “ಸಾವಿರದ ಮುನ್ನೂರು ಜನ ಸಂನ್ಯಾಸಿನಿಯರು ನಿಮ್ಮನ್ನು ಸೇವಿಸುವರು. ನೀವು ಅವರನ್ನು ಮದುವೆ ಮಾಡಿಕೊಳ್ಳಬೇಕು” ಎಂದನು. “ತಾವು ಹೇಳಿದ ಮೇಲೆ ನಿಮ್ಮ ಅಪ್ಪಣೆಯನ್ನು ಪರಿಪಾಲಿಸುವೆವು. ಆದರೆ ನಮ್ಮಲ್ಲಿ ನೂರು ಜನ ಗೈರು ಹಾಜರ್ ಆಗಿರುವರು.” ಉಳಿದ ಸಾವಿರದ ಇನ್ನೂರು ನೇಡರು ನೇಡಿಯರನ್ನು ಮದುವೆಯಾದರು. ಇದರ ಪರಿಣಾಮವಾಗಿ ಹಿಂದೆ ಇದ್ದ ತಪೋಶಕ್ತಿಯೆಲ್ಲ ನಾಶವಾಯಿತು. ಹೆಂಗಸಿನ ಸಹವಾಸದಿಂದ ಇವರೆಲ್ಲ ಪತಿತರಾದರು. ಏಕೆಂದರೆ ಅವರಿಗೆ ಹಿಂದೆ ಇದ್ದ ಸ್ವಾತಂತ್ರ್ಯ ಇರಲಿಲ್ಲ.


\section{\num{೨೪. } ಎಲ್ಲಕ್ಕೂ ಒಡೆಯರು, ಕಾಮಕ್ಕೆ ಅಧೀನರು}

ಒಬ್ಬ ಚಾಕರಿ ಹುಡುಕುವವನು ಚಾಕರಿಗಾಗಿ ಒಂದು ಆಫೀಸಿಗೆ ಅಲೆದು ಅಲೆದು ಬೇಸತ್ತಿದ್ದ. ಆಫೀಸ್ ಮ್ಯಾನೇಜರ್ ಯಾವಾಗಲೂ, “ಈಗ ಕೆಲಸ ವಿಲ್ಲ, ಮತ್ತೆ ಯಾವಾಗಲಾದರೂ ಬಂದು ನೋಡು,” ಎಂದೆನ್ನುತ್ತಿದ್ದನು. ಹಲವು ದಿನ ಹೀಗೆಯೇ ಕಳೆಯಿತು. ಒಂದು ದಿನ ಅವನು ತನ್ನ ಗೋಳನ್ನು ತನ್ನ ಗೆಳೆಯನಿಗೆ ಹೇಳಿಕೊಂಡ. ಸ್ನೇಹಿತನು ಅವನಿಗೆ “ಎಂತಹ ಮೂರ್ಖ ನೀನು, ಅವನ ಹತ್ತಿರ ಹೋಗಿ ಏತಕ್ಕೆ ಕಾಲನ್ನು ಸವೆಸುತ್ತಿರುವಿ. ನೀನು ಗುಲಾಬಳ ಹತ್ತಿರ ಹೋಗು. ನಿನ್ನ ಕೆಲಸ ಈಡೇರುವುದು. ನಾಳೆಯೇ ನಿನಗೆ ಕೆಲಸ ಸಿಕ್ಕುವುದು” ಎಂದನು. “ಹಾಗೇನು ಪರಿಸ್ಥಿತಿ! ಈಗಲೇ ನಾನು ಹೋಗುತ್ತೇನೆ.” ಗುಲಾಬಳು ಮ್ಯಾನೇ ಜರರ ಉಪಪತ್ನಿ. ಅವಳ ಬಳಿಗೆ ಹೋಗಿ “ತಾಯಿ, ನಾನು ತುಂಬ ಕಷ್ಟದಲ್ಲಿ ಸಿಕ್ಕಿ ಬಿದ್ದಿರುವೆನು. ನೀನು ನನ್ನನ್ನು ಈ ಪರಿಸ್ಥಿತಿಯಿಂದ ಪಾರುಮಾಡಬೇಕು. ನಾನು ಒಬ್ಬ ಬಡ ಬ್ರಾಹ್ಮಣನ ಮಗ. ನನಗೆ ಇನ್ನು ಯಾರು ಸಹಾಯ ಮಾಡುತ್ತಾರೆ, ತಾಯಿ, ಹಲವು ದಿನಗಳಿಂದ ನನಗೆ ಕೆಲಸ ಇಲ್ಲ. ನನ್ನ ಮಕ್ಕಳು ಉಪವಾಸದಿಂದ ಸಾಯುವ ಸ್ಥಿತಿಯಲ್ಲಿರುವರು. ನೀವು ಒಂದು ಮಾತನ್ನು ಹೇಳಿದರೆ ನನಗೆ ಕೆಲಸ ಸಿಕ್ಕುವುದು” ಎಂದನು. ಗುಲಾಬಳು, “ಅಯ್ಯ, ನಾನು ಯಾರಿಗೆ ಹೇಳಬೇಕು, ಹೇಳು” ಎಂದು ಕೇಳಿದಳು. ‘ಬಡ ಬ್ರಾಹ್ಮಣ, ತುಂಬಾ ಕಷ್ಟಪಟ್ಟಿದ್ದಾನೆ,’ ಎಂದು ಅವಳು ಮನಸ್ಸಿನಲ್ಲೇ ಅಂದುಕೊಂಡಳು. ಚಾಕರಿ ಗಾಗಿ ಬಂದವನು ಅವಳಿಗೆ, “ನೀವು ಯಜಮಾನರಿಗೆ ಒಂದು ಮಾತು ಹೇಳಿದರೆ ನನಗೆ ನಿಸ್ಸಂದೇಹವಾಗಿ ಕೆಲಸ ಸಿಕ್ಕುವುದು” ಎಂದನು. ಗುಲಾಬಳು “ಇವತ್ತೇ ನಾನು ಅವರಿಗೆ ಹೇಳಿ ನಿನಗೆ ಕೆಲಸ ಕೊಡಿಸುವೆನು” ಎಂದಳು. ಮಾರನೇ ದಿನ ಒಬ್ಬ ಬಂದು ಇವನಿಗೆ “ನೀನು ಮ್ಯಾನೇಜರ್ ಆಫೀಸಿನಲ್ಲಿ ಇವತ್ತಿನಿಂದಲೇ ಕೆಲಸಕ್ಕೆ ಹೋಗು” ಎಂದನು. ಮ್ಯಾನೇಜರ್ ತನ್ನ ಮೇಲಿನ ಇಂಗ್ಲೀಷ್ ಅಧಿಕಾರಿಗೆ “ಈ ಮನುಷ್ಯ ನುರಿತ ಕೆಲಸಗಾರ. ಇವನನ್ನು ನಾನು ಕೆಲಸಕ್ಕೆ ನೇಮಕ ಮಾಡಿಕೊಂಡಿದ್ದೇನೆ. ಇವನಿಂದ ನಮ್ಮ ಆಫೀಸಿಗೆ ಗೌರವ ಬರುತ್ತದೆ” ಎಂದ.


\section{\num{೨೫.} ಕಿವಿಯಲ್ಲಿ ಭಾಗವತ ಶ್ರವಣ, ಮನಸ್ಸಿನಲ್ಲಿ ವೇಶ್ಯೆ}

ಒಂದು ಸಲ ಇಬ್ಬರು ಸ್ನೇಹಿತರು ದಾರಿಯಲ್ಲಿ ಹೋಗುತ್ತಿದ್ದರು. ಅಲ್ಲಿ ಕೆಲವರು ಭಾಗವತ ಕೇಳುತ್ತಿದ್ದರು. ಒಬ್ಬ ಮತ್ತೊಬ್ಬನಿಗೆ, “ಗೆಳೆಯ, ಪವಿತ್ರ ಭಾಗವತವನ್ನು ಕೇಳೋಣ ಬಾ” ಎಂದು ಇಬ್ಬರೂ ಕೇಳಲು ಹೋದರು. ಎರಡನೆಯವನು ಸ್ವಲ್ಪ ಇಣುಕು ಹಾಕಿ ಅಲ್ಲಿಂದ ಹೊರ ಬಂದ. ಅವನು ಒಬ್ಬ ವೇಶ್ಯೆಯ ಮನೆಗೆ ಹೋದ. ಸ್ವಲ್ಪದರಲ್ಲೇ ಅವನಿಗೆ ಅಸಹ್ಯವಾಯಿತು. “ಎಂತಹ ತಿಳಿಗೇಡಿ ನಾನು. ನನ್ನ ಸ್ನೇಹಿತ ಪವಿತ್ರ ಹರಿನಾಮವನ್ನು ಕೇಳುತ್ತಿರುವನು. ಆದರೆ ನಾನು ಎಂತಹ ಅನಿಷ್ಟ ಸ್ಥಳದಲ್ಲಿರುವೆ” ಎಂದು ವ್ಯಥೆಪಟ್ಟ. ಆದರೆ ಭಾಗವತವನ್ನು ಕೇಳು ತ್ತಿದ್ದ ಸ್ನೇಹಿತನಿಗೂ ಬೇಜಾರಾಯಿತು. “ಇವನು ಏನೇನೂ ಕೆಲಸಕ್ಕೆ ಬಾರದುದನ್ನು ಹೇಳುತ್ತಿರುವನು. ಆದರೆ ನನ್ನ ಸ್ನೇಹಿತ ಮಜ ಮಾಡುತ್ತಿರುವನು” ಎನ್ನಿಸಿತು. ಕಾಲ ಕ್ರಮೇಣ ಇಬ್ಬರೂ ಸತ್ತರು. ಯಮ ದೂತರು ಭಾಗವತವನ್ನು ಕೇಳುತ್ತಿದ್ದವನನ್ನು ಯಮಲೋಕಕ್ಕೆ ಕರೆದುಕೊಂಡು ಹೋದರು. ವೇಶ್ಯೆಯ ಮನೆಗೆ ಹೋದವನನ್ನು ಸ್ವರ್ಗಕ್ಕೆ ಕರೆದುಕೊಂಡು ಹೋದರು.

ದೇವರು ನಿಜವಾಗಿ ಹೃದಯಲ್ಲಿರುವುದನ್ನು ನೋಡುತ್ತಾನೆ. ವ್ಯಕ್ತಿಯು ಏನು ಮಾಡುತ್ತಾನೊ, ಎಲ್ಲಿರುವನೋ ಅದನ್ನು ಗಮನಿಸುವುದಿಲ್ಲ.


\section{\num{೨೬. } ಗುರುವಿಗಿಂತ ಅಧಿಕ}

ಒಬ್ಬ ಬಡ ಬ್ರಾಹ್ಮಣನಿಗೆ ಶ್ರೀಮಂತ ಬಟ್ಟೆ ವ್ಯಾಪಾರಿಯೊಬ್ಬ ಶಿಷ್ಯನಾಗಿ ಸಿಕ್ಕಿದ. ಸ್ವಭಾವತಃ ವರ್ತಕ ಜಿಪುಣ. ಒಮ್ಮೆ ಬ್ರಾಹ್ಮಣನಿಗೆ ಪಾರಾಯಣ ಗ್ರಂಥಗಳನ್ನು ಕಟ್ಟುವುದಕ್ಕೆ ಒಂದು ಸಣ್ಣ ವಸ್ತ್ರ ಬೇಕಾಗಿತ್ತು. ಶಿಷ್ಯನ ಮನೆಗೆ ಹೋಗಿ ಒಂದು ತುಂಡು ಬಟ್ಟೆಯನ್ನು ಕೇಳಿದ. ಆದರೆ ವರ್ತಕ ಹೇಳಿದ “ಛೇ, ಎಂಥ ಅನ್ಯಾಯ! ನೀವು ಏನಾದರೂ ಸ್ವಲ್ಪ ಹೊತ್ತಿನ ಹಿಂದೆ ಬಂದು ಕೇಳಿದ್ದರೆ ನಾನು ಕೊಡುತ್ತಿದ್ದೆ. ದುರದೃಷ್ಟವಶಾತ್ ನೀವು ಕೇಳುವುದು ಈಗ ನನ್ನಲ್ಲಿ ಇಲ್ಲ. ನಿಮಗೆ ಬೇಕಾದುದನ್ನು ಜ್ಞಾಪಕದಲ್ಲಿಟ್ಟಿರುವೆನು. ಮಧ್ಯೆ ಮಧ್ಯೆ ಬಂದು ಜ್ಞಾಪಿಸಿ” ಎಂದು. ನಿರಾಶೆಯಿಂದ ಬ್ರಾಹ್ಮಣ ಹಿಂತಿರುಗಿದ. ತೆರೆಯ ಹಿಂದೆ ಇದ್ದ ವರ್ತಕನ ಹೆಂಡತಿ, ಗುರುಗಳಿಗೂ ವರ್ತಕನಿಗೂ ಆದ ಸಂಭಾಷಣೆಯನ್ನು ಕೇಳುತ್ತಿದ್ದಳು. ತಕ್ಷಣವೇ ಒಬ್ಬನ ಮೂಲಕ ಬ್ರಾಹ್ಮಣನನ್ನು ಕರೆದುಕೊಂಡು ಬರುವಂತೆ ಹೇಳಿದಳು. “ಪೂಜ್ಯ ಗುರುಗಳೆ, ನೀವು ಯಜಮಾನರಿಂದ ಏನನ್ನು ಕೇಳುವುದಕ್ಕೆ ಹೋಗುತ್ತಿದ್ದಿರಿ” ಎಂದು ವಿಚಾರಿಸಿದಳು. ಬ್ರಾಹ್ಮಣ ತನಗೆ ಆದ ಅನುಭವ ವನ್ನು ಹೇಳಿದ. ವರ್ತಕನ ಹೆಂಡತಿ ಗುರುಗಳಿಗೆ, “ದಯವಿಟ್ಟು ನೀವು ಮನೆಗೆ ಹೋಗಿ. ನಿಮಗೆ ನಾಳೆ ಬೆಳಿಗ್ಗೆ ಬಟ್ಟೆ ಸಿಕ್ಕುವುದು” ಎಂದಳು. ವರ್ತಕ ರಾತ್ರಿ ಹಿಂತಿರುಗಿ ಬಂದಾಗ ಹೆಂಡತಿ, “ಅಂಗಡಿಯನ್ನು ಮುಚ್ಚಿಬಿಟ್ಟಿರಾ?” ಎಂದು ಕೇಳಿದಳು. “ಮುಚ್ಚಿ ಆಗಿದೆ. ಏನು ಸಮಾಚಾರ” ಎಂದು ಕೇಳಿದ. “ತಕ್ಷಣವೆ ಅಂಗಡಿಗೆ ಹೋಗಿ ಅತ್ಯಂತ ಬೆಲೆಬಾಳುವ ಎರಡು ವಸ್ತ್ರಗಳನ್ನು ತೆಗೆದುಕೊಂಡು ಬನ್ನಿ” ಎಂದಳು. ವರ್ತಕ “ಇಷ್ಟೊಂದು ಆತುರ ಏತಕ್ಕೆ? ನಾಳೆಯೇ ನಿನಗೆ ಶ್ರೇಷ್ಠವಾದ ಬಟ್ಟೆಯನ್ನು ಕೊಡುವೆ” ಎಂದ. ಆಗ ಹೆಂಡತಿ “ನನಗೆ ಈ ಕ್ಷಣ ಅದು ಬೇಕು, ಕೊಡದೆ ಇದ್ದರೆ ನನಗೆ ಬೇಡವೇ ಬೇಡ” ಎಂದು ಹಟ ಹಿಡಿದಳು. ಪಾಪ, ವರ್ತಕ ಏನು ಮಾಡಲು ಸಾಧ್ಯ? ಈಗ ಅವನು ಸಮಾಧಾನ ಮಾಡ ಬೇಕಾಗಿರುವುದು ತನ್ನ ಗುರುವನ್ನು ಅಲ್ಲ. ಗುರುಗಳಿಗೆ ಏನೇನೋ ನೆಪ ಹೇಳಿ ಕಳುಹಿಸಿಕೊಟ್ಟ. ತೆರೆಯ ಹಿಂದೆ ಇರುವ ಗುರುವಿನ ಆಜ್ಞೆಯನ್ನು ತಕ್ಷಣವೇ ಪಾಲಿಸಬೇಕಾಗಿದೆ. ಇಲ್ಲದಿದ್ದರೆ ಮನೆಯಲ್ಲಿ ಶಾಂತಿ ಇರುವುದಿಲ್ಲ. ಕೊನೆಗೆ ವರ್ತಕನು ತಾನೇ ಅಂಗಡಿಯನ್ನು ತೆರೆದು ಬಟ್ಟೆಯನ್ನು ತಂದು ಕೊಟ್ಟ. ಮಾರನೇ ದಿನ ಬೆಳಿಗ್ಗೆ ಆಳು ಹೊತ್ತಿಗೆ ಮುಂಚೆಯೆ ಎದ್ದು, ಬ್ರಾಹ್ಮಣನ ಮನೆಗೆ ಹೋಗಿ ಬಟ್ಟೆಯನ್ನಿತ್ತು, “ನಿಮಗೆ ಮುಂದೆ ಏನಾದರೂ ಬೇಕಾದರೆ ಅಮ್ಮನವರಿಗೆ ಒಂದು ಮಾತನ್ನು ಹೇಳಿದರೆ ಸಾಕು” ಎಂದನು.


\section{\num{೨೭. } ಆಧುನಿಕ ಜನಕರು}

ಆಧುನಿಕ ವಿದ್ಯೆಯನ್ನು ಪಡೆದ ಒಬ್ಬ ದೊಡ್ಡ ಮನುಷ್ಯ, ಪ್ರಾಪಂಚಿಕತೆಯ ಸೋಂಕಿಲ್ಲದ ಗೃಹಸ್ಥರ ವಿಷಯವಾಗಿ ಶ್ರೀರಾಮಕೃಷ್ಣರೊಡನೆ ಚರ್ಚಿಸುತ್ತಿ ದ್ದನು. ಶ್ರೀರಾಮಕೃಷ್ಣರು ಅವನಿಗೆ ಹೇಳಿದರು: “ಪ್ರಾಪಂಚಿಕತೆಯ ಸೋಂಕಿ ಲ್ಲದ ವ್ಯಕ್ತಿ ಎಂಥವನು ಎಂಬುದು ನನಗೆ ಗೊತ್ತು! ಅವನು ಸಂಸಾರದಲ್ಲಿ ಅನಾಸಕ್ತನಾದ್ದರಿಂದ ಹಣದ ವಿಚಾರದಲ್ಲಿ ಅವನು ತಲೆಹಾಕುವುದಿಲ್ಲ. ಅದನ್ನೆಲ್ಲ ನೋಡಿಕೊಳ್ಳುವವಳು ಅವನ ಹೆಂಡತಿ ತಾನೆ! ಯಾರಾದರೂ ಬಡ ಬ್ರಾಹ್ಮಣ ಭಿಕ್ಷೆಯನ್ನು ಬೇಡಲು ಬಂದರೆ ಯಜಮಾನ ಬ್ರಾಹ್ಮಣನಿಗೆ “ನಾನು ಹಣದ ತಂಟೆಗೆ ಬರುವುದಿಲ್ಲ. ನನ್ನನ್ನು ಬೇಡಿ ಏನು ಪ್ರಯೋಜನ?” ಎಂದು ಹೇಳುತ್ತಾನೆ. ಆದರೆ ಬ್ರಾಹ್ಮಣ ಭಂಡ. ಇವನ ಕಾಟವನ್ನು ತಾಳಲಾರದೆ ಅನಾಸಕ್ತನಾದ ಗೃಹಸ್ಥ ಅವನಿಗೆ



\section{\num{೨೮. } ಭ್ರಷ್ಟನಾದ ಸಂನ್ಯಾಸಿಯ ಬಾಳು}

ಸಂನ್ಯಾಸಿ ಹಣವನ್ನು ಸ್ವೀಕರಿಸಿದರೆ, ಪ್ರಲೋಭನೆಗೆ ಬಿದ್ದರೆ ಅವನು ಹೇಗಿರುತ್ತಾನೆ ಗೊತ್ತೆ? ಬ್ರಾಹ್ಮಣ ವಿಧವೆ ಬಹಳ ಕಾಲ ಪತಿವ್ರತಳಾಗಿದ್ದು, ಹವಿಷ್ಯಾನ್ನ, ಹಾಲು–ಇವನ್ನು ತಿಂದುಕೊಂಡಿದ್ದು ಕೊನೆಗೆ ಒಬ್ಬ ಉಪಪತಿ ಯನ್ನು ಸೇರಿಕೊಂಡಂತೆ.

ನಮ್ಮ ಊರಿನಲ್ಲಿ ಭಾಗಿ ತೇಲಿ ಎಂಬ ಹರಿಜನ ಮಹಿಳೆ ಇದ್ದಳು. ಅವಳಿಗೆ ಅನೇಕ ಜನ ಶಿಷ್ಯರಿದ್ದರು. ಶೂದ್ರಳೊಬ್ಬಳನ್ನು ಅನೇಕ ಜನ ಗೌರವಿಸು ತ್ತಿದ್ದುದನ್ನು ನೋಡಿ ಜಮೀನ್​ದಾರನಿಗೆ ತುಂಬಾ ಅಸೂಯೆ ಬಂತು. ಒಬ್ಬ ದುಷ್ಟನನ್ನು ಅವಳನ್ನು ಕೆಣಕಲು ಬಿಟ್ಟ. ಅವಳನ್ನು ಕೆಡಿಸುವುದರಲ್ಲಿ ಅವನು ಜಯಶೀಲನಾದ. ಅಂದಿನಿಂದ ಅವಳ ತಪಸ್ಸೆಲ್ಲ ವ್ಯರ್ಥವಾಯಿತು. ಭ್ರಷ್ಟನಾದ ಸಂನ್ಯಾಸಿಯೂ ಹೀಗೆಯೆ.


\section{\num{೨೯. } ನೀನು ಕಾಮವನ್ನು ಗೆಲ್ಲಬೇಕಾದರೆ ಹೆಂಗಸನ್ನು ತಾಯಿಯಂತೆ ನೋಡು}

ಯಾರೋ ಶ್ರೀರಾಮಕೃಷ್ಣರನ್ನು ನೀವು ಏತಕ್ಕೆ ಸಾಧಾರಣ ಗೃಹಸ್ಥರಂತೆ ನಿಮ್ಮ ಪತ್ನಿಯೊಂದಿಗಿರಬಾರದು ಎಂದು ಕೇಳಿದರು. ಅದಕ್ಕೆ ಶ್ರೀರಾಮಕೃಷ್ಣರು ಒಂದು ಕಥೆಯನ್ನು ಹೇಳಿದರು: ಶಿವನ ಮಗನಾದ ಕಾರ್ತಿಕೇಯ ಒಂದು ಸಲ ಬೆಕ್ಕನ್ನು ತನ್ನ ಉಗುರಿನಿಂದ ಪರಚಿದ. ಮನೆಗೆ ಬಂದಾಗ ತನ್ನ ತಾಯಿ ಪಾರ್ವತಿಯ ಗಲ್ಲದ ಮೇಲೆ ಆ ಬರೆಯನ್ನು ಕಂಡ. ಕಾರ್ತಿಕೇಯ, “ಅಮ್ಮ, ನಿನ್ನ ಕಪೋಲದ ಮೇಲೆ ಈ ಗಾಯ ಹೇಗಾಯಿತು?” ಎಂದು ಕೇಳಿದ. ಪಾರ್ವತಿ, “ಇದನ್ನು ನೀನೇ ಮಾಡಿದ್ದು. ನಿನ್ನ ಉಗುರಿನ ಗೆರೆಯೆ ಅದು” ಎಂದಳು. ಕಾರ್ತಿಕೇಯ ಆಶ್ಚರ್ಯಭರಿತ ನಾಗಿ, “ಅದ್ಹೇಗಮ್ಮ? ನಾನು ನಿನ್ನನ್ನು ಎಂದಿಗೂ ಪರಚಿದ ಜ್ಞಾಪಕವೇ ಇಲ್ಲ” ಎಂದ. ಅದಕ್ಕೆ ಪಾರ್ವತಿ: “ಮಗು, ಇವತ್ತು ಬೆಳಗ್ಗೆ ನೀನು ಬೆಕ್ಕನ್ನು ಪರಚಿದ್ದು ಮರೆತುಬಿಟ್ಟಿರುವೆ.” ಕಾರ್ತಿಕೇಯ: “ನಾನು ಬೆಕ್ಕನ್ನು ಪರಚಿದೆ. ಆದರೆ ನಿನ್ನ ಕೆನ್ನೆಯ ಮೇಲೆ ಗಾಯವುಂಟಾದುದು ಹೇಗೆ?” ಆಗ ತಾಯಿ, “ಮಗು ಈ ಪ್ರಪಂಚದಲ್ಲಿ ನನ್ನ ವಿನಾ ಬೇರೆ ಯಾವುದೂ ಇಲ್ಲ. ಸೃಷ್ಟಿಯೆಲ್ಲ ನಾನೆ. ನೀನು ಯಾರಿಗೆ ಹಿಂಸೆ ಮಾಡಿದರೂ ನನಗೇ ಹಿಂಸೆ ಮಾಡುವೆ” ಎಂದಳು. ಕಾರ್ತಿ ಕೇಯ ಇದನ್ನು ಕೇಳಿದಾಗ ಅವನಿಗೆ ಆಶ್ಚರ್ಯವಾಯಿತು. ಅಂದಿನಿಂದ ತಾನು ಮದುವೆಯಾಗುವುದಿಲ್ಲ ಎಂದು ನಿಶ್ಚಯಿಸಿದ. ಅವನು ಯಾರನ್ನು ಮದುವೆ ಯಾದಾನು? ಪ್ರತಿಯೊಬ್ಬ ಸ್ತ್ರೀಯೂ ಅವನಿಗೆ ತಾಯಿಯಾದಳು. ಎಲ್ಲ ಸ್ತ್ರೀಯರೂ ಮಾತೃಸ್ವರೂಪರು ಎಂದು ತಿಳಿದು ಅವನು ಮದುವೆ ಮಾಡಿಕೊಳ್ಳ ಲಿಲ್ಲ. ನಾನು ಕಾರ್ತಿಕೇಯನಂತೆ ಪ್ರತಿಯೊಬ್ಬ ಹೆಂಗಸರಲ್ಲೂ ಜಗನ್ಮಾತೆ ಯನ್ನೇ ನೋಡುವೆ.


\section{\num{೩೦. } ಹಣವೂ ಒಂದು ದೊಡ್ಡ ಉಪಾಧಿ}

ಹಣವೂ ಒಂದು ಉಪಾಧಿ. ಉಪಾಧಿಗಳಲ್ಲಿ ಅತಿ ಪ್ರಬಲವಾದ್ದು. ಮನುಷ್ಯ ಶ್ರೀಮಂತನಾದ ಮೇಲೆ ಅವನು ಸಂಪೂರ್ಣ ಬೇರೆ ವ್ಯಕ್ತಿಯಾಗುತ್ತಾನೆ.

ದಕ್ಷಿಣೇಶ್ವರಕ್ಕೆ ಒಬ್ಬ ದೀನನಾದ ಬಡಬ್ರಾಹ್ಮಣ ಅನೇಕ ವೇಳೆ ಬರುತ್ತಿದ್ದ. ಸ್ವಲ್ಪ ಕಾಲವಾದ ಮೇಲೆ ಅವನು ಬರುವುದನ್ನು ನಿಲ್ಲಿಸಿದ. ಅವನು ಏನಾದನೋ ನಮಗೆ ಗೊತ್ತಾಗಲಿಲ್ಲ. ನಾವು ಒಂದು ದಿನ ದೋಣಿಯಲ್ಲಿ ಹತ್ತಿರವಿದ್ದ ಕೊನ್ನಗರಕ್ಕೆ ಹೋದೆವು. ನಾವು ದೋಣಿಯಿಂದ ಇಳಿಯುತ್ತಿದ್ದಂತೆ, ಗಂಗಾ ತೀರದಲ್ಲಿ ಆ ಬ್ರಾಹ್ಮಣ, ದೊಡ್ಡ ಮನುಷ್ಯರಂತೆ ಗಾಳಿಸೇವನೆ ಮಾಡುತ್ತ ಕುಳಿತಿದ್ದ. ನನ್ನನ್ನು ನೋಡಿ ಅವನು ಠೀವಿಯಿಂದ “ಏನೂ ಠಾಕೂರರೇ, ಈಗ ನೀವು ಹೇಗಿದ್ದೀರಿ?” ಎಂದು ಕೇಳಿದ. ನಾನು ಅವನನ್ನು ಕಂಡೊಡನೆಯೆ ಅವನಲ್ಲಿ ಆದ ಬದಲಾವಣೆ ನನ್ನ ಗಮನಕ್ಕೆ ಬಂತು. ನನ್ನ ಹತ್ತಿರ ಇದ್ದ ಹೃದಯನಿಗೆ “ಇವನಿಗೆ ಏನೊ ಆಸ್ತಿ ಸಿಕ್ಕಿರಬೇಕು. ಅವನಲ್ಲಿ ಆದ ಬದಲಾವಣೆಯನ್ನು ನೀನು ಗಮನಿಸು,” ಎಂದೆ. ಹೃದಯ ಗಟ್ಟಿಯಾಗಿ ನಕ್ಕ. ಹಣ ವ್ಯಕ್ತಿಯಲ್ಲಿ ಎಂಥ ಬದಲಾವಣೆಯನ್ನು ಉಂಟುಮಾಡುತ್ತದೆ!


\section{\num{೩೧. } ಹಣದಿಂದ ಬರುವ ಮದ}

ಒಂದು ಕಪ್ಪೆಯ ಹತ್ತಿರ ಒಂದು ರೂಪಾಯಿ ಇತ್ತು. ಅದನ್ನು ರಂಧ್ರದಲ್ಲಿ ಬಚ್ಚಿಟ್ಟಿತ್ತು. ಒಂದು ದಿನ ಆನೆಯೊಂದು ಕಪ್ಪೆಯ ಬಿಲದ ಮೇಲೆ ಹೋಗು ತ್ತಿತ್ತು. ಕಪ್ಪೆಗೆ ತುಂಬಾ ಕೋಪ ಬಂದು, ಕಾಲನ್ನೆತ್ತಿ ಆನೆಯನ್ನು ಒದೆಯು



\chapter{ಮಾಯೆ}

\section{\num{೩೨. } ಮಾಯೆಯಲ್ಲಿ ಸಿಕ್ಕಿ ಬಿದ್ದು ಬ್ರಹ್ಮನೂ ಅಳುವನು}

ವಿಷ್ಣು ಹಿರಣ್ಯಾಕ್ಷನನ್ನು ಕೊಲ್ಲುವುದಕ್ಕಾಗಿ ಒಂದು ಹಂದಿಯಾಗಿ ಅವತರಿಸಿದ. ರಾಕ್ಷಸನನ್ನು ಕೊಂದಾದ ಮೇಲೆ ಹಂದಿ ತನ್ನ ಮರಿಗಳೊಂದಿಗೆ ಸಂತೋಷದಿಂದ ಕಾಲ ಕಳೆಯುತ್ತಿತ್ತು. ಹಂದಿ ತನ್ನ ನಿಜಸ್ವರೂಪವನ್ನು

ಪ್ರತಿಯೊಬ್ಬರೂ ಮಾಯೆಯ ಅಧೀನದಲ್ಲಿರುವರು. ದೇವರ ಅವತಾರಗಳು ಕೂಡ ಮಾಯೆಯ ಆಶ್ರಯವನ್ನು ಪಡೆದು ತಮ್ಮ ಪಾಲಿನ ಕೆಲಸವನ್ನು ಮಾಡುತ್ತಾರೆ. ಅದಕ್ಕಾಗಿಯೇ ಮಹಾಮಾಯೆಯನ್ನು ಕೊಂಡಾಡುವುದು.


\section{\num{೩೩. } ಮಾಯೆ ಹೇಗಿದೆ?}

ಒಬ್ಬ ಸಾಧು ನಹಬತ್​ಖಾನೆಯ ಒಂದು ಮೇಲ್ಕೋಣೆಯಲ್ಲಿದ್ದ. ಅದು ದಕ್ಷಿಣೇಶ್ವರ ದೇವಸ್ಥಾನದಲ್ಲಿತ್ತು. ಅವನು ಯಾರೊಂದಿಗೂ ಮಾತನಾಡುತ್ತಿರ ಲಿಲ್ಲ. ಭಗವಂತನ ಧ್ಯಾನದಲ್ಲಿ ಕಾಲ ಕಳೆಯುತ್ತಿದ್ದ. ಒಂದು ದಿನ ಇದ್ದಕ್ಕಿ ದ್ದಂತೆ ಆಕಾಶದಲ್ಲಿ ಮೋಡ ಆವರಿಸಿತು. ಸ್ವಲ್ಪ ಕಾಲದ ಮೇಲೆ ದೊಡ್ಡ ಗಾಳಿ ಬೀಸಿ ಮೋಡವನ್ನು ಚದುರಿಸಿತು. ಸಾಧು ತನ್ನ ಕೋಣೆಯಿಂದ ಹೊರಗೆ ಬಂದು ವಾದ್ಯಶಾಲೆಯ ಎದುರಿಗೆ ಕುಣಿದು ಕುಪ್ಪಳಿಸುತ್ತಿದ್ದ. “ಯಾವಾಗಲೂ ಮೌನ ದಲ್ಲಿ ತಲ್ಲೀನನಾಗಿದ್ದವನು ಇವತ್ತು ಏಕೆ ಅಷ್ಟೊಂದು ಆನಂದದಿಂದ ಕುಣಿದು ಕುಪ್ಪಳಿಸುತ್ತಿರುವೆ?” ಎಂದು ಕೇಳಿದೆ. ಆಗ ಸಾಧು, “ಪ್ರಪಂಚವನ್ನು ಆವರಿಸಿ ರುವ ಮಾಯೆಯ ಸ್ವಭಾವ ಇದು. ಮೊದಲು ಆಕಾಶ ತಿಳಿಯಾಗಿತ್ತು. ಸ್ವಲ್ಪ ಹೊತ್ತಾದ ಮೇಲೆ ಮೋಡ ಆವರಿಸಿತು. ಈಗ ಅದೆಲ್ಲ ಹೋಗಿ ಆಕಾಶ ತಿಳಿ ಯಾಗಿದೆ” ಎಂದ. ಹೀಗೆಯೇ ಮನಸ್ಸನ್ನೂ ಅಜ್ಞಾನದ ಮೋಡ ಕವಿಯುತ್ತದೆ.


\section{\num{೩೪. } ನಿಜವಾಗಿ ಮಾಯೆಯ ಸ್ವಭಾವವೇ ಹೀಗೆ}

ಒಂದು ಸಲ ನಾರದರು ಶ್ರೀಕೃಷ್ಣನನ್ನು, “ಅಘಟಿತಘಟನಾ ಸಾಮರ್ಥ್ಯ ವಿರುವ ನಿನ್ನ ಮಾಯಾಶಕ್ತಿಯನ್ನು ತೋರು” ಎಂದು ಕೇಳಿಕೊಂಡರು. ಕೃಷ್ಣನು “ಆಗಲಿ” ಎಂದ. ಒಂದು ಸಲ ಕೃಷ್ಣನು ನಾರದರೊಂದಿಗೆ ತಿರುಗಾಡುತ್ತಿದ್ದನು. ಸ್ವಲ್ಪ ದೂರ ಹೋದ ಮೇಲೆ ಅವನಿಗೆ ಬಹಳ ಬಾಯಾರಿಕೆಯಾಗಿ ಅತ್ಯಂತ ಆಯಾಸದಿಂದ ಕುಳಿತುಕೊಂಡು ನಾರದರಿಗೆ, “ನನಗೆ ತುಂಬಾ ಬಾಯಾರಿಕೆ ಆಗಿದೆ. ಎಲ್ಲಿಂದಲಾದರೂ ಸ್ವಲ್ಪ ನೀರನ್ನು ತನ್ನಿ” ಎಂದನು.

ಹತ್ತಿರ ಎಲ್ಲೂ ನೀರು ಸಿಕ್ಕದೆ ಇದ್ದುದರಿಂದ ನಾರದರು ಸ್ವಲ್ಪ ದೂರ ಹೋದರು. ಅಲ್ಲಿ ಒಂದು ನದಿ ಇತ್ತು. ಅವರು ನದಿ ತೀರಕ್ಕೆ ಬಂದಾಗ ಒಬ್ಬ ಸುಂದರ ಯುವತಿ ಅಲ್ಲಿ ಕುಳಿತಿರುವುದನ್ನು ಕಂಡರು. ಅವಳ ಸೌಂದರ್ಯಕ್ಕೆ ಮನಸೋತರು. ನಾರದರು ಅವಳ ಬಳಿಗೆ ಹೋದಾಗ, ಅವರನ್ನು ಪ್ರೀತಿಯಿಂದ ಮಾತನಾಡಿಸಿದಳು. ಹೀಗೆ ಮಾತುಕತೆ ಪ್ರೇಮದಲ್ಲಿ ಪರ್ಯವಸಾನಗೊಂಡು ಇಬ್ಬರೂ ಮದುವೆಯಾಗಿ ಗೃಹಸ್ಥಾಶ್ರಮಿಗಳಾದರು. ಅವಳಿಂದ, ಕಾಲಾನಂತರ ಕೆಲವು ಮಕ್ಕಳಾದವು. ತನ್ನ ಹೆಂಡತಿ ಮಕ್ಕಳೊಂದಿಗೆ ಸಂತೋಷದಿಂದಿರುವಾಗ ಆ ದೇಶಕ್ಕೆ ಕ್ಷಾಮ ಬಂತು. ಅನೇಕ ಜನ ಸತ್ತರು. ಆಗ ನಾರದರು ಆ ಸ್ಥಳವನ್ನು ಬಿಟ್ಟು ಬೇರೆ ಕಡೆಗೆ ಹೋಗಲು ನಿರ್ಧರಿಸಿದರು. ಹೆಂಡತಿಯೂ ಅದಕ್ಕೆ ಸಮ್ಮತಿ ಕೊಟ್ಟಳು. ಇಬ್ಬರೂ ಮಕ್ಕಳೊಂದಿಗೆ ಇದ್ದ ಸಾಮಾನನ್ನು ಹೊತ್ತುಕೊಂಡು ಹೊರಟರು. ಆದರೆ ಹೋಗುವಾಗ ಒಂದು ಸೇತುವೆ ಬಂತು. ಅದನ್ನು ದಾಟ ಬೇಕಾಯಿತು. ಅದೇ ಸಮಯದಲ್ಲಿ ದೊಡ್ಡ ಪ್ರವಾಹವೊಂದು ಬಂತು. ಆ ಪ್ರವಾಹದ ರಭಸದಲ್ಲಿ ಒಬ್ಬರಾದ ಮೇಲೆ ಒಬ್ಬರು ಮಕ್ಕಳೆಲ್ಲ ಕೊಚ್ಚಿಕೊಂಡು ಹೋದರು. ಕೊನೆಗೆ ಹೆಂಡತಿಯೂ ಮುಳುಗಿ ಹೋದಳು. ಹೆಂಡತಿ ಮಕ್ಕಳ ಸಾವಿನ ದುಃಖದಿಂದ ನಾರದರು ಒಂದು ದಂಡೆಯ ಮೇಲೆ ಕುಳಿತು, ಸಾಕಾಗಿ ಗೊಳೋ ಎಂದು ಅಳುವುದಕ್ಕೆ ಶುರುಮಾಡಿದರು. ಆಗ ಕೃಷ್ಣನು ಬಂದು “ನಾರದರೇ ನೀರೆಲ್ಲಿ? ನೀವು ಏತಕ್ಕೆ ಅಳುತ್ತಿರುವಿರಿ?” ಎಂದು ಕೇಳಿದನು. ಭಗವಂತನ ದರ್ಶನದಿಂದ ನಾರದರು ಆಶ್ಚರ್ಯಚಕಿತರಾದರು. ಎಲ್ಲವೂ ಅರ್ಥವಾಯಿತವರಿಗೆ. ನಾರದರು, “ಹೇ ದೇವ! ನಿನಗೆ ನಮಸ್ಕಾರ. ನಿನ್ನ ಅದ್ಭುತವಾದ ಮಾಯಾಶಕ್ತಿಗೂ ನಮಸ್ಕಾರ” ಎಂದರು.


\section{\num{೩೫. } ಮಾಯೆಯನ್ನು ಅರಿತೊಡನೆಯೆ ಅದು ಕಂಬಿ ಕೀಳುವುದು}

ಒಬ್ಬ ಪುರೋಹಿತರು ಒಂದು ಹಳ್ಳಿಯಲ್ಲಿದ್ದ ಶಿಷ್ಯನ ಮನೆಗೆ ಹೊರಟರು. ಅವರೊಂದಿಗೆ ಸೇವಕರಾರೂ ಇರಲಿಲ್ಲ. ಒಬ್ಬ ಚಮ್ಮಾರನನ್ನು ಕಂಡ ಒಡನೆಯೆ, “ನೋಡಯ್ಯ, ನೀನು ಬಹಳ ಒಳ್ಳೆಯವನು. ನನ್ನೊಂದಿಗೆ ಆಳಿನಂತೆ ಬರುವೆಯಾ? ನೀನು ನನ್ನೊಂದಿಗೆ ಬಂದರೆ ನಿನ್ನನ್ನು ಚೆನ್ನಾಗಿ ನೋಡಿಕೊಳ್ಳು ವರು, ಊಟ ಉಪಚಾರವೆಲ್ಲ ಆಗುವುದು” ಎಂದು ಕೇಳಿದರು. ಅದಕ್ಕೆ ಮೋಚಿ ಹೇಳಿದ, “ಸ್ವಾಮಿ ನಾನು ತುಂಬಾ ಕೀಳು ಜಾತಿಗೆ ಸೇರಿದವನು. ನಾನು ಹೇಗೆ ನಿಮ್ಮ ಆಳಿನಂತೆ ಬರಲಿ?” ಎಂದು. ಪುರೋಹಿತರು “ಪರವಾಗಿಲ್ಲ. ನಿನ್ನ ಬಗ್ಗೆ ಯಾರಿಗೂ ಹೇಳಬೇಡ. ನೀನು ಯಾರೊಂದಿಗೂ ಮಾತನಾಡಬೇಡ. ಇನ್ನೊ ಬ್ಬರ ಪರಿಚಯವನ್ನೂ ಮಾಡಿಕೊಳ್ಳಬೇಡ” ಎಂದರು. ಮೋಚಿ ಇದಕ್ಕೆ ಒಪ್ಪಿ ಕೊಂಡ. ಸಂಜೆ ಪುರೋಹಿತರು ಶಿಷ್ಯನ ಮನೆಯಲ್ಲಿ ಪೂಜೆ ಮಾಡುತ್ತಿದ್ದರು. ಮತ್ತೊಬ್ಬ ಬ್ರಾಹ್ಮಣನು ಪುರೋಹಿತನ ಆಳಿಗೆ, “ಅಲ್ಲಿರುವ ನನ್ನ ಪಾದರಕ್ಷೆ ಗಳನ್ನು ತೆಗೆದುಕೊಂಡು ಬಾ” ಎಂದ. ತನ್ನ ಯಜಮಾನನ ಅಪ್ಪಣೆಯಂತೆ ಮೋಚಿ ಏನನ್ನೂ ಮಾಡಲಿಲ್ಲ. ಬ್ರಾಹ್ಮಣ ಮತ್ತೊಮ್ಮೆ ಆಳಿಗೆ ಹೇಳಿದ. ಆದರೂ ಆಳು ಸುಮ್ಮನೆ ಇದ್ದ. ಪುನಃ ಪುನಃ ಬ್ರಾಹ್ಮಣ ಪಾದರಕ್ಷೆಯನ್ನು ತರಲು ಹೇಳಿದ. ಆದರೆ ಮೋಚಿ ತನ್ನ ಸ್ಥಳವನ್ನು ಬಿಟ್ಟು ಸ್ವಲ್ಪವೂ ಕದಲಲಿಲ್ಲ. ಬ್ರಾಹ್ಮಣ ರೇಗಿ, “ಮೂಢ, ಬ್ರಾಹ್ಮಣನ ಆಣತಿಯನ್ನು ಪಾಲಿಸದೆ ಇರುವುದಕ್ಕೆ ನಿನಗೆಷ್ಟು ಧೈರ್ಯ? ನಿನ್ನ ಹೆಸರೇನು, ನೀನು ನಿಜವಾಗಿ ಮೋಚಿಯೇ?” ಎಂದ. ಮೋಚಿ ಇದನ್ನು ಕೇಳಿ ಗಡಗಡ ನಡುಗತೊಡಗಿದನು. ಪುರೋಹಿತನನ್ನು ನೋಡಿ, “ಮಹಾಸ್ವಾಮಿ, ಅವರು ನನ್ನನ್ನು ಗುರುತಿಸಿಬಿಟ್ಟರು. ನಾನು ಇಲ್ಲಿ ಇರಲಾರೆ. ಇಲ್ಲಿಂದ ಓಡಿಹೋಗುತ್ತೇನೆ” ಎಂದು ಓಡಿಹೋದ. ಹೀಗೆಯೇ ಮಾಯೆಯನ್ನು ಗುರುತಿಸಿದ ಕೂಡಲೆ ಅದು ಮಾಯವಾಗುತ್ತದೆ.


\section{\num{೩೬. } ನಮ್ಮ ಜೀವನವೆಂಬುದು ಒಂದು ದೀರ್ಘ ಸ್ವಪ್ನ}

ಒಂದು ಹಳ್ಳಿಯಲ್ಲಿ ಒಬ್ಬ ರೈತನಿದ್ದ. ಅವನು ನಿಜವಾದ ಜ್ಞಾನಿ. ಅವನು ಬೇಸಾಯದಿಂದ ಜೀವಿಸುತ್ತಿದ್ದ. ಅವನು ಮದುವೆಯಾಗಿದ್ದ. ಬಹಳ ಕಾಲದಮೇಲೆ ಅವನಿಗೆ ಒಂದು ಗಂಡು ಮಗುಆಯಿತು. ಮಗುವಿಗೆ ಹಾರು ಎಂದು ಹೆಸರಿಟ್ಟರು. ತಂದೆತಾಯಿಗಳು ಮಗು ವನ್ನು ಬಹಳ ಪ್ರೀತಿಸುತ್ತಿದ್ದರು. ಇದು ಸ್ವಾಭಾವಿಕವೇ ಆಗಿತ್ತು, ಏಕೆಂದರೆ

ಆದರೆ ರೈತ ಏನೂ ಆಗಿಲ್ಲವೇನೋ ಎಂಬಂತೆ ಇದ್ದ. ಅವನು ತನ್ನ ಮನೆಯವರಿಗೆ ಸಮಾಧಾನ ಮಾಡಿ, “ಈಗ ಅತ್ತು ಪ್ರಯೋಜನವೇನು” ಎಂದ. ಅವನು ಅನಂತರ ಹೊಲಕ್ಕೆ ಕೆಲಸ ಮಾಡಲು ಹೋದ. ಮನೆಗೆ ಹಿಂತಿರುಗಿದಾಗ ಹೆಂಡತಿ ಮೊದಲಿಗಿಂತ ಹೆಚ್ಚಾಗಿ ಅಳುತ್ತಿದ್ದಳು. ಅವಳು “ನೀವು ಎಷ್ಟು ನಿರ್ದಯಿ, ಮಗುವಿಗಾಗಿ ಒಂದು ತೊಟ್ಟೂ ಕಣ್ಣೀರನ್ನು ಹಾಕಲಿಲ್ಲವಲ್ಲ” ಎಂದಳು. ಆಗ ರೈತನು, “ನಾನು ಏತಕ್ಕೆ ಅಳಲಿಲ್ಲ ಗೊತ್ತೆ? ನಿನ್ನೆ ನನಗೊಂದು ಚೆನ್ನಾದ ಕನಸಾಯಿತು. ಅಲ್ಲಿ ನಾನು ಒಬ್ಬ ರಾಜನಾಗಿದ್ದೆ. ನನಗೆ ಎಂಟು ಜನ ಮಕ್ಕಳಿದ್ದರು. ಅವರೊಂದಿಗೆ ಸಂತೋಷದಿಂದ ಇದ್ದೆ. ಅನಂತರ ಎಚ್ಚರವಾಯಿತು. ಈಗ ನನಗೆ ಏನು ಮಾಡಬೇಕೆಂದು ತೋಚುತ್ತಿಲ್ಲ. ಆ ಎಂಟು ಮಕ್ಕಳಿಗಾಗಿ ಅಳಲೆ, ಈ ಒಬ್ಬ ಹಾರುವಿಗಾಗಿ ಅಳಲೆ?” ರೈತನು ಒಬ್ಬ ಜ್ಞಾನಿ, ಜಾಗೃತಾವಸ್ಥೆಯೂ ಕನಸಿ ನಂತೆಯೇ ಸುಳ್ಳು, ಎಂದೆಂದಿಗೂ ಇರುವ ಸತ್ಯ ಒಂದೇ; ಅದೇ ಆತ್ಮ ಎಂಬ ಸಾಕ್ಷಾತ್ಕಾರ ಅವನಿಗುಂಟಾಗಿತ್ತು.


\section{\num{೩೭. } “ಇದೇನೂ ಅಲ್ಲ, ಇದೇನೂ ಅಲ್ಲ!”}

ಮಾಯೆಯಿಂದ ಪಾರಾಗುವುದು ಸುಲಭವಲ್ಲ. ಜ್ಞಾನಪ್ರಾಪ್ತಿಯಾದ ಮೇಲೂ ಅದರ ವಾಸನೆ ಉಳಿಯುವುದು. ಒಬ್ಬ ಕನಸಿನಲ್ಲಿ ಒಂದು ಹುಲಿ ಯನ್ನು ಕಂಡ. ಅವನು ಎದ್ದೊಡನೆಯೆ ಕನಸು ಮಾಯವಾಯಿತು. ಆದರೆ ಅವನ ಎದೆ ಇನ್ನೂ ಡವಡವ ಬಡಿಯುತ್ತಿತ್ತು.

ಕೆಲವು ಕಳ್ಳರು ಒಂದು ಗದ್ದೆಗೆ ಬಂದರು. ಅಲ್ಲಿ ಕಳ್ಳಕಾಕರನ್ನು ಅಂಜಿಸಲು ಮನುಷ್ಯಾಕಾರದ ಗೊಂಬೆಯೊಂದನ್ನು ಇಟ್ಟಿದ್ದರು. ಇದನ್ನು ನೋಡಿ ಕಳ್ಳರಿಗೆ ಭಯವಾಗಿ ಮುಂದೆ ಹೋಗಲು ಧೈರ್ಯ ಬರಲಿಲ್ಲ. ಅವರಲ್ಲಿ ಒಬ್ಬ ಸಮೀಪಕ್ಕೆ ಹೋಗಿ ನೋಡಿದಾಗ, ಅದು ಬರೀ ಬಟ್ಟೆಯಿಂದ ಮಾಡಿದ ಮನುಷ್ಯಾಕೃತಿ ಎಂದು ಅವನಿಗೆ ತಿಳಿಯಿತು. ಅವನು ತನ್ನ ಸಂಗಡಿಗರಿಗೆ “ಅದಕ್ಕೆ ಅಂಜಬೇಕಾಗಿಲ್ಲ” ಎಂದು ಹೇಳಿದ. ಆದರೂ ಅವರು ಮುಂದು ವರಿಯಲು ಅಂಜಿದರು. ಅವರಲ್ಲಿ ಒಬ್ಬ ಧೈರ್ಯಶಾಲಿ ಡಕಾಯಿತ ಆ ಗೊಂಬೆಯ ಮೇಲೆ ಕುಳಿತು, “ಇದೇನೂ ಅಲ್ಲ; ಇದೇನೂ ಅಲ್ಲ,” ಎಂದನು. ಇದೇ “ನೇತಿ ನೇತಿ” ಮಾರ್ಗ.


\section{\num{೩೮. } ಇದೆಲ್ಲ ನಿಜವಾಗಿ ಸುಳ್ಳಾದರೆ!}

ರಾಮಲಕ್ಷ್ಮಣರು ಲಂಕೆಗೆ ಹೋಗಲು ಇಚ್ಛಿಸಿದರು. ಆದರೆ ಅವರೆದುರಿಗೆ ಸಮುದ್ರ ಇತ್ತು. ಲಕ್ಷ್ಮಣನಿಗೆ ತುಂಬಾ ಕೋಪ ಬಂತು. ಧನುರ್ಬಾಣಗಳನ್ನು ತೆಗೆದುಕೊಂಡು “ನಾನು ವರುಣ ನನ್ನು ಕೊಲ್ಲುತ್ತೇನೆ. ಈ ಸಮುದ್ರ ಲಂಕೆಗೆ ಹೋಗಲು ಆತಂಕವಾಗಿದೆ” ಎಂದ. ರಾಮ ಅವನಿಗೆ ಪರಿಸ್ಥಿತಿಯನ್ನು ಸರಿಯಾಗಿ ವಿವರಿಸಿದ. “ಲಕ್ಷ್ಮಣ ನೀನು ನೋಡುತ್ತಿರುವುದೆಲ್ಲ ಕನಸಿನಂತೆ ಒಂದು ಭ್ರಾಂತಿ. ಎದುರಿಗಿರುವ ಸಮುದ್ರ ಸುಳ್ಳು. ನಿನ್ನ ಕೋಪವೂ ಒಂದು ಕನಸು. ಒಂದು ಸುಳ್ಳಿನಿಂದ ಮತ್ತೊಂದು ಸುಳ್ಳನ್ನು ನಾಶಮಾಡುವೆ ಎನ್ನುವುದೂ ಸುಳ್ಳು.”

\chapter{ಅವನತಿಗಳು}

\section{\num{೩೯. } ಸಿದ್ಧನು ಬಿರುಗಾಳಿಯನ್ನು ನಿಲ್ಲಿಸುವುದು}

ಒಮ್ಮೆ ಒಬ್ಬ ಮಹಾ ಸಿದ್ಧಪುರುಷ ಸಮುದ್ರತೀರದಲ್ಲಿದ್ದ. ಆಗ ಒಂದು ಬಿರುಗಾಳಿ ಬಂತು. ಇದರಿಂದ ಸಿದ್ಧನಿಗೆ ತುಂಬಾ ಕೋಪ ಬಂತು. ಈ ಬಿರುಗಾಳಿ ನಿಲ್ಲಲಿ ಎಂದು ಮಂತ್ರ ಹಾಕಿದ. ಇವನ ಮಾತು ಸಿದ್ಧಿಸಿತು. ಆಗ ದೂರದಲ್ಲಿ ಒಂದು ಹಡಗು ಗಾಳಿಯ ವೇಗಕ್ಕೆ ಜೋರಾಗಿ ಹೋಗುತ್ತಿತ್ತು. ಬಿರುಗಾಳಿ ಇದ್ದಕ್ಕಿದ್ದಂತೆ ನಿಂತುಹೋದದ್ದರಿಂದ ದೋಣಿಯಲ್ಲಿದ್ದವರೆಲ್ಲ ಸಮುದ್ರದ ಪಾಲಾದರು.

ಇಷ್ಟೊಂದು ಜನರ ಸಾವಿಗೆ ಕಾರಣನಾದ ಸಿದ್ಧನಿಗೆ ಪಾಪ ಬಂತು. ಈ ಕಾರಣದಿಂದ ಅವನು ತನ್ನ ಮಾಯಾಶಕ್ತಿಯನ್ನು ಕಳೆದುಕೊಂಡು ನರಕದಲ್ಲಿ ನರಳಬೇಕಾಯಿತು.


\section{\num{೪೦. } ಅತೀಂದ್ರಿಯ ಶಕ್ತಿಗಳು ನಮಗೆ ಪ್ರಯೋಜನಕಾರಿಯ ಬದಲು ಒಂದು ಆತಂಕ}

ಒಮ್ಮೆ ಒಬ್ಬ ಸಾಧುವಿಗೆ ಅದ್ಭುತ ಸಿದ್ಧಿಗಳು ಪ್ರಾಪ್ತವಾದವು. ಇದಕ್ಕಾಗಿ ಅವನು ಅಹಂಕಾರಪಟ್ಟನು. ಆದರೆ ಅವನು ಒಳ್ಳೆಯ ಮನುಷ್ಯನಾಗಿದ್ದ, ಸ್ವಲ್ಪ ತಪಸ್ಸನ್ನೂ ಮಾಡಿದ್ದ. ಒಂದು ದಿನ ದೇವರು ಒಬ್ಬ ಸಾಧುವಿನಂತೆ ಇವನ ಎದುರಿಗೆ ಬಂದು “ಸ್ವಾಮಿಗಳೆ, ನಿಮಗೆ ಅದ್ಭುತ ಪವಾಡಗಳನ್ನು ಮಾಡುವ ಶಕ್ತಿ ಇದೆಯಂತೆ, ಹೌದೆ?” ಎಂದ. ಸಾಧು ಅವನನ್ನು ಗೌರವದಿಂದ ಬರಮಾಡಿ ಕೊಂಡು ಕುಳಿತುಕೊಳ್ಳುವುದಕ್ಕೆ ಒಂದು ಆಸನವನ್ನು ನೀಡಿದ. ಆಗ ಒಂದು ಆನೆ ಎದುರಿಗೆ ಹೋಗುತ್ತಿತ್ತು. ಸಾಧುವಿನ ವೇಶದಲ್ಲಿದ್ದ ಭಗವಂತ “ನಿಮಗೆ ಈ ಆನೆಯನ್ನು ಕೊಲ್ಲುವುದಕ್ಕೆ ಸಾಧ್ಯವೆ?” ಎಂದು ಕೇಳಿದ. “ಇದು ನನಗೆ ಸಾಧ್ಯ” ಎಂದು ಅವನು ಹೇಳಿದ. ಹಾಗೆ ಹೇಳುತ್ತ ಒಂದು ಚಿಟಿಕೆ ಮಣ್ಣನ್ನು ಮಂತ್ರಿಸಿ ಆನೆಯ ಮೇಲೆ ಎಸೆದ. ಆನೆ ಕೆಲವು ಕಾಲ ನರಳಿ ಸತ್ತುಬಿತ್ತು. ದೇವರು “ನಿನಗೆ ಅದ್ಭುತವಾದ ಶಕ್ತಿ ಇದೆ. ನೀನು ಆನೆಯನ್ನು ಕೊಂದುಬಿಟ್ಟೆ” ಎಂದನು. ಸಾಧು ಹೆಮ್ಮೆಯಿಂದ ಮುಗುಳ್ನಕ್ಕ. ಆಗ ಭಗವಂತ “ನೀನು ಸತ್ತ ಆನೆಯನ್ನು ಬದುಕಿಸಬಲ್ಲೆಯಾ?” ಎಂದು ಕೇಳಿದ. “ಅದು ಕೂಡ ಸಾಧ್ಯ” ಎಂದ ಸಾಧು. ಮತ್ತೊಂದು ಚಿಟಿಕೆ ಮಣ್ಣನ್ನು ಅದರ ಮೇಲೆ ಹಾಕಿ ಮಂತ್ರಿಸಿದ. ಆನೆ ಸ್ವಲ್ಪ ಹೊತ್ತು ನರಳಾಡಿ ಪುನಃ ಬದುಕಿ ಬಂತು. ಆಗ ದೇವರು, “ನಿನಗೇನೊ ಅದ್ಭುತ ಶಕ್ತಿ

ಇದನ್ನು ಹೇಳಿ ದೇವರು ಮಾಯವಾದ.

ಧರ್ಮದ ರೀತಿ ಬಹಳ ಸೂಕ್ಷ್ಮವಾದುದು. ಒಬ್ಬನಿಗೆ ಒಂದು ಸಾಸಿವೆ ಕಾಳಿನಷ್ಟು ಅಹಂಕಾರವಿದ್ದರೂ ಅವನು ಭಗವಂತನನ್ನು ಸಾಕ್ಷಾತ್ಕರಿಸಿಕೊಳ್ಳ ಲಾರ. ದಾರದಲ್ಲಿ ಒಂದು ಸಣ್ಣ ಎಳೆ ಹೊರಚಾಚಿದ್ದರೂ ಅದನ್ನು ಸೂಜಿಗೆ ಪೋಣಿಸಲಾಗುವುದಿಲ್ಲ.

ಅಹಂಕಾರದ ದೊಡ್ಡ ಮೂಟೆಯನ್ನೇ ಹೊತ್ತು ಶಿಥಿಲವಾದ ಸೇತುವೆಯನ್ನು ದಾಟುವುದು ಸಾಧ್ಯವೇ?


\section{\num{೪೧. } ಈಜು ಬಾರದ ಪಂಡಿತ}

ಒಂದು ಸಲ ಕೆಲವರು ದೋಣಿಯಲ್ಲಿ ಗಂಗಾನದಿಯನ್ನು ದಾಟುತ್ತಿದ್ದರು. ಅವರಲ್ಲಿ ಒಬ್ಬ ಪಂಡಿತ ತನ್ನ ಪಾಂಡಿತ್ಯವನ್ನು ಅವರೆದುರು ಕೊಚ್ಚಿಕೊಳ್ಳು ತ್ತಿದ್ದ. “ನಾನು ವೇದವೇದಾಂತ ಷಡ್ದರ್ಶನಗಳೆಲ್ಲವನ್ನೂ ಓದಿರುವೆನು” ಎಂದ. ಹತ್ತಿರವಿದ್ದ ಪ್ರಯಾಣಿಕನನ್ನು ಕೇಳಿದ, “ನಿನಗೆ ಏನಾದರೂ ವೇದಾಂತ ಗೊತ್ತಿದೆಯೆ?” ಎಂದು. “ಇಲ್ಲ ಮಹಾಶಯರೆ ನನಗೆ ಅದು ಗೊತ್ತಿಲ್ಲ” ಎಂದ. “ನೀನು ಯಾವ ತತ್ವಶಾಸ್ತ್ರಗಳನ್ನೂ ಓದಿಲ್ಲವೆ?” ಎಂದು ಕೇಳಿದ. “ಇಲ್ಲ ಮಹಾಶಯರೆ ನನಗೆ ಗೊತ್ತಿಲ್ಲ” ಎಂದ. ಪಂಡಿತ ಅಹಂಕಾರದಿಂದ ಹೀಗೆ ಮಾತನಾಡುತ್ತಿರುವಾಗ ಪ್ರಯಾಣಿಕನು ಸುಮ್ಮನೆ ಇದ್ದನು. ಆಗ ದೊಡ್ಡ ಬಿರುಗಾಳಿ ಎದ್ದಿತು. ದೋಣಿ ಇನ್ನೇನು ಮುಳುಗುವ ಸ್ಥಿತಿಗೆ ಬಂತು. ಪ್ರಯಾಣಿಕನು “ಮಹಾಶಯರೆ, ನಿಮಗೆ ಈಜು ಬರುವುದೆ?” ಎಂದು ಕೇಳಿದ. ಪಂಡಿತನು “ನನಗೆ ಈಜು ಬರುವುದಿಲ್ಲ” ಎಂದ. ಆಗ ಪ್ರಯಾಣಿಕ, “ನನಗೆ ಸಾಂಖ್ಯ, ಪಾತಂಜಲ ಇವುಗಳಾವುದೂ ಗೊತ್ತಿಲ್ಲ. ಆದರೆ ನನಗೆ ಈಜು ಬರುವುದು” ಎಂದ.

ಹಲವಾರು ಶಾಸ್ತ್ರಗಳನ್ನು ಓದಿ ಏನು ಪ್ರಯೋಜನ? ಭವಸಾಗರವನ್ನು ಹೇಗೆ ದಾಟುವುದು ಎಂಬುದನ್ನು ಒಬ್ಬ ತಿಳಿದಿರಬೇಕು. ದೇವರೊಬ್ಬನೇ ಸತ್ಯ, ಮಿಕ್ಕಿದೆಲ್ಲವೂ ಅಸತ್ಯ.


\section{\num{೪೨. } ಮನುಷ್ಯ ಒಂದು ಬಗೆದರೆ ದೇವರೊಂದು ಬಗೆಯುವನು}

ಶ್ರೀರಾಮಕೃಷ್ಣರು ಪ್ರತಾಪಚಂದ್ರ ಮಜುಂದಾರನಿಗೆ (ಪ್ರಸಿದ್ಧ ಬ್ರಹ್ಮ ಸಮಾಜವಾದಿ), “ನೀನು ವಿದ್ಯಾವಂತ ಮತ್ತು ಬುದ್ಧಿವಂತ ಮತ್ತು ಮೇಧಾವಿಯೂ ಆಗಿರುವೆ. ಕೇಶವಚಂದ್ರ ಮತ್ತು ನೀನು ಇಬ್ಬರೂ ಗೌರ್ ಮತ್ತು ನಿತಾಯಿಯ (ಶ್ರೀ ಚೈತನ್ಯದೇವ ಮತ್ತು ನಿತ್ಯಾನಂದರು) ಹಾಗೆ ಇದ್ದ ವರು. ನಿನಗೆ ಸಾಕಾದಷ್ಟು ಪ್ರಪಂಚದ ಪರಿಚಯವಿದೆ. ಬೇಕಾದಷ್ಟು ಉಪನ್ಯಾಸ ಮಾಡಿರುವೆ, ವಾದ ಮಾಡಿರುವೆ, ಹಲವು ಸಿದ್ಧಾಂತಗಳನ್ನು ಓದಿರುವೆ. ನಿನಗೆ ಇನ್ನೂ ಇವು ಬೇಜಾರಾಗಿಲ್ಲವೆ? ಈಗ ನೀನು ಇಂದ್ರಿಯ ನಿಗ್ರಹಮಾಡಿ, ಭಗ ವಂತನ ಕಡೆ ಮನಸ್ಸನ್ನು ತಿರುಗಿಸಿ, ಭಗವಂತನಲ್ಲಿ ಮುಳುಗಬೇಕು,” ಎಂದರು.

ಮಜುಂದಾರ–“ಹೌದು ನಾನದನ್ನು ಮಾಡಬೇಕು. ಅದರ ವಿಷಯದಲ್ಲಿ ಸಂಶಯವಿಲ್ಲ. ನಾನು ಇದನ್ನೆಲ್ಲ ಮಾಡುತ್ತಿರುವುದು ಕೇಶವಚಂದ್ರನ ಹೆಸರು ಪ್ರಖ್ಯಾತಿಗೆ ಬರಲಿ ಎಂದು.”

ಶ್ರೀರಾಮಕೃಷ್ಣರು ಮಂದಹಾಸದಿಂದ ಹೇಳಿದರು: “ನಾನು ನಿನಗೆ ಒಂದು ಕಥೆಯನ್ನು ಹೇಳುತ್ತೇನೆ, ಕೇಳು. ಒಬ್ಬ ಒಂದು ಬೆಟ್ಟದ ಮೇಲೆ ಒಂದು ಮನೆಯನ್ನು ಕಟ್ಟಿಸಿಕೊಂಡ. ಅದೊಂದು ಮಣ್ಣಿನಿಂದ ಮಾಡಿದ ಗುಡಿಸಲು. ಆದರೆ ಅದನ್ನು ತುಂಬಾ ಜೋಪಾನವಾಗಿ ಕಟ್ಟಿದ್ದ. ಕೆಲವು ದಿನಗಳಾದ ಮೇಲೆ ಒಂದು ಬಿರುಗಾಳಿ ಬೀಸ ತೊಡಗಿತು. ಗುಡಿಸಲು ಅಲುಗಾಡಲು ಪ್ರಾರಂಭವಾಯಿತು. ಮನುಷ್ಯ ಗುಡಿಸ ಲನ್ನು ಉಳಿಸಲು ಪ್ರಯತ್ನಪಟ್ಟ. ವಾಯುದೇವರಿಗೆ ಪ್ರಾರ್ಥನೆ ಮಾಡಿದ:

‘ವಾಯುದೇವರೆ, ದಯವಿಟ್ಟು ಈ ಗುಡಿಸಲನ್ನು ಉಳಿಸಪ್ಪ.’ ಆದರೆ ವಾಯುದೇವರು ಇದನ್ನು ಗಣನೆಗೆ ತರಲಿಲ್ಲ. ಗುಡಿಸಲು ಬೀಳುವುದರಲ್ಲಿತ್ತು. ಆಗ ಅವನು ಒಂದು ಉಪಾಯವನ್ನು ಯೋಚಿಸಿದ. ವಾಯುದೇವರ ಮಗ ಹನುಮಂತ ಎಂಬುದು ಜ್ಞಾಪಕಕ್ಕೆ ಬಂತು. ಅವನು ತುಂಬಾ ಭಕ್ತಿಯಿಂದ

“ಈಗ ನೀನು ಕೇಶವಚಂದ್ರನ ಕೀರ್ತಿಯನ್ನು ಉಳಿಸಲು ಪ್ರಯತ್ನಿಸ ಬಹುದು. ಆದರೆ ಭಗವಂತನ ದಯೆಯಿಂದ ಈ ಸಮಾಜ ಬಂತು, ಈಗ ಅದರ ಕಾಲ ಬಂದರೆ ಭಗವಂತನ ಇಚ್ಛೆಯಿಂದಲೇ ಅದು ಆಯಿತು ಎಂಬುದನ್ನು ಮರೆಯಬೇಡ. ಆದಕಾರಣ ಭಗವಂತನಲ್ಲಿ ತನ್ಮಯನಾಗು.”


\section{\num{೪೩. } ಒಬ್ಬ ಆಲೋಚಿಸಿದಂತೆ ಆಗುವುದು}

ಒಬ್ಬ ಮಂತ್ರವಾದಿ ತನಗೆ ಗೊತ್ತಿರುವ ಮಾಯಾ ಮಂತ್ರಗಳನ್ನು ರಾಜನ ಎದುರಿಗೆ ತೋರಿಸುತ್ತಿದ್ದ. ಮಂತ್ರವಾದಿ “ಮಾಯೆಯೇ ಬಾ, ಭ್ರಾಂತಿಯೇ ಬಾ, ರಾಜರೇ ನನಗೆ ಹಣ ಕೊಡಿ, ಬಟ್ಟೆ ಕೊಡಿ” ಎಂದು ಕೇಳುತ್ತಿದ್ದ. ಇದ್ದಕ್ಕಿ ದ್ದಂತೆಯೆ ಅವನ ನಾಲಿಗೆ ತಳಕೆಳಗಾಯಿತು. ಅವನಿಗೆ ಕುಂಭಕ ಪ್ರಾಪ್ತ ವಾಯಿತು. ಅವನು ಯಾವ ಮಾತನ್ನೂ ಆಡಲಿಲ್ಲ. ಯಾವ ಶಬ್ದವೂ ಬರುವು ದಕ್ಕೆ ಆಗಲಿಲ್ಲ. ಅವನು ಸತ್ತು ಹೋದ ಎಂದು ಜನ ಭಾವಿಸಿದರು. ಅವನನ್ನು ಕುಳ್ಳಿರಿಸಿ ಅವನ ಸುತ್ತಲೂ ಗೋಡೆ ಕಟ್ಟಿದರು. ಒಂದು ಸಾವಿರ ವರುಷಗಳಾದ ಮೇಲೆ ಯಾರೋ ಅವನನ್ನು ಹೂಳಿದ್ದ ಸ್ಥಳದಿಂದ ಮೇಲಕ್ಕೆತ್ತಿದ್ದರು. ಅಲ್ಲಿ ಅವನು ಸಮಾಧಿಸ್ಥಿತಿಯಲ್ಲಿದ್ದ. ಅವನನ್ನು ಅಲ್ಲಾಡಿಸಿದಾಗ ಅವನ ನಾಲಿಗೆಯು ಸ್ವಸ್ಥಿತಿಗೆ ಮರಳಿದ ಕೂಡಲೇ ಅವನಿಗೆ ಪ್ರಜ್ಞೆ ಬಂತು. ಅವನು ಒಂದು ಸಾವಿರ ವರುಷದ ಹಿಂದೆ ಹೇಗೆ ಕೂಗುತ್ತಿದ್ದನೋ ಹಾಗೆಯೇ ಈಗಲೂ ಶುರುಮಾಡಿದ: “ಭ್ರಾಂತಿಯೇ ಬಾ, ಮಾಯೆಯೇ ಬಾ. ರಾಜರೇ ನನಗೆ ಹಣಕೊಡಿ. ಬಟ್ಟೆ ಕೊಡಿ” ಅಂತ.

ದೇವರು ಕಲ್ಪತರು. ನಾನು ಏನೇ ಕೇಳಿದರೂ ಅವನು ಅದನ್ನು ಕೊಡುವನು. ಆದರೆ ನೀನು ಕಲ್ಪತರುವಿನ ಕೆಳಗೆ ನಿಂತು ಪ್ರಾರ್ಥಿಸಬೇಕು. ಆಗಲೆ ನಿನ್ನ ಪ್ರಾರ್ಥನೆ ಈಡೇರುವುದು. ನೀನು ಮತ್ತೊಂದನ್ನು ತಿಳಿದುಕೊಂಡಿರಬೇಕು. ದೇವರಿಗೆ ನಮ್ಮ ಮನಸ್ಸಿನಲ್ಲಿ ಏನಿದೆಯೊ ಅದು ಗೊತ್ತಿದೆ. ಒಬ್ಬ ಮನುಷ್ಯ ಸಾಧನೆ ಮಾಡುತ್ತಿದ್ದಾಗ ಅವನ ಮನಸ್ಸಿನಲ್ಲಿ ಯಾವ ಬಯಕೆ ಇದೆಯೋ ಅದು ಸಿದ್ಧಿಸುವುದು. ಆಲೋಚನೆಯಂತೆ ಫಲ.


\section{\num{೪೪. } ಈಗ ಅವಳು ಚೆನ್ನಾಗಿದ್ದಾಳೆ}

ಮಾಯಾಮಂತ್ರ ಸ್ವಲ್ಪ ಗೊತ್ತಿದ್ದವರಿಗೆ ಹೆಸರು ಕೀರ್ತಿಗಳೆಲ್ಲ ಬರುವುದು. ಅವರಲ್ಲಿ ಅನೇಕರು ಗುರುಗಳಾಗಲು ಇಚ್ಛಿಸುವರು. ಅವರಿಗೆ ಅನೇಕರ ಪರಿ ಚಯವಾಗುವುದು. ಅನೇಕ ಭಕ್ತರು, ಶಿಷ್ಯರು ಸಿಕ್ಕುವರು. ಅಂತಹ ಗುರುಗಳನ್ನು ಕಂಡಾಗ “ಅವನ ದೆಸೆ ಖುಲಾಯಿಸಿದೆ, ಎಷ್ಟೊಂದು ಜನ ಅವನನ್ನು ನೋಡಲು ಬರುತ್ತಾರೆ, ಅವನಿಗೆ ಅನೇಕ ಶಿಷ್ಯರು ಮತ್ತು ಅನುಯಾಯಿಗಳು ಇದ್ದಾರೆ, ಅವನ ಮನೆಯಲ್ಲಿ ಕುರ್ಚಿ, ಸೋಫ ಇವುಗಳೆಲ್ಲ ತುಂಬಿಹೋಗಿವೆ. ಅನೇಕರು ಅವನಿಗೆ ಬಹುಮಾನವನ್ನು ಕೊಡುವರು, ಅವನಿಗೆ ಅಷ್ಟೊಂದು ಐಶ್ವರ್ಯ ಇದೆ, ಅವನು ಬೇಕಾದರೆ ಎಷ್ಟೊಂದು ಜನರಿಗೆ ಊಟ ಕೊಡಬಲ್ಲ” ಎಂದು ಜನ ಆಡಿಕೊಳ್ಳುತ್ತಾರೆ.

ಗುರುವಿನ ವೃತ್ತಿ ವೇಶ್ಯೆಯ ವೃತ್ತಿಯಂತೆ. ಹಣಕ್ಕೆ, ಗೌರವಕ್ಕೆ ತನ್ನನ್ನು ತಾನೇ ಮಾರಿಕೊಂಡಂತೆ. ಇಂತಹ ಕ್ಷುಲ್ಲಕ ವಸ್ತುಗಳಿಗೆ ದೇಹ, ಮನಸ್ಸು ಮತ್ತು ಜೀವನವನ್ನು ಒತ್ತೆ ಇಡಬೇಕಾಗಿಲ್ಲ. ಅದರಿಂದ ಭಗವಂತನನ್ನೇ ಸಾಕ್ಷಾತ್ಕಾರ ಮಾಡಿಕೊಳ್ಳಬಹುದು. ಒಬ್ಬ ಮನುಷ್ಯ ಹೆಂಗಸಿನ ವಿಷಯದಲ್ಲಿ ಹೇಳುತ್ತಿದ್ದ, “ಓ ಈಗ ಅವಳ ಅದೃಷ್ಟ ಖುಲಾಯಿಸಿದೆ. ಅವಳಿಗೆ ಎಲ್ಲಾ ಅನುಕೂಲಗಳು ಇವೆ. ಅವಳು ಒಂದು ಮನೆಯನ್ನು ಬಾಡಿಗೆಗೆ ತೆಗೆದುಕೊಂಡಿದ್ದಾಳೆ. ಅಲ್ಲಿ ಹಾಸಿಗೆ, ಸುಪ್ಪತ್ತಿಗೆ, ದಿಂಬು ಇನ್ನೂ ಇತರ ವಸ್ತುಗಳು ಇವೆ. ಎಷ್ಟೊಂದು ಜನರನ್ನು ತನ್ನ ಕೆಳಗೆ ಇಟ್ಟುಕೊಂಡಿದ್ದಾಳೆ! ಅವರು ಯಾವಾಗಲೂ ಅವಳನ್ನು ಕಾಣುವುದಕ್ಕೆ ಬರುತ್ತಿದ್ದಾರೆ,” ಎಂದು. ಬೇರೆ ಮಾತಿನಲ್ಲಿ ಹೇಳು ವುದಾದರೆ ಅವಳು ಈಗ ವೇಶ್ಯೆ ಆಗಿದ್ದಾಳೆ. ಅದಕ್ಕೇ ಅವಳಿಗೆ ಇಷ್ಟೊಂದು ಸೌಕರ್ಯಗಳು. ಮೊದಲು ಅವಳು ಗೌರವಸ್ಥರ ಮನೆ ಯಲ್ಲಿ ಕೆಲಸ ಮಾಡುತ್ತಿದ್ದಳು. ಈಗ ಅವಳು ಸೂಳೆಯಾಗಿದ್ದಾಳೆ. ಕೆಲಸಕ್ಕೆ ಬಾರದ ವಸ್ತುವಿಗೆ ತನ್ನ ಬಾಳನ್ನೇ ಹಾಳು ಮಾಡಿಕೊಂಡಿದ್ದಾಳೆ.


\section{\num{೪೫. } ಹುಚ್ಚನಂತೆ ನಟಿಸುವುದು ಕೂಡ ಅಪಾಯಕರ}

ಒಬ್ಬನಿಗೆ ತುಂಬಾ ಸಾಲವಿತ್ತು. ಅದರಿಂದ ಪಾರಾಗುವುದಕ್ಕೆ ಅವನು ಹುಚ್ಚನಂತೆ ನಟಿಸತೊಡಗಿದನು. ವೈದ್ಯರಿಗೆ ಅವನ ರೋಗವನ್ನು ಗುಣ ಮಾಡಲು ಆಗಲಿಲ್ಲ. ಅವನಿಗೆ ಹೆಚ್ಚು ಚಿಕಿತ್ಸೆ ಮಾಡಿದಷ್ಟೂ ಅವನ ಹುಚ್ಚು ಹೆಚ್ಚಾಯಿತು. ಕೊನೆಗೆ ಒಬ್ಬ ಬುದ್ಧಿವಂತನಾದ ವೈದ್ಯ ಸತ್ಯವನ್ನು ಕಂಡುಹಿಡಿದ. ಹುಚ್ಚನಂತೆ ನಟಿಸುತ್ತಿದ್ದವನ ಹತ್ತಿರ ಹೋಗಿ, ಅವನಿಗೆ ಹೀಗೆ ಬುದ್ಧಿ ಹೇಳಿದನು: “ಸ್ನೇಹಿತನೆ, ನೀನು ಏನು ಮಾಡುತ್ತಿರುವೆ? ಹುಚ್ಚನಂತೆ ನಟಿಸಿದರೆ ಕೊನೆಗೆ ಹುಚ್ಚನಾಗುವೆ. ಜೋಪಾನ! ಆಗಲೇ ನಿನ್ನಲ್ಲಿ ಕೆಲವು ಹುಚ್ಚನ ಲಕ್ಷಣಗಳು ಕಾಣಬರುತ್ತಿವೆ.” ಈ ಬುದ್ಧಿವಾದದಿಂದ ಆತ ಹುಚ್ಚನಂತೆ ನಟಿಸು ವುದನ್ನು ಬಿಟ್ಟುಬಿಟ್ಟ. ಒಬ್ಬ ಮತ್ತೊಬ್ಬನಂತೆ ನಟಿಸಿದರೆ ಅವನ ಹಾಗೆಯೇ ಆಗುತ್ತಾನೆ.


\section{\num{೪೬. } ನೀನು ಒಳ್ಳೆಯದನ್ನು ಸ್ವಾಗತಿಸಿದರೆ ಕೆಟ್ಟದ್ದೂ ಬರುವುದು}

ಒಬ್ಬ ಬ್ರಾಹ್ಮಣ ಒಂದು ತೋಟವನ್ನು ಇಟ್ಟುಕೊಂಡಿದ್ದ. ಅವನು ಅತ್ಯಂತ ಆದರದಿಂದ ಅದನ್ನು ನೋಡಿಕೊಳ್ಳುತ್ತಿದ್ದ. ಒಂದು ಸಲ ಹಸುವೊಂದು ಅವನ ತೋಟಕ್ಕೆ ನುಗ್ಗಿ ಅವನು ಅಧಿಕವಾಗಿ ಪ್ರೀತಿಸುತ್ತಿದ್ದ ಮಾವಿನ ಸಸಿಗಳನ್ನು ಮೇಯಿತು. ಅದನ್ನು ನೋಡಿದ ಬ್ರಾಹ್ಮಣನಿಗೆ ಕೋಪವನ್ನು ತಡೆಯಲಾಗ ಲಿಲ್ಲ. ಹಸುವನ್ನು ಚೆನ್ನಾಗಿ ಹೊಡೆದನು. ಇದರಿಂದ ಹಸು ಸತ್ತುಹೋಯಿತು. ಈ ಸುದ್ದಿ ಕಾಡುಗಿಚ್ಚಿನಂತೆ ಊರಲ್ಲೆಲ್ಲ ಹರಡಿತು. ಬ್ರಾಹ್ಮಣ ಒಂದು ಗೋವನ್ನು ಹತ್ಯೆಮಾಡಿದ್ದಾನೆ ಎಂಬ ವದಂತಿ ಊರಿನಲ್ಲೆಲ್ಲ ಹಬ್ಬಿತು. ಗೋಹತ್ಯಾ ಪಾಪಕ್ಕೆ ಒಳಗಾಗಿರುವೆ ಎಂದು ಯಾರಾದರೂ ಹೇಳಿದರೆ ವೇದಾಂತಿಯಾದ ಆ ಬ್ರಾಹ್ಮಣ ಹೇಳುತ್ತಿದ್ದ: “ಗೋವಿನ ಸಾವಿಗೆ ಕಾರಣ ನಾನಲ್ಲ. ನನ್ನ ಕೈ ಅದನ್ನು ಮಾಡಿತು. ದೇವೇಂದ್ರನು ಈ ಕೈಗಳಿಗೆ ಅಧಿದೇವತೆ, ಅವನೇ ಈ ಪಾಪಕ್ಕೆ ಕಾರಣ” ಎಂದು. ಸ್ವರ್ಗದಲ್ಲಿದ್ದ ಇಂದ್ರನಿಗೆ ಈ ಸಮಾಚಾರ ತಿಳಿಯಿತು. ಇಂದ್ರ ಒಬ್ಬ ವೃದ್ಧಬ್ರಾಹ್ಮಣನಂತೆ ವೇಷ ಬದಲಾಯಿಸಿಕೊಂಡು ತೋಟದ ಮಾಲಿಕನ ಹತ್ತಿರ ಬಂದು “ಯಾರದಯ್ಯ ಈ ತೋಟ?” ಎಂದ. ಬ್ರಾಹ್ಮಣ “ನನ್ನದು” ಎಂದ. ಇಂದ್ರ, “ಇದು ಸುಂದರವಾದ ತೋಟ. ನಿನ್ನ ಬಳಿ ಚೆನ್ನಾಗಿ ಕೆಲಸ ಮಾಡುವ ತೋಟಗಾರನಿರಬೇಕು” ಎಂದು ಅವನನ್ನು ಹೊಗಳಿದ. ಆಗ ಬ್ರಾಹ್ಮಣ “ಇದನ್ನೆಲ್ಲ ನಾನೇ ಮಾಡಿದ್ದು. ಈ ಗಿಡಗಳನ್ನು ನಾನೇ ನೆಟ್ಟವನು”



\section{\num{೪೭. } ಪವಾಡ ಶಕ್ತಿಗಳು ಹೇಗಿರುತ್ತವೆ?}

ಒಂದು ಕಾಲದಲ್ಲಿ ನಾನು ಹೃದಯನು ಹೇಳಿದಂತೆ ಕೇಳುತ್ತಿದ್ದೆ. ಅವನು “ಪವಾಡಶಕ್ತಿಗಳನ್ನು ನೀಡೆಂದು ಭಗವತಿಯನ್ನು ಪ್ರಾರ್ಥಿಸು” ಎಂದನು. ನಾನು ದೇವಸ್ಥಾನಕ್ಕೆ ಹೋದೆ. ಅಲ್ಲಿ ನಾನೊಂದು ದಿವ್ಯದರ್ಶನದಲ್ಲಿ ಮೂವತ್ತು, ಮೂವತ್ತೈದು ವಯಸ್ಸಿನ ವಿಧವೆ ಹೇಸಿಗೆಯನ್ನು ತನ್ನ ಮೈಯೆಲ್ಲ ಬಳಿದುಕೊಂಡಿದ್ದನ್ನು ಕಂಡೆ. ಆಗ ಪವಾಡ ಶಕ್ತಿಗಳು ಅಮೇಧ್ಯಕ್ಕೆ ಸಮ ಎಂದು ಗೊತ್ತಾಯಿತು. ನನಗೆ ಹೃದಯನ ಮೇಲೆ ತುಂಬಾ ಕೋಪ ಬಂತು, ಏಕೆಂದರೆ ಅವನೇ ನನಗೆ ಪವಾಡಶಕ್ತಿಗಳಿಗಾಗಿ ಪ್ರಾರ್ಥಿಸು ಎಂದು ಪ್ರಚೋದಿಸಿದವನು.


\section{\num{೪೮. } ಗೋಶಾಲೆಯಲ್ಲಿ ಕುದುರೆಗಳು}

ದೇವರನ್ನು ಕಂಡಿಲ್ಲದ ಮನುಷ್ಯನ ಬೋಧನೆ ಸರಿಯಾದ ಪ್ರಭಾವ ಬೀರು ವುದಿಲ್ಲ. ಅವನು ಒಂದು ವಿಷಯವನ್ನು ಸರಿಯಾಗಿರುವುದನ್ನು ಹೇಳಿದರೆ ಮರುಕ್ಷಣವೆ ಏನೇನೊ ಅಸಂಬದ್ಧವನ್ನು ಹೇಳುವನು.

ಬ್ರಹ್ಮಸಮಾಜದ ನೇತಾರನೊಬ್ಬನಾದ ಸಮಾಧ್ಯಾಯಿಯು ಒಂದು ಉಪ ನ್ಯಾಸವನ್ನು ಕೊಟ್ಟ. ಅವನು “ದೇವರು ಮಾತು ಮತ್ತು ಮನಸ್ಸಿಗೆ ಅತೀತ. ಅವನು ನೀರಸ. ನಿನ್ನಲ್ಲಿರುವ ಭಕ್ತಿ ಮತ್ತು ಪ್ರೀತಿಯ ರಸವನ್ನು ಅವನಿಗೆ ಲೇಪಿಸಿ ಅವನನ್ನು ಪೂಜಿಸು” ಎಂದು ಹೇಳಿದ. ನೋಡಿ! ದೇವರನ್ನು ನೀರಸ ಎನ್ನುತ್ತಾನೆ! ಭಗವಂತನೇ ಎಲ್ಲಾ ರಸದ ಮೂಲ. ಅವನು ಸಚ್ಚಿದಾನಂದ ಸ್ವರೂಪ. ಅಂತಹ ಉಪನ್ಯಾಸದಿಂದ ಏನು ಪ್ರಯೋಜನ? ಜನರಿಗೆ ಅವನು ಏನನ್ನಾದರೂ ಬೋಧಿಸಲು ಸಾಧ್ಯವೆ? ಅಂತಹ ಉಪನ್ಯಾಸಕ ತನ್ನ ಮಾವನ ಮನೆಯ ಗೊಂತಿನ ತುಂಬ ಕುದುರೆಗಳು ಇವೆ ಎಂದು ಹೇಳಿದ ವ್ಯಕ್ತಿಯಂತೆ. ಕುದುರೆಗಳು ಎಂದಾದರೂ ಹಸುವಿನ ಗೊಂತಿನಲ್ಲಿರುತ್ತವೆಯೆ? ಇದರಿಂದ ಅಲ್ಲಿ ಕುದುರೆಗಳೇ ಇಲ್ಲವೆನ್ನಬಹುದು. ಅದರಂತೆಯೇ ಅಲ್ಲಿ ಹಸುಗಳೂ ಇಲ್ಲವೆನ್ನಬಹುದು.


\section{\num{೪೯. } ಆಕರ್ಷಣೀಯ ಆತಂಕಗಳು}

ಶ್ರೀರಾಮಕೃಷ್ಣರು ಮಹೇಂದ್ರ ಮುಖರ್ಜಿ ಎಂಬ ಭಕ್ತನಿಗೆ ಹೇಳುತ್ತಾರೆ, “ನಿನಗೆ ಮಕ್ಕಳಿಲ್ಲ. ನೀನು ಯಾರಿಗೂ ದಾಸನಾಗಿಲ್ಲ. ಆದರೂ ನಿನಗೆ ವಿರಾಮವೇ ಇಲ್ಲವಲ್ಲ! ದೇವರೇ ನಿನ್ನನ್ನು ಕಾಪಾಡ


\chapter{ಅಹಂಕಾರ}

\section{\num{೫೦. } ನನ್ನ ಸಮಾನ ಯಾರೂ ಇಲ್ಲ ಎನ್ನುವ ದುರಹಂಕಾರ}

ಹಸು ಹಂಬಾ ಎಂದು ಕೂಗುವುದು. ಹಂಬಾ ಎಂದರೆ ‘ನಾನು’ ಎಂದು. ಅದಕ್ಕಾಗಿ ಅದು ಅಷ್ಟೊಂದು ದುಃಖಕ್ಕೆ ಒಳಗಾಗುವುದು. ಅದನ್ನು ನೇಗಿಲಿಗೆ ಕಟ್ಟಿ ಮಳೆ ಬಿಸಿಲೆನ್ನದೆ ಉಳುತ್ತಾರೆ. ಅನಂತರ ಒಂದು ಕಟುಕ ಬಂದು ಕೊಲ್ಲಬಹುದು. ಅದರ ಚರ್ಮದಿಂದ ಎಕ್ಕಡ ಮಾಡುತ್ತಾರೆ. ತಮ್ಮಟೆ ಮಾಡು ತ್ತಾರೆ. ಆಗ ಅದನ್ನು ಚೆನ್ನಾಗಿ ಹೊಡೆಯುತ್ತಾರೆ. ಅಲ್ಲಿಗೇ ಅದರ ಶಿಕ್ಷೆ ನಿಲ್ಲುವುದಿಲ್ಲ. ಅನಂತರ ಅದರ ಕರುಳಿನಿಂದ ದಾರ ಮಾಡಿ ಅದನ್ನು ಹತ್ತಿ ಹೆಕ್ಕುವ ಬಿಲ್ಲಿಗೆ ಬಿಗಿಯುತ್ತಾರೆ. ಆಗ ಅದು, ಹಂಬ ಹಂಬ, ಅಂದರೆ ನಾನು ನಾನು ಎಂದು ಹೇಳುವುದಿಲ್ಲ. ತುಹು ತುಹು, ನೀನು ನೀನು ಎನ್ನುವುದು. ಆಗ ಮಾತ್ರ ಅದರ ಕಷ್ಟ ಪರಿಹಾರ.

ದೇವರೇ, ನಾನು ನಿನ್ನ ಸೇವಕ, ನೀನು ನನ್ನ ಯಜಮಾನ. ನಾನು ಮಗು, ನೀನು ತಾಯಿ. ಎಲ್ಲ ದುಃಖಕ್ಕೂ ಅಹಂಕಾರವೇ ಮೂಲ.


\section{\num{೫೧. } ದುರಹಂಕಾರ ಹಾನಿಕರವಾದುದು}

ಗುರುವಿನಲ್ಲಿ ಅಗಾಧವಾದ ಭಕ್ತಿಯಿದ್ದ ಶಿಷ್ಯನೊಬ್ಬ ಗುರುವಿನ ನಾಮ ಸ್ಮರಣೆಯ ಸಹಾಯದಿಂದಲೇ ನದಿಯ ಮೇಲೆ ನಡೆದುಕೊಂಡು ಹೋದ.



\section{\num{೫೨. } ಶಂಕರಾಚಾರ್ಯರು ಮತ್ತು ಅವರ ದಡ್ಡ ಶಿಷ್ಯ}

ಶಂಕರಾಚಾರ್ಯರಿಗೆ ಒಬ್ಬ ದಡ್ಡ ಶಿಷ್ಯನಿದ್ದ. ಶಂಕರಾಚಾರ್ಯರು ಏನು ಮಾಡಿ ದರೂ ಅದನ್ನೆಲ್ಲ ಅನುಕರಿಸುತ್ತಿದ್ದ. ಶಂಕರಾಚಾರ್ಯರು ‘ಶಿವೋಽಹಂ’ ಅಂದರೆ, ಶಿಷ್ಯನು ಕೂಡ ‘ಶಿವೋಽಹಂ’ ಎಂದನು. ಇವನಿಗೆ ಬುದ್ಧಿ ಕಲಿಸಬೇಕೆಂದು ಶಂಕರಾಚಾರ್ಯರು ಅಂದುಕೊಂಡರು. ಒಮ್ಮೆ ಅವರು ಕಮ್ಮಾರನ ಕುಲುಮೆ ಹತ್ತಿರ ಹೋಗುತ್ತಿದ್ದರು. ಶಂಕರಾಚಾರ್ಯರು ದ್ರವರೂಪದಲ್ಲಿರುವ ಬಿಸಿ ಕಬ್ಬಿಣ ವನ್ನು ಕುಡಿದರು. ಶಿಷ್ಯನಿಗೂ ಹಾಗೆ ಮಾಡಲು ಹೇಳಿದರು. ಆದರೆ ಗುರುಗಳ ಈ ಕೆಲಸವನ್ನು ಅವನಿಗೆ ಮಾಡಲು ಸಾಧ್ಯವಾಗಲಿಲ್ಲ. ಅಂದಿನಿಂದ ‘ಶಿವೋಽಹಂ’ ಹೇಳುವುದನ್ನು ನಿಲ್ಲಿಸಿದ. ಅನುಕರಣೆ ಅವನತಿಯ ಚಿಹ್ನೆ.


\section{\num{೫೩. } ಶಿವನ ನಂದಿ ಹಲ್ಲನ್ನು ತೋರಿಸಿದಾಗ}

ದೇವರೊಬ್ಬನೇ ಎಲ್ಲವನ್ನು ಮಾಡುವವನು. ನಾವುಗಳೆಲ್ಲ ಅವನ ನಿಮಿತ್ತ. ಜ್ಞಾನಿಗೂ ಕೂಡ ಅಹಂಕಾರಪಡಲು ಆಸ್ಪದವಿಲ್ಲ. ಶಿವನ ಮಹಿಮೆಯನ್ನು ರಚಿಸಿದ ಕವಿ ತಾನು ತುಂಬ ಶ್ರೇಷ್ಠವಾದುದನ್ನು ರಚಿಸಿರುವೆನು ಎಂದು ಅಹಂಕಾರಪಟ್ಟ. ಆದರೆ ಶಿವನ ನಂದಿ ತನ್ನ ಹಲ್ಲುಗಳನ್ನು ತೋರಿಸಿದಾಗ ಅವನ ಅಹಂಕಾರ ತಗ್ಗಿತು. ನಂದಿಯ ಬಾಯಲ್ಲಿ ಇದ್ದ ಪ್ರತಿಯೊಂದು ಹಲ್ಲೂ ಅವನು ರಚಿಸಿದ ಸ್ತೋತ್ರದ ಶಬ್ದವಾಗಿತ್ತು. ನಿನಗೆ ಇದರ ಅರ್ಥ ಗೊತ್ತಾ ಯಿತೆ? ಆ ಶಬ್ದಗಳು ಅನಾದಿ ಕಾಲದಿಂದ ಇದ್ದವು. ಲೇಖಕನು ಅವನ್ನು ಕಂಡು ಹಿಡಿದನು, ಅಷ್ಟೆ.


\section{\num{೫೪. } ಅಹಂಕಾರ}

ನನ್ನ ಸಮಾನ ಯಾರೂ ಇಲ್ಲ ಎಂದು ತಿಳಿಯುವುದು–ಕೆಲವು ಗ್ರಂಥ ಗಳನ್ನು ಓದಿದವರು ಈ ಅಹಂಕಾರದಿಂದ ಪಾರಾಗಲಾರರು. ಒಂದು ಸಲ ಕಾಳೀಕೃಷ್ಣ ಠಾಕೂರರೊಂದಿಗೆ ದೇವರ ವಿಷಯವನ್ನು ಕುರಿತು ಮಾತನಾಡಿದೆ. ಆತ “ನನಗೆ ಇದೆಲ್ಲ ಗೊತ್ತಿದೆ” ಎಂದ. ಆಗ ನಾನು “ಡೆಲ್ಲಿಯನ್ನು ನೋಡಿ ಬಂದವನು ಆ ಬಗ್ಗೆ ಜಂಬ ಕೊಚ್ಚಿಕೊಳ್ಳುವನೇ? ಒಬ್ಬ ಸಭ್ಯವ್ಯಕ್ತಿ ತಾನು ಸಭ್ಯವ್ಯಕ್ತಿ ಎಂದು ಎಲ್ಲರೆದುರಿಗೆ ಡಂಗುರ ಹೊಡೆಯುವುದಿಲ್ಲ,” ಎಂದೆ.

ಅಹಂಕಾರ ಮನುಷ್ಯನ ತಲೆಯನ್ನು ಹೇಗೆ ತಿರುಗಿಸುತ್ತದೆ! ದಕ್ಷಿಣೇಶ್ವರ ದೇವಸ್ಥಾನದಲ್ಲಿ ಕೆಲಸ ಮಾಡುವ ಝಾಡಮಾಲಿ ಇದ್ದಳು. ಅವಳ ಅಹಂಕಾರಕ್ಕೆ ಪಾರವೇ ಇರಲಿಲ್ಲ. ಎಲ್ಲಾ ಕೆಲವು ಒಡವೆಗಳಿಗಾಗಿ. ಒಂದು ದಿನ ಎದುರಿಗೆ ದಾರಿಯಲ್ಲಿ ಕೆಲವು ಗಂಡಸರು ಹೋಗುತ್ತಿದ್ದರು. ಆಕೆ ಅವರಿಗೆ “ಏಯ್! ರಸ್ತೆಯ ಆಚೆ ಬದಿಗೆ ಹೋಗಿ” ಎಂದು ಅರಚಿಕೊಂಡಳು. ಕಸ ಗುಡಿಸುವವಳೇ ಹೀಗೆ ಮಾತನಾಡಿದರೆ ಇನ್ನು ಬೇರೆಯವರ ಅಹಂಕಾರದ ವಿಷಯ ಕೇಳಬೇಕೆ?

\chapter{ಪೂರ್ವ ಸಂಸ್ಕಾರಗಳು}

\section{\num{೫೫. } ಪೂರ್ವ ಸಂಸ್ಕಾರಗಳು ಬಹಳ ಬಲವಾದವು}

ಹುಟ್ಟಿನಿಂದ ಬಂದ ಸ್ವಭಾವಗಳು ಎಷ್ಟು ಬಲಶಾಲಿ ಎಂಬುದನ್ನು ಹೇಳುತ್ತೇನೆ ಕೇಳಿ. ಒಬ್ಬ ರಾಜಕುಮಾರ ಹಿಂದಿನ ಜನ್ಮದಲ್ಲಿ ಅಗಸರವ ನಾಗಿದ್ದ. ಈ ಜನ್ಮದಲ್ಲಿ ತನ್ನ ಸಂಗಡಿಗರೊಂದಿಗೆ ಆಟವಾಡುತ್ತಿದ್ದಾಗ ಅವನು ಹೇಳಿದ: “ಈ ಆಟಗಳೆಲ್ಲ ಸಾಕು. ನಾನು ಹೊಸದೊಂದು ಆಟವನ್ನು ತೋರಿಸಿ



\section{\num{೫೬. } ಹಿಂದೂ ಬಲಾತ್ಕಾರದಿಂದ ಮಹಮ್ಮದೀಯನಾದ}

ಒಂದು ಸಲ ಒಬ್ಬ ಒಳ್ಳೆಯ ಹಿಂದೂ ಭಕ್ತನಿದ್ದ. ಅವನು ಯಾವಾಗಲೂ ಭಗವತಿಯನ್ನು ಪೂಜಿಸುತ್ತಿದ್ದ ಮತ್ತು ನಾಮ ಸ್ಮರಣೆಯನ್ನು ಮಾಡುತ್ತಿದ್ದ. ಮುಸಲ್ಮಾನರು ದೇಶವನ್ನು ಗೆದ್ದ ಮೇಲೆ ಈ ಹಿಂದೂವನ್ನು ಮಹಮ್ಮದೀಯ ಧರ್ಮಕ್ಕೆ ಬಲಾತ್ಕಾರದಿಂದ ಮತಾಂತರಿಸಿದರು. “ಇಂದಿನಿಂದ ನೀನು ಮಹಮ್ಮದೀಯ. ‘ಅಲ್ಲಾ’ ಎಂದೆನ್ನು. ಅಲ್ಲಾನ ಹೆಸರನ್ನೇ ಉಚ್ಚಾರ ಮಾಡಬೇಕು” ಎಂದರು. ಅವನು ಬಹಳ ಕಷ್ಟದಿಂದ ಅಲ್ಲಾನ ಹೆಸರನ್ನು ಹೇಳಿದ. ಆದರೆ ಮಧ್ಯೆಮಧ್ಯೆ ‘ಜಗದಂಬಾ’ ಎಂದು ಹೇಳುತ್ತಿದ್ದ. ಮಹಮ್ಮದೀಯರು ಇದಕ್ಕಾಗಿ ಅವನನ್ನು ಹೊಡೆಯಲು ಯತ್ನಿಸಿದರು. ಆಗ ಹೊಸದಾಗಿ ಮಹಮ್ಮದೀಯನಾದವನು “ದಯವಿಟ್ಟು ನನ್ನನ್ನು ಕ್ಷಮಿಸಿ. ನನ್ನನ್ನು ಕೊಲ್ಲಬೇಡಿ. ನಾನು ಅಲ್ಲಾನ ಹೆಸರನ್ನು ಉಚ್ಚರಿಸುವುದಕ್ಕೆ ಬಹಳ ಪ್ರಯತ್ನ ಮಾಡುತ್ತಿರುವೆ. ಆದರೆ ನನ್ನ ಜಗದಂಬ ನನ್ನ ಕಂಠದ ತನಕ ತುಂಬಿಕೊಂಡಿರುವಳು. ಅವಳು ಅಲ್ಲಾನ ಹೆಸರನ್ನು ಹೊರಗಟ್ಟುತ್ತಿರುವಳು.” ಹಿಂದಿನ ಸಂಸ್ಕಾರವನ್ನು ಬಿಡುವುದು ಅಷ್ಟು ಸುಲಭವಲ್ಲ.


\section{\num{೫೭. } ಭಗವಂತನ ದೃಷ್ಟಿಯಲ್ಲಿ ಯಾವುದೂ ನಾಶವಾಗುವುದಿಲ್ಲ}

ಒಬ್ಬನು ಶವಸಾಧನೆಯಲ್ಲಿ ನಿರತನಾಗಿದ್ದ. ಭಗವತಿಯನ್ನು ಒಂದು ಅರಣ್ಯದ ಮಧ್ಯದಲ್ಲಿ ಪೂಜಿಸುತ್ತಿದ್ದ. ಮೊದಮೊದಲು ಭಯಂಕರವಾದ ದೃಶ್ಯ ಗಳನ್ನು ಕಂಡ. ಕೊನೆಗೆ ಒಂದು ಹುಲಿ ಅವನ ಮೇಲೆ ಬಿದ್ದು ಅವನನ್ನು ಕೊಂದು ಹಾಕಿತು. ಮತ್ತೊಬ್ಬ ಆ ದಾರಿಯಲ್ಲಿ ಹೋಗುತ್ತಿದ್ದವನು ಹುಲಿಯನ್ನು ಕಂಡು ರಕ್ಷಣೆಗಾಗಿ ಒಂದು ಮರವನ್ನು ಹತ್ತಿದ. ಅವನು ಕೆಳಗೆ ಇಳಿದ ಮೇಲೆ ಭಗವತಿಯನ್ನು ಪೂಜಿಸುವುದಕ್ಕೆ ಎಲ್ಲಾ ಅಣಿಯಾಗಿರುವುದನ್ನು ಕಂಡ. ಅವನು ಆಚಮನ ಮಾಡಿದ ಮೇಲೆ ಶವದ ಮೇಲೆ ಕುಳಿತುಕೊಂಡ. ಅವನು ಸ್ವಲ್ಪ ಜಪವನ್ನು ಮಾಡಿದ್ದೆ ತಡ ತಕ್ಷಣವೆ ಭಗವತಿ ಅವನಿಗೆ ಪ್ರತ್ಯಕ್ಷಳಾದಳು. ದೇವಿ ಭಕ್ತನಿಗೆ “ನಿನ್ನ ಭಕ್ತಿಗೆ ನಾನು ಮೆಚ್ಚಿರುವೆನು. ನಿನಗೆ ಏನಾದರೂ ವರ ಬೇಕಾದರೆ ಕೇಳು” ಎಂದಳು. ಭಗವತಿಯ ಪಾದಪದ್ಮಗಳಿಗೆ ನಮಸ್ಕರಿಸಿ “ತಾಯಿ, ನಿನ್ನನ್ನು ಒಂದು ಪ್ರಶ್ನೆ ಕೇಳಲೆ? ನಿನ್ನ ಲೀಲೆಯನ್ನು ನೋಡಿ ನನಗೆ ಆಶ್ಚರ್ಯವಾಗಿದೆ. ಮತ್ತೊಬ್ಬನು ನಿನ್ನನ್ನು ಪಡೆಯುವುದಕ್ಕೆ ಹಲವು ವಸ್ತುಗಳನ್ನು ಸಂಗ್ರಹ ಮಾಡಿಕೊಂಡನು. ಇದ ಕ್ಕಾಗಿ ಅವನು ಅಷ್ಟೊಂದು ಶ್ರಮಪಟ್ಟನು. ಆದರೆ ನೀನು ಅನು ಗ್ರಹ ಮಾಡಲಿಲ್ಲ. ನನಗೆ ಪೂಜೆಯ ವಿಷಯವೇ ತಿಳಿಯದು. ನಾನು ಯಾವ ಸಾಧನೆಯನ್ನೂ ಮಾಡಿಲ್ಲ. ನನಗೆ ಭಕ್ತಿ, ಜ್ಞಾನ, ಪ್ರೇಮಗಳೇ ಗೊತ್ತಿಲ್ಲ. ಆದರೂ ನೀನು ನನಗೆ ಒಲಿದೆಯಲ್ಲ. ಇದು ಹೇಗೆ?” ಎಂದು ಕೇಳಿದನು. ಭಗ ವತಿ ಮಂದಹಾಸದಿಂದ “ನನ್ನ ಮಗುವೆ, ನೀನು ಹಿಂದಿನ ಜನ್ಮದಲ್ಲಿ ಏನೇನು ಮಾಡಿದ್ದೀಯೊ ಅದು ನಿನಗೆ ಗೊತ್ತಿಲ್ಲ. ಹಿಂದೆ ನನ್ನನ್ನು ಒಲಿಸಿಕೊಳ್ಳುವುದಕ್ಕೆ ಹಲವು ಜನ್ಮಗಳು ಪ್ರಯತ್ನಪಟ್ಟೆ. ಆ ತಪಸ್ಸಿನ ಫಲವಾಗಿ ಇವುಗಳೆಲ್ಲ ನಿನಗೆ ಬಂತು. ಆದ್ದರಿಂದಲೇ ನಾನು ಪ್ರತ್ಯಕ್ಷವಾದೆನು. ಈಗ ನಿನಗೇನು ಬೇಕೊ ಆ ವರವನ್ನು ಕೇಳು” ಎಂದಳು. ಒಬ್ಬ ಹಿಂದಿನ ಜನ್ಮದಲ್ಲಿ ಮಾಡಿದ ಕರ್ಮಗಳ ಸ್ವಭಾವವನ್ನು ಒಪ್ಪಬೇಕಾಗಿದೆ.


\section{\num{೫೮.} ಕೆಲವು ಅನಿವಾರ್ಯಗಳು}

ಪ್ರತಿಯೊಬ್ಬನೂ ತಾನು ಮಾಡಿದ ಹಿಂದಿನ ಜನ್ಮದ ಕರ್ಮಗಳ ಪರಿಣಾಮ ವನ್ನು ಅನುಭವಿಸಲೇಬೇಕು. ಪೂರ್ವಜನ್ಮ ಸಂಸ್ಕಾರಗಳ ಮತ್ತು ಪ್ರಾರಬ್ಧ ಕರ್ಮಗಳ ಪ್ರಭಾವವನ್ನು ಎಲ್ಲರೂ ಒಪ್ಪಲೇಬೇಕು. ಸುಖದುಃಖಗಳು ಮಾನವಜನ್ಮದ ಅವಿಭಾಜ್ಯ ಅಂಗಗಳೆಂಬುದನ್ನು ಯಾರೂ ಮರೆಯಬಾರದು. ಕವಿ ಕಂಕಣ ಬರೆದ ‘ಚಂಡಿ’ಯಲ್ಲಿ ಹೀಗೆ ಹೇಳಿದೆ: ಕಲುವೀರನನ್ನು ಜೈಲಿಗೆ ಹಾಕಿ ಅವನ ಎದೆಯ ಮೇಲೆ ದೊಡ್ಡ ಭಾರವನ್ನು ಹೊರಿಸಿದರು. ಆದರೆ ಭಗವತಿಯ ಕೃಪೆಯಿಂದ ಅವನು ಜನ್ಮವೆತ್ತಿದ್ದ. ಯಾವಾಗ ಒಬ್ಬ ದೇಹವನ್ನು ಧರಿಸುವನೋ ಆಗ ಸುಖದುಃಖಗಳು ತಪ್ಪಿದ್ದಲ್ಲ.

ಶ್ರೀಮಂತನ ಕಥೆಯನ್ನು ಓದು. ಅವನು ದೊಡ್ಡ ಭಕ್ತನಾಗಿದ್ದ. ಅವನ ತಾಯಿ ಖುಲ್ಲನಾ ಭಗವತಿಯ ದೊಡ್ಡ ಭಕ್ತೆ. ಆದರೂ ಅವನ ಕಷ್ಟಗಳಿಗೆ ಕೊನೆಯಿರಲಿಲ್ಲ. ಅವನ ಶಿರಚ್ಛೇದನವಾಗುವುದರಲ್ಲಿತ್ತು. ಇನ್ನೊಂದೆಡೆ ಭಗ ವತಿಯ ದೊಡ್ಡ ಭಕ್ತನಾದ ಒಬ್ಬ ಸೌದೆ ಒಡೆಯುವವನಿಗೆ ಸಂಬಂಧಿಸಿದ ಕಥೆಯಿದೆ. ಅವನಿಗೆ ಭಗವತಿಯ ದರ್ಶನವಾಗಿತ್ತು. ಅವಳ ವಿಶೇಷ ಕೃಪೆಗೆ ಮತ್ತು ಪ್ರೀತಿಗೆ ಪಾತ್ರನಾಗಿದ್ದ. ಆದರೆ ಜೀವನೋಪಾಯ ಕ್ಕಾಗಿ ಅವನು ಅನಂತರವೂ ಕಠಿಣವಾದ ಸೌದೆ ಒಡೆಯುವ ಕೆಲಸವನ್ನೇ ಮಾಡಬೇಕಾಯಿತು. ಶ್ರೀಕೃಷ್ಣನ ತಾಯಿ ದೇವಕಿ ಸೆರೆಮನೆಯಲ್ಲಿದ್ದಾಗ ಶಂಖಚಕ್ರಗದಾಧಾರಿಯಾದ ಭಗವಂತನನ್ನೇ ಕಂಡಳು. ಇವೆಲ್ಲ ಆದರೂ ಸೆರೆಯಿಂದ ಮುಕ್ತಳಾಗಲಿಲ್ಲ.

\chapter{ಪಾರಾಗಲು ಮಾರ್ಗ}

\section{\num{೫೯. } ಒಂದೇ ದಾರಿ}

ಸಂಸಾರದಲ್ಲಿದ್ದು ಆಧ್ಯಾತ್ಮಿಕ ಜೀವನವನ್ನು ನಡೆಸುವುದು ಏಕೆ ಸಾಧ್ಯವಿಲ್ಲ? ಆದರೆ ಅದು ಬಹಳ ಕಷ್ಟ. ಒಂದು ಸಲ ನಾನು ಬಾಗಬಜಾರ್ ಸೇತುವೆಯ ಮೇಲೆ ಹೋದೆ. ಅದನ್ನು ಎಷ್ಟೊಂದು ಸರಪಳಿಗಳಿಂದ ಬಿಗಿದಿರು ವರು! ಅವುಗಳಲ್ಲಿ ಯಾವುದೋ ಒಂದು ಸರಪಳಿ ಕಳಚಿಕೊಂಡರೆ ಏನೂ ಆಗುವುದಿಲ್ಲ, ಏಕೆಂದರೆ ಇನ್ನೂ ಹಲವು ಸರಪಣಿಗಳು ಅದನ್ನು ಹಿಡಿದಿರುತ್ತವೆ. ಅದರಂತೆಯೇ ಸಾಂಸಾರಿಕನಿಗೂ ಹಲವು ಬಂಧನಗಳಿವೆ. ಭಗವಂತನ ಅನು ಗ್ರಹವಿಲ್ಲದೆ ಅವುಗಳಿಂದ ಪಾರಾಗಲಾರ.

