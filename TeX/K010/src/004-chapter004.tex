
\chapter{ನಾವು ಮಾಡಲೇಬೇಕಾಗಿರುವುದು}

\section{\num{೧೭೧. } ಮುಂದೆ ಮುಂದೆ ಹೋಗಿ}

ಒಂದು ಸಲ ಸೌದೆ ಒಡೆಯುವವನು ಮರವನ್ನು ಕಡಿಯಲು ಒಂದು ಕಾಡಿಗೆ ಹೋದ. ಅವನು ಅಕಸ್ಮಾತ್ತಾಗಿ ಒಬ್ಬ ಬ್ರಹ್ಮಚಾರಿಯನ್ನು ಕಂಡ. ಆ ಸಾಧು ಇವನಿಗೆ ಹೇಳಿದ, “ಮುಂದೆ ಹೋಗು” ಎಂದು. ಸೌದೆ ಒಡೆಯುವವನು ಮನೆಗೆ ಬಂದಮೇಲೆ, “ಸಾಧುಗಳು ನನಗೆ ಮುಂದೆ ಮುಂದೆ ಹೋಗು ಎಂದಿದ್ದಾರೆ. ಇದನ್ನು ಏತಕ್ಕೆ ಹೇಳಿದರು?” ಎಂದು ಯೋಚಿಸಿದ. ಕೆಲವು ಕಾಲವಾಯಿತು. ಒಂದು ದಿನ ಸೌದೆ ತರಲು ಕಾಡಿಗೆ ಹೋಗಿದ್ದಾಗ, “ಇವತ್ತು ಕಾಡಿನಲ್ಲಿ ಇನ್ನೂ ಮುಂದಕ್ಕೆ ಹೋಗುತ್ತೇನೆ” ಎಂದು ಆಲೋಚಿ ಸಿದ. ಅರಣ್ಯದ ಒಳಗೆ ಇನ್ನೂ ಮುಂದೆ ಹೋಗಿ ಬೇಕಾದಷ್ಟು ಗಂಧದ ಮರ ಗಳನ್ನು ಕಂಡ. ಅದನ್ನು ನೋಡಿ ಅವನಿಗೆ ಸಂತೋಷವಾಯಿತು. ಗಾಡಿಗಟ್ಟಳೆ ಗಂಧದ ಕಟ್ಟಿಗೆಯನ್ನು ಅವನು ತೆಗೆದುಕೊಂಡು ಬಂದ. ಅದನ್ನು ಅಂಗಡಿ ಯಲ್ಲಿ ಮಾರಿ ಶ್ರೀಮಂತನಾದ.

ಸ್ವಲ್ಪ ದಿನಗಳಾದ ಮೇಲೆ, “ಸಾಧುಗಳು ನನಗೆ ಇನ್ನೂ ಮುಂದೆ ಹೋಗು” ಎಂದಿರುವರು–ಎಂದು ಆಲೋಚಿಸಿ ಅವನು ಅರಣ್ಯದೊಳಗೆ ಮತ್ತೂ ಮುಂದೆ ಹೋದ. ಅಲ್ಲಿ ಒಂದು ನದೀ ತೀರದಲ್ಲಿ ಬೆಳ್ಳಿಯ ಗಣಿಯನ್ನು ಕಂಡ. ಇದನ್ನು ಅವನು ಕನಸಿನಲ್ಲಿಯೂ ಭಾವಿಸಿರಲಿಲ್ಲ. ಅವನು ಗಣಿಯಿಂದ ಬೆಳ್ಳಿ ಯನ್ನು ಅಗೆದು ತೆಗೆದುಕೊಂಡು ಹೋಗಿ ಸಂತೆಯಲ್ಲಿ ಮಾರಿದ. ಆಗ ಅವನಿಗೆ ಎಷ್ಟೊಂದು ಹಣ ಸಿಕ್ಕಿತು ಎಂದರೆ ಅದನ್ನು ಎಣಿಸುವುದಕ್ಕೆ ಅವನಿಗೆ ಆಗಲಿಲ್ಲ.

ಇನ್ನೂ ಕೆಲವು ದಿನಗಳು ಆದವು. ಅವನು ಆಲೋಚಿಸಿದ: “ಸಾಧುಗಳು ಹತ್ತಿರದಲ್ಲಿಯೇ ನಿಲ್ಲು ಎನ್ನಲಿಲ್ಲ. ಮುಂದೆ ಹೋಗು ಎಂದಿದ್ದರು.” ಈ ಸಲ ಅವನು ನದಿಯ ಆಚೆಯ ಕಡೆ ಹೋದ. ಅಲ್ಲಿ ಚಿನ್ನದ ಗಣಿ ಸಿಕ್ಕಿತು. ನೋಡಿ ಆಶ್ಚರ್ಯವಾಯಿತು. “ಆದಕಾರಣವೆ ಸಾಧು ನನಗೆ ಮುಂದೆ ಮುಂದೆ ಹೋಗು ಎಂದಿದ್ದು” ಎಂದು ಉದ್ಗರಿಸಿದ.

ಇನ್ನೂ ಕೆಲವು ದಿನಗಳಾದ ಮೇಲೆ ಅವನು ಮತ್ತೂ ಮುಂದಕ್ಕೆ ಹೊರಟ. ಅಲ್ಲಿ ವಜ್ರ, ಪುಷ್ಪರಾಗ ಮುಂತಾದ ಬೆಲೆಬಾಳುವ ವಸ್ತುಗಳುಸಿಕ್ಕಿದವು. ಅವನು ಇವುಗಳನ್ನು ತೆಗೆದುಕೊಂಡು ಐಶ್ವರ್ಯದೇವತೆಗೆ ಸಮನಾಗಿ ಐಶ್ವರ್ಯವಂತನಾದ.

ನೀವು ಏನು ಮಾಡಿದರೂ ಮುಂದೆ ಹೋದಂತೆಲ್ಲ ಉತ್ತಮವಾದ ವಸ್ತುಗಳು ದೊರೆಯುವುವು. ನೀವು ಜಪವನ್ನು ಮಾಡಿದರೆ ಸ್ವಲ್ಪ ಆನಂದ ಸಿಕ್ಕಬಹುದು. ಇಷ್ಟರಿಂದ ಆಧ್ಯಾತ್ಮಿಕ ಜೀವನದಲ್ಲಿ ಏನೇನು ಬೇಕೋ ಅವೆಲ್ಲ ದೊರೆತಿದೆ ಎಂದು ಭಾವಿಸಬೇಡಿ. ಬರೀ ಕರ್ಮವೇ ನಮ್ಮ ಜೀವನದ ಗುರಿ ಯಲ್ಲ. ನೀವು ಮುನ್ನಡೆಯಿರಿ. ಅನಂತರ ನಿಃಸ್ವಾರ್ಥವಾದ ಕೆಲಸವನ್ನು ಮಾಡು ವುದಕ್ಕೆ ಸಾಧ್ಯ.


\section{\num{೧೭೨. } ಎಲೆಗಳನ್ನು ಎಣಿಸಬೇಡಿ, ಹಣ್ಣನ್ನು ತಿನ್ನಿ}

ಇಬ್ಬರು ಸ್ನೇಹಿತರು ಒಂದು ಮಾವಿನ ತೋಟಕ್ಕೆ ಹೋದರು. ಒಬ್ಬನಿಗೆ ಬೇಕಾದಷ್ಟು ವ್ಯವಹಾರಜ್ಞಾನವಿತ್ತು. ತೋಟಕ್ಕೆ ಹೋದ ತಕ್ಷಣವೆ ಅಲ್ಲಿ ಎಷ್ಟು ಮರಗಳು ಇವೆ, ಎಷ್ಟು ಎಲೆಗಳು ಇವೆ, ಎಷ್ಟು ಮಾವಿನ ಹಣ್ಣುಗಳು ಇವೆ, ಇವುಗಳ ಒಟ್ಟು ಆದಾಯ ಎಷ್ಟು ಇರಬಹುದು ಎಂಬುದನ್ನು ಲೆಕ್ಕಾಚಾರ ಮಾಡುವುದಕ್ಕೆ ಹೊರಟ. ಮತ್ತೊಬ್ಬ ಮನುಷ್ಯ ತೋಟದ ಮಾಲಿಕನ ಹತ್ತಿರ ಹೋಗಿ ಅವನ ಸ್ನೇಹ ಬೆಳೆಸಿ ಕೊಂಡು, ಅವನ ಅನುಮತಿಯನ್ನು ಪಡೆದು ಅಲ್ಲಿರುವ ಕೆಲವು ಹಣ್ಣುಗಳನ್ನು ತಿನ್ನಲು ಮೊದಲು ಮಾಡಿದ. ಇವರಿಬ್ಬರಲ್ಲಿ ಯಾರು ಬುದ್ಧಿ ವಂತರು? ಮಾವಿನ ಹಣ್ಣನ್ನು ತಿಂದರೆ ಹಸಿವಿನಿಂದ ಪಾರಾಗುತ್ತೀಯೆ. ಬರೀ ಎಲೆ ಮರ ಇವುಗಳ ಲೆಕ್ಕಾಚಾರದಿಂದ ಏನು ಪ್ರಯೋಜನ? ಒಣ ಪಂಡಿತನು ಸೃಷ್ಟಿ ಹೇಗಾಯಿತು, ಯಾವುದರಿಂದ ಆಯಿತು ಎಂಬುದನ್ನು ತಿಳಿಯಲು ಆಶಿಸುವನು. ಆದರೆ ದೀನ ಭಕ್ತನಾದರೋ ಸೃಷ್ಟಿಸಿದವನೊಂದಿಗೆ ಸಂಬಂಧ ವನ್ನು ಬೆಳೆಸಿ ಅವನ ಸಾಕ್ಷಾತ್ಕಾರವನ್ನು ಅನುಭವಿಸುವನು.


\section{\num{೧೭೩. } ಸಚ್ಚಿದಾನಂದ ಸಾಗರದಲ್ಲಿ ಮುಳುಗು}

ಒಂದು ಸಲ ನರೇಂದ್ರನಿಗೆ (ಸ್ವಾಮಿ ವಿವೇಕಾನಂದ) ನಾನು ಹೀಗೆ ಹೇಳಿದೆ: “ನೋಡು ಮಗು, ದೇವರು ಅಮೃತಸಾಗರ. ನೀನು ಆ ಸಾಗರದಲ್ಲಿ ಮುಳುಗಲು ಇಚ್ಛಿಸುವುದಿಲ್ಲವೆ? ಒಂದು ಪಾನಕದ ಬಟ್ಟಲು ಇದೆ, ನೀನು ಒಂದು ನೊಣ ಎಂದು ತಿಳಿ ದುಕೊ. ನೀನು ಪಾನಕವನ್ನು ಎಲ್ಲಿ ಕುಳಿತುಕೊಂಡು ಕುಡಿಯುವೆ?” ಅದಕ್ಕೆ ನರೇಂದ್ರ, “ನಾನು ಬಟ್ಟಲಿನ ಅಂಚಿನಲ್ಲಿ ಕುಳಿತುಕೊಂಡು ಅದನ್ನು ಕುಡಿಯುತ್ತೇನೆ,” ಎಂದನು. “ಯಾತಕ್ಕೆ ನೀನು ಅಂಚಿನಲ್ಲಿ ಕುಳಿತುಕೊಂಡು ಕುಡಿಯಬೇಕು?” ಎಂದು ಕೇಳಿದೆ. “ಪಾನಕದ ಒಳಗೆ ಇಳಿದರೆ ನಾನು ಅದರಲ್ಲಿ ಮುಳುಗಿ ಸತ್ತು ಹೋಗಬಹುದು” ಎಂದ ನರೇಂದ್ರ. ಅದಕ್ಕೆ ನಾನು ಹೇಳಿದೆ, “ಅದು ಅಮೃತದ ಮಡು. ನೀನು ಅದರಲ್ಲಿ ಮುಳುಗಿದರೆ ಸಾಯುವುದಿಲ್ಲ. ನೀನೂ ಅಮೃತನಾಗುವೆ. ಒಬ್ಬನು ದೇವರನ್ನು ಕುರಿತು ಚಿಂತಿಸುತ್ತಿದ್ದರೆ ಅವನ ಪ್ರಜ್ಞೆ ಹೋಗುವುದಿಲ್ಲ. ಅವನೂ ಕೂಡ ದೇವರಂತೆಯೇ ಆಗುತ್ತಾನೆ” ಎಂದೆ.


\section{\num{೧೭೪. } ನಿನ್ನ ಧರ್ಮವನ್ನು ನೀನು ಬಿಡಬೇಡ}

ಒಂದಾನೊಂದು ಕಾಲದಲ್ಲಿ ಒಬ್ಬ ಒಂದು ಬಾವಿಯನ್ನು ತೋಡಲು ಯತ್ನಿಸಿದ. ಯಾರೋ ಒಂದು ಸ್ಥಳದಲ್ಲಿ ಅಗೆ ಎಂದರು. ಅವರು ಹೇಳಿದ ಕಡೆ ಅಗೆದ. ಅವನೊಂದು ಹದಿನೈದು ಅಡಿ ಅಗೆದ ಮೇಲೆ ಅಲ್ಲಿ ನೀರು ಸಿಕ್ಕುವ ಚಿಹ್ನೆ ಕಾಣಲಿಲ್ಲ. ಅವನಿಗೆ ಬೇಜಾ ರಾಯಿತು. ಇನ್ನೊಬ್ಬ ಬಂದ ಅಷ್ಟು ಹೊತ್ತಿಗೆ. ಅವನು ಇವನ ಹುಚ್ಚುತನವನ್ನು ನೋಡಿ ಬೇರೆ ಕಡೆ ಅಗೆ ಎಂದ. ಅಲ್ಲಿ ಮೂವತ್ತು ಅಡಿಯವರೆಗೆ ಅಗೆದ. ಅಲ್ಲಿಯೂ ನೀರು ಬರಲಿಲ್ಲ ಎಂಬ ನಿರಾಶೆಯಿಂದ ಅಗೆಯುವುದನ್ನು ಬಿಡುವು ದರಲ್ಲಿದ್ದ. ನಾಲ್ಕನೆಯವನು ಬಂದು ನಗುತ್ತ “ಮಗು, ನೀನು ನೀರಿಗಾಗಿ ಬಹಳ ಆಯಾಸ ಪಟ್ಟಿರುವೆ. ನಿನಗೆ ತಪ್ಪು ದಾರಿಯನ್ನು ಹೇಳಿದ್ದರಿಂದ ನೀರು ಬರಲಿಲ್ಲ. ಆಗಲಿ ನನ್ನನ್ನು ಅನುಸರಿಸು. ನಾನು ಒಂದು ಕಡೆಗೆ ಕರೆದುಕೊಂಡು ಹೋಗುತ್ತೇನೆ. ಅಲ್ಲಿ ನೀನು ಪಿಕಾಸಿಯಿಂದ ಅಗೆಯುವುದಕ್ಕೆ ಪ್ರಾರಂಭ ಮಾಡಿದರೆ ಸಾಕು ಬೇಕಾದಷ್ಟು ನೀರು ಬರುವುದು” ಎಂದ. ಅವನಿಗೆ ತುಂಬಾ ಆಸೆಯಾಗಿ ಈಗ ಬರುತ್ತದೆ ಆಗ ಬರುತ್ತದೆ ನೀರು ಎಂದು ಅಗೆಯುತ್ತಿದ್ದ. ಕೊನೆಗೆ ಅವನು ಇಪ್ಪತ್ತೈದು ಅಡಿಗಳಷ್ಟು ಆಳದವರೆಗೂ ಹೋದ. ಆದರೂ ನೀರು ಬರಲಿಲ್ಲ. ಅನಂತರ ನಿರಾಸೆಯಿಂದ ಈ ಬಾವಿ ತೋಡುವ ಪ್ರಯತ್ನ ವನ್ನೇ ಕೈಬಿಟ್ಟ. ಇಷ್ಟು ಹೊತ್ತಿಗೆ ಅವನು ಬೇರೆ ಬೇರೆ ಕಡೆ ಎಪ್ಪತ್ತೈದು ಅಡಿಗಳಷ್ಟು ಅಗೆದಿದ್ದನು. ಆದರೆ ಅವನು ಯಾವುದಾದರೂ ಒಂದು ಸ್ಥಳದಲ್ಲಿ ಅಷ್ಟು ಅಗೆದಿದ್ದರೆ ಅದರಲ್ಲಿ ಅರ್ಧದಷ್ಟನ್ನು ಅಗೆಯುವುದರೊಳಗೆ ಅವನಿಗೆ ನಿಜವಾಗಿ ನೀರು ಸಿಕ್ಕುತ್ತಿತ್ತು.

ಒಂದು ಧರ್ಮವನ್ನು ನೆಚ್ಚದೆ ಹಲವು ಧರ್ಮಗಳನ್ನು ಅನುಸರಿಸುವವರು ಎಲ್ಲಿಯೂ ಸತ್ಯ ಸಿಕ್ಕದೆ ತಮ್ಮ ಇಳಿವಯಸ್ಸಿನಲ್ಲಿ ನಾಸ್ತಿಕರಾಗುತ್ತಾರೆ.


\section{\num{೧೭೫. } ನಿಮ್ಮ ಎರಡು ಕೈಗಳನ್ನು ಸಡಿಲಬಿಡಿ}

ಒಂದು ಸಲ ಒಬ್ಬ ಹೆಂಗಸು ನೇಯ್ಗೆ ಕೆಲಸ ಮಾಡುತ್ತಿದ್ದ ತನ್ನ ಸ್ನೇಹಿತೆ ಯನ್ನು ನೋಡಲು ಹೋದಳು. ಆ ಸಮಯದಲ್ಲಿ ಅವಳು ಹಲವು ಬಣ್ಣದ ರೇಷ್ಮೆ ದಾರವನ್ನು ಹೆಣೆಯುತ್ತಿದ್ದಳು. ಇವಳನ್ನು ನೋಡಿ ಆಕೆ ತುಂಬಾ ಸಂತೋಷ ಪಟ್ಟಳು. “ತಂಗಿ ನಿನ್ನನ್ನು ನೋಡಿ ನನಗೆ ಎಷ್ಟು ಆನಂದವಾಗಿದೆಯೊ ಅದನ್ನು ವಿವರಿಸಲು ಅಸದಳ. ನಾನು ಸ್ವಲ್ಪ ನಿನಗೆ ತಿಂಡಿ ಮಾಡಿಕೊಂಡು ಬರುವೆ” ಎಂದು ಹೇಳಿ ಒಳಗೆ ಹೋದಳು. ಬಂದ ಹೆಂಗಸು ಹಲವು ಬಣ್ಣದ ದಾರಗಳನ್ನು ಸಂತೋಷಪಟ್ಟಳು. “ಸಖಿ! ನಿನ್ನನ್ನು ನೋಡಿ ನನಗೆ ಎಷ್ಟು ಆನಂದವಾಗಿದೆಯೊ ಅದನ್ನು ವಿವರಿಸಲು ಅಸದಳ. ನಾನು ಸ್ವಲ್ಪ ನಿನಗೆ ತಿಂಡಿ ಮಾಡಿಕೊಂಡು ಬರುವೆ” ಎಂದು ಹೇಳಿ ಒಳಗೆ ಹೋದಳು. ಬಂದ ಹೆಂಗಸು ಹಲವು ಬಣ್ಣದ ದಾರಗಳನ್ನು ನೋಡಿ ಅವುಗಳಲ್ಲಿ ಕೆಲವನ್ನು ಕದಿಯಬೇಕೆಂದು ಆಶಿಸಿದಳು. ಅವಳು ತನ್ನ ಕಂಕುಳಲ್ಲಿ ಒಂದು ದಾರದ ಉಂಡೆಯನ್ನು ಬಚ್ಚಿಟ್ಟುಕೊಂಡಳು. ನೇಯಿಗೆಯವಳು ತಿಂಡಿ ಯನ್ನು ತಂದುಕೊಟ್ಟಳು, ತುಂಬಾ ಆದರದಿಂದ ಉಪಚರಿಸಿದಳು. ಯಾವಾಗ ದಾರದ ಕಡೆ ನೋಡಿದಳೊ ಆಗ ಗೊತ್ತಾಯಿತು ತನ್ನ ಸ್ನೇಹಿತೆಯು ಒಂದು ದಾರದ ಉಂಡೆಯನ್ನು ಬಚ್ಚಿಟ್ಟಿರುವಳು ಎಂದು. ಅದನ್ನು ಹೇಗಾದರೂ ಹೊರಕ್ಕೆ ಬರುವಂತೆ ಮಾಡಲು ಒಂದು ಉಪಾಯವನ್ನು ಯೋಚಿಸಿದಳು. “ಗೆಳತಿ, ನಿನ್ನನ್ನು ನೋಡಿ ತುಂಬಾ ದಿನವಾಯಿತು. ಇವತ್ತು ನಾವಿಬ್ಬರೂ ಕುಣಿ ಯೋಣ” ಎಂದಳು. ಅವಳ ಸ್ನೇಹಿತಳು “ನನಗೂ ಕೂಡ ಸಂತೋಷವೆ” ಎಂದಳು. ಇಬ್ಬರೂ ಕುಣಿಯುವುದಕ್ಕೆ ಶುರುಮಾಡಿದರು. ನೇಯ್ಗೆಯವಳು ತನ್ನ ಸ್ನೇಹಿತೆ ಕುಣಿಯುವಾಗ ಒಂದು ಕೈಯನ್ನು ಮೇಲಕ್ಕೆ ಎತ್ತುತ್ತಿಲ್ಲದುದನ್ನು ಗಮನಿಸಿದಳು. ಆಗ ನೇಯ್ಗೆಯವಳು, “ನಾವಿಬ್ಬರೂ ಎರಡೂ ಕೈಗಳನ್ನು ಎತ್ತಿಕೊಂಡು ಆನಂದದಿಂದ ಕುಣಿಯೋಣ” ಎಂದಳು. ಆದರೆ ಸ್ನೇಹಿತೆ ಒಂದು ಕೈಯನ್ನು ಕೆಳಗೆ ಒತ್ತಿಹಿಡಿದುಕೊಂಡು ಮತ್ತೊಂದನ್ನು ಮೇಲೆತ್ತಿಕೊಂಡು ನರ್ತಿಸುತ್ತಿದ್ದಳು. “ತಂಗಿ, ನೀನು ಒಂದೇ ಕೈಯನ್ನು ಎತ್ತಿ ಕುಣಿಯುತ್ತಿರು ವೆಯಲ್ಲ? ಎರಡೂ ಕೈಗಳನ್ನು ಎತ್ತು. ನನ್ನನ್ನು ನೋಡು, ನಾನು ಹೇಗೆ ಎರಡು ಕೈಗಳನ್ನು ಮೇಲಕ್ಕೆತ್ತಿ ಕುಣಿಯುತ್ತಿರುವೆ,” ಎಂದಳು ನೇಯ್ಗೆಯವಳು. ಆದರೆ ಅವಳ ಗೆಳತಿ ಒಂದು ಕೈಯನ್ನು ಮಡಿಸಿಕೊಂಡೇ ಕುಣಿಯುತ್ತಿದ್ದಳು. ಅವಳು ಒಂದನ್ನೇ ಎತ್ತಿ ಕುಣಿಯುತ್ತ “ನನಗೆ ಕುಣಿಯುವುದರ ಬಗ್ಗೆ ಇಷ್ಟೇ ಗೊತ್ತಿರು ವುದು” ಎಂದಳು. ಇನ್ನೊಂದು ಕೈಯನ್ನು ಸಡಿಲಬಿಡು. ಯಾವುದಕ್ಕೂ ಅಂಜ ಬೇಡ. ನಿತ್ಯ, ಲೀಲೆ ಎರಡನ್ನೂ ಒಪ್ಪಿಕೊಳ್ಳಿ.


\section{\num{೧೭೬. } ಮತಾಂಧರಾಗಬೇಡಿ}

ಘಂಟಾಕರ್ಣನಂತೆ ಧರ್ಮಾಂಧನಾಗಿರಬೇಡಿ. ಒಬ್ಬನಿದ್ದ. ಅವನು ಶಿವ ನನ್ನು ಮಾತ್ರ ಪೂಜಿಸುತ್ತಿದ್ದ. ಇತರ ದೇವದೇವತೆಗಳನ್ನು ದ್ವೇಷಿಸುತ್ತಿದ್ದ. ಒಂದು ದಿನ ಶಿವ ಅವನೆದುರಿಗೆ ಪ್ರತ್ಯಕ್ಷನಾಗಿ, “ನೀನು ಎಲ್ಲಿಯವರೆಗೆ ಅನ್ಯ ದೇವದೇವತೆಗಳನ್ನು ದ್ವೇಷಿಸುವೆಯೊ ಅಲ್ಲಿಯವರೆಗೆ ನಾನು ನಿನ್ನನ್ನು ಪ್ರೀತಿಸಲಾರೆ” ಎಂದನು. ಎಷ್ಟೇ ಬೋಧನೆ ಮಾಡಿದರೂ ಘಂಟಾಕರ್ಣ ಮಾತ್ರ ಈ ಬೋಧನೆಗೆ ಕಿವಿಗೊಡ ಲಿಲ್ಲ. ಸ್ವಲ್ಪಕಾಲದ ಮೇಲೆ ಶಿವನು ಪುನಃ ಪ್ರತ್ಯಕ್ಷನಾದ. ಈ ಸಲ ಅವನು ಹರಿ-ಹರನಂತೆ ಕಾಣಿಸಿಕೊಂಡನು. ಒಂದು ಭಾಗ ಹರಿ ಮತ್ತೊಂದು ಭಾಗ ಹರ. ಇದರಿಂದ ಭಕ್ತನಿಗೆ ಅರ್ಧ ಮಾತ್ರ ತೃಪ್ತಿಯಾಯಿತು, ಅರ್ಧ ಅತೃಪ್ತಿ ಯಾಯಿತು. ಶಿವನಿದ್ದ ಕಡೆಗೆ ಪೂಜೆ ಮಾಡಿದ. ವಿಷ್ಣುವಿದ್ದ ಕಡೆ ಪೂಜೆ ಮಾಡ ಲಿಲ್ಲ. ಶಿವನ ಮುಂದೆ ಊದಿನ ಕಡ್ಡಿಯನ್ನು ಇಟ್ಟ. ವಿಷ್ಣುವಿನ ಕಡೆಗೆ ಅದನ್ನು ತೋರಿಸಲೇ ಇಲ್ಲ. ಊದಿನ ಕಡ್ಡಿಯ ಹೊಗೆ ವಿಷ್ಣುವಿನ ಮೂಗಿಗೆ ಹೋಗದಿರ ಲೆಂದು ಅದನ್ನು ಮುಚ್ಚುವುದಕ್ಕೂ ಅವನು ಹಿಂಜರಿಯಲಿಲ್ಲ. ಶಿವಹೇಳಿದ, “ನಿನ್ನ ಮತಾಂಧತೆಗೆ ಮದ್ದಿಲ್ಲ. ನಾನು ಇಬ್ಬರಾಗಿ ಕಾಣಿಸಿಕೊಳ್ಳುವುದಕ್ಕೆ ಕಾರಣ ಎಲ್ಲ ದೇವರೂ ಒಂದೇ ಎಂಬುದನ್ನು ತೋರಿಸುವುದಕ್ಕಾಗಿ. ನೀನು ಆ ಬೋಧನೆಯನ್ನು ಸರಿಯಾಗಿ ತಿಳಿದುಕೊಳ್ಳಲಿಲ್ಲ. ಬಹಳ ಕಾಲ ನೀನು ಇದಕ್ಕಾಗಿ ವ್ಯಥೆ ಪಡಬೇಕಾಗಿದೆ” ಎಂದ. ಅವನು ಮನೆಗೆ ಹೋದ. ಅವನು ಇನ್ನೂ ಹೆಚ್ಚು ವಿಷ್ಣುವನ್ನು ದ್ವೇಷಿಸತೊಡಗಿದ. ಇವನ ವಿಚಿತ್ರ ಸ್ವಭಾವವನ್ನು ಬಲ್ಲ ಮಕ್ಕಳು ಆ ಭಕ್ತನಿಗೆ ಕೇಳುವಂತೆ ವಿಷ್ಣುವಿನ ನಾಮವನ್ನು ಉಚ್ಚರಿಸತೊಡಗಿದರು. ಇದನ್ನು ನೋಡಿ ಘಂಟಾಕರ್ಣ ತನ್ನ ಎರಡೂ ಕಿವಿಗಳಿಗೆ ಎರಡು ಗಂಟೆಯನ್ನು ಕಟ್ಟಿಕೊಂಡ. ವಿಷ್ಣುವಿನ ಹೆಸರು ಕೇಳಿಸದಿರಲಿ ಎಂದು ಅವನು ಗಂಟೆಯನ್ನು ಬಾರಿಸುತ್ತಿದ್ದ. ಇದರಿಂದ ಅವನಿಗೆ ಘಂಟಾಕರ್ಣ ಎಂಬ ಹೆಸರುಬಂತು.


\section{\num{೧೭೭. } ಒಂದನ್ನೇ ನೋಡು ಇಲ್ಲದೆ ಇದ್ದರೆ ಏನನ್ನೂ ನೋಡಬೇಡ}

ಒಬ್ಬ ರಾಜನಿಗೆ ಅವನ ಗುರುವು ಪರಮ ಪವಿತ್ರವಾದ ಅದ್ವೈತವನ್ನು, ಎಂದರೆ ಪ್ರಪಂಚವೆಲ್ಲ ಬ್ರಹ್ಮನಿಂದ ತುಂಬಿದೆ ಎಂಬುದನ್ನು ತೋರಿಸಿದರು. ರಾಜನಿಗೆ ಈ ಸಿದ್ಧಾಂತ ತುಂಬಾ ಹಿಡಿಸಿತು. ಅಂತಃಪುರಕ್ಕೆ ಹೋದಾಗ ತನ್ನ ರಾಣಿಗೆ, “ರಾಣಿಗೂ, ರಾಣಿಯ ಪರಿಚಾರಿಕೆಗೂ ಏನೂ ವ್ಯತ್ಯಾಸವಿಲ್ಲ. ಇಂದಿ ನಿಂದ ಪರಿಚಾರಿಕೆಯೇ ನನ್ನ ರಾಣಿ” ಎಂದನು. ರಾಣಿಗೆ ಈ ಮಾತನ್ನು ಕೇಳಿದಾಗ ಸಿಡಿಲು ಬಡಿದಂತೆ ಆಯಿತು. ಅವಳು ತನ್ನ ಗುರುವನ್ನು ಕರೆದು ಅಳುತ್ತ ಹೇಳಿದಳು: “ನೋಡಿ, ನಿಮ್ಮ ಅದ್ವೈತ ಬೋಧನೆ ಎಂತಹ ದುರಂತಕ್ಕೆ ಕಾರಣವಾಗಿದೆ” ಎಂದು ಹೇಳಿ ನಡೆದ ಘಟನೆಯನ್ನು ಹೇಳಿ ದಳು. ಗುರುಗಳು ರಾಣಿಗೆ ಸಮಾಧಾನ ಹೇಳಿ, “ಊಟದ ಸಮಯದಲ್ಲಿ ಬಡಿಸುವಾಗ ಪಕ್ಕದಲ್ಲಿ ಕುಳಿತುಕೊಳ್ಳುವ ರಾಜನ ತಟ್ಟೆಗೆ ಒಂದು ಸಗಣಿ ಮುದ್ದೆಯನ್ನು ಹಾಕು” ಎಂದರು. ಊಟದ ಸಮಯ ದಲ್ಲಿ ಗುರು ಶಿಷ್ಯರಿಬ್ಬರೂ ಊಟಕ್ಕೆ ಒಟ್ಟಿಗೆ ಕುಳಿತು ಕೊಂಡರು. ರಾಜನ ತಟ್ಟೆಗೆ ಒಂದು ಸಗಣಿ ಮುದ್ದೆ ಇಟ್ಟಾಗ ಅವನು ಕೋಪದಿಂದ ಗುಡಿಗುಡಿಸಿದನು. ಆದರೆ ಪಕ್ಕದಲ್ಲೇ ಇದ್ದ ಗುರುಗಳಿಗೆ ಮೃಷ್ಟಾನ್ನ ಭೋಜನ. ಗುರುಗಳು ರಾಜನ ಕೋಪವನ್ನು ನೋಡಿ, “ರಾಜರೆ, ನೀವು ಅದ್ವೈತದಲ್ಲಿ ಪಾರಂಗತರು. ಸಗಣಿಗೂ ಮಿಕ್ಕಿರುವುದಕ್ಕೂ ಏತಕ್ಕೆ ಭೇದವನ್ನು ಮಾಡುವಿರಿ” ಎಂದರು.

ರಾಜನಿಗೆ ತುಂಬಾ ಕೋಪ ಬಂತು. “ತುಂಬಾ ದೊಡ್ಡ ಅದ್ವೈತಿಗಳು ಎಂದು ಹೆಮ್ಮೆ ಕೊಚ್ಚಿಕೊಳ್ಳುವ ನೀವು ಈ ಸಗಣಿಯನ್ನು ತಿನ್ನಿ” ಎಂದು ಗುರುಗಳಿಗೆ ಹೇಳಿದನು. ಗುರುಗಳು ಆಗಲಿ ಎಂದು ತಕ್ಷಣವೆ ಒಂದು ಹಂದಿಯಾಗಿ ಸಗಣಿಯ ಮುದ್ದೆಯನ್ನೆಲ್ಲ ತಿಂದರು. ಅನಂತರ ಹಿಂದಿನಂತೆಯೇ ಗುರುಗಳು ಮಾನವ ಆಕಾರವನ್ನು ಧರಿಸಿದರು. ರಾಜನಿಗೆ ಇದನ್ನು ಕಂಡು ತುಂಬಾ ನಾಚಿಕೆ ಯಾಗಿ, ಮತ್ತೊಮ್ಮೆ ರಾಣಿಯ ಹತ್ತಿರ ಈ ಪ್ರಸಂಗವನ್ನೇ ಎತ್ತಲಿಲ್ಲ. ವಿವಿಧ ಆಟಿಕೆಗಳು ಒಂದೇ ಮಣ್ಣಿನಿಂದ ಮಾಡಿದುದಾದರೂ ಆಕಾರದಲ್ಲಿ ಭೇದವಿರು ತ್ತದೆ.


\section{\num{೧೭೮. } ಜ್ಞಾನ ಅಜ್ಞಾನಗಳಿಗೆ ಅತೀತನಾಗು}

ನೀನು ಜ್ಞಾನ ಮತ್ತು ಅಜ್ಞಾನಕ್ಕೆ ಅತೀತನಾಗಬೇಕು. ಆಗಲೇ ನೀನು ದೇವರನ್ನು ನೋಡಬಲ್ಲೆ. ಹಲವನ್ನು ನೋಡುವುದು ಅಜ್ಞಾನ. ನನ್ನ ಸಮಾನ ವಿದ್ಯಾವಂತನಿಲ್ಲ ಎಂದು ಭಾವಿಸುವುದು ಕೂಡ ಒಂದು ಅಜ್ಞಾನವೇ. ಎಲ್ಲಾ ಕಡೆಯಲ್ಲಿಯೂ ದೇವರೊಬ್ಬನೇ ಇರುವನು –ಎಂದರಿಯುವುದು ಜ್ಞಾನ. ಅವನನ್ನು ನಿಕಟವಾಗಿ ಅನುಭವಿಸುವುದು ವಿಜ್ಞಾನ. ಇದು ಜ್ಞಾನಕ್ಕಿಂತ ಮೇಲಿನದು. ಒಂದು ಮುಳ್ಳು ನಿನ್ನ ಕಾಲಿನ ಒಳಗೆ ಹೋದರೆ, ಅದನ್ನು ಹೊರಗೆ ತೆಗೆಯಲು ಮತ್ತೊಂದು ಮುಳ್ಳು ಬೇಕಾಗುತ್ತದೆ. ಮುಳ್ಳನ್ನು ತೆಗೆದ ಮೇಲೆ ಎರಡನ್ನೂ ಆಚೆಗೆ ಎಸೆಯಬೇಕು. ಜ್ಞಾನದ ಮುಳ್ಳಿ ನಿಂದ ಅಜ್ಞಾನದ ಮುಳ್ಳನ್ನು ತೆಗೆಯಬೇಕು. ಅನಂತರ ಜ್ಞಾನ ಮತ್ತು ಅಜ್ಞಾನ ಎರಡನ್ನೂ ಬಿಡಬೇಕು. ದೇವರು ಜ್ಞಾನ-ಅಜ್ಞಾನಗಳಿಗೆ ಅತೀತನು.

ಒಂದು ಸಲ ಲಕ್ಷ್ಮಣ ರಾಮನಿಗೆ ಹೇಳಿದ, “ಇದು ಎಂತಹ ವಿಚಿತ್ರ! ಅಂತಹ ಜ್ಞಾನಿಗಳಾದ ವಸಿಷ್ಠರು ಮಕ್ಕಳ ಸಾವಿಗೆ ದುಃಖಪಟ್ಟರು.” ಆಗ ರಾಮ ಹೇಳಿದ, “ಸಹೋದರನೆ, ಯಾರಿಗೆ ಜ್ಞಾನವಿದೆಯೋ ಅವನಿಗೆ ಅಜ್ಞಾನವೂ ಇರುತ್ತದೆ. ಯಾರಿಗೆ ಒಂದರ ಅನುಭವವಿದೆಯೋ ಅವನಿಗೆ ಹಲವಿನ ಅನು ಭವವೂ ಇರಬೇಕು. ಯಾರಿಗೆ ಬೆಳಕು ಗೊತ್ತಿದೆಯೊ ಅವನಿಗೆ ಕತ್ತಲೆಯ ಅರಿವೂ ಇರಬೇಕು.”

ಬ್ರಹ್ಮನು ಜ್ಞಾನ, ಅಜ್ಞಾನಗಳಿಗೆ ಅತೀತ. ಧರ್ಮ, ಅಧರ್ಮ; ಶುಚಿ, ಅಶುಚಿ ಗಳಿಗೆ ಅತೀತ.


\section{\num{೧೭೯. } ಪ್ರಾಪಂಚಿಕರೊಡನೆ ಬೆರೆಯಬೇಡಿ–ಜೋಪಾನ!}

(ಪುರೋಹಿತರನ್ನು ಕುರಿತು ಶ್ರೀರಾಮಕೃಷ್ಣರು ಒಂದು ಘಟನೆಯನ್ನು ಶ್ರೀ ಗೌರಾಂಗನ ಜೀವನದಿಂದ ಹೇಳುತ್ತಿದ್ದರು.) ಗೌರಾಂಗ ಭಾವಸಮಾಧಿಯಲ್ಲಿ ತನ್ಮಯನಾದಾಗ ಸಮುದ್ರಕ್ಕೆ ಬಿದ್ದನು. ತೀರದಲ್ಲಿದ್ದ ಬೆಸ್ತರು ಅವನನ್ನು ತಮ್ಮ ಬಲೆಯಲ್ಲಿ ಹಿಡಿದು ಮೇಲಕ್ಕೆತ್ತಿದ್ದರು. ಅವನನ್ನು ಬಲೆಯಿಂದ ಮೇಲೆ ಎತ್ತುತ್ತಿದ್ದಾಗ ಆ ಮಹಾಪುರುಷನ ಸಂಸರ್ಗದಿಂದ ಅವರೂ ಕೂಡ ಭಾವಾ ವಸ್ಥೆಗೆ ಹೋದರು. ಅವರು ತಮ್ಮ ಕೆಲಸವನ್ನೆಲ್ಲ ಬಿಟ್ಟು ‘ಹರಿಬೋಲ್, ಹರಿಬೋಲ್​’ ಎಂದು ಭಜನೆ ಮಾಡುತ್ತ ಅಲೆಯತೊಡಗಿದರು. ಅವರ ನೆಂಟರಿಷ್ಟರಿಗೆ ಇದನ್ನು ಕಂಡು ಭಯವಾಯಿತು. ಯಾವುದರಿಂದಲೂ ಹರಿನಾಮದ ಜಾಡ್ಯವನ್ನು ಬಿಡಿಸಲಾಗಲಿಲ್ಲ. ಕೊನೆಗೆ ಅವರೆಲ್ಲ ಗೌರಾಂಗದೇವನ ಹತ್ತಿರ ಬಂದು ತಮ್ಮ ಗೋಳನ್ನು ಹೇಳಿಕೊಂಡರು. ಆಗ ಗೌರಾಂಗದೇವ ಅವರಿಗೆ, “ಒಬ್ಬ ಪುರೋಹಿತನ ಮನೆಯಿಂದ ಸ್ವಲ್ಪ ಅನ್ನವನ್ನು ತಂದು ಅವರ ಬಾಯಿಗೆ ಹಾಕಿ. ಅವರು ಗುಣ ವಾಗುವರು” ಎಂದನು. ಅದರಂತೆ ಮಾಡಿದಾಗ ಅವರು ದಿವ್ಯಭಾವವನ್ನು ಕಳೆದುಕೊಂಡರು. ಪ್ರಾಪಂಚಿಕರ ಸಂಪರ್ಕದಿಂದ ಆಗುವ ದುರವಸ್ಥೆ ಇದು.


\section{\num{೧೮೦. } ಆಧ್ಯಾತ್ಮಿಕ ಪ್ರಗತಿಯನ್ನು ಲೌಕಿಕ ದೃಷ್ಟಿಯಿಂದ ನೋಡಬೇಡಿ}

ಒಂದು ಸಲ ಪುಷಿಯೊಬ್ಬ ಗಾಢ ಧ್ಯಾನಾಸಕ್ತನಾಗಿ ರಸ್ತೆಯ ಪಕ್ಕದಲ್ಲಿ ಮಲಗಿದ್ದ. ದಾರಿಯಲ್ಲಿ ಹೋಗುತ್ತಿದ್ದ ಕಳ್ಳನೊಬ್ಬನು ಇದನ್ನು ನೋಡಿ, “ಈ ಮನುಷ್ಯ ಒಬ್ಬ ಕಳ್ಳನಿರಬೇಕು. ಯಾರದೋ ಮನೆಗೆ ರಾತ್ರಿ ಕನ್ನಹಾಕಿ ಈಗ ಸುಸ್ತಾಗಿ ಮಲಗಿರುವನು. ಇನ್ನು ಸ್ವಲ್ಪ ಹೊತ್ತಿನಲ್ಲಿ ಪೋಲೀಸಿನವರು ಬಂದು ಅವನನ್ನು ದಸ್ತಗಿರಿ ಮಾಡುತ್ತಾರೆ. ಅವರು ಬರುವುದಕ್ಕಿಂತ ಮುಂಚೆ ನಾನು ಓಡಿಹೋಗುವೆನು” ಎಂದು ಭಾವಿಸಿ ಅಲ್ಲಿಂದ ಓಟಕಿತ್ತ. ಸ್ವಲ್ಪ ಕಾಲ ಕಳೆದ ಮೇಲೆ ಒಬ್ಬ ಕುಡುಕ ಬಂದ. ಈ ಸಾಧುವನ್ನು ನೋಡಿ, “ನೀನು ಚೆನ್ನಾಗಿ ಕುಡಿದು ಈಗ ಚರಂಡಿಯಲ್ಲಿ ಬಿದ್ದಿರುವೆ. ಆದರೆ ನಾನು ನಿನಗಿಂತ ಮೇಲು, ಇನ್ನೂ ಸಂಚರಿಸುತ್ತಿರುವೆ. ಕೆಳಗೆ ಬೀಳುವುದಿಲ್ಲ” ಎಂದ. ಕೊನೆಗೆ ಒಬ್ಬ ಸಾಧು ಬಂದ. ಗಾಢಸಮಾಧಿಯಲ್ಲಿ ಮುಳುಗಿರುವ ಸಂತನೆಂದು ತಿಳಿದು ಅವನ ಹತ್ತಿರ ಕುಳಿತು ಅವನ ಕಾಲುಗಳನ್ನು ಒತ್ತತೊಡಗಿದ.

ಹೀಗೆ ನಮ್ಮ ಪ್ರಾಪಂಚಿಕ ಸ್ವಭಾವ ಆಧ್ಯಾತ್ಮಿಕ ವ್ಯಕ್ತಿಯನ್ನು ಗುರುತಿಸ ಲಾರದು.


\section{\num{೧೮೧. } ಯಾವಾಗಲೂ ಜೋಪಾನವಾಗಿರು}

ಯಾವಾಗಲೂ ಜಾಗೃತನಾಗಿರಬೇಕು. ಬಟ್ಟೆಯೂ ಕೂಡ ನಮ್ಮಲ್ಲಿ ಅಹಂ ಕಾರವನ್ನು ತರುವುದು. ಮಲೇರಿಯಾ ರೋಗದಿಂದ ನರಳುತ್ತಿರುವವನು ಕೂಡ, ಕಪ್ಪು ಅಂಚಿನ ಬಟ್ಟೆಯನ್ನು ಹಾಕಿಕೊಂಡಾಗ ಕಾಮುಕನ ಹಾಡುಗಳು ಅವನ ಬಾಯಿಗೆ ಬರುತ್ತವೆ. ಠೀವಿಯಿಂದ ಎತ್ತರವಾಗಿರುವ ಬೂಟುಗಳನ್ನು ಹಾಕಿಕೊಂಡಾಗ ಬರೀ ಇಂಗ್ಲಿಷ್ ಭಾಷೆಯನ್ನೇ ಮಾತನಾಡುತ್ತಾರೆ. ಅಯೋಗ್ಯ ಕಾವಿ ಬಟ್ಟೆಯನ್ನು ಧರಿಸಿದಾಗ ತನ್ನ ಸಮಾನ ಇಲ್ಲ ಎಂದು ಮೆರೆಯು ವನು. ಏನಾದರೂ ಅವನಿಗೆ ಗೌರವ ತೋರಿಸು ವುದರಲ್ಲಿ ಕಡಿಮೆಯಾದರೆ ಅವನಿಗೆ ಕೋಪ ಬರುವುದು.


\section{\num{೧೮೨. } ಕೆಲವು ವೇಳೆ ನಾಯಿಗೆ ಚೆನ್ನಾಗಿ ಹೊಡೆಯಬೇಕಾಗುವುದು}

ಒಬ್ಬ ಮನುಷ್ಯ ಒಂದು ನಾಯಿಯನ್ನು ಸಾಕಿದ್ದ. ಅದನ್ನು ತುಂಬ ಮುದ್ದಿಸುತ್ತಿದ್ದ. ತನ್ನ ಕೈಗಳಿಂದ ಅದನ್ನು ಎತ್ತಿಕೊಂಡು ಆಡಿಸುತ್ತಿದ್ದ. ಒಬ್ಬ ಬುದ್ಧಿವಂತ ಇವನ ಸ್ವಭಾವವನ್ನು ಕಂಡು, “ನಾಯಿಯೊಂದಿಗೆ ಅಷ್ಟು ಸಲಿಗೆ ತೋರಿಸಬೇಡ. ಎಷ್ಟಾದರೂ ಅದು ಬುದ್ಧಿಯಿಲ್ಲದ ನಾಯಿ, ಯಾವತ್ತಾದರೂ ನಿನ್ನನ್ನು ಕಚ್ಚಬಹುದು” ಎಂದು ಎಚ್ಚರಿಸಿದ. ಮನೆಯವನು ಇದನ್ನು ತಿಳಿದುಕೊಂಡು, ನಾಯಿಯನ್ನು ತನ್ನ ಕಂಕುಳಿಂದ ಕೆಳಕ್ಕೆ ಇಳಿಸಿದ. ಇನ್ನು ಮೇಲೆ ಅದನ್ನು ಮುದ್ದಿಸುವುದಿಲ್ಲ, ಪ್ರೀತಿಸುವುದಿಲ್ಲ ಎಂದು ಪ್ರತಿಜ್ಞೆ ಮಾಡಿದ. ಆದರೆ ನಾಯಿಗೆ ಯಜಮಾನ ತನ್ನ ಮನಸ್ಸನ್ನು ಬದಲಾಯಿಸಿರುವನು ಎಂದು ಗೊತ್ತಾಗಲಿಲ್ಲ. ಪುನಃ ಪುನಃ ಅದು, ತನ್ನನ್ನು ಮುದ್ದಿಸಲಿ ಎಂದು ಯಜಮಾನನ ಹತ್ತಿರ ಬರುತ್ತಿತ್ತು. ಹಲವು ವೇಳೆ ಅದನ್ನು ಹೊಡೆದ ಮೇಲೆ ಯಜಮಾನನ ಹತ್ತಿರ ಹೋಗುವುದನ್ನು ಬಿಟ್ಟಿತು.

ಪ್ರತಿಯೊಬ್ಬರೂ ಇದರಂತೆಯೆ. ಯಾವ (ಕಾಮವೆಂಬ) ನಾಯಿಯನ್ನು ನೀನು ಅಷ್ಟೊಂದು ಮುದ್ದಿಸುತ್ತಿದ್ದೆಯೋ ಅದು ಬೇಡ ಎಂದ ತಕ್ಷಣವೇ ಯಜಮಾನನನ್ನು ಬಿಡುವುದಿಲ್ಲ. ನಮಗೆ ಅದು ಬೇಡದೆ ಇದ್ದರೂ ಹತ್ತಿರ ಬರುವುದು. ಬಂದರೆ ಅದರಿಂದ ಅಪಾಯವೇನೂ ಇಲ್ಲ. ಮತ್ತೊಮ್ಮೆ ನಾಯಿಯನ್ನು ಮುದ್ದಿಸಬೇಡ. ಮುದ್ದಿಸಲಿ ಎಂದು ನಿನ್ನ ಹತ್ತಿರ ಬಂದಾಗ ಅದನ್ನು ಹೊಡೆದೋಡಿಸು. ಕಾಲಕ್ರಮೇಣ ಅದು ನಿನ್ನ ಹತ್ತಿರ ಬರುವುದನ್ನು ಬಿಡುವುದು.


\section{\num{೧೮೩. } ಮಧ್ಯ ಮಧ್ಯ ಮುಳುಗು}

ನೀವು ಮುಂದುವರಿದಂತೆ, ಚಂದನದ ಮರದಾಚೆ ಇನ್ನೂ ಏನೇನೋ ಇದೆ ಎನ್ನುವುದು ಗೊತ್ತಾಗುವುದು. ಬೆಳ್ಳಿಯ ಗಣಿ, ಚಿನ್ನದ ಗಣಿ, ವಜ್ರ ವೈಡೂರ್ಯ ಗಳು ಎಲ್ಲವೂ ಇದೆ. ಅದಕ್ಕಾಗಿ ಮುಂದೆ ಹೋಗಿ.

ಜನರಿಗೆ ಮುಂದೆ ಹೋಗಿ ಎಂದು ನಾನು ಹೇಗೆ ಹೇಳಲು ಸಾಧ್ಯ? ಪ್ರಾಪಂಚಿಕರು ಬಹಳ ಮುಂದುವರಿದರೆ ಅವರ ನೆಲೆಯೇ ತಪ್ಪುತ್ತದೆ. ಒಂದು ದಿನ ಕೇಶವಚಂದ್ರಸೇನ ಪ್ರಾರ್ಥನೆ ನಡೆಸುತ್ತಿದ್ದ. “ದೇವರೆ, ನಾವುಗಳೆಲ್ಲ ಭಗವಂತನಲ್ಲಿ ಮುಳುಗಿ ತನ್ಮಯರಾಗಿ ಹೋಗೋಣ” ಎಂದು ಹೇಳುತ್ತಿದ್ದ. ಪ್ರಾರ್ಥನೆ ಆದಮೇಲೆ ನಾನು ಅವನಿಗೆ ಹೇಳಿದೆ, “ನೋಡು, ನೀವು ಭಕ್ತಿ ನದಿಯಲ್ಲಿ ಹೇಗೆ ಮುಳುಗಿ ಹೋಗುವುದಕ್ಕೆ ಸಾಧ್ಯ? ನೀವು ಹಾಗೆ ಮಾಡಿದರೆ ತೆರೆಯ ಹಿಂದೆ ಇರುವ ನಿಮ್ಮ ಮನೆಯ ಹೆಂಗಸರ ಪಾಡು ಏನಾಗಬೇಕು? ಆದರೆ ಹೀಗೆ ಮಾಡಬಹುದು, ಮಧ್ಯೆ ಮಧ್ಯೆ ನೀರಿನಲ್ಲಿ ಸ್ವಲ್ಪ ಕಾಲವಿದ್ದು ಅನಂತರ ಹೊರಗೆ ಬನ್ನಿ.”


\section{\num{೧೮೪. } ಕೈ ಮುಷ್ಟಿಯನ್ನು ಸ್ವಲ್ಪ ಸಡಿಲಬಿಡಿ}

ಕಾಮಿನಿಕಾಂಚನವೇ ಸಂಸಾರ. ಅನೇಕರಿಗೆ ಹಣವೇ ಪ್ರಾಣ. ನೀವು ಹಣ ವನ್ನು ಎಷ್ಟು ಪ್ರೀತಿಸಿದರೂ ಒಂದು ದಿನ ಎಲ್ಲವೂ ಹೊರಟುಹೋಗುವುದು. ನಮ್ಮ ಊರಿನ ಕಡೆ ಗದ್ದೆಯ ಸುತ್ತಲೂ ಕಟ್ಟೆ ಕಟ್ಟುತ್ತಾರೆ. ಅದು ಏನೆಂಬುದು ನಿಮಗೆ ಗೊತ್ತು. ಕೆಲವರು ಬಹಳ ಜಾಗರೂಕತೆಯಿಂದ ತಮ್ಮ ಹೊಲದ ಸುತ್ತಲೂ ಎತ್ತರವಾದ ಕಟ್ಟೆಯನ್ನು ಕಟ್ಟುತ್ತಾರೆ. ಮಳೆ ನೀರು ಬಂದೊಡನೆ ಕಟ್ಟಿದ ಕಟ್ಟೆಯೆಲ್ಲ ಕೊಚ್ಚಿಕೊಂಡು ಹೋಗುವುದು. ಕೆಲವು ರೈತರು ನೀರು ಹರಿದುಕೊಂಡು ಹೋಗುವುದಕ್ಕೆ ದಾರಿ ಮಾಡಿರುತ್ತಾರೆ. ಅದನ್ನು ಹುಲ್ಲಿನಿಂದ ಮುಚ್ಚಿರುತ್ತಾರೆ. ನೀರು ಹುಲ್ಲಿನ ಮೂಲಕ ಹರಿದು ಹೋಗುವಾಗ ಮೆಕ್ಕಲು ಮಣ್ಣನ್ನು ಅಲ್ಲೆ ನಿಲ್ಲಿಸುತ್ತದೆ. ಆಗ ಚೆನ್ನಾದ ಬೆಳೆ ಬರುವುದು. ಭಗವಂತನ ಪೂಜೆಗೆ ಮತ್ತು ಸಾಧು ಸಂತರಿಗಾಗಿ ಯಾರು ಹಣವನ್ನು ಖರ್ಚುಮಾಡುವರೋ, ಅವರೇ ಹಣದಿಂದ ಪ್ರಯೋ ಜನವನ್ನು ಪಡೆಯುವರು.


\section{\num{೧೮೫. } ನಮಗೆ ಗೊತ್ತಿಲ್ಲದ ಭವಿಷ್ಯವನ್ನು ನೆಚ್ಚಬೇಡ}

ಒಂದು ಸಲ ಜೂನ್ ತಿಂಗಳಲ್ಲಿ ಆಡಿನಮರಿಯೊಂದು ತಾಯಿಯ ಹತ್ತಿರ ಆಟವಾಡುತ್ತಿತ್ತು. ಆ ಮರಿ ಸಂತೋಷದಿಂದ ತಾಯಿಗೆ, ನಾನು ರಾಸ ಹೂವನ್ನು (ನವೆಂಬರ್​ನಲ್ಲಿ, ರಾಸಲೀಲೆಯ ಸಮಯದಲ್ಲಿ ಬಿಡುವ ಒಂದು ಬಗೆಯ ಹೂ) ತಿನ್ನಬೇಕೆಂದಿರುವೆ” ಎಂದಿತು. ಆಗ ಮರಿಗೆ ಆಡು ಹೇಳಿತು: “ನೀನು ತಿಳಿದುಕೊಂಡಿರುವಷ್ಟು ಸುಲಭವಲ್ಲ ಅದು. ಮುಂದೆ ಬರುವ ಸೆಪ್ಟೆಂಬರ್ ಮತ್ತು ಅಕ್ಟೋಬರ್ ತಿಂಗಳುಗಳು ನಿನಗೆ ಒಳ್ಳೆಯದಲ್ಲ. ಕೆಲವರು ನಿನ್ನನ್ನು ದುರ್ಗೆಗೆ ಬಲಿ ಕೊಡಲು ತೆಗೆದುಕೊಂಡು ಹೋಗಬಹುದು. ಅನಂತರ ಭಯಂಕರವಾದ ಕಾಳಿ ಪೂಜೆ ಇದೆ. ಆಗಲೂ ನೀನು ಬದುಕಿ ಉಳಿದರೆ ಜಗ ದ್ಧಾತ್ರಿ ಪೂಜೆ ಇದೆ. ಆ ಸಮಯದಲ್ಲಿ ಅಳಿದುಳಿದ ನಮ್ಮ ವಂಶದವರನ್ನೆಲ್ಲ ಬಲಿಕೊಡುತ್ತಾರೆ. ಈ ಎಲ್ಲಾ ಆಪತ್ತುಗಳಿಂದ ನೀನು ಪಾರಾಗುವಷ್ಟು ಅದೃಷ್ಟ ಶಾಲಿಯಾಗಿದ್ದರೆ, ಆಗ ನವೆಂಬರ್ ತಿಂಗಳಲ್ಲಿ ರಾಸ ಹೂವನ್ನು ತಿನ್ನಬಹುದು.”

ಯೌವನದ ಭರದಲ್ಲಿ ನಮ್ಮ ಮನಸ್ಸಿನಲ್ಲಿರುವ ಯೋಚನೆಗಳೆಲ್ಲ ಫಲಕಾರಿ ಯಾಗುವುದು ಎಂದು ಭಾವಿಸಬಾರದು. ನಮ್ಮ ಜೀವನದಲ್ಲಿ ನಾವು ಎಷ್ಟೋ ಗಂಡಾಂತರಗಳ ಮೂಲಕ ಹೋಗಬೇಕಾಗಿದೆ.


\section{\num{೧೮೬. } ದಾನ ಮಾಡುವಾಗಲೂ ವಿವೇಚನೆ ಇರಲಿ}

ಒಂದು ಸಲ ಒಬ್ಬ ಕಟುಕ ಹಸುವನ್ನು ದೂರದಲ್ಲಿರುವ ಕಸಾಯಿಖಾನೆಗೆ ತೆಗೆದುಕೊಂಡು ಹೋಗುತ್ತಿದ್ದ. ಕಟುಕ ಹಸುವನ್ನು ನಿರ್ದಯದಿಂದ ನೋಡು ತ್ತಿದ್ದುದರಿಂದ ಅದು ಇವನು ಹೇಳಿದ ಮಾತನ್ನು ಕೇಳುತ್ತಿರಲಿಲ್ಲ. ಸ್ವಲ್ಪ ದೂರ ಹಾಗೆಯೇ ಬಲಾತ್ಕಾರದಿಂದ ಎತ್ತನ್ನು ಅಟ್ಟಿಸಿಕೊಂಡು ಹೋಗುತ್ತಿದ್ದಾಗ ಕಟುಕನಿಗೆ ಹಸಿವಾಯಿತು. ಹತ್ತಿರದಲ್ಲಿರುವ ಛತ್ರಕ್ಕೆ ಹೋಗಿ ಅಲ್ಲಿ ಅವರು ಕೊಡುತ್ತಿದ್ದ ಬಿಟ್ಟಿ ಊಟವನ್ನು ಮಾಡಿದ. ಊಟದಿಂದ ಅವನಿಗೆ ಸ್ವಲ್ಪ ತ್ರಾಣ ಬಂದಾದ ಮೇಲೆ ಹಸುವನ್ನು ಕಸಾಯಿ ಖಾನೆಗೆ ಸುಲಭವಾಗಿ ಸಾಗಿಸಿಕೊಂಡು ಹೋದ. ಆ ಹಸುವನ್ನು ಕೊಂದ ಪಾಪದ ಭಾಗ, ಆ ಅನ್ನಛತ್ರದಲ್ಲಿ ನೀಡಿದ ಊಟದ ದಾನಿಗೂ ಸೋಂಕಿತು.

ಆದಕಾರಣವೇ ದಾನ ಧರ್ಮ ಮಾಡುವಾಗ ಯಾರಿಗೆ ಅದನ್ನು ಕೊಡು ತ್ತೇವೆಯೋ ಅವನು ಧರ್ಮಾತ್ಮನೇ ಅಲ್ಲವೇ–ಎಂಬುದನ್ನು ಯೋಚಿಸಬೇಕು.


\section{\num{೧೮೭. } ನೀನು ಬೇಕಾದರೆ ಬುಸುಗುಟ್ಟು, ಆದರೆ ಎಂದಿಗೂ ಕಚ್ಚಬೇಡ}

ಕೆಲವು ಗೊಲ್ಲರು ಒಂದು ಬಯಲಿನಲ್ಲಿ ದನ ಕಾಯುತ್ತಿದ್ದರು. ಅಲ್ಲಿ ಒಂದು ಘಟಸರ್ಪ ಇತ್ತು. ಎಲ್ಲರೂ ಅದಕ್ಕೆ ಅಂಜುತ್ತಿದ್ದರು. ಒಂದು ಸಲ ಒಬ್ಬ ಬ್ರಹ್ಮಚಾರಿ ಆ ದಾರಿಯಲ್ಲಿ ಹೋಗುತ್ತಿದ್ದ. ಹುಡುಗರು ಓಡಿ ಹೋಗಿ ಬ್ರಹ್ಮಚಾರಿಗೆ, “ಅಲ್ಲಿ ಒಂದು ವಿಷಸರ್ಪವಿದೆ. ಅದರ ಬಳಿಗೆ ಹೋಗಬೇಡಿ” ಎಂದರು. “ಆದರೇನು ಮಕ್ಕಳೆ, ನಾನು ಸರ್ಪಕ್ಕೆ ಅಂಜುವುದಿಲ್ಲ. ನನ್ನಲ್ಲಿ ಮಂತ್ರಶಕ್ತಿ ಇದೆ” ಎಂದನು ಬ್ರಹ್ಮಚಾರಿ. ಹೀಗೆ ಹೇಳುತ್ತ ಬಯಲುಗಾವಲಿಗೆ ಬಂದನು. ಗೊಲ್ಲ ಹುಡುಗರು ಸರ್ಪಕ್ಕೆ ಅಂಜಿ ಬ್ರಹ್ಮಚಾರಿಯ ಜೊತೆಗೆ ಬರ ಲಿಲ್ಲ. ಅಷ್ಟು ಹೊತ್ತಿಗೆ ಸರಿಯಾಗಿ ಆ ಸರ್ಪ ತನ್ನ ಹೆಡೆಯನ್ನು ಎತ್ತಿಕೊಂಡು ಬ್ರಹ್ಮಚಾರಿಯನ್ನು ಕಚ್ಚಲು ಬಂದಿತು. ಆ ಸರ್ಪ ಬಂದಾಗ ಬ್ರಹ್ಮಚಾರಿ ಒಂದು ಮಂತ್ರವನ್ನು ಹಾಕಿದ. ಹಾವು ಒಂದು ಎರೆಹುಳದಂತೆ ಅಲ್ಲಿ ತೆಪ್ಪಗೆ ಬಿದ್ದಿತು. ಬ್ರಹ್ಮಚಾರಿ ಆ ಸರ್ಪಕ್ಕೆ, “ಏತಕ್ಕೆ ಎಲ್ಲರಿಗೂ ಕಚ್ಚುತ್ತಿರುವೆ? ಬಾ, ನಾನು ನಿನಗೆ ಒಂದು ಪವಿತ್ರ ಮಂತ್ರವನ್ನು ಉಪದೇಶ ಮಾಡುತ್ತೇನೆ. ನೀನು ಅದನ್ನು ಜಪಿಸುತ್ತಿದ್ದರೆ ಭಗವಂತನ ಬಳಿಗೆ ಹೋಗುವೆ, ಅವನನ್ನು ಪ್ರೀತಿಸುವೆ. ಕೊನೆಗೆ ಅವನ ಸಾಕ್ಷಾತ್ಕಾರವನ್ನು ಪಡೆದು ನಿನ್ನ ದುಷ್ಟಸ್ವಭಾವವನ್ನು ಬಿಟ್ಟು ಬಿಡುವೆ” ಎಂದನು. ಹೀಗೆ ಹೇಳುತ್ತ ಹಾವಿಗೆ ಮಂತ್ರೋಪದೇಶ ಮಾಡಿದನು. ಸರ್ಪ ಗುರುಗಳಿಗೆ ನಮಸ್ಕರಿಸಿ, “ಸ್ವಾಮಿಗಳೇ, ನಾನು ಹೇಗೆ ಸಾಧನೆ ಮಾಡಲಿ?” ಎಂದು ಕೇಳಿತು. ಬ್ರಹ್ಮಚಾರಿ “ಉಪದೇಶ ಮಾಡಿದ ಮಂತ್ರವನ್ನು ಉಚ್ಚರಿಸು ಮತ್ತು ಯಾರಿಗೂ ತೊಂದರೆಯನ್ನು ಕೊಡಬೇಡ” ಎಂದನು. ಹೋಗುವ ಸಮಯದಲ್ಲಿ, “ನಾನು ಪುನಃ ಬಂದು ನಿನ್ನನ್ನು ನೋಡುತ್ತೇನೆ” ಎಂದನು.

ಕೆಲವು ದಿನಗಳಾದವು. ದನ ಕಾಯುವವರು ಈ ಸರ್ಪ ಈಗ ಕಚ್ಚುತ್ತಿಲ್ಲ ಎಂಬುದನ್ನು ಕಂಡುಹಿಡಿದರು. ಅದರ ಕಡೆಗೆ ಕಲ್ಲನ್ನು ಎಸೆದರು. ಆದರೂ ಅದು ಕೋಪವನ್ನು ತೋರಲಿಲ್ಲ. ಮಣ್ಣಿನ ಹುಳುವಿನಂತೆ ಸುಮ್ಮ ನಿತ್ತು. ಒಂದು ದಿನ ಒಬ್ಬ ಹುಡುಗ ಹಾವನ್ನು ಕೈಯಿಂದ ತೆಗೆದುಕೊಂಡು ಅದನ್ನು ಗರಗರ ಸುತ್ತಿ ಎಸೆದನು. ಹಾವು ರಕ್ತಕಾರಿ ಪ್ರಜ್ಞೆ ಇಲ್ಲದೆ ಬಿತ್ತು. ಅದಕ್ಕೆ ಚಲಿಸುವುದಕ್ಕೇ ಆಗಲಿಲ್ಲ. ಅದು ಸತ್ತು ಹೋಯಿತು ಎಂದು ದನ ಕಾಯುವವರು ಹೊರಟುಹೋದರು.

ರಾತ್ರಿ ಬಹಳ ಹೊತ್ತಾದ ಮೇಲೆ ಹಾವಿಗೆ ಪ್ರಜ್ಞೆ ಬಂತು. ತುಂಬಾ ಕಷ್ಟಪಟ್ಟು ತನ್ನ ಬಿಲಕ್ಕೆ ಹೋಯಿತು. ಅದರ ಮೂಳೆಗಳೆಲ್ಲ ಮುರಿದು ಹೋಗಿತ್ತು. ಅದಕ್ಕೆ ಸಂಚಾರ ಮಾಡುವುದಕ್ಕೆ ಸಾಧ್ಯವಾಗಲಿಲ್ಲ. ಕೆಲವು ದಿನಗಳಾದ ಮೇಲೆ ಹಾವು ಮೂಳೆಚಕ್ಕಳ ಆಯಿತು. ಯಾವಾಗಲಾದರೊಮ್ಮೆ ಅದು ಆಹಾರಕ್ಕೆ ರಾತ್ರಿ ಹೊತ್ತು ಹೊರ ಬರುತ್ತಿತ್ತು. ದನಕಾಯುವವರು ತನ್ನನ್ನು ಹೊಡೆದಾರು ಎಂದು ಹಗಲು ಹೊರಗೆ ಬರುತ್ತಿರಲಿಲ್ಲ. ಗುರುಗಳಿಂದ ಉಪದೇಶ ಪಡೆದ ಮೇಲೆ ಇತರರಿಗೆ ಹಿಂಸೆ ಕೊಡುವುದನ್ನು ಬಿಟ್ಟುಬಿಟ್ಟಿತ್ತು. ರಾತ್ರಿ ಏನಾದರೂ ಕೊಳೆ ಕಶ್ಮಲ, ಎಲೆ, ಕೆಳಗೆ ಬಿದ್ದ ಹಣ್ಣು ಇವುಗಳನ್ನು ತಿನ್ನುತ್ತಿತ್ತು. 

ಒಂದು ವರುಷ ಆದಮೇಲೆ ಬ್ರಹ್ಮಚಾರಿ ಅದೇ ದಾರಿಯಲ್ಲಿ ಬಂದನು. “ಈಗ ಹಾವು ಹೇಗಿದೆ” ಎಂದು ಬ್ರಹ್ಮಚಾರಿ ದನಕಾಯುವವರನ್ನು ಕೇಳಿದನು. ದನಕಾಯುವವರು “ಅದು ಸತ್ತು ಹೋಯಿತು” ಎಂದರು. ಆದರೆ ಬ್ರಹ್ಮಚಾರಿ ಅವರ ಮಾತನ್ನು ನಂಬಲಿಲ್ಲ. ಅದಕ್ಕೆ ಯಾವ ಮಂತ್ರೋಪದೇಶವನ್ನು ಮಾಡಿ ದ್ದನೋ ಅದು ಫಲಿಸುವವರೆಗೆ ಅದು ಸಾಯಲಾರದು ಎಂಬುದು ಅವನಿಗೆ ತಿಳಿದಿತ್ತು. ಅವನು ಹಾವು ಎಲ್ಲಿಯಾದರೂ ಇದೆಯೇ ಎಂದು ಹುಡುಕಾಡಿದ. ಅವನು ಅದಕ್ಕೆ ಯಾವ ಹೆಸರನ್ನು ಇಟ್ಟಿದ್ದನೊ ಆ ಹೆಸರಿನಿಂದ ಅದನ್ನು ಕರೆದ. ಗುರುಗಳ ಧ್ವನಿಯನ್ನು ಆಲಿಸಿ ಅದು ತನ್ನ ಬಿಲದಿಂದ ಹೊರಗೆ ಬಂತು. ಬಹಳ ಭಕ್ತಿಯಿಂದ ಸರ್ಪವು ನಮಸ್ಕಾರ ಮಾಡಿತು. ಬ್ರಹ್ಮಚಾರಿ ಹಾವಿಗೆ, “ನೀನು ಈಗ ಹೇಗಿರುವೆ?” ಎಂದು ಕೇಳಿದನು. ಹಾವು, “ನಾನೇನೋ ಚೆನ್ನಾಗಿದ್ದೇನೆ” ಎಂದಿತು. ಆದರೆ ಗುರುಗಳು, “ಏತಕ್ಕೆ ಇಷ್ಟೊಂದು ಸವೆದು ಹೋಗಿರುವೆ?” ಎಂದು ಕೇಳಿದರು. ಅದಕ್ಕೆ ಹಾವು, “ಯಾರಿಗೂ ಕಚ್ಚಬೇಡ ಎಂದು ನನಗೆ ಹೇಳಿದಿರಿ. ಆದ್ದರಿಂದ ನಾನು ಎಲೆ, ಹಣ್ಣು ಇದರ ಮೇಲೆ ಜೀವಿಸುತ್ತಿರುವೆ. ಬಹುಶಃ ಅದರಿಂದಲೇ ನಾನು ಬಡವಾಗಿರಬಹುದು” ಎಂದಿತು. ಸರ್ಪದಲ್ಲಿ ಸತ್ತ್ವಗುಣ ವೃದ್ಧಿಯಾಗಿ ಅದು ಯಾರನ್ನೂ ದೂರಲಿಲ್ಲ. ದನಕಾಯುವವರು ತನ್ನನ್ನು ಸಾಯಿಸುವುದರಲ್ಲಿದ್ದರು ಎಂಬುದನ್ನು ಅದು ಮರೆತುಬಿಟ್ಟಿತ್ತು.

ಬ್ರಹ್ಮಚಾರಿ ಹೇಳಿದ, “ನೀನು ಬರೀ ಆಹಾರದ ಅಭಾವದಿಂದ ಈ ಸ್ಥಿತಿಗೆ ಬರಲಿಲ್ಲ. ಬೇರೆ ಏನೋ ಕಾರಣ ಇರಬೇಕು. ಆಲೋಚನೆ ಮಾಡಿ ಹೇಳು” ಎಂದನು. ಆಗ ದನಕಾಯುವವರು ತನ್ನನ್ನು ತಿರುಗಿಸಿ ಚೆನ್ನಾಗಿ ನೆಲದ ಮೇಲೆ ಅಪ್ಪಳಿಸಿದ ವಿಷಯ ನೆನಪಿಗೆ ಬಂತು. “ಪೂಜ್ಯರೆ, ಈಗ ನನಗೆ ಗೊತ್ತಾಗುತ್ತಿದೆ. ಒಂದು ದಿನ ದನ ಕಾಯುವವರು ನನ್ನನ್ನು ನೆಲದ ಮೇಲೆ ಅಪ್ಪಳಿ ಸಿದರು. ಪಾಪ ಅವರಿಗೇನೂ ಗೊತ್ತಿಲ್ಲ, ನನ್ನ ಮನಸ್ಸಿನ ಮೇಲೆ ಆದ ಪರಿಣಾಮ, ಪಾಪ ಅವರಿಗೆ ಹೇಗೆ ಗೊತ್ತಾಗಬೇಕು –ನಾನು ಕಚ್ಚುವುದಿಲ್ಲ, ಇನ್ನೊಬ್ಬರಿಗೆ ತೊಂದರೆ ಮಾಡುವುದಿಲ್ಲ ಎಂಬುದು,” ಎಂದು ವಿನಯದಿಂದ ನುಡಿಯಿತು.

ಇದನ್ನು ಕೇಳಿ ಬ್ರಹ್ಮಚಾರಿ, “ಎಂತಹ ಮೂಢ ನೀನು! ನಾನು ನಿನಗೆ ಇನ್ನೊಬ್ಬರನ್ನು ಕಚ್ಚಬೇಡ ಎಂದೆ, ಆದರೆ ಬುಸುಗುಟ್ಟಬೇಡ ಎಂದು ಹೇಳಿ ದೆನೆ? ನೀನು ಏಕೆ ಬುಸುಗುಟ್ಟಿ ಅವರನ್ನು ಆಚೆಗೆ ಓಡಿಹೋಗುವಂತೆ ಮಾಡ ಲಿಲ್ಲ?” ಎಂದು ಹೇಳಿದ. ದುರ್ಜನರು ಬಂದಾಗ ನಾವು ಬುಸುಗುಟ್ಟಬೇಕು –ಅವರು ನಮಗೆ ಅಪಾಯ ಮಾಡದಿರಲಿ ಎಂದು. ಭುಸುಗುಟ್ಟಬೇಕು, ಆದರೆ ವಿಷವನ್ನು ಕಾರಬೇಕಾಗಿಲ್ಲ, ಇತರರಿಗೆ ಹಿಂಸೆಯನ್ನು ಕೊಡಬೇಕಾಗಿಲ್ಲ.


\section{\num{೧೮೮. } ಮೊದಲು ನೀನು ಶುದ್ಧನಾಗಿ ಅನಂತರ ಇತರರಿಗೆ ಬೋಧಿಸು}

ಒಂದು ಹಳ್ಳಿಯಲ್ಲಿ ಪದ್ಮಲೋಚನನೆಂಬುವನು ಇದ್ದನು. ಅವನನ್ನು ಪೊದೆ ಎಂದು ಕರೆಯುತ್ತಿದ್ದರು. ಆ ಊರಿನಲ್ಲಿ ಒಂದು ಪಾಳುಬಿದ್ದ ದೇವಸ್ಥಾನ ವಿತ್ತು. ಅಲ್ಲಿ ಪೂಜಿಸಲು ದೇವರು ಮುಂತಾದವುಗಳು ಯಾವುದೂ ಇರಲಿಲ್ಲ. ಅದರ ಗೋಡೆಯ ಮೇಲೆ ಅಶ್ವತ್ಥ ಮುಂತಾದ ಮರಗಳು ಬೆಳೆದಿದ್ದವು. ಅಲ್ಲಿ ಬಾವಲಿಗಳು ವಾಸವಾಗಿದ್ದವು. ನೆಲವೆಲ್ಲ ಧೂಳು ಮತ್ತು ಬಾವಲಿಗಳ ಹೇಸಿಗೆ ಯಿಂದ ಕೊಳೆಯಾಗಿತ್ತು. ಊರಿನ ಜನ ಆ ದೇವಸ್ಥಾನಕ್ಕೆ ಹೋಗುತ್ತಿರಲಿಲ್ಲ. ಅದೊಂದು ಪಾಳು ದೇವಾಲಯವಾಗಿತ್ತು. ಒಂದು ಸಂಜೆ ಹಳ್ಳಿಯವರು ದೇವಸ್ಥಾನದಿಂದ ಶಂಖದ ಧ್ವನಿಯನ್ನು ಕೇಳಿದರು. ಯಾರೋ ಬಂದು ದೇವರನ್ನು ಪ್ರತಿಷ್ಠೆ ಮಾಡಿ ಪೂಜೆ ಮಾಡುತ್ತಿರಬಹುದು ಎಂದು ಭಾವಿಸಿದರು. ಒಬ್ಬ ನಿಧಾನವಾಗಿ ಬಾಗಿಲನ್ನು ತೆರೆದನು. ಅಲ್ಲಿ ಪದ್ಮಲೋಚನ ಒಂದು ಮೂಲೆಯಲ್ಲಿ ನಿಂತುಕೊಂಡು ಶಂಖವನ್ನು ಊದುತ್ತಿದ್ದನು. ಯಾವ ವಿಗ್ರಹ ವನ್ನೂ ಸ್ಥಾಪನೆ ಮಾಡಿರಲಿಲ್ಲ. ದೇವಸ್ಥಾನವನ್ನು ಗುಡಿಸಿರಲಿಲ್ಲ. ಸಾರಿಸಿರ ಲಿಲ್ಲ. ಎಲ್ಲಾ ಕಡೆಯಲ್ಲಿಯೂ ಕೊಳೆ ಕಸ ತುಂಬಿತ್ತು. ಆಗ ಅವನು ಪದ್ಮ ಲೋಚನನಿಗೆ ಹೇಳಿದ:

“ದೇವಸ್ಥಾನದಲ್ಲಿ ನೀನು ಯಾವ ವಿಗ್ರಹವನ್ನೂ ಪ್ರತಿಷ್ಠೆ ಮಾಡಲಿಲ್ಲ. ಬರೀ ಶಂಖವನ್ನು ಮಾತ್ರ ಊದುತ್ತಿರುವೆ. ಆಗಲೇ\\ಇರುವ ಗೊಂದಲವನ್ನು ಇನ್ನೂ ಹೆಚ್ಚಿಸುತ್ತಿರುವೆ, ಹಗಲಿರುಳು ಹನ್ನೊಂದು ಬಾವಲಿಗಳು ಕಿರಿಚುತ್ತಿರುವವು ಅಲ್ಲಿ.”

ನಿನ್ನ ಹೃದಯಮಂದಿರದಲ್ಲಿ ಭಗವಂತನನ್ನು ಪ್ರತಿಷ್ಠೆ ಮಾಡದೆ ಸುಮ್ಮನೆ ಗಲಾಟೆ ಮಾಡುವುದರಿಂದ ಏನೂ ಪ್ರಯೋಜನವಿಲ್ಲ. ಮೊದಲು ನಿನ್ನನ್ನು ಶುದ್ಧಗೊಳಿಸು. ನಿರ್ಮಲ ಹೃದಯದಲ್ಲಿ ಭಗವಂತ ಬಂದು ಮಂಡಿಸುತ್ತಾನೆ. ಬಾವಲಿಯ ಹಿಕ್ಕೆಗಳು ಗುಡಿಯಲ್ಲೆಲ್ಲ ಇದ್ದರೆ, ದೇವರನ್ನು ಅಲ್ಲಿಗೆ ತೆಗೆದು ಕೊಂಡು ಬರಲು ಆಗುವುದಿಲ್ಲ. ಹನ್ನೊಂದು ಬಾವಲಿಗಳೇ ಹನ್ನೊಂದು ಇಂದ್ರಿಯಗಳು. ಐದು ಜ್ಞಾನೇಂದ್ರಿಯಗಳು, ಐದು ಕರ್ಮೇಂದ್ರಿಯಗಳು ಮತ್ತು ಮನಸ್ಸೇ ಹನ್ನೊಂದನೆಯ ಇಂದ್ರಿಯ.

ಮೊದಲು ದೇವರನ್ನು ಪ್ರತಿಷ್ಠೆ ಮಾಡು. ಅನಂತರ ನೀನು ಬೇಕಾದಷ್ಟು ಪ್ರವಚನಗಳನ್ನು ಮಾಡು. ಮುಂಚೆ ನೀರಿನ ಒಳಗೆ ಮುಳುಗು. ಅಲ್ಲಿರುವ ಮುತ್ತುರತ್ನಗಳನ್ನು ಹೊರಗೆ ತೆಗೆದು ಕೊಂಡು ಬಾ. ಅನಂತರ ನೀನು ಬೇರೆ ಕೆಲಸಗಳನ್ನು ಮಾಡು.


\section{\num{೧೮೯. } ‘ನಾನು’ ಅನಂತವಾಗಲಿ, ಇಲ್ಲವೆ ಅಳಿಸಿಹೋಗಲಿ}

ಶಂಕರಾಚಾರ್ಯರಿಗೆ ಒಬ್ಬ ಶಿಷ್ಯನಿದ್ದ. ಅವರಿಂದ ಯಾವ ಬೋಧನೆಯನ್ನೂ ಪಡೆದುಕೊಳ್ಳದೆ ಅವರ ಸೇವೆ ಮಾಡುತ್ತಿದ್ದ. ಒಂದು ಸಲ ಯಾರೋ ಅವರ ಹಿಂದೆ ಬರುತ್ತಿದ್ದರು. “ಯಾರದು?” ಎಂದು ಕೇಳಿದರು. ಆಗ ಶಿಷ್ಯ “ನಾನು” ಎಂದ. ಆಗ ಗುರುಗಳು “ನಾನು ಎಂಬುದು ನಿನಗೆ ಅಷ್ಟು ಪ್ರಿಯವಾಗಿದ್ದರೆ ಅದನ್ನು ಅನಂತವನ್ನಾಗಿ ಮಾಡು, ಇಲ್ಲವೇ ಅಲ್ಲಗಳೆ” ಎಂದರು.

\chapter{ಕೆಲವು ಸಲಹೆಗಳು}

\section{\num{೧೯೦. } ನೀನು ತಮಾಷೆಯನ್ನು ನೋಡುವುದಕ್ಕೆ ಇಚ್ಛೆಪಟ್ಟರೆ}

ಯಾವಾಗ ಒಬ್ಬನಿಗೆ ಬ್ರಹ್ಮಜ್ಞಾನವಾಗುವುದೊ, ಆಗ ಅವನು ದೇವರೇ ಎಲ್ಲವೂ ಆಗಿರುವನು ಎಂಬುದನ್ನು ನಿಜವಾಗಿ ಅನುಭವಿಸುವನು. ಅವನು ಯಾವುದನ್ನೂ ತ್ಯಜಿಸಬೇಕಾಗಿಲ್ಲ, ಯಾವುದನ್ನೂ ಸ್ವೀಕರಿಸಬೇಕಾಗಿಲ್ಲ. ಅವನು ಯಾರ ಮೇಲೂ ಕೋಪಿಸಿಕೊಳ್ಳಲಾರ.

ನಾನು ಒಂದು ದಿನ ಗಾಡಿಯಲ್ಲಿ ಹೋಗುತ್ತಿದ್ದೆ. ವರಾಂಡದಲ್ಲಿ ಇಬ್ಬರು ವೇಶ್ಯೆಯರು ನಿಂತಿರುವು ದನ್ನು ನೋಡಿದೆ. ಅವರು ಜಗ ಜ್ಜನನಿಯ ರೂಪಗಳಂತೆ ಕಂಡರು. ನಾನು ಅವರಿಗೆ ನಮಸ್ಕಾರ ಮಾಡಿದೆ.

ನನಗೆ ಮೊದಲು ಇಂತಹ ಪರಮ ಸ್ಥಿತಿ ಬಂದಾಗ, ನಾನು ಕಾಳಿಕಾಮಾತೆ ಯನ್ನು ಪೂಜಿಸುವುದಕ್ಕಾಗಲೀ ನೈವೇದ್ಯವನ್ನು ಕೊಡುವುದಕ್ಕಾಗಲೀ ಸಾಧ್ಯವಾಗಲಿಲ್ಲ. ಇದಕ್ಕಾಗಿ ದೇವ ಸ್ಥಾನದ ಅಧಿಕಾರಿಯು ನನ್ನನ್ನು ನಿಂದಿಸುತ್ತಿರುವನು ಎಂದು ಹಲಧಾರಿ ಮತ್ತು ಹೃದಯ ಹೇಳಿದರು. ನನಗೆ ಅದರಿಂದ ಸ್ವಲ್ಪವೂ ಕೋಪ ಬರಲಿಲ್ಲ.

ಒಬ್ಬ ಸಾಧು ಒಂದೂರಿಗೆ ಹೋಗಿ ಅಲ್ಲಿ ನೋಡಬೇಕಾದುದನ್ನೆಲ್ಲ ನೋಡಿದ. ಅವನಿಗೆ ಪರಿಚಿತವಾದ ಮತ್ತೊಬ್ಬ ಸಾಧುವನ್ನು ಕಂಡು, “ನೀನು ಸಂತೋಷದಿಂದ ಅಲೆಯುತ್ತಿರುವೆ. ನಿನ್ನ ಗಂಟುಮೂಟೆಗಳನ್ನೆಲ್ಲ ಯಾರೋ ಕಳ್ಳರು ತೆಗೆದುಕೊಂಡಿಲ್ಲ ತಾನೆ?” ಎಂದು ಕೇಳಿದನು. “ಇಲ್ಲ, ನಾನು ಮುಂಚೆ ಒಂದು ಕೋಣೆಯನ್ನು ಆರಿಸಿಕೊಂಡು ನನ್ನ ಸಾಮಾನುಗಳನ್ನೆಲ್ಲ ಅಲ್ಲಿಟ್ಟು, ಕೋಣೆಗೆ ಬೀಗ ಹಾಕಿ ಈಗ ಊರನ್ನು ಸಂತೋಷದಿಂದ ನೋಡುತ್ತಿರುವೆ” ಎಂದ. ಹಾಗೆಯೇ ಮೊದಲು ಬ್ರಹ್ಮಜ್ಞಾನವನ್ನು ಪಡೆದು ಅನಂತರ ಅಲೆದಾಡುತ್ತ ಅವನ ಲೀಲೆಯನ್ನು ಅನುಭವಿಸಿ. ಕಂಬವನ್ನು ಹಿಡಿದು ಸುತ್ತಿದರೆ ಬೀಳುವ ಸಾಧ್ಯತೆಯಿಲ್ಲ. ಅಂತೆಯೇ ದೇವರನ್ನು ತಿಳಿದನಂತರ ಯಾವ ಭಯವೂ ಇಲ್ಲ.


\section{\num{೧೯೧ } ಯಾವುದಕ್ಕೆ ಪ್ರಾರ್ಥಿಸಬೇಕು?}

ಭಗವಂತನನ್ನು ಪ್ರಾರ್ಥಿಸುವಾಗ ಅವನ ಪಾದಪದ್ಮಗಳಲ್ಲಿ ಭಕ್ತಿಯನ್ನು ಮಾತ್ರ ಕೇಳು. ರಾಮ ಅಹಲ್ಯೆಯನ್ನು ಶಾಪದಿಂದ ಪಾರು ಮಾಡಿದ ಮೇಲೆ, “ನನ್ನಿಂದ ಯಾವ ವರವನ್ನಾದರೂ ಕೇಳು” ಎಂದನು. ಅದಕ್ಕೆ ಅಹಲ್ಯೆ, “ರಾಮ, ನನಗೆ ಏನನ್ನಾದರೂ ಕೊಡಬೇಕೆಂದು ಇದ್ದರೆ, ನಾನು ಯಾವಾಗಲೂ ನಿನ್ನ ಪಾದಪದ್ಮಗಳನ್ನೇ ಚಿಂತಿಸುವಂತೆ ಮಾಡು, ನಾನು ಒಂದು ಹಂದಿಯ ದೇಹ ದಲ್ಲಿ ಜನ್ಮವೆತ್ತಬೇಕಾದರೂ ಚಿಂತೆಯಿಲ್ಲ” ಎಂದಳು.

ನಾನು ಭಗವತಿಯನ್ನು ಭಕ್ತಿಗಾಗಿ ಮಾತ್ರ ಯಾಚಿಸಿದೆ. ನಾನು ಪುಷ್ಪಗಳನ್ನು ಭಗವತಿಯ ಅಡಿದಾವರೆಗಳಿಗೆ ಇಟ್ಟು, “ತಾಯಿ ಇಲ್ಲಿ ನಿನ್ನ ಅಜ್ಞಾನವಿದೆ, ಇಲ್ಲಿ ನಿನ್ನ ಜ್ಞಾನವಿದೆ. ಇವೆರಡನ್ನೂ ತೆಗೆದುಕೊಂಡು ನನಗೆ ಶುದ್ಧಭಕ್ತಿಯನ್ನು ಕೊಡು. ಇಲ್ಲಿ ನಿನ್ನ ಪಾವಿತ್ರ್ಯವಿದೆ, ಇಲ್ಲಿ ನಿನ್ನ ಅಪಾವಿತ್ರ್ಯವಿದೆ. ಇವೆರಡನ್ನೂ ತೆಗೆದುಕೊಂಡು ನನಗೆ ಪರಮಭಕ್ತಿಯನ್ನು ನೀಡು. ಇಲ್ಲಿ ನಿನ್ನ ಪುಣ್ಯವಿದೆ, ಇಲ್ಲಿ ನಿನ್ನ ಪಾಪವಿದೆ. ಎರಡನ್ನೂ ತೆಗೆದುಕೊಂಡು ನನಗೆ ಪರಮಭಕ್ತಿಯನ್ನು ಕೊಡು. ಇಲ್ಲಿ ನಿನ್ನ ಧರ್ಮವಿದೆ, ಇಲ್ಲಿ ನಿನ್ನ ಅಧರ್ಮವಿದೆ. ಎರಡನ್ನೂ ತೆಗೆದುಕೊಂಡು ನನಗೆ ಶುದ್ಧಭಕ್ತಿಯನ್ನು ಕೊಡು.”


\section{\num{೧೯೨. } ನಾವು ಪ್ರಾರಬ್ಧ ಕರ್ಮದಿಂದ ಪಾರಾಗುವುದು ಹೇಗೆ?}

ಪ್ರಶ್ನೆ: ಮಹಾಶಯರೆ, ಹೇಗೆ ಪ್ರಾರಬ್ಧಕರ್ಮದಿಂದ ಪಾರಾಗುವುದು?

ಶ್ರೀರಾಮಕೃಷ್ಣ: ಪ್ರಾರಬ್ಧಕರ್ಮವನ್ನು ಎಲ್ಲರೂ ಸ್ವಲ್ಪ ಅನುಭವಿಸಲೇ ಬೇಕು. ಆದರೆ ಭಗವಂತನ ನಾಮದ ಬಲದಿಂದ ಅದರ ಬಹುಭಾಗ ನಾಶ ವಾಗಿಬಿಡುತ್ತದೆ. ಒಬ್ಬ ಹುಟ್ಟುಕುರುಡನಾಗಿದ್ದ. ಹಿಂದಿನ ಜನ್ಮದಲ್ಲಿ ಅವನು ಏನೋ ಮಾಡಿದ ಕರ್ಮಫಲವಿರಬೇಕು, ಅದಕ್ಕಾಗಿ ಅವನು ಇನ್ನೂ ಆರು ಜನ್ಮಗಳು ಅದನ್ನು ಅನುಭವಿಸಬೇಕಾಗಿತ್ತು. ಅವನು ಮೋಕ್ಷದಾಯಿನಿಯಾದ ಗಂಗಾನದಿಯಲ್ಲಿ ಸ್ನಾನ ಮಾಡಿದ. ಈ ಪುಣ್ಯ ಕಾರ್ಯ ಅವನ ಕುರುಡುತನವನ್ನು ಹೋಗಲಾಡಿಸಲಿಲ್ಲ. ಆದರೆ ಪುನರ್ಜನ್ಮದಿಂದ ಪಾರುಮಾಡಿತು.


\section{\num{೧೯೩. } ಹಾಗಾದರೆ ಏನು ಮಾರ್ಗ?}

ಪ್ರಪಂಚದಲ್ಲಿ ಇಷ್ಟು ಕಷ್ಟ ಇದ್ದರೆ ಪಾರಾಗುವುದು ಹೇಗೆ? ಎಂದು ನೀವು ಕೇಳಬಹುದು. ಇದಕ್ಕೆ ನಿರಂತರ ಸಾಧನೆಯೇ ಉಪಾಯ. ನಾನು ಕಾಮಾರ ಪುಕುರದಲ್ಲಿ ಅಕ್ಕಸಾಲಿಗ ಹೆಂಗಸರು ಯಾತದಿಂದ ಅವಲಕ್ಕಿ ಕುಟ್ಟುವುದನ್ನು ನೋಡಿರುವೆನು. ಹಾಗೆ ಕೆಲಸ ಮಾಡುವಾಗ ಯಾತ ತನ್ನ ಕೈಮೇಲೆ ಬೀಳದಂತೆ ನೋಡಿಕೊಳ್ಳುವಳು. ಏಕಕಾಲದಲ್ಲೇ ಮಗುವಿಗೆ ಹಾಲು ಕೊಡುವಳು, ಗಿರಾಕಿ ಗಳೊಡನೆ ಚೌಕಾಶಿ ಮಾಡುತ್ತಿರುತ್ತಾಳೆ. ಗಿರಾಕಿಗಳಿಗೆ ಹೇಳುತ್ತಾಳೆ, “ನೀನು ಹೋಗುವುದಕ್ಕೆ ಮುಂಚೆ ನಮಗೆ ಕೊಡಬೇಕಾಗಿರುವ ಬಾಕಿಯನ್ನು ಕೊಟ್ಟು ಬಿಡು” ಎಂದು.


\section{\num{೧೯೪. } ಯಾರು ಆನೆಯಲ್ಲಿ ದೇವರನ್ನು ನೋಡುವನೊ, ಅವನು ಮಾಹುತನಲ್ಲೂ ದೇವರನ್ನು ನೋಡಬೇಕು }

ಒಂದು ಕಾಡಿನಲ್ಲಿ ಒಬ್ಬ ಸಾಧು ಹಲವು ಶಿಷ್ಯರೊಂದಿಗೆ ಇದ್ದನು. ಒಂದು ಸಲ ಗುರುಗಳು, “ಎಲ್ಲರಲ್ಲಿಯೂ ದೇವರನ್ನು ನೋಡಿ, ಎಲ್ಲರಿಗೂ ನಮ ಸ್ಕರಿಸಿ” ಎಂದು ಬೋಧನೆ ಮಾಡಿದರು. ಒಬ್ಬ ಶಿಷ್ಯ ಸಮಿತ್ತನ್ನು ತರಲು ಕಾಡಿಗೆ ಹೋದ. ಆಗ ದೊಡ್ಡ ಗದ್ದಲವೆದ್ದಿತು. “ಈ ಸ್ಥಳದಿಂದ ಬೇಗ ಹೋಗಿ, ಮದ್ದಾನೆ ಓಡಿ ಬರುತ್ತಿದೆ” ಎಂಬ ಕೂಗು ಬಂತು. ಈ ಗುರುವಿನ ಶಿಷ್ಯನನ್ನು ಬಿಟ್ಟು ಎಲ್ಲರೂ ಅಲ್ಲಿಂದ ಓಡಿಹೋದರು. “ಆನೆಯು ಕೂಡ ದೇವರ ಮತ್ತೊಂದು ರೂಪ. ಹಾಗಿರುವಾಗ ಏತಕ್ಕೆ ಇಲ್ಲಿಂದ ಓಡಿಹೋಗಬೇಕು” ಎಂದು ಅವನು ಅಲ್ಲೇ ನಿಂತ, ಆನೆಗೆ ಬಗ್ಗಿ ನಮಸ್ಕರಿಸಿ, ಸ್ತುತಿಸಲಾರಂಭಿಸಿದ. ಮಾಹುತ “ದೂರ ಹೋಗಿ! ದೂರ ಹೋಗಿ!” ಎಂದು ಅರಚಿಕೊಳ್ಳುತ್ತಿದ್ದ. ಆದರೆ ಶಿಷ್ಯ ಆ ಸ್ಥಳವನ್ನು ಬಿಟ್ಟು ಹೋಗಲಿಲ್ಲ. ಆನೆ ಶಿಷ್ಯನನ್ನು ಹಿಡಿದುಕೊಂಡು ಸೊಂಡಲಿನಿಂದ ತಿರುಗಿಸಿ ಅವನನ್ನು ಎಸೆದು ಮುಂದೆ ಹೋಯಿತು. ಗಾಯವಾಗಿ ನೋವಿನಿಂದ ಶಿಷ್ಯ ಪ್ರಜ್ಞೆ ತಪ್ಪಿ ಒಂದು ಕಡೆ ಬಿದ್ದನು. ನಡೆದ ಸಮಾಚಾರವನ್ನು ಕೇಳಿ ಗುರುಗಳು ಮತ್ತು ಅವನ ಶಿಷ್ಯರು ಅಲ್ಲಿಗೆ ಬಂದರು. ಅವನನ್ನು ಎತ್ತಿಕೊಂಡು ಪರ್ಣ ಶಾಲೆಗೆ ಹೋದರು. ಕೆಲವು ಔಷಧಗಳ ಸೇವನೆಯಿಂದ ಅವನಿಗೆ ಪ್ರಜ್ಞೆ ಬಂತು. “ಆನೆ ಬರುತ್ತಿದೆ ಎಂಬುದನ್ನು ಕೇಳಿದರೂ ನೀನು ಏತಕ್ಕೆ ಓಡಿ ಹೋಗಲಿಲ್ಲ?” ಎಂದು ಯಾರೋ ಕೇಳಿದರು. ಶಿಷ್ಯ, “ಆದರೆ ನಮ್ಮ ಗುರುಗಳು ದೇವರೇ ಎಲ್ಲ ರೂಪಗಳನ್ನೂ ಧರಿಸಿರುವವನು, ಆನೆಯಲ್ಲಿಯೂ ದೇವರು ಇರುವನು ಎಂದು ಹೇಳಿದ್ದರಿಂದ ನಾನು ಆ ಸ್ಥಳವನ್ನು ಬಿಟ್ಟು ಹೋಗಲಿಲ್ಲ” ಎಂದನು. ಅದಕ್ಕೆ ಗುರುಗಳು ಹೇಳಿದರು, “ಹೌದು ಮಗು, ಆನೆ ದೇವರು ಬರುತ್ತಿದ್ದ. ಆದರೆ ಮಾಹುತನಲ್ಲಿರುವ ದೇವರು ಸ್ಥಳ ಬಿಟ್ಟು ಹೋಗು ಎನ್ನುತ್ತಿದ್ದನಲ್ಲ! ದೇವರು ಎಲ್ಲರಲ್ಲಿಯೂ ಇರುವುದರಿಂದ ಮಾಹುತದೇವರ ಮಾತನ್ನು ಏತಕ್ಕೆ ನೀನು ಕೇಳಲಿಲ್ಲ?” ಎಂದರು. ದೇವರು ಎಲ್ಲರಲ್ಲಿಯೂ ಇರುವನು. ಆದರೆ ಒಳ್ಳೆಯವರೊಡನೆ ಮಾತ್ರ ನಿಕಟವಾಗಿ ಬೆರೆಯಬಹುದು. ಕೆಟ್ಟವರಿಂದ ದೂರ ವಿರಬೇಕು. ದೇವರು ವ್ಯಾಘ್ರನಲ್ಲಿಯೂ ಇರುವನು ಎಂದು ಹುಲಿಯನ್ನು ಅಪ್ಪಿಕೊಳ್ಳಲು ಹೋಗಬಾರದು. ವ್ಯಾಘ್ರ ನಾರಾಯಣನಿಂದ ದೂರಹೋಗು ಎಂದು ಹೇಳುವವರಲ್ಲಿಯೂ ದೇವರಿರುವನು. ನೀವು ಏತಕ್ಕೆ ಅವರ ಮಾತನ್ನು ಕೇಳಬಾರದು?

ಭಗವಂತನೇ ಎಲ್ಲರಲ್ಲಿಯೂ ಇರುವನು ನಿಜ, ಒಳ್ಳೆಯವನಲ್ಲಿ ಕೆಟ್ಟ ವನಲ್ಲಿ, ಧರ್ಮಾತ್ಮನಲ್ಲಿ, ಅಧರ್ಮಿಯಲ್ಲಿ ಎಲ್ಲರಲ್ಲಿಯೂ ಅಂತರ್ಯಾಮಿ ಯಾಗಿ ಇರುವನು. ಆದರೆ ಕೆಟ್ಟವರ ಸಹವಾಸದಿಂದ ದೂರವಿರಬೇಕು. ಅವ ರೊಂದಿಗೆ ನಿಕಟ ಸಂಬಂಧವನ್ನು ಬಿಡಬೇಕು. ಕೆಲವರ ಹತ್ತಿರ ಮಾತನಾಡ ಬಹುದು. ಮತ್ತೆ ಕೆಲವರ ಹತ್ತಿರ, ಮಾತನಾಡುವುದಕ್ಕಿಂತ ದೂರವಿರುವುದೇ ಒಳಿತು.


\section{\num{೧೯೫ } ಡ್ಯಾಮ್ ಡ್ಯಾಮ್​}

ಒಂದು ಸಲ ಕ್ಷೌರಿಕನು ಒಬ್ಬ ಮನುಷ್ಯನಿಗೆ ಕ್ಷೌರ ಮಾಡುತ್ತಿದ್ದ. ಕ್ಷೌರ ಮಾಡಿಸಿಕೊಳ್ಳುತ್ತಿದ್ದವನಿಗೆ ಕತ್ತಿಯಿಂದ ಸ್ವಲ್ಪ ಗಾಯ ವಾಯಿತು. ತಕ್ಷಣವೆ ಅವನು ‘ಡ್ಯಾಮ್​’ ಎಂದನು. ಆದರೆ ಕ್ಷೌರಿಕನಿಗೆ ಅದರ ಅರ್ಥ ಗೊತ್ತಾಗ ಲಿಲ್ಲ. ಅವನು ಕ್ಷೌರದ ಕತ್ತಿ ಮತ್ತು ಇತರ ವಸ್ತುಗಳನ್ನು ಕೆಳಗಿಟ್ಟ. ಅಂಗಿಯ ಕೈಯನ್ನು ಮಡಚಿ ಕೊಳ್ಳುತ್ತ “ನೀವು, ‘ಡ್ಯಾಮ್​’ ಎಂದಿರಲ್ಲ, ಅದಕ್ಕೇನು ಅರ್ಥ ಹೇಳಿ” ಎಂದ. “ಮೂಢನಂತಿರ ಬೇಡ. ಸುಮ್ಮನೆ ಕ್ಷೌರ ಮಾಡು. ಅದಕ್ಕೆ ಏನೂ ಅರ್ಥವಿಲ್ಲ. ಸ್ವಲ್ಪ ಜೋಪಾನವಾಗಿ ಕ್ಷೌರ ಮಾಡು” ಎಂದು ಗಿರಾಕಿ ನುಡಿದ. ಆದರೆ ಅಷ್ಟಕ್ಕೆ ಬಿಡುವ ಅಸಾಮಿಯಲ್ಲ ಕ್ಷೌರಿಕ. ಅವನು ಹೇಳಿದ, “‘ಡ್ಯಾಮ್​’ ಪದಕ್ಕೇನಾದರೂ ಒಳ್ಳೆಯ ಅರ್ಥ ಇದ್ದರೆ ನಾನು ಡ್ಯಾಮ್, ನಮ್ಮ ಅಪ್ಪ ಡ್ಯಾಮ್, ನನ್ನ ಅಜ್ಜ ಡ್ಯಾಮ್, ನಮ್ಮ ಪೂರ್ವಿಕರೆಲ್ಲ ಡ್ಯಾಮ್. ಅದಕ್ಕೇನಾದ್ರೂ ಕೆಟ್ಟ ಅರ್ಥ ಇದ್ರೆ, ನೀನು ಡ್ಯಾಮ್, ನಿಮ್ಮಪ್ಪ ಡ್ಯಾಮ್, ನಿಮ್ಮ ಪೂರ್ವಿಕರೆಲ್ಲ ಡ್ಯಾಮ್, ಅವರು ಬರೀ ಡ್ಯಾಮ್ ಮಾತ್ರವಲ್ಲ ಡ್ಯಾಮ್ ಡ್ಯಾಮ್ ಡ್ಯಾಮ್​” ಎಂದ. ಇತರ ರೊಡನೆ ಇದ್ದಾಗ ಅವರಿಗೆ ಅರ್ಥವಾಗದ ಮಾತಿನಿಂದ ನೋವುಂಟಾಗದಂತೆ ಜಾಗರೂಕನಾಗಿರಬೇಕು.


\section{\num{೧೯೬ } ಇತರರ ತಪ್ಪನ್ನು ಮೆಲುಕುತ್ತಿದ್ದರೆ ನೀನೇ ಆ ತಪ್ಪನ್ನು ಮಾಡುವೆ}

ಸಂನ್ಯಾಸಿಯೊಬ್ಬ ದೇವಸ್ಥಾನದ ಹತ್ತಿರ ಇದ್ದ. ಮನೆಯೆದುರಿಗೆ ಒಬ್ಬ ವೇಶ್ಯೆ ಇದ್ದಳು. ವ್ಯಭಿಚಾರಿಣಿಯ ಮನೆಗೆ ಅಷ್ಟೊಂದು ಜನ ಹೋಗುತ್ತಿದ್ದುದನ್ನು ನೋಡಿ, ಒಂದು ದಿನ ವೇಶ್ಯೆಯನ್ನು ಕರೆದು, “ನೀನು ಒಬ್ಬ ಮಹಾಪಾಪಿ, ಹಗಲು ರಾತ್ರಿ ಪಾಪಕೃತ್ಯಗಳನ್ನು ಮಾಡುತ್ತಿರುವೆ. ನಿನಗೆ ಅನಂತರ ಎಂತಹ ನರಕ ಕಾದಿದೆಯೊ” ಎಂದು ಮುನ್ನೆಚ್ಚರಿಕೆಯನ್ನು ಕೊಟ್ಟ. ವೇಶ್ಯೆ ತನ್ನ ಪಾಪಕೃತ್ಯಗಳಿಗೆ ಹೃತ್ಪೂರ್ವಕ ಪರಿತಾಪವನ್ನು ಪಟ್ಟಳು. ಪಶ್ಚಾತ್ತಾಪದಿಂದ ದೇವರಿಗೆ ನನ್ನನ್ನು ಕ್ಷಮಿಸಿ ಎಂದು ಪ್ರಾರ್ಥನೆ ಮಾಡಿಕೊಂಡಳು. ವೇಶ್ಯೆಗೆ ಅದೇ ಅವಳ ವೃತ್ತಿಯಾದುದರಿಂದ, ಬೇರೆ ವೃತ್ತಿಯಿಂದ ಜೀವನೋಪಾಯವನ್ನು ಮಾಡಲು ಆಗಲಿಲ್ಲ. ಪಶ್ಚಾತ್ತಾಪ ದಿಂದ ಭಗವಂತನಲ್ಲಿ ಮತ್ತೆ ಮತ್ತೆ ಮೊರೆ ಇಟ್ಟಳು. ಸಂನ್ಯಾಸಿ ನೋಡಿದ, ತಾನು ಹೇಳಿದ ಬುದ್ಧಿವಾದ ಫಲಕಾರಿಯಾಗಲಿಲ್ಲ. ಸಂನ್ಯಾಸಿ ಅವಳ ಮನೆಗೆ ಎಷ್ಟು ಜನ ಹೋಗುತ್ತಾರೋ ಲೆಕ್ಕಹಾಕೋಣ ಎಂದು ಯೋಚಿಸಿದ. ಅಂದಿ ನಿಂದ, ಯಾರಾದರೂ ವೇಶ್ಯೆ ಮನೆಗೆ ಹೋದರೆ ಒಂದು ಕಲ್ಲನ್ನು ಒಂದು ಕಡೆ ಇಡುತ್ತ ಬಂದ. ಸ್ವಲ್ಪಕಾಲ ವಾದ ಮೇಲೆ ದೊಡ್ಡ ಒಂದು ಕಲ್ಲಿನ ರಾಶಿಯೆ ಅವನೆದುರಿಗೆ ಇತ್ತು. ಅಲ್ಲಿರುವ ಕಲ್ಲುರಾಶಿಯನ್ನು ವೇಶ್ಯೆಗೆ ತೋರಿ, “ಇಲ್ಲಿ ಇರುವ ಕಲ್ಲಿನ ರಾಶಿ ನಿನಗೆ ಕಾಣುವುದಿಲ್ಲವೆ? ಇಲ್ಲಿರುವ ಪ್ರತಿಯೊಂದು ಕಲ್ಲೂ ನೀನು ಮಾಡಿದ ತಪ್ಪಿಗೆ ಸಾಕ್ಷಿಯಾಗಿದೆ. ನಾನು ಕಳೆದಸಾರಿ ನಿನಗೆ ಹೇಳಿದೆ. ಇಂತಹ ಕೃತ್ಯವನ್ನು ಮಾಡಬೇಡ ಎಂದು. ಈಗಲೂ ಕೂಡ ಮುಂದಾ ಗುವುದನ್ನು ಯೋಚಿಸು” ಎಂದು ಮುನ್ನೆ ಚ್ಚರಿಕೆ ಕೊಟ್ಟನು. ತಾನು ಮಾಡಿದ ಪಾಪರಾಶಿಗೆ ತಾನೇ ನಡುಗುತ್ತ ಹೃತ್ಪೂರ್ವಕವಾಗಿ ಭಗವಂತನನ್ನು, “ದೇವರೇ, ನನ್ನನ್ನು ಈ ಪಾಪಕೃತ್ಯದಿಂದ ಪಾರು ಮಾಡಲಾರೆಯ?” ಎಂದು ಕಂಬನಿ ತುಂಬಿ ಬೇಡಿಕೊಂಡಳು. ಭಗವಂತ ಈ ಪ್ರಾರ್ಥನೆಯನ್ನು ಕೇಳಿದ. ಅಂದಿನ ರಾತ್ರಿಯೇ ಆಕೆ ಮೃತ್ಯುವಶಳಾದಳು. ಭಗವಂತನ ಇಚ್ಛೆಯಂತೆ ಸಂನ್ಯಾಸಿ ಕೂಡ ಅಂದೇ ತೀರಿಕೊಂಡನು. ವಿಷ್ಣುದೂತರು ಬಂದು ಪಶ್ಚಾತ್ತಾಪಪಡು ತ್ತಿದ್ದ ಆ ವೇಶ್ಯೆಯ ಜೀವವನ್ನು ಭಗವಂತನೆಡೆಗೆ ಕೊಂಡೊಯ್ದರು. ಆದರೆ ಯಮನ ಭಟರು ಬಂದು ಸಂನ್ಯಾಸಿಯನ್ನು ಯಮಪಾಶದಿಂದ ಬಿಗಿದು ನರಕಕ್ಕೆ ಒಯ್ದರು. ಆ ವೇಶ್ಯೆಯ ಅದೃಷ್ಟವನ್ನು ಕಂಡು ಸಂನ್ಯಾಸಿ ಜೋರಾಗಿ ಕೂಗಿ ದನು: “ಇದೇನೆ ದೇವರ ನ್ಯಾಯ? ನಾನು ಜೀವಾವಧಿ ಪ್ರಾರ್ಥನೆಯಲ್ಲಿ ಕಳೆದೆ. ಬಡತನದಲ್ಲಿ ಕಳೆದೆ. ನನ್ನನ್ನು ನರಕಕ್ಕೆ ಒಯ್ಯುತ್ತಿರುವರು. ಆ ವೇಶ್ಯೆಯಾದರೂ ಹುಟ್ಟಿದಂದಿನಿಂದ ಪಾಪಕೃತ್ಯವನ್ನು ಮಾಡುತ್ತಿ ದ್ದವಳು. ಈಗ ಸ್ವರ್ಗಕ್ಕೆ ಹೋಗುತ್ತಿರುವಳು.” ಇದನ್ನು ಕೇಳಿ ವಿಷ್ಣುವಿನ ದೂತರು ಹೇಳಿದರು: “ಭಗವಂತ ಯಾವಾಗಲೂ ಸರಿಯಾಗಿರುವುದನ್ನೇ ಮಾಡುವನು. ನೀನು ಏನನ್ನು ಬಿತ್ತಿದ್ದೆಯೋ ಅದಕ್ಕೆ ಪ್ರಾಯಶ್ಚಿತ್ತ ಇದು. ನಿನ್ನ ಧಾರ್ಮಿಕತೆ ಕೇವಲ ಬಾಹ್ಯಾಡಂಬರವಾಗಿತ್ತು. ನನ್ನ ಸಮಾನ ಪುಣ್ಯಾತ್ಮರಿಲ್ಲ ಎಂದು ಭಾವಿಸಿದೆ. ಗೌರವ ಕೀರ್ತಿಗೆ ನಿನ್ನ ಬಾಳನ್ನು ಮಾರಿದೆ. ಎಂದಿಗೂ ನೀನು ಭಗವಂತನನ್ನು ಹೃತ್ಪೂರ್ವಕ ಪ್ರಾರ್ಥಿಸಲಿಲ್ಲ. ವೇಶ್ಯೆಯಾದರೋ ಹಗಲು ರಾತ್ರಿ ಭಗವಂತ ನಲ್ಲಿ ಮೊರೆಯಿಟ್ಟಳು. ಅವಳ ದೇಹ ಕೆಟ್ಟದನ್ನು ಮಾಡಿತು. ನಿನ್ನ ಶರೀರಕ್ಕೆ ಮತ್ತು ಅವಳ ಶರೀರಕ್ಕೆ ಸಿಗುತ್ತಿರುವ ಮನ್ನಣೆಯನ್ನು ನೋಡು. ನೀನು ದೇಹದ ಮೂಲಕ ಯಾವ ಕೆಟ್ಟದನ್ನೂ ಮಾಡದೆ ಇದ್ದುದರಿಂದ ನಿನ್ನ ಶವವನ್ನು ಹೂವಿನ ರಾಶಿಯಲ್ಲಿ ಮುಚ್ಚಿ ತಾಳಮೇಳದೊಂದಿಗೆ ಗಂಗಾನದಿಗೆ ಒಯ್ಯುತ್ತಿರುವರು. ವೇಶ್ಯೆಯ ದೇಹ ಕೆಟ್ಟದ್ದನ್ನು ಮಾಡಿದ್ದರಿಂದ ನಾಯಿ ನರಿಗಳು ಅವಳ ಶವವನ್ನು ಕಿತ್ತು ತಿನ್ನುತ್ತಿವೆ. ಆದರೆ ಅವಳ ಹೃದಯ ಪರಿಶುದ್ಧವಾಗಿತ್ತು. ಅದಕ್ಕೇ ಅವಳು ದೇವ ಲೋಕಕ್ಕೆ ಹೋಗುತ್ತಿರುವಳು. ನೀನು ಯಾವಾಗಲೂ ಮತ್ತೊಬ್ಬರ ತಪ್ಪನ್ನು ಎಣಿಸುವುದರಲ್ಲಿ ನಿರತನಾಗಿದ್ದುದರಿಂದ ನರಕಕ್ಕೆ ಹೋಗುತ್ತಿರುವೆ. ನೀನೇ ನಿಜವಾದ ವೇಶ್ಯೆ, ಅವಳಲ್ಲ.”

\chapter{ಆಳವಾದ ಸತ್ಯ}

\section{\num{೧೯೭. } ಅಲ್ಲಿಲ್ಲ, ಇಲ್ಲಿ}

ಒಂದು ಸಲ ಒಂದು ಹಡಗಿನ ಪಟಸ್ತಂಭದ ಮೇಲೆ ಒಂದು\\ಹಕ್ಕಿ ಕುಳಿತಿತ್ತು. ಹಡಗು ಯಾವಾಗ ಸಮುದ್ರದಲ್ಲಿ ಯಾನ ಹೊರಟಿತೋ ಅದಕ್ಕೆ ಗೊತ್ತಾಗ ಲಿಲ್ಲ. ಯಾವಾಗ ಅದಕ್ಕೆ ಅನಂತ ಸಾಗರದ ಅರಿವಾಯಿತೊ, ಪಟ ಸ್ತಂಭವನ್ನು ಬಿಟ್ಟು ದಡವನ್ನು ಹುಡುಕಿಕೊಂಡು ಉತ್ತರದ ಕಡೆ ಹಾರಿತು. ಆದರೆ ಕರೆಯ ಚಿಹ್ನೆಯು ಇಲ್ಲದೆ ಹಿಂತಿರುಗಿ ಬಂತು. ಸ್ವಲ್ಪಕಾಲ ಸ್ತಂಭದ ಮೇಲಿದ್ದು ದಕ್ಷಿಣದಿಕ್ಕಿಗೆ ಹೋಯಿತು. ಅಲ್ಲಿಯೂ ಕೂಡ ಅಸೀಮವಾದ ಸಾಗರವೇ ಇತ್ತು. ಪುನಃ ಸುಸ್ತಾಗಿ ಸ್ತಂಭಕ್ಕೆ ಬಂತು. ಸ್ವಲ್ಪಕಾಲ ಅಲ್ಲಿದ್ದು ಪೂರ್ವ ಮತ್ತು ಪಶ್ಚಿಮ ದಿಕ್ಕುಗಳಿಗೆ ಹೋಯಿತು. ಎಲ್ಲಿಯೂ ದಡದ ಸುಳಿವೇ ಇಲ್ಲದೆ ಅದು ಪಟಸ್ತಂಭದ ಮೇಲೆ ಸುಮ್ಮನೆ ಕುಳಿತುಕೊಂಡಿತು.

ಯಾವುದನ್ನು ಮನುಷ್ಯ ಅರಸಿಕೊಂಡು ಹೋಗುವನೊ ಅದು ಅವನ ಬಳಿಯೇ ಇದೆ. ಆದರೂ ಅವನು ಎಲ್ಲೆಲ್ಲೋ ಅಲೆಯುವನು. ದೇವರು ಅಲ್ಲಿರುವನು ಎಂದು ಭಾವಿಸಿದರೆ ಅವನು ಅಜ್ಞಾನಿ. ದೇವರು ಇಲ್ಲೇ ಇರುವನು ಎಂದು ಅರಿತರೆ ಅವನೇ ಜ್ಞಾನಿ.


\section{\num{೧೯೮. } ನೀನರಸುವುದು ನಿನ್ನಲ್ಲಿಯೇ ಇದೆ}

ಒಬ್ಬ ತಂಬಾಕನ್ನು ಸೇದಬೇಕೆಂದು ಆಶಿಸಿದನು. ಅವನು ನೆರೆ ಮನೆಗೆ ಸ್ವಲ್ಪ ಬೆಂಕಿ ಕೇಳಲು ಹೋದ. ಅದು ಅವೇಳೆ, ರಾತ್ರಿ. ಮನೆಯಲ್ಲಿ ಎಲ್ಲರೂ ಮಲಗಿದ್ದರು. ಅವನು ಬಾಗಿಲನ್ನು ಹಲವು ವೇಳೆ ತಟ್ಟಿದ ಮೇಲೆ ಮನೆಯವನೊಬ್ಬನು ಬಾಗಿಲನ್ನು ತೆಗೆದನು. “ಏನು ಇಷ್ಟು ಹೊತ್ತಿನಲ್ಲಿ ಬಂದಿರಲ್ಲ?” ಎಂದು ಕೇಳಿದ. “ನೀನು ಅದನ್ನು ಊಹಿ ಸುವುದಕ್ಕೆ ಆಗಲಿಲ್ಲವೇ? ತಂಬಾಕನ್ನು ಸೇದುವುದು ನನಗೆ ಬಹಳ ಆಸೆ. ಅದಕ್ಕಾಗಿ ಸ್ವಲ್ಪ ಬೆಂಕಿಗಾಗಿ ಬಂದೆ.” ನೆರೆಯವನು, “ನೀನು ಒಳ್ಳೆಯ ಮನುಷ್ಯ. ಅದಕ್ಕಾಗಿ ರಾತ್ರಿ ಮನೆಯವರನ್ನೆಲ್ಲ ಎಬ್ಬಿಸಿದೆಯಲ್ಲ! ಉರಿಯುತ್ತಿರುವ ದೀಪ ನಿನ್ನ ಹತ್ತಿರವೆ ಇದೆಯಲ್ಲಯ್ಯ” ಎಂದ. ನಾವು ಯಾವುದನ್ನು ಅರಸುತ್ತಿರು ವೆವೋ ಅದು ನಮ್ಮ ಹತ್ತಿರವೇ ಇದೆ. ಆದರೂ ಅದಕ್ಕಾಗಿ ಎಲ್ಲೆಲ್ಲೋ ಹುಡುಕುವೆವು.


\section{\num{೧೯೯. } ಭಗವದ್ರಾಜ್ಯವನ್ನು ಪ್ರವೇಶಿಸುವುದು ಹೇಗೆ?}

ದುರಹಂಕಾರ ನಮ್ಮಲ್ಲಿರುವುದ ರಿಂದ ನಾವು ದೇವರನ್ನು ನೋಡ ಲಾರೆವು. ಭಗವಂತನ ಮನೆ ಎದುರಿಗೆ ಅಹಂಕಾರದ ಗೂಟ ಇದೆ. ಅಹಂಕಾರದ ಗೂಟವನ್ನು ದಾಟಿ ದಲ್ಲದೆ ಭಗವಂತನ ಬಳಿಗೆ ಹೋಗಲು ಸಾಧ್ಯವಿಲ್ಲ.

ಒಂದೂರಿನಲ್ಲಿ ಒಬ್ಬ ಇದ್ದ. ಅವನಿಗೆ ಭೂತ ವಶವಾಗಿತ್ತು. ಒಂದು ದಿನ ಅವನ ಅಪ್ಪಣೆ ಪ್ರಕಾರ ಒಂದು ಭೂತ ಬಂತು. “ಈಗ ಹೇಳು, ನಿನಗೇನು ಬೇಕು? ನೀನು ನನಗೆ ಕೆಲಸವನ್ನು ಕೊಡದಿದ್ದರೆ ನಾನು ನಿನ್ನ ಕತ್ತನ್ನು ಮುರಿದು ಹಾಕುವೆ” ಎಂದಿತು. ಮನುಷ್ಯನಿಗೆ ಹಲವಾರು ಆಸೆಗಳಿದ್ದವು. ಆ ಭೂತ ಇವನು ಏನೇ ಹೇಳಿದರೂ ಅದನ್ನೆಲ್ಲ ಮಾಡಿಬಿಡುತ್ತಿತ್ತು ಕ್ಷಣಾರ್ಧದಲ್ಲಿ. ಅನಂತರ ಭೂತಕ್ಕೆ ಏನು ಕೆಲಸ ಹೇಳಬೇಕೊ ತೋಚಲಿಲ್ಲ. “ನೋಡು, ಈಗ ನಾನು ನಿನ್ನ ಕತ್ತನ್ನು ಮುರಿಯುವೆ” ಎಂದಿತು ಭೂತ. ಸ್ವಲ್ಪ ತಾಳು, ನಾನು ಬರು ತ್ತೇನೆ–ಎಂದು ಭೂತವನ್ನು ವಶಮಾಡಿಕೊಂಡವನು ತನ್ನ ಗುರುಗಳೆಡೆಗೆ ಓಡಿ ಹೋಗಿ, “ಸ್ವಾಮಿಗಳೆ, ನಾನು ಈಗ ಬಹಳ ಅಪಾಯದಲ್ಲಿ ಬಿದ್ದಿರುವೆ. ಈಗ ಏನು ಮಾಡಲಿ ಹೇಳಿ?” ಎಂದು ತನ್ನ ತೊಂದರೆಯನ್ನು ವಿವರಿಸಿದ. ಗುರು, “ಬಾಗಿರುವ ಕೂದಲನ್ನು ನೇರಮಾಡು ಎಂದು ಅದಕ್ಕೆ ಹೇಳು” ಎಂದನು. ಅದರಂತೆ ಭೂತ ಸುರುಳಿ ಸುತ್ತಿದ ಕೂದಲನ್ನು ನೇರಮಾಡಲು ಬೇಕಾದಷ್ಟು ಪ್ರಯತ್ನಪಟ್ಟಿತು. ಆದರೆ ಆ ಸುರುಳಿ ಸುತ್ತಿದ ಕೂದಲನ್ನು ನೇರವಾಗಿ ಹೇಗೆ ಮಾಡುವುದು? ಎಷ್ಟು ಪ್ರಯತ್ನ ಮಾಡಿದರೂ ಕೂದಲು ಸುರುಳಿ ಹಾಗೇ ಇತ್ತು.

ಅದರಂತೆಯೇ ಅಹಂಕಾರ ಹೋದಂತೆ ಕಾಣುವುದು. ಮರುಕ್ಷಣವೇ ಮತ್ತೆ ಕಾಣಿಸಿಕೊಳ್ಳುವುದು. ಅಹಂಕಾರವನ್ನು ಬಿಟ್ಟ ಹೊರತು ಭಗವಂತನ ಕೃಪೆ ಬರುವುದಿಲ್ಲ.


\section{\num{೨೦೦. } ಈಗ ಕೆಲಸ ಮಾಡುವುದಕ್ಕೆ ಸಮಯ ಬಂದಿದೆ}

ನನ್ನ ಸ್ವಭಾವ ಏನು ಎಂಬುದು ನಿಮಗೆ ಗೊತ್ತಿದೆಯೆ? ಶಾಸ್ತ್ರಾದಿಗಳು ಭಗವಂತನ ಕಡೆಗೆ ಹೇಗೆ ಹೋಗಬೇಕು ಎಂಬುದನ್ನು ವಿವರಿಸುತ್ತವೆ. ಅದನ್ನು ತಿಳಿದಾದ ಮೇಲೆ ಶಾಸ್ತ್ರಾದಿಗಳಿಂದ ಏನು ಪ್ರಯೋಜನ? ಆಗ ಬರುವುದು ಸಾಧನೆಗೆ ಸಮಯ. ಒಬ್ಬನಿಗೆ ಊರಿನಿಂದ ಒಂದು ಕಾಗದ ಬಂತು. ಅದರಲ್ಲಿ, ಅವನ ಬಂಧುಗಳಿಗೆ ಕೆಲವು ಸಾಮಾನನ್ನು ತೆಗೆದುಕೊಂಡು ಬನ್ನಿ ಎಂದು ಬರೆದಿತ್ತು. ಸಾಮಾನುಗಳ ಹೆಸರನ್ನು ಆ ಕಾಗದದಲ್ಲಿ ಬರೆದಿದ್ದರು. ಆ ಸಾಮಾನುಗಳನ್ನು ಖರೀದಿಸೋಣ ಎನ್ನುವಾಗ ಆ ಕಾಗದವೇ ಕಳೆದು ಹೋಗಿತ್ತು. ಆ ಕಾಗದಕ್ಕಾಗಿ ಅವನು ಹುಡುಕಾಡಿದ. ಇತರರೂ ಸಹಾಯ ಮಾಡಿದರು. ಕಾಗದ ಸಿಕ್ಕಿದಾಗ ಅವನ ಸಂತೋಷಕ್ಕೆ ಪಾರವೇ ಇರಲಿಲ್ಲ. ತುಂಬಾ ಆಸಕ್ತಿಯಿಂದ ಕಾಗದವನ್ನು ಓದಿದ. ಅದರಲ್ಲಿ ಐದು ಸೇರು ಮಿಠಾಯಿ, ಒಂದು ಪಂಚೆ ಮತ್ತು ಇತರ ವಸ್ತುಗಳನ್ನು ಅವರು ಬರೆದಿದ್ದರು. ಅದನ್ನು ಓದಿದ ಮೇಲೆ ಕಾಗದದ ಅಗತ್ಯ ಅವನಿಗೆ ಇರಲಿಲ್ಲ. ಅದರ ಕಾರ್ಯ ಮುಗಿದುಹೋಗಿತ್ತು. ಕಾಗದ ವನ್ನು ಅವನು ಆಚೆಗೆ ಎಸೆದು ಸಾಮಾನನ್ನು ತರಲು ಪೇಟೆಗೆ ಹೋದ.

ಶಾಸ್ತ್ರದಲ್ಲಿ ಹೇಗೆ ಭಗವಂತನನ್ನು ಪಡೆಯಬಹುದು ಎಂಬುದನ್ನು ಹೇಳಿದೆ. ಅದನ್ನು ತಿಳಿದ ಮೇಲೆ ಅದರಲ್ಲಿ ಹೇಳಿರುವಂತೆ ಒಬ್ಬ ಸಾಧನೆ ಮಾಡಬೇಕು. ಆಗ ಮಾತ್ರ ನೀನು ಗುರಿಯನ್ನು ತಲುಪಲು ಸಾಧ್ಯ.


\section{\num{೨೦೧. } ಅಲ್ಪಜ್ಞಾನದಿಂದ ಮತಾಂಧತೆ ಬರುವುದು}

ನಾಲ್ಕು ಜನ ಕುರುಡರು ಆನೆ ಹೇಗಿದೆ ಎಂಬುದನ್ನು ಕಂಡುಹಿಡಿಯಲು ಹೋದರು. ಒಬ್ಬ ಆನೆಯ ಕಾಲನ್ನು ಮುಟ್ಟಿದ. ಅದೊಂದು ಕಂಬದಂತೆ ಇದೆ ಎಂದ. ಎರಡನೆಯವನು ಅದರ ಸೊಂಡಿಲನ್ನು ಮುಟ್ಟಿದ. ಅವನು ಅದು ದೊಡ್ಡ ಗದೆಯಂತೆ ಇದೆ ಎಂದ. ಮೂರನೆಯವನು ಅದರ ಹೊಟ್ಟೆಯನ್ನು ಮುಟ್ಟಿದ. ಅದೊಂದು ದೊಡ್ಡ ಪೀಪಾಯಿಯಂತಿದೆ ಎಂದ. ನಾಲ್ಕನೆಯವನು ಅದರ ಕಿವಿಯನ್ನು ಮುಟ್ಟಿದ. ಅದೊಂದು ಕೇರುವ ಮೊರದಂತೆ ಇದೆ ಎಂದ. ನಾಲ್ಕು ಜನರೂ ಆನೆ ಹಾಗಿದೆ ಹೀಗಿದೆ ಎಂದು ವಾದ ಮಾಡುತ್ತಿದ್ದರು. ದಾರಿ ಯಲ್ಲಿ ಹೋಗುತ್ತಿದ್ದವನು, “ಏತಕ್ಕೆ ಜಗಳವಾಡುತ್ತಿರುವಿರಿ?” ಎಂದು ಕೇಳಿದ. ಅವರು ತಮಗೆ ತೋಚಿದ ಎಲ್ಲವನ್ನು ಹೇಳಿ, ಈಗ ನಿಮ್ಮ ನಿರ್ಣಯವನ್ನು ಕೊಡಿ ಎಂದು ಕೇಳಿದರು. “ನಿಮ್ಮಲ್ಲಿ ಪೂರ್ತಿ ಆನೆಯನ್ನು ಯಾರೂ ಕಂಡಿಲ್ಲ. ಆನೆ ಕಂಬದಂತೆ ಇಲ್ಲ. ಅದರ ಕಾಲುಗಳು ಕಂಬದಂತೆ ಇವೆ. ಅದು ಮೊರ ದಂತೆ ಇಲ್ಲ. ಅದರ ಕಿವಿ ಮೊರದಂತೆ ಇದೆ. ಅದು ಒಂದು ಗದೆಯಂತೆ ಇಲ್ಲ. ಅದರ ಸೊಂಡಿಲು ಗದೆಯಂತೆ ಇದೆ. ಅದು ಪೀಪಾಯಿಯಂತೆ ಇಲ್ಲ. ಅದರ ಹೊಟ್ಟೆ ಪೀಪಾಯಿಯಂತಿದೆ. ಇವುಗಳೆಲ್ಲ ಸೇರಿ ಆನೆಯಾಗಿದೆ. ಕಾಲು, ಕಿವಿ, ಹೊಟ್ಟೆ, ಸೊಂಡಿಲು ಎಲ್ಲ ಅದರ ಭಾಗಗಳು” ಎಂದನು. ಇದರಂತೆಯೆ ಯಾರು ದೇವರ ವಿಷಯದಲ್ಲಿ ವಾದ ಮಾಡುತ್ತಾರೊ ಅವರಿಗೆ ದೇವರ ಯಾವುದೋ ಒಂದು ಅಂಶ ಮಾತ್ರ ತಿಳಿದಿರುತ್ತದೆ.


\section{\num{೨೦೨. } ಮತಾಂಧತೆಗೆ ಮತ್ತೊಂದು ಹೆಸರೇ ಅಜ್ಞಾನ}

ಒಂದು ಕಪ್ಪೆ ಬಾವಿಯಲ್ಲಿತ್ತು. ಅದು ಅಲ್ಲೇ ಬಹಳ ಕಾಲ\\ದಿಂದ ವಾಸವಾಗಿತ್ತು. ಅದು ಅಲ್ಲೇ ಹುಟ್ಟಿಬೆಳೆದ ಕಪ್ಪೆ. ಅದು\\ಒಂದು ಪುಟ್ಟ ಕಪ್ಪೆ. ಒಂದು ಸಲ ಸಮುದ್ರದಿಂದ ಒಂದು\\ಕಪ್ಪೆ ಅಲ್ಲಿಗೆ ಬಂತು. ಬಾವಿ ಯಲ್ಲಿದ್ದ ಕಪ್ಪೆ ಸಮುದ್ರದ ಕಪ್ಪೆ ಯನ್ನು, “ನೀನು ಎಲ್ಲಿಂದ ಬಂದೆ?” ಎಂದು ಕೇಳಿತು. ಸಮುದ್ರದ ಕಪ್ಪೆ, “ಸಮುದ್ರ ದಿಂದ” ಎಂದಿತು. ಬಾವಿಯ ಕಪ್ಪೆ “ಸಮುದ್ರ ಎಷ್ಟು ದೊಡ್ಡದು?” ಎಂದು ಕೇಳಿತು. ಸಮುದ್ರದ ಕಪ್ಪೆ ಹೇಳಿತು, “ಸಮುದ್ರ ತುಂಬಾ ದೊಡ್ಡದು” ಎಂದು. ಬಾವಿಯ ಕಪ್ಪೆ ತನ್ನ ಕಾಲುಗಳನ್ನು ಅಗಲ ವಾಗಿ ಚಾಚಿ, “ಇಷ್ಟು ದೊಡ್ಡ ದಿದೆಯೆ ನಿನ್ನ ಸಮುದ್ರ?” ಎಂದು ಕೇಳಿತು. ಸಮುದ್ರದ ಕಪ್ಪೆ ಹೇಳಿತು, “ಇಲ್ಲ, ಇನ್ನೂ ತುಂಬ ದೊಡ್ಡದು,” ಎಂದು. ಬಾವಿಯ ಕಪ್ಪೆ ಬಾವಿಯ ಒಂದು ಕಡೆಯಿಂದ ಮತ್ತೊಂದು ಕಡೆಗೆ ನೆಗೆದಾಡಿ, “ಈ ನನ್ನ ಬಾವಿಯಷ್ಟು ದೊಡ್ಡದೆ?” ಎಂದಿತು. ಸಮುದ್ರದ ಕಪ್ಪೆ ಬಾವಿಯ ಕಪ್ಪೆಗೆ, “ನನ್ನ ಸ್ನೇಹಿತನೆ, ಸಮುದ್ರವನ್ನು ನಿನ್ನ ಬಾವಿಯೊಂದಿಗೆ ಹೇಗೆ ಹೋಲಿಸು ವುದು?” ಎಂದು ಹೇಳಿತು. ಆಗ ಬಾವಿಯ ಕಪ್ಪೆ, “ನನ್ನ ಬಾವಿಗಿಂತ ಯಾವುದೂ ದೊಡ್ಡದಾಗಿರಲಾರದು. ಈ ಸಮುದ್ರದ ಕಪ್ಪೆ ಸುಳ್ಳುಗಾರ. ಇವನನ್ನು ಆಚೆಗೆ ತಳ್ಳಬೇಕು” ಎಂದಿತು.

ಅಲ್ಪಮನುಷ್ಯರ ಮನಸ್ಸು ಇದರಂತೆ. ತನ್ನ ಕಿರಿದಾದ ಬಾವಿಯಲ್ಲಿ ಇದ್ದು ಕೊಂಡು, ಪ್ರಪಂಚದಲ್ಲಿ ಇದನ್ನು ಮೀರಿದ ವಸ್ತುವೇ ಇಲ್ಲ ಎನ್ನುತ್ತಾರೆ.


\section{\num{೨೦೩. } ಶಾಸ್ತ್ರಜ್ಞ ಎಂದಿಗೂ ಜಂಭ ಕೊಚ್ಚಿಕೊಳ್ಳುವುದಿಲ್ಲ}

ಪಂಡಿತ ಬ್ರಾಹ್ಮಣನೊಬ್ಬ ರಾಜನ ಬಳಿಗೆ ಹೋಗಿ, “ನನಗೆ ಶಾಸ್ತ್ರಗಳು ಚೆನ್ನಾಗಿ ಗೊತ್ತಿವೆ. ನಾನು ನಿಮಗೆ ಭಾಗವತವನ್ನು ಬೋಧನೆ ಮಾಡಬೇಕೆಂದಿರುವೆನು” ಎಂದನು. ರಾಜನು ಪಂಡಿತ ನಿಗಿಂತ ಬುದ್ಧಿವಂತನಾಗಿದ್ದ. ಯಾರು ನಿಜವಾಗಿ ಭಾಗವತವನ್ನು ಕಲಿತಿರು ವನೋ, ಅವನು ತನ್ನಾತ್ಮವನ್ನು ತಾನೇ ಅರಿಯಲಿಚ್ಛಿಸುವನು; ರಾಜರ ಆಸ್ಥಾನಕ್ಕೆ ಹೋಗಿ ಐಶ್ವರ್ಯ ಅಧಿಕಾರಗಳನ್ನು ಕೇಳುವುದಿಲ್ಲ ಎಂಬುದನ್ನು ಚೆನ್ನಾಗಿ ಬಲ್ಲವನು. ಅದಕ್ಕೆ ರಾಜನು, “ನೀವೆ ಭಾಗವತವನ್ನು ಚೆನ್ನಾಗಿ ಓದಿಲ್ಲ. ಮೊದಲು ನೀವು ಭಾಗವತವನ್ನು ಚೆನ್ನಾಗಿ ಕಲಿತುಕೊಳ್ಳಿ. ಅನಂತರ ನಿಮ್ಮನ್ನು ನನ್ನ ಗುರುವಾಗಿ ಸ್ವೀಕರಿಸುವೆನು” ಎಂದನು. ಬ್ರಾಹ್ಮಣ ಮನೆಗೆ ಹೋದ. “ನನಗೇ ಭಾಗವತ ಚೆನ್ನಾಗಿ ಗೊತ್ತಿಲ್ಲ ಎಂದು ಹೇಳುತ್ತಿರುವನಲ್ಲ! ನಾನು ಭಾಗವತವನ್ನು ಎಷ್ಟೋ ಸಾರಿ ಓದಿರುವೆನು,” ಎಂದು ಭಾವಿಸತೊಡಗಿದ. ಆದರೂ ಮನೆಗೆ ಹೋಗಿ ಭಾಗವತವನ್ನು ಮತ್ತೊಮ್ಮೆ ಓದಿದ. ಇನ್ನೊಂದು ಸಲ ರಾಜರ ಬಳಿಗೆ ಬಂದು, “ಮತ್ತೊಮ್ಮೆ ನಾನು ಓದಿ ಬಂದಿರುವೆನು” ಎಂದನು. ಈಗಲೂ ಕೂಡ “ಇನ್ನೊಮ್ಮೆ ಭಾಗವತವನ್ನು ಓದಿ ಬನ್ನಿ” ಎಂದ ರಾಜ. ಬ್ರಾಹ್ಮಣನಿಗೆ ತುಂಬಾ ಬೇಜಾರಾಯಿತು. ಆದರೂ ರಾಜರು ಹೇಳುವುದ ರಲ್ಲಿ ಏನೋ ಮರ್ಮವಿರಬೇಕು ಎಂದು ಭಾವಿಸಿದನು. ಮನೆಗೆ ಹೋಗಿ ಕೋಣೆ ಯಲ್ಲಿ ತಾನೊಬ್ಬನೆ ಕುಳಿತು ಎಂದಿಗಿಂತ ಹೆಚ್ಚು ಆಸಕ್ತಿಯಿಂದ ಭಾಗವತವನ್ನು ಓದಲು ಉಪಕ್ರಮಿಸಿದ. ಓದುತ್ತಿರುವಂತೆ ಭಾಗವತದಲ್ಲಿರುವ ಗೂಢಾರ್ಥ ಗಳು ಬೆಳಕಿಗೆ ಬಂದವು. ಐಶ್ವರ್ಯ, ಗೌರವ, ರಾಜರು, ಆಸ್ಥಾನ ಇವುಗಳೆಲ್ಲ ವ್ಯರ್ಥ ಎಂಬುದನ್ನು ತಿಳಿಯುತ್ತ ಬಂದ. ಅಂದಿನಿಂದ ಭಾಗವತದಲ್ಲಿ ಇರುವು ದನ್ನು ಅನುಷ್ಠಾನ ಮಾಡಲೆತ್ನಿಸಿದ. ರಾಜನ ಬಳಿಗೆ ಬರುವುದನ್ನು ನಿಲ್ಲಿಸಿದ. ಕೆಲವು ವರ್ಷಗಳಾದ ಮೇಲೆ ರಾಜನು, ಪಂಡಿತರು ಈಗ ಏನು ಮಾಡುತ್ತಿರು ವರು ಎಂಬುದನ್ನರಿಯಲು ಅವನ ಬಳಿಗೆ ಹೋದ. ಈಗ ಪಂಡಿತನ ಮುಖ ಬ್ರಹ್ಮಕಳೆ ಮತ್ತು ಭಗವತ್​ಪ್ರೇಮದಿಂದ ರಂಜಿಸುತ್ತಿತ್ತು. ರಾಜ ಆಗ ಪಂಡಿತ ನಿಗೆ ನಮಸ್ಕಾರ ಮಾಡಿ “ಈಗ ನಿಮಗೆ ಶಾಸ್ತ್ರದ ಸಾರ ನಿಜವಾಗಿ ಗೊತ್ತಿದೆ. ನೀವು ನನ್ನನ್ನು ಸ್ವೀಕರಿಸುವುದಾದರೆ ನಾನು ನಿಮ್ಮ ಶಿಷ್ಯನಾಗುವೆನು” ಎಂದನು.


\section{\num{೨೦೪. } ಕೇರೆಹಾವಿನ ಬಾಯಿಗೆ ಬೀಳುವುದು ದೊಡ್ಡ ದುರಂತ}

ನಾನು ಒಂದು ಸಲ ಪಂಚವಟಿಯ ಮೂಲಕ ಸರ್ವೇ ಮರದ\\ತೋಪಿನ ಬಳಿಗೆ ಹೋಗುವಾಗ ಒಂದು ದೊಡ್ಡ ಕಪ್ಪೆ ಅರಚುತ್ತ\\ಇದ್ದುದನ್ನು ನೋಡಿದೆನು. ಅದು ಯಾವುದೋ ಹಾವಿನ ಬಾಯಿಗೆ ಬಿದ್ದಿರಬೇಕು ಎಂದು ಭಾವಿಸಿದೆ. ಹಿಂತಿರುಗಿ ಬರುವಾಗಲೂ ಕಪ್ಪೆ ಕಿರುಚಿ ಕೊಳ್ಳುತ್ತಲೇ ಇತ್ತು. ಏತಕ್ಕೆ ಎಂದು ನೋಡಿದಾಗ ಕೇರೆ ಹಾವಿನ ಬಾಯಿಗೆ ಕಪ್ಪೆ ಬಿದ್ದಿದೆ. ಹಾವು ಅದನ್ನು ತಿನ್ನಲೂ ಆಗದು, ಕಕ್ಕಲೂ ಆಗದು. ಆಗ ಕಪ್ಪೆಯ ಗೋಳು ಹೇಳತೀರದು. ಅದೇನಾದರೂ ನಾಗರಹಾವಿನ ಬಾಯಿಗೆ ಬಿದ್ದಿ ದ್ದರೆ ಮೂರು ಸಲ ವಟಗುಟ್ಟು ವುದರೊಳಗೆ ಅದು ಸತ್ತು ಹೋಗುತ್ತಿತ್ತು. ಅದು ಕೇರೆ ಹಾವು. ಆದುದರಿಂದ ಹಾವು ಕಪ್ಪೆ ಇಬ್ಬರೂ ಪ್ರಾಣಸಂಕಟ ಪಡಬೇಕಾಯಿತು.

ನಿಜವಾದ ಗುರುವಿನ ಬಳಿಗೆ ಹೋದರೆ ಮೂರು ವೇಳೆ ಶಬ್ದಮಾಡು ವುದರೊಳಗೆ ಶಿಷ್ಯನ ಅಹಂಕಾರ ನಾಶವಾಗುವುದು. ಗುರು ಏನಾದರೂ ಇನ್ನೂ ಪೂರ್ಣತೆಯನ್ನು ಸಾಧಿಸದಿದ್ದರೆ ಆಗ ಗುರು, ಶಿಷ್ಯ ಇಬ್ಬರೂ ಸಂಕಟಪಡಬೇಕಾ ಗುವುದು. ಶಿಷ್ಯನ ಅಹಂಕಾರವೂ ನಾಶವಾಗುವುದಿಲ್ಲ, ಪ್ರಪಂಚದ ಬಂಧನ ದಿಂದಲೂ ಪಾರಾಗಲಾರ. ಪರಿಪೂರ್ಣತೆಯನ್ನು ಮುಟ್ಟದ ಗುರುವಿನ ಬಾಯಿಗೆ ಬಿದ್ದರೆ, ಅವನು ಮುಕ್ತಿಯನ್ನು ಪಡೆಯಲಾರ.


\section{ನಮ್ಮ ಕೆಲವು ಪ್ರಕಟಣೆಗಳು}

ಗುರುದೇವ ಶ್ರೀರಾಮಕೃಷ್ಣ–ಸ್ವಾಮಿ ಸೋಮನಾಥಾನಂದ

ಭಗವಾನ್ ಶ್ರೀರಾಮಕೃಷ್ಣ–ಸ್ವಾಮಿ ಸೋಮನಾಥಾನಂದ

ಶ್ರೀರಾಮಕೃಷ್ಣ ಪರಮಹಂಸ–ಕುವೆಂಪು

ನನ್ನ ಗುರುದೇವ–ಸ್ವಾಮಿ ವಿವೇಕಾನಂದ

ಶ್ರೀರಾಮಕೃಷ್ಣ ಲೀಲಾಪ್ರಸಂಗ (೨ ಭಾಗಗಳು)–\\ಮೂಲ ಬಂಗಾಳಿ: ಸ್ವಾಮಿ ಶಾರದಾನಂದ \\ಕನ್ನಡಾನುವಾದ: ಸ್ವಾಮಿ ನಿತ್ಯಸ್ಥಾನಂದ

ಶ್ರೀರಾಮಕೃಷ್ಣ ವಚನವೇದ (೨ ಭಾಗಗಳು)–‘ಮ’

ಶ್ರೀರಾಮಕೃಷ್ಣ ಉಪದೇಶಾಮೃತ

ಶ್ರೀರಾಮಕೃಷ್ಣ ಉಪನಿಷತ್​

ನುಡಿನಮನ–ಕುವೆಂಪು

ಶ್ರೀರಾಮಕೃಷ್ಣರ ಜೀವನ ಕಥೆ–ಮಕ್ಕಳ ಪುಸ್ತಕ

ಶ್ರೀರಾಮಕೃಷ್ಣ ದೃಷ್ಟಾಂತ ಕಥೆಗಳು–ಮಕ್ಕಳ ಪುಸ್ತಕ

