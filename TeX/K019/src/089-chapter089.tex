
\chapter{೮೯. ಶ್ರೀಹರಿ! ನಿನಗಾರು ಸಮ!}

ಒಮ್ಮೆ ಮಹರ್ಷಿಗಳೆಲ್ಲರೂ ಸೇರಿ ಸರಸ್ವತೀನದಿಯ ತೀರದಲ್ಲಿ ಒಂದು ಸತ್ರಯಾಗವನ್ನು ಮಾಡುತ್ತಿದ್ದರು. ಅವರೆಲ್ಲ ಕಲೆತು ಪರಸ್ಪರ ವಿನೋದವಾಗಿ ಮಾತನಾಡುತ್ತಿರುವಾಗ ‘ತ್ರಿಮೂರ್ತಿ ಗಳಲ್ಲಿ ಯಾರು ಶ್ರೇಷ್ಠರು?’ ಎಂಬ ವಿವಾದ ಹುಟ್ಟಿತು. ಅದನ್ನು ನಿರ್ಧರಿಸಿ ತಿಳಿಸುವುದಕ್ಕಾಗಿ ಅವರು ಭೃಗುಮುನಿಯನ್ನು ಬೇಡಿಕೊಂಡರು. ಆತನು ಪ್ರತ್ಯಕ್ಷ ಪ್ರಮಾಣದಿಂದ ಅದನ್ನು ನಿರ್ಧರಿಸಬೇಕೆಂದು ಕೊಂಡು, ಅಲ್ಲಿಂದ ನೇರವಾಗಿ ಬ್ರಹ್ಮಲೋಕಕ್ಕೆ ಹೋದನು. ಅಲ್ಲಿ ಬ್ರಹ್ಮಸಭೆ ನಡೆಯುತ್ತಿತ್ತು. ಋಷಿಯು ಸಭೆಯನ್ನು ಪ್ರವೇಶಿಸಿ ಬ್ರಹ್ಮನ ಇದಿರಿಗೆ ನಿಂತುಕೊಂಡನು. ತನ್ನನ್ನು ಪ್ರತ್ಯಕ್ಷವಾಗಿ ಕಂಡರೂ ನಮಸ್ಕಾರವನ್ನೂ ಕೂಡ ಮಾಡದ ಆ ಬ್ರಾಹ್ಮಣನ ಗಂಡಗರ್ವವನ್ನು ಕಂಡು ಬ್ರಹ್ಮನಿಗೆ ರೇಗಿಹೋಯಿತು. ಆತನು ಆ ಋಷಿಯನ್ನು ಕೆಂಗಣ್ಣಿನಿಂದ ಕೆಕ್ಕರಿಸಿಕೊಂಡು ನೋಡಿದನು. ಒಡನೆಯೆ ಭೃಗುವು ಅಲ್ಲಿಂದ ಕೈಲಾಸಕ್ಕೆ ಹೋದನು. ಆತನನ್ನು ಕಾಣುತ್ತಲೆ ಪರಶಿವನು ಆದರದಿಂದ ಆತನನ್ನು ಇದಿರುಗೊಂಡು, ಆಲಿಂಗಿಸುವುದಕ್ಕೆ ಹೋದನು. ಆದರೆ ಭೃಗುಋಷಿಯು ಆತನನ್ನು ಕೋಪದಿಂದ ದುರುದುರು ನೋಡುತ್ತಾ, ‘ನೀನು ನನ್ನನ್ನು ಮುಟ್ಟಬೇಡ’ ಎಂದು ಗದರಿಸಿದನು. ಆತನ ಚರ್ಯೆಯನ್ನು ಕಂಡು ರುದ್ರನಿಗೆ ಕೋಪ ಬಂದಿತು. ಕಣ್ಣಿನಲ್ಲಿ ಕೆಂಡವನ್ನು ಸುರಿಸುತ್ತಾ ತನ್ನ ತ್ರಿಶೂಲವನ್ನು ಕೈಗೆತ್ತಿಕೊಂಡನು. ಪಕ್ಕದಲ್ಲಿದ್ದ ಪಾರ್ವತಿ ಆತನನ್ನು ತಡೆದು ಅನಾಹುತವನ್ನು ತಪ್ಪಿಸಿದಳು. ಭೃಗುವು ಅಲ್ಲಿಂದ ವೈಕುಂಠಕ್ಕೆ ಬಂದ. ಅಲ್ಲಿ ಶ್ರೀಹರಿ ಶೇಷಶಾಯಿಯಾಗಿ ನಿದ್ರೆಹೋಗಿದ್ದ. ಭೃಗುಋಷಿ ನೇರವಾಗಿ ಆತ ಮಲಗಿದ್ದಲ್ಲಿಗೆ ಹೋಗಿ ಆತನ ಎದೆಯಮೇಲೆ ತನ್ನ ಕಾಲಿಂದ ಒದೆದನು. ಒಡನೆಯೆ ಶ್ರೀಹರಿಯು ದಿಗ್ಗನೆ ಮೇಲಕ್ಕೆದ್ದು, ತನ್ನ ಮಡದಿಯಾದ ಮಹಾಲಕ್ಷ್ಮಿಯೊಡನೆ ಆ ಋಷಿಯ ಪಾದಕ್ಕೆ ಅಡ್ಡಬಿದ್ದು, ‘ಸ್ವಾಮಿ, ತಮ್ಮ ಪಾದ ತಗುಲಿ ನನ್ನ ಜನ್ಮ ಪಾವನ ವಾಯಿತು. ನೀವು ಬಂದಾಗ ನಾನು ಮೈಮರೆತು ಮಲಗಿದ್ದ ಅಪರಾಧವನ್ನು ಕ್ಷಮಿಸಬೇಕು’ ಎಂದನು. ಈ ಮಾತನ್ನು ಕೇಳುತ್ತಲೆ ಭೃಗು ಆಶ್ಚರ್ಯದಿಂದ ‘ಶ್ರೀಹರಿ, ನಿನಗಾರು ಸಮ!’ ಎಂದು ಹೇಳಿ, ಅಲ್ಲಿಂದ ನೇರವಾಗಿ ಋಷಿಗಳ ಬಳಿಗೆ ಬಂದು ನಡೆದ ಸಮಾಚಾರವನ್ನೆಲ್ಲ ಅವರಿಗೆ ವರದಿ ಮಾಡಿದ. ಎಲ್ಲರೂ ಒಕ್ಕೊರಲಿನಿಂದ ‘ಶ್ರೀಹರಿ, ನಿನಗಾರು ಸಮ!’ ಎಂದರು.

ಶ್ರೀಹರಿಯೆ ಪರಬ್ರಹ್ಮವೆಂದು ಸಾರುವ ಮತ್ತೊಂದು ಸೋಜಿಗದ ಸಂಗತಿಯನ್ನು ಕೇಳಿ. ದ್ವಾರಕಿಯಲ್ಲಿ ಒಬ್ಬ ಬಡಬ್ರಾಹ್ಮಣನಿದ್ದ. ಆತನ ಹೆಂಡತಿ ಹೆತ್ತ ಮಗು ಹುಟ್ಟುತ್ತಲೆ ಸತ್ತು ಹೋಯಿತು. ಆ ಬ್ರಾಹ್ಮಣ ತನ್ನ ಮಗುವನ್ನು ಅರಮನೆಯ ಹತ್ತಿರಕ್ಕೆ ಎತ್ತಿಕೊಂಡು ಬಂದು, ‘ರಾಜನು ಧರ್ಮಪರನಾದರೆ ತಂದೆಯ ಕಣ್ಣೆದುರಿಗೆ ಮಗ ಸಾಯುವುದು ಎಂದ ರೇನು? ರಾಜರ ಪಾಪ ಪ್ರಜೆಗಳಿಗೆ ಮೃತ್ಯು. ರಾಜರು ನಿರ್ದಯರೂ ಕಾಮುಕರೂ ಆಗಿರುವ ಕಡೆ ಪ್ರಜೆಗಳು ದರಿದ್ರರೂ ದುಃಖಿತರೂ ಆಗುತ್ತಾರೆ’ ಎಂದು ಕಟುವಾಗಿ ನಿಂದಿಸಿ ಹೋದ. ಆ ಬ್ರಾಹ್ಮಣನ ಎರಡನೆಯ ಮಗುವೂ ಹುಟ್ಟುತ್ತಲೇ ಸತ್ತಿತು. ಬ್ರಾಹ್ಮಣ ಹಿಂದಿನಂತೆಯೆ ಆ ಸತ್ತ ಮಗುವನ್ನು ಅರಮನೆಯ ಹತ್ತಿರಕ್ಕೆ ಎತ್ತಿಕೊಂಡು ಹೋಗಿ, ರಾಜರನ್ನು ನಿಂದಿಸಿ ಬಂದ. ಹೀಗೆಯೆ ಆತ ಎಂಟು ಸಲ ಮಾಡಿದ. ಯಾರೂ ಆತನನ್ನು ಗಮನಿಸಲಿಲ್ಲ. ಒಂಬ ತ್ತನೆಯ ಬಾರಿ ಆತನು ತನ್ನ ಸತ್ತ ಮಗುವಿನೊಡನೆ ಬಂದು, ರಾಜರನ್ನು ನಿಂದಿಸುತ್ತಿರು ವಾಗ, ಕೃಷ್ಣ ಅರ್ಜುನರು ಒಳಗೆ ಮಾತಾಡುತ್ತಿದ್ದರು. ಅರ್ಜುನನು ಆ ಬ್ರಾಹ್ಮಣನ ಕಿಡಿ ನುಡಿಗಳನ್ನು ಸಹಿಸಲಾರದೆ ಹೊರಕ್ಕೆ ಬಂದು ‘ಅಯ್ಯಾ ಬ್ರಾಹ್ಮಣ, ನೀನಿಷ್ಟು ಕಿಡಿನುಡಿ ಗಳನ್ನಾಡಿದರೂ ಇಲ್ಲಿನ ರಾಜರೆಲ್ಲ ಸುಮ್ಮನಿರುವರಲ್ಲಾ, ಏನನ್ಯಾಯ? ಇಲ್ಲಿನ ರಾಜ ವಂಶದವರೆಲ್ಲ ಸತ್ರಯಾಗದ ಬ್ರಾಹ್ಮಣರಂತೆ ತಿಂದುತಿಂದು ಮಲಗುತ್ತಾರೆಂದು ತೋರುತ್ತದೆ. ಇರಲಿ, ನೀನು ಯೋಚಿಸಬೇಡ. ನಾನು ಇಲ್ಲಿಯೆ ಇದ್ದು, ಮುಂದಿನ ಸಲ ಹುಟ್ಟುವ ನಿನ್ನ ಮಗನನ್ನು ನಾನು ಕಾಪಾಡುತ್ತೇನೆ’ ಎಂದನು. ಅದನ್ನು ಕೇಳಿ ಆ ಬ್ರಾಹ್ಮಣ ನಕ್ಕು ‘ಅಯ್ಯಾ, ಭಗವಂತನ ಅವತಾರವೆನಿಸಿಕೊಂಡಿರುವ ಈ ಬಲರಾಮಕೃಷ್ಣರೆ ನಾನು ಏನೆಂದರೂ ಬಾಯಿಮುಚ್ಚಿಕೊಂಡು ಕುಳಿತಿದ್ದಾರೆ! ಅವರ ಕೈಲಾಗದ ಕೆಲಸವನ್ನು ನೀನು ಮಾಡುತ್ತೀಯಾ?’ ಎಂದ. ಅರ್ಜುನನು ‘ಅಯ್ಯಾ, ನಾನು ಬಲರಾಮ ಕೃಷ್ಣರಲ್ಲ, ಸಾಕ್ಷಾತ್ ಅರ್ಜುನ. ನನ್ನ ಪರಾಕ್ರಮಕ್ಕೆ ಪರಮೇಶ್ವರನೆ ಬೆರಗಾಗಿದ್ದಾನೆ! ನಾನು ಸಾಧಾರಣನಲ್ಲ; ಮೃತ್ಯುವಿನೊಡನೆ ಹೋರಾಡಿ ನಿನ್ನ ಮಗನನ್ನು ಉಳಿಸುತ್ತೇನೆ. ನೀನು ಸಂದೇಹಪಡಬೇಡ. ನಿನ್ನ ಮಗನನ್ನು ಉಳಿಸಲಾರದೆ ಹೋದರೆ ನನ್ನ ದೇಹವನ್ನು ಅಗ್ನಿಗೆ ಆಹುತಿಯಾಗಿ ಕೊಡು ತ್ತೇನೆ. ಇದು ನನ್ನ ಪ್ರತಿಜ್ಞೆ’ ಎಂದ. ಬ್ರಾಹ್ಮಣನು ಆತನ ಮಾತುಗಳಿಂದ ಸಮಾಧಾನ ಹೊಂದಿ ಹಿಂದಕ್ಕೆ ಹೊರಟುಹೋದ.

ಅರ್ಜುನನ ಸತ್ವಪರೀಕ್ಷೆಯ ಸಮಯ ಬಹು ಬೇಗನೆ ಬಂತು. ಆ ಬ್ರಾಹ್ಮಣನ ಹೆಂಡತಿ ಹತ್ತನೆಯ ಗರ್ಭವನ್ನು ಧರಿಸಿ, ಹೆರುವುದಕ್ಕೆ ಸಿದ್ಧವಾದಳು. ಬ್ರಾಹ್ಮಣನಿಂದ ಆ ಸುದ್ದಿ ಯನ್ನು ಅರಿತ ಅರ್ಜುನನು, ತನ್ನ ಬಿಲ್ಲು ಬಾಣಗಳೊಡನೆ ಅವನ ಮನೆಗೆ ಹೋಗಿ, ಹೆರಿಗೆಯ ಮನೆಯ ಸುತ್ತ ಬಾಣಗಳ ಕೋಟೆಯನ್ನು ಕಟ್ಟಿದನು. ಸ್ವಲ್ಪವೂ ಪ್ರವೇಶಕ್ಕೆ ಆಸ್ಪದವಿಲ್ಲದಂತೆ ಆ ಮನೆಯ ಮೇಲೆ ಕೆಳಗೆಲ್ಲ ಶರಪಂಜರವನ್ನು ಬಿಗಿದನು. ಅನಂತರ ಗಾಂಡೀವಧನುಸ್ಸಿನಲ್ಲಿ ದಿವ್ಯಾಸ್ತ್ರವನ್ನು ಹೂಡಿ. ಯಾರು ಬರುವರೋ ನೋಡೋಣ ವೆಂದು ಸಿದ್ಧವಾಗಿ ನಿಂತನು. ಇದಾದ ಸ್ವಲ್ಪ ಹೊತ್ತಿನಲ್ಲೆ ಬ್ರಾಹ್ಮಣನ ಮಡದಿ ಗಂಡು ಮಗುವನ್ನು ಹೆತ್ತಳು. ನೆಲಕ್ಕೆ ಬಿದ್ದ ಮಗು ಒಮ್ಮೆ ಗಟ್ಟಿಯಾಗಿ ಅತ್ತಿತು, ಮರುನಿಮಿಷವೆ ಸತ್ತಿತು. ಇಷ್ಟೇ ಅಲ್ಲ, ಅ ಸತ್ತ ಮಗು ಶರಪಂಜರವನ್ನು ಭೇದಿಸಿಕೊಂಡು, ಆಕಾಶದ ಲ್ಲೆಲ್ಲೊ ಅದೃಶ್ಯವಾಗಿ ಹೋಯಿತು. ಆ ಬ್ರಾಹ್ಮಣ, ‘ಕುರುವಿನ ಮೇಲೆ ಬೊಕ್ಕೆ’ ಎಂಬಂತೆ ಅರ್ಜುನ ನನ್ನು ಪರಿಪರಿಯಾಗಿ ಹೀಯಾಳಿಸಿ, ನಿಂದಿಸಿದ. ಇದನ್ನು ಕೇಳಿ ಅರ್ಜುನನ ಕ್ಷಾತ್ರ ಕೆರಳಿತು. ಆ ಮಗುವನ್ನು ಎಲ್ಲಿದ್ದರೂ ತರಬೇಕೆಂದು ನಿಶ್ಚಯಿಸಿ, ಆತನು ಯೋಗವಿದ್ಯೆಯ ಬಲದಿಂದ ಯಮನ ಬಳಿಗೆ ಹೋದನು. ಅಲ್ಲಿ ಆ ಮಗುವಿರಲಿಲ್ಲ; ಇಂದ್ರನ ಅಮರಾ ವತಿಗೆ ಹೋದ, ಅಲ್ಲಿಯೂ ಇಲ್ಲ. ಆತ ಲೋಕಗಳನ್ನೆಲ್ಲ ಶೋಧಿಸಿದರೂ ಆ ಮಗುವಿನ ನೆಲೆ ಹತ್ತಲಿಲ್ಲ. ಹಾಗಾದರೆ ಗತಿ? ಅರ್ಜುನನು ತನ್ನ ಮಾತಿನಂತೆ ಅಗ್ನಿಪ್ರವೇಶಕ್ಕೆ ಸಿದ್ಧ ನಾದ. ಆಗ ಶ್ರೀಕೃಷ್ಣ ಅವನನ್ನು ತಡೆದು ‘ಮಿತ್ರ, ಆತುರ ಪಡಬೇಡ. ನಾನು ಆ ಮಗುವನ್ನು ನಿನಗೆ ತೋರಿಸುತ್ತೇನೆ’ ಎಂದು ಹೇಳಿ, ಅವನನ್ನು ತನ್ನ ರಥದಲ್ಲಿ ಕುಳ್ಳಿರಿಸಿ ಕೊಂಡು, ಪಶ್ಚಿಮ ದಿಕ್ಕಿಗೆ ಪ್ರಯಾಣ ಹೊರಟ.

ಪ್ರಯಾಣ ಹೊರಟ ಕೃಷ್ಣಾರ್ಜುನರು ಸಪ್ತದ್ವೀಪಗಳನ್ನೂ, ಸಪ್ತಸಮುದ್ರಗಳನ್ನೂ ದಾಟಿ ಇನ್ನೂ ಮುಂದಕ್ಕೆ ಹೋದರು. ಅಲ್ಲಿ ಭೂಮಿಗೂ ಆಕಾಶಕ್ಕೂ ಏಕವಾಗಿದ್ದ ಚಕ್ರ ವಾಳ ಪರ್ವತವಿತ್ತು. ಅದನ್ನೂ ದಾಟಿದರು. ಅಲ್ಲಿಂದ ಮುಂದೆ ಬರಿಯ ಕಗ್ಗತ್ತಲು. ಅದ ರಲ್ಲಿ ಹಾದಿ ತಿಳಿಯದೆ ರಥದ ಕುದುರೆಗಳು ದಾರಿತಪ್ಪಿದವು. ಆಗ ಶ್ರೀಕೃಷ್ಣನು ಕೋಟಿ ಸೂರ್ಯ ಪ್ರಕಾಶಮಾನವಾದ ತನ್ನ ಚಕ್ರವನ್ನು ಮುಂದೆ ಹೋಗುವಂತೆ ಅಪ್ಪಣೆ ಮಾಡಿದನು. ಆ ಕಗ್ಗತ್ತಲಿನಲ್ಲಿ ಚಕ್ರವು ಚೆಲ್ಲುತ್ತಿದ್ದ ಬೆಳಕನ್ನು ನೋಡಲಾರದೆ ಅರ್ಜುನನು ಕಣ್ಮುಚ್ಚಿದನು. ರಥವು ಮುಂದೆ ಮುಂದೆ ಸಾಗಿ, ಕತ್ತಲೆಗಿಂತಲೂ ಭಯಂಕರವಾಗಿದ್ದ ಮಹಾಜಲವನ್ನು ಪ್ರವೇಶಿಸಿತು. ಪ್ರಚಂಡವಾದ ಗಾಳಿಯಿಂದ ಆ ಮಹಾಜಲ ಅಲ್ಲೋಲ ಕಲ್ಲೋಲವಾಗಿ, ಕತಕತ ಕುದಿಯುವಂತೆ ಕಾಣಿಸುತ್ತಿತ್ತು. ಆ ಮಹಾ ಜಲಮಧ್ಯದಲ್ಲಿ ಸಾವಿ ರಾರು ರತ್ನದ ಕಂಬಗಳಿಂದ ಕೂಡಿದ ದಿವ್ಯ ಭವನವಿತ್ತು. ಅದರಲ್ಲಿ ಆದಿಶೇಷನು ತನ್ನ ಸಾವಿರ ಹೆಡೆಗಳನ್ನೂ ಅರಳಿಸಿಕೊಂಡು, ಅವುಗಳ ರತ್ನಗಳಿಂದ ದಿಕ್​ದಿಗಂತಗಳನ್ನು ಬೆಳಗುತ್ತ ಮಂಡಿಸಿದ್ದನು. ಆತನ ಶರೀರದ ಮೇಲೆ ನೀಲಮೇಘಶ್ಯಾಮನಾಗಿ, ಪೀತಾಂಬರ ಧಾರಿಯಾದ ಪುರುಷೋತ್ತಮನು ತನ್ನ ತುಟಿಗಳಲ್ಲಿ ಮಂದಹಾಸವನ್ನು ಚೆಲ್ಲುತ್ತಾ ಇದ್ದನು. ವಿಶಾಲವಾದ ಆತನ ಕಣ್ಣುಗಳು ಶಾಂತಿಗೆ ತೌರುಮನೆಯಾಗಿದ್ದವು. ಆತನ ಎಂಟು ತೋಳುಗಳು, ಎದೆಯ ಕೌಸ್ತುಭಮಣಿ, ಕೊರಳಹಾರ, ಸುಳಿಯಾದ ಮುಂಗು ರುಳು, ತಲೆಯ ಕಿರೀಟ–ಒಂದೊಂದೂ ಆತನ ಮಹತ್ತನ್ನು ಸಾರುತ್ತಿದ್ದವು. ಲಕ್ಷ್ಮಿಯು ಕೀರ್ತಿಲಕ್ಷ್ಮಿ ಮತ್ತು ಅಷ್ಟಸಿದ್ಧಿಗಳೊಡನೆ ಆತನನ್ನು ಸೇವಿಸುತ್ತಿದ್ದಳು. ನಂದ, ಸುನಂದ ಮೊದಲಾದ ಭಕ್ತರು ಆತನ ಸುತ್ತ ಕೈಕಟ್ಟಿ ನಿಂತಿದ್ದರು. ಶ್ರೀಕೃಷ್ಣನು ಆತನಿಗೆ ನಮಸ್ಕರಿಸಿ ದನು. ಆತನ ತೇಜಸ್ಸನ್ನು ಕಂಡು ಭಯದಿಂದ ನಡುಗುತ್ತಾ ಇದ್ದ ಅರ್ಜುನನೂ ಆತನಿಗೆ ನಮಸ್ಕರಿಸಿದನು.

ತನ್ನೆದುರಿಗೆ ಭಕ್ತಿಯಿಂದ ಕೈಜೋಡಿಸಿ ನಿಂತಿರುವ ಕೃಷ್ಣಾರ್ಜುನರನ್ನು ಕುರಿತು ಆ ಮಹಾ ಪುರುಷನು ಧೀರಗಂಭೀರವಾಣಿಯಿಂದ ‘ಅಯ್ಯಾ, ನಿಮ್ಮಿಬ್ಬರನ್ನೂ ಇಲ್ಲಿಗೆ ಕರೆತರಿಸ ಬೇಕೆಂಬ ಉದ್ದೇಶದಿಂದಲೆ ನಾನು ಆ ಬ್ರಾಹ್ಮಣನ ಮಕ್ಕಳೆಲ್ಲ ಇಲ್ಲಿಗೆ ಬರುವಂತೆ ಮಾಡಿದೆ. ನೀವಿಬ್ಬರೂ ಧರ್ಮರಕ್ಷಣೆಗಾಗಿ ಭೂಮಿಯಲ್ಲಿ ಅವತರಿಸಿರುವಿರಿ. ನಾನೆ ಈ ಶ್ರೀಕೃಷ್ಣ, ನೀನು ನನ್ನ ಅಂಶ. ನೀವಿಬ್ಬರೂ ಭೂಭಾರವನ್ನು ಇಳಿಸಿ, ನನ್ನ ಬಳಿಗೆ ಹಿಂದಿರು ಗುವ ಕಾಲ ಹತ್ತಿರವಾಗುತ್ತಾ ಇದೆ. ಇದನ್ನು ತಿಳಿಸುವುದಕ್ಕಾಗಿಯೆ ನಿಮ್ಮನ್ನು ನಾನು ಇಲ್ಲಿಗೆ ಕರೆತರಿಸಿದುದು’ ಎಂದನು. ಕೃಷ್ಣಾರ್ಜುನರು ಆ ಬ್ರಾಹ್ಮಣನ ಮಕ್ಕಳನ್ನೆಲ್ಲ ತಮ್ಮೊಡನೆ ಕರೆದುಕೊಂಡು, ಆದಿಪುರುಷನಿಗೆ ನಮಸ್ಕರಿಸಿ, ಅಲ್ಲಿಂದ ಹಿಂದಿರುಗಿದರು. ಬರುತ್ತಾ ದಾರಿಯಲ್ಲಿ ಅರ್ಜುನನು ‘ಪ್ರಭು, ಶ್ರೀಹರಿ! ನಿನಗಾರು ಸಮ?’ ಎಂದು ಮನಸ್ಸಿನಲ್ಲೆ ಶ್ರೀಕೃಷ್ಣನನ್ನು ಸ್ತುತಿಸಿದನು. ಆತನು ದ್ವಾರಕಿಗೆ ಹಿಂದಿರುಗುತ್ತಲೆ ಬ್ರಾಹ್ಮಣನ ಮಕ್ಕಳ ನ್ನೆಲ್ಲ ಆತನಿಗೆ ಒಪ್ಪಿಸಿದನು. ಆ ಬ್ರಾಹ್ಮಣ ಅರ್ಜುನನಿಗೆ ಬದಲಾಗಿ ಶ್ರೀಕೃಷ್ಣನಿಗೆ ‘ಶ್ರೀಹರಿ, ನಿನಗಾರು ಸಮ?’ ಎಂದು ನಮಸ್ಕರಿಸಿ ಸ್ತುತಿಸಬೇಕೆ?

