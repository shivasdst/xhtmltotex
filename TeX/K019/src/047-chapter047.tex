
\chapter{೪೭. ಮತ್ತಿಯ ಮರಗಳು ಮನುಷ್ಯರಾದುವು}

ನಂದನ ಮನೆಯಲ್ಲಿ ಒಂದು ದಿನ ಆಳುಕಾಳುಗಳೆಲ್ಲ ಬೇರೆಬೇರೆ ಕೆಲಸಗಳಲ್ಲಿ ತೊಡಗಿ ದ್ದುದರಿಂದ ಯಜಮಾನಿಯಾದ ಯಶೋದಾದೇವಿಯೆ ಮೊಸರನ್ನು ಕಡೆಯಬೇಕಾಯಿತು. ಆಕೆ ತನ್ನ ಮಗನ ಬಾಲಲೀಲೆಗಳನ್ನು ರಾಗರಾಗವಾಗಿ ಹಾಡಿಕೊಳ್ಳುತ್ತಾ ತನ್ನ ಕೆಲಸದಲ್ಲಿ ತೊಡಗಿರಲು, ಶ್ರೀಕೃಷ್ಣ ಅಲ್ಲಿಗೆ ಬಂದು ‘ಅಮ್ಮ, ಹೊಟ್ಟೆಗೆ ಕೊಡು’ ಎಂದು ಆಕೆಯ ನೆರಿಗೆಯನ್ನು ಹಿಡಿದೆಳೆದು ಕಾಡುತ್ತಾ, ಕಡೆಗೆ ಕಡೆಗೋಲನ್ನೇ ಹಿಡಿದು ನಿಲ್ಲಿಸಿದನು. ಆಕೆ ‘ನೀನೊಬ್ಬ ಶುದ್ಧ ತುಂಟ’ ಎಂದು ಬೈದು, ಅವನನ್ನು ತೊಡೆಯ ಮೇಲೆ ಮಲಗಿಸಿ ಕೊಂಡು ಮೊಲೆಯುಣಿಸುತ್ತಿದ್ದಳು. ಅಷ್ಟರಲ್ಲಿ ಒಲೆಯ ಮೇಲಿಟ್ಟಿದ್ದ ಹಾಲು ಉಕ್ಕು ತ್ತಿರಲು, ಆಕೆ ಮಗನನ್ನು ಕೆಳಗೆ ಕೂಡಿಸಿ, ಒಲೆಯ ಬಳಿಗೆ ಹೋದಳು. ಇದನ್ನು ಕಂಡ ಪುಟ್ಟ ಕೃಷ್ಣನಿಗೆ ರೇಗಿಹೋಯಿತು. ಆತ ತುಟಿಯನ್ನು ಕಚ್ಚಿ, ಒಂದು ಗುಂಡುಕಲ್ಲಿನಿಂದ ಮೊಸರಿನ ಗಡಿಗೆಯನ್ನು ಒಡೆದುಹಾಕಿ, ಅದರಿಂದ ಆಗತಾನೆ ತೆಗೆದಿದ್ದ ಬೆಣ್ಣೆಯನ್ನೆಲ್ಲ ತಿಂದುಹಾಕಿದನು. ಅನಂತರ ಮನೆಯ ಒಂದು ಮೂಲೆಯಲ್ಲಿದ್ದ ಒರಳು ಕಲ್ಲಮೇಲೆ ಕುಳಿತು, ತನ್ನ ಕೈಲಿ ಉಳಿದಿದ್ದ ಬೆಣ್ಣೆಯನ್ನು ಒಂದು ಕೋತಿಗೆ ತಿನ್ನಿಸುತ್ತಾ ಇದ್ದನು. ಯಶೋದೆ ಹಾಲನ್ನು ಕೆಳಗಿಳಿಸಿ ಹಿಂದಿರುಗಿ ಬಂದು ನೋಡುತ್ತಾಳೆ, ಮೊಸರಿನ ಗಡಿಗೆ ಒಡೆದುಹೋಗಿದೆ; ಹೊಸದಾಗಿ ತೆಗೆದ ಬೆಣ್ಣೆಯೆಲ್ಲ ಮಾಯವಾಗಿದೆ. ಈ ಕಿಡಿಗೇಡಿತನ ತನ್ನ ಮಗರಾಯನದೇ ಎಂದು ಆಕೆಗೆ ಗೊತ್ತಾಯಿತು. ಆಕೆ ಅವನನ್ನು ಹುಡುಕಿಕೊಂಡು ಹೊರಟಳು. ಅಕೆ ತನ್ನ ಬಳಿಗೆ ಬರುವ ಸುಳಿವನ್ನು ತಿಳಿಯುತ್ತಲೆ ಶ್ರೀಕೃಷ್ಣ ತನ್ನ ಕಾಲಿಗೆ ಬುದ್ಧಿ ಹೇಳಿದ. ಯಶೋದೆ ಒಂದು ಸಣ್ಣ ಕಡೆಗೋಲನ್ನು ಕೈಲಿ ಹಿಡಿದು, ಅವನನ್ನು ಅಟ್ಟಿಸಿಕೊಂಡು ಹೊರಟಳು. ಅವನು ಸ್ವಲ್ಪ ಹೊತ್ತು ಗುಡುಗಾಡಿಸಿ, ಕಡೆಗೆ ತಾಯಿಯ ಕೈಗೆ ಸಿಕ್ಕಿಬಿದ್ದ. ಆಕೆ ಅವನಿಗೆ ಎರಡು ಬಿಗಿಯಬೇಕೆಂದು ಕಡೆಗೋಲನ್ನು ಎತ್ತುವ ಮುನ್ನವೆ, ಅವನು ಬಹಳ ಭಯಪಟ್ಟವನಂತೆ ‘ಅಮ್ಮಾ, ಹೊಡೆಯಬೇಡಮ್ಮ’ ಎಂದು ಅಳುವುದಕ್ಕೆ ಪ್ರಾರಂಭಿಸಿದ. ಅದನ್ನು ಕಂಡೊಡನೆ ಆ ತಾಯಿಯ ಕರುಳು ಕರಗಿ ಹೋಯಿತು. ಆಕೆ ಕೈಲಿದ್ದ ಕಡೆಗೋಲನ್ನು ಅತ್ತ ಬಿಸುಟು ‘ತುಂಟ, ಬಾ ನಿಂಗೆ ಮಾಡು ತ್ತೇನೆ, ಈ ಒರಳಿಗೆ ನಿನ್ನ ಕಟ್ಟಿಹಾಕುತ್ತೇನೆ. ಬಿದ್ದಿರು ಇಲ್ಲಿಯೆ’ ಎಂದು ಹೇಳಿ, ಅವನನ್ನು ಒರಳು ಕಲ್ಲಿನ ಹತ್ತಿರಕ್ಕೆ ಎಳೆತಂದಳು. ಅನಂತರ ಅಲ್ಲಿಯೇ ಬಿದ್ದಿದ್ದ ಒಂದು ಮಾರುದ್ದದ ಹಗ್ಗವನ್ನು ತೆಗೆದುಕೊಂಡು ಅಕೆ ಅವನ ಸೊಂಟಕ್ಕೆ ಸುತ್ತಿದಳು. ಆದರೆ ಆ ಹಗ್ಗ ಎರಡು ಬೆರಳಷ್ಟು ಕಡಿಮೆಯಾಗಿತ್ತು. ಯಶೋದೆ ಬೇರೊಂದು ಹಗ್ಗವನ್ನು ತಂದು ಅದರೊಡನೆ ಸೇರಿಸಿ ಕಟ್ಟಿದಳು. ಅದೂ ಸಾಲದೆ ಬಂತು. ಮತ್ತೆಮತ್ತೆ ಹಗ್ಗಗಳನ್ನು ತಂದು ಜೋಡಿಸಿದರೂ ಆ ಮುಗುವಿನ ಸೊಂಟಕ್ಕೆ ಸಾಲದೆ ಹೋಯಿತು. ಇಷ್ಟಾದರೂ ಆ ತಾಯಿಗೆ ವಿವೇಕ ಬರಲಿಲ್ಲ. ಆ ಸಚ್ಚಿದಾನಂದನ ಸ್ವರೂಪವನ್ನು ತಿಳಿಯುವ ಶಕ್ತಿ ಆಕೆಗೆಲ್ಲಿ ಬರ ಬೇಕು? ಹೇಗಾದರೂ ಮಾಡಿ ಕೃಷ್ಣನನ್ನು ಕಟ್ಟಿ ಹಾಕಲೇಬೇಕೆಂದು ಆಕೆ ಬವಣೆಪಡುವು ದನ್ನು ಕಂಡು, ಶ್ರೀಕೃಷ್ಣನ ಮನಸ್ಸು ಕರಗಿತು. ಆತನು ಆಕೆಯ ಕಟ್ಟಿಗೆ ಒಳಗಾದ. ಆಕೆ ಮಗನನ್ನು ಒರಳಿಗೆ ಬಿಗಿದು, ‘ಬಿದ್ದಿರು ಇಲ್ಲಿ’ ಎಂದು ಮತ್ತೊಮ್ಮೆ ಮೂದಲಿಸಿ, ಮನೆ ಗೆಲಸಕ್ಕೆಂದು ಹೊರಟುಹೋದಳು.

ಯಶೋದೆ ಅತ್ತ ಹೋಗುತ್ತಲೆ ಇತ್ತ ಶ್ರೀಕೃಷ್ಣನು, ಆ ಒರಳುಕಲ್ಲನ್ನೂ ಎಳೆದು ಕೊಂಡು ಅಂಗಳಕ್ಕೆ ಬಂದನು. ಅಲ್ಲಿ ಎರಡು ಮತ್ತಿಯ ಮರಗಳು ಒಂದರ ಪಕ್ಕದಲ್ಲಿ ಒಂದು ಜೋಡಿಯಾಗಿ ಬೆಳೆದಿದ್ದವು. ಮಗುವಾದ ಶ್ರೀಕೃಷ್ಣ ಆ ಮರಗಳ ಮಧ್ಯೆ ತೂರಿ ಕೊಂಡು ಹೊರಟ. ಅವನನ್ನು ಕಟ್ಟಿದ್ದ ಒರಳು ಆ ಮರಗಳ ಮಧ್ಯೆ ಸಿಕ್ಕಿಕೊಂಡಿತು. ಆ ವಿಚಿತ್ರ ಮಗು ಮುಂದಕ್ಕೆ ಹೋಗಬೇಕೆಂದು ಬಲವಾಗಿ ಜಗ್ಗಿತು. ಒಡನೆಯೆ ಮತ್ತಿಯ ಮರಗಳೆರಡೂ ಮುರಿದು ಬಿದ್ದವು. ಅವುಗಳಿದ್ದ ಸ್ಥಳದಲ್ಲಿ ಇಬ್ಬರು ಮಹಾಪುರುಷರು ಕಾಣಿಸಿಕೊಂಡರು. ಅವರಿಬ್ಬರೂ ಕುಬೇರನ ಮಕ್ಕಳು. ನಳಕೂಬರ, ಮಣಿಗ್ರೀವರೆಂದು ಅವರ ಹೆಸರು. ಅವರು ಒಂದಾನೊಂದು ಕಾಲದಲ್ಲಿ ಮದ್ಯಪಾನದಿಂದ ಮತ್ತೇರಿ ಚೈತ್ರ ರಥವೆಂಬ ಕುಬೇರನ ಉದ್ಯಾನವನದಲ್ಲಿ ತಮ್ಮ ಮಡದಿಯರೊಡನೆ ವಿಹರಿಸುತ್ತಾ ಅಲ್ಲಿದ್ದ ಸರೋವರದಲ್ಲಿ ನೀರಾಟಕ್ಕಿಳಿದಿದ್ದರು. ಹೆಣ್ಣಾನೆಗಳೊಡನೆ ಕೂಡಿದ ಮದ್ದಾನೆ ಗಳಂತೆ ಅವರು ಆಟವಾಡುತ್ತಿರುವಾಗ ಮಹರ್ಷಿಗಳಾದ ನಾರದರು ಸಂಚಾರ ಮಾಡುತ್ತಾ ಅಲ್ಲಿಗೆ ಬಂದರು. ಅವರನ್ನು ಕಂಡು ಹೆಣ್ಣುಗಳೆಲ್ಲ ನಾಚಿಕೆಯಿಂದ ಬುಡುಬುಡು ಓಡಿ ಬಂದು ತಮ್ಮ ಬಟ್ಟೆಗಳನ್ನು ಮೈಗೆ ಸುತ್ತಿಕೊಂಡರು. ಆದರೆ ನಳಕೂಬರ ಮಣಿಗ್ರೀವರು ಪುಷಿಗಳನ್ನು ಸ್ವಲ್ಪವೂ ಲಕ್ಷಿಸದೆ ಬೆತ್ತಲೆಯಾಗಿ ಅವರಿದುರಿಗೆ ನಿಂತರು. ಇದನ್ನು ಕಂಡು ನಾರದರಿಗೆ ಕೋಪ ಬಂತು. “ಎಲಾ, ನೀವು ಮನುಷ್ಯರಾಗಿರಲು ಯೋಗ್ಯರಲ್ಲ; ಕಾಡಿನಲ್ಲಿ ಮರಗಳಾಗಿರುವುದಕ್ಕೆ ತಕ್ಕವರು. ನೀವಿಬ್ಬರೂ ನೂರು ವರ್ಷಕಾಲ ಮರವಾಗಿ ಬಿದ್ದಿರಿ. ನೂರು ವರ್ಷಗಳಾದ ಮೇಲೆ ಶ್ರೀಕೃಷ್ಣನ ಪಾದ ಸೋಕಿ, ನಿಮ್ಮ ಶಾಪ ನಿವಾರಣೆಯಾಗು ತ್ತದೆ” ಎಂದು ಹೇಳಿದರು. ನಾರದರ ಶಾಪದಿಂದ ಮತ್ತಿಯ ಮರವಾಗಿ ಹುಟ್ಟಿದ್ದ ಕುಬೇರನ ಆ ಮಕ್ಕಳಿಬ್ಬರೂ ಈಗ ಶ್ರೀಕೃಷ್ಣನ ಅನುಗ್ರಹದಿಂದ ಸ್ವಸ್ವರೂಪವನ್ನು ಪಡೆದು, ಶಿಶುರೂಪದಲ್ಲಿರುವ ಆ ಭಗವಂತನನ್ನು ‘ಹೇ ದೇವದೇವ, ನಮ್ಮನ್ನು ಅನು ಗ್ರಹಿಸು. ಇನ್ನು ಮುಂದೆ ನಿನ್ನ ಸ್ತೋತ್ರವೇ ನಮ್ಮ ಮಾತಾಗಿರಲಿ. ನಮ್ಮ ಕಿವಿ ನಿನ್ನ ಕೀರ್ತನೆಗೂ, ಕಣ್ಣುಗಳು ನಿನ್ನ ಮೂರ್ತಿಯ ದರ್ಶನಕ್ಕೂ, ಕೈಗಳು ನಿನ್ನ ಪೂಜೆಗೂ, ಮನಸ್ಸು ನಿನ್ನ ಧ್ಯಾನಕ್ಕೂ ಮೀಸಲಾಗಿರಲಿ; ನಮ್ಮ ತಲೆ ಸದಾ ನಿನ್ನ ಪಾದವಂದನಕ್ಕೆ ಮುಡಿಪಾಗಲಿ’ ಎಂದು ನಮಸ್ಕರಿಸಿ ಬೇಡಿಕೊಂಡರು. ಶ್ರೀಕೃಷ್ಣನು ‘ತಥಾಸ್ತು’ ಎಂದು ಹೇಳಿ ಅವರನ್ನು ಬೀಳ್ಕೊಟ್ಟನು.

ಅಷ್ಟರಲ್ಲಿ ದೊಡ್ಡ ಮರಗಳೆರಡು ಮುರಿದು ಬಿದ್ದುದರಿಂದಾದ ದೊಡ್ಡ ಶಬ್ದವನ್ನು ಕೇಳಿ, ನಂದನೇ ಮೊದಲಾದವರೆಲ್ಲ ಅಲ್ಲಿಗೆ ಓಡಿಬಂದರು. ಆ ಮರಗಳು ಇದ್ದಕ್ಕಿದ್ದಂತೆ ಹೇಗೆ ಬಿದ್ದುವೆಂದು ಅವರು ಯೋಚಿಸುತ್ತಿರುವಾಗ, ತನಗೆ ಕಟ್ಟಿದ್ದ ಒರಳನ್ನು ಎಳೆಯುತ್ತಾ ಅಲ್ಲಿಯೇ ಸುತ್ತಾಡುತ್ತಿರುವ, ಶ್ರೀಕೃಷ್ಣ ಅವರ ಕಣ್ಣಿಗೆ ಬಿದ್ದ. ‘ಈ ಮರಗಳು ಹೇಗೆ ಬಿದ್ದಿರಬಹುದು?’ ಎಂದು ಹಿರಿಯರು ಮಾತನಾಡುತ್ತಿದ್ದುದನ್ನು ಅಲ್ಲಿಯೇ ನಿಂತಿದ್ದ ಒಬ್ಬ ಗೊಲ್ಲರ ಹುಡುಗ ಕೇಳಿ, ತಾನು ಅದನ್ನು ಪ್ರತ್ಯಕ್ಷವಾಗಿ ಕಂಡಿದ್ದುದರಿಂದ, ಕಂಡ ದ್ದನ್ನು ಕಂಡಹಾಗೆ ತಿಳಿಸಿದ. ಆದರೆ ಅವನ ಮಾತನ್ನು ಯಾರೂ ನಂಬಲಿಲ್ಲ. ‘ಈ ಸಣ್ಣ ಮಗು ಎಳೆಯುವುದೆಂದರೇನು? ಆ ದೊಡ್ಡ ಮರಗಳು ಬೀಳುವುದೆಂದರೇನು?’ ಎಂದು ಹೇಳಿ ನಂದನು ತನ್ನ ಮಗನ ಕಡೆ ನೋಡಿದನು. ಬಹು ಕಷ್ಟದಿಂದ ಒರಳನ್ನು ಎಳೆಯುತ್ತಾ ನಡೆಯುತ್ತಿರುವ ಮಗನನ್ನು ಕಂಡು, ಕರುಣೆಯಿಂದ ‘ಎಲ ತುಂಟ, ಹೋಗಿ ಆಡಿಕೊ’ ಎಂದು ಹೇಳಿ, ನಗುತ್ತಾ, ಅವನಿಗೆ ಕಟ್ಟಿದ್ದ ಹಗ್ಗವನ್ನು ಬಿಚ್ಚಿಹಾಕಿದನು. ಮತ್ತಿಯ ಮರ ಗಳು ಮಾನವರಾದುದನ್ನು ಅವನೇನು ಬಲ್ಲ? ಶ್ರೀಕೃಷ್ಣ ಯಾರೆಂಬುದನ್ನು ತಾನೆ ಅವ ನೆಂತು ತಿಳಿಯಬಲ್ಲ?

