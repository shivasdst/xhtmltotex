
\chapter{೭೦. ಹದಿನಾರು ಸಾವಿರ ಹೆಂಡತಿಯರು}

ಪ್ರಾಗ್ಜೋತಿಷವೆಂಬ ಪಟ್ಟಣದಲ್ಲಿ ಮುರಾಸುರನೆಂಬ ರಕ್ಕಸನಿದ್ದ. ಅವನು ಮಹಾ ಶೂರ, ಅಷ್ಟೇ ಕೇಡಿಗ. ಅವನು ದೇವೇಂದ್ರನ ಅಧಿಕಾರಸೂಚಕವಾದ ಬಿಳಿಯ ಕೊಡೆಯನ್ನೆ ಕಿತ್ತುಕೊಂಡು ಹೋಗಿದ್ದನು, ಸ್ವರ್ಗಲೋಕದ ಮುಖ್ಯವಿಹಾರವಸ್ತುವಾದ ಮಣಿಪರ್ವತ ವನ್ನೆ ಹೊತ್ತುಕೊಂಡು ಹೋಗಿದ್ದನು, ದೇವೇಂದ್ರನ ತಾಯಿಯಾದ ಅದಿತಿಯ ಕಿವಿಯ ಓಲೆಗಳನ್ನೆ ಕಿತ್ತುಕೊಂಡು ಹೋಗಿದ್ದನು, ಭೂಮಿಯ ಮೇಲಿರುವ ರಾಜಕುಮಾರಿಯರ ನ್ನೆಲ್ಲ ಹೊತ್ತುಕೊಂಡುಹೋಗಿ ತನ್ನ ಸೆರೆಯಲ್ಲಿ ಹಾಕಿದ್ದನು. ಇವನ ಹಿಂಸೆಯನ್ನು ತಾಳ ಲಾರದೆ ದೇವೇಂದ್ರನು ಶ್ರೀಕೃಷ್ಣನಲ್ಲಿ ಬಂದು ಮೊರೆಯಿಟ್ಟನು. ಒಡನೆಯೆ ಶ್ರೀಕೃಷ್ಣನು ತನ್ನ ಮೋಹದ ಮಡದಿಯಾದ ಸತ್ಯಭಾಮೆಯೊಡನೆ ಗರುಡನನ್ನೇರಿ ಪ್ರಾಗ್ಜೋತಿಷಪುರದ ಬಳಿಗೆ ಬಂದನು. ಅದು ಎಂತಹವರಿಗೂ ಪ್ರವೇಶಿಸಲು ಅಸಾಧ್ಯವಾಗಿತ್ತು. ಅದರ ಸುತ್ತಲೂ ಬೆಟ್ಟದ ಕೋಟೆ, ಅದನ್ನು ಭೇದಿಸಿದರೆ ನೀರಿನ ಕೋಟೆ, ಅದರ ಮುಂದೆ ಬೆಂಕಿಯ ಕೋಟೆ, ಅದಕ್ಕೆ ಆಚೆ ಶಸ್ತ್ರಗಳ ಕೋಟೆ, ಅದನ್ನು ದಾಟಿದರೆ ಶತ್ರುಗಳನ್ನು ಸುತ್ತಿ ಕೊಂಡು ನೆಲಕ್ಕೆ ಕೆಡಹಿ ಕೊಲ್ಲುವ ಮುರಪಾಶ. ಶ್ರೀಕೃಷ್ಣನು ಅಲ್ಲಿಗೆ ಬಂದವನೆ ತನ್ನ ಗದೆಯಿಂದ ಬೆಟ್ಟದ ಕೊನೆಯನ್ನು ಹುಡಿಗುಟ್ಟಿದನು; ತನ್ನ ಚಕ್ರಾಯುಧದಿಂದ ಉಳಿದ ಕೋಟೆಗಳನ್ನೂ ಧ್ವಂಸಮಾಡಿ, ಒಮ್ಮೆ ತನ್ನ ಶಂಖವನ್ನು ಊದಿದನು. ಒಡನೆಯೆ ಸಿಡಿಲು ಬಡಿದಂತಾಯಿತು. ನೀರಿನ ಮಧ್ಯದಲ್ಲಿ ನಿಶ್ಚಿಂತೆಯಿಂದ ಮಲಗಿದ್ದ ಮುರನೆಂಬ ಐದು ತಲೆಯ ರಕ್ಕಸನು ಶಂಖದ ದನಿಯನ್ನು ಕೇಳುತ್ತಲೆ ಮೆಟ್ಟಿಬಿದ್ದು ಮೇಲಕ್ಕೆದ್ದು ಬಂದನು. ತನ್ನ ನಿದ್ದೆ ಭಂಗವಾದುದಕ್ಕಾಗಿ ಅವನಿಗೆ ಬಲು ಕೋಪ ಬಂದಿತ್ತು. ಅವನು ತನ್ನ ದೊಡ್ಡ ಗದೆಯನ್ನು ಗರಗರ ತಿರುಗಿಸುತ್ತಾ, ಮೂರು ಲೋಕಗಳನ್ನೂ ನುಂಗುವವನಂತೆ ತನ್ನ ಐದು ಬಾಯಿಗಳನ್ನೂ ತೆರೆದುಕೊಂಡು ಶ್ರೀಕೃಷ್ಣನ ಮೇಲೆ ಏರಿಹೋದನು. ಆದರೆ, ಗರುಡನನ್ನು ಹಿಡಿಯಹೋದ ಹಾವಿನಂತಾಯಿತು, ಅವನ ಸ್ಥಿತಿ. ಐದು ಬಾಯಿಗಳಿಂದಲೂ ಅಬ್ಬರಿ ಸುತ್ತಾ ಬಂದ ಆ ಭಯಂಕರ ರಕ್ಕಸನನ್ನು, ಶ್ರೀಕೃಷ್ಣನು ಒಂದೇ ಸಲ ಐದು ಬಾಣಗಳನ್ನು ಬಿಟ್ಟು ಐದು ತಲೆಗಳನ್ನೂ ಕತ್ತರಿಸಿಹಾಕಿದನು. ವಜ್ರಾಯುಧದಿಂದ ಉರುಳುವ ಬೆಟ್ಟ ದಂತೆ ಅವನ ದೇಹ ಕೆಳಕ್ಕುರುಳಿತು.

ಮುರಾಸುರನು ಸತ್ತುದುದನ್ನು ಕೇಳಿ, ಅವನ ಏಳು ಜನ ಮಕ್ಕಳೂ ನರಕಾಸುರನಿಂದ ಅಪ್ಪಣೆ ಪಡೆದು, ಪೀಠಾಸುರನೆಂಬ ಸೇನಾಪತಿಯೊಡನೆ ದೊಡ್ಡ ಸೇನೆಯನ್ನು ತೆಗೆದು ಕೊಂಡು ಶ್ರೀಕೃಷ್ಣನ ಮೇಲೆ ಯುದ್ಧಕ್ಕೆ ಬಂದರು. ಆದರೆ ಅವರೆಲ್ಲ ಕ್ಷಣಮಾತ್ರದಲ್ಲಿ ಶ್ರೀಕೃಷ್ಣನ ಚಕ್ರಕ್ಕೆ ಆಹುತಿಯಾಗಿ ಹೋದರು. ಸೋಲನ್ನೆ ಕಂಡರಿಯದ ತನ್ನ ಸೇನೆ ನಾಶ ವಾದುದನ್ನು ಕಂಡು ನರಕಾಸುರನಿಗೆ ರೋಷವುಕ್ಕಿತು. ಆತನು ತನ್ನ ಆನೆಗಳ ಸೈನ್ಯದೊಡನೆ ಯುದ್ಧರಂಗಕ್ಕೆ ಬಂದನು. ಅಲ್ಲಿ ಮಿಂಚಿನೊಡನೆ ಕೂಡಿದ ಮೋಡವು ಸೂರ್ಯನ ಮೇಲೆ ಕುಳಿತಿರುವಂತೆ ಸತ್ಯಭಾಮಸಹಿತನಾದ ಶ್ರೀಕೃಷ್ಣನು ಗರುಡನನ್ನೇರಿ ಆಕಾಶದಲ್ಲಿ ಕುಳಿತಿ ದ್ದಾನೆ. ಆತನ ತುಟಿಯಲ್ಲಿ ತಾಂಡವವಾಡುತ್ತಿದ್ದ ಮುಗುಳ್ನಗೆಯನ್ನು ಕಂಡು ನರಕಾ ಸುರನಿಗೆ ತಡೆಯಲಾರದಷ್ಟು ಕೋಪ ಬಂತು. ತನ್ನ ಶಕ್ತ್ಯಾಯುಧವನ್ನು ಆತನ ಮೇಲೆ ಪ್ರಯೋಗಿಸಿದ. ಅದೇ ನಿಮಿಷದಲ್ಲಿ ಅವನ ಸೈನಿಕರೂ ಸಹಸ್ರಾರು ಬಾಣಗಳನ್ನು ಕೃಷ್ಣನಿಗೆ ಗುರಿಯಿಟ್ಟು ಬಿಟ್ಟರು, ಆದರೇನು? ಶ್ರೀಕೃಷ್ಣನು ಅವುಗಳನ್ನೆಲ್ಲ ಮಧ್ಯ ಮಾರ್ಗದಲ್ಲಿಯೇ ಕತ್ತರಿಸಿ ಹಾಕಿ, ಅವರ ಮೇಲೆಲ್ಲ ತನ್ನ ಬಾಣಗಳ ಮಳೆಗರೆದನು. ಆತನನ್ನು ಹೊತ್ತಿದ್ದ ಗರುಡನೂ ಸುಮ್ಮನಿರಲಿಲ್ಲ; ತನ್ನ ರೆಕ್ಕೆ ಕೊಕ್ಕುಗಳಿಂದ ನರಕಾಸುರನ ಆನೆಗಳನ್ನೆಲ್ಲ ಕೆಡಹಿದ. ಇದನ್ನು ಕಂಡು ಕೆರಳಿದ ನರಕಾಸುರ ತನ್ನ ಶಕ್ತ್ಯಾಯುಧವನ್ನು ಗರುಡನ ಮೇಲೆ ಪ್ರಯೋಗಿಸಿದ. ಇಂದ್ರನನ್ನು ನಡುಗಿಸಿದ ಆ ಆಯುಧ ಗರುಡನಿಗೆ ಹೂವಿನಷ್ಟು ಹಗುರ ವಾಗಿತ್ತು. ಈ ಅಪಮಾನವನ್ನು ಸಹಿಸಲಾರದೆ ನರಕಾಸುರನು ಮೊದಲು ಶ್ರೀಕೃಷ್ಣನನ್ನು ಕೊಂದುಬಿಡಬೇಕೆಂದು ತನ್ನ ಶೂಲವನ್ನು ಕೈಗೆತ್ತಿ ಕೊಂಡನು. ಆದರೆ ಅದನ್ನು ಬಳಸುವ ಮೊದಲೆ ಶ್ರೀಕೃಷ್ಣನ ಚಕ್ರ ಅವನ ತಲೆಯನ್ನು ಕತ್ತರಿಸಿಹಾಕಿತು.

ನರಕಾಸುರನ ತಲೆ ನೆಲಕ್ಕುರುಳುತ್ತಲೆ ಅವನ ತಾಯಿಯಾದ ಭೂದೇವಿ ಶ್ರೀಕೃಷ್ಣನ ಬಳಿಗೆ ಬಂದು, ತನ್ನ ಮಗ ತಂದಿದ್ದ ಅದಿತಿದೇವಿಯ ಕಿವಿಯೋಲೆಗಳೆ ಮೊದಲಾದ ದೇವತೆ ಗಳ ವಸ್ತುಗಳನ್ನೆಲ್ಲ ಆತನಿಗೊಪ್ಪಿಸಿ, ಭಕ್ತಿಯಿಂದ ಆತನನ್ನು ಸ್ತುತಿಸಿದಳು. ಆಕೆಯ ಅಪೇಕ್ಷೆಯಂತೆ ಶ್ರೀಕೃಷ್ಣನು ನರಕಾಸುರನ ಮಗನಾದ ಭಗದತ್ತನನ್ನು ಕ್ಷಮಿಸಿ, ಅವನಿಗೆ ಜೀವದಾನಮಾಡಿದನು. ಅನಂತರ ಆತನು ನರಕಾಸುರನ ಅರಮನೆಗೆ ಹೋದನು. ಅಲ್ಲಿ ಸೆರೆಯಲ್ಲಿದ್ದ ಹದಿನಾರುಸಾವಿರ ರಾಜಪುತ್ರಿಯರನ್ನು ಕಂಡು ಆತನ ಕರುಳು ಕರಗಿತು. ಆ ಹೆಣ್ಣುಗಳೆಲ್ಲ ಶ್ರೀಕೃಷ್ಣನನ್ನು ತಮ್ಮ ಪತಿಯಾಗುವಂತೆ ಬೇಡಿಕೊಂಡರು. ಶ್ರೀಕೃಷ್ಣನು ಸುಂದರಿಯರಾದ ಆ ಹೆಣ್ಣುಗಳನ್ನೆಲ್ಲ ಕೂಡಿಸಿ, ದ್ವಾರಕಿಗೆ ಕಳುಹಿಸಿದನು. ಅವರ ಜೊತೆ ಯಲ್ಲಿ ನರಕನ ಅರಮನೆಯಲ್ಲಿದ್ದ ಧನಧಾನ್ಯಗಳೂ ದ್ವಾರಕಿಗೆ ಹೊರಟವು. ಎಲ್ಲಕ್ಕೂ ಹೆಚ್ಚಾಗಿ, ಅಲ್ಲಿದ್ದ ನಾಲ್ಕು ಕೊಂಬಿನ ಅರುವತ್ತು ನಾಲ್ಕು ಬಿಳಿಯ ಆನೆಗಳು–ಅವೆಲ್ಲ ಸ್ವರ್ಗದಲ್ಲಿರುವ ಐರಾವತದ ಪೀಳಿಗೆಯವು–ದ್ವಾರಕಿಗೆ ಸಾಗಿ ಹೋದವು.

ಶ್ರೀಕೃಷ್ಣನು ಪ್ರಾಗ್ಜೋತಿಷಪುರದಿಂದ ನೇರವಾಗಿ ಸ್ವರ್ಗಕ್ಕೆ ಹೋಗಿ, ಅದಿತಿದೇವಿಯ ಕಿವಿಯೋಲೆಗಳನ್ನು ದೇವೇಂದ್ರನ ಕೈಗಿತ್ತನು. ಇದನ್ನು ಕಂಡು ಅತ್ಯಾನಂದಗೊಂಡ ದೇವೇಂದ್ರ ತನ್ನ ಮಡದಿಯಾದ ಶಚಿಯೊಡನೆ ಶ್ರೀಕೃಷ್ಣ ಸತ್ಯಭಾಮೆಯರನ್ನು ಪರಿಪರಿ ಯಾಗಿ ಸತ್ಕರಿಸಿದನು. ಸತ್ಯಭಾಮೆ ಸ್ವರ್ಗದಲ್ಲಿದ್ದ ಪಾರಿಜಾತದ ಮರವನ್ನು ಕಂಡು ಮೋಹಿಸಿದಳು. ಆಕೆಯನ್ನು ಸಂತೋಷಪಡಿಸುವುದಕ್ಕಾಗಿ ಶ್ರೀಕೃಷ್ಣ ಅದನ್ನು ಕಿತ್ತು ಗರುಡನ ಮೇಲಿಟ್ಟುಕೊಂಡು ದ್ವಾರಕಿಗೆ ಹಿಂದಿರುಗಿದನು. ಪಾರಿಜಾತ ಸತ್ಯಭಾಮೆಯ ಹಿತ್ತಿಲನ್ನು ಅಲಂಕರಿಸಿತು. ಆದರೆ ದೇವೇಂದ್ರ ಇದನ್ನು ಸಹಿಸಲಿಲ್ಲ. ಆ ಮರ ಸ್ವರ್ಗ ದಲ್ಲಿಯೆ ಇರಬೇಕೆಂದು ಹೇಳಿ, ಅದಕ್ಕಾಗಿ ಶ್ರೀಕೃಷ್ಣನ ಮೇಲೆ ಯುದ್ಧಕ್ಕೂ ಸಿದ್ಧನಾದ. ಆದರೆ ಆತ ಯುದ್ಧದಲ್ಲಿ ಸೋತು, ಶ್ರೀಕೃಷ್ಣನಿಗೆ ಶರಣಾಗಿ, ಪಾರಿಜಾತವನ್ನು ತನಗೆ ಹಿಂದಿರುಗಿಸಬೇಕೆಂದು ದೈನ್ಯದಿಂದ ಬೇಡಿಕೊಂಡ. ಶ್ರೀಕೃಷ್ಣನು ನಕ್ಕು ಅದನ್ನು ಹಿಂದಕ್ಕೆ ಕೊಟ್ಟ.

ಶ್ರೀಕೃಷ್ಣನು ನರಕಾಸುರನ ಅರಮನೆಯಿಂದ ಬಂದ ಹದಿನಾರು ಸಾವಿರ ಹೆಣ್ಣುಗಳಿಗೂ ಹದಿನಾರು ಸಾವಿರ ಅರಮನೆಗಳನ್ನು ಕಟ್ಟಿಸಿಕೊಟ್ಟನು. ಅನಂತರದಲ್ಲಿ ಜಗತ್ತಿನಲ್ಲೆಲ್ಲ ಅತ್ಯಂತ ಅಪೂರ್ವವೆನ್ನುವಂತೆ ಆ ಹದಿನಾರು ಸಾವಿರ ಹೆಣ್ಣುಗಳನ್ನೂ ಒಂದೇ ಹಸೆಯ ಮೇಲೆ ಏಕಕಾಲಕ್ಕೆ ಮದುವೆಯಾದನು. ಅಷ್ಟೇ ಅಲ್ಲ; ಹದಿನಾರು ಸಾವಿರ ಶ್ರೀಕೃಷ್ಣರು, ಏಕಕಾಲದಲ್ಲಿ ಆ ಹದಿನಾರು ಸಾವಿರ ಮನೆಗಳಲ್ಲಿದ್ದ ಹದಿನಾರು ಸಾವಿರ ಸುಂದರಿಯ ರನ್ನೂ ಸಂತೋಷಪಡಿಸುತ್ತಿದ್ದರು. ಆ ದಾಂಪತ್ಯ ಜೀವನದ ಸೌಭಾಗ್ಯವನ್ನು ಯಾರು ಬಣ್ಣಿಸಬಲ್ಲರು! ಆ ಹೆಣ್ಣುಗಳ ಅನುರಾಗವೇನು, ನಗೆ ನೋಟವೇನು, ಸರಸ ವಿನೋದ ವೇನು; ಅವರು ತಮ್ಮ ಪತಿಸೇವೆಯನ್ನು ತಾವೇ ಮಾಡಬೇಕು, ದಾಸದಾಸಿಯರು ಮಾಡ ಕೂಡದು. ತಾವೆ ಆತನಿಗೆ ಪೀಠಹಾಕಬೇಕು, ತಾವೆ ಬಡಿಸಬೇಕು, ಕಾಲೊತ್ತಬೇಕು, ಗಾಳಿ ಹಾಕಬೇಕು, ಹಾಸಿಗೆ ಹಾಕಬೇಕು, ಎಣ್ಣೆಯೊತ್ತಿ ನೀರೆರೆಯಬೇಕು. ಶ್ರೀಕೃಷ್ಣನನ್ನು ಕೈಹಿಡಿದ ಹೆಣ್ಣುಗಳು ಧನ್ಯರು; ಅವರನ್ನು ಕೈಹಿಡಿದ ಆತನೂ ಧನ್ಯ!

