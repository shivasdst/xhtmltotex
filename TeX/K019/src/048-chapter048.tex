
\chapter{೪೮. ಬೃಂದಾವನದಲ್ಲಿ ಬಾಲಕ ಕೃಷ್ಣ}

ಗೋಕುಲದಲ್ಲಿ ಮೇಲಿಂದ ಮೇಲೆ ಅನಾಹುತಗಳಾಗುವುದನ್ನು ಕಂಡು ನಂದನಿಗೆ ದಿಗಿಲಾಯಿತು. ಆತನು ಹಿರಿಯರೊಡನೆಯೂ ಗೆಳೆಯರೊಡನೆಯೂ ಆಲೋಚಿಸಿ, ಅವರ ಬುದ್ಧಿವಾದದಂತೆ ತಾನಿದ್ದ ಸ್ಥಳವನ್ನು ಬಿಟ್ಟು, ಅಲ್ಲಿಗೆ ಸಮೀಪದಲ್ಲಿದ್ದ ಬೃಂದಾವನವೆಂಬ ಸ್ಥಳಕ್ಕೆ ತನ್ನ ಸಂಗಡಿಗರನ್ನೆಲ್ಲ ಕರೆದುಕೊಂಡು ಹೊರಟನು. ಎಲ್ಲರೂ ತಮ್ಮ ತಮ್ಮ ಸಾಮಾನುಗಳೊಡನೆ ಹೆಂಗಸರು ಮಕ್ಕಳನ್ನೂ ಮುಪ್ಪಿನ ಮುದುಕರನ್ನೂ ಗಾಡಿಗಳಲ್ಲಿ ತುಂಬಿಕೊಂಡು ಪ್ರಯಾಣ ಬೆಳೆಸಿದರು. ಮುಂದೆ ಕೊಂಬು ಕಹಳೆಗಳು, ಅದರ ಹಿಂದೆ ತುರುಗಳ ಮಂದೆ, ಅವುಗಳ ತರುವಾಯ ಗಾಡಿಗಳು, ಕೊನೆಯಲ್ಲಿ ಬಿಲ್ಲು ಬಾಣಗಳನ್ನು ಹಿಡಿದ ಗೋಪಾಲರು–ಹೀಗೆ ಸಾಗಿದ ಮೆರವಣಿಗೆ ಬೃಂದಾವನವನ್ನು ಸೇರುತ್ತಲೆ ಅರ್ಧಚಂದ್ರಾಕಾರವಾಗಿ ಗಾಡಿಗಳನ್ನು ನಿಲ್ಲಿಸಿ, ಅವುಗಳ ಮಧ್ಯೆ ತಮ್ಮ ತುರುಗಳಿಗೆ ಅನುಕೂಲವಾದ ವಾಸಸ್ಥಾನವನ್ನು ಕಲ್ಪಿಸಿದರು. ಅನಂತರ ತಮ್ಮ ವಸತಿಗೂ ತಕ್ಕ ಅನುಕೂಲಗಳನ್ನು ಮಾಡಿಕೊಂಡು ಅಲ್ಲಿಯೆ ನೆಲಸಿದರು.

ಬಾಲಕರಾದ ಬಲರಾಮ ಕೃಷ್ಣರಿಗೆ ಬೃಂದಾವನದ ಸುಂದರ ಸನ್ನಿವೇಶ ಬಹು ಸಂತೋಷವನ್ನುಂಟುಮಾಡಿತು. ಸುತ್ತಲೂ ಸುಂದರವಾದ ಕಾಡು, ಹತ್ತಿರದಲ್ಲಿಯೇ ಹಸಿರನ್ನು ಹೊತ್ತ ಗೋವರ್ಧನ ಪರ್ವತ, ಅದರ ಪಕ್ಕದಲ್ಲಿಯೇ ವಿಸ್ತಾರವಾದ ಮರಳು ದಿಣ್ಣೆಯ ಮಧ್ಯೆ ಹರಿಯುತ್ತಿರುವ ಯಮುನಾ ನದಿ; ಹುಡುಗರಿಗೆ ವಿಹರಿಸುವುದಕ್ಕೆ ಬೇಕಾದಷ್ಟು ಅವಕಾಶವಿತ್ತು. ಯಮುನಾ ತೀರದಲ್ಲಿ ಆಟವಾಡುವುದೆಂದರೆ ಶ್ರೀಕೃಷ್ಣನಿ ಗಂತೂ ಜಗತ್ತೆ ಮರೆತುಹೋಗುತ್ತಿತ್ತು. ಒಮ್ಮೆ ಆತ ಮಕ್ಕಳೊಡನೆ ಅಲ್ಲಿ ಆಟವಾಡು ತ್ತಿರಲು ಊಟದ ಹೊತ್ತಾಯಿತು. ಆದರೂ ಮನೆಗೆ ಹಿಂದಿರುಗಲಿಲ್ಲ. ರೋಹಿಣಿ ಬಂದು ‘ಅಪ್ಪ, ಹೊತ್ತಾಯಿತು ಊಟಕ್ಕೆ ಬನ್ನಿರೊ’ ಎಂದು ಮಕ್ಕಳನ್ನು ಕರೆದಳು. ಆಕೆಯ ಮಾತು ಅವರ ಕಿವಿಗೆ ಬೀಳಲಿಲ್ಲ. ಆಕೆ ಹಿಂದಿರುಗಿ, ಯಶೋದೆಯನ್ನು ಕಳುಹಿಸಿದಳು. ಆಕೆ ಅಲ್ಲಿಗೆ ಬಂದು ‘ಮಗೂ, ಕೃಷ್ಣ, ಬಾಪ್ಪ ಚಿನ್ನ, ಆಟ ಸಾಕು, ಮಮ್ಮು ತಿನ್ನುವಂತೆ ಬಾ; ಬಾಮ್ಮ, ತುಂಬ ಹಸಿವಾಗಿದೆ, ಊಟ ಮಾಡಿ ಬಂದು ಆಟವಾಡುವಿಯಂತೆ, ಬಾ ನನ್ನ ದೇವರು; ನೋಡು, ಹಸಿದು ನಿನ್ನ ಹೊಟ್ಟೆ ಬೆನ್ನಿಗೆ ಹತ್ತಿಹೋಗಿದೆ, ಮುಖವೆಲ್ಲ ಬಾಡಿಹೋಗಿದೆ, ಬಾರೋ ನನ್ನ ಕಂದ’ ಎಂದು ರಮಿಸಿದಳು. ಮಗ ತನ್ನ ಕಡೆ ಕಣ್ಣೆತ್ತಿಯೂ ನೋಡದಿರು ವುದನ್ನು ಕಂಡು ಆಕೆ ಬಲರಾಮನನ್ನು ಕೂಗಿ ಕರೆದಳು. ‘ಮಗು ಬಲರಾಮ, ನೀನು ಬಾಪ್ಪ, ಆ ತುಂಟ ಅಲ್ಲೆ ಬಿದ್ದಿರಲಿ. ಅವರಪ್ಪ ಅವನಿಗಾಗಿ ಕಾದುಕೊಂಡಿದ್ದಾರೆ, ಪಾಪ. ಹೋಗಲಿ, ನೀನು ಬಾರಣ್ಣ’ ಎಂದಳು. ಕೃಷ್ಣನ ಜೊತೆ ಬಿಟ್ಟು ಅವನೆಲ್ಲಿ ಬರುತ್ತಾನೆ? ಇದನ್ನು ಕಂಡು ಆಕೆಗೆ ರೇಗಿಹೋಯಿತು. ಆಕೆ ಅಲ್ಲಿದ್ದ ಹುಡುಗರನ್ನೆಲ್ಲ, ‘ಏ ಹುಡುಗರಾ, ಹೋಗುತ್ತೀ ರಿಲ್ಲೊ ನಿಮ್ಮ ನಿಮ್ಮ ಮನೆಗಳಿಗೆ? ನಿಮಗೇನು ಹೇಳುವವರು ಕೇಳುವವರು ಯಾರೂ ಇಲ್ಲವೇನು? ಹೊರಡಿರಿ ಮೊದಲು ಇಲ್ಲಿಂದ’ ಎಂದು ಗದರಿಸಿದಳು. ಒಡನೆಯೇ ಹುಡುಗ ರೆಲ್ಲರೂ ಬುಡುಬುಡು ಓಡಿಹೋದರು. ಆಗ ಆಕೆ ಅಲ್ಲಿಯೇ ಇನ್ನೂ ನಿಂತಿದ್ದ ಕೃಷ್ಣನ ಕೈಯನ್ನು ಹಿಡಿದುಕೊಂಡು, ‘ಅಯ್ಯಯ್ಯೊ, ಬೀದಿಯ ಮಣ್ಣೆಲ್ಲ ನಿನ್ನ ಮೈಯಲ್ಲೆ ಇದೆ ಯಲ್ಲೊ! ಹೀಗಾದರೆ ಏನು ಗತಿ? ನೋಡಿದವರೆಲ್ಲ ನಗುವುದಿಲ್ಲವೆ? ಬಾ, ಸ್ನಾನಮಾಡಿಸಿ, ಹೊಸಬಟ್ಟೆ ಹಾಕಿ, ಹೂ ಮುಡಿಸಿ ಅಲಂಕಾರ ಮಾಡುತ್ತೀನಿ. ಊಟ ಮಾಡಿ ಬಂದು ಬೇಕಾದ ಹಾಗೆ ಆಡಿಕೊ’ ಎಂದು ರಮಿಸಿ, ಅವನನ್ನೂ ಬಲರಾಮನನ್ನೂ ಕೈಹಿಡಿದು ಮನೆಗೆ ಕರೆತಂದಳು.

ಬಲರಾಮ ಕೃಷ್ಣರು ಮತ್ತಷ್ಟು ದೊಡ್ಡವರಾದರು. ಅವರೀಗ ಇತರ ಗೋಪಾಲಕ ರೊಡನೆ ದನದ ಮಂದೆಯನ್ನು ಮೇಯಿಸುವುದಕ್ಕಾಗಿ ಅಡವಿಗೆ ಹೋಗುವರು. ದನಗಳನ್ನು ಮೇಯುವುದಕ್ಕೆ ಬಿಟ್ಟು, ಬಾಲಕರೆಲ್ಲ ನಾನಾ ಬಗೆಯಾದ ಆಟಗಳಲ್ಲಿ ಮಗ್ನರಾಗುವರು. ಒಮ್ಮೆ ಕೊಳಲೂದುವರು, ಒಮ್ಮೆ ಕೋಲಾಟವಾಡುವರು, ಇನ್ನೊಮ್ಮೆ ಕಾಡುಹಣ್ಣು ಗಳನ್ನು ಕಿತ್ತು ಸ್ಪರ್ಧೆಯಿಂದ ದೂರಕ್ಕೆ ಎಸೆಯುವರು, ಕಂಬಳಿಯನ್ನು ಹೊದ್ದು ದನ ಗಳಂತೆ ನಾಲ್ಕು ಕಾಲುಗಳಲ್ಲಿ ನಡೆಯುತ್ತಾ ಒಬ್ಬರನ್ನೊಬ್ಬರು ತಿವಿದಾಡುವರು, ನೆಗೆವರು, ಕೆಲವರು ಹಕ್ಕಿಗಳಂತೆ ಹಾಡಿ ನಲಿವರು. ಹೀಗೆ ದಿನಗಳು ಉರುಳಿ ಹೋಗುತ್ತಿರಲು ಒಂದು ದಿನ ಅವರ ಮಂದೆಯಲ್ಲಿ ಹೊಸದೊಂದು ಕರು ಬಂದು ಸೇರಿತು. ಶ್ರೀಕೃಷ್ಣ ಅದನ್ನು ನೋಡಿದ. ಮೆಲ್ಲಗೆ ಅವನು ಅದರ ಹಿಂದೆ ಹೋಗಿ ನಿಂತ. ಕರುವೂ ಅವನನ್ನು ನೋಡಿತು. ಅದು ಅವನನ್ನು ಇರಿದು ಕೊಲ್ಲಬೇಕೆಂದು ಹೊಂಚು ಹಾಕಿತು. ಅದರ ಸಂಚನ್ನು ತಿಳಿದ ಶ್ರೀಕೃಷ್ಣ ಅದರ ಹಿಂದಿನ ಕಾಲೆರಡನ್ನೂ ತಿರುಗಿಸಿ, ಹತ್ತಿರದಲ್ಲಿದ್ದ ಬೇಲದಮರಕ್ಕೆ ಎಸೆದ. ಮರದಿಂದ ದಪದಪ ಹಣ್ಣುಗಳುದುರಿದವು. ಅದರ ಜೊತೆಯಲ್ಲಿಯೇ ದೊಡ್ಡ ರಾಕ್ಷಸ ದೇಹ ನೆಲಕ್ಕೆ ಬಿತ್ತು. ಹುಡುಗರೆಲ್ಲ ಈ ವಿನೋದವನ್ನು ಕಂಡು ಗಹಗಹಿಸಿ ನಗುತ್ತಾ ‘ಅಯ್ಯಯ್ಯೊ, ಇವನು ಕರುವಲ್ಲ, ಕ್ರೂರರಾಕ್ಷಸ’ ಎಂದು ಚಪ್ಪಾಳೆ ತಟ್ಟಿಕೊಂಡು ಕುಣಿ ದಾಡಿದರು. ಅಹುದು ಅವನು ನಿಜವಾಗಿಯೂ ಕಂಸ ಕಳುಹಿಸಿದ್ದ ‘ವತ್ಸಕ’ನೆಂಬ ರಕ್ಕಸನೇ. ಕೃಷ್ಣನನ್ನು ಕೊಲ್ಲಲೆಂದು ಬಂದವನು ಆತನಿಂದ ಹತನಾಗಿ ಬಿದ್ದ.

ಮತ್ತೊಂದು ದಿನ ಬಲರಾಮ ಕೃಷ್ಣರು ಇತರ ಗೊಲ್ಲರ ಹುಡುಗರೊಡನೆ ದನಗಳನ್ನು ಮೇಯಿಸುತ್ತಾ ಆಟವಾಡುತ್ತಿದ್ದವರು, ಬಾಯಾರಿಕೆಯಿಂದ ನೀರು ಕುಡಿಯುವುದಕ್ಕಾಗಿ ಯಮುನಾ ನದಿಗೆ ಹೋದರು. ಅಲ್ಲಿ ನೀರು ಕುಡಿದು ದಡಕ್ಕೆ ಏರಿದಾಗ ಅವರಿಗೆ ಒಂದು ದೊಡ್ಡ ಬಕಪಕ್ಷಿ ಕಾಣಿಸಿತು. ಅದು ಇದ್ದಕ್ಕಿದ್ದಂತೆ ನೇರವಾಗಿ ಕೃಷ್ಣನ ಬಳಿಗೆ ಹಾರಿಬಂದು ಅವನನ್ನು ನುಂಗುವುದಕ್ಕೆ ಬಾಯಿಹಾಕಿತು. ಆ ರಕ್ಕಸ ಹಕ್ಕಿಯನ್ನು ಕಂಡು ಗೋಪಾಲಕ ರೆಲ್ಲ ಗಡಗಡ ನಡುಗಿದರು. ಆದರೆ ಶ್ರೀಕೃಷ್ಣನು ಸ್ವಲ್ಪವೂ ಅಳುಕದೆ ಅದರ ಕೊಕ್ಕುಗಳನ್ನು ತನ್ನ ಎರಡು ಕೈಗಳಲ್ಲಿಯೂ ಹಿಡಿದುಕೊಂಡು, ಹಂಚಿಯ ಕಡ್ಡಿಯನ್ನು ಸೀಳಿಹಾಕುವಂತೆ ಅದನ್ನು ಸೀಳಿಹಾಕಿದನು. ಒಡನೆಯೆ ರಕ್ಕಸದೇಹದ ಎರಡು ತುಂಡುಗಳು ನೆಲದ ಮೇಲೆ ಬಿದ್ದವು. ಹೆದರಿ ನಿಂತಿದ್ದ ಗೋಪಾಲರು ಈ ಪವಾಡವನ್ನು ಕಂಡು ಚಪ್ಪಾಳೆ ತಟ್ಟುತ್ತಾ ‘ಇದು ಬಕಪಕ್ಷಿಯಲ್ಲವಪ್ಪ, ಬಕಾಸುರ’ ಎಂದು ಕುಣಿದಾಡಿದರು. ಅವರು ಹೇಳಿದುದು ಸುಳ್ಳೇನೂ ಅಲ್ಲ. ಕಂಸನ ಗೆಳೆಯನಾದ ಬಕನೆಂಬ ರಕ್ಕಸನು ಶ್ರೀಕೃಷ್ಣನನ್ನು ಕೊಲ್ಲ ಲೆಂದು ಬಕಪಕ್ಷಿಯಾಗಿ ಬಂದು ಅವನಿಂದ ಹತನಾಗಿದ್ದ.

ಇನ್ನೊಂದು ದಿನ ಶ್ರೀಕೃಷ್ಣಬಲರಾಮರೊಡನೆ ಗೋಪಾಲಕರೆಲ್ಲ ದನಗಳನ್ನು ಮೇಯಿ ಸುತ್ತಾ ಅರಣ್ಯದಲ್ಲಿ ಬಹುದೂರ ಹೋದರು. ಮನಬಂದಂತೆ ಅವರೆಲ್ಲ ಸುತ್ತಾಡುತ್ತಿರು ವಾಗ ದೊಡ್ಡದೊಂದು ಹೆಬ್ಬಾವು ಅಡ್ಡ ಮಲಗಿರುವುದು ಅವರಿಗೆ ಕಾಣಿಸಿತು. ಅದು ಹೆಬ್ಬಾವೆಂದರೆ ಸಾಮಾನ್ಯವಾದ ಹೆಬ್ಬಾವಲ್ಲ. ಅಘಾಸುರನೆಂಬ ಭಯಂಕರ ರಾಕ್ಷಸ ಆ ರೂಪವನ್ನು ತಾಳಿದ್ದ. ಅಮೃತವನ್ನು ಕುಡಿದ ದೇವತೆಗಳು ಕೂಡ ಅವನನ್ನು ಕಂಡರೆ ಪ್ರಾಣಭಯದಿಂದ ನಡುಗುತ್ತಿದ್ದರು. ಶ್ರೀಕೃಷ್ಣನು ಕೊಂದುಹಾಕಿದ ಬಕಾಸುರನ ತಮ್ಮ ಅವನು. ತನ್ನ ಅಣ್ಣನನ್ನು ಕೊಂದ ಶ್ರೀಕೃಷ್ಣನನ್ನು ನುಂಗಲೆಂದೆ ಅವನು ಅಲ್ಲಿ ಕಾದಿದ್ದುದು. ಗಾವುದಗಳಷ್ಟು ಉದ್ದ, ಬೆಟ್ಟದಷ್ಟು ದಪ್ಪವಿದ್ದ ಆ ಹೆಬ್ಬಾವು ಬಾಯಿ ತೆರೆದು ಮಲಗಿರಲು, ಗೋಪಾಲಕರೆಲ್ಲ ಅದನ್ನು ಬೆಟ್ಟದ ಗವಿಯೆಂದು ಭಾವಿಸಿದರು. ತಮಗೆ ಆಟವಾಡಲು ಸೊಗಸಾದ ಸ್ಥಳ ಸಿಕ್ಕಿತೆಂದುಕೊಂಡು, ಆ ಬಾಲಕರೆಲ್ಲ ಅದರ ಬಾಯೊಳಗೆ ಕಾಲಿಟ್ಟರು. ಶ್ರೀಕೃಷ್ಣನು ಅವರನ್ನು ತಡೆಯುವಷ್ಟರಲ್ಲಿ ಆ ಹೆಬ್ಬಾವು ಉಸಿರನ್ನೆಳೆದು ಕೊಳ್ಳಲು, ಅವರೆಲ್ಲ ಅದರ ಹೊಟ್ಟೆಯನ್ನು ಸೇರಿದರು. ಕಾರ್ಯ ಕೈಮಿಂಚಿದುದನ್ನು ಕಂಡು ಶ್ರೀಕೃಷ್ಣ ಮಮ್ಮಲ ಮರುಗಿದನು. ತನ್ನ ಜೊತೆಯಲ್ಲಿ ಬಂದವರನ್ನು ಹಾವಿಗೆ ಆಹುತಿಗೊಟ್ಟು ಆತನು ಹಿಂದಿರುಗುವುದಕ್ಕೆ ಸಾಧ್ಯವೆ? ಆತನು ಆ ಹಾವಿನ ಬಾಯನ್ನು ಹೊಕ್ಕನು. ಆತನು ಅದರ ಕುತ್ತಿಗೆಯ ತನಕ ನಿರ್ಭಯವಾಗಿ ಪ್ರವೇಶಿಸಿ, ಅಲ್ಲಿ ನಿಂತವನೆ ತ್ರಿವಿಕ್ರಮನಂತೆ ಬೆಳೆಯುತ್ತ ಹೋದನು. ಆಗ ಆ ಹಾವಿನ ಗಂಟಲು ಕಟ್ಟಿತು. ಅದರ ಉಸಿರು ನಿಂತಿತು. ಅದು ಪ್ರಾಣಸಂಕಟದಿಂದ ಹೊರಳಾಡಿತು. ಅದರ ದೇಹದಲ್ಲಿ ತುಂಬಿ ಕೊಂಡ ಗಾಳಿ ಹೊರ ಬರುವುದಕ್ಕೆ ಅವಕಾಶವಿಲ್ಲದೆ, ಅದರ ಹೊಟ್ಟೆ ಒಡೆದುಹೋಯಿತು. ಅಲ್ಲಿದ್ದ ಗಾಳಿಯೊಡನೆ ಅದರ ಪ್ರಾಣವೂ ಹಾರಿಹೋಯಿತು. ಗಣಪತಿಯ ಹೊಟ್ಟೆಯಿಂದ ಕಡುಬುಗಳು ಕೆಳಗುರುಳುವಂತೆ ಗೋಪಾಲಕರೆಲ್ಲ ಅದರ ಹೊಟ್ಟೆಯಿಂದ ಕೆಳಕ್ಕೆ ಬಿದ್ದರು. ಗಾಬರಿಯಿಂದ ಕಂಗಾಲಾಗಿದ್ದ ಅವರನ್ನೆಲ್ಲ ಶ್ರೀಕೃಷ್ಣನು ಸಮಾಧಾನಮಾಡಿ, ಇನ್ನೊಮ್ಮೆ ಹಾಗೆ ಎಚ್ಚರದಪ್ಪಿ ನುಗ್ಗದಂತೆ ಬುದ್ಧಿವಾದ ಹೇಳಿದನು.

