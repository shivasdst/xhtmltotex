
\chapter{೯೦. ವೇಷಧಾರಿ ಒನಕೆಯನ್ನು ಹೆತ್ತ}

ಶ್ರೀಹರಿಯು ಕೃಷ್ಣಾವತಾರವೆತ್ತಿ ನೂರಿಪ್ಪತ್ತೈದುವರ್ಷಗಳು ಸಂದವು. ಲೆಕ್ಕವಿಲ್ಲದಷ್ಟು ಜನ ರಕ್ಕಸರೂ ಪಾಪಿ ಕ್ಷತ್ರಿಯರೂ ನಾಮಾವಶೇಷವಾಗಿ ಹೋಗಿ, ಭೂಭಾರಹರಣ ಕಾರ್ಯ ಬಹುಮಟ್ಟಿಗೆ ಮುಗಿದಂತಾಗಿತ್ತು. ಶ್ರೀಕೃಷ್ಣನು ತನ್ನ ಮನಸ್ಸಿನಲ್ಲಿಯೆ “ಆಹಾ, ಈ ಯಾದವರೆಲ್ಲ ನಿಶ್ಶೇಷರಾಗದ ಹೊರತು ಭೂಭಾರಹರಣಕಾರ್ಯ ಕೊನೆ ಮುಟ್ಟಿದಂತಾಗು ವುದಿಲ್ಲ. ಈ ಪಾಪಿಗಳು ನನ್ನ ಆಶ್ರಯದಲ್ಲಿರುವುದರಿಂದ ದುರ್ಜಯರಾಗಿಹೋಗಿದ್ದಾರೆ. ಅಹಂಕಾರದಿಂದ ಬೀಗಿಬಿರಿಯುತ್ತಿರುವ ಈ ಮದಾಂಧರ ನಾಶಕ್ಕೆ ಇರುವುದೊಂದೇ ದಾರಿ–ಸ್ವಯೂಥಕಲಹ. ಬಿದಿರಿನಲ್ಲಿ ಹುಟ್ಟಿದ ಬೆಂಕಿ ಮೆಳೆಯನ್ನೆಲ್ಲ ಸುಡುವಂತೆ, ಇವರಿವರಲ್ಲಿ ಹುಟ್ಟಿದ ಜಗಳ ಈ ಪೀಳಿಗೆಯನ್ನೆ ನಾಶಮಾಡುವಂತಾಗಬೇಕಾಗಿದೆ. ಇಷ್ಟಾಗಿ ಹೋದರೆ ನಾನು ನಿಶ್ಚಿಂತನಾಗಿ ವೈಕುಂಠಕ್ಕೆ ಹಿಂದಿರುಗಬಹುದು. ಅಷ್ಟರೊಳ ಗಾಗಿ ನಾನು ಭಕ್ತರಿಗೆ ಆತ್ಮತತ್ವವನ್ನು ಉಪದೇಶಮಾಡಿ ಮುಗಿಸಬೇಕು” ಎಂದು ಕೊಂಡನು. ಯೋಗೇಶ್ವರನಾದ ಆತನ ಉದ್ದೇಶ ನೆರವೇರದೆ ನಿಲ್ಲುತ್ತದೆಯೆ? ಯಾದವ ವಂಶದ ಪ್ರಳಯಕ್ಕೆ ಬೀಜರೂಪವಾದ ಘಟನೆಯೊಂದು ಬಹುಬೇಗ ನಡೆಯಿತು.

ದ್ವಾರಕಾಪಟ್ಟಣದ ಹತ್ತಿರದಲ್ಲಿಯೆ ಪಿಂಡಾರಕವೆಂಬ ಒಂದು ತೀರ್ಥಕ್ಷೇತ್ರವಿತ್ತು. ಅಲ್ಲಿಗೆ ಅಸಿತ, ಕಣ್ವ, ದುರ್ವಾಸ, ಅತ್ರಿ, ವಸಿಷ್ಠ, ವಾಮದೇವ ಮೊದಲಾದ ಮಹರ್ಷಿಗಳು ಯಾತ್ರೆಗೆಂದು ಬಂದರು. ಆ ಕ್ಷೇತ್ರದ ಬಳಿಯಲ್ಲಿಯೆ ವಿನೋದಕೇಳಿಗೆಂದು ನೆರೆದಿದ್ದ ಯಾದವ ಯುವಕರಿಗೆ ಆ ಪುಷಿಗಳನ್ನೆಲ್ಲ ಗೇಲಿಗೆಬ್ಬಿಸಬೇಕೆಂಬ ದುರ್ಬುದ್ಧಿ ಹುಟ್ಟಿತು. ‘ವಿನಾಶಕಾಲಕ್ಕೆ ವಿಪರೀತ ಬುದ್ಧಿ.’ ಅವರೆಲ್ಲ ಸೇರಿಕೊಂಡು, ಜಾಂಬವತಿಯ ಮಗನಾದ ಸಾಂಬನಿಗೆ ಗರ್ಭಿಣಿಯಾದ ಹೆಂಗಸಿನಂತೆ ವೇಷಹಾಕಿ, ಆ ಪುಷಿಗಳ ಬಳಿಗೆ ಕರೆತಂದು ‘ಸ್ವಾಮಿ, ಪುಷಿಗಳೆ, ಈ ನಮ್ಮ ಹುಡುಗಿಗೆ ಗಂಡುಮಗು ಬೇಕಂತೆ; ತಾನೆ ಬಂದು ನಿಮ್ಮನ್ನು ಕೇಳುವುದಕ್ಕೆ ಅವಳಿಗೆ ನಾಚಿಕೆ; ಆಕೆಯ ಪರವಾಗಿ ನಾವು ನಿಮ್ಮನ್ನು ಬೇಡಿಕೊಳ್ಳು ತ್ತಿದ್ದೇವೆ. ಹೇಳಿ, ಈ ಹುಡುಗಿಗೆ ಎಂತಹ ಮಗು ಹುಟ್ಟುತ್ತದೆ?’ ಎಂದು ಕೇಳಿದರು. ತಾವು ಬೆಂಕಿಯ ಕೂಡ ಸರಸವಾಡುತ್ತಿರುವೆವೆಂದು ಆ ದುರ್ಮಾರ್ಗರಿಗೇನು ಗೊತ್ತು? ಸರ್ವಜ್ಞ ರಾದ ಮಹರ್ಷಿಗಳು ಅವರ ದುರಹಂಕಾರವನ್ನು ಕಂಡು ಕೆರಳಿ ‘ಎಲ ಮದಾಂಧರಿರಾ, ಇವಳು ನಿಮ್ಮ ವಂಶವನ್ನು ನಿರ್ಮೂಲಮಾಡುವ ಒನಕೆಯೊಂದನ್ನು ಹೆರುತ್ತಾಳೆ’ ಎಂದು ಹೇಳಿ, ಕೋಪದಿಂದ ಬುಸುಗುಟ್ಟುತ್ತಾ ಅಲ್ಲಿಂದ ಹೊರಟುಹೋದರು.

ಬರಸಿಡಿಲಿನಂತಹ ಪುಷಿಶಾಪವನ್ನು ಕೇಳಿ ಯದುಕುಮಾರರ ಜೀವಗಳು ತತ್ತರಿಸಿ ತಲ್ಲಣಗೊಂಡುವು. ಅವರು ತಮ್ಮ ಅವಿವೇಕಕ್ಕಾಗಿ ಪಶ್ಚಾತ್ತಾಪಪಡುತ್ತಾ ಸಾಂಬ ಕುಮಾರನ ವೇಷವನ್ನು ಕಿತ್ತೆಸೆಯಲು, ಅವನ ಹೊಟ್ಟೆಯ ಬಳಿಯಲ್ಲಿ ಕಬ್ಬಿಣದ ಗೊಣಸು ಹಾಕಿದ ಒಂದು ಒನಕೆಯಿತ್ತು. ಅವರು ಗಡಗಡ ನಡುಗುತ್ತಾ, ಆ ಒನಕೆಯೊಡನೆ ರಾಜಾಸ್ಥಾನಕ್ಕೆ ಹೋಗಿ, ಉಗ್ರಸೇನ ಮಹಾರಾಜನ ಮುಂದೆ ನಡೆದುದನ್ನೆಲ್ಲ ನಿವೇದಿಸಿ ದರು. ಪುಷಿ ಶಾಪವನ್ನೂ ಅದರ ಪ್ರತ್ಯಕ್ಷ ಪ್ರಮಾಣದಂತಿದ್ದ ಒನಕೆಯನ್ನೂ ಕಂಡು ರಾಜಸಭೆ ಭಯದಿಂದ ತಲ್ಲಣಿಸಿತು. ಉಗ್ರಸೇನ ಮಹಾರಾಜನು ಬಹುಕಾಲ ಮೌನದಿಂದ ಚಿಂತಿಸಿ ಕೊನೆಗೆ ಆ ಒನಕೆಯನ್ನು ಪುಡಿಮಾಡಿಸಿ, ಸಮುದ್ರದಲ್ಲಿ ಕದಡಿಸಿದನು. ಅದರ ಗೊಣಸನ್ನೂ ಆದಷ್ಟು ಅರೆಸಿ, ಉಳಿದ ಸಣ್ಣ ತುಂಡನ್ನು ಸಮುದ್ರದಲ್ಲಿ ಬಿಸಾಡಿದ. ಇದನ್ನು ಕೇಳಿದರೆ ಶ್ರೀಕೃಷ್ಣನೇನೆನ್ನುವನೊ ಎಂಬ ಭಯದಿಂದ, ಆತನಿಗೆ ಈ ಸುದ್ದಿಯನ್ನೇ ತಿಳಿಸಲಿಲ್ಲ, ಆದರೆ ಸರ್ವಾಂತರ್ಯಾಮಿಯಾದ ಶ್ರೀಕೃಷ್ಣನ ಸಂಕಲ್ಪಕ್ಕೆ ಅನುಸಾರಿ ಯಾಗಿಯೆ ಯದುಕುಮಾರರ ಅವಿವೇಕವೂ ಮಹರ್ಷಿಶಾಪವೂ ನಡೆದಿದ್ದುವೆಂಬುದನ್ನು ಆ ಮಂಕುಗಳೇನು ಬಲ್ಲರು? ಅರೆದು ನೀರಿನಲ್ಲಿ ಕಲೆಸಿದ ಒನಕೆ ಪ್ರಭಾಸಕ್ಷೇತ್ರದ ಸಮುದ್ರ ತೀರದಲ್ಲಿ ಜೊಂಡುಹುಲ್ಲಾಗಿ ಬೆಳೆದು, ಅದರಿಂದ ಹೊಡೆದಾಡಿಯೆ ಯಾದವ ರೆಲ್ಲ ಮುಂದೆ ನಾಶವಾಗುವುದು. ಅವರು ಅರೆದು ಸಮುದ್ರಕ್ಕೆ ಒಗೆದ ಗೊಣಸಿನ ಚೂರನ್ನು ಮೀನೊಂದು ನುಂಗಿತು. ಅದು ಒಬ್ಬ ಬೇಡನ ಬಲೆಗೆ ಬೀಳಲು, ಅವನು ಅದನ್ನು ಸೀಳಿ, ಅದರ ಹೊಟ್ಟೆಯಲ್ಲಿದ್ದ ಆ ಚೂರನ್ನು ಚೂಪಾಗಿ ಮಾಡಿಕೊಂಡು, ತನ್ನ ಬಾಣದ ಮೊನೆಯಾಗಿ ಮಾಡಿಕೊಂಡನು.

ಯದುವಂಶದ ಸರ್ವನಾಶಕ್ಕೆ ಬೀಜವನ್ನು ನಾಟಿದ ಶ್ರೀಕೃಷ್ಣನು ತನ್ನ ಅವತಾರಕಾರ್ಯದ ಕೊನೆಯ ಘಟ್ಟವನ್ನು ಮುಟ್ಟಿದುದನ್ನು ಕಂಡು ಚತುರ್ಮುಖಬ್ರಹ್ಮನು ರುದ್ರನೇ ಮೊದಲಾದ ದೇವತೆಗಳನ್ನೂ ಸನಕ ಸನಂದಾದಿ ಮಹರ್ಷಿಗಳನ್ನೂ ತನ್ನ ಜೊತೆಯಲ್ಲಿ ಕರೆದುಕೊಂಡು ಶ್ರೀಕೃಷ್ಣನ ಬಳಿಗೆ ಬಂದನು. ಅವರೆಲ್ಲರೂ ಅನನ್ಯ ಭಕ್ತಿಯಿಂದ ಆತ ನನ್ನು ಸ್ತುತಿಸಿದ ಮೇಲೆ ಬ್ರಹ್ಮನು ‘ಪ್ರಭು, ನಮ್ಮ ಪ್ರಾರ್ಥನೆಯಂತೆ ನೀನು ಮಾನವ ರೂಪದಿಂದ ಅವತರಿಸಿಬಂದು, ಭೂಭಾರಹರಣ ಮಾಡಿ ಕಾರ್ಯವನ್ನು ಮುಗಿಸಿರುವೆ. ಭೂದೇವಿ ತೃಪ್ತಳಾಗಿರುವಳು. ಸರ್ವೇಶ್ವರನಾದ ನಿನ್ನ ಇಷ್ಟಕ್ಕೆ ವಿರೋಧವಲ್ಲವಾದರೆ ನೀನಿನ್ನು ಪರಂಧಾಮಕ್ಕೆ ತೆರಳಬಹುದು’ ಎಂದನು. ಆಗ ಶ್ರೀಕೃಷ್ಣಭಗವಾನನು ‘ಬ್ರಹ್ಮ ದೇವ, ನಿನ್ನ ಅಪೇಕ್ಷೆ ನನ್ನ ಅಪೇಕ್ಷೆಯೆ. ನಾನು ಇಲ್ಲಿಂದ ಪರಂಧಾಮಕ್ಕೆ ಹೊರಡುವ ಮುನ್ನ ಈ ಯಾದವಕುಲವೊಂದನ್ನು ಸಮೂಲವಾಗಿ ನಾಶಮಾಡಬೇಕಾಗಿದೆ. ಧೈರ್ಯ ಶೌರ್ಯ ಐಶ್ವರ್ಯಗಳಿಂದ ಕೊಬ್ಬಿಹೋಗಿರುವ ಈ ಯಾದವರು ಕೇವಲ ನನ್ನ ಸಂಕಲ್ಪದಿಂದ ಇನ್ನೂ ಮೇರೆ ಮೀರಿಲ್ಲ. ಇವರನ್ನು ಹಾಗೆಯೆ ಬಿಟ್ಟರೆ ಲೋಕಕಂಟಕರಾಗುತ್ತಾರೆ. ಇವರನ್ನು ಇಲ್ಲದಂತೆ ಮಾಡಿ, ಆಮೇಲೆ ನಾನು ಹೊರಟುಬರುತ್ತೇನೆ. ಇದಕ್ಕೆ ಹೆಚ್ಚು ಕಾಲವೇನೂ ಬೇಡ. ಹೆಣ್ಣುವೇಷವನ್ನು ಧರಿಸಿದ ಸಾಂಬ ಪುಷಿಶಾಪದಿಂದ ಹೆತ್ತಿರುವ ಒನಕೆಯೆ ಈ ವಂಶವನ್ನು ನಿರ್ವಂಶವಾಗಿ ಮಾಡುವ ಆಯುಧವಾಗುತ್ತದೆ’ ಎಂದು ಹೇಳಿದನು. ಈ ಮಾತನ್ನು ಕೇಳಿ ಬ್ರಹ್ಮನೇ ಮೊದಲಾದವರು ಭಕ್ತಿಯಿಂದ ಆತನಿಗೆ ನಮಸ್ಕರಿಸಿ, ಸ್ವಸ್ಥಳ ಗಳಿಗೆ ಹಿಂದಿರುಗಿದರು.

ಬ್ರಹ್ಮನು ಇತ್ತ ಹಿಂದಿರುಗುತ್ತಲೆ ಅತ್ತ ದ್ವಾರಕಿಯಲ್ಲಿ ಭಯಂಕರವಾದ ಉತ್ಪಾತಗಳಾ ದವು. ಭೂಮಿ ನಡುಗಿತು, ದಿಕ್ಕುಗಳು ಕೆಂಡವನ್ನು ಕಾರಿದವು, ಸೂರ್ಯನ ಸುತ್ತ ಪರಿವೇಷ ಕಟ್ಟಿತು. ಇದನ್ನು ಕಂಡು ಪುರಜನರೆಲ್ಲ ಭಯದಿಂದ ತಲ್ಲಣಗೊಳ್ಳುತ್ತಿರಲು, ಶ್ರೀಕೃಷ್ಣನು ಯಾದವ ವೃದ್ಧರನ್ನೆಲ್ಲ ಕುರಿತು ‘ಪೂಜ್ಯರೆ, ಈ ಉತ್ಪಾತಗಳನ್ನು ಕಂಡರೆ ಯಾವುದೊ ದೊಡ್ಡ ಗಂಡಾಂತರವೊಂದು ನಮಗೆ ಕಾದಿದೆಯೆಂದು ತೋರುತ್ತದೆ. ಇದಕ್ಕೆ ಸರಿಯಾಗಿ, ನಮ್ಮ ಹುಡುಗರ ಅವಿವೇಕದಿಂದ ಬ್ರಾಹ್ಮಣ ಶಾಪವೊಂದು ಬೇರೆ ನಮಗೆ ಗಂಟು ಬಿದ್ದಿದೆ. ಜೀವದ ಆಶೆ ಇರುವವರಾರೂ ಇನ್ನು ಇಲ್ಲಿರುವುದು ಸರಿಯಲ್ಲ. ನಾವೆಲ್ಲ ಆದಷ್ಟು ಬೇಗ ಪ್ರಭಾಸಕ್ಷೇತ್ರಕ್ಕೆ ಹೋಗೋಣ. ಅದು ಬಹು ಪುಣ್ಯಸ್ಥಳ. ಹಿಂದೆ ದಕ್ಷನ ಶಾಪದಿಂದ ಕ್ಷಯರೋಗಿಯಾಗಿದ್ದ ಚಂದ್ರನು ಈ ಕ್ಷೇತ್ರದ ತೀರ್ಥದಲ್ಲಿ ಸ್ನಾನ ಮಾಡಿ ಆರೋಗ್ಯವಂತನಾದ. ನಾವೂ ಆತನಂತೆಯೇ ಅಲ್ಲಿ ಸ್ನಾನಮಾಡಿ, ದೇವ ಪಿತೃ ತರ್ಪಣಗಳ ನ್ನಾಚರಿಸಿ, ದಾನ ಧರ್ಮಗಳನ್ನು ಆಚರಿಸಿದರೆ, ಆ ಪುಣ್ಯವೆಂಬ ನಾವೆ ನಮ್ಮನ್ನು ಪುಷಿ ಶಾಪದ ಸಾಗರದಿಂದ ದಾಟಿಸಬಹುದು’ ಎಂದನು. ಆತನ ಆದೇಶದಂತೆ ಯಾದವ ರೆಲ್ಲರೂ ದ್ವಾರಕಿಯನ್ನು ಬಿಟ್ಟು ಹೋಗುವುದಕ್ಕಾಗಿ ತಮ್ಮ ತಮ್ಮ ವಾಹನಗಳನ್ನು ಸಿದ್ಧ ಪಡಿಸಿಕೊಳ್ಳಹೊರಟರು.

