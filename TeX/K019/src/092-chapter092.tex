
\chapter{೯೨. ಕಣ್ಣಿಗೆ ಕಾಣಿಸಿದುದೆಲ್ಲ ಗುರು}

ಯಾದವರೆಲ್ಲ ಪ್ರಯಾಣಸನ್ನದ್ಧರಾಗುತ್ತಿರಲು, ಭಗವದ್ಭಕ್ತನಾದ ಉದ್ಧವನು ಶ್ರೀ ಕೃಷ್ಣನನ್ನು ಏಕಾಂತದಲ್ಲಿ ಕಂಡು, ಭಕ್ತಿಯಿಂದ ನಮಸ್ಕರಿಸಿ ‘ಹೇ ಪರಮಾತ್ಮ, ಪುಷಿ ಶಾಪದಿಂದ ಯಾದವಕುಲ ನಾಶವಾಗುವುದನ್ನು ನೀನು ತಪ್ಪಿಸಬಲ್ಲೆಯಾದರೂ ಹಾಗೆ ಮಾಡದಿರುವುದನ್ನು ನೋಡಿದರೆ, ಆ ನಾಶ ನಿನಗೆ ಸಮ್ಮತವೆಂದೆ ಅರ್ಥವಾಗುತ್ತದೆ. ಈ ಕಾರ್ಯವನ್ನು ಮುಗಿಸಿ, ನೀನು ಪರಂಧಾಮಕ್ಕೆ ತೆರಳುವುದು ನಿಶ್ಚಯವಾದ ಹಾಗಾಯಿತು. ಹಾಗಾದರೆ ನನ್ನನ್ನೂ ನಿನ್ನೊಡನೆ ಕರೆದೊಯ್ಯಿ. ನಾನು ನಿನ್ನ ಪಾದಗಳನ್ನು ಅಗಲಿ ಜೀವಿಸ ಲಾರೆ. ಸ್ನಾನ, ಊಟ, ನಿದ್ರೆ–ಎಲ್ಲದರಲ್ಲಿಯೂ ನಾನು ನಿನ್ನ ಜೊತೆಯಲ್ಲಿದ್ದವನು; ನೀನು ಅನುಭವಿಸಿ ಬಿಟ್ಟುದನ್ನು ಮಹಾಪ್ರಸಾದವೆಂದು ಪರಿಗ್ರಹಿಸುತ್ತಿದ್ದವನು. ನನ್ನ ಆತ್ಮ ಸ್ವರೂಪಿಯಾದ ನಿನಗಿಂತಲೂ ಪ್ರಿಯವಾದ ವಸ್ತು ನನಗೆ ಮತ್ತೊಂದಿಲ್ಲ. ಸ್ವಾಮಿ, ನಿನ್ನ ಅಗಲಿಕೆ ನನಗೆ ಅತ್ಯಂತ ಭಯಂಕರ. ನಾನದನ್ನು ಸಹಿಸಲಾರೆ’ ಎಂದು ಕಣ್ಣೀರುಗರೆದನು.

ಉದ್ಧವನ ಭಕ್ತಿಗೆ ಮೆಚ್ಚಿದ ಶ್ರೀಕೃಷ್ಣನು ಆತನನ್ನು ಕುರಿತು ‘ಮಿತ್ರ, ನೀನು ಊಹಿಸಿರು ವುದೆಲ್ಲ ನಿಜ. ಈ ಯಾದವ ಕುಲ ಪರಸ್ಪರ ಕಲಹದಿಂದ ಇಷ್ಟರಲ್ಲಿಯೆ ನಾಶವಾಗಿಹೋಗು ತ್ತದೆ. ಅಷ್ಟಾಗುತ್ತಲೆ ನಾನು ಪರಂಧಾಮಕ್ಕೆ ತೆರಳುತ್ತೇನೆ. ಇಂದಿಗೆ ಏಳನೆಯ ದಿನ ಈ ದ್ವಾರಕಿಯನ್ನು ಸಮುದ್ರವು ನುಂಗಿಹಾಕುತ್ತದೆ. ನಾನು ಈ ಲೋಕವನ್ನು ಬಿಟ್ಟು ತೆರಳು ತ್ತಲೆ ಕಲಿಪುರುಷನು ಈ ಲೋಕವನ್ನು ಆಕ್ರಮಿಸುತ್ತಾನೆ. ಅಧರ್ಮ ತಾನೇತಾನಾಗಿ ತಲೆ ಯೆತ್ತಿ ವಿಜೃಂಭಿಸುತ್ತದೆ. ಕ್ಷಣಮಾತ್ರವೂ ನೀನಿಲ್ಲಿ ನಿಲ್ಲಬೇಡ; ಬಂಧುಬಾಂಧವರೆಂಬ ಮೋಹವನ್ನು ಬಿಟ್ಟು ಭೂಸಂಚಾರಕ್ಕೆ ಹೊರಡು; ಇಂದ್ರಿಯಗಳನ್ನು ನಿಗ್ರಹಿಸಿ, ಮನ ವನ್ನು ನನ್ನಲ್ಲಿ ನೆಲೆಗೊಳಿಸು; ವಿಸ್ತಾರವಾದ ಈ ಜಗತ್ತನ್ನು ಆತ್ಮಸ್ವರೂಪವಾಗಿಯೂ, ಆತ್ಮವನ್ನು ಬ್ರಹ್ಮಸ್ವರೂಪವಾಗಿಯೂ ತಿಳಿ. ಈ ತಿಳಿವಳಿಕೆ ದೃಢವಾಯಿತೆಂದರೆ ಮಾಯೆ ಇಂಗುತ್ತದೆ, ಸಂಸಾರ ನೀಗುತ್ತದೆ’ ಎಂದು ಉಪದೇಶಿಸಿದನು. ಈ ಉಪದೇಶದಿಂದ ಉದ್ಧವನ ಜ್ಞಾನದಾಹ ಹೆಚ್ಚಿತು. ಆತನು ಶ್ರೀಕೃಷ್ಣನೊಡನೆ ‘ದೇವದೇವನಾದ ಯೋಗೇ ಶ್ವರ, ನೀನು ಸರ್ವಸಂಗ ಪರಿತ್ಯಾಗವನ್ನು–ಸನ್ಯಾಸವನ್ನು–ಬೋಧಿಸುತ್ತಿರುವೆ. ನನ್ನಂ ತಹವನಿಗೆ ಅದು ಸಾಧ್ಯವೆ? ಅದನ್ನು ಸಾಧಿಸುವ ಉಪಾಯವನ್ನೂ ನನಗೆ ಹೇಳಿಕೊಡು. ನೀನು ಜಗದ್ಗುರು, ನಿನ್ನಂತಹ ಗುರು ಮತ್ತೆಲ್ಲಿ ಸಿಕ್ಕುವುದು ಸಾಧ್ಯ?’ ಎಂದು ಬೇಡಿ ಕೊಂಡನು. 

ಉದ್ಧವನ ಪ್ರಾರ್ಥನೆಯಿಂದ ಸುಪ್ರೀತನಾದ ಶ್ರೀಕೃಷ್ಣನು ಆತನನ್ನು ಕುರಿತು ‘ಹೇ ಭಕ್ತೋತ್ತಮ, ಗುರು ಬೇಕೆನ್ನುವವನು ಅವನನ್ನು ಅರಸುತ್ತಾ ಬಹುದೂರ ಹೋಗಬೇಕಾ ಗಿಲ್ಲ. ನನ್ನ ಉದ್ಧಾರದ ಮಾರ್ಗವನ್ನು ತೋರಿಸುವ ಬಹು ದೊಡ್ಡ ಗುರುವೆಂದರೆ ನಿನ್ನ ಮನಸ್ಸೆ. ಮಾನವನ ಮನಸ್ಸೆ ಬಂಧ ಮೋಕ್ಷಗಳೆರಡಕ್ಕೂ ಮೂಲ. ಮನಸ್ಸು ವಿಷಯಾ ಸಕ್ತವಾದರೆ ಬಂಧನ, ವಿಷಯ ವಿಮುಖವಾದರೆ ಮೋಕ್ಷ. ಆದ್ದರಿಂದ ಮನುಷ್ಯನು ಮನಸ್ಸನ್ನು ನಿಯಂತ್ರಿಸುವ ಬುದ್ಧಿಗೆ ತಕ್ಕ ಶಿಕ್ಷಣವನ್ನು ನೀಡಬೇಕು. ಶ್ರುತಿ ಸ್ಮೃತಿಗಳ ಶ್ರವಣ, ಪ್ರತ್ಯಕ್ಷಪ್ರಮಾಣ, ಅನುಮಾನ ಎಂಬ ಮೂರು ಮಾರ್ಗಗಳಿಂದಲೂ ಬುದ್ಧಿ ಯನ್ನು ಹದಗೊಳಿಸಿದಾಗ ಇಂದ್ರಿಯಜಯ, ಪರಮಾತ್ಮನ ಉಪಾಸನೆಗಳ ಮೂಲಕ ಸತ್ಯ ಸಾಕ್ಷಾತ್ಕಾರದ ಸಿದ್ಧಿ ಸಾಧ್ಯವಾಗುತ್ತದೆ. ಅಯ್ಯಾ ಉದ್ಧವ, ತಿಳಿಯಬೇಕೆಂಬ ತೀವ್ರಾಸಕ್ತಿ ಯೊಂದಿದ್ದರೆ ಕಣ್ಣಿಗೆ ಕಾಣಿಸಿದುದೆಲ್ಲ ಗುರುವಾಗುತ್ತದೆ. ಈ ವಿಚಾರವಾಗಿ ಹಿಂದೆ ನಡೆದ ಒಂದು ಸಣ್ಣ ಕಥೆಯನ್ನು ಹೇಳುತ್ತೇನೆ, ಕೇಳು. ಹಿಂದೆ, ಬಹುಹಿಂದೆ, ನಮ್ಮ ವಂಶದ ಮೂಲಪುರುಷನಾದ ಯದುಮಹಾರಾಜನು ರಾಜ್ಯಭಾರಮಾಡುತ್ತಿದ್ದಾಗ, ಒಬ್ಬ ಬ್ರಾಹ್ಮಣನಿದ್ದ. ಆತ ಮಹಾಪ್ರಾಜ್ಞ; ಆದರೆ ಕೇವಲ ಅಜ್ಞನಂತೆ ವ್ಯವಹರಿಸುತ್ತಾ, ಮನ ಬಂದಂತೆ ಅಲೆಯುತ್ತಿದ್ದ. ಯದುಮಹಾರಾಜ ಒಮ್ಮೆ ಆತನನ್ನು ಕಂಡು ‘ಅಯ್ಯಾ ಬ್ರಾಹ್ಮಣ, ನೀನು ಒಳ್ಳೆಯ ಯುವಕ, ಬಲಶಾಲಿ, ಚೆಲುವ; ಆದರೆ ಭೋಗಭಾಗ್ಯಗಳ ಕಡೆ ಗಮನವೇ ಇಲ್ಲ. ನೀನು ಮಹಾಪ್ರಾಜ್ಞ; ಆದರೆ ಹುಚ್ಚನಂತೆ, ಜಡನಂತೆ, ದೆವ್ವಬಡಿ ದವನಂತೆ ಸುಮ್ಮನೆ ತಿರುಗುತ್ತಿದ್ದಿ! ಸ್ನಾನ ಸಂಧ್ಯೆಗಳಾಗಲಿ, ಊಟ ಉಪಚಾರಗಳಾಗಲಿ ನಿನಗೆ ಬೇಕಿಲ್ಲ. ಲೋಕದ ಜನರೆಲ್ಲ ಕಾಮಲೋಭಗಳೆಂಬ ಕಾಡುಗಿಚ್ಚಿನಲ್ಲಿ ಬೆಂದು ಹೋಗುತ್ತಿರುವಾಗ, ನೀನು ಗಂಗಾನದಿಯಲ್ಲಿ ನೀರಾಟವಾಡುವ ಮದ್ದಾನೆಯಂತೆ ಆನಂದವಾಗಿರುವಿ! ಹೀಗೆ ಸದಾ ಆನಂದವಾಗಿರುವ ಶಕ್ತಿಯನ್ನು ನೀನು ಯಾರಿಂದ ಕಲಿತೆ? ನಿನ್ನ ಗುರುವಾರು?’ ಎಂದು ನಮ್ರನಾಗಿ ಕೇಳಿದ.

ಯದುಮಹಾರಾಜನು ದೊಡ್ಡ ದೈವಭಕ್ತ, ಬಹುಬುದ್ಧಿಶಾಲಿ, ಪುಷಿ ಬ್ರಾಹ್ಮಣರಲ್ಲಿ ತುಂಬ ಆದರವುಳ್ಳವನು. ಆದ್ದರಿಂದ ಆ ಬ್ರಾಹ್ಮಣ ಆತನ ಸಂದೇಹ ನಿವಾರಣೆಯಾಗು ವಂತೆ ಉತ್ತರವಿತ್ತ: “ಅಯ್ಯಾ ಮಹಾರಾಜ, ನನಗೆ ಗುರು ಒಬ್ಬರಲ್ಲ, ಹಲವಾರು ಜನ. ನಾನು ಒಬ್ಬೊಬ್ಬ ಗುರುವಿನಿಂದ ಒಂದೊಂದು ಗುಣವನ್ನು ಕಲಿತೆ. ನನ್ನ ಮೊದಲ ಗುರು ನಾನು ನಿಂತಿರುವ ಈ ನೆಲವೆ. ನೋಡು, ನಾವೆಲ್ಲ ಈ ನೆಲವನ್ನು ಹೇಗೆ ಕಾಲಿನಿಂದ ತುಳಿ ಯುತ್ತಿದ್ದೇವೆ. ಆದರೂ ಅದೇನಾದರೂ ಸಹನೆಯನ್ನು ಕಳೆದುಕೊಳ್ಳುತ್ತದೆಯೆ? ಇಲ್ಲ; ಅದಕ್ಕೆ ಬದಲಾಗಿ, ಆ ಭೂಮಿತಾಯಿಯಿಂದ ಹುಟ್ಟಿದ ಪರ್ವತಗಳೂ ಮರಗಳೂ ತಮ್ಮ ಜನ್ಮವನ್ನೆಲ್ಲ ಪರೋಪಕಾರಕ್ಕಾಗಿ ಮೀಸಲಿರಿಸಿವೆ. ಅವರನ್ನು ಕಂಡು, ಆ ತಾಯಿಯ ಸಹನೆ ಯನ್ನೂ ಆ ಮಕ್ಕಳ ಪರೋಪಕಾರ ಬುದ್ಧಿಯನ್ನೂ ನಾನು ಕಲಿತೆ. ನನ್ನ ಎರಡನೆಯ ಗುರು ಗಾಳಿ. ಈ ಗಾಳಿಯ ಬಹು ಸಣ್ಣ ಭಾಗವೆ ನಮ್ಮ ಪ್ರಾಣವಾಯು. ಅಷ್ಟು ಸ್ವಲ್ಪವೆ ಸಾಕು ನಮ್ಮ ಪ್ರಾಣಧಾರಣೆಗೆ. ಇದನ್ನು ಕಂಡು ನನಗನ್ನಿಸುತ್ತದೆ, ಆಹಾರವನ್ನೂ ಪ್ರಾಣ ಧಾರಣೆಗೆ ಅಗತ್ಯವಾದಷ್ಟನ್ನು ಮಾತ್ರ ತೆಗೆದುಕೊಳ್ಳಬೇಕೆಂದು. ಇನ್ನೊಂದು ಮುಖ್ಯ ಪಾಠ ವನ್ನೂ ಕಲಿಯಬಹುದು ಈ ಗುರುವಿನಿಂದ. ನೋಡು, ಈ ಹೊರಗಿನ ಗಾಳಿ ಎಲ್ಲೆಲ್ಲಿಯೋ ಬೀಸುತ್ತದೆ. ಶೀತ, ಉಷ್ಣ, ಸುಗಂಧ, ದುರ್ಗಂಧ–ಇತ್ಯಾದಿಗಳಲ್ಲಿ ತೂರಿಬಂದರೂ ಆ ಗುಣಗಳು ಅದರಲ್ಲಿ ನೆಲೆಯಾಗಿ ನಿಲ್ಲುವುದಿಲ್ಲ. ಯೋಗಿಯಾದವನು ಪ್ರಿಯಾಪ್ರಿಯ ವಾದ ವಿಷಯಗಳನ್ನು ಇಂದ್ರಿಯಗಳಿಂದ ಅನಿವಾರ್ಯವಾಗಿ ಅನುಭವಿಸಬೇಕಾಗಿ ಬಂದಾ ಗಲೂ, ಅದರ ಸುಖದುಃಖಗಳಿಗೆ ಒಳಗಾಗಬಾರದು. ಗಾಳಿಯಿಂದ ಈ ಪಾಠವನ್ನು ನಾನು ಕಲಿತೆ. ನನ್ನ ಮೂರನೆಯ ಗುರು ಆಕಾಶ. ಇದು ಜಗತ್ತಿನ ಒಳಗೆ, ಹೊರಗೆ ಎಲ್ಲೆಲ್ಲಿಯೂ ವ್ಯಾಪಿಸಿದೆ; ಆದರೂ ಅದಕ್ಕೆ ಯಾವ ವಸ್ತುವಿನ ಸಂಪರ್ಕವೂ ಇಲ್ಲ. ಇದನ್ನು ಕಂಡು ನಾನು, ಭಗವಂತನು ವಿಶ್ವವ್ಯಾಪಿಯಾದರೂ ಸೃಷ್ಟಿಯ ಗುಣದೋಷಗಳಿಗೆ ಅತೀತನೆಂಬು ದನ್ನು ಅರ್ಥಮಾಡಿಕೊಂಡೆನು. ನಾನು ನೀರನ್ನು ಕಂಡಾಗ ನನಗೆ ಅನಿಸಿತು–ನನ್ನ ಮನಸ್ಸು ಅದರಂತೆ ಸ್ವಚ್ಛವಾಗಿರಬೇಕು, ನನ್ನ ಹೃದಯ ಅದರಂತೆ ಸ್ನಿಗ್ಧವಾಗಿ ಸ್ನೇಹಮಯವಾಗಿರ ಬೇಕು, ನನ್ನ ಮಾತು ಅದರಂತೆ ಮಧುರವಾಗಿರಬೇಕು, ಎಂದು. ಪಾವನವಾದ ಪುಣ್ಯ ತೀರ್ಥದಂತೆ ಮನುಷ್ಯನು ತನ್ನ ಸ್ಪರ್ಶನ ದರ್ಶನಗಳಿಂದ ಜನರನ್ನು ಪುನೀತರನ್ನಾಗಿ ಮಾಡುವಂತಹನಾಗಬೇಕು. ನನ್ನ ಮುಂದಿನ ಗುರು ಬೆಂಕಿ. ನಿಜವಾಗಿಯೂ ಈ ಬೆಂಕಿ ಎಂತಹ ಮಹಾ ತತ್ವವನ್ನು ಬೋಧಿಸುತ್ತದೆ! ಅದು ಹಾಕಿದುದನ್ನೆಲ್ಲ ತಿಂದು ಹಾಕುತ್ತದೆ, ಆದರೂ ಅದು ಶುದ್ಧ, ಪರಿಶುದ್ಧ; ಯೋಗಿಯಾದವನು ಅದರಂತಿರಬೇಕು. ಮೃಷ್ಟಾ ನ್ನವೊ, ಕದನ್ನವೊ, ಸಿಕ್ಕುದನ್ನು ತಿಂದು ಶುದ್ಧ ಮನಸ್ಸಿನವನಾಗಿರಬೇಕು; ಒಂದು ಕಡೆ ಬೂದಿ ಮುಚ್ಚಿದ್ದರೆ ಮತ್ತೊಂದು ಕಡೆ ಧಗಧಗ ಉರಿಯುತ್ತಿರುತ್ತದೆ, ಬೆಂಕಿ. ಯೋಗಿ ಯಾದವನು ಬೂದಿ ಮುಚ್ಚಿದ ಕೆಂಡದಂತೆ ತನ್ನ ಮಹಿಮೆಯನ್ನು ಹುದುಗಿಟ್ಟುಕೊಂಡಿರ ಬೇಕು; ಸಮಯ ಬಂದಾಗ ಹವಿಸ್ಸನ್ನು ಪಡೆದ ಅಗ್ನಿ ತೇಜೋಮೂರ್ತಿಯಾಗಿ ಕಾಣಿಸಿ ಕೊಂಡು, ತನ್ನ ಮೊರೆಹೊಕ್ಕವನ ಪಾಪರಾಶಿಯನ್ನು ಹೋಗಲಾಡಿಸಬೇಕು. ಇದಿಷ್ಟು ಪಂಚಭೂತಗಳಿಂದ ನಾನು ಕಲಿತ ಪಾಠ. 

“ಯದುಮಹಾರಾಜ, ನಾವು ನಿತ್ಯವೂ ಸೂರ್ಯ ಚಂದ್ರರನ್ನು ನೋಡುತ್ತೇವೆ. ಅವರಿ ಗಿಂತಲೂ ದೊಡ್ಡ ಗುರುಗಳಾರು? ದಿನದಿನವೂ ಸೂರ್ಯನು ಸಮುದ್ರದ ನೀರನ್ನು ಸೆಳೆದು ಕೊಳ್ಳುತ್ತಾನೆ, ಆದರೆ ಅದೆಲ್ಲವನ್ನೂ ಪರೋಪಕಾರಕ್ಕಾಗಿ ಬಳಸುತ್ತಾನೆ. ಆತನಿಗೆ ಸ್ವಾರ್ಥ ಸ್ವಲ್ಪವೂ ಇಲ್ಲ. ನಾವು ಭೋಗ್ಯವಸ್ತುಗಳನ್ನು ಸಂಗ್ರಹಿಸಿ, ಅತಿಥಿಗಳಿಗೆ ದಾನಮಾಡ ಬೇಕು. ಕನ್ನಡಿಯಲ್ಲಿ ಸೂರ್ಯನ ಪ್ರತಿಬಿಂಬವನ್ನು ಕಂಡಾಗ ನನಗೆ ಮಹಾ ತತ್ವವೊಂದರ ಅರಿವಾಗುತ್ತದೆ; ಸೂರ್ಯನು ಕನ್ನಡಿಯಲ್ಲಿ ಇರುವಂತೆ ಕಾಣಿಸಿದರೂ ಆತನಿಗೆ ಅದರ ಸಂಪರ್ಕವಿಲ್ಲ. ಅದರಂತೆಯೆ ಆತ್ಮನು ದೇಹದಲ್ಲಿಯೆ ಇದ್ದರೂ ದೇಹ ಸಂಬಂಧಿ ಯಲ್ಲ. ಚಂದ್ರನನ್ನು ಕಂಡಾಗಲೂ ಇಂತಹ ದೊಡ್ಡ ತತ್ವವೊಂದು ಗೋಚರಿಸುತ್ತದೆ. ಚಂದ್ರನ ವೃದ್ಧಿಕ್ಷಯಗಳು ಕೇವಲ ಅದರ ಕಳೆಗಳ ವಿಕಾರಗಳೇ ಹೊರತು ಚಂದ್ರನ ವಿಕಾರ ಗಳಲ್ಲ; ಹಾಗೆಯೆ ಮಾನವವಿಕಾರಗಳೆಲ್ಲ ದೇಹಕ್ಕೆ ಸಂಬಂಧಪಟ್ಟುವೆ ಹೊರತು ಆತ್ಮನಿ ಗಲ್ಲ. ಮಹಾರಾಜ, ಅತಿ ಸ್ನೇಹ ಬಹು ದುಃಖಕ್ಕೆ ಕಾರಣ–ಎಂಬ ಸತ್ಯವನ್ನು ನಾನೊಂದು ಪಾರಿವಾಳದಿಂದ ಕಲಿತೆ. ಅದರ ಕಥೆಯನ್ನು ಹೇಳುತ್ತೇನೆ ಕೇಳು. ಒಂದು ಪಾರಿವಾಳ ತನ್ನ ಮಡದಿಯೊಡನೆ ಒಂದು ಕಾಡಿನಲ್ಲಿ ಗೂಡು ಕಟ್ಟಿಕೊಂಡಿತ್ತು. ಆ ಗಂಡು ಹೆಣ್ಣು ಹಕ್ಕಿ ಗಳಿಗೆ ಪರಸ್ಪರ ಬಲು ಪ್ರೇಮ. ಕುಳಿತಾಗ, ಮಲಗುವಾಗ, ಆಟವಾಡುವಾಗ, ಊಟಮಾಡು ವಾಗ, ಹಾರಾಡುವಾಗ ಅವು ಜೊತೆಜೊತೆಯಾಗಿಯೆ ಇರುತ್ತಿದ್ದವು. ಅವುಗಳ ಪ್ರೇಮಫಲ ದಂತೆ ಕೆಲವು ಮರಿಗಳು ಹುಟ್ಟಿದವು. ಅವುಗಳನ್ನು ಅತ್ಯಂತ ಪ್ರೇಮದಿಂದ ಲಾಲಿಸಿ, ಪಾಲಿಸುತ್ತಾ, ಆ ದಂಪತಿ ಪಕ್ಷಿಗಳು ಅವುಗಳ ಬಾಲಲೀಲೆಯಿಂದ ಧನ್ಯತೆಯನ್ನು ಪಡೆ ದವು. ಅದೆಂತಹ ಮಾಯೆಯೊ, ಅದೆಂತಹ ಮೋಹವೊ! ತಮ್ಮ ಮರಿಗಳಿಗಾಗಿ ಎಂತಹ ಕಷ್ಟವನ್ನಾದರೂ ಇದಿರಿಸಲು ಅವು ಸಿದ್ಧವಾಗಿದ್ದವು. ಒಮ್ಮೆ ಆ ಹಕ್ಕಿಗಳೆರಡೂ ತಮ್ಮ ಮರಿಗಳಿಗೆ ಆಹಾರವನ್ನು ಹುಡುಕುತ್ತಾ ಅಡವಿಯಲ್ಲಿ ತಿರುಗುತ್ತಿರಲು, ಬೇಡನೊಬ್ಬನು ಅವುಗಳ ಗೂಡಿನ ಬಳಿಗೆ ಹೋಗಿ, ಸಂತೋಷದಿಂದ ಮೈಮರೆತು ಹಾರಾಡುತ್ತಿರುವ ಮರಿ ಗಳನ್ನು ಬಲೆಯಲ್ಲಿ ಬೀಳಿಸಿದನು. ಆ ವೇಳೆಗೆ ಸರಿಯಾಗಿ ಪಕ್ಷಿದಂಪತಿಗಳು ಆಹಾರ ದೊಡನೆ ಗೂಡಿಗೆ ಹಿಂದಿರುಗಿದವು. ತಮ್ಮ ಮರಿಗಳು ಬೇಡನ ಬಲೆಯಲ್ಲಿ ಬಿದ್ದಿರುವು ದನ್ನು ಕಾಣುತ್ತಲೆ ಅವು ದುಃಖದಿಂದ ಕರುಳು ಕರಗುವಂತೆ ಕಿರಿಚಿಕೊಂಡವು. ತಾಯಿ ಹಕ್ಕಿಯಂತೂ ತನ್ನ ದುಃಖವನ್ನು ತಡೆಯಲಾರದೆ ಜೀವದ ಹಂಗುತೊರೆದು, ಆ ಸಮೀಪ ದಲ್ಲಿಯೆ ಸುಳಿದಾಡುತ್ತಾ, ಕೊನೆಗೆ ಮೈಮರೆತು ತಾನೂ ಬಲೆಗೆ ಬಿದ್ದಿತು. ಆಗ ಇದನ್ನು ಕಂಡ ಗಂಡು ಪಾರಿವಾಳವು ಕಲ್ಲುಕೂಡ ಕರಗುವಂತೆ ಗೋಳಾಡುತ್ತಾ ‘ಅಯ್ಯೋ, ಅಂತಹ ಪ್ರೇಮಮಯಿ ಅಗಲಿದಮೇಲೆ ನನ್ನ ಜೀವಕ್ಕೆ ಏನು ಬೆಲೆ?’ ಎಂದು ಹೇಳಿಕೊಂಡು, ತಾನಾಗಿಯೆ ಹೋಗಿ ಆ ಬಲೆಯಲ್ಲಿ ಬಿದ್ದಿತು. ಆದ್ದರಿಂದ, ಯದುಮಹಾರಾಜ, ಸಂಸಾರ ಕ್ಕಾಗಿಯೆ ಹಗಲಿರುಳು ದುಡಿಯು ವವರ ಹಣೆಯ ಬರಹ ಇಷ್ಟೇ ಸರಿ–ಅಧೋಗತಿ. ಮುಕ್ತಿಯ ಬಾಗಿಲನ್ನು ತೆರೆಯುವುದಕ್ಕೆ ಸಾಧನವೆನಿಸಿದ ಮಾನವಜನ್ಮವನ್ನು ಪಡೆದವ ನಾವನೂ ಆ ಪಾರಿವಾಳದಂತೆ ಮತಿಗೆಡಬಾರದು.

“ಮಹಾರಾಜ, ನಾನೊಂದು ದಿನ ದೊಡ್ಡದೊಂದು ಹೆಬ್ಬಾವನ್ನು ಕಂಡೆ. ತಕ್ಷಣವೆ ನನಗನಿಸಿತು–ಎಲಾ, ಇದು ಬಿದ್ದಕಡೆ ಬಿದ್ದಿರುತ್ತದೆ. ತಾನಾಗಿ ಸಿಕ್ಕುದನ್ನು ತಿಂದು ತೃಪ್ತಿ ಪಡುತ್ತದೆ. ಮನುಷ್ಯನೂ ಹಾಗೇಕೆ ಇರಬಾರದು? ಸುಖದುಃಖಗಳು ಅನಪೇಕ್ಷಿತವಾಗಿ ಬರುತ್ತವೆ, ಅವು ಕರ್ಮಾಧೀನ. ಆದ್ದರಿಂದ ಬಂದುದನ್ನು ಗೊಣಗದೆ ಅನುಭವಿಸುವುದು ತಾನೆ? ಸಿಕ್ಕಿದಷ್ಟನ್ನು, ಸಿಕ್ಕಿದಂತಹುದನ್ನು ಪಡೆದು ತೃಪ್ತವಾಗಿರುವುದೇ ಅಜಗರವೃತ್ತಿ. ನಾವು ಹೆಬ್ಬಾವಿನಂತೆ ಸುಮ್ಮನೆ ಮಲಗಿ ದೈವಚಿಂತನೆ ಮಾಡುತ್ತಾ ಸಿಕ್ಕಿದುದರಿಂದ ತೃಪ್ತ ರಾಗಬೇಕು. ಸಮುದ್ರವನ್ನು ನೋಡು, ಹೇಗೆ ಅದು ಅರಕ್ಕೆ ಹಿಗ್ಗದೆ ಬರಕ್ಕೆ ತಗ್ಗದೆ ಇರು ತ್ತದೆ! ಎಷ್ಟು ನದಿಗಳು ಹರಿದು ಹರಿದು ತನ್ನೊಳಗೆ ಸೇರಿದರೂ ಅದು ಉಕ್ಕುವುದಿಲ್ಲ. ಎಂತಹ ಬಿರುಬೇಸಗೆ ಬಂದರೂ ಬತ್ತುವುದಿಲ್ಲ. ಮನುಷ್ಯ ಅದನ್ನು ಕಂಡು ಪಾಠ ಕಲಿಯ ಬೇಕು–ಐಶ್ವರ್ಯಕ್ಕೆ ಹಿಗ್ಗಬಾರದು, ಬಡತನಕ್ಕೆ ಕುಗ್ಗಬಾರದು. ಅದು ಹೊರಗೆ ಪ್ರಸನ್ನ, ಒಳಗೆ ಗಂಭೀರ. ಜ್ಞಾನಿಯಾದವನು ಹೇಗಿರಬೇಕೆಂಬದನ್ನು ಅದು ಬೋಧಿಸುತ್ತದೆ. ಪತಂಗವನ್ನು ಕಂಡಾಗ ನಮಗೆ ಎಂತಹ ತತ್ವ ಬೋಧನೆಯಾಗುತ್ತದೆ! ಪಾಪ, ಅದು ಬೆಂಕಿ ಯನ್ನು ಭೋಗ್ಯವಸ್ತುವೆಂದು ಭ್ರಮಿಸಿ ದಗ್ಧವಾಗಿ ಹೋಗುತ್ತದೆ. ಅದರಂತೆಯೆ ಹೆಣ್ಣಿನ ಮೋಹಕ್ಕೆ ಮರುಳಾಗಿ, ಇಂದ್ರಿಯಸುಖಕ್ಕೆ ಎಳಸಿದವರು ತಮಸ್ಸಿನಲ್ಲಿ ಮುಳುಗಿಹೋಗು ವರೆಂಬ ವಿವೇಕವನ್ನು ನಾವು ಕಲಿಯಬಹುದಲ್ಲವೆ? ಮಹಾರಾಜ, ಜೇನನ್ನು ಶೇಖರಿಸುವ ಜೇನುಹುಳುವನ್ನು ನೋಡಿದಾಗಲೆಲ್ಲ ನಾವು ಜಗತ್ತಿನಲ್ಲಿ ಹೇಗೆ ವ್ಯವಹರಿಸಬೇಕೆಂಬ ಜ್ಞಾನಬೋಧೆಯಾಗುತ್ತದೆ, ನನಗೆ. ಅದು ಹೂವಿಗೆ ಸ್ವಲ್ಪವೂ ನೋವಾಗದಂತೆ ಬಹುಸ್ವಲ್ಪ ಮಧುವನ್ನು ಹೀರಿ, ಬೇರೆಯ ಹೂವಿಗೆ ಹೊರಟುಹೋಗುತ್ತದೆ. ತನ್ನ ಆಹಾರವನ್ನು ಅನೇಕ ಹೂಗಳಿಂದ ಅದು ಪಡೆಯುವಂತೆಯೆ, ನಾವು ಒಬ್ಬರಿಗೆ ಹಿಂಸೆಯಾಗದಂತೆ ಹಲ ವಾರು ಜನರಿಂದ ಸ್ವಲ್ಪಸ್ವಲ್ಪವನ್ನು ಸ್ವೀಕರಿಸಿ, ನಮ್ಮ ಅಗತ್ಯಗಳನ್ನು ಪೂರೈಸಿಕೊಳ್ಳಬೇಕು. ಅಷ್ಟೇ ಅಲ್ಲ, ಅದು ಪ್ರತಿಯೊಂದು ಹೂವಿನ ಸಾರವನ್ನೂ ಗ್ರಹಿಸುವಂತೆ ನಾವೂ ಪ್ರತಿ ಯೊಂದು ಶಾಸ್ತ್ರದ ಸಾರವನ್ನೂ ಸ್ವೀಕರಿಸಬೇಕು. ಆದರೆ ಜೇನುಹುಳದಂತೆ ಆಹಾರವನ್ನು ಸಂಗ್ರಹಿಸಿಡುವ ಕಾರ್ಯವನ್ನು ಮಾತ್ರ ಮಾಡಬಾರದು. ಹಾಗೆ ಕೂಡಿಟ್ಟ ಜೇನು ಬೇಡನ ಪಾಲಾಗುತ್ತದೆ. ಸುಖಿಯಾಗಬಯಸುವವನಿಗೆ ಅವನ ಬೊಗಸೆಯೆ ಭೋಜನಪಾತ್ರೆ, ಹೊಟ್ಟೆಯೆ ಉಗ್ರಾಣ.

“ಮಹಾರಾಜ, ಆನೆಯನ್ನು ಕಂಡಾಗಲೆಲ್ಲ, ಹೆಣ್ಣಿನಿಂದ–ಮರದಿಂದ ಮಾಡಿದ ಹೆಣ್ಣು ಬೊಂಬೆಯಿಂದ ಕೂಡ–ದೂರವಾಗಿರಬೇಕೆನಿಸುತ್ತದೆ. ಪಾಪ ಮರದಿಂದ ಮಾಡಿದ ಹೆಣ್ಣಾನೆಯನ್ನು ಕಂಡೇ ಅದು ಗುಂಡಿಗೆ ಬೀಳುವುದು. ಅಷ್ಟೇ ಅಲ್ಲ, ಹೆಣ್ಣಾನೆಗೆ ಆಸೆಪಟ್ಟ ಗಂಡಾನೆ ತನಗಿಂತಲೂ ಬಲಶಾಲಿಯಾದ ಆನೆಯ ಕೈಗೆ ಸಿಕ್ಕಿ ಸಾಯಬೇಕಾಗುತ್ತದೆ. ಮಾನವನಾದರೂ ಅಷ್ಟೇ ತಾನೆ! ಇಂದ್ರಿಯ ಸುಖದ ಆಸೆ ಎಷ್ಟು ಭಯಂಕರ! ನೋಡು, ಜಿಂಕೆ ಬೇಟೆಗಾರನ ಸಂಗೀತಕ್ಕೆ ಮರುಳಾಗಿ ಪ್ರಾಣಕ್ಕೆ ಸಂಚಕಾರ ತಂದುಕೊಳ್ಳುತ್ತದೆ. ಮಾನವನೂ ಅಷ್ಟೆ. ಪುಷ್ಯಶೃಂಗನ ಕಥೆಯನ್ನು ಕೇಳಿರುವೆಯಲ್ಲ. ವೇಶ್ಯೆಯರ ಸಂಗೀತ ನೃತ್ಯಗಳಿಗೆ ಮರುಳಾಗಿ ಆತ ಅವರ ಕೈಗೊಂಬೆಯಾದ? ಮೀನನ್ನು ನೋಡು, ನಾಲಗೆಯ ಚಪಲದಿಂದ ಅದು ಗಾಳಕ್ಕೆ ಸಿಕ್ಕಿ ಸಾಯುತ್ತದೆ. ಮಹಾರಾಜ, ಎಲ್ಲ ಚಪಲಗಳಿಗಿಂತ ಈ ನಾಲಗೆ ಚಪಲ ಬಹು ಭಯಂಕರ. ಇದೊಂದನ್ನು ಜಯಿಸಿದರೆ ಎಲ್ಲವನ್ನೂ ಜಯಿಸಿ ದಂತೆಯೆ! ಈ ನಾಲಗೆಯ ಸವಿಯನ್ನು ತಣಿಸುವುದಕ್ಕಾಗಿಯೇ ನಾವು ಅನೇಕ ವಸ್ತುವನ್ನು ಸಂಗ್ರಹಿಸುವುದು. ಹೀಗೆ ಸಂಗ್ರಹಿಸಿದಾಗ ಎಂತಹ ಸಂಕಟಗಳು ಬರುತ್ತವೆ! ಒಂದು ಕಡಲ ಹದ್ದು ಒಂದು ಮೀನನ್ನು ಬಾಯಲ್ಲಿ ಕಚ್ಚಿಕೊಂಡು, ಅದನ್ನು ಯಾರಾದರೂ ಕಿತ್ತು ಕೊಂಡಾರೆಂಬ ಭಯದಿಂದ ಹಾರಿಹೋಗುತ್ತಿತ್ತು. ಅದಕ್ಕಿಂತಲೂ ಬಲವಾದ ಒಂದು ಹದ್ದು ಬಂದು ಹೊಡೆದು ಮೀನನ್ನು ಕಿತ್ತುಕೊಂಡಿತು. ಆಗ ಆ ಕಡಲ ಹದ್ದಿಗೆ ಇದ್ದ ಭಯ ಹೋಯಿತು. ಕೂಡಿಟ್ಟುಕೊಂಡಿದ್ದಾಗ ಭಯ. ಆದ್ದರಿಂದ ಕೂಡಿಡುವುದೇ ತಪ್ಪು.

“ಯದುಮಹಾರಾಜ, ನಿನಗೊಂದು ಸಣ್ಣ ಕಥೆಯನ್ನು ಹೇಳುತ್ತೇನೆ, ಕೇಳು. ವಿದೇಹ ನಗರದಲ್ಲಿ ಬಹುಕಾಲದ ಹಿಂದೆ ಪಿಂಗಳೆಯೆಂಬ ವೇಶ್ಯೆಯೊಬ್ಬಳಿದ್ದಳು. ತನ್ನ ಬಳಿಗೆ ಬರುವ ವಿಟರು ಇಷ್ಟೇ ಹಣ ಕೊಡಬೇಕೆಂದು ಅವಳು ಗೊತ್ತು ಮಾಡಿದ್ದಳು. ಸಂಜೆ ಯಾಗುತ್ತಲೆ ಅವಳು ಸರ್ವಾಲಂಕಾರಭೂಷಿತೆಯಾಗಿ ತನ್ನ ಮನೆಯ ಬಾಗಿಲಲ್ಲಿ ನಿಲ್ಲುತ್ತಿ ದ್ದಳು. ಅನೇಕ ವಿಟಪುರುಷರು ಅವಳ ಬಾಗಿಲಿಗೆ ಬರುತ್ತಿದ್ದರಾದರೂ, ಅವಳ ಆಶೆಯಷ್ಟು ಹಣವನ್ನು ಕೊಡಲಾರದೆ ಹಿಂದಿರುಗುತ್ತಿದ್ದರು. ತಾನು ಕೇಳಿದಷ್ಟನ್ನು ಕೊಡುವ ಶ್ರೀಮಂತ ಬಂದಾನೆಂಬ ಆಶೆಯಿಂದ ಆಕೆ ಕಾದಳು, ಮಧ್ಯರಾತ್ರಿವರೆಗೆ ಕಾದಳು. ಯಾರೂ ಬರಲಿಲ್ಲ. ನಿರಾಶೆಯಿಂದ ಆಕೆಗೆ ವೈರಾಗ್ಯ ಹುಟ್ಟಿತು. ಆಕೆ ತನ್ನ ಮನಸ್ಸಿನಲ್ಲಿ ‘ಆಹಾ, ನನ್ನ ಆಸೆ ಯಿಂದ ನಾನು ಎಂತಹ ಹೀನಕೆಲಸ ಮಾಡುತ್ತಿದ್ದೇನೆ! ಕೇಳಿದುದೆಲ್ಲವನ್ನೂ ಕೊಡಬಲ್ಲ ಭಗವಂತ ನನ್ನ ಹೃದಯದಲ್ಲಿಯೆ ಕುಳಿತಿರುವಾಗ ಆತನನ್ನು ಬಿಟ್ಟು ಸ್ತ್ರೀಲಂಪಟನಾದ ಮನುಷ್ಯನಿಗೆ ಬಾಯಿ ಬಿಡುವುದೆ? ಪರಮಾತ್ಮನಿಗೆ ನನ್ನ ಆತ್ಮವನ್ನು ಒಪ್ಪಿಸಿ ನಾನು ಧನ್ಯ ಳಾಗುತ್ತೇನೆ. ಇನ್ನು ಮುಂದೆ ತಾನಾಗಿ ದೊರೆತಷ್ಟೆ ನನಗೆ ಸಾಕು” ಎಂದು ಹೇಳಿಕೊಂಡು ಅವಳು ಒಳಗೆ ಹೋಗಿ ನಿಶ್ಚಿಂತಳಾಗಿ ಮಲಗಿಕೊಂಡಳು. ಅವಳಿಂದ ನಾನು ‘ಆಶಾಶಾಃ ಪರಮಂ ದುಃಖಂ, ನಿರಾಶಾಃ ಪರಮಂ ಸುಖಂ’ ಎಂಬ ನೀತಿಯನ್ನು ಕಲಿತೆ. ನಾವು ಮಗುವಿನಂತೆ ಇರುವುದನ್ನು ಕಲಿಯಬೇಕು. ಅದಕ್ಕೆ ಮಾನಾವಮಾನದ ಪ್ರಶ್ನೆಯಿಲ್ಲ, ಯಾವ ಚಿಂತೆಯೂ–ಊಟದ ಚಿಂತೆ ಕೂಡ–ಇಲ್ಲ. ಅದು ಸದಾ ಪರಮಾನಂದವಾಗಿರುತ್ತದೆ.

“ಮಹಾರಾಜ, ನಾನು ಸದಾ ಏಕಾಕಿಯಾಗಿರುತ್ತೇನೆ. ಹಾಗಿರುವುದರಿಂದ ನನ್ನ ಯೋಗ ಸಿದ್ಧಿಗೆ ಯಾವ ಅಡ್ಡಿಯೂ ಇಲ್ಲದಂತಾಗಿದೆ. ಇದನ್ನು ನಾನು ಒಂದು ಕನ್ನೆಯಿಂದ ಅರಿತೆ. ಅವಳೊಬ್ಬ ಹಳ್ಳಿಯ ಹುಡುಗಿ. ಅವಳಿಗೆ ಮದುವೆಯ ವಯಸ್ಸು. ಅವಳನ್ನು ನೋಡ ಬೇಕೆಂದು ಗಂಡಿನ ಕಡೆಯವರು ಮನೆಗೆ ಬಂದರು. ಆಗ ಮನೆಯವರೆಲ್ಲರೂ ಹೊರಗೆ ಹೋಗಿದ್ದರು. ಆ ಹೆಣ್ಣು ತಾನೆ ಅಡಿಗೆ ಮಾಡಿ ಬಂದವರಿಗೆಲ್ಲ ಬಡಿಸಬೇಕಾಗಿತ್ತು. ಮನೆ ಯಲ್ಲಿ ಬತ್ತವಿತ್ತು, ಅಕ್ಕಿ ಇರಲಿಲ್ಲ. ಹುಡುಗಿಯು ಬಂದವರಿಗೆ ಗೊತ್ತಾಗದಂತೆ ಆ ಕೆಲಸ ಮಾಡಬೇಕೆಂದುಕೊಂಡು, ಒಳಮನೆಯಲ್ಲಿ ಬತ್ತ ಕುಟ್ಟುವುದಕ್ಕೆ ಹೊರಟಳು. ಅವಳ ಕೈ ತುಂಬ ಬಳೆಗಳಿದ್ದುದರಿಂದ, ಅವು ಪರಸ್ಪರ ತಗಲಿ ಘಲ್​ಘಲ್ ಎಂಬ ಶಬ್ದವಾಯಿತು. ಜಾಣೆಯಾದ ಹುಡುಗಿ ಆ ಶಬ್ದವನ್ನು ನಿಲ್ಲಿಸುವುದಕ್ಕಾಗಿ ಎರಡೆರಡು ಬಳೆಗಳನ್ನು ಕೈಲಿ ಉಳಿಸಿಕೊಂಡು ಉಳಿದುವನ್ನು ಒಡೆದು ಹಾಕಿದಳು. ಆ ಎರಡು ಬಳೆಗಳೆ ಪರಸ್ಪರ ತಗುಲಿ ಶಬ್ದವಾಗುತ್ತಿರಲು ಆ ಹುಡುಗಿ ಅವುಗಳಲ್ಲಿ ಒಂದೊದನ್ನು ಒಡೆದು ಹಾಕಿ, ತನ್ನ ಕೆಲಸ ವನ್ನು ನಿಶ್ಶಬ್ದವಾಗಿ ನೆರವೇರಿಸಿದಳು. ಭೂಸಂಚಾರ ಮಾಡುತ್ತಿದ್ದ ನಾನು ಆ ಹುಡುಗಿಯ ಕಾರ್ಯವನ್ನು ಕಂಡು ‘ಬಹು ಜನ ಸೇರಿದರೆ ಜಗಳ, ಇಬ್ಬರಿದ್ದರೆ ಕಾಡುಹರಟೆ, ಏಕಾಕಿ ಯಾದರೆ ಮಾತ್ರ ಯೋಗಸಿದ್ಧಿ’ ಎಂಬ ತತ್ವವನ್ನು ಅರ್ಥಮಾಡಿಕೊಂಡೆ.

“ಮಹಾರಾಜ, ನಾನು ಪ್ರಾರಂಭದಲ್ಲಿಯೆ ಹೇಳಿದಂತೆ ಕಣ್ಣಿಗೆ ಕಾಣಿಸಿದುದೆಲ್ಲ ಗುರು ವಾಗಬಲ್ಲದು. ನೋಡು, ನಾನೊಮ್ಮೆ ಹುತ್ತವನ್ನು ಹೊಗುತ್ತಿರುವ ಹಾವನ್ನು ಕಂಡೆ. ಇರುವೆಗಳು ಕಟ್ಟಿದ ಹುತ್ತದಲ್ಲಿ ಅದು ಹಾಯಾಗಿರುತ್ತದೆ. ನಾವಾದರೂ ಸಜ್ಜನರ ಮನೆ ಯಲ್ಲಿ ಹಾಯಾಗಿರಬೇಕೆನಿಸಿತು; ಹಾವಿನಂತೆ ಒಂಟಿಯಾಗಿರಬೇಕೆಂದೂ ಬೋಧೆಯಾ ಯಿತು. ನಾನೊಬ್ಬ ಬಿಲ್ಲುಗಾರ ಬೇಟೆಗೆ ಗುರಿಯಿಡುವುದನ್ನು ಕಂಡೆ. ಆಗ ಸಮೀಪದಲ್ಲೆ ರಾಜನ ಮೆರವಣಿಗೆ ಹೋಗುತ್ತಿತ್ತು. ಆದರೆ ಬೇಟೆಗಾರನ ದೃಷ್ಟಿ ಅತ್ತ ತಿರುಗಲಿಲ್ಲ. ನಾನಾಗ ನನ್ನಲ್ಲಿಯೇ ‘ನನ್ನ ಮನಸ್ಸು ಅವನಂತೆ ಏಕಾಗ್ರವಾಗಬೇಕು’ ಎಂದುಕೊಂಡೆ. ಜೇಡರಹುಳುವನ್ನು ಕಂಡಾಗಲೆಲ್ಲ ನನಗೆ ‘ಎಲ ಎಲ, ಈ ಹುಳ ತನ್ನ ಹೃದಯದಿಂದ ದಾರವನ್ನು ನಿರ್ಮಿಸಿ ಅದರಿಂದ ಬಲೆಯನ್ನು ಕಟ್ಟಿ, ತನ್ನ ಕಾರ್ಯವಾಗುತ್ತಲೆ ಅದನ್ನೆಲ್ಲ ತನ್ನಲ್ಲಿಯೆ ಅಡಗಿಸಿಕೊಳ್ಳುತ್ತದೆಯಲ್ಲ! ಭಗವಂತ ತನ್ನ ಸೃಷ್ಟಿಯನ್ನು ಹಬ್ಬಿ ಹರಡಿ, ಮತ್ತೆ ತನ್ನಲ್ಲಿಯೆ ಅಡಗಿಸಿಕೊಳ್ಳುವುದು ಹೀಗೆಯೆ ಅಲ್ಲವೆ? ಸ್ಥೂಲ ಸೂಕ್ಷ್ಮಗಳೆರಡಕ್ಕೂ ಅವನೆ ಆಶ್ರಯವಲ್ಲವೆ?’ ಎಂದು ಹೇಳಿದನು. ಕಣಜದ ಹುಳವನ್ನು ಕಂಡಾಗ, ಬ್ರಹ್ಮಧ್ಯಾನ ದಿಂದ ಪರಬ್ರಹ್ಮನಾಗುವುದು ಹೇಗೆಂಬುದು ಬೋಧೆಯಾಯಿತು. (ಆ ಕಣಜದ ಹುಳು ತನ್ನ ಆಹಾರಕ್ಕೆಂದು ತಂದಿಟ್ಟ ಹುಳು ಪ್ರಾಣಭಯದಿಂದ ಸದಾ ಕಣಜದ ಹುಳವನ್ನೆ ಧ್ಯಾನಿ ಸುತ್ತದೆ. ಇದರ ಫಲವಾಗಿ ಅದು ಕಣಜದ ಹುಳವೆ ಆಗಿಹೋಗುತ್ತದೆ.)

“ಮಹಾರಾಜ, ಹೀಗೆ ನಾನು ಕಂಡುದೆಲ್ಲವೂ ನನಗೆ ಗುರುವಾಯಿತು. ನಾನು ವೈರಾಗ್ಯ ಪರನಾಗಿ ಜಗತ್ತಿನಲ್ಲೆಲ್ಲ ಸಂಚರಿಸುತ್ತಿದ್ದೇನೆ. ಒಬ್ಬ ಗುರುವಿನಿಂದ ಸಮಸ್ತ ಜ್ಞಾನವೂ ಬರುವುದಕ್ಕೆ ಸಾಧ್ಯವಿಲ್ಲ. ಇಷ್ಟೇ ಅಲ್ಲ ಬರಿ ಪುಸ್ತಕಪಾಠ ಸ್ಥಿರವಾಗಿ ನಿಲ್ಲುವುದೂ ಇಲ್ಲ. ಆದ್ದರಿಂದ ನಾನು ಕಲಿತಂತೆ ಕಲಿತ ಜ್ಞಾನ, ಅದ್ವಿತೀಯವಾದುದು, ಸ್ಥಿರವಾದುದು” ಎಂದು ಹೇಳಿದನು. ಇದನ್ನು ಕೇಳಿದ ಯದುರಾಜನು ಸರ್ವಸಂಗವನ್ನು ತ್ಯಜಿಸಿ, ಶಾಂತಚಿತ್ತನಾದನು.

ಶ್ರೀಕೃಷ್ಣನಿಂದ ಇದೆಲ್ಲವನ್ನೂ ಕೇಳಿದ ಉದ್ಧವನು ಗುರುಸ್ವರೂಪವನ್ನು ಅರಿತವನಾ ದನು. ಅನಂತರ ಶ್ರೀಕೃಷ್ಣನು ಆತನಿಗೆ ಕರ್ಮ, ಭಕ್ತಿ, ಜ್ಞಾನಮಾರ್ಗಗಳ ಸ್ವರೂಪವನ್ನು ವಿವರಿಸಿ, ತನ್ನ ಉಪಾಸನಾರೂಪವಾದ ಯೋಗಮಾರ್ಗವನ್ನು ಉಪದೇಶಿಸಿದನು. ಆ ಉಪ ದೇಶದಿಂದ ಉದ್ಧವನ ಅಜ್ಞಾನ ತೊಲಗಿತು. ಆತನು ಆನಂದಬಾಷ್ಪಗಳನ್ನು ಸುರಿಸುತ್ತಾ, ಶ್ರೀಕೃಷ್ಣನಿಗೆ ಅಡ್ಡಬಿದ್ದು, ಸ್ವಾಮಿ, ನನ್ನ ಮೋಹಪಾಶವನ್ನು ನಿನ್ನ ಜ್ಞಾನೋಪದೇಶವೆಂಬ ಕತ್ತಿಯಿಂದ ಕತ್ತರಿಸಿರುವೆ. ನಿನ್ನ ಮೇಲಿನ ಭಕ್ತಿ ನನ್ನಲ್ಲಿ ನೆಲೆಯಾಗಿರುವಂತೆ ಅನುಗ್ರಹಿಸು’ ಎಂದು ಬೇಡಿದನು. ಶ್ರೀಕೃಷ್ಣನು ‘ತಥಾಸ್ತು’ ಎಂದು ಹೇಳಿ, ಆತನನ್ನು ಬದರಿಕಾಶ್ರಮಕ್ಕೆ ಹೋಗಿ, ಪುಷಿಜೀವನವನ್ನು ನಡೆಸುತ್ತಾ, ಆತ್ಮಾನಂದದಿಂದ ಕೂಡಿ, ತನ್ನ ಬಾಳನ್ನು ಸವೆಸುವಂತೆ ಅಪ್ಪಣೆ ಮಾಡಿ, ಅವನನ್ನು ಅಲ್ಲಿಂದ ಬೀಳ್ಕೊಟ್ಟನು.

