
\chapter{೬೧. ನಂದಗೋಕುಲದಲ್ಲಿ ಉದ್ಧವ}

ಸಕಲ ವಿದ್ಯಾಪಾರಂಗತನಾಗಿ ಮಧುರೆಗೆ ಹಿಂದಿರುಗಿದ ಶ್ರೀಕೃಷ್ಣನು ನಂದಗೋಕುಲದ ತನ್ನ ಗೆಳೆಯರನ್ನೆಲ್ಲ ಒಮ್ಮೆ ಹೋಗಿ ನೋಡಿಬರಬೇಕೆಂದುಕೊಂಡನು. ಆದರೆ ಆತನು ಮಧುರೆಯಲ್ಲಿಯೇ ಕುಳಿತು ಮಾಡಬೇಕಾದ ಕಾರ್ಯ ಬೆಟ್ಟದಷ್ಟಿತ್ತು. ಆದ್ದರಿಂದ ತನ್ನ ಜೀವದ ಗೆಳೆಯನಾದ ಉದ್ಧವನನ್ನು ಅಲ್ಲಿ ಹೋಗಿ ಬರುವಂತೆ ಕೇಳಿಕೊಂಡನು. ಉದ್ಧವ ನೆಂದರೆ ಸಾಮಾನ್ಯನಲ್ಲ, ದೇವಗುರುವಾದ ಬೃಹಸ್ಪತಿಯಲ್ಲಿ ವಿದ್ಯಾಭ್ಯಾಸಮಾಡಿ ಮಹಾ ಬುದ್ಧಿಶಾಲಿಯೆಂದು ಹೆಸರು ಪಡೆದಿದ್ದವನು. ಆತನಿಗೆ ಶ್ರೀಕೃಷ್ಣನಲ್ಲಿ ಬಹುಭಕ್ತಿ. ಶ್ರೀಕೃಷ್ಣನೂ ತನ್ನ ಕೆಲಸ ಕಾರ್ಯಗಳಲ್ಲಿ ಆತನ ಸಲಹೆ ಬುದ್ಧಿವಾದಗಳನ್ನು ಬೇಡುತ್ತಿದ್ದನು. ಶ್ರೀಕೃಷ್ಣನು ಆತನನ್ನು ಕುರಿತು ‘ಮಿತ್ರ, ನೀನು ಈಗ ನನಗಾಗಿ ಒಮ್ಮೆ ನಂದಗೋಕುಲಕ್ಕೆ ಹೋಗಿಬರಬೇಕಾಗಿದೆ. ನೋಡು, ನಾನು ಅಲ್ಲಿಂದ ಬರುವಾಗ ತಕ್ಷಣವೇ ಹಿಂದಿರುಗಿ ಬರುವೆನೆಂದು ಅಲ್ಲಿನವರಿಗೆ ತಿಳಿಸಿ ಬಂದಿದ್ದೆ. ಆದರೆ ಹಾಗೆ ಮಾಡಲಾಗಲಿಲ್ಲ. ಅವರೆಲ್ಲ ಏನೆಂದುಕೊಂಡಿರುವರೋ! ನೀನು ಹೋಗಿ ನನ್ನ ತಾಯ್ತಂದೆಗಳಾದ ಯಶೋದೆ ನಂದ ರಿಗೆ ನನ್ನ ನಮಸ್ಕಾರಗಳನ್ನು ಹೇಳಿ, ನಾನು ಕ್ಷೇಮವಾಗಿರುವೆನೆಂದು ತಿಳಿಸು. ಅಲ್ಲಿನ ಗೋಪಿಯರಂತೂ, ನನ್ನಲ್ಲಿಯೇ ಪ್ರಾಣವಿಟ್ಟುಕೊಂಡಿರುವರು. ನನ್ನ ಅಗಲಿಕೆಯಿಂದ ಅವರು ಎಷ್ಟು ಸಂಕಟಪಡುತ್ತಿರುವರೊ! ಪತಿಪುತ್ರರನ್ನೂ ಮನೆಮಠಗಳನ್ನೂ ತೊರೆದು, ನನ್ನನ್ನೆ ಪರದೈವವೆಂದು ಭಾವಿಸಿದ್ದ ಆ ಹೆಣ್ಣುಗಳನ್ನು ಮರೆತರೆ ಹೇಗೆ? ನೀನು ಹೋಗಿ, ನನ್ನ ಪರವಾಗಿ ಅವರನ್ನು ಸಮಾಧಾನಮಾಡಿ ಬಾ’ ಎಂದನು. ಆತನ ಅಪ್ಪಣೆಯಂತೆ ಉದ್ಧವನು ರಥವೇರಿ ನಂದಗೋಕುಲಕ್ಕೆ ತೆರಳಿದನು.

ಉದ್ಧವನು ನಂದಗೋಕುಲವನ್ನು ಸೇರಿದಾಗ ಸಂಜೆಯಾಗಿತ್ತು. ಒಂದು ಕಡೆ ಹಾಲು ಕರೆಯುವ ದನಿ, ಇನ್ನೊಂದು ಕಡೆ ಕೊಳಲದನಿ, ಮತ್ತೊಂದು ಕಡೆ ಗೋಪಿಯರು ಶ್ರೀಕೃಷ್ಣನ ಲೀಲೆಗಳನ್ನು ಹಾಡುತ್ತಿರುವ ಇನಿದನಿ, ಪ್ರಕೃತಿಯ ಮಧ್ಯದಲ್ಲಿ ಮಾನವ ದ್ವೀಪದಂತಿದ್ದ ಆ ಗೋಕುಲದ ಸುತ್ತಲಿಂದಲೂ ಕೇಳಿ ಬರುತ್ತಿದ್ದ, ಸಹಸ್ರಾರು ಬಗೆಯ ಹಕ್ಕಿಗಳ ಕಲಕಲ ಧ್ವನಿ–ಇವುಗಳನ್ನು ಕೇಳುತ್ತಾ, ಅಲ್ಲಲ್ಲಿಯೇ ಜಿಂಕೆಯ ಮರಿಗಳಂತೆ ಹಾರಾಡುತ್ತಿರುವ ಆಕಳ ಕರುಗಳನ್ನು ನೋಡುತ್ತಾ, ಉದ್ಧವನು ಗೋಕುಲವನ್ನು ಪ್ರವೇಶಿಸಿ ದನು. ಆತನನ್ನು ಕಾಣುತ್ತಲೆ ನಂದನು ಸಾಕ್ಷಾತ್ ಶ್ರೀಕೃಷ್ಣನನ್ನೇ ಕಂಡಷ್ಟು ಸಂತೋಷ ಸಡಗರಗಳಿಂದ ಆತನನ್ನು ಅಪ್ಪಿಕೊಂಡು, ಊಟ ಉಪಚಾರಗಳಿಂದ ಆತನನ್ನು ತಣಿಸಿದ ಮೇಲೆ ಮಂಚದ ಮೇಲೆ ಆತನನ್ನು ಕುಳ್ಳಿರಿಸಿ ಕಾಲೊತ್ತುತ್ತ ‘ಏನಪ್ಪ ಉದ್ಧವ, ಮಧುರೆಯ ಸಮಾಚಾರ? ಕಂಸ ಸತ್ತುಹೋದಮೇಲೆ ನನ್ನ ಗೆಳೆಯ ವಸುದೇವ ಈಗ ಸುಖವಾಗಿದ್ದಾನೆ ತಾನೆ? ನಮ್ಮ ಕೃಷ್ಣ ಕ್ಷೇಮವಾಗಿದ್ದಾನೆಯೊ? ಅವನಿಗೆ ಈ ನಂದಗೋಕುಲ ನೆನಪಿದೆಯೋ ಇಲ್ಲವೇ ಇಲ್ಲವೋ? ಅವನು ಇತ್ತಕಡೆ ಒಮ್ಮೆಯಾದರೂ ಬರುತ್ತಾನಂತೋ? ಇಲ್ಲಿನ ಅಪ್ಪ, ಅಮ್ಮ, ಗೆಳೆಯರು, ಗೋಗಳು, ಗೋವರ್ಧನಪರ್ವತ, ಬೃಂದಾವನ–ಇವುಗಳ ನ್ನೆಲ್ಲಾ ಅವನೀಗ ನೆನಸಿಕೊಳ್ಳುತ್ತಾನೊ, ಇಲ್ಲವೆ ಇಲ್ಲವೊ! ಅವನು ನೆನಸಲಿ, ಬಿಡಲಿ; ನಾವಂತೂ ಅವನನ್ನು ಸದಾ ನೆನೆಸಿಕೊಳ್ಳುತ್ತೇವೆ. ಅವನು ತೋರಿಸಿದ ಅದ್ಭುತಗಳು ಒಂದೆ, ಎರಡೆ? ಎಂತೆಂತಹ ಅಪಾಯಗಳಿಂದ ಆತ ನಮ್ಮನ್ನು ಪಾರುಮಾಡಿದ! ನೆನೆಸಿಕೊಂಡರೆ ನಮ್ಮ ಮನೆಕೆಲಸಗಳೊಂದೂ ತೋಚದಂತಾಗುತ್ತದೆ. ಅಲ್ಲ, ಹತ್ತು ಸಾವಿರ ಆನೆಯ ಬಲವುಳ್ಳ ಆ ಕಂಸನನ್ನು ಲೀಲಾಜಾಲವಾಗಿ ಕೊಂದುಹಾಕಿದನಲ್ಲಾ! ಅವ ನನ್ನು ಹುಡುಗ ಎಂದು ಹೇಳುವುದಕ್ಕೆ ಸಾಧ್ಯವೇನಪ್ಪ? ಅವನು ನಿಜವಾಗಿಯೂ ದೇವರೆ ಹೊರತು ಮನುಷ್ಯನಲ್ಲಪ್ಪ’ ಹೀಗೆ ಮಾತನಾಡುತ್ತಿದ್ದಂತೆ ಆತನ ಕಣ್ಣಲ್ಲಿ ನೀರು ತುಳು ಕಿತು, ಗಂಟಲು ಕಟ್ಟಿಕೊಂಡಿತು, ಆತನು ಮುಂದಕ್ಕೆ ಮಾತನಾಡಲಾರದೆ ಮೌನಿಯಾದನು. ಹತ್ತಿರದಲ್ಲೆ ನಿಂತು ಇದನ್ನು ಕೇಳುತ್ತಿದ್ದ ಯಶೋದೆಗೂ ಆನಂದಬಾಷ್ಪ ಸುರಿಯಿತು.

ನಂದ ಯಶೋದೆಯರಿಗೆ ಶ್ರೀಕೃಷ್ಣನಲ್ಲಿರುವ ವಾತ್ಸಲ್ಯ ಭಕ್ತಿಯನ್ನು ಕಂಡು ಉದ್ಧವನಿಗೆ ಪರಮ ಸಂತೋಷವಾಯಿತು. ಆತನು ನಂದನೊಡನೆ ‘ಅಯ್ಯಾ, ಲೋಕ ಗುರುವಾದ ಶ್ರೀಕೃಷ್ಣನಲ್ಲಿ ಇಷ್ಟು ಪ್ರೇಮದಿಂದಿರುವ ನೀವು ಧನ್ಯರು. ನೀವು ಆಗಲೆ ತಿಳಿದಿರುವಂತೆ ಆತನು ಸಾಮಾನ್ಯನಲ್ಲ, ಸಾಕ್ಷಾತ್ ಪರಮೇಶ್ವರ. ಆತನು ಸ್ವಲ್ಪ ಕಾಲ ದಲ್ಲಿಯೇ ಇಲ್ಲಿಗೆ ಬಂದು ನಿಮ್ಮನ್ನು ಸಂತೋಷಪಡಿಸುತ್ತಾನೆ. ಅಪ್ಪ, ಆ ಶ್ರೀಕೃಷ್ಣ ನಿಮಗೆ ಮಾತ್ರ ಮಗನೆ? ಲೋಕಕ್ಕೆಲ್ಲ ಆತ ಮಗ, ಅಪ್ಪ, ಅಮ್ಮ, ದೇವರು ಅಷ್ಟೇಕೆ? ಹಿಂದೆ ಇದ್ದ, ಈಗ ಇರುವ, ಮುಂದೆ ಬರುವ, ನಾವು ಕಾಣುವ, ಕೇಳುವ–ಎಲ್ಲ, ಎಲ್ಲವೂ ಆತನೆ’ ಎಂದನು. ಶ್ರೀಕೃಷ್ಣಮಹಿಮೆಯನ್ನು ಹೇಳುವ ಉದ್ಧವ, ಕೇಳುವ ನಂದ–ಇಬ್ಬರೂ ಭಕ್ತಿಭಾವದಿಂದ ಮೈಮರೆತಿದ್ದರು. ಅವರ ಅರಿವಿಲ್ಲದಂತೆಯೇ ಇರುಳು ಕಳೆದು ಬೆಳಗಾಯಿತು. ಹಕ್ಕಿಗಳ ಇಂಚರಕ್ಕಿಂತಲೂ ಇನಿದಾದ ದನಿಯಲ್ಲಿ ಗೋಪಿ ಯರು ಶ್ರೀಕೃಷ್ಣನ ಲೀಲೆಗಳನ್ನು ಹಾಡುತ್ತಾ ಮೊಸರು ಕಡೆಯುವ ದನಿ ಕೇಳಿ ಬಂತು. ಮಾತಾಡುತ್ತಿದ್ದವರಿಬ್ಬರೂ ಮೇಲಕ್ಕೆದ್ದು ಪ್ರಾತಃಕೃತ್ಯಗಳಿಗೆಂದು ಹೊರಟುಹೋದರು.

ಮೂಡಲ ಕೆಂಪು ಹರಿದು ಬೆಳ್ಳಗೆ ಬೆಳಕಾಯಿತು. ಗೋಪಿಯರು ತಮ್ಮ ಮನೆಗಳಿಂದ ಹೊರಗೆ ಬಂದರು. ಗೋಕುಲದ ಬಾಗಿಲಲ್ಲಿ ಸುಂದರವಾದ ರಥವೊಂದು ಬಂದು ನಿಂತಿ ರುವುದು ಅವರಿಗೆ ಕಾಣಿಸಿತು. ತಕ್ಷಣವೇ ಅವರ ಮನಸ್ಸಿನಲ್ಲಿ ಶ್ರೀಕೃಷ್ಣನನ್ನು ಮಧುರೆಗೆ ಕರೆದೊಯ್ದ ರಥ ಜ್ಞಾಪಕಕ್ಕೆ ಬಂತು. ಅವರು ‘ಓಹೋ, ಆ ಕ್ರೂರನಾದ ಅಕ್ರೂರ ಮತ್ತೆ ಬಂದನೋ? ಅವನಿಗೇನು ಕೆಲಸ ಇnಲ್ಲಿ? ಸತ್ತ ಕಂಸನಿಗೆ ನಮ್ಮ ಮಾಂಸದಿಂದ ಪಿಂಡ ಹಾಕಬೇಕಂತೋ!’ ಎಂದು ರೇಗಿದರು. ಆ ವೇಳೆಗೆ ಸರಿಯಾಗಿ ಉದ್ಧವ ತನ್ನ ಬೆಳಗಿನ ಸಂಧ್ಯೆಯನ್ನು ಮುಗಿಸಿ ಅಲ್ಲಿಗೆ ಬಂದನು. ಆತನನ್ನು ಕಾಣುತ್ತಲೇ ಗೋಪಿಯರಿಗೆ ಶ್ರೀಕೃಷ್ಣನನ್ನೆ ಕಂಡಂತಾಯಿತು. ಅವರು ಪರಸ್ಪರ ‘ಯಾರೀತ? ನಮ್ಮ ಶ್ರೀಕೃಷ್ಣನಂತೆಯೇ ವೇಷ, ಭೂಷಣ. ಅದೇ ಮುಗುಳ್ನಗೆ; ಅದೇ ಸುಂದರ ನೋಟ! ಇವನು ಎಲ್ಲಿಂದ ಬಂದ? ಯಾರ ಕಡೆಯವನು?’ ಎಂದು ಮಾತನಾಡಿಕೊಳ್ಳುತ್ತಾ ಬಂದು ಆತನ ಸುತ್ತ ನೆರೆದರು. ಅವರ ಮನಸ್ಸು ಹೇಳಿತು–‘ಈತ ಶ್ರೀಕೃಷ್ಣನ ಕಡೆಯವನೇ! ಆತನ ವಿಚಾರವನ್ನು ತಿಳಿಸುವುದಕ್ಕಾಗಿಯೇ ಇಲ್ಲಿಗೆ ಬಂದಿದ್ದಾನೆ’ ಎಂದು. ಅವರ ಮನಸ್ಸು ನಿಲ್ಲದೆ ಬಾಯ್ಬಿಟ್ಟು ಅವನನ್ನು ಕೇಳಿದರು ‘ಅಯ್ಯಾ ಮಹಾತ್ಮಾ, ನೀನು ಶ್ರೀಕೃಷ್ಣನ ಕಡೆಯಿಂದ ಬಂದಿರು ವೆಯಾ? ಬಹುಶಃ ಆತ ತನ್ನ ತಾಯ್ತಂದೆಗಳ ಯೋಗಕ್ಷೇಮವನ್ನು ವಿಚಾರಿಸುವುದಕ್ಕಾಗಿ ನಿನ್ನನ್ನು ಕಳುಹಿಸಿರಬೇಕು. ಆತನಿಗೆ ಇಲ್ಲಿ ಇನ್ನಾರ ಜ್ಞಾಪಕವಿರುತ್ತದೆ? ಇಲ್ಲಿರುವ ನಾವೆಲ್ಲ ಏನು ಆತನ ಬಂಧುಗಳೇ? ಕೇವಲ ರುಚಿ ನೋಡಿ ಬಿಸುಟ ಅಡವಿಯ ಹಣ್ಣುಗಳು. ದುಂಬಿ ಬಂಡನ್ನು ಕುಡಿದ ಮೇಲೆ ಆ ಹೂವನ್ನು ಮತ್ತೆ ಮೂಸುತ್ತದೆಯೆ? ಹಾಗೆಯೇ ವಿಟ ಪುರುಷರು. ಹಣ್ಣು ತಿನ್ನುವುದು ಮುಗಿಯುತ್ತಲೆ ಹಕ್ಕಿ ಆ ಮರವನ್ನು ಬಿಟ್ಟು ಹೋಗುವಂತೆ ಜಾರನು ಕಾಮತೃಪ್ತಿಯಾಗುತ್ತಲೆ ಆ ಹೆಣ್ಣನ್ನು ಬಿಟ್ಟು ಹೋಗುತ್ತಾನೆ. ಶ್ರೀಕೃಷ್ಣನೂ ನಮ್ಮನ್ನು ಹಾಗೆಯೇ ಮಾಡಿದ. ಅಯ್ಯೋ, ಮತ್ತೆ ಅವನು ನಮಗೆ ಗೋಚರಿಸುತ್ತಾ ನೆಯೇ?’ ಎಂದು ಗಳಗಳ ಅತ್ತರು. ಅವರಲ್ಲಿ ಒಬ್ಬಳಂತೂ ತನ್ನ ಮುಂದೆ ಹಾರಾಡುತ್ತಿದ್ದ ದುಂಬಿಯೊಂದನ್ನು ಕಂಡು, ಅದನ್ನೇ ಶ್ರೀಕೃಷ್ಣನ ದೂತನೆಂದು ಭ್ರಮಿಸಿ, ಅದರ ಮುಂದೆ ತನ್ನ ದುಃಖವನ್ನೆಲ್ಲಾ ತೋಡಿಕೊಂಡಳು. (ಪರಿಶಿಷ್ಟ–‘ಭ್ರಮರಗೀತ’ ನೋಡಿ)

ಗೋಪಿಯರ ವಿರಹವೇದನೆಯನ್ನು ಅರ್ಥಮಾಡಿಕೊಂಡ ಉದ್ಧವನು ಅವರನ್ನು ಕುರಿತು “ಅಮ್ಮ ಗೋಪಿಯರೆ, ಲೋಕದಲ್ಲಿ ನೀವೇ ಪುಣ್ಯಶಾಲಿಗಳು, ನೀವೇ ಧನ್ಯರು! ಭಗವಂತನಾದ ಶ್ರೀಕೃಷ್ಣನಲ್ಲಿ ನಿಮಗಿರುವಂತಹ ಪ್ರೇಮ ಮೂರು ಲೋಕದಲ್ಲಿ ಹುಡುಕಿ ದರೂ ಮತ್ತಾರಲ್ಲಿಯೂ ಸಿಕ್ಕುವುದಿಲ್ಲ. ಅನೇಕ ಕಾಲದವರೆಗೆ ಜಪ, ತಪ, ಹೋಮಾದಿ ಗಳನ್ನು ಮಾಡಿದವರಲ್ಲಿಯೂ ಇಂತಹ ಏಕಾಗ್ರತೆ ಹುಟ್ಟುವುದಿಲ್ಲ. ಇದು ಮಹಾಯೋಗಿ ಗಳಿಗೂ ದುರ್ಲಭ. ವಿರಹತಾಪದ ಹೆಸರಿನಲ್ಲಿ ಕಾಯಾ ವಾಚಾ ಮನಸಾ ಶ್ರೀಕೃಷ್ಣನ ಆರಾ ಧನೆಯನ್ನು ನಡೆಸುತ್ತಿರುವ ನೀವು ಧನ್ಯರು; ನಿಮ್ಮಂತಹ ಮಹಾನುಭಾವರನ್ನು ಕಂಡು ನಾನು ಧನ್ಯನಾದೆ, ಶ್ರೀಕೃಷ್ಣನು ನಿಮ್ಮನ್ನು ಕುರಿತು ‘ಎಲೆ ಗೋಪಿಯರೆ, ನನಗೂ ನಿಮಗೂ ಅಗಲಿಕೆಯೆಂಬುದೇ ಇಲ್ಲ. ನಾನು ಆತ್ಮರೂಪನಾಗಿ ಸದಾ ನಿಮ್ಮಲ್ಲಿಯೇ ನೆಲೆಸಿದ್ದೇನೆ. ಕಾರಣವನ್ನು ಬಿಟ್ಟು ಕಾರ್ಯ ಹೇಗೆ ಇರಲು ಸಾಧ್ಯವಿಲ್ಲವೋ, ಹಾಗೆಯೇ ನನ್ನನ್ನು ಬಿಟ್ಟು ನೀವಿರುವುದಕ್ಕೆ ಸಾಧ್ಯವಿಲ್ಲ. ಪ್ರಿಯ ಸಖಿಯರೆ, ನಾನು ಈಗ ನಿಮ್ಮ ಕಣ್ಣಿಗೆ ದೂರವಾಗಿ ದ್ದೇನೆ. ಅದಕ್ಕೆ ಕಾರಣ, ನಿಮ್ಮ ಮನಸ್ಸನ್ನು ನನಗೆ ಹತ್ತಿರವಾಗಿ ಮಾಡಿಕೊಳ್ಳುವುದು. ನೀವು ನಿಮ್ಮ ಮನಸ್ಸನ್ನು ಸದಾ ನನ್ನ ಮೇಲಿಟ್ಟರೆ, ಬಹು ಬೇಗ ನನ್ನ ನಿತ್ಯಸಾನಿಧ್ಯವು ನಿಮಗೆ ಸಾಧ್ಯವಾಗುತ್ತದೆ’ ಎಂದು ಹೇಳಿಕಳುಹಿಸಿದ್ದಾನೆ” ಎಂದನು. ಅದನ್ನು ಕೇಳಿ ಗೋಪಿಯರ ಹೃದಯಕ್ಕೆ ಎಷ್ಟೋ ತಂಪಾಯಿತು. ಅವರು ಹೇಳಿದರು, ‘ಮಹಾನುಭಾವನಾದ ಉದ್ಧವಾ! ಕಂಸನನ್ನು ಕೊಂದು ಮಧುರೆಗೆ ಒಡೆಯನಾಗಿರುವ ಶ್ರೀಕೃಷ್ಣನಿಗೆ ಈಗ ರಾಜಕುಮಾರಿ ಯರೆ ಹಾರ ಹಿಡಿದು ಕಾದಿದ್ದಾರೆ. ಎಂದಮೇಲೆ ನಮ್ಮಂತಹ ಕಾಡು ಹೆಣ್ಣುಗಳು ಏಕೆ ಬೇಕು? ಹಿಂದೆ ಕೃಷ್ಣನೊಡನೆ ಕಲೆತ ಪಿಂಗಳೆಯೆಂಬ ವೇಶ್ಯೆ ‘ಆತನ ವಿಷಯದಲ್ಲಿ ನಿರಾಶೆಯಿಂದಿರುವುದೇ ಪರಮಸುಖ’ ಎಂದು ತನ್ನ ಅನುಭವವನ್ನು ಹೇಳಿಕೊಂಡಿದ್ದಳು. ಆ ಮಾತು ಎಂತಹ ಸತ್ಯ! ಆದರೂ ನಮ್ಮ ಹಾಳು ಮನಸ್ಸು ಮತ್ತೆ ಮತ್ತೆ ಆತನನ್ನು ನೆನೆ ಯುತ್ತದೆ. ರಾಸಕ್ರೀಡೆಯಲ್ಲಿ ಆತನೊಡನೆ ಕಳೆದ ರಸನಿಮಿಷಗಳು ಮತ್ತೆಮತ್ತೆ ನೆನಪಿನಲ್ಲಿ ನುಗ್ಗಿ ಬಂದು, ನಮ್ಮ ಹೃದಯಗಳಲ್ಲಿ ತಕತಕ ಕುಣಿಯುತ್ತವೆ. ಬೇಸಗೆಯಿಂದ ಒಣಗಿ ಹೋದ ಅಡವಿಗೆ ಮಳೆ ಬಂದಂತೆ, ವಿರಹವೇದನೆಯಿಂದ ಬರಡಾಗಿರುವ ನಮ್ಮ ಹೃದಯ ಗಳಿಗೆ ಆತನ ಆಲಿಂಗನ ಸುಖ ಎಂದು ದೊರೆಯುವುದೋ! ಹೇ ಕೃಷ್ಣ, ನಮ್ಮನ್ನು ಉದ್ಧರಿಸು’ ಹೀಗೆಂದು ಅವರು ಕಣ್ಮುಚ್ಚಿ ಧ್ಯಾನ ಮಾಡಿದರು.

ಗೋಪಿಯರ ಮಧುರಭಕ್ತಿಯನ್ನು ಕಂಡು ಮುಗ್ಧನಾದ ಉದ್ಧವನು ಶ್ರೀಕೃಷ್ಣನ ಕಥೆ ಗಳನ್ನು ಗೋಪಿಯರಿಗೆ ಹೇಳುತ್ತಾ ಕೆಲಕಾಲ ಗೋಕುಲದಲ್ಲಿಯೇ ನೆಲಸಿದ್ದು, ಅನಂತರ ಅವರೆಲ್ಲರಿಂದ ಬೀಳ್ಕೊಂಡು ಮಧುರೆಗೆ ಹಿಂದಿರುಗಿದನು.

