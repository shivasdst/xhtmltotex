
\chapter{೫೫. ಕುಚೇಷ್ಟೆಯಿಂದ ಪ್ರತಿಷ್ಠೆ ಬರುತ್ತದೆಯೆ?}

ಗೋಕುಲದಲ್ಲಿ ‘ಇಂದ್ರಯಾಗ’ವು ಪ್ರತಿವರ್ಷವೂ ನಡೆಯುವ ಬಹುದೊಡ್ಡ ಉತ್ಸವ. ಕಾಲ ಕಾಲಕ್ಕೆ ಮಳೆಯನ್ನು ಕರೆದು, ಜನರ ತುಷ್ಟಿ ಪುಷ್ಟಿಗಳಿಗೆ ಕಾರಣನಾದ ದೇವೇಂದ್ರನಿಗೆ ಪ್ರೀತಿಯಾಗಲೆಂದು ಆ ಉತ್ಸವ ನಡೆಯುತ್ತಿತ್ತು. ಅನಾದಿಕಾಲದಿಂದ ವಂಶಪರಂಪರೆ ಯಾಗಿ ನಡೆದು ಬರುತ್ತಿದ್ದ ಆ ಹಬ್ಬವನ್ನು ವೈಭವದಿಂದ ಆಚರಿಸಬೇಕೆಂದು ಗೋಕುಲ ದವರೆಲ್ಲ ಸಡಗರ ಪಡುತ್ತಿರಲು, ಶ್ರೀಕೃಷ್ಣನು ತನ್ನ ತಂದೆಯಾದ ನಂದನೊಡನೆ ‘ಅಪ್ಪ, ನಿಮ್ಮದು ಮೂಢನಂಬಿಕೆ. ನಮಗೂ ದೇವೇಂದ್ರನಿಗೂ ಏನು ಸಂಬಂಧ? ನಾವೇಕೆ ಅವನನ್ನು ಪೂಜಿಸಬೇಕು? ನಾವು ಗೊಲ್ಲರು, ಗೋವುಗಳಿಂದ ನಮ್ಮ ಜೀವನ; ಕಾಡು ಬೆಟ್ಟಗಳು ನಮ್ಮ ವಾಸಸ್ಥಾನ; ಬ್ರಾಹ್ಮಣರು ನಮ್ಮ ಕುಲಗುರುಗಳು; ಆದ್ದರಿಂದ ಗೋವು ಗಳನ್ನೂ, ಕಾಡುಬೆಟ್ಟಗಳನ್ನೂ, ಬ್ರಾಹ್ಮಣರನ್ನೂ ಪೂಜಿಸುವುದರಲ್ಲಿ ಅರ್ಥವಿದೆ. ನಮಗೂ ದೇವೇಂದ್ರನಿಗೂ ಏನು ಸಂಬಂಧ? ದನ ಕಾಯುವುದು ನಮ್ಮ ಕರ್ಮವಾದರೆ, ಮಳೆಗರೆಯುವುದು ಅವನ ಕರ್ಮ. ಅವನವನ ಕರ್ಮವನ್ನು ಅವನು ಮಾಡಬೇಕು. ಆ ಕರ್ಮಕ್ಕೆ ತಕ್ಕ ಫಲ ಅವನಿಗೆ ದೊರೆಯುತ್ತದೆ. ಸರ್ವೇಶ್ವರನೂ ಕೂಡ ಫಲವನ್ನು ಕೊಡು ವಾಗ ಕರ್ಮಕ್ಕೆ ತಕ್ಕ ಫಲವನ್ನು ಮಾತ್ರ ಕೊಡಬಲ್ಲ. ಎಲ್ಲ ವಿಚಾರಗಳಲ್ಲಿಯೂ ಸರ್ವ ತಂತ್ರಸ್ವತಂತ್ರನಾದ ಆತನು ಈ ವಿಚಾರದಲ್ಲಿ ಮಾತ್ರ ಕೇವಲ ಪರತಂತ್ರ. ಎಂದಮೇಲೆ, ನಮ್ಮ ಕರ್ಮಕ್ಕೆ ತಕ್ಕ ಫಲವನ್ನು ನಾವು ನಿರೀಕ್ಷಿಸಬೇಕೇ ಹೊರತು, ದೇವೇಂದ್ರನ ದಯೆ ಯನ್ನಲ್ಲ. ಜಗತ್ತಿನ ವ್ಯಾಪಾರವೆಲ್ಲ ನಡೆಯುವುದು ಸತ್ವ, ರಜ, ತಮ–ಎಂಬ ಮೂರು ಗುಣಗಳಿಂದ. ಮೇಘಗಳಿಂದ ಮಳೆ ಬರುವುದಕ್ಕೆ ರಜೋಗುಣವೆ ಕಾರಣವಲ್ಲದೆ ದೇವೇಂದ್ರನಲ್ಲ. ಆದ್ದರಿಂದ ದೇವೇಂದ್ರನ ಪೂಜೆಯನ್ನು ನಿಲ್ಲಿಸಿ, ಆ ಪೂಜಾವಸ್ತು ಗಳಿಂದಲೆ ಗೋವರ್ಧನ ಪರ್ವತವನ್ನು ಪೂಜೆಮಾಡಿರಿ. ಬ್ರಹ್ಮಜ್ಞರಾದ ಬ್ರಾಹ್ಮಣರಿಗೆ ಅನ್ನವನ್ನಿಕ್ಕಿರಿ, ಗೋದಾನ ಮಾಡಿರಿ. ಹಾಗೆಯೆ ನಮ್ಮ ಮಂದೆಯನ್ನು ಕಾಯುವ ನಾಯಿ ಗಳಿಗೂ, ಅವರಿವರೆನ್ನದೆ ಮಾನವರೆಲ್ಲರಿಗೂ ಹೊಟ್ಟೆತುಂಬ ಊಟಮಾಡಿಸಿರಿ. ನಮ್ಮ ಗೋಗಳಿಗೆಲ್ಲ ಬೇಕಾದಷ್ಟು ಮೇವನ್ನು ಹಾಕಿರಿ. ಇದು ನನ್ನ ಅಭಿಪ್ರಾಯ’ ಎಂದನು. 

ನಂದನಿಗೆ ಮಗನ ಮಾತು ಹಿಡಿಸಿತು. ಶ್ರೀಕೃಷ್ಣನ ಅಭಿಪ್ರಾಯದಂತೆ ಗೊಲ್ಲರ ಗಂಡು ಹೆಣ್ಣುಗಳೆಲ್ಲ ಹೊಸ ಬಟ್ಟೆಗಳನ್ನುಟ್ಟು ಗೋವರ್ಧನಪರ್ವತವನ್ನು ಪೂಜಿಸಿ, ತಮ್ಮ ಗೋಗಣದೊಡನೆ ಅದನ್ನು ಪ್ರದಕ್ಷಿಣೆ ಮಾಡಿದರು. ಅವರಲ್ಲಿ ಕೆಲವರಿಗೆ ಈ ಹೊಸ ಪದ್ಧತಿಯಲ್ಲಿ ಅಷ್ಟು ನಂಬಿಕೆಯಿಲ್ಲ. ಬೆಟ್ಟಕ್ಕೆಂಥ ಪೂಜೆಯೆಂದು ಮನಸ್ಸಿನಲ್ಲಿಯೆ ಮಿಡುಕಿಕೊಂಡರು. ಸರ್ವಾಂತರ್ಯಾಮಿಯಾದ ಶ್ರೀಕೃಷ್ಣನು ಇದನ್ನು ಅರಿತು, ಅವರ ಸಂದೇಹ ನಿವಾರಣೆಗಾಗಿ ಒಂದು ಉಪಾಯ ಮಾಡಿದನು. ತಾನು ನಿಂತ ಕಡೆಯೆ ನಿಂತಿದ್ದು, ಮತ್ತೊಂದು ಮಹದಾಕಾರದಿಂದ ಅವರಿಗೆ ಕಾಣಿಸಿಕೊಂಡು ‘ಹೇ ಗೋಪಾಲ, ನಾನೆ ಗೋವರ್ಧನಪರ್ವತ. ನಿಮ್ಮ ಪೂಜೆಯಿಂದ ನಾನು ಸಂತುಷ್ಟನಾಗಿದ್ದೇನೆ. ಇಗೋ, ನೀವು ನೈವೇದ್ಯಕ್ಕಿಟ್ಟಿರುವ ಆಹಾರವನ್ನೆಲ್ಲ ತಿನ್ನುತ್ತೇನೆ’ ಎಂದು ಹೇಳಿ, ತಾನು ಬಾಯಿಂದ ಹೇಳಿದಂತೆ ಮಾಡಿ ತೋರಿಸಿದನು. ಆಗ ಬಾಲರೂಪಿಯಾದ ಶ್ರೀಕೃಷ್ಣನು ನಂದನೊಡನೆ ‘ಅಪ್ಪ, ನೋಡಿದಿರಾ? ಪರ್ವತ ಹೇಗೆ ಮನುಷ್ಯರೂಪದಿಂದ ಬಂದು ಭಕ್ತರಾದ ನಮ್ಮನ್ನು ಅನುಗ್ರಹಿಸುತ್ತಾ ಇದೆ! ನಾವು ಅದನ್ನು ಪೂಜಿಸದೆ ಉದಾಸೀನ ಮಾಡಿದರೆ ನಮ್ಮನ್ನು ಧ್ವಂಸ ಮಾಡಿಬಿಡುತ್ತದೆ. ನಾವು ಭಕ್ತಿಯಿಂದ ಅದಕ್ಕೆ ನಮಸ್ಕಾರ ಮಾಡೋಣ’ ಎಂದು ಹೇಳಿ, ಎಲ್ಲರಿಗಿಂತ ಮೊದಲು ತಾನೆ ಅದಕ್ಕೆ ಅಡ್ಡಬಿದ್ದನು. ಇದನ್ನು ಕಂಡ ಉಳಿದವರೂ ಅದಕ್ಕೆ ಅಡ್ಡಬಿದ್ದರು.

ಹೀಗೆ ‘ಇಂದ್ರಯಾಗ’ವು ‘ಗೋವರ್ಧನಪೂಜೆ’ಯಾಗಿ ಬದಲಾಯಿಸಿದುದನ್ನು ಕಂಡು ದೇವೇಂದ್ರನಿಗೆ ರೇಗಿಹೋಯಿತು. ಕೇವಲ ಏಳು ವರ್ಷದ ಬಾಲಕನೊಬ್ಬ ಅರ್ಥವಿಲ್ಲದೆ ಆಡಿದ ಮಾತಿಗೆ, ನಂದನೇ ಮೊದಲಾದ ಗೋಪಾಲರೆಲ್ಲ ತಲೆದೂಗಿ ತನ್ನ ಪೂಜೆಯನ್ನು ಬಿಟ್ಟುಬಿಡುವುದೆಂದರೆ ತನಗೆಂತಹ ಅಪಮಾನ! ಇದಕ್ಕೆ ತಕ್ಕ ಶಿಕ್ಷೆ ಮಾಡಬೇಕೆಂದು ಕೊಂಡ, ಆ ದೇವೇಂದ್ರ. ತಕ್ಷಣವೆ ಆತ ‘ಸಂವರ್ತಕ’ವೆಂಬ ಹೆಸರಿನ ಪ್ರಳಯಕಾಲದ ಮೇಘಗಳನ್ನು ಕರೆಸಿ ‘ಎಲೆ ಮೇಘಗಳೆ, ನೀವು ಈಗಲೆ ಗೋಕುಲದ ಮೇಲೆ ಭಯಂಕರ ವಾದ ಮಳೆಯನ್ನು ಕರೆದು, ಅದನ್ನು ಜಲಪ್ರಳಯ ಮಾಡಿ ಬನ್ನಿ’ ಎಂದು ಅಪ್ಪಣೆ ಮಾಡಿ ದನು. ಒಡೆಯನ ಅಪ್ಪಣೆಯಂತೆ ಅವು ಗುಡುಗು ಮಿಂಚು ಸಿಡಿಲುಗಳೊಡನೆ ಆಲಿಕಲ್ಲಿನ ಮಳೆಯನ್ನು ಸುರಿಸಲು ಪ್ರಾರಂಭಿಸಿದವು. ಕ್ಷಣ ಮಾತ್ರದಲ್ಲಿ ಗೋಕುಲವೆಲ್ಲ ನೀರಿನ ಮಡುವಿನಂತಾಯಿತು. ದನಗಳೆಲ್ಲ ನೀರಿನಲ್ಲಿ ನೆನೆದು ಗಡಗಡ ನಡುಗುತ್ತ ನಿಂತವು. ಜನರೂ ಆ ಚಳಿಯನ್ನು ತಡೆಯಲಾರದೆ ‘ಶ್ರೀಕೃಷ್ಣ, ನೀನೇ ಗತಿ’ ಎಂದು ಕೂಗಿಕೊಂಡರು. ಅವರ ಆರ್ತನಾದಕ್ಕೆ ಮನ ಕರಗಿದ ಶ್ರೀಕೃಷ್ಣನು ತನ್ನಲ್ಲಿಯೇ ‘ಎಲ ಎಲ, ಈ ದೇವೇಂದ್ರ ಎಂತಹ ಕುಚೇಷ್ಟೆಯನ್ನು ಪ್ರಾರಂಭಿಸಿದ್ದಾನೆ! ಕುಚೇಷ್ಟೆಯಿಂದ ಪ್ರತಿಷ್ಠೆ ಬರುತ್ತದೆಯೆ? ತಾನೆ ಲೋಕಪಾಲನೆಂಬ ಗರ್ವ ಇವನ ನೆತ್ತಿಗೇರಿದೆ. ಇವನ ಪಿತ್ತವನ್ನು ಇಳಿಸುವುದಕ್ಕೆ ಇದು ಒಳ್ಳೆಯ ಸಮಯ’ ಎಂದುಕೊಂಡು, ತನ್ನ ಇದಿರಿಗಿದ್ದ ಗೋವರ್ಧನಪರ್ವತವನ್ನು ಕಾಲಿನಿಂದ ಮೀಟಿ, ತನ್ನ ಕೈಯಿಂದ ಅದನ್ನು ಮೇಲಕ್ಕೆ ಎತ್ತಿದನು. ಆ ದೊಡ್ಡ ಪರ್ವತ ಆತನ ಕೈಯ ಕೊಡೆಯಾಯಿತು. ಗೋಕುಲದ ಗೊಲ್ಲರೂ ಗೋಗಳೂ ಅದರ ಕೆಳಗೆ ಆಶ್ರಯ ಪಡೆದರು. ಏಳು ದಿನಗಳ ಕಾಲ ಹಗಲೂ ರಾತ್ರಿ ಮಳೆ ಸುರಿಯಿತು; ಶ್ರೀಕೃಷ್ಣನು ಏಳು ದಿನಗಳವರೆಗೆ ನಿಂತಲ್ಲಿ ಕದಲದೆ ನಿಂತು, ಗೋಕುಲದ ಜೀವಿಗಳಿಗೆ ಗೋವರ್ಧನ ಪರ್ವತದ ಕೊಡೆ ಹಿಡಿದನು. ಜನರು ನೆಮ್ಮದಿಯಾಗಿ ಅದರ ಕೆಳಗೆ ತಮ್ಮ ನಿತ್ಯದ ಕಾರ್ಯ ಗಳನ್ನು ನಡೆಸುತ್ತಿದ್ದರು.

 ಶ್ರೀಕೃಷ್ಣನ ಮಹಿಮೆಯಿಂದ ಗೋಪಾಲರ ಕೂದಲು ಕೊಂಕದಿರುವುದನ್ನು ಕಂಡು ದೇವೇಂದ್ರನಿಗೆ ನಾಚಿಕೆಯಾಯಿತು. ಆತ ಮಳೆಯ ಮೇಘಗಳನ್ನು ಅಲ್ಲಿಂದ ಹಿಂದಿರುಗು ವಂತೆ ಅಪ್ಪಣೆ ಮಾಡಿದನು. ಒಡನೆಯೆ ಮಳೆ ನಿಂತಿತು, ಮೋಡಗಳು ಚದುರಿದವು, ಬಿಸಿಲು ಕಾಣಿಸಿತು. ಗೋಪಾಲರು ತಮ್ಮ ತುರುಮಂದೆಗಳನ್ನು ಅಟ್ಟಿಕೊಂಡು ಹೊರಕ್ಕೆ ಬಂದರು. ಶ್ರೀಕೃಷ್ಣನು ತಾನು ಎತ್ತಿ ಹಿಡಿದಿದ್ದ ಪರ್ವತವನ್ನು ಅದರ ಸ್ಥಳದಲ್ಲಿ ಮತ್ತೆ ಸ್ಥಾಪಿಸಿದನು. ನಂದಗೋಪನೂ ಬಲರಾಮನೂ ಬಂದು ಆತನನ್ನು ಅಪ್ಪಿಕೊಂಡು ಮುದ್ದಾಡಿದರು. ಹೆಂಗಸರು ಮಕ್ಕಳೊಡನೆ ಎಲ್ಲರೂ ಗೋಕುಲಕ್ಕೆ ಹಿಂದಿರುಗಿದರು. ಎಲ್ಲ ಗೋಪಾಲರ ಮನಸ್ಸಿನಲ್ಲಿಯೂ ಒಂದೇ ವಿಚಾರ. ‘ಏಳು ವರ್ಷದ ಬಾಲಕನಾದ ಶ್ರೀಕೃಷ್ಣ ಗೋವರ್ಧನ ಪರ್ವತವನ್ನು ಕಿತ್ತೆತ್ತಿ ಹಿಡಿದನಲ್ಲಾ! ಇದು ಮನುಷ್ಯ ಮಾತ್ರರಿಂದ ಸಾಧ್ಯವೆ? ಮೊದಲಿನಿಂದ ಶ್ರೀಕೃಷ್ಣ ಇಂತಹ ಮಹಾಕಾರ್ಯಗಳನ್ನೆ ಮಾಡುತ್ತಾ ಬಂದಿದ್ದಾನೆ. ಎಳೆಯ ಕೂಸು ಪೂತನಿಯನ್ನು ಕೊಂದಿತು, ಮೂರು ತಿಂಗಳ ಮಗು ತುಂಬಿದ ಬಂಡಿಯನ್ನು ತಲೆಕೆಳಗು ಮಾಡಿತು, ಒಂದು ವರ್ಷದ ಕೂಸು ತೃಣಾವರ್ತನನ್ನು ಕತ್ತು ಹಿಸುಕಿ ಕೊಂದಿತು, ಅಂಬೆಗಾಲಿಡುವ ಮಗು ಮರಗಳನ್ನು ಮುರಿದು ಹಾಕಿತು, ಬಕಾಸುರ, ವತ್ಸಾ ಸುರ, ಗರ್ದಭಾಸುರ, ಪ್ರಲಂಬಾಸುರರೆಲ್ಲ ಆರು ವರ್ಷದ ಬಾಲಕನಿಗೆ ಆಹುತಿಯಾದರು, ಕಾಳಿಯ ಸರ್ಪ ಇವನ ಕಾಲ್ತುಳಿತಕ್ಕೆ ಸೊರಗಿ ಸೊಪ್ಪಾಯಿತು. ಈಗ ನೋಡಿದರೆ ಈ ಏಳು ವರ್ಷದ ಪೋರ ಗೋವರ್ಧನ ಪರ್ವತವನ್ನೆ ಅಂಗೈ ಬೆರಳಲ್ಲಿ ಆಡಿಸಿದ. ಇವನು ನಿಜವಾಗಿ ಮನುಷ್ಯನಲ್ಲ. ಯಾರೋ ನಮ್ಮ ಭಾಗ್ಯ ದೇವತೆ’–ಅವರ ಈ ವಿಚಾರ ಸರಣಿಯನ್ನು ನಂದ ನಿಸ್ಸಂದೇಹವಾಗಿ ಸ್ಥಾಪಿಸುತ್ತ ‘ನನಗೆ ಗರ್ಗಮುನಿಗಳು ಹಿಂದೆಯೇ ಹೇಳಿದ್ದರು, ಈತ ಮಹಾಪುರುಷನೆಂದು. ನಾನು ಈತನನ್ನು ಸಾಕ್ಷಾತ್ ಶ್ರೀಹರಿಯೆಂದೆ ನಂಬಿದ್ದೇನೆ’ ಎಂದು ಹೇಳಿದನು.

ಅತ್ತ ದೇವೇಂದ್ರನು ತನ್ನ ಅವಿವೇಕಕ್ಕಾಗಿ ನಾಚಿದನು. ಅವನ ಗಂಡಗರ್ವವೆಲ್ಲ ಸೋರಿ ಹೋಯಿತು. ಅಷ್ಟೇ ಅಲ್ಲ. ಆತನ ಮಹಿಮೆಯನ್ನು ಕೇಳಿದ್ದರೂ ಅಹಂಕಾರದಿಂದ ತಾನು ಕೃಷ್ಣನಿಗೆ ಮಾಡಿದ ಅಪಚಾರಕ್ಕಾಗಿ ಅವನಿಗೆ ಭಯವೂ ಹುಟ್ಟಿತು. ಆತನು ಶ್ರೀಕೃಷ್ಣನಲ್ಲಿ ಕ್ಷಮೆಯನ್ನು ಯಾಚಿಸುವುದಕ್ಕಾಗಿ ಸ್ವರ್ಗದಿಂದ ಇಳಿದು ಬಂದನು. ಆತನ ಜೊತೆಯಲ್ಲಿ ಕಾಮಧೇನುವೂ ಬಂದಿತು. ಆತನು ರಹಸ್ಯವಾಗಿ ಶ್ರೀಕೃಷ್ಣನನ್ನು ಕಂಡು ತನ್ನ ತಪ್ಪನ್ನು ಮನ್ನಿಸುವಂತೆ ಬೇಡಿ, ಆತನ ಪಾದಗಳಿಗೆ ಅಡ್ಡ ಬಿದ್ದನು. ಅನಂತರ ಆತನ ಇದಿರಿಗೆ ಕೈಮುಗಿದು ನಿಂತು ‘ಸ್ವಾಮಿ, ನಿನ್ನ ಮಹಿಮೆಯನ್ನು ಅರಿಯದೆ ಅಹಂಕಾರದಿಂದ ನಾನು ಅಪರಾಧ ಮಾಡಿದೆ. ನನ್ನನ್ನು ಕ್ಷಮಿಸು. ನಿನ್ನ ಆಗ್ರಹವೆ ಅನುಗ್ರಹವಾಗಿ ನನ್ನ ಕಣ್ಣು ತೆರೆಸಿತು. ನಾನು ನೀನಾರೆಂಬುದನ್ನು ಅರಿತು ಧನ್ಯನಾದೆ’, ಎಂದನು. ಶ್ರೀಕೃಷ್ಣನು ಧೀರ ಗಂಭೀರವಾಣಿಯಿಂದ ‘ಏನಯ್ಯ, ಕುಚೇಷ್ಟೆಯಿಂದ ಪ್ರತಿಷ್ಠೆ ಬರುವುದು ಎಂದು ಕೊಂಡೆಯಾ? ನಿನ್ನ ಅಹಂಕಾರವನ್ನು ಮುರಿಯುವುದಕ್ಕಾಗಿಯೆ ನಾನು ಗೋವರ್ಧನವನ್ನು ಎತ್ತಿ ಹಿಡಿಯಬೇಕಾಯಿತು. ಹೋಗಲಿ, ಇನ್ನು ಮುಂದೆ ಅಹಂಕಾರಿಯಾಗದಂತೆ ನಡೆದುಕೊ. ನಿನಗೆ ಮಂಗಳವಾಗಲಿ ಹೋಗಿ ಬಾ’ ಎಂದನು.

ದೇವೇಂದ್ರನು ಹೊರಟು ಹೋಗುತ್ತಲೆ ಕಾಮಧೇನುವು ಶ್ರೀಕೃಷ್ಣನನ್ನು ಕಂಡು ‘ದೇವ ದೇವ, ಬ್ರಹ್ಮನ ಅಪ್ಪಣೆಯಂತೆ ನಿನಗೆ ಭೂಲೋಕದ ಇಂದ್ರನ ಪದವಿಯಲ್ಲಿ ಪಟ್ಟಾಭಿ ಷೇಕ ಮಾಡುವುದಕ್ಕಾಗಿ ನಾನು ಬಂದಿದ್ದೇನೆ. ಇದಕ್ಕೆ ಅಪ್ಪಣೆಯನ್ನು ಕೊಡು’ ಎಂದು ಬೇಡಿ, ಆತನ ಅಪ್ಪಣೆಯನ್ನು ಪಡೆದ ಮೆಲೆ, ತನ್ನ ಹಾಲಿನಿಂದ ಆತನಿಗೆ ಅಭಿಷೇಕ ಮಾಡಿತು. ಅಷ್ಟರಲ್ಲಿ ದೇವೇಂದ್ರ ಮತ್ತೆ ಅಲ್ಲಿ ಕಾಣಿಸಿಕೊಂಡು, ದೇವಗಂಗೆಯಿಂದ ಆತನಿಗೆ ಅಭಿಷೇಕ ಮಾಡಿ ‘ಗೋವಿಂದ’ ಎಂದು ಆತನಿಗೆ ಹೊಸ ಹೆಸರೊಂದನ್ನು ಕೊಟ್ಟನು. ಹೀಗೆ ಆತನಿಗೆ ಪಟ್ಟಾಭಿಷೇಕವಾಗುವ ಮಂಗಳ ಮುಹೂರ್ತದಲ್ಲಿ ಸಿದ್ಧ ವಿದ್ಯಾ ಧರರು ಆತನ ಕೀರ್ತಿಯನ್ನು ಹಾಡಿದರು, ಹೂವಿನ ಮಳೆ ಸುರಿಯಿತು, ದೇವಕನ್ಯೆಯರು ನರ್ತನ ಮಾಡಿದರು, ಮೂರು ಲೋಕವೂ ಸಂತೋಷದಿಂದ ನಲಿದಾಡಿತು. ಇಂದ್ರನ ಕುಚೇಷ್ಟೆ ಶ್ರೀಕೃಷ್ಣನ ಪ್ರತಿಷ್ಠೆಗೆ ಕಾರಣವಾಯಿತು.

