
\chapter{೨೫. ಪ್ರಾಚೇತಸ}

ಪ್ರಚೇತಸರು ತಾರ್ಕ್ಷ್ಯೆಯನ್ನು ಮದುವೆಯಾಗಿ ಆಕೆಯಲ್ಲಿ ಪ್ರಾಚೇತಸನೆಂಬ ಮಗ ನನ್ನು ಪಡೆದರಷ್ಟೆ! ತನ್ನ ಯಾಗಕಾಲದಲ್ಲಿ ರುದ್ರನಿಗೆ ದ್ರೋಹಮಾಡಿ ಆಡಿನ ತಲೆಯನ್ನು ಪಡೆದ ದಕ್ಷಬ್ರಹ್ಮನೇ ಸತ್ತು ಈ ಪ್ರಾಚೇತಸನಾಗಿ ಹುಟ್ಟಿದನು. ಈತನ ಸಂತತಿಯೇ ಮೂರು ಲೋಕಗಳಲ್ಲಿಯೂ ಹಬ್ಬಿ ಹರಡಿದೆ. ಪ್ರಾರಂಭದಲ್ಲಿ ಈತನು ತನ್ನ ಮನಸ್ಸಿ ನಿಂದಲೆ ಮೂರು ಲೋಕಗಳ ದೇವ ಮನುಷ್ಯಾದಿ ಪ್ರಜೆಗಳನ್ನು ಸೃಷ್ಟಿಸುತ್ತಿದ್ದನು. ಆದರೆ ಆ ಕಾರ್ಯ ತೃಪ್ತಿಕರವಾಗಿ ನಡೆಯುತ್ತಿರಲಿಲ್ಲವಾದ್ದರಿಂದ ವಿಂಧ್ಯಪರ್ವತ್ಕೆ ಹೋಗಿ, ಶ್ರೀಹರಿ ಯನ್ನು ಕುರಿತು ತಪಸ್ಸು ಮಾಡಿದನು. ಆತನ ‘ಹಂಸಗುಹ್ಯ’ವೆಂಬ ಸ್ತೋತ್ರವನ್ನು ಕೇಳಿ ಮೆಚ್ಚಿದ ಶ್ರೀಹರಿಯು ಆತನಿಗೆ ಪ್ರತ್ಯಕ್ಷವಾದನು. ಓ! ಆತನ ಸುಂದರರೂಪವು ವರ್ಣನೆಗೆ ಸಿಕ್ಕುತ್ತದೆಯೆ! ಮೇಘದಂತೆ ಶ್ಯಾಮಲವಾದ ದೇಹಕಾಂತಿ! ವಿಶಾಲವಾದ ಆ ಕಣ್ಣುಗಳಲ್ಲಿ ಕರುಣೆಯೇ ಪ್ರತ್ಯಕ್ಷವಾಗಿದೆ! ಕೊರಳಲ್ಲಿ ತುಲಸಿಯ ಮಾಲೆ! ಅ ಎದೆಯಲ್ಲಿ ಕೌಸ್ತುಭರತ್ನ! ತಲೆಯಲ್ಲಿ ರತ್ನಕಿರೀಟ, ಕಿವಿಯಲ್ಲಿ ಕುಂಡಲಗಳು, ನಿಡಿದಾದ ಆ ತೋಳುಗಳಲ್ಲಿ ಶಂಖ ಚಕ್ರ ಗದೆ ಪದ್ಮಗಳು! ನಡುವಿನಲ್ಲಿ ಬಂಗಾರದ ಒಡ್ಯಾಣ, ಮೈಮೇಲೆ ಪೀತಾಂಬರ! ಎಲ್ಲಕ್ಕೂ ಹೆಚ್ಚಾಗಿ ಮೂರು ಲೋಕಗಳನ್ನೂ ಮೋಹಗೊಳಿಸುವ ಆ ಮುಗುಳುನಗೆ! ದಿವ್ಯ ಸುಂದರನಾದ ಆ ಶ್ರೀಹರಿ ಭಕ್ತಸಮೂಹದೊಡನೆ ಪ್ರಾಚೇತಸನಿಗೆ ಕಾಣಿಸಿಕೊಂಡನು. ಆತನ ಸಾಷ್ಟಾಂಗ ನಮಸ್ಕಾರವನ್ನು ಸಾದರದಿಂದ ಸ್ವೀಕರಿಸುತ್ತಾ ‘ಅಯ್ಯಾ ಪ್ರಾಚೇತಸ! ನಿನ್ನ ಭಕ್ತಿಗೆ ನಾನು ಮೆಚ್ಚಿದ್ದೇನೆ. ನಿನ್ನ ಅಪೇಕ್ಷೆಯೇ ನನ್ನ ಅಪೇಕ್ಷೆ. ಅಗತ್ಯವಾಗಿಯೂ ನೀನು ಪ್ರಜಾಸೃಷ್ಟಿಯನ್ನು ಮಾಡು, ‘ಪಂಚಜನ’ ಎಂಬ ಪ್ರಜಾಧಿಪತಿಗೆ ಅಸಿಕ್ನಿ ಎಂಬ ಮಗಳಿದ್ದಾಳೆ; ನೀನು ಅವಳನ್ನು ಮದುವೆಯಾಗು. ನಿನ್ನಿಂದಲೂ ನಿನ್ನ ಪೀಳಿಗೆಯಿಂದಲೂ ಜಗತ್ತು ತುಂಬಿಹೋಗುವಷ್ಟು ಪ್ರಜೆಗಳು ಹುಟ್ಟುತ್ತಾರೆ’ ಎಂದು ಹೇಳಿ ಮಾಯವಾದನು. ಪ್ರಾಚೇತಸನಿಗೆ ಕನಸಿನಲ್ಲಿ ಕಂಡ ನಿಧಿ ಕಣ್ಮರೆಯಾದಂತಾಯಿತು.

ಶ್ರೀಹರಿಯ ಅಪ್ಪಣೆಯಂತೆ ಪ್ರಾಚೇತಸನು ಅಸಿಕ್ನಿಯನ್ನು ಮದುವೆಮಾಡಿಕೊಂಡನು. ಭಗವಂತನ ಮಾತಿನಂತೆ ಅವನಿಗೆ ಹರ್ಯಶ್ವರೆಂಬ ಹತ್ತುಸಾವಿರಜನ ಗಂಡುಮಕ್ಕಳು ಹುಟ್ಟಿದರು. ಅವರು ತಂದೆಯ ಅಪ್ಪಣೆಯಂತೆ ತಮ್ಮ ಸಂತತಿಯನ್ನು ಬೆಳಸುವುದಕ್ಕಾಗಿ ತಪಸ್ಸನ್ನು ಕೈಕೊಳ್ಳಲೆಂದು ಪಶ್ಚಿಮಸಮುದ್ರತೀರಕ್ಕೆ ಹೋದರು. ಅಲ್ಲಿ ಸಿಂಧೂ ನದಿಯ ಸಂಗಮಸ್ಥಾನದಲ್ಲಿ, ಅವರೆಲ್ಲರೂ ತಪಸ್ಸಿಗಾಗಿ ಕುಳಿತರು. ಆಗ ನಾರದ ಮಹರ್ಷಿ ಅವರ ಬಳಿಗೆ ಬಂದು ‘ಅಯ್ಯೋ ಮಕ್ಕಳೆ, ಸಂತಾನವನ್ನು ಬಯಸಿ ತಪಸ್ಸನ್ನು ಮಾಡುತ್ತಾರೆಯೆ? ಇದು ಎಂತಹ ಅನ್ಯಾಯ! ಕರ್ಮಮಾರ್ಗಕ್ಕೆ ಹೋದವನು ಗುಹೆಯೊಳಗೆ ಸಿಕ್ಕಿಕೊಂಡು ಅಲ್ಲಿಯೇ ಕೊಳೆಯಬೇಕಾಗುತ್ತದೆ. ಹಾದಿಯಲ್ಲಿ ವೇಶ್ಯೆ ಬೇರೆ ಕಾದಿದ್ದಾಳೆ. ದಾರಿಯಲ್ಲಿ ದಡಗಳನ್ನು ಮೀರಿ ಹರಿಯುತ್ತಿರುವ ನದಿಯೊಂದು ಸಿಕ್ಕಿದವರನ್ನು ಕೊಚ್ಚಿಕೊಂಡು ಹೋಗುತ್ತದೆ. ಎರಡು ರೆಕ್ಕೆಗಳ ಹಂಸ ಹಾದಿಯಲ್ಲಿ ಕಣ್ಣಿಟ್ಟು ಕಾಯುತ್ತಿರುತ್ತದೆ. ಚೂಪಾದ ಅಲಗುಗಳಿಂದ ಕೂಡಿದ ಚಕ್ರವೊಂದು ತನಗೆ ತಾನೆ ತಿರುಗುತ್ತಿರುತ್ತದೆ. ಎಚ್ಚರಿಕೆ, ಎಚ್ಚರಿಕೆ!’ ಎಂದು ಹೇಳಿ ಆತ ಹೊರಟುಹೋದ. ವಿಚಾರಪರರಾದ ಹರ್ಯ ಶ್ವರು ನಾರದರ ಮಾತಿನ ಗೂಢಾರ್ಥವನ್ನು ತಿಳಿದುಕೊಂಡರು. ‘ಗುಹೆ’ ಎಂದರೆ ಹೃದಯ, ‘ವೇಶ್ಯೆ’ ಎಂದರೆ ಬುದ್ಧಿ, ‘ನದಿ’ ಎಂದರೆ ಮಾಯೆ, ‘ಪ್ರವಾಹ’ವೆಂದರೆ ಹುಟ್ಟುಸಾವು, ‘ದಡ’ಗಳೆಂದರೆ ಇಹಪರಗಳು, ‘ಎರಡು ರೆಕ್ಕೆಯ ಹಂಸ’ವೆಂದರೆ ಸಂಸಾರದ ಬಂಧನಕ್ಕೂ ಮೋಕ್ಷಕ್ಕೂ ಕಾರಣವಾದ ಪ್ರವೃತ್ತಿ ನಿವೃತ್ತಿ ಮಾರ್ಗಗಳನ್ನು ತೋರಿಸತಕ್ಕ ಪರಮೇಶ್ವರ, ‘ಚಕ್ರ’ವೆಂದರೆ ಕಾಲಚಕ್ರ–ಹೀಗೆಂದುಕೊಂಡು, ಅವರು ಫಲಾಪೇಕ್ಷೆಯಿಲ್ಲದ ತಪಸ್ಸನ್ನಾ ಚರಿಸಿ, ಮುಕ್ತಿಯನ್ನು ಪಡೆದರು.

ನಾರದನ ಬುದ್ಧಿವಾದಕ್ಕೆ ಮನಸೋತು, ತನ್ನ ಮಕ್ಕಳೆಲ್ಲ ಮೋಕ್ಷಕ್ಕೆ ಹೋದ ಸುದ್ದಿ ಯನ್ನು ಕೇಳಿ, ಪ್ರಾಚೇತಸನಿಗೆ ಅಪಾರವಾದ ಸಂಕಟವಾಯಿತು. ‘ಒಳ್ಳೆಯ ಮಕ್ಕಳು ಹುಟ್ಟು ವುದು ದುಃಖಕ್ಕೆ ಕಾರಣ’ ಎಂದು ಆತ ರೋದಿಸುತ್ತಿರಲು, ಬ್ರಹ್ಮನು ಆತನಲ್ಲಿಗೆ ಬಂದು ಆತನನ್ನು ಸಮಾಧಾನಮಾಡಿದನು. ಪ್ರಾಚೇತಸನು ಅಸಿಕ್ನಿಯಲ್ಲಿ ಮತ್ತೆ ಒಂದು ಸಾವಿರ ಮಕ್ಕಳನ್ನು ಪಡೆದು, ಅವರನ್ನೂ ವಂಶಾಭಿವೃದ್ಧಿಗಾಗಿ ತಪಸ್ಸು ಮಾಡುವಂತೆ ಹೇಳಿ ಕಳುಹಿ ಸಿದನು. ಅವರೂ ಅಣ್ಣಂದಿರಂತೆಯೆ ಸಿಂಧೂ ನದಿಯ ಸಂಗಮದಲ್ಲಿ ತಪಸ್ಸಿಗೆಂದು ಕುಳಿತಿರಲು, ನಾರದನು ಮತ್ತೊಮ್ಮೆ ಅಲ್ಲಿ ಕಾಣಿಸಿಕೊಂಡು, ಅವರ ಅಣ್ಣಂದಿರಿಗೆ ಹೇಳಿದ ಒಗಟನ್ನೆ ಹೇಳಿ, ‘ಅಯ್ಯಾ, ನಿಮಗೆ ಅಣ್ಣಂದಿರಲ್ಲಿ ಪ್ರೀತಿಯಿದ್ದರೆ ಅವರು ಹೋದ ಹಾದಿ ಯನ್ನೇ ಅನುಸರಿಸಿರಿ’ ಎಂದನು. ಅವರೂ ಅಣ್ಣಂದಿರಂತೆ ತಪಸ್ಸನ್ನು ಕೈಕೊಂಡು ಮುಕ್ತಿಮಾರ್ಗವನ್ನೆ ಹಿಡಿದರು. ಇದನ್ನು ಕೇಳಿ ಪ್ರಾಚೇತಸನ ಪಿತೃಹೃದಯ ಕುತಕುತ ಕುದಿಯಿತು. ಆತನು ಆ ಸಂಕಟದಲ್ಲಿರುವಾಗಲೆ ನಾರದರ ಸವಾರಿ ಅಲ್ಲಿಗೆ ಬಂದಿತು. ಆತನನ್ನು ಕಾಣುತ್ತಲೆ ಪ್ರಾಚೇತಸನು ಕೆರಳಿ ‘ಅಯ್ಯಾ ನೀನು ಕೇವಲ ವೇಷಧಾರಿ; ಶುದ್ಧ ಮೋಸಗಾರ. ನನ್ನ ಮಕ್ಕಳಿಗೆ ಇಲ್ಲದ ಉಪದೇಶಮಾಡಿ ಅವರನ್ನು ಹಾಳುಮಾಡಿದೆ ಯಲ್ಲಾ! ಅವರು ಶಾಸ್ತ್ರಗಳನ್ನು ಓದಿ ಪುಷಿ ಪುಣವನ್ನು ತೀರಿಸಲಿಲ್ಲ, ಮಕ್ಕಳನ್ನು ಪಡೆದು ಪಿತೃಪುಣವನ್ನು ತೀರಿಸಲಿಲ್ಲ. ಅವರಿಗೆ ಇಹಪರಗಳೆರಡೂ ಇಲ್ಲದಂತೆ ಮಾಡಿದೆಯಲ್ಲ! ನೀನು ಮಹಾಕ್ರೂರಿ. ನೀನು ನಿಂತಕಡೆ ನಿಲ್ಲದೆ ಸದಾ ಸುತ್ತುತ್ತಿರು. ಇದು ನನ್ನ ಶಾಪ’ ಎಂದು ಕೂಗಿ ರೇಗಿದನು. ಅವನು ಎಷ್ಟು ರೇಗಿದರೇನು? ನಾರದನು ಪ್ರಶಾಂತನಾಗಿದ್ದನು. ತನಗೆ ಪ್ರತಿಶಾಪ ಕೊಡುವ ಶಕ್ತಿಯಿದ್ದರೂ ಆತ ಹಾಗೆ ಮಾಡಲಿಲ್ಲ. ಸಾಧುವಾದ ಆತನು ಸಾಧುವಿನಂತೆ ನಡೆದುಕೊಂಡನು.

ಪ್ರಾಚೇತಸನ ಪುತ್ರದುಃಖವನ್ನು ಮತ್ತೊಮ್ಮೆ ಬ್ರಹ್ಮನೇ ಬಂದು ಹೋಗಲಾಡಿಸ ಬೇಕಾಯಿತು. ಆತನು ಮತ್ತೊಮ್ಮೆ ಪ್ರಜಾಸೃಷ್ಟಿಗೆ ಪ್ರಯತ್ನಿಸಿ, ಅರವತ್ತು ಮಂದಿ ಹೆಣ್ಣು ಮಕ್ಕಳನ್ನು ಪಡೆದನು. ಅವರಲ್ಲಿ ಹತ್ತು ಮಂದಿಯನ್ನು ಯಮಧರ್ಮನಿಗೂ, ಹದಿ ಮೂರು ಮಂದಿಯನ್ನು ಕಶ್ಯಪನಿಗೂ, ಇಪ್ಪತ್ತೇಳು ಮಂದಿಯನ್ನು ಚಂದ್ರನಿಗೂ, ಇಬ್ಬಿಬ್ಬರನ್ನು ಕ್ರಮವಾಗಿ ರುದ್ರ, ಅಂಗಿರಸ್ಸು, ಕೃಶಾಶ್ವರಿಗೂ ಕೊಟ್ಟು ವಿವಾಹ ಮಾಡಿ ದನು. ಉಳಿದ ನಾಲ್ವರು ಕನ್ಯೆಯರನ್ನು ತಾರ್ಕ್ಷ್ಯನೆಂಬ ಪುಷಿಯು ವಿಹಾಹವಾದನು. ಇವರಲ್ಲಿ ಚಂದ್ರನ ಪತ್ನಿಯರಾದ ಕೃತ್ತಿಕೆಯೇ ಮೊದಲಾದ ಇಪ್ಪತ್ತೇಳು ಜನ ಮಾತ್ರ ಪುತ್ರವತಿಯರಾಗಲಿಲ್ಲ. ಎಲ್ಲ ಮಡದಿಯರನ್ನೂ ಸಮಾನಪ್ರೀತಿಯಿಂದ ಕಾಣಬೇಕೆಂದು ಪ್ರಾಚೇತಸನು ತಿಳಿಸಿದ್ದರೂ ಚಂದ್ರನು ರೋಹಿಣಿಯಲ್ಲಿಯೇ ಹೆಚ್ಚು ಪ್ರೇಮವುಳ್ಳನಾಗಿ, ಉಳಿದವರನ್ನು ಉಪೇಕ್ಷೆ ಮಾಡಿದುದರಿಂದ ಆತನು ಕೋಪಗೊಂಡು ಅವನನ್ನು ಶಪಿಸಿ ದನು. ಆ ಶಾಪದಿಂದ ಚಂದ್ರನು ಕ್ಷಯರೋಗಿಯಾಗಿ ಅಪುತ್ರವಂತನಾದನು. ಮಾವ ನನ್ನು ಶರಣುಹೋಗಿ, ಆತನ ಅನುಗ್ರಹದಿಂದ ಚಂದ್ರನು ತನ್ನ ಕಳೆಗಳು ಪುನಃ ಬೆಳೆಯು ವಂತೆ ವರವನ್ನು ಪಡೆದನಾದರೂ ಮಕ್ಕಳನ್ನು ಮಾತ್ರ ಪಡೆಯಲಿಲ್ಲ. ಉಳಿದ ಹೆಣ್ಣು ಮಕ್ಕಳು ಅನೇಕ ಪುತ್ರವತಿಯರಾಗಿ ಜಗತ್ತನ್ನೆಲ್ಲ ತಮ್ಮ ಪೀಳಿಗೆಯಿಂದ ಸಮೃದ್ಧವಾಗು ವಂತೆ ಮಾಡಿದರು.

