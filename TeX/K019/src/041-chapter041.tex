
\chapter{೪೧. ಶ್ರೀ ಕೃಷ್ಣಾವತರಣ}

ಯಯಾತಿಯ ಹಿರಿಯ ಮಗ ಯದು. ಆತನಿಂದಲೆ ಯಾದವವಂಶ ಪ್ರಾರಂಭ ವಾದುದು. ಆತನ ನಾಲ್ವರು ಮಕ್ಕಳಲ್ಲಿ ಹಿರಿಯನಾದವನು ಸಹಸ್ರಜಿತ್ತು. ಈತನ ಪೀಳಿಗೆ ಯಲ್ಲಿ ಎಣೆಯಿಲ್ಲದ ಮಹಾಪುರುಷರು ಎಷ್ಟೊ ಜನ ಹುಟ್ಟಿ ಹೆಸರಾದರು. ಕಾರ್ತವೀರ್ಯಾರ್ ಜುನನ ಕಥೆಯನ್ನು ಈ ಹಿಂದೆಯೆ ಹೇಳಿದೆಯಷ್ಟೆ. ಸಪ್ತದ್ವೀಪಗಳಿಗೆ ಚಕ್ರವರ್ತಿ ಯಾಗಿದ್ದ ಆ ಮಹಾಪುರುಷನು ಈ ಪೀಳಿಗೆಗೆ ಸೇರಿದವನೆ. ಆತನ ಮರಿಮಗನಾದ ಶೂರಸೇನನೆಂಬುವನು ಮಾಥುರ, ಶೂರಸೇನ ಎಂಬ ಎರಡು ರಾಜ್ಯಗಳ ಒಡೆಯನಾಗಿ, ಮಧುರಾಪುರಿಯನ್ನು ತನ್ನ ರಾಜಧಾನಿಯಾಗಿ ಮಾಡಿಕೊಂಡನು. ಅಂದಿನಿಂದ ಆ ಪಟ್ಟಣವೇ ಯಾದವರ ರಾಜಧಾನಿಯಾಯಿತು. ಈ ಶೂರಸೇನನ ವಂಶಪರಂಪರೆಯಲ್ಲಿ ಹುಟ್ಟಿದ ಶೂರನೆಂಬುವನು ಮಾರಿಷಾ ಎಂಬ ಮಡದಿಯಲ್ಲಿ ವಸುದೇವನೆ ಮೊದಲಾದ ಹಲವು ಮಕ್ಕಳನ್ನು ಪಡೆದನು, ಅವರೆಲ್ಲ ಮಧುರೆಯಲ್ಲಿಯೇ ನೆಲಸಿದ್ದರು. ಆದರೆ ರಾಜ್ಯ ಭಾರ ಮಾತ್ರ ಅವರ ಕೈ ತಪ್ಪಿ, ಯದುವಿನ ಎರಡನೆಯ ಮಗನಾದ ಕ್ರೋಷ್ಟುವಿನ ವಂಶ ದವರ ಪಾಲಾಗಿತ್ತು. ಆ ವಂಶ ಪರಂಪರೆಯಲ್ಲಿ ಬಂದ ಅಹುಕನೆಂಬುವನಿಗೆ ದೇವುಕ, ಉಗ್ರಸೇನ ಎಂಬ ಇಬ್ಬರು ಮಕ್ಕಳಿದ್ದರು. ಆ ಮಕ್ಕಳಲ್ಲಿ ಹಿರಿಯನಾದ ಉಗ್ರಸೇನನೇ ಮಧುರಾಪುರಿಯ ಅರಸನಾಗಿದ್ದ. ದೇವುಕನಿಗೆ ದೇವಲನೆ ಮೊದಲಾದ ನಾಲ್ವರು ಗಂಡು ಮಕ್ಕಳೂ ದೇವಕಿಯೇ ಮೊದಲಾದ ಏಳು ಜನ ಹೆಣ್ಣುಮಕ್ಕಳೂ ಹುಟ್ಟಿದರು. ಉಗ್ರ ಸೇನನಿಗೆ ಕಂಸನೇ ಮೊದಲಾದ ಒಂಬತ್ತು ಜನ ಗಂಡುಮಕ್ಕಳೂ, ಕಂಸೆಯೇ ಮೊದಲಾದ ಐವರು ಹೆಣ್ಣುಮಕ್ಕಳೂ ಇದ್ದರು. ಇವರೆಲ್ಲರೂ ಯಾವ ಭೇದಭಾವನೆಯೂ ಇಲ್ಲದೆ ಒಂದೇ ಕುಟುಂಬದಲ್ಲಿ ಪರಸ್ಪರ ಪ್ರೀತಿ ವಿಶ್ವಾಸಗಳಿಂದ ಜೀವಿಸುತ್ತಿದ್ದರು.

ಹೀಗಿರುತ್ತಿರಲು ಒಮ್ಮೆ ದೇವುಕನ ಹಿರಿಯಮಗಳಾದ ದೇವಕಿಯನ್ನು ಶೂರನ ಮಗ ನಾದ ವಸುದೇವನಿಗೆ ಕೊಟ್ಟು ಅತ್ಯಂತ ವೈಭವದಿಂದ ವಿವಾಹ ನಡೆಯಿತು. ಉಗ್ರಸೇನ ಮಹಾರಾಜನಿಗೆ ದೇವಕಿಯಲ್ಲಿ ಹೊಟ್ಟೆಯ ಮಗಳಿಗಿಂತಲೂ ಹೆಚ್ಚು ಪ್ರೀತಿ. ಆದ್ದರಿಂದ ಆತನು ನಾನೂರು ಆನೆ, ಹತ್ತುಸಾವಿರ ಕುದುರೆ, ಸಾವಿರದೆಂಟುನೂರು ರಥ, ಇನ್ನೂರು ದಾಸಿಯರನ್ನು ಬಂಗಾರದಿಂದ ಅಲಂಕರಿಸಿ, ಆ ಮಗಳಿಗೆ ಬಳುವಳಿಯಾಗಿ ಕೊಟ್ಟನು. ವಸುದೇವನು ಮಡದಿಯೊಡನೆ ತನ್ನ ಅರಮನೆಗೆ ಹೊರಟಾಗ ಈ ಬಳುವಳಿಯೂ ಆತನನ್ನು ಹಿಂಬಾಲಿಸಿತು. ಶಂಖ, ಭೇರಿ, ಮೃದಂಗಗಳ ಧ್ವನಿ ದಿಕ್ಕುದಿಕ್ಕುಗಳಲ್ಲಿಯೂ ಮಾರ್ಮೊಳಗುತ್ತಿರಲು ನೂತನ ದಂಪತಿಗಳು ರಥವೇರಿದರು. ರಾಜಕುಮಾರನಾದ ಕಂಸನೇ ಸ್ವತಃ ಅದರ ಸಾರಥಿಯಾದನು. ಆ ದಿಬ್ಬಣದ ವೈಭವವನ್ನು ವರ್ಣಿಸುವುದಕ್ಕೆ ಸಾಧ್ಯವೆ? ನೂರಾರು ರಥಗಳು ಆ ದಂಪತಿಗಳ ರಥದ ಹಿಂದೆ ಮುಂದೆ ಚಲಿಸಿದವು. ಸೇವಕ ಜನರ ಸಡಗರಕ್ಕಂತೂ ಕೊನೆಮೊದಲಿಲ್ಲ. ದಿವ್ಯವಾಗಿ ಅಲಂಕರಿಸಿದ ರಥದಲ್ಲಿ ರತಿ ಮನ್ಮಥರಂತೆ ಕುಳಿತಿದ್ದ ದಂಪತಿಗಳನ್ನು ಕಂಡು ಪುರಜನರ ಕಣ್ಣು ಅರಳಿದವು. ಅವರು ಪುಲಕಗೊಂಡರು. ರಾಜಕುಮಾರನಾದ ಕಂಸನಿಗಂತೂ ಹಿಗ್ಗೋ ಹಿಗ್ಗು. ಹೆಜ್ಜೆಹೆಜ್ಜೆಗೂ ಹಿಂದಿರುಗಿ ನೋಡಿ, ತಂಗಿ ಭಾವನನ್ನು ಉಪಚರಿಸುವನು, ಹಾಸ್ಯದ ಚಟಾಕಿಯಿಂದ ಅವರನ್ನು ನಗಿಸುವನು, ಕಣ್ತುಂಬ ಅವರನ್ನು ನೋಡಿ ಆನಂದಿಸುವನು. ಆ ಈಡು ಜೋಡಿಯು ಮಂಡಿಸಿರುವ ರಥವನ್ನು ತಾನು ನಡೆಸುತ್ತಿರುವೆನಲ್ಲಾ ಎಂಬ ಹೆಮ್ಮೆ, ಆತನಿಗೆ.

ನೂತನ ವಧೂವರರ ಮೆರವಣಿಗೆ ಮಂದಗತಿಯಿಂದ ಮುಂದೆ ಮುಂದೆ ಸಾಗುತ್ತಿತ್ತು. ಸ್ವಲ್ಪ ದೂರ ಸಾಗುವಷ್ಟರಲ್ಲಿ ಇದ್ದಕ್ಕಿದ್ದಂತೆಯೆ, ಕಂಸನ ಉತ್ಸಾಹವೆಲ್ಲ ಒಮ್ಮೆಗೆ ಕಮರಿ ಹೋಗುವಂತೆ, ಆಕಾಶವಾಣಿ ಕೇಳಿಬಂತು. ‘ಹೇ ಕಂಸ, ನೀನು ಇಷ್ಟು ಉತ್ಸವ ಉತ್ಸಾಹ ಗಳಿಂದ ಗಂಡನ ಮನೆಗೆ ಕರೆದೊಯ್ಯುತ್ತಿರುವ ಈ ದೇವಕಿಯ ಎಂಟನೆಯ ಮಗುವೇ ನಿನ್ನ ಮೃತ್ಯು.’ ಈ ಮಾತುಗಳು ಕಿವಿಗೆ ಬೀಳುತ್ತಲೆ ಕಂಸನ ಪ್ರೇಮ ಬತ್ತಿತು; ಕೋಪ ಹೊತ್ತಿತು. ಆತನು ಕೈಲಿ ಹಿಡಿದಿದ್ದ ಹಗ್ಗವನ್ನು ಅತ್ತ ಒಗೆದು, ಒರೆಯಲ್ಲಿದ್ದ ಕತ್ತಿಯನ್ನು ಹೊರಕ್ಕೆ ಕಿತ್ತನು. ಅಲ್ಲಿಯವರೆಗೆ ಮಮತೆಯನ್ನು ತುಳುಕಿಸುತ್ತಿದ್ದ ಕಣ್ಣುಗಳಲ್ಲಿ ಕೆಂಡ ವನ್ನು ಕಾರುತ್ತಾ, ದೇವಕಿಯ ತುರುಬನ್ನು ಹಿಡಿದು, ಅವಳ ಕತ್ತನ್ನು ಕತ್ತರಿಸಲೆಂದು ಕತ್ತಿ ಯೆತ್ತಿದನು. ಕ್ಷಣಾರ್ಧದಲ್ಲಿ ನಡೆದುಹೋದ ಈ ಪರಿವರ್ತನೆಯನ್ನು ಕಂಡು ಕಂಗಾಲಾಗಿ ಹೋದ ವಸುದೇವನು ಕಂಸನ ಕೈಯನ್ನು ಹಿಡಿದು ‘ಕಂಸಕುಮಾರ ಇದೇನನ್ಯಾಯ! ನೀನು ಮಹಾ ಶೂರನೆಂದು ಲೋಕವೆಲ್ಲ ಹೊಗಳುತ್ತಿದೆ, ಇದೇ ಏನು ನಿನ್ನ ಶೌರ್ಯ? ನೀನೇನು, ನಿನ್ನ ವಂಶವೇನು, ನೀನು ಹೆಣ್ಣನ್ನು–ಅದರಲ್ಲಿಯೂ ತಂಗಿಯನ್ನು–ಕೊಲ್ಲುವುದೆಂದ ರೇನು? ಈಗತಾನೆ ಮದುವೆಯಾಗಿ ಹಸಿಯ ಮೈಯ್ಯಲ್ಲಿರುವ ಈ ಮದುವಣಗಿತ್ತಿಯನ್ನು ಕೊಂದುಹಾಕುವುದೆಂದರೆ ಎಂತಹ ಪಾಪಕಾರ್ಯ! ಸ್ವಲ್ಪ ಸಮಾಧಾನ ಮಾಡಿಕೊ. ಸಾವು ಯಾರಿಗೆ ತಪ್ಪಿದುದು? ಒಂದಲ್ಲ ಒಂದು ದಿನ ಎಲ್ಲರೂ ಸಾಯಲೇಬೇಕು. ಜಿಗಣೆ ಮುಂದಣ ಹುಲ್ಲುಕಡ್ಡಿಯನ್ನು ಬಿಗಿಯಾಗಿ ಹಿಡಿದುಕೊಂಡು, ಹಿಂದಿನ ಹುಲ್ಲುಕಡ್ಡಿ ಯನ್ನು ಬಿಡುವ ಹಾಗೆ ಈ ಜೀವ ಮುಂದಿನ ಜನ್ಮವನ್ನು ಹಿಡಿದೇ ಈ ಜನ್ಮವನ್ನು ಬಿಡು ತ್ತಾನೆ. ಆದ್ದರಿಂದ ಸಾವಿಗೆ ಹೆದರಿಯಲ್ಲ, ನಾನು ಈ ಮಾತಾಡುತ್ತಿರುವುದು. ಕೇವಲ ಮರಣಭಯದಿಂದ ಮಾಡಬಾರದ ಪಾಪಕಾರ್ಯವನ್ನು ನೀನು ಮಾಡದಿರಲೆಂಬ ಕಾರಣ ದಿಂದ ನಾನು ಹೇಳುತ್ತಿದ್ದೇನೆ. ಇವಳು ಕೇವಲ ಬಾಲೆ, ನಿನಗೆ ತಂಗಿ, ನಿನ್ನ ಮಗಳಂತೆ ಇರುವವಳು. ಇವಳನ್ನು ಕೊಲ್ಲುವುದು ತರವಲ್ಲ. ಸ್ವಲ್ಪ ಯೋಚಿಸಿ ನೋಡು, ಇವಳೇನೂ ನಿನ್ನನ್ನು ಕೊಲ್ಲುವವಳಲ್ಲ; ಇವಳ ಮಕ್ಕಳು ತಾನೆ ನಿನ್ನನ್ನು ಕೊಲ್ಲುವವರು? ಆ ಮಕ್ಕಳ ನ್ನೆಲ್ಲ ನಿನ್ನ ಕೈಗೆ ಕೊಡುತ್ತೇವೆ. ಅಷ್ಟಾದರೆ ಸಾಕು ತಾನೆ?’ ಎಂದು ಸಮಾಧಾನ ಮಾಡಿದನು. ಈ ಮಾತಿಗೆ ಒಪ್ಪಿ, ಕಂಸನು ಹಾಗೆಯೆ ಮನೆಗೆ ಹಿಂದಿರುಗಿದನು. ವಸುದೇವನು ತನ್ನ ಮಡದಿಯೊಡನೆ ತನ್ನ ಅರಮನೆಯನ್ನು ಸೇರಿದನು.

ಕಾಲಚಕ್ರ ಉರುಳಿತು. ದೇವಕಿದೇವಿ ಗರ್ಭಿಣಿಯಾಗಿ ಒಂಬತ್ತು ತಿಂಗಳು ತುಂಬುತ್ತಲೆ ಸುಂದರನಾದ ಒಬ್ಬ ಸುಕುಮಾರನನ್ನು ಹೆತ್ತಳು. ದುಷ್ಟರಿಗೆ ಕೆಟ್ಟುದಿಲ್ಲ, ಜ್ಞಾನಿಗಳಿಗೆ ಮೋಹವಿಲ್ಲ. ಅದರಂತೆ ಸತ್ಯಸಂಧರಿಗೆ ಸತ್ಯಕ್ಕಾಗಿ ಮಾಡಲಾರದ ತ್ಯಾಗವಿಲ್ಲ. ತಾನು ಕೊಟ್ಟ ಮಾತನ್ನು ಉಳಿಸಿಕೊಳ್ಳುವುದಕ್ಕಾಗಿ, ವಸುದೇವನು ತನ್ನ ಕಣ್ಮಣಿಯನ್ನು ತಾನೆ ಎತ್ತಿ ತಂದು ಮೃತ್ಯುವಿನಂತಿದ್ದ ಕಂಸನ ಕೈಗಿತ್ತನು. ಆತನ ಸತ್ಯವನ್ನು ಕಂಡು ಮೆಚ್ಚಿದ ಕಂಸ ರಾಜನು ಮಂದಹಾಸದೊಡನೆ ‘ಭಾವಾ, ಈ ಮಗುವಿನಿಂದ ನನಗೇನು ಭಯ? ಮುದ್ದಾ ಗಿರುವ ಇದನ್ನು ನಾನು ಕೊಲ್ಲಲಾರೆ. ಇದನ್ನು ಮನೆಗೆ ಕೊಂಡೊಯ್ಯಿ. ಆದರೆ ನನ್ನ ಮೃತ್ಯು ವೆನಿಸಿಕೊಂಡ ಎಂಟನೆಯ ಮಗುವನ್ನು ಮಾತ್ರ ನನಗೆ ತಂದುಕೊಡಬೇಕೆಂಬುದನ್ನು ಮರೆಯಬೇಡ’ ಎಂದು ಹೇಳಿದನು. ಇದನ್ನು ಕೇಳಿ ವಸುದೇವನಿಗೆ ಹಿಡಿಸಲಾರದಷ್ಟು ಸಂತೋಷವಾಯಿತು. ಆತನು ತನ್ನ ಮನೆಗೆ ಹಿಂದಿರುಗಿ, ಮಗುವನ್ನು ಅದರ ತಾಯಿಗೆ ಒಪ್ಪಿಸಿದನು. ದೇವಕಿಗೆ ತನ್ನ ಕಣ್ಣನ್ನು ತಾನೆ ನಂಬದಷ್ಟು ಆಶ್ಚರ್ಯವಾಯಿತು. ತನ್ನ ಅಣ್ಣ ನಾದ ಕಂಸನು ಎಂತಹ ಶೂರನೋ ಅಂತಹ ಕ್ರೂರಿ. ಆತನು ಬಂಧುಗಳಲ್ಲಿ ಬಹು ಪ್ರೇಮವುಳ್ಳವನಾದರೂ, ದ್ವೇಷ ಹುಟ್ಟಿದರೆ ಕ್ರೋಧದಿಂದ ಕುರುಡನಾಗುವನು. ಕಂಸನ ಈ ಸ್ವಭಾವವನ್ನರಿತ ದೇವಕಿ ವಸುದೇವರು ಅವನ ಮನಸ್ಸು ಯಾವ ಕಾಲಕ್ಕೆ ಹೇಗೆ ಬದಲಾ ಯಿಸುವದೋ ಎಂಬ ಭಯದಿಂದ ಒಳಗೊಳಗೆ ಕಳವಳಗೊಳ್ಳುತ್ತಿದ್ದರು. ಮಗುವಿಗೆ ಕೀರ್ತಿವಂತನೆಂದು ನಾಮಕರಣವಾಯಿತು.

ದೇವಕಿದೇವಿ ಮಗನನ್ನು ಹೆತ್ತ ಕೆಲದಿನಗಳ ಮೇಲೆ ತ್ರಿಲೋಕಸಂಚಾರಿಯಾದ ನಾರದ ಮಹರ್ಷಿ ಮಧುರಾಪುರಿಗೆ ಬಂದು ಅರಮನೆಯಲ್ಲಿ ನಿಂತನು. ಆತನು ಕಂಸನನ್ನು ಗುಟ್ಟಾಗಿ ಕರೆದು ‘ಅಯ್ಯಾ ರಾಜಕುಮಾರ, ನಿನ್ನ ಮೇಲಿನ ಪ್ರೀತಿಯಿಂದ ಈ ಮಾತನ್ನು ನಿನಗೆ ಹೇಳುತ್ತಿದ್ದೇನೆ. ಗೋಕುಲದಲ್ಲಿರುವ ನಂದನೇ ಮೊದಲಾದ ಯಾದವರೂ ಅವರ ಹೆಂಡಿರು ಮಕ್ಕಳೂ ದೇವತೆಗಳ ಅಂಶದಿಂದ ಹುಟ್ಟಿದವರು. ನಿನ್ನ ತಂಗಿಯಾದ ದೇವಕಿ ಮತ್ತು ಭಾವನಾದ ವಸುದೇವ–ಇವರೂ ಅದೇ ಗುಂಪಿಗೆ ಸೇರಿದವರು. ನೀನೂ ನಿನ್ನ ಅನುಯಾಯಿಗಳೂ ದೈತ್ಯರ ಅಂಶದಿಂದ ಹುಟ್ಟಿದವರು. ಆದ್ದರಿಂದ ಆ ದೇವತೆಗಳ ಹಿತ ಕ್ಕಾಗಿ ದಾನವರಾದ ನಿಮ್ಮೆಲ್ಲರನ್ನೂ ಕೊಲ್ಲುವ ಪ್ರಯತ್ನ ನಡೆಯುತ್ತಾಇದೆ. ಸಾಕ್ಷಾತ್ ಮಹಾವಿಷ್ಣುವೆ ದೇವಕಿಯ ಎಂಟನೆಯ ಮಗನಾಗಿ ಹುಟ್ಟಿ ನಿನ್ನನ್ನೂ ನಿನ್ನ ಕಡೆಯವರನ್ನೂ ಕೊಲ್ಲುವವನಾಗಿದ್ದಾನೆ. ಈ ಯಾದವರೆಲ್ಲರೂ ಅವನ ಸಹಾಯಕರಾಗಿದ್ದಾರೆ, ಇವರ ವಿಚಾರದಲ್ಲಿ ನೀನು ಸದಾ ಎಚ್ಚರದಿಂದ ಇರುವುದು ಒಳ್ಳೆಯದು’ ಎಂದು ಅವನ ಕಿವಿ ಯಲ್ಲಿ ಭಯವನ್ನು ಬಿತ್ತಿದನು. ಈ ಮಾತನ್ನು ಕೇಳುತ್ತಲೆ ಕಂಸನ ಶಾಂತಿ ಕದಡಿ ಹೋಯಿತು. ಆತನು ಆ ಕ್ಷಣವೆ ದೇವಕಿ ವಸುದೇವರನ್ನು ಹಿಡಿದು, ಅವರ ಕೈಕಾಲುಗಳಿಗೆ ಸಂಕಲೆಯನ್ನು ತೊಡಿಸಿ, ಸೆರೆಮನೆಗೆ ನೂಕಿದನು. ಅವರ ಮಗು, ಕೀರ್ತಿವಂತ, ಕಂಸನ ಕೋಪಕ್ಕೆ ಆಹುತಿಯಾದನು. ಇಷ್ಟೆ ಅಲ್ಲ, ಅಲ್ಲಿಂದ ಮುಂದೆ ದೇವಕಿಯ ಗರ್ಭದಲ್ಲಿ ಹುಟ್ಟಿದ ಒಂದೊಂದು ಮಗುವನ್ನೂ ಆ ಕ್ಷಣದಲ್ಲಿಯೆ ಕೊಲ್ಲುತ್ತಾ ಬಂದನು. ಹೀಗೆ ಆಕೆ ಹೆತ್ತ ಆರು ಮಕ್ಕಳು ಕಂಸನ ಕೋಪಕ್ಕೆ ತುತ್ತಾಗಿ ಸತ್ತುಹೋದವು. ಯಾದವರ ಮೇಲೆ ಆತನ ವೈರ ದಿನದಿನಕ್ಕೆ ಹೆಮ್ಮರವಾಗಿ ಬೆಳೆಯತೊಡಗಿತು. ಯಾದವರಿಗೆ ರಾಜನಾಗಿದ್ದ ತನ್ನ ತಂದೆಯಲ್ಲಿ ಕೂಡ ಆತನಿಗೆ ನಂಬಿಕೆ ತಪ್ಪಿಹೋಯಿತು. ಆತನು ರಾಜ್ಯಾಧಿಕಾರವನ್ನು ತಂದೆಯಿಂದ ಕಿತ್ತುಕೊಂಡು, ತಾನೆ ಸಾಮ್ರಾಜ್ಯದ ಸರ್ವಾಧಿಕಾರಿಯಾದನು. 

ಹೀಗೆ ನಿರಂಕುಶಪ್ರಭುವಾಗಿದ್ದ ಕಂಸನ ಸುತ್ತ ಪ್ರಲಂಬ, ಬಕ, ಚಾಣೂರ, ತೃಣಾ ವರ್ತ, ಅಘ, ಮುಷ್ಟಿಕ, ಅರಿಷ್ಟ, ದ್ವಿವಿಧ, ಕೇಶಿ, ಧೇನುಕ, ಬಾಣ, ಜರಾಸಂಧ ಮೊದ ಲಾದ ಕಿಡಿಗೇಡಿಗಳು ಸೇರಿಕೊಂಡರು. ಮಹಾಶೂರರಾದ ಇವರೆಲ್ಲರೂ ಯಾವುದಾದ ರೊಂದು ನೆಪದಿಂದ ಯಾದವರೊಡನೆ ಕಾಲ್ಕೆರೆದು ಜಗಳವಾಡುತ್ತಾ, ಅವರನ್ನು ಹಿಂಸಿಸು ತ್ತಿದ್ದರು. ಇವರ ಕಾಟವನ್ನು ತಡೆಯಲಾರದೆ ಯಾದವರಲ್ಲಿ ಹಲವರು ತಲೆ ಮರೆಸಿ ಕೊಂಡು ದೇಶಾಂತರಗಳಿಗೆ ಓಡಿಹೋದರು. ಉಳಿದವರು ಬೇರೆಯ ಮಾರ್ಗವಿಲ್ಲದೆ ಕಂಸ ನನ್ನೆ ಅನುಸರಿಸಿಕೊಂಡು, ಮಧುರೆಯಲ್ಲಿಯೆ ಕಾಲವನ್ನು ತಳ್ಳುತ್ತಿದ್ದರು. ಹೀಗಿರಲು ದೇವಕಿಗೆ ಏಳನೆಯ ಗರ್ಭ ಮೂಡಿ ಬೆಳೆಯಲಾರಂಭವಾಯಿತು. ಅದನ್ನು ಹೊತ್ತ ತಾಯಿಗೆ ಮಗು ಹುಟ್ಟುವುದೆಂಬ ಸಂತೋಷಕ್ಕಿಂತಲೂ ಕಂಸನಿಂದ ಅದು ಸಾಯುವುದಲ್ಲಾ ಎಂಬ ಸಂಕಟ ಹೆಚ್ಚಾಯಿತು. ತನ್ನ ತಾಯಾಗಲಿರುವವಳ ಈ ಸಂಕಟವನ್ನು ನೋಡಲಾರದೆ ಭಗ ವಂತನು ತನ್ನ ಯೋಗಮಾಯೆಯನ್ನು ಕರೆದು ‘ಮಾಯಾ, ದೇವಕಿಯ ಗರ್ಭದಲ್ಲಿ ಆದಿ ಶೇಷನು ಮಗುವಾಗಿ ಬೆಳೆಯುತ್ತಿದ್ದಾನೆ. ನೀನು ಆ ಮಗುವನ್ನು ತೆಗೆದುಕೊಂಡು ನಂದಗೋಕುಲಕ್ಕೆ ಹೋಗು. ಅಲ್ಲಿ ವಸುದೇವನ ಹಿರಿಯ ಹೆಂಡತಿಯಾದ ರೋಹಿಣಿಯು ಕಂಸನ ಭಯದಿಂದ ತಲೆ ಮರೆಸಿಕೊಂಡಿದ್ದಾಳೆ. ಅವಳ ಗರ್ಭದಲ್ಲಿ ಆ ಮಗುವನ್ನಿಡು. ನೀನು ಹಾಗೆ ಮಾಡಿದಮೇಲೆ ನಾನು ದೇವಕಿಯ ಗರ್ಭವನ್ನು ಪ್ರವೇಶಿಸಿ, ಆಕೆಯ ಮಗನಾಗಿ ಹುಟ್ಟುತ್ತೇನೆ. ಅಷ್ಟರಲ್ಲಿ ನೀನು ನಂದನ ಮಡದಿಯಾದ ಯಶೋದೆಯ ಗರ್ಭವನ್ನು ಪ್ರವೇಶಿಸಿ, ನಾನು ಹುಟ್ಟುವ ವೇಳೆಗೆ ಸರಿಯಾಗಿ ನೀನೂ ಹುಟ್ಟಬೇಕು. ನಿನ್ನನ್ನು ಲೋಕದ ಜನ ಭದ್ರಕಾಳಿ, ವೈಷ್ಣವಿ, ನಾರಾಯಣಿ, ಮಾಧವೀ, ಶಾರದೆ, ಅಂಬಿಕೆ ಇತ್ಯಾದಿ ನಾನಾ ಹೆಸರುಗಳಿಂದ ಕರೆದು ಪೂಜಿಸುತ್ತಾರೆ’ ಎಂದು ಹೇಳಿದನು. ಆತನ ಅಪ್ಪಣೆ ಯಂತೆ ಮಾಯಾದೇವಿಯು ದೇವಕಿಯ ಗರ್ಭವನ್ನು ಸೆಳೆದು ರೋಹಿಣಿಯ ಹೊಟ್ಟೆಯ ಲ್ಲಿಟ್ಟಳು. ಒಡನೆಯೆ ದೇವಕಿಯ ಗರ್ಭ ಬರಿದಾಯಿತು. ಇದನ್ನು ಕಂಡ ಜನ ದೇವಕಿಗೆ ಗರ್ಭಸ್ರಾವವಾಯಿತೆಂದು ಮಮ್ಮಲ ಮರುಗಿದರು. ಅತ್ತ ರೋಹಿಣಿಯು ಗರ್ಭಿಣಿ ಯಾಗಿ, ದಿನ ತುಂಬುತ್ತಲೆ ಸುಂದರನಾದ ಸುಕುಮಾರನನ್ನು ಹೆತ್ತಳು. ಲೋಕರಂಜಕನಾದ ಆತನಿಗೆ ರಾಮನೆಂದು ನಾಮಕರಣವಾಯಿತು. ಮುಂದೆ ಆತನೆ ತನ್ನ ಪರಾಕ್ರಮದಿಂದ ಬಲರಾಮನೆಂದೂ ಬಲಭದ್ರನೆಂದೂ ಪ್ರಖ್ಯಾತನಾದನು. ಗರ್ಭವನ್ನು ಸೆಳೆದು ಹುಟ್ಟಿ ದುದರಿಂದ ಆತನಿಗೆ ‘ಸಂಕರ್ಷಣ’ನೆಂದೂ ಹೆಸರಾಯಿತು.

ಆದಿಶೇಷನು ಬಲರಾಮನಾಗಿ ಭೂಮಿಗೆ ಅವತರಿಸುತ್ತಲೆ ಶ್ರೀಹರಿಯೂ ತನ್ನ ಅವ ತರಣಕ್ಕೆ ಸಿದ್ಧನಾದನು. ಸರ್ವವ್ಯಾಪಿಯಾದ ಆತನು ವಸುದೇವನ ಹೃದಯವನ್ನು ಹೊಕ್ಕು ಅಲ್ಲಿ ನೆಲಸಿದನು. ಭಗವಂತನ ತೇಜಸ್ಸು ಬಂದು ನೆಲಸುತ್ತಲೆ ವಸುದೇವನು ಸೂರ್ಯನಂತೆ ಕಾಂತಿಯುಕ್ತನಾದನು. ಆ ಸಮಯದಲ್ಲಿ ಆತನನ್ನು ಕಂಡರೆ ಕಂಸನಿಗೆ ಸಹ ಏನೋ ಒಂದು ಬಗೆಯ ದಿಗಿಲು; ಆತನ ತೇಜಸ್ಸು ಅಷ್ಟು ಪ್ರಖರವಾಗಿತ್ತು. ಕೆಲವು ದಿನಗಳು ಕಳೆಯುವಷ್ಟರಲ್ಲಿ ವಸುದೇವನಲ್ಲಿ ನೆಲಸಿದ್ದ ಆ ತೇಜಸ್ಸು ದೇವಕಿಯ ಗರ್ಭವನ್ನು ಪ್ರವೇಶಿಸಿತು. ಒಡನೆಯೆ ಆಕೆ ಚಂದ್ರಕಳೆಯನ್ನು ಧರಿಸಿದ ಪೂರ್ವದಿಕ್ಕಿನಂತೆ ತೊಳಗಿ ಬೆಳಗಿದಳು. ಆಕೆಯ ದೇಹ ಕಾಂತಿಯಿಂದ ತುಂಬಿತು. ಇದನ್ನು ಕಂಡು ಕಂಸನಿಗೆ ಕಳವಳ ವಾಯಿತು. ಆತ ತನ್ನಲ್ಲಿಯೆ ತಾನು ‘ಎಲ ಎಲ, ಇದೆಂತಹ ಸೋಜಿಗ! ಹಿಂದೆಲ್ಲ ಗರ್ಭಿಣಿ ಯಾಗಿದ್ದಾಗ ಈಕೆಯಲ್ಲಿ ಇಂತಹ ತೇಜಸ್ಸನ್ನು ನಾನು ಕಂಡಿರಲಿಲ್ಲ. ಬಹುಶಃ ನನ್ನ ಮೃತ್ಯು ವಾದ ವಿಷ್ಣು ಈಗ ಇವಳ ಗರ್ಭವನ್ನು ಹೊಕ್ಕಿರವನೋ ಏನೋ! ಈಗ ನಾನೇನು ಮಾಡಲಿ? ಇವಳನ್ನು ಕೊಂದುಹಾಕಿಬಿಡಲೆ? ಛೆ, ಛೆ, ಅದು ಶುದ್ಧ ಅನ್ಯಾಯ. ಇವಳು ನನ್ನ ತಂಗಿ, ಗರ್ಭಿlಣಿಯಾಗಿದ್ದಾಳೆ; ಈಗ ಇವಳನ್ನು ಕೊಂದರೆ ನನ್ನ ಕೀರ್ತಿಗೆ ಮಸಿ ಬಳಿದಂತಾಗುತ್ತದೆ. ನನ್ನ ಜೀವಿತಕಾಲದಲ್ಲಿ ಮಾತ್ರವೇ ಅಲ್ಲ, ಸತ್ತಮೇಲೆ ಕೂಡ ಜನ ನನ್ನನ್ನು ಪಾಪಿಯೆಂದು ನಿಂದಿಸುತ್ತಾರೆ. ಆಯಿತು, ಅವಳ ಹೊಟ್ಟೆಯಿಂದ ಮಗು ಹುಟ್ಟುವ ತನಕ ಕಾದಿದ್ದು, ಆಮೇಲೆ ಆ ಮಗುವನ್ನು ತೀರಿಸಿದರಾಯಿತು’ ಎಂದುಕೊಂಡನು. ಅಲ್ಲಿಂದ ಮುಂದೆ ಅವ ನಿಗೆ ಹಗಲಿರುಳೂ–ನಿಂತಿರುವಾಗ, ಕುಳಿತಿರುವಾಗ, ಓಡಾಡುವಾಗ, ಊಟಮಾಡುವಾಗ –ವಿಷ್ಣುವಿನ ಮೇಲಿನ ದ್ವೇಷದಿಂದ, ಆ ವಿಷ್ಣುವಿನ ಚಿಂತೆಯೇ ಆಯಿತು. ಅವನ ಪಾಲಿಗೆ ಜಗತ್ತೆಲ್ಲ ವಿಷ್ಣುಮಯವಾಯಿತು.

ಶ್ರೀಹರಿಯು ದೇವಕಿಯ ಗರ್ಭದಲ್ಲಿ ಬೆಳೆಯುತ್ತಿರುವುದನ್ನು ಅರಿತ ಬ್ರಹ್ಮರುದ್ರಾದಿ ದೇವತೆಗಳೂ ಮಹರ್ಷಿಗಳೂ ದೇವಕಿಯಿದ್ದ ಸೆರೆಮನೆಯನ್ನು ಹೊಕ್ಕು, ಶ್ರೀಹರಿಯನ್ನು ಸ್ತುತಿಸಹೊರಟರು–“ಹೇ ದೇವ ದೇವ! ಸತ್ಯಸ್ವರೂಪನಾದ ನೀನು ಕೊಟ್ಟ ಮಾತಿನಂತೆ ದೇವಕೀದೇವಿಯ ಗರ್ಭಕ್ಕೆ ಇಳಿದು ಬಂದಿರುವೆ. ನಿನ್ನನ್ನು ಪಡೆಯಲು ಸತ್ಯವೇ ಶರಣು. ಹೇ ಭಗವಂತ! ಈ ಪ್ರಪಂಚವೆಂಬ ಅನಾದಿಯಾದ ವೃಕ್ಷಕ್ಕೆ ನೀನೆ ಊರುಗೋಲು; ಸತ್ವ, ರಜ, ತಮಗಳೆಂಬ ಬೇರುಗಳ ಮೇಲೆ ನಿಂತಿರುವ ಈ ವೃಕ್ಷದಲ್ಲಿ ಸುಖದುಃಖಗಳೆಂಬ ಎರಡು ಹಣ್ಣುಗಳು ಇವೆ. ಇವನ್ನು ತಿನ್ನುವುದಕ್ಕಾಗಿ ಜೀವಾತ್ಮ, ಪರಮಾತ್ಮ ಎಂಬ ಎರಡು ಹಕ್ಕಿಗಳು ಕುಳಿತಿವೆ. ಸ್ವಾಮಿ, ಆ ಮರವನ್ನು ನೆಟ್ಟವನು ನೀನೆ, ರಕ್ಷಿಸುವವನು ನೀನೆ, ಕೊನೆಗದನ್ನು ಕಿತ್ತೆಸೆದು ಆಟವಾಡುವವನೂ ನೀನೆ. ಸಚ್ಚಿದಾನಂದ ಸ್ವರೂಪಿಯಾದ ನೀನು ಈ ಜಗತ್ತನ್ನು ಸೃಷ್ಟಿಸಿ, ದುಷ್ಟಶಿಕ್ಷಣ ಶಿಷ್ಟರಕ್ಷಣೆಗಾಗಿ ಅವತಾರವೆತ್ತುವ ನಾಟಕ ವನ್ನಾಡುವೆ. ಕೇವಲ ಲೋಕರಕ್ಷಣೆಗಾದರೆ ನೀನು ಅವತಾರವೆತ್ತುವುದೇಕೆ? ಸಂಕಲ್ಪಮಾತ್ರ ದಿಂದಲೆ ನೀನು ಅದನ್ನು ನಿರ್ವಹಿಸಬಲ್ಲೆ. ಆದರೆ, ಲೋಕದ ಜನ ನಿನ್ನ ಅವತಾರದ ಕಥೆ ಯನ್ನು ಕೇಳಿ, ಭಕ್ತಿಯಿಂದ ನಿನ್ನನ್ನು ಪೂಜಿಸಿ, ಮುಕ್ತಿಯನ್ನು ಪಡೆಯಲೆಂಬುದೆ ನಿನ್ನ ಅವತಾರದ ಗುಟ್ಟು. ಅಲ್ಲದೆ ನೀನು ಮಾನವನಾಗಿ ಅವತರಿಸಿದಾಗ ಮಾನವನಂತೆಯೇ ನಟಿಸುವೆಯಾದರೂ, ಆಗಾಗ ಕೆಲವು ಅದ್ಭುತಕಾರ್ಯಗಳನ್ನು ಮಾಡಿ, ಅಜ್ಞಾನಿಗಳ ಕಣ್ಣನ್ನು ತೆರಸುವೆ. ಅವರು ಅದ್ಭುತಕಾರ್ಯಗಳಿಗಾಗಿ ನಿನ್ನನ್ನು ಪೂಜಿಸುವರು; ಆ ಮೂಲಕ ಅವರಲ್ಲಿ ದೈವಭಕ್ತಿ ಮೂಡುವುದು. ಅವರು ಯಾವ ವಿಗ್ರಹದ ರೂಪಿನಿಂದಲೇ ನಿನ್ನನ್ನು ಪೂಜಿಸಲಿ, ಅದು ನಿನಗೆ ಸಲ್ಲುತ್ತದೆ. ಸ್ವಾಮಿ, ನೀನೀಗ ದೇವಕೀದೇವಿಯ ಮಗನ ರೂಪದಿಂದ ಅವತರಿಸುತ್ತಿರುವೆ. ನಿರ್ವಿಕಾರನಾದ ನೀನು ಈ ವಿಕಾರವನ್ನು ತೋರಿ, ಭೂಭಾರವನ್ನು ತಗ್ಗಿಸುವ ಲೀಲೆಯನ್ನು ಕೈಕೊಂಡಿರುವೆ. ನಿನಗೆ ನಮೋನಮೋ!” ಹೀಗೆ ಶ್ರೀಹರಿಯನ್ನು ಸ್ತುತಿಸಿದ ಮೇಲೆ, ಅವರು ದೇವಕೀದೇವಿಯನ್ನು ಕುರಿತು ‘ಅಮ್ಮ, ನಿನ್ನ ಮತ್ತು ಯಾದವ ರೆಲ್ಲರ ಕಷ್ಟಗಳೂ ಇನ್ನು ಕೊನೆಯಾದುವೆಂದು ತಿಳಿ’ ಎಂದು ಸಮಾಧಾನ ಹೇಳಿ, ತಮ್ಮ ತಮ್ಮ ಲೋಕಗಳಿಗೆ ಹಿಂದಿರುಗಿದರು.

ಕಾಲಚಕ್ರ ಮತ್ತಷ್ಟು ಮುಂದುರುಳಿತು. ಶ್ರೀಹರಿಯು ಮಾನವನಾಗಿ ಅವತರಿಸುವ ಮಂಗಳ ಮುಹೂರ್ತವು ಸನ್ನಿಹಿತವಾಯಿತು. ಒಡನೆಯೆ ಗ್ರಹ ನಕ್ಷತ್ರಗಳೆಲ್ಲ ಶುಭಸ್ಥಾನ ವನ್ನು ಸೇರಿದವು. ಸೋದರಮಾವನ ಸಾವನ್ನು ಸಾರುವ ರೋಹಿಣೀ ನಕ್ಷತ್ರವು ಜನ್ಮನಕ್ಷತ್ರ ವಾಗಿ ಬಂದು ಸಂಧಿಸಿತು; ದಿಕ್ಕುಗಳೆಲ್ಲವೂ ಶುಭ್ರಧವಳಕಾಂತಿಯಿಂದ ತುಂಬಿದವು, ಆಕಾಶವು ನಿರ್ಮಲವಾಯಿತು, ನದಿ ತೊರೆಗಳ ನೀರು ತಿಳಿಯಾಯಿತು, ವನಕಾನನಗಳ ಗಿಡಮರಗಳು ಹೂ ಹಣ್ಣುಗಳಿಂದ ತುಂಬಿ ಹೋದವು, ತಂಗಾಳಿಯು ಸುವಾಸನೆಯನ್ನು ಹೊತ್ತು ಮಂದಮಂದವಾಗಿ ಬೀಸಿತು, ಜನಮನಗಳು ಆನಂದದಿಂದ ತುಂಬಿದವು. ಹಕ್ಕಿಗಳು ಇಂಪಾಗಿ ಗಾನಮಾಡಿದವು. ಆನಂದದ ಲಹರಿಯೊಂದು ಇಡೀ ಬ್ರಹ್ಮಾಂಡ ವನ್ನೆ ಅಲೆಅಲೆಯಾಗಿ ಹಬ್ಬಿ ತುಂಬಿದಂತಾಯಿತು. ಗಂಧರ್ವರು ಗಾನಮಾಡುತ್ತಿರಲು ಅಪ್ಸರೆಯರು ನರ್ತನ ಮಾಡಿದರು, ದೇವತೆಗಳು ತಮ್ಮ ನಗಾರಿಯನ್ನು ಬಾರಿಸಿ ಹೂಮಳೆ ಯನ್ನು ಕರೆದರು. ಆಗ ನಡುರಾತ್ರಿ, ಇದ್ದಕಿದ್ದಂತೆಯೆ ಮೋಡಗಳು ಆಕಾಶಕ್ಕೆ ಏರಿಬಂದು, ಗಂಭೀರವಾದ ಗುಡುಗಿನ ದನಿಯನ್ನು ಹೊರಸೂಸಿದವು. ವಿಷ್ಣುಮಾಯೆಯಂತೆ ಕಗ್ಗತ್ತಲು ಜಗತ್ತನ್ನೆಲ್ಲ ತುಂಬಿತು. ಇಂತಹ ಕಗ್ಗತ್ತಲೆಯ ನಟ್ಟಿರುಳಿನಲ್ಲಿ ಸಕಲ ಕಲ್ಯಾಣಗುಣ ಪರಿಪೂರ್ಣನಾದ ಶ್ರೀಹರಿಯು ಪೂರ್ವ ದಿಕ್ಕಿನಿಂದ ಮೂಡಿಬರುವ ಪೂರ್ಣಚಂದ್ರನಂತೆ ದೇವಕಿಯ ಗರ್ಭದಿಂದ ಇಳೆಗೆ ಅವತರಿಸಿದನು.

ತನ್ನ ಪೂರ್ಣಾಂಶದಿಂದ ಹುಟ್ಟಿರುವ ಆ ಪರಮ ಪುರುಷೋತ್ತಮನ ರೂಪವೈಭವ ವನ್ನು ಯಾವನು ತಾನೆ ಮಾತುಗಳಿಂದ ವರ್ಣಿಸಿ ಹೇಳಬಲ್ಲನು? ಅರಳಿದ ಕಮಲದಂತೆ ವಿಶಾಲವಾದ ಆ ಕಣ್ಣುಗಳು, ನಾಲ್ಕು ತೋಳುಗಳು, ಆ ನಾಲ್ಕರಲ್ಲಿಯೂ ಶಂಖ ಚಕ್ರ ಇತ್ಯಾದಿ ಆಯುಧಗಳು, ಎದೆಯಲ್ಲಿ ಶ್ರೀವತ್ಸವೆಂಬ ಮಚ್ಚೆ, ಕೊರಳಲ್ಲಿ ಕೌಸ್ತುಭರತ್ನ, ನಡುವಿನಲ್ಲಿ ದಿವ್ಯವಾದ ಪೀತಾಂಬರ, ಮುಂಗಾರ ಮುಗಿಲಿನಂತಿರುವ ದೇಹಕಾಂತಿ, ವಜ್ರವೈಢೂರ್ಯಗಳಿಂದ ಕೂಡಿದ ಕಿರೀಟಕುಂಡಲಗಳು, ಅವುಗಳ ಕಾಂತಿಯಿಂದ ಝಗ ಝಗಿಸುವ ಮುಂಗುರುಳು, ನಡುವಿನ ಒಡ್ಯಾಣ, ತೋಳ್ಬಳೆ, ಕಾಲ್ಗಡಗ–ಹೀಗೆ ಸರ್ವಾ ಲಂಕಾರ ಭೂಷಿತವಾದ ದಿವ್ಯಸುಂದರ ವಿಗ್ರಹವನ್ನು ಕಂಡ ವಸುದೇವ ದೇವಕಿಯರು ಅಚ್ಚರಿಯಿಂದ ಬಿಟ್ಟ ಕಣ್ಣನ್ನು ಮುಚ್ಚಲಾರದೆ ಮಗುವನ್ನು ದಿಟ್ಟಿಸಿ ನೋಡಿದರು. ಸಾಕ್ಷಾತ್ ಮಹಾವಿಷ್ಣುವೆ ಮಗನಾಗಿ ಹುಟ್ಟಿರುವಾಗ ಅವರಿಗಾದ ಸಂಭ್ರಮವನ್ನು ಕೇಳ ಬೇಕೆ! ಅವರು ಶಿಶುವಿನ ಮಂದೆ ಕೈಜೋಡಿಸಿ ನಿಂತು ಭಕ್ತಿಯಿಂದ ‘ಹೇ ಸಚ್ಚಿದಾನಂದ! ಸರ್ವಾಂತರ್ಯಾಮಿ! ಸರ್ವವ್ಯಾಪಕ! ಸರ್ವಸ್ವರೂಪ! ಮಾಯಾತೀತ! ಪರಮೇಶ್ವರ! ನೀನು ನಿರ್ಗುಣ, ನಿರ್ವಿಕಾರ, ನಿರಪೇಕ್ಷ! ಪರಬ್ರಹ್ಮಸ್ವರೂಪಿಯಾದ ನೀನು ನಿನ್ನ ಮಾಯೆ ಯಿಂದ ಮೂರು ಲೋಕಗಳನ್ನೂ ಹುಟ್ಟಿಸುವೆ, ರಕ್ಷಿಸುವೆ, ಅಳಿಸುವೆ! ಆದರೂ ನಾವು ನಿನ್ನ ಮಾಯೆಗೆ ಒಳಗಾದವರು. ಅದರಿಂದಲೆ ಕಂಸನೆಲ್ಲಿ ಬಂದು ನಿನಗೆ ಕೇಡನ್ನು ಮಾಡು ವನೋ ಎಂಬ ಭಯ. ಮಹಾಯೋಗಿಗಳಿಗೆ ಮಾತ್ರ ಗೋಚರವಾಗುವ ಈ ದಿವ್ಯಮಂಗಳ ರೂಪವನ್ನು ಬೇಗ ಮರೆಗೊಳಿಸು. ಸಕಲ ಬ್ರಹ್ಮಾಂಡವನ್ನೂ ಹೊಟ್ಟೆಯಲ್ಲಿಟ್ಟುಕೊಂಡಿ ರುವ ನೀನು ನಮ್ಮ ಹೊಟ್ಟೆಯ ಮಗುವೆಂದು ಹೇಳಿಕೊಳ್ಳುವುದು ನಗೆಪಾಟಲಾದರೂ, ಹಾಗೆ ಹೇಳಿಕೊಳ್ಳುವಂತೆ ನಮ್ಮನ್ನು ಅನುಗ್ರಹಿಸಿರುವ ನಿನಗೆ ನಮೋ ನಮೋ’ ಎಂದು ಬೇಡಿಕೊಂಡರು.

ದೇವಕಿ ವಸುದೇವರ ವಾತ್ಸಲ್ಯಭಕ್ತಿಯನ್ನು ಕಂಡು ಮೆಚ್ಚಿದ ಭಗವಂತನು ಅವರನ್ನು ಕುರಿತು ಅವರ ಪೂರ್ವಜನ್ಮದ ವೃತ್ತಾಂತವನ್ನೆಲ್ಲ ವಿಶದವಾಗಿ ತಿಳಿಸಿದನು. ಹಿಂದೆ ಸ್ವಾಯಂಭುಮನುವಿನ ಕಾಲದಲ್ಲಿ ಅವರು ಸುತಪ-ಪೃಶ್ನಿ ಎಂಬ ದಂಪತಿಗಳಾಗಿದ್ದರು. ಅವರು ಬ್ರಹ್ಮನ ಅಪ್ಪಣೆಯಂತೆ ಸಂತಾನವನ್ನು ಪಡೆಯುವುದಕ್ಕಾಗಿ ಶ್ರೀಹರಿಯನ್ನು ಕುರಿತು ಒಂದೆ ಮನಸ್ಸಿನಿಂದ ಬಹುಕಾಲ ತಪಸ್ಸು ಮಾಡಿದರು. ಮಹಾವಿಷ್ಣುವು ಪ್ರತ್ಯಕ್ಷ ನಾಗಿ ವರವನ್ನು ಬೇಡೆನ್ನಲು ಅವರು ‘ನಿನ್ನಂತಹ ಪುತ್ರನಾಗಬೇಕು’ ಎಂದು ಬೇಡಿ ಕೊಂಡರು. ತನಗೆ ಸಮಾನನಾದವನು ಮತ್ತೊಬ್ಬನಿಲ್ಲದುದರಿಂದ ಶ್ರೀಹರಿಯೇ ಅವರ ಮಗನಾಗಿ ಪೃಶ್ನಿಗರ್ಭನೆಂಬ ಹೆಸರಿನಿಂದ ಹುಟ್ಟಿದ. ಆ ದಂಪತಿಗಳು ಮರುಜನ್ಮದಲ್ಲಿ ಕಶ್ಯಪ ಅದಿತಿಯರಾಗಿ ಹುಟ್ಟಿದರು. ಆಗಲೂ ಶ್ರೀಹರಿ ವಾಮನನೆಂಬ ಹೆಸರಿನಿಂದ ಅವರ ಮಗನಾಗಿ ಹುಟ್ಟಿದ. ಆ ಕಶ್ಯಪ ಅದಿತಿಯರೆ ಈಗ ವಸುದೇವ ದೇವಕಿಯರಾಗಿ ಹುಟ್ಟಿ ದ್ದಾರೆ. ಈ ಜನ್ಮದಲ್ಲಿಯೂ ಶ್ರೀಹರಿ ಅವರ ಮಗನಾಗಿ ಹುಟ್ಟಿದ್ದಾನೆ. ಆ ದಂಪತಿಗಳಿಗೆ ಅವರ ಪೂರ್ವಜನ್ಮದ ಕಥೆಯನ್ನು ಜ್ಞಾಪಿಸುವುದಕ್ಕಾಗಿಯೇ ಶ್ರೀಹರಿ ಈಗ ಸ್ವಸ್ವರೂಪ ದಿಂದ ಹುಟ್ಟಿರುವುದು. ಇದನ್ನು ತಿಳಿಸಿದ ಮೇಲೆ ಶ್ರೀಹರಿಯು ವಸುದೇವನನ್ನು ಕುರಿತು ‘ಕಂಸನಿಗೆ ಅಷ್ಟು ಹೆದರುವುದಾದರೆ ನೀನು ಈಗಲೆ ನನ್ನನ್ನು ನಂದಗೋಕುಲಕ್ಕೆ ತೆಗೆದು ಕೊಂಡು ಹೋಗು. ಅಲ್ಲಿ ನನ್ನ ಮಾಯೆಯು ನಂದನ ಮಡದಿಯಾದ ಯಶೋದೆಯ ಗರ್ಭದಲ್ಲಿ ಈಗತಾನೆ ಹುಟ್ಟಿದ್ದಾಳೆ. ನನ್ನನ್ನು ಅಲ್ಲಿಟ್ಟು, ಅವಳನ್ನು ಇಲ್ಲಿಗೆ ಕರೆದು ತಾ. ಇದು ಹೇಗೆ ಸಾಧ್ಯ ಎಂಬ ಚಿಂತೆಬೇಡ. ನನ್ನ ಮಹಿಮೆಯಿಂದ ಅಸಾಧ್ಯವಾದುದೂ ಸಾಧ್ಯ ವಾಗುವುದು’ ಎಂದು ಹೇಳಿದನು. ಮರುಕ್ಷಣದಲ್ಲಿ, ದೇವಕಿ ವಸುದೇವರು ನೋಡುತ್ತಿ ದ್ದಂತೆಯೆ ಶ್ರೀಹರಿಯು ಲೋಕಸಾಮಾನ್ಯವಾದ ಶಿಶುವಿನಂತೆ ಕಾಣಿಸಿಕೊಂಡನು.

ವಸುದೇವನು ಶಿಶುರೂಪಿಯಾದ ಭಗವಂತನ ಅಪ್ಪಣೆಯಂತೆ ಆ ಶಿಶುವನ್ನು ಎತ್ತಿ ಕೊಂಡು ಸೆರೆಮನೆಯಿಂದ ಹೊರಟನು. ಸೆರೆಮನೆಯೆಂದಮೇಲೆ ಕೇಳಬೇಕೆ? ಅದಕ್ಕೆ ಬಲ ವಾದ ಬಾಗಿಲುಗಳು, ದಪ್ಪನಾದ ಅಗಳಿಗಳು, ಉಕ್ಕಿನ ಸರಪಳಿಗಳಿಂದ ಆ ಬಾಗಿಲುಗಳನ್ನು ಬಂಧಿಸಲಾಗಿತ್ತು. ಆದರೇನು? ಸೂರ್ಯನ ಮುಂದೆ ಕಗ್ಗತ್ತಲೆಯೆ? ವಸುದೇವನು ಮಗುವಿ ನೊಡನೆ ಬಾಗಿಲ ಬಳಿಗೆ ಬರುತ್ತಲೆ, ಆ ಬಾಗಿಲುಗಳು ತಮಗೆ ತಾವೆ ತೆರೆದುಕೊಂಡವು. ಹಗಲಿರುಳೆನ್ನದೆ ಕಾವಲು ಕಾಯುತ್ತಿದ್ದವರೆಲ್ಲ ಒಮ್ಮೆಗೆ ಗಾಢನಿದ್ರೆಯಲ್ಲಿ ಮೈಮರೆತು, ಹೆಣಗಳಂತೆ ನೆಲಕ್ಕೊರಗಿದರು. ಊರಜನಕ್ಕೆಲ್ಲ ಎಲ್ಲಿಲ್ಲದ ನಿದ್ರೆ ಅಂದು. ಜನರಾರೂ ಹೊರಗೆ ಸಂಚರಿಸಲಾರದಂತೆ ಕಗ್ಗತ್ತಲು ಕವಿದಿದೆ, ದೊಡ್ಡ ಮಳೆ ಸುರಿಯುತ್ತಿದೆ. ವಸುದೇವನು ಮಗುವನ್ನೆತ್ತಿಕೊಂಡು ಹೊರಗೆ ಕಾಲಿಡುತ್ತಿದಂತೆಯೇ, ಶಿಶುವಿನ ಮೇಲೆ ಮಳೆಯ ಹನಿ ಬೀಳದಂತೆ ಆದಿಶೇಷನು ತನ್ನ ಹೆಡೆಗಳನ್ನು ಕೊಡೆಯಾಗಿ ಮಾಡಿ ಆತನನ್ನು ಹಿಂಬಾಲಿಸಿದನು. ಮಧುರಾಪುರಿಯನ್ನು ಬಿಟ್ಟು ಹೊರಡುತ್ತಲೆ ಪೂರ್ಣಪ್ರವಾಹದಿಂದ ಭಯಂಕರಾಕಾರದಲ್ಲಿ ಹರಿಯುತ್ತಿದ್ದ ಯಮುನಾ ನದಿ ಅವರನ್ನು ಇದಿರಿಸಿತು. ಆದರೆ ಅವರು ತನ್ನ ಬಳಿಗೆ ಬರುತ್ತಿದ್ದಂತೆಯೆ ಆ ನದಿ ಇಬ್ಭಾಗವಾಗಿ ಅವರಿಗೆ ಹಾದಿಯನ್ನು ಮಾಡಿಕೊಟ್ಟಿತು. ವಸುದೇವನು ನದಿಯನ್ನು ದಾಟಿ ನಂದಗೋಕುಲಕ್ಕೆ ಕಾಲಿಡುತ್ತಿದ್ದಂತೆ ಅಲ್ಲಿನವರೆಲ್ಲರೂ ಗಾಢನಿದ್ರೆಯಲ್ಲಿ ಮೈ ಮರೆತರು. ವಸುದೇವನು ನಂದನ ಮನೆಯನ್ನು ಹೊಕ್ಕು ನೋಡುತ್ತಾನೆ, ಯಶೋದೆ ಆಗತಾನೆ ಮಗುವನ್ನು ಹೆತ್ತು, ಆಯಾಸದಿಂದ ಮೈ ಮರೆತು ಮಲಗಿದ್ದಳು. ತನಗೆ ಹುಟ್ಟಿದ ಮಗು ಹೆಣ್ಣೊ, ಗಂಡೊ ಎಂಬುದನ್ನು ಸಹ ಅವಳು ಅರಿಯಳು. ವಸುದೇವನು ತನ್ನ ಶಿಶುವನ್ನು ಪಕ್ಕದಲ್ಲಿ ಮಲಗಿಸಿ, ಅಲ್ಲಿದ್ದ ಹೆಣ್ಣು ಶಿಶುವನ್ನು ತಾನೆತ್ತಿಕೊಂಡು ಹಿಂದಿರುಗಿ, ದೇವಕಿಯ ಪಕ್ಕದಲ್ಲಿ ಮಲಗಿಸಿದನು.

