
\chapter{೭೭. ಶ್ರೀಕೃಷ್ಣಮಾಯೆಯ ವೈಭವ}

ಶ್ರೀಕೃಷ್ಣನು ನರಕಾಸುರನನ್ನು ಕೊಂದು, ಅವನ ಬಂಧನದಲ್ಲಿದ್ದ ಹದಿನಾರು ಸಾವಿರ ರಾಜಪುತ್ರಿಯರನ್ನು ಮದುವೆಯಾದನೆಂಬ ಸುದ್ದಿಯನ್ನು ಕೇಳಿ, ಮಹರ್ಷಿಯಾದ ನಾರದರು ಆ ಕಪಟನಾಟಕ ಸೂತ್ರಧಾರಿಯ ಕುಟುಂಬಲೀಲೆಯನ್ನು ನೋಡಬೇಕೆಂಬ ಕುತೂಹಲದಿಂದ ದ್ವಾರಕಾನಗರಕ್ಕೆ ಬಂದರು. ಅದರ ಸುತ್ತಲಿದ್ದ ಉಪವನ, ಹಕ್ಕಿಗಳ ಗಾಯನ, ಸರೋವರಗಳ ಸಿಂಗಾರಗಳಿಂದ ಕಣ್ಣು ಕಿವಿಗಳನ್ನು ತಣಿಸುತ್ತಾ ಅವರು ನಗರ ವನ್ನು ಪ್ರವೇಶಿಸಿದರು. ಅಲ್ಲದೆ ರಸ್ತೆಗಳು, ಮನೆಗಳು, ಅಂಗಡಿಗಳು, ದೇವಾಲಯಗಳು– ಇವೆಲ್ಲ ಸ್ವರ್ಗದ ಸೌಂದರ್ಯವನ್ನೂ ನಗುವಂತಿದ್ದವು. ಬಿಡುಗಣ್ಣುಗಳಿಂದ ಅವುಗಳನ್ನು ನೋಡುತ್ತಾ ನಾರದರು ಶ್ರೀಕೃಷ್ಣನ ಅಂತಃಪುರಕ್ಕೆ ಬಂದರು. ದ್ವಾರಕಿಯನ್ನು ನಿರ್ಮಿಸಿದ ವಿಶ್ವಕರ್ಮನು ತನ್ನ ಕಲಾಕೌಶಲ್ಯವನ್ನೆಲ್ಲ ಇಲ್ಲಿ ವೆಚ್ಚಮಾಡಿದ್ದನು. ಆ ಅಂತಃಪುರದಲ್ಲಿ ಹದಿನಾರು ಸಾವಿರ ಉಪ್ಪರಿಗೆಯ ಮನೆಗಳು. ನಾರದರು ನೇರವಾಗಿ ಒಂದು ಮನೆಯನ್ನು ಹೊಕ್ಕರು. ಅದೇನು ವೈಭವ ಅಲ್ಲಿ! ಕಂಬಗಳೆಲ್ಲ ಹವಳ, ಬೋದಿಗೆಗಳೆಲ್ಲ ವಜ್ರ ವೈಡೂರ್ಯ, ಗೋಡೆಗಳೆಲ್ಲ ಇಂದ್ರನೀಲಮಣಿ, ಮುತ್ತಿನ ಕುಚ್ಚುಗಳುಳ್ಳ ಮೇಲ್ಕಟ್ಟುಗಳು, ಅಲ್ಲಲ್ಲಿ ರತ್ನಖಚಿತವಾದ ಬಂಗಾರದ ಮಂಚಗಳು; ಅಲ್ಲಿನ ದಾಸಿಯರೆಲ್ಲ ರಾಣಿಯರಂತೆ ರತ್ನಾಭರಣಗಳಿಂದ ಅಲಂಕೃತರಾಗಿದ್ದಾರೆ, ಅವರ ಬಟ್ಟೆಗಳೆಲ್ಲ ಹಾಲಿನ ನೊರೆಯಂತೆ ಬೆಳ್ಳಗಿರುವ ಮಗುಟಗಳು. ಅಲ್ಲಿನ ರತ್ನದ ದೀಪಗಳಿಂದ ಸದಾ ಬೆಳ್ಳಂಬೆಳಗು, ಅಲ್ಲ ಲ್ಲಿಯೆ ಹೊತ್ತಿ ಉರಿಯುತ್ತಿರುವ ಅಗರು ಧೂಪಗಳು. ನಾರದರು ಈ ಐಶ್ವರ್ಯ ವೈಭವ ಗಳನ್ನು ಅಚ್ಚರಿಯಿಂದ ನೋಡುತ್ತಾ ಒಳಗೆ ಪ್ರವೇಶಿಸಿದರು. ಅಗೋ, ಇದಿರಿನ ಮಂಚದ ಮೇಲೆ ಶ್ರೀಕೃಷ್ಣ ಕುಳಿತಿದ್ದಾನೆ. ಸಹಸ್ರಾರು ದಾಸಿಯರಿದ್ದರೂ ರುಕ್ಮಿಣಿ ತಾನೆ ಕೈಲಿ ಚಾಮರ ವನ್ನು ಹಿಡಿದು ಆತನಿಗೆ ಗಾಳಿ ಹಾಕುತ್ತಿದ್ದಾಳೆ. ಅದನ್ನು ಕಂಡು ಆನಂದಪರವಶರಾದರು ನಾರದರು.

 ಶ್ರೀಕೃಷ್ಣನು ನಾರದರನ್ನು ಕಾಣುತ್ತಲೆ ದಿಗ್ಗನೆ ಮಂಚದಿಂದ ಧುಮ್ಮಿಕ್ಕಿ ಆತನಿಗೆ ಅಡ್ಡ ಬಿದ್ದನು; ಅನಂತರ ಆತನನ್ನು ಕರೆತಂದು ತನ್ನ ಮಂಚದ ಮೇಲೆ ಕುಳ್ಳಿರಿಸಿ, ಬಂಗಾರದ ತಟ್ಟೆಯಲ್ಲಿ ಆತನ ಪಾದಗಳನ್ನಿಟ್ಟು ತೊಳೆದನು. ಆ ತೀರ್ಥವನ್ನು ತಲೆಯ ಮೇಲೆ ಪ್ರೋಕ್ಷಿಸಿಕೊಂಡು ‘ಸ್ವಾಮಿ, ತಮ್ಮ ಆಗಮನದಿಂದ ಧನ್ಯನಾದೆ. ತಮಗೆ ನನ್ನಿಂದ ಆಗ ಬೇಕಾದ ಕಾರ್ಯವೇನಾದರೂ ಇದೆಯೆ?’ ಎಂದು ಕೇಳಿದನು. ಅಮೃತದಂತಿರುವ ಆತನ ಮಾತುಗಳಿಂದ ಮುಗ್ಧರಾದ ನಾರದರು ‘ಹೇ ಲೋಕೇಶ್ವರ, ನನಗೆ ಬೇಕಾಗಿರುವುದು ನಿನ್ನ ಪಾದದರ್ಶನ. ಅದನ್ನು ಅನುಗ್ರಹಿಸಿರುವೆ. ನಾನು ಧನ್ಯನಾದೆ. ಆ ಪಾದಗಳನ್ನು ಸದಾ ಧ್ಯಾನಮಾಡುತ್ತಿರುವಂತೆ ನನ್ನನ್ನು ಹರಸು. ಇಷ್ಟೆ ನನ್ನ ಕೋರಿಕೆ’ ಎಂದು ಹೇಳಿ, ಆತನಿಂದ ಬೀಳ್ಕೊಂಡು, ಅಂತಃಪುರದ ಮತ್ತೊಂದು ಮನೆಗೆ ಬಂದರು. ಅಲ್ಲಿ ಶ್ರೀಕೃಷ್ಣನು ತನ್ನ ಮಡದಿಯೊಡನೆ ಪಗಡೆಯಾಟದಲ್ಲಿ ಮಗ್ನನಾಗಿದ್ದಾನೆ. ಆತನು ನಾರದರನ್ನು ಕಾಣುತ್ತಲೆ, ಆಗತಾನೆ ಆತನನ್ನು ಕಾಣುವವನಂತೆ ಭಕ್ತಿಯಿಂದ ಆತನನ್ನು ಇದಿರುಗೊಂಡು ನಮ ಸ್ಕರಿಸಿ, ‘ಪೂಜ್ಯರೆ, ಯಾವಾಗ ಬಂದಿರಿ? ನನ್ನಿಂದೇನಾಗಬೇಕು?’ ಎಂದನು. ನಾರದರು ಮುಗುಳ್ನಗುವನ್ನು ಪ್ರತ್ಯುತ್ತರವಾಗಿತ್ತು ಮತ್ತೊಂದು ಮನೆಗೆ ಹೋದರು. ಅಲ್ಲಿ ಶ್ರೀಕೃಷ್ಣ ತನ್ನ ಮುದ್ದು ಮಕ್ಕಳೊಡನೆ ಆಟವಾಡುತ್ತಿದ್ದ. ಅಲ್ಲಿಯೂ ಅವರಿಗೆ ‘ಪೂಜ್ಯರೆ, ಯಾವಾಗ ಬಂದಿರಿ? ನನ್ನಿಂದೇನಾಗಬೇಕು?’ ಎಂಬ ಪ್ರಶ್ನೆ! ಇನ್ನೊಂದು ಮನೆಗೆ ಹೋದರೆ ಅಲ್ಲಿ ಶ್ರೀಕೃಷ್ಣನು ನೀರೆರೆದುಕೊಳ್ಳುತ್ತಿದ್ದಾನೆ, ಮತ್ತೊಂದು ಮನೆಯಲ್ಲಿ ದೇವರಪೂಜೆ ಮಾಡು ತ್ತಿದ್ದಾನೆ, ಮಗುದೊಂದರಲ್ಲಿ ಸಂಧ್ಯಾವಂದನೆಗೆ ಕುಳಿತಿದ್ದಾನೆ. ಹೀಗೆಯೆ ಊಟಕ್ಕೆ, ಧ್ಯಾನಕ್ಕೆ, ಮಂತ್ರಾಲೋಚನೆಗೆ, ಮತ್ತಾವುದೊ ಕಾರ್ಯಕ್ಕೆ! ಮನೆಮನೆಯಲ್ಲಿಯೂ ಬೇರೆ ಬೇರೆ ಕಾರ್ಯಗಳಲ್ಲಿ ಮಗ್ನನಾಗಿ ಕುಳಿತಿರುವ ಶ್ರೀಕೃಷ್ಣನನ್ನು ಕಂಡು ನಾರದರು ಆಶ್ಚರ್ಯಗೊಂಡರು.

ಹದಿನಾರು ಸಾವಿರ ಹೆಂಡಿರಲ್ಲಿ ಹದಿನಾರು ಸಾವಿರ ಶ್ರೀಕೃಷ್ಣರೂಪದಿಂದ ಕಾಣಿಸಿ ಕೊಂಡು, ಅವರೆಲ್ಲರನ್ನೂ ಏಕಕಾಲದಲ್ಲಿಯೆ ಸಂತೋಷಪಡಿಸುತ್ತಾ, ಮಾನವರೂಪದಲ್ಲಿ ಇರುವಾಗಲೂ ಈ ಅದ್ಭುತಲೀಲೆಯನ್ನು ತೋರುತ್ತಿರುವ ಈ ಶ್ರೀಕೃಷ್ಣಮಾಯೆಯ ವೈಭವವನ್ನು ಕಣ್ಣಾರೆ ಕಂಡು ಧನ್ಯರಾದ ನಾರದರು ಅಂತಃಪುರದಿಂದ ಹೊರಕ್ಕೆ ಬರು ವಷ್ಟರಲ್ಲಿ ಶ್ರೀಕೃಷ್ಣನೂ ಅವರನ್ನು ಬೀಳ್ಕೊಳ್ಳಲೆಂದು ಹೊರಗೆ ಬಂದನು. ಆಗ ನಾರದರು ‘ಯೋಗೀಶ್ವರ, ಯೋಗಿಗಳಿಗೂ ಅಗೋಚರವಾದ ನಿನ್ನ ಲೀಲೆಯನ್ನು ಕಣ್ಣಾರೆ ಕಂಡು ನಾನಿಂದು ಧನ್ಯನಾದೆ’ ಎಂದರು. ಶ್ರೀಕೃಷ್ಣನು ‘ನಾರದ, ಕಾಲಕ್ಕೆ ತಕ್ಕಂತೆ ಧರ್ಮವನ್ನು ಉಪದೇಶಿಸುವವನೂ ನಾನೆ, ಅದನ್ನು ಮಾಡಿ ತೋರಿಸುವವನೂ ನಾನೆ. ಹಾಗೆ ಮಾಡಿ ತೋರಿಸುವುದಕ್ಕಾಗಿಯೆ ನಾನೀಗ ಮನುಷ್ಯನಾಗಿ ಅವತರಿಸಿದ್ದೇನೆ. ಇದನ್ನು ಕಂಡು ನೀನು ಅಚ್ಚರಿಪಡಬೇಕಾದುದಿಲ್ಲ’ ಎಂದ. ಆ ಮಾತು ಆತನ ಬಾಯಿಂದ ಬರುತ್ತಿದ್ದಂತೆಯೆ ನಾರದರಿಗೆ ಜ್ಞಾನದೃಷ್ಟಿ ಮೂಡಿತು. ಸಮಸ್ತ ಜೀವ ದೇಹಗಳಲ್ಲಿಯೂ ಆತ್ಮರೂಪಿನಿಂದ ಇರುವ ಆತ ಕಾಣಿಸಿದ. ಆಗ ನಾರದರು ‘ಆಹಾ, ಶ್ರೀಕೃಷ್ಣ ಮಾಯೆಯ ವೈಭವವೆ!’ ಎಂದು ಕೊಂಡು ಅಲ್ಲಿಂದ ಹೊರಟರು.

