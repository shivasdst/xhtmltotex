
\chapter{೩೨. ಮಹೇಶ್ವರನಿಗೂ ಮಂಕುಬೂದಿ}

ಶ್ರೀಹರಿಯು ಮೋಹಿನಿಯಾಗಿ ದೈತ್ಯದಾನವರಿಗೆಲ್ಲ ಮಂಕುಬೂದಿ ಎರಚಿದನೆಂದು ಕೇಳಿದ ಪರಶಿವನಿಗೆ, ತಾನೂ ಆ ಮೋಹಕರೂಪವನ್ನು ಕಾಣಬೇಕೆಂಬ ಚಪಲ ಹುಟ್ಟಿತು. ಆತನು ಪರಶಿವೆಯೊಡನೆ ನಂದಿಯನ್ನೇರಿ ಶ್ರೀಹರಿಯ ಬಳಿಗೆ ಬಂದನು. ಜಗತ್ತಿನ ತಾಯ್ತಂದೆಗಳಂತಿರುವ ಶಿವ-ಶಿವೆಯರು ಮನೆಯ ಬಾಗಿಲಿಗೆ ಬಂದುದನ್ನು ಕಂಡು ಶ್ರೀಹರಿಗೆ ಪರಮಸಂತೋಷವಾಯಿತು. ಆತನು ಅವರನ್ನು ಆದರದಿಂದ ಬರಮಾಡಿ ಕೊಂಡು, ಭಕ್ತಿಯಿಂದ ನಮಸ್ಕರಿಸಿದನು. ಪರಮೇಶ್ವರನೂ ಆತನಿಗೆ ಪ್ರತಿ ನಮಸ್ಕಾರ ಮಾಡಿ ‘ದೇವದೇವ, ನಿನ್ನ ಲೀಲೆಯೇ ಅರ್ಥವಾಗುವುದಿಲ್ಲ. ನೀನು ನಿರ್ಗುಣ, ನಿರ್ವಿಕಾರ, ಆನಂದಮಯ; ಆದರೂ ಜಗತ್ತಿನ ಸೃಷ್ಟಿ ಸ್ಥಿತಿ ಲಯಗಳಿಗೆ ಕಾರಣನಾಗಿರುವೆ; ನಿರಾಕಾರ, ಆದರೂ ಭಕ್ತರ ಉದ್ಧಾರಕ್ಕಾಗಿ ಅನೇಕ ಅವತಾರಗಳನ್ನು ಎತ್ತುತ್ತಿ. ಮೊನ್ನೆತಾನೆ ದೇವತೆ ಗಳ ಉದ್ಧಾರಕ್ಕಾಗಿ ಮೋಹಿನಿಯ ಅವತಾರವನ್ನು ಎತ್ತಿದ್ದೆಯಂತೆ! ನಾನು ನಿನ್ನ ಎಲ್ಲ ಅವತಾರಗಳನ್ನೂ ನೋಡಿದ್ದೇನೆ; ಮೋಹಿನಿಯ ಅವತಾರವನ್ನು ಮಾತ್ರ ನೋಡಲಿಲ್ಲ. ಅದನ್ನೂ ನೋಡಬೇಕೆಂದು ನನಗೆ ಆಶೆ’ ಎಂದು ತಿಳಿಸಿದನು. 

ಪರಶಿವನ ನುಡಿಗಳನ್ನು ಕೇಳಿ ಶ್ರೀಹರಿ ಒಮ್ಮೆ ಮುಗುಳ್ನಗೆ ನಕ್ಕನು. ಮರು ನಿಮಿಷ ದಲ್ಲಿ ಆತ ಅಲ್ಲಿರಲಿಲ್ಲ, ಮಾಯಾವಾಗಿ ಹೋಗಿದ್ದ. ಪರಶಿವನು ಕಕ್ಕಾಬಿಕ್ಕಿಯಾಗಿ ಸುತ್ತಲೂ ನೋಡಿದನು. ಆತನ ಇದಿರಿಗಿದ್ದ ಉದ್ಯಾನವನದಲ್ಲಿ, ಹಚ್ಚಹಸುರಾದ ಮರಗಿಡ ಗಳ ಮಧ್ಯೆ ದಿವ್ಯಸುಂದರಿಯಾದ ಹೆಣ್ಣೊಬ್ಬಳು ಕಾಣಿಸಿದಳು. ಅವಳು ಹೊಸ ಪೀತಾಂಬರ ವೊಂದನ್ನು ಉಟ್ಟಿದ್ದಾಳೆ, ನಡುವಿನಲ್ಲಿ ರತ್ನ ಖಚಿತವಾದ ಒಡ್ಯಾಣ, ಆಕೆಯು ಪುಟ ಚೆಂಡಿನ ಆಟದಲ್ಲಿ ಮಗ್ನಳಾಗಿದ್ದಾಳೆ. ಇದರಿಂದ ಆಕೆ ಮತ್ತೆ ಮತ್ತೆ ಬಗ್ಗಿ ಏಳುತ್ತಿದ್ದಾಳೆ. ಆಗ ಆಕೆಯ ಮುತ್ತಿನ ಹಾರ ಎದೆಯ ಮೇಲೆ ತೊನೆದಾಡುತ್ತಿದೆ. ಚೆಂಡು ಹೋದತ್ತ ಅವಳ ಚೆಲ್ಲೆಗಂಗಳು ಚಲಿಸುತ್ತಿದೆ, ಕಿವಿಯ ಓಲೆಗಳು ಕುಣಿಯುತ್ತಿವೆ. ಆಟದಲ್ಲಿ ಮುಳುಗಿ ಹೋಗಿರುವ ಆ ಹೆಣ್ಣಿನ ಸೆರಗು ಜಾರುತ್ತಿದೆ, ಕೂದಲ ರಾಶಿ ಸೋರ್ಮುಡಿಯಾಗಿ ನೆಲ ವನ್ನು ಮುಟ್ಟುತ್ತಿದೆ. ಆಕೆ ಸೆರಗನ್ನೂ ಕೂದಲನ್ನೂ ಎಡಗೈಯಿಂದ ಹಿಂದಕ್ಕೆಳೆದು ಕೊಳ್ಳುತ್ತಾ ಚಲಿಸುವ ಮಿಂಚಿನ ಬಳ್ಳಿಯಂತೆ ಚೆಂಡಿನ ಹಿಂದೆ ಓಡುತ್ತಿದ್ದಾಳೆ. ಮುಗುಳು ನಗೆಯಿಂದ ಕೂಡಿದ ಅವಳ ಮುಖ, ಜಿಂಕೆಯ ಕಣ್ಣಿನಂತೆ ಚಂಚಲವಾದ ಅವಳ ಕಣ್ಣಿನ ಕುಡಿನೋಟ–ಇವುಗಳನ್ನು ಕಂಡ ಪರಶಿವನಿಗೆ ತಡೆಯಲಾರದಷ್ಟು ಮೋಹ ಉಕ್ಕಿತು. ತಾನು ಎಲ್ಲಿರುವೆನೆಂಬುದರ ಅರಿವೇ ಆತನಿಂದ ಹಾರಿಹೋಯಿತು. ಪಕ್ಕದಲ್ಲಿ ಪರಶಿವೆ ಕುಳಿತಿರುವುದನ್ನೂ ಆತ ಮರೆತುಬಿಟ್ಟ. ಆತನ ಕಾಲುಗಳು ಆ ಹೆಣ್ಣಿನತ್ತ ಸಾಗಿದವು. ಅಷ್ಟರಲ್ಲಿ ಆಟದ ಸಡಗರದಲ್ಲಿದ್ದ ಆ ಹೆಣ್ಣಿನ ಸೀರೆ ಬಿಚ್ಚಿಹೋಯಿತು. ಆ ಹೆಣ್ಣು ನಾಚಿಕೆ ಯಿಂದ ತನ್ನ ಮೈಯನ್ನು ಹುದುಗಿಸಿಕೊಂಡು, ಸುತ್ತಲೂ ನೋಡಿದಳು. ಕಾಮದ ಹಸಿವಿ ನಿಂದ ತನ್ನನ್ನು ನುಂಗುವುದಕ್ಕೆಂಬಂತೆ ಬರುತ್ತಿದ್ದ ಪರಶಿವನನ್ನು ಕಾಣುತ್ತಲೆ, ಆಕೆ ನಾಚಿಕೆ ಭಯಗಳಿಂದ, ಬಂಗಾರದ ಬಳ್ಳಿಯಂತಿರುವ ತನ್ನ ದೇಹವನ್ನು ಗಿಡಬಳ್ಳಿಗಳ ಮಧ್ಯದಲ್ಲಿ ಅವಿತಿಟ್ಟಳು. 

ಸ್ವಪ್ನಸುಂದರಿಯಂತಿದ್ದ ಹೆಣ್ಣು ಕಣ್ಮರೆಯಾಗುತ್ತಲೆ ಪರಶಿವನ ಹಂಬಲು ಹುಚ್ಚಾ ಯಿತು. ಆತ ಮದ್ದಾನೆಯಂತೆ ಗಿಡಗಳ ಮಧ್ಯೆ ನುಗ್ಗಿಹೋದನು. ಇದನ್ನು ಕಂಡು ಆ ಹೆಣ್ಣು ಕಿಲಿಕಿಲಿ ನಗುತ್ತಾ ಓಡುವುದಕ್ಕೆ ಮೊದಲುಮಾಡಿದಳು. ಆದರೆ ಶಿವನು ಒಮ್ಮೆಗೆ ಅವಳ ಬಳಿಗೆ ಹಾರಿ, ನೀಳವಾದ ಅವಳ ಮುಡಿಯನ್ನು ಹಿಡಿದು, ಅವಳನ್ನು ತನ್ನ ಬಳಿಗೆ ಸೆಳೆದುಕೊಂಡನು. ಆದರೆ ಅವಳ ನುಣುಪಾದ ದೇಹ ಆತನ ಆಲಿಂಗನಕ್ಕೆ ಸಿಕ್ಕದೆ ನುಣುಚಿ ಕೊಂಡಿತು. ಅವಳು ಮತ್ತೆ ಓಡತೊಡಗಿದಳು. ಕೈಗೆ ಬಂದ ತುತ್ತು ಬಾಯಿಗೆ ಬರದಂತಾ ದುದರಿಂದ ಕಳವಳಗೊಂಡ ಪರಶಿವನು ಆ ಹೆಣ್ಣನ್ನು ಅಟ್ಟಿಸಿಕೊಂಡು ನದಿ, ಕೊಳ, ಬೆಟ್ಟ, ವನಗಳಲ್ಲೆಲ್ಲ ತಿರುಗುತ್ತಾ ಕಡೆಗೆ ಒಂದು ಪುಷ್ಯಾಶ್ರಮಕ್ಕೆ ಬಂದನು. ಅಲ್ಲಿದ್ದ ಪುಷಿ ಗಳನ್ನು ಕಾಣುತ್ತಲೆ ಪರಶಿವನ ಕಾಮವಿಕಾರ ಅಡಗಿತು. ಆತನು ತನ್ನ ಮನಸ್ಸಿನಲ್ಲಿ ‘ಆಹಾ!ಇದು ವಿಷ್ಣು ಮಾಯೆ. ಆತನ ಮೋಹಿನೀರೂಪವನ್ನು ಕಂಡು ನಾನು ಮರುಳಾದೆ!’ ಎಂದುಕೊಂಡನು. ಆಗ ಶ್ರೀಹರಿ ತನ್ನ ಸ್ವಸ್ವರೂಪದಿಂದ ಆತನ ಇದಿರಿಗೆ ನಿಂತು ‘ಹೇ ದೇವದೇವ, ನನ್ನ ಮಾಯೆಗೆ ಒಳಗಾದರೂ ಆದರಿಂದ ನೀನಾಗಿಯೇ ಪಾರಾದುದು ಮಹಾ ಮಹಿಮನಾದ ನಿನಗೆ ಮಾತ್ರ ಸಾಧ್ಯ. ಸೃಷ್ಟಿಯೇ ಇತ್ಯಾದಿ ಕಾರ್ಯಗಳಲ್ಲಿ ಕಾಲರೂಪದಿಂದ ನನ್ನ ಜೊತೆಯಲ್ಲಿಯೇ ಇರುವ ಈ ಮಾಯೆ ನಿನ್ನನ್ನು ಮಾತ್ರ ಮುಸುಕಲಾರದು’ ಎಂದು ಹೇಳಿದನು. 

ಅನಂತರ ಮಹಾದೇವನು ವಿಷ್ಣುವಿನಿಂದ ಬೀಳ್ಕೊಂಡು ಗಿರಿಜೆಯೊಡನೆ ಕೈಲಾಸಕ್ಕೆ ಹಿಂದಿರುಗಿದನು. ಹಿಂದಿರುಗುತ್ತಾ ಆತ ಮನಸ್ಸಿನಲ್ಲಿ ‘ನನಗೇ ಇಂತಹ ಮಂಕು ಮುಸುಕಿ ಕೊಂಡಿತೆಂದಮೇಲೆ, ಪಾಪ, ರಾಕ್ಷಸರಿಗೆ ಅದು ಎಂತಹ ಮಂಕುಬೂದಿಯಾಗಿರಬೇಕು!’ ಎಂದುಕೊಂಡ.

