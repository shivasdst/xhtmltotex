
\chapter{೧೮. ಪುರಂಜನರಾಜ}

ಪ್ರಚೇತಸರನ್ನು ತಪಸ್ಸಿಗೆ ಕಳುಹಿಸಿದ ಪ್ರಾಚೀನಬರ್ಹಿ ಎಂದಿನಂತೆ ತನ್ನ ಯಾಗ ಕಾರ್ಯಕ್ಕೆ ಕೈಹಾಕಿದ. ಇದನ್ನು ಕಂಡು ನಾರದರಿಗೆ ‘ಅಯ್ಯೋ’ ಎನಿಸಿತು. ತನ್ನ ಶಿಷ್ಯನಾದ ಧ್ರುವನ ವಂಶದಲ್ಲಿ ಹುಟ್ಟಿದ ಈ ರಾಜ ಕೇವಲ ಕಾಮ್ಯಕರ್ಮಗಳಲ್ಲಿಯೇ ಮಗ್ನನಾಗಿರುವ ನಲ್ಲಾ ಎಂದುಕೊಂಡು, ಆತನು ನೇರವಾಗಿ ಆ ರಾಜನ ಬಳಿಗೆ ಬಂದು ‘ಅಯ್ಯಾ, ದಿನ ದಿನವೂ ಈ ಯಾಗಗಳನ್ನು ಮಾಡುತ್ತಿರುವೆಯಲ್ಲಾ, ಇದರಿಂದ ನಿನಗೇನು ಪ್ರಯೋಜನ? ಫಲಪ್ರಾಪ್ತಿಗಳಲ್ಲಿ ದುಃಖನಿವೃತ್ತಿ, ಸುಖಪ್ರಾಪ್ತಿ ಎಂದು ಎರಡು ಬಗೆ. ನಿನ್ನ ಯಾಗದಿಂದ ಇವೆರಡರಲ್ಲಿ ಯಾವುದೂ ನಿನಗೆ ಸಿಕ್ಕುವುದಿಲ್ಲ. ಕೇವಲ ಪಾಪವನ್ನು ಗಳಿಸಿಕೊಳ್ಳುತ್ತಿರುವೆ. ನಿನಗೆ ದೊರೆವ ಫಲವೆಂತಹುದೆಂಬುದನ್ನು ನನ್ನ ಯೋಗಶಕ್ತಿಯಿಂದ ನಿನಗೆ ತೋರಿಸು ತ್ತೇನೆ ನೋಡು! ಯಾಗಗಳಲ್ಲಿ ನೀನು ಕೊಂದ ಪಶುಗಳೆಲ್ಲ ಯಮಲೋಕದಲ್ಲಿ ಹಿಂಡು ಹಿಂಡಾಗಿ ನಿಂತು, ನಿನ್ನ ಸಾವನ್ನೇ ನಿರೀಕ್ಷಿಸುತ್ತಿವೆ. ನೀನು ಸತ್ತು ಯಮಲೋಕಕ್ಕೆ ಹೋಗುವುದೇ ತಡ, ಅವು ಉಕ್ಕಿನಂತಿರುವ ತಮ್ಮ ಕೊಂಬುಗಳಿಂದ ನಿನ್ನನ್ನು ತಿವಿದು ಹಿಂಸಿಸಲು ಹೊಂಚುಹಾಕುತ್ತಿವೆ’ ಎಂದನು.

ನಾರದರ ನುಡಿಗಳನ್ನು ಕೇಳಿ, ತನಗೆ ಕಾದಿರುವ ವಿಪತ್ತನ್ನು ಪ್ರತ್ಯಕ್ಷವಾಗಿ ಕಂಡು, ಪ್ರಾಚೀನ ಬರ್ಹಿಯ ಜೀವ ತಲ್ಲಣಿಸಿತು. ಆತನು ‘ಸ್ವಾಮಿ, ಸಂಸಾರದ ಬಲೆಗೆ ಸಿಕ್ಕಿ ಒದ್ದಾಡುತ್ತಿರುವ ನನಗೆ ಗೊತ್ತಿರುವುದು ಕಾಮ್ಯ ಕರ್ಮವೊಂದೆ. ಯಾವುದು ನನಗೆ ಶ್ರೇಯಸ್ಸೋ ನನಗೆ ಗೊತ್ತಿಲ್ಲ. ಮಹಾತ್ಮರಾದ ನೀವು ನನಗೆ ದಾರಿತೋರಬೇಕು’ ಎಂದು ಕೇಳಿಕೊಂಡನು. ನಾರದರು ಒಂದು ಕಥೆಯ ಮೂಲಕ ಅವನಿಗೆ ಉಪದೇಶ ಮಾಡಲು ಪ್ರಾರಂಭಿಸಿದರು:

‘ಅಯ್ಯಾ, ರಾಜ! ಪೂರ್ವಕಾಲದಲ್ಲಿ ಪುರಂಜನನೆಂಬ ಒಬ್ಬ ರಾಜನಿದ್ದ. ಆತನಿಗೆ ಭೋಗ ಜೀವನದಲ್ಲಿ ಬಹು ಆಶೆ. ಅದಕ್ಕೆ ಅನುಕೂಲವಾದಂತಹ ಒಂದು ಸ್ಥಳವನ್ನು ಹುಡುಕುತ್ತಾ ಭೂಮಂಡಲದಲ್ಲೆಲ್ಲ ಅಲೆಯುತ್ತಿರಲು, ಹಿಮಾಲಯಪರ್ವತದ ದಕ್ಷಿಣ ದಲ್ಲಿ ಒಂದು ಸುಂದರವಾದ ನಗರ ಕಣ್ಣಿಗೆ ಬಿತ್ತು. ಒಂಬತ್ತು ಹೆಬ್ಬಾಗಿಲುಗಳುಳ್ಳ ಆ ಪಟ್ಟಣಕ್ಕೆ ಸುತ್ತಲೂ ಎತ್ತರವಾದ ಕೋಟೆ; ಅದರ ಸುತ್ತಲೂ ಆಳವಾದ ಕಂದಕ. ಕೋಟೆಯ ಒಳಗಡೆ, ದಿವ್ಯವಾದ ಮಹಡಿಮನೆಗಳು ರತ್ನ ಖಚಿತವಾದ ಗೋಪುರಗಳೊಡನೆ ಹೊಳೆಯುತ್ತಿದ್ದವು. ರಾಜಬೀದಿಗಳ ಇಕ್ಕೆಲಗಳಲ್ಲಿಯೂ ಸಾಲು ಸಾಲಾಗಿರುವ ಈ ಮನೆ ಗಳ ಮುಂದೆ ಹವಳದ ಜಗುಲಿಗಳು, ಅಲ್ಲಲ್ಲೆ ಅಂಗಡಿಗಳ ಸಾಲುಗಳು, ಆಟದ ಮೈದಾನ ಗಳು, ಧ್ವಜಪತಾಕೆಗಳು. ಪಟ್ಟಣದ ಮಧ್ಯಭಾಗದಲ್ಲಿ ಸೊಗಸಾದ ಸಭಾಮಂಟಪ, ಅರ ಮನೆ. ಅದರ ಸುತ್ತಲೂ ಕಣ್ಣಿಗೆ ಇಂಪನ್ನೂ ಹೃದಯಕ್ಕೆ ತಂಪನ್ನೂ ನೀಡುವ ಒಂದು ಉದ್ಯಾನವನ. ಅದರ ಒಂದು ಭಾಗದಲ್ಲಿ ಕನ್ನಡಿಯಂತೆ ತಿಳಿನೀರನ್ನು ಒಳಗೊಂಡ ಒಂದು ಸರೋವರ, ಬಗೆಬಗೆಯ ಬಣ್ಣದ ಹೂಗಳಿಂದ ಕಣ್ಣಿಗೆ ಹಬ್ಬವನ್ನು ಮಾಡುವ ಹೂಗಿಡ ಗಳು, ಕಿವಿಗೆ ಅಮೃತವನ್ನು ಸುರಿಯುವ ಹಕ್ಕಿಗಳ ಗಾನ, ತಣ್ಣಗೆ, ಸುವಾಸನೆಯನ್ನು ಹೊತ್ತು ಮೆಲ್ಲಗೆ ಬೀಸುತ್ತಿರುವ ಗಾಳಿ. ಅದು ನಾಗರಾಜನ ಭೋಗವತಿಯಂತೆ ರಮಣೀಯವಾಗಿತ್ತು’

ಪುರಂಜನನು ಆ ಉದ್ಯಾನದ ಸೌಖ್ಯವನ್ನು ಪಂಚೇಂದ್ರಿಯಗಳಿಂದಲೂ ಸವಿಯುತ್ತಾ ಬರುತ್ತಿರಲು, ಈ ಸುಖಗಳಿಗೆಲ್ಲ ಕಲಶದಂತಿರುವ ಕನ್ಯಾಮಣಿಯೊಬ್ಬಳು ಆತನಿಗೆ ಕಾಣಿಸಿ ದಳು. ಆಕೆಯ ಸುತ್ತ ಹತ್ತು ಜನ ಅಂಗರಕ್ಷಕರು, ಅವರೊಬ್ಬಬ್ಬರಿಗೂ ನೂರಾರು ಜನ ದಾಸಿಯರು; ಆ ಕನ್ಯೆಯ ಮುಂದೆ ಐದು ಹೆಡೆಗಳ ಸರ್ಪವೊಂದು ಆಕೆಯ ರಕ್ಷಣೆಯಲ್ಲಿ ತತ್ಪರವಾಗಿದೆ. ಆ ಹೆಣ್ಣಿನ ಚೆಲುವು ಬಣ್ಣನೆಗೆ ಮೀರಿದುದು. ಸಂಪಗೆಯಂತೆ ಎಸಳಾದ ಆ ಮೂಗೊ! ಕನ್ನಡಿಯಂತೆ ನುಣುಪಾದ ಆ ಕೆನ್ನೆಗಳೊ! ಬೆಳಕನ್ನು ಚೆಲ್ಲುವ ಕಣ್ಣು, ಕೆಂಪನ್ನು ಉಗುಳುವ ಆ ತುಟಿಗಳು, ರತ್ನಕುಂಡಲಗಳನ್ನು ಧರಿಸಿದ ಆ ಕಿವಿಗಳು, ಕೊಂಕಾದ ಆ ಹುಬ್ಬು, ಸುರುಳಿಸುರುಳಿಯಾದ ಆ ಮುಂಗುರುಳು, ನಸು ನಾಚಿಕೆಯನ್ನು ಹೊತ್ತ ಮುಗುಳ್ನಗೆಯ ಆ ಮುದ್ದು ಮುಖ; ಬಂಗಾರದ ಬಣ್ಣದ ಸೀರೆಯುಟ್ಟು, ಚಿನ್ನದ ಡಾಬನ್ನು ಧರಿಸಿ, ಉಬ್ಬಿದ ಎದೆಯ ಸೆರಗನ್ನು ಒಯ್ಯಾರದಿಂದ ಸರಿಪಡಿಸಿಕೊಳ್ಳುತ್ತಾ, ಮಂದಗಮನದಿಂದ ಬರುತ್ತಿದ್ದ ಆಕೆ ಒಮ್ಮೆ ರಾಜನತ್ತ ಕಡೆಗಣ್ಣಿನ ನೋಟವನ್ನು ಎಸೆದಳು. ಇದರಿಂದ ರಾಜನ ಹೃದಯ ಸೂರೆಹೋಯಿತು. ಆತನು ತನ್ನ ಮನಸ್ಸನ್ನು ಬಿಗಿಹಿಡಿಯಲಾರದೆ ಬಾಯಿಬಿಟ್ಟು ಕೇಳಿಯೇಬಿಟ್ಟ: ‘ಮನೋಹರಿ, ನೀನಾರು? ಎಲ್ಲಿಂದ ಬಂದೆ? ಏಕಾಂಗಿಯಾಗಿ ಇಲ್ಲೇಕೆ ಸುಳಿದಾಡುತ್ತಿರುವೆ? ನಿನ್ನ ಈ ಅಂಗರಕ್ಷಕರು, ನಿನಗೆ ಕಾವಲಾಗಿ ನಿನ್ನ ಮುಂದೆ ನಿಂತಿರುವ ಈ ಸರ್ಪರಾಜ–ಇವುಗಳ ಅರ್ಥವೇನು? ನಿನ್ನ ಈ ಚೆಲುವು ಮಾನವರಿಗೆಲ್ಲಿಯದು? ನೀನು ಭೂದೇವಿಯೊ, ಪಾರ್ವತಿಯೊ, ಲಕ್ಷ್ಮಿಯೊ! ಆದರೆ ನಿನ್ನ ಪಾದಗಳು ಭೂಮಿಯನ್ನು ಸೋಕುತ್ತಿರುವುದರಿಂದ ನೀನು ದೇವತೆಯಲ್ಲ, ಮಾನವಳೇ ಇರಬೇಕು. ಹಾಗಾದರೆ ವೈಕುಂಠದಲ್ಲಿ ನಾರಾಯಣನೊಡನೆ ನೆಲಸಿರುವ ಲಕ್ಷ್ಮಿ ಯಂತೆ ನೀನು ನನ್ನೊಡನೆ ಈ ನಗರದಲ್ಲಿ ಏಕೆ ನೆಲಸಿರಬಾರದು? ನನ್ನ ಮನಸ್ಸು ನಿನ್ನ ರೂಪಕ್ಕೆ ಮರುಳಾಗಿದೆ. ಮನ್ಮಥನ ಬಾಣಕ್ಕೆ ನಿನ್ನ ಮುಗುಳ್ನಗೆ ಮತ್ತು ಕಡೆಗಣ್ ನೋಟ ಗಳು ಸಹಾಯಕವಾಗಿ ನನ್ನನ್ನು ಅನಾಥನನ್ನಾಗಿ ಮಾಡಿವೆ. ನಾನು ನಿನಗೆ ಶರಣಾಗಿದ್ದೇನೆ. ಅನುಗ್ರಹಿಸು.’

ರಾಜನ ಮನಸ್ಸನ್ನು ಕೊಳ್ಳೆಹೊಡೆಯುವ ಲಜ್ಜೆಯನ್ನು ನಟಿಸುತ್ತಾ ಆ ರಮಣಿ ಉತ್ತರ ಕೊಟ್ಟಳು–‘ಮಹಾರಾಜ, ನನಗಾಗಲಿ ನಿಮಗಾಗಲಿ ತಂದೆತಾಯಿಗಳಾರೊ ಗೊತ್ತಿಲ್ಲ, ನಮ್ಮ ಹೆಸರೂ ಗೊತ್ತಿಲ್ಲ, ಕುಲವೂ ಗೊತ್ತಿಲ್ಲ; ನಾವಿಬ್ಬರೂ ಆಶ್ರಯಿಸಬೇಕೆಂದು ನೀವು ಸೂಚಿಸುತ್ತಿರುವ ಈ ನಗರವಾವುದೋ, ಆರು ಕಟ್ಟಿದರೊ ನಾವರಿಯೆವು. ನನ್ನ ಜೊತೆ ಗಿರುವ ಈ ಅಂಗರಕ್ಷಕರೂ ಅವರ ದಾಸಿಯರು ನನ್ನ ಗೆಳೆಯರು. ಈ ಸರ್ಪರಾಜನು ನಾನು ಮಲಗಿರುವಾಗ ಈ ಪಟ್ಟಣವನ್ನು ಕಾಪಾಡುತ್ತಿರುವನು. ಅದೃಷ್ಟವಶದಿಂದ ನೀವು ಇಲ್ಲಿಗೆ ಬಂದಿರುವಿರಿ. ನಿಮ್ಮ ಆಶೆ ಭೋಗಭಾಗ್ಯ. ನಾನು ನನ್ನ ಪರಿವಾರದೊಡನೆ ನಿಮಗೆ ಆ ಸುಖವನ್ನು ಅಗತ್ಯವಾಗಿಯೂ ಒದಗಿಸಿಕೊಡುತ್ತೇನೆ. ನೀವು ನಿಮ್ಮ ಇಷ್ಟಾರ್ಥಸಿದ್ಧಿಯನ್ನು ಪಡೆದು ನೂರುವರ್ಷಗಳವರೆಗೆ ಇಲ್ಲಿ ಸುಖದಿಂದ ಇರಬಹುದು. ನಿಮ್ಮಂತಹ ರಸಿಕರು ನನ್ನನ್ನು ವರಿಸುವೆನೆಂದಾಗ, ಅದನ್ನು ನಿರಾಕರಿಸಿ, ಪಶುಗಳಂತೆ ಜಡರಾದ ವಿರಕ್ತರನ್ನು ವರಿಸಲೆ? ಮಹಾರಾಜ, ಗೃಹಸ್ಥಾಶ್ರಮವು ಆಶ್ರಮಗಳಲ್ಲೆಲ್ಲ ಅತ್ಯಂತ ಶ್ರೇಷ್ಠವಾದುದು. ಧರ್ಮ, ಅರ್ಥ, ಕಾಮ, ಮೋಕ್ಷಗಳೂ, ಇಹದಲ್ಲಿ ಕೀರ್ತಿ ಪರದಲ್ಲಿ ಪುಣ್ಯಲೋಕವೂ ದೊರೆಯಬೇಕಾದರೆ ಗೃಹಸ್ಥಾಶ್ರಮವನ್ನೇ ಆಶ್ರಯಿಸಬೇಕು. ನಿಮ್ಮಂತಹ ಶೂರ ಧೀರ ರಾದ ಸುಂದರಪುರುಷರು ಪತಿಯಾಗಿ ದೊರೆವುದು ನನ್ನ ಭಾಗ್ಯ. ಸರ್ಪಗಳಂತಿರುವ ನಿಮ್ಮ ತೋಳ ಬಂಧನದಲ್ಲಿ ನಾನು ಧನ್ಯಳಾಗುವೆನು.’

ಪುರಂಜನನು ಆ ಸುಂದರಿಯೊಂದಿಗೆ ಸ್ವರ್ಗಸುಖವನ್ನು ಸೂರೆಗೊಳ್ಳುತ್ತಾ, ವರ್ಷ ಗಳನ್ನು ಕ್ಷಣವೆಂಬಂತೆ ಕಳೆದನು. ಆ ಕಾಲದಲ್ಲಿ ಆತನಿಗೆ ಆ ಹೆಣ್ಣೇ ಜೀವಸರ್ವಸ್ವವಾಗಿ ದ್ದಳು. ಅವನು ಅವಳ ಪಡಿನೆರಳಂತೆ ವ್ಯವಹರಿಸುತ್ತಿದ್ದನು. ಅವಳು ನಿಂತರೆ ನಿಲ್ಲುವನು, ಕೂತರೆ ಕೂಡುವನು, ಮಲಗಿದರೆ ಮಲಗುವನು; ಅವಳು ತಿಂದರೆ ತಿನ್ನುವನು, ಕುಡಿದರೆ ಕುಡಿಯುವನು; ನಕ್ಕರೆ ನಗುವನು, ಅತ್ತರೆ ಅಳುವನು; ಅವಳ ರುಚಿ ಇವನ ಸವಿ, ಅವಳ ಮಾತು ಇವನ ನುಡಿ; ಅವನು ವ್ಯಕ್ತಿತ್ವವಿಲ್ಲದವನಂತೆ ಅವಳ ಕೈಗೊಂಬೆಯಾಗಿದ್ದನು. ಹೀಗೆ ಹಲವು ವರ್ಷಗಳು ಉರುಳಿಹೋದವು.

ಪುರಂಜನನು ಒಂದು ದಿನ ಕೈಲಿ ದೊಡ್ಡ ಬಿಲ್ಲನ್ನು ಹಿಡಿದು ಬೇಟೆಗೆ ಹೊರಟನು. ಅವನ ರಥ ಬಹು ವಿಚಿತ್ರವಾದುದು. ಎರಡು ನೊಗ ಅದಕ್ಕೆ; ಎರಡು ಗಾಲಿ, ಒಂದು ಅಚ್ಚು, ಆದರೆ ಮೂರು ಧ್ವಜಗಳು; ಐದು ಕಟ್ಟುಗಳು; ಅದಕ್ಕೆ ಹೂಡಿದ್ದುದು ಐದು ಕುದುರೆಗಳು, ಆದರೆ ಒಂದೇ ಲಗಾಮು ಮತ್ತು ಒಬ್ಬನೇ ಸಾರಥಿ; ಆ ರಥಕ್ಕೆ ಏಳು ಹೊದಿಕೆ ಗಳು, ಅದರಲ್ಲಿ ಐದು ಬಗೆಯ ಆಯುಧಗಳು; ಆ ರಥದ ನಡಗೆಯಲ್ಲಿಯೂ ಐದು ವಿಧ. ರಾಜನು ಬಂಗಾರದ ಕವಚವನ್ನು ತೊಟ್ಟು, ಕೊನೆಯಿಲ್ಲದ ಬಾಣಗಳಿಂದ ತುಂಬಿದ ಮೂರು ಬತ್ತಳಿಕೆಗಳನ್ನು ಬೆನ್ನಿಗೆ ಕಟ್ಟಿಕೊಂಡು ರಥದಲ್ಲಿ ಮಂಡಿಸಿದ. ಆತನೊಡನೆ ಹನ್ನೊಂದು ಜನ ಸೇನಾನಾಯಕರು ಹೊರಟರು. ಎಲ್ಲರೂ ಪಂಚಪ್ರಸ್ಥವೆಂಬ ಅರಣ್ಯಕ್ಕೆ ಹೋಗಿ, ಬೇಟೆಯಾಡಿದರು. ರಾಜನಂತೂ ಬೇಟೆಯಲ್ಲಿ ಮುಳುಗಿ, ತನ್ನ ಜೀವಕ್ಕಿಂತಲೂ ಮಿಗಿಲಾದ ಹೆಂಡತಿಯನ್ನೂ ಕೂಡ ಮರೆತ. ಅನೇಕ ಪ್ರಾಣಿಗಳು ಆತನ ಬಾಣಕ್ಕೆ ತುತ್ತಾ ದವು. ಬೇಟೆಯಾಡಿ ಆಡಿ, ಮೃಗಗಳನ್ನು ಕೊಂದು ಕೊಂದು, ಆತನು ಬಳಲಿಹೋದ.

ಬೇಟೆಯಿಂದ ಬಳಲಿ ಬಾಯಾರಿಕೆಯಾದಾಗ ಅವನಿಗೆ ಅರಮನೆಯ ನೆನಪಾಯಿತು. ಅವನು ರಾಜಧಾನಿಗೆ ಹಿಂದುರುಗಿದ. ಸ್ನಾನಮಾಡಿ, ಬಳಲಿಕೆಯನ್ನು ಕಳೆದುಕೊಂಡು, ಊಟಮಾಡಿ, ವಿಶ್ರಾಂತಿ ಪಡೆದಮೇಲೆ ಗಂಧ ಪುಷ್ಪ ತಾಂಬೂಲಗಳನ್ನು ಸ್ವೀಕರಿಸುತ್ತಿರು ವಾಗ ಅವನಿಗೆ ಕಾಮೋದ್ರೇಕವಾಗಿ ಹೆಂಡತಿಯ ನೆನಪು ಬಂತು. ಆತ ನೇರವಾಗಿ ಮಲಗುವ ಮನೆಗೆ ಹೋದ. ಆದರೆ ಅಲ್ಲಿ ಆತನ ಮಡದಿ ಇರಲಿಲ್ಲ. ಅವನಿಗೆ ಕಳವಳವಾಯಿತು. ರಾಣಿವಾಸದವರನ್ನು ವಿಚಾರಿಸಿದ. ಅವರು ಅರಮನೆಯ ಒಂದು ಕಡೆ ದುಃಖದಿಂದ ನೆಲದ ಮೇಲೆ ಮಲಗಿದ್ದ ರಾಣಿಯನ್ನು ತೋರಿಸಿದರು. ರಾಜನ ಜೀವ ತತ್ತರಿಸಿತು. ಆತನು ಆಕೆಯ ಬಳಿಗೆ ಓಡಿಹೋಗಿ, ಪಕ್ಕದಲ್ಲಿ ಕುಳಿತು, ಮೃದು ಮಧುರವಾಗಿ ಆಕೆಯ ದುಃಖಕ್ಕೆ ಕಾರಣ ವನ್ನು ಕೇಳಿದನು. ಆದರೆ ಆಕೆ ಮೌನಧಾರಿಯಾಗಿದ್ದಳು. ರಾಜನು ಆಕೆಯ ಕಾಲನ್ನು ಹಿಡಿದು ಬೇಡಿಕೊಂಡ. ಅಲ್ಲಿಗೂ ಉತ್ತರ ಬರದಿರಲು, ಆಕೆಯನ್ನು ಎತ್ತಿ ತೊಡೆಯ ಮೇಲೆ ಕೂಡಿಸಿಕೊಂಡು ‘ದೇವಿ! ನೀನು ಒಡತಿ, ನಾನು ದಾಸ. ನಾನು ತಪ್ಪು ಮಾಡಿದಾಗ ದಂಡಿಸುವ ಅಧಿಕಾರ ನಿನ್ನದು. ನಾನು ಕ್ಷಮಿಸೆಂದು ಕೇಳುವುದಿಲ್ಲ, ಅಗತ್ಯವಾಗಿಯೂ ದಂಡಿಸು. ಆದರೆ ಒಮ್ಮೆ ನಿನ್ನ ಸುಂದರವಾದ ಮುಖವನ್ನು ನನ್ನತ್ತ ತಿರುಗಿಸು, ನಿನ್ನ ಮಧುರವಾದ ದನಿಯಿಂದ ಒಮ್ಮೆ ಮಾತಾಡು. ನಾನು ನಿನ್ನನ್ನು ಕೇಳದೆ ಬೇಟೆಗೆ ಹೋದೆ ನೆಂಬುದೇ ನಿನ್ನ ಕೋಪಕ್ಕೆ ಕಾರಣವಾದರೆ, ಇಗೋ ನಾನು ಕ್ಷಮಿಸುವಂತೆ ನಿನ್ನನ್ನು ಬೇಡು ತ್ತಿದ್ದೇನೆ. ಇನ್ನು ಮೇಲೆ ಎಂದೆಂದೂ ನಿನ್ನ ಅಪ್ಪಣೆಯಿಲ್ಲದೆ ಯಾವ ಕಾರ್ಯವನ್ನೂ ಮಾಡು ವುದಿಲ್ಲ. ಸದಾ ನಿನ್ನ ಸೇವೆಯಲ್ಲಿಯೇ ತೊಡಗಿರುವೆನು’ ಎಂದನು.

ರಾಜನ ಯಾಚನೆ ರಾಣಿಯ ಕೋಪವನ್ನು ತಗ್ಗಿಸಿತು. ಆಕೆ ಮೇಲಕ್ಕೆದ್ದು, ಮಂಗಳಸ್ನಾನ ಮಾಡಿ, ಪಟ್ಟೆವಸ್ತ್ರಗಳನ್ನು ಧರಿಸಿ, ಆಭರಣಗಳನ್ನು ತೊಟ್ಟು, ಊಟ ಮಾಡಿ, ಸಿಂಗರದ ಬೊಂಬೆಯಂತೆ ರಾಜನ ಬಳಿಗೆ ಬಂದಳು. ಆಕೆಯನ್ನು ಕಂಡು ರಾಜನಿಗೆ ‘ಸ್ವರ್ಗ ಮೂರೇ ಬೆರಳು ದೂರ’ ಎನ್ನುವಂತಾಯಿತು. ಆತನು ಆಕೆಯನ್ನು ಬರಸೆಳೆದು, ಬಿಗಿಯಪ್ಪಿ, ಧನ್ಯ ನಾದೆನೆಂದುಕೊಂಡನು. ಹೀಗೆ ಕಾಮಾಂಧನಾಗಿದ್ದ ಅವನ ಆಯುಷ್ಯದಲ್ಲಿ ಮೊದಲರ್ಧ ವಾದ ಐವತ್ತು ವರ್ಷ ಕಳೆದುಹೋಯಿತು. ಅಷ್ಟರಲ್ಲಿ ಅವನಿಗೆ ನೂರ ಹನ್ನೊಂದು ಗಂಡು ಮಕ್ಕಳೂ ನೂರ ಹತ್ತು ಹೆಣ್ಣುಮಕ್ಕಳೂ ಹುಟ್ಟಿದರು. ಪುರಂಜನ ಅವರೆಲ್ಲರ ವಿವಾಹ ವನ್ನೂ ವೈಭವದಿಂದ ನಡೆಸಿದನು. ಗಂಡು ಮಕ್ಕಳಲ್ಲಿ ಒಬ್ಬೊಬ್ಬನೂ ನೂರು ಮಕ್ಕಳ ತಂದೆಯಾದನು. ಅವನ ಮನೆ ಮಕ್ಕಳು, ಮೊಮ್ಮಕ್ಕಳಿಂದ ತುಂಬಿ ನಂದಗೋಕುಲ ದಂತಿತ್ತು. ರಾಜನು ಅವಅಅರನ್ನೆಲ್ಲ ನೋಡಿ ಹಿಗ್ಗುತ್ತಾ, ತನ್ನ ಸಮಾನರಿಲ್ಲವೆಂದು ಹೆಮ್ಮೆ ಪಡುತ್ತಿದ್ದನು. ಇಷ್ಟಾದರೂ ಅವನಿಗೆ ಭೋಗದ ಇಚ್ಛೆ ತೀರಲಿಲ್ಲ. ಹೊಸ ಹೊಸ ಭೋಗ ಗಳನ್ನು ಪಡೆಯುವುದಕ್ಕಾಗಿ ಯಾಗಗಳನ್ನು ಕೈಕೊಂಡನು.

ಕಾಲಚಕ್ರ ಉರುಳಿತು. ಪುರಂಜನ ಮುದುಕನಾದ. ಆದರೇನು? ಅವನ ಭೋಗದ ಆಶೆ ತಗ್ಗಲಿಲ್ಲ. ಆದರೆ ಅಷ್ಟರಲ್ಲಿಯೇ ಚಂಡವೇಗನೆಂಬ ಗಂಧರ್ವರಾಜನೊಬ್ಬ ಅವನ ನಗರಕ್ಕೆ ಮುತ್ತಿಗೆ ಹಾಕಿದ. ಅವನಿಗೆ ಮುನ್ನೂರ ಅರವತ್ತು ಸೇವಕರು. ಅವರು ಮಹಾ ಬಲಾಢ್ಯರು. ಅವರು ಪುರಂಜನನ ಪಟ್ಟಣವನ್ನು ಕೊಳ್ಳೆ ಹೊಡೆಯಲೆಂದು ಬಂದಾಗ, ಅಲ್ಲಿ ಕಾವಲಿದ್ದ ಪ್ರಜಾಗರನೆಂಬುವನು ತಾನೊಬ್ಬನೇ ಇಪ್ಪತ್ತೇಳು ಬಾಣಗಳನ್ನು ಬಿಲ್ಲಿ ನಲ್ಲಿ ಹೂಡಿ ಬಹುಕಾಲದವರೆಗೆ ಅವರೊಡನೆ ಹೋರಾಡುತ್ತಿದ್ದನು. ಕ್ರಮೇಣ ಅವನ ಶಕ್ತಿ ಕುಂದುತ್ತಾ ಬಂದಿತು. ಅದೇ ವೇಳೆಯಲ್ಲಿ ಭಯನೆಂಬವನು ಜ್ವರ ಪ್ರಜ್ವರರೆಂಬ ತನ್ನ ಸೋದರರೊಡನೆಯೂ ಭಯಂಕರವಾದ ಸೇನೆಯೊಡನೆಯೂ ಪುರಂಜನನ ಪಟ್ಟಣ ವನ್ನು ಪ್ರವೇಶಿಸಿದನು. ಪುರಂಜನನಿಗೆ ತುಂಬ ಭಯವಾಯಿತು. ಆದರೂ ಅವನು ಸ್ತ್ರೀ ಲೋಲನಾಗಿಯೇ ಇದ್ದನು. ಜರೆಯೆಂಬ ವೇಶ್ಯೆ ಅವನನ್ನು ತನ್ನ ಬುಟ್ಟಿಯಲ್ಲಿ ಹಾಕಿ ಕೊಂಡಿದ್ದಳು. ಅವನು ತನ್ನ ಶಕ್ತಿಯನ್ನೆಲ್ಲ ಕಳೆದುಕೊಂಡು ಉಸಿರಾಡುವ ಹೆಣ ದಂತಿದ್ದನು. ಆದರೂ ಅವನಿಗೆ ಸಂಸಾರದ ಮೋಹ. ತನ್ನವರಿಂದ ತೊಲಗಿಹೋಗಲು ಇಷ್ಟವಿಲ್ಲದೆ ಅರಮನೆಯಲ್ಲಿ ಅಡಗಿಕೊಂಡಿದ್ದನು. ಅಷ್ಟರಲ್ಲಿ ಪ್ರಜ್ವರನು ಊರನ್ನೆಲ್ಲಾ ಸುಡುತ್ತಾ ಬರಲು, ಪುರಂಜನನು ಪ್ರಾಣಭಯದಿಂದ ಅರಮನೆಯಿಂದ ಓಡಿಹೋಗುತ್ತಾ ಸತ್ತುಹೋದನು.

ಪುರಂಜನನು ಸಾಯುವ ಮುನ್ನ ತನ್ನ ಮಡದಿಯನ್ನೇ ನೆನೆನೆನೆದು ದುಃಖಿಸುತ್ತಿದ್ದನು. ‘ಅವಳಂತಹ ಪತಿವ್ರತೆಯರಿಲ್ಲ, ನನ್ನನ್ನು ಕಂಡರೆ ಅವಳಿಗೆಷ್ಟು ಪ್ರೀತಿ! ನಾನಿಲ್ಲದಾಗ ಅವಳ ಗತಿಯೇನು?’–ಎಂದು ಅವಳನ್ನೇ ಧ್ಯಾನಮಾಡುತ್ತಾ ಸತ್ತುದರಿಂದ ಅವನು ಮರು ಜನ್ಮದಲ್ಲಿ ವಿದರ್ಭರಾಜನಾದ ರಾಜಸಿಂಹನ ಮಗಳಾಗಿ ಹುಟ್ಟಬೇಕಾಯಿತು. ಈ ಹೆಣ್ಣು ಬೆಳೆದು ಮಲಯಧ್ವಜನೆಂಬ ರಾಜನನ್ನು ಮದುವೆಯಾಗಿ ಎಂಟು ಮಕ್ಕಳ ತಾಯಾದಳು. ಈ ಮಕ್ಕಳು ಬೆಳೆದು ದೊಡ್ಡವರಾದಮೇಲೆ ಮಲಯಧ್ವಜನು ಅವರಿಗೆ ರಾಜ್ಯವನ್ನು ವಹಿಸಿ, ವೈರಾಗ್ಯದಿಂದ ತಪಸ್ಸಿಗೆ ಹೊರಟನು. ಆತನ ಪತ್ನಿಯೂ ಆತನನ್ನು ಅನುಸರಿಸಿ, ಅತ್ಯಂತ ಭಕ್ತಿಯಿಂದ ಆತನ ಸೇವೆ ಮಾಡುತ್ತಿದ್ದಳು. ಆತನು ಬಹುಕಾಲ ತಪಸ್ಸು ಮಾಡಿ, ಬ್ರಹ್ಮಜ್ಞಾನಿಯಾಗಿ ದೇಹವನ್ನು ತ್ಯಾಗ ಮಾಡಿದನು. ಆತನ ಮಡದಿ ಗಂಡನ ದೇಹವನ್ನು ಚಿತೆಯ ಮೇಲಿಟ್ಟು, ತಾನೂ ಸಹಗಮನ ಮಾಡಬೇಕೆಂದಿರಲು, ಆಕೆಯ ಪುರಂಜನ ಜನ್ಮದಲ್ಲಿ ಗೆಳೆಯನಾಗಿದ್ದ ಅವಿಜ್ಞಾತನು ಅಲ್ಲಿಗೆ ಬಂದು ಆಕೆಗೆ ವಿವೇಕಬೋಧೆ ಮಾಡಿದನು. ಯಾರಿಗೂ ತಿಳಿಯದಂತೆ ಗೆಳೆಯನಿಗೆ ಹಿತವನ್ನು ಮಾಡುತ್ತಿದ್ದುದರಿಂದಲೆ ಆತನಿಗೆ ಆ ಹೆಸರು ಬಂದಿತ್ತು. ಆತನು ಆಕೆಯೊಡನೆ ‘ಅಮ್ಮಾ, ನಿನ್ನ ಪೂರ್ವಜನ್ಮವನ್ನು ಜ್ಞಾಪಿಸಿಕೊ. ನೀನು ಪುರಂಜನನಾಗಿದ್ದಾಗ ನಾನು ನಿನ್ನ ಗೆಳೆಯನಾಗಿದ್ದೆ. ನಾನು ಹೇಳಿದ ಬುದ್ಧಿವಾದವನ್ನು ಕೇಳದೆ ವಿಷಯಸುಖಗಳನ್ನು ಅರಸುತ್ತಾ, ನೀನು ಆ ಬಾಳನ್ನು ಹಾಳು ಮಾಡಿಕೊಂಡೆ. ಆ ಜನ್ಮಕ್ಕೂ ಮೊದಲು ನಾವು ಮಾನಸಸರೋವರದಲ್ಲಿ ಹಂಸಗಳಾಗಿ ದ್ದೆವು. ಆಗಲೂ ನೀನು ಭೋಗಗಳನ್ನು ಬಯಸಿಯೇ ಪುರಂಜನ ಜನ್ಮವನ್ನು ಪಡೆದುದು. ಸ್ವಲ್ಪ ಯೋಚಿಸಿ ನೋಡು. ಪೂರ್ವಜನ್ಮದಲ್ಲಿ ನೀನು ಗಂಡೆಂದು, ಈ ಜನ್ಮದಲ್ಲಿ ಹೆಣ್ಣೆಂದು ಭ್ರಾಂತಿಗೊಂಡಿರುವುದಕ್ಕೆ ಮಾಯೆಯೇ ಕಾರಣ. ಆ ಮಾಯೆಯನ್ನು ಹರಿ. ನೋಡು, ನಾನು ನೀನು ಬೇರೆಯಲ್ಲ. ಒಂದಾದರೂ ಬೇರೆಬೇರೆಯಾಗಿ ಕಾಣುವುದಕ್ಕೆ ಕಾರಣವೇನೆಂಬುದು ಜ್ಞಾನಿಗೆ ಅರ್ಥವಾಗುತ್ತದೆ. ಈಗ ಸಧ್ಯಕ್ಕೆ ಇಷ್ಟು ತಿಳಿದುಕೊ. ನೀನು ದೇಹ, ನಾನು ಆತ್ಮ. ನಾನು ನಿನ್ನಲ್ಲಿಯೇ ಇದ್ದೇನೆ’ ಎಂದನು.

ಕಥೆಯನ್ನು ಹೇಳುತ್ತಿದ್ದ ನಾರದರು ಈಗ ಪ್ರಾಚೀನಬರ್ಹಿಯನ್ನು ಕುರಿತು–“ಮಹಾ ರಾಜ, ನಾನು ಇದುವರೆಗೆ ಹೇಳಿದ ಕಥೆ ಆತ್ಮತತ್ವವನ್ನು ಒಳಗೊಂಡ ಒಂದು ಗೂಢಕಥೆ. ಆ ಕಥೆಯಲ್ಲಿ ಬರುವ ಪುರಂಜನನೇ ಪುರುಷ. ಪುರ ಎಂದರೆ ದೇಹ, ಅದರಲ್ಲಿ ಪ್ರೀತಿ ಯುಳ್ಳವನು ಪುರಂಜನ. ಅವನ ಗೆಳೆಯನಾದ ಅವಿಜ್ಞಾತ ಪರಮಾತ್ಮ. ಅವನ ಗುಣ, ಕಾರ್ಯಗಳು ಯಾರಿಗೂ ತಿಳಿಯವಾದ್ದರಿಂದ ಅವನಿಗೆ ಆ ಹೆಸರು. ವಿಷಯಸುಖಕ್ಕೆ ಆಶೆ ಪಟ್ಟು, ಪುರುಷನು ಮನುಷ್ಯ ಶರೀರವನ್ನು ಪ್ರವೇಶಿಸುತ್ತಾನೆ. ಪುರಂಜನ ಪ್ರವೇಶಿಸಿದ ನಗರ ಅದೇ. ಅಲ್ಲಿದ್ದ ಅವನ ಮಡದಿ ಎಂದರೆ ಬುದ್ಧಿ; ಅದರಿಂದಲೆ ಮನುಷ್ಯನಿಗೆ ಅಹಂಕಾರ ಮಮಕಾರಗಳು, ಇಂದ್ರಿಯಸುಖಗಳ ಮೇಲಿನ ಆಶೆಹುಟ್ಟುವುದು. ಅಂಗ ರಕ್ಷಕರಾಗಿದ್ದವರೇ ಜ್ಞಾನೇಂದ್ರಿಯ ಕರ್ಮೇಂದ್ರಿಯ ಮತ್ತು ಮನಸ್ಸು. ಅವು ಬುದ್ಧಿ ಯನ್ನು ಅನುಸರಿಸುತ್ತವೆ. ನೂರಾರು ದಾಸಿಯರೆಂದು ಹೇಳಿರುವುದು ಇಂದ್ರಿಯ ಪ್ರವೃತ್ತಿ ಗಳನ್ನು. ಐದು ಹೆಡೆಯ ಸರ್ಪ ದೇಹವನ್ನು ರಕ್ಷಿಸುವ ಪ್ರಾಣ. ನಗರಕ್ಕಿದ್ದ ಹೆಬ್ಬಾಗಿಲು ಗಳೆಂದರೆ ದೇಹದ ನವ ದ್ವಾರಗಳು: ಕಣ್ಣು, ಕಿವಿ, ಮೂಗು, ಬಾಯಿ, ಮಲಮೂತ್ರ ದ್ವಾರ ಗಳು. ರಾಜ್ಯಕ್ಕೆ ಮುತ್ತಿಗೆ ಹಾಕಿದ ಚಂಡವೇಗನೆಂದರೆ ಕಾಲಪುರುಷ. ವರುಷಕ್ಕೆ ಇರುವ ಮುನ್ನೂರರವತ್ತು ದಿನಗಳೇ ಅವನ ಸೇವಕರು. ರಾಜನನ್ನು ವಶಪಡಿಸಿಕೊಂಡ ಜರೆ ಎಂದರೆ ಮುಪ್ಪು.”

“ರಾಜನು ಬೇಟೆಗೆ ಹೊರಟ ಚಿತ್ರವೂ ಅರ್ಥವತ್ತಾಗಿದೆ. ದೇಹವೇ ರಥ, ಜ್ಞಾನೇಂ ದ್ರಿಯಗಳೇ ಕುದುರೆಗಳು, ಪುಣ್ಯಪಾಪಗಳು ಅದರ ಚಕ್ರಗಳು,–ಸತ್ವ, ರಜ, ತಮ– ಎಂಬುವು ಮೂರು ಧ್ವಜಗಳು. ಪಂಚಪ್ರಾಣಗಳೇ ಐದು ಕಟ್ಟುಗಳು, ಮನಸ್ಸೇ ಲಗಾಮು, ಬುದ್ಧಿಯೇ ಸಾರಥಿ, ಸುಖ ದುಃಖ ಎಂಬ ಎರಡು ನೊಗಗಳು, ಶಬ್ದಾದಿ ವಿಷಯಗಳು ಐದು ಬಾಣಗಳು. ರಥಕ್ಕೆ ಹೊದಿಸಿದ ಏಳು ಆವರಣಗಳೆಂದರೆ ಚರ್ಮವೇ ಮೊದಲಾದ ಸಪ್ತ ಧಾತುಗಳು. ವಿಷಯಾಸಕ್ತಿಯೇ ಬೇಟೆ. ಹನ್ನೊಂದು ಇಂದ್ರಿಯಗಳು ಸೇನಾನಿಗಳು. ಪ್ರಾಣಿಹಿಂಸೆಯೇ ಬೇಟೆಯ ವಿನೋದ.”

“ಮಹಾರಾಜ! ನಾನು ಪುರಂಜನನ ಕಥೆ ಹೇಳಿದುದು ನಿನ್ನ ಮನಸ್ಸು ವೈರಾಗ್ಯದತ್ತ ತಿರುಗಲೆಂದು. ನೀನು ಮಾಡುವ ಯಾಗಗಳು ಪುರುಷಾರ್ಥಗಳನ್ನು ಕೊಡುವುದಿಲ್ಲ. ವೇದ ದಲ್ಲಿ ಹೇಳಿರುವ ಕರ್ಮಗಳು ಸ್ವರ್ಗವೇ ಮೊದಲಾದ ಸುಖಗಳನ್ನು ಕೊಡುವವೆಂದು ವೇದಪಾಠಕರು ಹೇಳಬಹುದಾದರೂ, ವೇದಗಳನ್ನು ಕರ್ಮಪ್ರತಿಪಾದಕವೆಂದು–ವೇದ ದಲ್ಲಿರುವ ಕರ್ಮಕಾಂಡವನ್ನು ಭಗವಂತನ ಆರಾಧನೆಗಾಗಿ ಹೇಳಿರುವುದು–ಹೇಳುವವರು ವೇದಾರ್ಥವನ್ನು ಸರಿಯಾಗಿ ತಿಳಿಯದ ಅಜ್ಞಾನಿಗಳೇ ಸರಿ. ಕರ್ಮದ ಅರ್ಥವನ್ನು ಸರಿ ಯಾಗಿ ತಿಳಿದುಕೊ. ಭಗವಂತನಿಗೆ ಮೆಚ್ಚುಗೆಯಾಗುವ ಕರ್ಮವೇ ನಿಜವಾದ ಕರ್ಮ; ಭಗವಂತನನ್ನು ನಿಶ್ಚಲವಾಗಿ ಧ್ಯಾನಮಾಡುವ ಜ್ಞಾನವೇ ನಿಜವಾದ ಜ್ಞಾನ. ಭಗವಂತನೇ ಜಗತ್ತಿಗೆ ಮೂಲಕಾರಣನು, ಆತನೇ ಸಕಲ ಪ್ರಾಣಿಗಳ ಆತ್ಮನು, ಆತನೇ ರಕ್ಷಕನು. ಇದನ್ನು ತಿಳಿದವನೇ ವಿದ್ವಾಂಸ, ಗುರು; ಆ ಗುರುವೇ ಸಾಕ್ಷಾತ್ ಭಗವಂತ. ಆದ್ದರಿಂದ ನಿನ್ನ ಮನಸ್ಸು ಇನ್ನು ಭಗವದ್​ಭಕ್ತಿಯ ಕಡೆ ತಿರುಗಲಿ. ಗೃಹಸ್ಥಾಶ್ರಮವೆಂಬುದು ಸುಂದರವಾದ ಹೂವಿನ ತೋಟದಂತೆ ಆಕರ್ಷಕವಾಗಿದೆ. ಅಲ್ಲಿನ ಹೂವಿನಂತೆ ಇಲ್ಲಿನ ಹೆಣ್ಣುಗಳು. ಅಲ್ಲಿನ ಸುವಾಸನೆಯಂತೆ ಇಲ್ಲಿನ ಸುಖಾಭಿಲಾಷೆ. ಆದರೆ ಅಲ್ಲಿ ಅಲೆದಾಡ ಹೊರಟಿರುವ ಜಿಂಕೆಗೆ ಹಿಂದೆ ಬೇಡ, ಮುಂದೆ ತೋಳ. ಎರಡೂ ಎದುರಿಗೆ ಕಾಣುತ್ತಿಲ್ಲ. ಆದರೆ ಪ್ರಾಣಹೀರಲು ಕಾದಿವೆ. ಹಾಗೆಯೇ ಜೀವನದಲ್ಲಿ ಅಹೋರಾತ್ರಿಗಳು ಮುಂದೆ ಕಾದಿವೆ, ಹಿಂದೆ ಮೃತ್ಯು. ಆದ್ದರಿಂದ ಮನಸ್ಸನ್ನು ಸಂಸಾರದಿಂದ ಎತ್ತಿ ಭಗವಂತನ ಮೇಲೆ ಇಡು ಎಂದನು.

ನಾರದರ ಉಪದೇಶದಿಂದ ಪ್ರಾಚೀನಬರ್ಹಿಗೆ ಆತ್ಮತತ್ವವೆಲ್ಲ ಗೋಚರವಾದಂತಾ ಯಿತು. ಅವನಿಗೆ ಒಂದೇ ಸಂದೇಹ. ಪಾಪ ಪುಣ್ಯ ಕರ್ಮಗಳನ್ನು ಮಾಡಿದ ದೇಹ ಇಲ್ಲಿಯೇ ಬಿದ್ದು ಹೋಗುತ್ತದೆ. ಆ ಕರ್ಮದ ಫಲವಾಗಿ ಬೇರೊಂದು ದೇಹ ಬರುತ್ತದೆ. ಆ ದೇಹ ಹಿಂದಿನ ಜನ್ಮದ ಕರ್ಮಫಲಗಳನ್ನು ಏಕೆ ಅನುಭವಿಸಬೇಕು? ಹಿಂದಿನ ಜನ್ಮ ದಲ್ಲಿ ಮಾಡಿದ ಕರ್ಮಗಳು ಆ ಕ್ಷಣದಲ್ಲಿಯೇ ಮರೆಯಾಗಿ ಹೋಗುತ್ತವೆ. ಪೂರ್ವಕರ್ಮ ವಾಗಲಿ, ಅದಕ್ಕೆ ಕಾರಣವಾದ ಮನಸ್ಸಾಗಲಿ, ಆ ಮನಸ್ಸನ್ನು ಧರಿಸಿದ್ದ ದೇಹವಾಗಲಿ ಮುಂದಿನ ಜನ್ಮದಲ್ಲಿ ಕಾಣದಿರುವಾಗ, ಪೂರ್ವಕರ್ಮಗಳಿಗೆ ಈ ಜನ್ಮದಲ್ಲಿ ಸುಖದುಃಖ ಗಳೇಕೆ ಉಂಟಾಗಬೇಕು? ಆತನ ಈ ಪ್ರಶ್ನೆಗೆ ನಾರದರು ಸದುತ್ತರವಿತ್ತರು: ‘ಕರ್ಮಕ್ಕೆ ಕಾರಣ ಮನಸ್ಸು. ಈ ಮನಸ್ಸು ದೇಹದೊಡನೆ ಸಾಯುವುದಿಲ್ಲ; ಲಿಂಗ ಶರೀರದೊಡನೆ ಅದು ಮತ್ತೊಂದು ಜನ್ಮಕ್ಕೂ ಹೋಗುತ್ತದೆ. ಮಾಡಿದ ಕರ್ಮ ಅಲ್ಲಿಗೆ ಮುಗಿದು ಹೋಯಿತೆಂದು ತಿಳಿಯುವಂತಿಲ್ಲ. ಏಕೆಂದರೆ ‘ನಾನು, ನನ್ನದು’ ಎಂಬ ಅಹಂಕಾರ ಮಮಕಾರಗಳಿಂದ ಮಾಡಿದ ಕರ್ಮ ಮನಸ್ಸಿಗೆ ಅಂಟಿಕೊಂಡಿರುವುದರಿಂದ, ಆ ಕರ್ಮ ವಾಸನೆಯನ್ನು ಹೊತ್ತೇ ಮನಸ್ಸು ಮುಂದಿನ ಜನ್ಮಕ್ಕೆ ಪಯಣ ಮಾಡುತ್ತದೆ. ಆದ್ದರಿಂದ ಎಲ್ಲಕ್ಕೂ ಮನಸ್ಸೇ ಮೂಲ.’

ನಾರದರ ಉಪದೇಶದಿಂದ ಕಣ್ಣು ತೆರೆದ ಪ್ರಾಚೀನಬರ್ಹಿಯು, ಸಂಸಾರವನ್ನು ತ್ಯಜಿಸಿ ನೇರವಾಗಿ ಕಪಿಲ ಪುಷಿಗಳ ಆಶ್ರಮಕ್ಕೆ ಹೋದನು. ಅಲ್ಲಿ ಆತನು ಭಗವಂತನನ್ನು ಕುರಿತು ಧ್ಯಾನಮಗ್ನನಾಗಿ ಬ್ರಹ್ಮೀಭಾವವನ್ನು ಪಡೆದನು.

ಮೈತ್ರೇಯ ಪುಷಿಯಿಂದ ಇಷ್ಟನ್ನು ಕೇಳಿ ತೃಪ್ತಿಹೊಂದಿದ ವಿದುರನು ಆ ಪುಷಿಗೆ ನಮಸ್ಕರಿಸಿ, ಆತನಿಂದ ಬೀಳ್ಕೊಂಡು ಹಸ್ತಿನಾವತಿಗೆ ತೆರಳಿದನು–ಎಂದು ಶುಕಮುನಿಯು ಪರೀಕ್ಷಿದ್ರಾಜನಿಗೆ ತಿಳಿಸಿದನು.

