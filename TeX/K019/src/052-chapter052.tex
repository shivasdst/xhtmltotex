
\chapter{೫೨. ಭಲೆ ಬಲರಾಮ! ಜಯ್ ಶ್ರೀಕೃಷ್ಣ!}

ವಸಂತಋತು ಕಳೆದುಹೋಗಿ ಗ್ರೀಷ್ಮಋತು ಬಂದಿತ್ತು. ಸಾಮಾನ್ಯವಾಗಿ ಗ್ರೀಷ್ಮ ವೆಂದರೆ ಬಿರು ಬೇಸಿಗೆಯ ತಾಪಕ್ಕೆ ಹೆಸರಾದುದು. ಆದರೆ ಶ್ರೀಕೃಷ್ಣಪರಮಾತ್ಮನ ಪ್ರಭಾವ ದಿಂದ ಬೃಂದಾವನದ ಗ್ರೀಷ್ಮವೂ ವಸಂತದಂತೆಯೇ ಮನೋಹರವಾಗಿತ್ತು. ಅಲ್ಲಿನ ಗಿಡ ಮರ ಬಳ್ಳಿಗಳು ಹಚ್ಚಹಸುರಾಗಿ ಹೂ, ಹಣ್ಣುಗಳಿಂದ ಸಮೃದ್ಧವಾಗಿದ್ದವು; ಸುತ್ತಲಿನ ಕೆರೆ ತೊರೆಗಳು ನೀರಿನಿಂದ ತುಂಬಿ ಕೊಳ್ಳುತ್ತಿದ್ದವು; ನೆಲದ ತಂಪಾಗಲಿ, ಗಿಡಗಳ ಸೊಂಪಾ ಗಲಿ, ಅಚ್ಚಳಿಯದಂತೆ ಇದ್ದವು; ಅಲ್ಲಿನ ಗಾಳಿ ಕೂಡ ತಂಪಾಗಿ ಹೂವಿನ ಸುವಾಸನೆಯನ್ನು ಹೊತ್ತು, ಮಂದವಾಗಿ ಬೀಸುತ್ತಿತ್ತು; ಹಕ್ಕಿಗಳು ಇಂಪಾಗಿ ಗಾನ ಮಾಡುತ್ತಿದ್ದವು; ನೆಲವೆಲ್ಲ ಹಸಿರು ಗರಿಕೆಯಿಂದ ತುಂಬಿ ನೋಡುವವರ ಕಣ್ಣಿಗೆ ತಂಪನ್ನೂ, ಅದನ್ನು ತಿನ್ನುವ ಗೋಗಳ ಹೊಟ್ಟೆಗೆ ಸೊಂಪನ್ನೂ ನೀಡುತ್ತಿತ್ತು. ಇಂತಹ ಸುಂದರ ದಿನವೊಂದರ ಬೆಳಗ್ಗೆ ಶ್ರೀಕೃಷ್ಣ ಬಲರಾಮರು ಗೋಪಾಲಬಾಲಕರೊಡನೆ ತುರುಗಳನ್ನು ಮೇಯಿಸಲು ಅಡವಿಗೆ ಹೊರಟರು. ವನದ ಸೊಬಗನ್ನು ಸವಿಯುತ್ತ ಅವರು ಹಕ್ಕಿಗಳಂತೆ ಹಾಡುವರು; ಮೃಗ ಗಳಂತೆ ಆರ್ಭಟಿಸುವರು. ಒಮ್ಮೆ ಕಣ್ಣುಮುಚ್ಚಾಲೆ, ಮತ್ತೊಮ್ಮೆ ಮುಟ್ಟಿಸಾಟ, ಇನ್ನೊಮ್ಮೆ ಕುಸ್ತಿ, ಮಗದೊಮ್ಮೆ ಜೋಕಾಲಿ, ರಾಜ ಮಂತ್ರಿಗಳ ಆಟ–ಹೀಗೆ ಬಗೆ ಬಗೆಯ ಆಟಗಳಿಂದ ಅವರ ಗಂಟೆಗಳು ಗಳಿಗೆಗಳಂತೆ ಹಾರಿಹೋಗುವುವು. ಹೀಗೆ ಅವರೆಲ್ಲರೂ ಆಟದಲ್ಲಿ ಮೈಮರೆತಿರುವ ಸಮಯವನ್ನು ನೋಡಿಕೊಂಡು ಪ್ರಲಂಬನೆಂಬ ರಕ್ಕಸನು ಗೊಲ್ಲರ ಹುಡುಗನಂತೆ ವೇಷಹಾಕಿಕೊಂಡು, ಅವರ ಮಧ್ಯದಲ್ಲಿ ಸೇರಿ ಕೊಂಡನು. ಹಾವಿಗೆ ನವಿಲಿನಲ್ಲಿರುವಂತೆ ಅವನಿಗೆ ಶ್ರೀಕೃಷ್ಣನಲ್ಲಿ ಸಹಜದ್ವೇಷ. ಶ್ರೀಕೃಷ್ಣ ನನ್ನು ಉಪಾಯದಿಂದ ಕೊಲ್ಲಬೇಕೆಂದೆ ಅಲ್ಲಿಗೆ ಬಂದಿದ್ದುದು. ಶ್ರೀಕೃಷ್ಣನು ಅದನ್ನು ಬಲ್ಲ. ಅವನ ಮಂತ್ರವನ್ನು ಅವನ ಮೇಲೆ ತಿರುಗಿಸಿ, ಅವನನ್ನು ಕೊಲ್ಲಬೇಕೆಂಬುದೇ ಆತನ ಯೋಚನೆ. ಆದ್ದರಿಂದಲೆ ಆತ ಸಂತೋಷವಾಗಿ ಆ ಹೊಸಬನನ್ನು ತಮ್ಮ ಆಟಕ್ಕೆ ಸೇರಿಸಿಕೊಂಡ.

 ಹುಡುಗರೆಲ್ಲ ಸೇರಿಕೊಂಡು ಕುದುರೆಚೆಂಡಿನ ಆಟವಾಡಬೇಕೆಂದು ನಿಶ್ಚಯಿಸಿದರು. ಇದಕ್ಕಾಗಿ ಅವರೆಲ್ಲ ಎರಡು ಗುಂಪುಗಳಾಗಿ ವಿಭಾಗವಾದರು. ಶ್ರೀಕೃಷ್ಣ ಬಲರಾಮರು ಒಂದೊಂದು ಗುಂಪಿನ ನಾಯಕರಾದರು. ಆ ಆಟದಲ್ಲಿ ಸೋತವರು ಗೆದ್ದವರನ್ನು ಹೊತ್ತುಕೊಂಡು ಕುದುರೆಯಂತೆ ಓಡಬೇಕು. ಇದಕ್ಕೆ ಒಪ್ಪಿಕೊಂಡು ಹುಡುಗರೆಲ್ಲ ಒಂದು ಆಲದ ಮರದ ಕೆಳಗೆ ಲಗ್ಗೆ ಹೂಡಿದರು. ಕೆಲಹೊತ್ತು ಆಡುವಷ್ಟರಲ್ಲಿ ಶ್ರೀಕೃಷ್ಣನ ಗುಂಪಿಗೆ ಸೋಲಾಯಿತು. ಅವನ ಕಡೆ ಸೇರಿಕೊಂಡಿದ್ದ ಪ್ರಲಂಬನು ಎದುರು ಗುಂಪಿನ ಬಲರಾಮನನ್ನು ಹೊತ್ತುಕೊಂಡು ಓಡಬೇಕಾಯಿತು. ಅವನು ಬಂದಿದ್ದುದು ಕೃಷ್ಣನನ್ನು ಕೊಲ್ಲಲೆಂದು. ಆದರೆ ಕೃಷ್ಣನು ತೃಣಾವರ್ತನನ್ನು ಕೊಂದ ಕಥೆ ಅವನಿಗೆ ಗೊತ್ತಿತ್ತು. ಆದ್ದರಿಂದ ಅವನನ್ನು ಕಂಡರೆ ಸ್ವಲ್ಪ ಅಳುಕು. ಈಗ ಸಮಯ ಸಿಕ್ಕಿರುವಾಗ ಬಲರಾಮ ನನ್ನಾದರೂ ಕೊಲ್ಲಬೇಕೆಂದು ಅವನು ನಿಶ್ಚಯಿಸಿದ. ಅವನು ಬಲರಾಮನನ್ನು ಹೊತ್ತವನೆ ಕುದುರೆಯಂತೆ ಓಡತೊಡಗಿದ. ಎಲ್ಲರೂ ಚಪ್ಪಾಳೆ ತಟ್ಟಿ ಆನಂದಿಸಿದರು. ಆದರೆ ಅವನು ಗೊತ್ತಾದ ನೆಲೆಯಲ್ಲಿ ನಿಲ್ಲದೆ ಇನ್ನೂ ಮುಂದಕ್ಕೆ ಓಡಿದ. ಅವನು ಓಡಿದ, ಓಡಿದ, ಗೋಪಾಲಕರ ಕಣ್ಣಿಗೆ ಕಾಣದಷ್ಟು ದೂರ ಓಡಿದ. ಮೊದಲು ವಿನೋದವಾಗಿದ್ದುದು ಈಗ ಭಯಕ್ಕೆ ಕಾರಣವಾಯಿತು. ಇತ್ತ ಅವರೆಲ್ಲ ಬಲರಾಮನಿಗಾಗಿ ಮಿಡುಕುತ್ತಿದ್ದರೆ, ಅತ್ತ ಪ್ರಲಂಬ, ಬಲರಾಮನ ಭಾರ ಹೆಜ್ಜೆಹೆಜ್ಜೆಗೂ ಹೆಚ್ಚುತ್ತಾ ಹೋದುದರಿಂದ, ತನ್ನ ಸಹಜ ರೂಪವನ್ನು ಧರಿಸಿ ಆಕಾಶಕ್ಕೆ ಹಾರಿದ. ಅಬ್ಬ ಅವನ ಆಕೃತಿಯೆ! ಬೆಟ್ಟದಂತೆ ದೊಡ್ಡ ದೇಹ, ಗಿರಿಗಿರಿ ತಿರುಗುತ್ತಿರುವ ಚಕ್ರದಂತಹ ಕಣ್ಣುಗಳು, ನೇಗಿಲಿನಂತೆ ಹೊರ ಹೊರಟ ಕೋರೆ ದಾಡೆಗಳು, ಬೆಂಕಿಯ ನಾಲಗೆಯಂತಿರುವ ಕೆಂಗೂದಲು, ಕಪ್ಪಾದ ಮೋಡವೊಂದು ಮಿಂಚಿನೊಡನೆ ತೇಲಿಹೋಗುತ್ತಿರುವಾಗ, ಅದರ ಮೇಲೆ ಚಂದ್ರಬಿಂಬ ಮೂಡಿದಂತಾ ಗಿತ್ತು. ಆ ರಕ್ಕಸನ ಆಕಾರವನ್ನು ನೋಡಿ ಬಲರಾಮನಿಗೂ ಮೊದಲು ದಿಗಿಲಾಯಿತು. ಆದರೆ ಆತನು ತನ್ನ ಭಯವನ್ನು ಬದಿಗೊತ್ತಿ, ಕೋಪವನ್ನು ಕೈಗೊಂಡು, ತನ್ನ ಶಕ್ತಿ ಯನ್ನೆಲ್ಲ ಒಟ್ಟುಗೂಡಿಸಿಕೊಂಡು ಆ ರಕ್ಕಸನ ತಲೆಯ ಮೇಲೆ ಗುದ್ದಿದನು. ಆ ಒಂದು ಹೊಡೆತಕ್ಕೆ ಅವನ ತಲೆ ಎರಡು ಹೋಳಾಯಿತು. ಅವನು ‘ಹೋ’ ಎಂದು ಚೀರಿ ನೆಲಕ್ಕೆ ಬಿದ್ದನು. ಆ ಸದ್ದನ್ನು ಕೇಳಿ ಅವನ ಗೆಳೆಯರೆಲ್ಲ ಅಲ್ಲಿಗೆ ಓಡಿಬಂದರು. ಸತ್ತ ರಕ್ಕಸನನ್ನೂ ಸುಕ್ಷೇಮನಾಗಿದ್ದ ಬಲರಾಮನನ್ನೂ ಕಂಡು ಅವರು ‘ಜೈ ಬಲರಾಂ, ಭಲೆ ಬಲರಾಂ’ ಎಂದು ಹೇಳಿ, ಆನಂದದಿಂದ ಅವನನ್ನು ಆಲಂಗಿಸಿದರು.

ಗೋಪಾಲಬಾಲಕರೆಲ್ಲ ಆಟದಲ್ಲಿ ಮಗ್ನರಾಗಿರುವಾಗ ದನಗಳೆಲ್ಲ ಹೊಸ ಹುಲ್ಲನ್ನು ಅರಸುತ್ತಾ, ಅಡವಿಯಲ್ಲಿ ಬಹುದೂರ ಹೊರಟುಹೋದವು. ಗೋಪಾಲ ಬಾಲಕರು ಗಾಬರಿಯಾಗಿ, ಅವುಗಳ ಗೊರಸಿನ ಗುರುತನ್ನು ನೋಡಿಕೊಂಡು ಹಿಂಬಾಲಿಸಿ ಹೊರಟರು; ಆದರೆ ಎಷ್ಟು ಹುಡುಕಿದರೂ ದನಗಳು ಕಾಣಿಸಲಿಲ್ಲ. ಅವರು ಭಯದಿಂದ ಕಂಗೆಟ್ಟರು. ಆಗ ಶ್ರೀಕೃಷ್ಣನು ಅವರಿಗೆಲ್ಲ ಧೈರ್ಯ ಹೇಳಿ, ಒಂದು ಎತ್ತರವಾದ ಮರವನ್ನು ಏರಿದವನೆ ಆ ಹಸುಗಳನ್ನು ಹೆಸರು ಹಿಡಿದು ಕೂಗಿದನು. ಅವುಗಳಿಂದ ‘ಅಂಬಾ’ ಎಂಬ ಪ್ರತಿ ಧ್ವನಿಯೂ ಬಂದಿತು. ಆದರೆ ಅಷ್ಟರಲ್ಲಿ ಭಯಂಕರವಾದ ಒಂದು ಕಾಡುಕಿಚ್ಚು ಸುತ್ತಲೂ ಹೊತ್ತಿ, ಗಾಳಿಯಿಂದ ಹಬ್ಬಿ ಬರುತ್ತಿರುವುದು ಕಾಣಿಸಿತು. ಅದನ್ನು ಕಾಣುತ್ತಲೆ ಗೋಪಾಲರೆಲ್ಲ, ‘ಶ್ರೀಕೃಷ್ಣ, ಕಾಪಾಡು ಕಾಪಾಡು’ ಎಂದು ಕೂಗಿಕೊಂಡರು. ಶ್ರೀಕೃಷ್ಣನು ‘ಭಯಪಡಬೇಡಿ’ ಎಂದು ಹೇಳಿ, ಮರದಿಂದ ಕೆಳಗಿಳಿದು ಬಂದವನೆ ‘ನೀವೆಲ್ಲ ಕ್ಷಣಕಾಲ ಕಣ್ಣು ಮುಚ್ಚಿಕೊಳ್ಳಿರಿ’ ಎಂದನು. ಅವರು ಆತನ ಅಪ್ಪಣೆಯಂತೆ ಕಣ್ಣು ಮುಚ್ಚಿ ಕೊಂಡರು. ಒಡನೆಯೆ ಶ್ರೀಕೃಷ್ಣನು ಆ ಕಾಡುಕಿಚ್ಚನ್ನು ತನ್ನ ಅಂಗೈಗೆ ಸೆಳೆದು ಅದನ್ನು ನೀರಿನಂತೆ ಕುಡಿದುಬಿಟ್ಟನು. ಗೋಪಾಲಬಾಲರು ಕಣ್ಣು ತೆರೆದು ನೋಡುತ್ತಾರೆ, ಕಾಡು ಕಿಚ್ಚೂ ಇಲ್ಲ, ಯಾವುದೂ ಇಲ್ಲ. ಅವರ ಗೋವುಗಳೆಲ್ಲ ಸುತ್ತಲೂ ಬಂದು ನಿಂತಿವೆ. ಅದನ್ನು ಕಂಡ ಗೋಪಾಲಬಾಲರು ‘ಜಯ್ ಶ್ರೀಕೃಷ್ಣ!’ ಎಂದು ಜಯಕಾರ ಮಾಡಿದರು. ಅನಂತರ ಶ್ರೀಕೃಷ್ಣನು ಕೊಳಲನ್ನು ಬಾರಿಸುತ್ತಾ ಎಲ್ಲರೊಡನೆ ಗೋವುಗಳನ್ನು ಕರೆದು ಕೊಂಡು ಗೋಕುಲಕ್ಕೆ ಹಿಂದಿರುಗಿದನು. ಅಡವಿಯಲ್ಲಿ ನಡೆದುದನ್ನೆಲ್ಲ ಕೇಳಿದ ಗೋಕುಲ ದವರು ರಾಮಕೃಷ್ಣರನ್ನು ಪರಮಪುರುಷರೆಂದು ಭಾವಿಸಿದರು.

