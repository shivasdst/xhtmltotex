
\chapter{೫೮. ಸುದರ್ಶನ, ಶಂಖಚೂಡ, ಅರಿಷ್ಟ}

ಒಮ್ಮೆ ಗೋಕುಲದವರೆಲ್ಲ ತೀರ್ಥಯಾತ್ರೆಗೆಂದು ಹೊರಟು, ಸರಸ್ವತೀನದಿಯ ತೀರ ದಲ್ಲಿದ್ದ ‘ಅಂಬಿಕಾವನ’ವೆಂಬ ಪುಣ್ಯಕ್ಷೇತ್ರಕ್ಕೆ ಹೋದರು. ಎಲ್ಲರೂ ಅಲ್ಲಿನ ಪುಣ್ಯನದಿ ಯಲ್ಲಿ ಸ್ನಾನ ಮಾಡಿ, ಭಕ್ತಿಯಿಂದ ಪಾರ್ವತೀ ಪರಮೇಶ್ವರರನ್ನು ಪೂಜೆ ಮಾಡಿದ ಮೇಲೆ ಗೋದಾನ ಶಾಸ್ತ್ರಗಳನ್ನು ಮಾಡಿ, ತಾವು ತಂದಿದ್ದ ಹಾಲು, ಮೊಸರು, ಬೆಣ್ಣೆ, ತುಪ್ಪ, ಜೇನುತುಪ್ಪಗಳೊಡನೆ ಬೇಕಾದಷ್ಟು ಅನ್ನದಾನವನ್ನೂ ಮಾಡಿದರು. ಗೋಪಾಲರಿಗೆಲ್ಲ ಅಂದು ಉಪವಾಸವ್ರತ; ಕೇವಲ ನೀರನ್ನು ಮಾತ್ರ ಕುಡಿದು, ಅವರೆಲ್ಲ ನದಿಯ ತೀರ ದಲ್ಲಿಯೆ ಬಿಡಾರ ಹೂಡಿದರು. ಆ ರಾತ್ರಿ ಅವರು ದಣಿದು ಮಲಗಿರುವಾಗ, ಹಸಿದ ಹೆಬ್ಬಾವೊಂದು ಅವರ ಬಿಡಾರವನ್ನು ಪ್ರವೇಶಿಸಿತು. ಅದು ಅಲ್ಲಿಯೇ ಮಲಗಿದ್ದ ನಂದ ಗೋಪನನ್ನು ನುಂಗ ಬಗೆದು ಅವನ ಕಾಲನ್ನು ಕಚ್ಚಿತು. ಹಾವಿನ ಬಾಯಿಗೆ ಸಿಕ್ಕಿಬಿದ್ದ ನಂದನು ‘ಕೃಷ್ಣಾ, ಕೃಷ್ಣಾ ಇನ್ನೇನು ಗತಿ? ಹೆಬ್ಬಾವು ನನ್ನನ್ನು ನುಂಗುತ್ತಿದೆಯಲ್ಲಾ!’ ಎಂದು ಕಿರಿಚಿಕೊಂಡ. ಒಡನೆಯೆ ಮಲಗಿದ್ದವರೆಲ್ಲ ಕೊಡವಿಕೊಂಡು ಮೇಲೆದ್ದರು. ಒಬ್ಬರು ಅದನ್ನು ದೊಣ್ಣೆಯಿಂದ ಹೊಡೆದರು, ಮತ್ತೊಬ್ಬರು ಕೊಳ್ಳಿಯಿಂದ ತಿವಿದರು, ಇನ್ನೊ ಬ್ಬರು ಕಲ್ಲನ್ನೆತ್ತಿ ಅದರ ಮೇಲೆ ಹಾಕಿದರು. ಯಾರು ಏನು ಮಾಡಿದರೂ ಅದು ಬಿಡಲಿಲ್ಲ; ತನ್ನ ನುಂಗುವ ಕೆಲಸವನ್ನು ಮುಂದುವರಿಸಿತು. ಅಷ್ಟರಲ್ಲಿ ಶ್ರೀಕೃಷ್ಣ ಅಲ್ಲಿಗೆ ಬಂದ. ಆತ ಅದನ್ನು ತನ್ನ ಕಾಲಿನಿಂದ ಮೆಟ್ಟಿದ. ಒಡನೆಯೆ ಹಾವು ಸತ್ತುಬಿತ್ತು. ಅದರ ಪಕ್ಕದಲ್ಲಿ ಸುಂದರನಾದ ಯುವಕನೊಬ್ಬನು ಕಾಣಿಸಿಕೊಂಡ. ಅವನು ಶ್ರೀಕೃಷ್ಣನ ಪಾದಕ್ಕೆ ಅಡ್ಡ ಬಿದ್ದು ಕೈಮುಗಿದು ನಿಂತುಕೊಂಡನು. ಶ್ರೀಕೃಷ್ಣನು ಮುಗುಳ್​ನಗುತ್ತಾ ‘ಎಲಾ, ನೀನು ಯಾರು, ಏಕೆ ಹಾವಾದೆ, ಎಂಬುದನ್ನು ಹೇಳು; ಇವರೆಲ್ಲ ಕೇಳಲಿ’ ಎಂದ. ಅವನು ತನ್ನ ಕಥೆಯನ್ನು ಹೇಳಿದ–

“ಸ್ವಾಮಿ, ಶ್ರೀಕೃಷ್ಣಪರಮಾತ್ಮ, ನಾನೊಬ್ಬ ಗಂಧರ್ವ. ನನ್ನ ಹೆಸರು ಸುದರ್ಶನ. ಧನಮದ ರೂಪಮದದಿಂದ ಕೊಬ್ಬಿದ ನಾನು ಮನ ಬಂದಂತೆ ವಿಹರಿಸುತ್ತಾ ಅಲೆಯು ತ್ತಿರುವಾಗ ಒಮ್ಮೆ ಅಂಗಿರಸನೆಂಬ ಪುಷಿಗಳು ನನಗೆ ಇದಿರಾದರು. ಅವರ ರೂಪವನ್ನೂ ವೇಷವನ್ನೂ ಕಂಡು ನನಗೆ ನಗೆಬಂತು. ನಾನು ಅವರನ್ನು ಪರಿಪರಿಯಾಗಿ ಹಾಸ್ಯಮಾಡಿ ನಕ್ಕೆ. ಅವರು ಕೋಪಗೊಂಡು ಹಾವಾಗೆಂದು ನನ್ನನ್ನು ಶಪಿಸಿದರು. ನಾನು ಹಾವಾದೆ. ಈಗ ನೋಡಿದರೆ ಅವರ ಶಾಪವು ದೊಡ್ಡ ಅನುಗ್ರಹವೆನಿಸುತ್ತಿದೆ. ಅವರು ಶಪಿಸದಿದ್ದರೆ ನಿನ್ನ ಪಾದದ ಸೊಂಕು ನನಗೆಲ್ಲಿ ಲಭಿಸುತ್ತಿತ್ತು? ದೇವ ದೇವನಾದ ನಿನ್ನ ಪಾದ ಸೋಕುತ್ತಲೆ ನನ್ನ ಶಾಪ ಹಾರಿಹೋಯಿತು. ಅಷ್ಟೇ ಅಲ್ಲ, ಜನ್ಮಜನ್ಮಾಂತರಗಳಿಂದ ನನಗೆ ಅಂಟಿ ಬಂದ ಪಾಪಕರ್ಮವೆಲ್ಲ ನಿರ್ನಾಮವಾಯಿತು. ಪ್ರಭು, ನನಗೆ ಅಪ್ಪಣೆಕೊಡು. ನಾನು ಹೊರಡುತ್ತೇನೆ’ ಎಂದು ಬೇಡಿದನು. ಅವನು ಶ್ರೀಕೃಷ್ಣನ ಅಪ್ಪಣೆ ಪಡೆದು ಅತ್ತ ಹೋದನು. ಇತ್ತ ಗೋಪಾಲರು ಶ್ರೀಕೃಷ್ಣನ ಮಹಿಮೆಯನ್ನು ಕೊಂಡಾಡುತ್ತಾ ಗೋಕುಲಕ್ಕೆ ಹಿಂದಿರುಗಿದರು.

ಗೋಪಾಲರು ತೀರ್ಥಯಾತ್ರೆಯಿಂದ ಹಿಂದಿರುಗಿದ ಕೆಲವು ದಿನಗಳ ಮೇಲೆ, ಒಂದು ರಾತ್ರಿ ಬಲರಾಮ ಕೃಷ್ಣರಿಗೆ ಬೆಳುದಿಂಗಳಲ್ಲಿ ವಿಹರಿಸಬೇಕೆಂದು ಆಶೆಯಾಯಿತು. ಅವರು ಕೆಲ ಗೋಪಿಯರನ್ನೂ ಜೊತೆಯಲ್ಲಿ ಕರೆದುಕೊಂಡು ಯಮುನಾನದಿಯ ತೀರದಲ್ಲಿದ್ದ ಒಂದು ಉದ್ಯಾನಕ್ಕೆ ಬಂದರು. ಸುತ್ತಲೂ ಹಾಲು ಚೆಲ್ಲಿದಂತಹ ಬೆಳುದಿಂಗಳು, ಯಮುನಾನದಿಯ ಮೇಲಿನಿಂದ ಬೀಸಿಕೊಂಡು ಬರುತ್ತಿರುವ ತಂಗಾಳಿ, ಸುತ್ತುಮುತ್ತಿನ ಗಿಡಗಳಲ್ಲಿ ಅರೆಯರಳಿರುವ ಹೂಗಳ ಕಂಪು–ಈ ರಸಮಯ ಸನ್ನಿವೇಶದಲ್ಲಿ ಬಲರಾಮ ಕೃಷ್ಣರು ಮೈದುಂಬಿ ಹಾಡುತ್ತಾ ಆಲೆದಾಡುತ್ತಿದ್ದರು. ಅವರ ಸಂಗೀತದ ಅಮೃತವನ್ನು ಸವಿಯುತ್ತಲೆ ಗೋಪಿಯರು ಉಟ್ಟಸೀರೆ ಜಾರುವುದನ್ನಾಗಲಿ, ತೊಟ್ಟ ಆಭರಣ ಬೀಳುವು ದನ್ನಾಗಲಿ ಗಮನಿಸದಷ್ಟು ಮೈಮರೆತು, ಎಲ್ಲಿದ್ದವರಲ್ಲಿಯೆ ಚಿತ್ರಪ್ರತಿಮೆಗಳಂತೆ ನಿಂತು ಕೊಂಡರು. ಇಂತಹ ಸಮಯವನ್ನೆ ಹೊಂಚುಕಾಯುತ್ತಿದ್ದ ಶಂಖಚೂಡನೆಂಬ ಗಂಧರ್ವ ಒಬ್ಬನು ಬಲರಾಮಕೃಷ್ಣರಿಂದ ತುಸುದೂರದಲ್ಲಿದ್ದ ಗೋಪಿಯರಲ್ಲಿ ಕೆಲವರನ್ನು ನಿರ್ಭಯವಾಗಿ ಎಳೆದುಕೊಂಡು ಹೋದನು. ಅವರು ಹುಲಿಹಿಡಿದ ಹುಲ್ಲೆಯಂತೆ ಗಟ್ಟಿ ಯಾಗಿ ಅರಚಿಕೊಂಡರು. ‘ಶ್ರೀಕೃಷ್ಣ, ಬೇಗ ಬಾ, ಕಾಪಾಡು’ ಎಂದು. ಒಡನೆಯೆ ಬಲ ರಾಮ ಕೃಷ್ಣರು ಹತ್ತಿರದಲ್ಲಿದ್ದ ಒಂದೊಂದು ಮರವನ್ನೆ ಕಿತ್ತುಕೊಂಡು, ದನಿ ಬಂದ ದಿಕ್ಕಿಗೆ ಓಡಿದರು. ಅವರಿಬ್ಬರೂ ಯಮದೂತರಂತೆ ಅಟ್ಟಿಸಿಕೊಂಡು ಬರುವುದನ್ನು ಕಂಡು, ಶಂಖಚೂಡನು ಪ್ರಾಣಭಯದಿಂದ ಆ ಹೆಂಗಸರನ್ನು ಅಲ್ಲಿಯೆ ಬಿಟ್ಟು ಓಡಿ ಹೋದನು. ಶ್ರೀಕೃಷ್ಣನು ಅವನ ಬೆನ್ನು ಬಿಡದೆ ಓಡಿ ಕೊನೆಗೆ ಅವನನ್ನು ಹಿಡಿದವನೆ ಅವನ ತಲೆಯ ಮೇಲೆ ಬಲವಾಗಿ ಗುದ್ದಿದನು. ಆ ಹೊಡೆತಕ್ಕೆ ಅವನ ತಲೆ ಎರಡು ಹೋಳಾಗಿ, ಅದರಲ್ಲಿದ್ದ ಒಂದು ಅಮೂಲ್ಯವಾದ ರತ್ನ ಕೆಳಕ್ಕೆ ಬಿತ್ತು. ಶ್ರೀಕೃಷ್ಣನು ಅದನ್ನು ತಂದು ಅಣ್ಣನಾದ ಬಲರಾಮನಿಗೆ ಕಾಣಿಕೆಯಾಗಿ ಕೊಟ್ಟನು.

ಮತ್ತೊಂದು ದಿನ ಶ್ರೀಕೃಷ್ಣನು ಗೋಕುಲದಲ್ಲಿರುವ ತುರುಗಳ ಮಂದೆಯಲ್ಲಿ ವಿನೋದದಿಂದ ಅಡ್ಡಾಡುತ್ತಿರುವಾಗ ಅರಿಷ್ಟನೆಂಬ ರಕ್ಕಸನೊಬ್ಬನು ಗೂಳಿಯ ರೂಪ ದಿಂದ ಗೋಕುಲಕ್ಕೆ ನುಗ್ಗಿ ಬಂದನು. ದೊಡ್ಡ ಬೆಟ್ಟದಂತಿದ್ದ ಆ ಗೂಳಿ ಭಯಂಕರವಾಗಿ ಗುಟುರುಹಾಕುತ್ತಾ, ಗೊರಸಿನಿಂದ ನೆಲವನ್ನು ಕೆರೆಯುತ್ತಿತ್ತು; ಬಾಲವನ್ನು ಮೇಲಕ್ಕೆತ್ತಿ. ಚೂಪಾದ ತನ್ನ ಕೊಂಬುಗಳಿಂದ ಕಲ್ಲು ಗುಂಡುಗಳನ್ನು ಮೀಟಿ ಮೇಲಕ್ಕೆಸೆಯುತ್ತಿತ್ತು. ಅದರ ಆರ್ಭಟಕ್ಕೆ ಗೋಪ, ಗೋಪಿಯರು ನಡುಗಿಹೋದರು; ಮಂದೆಯಲ್ಲಿದ್ದ ಗೋವು ಗಳು ದಿಕ್ಕಾಪಾಲಾಗಿ ಚೆದರಿಹೋದವು. ಗೋಕುಲದವರೆಲ್ಲ ‘ಕೃಷ್ಣ, ಇನ್ನೇನು ಗತಿ?’ ಎಂದು ಚೀರಿದರು. ಇದನ್ನು ಕಂಡು ಶ್ರೀಕೃಷ್ಣನಿಗೆ ರೇಗಿಹೋಯಿತು. ಆತನು ಆ ರಕ್ಕಸ ನನ್ನು ಕುರಿತು ‘ಎಲೊ, ಗೂಳಿಯಾಗಿರುವ ಕೆಟ್ಟ ರಕ್ಕಸ, ಈ ಹೆಂಗಸರನ್ನೂ ಮಕ್ಕಳನ್ನೂ ದನಗಳನ್ನೂ ಹೆದರಿಸುತ್ತಿರುವೆಯಲ್ಲಾ, ನಿನಗೆ ನಾಚಿಕೆಯಾಗುವುದಿಲ್ಲವೆ? ಇಲ್ಲಿ ನೋಡು, ನಿನ್ನಂತಹ ದುಷ್ಟರನ್ನು ದಮನ ಮಾಡುವುದಕ್ಕಾಗಿಯೆ ನಾನು ಹುಟ್ಟಿರುವುದು; ನನ್ನ ಹತ್ತಿರ ಬಾ’ ಎಂದು ಹೇಳಿ ಭುಜವನ್ನು ತಟ್ಟಿದನು. ಇದನ್ನು ಕಂಡು ಅರಿಷ್ಟನಿಗೆ ಅತಿಶಯವಾದ ಕೋಪ ಬಂತು. ಅವನು ತನ್ನ ಬಾಲವನ್ನೆತ್ತಿಕೊಂಡು, ಕೆಂಗಣ್ಣಿನಿಂದ ಕೃಷ್ಣನನ್ನು ದುರು ದುರು ನೋಡುತ್ತಾ, ತನ್ನ ಕೊಂಬುಗಳನ್ನು ತಿವಿಯುವುದಕ್ಕೆ ಅಣಿಯಾಗಿ ಚಾಚಿಕೊಂಡು, ಅಂಬಿನಂತೆ ನೇರವಾಗಿ ನುಗ್ಗಿ ಬಂದನು. ಅವನು ಹತ್ತಿರಕ್ಕೆ ಬರುತ್ತಲೆ ಶ್ರೀಕೃಷ್ಣನು ಅವನ ಕೊಂಬುಗಳನ್ನು ಹಿಡಿದುಕೊಂಡು ಹಿಂದಕ್ಕೆ ತಳ್ಳಿದನು. ಅರಿಷ್ಟನು ಹದಿನೆಂಟು ಹೆಜ್ಜೆ ಹಿಂದಕ್ಕೆ ಹೋಗಿ ಬಿದ್ದನು. ಇದರಿಂದ ಅವನ ಕೋಪ ಮತ್ತಷ್ಟು ಹೆಚ್ಚಿತು. ಅವನು ಮತ್ತೊಮ್ಮೆ ಇನ್ನೂ ರಭಸದಿಂದ ನುಗ್ಗಿ ಬಂದನು. ಈ ಬಾರಿ ಶ್ರೀಕೃಷ್ಣನು ಅವನ ಕೊಂಬುಗಳನ್ನು ಹಿಡಿದು ಕತ್ತನ್ನು ತಿರುಚಿದನು; ನೆನೆದ ಬಟ್ಟೆಯನ್ನು ಹಿಂಡುವಂತೆ ಅವನ ಕತ್ತನ್ನು ತಿರುಚುತ್ತಲೆ ಆ ರಕ್ಕಸ ಕತ್ತು ಮುರಿದು ಕೆಳಕ್ಕೆ ಬಿದ್ದನು. ಶ್ರೀಕೃಷ್ಣನು ಅವನನ್ನು ಅಲ್ಲಿಗೆ ಬಿಡದೆ, ಅವನ ಕೊಂಬುಗಳನ್ನು ತಲೆಯಿಂದ ಕಿತ್ತು, ಅವುಗಳಿಂದಲೆ ಅವನನ್ನು ಚುಚ್ಚಿ ಹೊಡೆದನು. ಆ ಹೊಡೆತಕ್ಕೆ ರಕ್ಕಸ ನೆತ್ತರನ್ನು ಕಕ್ಕಿ, ಕಾಲುಗಳನ್ನು ಒದರುತ್ತಾ ಪ್ರಾಣವನ್ನು ಬಿಟ್ಟನು. ಹೀಗೆ ರಾಕ್ಷಸ ಸತ್ತು ಬಿದ್ದುದನ್ನು ಕಂಡು ಗೋಪಾಲರು ‘ಭಾಪು’ ಎಂದರು; ದೇವತೆಗಳು ಹೂಮಳೆಗರೆದರು.

