
\chapter{೩೯. ರಂತಿದೇವ}

ಭರದ್ವಾಜನು ಭರತನ ಸಂತತಿಯನ್ನು ಮುಂದುವರಿಸಿದನು. ಆ ಪೀಳಿಗೆಯಲ್ಲಿ ಹುಟ್ಟಿದ ಸಂಕೃತಿಯೆಂಬುವನಿಗೆ ಗುರು, ರಂತಿದೇವ ಎಂಬ ಇಬ್ಬರು ಮಕ್ಕಳಾದರು. ಇದರಲ್ಲಿ ರಂತಿದೇವನ ಪುಣ್ಯಚರಿತ್ರೆ ಬುವಿಬಾನುಗಳೆರಡರಲ್ಲಿಯೂ ತನ್ನ ಕೀರ್ತಿಯ ಹೊಳೆಯನ್ನು ಹರಿಸಿತು. ರಂತಿದೇವನು ಚಿನ್ನದ ಚಮಚವನ್ನು ಕಚ್ಚಿಕೊಂಡೇ ಹುಟ್ಟಿದವ ನಾದರೂ, ಆತನ ಕೊಡುಗೈಯ ಮುಂದೆ ಯಾವ ಐಶ್ವರ್ಯವೂ ನಿಲ್ಲಲಾರದೆ ಹೋಯಿತು. ಈತನಿಂದ ದಾನವನ್ನು ಪಡೆದವರು ಶ್ರೀಮಂತರಾದರು; ರಂತಿದೇವ ದಟ್ಟದರಿದ್ರನಾದ. ಆದರೇನು? ಹೊತ್ತಿಗೆ ಸಿಕ್ಕಷ್ಟರಿಂದಲೆ ತೃಪ್ತನಾಗಬಲ್ಲ ಆತನಿಗೆ ಬಡತನದ ಕಿರಿಕಿರಿಯೇ ಕಾಣಿಸಲಿಲ್ಲ. ಪುಣ್ಯಕ್ಕೆ ಆತನ ಮಡದಿಮಕ್ಕಳೂ ಆತನಿಗೆ ಸರಿಯಾದವರೇ. ಮುಂದಿನ ಹೊತ್ತಿಗೆ ತಮ್ಮ ಹೊಟ್ಟೆಗಿಲ್ಲವಲ್ಲ ಎಂಬ ಯೋಚನೆ ಆ ಕುಟುಂಬದಲ್ಲಿ ಯಾರಿಗೂ ಇರಲಿಲ್ಲ. ಹೀಗಿರಲು ಒಮ್ಮೆ ಆ ಸಂಸಾರಕ್ಕೆ ನಲವತ್ತೆಂಟು ದಿನಗಳವರೆಗೆ ಕುಡಿಯಲು ನೀರು ಕೂಡ ಇಲ್ಲದಂತಾಯಿತು. ನಲವತ್ತೊಂಬತ್ತನೆಯ ದಿನ ಬೆಳಗ್ಗೆ ದೇವರ ದಯ ದಿಂದ ಸ್ವಲ್ಪ ತುಪ್ಪ, ಪಾಯಸ, ಗೋದಿಯ ಅನ್ನ, ಸೀನೀರು ಸಿಕ್ಕವು. ರಂತಿದೇವ ಸ್ನಾನ ದೇವರಪೂಜೆಗಳನ್ನು ಮುಗಿಸಿ, ತನ್ನ ಮಡದಿ ಮಕ್ಕಳೊಡನೆ ಊಟಕ್ಕೆ ಕುಳಿತ. ಆತ ತುತ್ತನ್ನೆತ್ತಿ ಬಾಯಿಗೆ ಹಾಕಿಕೊಳ್ಳಬೇಕೆನ್ನುವಷ್ಟರಲ್ಲಿ ಒಬ್ಬ ಬ್ರಾಹ್ಮಣ ಮನೆಗೆ ಬಂದ. ರಂತಿದೇವನ ದೃಷ್ಟಿಯಲ್ಲಿ ಅತಿಥಿಯೆಂದರೆ ದೇವರು. ಆದ್ದರಿಂದ ಆತ ಆ ಬ್ರಾಹ್ಮಣನನ್ನು ಭಕ್ತಿಯಿಂದ ಪೂಜಿಸಿ, ತನ್ನ ಪಾಲಿನ ಅನ್ನ ಪಾಯಸಗಳನ್ನು ಆತನಿಗೆ ನೀಡಿ ತೃಪ್ತಿಪಡಿಸಿದ. ಆ ಬ್ರಾಹ್ಮಣ ಆ ಸಂಸಾರದವರನ್ನೆಲ್ಲ ಹರಸಿ, ಹೊಟ್ಟೆ ನೀವಿಕೊಳ್ಳುತ್ತಾ ಹೊರಟು ಹೋದ. ಮನೆಯಲ್ಲಿ ಉಳಿದುದನ್ನು ಮತ್ತೆ ಎಲ್ಲರೂ ಹಂಚಿಕೊಂಡು, ಊಟಕ್ಕೆ ಕುಳಿತು ಕೊಳ್ಳಬೇಕೆಂದು ಸಡಗರಿಸುವಷ್ಟರಲ್ಲಿ ಶೂದ್ರನೊಬ್ಬನು ‘ಅಮ್ಮ, ಕವಳ’ ಎಂದು ಬಾಗಿಲಿನ ಮುಂದೆ ಬಂದು ನಿಂತ. ಹಸಿದು ಬಂದವನಿಗೆ ಅನ್ನವಿಕ್ಕದೆ ಕಳಿಸುತ್ತಾರೆಯೆ? ರಂತಿದೇವ ತಮ್ಮಲ್ಲಿದ್ದುದರಲ್ಲಿಯೆ ಅರ್ಧವನ್ನು ಅವನಿಗೆ ಕೊಟ್ಟು ಕಳುಹಿಸಿದ. ಅತ್ತ ಅವನು ತೇಗುತ್ತಾ ಹೋಗುತ್ತಲೆ, ಇತ್ತ ರಂತಿದೇವ ಮಡದಿಮಕ್ಕಳೊಡನೆ ಉಳಿದ ಸ್ವಲ್ಪವನ್ನು ಹೊಟ್ಟೆಗೆ ಹಾಕಿಕೊಳ್ಳುವ ಯೋಚನೆ ಮಾಡುತ್ತಿರುವಷ್ಟರಲ್ಲಿ ಮತ್ತೊಬ್ಬ ನಾಲ್ಕು ನಾಯಿಗಳನ್ನೂ ಕಟ್ಟಿಕೊಂಡು ಮನೆಯ ಬಾಗಿಲಲ್ಲಿ ಪ್ರತ್ಯಕ್ಷನಾದ. ‘ಸ್ವಾಮಿ, ಹಸಿವೆಯಿಂದ ಪ್ರಾಣಹೋಗುತ್ತಿದೆ. ನನಗೂ ನಾಯಿಗಳಿಗೂ ಸ್ವಲ್ಪ ಅನ್ನ ಹಾಕಿ ಪುಣ್ಯ ಕಟ್ಟಿಕೊಳ್ಳಿ’ ಎಂದು ಅವನು ಅಂಗಲಾಚುವುದನ್ನು ಕಂಡು, ರಂತಿದೇವನ ಕರುಳು ಕರಗಿತು. ಆತನು ಮನೆಯಲ್ಲಿ ಉಳಿದಿದ್ದ ಆಹಾರವನ್ನೆಲ್ಲ ಅವನಿಗೂ ಅವನ ನಾಯಿ ಗಳಿಗೂ ಹಾಕಿ ‘ಕೃಷ್ಣಾರ್ಪಣ’ ಎಂದು ಕೈಮುಗಿದ. ಇನ್ನು ಆ ಕುಟುಂಬದ ಪಾಲಿಗೆ ಉಳಿದದ್ದು ಸ್ವಲ್ಪ ನೀರು ಮಾತ್ರ. ಅದೂ ಅವರಿಗೆ ದಕ್ಕಲಿಲ್ಲ. ನೀರನ್ನು ಕುಡಿಯಲೆಂದು ಅವರು ಹವಣಿಸುತ್ತಿರುವಷ್ಟರಲ್ಲಿ ಒಬ್ಬ ಹೊಲೆಯ ಹಾದಿಯಲ್ಲಿ ನಿಂತು ಕೂಗಿಕೊಂಡ. ‘ಅಯ್ಯೋ, ಬಾಯಾರಿಕೆಯಿಂದ ಪ್ರಾಣ ಹೋಗುತ್ತಿದೆ. ಸ್ವಲ್ಪ ನೀರು ಬಿಡಿರಿ’ ಎಂದ. ರಂತಿ ದೇವ ನೀರಿನ ಪಾತ್ರೆಯೊಡನೆ ಹೊರಕ್ಕೆ ಓಡಿಬಂದ. ‘ಅಯ್ಯಾ, ನಿನ್ನ ಸಂಕಟ ನಾನು ನೋಡ ಲಾರೆ; ಇಗೋ ಈ ನೀರನ್ನು ಕುಡಿ’ ಎಂದು ಹೇಳಿ ಆ ಪಾತ್ರೆಯನ್ನು ಅವನಿಗೆ ಕೊಟ್ಟನು.

ನೀರನ್ನು ಕುಡಿದು ತೃಪ್ತನಾದ ಚಂಡಾಲನತ್ತ ರಂತಿದೇವ ನೋಡುತ್ತಾನೆ, ಅವನು ಚಂಡಾಲನಲ್ಲ, ಸಾಕ್ಷಾತ್ ಪರಬ್ರಹ್ಮ. ನಾಯಿಗಳೊಡನೆ ಊಟಮಾಡಿಕೊಂಡು ಹೋಗಿ ದ್ದವನೂ ಅಲ್ಲಿ ಕಾಣಿಸಿಕೊಂಡ. ಅವನು ಸಾಕ್ಷಾತ್ ದತ್ತಾತ್ರೇಯ. ಊಟಮಾಡಿ ಹೋದ ಉಳಿದಿಬ್ಬರು ಇಂದ್ರ, ಅಗ್ನಿ; ಅವರೂ ಅಲ್ಲಿ ಪ್ರತ್ಯಕ್ಷರಾದರು. ಆ ದೇವತೆಗಳೆಲ್ಲ ಸೇರಿ ಕೊಂಡು ರಂತಿದೇವನ ಸತ್ವಪರೀಕ್ಷೆ ಮಾಡಿದ್ದರು. ಆತನ ಸತ್ವ ದೇವತೆಗಳ ಸತ್ವವನ್ನೂ ಮೀರಿಸಿತ್ತು. ಇಷ್ಟೇ ಅಲ್ಲ, ಆತ ಆ ದೇವತೆಗಳಿಂದ ಯಾವ ವರವನ್ನೂ ಕೇಳುವುದಕ್ಕೆ ಹೋಗಲಿಲ್ಲ. ಈ ಪವಾಡವನ್ನು ಕಂಡು ಆತನ ಮನಸ್ಸು ಭಗವಂತನಲ್ಲಿ ಲೀನವಾಗಿತ್ತು. ಇದರ ಫಲವಾಗಿ ಸಮಸ್ತ ಪ್ರಾಣಿಗಳನ್ನೂ ಕುಣಿಸುವ ಮಾಯೆ ಆತನಿಂದ ದೂರಕ್ಕೆ ಓಡಿ ಹೋಯಿತು. ಆತನು ಮೋಕ್ಷವನ್ನು ಪಡೆದನು.

ಮಹಾನುಭಾವನಾದ ರಂತಿದೇವನು ಮೋಕ್ಷವನ್ನು ಪಡೆದನೆಂಬುದು ಏನು ಆಶ್ಚರ್ಯ? ಆತನ ಕಾಲದಲ್ಲಿ ಆತನೊಡನೆ ನುಡಿದವರು, ನಡೆದವರು ಮಹಾಯೋಗಿಗಳೆನಿಸಿ ಕೊಂಡರು. ಇಂದಿಗೂ ಆತನ ಹೆಸರು ಹೇಳಿದವರು ಸಜ್ಜನರಾಗುತ್ತಾರೆ; ಆತನ ಕಥೆ ಕೇಳಿ ದವರು ಸದ್ಗತಿ ಪಡೆಯುತ್ತಾರೆ; ಅದನ್ನು ಬರೆದವರು ಉದ್ಧಾರವಾಗುತ್ತಾರೆ.

