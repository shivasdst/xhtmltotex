
\chapter{೩. ಶುಕಮುನಿಯ ಉಪದೇಶ}

ಶುಕಮುನಿಯು ಪರೀಕ್ಷಿದ್ರಾಜನನ್ನು ಕುರಿತು ‘ಮಹಾರಾಜ, ಸಾವು ಹತ್ತಿರವಾದಾಗ ಮಾಡಬೇಕಾದ ಮೊದಲ ಕೆಲಸವೆಂದರೆ ಮೃತ್ಯುವಿಗೆ ಸ್ವಲ್ಪವೂ ಹೆದರದೆ ನಿರ್ಭಯ ವಾಗಿದ್ದು, ದಾರಾ ಪುತ್ರಾದಿಗಳಲ್ಲಿನ ಮೋಹಬಂಧನವನ್ನು ವೈರಾಗ್ಯವೆಂಬ ಕತ್ತಿಯಿಂದ ಕತ್ತರಿಸಿಹಾಕುವುದು; ಈ ಮೋಹಬಂಧನ ಮತ್ತೆ ಬಂದು ತೊಡರಿಕೊಳ್ಳದಂತೆ ಮನೆಮಠ ಗಳನ್ನು ತೊರೆದು ಹೊರಟುಹೋಗುವುದು. ಇದನ್ನು ಸಂನ್ಯಾಸವೆಂದು ಕರೆಯುತ್ತಾರೆ. ಹೀಗೆ ಸಂನ್ಯಾಸಿಯಾಗಿ ಹೊರಟವನು ಪುಣ್ಯತೀರ್ಥಗಳಲ್ಲಿ ಮಿಂದು, ವಿವಿಕ್ತವಾದ ಒಂದು ಸ್ಥಳದಲ್ಲಿ ದರ್ಭೆಯಿಂದ ಮಾಡಿದ ಆಸನದ ಮೇಲೆ ಕುಳಿತು, ಪವಿತ್ರವಾದ ಓಂಕಾರವನ್ನು ಮನನ ಮಾಡಬೇಕು. ಪ್ರಾಣಾಯಾಮದಿಂದ ಪ್ರಾಣವಾಯುವನ್ನು ನಿಗ್ರಹಿಸಿ, ಮನಸ್ಸನ್ನು ಬ್ರಹ್ಮಬೀಜಾಕ್ಷರವಾದ ಓಂಕಾರದಲ್ಲಿ ನೆಲೆಗೊಳಿಸಬೇಕು. ಅಯ್ಯಾ, ಈ ದೇಹವೆಂಬ ರಥಕ್ಕೆ ಇಂದ್ರಿಯಗಳು ಕುದುರೆಗಳಿದ್ದಂತೆ; ಬುದ್ಧಿಯೇ ಅವುಗಳ ಸಾರಥಿ; ಮನಸ್ಸೇ ಹಗ್ಗ. ಇಂದ್ರಿಯ ರೂಪದ ಕುದುರೆಗಳು ವಿಷಯಸುಖದ ಬೀದಿಯಲ್ಲಿ ಮನಬಂದಂತೆ ಓಡ ದಿರುವ ಹಾಗೆ ತಡೆಯಬೇಕಾದುದು ರಥಿಕನಾಗಿ ಕುಳಿತಿರುವ ಜೀವಾತ್ಮನ ಕಾರ್ಯ. ಪೂರ್ವ ಕರ್ಮದ ಬಲದಿಂದ ತಡೆ ಹಿಡಿದು, ಭಗವಂತನ ಧ್ಯಾನದಲ್ಲಿ ನಿಲ್ಲಿಸಬೇಕು. ಇದನ್ನು ಧಾರಣೆಯೆಂದು ಕರೆಯುವರು. ಇದು ಕಷ್ಟಸಾಧ್ಯವಾದುದು. ಮನಸ್ಸು ಬಹು ಚಂಚಲ ವಾದುದು; ರಜೋಗುಣಕ್ಕೆ ಸಿಕ್ಕಿ ಓಡುತ್ತದೆ, ತಮೋಗುಣಕ್ಕೆ ಸಿಕ್ಕಿ ಮಂಕಾಗುತ್ತದೆ. ಆದ್ದರಿಂದ ಮತ್ತೆ ಮತ್ತೆ ಅದನ್ನು ಹಿಡಿದು, ಭಗವಂತನ ವಿರಾಡ್ರೂಪದಲ್ಲಿ ನಿಲ್ಲಿಸಬೇಕು. ಹೀಗೆ ಮಾಡಿದ ಯೋಗಿಗೆ ಭಕ್ತಿಯೋಗ ಕರಗತವಾಗುತ್ತದೆ.

ರಾಜ! ಧಾರಣೆಗೆ ಆಧಾರವಾದ ವಿರಾಡ್ರೂಪದ ಸ್ವರೂಪವನ್ನು ಕುರಿತು ಹೇಳುತ್ತೇನೆ, ಕೇಳು. ಭೂಮಿ, ನೀರು, ತೇಜಸ್ಸು, ವಾಯು, ಆಕಾಶ, ಅಹಂಕಾರ, ಮಹತ್ತತ್ವ–ಎಂಬ ಏಳು ಆವರಣಗಳೇ ಆ ವಿರಾಟ್ ಪುರುಷನ ದೇಹ. ಪಾತಾಳವೇ ಆತನ ಅಂಗಾಲು, ಸತ್ಯಲೋಕವೇ ಆತನ ಶಿರಸ್ಸು, ಇಂದ್ರಾದಿ ದೇವತೆಗಳು ಆತನ ಬಾಹುಗಳು, ದಿಕ್ಕುಗಳೇ ಕಿವಿಗಳು, ಅಶ್ವಿನೀ ದೇವತೆಗಳು ಆತನ ನಾಸಿಕಪುಟ, ಅಗ್ನಿಹೋತ್ರಾ ಆತನ ಬಾಯಿ, ನಕ್ಷತ್ರಲೋಕ ಕಣ್ಣುಗಳು, ಹಗಲಿರಳುಗಳು ರೆಪ್ಪೆಗಳು, ಬ್ರಹ್ಮಪದವಿ ಆತನ ಭ್ರೂಭಂಗಿ, ಯಮನೇ ಕೋರೆಹಲ್ಲು, ಮಾಯೆ ಮಂದಹಾಸ, ಪ್ರಪಂಚದ ಸೃಷ್ಟಿ ಕಡೆಗಣ್​ದೃಷ್ಟಿ, ಸಮುದ್ರ ಆತನ ಜಠರ, ಬೆಟ್ಟಗಳು ಎಲುಬು, ನದಿಗಳು ನಾಡಿಗಳು, ಮರಗಳು ಮೈ ಗೂದಲು, ಕಾಲವೇ ನಡುಗೆ, ಮೋಡಗಳೇ ತಲೆಗೂದಲು, ಸಂಧ್ಯೆಗಳೇ ವಸ್ತ್ರಗಳು, ಪ್ರಕೃತಿಯೇ ಹೃದಯ, ಚಂದ್ರನೇ ಮನಸ್ಸು, ರುದ್ರನೇ ಅಹಂಕಾರ, ಮಂತ್ರಗಳೇ ಬುದ್ಧಿ, ಮಾನವ ಜಾತಿಯೇ ಮನೆ, ಹಕ್ಕಿಗಳೇ ಕಲಾನೈಪುಣ್ಯ, ಗಂಧರ್ವಾದಿಗಳೇ ಕಂಠಧ್ವನಿ; ಅಯ್ಯಾ, ಆ ಮಹಾತ್ಮನಿಗೆ ಬ್ರಾಹ್ಮಣರೇ ಮುಖ, ಕ್ಷತ್ರಿಯರೇ ಭುಜ, ವೈಶ್ಯರೇ ತೊಡೆ, ಶೂದ್ರರೇ ಪಾದಗಳು. ಇಂತಹ ರೂಪದಲ್ಲಿ ಮನಸ್ಸನ್ನು ನಿಲ್ಲಿಸಿದ ಮೇಲೆ ಎಲ್ಲವೂ ಆತ ನಾಗಿಯೇ ಕಾಣುವುದೇ ಹೊರತು, ಆತನಿಂದ ಭಿನ್ನವಾದ ಯಾವುದೂ ಕಂಡುಬರುವು ದಿಲ್ಲ. ಆದ್ದರಿಂದ ಸರ್ವಾತ್ಮಕನಾದ ಆ ಭಗವಂತನನ್ನು ಕುರಿತು ಧ್ಯಾನ ಮಾಡಬೇಕೆ ಹೊರತು, ಸಂಸಾರಕ್ಕೆ ಕಾರಣವಾಗುವ ಇತರ ವಸ್ತುವಾವುದನ್ನೂ ಭಜಿಸಬಾರದು.

ಧಾರಣಾಯೋಗದ ಮಹಿಮೆ ಮಹತ್ತರವಾದುದು. ಬ್ರಹ್ಮದೇವನು ಧಾರಣೆಯಿಂದ ಮಹಾ ವಿಷ್ಣುವನ್ನು ಮೆಚ್ಚಿಸಿ, ಪ್ರಳಯದಲ್ಲಿ ತನೆಗೆ ಮರೆತು ಹೋಗಿದ್ದ ಸೃಷ್ಟಿರಹಸ್ಯವನ್ನು ಮತ್ತೆ ಪಡೆದನು. ಸಾಮಾನ್ಯವಾಗಿ ಜನರು ವೇದವಿಹಿತವಾದ ಕರ್ಮಗಳನ್ನು ಆಚರಿಸಿ, ಸ್ವರ್ಗಾದಿ ಸುಖಗಳನ್ನು ಪಡೆಯುವುದರಲ್ಲಿಯೇ ಆಸಕ್ತರಾಗಿರುವರು; ಮೋಕ್ಷಕ್ಕಾಗಿ ಪ್ರಯತ್ನಿಸುವವರೇ ವಿರಳ. ಇದಕ್ಕಾಗಿ ವೇದವನ್ನು ನಿಂದಿಸಬೇಕಾಗಿಲ್ಲ. ವೇದದಲ್ಲಿ ಸತ್ವ, ರಜಸ್, ತಮೋಗುಣವುಳ್ಳ ಮೂರು ಬಗೆಯ ಜನರಿಗೂ ಅವರವರ ಗುಣಕ್ಕೆ ಅನುಸಾರ ವಾದ ಮಾರ್ಗವನ್ನು ಬೋಧಿಸಿದೆ. ಹಾಗೆ ಮಾಡದಿದ್ದರೆ ರಜಸ್ತಮೋಗುಣವುಳ್ಳವರು ಮೋಕ್ಷವನ್ನು ಅಪೇಕ್ಷಿಸುವವರಲ್ಲವಾದ್ದರಿಂದ, ಸ್ವರ್ಗಾದಿಗಳಿಗೂ ದಾರಿ ತೋರದಿದ್ದರೆ ಕಾಮಪರಾಯಣರಾಗಿ ಕೆಟ್ಟುಹೋಗುವರು. ಆದ್ದರಿಂದ ವಿವೇಕಿಗಳಾದ ಸಾತ್ವಿಕರು ಅಶಾಶ್ವತಗಳಾದ ಸ್ವರ್ಗಾದಿ ಸುಖಗಳನ್ನು ಬಯಸದೆ ಮೋಕ್ಷಸಾಧನೆಗಾಗಿಯೇ ಪ್ರಯತ್ನಿಸ ಬೇಕು. ನಿಜ, ಕರ್ಮಫಲಗಳನ್ನು ಸಂಪೂರ್ಣವಾಗಿ ತ್ಯಾಗಮಾಡಿದರೆ ಜೀವಧಾರಣೆಯೇ ಕಷ್ಟವಾಗುತ್ತದೆ. ಆದ್ದರಿಂದ ದೇಹಧಾರಣೆಗೆ ಅತ್ಯಗತ್ಯವಾದಷ್ಟು ಮಾತ್ರ ಭೋಗವಸ್ತು ಗಳನ್ನು ಪಡೆಯಬಹುದಾದರೂ ಅದರಲ್ಲಿ ಆಸಕ್ತಿ ಸಲ್ಲದು. ಮಲಗುವುದಕ್ಕೆ ವಿಶಾಲವಾದ ಈ ಭೂಮಿಯೇ ಇರುವಾಗ ಹಾಸಿಗೆಯನ್ನು ಹುಡುಕಿಕೊಂಡು ಅಲೆಯುವುದೇಕೆ? ಮೃದುವಾದ ತೋಳುಗಳಿರುವಾಗ ದಿಂಬುಗಳನ್ನು ಹುಡುಕುವುದೇಕೆ? ಕೈಬೊಗಸೆ ಇರು ವಾಗ ಪಾತ್ರೆಗಳನ್ನು ಸಂಗ್ರಹಿಸುವುದೇಕೆ? ದಿಕ್ಕುಗಳೇ ಇರುವಾಗ ಪಟ್ಟೆ ಪೀತಾಂಬರ ಗಳೇಕೆ? ಅಗತ್ಯವಾದರೆ, ದಾರಿಯಲ್ಲಿ ಬಿದ್ದಿರುವ ಹರಕುಬಟ್ಟೆಯನ್ನು ಕೌಪೀನವಾಗಿ ಧರಿಸಿದರಾಯಿತು. ಮರಗಳೆಲ್ಲವೂ ಪರರಿಗಾಗಿಯೇ ಬಿಟ್ಟಿರುವ ಫಲಗಳನ್ನು ತಿಂದರೆ ನಮ್ಮ ಹೊಟ್ಟೆ ತುಂಬುವುದಿಲ್ಲವೆ? ಪವಿತ್ರವಾದ ನದಿಗಳ ನೀರನ್ನು ಕುಡಿದರೆ ತೃಪ್ತಿ ಯಾಗುವುದಿಲ್ಲವೆ? ನಿರಾತಂಕವಾಗಿರುವ ಗವಿಗಳು ವಾಸಕ್ಕೆ ಸಾಲವೆ? ಇವೆಲ್ಲವೂ ಒಂದು ಪಕ್ಷ ದುರ್ಲಭವಾದರೂ ಶರಣಾಗತ ರಕ್ಷಕನಾದ ಭಗವಂತನು ಕಾಪಾಡಲಾರನೆ? ಆದ್ದರಿಂದ ಜೀವನಾಧಾರಕ್ಕೆ ಅಗತ್ಯವಾದ ವಸ್ತುಗಳಿಗಾಗಿ ಮಾನವ ಹಲುಬಿ ಹಂಬಲಿಸ ಬೇಕಾದುದಿಲ್ಲ. ಇವುಗಳಲ್ಲಿ ವಿರಕ್ತನಾಗಿ, ಪರಮಾತ್ಮನನ್ನು ಧ್ಯಾನ ಮಾಡುತ್ತಾ ಹೋದರೆ, ಸಂಸಾರಕ್ಕೆ ಕಾರಣವಾದ ಅವಿದ್ಯೆ ನಾಶವಾಗುತ್ತದೆ; ನಿರತಿಶಯವಾದ ಆನಂದ ಲಭಿಸು ತ್ತದೆ. 

ಅಯ್ಯಾ ಮಹಾರಾಜ, ಮೂಢರಾದ ಜನ ಭಗವಂತನನ್ನು ಬಿಟ್ಟು, ಬೇರೆಯ ವಸ್ತುಗಳಿ ಗಾಗಿ ಬಾಯಿಬಿಟ್ಟು, ಸಂಸಾರದ ಸುಳಿಗೆ ಸಿಕ್ಕಿ ಸಂಕಟಪಡುತ್ತಾರೆ. ಇದನ್ನು ಕಣ್ಣಾರೆ ಕಂಡರೂ ಅದಕ್ಕಾಗಿಯೇ ಪ್ರಯತ್ನಿಸುವವನು ಪಶುಪ್ರಾಯನೇ ಸರಿ. ಜ್ಞಾನಿಯಾದವನು ಮತ್ತೊಂದಕ್ಕೆ ಆಶೆ ಪಡದೆ, ತನ್ನ ಹೃದಯಾಕಾಶದಲ್ಲಿರುವ ಪರಮಾತ್ಮನನ್ನು ಸದಾ ಧ್ಯಾನಿ ಸುತ್ತಿರಬೇಕು. ಹೀಗೆ ಮಾಡುತ್ತಾ ಭಕ್ತಿಯೋಗವನ್ನು ಸಾಧಿಸಿದವನಾಗಿ, ಶರೀರಕ್ಕೆ ಕಾರಣ ವಾದ ಪ್ರಾರಬ್ಧಕರ್ಮವನ್ನು ಸವೆಸುತ್ತಾನೆ. ಅದು ಮುಗಿದೊಡನೆಯೇ ದೇಹವನ್ನು ತ್ಯಜಿಸಲು ನಿಶ್ಚಯಿಸಿ, ಸ್ವಸ್ಥಚಿತ್ತದಿಂದ ಕುಳಿತು, ಮನಸ್ಸನ್ನು ಬುದ್ಧಿಯಿಂದ ಕಟ್ಟಿಹಾಕಿ, ಆ ಬುದ್ಧಿಯನ್ನು ಜೀವಾತ್ಮನಲ್ಲಿ ಸೇರಿಸಬೇಕು. ಅನಂತರ ಜೀವಾತ್ಮನನ್ನು ಪರಮಾತ್ಮ ನಲ್ಲಿ ಸಮರ್ಪಿಸಿ, ಬ್ರಹ್ಮಾನಂದದಲ್ಲಿ ತಲ್ಲೀನನಾಗಬೇಕು. ಇಲ್ಲಿಗೆ ಸಮಸ್ತ ಕರ್ಮಗಳಿಂದಲೂ ನಿವೃತ್ತನಾದಂತಾಯಿತು. ಇನ್ನು ದೇಹತ್ಯಾಗಕ್ಕೂ ಒಂದು ಕ್ರಮವಿದೆ. ಮೂಲಾಧಾರದಲ್ಲಿರುವ ಪ್ರಾಣವಾಯುವನ್ನು ಕ್ರಮವಾಗಿ ನಾಭಿಯಲ್ಲಿರುವ \textbf{ಮಣಿಪೂರಕ}, ಹೃದಯದಲ್ಲಿರುವ \textbf{ಅನಾಹುತ} ಕಂಠದ ಕೆಳಗಿರುವ \textbf{ವಿಶುದ್ಧ}–ಎಂಬ ಚಕ್ರಗಳಲ್ಲಿ ಸೇರಿಸಿ, ಕಡೆಗೆ ಹುಬ್ಬುಗಳ ಮಧ್ಯದಲ್ಲಿರುವ ಆಜ್ಞಾಚಕ್ರಕ್ಕೆ ತಂದು ಸೇರಿಸಬೇಕು. ಆಗ ಕಿವಿ, ಮೂಗು, ಕಣ್ಣು, ಬಾಯಿ ಮೊದಲಾದ ದ್ವಾರಗಳೆಲ್ಲವೂ ಮುಚ್ಚಿಹೋಗುವವು. ಮುಹೂರ್ತಕಾಲ ಪ್ರಾಣವಾಯುವನ್ನು ಆಜ್ಞಾಚಕ್ರದಲ್ಲಿಯೇ ನಿಲ್ಲಿಸಿ, ಪರಮಾತ್ಮಪ್ರಾಪ್ತಿ ಗಾಗಿ ನಿರೀಕ್ಷಿಸುತ್ತಿದ್ದರೆ, ಆ ಪ್ರಾಣವಾಯು ಬ್ರಹ್ಮರಂಧ್ರವನ್ನು ಭೇದಿಸಿಕೊಂಡು ಹೋಗುತ್ತದೆ. ಯೋಗಿಯಾದವನು ದೇಹತ್ಯಾಗ ಮಾಡಬೇಕಾದುದೇ ಹೀಗೆ.

ದೇಹತ್ಯಾಗ ಮಾಡಿದ ಜೀವಕ್ಕೆ ಅರ್ಚಿರಾದಿಗತಿ, ಧೂಮಾದಿಗತಿ–ಎಂಬ ಎರಡು ಮಾರ್ಗಗಳುಂಟು. ಇವುಗಳಲ್ಲಿ ಅರ್ಚಿರಾದಿಗತಿಯೇ ಶ್ರೇಷ್ಠವಾದುದು. ಈ ಗತಿಯನ್ನು ಅನುಸರಿಸುವ ಜೀವನು ಮೊದಲು ಆಕಾಶವನ್ನು ಸೇರಿ ಅಲ್ಲಿಂದ ಅಗ್ನಿಯನ್ನು ಹೊಂದು ವನು. ಅಲ್ಲಿ ಜೀವನ ಪಾಪ ಪುಣ್ಯಗಳು ಇಲ್ಲವಾಗುವವು. ಹೀಗೆ ಕರ್ಮವಿಮೋಚನೆಯಾದ ಮೇಲೆ ಜೀವನು ಸೂರ್ಯನಿಗೂ ಮೇಲ್ಭಾಗದಲ್ಲಿರುವ ಶಿಂಶುಮಾರ ಚಕ್ರವನ್ನೂ, ಅಲ್ಲಿಂದ ಮುಂದೆ ಮಹರ್ಲೋಕವನ್ನೂ ದಾಟಿ ಬ್ರಹ್ಮಲೋಕವನ್ನು ಸೇರುವನು. ಆ ಲೋಕದಲ್ಲಿ ಭಯವಿಲ್ಲ, ದುಃಖವಿಲ್ಲ, ಮುಪ್ಪಿಲ್ಲ, ಸಾವಿಲ್ಲ. ಆದರೆ ಆಗಾಗ ‘ಮತ್ತೆಲ್ಲಿ ದುರಂತ ದುಃಖಕ್ಕೆ ಕಾರಣನಾದ ಸಂಸಾರ ನಮಗೆ ಗಂಟುಬೀಳುವುದೊ!’ ಎಂಬ ಕಳವಳ ಹುಟ್ಟು ತ್ತದೆ. ಅದರಿಂದ ಆ ಜೀವ ಅಲ್ಲಿಯೂ ನಿಲ್ಲದೆ ಮುಂದೆ ಹೊರಟು ಪೃಥ್ವಿ, ಅಪ್ಪು, ತೇಜಸ್ಸು, ವಾಯು, ಆಕಾಶ, ಮಹತ್, ಅಹಂಕಾರ ಎಂಬ ಏಳು ಆವರಣಗಳನ್ನು ಕ್ರಮ ವಾಗಿ ಭೇದಿಸಿಕೊಂಡು ಹೋಗಿ ಆನಂದಘನವಾದ ಬ್ರಹ್ಮಸ್ವರೂಪವನ್ನು ಪಡೆಯುವನು. ಚತುರ್ಮುಖ ಬ್ರಹ್ಮನೇ ಮೊದಲಾದ ದೇವತೆಗಳನ್ನು ಸ್ವತಂತ್ರರೆಂದು ನಂಬಿ ಭಜಿಸ ತಕ್ಕವರು ಧೂಮಾದಿಗತಿಯನ್ನು ಪಡೆಯುವರು. ಇವರಿಗೆ ಸ್ಥೂಲದೇಹವು ಬಿದ್ದು ಹೋದರೂ ಸೂಕ್ಷ್ಮದೇಹವು ಮಾತ್ರ ಇದ್ದೇ ಇರುವುದು. ಆ ದೇಹದಿಂದಲೇ ಅವರು ಲೋಕಾಂತರಗಳನ್ನು ಪಡೆಯುವರು. ಇವರು ಕಾಮ್ಯಕರ್ಮಗಳನ್ನು ಮಾಡುವವರಿಗಿಂತಲೂ ಮೇಲು.

ಅಯ್ಯಾ ಮಹಾರಾಜ, ಲೋಕದಲ್ಲಿ ಮನುಷ್ಯಜನ್ಮವೇ ದುರ್ಲಭ. ಮನುಷ್ಯನಾಗಿ ಹುಟ್ಟಿದರೂ ವಿವೇಕಿಯಾಗಿರುವದಂತೂ ಇನ್ನೂ ದುರ್ಲಭ. ಸಾಮಾನ್ಯವಾಗಿ ಫಲಾಪೇಕ್ಷೆ ಯಿಂದ ದೇವತೆಗಳನ್ನು ಪೂಜಿಸುವವರೇ ಹೆಚ್ಚು. ಅಂತಹವರೇ ಆಗಲಿ, ವಿರಕ್ತರೇ ಆಗಲಿ, ಮೋಕ್ಷವನ್ನು ಬಯಸುವುದಾದರೆ ಭಕ್ತಿಯೋಗವನ್ನು ಅವಲಂಬಿಸಿ, ಪರಮಪುರುಷನನ್ನು ಭಜಿಸಬೇಕು. ಹಾಗೆ ಭಜಿಸುವವರಿಗೆ ಭಗವದ್ಭಕ್ತರ ಸಹವಾಸ ಲಭಿಸಿದೊಡನೆಯೆ ಅವರ ಮನಸ್ಸು ಭಗವಂತನಲ್ಲಿ ನಿಶ್ಚಲವಾಗಿ ನಿಂತು, ಮುಕ್ತಿ ದೊರೆಯುತ್ತದೆ. ಭಗವಂತನ ಸಂಕೀರ್ತನದಿಂದ ಮನಸ್ಸು ಶುದ್ಧವಾಗುತ್ತದೆ, ವಿರಕ್ತಿ ಹುಟ್ಟುತ್ತದೆ, ಭಕ್ತಿಯೋಗ ತಾನಾಗಿಯೇ ಲಭಿಸುತ್ತದೆ. ಆದ್ದರಿಂದ ಸುಖವನ್ನು ಬಯಸುವವನು ಹರಿಕಥೆಯಲ್ಲಿ ಅನು ರಾಗವನ್ನಿಡಬೇಕು. ಇಲ್ಲದಿದ್ದರೆ ಮನುಷ್ಯನಿಗೂ ನಾಯಿ ಹಂದಿ ಮೊದಲಾದ ಪ್ರಾಣಿ ಗಳಿಗೂ ವ್ಯತ್ಯಾಸವಿಲ್ಲದಂತಾಗುತ್ತದೆ. ಭಗವಂತನ ಚರಿತ್ರೆಯನ್ನು ಕೇಳದ ಕಿವಿಗಳು ಹಾಳುಗವಿಗಳು, ಗಾನ ಮಾಡದ ನಾಲಗೆ ಕಪ್ಪೆಯ ನಾಲಗೆ, ಭಗವಂತನ ದಿವ್ಯಮಂಗಳ ವಿಗ್ರಹವನ್ನು ಕಾಣದ ಕಣ್ಣು ನವಿಲಕಣ್ಣು, ಪೂಜೆಮಾಡದ ಕೈ ಹೆಣದ ಕೈ, ಯಾತ್ರೆ ಮಾಡದ ಕಾಲು ಮರದ ಕಾಲು. ಭಗವಂತನ ನಾಮಾಮೃತವನ್ನು ಕೇಳಿ ರೋಮಾಂಚನಗೊಳ್ಳದ, ಆನಂದಬಾಷ್ಪಗಳನ್ನು ಸುರಿಸದ ಮನುಷ್ಯ ಮನುಷ್ಯನಲ್ಲ, ಒಂದು ಕಗ್ಗಲ್ಲು.

ಶುಕಮುನಿಯ ಉಪದೇಶದಿಂದ ಸಂತೋಷಗೊಂಡ ಪರೀಕ್ಷಿದ್ರಾಜನು ಆ ಮುನಿ ಯನ್ನು ಕುರಿತು ‘ಹೇ ಬ್ರಾಹ್ಮಣೋತ್ತಮ, ನಿಮ್ಮ ಉಪದೇಶದಿಂದ ನನ್ನ ಅಜ್ಞಾನ ತೊಲಗಿತು. ನಿಮ್ಮ ಪವಿತ್ರವಾದ ನುಡಿಗಳಿಂದ ನನ್ನ ಇನ್ನು ಕೆಲವು ಸಂದೇಹಗಳನ್ನೂ ನಿವಾರಿಸಿರಿ. ಭಗವಂತನು ತನ್ನ ಮಾಯೆಯಿಂದ ಸೃಷ್ಟಿ ಸ್ಥಿತಿ ಲಯಗಳನ್ನು ನಡೆಸುವುದು ಹೇಗೆ? ಆತನು ತನ್ನ ಲೀಲೆಗಳನ್ನು ನಟಿಸುವುದಕ್ಕಾಗಿ ಯಾವ ಯಾವ ಅವತಾರಗಳನ್ನು ತಾಳಿದ? ತತ್ವಜ್ಞಾನಿಗಳಲ್ಲಿ ಅಗ್ರಗಣ್ಯರಾದ ನೀವು ನನ್ನಲ್ಲಿ ಕೃಪೆದೋರಿ ಈ ಸಂದೇಹ ಗಳನ್ನು ಹೋಗಲಾಡಿಸಿ’ ಎಂದು ಬೇಡಿಕೊಂಡನು. ಆತನ ಜ್ಞಾನದಾಹವನ್ನು ಕಂಡು ಶುಕ ಮುನಿಗೆ ಸಂತೋಷವಾಯಿತು. ಆತನು ರಾಜನಿಗೆ ಉತ್ತರಕೊಡುವ ಮುನ್ನ ಕಣ್ಮುಚ್ಚಿ ಶ್ರೀಹರಿಯನ್ನು ಧ್ಯಾನಿಸಿದನು–‘ಹೇ ಅಪಾರಮಹಿಮ, ತ್ರಿಮೂರ್ತಿಸ್ವರೂಪ, ಸರ್ವಾಂತರ್ ಯಾಮಿ, ಅಗೋಚರ ಸ್ವರೂಪ, ನಿನಗೆ ನಮಸ್ಕಾರ; ಹೇ ಸಜ್ಜನ ರಕ್ಷಕ, ದುರ್ಜನ ಶಿಕ್ಷಕ, ಸತ್ವಸ್ವರೂಪಿ, ಪರಮಹಂಸಾಶ್ರಮದಲ್ಲಿರುವವರಿಗೆ ಆತ್ಮಸ್ವರೂಪವನ್ನು ಅನುಗ್ರಹಿಸುವ ಪರಮಾತ್ಮ, ನಿನಗೆ ನಮಸ್ಕಾರ; ಹೇ ಭಕ್ತಪಾಲ, ನಿನ್ನ ಕೀರ್ತನ ಸ್ಮರಣ ಈಕ್ಷಣ ವಂದನ ಶ್ರವಣ ಪೂಜೆಗಳಿಂದ ತಕ್ಷಣವೇ ಪಾಪಗಳನ್ನು ಕಳೆಯಬಲ್ಲ ಮಂಗಳ ಮೂರ್ತಿ, ನಿನಗೆ ನಮಸ್ಕಾರ; ಶ್ರಿಯಃಪತಿ, ಯಜ್ಞಪತಿ, ಪ್ರಜಾಪತಿ, ಧಿಯಾಂಪತಿ, ಲೋಕಪತಿ, ಧರಾಪತಿ, ಸತಾಂಪತಿ, ಅಂಧಕವೃಷ್ಣಿ, ಸಾತ್ವತಾಂಪತಿ ಎಂಬ ಕೀರ್ತಿಗೆ ಪಾತ್ರನಾದ ಹೇ ಭಗವಂತ, ನಿನಗೆ ನಮಸ್ಕಾರ–ಇತ್ಯಾದಿಯಾಗಿ ಶ್ರೀಹರಿಯನ್ನು ಸ್ತೋತ್ರಮಾಡಿದ ಮೇಲೆ, ತಂದೆ ಯಾದ ವ್ಯಾಸರನ್ನು ಭಕ್ತಿಯಿಂದ ಸ್ಮರಿಸಿಕೊಂಡು ಪರೀಕ್ಷಿದ್ರಾಜನ ಪ್ರಶ್ನೆಗಳಿಗೆ ಉತ್ತರ ಕೊಡಲು ಪ್ರಾರಂಭಿಸಿದನು–

“ಮಹಾರಾಜ, ಹಿಂದೆ ಬ್ರಹ್ಮದೇವರು ಏಕಾಗ್ರಮನಸ್ಸಿನಿಂದ ತಪಸ್ಸನ್ನು ಆಚರಿಸಿದು ದನ್ನು ಕಂಡು ನಾರದನು ಆತನನ್ನು ಕುರಿತು ‘ಸ್ವಾಮಿ, ನೀನೇ ಸರ್ವೇಶ್ವರನೂ ಸರ್ವಜ್ಞನೂ ಆಗಿರುವಾಗ ನೀನಾರನ್ನು ಕುರಿತು ತಪಸ್ಸು ಮಾಡುವೆ?’ ಎಂದು ಕೇಳಿದನು. ಆಗ ಬ್ರಹ್ಮನು, ‘ಮಗು, ನಾನು ನಿನಗೆ ಲೋಕೇಶ್ವರನಂತೆ ಕಾಣಿಸಿದುದು ಸಹಜ. ಆದರೆ ಆ ಶಕ್ತಿ ನನ್ನದಲ್ಲ; ಭಗವಂತನಾದ ವಾಸುದೇವನದು. ಆತನೊಬ್ಬನೇ ಮಾಯಾತೀತನು. ವೇದಗಳೆಲ್ಲವೂ ಹೊಗಳುವುದು ಆತನನ್ನೇ. ಸಕಲ ದೇವತೆಗಳೂ ಆತನಿಂದಲೇ ಜನಿಸಿದವರು, ಆತನಿಂದ ಬೇರೆಯಲ್ಲ. ಸ್ವರ್ಗಾದಿ ಲೋಕಗಳೆಲ್ಲವೂ ಆತನ ಆನಂದಾಂಶಗಳೇ. ಯೋಗ, ತಪಸ್ಸು, ಜ್ಞಾನ, ಮೋಕ್ಷ–ಎಲ್ಲವೂ ಆತನ ಸ್ವರೂಪವೇ. ಆತನೇ ಸರ್ವ ಸ್ವರೂಪ, ಸರ್ವಾಧಾರ, ಸರ್ವಜ್ಞ, ಸರ್ವಸಾಕ್ಷಿ, ಸರ್ವೇಶ್ವರ, ಸರ್ವಾತ್ಮಕ, ನಿರ್ವಿಕಾರ. ಆತನ ಸಂಕಲ್ಪಮಾತ್ರ ದಿಂದ ಪ್ರೇರಿತನಾಗಿ ಜನಿಸಿರುವ ನಾನು ಆತ ಸೃಷ್ಟಿಸಿರುವ ಪದಾರ್ಥಗಳನ್ನೇ ಸೃಷ್ಟಿಸು ತ್ತೇನೆ’ ಎಂದು ಹೇಳಿದನು. ಅನಂತರ ಆತನು ನಾರದನಿಗೆ ಸೃಷ್ಟಿಕ್ರಮವನ್ನು ವಿವರಿಸಿ ಹೇಳಿದನು–

‘ಮಾಯಾನಿಯಾಮಕನಾದ ಭಗವಂತನು ತನ್ನ ಸಂಕಲ್ಪರೂಪವಾದ ಮಾಯೆಯಿಂದ ನಾನಾ ರೂಪನಾಗಲು ಇಚ್ಛಿಸಿದನು. ಆತನ ಇಚ್ಛಾಮಾತ್ರದಿಂದ ಆತನಲ್ಲಿ ಕಾಲ, ಕರ್ಮ, ಸ್ವಭಾವಗಳಾದವು. ಕಾಲವು ಸತ್ವ, ರಜ ಮತ್ತು ತಮೋಗುಣಗಳ ವ್ಯತ್ಯಾಸಕ್ಕೆ ಕಾರಣ ವಾದುದು. ಈ ವ್ಯತ್ಯಾಸದಿಂದಲೂ ಜೀವಾತ್ಮರ ಅದೃಷ್ಟದಿಂದಲೂ ಈಶ್ವರಾಧಿಷ್ಠಿತವಾದ ಅವ್ಯಕ್ತದಿಂದ ಮಹತ್​ತತ್ವ ಹುಟ್ಟಿತು. ಸತ್ವ ರಜೋಗುಣಗಳಿಗೆ ಮಾತ್ರವೇ ನೆಲೆಯಾದ ಆ ಮಹತ್ತತ್ವದಿಂದ ತಮೋಗುಣ ಪ್ರಧಾನವಾದ ಅಹಂಕಾರ ಜನಿಸಿತು. ಈ ಅಹಂಕಾರ ತತ್ವವು ಸಾತ್ವಿಕಾಹಂಕಾರ, ರಾಜಸಾಹಂಕಾರ, ತಾಮಸಾಹಂಕಾರವೆಂಬ ಮೂರು ವಿಕಾರ ಗಳನ್ನು ಪಡೆಯಿತು. ಈ ಮೂರರಲ್ಲಿ ತಾಮಸಾಹಂಕಾರದಿಂದ ಶಬ್ದವು ಜನಿಸಿ, ಅದರಿಂದ ಆಕಾಶ ಹುಟ್ಟಿತು. ಆಕಾಶದಿಂದ ವಾಯು, ವಾಯುವಿನಿಂದ ತೇಜಸ್ಸು, ತೇಜಸ್ಸಿನಿಂದ ಜಲ, ಜಲದಿಂದ ಪೃಥ್ವಿ ಜನಿಸಿದವು. ಬಳಿಕ ಸಾತ್ವಿಕಾಹಂಕಾರದಿಂದ ಮನಸ್ಸೂ, ಐದು ಕರ್ಮೇಂದ್ರಿಯಗಳು ಮತ್ತು ಐದು ಜ್ಞಾನೇಂದ್ರಿಯಗಳ ಅಧಿದೇವತೆಗಳೂ ಜನಿಸಿದರು. ರಾಜಸಾಹಂಕಾರದಿಂದ ಪ್ರಾಣ, ಐದು ಜ್ಞಾನೇಂದ್ರಿಯಗಳು ಮತ್ತು ಐದು ಕರ್ಮೇಂದ್ರಿಯಗಳು ಹುಟ್ಟಿದವು. ಭಗವಂತನ ಸಂಕಲ್ಪರೂಪವಾದ ಶಕ್ತಿಯಿಂದ ಈ ಗುಣ ಗಳೆಲ್ಲವೂ ಪರಸ್ಪರ ಸೇರಿ ಬ್ರಹ್ಮಾಂಡವೆಂಬ ಶರೀರ ನಿರ್ಮಾಣವಾಗಿ, ಅದು ಮಹಾಜಲ ದಲ್ಲಿ ತೇಲುತ್ತಿತ್ತು. ಅಚೇತನವಾಗಿದ್ದ ಈ ಬ್ರಹ್ಮಾಂಡವನ್ನು ಭಗವಂತನು ಚೇತನಗೊಳಿ ಸಿದನು. ಅನಂತರ ಆ ಭಗವಂತನೇ ಆ ಬ್ರಹ್ಮಾಂಡವನ್ನು ಒಡೆದು ಕೊಂಡು, ಸಹಸ್ರಾರು ಕಾಲು, ತೊಡೆ,ತೋಳು, ಕಣ್ಣು, ತಲೆಗಳೊಡನೆ ಆವಿರ್ಭವಿಸಿದನು. ಆತನ ಮುಖದಿಂದ ಬ್ರಾಹ್ಮಣನೂ, ತೋಳುಗಳಿಂದ ಕ್ಷತ್ರಿಯನೂ, ತೊಡೆಗಳಿಂದ ವೈಶ್ಯನೂ, ಪಾದಗಳಿಂದ ಶೂದ್ರನೂ ಉದಯಿಸಿದರು. ಆತನ ಅವಯವ ಸ್ವರೂಪವಾಗಿ ಏಳು ಊರ್ಧ್ವ ಲೋಕ ಗಳೂ ಏಳು ಅಧೋಲೋಕಗಳೂ ಉಂಟಾದುವು. ಈ ವಿರಾಟ್​ಪುರುಷನೇ ಭಗವಂತನ ಮೊದಲ ಅವತಾರ.

“ ಅಯ್ಯಾ, ನಾರದ, ಲೋಕದ ಮಾತಿಗೆಲ್ಲಾ ಆತನ ಮುಖವೇ ಮೂಲ; ಹವ್ಯಕವ್ಯ ಅಮೃತಾದಿಗಳು ಆತನ ನಾಲಗೆಯಿಂದ ಜನಿಸಿದವು; ವಾಯುವೂ ಪ್ರಾಣವಾಯುವೂ ಔಷಧಿಗಳೂ ವಾಸನೆಯೂ ಆತನ ಮೂಗಿನಿಂದ ಬಂದವು; ರೂಪು, ತೇಜಸ್ಸು, ಸೂರ್ಯ, ಸ್ವರ್ಗಗಳು ಆತನ ಕಣ್ಣುಗಳಿಂದ ಹುಟ್ಟಿದವು; ದಿಕ್ಕು, ಶಾಸ್ತ್ರಗಳು, ಶಬ್ದ ಆತನ ಕಿವಿಯಿಂದ ಉದಿಸಿದವು; ಚರ್ಮ, ಸ್ಪರ್ಶಗುಣ, ಯಜ್ಞಗಳಿಗೆ ಆತನ ಚರ್ಮವೇ ಕಾರಣ. ವೃಕ್ಷಗಳೇ ಆತನ ರೋಮಗಳು, ಮೇಘಗಳೇ ಕೇಶಗಳು, ಮಿಂಚು ಗಡ್ಡ ಮೀಸೆ; ಶಿಲೆ, ಲೋಹಗಳು ಆತನ ಉಗುರುಗಳು, ಇಂದ್ರಾದಿಗಳು ಬಾಹುಗಳು, ಮೂರು ಲೋಕಗಳೇ ಆತನ ನಡಗೆ; ಕ್ಷೇಮ, ಅಭಯ, ವರಪ್ರಧಾನಗಳು ಆತನ ಪಾದಗಳು, ನದಿಗಳು ನರಗಳು, ಬೆಟ್ಟಗಳು ಎಲುಬುಗಳು; ಸಮುದ್ರ, ಸಕಲ ಭೂತಗಳೂ ಮತ್ತು ಪ್ರಳಯ ಆತನ ಹೊಟ್ಟೆ; ಪ್ರಾಣಿಗಳ ಲಿಂಗಶರೀರ ಆತನ ಹೃದಯ; ನೀರು, ಶುಕ್ಲ, ಸೃಷ್ಟಿ ಮೊದಲಾದುವುಗಳಿಗೂ, ಯಮ ಮಿತ್ರರಿಗೂ, ಮೃತ್ಯು ನರಕಗಳಿಗೂ ಆತನ ದೇಹದ ಬೇರೆ ಬೇರೆ ಭಾಗಗಳೇ ಕಾರಣ. ಅಯ್ಯಾ ನಾರದ, ನಾನು, ನೀನು, ಸನಕಾದಿಗಳು, ರುದ್ರ–ಎಲ್ಲರೂ ಆತನ ಚಿತ್ತದಿಂದ ಜನಿಸಿದವರು. ಆದ್ದರಿಂದ ಆ ವಿರಾಡ್ರೂಪನಿಂದ ಜನಿಸಿದ ಈ ಜಗತ್ತೆಲ್ಲವೂ ಆತನಿಂದ ಬೇರೆಯಲ್ಲ; ಎಲ್ಲವೂ ಆತನೇ. ಹಿಂದಣ ಕಲ್ಪದಲ್ಲಿದ್ದ ಜಗತ್ತೂ, ಮುಂದಣ ಕಲ್ಪದಲ್ಲಿ ಬರಬಹುದಾದ ಜಗತ್ತೂ, ಇಂದಿನ ಜಗತ್ತೂ ಇದೆಲ್ಲವೂ ಆತನೇ. ಅಷ್ಟೇ ಅಲ್ಲ, ಇದಕ್ಕೂ ಮಿಗಿಲಾಗಿರುವನು. ಆತ ಪ್ರಪಂಚಾತ್ಮಕ, ಪ್ರಪಂಚಾ ತೀತ, ಮುಕ್ತಿನಾಯಕ. ನಾನು ಆತನ ಅಪ್ಪಣೆಯಂತೆ ಸೃಷ್ಟಿಕಾರ್ಯವನ್ನು ನಡಸುವೆನು; ರುದ್ರನು ಆತನ ಆಜ್ಞೆಯಂತೆ ಸಂಹಾರ ಕಾರ್ಯದಲ್ಲಿ ತೊಡಗಿರುವನು; ಆತನೇ ವಿಷ್ಣುರೂಪವನ್ನು ಧಅರಿಸಿ ಜಗತ್ತನ್ನು ರಕ್ಷಿಸುತ್ತಿರು ವನು. ಆತನ ಮಾಯೆಯು ಜಗತ್ತನ್ನೆಲ್ಲಾ ಆವರಿಸಿರುವುದರಿಂದ ಯಾರೂ ಆತನನ್ನು ಅರಿಯಲಾರರು. ಈ ಮಾಯೆಯಿಂದ ಉದ್ಧಾರವಾಗಲು ಒಂದೇ ದಾರಿ, ಆತನ ನಾಮ ಸಂಕೀರ್ತನೆ’ ಎಂದು ಹೇಳಿದನು. ಅನಂತರ ಬ್ರಹ್ಮನು ನಾರದನಿಗೆ ಭಗವದವತಾರ ಗಳನ್ನೂ, ಬೇರೆಬೇರೆ ಅವತಾರಗಳಲ್ಲಿ ಆದ ಕಾರ್ಯಕಲಾಪಗಳನ್ನೂ ವಿವರಿಸಿ ಹೇಳಿದನು.

ಶುಕಮುನಿಯು ಹೇಳಿದ ಬ್ರಹ್ಮ-ನಾರದರ ಸಂಭಾಷಣೆಯನ್ನು ಕೇಳಿ ಪರೀಕ್ಷಿದ್ರಾಜನು ಅತ್ಯಂತ ಸಂತೋಷದಿಂದ ‘ಮಹಾತ್ಮಾ, ಸಾಧುರಕ್ಷಕನಾದ ಭಗವಂತನ ಚರಿತ್ರೆಯನ್ನು ನನಗೆ ಸಮಗ್ರವಾಗಿ ತಿಳಿಸು. ಬ್ರಹ್ಮ, ನಾರದ, ವ್ಯಾಸರ ಪರಂಪರೆಯಿಂದ ನೀನು ಕೇಳಿರು ವವನಾದ್ದರಿಂದ ನನಗೆ ಕ್ರಮವಾಗಿ ಅದನ್ನು ತಿಳಿಸು. ಅಚ್ಯುತಚರಿತ್ರೆಯೆಂಬ ಅಮೃತಪಾನ ಮಾಡುತ್ತಿರುವ ನನಗೆ ಹಸಿವು ಬಾಯಾರಿಕೆಗಳಿಲ್ಲ; ಬ್ರಾಹ್ಮಣ ಶಾಪದ ಭಯವೂ ಹಾರಿ ಹೋಯಿತು’ ಎಂದು ಹೇಳಿದನು. ಆಗ ಶುಕ ಮಹರ್ಷಿಯು ಭಗವಂತನು ಬ್ರಹ್ಮನಿಗೆ ಉಪದೇಶಿಸಿದ, ವೇದಕ್ಕೆ ಸಮಾನವಾದ ಭಾಗವತವನ್ನು ಹೇಳಲು ಪ್ರಾರಂಭಿಸಿದನು.

