
\chapter{೯೧. ಭಾಗವತಧರ್ಮ}

ಬ್ರಹ್ಮನೇ ಮೊದಲಾದ ದೇವತೆಗಳು ಶ್ರೀಕೃಷ್ಣನಿಂದ ಬೀಳ್ಕೊಂಡು ಅತ್ತ ಹೋಗು ತ್ತಲೆ, ಇತ್ತ ನಾರದರ ಸವಾರಿ ಶ್ರೀಕೃಷ್ಣದರ್ಶನಕ್ಕಾಗಿ ದ್ವಾರಕಿಗೆ ಬಂತು. ಆತನು ಮನೆಯ ಬಾಗಿಲಿಗೆ ಹೋಗುತ್ತಿದ್ದಂತೆಯೆ ವಸುದೇವನು ಸಡಗರದಿಂದ ಆತನನ್ನು ಇದಿರು ಗೊಂಡು, ಭಕ್ತಿಯಿಂದ ಅವರನ್ನು ಉಪಚರಿಸಿದ ಮೇಲೆ, ‘ಸ್ವಾಮಿ, ತಾವು ದಯಮಾಡಿಸಿ ದಿರೆಂದರೆ ತಂದೆಯೆ ಬಂದಷ್ಟು ಸಂತೋಷ ನನಗೆ. ನೀವು ಸಹಜ ದಯಾಳುಗಳು, ಲೋಕಾನುಗ್ರಹವೇ ನಿಮ್ಮ ನಿತ್ಯನಿಧಿ. ನಿಮ್ಮ ದರ್ಶನದಿಂದಲೆ ನಾನು ಧನ್ಯ. ನೀವು ದಯಮಾಡಿ, ಮೋಕ್ಷಪ್ರದವಾದ ಭಾಗವತ ಧರ್ಮವನ್ನು ನನಗೆ ಬೋಧಿಸಬೇಕು’ ಎಂದು ಬೇಡಿಕೊಂಡನು. ಕೇಳುವವನ ಆಸಕ್ತಿಗೆ ತಕ್ಕಂತೆ ಹೇಳುವವನ ಉತ್ಸಾಹ. ನಾರದರು ಅತ್ಯಂತ ಸಂತೋಷದಿಂದ ಆತನ ಅಪೇಕ್ಷೆಯನ್ನು ಸಲ್ಲಿಸಲು ಸಿದ್ಧರಾಗಿ “ಅಯ್ಯಾ, ವಸುದೇವ! ಸ್ವಾಯಂಭುವ ಮನುವಿನ ವಂಶದಲ್ಲಿ ಸಾಕ್ಷಾತ್ ಭಗವಂತನೆ ಮೋಕ್ಷಧರ್ಮ ವನ್ನು ಪ್ರಚಾರ ಮಾಡುವುದಕ್ಕಾಗಿ ಪುಷಭನೆಂಬ ಹೆಸರಿನಿಂದ ಅವತರಿಸಿದ. ಆತನಿಗೆ ನೂರು ಜನ ಮಕ್ಕಳು. ಅವರಲ್ಲಿ ಹಿರಿಯ ಭರತ. ಪರಮ ಭಾಗವತನಾದ ಆತನಿಂದಲೆ ಈ ನಮ್ಮ ದೇಶಕ್ಕೆ ಭರತಖಂಡವೆಂಬ ಹೆಸರು ಬಂದುದು. ಅವನ ತೊಂಬತ್ತೊಂಬತ್ತು ಜನ ತಮ್ಮಂದಿರಲ್ಲಿ ಒಂಬತ್ತು ಜನ ಒಂಬತ್ತು ದ್ವೀಪಗಳ ಚಕ್ರವರ್ತಿಗಳಾದರು; ಎಂಬ ತ್ತೊಂದು ಜನ ಶ್ರೋತ್ರಿಯ ಬ್ರಾಹ್ಮಣರಾದರು; ಉಳಿದ ಒಂಬತ್ತು ಜನ ಪರಿವ್ರಾಜಕ ರಾಗಿ, ಅಧ್ಯಾತ್ಮ ಜ್ಞಾನವನ್ನು ಲೋಕಕ್ಕೆ ಬೋಧಿಸುವವರಾದರು. ಅವರ ಹೆಸರು ಕ್ರಮ ವಾಗಿ ಕವಿ, ಹರಿ, ಅಂತರಿಕ್ಷ, ಪ್ರಬುದ್ಧ, ಪಿಪ್ಪಲಾಯನ, ಆವಿರ್ಹೋತ, ದ್ರಮಿಳ, ಚಮಸ, ಕರಭಾಜನ ಎಂದು. ಸದಾ ಸಂಚರಿಸುತ್ತಿದ್ದ ಅವರಿಗೆ ಯಾವ ಲೋಕಕ್ಕೆ ಬೇಕಾ ದರೂ ಹೋಗುವ ಶಕ್ತಿಯಿತ್ತು. ಅವರೊಮ್ಮೆ ಜನಕರಾಜನ ವಂಶದ ನಿಮಿ ಮಹಾರಾಜನು ಸತ್ರಯಾಗ ಮಾಡುತ್ತಿದ್ದ ವಿದೇಹ ನಗರಿಗೆ ಹೋದರು. ಅಲ್ಲಿ ಅವರು ನಿಮಿಯ ಪ್ರಾರ್ಥನೆ ಯಂತೆ ಆತನಿಗೆ ಭಾಗವತ ಧರ್ಮವನ್ನು ಉಪದೇಶಿಸಿದರು. ಅದನ್ನೇ ನಾನೀಗ ನಿನಗೆ ತಿಳಿಸುತ್ತೇನೆ.

“ಜೀವನಿಗೆ ಮುಕ್ತಿಯೆ ಹೆಗ್ಗುರಿ. ಮುಕ್ತಿಯೆಂದರೆ ಲೋಕದಲ್ಲಿ ಯಾವ ಭಯವೂ ಇಲ್ಲದೆ ನಿಶ್ಚಿಂತೆಯಾಗಿರುವುದು. ಭಯಕ್ಕೆ ಮೂಲಕಾರಣವೇನು? ಈ ದೇಹವೇ ನಾನು, ಆತ್ಮವೆಂಬ ಭ್ರಮೆ. ಈ ಭ್ರಮೆ ನೀಗಿ, ಜೀವನು ನಿರ್ಭಯನಾಗಬೇಕಾದರೆ ಆತ್ಮ, ಪರ ಮಾತ್ಮರ ಸ್ವರೂಪವನ್ನು ತಿಳಿದು ಕೊಳ್ಳಬೇಕು. ಹಾಗೆ ತಿಳಿದುಕೊಳ್ಳಲಾರದ ಅಜ್ಞರಿಗಾಗಿ ಅತ್ಯಂತ ಸುಲಭವಾದ ಒಂದು ಹಾದಿಯಿದೆ. ಅದೇ ಭಗವಂತನ ಕಥಾಶ್ರವಣ, ಸ್ಮರಣ, ಕೀರ್ತನ. ಇದನ್ನೇ ಭಾಗವತಧರ್ಮವೆಂದು ಹೇಳುತ್ತಾರೆ. ಮೋಕ್ಷವನ್ನು ಮುಟ್ಟಬೇಕೆನ್ನು ವವನಿಗೆ ಇತರ ಉಪಾಯಗಳಿವೆಯಾದರೂ ಇದು ಅತ್ಯಂತ ಸುಲಭವಾದ ಉಪಾಯ, ಕಣ್ಣು ಮುಚ್ಚಿಕೊಂಡು ಓಡಿದರೂ ಇಲ್ಲಿ ಎಡವುತ್ತೇನೆಂಬ ಭಯವಿಲ್ಲ. ತಾನು ಮಾಡುವು ದೆಲ್ಲವನ್ನೂ ‘ಕೃಷ್ಣಾರ್ಪಣ’ ಬುದ್ಧಿಯಿಂದ ಮಾಡಿದರಾಯಿತು, ಅಷ್ಟೆ. ‘ನಾನು ಮಾಡು ತ್ತೇನೆ’ ಎಂದುಕೊಂಡಾಗ ಭಯ. ಅದೇ ಮಾಯೆ, ಅದರಿಂದ ತಪ್ಪಿಸಿಕೊಳ್ಳಬೇಕಾದರೆ ‘ಇದನ್ನು ಮಾಡುತ್ತಿರುವವನು ‘ನಾನು’ ಎಂಬ ಈ ದೇಹವಲ್ಲ ಎಂಬ ನಿಶ್ಚಯಜ್ಞಾನ ಹುಟ್ಟಬೇಕು. ದೇವದೇವನ ಕಥಾಶ್ರವಣ, ಸ್ಮರಣ, ಕೀರ್ತನೆಗಳಿಂದ ಈ ನಿಶ್ಚಯಜ್ಞಾನ ಹುಟ್ಟುತ್ತದೆ. ಇದರ ಸಾಧನೆಗೆ ‘ಭಕ್ತಿ’ ಎಂದು ಹೇಳುತ್ತಾರೆ. ಭಕ್ತಿಯೆಂದರೆ ಭಗವಂತನ ಮೇಲಿನ ಪ್ರೇಮ. ಈ ಪ್ರೇಮ ಹೆಚ್ಚಿದಂತೆಲ್ಲ ದೇಹದ ಮೇಲಿನ ಪ್ರಜ್ಞೆ ತಪ್ಪುತ್ತದೆ. ಅವನು ಕುಣಿಯುತ್ತಾನೆ, ಹಾಡುತ್ತಾನೆ, ನಗುತ್ತಾನೆ, ಅಳುತ್ತಾನೆ, ಭೂತಹಿಡಿದವನಂತೆ ಆಡುತ್ತಾನೆ. ಹರಿಸಂಕೀರ್ತನೆಯ ಜೊತೆಗೆ ಜಗತ್ತಿನಲ್ಲಿ ಗೋಚರಿಸುವುದೆಲ್ಲವೂ ಭಗವತ್ ಸ್ವರೂಪಿ ಯೆಂಬ ಭಾವನೆ ಬೆಳೆಯಬೇಕು. ಕಣ್ಣಿಗೆ ಕಂಡುದೆಲ್ಲ ಭಗವಂತನೆಂದೆ ಭಾವಿಸಿ, ನಮಸ್ಕ ರಿಸಬೇಕು. ತನಗೆ ತಾನೆ ನಮಸ್ಕರಿಸಿಕೊಳ್ಳಬಹುದು. ಹೀಗೆ ಸದಾ ಹರಿಧ್ಯಾನತತ್ಪರ ನಾಗಿದ್ದರೆ ತತ್ವಜ್ಞಾನ, ಭಕ್ತಿ, ವಿರಕ್ತಿಗಳೂ ಏಕಕಾಲಕ್ಕೆ ಹುಟ್ಟುತ್ತವೆ. ಇಷ್ಟಾಯಿತೆಂದರೆ ಮನಶ್ಶಾಂತಿ ನೆಲೆಗೊಳ್ಳುತ್ತದೆ. ಮೋಕ್ಷ ಕರಗತವಾಗುತ್ತದೆ.

“ಭಗವಂತನ ಭಕ್ತರು ಭಾಗವತರು. ಇವರಲ್ಲಿ ಮೂರು ವಿಧ. ಜಗತ್ತೆಲ್ಲವನ್ನೂ– ತನ್ನನ್ನು ಕೂಡ–ಪರಮೇಶ್ವರನೆಂದು ಭಾವಿಸಿ ಪೂಜಿಸುವವನು ಭಾಗವತೋತ್ತಮ. ಹಾಗೆ ತಿಳಿಯಲಾರದೆ ಹೋದರೂ, ಭಕ್ತಿಯಿಂದ ಭಗವಂತನನ್ನು ಪೂಜೆ ಮಾಡುತ್ತಾ, ಭಾಗವತರ ಸಹವಾಸದಲ್ಲಿದ್ದುಕೊಂಡು, ಅಜ್ಞರಲ್ಲಿ ಕನಿಕರವನ್ನೂ ಶತ್ರುಗಳಲ್ಲಿ ಔದಾರ್ಯವನ್ನೂ ತೋರಬಲ್ಲವನು ಭಾಗವತರಲ್ಲಿ ಮಧ್ಯಮ. ಕೇವಲ ಭಗವಂತನ ವಿಗ್ರಹವೇ ಭಗವಂತ ನೆಂದು ಭಾವಿಸಿ, ಅದರ ಪೂಜೆಯಲ್ಲಿ ಶ್ರದ್ಧೆಯಿಂದ ತತ್ಪರನಾಗುವವನು ಮೂಢಭಕ್ತ ನೆನಿಸಿಕೊಳ್ಳುವನು. ಉತ್ತಮನಾದ ಭಾಗವತನು ನಡೆಯುವುದೆಲ್ಲ ವಿಷ್ಣುಮಾಯೆಯೆಂದು ಗ್ರಹಿಸಿ, ಇಷ್ಟಾನಿಷ್ಟಗಳಲ್ಲಿ ಆಶೆ ಜಿಹಾಸೆಗಳಿಲ್ಲದೆ ನಿರ್ವಿಕಾರನಾಗಿರುವನು. ಆತನು ಹಸಿವು, ಬಾಯಾರಿಕೆ, ಭಯ, ಶ್ರಮ, ಇವಾವನ್ನೂ ಗಮನಿಸದೆ ಹರಿಧ್ಯಾನದಲ್ಲಿ ತತ್ಪರ ನಾಗಿರುವನು; ಭಗವಂತನೆ ಗತಿಯೆಂದು ನಂಬಿ ನಡೆಯುತ್ತಾ ಕರ್ಮಬಂಧಕ್ಕೆ ಅವಕಾಶ ವನ್ನೆ ಕೊಡುವುದಿಲ್ಲ, ಆತ. ಜಾತಿ, ವರ್ಣ, ಆಶ್ರಮ–ಎಂಬ ಯಾವ ಅಹಂಕಾರವೂ ಆತನಿಗಿರುವುದಿಲ್ಲ; ತನ್ನದು ಪರರದೆಂಬ ಭೇದಬುದ್ಧಿ ಆತನಲ್ಲಿ ಸುಳಿಯದು; ಆತನು ಸದಾ ಸಮಚಿತ್ತನಾಗಿ ಸರ್ವಸಮಾನಭಾವದಿಂದಿರುವನು. ಎಂತಹ ಸುಖಸೌಭಾಗ್ಯಗಳು ಬಂದರೂ ಆತನ ಹರಿಭಜನೆಗೆ ಅಪೋಹವಿಲ್ಲ. ಚಂದ್ರೋದಯವಾಗುತ್ತಲೆ ಬಿಸಿಲ ತಾಪ ಅಡಗುವಂತೆ ಭಗವಂತನ ಪಾದಕಮಲ ಚಂದ್ರಿಕೆ ಹೃದಯದಲ್ಲಿ ಮೂಡುತ್ತಲೆ ಸಮಸ್ತ ಸಂಸಾರತಾಪಗಳೂ ನಿವಾರಣೆಯಾಗುತ್ತವೆ. ಭಾಗವತೋತ್ತಮನು ತನ್ನ ಭಕ್ತಿಪಾಶದಿಂದ ತನ್ನ ಇಷ್ಟದೇವತೆಯನ್ನು ತನ್ನ ಹೃದಯದಲ್ಲಿಯೆ ಕಟ್ಟಿ ಹಾಕುತ್ತಾನೆ.

“ಜಗತ್ತೆಲ್ಲವೂ ಮಾಯಾಮಯ. ಆತ್ಮನು ಸ್ವಭಾವತಃ ಶುದ್ಧ, ನಿರ್ಗುಣ, ಸ್ವಯಂ ಪ್ರಕಾಶ; ಆದರೆ ಶರೀರಸಂಬಂಧ ಬರುತ್ತಲೆ ಜೀವಾತ್ಮನು ಗುಣವಿಕಾರಗಳಿಗೆ ವಶನಾಗು ತ್ತಾನೆ; ದೇಹವನ್ನು ಆತ್ಮವೆಂದು ಭ್ರಮಿಸುತ್ತಾನೆ. ಇದೇ ಮಾಯೆ. ಭಗವಂತನ ಸೃಷ್ಟಿಗೆಲ್ಲ ಈ ಮಾಯೆಯೆ ಮೂಲವಸ್ತು. ಎಚ್ಚರ, ಕನಸು, ನಿದ್ರೆ–ಈ ಮೂರು ಅವಸ್ಥೆಗಳೂ ಮಾಯೆಯ ಕೆಲಸವೆ. ಎಚ್ಚರವಿದ್ದಾಗ ದೇಹಾಭಿಮಾನ ಕಾಡುತ್ತದೆ; ಸ್ವಪ್ನದಲ್ಲಿ ಬಾಹ್ಯೇಂ ದ್ರಿಯಗಳು ಸುಪ್ತಸ್ಥಿತಿಯಲ್ಲಿದ್ದರೂ, ಅಂತಃಕರಣ ಆ ಸ್ವಪ್ನಜಗತ್ತನ್ನು ಸತ್ಯವೆಂದೇ ಭ್ರಮಿಸುತ್ತದೆ; ಸುಷುಪ್ತಿಯಲ್ಲಿ ಒಳ–ಹೊರಗಿನ ಇಂದ್ರಿಯಜ್ಞಾನವಿರುವುದಿಲ್ಲವಾದರೂ ಕರ್ಮದ ಬೀಜಮಾತ್ರ ಇದ್ದೇ ಇರುತ್ತದೆ. ಆದ್ದರಿಂದ ಪ್ರಕೃತಿಸಂಬಂಧ ಅದಕ್ಕೆ ತಪ್ಪು ವುದಿಲ್ಲ. ಈ ಮೂರು ಅವಸ್ಥೆಗಳನ್ನೂ ಮೀರಿದುದು ಮುಕ್ತಾವಸ್ಥೆ. ಸ್ವಭಾವತಃ ಶುದ್ಧನಾದ ಆತ್ಮನಿಗೆ ಈ ಗುಣಗಳೆಲ್ಲ ಬರುವುದು ಅನಾದಿಯಾದ ಕರ್ಮವಾಸನೆಯಿಂದ, ಈ ವಾಸನೆಯ ಫಲವಾಗಿ ಜೀವಾತ್ಮನು ಪುಣ್ಯ-ಪಾಪಕರ್ಮಗಳನ್ನು ಮಾಡುತ್ತಾನೆ. ಇದು ಅವನ ಪುನರ್ಜನ್ಮಕ್ಕೆ ಕಾರಣವಾಗುತ್ತದೆ. ಕರ್ಮ ಕರ್ಮಕ್ಕೆ ತಕ್ಕ ಜನ್ಮ; ಪುನರಪಿ ಜನನಂ ಪುನರಪಿ ಮರಣಂ. ಪ್ರಳಯಕಾಲದವರೆಗೂ ಇದು ನಡೆಯುತ್ತಾ ಹೋಗುತ್ತದೆ. ಪ್ರಳಯ ಬಂದಾಗ ಸ್ಥೂಲ ಪ್ರಪಂಚವೆಲ್ಲವೂ ಮಾಯೆಯಲ್ಲಿ ಲೀನವಾಗುತ್ತದೆ.

ಹಾಗಾದರೆ ಈ ಮಾಯೆಯಿಂದ ಉದ್ಧಾರವೆಂತು? ಮನುಷ್ಯ ಮಾಯೆಯ ಮಬ್ಬಿನಲ್ಲಿ ಸುಖಾಭಿಲಾಷಿಯಾಗಿ ಅದನ್ನು ಅರಸುತ್ತಾ ಹೊರಡುತ್ತಾನೆ. ಆ ಬಿಸಿಲ್​ಗುದುರೆಯ ಬೆನ್ನು ಹತ್ತಿ ದೇಹ ದಂಡನೆಯಿಂದ ಹಣ ಸಂಪಾದಿಸುತ್ತಾನೆ, ಹೆಂಡತಿ ಮಕ್ಕಳೆಂದು ಹಗಲಿರುಳು ಹಿಂಸೆಪಡುತ್ತಾನೆ. ಸುಖದಂತೆ ಕಂಡ ಇದೆಲ್ಲ ದುಃಖದಲ್ಲಿ ಪರ್ಯವಸಾನವಾಗುತ್ತದೆ. ಇದರಂತೆಯೆ ಯಜ್ಞಯಾಗಾದಿ ಕರ್ಮಗಳ ಫಲವೂ. ಯಜ್ಞಮಾಡಿ ಸ್ವರ್ಗವೇರಿದವರು ‘ಕ್ಷೀಣೇ ಪುಣ್ಯೇ ಮರ್ತ್ಯಲೋಕಂ ವಿಶಂತಿ’ ಯಜ್ಞಕರ್ಮವು ವೇದವಿಹಿತವಾಗಿದ್ದರೂ, ಮಗುವಿಗೆ ಸಕ್ಕರೆ ತೋರಿಸಿ ಔಷಧ ಕುಡಿಸುವಂತೆ, ಫಲಾಪೇಕ್ಷೆಯಿಂದ ಕರ್ಮಾಚರಣೆಗೆ ಪ್ರೋತ್ಸಾಹಿಸುವುದಕ್ಕಾಗಿಯೇ ಹೊರತು, ಕಾಮ್ಯಫಲಕ್ಕಲ್ಲ. ನ್ಯಾಯವಾಗಿ ಯಜ್ಞಕರ್ಮವು ಭಕ್ತಿಯೋಗಕ್ಕೆ ಸಹಾಯಕವಾಗಿ ಕರ್ಮಕ್ಷಯದಿಂದ ಮೋಕ್ಷಹೇತುವಾಗಬೇಕು. ಭಗವ ದರ್ಪಣರೂಪವಾಗಿ ಮಾಡದ ಕರ್ಮ ವ್ಯರ್ಥ. ಯೋಗಕ್ಕಿಂತಲೂ ಶೀಘ್ರವಾಗಿ ಭಕ್ತಿ ಯೋಗವನ್ನು ಹುಟ್ಟಿಸುವ ಕರ್ಮಗಳು ಪಂಚರಾತ್ರದಲ್ಲಿ ನಿರೂಪಿತವಾಗಿವೆ. ಅವು ತಾಂತ್ರಿಕ ಕರ್ಮಗಳು. ಪಂಚರಾತ್ರವಿಧಿಯಿಂದ ಭಗವಂತನನ್ನು ಆರಾಧಿಸುವ ಮುನ್ನ, ಗುರುವನ್ನು ಆಶ್ರಯಿಸಿ, ಆತನೆ ಪರದೈವವೆಂಬ ಭಾವನೆಯಿಂದ ಸೇವಿಸಿ, ಪೂಜಾವಿಧಿ ಗಳನ್ನು ಉಪದೇಶವನ್ನು ಆತನಿಂದ ಪಡೆಯಬೇಕು. ಪ್ರಾಣಾಯಾಮದಿಂದ ದೇಹಶುದ್ಧಿ ಯನ್ನು ಪಡೆದು, ಪ್ರತಿಮೆಯಲ್ಲಿ ಭಗವಂತನನ್ನು ಭಾವಿಸಿ, ಪೂಜಾ ಸಾಮಗ್ರಿಗಳೊಡನೆ ಕ್ರಮಪೂಜೆಯನ್ನು ಮಾಡಬೇಕು. ಭಗವಂತನಿಗೆ ನಿವೇದಿಸಿದ ಆಹಾರವನ್ನು ತಿನ್ನಬೇಕು. ಪೂಜಾಮೂರ್ತಿಯನ್ನು ಹೃದಯದಲ್ಲಿ ನೆಲೆಗೊಳಿಸಬೇಕು. ಹೀಗೆ ಮಾಡಿದಾಗ ಮಾಯೆ ಮಾಯವಾಗಿಹೋಗುತ್ತದೆ.

“ಭಗವಂತನ ಸ್ವರೂಪವೇನು? ಆತನ ಸ್ವರೂಪ ಸ್ವಭಾವಗಳನ್ನು ತಿಳಿಸುವುದು ಅಸಾಧ್ಯ. ‘ನ ಚಕ್ಷುಷಾ ದೃಶ್ಯತೇ’, ‘ನಾಪಿವಾಚಾ’ ಎಂದು ವೇದಾಂತ ಹೇಳುತ್ತದೆ. ಚೇತನಾ ಚೇತನ ಜಗತ್ತೆಲ್ಲವೂ ಆತನೆ. ಅದರ ಸೃಷ್ಟಿ ಸ್ಥಿತಿ ಲಯಗಳನ್ನು ಮಾಡುವವನೂ ಆತನೆ. ಆತ ಸರ್ವಾಂತರ್ಯಾಮಿಯಾಗಿಯೂ ಇದ್ದಾನೆ. ದೇಹ, ಇಂದ್ರಿಯ, ಪ್ರಾಣ, ಮನಸ್ಸು– ಇವೆಲ್ಲ ಯಾರ ಪ್ರೇರಣೆಯಿಂದ ನಡೆಯುತ್ತವೆಯೊ ಆತನೇ ಪರಬ್ರಹ್ಮ, ಪರಾತ್ಪರ, ಪರ ಮೇಶ್ವರ. ಎಲ್ಲ ಆಸೆಗಳನ್ನೂ ಜಯಿಸಿ, ಚಿತ್ತಚಾಂಚಲ್ಯವಿಲ್ಲದವನ ಶುದ್ಧಮನಸ್ಸಿನಲ್ಲಿ ಆತ ಗೋಚರಿಸುವನು. ಆದರೆ ಆತನ ಸ್ವರೂಪವನ್ನು ಬಣ್ಣಿಸುವುದು ಮಾತಿಗೆ ಅತೀತ. ಆತನ ಸ್ವರೂಪ ಸ್ವಭಾವಗಳು ಅಗಣಿತ, ಅನಂತ. ಭೂಮಿಯ ಧೂಳಿನ ಕಣಗಳನ್ನು ಎಣಿಸ ಬಹುದು; ಆತನ ಗುಣರೂಪಗಳ ಗಣನೆ ಅಸಾಧ್ಯ. ಆದರೆ ಆತನು ಹಲವಾರು ಅವತಾರ ಗಳನ್ನು ಎತ್ತಿ, ಮಾನವರಿಗೆ ಗೋಚರಿಸುತ್ತಾನೆ. ಈ ಬ್ರಹ್ಮಾಂಡವೆ ಆತನ ಮೊದಲ ಅವತಾರ. ಅದನ್ನು ವಿರಾಟ್​ರೂಪವೆನ್ನುತ್ತಾರೆ. ಆತನ ಎರಡನೆಯ ಅವತಾರವೆ ನರ- ನಾರಾಯಣಾವತಾರ. ಇಂದಿಗೂ ಈ ಅವತಾರಪುರುಷರು ಬದರಿಕಾಶ್ರಮದಲ್ಲಿ ತಪೋ ನಿರತರಾಗಿದ್ದಾರೆ. ದತ್ತಾತ್ರೇಯ, ಹಯಗ್ರೀವ, ಮತ್ಸ್ಯ, ಕೂರ್ಮ, ವರಾಹ, ನಾರಸಿಂಹ, ರಾಮ, ಕೃಷ್ಣ ಇತ್ಯಾದಿ ಅವತಾರಗಳನ್ನೆತ್ತಿ, ಆತ ಧರ್ಮ ಸಂಸ್ಥಾಪನೆಯನ್ನು ಮಾಡಿದ್ದಾನೆ. ಭಾಗವತನಾದವನು ಆತನ ಅವತಾರಗಳನ್ನೂ, ಆಯಾ ಅವತಾರಗಳಲ್ಲಿ ನಡೆಸಿದ ಮಹ ತ್ಕಾರ್ಯಗಳನ್ನೂ ಹಾಡಿ ಹೊಗಳಿ ಧನ್ಯನಾಗುತ್ತಾನೆ.

“ಭಗವಂತನ ಸಾಕ್ಷಾತ್ಕಾರಕ್ಕೆ ಇಂದ್ರಿಯಜಯ ಅಗತ್ಯ. ಹಾಗೆ ಜಯಿಸಲಾರದ ನರಾ ಧಮರ ಗತಿಯೇನು? ಅಂತಹರೂ ನಿರಾಶರಾಗಬೇಕಾಗಿಲ್ಲ. ನಮ್ಮ ಕಲಿಯುಗದಲ್ಲಿ ಕೇವಲ ಭಗವತ್ ಸಂಕೀರ್ತನೆಯೊಂದರಿಂದಲೆ ಸಕಲ ಪುರುಷಾರ್ಥಗಳ ಸಾಧನೆಯಾಗಿ ಹೋಗುತ್ತದೆ. ದೋಷಗಳೆಷ್ಟಿದ್ದರೂ ಕಲಿಯುಗಕ್ಕೆ ಎಲ್ಲ ಯುಗಗಳಿಗಿಂತಲೂ ಹೆಚ್ಚು ಮಹತ್ವವಿದೆ. ಕೃತಯುಗದವರು ಕಲಿಯುಗದಲ್ಲಿ ಜನ್ಮವೆತ್ತಬೇಕೆಂದು ಬಯಸುತ್ತಾರೆ; ಏಕೆಂದರೆ ಈ ಯುಗದಲ್ಲಿ ಮೋಕ್ಷಸಂಪಾದನೆ ಹೆಚ್ಚು ಸುಲಭ. ಹಾಗೆ ಬಯಸಿದವರು ಪರಮ ಭಾಗವತೋತ್ತಮರಾಗಿ ಹುಟ್ಟುತ್ತಾರೆ. ಅದರಲ್ಲಿಯೂ ದಕ್ಷಿಣಭಾರತದಲ್ಲಿ ಅವರ ಸಂಖ್ಯೆ ಹೆಚ್ಚು. ಇಲ್ಲಿನ ನದಿಗಳ ನೀರನ್ನು ಪಾನಮಾಡಿದರೆ ಸಾಕು, ಮನಸ್ಸು ಪರಿಶುದ್ಧ ವಾಗುತ್ತದೆ. ಇಲ್ಲಿ ಹರಿಯನ್ನು ಸ್ತೋತ್ರ ಮಾಡಿದವರು ಮುಕ್ತರಾಗುತ್ತಾರೆ.”

ಇಷ್ಟು ಹೇಳಿದ ನಾರದರು ವಸುದೇವನನ್ನು ಕುರಿತು, “ಅಯ್ಯಾ, ನೀನು ಭಾಗವತನಾಗಿ ದೇಹ ಮೋಹವನ್ನು ತ್ಯಜಿಸಿದರೆ ಮುಕ್ತನಾಗುತ್ತಿ. ಇದು ಶಾಸ್ತ್ರರೀತಿಯಾಗಿ ಹೇಳುವ ಮಾತು. ಆದರೆ ನಿನಗೆ ಶಾಸ್ತ್ರ, ಗೀಸ್ತ್ರ ಒಂದೂ ಬೇಡ. ಸರ್ವೇಶ್ವರನಾದ ಶ್ರೀಹರಿ ನಿನ್ನ ಮಗನಾಗಿ ಹುಟ್ಟಿದ್ದಾನೆ. ಆತನನ್ನು ನೀನು ಪುತ್ರಪ್ರೇಮದಿಂದ ಆಲಿಂಗಿಸುತ್ತಿ, ಮಾತನಾಡುತ್ತಿ; ಹಗಲಿರುಳೂ–ಕೂತಾಗ, ನಿಂತಾಗ, ಊಟಮಾಡುವಾಗ–ಜೊತೆಯಲ್ಲಿ ಇರುತ್ತಿ. ಎಂದಮೇಲೆ ನಿನ್ನ ಮನಸ್ಸು ಪರಮಪಾವನವಾಗಿದೆ ಯೆಂಬುದರಲ್ಲಿ ಸಂದೇಹ ವಿಲ್ಲ. ಅಯ್ಯಾ, ನಿನ್ನ ಮಗ ಸರ್ವೇಶ್ವರನಯ್ಯ. ಇನ್ನು ಮುಂದೆ ಆತ ನಿನ್ನ ಮಗನೆಂಬ ಮೋಹವನ್ನು ತೊರೆ” ಎಂದರು. ಇದನ್ನು ಕೇಳಿ ವಸುದೇವನ ಕಣ್ಣು ತೆರೆಯಿತು. ಅಂದಿ ನಿಂದ ಆತನೂ ದೇವಕೀದೇವಿಯೂ ಶ್ರೀಕೃಷ್ಣನಲ್ಲಿ ಮೋಹವನ್ನು ತೊರೆದು, ಆತನನ್ನು ಪರಾತ್ಪರ ವಸ್ತುವೆಂದು ಗ್ರಹಿಸಿ, ಆರಾಧಿಸಹೊರಟರು.

