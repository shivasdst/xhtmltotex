
\chapter{೩೦. ಗಜೇಂದ್ರಮೋಕ್ಷ}

ಸ್ವಾಯಂಭುವಮನುವಿನ ವಂಶಾವಳಿಯನ್ನೂ ಆ ವಂಶದವರ ಇತಿಹಾಸವನ್ನೂ ಸಮಗ್ರವಾಗಿ ಕೇಳಿ ಸಂತೋಷಗೊಂಡ ಪರೀಕ್ಷಿದ್ರಾಜನು ಉಳಿದ ಮನುಗಳ ಇತಿಹಾಸ ಪರಂಪರೆಯನ್ನೂ ತಿಳಿಸುವಂತೆ ಶುಕಮುನಿಯನ್ನು ಬೇಡಿಕೊಂಡ. ಕರುಣಾಳುವಾದ ಆ ಮುನಿ ಇಲ್ಲಿಯವರೆಗೆ ಆಗಿ ಹೋಗಿರುವ ಮನುಗಳನ್ನು ಸ್ಥೂಲವಾಗಿ ಪರಿಚಯ ಮಾಡಿ ಕೊಡುತ್ತಾ, ನಾಲ್ಕನೆಯ ಮನುವಿನ ಕಾಲದಲ್ಲಿ ‘ಗಜರಾಜನನ್ನು ಮೊಸಳೆಯಿಂದ ಉದ್ಧರಿ ಸಿದ ಶ್ರೀಹರಿಯು ಈ ಮನ್ವಂತರದಲ್ಲಿ ಹರಿಮೇಧನ ಮಡದಿ ಹರಿಣೀದೇವಿಗೆ ಮಗನಾಗಿ ಹುಟ್ಟಿದನು’ ಎಂದು ತಿಳಿಸಿದನು. ಆ ಮಾತನ್ನು ಕೇಳುತ್ತಲೆ ಪರೀಕ್ಷಿದ್ರಾಜನಿಗೆ ಗಜರಾಜನು ಮೊಸಳೆಯಿಂದ ಉದ್ಧಾರವಾದ ಕಥೆಯನ್ನು ಕೇಳಬೇಕೆನ್ನಿಸಿತು. ಅದನ್ನು ವಿವರವಾಗಿ ತಿಳಿಸುವಂತೆ ಆತ ಶುಕಮುನಿಯನ್ನು ಕೇಳಿಕೊಂಡ. ಶುಕಮುನಿ ಅದನ್ನು ಹೇಳಿದ–

“ಹಾಲಿನ ಸಮುದ್ರದ ಮಧ್ಯದಲ್ಲಿ ತ್ರಿಕೂಟವೆಂಬ ಒಂದು ಪರ್ವತವಿದೆ. ಅದಕ್ಕೆ ರತ್ನ ಖಚಿತವಾದ ಅನೇಕ ಶಿಖರಗಳಿದ್ದರೂ, ಬೆಳ್ಳಿ, ಬಂಗಾರ, ಕಬ್ಬಿಣದಿಂದ ಆದ ಮೂರು ಶಿಖರಗಳು ಬಹು ಮುಖ್ಯ. ಆ ಶಿಖರಗಳ ಕಾಂತಿಯಿಂದ ಸುತ್ತುಮುತ್ತಿನ ಸಮುದ್ರವೆಲ್ಲ ತೊಳಗಿ ಬೆಳಗುತ್ತಿರುತ್ತದೆ. ಸಿದ್ಧ, ವಿದ್ಯಾಧರ, ಕಿನ್ನರ ಕಿಂಪುರುಷಾದಿಗಳಿಗೆ ನೆಲೆವನೆಯಾ ಗಿರುವ ಆ ಪರ್ವತದ ತಪ್ಪಲಲ್ಲಿ ವರುಣದೇವನ ಉದ್ಯಾನವನವಿದೆ. ಅದರ ಸೊಬಗು ವರ್ಣನಾತೀತ. ಮಂದಾರ, ಪಾರಿಜಾತ ಮೊದಲಾದ ಹೂ ಗಿಡಗಳೂ; ಮಾವು, ಹಲಸು ಮೊದಲಾದ ಹಣ್ಣಿನ ಗಿಡಗಳೂ ಸದಾ ಹೂ ಹಣ್ಣುಗಳಿಂದ ತುಂಬಿ ತುಳುಕಾಡುತ್ತಿರುತ್ತವೆ. ಅವುಗಳಲ್ಲಿ ಮನೆಮಾಡಿಕೊಂಡಿರುವ ನೂರಾರು ಬಗೆಯ ಬಣ್ಣಬಣ್ಣದ ಹಕ್ಕಿಗಳು ತಮ್ಮ ಇಂಪಾದ ಗಾನದಿಂದ ದಿಕ್ಕುಗಳನ್ನು ತುಂಬುತ್ತವೆ. ಆ ವನದ ಮಧ್ಯದಲ್ಲಿ ಬನದೇವಿಯ ಕೈಗನ್ನಡಿಯಂತಿರುವ ದಿವ್ಯವಾದ ಒಂದು ಸರೋವರ. ಅದರಲ್ಲಿ ಕಮಲ ಕನ್ನೈದಿಲೆಗಳು ಸದಾ ಅರಳಿ ದುಂಬಿಗಳನ್ನು ಆಹ್ವಾನಿಸುತ್ತವೆ. ಹಂಸ, ಚಕ್ರವಾಕ ಮೊದಲಾದ ನೀರು ಹಕ್ಕಿಗಳು ಅವುಗಳ ಮಧ್ಯೆ ಓಡಾಡುವಾಗ ಹೂವಿನ ಪರಾಗ ನೀರಿನಲ್ಲಿ ಸುರಿದು, ಆ ನೀರೆಲ್ಲವೂ ಪನ್ನೀರಿನಂತೆ ಪರಿಮಳಯುಕ್ತವಾಗಿರುತ್ತದೆ. ಆ ಸರೋವರದ ಸುತ್ತ ಹೂವಿ ನಿಂದ ತುಂಬಿದ ಮೊಲ್ಲೆ ಮಲ್ಲಿಗೆ ಜಾಜಿ ಮೊದಲಾದ ಹೂ ಬಳ್ಳಿಗಳು ದಟ್ಟವಾಗಿ ಬೆಳೆದು, ಅವುಗಳ ನೆರಳನ್ನು ನೀರಿನಲ್ಲಿ ಕಂಡಾಗ ಅದೊಂದು ಗಂಧರ್ವಲೋಕವೆಂಬಂತೆ ಕಣ್ಣಿಗೆ ಹಬ್ಬವನ್ನು ಮಾಡುತ್ತದೆ. ತ್ರಿಕೂಟಪರ್ವತದಲ್ಲಿ ನೆಲೆಸಿದ ದೇವತೆಗಳು ಆ ಸರೋವರ ದಲ್ಲಿ ನೀರಾಟವಾಡಿ ಧನ್ಯರಾಗುತ್ತಾರೆ.

ವರುಣದೇವನ ಈ ಸುಂದರವನದಲ್ಲಿ ಅನೇಕ ಮೃಗಗಳು ವಾಸಿಸುತ್ತಿದ್ದವು. ಅವುಗಳಲ್ಲಿ ಮದ್ದಾನೆಯೊಂದು ಮೃಗಗಳ ಮಹಾರಾಜನಂತೆ ಮೆರೆಯುತ್ತಿತ್ತು. ಉಳಿದ ಮೃಗಗಳು ಹಾಗಿರಲಿ, ಹುಲಿ, ಖಡ್ಗಮೃಗ, ಸಿಂಹಗಳು ಕೂಡ ಅದರ ವಾಸನೆ ಬರುತ್ತಲೆ ಹೆದರಿ ದೂರ ಸರಿಯುತ್ತಿದ್ದವು. ಕೋತಿ, ಹುಲ್ಲೆ, ಮೊಲ ಮೊದಲಾದ ಪ್ರಾಣಿಗಳು ಅದರ ಸುತ್ತಮುತ್ತ ಸಂಚರಿಸುತ್ತ ಶತ್ರು ಭಯವಿಲ್ಲದೆ ನಿರ್ಭಯವಾಗಿರುತ್ತಿದ್ದವು. ಆ ಗಜ ರಾಜನು ತನ್ನ ಹೆಣ್ಣಾನೆಗಳನ್ನೂ ಮರಿಯಾನೆಗಳನ್ನೂ ಕೂಡಿಕೊಂಡು ಬಿದಿರು ಮೆಳೆ ಗಳನ್ನೂ ಪೊದರುಗಳನ್ನೂ ನೆಲಸಮವಾಗುವಂತೆ ನಡೆಯುತ್ತಿದ್ದರೆ ತ್ರಿಕೂಟಗಿರಿಯೆಲ್ಲವೂ ನಡುಗಿಹೋಗುತ್ತಿತ್ತು. ಹೀಗಿರುತ್ತಿರಲು, ಒಂದು ಬಿರುಬೇಸಗೆಯ ದಿನ ಅದು ಬಾಯಾರಿಕೆ ಯಿಂದ ನೀರನ್ನು ಅರಸುತ್ತಾ, ವರುಣೋದ್ಯಾನದ ಸರೋವರಕ್ಕೆ ಬಂತು. ಅಮೃತದಂ ತಿರುವ ಅದರ ನೀರನ್ನು ಹೊಟ್ಟೆತುಂಬ ಕುಡಿದು ತೃಪ್ತಿಯಾದ ಮೇಲೆ ಅದು ಸರೋವರದ ಆಳಕ್ಕಿಳಿದು, ಸೊಂಡಿಲಿನಿಂದ ನೀರನ್ನು ತನ್ನ ಮೈಮೇಲೆಲ್ಲ ಸುರಿದುಕೊಂಡು, ತನ್ನ ಬೇಗೆಯನ್ನು ನೀಗಿಕೊಂಡಿತು. ಅನಂತರ ಅದು ಒಳ್ಳೆಯ ಸಂಸಾರಿಯಂತೆ ತನ್ನ ಪರಿವಾರಕ್ಕೆ ಸೇರಿದ ಇತರ ಆನೆಗಳಿಗೂ ನೀರನ್ನು ಕುಡಿಸಿ, ಸ್ನಾನಮಾಡಿಸಿತು.

ಹೀಗೆ ಆ ಗಜೇಂದ್ರನು ನೀರಾಟವಾಡುತ್ತಾ ವಿನೋದವಾಗಿ ವಿಹರಿಸುತ್ತಿರಲು, ನೀರೊಳ ಗಿದ್ದ ದೊಡ್ಡ ಮೊಸಳೆಯೊಂದು ಅದರ ಕಾಲನ್ನು ಹಿಡಿದು, ನೀರೊಳಕ್ಕೆ ಎಳೆಯ ಹೊರ ಟಿತು. ಕೋಪಗೊಂಡ ಗಜರಾಜನು ಒಮ್ಮೆ ಘೇಂಕರಿಸಿ, ಕಾಲನ್ನು ಕಿತ್ತುಕೊಳ್ಳಲು ಪ್ರಯತ್ನಿಸಿತು. ಆದರೆ ಅದು ಸಾಧ್ಯವಾಗಲಿಲ್ಲ. ಮತ್ತೆ ಮತ್ತೆ ತನ್ನ ಶಕ್ತಿಯನ್ನೆಲ್ಲ ವೆಚ್ಚ ಮಾಡಿ ಯತ್ನಿಸಿದರೂ ಮೊಸಳೆಯ ಹಿಡಿತ ಬಿಟ್ಟುಹೋಗಲಿಲ್ಲ. ಆನೆಯ ಶಕ್ತಿ ಕುಗ್ಗುತ್ತಾ ಹೋದಂತೆ ಮೊಸಳೆಯ ಶಕ್ತಿ ಹೆಚ್ಚುತ್ತಾ ಹೋಯಿತು. ಭಯದಿಂದ ಆನೆ ಪರಿತಪಿಸಿತು. ಅದರ ದುಸ್ಥಿತಿಯನ್ನು ಕಂಡು ಉಳಿದ ಆನೆಗಳೆಲ್ಲ ದುಃಖದಿಂದ ಕಣ್ಣೀರ್ಗರೆಯುತ್ತಾ ಘೀಳಿಟ್ಟವು. ಆದರೆ ಅದರಿಂದ ಏನು ಪ್ರಯೋಜನ? ಭಯಂಕರವಾದ ಆ ಮೊಸಳೆಯನ್ನು ಯಾರು ಬಿಡಿಸುವವರು? ಜೀವವನ್ನು ಉಳಿಸಿಕೊಳ್ಳುವುದಕ್ಕಾಗಿ ಆನೆ ಹೋರಾಡುತ್ತಿದ್ದರೆ, ಅದರ ಜೀವವನ್ನು ಹೀರುವುದಕ್ಕಾಗಿ ಮೊಸಳೆ ಕಾದುತ್ತಿದೆ. ಬಹುಕಾಲದವರೆಗೆ ಅವು ಪರಸ್ಪರ ಹೋರಾಡಿದವು. ಕೊನೆಗೆ ಸೋತು, ಸಂಕಟದಿಂದ ಸಾವನ್ನು ಇದಿರುನೋಡು ತ್ತಿದ್ದ ಆನೆಗೆ ತನ್ನ ಪುನರ್ಜನ್ಮದ ಸ್ಮರಣೆಯುಂಟಾಯಿತು. ಅದರೊಡನೆ ಅದಕ್ಕೆ ಎಲ್ಲೆ ಯಿಲ್ಲದ ಧೈರ್ಯವೂ ಮೂಡಿತು.

ಗಜರಾಜನು ಪೂರ್ವಜನ್ಮದಲ್ಲಿ ಪಾಂಡ್ಯದೇಶದ ರಾಜನಾಗಿದ್ದ. ಆತನ ಹೆಸರು ಇಂದ್ರ ದ್ಯುಮ್ನ ಎಂದು. ಆತನು ಬಹು ದೊಡ್ಡ ದೈವಭಕ್ತ. ಆತನು ತನ್ನ ರಾಜ್ಯಕೋಶಗಳನ್ನೆಲ್ಲ ತೊರೆದು, ತಪಸ್ಸು ಮಾಡುವುದಕ್ಕಾಗಿ ಮಲಯ ಪರ್ವತಕ್ಕೆ ಹೊರಟುಹೋದನು. ಅಲ್ಲಿ ಆತನು ಒಂದು ಆಶ್ರಮವನ್ನು ಮಾಡಿಕೊಂಡು ಬಹುಕಾಲದವರೆಗೆ ಶ್ರೀಹರಿಯನ್ನು ಆರಾಧಿಸುತ್ತಿದ್ದನು. ಹೀಗಿರಲು ಒಂದು ದಿನ ಆತ ಎಂದಿನಂತೆ ಸ್ನಾನ ಸಂಧ್ಯೆಗಳನ್ನು ಮುಗಿಸಿ, ಶ್ರೀಹರಿಯ ಧ್ಯಾನದಲ್ಲಿ ಮುಳುಗಿಹೋಗಿದ್ದಾಗ ಮಹರ್ಷಿಯಾದ ಅಗಸ್ತ್ಯನು ತನ್ನ ಶಿಷ್ಯರೊಡನೆ ಆತನ ಆಶ್ರಮಕ್ಕೆ ಬಂದನು. ಧ್ಯಾನದಲ್ಲಿದ್ದ ಇಂದ್ರದ್ಯುಮ್ನನಿಗೆ ಇದು ಗೊತ್ತಾಗಲಿಲ್ಲ. ಆಶ್ರಮದ ಬಾಗಿಲಿಗೆ ಬಂದ ತನ್ನನ್ನು ತಕ್ಕರೀತಿಯಲ್ಲಿ ಮರ್ಯಾದೆ ಮಾಡಲಿಲ್ಲವಲ್ಲ ಎಂದು ಅಗಸ್ತ್ಯನಿಗೆ ಕೋಪ ಬಂತು. ಆತ ‘ಆನೆಯಂತೆ ಸೊಕ್ಕಿ ನಡೆ ಯುವ ಈ ಇಂದ್ರದ್ಯುಮ್ನ ಆನೆಯಾಗಿ ಹೋಗಲಿ’ ಎಂದು ಶಾಪವಿತ್ತು, ತನ್ನ ಶಿಷ್ಯರೊಡನೆ ಅಲ್ಲಿಂದ ಹೊರಟುಹೋದ. ಇದರ ಫಲವಾಗಿ ಆ ರಾಜಪುಷಿಗೆ ಆನೆಯ ಜನ್ಮ ಪ್ರಾಪ್ತ ವಾಯಿತು. ಅದು ತ್ರಿಕೂಟಪರ್ವತದಲ್ಲಿ ಹುಟ್ಟಿ, ಗಜರಾಜನೆನಿಸಿಕೊಂಡು, ಮೇಲೆ ಹೇಳಿ ದಂತೆ ಒಂದು ದಿನ ಸರೋವರದಲ್ಲಿ ನೀರು ಕುಡಿಯಲೆಂದು ಹೋದಾಗ ಮರಣಸಂಕಟಕ್ಕೆ ಗುರಿಯಾಯಿತು.

ಪೂರ್ವಜನ್ಮದ ಜ್ಞಾಪಕ ಮೂಡುತ್ತಲೆ ಗಜರಾಜನು ತನ್ನ ಆರಾಧ್ಯ ದೈವವನ್ನು ಮರೆ ಹೊಕ್ಕಿತು. ಹಿಂದಿನ ಜನ್ಮದಲ್ಲಿ ದಿನವೂ ಹೇಳುತ್ತಿದ್ದ ಪರಬ್ರಹ್ಮ ಸ್ತುತಿಯನ್ನು ಅದು ಇಂದು ಜಪಿಸಿತು. ಹಿಂದಿನ ಜನ್ಮದಂತೆ ಈಗ ನಮಸ್ಕರಿಸುವುದಕ್ಕೆ ಸಾಧ್ಯವಿಲ್ಲವಾದುದ ರಿಂದ ಮನಸ್ಸಿನಲ್ಲಿಯೇ ಅದು ದೈವಕ್ಕೆ ನಮಸ್ಕಾರವನ್ನು ಸಲ್ಲಿಸಿತು. ಯಾವ ನಾಮ ರೂಪಗಳನ್ನೂ ನೆನೆಯದೆ ಪರಾತ್ಪರ ವಸ್ತುವನ್ನೆ ಧ್ಯಾನ ಮಾಡುತ್ತಿದ್ದರಿಂದ ಬ್ರಹ್ಮ, ಇಂದ್ರ ಇತ್ಯಾದಿ ದೇವತೆಗಳಾರೂ ಪ್ರತ್ಯಕ್ಷವಾಗದೆ, ಸಾಕ್ಷಾತ್ ಪರಬ್ರಹ್ಮ ಸ್ವರೂಪಿಯಾದ ಶ್ರೀಹರಿಯೇ ಆ ನಿಮಿಷದಲ್ಲಿ ಪ್ರತ್ಯಕ್ಷನಾದನು. ವೇದಸ್ವರೂಪನಾದ ಗರುಡನ ಮೇಲೆ ಕುಳಿತು ಬರುತ್ತಿದ್ದ ಆತನು, ಗರುಡನ ಗಮನ ತಡವಾದೀತೆಂದು ತಟ್ಟನೆ ಆನೆಯ ಬಳಿಗೆ ನೆಗೆದು, ಗಜರಾಜನನ್ನು ಹಿಡಿದಿದ್ದ ಮೊಸಳೆಯೊಡನೆ ಅದನ್ನು ಕೊಳದಿಂದ ಹೊರಕ್ಕೆ ಎಳೆದು ಹಾಕಿದನು. ಆತನ ಚಕ್ರಾಯುಧವು ಮೊಸಳೆಯ ಬಾಯನ್ನು ಸೀಳಿಹಾಕಿತು. ಇದನ್ನು ಕಂಡು ಸ್ವರ್ಗದ ದೇವತೆಗಳು ಹೂಮಳೆಯನ್ನು ಕರೆದರು. ಅಪ್ಸರೆಯರು ನರ್ತನ ಮಾಡಿದರು. ಗಂಧರ್ವರು ಗಾನ ಮಾಡಿದರು.

ಸತ್ತುಬಿದ್ದ ಮೊಸಳೆಯ ದೇಹದಿಂದ ದಿವ್ಯಸುಂದರನಾದ ದೇವತೆಯೊಬ್ಬ ಮೂಡಿ ದನು. ಆತನು ‘ಹೂ ಹೂ’ ಎಂಬ ಹೆಸರಿನ ಗಂಧರ್ವರಾಜ; ದೇವಲ ಮುನಿಯ ಶಾಪ ದಿಂದ ಮೊಸಳೆಯಾಗಿದ್ದ. ಶ್ರೀಹರಿಯ ಕೈಯಿಂದ ಸಾವು ಬರುತ್ತಲೆ ಆತನ ಶಾಪ ಹಾರಿ ಹೋಯಿತು. ಆತನು ಸ್ವಸ್ವರೂಪವನ್ನು ಪಡೆದು, ಭಕ್ತಿಯಿಂದ ಭಗವಂತನನ್ನು ಸ್ತೋತ್ರ ಮಾಡುತ್ತ ಅಡ್ಡಬಿದ್ದನು. ಅನಂತರ ಆತನು ಶ್ರೀಹರಿಯಿಂದ ಬೀಳ್ಕೊಂಡು ಎಲ್ಲರೂ ನೋಡುತ್ತಿರುವಂತೆಯೆ ತನ್ನ ಲೋಕಕ್ಕೆ ಹೊರಟು ಹೋದನು. ಇತ್ತ ಶ್ರೀಹರಿಯ ಕೈಸೋಕಿ ನಿಂದ ಗಜರಾಜನ ಅಜ್ಞಾನದೊಡನೆ ಆನೆಯ ದೇಹವೂ ಹೋಯಿತು; ಅದಕ್ಕೆ ಬದಲಾಗಿ ಪೀತಾಂಬರವನ್ನು ಧರಿಸಿದ, ನಾಲ್ಕು ತೋಳುಗಳ ಮಹಾಪುರುಷನೊಬ್ಬನು ಕಾಣಿಸಿ ಕೊಂಡನು. ತನ್ನನ್ನು ಭಕ್ತಿಯಿಂದ ನಮಸ್ಕರಿಸಿದ ಆತನನ್ನು ಕುರಿತು ಶ್ರೀಹರಿಯು ‘ಅಯ್ಯಾ, ಹಾಲ್ಗಡಲು, ಶಂಖ ಚಕ್ರ ಗದೆ ಪದ್ಮಗಳು, ಗರುಡಶೇಷರು, ರಮಾದೇವಿ, ಬ್ರಹ್ಮ ರುದ್ರರು–ನನಗೆ ಅತ್ಯಂತ ಪ್ರಿಯರಾದವರು. ನೀನು ಅವರ ಸಾಲಿಗೆ ಸೇರಿರುವೆ. ನಿನಗೆ ಸಾಯುಜ್ಯಪದವಿಯನ್ನು ಕೊಟ್ಟಿರುವೆನು. ಬೆಳಿಗ್ಗೆ ಎದ್ದು ನಿನ್ನ ಹೆಸರು ಹೇಳಿದವರ, ನಿನ್ನ ಸ್ತೋತ್ರವನ್ನು ಹಾಡಿದವರ ಪಾಪಗಳು ನಿವಾರಣೆಯಾಗುತ್ತವೆ, ಸಾಯುವ ಕಾಲದಲ್ಲಿ ಅವರಿಗೆ ನಿರ್ಮಲವಾದ ಮನಸ್ಸು ಬುದ್ಧಿಗಳುಂಟಾಗುತ್ತವೆ’ ಎಂದು ಹೇಳಿ, ಆತನನ್ನೂ ತನ್ನ ಜೊತೆಯಲ್ಲಿಯೇ ಗರುಡನ ಮೇಲೆ ಕೂಡಿಸಿಕೊಂಡು ವೈಕುಂಠಕ್ಕೆ ಹೊರಟುಹೋದನು.”

