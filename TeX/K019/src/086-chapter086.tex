
\chapter{೮೬. ಅರ್ಜುನ ಸಂನ್ಯಾಸಿ}

ಒಮ್ಮೆ ಅರ್ಜುನನು ತೀರ್ಥಯಾತ್ರೆಗಾಗಿ ಸಂಚಾರಮಾಡುತ್ತಾ ಪ್ರಭಾಸಕ್ಷೇತ್ರಕ್ಕೆ ಬಂದನು. ಅಲ್ಲಿ ತನ್ನ ಸೋದರಮಾವನಾದ ವಸುದೇವನ ಮಗಳು ಸುಭದ್ರೆಯನ್ನು ದುರ್ಯೋಧನನಿಗೆ ಕೊಟ್ಟು ಮದುವೆಮಾಡಬೇಕೆಂದು ಬಲರಾಮ ಪ್ರಯತ್ನಿಸುತ್ತಿರುವು ದಾಗಿ ತಿಳಿದುಬಂತು. ಸುಂದರಿಯರಲ್ಲಿ ಸುಂದರಿಯೆಂದು ಪ್ರಖ್ಯಾತಳಾಗಿದ್ದ ಆ ಹೆಣ್ಣನ್ನು ತಾನೆ ಮದುವೆಮಾಡಿಕೊಳ್ಳಬೇಕೆಂದು ಅರ್ಜುನನ ಆಸೆಯಾಗಿತ್ತು. ಈಗ ತನ್ನ ಆಸೆ ನಿರಾಶೆಯಾಗುವ ಸಂದರ್ಭವನ್ನು ಕಂಡು, ಅವನಿಗೆ ಆ ಹುಡುಗಿಯನ್ನು ಯಾವುದಾದರೂ ಉಪಾಯದಿಂದ ಹಾರಿಸಿಕೊಂಡು ಹೋಗಬೇಕೆಂದು ಬುದ್ಧಿ ಹುಟ್ಟಿತು. ಅವನು ಸಂನ್ಯಾಸಿಯ ವೇಷವನ್ನು ಧರಿಸಿ, ದ್ವಾರಕಿಗೆ ಹೋದನು. ಅದು ಮಳೆಗಾಲವಾದುದರಿಂದ ಆತ ಅಲ್ಲಿಯೇ ಚಾತುರ್ಮಾಸ್ಯದ ವ್ರತವನ್ನು ಕೈಕೊಂಡನು. ಆತನನ್ನು ನಿಜವಾದ ಸಂನ್ಯಾಸಿಯೆಂದೆ ಮೋಸಹೋದ ಜನ ಆತನಿಗೆ ಅತ್ಯಂತ ಭಕ್ತಿಯಿಂದ ಸತ್ಕರಿಸುತ್ತಿದ್ದರು. ಆತನ ಕೀರ್ತಿಯನ್ನು ಕೇಳಿದ ಬಲರಾಮನು ಆತನನ್ನು ಭಿಕ್ಷೆಗಾಗಿ ಮನೆಗೆ ಕರೆತಂದು, ಭಕ್ತಿಯಿಂದ ಭೋಜನಮಾಡಿಸಿದನು. ಭೋಜನಾನಂತರ ಅಲ್ಲಿಯೆ ಅವನು ವಿಶ್ರಮಿ ಸಿರಲು, ಸುಭದ್ರೆಯ ಲೋಕಮೋಹಕ ಮೂರ್ತಿ ಅವನ ಕಣ್ಣಿಗೆ ಬಿತ್ತು. ಅವಳನ್ನು ಕಾಣು ತ್ತಲೆ ಅರ್ಜುನನ ಮನಸ್ಸು ದುರ್ದಮನೀಯವಾದ ಕಾಮವಿಕಾರಕ್ಕೆ ಒಳಗಾಯಿತು. ಅರ್ಜುನನ ಸುಂದರರೂಪವನ್ನು ಕಂಡು ಸುಭದ್ರೆಗೂ ಹಾಗೆಯೆ ಆಯಿತು. ಅವಳು ಮುಖ ದಲ್ಲಿ ಮಂದಹಾಸವನ್ನು ತುಳುಕಿಸುತ್ತಾ ಕಡೆಗಣ್ ನೋಟದಿಂದ ಆತನನ್ನು ನೋಡಿ, ಲಜ್ಜೆ ಯಿಂದ ತಲೆಬಾಗಿದಳು. ಇಬ್ಬರ ಮನಸ್ಸೂ ಒಂದಾಯಿತು. ಪರಸ್ಪರ ಸರಸಸಲ್ಲಾಪಕ್ಕೆ ಅವಕಾಶವಿಲ್ಲದೆ ಅವರು ಚಡಪಡಿಸುತ್ತಿದ್ದರು.

ಹೀಗಿರಲು, ಒಂದು ದಿನ ಅರ್ಜುನನು ದ್ವಾರಕಾಪುರದ ಹೊರಭಾಗದಲ್ಲಿದ್ದ ಒಂದು ಪರ್ವತದ ಗುಹೆಯಲ್ಲಿ ಸುಭದ್ರೆಯನ್ನೆ ಧ್ಯಾನಮಾಡುತ್ತಾ ಕುಳಿತಿದ್ದಾನೆ; ವಸುದೇವನೆ ಮೊದಲಾದ ಅರಮನೆಯವರೆಲ್ಲರೂ ಊರ ಹೊರಗೆ ನಡೆಯುವ ಒಂದು ಜಾತ್ರೆಯನ್ನು ನೋಡಲೆಂದು ಕೋಟೆಯಿಂದ ಹೊರಗೆ ಬಂದರು. ಸುಭದ್ರೆಯೂ ಒಂದು ರಥದಲ್ಲಿ ಕುಳಿತು ಅವರೊಡನೆ ಬಂದಳು. ಇಂತಹ ಸಮಯಕ್ಕಾಗಿಯೆ ಕಾದಿದ್ದ ಅರ್ಜುನನು ಅವಳನ್ನು ಎತ್ತಿ ತನ್ನ ರಥದಲ್ಲಿ ಕುಳ್ಳಿರಿಸಿಕೊಂಡು ಇಂದ್ರಪ್ರಸ್ಥದ ಹಾದಿಯನ್ನು ಹಿಡಿ ದನು. ಇದನ್ನು ಕಂಡು ಯಾದವರೆಲ್ಲ ‘ಹೋ’ ಎಂದು ಕೂಗಿಕೊಂಡರು. ಶ್ರೀಕೃಷ್ಣನಿಗೂ ವಸುದೇವನಿಗೂ ಹೆಣ್ಣಿನ ಕಳ್ಳನಾರೆಂಬುದು ಗೊತ್ತು; ಅವನು ಹೋದುದು ಸಮ್ಮತ ವಾಗಿಯೂ ಇತ್ತು. ಆದರೆ ಬಲರಾಮನು ಆ ಸುದ್ದಿಯನ್ನು ಕೇಳಿ ಕಿಡಿಕಿಡಿಯಾದನು. ಆತನು ತನ್ನ ಯಾದವ ಸೇನೆಯನ್ನು ಆ ಕಳ್ಳ ಸಂನ್ಯಾಸಿಯನ್ನು ಹಿಡಿತರುವುದಕ್ಕಾಗಿ ಅಟ್ಟಿದನು. ಆದರೆ ಅರ್ಜುನನು ಬಿಟ್ಟ ಬಾಣಗಳ ಮಳೆಯನ್ನು ಸಹಿಸಲಾರದೆ ಅವರೆಲ್ಲ ಓಡಿಬಂದರು. ಮೃಗಗಳ ಮಧ್ಯದಲ್ಲಿದ್ದ ತನ್ನ ಆಹಾರವನ್ನು ಸಿಂಹರಾಜನು ನಿರ್ಭಯವಾಗಿ ಹೊತ್ತು ಕೊಂಡು ಹೋಯಿತು.

ತನ್ನ ತಂಗಿಯನ್ನು ಕದ್ದೊಯ್ದ ಸಂನ್ಯಾಸಿ, ಅರ್ಜುನನೆಂದು ಕೇಳುತ್ತಲೆ ಬಲರಾಮನು ‘ಎಲ, ಅರ್ಜುನ ಸಂನ್ಯಾಸಿ, ನಿನ್ನನ್ನು ಬಲಿಹಾಕುತ್ತೇನೆ’ ಎಂದು ಅಬ್ಬರಿಸಿ, ತನ್ನ ನೇಗಿ ಲನ್ನೂ ಒನಕೆಯನ್ನೂ ಕೈಗೆ ತೆಗೆದುಕೊಂಡನು. ಆಗ ಶ್ರೀಕೃಷ್ಣನು ಅವನನ್ನು ತಡೆದು ‘ಅಣ್ಣ, ಮಿಂಚಿಹೋದ ಕಾರ್ಯಕ್ಕೆ ಚಿಂತಿಸಿ ಫಲವಿಲ್ಲ. ಸುಭದ್ರೆ ಅವನನ್ನು ಮೆಚ್ಚಿ, ಅವನ ಹಿಂದೆ ಓಡಿಹೋಗಿರುವಾಗ, ನಾವು ಅರ್ಜುನನನ್ನು ಶಿಕ್ಷಿಸಿ ಫಲವೇನು?’ ಎಂದನು. ಅವನ ಮಾತಿನ ಮೋಡಿಗೆ ಮರುಳಾದ ಬಲರಾಮನು ಶಾಂತನಾದುದಷ್ಟೇ ಅಲ್ಲ; ನೂತನ ವಧೂ ವರರಿಗೆ ಬೇಕಾದಷ್ಟು ಬಹುಮಾನ ಬಳುವಳಿಗಳನ್ನು ಕೊಟ್ಟು ಕಳುಹಿಸಿದನು. ಈ ಸಂಬಂಧ ದಿಂದ ಯಾದವರೂ ಪಾಂಡವರೂ ಸಂತೋಷಗೊಂಡರು. ‘ಅರ್ಜುನ ಸಂನ್ಯಾಸಿ’ ಎಂಬ ಮಾತು ಆ ಬಂಧುವರ್ಗದಲ್ಲಿ ಕಪಟತನಕ್ಕೆ ಒಂದು ಗಾದೆಯೆ ಆಗಿಹೋಯಿತು.

