
\chapter{೫೧. ವಿಷದ ಮಡು ಅಮೃತದ ಸರವಾಯಿತು}

ಒಂದಾನೊಂದು ಕಾಲದಲ್ಲಿ ಹಾವುಗಳ ಹಾವಳಿಯಿಂದ ಜಗತ್ತೆಲ್ಲ ಕಳವಳಕ್ಕೆ ಒಳಗಾಯಿತು. ಆಗ ಜನರೆಲ್ಲ ಸೇರಿ, ಪ್ರತಿ ಅಮಾವಾಸ್ಯೆಯಂದು ಹಾವುಗಳಿಗೆ ಬೇಕಾದ ಆಹಾರವನ್ನು ತಾವು ಸಾಕಷ್ಟು ಒದಗಿಸುವುದಾಗಿಯೂ, ಹಾವುಗಳು ಇನ್ನು ಮುಂದೆ ಮಾನವರನ್ನು ಕಚ್ಚಬಾರದೆಂಬುದಾಗಿಯೂ ಹಾವುಗಳೊಡನೆ ಒಪ್ಪಂದಮಾಡಿಕೊಂಡರು. ಅಂದಿನಿಂದ ಮಾನವರಿಗೆಲ್ಲ ಭಯ ತಪ್ಪಿತು. ಹಾವುಗಳು ಗರುಡನ ಭಯದಿಂದ ತಪ್ಪಿಸಿ ಕೊಳ್ಳುವುದಕ್ಕಾಗಿ, ತಮಗೆ ಮಾನವರಿಂದ ಬಂದ ಆಹಾರದಲ್ಲಿ ಸ್ವಲ್ಪ ಭಾಗವನ್ನು ಗರುಡ ನಿಗೆ ಸಲ್ಲಿಸುವುದಾಗಿ ಗರುಡನೊಂದಿಗೆ ಒಪ್ಪಂದ ಮಾಡಿಕೊಂಡವು. ಅಂದಿನಿಂದ ಅವು ಗಳಿಗೆ ಗರುಡನಿಂದ ಆಗುವ ಅಪಾಯ ತಪ್ಪಿತು. ಕೆಲಕಾಲ ಎಲ್ಲರೂ ಸುಖವಾಗಿದ್ದರು, ‘ಊರಿಗೊಂದು ಹೊಲೆಗೇರಿ’ ಎಂಬಂತೆ. ಕದ್ರುವಿನ ಮಗನಾದ ಕಾಳಿಯನೆಂಬ ಹಾವು ಮಹಾಗರ್ವದಿಂದ, ತಾನು ಗರುಡನಿಗೆ ಆಹಾರವನ್ನು ಒಪ್ಪಿಸುವುದಿಲ್ಲವೆಂದು ಹಟಹಿಡಿ ಯಿತು. ಇದನ್ನು ಕೇಳಿ ಗರುಡನಿಗೆ ಕೋಪಬಂತು. ಆತ ಸಮಯವನ್ನು ಕಾದಿದ್ದು ಒಮ್ಮೆ ಕಾಳಿಯನಮೇಲೆ ಎರಗಿದ. ಕಾಳಿಯನೇನೂ ಸಾಮಾನ್ಯವಾದ ಹಾವಲ್ಲ. ಅವನಿಗೆ ಸಹಸ್ರ ಹೆಡೆಗಳು, ಮಹಾ ವಿಷದಿಂದ ಕೂಡಿದ ಹಲ್ಲುಗಳು. ಅದು ಗರುಡನ ಮೇಲೆ ಯುದ್ಧಕ್ಕೆ ನಿಂತಿತು. ಆದರೇನು? ಇಲಿ ಕೊಬ್ಬಿದರೆ ಬೆಕ್ಕಿಗೆ ಸಾಟಿಯೆ? ಗರುಡನು ಕೋಪದಿಂದ ಕಾಳಿಯನನ್ನು ತನ್ನ ರೆಕ್ಕೆಯಿಂದ ಒಮ್ಮೆ ಅಪ್ಪಳಿಸುತ್ತಲೆ ಅದರ ಗಂಡಗರ್ವವೆಲ್ಲ ಅಡಗಿ ಹೋಯಿತು. ಒಡನೆಯೆ ಅದು ಪ್ರಾಣಭಯದಿಂದ ತಾನಿದ್ದ ರಮಣಕ ದ್ವೀಪವನ್ನು ಬಿಟ್ಟು, ಬೃಂದಾವನದ ಬಳಿಯಿದ್ದ ಕಾಳಿಂದಿಯ ಮಡುವಿಗೆ ಓಡಿಹೋಯಿತು. ಗರುಡನು ಆ ಮಡುವಿನ ಹತ್ತಿರಕ್ಕೆ ಹೋಗುವಂತಿಲ್ಲ. ಸೌಭರಿ ಮಹರ್ಷಿಯ ಶಾಪಭಯ ಅವನಿಗೆ. ಹಿಂದೊಮ್ಮೆ ಆ ಮಹರ್ಷಿ ಕಾಳಿಂದಿಯ ತಡಿಯಲ್ಲಿ ತಪಸ್ಸು ಮಾಡುತ್ತಿದ್ದಾಗ ಗರುಡನು ಮಡುವಿನಲ್ಲಿದ್ದ ಮೀನೊಂದನ್ನು ಹಿಡಿದು ತಿಂದುದರಿಂದ, ಪುಷಿಯು ಕೋಪಗೊಂಡು ಮತ್ತೊಮ್ಮೆ ಹಾಗೆ ಮಾಡಿದರೆ ಅವನಿಗೆ ಮರಣವಾಗಲೆಂದು ಶಾಪಕೊಟ್ಟಿದ್ದನು. ಆದ್ದ ರಿಂದ ಗರುಡ ಅಲ್ಲಿಗೆ ಹೋಗುವಂತಿಲ್ಲ. ಇದನ್ನು ಅರಿತಿದ್ದ ಕಾಳಿಯ ಅಲ್ಲಿಗೆ ಹೋಗಿ ಸುಖವಾಗಿದ್ದನು. ಅವನು ಆ ಮಡುವನ್ನು ಸೇರುತ್ತಲೆ ಅದೊಂದು ಘೋರವಾದ ವಿಷದ ಮಡುವಾಯಿತು. ಆ ಮಡುವಿನ ನೀರೆಲ್ಲ ವಿಷದ ಬೇಗೆಯಿಂದ ಕುದಿಯುತ್ತಿತ್ತು. ಸುತ್ತಮುತ್ತಲಿನ ಮರಗಳೆಲ್ಲ ಸುಟ್ಟು ಕರಿಕಾಗಿದ್ದವು, ಅದರ ಸುತ್ತಮುತ್ತ ಯಾವ ಪ್ರಾಣಿಯೂ ಸಂಚರಿಸುವಂತಿರಲಿಲ್ಲ, ಆ ಮಡುವಿನ ಮೇಲೆ ಬಹು ಎತ್ತರದವರೆಗೆ ಯಾವ ಹಕ್ಕಿಯೂ ಹಾರಿಹೋಗುವಂತಿರಲಿಲ್ಲ.

ಎಂದಿನಂತೆ ಒಂದು ದಿನ ಗೋಪಾಲಕರು ಗೋಗಳನ್ನು ಮೇಯಿಸುವುದಕ್ಕಾಗಿ ಅಡವಿಗೆ ಹೊರಟರು. ಅಂದು ಬಲರಾಮ ಬಂದಿರಲಿಲ್ಲ; ಶ್ರೀಕೃಷ್ಣನೊಡನೆ ಅವರು ವಿನೋದ ದಿಂದ ವಿಹರಿಸುತ್ತಾ ಇದ್ದು, ಮಧ್ಯಾಹ್ನದ ವೇಳೆಗೆ ಬಾಯಾರಿಕೆಯನ್ನು ತೀರಿಸಿಕೊಳ್ಳ ಲೆಂದು ಯಮುನಾ ನದಿಯ ತಡಿಗೆ ಹೋದರು. ದುರದೃಷ್ಟದಿಂದ ಅವರು ಹೋದುದು ಕಾಳಿಂದಿಯ ಮಡುವಾಗಿತ್ತು. ಪಾಪ, ಅವರಿಗೆ ಅದರ ಕಥೆ ಗೊತ್ತಿಲ್ಲ. ನಗುತ್ತ ಕೆಲೆಯುತ್ತ ಅವರು ದುಡುದುಡು ಮಡುವಿನೊಳಕ್ಕೆ ಇಳಿದು ಬೊಗಸೆಯಿಂದ ನೀರನ್ನು ಹೀರಿದರು. ನೀರು ನಾಲಿಗೆಗೆ ಸೋಕುತ್ತಿದ್ದಂತೆಯೇ ಅವರ ಆಯಸ್ಸು ಮುಗಿದು ಹೋದಂತೆ ಅವರು ದೊಪ್ಪನೆ ನೆಲಕ್ಕುರುಳಿದರು. ದಡದಲ್ಲಿ ನಿಂತಿದ್ದ ಶ್ರೀಕೃಷ್ಣನಿಗೆ ಅದರ ಕಾರಣ ಅರ್ಥ ವಾಯಿತು. ದೇವದೇವನಾದ ಆತ ತನ್ನ ಕಣ್​ದಿಟ್ಟಿಯ ಅಮೃತದಿಂದಲೆ ಅವರನ್ನು ಬದುಕಿಸಿದನು. ಎಲ್ಲರೂ ಮೇಲಕ್ಕೆದ್ದು ಒಬ್ಬರ ಮುಖವನ್ನು ಒಬ್ಬರು ನೋಡುತ್ತಾ ಇದು ಹೇಗೆ ನಡೆಯಿತೆಂದು ಆಶ್ಚರ್ಯಪಡುತ್ತಿದ್ದರು. ಕ್ರಮೇಣ ಅವರಿಗೆ ಅರ್ಥವಾಯಿತು, ಅದು ವಿಷದ ಮಡುವೆಂದು; ತಾವು ಸಾಯದೆ ಉಳಿದುದು ಶ್ರೀಕೃಷ್ಣನ ಕೃಪೆಯಿಂದ, ಎಂದು.

ಶ್ರೀಕೃಷ್ಣನು ಆ ವಿಷದ ಮಡುವನ್ನು ಅಮೃತದ ಸರೋವರವನ್ನಾಗಿ ಮಾಡಬೇಕೆಂದು ನಿಶ್ಚಯಿಸಿದನು. ಮನಸ್ಸಿನಲ್ಲಿ ಹುಟ್ಟಿದ ಭಾವನೆ ತಕ್ಷಣವೇ ಕಾರ್ಯರೂಪವನ್ನೂ ಪಡೆ ಯಿತು. ಆತನು ಸುತ್ತಲೂ ನೋಡಿದನು. ಆ ಮಡುವಿನ ತೀರದಲ್ಲಿ ಒಂದೆ ಒಂದು ಮರ ಮಾತ್ರ ಎತ್ತರವಾಗಿ ಬೆಳೆದು ನಿಂತಿದೆ, ಹಿಂದೆ ಗರುಡ ಅಮೃತವನ್ನು ಕೊಂಡೊಯ್ಯುತ್ತಿ ದ್ದಾಗ ಒಂದು ತೊಟ್ಟು ಆ ಸ್ಥಳದಲ್ಲಿ ಬಿದ್ದಿತಂತೆ! ಆದ್ದರಿಂದಲೇ ಆ ಮರ ಅಲ್ಲಿ ಉಳಿ ಯುವುದಕ್ಕೆ ಸಾಧ್ಯವಾಯಿತು. ಆ ಮರದ ಒಂದು ರೆಂಬೆ ಬಹು ಎತ್ತರದಲ್ಲಿ ಆ ಮಡುವಿನ ಮೇಲೆಯೆ ಹಾದುಬಂದಿತ್ತು. ಶ್ರೀಕೃಷ್ಣನು ತನ್ನ ಚಡ್ಡಿಯನ್ನು ಎಳೆದು ಬಿಗಿಯಾಗಿ ಕಟ್ಟಿ ಕೊಂಡು ಆ ಮರವನ್ನು ಏರಿದವನೆ, ಮಡುವಿನ ಮೇಲಿದ್ದ ಕೊಂಬೆಯಿಂದ ಮುಡುವಿ ನೊಳಕ್ಕೆ ಧುಮ್ಮುಕ್ಕಿದನು. ಒಡನೆಯೇ ಆ ಮಡುವಿನ ನೀರು ಅಲ್ಲೋಲಕಲ್ಲೋಲ ವಾಯಿತು; ನೀರಿನ ತಳದಲ್ಲಿ ತೆಪ್ಪಗೆ ಮಲಗಿದ್ದ ಕಾಳಿಯನಿಗೆ ಯಾರೋ ಬಡಿದೆಬ್ಬಿಸಿದಂ ತಾಯಿತು. ಅದು ರೋಷದಿಂದ ಬುಸುಗುಟ್ಟುತ್ತಾ ಮೇಲೆದ್ದು ಬಂದು, ಶ್ರೀಕೃಷ್ಣನ ಮೈಗೆಲ್ಲಾ ಸುತ್ತಿಕೊಂಡಿತು. ಇದನ್ನು ಕಂಡೊಡನೆಯೇ ಗೋಪಾಲಬಾಲರೆಲ್ಲ ಭಯದುಃಖ ಗಳಿಂದ ‘ಅಯ್ಯೋ’ ಎಂದು ಅಳುತ್ತಾ ನಿಂತುಕೊಂಡರು. ಅದೇ ವೇಳೆಗೆ ಬೃಂದಾವನದ ಗೋಕುಲದಲ್ಲಿ ಭಯಂಕರವಾದ ಉತ್ಪಾತಗಳಾದವು. ಅಂತರಿಕ್ಷದಿಂದ ಬೆಂಕಿಯ ಮಳೆ ಸುರಿಯಿತು, ಭೂಮಿ ನಡುಗಿತು, ಜನರ ಎಡಗಣ್ಣುಗಳು ಹಾರಿದವು. ಇದನ್ನು ಕಂಡು ನಂದನೇ ಮೊದಲಾದವರೆಲ್ಲ ಭಯದಿಂದ ನಡುಗಿಹೋದರು. ಬಲರಾಮನು ಅಂದು ಗೋಪಾಲರೊಡನೆ ಹೋಗದೆ ಗೋಕುಲದಲ್ಲಿಯೇ ಇರುವನೆಂಬುದನ್ನು ಕೇಳಿ, ಅವರ ಭಯ ಮತ್ತಷ್ಟು ಹೆಚ್ಚಿತು. ಅಡವಿಗೆ ಹೋಗಿರುವ ಶ್ರೀಕೃಷ್ಣನಿಗೆ ಯಾವುದೋ ಅಪಾಯ ವಾಗಿರ ಬೇಕೆಂದು ಅವರ ಮನಸ್ಸಿಗೆ ತೋಚಿತು. ಆ ಭಾವನೆ ಬಂದುದೇ ತಡ ಗೋಕುಲ ದವರೆಲ್ಲ–ಹೆಂಗಸರು ಮಕ್ಕಳೂ ಕೂಡ–ಶ್ರೀಕೃಷ್ಣನನ್ನು ಹುಡುಕಿಕೊಂಡು ಹೊರಟರು.

ಹೀಗೆ ಗೋಕುಲದವರೆಲ್ಲ ಶ್ರೀಕೃಷ್ಣನನ್ನು ಅರಸುತ್ತಾ ಕಾಳಿಂದಿ ಮಡುವಿನ ಬಳಿಗೆ ಬಂದರು. ಅಲ್ಲಿ ನೋಡುತ್ತಾರೆ–ಗೋಪಾಲಬಾಲರೆಲ್ಲ ಗಟ್ಟಿಯಾಗಿ ಅಳುತ್ತಾ ನಿಂತಿ ದ್ದಾರೆ; ಗೋವುಗಳೆಲ್ಲ ಮಡುವಿನ ಕಡೆ ಮುಖ ಮಾಡಿಕೊಂಡು ಕಣ್ಣೀರು ಸುರಿಸುತ್ತಾ ನಿಂತಿವೆ; ಮಡುವಿನೊಳಗೆ ಶ್ರೀಕೃಷ್ಣನು ಕಾಳಿಯನ ಕಟ್ಟಿಗೆ ಸಿಕ್ಕಿ ಕಣ್ಣುಕಣ್ಣು ಬಿಡುತ್ತಾ ನೀರಿನ ಮಧ್ಯೆ ತೇಲುತ್ತಿದ್ದಾನೆ. ಅವರು ತಾನೆ ಏನು ಮಾಡುವುದಕ್ಕೆ ಸಾಧ್ಯ? ಹೆಂಗಸರು ಗಂಡಸರೆಲ್ಲ ಒಟ್ಟಾಗಿ ಗಳಗಳ ಅತ್ತರು. ಶ್ರೀಕೃಷ್ಣನ ಚೆಲುವನ್ನು, ಮಗುಳ್​ನಗೆಯನ್ನು, ಮುದ್ದು ಮಾತುಗಳನ್ನು ನೆನೆದು ಹಾಡಿಕೊಂಡು ಗೋಳಾಡಿದರು. ಶ್ರೀಕೃಷ್ಣನಿಲ್ಲದೆ ತಾವು ಜೀವಿಸಲಾರೆವೆಂದುಕೊಂಡು ಅವರೆಲ್ಲ ಸಾಯುವುದಕ್ಕೆ ಸಿದ್ಧರಾದರು. ಕೊನೆಗೆ ಶ್ರೀ ಕೃಷ್ಣನೇ ಅವರ ಗೋಳನ್ನು ನೋಡಲಾರದೆ, ಆ ಹಾವಿನ ಕಟ್ಟಿನಿಂದ ಮೆಲ್ಲನೆ ಬಿಡಿಸಿ ಕೊಂಡನು. ಅದರ ಕಟ್ಟು ಬಿಟ್ಟುದೇ ತಡ, ಶ್ರೀಕೃಷ್ಣನು ಅದನ್ನು ಹಿಡಿದು ಅಪ್ಪಳಿಸ ಬೇಕೆಂದು ಅದರ ಸುತ್ತ ಚಕ್ರದಂತೆ ತಿರುಗತೊಡಗಿದನು. ಕಾಳಿಯನು ಅದಕ್ಕೆ ಅವಕಾಶ ಕೊಡದೆ ವಿಷವುಕ್ಕುವಂತೆ ಬುಸುಗುಟ್ಟುತ್ತಾ ತನ್ನ ಹೆಡೆಗಳಿಂದ ಅವನನ್ನು ಅಪ್ಪಳಿಸಲು ಹೊಂಚುಹಾಕುತ್ತಿತ್ತು. ಈ ಸಂಚನ್ನು ತಿಳಿದ ಕೃಷ್ಣನು ತಟ್ಟನೆ ಅದರ ಹೆಡೆಗಳ ಮೇಲೆ ಹಾರಿ ನಿಂತನು. ಆಗ ಹೆಡೆಗಳಲ್ಲಿದ್ದ ರತ್ನಗಳ ಕಾಂತಿ ಆತನ ಮೃದುಪಾದಗಳ ಮೇಲೆ ಬಿದ್ದು, ಚಿಗುರಿಗೆ ಅರಗಿನ ಬಣ್ಣವನ್ನು ಬಳಿದಂತಾಯಿತು. ಕಲಾಚತುರನಾದ ಶ್ರೀಕೃಷ್ಣನು ಕಾಳಿಯನ ಬಾಲವನ್ನು ಎಡಗೈಲಿ ಹಿಡಿದು, ಅದರ ಹೆಡೆಯ ಮೇಲೆ ಕುಣಿಯಲು ಪ್ರಾರಂಭಿ ಸಿದನು. ಆ ಹಾವು ಮೊದಮೊದಲು ಅವನನ್ನು ಕೆಳಕ್ಕೆ ಕೆಡವಲು ಪ್ರಯತ್ನಿಸಿತಾದರೂ ಅದರ ಆಟ ಸಾಗಲಿಲ್ಲ. ಕ್ರಮಕ್ರಮವಾಗಿ ಅದರ ಶಕ್ತಿ ಕುಂದುತ್ತಾ ಬಂದಿತು. ಅದರ ಹೆಡೆಗಳೆಲ್ಲ ನುಜ್ಜುಗುಜ್ಜಾದವು. ಅದರ ಸಾವಿರ ಬಾಯಿಗಳಿಂದಲೂ ನೆತ್ತರು ಸುರಿಯುವು ದಕ್ಕೆ ಪ್ರಾರಂಭವಾಯಿತು. ಅದರ ಹಮ್ಮು ಅಡಗಿ, ಅದು ಶ್ರೀಕೃಷ್ಣನಿಗೆ ಶರಣಾಯಿತು.

ಕಾಳಿಯನು ಸೋತು ಸೊಪ್ಪಾಗಿ ನೆಲಕ್ಕುರುಳುತ್ತಲೆ, ಅದರ ಹೆಂಡತಿರೆಲ್ಲ ಶ್ರೀಕೃಷ್ಣನಿಗೆ ಅಡ್ಡ ಬಿದ್ದು, ತಮ್ಮ ಮಾಂಗಲ್ಯಭಾಗ್ಯವನ್ನು ಕರುಣಿಸವಂತೆ ಬೇಡಿಕೊಂಡವು. ‘ಗಂಟು ಹಾಕಿದವನೇ ಅದನ್ನು ಬಿಚ್ಚಬೇಕು. ಅದರಂತೆ ಹಾವುಗಳಿಗೆ ಕ್ರೂರ ಸ್ವಭಾವವನ್ನು ಕೊಟ್ಟ ನೀನೆ ಅದನ್ನು ಬಿಡಿಸಬೇಕು. ನೀನು ಹೇಳಿದಂತೆ ನಾವು ಕೇಳುತ್ತೇವೆ’ ಎಂದು ಕಾಳಿಯನೂ ಶ್ರೀಕೃಷ್ಣ ಪರಮಾತ್ಮನಲ್ಲಿ ಹೇಳಿಕೊಂಡನು. ಇದನ್ನು ಕೇಳಿ ಶ್ರೀಕೃಷ್ಣನಿಗೆ ಕರುಣೆ ಹುಟ್ಟಿತು. ಆತನು ‘ಎಲಾ ಕಾಳಿಯ, ನಾನು ನಿನ್ನನ್ನೇನು ಕೊಲ್ಲುವುದಿಲ್ಲ. ಆದರೆ ನೀನು ಇನ್ನು ಮುಂದೆ ಇಲ್ಲಿ ನಿಲ್ಲಕೂಡದು. ನೀನು ಈ ಕ್ಷಣವೆ ಇಲ್ಲಿಂದ ಸಮುದ್ರಕ್ಕೆ ಹೊರಟು ಹೋಗು. ಈ ವಿಷದ ಮಡು ಇನ್ನು ಮುಂದೆ ಅಮೃತದ ಸರೋವರವಾಗಬೇಕು. ಇದರ ಸುತ್ತ ಗೋಗಳೂ ಮನುಷ್ಯರೂ ನಿರ್ಭಯವಾಗಿ ಸಂಚರಿಸುವಂತಾಗಬೇಕು. ಈ ಮಡುವಿನ ನೀರು ಅವರ ದಾಹವನ್ನು ಅಡಗಿಸುವ ಅಮೃತವಾಗಬೇಕು’ ಎಂದನು. ಆತನ ಅಪ್ಪಣೆ ಯಂತೆ ಕಾಳಿಯನು ಆ ಕ್ಷಣವೆ ಅಲ್ಲಿಂದ ಹೊರಟುಹೋದನು. ಆಗ ಗೋಪಾಲರಿಗೆಲ್ಲ ಆದ ಆನಂದವನ್ನು ಮಾತುಗಳಿಂದ ವರ್ಣಿಸಲು ಸಾಧ್ಯವೆ? ಅಂದಿನಿಂದ ಅದು ಗೋಪಾಲ ಬಾಲಕರ ಆಟದ ತಾಣವಾಯಿತು.

