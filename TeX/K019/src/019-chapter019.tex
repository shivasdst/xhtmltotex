
\chapter{೧೯. ಪ್ರಿಯವ್ರತ ರಾಜ}

ಪರೀಕ್ಷಿದ್ರಾಜನು ಶುಕಮುನಿಯನ್ನು ಕುರಿತು ‘ಮಹಾನುಭಾವ, ಸ್ವಾಯಂಭುವಮನು ವಿನ ಇಬ್ಬರು ಮಕ್ಕಳಲ್ಲಿ ಉತ್ತಾನಪಾದ ಮತ್ತು ಅವನ ವಂಶದವರ ಕಥೆಯನ್ನು ಮಾತ್ರ ತಿಳಿಸಿದೆ. ಹಿರಿಯ ಮಗನಾದ ಪ್ರಿಯವ್ರತನ ಕಥೆಯನ್ನು ಕೇಳಬೇಕೆಂದು ನನಗೆ ಆಶೆ. ಆತನು ರಾಜ್ಯಭಾರ ಮಾಡುತ್ತಿದ್ದನೆಂದು ಮಾತ್ರ ತಿಳಿಸಿದೆ. ಆತನನ್ನು ಕುರಿತು ವಿಸ್ತಾರವಾಗಿ ತಿಳಿಸು’ ಎಂದು ಕೇಳಿಕೊಂಡನು. ಶುಕಮುನಿಯು ಪುಣ್ಯಪುರುಷನಾದ ಪ್ರಿಯವ್ರತನ ಕಥೆಯನ್ನು ಆದ್ಯಂತವಾಗಿ ತಿಳಿಸಿದನು:

ಮನುಪುತ್ರನಾದ ಪ್ರಿಯವ್ರತನು ಚಿಕ್ಕಂದಿನಿಂದಲೂ ಭಗವದ್​ಭಕ್ತನಾಗಿದ್ದನು. ಆತನು ಮಹರ್ಷಿ ನಾರದರಿಂದ ಉಪದೇಶವನ್ನು ಪಡೆದು, ಜ್ಞಾನಯೋಗವನ್ನು ಸಾಧಿಸ ಬೇಕೆಂಬ ಆಶೆಯಿಂದ ಸದಾ ಭಗವಂತನ ಧ್ಯಾನದಲ್ಲಿ ಮಗ್ನನಾಗಿದ್ದನು. ತಂದೆಯಾದ ಮನುವಿಗೆ ಈ ಮಗನು ರಾಜನಾಗಬೇಕೆಂಬ ಅಪೇಕ್ಷೆಯಿತ್ತು. ಆದರೆ ಮಗ ಅದಕ್ಕೆ ಒಪ್ಪಲಿಲ್ಲ. ಇದನ್ನು ಕಂಡು ಸರ್ವಜ್ಞನಾದ ಬ್ರಹ್ಮನು ಅವನ ಬಳಿಗೆ ಇಳಿದು ಬಂದನು. ಒಡನೆಯೆ ಮನು ಪ್ರಿಯವ್ರತರು ಆತನನ್ನು ಭಕ್ತಿಯಿಂದ ಪೂಜಿಸಿ, ಗೌರವಿಸಿದರು. ಬ್ರಹ್ಮನು ಸಂತುಷ್ಟನಾಗಿ ಪ್ರಿಯವ್ರತನೊಡನೆ ‘ಮಗು ಪ್ರಿಯವ್ರತಾ! ನೀನಾಗಲೀ, ನಿಮ್ಮಪ್ಪನಾಗಲೀ ಅಥವಾ ನಿಮ್ಮಜ್ಜನಾದ ನಾನಾಗಲಿ, ಭಗವಂತನ ಇಷ್ಟಕ್ಕೆ ತಕ್ಕಂತೆ ನಡೆಯಬೇಕೇ ಹೊರತು, ಅದನ್ನು ಮೀರುವಂತಿಲ್ಲ. ಹಾಗೆ ಮೀರಿದರೆ ಭಗವಂತನ ಕೋಪಕ್ಕೆ ಪಾತ್ರರಾಗಬೇಕಾಗುತ್ತದೆ. ನೋಡು, ಭಗವಂತನ ಸಂಕಲ್ಪದಿಂದ ಸಮಸ್ತ ಜೀವರಾಶಿಗಳೂ ಹುಟ್ಟಿ, ಸುಖ, ದುಃಖ ಮೊದಲಾದುವನ್ನು ಅನುಭವಿಸಬೇಕಾಗುತ್ತದೆ. ಇದರಲ್ಲಿ ಆತನ ಉದ್ದೇಶವೇನೆಂಬುದು ನಮಗೆ ಗೊತ್ತಿಲ್ಲ. ಮೂಗುದಾರಹಾಕಿದ ದನ ದಂತೆ ನಾವು ಆತ ಎಳೆದ ದಾರಿಯಲ್ಲಿ ಹೋಗಬೇಕಾಗಿದೆ. ನಮ್ಮ ಕರ್ಮಕ್ಕೆ ತಕ್ಕಂತೆ ನಾವು ಯಾವ ದೇಹವನ್ನು ಧರಿಸಬೇಕೋ, ಯಾವ ಸುಖ ಕಷ್ಟಗಳನ್ನು ಅನುಭವಿಸಬೇಕೋ ಅದ ರಿಂದ ತಪ್ಪಿಸಿಕೊಳ್ಳಲು ಸಾಧ್ಯವೇ ಇಲ್ಲ. ಮತ್ತೊಬ್ಬನ ಸಹಾಯದಿಂದ ಹಾದಿ ನಡೆಯು ತ್ತಿರುವ ಕುರುಡನಂತೆ, ನಾವು ದೇವರು ಕೊಂಡೊಯ್ದ ಹಾದಿಯಲ್ಲಿ ಹೋಗಬೇಕೇ ಹೊರತು ಗತ್ಯಂತರವಿಲ್ಲ. ಈ ಪಾರತಂತ್ರ್ಯ ಅಜ್ಞಾನಿಯಂತೆ ಮುಕ್ತನಾದವನಿಗೂ ತಪ್ಪಿದು ದಲ್ಲ. ಅವನೂ ಕೂಡ ದೇಹದಿಂದಿರುವಷ್ಟು ಕಾಲ ಕರ್ಮಫಲವನ್ನು ಅನುಭವಿಸಿಯೇ ತೀರಬೇಕು. ಮಗು, ನೀನು ಮುಮುಕ್ಷುವೇ ನಿಜ, ಆದರೂ ಈ ದೇಹವಿರುವವರೆಗೆ ನೀನು ಪ್ರಾರಬ್ಧ ಕರ್ಮವನ್ನು ಮಾಡಿ ಮುಗಿಸಲೇಬೇಕು.

ಇನ್ನು, ‘ಸಂಸಾರದಲ್ಲಿದ್ದುಕೊಂಡು ಸುಖಭೋಗಗಳನ್ನು ಅನುಭವಿಸುತ್ತಾ ಹೊರಟರೆ ವೈರಾಗ್ಯವಾಗಲಿ, ಮೋಕ್ಷವಾಗಲಿ ಹೇಗೆ ಸಾಧ್ಯ?’ ಎಂಬ ಪ್ರಶ್ನೆ ಏಳುತ್ತದೆ. ಇದಕ್ಕಾಗಿಯೇ ನೀನು ವಿರಕ್ತನಾಗಿ ಕಾಡಿಗೆ ಹೋಗುತ್ತೇನೆಂದು ಹೇಳುತ್ತಿರುವುದು. ಅಲ್ಲವೆ? ಆದರೆ ಒಂದು ವಿಷಯ: ಮನಸ್ಸು ಮತ್ತು ಇಂದ್ರಿಯಗಳನ್ನು ಗೆದ್ದವನು ಮನೆಯಲ್ಲಿದ್ದರೇನು, ಅಡವಿಯಲ್ಲಿದ್ದರೇನು? ಎಚ್ಚರದಿಂದಿದ್ದವನು ಮನೆಯಲ್ಲಿದ್ದರೂ ಭಯವಿಲ್ಲ, ಎಚ್ಚರ ತಪ್ಪಿದವನು ಅಡವಿಯಲ್ಲಿದ್ದರೂ ಪ್ರಯೋಜನವಿಲ್ಲ. ಇಂದ್ರಿಯಗಳನ್ನು ಗೆದ್ದವನು ಆತ್ಮಜ್ಞಾನವನ್ನು ಅರಸುತ್ತಾ ಮನೆಯಲ್ಲಿಯೇ ನೆಲಸಿದ್ದರೂ ಯಾವ ಬಾಧಕವೂ ಇಲ್ಲ. ಅಷ್ಟೇ ಅಲ್ಲ, ಅಂತಹನು ಸಂಸಾರದಲ್ಲಿ ನಿಲ್ಲುವುದೇ ಹೆಚ್ಚು ಶ್ರೇಯಸ್ಸು. ರಾಜನಾದವನು ಶತ್ರುಗಳನ್ನು ಗೆಲ್ಲಲು ಕೋಟೆಯನ್ನು ಆಶ್ರಯಿಸುವಂತೆ, ಇಂದ್ರಿಯಗಳನ್ನು ಗೆಲ್ಲಬೇಕೆಂ ಬುವನು ಸಂಸಾರವನ್ನು ಆಶ್ರಯಿಸಬೇಕು. ಸಂಸಾರದಲ್ಲಿದ್ದು ಇಂದ್ರಿಯಗಳ ಸೊಕ್ಕನ್ನು ಅಡಗಿಸಲು ಸಮರ್ಥನಾದಮೇಲೆ ಅವನು ಎಲ್ಲಿಯಾದರೂ ನಿಶ್ಚಿಂತೆಯಿಂದಿರಬಹುದು. ಆದ್ದರಿಂದ ನೀನು ನಿನ್ನ ತಂದೆಯ ಅಪೇಕ್ಷೆಯಂತೆ ಸಂಸಾರಿಯಾಗಿ ರಾಜ್ಯಭಾರವನ್ನು ವಹಿಸು. ನಿನ್ನನ್ನು ರಾಜನ ಮಗನಾಗಿ ಹುಟ್ಟಿಸಿದ, ಭಗವಂತ. ಆದ್ದರಿಂದ ನಿರ್ದಿಷ್ಟ ಕಾಲದವರೆಗೆ ನೀನು ಈ ರಾಜ್ಯಭೋಗಗಳನ್ನು ಅನುಭವಿಸಲೇಬೇಕು. ಭಗವಂತನ ಪಾದ ಕಮಲಗಳನ್ನು ಆಶ್ರಯಿಸಿದ ನಿನಗೆ ಇಂದ್ರಿಯಗಳಿಂದ ಬಾಧೆಯೇನೂ ಆಗುವುದಿಲ್ಲ. ನಿನ್ನ ಕರ್ಮಫಲವನ್ನು ಅನುಭವಿಸಿ ಮುಗಿಸಿದಮೇಲೆ ನೀನು ಸರ್ವಸಂಗ ಪರಿತ್ಯಾಗ ಮಾಡಿ ನಿನ್ನ ಸ್ವಸ್ವರೂಪವನ್ನು ಪಡೆಯಬಹುದು’ ಎಂದನು. ಈ ಉಪದೇಶವನ್ನು ಪ್ರಿಯವ್ರತನು ಅಪ್ಪಣೆಯೆಂದೇ ಭಾವಿಸಿ, ಒಪ್ಪಿಕೊಂಡನು. ಬ್ರಹ್ಮನು ಎಲ್ಲರೂ ನೋಡುತ್ತಿರುವಂತೆಯೇ ಮಾಯವಾಗಿ ಹೋದನು.

ಬ್ರಹ್ಮನು ಕಣ್ಮರೆಯಾಗುತ್ತಲೆ ಸ್ವಾಯಂಭುವಮನುವು ಪ್ರಿಯವ್ರತನಿಗೆ ಪಟ್ಟಗಟ್ಟಿ, ತಾನು ತಪಸ್ಸಿಗಾಗಿ ಅರಣ್ಯಕ್ಕೆ ಹೊರಟುಹೋದನು. ಪ್ರಿಯವ್ರತರಾಜನು ದೈವಭಕ್ತಿಯುಕ್ತ ನಾಗಿ ಧರ್ಮದಿಂದ ರಾಜ್ಯಭಾರ ಮಾಡುತ್ತಾ ವಿಶ್ವಕರ್ಮನ ಮಗಳಾದ ಬರ್ಹಿಷ್ಮತಿಯನ್ನು ಮದುವೆಯಾದನು. ಆಕೆಯಲ್ಲಿ ಎಲ್ಲ ವಿಧದಲ್ಲಿಯೂ ತನಗೆ ಸಮಾನರಾದ ಹತ್ತು ಜನ ಗಂಡು ಮಕ್ಕಳೂ ಸುಂದರಿಯಾದ ಒಬ್ಬ ಹೆಣ್ಣುಮಗಳೂ ಹುಟ್ಟಿದರು. ಇವರಲ್ಲಿ ಕವಿ, ಮಹಾವೀರ, ಸವನ–ಎಂಬ ಮೂವರು ಬ್ರಹ್ಮವಿದ್ಯೆಯಲ್ಲಿ ಆಸಕ್ತರಾಗಿ, ಸಂಸಾರಜೀವನ ವನ್ನು ಬಿಟ್ಟು ಸಂನ್ಯಾಸಿಗಳಾದರು; ತಪಸ್ಸಿನಿಂದ ಮುಕ್ತಿಯನ್ನು ಪಡೆದರು. ಪ್ರಿಯವ್ರತ ರಾಜನಿಗೆ ಮತ್ತೊಬ್ಬ ಹೆಂಡತಿಯಿದ್ದಳು. ಆಕೆಯಲ್ಲಿ ಉತ್ತಮ, ತಾಮಸ, ರೈವತ–ಎಂಬ ಮೂರು ಮಕ್ಕಳು ಹುಟ್ಟಿ, ಅವರು ಮನ್ವಂತರಾಧಿಪತಿಗಳಾದರು. ತನ್ನ ಮಕ್ಕಳು ಮಡದಿ ಯರೊಡನೆ ಪ್ರಿಯವ್ರತರಾಜನು ಸುಖಸಂತೋಷಗಳಿಂದ ಕೂಡಿ, ಹನ್ನೊಂದು ಕೋಟಿ ವರ್ಷಗಳವರೆಗೆ ಅತ್ಯಂತ ವೈಭವದಿಂದ ರಾಜ್ಯಭಾರಮಾಡುತ್ತಿದ್ದನು.

ಪ್ರಿಯವ್ರತರಾಜನು ಸ್ವತಃ ಶೂರ; ಆತನ ಮಕ್ಕಳು ಆತನಿಗೆ ಸಮನಾದ ಬಲಶಾಲಿಗಳು. ಎಂದಮೇಲೆ ಆತನ ಕತ್ತಿಗೆ ಎದುರುಂಟೆ? ಆತನು ಬಿಲ್ಲಿಗೆ ಹಗ್ಗವನ್ನು ಕಟ್ಟಿ ಠೇಂಕಾರ ಮಾಡಿದರೆ ಸಾಕು, ಧರ್ಮಕಂಟಕರು ನಡುಗಿಹೋಗುತ್ತಿದ್ದರು. ಆದ್ದರಿಂದ ಆತನ ರಾಜ್ಯ ದಲ್ಲಿ ಸುಖಶಾಂತಿಗಳು ನೆಲಸಿದ್ದವು. ಹೀಗಿರಲು ಒಮ್ಮೆ ಆತನಿಗೊಂದು ಆಲೋಚನೆ ಹುಟ್ಟಿತು. ‘ಸೂರ್ಯನು ಮೇರು ಗಿರಿಯನ್ನು ಸುತ್ತುವಾಗ ಜಗತ್ತಿನ ಅರ್ಧಭಾಗಕ್ಕೆ ಮಾತ್ರ ಬೆಳಕಾಗುತ್ತದೆ; ಮತ್ತರ್ಧಭಾಗಕ್ಕೆ ಕತ್ತಲು. ಒಂದೆಡೆ ಬೆಳಕು, ಮತ್ತೊಂದೆಡೆ ಕತ್ತಲು ಏಕಿರಬೇಕು? ಪ್ರಪಂಚಕ್ಕೆಲ್ಲ ಯಾವಾಗಲೂ ಬೆಳಕಿರುವಂತೆ ಮಾಡಬೇಕು.’ ಆಲೋಚನೆ ಮಾಡುತ್ತಲೇ ಆತನು ಭಗವಂತನ ಧ್ಯಾನಮಾಡಿ, ಆ ದಿವ್ಯ ಶಕ್ತಿಯನ್ನು ಪಡೆದನು. ಆ ಶಕ್ತಿಯ ಪ್ರಭಾವದಿಂದ ಸೂರ್ಯನ ರಥದಂತೆ ತೇಜಸ್ಸುಳ್ಳ ಮತ್ತೊಂದು ರಥವನ್ನೇರಿ ಎರಡನೆಯ ಸೂರ್ಯನಂತೆ ಮೇರುಗಿರಿಯನ್ನು ಸುತ್ತಹೊರಟನು. ಸೂರ್ಯ ಪರ್ವತದ ಉತ್ತರ ದಲ್ಲಿರುವಾಗ ತಾನು ದಕ್ಷಿಣದಲ್ಲಿ, ಆತ ದಕ್ಷಿಣದಲ್ಲಿರುವಾಗ ತಾನು ಉತ್ತರದಲ್ಲಿ ಸಂಚರಿಸಿ ಜಗತ್ತಿಗೆಲ್ಲ ಏಕಕಾಲದಲ್ಲಿ ಬೆಳಕನ್ನುಂಟುಮಾಡಿದನು. ಹೀಗೆ ಆತನು ಏಳು ಬಾರಿ ಆ ಮೇರುಪರ್ವತವನ್ನು ಸುತ್ತಿದನು. ಹೀಗೆ ಆತನು ಸುತ್ತಿದಾಗ, ರಥದ ಚಕ್ರಗಳಿಂ ದಾದ ಹಳ್ಳಗಳೇ ಏಳು ಸಮುದ್ರಗಳು\footnote{೧. ಉಪ್ಪು ನೀರಿನ ಸಮುದ್ರ, ಕಬ್ಬಿನ ಹಾಲಿನದು, ಮದ್ಯ, ತುಪ್ಪ, ಮೊಸರು, ಹಾಲು, ಸೀ ನೀರು.}. ಆ ಸಮುದ್ರಗಳ ನಡುಗಡ್ಡೆಗಳೇ ಏಳು ದ್ವೀಪಗಳು.\footnote{೧. ಜಂಬೂ, ಪ್ಲಕ್ಷ, ಶಾಲ್ಮಲಿ, ಕುಶ, ಕ್ರೌಂಚ, ಶಾಕ, ಪುಷ್ಕರ.} ಇವು ಒಂದಕ್ಕಿಂತ ಮತ್ತೊಂದು ಕ್ರಮವಾಗಿ ಎರಡರಷ್ಟು ದೊಡ್ಡವು. ಪ್ರಿಯವ್ರತನು ಆ ಏಳು ದ್ವೀಪಗಳಿಗೂ ತನ್ನ ಏಳು ಜನ ಮಕ್ಕಳನ್ನು\footnote{೨. ಅಗ್ನೀಧ್ರ, ಇಧ್ಮಜಿಹ್ವ, ಯಜ್ಞಬಾಹು, ಹಿರಣ್ಯರೇತ, ಘೃತಪೃಷ್ಠ, ಮೇಧಾತಿಥಿ, ವೀತಿಹೋತ್ರ.} ಒಡೆಯರನ್ನಾಗಿ ಮಾಡಿದನು. ಕಡೆಯ ಮಗಳಾದ ಊರ್ಜಸ್ವತಿಯನ್ನು ಶುಕ್ರಾಚಾರ್ಯನಿಗೆ ಕೊಟ್ಟು ಮದುವೆ ಮಾಡಿದನು. ಆಕೆಯ ಹೊಟ್ಟೆಯಲ್ಲಿ ದೇವಯಾನಿಯೆಂಬ ಮಗಳು ಹುಟ್ಟಿದಳು.

ಪ್ರಿಯವ್ರತರಾಜನು ವೈಭವಶಿಖರದಮೇಲೆ ಕುಳಿತಿದ್ದರೂ ಆತನ ಮನಸ್ಸು ಒಳ ಗೊಳಗೇ ಮರುಗುತ್ತಿತ್ತು. ‘ಅಯ್ಯೋ, ನಾನೆಂತಹ ಅವಿವೇಕಿ! ಹುಲ್ಲು ಮುಚ್ಚಿದ ಹಾಳು ಬಾವಿಯಂತಿರುವ ಈ ಸಂಸಾರಸುಖಕ್ಕೆ ಸಿಕ್ಕಿಬಿದ್ದು ಹೆಣ್ಣಿನ ಕೈಗೊಂಬೆಯಾದೆ! ಇನ್ನು ಇದು ಸಾಕು’ ಎಂದುಕೊಂಡು, ರಾಜ್ಯಭಾರವನ್ನು ಮಕ್ಕಳಿಗೆ ವಹಿಸಿ, ತಾನು ದೇವರ ಧ್ಯಾನ ದಲ್ಲಿ ನಿರತನಾದನು. ಇದರಿಂದ ಆತನ ಮನಸ್ಸಿಗೆ ಶಾಂತಿ ದೊರೆತಂತಾಯಿತು. ಆತನು ತಪಸ್ಸಿದ್ಧಿಯನ್ನು ಪಡೆಯುವುದಕ್ಕಾಗಿ ಗಂಧಮಾದನ ಪರ್ವತಕ್ಕೆ ಹೋದನು. ಆತನನ್ನು ಜಗತ್ತು ಹೀಗೆಂದು ಸ್ತುತಿಸಿತು:

\begin{verse}
ಪ್ರಿಯವ್ರತಕೃತಂ ಕರ್ಮ ಕೋ ನು ಕುರ್ಯಾದ್ವಿನೇಶ್ವರಂ ।\\ಯೋ ನೇಮಿನಿಮ್ನೈರಕರೋಚ್ಛಾಯಾಂ ಘ್ನನ್ ಸಪ್ತವಾರಿಧೀನ್​\\ಭೂಸಂಸ್ಥಾನಂ ಕೃತಂ ಯೇನ ಸರಿದ್ಗಿರಿವನಾದಿಭಿಃ ॥\\ಸೀಮಾ ಚ ಭೂತನಿರ್ವೃತ್ಯೈ ದ್ವೀಪೇ ದ್ವೀಪೇ ವಿಭಾಗಶಃ ॥\\ಭೌಮಂ ದಿವ್ಯಂ ಮಾನುಷಂ ಚ ಮಹತ್ವಂ ಕರ್ಮಯೋಗಜಂ ।\\ಯಶ್ಚಕ್ರೇ ನಿರಯೌಪಮ್ಯಂ ಪುರುಷೋ ನು ಜನಪ್ರಿಯಃ ॥
\end{verse}

ಯಾವ ಪುಣ್ಯಾತ್ಮನು ರಾತ್ರಿಗಳೇ ಇಲ್ಲದಂತೆ ಮಾಡಿ, ತನ್ನ ರಥದ ಗಾಲಿಗಳ ಹಳಿ ಗಳಿಂದ ಸಪ್ತಸಮುದ್ರಗಳನ್ನು ಮಾಡಿದನೋ, ಯಾವ ದಿವ್ಯಪುರುಷನು ಬೇರೆ ಬೇರೆ ದ್ವೀಪ ಗಳನ್ನು ನಿರ್ಮಿಸಿ, ಅವುಗಳಲ್ಲಿ ನದಿ, ಬೆಟ್ಟ, ಕಾಡುಗಳನ್ನೂ ಪ್ರಾಣಿಗಳಿಗೆ ವಸತಿಯನ್ನೂ ಮಾಡಿದನೋ, ಮೂರು ಲೋಕಗಳ ಭೋಗಭಾಗ್ಯಗಳನ್ನೂ ನರಕಸಮಾನವಾಗಿ ಭಾವಿಸಿ ಭಗವದ್​ಭಕ್ತಿಯಿಂದ ಆತ್ಮಜ್ಞಾನವನ್ನು ಪಡೆದನೋ ಆ ಮಹಾಪುರುಷನು ಪರಮೇಶ್ವರ ನಲ್ಲದೆ ಮತ್ತಾವನಾದಾನು?

