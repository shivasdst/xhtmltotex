
\chapter{೬೫. ರುಕ್ಮಿಣೀ, ಭಯಪಡಬೇಡ!}

ಜರಾಸಂಧನ ಕಣ್ಣಿಗೆ ಮಣ್ಣೆರಚಿ ದ್ವಾರಕಾಪುರವನ್ನು ಸೇರಿದ ಬಲರಾಮಕೃಷ್ಣರು ಯಾವ ಆತಂಕವೂ ಇಲ್ಲದೆ ಕೆಲಕಾಲ ಸುಖವಾಗಿದ್ದರು. ಆವರ್ತ ದೇಶದ ರಾಜನಾದ ರೈವತನು ತನ್ನ ಮಗಳಾದ ರೇವತಿಯೆಂಬ ಕನ್ಯಾರತ್ನವನ್ನು ಬಲರಾಮನಿಗೆ ಮದುವೆಮಾಡಿಕೊಟ್ಟನು. ಅನುರೂಪವಾದ ಈ ಗಂಡಹೆಂಡಿರು ಸ್ವರ್ಗಸುಖವನ್ನು ಸೂರೆಗೊಳ್ಳುತ್ತಾ ಜಕ್ಕವಕ್ಕಿ ಗಳಂತೆ ನಲಿದಾಡುತ್ತಿದ್ದರು. ಶ್ರೀಕೃಷ್ಣನಿಗೆ ತಕ್ಕ ಹೆಣ್ಣೂ ಹುಟ್ಟಿ ಬೆಳೆಯುತ್ತಿತ್ತು. ವಿದರ್ಭ ದೇಶದ ರಾಜನಾದ ಭೀಷ್ಮಕನಿಗೆ ರುಕ್ಮಿಯೆ ಮೊದಲಾದ ಐವರು ಗಂಡುಮಕ್ಕಳೂ, ಕಡೆಯ ವಳಾಗಿ ರುಕ್ಮಿಣಿ ಎಂಬ ಹೆಣ್ಣುಮಗಳೂ ಹುಟ್ಟಿದರು. ಈ ರುಕ್ಮಿಣಿ ತನ್ನ ರೂಪು, ಗುಣ, ಶೀಲಗಳಲ್ಲಿ ಎಣೆಯಿಲ್ಲದವಳೆಂದು ಹೆಸರು ಪಡೆದಿದ್ದಳು. ಆಕೆ ಶ್ರೀಕೃಷ್ಣನ ರೂಪು, ಧೈರ್ಯ, ವೀರ್ಯ, ಸಾಹಸಗಳ ಕಥೆಯನ್ನು ಕೇಳಿ ಆತನೆ ತನ್ನ ಪತಿಯೆಂದು ಮನಸಾ ವರಿಸಿದ್ದಳು. ತಂದೆಯಾದ ಭೀಷ್ಮಕನಿಗೂ ಇದು ಒಪ್ಪಿಗೆಯಾಗಿತ್ತು. ಆತನ ಬಂಧುಗಳೆಲ್ಲರೂ ಇದು ಬಹು ಹೊಂದಿಕೆಯಾದ ಜೋಡಿಯಾಗುವುದೆಂದು ಹೇಳಿ, ಈ ವಿವಾಹಕ್ಕೆ ಒತ್ತಾಸೆ ನೀಡಿ ದ್ದರು. ಆದರೆ ಭೀಷ್ಮಕನ ಹಿರಿಯ ಮಗನಾದ ರುಕ್ಮಿ ಮಾತ್ರ ಈ ಅನುಕೂಲದಾಂಪತ್ಯಕ್ಕೆ ಅಡ್ಡಿಯಾಗಿ ಕುಳಿತಿದ್ದನು. ಏನು ಮಾಡಿದರೂ ತನ್ನ ತಂಗಿಯನ್ನು ಶ್ರೀಕೃಷ್ಣನಿಗೆ ಕೊಡುವು ದಿಲ್ಲವೆಂದು ಅವನ ಹಟ. ಅವನು ರುಕ್ಮಿಣಿಯನ್ನು ಚೇದಿದೇಶದ ರಾಜನಾದ ಶಿಶುಪಾಲ ನಿಗೆ ಕೊಟ್ಟು ವಿವಾಹ ಮಾಡುವುದಾಗಿ ನಿಶ್ಚಯಿಸಿ, ಲಗ್ನದ ದಿನವನ್ನೂ ಗೊತ್ತುಮಾಡಿದನು.

ಅಣ್ಣನ ಹಟವನ್ನು ಕಂಡು ರುಕ್ಮಿಣಿಗೆ ಮುಂದೋರದಂತಾಯಿತು. ಆಕೆ ತನಗೆ ಆಪ್ತ ನಾಗಿದ್ದ ಬ್ರಾಹ್ಮಣನೊಬ್ಬನಲ್ಲಿ ತನ್ನ ದುಃಖವನ್ನೆಲ್ಲ ಹೇಳಿಕೊಂಡು, ಒಂದು ಪ್ರೇಮಪತ್ರ ದೊಡನೆ ಅವನನ್ನು ಶ್ರೀಕೃಷ್ಣನ ಬಳಿಗೆ ಕಳುಹಿಸಿದಳು. ಆ ಬ್ರಾಹ್ಮಣ ನೇರವಾಗಿ ದ್ವಾರಕಾ ಪಟ್ಟಣಕ್ಕೆ ಹೋಗಿ, ಶ್ರೀಕೃಷ್ಣನನ್ನು ಕಂಡು ‘ಸ್ವಾಮಿ, ನಾನು ರುಕ್ಮಿಣೀದೇವಿಯ ದೂತ. ಆಕೆ ಈ ಪತ್ರವನ್ನು ತಮಗೆ ಬರೆದು ಕಳುಹಿಸಿದ್ದಾಳೆ’ ಎಂದು ಹೇಳಿ, ರುಕ್ಮಿಣಿಯ ಪತ್ರ ವನ್ನು ಆತನ ಕೈಗಿತ್ತನು. ಶ್ರೀಕೃಷ್ಣನು ಅದನ್ನು ಓದುವಂತೆ ಆ ಬ್ರಾಹ್ಮಣನ ಕೈಗೇ ಕೊಟ್ಟನು. ಆ ಪತ್ರದಲ್ಲಿ ರುಕ್ಮಿಣಿ ತನ್ನ ಹೃದಯದ ನೋವನ್ನೆಲ್ಲ ತೋಡಿಕೊಂಡಿದ್ದಳು. “ಹೇ ನನ್ನ ಜೀವದ ಜೀವ! ಹೇ ಲೋಕಸುಂದರ! ನಿನ್ನ ಗುಣ ರೂಪಗಳಿಗೆ ಮನಸೋತ ನಾನು ನಾಚಿಕೆಯನ್ನು ಬದಿಗೊತ್ತಿ ನನ್ನ ಬಯಕೆಯನ್ನು ನಿನ್ನ ಮುಂದೆ ಇಡುತ್ತಿದ್ದೇನೆ. ಮಹಾಮಹಿಮನಾದ ನಿನಗೆ ನಾನು ಅನುರೂಪಳಲ್ಲವೆಂಬುದನ್ನು ನಾನು ಬಲ್ಲೆ. ನಾನು ಮಾತ್ರವೇ ಅಲ್ಲ; ರೂಪ ಗುಣ ಶೀಲಗಳಲ್ಲಿ ನಿನಗೆ ಅನುರೂಪರಾದವರು ಜಗತ್ತಿನಲ್ಲಿ ಯಾರೂ ಇಲ್ಲ. ನಿನಗೆ ನೀನೆ ಸಾಟಿ, ಅಷ್ಟೆ. ಆದರೂ ನಾನು ನಿನ್ನನ್ನು ಪತಿಯೆಂದು ಮನಸಾ ವರಿಸಿದ್ದೇನೆ. ನನ್ನ ಆತ್ಮವನ್ನೇ ನಿನಗೆ ಅರ್ಪಿಸಿದ್ದೇನೆ. ನೀನು ತಕ್ಷಣವೇ ಇಲ್ಲಿಗೆ ಬಂದು ನನ್ನ ಕೈಹಿಡಿಯಬೇಕು. ಇಲ್ಲದಿದ್ದರೆ ಸಿಂಹದ ಭಾಗ ನರಿಯ ಪಾಲಾಗುತ್ತದೆ. ನನ್ನಣ್ಣ ರುಕ್ಮಿ ನನ್ನನ್ನು ಶಿಶುಪಾಲನ ಕೈಲಿಡಬೇಕೆಂದು ಬಗೆದಿದ್ದಾನೆ. ನಾಳೆ ಆ ಕಾರ್ಯ ನಡೆಯುವು ದಾಗಿದೆ. ವಿವಾಹಕ್ಕೆ ಮುಂಚೆ ಪಟ್ಟಣದ ಹೊರಗಿರುವ ಕುಲದೇವತೆಯ ಗುಡಿಗೆ ಪೂಜೆ ಗೆಂದು ನನ್ನನ್ನು ಕರೆತರುತ್ತಾರೆ. ಆ ಸಮಯದಲ್ಲಿ ನೀನು ನನ್ನನ್ನು ಕರೆದುಕೊಂಡು ಹೋಗುವುದು ಸುಲಭ. ಆಗ ಹೆಚ್ಚು ಕಾವಲೇನೂ ಇರುವುದಿಲ್ಲ. ನೀನು ನನ್ನನ್ನು ಕರೆ ದೊಯ್ಯದಿದ್ದರೆ ನಾನು ಆತ್ಮಹತ್ಯೆಮಾಡಿಕೊಳ್ಳುತ್ತೇನೆ, ಹೆಣ್ಣಿನ ಕೊಲೆಯ ಪಾಪ ನಿನಗೆ ತಟ್ಟುತ್ತದೆ. ಸ್ವಾಮಿ, ನನ್ನ ಪೂರ್ವಪುಣ್ಯವೇನಾದರೂ ಅಲ್ಪಸ್ವಲ್ಪವಿದ್ದರೆ, ಆ ಪುಣ್ಣ ನಿನ್ನನ್ನು ಇಲ್ಲಿಗೆ ಕರೆತರಲಿ. ಇದು ತಪ್ಪಿದರೆ ಜನ್ಮಜನ್ಮಾಂತರಗಳಲ್ಲಿಯಾದರೂ ನೀನೆ ನನ್ನ ಗಂಡನಾಗುವಂತೆ ಅನುಗ್ರಹಿಸಲಿ!”

ಪತ್ರವನ್ನು ಓದಿದ ಬ್ರಾಹ್ಮಣ ಅದರಲ್ಲಿ ಇದ್ದುದನ್ನೆ ರಂಗುರಂಗಾಗಿ ಬಣ್ಣಿಸಿ ಶ್ರೀಕೃಷ್ಣ ನಿಗೆ ಹೇಳಿದ. ರುಕ್ಮಿಣಿಯ ರೂಪ ಗುಣ ಶೀಲಗಳನ್ನೂ ಬಾಯಲ್ಲಿ ನೀರೂರುವಂತೆ ಬಣ್ಣಿ ಸಿದ. ಶ್ರೀಕೃಷ್ಣನು ರುಕ್ಮಿಣಿಯ ದಿವ್ಯ ಸೌಂದರ್ಯವನ್ನು ಕೇಳಿ ಮೊದಲೆ ಮೋಹಗೊಂಡಿದ್ದ, ಈಗ ಬ್ರಾಹ್ಮಣನ ಮಾತಿನಿಂದ ಅದು ಮತ್ತಷ್ಟು ಹೆಚ್ಚಾಯಿತು. ಆತನು ಮುಗುಳ್ ನಗುತ್ತಾ ‘ಪೂಜ್ಯನಾದ ಬ್ರಾಹ್ಮಣ! ನಾನೂ ರುಕ್ಮಿಣಿಯನ್ನು ಮನಸಾ ವರಿಸಿದ್ದೇನೆ. ಅವಳ ಚಿಂತೆಯಿಂದ ನನಗೆ ಹಗಲು ರಾತ್ರಿ ನಿದ್ರೆ ಬರುತ್ತಿಲ್ಲ. ನಾನು ಬಯಸುತ್ತಿದ್ದುದು ತಾನಾಗಿ ಕೈಸೇರುತ್ತಿರುವಾಗ ನನ್ನ ಭಾಗ್ಯವೆ ಭಾಗ್ಯ. ಇನ್ನು ರುಕ್ಮಿಣಿ ಭಯಪಡುವುದು ಬೇಡ. ನಾನು ಸಕಾಲಕ್ಕೆ ಅಲ್ಲಿಗೆ ಹೋಗಿ, ನನ್ನ ಕಾರ್ಯಕ್ಕೆ ಅಡ್ಡಿ ಮಾಡುವವರನ್ನೆಲ್ಲ ಧ್ವಂಸಮಾಡಿ, ರುಕ್ಮಿಣಿಯನ್ನು ಕರೆತರುತ್ತೇನೆ’ ಎಂದು ಹೇಳಿ, ತನ್ನ ರಥವನ್ನು ಸಿದ್ಧಪಡಿಸುವಂತೆ ದಾರುಕ ನಿಗೆ ಅಪ್ಪಣೆಮಾಡಿದನು. ಮರುಕ್ಷಣದಲ್ಲಿಯೆ ನಾಲ್ಕು ಕುದುರೆಗಳನ್ನು ಹೂಡಿದ ಆತನ ರಥ ಸಿದ್ಧವಾಯಿತು. ಶ್ರೀಕೃಷ್ಣನು ಆ ಬ್ರಾಹ್ಮಣನೊಡನೆ ತನ್ನ ರಥವನ್ನೇರಿ ಗಾಳಿಗೆ ಗರಿ ಮೂಡಿದಂತೆ ಪ್ರಯಾಣಮಾಡಿದನು. ಅಷ್ಟರಲ್ಲಿ ಆ ಸುದ್ದಿ ಬಲರಾಮನ ಕಿವಿಗೆ ಬಿತ್ತು. ಆತನು ಸಂದರ್ಭ ಹೇಗಿರುವುದೊ ಎಂಬ ಶಂಕೆಯಿಂದ ತನ್ನ ಸೈನ್ಯವೆಲ್ಲವನ್ನೂ ತೆಗೆದು ಕೊಂಡು ಶ್ರೀಕೃಷ್ಣನನ್ನು ಹಿಂಬಾಲಿಸಿದನು.

ಅತ್ತ ವಿದರ್ಭರಾಜ್ಯದ ರಾಜಧಾನಿಯಾದ ಕುಂಡಿನಪುರದಲ್ಲಿ ರುಕ್ಮಿಣಿಯ ವಿವಾಹ ಕ್ಕಾಗಿ ಸಕಲ ಸಿದ್ಧತೆಗಳೂ ನಡೆದಿದ್ದವು. ಬೀದಿಗಳಿಗೆಲ್ಲ ಪನ್ನೀರನ್ನು ಚೆಲ್ಲಿ, ಅಲ್ಲಲ್ಲಿಯೇ ಚಿಗುರಿನ ತೋರಣಗಳನ್ನು ಕಟ್ಟಿದ್ದರು. ಮನೆಮನೆಯ ಮುಂದೆಯೂ ಸಾರಣೆ ಕಾರಣೆ ರಂಗವಲ್ಲಿಗಳ ಅಲಂಕಾರ ನಡೆದಿತ್ತು. ಮರುದಿನ ಬೆಳಕು ಹರಿಯುತ್ತಿದ್ದಂತೆ ಎಲ್ಲೆ ಲ್ಲಿಯೂ ಸಂಭ್ರಮ, ಸಡಗರ. ಗಂಡು ಹೆಣ್ಣುಗಳು ಹೊಸಬಟ್ಟೆಗಳನ್ನುಟ್ಟು ರತ್ನಾಭರಣ ಗಳನ್ನು ತೊಟ್ಟು ಬೀದಿಗಳಲ್ಲೆಲ್ಲ ಸಡಗರದಿಂದ ಓಡಾಡುತ್ತಿದ್ದಾರೆ. ವರನಾಗಲಿರುವ ಶಿಶುಪಾಲನು ತನ್ನ ಬಂಧುಬಾಂಧವರೊಡನೆ ದೊಡ್ಡ ಸೇನೆಯನ್ನೂ ತೆಗೆದುಕೊಂಡು ಊರ ಬಳಿ ಬಂದಿರುವನೆಂದು ಕೇಳುತ್ತಲೆ ಭೀಷ್ಮಕರಾಜನು ತನ್ನ ಪರಿವಾರದೊಡನೆ ಮಂತ್ರಜ್ಞ ರಾದ ಬ್ರಾಹ್ಮಣರನ್ನು ಮುಂದೆಮಾಡಿಕೊಂಡು ಹೋಗಿ ಆತನನ್ನು ಇದಿರುಗೊಂಡನು. ಮದುವೆ ಗಂಡಿಗೆ ನಡೆಯಬೇಕಾದ ಉಪಚಾರಗಳೆಲ್ಲ ನಡೆದವು. ಆತನನ್ನೂ ಆತನ ಪರಿವಾರವನ್ನೂ ಸೊಗಸಾಗಿ ಅಲಂಕರಿಸಿದ ಒಂದು ಅರಮನೆಯಲ್ಲಿ ಇಳಿಸಿದುದಾಯಿತು. ಅಷ್ಟರಲ್ಲಿ ಮದುವಣಿಗನ ಗೆಳೆಯರಾದ ಜರಾಸಂಧ, ಸಾಲ್ವ, ದಂತವಕ್ತ್ರ, ಪೌಂಡ್ರಕ ಮೊದಲಾದ ರಾಜರು ತಮ್ಮ ತಮ್ಮ ಬಂಧುಗಳೊಡನೆ ಸೇನೆಗಳನ್ನೂ ತೆಗೆದುಕೊಂಡು ಕುಂಡಿನ ನಗರಕ್ಕೆ ಬಂದಿಳಿದರು. ರಾಜಕುಮಾರನಾದ ರುಕ್ಮಿ ಅವರನ್ನೆಲ್ಲ ಗೌರವದಿಂದ ಕಂಡು, ಭವ್ಯ ಭವನಗಳಲ್ಲಿ ಅವರಿಗೆಲ್ಲ ಬಿಡಾರಕ್ಕೆ ಏರ್ಪಡಿಸಿದನು. ಆ ರಾಜರಿಗೆಲ್ಲ ಮನಸ್ಸಿನಲ್ಲಿ ಸ್ವಲ್ಪ ಆತಂಕ. ರುಕ್ಮಿಣಿ ಶ್ರೀಕೃಷ್ಣನನ್ನು ಮೋಹಿಸಿರುವಳೆಂಬ ಗಾಳಿ ವರ್ತ ಮಾನ ಅವರಿಗೆಲ್ಲ ತಿಳಿದಿತ್ತು. ಆದ್ದರಿಂದ ಈ ಮಂಗಳ ಸಮಯದಲ್ಲಿ ಬಲರಾಮ ಕೃಷ್ಣರೆಲ್ಲಿ ಪಿಡಿಗಿನಂತೆ ಎರಗಿ ರಸಭಾಸಮಾಡುತ್ತಾರೊ ಎಂದು ಅವರೆಲ್ಲರ ಅಳುಕು. ಅದಕ್ಕಾಗಿಯೇ ಅವರು ತಮ್ಮ ಸೇನೆಗಳೊಡನೆ ಸಜ್ಜಾಗಿ ಬಂದಿದ್ದುದು. ಅವರೆಲ್ಲ ಒಟ್ಟಾಗಿ ಸೇರಿ ‘ಆ ಕೃಷ್ಣ ಈ ಸಮಯದಲ್ಲಿ ಮದುಮಗಳನ್ನು ಹೊತ್ತುಕೊಂಡು ಹೋದರೂ ಹೋದನೆ! ನಾವು ಎಚ್ಚರದಿಂದಿರಬೇಕು’ ಎಂದು ಮಾತನಾಡಿಕೊಂಡರು.

ಮದುವೆಯ ಮುಹೂರ್ತ ಸಮೀಪಿಸುತ್ತಿದೆಯೆಂದು ಪುರೋಹಿತರು ಎಚ್ಚರಿಕೆ ಯಿತ್ತರು. ಮಂಗಳಸ್ನಾನವನ್ನು ಮಾಡಿದ್ದ ರುಕ್ಮಿಣಿಗೆ ಅರಮನೆಯ ಹೆಣ್ಣುಗಳು ದಿವ್ಯವಾದ ಪೀತಾಂಬರವನ್ನುಡಿಸಿ, ಒಡವೆಗಳನ್ನಿಟ್ಟು, ಹೂ ಮುಡಿಸಿ ಅಲಂಕಾರ ಮಾಡಿದರು. ಪುರೋಹಿತರು ಮಂತ್ರವನ್ನು ಹೇಳಿ ಆಕೆಗೆ ಆಶೀರ್ವಾದಮಾಡಿದರು. ಆಕೆಯ ಕೈಯಿಂದ ಗೋದಾನವೇ ಮೊದಲಾದ ಅನೇಕ ದಾನಗಳನ್ನು ಕೊಡಿಸಿದುದಾಯಿತು. ಆಕೆಯ ದೇಹ ಯಂತ್ರದಂತೆ ಹೇಳಿದ ಕೆಲಸ ಮಾಡುತ್ತಿತ್ತು; ಆದರೆ ಮನಸ್ಸು ಮಾತ್ರ ‘ಅಯ್ಯೋ ಇನ್ನೇನು ಗತಿ? ಇದುವರೆಗೆ ಶ್ರೀಕೃಷ್ಣನ ಸುದ್ದಿ ಸಮಾಚಾರಗಳೊಂದೂ ಇಲ್ಲ. ನಾನು ಕಳುಹಿಸಿದ ಬ್ರಾಹ್ಮಣ ಎಲ್ಲಿ ಹೋದನೊ, ಏನು ಮಾಡಿದನೊ! ಶ್ರೀಕೃಷ್ಣನು ನನ್ನ ಬೇಡಿಕೆಯನ್ನು ಈಡೇರಿಸುವನೊ, ನಿರಾಕರಿಸುವನೊ!’ ಎಂದು ತಳಮಳಗೊಳ್ಳುತ್ತಿತ್ತು. ಅಷ್ಟರಲ್ಲಿ ಅವಳ ಎಡಭುಜ ಅದುರಿತು; ‘ರುಕ್ಮಿಣಿ, ಹೆದರಬೇಡ’ ಎಂದು ಭರವಸೆ ಕೊಟ್ಟಂತಾಯಿತು. ಅದರ ಹಿಂದೆಯೇ ತಾನು ಕಳುಹಿಸಿದ್ದ ಬ್ರಾಹ್ಮಣನೂ ಅಲ್ಲಿ ಪ್ರತ್ಯಕ್ಷನಾದ. ಅವನ ನಗು ಮುಖವೂ ‘ರುಕ್ಮಿಣಿ, ನೀನು ಹೆದರಬೇಡ’ ಎಂದೇ ಸಂದೇಶವನ್ನು ಸಾರುತ್ತಿತ್ತು. ಅವನು ಶ್ರೀಕೃಷ್ಣನಿಂದ ತಂದ ಸುದ್ದಿಯೂ ಅದೇ ಆಗಿತ್ತು. ‘ರುಕ್ಮಿಣಿ, ನೀನು ಹೆದರಬೇಡ. ಸಕಾಲಕ್ಕೆ ನಾನಲ್ಲಿಗೆ ಬರುವೆನು’ ಎಂಬ ಶ್ರೀಕೃಷ್ಣನ ವಾಕ್ಯಗಳನ್ನು ಕೇಳಿ ಅವಳ ಮುಖ ಅರಳಿತು, ಆಕೆಗೆ ಬ್ರಾಹ್ಮಣನನ್ನು ಹೇಗೆ ಬಹುಮಾನಿಸಬೇಕೆಂಬುದೇ ಗೊತ್ತಾಗದೆ, ಆತ ನಿಗೆ ಅಡ್ಡಬಿದ್ದಳು.

ಮದುಮಗಳಾದ ರುಕ್ಮಿಣಿಯನ್ನು, ಮನೆದೇವತೆಯಾದ ಅಂಬಿಕೆಯ ಪೂಜೆಗೆಂದು ಊರ ಹೊರಗಿದ್ದ ದೇವಾಲಯಕ್ಕೆ ಕರೆದುಕೊಂಡು ಹೊರಟರು. ಆ ವೇಳೆಗೆ ಸರಿಯಾಗಿ ಶ್ರೀಕೃಷ್ಣ ಬಲರಾಮರು ಬಂದರೆಂಬ ಸುದ್ದಿ ಬಂತು. ಸರಳ ಹೃದಯನಾದ ಭೀಷ್ಮಕನು, ಇತರರಂತೆ ಅವರೂ ತನ್ನ ಮಗಳ ಮದುವೆಯನ್ನು ನೋಡಬಂದವರೆಂದು ಭಾವಿಸಿ, ಅವರನ್ನು ಸತ್ಕರಿಸಿ ಬಿಡಾರದಲ್ಲಿ ಇಳಿಸುವಂತೆ ಏರ್ಪಡಿಸಿದನು. ಶ್ರೀಕೃಷ್ಣ ಬಂದನೆಂಬ ಸುದ್ದಿ ಕ್ಷಣಮಾತ್ರದಲ್ಲಿ ಊರಲ್ಲೆಲ್ಲ ಹಬ್ಬಿತು. ಹೆಣ್ಣು ಗಂಡುಗಳು ಮನೆಯಿಂದ ಹೊರ ಕ್ಕೋಡಿ ಬಂದು ಅವನ ಚೆಲುವನ್ನು ಸವಿಯುತ್ತಾ ‘ನಮ್ಮ ರಾಜಪುತ್ರಿ ಈತನ ಕೈಹಿಡಿದಿದ್ದರೆ ಜೋಡಿ ಎಷ್ಟು ಚೆನ್ನಾಗಿರುತ್ತಿತ್ತು!’ ಎಂದು ಗುಸುಗುಟ್ಟುತ್ತಿದ್ದರು. ಅಷ್ಟರಲ್ಲಿ ಅರಮನೆ ಯಿಂದ ಮದುಮಗಳ ಸವಾರಿ ಅಂಬಿಕಾಗುಡಿಗೆಂದು ಕಾಲ್ನಡಿಗೆಯಲ್ಲಿಯೆ ಹೊರಟಿತು. ಆಕೆಯ ಮುಂದುಗಡೆಯಲ್ಲಿ ಮಂಗಳವಾದ್ಯಗಳು ಭೋರ್ಗರೆಯುತ್ತಿದ್ದವು. ಅವಳ ಸುತ್ತಲೂ ಪೂಜೆಯ ಸಾಮಾನನ್ನು ಹೊತ್ತ ಸಖಿಯರು ಮಂಗಳಗೀತಗಳನ್ನು ಹಾಡುತ್ತಿ ದ್ದರು. ಆಕೆಯ ತಾಯ್ತಂದೆಗಳು ಆಕೆಯನ್ನು ಹಿಂಬಾಲಿಸಿ ಬರುತ್ತಿದ್ದರು. ಅವರ ಹಿಂದೆ ಹಣ್ಣುಕಾಯಿಗಳನ್ನು ಹೊತ್ತ ದಾಸದಾಸಿಯರೂ ಒರೆಗಳಚಿದ ಕತ್ತಿಗಳೊಡನೆ ಸೈನ್ಯದ ಭಟರೂ ಬರುತ್ತಿದ್ದರು. ದಿಬ್ಬಣವು ಅತ್ಯಂತ ವೈಭವದಿಂದ ನಡೆದು ಅಂಬಿಕೆಯ ಗುಡಿ ಯನ್ನು ಸೇರಿತು. ರುಕ್ಮಿಣಿಯು ಕೈಕಾಲುಗಳನ್ನು ತೊಳೆದುಕೊಂಡು ಗುಡಿಯನ್ನು ಹೊಕ್ಕ ವಳೆ ಅಂಬಿಕೆಗೆ ಅಡ್ಡಬಿದ್ದು ‘ಅಮ್ಮ, ಶ್ರೀಕೃಷ್ಣನೆ ನನಗೆ ಗಂಡನಾಗುವಂತೆ ಅನುಗ್ರಹಿಸು’ ಎಂದು ಕಣ್ಮುಚ್ಚಿ ಪ್ರಾರ್ಥಿಸಿದಳು. ಹಿರಿಯ ಮುತ್ತೈದೆಯರು ಆಕೆಯಿಂದ ವಿಧಿಪೂರ್ವಕ ವಾಗಿ ಪೂಜೆಮಾಡಿಸಿದರು. ಅಲ್ಲಿ ನೆರೆದಿದ್ದ ಮುತ್ತೈದೆಯರಿಗೆಲ್ಲ ಫಲತಾಂಬೂಲಗಳನ್ನು ಹಂಚಿದುದಾಯಿತು. ಕಡೆಯಲ್ಲಿ ಅಂಬಿಕೆಯ ಪ್ರಸಾದವನ್ನು ಪಡೆದು ಎಲ್ಲರೂ ಗುಡಿ ಯಿಂದ ಹೊರಗೆ ಬಂದರು. ರುಕ್ಮಿಣಿಯನ್ನು ಹಿಂದಕ್ಕೆ ಕರೆದೊಯ್ಯುವುದಕ್ಕಾಗಿ ಅರ ಮನೆಯ ರಥ ಬಂದು ಗುಡಿಯ ಬಾಗಿಲಲ್ಲಿ ನಿಂತಿತ್ತು. ರುಕ್ಮಿಣಿಗೆ ಅದೊಂದು ಸಾವು ಬದುಕಿನ ಘಳಿಗೆ. ಅವಳು ಹಂಸದಂತೆ ಹೆಜ್ಜೆಯನ್ನಿಡುತ್ತಾ, ಹೆಜ್ಜೆ ಹೆಜ್ಜೆಗೂ ತಲೆಯೆತ್ತಿ ಸುತ್ತಲೂ ನೋಡುತ್ತಾ ರಥದ ಬಳಿಗೆ ನಡೆದು ಬರುತ್ತಿದ್ದಳು. ‘ಕೃಷ್ಣ, ಕೃಷ್ಣ, ಎಲ್ಲಿರುವೆ?’ ಎಂದು ಮನಸ್ಸು ಒಳಗೆ ಮೊರೆಯುತ್ತಿದೆ. ಆಕೆ ರಥದ ಹತ್ತಿರಕ್ಕೆ ಬಂದಳು. ಇನ್ನೂ ಕೃಷ್ಣನ ಸುಳಿವಿಲ್ಲ. ಆಕೆ ಬಾಯಾರಿದ ಕಣ್ಣಿನಿಂದ ಮತ್ತೊಮ್ಮೆ ಸುತ್ತಲೂ ನೋಡಿ, ತನ್ನ ಕಾಲನ್ನೆತ್ತಿ ರಥದಲ್ಲಿಡಬೇಕು, ಅಷ್ಟರಲ್ಲಿ ಶ್ರೀಕೃಷ್ಣನು ರಭಸದಿಂದ ಅಲ್ಲಿಗೆ ನುಗ್ಗಿ ಬಂದು, ರುಕ್ಮಿಣಿ ಯನ್ನು ಅನಾಮತ್ತಾಗಿ ಎತ್ತಿಕೊಂಡವನೆ ಕ್ಷಣಮಾತ್ರದಲ್ಲಿ ಮಿಂಚಿನಂತೆ ಮಾಯ ವಾದನು. ಸುತ್ತಲೂ ನೆರೆದಿದ್ದವರು ಮಂಕರಂತೆ ನೋಡುತ್ತಿರಲು ಆತನು ರುಕ್ಮಿಣಿಯನ್ನು ರಥದಲ್ಲಿ ಕುಳ್ಳಿರಿಸಿಕೊಂಡು, ನಾಗಾಲೋಟದಿಂದ ರಥವನ್ನು ಓಡಿಸಿದನು. ನರಿಗಳ ಮಧ್ಯದಲ್ಲಿದ್ದ ಆಹಾರ ಸಿಂಹಕ್ಕೆ ದಕ್ಕಿ ಹೋಯಿತು.

ಶ್ರೀಕೃಷ್ಣನು ರುಕ್ಮಿಣಿಯನ್ನು ಕದ್ದೊಯ್ದನೆಂಬ ಸುದ್ದಿ ಕ್ಷಣಮಾತ್ರದಲ್ಲಿ ಕಾಡುಕಿಚ್ಚಿ ನಂತೆ ಹಬ್ಬಿತು. ಮದುಮಗನಾಗಲು ಅಲಂಕರಿಸಿಕೊಂಡು ಅಣಿಯಾಗಿದ್ದ ಶಿಶುಪಾಲನು ಆ ಸುದ್ದಿಯನ್ನು ಕೇಳುತ್ತಲೆ ಕನಲಿ ಕೆಂಗೆಂಡನಾದನು. ಅವನು ತನ್ನ ಅಲಂಕಾರವನ್ನೆಲ್ಲ ಕಿತ್ತೆಸೆದು, ಯುದ್ಧಕವಚವನ್ನು ಧರಿಸಿ, ತನ್ನ ಸೈನ್ಯದೊಡನೆ ಶ್ರೀಕೃಷ್ಣನನ್ನು ಬೆನ್ನಟ್ಟಿದನು. ಉಳಿದ ರಾಜರೂ ತಮ್ಮ ಸೈನ್ಯಗಳೊಡನೆ ಆತನನ್ನು ಹಿಂಬಾಲಿಸಿದರು. ಆದರೆ ಅವರು ಊರುಬಾಗಿಲನ್ನು ದಾಟುತ್ತಿದ್ದಂತೆಯೆ ಯಾದವಸೇನೆ ಅವರಿಗೆ ಇದಿರಾಯಿತು. ಕೋಪ ಗೊಂಡಿದ್ದ ರಾಜರು ಅವರ ಮೇಲೆ ಬಾಣಗಳ ಮಳೆಯನ್ನೆ ಸುರಿಸಿದರು. ಆದರೇನು? ಅದೆಲ್ಲ ಬೆಟ್ಟದ ಮೇಲೆ ಮಳೆ ಸುರಿದಂತಾಯಿತು. ಆದರೂ ಶ್ರೀಕೃಷ್ಣನೊಡನೆ ರಥದಲ್ಲಿ ಕುಳಿತು ಇದನ್ನು ನೋಡುತ್ತಿದ್ದ ರುಕ್ಮಿಣಿ ಭಯದಿಂದ ನಡುಗುತ್ತಾ ತನ್ನ ಗಂಡನ ಮುಖ ವನ್ನು ನೋಡಿದಳು. ಆಗ ಶ್ರೀಕೃಷ್ಣನು ‘ರುಕ್ಮಿಣಿ, ನೀನು ಭಯಪಡಬೇಡ. ನಮ್ಮ ಸೇನೆ ಶತ್ರು ಸೇನೆಯನ್ನು ಕ್ಷಣಮಾತ್ರದಲ್ಲಿ ನಾಶಮಾಡುತ್ತದೆ’ ಎಂದನು. ಆತ ಹೇಳಿದಂತೆಯೆ ನಡೆಯಿತು. ಶತ್ರುಗಳ ಆನೆ ಕುದುರೆ ಕಾಲಾಳುಗಳು ಲೆಕ್ಕವಿಲ್ಲದಷ್ಟು ಸತ್ತುಬಿದ್ದವು, ರಥ ಗಳೆಲ್ಲ ಪುಡಿಪುಡಿಯಾದವು. ತಮ್ಮ ಸೈನ್ಯವೆಲ್ಲವೂ ದಿಕ್ಕಾಪಾಲಾಗಿ ಓಡುತ್ತಿರುವುದನ್ನು ಕಂಡು ಜರಾಸಂಧನೆ ಮೊದಲಾದ ರಾಜರೆಲ್ಲ ಯುದ್ಧಭೂಮಿಯಿಂದ ಕಾಲಿಗೆ ಬುದ್ಧಿ ಹೇಳಿದರು. ಆಶಾಭಂಗವಾದ ಶಿಶುಪಾಲನ ಮುಖ ಕಳೆಗುಂದಿ, ಆತನು ತಲೆತಗ್ಗಿಸಿ ನಿಂತನು. ಆಗ ಉಳಿದ ರಾಜರು ಆತನೊಡನೆ ‘ಅಯ್ಯಾ, ಯುದ್ಧದಲ್ಲಿ ಜಯ, ಅಪಜಯ ಸ್ಥಿರವಲ್ಲ. ಈಗ ನಾವು ಸೋತೆವೆಂದು ಸಂಕಟಪಡುವುದು ಬೇಡ. ಇಂದು ಸೋತರೇನು? ನಾಳೆ ಗೆಲ್ಲುತ್ತೇವೆ. ನೋಡು, ಜರಾಸಂಧ ಹದಿನೇಳು ಸಲ ಸೋತರೂ ಹದಿನೆಂಟನೆಯ ಸಲ ಗೆದ್ದ. ಸೋತಾಗ ಎಂದೂ ಆತ ಕಂಗಾಲಾಗಲಿಲ್ಲ. ನೀನೂ ಆತನಂತೆ ಧೈರ್ಯಶಾಲಿ ಯಾಗಬೇಕು’ ಎಂದರು. ಇದರಿಂದ ಸಮಾಧಾನಗೊಂಡ ಶಿಶುಪಾಲ ತನ್ನ ಪರಿವಾರ ದೊಡನೆ ಅಲ್ಲಿಂದ ಹಿಂದಿರುಗಿದ. ಉಳಿದ ರಾಜರೂ ಭಾರವಾದ ಮನಸ್ಸಿನಿಂದ ಹಿಂದಿರುಗಿದರು.

ಶ್ರೀಕೃಷ್ಣನು ರುಕ್ಮಿಣಿಯನ್ನು ಕದ್ದೊಯ್ದ ಅಪರಾಧವನ್ನು ಯಾರು ಕ್ಷಮಿಸಿದರೂ ರುಕ್ಮಿ ಮಾತ್ರ ಕ್ಷಮಿಸಲಿಲ್ಲ. ಆತನು ತನ್ನ ಒಂದು ಅಕ್ಷೋಹಿಣಿ ಸೈನ್ಯದೊಡನೆ ಕೃಷ್ಣನ ಬೆನ್ನಟ್ಟಿದ. ಶ್ರೀಕೃಷ್ಣನನ್ನು ಕೊಂದು, ರುಕ್ಮಿಣಿಯನ್ನು ಬಿಡಿಸಿ ತರದೆ ತಾನು ಮತ್ತೆ ಊರನ್ನು ಪ್ರವೇಶಿ ಸುವುದಿಲ್ಲವೆಂಬುದು ಆತನ ಪ್ರತಿಜ್ಞೆ. ಆತನು ತನ್ನ ಸಾರಥಿಯನ್ನು ಕುರಿತು ‘ಅಯ್ಯಾ, ನನ್ನ ಗುರಿ ಆ ಕಳ್ಳ ಕೃಷ್ಣ. ಅವನ ರಥ ಎತ್ತ ಹೋಗುವುದೋ ಅತ್ತ ಕಡೆ ನನ್ನ ರಥ ಹರಿಯಲಿ. ಆ ಗೊಲ್ಲನಿಗೆ ತಕ್ಕ ಶಿಕ್ಷೆ ಮಾಡದ ಹೊರತು ನನ್ನ ಜೀವಕ್ಕೆ ಸಮಾಧಾನವಿಲ್ಲ’ ಎಂದನು. ಆತನ ಅಪ್ಪಣೆಯಂತೆ ಸಾರಥಿ ರಥವನ್ನು ವೇಗವಾಗಿ ನಡೆಸಿ, ಶ್ರೀಕೃಷ್ಣನ ರಥ ವನ್ನು ಬೆನ್ನಟ್ಟಿದನು. ರುಕ್ಮಿಯು ದೂರದಿಂದಲೆ ‘ಎಲ ಕಳ್ಳ, ನಿಲ್ಲು ನಿಲ್ಲು’ ಎಂದು ಕೂಗಿ, ಹರಿತವಾದ ಮೂರು ಬಾಣಗಳನ್ನು ಶ್ರೀಕೃಷ್ಣನ ಮೇಲೆ ಬಿಟ್ಟನು. ಇದನ್ನು ಕಂಡು ರುಕ್ಮಿಣಿ ಬೆದರಿದ ಹುಲ್ಲೆಯಂತಾದಳು. ಶ್ರೀಕೃಷ್ಣನು ಮುಗುಳ್ನಗುತ್ತಾ ಆಕೆಗೆ ‘ರುಕ್ಮಿಣಿ, ನೀನು ಹೆದರಬೇಡ’ ಎಂದು ಹೇಳಿ, ತನ್ನ ಬಿಲ್ಲಿನಿಂದ ಬಾಣವನ್ನು ಹೂಡಿದವನೆ ಮೊದಲ ಬಾಣದಿಂದ ರುಕ್ಮಿಯ ಬಿಲ್ಲನ್ನು ಕತ್ತರಿಸಿ ಹಾಕಿ, ಆಮೇಲೆ ಕುದುರೆ ಸಾರಥಿಗಳೊಡನೆ ರುಕ್ಮಿಯನ್ನೂ ತನ್ನ ಬಾಣಗಳಿಂದ ಹೊಡೆದು ನೋಯಿಸಿದನು. ಇದನ್ನು ಕಂಡು ಕನಲಿದ ರುಕ್ಮಿ ಬೇರೊಂದು ಬಿಲ್ಲನ್ನು ತೆಗೆದುಕೊಂಡು ಯುದ್ಧಕ್ಕೆ ನಿಂತನು. ಬುಡಕ್ಕೆ ಕೊಡಲಿ ಹಾಕಿದಂತೆ ಶ್ರೀಕೃಷ್ಣ ಮತ್ತೆ ಆತನ ಬಿಲ್ಲನ್ನು ಕತ್ತರಿಸಿ ಹಾಕಿದನು. ರುಕ್ಮಿ ಯಾವ ಆಯುಧ ವನ್ನು ಕೈಲಿ ಹಿಡಿದರೂ ಮರುನಿಮಿಷದಲ್ಲಿ ಅದನ್ನು ಕತ್ತರಿಸಿಹಾಕುತ್ತಿದ್ದ ಶ್ರೀಕೃಷ್ಣ. ರುಕ್ಮಿಗೆ ರೇಗಿಹೋಯಿತು; ಆತ ತನ್ನ ರಥದಿಂದ ಧುಮ್ಮಿಕ್ಕಿದವನೆ ಹಿರಿದ ಕತ್ತಿಯೊಡನೆ ಶ್ರೀಕೃಷ್ಣನ ರಥದೊಳಕ್ಕೆ ನುಗ್ಗಿದನು. ಶ್ರೀಕೃಷ್ಣನು ಅವನ ಕೈಯ ಕತ್ತಿಯನ್ನು ಕಿತ್ತುಕೊಂಡು, ಅದ ರಿಂದಲೆ ಅವನನ್ನು ಕೊಲ್ಲಲೆಂದು ಕತ್ತಿಯನ್ನು ಮೇಲಕ್ಕೆತ್ತಿದನು. ಇದನ್ನು ಕಂಡು ರುಕ್ಮಿಣಿ ಗಡಗಡ ನಡುಗುತ್ತಾ ‘ದೇವದೇವ, ಲೋಕೇಶ್ವರ, ಮಹಾಮಹಿಮ, ಈ ನನ್ನ ಅಣ್ಣ ಅಜ್ಞ; ನಿನ್ನನ್ನು ದ್ವೇಷಿಸುತ್ತಾನೆ. ಅವನು ನಿನಗೆ ಸರಿಯೆ, ಸಮನೆ? ನನಗಾಗಿ ಅವನನ್ನು ಕ್ಷಮಿಸು’ ಎಂದು ಬೇಡಿಕೊಂಡಳು. ಆಗ ಶ್ರೀಕೃಷ್ಣನು ‘ರುಕ್ಮಿಣಿ, ಹೆದರಬೇಡ. ನಾನು ಇವನನ್ನು ಕೊಲ್ಲುವುದಿಲ್ಲ. ಆದರೆ ಇವನಿಗೆ ಅವಮಾನವಾಗುವಂತೆ ಮಾಡಿ, ಇವನನ್ನು ಕೈಬಿಡುತ್ತೇನೆ’ ಎಂದು ಹೇಳಿ, ರುಕ್ಮಿ ಹೊದೆದಿದ್ದ ಬಟ್ಟೆಯಿಂದಲೇ ಅವನ ಕೈಕಾಲುಗಳನ್ನು ಕಟ್ಟಿ ಹಾಕಿ, ತನ್ನ ಕೈಲಿದ್ದ ಕತ್ತಿಯಿಂದಲೆ ಅವನ ತಲೆಗೂದಲನ್ನೂ ಮೀಸೆಯನ್ನೂ ಬೋಳಿಸಿಹಾಕಿದನು.

ಶ್ರೀಕೃಷ್ಣನು ರುಕ್ಮಿಗೆ ತಕ್ಕ ಶಿಕ್ಷೆಯನ್ನು ವಿಧಿಸುವ ವೇಳೆಗೆ ಬಲರಾಮನು ತನ್ನ ಯದು ಸೇನೆಯ ಸಹಾಯದಿಂದ ರುಕ್ಮಿಯ ಸೇನೆಯನ್ನೆಲ್ಲ ಬಡಿದೋಡಿಸಿ, ಶ್ರೀಕೃಷ್ಣನ ರಥದ ಬಳಿಗೆ ಬಂದನು. ಅಲ್ಲಿ ಬದುಕಿದ ಹೆಣದಂತೆ ಬಿದ್ದಿದ್ದ ರುಕ್ಮಿಯನ್ನು ಕಂಡು ಆತನಿಗೆ ‘ಅಯ್ಯೋ’ ಎನಿಸಿತು. ಆತನು ಅವನ ಕಟ್ಟುಗಳನ್ನೆಲ್ಲ ಬಿಚ್ಚಿ, ‘ಏನಯ್ಯಾ, ಕೃಷ್ಣ, ಬಂಧು ವಾದ ಈತನಿಗೆ ಹೀಗೆ ಅವಮಾನಮಾಡಿದುದು ಶುದ್ಧ ತಪ್ಪು. ಇದು ನಮಗೆ ಯೋಗ್ಯವಾದ ಕೆಲಸವಲ್ಲ’ ಎಂದು ಅವನನ್ನು ಚೀಮಾರಿ ಮಾಡಿದನು. ಅನಂತರ ರುಕ್ಮಿಣಿಯ ಕಡೆ ತಿರುಗಿ ಕೊಂಡು ‘ಮಗಳೆ, ನಿನ್ನ ಅಣ್ಣನಿಗಾದ ಅವಮಾನವನ್ನು ಮನಸ್ಸಿಗೆ ಹಚ್ಚಿಕೊಳ್ಳಬೇಡ. ಹಾಳು ಕ್ಷತ್ರಿಯರ ಸ್ವಭಾವವೇ ಹೀಗೆ. ಅವರಿಗೆ ಬಂಧು ಬಳಗ ಒಂದೂ ಇಲ್ಲ. ಅನ್ಯಾಯ ವಾಯಿತು ಎನಿಸಿದರೆ ಅವನನ್ನು ಹೊಡೆದುಹಾಕಬೇಕು–ಎಂಬುದೇ ಅವರ ಸ್ವಭಾವ. ಶ್ರೀಕೃಷ್ಣ ಹೀಗೆ ಮಾಡಬಾರದಿತ್ತು; ಮಾಡಿಬಿಟ್ಟನಲ್ಲ! ಏನು ಮಾಡುವುದು? ನಡೆದು ಹೋದುದಕ್ಕಾಗಿ ನೀನು ಚಿಂತಿಸಬೇಡ. ಇದೆಲ್ಲ ಪೂರ್ವಾರ್ಜಿತ ಕರ್ಮ ಎಂದುಕೊಂಡು ಸಮಾಧಾನ ಮಾಡಿಕೊ’ ಎಂದು ಸಮಾಧಾನ ಮಾಡಿದನು. ರುಕ್ಮಿಣಿಯೇನೊ ಸಮಾಧಾನ ತಂದುಕೊಂಡಳು. ಆದರೆ ರುಕ್ಮಿ ತನಗಾದ ಅವಮಾನದಿಂದ ಊರಿನವರಿಗೆ ತನ್ನ ಮುಖ ವನ್ನು ತೋರಿಸಲಾರದೆ, ಆತನು ಅಲ್ಲಿಂದ ಹೊರಟು ಭೋಜಕಟವೆಂಬ ಒಂದು ಹಳ್ಳಿ ಯನ್ನು ಸೇರಿ, ಅಲ್ಲಿಯೇ ನೆಲಸಿದನು. ಇದರಿಂದ ಆತನು ಊರು ಬಿಡುವ ಮುನ್ನ ಮಾಡಿದ ಪ್ರತಿಜ್ಞೆಯೂ ನೆರವೇರಿದಂತಾಯಿತು.

ಇತ್ತ ಶ್ರೀಕೃಷ್ಣನು ರುಕ್ಮಿಣಿಯೊಡನೆ ದ್ವಾರಕಾನಗರಿಗೆ ಸುಖವಾಗಿ ಬಂದು ಸೇರಿದನು. ಊರ ಜನರು ಅತ್ಯಂತ ಉತ್ಸಾಹದಿಂದ ಅವರನ್ನು ಇದಿರುಗೊಂಡು, ಮೆರವಣಿಗೆಯಲ್ಲಿ ಅವರನ್ನು ಅರಮನೆಗೆ ಕರೆದೊಯ್ದರು. ಮರುದಿವಸ ಕೃಷ್ಣರುಕ್ಮಿಣಿಯರ ವಿವಾಹವು ಅತ್ಯಂತ ವೈಭವದಿಂದ ನಡೆಯಿತು. ದ್ವಾರಕಾನಗರಿಯು ದೇವೇಂದ್ರನ ಅಮರಾವತಿ ಗಿಂತಲೂ ಹೆಚ್ಚು ಸುಂದರವಾಗಿ ಅಲಂಕೃತವಾಯಿತು. ಮನೆಮನೆಯ ಮುಂದೆಯೂ ತಳಿರು ತೋರಣ, ಬಾಳೆಯ ಕಂಬ, ಅಡೆಕೆಯ ಗೊನೆ; ಅಂಗಳಗಳೆಲ್ಲ ಸಾರಣೆ ಕಾರಣೆ ರಂಗವಲ್ಲಿ ಅರಳು ಅಕ್ಷತೆಗಳಿಂದ ಅಲಂಕೃತವಾಗಿದ್ದವು. ಎಲ್ಲಿ ನೋಡಿದರೂ ಧ್ವಜಗಳು, ಪತಾಕಿಗಳು, ಹೂಹಾರಗಳು; ಎಲ್ಲಿ ನೋಡಿದರೂ ಸಿಂಗಾರ ಮಾಡಿಕೊಂಡ ಗಂಡುಹೆಣ್ಣು ಗಳು! ಎಲ್ಲಿ ನೋಡಿದರೂ ಅಗರು, ಚಂದನ, ಶ್ರೀಗಂಧಗಳ ಸುವಾಸನೆ! ಕುರು, ಕೇಕಯ, ವಿದರ್ಭ, ಯದು, ಕುಂತಿ ಮೊದಲಾದ ದೇಶಗಳ ಅರಸರು ಮದುವೆಯ ಮಂಟಪದಲ್ಲಿ ಸಡಗರದಿಂದ ಓಡಾಡುತ್ತಾ ನೂತನ ದಂಪತಿಗಳಿಗೆ ಕೈಗಾಣಿಕೆಗಳನ್ನು ಕೊಟ್ಟರು. ಶ್ರೀ ಕೃಷ್ಣನ ಸಾಹಸವು ಬಹುಕಾಲದವರೆಗೆ, ದೇಶದೇಶದ ಜನರಿಗೆಲ್ಲ ಸವಿನೆನಪಾಗಿ ಪ್ರಚಾರದಲ್ಲಿತ್ತು.

