
\chapter{೬೦. ಗುರುದಕ್ಷಿಣೆ}

ನಂದಗೋಕುಲದಲ್ಲಿ ಇದುವರೆಗೂ ಗೊಲ್ಲರ ಮಧ್ಯದ ಗೊಲ್ಲರಂತೆ ಬೆಳೆದ ಬಲ ರಾಮಕೃಷ್ಣರು ಈಗ ಮಧುರಾಪುರಿಯಲ್ಲಿ ಯದುವಂಶದ ಹಿರಿಮೆ ಗರಿಮೆಗಳಿಗೆ ತಕ್ಕಂತೆ ಹೊಂದಿಕೊಳ್ಳಬೇಕಾಯಿತು. ವಸುದೇವನು ತನ್ನ ಮಕ್ಕಳಿಗೆ ಉಪನಯನದ ವಯಸ್ಸಾಗಿರು ವುದನ್ನು ಕಂಡು ಕುಲಪುರೋಹಿತರಾದ ಗರ್ಗರ ಸಹಾಯದಿಂದ ಒಂದು ಶುಭಮುಹೂರ್ತ ದಲ್ಲಿ ಅವರಿಬ್ಬರಿಗೂ ವೈಭವದಿಂದ ಉಪನಯನ ಮಹೋತ್ಸವವನ್ನು ನೆರವೇರಿಸಿದನು. ಅನಂತರ ವಸುದೇವನು ಬ್ರಹ್ಮಚಾರಿಗಳಾದ ಅವರೀರ್ವರನ್ನೂ ಅವಂತೀಪುರದಲ್ಲಿದ್ದ ಸಾಂದೀಪನರೆಂಬ ಗುರುಗಳ ಬಳಿಯಲ್ಲಿ ವಿದ್ಯಾಭ್ಯಾಸಕ್ಕಾಗಿ ಕಳುಹಿಸಿಕೊಟ್ಟನು. ಸಕಲ ವಿದ್ಯೆಗಳಿಗೂ ತಾವೇ ತೌರುಮನೆಯಾದರೂ ಆ ಮಕ್ಕಳು ಲೋಕದ ನಟನೆಗಾಗಿ ವಿದ್ಯಾರ್ಥಿ ಗಳಂತೆ ಗುರುವನ್ನು ಭಕ್ತಿಯಿಂದ ಸೇವಿಸುತ್ತಿದ್ದರು. ಅವರ ಭಕ್ತಿಗೆ ಮೆಚ್ಚಿದ ಸಾಂದೀಪನರು ವೇದವೇದಾಂತಗಳೊಡನೆ ಧರ್ಮಶಾಸ್ತ್ರ, ಧನುರ್ವೇದ, ರಾಜನೀತಿ ಮೊದ ಲಾದ ಕ್ಷತ್ರಿಯ ವಿದ್ಯೆಗಳನ್ನೂ ಅವರಿಗೆ ಉಪದೇಶಿಸಿದರು. ಅವರೇನು ಸಾಮಾನ್ಯ ಮಕ್ಕಳೆ? ಒಮ್ಮೆ ಗುರು ಹೇಳಿದುದನ್ನು ಕೇಳುತ್ತಲೆ ಅವರು ಅದನ್ನು ಗ್ರಹಿಸಿಬಿಡುತ್ತಿದ್ದರು; ಅರುವತ್ತುನಾಲ್ಕು ವಿದ್ಯೆಗಳಲ್ಲೂ ಪಾರಂಗತರಾದರು. ಹೀಗೆ ತಮ್ಮ ವಿದ್ಯಾಭ್ಯಾಸವನ್ನು ಮಾಡಿ ಮುಗಿಸಿದ ಬಲರಾಮ ಕೃಷ್ಣರು ತಾವು ಸಲ್ಲಿಸಬೇಕಾದ ಗುರುದಕ್ಷಿಣೆಯೇನೆಂದು ಗುರುಗಳನ್ನು ಪ್ರಾರ್ಥಿಸಿದರು. ಆ ವೇಳೆಗೆ ಅವರ ಮಹಿಮೆಯನ್ನು ಅರಿತಿದ್ದ ಗುರುಗಳು ಆ ಮಕ್ಕಳ ಯೋಗ್ಯತೆಗೆ ತಕ್ಕ ಗುರುದಕ್ಷಿಣೆಯನ್ನು ಕೇಳಿದರು– ‘ಮಕ್ಕಳೆ, ಈಗ ಕೆಲವು ಕಾಲದ ಹಿಂದೆ ನನ್ನ ಮಗ ಪ್ರಭಾಸಕ್ಷೇತ್ರದಲ್ಲಿರುವ ಸಮುದ್ರದಲ್ಲಿ ಬಿದ್ದುಹೋದ. ನೀವು ಅವನನ್ನು ಕರೆತಂದು ನನಗೆ ಒಪ್ಪಿಸಬೇಕು. ಮಕ್ಕಳಿಬ್ಬರೂ ‘ತಥಾಸ್ತು’ ಎಂದು ಹೇಳಿ ಅಲ್ಲಿಂದ ಹೊರಟರು.

ಸಾಂದೀಪನರಿಂದ ಹೊರಟ ಬಲರಾಮಕೃಷ್ಣರು ನೇರವಾಗಿ ಪ್ರಭಾಸಕ್ಷೇತ್ರಕ್ಕೆ ಬಂದು, ಸಮುದ್ರತೀರಕ್ಕೆ ಹೋದರು. ಅವರ ಮಹಿಮೆಯನ್ನರಿತ ಸಮುದ್ರರಾಜನು ಕೈಗಾಣಿಕೆ ಗಳೊಡನೆ ಅವರ ಬಳಿಗೆ ಓಡಿಬಂದನು. ಶ್ರೀಕೃಷ್ಣನು ಆತನನ್ನು ಕುರಿತು ‘ಅಯ್ಯಾ, ನಿನ್ನ ಕಪ್ಪಕಾಣಿಕೆ, ಆದರೋಪಚಾರ ಹಾಗಿರಲಿ. ನಮ್ಮ ಗುರುಗಳ ಮಗನನ್ನು ನೀನು ನಿನ್ನ ತೆರೆ ಗಳಿಂದ ಸೆಳೆದು ನುಂಗಿಹಾಕಿದ್ದಿ. ಅವನನ್ನು ಈಗ ನಮ್ಮ ವಶಕ್ಕೆ ಕೊಟ್ಟರೆ ಸಾಕು’ ಎಂದ. ಸಮುದ್ರರಾಜನು ‘ಭಗವಂತ, ನಾನಲ್ಲ ಆ ಬ್ರಾಹ್ಮಣನ ಮಗನನ್ನು ನುಂಗಿದವನು. ಪಂಚಜನನೆಂಬ ರಕ್ಕಸನು ಶಂಖದ ರೂಪಿನಿಂದ ನನ್ನ ನೀರಿನಲ್ಲಿ ಸಂಚರಿಸುತ್ತಿದ್ದಾನೆ. ಅವನೇ ನಿಮ್ಮ ಗುರುಪುತ್ರನನ್ನು ನುಂಗಿದವನು. ನೀನೀಗ ನೀರನ್ನು ಹೊಕ್ಕು, ಆ ರಕ್ಕಸ ನನ್ನು ಕೊಂದು ಮಗುವನ್ನು ಕರೆದುಕೊಂಡು ಹೋಗು’ ಎಂದನು. ಒಡನೆಯೆ ಶ್ರೀಕೃಷ್ಣನು ಸಮುದ್ರವನ್ನು ಹೊಕ್ಕು, ಪಂಚಜನನನ್ನು ಕಂಡುಹಿಡಿದು ಅವನನ್ನು ಸೀಳಿಹಾಕಿದನು. ಅವನ ಹೊಟ್ಟೆಯಲ್ಲಿ ಗುರುಪುತ್ರನಿರಲಿಲ್ಲ. ಸೊಗಸಾದ ಒಂದು ಶಂಖಮಾತ್ರ ಅಲ್ಲಿ ಸಿಕ್ಕಿತು. ಶ್ರೀಕೃಷ್ಣನು ಆ ಶಂಖವನ್ನು ತೆಗೆದುಕೊಂಡು, ಅಣ್ಣನೊಡನೆ ಯಮನ ರಾಜಧಾನಿ ಯಾದ ‘ಸಂಯಮನಿ’ಗೆ ಬಂದನು. ಆ ಊರಿನ ಹೆಬ್ಬಾಗಿಲಲ್ಲಿ ನಿಂತು ಶ್ರೀಕೃಷ್ಣನು ತನ್ನ ಕೈಲಿದ್ದ ಶಂಖವನ್ನು ಗಟ್ಟಿಯಾಗಿ ಊದಿದನು. ಭಯಂಕರವಾದ ಆ ಧ್ವನಿಯನ್ನು ಕೇಳು ತ್ತಲೆ ಯಮನು ಕಪ್ಪಕಾಣಿಕೆಗಳೊಡನೆ ಶ್ರೀಕೃಷ್ಣನ ಇದಿರಿಗೆ ಬಂದು ನಿಂತನು. ಆಗ ಶ್ರೀಕೃಷ್ಣನು ‘ಯಮಧರ್ಮರಾಯ, ನಿನ್ನ ಆಳುಗಳು ನಮ್ಮ ಗುರುವಿನ ಮಗನನ್ನು ಇಲ್ಲಿಗೆ ಕರೆತಂದಿದ್ದಾರೆ. ದಯವಿಟ್ಟು ಈಗ ಅವನನ್ನು ನನಗೊಪ್ಪಿಸು’ ಎಂದನು. ಯಮನು ಆ ಬಾಲಕನನ್ನು ಕರೆತಂದು ಬಲರಾಮಕೃಷ್ಣರಿಗೆ ಒಪ್ಪಿಸಿದನು. ಒಡನೆಯೆ ಅವರು ಆ ಬಾಲಕನನ್ನು ತಮ್ಮ ಗುರುಗಳಿಗೆ ಒಪ್ಪಿಸಿ ‘ಸ್ವಾಮಿ, ನಮ್ಮಿಂದ ಇನ್ನೇನಾಗಬೇಕು?’ ಎಂದು ಕೈಮುಗಿದು ಕೇಳಿಕೊಂಡರು. ಸತ್ತ ಮಗನನ್ನು ಮತ್ತೆ ಪಡೆದ ಸಾಂದೀಪನರು ಅತ್ಯಂತ ಸಂತೋಷದಿಂದ ‘ಅಪ್ಪ, ನಿಮ್ಮಂತಹ ಶಿಷ್ಯರನ್ನು ಪಡೆದ ಗುರುವು ಪರಮ ಧನ್ಯ. ಅವನಿಗೆ ಯಾವುದರ ಕೊರತೆಯಿದ್ದೀತು? ನನಗಿನ್ನೇನು ಬೇಡ. ನೀವು ಚಿರಕಾಲ ಸುಖವಾಗಿ ಬಾಳಿ ಲೋಕೋತ್ತರವಾದ ಕೀರ್ತಿಶಾಲಿಗಳಾಗಿರಿ’ ಎಂದು ಹರಸಿದರು. ಬಲರಾಮಕೃಷ್ಣರು ಗುರು ವಿಗೆ ಅಡ್ಡಬಿದ್ದು, ಅವರಿಂದ ಬೀಳ್ಕೊಂಡವರೆ ಮಧುರೆಗೆ ಹಿಂದಿರುಗಿದರು.

