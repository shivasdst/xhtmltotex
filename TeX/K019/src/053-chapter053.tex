
\chapter{೫೩. ಶ್ರೀಕೃಷ್ಣನ ಕೊಳಲಗಾನ}

ಅದೊಂದು ಶರದೃತುವಿನ ಸುಂದರವಾದ ಮುಂಜಾನೆ. ಬಲರಾಮಕೃಷ್ಣರು ಗೊಲ್ಲರ ಹುಡುಗರೊಡನೆ ತುರುಮಂದೆಯನ್ನು ಹೊಡೆದುಕೊಂಡು ಕಾನನದತ್ತ ಹೊರಟರು. ಶುಭ್ರವಾದ ಆಕಾಶ, ಕಮಲಪುಷ್ಪಗಳ ಸುವಾಸನೆಯನ್ನು ಹೊತ್ತು ಮಂದಮಂದವಾಗಿ ಬೀಸುವ ತಂಗಾಳಿ, ನೂರಾರು ಬಗೆಯ ಹಕ್ಕಿಗಳು ತಮ್ಮ ಕಲರವದಿಂದ ಪ್ರಕೃತಿದೇವಿಗೆ ಉದಯರಾಗವನ್ನು ಹಾಡುತ್ತಿವೆ, ಅರಳಿದ ಹೂವನ್ನು ಮುಡಿದ ಗಿಡಬಳ್ಳಿಗಳು ತಲೆದೂಗು ತ್ತಿವೆ. ವನಲಕ್ಷ್ಮಿಯ ಈ ಸೊಬಗನ್ನು ಕಂಡು ಶ್ರೀಕೃಷ್ಣನ ಮನಸ್ಸು ಅನಿರ್ವಚನೀಯವಾದ ಆನಂದದಿಂದ ತುಂಬಿತು. ಆತನು ಸೊಂಟದಲ್ಲಿ ಸಿಕ್ಕಿಸಿಕೊಂಡಿದ್ದ ಕೊಳಲನ್ನು ಎತ್ತಿ ತುಟಿಗೊಯ್ದನು. ಒಡನೆಯೆ ಆತನ ಆನಂದ ಕೊಳಲಗಾನದ ತುಂಬುಹೊಳೆಯಾಗಿ ಹರಿ ಯಿತು. ಆ ಹೊಳೆ ಹತ್ತಿರದಲ್ಲಿಯೆ ಇದ್ದ ಗೋಕುಲಕ್ಕೆ ನುಗ್ಗಿ ಮನೆಮನೆಯನ್ನೂ ಪ್ರವೇಶಿ ಸಿತು. ಮುಸುಕಿದ ನಿದ್ದೆಯ ಮಬ್ಬಿನಲ್ಲಿದ್ದ ಗೋಪಬಾಲೆಯರು ಗಾನದ ತಂಪು ಹೃದಯ ವನ್ನು ತಾಕುತ್ತಲೆ, ಕೊಡವಿಕೊಂಡು ಮೇಲೆದ್ದರು. ಪುಂಗಿಯನ್ನು ಕೇಳಿ ತಲೆದೂಗುವ ಎಳೆನಾಗರದಂತೆ ಅವರ ತಲೆಗಳು ತೂಗಿದವು. ಅವರ ಕಣ್ಣೆದುರಿಗೆ ಕೊಳಲನ್ನು ಹಿಡಿದ ಶ್ರೀಕೃಷ್ಣನ ಮೋಹಕಮೂರ್ತಿ ನಲಿದು ನರ್ತಿಸಿದಂತಾಯಿತು. ಉಟ್ಟ ಪಟ್ಟೆಯ ಮಡಿ, ಕೊರಳ ಹಾರ, ಕಿವಿಯ ಕರ್ಣಿಕಾಪುಷ್ಪ, ತಲೆಯಮೇಲಿನ ನವಿಲುಗರಿ–ಎಲ್ಲವೂ ಅವನ ನೆನಪನ್ನು ಮೆಲುಕುಹಾಕುತ್ತಾ ಅವರ ಮುಂದೆ ಬಂದು ನಿಂತವು. ಅವರ ಕಾಲುಗಳು ಯಾವ ಪ್ರಯತ್ನವೂ ಇಲ್ಲದೆ, ತಾವಾಗಿಯೆ ಅವರನ್ನು ಮನೆಯಿಂದ ಹೊರಕ್ಕೆ ಎಳೆತಂದವು. ಅವರು ಗಾನ ಬಂದ ದಿಕ್ಕಿನತ್ತ ತಿರುಗಿನಿಂತು, ಕಣ್ಣಿನಲ್ಲಿದ್ದ ಕೃಷ್ಣನನ್ನು ಹೃದಯದಲ್ಲಿ ಹೊಗಿಸಿ, ಅವನ ಆಲಿಂಗನದ ಪರಮಸುಖದಿಂದ ಪುಳುಕಿತರಾದರು. ಅವರ ಕಿವಿಯ ಪುಣ್ಯದಿಂದ ಕಣ್ಣು ಧನ್ಯವಾಯಿತು; ಹೃದಯ ಸಂತಸಗೊಂಡಿತು.

ಮನೆಯಿಂದ ಹೊರಗೆ ಬಂದ ಗೋಪಬಾಲೆಯರು ಪರಸ್ಪರ ಲಲ್ಲೆವಾತುಗಳಿಂದ ತಮ್ಮ ಹೃದಯವನ್ನು ಹೊರದೋರಿದರು. ಅವರಲ್ಲಿ ಒಬ್ಬಳು ‘ಅಕ್ಕ, ಅವನ ಕೈಯ ಕೊಳಲು ಏನು ಪುಣ್ಯಮಾಡಿತ್ತೊ! ಅವನ ತುಟಿಯ ಅಮೃತವೆಲ್ಲ ಅದರ ಪಾಲೆ’ ಎಂದಳು. ಮತ್ತೊಬ್ಬಳು ‘ಎಲೆ ಗೆಳತಿ, ಆ ಕೃಷ್ಣನ ಗಾನವನ್ನು ಕೇಳಿ ನವಿಲುಗಳೆಲ್ಲ ಕುಣಿಯುತ್ತವಂತೆ! ಆಗ ಕಾಡಿನ ಇತರ ಮೃಗಗಳೆಲ್ಲ ಹತ್ತಿರದ ಬೆಟ್ಟವನ್ನೇರಿ, ಅದನ್ನೆ ನೋಡುತ್ತಾ ನಿಲ್ಲುತ್ತವಂತೆ! ಆಗ ದೇವತೆಗಳು ಕೂಡ ಆಕಾಶದಲ್ಲಿ ನಿಂತು ಆ ಸಂಗೀತ ನರ್ತನಗಳನ್ನು ನೋಡುತ್ತಾರಂತೆ!’ ಎಂದಳು. ಇನ್ನೊಬ್ಬಳು ‘ಅಯ್ಯೊ, ಅವನ ಸಂಗೀತವನ್ನು ಕೇಳಿ ಹೆಣ್ಣು ಜಿಂಕೆಗಳೆಲ್ಲ ಪ್ರೇಮದಿಂದ ಅವನನ್ನು ನೋಡುತ್ತಾ ನಿಲ್ಲುತ್ತವಂತೆ! ಗಂಡು ಜಿಂಕೆಗಳೂ ಆಗ ಪಕ್ಕದಲ್ಲಿಯೇ ಸುಮ್ಮನೆ ನಿಂತಿರುತ್ತವಂತೆ! ಹೆಂಗಸೊಬ್ಬಳು ಹಾಗೆ ನಿಂತಿದ್ದರೆ ಅವಳ ಗಂಡ ಸುಮ್ಮನೆ ಇರುತ್ತಿದ್ದನೆ?’ ಎಂದಳು. ಮಗದೊಬ್ಬಳು ‘ಅಮ್ಮ, ಏನು ಹೇಳಲೆ ತಾಯಿ, ಆ ಶ್ರೀಕೃಷ್ಣ ಕೊಳಲು ಬಾರಿಸುತ್ತಾ ನಿಂತನೆಂದರೆ ಆಕಳುಗಳೆಲ್ಲ ಹುಲ್ಲು ಮೇಯು ವುದನ್ನು ಮರೆತು, ತೆರೆದ ಕಿವಿಗಳಿಂದ ಸಂಗೀತವನ್ನು ಆಲಿಸುತ್ತಾ ನಿಲ್ಲುತ್ತವೆ! ಅವುಗಳ ಕರುಗಳೂ ಅಷ್ಟೆ. ಮೊಲೆಕುಡಿಯುವುದನ್ನೂ ಬಿಟ್ಟು, ಬಾಯಿಂದ ಹಾಲು ಚೆಲ್ಲುತ್ತಿದ್ದರೂ ತಿಳಿಯದೆ, ಕಿವಿ ಮೂತಿಗಳನ್ನು ಕೃಷ್ಣನತ್ತ ತಿರುಗಿಸಿಕೊಂಡು ನಿಲ್ಲುತ್ತವೆ! ಅಲ್ಲಿನ ಹಕ್ಕಿ ಗಳೂ ಅಷ್ಟೆ, ಕೊಳಲದನಿ ಕಿವಿಗೆ ಬೀಳುತ್ತಲೆ ಮೈಮರೆತು ಮರಗಳಲ್ಲಿ ಕುಳಿತುಕೊಳ್ಳು ತ್ತವೆ. ಅವೆಲ್ಲ ಏನು ಪುಣ್ಯ ಮಾಡಿದ್ದವೊ! ಪೂರ್ವಜನ್ಮದಲ್ಲಿ ಅವೆಲ್ಲ ಪುಷಿಗಳಾಗಿದ್ದವೊ ಏನೊ! ಅವುಗಳಂತೆ ಹಗಲೆಲ್ಲ ಅದನ್ನು ಸವಿಯುವ ಪುಣ್ಯ ನಮಗಿಲ್ಲವಲ್ಲ!’ ಎಂದು ಹೇಳಿ, ನಿಟ್ಟುಸಿರುಬಿಟ್ಟಳು. ಆಗ ಇನ್ನೊಬ್ಬಳು ‘ಅಯ್ಯೊ ಅದೆಲ್ಲ ಏನು ಮಹಾ! ನಾನು ಕಣ್ಣಾರ ನೋಡಿದ್ದೀನಿ. ಶ್ರೀಕೃಷ್ಣ ಕೊಳಲು ಬಾರಿಸಿದನೆಂದರೆ ಯಮುನೆ ತನ್ನ ವೇಗವನ್ನು ತಗ್ಗಿಸಿ, ನಿಧಾನವಾಗಿ ಹರಿಯುತ್ತಾಳೆ. ಮರಗಿಡಗಳೆಲ್ಲ ಆನಂದದಿಂದ ರೋಮಾಂಚನ ಗೊಳ್ಳುತ್ತವೆ. ಆ ಕಾಡು, ಆ ಮೃಗ, ಆ ತೊರೆ, ಆ ತುರುಗಳು, ಆ ಗೊಲ್ಲರು ಏನು ಪುಣ್ಯ ಮಾಡಿದ್ದಾರೊ! ಸದಾ ಅವರಿಗೆ ಶ್ರೀಕೃಷ್ಣನ ಸಹವಾಸ’ ಎಂದು ಹೇಳಿ ಬಳಬಳ ಕಣ್ಣೀರನ್ನೆ ಸುರಿಸಿದಳು.

ಶ್ರೀಕೃಷ್ಣನ ಕೊಳಲದನಿಗೆ ಮನಸೋತು ಮರುಳಾದ ಆ ಗೊಲ್ಲ ಹುಡುಗಿಯರು ಇನ್ನೂ ಎಳೆಯ ಬಾಲಕನಾಗಿರುವ ಆ ಕೃಷ್ಣನನ್ನು ಮನಸಾ ವರಿಸಿ, ಆತನೇ ತಮಗೆ ಗಂಡನಾಗ ಬೇಕೆಂದು ಬಯಸಿದರು. ಅದಕ್ಕಾಗಿ ಅವರೆಲ್ಲರೂ ಸೇರಿಕೊಂಡು, ಒಂದು ತಿಂಗಳ ಕಾಲ ಕಾತ್ಯಾಯಿನೀ ವ್ರತವನ್ನು ಆಚರಿಸಿದರು. ಬೆಳಗ್ಗೆ ಎದ್ದವರೆ ಯಮುನಾ ನದಿಗೆ ಹೋಗಿ ಸ್ನಾನಮಾಡಿ ಬಂದು, ನದಿಯ ದಡದಲ್ಲಿದ್ದ ಮರಳಿನಲ್ಲಿ ಗೌರಿದೇವಿಯ ವಿಗ್ರಹವನ್ನು ಮಾಡಿ ಅದಕ್ಕೆ ಭಕ್ತಿಯಿಂದ ಪೂಜೆಮಾಡಿದಮೇಲೆ ‘ತಾಯಿ, ಶ್ರೀಕೃಷ್ಣನು ನಮಗೆ ಗಂಡ ನಾಗುವಂತೆ ಅನುಗ್ರಹಿಸು’ ಎಂದು ಅಡ್ಡಬಿದ್ದು ಬೇಡಿಕೊಳ್ಳುತ್ತಿದ್ದರು. ಆ ವ್ರತದ ಕಡೆಯ ದಿವಸ ಅವರೆಲ್ಲ ಎಂದಿನಂತೆ ನದಿಗೆ ಹೋಗಿ, ತಮ್ಮ ಸೀರೆಗಳನ್ನೆಲ್ಲ ನದಿಯ ದಡದ ಲ್ಲಿಟ್ಟು, ಸ್ನಾನಕ್ಕಿಳಿದರು. ಆ ವೇಳೆಗೆ ಸರಿಯಾಗಿ ಶ್ರೀಕೃಷ್ಣನು ತನ್ನ ಗೆಳೆಯರೊಡನೆ ಅಲ್ಲಿಗೆ ಹೋದನು. ಆತನು ಆ ಹುಡುಗಿಯರೊಬ್ಬರಿಗೂ ಗೊತ್ತಾಗದಂತೆ ಅವರ ಸೀರೆಗಳನ್ನೆಲ್ಲ ಎತ್ತಿ ಕೊಂಡು, ಸಮೀಪದಲ್ಲಿದ್ದ ದೊಡ್ಡ ಮರವೊಂದನ್ನೇರಿ ಕುಳಿತನು. ಒಡನೆಯೆ ಅವನ ಗೆಳೆಯರೆಲ್ಲ ಚಪ್ಪಾಳೆ ತಟ್ಟಿಕೊಂಡು ಗಟ್ಟಿಯಾಗಿ ನಕ್ಕರು. ಶ್ರೀಕೃಷ್ಣನು ಅವರೊಡನೆ ಕೇಕೆ ಹಾಕಿಕೊಂಡು ನಕ್ಕನು. ಇದು ಹುಡುಗಿಯರಿಗೆ ಕೇಳಿಸುತ್ತಲೆ ಅವರು ನಿಟ್ಟುಬಿದ್ದು ಸುತ್ತಲೂ ನೋಡಿದರು. ಅವರ ಸೀರೆಗಳೆಲ್ಲ ಮರದಮೇಲೆ ನೇತಾಡುತ್ತಿವೆ! ಅವುಗಳ ಮಧ್ಯದಲ್ಲಿ ಶ್ರೀಕೃಷ್ಣ ನಗುತ್ತಾ ಕುಳಿತಿದ್ದಾನೆ. ನದಿಯ ದಡದಲ್ಲಿ ನಾಲ್ಕಾರು ಜನ ಅವನ ಗೆಳೆಯರು ನಿಂತು ನಗುತ್ತಿದ್ದಾರೆ. ಅದನ್ನು ಕಂಡು ಆ ಹುಡುಗಿಯರಿಗೆಲ್ಲ ತುಂಬ ನಾಚಿಕೆ ಯಾಯಿತು. ಅವರು ಬಗ್ಗಿಸಿದ ತಲೆಯನ್ನು ಮೇಲಕ್ಕೆತ್ತದೆ ನೀರಿನಲ್ಲಿ ಕುಳಿತರು.

ಹೆಣ್ಣುಮಕ್ಕಳ ನಾಚಿಕೆಯನ್ನು ಕಂಡು, ಶ್ರೀಕೃಷ್ಣ ಅವರೊಡನೆ ಸರಸವನ್ನಾಡುತ್ತಾ ‘ಎಲೆ, ಹುಡುಗಿಯರೆ, ಇಲ್ಲಿ ನೋಡಿ, ನಿಮ್ಮ ಸೀರೆಗಳೆಲ್ಲ ನಮ್ಮ ಬಳಿ ಇವೆ. ನೀವು ಇಲ್ಲಿಗೆ ಬಂದು ನಿಮ್ಮ ಸೀರೆ ಯಾವುದೆಂದು ತೋರಿಸಿದರೆ ಅದನ್ನು ನಿಮಗೆ ಕೊಡುತ್ತೇನೆ. ನನ್ನ ಮಾತನ್ನು ಹುಡುಗಾಟವೆಂದು ತಿಳಿಯಬೇಡಿ, ನನ್ನ ಮಾತನ್ನು ನಂಬಿ. ನೀವು ಒಬ್ಬೊಬ್ಬ ರಾಗಿಯಾದರೂ ಬನ್ನಿ, ಒಟ್ಟಾಗಿಯಾದರೂ ಬನ್ನಿ, ನಿಮ್ಮ ನಿಮ್ಮ ಸೀರೆಗಳನ್ನು ನೀವು ತೆಗೆದುಕೊಳ್ಳಿ’ ಎಂದನು. ಆದರೆ ಆ ಹೆಣ್ಣುಮಕ್ಕಳು ಬೆತ್ತಲೆಯಾಗಿ ಅವನ ಬಳಿಗೆ ಹೇಗೆ ಹೋದಾರು, ಪಾಪ? ಒಂದು ಕಡೆ ನಾಚಿಕೆ, ಭಯ; ಮತ್ತೊಂದು ಕಡೆ ಅವನ ಮೇಲೆ ಪಂಚ ಪ್ರಾಣ. ಅವರು ಒಬ್ಬರ ಮುಖವನ್ನು ಒಬ್ಬರು ನೋಡುತ್ತಾ ನಿಂತಿದ್ದರು. ಆದರೆ ಬೆಳಗಿನ ಚಳಿಯಲ್ಲಿ ಅವರು ನೀರಿನಲ್ಲಿ ಎಷ್ಟುಹೊತ್ತು ನಿಲ್ಲಲು ಸಾಧ್ಯ? ಅವರಲ್ಲೊಬ್ಬಳು ಮೆಲ್ಲನೆ ತನ್ನ ತಲೆಯನ್ನೆತ್ತಿ ಶ್ರೀಕೃಷ್ಣನೊಡನೆ ‘ಅಯ್ಯೋ, ಇದೆಂತಹ ಅನ್ಯಾಯದ ಕೆಲಸ. ನೀನು ನಂದರಾಜನ ಮುದ್ದುಮಗ. ಆದ್ದರಿಂದಲೆ ನೀನು ಸ್ವಲ್ಪವೂ ಭಯವಿಲ್ಲದೆ ಹೀಗೆ ಮಾಡು ತ್ತಿದ್ದಿ. ನಾವು ಚಳಿಯಲ್ಲಿ ನಡುಗುತ್ತ ಎಷ್ಟುಹೊತ್ತು ಹೀಗೆ ನಿಂತಿರಬೇಕು? ಇದು ತರವಲ್ಲ, ನಮ್ಮ ಸೀರೆಗಳನ್ನು ಕೊಟ್ಟುಬಿಡು’ ಎಂದಳು. ಒಬ್ಬಳು ಮಾತಾಡುತ್ತಲೆ ಉಳಿದವರಿಗೂ ಧೈರ್ಯಬಂತು. ಅವರಲ್ಲೊಬ್ಬಳು ನಗೆಚೆಲ್ಲಹೊರಟ ತುಟಿಯನ್ನು ಕಚ್ಚಿ ‘ಕೃಷ್ಣ, ನಮ್ಮ ಬಟ್ಟೆಗಳನ್ನು ಕೊಡುತ್ತೀಯೋ, ನಂದರಾಜನಿಗೆ ಹೇಳಬೇಕೊ?’ ಎಂದಳು. ಅವಳ ಮಾತಿಗೆ ಪಕಪಕ ನಗುತ್ತಾ ಶ್ರೀಕೃಷ್ಣನು ‘ಎಲೆ ಹುಡುಗಿ ನನ್ನನ್ನು ಹೆದರಿಸುತ್ತೀಯಾ? ನೀರಿನಿಂದ ಹೊರಬರಲು ನಾಚಿ ಸಾಯುತ್ತಿರುವ ನೀವು ಬರಿಮೈಲಿ ನಂದರಾಜನ ಬಳಿಗೆ ಹೋಗು ತ್ತೀರಾ?’ ಎಂದ. ಆಗ ಮತ್ತೊಬ್ಬಳು ಆತನಿಗೆ ಶರಣಾದವಳಂತೆ ‘ಶ್ರೀಕೃಷ್ಣ, ನೀನು ರಾಜ ಕುಮಾರ, ನಾವು ನಿನ್ನ ದಾಸಿಯರು; ನೀನು ಹೇಳಿದಂತೆ ಕೇಳುತ್ತೇವೆ, ನಾವು; ಮೊದಲು ನಮ್ಮ ಸೀರೆಗಳನ್ನು ಕೊಟ್ಟುಬಿಡು. ನೀನೆ ಯೋಚಿಸು, ಹೆಣ್ಣು ಬರಿಮೈಯಲ್ಲಿರುವಾಗ ಅವಳನ್ನು ನೋಡುವುದು ಪಾಪವಲ್ಲವೆ?’ ಎಂದಳು.

ಶ್ರೀಕೃಷ್ಣನೇನು ಸಾಮಾನ್ಯವಾದ ಮೀನೆ, ಅವರೊಡ್ಡಿದ ಬಲೆಗೆ ಬೀಳುವುದಕ್ಕೆ? ಅವನು ಸೇರಿಗೆ ಸವಾಸೇರಾಗಿ ಉತ್ತರವಿತ್ತ: ‘ಎಲೆ ಹುಡುಗಿಯರೆ, ನನಗೆ ಪಾಪಪುಣ್ಯವನ್ನು ಹೇಳಿ ಕೊಡುತ್ತೀರಾ? ವ್ರತವನ್ನು ಹಿಡಿದು ಗೌರಿಯನ್ನು ಪೂಜಿಸುತ್ತಿರುವ ನೀವು ಬರಿಮೈಲಿ ನೀರಿ ಗಿಳಿದುದು ಎಂಥ ದೊಡ್ಡ ಪಾಪ! ಆ ಪಾಪ ಹೋಗಬೇಕಾದರೆ ನೀವೆಲ್ಲ ನೀರಿನಿಂದ ಹೊರಗೆ ಬಂದು, ತಲೆಯಮೇಲೆ ಕೈಜೋಡಿಸಿ, ನನಗೆ ನಮಸ್ಕಾರ ಮಾಡಬೇಕು. ಹಾಗೆ ಮಾಡಿದೊಡನೆ ನಿಮ್ಮ ಸೀರೆಗಳು ನಿಮ್ಮ ಕೈಗೆ ಬರುತ್ತವೆ’ ಎಂದನು. ಆ ಹೆಣ್ಣುಮಕ್ಕಳಿಗೆ ಅವನ ಮಾತು ನ್ಯಾಯವೆನಿಸಿತು. ಅವರು ಕೈಗಳಿಂದ ಮೈಮುಚ್ಚಿಕೊಂಡು ಹೊರಗೆ ಬಂದವರೆ, ಎದೆಯ ಮೇಲಿದ್ದ ತಮ್ಮ ಕೈಯೆತ್ತಿ ಅವನಿಗೆ ನಮಸ್ಕರಿಸಿದರು. ಆದರೆ ಅವರು ಒಂದೇ ಒಂದು ಕೈಯಿಂದ ನಮಸ್ಕಾರ ಮಾಡಿದುದು ಶ್ರೀಕೃಷ್ಣನಿಗೆ ಒಪ್ಪಿಗೆಯಾಗಲಿಲ್ಲ. ‘ಛಿ, ಛೀ, ಒಂದು ಕೈಯಿಂದ ನಮಸ್ಕಾರ ಮಾಡಿದರೆ ಮತ್ತೂ ಪಾಪ. ಆ ಕೈಯನ್ನೆ ಕತ್ತರಿಸಿ ಹಾಕಬೇಕೆಂದು ಧರ್ಮಶಾಸ್ತ್ರ ಹೇಳುತ್ತದೆ. ನೀವು ಎರಡು ಕೈಯೆತ್ತಿ ನಮಸ್ಕರಿಸಿದ ಹೊರತು ನಾನು ಮೆಚ್ಚುವುದಿಲ್ಲ’ ಎಂದು ಆತ ಅವರನ್ನು ಛೀಗುಟ್ಟಿದ. ಕಡೆಗೆ ಆ ಹೆಣ್ಣು ಮಕ್ಕಳು ಗತ್ಯಂತರವಿಲ್ಲದೆ ಎರಡು ಕೈಗಳಿಂದಲೂ ಆತನಿಗೆ ನಮಸ್ಕರಿಸಿದರು. ಆಗ ಅವರ ಸೀರೆಗಳು ಅವರಿಗೆ ಸಿಕ್ಕವು.

ಗೊಲ್ಲ ಹುಡುಗಿಯರು ಸೀರೆಗಳನ್ನು ಧರಿಸಿ ಮನೆಗೆ ಹೊರಡುವುದಕ್ಕೆ ಬದಲು ಕೃಷ್ಣ ನತ್ತ ನೋಡುತ್ತಾ ಅಲ್ಲಿಯೆ ನಿಂತರು. ಅವರ ಆಶೆಯನ್ನು ಅರಿತ ಶ್ರೀಕೃಷ್ಣ ಅವರನ್ನು ಕುರಿತು ‘ಎಲೆ ಹೆಣ್ಣುಗಳೆ, ನೀವೆಲ್ಲ ನನ್ನನ್ನು ಮದುವೆಯಾಗಬೇಕೆಂದು ಬಯಸಿ, ಗೌರಿಯ ವ್ರತವನ್ನು ಮಾಡಿ ಮುಗಿಸಿರುವಿರಿ. ಅದು ನನಗೆ ಗೊತ್ತು. ನಾನು ನಿಮ್ಮ ಇಷ್ಟವನ್ನು ಸಲ್ಲಿಸಲು ಸಿದ್ಧ. ಆದರೆ ಒಂದು ವಿಚಾರ. ನನ್ನ ಮೇಲಿನ ಪ್ರೇಮವೆಂದರೆ ಅದು ಬರಿಯ ಇಂದ್ರಿಯಸುಖವಲ್ಲ; ಅದು ಪರಮಸುಖ. ಕಾಮವೆಂಬುದು ಸಾಮಾನ್ಯವಾಗಿ ಸಂಸಾರಕ್ಕೆ ಬೀಜರೂಪವಾದುದು. ಆದರೆ ನನ್ನ ಮೇಲಿನ ಕಾಮ ಇತರ ಸಕಲ ಕಾಮಗಳನ್ನು ನಾಶ ಮಾಡುತ್ತದೆಯೆ ಹೊರತು ಕಾಮಭೋಗವನ್ನು ಕೊಡುವುದಿಲ್ಲ. ಅದು ಕೇವಲ ಮುಕ್ತಿ ಯನ್ನೆ ನೀಡತಕ್ಕುದು. ಆದ್ದರಿಂದ ನಿಮ್ಮ ಅಪೇಕ್ಷೆ ಯೋಗ್ಯವಾದುದೆ. ನೀವು ಈ ಶರತ್ಕಾಲದ ರಾತ್ರಿಗಳಲ್ಲೆಲ್ಲ ನನ್ನ ಸುಖವನ್ನು ಪಡೆಯುವಿರಿ’ ಎಂದನು. ಸಾಕ್ಷಾತ್ ಭಗವಂತನಾದ ಆತನ ಮಾತು ಆ ಗೊಲ್ಲ ಹುಡುಗಿಯರಿಗೆ ಎಷ್ಟು ಅರ್ಥವಾಯಿತೊ! ಅವರು ಆ ಲೋಕಮೋಹಕ ಮೂರ್ತಿಯನ್ನು ಕಣ್ಣಲ್ಲಿ ತುಂಬಿಕೊಂಡು ಗೋಕುಲಕ್ಕೆ ಹಿಂದಿರುಗಿದರು.

