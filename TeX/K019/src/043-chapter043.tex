
\chapter{೪೩. ಹಗ್ಗವೆಂದುದು ಹಾವಾಯಿತು}

ವಸುದೇವನು ಮಧ್ಯರಾತ್ರಿಯಲ್ಲಿ ತಂದಿಟ್ಟುಹೋದ ಮಗು ರಾತ್ರಿಯೆಲ್ಲ ಯಶೋದೆಯ ಮಡಿಲಲ್ಲಿ ಹಾಯಾಗಿ ಮಲಗಿ ನಿದ್ದೆಹೋಯಿತು. ಮೈಮರೆತು ಮಲಗಿದ್ದ ಯಶೋದೆ ಬೆಳಗ್ಗೆ ಎದ್ದು ನೋಡುತ್ತಾಳೆ, ಲೋಕಮೋಹಕವಾದ ಗಂಡುಮಗು! ಹಿಡಿಸ ಲಾರದ ಸಂತೋಷದಿಂದ ಆಕೆ ಗಂಡನಿಗೆ ಹೇಳಿಕಳುಹಿಸಿದಳು. ನಂದನು ಓಡಿಬಂದು, ಕಂದನಾದ ಮುಕುಂದನನ್ನು ಕಾಣುತ್ತಲೆ ಆನಂದದಿಂದ ಕುಣಿದಾಡಿದನು. ಕ್ಷಣಮಾತ್ರ ದಲ್ಲಿ ಆ ಸುದ್ದಿ ಗೋಕುಲಕ್ಕೆಲ್ಲ ಹರಡಿತು. ತಮ್ಮ ಒಡೆಯನಿಗೆ ಗಂಡುಮಗು ಹುಟ್ಟಿ ದುದನ್ನು ಕೇಳಿ ಗೋಪಾಲಕರಿಗೆಲ್ಲ ಸಡಗರವೋ ಸಡಗರ. ಮನೆಮನೆಯ ಮುಂದು ಗಡೆಯೂ ಸಾರಣೆ ಕಾರಣೆ ರಂಗವಲ್ಲಿಗಳಿಂದ ಅಲಂಕಾರ; ಎಲ್ಲೆಲ್ಲಿಯೂ ತಳಿರು, ತೋರಣ, ಬಾವುಟ, ಹೂಮಾಲೆಗಳ ಸಿಂಗಾರ! ಗೋಪಾಲಕರು ತಮ್ಮ ದನಕರುಗಳಿಗೆಲ್ಲ ಸ್ನಾನ ಮಾಡಿಸಿ, ಅರಿಸಿನ ಕುಂಕುಮಗಳಿಂದಲೂ ಬಗೆಬಗೆಯ ಬಣ್ಣಗಳಿಂದಲೂ ಅವು ಗಳನ್ನು ಅಂದಗೊಳಿಸಿದರು; ಹೂಮಾಲೆಗಳಿಂದಲೂ ನವಿಲುಗರಿಗಳಿಂದಲೂ ಅವನ್ನು ಸಿಂಗರಿಸಿದರು; ಹೊಸ ಬಟ್ಟೆಗಳನ್ನು ಅವುಗಳ ಮೇಲೆ ಹೊದಿಸಿ ಕಾಲ್ಗೆಜ್ಜೆಗಳನ್ನು ಕಟ್ಟಿ ಕುಣಿಸಿದರು. ಅನಂತರ ಅವರೂ ಹೊಸ ಬಟ್ಟೆಗಳನ್ನು ಧರಿಸಿ, ಬಗೆಬಗೆಯ ಕೈಗಾಣಿಕೆ ಗಳನ್ನು ಕೊಂಡೊಯ್ದು ನಂದನಿಗೆ ಒಪ್ಪಿಸಿದರು. ಗಂಡಸರಂತೆ ಹೆಂಗಸರೂ ಸಿಂಗರಿಸಿ ಕೊಂಡು, ತಮ್ಮತಮ್ಮ ಯೋಗ್ಯತೆಗೆ ತಕ್ಕ ಕೈಗಾಣಿಕೆಗಳ ತಟ್ಟೆಯೊಡನೆ ಯಶೋದೆಯ ಬಳಿಗೆ ಹೊರಟರು. ಅವರ ಪಟ್ಟೆಯ ಸೀರೆ, ಜರತಾರಿಯ ಕುಪ್ಪುಸ, ಮುತ್ತಿನ ಹಾರ, ರತ್ನದ ಓಲೆ, ಕೈಗಡಗ, ಕಾಲಂದುಗೆಗಳು ಅವರ ಸಡಗರ ಸಂಭ್ರಮಗಳನ್ನು ಸಾರಿ ಹೇಳುತ್ತಿದ್ದರೆ, ಅವರ ಮಂದಗಮನ, ಓರೆಗಣ್​ನೋಟ, ಕಿರುನಗೆಗಳು ಅವರ ಹೃದಯದ ಆನಂದವನ್ನು ಸೂರೆಮಾಡುತ್ತಿದ್ದವು. ಯಶೋದೆಯ ಕಂದನನ್ನು ಕಾಣುತ್ತಲೆ ಅವರ ಉಲ್ಲಾಸ ಉತ್ಸಾಹ ಗಳು ಒಂದಕ್ಕೆ ನೂರರಷ್ಟಾದವು. ‘ಮುದ್ದುಕುಮಾರ, ನಮ್ಮೆಲ್ಲರ ಅರಸನಾಗಿ ನೂರುಕಾಲ ಸುಖವಾಗಿ ಬಾಳು’ ಎಂದು ಬಾಯ್ತುಂಬ ಹರಸಿದರು. ಮಗುವಿಗೂ ತಾಯಿಗೂ ಆರತಿ ಯೆತ್ತಿ ಮಂಗಳ ಗೀತೆಗಳನ್ನು ಹಾಡಿದರು. ಅನೇಕ ಬಗೆಯ ಮಂಗಳವಾದ್ಯಗಳು ಭೋರ್ಗರೆದವು. ನಂದನು ಆ ಉತ್ಸವ ಸಮಯದಲ್ಲಿ ಬಂದವರಿಗೆಲ್ಲ ಬೇಕಾದಷ್ಟು ಬಹುಮಾನಗಳನ್ನಿತ್ತು ಮನ್ನಿಸಿದುದಲ್ಲದೆ, ಯಥೇಷ್ಟವಾಗಿ ದಾನಧರ್ಮಗಳನ್ನು ಮಾಡಿ ದನು. ನಾಲ್ಕು ದಿನಗಳವರೆಗೆ ನಂದಗೋಕುಲದಲ್ಲಿ ಎಡೆಬಿಡದೆ ಉತ್ಸವ ನಡೆಯಿತು.

ಇದಾದ ಕೆಲವು ದಿನಗಳಮೇಲೆ ನಂದನು ಪ್ರತಿವರ್ಷವೂ ತಾನು ಸಲ್ಲಿಸಬೇಕಾದ ಕಪ್ಪ ಕಾಣಿಕೆಗಳನ್ನು ಕಂಸನಿಗೆ ಸಲ್ಲಿಸುವುದಕ್ಕಾಗಿ ಮಧುರೆಗೆ ಹೋದನು. ಇದನ್ನು ಕೇಳಿದ ವಸುದೇವನು ಆತನನ್ನು ಕಾಣುವುದಕ್ಕಾಗಿ ಆತನ ಬಿಡಾರಕ್ಕೆ ಹೋದನು. ಹೀಗೆ ಬಂದ ವಸುದೇವನನ್ನು ಕಾಣುತ್ತಲೆ ನಂದನು ಆದರದಿಂದ ಆತನನ್ನು ಇದಿರುಗೊಂಡು, ಆಲಿಂಗಿಸಿ, ಪಕ್ಕದಲ್ಲಿ ಕುಳ್ಳಿರಿಸಿಕೊಂಡು ಉಪಚರಿಸಿದನು. ವಸುದೇವನಿಗೆ ತನ್ನ ಮಕ್ಕಳ ಯೋಗಕ್ಷೇಮವನ್ನು ಕೇಳಬೇಕೆಂಬ ಆಶೆ. ಆದರೆ ನೇರವಾಗಿ ಕೇಳುವಂತಿಲ್ಲ. ಆದ್ದರಿಂದ ಆತನು ವ್ಯಾಜ್ಯಾಂತರವಾಗಿ ‘ಅಣ್ಣ, ನಿನ್ನ ಈ ಮುಪ್ಪಿನಕಾಲದಲ್ಲಿ ನಿನಗೆ ಒಬ್ಬ ಮಗ ಹುಟ್ಟಿದನೆಂದು ಕೇಳಿ, ತುಂಬ ಸಂತೋಷವಾಯಿತು. ಆ ನಿನ್ನ ಮಗ ಸುಖವಾಗಿದ್ದಾನೆಯೆ? ನನ್ನ ಮಗನಾದ ಬಲರಾಮ ಮತ್ತು ಅವನ ತಾಯಿ ನಿಮ್ಮ ಮನೆಯಲ್ಲಿ ಎಲ್ಲರಿಗೂ ಹೊಂದಿಕೊಂಡು ಸುಖವಾಗಿದ್ದಾರೆಯೊ? ಕಂಸನ ದೆಸೆಯಿಂದ ನಾವು ಸತ್ತುಹುಟ್ಟಿದಂತಾ ಯಿತು. ದೇವರ ದಯೆಯಿಂದ ಇಷ್ಟಾದರೂ ಆಯಿತಲ್ಲ’ ಎಂದು ಹೇಳಿ ನಿಟ್ಟುಸಿರನ್ನು ಬಿಟ್ಟನು. ಆತನ ದುರದೃಷ್ಟಕ್ಕಾಗಿ ನಂದನೂ ಮರುಗಿ, ಆತನಿಗೆ ಸಮಾಧಾನದ ಮಾತು ಗಳನ್ನಾಡಿ ತನ್ನೂರಿಗೆ ಹಿಂದಿರುಗಿದನು. ವಸುದೇವನು ಆತನನ್ನು ಬೀಳ್ಕೊಳ್ಳುವ ಮುನ್ನ ‘ಅಣ್ಣ, ನೀನು ಇನ್ನು ಸ್ವಲ್ಪಕಾಲ ಬಹು ಎಚ್ಚರದಿಂದಿರಬೇಕೆಂದು ತೋರುತ್ತದೆ. ಇಷ್ಟರ ಲ್ಲಿಯೇ ಗೋಕುಲದಲ್ಲಿ ಕೆಲವು ಉತ್ಪಾತಗಳು ನಡೆಯುವ ಸಂಭವವಿದೆ’ ಎಂದು ಎಚ್ಚರಿಸಿ ದನು. ಇದನ್ನು ಕೇಳಿ ನಂದನ ಮನಸ್ಸಿಗೆ ಏನೋ ಒಂದು ಬಗೆಯ ಕಸಿವಿಸಿಯಾದಂ ತಾಯಿತು. ಆತನು ದುಗುಡವನ್ನು ಹೊತ್ತುಕೊಂಡೇ ಗೋಕುಲದ ಹಾದಿಯನ್ನು ಹಿಡಿದನು.

 ಇತ್ತ ಗೋಕುಲದಲ್ಲಿ ನಂದನಿಲ್ಲದಿರುವಾಗ ಭಯಂಕರವಾದ ಒಂದು ಘಟನೆ ನಡೆದು ಹೋಗಿತ್ತು. ಕಂಸರಾಜನ ಅಪ್ಪಣೆಯಂತೆ ಹುಡುಕಿ ಹುಡುಕಿ ಎಳೆಯ ಮಕ್ಕಳನ್ನೆಲ್ಲ ಕೊಂದುಹಾಕುವುದಕ್ಕಾಗಿ ಪೂತನಿಯೆಂಬ ರಕ್ಕಸಿ ಊರೂರು ಅಲೆಯುತ್ತಾ, ಗೋಕುಲಕ್ಕೂ ಬಂದಳು. ನೋಡುವುದಕ್ಕೆ ಅವಳು ಒಳ್ಳೆಯ ಮನೆತನದ ಹೆಣ್ಣಿನಂತೆ ಕಾಣಬರುತ್ತಿದ್ದಳು. ಸೊಗಸಾದ ಪೀತಾಂಬರವನ್ನುಟ್ಟು, ಬಗೆಬಗೆಯ ಆಭರಣಗಳನ್ನು ತೊಟ್ಟಿದ್ದಾಳೆ; ಕೂದ ಲನ್ನು ತುರುಬುಗಟ್ಟಿ ಮಲ್ಲಿಗೆ ದಂಡೆಯನ್ನು ಮುಡಿದಿದ್ದಾಳೆ; ಅವಳ ಬಡನಡು, ಬಳಕುವ ದೇಹ, ಮುಗುಳ್ನಗೆ, ಕೈಲಿರುವ ಲೀಲಾಕಮಲವನ್ನು ನೋಡಿದರೆ ಸಾಕ್ಷಾತ್ ಲಕ್ಷ್ಮಿಯೇ ಇರ ಬೇಕೆಂಬ ಭ್ರಾಂತಿ ಬರುವಂತಿದೆ. ಅವಳು ಜನಮನವನ್ನು ಸೂಜಿಗಲ್ಲಿನಂತೆ ಸೆಳೆಯುತ್ತಾ ನಡೆದುಬಂದು ನಂದನ ಮನೆಯನ್ನು ಹೊಕ್ಕಳು. ಅಲ್ಲಿ ಬೂದಿ ಮುಚ್ಚಿದ ಕೆಂಡದಂತೆ ಎಳೆಯ ಮಗುವಾಗಿ ಮಲಗಿದ್ದ ಯಶೋದೆಯ ಕಂದ ಅವಳ ಕಣ್ಣಿಗೆ ಕಾಣಿಸಿದ. ಗಾಢನಿದ್ರೆ ಯಲ್ಲಿದ್ದ ಆ ಮಗುವಿನ ಬಳಿಗೆ ಅವಳು ಮೆಲ್ಲಗೆಹೋಗಿ, ಹಾವನ್ನು ಹಗ್ಗವೆಂದು ಕೈಗೆತ್ತಿ ಕೊಳ್ಳುವ ಎಗ್ಗನಂತೆ, ಆ ಮಗುವನ್ನು ಎತ್ತಿ ತನ್ನ ತೊಡೆಯಮೇಲೆ ಮಲಗಿಸಿಕೊಂಡಳು. ರೋಹಿಣಿ ಯಶೋದೆಯರಿಬ್ಬರೂ ಅಲ್ಲಿಯೇ ಇದ್ದರಾದರೂ ಅವರು ಅವಳನ್ನು ತಡೆಯ ಲಿಲ್ಲ. ಹೆತ್ತ ತಾಯಿಗಿಂತಲೂ ಹೆಚ್ಚು ಪ್ರೇಮ ತೋರುತ್ತಿತ್ತು, ಆ ಪೂತನಿಯ ಮುಖದಲ್ಲಿ. ಮೃದುವಾದ ಒರೆಯಲ್ಲಿ ಮುಚ್ಚಿಟ್ಟ ಕತ್ತಿಯಂತಿದ್ದ ಆ ಪೂತನಿ ಭಯಂಕರವಿಷದಿಂದ ತುಂಬಿದ ತನ್ನ ಮೊಲೆಯನ್ನು ತನ್ನ ತೊಡೆಯಮೇಲಿದ್ದ ಮಗುವಿನ ಬಾಯಲ್ಲಿಟ್ಟಳು. ಆ ರಕ್ಕಸಿ ಭಯಂಕರಳಾದರೆ ಆ ಮಗು ಭಯಕ್ಕೆಯೇ ಭಯಂಕರ. ಅದು ಅವಳ ಮೊಲೆಯನ್ನು ತನ್ನೆರಡು ಕೈಗಳಿಂದಲೂ ಹಿಡಿದುಕೊಂಡು ಕುಡಿಯಲು ಪ್ರಾರಂಭಿಸಿತು. ಮೊಲೆಯಲ್ಲಿದ್ದ ವಿಷ ಮುಗಿಯಿತು, ನೆತ್ತರು ಮುಗಿಯಿತು; ಮಗುವು ಮೊಲೆಗುಡಿಯುವುದು ಮಾತ್ರ ಮುಗಿಯಲಿಲ್ಲ. ಅವಳು ‘ಅಯ್ಯೋ ಅಯ್ಯೋ, ಸಾಕು ಸಾಕು, ಬಿಡು’ ಎಂದು ಅರಚಿ ಕೊಳ್ಳುತ್ತಾ, ಅವನ ಬಾಯಿಂದ ತನ್ನ ಮೊಲೆಯನ್ನು ಕಿತ್ತುಕೊಳ್ಳುವುದಕ್ಕೆ ಪ್ರಯತ್ನಿಸಿದಳು. ಆದರೆ ಅದು ಸಾಧ್ಯವಾಗಲಿಲ್ಲ. ಅವಳು ಹಗ್ಗವೆಂದು ಹಿಡಿದುದು ಹಾವಾಗಿತ್ತು. ಆ ಎಳೆಯ ಮಗು ಅವಳ ಪ್ರಾಣವಾಯುವನ್ನೇ ಹೀರಿತ್ತು. ಅವಳ ಕಣ್ಣುಗುಡ್ಡೆಗಳು ಮೇಲೆ ನಾಟಿದವು, ಮೈ ಬೆವರಿತು, ಅವಳು ಹೋಯೆಂದು ಕೈಕಾಲುಗಳನ್ನು ಒದರುತ್ತಾ ಕೆಳಕ್ಕೆ ಬಿದ್ದು ಸತ್ತು ಹೋದಳು.

ಪೂತನಿಯು ಕೆಳಕ್ಕೆ ಬೀಳುತ್ತಲೆ ಅವಳ ಸೌಮ್ಯರೂಪವು ಅಳಿಸಿಹೋಗಿ, ಭಯಂಕರ ವಾದ ಆಕಾರ ಕಾಣಿಸಿತು. ನೇಗಿಲಿನಂತೆ ಉದ್ದವಾದ ಆ ಕೋರೆದಾಡೆಗಳು, ಬೆಟ್ಟದ ಗವಿ ಯಂತಿರುವ ಮೂಗಿನ ಹೊಳ್ಳೆಗಳು, ದೊಡ್ಡ ಗುಂಡಿನಂತಿರುವ ಮೊಲೆಗಳು, ಹಾಳುಬಾವಿ ಯಂತಿರುವ ಕಣ್ಣುಗಳು, ಆ ಕೆದರಿದ ತಲೆ, ಸೇತುವೆಯಂತಿರುವ ತೊಡೆಗಳು–ನೋಡಿ ದವರು ನಡುಗುವಂತಿದ್ದ ಆ ಹೆಣದಮೇಲೆ ನಗುತ್ತ, ಕೈಕಾಲು ಝಾಡಿಸುತ್ತಿರುವ ಮಗುವನ್ನು ಕಂಡು, ಯಶೋದೆ ರೋಹಿಣಿಯರು ಪ್ರಾಣಭಯವನ್ನು ಬದಿಗೊತ್ತಿ ಆ ಮಗುವನ್ನು ಎತ್ತಿಕೊಂಡರು. ಆಮೇಲೆ ಅಲ್ಲಿದ್ದ ಹೆಂಗಸರೆಲ್ಲ ಸೇರಿಕೊಂಡು, ಆ ಮಗು ವನ್ನು ಹಸುವಿನ ಬಾಲದಿಂದ ಇಳೆತೆಗೆದು, ಗೋಪಾದದ ಧೂಳಿಯನ್ನು ಮೈಗೆ ಹಚ್ಚಿ, ಭಗವಂತನ ಹೆಸರು ಹೇಳುತ್ತಾ ರಕ್ಷೆಯನ್ನು ಮಾಡಿದರು. ಅನಂತರ ಅವರೆಲ್ಲ ಪರಸ್ಪರ ಮಾತನಾಡುತ್ತಾ ‘ಅಮ್ಮ ಹಗ್ಗವೆಂದುದು ಹಾವಾಯಿತಲ್ಲೆ! ಆ ಗಯ್ಯಾಳಿ ಎಂತಹ ನಟನೆ ಯಿಂದ ಹೇಗೆ ಮಾಡಿದಳಲ್ಲಮ್ಮ!’ ಎಂದುಕೊಂಡರು. ತಾಯಿಯಾದ ಯಶೋದೆ ತನ್ನ ಮುದ್ದುಮಗನಿಗೆ ಮೊಲೆಹಾಲು ಕೊಟ್ಟು, ತೊಟ್ಟಿಲಲ್ಲಿಟ್ಟು, ಜೋಗುಳ ಹಾಡುತ್ತಾ ಅವ ನನ್ನು ಮಲಗಿಸಿದಳು.

ಯಶೋದೆಯು ನಿದ್ರೆಹೋಗುವ ಹೊತ್ತಿಗೆ ಸರಿಯಾಗಿ ಮಧುರೆಗೆ ಹೋಗಿದ್ದ ನಂದ ಹಿಂದಿರುಗಿದ. ಮೊದಲೆ ಆತಂಕಗೊಂಡಿದ್ದ ಅವನ ಮನಸ್ಸು ಮನೆಯ ಮುಂದೆ ಬಿದ್ದಿದ್ದ ರಕ್ಕಸಿಯ ಹೆಣವನ್ನು ಕಾಣುತ್ತಲೆ ಹೌಹಾರಿತು. ಆತ ತನ್ನ ಸಂಗಡಿಗರೊಡನೆ ‘ಅಬ್ಬ, ವಸು ದೇವ ಹೇಳಿದ ಮಾತು ಎಷ್ಟು ಬೇಗ ನಿದರ್ಶನಕ್ಕೆ ಬಂತು! ಆತ ನಿಜವಾಗಿ ಮಹಾನುಭಾವನೆ ಸರಿ!’ ಎಂದು ಹೇಳಿದ. ಅವನಿಗೆ ಹೆಚ್ಚು ಮಾತನಾಡಲು ಅವಕಾಶವಿಲ್ಲ. ಮೊದಲು ಮನೆಯ ಮುಂದಿನ ಹೆಣವನ್ನು ದೂರ ಸಾಗಿಸಬೇಕು. ಆದರೆ ಭಯಂಕರವಾದ ಆ ದೇಹ ವನ್ನು ಹೊರುವುದಕ್ಕೆಲ್ಲಿ ಸಾಧ್ಯ? ಗೋಪಾಲರು ಆ ಹೆಣದ ಒಂದೊಂದೇ ಅಂಗವನ್ನು ಕಡಿದು ಕಡಿದು, ದೂರ ಸಾಗಿಸಿ ಅವನ್ನು ಸುಟ್ಟರು. ಆಗ, ಅಬ್ಬಾ! ಇದೇನಾಶ್ಚರ್ಯ? ಆ ಹೆಣ ವನ್ನು ಸುಟ್ಟರೆ ಶ್ರೀಗಂಧದ ಸುವಾಸನೆಯೆ! ಈ ಅಚ್ಚರಿಯನ್ನು ಜ್ಞಾನಿಗಳು ಮಾತ್ರ ಅರ್ಥ ಮಾಡಿಕೊಳ್ಳಬಲ್ಲರು. ಪರಮಾತ್ಮನು ಶಿಶುರೂಪಿನಿಂದ ಮೊಲೆಗುಡಿದು ಪ್ರಾಣವನ್ನು ಹೀರಿದಮೇಲೆ ಅವಳಲ್ಲಿ ಪಾಪಲೇಪ ಎಲ್ಲಿಯದು? ಆಕೆ ಮೋಕ್ಷವನ್ನು ಪಡೆದಳೆಂಬುದಕ್ಕೆ ಆ ಸುವಾಸನೆಯೆ ಸಾಕ್ಷಿ!

ಪೂತನಿಯನ್ನು ಸುಟ್ಟುಹಾಕಿದ ಮೇಲೆ, ನಂದನು ತನ್ನ ಮಡದಿಯಿಂದ ನಡೆದ ಘಟನೆ ಯನ್ನೆಲ್ಲ ತಿಳಿದು ‘ಹಗ್ಗವೆಂದುದು ಹಾವಾಯಿತಲ್ಲ, ಏನು ಗತಿ? ಇನ್ನು ಮುಂದೆ ಎಚ್ಚರ ದಿಂದಿರು’ ಎಂದ. ಯಾವ ಹಗ್ಗ, ಯಾರಿಗೆ ನಿಜವಾಗಿಯೂ ಹಾವಾಯಿತೆಂದು ಆತನೇನು ಬಲ್ಲ?

