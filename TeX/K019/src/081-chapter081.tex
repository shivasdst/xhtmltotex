
\chapter{೮೧. ಬಲರಾಮನ ತೀರ್ಥಯಾತ್ರೆ}

ಪಾಂಡವ ಕೌರವರು ದಾಯಾದಿಮಾತ್ಸರ್ಯದಿಂದ ಪರಸ್ಪರ ಯುದ್ಧಕ್ಕೆ ಸಿದ್ಧರಾಗು ತ್ತಿರುವರೆಂಬ ಸುದ್ದಿ ದ್ವಾರಕಿಯನ್ನು ಮುಟ್ಟಿತು. ಬಲರಾಮನಿಗೆ ಎರಡು ಪಕ್ಷದವ ರಲ್ಲಿಯೂ ಸಮಾನವಾದ ಪ್ರೇಮ. ಆ ಇತ್ತಂಡದಲ್ಲಿ ಯಾರೊಬ್ಬರೊಡನೆ ಸೇರಿಯೂ ಮತ್ತೊಬ್ಬರೊಡನೆ ಹೋರಾಡುವುದು ಆತನಿಗೆ ಇಷ್ಟವಿರಲಿಲ್ಲ. ಆದ್ದರಿಂದಲೆ ಅವನು ತೀರ್ಥಯಾತ್ರೆಯ ನೆಪದಿಂದ ತಲೆ ಮರೆಸಿಕೊಂಡು ಹೋದನು. ಮೊಟ್ಟಮೊದಲು ಆತನು ಪ್ರಭಾಸಕ್ಷೇತ್ರಕ್ಕೆ ಹೋಗಿ, ಅಲ್ಲಿನ ತೀರ್ಥಕುಂಡದಲ್ಲಿ ಸ್ನಾನ ಮಾಡಿ, ದೇವ ಪುಷಿ ಪಿತೃ ಗಳ ತರ್ಪಣಕಾರ್ಯವನ್ನು ಮಾಡಿ ಮುಗಿಸಿದನು. ಅನಂತರ ಆತನು ಉತ್ತರಭಾರತದ ಲ್ಲಿರುವ ಗಂಗಾ, ಯಮುನಾ, ಸರಸ್ವತೀ ಮೊದಲಾದ ಪುಣ್ಯನದಿಗಳಲ್ಲಿ ಸ್ನಾನಾದಿ ಸಕಲ ಕರ್ಮಗಳನ್ನೂ ನಿರ್ವಹಿಸಿ, ಪುಣ್ಯಕ್ಷೇತ್ರಗಳಲ್ಲಿನ ದೇವರುಗಳನ್ನು ಸಂದರ್ಶಿಸಿ ಯಾತ್ರೆ ಮಾಡುತ್ತಾ ನೈಮಿಶಾರಣ್ಯಕ್ಕೆ ಬಂದನು. ಅಲ್ಲಿ ಪುಷಿಗಳೆಲ್ಲ ಸತ್ರಯಾಗವನ್ನು ಮಾಡುತ್ತಿ ದ್ದರು. ಮಹಾನುಭಾವನಾದ ಬಲರಾಮನನ್ನು ಕಾಣುತ್ತಲೆ ಅವರೆಲ್ಲ ಸಂತೋಷದಿಂದ ಮೇಲಕ್ಕೆದ್ದು, ಇದಿರುಗೊಂಡು ಉಚಿತಾಸನವನ್ನಿತ್ತು ಆತನನ್ನು ಸತ್ಕರಿಸಿದರು. ಆದರೆ ಪುರಾಣವನ್ನು ಹೇಳುತ್ತಾ ಕುಳಿತಿದ್ದ ಸೂತಮಹರ್ಷಿ ತಾನು ಕೂತಲ್ಲಿಂದ ಮೇಲಕ್ಕೇಳಲಿಲ್ಲ; ಬಲರಾಮನನ್ನು ಕಣ್ಣೆತ್ತಿಯೂ ನೋಡಲಿಲ್ಲ. ಇದನ್ನು ಕಂಡ ಬಲರಾಮನು ಕೋಪದಿಂದ ಕಿಡಿಕಿಡಿಯಾಗಿ ‘ಎಲೆ ಮಹರ್ಷಿಗಳೆ, ಇವನು ಸೂತ ಕುಲದವನು. ಬ್ರಾಹ್ಮಣರಾದ ನಿಮ ಗೆಲ್ಲರಿಗಿಂತಲೂ ಎತ್ತರವಾದ ಪೀಠದಲ್ಲಿ ಕುಳಿತಿದ್ದಾನೆ; ಧರ್ಮರಕ್ಷಕನಾದ ನಾನು ಬಂದರೆ ಮೇಲಕ್ಕೇಳುವಷ್ಟು ಮರ್ಯಾದೆ ಕೂಡ ಇವನಿಗೆ ಗೊತ್ತಿಲ್ಲ. ವ್ಯಾಸರಿಗೆ ಶಿಷ್ಯನಾಗಿ, ಧರ್ಮ ಶಾಸ್ತ್ರವನ್ನು ಸಾಂಗವಾಗಿ ತಿಳಿದವನಾದರೂ ಇವನಿಗೆ ಲೋಕ ಮರ್ಯಾದೆ ತಿಳಿದಿಲ್ಲ. ಇವನಿಗೆ ತಾನೇ ಪಂಡಿತನೆಂಬ ಗರ್ವ ನೆತ್ತಿಗೇರಿದೆ. ಧಾರ್ಮಿಕರಂತೆ ವೇಷ ಹಾಕಿಕೊಂಡು, ಅಹಂಕಾರ ದಿಂದ ಮೆರೆಯುವವರನ್ನು ಮುರಿಯುವುದಕ್ಕಾಯೆ ನಾನು ಅವತರಿಸಿರುವುದು’ ಎಂದು ಹೇಳಿ, ಅಲ್ಲಿಯೆ ಇದ್ದ ಒಂದು ದರ್ಭೆಯನ್ನು ತೆಗೆದುಕೊಂಡು, ಅದರಿಂದ ಆತನನ್ನು ಸಂಹರಿಸಿದನು.

ಬಲರಾಮನ ಅಕಾರ್ಯವನ್ನು ಕಂಡು ಅಲ್ಲಿದ್ದ ಪುಷಿಗಳೆಲ್ಲ ಹಾಹಾಕಾರ ಮಾಡಿದರು. ಅವರು ಆತನನ್ನು ಕುರಿತು ‘ಬಲದೇವ, ನೀನು ಅನ್ಯಾಯ ಮಾಡಿದೆ. ನಾವೇ ಆತನನ್ನು ಬ್ರಹ್ಮಾಸನದಲ್ಲಿ ಕೂಡಿಸಿದ್ದೆವು. ಅಲ್ಲಿ ಕೂತವರು ಯಾರು ಬಂದರೂ ಮೇಲಕ್ಕೇಳಬೇಕಾ ದುದಿಲ್ಲ. ಇದನ್ನು ಅರಿಯದೆ ನೀನು ಆತನನ್ನು ಕೊಂದು ಬ್ರಹ್ಮಹತ್ಯೆಯನ್ನು ಮಾಡಿದೆ. ನೀನು ಭಗವಂತನ ಅವತಾರವೇ ನಿಜವಾದರೂ ಬ್ರಹ್ಮಹತ್ಯೆಗೆ ಪ್ರಾಯಶ್ಚಿತ್ತ ಮಾಡಿ ಕೊಳ್ಳಲೇಬೇಕು. ಏಕೆಂದರೆ ದೊಡ್ಡವರು ಅನುಸರಿಸಿದ ಮಾರ್ಗವನ್ನೆ ಇತರರು ಅನುಸರಿಸು ತ್ತಾರೆ’ ಎಂದರು. ಬಲರಾಮನು ಕೃತಕಾರ್ಯಕ್ಕಾಗಿ ಪಶ್ಚಾತ್ತಾಪ ಪಟ್ಟು, ಪ್ರಾಯಶ್ಚಿತ್ತ ಮಾಡಿ ಕೊಳ್ಳಲು ಒಪ್ಪಿದನು. ಪುಷಿಗಳು ‘ಸತ್ರಯಾಗ ಮುಗಿಯುವವರೆಗೆ ಧೀರ್ಘಾಯುಷಿ ಯಾಗಿರುವಂತೆ ನಾವು ಈತನಿಗೆ ಮಾತು ಕೊಟ್ಟಿದ್ದೆವು. ಆ ಮಾತು ಉಳಿಯಬೇಕು. ಮೊದಲು ಇದನ್ನು ನಡೆಸಿಕೊಟ್ಟರೆ ಆಮೇಲೆ ನಿನ್ನ ಪ್ರಾಯಶ್ಚಿತ್ತದ ವಿಚಾರ’ ಎಂದರು. ಬಲರಾಮ ಹೇಳಿದ–“ತಂದೆಯೆ ಪುತ್ರನಾಗುತ್ತಾನೆ. ‘ಆತ್ಮಾ ವೈ ಪುತ್ರನಾಮಸಿ’ ಎಂದು ವೇದಗಳು ಹೇಳುತ್ತವೆ. ಆದ್ದರಿಂದ ಈ ಸೂತನ ಮಗನಾದ ಉಗ್ರಶ್ರವನು ಇದೇ ಬ್ರಹ್ಮಾ ಸನದಲ್ಲಿ ಕುಳಿತು ನಿಮಗೆ ಪುರಾಣ ಹೇಳಲಿ. ತಂದೆಯ ಶಕ್ತಿಯೆಲ್ಲವೂ ಅವನಲ್ಲಿ ಕಾಣಿಸಿ ಕೊಳ್ಳುವಂತೆ ನಾನು ವರ ಕೊಡುತ್ತೇನೆ” ಎಂದನು. ಪುಷಿಗಳು ಆತನ ಮಾತನ್ನು ಸ್ವೀಕರಿಸಿ ‘ಅಯ್ಯಾ, ನೀನು ಬ್ರಹ್ಮಚರ್ಯವ್ರತದಿಂದ ಹನ್ನೆರಡುವರ್ಷ ತೀರ್ಥಯಾತ್ರೆ ಮಾಡ ಬೇಕು. ಇದೇ ನಿನ್ನ ಪಾಪಕ್ಕೆ ಪ್ರಾಯಶ್ಚಿತ್ತ. ನೀನು ಯಾತ್ರೆ ಹೊರಡುವ ಮುನ್ನ ನಮ್ಮ ಯಾಗದಲ್ಲಿ ಮಲಮೂತ್ರಗಳನ್ನು ತಂದು ಚೆಲ್ಲಿ ಹಿಂಸೆ ಮಾಡುತ್ತಿರುವ ಪಲ್ವಲನೆಂಬ ರಕ್ಕಸನನ್ನು ಕೊಂದುಹಾಕು’ ಎಂದು ಕೇಳಿಕೊಂಡರು. ಬಲರಾಮನು ಅದಕ್ಕೊಪ್ಪಿ ಅಲ್ಲಿಯೆ ನಿಂತನು.

ಇದಾದ ಕೆಲವು ದಿನಗಳ ಮೇಲೆ ಒಂದು ಅಮಾವಾಸ್ಯೆಯ ದಿನ ಇದ್ದಕ್ಕಿದ್ದಂತೆಯೆ ಆಶ್ರಮದಲ್ಲೆಲ್ಲ ಬಿರುಗಾಳಿ ಎದ್ದಿತು. ಅದರ ಜೊತೆಯಲ್ಲಿಯೆ ಮೂಗು ಮುಚ್ಚಿಕೊಳ್ಳು ವಂತಹ ಕೆಟ್ಟವಾಸನೆ ಎಲ್ಲೆಲ್ಲಿಯೂ ಹಬ್ಬಿತು. ಇದಾದ ಒಂದೆರಡು ಗಳಿಗೆಗಳಲ್ಲಿಯೇ ಮಲಮೂತ್ರಗಳ ಮಳೆ ಸುರಿಯಲು ಪ್ರಾರಂಭವಾಯಿತು. ಅದರ ಹಿಂದೆಯೇ ಕಾಣಿಸಿ ಕೊಂಡ, ಪಲ್ವಲರಾಕ್ಷಸ. ಇದ್ದಲಿನಂತೆ ಕಪ್ಪಾಗಿರುವ ಅವನ ಬೆಟ್ಟದಂತಹ ದೊಡ್ಡದೇಹ, ಕಾಸಿದ ತಾಮ್ರದ ತಂತಿಯಂತೆ ಕೆಂಪಾದ ಕೂದಲು, ನೇಗಿಲಿನಂತಹ ಉದ್ದವಾದ ಕೋರೆ ದಾಡೆ–ಈ ಭಯಂಕರನಾದ ರಕ್ಕಸನನ್ನು ಕಂಡು ಪುಷಿಗಳೆಲ್ಲ ನಡುಗಿದರು. ಬಲ ರಾಮನು ತನ್ನ ನೇಗಿಲಿನಿಂದ ಅವನನ್ನು ಹತ್ತಿರಕ್ಕೆ ಎಳೆದುಕೊಂಡು, ತನ್ನ ಒನಕೆಯಿಂದ ಅವನ ತಲೆಯ ಮೇಲೆ ಅಪ್ಪಳಿಸಿದನು. ಒಡನೆಯೆ ಅವನ ತಲೆ ಹೋಳಾಗಿ, ಅವನು ಸತ್ತು ಬಿದ್ದನು. ಪುಷಿಗಳಿಗೆ ಆ ಪೀಡೆ ತೊಲಗಿದುದಕ್ಕಾಗಿ ಅತ್ಯಂತ ಸಂತೋಷವಾಯಿತು. ಅವರು ಹೊಗಳಿ ಆಶೀರ್ವದಿಸಿದ ಮೇಲೆ ಬಲರಾಮನಿಗೆ ಮಂಗಳಸ್ನಾನವನ್ನು ಮಾಡಿಸಿ, ವೈಜಯಂತಿಮಾಲೆಯನ್ನೂ, ದಿವ್ಯವಸ್ತ್ರಗಳನ್ನೂ ಕಾಣಿಕೆಯಾಗಿ ಕೊಟ್ಟರು. ಬಲರಾಮನು ಅವರಿಗೆ ನಮಸ್ಕರಿಸಿ, ಅವರಿಂದ ಬೀಳ್ಕೊಂಡು ತನ್ನ ಹನ್ನೆರಡು ವರ್ಷಗಳ ತೀರ್ಥಯಾತ್ರೆ ಯನ್ನು ಆರಂಭಿಸಿದನು.

ಯಾತ್ರೆಯನ್ನು ಕೈಕೊಂಡ ಬಲರಾಮನು ಪ್ರಯಾಗ, ಗಯೆ ಮೊದಲಾದ ಉತ್ತರ ಭಾರತದ ಪುಣ್ಯ ಕ್ಷೇತ್ರಗಳಲ್ಲಿ ಸಂಚರಿಸಿ, ದಕ್ಷಿಣ ಭಾರತದ ಶ್ರೀಶೈಲಕ್ಕೆ ಬಂದನು. ಅಲ್ಲಿ ಪರಶಿವನನ್ನು ಪೂಜಿಸಿ, ತಿರುಪತಿ, ಕಾಮಕೋಟಿ, ಕಾಂಚಿ, ಶ್ರೀರಂಗ, ಮಧುರೆಗಳ ಮಾರ್ಗವಾಗಿ ರಾಮಸೇತುವೆಗೆ ಬಂದು, ಅಲ್ಲಿ ತೀರ್ಥಸ್ನಾನ ಮಾಡಿ ಸಹಸ್ರ ಗೋದಾನ ಮಾಡಿದನು. ಕುಲಪರ್ವತಗಳಲ್ಲಿ ಒಂದಾದ ಮಲಯ ಪರ್ವತಕ್ಕೆ ಬಂದು ಅಲ್ಲಿದ್ದ ಅಗಸ್ತ್ಯಮಹರ್ಷಿಗಳಿಗೆ ನಮಸ್ಕರಿಸಿ, ಕನ್ಯಾಕುಮಾರಿಗೆ ಬಂದು ಅಲ್ಲಿ ಪಾರ್ವತೀದೇವಿಗೆ ಭಕ್ತಿಯಿಂದ ನಮಸ್ಕರಿಸಿದ; ಗೋಕರ್ಣಕ್ಷೇತ್ರದಲ್ಲಿ ಪರಶಿವನ ದರ್ಶನ ತೆಗೆದುಕೊಂಡು ದಕ್ಷಿಣದೇಶದ ಎಲ್ಲ ಪುಣ್ಯಕ್ಷೇತ್ರಗಳಲ್ಲಿಯೂ ಸಂಚರಿಸಿ, ಇಲ್ಲಿನ ಕಾವೇರಿ ಕಪಿಲೆ ಮುಂತಾದ ಪುಣ್ಯತೀರ್ಥಗಳಲ್ಲಿ ಮಿಂದು ಪುನೀತನಾದ. ಅಲ್ಲಿಂದ ಉತ್ತರಕ್ಕೆ ತಿರುಗಿ ನರ್ಮದೆಯಲ್ಲಿ ಸ್ನಾನಮಾಡಿ, ಮತ್ತೆ ಪ್ರಭಾಸತೀರ್ಥಕ್ಕೆ ಬಂದ. ಆ ವೇಳೆಗೆ ಭಾರತ ಯುದ್ಧವೆಲ್ಲ ಮುಗಿದು, ಭೀಮ ದುರ್ಯೋಧನರು ಗದಾಯುದ್ಧದಲ್ಲಿ ತೊಡಗಿರುವುದಾಗಿ ಆತನಿಗೆ ಗೊತ್ತಾಯಿತು. ಸಾಧ್ಯವಾದರೆ ಆ ಯುದ್ಧವನ್ನಾದರೂ ತಪ್ಪಿಸೋಣವೆಂದು ಕೊಂಡು ಆತನು ನೆಟ್ಟಗೆ ಆ ಯುದ್ಧ ನಡೆಯುತ್ತಿದ್ದ ಸ್ಥಳಕ್ಕೆ ಹೋದ. ಆತನನ್ನು ಕಾಣುತ್ತಲೆ ಅಲ್ಲಿದ್ದ ಶ್ರೀಕೃಷ್ಣ ಧರ್ಮರಾಯ ಮೊದಲಾದವರು ಅತ್ಯಂತ ಸಡಗರದಿಂದ ಆತನನ್ನು ಇದಿರುಗೊಂಡು ನಮಸ್ಕರಿಸಿದರು. ಬಲರಾಮನು ಘೋರ ಯುದ್ಧದಲ್ಲಿ ತೊಡಗಿದ್ದ ಭೀಮ ದುರ್ಯೋಧನರನ್ನು ಕುರಿತು ‘ಎಲಾ, ನೀವಿಬ್ಬರೂ ಸಮಬಲರು. ಶಕ್ತಿಯಲ್ಲಿ ಭೀಮ ಮೇಲ್ಗೈಯಾದರೆ ದುರ್ಯೋಧನ ಗದಾಯುದ್ಧ ವಿದ್ಯೆಯಲ್ಲಿ ಅಧಿಕನಾದವನು. ಆದ್ದ ರಿಂದ ನಿಮ್ಮಿಬ್ಬರಲ್ಲಿ ಯಾರೂ ಗೆಲ್ಲುವುದಿಲ್ಲ, ಯಾರೂ ಸೋಲುವುದಿಲ್ಲ. ಸಾಕು, ಈ ಯುದ್ಧವನ್ನು ನಿಲ್ಲಿಸಿರಿ’ ಎಂದು ಹೇಳಿದ. ಆದರೆ ಅವರಿಬ್ಬರೂ ಆತನ ಮಾತಿಗೆ ಮನ್ನಣೆ ಕೊಡುವಂತಿರಲಿಲ್ಲ. ಇದನ್ನು ಕಂಡು ಬಲರಾಮನು ‘ನಿಮ್ಮ ಪ್ರಾಚೀನ ಕರ್ಮ’ ಎಂದು ಕೊಂಡು ಅಲ್ಲಿಂದ ದ್ವಾರಕೆಗೆ ಹಿಂದಿರುಗಿ ಹೋದನು.

