
\chapter{೨೮. ದಿತಿಯ ಮಕ್ಕಳು ದೇವತೆಗಳಾದರು!}

ದಿತಿಯ ಮಕ್ಕಳಾದ ಹಿರಣ್ಯಕಶಿಪು, ಹಿರಣ್ಯಾಕ್ಷರು ತಮ್ಮ ದಾಯಾದಿಗಳಾದ ದೇವತೆಗಳ ಪರಮ ಶತ್ರುಗಳು. ಭಗವಂತನು ದೇವತೆಗಳ ಪಕ್ಷವನ್ನು ವಹಿಸಿ, ಹಿರಣ್ಯಾಕ್ಷನನ್ನು ವರಾಹರೂಪಿನಿಂದಲೂ, ಹಿರಣ್ಯಕಶಿಪುವನ್ನು ನರಸಿಂಹರೂಪದಿಂದಲೂ (ಈ ಕಥೆ ಮುಂದೆ ಬರುತ್ತದೆ) ಕೊಂದುಹಾಕಿದನು. ಶೂರರಾದ ತನ್ನಿಬ್ಬರು ಮಕ್ಕಳನ್ನೂ ಹೀಗೆ ಕೊಲ್ಲಿಸಿದ ದೇವತೆಗಳಲ್ಲಿ–ಅದರಲ್ಲಿಯೂ ಅವರ ರಾಜನಾದ ದೇವೇಂದ್ರನಲ್ಲಿ–ದಿತಿಗೆ ತುಂಬ ದ್ವೇಷ ಹುಟ್ಟಿತು. ಅವನನ್ನು ಕೊಂದ ಹೊರತು ಆಕೆಯ ಮನಸ್ಸಿಗೆ ಶಾಂತಿ ದೊರೆಯದಂತಾಯಿತು. ಅವನನ್ನು ಕೊಲ್ಲುವಂತಹ ಮಗನೊಬ್ಬನು ತನ್ನ ಹೊಟ್ಟೆಯಲ್ಲಿ ಹುಟ್ಟಬೇಕೆಂದು ಆಕೆಯ ಹಗಲಿರುಳಿನ ಕನಸಾಯಿತು. ಈ ಕನಸು ನನಸಾಗಬೇಕಾದರೆ ತನ್ನ ಗಂಡನಾದ ಕಶ್ಯಪಋಷಿಯ ಅನುಗ್ರಹವಾಗಬೇಕು. ಆದ್ದರಿಂದ ಆಕೆ ಗಂಡನ ಮನಸ್ಸನ್ನು ಒಲಿಸಿಕೊಳ್ಳುವ ಕಾರ್ಯದಲ್ಲಿ ತತ್ಪರಳಾದಳು. ಆಕೆಯ ನಯ, ವಿನಯ, ಉಪಚಾರ, ಅನು ರಾಗ ಭಕ್ತಿ–ಇವುಗಳನ್ನು ಕಂಡು ಕಶ್ಯಪಋಷಿ ಹಿರಿಹಿರಿ ಹಿಗ್ಗಿಹೋದ. ಹೆಣ್ಣಿನ ಮೋಹಕ್ಕೆ ಸಿಕ್ಕದ ಅಣ್ಣಗಳಾರು? ಕಶ್ಯಪಋಷಿಯು ಆಕೆಯ ಮೋಹವನ್ನು ಸವಿದು ಮೆಲುಕು ಹಾಕುತ್ತಾ ‘ರಮಣಿ, ನನ್ನ ಮನಸ್ಸು ನಿನ್ನ ಪ್ರೇಮ ಸುಖದಿಂದ ತಣಿದುಹೋಗಿದೆ. ನಿನಗೆ ಬೇಕಾದ ವರವನ್ನು ಕೇಳಿಕೊ, ಕೊಡುತ್ತೇನೆ’ ಎಂದು ಮಾತು ಕೊಟ್ಟನು. ಇಂತಹ ರಸ ನಿಮಿಷವನ್ನೇ ಕಾಯುತ್ತಿದ್ದ ದಿತಿ ‘ಸ್ವಾಮಿ, ನನ್ನ ಹೊಟ್ಟೆಯಲ್ಲಿ ಹುಟ್ಟಿದ ಮಗ ದೇವೇಂದ್ರ ನನ್ನು ಕೊಲ್ಲಬೇಕು. ಇದೇ ನಾನು ಕೇಳುವ ವರ’ ಎಂದಳು.

ಆಕಳಿನಂತೆ ಸಾಧುವಾದ ಕಶ್ಯಪಋಷಿ ಹೆಂಡತಿಯ ಬೇಡಿಕೆಯನ್ನು ಕೇಳಿ ನಿಟ್ಟುಸಿರನ್ನು ಬಿಟ್ಟ. ಆತ ನಿಟ್ಟುಸಿರನ್ನು ಬಿಟ್ಟು, ‘ಅಯ್ಯೋ ನನ್ನ ಅವಿವೇಕವೆ! ನಾನಾಗಿ ನಾನೆ ನನ್ನ ತಲೆಯ ಮೇಲೆ ಬಂಡೆಯನ್ನು ಎಳೆದುಕೊಂಡಂತಾಯಿತು. ಇಂದ್ರಿಯಕ್ಕೆ ವಶನಾಗಿ ಹೆಣ್ಣಿನ ಮೋಸಕ್ಕೆ ಬಲಿಯಾದೆ. ಹೆಣ್ಣಿನ ಮಾತು ಮಧುರ, ಹೃದಯ ವಿಷ ಎಂಬ ವಿವೇಕ ನನ್ನ ನೆರವಿಗೆ ಬರಲಿಲ್ಲ. ನಾನು ಇವಳಿಗೆ ಭಾಷೆಯನ್ನು ಕೊಟ್ಟು ಕೆಟ್ಟೆ. ನನ್ನ ಮಾತಿಗೆ ತಪ್ಪಲಾರೆ; ಹಾಗೆಂದು ದೇವೇಂದ್ರನನ್ನು ಕೊಲ್ಲಿಸಲೆ? ಅದೂ ಸಾಧ್ಯವಿಲ್ಲ’ ಹೀಗೆಂದು ಆತ ಮನಸ್ಸಿನಲ್ಲೇ ಪಶ್ಚಾತ್ತಾಪದಿಂದ ಮಿಡುಕಿಕೊಂಡ. ಆತ ಬೆಂಕಿಯಲ್ಲಿ ಬಿದ್ದ ಹುಳುವಿ ನಂತೆ ಲಿಬಿಲಿಬಿ ಒದ್ದಾಡುತ್ತಿರಲು, ಆತನಿಗೊಂದು ಉಪಾಯ ಹೊಳೆಯಿತು. ಆತನು ಮಡದಿಯನ್ನು ಕುರಿತು ‘ಎಲೆ ಹೆಣ್ಣೆ! ನೀನು ಕೇಳಬಾರದುದನ್ನು ಕೇಳಿದ್ದೀಯೆ; ಚಿಂತೆ ಯಿಲ್ಲ. ನಾನು ಕೊಟ್ಟ ಮಾತಿಗೆ ತಪ್ಪಲಾರೆ. ಆದ್ದರಿಂದ ಇಂದ್ರನನ್ನು ಕೊಲ್ಲುವಂತಹ ಶಕ್ತಿಯುಳ್ಳ ಮಗನೇ ನಿನಗೆ ಹುಟ್ಟುತ್ತಾನೆ. ಆದರೆ ನೀನು ಇಂದಿನಿಂದ ಒಂದು ವರ್ಷದ ವರೆಗೆ ಸದಾ ಶುಚಿಯಾಗಿರಬೇಕು; ನಿನ್ನ ಬಾಯಿಂದ ಸುಳ್ಳು ಬರಬಾರದು; ನೀನು ಕೋಪ ತಾಪಕ್ಕೆ ಒಳಗಾಗಬಾರದು; ಹೆಚ್ಚು ಮಾತಾಡಬಾರದು; ಉತ್ತರ ಪಶ್ಚಿಮ ದಿಕ್ಕುಗಳಿಗೆ ತಲೆಯಿಟ್ಟು ಮಲಗಬಾರದು; ಸಂಧ್ಯಾಕಾಲದಲ್ಲಿ ನಿದ್ದೆ ಮಾಡಬಾರದು; ಗುರು ಹಿರಿಯರನ್ನೂ, ಗೋಬ್ರಾಹ್ಮಣರನ್ನೂ, ಪತಿಯನ್ನೂ, ಭಕ್ತಿಯಿಂದ ಪೂಜಿಸಬೇಕು. ಇದೊಂದು ವ್ರತ ವಾಗಿ ಸ್ವೀಕರಿಸಿ (ಪುಂಸವನ) ನಡೆಸಬೇಕು. ಹೀಗೆ ಮಾಡಿದರೆ ಇಂದ್ರನನ್ನು ಕೊಲ್ಲುವ ಮಗ ನಿನಗೆ ಹುಟ್ಟುತ್ತಾನೆ. ಈ ವ್ರತದಲ್ಲಿ ಲೋಪವಾದರೆ ನಿನಗೆ ಹುಟ್ಟುವವನು ದೇವತೆಗಳ ಸಾಲಿಗೆ ಸೇರಿಹೋಗುತ್ತಾನೆ’ ಎಂದನು.

ಇಂದ್ರನನ್ನು ಕೊಲ್ಲುವ ಮಗನಿಗಾಗಿ ದಿತಿಯು ಎಂತಹ ನಿಯಮವನ್ನು ನಡೆಸುವು ದಕ್ಕೂ ಸಿದ್ಧಳಾಗಿದ್ದಳು; ಗಂಡನ ಅಪ್ಪಣೆಯಂತೆ ನಡೆಯುವುದಾಗಿ ಆಕೆ ಒಪ್ಪಿಕೊಂಡಳು. ಇದನ್ನು ಕೇಳಿ ದೇವೇಂದ್ರನ ಜೀವ ತಲ್ಲಣಿಸಿತು. ಆತ ಯಾವುದಾದರೂ ಉಪಾಯದಿಂದ ದಿತಿಯ ವ್ರತವನ್ನು ಕೆಡಿಸಬೇಕೆಂದು ನಿಶ್ಚಯಮಾಡಿಕೊಂಡು, ಕಶ್ಯಪಋಷಿಯ ಆಶ್ರಮಕ್ಕೆ ಬಂದನು. ಅಲ್ಲಿ ಆತ ತನ್ನ ಚಿಕ್ಕಮ್ಮನಿಗೆ ಅಡವಿಯಿಂದ ಹೂ, ಹಣ್ಣು ಇತ್ಯಾದಿ ಪೂಜಾ ವಸ್ತುಗಳನ್ನು ತಂದುಕೊಡುತ್ತಾ, ಸಿಂಹವು ಜಿಂಕೆಯ ಮೇಲೆ ಬೀಳುವುದಕ್ಕೆ ಹೊಂಚು ಹಾಕುವಂತೆ, ತನ್ನ ಚಿಕ್ಕಮ್ಮನ ವ್ರತವನ್ನು ಹಾಳು ಮಾಡುವುದಕ್ಕೆ ತಕ್ಕ ಸಂದರ್ಭಕ್ಕಾಗಿ ಕಾಯುತ್ತಿದ್ದನು. ಬಹುಕಾಲದಮೇಲೆ ದೇವರ ದಯೆಯಿಂದ ಅಂತಹ ಸಂದರ್ಭ ಸಿಕ್ಕಿತು, ಅವನಿಗೆ. ಒಂದು ದಿನ ದಿತಿಯು ಬೆಳಗಿನಿಂದ ಸಂಜೆಯವರೆಗೆ ನಿತ್ಯನಿಯಮಗಳನ್ನು ಆಚರಿಸಿ ದಣಿದು ಹೋಗಿದ್ದಳು. ಸಂಜೆ ಜಲಬಾಧೆಯನ್ನು ಮುಗಿಸಿ ಬಂದು, ದೇಹಕ್ಕೆ ತುಂಬ ಆಯಾಸವಾಗಿದ್ದುದರಿಂದ ಕಾಲನ್ನೂ ತೊಳೆಯದೆ ನೆಲದ ಮೇಲೆ ಉರುಳಿ ಕೊಂಡಳು. ಆಕೆಗೆ ಗಾಢನಿದ್ರೆ ಬಂದಿತು. ‘ಸಿಕ್ಕಿತು ಸಮಯ’ ಎಂದುಕೊಂಡ, ದೇವೇಂದ್ರ. ತನ್ನ ಯೋಗಶಕ್ತಿಯಿಂದ ಆತ ದಿತಿಯ ಗರ್ಭವನ್ನು ಹೊಕ್ಕ. ಅಲ್ಲಿ ಬಂಗಾರದ ಗಟ್ಟಿಯಂತೆ ಹೊಳೆಯುತ್ತಿದ್ದ ದಿತಿಯ ಗರ್ಭವನ್ನು ತನ್ನ ವಜ್ರಾಯುಧದಿಂದ ಏಳು ತುಂಡುಗಳಾಗಿ ಕತ್ತರಿಸಿದ. ಒಡನೆಯೇ ತುಂಡುಗಳು ಗಟ್ಟಿಯಾಗಿ ಅಳುವುದಕ್ಕೆ ತೊಡಗಿದವು. ಇಂದ್ರನು ‘ಮಾರುದ’–ಅಳಬೇಡಿ–ಎಂದು ಹೇಳುತ್ತಾ, ಆ ಒಂದೊಂದು ತುಂಡನ್ನೂ ಮತ್ತೆ ಏಳು ತುಂಡುಗಳಾಗಿ ಕತ್ತರಿಸಿದ. ಆದರೇನು? ಆ ತುಂಡುಗಳು ಒಂದೊಂದು ಕೈ, ಕಾಲು, ಕಣ್ಣು, ಮೂಗು–ಇತ್ಯಾದಿ ಎಲ್ಲ ಅವಯವಗಳನ್ನೂ ಪಡೆದವು. ಹೀಗೆ ಒಂದು ಗರ್ಭದಿಂದ ಕಾಣಿಸಿಕೊಂಡ ಈ ನಲವತ್ತೊಂಬತ್ತು ಶಿಶುಗಳೂ ಇಂದ್ರನಿಗೆ ಕೈ ಮುಗಿದು, ‘ಅಣ್ಣ, ನಾವೆಲ್ಲ ನಿನ್ನ ತಮ್ಮಂದಿರು. ನಮ್ಮನ್ನೇಕೆ ಕೊಲ್ಲುತ್ತಿ?’ ಎಂದು ಕೇಳಿದವು; ಆಗ ಇಂದ್ರನು ‘ಮಕ್ಕಳೆ, ಹೆದರಬೇಡಿ. ನೀವು ನನ್ನ ತಮ್ಮಂದಿರಂತೆ ಬಾಳುವು ದಾದರೆ ನನಗೆ ತುಂಬ ಸಂತೋಷ’ ಎಂದು ಹೇಳಿ ಹೊರಕ್ಕೆ ಬಂದನು. 

ದಿತಿಯು ಭಗವಂತನ ಪೂಜೆಯನ್ನು ಮಾಡಿದುದರ ಫಲವಾಗಿ ಆಕೆಯ ಗರ್ಭವನ್ನು ಕತ್ತರಿಸಿದರೂ ಅದು ವ್ಯರ್ಥವಾಗಲಿಲ್ಲ. ಅಷ್ಟೇ ಅಲ್ಲ, ಹೊಟ್ಟೆಯಲ್ಲಿದ್ದ ನಲವತ್ತೊಂ ಬತ್ತು ಮಕ್ಕಳೂ ರಾಕ್ಷಸ ಸ್ವಭಾವವನ್ನು ಕಳೆದುಕೊಂಡ ದೇವತೆಗಳಾದರು. ‘ಮಾರುದ’:–ಅಳಬೇಡ–ಎಂದು ದೇವೇಂದ್ರ ಅವರಿಗೆ ಹೇಳಿದುದರಿಂದ ಅವರ ಹೆಸರು ‘ಮಾರುತ’ ರೆಂದಾಯಿತು. ಅವರೆಲ್ಲರೂ ದೇವೇಂದ್ರನ ಹಿಂದೆಯೇ ದಿತಿಯ ಗರ್ಭದಿಂದ ಹೊರಗೆ ಬಂದು ಆತನ ಪಕ್ಕದಲ್ಲಿ ನಿಂತುಕೊಂಡರು. ಆ ವೇಳೆಗೆ ದಿತಿಯು ನಿದ್ರೆಯಿಂದ ಎಚ್ಚೆ ತ್ತಳು. ಆಕೆ ಕಣ್ಣು ಬಿಟ್ಟುನೋಡುತ್ತಾಳೆ–ನಿಗಿನಿಗಿ ಹೊಳೆಯುವ ಕೆಂಡದಂತಹ ಮೈ ಬಣ್ಣದ ಮಕ್ಕಳು ಇದಿರಿಗೆ ನಿಂತಿದ್ದಾರೆ! ಅವರನ್ನು ಕಂಡು ಆಕೆಗೆ ತುಂಬ ಸಂತೋಷ ವಾಯಿತು. ಆಕೆ ದೇವೇಂದ್ರನನ್ನು ಕುರಿತು ‘ಮಗು, ದೇವೇಂದ್ರ! ನಾನು ದೇವತೆಗಳಿಗೆ ಭಯವನ್ನು ಹುಟ್ಟಿಸುವ ಒಬ್ಬ ಮಗನನ್ನು ಬಯಸಿದೆ. ಇದೇನು ನಲವತ್ತೊಂಬತ್ತು ಮಕ್ಕಳು ಹುಟ್ಟಿರುವರಲ್ಲಾ!’ಎಂದು ಕೇಳಿದಳು. ದೇವೇಂದ್ರನು ನಡೆದುದನ್ನು ನಡೆದ ಹಾಗೆ ಆಕೆಗೆ ತಿಳಿಸಿ ‘ಅಮ್ಮ! ನಿನ್ನ ಹೊಟ್ಟೆಯ ಪುಣ್ಯ ಬಹು ದೊಡ್ಡದು. ನಾನು ಕತ್ತರಿಸಿದರೂ ಗರ್ಭನಾಶವಾಗಲಿಲ್ಲ. ಇದು ನಿನ್ನ ಪೂಜಾಫಲ. ದಯವಿಟ್ಟು ನನ್ನ ತಪ್ಪನ್ನು ಕ್ಷಮಿಸು’ ಎಂದು ದೈನ್ಯದಿಂದ ಬೇಡಿಕೊಂಡನು. ಕಪಟವಿಲ್ಲದೆ ನುಡಿದ ಇಂದ್ರನ ಮಾತು ಗಳನ್ನು ಕೇಳಿ ದಿತಿಗೆ ಒಂದು ಬಗೆಯ ಸಮಾಧಾನವಾದಂತಾಯಿತು. ಇಂದ್ರನು ಆಕೆಗೆ ನಮಸ್ಕರಿಸಿ, ತನ್ನ ಹೊಸ ತಮ್ಮಂದಿರೊಡನೆ ಸ್ವರ್ಗಕ್ಕೆ ಹಿಂದಿರುಗಿದನು.

