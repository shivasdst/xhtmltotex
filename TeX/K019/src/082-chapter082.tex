
\chapter{೮೨. ಹಿಡಿಯವಲಕ್ಕಿಯಿಂದ ಹಿರಿಹಿರಿ ಹಿಗ್ಗಿದ}

ಶ್ರೀಕೃಷ್ಣನು ಸಾಂದೀಪರ ಬಳಿ ವಿದ್ಯಾಭ್ಯಾಸ ಮಾಡುತ್ತಿದ್ದಾಗ ಸುಧಾಮ ಆತನ ಸಹಪಾಠಿಯಾಗಿದ್ದ. ಶ್ರೀಕೃಷ್ಣ ರಾಜಕುಮಾರ, ಸುಧಾಮ ಬಡ ಬ್ರಾಹ್ಮಣನ ಮಗ; ಆದರೂ ಅವರಿಬ್ಬರಲ್ಲಿ ತುಂಬ ಗೆಳೆತನವಿತ್ತು. ವಿದ್ಯಾಭ್ಯಾಸ ಮುಗಿದಮೇಲೆ ಶ್ರೀಕೃಷ್ಣ ಮಧುರೆಗೆ ಹಿಂದಿರುಗಿ ತನ್ನ ರಾಜಕಾರ್ಯಗಳಲ್ಲಿ ಮಗ್ನನಾದ. ಸುಧಾಮ ತನ್ನ ಅಗ್ರಹಾರಕ್ಕೆ ಹಿಂದಿರುಗಿ ವೇದಕರ್ಮಗಳಲ್ಲಿ ನಿರತನಾದ. ಕಾಲಕ್ರಮದಲ್ಲಿ ಆತನಿಗೆ ಮದುವೆಯಾಗಿ ಮನೆ ತುಂಬ ಮಕ್ಕಳಾದುವು. ಸುಧಾಮನು ವೇದವೇದಾಂತಗಳಲ್ಲಿ ಪಾರಂಗತನಾದರೂ ಹೊಟ್ಟೆಗೆ ಸಾಕಷ್ಟು ಹಿಟ್ಟಿಗೆ ಮಾತ್ರ ಗತಿಯಿರಲಿಲ್ಲ. ಉಡಲು ಸರಿಯಾದ ಬಟ್ಟೆಗೆ ಕೂಡ ಗತಿಯಿಲ್ಲದೆ ಚಿಂದಿಗಳನ್ನು ಧರಿಸುತ್ತಿದ್ದುದರಿಂದ ಜನ ಈತನನ್ನು ‘ಕುಚೇಲ’ ಎಂಬ ಅಡ್ಡ ಹೆಸರಿನಿಂದ ಕರೆಯುತ್ತಿದ್ದರು. ಆದರೆ ಸುಧಾಮನು ಅದನ್ನು ಗಮನಿಸುತ್ತಿರಲಿಲ್ಲ. ವೈರಾಗ್ಯ ಪರನಾದ ಆತನು ಸದಾ ಶಾಂತಚಿತ್ತನಾಗಿರುತ್ತಿದ್ದನು. ಆತನ ಪುಣ್ಯಕ್ಕೆ ಆತನ ಮಡದಿಯೂ ಆತನಿಗೆ ತಕ್ಕವಳೆ. ಗಂಡ ತಂದುದನ್ನು ಆಕೆ ಆತನ ಮತ್ತು ಮಕ್ಕಳ ಪೋಷಣೆಗಾಗಿ ಬಳಸಿ, ತಾನು ಎಷ್ಟೋ ದಿನ ಉಪವಾಸ ಮಾಡುತ್ತಿದ್ದಳು. ಇದನ್ನು ಕಂಡು ಜನ ಆಕೆಗೆ ‘ಕ್ಷುತ್ ಕ್ಷಾಮೆ’ ಎಂದು ಅಡ್ಡಹೆಸರಿಟ್ಟಿದ್ದರು. ಮಹಾಪತಿವ್ರತೆಯಾದ ಆಕೆ ಒಮ್ಮೆ ಗಂಡನನ್ನು ಕುರಿತು ‘ಅಲ್ಲ, ಸಮುದ್ರದ ನೆಂಟಸ್ತನ, ಉಪ್ಪಿಗೆ ಬಡತನ–ಎಂಬ ಗಾದೆಯಂತೆ, ಸಾಕ್ಷಾತ್ ಲಕ್ಷ್ಮೀಪತಿಯೆ ನಿಮಗೆ ಜೀವದ ಗೆಳೆಯ ಎಂದು ಹೇಳಿಕೊಳ್ಳುವವರು ಹೊಟ್ಟೆಗೆ ಹಿಟ್ಟಿಲ್ಲದೆ ಸಂಕಟಪಡುತ್ತಿರುವಿರಲ್ಲಾ! ಆ ಶ್ರೀಕೃಷ್ಣನಿಗೆ ಬ್ರಾಹ್ಮಣರನ್ನು ಕಂಡರೆ ತುಂಬ ಅಕ್ಕರೆಯಂತೆ; ಹೋಗಿ ಆತನನ್ನು ಕೇಳಿಕೊಂಡರೆ ನಿಮ್ಮ ಸಂಸಾರಕ್ಕೆ ಆಗುವಷ್ಟು ಹಣ ವನ್ನು ಕೊಡಲಾರೆನೆ? ಆತ ಮಹಾ ಉದಾರಿ, ಬಡವರಿಗೆ ಆಧಾರಿ–ಎಂದು ಕೇಳಿದ್ದೇನೆ. ನೀವೊಮ್ಮೆ ದ್ವಾರಕಿಗೆ ಹೋಗಿ ಬರಬಾರದೆ?’ ಎಂದಳು.

‘ಕುಚೇಲ’ನಿಗೆ ತನ್ನ ಮಡದಿಯಾದ ‘ಕ್ಷುತ್ ಕ್ಷಾಮೆ’ಯ ಮಾತುಗಳನ್ನು ಕೇಳಿ ಯಾವುದೋ ಹೊಸ ಬೆಳಕೊಂದು ಕಣ್ಣಿಗೆ ಗೋಚರವಾದಂತಾಯಿತು. ‘ಉಳಿದದ್ದು ಹೇಗಾ ದರೂ ಇರಲಿ; ಈ ನೆಪದಿಂದ ಶ್ರೀಕೃಷ್ಣನ ದರ್ಶನಭಾಗ್ಯ ದೊರೆತಂತಾಗುತ್ತದೆ’ ಎಂದು ಕೊಂಡು, ಆತ ಮಡದಿಯೊಡನೆ ‘ಎಲೆ ಹೆಣ್ಣೆ, ದೊಡ್ಡವರ ಮನೆಗೆ ಬರಿಗೈಲಿ ಹೇಗೆ ಹೋಗಲಿ? ಏನಾದರೂ ತಿನಿಸನ್ನು ಆತನಿಗಾಗಿ ಮಾಡಿ ಕೊಟ್ಟೀಯ?’ ಎಂದು ಕೇಳಿದನು. ಆಕೆ ಮನೆಯಲ್ಲಿ ಏನಿದೆಯೆಂದು ಏನಾದರೂ ಮಾಡಿಕೊಟ್ಟಾಳು? ಪಾಪ, ನೆರೆಹೊರೆಯ ಬ್ರಾಹ್ಮಣರ ಮನೆಗಳಲ್ಲಿ ಕೇಳಿ ನಾಲ್ಕು ಹಿಡಿ ಅವಲಕ್ಕಿಯನ್ನು ತಿರಿದು ತಂದು, ಅವನು ಹೊದ್ದಿದ್ದ ಚಿಂದಿಯ ಸೆರಗಿನಲ್ಲಿ ಅದನ್ನು ಕಟ್ಟಿದಳು. ಕುಚೇಲನು ಈ ಕೈಗಾಣಿಕೆಯೊಡನೆ ದ್ವಾರಕಿಯ ಹಾದಿಯನ್ನು ಹಿಡಿದ. ಪ್ರಯಾಣಮಾಡುತ್ತಾ ಆತನು ‘ಭಗವಂತನಾದ ಶ್ರೀಕೃಷ್ಣನನ್ನು ಸ್ತ್ರೋತ್ರಮಾಡುವ ಮಾತುಗಳೆ ನಿಜವಾದ ಮಾತುಗಳು, ಆತನ ಸೇವೆ ಮಾಡುವ ಕೈಗಳೆ ಕೈಗಳು, ಆತನನ್ನು ಧ್ಯಾನಿಸುವ ಮನಸ್ಸೆ ಮನಸ್ಸು, ಆತನ ಪುಣ್ಯ ಕಥೆಯನ್ನು ಕೇಳುವ ಕಿವಿಗಳೆ ಕಿವಿಗಳು, ಆತನಿಗೆ ನಮಸ್ಕರಿಸುವ ತಲೆಯೆ ತಲೆ, ಆತನನ್ನು ನೋಡುವ ಕಣ್ಣೆ ಕಣ್ಣು! ಆದರೆ ಅರಮನೆಯಲ್ಲಿರುವ ಆತನ ದರ್ಶನ ನನಗೆ ಆಗುವುದು ಹೇಗೆ?’ ಎಂದು ಚಿಂತಿಸುತ್ತಾ ಬಂದು ದ್ವಾರಕೆಯನ್ನು ಸೇರಿದನು. ಹಾಗೂ ಹೀಗೂ ಮಾಡಿ ಆತನು ಶ್ರೀಕೃಷ್ಣನ ಹದಿನಾರು ಸಾವಿರ ಹೆಂಡಿರ ಅಂತಃಪುರದ ಬಳಿಗೆ ಬಂದನು. ಅಲ್ಲಿನ ಹದಿನಾರು ಸಾವಿರ ಮನೆಗಳಲ್ಲಿ ತನಗೆ ಅತ್ಯಂತ ಸುಂದರವಾಗಿರುವಂತೆ ಕಾಣಿಸಿದ ಒಂದು ಅರಮನೆಯ ಬಾಗಿಲಿಗೆ ಹೋಗಿ, ಒಳಗೆ ಇಣಿಕಿ ನೋಡಿದನು. ಅಲ್ಲಿ ಶ್ರೀಕೃಷ್ಣ ರುಕ್ಮಿಣಿ ಯೊಡನೆ ಪಟ್ಟೆ ಮಂಚದ ಮೇಲೆ ಕುಳಿತು ಸರಸವಾಡುತ್ತಿದ್ದನು. ಆ ಮಹಾನುಭಾವನು ಎಂತಹ ಭಕ್ತವತ್ಸಲನೊ! ಆ ಬ್ರಾಹ್ಮಣನನ್ನು ಕಾಣುತ್ತಲೆ ಆತನು ಮಂಚದಿಂದ ಕೆಳಕ್ಕೆ ಧುಮ್ಮಿಕ್ಕಿ ಓಡಿ ಬಂದವನೆ ಕುಚೇಲನನ್ನು ತನ್ನೆರಡು ತೋಳುಗಳಿಂದಲೂ ಆಲಿಂಗಿಸಿ ಕೊಂಡು, ಕೈಹಿಡಿದು ಕರೆತಂದು, ತನ್ನ ಮಂಚದ ಮೇಲೆ ಕೂಡಿಸಿ, ಆತನ ಪಾದವನ್ನು ತೊಳೆದು, ಆ ಪಾದತೀರ್ಥವನ್ನು ತನ್ನ ತಲೆಯ ಮೇಲೆ ಪ್ರೋಕ್ಷಿಸಿಕೊಂಡನು. ನಂತರ ಉಪಹಾರವಿತ್ತು ಸತ್ಕರಿಸಿದ ಮೇಲೆ ಆತನ ಕುಶಲವನ್ನು ವಿಚಾರಿಸಿದನು. ಶ್ರೀಕೃಷ್ಣನ ಸೂಚನೆಯಿಂದ ರುಕ್ಮಿಣೀದೇವಿಯು ಆತನಿಗೆ ಗಾಳಿ ಬೀಸುತ್ತಿದ್ದಳು.

ಹೊಟ್ಟೆಗೆ ಹಿಟ್ಟಿಲ್ಲದೆ ಬಡವಾಗಿ ಕಡ್ಡಿಯಂತಿದ್ದ ಕುಚೇಲನಿಗೆ ಶ್ರೀಕೃಷ್ಣನು ಮಾಡು ತ್ತಿದ್ದ ಆದರೋಪಚಾರಗಳನ್ನು ಕಂಡು ಅಲ್ಲಿದ್ದ ಅಂತಃಪುರದವರಿಗೆಲ್ಲ ಅತ್ಯಂತ ಆಶ್ಚರ್ಯವಾಯಿತು; ‘ಆಹಾ, ಈ ಬಡಬ್ರಾಹ್ಮಣ ಎಂತಹ ಪುಣ್ಯಶಾಲಿ! ಈ ತಿರುಕನನ್ನು ಕಾಣುತ್ತಲೆ, ಅಣ್ಣನಾದ ಬಲರಾಮನಿಗಿಂತಲೂ ಹೆಚ್ಚು ಅಕ್ಕರೆಯಿಂದವನನ್ನು ಆಲಿಂಗಿಸಿ ಕರೆತಂದನಲ್ಲಾ! ಸಾಕ್ಷಾತ್ ಲಕ್ಷ್ಮಿಯಾದ ರುಕ್ಮಿಣೀದೇವಿ ಈತನಿಗೆ ಗಾಳಿ ಹಾಕುವಳಲ್ಲಾ!’ ಎಂದುಕೊಂಡರು. ಶ್ರೀಕೃಷ್ಣ ಸುಧಾಮನ ಕೈಹಿಡಿದುಕೊಂಡು ‘ಗೆಳೆಯ, ಗುರುಕುಲದಿಂದ ನಾವು ಅಗಲಿದಮೇಲೆ ನಡೆದುದನ್ನೆಲ್ಲ ಹೇಳಯ್ಯ. ನಿನಗೆ ಮದುವೆಯಾಯಿತೆ? ಹೆಂಡತಿ ಅನುಕೂಲಳಾಗಿರುವಳೊ? ನೀನು ಮೊದಲಿಂದಲೂ ವೈರಾಗ್ಯಪರ. ನಿನಗೆ ಹಣಕಾಸು ಬೇಡ, ಬಟ್ಟೆಬರೆ ಬೇಡ. ನೀನು ಮಹಾನುಭಾವ ಕಣಯ್ಯ. ನಿನ್ನನ್ನು ಕಟ್ಟಿಕೊಂಡು ನಮ್ಮ ಅತ್ತಿಗೆ ಗತಿ ಏನಾಗಿದೆಯೊ! ಅಲ್ಲಯ್ಯ, ನಾವು ಗುರುಕುಲದಲ್ಲಿದ್ದಾಗ ನಡೆದ ಆ ಸಂಗತಿ ಜ್ಞಾಪಕವಿದೆಯೆ? ಗುರುಗಳ ಮಡದಿ ಒಂದು ಸಂಜೆ ನಮ್ಮಿಬ್ಬರಿಗೂ ಕಟ್ಟಿಗೆ ತರುವಂತೆ ಹೇಳಿದಳು, ನಾವು ಆ ಕೆಲಸಕ್ಕಾಗಿ ಅಡವಿಗೆ ಹೋದಾಗ ಮಳೆ ಬಂತು; ನಾವಿಬ್ಬರೂ ಒಂದು ಮರದ ಕೆಳಗೆ ನಡುಗುತ್ತಾ ನಿಂತಿದ್ದೆವು; ಆಗ ರಾತ್ರಿಯಾಯಿತು; ಕಗ್ಗತ್ತಲಿನಲ್ಲಿ ಕಟ್ಟಿಗೆ ಹೊತ್ತು ನಾವು ದಾರಿತಪ್ಪಿದೆವು; ರಾತ್ರಿಯೆಲ್ಲ ಅಡವಿಯಲ್ಲಿ ಸುತ್ತಾಡಿ ಸಂಕಟಪಟ್ಟೆವು; ಮರುದಿನ ಬೆಳಕು ಹರಿಯುತ್ತಿರುವಂತೆಯೆ ಗುರುಗಳು ನಮ್ಮನ್ನು ಹುಡುಕಿಕೊಂಡು ಬಂದು ಆಶ್ರಮಕ್ಕೆ ಕರೆದುಕೊಂಡು ಹೋದರು. ಆಮೇಲೆ ಗುರುಗಳು “ಎಲಾ ಹುಡುಗರೆ, ನೀವು ಗುರುಭಕ್ತಿಯಿಂದ ಜೀವದ ಭಯವನ್ನೂ ಬಿಟ್ಟು ನನಗಾಗಿ ದುಡಿದಿದ್ದೀರಿ, ಇದನ್ನು ಕಂಡು ನನಗೆ ತುಂಬ ಸಂತೋಷವಾಗಿದೆ. ನೀವು ನನ್ನಿಂದ ಕಲಿತ ವಿದ್ಯೆ ಶಾಶ್ವತವಾಗಲಿ. ನಿಮ್ಮ ಕೋರಿಕೆಗಳೆಲ್ಲ ನೆರವೇರಲಿ!” ಎಂದು ಆಶೀರ್ವಾದ ಮಾಡಿದರು’ ಎಂದ. ಅದನ್ನು ಕೇಳಿ ಸುಧಾಮ ‘ಹೇ ಜಗದ್ಗುರು, ನಿನಗೆ ಗುರುವಿನ ಅನುಗ್ರಹವೆ! ಸ್ವಾಮಿ, ಎಷ್ಟು ಜನ್ಮ ಗಳ ಪುಣ್ಯದಿಂದಲೊ ನಿನಗೆ ಸಹಪಾಠಿಯಾಗುವ ಭಾಗ್ಯ ನನಗೆ ದೊರೆಯಿತು. ವೇದ ಮೂರ್ತಿಯಾದ ನೀನು ಗುರುಕುಲವಾಸ ಮಾಡಿದೆಯೆಂಬುದು ಕೇವಲ ನಿನ್ನದೊಂದು ಲೀಲೆ’ ಎಂದನು.

ಸುಧಾಮನ ನುಡಿಗಳನ್ನು ಕೇಳಿಯೂ ಕೇಳದವನಂತೆ ಶ್ರೀಕೃಷ್ಣನು ‘ಸುಧಾಮ, ನಾವಿ ಬ್ಬರೂ ಹಳೆಯ ಗೆಳೆಯರು; ಬಹು ದಿನಕ್ಕೆ ಇಂದು ಸೇರಿದ್ದೇವೆ. ಎಲ್ಲಿ, ನಿಮ್ಮ ಮನೆ ಯಿಂದ ನನಗೇನು ತಿಂಡಿ ಮಾಡಿಸಿಕೊಂಡು ಬಂದಿದ್ದೀಯೆ? ತೆಗೆ ಹೊರಕ್ಕೆ. ನೀನು ಪ್ರೀತಿ ಯಿಂದ ಏನು ತಂದಿದ್ದರೂ ಅದು ನನಗೆ ಅಮೃತ. ನಾನೇನಯ್ಯ ಅಲ್ಪತೃಪ್ತ. ಒಂದು ಹಣ್ಣು, ಒಂದು ಹೂವು, ಕಡೆಗೆ ಸ್ವಲ್ಪ ನೀರು–ಏನು ಕೊಟ್ಟರೂ ಆನಂದದಿಂದ ತೆಗೆದು ಕೊಳ್ಳುತ್ತೇನೆ’ ಎಂದ. ಆತ ಹಾಗೆ ಹೇಳಿದರೂ ಕುಚೇಲನಿಗೆ ತಾನು ತಂದಿರುವ ಅವಲಕ್ಕಿ ಯನ್ನು ಹೊರತೆಗೆಯಲು ನಾಚಿಕೆ. ಆತ ತಲೆ ಬಾಗಿಸಿ ಸುಮ್ಮನೆ ಕೂತಿದ್ದ. ಅವನು ಸುಮ್ಮನಿ ದ್ದರೆ ಶ್ರೀಕೃಷ್ಣ ಸುಮ್ಮನಿದ್ದಾನೆ? ಅವನ ಕೊಂಕಳಲ್ಲಿದ್ದ ಪುಟ್ಟ ಗಂಟನ್ನು ಕಂಡು, ‘ಇದೇನು?’ ಎಂದು ಅದನ್ನು ಹೊರಕ್ಕೆಳೆದ; ಆ ಗಂಟನ್ನು ಬಿಚ್ಚಿ, ಅದರಲ್ಲಿದ್ದ ಅವಲಕ್ಕಿ ಯನ್ನು ಕಂಡು ‘ಓಹೋ ಅವಲಕ್ಕಿ! ಸುಧಾಮ, ನನಗೆ ಅವಲಕ್ಕಿ ಎಂದರೆ ಪಂಚಪ್ರಾಣ! “ತಾನುಂಟೊ ಮೂರು ಲೋಕವುಂಟೊ” ಎಂಬ ಗಾದೆ ಕೇಳಿಲ್ಲವೆ? ಇದನ್ನು ನಾನು ತಿಂದರೆ ಮೂರು ಲೋಕವೂ ತಿಂದಹಾಗೆ!’ ಎಂದು ಹೇಳುತ್ತಾ, ಒಂದು ಹಿಡಿಯಷ್ಟನ್ನು ತೆಗೆದು ಕೊಂಡು ಹಿರಿಹಿರಿ ಹಿಗ್ಗುತ್ತಾ ಮುಕ್ಕಿದನು. ಅನಂತರ ಆತ ಇನ್ನೊಂದು ಹಿಡಿಯನ್ನು ತಿನ್ನು ವುದಕ್ಕಾಗಿ ಕೈಹಾಕಲು, ರುಕ್ಮಿಣಿಯು ಆತನ ಕೈಯನ್ನು ಹಿಡಿದು ‘ಸ್ವಾಮಿ, ನೀನು ತಿಂದ ಒಂದು ಹಿಡಿ ಈತನ ಇಹ ಪರಗಳೆರಡರ ಸುಖಕ್ಕೂ ಸಾಕಾಗಿದೆ. ಸಾಕು, ನಿಲ್ಲಿಸು’ ಎಂದು ಆತನನ್ನು ತಡೆದಳು.

ಸುಧಾಮನು ಆ ದಿನವೆಲ್ಲ ಶ್ರೀಕೃಷ್ಣನ ಅರಮನೆಯಲ್ಲಿ ರಾಜಭೋಗಗಳನ್ನು ಅನುಭವಿ ಸಿದನು. ಮರುದಿನ ಬೆಳಗ್ಗೆ ಶ್ರೀಕೃಷ್ಣನು ಆ ಬ್ರಾಹ್ಮಣನನ್ನು ಬಾಯಿಮಾತಿನಿಂದ ಬೇಕಾ ದಷ್ಟು ಉಪಚರಿಸಿ ‘ಗೆಳೆಯ, ನೀನಿನ್ನು ಊರಿಗೆ ಹೊರಡಬಹುದು’ ಎಂದು ಹೇಳಿ, ಅವ ನನ್ನು ಊರ ಹೊರಗಿನವರೆಗೆ ಸಾಗಕಳುಹಿಸಿ, ಹಿಂದಿರುಗಿದನು, ‘ನೀನು ಏಕೆ ಬಂದೆ?’ ಎಂದು ಶ್ರೀಕೃಷ್ಣ ಕೇಳಲೂ ಇಲ್ಲ, ಆ ಬ್ರಾಹ್ಮಣ ‘ಇಂತಹುದನ್ನು ಕೊಡು’ ಎಂದು ಕೇಳಲೂ ಇಲ್ಲ. ತಾನು ಹಾಗೆ ಕೇಳಲಿಲ್ಲವಲ್ಲಾ ಎಂಬ ಅತೃಪ್ತಿಯೂ ಅವನಿಗಾಗಲಿಲ್ಲ. ಅವನು ತನ್ನ ಮನಸ್ಸಿನಲ್ಲಿ ‘ಆಹಾ, ಲಕ್ಷ್ಮೀಪತಿಯಾದ ಆ ಶ್ರೀಕೃಷ್ಣನು ದಟ್ಟದರಿದ್ರನಾದ ನನ್ನನ್ನು ಆಲಿಂಗಿಸಿದನಲ್ಲಾ! ನನ್ನನ್ನು ತನ್ನ ಮಂಚದ ಮೇಲೆ ಕೂಡಿಸಿಕೊಂಡು, ತನ್ನ ಪಟ್ಟದ ರಾಣಿಯಿಂದ ಗಾಳಿ ಹಾಕಿಸಿದನಲ್ಲಾ! ದೇವದೇವನಾದ ಆತ ನನ್ನ ಕಾಲನ್ನು ತೊಳೆದು, ನಾನೇ ದೇವರೇನೋ ಎನ್ನುವಂತೆ ಉಪಚರಿಸಿದನಲ್ಲಾ! ಸರ್ವಾಂತರ್ಯಾಮಿಯಾದ ಆತನಿಗೆ ನನ್ನ ಮನಸ್ಸಿನ ಆಶೆ ಗೊತ್ತಾಗಿರಬೇಕು; ಆದರೂ ಆತ ನನಗೆ ಧನವೇನನ್ನೂ ಕೊಡ ದಿದ್ದುದು ನೋಡಿದರೆ ನನ್ನ ಉದ್ಧಾರಕ್ಕಾಗಿಯೆ ಆತ ಹಾಗೆ ಮಾಡಿರಬೇಕು! ಭಾಗ್ಯವಂತ ನಾದರೆ ಭಗವಂತ ಮರೆತು ಹೋಗುತ್ತಾನೆ’ ಎಂದುಕೊಳ್ಳುತ್ತಾ ಆತ ತನ್ನ ಅಗ್ರಹಾರವನ್ನು ಸೇರಿ, ತನ್ನ ಮನೆಗೆ ಹೋದ. ಆದರೆ ಇದೇನು? ಅವನ ಗುಡಿಸಲಿದ್ದ ಕಡೆ ಅರಮನೆ ನಿಂತಿದೆ! ಅದರ ಸುತ್ತ ಸೊಗಸಾದ ಉದ್ಯಾನವನ! ಮನೆಯೊಳಗೆ ದಿವ್ಯವಾದ ವಸ್ತ್ರಾಭರಣಗಳನ್ನು ಧರಿಸಿದ ಹೆಣ್ಣುಗಂಡುಗಳು ಓಡಾಡುತ್ತಿದ್ದಾರೆ! ಕುಚೇಲನಿಗೆ ದಾರಿತಪ್ಪಿ ದೇವೇಂದ್ರನ ಅರ ಮನೆಗೆ ಬಂದಿರಬೇಕೆನ್ನಿಸಿತು. ಅಷ್ಟರಲ್ಲಿ ಆತನ ಹೆಂಡತಿ ಮಹಾಲಕ್ಷ್ಮಿಯಂತೆ ವಸ್ತ್ರಾ ಲಂಕಾರಗಳಿಂದ ಕಂಗೊಳಿಸುತ್ತಾ ಮಂಗಳಗೀತ ವಾದ್ಯಗಳೊಡನೆ ಗಂಡನನ್ನು ಇದಿರು ಗೊಂಡು, ಆತನನ್ನು ಕೈಹಿಡಿದು ಅರಮನೆಯೊಳಕ್ಕೆ ಕರೆದೊಯ್ದಳು. ಅಬ್ಬಾ ಮನೆಯೆ! ಅಲ್ಲಿನ ಕಂಬಗಳೆಲ್ಲ ರತ್ನದವು, ಗೋಡೆಗಳೆಲ್ಲ ಸ್ಫಟಿಕ, ಮೇಲ್ಕಟ್ಟೆಲ್ಲ ಮುತ್ತಿನ ಜಾಲರಿ, ಅಲ್ಲಲ್ಲಿರುವ ಮಂಚಗಳೆಲ್ಲ ಬಂಗಾರದವು, ಆ ಮಂಚಗಳ ಮೇಲೆ ಹಾಲಿನ ನೊರೆಯಂ ತಿರುವ ಸುಪ್ಪತ್ತಿಗೆಗಳು. ನೆಲದ ಮೇಲೆ ಅಲ್ಲಲ್ಲಿಯೆ ಬಂಗಾರದ ಮಣೆಗಳು, ಅವುಗಳ ಮೇಲೆ ಮೆತ್ತನೆಯ ಸುಪ್ಪತ್ತಿಗೆಗಳು. ಬಂಗಾರದ ಹಿಡಿಯುಳ್ಳ ಚಾಮರಗಳನ್ನು ಹಿಡಿದು ದಾಸದಾಸಿಯರು ನಿಂತಿದ್ದಾರೆ. ರತ್ನದ ದೀವಿಗೆಗಳಿಂದ ತಂಪಾದ ಬೆಳಕು ಅರಮನೆಯನ್ನು ಬೆಳಗುತ್ತಿದೆ. ಕುಚೇಲ ದಂಗುಬಡಿದುಹೋದ. ಆತ ‘ಆಹಾ ಶ್ರೀಕೃಷ್ಣನ ಮಹಿಮೆಯೆ! ನಾನು ಕೇಳದೆಯೆ ನೀನಿಷ್ಟು ಐಶ್ವರ್ಯವನ್ನಿತ್ತೆ, ಸ್ವಾಮಿ! ಅರೆಕಾಸಿನ ಭಕ್ತಿಗೆ ಅರಮನೆಯನ್ನೆ ಕೊಡುವ ಉದಾರಿ ನೀನು! ಹಿಡಿಯವಲಕ್ಕಿಗೆ ಹಿರಿಹಿರಿ ಹಿಗ್ಗಿ ಈ ಒಡೆತನ ಇತ್ತೆಯಲ್ಲಾ! ನಾನು ಇದನ್ನು ಕಟ್ಟಿಕೊಂಡು ಏನುಮಾಡಲಿ? ಈ ಭೋಗಭಾಗ್ಯಗಳಿಂದ ನನಗೆ ಅಹಂಕಾರ ಮೂಡದಂತೆ ನನ್ನನ್ನು ಅನುಗ್ರಹಿಸು. ನಿನ್ನ ಮೇಲಿನ ಭಕ್ತಿ ನನ್ನಲ್ಲಿ ನಿರಂತರವಾಗಿರಲಿ’ ಎಂದು ಭಗವಂತನನ್ನು ಬೇಡಿಕೊಂಡ. ಜ್ಞಾನಿಯಾದ ಆತನು ತನಗೆ ಮಹದೈಶ್ವರ್ಯ ಬಂದಾಗಲೂ ಅವಿವೇಕಿಯಾಗಲಿಲ್ಲ. ಸಕಲ ಸುಖಭೋಗಗಳ ಮಧ್ಯದಲ್ಲಿಯೂ ಆ ಬ್ರಹ್ಮಜ್ಞಾನಿ ತಾವರೆ ಎಲೆಯ ಮೇಲಿನ ಜಲಬಿಂದುವಿನಂತಿದ್ದು, ಕಡೆಯಲ್ಲಿ ಭಗವಂತನ ಪಾದಾರವಿಂದದಲ್ಲಿ ಒಂದಾಗಿಹೋದ.

