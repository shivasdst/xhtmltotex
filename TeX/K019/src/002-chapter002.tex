
\chapter{೨. ಪರೀಕ್ಷಿದ್ರಾಜ}

ಸೂತಮಹರ್ಷಿ ಶೌನಕಾದಿಗಳನ್ನು ಕುರಿತು ‘ಅಯ್ಯಾ, ಪರೀಕ್ಷಿದ್ರಾಜನು ಪ್ರಾಯೋಪ ವೇಶವನ್ನು ಏಕೆ ಕೈಗೊಂಡನೆಂದು ಕೇಳಿದಿರಲ್ಲವೇ? ಆತನ ಕಥೆಯನ್ನು ಆದ್ಯಂತವಾಗಿ ಹೇಳುತ್ತೇನೆ, ಕೇಳಿರಿ. ಭಾರತಯುದ್ಧದ ಕಡೆಯ ದಿನ ಭೀಮಸೇನನು ದುರ್ಯೋಧನನ ತೊಡೆಗಳನ್ನು ತನ್ನ ಗದಾ ದಂಡದಿಂದ ಮುರಿದು, ಅವನನ್ನು ಕೆಳಕ್ಕೆ ಕೆಡವಿದನು. ಧೂಳಿ ನಲ್ಲಿ ಹೊರಳಿ ನರಳುತ್ತಿದ್ದ ಆ ಕೌರವನನ್ನು ಕಂಡು ಅಶ್ವತ್ಥಾಮನ ಕ್ರೋಧ ಹೊತ್ತಿತು. ಅವನು ತನ್ನ ಸ್ವಾಮಿಗೆ ಸಂತೋಷವುಂಟುಮಾಡಬೇಕೆಂದು ಬಗೆದು, ಪಾಂಡವರ ಪಾಳೆಯಕ್ಕೆ ನುಗ್ಗಿ, ದ್ರೌಪದಿಯ ಮಕ್ಕಳಾದ ಉಪಪಾಂಡವರನ್ನು ಕೊಂದು, ಅವರ ತಲೆ ಗಳನ್ನು ಕೌರವನಿಗೆ ತೋರಿದನು. ಆದರೆ ಕೌರವನು ಅಶ್ವತ್ಥಾಮನ ಕಾರ್ಯವನ್ನು ಮೆಚ್ಚದೆ ಅವನನ್ನು ನಿಂದಿಸಿದನು. ಅಷ್ಟರಲ್ಲಿ ದ್ರೌಪದಿ ತನ್ನ ಐವರು ಮಕ್ಕಳೂ ಹತರಾದುದನ್ನು ಕಂಡು ಅಸಹ್ಯ ವೇದನೆಯಿಂದ ಗಳಗಳ ಅತ್ತಳು. ಇದನ್ನು ಕೇಳಿದ ಅರ್ಜುನನು, ಆಕೆಯ ಬಳಿಗೆ ಓಡಿಬಂದು, ನಡೆದ ಸಂಗತಿಗಳನ್ನು ಅರಿತವನಾಗಿ, ಕಣ್ಣುಗಳಿಂದ ಕಿಡಿಗಳನ್ನು ಸುರಿಸುತ್ತಾ ‘ಈ ನೀಚಕಾರ್ಯವನ್ನು ಮಾಡಿದ ಬ್ರಾಹ್ಮಣಾಧಮನನ್ನು ವಧಿಸುವೆನು’ ಎಂದು ಪ್ರತಿಜ್ಞೆ ಮಾಡಿ, ಅಶ್ವತ್ಥಾಮನನ್ನು ಹುಡುಕ ಹೊರಟನು. ಅವನನ್ನು ದೂರದಿಂದ ಕಂಡ ಅಶ್ವತ್ಥಾಮನು ಜೀವ ಭಯದಿಂದ ಪಲಾಯನ ಮಾಡಿದನು. ಆದರೆ ಸಾಕ್ಷಾತ್ ಶ್ರೀಕೃಷ್ಣನೇ ನಡೆಸುತ್ತಿದ್ದ ರಥದ ವೇಗಕ್ಕೆ ಅಶ್ವತ್ಥಾಮ ಎಲ್ಲಿಯ ಸಾಟಿ? ತನಗಿನ್ನು ಉಳಿಗಾಲವಿಲ್ಲವೆಂದುಕೊಂಡ ಅಶ್ವತ್ಥಾಮನು, ತನಗೆ ಅದರ ಉಪಸಂಹಾರ ಗೊತ್ತಿಲ್ಲ ದಿದ್ದರೂ ಬ್ರಹ್ಮಾಸ್ತ್ರವನ್ನು ಪ್ರಯೋಗಿಸಿದನು. ಆ ಅಸ್ತ್ರಕ್ಕೆ ಇದಿರೇ ಇಲ್ಲವಲ್ಲ! ಮುಂದೋರದ ಅರ್ಜುನನು ಶ್ರೀಕೃಷ್ಣನ ಕಡೆ ನೋಡಿದನು. ಆತ ‘ಅಯ್ಯಾ, ಇದಕ್ಕೆ ಇನ್ನಾವ ಪ್ರತೀಕಾರವೂ ಇಲ್ಲ. ನೀನೂ ಬ್ರಹ್ಮಾಸ್ತ್ರವನ್ನೇ ಬಿಟ್ಟು ಇದನ್ನು ತಡೆಯಬೇಕು’ ಎಂದನು. ಅರ್ಜುನನೂ ಬ್ರಹ್ಮಾಸ್ತ್ರವನ್ನು ಬಿಟ್ಟನು. ಇವೆರಡರ ಹೋರಾಟದಲ್ಲಿ ಜಗತ್ತೇ ಪ್ರಳಯವಾಗುವಂತಾಯ್ತು. ಆಗ ಶ್ರೀಕೃಷ್ಣನ ಸಲಹೆಯಂತೆ ಅರ್ಜುನನು ಎರಡು ಅಸ್ತ್ರ ಗಳನ್ನೂ ಉಪಸಂಹಾರ ಮಾಡಿದನು. ಅನಂತರ ಆತನು ಅಶ್ವತ್ಥಾಮನ ಕಡೆ ನುಗ್ಗಿ ಹೋಗಿ, ಯಜ್ಞಪಶುವನ್ನು ಕಟ್ಟುವಂತೆ ಹಗ್ಗದಿಂದ ಅವನನ್ನು ಕಟ್ಟಿ ದ್ರೌಪದಿಯ ಬಳಿಗೆ ಎಳೆದುಕೊಂಡುಹೋದನು. ಅಳುಮುಖದಿಂದ ತನ್ನೆದುರಿಗೆ ನಿಂತಿದ್ದ ಆ ಬ್ರಾಹ್ಮಣನನ್ನು ಕಂಡು ದ್ರೌಪದಿಯ ಕರುಳು ಕರಗಿತು. ಆಕೆ ಅರ್ಜುನನೊಡನೆ–‘ಸ್ವಾಮಿ, ಇವನು ಬ್ರಾಹ್ಮಣ, ಗುರುಪುತ್ರ; ಹೋಗಲಿ ಬಿಟ್ಟುಬಿಡು. ಪಾಪ, ಇವನ ತಾಯಿ ಮಗನ ಮೇಲಿನ ಮೋಹದಿಂದ ಸಹಗಮನಮಾಡದೆ ಬದುಕಿದ್ದಾಳೆ. ನಾನು ಮಕ್ಕಳನ್ನು ಕಳೆದುಕೊಂಡು ಅಳುತ್ತಿರುವಂತೆ ಮಹಾ ಪತಿವ್ರತೆಯಾದ ಇವನ ತಾಯಿಯೂ ಅಳುವುದು ಬೇಡ’ ಎಂದಳು. ಧರ್ಮರಾಯನೂ ನಕುಲ ಸಹದೇವರೂ ಈ ಮಾತನ್ನು ಅನುಮೋದಿಸಿದರು. ಭೀಮನು ಮಾತ್ರ ಗದೆಯನ್ನು ತಿರುಗಿಸುತ್ತಾ ‘ಛೆ, ಛೆ, ಮಲಗಿ ನಿದ್ರಿಸುತ್ತಿದ್ದ ಎಳೆಯ ಮಕ್ಕಳನ್ನು ಅಕಾರಣವಾಗಿ ಕೊಂದ ಈ ಭ್ರಷ್ಟನನ್ನು ಬಲಿಹಾಕಲೇಬೇಕು’ ಎಂದು ಅಬ್ಬರಿಸಿ ದನು. ಶ್ರೀಕೃಷ್ಣನು ನಸುನಗುತ್ತಾ ‘ಅರ್ಜುನ, ನಿನ್ನ ಪ್ರತಿಜ್ಞೆಯೂ ನೆರವೇರಬೇಕು; ದ್ರೌಪದಿಯೇ ಮೊದಲಾದವರ ಮಾತೂ ನಡೆಯ ಬೇಕು–ಹಾಗೆ ಮಾಡು’ ಎಂದನು. ಅರ್ಜುನನಿಗೆ ಶ್ರೀಕೃಷ್ಣನ ಅಭಿಪ್ರಾಯ ಅರ್ಥವಾಯಿತು. ಅವನು ಕತ್ತಿಯನ್ನು ಹಿರಿದು, ಅಶ್ವತ್ಥಾಮನ ತಲೆ ಕೂದಲನ್ನು ಕತ್ತರಿಸಿ, ಅವನ ತಲೆಯಲ್ಲಿದ್ದ ಮಣಿಯನ್ನು ಹೊರತೆಗೆ ದನು. ಆ ರಕ್ಷಾಮಣಿ ಹೋದ ಮೇಲೆ ಆ ಬ್ರಾಹ್ಮಣ ಇದ್ದೂ ಸತ್ತಂತಾಯಿತು. ಅನಂತರ ಅರ್ಜುನನು ಅವನ ಕಟ್ಟುಗಳನ್ನು ಬಿಚ್ಚಿ, ಅವನನ್ನು ಶಿಬಿರದಿಂದ ಹೊರಗೆ ಹಾಕಿದನು.

ಅಪಮಾನದಿಂದ ನೊಂದು, ಹೆಡೆತುಳಿದ ಹಾವಿನಂತೆ ಬುಸುಗುಟ್ಟುತ್ತಾ ಹೊರಟ ಅಶ್ವ ತ್ಥಾಮನು ಭೂಮಿಯಲ್ಲಿ ಪಾಂಡವರ ಸಂತತಿಯೇ ಇಲ್ಲದಂತೆ ಮಾಡಬೇಕೆಂದು ಬಗೆದು, ಭಯಂಕರವಾದ ನಿಷ್ಪಾಂಡವಾಸ್ತ್ರವನ್ನು ಪ್ರಯೋಗಿಸಿದನು. ಅದು ಅಭಿಮನ್ಯುವಿನ ಹೆಂಡತಿಯಾದ ಉತ್ತರೆಯ ಗರ್ಭವನ್ನು ಭೇದಿಸಹೊರಟಿತು. ಇದನ್ನು ದೂರದಿಂದಲೆ ಕಂಡ ಉತ್ತರೆ ಶ್ರೀಕೃಷ್ಣನ ಬಳಿಗೆ ಓಡಿಬಂದು ‘ಸ್ವಾಮಿ, ಶ್ರೀಕೃಷ್ಣ ಪರಮಾತ್ಮ, ಅಲ್ಲಿ ನೋಡು, ಕಾದ ಕಬ್ಬಿಣದ ಅಲಗುಗಳಿಂದ ಕೂಡಿದ ಬಾಣವೊಂದು ನನ್ನನ್ನು ಅಟ್ಟಿ ಬರು ತ್ತಿದೆ. ಅದು ನನ್ನನ್ನು ಕೊಂದರೂ ಚಿಂತೆಯಿಲ್ಲ; ನನ್ನ ಗರ್ಭವನ್ನು ಮಾತ್ರ ಕಾಪಾಡು’ ಎಂದು ಕಣ್ಣೀರುಗರೆದಳು. ಆಗ ಕರುಣಾಮಯನಾದ ಶ್ರೀಕೃಷ್ಣನು, ಮತ್ತಾವ ಅಸ್ತ್ರವೂ ಅದನ್ನು ತಡೆಯಲಾರದಾದುದರಿಂದ ತನ್ನ ಚಕ್ರವನ್ನೇ ಪ್ರಯೋಗಿಸಿ ಅದನ್ನು ಕತ್ತರಿಸಿ ಹಾಕಿದನು.

ಹೀಗೆ ಪಾಂಡವರಿಗೆ ಬಂದಿದ್ದ ಅಪಾಯಗಳನ್ನು ಹೋಗಲಾಡಿಸಿ, ತಾನಿನ್ನು ದ್ವಾರಕೆಗೆ ತೆರಳಬೇಕೆಂದು ಶ್ರೀಕೃಷ್ಣನು ನಿಶ್ಚಯಿಸಿದನು. ಆದರೆ ಕುಂತಿದೇವಿಯೂ ಧರ್ಮರಾಯನೂ ಅದಕ್ಕೆ ಒಪ್ಪಲಿಲ್ಲ. ಇನ್ನು ಕೆಲವು ದಿನಗಳು ಹಸ್ತಿನಾವತಿಯಲ್ಲಿ ನಿಲ್ಲಬೇಕೆಂದು ಬಲವಂತ ಮಾಡಿ ಬೇಡಿಕೊಂಡರು. ಶ್ರೀಕೃಷ್ಣನು ಒಪ್ಪಿಕೊಂಡನು. ಆಗ ಉತ್ತರಾಯಣ ಪ್ರಾಪ್ತ ವಾದುದರಿಂದ ಅದು ಭೀಷ್ಮನು ದೇಹತ್ಯಾಗಮಾಡುವ ಕಾಲ. ಶ್ರೀಕೃಷ್ಣನು ಪಾಂಡವ ರೊಡನೆ ಯುದ್ಧರಂಗಕ್ಕೆ ಹೋಗಿ, ಶರಶಯ್ಯೆಯಲ್ಲಿ ಮಲಗಿದ್ದ ಭೀಷ್ಮನನ್ನು ಕಂಡನು. ಆ ವೃದ್ಧಪಿತಾಮಹನಿಗೆ ಕೊನೆಗಾಲದಲ್ಲಿ ಶ್ರೀಕೃಷ್ಣನ ದರ್ಶನವಾದುದರಿಂದ ಅತ್ಯಂತ ಸಂತೋಷವಾಯಿತು. ಆತನನ್ನು ಭಕ್ತಿಯಿಂದ ಸ್ತೋತ್ರ ಮಾಡುತ್ತಾ, ತನ್ನ ಮನಸ್ಸನ್ನು ಆತನಲ್ಲಿಯೇ ನೆಲೆಯಾಗಿ ನಿಲ್ಲಿಸಿ, ತನ್ನ ಪ್ರಾಣವಾಯುಗಳನ್ನು ದೇಹದಲ್ಲಿಯೇ ಅಡಗಿಸಿ ಕೊಂಡು, ಮುಕ್ತಿಯನ್ನು ಪಡೆದನು.

ಭೀಷ್ಮನು ಮುಕ್ತಿಪದವಿಗೆ ಏರಿದಮೇಲೆ, ಶ್ರೀಕೃಷ್ಣನು ತಾನಿನ್ನು ದ್ವಾರಕೆಗೆ ಹೊರಡುವೆ ನೆಂದು ನಿಶ್ಚಯಿಸಿದನು. ಪಾಂಡವರು ಆತನನ್ನು ಅಗಲಿರಲಾರದೆ ದುಗುಡಗೊಂಡರೂ ಅನಿವಾರ್ಯವಾಗಿ ಆತನನ್ನು ಬೀಳ್ಕೊಟ್ಟರು. ಅವರು ಕೊಟ್ಟ ಕಾಣಿಕೆಗಳನ್ನೂ ವಂದನೆ ಗಳನ್ನೂ ನಗೆಮೊಗದಿಂದ ಸ್ವೀಕರಿಸಿ, ಹೇಳಿಕಳುಹಿಸಿದೊಡನೆಯೇ ಮತ್ತೆ ಬರುವುದಾಗಿ ಭರವಸೆಯಿತ್ತು ಶ್ರೀಕೃಷ್ಣನು ಪ್ರಯಾಣ ಬೆಳೆಸಿದನು. ಪಾಂಡವರು ಧರ್ಮದಿಂದ ರಾಜ್ಯ ಪರಿಪಾಲನೆ ಮಾಡುತ್ತಾ, ಭಗವದ್ಧ್ಯಾನತತ್ಪರರಾಗಿರಲು, ಉತ್ತರೆಗೆ ನವಮಾಸ ತುಂಬಿ ಒಳ್ಳೆಯ ಶುಭದಿನ ಶುಭಲಗ್ನದಲ್ಲಿ ಗಂಡು ಮಗುವನ್ನು ಹೆತ್ತಳು. ತನ್ನ ವಂಶವೃಕ್ಷದಲ್ಲಿ ಮೂಡಿದ ಈ ಹೊಸ ಕುಡಿಯನ್ನು ಕಂಡು ಧರ್ಮರಾಯನ ಸಂತೋಷಕ್ಕೆ ಕೊನೆ ಮೊದಲಿಲ್ಲವಾಯಿತು. ಆತನು ಅತ್ಯಂತ ಸಂಭ್ರಮದಿಂದ ಜಾತಕರ್ಮ ಮಹೋತ್ಸವ ವನ್ನು ನೆರವೇರಿಸಿ, ಯಥೇಚ್ಛವಾಗಿ ದಾನಧರ್ಮಾದಿಗಳನ್ನು ನಡೆಸಿದನು. ಪುರೋಹಿತರು ಆ ಮಗುವಿನ ಜಾತಕವನ್ನು ಬರೆದು, ಧರ್ಮರಾಯನೊಡನೆ–‘ಮಹಾರಾಜ, ಈ ನಿನ್ನ ಮೊಮ್ಮಗನು ಸಾಮಾನ್ಯನಲ್ಲ. ಇಕ್ಷ್ವಾಕುವಿನಂತೆ ಪ್ರಜಾರಂಜಕನು, ಶ್ರೀರಾಮನಂತೆ ಗೋಬ್ರಾಹ್ಮಣ ಸಂರಕ್ಷಕನು, ಶಿಬಿ ಚಕ್ರವರ್ತಿಯಂತೆ ಬಂಧುಜನಪ್ರಿಯನು, ನಿನ್ನ ತಮ್ಮ ನಾದ ಅರ್ಜುನನಂತೆ ಬಿಲ್ ವಿದ್ಯಾ ಚತುರನು, ಅಗ್ನಿಯಂತೆ ಭಯಂಕರನು, ಸಮುದ್ರ ದಂತೆ ಗಂಭೀರನು, ಭೂಮಿಯಂತೆ ತಾಳ್ಮೆಯುಳ್ಳವನು, ಪರಮೇಶ್ವರನಂತೆ ವರ ಪ್ರದಾತನು, ವಿಷ್ಣುವಿನಂತೆ ಎಲ್ಲರಿಗೂ ಆಶ್ರಯನು, ಪ್ರಹ್ಲಾದನಂತೆ ಭಗವದ್ಭಕ್ತನಾದ ಈತನು ಗುರುಜನರ ಸೇವೆಯನ್ನು ಮಾಡಿ ಅತ್ಯಂತ ಕೀರ್ತಿಶಾಲಿಯಾಗುವನು. ಈತನಿಂದ ನಿಮ್ಮ ವಂಶ ಬೆಳೆದು ಸ್ಥಿರವಾಗುವುದು’ ಎಂದು ಭವಿಷ್ಯವನ್ನು ಹೇಳಿದರು. ಈ ಮಗು ಗರ್ಭದಲ್ಲಿರುವಾಗ ಅಶ್ವತ್ಥಾಮನ ಬಾಣವನ್ನು ಕಂಡು ನಡುಗುತ್ತಿರಲು, ಶ್ರೀ ಮಹಾ ವಿಷ್ಣುವು ಅಂಗುಷ್ಠ ಮಾತ್ರ ರೂಪದಿಂದ ಕಾಣಿಸಿಕೊಂಡು ಇವನಿಗೆ ಅಭಯವನ್ನು ನೀಡಿದನು. ಆಗ ಕಂಡ ದಿವ್ಯಪುರುಷನು ಎಲ್ಲಿರುವನೆಂದು ಎಲ್ಲರಲ್ಲಿಯೂ ಪರೀಕ್ಷಿಸು ತ್ತಿದ್ದುದರಿಂದ ಈ ಮಗುವಿಗೆ ಪರೀಕ್ಷಿದ್ರಾಜನೆಂದೇ ಹೆಸರಾಯಿತು. ಮಗುವು ಶುಕ್ಲಪಕ್ಷದ ಚಂದ್ರನಂತೆ ದಿನದಿನಕ್ಕೂ ಬೆಳೆಯುತ್ತಾ ಎಲ್ಲರ ಕಣ್ಮಣಿಯಾಯಿತು.

ಪರೀಕ್ಷಿತಕುಮಾರನು ಬೆಳೆದು ಬಾಲಕನಾದನು. ಆತನ ವಿದ್ಯಾಭ್ಯಾಸ ಸಾಂಗವಾಗಿ ನೆರವೇರುತ್ತಿತ್ತು. ಆಗ ಧರ್ಮರಾಯನು ಭಾರತ ಯುದ್ಧದಲ್ಲಿ ನಡೆದ ಬಂಧುಹತ್ಯಾ ದೋಷವನ್ನು ಹೋಗಲಾಡಿಸಿಕೊಳ್ಳುವುದಕ್ಕಾಗಿ ಶ್ರೀಕೃಷ್ಣನ ಸಹಾಯದಿಂದ ಮೂರು ಅಶ್ವಮೇಧ ಯಾಗಗಳನ್ನು ವಿಜೃಂಭಣೆಯಿಂದ ನಡೆಸಿ, ಭಗವಂತನನ್ನು ತೃಪ್ತಿಪಡಿಸಿದನು. ಇದಾದ ಮೇಲೆ ಶ್ರೀಕೃಷ್ಣನು ಅರ್ಜುನನ್ನೂ ತನ್ನ ಜೊತೆಯಲ್ಲಿ ಕರೆದುಕೊಂಡು ದ್ವಾರಕೆಗೆ ಹಿಂದಿರುಗಿದನು. ಆತನು ಹಿಂದಿರುಗಿದ ಕೆಲವು ದಿನಗಳ ಮೇಲೆ ವಿದುರನು ತೀರ್ಥಯಾತ್ರೆಯಿಂದ ಹಿಂದಿರುಗಿದನು. ಭಾರತಯುದ್ಧ ನಡೆಯುವುದಕ್ಕೂ ಪೂರ್ವ ದಲ್ಲಿಯೇ ಆತನು ಪಾಂಡವರಿಗೆ ಅರ್ಧರಾಜ್ಯ ಕೊಡಿಸಬೇಕೆಂಬ ಪ್ರಯತ್ನದಲ್ಲಿ ದುರ್ಯೋಧನನಿಂದ ಅಪಮಾನಿತನಾಗಿ, ತೀರ್ಥಯಾತ್ರೆ ಹೋಗಿದ್ದವನು; ಆ ಕಾಲದಲ್ಲಿ ಆತನು ಮೈತ್ರೇಯ ಪುಷಿಗಳಿಂದ ಉಪದೇಶವನ್ನು ಪಡೆದು ಆತ್ಮಜ್ಞಾನಿಯಾಗಿದ್ದ. ಬಹು ಕಾಲದ ಮೇಲೆ ಹಿಂದಿರುಗಿದ ಈ ಚಿಕ್ಕಪ್ಪನನ್ನು ಕಂಡು ಪಾಂಡವರಿಗೆ ಬಹು ಸಂತೋಷ ವಾಯಿತು. ಧರ್ಮರಾಯನು ಆತನನ್ನು ಉಚಿತ ರೀತಿಯಲ್ಲಿ ಪೂಜಿಸಿ ಗೌರವಿಸಿದನು. ಪಾಂಡವರ ಶ್ರೇಯಸ್ಸನ್ನು ಕಂಡು ವಿದುರನಿಗೆ ಸಂತೋಷವಾಯಿತಾದರೂ, ಮಕ್ಕಳು ಮರಿ ಗಳನ್ನೆಲ್ಲಾ ಕಳೆದುಕೊಂಡು ಕುರುಡ ಕಬೋಜಿಯಾಗಿರುವ ಧೃತರಾಷ್ಟ್ರನನ್ನು ಕಂಡು ‘ಅಯ್ಯೋ’ ಎನಿಸಿತು. ಆ ಮುಪ್ಪಿನ ಮುದುಕನನ್ನು ಕುರಿತು ಆತನು ‘ಮಹಾರಾಜ, ನಿನಗೆ ಮುಪ್ಪು ಬಂದಿತು; ಸಾಯುವ ಕಾಲ ಹತ್ತಿರ ಸಾರಿದೆ. ನೀನು ಮೊದಲೇ ಕುರುಡ, ಈಗ ಕಿವಿ ಕೇಳುವುದಿಲ್ಲ, ಅರಿವು ಮರುಳಾಗಿದೆ, ಹಲ್ಲು ಉದುರಿವೆ, ಆಹಾರ ಜೀರ್ಣಿಸುವುದಿಲ್ಲ, ರೋಗಗಳು ಕಾಣಿಸಿಕೊಂಡಿವೆ. ಆದರೂ ಈ ಅರಮನೆಯಲ್ಲಿ ವಾಸಿಸುತ್ತಿರುವೆ. ನಿನ್ನ ಮಕ್ಕಳನ್ನೆಲ್ಲಾ ಕೊಂದ ಭೀಮನು ಅಸಡ್ಡೆಯಿಂದ ಹಾಕುವ ಕೂಳಿಗೆ ಬಾಲವಲ್ಲಾಡಿಸುತ್ತಾ ಬದುಕಿರುವೆ. ನಿನ್ನವರಿಂದ ಪಾಂಡವರಿಗಾದ ಅನ್ಯಾಯಗಳನ್ನು ಸ್ವಲ್ಪ ಜ್ಞಾಪಿಸಿಕೊ. ಅರಗಿನ ಮನೆಯಲ್ಲಿಟ್ಟು ಬೆಂಕಿ ಹಾಕಿದುದು, ವಿಷದ ಅನ್ನವನ್ನು ಇಕ್ಕಿದುದು, ದ್ರೌಪದಿಯ ಮಾನ ಭಂಗವನ್ನು ಮಾಡಲು ಯತ್ನಿಸಿದುದು, ಕಪಟದ ಜೂಜಿನಲ್ಲಿ ಎಲ್ಲವನ್ನೂ ಕವರಿಕೊಂಡು ಅವರನ್ನು ಕಾಡಿಗಟ್ಟಿದುದು–ಈ ಅನ್ಯಾಯಗಳು ಒಂದೆ, ಎರಡೆ? ಇಂತಹ ಅನ್ಯಾಯಕ್ಕೆ ಒಳಗಾದವರ ಕೈಯಿಂದಲೇ ನೀನು ಹೊಟ್ಟೆಹೊರೆಯುತ್ತ ಬದುಕಿರಬಹುದೆ? ಅಣ್ಣ, ಇನ್ನು ಈ ದುಸ್ಥಿತಿಯನ್ನು ಕೊನೆಗಾಣಿಸು. ಬಾ ನನ್ನೊಡನೆ. ನಡಿ, ಅರಣ್ಯಕ್ಕೆ ಹೋಗೋಣ’ ಎಂದು ಹೇಳಿದನು.

ವಿದುರನ ನುಡಿಗಳಿಂದ ಕುರುಡರಾಜನ ಒಳಗಣ್ಣು ತೆರೆಯಿತು. ಆತನು ಉದ್ಧಾರಕ್ಕೆ ಹೆದ್ದಾರಿಯಂತಿರುವ ಅರಣ್ಯಮಾರ್ಗವನ್ನು ಹಿಡಿದನು. ಮಹಾಪತಿವ್ರತೆಯಾದ ಗಾಂಧಾರಿ ದೇವಿಯೂ ಗಂಡನನ್ನು ಹಿಂಬಾಲಿಸಿದಳು. ಮರುದಿನ ಬೆಳಿಗ್ಗೆ ಧರ್ಮರಾಯನು ಎಂದಿ ನಂತೆ ಪ್ರಾತರಾಹ್ನಿಕಗಳನ್ನು ನೆರವೇರಿಸಿ, ಗುರುವಂದನೆಗೆಂದು ಧೃತರಾಷ್ಟ್ರನ ಅರಮನೆಗೆ ಹೋದನು. ಅದು ಬರಿದಾಗಿತ್ತು. ಅವರು ಎಲ್ಲಿಗೆ ಹೋದರೆಂಬುದೇ ತಿಳಿಯದೆ ಆತನು ಪೇಚಾಡುತ್ತಿರಲು, ದುಃಖಸಮುದ್ರದಲ್ಲಿ ಮುಳುಗಿದ್ದ ಆತನನ್ನು ತೇಲಿಸುವ ನಾವೆಯಂತೆ ನಾರದರು ಅಲ್ಲಿ ಕಾಣಿಸಿಕೊಂಡು ‘ಅಯ್ಯಾ, ಏಕೆ ದುಃಖಿಸುವೆ? ಜನರನ್ನು ಜೊತೆಗೂಡಿಸು ವವನೂ ಭಗವಂತ, ಅಗಲಿಸುವವನೂ ಭಗವಂತ. ಜಗತ್ತು ಸ್ಥಿರವೇ? ಸ್ಥಿರವಾದುದಕ್ಕೆ ನಾಶವೆಲ್ಲಿಯದು? ಇದು ಅಸ್ಥಿರವೆನ್ನುವುದಾದರೆ ಅದಕ್ಕಾಗಿ ಅಳುವುದು ಅವಿವೇಕ. ನಿನ್ನನ್ನು ಅಗಲಿ ಮುಪ್ಪಿನ ಮುದುಕರಾದ ಧೃತರಾಷ್ಟ್ರ ಗಾಂಧಾರಿಯರು ಹೇಗೆ ಬದುಕಬಲ್ಲ ರೆಂದು ಅಳುವೆಯಲ್ಲವೆ? ಅಜ್ಞಾನದಿಂದ ಬಂದಿರುವ ಈ ಮೋಹವನ್ನು ಬಿಡು. ನಿನ್ನ ದೇಹವೇ ನಿತ್ಯವಲ್ಲ, ಇನ್ನು ಉಳಿದವರಿಗಾಗಿ ಏಕೆ ಅಳುವೆ? ಪಂಚಭೂತಗಳಿಂದಾದ ಈ ದೇಹ ಕಾಲ, ಕರ್ಮ, ಗುಣಗಳಿಗೆ ಅಧೀನವಾದುದು. ವಿವೇಕಿಗಳಾದವರು ಅದರ ನಾಶಕ್ಕಾಗಿ ಅಳುವುದಿಲ್ಲ. ಎಲ್ಲರೂ ಮೃತ್ಯುವಿಗೆ ತುತ್ತಾಗುವವರೇ. ಸಾಕ್ಷಾತ್ ಭಗವಂತನೇ ಆಗಿರುವ ಶ್ರೀಕೃಷ್ಣನೂ ಇನ್ನು ಕೆಲವು ದಿನಗಳಲ್ಲಿಯೇ ದೇಹತ್ಯಾಗ ಮಾಡುವನು. ಆದ್ದರಿಂದ ಅಸ್ಥಿರ ವಾದ ದೇಹಕ್ಕಾಗಿ ಅಳುವುದು ಸಲ್ಲದು. ನಿಮ್ಮ ತಂದೆಯಾದ ಧೃತರಾಷ್ಟ್ರನು ಗಾಂಧಾರಿ ವಿದುರರೊಡನೆ ಹಿಮಾಲಯದ ತಪ್ಪಲಿನಲ್ಲಿರುವ ಪುಷ್ಯಾಶ್ರಮದಲ್ಲಿ ನೆಲಸಿ, ಸಕಲೇಂ ದ್ರಿಯ ವಿಷಯಗಳನ್ನು ಗೆದ್ದು ಆತ್ಮಾರಾಮನಾಗಿರುವನು. ಇಂದಿಗೆ ಐದನೆಯ ದಿನ ಆತನ ದೇಹವು ತ್ರೇತಾಗ್ನಿಗೆ ಆಹುತಿಯಾಗುವುದು. ಗಾಂಧಾರಿಯೂ ಅದೇ ಅಗ್ನಿಯನ್ನೇ ಪ್ರವೇಶಿಸಿ ದೇಹತ್ಯಾಗ ಮಾಡುವಳು. ವಿದುರನು ತೀರ್ಥಯಾತ್ರೆ ಹೋಗುವನು’ ಎಂದು ತಿಳಿಸಿದರು. ಮಹಾನುಭಾವರಾದ ಅವರ ಮಾತಿನಿಂದ ಧರ್ಮರಾಯನ ಚಿಂತೆ ದೂರ ವಾಯಿತು.

ಸಂಸಾರದಲ್ಲಿ ಚಿಂತೆಗೆ ಕೊರತೆಯೆ? ಧೃತರಾಷ್ಟ್ರನ ವಿಷಯದಲ್ಲಿ ದೂರವಾದ ಧರ್ಮ ರಾಯನ ಚಿಂತೆ ತಮ್ಮನಾದ ಅರ್ಜುನನ ವಿಚಾರದಲ್ಲಿ ಉಲ್ಬಣಿಸಿತು. ಆತನು ಶ್ರೀಕೃಷ್ಣ ನೊಡನೆ ದ್ವಾರಕೆಗೆ ಹೋಗಿ ಏಳು ತಿಂಗಳಾದುವು. ಆದರೂ ಹಿಂದಿರುಗಲಿಲ್ಲ. ನಾರದರು ಬೇರೆ ಶ್ರೀಕೃಷ್ಣನ ಅಂತ್ಯಕಾಲ ಸನ್ನಿಹಿತವಾಗಿರುವುದಾಗಿ ಹೇಳಿ ಹೋಗಿದ್ದಾರೆ. ಧರ್ಮ ರಾಯನು ಈ ಚಿಂತೆಯಲ್ಲೆ ದಿನಗಳನ್ನು ನೂಕುತ್ತಿರಲು, ಅನೇಕ ಅಪಶಕುನಗಳೂ ಉತ್ಪಾತ ಗಳೂ ಆತನಿಗೆ ಕಾಣಿಸಿಕೊಂಡವು. ಆತನ ಎಡಗಣ್ಣು ಎಡಭುಜಗಳು ಮತ್ತೆ ಮತ್ತೆ ಹಾರಿದವು; ನಾಯಿಯು ವಿಕಾರಧ್ವನಿಯಿಂದ ಕೂಗುತ್ತಾ ಇದಿರಾಯಿತು; ಕಾಗೆ ಗೂಗೆಗಳು ಕರ್ಕಶವಾಗಿ ಕೂಗಿಕೊಂಡವು. ಆನೆ ಕುದುರೆ ಮೊದಲಾದ ವಾಹನಗಳು ಕಣ್ಣೀರುಗರೆದವು, ದಿಕ್ಕುಗಳೆಲ್ಲಾ ಹೊಗೆಯಿಂದ ತುಂಬಿದುವು, ಧೂಳಿನಿಂದ ತುಂಬಿದ ಬಿರುಗಾಳಿ ಬೀಸಿತು, ಮೋಡಗಳಿಂದ ನೆತ್ತರ ಮಳೆ ಸುರಿಯಿತು, ಸೂರ್ಯನು ಕಾಂತಿಹೀನನಾದನು, ಅಗ್ನಿ ಹೋತ್ರದ ಬೆಂಕಿ ಆರಿತು, ದೇವತಾವಿಗ್ರಹಗಳೆಲ್ಲವೂ ಗಳಗಳ ಅತ್ತವು. ಇದನ್ನು ಕಂಡು ಧರ್ಮರಾಯನಿಗೆ ‘ಹೆದರುವವರ ಮೇಲೆ ಕಪ್ಪೆ ಎಸೆ’ದಂತಾಯಿತು.

ಧರ್ಮರಾಯನು ಅಧೀರನಾಗಿ, ಒಳಗೊಳಗೆ ನಡುಗುತ್ತ ಕುಳಿತಿರಲು, ಅರ್ಜುನನು ಅಳು ಮೋರೆಯಿಂದ ಹಿಂದಿರುಗಿದನು. ಅಣ್ಣನನ್ನು ಕಾಣುತ್ತಲೇ ಆತನು ಕಣ್ಣೀರುಗರೆ ಯುತ್ತಾ, ಅಣ್ಣನಿಗೆ ನಮಸ್ಕರಿಸಿ, ನಿಟ್ಟುಸಿರು ಬಿಡುತ್ತಾ ನೆಲದಮೇಲೆ ಕುಳಿತನು. ಅವನ ಅಳುಮೋರೆಯನ್ನು ಕಾಣುತ್ತಲೆ ಧರ್ಮರಾಯನ ಮನಸ್ಸು ತಳಮಳಗೊಂಡಿತು. ಆದರೂ ‘ಏನಪ್ಪ ದ್ವಾರಕೆಯಲ್ಲಿ ಎಲ್ಲರೂ ಕ್ಷೇಮವೆ?’ ಎಂದು ಆತನು ಯಾಂತ್ರಿಕವಾಗಿ ತಮ್ಮ ನನ್ನು ಕೇಳಿದನು. ಹಾಗೆ ಕೇಳಿದುದೇ ತಡ, ಅರ್ಜುನನ ದುಃಖದ ಕಟ್ಟೆ ಒಡೆದುಹೋಯಿತು; ಆತನು ಗಟ್ಟಿಯಾಗಿ ಅತ್ತನು. ಸ್ವಲ್ಪ ಹೊತ್ತಿನ ಮೇಲೆ ಆತನು ಹಾಗೆಯೆ ಸಮಾಧಾನಮಾಡಿ ಕೊಂಡು ಅಣ್ಣನೊಡನೆ ಗದ್ಗದ ಸ್ವರದಿಂದ ‘ಅಣ್ಣ, ಜಗತ್ತಿಗೆ ಸ್ವಾಮಿಯಾದರೂ ನಮಗೆ ಬಂಧುವಿನಂತಿದ್ದ ಶ್ರೀಕೃಷ್ಣ ಪರಮಾತ್ಮ ಇನ್ನಿಲ್ಲ. ಆತನ ಪ್ರಾಣಗಳೊಡನೆ ನನ್ನ ತೇಜಸ್ಸೂ ಬತ್ತಿಹೋಯಿತು. ಜೀವನದ ಉದ್ದಕ್ಕೂ ಆತನೇ ನಮ್ಮ ರಕ್ಷಕನಾಗಿದ್ದ. ಅಣ್ಣ, ದ್ರೌಪದಿಯ ಸ್ವಯಂವರ ಕಾಲದಲ್ಲಿ ನನ್ನ ಮೇಲೆ ಬಂದು ಬಿದ್ದ ರಾಜರುಗಳನ್ನೆಲ್ಲಾ ದಿಗ್ಬಲಿ ಹಾಕಿದುದು ಯಾರ ಕೃಪೆಯಿಂದ? ಅಗ್ನಿಗೆ ಖಾಂಡವವನವನ್ನು ಆಹುತಿಯಾಗಿ ಕೊಟ್ಟಾಗ, ನನ್ನನ್ನು ಇದಿರಿಸಿದ ದೇವೇಂದ್ರನನ್ನು ಗೆದ್ದುದು ಯಾರ ಬಲದಿಂದ? ರಾಜ ಸೂಯಯಾಗದಲ್ಲಿ ಜಗತ್ತಿನ ರಾಜರೆಲ್ಲರೂ ನಿನಗೆ ಮಣಿಯುವಂತೆ ಮಾಡಿದುದು ಯಾರ ಮಹಿಮೆಯಿಂದ? ಅಣ್ಣನಾದ ಭೀಮಸೇನನು ಜರಾಸಂಧನನ್ನು ಕೊಂದುದು ಯಾರ ಕರುಣೆಯಿಂದ? ದ್ರೌಪದಿಯು ಸೆಳೆಸೀರೆಗೆ ಸಿಕ್ಕಾಗ ಅವಳ ಮಾನವನ್ನು ಕಾಪಾಡಿದ ಮಹಾನುಭಾವನಾರು? ಕಿರಾತರೂಪಿನಿಂದ ಕಾಣಿಸಿಕೊಂಡ ಸಾಕ್ಷಾತ್ ಪರಮೇಶ್ವರನೊಡನೆ ನಾನು ಸಮಸಮವಾಗಿ ಯುದ್ಧಮಾಡಿ ಪಾಶುಪತಾಸ್ತ್ರವನ್ನು ಪಡೆದುದು ಯಾರ ದಯೆ ಯಿಂದ? ಇಂದ್ರನ ಅರ್ಧಾಸನವನ್ನು ಏರಿ ವೈಭವದಿಂದ ಮೆರೆಯುವಂತೆ ಮಾಡಿದ ಕೃಪಾಳು ಯಾರು? ಉತ್ತರಗೋಗ್ರಹಣದಲ್ಲಿ ನಾನೋರ್ವನೆ ಕುರುಸೇನೆಯನ್ನು ಗೆಲ್ಲುವ ಸಾಹಸವನ್ನು ನನಗೆ ಕೊಟ್ಟವರು ಯಾರು? ಕುರುಕ್ಷೇತ್ರದ ರಣರಂಗದಲ್ಲಿ ನನಗೆ ಸಾರಥಿ ಯಾಗಿ ಕುಳಿತು ನನ್ನಿಂದ ಎಲ್ಲರನ್ನೂ ಧ್ವಂಸಮಾಡಿಸಿದ ಆ ಮಹಾಪ್ರಭು ಯಾರು? ಆ ಶ್ರೀಕೃಷ್ಣ ಪರಮಾತ್ಮ ಇನ್ನು ಇಲ್ಲವಣ್ಣ. ಜಯದ್ರಥನನ್ನು ಸೂರ್ಯಾಸ್ತದೊಳಗಾಗಿ ಕೊಲ್ಲುವೆನೆಂಬ ನನ್ನ ಅಸಾಧ್ಯ ಪ್ರತಿಜ್ಞೆಯನ್ನು ಸಾಧ್ಯವಾಗುವಂತೆ ಮಾಡಿದ ಮಹಾ ಮಹಿಮ ನಮ್ಮನ್ನು ಅಗಲಿ ಹೋದ. ಅಣ್ಣ, ಸಾಕ್ಷಾತ್ ಲಕ್ಷ್ಮೀಪತಿಯಾದ ಆತನು ನನ್ನನ್ನು ಅತ್ಯಂತ ಪ್ರೀತಿಯಿಂದ ‘ಅಯ್ಯಾ ಪಾರ್ಥ, ಹೇ ಅರ್ಜುನ, ಎಲೋ ಗೆಳೆಯ’ ಇತ್ಯಾದಿ ಯಾಗಿ ಮಾತನಾಡಿಸುತ್ತಿದ್ದುದನ್ನು ನೆನೆಸಿಕೊಂಡರೆ ಸಂಕಟದಿಂದ ನನ್ನ ಹೊಟ್ಟೆ ಉರಿದು ಹೋಗುತ್ತದೆ. ತಿರುಗಾಟ, ಊಟ, ನಿದ್ರೆ, ಹಾಸ್ಯಗಳಲ್ಲಿ ಸದಾ ಜೊತೆಯಲ್ಲಿರುತ್ತಿದ್ದ ಆತನನ್ನು, ಮೂಢನಾದ ನಾನು ಗೆಳೆಯನೆಂದೇ ಭಾವಿಸಿ, ಅನೇಕ ವೇಳೆ ಹಾಸ್ಯ ಮಾಡು ತ್ತಿದ್ದೆ; ಕೆಲವು ವೇಳೆ ‘ಅಹಹ ನೀನು ಬಹು ಸತ್ಯವಂತ, ಮೋಸವೆಂಬುದು ನಿನ್ನ ಹತ್ತಿರಕ್ಕೇ ಬರದು’ ಎಂದು ಹಂಗಿಸುತ್ತಿದ್ದೆ. ತಂದೆ ಮಗನ ತಪ್ಪನ್ನು ಕ್ಷಮಿಸುವಂತೆ ಆತನು ಎಲ್ಲವನ್ನೂ ಕ್ಷಮಿಸುತ್ತಿದ್ದ. ಅಣ್ಣ, ದೇವರ ದೇವನಾಗಿ, ನಮ್ಮ ಬಂಧುವಾಗಿ, ಮಿತ್ರನಾಗಿ ಇದ್ದ ವಾಸುದೇವನು ನಮ್ಮ ಅಗಲಿ ಹೋದನು. ಅಣ್ಣ, ನಾನೇನು ಹೇಳಲಿ? ನನ್ನ ಬಿಲ್​ದನಿಗೆ ಬಾಗದ ಶೂರನಿರಲಿಲ್ಲ, ಕೃಷ್ಣನಿರುವವರೆಗೆ. ಈಗ ಅದೇ ಬಿಲ್ಲು, ಅದೇ ಬಾಣ ಗಳು, ಅದೇ ರಥ, ಅದೇ ಕುದುರೆಗಳು ಇವೆ. ಆದರೇನು ಪ್ರಯೋಜನ? ಶ್ರೀಕೃಷ್ಣನ ರಾಣಿವಾಸದವರನ್ನು ಇಲ್ಲಿಗೆ ಕರೆತರಬೇಕೆಂದು ಹೊರಟ ನಾನು ದಾರಿಯಲ್ಲಿ ದನ ಕಾಯು ವವರಿಗೆ ಸೋತು ಓಡಿಬಂದೆ. ದ್ವಾರಕೆಯಲ್ಲಿದ್ದ ಶೂರರೆಲ್ಲರೂ ಪರಸ್ಪರ ಹೊಡೆದಾಡಿ ಮಡಿದರು. ದ್ವಾರಕೆ ಈಗ ಶ್ಮಶಾನವಾಗಿದೆ’ ಎಂದು ಹೇಳಿದನು.

ಶ್ರೀಕೃಷ್ಣನ ನಾಮಾಮೃತವನ್ನು ಮತ್ತೆ ಮತ್ತೆ ಸವಿದುದರಿಂದ ಅರ್ಜುನನ ಕಳವಳ ಕಡಮೆಯಾಯಿತು. ಶ್ರೀಕೃಷ್ಣನು ಬೋಧಿಸಿದ್ದ ಗೀತೆಯು ಆತನ ಜ್ಞಾಪಕಕ್ಕೆ ಬಂತು; ಆತನು ವೈರಾಗ್ಯಪರನಾದನು. ಶ್ರೀಕೃಷ್ಣನು ಪರಂಧಾಮಕ್ಕೆ ತೆರಳಿದುದನ್ನೂ, ಯಾದವರು ನಷ್ಟ ರಾದುದನ್ನೂ ಕೇಳಿ ಧರ್ಮರಾಯನಿಗೂ ವೈರಾಗ್ಯ ಉದಿಸಿತು. ಶ್ರೀಕೃಷ್ಣನು ದಿವಂಗತನಾದ ದಿನವೇ ಕಲಿಯು ಭೂಲೋಕವನ್ನು ಹೊಕ್ಕಿರುವನೆಂಬುದು ಆತನಿಗೆ ಅರಿವಾಯಿತು. ಆದ್ದರಿಂದ ಆತನು ಹಸ್ತಿನಾವತಿಯಲ್ಲಿ ತನ್ನ ಮೊಮ್ಮಗನಾದ ಪರೀಕ್ಷಿತನಿಗೂ, ಮಧುರೆ ಯಲ್ಲಿ ಶ್ರೀಕೃಷ್ಣನ ಮೊಮ್ಮಗನಾದ ವಜ್ರನಿಗೂ ಪಟ್ಟವನ್ನು ಕಟ್ಟಿ, ತಾನು ಸರ್ವಸಂಗಪರಿ ತ್ಯಾಗ ಮಾಡಿದವನಾದನು. ನಾರುಮಡಿಯನ್ನುಟ್ಟು, ಮೌನವನ್ನು ತೊಟ್ಟು, ಆಹಾರವನ್ನು ಬಿಟ್ಟು, ಮನಸ್ಸಿನಲ್ಲಿ ಈಶ್ವರನನ್ನು ಅನುಸಂಧಾನ ಮಾಡುತ್ತಾ ಆತನು ಒಂಟಿಯಾಗಿ ಉತ್ತರದಿಕ್ಕಿಗೆ ಪ್ರಯಾಣಮಾಡಿದನು. ಆತನ ತಮ್ಮಂದಿರೂ ಆತನನ್ನು ಅನುಸರಿಸಿದರು. ದ್ರೌಪದಿಯೂ ಶ್ರೀಕೃಷ್ಣನ ಪಾದಧ್ಯಾನದಲ್ಲಿ ತತ್ಪರಳಾಗಿ ಅವರನ್ನು ಅನುಸರಿಸಿದಳು.

ಅತ್ತ ಪಾಂಡವರು ಮಹಾಪ್ರಸ್ಥಾನಕ್ಕೆ ಪ್ರಯಾಣಮಾಡಿದ ಮೇಲೆ, ಇತ್ತ ಪರೀಕ್ಷಿ ದ್ರಾಜನು ಧರ್ಮದಿಂದ ರಾಜ್ಯಭಾರಮಾಡುತ್ತಾ, ತನ್ನ ಸೋದರಮಾವನಾದ ಉತ್ತರನ ಮಗಳು ಇರಾವತಿಯನ್ನು ಮದುವೆಯಾಗಿ, ಆಕೆಯಲ್ಲಿ ಜನಮೇಜಯನೇ ಮೊದಲಾದ ನಾಲ್ವರು ಪುತ್ರರನ್ನು ಪಡೆದನು. ಕೃಪಾಚಾರ್ಯನು ಆತನ ಪರೋಹಿತನಾಗಿದ್ದನು. ಈ ಪುರೋಹಿತನ ಸಹಾಯದಿಂದ ಪರೀಕ್ಷಿದ್ರಾಜನು ಮೂರು ಅಶ್ವಮೇಧಯಾಗಗಳನ್ನು ಮಾಡಿದನು. ಈ ಯಾಗಗಳಲ್ಲಿ ಇಂದ್ರನೇ ಮೊದಲಾದ ದೇವತೆಗಳು ಪ್ರತ್ಯಕ್ಷರಾಗಿ ಬಂದು ಹವಿಸ್ಸನ್ನು ಸ್ವೀಕರಿಸಿದರು. ಇಂತಹ ಮಹಿಮೆಯುಳ್ಳ ಪರೀಕ್ಷಿದ್ರಾಜನಿಗೆ ತನ್ನ ರಾಜ್ಯದಲ್ಲಿ ಕಲಿಯು ಪ್ರವೇಶಿಸಿರುವನೆಂಬ ಸುದ್ದಿ ಬಂದಿತು. ಇದರಿಂದ ಕೋಪಗೊಂಡ ರಾಜನು ಅವನನ್ನು ಹೋಡೆದೋಡಿಸಬೇಕೆಂದು ಚತುರಂಗಸೇನೆಯೊಡನೆ ದಿಗ್ವಿಜಯ ಹೊರಟನು. ಆತನ ಪ್ರಯಾಣ ಮಧ್ಯದಲ್ಲಿ ಒಂದು ಆಶ್ಚರ್ಯ ಕಾಣಿಸಿತು. ಗೋರೂಪನ್ನು ಧರಿಸಿದ ಭೂದೇವಿ ಅಳುತ್ತಾ ನಿಂತಿರಲು, ವೃಷಭರೂಪಿನ ಧರ್ಮ ಪುರುಷನು ಒಂಟಿ ಕಾಲಿ ನಲ್ಲಿ ಕುಂಟುತ್ತಾ ಆಕೆಯ ಬಳಿಗೆ ಬಂದು ‘ಭದ್ರೆ, ನಿನ್ನ ಕಣ್ಣೀರನ್ನು ಕಂಡು ನನ್ನ ಮನಸ್ಸು ಕರಗುತ್ತಿದೆ. ನಿನಗೆ ಬಂದಿರುವ ಸಂಕಟ ಎಂತಹುದು? ಇನ್ನು ಮುಂದೆ ಕಲಿಯ ಮಹಿಮೆ ಯಿಂದ ನಡೆಯಬಹುದಾದ ಅನ್ಯಾಯಗಳನ್ನು ನೆನೆದು ಸಂಕಟಪಡುತ್ತಿರುವೆಯಾ?’ ಎಂದು ಕೇಳಿದನು. ಆಗ ಭೂದೇವಿಯು ‘ಅಪ್ಪ, ನನ್ನ ದುಃಖಕ್ಕೆ ಕಾರಣ ನಿನಗೂ ಗೊತ್ತು. ಸಕಲ ಕಲ್ಯಾಣಗುಣಪರಿಪೂರ್ಣನಾದ ಶ್ರೀಕೃಷ್ಣನು ಈ ಲೋಕವನ್ನು ಬಿಟ್ಟು ಹೋದನು; ಪಾಪಗಳಿಗೆ ನೆಲೆಮನೆಯಾದ ಕಲಿಯು ಈ ಜಗತ್ತನ್ನು ಪ್ರವೇಶಿಸಿರುವನು. ಆದ್ದರಿಂದ ನಾನು ನಿನಗಾಗಿಯೂ, ಮಹರ್ಷಿಗಳಿಗಾಗಿಯೂ, ಪಿತೃ, ದೇವ, ಬ್ರಾಹ್ಮಣರಿಗಾಗಿಯೂ, ವರ್ಣಾಶ್ರಮಧರ್ಮಗಳಿಗಾಗಿಯೂ ದುಃಖಿಸುತ್ತಿರುವೆನು. ಶ್ರೀಕೃಷ್ಣನ ಪಾದಪದ್ಮಗಳಿಂದ ಪುಲಕಿತಳಾಗುತ್ತಿದ್ದ ನಾನು ಆತನ ಅಗಲಿಕೆಯನ್ನು ಹೇಗೆ ಸಹಿಸಲಿ?’ ಎಂದು ಹೇಳಿ ಕಣ್ಣೀರುಗರೆದಳು.

ಪರಸ್ಪರ ಸಂಭಾಷಿಸುತ್ತಿದ್ದ ಭೂದೇವಿ-ಧರ್ಮಪುರುಷರ ಬಳಿಗೆ ಪರೀಕ್ಷಿದ್ರಾಜನು ಬರುವ ಹೊತ್ತಿಗೆ ಸರಿಯಾಗಿ ರಾಜವೇಷವನ್ನು ಧರಿಸಿದ್ದ ಶೂದ್ರನೊಬ್ಬನು ಅವುಗ ಳೆರಡನ್ನೂ ಕೋಲಿನಿಂದಲೂ ತನ್ನ ಕಾಲಿನಿಂದಲೂ ಹೊಡೆದು ಹಿಂಸಿಸುತ್ತಿದ್ದನು. ಆ ಪ್ರಾಣಿಗಳೆರಡೂ ಗಡಗಡ ನಡುಗುತ್ತಾ ನಿಂತಿದ್ದವು. ಇದನ್ನು ಕಂಡು ಪರೀಕ್ಷಿದ್ರಾಜನ ಕೋಪ ಕೆರಳಿತು. ಆತನು ಕೆಂಗಣ್ಣುಗಳಿಂದ ಈ ಕ್ರೂರಿಯನ್ನು ಕೆಕ್ಕರಿಸಿ ನೋಡುತ್ತಾ, ‘ಎಲಾ ಪಾಪಿ, ನೀನಾರು? ಈ ಪ್ರಾಣಿಗಳನ್ನು ಏಕೆ ಹಿಂಸಿಸುತ್ತಿರುವೆ? ನಿನ್ನ ಈ ಅಪರಾಧ ಕ್ಕಾಗಿ ನಿನಗೆ ಮರಣ ದಂಡನೆ ವಿಧಿಸುವೆನು’ ಎಂದು ಗದರಿಸಿ, ಆ ಪ್ರಾಣಿಗಳಿಗೆ ಅಭಯ ವನ್ನು ನೀಡಿದನು. ಆಗ ಧರ್ಮಪುರುಷನು ಪರೀಕ್ಷಿತನನ್ನು ಕುರಿತು ‘ಮಹಾರಾಜ, ಪಾಂಡವ ವಂಶದಲ್ಲಿ ಹುಟ್ಟಿರುವ ನೀನು ನಿನ್ನ ವಂಶಕ್ಕೆ ತಕ್ಕ ಮಾತುಗಳನ್ನಾಡಿದೆ. ಇಂತಹ ಸದ್ಗುಣಗಳು ಇಲ್ಲದಿದ್ದರೆ ಶ್ರೀಕೃಷ್ಣ ಪರಮಾತ್ಮನು ನಿಮಗೆ ಊಳಿಗವನ್ನು ಮಾಡುತ್ತಿ ದ್ದನೆ? ಅದಿರಲಿ, ಇಂದಿನ ನಮ್ಮ ದುಸ್ಥಿತಿಗೆ ಯಾರೂ ಕಾರಣರಲ್ಲ. ಸುಖದುಃಖಗಳಿಗೆಲ್ಲಾ ಕಾರಣನಾದವನು ಒಬ್ಬನೇ, ಅವಾಙ್ಮಾನಸಗೋಚರನಾದ ಆ ಪರಮೇಶ್ವರ’ ಎಂದು ಹೇಳಿ ದನು. ಅದನ್ನು ಕೇಳಿ ಸಂತೋಷಗೊಂಡ ಪರೀಕ್ಷಿದ್ರಾಜನು ಧರ್ಮಪುರುಷನೊಡನೆ ‘ಮಹಾನುಭಾವ, ಕೃತಯುಗದಲ್ಲಿ ನಾಲ್ಕು ಪಾದಗಳೂ ನೆಟ್ಟಗಿದ್ದ ನೀನು ಅಧರ್ಮಗಳು ಹೆಚ್ಚಿದಂತೆಲ್ಲಾ ಒಂದೊಂದು ಪಾದವನ್ನು ಕಳೆದುಕೊಂಡು ಈಗ ಏಕ ಪಾದನಾಗಿರುವೆ. ಕಲಿರೂಪನಾದ ಅಧರ್ಮನು ಅದನ್ನು ಮುರಿಯಲು ಯತ್ನಿಸುತ್ತಿರುವನು. ನಿನಗೂ ಭೂದೇವಿಗೂ ಈಗ ಬಂದಿರುವ ದುಃಖವನ್ನು ನಾನು ಹೋಗಲಾಡಿಸುತ್ತೇನೆ’ ಎಂದು ಹೇಳಿ, ಬಳಿಯಲ್ಲಿಯೇ ನಿಂತಿದ್ದ ಕಲಿಯನ್ನು ಕತ್ತರಿಸಿ ಹಾಕಬೇಕೆಂದು ಕತ್ತಿಯನ್ನು ಒರೆ ಯಿಂದ ಹೊರಗೆಳೆದನು. ಕೂಡಲೇ ಕಲಿಯು ಗಡಗಡ ನಡುಗುತ್ತಾ ತನ್ನ ಕಿರೀಟವನ್ನು ರಾಜನ ಪದತಲದಲ್ಲಿಟ್ಟು ಆತನಿಗೆ ಅಡ್ಡಬಿದ್ದನು. ದಯಾಳುವಾದ ಪರೀಕ್ಷಿದ್ರಾಜನು ಕಲಿಗೆ ಜೀವ ದಾನವನ್ನು ನೀಡಿ, ಒಡನೆಯೇ ತನ್ನ ರಾಜ್ಯವನ್ನು ಬಿಟ್ಟು ಹೊರಟುಹೋಗು ವಂತೆ ಅಪ್ಪಣೆ ಮಾಡಿದನು. ಕಲಿಪುರುಷನು ರಾಜನ ಅಪ್ಪಣೆಯಂತೆ ನಡೆದುಕೊಳ್ಳುವುದಾಗಿ ತಿಳಿಸಿ, ‘ಅಯ್ಯಾ, ನಾನು ನೆಲಸುವುದಕ್ಕೆ ಯಾವುದಾದರೂ ಒಂದು ಸ್ಥಳವನ್ನು ತೋರಿಸು’ ಎಂದು ಕೇಳಿಕೊಂಡನು. ರಾಜನು ‘ಅಯ್ಯಾ ಕಲಿಪುರುಷ, ನೀನು ಸುಳ್ಳಿಗೆ ತವರುಮನೆ ಯಾದ ಜೂಜು, ದಯೆಗೆ ವೈರಿಯಾದ ಮದ್ಯ, ಶುದ್ಧ ನಡತೆಯನ್ನು ಹಾಳುಮಾಡುವ ಹೆಣ್ಣು, ಎಲ್ಲ ದೋಷಗಳಿಗೂ ಗಣಿಯಾದ ಹಿಂಸೆ–ಈ ನಾಲ್ಕರಲ್ಲಿ ನೆಲಸು’ ಎಂದನು. ಕಲಿಯು ‘ನಾಲ್ಕು ಅಧರ್ಮಗಳೂ ನೆಲೆಸಿರುವ ಒಂದೇ ಸ್ಥಳವನ್ನು ತೋರಿಸು’ ಎಂದು ಕೇಳಿಕೊಳ್ಳಲು, ರಾಜನು ಮೇಲಿನ ನಾಲ್ಕು ಅಧರ್ಮಗಳಲ್ಲದೆ ವೈರವೆಂಬ ಐದನೆಯ ಅಧರ್ಮಕ್ಕೂ ನೆಲೆವನೆಯಾದ ಸುವರ್ಣವನ್ನು ತೋರಿಸಿದನು. ಕಲಿಯು ಆತನು ತೋರಿದ ನೆಲೆಗಳನ್ನು ಸೇರಿ, ಅಲ್ಲಿ ವಿಹರಿಸುತ್ತಿದ್ದನು.

ಕಲಿಯನ್ನು ನಿಗ್ರಹಿಸಿದ ಮಹಾನುಭಾವನೆಂದು ಪ್ರಸಿದ್ಧನಾದ ಪರೀಕ್ಷಿದ್ರಾಜನು ಒಂದು ದಿನ ಬೇಟೆಗೆಂದು ಅಡವಿಗೆ ಹೋದನು. ಬಹುಕಾಲ ಮೃಗಗಳನ್ನು ಅಟ್ಟಿಕೊಂಡು ಹೋಗಿ ಬೇಟೆಯಾಡಿದ ಬಳಿಕ ಆತನು ಬಳಲಿ, ಬಾಯಾರಿ, ನೀರನ್ನು ಹುಡುಕುತ್ತಾ ಶಮೀಕ ಪುಷಿಯ ಆಶ್ರಮವನ್ನು ಹೊಕ್ಕನು. ಆ ಸಮಯದಲ್ಲಿ ಪುಷಿಯು ತಪೋಮಗ್ನನಾಗಿ ಸಮಾಧಿಯಲ್ಲಿದ್ದನು. ಸ್ಥಾಣುವಿನಂತೆ ಕುಳಿತಿದ್ದ ಆತನ ಇದಿರಿನಲ್ಲಿ ನಿಂತುಕೊಂಡು ನೀರನ್ನು ಕೊಡುವಂತೆ ರಾಜನು ಆತನನ್ನು ಬೇಡಿದನು. ಬಾಹ್ಯಪ್ರಜ್ಞೆಯಿಲ್ಲದ ಪುಷಿಯು ಮೌನದಿಂದಿರಲು, ರಾಜನು ಅದನ್ನು ಅರ್ಥ ಮಾಡಿಕೊಳ್ಳದೆ, ಪುಷಿಯು ಬೇಕೆಂದೇ ತನಗೆ ಅವಮಾನಮಾಡುತ್ತಿರುವನೆಂದು ಬಗೆದು, ರೋಷದಿಂದ ಬುಸುಗುಟ್ಟುತ್ತಾ ಹಿಂದಿರುಗಿ, ಅಲ್ಲಿಯೇ ಸತ್ತು ಬಿದ್ದಿದ್ದ ಒಂದು ಹಾವನ್ನು ಬಿಲ್ಲಿನ ಕೊನೆಯಲ್ಲಿ ಎತ್ತಿ ಪುಷಿಯ ಕೊರಳಲ್ಲಿ ಹಾಕಿ, ತನ್ನ ರಾಜಧಾನಿಗೆ ಹಿಂದಿರುಗಿದನು. ರಾಜನು ಅತ್ತ ಹೋಗುತ್ತಲೆ ಇತ್ತ ಶಮೀಕ ಪುಷಿಯ ಪುತ್ರನಾದ ಶೃಂಗಿಯು ಅಲ್ಲಿಗೆ ಬಂದನು. ಆತನು ತನ್ನ ತಂದೆಗಾದ ಅಪಮಾನವನ್ನು ಕಂಡು ಅತ್ಯಂತ ಕೋಪದಿಂದ ‘ಧರ್ಮವನ್ನು ಅತಿಕ್ರಮಿಸಿ, ನನ್ನ ತಂದೆಗೆ ಅಪಮಾನಮಾಡಿದ ದ್ರೋಹಿಯು ಇಂದಿನಿಂದ ಏಳು ದಿನಗಳೊಳಗಾಗಿ ಸರ್ಪದಷ್ಟನಾಗಿ ಸಾಯಲಿ’ ಎಂದು ಶಪಿಸಿ, ಗಟ್ಟಿಯಾಗಿ ಅಳುತ್ತಾ ನಿಂತನು. ಸಮಾಧಿಯಿಂದ ಕೆಳಕ್ಕಿಳಿದ ಶಮೀಕ ಪುಷಿಯು, ಕೊರಳಲ್ಲಿದ್ದ ಹಾವನ್ನೂ ಅಳುತ್ತಿದ್ದ ಮಗನನ್ನೂ ಕಂಡು, ನಡೆದ ಸಮಾಚಾರವನ್ನೆಲ್ಲಾ ಮಗನಿಂದ ತಿಳಿದವನಾಗಿ, ಬಹಳವಾಗಿ ಮರುಗುತ್ತಾ ‘ಅಯ್ಯಾ, ಪರೀಕ್ಷಿದ್ರಾಜನು ಸಜ್ಜನನಾದ ಧರ್ಮಿಷ್ಠ. ಆತನ ಸಣ್ಣತಪ್ಪಿಗೆ ಎಂತಹ ಭಯಂಕರವಾದ ಶಿಕ್ಷೆ! ರಾಜನೆಂದರೆ ವಿಷ್ಣುವಿನ ಅಂಶ; ಆತನ ದೆಸೆಯಿಂದಲೇ ಜಗತ್ತಿನಲ್ಲಿ ಶಾಂತಿ. ಅರಾಜಕವಾದ ರಾಜ್ಯದಲ್ಲಿ ವರ್ಣಾಶ್ರಮ ಧರ್ಮಗಳಿಗೂ ಸಜ್ಜನರಿಗೂ ಉಳಿಗಾಲವುಂಟೆ? ನೀನು ಎಂತಹ ಅಕಾರ್ಯವನ್ನು ಮಾಡಿದೆ? ನಿನ್ನ ಹುಡುಗಬುದ್ಧಿಗೆ ಧಿಕ್ಕಾರ!’ ಎಂದು ಮಗ ನನ್ನು ಛೀಗುಟ್ಟಿದನು. ಏನಾದರೇನು? ಕಾರ್ಯ ಕೈಮಿಂಚಿಹೋಗಿತ್ತು.

ಇತ್ತ ಪರೀಕ್ಷಿದ್ರಾಜನು ಶಮೀಕಪುಷಿಗೆ ತನ್ನಿಂದಾದ ಅಪಚಾರವನ್ನೇ ಮನಸ್ಸಿನಲ್ಲಿ ತಿರುವಿಹಾಕುತ್ತಾ ‘ಅಯ್ಯೋ ತುಂಟಹುಡುಗನಂತೆ ನಾನು ಮಾಡಬಾರದುದನ್ನು ಮಾಡಿದೆ ನಲ್ಲಾ! ದೇವ ಬ್ರಾಹ್ಮಣರಿಗೆ ಮಾಡಿದ ಅಪರಾಧ ಸಾಂಕ್ರಾಮಿಕದಂತೆ ವಂಶಪಾರಂಪರ್ಯ ವಾಗಿ ಹಬ್ಬುವುದಲ್ಲಾ! ಇದಕ್ಕೆ ನಿಷ್ಕೃತಿಯೆಲ್ಲಿ?’ ಎಂದು ಪಶ್ಚಾತ್ತಾಪ ಪಡುತ್ತಿರಲು, ಶೃಂಗಿಯು ತನ್ನನ್ನು ಶಪಿಸಿದ ವೃತ್ತಾಂತ ಆತನಿಗೆ ಗೊತ್ತಾಯಿತು. ರಾಜನು ಅದಕ್ಕಾಗಿ ವ್ಯಥೆಪಡಲಿಲ್ಲ; ತನಗೆ ವೈರಾಗ್ಯ ಹುಟ್ಟಲು ಸಕಾರಣವೇ ದೊರೆಯಿತೆಂದು ಆತನು ಸಮಾ ಧಾನಗೊಂಡನು. ಇಹಲೋಕ ಪರಲೋಕಗಳೆರಡೂ ನಶ್ವರವೆಂದು ನಿಶ್ಚಯಿಸಿದ ಆತನು, ಭಗವಂತನ ಪಾದಸೇವೆಯೇ ಸಕಲ ಪುರುಷಾರ್ಥಗಳಿಗಿಂತಲೂ ಶ್ರೇಷ್ಠವಾದುದೆಂದು ಬಗೆದು, ನಿರಶನವ್ರತದಿಂದ ದೇವಗಂಗೆಯಲ್ಲಿ ಕುಳಿತನು. ಹಾಗೆ ಕುಳಿತು ಆತನು ಭಗವಂತನ ಪಾದಕಮಲಗಳನ್ನು ಧ್ಯಾನಿಸುತ್ತಿರಲು ಅತ್ರಿ, ವಸಿಷ್ಠ, ವಿಶ್ವಾಮಿತ್ರ, ಭೃಗು, ಪರಶುರಾಮ ಮೊದಲಾದ ಪುಷಿಗಳೆಲ್ಲರೂ ಅಲ್ಲಿಗೆ ಬಂದರು. ಪರೀಕ್ಷಿದ್ರಾಜನು ಅವ ರೆಲ್ಲರನ್ನೂ ಉಚಿತ ರೀತಿಯಲ್ಲಿ ಸತ್ಕರಿಸಿ ಪೂಜಿಸಿದನು. ಅನಂತರ ಆತನು ಅವರೆಲ್ಲ ರನ್ನೂ ಕುರಿತು ‘ಪೂಜ್ಯರೆ, ಒಂದೇ ಮನಸ್ಸಿನಿಂದ ಭಗವಂತನನ್ನು ಧ್ಯಾನಿಸುತ್ತಿರುವ ನಾನು ನಿಮಗೆ ಶರಣಾಗತನಾಗಿದ್ದೇನೆ. ನೀವೆಲ್ಲರೂ ಭಗವಂತನ ನಾಮಾವಳಿಗಳನ್ನು ಗಾನ ಮಾಡಿರಿ. ನಾನು ಮುಂದೆ ಯಾವ ಜನ್ಮವನ್ನೇ ಪಡೆಯಲಿ, ಭಗವಂತನಲ್ಲಿಯೂ ಭಗವ ದ್ಭಕ್ತರಲ್ಲಿಯೂ ನನಗೆ ನಿರಂತರವಾದ ಪ್ರೇಮವಿರುವಂತೆ ನನ್ನನ್ನು ಆಶೀರ್ವದಿಸಿರಿ’ ಎಂದು ಕೇಳಿಕೊಂಡನು. ಪುಷಿಗಳು ‘ತಥಾಸ್ತು’ ಎಂದು ಆಶೀರ್ವಾದ ಮಾಡಿದರು.

ಪುಷಿಗಳ ಆಶೀರ್ವಾದವನ್ನು ಪಡೆದ ಪರೀಕ್ಷಿತನು ತನ್ನ ಮಗನಾದ ಜನಮೇಜಯನಿಗೆ ರಾಜ್ಯವನ್ನು ವಹಿಸಿಕೊಟ್ಟು, ಗಂಗೆಯ ದಕ್ಷಿಣದಲ್ಲಿ ದರ್ಭೆಗಳನ್ನು ಹಾಸಿಕೊಂಡು ಪ್ರಾಯೋಪವೇಶಕ್ಕಾಗಿ ಕುಳಿತನು. ಪುಷಿಗಳೆಲ್ಲರೂ ಆತನ ಸುತ್ತ ನೆರೆದು, ಭಗವಂತನ ನಾಮಸಂಕೀರ್ತನೆಯನ್ನು ಮಾಡುತ್ತ ಕುಳಿತರು. ಆ ವೇಳೆಗೆ ಸರಿಯಾಗಿ ಪರಮಜ್ಞಾನಿಯಾದ ಶುಕಮಹರ್ಷಿ ಅಲ್ಲಿಗೆ ಆಗಮಿಸಿದನು. ಆ ಮಹಾನುಭಾವನನ್ನು ಕಾಣುತ್ತಲೆ ನೆರೆದಿದ್ದ ಪುಷಿಗಳೆಲ್ಲರೂ ದಿಗ್ಗನೆ ಮೇಲಕ್ಕೆದ್ದು ಗೌರವಿಸಿದರು. ಪರೀಕ್ಷಿದ್ರಾಜನು ಆತನಿಗೆ ಕೈ ಜೋಡಿಸಿ ಅಡ್ಡ ಬಿದ್ದನು; ಅನಂತರ ಆತನನ್ನು ಕುರಿತು ‘ಸ್ವಾಮಿ, ನಿನ್ನ ದರ್ಶನದಿಂದ ನಾನು ಧನ್ಯನಾದೆ. ಮರಣಸನ್ನಿಹಿತವಾಗಿರುವ ನಾನು ನಿನ್ನ ಉಪದೇಶವನ್ನು ಬೇಡುವೆನು. ನಾನೀಗ ಏನು ಮಾಡಲಿ? ಏನನ್ನು ಜಪಿಸಲಿ, ಧ್ಯಾನಮಾಡಲಿ? ನನ್ನ ಕರ್ತವ್ಯವೇನೆಂಬು ದನ್ನು ತಿಳಿಸಿ, ನನ್ನನ್ನು ಉದ್ಧರಿಸು’ ಎಂದು ಬೇಡಿಕೊಂಡನು. ಆಗ ಶುಕಮುನಿಯು, ಆತನೊಡನೆ ‘ಅಯ್ಯಾ, ನಿನ್ನ ಪ್ರಶ್ನೆಯಿಂದ ನಿನಗೆ ಮಾತ್ರವೇ ಅಲ್ಲ, ಲೋಕಕ್ಕೆ ಹಿತವಾಗು ತ್ತದೆ. ಮನುಷ್ಯನು ಕೇಳಿ, ತಿಳಿದು, ನಡೆದುಕೊಳ್ಳಬೇಕಾದ ವಿಷಯಗಳಲ್ಲೆಲ್ಲಾ ಇದೇ ಅತ್ಯಂತ ಶ್ರೇಷ್ಠವಾದುದು. ಸಾಮಾನ್ಯವಾಗಿ ಲೋಕದ ಜನ ಹಗಲನ್ನು ತಮ್ಮ ಕುಟುಂಬ ಪೋಷಣೆಯ ಚಿಂತೆಯಲ್ಲಿಯೂ ರಾತ್ರಿಯನ್ನು ವಿಷಯಸುಖ, ನಿದ್ರೆಗಳಲ್ಲಿಯೂ ಕಳೆಯು ತ್ತಾರೆ; ಆತ್ಮಜ್ಞಾನದಲ್ಲಿ ಅವರಿಗೆ ಆಸಕ್ತಿಯೇ ಇರುವುದಿಲ್ಲ. ತಾಯಿ ತಂದೆ ಮೊದಲಾದವರ ಸಾವನ್ನು ಕಣ್ಣಾರೆ ಕಾಣುತ್ತಿದ್ದರೂ ಅವರಿಗೆ ದೇಹ ಅನಿತ್ಯವೆಂಬುದು ಅರ್ಥವಾಗುವುದಿಲ್ಲ. ಇದಕ್ಕೆ ಏನು ಹೇಳೋಣ? ಸಂಸಾರದ ಸಂಕಟದಿಂದ ಪಾರಾಗಬೇಕೆನ್ನುವವನು ಭಗವಂತನ ಗುಣಗಳನ್ನು ಬಾಯಿಂದ ಹಾಡಬೇಕು, ಕಿವಿಯಿಂದ ಕೇಳಬೇಕು, ಮನಸ್ಸಿನಲ್ಲಿ ಸ್ಮರಿಸ ಬೇಕು. ಮಹಾರಾಜ, ಭಗವಂತನ ಕಲ್ಯಾಣಗುಣಗಳನ್ನು ವರ್ಣಿಸುವ ‘ಭಾಗವತ’ವೆಂಬ ಒಂದು ಗ್ರಂಥವಿದೆ. ಇದು ಆತ್ಮಜ್ಞಾನಿಗಳಿಗೆ ಅತ್ಯಂತ ಪ್ರಿಯವಾದುದು. ನನ್ನ ತಂದೆ ಯಾದ ವ್ಯಾಸಮಹರ್ಷಿ ಕಲಿಯುಗದ ಆರಂಭದಲ್ಲಿ ಇದನ್ನು ರಚಿಸಿ ನನಗೆ ಬೋಧಿಸಿದನು. ನಾನು ನಿರ್ಗುಣಬ್ರಹ್ಮನನ್ನು ಸಾಕ್ಷಾತ್ಕಾರಮಾಡಿಕೊಂಡವನಾದರೂ ಭಗವಂತನ ಪವಿತ್ರ ಲೀಲೆಗಳಿಗೆ ಮನಸೋತು ಅದನ್ನು ಕಲಿತುಕೊಂಡೆನು. ನೀನು ಭಗವದ್ಭಕ್ತನಾದುದರಿಂದ ನಾನು ಇದನ್ನು ನಿನಗೆ ಉಪದೇಶಿಸುವೆನು. ಮೋಕ್ಷಾಪೇಕ್ಷಿಯಾದ ನಿನಗೆ ಇದರಿಂದ ಮೋಕ್ಷ ಸಿಗುವುದು. ನಿನಗೆ ಸಾವು ಸಮೀಪಿಸಿರುವುದೆಂದು ನೀನೇನೂ ಆತಂಕ ಪಡಬೇಕಾದುದಿಲ್ಲ. ಶ್ರೇಯಸ್ಸನ್ನು ಪಡೆಯಹೊರಟವನಿಗೆ ಉದ್ಧಾರವಾಗಲು ಒಂದು ಕ್ಷಣ ಸಾಕು. ಹಿಂದೆ ಖಟ್ವಾಂಗನೆಂಬ ರಾಜನಿದ್ದ. ಅವನು ದೇವತೆಗಳ ಪರವಾಗಿ ನಿಂತು ಯುದ್ಧಮಾಡಿ ರಾಕ್ಷಸ ರನ್ನು ಸೋಲಿಸಿದ. ಸಂತುಷ್ಟರಾದ ದೇವತೆಗಳು ‘ನಿನಗೆ ಏನು ವರ ಬೇಕು?’ ಎಂದು ಕೇಳಿದರು. ಖಟ್ವಾಂಗ ‘ನನಗೆ ಇನ್ನೂ ಆಯುಸ್ಸು ಎಷ್ಟಿದೆ?’ ಎಂದು ಕೇಳಿದ. ದೇವತೆಗಳು ‘ಇನ್ನೊಂದು ಮುಹೂರ್ತ ಮಾತ್ರ’ ಎಂದರು. ಖಟ್ವಾಂಗ ಮರು ಮಾತಾಡಲಿಲ್ಲ. ಆ ಕ್ಷಣವೇ ಭೂಮಿಗಿಳಿದು, ಭಗವಂತನ ಧ್ಯಾನ ಮಾಡುತ್ತಾ ಮೋಕ್ಷವನ್ನು ಪಡೆದ. ಆದ್ದ ರಿಂದ ನೀನೇನೂ ಯೋಚಿಸಬೇಡ. ನಿನಗಿನ್ನೂ ಏಳು ದಿನಗಳ ಅವಧಿಯಿದೆ. ನಿನ್ನ ಉದ್ಧಾರವನ್ನು ಸಾಧಿಸಿಕೊಳ್ಳುವುದು ನಿನಗೇನು ಕಷ್ಟವಲ್ಲ ಎಂದನು.

