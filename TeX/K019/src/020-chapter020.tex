
\chapter{೨೦. ಅಗ್ನೀಧ್ರ}

ಪ್ರಿಯವ್ರತರಾಜನ ಹಿರಿಯಮಗನಾದ ಅಗ್ನೀಧ್ರನು ಜಂಬೂದ್ವೀಪದ ರಾಜನಾಗಿ, ಪ್ರಜೆಗಳನ್ನು ತನ್ನ ಮಕ್ಕಳಂತೆ ಕಾಪಾಡುತ್ತಿದ್ದನು. ಕೆಲವು ಕಾಲದ ಮೇಲೆ ಆತನಿಗೆ ಮಕ್ಕಳನ್ನು ಪಡೆಯಬೇಕೆಂಬ ಆಶೆ ಹುಟ್ಟಿತು. ಆತನು ಮಂದರ ಪರ್ವತಕ್ಕೆ ಹೋಗಿ, ಬ್ರಹ್ಮನನ್ನು ಕುರಿತು ತಪಸ್ಸು ಮಾಡಲು ಪ್ರಾರಂಭಿಸಿದನು. ಆ ಪರ್ವತ, ದೇವತೆಯರು ಆಟವಾಡುವ ಸ್ಥಳ. ಆದ್ದರಿಂದ ಅದು ಅತ್ಯಂತ ರಮ್ಯವಾಗಿತ್ತು. ಅಲ್ಲಿ ಎಲ್ಲಿ ನೋಡಿದರೂ ಹಣ್ಣಿನ ಭಾರದಿಂದ ಬಳುಕುತ್ತಿರುವ ಮರಗಳು, ಅವುಗಳಿಗೆ ಹಬ್ಬಿರುವ ಬಳ್ಳಿಗಳು, ಆ ಮರ ಬಳ್ಳಿಗಳ ಮೇಲೆ ಕುಳಿತು ಮಧುರವಾಗಿ ಹಾಡುತ್ತಿರುವ ಜೋಡುಹಕ್ಕಿಗಳು; ಅಲ್ಲಲ್ಲಿಯೇ ಬಗೆಬಗೆಯ ನೀರುಹೂಗಳಿಂದ ಕೂಡಿದ ನಿರ್ಮಲಜಲದ ಸರೋವರಗಳು! ಇಂತಹ ಸರೋವರವೊಂದರ ತಡಿಯಲ್ಲಿ ಅಗ್ನೀಧ್ರನು ತಪಸ್ಸಿಗೆ ಕುಳಿತಿದ್ದನು. ಆತನ ತಪಸ್ಸಿಗೆ ಮೆಚ್ಚಿದ ಬ್ರಹ್ಮದೇವನು ತನ್ನ ಸಭೆಯ ಸಂಗೀತಗಾತಿ ‘ಪೂರ್ವಚಿತ್ತಿ’ಯನ್ನು ಆತನ ಬಳಿಗೆ ಕಳುಹಿಸಿದನು. ಆಕೆ ‘ಝಣಕ್, ಝಣಕ್​’ ಎಂದು ಕಾಲಂದುಗೆಗಳ ಶಬ್ದ ಮಾಡುತ್ತಾ ಅಗ್ನೀಧ್ರನ ಸಮೀಪದಲ್ಲಿ ಓಡಾಡಿದಳು. ಇಂಪಾದ ಆ ದನಿಯನ್ನು ಕೇಳುತ್ತಲೆ ಆತನ ಕಣ್ಣು ತೆರೆದವು. ಇದಿರಿನಲ್ಲಿ ಹೂವಿಂದ ಹೂವಿಗೆ ಹಾರುತ್ತಿರುವ ದುಂಬಿಯಂತೆ ಬಳ್ಳಿಯಿಂದ ಬಳ್ಳಿಗೆ ಒಯ್ಯನೆ ವೈಯ್ಯಾರದಿಂದ ಸಾಗುತ್ತಿರುವ ಮೋಹಕ ಮೂರ್ತಿ! ಅದನ್ನು ಕಾಣುತ್ತಲೆ ಅಗ್ನೀಧ್ರನ ತಪಸ್ಸು ಹಾರಿಹೋಯಿತು. ಅವನು ಎಂದೂ ಹೆಣ್ಣನ್ನೇ ಕಾಣದವನಂತೆ ಮತಿಗೆಟ್ಟ ಮುಗ್ಧನಾಗಿ ಹೋದನು. ಆವನು ಆಕೆಯನ್ನು ಕಾಡಿನಲ್ಲಿರುವ ಪುಷಿಯನ್ನು ಮಾತನಾಡಿಸುವಂತೆ ಮಾತನಾಡಿಸುತ್ತಾ ಪ್ರಶ್ನಿಸಿದ–

‘ಹೇ ಮುನೀಶ್ವರಾ, ನೀನಾರು? ಆ ಭಗವಂತನ ಮಾಯಾ ಶಕ್ತಿಯೇ ನೀನೋ ಏನು ಕಥೆ? ಹೆದೆಯೇರಿಸಿದ ಬಿಲ್ಲಿನಂತೆ ಇರುವ ಆ ನಿನ್ನ ಹುಬ್ಬುಗಳಿಂದ ಏನನ್ನು ಸಾಧಿಸಬೇಕೆಂ ದಿರುವೆ? ಕಾಡಿನಲ್ಲಿ ಮೃಗದಂತಿರುವ ನನ್ನನ್ನು ಹೊಡೆದು ಉರುಳಿಸಬೇಕೆಂದೆ? ಆ ಬಿಲ್ಲಿನ ಕೆಳಗೆ ಗರಿಸಹಿತವಾದ ಬಾಣದಂತಿದೆ ನಿನ್ನ ಕಣ್ಣಿನ ನೋಟ. ಹರಿತವಾದ ಆ ಬಾಣದಿಂದ ದೇವರೇ ನನ್ನನ್ನು ಕಾಪಾಡಬೇಕು! ನಿನ್ನ ಶಿಷ್ಯರಂತಿರುವ ಈ ದುಂಬಿಗಳು ನಿನ್ನನ್ನು ಅಗಲಲಾರದೆ ಸುತ್ತಲೂ ಮುತ್ತಿಕೊಂಡು, ಜೇಂಕಾರದ ಸಾಮಗಾನದಿಂದ ನಿನ್ನನ್ನು ಸ್ತುತಿಸು ತ್ತಿವೆ. ನಿನ್ನ ಕಾಲಂದುಗೆಯ ಶಬ್ದ ತಿತ್ತಿರಿಪಕ್ಷಿಗಳ ಗಾನದಂತಿದೆ. ನೀನುಟ್ಟ ಉಡಿಗೆ ಕದಂಬ ಪುಷ್ಪದ ಕಾಂತಿಯಂತಿದೆ. ರತ್ನದ ನಿನ್ನ ನಡುಕಟ್ಟು ಅಗ್ನಿವಲಯದಂತಿದೆ. ನಿನ್ನ ಎದೆಯ ಮೇಲೆ ಎರಡು ಕೊಂಬುಗಳು ಮೂಡಿರುವಂತೆ ಕಾಣುತ್ತಿದೆಯಲ್ಲಾ! ಅಷ್ಟು ದಪ್ಪನಾದ ಕೊಂಬುಗಳ ಭಾರವನ್ನು ನಿನ್ನ ಬಡನಡು ಹೇಗೆ ತಾನೆ ಹೊರಬಲ್ಲದು? ನನ್ನ ಕಣ್ಣು ಆ ಕೊಂಬುಗಳಲ್ಲಿಯೇ ನಾಟಿಹೋಗಿವೆ. ನೀನು ಇಲ್ಲಿಗೆ ಬರುತ್ತಲೆ ನನ್ನ ಆಶ್ರಮವೆಲ್ಲ ಸುವಾಸನೆಯಿಂದ ತುಂಬಿಹೋಯಿತು. ನೀನು ಯಾರು? ಎಲ್ಲಿಂದ ಬಂದೆ? ನೀನಿರುವ ಸ್ಥಳದಲ್ಲಿ ಎಲ್ಲರೂ ನಿನ್ನಂತೆಯೇ ಇರುವರೇನು? ಅವರೆಲ್ಲರ ಮುಖವೂ ನಿನ್ನಂತೆಯೇ ಮಧುರವಾಗಿ, ಮನೋಹರವಾಗಿ ಇರುವುದೇನು? ನೀನು ತಿನ್ನುವ ಆಹಾರವಾವುದು? ಅಥವಾ ಆಹಾರವಿಲ್ಲದೆಯೇ ಇರುವಿಯೊ? ಮೀನಿನಂತಿರುವ ನಿನ್ನ ಕಣ್ಣು, ದುಂಬಿಯಂ ತಿರುವ ಮುಂಗುರುಳು–ನಿನ್ನ ಮುಖ ತಾವರೆಗೊಳದಂತಿದೆ! ನಿನ್ನ ರೂಪನ್ನು ಕಂಡು ನನ್ನ ತಪಸ್ಸು ಹಾರಿಹೋಯಿತು. ಇಂತಹ ಸುಂದರ ರೂಪನ್ನು ಯಾವ ತಪಸ್ಸಿನಿಂದ ನೀನು ಸಂಪಾದಿಸಿದೆ? ಬಾ ಮಿತ್ರ! ನೀನು ಇಲ್ಲಿಯೇ ತಪಸ್ಸು ಮಾಡು. ನಿನ್ನ ಮೇಲೆ ನಟ್ಟ ನನ್ನ ಕಣ್ಣು ಮನಸ್ಸುಗಳು ಹಿಂದಕ್ಕೆ ಬರುತ್ತಿಲ್ಲ. ನನ್ನ ತಪಸ್ಸಿಗೆ ಮೆಚ್ಚಿ ಬ್ರಹ್ಮನೆ ನಿನ್ನನ್ನು ನನಗೆ ಕರುಣಿಸಿರಬೇಕು. ಆದ್ದರಿಂದ ನಾನು ನಿನ್ನನ್ನು ಇಲ್ಲಿಯೇ ಇಟ್ಟುಕೊಳ್ಳುತ್ತೇನೆ. ಬಿಡುವುದಿಲ್ಲ.’

ಅಗ್ನೀಧ್ರನು ಬೇಡುತ್ತಿರುವುದನ್ನು ನೀಡುವುದಕ್ಕಾಗಿಯೇ ಬಂದಿದ್ದ ಪೂರ್ವಚಿತ್ತಿ ಆತನ ಬಳಿಯಲ್ಲಿಯೇ ನಿಂತು ಆತನ ಮಡದಿಯಾದಳು. ಗಂಡ ಹೆಂಡಿರು ಸಮಸ್ತ ಸುಖಭೋಗ ಗಳಿಗೂ ತೌರುಮನೆಯಾದ ಜಂಬೂದ್ವೀಪದಲ್ಲಿ ಬಹುಕಾಲ ಸುಖದಿಂದಿದ್ದರು. ಆಮೇಲೆ ಪೂರ್ವಚಿತ್ತಿಯು ವರ್ಷಕ್ಕೊಂದರಂತೆ ಒಂಬತ್ತು ವರ್ಷಕ್ಕೆ ಒಂಬತ್ತು ಮಕ್ಕಳನ್ನು\footnote{೧. ನಾಭಿ, ಕಿಂಪುರುಷ, ಹರಿವರ್ಷ, ಇಲಾವೃತ, ರಮ್ಯಕ, ಹಿರಣ್ಮಯ, ಕುರು, ಭದ್ರಾಶ್ವ, ಕೇತುಮೂಲ.} ಹೆತ್ತು, ರಾಜನ ಅಪೇಕ್ಷೆಯನ್ನು ಸಫಲಗೊಳಿಸಿದಳು. ಅಲ್ಲಿಗೆ ಆಕೆ ಬಂದ ಕಾರ್ಯ ಮುಗಿಯಿತು; ಬ್ರಹ್ಮದೇವನ ಸೇವೆಗಾಗಿ ಆಕೆಯು ಸತ್ಯಲೋಕಕ್ಕೆ ಹಿಂದಿರುಗಿದಳು. ಆದರೆ ಅಗ್ನೀಧ್ರನ ಕಾಮಾಭಿಲಾಷೆ ಮಾತ್ರ ಇನ್ನೂ ಹಚ್ಚ ಹಸುರಾಗಿಯೇ ಇತ್ತು. ಆತ ಪೂರ್ವಚಿತ್ತಿಯನ್ನು ಮತ್ತೆ ಪಡೆಯಬೇಕೆಂಬ ಆಶೆಯಿಂದ ಅನೇಕ ಕಾಮ್ಯ ಕರ್ಮಗಳನ್ನು ಮಾಡುತ್ತಾ, ಸತ್ತು, ಪಿತೃಲೋಕವನ್ನು ಸೇರಿದನು.

