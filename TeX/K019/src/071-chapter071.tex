
\chapter{೭೧. ಸಂಸಾರಿ ಶ್ರೀಕೃಷ್ಣ}

ಇಗೋ, ಇದೇ ಶ್ರೀಕೃಷ್ಣನ ಪಟ್ಟದ ರಾಣಿಯಾದ ರುಕ್ಮಿಣಿಯ ಅಂತಃಪುರ. ಸುತ್ತಲೂ ಮುತ್ತಿನ ಜಾಲರಿಗಳಿಂದ ಕೂಡಿದ ಪಟ್ಟೆಯ ಮಂಚದ ಮೇಲೆ ಮೆತ್ತನೆಯ ರೇಷ್ಮೆ ಸುಪ್ಪ ತ್ತಿಗೆಗಳು, ಅವುಗಳ ಮೇಲೆ ಹಾಲಿನ ನೊರೆಯಂತಿರುವ ಮೇಲುಹೊದಿಕೆ, ಅದರ ಮೇಲೆ ಆಗತಾನೆ ಅರಳಿದ ಮಲ್ಲಿಗೆ ಜಾಜಿಯ ಹೂಗಳು, ಬಗೆಬಗೆಯ ಹೂವಿನ ಹಾರಗಳು ಆ ಮಂಚಕ್ಕೆ ಹಾಕಿದ ಪರದೆಯಂತಿವೆ; ಎಂಟು ಮೂಲೆಗಳಲ್ಲಿಯೂ ಎಂಟು ಎತ್ತರವಾದ ಪೀಠಗಳ ಮೇಲೆ ಅಗರುಬತ್ತಿಗಳು ಸುವಾಸನೆಯನ್ನು ಚೆಲ್ಲುತ್ತಾ ಉರಿಯುತ್ತಿವೆ; ಹಾಸಿಗೆಯ ಪಕ್ಕದಲ್ಲಿನ ಒಂದು ಪೀಠದಲ್ಲಿ ಬಂಗಾರದ ತಟ್ಟೆಯ ತುಂಬ ಹಣ್ಣುಗಳು, ಅದರ ಪಕ್ಕದಲ್ಲಿ ಹಾಲಿನ ಚೆಂಬು; ತಾಂಬೂಲದ ತಟ್ಟೆ. ಗೋಡೆಗಳ ಮೇಲೆ ನೆಟ್ಟಿರುವ ರತ್ನಗಳು–ಅವೇ ಅಲ್ಲಿನ ದೀವಿಗೆಗಳು–ತಮ್ಮ ಕಾಂತಿಯಿಂದ ಬೆಳಕನ್ನು ಚೆಲ್ಲುತ್ತಿವೆ; ಮೇಲ್ಗಡೆಯ ಗವಾಕ್ಷೆಯಿಂದ ಬೆಳುದಿಂಗಳ ತಂಪುಬೆಳಕು ಇಣಿಕಿ ಹಾಕುತ್ತಿದೆ; ಮನೆಯ ಕೈತೋಟದಲ್ಲಿ ಆಗತಾನೆ ಅರಳುತ್ತಿರುವ ಹೂಗಳ ಸುವಾಸನೆಯನ್ನು ಹೊತ್ತ ತಂಗಾಳಿ ಕಿಟಕಿಯಿಂದ ಮಂದ ಮಂದವಾಗಿ ಬೀಸುತ್ತಿದೆ. ಶ್ರೀಕೃಷ್ಣನು ಶೇಷಶಾಯಿಯಾದ ನಾರಾ ಯಣನಂತೆ ದಿಂಬಿನ ಮೇಲೆ ಮಲಗಿ ಮಡದಿಯಾದ ರುಕ್ಮಿಣಿ ಮಡಿಸಿ ಮಡಿಸಿ ಬಾಯಲ್ಲಿ ಇಡುತ್ತಿರುವ ತಾಂಬೂಲವನ್ನು ಮೆಲ್ಲುತ್ತಾ, ಕಣ್ಣಿನಿಂದ ಆ ಹೆಣ್ಣಿನ ರೂಪವನ್ನು ಹೀರು ತ್ತಿದ್ದಾನೆ. ಇದನ್ನು ಕಂಡು ಆನಂದಗೊಂಡ ರುಕ್ಮಿಣಿ ತನ್ನ ಚೆಲುವಿಗಾಗಿ ಹೆಮ್ಮೆಪಟ್ಟು ಕೊಳ್ಳುತ್ತಾ, ಕೈಲಿ ಬೀಸಣಿಗೆಯನ್ನು ಹಿಡಿದು ಕೈಬಳೆಗಳನ್ನು ಘಲಿಘಲಿರೆನಿಸುತ್ತಾ ಆತನಿಗೆ ಗಾಳಿ ಹಾಕುತ್ತಿದ್ದಾಳೆ.

ರಸಿಕರಸಪುಷಿಯಾದ ಶ್ರೀಕೃಷ್ಣನಿಗೆ ಈ ಸುಂದರ ಸನ್ನಿವೇಶದಲ್ಲಿ ತನ್ನ ಮಡದಿಯನ್ನು ಸ್ವಲ್ಪ ರೇಗಿಸಿ ತಮಾಷೆನೋಡಬೇಕೆನ್ನಿಸಿತು. ಆತನು ‘ರುಕ್ಮಿಣಿ, ನೀನು ರಾಜಕುಮಾರಿ, ಪರಮ ಸುಂದರಿ! ನಿನ್ನನ್ನು ವರಿಸಬೇಕೆಂದು ಎಷ್ಟೋ ರಾಜರು ಕಾತರರಾಗಿದ್ದರು. ಶಿಶು ಪಾಲನಂತಹ ಮಹಾಶೂರ ನಿನ್ನ ದಾಸಾನುದಾಸನಾಗಲು ಸಿದ್ಧನಾಗಿದ್ದ. ಅವರನ್ನೆಲ್ಲ ಬಿಟ್ಟು ನೀನು ನನ್ನನ್ನು ವರಿಸಿದೆಯಲ್ಲ, ನೀನೆಂತಹ ದಡ್ಡಿ! ನನ್ನ ಕುಲಗೋತ್ರಗಳೊಂದೂ ನನಗೇ ಗೊತ್ತಿಲ್ಲ. ನಾನೇನು ವಸುದೇವನ ಮಗನೊ, ಗೊಲ್ಲನಾದ ನಂದನ ಮಗನೊ! ನಾನು ಮಹಾಪರಾಕ್ರಮಿಯೆ? ಅದೂ ಇಲ್ಲ; ರಾಜರಿಗೆ ಹೆದರಿ ಈ ಸಮುದ್ರಮಧ್ಯದಲ್ಲಿ, ತಲೆ ಮರೆಸಿಕೊಂಡಿದ್ದೇನೆ. ನಾನು ಹುಟ್ಟಾ ಬಡವ; ನನ್ನ ಗೆಳೆಯರೂ, ನನ್ನನ್ನು ಹೊಗಳು ವವರೂ ಕೇವಲ ಬಡವರೇ! ನನಗೆ ಮನೆ ಮಕ್ಕಳು ಎಂಬ ಅಭಿಮಾನ ಕೂಡ ಇಲ್ಲ; ನನ್ನನ್ನು ನಂಬಿದವರಿಗೆ ಸ್ವಲ್ಪವೂ ಸುಖವಿಲ್ಲ. ಸ್ವಲ್ಪ ಯೋಚಿಸಿ ನೋಡು, ವಿಲಾಸವತಿ ಯಾದ ನೀನೆಲ್ಲಿ? ಒರಟನಾದ ನಾನೆಲ್ಲಿ? ಇತರರನ್ನಾದರೂ ವರಿಸಿದ್ದರೆ ನಿನ್ನನ್ನು ಸುಪ್ಪ ತ್ತಿಗೆಯ ಮೇಲೆ ಕೂಡಿಸಿ ಮುಟ್ಟಿದರೆಲ್ಲಿ ಮಾಸುವೆಯೊ ಎಂಬಂತೆ ಸುಖವಾಗಿರಿಸು ತ್ತಿದ್ದರು. ಅನುರೂಪನಲ್ಲದ ನನ್ನನ್ನು ಕೈಹಿಡಿದು ನೀನು ಕೆಟ್ಟೆ! ಈಗಲಾದರೂ ಏನಾ ಯಿತು? ನಿನಗೆ ಅನುರೂಪನಾದ ಮಹಾರಾಜನೊಬ್ಬನನ್ನು ಕೈಹಿಡಿದು, ಇಹಲೋಕದ ಸುಖವನ್ನು ಸೂರೆಗೊಳ್ಳಬಾರದೆ?’ ಎಂದ. 

ರುಕ್ಮಿಣಿಗೆ ತನ್ನ ಸಮಾನರಾದ ಅದೃಷ್ಟವಂತರೇ ಇಲ್ಲ, ತನ್ನ ಗಂಡ ತನ್ನನ್ನು ಕ್ಷಣ ಕಾಲವೂ ಬಿಟ್ಟಿರಲಾರ–ಎಂಬ ಹೆಮ್ಮೆ ತುಂಬಿಕೊಂಡಿತ್ತು. ಗಂಡನ ಮಾತುಗಳನ್ನು ಕೇಳು ತ್ತಲೆ ಆ ಹೆಮ್ಮೆಯೆಲ್ಲ ಸೋರಿಹೋಯಿತು. ಆಕೆ ಭಯದಿಂದ ನಡುಗಿ ಭೂಮಿಗಿಳಿದು ಹೋದಳು, ಕಣ್ಣುಗಳಿಂದ ಬಳಬಳನೆ ನೀರು ಸುರಿಯಿತು, ಕೈಲಿ ಹಿಡಿದಿದ್ದ ಬೀಸಣಿಗೆ ಜಾರಿ ಕೆಳಕ್ಕೆ ಬಿತ್ತು, ಅದರೊಡನೆ ಆಕೆಯೂ ಕುಸಿದು ನೆಲಕ್ಕುರುಳಿದಳು. ತಡೆಯಲಾರದ ದುಃಖ ದಿಂದ ಆಕೆ ಮೂರ್ಛಿತಳಾದಳು. ಇದನ್ನು ಕಂಡು ಭಯಗೊಂಡ ಶ್ರೀಕೃಷ್ಣನು ದಿಗ್ಗನೆ ಮೇಲಕ್ಕೆದ್ದು, ಆಕೆಯನ್ನು ಮಂಚದ ಮೇಲೆ ಮಲಗಿಸಿ, ಕೆದರಿದ ಆಕೆಯ ತಲೆಗೂದಲನ್ನು ನೇವರಿಸಿದನು, ಕಣ್ಣೀರನ್ನು ತೊಡೆದನು, ಮುಖದ ಬೆವರನ್ನು ಒರೆಸಿದನು, ಎರಡು ತೋಳುಗಳಿಂದಲೂ ಆಕೆಯನ್ನು ಆಲಿಂಗಿಸಿಕೊಂಡು ಮುಖಕ್ಕೆ ಮುತ್ತಿತ್ತನು. ತನ್ನ ಅಪ ಹಾಸ್ಯದಿಂದ ನೊಂದ ಆ ಸುಕುಮಾರಿಯನ್ನು ಪರಿಪರಿಯಾಗಿ ರಮಿಸುತ್ತಾ ‘ಮನೋಹರಿ, ನೀನು ನನ್ನ ಹೃದಯೇಶ್ವರಿ! ನಿನ್ನನ್ನು ನಾನು ಅಗಲಿ ಜೀವಿಸಬಲ್ಲೆನೆ? ಹುಚ್ಚಿ, ನಿನ್ನೊಡನೆ ಸ್ವಲ್ಪ ಪ್ರಣಯಕಲಹ ಮಾಡಬೇಕೆನ್ನಿಸಿತು; ನಡುಗುತ್ತಿರುವ ನಿನ್ನ ತುಟಿ, ಗಂಟುಹಾಕಿದ ಹುಬ್ಬು, ಕೆಕ್ಕರಗಣ್ಣಿನ ನೋಟಗಳನ್ನು ನೋಡೋಣವೆಂದು ಬಯಸಿ ನಿನ್ನನ್ನು ಅಪಹಾಸ್ಯ ಮಾಡಿದೆ. ಇಂತಹ ಪ್ರಣಯಕಲಹಗಳಿಂದ ಸಂಸಾರಜೀವನ ರಸವತ್ತಾಗುವುದೆಂದು ನೆನೆದೆ. ನೀನೋ, ದೀಪದ ಬೆಳಕಿಗೆ ಬಾಡುವ ಜಾಜಿಯಷ್ಟು ಕೋಮಲ. ವಿನೋದಕ್ಕೂ ಅಪಹಾಸ್ಯವನ್ನು ಸಹಿಸಲಾರೆ’ ಎಂದ.

ಶ್ರೀಕೃಷ್ಣನ ಮಾತು ಕೇವಲ ವಿನೋದವೆಂದು ಗೊತ್ತಾದಮೇಲೆ ಆಕೆಗೆ ಧೈರ್ಯಬಂತು. ಜಾಣೆಯಾದ ಆ ಹೆಣ್ಣು ಗಂಡನ ಮಾತುಗಳನ್ನೆಲ್ಲ ಒಪ್ಪಿಕೊಂಡು, ಅವುಗಳಿಗೆ ತನ್ನದೆ ಆದ ಕಾರಣಗಳನ್ನೂ ಕೊಟ್ಟಳು. “ಪ್ರಭು, ನಿನ್ನ ಮಾತೆಲ್ಲ ನಿಜ. ನಾನು ಹೇಗೆ ತಾನೆ ನಿನಗೆ ತಕ್ಕವಳಾದೇನು? ನಿನ್ನ ಆ ಕಣ್ಣೆ ಸಾಕು; ಅದಕ್ಕೆ ಸಮಾನವಾದ ಚೆಲುವುಳ್ಳವರು ಮೂರು ಲೋಕದಲ್ಲಿಯೂ ಇಲ್ಲ. ನಿನಗೆ ನಿನ್ನ ಕುಲವೆ ಗೊತ್ತಿಲ್ಲವೆಂದು ಹೇಳುವೆಯಲ್ಲಾ! ಅದು ಯಾರಿಗೂ ಗೊತ್ತಿಲ್ಲ. ಮಹಾಯೋಗಿಗಳಿಗೂ ನಿನ್ನ ಮೂಲ ಗೊತ್ತಿಲ್ಲ. ನೀನು ಶತ್ರುಗಳಿಗೆ ಹೆದರಿ ಸಮುದ್ರವನ್ನು ಹೊಕ್ಕಿರುವೆಯಾ? ಮಹಾನುಭಾವ, ಮೂರಡಿಗಳಿಂದ ಮೂರು ಲೋಕಗಳನ್ನೆ ಅಳೆದ ನಿನಗೆ ಶತ್ರುಗಳ ಭಯವೆ? ಲೋಕದಲ್ಲಿ ಕೆಲವು ಜನ ನನ್ನನ್ನು ಸಮುದ್ರರಾಜನ ಮಗಳಾದ ಲಕ್ಷ್ಮಿಯೆಂದು ಕರೆಯುತ್ತಾರೆ; ಬಹುಶಃ ಆ ಕಾರಣದಿಂದಲೆ ನೀನು ನನ್ನ ಮೇಲೆ ಪ್ರೇಮವನ್ನು ತೋರುವುದಕ್ಕಾಗಿ ಇಲ್ಲಿ ನೆಲೆಸಿರುವೆಯೊ ಏನೊ! ‘ಭಿಕಾರಿಗಳೆ ನನ್ನ ಗೆಳೆಯರು, ಅವರೇ ನನ್ನನ್ನು ಹೊಗಳುವವರು’ ಎಂಬ ನಿನ್ನ ಮಾತು ದಿಟವೆ. ಮದಾಂಧರಾದ ಶ್ರೀಮಂತರಿಗೆ ನೀನೆಲ್ಲಿ ಸಿಕ್ಕುತ್ತಿ? ಇಂದ್ರಿಯಗಳಿಗೆ ದಾಸರಾದ ಅವರು ನಿನ್ನನ್ನೆಲ್ಲಿ ನೆನೆಯುತ್ತಾರೆ? ಇನ್ನು ನೀನು ಮೋಹದೂರನೆಂಬುದನ್ನು ನೀನು ಹೇಳಬೇಕೆ? ಆತ್ಮಾರಾಮನಾದ ನಿನಗೆ ಮೋಹವೆಲ್ಲಿಯದು, ಕಾಮವೆಲ್ಲಿಯದು? ಆದರೆ ನಾನು ನಿನ್ನ ಮೋಹಕ್ಕಾಗಲಿ ಇಂದ್ರಿಯ ತೃಪ್ತಿಗಾಗಲಿ ನಿನ್ನನ್ನು ಆಶ್ರಯಿಸಿಲ್ಲ; ಶಾಶ್ವತ ಸುಖವನ್ನು ಅರಸಿಬಂದಿದ್ದೇನೆ. ನೀನು ಕಾಮಿಯಲ್ಲದಿದ್ದರೂ ನಾನು ನಿನ್ನ ಪಾದಸೇವೆ ಯಲ್ಲಿ ಅನುರಾಗವುಳ್ಳವಳು. ಜನ್ಮಜನ್ಮದಲ್ಲಿಯೂ ನೀನು ನನ್ನ ಪತಿಯಾಗಿ ನನ್ನನ್ನು ಉದ್ಧರಿಸಬೇಕು. ಹೇ, ಕಪಟನಾಟಕಸೂತ್ರಧಾರಿ, ನಿನ್ನ ನಟನೆಯಲ್ಲಿ ಒಮ್ಮೆ ನನ್ನತ್ತ ಪ್ರೇಮಧಾರೆ ಹರಿದರೆ ಸಾಕು, ನಾನು ಧನ್ಯೆ” ಎಂದಳು.

ರುಕ್ಮಿಣಿಯ ಮಾತುಗಳನ್ನು ಕೇಳಿ ಸುಪ್ರೀತನಾದ ಶ್ರೀಕೃಷ್ಣನು ‘ಹೇ ನನ್ನ ಮನದನ್ನೆ, ನಿನ್ನ ಬಾಯಿಂದ ಇಂತಹ ಮಧುರವಾದ ಮಾತುಗಳನ್ನು ಕೇಳಬೇಕೆಂದೆ ನಾನು ನಿನ್ನನ್ನು ಕೆಣಕಿದುದು. ಮಹಾಪತಿವ್ರತೆಯಾದ ನೀನು ನನ್ನ ಏಕಾಂತಭಕ್ತೆ. ನೀನು ಅತ್ಯಲ್ಪವಾಗಿರುವ ಲೌಕಿಕ ಸುಖಗಳನ್ನು ಬಯಸುವುದಿಲ್ಲವೆಂದು ನಾನು ಬಲ್ಲೆ. ಸಂಸಾರಿಗಳಲ್ಲಿ ನಿನ್ನಂತಹ ಹೆಣ್ಣು ಮೂರು ಲೋಕಗಳಲ್ಲಿ ಹುಡುಕಿದರೂ ಇನ್ನೊಬ್ಬಳಿಲ್ಲ. ನಾನು ನಿನ್ನ ಶುದ್ಧವಾದ ಪ್ರೇಮಕ್ಕೆ ಏನನ್ನು ಕೊಡಲಿ? ಇಗೋ ನನ್ನನ್ನೆ ನಿನಗೆ ಅರ್ಪಿಸಿದ್ದೇನೆ’ ಎಂದನು.

ಶ್ರೀಕೃಷ್ಣನಿಗೆ ಎಂಟು ಜನ ಪಟ್ಟದ ರಾಣಿಯರು. ಇವರ ಜೊತೆಗೆ ಹದಿನೆಂಟು ಸಹಸ್ರ ಮಂದಿ ಹೆಂಡತಿಯರು. ಇವರಲ್ಲಿ ಒಬ್ಬೊಬ್ಬಳಿಗೂ ತನ್ನಂತೆ ಪ್ರಿಯಳಾದವಳು ಶ್ರೀಕೃಷ್ಣ ನಿಗೆ ಮತ್ತೊಬ್ಬಳಿಲ್ಲವೆಂದೇ ಭಾವನೆ. ಅವರೆಲ್ಲರೂ ತಮ್ಮ ಚೆಲುವಿನಿಂದ ಆತನನ್ನು ಮೋಹಪರವಶನನ್ನಾಗಿ ಮಾಡಿರುವೆವೆಂದೇ ಭ್ರಮಿಸುವರು. ಆದರೆ ಅವರೆಲ್ಲ ಆತನ ರೂಪಿಗೆ ಮರುಳಾಗಿದ್ದವರೇ ಹೊರತು, ಆತನೇನೂ ಇವರ ಹಾವ ಭಾವ ವಿಲಾಸ ವಿಭ್ರಮ ಗಳಿಗೆ ಮರುಳಾಗಿದ್ದವನಲ್ಲ. ಅವರೆಲ್ಲ ಆತನ ಕೈಗೊಂಬೆಗಳಾಗಿ ಕುಣಿಯುತ್ತಿದ್ದವರೆ. ದಾಸದಾಸಿಯರಿದ್ದರೂ ಅವರು ಆತನ ಸೇವೆಯನ್ನು ಮಾತ್ರ ತಾವೆ ಕೈಯಾರೆ ಮಾಡುವರು. ಅದೇನು ಆತನ ಮೋಡಿಯೊ! ಗುಣದಲ್ಲಿ, ರೂಪದಲ್ಲಿ, ಯೋಗ್ಯತೆಯಲ್ಲಿ ಇವರಿಗೆಲ್ಲ ಹಿರಿಯಳಾದವಳು ರುಕ್ಮಿಣಿ. ಆಕೆಯ ಶೃಂಗಾರಜೀವನದ ಒಂದು ಉದಾಹರಣೆಯಿಂದ ಶ್ರೀಕೃಷ್ಣನ ರಸಿಕ ರಸಜೀವನವನ್ನು ಅರ್ಥಮಾಡಿಕೊಳ್ಳಬಹುದು.

ಶ್ರೀಕೃಷ್ಣನ ಒಬ್ಬೊಬ್ಬ ಮಡದಿಗೂ ಹತ್ತುಜನ ಗಂಡು ಮಕ್ಕಳು, ಒಬ್ಬೊಬ್ಬ ಹೆಣ್ಣುಮಗಳು. ಈ ಮಕ್ಕಳಿಗೆಲ್ಲ ಮದುವೆಯಾಗಿ, ಅವರಿಂದ ಸಂತಾನವೃದ್ಧಿಯಾದು ದನ್ನೆಲ್ಲ ವಿವರಿಸುತ್ತಾ ಹೋಗುವುದು ಅಸಾಧ್ಯ. ಅವರಲ್ಲಿ ಅಸಾಧಾರಣ ಪುರುಷರಾದ ಕೆಲವರನ್ನು ಮಾತ್ರ ಕುರಿತು ನಾವಿಲ್ಲಿ ಸಮಾಲೋಚಿಸಬಹುದು. ಅಂತಹರಲ್ಲಿ ಬಹು ಮುಖ್ಯನಾದವನು ರುಕ್ಮಿಣಿಯ ಮಗನಾದ ಪ್ರದ್ಯುಮ್ನ. ಆತನು ಶಂಬರಾಸುರನನ್ನು ಕೊಂದು ಮಾಯಾವತಿಯನ್ನು ಮದುವೆಯಾದುದು ಸರಿಯಷ್ಟೆ! ಈತನ ಗುಣರೂಪ ಗಳನ್ನು ಕೇಳಿ ರುಕ್ಮಿಯ ಮಗಳಾದ ರುಕ್ಮವತಿಯು ಈತನನ್ನು ಮನಸಾ ಪತಿಯೆಂದು ವರಿಸಿ ದ್ದಳು. ಪ್ರದ್ಯುಮ್ನನೂ ಆ ಹೆಣ್ಣಿನ ಸುಗುಣ ಸೌಂದರ್ಯಗಳನ್ನು ಕೇಳಿ ಅವಳನ್ನು ಮೋಹಿಸಿದ್ದನು. ಹೀಗಿರಲು ರುಕ್ಮಿಯು ತನ್ನ ಮಗಳ ಸ್ವಯಂವರವನ್ನು ಏರ್ಪಡಿಸಿದನು. ಆಗ ಪ್ರದ್ಯುಮ್ನ ಸ್ವಯಂವರ ಮಂಟಪಕ್ಕೆ ನುಗ್ಗಿ, ಅಲ್ಲಿದ್ದ ರಾಜರನ್ನೆಲ್ಲ ಸದೆಬಡಿದು ರುಕ್ಮವತಿಯನ್ನು ಹೊತ್ತುಕೊಂಡು ಹೋದನು. ರುಕ್ಮಿ ಶ್ರೀಕೃಷ್ಣನ ಪರಮದ್ವೇಷಿ ಯಾದರೂ ರುಕ್ಮಿಣಿಯ ಮೇಲಿನ ಅಭಿಮಾನದಿಂದ ತನ್ನ ಸೋದರಳಿಯನನ್ನು ಕ್ಷಮಿಸಿ ದನು. ಇಷ್ಟೇ ಅಲ್ಲ, ತನ್ನ ತಂಗಿಯ ಮೇಲಿನ ಪ್ರೇಮದಿಂದ ಆತನು ತನ್ನ ಮೊಮ್ಮಗಳಾದ ‘ರೋಚಿನಿ’ಯನ್ನು ಪ್ರದ್ಯುಮ್ನನ ಮಗನಾದ ಅನಿರುದ್ಧನಿಗೆ ಕೊಟ್ಟು ಮದುವೆ ಮಾಡಿದನು.

ರೋಚನಿ-ಅನಿರುದ್ಧರ ಮದುವೆಗೆಂದು ಬಂದಿದ್ದ ರಾಜರುಗಳೆಲ್ಲ ಇನ್ನೂ ರುಕ್ಮಿಯ ಅತಿಥಿಗಳಾಗಿ ಅರಮನೆಯಲ್ಲಿಯೆ ಇದ್ದರು. ಅವರಲ್ಲಿ ಕಳಿಂಗರಾಜನೂ ಒಬ್ಬ. ಅವನಿಗೆ ಕೃಷ್ಣನ ತಲೆ ಕಂಡರಾಗದು. ರುಕ್ಮಿಯು ಅವನ ಸಂಬಂಧವನ್ನು ಬೆಳಸಿ ಮುಂದುವರಿಸಿ ದುದು ಅವನಿಗೆ ಹಿಡಿಸಲಿಲ್ಲ. ಮೋಸವಿದ್ಯೆಯಲ್ಲಿ ಎತ್ತಿದ ಕೈ ಅವನು. ರುಕ್ಮಿಯನ್ನು ಗುಟ್ಟಾಗಿ ಕರೆದು ಅವನು ‘ಗೆಳೆಯ, ನೀನು ನಿನ್ನ ತಂಗಿಗಾಗಿ ಶತ್ರುವಿನೊಡನೆ ಸಂಬಂಧ ವನ್ನೇನೊ ಬೆಳೆಸಿದೆ, ಹೋಗಲಿ. ಅವರಿಗೆ ತಕ್ಕ ಶಿಕ್ಷೆ ಮಾಡುವ ಒಂದು ಉಪಾಯವನ್ನು ಹೇಳಿಕೊಡುತ್ತೇನೆ, ಕೇಳು. ಈ ಬಲರಾಮನಿಗೆ ಪಗಡೆಯೆಂದರೆ ಪಂಚಪ್ರಾಣ. ಆದರೆ ಆಡುವುದು ಮಾತ್ರ ಗೊತ್ತಿಲ್ಲ. ಅವನನ್ನು ಮೆಲ್ಲಗೆ ಪುಸಲಾಯಿಸಿ ಆಟಕ್ಕೆ ಕೂಡಿಸಿಕೊ. ಅವನ ಐಶ್ವರ್ಯವನ್ನೆಲ್ಲ ಕಿತ್ತುಕೊಂಡು ಕಳುಹಿಸೋಣ’ ಎಂದು ಸಲಹೆಯಿತ್ತ. ‘ಕೇಡು ಗಾಲಕ್ಕೆ ದುರ್ಬುದ್ಧಿ’ ಎಂಬಂತೆ ರುಕ್ಮಿಗೆ ಆ ಮಾತು ರುಚಿಸಿತು. ಅವನು ಬಲರಾಮನನ್ನು ಜೂಜಾಟಕ್ಕೆ ಕರೆದ. ಬಲರಾಮ ತಕ್ಷಣ ಒಪ್ಪಿ ಬಂದ. ಆಟ ಮೊದಲಾಯಿತು. ಪ್ರತಿ ಯೊಂದು ಆಟದಲ್ಲಿಯೂ ಬಲರಾಮ ಸೋಲುತ್ತಾ ಹೋದ. ಸೋತಂತೆಲ್ಲ ಅವನ ಹಟ, ರೋಷಗಳು ಬೆಳೆಯುತ್ತಾ ಹೋದವು. ಪಣವಾಗಿ ಸಾವಿರ ಒಡ್ಡಿದವನು ಹತ್ತು ಸಾವಿರಕ್ಕೆ ಏರಿಸಿದ, ಸಲಸಲವೂ ಸೋತ. ಪ್ರತಿಸಲ ಸೋತಾಗಲೂ ಕಳಿಂಗ ಹದಿನಾರು ಹಲ್ಲೂ ಕಾಣುವಂತೆ ನಕ್ಕು ಹಾಸ್ಯಮಾಡುವನು. ಆಟದ ಕಾವಿನಲ್ಲಿದ್ದ ಬಲರಾಮ ಅದನ್ನು ಲಕ್ಷಿಸ ಲಿಲ್ಲ. ಕೊನೆಗೆ ರುಕ್ಮಿ ಒಂದು ಲಕ್ಷವನ್ನು ಪಣವಾಗಿ ಒಡ್ಡಿದ, ಈ ಸಲ ಬಲರಾಮ ಗೆದ್ದ. ತಾನು ಸೋತುದೆಲ್ಲ ಹಿಂದಕ್ಕೆ ಬರುವುದೆಂದಿದ್ದರೆ, ರುಕ್ಮಿ ಕಳಿಂಗರು ತಮಗೇ ಜಯ ವಾಯಿತೆಂದು ಹಟ ಹಿಡಿದು ಕೂತರು. ಬಲರಾಮನಿಗೆ ಕೋಪ ಉಕ್ಕಿತಾದರೂ ಅದನ್ನು ತಡೆದುಕೊಂಡು ಮತ್ತೊಮ್ಮೆ ಒಂದು ಲಕ್ಷವನ್ನು ಪಣವಾಗಿ ಒಡ್ಡಿ ಗೆದ್ದ. ಆದರೆ ಈ ಸಲವೂ ರುಕ್ಮಿ ತಾನೆ ಗೆದ್ದುದಾಗಿ ಹಟ ಹಿಡಿದ; ಕಳಿಂಗ ಹೌದೆಂದು ಸಾಕ್ಷಿ ಹೇಳಿದ. ಇಷ್ಟೇ ಅಲ್ಲ, ಅವರಿಬ್ಬರೂ ಸೇರಿಕೊಂಡು ‘ಎಲೊ ಗೊಲ್ಲ, ನಿನಗೇನು ಗೊತ್ತೊ ಪಗಡೆಯಾಟ? ಪಗಡೆ, ಬಾಣಗಳ ವಿದ್ಯೆ ಕ್ಷತ್ರಿಯ ವಿದ್ಯೆಗಳು. ಗೊಲ್ಲರವಲ್ಲ’ ಎಂದರು. ಇದನ್ನು ಕೇಳಿ ಅವನ ಕಣ್ಣು ಕೆಂಪಗಾಯಿತು, ಕೋಪ ಹೊತ್ತಿತು, ರೋಷ ಉಕ್ಕಿತು. ಅಲ್ಲಿಯೇ ಇದ್ದ ಒಂದು ಗದೆಯನ್ನು ತೆಗೆದುಕೊಂಡು ರುಕ್ಮಿಯನ್ನು ಕೊಂದುಹಾಕಿದನು. ಒಡನೆಯೆ ಕಳಿಂಗನು ಅಲ್ಲಿಂದ ಓಡಿಹೋಗಲು, ಬಲರಾಮನು ಅವನ ಬೆನ್ನಟ್ಟಿ ಹೋಗಿ ಅವನ ಹಲ್ಲುಗಳನ್ನೆಲ್ಲ ಉದುರಿಸಿದನು. ಉಳಿದ ರಾಜರೆಲ್ಲ ಓಡಿಹೋದರು. ಯಾದವರು ನೂತನ ವಧೂವರರನ್ನು ಕರೆದುಕೊಂಡು ದ್ವಾರಕಿಗೆ ಹಿಂದಿರುಗಿದರು.

ಶ್ರೀಕೃಷ್ಣನಿಗೆ ಒಂದು ಸಮಾಧಾನ. ಯಾವುದೋ ಕಾರಣದಿಂದ ಹೇಗೋ, ಎಂತೋ, ದುಷ್ಟ ಕ್ಷತ್ರಿಯರ ಸಂಹಾರವಾಗುತ್ತಿದೆ, ಭೂಭಾರ ತಗ್ಗುತ್ತಿದೆ, ತನ್ನ ಅವತಾರದ ಉದ್ದೇಶ ನೆರವೇರುತ್ತಿದೆ.

