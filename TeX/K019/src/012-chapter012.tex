
\chapter{೧೨. ದತ್ತಾತ್ರೇಯಾವತಾರ}

ಕರ್ದಮ ದೇವಹೂತಿಯರು ತಪಸ್ಸಿನಿಂದ ಮೋಕ್ಷವನ್ನು ಪಡೆದರು. ಅವರ ಮಗನಾದ ಕಪಿಲಮುನಿ ಸಂಸಾರದಲ್ಲಿ ವಿರಕ್ತನಾಗಿ, ಸಮುದ್ರತೀರದಲ್ಲಿ ಆಶ್ರಮವನ್ನು ಸ್ಥಾಪಿಸಿ ಕೊಂಡು ತಪೋನಿರತನಾದನು. ಆದ್ದರಿಂದ ಕರ್ದಮನ ವಂಶ ಅವನ ಹೆಣ್ಣು ಮಕ್ಕಳಿಂದ ಬೆಳೆಯಬೇಕಾಯಿತು. ಆ ಹೆಣ್ಣುಮಕ್ಕಳು ಒಂಬತ್ತು ಜನರೂ ಮಹಾಪುರುಷರನ್ನು ಕೈಹಿಡಿದು ತಮ್ಮ ಸಂತಾನಸೌಭಾಗ್ಯದಿಂದ ಲೋಕಕಲ್ಯಾಣವನ್ನು ಸಾಧಿಸಿದರು. ಅವರ ಲ್ಲಿಯೂ ಎರಡನೆಯವಳಾದ ಅನುಸೂಯಾದೇವಿ ತ್ರಿಮೂರ್ತಿಗಳನ್ನೆ ತನ್ನ ಮಕ್ಕಳಾಗಿ ಪಡೆದ ಮಹಾನುಭಾವಳು. ಆಕೆಗೆ ಮಕ್ಕಳಾದ ಕಥೆ ಮನೋಹರವಾಗಿರುವಂತೆ ಮಂಗಳಕರ ವಾಗಿಯೂ ಇದೆ.

ಅನುಸೂಯೆಯ ಗಂಡ ಅತ್ರಿಮಹರ್ಷಿ. ಬ್ರಹ್ಮಜ್ಞಾನಿಗಳಲ್ಲಿ ಅಗ್ರಗಣ್ಯ, ಆತ. ಚತುರ್ಮುಖ ಬ್ರಹ್ಮನ ಅಪ್ಪಣೆಯಂತೆ ಪ್ರಜಾಸೃಷ್ಟಿಕಾರ್ಯವನ್ನು ಕೈಕೊಳ್ಳುವ ಮುನ್ನ ಆತನು ತನ್ನ ಮಡದಿಯೊಡನೆ ಪುಕ್ಷಪರ್ವತಕ್ಕೆ ಹೋಗಿ ಅಲ್ಲಿ ತಪಸ್ಸನ್ನು ಆಚರಿಸಿದನು. ಪರ್ವತದಲ್ಲಿ ನಿರ್ವಿಂಧ್ಯೆಯೆಂಬ ಪವಿತ್ರವಾದ ನದಿಯೊಂದು ಹರಿಯುತ್ತಿತ್ತು. ಅದರ ದಡದಲ್ಲಿ ಹೂ ಹಣ್ಣುಗಳ ಗಿಡಮರಗಳು ದಟ್ಟವಾಗಿ ಬೆಳೆದಿದ್ದವು. ಆ ಕಾನನದ ಮಧ್ಯದಲ್ಲಿಆತನು ಒಂಟಿಗಾಲಲ್ಲಿ ನಿಂತು, ಆಹಾರ ನಿದ್ರೆಗಳನ್ನಾಗಲಿ, ಚಳಿ ಗಾಳಿ ಮಳೆ ಬಿಸಿಲುಗಳನ್ನಾಗಲಿ ಗಮನಿಸದೆ, ನೂರು ವರ್ಷಗಳ ಕಾಲ ತಪಸ್ಸುಮಾಡಿದನು. ಆ ಕಾಲದಲ್ಲಿ ಆತನು ಪ್ರಾಣಾಯಾಮದಿಂದ ಉಸಿರಾಡುವುದನ್ನು ನಿಲ್ಲಿಸಿ, ‘ಹೇ ಭಗವಂತ, ನಿನಗೆ ಶರಣಾಗತನಾಗಿದ್ದೇನೆ. ನಿನಗೆ ಸಮಾನನಾದ ಮಗನನ್ನು ನನಗೆ ಕರುಣಿಸು’ ಎಂದು ಒಂದೇ ಮನಸ್ಸಿನಿಂದ ಪ್ರಾರ್ಥಿಸಿದನು. ಆತನ ನಡುನೆತ್ತಿಯಿಂದ ತಪೋಜ್ವಾಲೆ ಹೊರಹೊರಟು, ಮೂರು ಲೋಕಗಳನ್ನೂ ಸುಡಲು ಪ್ರಾರಂಭಿಸಿತು. ಇದನ್ನು ಕಂಡು ಹೆದರಿದ ಸಿದ್ಧ ವಿದ್ಯಾಧರಾದಿ ದೇವತೆಗಳು ಬ್ರಹ್ಮ, ವಿಷ್ಣು, ಮಹೇಶ್ವರರೆಂಬ ತ್ರಿಮೂರ್ತಿ ಗಳೊಡನೆ ಆತನ ಬಳಿಗೆ ಬಂದರು. ಅವರನ್ನು ಕಂಡ ಪುಷಿ ತನ್ನ ತಪಸ್ಸನ್ನು ನಿಲ್ಲಿಸಿ, ಅವರಿಗೆ ಅಡ್ಡಬಿದ್ದನು. ಅವರನ್ನು ಪೂಜಿಸಲೆಂದು ತಲೆಯೆತ್ತಿ ನೋಡುತ್ತಾನೆ, ಎತ್ತನ್ನು ಹತ್ತಿಕೊಂಡು ತ್ರಿಶೂಲವನ್ನು ಹಿಡಿದಿರುವ ಆ ಪರಶಿವ, ಹಂಸವನ್ನೇರಿ ಕಮಂಡಲವನ್ನು ಹಿಡಿದಿರುವ ಆ ಬ್ರಹ್ಮ, ಗರುಡವಾಹನನಾಗಿ ಶಂಖಚಕ್ರಧಾರಿಯಾಗಿರುವ ಆ ಮಹಾ ವಿಷ್ಣು–ಇವರು ಬರಿಯ ತೇಜಸ್ಸಿನಂತೆ ಕಾಣುತ್ತಿದ್ದಾರೆ. ದಯಾರಸವನ್ನು ತುಳುಕಿಸು ತ್ತಿರುವ ಆ ಕಡೆಗಣ್​ನೋಟ, ಕಿರುನಗೆಯನ್ನು ಚೆಲ್ಲುತ್ತಿರುವ ಆ ಮುದ್ದುಮೊಗಗಳ ದಿವ್ಯತೇಜಸ್ಸು ಆತನ ಕಣ್ಣುಗಳನ್ನು ಕೋರೈಸುತ್ತಿರಲು, ಆತನು ಅರ್ಧ ಕಣ್ಮುಚ್ಚಿಕೊಂಡೆ ಅವರ ಪಾದಗಳಿಗೆ ಪೂಜೆಯನ್ನು ಸಲ್ಲಿಸಿದನು. ಅನಂತರ ಆತನು ಕೈಗಳನ್ನು ಜೋಡಿಸಿ ಕೊಂಡು “ಹೇ ದಿವ್ಯಪುರುಷರೆ, ನಿಮಗೆ ನಮಸ್ಕಾರ. ನನ್ನ ಕಣ್ಣುಗಳು ನಿಮ್ಮ ದಿವ್ಯ ತೇಜಸ್ಸನ್ನು ಕಾಣಲು ಶಕ್ತವಾಗಿಲ್ಲ. ನೀವು ಯಾರು? ಏಕೆ ಬಂದಿದ್ದೀರಿ? ನಾನು ಮಗ ನಾಗಬೇಕೆಂದು ಬಯಸಿ ಒಬ್ಬ ದೇವದೇವನನ್ನು ಬೇಡಿಕೊಂಡೆ. ನಿಮ್ಮ ಮೂವರಲ್ಲಿ ಅವನಾರೋ ನಾನರಿಯೆ! ಅವನು ಯಾರೆಂಬುದನ್ನು ನೀವೆ ನನಗೆ ತಿಳಿಸುವ ಕೃಪೆಮಾಡ ಬೇಕು. ಮನಸ್ಸಿಗೆ ಕೂಡ ಗೋಚರಿಸಲು ಅಸಾಧ್ಯರಾದ ಮೂವರೂ ಏಕಕಾಲದಲ್ಲಿ ಪ್ರತ್ಯಕ್ಷ ರಾದುದು ಪರಮಾಶ್ಚರ್ಯ!” ಎಂದನು.

ಅತ್ರಿಮುನಿಯ ಮಾತುಗಳಿಂದ ಸಂತಸಗೊಂಡ ತ್ರಿಮೂರ್ತಿಗಳು ಆತನೊಡನೆ ‘ಮಹರ್ಷಿ, ನೀನು ಯಾವ ದೇವದೇವನನ್ನು ಕುರಿತು ತಪಸ್ಸು ಮಾಡಿದೆಯೋ ಅವನೇ ನಾವು. ಬೇರಬೇರೆಯಾಗಿ ಮೂವರಂತೆ ಕಾಣಿಸಿದರೂ ನಾವು ಬೇರೆಬೇರೆಯಲ್ಲ, ಒಂದೇ ತತ್ವ. ನಿನ್ನ ಅಪೇಕ್ಷೆಯನ್ನು ಈಡೇರಿಸಲೆಂದು ನಾವು ಬಂದಿದ್ದೇವೆ. ನಮ್ಮಲ್ಲಿ ಭೇದವೆಣಿಸಬೇಡ. ನಮ್ಮ ಅಂಶದಿಂದ ನಿನಗೆ ಮೂವರು ಮಕ್ಕಳು ಹುಟ್ಟುತ್ತಾರೆ. ಆ ಮೂವರೂ ಲೋಕೋತ್ತರವಾದ ಕೀರ್ತಿಶಾಲಿಗಳಾಗಿ, ನಿನಗೂ ಕೀರ್ತಿಯನ್ನು ತರುತ್ತಾರೆ’ ಎಂದರು. ಇದನ್ನು ಕೇಳಿ ಸಂತೋಷಗೊಂಡ ಅತ್ರಿಮುನಿಯು ಅನಸೂಯೆಯೊಡನೆ ಅವ ರನ್ನು ಭಕ್ತಿಯಿಂದ ಸತ್ಕರಿಸುತ್ತಿರಲು, ನೋಡುನೋಡುತ್ತಿದ್ದಂತೆ ಅವರು ಮಾಯವಾಗಿ ಹೋದರು. ಅವರಿತ್ತ ವರದಂತೆ ಅನುಸೂಯೆಯಲ್ಲಿ ಮೂವರು ಮಕ್ಕಳು ಜನಿಸಿದರು. ಬ್ರಹ್ಮನ ವರದಿಂದ ಚಂದ್ರನೂ, ಈಶ್ವರನ ವರದಿಂದ ದುರ್ವಾಸನೂ ಹುಟ್ಟಿದರು. ಮಹಾವಿಷ್ಣುವಿನ ವರದಿಂದ ಹುಟ್ಟಿದವನೇ ದತ್ತಾತ್ರೇಯ. ಈ ಪುತ್ರನನ್ನು ಪಡೆದ ಅತ್ರಿ ಧನ್ಯನಾದ.

