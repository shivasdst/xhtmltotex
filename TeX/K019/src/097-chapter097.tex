
\chapter{ಪರಿಶಿಷ್ಟಗಳು}

\section{೧. ಮಹಾವಿಷ್ಣುವಿನ ಅವತಾರಗಳು}

ಭಗವಂತನು ತನ್ನ ಲೀಲೆಗಾಗಿ ಪ್ರಪಂಚವನ್ನು ಸೃಷ್ಟಿಸಹೊರಟು, ಏಕಾದಶೇಂದ್ರಿಯ ಗಳು (ಐದು ಜ್ಞಾನೇಂದ್ರಿಯಗಳು, ಐದು ಕರ್ಮೇಂದ್ರಿಯಗಳು ಮತ್ತು ಮನಸ್ಸು) ಮತ್ತು ಪಂಚಭೂತಗಳೆಂಬ ಹದಿನಾರು ಅವಯವಗಳಿಂದ ಕೂಡಿದ ಪುರುಷಾವತಾರವಾದನು. ಈ ಪುರುಷಾವತಾರ ಇತರ ಎಲ್ಲ ಅವತಾರಗಳಿಗೂ ಮೂಲವಾದುದು; ಎಲ್ಲ ಅವತಾರ ಗಳೂ ಕಡೆಯಲ್ಲಿ ಇದರಲ್ಲಿಯೇ ಲೀನವಾಗುತ್ತವೆ. ಎಲ್ಲೆಲ್ಲಿಯೂ ತುಂಬಿದ ಜಲರಾಶಿಯ ಮೇಲೆ ಯೋಗನಿದ್ರೆಯಲ್ಲಿ ಮಲಗಿದ್ದ ಈ ಪರಮಪುರುಷನ ನಾಭೀಕಮಲದಿಂದ, ಮರೀಚಿಯೇ ಮೊದಲಾದ ಒಂಬತ್ತು ಜನ ಬ್ರಹ್ಮರಿಗೆ ಅಧಿಪತಿಯಾದ ಚತುರ್ಮುಖ ಬ್ರಹ್ಮನು ಉದಯಿಸಿದನು; ಪರಮಪುರುಷನ ಅವಯವಗಳಾದ ಪಂಚಭೂತಗಳಿಂದ ಪ್ರಪಂಚ ಬೆಳೆಯಿತು. ವಿರಾಟ್​ರೂಪನಾದ ಈ ಪರಮಪುರುಷನೇ ಅನೇಕ ಅವತಾರಗಳನ್ನು ಎತ್ತಿದನು. 

೧. ಆ ವಿರಾಟ್​ಪುರುಷನು ಸನಕ, ಸನಂದನ, ಸನತ್ಸುಜಾತ, ಸನತ್ಕುಮಾರ ಎಂಬ ಬ್ರಾಹ್ಮಣರ ರೂಪದಲ್ಲಿ ಅವತರಿಸಿ, ಬ್ರಹ್ಮಚರ್ಯವ್ರತವನ್ನು ಪಾಲಿಸಿದನು. ಈ ಮೊದಲ ನೆಯ ಅವತಾರವನ್ನು \textbf{ಕೌಮಾರಾವತಾರ}ವೆಂದು ಕರೆಯುತ್ತಾರೆ. 

೨. ಎರಡನೆಯದು \textbf{ಆದಿ ವರಾಹಾವತಾರ} ಪಾತಾಳಕ್ಕೆ ಇಳಿದಿದ್ದ ಭೂಮಿಯನ್ನು ಉದ್ಧರಿಸುವುದಕ್ಕಾಗಿ ಈ ಅವತಾರವಾಯಿತು.

೩. ಮೂರನೆಯ ಸಲ \textbf{ನಾರದನ} ರೂಪಿನಿಂದ ಅವತರಿಸಿ, ಕರ್ಮಬಂಧವನ್ನು ಹೋಗ ಲಾಡಿಸುವ ಪಂಚರಾತ್ರಾಗಮವನ್ನು ರಚಿಸಿದನು.

೪. ನಾಲ್ಕನೆಯ ಸಲ ಧರ್ಮನ ಹೆಂಡತಿ ಮೂರ್ತಿದೇವಿಯಲ್ಲಿ \textbf{ನರ ನಾರಾಯಣ}ರೆಂಬ ಋಷಿಗಳಾಗಿ ಅವತರಿಸಿ, ಇತರರಿಗೆ ಅಸಾಧ್ಯವಾದ ತಪಸ್ಸನ್ನು ಆಚರಿಸಿದನು.

೫. ಐದನೆಯದು \textbf{ಕಪಿಲಮುನಿಯ} ಅವತಾರ. ಆಗ ಸಾಂಖ್ಯಾಶಾಸ್ತ್ರವನ್ನು ರಚಿಸಿದನು.

೬. ಆರನೆಯ ಅವತಾರದಲ್ಲಿ ಅತ್ರಿ-ಅನಸೂಯೆಯರ ಮಗನಾಗಿ \textbf{ದತ್ತಾತ್ರೇಯ}ನೆಂಬ ಹೆಸರಿನಿಂದ ಹುಟ್ಟಿ, ಅಲರ್ಕನೆಂಬ ರಾಜನಿಗೂ ಪ್ರಹ್ಲಾದನೇ ಮೊದಲಾದ ಭಕ್ತರಿಗೂ ಆತ್ಮವಿದ್ಯೆಯನ್ನು ಉಪದೇಶಮಾಡಿದನು.

೭. ಏಳನೆಯಸಾರಿ ರುಚಿ-ಆಕೂತಿಗಳ ಮಗನಾಗಿ ಜನಿಸಿ, \textbf{ಯಜ್ಞ}ನೆಂಬ ಹೆಸರಿನಿಂದ ಇಂದ್ರ ಪಟ್ಟವನ್ನು ವಹಿಸಿ, ಮೂರು ಲೋಕಗಳನ್ನೂ ಕಾಪಾಡಿದನು.

೮. ಎಂಟನೆಯದು \textbf{ಪುಷಭಾವತಾರ}; ನಾಭಿ-ಮೇರುದೇವಿಯರ ಮಗನಾಗಿ ಪರಮಹಂಸ ಮಾರ್ಗವನ್ನು ಲೋಕಕ್ಕೆ ಉಪದೇಶಿಸಿದನು.

೯. ಋಷಿಗಳ ಪ್ರಾರ್ಥನೆಯಂತೆ ಭಗವಂತನು \textbf{ಪೃಥು}ವೆಂಬ ರಾಜನ ರೂಪದಿಂದ ಅವತರಿಸಿ, ಗೋರೂಪದಲ್ಲಿದ್ದ ಭೂಮಿಯಿಂದ ಔಷಧಿಗಳೇ ಮೊದಲಾದ ಸಮಸ್ತ ವಸ್ತು ಗಳನ್ನೂ ಕರೆದನು.

೧೦. ಹತ್ತನೆಯದು \textbf{ಮತ್ಸ್ಯಾವತಾರ}; ಜಲಪ್ರಳಯದಲ್ಲಿ ನಾವೆಯಂತೆ ತೇಲುತ್ತಿದ್ದ ಭೂಮಿಯ ಮೇಲೆ ವೈವಸ್ವತ ಮನುವನ್ನು ಕುಳ್ಳಿರಿಸಿ, ಕಾಪಾಡಿದನು.

೧೧. ದೇವದಾನವರು ಸಮುದ್ರಮಥನ ಮಾಡುವಾಗ ಹನ್ನೊಂದನೆಯ \textbf{ಕೂರ್ಮಾವತಾರ}ವನ್ನು ತಾಳಿದ ಭಗವಂತನು ಮಂದರಪರ್ವತವನ್ನು ಬೆನ್ನಮೇಲೆ ಹೊತ್ತು ಭಕ್ತರನ್ನು ರಕ್ಷಿಸಿದನು.

೧೨. ಹನ್ನೆರಡನೆಯ ಅವತಾರದಲ್ಲಿ \textbf{ಧನ್ವಂತರಿ}ಯಾಗಿ ಜನಿಸಿ, ಕ್ಷೀರಸಮುದ್ರದಿಂದ ಅಮೃತದ ಕಲಶವನ್ನು ತಂದನು.

೧೩. ಹದಿಮೂರನೆಯ \textbf{ಮೋಹಿನಿ} ಅವತಾರದಲ್ಲಿ ರಾಕ್ಷಸರನ್ನು ಮರುಳುಮಾಡಿ, ದೇವತೆಗಳಿಗೆ ಅಮೃತವನ್ನು ಕೊಟ್ಟನು.

೧೪. \textbf{ನೃಸಿಂಹಾವತಾರ್}ಅ ಹದಿನಾಲ್ಕನೆಯದು; ಈ ಅವತಾರದಲ್ಲಿ ಹಿರಣ್ಯಕಶಿಪುವಿನ ವಧೆ ಯಾಯಿತು.

೧೫. ಹದಿನೈದನೆಯ \textbf{ವಾಮನಾವತಾರ್}ಅದಲ್ಲಿ ಬಲಿಯಿಂದ ಮೂರಡಿ ಭೂಮಿಯನ್ನು ಬೇಡಿದನು.

೧೬. ಹದಿನಾರನೆಯ ಅವತಾರದಲ್ಲಿ \textbf{ಪರಶುರಾಮ}ನಾಗಿ ಹುಟ್ಟಿ ಕ್ಷತ್ರಿಯರನ್ನು ನಿರ್ಮೂಲ ಮಾಡಿದನು.

೧೭. ಪರಾಶರ-ಸತ್ಯವತಿಯರ ಮಗನಾದ \textbf{ವ್ಯಾಸ್}ಅನೆಂಬ ಹೆಸರಿನಿಂದ ಅವತರಿಸಿ ವೇದ ಗಳನ್ನು ವಿಂಗಡಿಸಿದನು.

೧೮. ಹದಿನೆಂಟನೆಯದು \textbf{ರಾಮಾವತಾರ್}ಅ. ರಾವಣನ ಸಂಹಾರ ಈ ಅವತಾರದ ಮುಖ್ಯ ಕಾರ್ಯ.

 ೧೯. ಹತ್ತೊಂಬತ್ತರಲ್ಲಿ \textbf{ಬಲರಾಮ}ನಾಗಿಯೂ ಇಪ್ಪತ್ತರಲ್ಲಿ \textbf{ಶ್ರೀಕೃಷ್ಣ}ನಾಗಿಯೂ ಹುಟ್ಟಿ, ಕಂಸನೇ ಮೊದಲಾದ ರಾಕ್ಷಸರನ್ನು ಸಂಹರಿಸಿ, ಭೂಭಾರವನ್ನು ಇಳಿಸಿದನು.

೨೧. \textbf{ಬುದ್ಧಾವತಾರ;} ರಾಕ್ಷಸರನ್ನು ಮರುಳುಮಾಡಿದನು.

೨೨. \textbf{ಕಲ್ಕಿ ಅವತಾರ;}ರಾಜರಾದವರೇ ಚೋರವೃತ್ತಿಗೆ ಇಳಿದಿರಲು, ವಿಷ್ಣುಯಶನೆಂಬ ಬ್ರಾಹ್ಮಣನ ಮಗನಾಗಿ ಹುಟ್ಟಿ ದುಷ್ಟನಿಗ್ರಹವನ್ನು ಮಾಡುವನು.


\section{೨. ಭಾಗವತಕಾರನು ಕಂಡ ಬ್ರಹ್ಮಾಂಡ}

ಜಗತ್ತು ಭಗವಂತನ ಸ್ಥೂಲರೂಪ. ಅದನ್ನು ಅರಿತರೆ ಭಗವಂತನ ಸ್ಥೂಲರೂಪವನ್ನು ತಿಳಿಯಲು ಸಾಧ್ಯವಾಗುತ್ತದೆ. ಆದ್ದರಿಂದ ಈಶ್ವರಸ್ವರೂಪವನ್ನು ತಿಳಿಯಬಯಸುವವನು ಬ್ರಹ್ಮಾಂಡದ ಸ್ವರೂಪವನ್ನು ಮೊದಲು ಅರ್ಥಮಾಡಿಕೊಳ್ಳಬೇಕು. ನಾವು ಭಗವಂತ ನನ್ನು ‘ಅಖಿಲಾಂಡಕೋಟಿ ಬ್ರಹ್ಮಾಂಡನಾಯಕ’ ಎಂದು ಕರೆಯುತ್ತೇವೆ. ಏಕೆಂದರೆ ಪರ ಬ್ರಹ್ಮನ ಲೀಲಾವಿಭೂತಿಗಳಂತಿರುವ ಬ್ರಹ್ಮಾಂಡಗಳು ಅನಂತವಾಗಿವೆ. ದೇವತೆಗಳಂತೆ ಚಿರಾಯುವಾದರೂ ಅವುಗಳನ್ನು ಎಣಿಸುವುದು ಸಾಧ್ಯವಿಲ್ಲ. ಅಪರಿಮಿತವಾದ ಆ ಸೃಷ್ಟಿಯ ಸ್ಥೂಲಪರಿಚಯವನ್ನು ಮಾತ್ರ ನಾವೀಗ ಮಾಡಿಕೊಳ್ಳಲು ಪ್ರಯತ್ನಿಸೋಣ.

ಸೂರ್ಯನು ಬ್ರಹ್ಮಾಂಡದ ಕೇಂದ್ರಸ್ಥಾನದಲ್ಲಿದ್ದಾನೆ. ಆತನ ಸುತ್ತ ಇಪ್ಪತ್ತೈದುಕೋಟಿ ಯೋಜನಗಳ ಅಳತೆಯಲ್ಲಿ ಸೃಷ್ಟಿಯು ಹಬ್ಬಿದೆ. ಈ ಬ್ರಹ್ಮಾಂಡ ಕಾಲಕಾಲಕ್ಕೆ ನಾಶವಾಗ ತಕ್ಕುದು. ಆದ್ದರಿಂದಲೇ ಸೂರ್ಯನಿಗೆ ‘ಮಾರ್ತಾಂಡ’ಎಂದು ಹೆಸರು. ಈ ಮಾರ್ತಾಂಡನ ನೆಲೆಯಿಂದಲೆ ದಿಕ್ಕುಗಳ ಮತ್ತು ಲೋಕಗಳ ವಿಭಾಗಕಾರ್ಯ ನಡೆಯಬೇಕಾ ಗಿದೆ. ಲೋಕದ ಮಳೆ, ಗಾಳಿ, ಬಿಸಿಲುಗಳಿಗೆ ಈತನೇ ಕಾರಣ. ಮೂರು ಲೋಕಗಳಿಗೂ ಆತನಿಂದಲೆ ಬೆಳಕು. ದೇವ, ಮನುಷ್ಯ ಇತ್ಯಾದಿ ಸಕಲಪ್ರಾಣಿವರ್ಗಕ್ಕೂ, ಗಿಡ ಮರ ಬಳ್ಳಿ ಗಳಿಗೂ ಕೂಡ, ಆತನೇ ಆತ್ಮ, ಪ್ರಭು, ಕಣ್ಣು. ಮಾನಸೋತ್ತರ ಪರ್ವತದಲ್ಲಿ ನೆಲೆಸಿರುವ ಆತನು ನಿತ್ಯವೂ ಒಂಬತ್ತು ಕೋಟಿ ಐವತ್ತೊಂದು ಲಕ್ಷ ಯೋಜನೆಗಳಷ್ಟು ದೂರ ಸಂಚರಿ ಸುತ್ತಾನೆ. ಆ ಮಾನಸೋತ್ತರ ಪರ್ವತದ ಪೂರ್ವ ದಕ್ಷಿಣ ಪಶ್ಚಿಮ ಉತ್ತರಗಳಲ್ಲಿ ಕ್ರಮ ವಾಗಿ ಇಂದ್ರ ಯಮ ವರುಣ ಸೋಮರೆಂಬ ದೇವತೆಗಳ ಪಟ್ಟಣಗಳಿವೆ. ಈ ಪಟ್ಟಣ ಗಳನ್ನು ಸೂರ್ಯನು ಹಾದುಹೋಗುವಾಗ ಕ್ರಮವಾಗಿ ಬೆಳಗು, ಮಧ್ಯಾಹ್ನ, ಸಂಜೆ, ರಾತ್ರಿಗಳಾಗುತ್ತವೆ.

ಸೂರ್ಯಮಂಡಲಕ್ಕೆ ಒಂದು ಲಕ್ಷಯೋಜನ ದೂರದಲ್ಲಿ ಚಂದ್ರಮಂಡಲವಿದೆ. ಸೂರ್ಯ ನಿಗಿಂತ ಹೆಚ್ಚು ವೇಗದಿಂದ ಸಂಚರಿಸುವ ಈತನು ತನ್ನ ಶುಕ್ಲಪಕ್ಷ, ಕೃಷ್ಣಪಕ್ಷಗಳಿಂದ ಪಿತೃಲೋಕಕ್ಕೆ ಹಗಲು ರಾತ್ರಿಗಳನ್ನು ಒದಗಿಸುತ್ತಾನೆ. ಈತನು ಜೀವಿಗಳ ಮನಸ್ಸಿಗೆ ಅಧಿಪತಿಯಾಗಿ ‘ಮನೋಮಯ’ನೆನಿಸಿದ್ದಾನೆ; ಸಸ್ಯಗಳಿಗೆ ಅಧಿಪತಿಯಾಗಿ ‘ಅನ್ನಮಯ’; ಜೀವಕ್ಕೆ ಕಾರಣನಾಗಿ ‘ಅಮೃತಮಯ’; ಪ್ರಾಣಸ್ವರೂಪಿಯಾಗಿ ‘ಸರ್ವಮಯ’. ಚಂದ್ರ ಮಂಡಲದಿಂದ ಮೂರುಲಕ್ಷ ಯೋಜನದಾಚೆ ‘ಅಭಿಜಿತ್​’ ಎಂಬ ನಕ್ಷತ್ರಗಳನ್ನೊಳ ಗೊಂಡ ಇಪ್ಪತ್ತೆಂಟು ನಕ್ಷತ್ರಗಳ ಮಂಡಲವಿದೆ. ಅದರಾಚೆ ಎರಡು ಲಕ್ಷ ಯೋಜನದಲ್ಲಿ ಶುಕ್ರಗ್ರಹವಿದೆ; ಈತನ ದೆಸೆಯಿಂದ ಜಗತ್ತಿಗೆ ಮಳೆ ಬರುತ್ತದೆ. ಈತನಿಂದಾಚೆ ಎರಡು ಲಕ್ಷಯೋಜನದಲ್ಲಿ ಬುಧಗ್ರಹವಿದೆ; ಈತ ಸೂರ್ಯನಿಂದ ದೂರ ಹೋದಾಗ ಅತಿವೃಷ್ಟಿ, ಅನಾವೃಷ್ಟಿ, ಬಿರುಗಾಳಿಗಳು ಉಂಟಾಗುತ್ತವೆ. ಈತನಿಂದ ಎರಡು ಲಕ್ಷ ಯೋಜನದಾಚೆ ಅಂಗಾರಕ; ಈತ ಅಶುಭಗ್ರಹವಾದರೂ ವಕ್ರಗತಿಯಿಲ್ಲದಾಗ ಶುಭದಾಯಕ. ಈತನಿಂದ ಎರಡು ಲಕ್ಷ ಯೋಜನ ದಾಚೆ ಬೃಹಸ್ಪತಿ; ಈತ ಬ್ರಾಹ್ಮಣರಿಗೆ ಹಿತೈಷಿ. ಈತನಿಂದ ಎರಡು ಲಕ್ಷ ಯೋಜನದಾಚೆ ಶನಿಗ್ರಹ; ಈತ ಜನ್ಮರಾಶಿ ಅಥವಾ ಅದರ ಹಿಂದು ಮುಂದಿನ ರಾಶಿಗಳಲ್ಲಿ ಸಂಚರಿಸುವಾಗ ಜೀವಿಗಳಿಗೆ ಹಾನಿಕರ; ಈತನ ಏಳೂವರೆ ವರ್ಷಗಳ ಪೀಡೆಯು ಎಲ್ಲರಿಗೂ ಗೊತ್ತು. ಈತನಿಂದಾಚೆ ಹನ್ನೊಂದು ಲಕ್ಷ ಯೋಜನ ದೂರದಲ್ಲಿ ಸಪ್ತಪುಷಿ ಮಂಡಲವಿದೆ. ಇದರ ಮೇಲೆ ಎಲ್ಲಕ್ಕೂ ಎತ್ತರದಲ್ಲಿ ಧ್ರುವ ಮಂಡಲವಿದೆ. ಇದು ಮಹಾವಿಷ್ಣುವಿನ ವಾಸಸ್ಥಾನ. ಆ ದೇವದೇವನ ಯೋಗಶಕ್ತಿಯಿಂದ ಎಲ್ಲ ಗ್ರಹ ನಕ್ಷತ್ರಗಳೂ ಕಾಲಚಕ್ರಕ್ಕೆ ಸಿಕ್ಕಿ ಈ ಧ್ರುವಮಂಡಲವನ್ನು ಸುತ್ತುತ್ತವೆ. ಕಾಲಚಕ್ರಕ್ಕೆ ಶಿಶುಮಾರ ಚಕ್ರವೆಂದೂ ಹೆಸರು. ಇದನ್ನು ತ್ರಿಕಾಲಗಳಲ್ಲಿಯೂ ‘ನಮೋ ಜ್ಯೋತಿರ್ಲೋ ಕಾಯ ಕಾಲಾಽಯ ನಾಯಾನಿಮಿಷಾಂ ಪತಯೇ ಮಹಾಪುರುಷಾಯಾಽಭಿ ಧೀಮಹಿ’ ಎಂಬ ಮಂತ್ರದಿಂದ ಜಪಿಸಿದರೆ ಪಾಪಹರ. ಇದು ಸೂರ್ಯನಿಂದ ಮೇಲಿರುವ ಲೋಕಗಳ ಸ್ಥೂಲ ಪರಿಚಯ.

ದ್ವಿದಳಧಾನ್ಯದ ಒಂದು ಅರ್ಧ ಮೇಲಿನ ಲೋಕವಾದರೆ, ಮತ್ತೊಂದು ಅರ್ಧ ಕೆಳಲೋಕ. ಸೂರ್ಯ ಮಂಡಲದಿಂದ ಹತ್ತುಸಾವಿರ ಯೋಜನಗಳ ಕೆಳಗೆ ರಾಹುಗ್ರಹವಿದೆ; ಹದಿಮೂರು ಸಾವಿರ ಯೋಜನದಷ್ಟು ವಿಸ್ತಾರವಾದ ಈ ನೀಚಗ್ರಹ ಸೂರ್ಯ ಚಂದ್ರ ಗ್ರಹಣಗಳಿಗೆ ಕಾರಣವಾದುದು. ಇದರ ಕೆಳಗೆ ಹತ್ತುಸಾವಿರ ಯೋಜನ ದೂರದಲ್ಲಿ ಸಿದ್ಧ, ಚಾರಣ, ವಿದ್ಯಾಧರರ ಸ್ಥಾನವೂ ಅದರ ಕೆಳಗೆ ಯಕ್ಷ ರಾಕ್ಷಸ ಭೂತ ಪ್ರೇತಗಳು ಸಂಚರಿ ಸುವ ಅಂತರಿಕ್ಷವೂ, ಅದರ ಕೆಳಗೆ ಭೂಮಂಡಲವೂ ಇವೆ. ಭೂಮಂಡಲದ ಕೆಳಗೆ ಅತಳ, ವಿತಳ, ಸುತಳ, ರಸಾತಳ, ತಳಾತಳ, ಮಹಾತಳ, ಪಾತಾಳ ಎಂಬ ಏಳು ಲೋಕಗಳು ಒಂದರ ಕೆಳಗೊಂದಿವೆ. ಇವು ಒಂದೊಂದೂ ಹತ್ತು ಸಾವಿರ ಯೋಜನ ವಿಸ್ತಾರವಾಗಿವೆ. ಇವು ಸ್ವರ್ಗಕ್ಕಿಂತಲೂ ಹೆಚ್ಚು ಸುಖಕರ ಸ್ಥಾನಗಳು. ಇವುಗಳಲ್ಲಿ ದೈತ್ಯರೂ ದಾನವರೂ ನಾಗರೂ ತಮ್ಮ ಸಂಸಾರದೊಡನೆ ಸದಾ ಸುಖಭೋಗಗಳಲ್ಲಿ ಮುಳುಗಿರುತ್ತಾರೆ. ಮಾಯಾವಿಯಾದ ಮಾಯಾಸುರನೆಂಬುವನು ಈ ಲೋಕವಾಸಿಗಳಿಗೆ ಕೇಳಿದ ಸುಖಭೋಗ ಗಳನ್ನು ಒದಗಿಸುತ್ತಾನೆ. ಇಲ್ಲಿ ಹಗಲುರಾತ್ರಿಗಳಿಲ್ಲ; ಹಾವುಗಳ ಹೆಡೆಗಳ ರತ್ನದಿಂದ ಅಲ್ಲಿ ಸದಾ ಬೆಳಕೇ. ಇಲ್ಲಿನ ಜನರಿಗೆ ರೋಗವಿಲ್ಲ, ಮುಪ್ಪಿಲ್ಲ, ಮರಣವಿಲ್ಲ. ಪಾತಾಳದಿಂದ ಕೆಳಗೆ ಮೂವತ್ತು ಸಾವಿರ ಯೋಜನಗಳ ಆಳದಲ್ಲಿ ಸಾವಿರ ಹೆಡೆಗಳ ಅನಂತನಿದ್ದಾನೆ. ಆತನು ಭಗವಂತನ ಅಂಶ. ಆತನ ತಲೆಯಲ್ಲಿ ಇಡೀ ಭೂಮಂಡಲವು ಅಣುಮಾತ್ರದಂತೆ ನಿಂತಿದೆ. ನಾಗರೂ ನಾಗಕನ್ನಿಕೆಯರೂ ಆತನನ್ನು ಭಕ್ತಿಯಿಂದ ಸೇವಿಸುತ್ತಾರೆ. ಆತನು ಪ್ರಳಯಕಾಲದಲ್ಲಿ ಕೋಪದಿಂದ ತನ್ನ ಹುಬ್ಬುಗಳನ್ನು ಗಂಟು ಹಾಕಿಕೊಳ್ಳುತ್ತಾನೆ. ಆಗ ಸಂಕರ್ಷಣರೆಂಬ ಏಕಾದಶ ರುದ್ರರು ಆತನ ಹುಬ್ಬಿನ ಮಧ್ಯದಿಂದ ಹುಟ್ಟಿಬಂದು ಜಗತ್ತನ್ನೆಲ್ಲ ನಾಶಮಾಡುತ್ತಾರೆ.

ಭೂಮಂಡಲದ ದಕ್ಷಿಣ ದಿಗ್ಭಾಗದಲ್ಲಿ ಭೂಮಿಗಿಂತ ಕೆಳಗೆ, ಸಮುದ್ರಕ್ಕಿಂತ ಮೇಲೆ ನರಕವಿದೆ. ಯಮಧರ್ಮರಾಯನು ಅದರ ರಾಜ. ಜಗತ್ತಿನ ಜೀವರಾಶಿಗಳ ಪಾಪಕ್ಕೆ ಶಿಕ್ಷೆ ಕೊಡುವುದು ಈತನ ಕೆಲಸ. ಜೀವಿಗಳು ಅಲ್ಲಿನ ಭಯಂಕರ ಯಾತನೆಗಳನ್ನು ಅನುಭವಿ ಸಿದಮೇಲೆ ತಮ್ಮ ಉಳಿದ ಕರ್ಮವನ್ನು ಭೋಗಿಸುವುದಕ್ಕಾಗಿ ಜಗತ್ತಿನಲ್ಲಿ ಜನ್ಮ ತಾಳ ಬೇಕಾಗುತ್ತದೆ. 

ಇನ್ನು ಭೂಮಂಡಲದ ಸ್ವರೂಪವನ್ನು ಸ್ವಲ್ಪ ವಿವೇಚಿಸೋಣ. ಇದು ಏಳು ಮಹಾ ದ್ವೀಪಗಳು ಸೇರಿ ಆಗಿರುವ ಒಂದು ಭೂಭಾಗ. ಇವುಗಳಲ್ಲಿ ಮೊದಲನೆಯದೇ ನಾವು ವಾಸಿಸುತ್ತಿರುವ ಜಂಬೂದ್ವೀಪ. ಭೂಮಂಡಲಕ್ಕೆಲ್ಲ ಸೂರ್ಯನು ಕೇಂದ್ರದಲ್ಲಿರುವಂತೆ ಮೇರುಪರ್ವತವು ಇದರ ಕೇಂದ್ರದಲ್ಲಿದೆ. ಅದರ ನೆಲೆಯಿಂದಲೆ ಜಂಬೂದ್ವೀಪವನ್ನು ಒಂಬತ್ತು ‘ವರ್ಷ’ಗಳಾಗಿ ವಿಭಾಗಿಸುತ್ತಾರೆ. ಕುಲಪರ್ವತಗಳಲ್ಲೆಲ್ಲ ಅತ್ಯಂತ ಶ್ರೇಷ್ಠವಾದ ಈ ಮೇರುಪರ್ವತ ‘ಇಲಾವೃತ ವರ್ಷ’ದ ಮಧ್ಯದಲ್ಲಿ ನೆಲೆಸಿ, ಒಂದು ಲಕ್ಷ ಯೋಜನ ಎತ್ತರವಾಗಿದೆ. ಅದು ತುದಿಯಲ್ಲಿ ಮೂವತ್ತೆರಡು ಸಹಸ್ರ ಯೋಜನ, ಬುಡದಲ್ಲಿ ಹದಿ ನಾರು ಸಹಸ್ರ ಯೋಜನ ವಿಸ್ತಾರವಾಗಿದೆ; ಹದಿನಾರು ಸಹಸ್ರ ಯೋಜನದಷ್ಟು ಆಳವಾಗಿ ಭೂಮಿಯಲ್ಲಿ ನೆಲೆಸಿದೆ. ಇದರ ಉತ್ತರಕ್ಕೆ ರಮ್ಯಕವರ್ಷ, ಹಿರಣ್ಮಯ ವರ್ಷ ಮತ್ತು ಕುರುವರ್ಷವೆಂಬ ಮೂರು ವರ್ಷಗಳಿವೆ; ಇದರ ಪೂರ್ವಕ್ಕೆ ಭದ್ರಾಶ್ವವರ್ಷ; ಪಶ್ಚಿಮಕ್ಕೆ ಕೇತುಮೂಲವರ್ಷ; ದಕ್ಷಿಣಕ್ಕೆ ಹರಿವರ್ಷ, ಕಿಂಪುರುಷವರ್ಷ, ಭಾರತವರ್ಷ. ಈ ಒಂಬತ್ತು ವರ್ಷಗಳಲ್ಲಿಯೂ ಭಗವಂತನು ಬೇರೆಬೇರೆ ಹೆಸರಿನಿಂದ ನೆಲಸಿ, ಅಲ್ಲಿನವರಿಂದ ಭಕ್ತಿಯಿಂದ ಕಾಣಿಕೆಯನ್ನು ಪಡೆಯುತ್ತಿದ್ದಾನೆ.

\textbf{‘ಇಲಾವೃತ್’} ದಲ್ಲಿ ರುದ್ರನ ಹೊರತು ಬೇರೆ ಗಂಡು ಪೀಳಿಗೆಯೇ ಇಲ್ಲ, ಪಾರ್ವತಿಯ ಶಾಪದಿಂದ ಇಲ್ಲಿ ಪ್ರವೇಶಿಸಿದವರೆಲ್ಲರೂ ಸ್ತ್ರೀಯರೇ ಆಗುತ್ತಾರೆ. ಅಲ್ಲಿ ರುದ್ರನು ಪಾರ್ವತಿ ಮತ್ತು ಆಕೆಯ ದಾಸಿಯರಿಂದ ಸೇವೆಯನ್ನು ಕೈಕೊಳ್ಳುತ್ತಾ, ಭಗವಂತನ ಸಂಕರ್ಷಣ ಮೂರ್ತಿಯನ್ನು "ಓಂ ನಮೋ ಭಗವತೇ ಮಹಾಪುರುಷಾಯ ಸರ್ವಗುಣ ಸಂಖ್ಯಾನಾಯಽನಂತಾ ಯಾಽವ್ಯಕ್ತಾಯ ನಮಃ" - ಮಹಾಮಹಿಮನೂ ಸಕಲಗುಣ ಪ್ರಕಾಶನೂ ಪುರುಷೋತ್ತಮನೂ ನಾಶರಹಿತನೂ ಅಪರಿಚ್ಛಿನ್ನನೂ ಆದ ಭಗವಂತನಿಗೆ ನಮಸ್ಕಾರ - ಎಂಬ ಮಂತ್ರದಿಂದ ಜಪಿಸಿ ಸ್ತೋತ್ರಮಾಡುತ್ತಾನೆ.

\textbf{‘ಭದ್ರಾಶ್ವ ವರ್ಷ’}ಕ್ಕೆ ಭದ್ರಶ್ರವನೆಂಬುವನು ಒಡೆಯ. ಆತನೂ ಆತನ ವಂಶದವರೂ ವಿಷ್ಣುವಿನ ಹಯಗ್ರೀವಮೂರ್ತಿಯನ್ನು ಆರಾಧಿಸುತ್ತಾ “ಓಂ ನಮೋ ಭಗವತೇ ಧರ್ಮಾ ಯಾತ್ಮ ವಿಶೋಧನಾಯ ನಮಃ”–ಧರ್ಮ ಸ್ವರೂಪನೂ ಅಂತಃಕರಣ ಶೋಧಕನೂ ಆದ ಭಗವಂತನಿಗೆ ನಮಸ್ಕಾರ–ಎಂಬ ಮಂತ್ರವನ್ನು ಜಪಿಸಿ, ಕೀರ್ತಿಸುತ್ತಾರೆ. ‘ಹರಿವರ್ಷ’ ದಲ್ಲಿ ಪ್ರಹ್ಲಾದನು ತನ್ನ ಪ್ರಜೆಗಳೊಡನೆ ನರಸಿಂಹನ ಭಕ್ತನಾಗಿದ್ದು, ಆ ದೇವದೇವನನ್ನು “ಓಂ ನಮೋ ಭಗವತೇ ನಾರಸಿಂಹಾಯ ನಮಸ್ತೇಜ ಸ್ತೇಜಸ್ತೇ, ಆವಿರಾವಿರ್ಭವ ವಜ್ರನಖ! ವಜ್ರದಂಷ್ಟ್ರ! ಕರ್ಮಾಶಯಾನ್ ರಂಧಯ ರಂಧಯ ತಮೋ ಗ್ರಸ ಗ್ರಸ, ಓಂ ಸ್ವಾಹಾ ಅಭಯಮಭಯಮಾತ್ಮನಿ ಭೂಯಿಷ್ಠ” –ಸೂರ್ಯ ಚಂದ್ರ ಅಗ್ನಿಗಳಿಗೆ ತೇಜ ಸ್ಸನ್ನು ಕೊಡುವ ತೇಜೋರಾಶಿಯಾದ ಭಗವಂತ ಶ್ರೀನರಸಿಂಹಮೂರ್ತಿಗೆ ನಮಸ್ಕಾರ. ವಜ್ರದಂತೆ ಕಠಿನವಾದ ಉಗುರು ಮತ್ತು ಕೋರೆದಾಡೆಗಳುಳ್ಳ ಹೇ ಭಗವಂತ! ನಮ್ಮ ಮನಸ್ಸಿನಲ್ಲಿರುವ ಕರ್ಮವಾಸನೆಗಳನ್ನು ದಹಿಸು, ಅಜ್ಞಾನವನ್ನು ಹೋಗಲಾಡಿಸು, ಅಭಯ ವನ್ನು ಕೊಡು–ಎಂಬ ಮಂತ್ರವನ್ನು ಜಪಿಸುತ್ತಾ ಧ್ಯಾನ ಮಾಡುವನು; ಸದಾ ಆತನನ್ನು ಸ್ತೋತ್ರ ಮಾಡುತ್ತಿರುವವನು. \textbf{‘ಕೇತುಮೂಲವರ್ಷ’}ದಲ್ಲಿ ಸಂವತ್ಸರನೆಂಬುವನು ಅಧಿಪತಿ. ಆತನಿಗೆ ರಾತ್ರಿಯ ಅಭಿಮಾನ ದೇವತೆಗಳಾದ ಮೂವತ್ತಾರು ಸಾವಿರ ಹೆಣ್ಣು ಮಕ್ಕಳೂ, ಹಗಲಿಗೆ ಅಭಿಮಾನ ದೇವತೆಗಳಾದ ಮೂವತ್ತಾರು ಸಾವಿರ ಗಂಡುಮಕ್ಕಳೂ ಇರುವರು. ಮಹಾವಿಷ್ಣುವು ದಿವ್ಯಸುಂದರವಾದ ಪ್ರದ್ಯುಮ್ನರೂಪದಿಂದ ಅಲ್ಲಿ ನೆಲೆಸಿರು ವನು. ಆ ಮೂರ್ತಿಗೆ ಮನಸೋತ ಮಹಾಲಕ್ಷ್ಮಿ ರಾತ್ರಿಯಲ್ಲಿ ಹೆಣ್ಣುಮಕ್ಕಳೊಡನೆಯೂ ಹಗಲು ಗಂಡುಮಕ್ಕಳೊಡನೆಯೂ ಸೇರಿ ಅಲ್ಲಿಯೇ ನೆಲೆಸಿರುವಳು. ಆಕೆಯು “ಓಂ ಹ್ರಾಂ ಹ್ರೀಂ ಓಂ ನಮೋ ಭಗವತೇ ಹೃಷೀಕೇಶಾಯ ಸರ್ವಗುಣ ವಿಶೇಷೈರ್ವಿಲಕ್ಷಿತಾತ್ಮನೇ ಆಕೂತೀನಾಂ ಚಿತ್ತೀನಾಂ ಚೇತಸಾಂ ವಿಶೇಷಣಾಂ ಚಾಽಧಿಪತಯೇ ಷೋಡಶಕಲಾಯ ಛಂದೋಮಯಾಽನ್ನ ಮಯಾಯಾಽಮೃತಮಯಾಯ ಸಹಸೇ ಓಜಸೇ ಬಲಾಯ ಕಾಂತಾಯ ಕಾಮಾಯ ನಮಸ್ತೇ ಉಭ ಯತ್ರ ಭೂಯಾತ್​”–ಹೇ ಇಂದ್ರಿಯಪ್ರೇರಕ! ಸಕಲ ಸದ್ವಸ್ತುಲಕ್ಷ್ಯ! ಕ್ರಿಯೆ, ಜ್ಞಾನ, ಸಂಕಲ್ಪ, ನಿಶ್ಚಯಗಳಿಗೂ ಪೃಥಿವ್ಯಾಧಿ ಪಂಚಭೂತಗಳಿಗೂ ನಿಯಾಮಕನಾದವನೇ! ಜ್ಞಾನೇಂದ್ರಿಯ, ಕರ್ಮೇಂದ್ರಿಯ, ಮನಸ್ಸು, ಶಬ್ದಾದಿ ಪಂಚವಿಷಯಗಳೆಂಬ ಹದಿನಾರು ಅಂಗಗಳುಳ್ಳವನೆ! ವೇದೋಕ್ತ ಕರ್ಮಕ್ಕೆ ಒಲಿಯು ವವನೇ! ಹೇ ಮನಶ್ಶಕ್ತಿ ರೂಪ! ಹೇ ಇಂದ್ರಿಯ ಶಕ್ತಿರೂಪ! ಹೇ ಶರೀರ ಶಕ್ತಿರೂಪ! ಹೇ ಪರಮಪ್ರಿಯ! ಹೇ ಭಕ್ತಪ್ರೇಮಿ! ಹೇ ಭಗವಂತ! ನಿನಗೆ ಪ್ರತ್ಯಕ್ಷವಾಗಿಯೂ ಪರೋಕ್ಷ ವಾಗಿಯೂ ನಮಸ್ಕಾರ–ಎಂಬ ಮಂತ್ರವನ್ನು ಜಪಿಸುತ್ತಾ ಆತನನ್ನು ಸ್ತೋತ್ರ ಮಾಡು ವಳು. \textbf{‘ರಮ್ಯಕವರ್ಷ’}ದಲ್ಲಿ ಮನುಚಕ್ರವರ್ತಿಯು ತನ್ನ ಪ್ರಜೆಗಳೊಡನೆ ಭಗವಂತನ ಮತ್ಸ್ಯರೂಪವನ್ನು ಆರಾಧಿಸುತ್ತಾ ‘ಓಂ ನಮೋ ಭಗವತೇ ಮುಖ್ಯತಮಾಯ ನಮಃ ಸತ್ತ್ವಾಯ ಪ್ರಾಣಾ ಯೌಜಸೇ ಸಹಸೇ ಬಲಾಯ ಮಹಾಮತ್ಸ್ಯಾಯ ನಮಃ’–ಷಡ್ಗುಣ ಪರಿಪೂರ್ಣನೂ ಸರ್ವಗುಣ ಪ್ರಧಾನನೂ ಇಂದ್ರಿಯಶಕ್ತಿ, ಮನಶ್ಶಕ್ತಿ, ದೇಹಶಕ್ತಿ ಸ್ವರೂಪನೂ ಮಹಾಶರೀರಿಯಾಗಿ ಪ್ರಥಮಾವತಾರರೂಪಿಯಾಗಿರುವ ನಿನಗೆ ನಮಸ್ಕಾರ –ಎಂಬ ಮಂತ್ರವನ್ನು ಜಪಿಸುವನು. \textbf{‘ಹಿರಣ್ಮಯ}ದಲ್ಲಿ ಆರ್ಯಮನೆಂಬುವನು ತನ್ನ ಪ್ರಜೆಗಳೊಡನೆ ಕೂರ್ಮಾವತಾರವನ್ನು ಆರಾಧಿಸುವನು. ಆತ ಜಪಿಸುವ ಮಂತ್ರ ಇದು: "ಓಂ ನಮೋ ಭಗವತೇ ಅಕೂಪಾರಾಯ ಸರ್ವಸತ್ವಗುಣ ವಿಶೇಷಣಾಽನುಪಲಕ್ಷಿತಸ್ಥಾನಾಯ ನಮೋ ವರ್ಷ್ಮಣೇ ನಮೋ ಭೂಮ್ನೇ ನಮೋ ನಮೋಽವಸ್ಥಾನಾಯ ನಮಸ್ತೇ" - ಸಕಲ ಸತ್ವಗುಣಾಶ್ರಯನೂ ಯಾರಿಗೂ ತಿಳಿಯಲಾಗದ ನೆಲೆಯುಳ್ಳವನೂ ಸರ್ವ ಶ್ರೇಷ್ಠನೂ ಆದಿಮಧ್ಯಾಂತರಹಿತನೂ ಸರ್ವಾಶ್ರಯನೂ ಭಗವಂತನೂ ಆದ ಕೂರ್ಮ ಮೂರ್ತಿಗೆ ನಮಸ್ಕಾರ - ಎಂಬ ಮಂತ್ರದಿಂದ ಭಗವಂತನನ್ನು ಧ್ಯಾನಮಾಡಿ ಸ್ತುತಿಸುವನು. ಉತ್ತರ ಕುರುವರ್ಷದಲ್ಲಿ ಭಗವಂತ ವರಾಹರೂಪದಿಂದ ನೆಲೆಸಿರುವನು. ಭೂದೇವಿಯು ಅಲ್ಲಿ ಆತನನ್ನು ಪೂಜಿಸುತ್ತಾ "ಓಂ ನಮೋ ಭಗವತೇ ಮಂತ್ರತತ್ವಲಿಂಗಾಯ ಯಜ್ಞಕ್ರತವೇ ಮಹಾಽಧ್ವರಾಽವಯವಾಯ ಮಹಾಪುರಾಣಾಯ ನಮಃ ಕರ್ಮಶುಕ್ಲಾಯ ತ್ರಿಯುಗಾಯ ನಮಸ್ತೇ" - ಮಂತ್ರಗಳಿಂದ ಹುಡುಕುತ್ತಿರುವ ತತ್ವರೂಪಿಯಾಗಿ, ಯಜ್ಞಕ್ರತುಗಳ ಸ್ವರೂಪವೆನಿಸಿ, ಯಜ್ಞಕ್ರತುಗಳೇ ಅವಯವನಾಗಿ ಉಳ್ಳ ಯಜ್ಞಾನುಷ್ಠಾತೃವಾದ, ಯುಗತ್ರಯಸ್ವರೂಪನಾದ ಮಹಾಪುರುಷನಾದ ಭಗವಂತನಿಗೆ ನಮಸ್ಕಾರ - ಎಂಬ ಮಂತ್ರವನ್ನು ಜಪಿಸುತ್ತಾ ಸ್ತೋತ್ರಮಾಡುವಳು. \textbf{'ಕಿಂಪುರುಷವರ್ಷ'}ದಲ್ಲಿ ಆಂಜನೇಯನು ಅಲ್ಲಿನ ಕಿಂಪುರುಷರೊಡನೆ ಶ್ರೀರಾಮಮೂರ್ತಿಯನ್ನು "ಓಂ ನಮೋ ಭಗವತೇ ಉತ್ತಮಶ್ಲೋಕಾಯ ನಮಃ ಆರ್ಯಲಕ್ಷಣಶೀಲವ್ರತಾಯ ನಮಃ ಉಪಶಿಕ್ಷಿತಾತ್ಮನೇ ಉಪಾಸಿತ ಲೋಕಾಯ ನಮಃ ಸಾಧುವಾದ ನಿಕರ್ಷಣಾಯ ನಮೋ ಬ್ರಹ್ಮಣ್ಯದೇವಾಯ ಮಹಾಪುರುಷಾಯ ಮಹಾರಾಜಾಯ ನಮಃ" - ಬ್ರಹ್ಮಾದಿ ದೇವತೆಗಳಿಂದ ಸ್ತುತ್ಯನಾಗಿ, ಸಜ್ಞನ ಲಕ್ಷಣಗಳಿಂದಲೂ ನಡೆವಳಿಕೆಯಿಂದಲೂ ನಿಯಮಗಳಿಂದಲೂ ಕೂಡಿದವನಾಗಿಯೂ, ಮನಸ್ಸನ್ನು ನಿಗ್ರಹಿಸಿದವನಾಗಿಯೂ, ಸಕಲ ಲೋಕಾಶ್ರಯನಾಗಿಯೂ, ಸಾಧುತ್ವಕ್ಕೆ ಒರೆಗಲ್ಲಿನಂತೆ ನಿರ್ಣಯಸ್ಥಾನವಾಗಿಯೂ, ಬ್ರಾಹ್ಮಣಪ್ರಿಯನಾಗಿಯೂ ಇರುವ ಪುರುಷೋತ್ತಮನಾದ ಮಹಾರಾಜನಿಗೆ ನಮಸ್ಕಾರ - ಎಂಬ ಮಂತ್ರವನ್ನು ಜಪಿಸುತ್ತಾ ಸ್ತೋತ್ರಮಾಡುತ್ತಿರುವನು.

ಇನ್ನು ಕಡೆಯದಾಗಿ \textbf{ಭಾರತವರ್ಷ}. ಇದು ಕರ್ಮಕ್ಷೇತ್ರ. ಉಳಿದ ವರ್ಷಗಳೆಲ್ಲ ಭೋಗ ಕ್ಷೇತ್ರಗಳು; ಪುಣ್ಯಲೋಕದಿಂದ ಬಂದ ಜೀವಿಗಳು ತಮ್ಮ ಪುಣ್ಯದ ಉಳಿದ ಭಾಗವನ್ನು ಕಳೆಯುವುದಕ್ಕಾಗಿ ಅಲ್ಲಿ ಜನಿಸುತ್ತಾರೆ; ಅವು ಅನೇಕ ಸುಖಗಳ ಮಾತೃಸ್ಥಾನವಾದ್ದರಿಂದ ಅವುಗಳನ್ನು ಭೂ ಸ್ವರ್ಗಗಳೆನ್ನುತ್ತಾರೆ. ಆದರೆ ಕರ್ಮಕ್ಷೇತ್ರವಾದ ಭಾರತವರ್ಷದಲ್ಲಿ ಪಾಪ ಮತ್ತು ಪುಣ್ಯಕರ್ಮಗಳು ನಡೆಯಬೇಕಾಗಿದೆ. ಇಲ್ಲಿ ಭಗವಂತನು ಲೋಕಕಲ್ಯಾಣಕ್ಕಾಗಿ ನಾರಾಯಣರೂಪಿನಿಂದ ಬದರಿಕಾಶ್ರಮದಲ್ಲಿ ತಪಸ್ಸುಮಾಡುತ್ತಿದ್ದಾನೆ. ವರ್ಣಾ ಶ್ರಮಧರ್ಮಗಳನ್ನು ಆಚರಿಸುತ್ತಿರುವ ಇಲ್ಲಿನ ಜನರೊಡನೆ ಸೇರಿಕೊಂಡು, ನಾರದನು ಆತನನ್ನು “ಓಂ ನಮೋ ಭಗವತೇ ಉಪಶಮಶೀಲಾಯೋ ಪರತಾಽನಾತ್ಮಾಯ ನಮೋಽಕಿಂಚನವಿತ್ತಾಯ ಋಷಿ ಋಷಭಾಯ ನರನಾರಾಯಣಾಯ ಪರಮಹಂಸ ಪರಮಗುರುವೇ ಆತ್ಮಾರಾಧಿಪತಯೇ ನಮೋ ನಮಃ”–ಷಡ್ಗುಣ ಪರಿಪೂರ್ಣನೂ ಶಾಂತ ಮೂರ್ತಿಯೂ ಅಹಂಕಾರ ರಹಿತನೂ ಮುಮುಕ್ಷಗಳ ನಿಧಿಯೂ, ಋಷಿಪುಂಗವನೂ, ಪರಮಹಂಸರಿಗೆ ಗುರುವೂ, ಜ್ಞಾನಿಗಳಿಗೆ ಒಡೆಯನೂ ಆದ ನಾರಾಯಣಮುನಿಗೆ ನಮ ಸ್ಕಾರ–ಎಂದು ಧ್ಯಾನಿಸಿ ಸ್ತುತಿಸುತ್ತಿರುವನು. ಇಲ್ಲಿನ ಜನರು ತಮ್ಮ ಕರ್ಮಗಳಿಗೆ ತಕ್ಕಂತೆ ದೇವಮಾನವಾದಿ ಜನ್ಮಗಳನ್ನು ಪಡೆಯುತ್ತಾರೆ. ಇಲ್ಲಿ ಜನ್ಮವೆತ್ತಿಯೇ ಮೋಕ್ಷಸಾಧನೆ ಮಾಡಿಕೊಳ್ಳಬೇಕು. ಆದ್ದರಿಂದ ದೇವತೆಗಳು ಕೂಡ ಅಲ್ಲಿ ಹುಟ್ಟಬಯಸುತ್ತಾರೆ. ಇಲ್ಲಿ ಅನೇಕ ಪರ್ವತಗಳಿವೆ. ಪುಣ್ಯನದಿಗಳು ಹರಿಯುತ್ತವೆ.

ಒಂದು ಲಕ್ಷಯೋಜನ ವಿಸ್ತೀರ್ಣವಿರುವ ಜಂಬೂದ್ವೀಪದ ಸುತ್ತಲೂ ಅಷ್ಟೇ ವಿಸ್ತೀರ್ಣವುಳ್ಳ ಉಪ್ಪುನೀರಿನ ಸಮುದ್ರವಿದೆ. ಅದರ ಸುತ್ತಲೂ ಅದಕ್ಕೆ ಎರಡರಷ್ಟು ವಿಸ್ತಾರವಾದ \textbf{ಪ್ಲಕ್ಷದ್ವೀಪ}ವಿದೆ. ಇಲ್ಲಿರುವ ಆಲದ ಮರವು ಮಹಾಪ್ರಮಾಣಕ್ಕೆ ಬೆಳೆದಿರು ವುದರಿಂದ ಈ ದ್ವೀಪಕ್ಕೆ ಆ ಹೆಸರು ಬಂದಿದೆ. ಪ್ರಿಯವ್ರತನ ಮಗನಾದ ಇಧ್ಮಜಿಹ್ವನು ಅಲ್ಲಿಯ ದೊರೆ. ಅವನು ಆ ದ್ವೀಪವನ್ನು ತನ್ನ ಏಳು ಮಕ್ಕಳಿಗೆ ಸಮನಾಗಿ ಹಂಚಿ ಕೊಟ್ಟಿದ್ದಾನೆ. ಅವಕ್ಕೆ ‘ವರ್ಷ’ ಎಂದು ಹೆಸರು. ಒಂದೊಂದು ವರ್ಷದಲ್ಲಿಯೂ ಒಂದು ಮಹಾಪರ್ವತ, ಒಂದು ಮಹಾನದಿ ಇದೆ. ಇಲ್ಲಿ ಹಂಸ, ಪತಂಗ, ಊರ್ಧ್ವಾಯನ, ಸತ್ಯಾಂಗ–ಎಂಬ ನಾಲ್ಕುವರ್ಣದವರಿದ್ದಾರೆ. ಅವರ ಆಯುಸ್ಸು ಸಾವಿರ ವರ್ಷ. ಅವರು ಸೂರ್ಯನ ಭಕ್ತರು. “ಪ್ರತ್ನಸ್ಯ ವಿಷ್ಣೋರೂಪಂ ಯತ್ಸತ್ಯಸ್ಯ ಬ್ರಹ್ಮಣಃ ಅಮೃತಸ್ಯ ಚ ಮೃತ್ಯೋಶ್ಚ ಸೂರ್ಯಮಾತ್ಮಾನಮೀಮಹೀತಿ”–ಹೇ ಸೂರ್ಯಭಗವಂತ! ಪುರಾಣಪುರುಷ ನಾದ ಶ್ರೀಹರಿಯ ಸ್ವರೂಪನಾಗಿಯೂ, ನಾವು ಆಚರಿಸುತ್ತಿರುವ ಧರ್ಮಗಳಿಗೂ, ಶಾಸ್ತ್ರ ನಿರೂಪಿಸುವ ಧರ್ಮಗಳಿಗೂ, ಆ ಧರ್ಮವನ್ನು ಪ್ರತಿಪಾದಿಸುವ ವೇದಗಳಿಗೂ ಆಶ್ರಯ ನಾಗಿ ಪಾಪಪುಣ್ಯ ಫಲ ನಿಯಾಮಕನಾಗಿರುವ ನಿನಗೆ ಶರಣಾಗಿದ್ದೇನೆ–ಎಂಬ ಮಂತ್ರ ವನ್ನು ಧ್ಯಾನ ಮಾಡುವರು. ಇವರು ಪೂರ್ಣಾಯುಷಿಗಳಾಗಿ ಸದಾ ಸುಖಿಗಳಾಗಿರುವರು.

ಪ್ಲಕ್ಷದ್ವೀಪದ ಸುತ್ತ ಅದರಷ್ಟೆ ವಿಸ್ತಾರವಾದ ಇಕ್ಷುಸಮುದ್ರವಿದೆ. ಅದರ ಸುತ್ತ ಅದಕ್ಕೆ ಎರಡರಷ್ಟು ವಿಸ್ತಾರವಾದ \textbf{ಶಾಲ್ಮಲಿ} ದ್ವೀಪವಿದೆ. ಅಲ್ಲಿರುವ ಬೂರುಗದ ಮಹಾವೃಕ್ಷ ದಿಂದ ಅದಕ್ಕೆ ಆ ಹೆಸರು ಬಂದಿದೆ. ಅಲ್ಲಿಯೂ ಪ್ರಿಯವ್ರತರಾಜನ ಮತ್ತೊಬ್ಬ ಮಗ– ಯಜ್ಞಬಾಹು–ಅಧಿಪತಿಯಾಗಿ, ತನ್ನ ಏಳು ಮಕ್ಕಳಿಗೆ ಅದರ ಏಳುವರ್ಷಗಳನ್ನು ಹಂಚಿ ಕೊಟ್ಟಿದ್ದಾನೆ. ಇಲ್ಲಿಯೂ ಏಳು ಮಹಾಪರ್ವತಗಳೂ ಏಳು ಮಹಾನದಿಗಳೂ ಇವೆ. ಇಲ್ಲಿನ ಜನ ಶ್ರುತಧರ, ವಿದ್ಯಾಧರ, ವಸುಂಧರ, ಇಧ್ಮಧರರೆಂಬ ನಾಲ್ಕುವರ್ಣದವ ರಾಗಿದ್ದು ಚಂದ್ರನನ್ನು ಆರಾಧಿಸುತ್ತಾರೆ. “ಸ್ವಗೋಭಿಃ ಪಿತೃ ದೇವೇಭ್ಯೋ ವಿಭಜನ್ ಕೃಷ್ಣಶುಕ್ಲಯೋಃ । ಅಂಧಃ ಪ್ರಜಾನಾಂ ಸರ್ವಾಸಾಂ ರಾಜಾ ನ ಸ್ಸೇಮ ಅಸ್ತ್ವಿತಿ”–ತನ್ನ ಕಿರಣಗಳಿಂದ ಕೃಷ್ಣಪಕ್ಷದಲ್ಲಿ ಪಿತೃಗಳಿಗೂ ಶುಕ್ಲಪಕ್ಷದಲ್ಲಿ ದೇವತೆಗಳಿಗೂ ಅನ್ನವನ್ನು ವಿಭಾಗಿಸಿಕೊಡುವ ಚಂದ್ರನು ಪ್ರಜೆಗಳಾದ ನಮಗೆಲ್ಲ ಸುಖವನ್ನು ಕೊಡಲಿ–ಎಂದು ಧ್ಯಾನಮಾಡುತ್ತಾರೆ.

ಶಾಲ್ಮಲಿದ್ವೀಪದ ಸುತ್ತ ಅದಕ್ಕೆ ಎರಡರಷ್ಟು ವಿಸ್ತಾರವಾದ ಸುರೆಯ ಸಮುದ್ರವೂ, ಅದರ ಸುತ್ತ ಅದಕ್ಕೆ ಎರಡರಷ್ಟು ವಿಸ್ತಾರವಾದ \textbf{ಕುಶದ್ವೀಪ}ವೂ ಇವೆ. ಬ್ರಹ್ಮನಿರ್ಮಿತವಾದ ಒಂದು ದರ್ಭೆಯ ಗಿಡ ತನ್ನ ಕಾಂತಿಯಿಂದ ದಶದಿಕ್ಕುಗಳನ್ನು ಇಲ್ಲಿ ಬೆಳಗುತ್ತಿರುವುದರಿಂದ ಈ ದ್ವೀಪಕ್ಕೆ ಈ ಹೆಸರು ಬಂದಿದೆ. ಪ್ರಿಯವ್ರತರಾಜನ ಮತ್ತೊಬ್ಬ ಮಗನಾದ ಹಿರಣ್ಯಕನೆಂಬುವನು ಈ ದ್ವೀಪದ ಏಳುವರ್ಷಗಳನ್ನು ತನ್ನ ಮಕ್ಕಳಿಗೆ ಹಂಚಿದ್ದಾನೆ. ಇಲ್ಲಿಯೂ ಹಿಂದಿನಂತೆಯೇ ಗಿರಿ, ನದಿಗಳಿವೆ. ಇಲ್ಲಿನವರಲ್ಲಿ ಕುಶಲ, ಕೋವಿದ, ಅಭಿಯುಕ್ತ, ಕುಲಕ–ಎಂಬ ನಾಲ್ಕು ವರ್ಣಗಳಿವೆ. ಅಗ್ನಿ ಇವರ ಆರಾಧ್ಯದೈವ. “ಪರಸ್ಯ, ಬ್ರಹ್ಮಣಸ್ಸಾಕ್ಷಾಜ್ಜಾತ ವೇದೋಸಿ ಹವ್ಯವಾಟ್, ದೇವಾನಂ ಪುರುಷಾಂಗನಾಮ್ ಯಜ್ಞೇನ ಪುರುಷಂ ಯಜೇತಿ”–ಹೇ ಅಗ್ನಿಪುರುಷ, ನೀನು ಸಾಕ್ಷಾತ್ ಪರಮಾತ್ಮನಾದ ವಾಸು ದೇವನಿಗೆ ಹವಿರ್ ಭಾಗಗಳನ್ನು ಅರ್ಪಿಸುವನಾದ್ದರಿಂದ, ಆತನಿಗೆ ಅಂಗಭೂತರಾದ ಇಂದ್ರಾದಿಗಳನ್ನು ಉದ್ದೇಶಿಸಿ ನಾವು ಮಾಡುವ ಯಜ್ಞ ಶ್ರೀಹರಿಗೆ ಪ್ರೀತಿಯುಂಟಾಗುವಂತೆ ಮಾಡು–ಎಂಬ ಮಂತ್ರವನ್ನು ಜಪಿಸುತ್ತಾರೆ.

ಕುಶದ್ವೀಪದ ಸುತ್ತ ಅದರಷ್ಟೇ ವಿಸ್ತಾರವಾದ ತುಪ್ಪದ ಸಮುದ್ರವಿದೆ. ಆ ಸಮುದ್ರದ ಸುತ್ತ ಅದಕ್ಕೆ ಎರಡರಷ್ಟು ವಿಸ್ತಾರವಾದ \textbf{ಕ್ರೌಂಚದ್ವೀಪ}ವಿದೆ. ಇಲ್ಲಿರುವ ಕ್ರೌಂಚವೆಂಬ ಮಹಾಪರ್ವತದಿಂದ ಇದಕ್ಕೆ ಈ ಹೆಸರು. ಇಲ್ಲಿಯೂ ಪ್ರಿಯವ್ರತನ ಮಗ–ಘೃತ ಪೃಷ್ಠ–ರಾಜ. ಹಿಂದಿನಂತೆಯೇ ಅವನ ಏಳು ಮಕ್ಕಳು ಇಲ್ಲಿನ ಏಳು ವರ್ಷಗಳನ್ನೂ ಆಳುತ್ತಾರೆ. ಇಲ್ಲಿನವರು ಪುರುಷ, ಪುಷಭ, ದ್ರವಿಣ, ದೇವ–ಎಂಬ ನಾಲ್ಕು ವರ್ಣ ದವರಾಗಿದ್ದು ಜಲದೇವತೆಯ ಆರಾಧಕರಾಗಿದ್ದಾರೆ. “ಆಪಃ ಪುರುಷ ವಿರ್ಯಾಸ್ಥ ಪುನಂತೀ ಭೂರ್ಭುವಸ್ಸುವಃ । ತಾ ನಃ ಪುನಂತ್ವಮೀವಘ್ನೀಃ ಸ್ಪೃಶತಾಮಾತ್ಮನಾ ಭುವಃ”–ಹೇ ಜಲದೇವತೆಗಳೇ! ನೀವು ಪರಮಾತ್ಮನ ಅನುಗ್ರಹದಿಂದ ಸರ್ವಶಕ್ತರು; ನೀವು ಸ್ವಭಾವತಃ ಪಾಪನಾಶಕ ಶಕ್ತಿಯುಳ್ಳವರು; ಆದ್ದರಿಂದ ಭೂ, ಭುವ, ಸ್ವರ್ಗವೆಂಬ ಮೂರುಲೋಕ ಗಳನ್ನೂ ಪಾವನಗೊಳಿಸುವ ನೀವು ನಿಮ್ಮನ್ನು ಸೇವಿಸುವ ನಮ್ಮನ್ನು ಪರಿಶುದ್ಧ ಗೊಳಿಸಿರಿ–ಎಂದು ಧ್ಯಾನಮಾಡುತ್ತಾರೆ.

ಕ್ರೌಂಚ ದ್ವೀಪದ ಸುತ್ತ ಅದರಷ್ಟೆ ವಿಸ್ತಾರವಾದ ಮೊಸರಿನ ಸಮುದ್ರವಿದೆ. ಅದರ ಸುತ್ತ ಅದಕ್ಕೆ ಎರಡರಷ್ಟು ವಿಸ್ತಾರವಾದ \textbf{ಶಾಕದ್ವೀಪ}ವಿದೆ. ಅಲ್ಲಿರುವ ‘ಶಾಕ’ವೆಂಬ ಮರ ದಿಂದ ಅದಕ್ಕೆ ಆ ಹೆಸರು ಬಂದಿದೆ. ಅಲ್ಲಿಯೂ ಮೇಧಾತಿಥಿಯೆಂಬ ಪ್ರಿಯವ್ರತಪುತ್ರನೇ ರಾಜ. ಉಳಿದೆಲ್ಲ ವಿಚಾರಗಳೂ ಹಿಂದಿನಂತೆಯೇ. ಆದರೆ ಇಲ್ಲಿನ ವರ್ಣಗಳಿಗೆ ಪುತ ವ್ರತ, ಸತ್ಯವ್ರತ, ದಾನವ್ರತ, ಅನುವ್ರತ–ಎಂದು ಹೆಸರು; ವಾಯುದೇವರು ಇಲ್ಲಿನ ಆರಾಧ್ಯದೈವ. ಆತನನ್ನು “ಅಂತಃ ಪ್ರವಿಶ್ಯ ಭೂತಾನಿ ಯೋಭಿ ಭರ್ತ್ಯಾತ್ಮಕೇತುಭಿಃ । ಅಂತರ್ಯಾಮೀಶ್ವರಸ್ಸಾಕ್ಷಾತ್ಪಾತು ನೋಯದ್ವಶೇ ಸ್ಫುಟಂ”–ಯಾರು ಸಕಲ ಪ್ರಾಣಿ ಗಳನ್ನೂ ಒಳಹೊಕ್ಕು, ಶ್ರುತಿಯಲ್ಲಿ ಹೇಳುವಂತೆ ತನ್ನ ಪ್ರಾಣಾದಿವೃತ್ತಿಗಳಿಂದ ಜಗತ್ತನ್ನು ಸಲಹುತ್ತಿರುವನೋ, ಯಾರಿಗೆ ಈ ಜಗತ್ತು ಸಂಪೂರ್ಣ ಅಧೀನವೆನಿಸಿರುವುದೋ ಅಂತಹ ಸರ್ವಾಂತರ್ಯಾಮಿಯಾದ ವಾಯುರೂಪೀ ಪರಮಾತ್ಮನು ಪ್ರತ್ಯಕ್ಷವಾಗಿ ನಮ್ಮನ್ನು ಕಾಪಾ ಡಲಿ–ಎಂಬ ಮಂತ್ರದಿಂದ ಧ್ಯಾನಿಸುತ್ತಾರೆ.

ಶಾಕದ್ವೀಪದ ಸುತ್ತ ವಿಸ್ತಾರವಾದ ಹಾಲಿನ ಸಮುದ್ರವಿದೆ. ಅದರ ಸುತ್ತ ಅದಕ್ಕೆ ಎರಡ ರಷ್ಟು ವಿಸ್ತಾರವಾದ \textbf{ಪುಷ್ಕರದ್ವೀಪ}ವಿದೆ. ಇಲ್ಲಿ ಸಹಸ್ರದಳಗಳಿಂದ ಕೂಡಿ ಅಗ್ನಿಜ್ವಾಲೆ ಯಂತೆ ಪ್ರಕಾಶಮಾನವಾದ ಒಂದು ಕಮಲವಿದೆ. ಅದರಿಂದಲೇ ಆ ದ್ಪೀಪಕ್ಕೆ ಈ ಹೆಸರು ಬಂದಿರುವುದು. ಆ ದ್ವೀಪದ ಮಧ್ಯದಲ್ಲಿರುವ ಮಾನಸೋತ್ತರ ಪರ್ವತದಲ್ಲಿ ಇಂದ್ರ, ಯಮ, ವರುಣ, ಕುಬೇರರ ರಾಜಧಾನಿಗಳಿವೆ. ಇಲ್ಲಿಯೂ ಪ್ರಿಯವ್ರತ ಪುತ್ರನಾದ ಮಿತಿ ಹೋತ್ರ ರಾಜನಾಗಿ ತನ್ನ ಇಬ್ಬರು ಮಕ್ಕಳಿಗೆ ಅದನ್ನು ಹಂಚಿಕೊಟ್ಟಿದ್ದಾನೆ. ಅಲ್ಲಿನ ಜನ ಬ್ರಹ್ಮನ ಭಕ್ತರು, “ಯತ್ತತ್ಕರ್ಮ ಮಯಂ ಲಿಂಗಂ ಬ್ರಹ್ಮಲಿಂಗಂ ಜನೋಽಚರ್ಯೆತ್​ । ಏಕಾಂತಮದ್ವಯಂ ಶಾಂತಂ ತಸ್ಮೈ ಭಗವತೇ ನಮಃ”–ಕರ್ಮಫಲ ಸ್ವರೂಪನೂ ಬ್ರಹ್ಮಸ್ವರೂಪ ಪ್ರಕಾಶನೂ, ವೇದಪ್ರತಿಪಾದ್ಯನೂ, ಅದ್ವಿತೀಯ ಪರಮೇ ಶ್ವರನಲ್ಲಿ ಏಕಾಗ್ರಮನಸ್ಕನೂ, ಅದ್ವಿತೀಯನೂ, ರಾಗರಹಿತನೂ, ಜನಸೇವ್ಯನೂ ಆದ ಬ್ರಹ್ಮನಿಗೆ ನಮಸ್ಕಾರ–ಎಂಬ ಮಂತ್ರವನ್ನು ಜಪಿಸುತ್ತಾರೆ.

ಪುಷ್ಕರ ದ್ವೀಪದ ಸುತ್ತ ಸಿಹಿನೀರಿನ ಸಮುದ್ರವಿದೆ. ಇದರ ಸುತ್ತ ಲೋಕಾಲೋಕವೆಂಬ ಪರ್ವತವು ಕೋಟೆಯಂತೆ ಹಬ್ಬಿದೆ. ಅದರಿಂದ ಆಚೆಗೆ ಜನ ಸಂಚಾರವಿಲ್ಲದುದರಿಂದ ಅದಕ್ಕೆ ಅಲೋಕವೆಂದು ಹೆಸರು; ಆ ಪರ್ವತ ಧ್ರುವನಕ್ಷತ್ರಕ್ಕಿಂತಲೂ ಎತ್ತರವಾಗಿರುವುದ ರಿಂದ ಯಾವ ನಕ್ಷತ್ರಕಾಂತಿಯೂ ಅದನ್ನು ದಾಟಿಹೋಗಲಾರದು. ಅಲೋಕದಲ್ಲಿ ಕಗ್ಗತ್ತಲು ತುಂಬಿದೆ. ದೇವದೇವನಾದ ಭಗವಂತನು ಲೋಕಾಲೋಕ ಪರ್ವತದ ಮೇಲೆ ನೆಲೆಸಿದ್ದಾನೆ.

\begin{center}
ಇದು ಭೂಮಂಡಲದ ಸ್ಥೂಲ ಚಿತ್ರ.
\end{center}


\section{೩. ನಾರಾಯಣ ಕವಚ}

ವಿಶ್ವರೂಪನು ಇಂದ್ರನಿಗೆ ನಾರಾಯಣ ಕವಚವನ್ನು ಉಪದೇಶಮಾಡಿದನು. ಇದ ರಿಂದ ಆತ ರಾಕ್ಷಸರನ್ನು ಗೆದ್ದು ಮತ್ತೆ ಇಂದ್ರಪದವಿಯ ವೈಭವಗಳನ್ನು ಗಳಿಸಿಕೊಂಡನು. ಇದನ್ನು ಉಪಾಸನೆ ಮಾಡುವವನು ಶುಚಿರ್ಭೂತನಾಗಿ, ಆಚಮನ ಮಾಡಿ ಅಷ್ಟಾಕ್ಷರ ದ್ವಾದಶಾಕ್ಷರಗಳಿಂದ ಅಂಗನ್ಯಾಸ ಕರನ್ಯಾಸಗಳನ್ನು ಮಾಡಿಕೊಳ್ಳಬೇಕು.

\begin{verse}
ಹರಿರ್ವಿದಧ್ಯಾನ್ಮಮ ಸರ್ವರಕ್ಷಾಂ ।\\ನ್ಯಸ್ತಾಂಘ್ರಿಪದ್ಮಃ ಪತಗೇಂದ್ರಪೃಷ್ಠೇ ॥\\ದರಾಽರಿಚರ್ಮಽಸಿಗದೇಷುಚಾಪ ।\\ಪಾಶಾನ್ ದಧಾನೋಽಷ್ಟಗುಣೋಷ್ಟಬಾಹುಃ \num{ ॥ ೧ ॥}
\end{verse}

ಗರುಡನ ಹೆಗಲಲ್ಲಿ ಪಾದಕಮಲವನ್ನೂರಿ, ಶಂಖ ಚಕ್ರ ಚರ್ಮ ಖಡ್ಗ ಗದೆ ಬಾಣ ಬಿಲ್ಲು ಪಾಶ–ಇವನ್ನು ಎಂಟು ಭುಜಗಳಲ್ಲಿ ಧರಿಸಿ, ಅಣಿಮಾದಿ ಶಕ್ತಿಗಳಿಂದ ಯುಕ್ತ ನಾದ ಶ್ರೀಹರಿಯು ನನ್ನನ್ನು ಎಲ್ಲ ಕಡೆಯೂ ಕಾಪಾಡಲಿ!

\begin{verse}
ಜಲೇಷು ಮಾಂ ರಕ್ಷತು ಮತ್ಸ್ಯಮೂರ್ತಿಃ ।\\ಯಾದೋ ಗಣೇಭ್ಯೋ ವರುಣಸ್ಯ ಪಾಶಾತ್ ॥\\ಸ್ಥಲೇಷು ಮಾಯಾವಟುವಾಮನೋಽವ್ಯಾತ್ ।\\ತ್ರಿವಿಕ್ರಮಃ ಖೇಽವತು ವಿಶ್ವರೂಪಃ \num{ ॥ ೨ ॥}
\end{verse}

ಮತ್ಸ್ಯರೂಪನಾದ ಭಗವಂತನು ಜಲಜಂತುಗಳಿಂದಲೂ ವರುಣನ ಪಾಶದಿಂದಲೂ ನನ್ನನ್ನು ಸಲಹಲಿ! ಕಪಟ ಬ್ರಹ್ಮಚಾರಿಯಾದ ವಾಮನನು ಭೂಮಿಯ ಮೇಲೆ ನನ್ನನ್ನು ಕಾಪಾಡಲಿ. ಆಕಾಶದಲ್ಲಿ ಸರ್ವಾತ್ಮಕ ತ್ರಿವಿಕ್ರಮನು ರಕ್ಷಿಸಲಿ!

\begin{verse}
ದುರ್ಗೇಷ್ವಟವ್ಯಾಜಿಮುಖಾದಿಷು ಪ್ರಭುಃ ।\\ಪಾಯಾನ್ನೃಸಿಂಹೋಽಸುರಯೂಥಪಾರಿಃ ॥\\ವಿಮುಂಚತೋ ಯಸ್ಯ ಮಹಾಟ್ಟಹಾಸಂ ।\\ದಿಶೋ ವಿನೇದುರ್ನ್ಯಪತಂಶ್ಚ ಗರ್ಭಾಃ \num{॥ ೩ ॥}
\end{verse}

ತನ್ನ ಅಟ್ಟಹಾಸದಿಂದ ದಶದಿಕ್ಕುಗಳೂ ಪ್ರತಿಧ್ವನಿಗೊಳ್ಳುವಂತೆ ಮಾಡಿ, ಅದರಿಂದ ಗರ್ಭಿಣಿಯರ ಗರ್ಭಗಳು ಕೆಳಕ್ಕೆ ಬೀಳುವಂತಾಗುವ, ರಾಕ್ಷಸರಾಜನಾದ ಹಿರಣ್ಯಕಶಿಪುವಿನ ಶತ್ರುವಾದ ನರಸಿಂಹ ದೇವರು ಅರಣ್ಯದಲ್ಲಿಯೂ, ಯುದ್ಧದಲ್ಲಿಯೂ, ಇತರ ಆಪತ್ತುಗಳಲ್ಲಿಯೂ ನಮ್ಮನ್ನು ರಕ್ಷಿಸಲಿ!

\begin{verse}
ರಕ್ಷತ್ವಸೌ ಮಾಽಧ್ವನಿ ಯಜ್ಞಕಲ್ಪ ।\\ಸ್ವದಂಷ್ಟ್ರಯೋನ್ನಿತಧರೋ ವರಾಹಃ ॥\\ರಾಮೋಽದ್ರಿಕೂಟೇಷ್ವಥ ವಿಪ್ರವಾಸೇ ।\\ಸಲಕ್ಷ್ಮಣೋಽವ್ಯಾದ್ಭರತಾಗ್ರಜೋ ಮಾಂ \num{॥ ೪ ॥}
\end{verse}

ಯಜ್ಞರೂಪದ ಅವಯವಗಳುಳ್ಳವನಾಗಿ, ಸಾಗರದಲ್ಲಿ ನುಗ್ಗಿ ಪಾತಾಳಕ್ಕಿಳಿದಿದ್ದ ಭೂಮಿಯನ್ನು ಕೋರೆದಾಡೆಗಳಿಂದ ಮೇಲಕ್ಕೆತ್ತಿ ತಂದ ವರಾಹಮೂರ್ತಿಯು ಪಯಣದ ಹಾದಿಯಲ್ಲಿ ನಮ್ಮನ್ನು ರಕ್ಷಿಸಲಿ! ಪರಶುರಾಮನು ಪರ್ವತಶಿಖರದಲ್ಲಿ ನಮ್ಮನ್ನು ರಕ್ಷಿ ಸಲಿ. ಲಕ್ಷ್ಮಣನೊಡಗೂಡಿದ, ಭರತಾಗ್ರಜನಾದ ಶ್ರೀರಾಮನು ಪ್ರಯಾಣಕಾಲದಲ್ಲಿ ನಮ್ಮನ್ನು ರಕ್ಷಿಸಲಿ!

\begin{verse}
ಮಾಮುಗ್ರಧರ್ಮಾದಖಿಲಾತ್ಪ್ರಮಾದ ।\\ನ್ನಾರಾಯಣಃ ಪಾತು ನರಶ್ಚ ಹಾಸಾತ್ ॥\\ದತ್ತಸ್ತ್ವಯೋಗಾದಥ ಯೋಗನಾಥಃ ।\\ಪಾಯುದ್ಗುಣೇಶಃ ಕಪಿಲಃ ಕರ್ಮಬಂಧಾತ್ \num{॥ ೫ ॥}
\end{verse}

ಶ್ರೀ ನಾರಾಯಣನು ಅಭಿಚಾರವೇ ಮೊದಲಾದ ಪ್ರಮಾದಗಳಿಂದ ನಮ್ಮನ್ನು ರಕ್ಷಿಸಲಿ! ನರ ಮೂರ್ತಿಯು ದುರಹಂಕಾರದಿಂದ ತಪ್ಪಿಸಿ ನಮ್ಮನ್ನು ರಕ್ಷಿಸಲಿ! ಯೋಗೇಶ್ವರನಾದ ದತ್ತಾತ್ರೇಯನು ಯೋಗ ಕೆಡದಂತೆ ನಮ್ಮನ್ನು ರಕ್ಷಿಸಲಿ! ಸತ್ವಾದಿ ಗುಣರೂಪದ ಪ್ರಕೃತಿಗೆ ಒಡೆಯನಾದ ಕಪಿಲಪುಷಿ ಕರ್ಮಬಂಧವಾಗದಂತೆ ನಮ್ಮನ್ನು ಕಾಪಾಡಲಿ!

\begin{verse}
ಸನತ್ಕುಮಾರೋಽವತು ಕಾಮದೇವಾತ್ ।\\ಹಯಾನನೋ ಮಾಂ ಪಥಿ ದೇವಹೇಳನಾತ್ ॥\\ದೇವರ್ಷಿವರ್ಯಃ ಪುರುಷಾರ್ಚನಾಂತರಾತ್ ।\\ಕೂರ್ಮೋ ಹರಿರ್ಮಾಂ ನಿರಯಾದಶೇಷಾತ್ \num{॥ ೬ ॥}
\end{verse}

ಜಿತೇಂದ್ರಿಯನಾದ ಸನತ್ಕುಮಾರನು ಕಾಮವಿಕಾರವಿಲ್ಲದಂತೆಯೂ, ಹಯಗ್ರೀವನು ದೇವರಿಗೆ ನಮಸ್ಕಾರಮಾಡದ ಪಾಪವನ್ನು ಸಂಪಾದಿಸಿಕೊಳ್ಳದಂತೆಯೂ, ನಾರದನು ಭಗ ವಂತನ ಆರಾಧನೆಗೆ ವಿಘ್ನವಾಗದಂತೆಯೂ, ಕೂರ್ಮನು ನರಬಾಧೆಯಾಗದಂತೆಯೂ ನಮ್ಮನ್ನು ರಕ್ಷಿಸಲಿ!

\begin{verse}
ಧನ್ವಂತರಿರ್ಭಗವಾನ್ ಪಾತ್ವಪಥ್ಯಾತ್ ।\\ದ್ವಂದ್ವಾದ್ಭಯಾದೃಷಭೋನಿರ್ಜಿತಾತ್ಮಾ ॥\\ಯಜ್ಞಶ್ಚ ಲೋಕಾದವತಾಜ್ಜನಾಂತಾತ್ ।\\ಬಲೋ ಗಣಾತ್ ಕ್ರೋಧವಶಾದಹೀಂದ್ರಃ \num{॥ ೭ ॥}
\end{verse}

ಭಗವಾನ್ ಧನ್ವಂತರಿಯು ಅಹಿತ ಪದಾರ್ಥಗಳನ್ನು ತಿನ್ನದಂತೆ, ಮಹಾ ತಪಸ್ವಿಯಾದ ಪುಷಭನು ಸುಖದುಃಖಾದಿ ದ್ವಂದ್ವಗಳಿಲ್ಲದಂತೆ, ಯಜ್ಞಮೂರ್ತಿಯು ಜನಾಪವಾದ ಬರದಂತೆ, ಬಲರಾಮನು ಜನರಿಂದ ಬಾಧೆಯಾಗದಂತೆ, ಆದಿಶೇಷನು ಮಹಾಕೋಪವುಳ್ಳ ಸರ್ಪಗಳ ಕಾಟವಿಲ್ಲದಂತೆ ನಮ್ಮನ್ನು ರಕ್ಷಿಸಲಿ!

\begin{verse}
ದ್ವೈಪಾಯನೋ ಭಗವಾನ್ ಸಂಪ್ರಮೋಹಾತ್ ।\\ಬದ್ಧಶ್ಚ ಪಾಶಂ ಚ ಗಣಾತ್ಪ್ರಮಾದಾತ್ ॥\\ಕಲ್ಕೀ ಕಲೇಃ ಕಾಲಮಲಾತ್ಪ್ರಪಾತು ।\\ಧರ್ಮಾವನಾಯೋರುಕೃತಾವತಾರಃ \num{॥ ೮ ॥}
\end{verse}

ಭಗವಾನ್ ವ್ಯಾಸಮುನಿಯು ಅಜ್ಞಾನ ಮುಸುಕದಂತೆ, ಬುದ್ಧನು ಮೋಹಗೊಳಿಸುವ ಪಾಷಂಡರ ಜೊತೆ ಸೇರದಂತೆ, ಧರ್ಮರಕ್ಷಣೆಗಾಗಿ ಅವತರಿಸಿರುವ ಕಲ್ಕಿಯು ಕಾಲದೋಷ ರೂಪನಾದ ಕಲಿಯ ಕಾಟವಿಲ್ಲದಂತೆ ನಮ್ಮನ್ನು ರಕ್ಷಿಸಲಿ!

\begin{verse}
ಮಾಂ ಕೇಶವೋ ಗದಯಾ ಪ್ರಾತರವ್ಯಾತ್ ।\\ಗೋವಿಂದ ಅಸಂಗವಮಾತ್ತವೇಣುಃ ॥\\ನಾರಾಯಣಃ ಪ್ರಾಹ್ಣ ಉದಾತ್ತಶಕ್ತಿಃ ।\\ಮಾಧ್ಯಂದಿನೇ ವಿಷ್ಣುರರೀಂದ್ರಪಾಣಿಃ \num{॥ ೯ ॥}
\end{verse}

ಗದಾಧಾರಿ ಕೇಶವನು ಪ್ರಾತಃಕಾಲದಲ್ಲಿಯೂ, ವೇಣುಧರ ಗೋವಿಂದನು ಸಂಗಮ ಕಾಲದಲ್ಲಿಯೂ, ಮಹಾ ಪರಾಕ್ರಮಿ ನಾರಾಯಣನು ಪೂರ್ವಾಹ್ನದಲ್ಲಿಯೂ, ಚಕ್ರಪಾಣಿ ವಿಷ್ಣುವು ಮಧ್ಯಾಹ್ನದಲ್ಲಿಯೂ ನಮ್ಮನ್ನು ಸಲಹಲಿ!

\begin{verse}
ದೇವೋಽಪರಾಹ್ಣೇ ಮಧುಹೋಗ್ರಧನ್ವಾ ।\\ಸಾಯಂ ತ್ರಿಧಾಮಽವತು ಮಾಧವೋ ಮಾಂ ॥\\ದೋಷೇ ಹೃಷೀಕೇಶ ಉತಾರ್ಧರಾತ್ರೇ ।\\ನೀಶೀಥ ಏಕೋಽವತು ಪದ್ಮನಾಭಃ \num{॥ ೧೦॥}
\end{verse}

ಮಹಾ ಧನುರ್ಧಾರಿಯಾದ ಮಧುಸೂದನನು ಅಪರಾಹ್ನದಲ್ಲಿಯೂ, ಇಂದ್ರಿಯಾಧಿ ಪತಿ ಹೃಷೀಕೇಶನು ಪ್ರದೋಷದಲ್ಲಿಯೂ, ಪದ್ಮನಾಭನು ಮಧ್ಯರಾತ್ರಿಯಲ್ಲಿಯೂ ನಮ್ಮನ್ನು ರಕ್ಷಿಸಲಿ!

\begin{verse}
ಶ್ರೀವತ್ಸ ಧಾಮಾಪರರಾತ್ರ ಈಶಃ ।\\ಪ್ರತ್ಯೂಷ ಈಶೋಽಸಿಧರೋ ಜನಾರ್ದನಃ ॥\\ದಾಮೋದರೋಽವ್ಯಾದನುಸಂಧ್ಯಂ ಪ್ರಭಾತೇ ।\\ವಿಶ್ವೇಶ್ವರೋ ಭಗವಾನ್ ಕಾಲಮೂರ್ತಿ \num{॥ ೧೧॥}
\end{verse}

ಶ್ರೀವತ್ಸ ಚಿಹ್ನೆಯನ್ನು ಎದೆಯಲ್ಲಿ ಧರಿಸಿರುವ ಲೋಕೇಶ್ವರ ಶ್ರೀಹರಿಯು ಅಪರಾತ್ರಿ ಯಲ್ಲಿಯೂ, ಖಡ್ಗಧಾರಿಯಾದ ಜನಾರ್ದನನು ಬೆಳಗಿನ ಝಾವದಲ್ಲಿಯೂ, ದಾಮೋ ದರನು ಸಂಧ್ಯಾಕಾಲಗಳಲ್ಲಿಯೂ, ಭಗವಂತನೂ ಕಾಲಸ್ವರೂಪಿಯೂ ಆದ ವಿಶ್ವೇಶ್ವರನು ಪ್ರಾತಃಕಾಲದಲ್ಲಿಯೂ ನಮ್ಮನ್ನು ರಕ್ಷಿಸಲಿ!

\begin{verse}
ಚಕ್ರಂ ಯುಗಾಂತಾನಲ ತಿಗ್ಮನೇಮಿಃ ।\\ಭ್ರಮತ್ಸಮಂತಾದ್ಭಗವತ್ಪ್ರಯುಕ್ತಂ ॥\\ದಂ ದಗ್ಧಿದಂ ದಗ್ದೃಧಿಸೈನ್ಯಮಾಶು ।\\ಕಕ್ಷಂ ಯಥಾ ವಾಯುಸಖೋ ಹುತಾಶಃ \num{ ॥ ೧೨ ॥}
\end{verse}

ಎಲೈ ಸುದರ್ಶನ ಚಕ್ರವೇ! ಪ್ರಳಯಾಗ್ನಿಯಂತೆ ತೀವ್ರವಾದ ಅಂಚುಳ್ಳದಾಗಿ ಗರಗರ ತಿರುಗುತ್ತಿರುವ ನೀನು ಗಾಳಿಯೊಡನೆ ಸೇರಿದ ಬೆಂಕಿಯು ಒಣ ಹುಲ್ಲನ್ನು ಸುಡುವಂತೆ, ನನ್ನ ಹಗೆಗಳ ಗುಂಪನ್ನು ಬೇಗ ಬೇಗಸುಡು, ಸುಡು.

\begin{verse}
ಗದೇಽಶನಿ ಸ್ಪರ್ಶನವಿಷ್ಫುಲಿಂಗೇ ।\\ನಿಷ್ಪಿಂಡಿ ನಿಷ್ಪಿಂಡ್ಯಜಿತಪ್ತಿಯಾಸಿ ॥\\ಕೂಶ್ಮಾಂಡ ವೈನಾಯಕಯಕ್ಷರಕ್ಷೋ ।\\ಭೂತಗ್ರಹಾಂಶ್ಚೂರ್ಣಯ ಚೂರ್ಣಯಾಽರೀನ್ \num{॥ ೧೩ ॥}
\end{verse}

ಬರಸಿಡಿಲಿನಂತೆ ಭಯಂಕರವಾಗಿ ಬಂದೆರಗಬಲ್ಲ, ಕಿಡಿಗರೆವ ಹೇ ಗದಾಯುಧವೇ, ಭಗವಂತನಿಗೆ ಅತ್ಯಂತ ಪ್ರಿಯಕರನಾದ ನೀನು ಕೂಶ್ಮಾಂಡ, ವೈನಾಯಕ, ಯಕ್ಷ, ರಾಕ್ಷಸ, ಭೂತ ಮೊದಲಾದ ಗ್ರಹಗಳನ್ನು ಪುಡಿಪುಡಿ ಮಾಡು. ನಮ್ಮ ಹಗೆಗಳನ್ನೆಲ್ಲ ಧೂಳೀಪಟ ಮಾಡು.

\begin{verse}
ತ್ವಂ ಯಾತುಧಾನಪ್ರಮಥಪ್ರೇತಮಾತೃ ।\\ಪಿಶಾಚವಿಪ್ರಗ್ರಹಘೋರದೃಷ್ಟೀನ್ ॥\\ಧರೇಂದ್ರ! ವಿದ್ರಾವಯ ಕೃಷ್ಣಪೂರಿತೋ ।\\ಭೀಮಸ್ವನೋಽರೇರ್ಹೃದಯಾನಿ ಕಂಪಯನ್ \num{॥ ೧೪ ॥}
\end{verse}

ಶ್ರೀಕೃಷ್ಣಮೂರ್ತಿಯಿಂದ ಮೊಳಗಿಸಲ್ಪಡುವ ಹೇ ಪಾಂಚಜನ್ಯ ಶಂಖವೇ! ನಿನ್ನ ಭಯಂಕರ ಧ್ವನಿಯಿಂದ ಶತ್ರುಗಳ ಹೃದಯವನ್ನು ತಲ್ಲಣಗೊಳಿಸುತ್ತಾ, ರಾಕ್ಷಸರು, ಪ್ರಮಥರು, ಭೂತ ಪ್ರೇತಗಳು, ಸಪ್ತಮಾತೃಕೆಯರು–ಮೊದಲಾದ ಕ್ರೂರದೃಷ್ಟಿಯ ಭೂತಗಳನ್ನು ದೂರ ತೊಲಗಿಸು!

\begin{verse}
ತ್ವಂ ತಿಗ್ಮಧಾರಾಸಿ ವರಾರಿಸೈನ್ಯ ।\\ಮೀಶಪ್ರಯುಕ್ತೋ ಮಮ ಛಿಂದಿ ಛಿಂದಿ ॥\\ಚಕ್ಷೂಂಷಿ ಚರ್ಮನ್! ಶತಚಂದ್ರ! ಛಾದಯ ।\\ದ್ವಿಷಾಮಘೋನಾಂ ಹರ ಪಾಪ ಚಕ್ಷುಷಾಂ \num{॥ ೧೫ ॥}
\end{verse}

ಹರಿತವಾದ ಮೊನೆಯುಳ್ಳ ನಂದಕಖಡ್ಗವೇ! ಶ್ರೀಹರಿಯಿಂದ ಪ್ರಯೋಗಿಸಲ್ಪಟ್ಟ ನೀನು ನಮ್ಮ ಹಗೆಗಳನ್ನು ಕಡಿಕಡಿದುಹಾಕು! ಚಂದ್ರಾಕಾರದ ನೂರು ಮಂಡಲಗಳುಳ್ಳ ಹೇ ಖಡ್ಗವೇ! ಪಾಪಿಷ್ಠ ರಾಗಿ ಪಾಪಕರವಾದ ಕಣ್ಣುಗಳ್ಳುಳ್ಳ ನಮ್ಮ ಶತ್ರುಗಳ ಕಣ್ಣುಗಳನ್ನು ಮುಚ್ಚಿಹಾಕು!

\begin{verse}
ಯನ್ನೋ ಭಯಂ ಗೃಹೇಭ್ಯೋಽಭೂತ್ಕೇತುಭ್ಯೋ ನೃಭ್ಯ ಏವ ಚ ।\\ಸರೀಸೃಪೇಭ್ಯೋ ದಂಷ್ಟ್ರಿಭ್ಯೋ ಭೂತೇಭ್ಯೋಂಹೇಭ್ಯ ಏವ ಚ \num{॥ ೧೬ ॥}
\end{verse}

ಭೂತಗಳಿಂದ, ಕೇತುಗಳಿಂದ, ಮನುಷ್ಯರಿಂದ, ಸರ್ಪಗಳಿಂದ, ಕೋರೆದಾಡೆಗಳುಳ್ಳ ದುಷ್ಟಮೃಗಗಳಿಂದ, ಇತರ ಭೂತಗಳಿಂದ, ಪಾಪಗಳಿಂದ ನಮಗುಂಟಾಗಬಹುದಾದ ಭಯ–

\begin{verse}
ಸರ್ವಾಣ್ಯೇತಾನಿ ಭಗವನ್ನಾಮರೂಪಾಸ್ತ್ರಕೀರ್ತನಾತ್ ॥\\ಪ್ರಯಾಂತು ಸಂಕ್ಷಯಂ ಸದ್ಯೋ ಯೇ ನಃ ಶ್ರೇಯಃಪ್ರತೀಪಕಾಃ \num{॥ ೧೭ ॥}
\end{verse}

ಈ ಎಲ್ಲವೂ ಭಗವಂತನ ನಾಮ, ರೂಪ ಮತ್ತು ಅಸ್ತ್ರಗಳ ಸಂಕೀರ್ತನದಿಂದ ತಕ್ಷಣವೇ ನಾಶವಾಗಲಿ! ನಮ್ಮ ಏಳಿಗೆಗೆ ವಿರೋಧವಾದ ಸಮಸ್ತ ಭೂತಗಳೂ ನಾಶ ವಾಗಲಿ!

\begin{verse}
ಗರುಡೋ ಭಗವಾನ್ ಸ್ತೋತ್ರಸ್ತೋಭಶ್ಛಂದೋಮಯಃ ಪ್ರಭುಃ ॥\\ರಕ್ಷತ್ವಶೇಷಕೃಚ್ಛೆ ಭ್ಯೋ ವಿಷ್ವಕ್ಸೇನಃ ಸ್ವನಾಮಭಿಃ \num{॥ ೧೮ ॥}
\end{verse}

ಸ್ತೋತ್ರರೂಪವಾದ ಸಾಮವೇದಮಂತ್ರಗಳಿಂದ ಹೊಗಳಿಸಿಕೊಳ್ಳುತ್ತಾ ವೇದಸ್ವರೂಪ ನೆನಿಸಿರುವ ಮಹಾಮಹಿಮನಾದ ಗರುಡನು ಸಕಲ ಸಂಕಟಗಳಿಂದಲೂ ನಮ್ಮನ್ನು ಉದ್ಧ ರಿಸಿ ರಕ್ಷಿಸಲಿ! ಭಗವಂತನ ಪಾರ್ಷದರಲ್ಲಿ ಅಗ್ರಗಣ್ಯನಾದ ವಿಷ್ವಕ್ಸೇನನು ನಮ್ಮನ್ನು ರಕ್ಷಿ ಸಲಿ!

\begin{verse}
ಸರ್ವಾಪದ್ಭ್ಯೋ ಹರೇರ್ನಾಮ ರೂಪಯಾನಾಯುಧಾನಿ ನಃ ॥\\ಬುದ್ಧೀಂದ್ರಯಮನಃ ಪ್ರಾಣಾನ್ ಪಾಂತು ಪಾರ್ಷದಭೂಷಣಾಃ \num{॥ ೧೯ ॥}
\end{verse}

ಭಗವಂತನ ದಿವ್ಯನಾಮ, ಆತನ ನಾನಾರೂಪ, ಗದೆಯೇ ಮೊದಲಾದ ಆಯುಧಗಳು, ಗರುಡ ವಿಷ್ವಕ್ಸೇನ ಮೊದಲಾದ ಭಕ್ತರು, ಶ್ರೀವತ್ಸ ಮೊದಲಾದ ಆಭರಣಗಳು– ಇವೆಲ್ಲವೂ ನಮ್ಮನ್ನು ಎಲ್ಲ ವಿಪತ್ತುಗಳಿಂದ ರಕ್ಷಿಸಲಿ! ನಮ್ಮ ಬುದ್ಧಿ, ಇಂದ್ರಿಯ, ಮನಸ್ಸು, ಪ್ರಾಣ–ಇವನ್ನು ರಕ್ಷಿಸಲಿ!

\begin{verse}
ಯಥಾ ಹಿ ಭಗವಾನೇವ ವಸ್ತುತಃ ಸದಸಚ್ಚಯತ್ ॥\\ಸತ್ಯೇನಾಽನೇನ ನಃ ಸರ್ವೇ ಯಾಂತು ನಾಶಮುಪದ್ರವಾಃ \num{॥೨೦ ।।}
\end{verse}

‘ಪ್ರತ್ಯಕ್ಷ ಪರೋಕ್ಷವಾಗಿರುವ ಈ ಜಗತ್ತೆಲ್ಲವೂ ಭಗವಂತನ ಸ್ವರೂಪವೇ ಸರಿ’– ಎಂಬ ಅರ್ಥ ದಿಟವಾದ ಪಕ್ಷದಲ್ಲಿ ಆ ಸತ್ಯವೇ ನಮ್ಮ ಸಕಲ ಉಪದ್ರವಗಳನ್ನು ನಾಶ ಮಾಡಲಿ!

\begin{verse}
ಯಥೈಕಾತ್ಮ್ಯಾನು ಭಾವಾನಾಂ ವಿಕಲ್ಪರಹಿತಸ್ಸ್ವಯಂ ॥\\ಭೂಷಣಾಯುಧಲಿಂಗಾಖ್ಯಾ ಧತ್ತೇ ಶಕ್ತೀ ಸ್ವಮಾಯಯಾ \num{॥ ೨೧ ॥}
\end{verse}

ಒಂದೇ ವಿಧವಾದ ಮಹಿಮೆಯನ್ನೂ ಸ್ವರೂಪವನ್ನೂ ಪಡೆದಿರುವ ಅಲಂಕಾರಾದಿ ಗಳಿಗೂ ತನಗೂ ಭೇದವಿಲ್ಲದಿದ್ದರೂ, ಭಗವಂತನು ತನ್ನ ಮಾಯಾಶಕ್ತಿಯಿಂದ ಭಿನ್ನ ವಾಗಿ ತೋರುವ ಭೂಷಣ, ಆಯುಧ, ಆಕೃತಿ, ನಾಮ–ಇವುಗಳ ಸ್ವರೂಪವೆನಿಸಿದ ಹಲವು ಶಕ್ತಿಗಳನ್ನು ಧರಿಸಿರುವನೆಂಬುದು ದಿಟವಾದರೆ

\begin{verse}
ತೇನೈವ ಸತ್ಯಮಾನೇನ ಸರ್ವಜ್ಞೋ ಭಗವಾನ್ ಹರಿಃ ॥\\ಪಾತು ಸರ್ವೈಸ್ಸ್ವರೂಪೈರ್ನಃ ಸದಾ ಸರ್ವತ್ರ ಸರ್ವಗಃ \num{॥ ೨೨ ॥}
\end{verse}

ಆ ಸತ್ಯ ಪ್ರಮಾಣದಿಂದಲೇ ಸರ್ವಜ್ಞನೂ, ಸರ್ವವ್ಯಾಪಕನೂ, ನಿರತಿಶಯಮಹಿಮನೂ ಆದ ಶ್ರೀಹರಿಯು ಸಕಲ ರೂಪಗಳಿಂದ, ಎಲ್ಲೆಲ್ಲಿಯೂ ಯಾವಾಗಲೂ ನಮ್ಮನ್ನು ಸಂರಕ್ಷಿಸಲಿ!

\begin{verse}
ವಿದಿಕ್ಷು ದಿಕ್ಷೂರ್ಧ್ವಮದಸ್ಸಮಂತಾತ್ ।\\ಅಂತರ್ಬಹಿರ್ಭಗವಾನ್ನಾರಸಿಂಹಃ ॥\\ಪ್ರಹಾಪಯನ್ ಲೋಕಭಯಂ ಸ್ವನೇನ ।\\ಸ್ವತೇಜಸಾಗ್ರಸ್ತ ಸಮಸ್ತತೇಜಾಃ \num{॥ ೨೩ ॥}
\end{verse}

ತನ್ನ ತೇಜಸ್ಸಿನಿಂದಲೇ ಸಕಲ ತೇಜಸ್ಸುಗಳನ್ನೂ ಹೀರಿ, ತನ್ನ ಅಟ್ಟಹಾಸದಿಂದಲೇ ಲೋಕದ ಭಯವನ್ನು ಪರಿಹರಿಸುತ್ತಾ, ಸರ್ವಾತ್ಮಕನಾಗಿರುವ ನರಸಿಂಹಮೂರ್ತಿಯು ಪೂರ್ವಾದಿ ದಿಕ್ಕುಗಳಲ್ಲಿಯೂ ಮೇಲೆ ಕೆಳಗೆ ಒಳಗೆ ಹೊರಗೆ ನಮ್ಮನ್ನು ಸದಾ ರಕ್ಷಿಸುತ್ತಿರಲಿ!

\begin{center}
\textbf{ಅಂಗನ್ಯಾಸ}
\end{center}

\begin{center}
ಓಂ ನಮೋ ನಾರಾಯಣಾಯ=(ಅಷ್ಟಾಕ್ಷರೀ)
\end{center}

\begin{tabular}{{@{}cc@{}}}
\textbf{ಸೃಷ್ಟಿ} & \textbf{ಸಂಹಾರ} \\
ಓಂ ಓಂ ಪಾದಾಭ್ಯಾನ್ನಮಃ (ಪಾದ) & ಓಂ ಓಂ ಶಿರಸೇನಮಃ \\
ಓಂ ನಂ ಜಾನುಭ್ಯಾನ್ನಮಃ (ಮೊಳಕಾಲು) & ಓಂ ನಂ ಮುಖಾಯ ನಮಃ \\
ಓಂ ಮೋಂ ಊರುಭ್ಯಾನ್ನಮಃ (ತೊಡೆ) & ಓಂ ಮೋಂ ಉರಸೇ ನಮಃ \\
ಓಂ ನಾಂ ಉದರಾಯ ನಮಃ (ಹೊಟ್ಟೆ) & ಓಂ ನಾಂ ಹೃದಯಾಯ ನಮಃ \\
ಓಂ ರಾಂ ಹೃದಯಾಯನಮಃ (ಹೃದಯ) & ಓಂ ರಾಂ ಉದರಾಯ ನಮಃ \\
ಓಂ ಯಂ ಉರಸೇ ನಮಃ (ಎದೆ) & ಓಂ ಯಂ ಊರುಭ್ಯಾನ್ನಮಃ \\
ಓಂ ಣಾಂ ಮುಖಾಯ ನಮಃ (ಮುಖ) & ಓಂ ಣಾಂ ಜಾನುಭ್ಯಾನ್ನಮಃ \\
ಓಂ ಯಂ ಶಿರಸೇ ನಮಃ (ಶಿರಸ್ಸು) & ಓಂ ಯಂ ಪಾದಾಭ್ಯಾನ್ನಮಃ \\
\end{tabular}

\begin{center}
\textbf{ಕರನ್ಯಾಸಃ}
\end{center}

\begin{center}
ಓಂ ನಮೋ ಭಗವತೇ ವಾಸುದೇವಾಯ (ದ್ವಾದಶಾಕ್ಷರೀ)
\end{center}

\begin{verse}
ಓಂ ನಂ ತರ್ಜಿನೀಭ್ಯಾಂ ನಮಃ\\ಮೋಂ ಭಂ ಮಧ್ಯಮಾಭ್ಯಾಂ ನಮಃ\\ಗಂ ವಂ ಅನಾಮಿಕಾಭ್ಯಾಂ ನಮಃ\\ತೇ ವಾಂ ಕನಿಷ್ಠಿಕಾಭ್ಯಾಂ ನಮಃ\\ಸುಂ ದೇಂ\\ವಾಂ ಯಂ \\ಅಂಗುಷ್ಠಾಭ್ಯಾಂ ನಮಃ
\end{verse}

ಇದರಲ್ಲಿ ಮೊದಲೆರಡಕ್ಷರಗಳನ್ನು ಅಂಗುಷ್ಠದ ಕೆಳಗಿನ ಎರಡು ಗಿಣ್ಣುಗಳಲ್ಲಿಯೂ ಕೊನೆಯೆರಡನ್ನು ಅಂಗುಷ್ಠದ ಕೊನೆಯ ಗಿಣ್ಣುಗಳಲ್ಲಿಯೂ ನ್ಯಾಸ ಮಾಡಬೇಕು.

\begin{center}
\textbf{ಹೃದಯಾದಿ ನ್ಯಾಸ}
\end{center}

\begin{center}
ಓಂ ವಿಷ್ಣುವೇ ನಮಃ (ಷಡಕ್ಷರೀ)
\end{center}

\begin{verse}
ಓಂ ಹೃದಯಾಯ ನಮಃ\\ವಿಂ ಶಿರಸೇ ಸ್ವಾಹಾ\\ಷಂ ಭ್ರೂಮಧ್ಯಾಯ ನಮಃ\\ಣಂ ಶಿಖಾಯೈ ವಷಟ್​\\ವೇಂ ನೇತ್ರತ್ರಯಾಯ ವೌಷಟ್​\\ನಂ ಕೀಲಕಂ\\ಮಃ ಅಸ್ತ್ರಾಯ ಫಟ್​\\ಓಂ ವಿಷ್ಣುವೇ ನಮಃ ಇತಿ ದಿಗ್ಬಂಧಃ
\end{verse}

\textbf{ಅಥ ಧ್ಯಾನಮ್​–}

\begin{verse}
ಆತ್ಮಾನಂ ಪರಮಂ ಧ್ಯಾಯೇ ದ್ಧ್ಯೇಯಂಷಟ್ಚಕ್ರಿಭಿರ್ಯುತಮ್​ ।\\ವಿದ್ಯಾತೇಸ್ತಜಪೋಮೂರ್ತಿಮಿಮಂ ಮಂತ್ರಮುದಾಹರೇತ್ ॥
\end{verse}


\section{೪. ನಾರದನು ಚಿತ್ರಕೇತುವಿಗೆ ಉಪದೇಶಿಸಿದ ಮಂತ್ರೋಪನಿಷತ್ತು}

\begin{verse}
ನಮಸ್ತುಭ್ಯಂ ಭಗವತೇ ವಾಸುದೇವಾಯ ಧೀಮಹಿ ।\\ಪ್ರದ್ಯುಮ್ನಾಯಾಽನಿರುದ್ಧಾಯ ನಮಸ್ಸಂಕರ್ಷಣಾಯ ಚ ॥
\end{verse}

ವಾಸುದೇವ, ಪ್ರದ್ಯುಮ್ನ, ಅನಿರುದ್ಧ, ಸಂಕರ್ಷಣ–ಎಂಬ ನಾಲ್ಕು ವ್ಯೂಹಗಳಿಂದ ಪ್ರಕಾಶಿಸುವ ಹೇ ಭಗವಂತ! ನಿನಗೆ ನಮಸ್ಕಾರ

\begin{verse}
ನಮೋ ವಿಜ್ಞಾನಮಾತ್ರಾಯ ಪರಮಾನಂದಮೂರ್ತಯೇ ।\\ಆತ್ಮಾರಾಮಾಯ ಶಾಂತಾಯ ನಿವೃತ್ತದ್ವೈತದೃಷ್ಟಯೇ ॥
\end{verse}

ಕೇವಲ ಜ್ಞಾನಸ್ವರೂಪನಾಗಿಯೂ, ಪರಮಾನಂದ ಶರೀರನಾಗಿಯೂ, ಆತ್ಮಾರಾಮ ನಾಗಿಯೂ, ಪರಮಶಾಂತನಾಗಿಯೂ, ಭೇದದೃಷ್ಟಿಯಿಲ್ಲದವನಾಗಿಯೂ ಇರುವ ನಿನಗೆ ನಮಸ್ಕಾರ.

\begin{verse}
ಆತ್ಮಾನಂದಾನುಭೂತ್ಯೈವ ನ್ಯಸ್ತಶಕ್ತ್ಯೂರ್ಮಯೇ ನಮಃ ।\\ಹೃಷೀಕೇಶಾಯ ಮಹತೇ ನಮಸ್ತೇಽನಂತ ಶಕ್ತಯೇ ॥
\end{verse}

ಆತ್ಮಾನಂದಾನುಭವದಿಂದಲೆ ಮಾಯಾನಿಮಿತ್ತಗಳಾದ ರಾಗದ್ವೇಷಾದಿಗಳನ್ನು ನಿರಾ ಕರಿಸಿದ, ಅನಂತ ಶಕ್ತಿಗಳುಳ್ಳ, ಇಂದ್ರಿಯ ಪ್ರೇರಕನಾದವನಿಗೆ ನಮಸ್ಕಾರ.

\begin{verse}
ವಚಸ್ಯುಪರತೇ ಪ್ರಾಪ್ಯ ಯ ಏಕೋ ಮನಸಾ ಸಹ ।\\ಅನಾಮರೂಪಶ್ಚಿನ್ಮಾತ್ರಸ್ಸೋವ್ಯಾನ್ನಸ್ಸದಸತ್ಪರಃ ॥
\end{verse}

ಮನಸ್ಸು, ವಾಕ್ಕು, ಇಂದ್ರಿಯಗಳು ಯಾವನನ್ನು ಹೊಂದಲಾರದೆ ಹಿಂತಿರುಗುವವೋ, ಯಾವನು ನಾಮರೂಪರಹಿತನಾಗಿ, ಚಿತ್ಸ್ವರೂಪನಾಗಿ, ಕಾರ್ಯಕಾರಣರೂಪವಾದ ಜಗತ್ತಿಗೆ ಮೂಲಕಾರಣನಾಗಿ, ಒಬ್ಬನೇ ಬೆಳಗುವನೋ ಆ ಶ್ರೀಹರಿ ನಮ್ಮನ್ನು ಸದಾ ರಕ್ಷಿಸಲಿ.

\begin{verse}
ಯಸ್ಮಿನ್ನಿದಂ ಯತಶ್ಚೇದಂ ತಿಷ್ಠತ್ಯಪ್ಯೇತಿ ಜಾಯತೇ ।\\ಮೃಣ್ಮಯೇಷ್ವಿವ ಮೃಜ್ಜಾತಿಸ್ತಸ್ಮೈ ತೇ ಬ್ರಹ್ಮಣೇ ನಮಃ ॥
\end{verse}

ಸೃಷ್ಟಿ ಸ್ಥಿತಿ ಲಯಗಳಿಗೆ ಕಾರಣನಾಗಿ, ಗಡಿಗೆಯಲ್ಲಿನ ಮೃತ್ತಿಕೆಯಂತೆ ಎಲ್ಲೆಲ್ಲಿಯೂ ವ್ಯಾಪಿಸಿರುವ ಪರಬ್ರಹ್ಮನಿಗೆ ನಮಸ್ಕಾರ.

\begin{verse}
ಯನ್ನಸ್ಪೃಶಂತಿ ನ ವಿದುರ್ಮನೋ ಬುದ್ಧೇಂದ್ರಿಯಾಽಸವಃ ।\\ಅಂತರ್ಬಹಿಶ್ಚ ವಿತತಂ ವ್ಯೋಮವತ್ತನ್ನತೋಽಸ್ಮ್ಯಹಮ್ ॥
\end{verse}

ಒಳಗೆ ಹೊರಗೆ ಎಲ್ಲೆಲ್ಲಿಯೂ ಆಕಾಶದಂತೆ ವ್ಯಾಪಿಸಿದ್ದರೂ ಮನಸ್ಸು, ಬುದ್ಧಿ, ಇಂದ್ರಿಯಗಳು ನಿನ್ನನ್ನು ಕಂಡುಕೊಳ್ಳಲಾರವು; ಅಂತಹ ಪರಬ್ರಹ್ಮನಾದ ನಿನಗೆ ನಮಸ್ಕಾರ.

\begin{verse}
ದೇಹೇಂದ್ರಿಯ ಪ್ರಾಣ ಮನೋ ಧಿಯೋಮೀ ।\\ಯದಂವಿದ್ಧಾಃ ಪ್ರಚರಂತಿ ಕರ್ಮಸು ॥\\ನೈವಾಽನ್ಯದಾಲೋಹಮಿವ ಪ್ರತಪ್ತಂ ।\\ಸ್ಥಾನೇಷು ತದ್ದ್ರಷ್ಟ್ರಪದೇಶಮೇತಿ ॥
\end{verse}

ಅಗ್ನಿಸಂಬಂಧದಿಂದ ಲೋಹಗಳಿಗೂ ದಹನಶಕ್ತಿಯುಂಟಾಗಿ, ಅವು ಇತರ ಪದಾರ್ಥ ಗಳನ್ನು ದಹಿಸಲಿಲ್ಲವಾಗಿದ್ದರೂ ಅಗ್ನಿ ಸಂಬಂಧವಿಲ್ಲದಾಗ ದಹನಶಕ್ತಿಯೂ ಇಲ್ಲದಿರು ವಂತೆಯೇ ದೇಹ, ಇಂದ್ರಿಯ, ಪ್ರಾಣ, ಮನಸ್ಸು, ಬುದ್ಧಿ ಇವೆಲ್ಲವೂ ನಿನ್ನ ಅಂಶ ವಿದ್ದಾಗಲೆ ಜ್ಞಾನ ಕ್ರಿಯಾಶಕ್ತಿಗಳನ್ನು ಹೊಂದಿ, ಕರ್ಮದಲ್ಲಿ ಪ್ರವರ್ತಿಸುತ್ತವೆ. ಪರ ಮಾತ್ಮನು ಜಾಗ್ರದವಸ್ಥೆಯಲ್ಲಿ ಇವಕ್ಕೆ ಸಾಕ್ಷಿಯೆಂಬ ಹೆಸರನ್ನು ಪಡೆಯುತ್ತಾನೆ. ಆತನಿಗೆ ನಮಸ್ಕಾರ.

ಓಂ ನಮೋ ಭಗವತೇ ಮಹಾಪುರುಷಾಯ, ಮಹಾನುಭಾವಾಯ, ಮಹಾವಿಭೂ ತಯೇ ಸಕಲ ಸಾತ್ತ್ವತ ಪರಿಬೃಥನಿಕರ ಕರಕಮಲ ಕುಟ್ಮಲೋಪಲಾಲಿತ ಚರಣಾರ ವಿಂದಯುಗಳ! ಪರಮ! ಪರಮೇಷ್ಠಿನ್​! ನಮಸ್ತೇ ॥

ಹೇ ಸರ್ವೋತ್ತಮ, ಸರ್ವೇಶ್ವರ, ಭಕ್ತ ಸಮೂಹದ ಕರಕಮಲದಿಂದ ಸೇವಿತವಾದ ಪಾದವುಳ್ಳವನೆ, ಷಡ್ಗುಣೈಶ್ವರ್ಯ ಸಂಪನ್ನನಾಗಿ, ಪುರುಷೋತ್ತಮನಾಗಿ, ಮಹಾಮಹಿಮ ನಾಗಿ, ಭಾಗ್ಯನಿಧಿಯಾಗಿರುವ ನಿನಗೆ ನಮಸ್ಕಾರ ಮಾಡುತ್ತೇನೆ.


\section{೫. ಭ್ರಮರ ಗೀತಾ}

\begin{verse}
\footnote{*ಇದು ಶ್ರುತಿ ಜೀವ ಸಂವಾದ ರೂಪದ ಒಂದು ಗೂಢಾರ್ಥವನ್ನು ಒಳಕೊಂಡಿರುವುದೆಂದು ಹೇಳುತ್ತಾರೆ. ‘ತತ್ತ್ವಮಸಿ’ ಎನ್ನುವಂತಹ ಯಾವುದೋ ಒಂದು ಮುಖ್ಯ ಶ್ರುತಿಯು ವಿಷಯಸುಖವೆಂಬ ಬಂಡುಂಡು ಮೈಮರೆತಿರುವ ಜೀವನನ್ನು ಕಂಡು ಕರುಣೆಯಿಂದ, ಆ ಜೀವಕ್ಕೆ ತತ್ವೋಪದೇಶ ಮಾಡಿ, ಪರಮಾತ್ಮನಲ್ಲಿ ಸೇರಿಸುವುದಕ್ಕಾಗಿ, ಆ ಜೀವವನ್ನೆ ಪರಮಾತ್ಮನು ಕಳುಹಿಸಿದ ದೂತನೆಂದು ಕಲ್ಪಿಸಿಕೊಂಡು ಹೀಗೆ ಹೇಳುತ್ತದೆ.

ಹೇ ಜೀವ! ವಿಷಯಸುಖಗಳೆಂಬ ಮದ್ಯವನ್ನು ಕುಡಿದು, ಮೋಸಗಾರನಾದ ಮನಸ್ಸಿಗೆ ಮರುಳಾಗಿ, ಪುಷ್ಪಹಾರ–ಕುಂಕುಮಕೇಸರಿ–ಶ್ರೀಗಂಧ ಇತ್ಯಾದಿಗಳಿಂದ ಅಲಂಕೃತವಾದ ಸ್ತನಗಳು, ಮೀಸೆಗಳು ಮೊದ ಲಾದವುಗಳಿಂದ ಕೂಡಿದ ಈ ಶರೀರವೇ ನಾನೆಂದು ಭ್ರಮಿಸಿ, ಸವತಿಯಾದ ಮಾಯೆಯ ಕಾಲನ್ನು ಹಿಡಿದು ಶರಣು ಹೋಗುವೆಯಲ್ಲ! ಹಾಗೆ ಮಾಡಬೇಡ. ನೀನು ನ್ಯಾಯವಾಗಿ ಮಾಯೆಗೆ ಒಡೆಯನೆ ಹೊರತು ಆಳಲ್ಲ. ಬ್ರಹ್ಮವೇ ಸತ್ಯ ಎಂಬ ಮಾತನ್ನು ಲಾಲಿಸು. ಬ್ರಹ್ಮನಿಂದ ನೀನು ಬೇರೆ ಎಂದು ಭಾವಿಸುವುದು ಕೇವಲ ವಿಡಂಬನೆ \num{॥ ೧ ॥}

ಮಾಯೆಗೆ ಮರುಳಾಗಿ ನಿಕೃಷ್ಟನಾಗಿರುವ ಹೇ ಜೀವ! ಪರಮೇಶ್ವರನು ತನ್ನ ಶಕ್ತಿರೂಪವಾದ ಅಮೃತವನ್ನು ಒಮ್ಮೆ ಆ ಲೋಕಮಾಯೆಗೆ ಕುಡಿಸಿದಂತೆ ಮಾಡಿ ತಕ್ಷಣವೇ ಅವಳನ್ನು ಬಿಟ್ಟುಬಿಡುವನು. ಆದರೆ ಶ್ರುತಿಗಳಾದ ನಮ್ಮನ್ನು ಮಾತ್ರ ಹಾಗೆ ಬಿಡಲಿಲ್ಲ. ಆ ಮಾಯೆ ಸದಾ ಆತನ ಪಾದಸೇವೆ ಮಾಡುತ್ತಿದ್ದಾಳೆ. ಅವಳಿಗೆ ನೆಲೆಸಲು ಮತ್ತೆಲ್ಲಿಯೂ ಅವಕಾಶವಿಲ್ಲವಾದುದರಿಂದ ಆತನನ್ನು ಆಶ್ರಯಿಸಿದ್ದಾಳೆ. ಮಾಯಾತೀತನಾದ ಆ ಭಗವಂತನು ನಿನ್ನಿಂದ ಬೇರೆಯೇನೂ ಅಲ್ಲ. ಆದ್ದರಿಂದ ನೀನೂ ಮಾಯಾತೀತನಾಗಿ ಪರಮೇಶ್ವರತ್ವವನ್ನು ಪಡೆ. \num{॥ ೨ ॥}

ಪಂಚೇಂದ್ರಿಯಗಳೊಡನೆ ಮನಸ್ಸು ಸೇರಿ ಆರು ಪಾದಗಳಾಗಿರುವ ಎಲೆ ಜೀವನೆ! ಪೂರ್ವಪುಣ್ಯದಿಂದ ಮನುಷ್ಯಜನ್ಮವನ್ನು ಪಡೆದಿರುವ ನೀನು ನಿಕೃಷ್ಟವಾದ, ತ್ರಿಗುಣಾತ್ಮಕವಾದ ಪ್ರಕೃತಿಯನ್ನು ನಮ್ಮ ಮುಂದೆ ಹೊಗಳುತ್ತಾ ಹೋಗುವುದು ಯೋಗ್ಯವಲ್ಲ. ಸರ್ವನಿಯಾಮಕನಾದ ಪರಮಾತ್ಮನಿಗೆ ಮಿತ್ರನಾಗಿರುವ ಹೇ ಜೀವನೆ, ಅಜ್ಞಾನದಿಂದ ಉಂಟಾಗುವ ಸಂಸಾರವನ್ನು ನಾಶ ಮಾಡುವವನೂ, ಸಕಲ ಶ್ರುತಿಗಳ ತಾತ್ಪರ್ಯಕ್ಕೂ ವಿಷಯನಾಗಿರುವವನೂ ಆದ ಆ ಪರಮಾತ್ಮನಿಗೆ ನಾವು ಪ್ರಿಯಸಖಿಯರು. ನಮ್ಮ ಕೀರ್ತಿಯನ್ನು ಹಾಡಿದರೆ ನಾವು ನಿನಗೆ ಮೋಕ್ಷವನ್ನು ಕೊಡುತ್ತೇವೆ. \num{॥ ೩ ॥}

ಹೇ ಜೀವ! ಮೂರು ಲೋಕಗಳಲ್ಲಿಯೂ ಮಹಾಪತಿವ್ರತೆಯಂತಿರುವ ಶ್ರುತಿಗಳು ಮಾಯಾಮನೋ ಹರವಾದ ಸಗುಣಬ್ರಹ್ಮನನ್ನಲ್ಲ ಪ್ರತಿಪಾದಿಸುವುದು, ಪರಬ್ರಹ್ಮವಸ್ತುವನ್ನೇ. ಅನೇಕ ಕೋಟಿ ಬ್ರಹ್ಮಾಂಡಗಳನ್ನು ನಿಮಿಷಮಾತ್ರದಲ್ಲಿ ಸೃಷ್ಟಿಮಾಡಿ ಲಯಗೊಳಿಸಬಲ್ಲ ಮಾಯಾಶಕ್ತಿ ಕೂಡ ಯಾವನನ್ನು ಸೋಂಕಲಾರದೆ, ಕೇವಲ ಅವನ ಪಾದಧೂಳಿಯನ್ನು ಸೇವಿಸುತ್ತಿರುವಳೊ ಅಂತಹ ನಿರ್ಗುಣ ಬ್ರಹ್ಮನನ್ನು ಪ್ರತಿ ಪಾದಿಸುವುದಕ್ಕೆ ನಮಗೆ ತಾನೆ ಶಕ್ತಿಯೆಲ್ಲಿದೆ? ಆದರೂ ಪರಮಾತ್ಮನನ್ನು ಪ್ರತಿಪಾದಿಸಹೊರಟ ನಾವು ಅಜ್ಞಾನಿಗಳ ಉದ್ಧಾರಕ್ಕಾಗಿಯೆ ಇರುವೆವು \num{॥ ೪ ॥}

ಎಲೆ ಜೀವ! ಪ್ರಿಯೋಕ್ತಿಗಳಾದ ಶ್ರುತಿಗಳು ಯಾವನಿಂದ ಬಂದಿವೆಯೋ, ಭಕ್ತರ ಮನೋಭಾವನೆಗಳನ್ನು ಯಾವನು ಅರಿಯಬಲ್ಲನೊ ಅಂತಹ ಭಗವಂತನ ಪಾದಗಳನ್ನು ತಲೆಯಲ್ಲಿ ಧರಿಸು. ಇದೇ ಶ್ರೇಯಸ್ಕರವೆಂದು ನಾನು ಎಣಿಸುತ್ತೇನೆ. ನಿನ್ನ ಭೋಗಕ್ಕಾಗಿ ಪುತ್ರ ಮಿತ್ರ ಕಳತ್ರರನ್ನೂ ಸ್ವರ್ಗಾದಿ ಭೋಗಗಳನ್ನೂ ನಿರ್ಮಿಸಿದ ಆ ಪರಮಾತ್ಮನಲ್ಲಿ ಮನಸ್ಸನ್ನಿಡದೆ ಮಾನವಜನ್ಮವನ್ನು ವ್ಯರ್ಥಗೊಳಿಸಬೇಡ. \num{॥ ೫ ॥ }

ಎಲೆ ಜೀವ! ಭಗವಂತನು ಭಕ್ತವತ್ಸಲ. ಆತನು ಭಕ್ತನಾದ ಸುಗ್ರೀವನಿಗಾಗಿ ಬೇಟೆಗಾರನಂತೆ ಮರದ ಹಿಂದೆ ನಿಂತು ವಾಲಿಯನ್ನು ಸಂಹರಿಸಿದ. ಪತಿಭಕ್ತಪರಾಯಣೆಯಾದ ಸೀತೆಯನ್ನು ಸಲಹುವುದಕ್ಕಾಗಿ ಶೂರ್ಪಣಖಿಯನ್ನು ವಿರೂಪಗೊಳಿಸಿದ. ತನ್ನನ್ನು ಮೊರೆಹೊಕ್ಕ ಇಂದ್ರಾದಿಗಳನ್ನು ಕಾಪಾಡುವುದಕ್ಕಾಗಿ ತನ್ನನ್ನು ಪೂಜಿಸಿದ ಬಲಿಯನ್ನೆ ಬಿಗಿದು ಕಟ್ಟಿದ. ಭಕ್ತ ರಕ್ಷಣೆಗಾಗಿ ಆತನು ಮಾಡಬಾರದ ಕೆಲಸವನ್ನು ಕೂಡ ಮಾಡುತ್ತಾನೆ. ಆದ್ದರಿಂದ ಅನನ್ಯ ಭಾವದಿಂದ ಆತನನ್ನು ಸ್ತುತಿಸಿ ಧನ್ಯನಾಗು. \num{॥ ೬ ॥}

ಎಲೆ ಜೀವ! ಭಗವಂತನ ಕಥಾಮೃತದಲ್ಲಿ ಒಂದು ತೊಟ್ಟನ್ನು ಸೇವಿಸಿದರೂ ಧನ್ಯನಾಗುತ್ತಾನೆ. ಇನ್ನು ಏಕಾಂತ ಭಕ್ತನಾದವನನ್ನು ಕುರಿತು ಹೇಳುವುದೇನು? \num{॥ ೭ ॥}

ಎಲೈ ಜೀವನೆ! ಕರ್ಮಫಲಗಳು ನಿತ್ಯವೆಂದು ಹೇಳುವ ಮಾತು ಶುದ್ಧ ಕಪಟ. ‘ಯಾವುದು ಜನ್ಮವೋ ಅದು ಅನಿತ್ಯ’ ಎಂಬ ವೇದೋಕ್ತಿಗೆ ವ್ಯಾಪ್ತಿ ಜ್ಞಾನವಿಲ್ಲದೆ ಯಾಗಾದಿ ಕರ್ಮಗಳು ನಿತ್ಯಫಲವೆಂದು ನಂಬಿ, ಬೇಡನ ಸಂಗೀತಕ್ಕೆ ಮೋಸಹೋಗುವ ಹುಲ್ಲೆಯಂತೆ ಸಂಸಾರವೆಂಬ ರೋಗಕ್ಕೆ ಸಿಕ್ಕಿ ನರಳುತ್ತಾರೆ. ನಮ್ಮ ಅಭಿಪ್ರಾಯವನ್ನು ಅರ್ಥಮಾಡಿಕೊಂಡು ಬ್ರಹ್ಮನೇ ಸತ್ಯವೆಂಬುದನ್ನು ನಂಬು. ಬೇರೆಯದನ್ನು ನಂಬಬೇಡ. \num{॥ ೮ ॥}

ಹೇ ಜೀವ, ನಿನ್ನ ಪ್ರಿಯನಾದ ಪರಮೇಶ್ವರನಿಂದ ಪ್ರೇರಿತನಾಗಿಯೇ ನೀನು ಪುನಃ ಸಂಸಾರಿಯಾಗಿರುವೆ. ಆದ್ದರಿಂದ ಆತನನ್ನೆ ಮರೆಹೋಗು. ಉಳಿದುದಕ್ಕೆ ಆಶೆಪಡಬೇಡ. ಭಕ್ತಿ ಇಲ್ಲದೆ ಭಗವದನುಗ್ರಹವಿಲ್ಲ. ಆ ಅನುಗ್ರಹವಿಲ್ಲದೆ ಬ್ರಹ್ಮಾತ್ಮಗಳ ಐಕ್ಯತೆಯನ್ನು ಸಾರುವ ನಮ್ಮಲ್ಲಿ (ಶ್ರುತಿ) ಮನಸ್ಸು ನೆಲೆಸದು. ರಮಾ ದೇವಿಯು ಭಗವಂತನ ಹೃದಯದಲ್ಲಿ ಸದಾ ನೆಲೆಸಿರುವುದರಿಂದ ಬೇರೆಯ ಭೋಗಕ್ಕೆ ಅಲ್ಲಿ ಅವಕಾಶವಿಲ್ಲ. \num{॥ ೯ ॥}

ಅಯ್ಯೋ, ಪರಮಾತ್ಮನ ಅಂಶವಾದ ಈ ಜೀವನು ಜೇನುಹನಿಯಂತಿರುವ ವಿಷಯಸುಖಕ್ಕೆ ಆಧಾರವಾದ ಈ ದೇಹದಲ್ಲಿ ನೆಲೆಸಿ, ಒಮ್ಮೆಯಾದರೂ ಪರಮಾತ್ಮನನ್ನು ಧ್ಯಾನಿಸುವುದಿಲ್ಲ. ಮನೆ ಮಠ ಮಡದಿ ಮಕ್ಕಳನ್ನು ಮಾತ್ರ ಸ್ಮರಿಸುತ್ತಾನೆ. ಶಮದಮಾದಿಗಳೆಂಬ ಶ್ರೀಗಂಧದಿಂದ ತುಂಬಿದ ತನ್ನ ಅಂತಃಕರಣವೆಂಬ ಭುಜವನ್ನು ನಮ್ಮ ತಲೆಯ ಮೇಲೆ ಯಾವಾಗ ಇಡುವನೋ–ಎಂದು ಶ್ರುತಿಗಳು ಜೀವನಿಗಾಗಿ ಮರುಗಿದವು. \num{॥ ೧೦ ॥}}ಕಾಚಿನ್ ಮಧುಕರಂ ದೃಷ್ಟ್ವಧ್ಯಾಯಂತೀ ಕೃಷ್ಟಸಂಗಮಂ ।\\ಪ್ರಿಯಪ್ರಸ್ಥಾಪಿತಂ ದೂತಂ ಕಲ್ಪಯಿತ್ವೇದಮಬ್ರವೀತ್ ॥
\end{verse}

ಶ್ರೀಕೃಷ್ಣನ ಸಮಾಗಮವನ್ನು ಬಯಸುವ ಗೋಪಿಯೊಬ್ಬಳು ತನ್ನ ಬಳಿ ಹಾರಾಡು ತ್ತಿರುವ ಭ್ರಮರ(ದುಂಬಿ)ವೊಂದನ್ನು ಕಂಡು, ಅದನ್ನು ತನ್ನ ಪ್ರಿಯನಾದ ಶ್ರೀಕೃಷ್ಣನು ಕಳುಹಿಸಿದ ದೂತನನ್ನಾಗಿ ಕಲ್ಪಿಸಿಕೊಂಡು, ಅದರೊಡನೆ ಹೀಗೆ ಹೇಳುತ್ತಾಳೆ:

\begin{verse}
ಮಧುಪ ಕಿತವಬಂಧೋ ಮಾ ಸ್ಪೃಶಾಂಘ್ರಿಂ ಸಪತ್ನ್ಯಾಃ ।\\ಕುಚವಿಲುಳಿತಮಾಲಾಕುಂಕುಮಶ್ಮಶ್ರುಭಿರ್ನಃ ॥\\ವಹತು ಮಧುಪತಿಸ್ತನ್ಮಾನನೀನಾಂ ಪ್ರಸಾದಂ ಯದು ।\\ಸದಸಿ ವಿಡಂಬಂ ಯಸ್ಯ ದೂತಸ್ತ್ವಮೀದೃಕ್ \num{॥ ೧ ॥}
\end{verse}

ಎಲೈ ವಂಚಕನಾದ ಶ್ರೀಕೃಷ್ಣನ ಬಂಧುವಾದ ಭ್ರಮರನೆ, ನನ್ನ ಸವತಿಯರ ಕುಚ ಮಂಡಲದಲ್ಲಿನ ಹೂಮಾಲೆಯಿಂದ, ಕುಂಕುಮದ ಧೂಳನ್ನು ಅಂಟಿಸಿಕೊಂಡು ಬಂದಿ ರುವ, ನಿನ್ನ ಆ ಮೀಸೆಯಿಂದ ನನ್ನ ಕಾಲನ್ನು ಮುಟ್ಟಬೇಡ. ಭ್ರಮರಪತಿಯಾದ (ಭ್ರಮರ ಹೊಸ ಹೂಗಳನ್ನು ಅರಸುವಂತೆ ಹೊಸ ಹೊಸ ಹೆಣ್ಣುಗಳನ್ನು ಅರಸುವ) ಆ ಶ್ರೀಕೃಷ್ಣ ಆ ನನ್ನ ಸವತಿಯರ ಅನುಗ್ರಹವನ್ನೆ ಸಂಪಾದಿಸಿಕೊಂಡು, ಯಾದವರ ಸಭೆಯಲ್ಲಿ ಎಲ್ಲರ ಹಾಸ್ಯಕ್ಕೆ ಗುರಿಯಾಗಲಿ.

\begin{verse}
ಸಕೃದಧರಸುಧಾಂ ಸ್ವಾಂ ಮೋಹಿನೀಂ ಪಾಯಯಿತ್ವಾ ।\\ಸುಮಸನ ಇವ ಸದ್ಯಸ್ತತ್ಯಜೇಽಸ್ಮಾನ್ ಭವಾದೃಕ್ ॥\\ಪರಿಚರತಿ ಕಥಂ ತತ್ಪಾದಪದ್ಮಂ ತು ಪದ್ಮಾ ।\\ಅಪಿ ಬತ! ಹೃತಚೇತಾ ಹ್ಯುತ್ತಮಶ್ಲೋಕಜಲ್ಪೈಃ \num{॥ ೨ ॥}
\end{verse}

ಶ್ರೀಕೃಷ್ಣನೆಂಬ ಆ ಭ್ರಮರನು ಒಮ್ಮೆ ಮಾತ್ರ ತನ್ನ ಅಧರಾಮೃತವನ್ನು ನಮಗೆ ಪಾನ ಮಾಡಿಸಿ, ಮರುಕ್ಷಣದಲ್ಲಿಯೇ, ದುಂಬಿ ಒಂದು ಹೂವಿನಿಂದ ಮತ್ತೊಂದು ಹೂವಿಗೆ ಹಾರಿಹೋಗುವಂತೆ, ನಮ್ಮಿಂದ ದೂರ ಹೋದ. ಅಂತಹ ಮೋಸಗಾರನಾದ ಗಂಡನನ್ನು ಮಹಾಲಕ್ಷ್ಮಿ ಹೇಗೆ ತಾನೇ ಸೇವೆ ಮಾಡುತ್ತಾಳೊ! ಅವನ ಸವಿನುಡಿಗೆ ಬೆರಗಾಗಿ ಆಕೆ ಅವನನ್ನು ಸೇವಿಸುತ್ತಾಳೆ, ಅಷ್ಟೆ.

\begin{verse}
ಕಿಮಿಹ ಬಹು ಷಡಂಘ್ರೇ! ಗಾಯಸಿ ತ್ವಂ ಯದೂನಾಂ ।\\ಅಧಿಪತಿಮಗೃಹಾಣಾಮಗ್ರತೋ ನಃ ಪುರಾಣಂ ॥\\ವಿಜಯಸಖೀನಾಂ ಗೀಯತಾಂ ತತ್ಪ್ರಸಂಗಃ ।\\ಕ್ಷುಪಿತಕುಚರುಜಸ್ತೇ ಕಲ್ಪಯಂತೀಷ್ಟಮಿಷ್ಟಾಃ \num{॥ ೩ ॥}
\end{verse}

ಹೇ ಭ್ರಮರವೆ! ಆ ಲಕ್ಷ್ಮಿಯಹಾಗೆ ನಾನೇನೂ ಮರುಳಾಗುವವಳಲ್ಲ. ನಾವು ಬಹು ಕಾಲದಿಂದ ಆತನನ್ನು ಬಲ್ಲೆವು. ನಮ್ಮ ಮುಂದೆ ಆತನನ್ನು ಹೊಗಳಿ ಏನು ಪ್ರಯೋಜನ? ಆತನ ಆಲಿಂಗನದಿಂದ ತಮ್ಮ ಸ್ತನಗಳ ನವೆಯನ್ನು ತೀರಿಸಿಕೊಳ್ಳುವ ನಮ್ಮ ಸವತಿಯರ ಮುಂದೆ ನೀನು ಆತನ ಗುಣಗಾನಮಾಡು. ಅವರು ಕೇಳಿ ಸಂತೋಷಪಡುತ್ತಾರೆ.

\begin{verse}
ದಿವಿ ಭುವಿ ಚರ ಸಾಯಾಂ ಕಾಸ್ತ್ರಿಯಸ್ತದ್ದುರಾಪಾಃ ।\\ಕಪಟ! ರುಚಿರಹಾಸಭ್ರೂವಿಜೃಂಭಸ್ಯ ಯಾ ಸ್ಯುಃ ॥\\ಚರಣರಜ ಉಪಾಸ್ತೇ ಯಸ್ಯ ಭೂತಿರ್ವಯಂ ಕಾ ।\\ಹ್ಯಪಿ ಚ ಕೃಪಣ ಪಕ್ಷೇ ಹ್ಯುತ್ತಮಶ್ಲೋಕಶಬ್ದಃ \num{॥ ೪ ॥}
\end{verse}

ಎಲಾ ಕಪಟಿಯಾದ ಭ್ರಮರ! ಭೂ ಸ್ವರ್ಗ ಪಾತಾಳಗಳೆಂಬ ಮೂರು ಲೋಕಗಳ ಲ್ಲಿಯೂ ಇರುವ ಹೆಣ್ಣುಗಳಲ್ಲಿ ಯಾರು ತಾನೆ ಆ ಶ್ರೀಕೃಷ್ಣನ ಮನೋಹರವಾದ ಬೆಡಗು ಬಿನ್ನಾಣದ ಭ್ರೂಭಂಗದಿಂದಲೂ ಮೋಸಹೋಗದಿರುತ್ತಾರೆ? ಸ್ವಭಾವಚಪಲೆಯಾದ ಆ ಲಕ್ಷ್ಮಿಯೇ ಸದಾ ಅವನ ಪದತಲದ ಧೂಳಿಯಲ್ಲಿ ಬಿದ್ದು ಹೊರಳಾಡುತ್ತಿರುವಾಗ ನಮ್ಮಂ ತಹರ ಪಾಡೇನು? ಇಷ್ಟೆಲ್ಲ ಅನ್ಯಾಯಗಳನ್ನು ಮಾಡುತ್ತಿದ್ದರೂ, ಅವನನ್ನು ‘ದೀನ ಬಂಧು’, ‘ಉತ್ತಮ ಶ್ಲೋಕ’ಎಂದು ಕರೆಯುತ್ತಾರೆ.

\begin{verse}
ವಿಸೃಜ ಶಿರಸಿ ಪಾದಂ ವೇದ್ಮ್ಯಹಂ ಚಾಟುಕಾರೈ ।\\ರನುನಯ ವಿದುಷಸ್ತೇಭ್ಯೇತ್ಯ ದೌತ್ಯೈರ್ಮುಕುಂದಾತ್ ॥\\ಸ್ವಕೃತ ಇಹ ವಿಸೃಷ್ಟಾಪತ್ಯ ಪತ್ಯನ್ನಲೋಕಾ ।\\ವ್ಯಸೃಜದಕೃತ ಚೇತಾಃ ಕಿನ್ನು ಸಂಧೇಯಮಸ್ಮಿನ್ \num{॥ ೫ ॥}
\end{verse}

ಎಲಾ ಭ್ರಮರ! ಆ ಮುಕುಂದನ ಬಳಿಯಲ್ಲಿ ಕಲಿತು ಬಂದಿರುವ ಚಮತ್ಕಾರದ ಮಾತು ಗಳಿಂದ ನಮ್ಮನ್ನು ಮರುಳುಗೊಳಿಸುವ ನಿನ್ನ ಮೋಸವಿದ್ಯೆ ನನ್ನಲ್ಲಿ ನಡೆಯುವುದಿಲ್ಲ. ನೀನೇನೂ ನನ್ನ ಪಾದಕ್ಕೆ ಅಡ್ಡಬೀಳಬೇಡ. ಅವನಿಗಾಗಿ ಪತಿಪುತ್ರಾದಿಗಳನ್ನೂ, ಲೋಕ ಧರ್ಮವನ್ನೂ, ಪರಲೋಕವನ್ನೂ ತಿರಸ್ಕರಿಸಿ, ಅವನನ್ನೆ ನಂಬಿ ಬಂದ ನಮ್ಮನ್ನು ಅವನು ತ್ಯಜಿಸಿಹೋದ. ಇನ್ನು ಅವನ ವಿಷಯದಲ್ಲಿ ಸಂಧಿಯ ಪ್ರಸ್ತಾಪ ಏಕೆ?

\begin{verse}
ಮೃಗಯುರಿವ ಕಪೀಂದ್ರಂ ವಿವ್ಯಧೇ ಲುಬ್ಧಧರ್ಮಾ ।\\ಸ್ತ್ರಿಯಮಕೃತವಿರೂಪಾಂ ಸ್ತ್ರೀಜಿತಃ ಕಾಮಯಾನಾಂ ॥\\ಬಲಿಮಪಿ ಬಲಿಮತ್ವಾವೇಷ್ಟಯಧ್ವಾಂಕ್ಷವದ್ಯಃ ।\\ತದಲಮಸಿತಸಖ್ಯೈರ್ದುಸ್ತ್ಯಜಸ್ತತ್ಕಥಾರ್ಥಃ \num{॥ ೬ ॥}
\end{verse}

ಆಹಾ! ಆ ಕೃಷ್ಣನು ಮಹಾ ಕ್ರೂರಿ. ಬೇಡನಂತೆ ಮರೆಯಲ್ಲಿ ನಿಂತು ವಾಲಿಯನ್ನು ಕೊಂದ; ತನ್ನ ಹೆಂಡತಿಗೆ ವಶವಾಗಿ ಹೆಂಗೊಲೆಗೆ ಹೇಸದೆ, ತನ್ನನ್ನು ಕಾಮಿಸಿಬಂದ ಶೂರ್ಪ ಣಖಿಯನ್ನು ವಿರೂಪಗೊಳಿಸಿದ; ಬಲಿಯು ಕೊಟ್ಟ ಆಹಾರವನ್ನೆಲ್ಲ ಕಾಗೆಯಂತೆ ತಿಂದು, ಅನ್ನವಿಕ್ಕಿದವನನ್ನೆ ಕಟ್ಟಿಹಾಕಿದ. ಅವನ ಸ್ನೇಹ ಸಾಕು. ಆದರೇನು ಮಾಡಲಿ? ಅವನ ಗುಣಗಳನ್ನು ಮಾತ್ರ ಹೊಗಳದಿರಲು ನನಗೆ ಸಾಧ್ಯವಾಗುತ್ತಾ ಇಲ್ಲ.

\begin{verse}
ಯದನುಚರಿತಲೀಲಾಕರ್ಣಪೀಯೂಷವಿಪ್ರುಟ್ ।\\ಸಕೃದದನ ವಿಭೂತದ್ವಂದ್ವಧರ್ಮಾ ವಿನಷ್ಟಾಃ ॥\\ಸಪದಿ ಗೃಹಕುಟುಂಬಂ ದೀನಮತ್ಸೃಜ್ಯ ದೀನಾ ।\\ಬಹವ ಇವ ವಿಹಂಗಾ ಭಿಕ್ಷುಚರ್ಯಾಂ ಚರಂತಿ \num{॥ ೭ ॥}
\end{verse}

ಎಲೆ ಭ್ರಮರ! ರಾಗದ್ವೇಷಗಳನ್ನು ಕಳೆದುಕೊಂಡು ಮಹಾ ಯೋಗಿಗಳೆನಿಸಿಕೊಂಡ ವರು ಕೂಡ ಆ ಶ್ರೀಕೃಷ್ಣನ ಲೀಲಾವಿಲಾಸಗಳೆಂಬ ಅಮೃತಬಿಂದುಗಳನ್ನು ಸೇವಿಸುತ್ತಲೆ ತಮ್ಮ ಮಡದಿ ಮಕ್ಕಳನ್ನೆಲ್ಲ ಭಿಕಾರಿಗಳನ್ನಾಗಿ ಮಾಡಿ, ತಾವೂ ಭಿಕ್ಷುಕರಾಗಿ ಹಕ್ಕಿಗಳಂತೆ ಮರದಿಂದ ಮರಕ್ಕೆ ಅಲೆಯಬೇಕಾಗುತ್ತದೆ ಎಂದಮೇಲೆ ಮುಗ್ಧೆಯರಾದ ನಮ್ಮ ಗತಿ ಹೇಗಿರಬೇಕು?

\begin{verse}
ವಯಮೃತಮಿವ ಜಿಹ್ಮವ್ಯಾಹೃತಂ ಶ್ರದ್ದಧಾನಾಃ ।\\ಕುಳಿಕರುತಮಿವಾಜ್ಞಾಃ ಕೃಷ್ಣ ವಧ್ವೋ ಹರಿಣ್ಯಃ ॥\\ದದೃಶಿಮ ಸಕೃದೇತತ್ತನ್ನಖಸ್ಪರ್ಶತೀವ್ರ ।\\ಸ್ಮರರುಜ ಉಪಮಂತ್ರಿನ್​! ಭಣ್ಯತಾಮನ್ಯವಾರ್ತಾ \num{॥ ೮ ॥}
\end{verse}

ಎಲೆ ಕೃಷ್ಣಮಂತ್ರಿ! ಬೇಟೆಗಾರನ ಕಪಟಗಾನಕ್ಕೆ ಮರುಳಾಗುವ ಜಿಂಕೆಗಳಂತೆ, ಶ್ರೀ ಕೃಷ್ಣನ ಬೆಲ್ಲಿ ಮಾತುಗಳಿಗೆ ಮರುಳಾಗಿ, ನಾವು ಅವನ ನಖಕ್ಷತಿಗೆ ಒಳಗಾಗಿ ಕಾಮಬಾಧೆ ಯಿಂದ ನರಳುತ್ತಿದ್ದೇವೆ. ಸಾಕು ಅವನ ಸಹವಾಸ, ನಮಗೆ. ಅವನ ಸುದ್ದಿಯನ್ನು ಬಿಟ್ಟು ಇನ್ನೇನಾದರೂ ಇದ್ದರೆ ನಮಗೆ ಹೇಳು.

\begin{verse}
ಪ್ರಿಯಸಖ ಪುನರಾಗಾಃ ಪ್ರೇಯಸಾ ಪ್ರೇಷಿತಃ ಕಿಂ ।\\ವರಯ ಕಿಮನುರುಂಧೇ ಮಾನನೀಯೋಸಿ ಮೇಂಗ ॥\\ನಯಸಿ ಕಥಮಿಹಾಸ್ಮಾನ್ ದುಸ್ತ್ಯಜದ್ಪಂದ್ವಪಾರ್ಶ್ವಂ ।\\ಸತತಮುರಸಿ ಸೌಮ್ಯ! ಶ್ರೀವಧೂಸ್ಸಾಕಮಾಸ್ತೇ \num{॥ ೯ ॥}
\end{verse}

ನಮ್ಮ ಪ್ರಾಣಪ್ರಿಯನಾದ ಶ್ರೀಕೃಷ್ಣನ ಗೆಳೆಯನೇ, ಭ್ರಮರನೆ, ಮತ್ತೆ ಮತ್ತೆ ನಮ್ಮ ಬಳಿ ಏಕೆ ಬರುತ್ತಿ? ಶ್ರೀಕೃಷ್ಣನು ಮತ್ತೆ ನಿನ್ನನ್ನು ಕಳುಹಿಸಿದನೊ? ಹಾಗಾದರೆ ನೀನು ನಮಗೆ ಪೂಜ್ಯನೇ ಸರಿ. ನಿನ್ನ ಕೋರಿಕೆಯೇನು ಹೇಳು. ನಮ್ಮನ್ನು ಅವನ ಬಳಿಗೆ ಬರುವಂತೆ ಹೇಳುತ್ತಿರುವೆಯಾ? ಅವನ ಬಳಿಗೆ ನಮ್ಮನ್ನು ಕರೆದುಕೊಂಡು ಹೋಗಲು ನಿನಗೆ ಹೇಗೆ ಸಾಧ್ಯ? ಇತರ ಹೆಣ್ಣುಗಳನ್ನು ನೀನು ಓಡಿಸಿದರೂ ಆತನ ಎದೆಯಲ್ಲಿ ನೆಲೆಸಿರುವ ಲಕ್ಷ್ಮೀ ದೇವಿಯನ್ನು ಏನು ಮಾಡುವಿ? ನಾವಲ್ಲಿಗೆ ಹೇಗೆ ಬರೋಣ?

\begin{verse}
ಅಪಿ ಬತ! ಮಧುಪುರ್ಯಾಮಾರ್ಯಪುತ್ರೋಽಧುನಾಸ್ತೇ ।\\ಸ್ಮರತಿ ಸಪಿತೃಗೇಹಾನ್ ಸೌಮ್ಯ! ಬಂಧೂಂಶ್ಚ ಗೋಪಾನ್ ॥\\ಕ್ವಚಿದಪಿ ಸ ಕಥಾ ನಃ ಕಿಂಕರೀಣಾಂ ಗೃಣೀತೇ ।\\ಭುಜಮಗರುಸುಗಂಧಂಮೂರ್ಧ್ನ್ಯ ಧಾಸ್ಯತ್ಕದಾ ನು? \num{॥ ೧೦ ॥}
\end{verse}

ಎಲೆ ಭ್ರಮರ! ಆರ್ಯಪುತ್ರನಾದ ಶ್ರೀಕೃಷ್ಣನು ಈಗ ಮಧುರಾನಗರಿಯಲ್ಲಿಯೆ ನೆಲೆಸಿರು ವನೆ? ಎಲೆ ಶಾಂತಮೂರ್ತಿ, ಆತನು ತನ್ನ ತಾಯ್ತಂದೆಗಳನ್ನೂ ಮನೆಯನ್ನೂ ಬಂಧು ಬಳಗವನ್ನೂ ಒಡನಾಡಿಗಳಾದ ಗೋಪಾಲರನ್ನೂ ಇನ್ನೂ ಜ್ಞಾಪಕದಲ್ಲಿಟ್ಟುಕೊಂಡಿರು ವನೊ? ದಾಸಿಯರಾದ ನಮ್ಮನ್ನು ಯಾವಾಗಲಾದರೂ ಒಂದೊಂದು ಸಾರಿಯಾದರೂ ಸ್ಮರಿಸಿಕೊಳ್ಳುತ್ತಾನೆಯೊ? ಅಯ್ಯೋ, ಆ ನಮ್ಮ ಪ್ರಾಣಪ್ರಿಯನು ಶ್ರೀಗಂಧ ಹಚ್ಚಿದ ತನ್ನ ಭುಜವನ್ನು ಪುನಃ ಎಂದಾದರೂ ನಮ್ಮ ತಲೆಯಮೇಲಿಡುವನೋ ಇಲ್ಲವೇ ಇಲ್ಲವೊ!


\section{೬. ವಸುದೇವನ ಸೋದರರು ಮತ್ತು ಅವರ ಮಡದಿ ಮಕ್ಕಳು}

ದೇವಭಾಗ = ಕಂಸೆ–ಚಿತ್ರಕೇತು, ಬೃಹದ್ಬಲ

ದೇವಶ್ರವ = ಕಂಸವತಿ–ಸುವೀರ, ಇಷುಮಂತ

ಅನಕ = ಕರ್ಣಿಕೆ–ಪುತಧಾಮ, ಜಯ

ಸೃಂಜಯ = ರಾಷ್ಟ್ರಪಾಲಿಕೆ–ವೃಷ, ದುರ್ಮರ್ಷಣ

ಶ್ಯಾಮಕ = ಸುರಭೂಮಿ–ಹರಿಕೇಶ, ಹಿರಣ್ಯಾಕ್ಷ

ಕಂಕ = ಕಂಕೆ–ಬಕ, ಸತ್ಯಜಿತ್, ಪುರುಜಿತ್​

ಅನೀಕ = ಸುಧಾಮ–ಸುಮಿತ್ರ. ಅನೀಕ, ಬಾಣ

ವತ್ಸಕ = ವಿಶ್ರಕೇಶಿ—ವೃಕ ಮೊದಲಾದವರು

ವೃಕ = ದೂರ್ವಾಕ್ಷಿ–ತಕ್ಷ, ಪುಷ್ಕರ ಮೊದಲಾದವರು

\begin{center}
\textbf{ವಸುದೇವನ ಹೆಂಡತಿಯರು ಮತ್ತು ಮಕ್ಕಳು}
\end{center}

\textbf{(೧) ಕಂಸನ ದೊಡ್ಡಪ್ಪನಾದ ದೇವುಕನ ಮಕ್ಕಳು}

೧. ಧೃತದೇವಾ–ತ್ರಿಪೃಷ್ಠ

೨. ಶಾಂತಿದೇವಾ–ಪ್ರಶ್ರಮ; ಪ್ರಶ್ರಿತ ಇತ್ಯಾದಿ

೩. ಉಪದೇವಾ–ಕಲ್ಪವರ್ಷ ಮೊದಲಾದ ಹತ್ತು ಮಂದಿ

೪. ದೇವರಕ್ಷಿತಾ–ಗದಾ ಮೊದಲಾದ ಒಂಬತ್ತು ಮಂದಿ

೫. ಶ್ರೀದೇವಾ–ವಸುಹಂಸ, ಸುಧನ್ವ ಇತ್ಯಾದಿ ಆರು ಜನ ಮಕ್ಕಳು

೬. ಸಹದೇವಾ–ಪ್ರರೂಢ, ಶೃತ ಮೊದಲಾದ ಎಂಟು ಮಂದಿ

೭. ದೇವಕಿ–ಕೀರ್ತಿಮಂತ, ಸುಷೇಣ, ಭದ್ರಸೇನ, ಪುಜು, ಸಮದನ, ಭದ್ರ, ಸಂಕರ್ಷಣ, ಶ್ರೀಕೃಷ್ಣ, ಸುಭದ್ರೆ (ಮಗಳು)

\textbf{(೨) ಇತರರು}

೮. ರೋಹಿಣಿ–ಬಲರಾಮ, ಗದಸಾರಣ, ದುರ್ಮದ, ವಿಪುಲ, ಧ್ರುವ, ಕೃತ

೯. ಪೌರವಿ–ಸುಭದ್ರ, ಭದ್ರಬಾಹು, ದುರ್ಮದ, ಭದ್ರ, ಭೂತ–ಇತ್ಯಾದಿ ಹನ್ನೆರಡು ಜನ

೧೦. ಮದಿರೆ–ನಂದ, ಉಪನಂದ, ಕೃತಕ, ಶೂರ

೧೧. ಕೌಸಲ್ಯೆ–ಕೇಶಿ

೧೨. ರೋಚನೆ–ಹಸ್ತ, ಹೇಮಾಂಗದ ಇತ್ಯಾದಿ

೧೩. ಇಳೆ–ಇರುವಲ್ಕ ಇತ್ಯಾದಿ

\begin{center}
\textbf{ವಸುದೇವನ ಸೋದರಿಯರು ಮತ್ತು ಅವರ ಪತಿ ಪುತ್ರರು}
\end{center}

೧. ಪೃಥಾ (ಕುಂತೀರಾಜನ ಸಾಕುಮಗಳು)=ಪಾಂಡು–ಕರ್ಣ, ಧರ್ಮರಾಯ, ಭೀಮ, ಅರ್ಜುನ.

೨. ಶ್ರುತದೇವಾ = ಕುರೂಷದೇಶದ ವೃದ್ಧಕರ್ಮ–ದಂತವಕ್ತ್ರ

೩. ಶ್ರುತಕೀರ್ತಿ = –ಸಂತದರ್ಅನ ಮೊದಲಾದ ಐವರು ಮಕ್ಕಳು\\ಧೃಷ್ಟಕೇತು = ಕೇಕಯ (ಇವರಿಬ್ಬರೂ ಸೇರಿ ಈಕೆಯನ್ನು ಕೆಡಿಸಿದರು)

೪. ಶ್ರತಶ್ರವಾ = ಚೇದಿರಾಜನಾದ ದಮಘೋಷ–ಶಿಶುಪಾಲ

೫. ರಾಜಾಧಿದೇವಿ = ಜಯಸೇನ–ವಿಂದ, ಅನುಮಿಂದ

\begin{center}
\textbf{ಶ್ರೀಕೃಷ್ಣನ ಅಷ್ಟಮಹಿಷಿಯರು ಮತ್ತು ಅವರ ಮಕ್ಕಳು}
\end{center}

೧. ರುಕ್ಮಿಣಿ–ಪ್ರದ್ಯುಮ್ನ, ಚಾರುಧೇಷ್ಣ, ಸುಧೇಷ್ಮ, ಚಾರುದೇವ, ಸುಚಾರು, ಚಾರು ಗುಪ್ತ, ಭದ್ರಚಾರು, ಚಾರುಭಧ್ರ, ವಿಚಾರು, ಚಾರುಮಂತ, ಚಾರುಮತಿ (ಮಗಳು) = ಕೃತವರ್ಮ.

೨. ಸತ್ಯಭಾಮೆ–ಭಾನು, ಸುರ್ಭಾನು, ಸ್ವರ್ಭಾನು, ಪ್ರಭಾನು, ಭಾನುಮಂತ, ಚಂದ್ರ ಭಾನು, ಬೃಹದ್ಭಾನು, ರವಿಭಾನು, ಶ್ರೀಭಾನು, ಪ್ರತಿಭಾನು.

೩. ಜಾಂಬವತಿ–ಸಾಂಬ, ಸುಮಿತ್ರ, ಪುರುಜಿತ್ತು, ಶತಚಿತ್ತು, ಸಹಸ್ರಜಿತ್ತು, ವಿಜಯ, ಚಿತ್ರಕೇತು, ಕೀರ್ತಿಮಂತ, ದ್ರವಿಣ, ಕ್ರತು.

೪. ನಾಗ್ನಜಿತಿ–ಭಾನು, ಚಂದ್ರ, ಅಶ್ವಸೇನ, ಚಿತ್ರಗು, ವೇಗವಂತ, ವೃಷಭ, ಆಮ, ಶಂಕು, ವಸು, ಕೃತಿ.

೫. ಕಾಳಿಂದಿ–ಕವಿ, ವೃಷ, ವೀರ, ಸುಬಾಹು, ಭದ್ರ, ಏಕಲ, ಶಾಂತಿ, ದರ್ಶ, ಪೂರ್ಣ ಮಾಸ, ಸೋಮಕ. 

೬. ಲಕ್ಷಣೆ–ಪ್ರಘೋಷ, ಗಾತ್ರವಂತ, ಸಿಂಬ, ಬಲ, ಪ್ರಬಲ, ಊರ್ಧ್ವಗ, ಮಹಾ ಶಕ್ತಿ, ಸಹ, ತೇಜ, ಅಪರಾಜಿತ.

೭. ಮಿತ್ರವಿಂದೆ–ವೃಕ, ಅರ್ಕ, ಅನಿಲ, ಗೃಧ್ರ, ಬಹ್ವನ್ನ, ನಾಥ, ಮಹಾಶ, ಪಾವನ, ವಹ್ನಿ, ಸುಧೀ.

೮. ಭದ್ರಾದೇವಿ–ಸಂಗ್ರಾಮಜಿತ್ತು, ಬೃಹತ್ಸೇನ, ಶೂರ, ಪ್ರಹರಣ, ಅರಿಜಿತ್ತು, ಯಜ್ಞ, ಸುಭದ್ರ, ವಾಮನ, ಆಯು, ಸತ್ಯಕ.


\section{೭. ಮಹಿಷೀ ಗೀತೆ}

ಶ್ರೀಕೃಷ್ಣನ ದ್ವಾರಕಿ ಭೂಲೋಕದ ಸ್ವರ್ಗದಂತಿತ್ತು. ಹೆಣ್ಣು ಗಂಡುಗಳು ಸುಖ ಸಂತೋಷಗಳಿಂದ ನಲಿಯುತ್ತಿದ್ದರು. ಶ್ರೀಕೃಷ್ಣನ ರಾಣಿವಾಸವಂತೂ ಆನಂದದ ತೆರೆ ಗಳನ್ನು ಸದಾ ಸೂಸುವ ಭೋಗ ಸರಸಿಯಾಗಿತ್ತು. ಒಂದು ದಿನ ಆತನು ತನ್ನ ಹದಿನಾರು ಸಹಸ್ರದ ಒಂದುನೂರು ರಾಣಿಯರೊಡನೆ ಅಷ್ಟೇ ಸಂಖ್ಯೆಯ ಶ್ರೀಕೃಷ್ಣಾಕೃತಿಗಳನ್ನು ತಾಳಿ ಅವರೊಡನೆ ಜಲಕ್ರೀಡೆಯಾಡಿದನು. ಆತನು ಸ್ವಲ್ಪ ಮರೆಯಾಗುತ್ತಲೆ ಆ ಹೆಣ್ಣುಗಳೆಲ್ಲ ಹುಚ್ಚರಂತೆ ಆಡಿ ಹಾಡಿಕೊಂಡುದೆ ಮಹಿಷೀಗೀತೆ.

ಅವರಲ್ಲಿ ಒಬ್ಬಳು ಕುರರಪಕ್ಷಿಯನ್ನು ಕುರಿತು ಹೇಳುತ್ತಾಳೆ–

\begin{verse}
ಕುರರಿ! ವಿಲಪಸಿ ತ್ವಂ ವೀತನಿದ್ರಾನಶೇಷೇ\\ಸ್ವಪಿತಿ ಜಗತಿ ರಾತ್ರ್ಯಾಮೀಶ್ವರೋ ಗುಪ್ತಬೋಧಃ ॥\\ವಯಮಿವ ಸಖಿ! ಕಚ್ಚಿದ್ಗಾಢನಿರ್ಭಿನ್ನ ಚೇತಾಃ \\ನಳಿನನಯನ ಹಾಸೋದಾರ ಲೀಲೇಕ್ಷಿತೇನ \num{॥ ೧ ॥}
\end{verse}

‘ಹೇ, ಕುರರ ಪಕ್ಷಿ! ಭಗವಂತನಾದ ಶ್ರೀಕೃಷ್ಣನು ಜಗತ್ತಿನ ಜಂಜಡವನ್ನೆಲ್ಲ ಮರೆತು ಈಗ ಸುಖವಾಗಿ ನಿದ್ರಿಸುತ್ತಿದ್ದಾನೆ. ನೀನೂ ನಿದ್ದೆ ಮಾಡುವುದು ಬಿಟ್ಟು, ಆತನ ನಿದ್ರಾ ಭಂಗವಾಗುವಂತೆ ಗೋಳಾಡುತ್ತಿರುವೆಯಲ್ಲಾ! ನೀನೂ ನಮ್ಮಂತೆಯೆ ಆ ಶ್ರೀಕೃಷ್ಣನ ನಗೆ ನೋಟಗಳಿಗೆ ಮನಸೋತು, ಆತನ ವಿರಹವೇದನೆಯಿಂದ ಅಳುತ್ತಿರುವೆಯೋ ಹೇಗೆ?’

ಮತ್ತೊಬ್ಬಳು ಚಕ್ರವಾಕಕ್ಕೆ ಹೇಳುತ್ತಾಳೆ–

\begin{verse}
ನೇತ್ರೇ ನಿಮೀಲಯಸಿ ನತ್ತಮದೃಷ್ಟ ಬಂಧುಃ ।\\ತ್ವಂ ರೋರವೀಷಿ ಕರುಣಂ ಬತ ಚಕ್ರವಾಕಿ!\\ದಾಸ್ಯಂ ಗತಾ ವಯಮಿವಾಚ್ಯುತಪಾದಜುಷ್ಟಾಂ ।\\ಕಿಂ ವಾಸ್ರಜಂ ಸ್ಪೃಹಯಸೇ ಕಬರೇಣ ವೋಢುಂ \num{॥ ೨ ॥}
\end{verse}

ಎಲೆ ಚಕ್ರವಾಕಿ! ಈ ರಾತ್ರಿ ನಿನ್ನ ಗಂಡನನ್ನು ಕಾಣದುದಕ್ಕಾಗಿ ನೀನು ಹೀಗೆ ಕಣ್ಣು ಮುಚ್ಚಿಕೊಂಡು ಅಳುತ್ತಿರುವೆಯೋ, ಅಥವಾ ನನ್ನಂತೆ ನೀನೂ ಆ ಕೃಷ್ಣನ ಪಾದಸೇವೆ ಯನ್ನು ಬಯಸಿ, ಆತನ ಪಾದಕ್ಕೆ ಅರ್ಪಿಸಿದ ಪುಷ್ಪಮಾಲೆಯನ್ನು ತುರುಬಿನಲ್ಲಿ ಧರಿಸ ಬೇಕೆಂಬ ಆಸೆಯಿಂದ ಮರುಗುತ್ತಿರುವೆಯೊ!

ಮತ್ತೊಬ್ಬಳು ಸಮುದ್ರವನ್ನು ಕುರಿತು ಹೇಳುತ್ತಾಳೆ–

\begin{verse}
ಭೋ ಭೋಸ್ಸದಾ ನಿಷ್ಪನಸೇ ಉದನ್ವನ್ ।\\ಅಲಬ್ಧನಿದ್ರೋಽಧಿಗತಾತ್ಮಜಾ ನರಃ ॥\\ಕಿಂ ವಾ ಮುಕುಂದಾಪಹೃತಾತ್ಮಲಾಂಭನಃ ।\\ಪ್ರಾಪ್ತಾಃ ದಶಾಂ ತ್ವಂ ಚ ಗತೋ ದುರತ್ಯಯಾಂ \num{॥ ೩ ॥}
\end{verse}

ಹೇ ಸಮುದ್ರ! ನೀನೂ ನಮ್ಮಂತೆಯೆ ಹಗಲು ರಾತ್ರಿ ಒಂದೇ ಸಮನಾಗಿ ನಿದ್ರೆಯಿಲ್ಲದೆ ಮೊರೆದು ಭೋರಿಡುತ್ತಿರುವೆಯಲ್ಲಾ! ಆ ಶ್ರೀಕೃಷ್ಣ ನಮ್ಮನ್ನು ಆಲಿಂಗಿಸಿಕೊಂಡಾಗ ನಮ್ಮ ಕುಚಕುಂಕುಮವನ್ನು ಹಾರಿಸಿಕೊಂಡುಹೋದಂತೆ, ನಿನ್ನಲ್ಲಿದ್ದ ಕೌಸ್ತುಭಮಣಿ ಯನ್ನೂ ಹೊತ್ತುಕೊಂಡುಹೋಗಿದ್ದಾನೆ. ಪಾಪ, ನೀನೂ ನಮ್ಮಂತೆ ವ್ಯಥೆಗೊಳಗಾದವನೆ.

ಇನ್ನೊಬ್ಬಳು ಚಂದ್ರನೊಡನೆ ಹೇಳುತ್ತಾಳೆ–

\begin{verse}
ತ್ವಂ ಯಕ್ಷ್ಮಣಾ ಬಲವತಾಸಿ ಗೃಹೀತ ಇಂದೊ!\\ಕ್ಷೀಣಸ್ತಮೋ ನ ನಿಜದೀಧಿತಿಭಿಃ ಕ್ಷಿಣೋಷಿ ॥\\ಕಚ್ಚಿನ್ಮುಕುಂದ ಗದಿತಾನಿ ಯಥಾ ವಯಂ ತ್ವಂ \\ವಿಸ್ಮೃತ್ಯ ಭೋ! ಸ್ಥಗಿತ ಗೀರುಪಲಕ್ಷ್ಯಸೇನಃ \num{॥ ೪ ।।}
\end{verse}

ಹೇ ಚಂದ್ರ! ನೀನು ಕ್ಷಯರೋಗದಿಂದ ದಿನದಿನಕ್ಕೆ ಬಡವಾಗುತ್ತಿರುವೆ! ಕತ್ತಲೆಯನ್ನು ಹೋಗಲಾಡಿಸುವ ನಿನ್ನ ಶಕ್ತಿ ಕುಗ್ಗುತ್ತಿದೆ. ನೀನೂ ನಮ್ಮಂತೆಯೆ ಆ ಶ್ರೀಕೃಷ್ಣನ ಮುದ್ದು ಮಾತುಗಳನ್ನು ನೆನೆದು, ಆ ಚಿಂತೆಯಿಂದಲೆ ಕೊರಗಿ ಬಡವಾಗುತ್ತಿರುವೆಯಾ? ಅಹುದು ಹಾಗೆಯೇ ಇರಬೇಕು.

ಬೇರೊಬ್ಬಳು ಮಲಯಮಾರುತವನ್ನು ಕುರಿತು ಹೇಳುತ್ತಾಳೆ–

\begin{verse}
ಕಿಂ ತ್ವಾಚರಿತಮಾಸ್ಮಾಭಿರ್ಮಲಯಾನಲ! ತೇ ಪ್ರಿಯಂ?\\ಗೋವಿಂದಾಪಾಂಗನಿರ್ಭಿನ್ನೇ ಹೃದೀರಯಸಿ ನಸ್ಸ್ಮರಂ \num{॥ ೫ ॥}
\end{verse}

ಎಲೆ ಮಂದಮಾರುತ! ನಿನಗೆ ನಾನಾವ ಅಪಕಾರ ಮಾಡಿದೆ? ಮೊದಲೆ ಶ್ರೀಕೃಷ್ಣನ ಕಡೆ ಗಣ್ ನೋಟದ ಬಾಣಗಳಿಂದ ಸೀಳಿ ಹೋಗಿರುವ ನನ್ನ ಹೃದಯವನ್ನು ನೀನು ಮತ್ತಷ್ಟು ನೋಯಿಸುತ್ತಿರುವೆಯಲ್ಲಾ!

ಮಗುದೊಬ್ಬಳು ಮೇಘವನ್ನು ಕುರಿತು ಹೇಳುತ್ತಾಳೆ–

\begin{verse}
ಮೇಘ! ಶ್ರೀಮಂಸ್ತ್ವಮತಿದಯಿತೋ ಯಾದವೇಂದ್ರಸ್ಯ ನೂನಂ ।\\ಶ್ರೀವತ್ಸಾಂಕಂ ವಯಮಿವ ಭವಾನ್ ಧ್ಯಾಯತಿ ಪ್ರೇಮಬದ್ಧಃ ।\\ಅತ್ಯುತ್ಕಂಠಶ್ಯಬಲಹೃದಯೋಸ್ಮದ್ವಿಧೋ ಬಾಷ್ಪಧಾರಾಃ ।\\ಸ್ಮೃತ್ವಾ ಸ್ಮೃತ್ವಾ ವಿಸೃಜಸಿ ಮುಹುಃ ದುಃಖದಸ್ತತ್ಪ್ರಸಂಗಃ \num{॥ ೬ ॥}
\end{verse}

ಎಲೆ ಮೇಘರಾಜ! ನೀನು ಭಾಗ್ಯಶಾಲಿ. ಶ್ರೀಕೃಷ್ಣನ ಮೈಮೇಲೆ ಬಿಸಿಲು ಬೀಳದಂತೆ ಮಾಡಿ ಆತನಿಗೆ ಪ್ರಿಯವನ್ನುಂಟುಮಾಡುವೆ. ನೀನೂ ನಮ್ಮಂತೆ ಆತನನ್ನು ನಿಶ್ಚಲವಾದ ಮನಸ್ಸಿನಿಂದ ಧ್ಯಾನಿಸುವೆಯೇನು? ಅದರಿಂದಲೆ ನೀನೂ ನಮ್ಮಂತೆ ಬಾರಿಬಾರಿಗೂ ಕಣ್ಣೀರನ್ನು ಸುರಿಸುವೆ! ಆ ಕೃಷ್ಣನ ಸಹವಾಸ ಎಂತವಹರಿಗೂ ದುಃಖಹೇತುವೆ!

ಬೇರೊಬ್ಬಳು ಕೋಗಿಲೆಗೆ ಹೇಳುತ್ತಾಳೆ–

\begin{verse}
ಪ್ರಿಯರಾವ! ಪದಾನಿ ಭಾಷಸೇ ಮೃತಸಂಜೀವಿಕಯಾನಯಾ ಗಿರಾ ।\\ಕರವಾಣಿ ಕಿಮದ್ಯ ತೇ ಪ್ರಿಯಂ ವದ ಮೇ ವಲ್ಲತಕಂಠ ಕೋಕಿಲ \num{॥ ೭ ॥}
\end{verse}

ಹೇ ಕೋಗಿಲೆ! ಸತ್ತವರನ್ನು ಕೂಡ ಬದುಕಿಸಬಲ್ಲ ನಿನ್ನ ಮಧುರಸ್ವರದಿಂದ, ಆ ಶ್ರೀಕೃಷ್ಣನಂತೆಯೆ ಪ್ರಿಯವಾಗಿ ನುಡಿಯುತ್ತಿರುವೆ! ಹೇಳು, ನಿನಗೆ ಏನು ಮಾಡಿದರೆ ಪ್ರಿಯವಾಗುತ್ತದೆ?

ಮತ್ತೊಬ್ಬ ರಮಣಿ ಗಿಳಿಗೆ ಹೇಳುತ್ತಾಳೆ–

\begin{verse}
ಶುಕರಾಜ! ಗಾಯ ಯದುದೇವ ಸತ್ಕಥಾಂ ।\\ಕುರ್ಮಸ್ತವಾಂಗಪಯಸೋಪಸೇಚನಂ ॥\\ವಿರಹಾಗ್ನಿತಪ್ತಹೃದಯಯೇಷ್ವಲಂ ರಮಾ ।\\ರಮಣಾಭಿಧಾನ ಸುಧಯಾ ಚ ಸಿಂಚ ನಃ \num{॥ ೮ ॥}
\end{verse}

ಹೇ ಅರಗಿಳಿ! ಶ್ರೀಕೃಷ್ಣನ ಶುಭಚರಿತ್ರೆಯನ್ನು ಗಾನಮಾಡು. ವಿರಹಾಗ್ನಿಯಿಂದ ಸುಟ್ಟು ಹೋಗುತ್ತಿರುವ ನಮ್ಮ ಮನಸ್ಸುಗಳನ್ನು ಶ್ರೀಕೃಷ್ಣಕಥಾಮೃತದಿಂದ ಶಾಂತಗೊಳಿಸಿದರೆ ನಿನಗೆ ಹಾಲಿನಿಂದ ಅಭಿಷೇಕಮಾಡುತ್ತೇನೆ.

ಬೇರೊಬ್ಬ ಸುಂದರಿ ಪರ್ವತಕ್ಕೆ ಹೇಳುತ್ತಾಳೆ–

\begin{verse}
ನ ಚಲಸಿ ನವದಸ್ಯುದಾರ ಬುದ್ಧೇ ।\\ಕ್ಷಿತಿಧರ! ಚಿಂತಯಸೇ ಮಹಾಂತಮರ್ಥಂ ॥\\ಅಪಿ ಬತ ವಸುದೇವನಂದನಾಂಘ್ರಿಂ ।\\ವಯಮಿವ ಕಾಮಯಸೇ ಸ್ತನೈರ್ವಿಧರ್ತು \num{॥ ೯ ॥}
\end{verse}

ಎಲೆ ಪರ್ವತರಾಜ! ಉದಾರಿಯಾದ ನೀನು ಅಲ್ಲಾಡುವುದಿಲ್ಲ, ಮಾತನಾಡುವುದಿಲ್ಲ, ಆದ್ದರಿಂದ ಯಾವುದೋ ಮಹತ್ತಾದ ವಿಷಯವನ್ನು ಯೋಚಿಸುತ್ತಿರುವಂತಿದೆ. ನಾವು ನಮ್ಮ ಸ್ತನಾಗ್ರಗಳಲ್ಲಿ ಶ್ರೀಕೃಷ್ಣನ ಪಾದಗಳನ್ನು ಧರಿಸುವುದಕ್ಕೆ ಆಸೆಪಡುವಂತೆಯೇ ನೀನು ನಿನ್ನ ಶಿಖರಗಳ ಕೊನೆಯಲ್ಲಿ ಶ್ರೀಕೃಷ್ಣನನ್ನು ಧರಿಸಬೇಕೆಂದು ಬಯಸುತ್ತಿರ ಬಹುದು. ಹಾಗಾದರೆ ನಿನ್ನ ಅವಸ್ಥೆಯೂ ನಮ್ಮಂತೆಯೇ ಆಗುತ್ತದೆ.

ಮಗುದೊಬ್ಬ ಕಾಮಿನಿ ನದಿಗೆ ಹೇಳುತ್ತಾಳೆ–

\begin{verse}
ಶುಷ್ಯದ್ಧ್ರದಾಃ ಕರ್ಶಿತಾ ಬತ ಸಿಂಧು ಪತ್ನ ್ಯಃ ।\\ಸಂಪ್ರತ್ಯಪಾಸ್ತಕಮಲಶ್ರಿಯ ಇಷ್ಟಭರ್ತುಃ ॥\\ಯದ್ವದ್ವಯಂ ಯದುಪತೇಃ ಪ್ರಣಯಾವಲೋಕಂ ।\\ಅಪ್ರಾಪ್ಯಮುಷ್ಟಹೃದಯಾಃ ಉರು ಕರ್ಶಿತಾಶ್ಚ \num{॥ ೧೦ ॥}
\end{verse}

ಹೇ ಸಾಗರಕಾಂತೆಯರೆ! ಇದೇನು ನೀವು ಹೀಗೆ ದಿನದಿನಕೂ ನೀರೊಣಗಿ ಬಡವಾಗು ತ್ತಿರುವಿರಿ? ನಾವು ಶ್ರೀ ಕೃಷ್ಣನ ಪ್ರಣಯಕಟಾಕ್ಷಕ್ಕೆ ಪಾತ್ರರಾಗದೆ ಶೂನ್ಯಹೃದಯೆಯರಾಗಿ ದುಃಖಿಸುತ್ತಿರುವಂತೆಯೆ, ನೀವೂ ನಿಮ್ಮ ಪ್ರಿಯರಮಣನಾದ ಸಮುದ್ರರಾಜನ ಪ್ರೇಮ ವೀಕ್ಷಣವಿಲ್ಲದೆ ಬಡವಾಗುತ್ತಿರುವಿರೇನು?

ಇನ್ನೊಬ್ಬಳು ಹಂಸಕ್ಕೆ ಹೇಳುತ್ತಾಳೆ–

\begin{verse}
ಹಂಸ! ಸ್ವಾಗತಮಾಸ್ಯತಾಂ ಪಿಬ ಪಯೋ ಬೂೃಹ್ಯಂಗ! ಶೌರೇಃ ಕಥಾಂ ।\\ದೂತಂ ತ್ವಾಂ ನು ವಿದಾಮ ಕಚ್ಚಿದಜಿತಃ ಸ್ವಸ್ತ್ಯಾಸ್ತ ಉಕ್ತಂ ಪುರಾ ॥\\ಕಿಂ ವಾ ನಶ್ಚಲಸೌಹೃದಃ ಸ್ಮರತಿ ತಂ ತಸ್ಮಾದ್ಭಜಾಮೋ ವಯಂ ।\\ಕ್ಷೌದ್ರಾಲಾಪಯ ಕಾಮದಂ ಶ್ರಿಯಮೃತೇ ಸೇವೈಕನಿಷ್ಠಾ ಪ್ರಿಯಾಂ \num{॥ ೧೧ ॥}
\end{verse}

ಎಲೆ ಹಂಸನೆ! ನಿನಗೆ ಸುಖಾಗಮನವೆ? ಬಾ, ಕುಳ್ಳಿರು ಹಾಲನ್ನು ಕುಡಿ. ನೀನು ಕೃಷ್ಣನ ದೂತನೆಂದು ಕೇಳಿದ್ದೇವೆ. ಆತನ ಸಮಾಚಾರವೇನು, ಹೇಳು. ಆತನ ಗೆಳತನ ಚಂಚಲ. ಆತ ನಮಗೆ ಹೇಳಿದುದನ್ನು ಜ್ಞಾಪಿಸಿಕೊಳ್ಳುತ್ತಾನೋ, ಇಲ್ಲವೊ, ಹೋಗಿ ಆತನನ್ನೆ ಇಲ್ಲಿಗೆ ಕರೆದುಕೊಂಡು ಬಾ. ಆ ಲಕ್ಷ್ಮಿಗೆ ಕಾಣದಂತೆ ಅವನನ್ನು ಕರೆದು ತಾ. ಅವಳಿಗೆ ಮಾತ್ರ ವೇನೋ ಶ್ರೀಕೃಷ್ಣನ ಸೇವೆ ಗೊತ್ತಿರುವುದು? ನಮಗೂಗೊತ್ತಿದೆ.

ಇಂತಹ ತದೇಕಧ್ಯಾನದಿಂದ ಅವರೆಲ್ಲರೂ ಕೃಷ್ಣಗತಿಯನ್ನು–ಎಂದರೆ ಮುಕ್ತಿಯನ್ನು ಪಡೆದರು.


\section{೮. ಪುರಾಣಗಳು ಮತ್ತು ಅವುಗಳ ಲಕ್ಷಣಗಳು}

ಪುರಾಣಗಳು ಹದಿನೆಂಟು. ಅವು ಕ್ರಮವಾಗಿ ಬ್ರಹ್ಮಪುರಾಣ, ಪದ್ಮಪುರಾಣ, ವಿಷ್ಣು ಪುರಾಣ, ಶಿವಪುರಾಣ, ಲಿಂಗಪುರಾಣ, ಗರುಡಪುರಾಣ, ನಾರದಪುರಾಣ, ಭಾಗವತ ಪುರಾಣ, ಅಗ್ನಿಪುರಾಣ, ಸ್ಕಾಂದಪುರಾಣ, ಭವಿಷ್ಯಪುರಾಣ, ಬ್ರಹ್ಮವೈವರ್ತಪುರಾಣ, ಮಾರ್ಕಂಡೇಯಪುರಾಣ, ವಾಮನ ಪುರಾಣ, ವರಾಹಪುರಾಣ, ಮತ್ಸ್ಯಪುರಾಣ, ಕೂರ್ಮಪುರಾಣ, ಬ್ರಹ್ಮಾಂಡಪುರಾಣ.

ಈ ಪುರಾಣಗಳಲ್ಲಿ ಮಹಾಪುರಾಣಗಳು, ಅಲ್ಪಪುರಾಣಗಳು ಎಂದು ಎರಡು ವಿಧ. ಮಹಾ ಪುರಾಣಕ್ಕೆ ಸರ್ಗ, ವಿಸರ್ಗ, ವೃತ್ತಿ, ರಕ್ಷೆ, ಮನ್ವಂತರ, ವಂಶ, ವಂಶಾನುಚರಿತ, ಸಂಸ್ಥೆ, ಹೇತು, ಅಪಾಶ್ರಯ–ಎಂಬ ಹತ್ತು ಲಕ್ಷಣಗಳು. ಅಲ್ಪಪುರಾಣಕ್ಕೆ ಸರ್ಗ, ಪ್ರತಿಸರ್ಗ, ವಂಶ, ಮನ್ವಂತರ, ವಂಶಾನುಚರಿತ–ಎಂಬ ಐದು ಲಕ್ಷಣಗಳು ಮಾತ್ರ ಇರುತ್ತವೆ. ಈ ಲಕ್ಷಣಗಳ ಸ್ವರೂಪ ಹೀಗಿರುತ್ತದೆ.

೧. \textbf{ಸರ್ಗ} – ಮೂಲ ಪ್ರಕೃತಿಯಿಂದ ಮಹತ್ ತತ್ವವೂ, ಅದರಿಂದ ಸತ್ವ, ರಜ, ತಮ ಗುಣಗಳೆಂಬ ಮೂರು ಬಗೆಯ ಅಹಂಕಾರವೂ, ಅದರಿಂದ ಆಕಾಶಾದಿ ಪಂಚಭೂತಗಳೂ ಶಬ್ದವೇ ಮೊದಲಾದ ತನ್ಮಾತ್ರಗಳೂ ಆಗುವ ಕ್ರಮಗಳನ್ನು ವಿವರಿಸುವುದು.

೨. \textbf{ವಿಸರ್ಗ} – ಮಹದಾದಿತತ್ವಗಳು ಪರಮೇಶ್ವರ ಪ್ರೇರಣೆಯಿಂದ ಸೃಷ್ಟಿ ಶಕ್ತಿಯನ್ನು ಹೊಂದಿ, ಜೀವರ ಪೂರ್ವಕರ್ಮದ ವಾಸನಾನುಸಾರವಾಗಿ, ಬೀಜದಿಂದ ಬೀಜ ಹುಟ್ಟು ವಂತೆ, ಚರಾಚರ ಪ್ರಪಂಚದ ಪ್ರವಾಹ ರೂಪದ ಸೃಷ್ಟಿ.

೩. \textbf{ವೃತ್ತಿ} – ಚರಾಚರ ಜಗತ್ತಿನಲ್ಲಿ ಚರಜೀವಿಗೆ ಅಚರವು ಆಹಾರವಾಗಿ ಕಲ್ಪಿಸಲ್ಪಟ್ಟಿದೆ. ಈ ಜಂಗಮ ಜೀವಿಗಳಲ್ಲಿಯೂ ಮನುಷ್ಯರಿಗೆ ಸ್ವಾಭಾವಿಕವಾಗಿ ಮಾತ್ರವೇ ಅಲ್ಲದೆ ಅಪೇಕ್ಷೆಯಿಂದಲೂ ಕಲ್ಪಿತವಾದ ಜೀವನೋಪಾಯ ಸಹ ಶಾಸ್ತ್ರಸಮ್ಮತವಾಗಿಯೇ ಇದೆ. ಇದನ್ನು ತಿಳಿಸುವುದು ವೃತ್ತಿ.

೪. \textbf{ರಕ್ಷೆ} – ಭಗವಂತನು ದುಷ್ಟಶಿಕ್ಷಣ ಶಿಷ್ಟರಕ್ಷಣೆಗಾಗಿ ಪ್ರತಿಯುಗದಲ್ಲಿಯೂ ದೇವತೆ, ಋಷಿ, ಮಾನವ, ತಿರ್ಯಕ್ ಮೊದಲಾದ ರೂಪಗಳಿಂದ ಅವತರಿಸುವ ಚರಿತ್ರೆ.

೫. \textbf{ಮನ್ವಂತರ} – ಮನುಗಳು, ದೇವತೆಗಳು, ಮನುಪುತ್ರರು, ಇಂದ್ರ, ಋಷಿಗಳು, ಭಗವಂತನ ಅಂಶಾವತಾರಗಳು – ಈ ಆರನ್ನೂ ಒಳಕೊಂಡ ಬೇರೆಬೇರೆ ಕಾಲವಿಭಾಗಗಳು.

೬. \textbf{ವಂಶ} – ಭೂತ, ಭವಿಷ್ಯತ್, ವರ್ತಮಾನಕಾಲಗಳಲ್ಲಿ ಪ್ರಸಿದ್ಧವಾದ ರಾಜವಂಶಗಳ ಪರಂಪರೆಯ ನಿರೂಪಣೆ.

೭. \textbf{ವಂಶಾನುಚರಿತ} – ಬೇರೆಬೇರೆ ರಾಜವಂಶಗಳ ರಾಜರ ಚರಿತ್ರೆ,

೮. \textbf{ಸಂಸ್ಥೆ} – ಕಾಲ, ಕರ್ಮ, ಸ್ವಭಾವಕ್ಕೆ ತಕ್ಕಂತೆ ಸಂಭವಿಸುವ ನೈಮಿತ್ತಿಕ, ಪ್ರಾಕೃತಿಕ, ನಿತ್ಯ, ಆತ್ಯಂತಿಕ ಎಂಬ ಪ್ರಳಯಗಳ ವಿವರಣೆ.\footnote{*ನೈಮಿತ್ತಿಕ ಪ್ರಳಯ–ಮನುಷ್ಯರ ಲೆಕ್ಕದಲ್ಲಿ ಒಂದು ಸಾವಿರ ಚತುರ್ಯುಗಗಳು ಬಹ್ಮನ ಒಂದು ಹಗಲಾಗುತ್ತದೆ. ಈ ಸಹಸ್ರ ಚತುರ್ಯುಗಗಳಾಗುತ್ತಲೆ ಪ್ರಳಯ ನಡೆಯುತ್ತದೆ. ಈ ಪ್ರಳಯಕಾಲವು ಒಂದು ಸಹಸ್ರ ಚತುರ್ಯುಗಗಳೆ. ಇದು ಬ್ರಹ್ಮನ ಒಂದು ರಾತ್ರಿ. ಆ ಕಾಲದಲ್ಲಿ ಮೂರು ಲೋಕಗಳೂ ಪ್ರಳಯ ವಾಗುತ್ತವೆ. ಇದು ನೈಮಿತ್ತಿಕ ಪ್ರಳಯ. ಬ್ರಹ್ಮನ ನಿದ್ರೆಗೆ ನಿಮಿತ್ತವಾದ್ದರಿಂದ ಇದು ನೈಮಿತ್ತಿಕ ಪ್ರಳಯ

ಪ್ರಾಕೃತಿಕ ಪ್ರಳಯ–ಚತುರ್ಮುಖಬ್ರಹ್ಮನ ಆಯುಸ್ಸು ಎರಡು ಪರಾರ್ಧಕಾಲ. ಅದು ಕಳೆಯುತ್ತಲೆ ಮಹತ್ ತತ್ವ, ಅಹಂಕಾರ, ಪಂಚತನ್ಮಾತ್ರಗಳು–ಈ ಏಳು ಪ್ರಕೃತಿಗಳೂ ಲಯವನ್ನು ಪಡೆಯುತ್ತವೆ. ಇದು ಪ್ರಾಕೃತಿಕ ಪ್ರಳಯ. ಈ ಪ್ರಳಯಕ್ಕೆ ಮುನ್ನ ನೂರು ವರ್ಷಗಳು ಮಳೆಯಾಗುವುದಿಲ್ಲ. ಜನ ಆಹಾರವಿಲ್ಲದೆ ಕಂಗೆಟ್ಟು ಒಬ್ಬರನ್ನೊಬ್ಬರು ಕಿತ್ತು ತಿನ್ನುತ್ತಾರೆ. ಸೂರ್ಯನು ಭಯಂಕರವಾದ ಬಿಸಿಲಿನಿಂದ ಲೋಕದ ಸಾರವನ್ನೆಲ್ಲ ಹೀರಿ ಬಿಡುತ್ತಾನೆ. ಪ್ರಳಯಾಗ್ನಿ ಉಲ್ಬಣಿಸಿ ಜನರನ್ನೆಲ್ಲ ಸುಟ್ಟುಹಾಕುತ್ತದೆ. ಬ್ರಹ್ಮಾಂಡವು ಸುಟ್ಟ ಸಗಣಿಯ ಉಂಡೆ ಯಂತಾಗುತ್ತದೆ. ಪ್ರಚಂಡವಾದ ಚಂಡಮಾರುತ ನೂರಾರು ವರ್ಷಗಳವರೆಗೆ ಎಡೆಬಿಡದೆ ಮಳೆ ಸುರಿಯುತ್ತದೆ. ಸಮುದ್ರಗಳೆಲ್ಲ ಒಂದಾಗಿ ಸೇರಿ ಜಳಪ್ರಳಯವಾಗುತ್ತದೆ. ಇದರಲ್ಲಿ ಪುರುಷನೂ ಪ್ರಕೃತಿಯೂ ಈಶ್ವರನಲ್ಲಿ ಲಯವಾಗಿ ಹೋಗುತ್ತವೆ.

ಆತ್ಯಂತಿಕ ಪ್ರಳಯ–ಬ್ರಹ್ಮಜ್ಞಾನದಿಂದ ಬರುವ ಮೋಕ್ಷವೇ ಆತ್ಯಂತಿಕ ಪ್ರಳಯ. ಪುರುಷನು ವಿವೇಕಜ್ಞಾನವೆಂಬ ಕತ್ತಿಯಿಂದ ಅಹಂಕಾರವೆಂಬ ಆತ್ಮಬಂಧವನ್ನು ಕತ್ತರಿಸಿ, ಪರಿಪೂರ್ಣವಾದ ಆತ್ಮನನ್ನು ಸಾಕ್ಷಾತ್ತಾಗಿ ಅನುಭವಿಸಿದಾಗ ಇದು ಸಂಭವಿಸುತ್ತದೆ. ಒಂದು ಮಾತಿನಲ್ಲಿ ಹೇಳುವುದಾದರೆ ದೇಹೋಪಾಧಿಯು ಸಂಪೂರ್ಣವಾಗಿ ತೊಲಗಿಹೋಗುವುದೆ ಆತ್ಯಂತಿಕ ಪ್ರಳಯ.

ನಿತ್ಯಪ್ರಳಯ–ಬ್ರಹ್ಮನಿಂದ ಹಿಡಿದು ಇರುವೆಯವರೆಗೆ ಸಕಲಭೂತಗಳೂ ನಿರಂತರವಾಗಿ ಉತ್ಪತ್ತಿ ಪ್ರಳಯಗಳನ್ನು ಪಡೆಯುತ್ತಿವೆಯೆಂದು ಜ್ಞಾನಿಗಳು ಹೇಳುತ್ತಾರೆ. ಬ್ರಹ್ಮನೆ ಮೊದಲಾದ ಸಮಸ್ತ ಶರೀರಗಳು ಯಾವಾಗಲೂ ಪರಿಣಾಮವನ್ನು ಪಡೆಯುತ್ತಿವೆ. ಅವು ತಮ್ಮ ಪೂರ್ಣಾವಸ್ಥೆಯನ್ನು ಬಿಡುವುದೆ ಪ್ರಳಯ. ಇದು ಸದಾ ನಡೆಯುತ್ತಿರುವುದರಿಂದ ಇದು ನಿತ್ಯಪ್ರಳಯ.}

೯. \textbf{ಹೇತು} – ಜೀವರು ಅಜ್ಞಾನದಿಂದ ಮಾಡಿದ ಕರ್ಮಕ್ಕೆ ತಕ್ಕ ಫಲವನ್ನು ಅನುಭವಿಸುವುದಕ್ಕಾಗಿ ಪ್ರಪಂಚಸೃಷ್ಟಿ ಎಂಬುದನ್ನು ತಿಳಿಸುವುದು.

೧೦. \textbf{ಅಪಾಶ್ರಯ} – ಮಾಯಾಮಯವಾದ ಜಾಗ್ರತ್ ಸ್ವಪ್ನ ಸುಷುಪ್ತಿಯೆಂಬ ಮೂರು ಬಗೆಯ ಜೀವವ್ಯಾಪಾರಗಳಿಗೆ ತಾನೆ ಕಾರಣನಾಗಿದ್ದರೂ, ಭಗವಂತನು ಆ ಅವಸ್ಥೆಗಳಿಗೆ ಈಡಾಗುವುದಿಲ್ಲ. ಅಂತಹ ಪರಬ್ರಹ್ಮನಿಗೆ 'ಅಪಾಶ್ರಯ' ವೆಂದು ಹೆಸರು.

