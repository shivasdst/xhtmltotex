
\chapter{೬೮. ಸ್ಯಮಂತಕ ಮಣಿ}

ಸುರಸುಂದರಿಯಾದ ಸತ್ಯಭಾಮೆಯನ್ನು ತಾನು ಮದುವೆ ಮಾಡಿಕೊಳ್ಳಬೇಕೆಂದು ಬಯಸಿದ್ದ, ಶತಧನ್ವ. ಸತ್ರಾಜಿತನೂ ಇದಕ್ಕೆ ಒಪ್ಪಿದ್ದ. ಆದರೆ ಆ ಮನೋಹರಿ ಶ್ರೀಕೃಷ್ಣನ ಪಾಲಾದುದನ್ನು ಕಂಡು ಅವನಿಗೆ ಸಹಿಸಲಾರದ ಆಶಾಭಂಗವಾಯಿತು. ಆತನು ಸತ್ರಾಜಿತನ ಮೇಲೆ ಸೇಡು ತೀರಿಸಿಕೊಳ್ಳಬೇಕೆಂದು ಸಮಯ ಕಾಯುತ್ತಿದ್ದ. ಅಕ್ರೂರ, ಶತಧನ್ವ ಎಂಬ ಆತನ ಗೆಳೆಯರಿಬ್ಬರು ಈ ದೆವ್ವಕ್ಕೆ ಧೂಪವನ್ನು ಹಾಕುತ್ತಿದ್ದರು. ಹೀಗಿರಲು, ಅರಗಿನ ಮನೆಯಲ್ಲಿ ಬೆಂಕಿಗೆ ಸಿಕ್ಕಿ ಕುಂತಿಯೂ ಪಂಚಪಾಂಡವರೂ ಉರಿದು ಹೋದರೆಂದು ಹಸ್ತಿನಾವತಿಯಿಂದ ಸುದ್ದಿ ಬಂದಿತು. ಅವರ ಕರ್ಮಾಂತರಗಳನ್ನು ಮಾಡುವುದಕ್ಕಾಗಿ ಶ್ರೀಕೃಷ್ಣನು ಬಲರಾಮನೊಡನೆ ಹಸ್ತಿನಾವತಿಗೆ ಹೊರಟುಹೋದನು. ಇಂತಹ ಸುಸಮಯವನ್ನೇ ಕಾಯುತ್ತಿದ್ದ ಶತಧನ್ವನು ಸತ್ರಾಜಿತನ ಅಂತಃಪುರಕ್ಕೆ ನುಗ್ಗಿ, ನಿದ್ರಿಸು ತ್ತಿದ್ದ ಸತ್ರಾಜಿತನನ್ನು ಅಲ್ಲಿದ್ದ ಹೆಣ್ಣುಗಳ ಇದಿರಿನಲ್ಲಿಯೇ ಕಟುಕನು ದನವನ್ನು ಕತ್ತರಿಸುವಂತೆ ಕತ್ತರಿಸಿ ಹಾಕಿದನು. ಇಷ್ಟೇ ಅಲ್ಲ, ಸತ್ತವನ ಹೆಂಡತಿ ಮಕ್ಕಳು ಗೊಳೋ ಎಂದು ಅಳುತ್ತಾ ತಡೆದರೂ ಲೆಕ್ಕಿಸದೆ ಸತ್ತವನಲ್ಲಿದ್ದ ಸ್ಯಮಂತಕ ಮಣಿಯನ್ನೂ ಕದ್ದೊ ಯ್ದನು. ತನ್ನ ತಂದೆಯ ಕೊಲೆಯನ್ನು ಕಂಡು ಸತ್ಯಭಾಮೆ ಕಲ್ಲು ಕರಗುವಂತೆ ಹಲುಬಿ ಹಂಬಲಿಸಿ ಅತ್ತಳು. ಅನಂತರ ತಂದೆಯ ದೇಹವನ್ನು ಎಣ್ಣೆಯ ಕೊಪ್ಪರಿಗೆಯಲ್ಲಿರಿಸಿ, ಹಸ್ತಿನಾವತಿಗೆ ಹೋದವಳೆ ತನ್ನ ಗಂಡನ ಮುಂದೆ ಅಳುತ್ತಾ ನಡೆದ ಸಂಗತಿಯನ್ನೆಲ್ಲ ತಿಳಿಸಿದಳು. ಮಡದಿಯ ಕಣ್ಣೀರನ್ನು ಕಂಡು ಶ್ರೀಕೃಷ್ಣನಿಗೆ ಕೆಂಡದಂತಹ ಕೋಪ ಬಂತು. ಒಡನೆಯೆ ಆತನು ಅಣ್ಣನೊಡನೆ ಸತ್ಯಭಾಮೆಯನ್ನೂ ಕರೆದುಕೊಂಡು ದ್ವಾರಕಿಗೆ ಧಾವಿಸಿದನು.

ಶ್ರೀಕೃಷ್ಣನು ಊರಿಗೆ ಹಿಂದಿರುಗುತ್ತಲೆ ತನ್ನನ್ನು ಬಲಿಹಾಕುವನೆಂದು ಶತಧನ್ವನಿಗೆ ಗೊತ್ತು. ಆದ್ದರಿಂದ ಆತ ತನ್ನ ಗೆಳೆಯರ ಬಳಿಗೆ ಓಡಿಹೋಗಿ ಅವರ ಸಹಾಯವನ್ನು ಬೇಡಿದನು. ಆದರೆ ಅವರಿಬ್ಬರೂ ಒಂದೇ ಹಾಡು ಹಾಡಿದರು–‘ಶ್ರೀಕೃಷ್ಣನಲ್ಲಿ ದ್ವೇಷ ಕಟ್ಟಿಕೊಂಡು ಬದುಕುವುದು ಸಾಧ್ಯವೆ?’ ಎಂದು. ಅಕ್ರೂರನಂತೂ ಶ್ರೀಕೃಷ್ಣನ ಹೆಸರನ್ನು ಕೇಳುತ್ತಲೆ ‘ಮಹಾಮಹಿಮನಾದ ಶ್ರೀಕೃಷ್ಣನಿಗೆ ನಮಸ್ಕಾರ!’ ಎಂದು ಕಣ್ಣುಗಳನ್ನೆ ಮುಚ್ಚಿ ಕೊಂಡ. ಇಂತಹ ಗೆಳೆಯರನ್ನು ಕಟ್ಟಿ ಕೊಂಡು, ಪಾಪ! ಶತಧನ್ವ ಏನುಮಾಡಬೇಕು? ಆತ ಬೇರೆ ದಾರಿಯೇನೂ ತೋಚದೆ ತನ್ನಲ್ಲಿದ್ದ ಸ್ಯಮಂತಕ ಮಣಿಯನ್ನು ಅಕ್ರೂರನ ಕೈಗೆ ಕೊಟ್ಟು, ದ್ವಾರಕಿಯಿಂದ ಓಡಿಹೋದನು. ಆದರೆ, ಎಲ್ಲಿ ಹೊಕ್ಕರೆ ತಾನೇ ಅವನು ಶ್ರೀಕೃಷ್ಣನಿಂದ ತಪ್ಪಿಸಿಕೊಳ್ಳುವುದಕ್ಕೆ ಸಾಧ್ಯ? ಅವನು ಓಡಿಹೋದನೆಂದು ಕೇಳುತ್ತಲೆ ಬಲರಾಮ ಕೃಷ್ಣರಿಬ್ಬರೂ ರಥದಲ್ಲಿ ಕುಳಿತು ಅವನನ್ನು ಬೆನ್ನಟ್ಟಿದರು. ಪಲಾಯನ ಮಾಡು ತ್ತಿದ್ದ ಶತಧನ್ವನು ಮಿಥಿಲಾನಗರಿಯನ್ನು ಮುಟ್ಟುವ ವೇಳೆಗೆ ಅವನ ರಥದ ಕುದುರೆಗಳು ಸೋತು ನೆಲಕ್ಕೆ ಬಿದ್ದವು. ಶತಧನ್ವನು ಕಾಲು ನಡಿಗೆಯಿಂದಲೆ ಓಡಲು ಮೊದಲು ಮಾಡಿ ದನು. ಶ್ರೀಕೃಷ್ಣನು ರಥದಿಂದ ಧುಮ್ಮಿಕ್ಕಿ ಅವನನ್ನು ಬೆಂಬತ್ತಿದನು. ಕ್ಷಣಮಾತ್ರದಲ್ಲಿ ಶ್ರೀಕೃಷ್ಣ ಅವನನ್ನು ಹಿಡಿದು, ತನ್ನ ಚಕ್ರದಿಂದ ಅವನ ತಲೆಯನ್ನು ಕತ್ತರಿಸಿಹಾಕಿದನು. ಆ ವೇಳೆಗೆ ಬಲರಾಮನೂ ಅಲ್ಲಿಗೆ ಬಂದನು. ಅಣ್ಣನ ಅಪ್ಪಣೆಯಂತೆ ಶ್ರೀಕೃಷ್ಣನು ಸ್ಯಮಂತಕಮಣಿಗಾಗಿ ಶತಧನ್ವನ ಬಟ್ಟೆಗಳನ್ನೆಲ್ಲ ಶೋಧಿಸಿ ನೋಡಿದನು. ಅದು ಅವನಲ್ಲಿ ಇದ್ದರಲ್ಲವೆ ಸಿಕ್ಕುವುದು? ಅವನಲ್ಲಿ ರತ್ನವಿಲ್ಲದುದನ್ನು ಕಂಡು ಬಲರಾಮನು ‘ಕೃಷ್ಣ, ಈ ನೀಚ ಆ ರತ್ನವನ್ನು ಮತ್ತಾರ ಬಳಿಯಲ್ಲಿಯೋ ಇಟ್ಟಿರಬೇಕು. ನೀನು ದ್ವಾರಕಿಗೆ ಹಿಂದಿ ರುಗಿ ಅದನ್ನು ಪತ್ತೆಹಚ್ಚು. ನಾನು ಮಿಥಿಲೆಗೆ ಹೋಗಿ ಮಹಾರಾಜನನ್ನು ಕಂಡುಬರುತ್ತೇನೆ’ ಎಂದು ಆತನನ್ನು ದ್ವಾರಕಿಗೆ ಕಳುಹಿಸಿ, ತಾನು ಮಿಥಿಲಾನಗರಿಗೆ ಹೋದನು. ಅಲ್ಲಿ ಮಿಥಿಲಾರಾಜನಿಂದ ಸತ್ಕೃತನಾಗಿ ಆತ ಅಲ್ಲಿಯೇ ಕೆಲವು ಕಾಲ ನಿಂತನು. ಆ ಸಮಯದಲ್ಲಿ ಹಸ್ತಿನಾವತಿಯ ರಾಜಕುಮಾರನಾದ ದುರ್ಯೋಧನನು ಬಲರಾಮನ ಶಿಷ್ಯನಾಗಿ, ಆತನಿಂದ ಗದಾಯುದ್ಧವನ್ನು ಕ್ರಮವಾಗಿ ಕಲಿತುಕೊಂಡನು.

ಇತ್ತ ಶ್ರೀಕೃಷ್ಣನು ದ್ವಾರಕಿಗೆ ಹಿಂದಿರುಗಿ, ತಾನು ಶತಧನ್ವನನ್ನು ಕೊಂದುದನ್ನೂ ಸ್ಯಮಂತಕಮಣಿ ಅವನ ಬಳಿ ಸಿಕ್ಕದೆ ಹೋದುದನ್ನೂ ತನ್ನ ಮಡದಿಯಾದ ಸತ್ಯಭಾಮೆಗೆ ತಿಳಿಸಿ, ಆಕೆಯನ್ನು ಸಮಾಧಾನ ಪಡಿಸಿದನು. ಈ ಸುದ್ದಿಯನ್ನು ಕೇಳುತ್ತಲೆ ಅಕ್ರೂರ ಕೃತವರ್ಮರಿಗೆ ದಿಗಿಲು ಹತ್ತಿತು. ಸತ್ರಾಜಿತನನ್ನು ಕೊಲೆಮಾಡುವುದಕ್ಕೆ ತಾವು ಶತಧನ್ವನಿಗೆ ಪ್ರೋತ್ಸಾಹವಿತ್ತುದು ಶ್ರೀಕೃಷ್ಣನಿಗೆ ಗೊತ್ತಾದರೆ ತಮ್ಮ ಗತಿಯೇನಾಗುತ್ತದೊ ಎಂಬ ಭಯದಿಂದ ಅವರಿಬ್ಬರೂ ದ್ವಾರಕಿಯಿಂದ ಓಡಿಹೋಗಿ ತಲೆಮರೆಸಿಕೊಂಡರು. ಇದಾದ ಕೆಲವು ದಿನಗಳೊಳಗಾಗಿ ದ್ವಾರಕಿಯಲ್ಲಿ ಹಲವಾರು ಉತ್ಪಾತಗಳು ಕಾಣಿಸಿಕೊಂಡವು. ಜನರ ದೇಹಕ್ಕೆ ಸುಖವಿಲ್ಲ, ಮನಸ್ಸಿಗೆ ನೆಮ್ಮದಿಯಿಲ್ಲ. ಕೆಲವು ಜನರು ‘ಸಜ್ಜನನಾದ ಅಕ್ರೂರ ಊರು ಬಿಟ್ಟುಹೋದುದರಿಂದ ಈ ಕೇಡು ಬಡಿದುಕೊಂಡಿದೆ. ಇವರಪ್ಪನಾದ ಶ್ವಫಲ್ಕ ಇದ್ದ ಕಡೆ ಮಳೆಯಾಗುತ್ತಿತ್ತಂತೆ! ಇವನೂ ಅಷ್ಟೆ. ಇವನಿದ್ದ ಕಡೆ ಮಳೆ ಬೆಳೆ ಗಳಾಗುತ್ತವಂತೆ, ಯಾವ ಉಪದ್ರವಗಳೂ ಇರುವುದಿಲ್ಲವಂತೆ!’ ಎಂದು ಗೊಣಗಿಕೊಳ್ಳು ತ್ತಿದ್ದರು. ಇದು ಕಿವಿಯಿಂದ ಕಿವಿಗೆ ಹರಡಿ ಶ್ರೀಕೃಷ್ಣನನ್ನು ಮುಟ್ಟಿತು. ಆತ ಜನರ ಮೂಢತನಕ್ಕಾಗಿ ಮನಸ್ಸಿನಲ್ಲೆ ನಕ್ಕ. ಅಕ್ರೂರನ ಮಹತ್ತೆಲ್ಲ ಅವನಲ್ಲಿದ್ದ ಸ್ಯಮಂತಕ ರತ್ನದ ಪ್ರಭಾವವೆಂದು ಆತನಿಗೆ ಗೊತ್ತು. ಆದ್ದರಿಂದ ಆತನು ದೂತರನ್ನು ಅವನಿದ್ದಲ್ಲಿಗೆ ಅಟ್ಟಿ ದ್ವಾರಕಿಗೆ ಕರೆಸಿದನು. ಅವನು ಬರುತ್ತಲೆ ಶ್ರೀಕೃಷ್ಣನು ಅವನ ಕುಶಲವನ್ನು ವಿಚಾರಿಸಿ, ಸರಸವಾದ ನುಡಿಗಳಿಂದ ಅವನನ್ನು ಸಂತೋಷಪಡಿಸುತ್ತಾ ‘ಅಯ್ಯಾ ಅಕ್ರೂರ, ಶತಧನ್ವ ಸತ್ತುಹೋದ. ಅವನ ಆಸ್ತಿಯೆಲ್ಲ ಅವನ ಬಂಧುಗಳಿಗೆ ಸೇರಿಹೋಯಿತು. ಅವನು ನಿನಗೆ ಕೊಟ್ಟುಹೋದ ಸ್ಯಮಂತಕಮಣಿ ಮಾತ್ರ ನಿನ್ನಲ್ಲಿಯೇ ಇರಲಿ. ಅದನ್ನು ಧರಿಸುವ ಯೋಗ್ಯತೆ ಯಾರಿಗೂ ಇಲ್ಲ. ಸದಾಚಾರ ಸಂಪನ್ನನಾದ ನೀನು ಮಾತ್ರ ಅದನ್ನು ಧರಿಸಬಲ್ಲೆ. ನಾನು ಈಗ ಆ ಮಣಿಯ ಸುದ್ದಿಯನ್ನು ಎತ್ತಿದುದಕ್ಕೆ ಕಾರಣ, ನಮ್ಮ ಅಣ್ಣ ನಾದ ಬಲರಾಮ ಅದು ನನ್ನಲ್ಲಿದೆಯೆಂದು ಸಂದೇಹಪಡುತ್ತಿದ್ದಾನೆ. ನೀನು ಒಮ್ಮೆ ಅದನ್ನು ಎಲ್ಲರೆದುರಿಗೆ ತೋರಿಸಿ, ನನಗೆ ಬಂದಿರುವ ಈ ಅಪಕೀರ್ತಿಯನ್ನು ಹೋಗ ಲಾಡಿಸು’ ಎಂದನು. ಒಡನೆಯೆ ಅಕ್ರೂರನು ತನ್ನ ಹೊದಿಕೆಯ ಸೆರಗಿನಲ್ಲಿ ಕಟ್ಟಿಕೊಂಡಿದ್ದ ಆ ರತ್ನವನ್ನು ಹೊರಕ್ಕೆ ತೆಗೆದು ಎಲ್ಲರಿಗೂ ತೋರಿಸಿದನು. ಅಲ್ಲಿಂದ ಮುಂದೆ ಅದು ಅಕ್ರೂರನಿಗೆ ಸೇರಿಹೋಯಿತು. ಆತನು ಅದರಿಂದ ಬಂದ ಧನದಿಂದ ಯಜ್ಞಯಾಗಾದಿ ಗಳನ್ನು ಮಾಡಿಕೊಂಡು ಸುಖವಾಗಿದ್ದನು.

ಈ ಸ್ಯಮಂತಕ ಮಣಿಯ ಕಥೆ ಬಹು ಮಂಗಳಕರವಾದುದು. ಇದನ್ನು ಭಕ್ತಿಯಿಂದ ಹೇಳುವವರೂ ಕೇಳುವವರೂ ತಮ್ಮ ಮೇಲಿನ ಅಪರಾಧಗಳನ್ನು ಕಳೆದುಕೊಂಡು ಕೀರ್ತಿಶಾಲಿಗಳಾಗುವರು.

