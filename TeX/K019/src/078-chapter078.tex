
\chapter{೭೮. ಜರಾಸಂಧನನ್ನು ಕೊಲ್ಲಿಸಿದ}

ಸಾಕ್ಷಾತ್ ನಾರಾಯಣನೆ ತಾನಾದರೂ ಶ್ರೀಕೃಷ್ಣನು ಲೋಕಕ್ಕೆ ಮಾದರಿಯಾಗುವಂತೆ ತನ್ನ ಬಾಳನ್ನು ಅಳವಡಿಸಿಕೊಂಡಿದ್ದನು. ಆತನು ಬೆಳಗಿನ ಜಾವ ಮೊದಲ ಕೋಳಿ ಕೂಗು ತ್ತಲೆ ಹಾಸಿಗೆಯಿಂದೆದ್ದು, ಮುಖ ತೊಳೆದು, ತನ್ನ ಸಚ್ಚಿದಾನಂದಮೂರ್ತಿಯನ್ನು ತಾನೆ ಧ್ಯಾನ ಮಾಡುವನು. ಸೂರ್ಯೋದಯಕ್ಕೆ ಮುಂಚೆಯೇ ಸ್ನಾನಮಾಡಿ, ಸಂಧ್ಯಾವಂದನೆ ಮೊದಲಾದ ನಿತ್ಯಕರ್ಮಗಳನ್ನು ಶ್ರದ್ಧೆಯಿಂದ ಆಚರಿಸುವನು. ಅನಂತರ ಉತ್ತಮರಾದ ಬ್ರಾಹ್ಮಣರಿಗೆ ದಾನಧರ್ಮಗಳನ್ನಿತ್ತು, ವಸ್ತ್ರಭೂಷಣಗಳಿಂದ ಅಲಂಕೃತನಾಗಿ ತನ್ನ ಸಭಾ ಮಂಟಪಕ್ಕೆ ಹೋಗಿ, ರಾಜಕಾರ್ಯಗಳಲ್ಲಿ ಮಗ್ನನಾಗುವನು. ಒಂದು ದಿನ ಎಂದಿನಂತೆ ಆತನು ಸುಧರ್ಮೆಯೆಂಬ ಸಭಾಮಂಟಪದಲ್ಲಿ ನಕ್ಷತ್ರಗಳ ಮಧ್ಯದ ಚಂದ್ರನಂತೆ ಯಾದವವೀರರ ಮಧ್ಯದಲ್ಲಿ ಕುಳಿತಿರಲು, ದ್ವಾರಪಾಲಕರು ಯಾರೋ ಒಬ್ಬ ಮಹಾ ಪುರುಷನು ಶ್ರೀಕೃಷ್ಣದರ್ಶನಕ್ಕಾಗಿ ಬಂದು ಬಾಗಿಲಲ್ಲಿ ನಿಂತಿರುವುದಾಗಿ ತಿಳಿಸಿದರು. ಶ್ರೀಕೃಷ್ಣನು ಆತನನ್ನು ಸಭೆಗೆ ಬರುವಂತೆ ಕರೆಸಿದನು. ಕೈಕಟ್ಟಿಕೊಂಡು ಭಯಭಕ್ತಿಯಿಂದ ಸಭೆಯನ್ನು ಪ್ರವೇಶಿಸಿದ ಆ ಮನುಷ್ಯ ಶ್ರೀಕೃಷ್ಣನಿಗೆ ಅಡ್ಡಬಿದ್ದು, ಭಕ್ತಿಯಿಂದ ಕೈ ಮುಗಿದುಕೊಂಡು “ಭಕ್ತವತ್ಸಲನಾದ ಶ್ರೀಕೃಷ್ಣ ಪರಮಾತ್ಮ, ಜರಾಸಂಧನೆಂಬ ದುಷ್ಟ ರಾಜನು ನಿನ್ನ ಭಕ್ತರಾದ ಇಪ್ಪತ್ತು ಸಾವಿರ ರಾಜರನ್ನು ಗಿರಿವ್ರಜವೆಂಬ ಕೋಟೆಯಲ್ಲಿ ಸೆರೆ ಹಾಕಿದ್ದಾನೆ. ಹತ್ತುಸಾವಿರ ಆನೆಗಳ ಬಲವುಳ್ಳ ಜರಾಸಂಧನ ದೆಸೆಯಿಂದ ಆ ರಾಜರೆಲ್ಲ ಹುಲಿಯ ಕೈಗೆ ಸಿಕ್ಕ ಹುಲ್ಲೆಗಳಂತೆ ಪರಿತಪಿಸುತ್ತಿದ್ದಾರೆ. ದುಷ್ಟಶಿಕ್ಷಣ ಶಿಷ್ಟರಕ್ಷಣೆಗಾಗಿ ಅವತರಿಸಿರುವ ನಿನ್ನ ಬಳಿಗೆ ಅವರು ನನ್ನನ್ನು ಕಳಿಸಿದ್ದಾರೆ. ನಿನ್ನಲ್ಲಿ ಶರಣಾಗತರಾಗಿರುವ ಅವರನ್ನು ರಕ್ಷಿಸುವ ಹೊಣೆ ನಿನ್ನದು. ನಿನ್ನ ಪಾದದರ್ಶನವನ್ನೆ ಸದಾ ಬಯಸುತ್ತಿರುವ ಆ ರಾಜರಿಗೆ ನಿನ್ನ ದರ್ಶನವಿತ್ತು ಕಾಪಾಡು” ಎಂದು ಬೇಡಿದನು.

ದೂತನು ತನ್ನ ಮಾತನ್ನು ಮುಗಿಸುತ್ತಿದ್ದಂತೆಯೆ ನಾರದ ಮಹರ್ಷಿಗಳು ಆಸ್ಥಾನವನ್ನು ಪ್ರವೇಶಿಸಿದರು. ಶ್ರೀಕೃಷ್ಣನೂ ಸಭೆಯಲ್ಲಿದ್ದವರೂ ತಟ್ಟನೆ ಮೇಲಕ್ಕೆದ್ದು ತಲೆಬಾಗಿ ಅವ ರಿಗೆ ನಮಸ್ಕರಿಸಿದರು. ಶ್ರೀಕೃಷ್ಣನು ಅವರನ್ನು ಪೀಠದ ಮೇಲೆ ಕುಳ್ಳಿರಿಸಿ, ಉಪಚರಿ ಸಿದಮೇಲೆ ‘ತ್ರಿಲೋಕಸಂಚಾರಿಯಾದ ಮುನೀಂದ್ರ, ವಿಶೇಷ ಸಮಾಚಾರವೇನು? ಈಚೆಗೆ ನಮ್ಮ ಪಾಂಡವರನ್ನೇನಾದರೂ ಕಂಡಿದ್ದೀರಾ? ಅವರು ಆರೋಗ್ಯವೆ?’ ಎಂದು ಕೇಳಿದ. ನಾರದರು ‘ಅಯ್ಯಾ, ಬ್ರಹ್ಮನಿಗೂ ಮರುಳು ಹಿಡಿಸುವ ಮಾಯೆ ನಿನ್ನದು! ನಿನಗೆ ನಮಸ್ಕಾರ. ನಾನು ಹೇಳಬೇಕೆಂದು ಬಂದಿರುವುದನ್ನೇ ನೀನು ಕೇಳುತ್ತಿದ್ದಿ. ನಾನೂ ಹೇಳು ತ್ತೇನೆ. ನಿನ್ನ ಪಾಂಡವರಲ್ಲಿ ಹಿರಿಯನಾಗಿ ನಿನ್ನ ಭಕ್ತನಾಗಿರುವ ಧರ್ಮರಾಯನು ರಾಜ ಸೂಯಯಾಗವನ್ನು ಮಾಡಿ, ರಾಜರಲ್ಲೆಲ್ಲ ಶ್ರೇಷ್ಠನಾಗಬೇಕೆಂದು ಬಯಸಿದ್ದಾನೆ. ಅನೇಕ ರಾಜರು ಆ ಯಾಗಕ್ಕೆ ಬರುವವರಾಗಿದ್ದಾರೆ. ನೀನು ಬಂದು, ಆ ಯಾಗವು ಸಾಂಗವಾಗಿ ನೆರವೇರುವಂತೆ ಅನುಗ್ರಹಿಸಿ, ಆತನನ್ನು ಕೃತಾರ್ಥನನ್ನಾಗಿ ಮಾಡಬೇಕು. ಪಾವನವಾದ ನಿನ್ನ ದರ್ಶನದಿಂದ ಯಾಗಕ್ಕೆ ಬಂದವರೆಲ್ಲ ಉದ್ಧಾರವಾಗಲಿ’ ಎಂದನು.

ಈಗ ಎರಡು ಆಹ್ವಾನಗಳು ಏಕಕಾಲದಲ್ಲಿ ಬಂದಂತಾಯಿತು: ಒಂದು ಜರಾಸಂಧನ ವಧೆಗೆ, ಮತ್ತೊಂದು ಧರ್ಮರಾಯನ ರಾಜಸೂಯಯಾಗಕ್ಕೆ. ಯಾದವವೀರರಲ್ಲಿ ಬಹು ಮಂದಿ ಜರಾಸಂಧನ ವಧೆಯೆ ಮುಖ್ಯಕಾರ್ಯವೆಂದು ಭಾವಿಸಿದ್ದರು. ಆದರೆ ಶ್ರೀಕೃಷ್ಣನ ಅಭಿಪ್ರಾಯವೆ ಬೇರೆ ಯಾಗಿತ್ತು. ಆದ್ದರಿಂದ ಶ್ರೀಕೃಷ್ಣನು ತನ್ನ ಮಂತ್ರಿಯಾದ ಉದ್ಧವ ನನ್ನು ಕುರಿತು ‘ನಿನ್ನ ಸಲಹೆಯೇನು?’ ಎಂದು ಕೇಳಿದನು. ಆತನು ಶ್ರೀಕೃಷ್ಣನ ಮಂತ್ರಿ ಮಾತ್ರವೇ ಅಲ್ಲ, ಅಂತರಂಗದ ಭಕ್ತ; ಬೂದಿ ಮುಚ್ಚಿದ ಕೆಂಡದಂತಿರುವ ಆ ಮಾಯಾ ಮಯನ ಮನಸ್ಸಿನಲ್ಲಿರುವುದನ್ನೆ ಆತ ಹೇಳಿದ: “ಪ್ರಭು, ಈಗ ಒದಗಿರುವ ಎರಡು ಕಾರ್ಯ ಗಳೂ ಮುಖ್ಯವೆ; ಯಾವುದನ್ನೂ ಉಪೇಕ್ಷೆ ಮಾಡುವಂತಿಲ್ಲ. ಆದರೆ ಆ ರಾಜಸೂಯ ಯಾಗಕ್ಕೆ ಹೋಗುವುದರಿಂದ ಎರಡು ಕೆಲಸಗಳೂ ಏಕಕಾಲದಲ್ಲಿ ನಡೆದರೂ ನಡೆಯ ಬಹುದು. ರಾಜಸೂಯಯಾಗವನ್ನು ಮಾಡಬೇಕಾದರೆ ಜಗತ್ತಿನ ರಾಜರನ್ನೆಲ್ಲ ಗೆಲ್ಲಬೇಕು; ಇದೇ ನೆಪದಿಂದ ಜರಾಸಂಧನನ್ನು ಕೊಲ್ಲುವ ಕೆಲಸವೂ ನಡೆಯಬಹುದು. ಅವನನ್ನು ಕೊಲ್ಲುವುದೆಂದರೆ ಹುಡುಗಾಟವೇನೂ ಅಲ್ಲ. ಹತ್ತುಸಾವಿರ ಆನೆಯ ಬಲವುಳ್ಳ ಅವ ನೊಡನೆ ಹೋರಾಡುವುದೆಂದರೆ ಒಬ್ಬ ಭೀಮನಿಗೆ ಮಾತ್ರ ಸಾಧ್ಯ. ಅವನನ್ನು ಕೊಲ್ಲುವು ದಕ್ಕೆ ಶಕ್ತಿ ಮಾತ್ರವೇ ಅಲ್ಲ, ಯುಕ್ತಿಯೂ ಬೇಕು. ದ್ವಂದ್ವಯುದ್ಧವಾಗದ ಹೊರತು ಅವ ನಿಗೆ ಸಾವಿಲ್ಲ. ಹಾಗೆ ದ್ವಂದ್ವಯುದ್ಧಕ್ಕೆ ಎಳೆಯಬೇಕಾದರೆ ಬ್ರಾಹ್ಮಣವೇಷದಿಂದ ಅವನ ಬಳಿಗೆ ಹೋಗಿ, ಯುದ್ಧಭಿಕ್ಷೆಯನ್ನು ಬೇಡಬೇಕು. ಬ್ರಾಹ್ಮಣರು ಕೇಳಿದುದನ್ನು ಅವನು ‘ಇಲ್ಲ’ ಎನ್ನುವುದಿಲ್ಲ. ಆದ್ದರಿಂದ ಬ್ರಾಹ್ಮಣವೇಷದ ಭೀಮನಿಂದ ಅವನನ್ನು ಕೊಲ್ಲಿಸ ಬೇಕು” ಎಂದನು. ಈ ಮಾತನ್ನು ಅಲ್ಲಿದ್ದ ನಾರದರು, ಯಾದವರು–ಎಲ್ಲರೂ ಒಪ್ಪಿ ದರು. ಶ್ರೀಕೃಷ್ಣನು ರಾಜಸೂಯಯಾಗಕ್ಕೆ ಹೊರಡುವುದು ನಿಶ್ಚಯವಾಗುತ್ತಲೆ ನಾರದರು ಆತನಿಂದ ಬೀಳ್ಕೊಂಡು ತಮ್ಮ ಸಂಚಾರಕ್ಕೆ ಹೊರಟರು. ಶ್ರೀಕೃಷ್ಣನು ತನ್ನಲ್ಲಿಗೆ ಬಂದಿದ್ದ ರಾಜದೂತನನ್ನು ಕುರಿತು ‘ಅಯ್ಯಾ, ನೀನಿನ್ನು ಹಿಂದಕ್ಕೆ ಹೊರಡು. ಇಷ್ಟರಲ್ಲೆ ನಾನಲ್ಲಿಗೆ ಬಂದು ಜರಾಸಂಧನನ್ನು ಕೊಲ್ಲುತ್ತೇನೆ. ಇದನ್ನು ರಾಜರಿಗೆ ತಿಳಿಸು’ ಎಂದು ಹೇಳಿ ಅವನನ್ನು ಬೀಳ್ಕೊಟ್ಟನು.

ಶ್ರೀಕೃಷ್ಣನು ತಕ್ಷಣವೇ ಇಂದ್ರಪ್ರಸ್ಥಕ್ಕೆ ಹೊರಡಲು ನಿಶ್ಚಯಿಸಿದುದರಿಂದ, ದಾರುಕನು ಆತನ ರಥವನ್ನು ತಂದು ನಿಲ್ಲಿಸಿದನು. ರುಕ್ಮಿಣಿಯೇ ಮೊದಲಾದ ರಾಣಿವಾಸದವರು ರಾಜಸೂಯ ಯಾಗವನ್ನು ನೋಡಬೇಕೆಂಬ ಕುತೂಹಲದಿಂದ ವಸ್ತ್ರಾಲಂಕಾರಭೂಷಿತ ರಾಗಿ ಹೊರಟು ನಿಂತರು. ಅವರಿಗಾಗಿ ಚಿನ್ನದ ಪಲ್ಲಕ್ಕಿಗಳೂ ರಥಗಳೂ ಪ್ರಯಾಣಸನ್ನದ್ಧ ವಾದುವು. ಅವರೆಲ್ಲರ ರಕ್ಷಣೆಗಾಗಿ ಸೇನೆಯೂ, ಸೇವೆಗಾಗಿ ದಾಸದಾಸಿಯರೂ ಹೊರಡ ಬೇಕಾಯಿತು. ಅವರೆಲ್ಲರ ಹಾಸಿಗೆ ಹೊದ್ದಿಕೆ ಗುಡಾರಗಳಿಗಾಗಿ ಅನೇಕ ವಾಹನಗಳು ಹೊರಟುನಿಂತವು. ಆ ದೊಡ್ಡ ಪರಿವಾರದೊಡನೆ ಪ್ರಯಾಣ ಹೊರಟ ಆ ವಿಶ್ವ ಕುಟುಂಬಿಯು ಅನೇಕ ನದಿ ಪರ್ವತಗಳನ್ನೂ ಹಳ್ಳಿ ಕಾಡು ಪಟ್ಟಣಗಳನ್ನೂ ದಾಟಿ ಇಂದ್ರ ಪ್ರಸ್ಥಪುರದ ಬಳಿಗೆ ಬಂದನು. ಆ ಸುದ್ದಿಯನ್ನು ಕೇಳಿದ ಧರ್ಮರಾಯನು ಸೋದರ ರನ್ನೂ, ಪುರೋಹಿತರನ್ನೂ, ಇಷ್ಟಮಿತ್ರರನ್ನೂ ಕರೆದುಕೊಂಡುಹೋಗಿ ಆತನನ್ನು ಇದಿರು ಗೊಂಡನು. ಪಂಚೇಂದ್ರಿಯಗಳು ಪ್ರಾಣದ ಬಳಿಗೆ ಹೋಗುವಂತೆ ಆ ಪಾಂಡವರೈವರೂ ಶ್ರೀಕೃಷ್ಣನ ಬಳಿಗೆ ಹೋಗಿ ಆತನನ್ನು ಆಲಿಂಗಿಸಿಕೊಂಡರು. ಆಗ ಅವರ ಮೈ ಪುಲಕಿತ ವಾಯಿತು, ಆನಂದಬಾಷ್ಪಗಳು ಸುರಿದವು, ಮನಸ್ಸು ಧನ್ಯತೆಯನ್ನು ಅನುಭವಿಸಿತು. ಶ್ರೀಕೃಷ್ಣನು ಅವರ ಕುಶಲವನ್ನು ವಿಚಾರಿಸಿ, ಅಲ್ಲಿದ್ದ ಪುರೋಹಿತರಿಗೂ ಹಿರಿಯರಿಗೂ ನಮಸ್ಕರಿಸಿದನು. ಪರಿಚಯದವರೆಲ್ಲರ ಕುಶಲವನ್ನು ವಿಚಾರಿಸಿದನು. ಅನಂತರ ವಂದಿ ಮಾಗಧರು ಶ್ರೀಕೃಷ್ಣನ ಗುಣಗಳನ್ನು ಹೊಗಳಿ ಪರಾಕು ಮಾಡಿದರು; ಮಂಗಳವಾದ್ಯಗಳು ಭೋರ್ಗರೆದವು; ವೇಶ್ಯೆಯರು ನರ್ತಿಸಿದರು; ಅತ್ಯಂತ ವೈಭವದಿಂದ ದಿಬ್ಬಣ ಹೊರಟು ಸಮಸ್ತ ಪರಿವಾರದೊಡನೆ ಶ್ರೀಕೃಷ್ಣನು ಇಂದ್ರಪ್ರಸ್ಥವನ್ನು ಪ್ರವೇಶಿಸಿದನು. ಇಕ್ಕಡೆಯ ಉಪ್ಪರಿಗೆಗಳಿಂದಲೂ ಪುರಸ್ತ್ರೀಯರು ಹೂಮಳೆಯನ್ನು ಕರೆಯುತ್ತಿರಲು ಶ್ರೀಕೃಷ್ಣನು ರಾಜಬೀದಿಯಲ್ಲಿ ಬಿಜಯಮಾಡಿ, ಅಲ್ಲಲ್ಲಿಯೆ ಪುರಜನರು ನೀಡುತ್ತಿದ್ದ ಕಾಣಿಕೆಗಳನ್ನು ಸ್ವೀಕರಿಸುತ್ತಾ ಅರಮನೆಯನ್ನು ಸೇರಿದನು. ಮನೆಯೊಳಗೆ ಕಾಲಿಡುತ್ತಿದ್ದಂತೆಯೇ ಕುಂತೀ ದೇವಿ ಕಾಣಿಸಿಕೊಂಡಳು. ಶ್ರೀಕೃಷ್ಣನು ಆಕೆಗೆ ಕಾಲ್ಮುಟ್ಟಿ ನಮಸ್ಕರಿಸಿದನು. ದ್ರೌಪದಿಯು ಆತನಿಗೆ ನಮಸ್ಕರಿಸಿ, ರುಕ್ಮಿಣಿಯೇ ಮೊದಲಾದ ಹೆಣ್ಣುಮಕ್ಕಳನ್ನು ಸತ್ಕರಿಸುವ ಕಾರ್ಯದಲ್ಲಿ ಮಗ್ನಳಾದಳು. ಬಂದವರಿಗೆಲ್ಲ ಸುಖವಸತಿಗಳನ್ನು ಕಲ್ಪಿಸಿ, ಅವರ ಷೋಡಶೋಪಚಾರಗಳಿಗೆ ಅಣಿ ಮಾಡಲಾಯಿತು.

ಮರುದಿನ ಸುಖಭೋಜನ ವಿಶ್ರಾಂತಿಗಳಾದ ಮೇಲೆ ಧರ್ಮರಾಯನು ತನ್ನ ತಮ್ಮಂದಿ ರೊಡನೆಯೂ ಬಂಧುಬಾಂಧವರೊಡನೆಯೂ ಸಭಾಭವನದಲ್ಲಿ ಕುಳಿತು ಶ್ರೀಕೃಷ್ಣ ನೊಡನೆ ‘ಪ್ರಭು, ಗೋವಿಂದ, ಯಾಗಗಳಲ್ಲೆಲ್ಲ ಶ್ರೇಷ್ಠವಾದ ರಾಜಸೂಯಯಾಗವನ್ನು ಮಾಡಿ, ದೇವತೆಗಳನ್ನು ಆರಾಧಿಸಬೇಕೆಂದುಕೊಂಡಿದ್ದೇನೆ. ನಿನ್ನ ನೆರವಿಲ್ಲದೆ ಅದು ನೆರ ವೇರದು. ಇದನ್ನು ಆಗಮಾಡಿಸುವ ಹೊಣೆ ನಿನ್ನದು. ನಿನ್ನ ಭಕ್ತನಾದ ನನ್ನನ್ನು ಅನು ಗ್ರಹಿಸಬೇಕು’ ಎಂದು ಬೇಡಿದನು. ಶ್ರೀಕೃಷ್ಣನು ಆತನ ಅಭಿಪ್ರಾಯವನ್ನು ಮೆಚ್ಚಿ ಬಾಯ್ತುಂಬ ಹೊಗಳುತ್ತಾ, ‘ಅಯ್ಯಾ, ಧರ್ಮರಾಜ, ನಿನ್ನ ಉದ್ದೇಶ ಸ್ತೋತ್ರಾರ್ಹ ವಾದುದು. ಇದರಿಂದ ದೇವತೆಗಳು ಸುಪ್ರೀತರಾಗುತ್ತಾರೆ. ನಿನಗೆ ಮಂಗಳವಾಗುತ್ತದೆ. ನನಗೂ ತೃಪ್ತಿಯಾಗುತ್ತದೆ. ಈ ಕ್ಷಣವೇ ನೀನು ದಿಗ್ವಿಜಯಕ್ಕಾಗಿ ನಿನ್ನ ಸೇನೆಯನ್ನು ಕಳು ಹಿಸು’ ಎಂದನು. ಶ್ರೀಕೃಷ್ಣನ ನುಡಿಗಳಿಂದ ಧರ್ಮರಾಯನಿಗೆ ಧೈರ್ಯ ಬಂದಿತು. ಆತನು ತನ್ನ ನಾಲ್ವರು ತಮ್ಮಂದಿರನ್ನೂ ಅಂದೇ ದಿಗ್ವಿಜಯಕ್ಕೆ ಹೊರಡುವಂತೆ ನೇಮಿಸಿದನು. ಸಹದೇವನು ದಕ್ಷಿಣಕ್ಕೂ, ನಕುಲನು ಪಶ್ಚಿಮಕ್ಕೂ, ಅರ್ಜುನನು ಉತ್ತರ ದಿಕ್ಕಿಗೂ, ಭೀಮನು ಪೂರ್ವದಿಕ್ಕಿಗೂ ಹೊರಟರು. ದೇವಾಂಶಸಂಭೂತರಾದ ಅವರ ಕತ್ತಿಗೆ ಇದಿರು ನಿಲ್ಲಬಲ್ಲವರಾರು? ಅವರು ಜಗತ್ತಿನ ರಾಜರನ್ನೆಲ್ಲ ಗೆದ್ದು, ಅಪಾರ ಧನವನ್ನು ತಂದು ಅಣ್ಣನ ಪಾದಕ್ಕೊಪ್ಪಿಸಿದರು. ಅವರಿಗೆ ದುರ್ಜಯನಾಗಿದ್ದವನು ಮಗಧ ದೇಶದ ರಾಜ ನಾದ ಜರಾಸಂಧನೊಬ್ಬನು ಮಾತ್ರ. ಅವನನ್ನು ಗೆಲ್ಲದ ಹೊರತು ರಾಜಸೂಯ ಯಾಗ ವನ್ನು ಪ್ರಾರಂಭಿಸುವಂತಿಲ್ಲ. ಅವನನ್ನು ಗೆಲ್ಲುವ ಉಪಾಯವೇನೆಂದು ಧರ್ಮರಾಯನು ಚಿಂತಿಸುತ್ತಿರಲು, ಶ್ರೀಕೃಷ್ಣನು ‘ಅಯ್ಯಾ, ನೀನೇನೂ ಚಿಂತಿಸಬೇಡ. ನಾನು ಉಪಾಯ ದಿಂದ ಅವನನ್ನು ಕೊಲ್ಲಿಸುತ್ತೇನೆ’ ಎಂದು ಸಮಾಧಾನಮಾಡಿದನು.

ಜರಾಸಂಧನನ್ನು ಕೊಲ್ಲಿಸುವ ಹೊಣೆ ಹೊತ್ತ ಶ್ರೀಕೃಷ್ಣನು ಭೀಮಾರ್ಜುನರೊಡನೆ ಬ್ರಾಹ್ಮಣ ವೇಷವನ್ನು ಧರಿಸಿ ಗಿರಿವ್ರಜಕ್ಕೆ ಹೋದನು. ಅಲ್ಲಿ ಜರಾಸಂಧನು ತನ್ನ ನಿತ್ಯ ಕರ್ಮಗಳನ್ನು ಮುಗಿಸಿ, ಭಕ್ತಿಯಿಂದ ಬ್ರಾಹ್ಮಣರಿಗೆ ದಾನ ನೀಡುತ್ತಿರುವ ಹೊತ್ತಿಗೆ ಸರಿ ಯಾಗಿ ಈ ಮೂವರೂ ಆತನಿಗೆ ಕಾಣಿಸಿಕೊಂಡು ‘ಅಯ್ಯಾ, ನಿನಗೆ ಮಂಗಳವಾಗಲಿ; ನಾವು ಬಹು ದೂರದಿಂದ ಬಂದಿರುವ ಬ್ರಾಹ್ಮಣರು; ನಿನ್ನಲ್ಲಿ ಭಿಕ್ಷೆಯನ್ನು ಬೇಡಲು ಬಂದಿದ್ದೇವೆ. ನಮ್ಮ ಇಷ್ಟಾರ್ಥವನ್ನು ನಡೆಸಿಕೊಡುವುದಾಗಿ ನೀನು ಮಾತು ಕೊಟ್ಟರೆ, ನಾವು ನಮಗೆ ಬೇಕಾದುದನ್ನು ಬೇಡುತ್ತೇವೆ’ ಎಂದರು. ಆತ ಅವರ ಮುಖವನ್ನು ನೋಡಿದ, ಅವರ ಮಾತನ್ನು ಕೇಳಿದ; ಆ ರೂಪ, ಆ ದನಿ ಆತನಿಗೆ ಪರಿಚಿತವಾದುದೆನ್ನಿಸಿತು. ಅಲ್ಲದೆ ಅವರ ಮಣಿಕಟ್ಟಿನಲ್ಲಿ ಬಿಲ್ಲಿನ ಹಗ್ಗ ಒತ್ತಿ ಜಡ್ಡುಗಟ್ಟಿದ ಗುರುತೂ ಕಾಣಿಸುತ್ತಿದೆ. ಇವರು ಬ್ರಾಹ್ಮಣರಲ್ಲ, ಕ್ಷತ್ರಿಯರು ಎಂಬುದು ಆತನಿಗೆ ಮನದಟ್ಟಾಯಿತು. ‘ಇವರು ವೇಷಮಾತ್ರದಿಂದಲೇ ಪೂಜ್ಯರು’ ಎಂದುಕೊಂಡನಾದರೂ, ಅವರು ಬಾಗಿಲಿಗೆ ಬಂದು ಬೇಡುತ್ತಿರುವಾಗ, ಇಲ್ಲವೆನ್ನಬಾರದೆಂದುಕೊಂಡನು. ಆದ್ದರಿಂದಲೆ ಆತನು ‘ಅಯ್ಯಾ, ಬ್ರಾಹ್ಮಣರೆ, ನೀವು ಬೇಕಾದುದನ್ನು ಕೇಳಿರಿ; ನೀವು ನನ್ನ ತಲೆಯನ್ನೆ ಕೇಳಿದರೂ ನಾನು ಸಂತೋಷದಿಂದ ನಿಮಗದನ್ನು ಕೊಡುತ್ತೇನೆ’ ಎಂದನು. ಹಾಗೆ ಅವನು ಮಾತು ಕೊಡು ತ್ತಲೆ ಶ್ರೀಕೃಷ್ಣನು ಇದ್ದ ಸಂಗತಿಯನ್ನು ಇದ್ದಂತೆ ಆತನಿಗೆ ತಿಳಿಸಿದನು–‘ನೋಡಯ್ಯ ಜರಾಸಂಧ, ನಾವು ಬ್ರಾಹ್ಮಣರಲ್ಲ, ಕ್ಷತ್ರಿಯರು. ನಾನು ನಿನಗೆ ಚಿರಪರಿಚಿತನಾದ ಪರಮ ಶತ್ರು ಶ್ರೀಕೃಷ್ಣ; ಇವನು ಕುಂತಿಯ ಮಗನಾದ ಭೀಮ, ಆ ಇನ್ನೊಬ್ಬನು ಅವನ ತಮ್ಮ ಅರ್ಜುನ. ನಾವು ಬಂದುದು ಅನ್ನಭಿಕ್ಷೆಗಾಗಿ ಅಲ್ಲ, ಯುದ್ಧಭಿಕ್ಷೆಗೆ. ನೀನು ನಮ್ಮ ಮೂವರಲ್ಲಿ ಒಬ್ಬನೊಡನೆ ದ್ವಂದ್ವಯುದ್ಧ ಮಾಡು’ ಎಂದನು.

ಶ್ರೀಕೃಷ್ಣನ ಮಾತುಗಳನ್ನು ಕೇಳಿ ಜರಾಸಂಧ ಗಹಗಹಿಸಿ ನಕ್ಕ. ‘ಎಲಾ ಕೃಷ್ಣ, ನೀನು ನನ್ನೊಡನೆ ಯುದ್ಧಮಾಡಲಾರದೆ ಓಡಿಹೋಗಿ ಸಮುದ್ರಮಧ್ಯದಲ್ಲಿ ಅವಿತುಕೊಂಡಿ ರುವೆ. ನಿನ್ನಂತಹ ಜೀವಗಳ್ಳನಾದ ಹೇಡಿಯೊಡನೆ ಯುದ್ಧಮಾಡುವುದು ನನಗೆ ಅವಮಾನ. ಈ ಅರ್ಜುನ ಇನ್ನೂ ಚಿಕ್ಕ ಮಗು, ಇವನೊಡನೆಯೂ ಬೇಡ. ಈ ಭೀಮ ನನಗೆ ಸಮನಾದ ಬಲಶಾಲಿ, ಆತನೊಡನೆ ಯುದ್ಧ ಮಾಡುತ್ತೇನೆ’ ಎಂದು ಹೇಳಿ ಆತ ತನ್ನ ಒಂದು ದೊಡ್ಡ ಗದೆಯನ್ನು ತಾನೆ ಭೀಮನ ಕೈಗೆ ತಂದುಕೊಟ್ಟು, ತಾನು ಮತ್ತೊಂದು ಗದೆಯನ್ನು ತೆಗೆದು ಕೊಂಡನು. ಅವರೆಲ್ಲರೂ ಊರ ಹೊರಗಿನ ಒಂದು ಮೈದಾನಕ್ಕೆ ಹೋದರು. ಅಲ್ಲಿ ಭೀಮ ಜರಾಸಂಧರ ಗದಾಯುದ್ಧ ಆರಂಭವಾಯಿತು. ಇಬ್ಬರೂ ಸೋಲದೆ ಬಹು ಹೊತ್ತು ಹೋರಾಡಿದರು. ಅವರ ಗದೆಗಳ ತಾಕಲಾಟವು ಸಿಡಿಲದನಿಯನ್ನೂ ಮೀರಿಸಿತ್ತು. ಎರಡು ಸಲಗಗಳಂತೆ ಹೋರಾಡುತ್ತಿದ್ದ ಅವರು ಪರಸ್ಪರ ಕೊಟ್ಟ ಪೆಟ್ಟಿನಿಂದ ಅವರ ಅವಯವ ಗಳೆಲ್ಲ ಜಜ್ಜಿ ಹೋಗಿದ್ದವು. ಆದರೂ ಅವರು ಅತ್ತ ಗಮನಿಸದೆ, ಯುದ್ಧೋತ್ಸಾಹದಿಂದ ವಿಜೃಂಭಿಸುತ್ತಿದ್ದರು. ಹೀಗಾದರೆ ಇವರ ಹೋರಾಟ ಮುಗಿಯದೆಂದುಕೊಂಡ ಶ್ರೀಕೃಷ್ಣನು ಭೀಮನಿಗೆ ಕಾಣುವಂತೆ ಹತ್ತಿರದಲ್ಲಿದ್ದ ಒಂದು ಮರದ ರೆಂಬೆಯನ್ನು ಸೀಳಿ, ಜರಾಸಂಧನನ್ನು ಸೀಳಿ ಹಾಕಬೇಕೆಂದು ಸೂಚಿಸಿದನು. ಇಷ್ಟೇ ಅಲ್ಲ, ತನ್ನ ತೇಜೋಬಲ ವನ್ನು ಭೀಮನಲ್ಲಿ ಆರೋಪಿಸಿದನು. ಇದರಿಂದ ಉತ್ಸಾಹಗೊಂಡ ಭೀಮನು ಜರಾಸಂಧ ನನ್ನು ನೆಲಕ್ಕುರುಳಿಸಿ, ಶ್ರೀಕೃಷ್ಣನ ಸೂಚನೆಯಂತೆ, ಅವನ ಒಂದು ಕಾಲನ್ನು ತನ್ನ ಕಾಲಿನಿಂದ ಮೆಟ್ಟಿಕೊಂಡು, ಮತ್ತೊಂದು ಕಾಲನ್ನು ತನ್ನ ಎರಡು ಕೈಗಳಿಂದಲೂ ಬಲವಾಗಿ ಹಿಡಿದು, ಆನೆಯು ಮರದ ಕೊಂಬೆಯನ್ನು ಸೀಳುವಂತೆ ಅವನನ್ನು ಎರಡು ಹೋಳಾಗಿ ಸೀಳಿಹಾಕಿದನು. ಒಡನೆಯೆ ಶ್ರೀಕೃಷ್ಣಾರ್ಜುನರು ಭೀಮನನ್ನು ಆಲಿಂಗಿಸಿಕೊಂಡು ‘ಭೇಷ್, ಭಲೆ’ ಎಂದು ಅವನನ್ನು ಕೊಂಡಾಡಿದರು.

ಜರಾಸಂಧನು ಸಾಯುತ್ತಲೆ ಆತನ ಮಗನಾದ ಸಹದೇವನೆಂಬುವನು ಶ್ರೀಕೃಷ್ಣನಿಗೆ ಶರಣಾಗತನಾದನು. ಇದನ್ನು ಕಂಡು ಭಕ್ತವತ್ಸಲನಾದ ಶ್ರೀಕೃಷ್ಣನು ಅವನನ್ನು ಮಗಧ ಸಿಂಹಾಸನದ ಮೇಲೆ ಕೂಡಿಸಿ, ಆತನಿಗೆ ಪಟ್ಟಾಭಿಷೇಕ ಮಾಡಿದನು. ಜರಾಸಂಧನು ಸೆರೆಯ ಲ್ಲಿಟ್ಟಿದ್ದ ರಾಜರ ಸಂಖ್ಯೆ ಇಪ್ಪತ್ತು ಸಾವಿರದ ಎಂಟನೂರರಷ್ಟಿತ್ತು. ಅವರೆಲ್ಲ ಬಹುಕಾಲ ದಿಂದ ಸರಿಯಾದ ಅನ್ನಬಟ್ಟೆಗಳಿಲ್ಲದೆ ಬಡವಾಗಿ ಬತ್ತಿಹೋಗಿದ್ದರು. ಅವರೆಲ್ಲ ತಮ್ಮನ್ನು ಸೆರೆಯಿಂದ ಬಿಡಿಸಿದ ಶ್ರೀಕೃಷ್ಣನ ಬಳಿಗೆ ಓಡಿ ಬಂದು ಆತನ ಪಾದಗಳ ಮೇಲೆ ಅಡ್ಡ ಬಿದ್ದರು. ಆ ಮಹಾಮಹಿಮನ ದಿವ್ಯರೂಪವನ್ನು ಕಣ್ಣಿಂದಲೆ ಕುಡಿಯುವಂತೆ ನೋಡುತ್ತ, ಅವರು ತಮಗಾದ ಮಹದಾನಂದದಿಂದ ತಮ್ಮ ಕಷ್ಟಗಳನ್ನೆಲ್ಲ ಕ್ಷಣಮಾತ್ರದಲ್ಲಿ ಮರೆತರು. ಅವರೆಲ್ಲರೂ ಕೈಜೋಡಿಸಿ ನಿಂತು ‘ಹೇ ದೇವದೇವ, ಜರಾಸಂಧನ ಬಂಧನ ದಿಂದ ನಮ್ಮನ್ನು ಬಿಡಿಸಿದೆ, ಅದರಂತೆ ಸಂಸಾರಬಂಧನದಿಂದಲೂ ನಮ್ಮನ್ನು ಉದ್ಧರಿಸು. ಜರಾಸಂಧನು ನಮ್ಮಲ್ಲಿ ತೋರಿದ ಆಗ್ರಹ ನಮಗಿಂದು ಅನುಗ್ರಹವನ್ನು ತೋರಿದಂತಾಗಿದೆ. ಅದರಿಂದ ನಾವಿಂದು ನಿನ್ನ ದರ್ಶನಭಾಗ್ಯವನ್ನು ಪಡೆದೆವು. ಸ್ವಾಮಿ, ನಾವೆಲ್ಲ ನಿನ್ನ ಮಾಯೆಗೆ ಒಳಗಾದವರು. ಅರಿಯದ ಮಕ್ಕಳು ಬಿಸಿಲ್ದೊರೆಯನ್ನು ನೀರೆಂದು ಭ್ರಮಿಸುವಂತೆ ನಾವು ಭೋಗವಸ್ತುಗಳನ್ನು ಸುಖಕರಗಳೆಂದು ಭ್ರಮಿಸುತ್ತೇವೆ. ಆ ರೋಗಕ್ಕೆ ನಿನ್ನ ಕೃಪೆಯೊಂದೆ ಮದ್ದು. ನಾವು ಸಂಸಾರಿಗಳಾಗಿದ್ದೂ ನಿನ್ನ ಪಾದವನ್ನು ಮರೆಯದಂತಹ ವರವನ್ನು ನಮಗೆ ಕರುಣಿಸು’ ಎಂದು ಬೇಡಿಕೊಂಡರು. ಶ್ರೀಕೃಷ್ಣನು ಅವರನ್ನು ಕುರಿತು ಸಕ್ಕರೆಯಂತಹ ಸವಿಯಾದ ದನಿಯಲ್ಲಿ ‘ಅಯ್ಯಾ, ರಾಜರೆ, ನೀವು ಪುಣ್ಯಶಾಲಿಗಳು, ಆದ್ದರಿಂದಲೇ ನಿಮಗೀ ಸದ್ಬುದ್ಧಿ ಹುಟ್ಟಿದೆ. ತುಂಬ ಸಂತೋಷ. ನಿಮಗೆ ಭಗವಂತನಲ್ಲಿ ಸ್ಥಿರವಾದ ಭಕ್ತಿ ನೆಲಸಲಿ. ನೀವು ಈ ದೇಹ ಶಾಶ್ವತವಲ್ಲವೆಂಬುದನ್ನು ಮರೆಯದೆ, ಬದುಕಿರುವಷ್ಟು ಕಾಲ ಯಜ್ಞಯಾಗಾದಿಗಳಿಂದ ದೇವದೇವನನ್ನು ಆರಾಧಿ ಸುತ್ತಾ, ಧರ್ಮದಿಂದ ಪ್ರಜೆಗಳನ್ನು ಕಾಪಾಡುತ್ತಾ ಇರಿ. ನಿಮ್ಮ ಪ್ರಾರಬ್ಧಕರ್ಮ ಕಳೆಯು ತ್ತಲೆ ನೀವು ಪರಬ್ರಹ್ಮ ಸ್ವರೂಪಿಯಾದ ನನ್ನನ್ನು ಸೇರುವಿರಿ’ ಎಂದು ಹೇಳಿ, ಸಹದೇವ ರಾಜನಿಂದ ಅವರಿಗೆಲ್ಲ ಉತ್ತಮವಾದ ವಸ್ತ್ರಾಭರಣಗಳನ್ನೂ ಮೃಷ್ಟಾನ್ನವನ್ನೂ ಕೊಡಿಸಿ, ಸಕಲ ಸನ್ಮಾನಗಳೊಡನೆ ಅವರನ್ನು ಅವರವರ ರಾಜ್ಯಗಳಿಗೆ ಬೀಳ್ಕೊಟ್ಟನು. ಅನಂತರ ಆತನು ಭೀಮಾರ್ಜುನರೊಡನೆ ಇಂದ್ರಪ್ರಸ್ಥಕ್ಕೆ ಹಿಂದಿರುಗಿದನು. ಧರ್ಮರಾಜನು ಜರಾ ಸಂಧ ಮಡಿದುದನ್ನು ಕೇಳಿ, ಅತ್ಯಂತ ಸಂತೋಷದಿಂದ ಶ್ರೀಕೃಷ್ಣನನ್ನು ಆಲಿಂಗಿಸಿಕೊಂಡು ಆನಂದಬಾಷ್ಪವನ್ನು ಸುರಿಸಿದನು.

