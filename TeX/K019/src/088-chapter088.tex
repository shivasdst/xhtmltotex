
\chapter{೮೮. ವೃಕ}

ಶಕುನಿಯ ಮಗ ವೃಕ. ಅವನಿಗೆ ಒಮ್ಮೆ ತ್ರಿಲೋಕಸಂಚಾರಿಗಳಾದ ನಾರದರು ಕಾಣಿಸಿ ದರು. ಅವನು ಅವರನ್ನು ‘ಸ್ವಾಮಿ, ತ್ರಿಮೂರ್ತಿಗಳಲ್ಲಿ ಬಹುಬೇಗ ಅನುಗ್ರಹ ಮಾಡು ವರಾರು?’ ಎಂದು ಕೇಳಿದ. ನಾರದರು ‘ಅಯ್ಯಾ ವೃಕಾಸುರ, ಬಹುಬೇಗ ಕೇಳಿದವರಿಗೆ ಕೇಳಿದ ವರವನ್ನು ಕೊಡುವವನು ಶಂಕರ. ನೋಡು, ರಾವಣನ ಭಕ್ತಿಗೆ ಮೆಚ್ಚಿ ಅವನು ಕೇಳಿದ ವರಗಳನ್ನು ಕೊಟ್ಟ, ಬಾಣಾಸುರನ ಭಕ್ತಿಗೆ ಮೆಚ್ಚಿ ಅವನ ಪಟ್ಟಣವನ್ನು ಕಾಯುವವನಾಗಿದ್ದ. ನೀನು ಆತನನ್ನು ಭಜಿಸು’ ಎಂದ. ಒಡನೆಯೆ ವೃಕಾಸುರನು ಕೇದಾರ ಕ್ಷೇತ್ರಕ್ಕೆ ಹೋಗಿ, ಅಗ್ನಿಯಲ್ಲಿ ತನ್ನ ದೇಹದ ಮಾಂಸವನ್ನೆ ಕೊಯ್ದು ಹೋಮಮಾಡುತ್ತಾ ರುದ್ರನನ್ನು ಆರಾಧಿಸತೊಡಗಿದ. ಆರು ದಿನಗಳಲ್ಲಿ ಅವನ ದೇಹದ ಮಾಂಸವೆಲ್ಲ ಅಗ್ನಿಗೆ ಆಹುತಿಯಾಗಿಹೋಯಿತು. ರುದ್ರನು ಇನ್ನೂ ಪ್ರತ್ಯಕ್ಷವಾಗಲಿಲ್ಲವಲ್ಲಾ ಎಂದು ತಳಮಳ ಗೊಳ್ಳುತ್ತಾ, ಆತನು ಏಳನೆಯ ದಿನ ಬೆಳಗ್ಗೆ ತನ್ನ ತಲೆಯನ್ನೆ ಕತ್ತರಿಸಿ ಹೋಮಮಾಡ ಹೊರಟನು. ಆಗ ಪರಮಕಾರುಣಿಕನಾದ ಶಂಕರನು ಹೋಮಕುಂಡದ ಮಧ್ಯದಿಂದ ಮೇಲಕ್ಕೆದ್ದು, ಆತನ ಕ್ರೂರಕಾರ್ಯವನ್ನು ನಿಲ್ಲಿಸಿ, ತನ್ನ ಕೈಯಿಂದ ಆತನ ದೇಹವನ್ನೆಲ್ಲ ಸವರಿದನು. ಒಡನೆಯೆ ಅವನ ದೇಹದ ಗಾಯಗಳೆಲ್ಲ ಮಾಯವಾಗಿ ಅವನು ವಜ್ರಕಾಯ ನಾದನು. ಅನಂತರ ಭಗವಂತನು ಅವನೊಡನೆ ‘ಅಯ್ಯಾ, ನಾನು ನಿನ್ನ ಭಕ್ತಿಗೆ ಮೆಚ್ಚಿದೆ. ಬೇಕಾದ ವರವನ್ನು ಬೇಡು. ನಾನು ಅಲ್ಪತೃಪ್ತ. ನನ್ನ ಭಕ್ತರು ನೀರಿನಿಂದ ಪೂಜಿಸಿದರೂ ಸಾಕು, ನಾನು ಪ್ರೀತನಾಗುತ್ತೇನೆ. ನೀನು ನಿನ್ನ ದೇಹವನ್ನೆ ಕತ್ತರಿಸಿಹಾಕಿ ನನ್ನನ್ನು ಆರಾಧಿಸಿರುವೆ. ನಿನಗೇನು ಬೇಕೊ ಕೇಳು’ ಎಂದನು. ಆಗ ಆ ರಕ್ಕಸನು ‘ಪ್ರಭು, ನಾನು ಯಾರ ತಲೆಯ ಮೇಲೆ ಕೈ ಇಡುತ್ತೇನೋ ಅವರು ಸತ್ತುಹೋಗಬೇಕು. ಇದೇ ನಾನು ಬೇಡುವ ವರ’ ಎಂದ. ವೃಕಾಸುರನು ಬೇಡಿದ ಈ ವರವನ್ನು ಕೇಳಿ ಶಂಕರ ಮಂಕಾದ. ಆತನಿಗೆ ಕೋಪವೂ ಹುಟ್ಟಿತು. ತಾನು ಮಾತು ಕೊಟ್ಟುದಕ್ಕಾಗಿ ಆತನು ಮನಸ್ಸಿನಲ್ಲಿಯೆ ಪರಿತಪಿಸಿದನು. ಆದರೂ ಆಡಿದ ಮಾತನ್ನು ತಪ್ಪುವಂತಿಲ್ಲ. ಆದ್ದರಿಂದ ಆತನು ‘ತಥಾಸ್ತು’ ಎಂದ.

ಶಂಕರನು ವೃಕಾಸುರನಿಗೆ ವರವಿತ್ತುದು ಹಾವಿಗೆ ಹಾಲೆರೆದಂತಾಯಿತು. ಆ ರಕ್ಕಸ ತಾನು ಪಡೆದ ಮಹಿಮೆಯನ್ನು ಪರೀಕ್ಷಿಸುವುದಕ್ಕಾಗಿ ವರವಿತ್ತ ಶಂಕರನ ತಲೆಯ ಮೇಲೆಯೆ ಕೈ ಇಡುವುದಕ್ಕೆ ಹೊರಟ. ಇದನ್ನು ಕಂಡು ಶಂಕರನು ಭಯದಿಂದ ಓಡುವುದಕ್ಕೆ ಆರಂಭಿ ಸಿದ. ಆದರೆ ಭಕ್ತ ಭಗವಂತನ ಬೆನ್ನಟ್ಟಿದ. ದಿಕ್ಕುಗಳ ಕೊನೆಯಾಯಿತು; ಭೂಮಿ, ಆಕಾಶ, ಪಾತಾಳಗಳಿಗೆ ಓಡಿದುದಾಯಿತು. ಎಲ್ಲಿಗೆ ಓಡಿದರೂ ಬೆನ್ನ ಹಿಂದಿನ ಬೇತಾಳದಂತೆ ರಕ್ಕಸ ಬೆನ್ನಟ್ಟಿ ಬರುತ್ತಿದ್ದಾನೆ. ಕೊನೆಗೆ ಶಂಕರನು ವೈಕುಂಠಕ್ಕೆ ಓಡಿಬಂದ. ನಾರಾಯಣನು ಆತನನ್ನೂ ವೃಕಾಸುರನನ್ನೂ ಕಂಡು ಒಡನೆಯೆ ತನ್ನ ಯೋಗಮಾಯೆಯಿಂದ ಒಬ್ಬ ವಟು ವಿನ ಆಕೃತಿಯನ್ನು ತಾಳಿ, ಆ ರಾಕ್ಷಸನಿಗೆ ಇದಿರಾದನು. ಆತನು ದೂರದಿಂದಲೆ ರಾಕ್ಷಸ ನನ್ನು ಕೂಗಿ ‘ಅಯ್ಯಾ, ರಾಕ್ಷಸರಾಜ, ನಮಸ್ಕಾರ! ಓಡಿ ಓಡಿ ತುಂಬ ಆಯಾಸಗೊಂಡಂತೆ ಕಾಣುತ್ತೀಯಲ್ಲ; ಸ್ವಲ್ಪ ನಿಂತು ಸುಧಾರಿಸಿಕೊ. ಈ ದೇಹಕ್ಕೆ ಇಷ್ಟು ಆಯಾಸ ಕೊಡ ಬಾರದು. ದೇಹವೇ ಸರ್ವ ಸುಖಕ್ಕೂ ಮೂಲವಲ್ಲವೆ? ನೀನು ಇಷ್ಟು ಆಯಾಸಮಾಡಿ ಕೊಳ್ಳುವುದಕ್ಕೆ ಕಾರಣವಾದರೂ ಏನು ಹೇಳು, ಕೇಳೋಣ’ ಎಂದ. ಬೆಣ್ಣೆಯಂತಹ ಅವನ ಮಾತುಗಳಿಗೆ ಮರುಳಾದ ವೃಕಾಸುರ ತನ್ನ ಕಥೆಯನ್ನೆಲ್ಲ ಹೇಳಿದ. ಆಗ ಆ ವಟು ಗಹಗಹಿಸಿ ನಗುತ್ತಾ ‘ಅಯ್ಯೋ ಹುಚ್ಚಪ್ಪ! ಆ ರುದ್ರನ ಮಾತನ್ನು ಕೇಳಿ ನೀನು ಕೆಟ್ಟೆ. ಅವನು ಪಿಶಾಚಿ ಗಳ ರಾಜ; ಪಿಶಾಚಿಗಳ ಮಾತನ್ನು ಯಾರಾದರೂ ನಂಬುತ್ತಾರೆಯೆ? ಬೇಕಾದರೆ ಈಗ ಪರೀಕ್ಷೆಮಾಡಿ ನೋಡು. ನಿನ್ನ ತಲೆಯ ಮೇಲೆ ನಿನ್ನ ಕೈಯಿಟ್ಟುಕೊ; ಆಗ ಶಿವನ ಮಾತು ಎಂತಹ ಹಸಿಯ ಸುಳ್ಳು ಎಂಬುದು ನಿನಗೇ ಗೊತ್ತಾಗುತ್ತದೆ. ಆಮೇಲೆ ಆ ರುದ್ರನನ್ನು ಹಿಡಿದು ಇಂತಹ ಸುಳ್ಳನ್ನು ಇನ್ನೊಮ್ಮೆ ಆಡದಂತೆ ಅವನನ್ನು ಶಿಕ್ಷಿಸಬಹುದು’ ಎಂದನು. ಅಲ್ಪಬುದ್ಧಿಯ ಆ ವೃಕಾಸುರ ಆ ಮಾತನ್ನು ನಂಬಿ ತನ್ನ ಕೈಯನ್ನು ತನ್ನ ತಲೆಯ ಮೇಲಿಟ್ಟುಕೊಂಡ, ತಕ್ಷಣವೇ ಸತ್ತು ನೆಲಕ್ಕೊರಗಿದ.

ಆದಿ ಶ್ರೀಹರಿ ಶಿವನನ್ನು ಕುರಿತು ‘ಹೇ ದೇವೋತ್ತಮ, ಈ ಪಾಪಿ ತನ್ನ ಪಾಪದಿಂದಲೆ ಸತ್ತ. ಲೋಕಗುರುವಾದ ನಿನ್ನಲ್ಲಿ ಅಪರಾಧಿಯಾದವನು ಹೀಗಾಗಬೇಕಾದುದು ಸಹಜ’ ಎಂದು ಹೇಳಿ, ಆತನನ್ನು ಕೈಲಾಸಕ್ಕೆ ಬೀಳ್ಕೊಟ್ಟ.

