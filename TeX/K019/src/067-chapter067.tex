
\chapter{೬೭. ನನಗೆ ಭಾಮಾಮಣಿಯೆ ಸಾಕು, ಮಣಿ ಬೇಡ}

ಶ್ರೀಕೃಷ್ಮನು ತನ್ನ ಕೂರ್ತಮಡದಿಯಾದ ರುಕ್ಮಿಣಿಯೊಡನೆ ಅಂತಃಪುರದಲ್ಲಿ ಪಗಡೆ ಯಾಟದ ವಿನೋದದಲ್ಲಿದ್ದಾನೆ. ಇದ್ದಕ್ಕಿದ್ದಂತೆ ಹಲವು ಪಟ್ಟಣಿಗರು ಅರಮನೆಯ ಬಳಿಗೆ ಓಡಿಬಂದು ‘ಪ್ರಭೋ, ಶ್ರೀಕೃಷ್ಣ!’ ಎಂದು ಕೂಗಿಕೊಂಡರು. ಶ್ರೀಕೃಷ್ಣನು ಹೊರಕ್ಕೋಡಿ ಬಂದು ‘ಏನು ವಿಚಾರ?’ ಎಂದು ಅವರನ್ನು ಕೇಳಿದ. ಅವರು ಹೇಳಿದರು–‘ಸ್ವಾಮಿ, ನಿನ್ನನ್ನು ಕಾಣುವುದಕ್ಕಾಗಿ ಸಾಕ್ಷಾತ್ ಸೂರ್ಯಭಗವಾನನೆ ಬರುತ್ತಿದ್ದಾನೆ. ನಮ್ಮ ಕಣ್ಣುಗಳೆಲ್ಲ ಆತನ ಕಾಂತಿಯಿಂದ ಸೀದು ಹೋಗುತ್ತಿವೆ. ನೀನು ಹೋಗಿ ಆತನಿಗೆ ದರ್ಶನವಿತ್ತು, ಆತನನ್ನು ಹಾಗೆಯೇ ಹಿಂದಕ್ಕೆ ಕಳುಹಿಸು’ ಎಂದು ಬೇಡಿಕೊಂಡರು. ಅವರ ಗಾಬರಿ ಯನ್ನು ಕಂಡು ಶ್ರೀಕೃಷ್ಣನಿಗೆ ನಗುಬಂತು. ಅವರನ್ನು ಕುರಿತು ಆತ ‘ಅಯ್ಯಾ ಭಯ ಪಡಬೇಡಿ. ಆತ ಸೂರ್ಯನಲ್ಲ, ನಮ್ಮ ಸತ್ರಾಜಿತ. ಸೂರ್ಯನನ್ನು ಕುರಿತು ತಪಸ್ಸು ಮಾಡಿ, ಸ್ಯಮಂತಕವೆಂಬ ರತ್ನವನ್ನು ಪಡೆದಿದ್ದಾನೆ. ಅದನ್ನು ಕುತ್ತಿಗೆಯಲ್ಲಿ ಧರಿಸಿದ ಆತ ಬರುತ್ತಿರುವಾಗ ನಿಮಗೆ ಸೂರ್ಯನೇ ಬಂದನೆಂಬ ಭ್ರಾಂತಿ ಬಂದಿದೆ. ಆ ರತ್ನ ಸೂರ್ಯ ನಂತೆಯೇ ಪ್ರಕಾಶವುಳ್ಳದ್ದು’ ಎಂದು ಸಮಾಧಾನ ಹೇಳಿ, ಅವರನ್ನು ಹಿಂದಕ್ಕೆ ಕಳುಹಿಸಿದನು.

ಸ್ಯಮಂತಕಮಣಿಯನ್ನು ಧರಿಸಿದ ಸತ್ರಾಜಿತನು ನೇರವಾಗಿ ನಡೆದು ತನ್ನ ಮನೆಯನ್ನು ಸೇರಿದನು. ಆತನನ್ನು ಕಂಡು ಆತನ ಬಂಧುಬಾಂಧವರೆಲ್ಲ ತುಂಬ ಸಂತೋಷಪಟ್ಟರು. ಸತ್ರಾಜಿತನು ತನ್ನ ಪುರೋಹಿತರನ್ನು ಕರೆಸಿ, ಅವರಿಂದ ಸ್ಯಮಂತಕ ರತ್ನಕ್ಕೆ ಪೂಜೆಮಾಡಿಸಿ, ಅದನ್ನು ತನ್ನ ಮನೆದೇವರ ಬಳಿಯಲ್ಲಿ ಇರಿಸಿದನು. ಅದು ರತ್ನವೆಂದರೆ ಸಾಮಾನ್ಯ ರತ್ನವಲ್ಲ; ಪ್ರತಿದಿನವೂ ಅದು ಎಂಟು ಭಾರದಷ್ಟು ತೂಕ ಚಿನ್ನವನ್ನು ಕೊಡುತ್ತಿತ್ತು. ಅದಿದ್ದ ಕಡೆ ಬಡತನವಿಲ್ಲ. ಅಷ್ಟೇ ಅಲ್ಲ, ಅದನ್ನು ಪೂಜಿಸುತ್ತಿದ್ದರೆ ರೋಗರುಜಿನಗಳ ಕಾಟವಿಲ್ಲ, ಶತ್ರುಭಯವಿಲ್ಲ, ಯಾವ ಅಮಂಗಳವೂ ಇಲ್ಲ. ಇದನ್ನು ತಿಳಿದ ಶ್ರೀಕೃಷ್ಣನು ಒಂದು ದಿನ ಸತ್ರಾಜಿತನನ್ನು ಕಂಡು ‘ಅಯ್ಯಾ, ಇಂತಹ ರತ್ನ ನ್ಯಾಯವಾಗಿ ರಾಜನ ಬಳಿ ಇರಬೇಕು. ಇದನ್ನು ನಮ್ಮ ಉಗ್ರಸೇನ ಮಹಾರಾಜರಿಗೆ ಕೊಡು’ ಎಂದ. ಆದರೆ ಹಾಗೆ ಕೊಡುವ ಮನಸ್ಸು ಯಾರಿಗೆ ಬಂದೀತು? ಸತ್ರಾಜಿತ ‘ಉಹು’ ಎಂದ. ಇದಾದ ಕೆಲವು ದಿನಗಳಮೇಲೆ ಸತ್ರಾಜಿತನ ತಮ್ಮನಾದ ಪ್ರಸೇನನೆಂಬುವನು ಒಂದು ದಿನ ಬೇಟೆಗೆ ಹೋಗುವಾಗ ರತ್ನವನ್ನು ಕೊರಳಲ್ಲಿ ಧರಿಸಿ ಹೊರಟ. ಆತ ಬೇಟೆಯಾಡುತ್ತಿರುವಾಗ ಒಂದು ಸಿಂಹ ಆತನ ಮೇಲೆ ಹಾರಿ ಅವನನ್ನು ಕೊಂದು ಹಾಕಿತು. ನಂತರ ಅವನ ಕೊರಳ ಲ್ಲಿದ್ದ ಸ್ಯಮಂತಕ ರತ್ನವನ್ನು ಬಾಯಲ್ಲಿ ಕಚ್ಚಿಕೊಂಡು ಹೋಗುತ್ತ ಒಂದು ಗುಹೆಯನ್ನು ಪ್ರವೇಶಿಸಿತು. ಆ ಗುಹೆಯಲ್ಲಿ ವಾಸವಾಗಿದ್ದ ಒಂದು ಕರಡಿ ಆ ಸಿಂಹದೊಡನೆ ಹೋರಾಡಿ ಅದನ್ನು ಕೊಂದುಹಾಕಿ, ಆ ರತ್ನವನ್ನು ತನ್ನ ಮಗುವಿಗೆ ಆಟವಾಡುವುದಕ್ಕಾಗಿ ಕೊಟ್ಟಿತು.

ಇತ್ತ ದ್ವಾರಕೆಯಲ್ಲಿ ಬೇಟೆಗೆ ಹೋದ ತಮ್ಮ ಹಿಂದಕ್ಕೆ ಬಾರದುದನ್ನು ಕಂಡು, ಸತ್ರಾ ಜಿತನಿಗೆ ಭಯವಾಯಿತು. ಆತ ತಮ್ಮನಿಗಾಗಿಯೂ ರತ್ನಕ್ಕಾಗಿಯೂ ಎಷ್ಟೋ ಹುಡುಕಿಸಿದ; ಎಲ್ಲಿಯೂ ಸಿಕ್ಕಲಿಲ್ಲ. ಆತನಿಗೆ ಶ್ರೀಕೃಷ್ಣನ ಮೇಲೆ ಸಂದೇಹ ಹುಟ್ಟಿತು. ರತ್ನದ ಆಶೆ ಯಿಂದ ಅವನೇ ಪ್ರಸೇನನನ್ನು ಕೊಂದುಹಾಕಿರಬೇಕೆಂದುಕೊಂಡ. ಅವನು ತನ್ನ ಆಪ್ತ ರಲ್ಲಿ ಅದನ್ನು ಹೇಳಿಕೊಳ್ಳಲು, ಆ ಮಾತು ಕಿವಿಯಿಂದ ಕಿವಿಗೆ ಹಬ್ಬಿ ಶ್ರೀಕೃಷ್ಣನ ಕಿವಿಗೂ ಬಡಿಯಿತು. ಈ ಅಪವಾದವನ್ನು ಕೇಳಿ ಆತನಿಗೆ ಬಹು ವ್ಯಥೆಯಾಯಿತು. ಆತನು ಊರಿನ ಕೆಲವು ಜನರನ್ನು ಕರೆದುಕೊಂಡು ಪ್ರಸೇನನನ್ನು ಅರಸುವುದಕ್ಕಾಗಿ ಹೊರಟನು. ಅವನು ಹೋದ ದಿಕ್ಕನ್ನೇ ಹಿಡಿದು ಅಡವಿಯಲ್ಲಿ ಬಹುದೂರಹೋದ ಮೇಲೆ, ಒಂದು ಕಡೆ ಪ್ರಸೇನ ಸತ್ತುಬಿದ್ದಿರುವುದು ಕಾಣಿಸಿತು. ಅಲ್ಲಿಂದ ಮುಂದೆ ಸಿಂಹದ ಹೆಜ್ಜೆಗಳು ಕಾಣಿಸಲು, ಆತ ಅದರ ಹೆಜ್ಜೆಗಳನ್ನೆ ಹಿಂಬಾಲಿಸಿ ಹೋದ. ಒಂದು ಗುಹೆಯ ಬಳಿ ಸಿಂಹ ಸತ್ತು ಬಿದ್ದಿತ್ತು; ಅಲ್ಲಿಂದ ಮುಂದೆ ಕರಡಿಯ ಹೆಜ್ಜೆ ಕಾಣಿಸಿದವು. ಅವನ್ನು ಅನುಸರಿಸಿ ಆತ ಗುಹೆಯನ್ನು ಹೊಕ್ಕ. ಆತನ ಜೊತೆಯವರೆಲ್ಲ ಹೊರಗೆ ನಿಂತರು. ಶ್ರೀಕೃಷ್ಣ ಗುಹೆ ಯೊಳಗೆ ಹೋಗಿ ನೋಡುತ್ತಾನೆ, ಒಂದು ಕರಡಿಯ ಮರಿ ಅದನ್ನು ಕೈಲಿ ಹಿಡಿದು ಆಡುತ್ತಿದೆ! ಆ ಮರಿಯ ಬಳಿ ನಿಂತಿದ್ದ ಹೆಣ್ಣು ಕರಡಿಯೊಂದು ಶ್ರೀಕೃಷ್ಣನನ್ನು ಕಂಡು, ಭಯದಿಂದ ಗಟ್ಟಿಯಾಗಿ ಕಿರಿಚಿಕೊಂಡಿತು. ಒಡನೆಯೆ ಅಲ್ಲಿ ಮಲಗಿದ್ದ ಗಂಡುಕರಡಿ ದಿಗ್ಗನೆದ್ದು ಶ್ರೀಕೃಷ್ಣನ ಮೇಲೆ ಹಾರಿತು.

ಶ್ರೀಕೃಷ್ಣನ ಮೇಲೆ ಹಾರಿದ ಕರಡಿ ಸಾಮಾನ್ಯ ಕರಡಿಯಲ್ಲ. ತ್ರೇತಾಯುಗದಲ್ಲಿ ಶ್ರೀರಾಮನ ಸೇವಕನಾಗಿದ್ದ ಜಾಂಬವಂತ, ಅದು. ಕೃಷ್ಣಜಾಂಬವಂತರಿಬ್ಬರೂ ಮಹಾ ಶೂರರೇ. ಅವರಿಬ್ಬರಿಗೂ ಭಯಂಕರವಾದ ಯುದ್ಧ ನಡೆಯಿತು. ಮಾಂಸಕ್ಕಾಗಿ ಹೋರಾ ಡುವ ಗಿಡಗಗಳಂತೆ ಅವರು ಸ್ಯಮಂತಕ ರತ್ನಕ್ಕಾಗಿ ಇಪ್ಪತ್ತೆಂಟು ದಿನಗಳವರೆಗೆ ಒಂದೇ ಸಮನಾಗಿ ಹೋರಾಡಿದರು. ಆ ವೇಳೆಗೆ ಜಾಂಬವಂತನ ಶಕ್ತಿ ಕುಂದುತ್ತಾಬಂತು. ಶ್ರೀಕೃಷ್ಣನು ಒಂದು ಸಲ ಗುದ್ದಿದನೆಂದರೆ ಆತನ ದೇಹದ ಕೀಲುಗಳೆಲ್ಲ ಸಡಲಿದಂತಾ ಗುತ್ತಿತ್ತು. ಇದನ್ನು ಕಂಡು ಜಾಂಬುವಂತನಿಗೆ ಆಶ್ಚರ್ಯವಾಯಿತು. ತನ್ನನ್ನು ಸೋಲಿಸ ಬೇಕಾದರೆ ಇವನು ಮನುಷ್ಯನಲ್ಲ, ದೇವರೆ ಸರಿ–ಎಂದು ಆತನಿಗೆ ನಿರ್ಧರವಾಯಿತು. ಆತನು ಶ್ರೀಕೃಷ್ಣನನ್ನು ಕುರಿತು ‘ಮಹಾತ್ಮಾ, ನೀನು ಮನುಷ್ಯನಲ್ಲ. ಯಾವನ ಕೋಪಕ್ಕೆ ಹೆದರಿ ಸಮುದ್ರರಾಜ ಶರಣಾಗತನಾಗಿ ಹಾದಿಯನ್ನು ಬಿಟ್ಟುಕೊಟ್ಟನೊ, ಯಾವನು ಅದರ ಮೇಲೆ ಸೇತುವೆಯನ್ನೆ ಕಟ್ಟಿದನೊ, ಯಾವನು ಕಪಿಗಳ ಸಹಾಯದಿಂದ ಲಂಕೆಯನ್ನು ನಡುಗಿಸಿ, ರಾವಣನ ಹತ್ತು ತಲೆಗಳನ್ನೂ ಕತ್ತರಿಸಿಹಾಕಿದನೂ, ಆ ಶ್ರೀರಾಮನೆ ನೀನು. ಆತನ ಹೊರತು ನನ್ನನ್ನು ಜಯಿಸುವವರು ಜಗತ್ತಿನಲ್ಲಿ ಮತ್ತೊಬ್ಬರಿಲ್ಲ. ಸ್ವಾಮಿ, ನಿನಗೆ ಶರಣಾಗಿದ್ದೇನೆ. ಇಗೋ ಈ ರತ್ನವನ್ನು ತೆಗೆದುಕೋ. ಈ ನನ್ನ ಮಗಳಾದ ಜಾಂಬವತಿಯನ್ನೂ ಸ್ವೀಕರಿಸು’ ಎಂದು ಬೇಡಿದನು. ಶ್ರೀಕೃಷ್ಣನು ಆತನನ್ನು ಕುರಿತು ‘ಅಯ್ಯಾ, ಈ ರತ್ನಕ್ಕೆ ನಾನು ಆಸೆಪಟ್ಟವನಲ್ಲ. ಕೇವಲ ಅಪವಾದವನ್ನು ತಪ್ಪಿಸಿಕೊಳ್ಳುವುದ ಕ್ಕಾಗಿ ನಾನು ಇಲ್ಲಿಯವರೆಗೆ ಬರಬೇಕಾಯಿತು. ಈಗ ಈ ರತ್ನವನ್ನು ಅದರ ಒಡೆಯನಾದ ಸತ್ರಾಜಿತನಿಗೆ ಒಪ್ಪಿಸುತ್ತೇನೆ. ನನಗೆ ಕನ್ಯಾರತ್ನವೇ ಸಾಕು, ಈ ರತ್ನವೇನೂ ಬೇಡ’ ಎಂದು ಹೇಳಿ ಆತನ ಮಗಳಾದ ಜಾಂಬವತಿಯೊಡನೆ ರತ್ನವನ್ನೂ ತೆಗೆದುಕೊಂಡು ಅಲ್ಲಿಂದ ಹೊರಟನು.

ಶ್ರೀಕೃಷ್ಣನು ಗುಹೆಯಿಂದ ಬರುವವೇಳೆಗೆ ಅಲ್ಲಿ ಕಾದಿದ್ದವರೆಲ್ಲ ಮನೆಗೆ ಹಿಂದಿರುಗಿ ದ್ದರು. ಅವರಿಗೆ ಶ್ರೀಕೃಷ್ಣನು ಸತ್ತೇಹೋಗಿರಬಹುದೆಂಬ ಸಂದೇಹ. ಅವರು ದ್ವಾರಕಿಗೆ ಹಿಂದಿರುಗಿ, ನಡೆದ ಸಮಾಚಾರವನ್ನೆಲ್ಲ ಅರಮನೆಗೆ ವರದಿಮಾಡಿದರು. ವಸುದೇವ ದೇವಕಿಯರೂ ರುಕ್ಮಣಿಯೂ ಆ ಸುದ್ದಿಯನ್ನು ಕೇಳಿ, ಸತ್ರಾಜಿತನನ್ನು ಬಾಯಿಗೆ ಬಂದಂತೆ ಬಯ್ದು, ಶ್ರೀಕೃಷ್ಣನು ಸುಖವಾಗಿ ಬರಲೆಂದು ತಮ್ಮ ಕುಲದೇವತೆಯಾದ ದುರ್ಗಾದೇವಿಗೆ ಹರಕೆ ಹೊತ್ತರು. ಅವರ ಹರಕೆ ಸಫಲವಾಯಿತೋ ಎಂಬಂತೆ ಸ್ಯಮಂತಕರತ್ನ ಲಲನಾ ರತ್ನಗಳೊಡನೆ ಆತ ಹಿಂದಿರುಗಿದ. ಆತನನ್ನು ಕಾಣುತ್ತಲೆ ಬಂಧುಗಳಿಗೆಲ್ಲ ಆತನು ಮತ್ತೊಮ್ಮೆ ಹುಟ್ಟಿಬಂದಷ್ಟು ಸಂತೋಷವಾಯಿತು. ಊರಿನ ಜನರೆಲ್ಲ ಆತನನ್ನು ಕಂಡು ಅಭಿನಂದಿಸಿದರು. ಉಗ್ರಸೇನ ಮಹಾರಾಜನು ಸತ್ರಾಜಿತನನ್ನು ರಾಜ ಸಭೆಗೆ ಕರೆಸಿ, ಶ್ರೀಕೃಷ್ಣನಿಂದ ನಡೆದ ಕಥೆಯನ್ನೆಲ್ಲ ಹೇಳಿಸಿ, ಆತನ ರತ್ನವನ್ನು ಆತನಿಗೆ ಕೊಡಿಸಿದನು. ಸತ್ರಾಜಿತನಿಗೆ ತಾನು ಮಾಡಿದ ತಪ್ಪಿಗಾಗಿ ತುಂಬ ಪಶ್ಚಾತ್ತಾಪವಾಯಿತು. ಆತನು ಶ್ರೀಕೃಷ್ಣನ ಬಳಿಗೆ ಹೋಗಿ ‘ಅಯ್ಯಾ, ನನ್ನ ಮಗಳಾದ ಸತ್ಯಭಾಮೆಯನ್ನು ಕೈಹಿಡಿದು ನನ್ನ ವಂಶವನ್ನು ಉದ್ಧರಿಸು’ ಎಂದು ಬೇಡಿಕೊಂಡನು. ದಿವ್ಯ ಸುಂದರಿಯಾದ ಆ ಕನ್ಯೆಯನ್ನು ವಿವಾಹವಾಗಲು ಶ್ರೀಕೃಷ್ಣನು ಸಂತೋಷದಿಂದ ಸಮ್ಮತಿಸಿದನು. ಶುಭದಿನ ಶುಭ ಮುಹೂರ್ತದಲ್ಲಿ ಸತ್ರಾಜಿತನು ತನ್ನ ಮಗಳಾದ ಸತ್ಯಭಾಮೆಯನ್ನು ಆತನಿಗೆ ಧಾರೆ ಯೆರೆದುಕೊಟ್ಟು, ಸ್ಯಮಂತಕ ಮಣಿಯನ್ನೂ ಆತನಿಗೆ ಒಪ್ಪಿಸಿದನು. ಆದರೆ ಶ್ರೀಕೃಷ್ಣನು ‘ನನಗೆ ಭಾಮಾಮಣಿಯೆ ಸಾಕು, ಮಣಿ ಬೇಡ’ ಎಂದು ಹೇಳಿ, ಅದನ್ನು ಸತ್ರಾಜಿತನಿಗೇ ಹಿಂದಕ್ಕೆ ಕೊಟ್ಟನು.

