
\chapter{೮೭. ಭಕ್ತವತ್ಸಲ ಶ್ರೀಕೃಷ್ಣ}

ಮಳೆಗಾಲ ನಾಲ್ಕು ತಿಂಗಳು ಕಳೆದಮೇಲೆ ಒಂದು ದಿನ ಶ್ರೀಕೃಷ್ಣನು ನಾರದ, ಪರಾಶರ, ವ್ಯಾಸ, ವಾಮದೇವ, ಮೊದಲಾದ ಮಹರ್ಷಿಗಳ ತಂಡದೊಡನೆ ಮಿಥಿಲಾನಗರಿಗೆ ಹೊರಟನು. ಹಾದಿಯಲ್ಲಿ ಸಿಕ್ಕ ಹಳ್ಳಿ, ಪಟ್ಟಣಗಳಲ್ಲಿ ಜನರು ತನಗೆ ನೀಡಿದ ಆದರಾತಿಥ್ಯ ಗಳನ್ನು ಕೈಕೊಂಡು, ಅವರನ್ನು ಅನುಗ್ರಹಿಸುತ್ತಾ ಆತನು ಪಯಣದ ಮೇಲೆ ಪಯಣವನ್ನು ಮಾಡಿ ಮಿಥಿಲೆಯನ್ನು ಸೇರಿದನು. ಅಲ್ಲಿನ ರಾಜನಾದ ಬಹುಳಾಶ್ವನು ಶ್ರೀಕೃಷ್ಣನ ಪರಮಭಕ್ತ. ಆ ಊರಲ್ಲಿದ್ದ ಶ್ರುತದೇವನೆಂಬ ಬ್ರಾಹ್ಮಣನೂ ಶ್ರೀಕೃಷ್ಣನ ಏಕಾಂತಭಕ್ತ. ಅವರನ್ನು ಅನುಗ್ರಹಿಸುವುದಕ್ಕಾಗಿಯೆ ಶ್ರೀಕೃಷ್ಣನು ಅಲ್ಲಿಗೆ ಬಂದುದು. ಶ್ರೀಕೃಷ್ಣನ ಆಗಮನವನ್ನು ಕೇಳುತ್ತಲೆ ಮಿಥಿಲೆಯ ನಾಗರಿಕರೆಲ್ಲರೂ ಅತ್ಯಂತ ಸಡಗರದಿಂದ ಕಾಣಿಕೆ ಸಹಿತವಾಗಿ ಆತನನ್ನು ಇದಿರುಗೊಂಡು, ಆತನಿಗೂ ಪುಷಿಗಳಿಗೂ ಭಕ್ತಿಯಿಂದ ನಮಸ್ಕರಿ ಸಿದರು. ಬಹುಳಾಶ್ವ ಶ್ರುತದೇವರೂ ಅತ್ಯಂತ ಸಂತೋಷದಿಂದ ಓಡಿಬಂದು ಶ್ರೀಕೃಷ್ಣ ನಿಗೂ ಆತನ ಸಂಗಡಿಗರಿಗೂ ಅಡ್ಡ ಬಿದ್ದು, ಮುಗಿದ ಕೈಗಳಿಂದ ಅವರ ಇದಿರಿಗೆ ನಿಂತು ತಮ್ಮ ಮನೆಗೆ ಅತಿಥಿಗಳಾಗಿ ಬರಬೇಕೆಂದು ಬೇಡಿಕೊಂಡರು. ಇಬ್ಬರೂ ಏಕಕಾಲದಲ್ಲಿ ಒಂದೇ ಬೇಡಿಕೆಯನ್ನು ಮುಂದಿಟ್ಟುದನ್ನು ಕಂಡು ಶ್ರೀಕೃಷ್ಣನು ಮುಗುಳ್​ನಗೆಯಿಂದ ‘ಓಹೋ, ಅಗತ್ಯವಾಗಿಯೂ ಆಗಲಿ’ ಎಂದ. ಆತನು ಅವರಿಗೆ ಗೊತ್ತಾಗದಂತೆ ಏಕಕಾಲ ದಲ್ಲಿ ಎರಡು ರೂಪಗಳನ್ನು ತಾಳಿ, ಅವರಿಬ್ಬರ ಮನೆಗೂ ಹೋದನು.

ಶ್ರೀಕೃಷ್ಣನು ಹಲವು ಮಹರ್ಷಿಗಳೊಡನೆ ತನ್ನ ಮನೆಗೆ ಬಂದುದನ್ನು ಕಂಡು ಬಹುಳಾಶ್ವನಿಗೆ ಹಿಡಿಯಲಾರದಷ್ಟು ಆನಂದವಾಯಿತು. ಆತನು ಆನಂದಬಾಷ್ಪಗಳನ್ನು ಸುರಿಸುತ್ತಾ, ಅವರೆಲ್ಲರನ್ನೂ ಸುಖಾಸನದಲ್ಲಿ ಕುಳ್ಳಿರಿಸಿ, ಪಾದಗಳನ್ನು ತೊಳೆದು, ಆ ಪಾದೋದಕವನ್ನು ತಾನು ಶಿರಸ್ಸಿನಲ್ಲಿ ಧರಿಸಿ, ತನ್ನ ಕುಟುಂಬವರ್ಗದವರ ಮೇಲೂ ಅದನ್ನು ಪ್ರೋಕ್ಷಿಸಿದನು. ಅತಿಥಿಗಳೆಲ್ಲರನ್ನೂ ಗಂಧ ಪುಷ್ಪ ಧೂಪ ದೀಪಗಳಿಂದ ಪೂಜಿಸಿ, ಮೃಷ್ಟಾನ್ನ ಭೋಜನವನ್ನು ಮಾಡಿಸಿದನು. ಆಮೇಲೆ ಶ್ರೀಕೃಷ್ಣನ ಪಾದಗಳನ್ನು ತನ್ನ ತೊಡೆಯಮೇಲಿಟ್ಟುಕೊಂಡು ಒತ್ತುತ್ತಾ ‘ಹೇ ಪರಮಾತ್ಮ, ನಿನ್ನ ಪಾದದರ್ಶನಕ್ಕಾಗಿ ತವಕಿ ಸುತ್ತಿದ್ದ ನನ್ನ ಅಪೇಕ್ಷೆಯನ್ನು ಇಂದು ಈಡೇರಿಸಿ ಅನುಗ್ರಹಿಸಿರುವೆ. ನಾನು ಧನ್ಯನಾದೆ. ಎಲ್ಲವನ್ನೂ ತೊರೆದು ನಿನ್ನನ್ನು ಆಶ್ರಯಿಸುವವರಿಗೆ ನಿನ್ನನ್ನೆ ನೀನು ಒಪ್ಪಿಸಿಬಿಡುವಷ್ಟು ದಯಾಮಯ! ಲೋಕದ ಜನರ ಸಂಕಟಗಳನ್ನು ನಿವಾರಿಸುವುದಕ್ಕಾಗಿಯೆ ನೀನೀಗ ಮಾನವನಾಗಿ ಹುಟ್ಟಿಬಂದಿರುವೆ. ಹೇ ಪರಮೇಶ್ವರಾ, ನೀನು ನನ್ನ ಅತಿಥಿಯಾಗಿಯೆ ಇನ್ನು ಕೆಲಕಾಲ ನನ್ನ ಬಳಿಯಲ್ಲಿ ನಿಂತು ನನ್ನ ವಂಶವನ್ನು ಪಾವನಗೊಳಿಸಬೇಕು’ ಎಂದು ಬೇಡಿದನು. ಭಕ್ತವತ್ಸಲನಾದ ಶ್ರೀಕೃಷ್ಣನು ‘ತಥಾಸ್ತು’ ಎಂದು ಹೇಳಿ ಆತನ ಮನೆಯಲ್ಲೆ ನಿಂತನು.

ಅತ್ತ, ಹಲವು ಮಹರ್ಷಿಗಳೊಡನೆ ತನ್ನ ಮನೆಗೆ ಬಂದ ಶ್ರೀಕೃಷ್ಣನನ್ನು ಕಾಣುತ್ತಲೆ, ಶ್ರುತದೇವನು ಸಂತೋಷಾತಿಶಯದಿಂದ ತಾನು ಹೊದ್ದ ಉತ್ತರೀಯವನ್ನೆ ಹಾರಿಸಿ ಕುಣಿದಾಡುತ್ತಾ, ಬಂದವರಿಗೆಲ್ಲ ನೊದೆ ಹುಲ್ಲಿನ ಆಸನಗಳನ್ನು ಹಾಸಿ ಕುಳ್ಳಿರಿಸಿದನು. ತನ್ನ ಮಡದಿಯೊಡನೆ ಅವರ ಪಾದಗಳನ್ನು ತೊಳೆದು, ತೀರ್ಥವನ್ನು ತಾನು ತಳೆದು, ತನ್ನ ಮಡದಿ ಮಕ್ಕಳಿಗೂ ಪ್ರೋಕ್ಷಿಸಿದನು. ಗಂಧ ಪುಷ್ಪ ಧೂಪ ದೀಪಗಳಿಂದ ಅವರನ್ನು ಪೂಜಿ ಸಿದನು. ತನ್ನ ಮನೆಯಲ್ಲಿದ್ದ ಶುದ್ಧಾನ್ನವನ್ನು ಅವರಿಗೆ ನೀಡಿ ತೃಪ್ತಿಪಡಿಸಿದನು. ಆತನ ಮನಸ್ಸು ‘ಆಹಾ ನಾನೆಂತಹ ಭಾಗ್ಯಶಾಲಿ! ಸಂಸಾರವೆಂಬ ಕಗ್ಗಾಡಿನಲ್ಲಿ ತೊಳಲುತ್ತಿರುವ ನನಗೆ, ಮೋಕ್ಷಮಾರ್ಗವನ್ನು ತೋರುವ ಪರಮಾತ್ಮ ಈ ಮಹರ್ಷಿಗಳೊಡನೆ ದರ್ಶನ ವಿತ್ತನಲ್ಲ!’ ಎಂದು ನಲಿದು ನರ್ತಿಸುತ್ತಿತ್ತು. ಊಟವಾಗುತ್ತಲೆ ಆತನು ತನ್ನ ಮಡದಿ ಯೊಡನೆ ಶ್ರೀಕೃಷ್ಣನ ಪಾದಗಳನ್ನು ಒತ್ತುತ್ತಾ ‘ಸ್ವಾಮಿ, ನೀನು ಪ್ರಕೃತಿ, ಪುರುಷರಿಂದ ವಿಲಕ್ಷಣನಾಗಿ, ಅವಕ್ಕಿಂತಲೂ ಶ್ರೇಷ್ಠನಾಗಿರುವವನು. ನೀನು ಎಲ್ಲರ ಹೃದಯ ದಲ್ಲಿಯೂ ನೆಲೆಸಿರುವೆ. ಆದರೆ ನಿನ್ನನ್ನು ಕಾಣಲು ಎಷ್ಟು ಜನಕ್ಕೆ ಸಾಧ್ಯ? ನಿನ್ನ ಮಾಯಾಶಕ್ತಿಯಿಂದ ನಿನ್ನ ಸ್ವರೂಪ ಅವರಿಗೆ ಗೋಚರವಾಗದಂತೆ ಮಾಡುವೆ. ಮಹಾಯೋಗಿಗಳು ಕೂಡ ನಿನ್ನ ನಿಜಸ್ವರೂಪವೇನೆಂಬುದನ್ನು ತಮ್ಮ ಹೃದಯದಲ್ಲಿ ನೋಡಿ ತಿಳಿಯಬೇಕೆ ಹೊರತು ಅದು ಯಾರ ಕಣ್ಣಿಗೂ ಗೋಚರವಾಗದು. ಅಂತಹವನು ಈಗ ನನ್ನ ಕಣ್ಣಿಗೆ ಕಾಣಿಸಿರುವೆ. ನಿನ್ನ ದರ್ಶನವೆ ಮುಕ್ತಿ. ಇದಕ್ಕಿಂತಲೂ ದೊಡ್ಡ ಶ್ರೇಯಸ್ಸೇನಿದೆ? ನಿನ್ನ ಪಾದಸೇವೆಯ ಭಾಗ್ಯವನ್ನು ನನಗೆ ಅನುಗ್ರಹಿಸು’ ಎಂದು ಕೇಳಿ ಕೊಂಡನು. ಶ್ರೀಕೃಷ್ಣನು ಮಂದಹಾಸದಿಂದ ಆತನ ಕೈಹಿಡಿದು ‘ಅಯ್ಯಾ ಬ್ರಾಹ್ಮಣ, ಈ ಪುಷಿಗಳೆಲ್ಲ ನಿನ್ನನ್ನು ಅನುಗ್ರಹಿಸಲೆಂದೆ ಬಂದಿರುವರು. ಅವರು ಸದಾ ನನ್ನ ಧ್ಯಾನ ಮಾಡುವವರಾದುದರಿಂದ ಅವರಿದ್ದೆಡೆ ನಾನು ಇರುತ್ತೇನೆ. ಅವರು ತಮ್ಮ ಪಾದಧೂಳಿ ಯಿಂದ ಜಗತ್ತನ್ನು ಪಾವನಮಾಡುವುದಕ್ಕಾಗಿಯೆ ಲೋಕಸಂಚಾರ ಮಾಡುತ್ತಾರೆ. ಮಿತ್ರ, ಈ ಪುಷಿಗಳನ್ನೆಲ್ಲ ನಾನೆಂದೆ ತಿಳಿದು ಪೂಜಿಸು. ಅವರು ವೇದಮಯರು, ನಾನು ದೇವಮಯ; ದೇವಮಯವಾದ ನನ್ನ ದೇಹಕ್ಕಿಂತ ವೇದಮಯವಾದ ಪುಷಿದೇಹವೇ ನನಗೆ ಹೆಚ್ಚು ಪ್ರಿಯವಾದುದು. ಆದ್ದರಿಂದ ಅವರ ಪೂಜೆಯಿಂದ ನಾನು ಸಂತುಷ್ಟ ನಾಗುವೆ’ ಎಂದನು. ಶ್ರುತದೇವನು ಭಗವಂತನ ಅಪ್ಪಣೆಯಂತೆ ಆ ಬ್ರಹ್ಮಪುಷಿಗಳನ್ನು ಪರಮಾತ್ಮ ದೃಷ್ಟಿಯಿಂದ ಆರಾಧಿಸಿ ಉತ್ತಮ ಗತಿಯನ್ನು ಪಡೆದನು.

