
\chapter{೭೪. ನಂದಗೋಕುಲದಲ್ಲಿ ಬಲರಾಮ}

ವಸಂತಋತುವಿನ ಒಂದು ಬೆಳಗು ಮುಂಜಾನೆ ಎಚ್ಚರವಾದರೂ ಕಣ್ಮುಚ್ಚಿ ಮಲಗಿದ್ದ ಬಲರಾಮನಿಗೆ ತನ್ನ ಬಾಲ್ಯಜೀವನದ ರಸಚಿತ್ರಗಳು ಕಣ್ಣಿಗೆ ಕಟ್ಟಿ ನಿಂತಂತಾಯಿತು. ಒಮ್ಮೆ ನಂದಗೋಕುಲಕ್ಕೆ ಹೋಗಿ ಬಂಧುಗಳನ್ನು ನೋಡಿಕೊಂಡು ಬರಬೇಕು, ಬೃಂದಾ ವನದಲ್ಲಿ ಮನಬಂದಂತೆ ತಿರುಗಿ ಬರಬೇಕು, ಗೆಳೆಯ ಗೆಳತಿಯರೊಡನೆ ವಿನೋದ ವಾಗಿದ್ದು ಬರಬೇಕು–ಎನ್ನಿಸಿತು. ಆತನು ಸಡಗರದಿಂದ ಮೇಲಕ್ಕೆದ್ದು ಮುಖ ತೊಳೆ ದವನೆ ತನ್ನ ರಥವೇರಿ ನೇರವಾಗಿ ನಂದಗೋಕುಲಕ್ಕೆ ನಡೆದನು. ಬಹುಕಾಲದಮೇಲೆ ಬಂದ ಆತನನ್ನು ಕಂಡು ನಂದ ಯಶೋದೆಯರು ಹಿರಿಹಿರಿ ಹಿಗ್ಗಿದರು. ತಮಗೆ ನಮಸ್ಕರಿಸಿದ ಆತನನ್ನು ಎದೆಗಪ್ಪಿಕೊಂಡು ‘ಕಂದ, ನಿನ್ನ ತಮ್ಮನಾದ ಕೃಷ್ಣಮೂರ್ತಿಯೊಡನೆ ನಮ್ಮನ್ನು ಸದಾ ಕಾಪಾಡುತ್ತಿರಪ್ಪ’ ಎಂದು ಆಶೀರ್ವದಿಸಿ, ಬಗೆಬಗೆಯ ತಿಂಡಿ ತಿನಿಸುಗಳಿಂದ ಆತ ನನ್ನು ಆನಂದಪಡಿಸಿದರು. ಬಲರಾಮನು ಬಂದನೆಂಬ ಸುದ್ದಿಯನ್ನು ಕೇಳಿ, ಗೋಕುಲದ ಹಿರಿಯ ಕಿರಿಯರೆಲ್ಲ ಅಲ್ಲಿ ಬಂದು ನೆರೆದರು. ಪರಸ್ಪರ ಕ್ಷೇಮಸಮಾಚಾರವನ್ನು ವಿಚಾರಿ ಸಿದಮೇಲೆ ಆತನ ಬಾಲ್ಯದ ಗೆಳೆಯರೆಲ್ಲ ಹಿಂದಿನ ತಮ್ಮ ಆಟಪಾಟಗಳನ್ನೆಲ್ಲ ನೆನೆದು ವಿನೋದದಿಂದ ನಕ್ಕು ನಲಿದರು. ‘ಏನಯ್ಯಾ, ನೀನೂ ಕೃಷ್ಣನೂ ದೊಡ್ಡವರಾದಿರಿ, ನಿಮಗೆಲ್ಲ ದೊಡ್ಡ ದೊಡ್ಡ ಕಡೆಯಿಂದ ಹೆಣ್ಣುಗಳು ಬಂದು ಮದುವೆಯಾಗಿವೆ. ಕಂಸ ನನ್ನು ಕೊಂದು ನಿಮ್ಮವರೆಲ್ಲರ ಆದರ ಗೌರವಗಳಿಗೆ ಪಾತ್ರರಾಗಿದ್ದೀರಿ. ಈಗಲಂತೂ ಸಮುದ್ರದ ಮಧ್ಯೆ ಮನೆ ಮಾಡಿಕೊಂಡು ಅಜೇಯರಾಗಿಬಿಟ್ಟಿದ್ದೀರಿ. ನಾವೇನು ಒಮ್ಮೆ ಯಾದರೂ ಜ್ಞಾಪಕಕ್ಕೆ ಬರುತ್ತೇವೋ ಇಲ್ಲವೆ ಇಲ್ಲವೋ! ರಾಜಕೀಯಕ್ಕೆ ಇಳಿದಿರುವ ನಿಮಗೆ ನಮ್ಮಂತಹ ಗೊಲ್ಲರೆಲ್ಲಿ ಜ್ಞಾಪಕ!’ ಎಂದು ಗೇಲಿಮಾಡಿದರು. ಅವರು ಅತ್ತ ಹೋಗುತ್ತಲೆ ಗೋಪಿಯರ ಗುಂಪು ಆತನ ಬಳಿಗೆ ಬಂದಿತು. ಅವರು ಆತನ ಸುತ್ತ ಮುತ್ತಿಕೊಂಡು, ಮುಗುಳ್ನಗುತ್ತಾ ‘ಏನು, ದ್ವಾರಕಿಯ ಹೆಣ್ಣುಗಳಿಗೆಲ್ಲ ಕಣ್ಣಹಬ್ಬವನ್ನು ಮಾಡುವ ನಮ್ಮ ಶ್ರೀಕೃಷ್ಣ ಸುಖವಾಗಿದ್ದಾನೆಯೊ? ಈ ಹಳ್ಳಿಯ ಹೆಣ್ಣುಗಳನ್ನು ಒಮ್ಮೆ ಯಾದರೂ ಜ್ಞಾಪಿಸಿಕೊಳ್ಳುತ್ತಾನೆಯೊ? ನಾವು ಎಲ್ಲವನ್ನೂ ತೊರೆದು, ಅವನೇ ಸರ್ವಸ್ವ ವೆಂದು ಆದರಿಸಿದೆವು; ಆತ ನಮ್ಮನ್ನು ಕಾಲಿನಿಂದ ತಳ್ಳಿ, ಇಲ್ಲಿಂದ ಹೊರಟುಹೋದ. ಬಹುಬೇಗ ಹಿಂದಿರುಗುವೆನೆಂದು ಹೇಳಿ ಹೋದವನು ಇದುವರೆಗೆ ಹಿಂದಿರುಗಲಿಲ್ಲ. ನಾವಂತೂ ಹಳ್ಳಿಯ ಗುಗ್ಗುಗಳು; ಅವನ ಮಾತನ್ನು ನಂಬಿದೆವು. ಪಟ್ಟಣದ ಹೆಣ್ಣುಗಳೂ ಅವನ ಬಲೆಗೆ ಬೀಳುತ್ತಾರಂತಲ್ಲಾ!’ ಎಂದರು. ಅವರಲ್ಲಿಯೇ ಒಬ್ಬಳು ಮಾತಿನ ಮಧ್ಯೆ ಪ್ರವೇಶಿಸಿ, ‘ಅಮ್ಮ, ಅವನ ಆ ವಿಚಿತ್ರವಾದ ಮಾತುಗಳಿಗೂ, ಅವನ ಆ ಮುಗುಳ್ ನಗೆಗೂ ಯಾರು ತಾನೆ ಮೋಸಹೋಗುವುದಿಲ್ಲಮ್ಮ!’ ಎಂದು ಅವರಿಗೆ ಉತ್ತರವಿತ್ತಳು. ಮತ್ತೊಬ್ಬಳು ಸೆಡಕುಗಾತಿ ‘ಸಾಕು, ನಿಲ್ಲಿಸಿರೆ ಅವನ ಸುದ್ದಿಯನ್ನು. ಅವನ ಕೋಳಿಯಿಲ್ಲದೆ ನಮಗೆ ಬೆಳಗಾಗದೇನು?’ ಎಂದು ರೇಗಿದಳು. ಆಗ ಬಲರಾಮನು ಅವರನ್ನು ಕುರಿತು ಶ್ರೀಕೃಷ್ಣನು ಕಳಿಸಿದ್ದ ಸಂದೇಶವನ್ನು ತಿಳಿಸಿ, ಅವರನ್ನೆಲ್ಲ ಸಮಾಧಾನ ಮಾಡಿದನು.

ಗೋಪಿಯರಲ್ಲಿ ಕೆಲವರು ಬಲರಾಮನ ಗೆಳತಿಯರೂ ಇದ್ದರು. ಆತನು ಅವರೊಡನೆ ಬೃಂದಾವನದಲ್ಲಿ ಮನಬಂದಂತೆ ವಿಹರಿಸುತ್ತಾ ವಸಂತಋತು ಕಳೆದುಹೋಗುವವರೆಗೂ ಅಲ್ಲಿಯೆ ನಿಂತನು. ಆತನು ಹಾಗೆ ವಿಹರಿಸುವಾಗ ‘ವಾರುಣಿ’ ಎಂಬ ಮದ್ಯವನ್ನು ಕುಡಿದು, ತನ್ನ ಗೆಳತಿಯರಿಗೂ ಅದನ್ನು ಕುಡಿಸಿದನು. ಅದನ್ನು ಕುಡಿದು ಮತ್ತೇರಿ ಅವರೆಲ್ಲ ಹಾಡು ವರು, ಕುಣಿಯುವರು, ವನದ ಹೂಗಳನ್ನು ಕಿತ್ತು ಮುಡಿಯುತ್ತ, ಪರಸ್ಪರ ಮುಡಿಸುತ್ತ, ಇಹ ಜಗತ್ತನ್ನೆ ಮರೆತು ಆನಂದಪಡುವರು. ಒಮ್ಮೆ ಅವರು ಹೀಗೆ ಮೈಮರೆತು ವಿಹರಿಸು ತ್ತಿರುವಾಗ ಬಲರಾಮನಿಗೆ ಯಮುನಾನದಿಯಲ್ಲಿ ಜಲಕ್ರೀಡೆಯಾಡಬೇಕೆನಿಸಿತು. ಅವನು ಆ ನದಿಯನ್ನು ‘ಬಾ ಇಲ್ಲಿ, ನನ್ನ ಹತ್ತಿರಕ್ಕೆ’ ಎಂದು ಕರೆದ. ಅದು ಬರಲಿಲ್ಲ. ಅವನಿಗೆ ರೇಗಿಹೋಯಿತು, ಅವನು ತನ್ನ ನೇಗಿಲಿನಿಂದ ಅದನ್ನು ತನ್ನ ಕಡೆಗೆ ಎಳೆದುಕೊಂಡು ‘ಪಾಪಿ, ನಾನು ಕರೆದರೆ ಬರುವುದಿಲ್ಲವೆ? ಇಗೋ ಈ ನನ್ನ ನೇಗಿಲಿನಿಂದ ನಿನ್ನನ್ನು ನೂರು ತುಂಡಾಗಿ ಸೀಳಿಬಿಡುತ್ತೇನೆ’ ಎಂದು ಗರ್ಜಿಸಿದ. ಒಡನೆಯೆ ಯಮುನೆ ಹೆಣ್ಣಿನ ಆಕಾರ ದಿಂದ ಅವನ ಬಳಿಗೆ ಬಂದು ತನ್ನನ್ನು ಕ್ಷಮಿಸುವಂತೆ ಬೇಡಿಕೊಂಡಳು. ಆಗ ಬಲರಾಮ ನಗುತ್ತಾ ತನ್ನ ನೇಗಿಲನ್ನು ಹಿಂದಕ್ಕೆ ತೆಗೆದುಕೊಂಡು, ತನ್ನ ಗೆಳತಿಯರೊಡನೆ ಆ ನದಿ ಯಲ್ಲಿ ಮನಸ್ತೃಪ್ತಿಯಾಗುವಷ್ಟು ಜಲಕ್ರೀಡೆಯಾಡಿದನು.

ವಸಂತಕಾಲದ ಎರಡು ತಿಂಗಳನ್ನು ಎರಡು ಘಳಿಗೆಯಂತೆ ಕಳೆದು, ಅನಂತರ ಬಲರಾಮನು ಎಲ್ಲರಿಂದಲೂ ಬೀಳ್ಕೊಂಡು ದ್ವಾರಕಿಗೆ ಹಿಂದಿರುಗಿದನು.

