
\chapter{೧. ಭಾಗವತಾವತರಣ}

ನೈಮಿಶಾರಣ್ಯವೆಂಬುದು ಗಂಗಾತೀರದಲ್ಲಿರುವ ಒಂದು ಪುಣ್ಯಕ್ಷೇತ್ರ. ಅಲ್ಲಿ ಅನೇಕ ಪುಷಿಗಳು ವಾಸಮಾಡುತ್ತಿದ್ದರು. ಒಮ್ಮೆ ಶೌನಕನೇ ಮೊದಲಾದ ಅಲ್ಲಿನ ಪುಷಿಗಳೆಲ್ಲ ಸಹಸ್ರ ವರ್ಷಗಳು ನಡೆಯಬೇಕಾದ ಒಂದು ಮಹಾಯಾಗವನ್ನು ಕೈಕೊಂಡರು. ಉತ್ತಮ ಗತಿಯನ್ನು ಪಡೆಯಬೇಕೆಂಬ ಗುರಿಯಿಂದ ಅವರು ದಿನದಿನವೂ ಭಕ್ತಿಯಿಂದ ಭಗವಂತ ನನ್ನು ಆರಾಧಿಸುತ್ತಿರುವಾಗ, ಒಂದು ದಿನ ಬೆಳಗ್ಗೆ ಸೂತಪುರಾಣಿಕರು ಅಲ್ಲಿಗೆ ಬಂದರು. ಆತ ವೇದವ್ಯಾಸರ ಶಿಷ್ಯ; ಅನೇಕ ಪುರಾಣ ಧರ್ಮಶಾಸ್ತ್ರಗಳನ್ನು ಚೆನ್ನಾಗಿ ತಿಳಿದವರು; ಅವುಗಳನ್ನು ಆಗಾಗ ಜನರಿಗೆ ಬೋಧಿಸುವುದೇ ಅವರ ಉದ್ಯೋಗ. ಅವರನ್ನು ಕಾಣುತ್ತಲೆ ನೈಮಿಶಾರಣ್ಯದ ಪುಷಿಗಳೆಲ್ಲರೂ ಆತನ ಸುತ್ತ ನೆರೆದರು; ಆದರ ಉಪಚಾರಗಳಿಂದ ಅವರು ಆತನನ್ನು ಸತ್ಕರಿಸಿ, ಮೆತ್ತನೆಯ ಆಸನದ ಮೇಲೆ ಆತನನ್ನು ಕುಳ್ಳಿರಿಸಿ, ಸುತ್ತಲೂ ತಾವು ಕುಳಿತರು. ಅವರೆಲ್ಲರೂ ಸೂತರನ್ನು ಕುರಿತು ‘ಅಯ್ಯಾ ಮಹಾನುಭಾವ, ಈಗ ಕಲಿ ಯುಗ ಪ್ರಾರಂಭವಾಗಿದೆ. ಇನ್ನುಮುಂದೆ ಈ ಯುಗದಲ್ಲಿ ಹುಟ್ಟುವ ಜನರೆಲ್ಲ ಅಲ್ಪಾಯು ಗಳು, ಜಡಬುದ್ಧಿಯವರು ಆಗುತ್ತಾರಂತೆ! ಅವರಿಗೆ ಸದಾ ತಾಪತ್ರಯ ತಪ್ಪಿದುದಲ್ಲ ವಂತೆ! ಅವರು ಉದ್ಧಾರಕ್ಕೆ ಯತ್ನಿಸಿದರೂ ಅಡ್ಡಿಗಳು ಬರುತ್ತವಂತೆ! ಪಾಪ, ಆ ನಿರ್ಭಾಗ್ಯರ ಗತಿಯೇನು? ಅವರು ಉದ್ಧಾರವಾಗುವುದು ಹೇಗೆ? ಇದಕ್ಕೆ ಏನಾದರೂ ಮಾರ್ಗವುಂಟೆ? ನೀನು ಎಲ್ಲ ಶಾಸ್ತ್ರಗಳನ್ನೂ ತಿಳಿದವನು, ಕುಶಲಮತಿ; ಧರ್ಮಗ್ರಂಥಗಳ ಆಧಾರದಿಂದಲೊ, ನಿನ್ನ ಸ್ವಬುದ್ಧಿಯಿಂದಲೊ ನಮ್ಮ ಪ್ರಶ್ನೆಗೆ ಉತ್ತರಕೊಡು’ ಎಂದು ಕೇಳಿಕೊಂಡರು.

ಪುಷಿಗಳು ಮಾಡುತ್ತಿದ್ದುದು ಸಾವಿರವರ್ಷಗಳ ಯಾಗ; ಪುರಾಣ ಪುಣ್ಯಕಥೆಗಳನ್ನು ಕೇಳಲು ಬೇಕಾದಷ್ಟು ಕಾಲಾವಕಾಶವಿತ್ತು. ಆದ್ದರಿಂದ ಶೌನಕಪುಷಿಯು ಪುಷಿಗಳೆಲ್ಲರ ಪರವಾಗಿ ಸೂತರನ್ನು ಕುರಿತು “ಮುನೀಶ್ವರ, ನಮಗೆಲ್ಲ ಶ್ರೀಕೃಷ್ಣನ ಕಥೆಯನ್ನು ಕೇಳಬೇಕೆಂದು ಆಶೆ. ಆತ ಸಾಕ್ಷಾತ್ ಪರಮೇಶ್ವರನ ಅವತಾರವಂತೆ. ಶತ್ರು ಮಿತ್ರರೆಂಬ ಭೇದವಿಲ್ಲದೆ ಆತ ಎಲ್ಲರನ್ನೂ ಉದ್ಧಾರ ಮಾಡಿದನಂತೆ! ಆತನ ಹೆಸರನ್ನು ಕೇಳಿದರೆ ಭಯವೇ ಭಯಪಡುತ್ತದೆಯಂತೆ! ಮರೆತಾದರೂ ಒಮ್ಮೆ ಆತನ ಹೆಸರನ್ನು ನೆನೆದರೆ ಸಂಸಾರದ ಕಗ್ಗಂಟು ಬಿಚ್ಚಿಹೋಗುವುದಂತೆ! ಆತನನ್ನು ಕೀರ್ತಿಸಿದರೆ ಕಲಿ ಕಾಲದ ಕಿರುಕುಳಗಳೆಲ್ಲವೂ ತಪ್ಪುತ್ತವೆಯಂತೆ! ಆತನ ಕಥೆಯನ್ನು ಕೇಳಿದರೆ ಮೂಢರಿಗೂ ಮನ ಕರಗುವುದಂತೆ! ನಾರದಾದಿ ಮಹಾಪುಷಿಗಳು ಕೂಡ ಆತನ ಕೀರ್ತಿಯನ್ನು ಹಾಡುತ್ತಾರೆ. ಸಾಕ್ಷಾತ್ ಭಗವಂತನಾದವನು ವಸುದೇವ ದೇವಕೀದೇವಿಯರ ಮಗನಾಗಿ ಈ ಭೂಲೋಕ ದಲ್ಲಿ ಏಕೆ ಅವತರಿಸಿದ? ಇಲ್ಲಿ ಏನೇನು ಲೀಲೆಗಳನ್ನು ತೋರಿಸಿದ? ಎಲ್ಲವನ್ನೂ ವಿಸ್ತಾರ ವಾಗಿ ನಮಗೆ ತಿಳಿಸಿ ಹೇಳು. ಕಲಿಬಾಧೆಯನ್ನು ನೆನೆದು ಚಿಂತಿಸುತ್ತಿದ್ದ ನಮಗೆ ನಿನ್ನ ದರ್ಶನ ವಾದುದು, ಸಮುದ್ರದಲ್ಲಿ ಮುಳುಗುವವನಿಗೆ ನಾವಿಕನ ದರ್ಶನವಾದಂತಾಯಿತು. ‘ಕಲಿ’ ಎಂಬ ಸಮುದ್ರದಿಂದ ನಮ್ಮನ್ನು ದಾಟಿಸುವ ನಾವಿಕನಾಗಿ, ‘ಕೃಷ್ಣಚರಿತ್ರೆ’ ಎಂಬ ಹಡಗಿ ನಲ್ಲಿ ನಮ್ಮನ್ನು ಉದ್ಧರಿಸಿ ರಕ್ಷಿಸು” ಎಂದು ಬೇಡಿಕೊಂಡನು.

ಶೌನಕಾದಿ ಮಹರ್ಷಿಗಳ ಮಾತುಗಳಿಂದ ಸೂತಮುನೀಶ್ವರನಿಗೆ ಬಹು ಸಂತೋಷ ವಾಯಿತು. ಆತನು ಕಣ್ಣುಗಳನ್ನು ಮುಚ್ಚಿಕೊಂಡು ಒಂದೇ ಮನಸ್ಸಿನಿಂದ ತನ್ನ ಗುರುವಾದ ಶುಕಮಹರ್ಷಿಗಳನ್ನು ಕುರಿತು ಧ್ಯಾನಮಾಡಿದನು–,‘ಹೇ ಗುರುದೇವ, ವ್ಯಾಸಪುತ್ರ, ಬ್ರಹ್ಮಜ್ಞಾನಿ, ಪರಮಯೋಗೀಶ್ವರ, ಕೃತಕೃತ್ಯ, ನಿನಗೆ ನಮಸ್ಕಾರ! ಆತ್ಮತತ್ವವನ್ನು ಬೆಳಗ ಬಲ್ಲ ಜ್ಯೋತಿ ನೀನು! ಸಮಸ್ತ ವೇದಗಳ ಸಾರವಾದ ಭಾಗವತವನ್ನು ಜಗತ್ತಿಗೆ ಮೊಟ್ಟ ಮೊದಲು ಕೊಟ್ಟವನು ನೀನು! ನಿನಗೆ ನಮಸ್ಕಾರ!’ ಹೀಗೆ ತನ್ನ ಗುರುವನ್ನು ಸ್ತುತಿಸಿ, ಪುರಾಣಪಠನದ ಮುನ್ನ ಸ್ಮರಿಸಬೇಕಾದ ನರ, ನಾರಾಯಣ, ಸರಸ್ವತಿ, ವ್ಯಾಸರನ್ನು ಸ್ತುತಿಸಿ ನಮಸ್ಕರಿಸಿದನು. ಅನಂತರ ಆತನು ಅಲ್ಲಿದ್ದ ಪುಷಿಗಳನ್ನು ಕುರಿತು ‘ಅಯ್ಯಾ ಮಹರ್ಷಿ ಗಳೆ, ನಿಮ್ಮ ಪ್ರಶ್ನೆಯಿಂದ ನಿಮಗೆ ಮಾತ್ರವೇ ಅಲ್ಲ, ಜಗತ್ತಿಗೇ ಕಲ್ಯಾಣವಾಗುತ್ತದೆ. ಕೇಳಿ, ಎಲ್ಲ ವೇದಶಾಸ್ತ್ರಗಳ ಸಾರವೇ ಧರ್ಮ. ಧರ್ಮದಲ್ಲಿ ಪ್ರವೃತ್ತಿಧರ್ಮ, ನಿವೃತ್ತಿ ಧರ್ಮ–ಎಂದು ಎರಡು ವಿಧ. ಸ್ವರ್ಗವೇ ಮೊದಲಾದ ಉತ್ತಮ ಲೋಕಗಳನ್ನು ಪಡೆಯ ಬೇಕೆಂಬ ಬಯಕೆಯಿಂದ ಮಾಡುವ ಧರ್ಮ ಪ್ರವೃತ್ತಿಧರ್ಮ; ಪ್ರತಿಫಲದ ಅಪೇಕ್ಷೆ ಯಿಲ್ಲದೆ, ಶುದ್ಧಮನಸ್ಸಿನಿಂದ ಭಗವಂತನಲ್ಲಿ ಭಕ್ತಿಯನ್ನು ತೋರುವುದೇ ನಿವೃತ್ತಿಧರ್ಮ. ಭಗವಂತನಲ್ಲಿ ಭಕ್ತಿಯನ್ನಿಡುವುದರಿಂದ ಬೇರೆಯ ವಿಷಯಗಳಲ್ಲಿ ವಿರಕ್ತಿ ಹುಟ್ಟುತ್ತದೆ; ದೈವಸಾಕ್ಷಾತ್ಕಾರಕ್ಕೆ ಅಗತ್ಯವಾದ ಜ್ಞಾನವೂ ಹುಟ್ಟುತ್ತದೆ. ಆದ್ದರಿಂದ ಭಕ್ತಿಯು ಮೋಕ್ಷಕ್ಕೆ ಹೆದ್ದಾರಿ. ದೇವರ ಕೀರ್ತಿಯನ್ನು ಕೇಳುವಾಗ, ಹೇಳುವಾಗ, ಆತನನ್ನು ಧ್ಯಾನಿಸುವಾಗ, ಪೂಜಿಸುವಾಗ ಮನಸ್ಸು ಏಕಾಗ್ರವಾಗಿರಬೇಕು. ಇದರಿಂದ ಮುಕ್ತಿ ಬರುವುದು. ಆದ್ದರಿಂದ ಭಗವಂತನ ಚರಿತ್ರೆಯನ್ನು ಕೇಳಲು ಬಯಸುವುದು ಉದ್ಧಾರದ ಗುರುತು. ಈ ಬಯಕೆ ಹುಟ್ಟುವುದು ಸಾಮಾನ್ಯವಲ್ಲ. ಗಂಗಾದಿ ತೀರ್ಥಗಳಲ್ಲಿ ಮಿಂದು ಪವಿತ್ರಾತ್ಮರಾಗಿ, ಮಹಾತ್ಮರ ಸೇವೆಯಿಂದ ಶ್ರದ್ಧೆಯನ್ನು ಪಡೆದ ಮೇಲೆ ಭಗವಂತನ ಕಥೆಯನ್ನು ಕೇಳ ಬೇಕೆಂಬ ಬಯಕೆ ಹುಟ್ಟುತ್ತದೆ. ಅಂತಹವರು ಭಗವತ್ಕಥೆಯನ್ನು ಕೇಳುವಾಗ ದೇವರು ಅವರ ಮನಸ್ಸಿನಲ್ಲಿ ನೆಲಸಿ, ಅವರ ಪಾಪಗಳನ್ನೆಲ್ಲಾ ನಿರ್ಮೂಲಮಾಡುವನು. ಆಮೇಲೆ ಭಗವದ್ಭಕ್ತರನ್ನು ಸೇವಿಸುತ್ತಾ ಭಗವಂತನ ಕಥೆಗಳನ್ನು ಕೇಳುತ್ತಿದ್ದರೆ ಭಕ್ತಿಯು ಸ್ಥಿರ ವಾಗುತ್ತದೆ. ಕ್ರಮೇಣ ವಿಷಯಾಸಕ್ತಿ ತೊಲಗಿ ಆತ್ಮಸಾಕ್ಷಾತ್ಕಾರವಾಗುತ್ತದೆ; ಒಡನೆಯೆ ಅಹಂಕಾರ ನಾಶವಾಗುತ್ತದೆ. ಆದ್ದರಿಂದ ವಿವೇಕಿಗಳಾದವರು ಭಗವಂತನ ಲೀಲೆಗಳನ್ನು ಕೇಳುವುದರಲ್ಲಿ ಸದಾ ತತ್ಪರರಾಗಿಬೇಕು’ ಎಂದು ಹೇಳಿದನು.

ಸೂತಮಹರ್ಷಿಯು ತನ್ನ ಮಾತನ್ನು ಮುಂದುವರಿಸುತ್ತಾ ‘ಅಯ್ಯಾ ಮುನಿಗಳೆ, ಮಹಾಸರಸ್ಸಿನಿಂದ ಸಾವಿರಾರು ಕಾಲುವೆಗಳು ಪ್ರವಹಿಸುವಂತೆ ಭಗವಂತನಿಂದ ಅನೇಕ ಅವತಾರಗಳು ಆಗಿವೆ (ಪರಿಶಿಷ್ಟ ೧ ನೋಡಿ). ಆದರೆ ಅವುಗಳಲ್ಲಿ ಕೆಲವು ಕಲಾರೂಪಗಳು, ಕೆಲವು ಅಂಶರೂಪಗಳು; ನೀವು ಕೇಳಿದ ಶ್ರೀಕೃಷ್ಣವತಾರವು ಮಾತ್ರ ಪೂರ್ಣಾವತಾರ. ಶ್ರೀಕೃಷ್ಣನು ಸಾಕ್ಷಾತ್ ನಾರಾಯಣನೇ ಹೊರತು ಆತನ ಅಂಶವಲ್ಲ. ಈ ಅವತಾರ ಚರಿತ್ರೆ ಯನ್ನು ಕೇಳಿದರೆ ಸಂಸಾರರೂಪದ ಬಂಧನ ಬಿಟ್ಟು ಹೋಗುತ್ತದೆ. ನಿತ್ಯಶುದ್ಧನಾದ ಜೀವ ನಿಗೆ ಮಾಯೆಯಿಂದ ದೇಹ ಕಲ್ಪಿತವಾಗುತ್ತದೆ. ಈ ದೇಹದಲ್ಲಿ ಸ್ಥೂಲ, ಸೂಕ್ಷ್ಮ ಎಂದು ಎರಡು ವಿದl:ಸ್ಥೂಲ ದೇಹಕ್ಕೆ ಕೈಕಾಲು ಮೊದಲಾದ ಅವಯವಗಳು ಕಾಣಬರುತ್ತವೆ; ಸೂಕ್ಷ್ಮ ದೇಹಕ್ಕೆ ಇವು ಕಾಣಬರುವುದಿಲ್ಲ. ಇದನ್ನು ಲಿಂಗ ದೇಹವೆಂದು ಕರೆಯುತ್ತಾರೆ. ಇದನ್ನು ಕಂಡವರಿಲ್ಲ. ಆದರೂ ಇದೇ ಪುನರ್ಜನ್ಮಕ್ಕೆ ಕಾರಣವಾದುದರಿಂದ ಇದನ್ನು ಇಲ್ಲವೆಂದು ಹೇಳುವಂತಿಲ್ಲ. ಈ ಸ್ಥೂಲಸೂಕ್ಷ್ಮ ಶರೀರಗಳ ಸಂಬಂಧವಿರುವವರೆಗೂ ಜೀವನಿಗೆ ಬಿಡುಗಡೆಯಿಲ್ಲ. ಭಗವತ್ಕಥೆಯನ್ನು ಕೇಳುವುದರಿಂದ ಬರುವ ಜ್ಞಾನವು ಅಜ್ಞಾನ ದಿಂದ ಬಂದ ದೇಹಸಂಬಂಧವನ್ನು ಹೋಗಲಾಡಿಸಬಲ್ಲುದು. ಸ್ಥೂಲಸೂಕ್ಷ್ಮ ಶರೀರ ಗಳೆರಡೂ ಮಾಯಾಕಲ್ಪಿತ, ಅಸತ್ಯ–ಎಂದು ಅರ್ಥ ಮಾಡಿಕೊಂಡೊಡನೆಯೇ ಜೀವನು ದೇವನಾಗುತ್ತಾನೆ. ನಿತ್ಯಶುದ್ಧನಾದ ಜೀವನಿಗೆ ಸ್ಥೂಲಸೂಕ್ಷ್ಮ ಶರೀರಗಳು ಕಲ್ಪಿತವಾಗಿರು ವಂತೆಯೇ ಜನ್ಮರಹಿತನಾಗಿ ಅಕರ್ತೃವಾಗಿರುವ ಭಗವಂತನಿಗೂ ಅವತಾರ ಮತ್ತು ಕರ್ಮಗಳು ಮಾಯಾಕಲ್ಪಿತಗಳೆಂದು ಜ್ಞಾನಿಗಳು ಹೇಳುತ್ತಾರೆ. ಹೀಗೆ ಜೀವ ಈಶ್ವರರಲ್ಲಿ ಹೋಲಿಕೆ ಕಂಡು ಬರುವುದಾದರೂ ಈಶ್ವರನು ಸರ್ವತಂತ್ರ ಸ್ವತಂತ್ರನಾದವನು. ಆತನು ಲೀಲೆಗಾಗಿ ಸೃಷ್ಟಿ ಸ್ಥಿತಿ ಲಯಗಳನ್ನು ನಡೆಸುತ್ತಿದ್ದರೂ ತಾನು ಅದರಲ್ಲಿ ಪಾಲುಗಾರನಾಗು ವುದಿಲ್ಲ. ಇಂದ್ರಿಯ ನಿಯಾಮಕನಾದ ಆತನು ಸ್ವತಂತ್ರ. ಸಕಲ ಭೂತಗಳಲ್ಲಿಯೂ ಆತನು ಅಂತರ್ಯಾಮಿ; ಇಂದ್ರಿಯ ಸುಖಗಳನ್ನು ಅನುಭವಿಸುವನು, ಆದರೂ ಅನಾಸಕ್ತ. ಇದನ್ನು ಅರಿತುಕೊಳ್ಳುವುದು ಬಹು ಕಷ್ಟ. ಆತನ ಮಹಿಮೆ ಅಗಾಧ, ಅಗೋಚರ. ಆದರೂ ಶುದ್ಧ ಮನಸ್ಸಿನಿಂದ ನಿರಂತರವೂ ಆತನನ್ನು ಭಜಿಸುವವರು ಆತನ ಲೀಲೆಗಳನ್ನು ಅರ್ಥ ಮಾಡಿಕೊಳ್ಳಬಲ್ಲರು. ಅಯ್ಯಾ ಮುನಿಗಳೆ, ನೀವು ಅಂತಹ ಭಗವದ್ಭlಕ್ತರಾದುದರಿಂದ ನೀವು ಧನ್ಯರೇ ಸರಿ. ಭಕ್ತಿಯ ವಿಚಾರವನ್ನು ಕುರಿತು ನಾನು ಹೇಳುತ್ತಿರುವುದು ಹೊಸ ದೇನೂ ಅಲ್ಲ, ಭಗವಂತನಾದ ವ್ಯಾಸಮಹರ್ಷಿಯು ವೇದಸಮಾನವಾದ ಭಾಗವತದಲ್ಲಿ ಇದನ್ನು ಸವಿಸ್ತಾರವಾಗಿ ವಿವರಿಸಿರುವನು. ಅದನ್ನೇ ತನ್ನ ಮಗನಾದ ಶುಕಮುನಿಗೆ ಬೋಧಿkಸಿದನು. ಶುಕಮುನಿಯು ಇದನ್ನು ಪ್ರಾಯೋಪವೇಶಮಾಡಿ ಗಂಗೆಯ ಮಧ್ಯದಲ್ಲಿ ಕುಳಿತಿದ್ದ ಪರೀಕ್ಷಿತ ಮಹಾರಾಜನಿಗೆ ಬೋಧಿಸಿದನು. ಆ ಸಮಯದಲ್ಲಿ ನಾನೂ ಅಲ್ಲಿಯೇ ಕುಳಿತಿದ್ದುದರಿಂದ, ಶುಕ ಮುನಿಯ ಅನುಗ್ರಹದಿಂದ ನನಗೆ ಇದು ಲಭ್ಯವಾಯಿತು. ಸಂಸಾರವೆಂಬ ಕಗ್ಗತ್ತಲೆಗೆ ಸಿಕ್ಕಿ, ಜ್ಞಾನವೆಂಬ ಕಣ್ಣು ಕಾಣದಂತಾಗಿರುವ ಜನರ ಉದ್ಧಾರ ಕ್ಕಾಗಿ ಈ ಭಾಗವತವೆಂಬ ಸೂರ್ಯನು ಹುಟ್ಟಿದ್ದಾನೆ’ ಎಂದು ಹೇಳಿದನು.

ಸೂತಪುರಾಣಿಕನ ನುಡಿಗಳಿಂದ ನೆರೆದಿದ್ದ ಪುಷಿಗಳಿಗೆಲ್ಲಾ ಸಂತೋಷವಾಯಿತು. ಅವರು ಭಾಗವತ ಕಥೆಯನ್ನು ಕೇಳಲು ಸಿದ್ಧರಾದರು. ಅವರಲ್ಲಿ ಹಿರಿಯವನಾದ ಶೌನಕ ಪುಷಿಯು ಪುರಾಣಿಕನನ್ನು ಕುರಿತು “ಸೂತಮುನಿ, ಈ ಭಾಗವತ ಕಥೆ ಯಾವಾಗ ಹುಟ್ಟಿತು? ಎಲ್ಲಿ ಹುಟ್ಟಿತು? ಏಕೆ ಹುಟ್ಟಿತು? ವ್ಯಾಸಮುನಿ ಇದನ್ನು ಯಾರ ಪ್ರೇರಣೆಯಿಂದ ರಚಿಸಿದ? ಅದಲ್ಲದೆ ಇನ್ನೊಂದು ಸಂದೇಹ ನನಗೆ. ಶುಕನು ಮಹಾಯೋಗಿ, ಬ್ರಹ್ಮಜ್ಞಾನಿ, ಮಾಯೆಗೆ ಸಿಕ್ಕದ ಮಹಾಮಹಿಮ. ಆತನು ಸಮಚಿತ್ತನೆಂದು ಪ್ರಸಿದ್ಧನಾಗಿದ್ದಾನೆ. ಒಮ್ಮೆ ಆತನು ನಗ್ನನಾಗಿ ಓಡಿಹೋಗುತ್ತಿರುವಾಗ ವ್ಯಾಸರು ಆತನನ್ನು ಹಿಂಬಾಲಿಸಿದರಂತೆ! ದಾರಿಯ ಲ್ಲಿದ್ದ ಒಂದು ಕೊಳದಲ್ಲಿ ಅಪ್ಸರಸ್ತ್ರೀಯರು ಜಲಕ್ರೀಡೆಯಾಡುತ್ತಿದ್ದರಂತೆ, ಯುವಕ ನಾದ ಶುಕಮುನಿಯು ಬೆತ್ತಲೆಯಿದ್ದರೂ ಆ ಹೆಣ್ಣುಗಳು ಆತನನ್ನು ಕಂಡು ನಾಚಿಕೆ ಪಡಲಿಲ್ಲವಂತೆ! ಆದರೆ ಮುದುಕನಾಗಿ ವಸ್ತ್ರಗಳನ್ನು ಧರಿಸಿದ್ದ ವ್ಯಾಸರನ್ನು ಕಂಡು ನಾಚಿಕೆ ಯಿಂದ ಬೇಗ ಬೇಗ ತಮ್ಮ ಬಟ್ಟೆಗಳನ್ನು ಸುತ್ತಿಕೊಂಡರಂತೆ! ವ್ಯಾಸರು ‘ಹೀಗೇಕೆ?’ ಎಂದು ಅವರನ್ನು ಕೇಳಿದಾಗ ‘ಅಯ್ಯಾ ಮಹರ್ಷಿ, ಹೆಣ್ಣುಗಂಡೆಂಬ ಭೇದ ನಿನಗಿದೆ, ಶುಕ ನಿಗಿಲ್ಲ’ ಎಂದರಂತೆ. ಇಂತಹ ಶುಕಮುನಿ ಪರೀಕ್ಷಿದ್ರಾಜನಿಗೆ ಭಾಗವತ ಕಥೆಯನ್ನು ಹೇಳಿದುದು ಹೇಗೆ? ಆತನು ಮನೆಗೆ ಬರುವವನೇ ಅಲ್ಲ; ಬಂದರೂ ಕ್ಷಣಮಾತ್ರ, ಮನೆ ಯನ್ನು ಪವಿತ್ರಗೊಳಿಸಲು ಮಾತ್ರ ನಿಲ್ಲುವವನೇ ಹೊರತು ಭಿಕ್ಷಕ್ಕೆ ಕೂಡ ನಿಲ್ಲುವವ ನಲ್ಲ. ಇಂತಹವನು ರಾಜನ ಹತ್ತಿರಕ್ಕೆ ಬಂದು ನಿಧಾನವಾಗಿ ಈ ಕಥೆಯನ್ನೆಲ್ಲಾ ಹೇಗೆ ಹೇಳಿದ? ಇದಲ್ಲದೆ, ಪರೀಕ್ಷಿದ್ರಾಜ ಮಹಾಶೂರ, ಧರ್ಮಪರ, ಆತನು ಪ್ರಾಯೋಪವೇಶ ಮಾಡಿದುದೇಕೆ?” ಎಂದು ಪ್ರಶ್ನೆ ಮಾಡಿದನು.

ಶೌನಕ ಪುಷಿಯ ಪ್ರಶ್ನೆಗೆ ಉತ್ತರ ಕೊಡುತ್ತಾ ಸೂತ ಪುರಾಣಿಕನು ‘ಅಯ್ಯಾ ಪುಷಿಗಳೆ ಕೇಳಿರಿ, ದ್ವಾಪರಯುಗದ ಕಡೆಯ ಭಾಗದಲ್ಲಿ ಪರಾಶರ ಸತ್ಯವತಿಯರ ಮಗನಾಗಿ ವ್ಯಾಸ ನೆಂಬ ಹೆಸರಿನಿಂದ ಮಹಾವಿಷ್ಣು ಅವತರಿಸಿದನು. ಆತನು ಒಂದು ದಿನ ಸರಸ್ವತೀ ನದಿ ಯಲ್ಲಿ ಸ್ನಾನಮಾಡಿ ಧ್ಯಾನಮಗ್ನನಾಗಿ ಏಕಾಂತದಲ್ಲಿ ಕುಳಿತಿರುವಾಗ ಜಗತ್ತಿನಲ್ಲಿ ನಡೆ ಯುವ ಧರ್ಮಸಾಂಕರ್ಯ, ಜನರಲ್ಲಿ ನೆಲೆಸಿರುವ ನಾಸ್ತಿಕತೆ, ಅಧರ್ಮಬುದ್ಧಿ, ಅವರ ಅಲ್ಪಾಯುಸ್ಸು–ಇವುಗಳ ಕಡೆ ಆತನ ಮನಸ್ಸು ಹರಿಯಿತು. ಆತನು ಲೋಕಹಿತಕ್ಕೆ ಯಾವುದು ಅಗತ್ಯವೆಂಬುದನ್ನು ಆಳವಾಗಿ ಆಲೋಚಿಸಿ, ಒಂದಾಗಿದ್ದ ವೇದವನ್ನು ನಾಲ್ಕು ಭಾಗಗಳಾಗಿ ವಿಂಗಡಿಸಿದನು. ಇದು ಸಾಲದೆಂದು ಐದನೆಯ ವೇದವೆಂದು ಕರೆಯುವ ಮಹಾಭಾರತವನ್ನೂ ಹದಿನೆಂಟು ಪುರಾಣಗಳನ್ನೂ ರಚಿಸಿದನು. ಆದರೂ ಆತನ ಮನಸ್ಸಿಗೆ ಶಾಂತಿಯುಂಟಾಗಲಿಲ್ಲ. ಆತನು ತನ್ನಲ್ಲಿ ತಾನೆ ‘ನಾನು ವೇದಗಳನ್ನು ಕಲಿತೆ, ಗುರುಗಳ ಸೇವೆ ಮಾಡಿದೆ, ಅಗ್ನಿಯನ್ನು ಪೂಜಿಸಿದೆ, ವೇದಗಳನ್ನು ವಿಭಾಗ ಮಾಡಿದುದು ಸಾಲದೆ ಸ್ತ್ರೀ ಶೂದ್ರರೂ ಕೂಡ ಸಕಲ ಧರ್ಮಗಳನ್ನೂ ಅರ್ಥಮಾಡಿಕೊಳ್ಳಲೆಂದು ವೇದದ ಅರ್ಥ ವನ್ನೆಲ್ಲಾ ಭಾರತದಲ್ಲಿ ಪ್ರಕಟಗೊಳಿಸಿದೆ. ಆದರೂ ನನ್ನ ಜೀವ ಶಾಂತಿಗೊಂಡಿಲ್ಲ. ನನಗೆ ಆತ್ಮಜ್ಞಾನ ಬಂದಂತೆ ಕಾಣುವುದಿಲ್ಲ, ಬಹುಶಃ ನಾನು ಭಾಗವತಧರ್ಮವನ್ನು ಚೆನ್ನಾಗಿ ಪ್ರತಿಪಾದಿಸಿಲ್ಲವೊ ಏನೊ!’ ಎಂದು ಚಿಂತಿಸಿದನು. ಆ ವೇಳೆಗೆ ಸರಿಯಾಗಿ ನಾರದ ಮಹರ್ಷಿ ಅಲ್ಲಿ ಪ್ರತ್ಯಕ್ಷನಾದನು. ವ್ಯಾಸಮುನಿಯು ಆತನನ್ನು ಭಕ್ತಿಯಿಂದ ಪೂಜಿಸಿ ಉಪಚರಿಸಿದನು. ಇದರಿಂದ ಸಂತೋಷಗೊಂಡ ನಾರದನು ವ್ಯಾಸಮುನಿಯನ್ನು ಕುರಿತು ‘ಅಯ್ಯಾ, ನೀನು ಸಕಲಧರ್ಮಗಳನ್ನೂ ತಿಳಿದವನು. ನ್ಯಾಯವಾಗಿ ಆತ್ಮಾನಂದವನ್ನು ಅನುಭವಿಸುತ್ತಾ ಸುಖವಾಗಿರಬೇಕು. ಆದರೆ ನಿನ್ನ ಮುಖ ಚಿಂತೆಯನ್ನು ಚೆಲ್ಲುತ್ತಿದೆ. ಇದಕ್ಕೆ ಕಾರಣವೇನು?’ ಎಂದು ಕೇಳಿದನು. ಆಗ ವ್ಯಾಸನು ‘ ಸ್ವಾಮಿ, ನೀವು ಹೇಳುವುದು ನಿಜ. ನನ್ನ ಮನಸ್ಸಿಗೆ ಏಕೋ ಶಾಂತಿಯಿಲ್ಲ. ತ್ರಿಕಾಲಜ್ಞಾನಿಗಳಾದ ನೀವೆ ಇದಕ್ಕೆ ಕಾರಣ ವನ್ನೂ ಪರಿಹಾರವನ್ನೂ ಹೇಳಬೇಕು’ ಎಂದು ಕೇಳಿಕೊಂಡನು. ಆಗ ನಾರದನು ‘ಅಯ್ಯಾ, ನೀನು ವಾಸುದೇವನ ಚರಿತ್ರೆಯನ್ನು ಇನ್ನೂ ಬರೆದಿಲ್ಲವಲ್ಲವೆ? ಆದರಿಂದಲೆ ನಿನಗೆ ಈ ಚಿಂತೆ! ಈಶ್ವರನಿಗೆ ಮೆಚ್ಚಿಗೆಯಾಗದ ಜ್ಞಾನ, ಧರ್ಮ, ಶಾಸ್ತ್ರ–ಇವೆಲ್ಲವೂ ವ್ಯರ್ಥ. ಈಗ ನಿನ್ನ ಮನಸ್ಸು ಲೋಕದ ಜನರ ಉದ್ಧಾರದ ಕಡೆ ತಿರುಗಿರುವುದರಿಂದ ನೀನು ಧನ್ಯ. ನಿನ್ನ ಜ್ಞಾನ ಈಗ ಸಾರ್ಥಕವಾಯಿತು. ನೀನು ಬ್ರಹ್ಮನಿಷ್ಠ, ಸತ್ಯವಾದಿ, ಪವಿತ್ರಾತ್ಮ. ನಿನ್ನ ಯೋಗದೃಷ್ಟಿಯಿಂದ ಭಗವಂತನ ಲೀಲೆಗಳನ್ನೆಲ್ಲಾ ಕಂಡು, ಅವುಗಳನ್ನು ವರ್ಣಿಸುವವ ನಾಗು. ಮಹಾಭಾರತವೇ ಮೊದಲಾದ ಗ್ರಂಥಗಳಲ್ಲಿ ನೀನು ಭಗವಂತನ ಚರಿತ್ರೆಯನ್ನು ಬಣ್ಣಿಸದೆ, ಧರ್ಮವನ್ನು ಮಾತ್ರ ನಿರೂಪಿಸಿದೆ. ಇದರಿಂದಲೆ ನಿನ್ನ ಮನಸ್ಸಿಗೆ ಶಾಂತಿ ಯಿಲ್ಲ’ ಎಂದು ಹೇಳಿದನು.

ನಾರದನು ಭಗವಂತನ ಲೀಲಾವರ್ಣನೆಯ ಮಹತ್ತನ್ನು ವಿಸ್ತಾರವಾಗಿ ವಿವರಿಸಿ, ಅದರ ಕಥನ ಶ್ರವಣಗಳ ಫಲ ಬರಬೇಕಾದರೆ ಸತ್ಸಂಗ ಅತ್ಯಗತ್ಯವೆಂದನು. ಇದಕ್ಕೆ ತಾನೇ ಸಾಕ್ಷಿ ಯೆಂದು ಹೇಳಿ, ತನ್ನ ಪೂರ್ವಜನ್ಮ ವೃತ್ತಾಂತವನ್ನು ಸುವಿಸ್ತಾರವಾಗಿ ನಿರೂಪಿಸಿದನು.

‘ಅಯ್ಯಾ, ಪೂರ್ವಜನ್ಮದಲ್ಲಿ ನಾನೊಬ್ಬ ದಾಸೀಪುತ್ರ. ಚಿಕ್ಕಂದಿನಲ್ಲಿಯೇ ನನ್ನನ್ನು ಒಬ್ಬ ಪುಷಿಗಳ ಸೇವೆಗಾಗಿ ನೇಮಿಸಿದರು. ನಾನು ಭಕ್ತಿಯಿಂದ ಅವರ ಸೇವೆ ಮಾಡು ತ್ತಿರಲು ಪುಷಿಗಳು ಸಂತೋಷಗೊಂಡು, ತಾವು ತಿಂದು ಮಿಕ್ಕುದನ್ನು ನನಗೆ ಕೊಟ್ಟು ಅನುಗ್ರಹಿಸಿದರು. ಅವರ ಈ ಪ್ರಸಾದವನ್ನು ತಿಂದು ನನ್ನ ಪಾಪಗಳೆಲ್ಲವೂ ದೂರ ವಾದುವು; ಮನಸ್ಸು ಭಗವಂತನ ಭಜನೆಯಲ್ಲಿ ಆಸಕ್ತವಾಯಿತು. ಪುಷಿಗಳು ಹೇಳುತ್ತಿದ್ದ ಭಗವಂತನ ಚರಿತ್ರೆಯನ್ನು ಕೇಳಿ ಕೇಳಿ ನನ್ನ ಮನಸ್ಸು ಆನಂದದಿಂದ ತುಂಬಿತು. ನಾನು ಈ ದೇಹವಲ್ಲ, ಪರಬ್ರಹ್ಮನೇ ನಾನು–ಎಂಬ ವಿವೇಕ ಉದಿಸಿತು. ನಾಲ್ಕು ತಿಂಗಳು ಆ ಪುಷಿಗಳ ಸೇವೆಯನ್ನು ಎಡೆಬಿಡದೆ ನಡೆಸುವಷ್ಟರಲ್ಲಿ ನನ್ನ ಭಕ್ತಿ ದೃಢವಾಯಿತು. ನನ್ನ ಈ ಸ್ಥಿತಿಯನ್ನು ಅರ್ಥಮಾಡಿಕೊಂಡ ಪುಷಿಗಳೂ ಬಹು ಸಂತೋಷಪಟ್ಟು, ತಾವು ಅಲ್ಲಿಂದ ಬೇರೊಂದೆಡೆಗೆ ಹೋಗುವ ಮುನ್ನ ಅತ್ಯಂತ ರಹಸ್ಯವಾದ ಜ್ಞಾನಶಾಸ್ತ್ರವನ್ನು ನನಗೆ ಉಪ ದೇಶ ಮಾಡಿದರು. ಇದರ ಪ್ರಭಾವದಿಂದ ನಾನು ಸರ್ವಾಂತರ್ಯಾಮಿಯಾದ ಭಗವಂತನ ಮಾಯಾ ಮಹಿಮೆಗಳನ್ನು ತಿಳಿದುಕೊಂಡೆನು. ಸಂಸಾರಕ್ಕೆ ಕಾರಣವಾದ ಕರ್ಮವು ಭಗ ವಂತನಿಗೆ ಅರ್ಪಿತವಾದಾಗ ತಾಪತ್ರಯಗಳನ್ನೂ ದೂರ ಮಾಡುವುದೆಂಬುದನ್ನು ನಾನು ಅರಿತುಕೊಂಡೆನು. ಫಲಾಪೇಕ್ಷೆಯಿಂದ ಕರ್ಮಗಳನ್ನು ಮಾಡಿದರೆ ಆ ಫಲಗಳನ್ನು ಅನು ಭವಿಸುವುದಕ್ಕಾಗಿ ಜನ್ಮವೆತ್ತಬೇಕು. ಹಾಗೆ ಜನ್ಮವೆತ್ತಿದ ಮೇಲೆ ಸಂಸಾರ ತಪ್ಪದು. ಆದ್ದ ರಿಂದ ಕರ್ಮದಲ್ಲಿ ಫಲಾಪೇಕ್ಷೆಯಿಡದೆ ಈಶ್ವರಾರ್ಪಿತವಾಗಿ ಅದನ್ನು ಮಾಡಬೇಕು. ಇದರಿಂದ ಈಶ್ವರನು ಸುಪ್ರೀತನಾಗುವನು. ಆತನ ಕೃಪೆಯಿಂದ ಆತ್ಮಜ್ಞಾನ ದೊರೆಯು ವುದು. ಹೀಗೆ ಭಕ್ತಿಯೋಗದಿಂದ ಜ್ಞಾನವೂ, ಜ್ಞಾನದಿಂದ ಕರ್ಮನಾಶವೂ ಉಂಟಾಗು ವುದು. ಭಗವಂತನಿಗೆ ಅರ್ಪಿಸಿ ಮಾಡುವ ಕರ್ಮವೇ ಭಕ್ತಿಗೆ ಕಾರಣವಾದುದರಿಂದ ಕರ್ಮವೇ ಕರ್ಮನಾಶಕ್ಕೆ ಕಾರಣವಾದಂತಾಗುತ್ತದೆ. ನನಗೆ ಹಾಗೆಯೇ ಆಯಿತು. ಫಲಾ ಪೇಕ್ಷೆಯಿಲ್ಲದೆ ಕರ್ಮಮಾಡಿದ ನನಗೆ ಭಗವಂತನು ಭಕ್ತಿಯೋಗವನ್ನು ಕರುಣಿಸಿದನು. ಅಷ್ಟರಲ್ಲಿ ನನ್ನ ತಾಯಿ ಹಾವು ಕಡಿದು ಸತ್ತಳು. ಇದರಿಂದ ದಿಕ್ಕಿಲ್ಲದವನಾದ ನಾನು ಉತ್ತರ ದಿಕ್ಕನ್ನು ಹಿಡಿದು ಹೊರಟು, ಅನೇಕ ಬೆಟ್ಟ ನದಿ ಊರುಗಳನ್ನು ಕಳೆದು ದಟ್ಟವಾದ ಒಂದು ಅರಣ್ಯಕ್ಕೆ ಬಂದೆನು. ಅಲ್ಲಿ ಹರಿಯುತ್ತಿದ್ದ ನದಿಯಲ್ಲಿ ಸ್ನಾನಮಾಡಿ, ನೀರನ್ನು ಕುಡಿದು, ಅಲ್ಲಿಯೇ ಇದ್ದ ಒಂದು ಅರಳಿಯ ಮರದ ಕೆಳಗೆ ಪುಷಿಗಳು ಬೋಧಿಸಿದ್ದ ಮಂತ್ರವನ್ನು ಜಪಿಸುತ್ತ ಕುಳಿತೆನು. ಕೆಲವು ಕಾಲದಮೇಲೆ ನನ್ನ ಹೃದಯದಲ್ಲಿ ಶ್ರೀಹರಿ ಗೋಚರಿಸಿದನು. ನನ್ನ ಆನಂದಕ್ಕೆ ಪಾರವಿಲ್ಲದಂತಾಯಿತು. ಆದರೆ ಕ್ಷಣ ಕಾಲವಾಗುತ್ತಲೆ ಆತನು ಕಣ್ಮರೆಯಾದನು. ನನಗೆ ಬಹು ಗಾಬರಿಯಾಯಿತು. ದಿಗ್ಗನೆ ಮೇಲಕ್ಕೆದ್ದು ಸುತ್ತಲೂ ನೋಡಿದೆ. ಮತ್ತೆ ಮತ್ತೆ ಧ್ಯಾನ ಮಾಡಿದರೂ ಆತನ ಮೂರ್ತಿ ಗೋಚರಿಸಲಿಲ್ಲ. ನಾನು ಅಶಾಂತನಾಗಿ ಚಡಪಡಿಸುತ್ತಿರಲು, ಆಕಾಶವಾಣಿಯೊಂದು ಕೇಳಿಬಂತು–‘ಮಗು, ನಿನ್ನ ಭಕ್ತಿ ಅಚಲವಾದುದು, ನೀನಿನ್ನು ಈ ಲೋಕವನ್ನು ಬಿಟ್ಟು, ನನ್ನ ಸಾಮೀಪ್ಯ ಮುಕ್ತಿ ಯನ್ನು ಪಡೆಯುವವನಾಗು. ಸೃಷ್ಟಿ ಸ್ಥಿತಿ ಲಯ ಮೂರರಲ್ಲಿಯೂ ನಿನ್ನ ಸ್ಮೃತಿ ಸ್ಥಿರ ವಾಗಲಿ.’ ನಾನು ಆ ಧ್ವನಿ ಬಂದ ದಿಕ್ಕಿಗೆ ನಮಸ್ಕರಿಸಿ, ಭಗವಂತನ ನಾಮಗಳನ್ನು ಹಾಡುತ್ತ, ಸಂಚರಿಸುತ್ತಿದ್ದು, ದೇಹಾವಸಾನದ ಕಾಲವನ್ನು ಕಾಯುತ್ತಿದ್ದೆ. ಒಂದು ದಿನ ನನ್ನ ಭೌತಿಕ ದೇಹ ಬಿದ್ದು ಹೋಯಿತು. ನನ್ನ ಜೀವ ಬ್ರಹ್ಮನಲ್ಲಿ ಐಕ್ಯವಾಯಿತು. ನಾಲ್ಕು ಸಾವಿರ ಯುಗಗಳಾದ ಮೇಲೆ ಬ್ರಹ್ಮದೇವನ ಪ್ರಾಣವಾಯುಗಳಿಂದ ನಾನೂ ಮರೀಚಿಯೇ ಮೊದಲಾದ ನವಬ್ರಹ್ಮರೂ ಜನಿಸಿದೆವು. ಅಂದಿನಿಂದ ನಾನು ಬ್ರಹ್ಮಚಾರಿಯಾಗಿ ಹರಿ ಚರಿತ್ರೆಯನ್ನು ಗಾನಮಾಡುತ್ತ ಮೂರು ಲೋಕಗಳಲ್ಲಿಯೂ ಸಂಚರಿಸುತ್ತಿರುವೆನು. ನಾನು ನೆನೆದೊಡನೆಯೇ ಆ ಶ್ರೀಹರಿ ನನ್ನ ಹೃದಯದಲ್ಲಿ ಗೋಚರನಾಗುವನು. ಅಯ್ಯಾ ವ್ಯಾಸ ಮುನಿ, ನಾನು ಮೊದಲು ಹೇಳಿದಂತೆ ನೀನು ಶ್ರೀಹರಿಯ ಲೀಲೆಗಳನ್ನು ವರ್ಣಿಸಿ, ಲೋಕದ ಜನರನ್ನು ಉದ್ಧರಿಸು. ನಾನಿನ್ನು ಹೋಗಿಬರುವೆನು’ ಎಂದು ಹೇಳಿ, ವೀಣೆಯಲ್ಲಿ ಭಗವಂತನ ನಾಮವನ್ನು ನುಡಿಸುತ್ತ ಗಾಳಿಯಲ್ಲಿ ತೇಲಿಹೋದನು.

ನಾರದನು ಅತ್ತಹೋಗುತ್ತಲೇ, ಇತ್ತ ವ್ಯಾಸಮಹರ್ಷಿಯು ಸರಸ್ವತೀ ನದಿಯಲ್ಲಿ ಮಿಂದು, ಸ್ನಾನಸಂಧ್ಯೆಗಳನ್ನು ನೆರವೇರಿಸಿ, ಭಕ್ತಿಯಿಂದ ಭಗವಂತನನ್ನು ಧ್ಯಾನ ಮಾಡುತ್ತಾ ಕುಳಿತನು. ಆತನ ಹೃದಯ ಕಮಲದಲ್ಲಿ ವಿಷ್ಣುವು ಪ್ರತ್ಯಕ್ಷನಾದನು. ಆತನ ಜೊತೆಯಲ್ಲಿ ಮಾಯೆಯೂ ಭಕ್ತಿ ಯೋಗವೂ ಪ್ರತ್ಯಕ್ಷವಾದುವು. ಹೀಗೆ ಹೃದಯದಲ್ಲಿ ಅವತರಿಸಿದ ಭಕ್ತಿಯೋಗವನ್ನು ಕಣ್ಣಾರೆ ಕಂಡು ಸಂತುಷ್ಟಗೊಂಡ ವ್ಯಾಸಮಹರ್ಷಿಯು ಭವಬಂಧವನ್ನು ನೀಗುವ ‘ಭಾಗವತ’ವೆಂಬ ಭಕ್ತಿಶಾಸ್ತ್ರವನ್ನು ರಚಿಸಿ, ಅದನ್ನು ವೈರಾಗ್ಯಮೂರ್ತಿಯಾದ ತನ್ನ ಮಗ ಶುಕನಿಗೆ ಬೋಧಿಸಿದನು. ಭಗವದ್ಭಕ್ತರಲ್ಲಿ ಅಪಾರ ಪ್ರೀತಿಯುಳ್ಳ ಶುಕಮುನಿಯು ತಾನು ಪರಮಜ್ಞಾನಿಯಾದರೂ ಭಗವಂತನ ಗುಣ ಗಳಲ್ಲಿಯೇ ನೆಟ್ಟಮನಸ್ಸುಳ್ಳವನಾಗಿ ಇದನ್ನು ಆದ್ಯಂತವಾಗಿ ಕಲಿತನು ಎಂಬುದಾಗಿ ಸೂತಮಹರ್ಷಿಯು ಭಾಗವತದ ಅವತರಣವನ್ನು ವಿವರಿಸಿ ಹೇಳಿದನು.

