
\chapter{೬೯. ಹೆಣ್ಣಿನ ಮೇಲೆ ಹೆಣ್ಣು}

ಪಂಚಪಾಂಡವರು ಅರಗಿನ ಮನೆಯಲ್ಲಿ ಬೆಂದುಹೋದರೆಂದು ಶ್ರೀಕೃಷ್ಣ ಅವರಿಗೆ ಅಪರ ಕರ್ಮಗಳನ್ನು ಮಾಡಿದನಷ್ಟೆ! ಆದರೆ ಅವರು ಸಾಯಲಿಲ್ಲ; ಕೆಲಕಾಲ ತಲೆಮರೆಸಿ ಕೊಂಡಿದ್ದು, ದ್ರೌಪದಿಯ ಸ್ವಯಂವರ ಕಾಲದಲ್ಲಿ ಕಾಣಿಸಿಕೊಂಡರು. ಅವರು ದ್ರೌಪದಿ ಯನ್ನು ಮದುವೆಯಾಗಿ, ಎಂದಿನಂತೆ ಇಂದ್ರಪ್ರಸ್ಥಪುರದಲ್ಲಿ ರಾಜ್ಯಭಾರಮಾಡು ತ್ತಿದ್ದರು. ಈ ಸುದ್ದಿಯನ್ನು ಕೇಳಿದ ಶ್ರೀಕೃಷ್ಣನು ಅತ್ಯಂತ ಸಂತೋಷದಿಂದ ಅವರನ್ನು ನೋಡಿಕೊಂಡು ಬರಲೆಂದು ಇಂದ್ರಪ್ರಸ್ಥಕ್ಕೆ ಬಂದನು. ತಮ್ಮ ಪಂಚಪ್ರಾಣದಂತಿದ್ದ ಆತನನ್ನು ಕಾಣುತ್ತಲೆ ಪಂಚಪಾಂಡವರು ಅತ್ಯಂತ ಸಂಭ್ರಮದಿಂದ ಆತನನ್ನು ಇದಿರು ಗೊಂಡು ಆನಂದದಿಂದ ಆಲಿಂಗಿಸಿಕೊಂಡರು. ಕಪಟನಾಟಕ ಸೂತ್ರಧಾರಿಯಾದ ಶ್ರೀಕೃಷ್ಣ ಆ ಪಂಚಪಾಂಡವರಲ್ಲಿ ತನಗಿಂತಲೂ ಹಿರಿಯರಾದ ಧರ್ಮ ಭೀಮರಿಗೆ ನಮಸ್ಕರಿಸಿದನು. ಸಮವಯಸ್ಕನಾದ ಅರ್ಜುನನನ್ನು ಅಪ್ಪಿಕೊಂಡನು. ಕಿರಿಯರಾದ ನಕುಲ ಸಹದೇವರಿಂದ ನಮಸ್ಕಾರಗೊಂಡು ಅವರನ್ನು ಆಶೀರ್ವದಿಸಿದನು. ಹೊಸದಾಗಿ ಅತ್ತೆಯ ಮನೆಗೆ ಬಂದಿದ್ದ ದ್ರೌಪದಿ ನಾಚಿಕೆಯಿಂದ ಬಳುಕುತ್ತಾ ಬಂದು ಆತನಿಗೆ ನಮಸ್ಕರಿಸಿದಳು. ಶ್ರೀಕೃಷ್ಣನು ಹಾಸ್ಯಭರಿತ ನುಡಿಗಳಿಂದ ಆಕೆಯನ್ನು ನಗಿಸಿ, ತಾನೂ ನಗುತ್ತಾ ತನ್ನ ಸೋದರತ್ತೆಯಾದ ಕುಂತಿಯ ಮುಂದೆ ಅಡ್ಡಬಿದ್ದನು. ಆಕೆ ಆನಂದಬಾಷ್ಪ ಗಳನ್ನು ಸುರಿಸುತ್ತಾ ತನ್ನ ತೌರಿನವರೆಲ್ಲರ ಯೋಗಕ್ಷೇಮವನ್ನೂ ವಿಚಾರಿಸಿ ‘ಅಪ್ಪಾ ಶ್ರೀಕೃಷ್ಣ, ನೀನೆ ನಮಗೆಲ್ಲ ದಿಕ್ಕು; ನಿನ್ನ ಭಾವ ಮೈದುನರನ್ನು ಕಾಪಾಡುವ ಭಾರ ನಿನ್ನದು’ ಎಂದಳು. ಅಲ್ಲಿಯೆ ನಿಂತಿದ್ದ ಧರ್ಮರಾಯನು ‘ನಾವು ಪೂರ್ವಜನ್ಮದಲ್ಲಿ ಏನು ಪುಣ್ಯ ಮಾಡಿದ್ದೆವೊ, ನೀನು ಬಂಧುವಾಗಿ ನಮಗೆ ದೊರೆತೆ. ಶ್ರೀಕೃಷ್ಣ, ಹೇಗಿದ್ದರೂ ಈಗ ಮಳೆ ಗಾಲ. ಈ ಮಳೆಗಾಲ ಮುಗಿಯುವವರೆಗೂ ನೀನು ಇಲ್ಲಿಯೆ ನಿಲ್ಲಬೇಕು’ ಎಂದು ಬೇಡಿ ಕೊಂಡ. ಶ್ರೀಕೃಷ್ಣನು ‘ಹಾಗೆಯೆ ಆಗಲಿ’ ಎಂದು ನಾಲ್ಕು ತಿಂಗಳು ಅಲ್ಲಿಯೇ ನಿಂತ.

ಶ್ರೀಕೃಷ್ಣನು ಇಂದ್ರಪ್ರಸ್ಥದಲ್ಲಿ ಕಾಲವನ್ನು ವ್ಯರ್ಥವಾಗಿ ಕಳೆಯಲಿಲ್ಲ. ಒಂದೊಂದು ದಿನ ಒಂದೊಂದು ಕಾರ್ಯಕ್ರಮ ಆತನಿಗೆ. ಪಾಂಡವರು ತಲೆಮರೆಸಿಕೊಂಡು ಹೋಗಿದ್ದಾಗ ಇಂದ್ರಪ್ರಸ್ಥಪುರ ಹಾಳುಬಿದ್ದು ಹೋಗಿತ್ತು. ಶ್ರೀಕೃಷ್ಣನು ತನ್ನ ಶಿಲ್ಪಿಯಾದ ವಿಶ್ವಕರ್ಮ ನಿಂದ ಆ ಊರನ್ನು ಚಿತ್ರವಿಚಿತ್ರವಾಗಿಯೂ ಮನೋಹರವಾಗಿಯೂ ಇರುವಂತೆ ಹೊಸ ದಾಗಿ ನಿರ್ಮಾಣ ಮಾಡಿಸಿದನು. ಸಮಾನ ವಯಸ್ಸಿನವನಾದ ಅರ್ಜುನನೊಡನೆ ಶ್ರೀ ಕೃಷ್ಣನು ದಿನವೂ ವಿಹಾರಕ್ಕೆ ಹೊರಡುವನು. ಒಂದು ದಿನ ಹಾಗೆ ಹೋಗಿದ್ದಾಗ ಅಗ್ನಿಯು ಅವರ ಬಳಿಗೆ ಬಂದು ಖಾಂಡವವನವನ್ನು ತನಗೆ ಆಹಾರವಾಗಿ ನೀಡಬೇಕೆಂದು ಬೇಡಿ ದನು. ಶ್ರೀಕೃಷ್ಣನ ಬೆಂಬಲದಿಂದ ಅರ್ಜುನನು ಅದಕ್ಕೆ ಒಪ್ಪಿದನು. ಆ ವನವನ್ನು ತಿಂದು ಸಂತುಷ್ಟನಾದ ಅಗ್ನಿಯು ಅರ್ಜುನನಿಗೆ ದಿವ್ಯವಾದ ಒಂದು ರಥ, ಸೊಗಸಾದ ನಾಲ್ಕು ಕುದುರೆಗಳು, ಗಾಂಡೀವವೆಂಬ ಬಿಲ್ಲು, ಅರಕೆಯಾಗದ ಎರಡು ಬತ್ತಳಿಕೆಗಳು, ಒಂದು ವಜ್ರ ಕವಚ–ಇವುಗಳನ್ನು ಅನುಗ್ರಹಿಸಿದನು. ಆ ಖಾಂಡವವನ ದಹನಕಾಲದಲ್ಲಿ ಅರ್ಜುನನು ಮಯನೆಂಬ ಶಿಲ್ಪಿಗೆ ಪ್ರಾಣದಾನ ಮಾಡಿದುದರಿಂದ ಆತನು ಪಾಂಡವರಿ ಗಾಗಿ ಸೊಗಸಾದ ಸಭಾಭವನವನ್ನು ನಿರ್ಮಿಸಿಕೊಟ್ಟನು. ಈ ಭವನವನ್ನು ನೋಡಲೆಂದು ಬಂದ ದುರ್ಯೋಧನನು ನೆಲವನ್ನು ನೀರೆಂದೂ, ನೀರನ್ನು ನೆಲವೆಂದೂ ಭ್ರಮಿಸಿ ಎಲ್ಲರ ನಗೆಪಾಟಲಿಗೆ ಗುರಿಯಾದ.

ಒಂದು ದಿನ ಕೃಷ್ಣಾರ್ಜುನರು ಬೇಟೆಗೆಂದು ಅಡವಿಗೆ ಹೋದರು. ಮಧ್ಯಾಹ್ನದ ವೇಳೆಗೆ ಅವರಿಬ್ಬರೂ ಸೇರಿ ಅನೇಕ ದುಷ್ಟಮೃಗಗಳನ್ನು ಹೊಡೆದುಹಾಕಿದರು. ಬಿಸಿಲಿನ ಬೇಗೆಯಿಂದ ಅವರಿಗೆ ತುಂಬ ಬಾಯಾರಿಕೆಯಾಯಿತು. ಇಬ್ಬರೂ ಯಮುನಾ ನದಿಗೆ ಬಂದು ಅಲ್ಲಿ ನೀರನ್ನು ಕುಡಿಯುತ್ತಿದ್ದರು. ಅವರ ಇದಿರಿನಲ್ಲಿ ದಿವ್ಯಸುಂದರಿಯಾದ ಒಬ್ಬ ಹುಡುಗಿ ಕಾಣಿಸಿದಳು. ಅವಳಾರೆಂದು ವಿಚಾರಿಸುವಂತೆ ಶ್ರೀಕೃಷ್ಣ ಅರ್ಜುನನಿಗೆ ಹೇಳಿದ. ಅರ್ಜುನ ಅವಳನ್ನು ಕುರಿತು ‘ಎಲೆ ಬಾಲೆ, ನೀನಾರು? ಎಲ್ಲಿಂದ ಬಂದೆ? ನೀನೊಬ್ಬಳೆ ಈ ನದಿಯಲ್ಲಿ ಏಕೆ ನಿಂತಿರುವೆ?’ ಎಂದು ಕೇಳಿದ. ಆ ಬಾಲೆ ಹೇಳಿದಳು–‘ಅಯ್ಯಾ, ನಾನು ಸೂರ್ಯನ ಮಗಳು, ನನ್ನ ಹೆಸರು ಕಾಳಿಂದಿ. ನಮ್ಮ ತಂದೆ ಈ ನದಿಯ ತಳದಲ್ಲಿ ನನಗೊಂದು ಅರಮನೆಯನ್ನು ನಿರ್ಮಿಸಿಕೊಟ್ಟಿದ್ದಾನೆ. ನಾನು ಅದರಲ್ಲಿ ವಾಸಿಸುತ್ತಾ, ಶ್ರೀಹರಿಯೆ ನನ್ನ ಪತಿಯಾಗಬೇಕೆಂದು ತಪಸ್ಸು ಮಾಡುತ್ತಿದ್ದೇನೆ’ ಎಂದಳು. ಅರ್ಜುನನು ಈ ವಿಚಾರವನ್ನು ಶ್ರೀಕೃಷ್ಣನಿಗೆ ತಿಳಿಸಲು, ಆತನು ಆಕೆಯನ್ನು ತನ್ನೊಡನೆ ರಥದಲ್ಲಿ ಕುಳ್ಳಿರಿಸಿಕೊಂಡು ಬಂದು ಧರ್ಮರಾಜನ ಬಳಿಯಲ್ಲಿ ಬಿಟ್ಟನು. ಮಳೆಗಾಲ ಕಳೆಯುತ್ತಲೆ ಆತನು ಕಾಳಿಂದಿಯೊಡನೆ ದ್ವಾರಕಿಗೆ ಹಿಂದಿರುಗಿ, ಒಂದು ಶುಭದಿನ ಶುಭಮುಹೂರ್ತ ದಲ್ಲಿ ವೈಭವದಿಂದ ಆಕೆಯನ್ನು ವಿವಾಹವಾದನು.

ವಿಂದ, ಅನುವಿಂದ ಎಂಬ ಸೋದರರು ಅವಂತಿದೇಶದ ದೊರೆಗಳು. ಅವರಿಗೆ ‘ಮಿತ್ರವಿಂದೆ’ ಎಂಬ ಸುಂದರಿಯಾದ ತಂಗಿಯಿದ್ದಳು. ವಿಂದಾನುವಿಂದರು ಆಕೆಗೆ ಮದುವೆ ಮಾಡಬೇಕೆಂದು ನಿಶ್ಚಯಿಸಿ, ಸ್ವಯಂವರವನ್ನು ಏರ್ಪಡಿಸಿದರು. ಆಕೆಯನ್ನು ಕೈಹಿಡಿಯಬೇಕೆಂಬ ಆಶೆಯಿಂದ ಅನೇಕ ರಾಜರು ಆ ಸ್ವಯಂವರಕ್ಕೆ ಬಂದಿದ್ದರು. ಶ್ರೀಕೃಷ್ಣನೂ ಅವರಲ್ಲಿ ಒಬ್ಬ. ಮಿತ್ರವಿಂದೆ ಶ್ರೀಕೃಷ್ಣನ ಕೀರ್ತಿಯನ್ನು ಕೇಳಿ, ಮನಸಾ ಆತನನ್ನು ವರಿಸಿದ್ದಳು. ಆದರೆ ಸ್ವಯಂವರ ಮಂಟಪದಲ್ಲಿ ಆಕೆ ಶ್ರೀಕೃಷ್ಣನನ್ನು ವರಿಸಹೊರಡಲು, ಆಕೆಯ ಸೋದರರು ಆಕೆಯನ್ನು ತಡೆದರು. ಅವರು ದುರ್ಯೋಧನನ ಗೆಳೆಯರು, ಶ್ರೀಕೃಷ್ಣನ ದ್ವೇಷಿಗಳು. ಶ್ರೀಕೃಷ್ಣನು ಸ್ವಯಂವರ ಮಂಟಪದಲ್ಲಿ ನಡೆದ ಈ ರಸಾಭಾಸವನ್ನು ಕಂಡು ಕೆರಳಿದನು. ಎಲ್ಲ ರಾಜರೂ ನೋಡುತ್ತಿರುವಂತೆಯೆ ಆತನು ಮಿತ್ರವಿಂದೆಯನ್ನು ಎತ್ತಿ ರಥದಲ್ಲಿ ಕೂಡಿಸಿಕೊಂಡು ದ್ವಾರಕಿಗೆ ಹೊರಟು ಬಂದನು. ಅಲ್ಲಿ ಅವರಿಬ್ಬರಿಗೂ ವಿಧಿಪೂರ್ವಕವಾಗಿ ವಿವಾಹವಾಯಿತು.

ಕೋಸಲದೇಶದ ಅಯೋಧ್ಯಾಪಟ್ಟಣದಲ್ಲಿ ನಗ್ನಜಿತ್ ಎಂಬ ರಾಜನು ರಾಜ್ಯಭಾರ ಮಾಡುತ್ತಿದ್ದನು. ಆತನಿಗೆ ಸುಂದರಿಯಾದ ‘ಸತ್ಯೆ’ ಎಂಬ ಮಗಳಿದ್ದಳು. ಆಕೆಯನ್ನು ವಿವಾಹವಾಗಬೇಕಾದರೆ ಆ ರಾಜನ ಬಳಿ ಇದ್ದ ಏಳು ಮದಿಸಿದ ಗೂಳಿಗಳನ್ನು ಹಿಡಿದು ಅವುಗಳಿಗೆ ಮೂಗುದಾರ ಹಾಕಬೇಕಾಗಿತ್ತು. ಎಷ್ಟೋ ರಾಜಕುಮಾರರು ಆ ಕೆಲಸ ಮಾಡ ಹೊರಟು, ಆ ಗೂಳಿಗಳ ಕೊಂಬಿನ ಮೊನೆಗೆ ಆಹುತಿಯಾಗಿದ್ದರು. ಈ ಸಂಗತಿ ಶ್ರೀಕೃಷ್ಣ ನಿಗೆ ತಿಳಿಯಿತು. ಆತನು ಅಯೋಧ್ಯೆಗೆ ಹೋಗಿ ನಗ್ನಜಿತ್ ರಾಜನನ್ನು ಕಂಡನು. ರಾಜನು ಶ್ರೀಕೃಷ್ಣನನ್ನು ಅತ್ಯಂತ ಭಕ್ತಿಯಿಂದ ಸತ್ಕರಿಸಿದನು. ಆತನ ಮಗಳು ಶ್ರೀಕೃಷ್ಣನನ್ನು ಕಾಣುತ್ತಲೆ ಆತನಲ್ಲಿ ಮೋಹಗೊಂಡಳು. “ಈತನೆ ನನಗೆ ಪತಿಯಾಗಲಿ” ಎಂದು ಆಕೆ ದೇವನನ್ನು ಬೇಡಿಕೊಂಡಳು. ಶ್ರೀಕೃಷ್ಣನು ಆಕೆಯನ್ನು ವಿವಾಹವಾಗಬೇಕೆಂಬ ತನ್ನ ಬಯಕೆಯನ್ನು ರಾಜನಿಗೆ ತಿಳಿಸಿದನು. ನಗ್ನಜಿತ್ತನು ಸಂತೋಷದಿಂದ ಅದಕ್ಕೆ ಒಪ್ಪಿದನು. ‘ಸ್ವಾಮಿ, ದೇವ ದೇವನಾದ ನಿನಗಿಂತಲೂ ಉತ್ತಮನಾದ ಗಂಡ ನನ್ನ ಮಗಳಿಗೆಲ್ಲಿ ಸಿಕ್ಕಾನು? ಆದರೆ ನನ್ನ ನಿಯಮದಂತೆ ಆ ಗೂಳಿಗಳನ್ನು ನೀನು ನಿಗ್ರಹಿಸಬೇಕಲ್ಲಾ!’ ಎಂದನು. ಒಡನೆಯೆ ಶ್ರೀಕೃಷ್ಣನು ನಡುಕಟ್ಟನ್ನು ಬಿಗಿದು ಆ ಗೂಳಿಗಳೊಡನೆ ಸೆಣಸಲು ಸಿದ್ಧನಾದನು. ಒಬ್ಬ ಕೃಷ್ಣ ಏಳು ಕೃಷ್ಣರಾದರು. ಅವರು ಆಟವಾಡುತ್ತಾ ಆ ಏಳು ಗೂಳಿ ಗಳನ್ನೂ ಕಟ್ಟಿ, ಅವುಗಳಿಗೆ ಮೂಗುದಾರವನ್ನು ಹಾಕಿದರು. ಆತನ ಸಾಹಸವನ್ನು ಕಂಡು ಸಂತೋಷಿಸಿದ ನಗ್ನಜಿತ್ತನು ತನ್ನ ಮಗಳನ್ನು ಆತನಿಗೆ ವೈಭವದಿಂದ ವಿವಾಹಮಾಡಿ ಕೊಟ್ಟು ಮಗಳಿಗೆ ಬೇಕಾದಷ್ಟು ಬಳುವಳಿಗಳನ್ನು ಕೊಟ್ಟನು.

ಶ್ರೀಕೃಷ್ಣನು ಸತ್ಯೆಯೊಡನೆ ದ್ವಾರಕಿಗೆ ಹಿಂದಿರುಗುವಷ್ಟರಲ್ಲಿ ಆತನನ್ನು ಕೈಹಿಡಿವ ಮತ್ತೊಂದು ಹೆಣ್ಣು ಸಿದ್ಧವಾಗಿತ್ತು. ಆತನ ಸೋದರತ್ತೆಯಾದ ಶ್ರುತಕೀರ್ತಿಯೆಂಬ ಕೇಕಯರಾಜನ ಮಡದಿಯು ತನ್ನ ಮಗಳಾದ ಭದ್ರೆಯೆಂಬುವಳನ್ನು ಆತನಿಗೆ ಕೊಟ್ಟು ವಿವಾಹಮಾಡಿದಳು. ಅದರ ಹಿಂದೆಯೆ ಮದ್ರರಾಜನ ಮಗಳಾದ ಲಕ್ಷಣೆಯನ್ನು ಶ್ರೀಕೃಷ್ಣ ಗರುಡನು ಅಮೃತವನ್ನು ಹೊತ್ತು ತಂದಂತೆ ಅಪಹರಿಸಿ ತಂದು, ಮದುವೆಯಾದನು. ಹೆಣ್ಣಿನ ಮೇಲೆ ಹೆಣ್ಣು ಬಂದು ಸೇರಿ ಶ್ರೀಕೃಷ್ಣನ ಅಂತಃಪುರ ಶ್ರೀಮಂತವಾಯಿತು.

