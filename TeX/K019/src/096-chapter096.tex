
\chapter{೯೬. ಮಾರ್ಕಂಡೇಯ ಮಹರ್ಷಿ}

ಭಾಗವತ ಕಥೆ ಮತ್ತು ಅದನ್ನು ಕೇಳುತ್ತಿದ್ದ ಪರೀಕ್ಷಿತನ ಕಥೆ–ಎರಡೂ ಮುಗಿದಂ ತಾಯಿತು. ಅವೆರಡನ್ನೂ ಹೇಳುತ್ತಿರುವವನು ಸೂತಪುರಾಣಿಕ, ಅದನ್ನು ಕೇಳುತ್ತಿರುವ ಶೌನಕಾದಿ ಮಹರ್ಷಿಗಳು. ಅಗೋ, ಶೌನಕ ಮಹರ್ಷಿ ಋಷಿಗಳೆಲ್ಲರ ಪರವಾಗಿ ಸೂತ ಪುರಾಣಿಕನನ್ನು ಏನೋ ಕೇಳುತ್ತಿದ್ದಾನೆ, ಅದೇನೆಂದು ತಿಳಿಯೋಣ. ‘ಅಯ್ಯಾ ಸೂತ, ನೀನು ಚಿರಂಜೀವಿಯಾಗಿ ಬಾಳು. ನೀನು ಮಾತುಬಲ್ಲವನಯ್ಯ. ನಿನ್ನ ಮಾತೆಂದರೆ ಅಮೃತ–ಕಿವಿಗೆ ಇಂಪು, ಜೀವಕ್ಕೆ ತಂಪು. ನಿನ್ನ ಬಾಯಿಂದ ಮೃಕಂಡು ಋಷಿಯ ಮಗನಾದ ಮಾರ್ಕಂಡೇಯನ ಕಥೆಯನ್ನು ಕೇಳಬೇಕೆಂದು ನಮಗೆ ಆಶೆ. ಆತ ಹರಕೃಪೆ ಯಿಂದ ಚಿರಂಜೀವಿಯಾದನಂತೆ! ಆ ಕಥೆಯನ್ನು ನಮಗೆ ಹೇಳು’ಎಂದನು. ಸೂತ ಪುರಾಣಿಕ ಅವರ ಇಷ್ಟಾರ್ಥ ಸಲ್ಲಿಸಿದ:

ಹಿಮಾಲಯದ ಉತ್ತರಪಾರ್ಶ್ವದಲ್ಲಿ ಪುಷ್ಪಭದ್ರೆ ನದಿಯು ಸಜ್ಜನರ ಮನಸ್ಸಿನಂತೆ ಶುಭ್ರ ಧವಳವಾದ ತಿಳಿನೀರಿನಿಂದ ಓಂಕಾರವನ್ನು ಉಚ್ಚರಿಸುತ್ತಾ ಹರಿಯುತ್ತಿದೆ. ಅದರ ದಡದಲ್ಲಿ ಸುಂದರವಾದ ಒಂದು ಉಪವನವು ಹಣ್ಣು ಹೂಗಳಿಂದ ಕೂಡಿದ ಗಿಡ ಮರ ಬಳ್ಳಿಗಳಿಂದ ತುಂಬಿ, ಕಣ್ಮನಗಳಿಗೆ ತಂಪನ್ನು ನೀಡುತ್ತದೆ. ಅಲ್ಲಿ ದುಂಬಿಯ ಜೇಂಕಾರದ ಶ್ರುತಿಯೊಡನೆ ಗಿಳಿ, ಕೋಗಿಲೆ ಮೊದಲಾದ ಹಕ್ಕಿಗಳು ಇಂಪಾಗಿ ಹಾಡು ತ್ತವೆ; ನವಿಲುಗಳು ನರ್ತಿಸುತ್ತವೆ; ಕಪಿ, ಕಡವೆ, ಜಿಂಕೆ ಮೊದಲಾದ ಪ್ರಾಣಿಗಳು ಪ್ರೇಕ್ಷಕ ರಾಗಿ ಸಂತಸಪಡುತ್ತವೆ. ಅಲ್ಲಿನ ಮೃಗಗಳಲ್ಲಿ ಪರಸ್ಪರ ವೈರವೆಂಬುದೆ ಇಲ್ಲ. ಇದು ನಿಸ್ಸಂಶಯವಾಗಿಯೂ ಒಂದು ಪುಷ್ಯಾಶ್ರಮ. ಅಗೋ, ಆ ಎತ್ತರವಾದ ಬಂಡೆಯ ನೆತ್ತಿಯ ಮೇಲೆ ಧ್ಯಾನಸ್ತಿಮಿತ ಮೂರ್ತಿಯೊಂದು ಕಾಣಬರುತ್ತಿದೆ. ತಲೆಯಲ್ಲಿ ಜಟೆ, ಮೈ ಯಲ್ಲಿ ವಲ್ಕಲ, ಎದೆಯಲ್ಲಿ ಯಜ್ಞೋಪವೀತ, ಸೊಂಟದಲ್ಲಿ ಮೌಂಜಿ, ಹತ್ತಿರದಲ್ಲಿಯೆ ದಂಡಕಮಂಡಲಗಳು! ಮಹಾತೇಜಸ್ವಿಯಾಗಿರುವ ಆ ಮಹರ್ಷಿಯ ಬಳಿಗೆ ಮೆಲ್ಲ ಮೆಲ್ಲಗೆ ಇಳಿದು ಬರುತ್ತಿದ್ದಾನೆ ಮಂದ ಮಾರುತ. ಅವನ ಹಿಂದೆಯೆ ಬಂದ ವಸಂತ. ಅವರಿಬ್ಬರೂ ಪರಸ್ಪರ ಸೇರುವಷ್ಟರಲ್ಲಿ ಚಂದಿರನು ತನ್ನ ಬೆಳುದಿಂಗಳ ಹೊಳೆಯನ್ನು ಹರಿಸುತ್ತಾ ಮೂಡಿಬಂದ. ಆ ಬೆಳುದಿಂಗಳಲ್ಲಿ ಸುತ್ತುಮುತ್ತಿನ ಗಿಡಮರಬಳ್ಳಿಗಳೆಲ್ಲ ಹೊಸ ಚಿಗುರೊಡೆದು ದಿವ್ಯಸುಂದರವಾದ ಸ್ವರ್ಣಸ್ವಪ್ನ ಪ್ರಪಂಚವೊಂದನ್ನು ಸೃಷ್ಟಿಸಿ ದವು. ತನ್ನ ರಂಗಸಜ್ಜಿಕೆ ಸಿದ್ಧವಾದುದನ್ನು ಕಂಡು ಸ್ವತಃ ಮನ್ಮಥನೆ ತನ್ನ ಕಬ್ಬಿನ ಬಿಲ್ಲಲ್ಲಿ ಹೂಬಾಣವಿಟ್ಟುಕೊಂಡು, ಸುಂದರಿಯರಾದ ಅಪ್ಸರೆಯರೊಡನೆ ಅಲ್ಲಿಗಿಳಿದು ಬಂದ. ಅವರೆಲ್ಲರೂ ತಪೋನಿರತನಾದ ಋಷಿಯನ್ನು ಸುತ್ತುಗಟ್ಟಿದರು. ಗಂಧರ್ವರ ಗಾಯನದೊಡನೆ ಅಪ್ಸರೆಯರ ನರ್ತನ ಪ್ರಾರಂಭವಾಯಿತು. ಮನ್ಮಥನು ತನ್ನ ಸೇವಕರಾದ ಕಾಮ, ಮದ, ಲೋಭರನ್ನು ಋಷಿಯನ್ನು ಆಕ್ರಮಿಸುವಂತೆ ಅಪ್ಪಣೆಮಾಡಿ, ತನ್ನ ಪುಷ್ಪಬಾಣವನ್ನು ಋಷಿಯ ಹೃದಯಕ್ಕೆ ಗುರಿಯಿಟ್ಟ. ‘ಪುಂಜಿಕಸ್ಥಲಿ’ ಎಂಬ ಅಪ್ಸರೆ ಪುಟಚೆಂಡಿನ ಆಟವನ್ನು ಋಷಿಯ ಇದಿರಿನಲ್ಲಿ ಪ್ರಾರಂಭಿಸಿದಳು. ಅವಳು ಕುಣಿದಂತೆ ಸ್ತನ ಕುಣಿಯುತ್ತವೆ, ನಡು ಬಳುಕುತ್ತದೆ, ತುರುಬಿನ ಹೂ ಉದುರುತ್ತದೆ, ಚಂಡಿನತ್ತ ಹೊರಳಿದ ಚಂಚಲದೃಷ್ಟಿ ಮಿಂಚನ್ನು ಚೆಲ್ಲುತ್ತದೆ. ಅವಳು ಹಾಗೆ ಕುಣಿಯುತ್ತಿರುವಾಗ ಒಳ್ಳೆಯ ಸಮಯ ನೋಡಿಕೊಂಡು ಮಲಯಾನಿಲ ಅವಳ ಅಂಗವಸ್ತ್ರವನ್ನು ಮೇಲಕ್ಕೆ ಹಾರಿಸಿದ. ಅದೇ ಸಮಯಕ್ಕೆ ಮನ್ಮಥ ತನ್ನ ಬಾಣವನ್ನು ಹೊಡೆದ. ಆದರೇನು? ಅವರೆಲ್ಲರ ಪ್ರಯತ್ನಗಳೂ ದರಿದ್ರನ ಪ್ರಯತ್ನದಂತೆ ವಿಫಲವಾದವು. ಹಾವನ್ನು ಕೆಣಕಿದ ಹುಡುಗರಂತೆ ಅವರೆಲ್ಲರೂ ಅವನ ತೇಜಸ್ಸಿಗೆ ಹೆದರಿ ಚಲ್ಲಾಪಿಲ್ಲಿಯಾಗಿ ಓಡಿಹೋದರು.

ಋಷಿಯ ತೇಜಸ್ಸಿಗೆ ಹೆದರಿ ಓಡಿದ ಮನ್ಮಥನು ತನ್ನ ಪರಿವಾರದೊಡನೆ ದೇವೇಂದ್ರನ ಬಳಿಗೆ ಓಡಿಹೋಗಿ ‘ಸ್ವಾಮಿ, ನಿನ್ನ ಅಪ್ಪಣೆಯಂತೆ ನಾವು ಮಾರ್ಕಂಡೇಯನ ಬಳಿಗೆ ಹೋಗಿ, ನಮ್ಮ ಸಾಹಸವನ್ನೆಲ್ಲ ಬರಿದುಮಾಡಿಕೊಂಡೆವು. ಆತನು ನಮಗೆ ದಕ್ಕುವವನಲ್ಲ’ ಎಂದನು. ಅದನ್ನು ಕೇಳಿ ಕಂಗಾಲಾದ ದೇವೇಂದ್ರನು ‘ಅಹಾ, ಆ ಋಷಿಯು ಆರು ಮನ್ವಂತರಗಳನ್ನು ಮುಗಿಸಿ, ಏಳನೆಯ ಮನ್ವಂತರದಲ್ಲಿಯೂ ತಪಸ್ಸಿಗೆ ತೊಡಗಿದ್ದಾನೆ. ಈತನೆಲ್ಲಿ ನನ್ನ ಪದವಿಗೆ ಸಂಚು ತರುವನೊ ಎಂಬ ಭಯದಿಂದ ನನ್ನ ಜೀವ ತತ್ತರಿಸು ತ್ತಿದೆ’ ಎಂದು ಮನಸ್ಸಿನಲ್ಲಿಯೆ ಮಿಡುಕಿಕೊಂಡನು. ಮೃಕಂಡು ಋಷಿಯ ಮಗನಾದ ಮಾರ್ಕಂಡೇಯನು ತಂದೆಯಿಂದ ಉಪನಯನ ಸಂಸ್ಕಾರವನ್ನು ಪಡೆದವನೆ, ಗುರುವಿನ ಬಳಿಯಲ್ಲಿ ಅಧ್ಯಯನಕ್ಕೆಂದು ಬಂದವನು; ಆತ ಅಲ್ಲಿಂದ ಹಿಂದಿರುಗಲಿಲ್ಲ. ಹಗಲು ರಾತ್ರಿ ಭಿಕ್ಷೆಯಿಂದ ತಂದ ಆಹಾರವನ್ನು ಗುರುವಿಗೊಪ್ಪಿಸಿ, ಆತನು ತಿಂದು ಮಿಕ್ಕುದನ್ನು ಆತನು ಅಪ್ಪಣೆ ಇತ್ತರೆ ತಿನ್ನುವನು, ಇಲ್ಲದಿದ್ದರೆ ಉಪವಾಸವಿರುವನು. ಹೀಗೆ ಲಕ್ಷೋಪ ಲಕ್ಷ ವರ್ಷಗಳ ಕಾಲ ಗುರುಸೇವೆಯನ್ನು ಮಾಡಿ, ತ್ರಿಕಾಲಗಳಲ್ಲಿಯೂ ಭಗವಂತನ ಧ್ಯಾನ ದಲ್ಲಿ ಮಗ್ನನಾಗುವನು. ಯೋಗನಿರತನಾದ ಆ ಬ್ರಹ್ಮಚಾರಿಯು ಮೃತ್ಯುವನ್ನು ಜಯಿಸಿ ದನು. ಆತನು ಸದಾ ತಪೋನಿರತನಾಗಿ ಆರು ಮನ್ವಂತರಗಳಷ್ಟು ದೀರ್ಘಕಾಲ ಕುಳಿತು ದನ್ನು ಕಂಡ ದೇವೇಂದ್ರ ಕಳವಳಗೊಂಡುದು. ಮಹರ್ಷಿ ಮಾರ್ಕಂಡೇಯನ ತಪಸ್ಸನ್ನು ಕೆಡಿಸಬೇಕೆಂದು ಅವನು ಮಾಡಿದ ಪ್ರಯತ್ನ ವಿಫಲವಾಗಲು, ಅವನು ದೇವರೇ ತನಗೆ ದಿಕ್ಕೆಂದುಕೊಂಡು ಸುಮ್ಮನಾದನು.

ಒಮ್ಮೆ ಮಾರ್ಕಂಡೇಯ ಧ್ಯಾನಮಗ್ನನಾಗಿ ಕುಳಿತಿರಲು, ನರನಾರಾಯಣರು ಅವನ ಹೃದಯದಲ್ಲಿ ಗೋಚರಿಸಿದರು. ನಾರಾಯಣನ ಮೈ ಬಣ್ಣ ಕಪ್ಪು, ನರನದು ಬಿಳುಪು; ಇಬ್ಬರೂ ಬಾಲ್ಯಸೂರ್ಯನಂತೆ ತೊಳಗಿ ಬೆಳಗುತ್ತಿದ್ದಾರೆ. ಧೀರ ಗಂಭಿರ ಆಕೃತಿಯುಳ್ಳ ಅವರಿಬ್ಬರೂ ನಾರುಮಡಿಯುಟ್ಟು, ಕೃಷ್ಣಾಜಿನವನ್ನು ಹೊದ್ದಿದ್ದಾರೆ. ಕೊರಳಲ್ಲಿ ಯಜ್ಞೋ ಪವೀತವನ್ನು ಧರಿಸಿ ದಂಡ, ಕಮಂಡಲು, ಜಪಸರಗಳನ್ನು ಕೈಲಿ ಹಿಡಿದಿದ್ದಾರೆ. ಅವರ ಬೆರಳಲ್ಲಿ ದರ್ಭೆಯ ಪವಿತ್ರವಿದೆ. ತಪಸ್ಸೆ ಮೂರ್ತಿವೆತ್ತಂತೆ ಕಾಣಿಸುತ್ತಿರುವ ಅವರ ಬ್ರಹ್ಮ ತೇಜಸ್ಸನ್ನು ಸಹಿಸಲಾರದೆ ಅವನ ಒಳಗಣ್ಣು ಮುಚ್ಚಿ, ಹೊರಗಣ್ಣು ತೆರೆಯಿತು. ನರ ನಾರಾಯಣರು ಆತನಿಗೆ ಪ್ರತ್ಯಕ್ಷರಾಗಿಯೆ ಇದಿರಿನಲ್ಲಿ ನಿಂತಿದ್ದಾರೆ. ಮಾರ್ಕಂಡೇಯ ಋಷಿಯ ಮೈ ರೋಮಾಂಚನಗೊಂಡಿತು. ಆತನ ಕಣ್ಣು ಆನಂದಬಾಷ್ಪಗಳನ್ನು ಸುರಿಸಿತು. ಆತನು ದಿಗ್ಗನೆ ಮೇಲಕ್ಕೆದ್ದು, ಅವರನ್ನು ಆಸನದಲ್ಲಿ ಕುಳ್ಳಿರಿಸಿ, ಕಾಲ್ತೊಳೆದು, ಅತಿಥಿಸತ್ಕಾರದಿಂದ ಅವರನ್ನು ತಣಿಸಿದ ಮೇಲೆ, ನಾರಾಯಣನೊಡನೆ ‘ಹೇ ಸರ್ವೇಶ್ವರ, ಪರಾತ್ಪರ, ಪರಮಾತ್ಮ, ಲೋಕಗುರು, ಮಹರ್ಷಿಗಳು ಅನವರತವೂ ಧ್ಯಾನಿಸುವ ನಿನ್ನ ಪಾದಪದ್ಮಗಳನ್ನು ಈ ದಾಸನಿಗೆ ದರ್ಶನ ಮಾಡಿಸಿದೆ. ನನ್ನನ್ನು ಉದ್ಧರಿಸುವಂತೆ ಆ ಪಾದಗಳಿಗೆ ನಾನು ಶರಣುಹೊಕ್ಕಿದ್ದೇನೆ. ನಿನಗೆ ನಮೋ ನಮಃ’ ಎಂದು ಹೇಳಿ, ಆತನ ಮುಂದೆ ಅಡ್ಡಬಿದ್ದನು. ಆಗ ನಾರಾಯಣನು ಆತನನ್ನು ಮೇಲಕ್ಕೆಬ್ಬಿಸಿ ‘ಋಷಿತಿಲಕ! ಭಕ್ತಿ, ತಪಸ್ಸು, ವೇದಾಧ್ಯಯನ, ಸಮಾಧಿ–ಇವುಗಳಿಂದ ಇಂದ್ರಿಯನಿಗ್ರಹಮಾಡಿ, ಬ್ರಹ್ಮರ್ಷಿ ಗಳಲ್ಲಿ ಅಗ್ರಗಣ್ಯನಾಗಿರುವೆ, ನೀನು. ನಿನ್ನ ಬ್ರಹ್ಮಚರ್ಯೆಗೆ ಮೆಚ್ಚಿ, ನಿನಗೆ ವರಕೊಡಲೆಂದು ನಾನು ಬಂದಿದ್ದೇನೆ. ಕೇಳು, ನಿನಗೇನು ಬೇಕು?’ ಎಂದನು. ಮಾರ್ಕಂಡೇಯನು ಕೈ ಮುಗಿದುಕೊಂಡು ‘ಹೇ ಅಚ್ಯುತ! ನನ್ನೆಲ್ಲ ತಪಸ್ಸಿನ ಸಿದ್ಧಿಯೇ ನಿನ್ನ ದರ್ಶನ. ನೀನು ಈಗ ಅದನ್ನು ಕರುಣಿಸಿರುವುದೇ ವರಗಳಲ್ಲಿ ವರ. ನಾನು ಧನ್ಯನಾದೆ. ನೀನು ವರವನ್ನು ಕರುಣಿಸಲೇಬೇಕೆಂದಿದ್ದರೆ, ನಿನ್ನ ಮಾಯಾಪ್ರಭಾವವನ್ನು ನನಗೆ ತೋರಿಸು’ ಎಂದು ಕೇಳಿ ಕೊಂಡನು. ನಾರಾಯಣನು ‘ತಥಾಸ್ತು’ ಎಂದು ಹೇಳಿ, ನರನೊಡನೆ ಬದರಿಕಾಶ್ರಮಕ್ಕೆ ಹೊರಟುಹೋದನು.

ವಿಷ್ಣುಮಾಯೆಯನ್ನು ಕಾಣಬೇಕೆಂಬ ಕುತೂಹಲದಿಂದ ಇದ್ದ ಮಾರ್ಕಂಡೇಯನು ಪುಷ್ಪಭದ್ರಾ ನದಿಯ ತಟದಲ್ಲಿ ಧ್ಯಾನಮಗ್ನನಾಗಿ ಕುಳಿತಿದ್ದಾನೆ. ಇದ್ದಕ್ಕಿದ್ದಂತೆಯೆ ಭಯಂಕರವಾದ ಮಳೆ ಸುರಿಯಲು ಪ್ರಾರಂಭವಾಯಿತು. ಹಗಲುರಾತ್ರಿ ಎಡೆಬಿಡದೆ ಮಳೆ ಸುರಿದು ಸುರಿದು ಸಮುದ್ರಗಳೆಲ್ಲ ಒಂದಾಗಿಹೋದವು. ನೀರಿನ ಮೇಲಿನ ತೆಪ್ಪದಂತೆ ತೇಲುತ್ತಿದ್ದ ಭೂಮಿ ಒಮ್ಮೆಗೆ ಮುಳುಗಿ ಹೋಯಿತು. ಎಡೆಬಿಡದೆ ಸುರಿಯುತ್ತಿರುವ ಮಳೆ ಯಿಂದ ಸಮುದ್ರದ ನೀರು ಉಬ್ಬಿ ಉಬ್ಬಿ, ಗ್ರಹ ನಕ್ಷತ್ರಗಳೆಲ್ಲವೂ ಅದರಲ್ಲಿ ಮುಳುಗಿ ಹೋದವು. ಎಲ್ಲೆಲ್ಲಿಯೂ ಕತ್ತಲು, ಕತ್ತಲು, ಕಗ್ಗತ್ತಲು. ಮಾರ್ಕಂಡೇಯನು ಏಕಾಕಿ ಯಾಗಿ ಮಂಕನಂತೆ ನೀರಿನಲ್ಲಿ ತೇಲುತ್ತಿದ್ದಾನೆ. ಭೂಮಿ, ಆಕಾಶ, ದಿಕ್ಕು–ಯಾವುದೂ ಕಾಣಿಸುವಂತಿಲ್ಲ. ಅವನಿಗೆ ಹಸಿವೋ ಹಸಿವು. ಇದರ ಜೊತೆಗೆ ಬಿರುಗಾಳಿಯಿಂದ ಎದ್ದ ಅಲೆಗಳು ಬಂದು ಅವನನ್ನು ಅಪ್ಪಳಿಸುತ್ತಿವೆ. ನೀರಿನ ಸುಳಿಗಳು ಅವನನ್ನು ಕೆಳಕ್ಕೆ ಎಳೆಯು ತ್ತಿವೆ, ಜಲಚರಗಳು ಪರಸ್ಪರ ಜಗಳವಾಡುತ್ತಾ ಬಂದು ಅವನ ಮೇಲೆ ಬೀಳುತ್ತಿವೆ. ಋಷಿಯ ಆತ್ಮಜ್ಞಾನ ಎಲ್ಲಿ ಹಾರಿಹೋಯಿತೊ! ಅವನು ಭಯ, ಮೋಹ, ದುಃಖಗಳಿಂದ ತತ್ತರಿಸುತ್ತಿದ್ದಾನೆ. ಅದೆಷ್ಟೋ ಸಹಸ್ರವರ್ಷಗಳವರೆಗೆ ಅವನು ಸಮುದ್ರಮಧ್ಯದಲ್ಲಿ ಸಿಕ್ಕಿ ಸಂಕಟಪಡುತ್ತಾ ತೇಲಿಹೋಗುತ್ತಿದ್ದನು! ಕಡೆಗೆ ಎಲ್ಲೋ ಒಂದು ಕಡೆ ಎತ್ತರವಾದ ನೆಲ ಸಿಕ್ಕಂತಾಯಿತು. ಮಬ್ಬು ಬೆಳಕಿನಲ್ಲಿ ಅಲ್ಲೊಂದು ದೊಡ್ಡ ಮರವಿರುವಂತೆ ಬೋಧೆಯಾ ಯಿತು. ಆ ಋಷಿ ಅದರ ಬಳಿಗೆ ನಡೆದು ಹೋದ.

ಮಾರ್ಕಂಡೇಯ ಋಷಿಗೆ ಕಾಣಿಸಿದ ಮರ ಫಲಸಮೃದ್ಧವಾದ ಒಂದು ಆಲದ ಮರ. ಅದು ನೆಲಕ್ಕೆ ಬಹು ಸಮೀಪದಲ್ಲಿಯೆ ಹರಡಿಕೊಂಡಿತ್ತು. ಅದರ ಒಂದು ಕೊಂಬೆ ಈಶಾನ್ಯದಿಕ್ಕಿಗೆ ಚಾಚಿಕೊಂಡು ಹೋಗಿತ್ತು. ಆ ಕೊಂಬೆಯಲ್ಲಿ ದೊನ್ನೆಯಂತಿರುವ ಒಂದು ದೊಡ್ಡ ಎಲೆ, ಅದರಲ್ಲಿ ಒಂದು ಕೂಸು. ಅದರ ದೇಹಕಾಂತಿಯಿಂದ ಅಲ್ಲೊಂದು ದೀಪವನ್ನು ಹೊತ್ತಿಸಿಟ್ಟಂತೆ ಆಗಿದೆ. ಋಷಿಯು ಅದರ ಬಳಿಗೆ ಹೋದ. ಎಂತಹ ಮುದ್ದು ಮಗು ಅದು! ಶ್ಯಾಮಲವರ್ಣದ ಆ ಕೂಸಿನ ಮುಖ ಕಮಲದಲ್ಲಿ ಎಳಸಾದ ಮೂಗು, ಅಂದವಾದ ಹುಬ್ಬು, ಉಸಿರಾಟದಿಂದ ಅಲಗುತ್ತಿರುವ ಮುಂಗುರುಳು, ಹವಳದಂತಿರುವ ತುಟಿಯಲ್ಲಿ ಮುಗುಳ್ನಗೆ, ವಿಶಾಲವಾದ ಕಣ್ಣುಗಳಲ್ಲಿ ಮಿಂಚಿನಂತಿರುವ ದೃಷ್ಟಿ! ಆ ಮಗು ಎಸಳಾದ ತನ್ನ ಬೆರಳುಗಳಿಂದ ಕಮಲದಂತಿರುವ ತನ್ನ ಬಲಪಾದವನ್ನು ಹಿಡಿದು, ಅದರ ಹೆಬ್ಬೆರಳನ್ನು ತನ್ನ ಬಾಯಲ್ಲಿ ಚೀಪುತ್ತಿದೆ. ಮುಗುಳ್ನಗೆಯ ನವಿಲು ನರ್ತಿಸುತ್ತಿರುವ ಆ ಮುಖವನ್ನು ಕಾಣುತ್ತಲೆ ಮಾರ್ಕಂಡೇಯಋಷಿಯ ಶ್ರಮವೆಲ್ಲ ಪರಿಹಾರವಾದಂತಾ ಯಿತು. ಆ ಮಹಾಪ್ರಳಯದಲ್ಲಿ ಏಕಾಂಗಿಯಾಗಿ ನಲಿಯುತ್ತಿರುವ ಆ ಕೂಸನ್ನು ಕಂಡು ಆತನು ಅಚ್ಚರಿಯಿಂದ ಅದರ ಬಳಿಗೆ ಹೋದ. ಒಡನೆಯೆ ಆ ಶಿಶುವಿನ ಉಸಿರು ಆತನನ್ನು ಸೊಳ್ಳೆಯಂತೆ ತನ್ನೊಳಕ್ಕೆ ಎಳೆದುಕೊಂಡಿತು. ಆತ ಆ ಮಗುವಿನ ಹೊಟ್ಟೆಯನ್ನು ಸೇರಿದ. ಅಲ್ಲಿ ನೋಡುತ್ತಾನೆ–ತಾನು ಹೊರಗೆ ಕಂಡಿದ್ದ ಸಮಸ್ತವೂ ಅಲ್ಲಿಯೇ ಇದೆ. ಅಚ್ಚರಿ ಯಲ್ಲಿ ಅಚ್ಚರಿಯೆಂದರೆ ಹೊರಗಿದ್ದ ಆಲದಮರವೂ, ಅದರಲ್ಲಿದ್ದ ಶಿಶುವೂ ಅಲ್ಲಿಯೂ ಇವೆ. ಕ್ಷಣಕಾಲ ಕಳೆಯುವಷ್ಟರಲ್ಲಿ ಭೂಮಿ, ಆಕಾಶ, ನಕ್ಷತ್ರಗಳು, ಸಮುದ್ರ, ದೇಶ, ನದಿ, ಪಟ್ಟಣ, ಜೀವರಾಶಿ–ಅಖಂಡ ಬ್ರಹ್ಮಾಂಡವೇ ಅಲ್ಲಿ ಗೋಚರಿಸಿತು. ತಾನಿದ್ದ ಹಿಮಾಲಯ ಪರ್ವತವು, ತನ್ನ ಆಶ್ರಮವೂ, ತನ್ನ ಜೊತೆಯ ಋಷಿಗಳೂ ಅಲ್ಲಿ ಕಾಣಿಸಿ ದರು. ಅಜ್ಞಾನಿಗಳು ನಿತ್ಯವೆಂದು ತಿಳಿಯುವ ಸಮಸ್ತ ಜಗತ್ತೂ ಅಲ್ಲಿತ್ತು. ಆತನು ಅದ ನ್ನೆಲ್ಲ ಕಾಣುತ್ತ ಮೈಮರೆತಿರುವಷ್ಟರಲ್ಲಿ ಮಗು ತನ್ನ ಉಸಿರನ್ನು ಹೊರಗೆ ಬಿಟ್ಟಿತು. ಋಷಿಯು ಮತ್ತೆ ಪ್ರಳಯ ಮಧ್ಯದ ಆಲದ ಮರದ ಬಳಿ ಬಂದು ಬಿದ್ದ. ಆಗ ಆತ ಅಲ್ಲಿದ್ದ ಮಗವನ್ನು ಎತ್ತಿಕೊಳ್ಳಬೇಕೆಂದು ಹೋದ. ಆ ಕ್ಷಣವೆ ಅದು ಮಾಯವಾಗಿ ಹೋಯಿತು. ಅದರೊಡನೆ ಪ್ರಳಯವೂ ಮಾಯವಾಯಿತು. ಋಷಿಯು ತನ್ನ ಸಮಾಧಿ ಯಿಂದ ಬಹಿರ್ಮುಖನಾದ. ಆತನಿಗೆ ವಿಷ್ಣುಮಾಯೆಯ ದರ್ಶನವಾದಂತಾಯಿತು. ಆತನ ಬಾಯಿಂದ ‘ಹೇ ಪ್ರಭು, ನಿನ್ನ ಮಾಯೆಗೆ ಸಿಲುಕದಂತೆ ನನ್ನನ್ನು ರಕ್ಷಿಸು’ ಎಂಬ ಮಾತುಗಳು ಹೊರಬಂದವು.

ಮಾರ್ಕಂಡೇಯಋಷಿ ಮಾಯೆಯನ್ನು ಕಂಡು ಬಹಿರ್ಮುಖನಾದವನು ಭಗವಂತ ನನ್ನು ಧ್ಯಾನಿಸುತ್ತಾ ಮತ್ತೆ ಸಮಾಧಿಗೇರಿದನು. ಆ ಸಮಯದಲ್ಲಿ ಪರಮೇಶ್ವರನು ಪಾರ್ವತಿಯೊಡನೆ ಆಕಾಶಮಾರ್ಗದಲ್ಲಿ ಹೋಗುತ್ತಿರಲು, ಪಾರ್ವತಿಯು ಆ ಋಷಿಯನ್ನು ಶಿವನಿಗೆ ತೋರಿಸುತ್ತಾ ‘ದೇವದೇವ, ಗಾಳಿಯಿಲ್ಲದ ಸಮುದ್ರದಂತೆ ನಿಶ್ಚಲಚಿತ್ತನಾಗಿ ಕುಳಿತಿರುವ ಆ ಋಷಿಯನ್ನು ನೋಡು. ನೀನು ಸಿದ್ಧಿಪ್ರದನಲ್ಲವೆ? ಬಾ, ನೀನೀಗ ಈತನಿಗೆ ವರಗಳನ್ನು ಕರುಣಿಸು’ ಎಂದಳು. ಆಗ ಶಿವನು ‘ಗಿರಿಜೆ, ಈತ ಯಾವ ವರವನ್ನೂ ಬಯಸು ವವನಲ್ಲ. ಇವನಿಗೆ ಮೋಕ್ಷ ಕೂಡ ಬೇಡ. ಆದರೂ ನಾವೀಗ ಹೋಗಿ ಮಾತನಾಡಿಸಿ ಬರೋಣ. ಇಂತಹ ಸಾಧುಸಂತರ ಸಮಾಗಮ ನಮಗೂ ಒಳ್ಳೆಯದೆ’ ಎಂದು ಹೇಳಿ, ಆತನು ಗಿರಿಜೆಯೊಡನೆ ಆತನ ಇದಿರಿಗೆ ಹೋಗಿ ನಿಂತನು. ಆದರೇನು? ನಿರಾಕಾರದಲ್ಲಿ ನೆಲೆಸಿದ್ದ ಆತನ ಮನಸ್ಸು ಇದಿರಿಗೆ ನಿಂತ ಉಮಾಮಹೇಶ್ವರನನ್ನು ಕಾಣದಾಯಿತು. ಆಗ ಪರಶಿವನು ತನ್ನ ಯೋಗಮಾಯೆಯಿಂದ ಆತನ ಹೃದಯವನ್ನು ಪ್ರವೇಶಿಸಿದನು. ಒಡನೆಯೆ ಋಷಿಯ ಹೃದಯದಲ್ಲಿ ಮಿಂಚು ಮಿಂಚಿದಂತಾಯಿತು. ಆ ಮಿಂಚಿನ ಮಧ್ಯ ದಲ್ಲಿ ಬಾಲಸೂರ್ಯನಂತೆ ತೇಜಸ್ವಿಯಾದ ಮುಕ್ಕಣ್ಣ ಕಾಣಿಸಿದ. ಆತನ ಹತ್ತು ತೋಳುಗಳು, ಆ ಒಂದೊಂದು ತೋಳಿನಲ್ಲಿಯೂ ಧರಿಸಿರುವ ಆಯುಧಗಳು, ಬೆನ್ನ ಮೇಲಿನ ಹುಲಿ ಗೂದಲು, ಮಹೋನ್ನತವಾದ ಆ ದಿವ್ಯಾಕಾರ–ಇವುಗಳನ್ನು ಕಂಡು ಮಾರ್ಕಂಡೇಯನು ಆಶ್ಚರ್ಯದಿಂದ ಕಣ್ಣುಗಳನ್ನು ತೆರೆದನು. ತನ್ನ ಇದಿರಿನಲ್ಲಿ ಗಿರಿಜೆಯೊಡನೆ ನಿಂತಿರುವ ಆ ಲೋಕಗುರುವನ್ನು ಕಾಣುತ್ತಲೆ ಆತನು ಭಕ್ತಿಯಿಂದ ಅವರನ್ನು ನಮಸ್ಕರಿಸಿ, ಪೂಜಿಸಿದನು. ಅನಂತರ ಆತನು ಕೈಮುಗಿದುಕೊಂಡು ‘ಪರಮೇಶ್ವರ, ಆನಂದರೂಪಿಯಾದ ನಿನ್ನನ್ನು ನಾನು ಹೇಗೆ ಸಂತೋಷಪಡಿಸಲಿ? ನಿತ್ಯತೃಪ್ತನಾದ ನಿನಗೆ ನಮಸ್ಕಾರ. ಸತ್ವರೂಪಿಯಾಗಿ ನೀನೆ ಮಹಾವಿಷ್ಣುವೆನಿಸುವೆ. ರಜೋಗುಣವನ್ನು ಧರಿಸಿ ಬ್ರಹ್ಮನೆನಿಸುವೆ. ಪರಬ್ರಹ್ಮ ಸ್ವರೂಪಿಯಾದ ನಿನಗೆ ನಮೋ ನಮಃ’ ಎಂದು ಅಡ್ಡಬಿದ್ದನು.

ಮಾರ್ಕಂಡೇಯನ ಭಕ್ತಿಯಿಂದ ಸುಪ್ರೀತನಾದ ಭಗವಂತನು ‘ಹೇ ಮಹರ್ಷಿ, ನಾವು ತ್ರಿಮೂರ್ತಿಗಳು ಬೇರೆಬೇರೆಯೇನೂ ಅಲ್ಲ. ನಮ್ಮ ದರ್ಶನ ಸದಾ ಸಫಲವಾದುದೇ. ನಮ್ಮ ದರ್ಶನದಿಂದ ಮುಕ್ತಿಯೂ ಲಭಿಸುತ್ತದೆ. ಕೇಳು, ನಿನಗೇನು ಬೇಕು? ಮಹಾ ಬ್ರಾಹ್ಮಣ, ತ್ರಿಮೂರ್ತಿಗಳಲ್ಲಿ ಭೇದವನ್ನೆಣಿಸದ ನೀನು ಲೋಕಪೂಜ್ಯ. ಅವರಿವರಿರಲಿ, ತ್ರಿಮೂರ್ತಿಗಳಾದ ನಾವು ಕೂಡ ನಿನ್ನಂತಹ ಸಾಧುಸಜ್ಜನರನ್ನು ವಂದಿಸುತ್ತೇವೆ. ಕಲ್ಲು ಮಣ್ಣಿನ ದೇವತಾ ಪ್ರತಿಮೆಗಳೂ, ಗಂಗಾದಿ ತೀರ್ಥಗಳೂ ಪೂಜೆ ಅಥವಾ ಸ್ನಾನ ಪಾನ ಗಳಿಂದ ಮನುಷ್ಯರನ್ನು ಪಾವನರನ್ನಾಗಿ ಮಾಡುತ್ತವೆಯೆ ಹೊರತು ಕೇವಲ ದರ್ಶನ ದಿಂದಲ್ಲ. ಆದರೆ ನಿನ್ನಂಥ ಸಜ್ಜನರ ದರ್ಶನವೇ ಪವಿತ್ರ. ಮಹಾಪಾಪಿ ಕೂಡ ನಿಮ್ಮಂ ತಹವರ ದರ್ಶನದಿಂದ ಧನ್ಯನಾಗುತ್ತಾನೆ’ ಎಂದನು. ಅಮೃತದಂತಿದ್ದ ಆತನ ವಾಣಿ ಯಿಂದ ಧನ್ಯತೆಯನ್ನು ಪಡೆದ ಮಾರ್ಕಂಡೇಯನು ಮನಸ್ಸಿನಲ್ಲಿಯೆ ‘ಆಹಾ ಭಗವಂತನ ಲೀಲೆಯೆ! ತನ್ನ ಭಕ್ತರಿಗೆ ತಾನೆ ನಮಸ್ಕರಿಸುತ್ತಾನೆ, ತನ್ನನ್ನು ಸ್ತೋತ್ರಮಾಡುವವರನ್ನು ತಾನೆ ಸ್ತೋತ್ರಮಾಡುತ್ತಾನೆ. ಲೋಕಕ್ಕೆ ಧರ್ಮಮಾರ್ಗವನ್ನು ತಿಳಿಸಲು ತಾನು ಅದನ್ನು ಮಾಡಿ ತೋರಿಸುತ್ತಾನೆ’ ಎಂದುಕೊಂಡು ಪರ ಶಿವನೊಡನೆ ‘ಹೆ ಭಗವಂತ, ನನ್ನ ಹರಿಭಕ್ತಿ ಸ್ಥಿರವಾಗಿರುವಂತೆ ಅನುಗ್ರಹಿಸು’ ಎಂದು ಬೇಡಿದನು. ಪರಮೇಶ್ವರನು ‘ತಥಾಸ್ತು’ ಎಂದು ಹೇಳಿ, ‘ಅಯ್ಯಾ, ನಿನ್ನ ಕೀರ್ತಿ ಕಲ್ಪಾಂತದವರೆಗಿರಲಿ, ನಿನಗೆ ಜರಾಮರಣಗಳಿಲ್ಲ ದಿರಲಿ, ನಿನಗೆ ತ್ರಿಕಾಲಜ್ಞಾನವಿರಲಿ’ ಎಂದು ಹರಸಿ, ಅಲ್ಲಿಂದ ಹೊರಟನು. ಮುಂದೆ ಪ್ರಯಾಣ ಮಾಡುತ್ತಾ ಪರಮೇಶ್ವರನು ಶಿವೆಯೊಡನೆ ಮಾರ್ಕಂಡೇಯನಿಗಾದ ವಿಷ್ಣು ಮಾಯೆಯ ಅನುಭವವನ್ನೆಲ್ಲ ವಿವರಿಸಿ ಹೇಳಿದನು. ಚಿರಂಜೀವಿಯಾದ ಮಾರ್ಕಂ ಡೇಯನು ಇಂದಿಗೂ ಭೂಮಿಯಲ್ಲಿ ಸಂಚರಿಸುತ್ತಾ ಇರುವನು.

ಮಾರ್ಕಂಡೇಯನ ಚರಿತ್ರೆಯನ್ನು ಹೇಳಿ ಮುಗಿಸಿದ ಸೂತಪುರಾಣಿಕನು ಭಾಗವತ ಕಥಾ ನಿರೂಪಣೆಯ ಫಲಶ್ರುತಿಯನ್ನೂ ಹೇಳಿದನು–ಹದಿನೆಂಟು ಸಹಸ್ರ ಗ್ರಂಥ ಪ್ರಮಾಣ ವುಳ್ಳ ಈ ಭಾಗವತವು ಹದಿನೆಂಟು ಪುರಾಣಗಳ ಮಾಲೆಯಲ್ಲಿ ನಾಯಕರತ್ನದಂತಿದೆ. ಹರಿಲೀಲೆಯ ಕಥನಾಮೃತವನ್ನು ಒಳಗೊಂಡ ಈ ಮಹಾಪುರಾಣವು ಮಾನವರಂತೆ ಬಾನವರನ್ನೂ ಉದ್ಧರಿಸಬಲ್ಲ ದಿವ್ಯಕೃತಿ. ಜೀವ ಬ್ರಹ್ಮರ ಐಕ್ಯವೇ ಇದರ ವಸ್ತು. ಮೋಕ್ಷವೇ ಇದಕ್ಕೆ ಪ್ರಯೋಜನ. ಸಕಲ ವೇದವೇದಾಂತಗಳ ಸಾರವನ್ನು ಒಳಗೊಂಡ ಈ ಗ್ರಂಥವನ್ನು ಓದಿ ತೃಪ್ತನಾದವನಿಗೆ ಮತ್ತಾವ ಪುರಾಣದಲ್ಲಿಯೂ ಅಭಿರುಚಿ ಹುಟ್ಟದು. ನದಿಗಳಲ್ಲಿ ಗಂಗೆ, ಕ್ಷೇತ್ರಗಳಲ್ಲಿ ಕಾಶಿ, ದೇವತೆಗಳಲ್ಲಿ ವಿಷ್ಣು, ಪುರಾಣಗಳಲ್ಲಿ ಭಾಗವತ ಸರ್ವಶ್ರೇಷ್ಠ. ಇದನ್ನು ಆದರದಿಂದ ಓದುವ, ಭಕ್ತಿಯಿಂದ ಕೇಳುವ ಜನ ಮುಕ್ತರಾಗುತ್ತಾರೆ. ಇದನ್ನು ಹೇಳಿ ಪರೀಕ್ಷಿದ್ರಾಜನಿಗೆ ಮುಕ್ತಿಮಾರ್ಗವನ್ನು ತೋರಿದ ಸರ್ವಜ್ಞನಾದ ಶುಕ್ರಮುನಿಗೆ ನಮಸ್ಕಾರ. ಯಾರ ನಾಮಸ್ಮರಣೆಯಿಂದ ಸಕಲ ದುರಿತಗಳೂ ದೂರವಾಗಿ ದುಃಖ ನಿವಾರಣೆಯಾಗುವುದೊ, ಆ ಪರಾತ್ಪರ ಮೂರುತಿಗೆ ನಮೋ ನಮಃ, ಲೋಕಾಸ್ಸಮಸ್ತಾಃ ಸುಖಿನೋ ಭವಂತು.

\begin{center}
\textbf{॥ ಓಂ ತತ್ ಸತ್ ॥}
\end{center}

