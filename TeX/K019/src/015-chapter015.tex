
\chapter{೧೫. ಪಾಪಿಗಳಲ್ಲಿ ಪಾಪಿ ವೇನ}

ಧ್ರುವಚಕ್ರವರ್ತಿಯನಂತರ ಆತನ ಮಗನಾದ ಉತ್ಕಲನು ಸಿಂಹಾಸನವನ್ನು ಏರಿದನಾದರೂ ಬಾಲ್ಯದಿಂದ ವೈರಾಗ್ಯಪರನಾಗಿದ್ದ ಆತನಿಗೆ ಚಕ್ರವರ್ತಿಪದವಿಯಲ್ಲಾ ಗಲಿ, ಸಂಸಾರ ಜೀವನದಲ್ಲಿಯಾಗಲಿ ಆಸಕ್ತಿ ಇರಲಿಲ್ಲ. ಆತನು ಸದಾ ಶಾಂತಚಿತ್ತ. ಜಗತ್ತಿನ ವ್ಯಾಪಾರದಲ್ಲಿ ಆತನು ಕಣ್ಣಿದ್ದರೂ ಕುರುಡನಂತೆ, ಕಿವಿಯಿದ್ದರೂ ಕಿವುಡನಂತೆ, ಬಾಯಿದ್ದರೂ ಮೂಕನಂತೆ ವ್ಯವಹರಿಸುತ್ತಿದ್ದನು. ಬೂದಿಮುಚ್ಚಿದ ಆ ಕೆಂಡವನ್ನು ಜನ ಅರ್ಥಮಾಡಿಕೊಳ್ಳಲಾರದೆ ‘ಮರುಳ’ನೆಂದೇ ಭಾವಿಸಿತು. ಅರಮನೆಯ ಹಿರಿಯರೂ ಮಂತ್ರಿಗಳೂ ಸೇರಿ, ಆತನನ್ನು ಸಿಂಹಾಸನದಿಂದ ಇಳಿಸಿ, ಭ್ರಮಿಯ ಮಗನಾದ ವತ್ಸರ ನನ್ನು ಪಟ್ಟಕ್ಕೇರಿಸಿದರು. ಆತನು ಗೃಹಸ್ಥನಾಗಿ ಹಲವು ಮಕ್ಕಳ ತಂದೆಯಾದನು. ಆತನ ನಂತರ ಹಲವು ತಲೆಗಳಾದಮೇಲೆ ಅಂಗನೆಂಬುವನು ಸಿಂಹಾಸನವನ್ನೇರಿ, ಧರ್ಮದಿಂದ ರಾಜ್ಯಭಾರಮಾಡುತ್ತಾ, ಮಹಾ ಕೀರ್ತಿಶಾಲಿಯಾದನು. ಆತನು ಒಮ್ಮೆ ಅತಿ ವೈಭವದಿಂದ ಅಶ್ವಮೇಧಯಾಗವನ್ನು ಮಾಡಿದನು. ಆಗ ದೇವತೆಗಳು ಆತನು ನೀಡಿದ ಹವಿಸ್ಸನ್ನು ಸ್ವೀಕರಿಸಲಿಲ್ಲ. ವೇದಮಂತ್ರಗಳನ್ನು ಹೇಳುತ್ತಿದ್ದ ಪುತ್ವಿಜರಿಗೆ ಆಶ್ಚರ್ಯವಾಯಿತು. ಅವರು ಹೇಳುತ್ತಿದ್ದ ಮಂತ್ರಗಳ ಸ್ವರಗಳಲ್ಲಿ ಯಾವ ಲೋಪವೂ ಇರಲಿಲ್ಲ; ಸಂಗ್ರಹಿ ಸಿರುವ ಹವಿಸ್ಸಿನ ವಸ್ತುಗಳಲ್ಲಿ ಯಾವ ದೋಷವೂ ಇರಲಿಲ್ಲ. ಎಂದಮೇಲೆ ಯಾಗ ಮಾಡುವವನಲ್ಲಿಯೇ ಏನೋ ದೋಷವಿರಬೇಕು. ಪುಣ್ಯಾತ್ಮನಾದ ಆತನಲ್ಲಿ ಯಾವ ದೋಷವಿದೆ? ಅಹುದು; ಆತನಲ್ಲಿ ಒಂದು ಸಣ್ಣ ದೋಷವಿತ್ತು. ಪೂರ್ವಜನ್ಮದ ಪಾಪ ಕರ್ಮದಿಂದ ಆತನಿಗೆ ಮಕ್ಕಳಾಗಿರಲಿಲ್ಲ. ಮಕ್ಕಳಿಲ್ಲದವರಿಂದ ದೇವತೆಗಳು ಹವಿಸ್ಸನ್ನು ಸ್ವೀಕರಿಸುವುದಿಲ್ಲ. ಇದನ್ನು ಕೇಳಿದ ರಾಜನು, ಪುರೋಹಿತರ ಬುದ್ಧಿವಾದದಂತೆ ಪುತ್ರಕಾ ಮೇಷ್ಟಿಯನ್ನು ಆಚರಿಸಿದನು. ಆಗ ಅಗ್ನಿಪುರುಷನು ಪ್ರತ್ಯಕ್ಷನಾಗಿ ಪಾಯಸದಿಂದ ತುಂಬಿದ ಒಂದು ಪಾತ್ರೆಯನ್ನು ರಾಜನಿಗೆ ಕೊಟ್ಟನು. ಆತನು ಅದನ್ನು ತನ್ನ ಮಡದಿ ಯಾದ ಸುನೀಥೆಗೆ ಕೊಟ್ಟನು. ಆಕೆ ಅದನ್ನು ತಿಂದು, ಗರ್ಭವತಿಯಾಗಿ, ನವಮಾಸ ತುಂಬು ತ್ತಲೆ ಗಂಡುಮಗನನ್ನು ಹೆತ್ತಳು. ಅವನ ಹೆಸರು ವೇನ.

ವೇನನು ಚಿಕ್ಕಂದಿನಲ್ಲಿ ತನ್ನ ತಾಯಿಯ ತಂದೆಯಾದ ಮೃತ್ಯುವಿನ ಮನೆಯಲ್ಲಿ ಇದ್ದು, ಅಲ್ಲಿಯೇ ದೊಡ್ಡವನಾದನು. ಕೆಟ್ಟ ವಂಶದವನಾದ ಮೃತ್ಯುವಿನ ಸಹವಾಸದಲ್ಲಿ ಬಹು ಕಾಲ ನೆಲೆಸಿದ್ದ ಆ ಹುಡುಗ ಕೇವಲ ನೀಚಬುದ್ಧಿಯವನಾದ. ವಯಸ್ಸು ಬೆಳೆದಂತೆಲ್ಲ ಅವನ ದುರ್ಬುದ್ಧಿಯೂ ಬೆಳೆಯಿತು. ಕೈಲಿ ಬಿಲ್ಲುಬಾಣಗಳನ್ನು ಹಿಡಿದು ಕಣ್ಣಿಗೆ ಬಿದ್ದ ಮೃಗಪಕ್ಷಿಗಳನ್ನೆಲ್ಲ ನಿರ್ದಯವಾಗಿ ಕೊಲ್ಲುತ್ತಿದ್ದನು. ಬೀದಿಯಲ್ಲಿ ಕಂಡ ಮಕ್ಕಳನ್ನು ಬಾಣದಿಂದ ಹೊಡೆದು ಬೀಳಿಸುತ್ತಿದ್ದನು. ಜನರು ಅವನನ್ನು ಕಂಡರೆ ಗಡಗಡ ನಡುಗುತ್ತಾ ‘ಅಯ್ಯಯ್ಯೊ, ಈ ಪಾಪಿ ವೇನನು ಬಂದನಲ್ಲಪ್ಪ!’ ಎಂದು ಗೋಳಾಡುತ್ತಿದ್ದರು. ತಂದೆ ಯಾದ ಅಂಗರಾಜನು ಅವನನ್ನು ತಿದ್ದುವುದಕ್ಕಾಗಿ ಎಷ್ಟೋ ಪ್ರಯತ್ನಿಸಿದನು. ಎಲ್ಲವೂ ವ್ಯರ್ಥವಾಯಿತು. ಇಂತಹ ಮಗ ಹುಟ್ಟುವುದಕ್ಕಿಂತ ಮಕ್ಕಳಿಲ್ಲದಿರುವುದೇ ಲೇಸೆನಿಸಿತು. ದೇವರು ತನಗೆ ವೈರಾಗ್ಯವನ್ನು ಹುಟ್ಟಿಸುವುದಕ್ಕಾಗಿಯೆ ಇಂತಹ ಮಗನನ್ನು ಕೊಟ್ಟಿರ ಬಹುದೆ–ಎಂದುಕೊಂಡನು. ಕೊನೆಗೆ ರೋಸಿ, ಒಂದು ದಿನ ಅರ್ಧರಾತ್ರಿಯಲ್ಲಿ ಯಾರಿಗೂ ಹೇಳದೆ ಕೇಳದೆ ಅಡವಿಗೆ ಹೊರಟು ಹೋದನು. ಮರುದಿನ ಮಂತ್ರಿಯೇ ಮೊದಲಾದವರು ಆತನಿಗಾಗಿ ಎಲ್ಲೆಲ್ಲಿಯೂ ಹುಡುಕಿ ನೋಡಿದರು. ಎಲ್ಲವೂ ವ್ಯರ್ಥ ವಾಯಿತು. ರಾಜನಿಲ್ಲದೆ ರಾಜ್ಯವು ಅನಾಯಕವಾಗುವುದಲ್ಲ! ಆದ್ದರಿಂದ ಋಷಿಗಳೆಲ್ಲರೂ ಸೇರಿ ವೇನನಿಗೆ ಪಟ್ಟಾಭಿಷೇಕ ಮಾಡಿದರು. ಪ್ರಜೆಗಳಾರಿಗೂ ಇದು ಒಪ್ಪಿಗೆಯಾಗದಿದ್ದರೂ ಅನಿವಾರ್ಯವಾಗಿ ಹಾಗೆ ಮಾಡಬೇಕಾಯಿತು. 

ವೇನನಿಗೆ ಪಟ್ಟಕಟ್ಟಿದುದು ಚೇಳಿಗೆ ಪಾರುಪತ್ಯ ಕೊಟ್ಟಂತಾಯಿತು. ಅವನು ಅಹಂಕಾರ ದಿಂದ ಬೀಗಿ ಬಿರಿಯುತ್ತಾ, ಕೊಬ್ಬಿದ ಕಾಡಾನೆಯಂತೆ ಮನಸ್ಸು ಬಂದಂತೆ ಜನರನ್ನು ಹಿಂಸಿಸತೊಡಗಿದನು. ‘ಯಾರೂ ಯಾಗಗಳನ್ನು ಮಾಡಕೂಡದು, ಯಾರೂ ದಾನಧರ್ಮ ಗಳನ್ನು ನಡೆಸಕೂಡದು’ ಎಂದು ರಾಜ್ಯದಲ್ಲೆಲ್ಲ ಡಂಗುರ ಹೊಡೆಸಿದನು. ದೇಶದಲ್ಲಿ ಕಳ್ಳ ಕಾಕರ ಹಾವಳಿ ಹೆಚ್ಚಿತು. ಎಲ್ಲೆಲ್ಲಿಯೂ ಹಾಹಾಕಾರ ಹುಟ್ಟಿತು. ಯಜ್ಞಯಾಗಾದಿಗಳನ್ನು ಮಾಡುತ್ತಿದ್ದ ಋಷಿಗಳು ದಿಕ್ಕು ತೋರದೆ ಕಳವಳಿಸಿದರು. ಎರಡು ಕಡೆಯಿಂದಲೂ ಉರಿ ಯುತ್ತಾ ಬರುತ್ತಿದ್ದ ಉರಿಯ ಮಧ್ಯದಲ್ಲಿ ಸಿಕ್ಕ ಹುಳುಗಳಂತೆ, ಪ್ರಜೆಗಳು ಅತ್ತ ರಾಜನ ಕಾಟ, ಇತ್ತ ಕಳ್ಳರ ಕಾಟಕ್ಕೆ ಸಿಕ್ಕಿ ತಳಮಳಿಸಿದರು. ಸಮಾಜದ ರಕ್ಷಕರಾದ ಋಷಿಗಳು ಒಂದೆಡೆ ಕಲೆತು ಮುಂದೇನು ಮಾಡಬೇಕೆಂದು ಆಳವಾಗಿ ಆಲೋಚಿಸಿದರು. ತಾವಾಗಿ ತಮ್ಮ ಮೇಲೆ ಈ ಸಂಕಟ ತಂದುಕೊಂಡುದಕ್ಕಾಗಿ ಅವರು ಪಶ್ಚಾತ್ತಾಪಪಟ್ಟರು. ಎಲ್ಲರೂ ಸೇರಿ ರಾಜನಿಗೆ ಒಳ್ಳೆಯ ಮಾತಿನಿಂದ ಬುದ್ಧಿವಾದ ಹೇಳಬೇಕೆಂದು ನಿರ್ಧರಿಸಿದರು. ಅದರಂತೆ ಅವರೆಲ್ಲರೂ ವೇನನ ಬಳಿಗೆ ಹೋಗಿ ‘ಮಹಾರಾಜ, ದಯವಿಟ್ಟು ನಮ್ಮ ಪ್ರಾರ್ಥನೆಯನ್ನು ಲಾಲಿಸು. ಪ್ರಜೆಗಳೆಲ್ಲರೂ ತಮ್ಮ ತಮ್ಮ ಧರ್ಮಗಳನ್ನು ಆಚರಿಸುವು ದಕ್ಕೆ ಅವಕಾಶವನ್ನು ಕೊಡು. ಅವರನ್ನು ಅಪಾಯವಿಲ್ಲದಂತೆ ರಕ್ಷಿಸುವುದು ನಿನ್ನ ಧರ್ಮ. ನೋಡು, ದೇಶದಲ್ಲಿ ಎಲ್ಲಿನೋಡಿದರೂ ಕಳ್ಳಕಾಕರ ಬಾಧೆ ಒಂದು ಕಡೆ; ದೇವರ ಆರಾಧನಾರೂಪವಾದ ಯಾಗಗಳಿಗೆ ತಡೆ ಮತ್ತೊಂದು ಕಡೆ. ಹೀಗಾದರೆ ಹೇಗೆ? ದೇವತಾ ರಾಧನೆಯಿಂದ ಇಹಪರ ಸುಖಗಳು ಮಾತ್ರವೇ ಅಲ್ಲ, ಅದರ ರಕ್ಷಕನಾದ ರಾಜನಿಗೂ ಶ್ರೇಯಸ್ಸುಂಟಾಗುತ್ತದೆ’ ಎಂದು ದೈನ್ಯದಿಂದ ಪ್ರಾರ್ಥಿಸಿದರು.

ಋಷಿಗಳ ಬುದ್ಧಿವಾದ ವೇನನಿಗೆ ಹಿಡಿಸಲಿಲ್ಲ. ಆತನು ಅವರನ್ನು ಗೇಲಿಮಾಡುತ್ತಾ ‘ಎಲೈ ಋಷಿಗಳೆ, ನಿಮ್ಮ ಅವಿವೇಕದ ಮಾತನ್ನು ನಿಲ್ಲಿಸಿ. ಅನ್ನ ಬಟ್ಟೆಗಳನ್ನು ಕೊಟ್ಟು ಕಾಪಾಡುವ ಗಂಡನನ್ನು ಬಿಟ್ಟು ಕಾಮಸುಖವನ್ನು ನೀಡುವ ವಿಟನನ್ನು ಪ್ರೀತಿಸುವ ಸೂಳೆಯೂ ಒಂದೆ, ನೀವೂ ಒಂದೆ. ನಿಮ್ಮನ್ನು ಕಾಪಾಡುತ್ತಿರುವವನು ನಾನು; ನನ್ನನ್ನು ಬಿಟ್ಟು ಕಣ್ಣಿಗೆ ಕಾಣದ ಅವನಾವನೊ ಯಜ್ಞಪುರುಷನೆಂಬುವನನ್ನು ಪೂಜಿಸುತ್ತಿರುವಿ ರಲ್ಲಾ! ಯಾವನು ಆ ಯಜ್ಞಪುರುಷ? ಎಲ್ಲಿ ಸ್ವಲ್ಪ ತೋರಿಸಿ ಅವನನ್ನು. ರಾಜನು ಸಕಲ ದೇವತೆಗಳ ಸ್ವರೂಪವೆಂದು ಹೇಳುತ್ತಾರೆ. ಅಂತಹ ನಾನು ಪ್ರತ್ಯಕ್ಷನಾಗಿರುವಾಗ ಬೇರೆ ದೇವತೆಗಳಿಗೆ ಪೂಜೆಯೆ? ಅದನ್ನು ಇನ್ನು ನಿಲ್ಲಿಸಿ. ನೀವು ಮಾಡಬೇಕೆಂದಿರುವ ಪೂಜೆಯ ನ್ನೆಲ್ಲ ನನಗೆ ಮಾಡಿರಿ. ದೇವತೆಗಳಿಗೆ ಕೊಡಬೇಕೆಂದಿರುವ ಹವಿಸ್ಸನ್ನೆಲ್ಲ ನನಗೆ ಕೊಡಿ. ಅಗ್ರಪೂಜೆಯನ್ನು ಪಡೆಯುವುದಕ್ಕೆ ನನಗಿಂತ ದೊಡ್ಡ ದೇವರಾರೂ ಇಲ್ಲ’ ಎಂದನು.

ತಮ್ಮ ಬುದ್ಧಿವಾದ ಕಲ್ಲುಬಂಡೆಯ ಮೇಲೆ ಮಳೆಗರೆದಂತಾದುದನ್ನು ಕಂಡು, ಋಷಿಗಳು ಮಮ್ಮಲ ಮರುಗಿದರು. ಅವರ ಸಂಕಟ ರೋಷಕ್ಕೆ ತಿರುಗಿತು. ಭಗವಂತನ ಅನು ಗ್ರಹದಿಂದ ರಾಜಪದವಿಯನ್ನು ಪಡೆದ ಈ ಪಾಪಿ ಭಗವಂತನನ್ನೆ ಮರೆತಿರುವುದರಿಂದ, ಇವನು ಇನ್ನು ಭೂಮಿಯ ಮೇಲಿರಲು ಯೋಗ್ಯನಲ್ಲ–ಎಂದು ನಿರ್ಧರಿಸಿ, ದೇವನಿಂದೆ ಯಿಂದ ಅರ್ಧ ಸತ್ತವನಂತಾಗಿದ್ದ ಅವನನ್ನು ಶಪಿಸಿ, ತಮ್ಮ ಹುಂಕಾರದಿಂದ ಕೊಂದು ಹಾಕಿದರು. ಆ ಸತ್ತ ದೇಹವನ್ನು ಅವನ ತಾಯಿಯಾದ ಸುನೀಥೆ, ತನ್ನ ಮಂತ್ರವಿದ್ಯೆಯ ಮಹಿಮೆಯಿಂದ ಕೆಡದಂತೆ ಕಾಪಾಡಿಕೊಂಡಿದ್ದಳು. ರಾಜನು ಸತ್ತು ಒಂದು ಬಗೆಯ ಕಂಟಕವೇನೋ ಪರಿಹಾರವಾಯಿತು, ಆದರೆ ರಾಜ್ಯವು ಅನಾಯಕವಾಯಿತು. ಜನರು ಒಬ್ಬರನ್ನೊಬ್ಬರು ಕೊಂದು ಆಸ್ತಿಗಳನ್ನು ಅಪಹರಿಸುತ್ತಿದ್ದರು. ದುಷ್ಟರನ್ನು ದಂಡಿಸು ವವರೆ ಇಲ್ಲದಂತಾಯಿತು. ಕಳ್ಳಕಾಕರ ಕೋಟಲೆಯಂತೂ ತಾನೇ ತಾನಾಯಿತು. ಇದನ್ನು ಕಂಡು ಋಷಿಗಳು ಮತ್ತೆ ಚಿಂತಾಕ್ರಾಂತರಾದರು. ಲೋಕ ರಕ್ಷಣೆಯಾಗದಿದ್ದರೆ ತಮ್ಮ ತಪಸ್ಸೆಲ್ಲ ಒಡೆದ ಮಡಕೆಯಲ್ಲಿನ ನೀರಿನಂತಾಗುವುದೆಂದು ಭಾವಿಸಿ, ಒಡನೆಯೇ ರಾಜ ನೊಬ್ಬನನ್ನು ಸಿಂಹಾಸನದ ಮೇಲೆ ಕೂಡಿಸಬೇಕೆಂದು ನಿಶ್ಚಯಿಸಿದರು. ಅತಿ ಪವಿತ್ರವಾದ ಅಂಗರಾಜನ ವಂಶವೇ ಮುಂದುವರಿಯುವಂತೆ ಮಾಡಬೇಕೆಂಬುದು ಅವರ ಆಶೆ. ಅದಕ್ಕಾಗಿ ವೇನನ ದೇಹವನ್ನು ತರಿಸಿ, ಅವನ ತೊಡೆಯನ್ನು ಕಡೆದರು. ಅದರಿಂದ ವಿಕಾರ ರೂಪಿನ ಒಬ್ಬ ಮನುಷ್ಯ ಹುಟ್ಟಿಬಂದ. ಕಾಗೆಯಂತೆ ಕಪ್ಪಾದ ಮೈ, ಕುಳ್ಳದೇಹ, ಮೋಟು ಕೈಗಳು, ಚಪ್ಪಟೆಯಾದ ಮೂಗು, ಉಬ್ಬಿದ ಗಲ್ಲ, ಚಿಕ್ಕ ಬಾಯಿ, ಕೆಂಡುಂಡೆಯಂತಹ ಕಣ್ಣು, ಕೆಂಪು ಕೂದಲು–ಇಂತಿರುವ ಆ ಬಾಹುಕ, ಅಲ್ಲಿದ್ದ ಋಷಿಗಳಿಗೆ ಭಕ್ತಿಯಿಂದ ನಮಸ್ಕರಿಸಿ “ಮಹಾಸ್ವಾಮಿ ನನಗೇನು ಅಪ್ಪಣೆ?” ಎಂದು ವಿನಯದಿಂದ ಬೇಡಿಕೊಂಡನು. ಋಷಿಗಳು ಅವನನ್ನು ‘ನಿಷೀದ’, (ಎಂದರೆ, ‘ಬೇಡ’) ಎಂದು ಕರೆದು, ಬೇಡರ ವಂಶಕ್ಕೆ ಮೂಲಪುರುಷನಾಗುವಂತೆ ನೇಮಿಸಿದರು.

ಬಾಹುಕನ ಜನನದೊಡನೆ ವೇನನ ದೇಹದ ಪಾಪವೆಲ್ಲವೂ ಹೊರಕ್ಕೆ ಬಂದಂತಾ ಯಿತು. ಅನಂತರ ಋಷಿಗಳು ವೇನನ ದೇಹದ ತೋಳುಗಳೆರಡನ್ನೂ ಕಡೆದರು. ಅವು ಗಳಿಂದ ಒಂದು ಗಂಡು, ಒಂದು ಹೆಣ್ಣು ಹುಟ್ಟಿ ಬಂದವು. ಗಂಡಿನ ಹೆಸರು ಪೃಥು, ಹೆಣ್ಣಿನ ಹೆಸರು ಅರ್ಚಿ. ಅವರು ನಾರಾಯಣ ಮತ್ತು ಲಕ್ಷ್ಮಿಯರ ಅಂಶದಿಂದ ಹುಟ್ಟಿ ದವರು. ಅವರು ಹುಟ್ಟುತ್ತಲೆ ದೇವತೆಗಳು ಹೂಮಳೆಯನ್ನು ಕರೆದರು, ಗಂಧರ್ವರು ಗಾನ ಮಾಡಿದರು, ಅಪ್ಸರೆಯರು ನರ್ತಿಸಿದರು. ಚತುರ್ಮುಖ ಬ್ರಹ್ಮನನ್ನು ಮುಂದಿಟ್ಟು ಕೊಂಡು ದೇವತೆಗಳೆಲ್ಲರೂ ಅಲ್ಲಿ ಬಂದು ನೆರೆದರು. ಬ್ರಹ್ಮನು ಪೃಥುವಿನ ಬಲಗೈಲಿದ್ದ ಚಕ್ರರೇಖೆಯನ್ನೂ ಪಾದಗಳಲ್ಲಿದ್ದ ಪದ್ಮರೇಖೆಯನ್ನೂ ತೋರಿಸಿ, ಆತನು ವಿಷ್ಣುವಿನ ಅಂಶವೆಂಬುದನ್ನು ನೆರೆದಿದ್ದವರಿಗೆಲ್ಲ ಮನದಟ್ಟು ಮಾಡಿಕೊಟ್ಟನು. ಋಷಿಗಳು ಆತ ನನ್ನು ಚಕ್ರವರ್ತಿಯನ್ನಾಗಿ ಮಾಡಬೇಕೆಂದು ನಿಶ್ಚಯಿಸಿದರು.

