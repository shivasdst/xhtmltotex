
\chapter{೫೦. ಧೇನುಕಾಸುರ}

ಕಾಲಚಕ್ರ ಉರುಳಿತು. ಮಕ್ಕಳಾಗಿದ್ದ ಬಲರಾಮಕೃಷ್ಣರು ಈಗ ಬೆಳೆದು ಬಾಲಕರಾಗಿ ದ್ದಾರೆ. ಅವರ ನಡೆ ನುಡಿಗಳಲ್ಲಿ ಬೇಕಾದಷ್ಟು ಬದಲಾವಣೆಯಾಗಿದೆ. ಶ್ರೀಕೃಷ್ಣನು ಕೊಳಲು ಬಾರಿಸುವುದರಲ್ಲಿ ಬಲು ಚದುರನಾಗಿದ್ದಾನೆ. ಅದರ ಗಾನಕ್ಕೆ ಅವನ ಗೆಳೆಯ ರಿರಲಿ, ದನಗಳು ಕೂಡ ಮೈಮರೆತು ಮುಗ್ಧವಾಗಿ ನಿಲ್ಲುತ್ತವೆ. ಆತನು ಗೋಪಾಲರ ನಾಯಕ. ಆತ ಕೊಳಲನ್ನು ಬಾರಿಸುತ್ತಾ ಹೊರಟನೆಂದರೆ ಗೋಪಾಲಕರೆಲ್ಲ ಆತನ ಸುತ್ತ ನೆರೆದು ಆತನನ್ನು ಹೊಗಳುತ್ತ ಸಂತೋಷಪಡಿಸಲು ಯತ್ನಿಸುವರು. ಪ್ರಕೃತಿ ಪ್ರೇಮಿ ಯಾದ ಶ್ರೀಕೃಷ್ಣ ಅಡವಿಯಲ್ಲಿ ಸುಂದರವಾದ ಪ್ರದೇಶಗಳನ್ನು ಆಯ್ದುಕೊಂಡು ತುರು ಗಳೊಡನೆ ಅಲ್ಲಿಗೆ ಹೋಗುವನು. ಪ್ರಕೃತಿಯ ಚೆಲುವನ್ನು ಹೀರಿ, ಹಿರಿಹಿರಿ ಹಿಗ್ಗುತ್ತಾ ಮನಬಂದಂತೆ ಅಲ್ಲಿ ವಿಹರಿಸುವನು. ಆತನಿಗೆ ಅಣ್ಣನಾದ ಬಲರಾಮನಲ್ಲಿ ಭಕ್ತಿ, ಗೌರವ. ಆತನನ್ನು ಕುರಿತು ‘ಅಣ್ಣ, ಈ ಮರಗಿಡಗಳನ್ನು ನೋಡು, ಪಾಪ, ಇದ್ದ ಕಡೆಯಿಂದ ಚಲಿಸುವಂತಿಲ್ಲ. ಯಾವ ಕರ್ಮದಿಂದಲೊ ಇವಕ್ಕೆ ಈ ಜನ್ಮ ಬಂದಿದೆ. ಈಗ ಆ ಪಾಪ ವನ್ನು ಕಳೆದುಕೊಳ್ಳುವುದಕ್ಕಾಗಿ ಹೂ ಹಣ್ಣುಗಳೆಂಬ ಕಾಣಿಕೆಗಳನ್ನು ಪೂಜ್ಯನಾದ ನಿನ್ನ ಪಾದಕ್ಕೆ ಒಪ್ಪಿಸುತ್ತಿವೆ. ಅವುಗಳ ಆ ಬಗ್ಗಿದ ಕೊನೆಗಳನ್ನು ನೋಡು, ನಿನಗೆ ನಮಸ್ಕರಿಸು ತ್ತಿರುವಂತೆ ಕಾಣುತ್ತವೆ. ಈ ದುಂಬಿಗಳು ನಿನ್ನ ಕೀರ್ತಿಯನ್ನು ಹಾಡುತ್ತಿರಬೇಕು; ದೈವಭಕ್ತರಾದ ಮಹರ್ಷಿಗಳೇ ಆ ರೂಪವನ್ನು ತಳೆದು ಬಂದಿರುವರೋ ಏನೊ! ಅಹಹ, ಆ ನವಿಲುಗಳನ್ನು ನೋಡು, ನಿನ್ನನ್ನು ನೋಡಿ ಸಂತೋಷದಿಂದ ಹೇಗೆ ಕುಣಿಯುತ್ತಾ ಇವೆ! ಅಣ್ಣ, ಆ ಹೆಣ್ಣು ಜಿಂಕೆಗಳು ಸುಂದರಿಯರಾದ ಗೋಪಿಕಾಕನ್ಯೆಯರಂತೆ ನಿನ್ನನ್ನು ಕುಡಿನೋಟದಿಂದ ನೋಡಿ ಆನಂದಪಡುತ್ತಿವೆ. ಇಲ್ಲಿನ ಗಿಡ ಮರ ಬಳ್ಳಿಗಳೆಲ್ಲ ನಿನ್ನ ಸ್ಪರ್ಷದಿಂದ ಪಾವನವಾದವು. ಇಡೀ ಬೃಂದಾವನವೆ ನಿನ್ನ ಕಾಲ್ತುಳಿತದಿಂದ ಉದ್ಧಾರವಾಗಿ ಹೋಯಿತು’ ಎಂದು ಹೊಗುಳುವನು. ಎಲ್ಲ ಕೆಲಸಗಳಲ್ಲಿಯೂ ಆತ ಅಣ್ಣನನ್ನೆ ಮುಂದಿಟ್ಟುಕೊಂಡು ಹೋಗುವನು. ಆತ ಬಳಲಿ ಮಲಗಿಕೊಂಡರೆ ತಾನು ಕೈಯಾರೆ ಆತನ ಪಾದಗಳನ್ನು ಒತ್ತಿ ಉಪಚರಿಸುವನು. ಇದನ್ನು ಕಂಡ ಗೋಪಾಲರು ಶ್ರೀಕೃಷ್ಣನಿಗೆ ದಣಿ ವಾಯಿತೆಂದರೆ ಆತನನ್ನು ತೊಡೆಯಮೇಲೆ ಮಲಗಿಸಿಕೊಂಡು, ಆತನ ಕಾಲೊತ್ತಿ ಉಪಚಾರ ಮಾಡುವರು.

ಹೀಗಿರಲು, ಒಂದು ದಿನ ಬಲರಾಮಕೃಷ್ಣರ ಜೀವದ ಗೆಳೆಯನಾದ ಶ್ರೀರಾಮನೆಂಬ ಗೋಪಾಲ ಅವರನ್ನು ಕುರಿತು ‘ಅಯ್ಯಾ, ಇಲ್ಲಿಗೆ ಸಮೀಪದಲ್ಲಿಯೆ ಒಂದು ದೊಡ್ಡ ತಾಳೆಯ ಹಣ್ಣಿನ ತೋಟವಿದೆ. ಅಲ್ಲಿ ಬೇಕಾದಷ್ಟು ಹಣ್ಣುಗಳು ನೆಲದ ಮೇಲೆಯೆ ಬಿದ್ದಿರು ತ್ತವೆ. ಆದರೇನು? ಅಲ್ಲಿ ಧೇನುಕನೆಂಬ ಒಬ್ಬ ರಕ್ಕಸ ಸೇರಿಕೊಂಡಿದ್ದಾನೆ. ಅವನು ಯಾರಿಗೂ ಅಲ್ಲಿಗೆ ಹೋಗುವುದಕ್ಕೆ ಅವಕಾಶ ಕೊಡುವುದಿಲ್ಲ. ಅವನೂ ಅವನ ಪರಿವಾರ ದವರೂ ಆ ಹಣ್ಣುಗಳನ್ನು ಯಥೇಚ್ಛವಾಗಿ ತಿನ್ನುತ್ತಾರೆ. ನನಗೆ ಆ ಹಣ್ಣುಗಳೆಂದರೆ ಪಂಚ ಪ್ರಾಣ; ಅದರ ಹೆಸರು ಹೇಳಿದರೆ ಸಾಕು ಬಾಯಲ್ಲಿ ನೀರೂರುತ್ತದೆ. ಹೇಗಾದರೂ ಮಾಡಿ ಆ ಹಣ್ಣುಗಳನ್ನು ತಿನ್ನಬೇಕಲ್ಲ! ಬಲರಾಮ, ನೀನು ಮನಸ್ಸು ಮಾಡಿದರೆ ಅದು ಸಾಧ್ಯ ವಪ್ಪ’ ಎಂದನು. ಅವನ ಮಾತುಗಳನ್ನು ಕೇಳಿ ಬಲರಾಮಕೃಷ್ಣರಿಬ್ಬರೂ ಆ ತೋಟದ ಕಡೆಗೆ ಹೊರಟರು. ಗೋಪಾಲರ ತಂಡ ಅವರನ್ನು ಅನುಸರಿಸಿತು. ಬಲರಾಮನು ಮದ್ದಾನೆ ಯಂತೆ ತೋಟದೊಳಕ್ಕೆ ನುಗ್ಗಿ, ನಾಲ್ಕಾರು ಮರಗಳನ್ನು ಬಲವಾಗಿ ಅಲ್ಲಾಡಿಸಿ, ಹಣ್ಣುಗಳ ಮಳೆಯನ್ನೆ ಸುರಿಸಿದನು. ಹಣ್ಣುಗಳು ಬಿದ್ದ ಸದ್ದನ್ನು ಕೇಳುತ್ತಲೆ ಧೇನುಕಾಸುರನು ಬಿರುಗಾಳಿಯಂತೆ ಅಲ್ಲಿಗೆ ನುಗ್ಗಿಬಂದನು. ಆ ರಕ್ಕಸ ಕತ್ತೆಯ ಆಕಾರದಲ್ಲಿದ್ದ; ಆದ್ದರಿಂದ ಕತ್ತೆಯಂತೆಯೆ ಕೂಗುತ್ತಾ, ತನ್ನ ಹಿಂಗಾಲುಗಳಿಂದ ಬಲರಾಮನ ಎದೆಗೆ ಝಾಡಿಸಿ ಒದೆದನು. ಬಲರಾಮನು ಅದನ್ನು ಲಕ್ಷಿಸದೆ ಆ ರಕ್ಕಸನ ಹಿಂಗಾಲುಗಳೆರಡನ್ನೂ ಹಿಡಿದು ಕೊಂಡು ಅದನ್ನು ಗಿರಿಗಿರಿ ತಿರುಗಿಸಿ, ತಾಳೆಯ ಮರಕ್ಕೆ ಬೀಸಿ ಬಡಿದನು. ಆ ಪೆಟ್ಟಿಗೆ ಮರ ಎರಡು ತುಂಡಾಗಿ ಬಿತ್ತು; ಇಷ್ಟೇ ಅಲ್ಲ, ಆ ಮುರಿದ ಮರ ತಗಲಿ ಮತ್ತೊಂದು ಮರ ಮುರಿಯಿತು; ಅದು ತಗಲಿ ಇನ್ನೊಂದು ಮುರಿಯಿತು. ಹೀಗೆ ಅನೇಕ ತಾಳೆಯ ಮರಗಳು ಮುರಿದು ಬಿದ್ದವು.

ಧೇನುಕಾಸುರ ಸತ್ತು ಮರಗಳು ಮುರಿದು ಬೀಳುತ್ತಿದ್ದ ಹಾಗೆಯೆ ಆ ರಕ್ಕಸನ ಪರಿವಾರ ದವರೂ ನೂರಾರು ಮಂದಿ ಅಲ್ಲಿ ಬಂದು ನೆರೆದರು. ಎಲ್ಲರ ಆಕಾರಗಳೂ ಕತ್ತೆಯವೆ. ಅವರೆಲ್ಲರೂ ತಮ್ಮ ಹಿಂಗಾಲುಗಳನ್ನು ಹಾರಿಸುತ್ತಾ ರಾಮಕೃಷ್ಣರನ್ನು ಕೊಲ್ಲಲೆಳಸಿದರು. ಆದರೆ ಆ ಸೋದರರು ಅವರುಗಳಿಗೆಲ್ಲ ಧೇನುಕಾಸುರನ ಗತಿಯನ್ನೆ ಕಾಣಿಸಿ, ಧ್ವಂಸ ಮಾಡಿದರು. ಅಂದಿನಿಂದ ಆ ತಾಳೆಯ ವನ ಗೋಪಾಲಕರ ಸ್ವತ್ತಾಯಿತು. ಅವರು ತಮ್ಮ ಮನಬಂದಷ್ಟು ಹಣ್ಣುಗಳನ್ನು ತಿಂದು, ಹೊರುವಷ್ಟನ್ನು ಹೊತ್ತು ಮನೆಗೆ ಹಿಂದಿರುಗಿ ದರು. ಶ್ರೀಕೃಷ್ಣನು ಕೊಳಲೂದುತ್ತಾ ಮುಂದೆ ಹೊರಟನು. ಅವನ ಗೆಳೆಯರು ಅವನ ಕೀರ್ತಿಯನ್ನು ಹಾಡುತ್ತಾ ಅವನ ಹಿಂದೆ ಹೊರಟರು. ಅವರು ಗೋಕುಲವನ್ನು ಸೇರಿದಾಗ, ಅಲ್ಲಿನ ಗೋಪಿಯರೆಲ್ಲ ಮುದ್ದುಕೃಷ್ಣನನ್ನು ಕಂಡು ಆನಂದಪಟ್ಟರು. ಕಮಲದ ಬಂಡನ್ನು ಕುಡಿಯಲು ಆತುರಪಡುವ ದುಂಬಿಗಳಂತೆ ಅವರ ಕಣ್ಣುಗಳು ಶ್ರೀಕೃಷ್ಣನ ಮುದ್ದುಮುಖವನ್ನು ನೋಡಲು ಕಾತರಿಸುತ್ತಿದ್ದವು. ದಿನ ದಿನವೂ ಆತನು ನಡೆಸುತ್ತಿದ್ದ ಅದ್ಭುತ ಲೀಲೆಗಳನ್ನು ಕೇಳಿ ಕೇಳಿ, ಅವರ ಮನಸ್ಸುಗಳೆಲ್ಲ ಆತನಲ್ಲಿ ಪರವಶವಾಗಿ ಹೋಗಿ ದ್ದವು.

