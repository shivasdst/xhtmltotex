
\chapter{೯೩. ಭೂಭಾರವಿಳಿಯಿತು, ನನ್ನ ಕೆಲಸ ಮುಗಿಯಿತು}

ಉದ್ಧವನು ಅತ್ತ ಹೋಗುತ್ತಲೆ ಇತ್ತ ಶ್ರೀಕೃಷ್ಣನು ತನಗಾಗಿ ಸಿದ್ಧರಾಗಿ ಕಾದಿದ್ದ ಯಾದವರೊಡನೆ ನಾವೆಯಲ್ಲಿ ಸಮುದ್ರವನ್ನು ದಾಟಿ, ಅಲ್ಲಿಂದ ಮುಂದೆ ರಥದಲ್ಲಿ ಪ್ರಯಾಣಮಾಡಿ, ಪ್ರಭಾಸಕ್ಷೇತ್ರಕ್ಕೆ ಹೋದನು. ಅಲ್ಲಿ ಸರಸ್ವತೀನದಿಯು ಪಶ್ಚಿಮದಿಕ್ಕಿ ನಲ್ಲಿ ಹರಿದು ಸಮುದ್ರವನ್ನು ಸೇರುತ್ತದೆ. ಪವಿತ್ರವಾದ ಆ ಸಂಗಮಸ್ಥಳದಲ್ಲಿ ಯಾದವ ರೆಲ್ಲ ಶ್ರೀಕೃಷ್ಣನ ಅಪ್ಪಣೆಯಂತೆ ತಮ್ಮ ಶ್ರೇಯಸ್ಸಿಗಾಗಿ ಪೂಜಾಕಾರ್ಯಗಳನ್ನು ನಡೆಸಿ, ಅನೇಕ ದಾನಧರ್ಮಗಳನ್ನು ನಡೆಸಿದರು. ಏನು ಮಾಡಿದರೇನು? ಶ್ರೀಕೃಷ್ಣನ ಸಂಕಲ್ಪಕ್ಕೆ ಅನುಸಾರಿಯಾದ ಪುಷಿಶಾಪ ತನ್ನ ಪ್ರತಾಪವನ್ನು ತೋರಿಸಿಯೇಬಿಟ್ಟಿತು. ಪೂಜಾದಿಕಾರ್ಯ ಗಳೆಲ್ಲ ಸಾಂಗವಾಗಿ ನೆರವೇರಿ ಸಂತೋಷಚಿತ್ತದಲ್ಲಿದ್ದ ಯಾದವರು ‘ವಿನಾಶಕಾಲಕ್ಕೆ ವಿಪ ರೀತ ಬುದ್ಧಿ’ ಎಂಬಂತೆ ಮೈರೇಯವೆಂಬ ಅತ್ಯಂತ ರುಚಿಕರವಾದ ಮದ್ಯವನ್ನು ಮಿತಿ ಮೀರಿ ಕುಡಿದು ಮೈಮರೆತರು; ಮದ್ಯದ ಮದಕ್ಕೆ ಶ್ರೀಕೃಷ್ಣನ ಮಾಯೆ ಸೇರಿ, ಅವರೆಲ್ಲ ಪರಸ್ಪರ ಜಗಳಕ್ಕೆ ಪ್ರಾರಂಭಿಸಿದರು. ಬಾಯಿಜಗಳ ಬಹುಬೇಗ ಕೈಗೆ ಕೈ ಹೋರಾಟ ವಾಯಿತು. ಬರಬರುತ್ತಾ ಅದು ಬಾಣ, ಕತ್ತಿ, ಗುರಾಣಿಗಳ ದೊಡ್ಡ ಯುದ್ಧವಾಗಿ ಪರಿಣಮಿ ಸಿತು. ಸಮುದ್ರತೀರದಲ್ಲಿ ನಡೆದ ಆ ಸ್ವಯೂಥಕಲಹದಲ್ಲಿ ನೆತ್ತರಿನ ಕೋಡಿ ಹರಿಯಿತು. ಮದಿಸಿದ ಆನೆ ಕಾಡನ್ನು ನುಗ್ಗಿ ಕೈಗೆ ಸಿಕ್ಕ ಮರವನ್ನು ಕಿತ್ತೆಸೆಯುವಂತೆ ಮದಿಸಿದ ವೀರರು ಜನರ ಗುಂಪಿನಲ್ಲಿ ನುಗ್ಗಿ ಕೈಗೆ ಸಿಕ್ಕಿದವರನ್ನು ಕೊಂದು ಹಾಕಿದರು. ಪ್ರದ್ಯುಮ್ನ-ಸಾಂಬ, ಅಕ್ರೂರ-ಭೋಜ, ಅನಿರುದ್ಧ-ಸಾತ್ಯಕಿ ಹೀಗೆ ಪ್ರತಿದ್ವಂದ್ವಿಗಳಾಗಿ ನಿಂತು, ಪರಸ್ಪರ ಬಡಿ ದಾಡಿ, ಇಬ್ಬರೂ ಸತ್ತುಬಿದ್ದರು. ಮದ್ಯದ ಮದವೇರಿದ ಅವರಿಗೆ ಉಚಿತಾನುಚಿತದ ಪ್ರಶ್ನೆಯೇ ಇರಲಿಲ್ಲ. ತಂದೆ, ಮಗ, ಸೋದರ, ಬಂಧು, ಗೆಳೆಯ–ಎಂಬ ವಿವೇಕ ಅವರ ತಲೆಯಿಂದ ಹಾರಿಹೋಗಿತ್ತು. ಕಾದುವುದಕ್ಕಾಗಿಯೆ ಕಾದುತ್ತಿದ್ದರು, ಕೊಲ್ಲುವುದಕ್ಕಾಗಿಯೆ ಕೊಂದುಹಾಕುತ್ತಿದ್ದರು. ಯುದ್ಧಮಾಡಿ ಮಾಡಿ ಅವರ ಆಯುಧಗಳೆಲ್ಲ ತೀರಿಹೋದವು. ಆಗ ಅವರು ಸಮುದ್ರತೀರದಲ್ಲಿ ಬೆಳೆದಿದ್ದ ಜೊಂಡು ಹುಲ್ಲನ್ನು ಹಿಡಿಹಿಡಿಯಾಗಿ ಕಿತ್ತು ಕೈಲಿ ಹಿಡಿದರು. ಹಾಗೆ ಹಿಡಿಯುತ್ತಲೆ ಅವು ವಜ್ರಸಮಾನವಾದ ಲೋಹದಂಡಗಳಾದವು. ಅವರನ್ನು ತಡೆಯಹೋದ ಶ್ರೀಕೃಷ್ಣನನ್ನೆ ಹೊಡೆಯಹೊರಟರು. ತಲೆಕೆಟ್ಟ ಆ ಜನರಿಗೆ ಬಲರಾಮ ಕೃಷ್ಣರಿಬ್ಬರೂ ತಮ್ಮ ಪರಮ ವೈರಿಗಳಂತೆ ಕಾಣಿಸಿದರು. ಅವರು ನಿರ್ದಾಕ್ಷಿಣ್ಯ ವಾಗಿ ಅವರ ಮೇಲೆ ಏರಿಬಂದರು. ಆಗ ಅನಿವಾರ್ಯವಾಗಿ ಬಲರಾಮಕೃಷ್ಣರೂ ಜೊಂಡು ಹುಲ್ಲನ್ನು ಕಿತ್ತುಕೊಂಡು, ಅವರನ್ನು ನಿವಾರಿಸಬೇಕಾಯಿತು. ಬಿದಿರುಮೆಳೆಯಲ್ಲಿ ಹುಟ್ಟಿದ ಬೆಂಕಿ ವನವನ್ನೆಲ್ಲ ಸುಡುವಂತೆ ಸಾಂಬನ ಹುಡುಗಾಟ ಯಾದವ ಕುಲವನ್ನು ನಾಶ ಮಾಡಿತು. ಶ್ರೀಕೃಷ್ಣನು ತನ್ನ ಅಂತರಂಗದಲ್ಲಿ ‘ಭೂಭಾರವಿಳಿಯಿತು, ನನ್ನ ಕೆಲಸ ಮುಗಿ ಯಿತು’ ಎಂದು ಸಂತೋಷಪಟ್ಟನು.

ತನ್ನ ಹೊಣೆ ಮುಗಿಯಿತೆಂದು ಶ್ರೀಕೃಷ್ಣನು ಸಂತೋಷಿಸುತ್ತಿರಲು, ಬಲರಾಮನು ತನ್ನ ಅವತಾರ ಕಾರ್ಯ ಮುಗಿಯಿತೆಂದು ನಿರ್ಧರಿಸಿ, ಸಮುದ್ರತೀರದಲ್ಲಿ ಕುಳಿತು, ತನ್ನ ಯೋಗ ಶಕ್ತಿಯಿಂದ ತನ್ನ ಜೀವಚೈತನ್ಯವನ್ನು ಪರಮಾತ್ಮನಲ್ಲಿ ವಿಲೀನಗೊಳಿಸಿದನು. ಅದನ್ನು ಕಂಡು ಶ್ರೀಕೃಷ್ಣಪರಮಾತ್ಮನು ಅಲ್ಲಿಯೆ ಸಮೀಪದಲ್ಲಿದ್ದ ಒಂದು ಆಲದಮರದ ಕೆಳಗೆ ಮೌನದಿಂದ ಕುಳಿತನು. ಆಗ ಆತನು ಚತುರ್ಭುಜಗಳಿಂದ ಕೂಡಿದ ದಿವ್ಯರೂಪವನ್ನು ಧರಿಸಿದ್ದನು. ಆತನ ತೇಜಸ್ಸು ಹೊಗೆಯಿಲ್ಲದ ಯಜ್ಞೇಶ್ವರನಂತೆ ದಶದಿಕ್ಕುಗಳನ್ನೂ ಬೆಳಗುತ್ತಿತ್ತು. ನೀರುಂಡ ಮೇಘದಂತೆ ಶ್ಯಾಮಲವಾದ ದೇಹ, ಎದೆಯಲ್ಲಿ ಶ್ರೀವತ್ಸ ವೆಂಬ ಮಚ್ಚೆ, ಚಿನ್ನದಂತೆ ಹೊಳೆಯುವ ಪೀತಾಂಬರ, ಅರಳಿದ ಕಮಲದಂತೆ ಮಂದ ಹಾಸವಾದ ಮುಖ, ಕಪ್ಪಾದ ಮುಂಗುರುಳು, ಮನೋಹರವಾದ ಕಣ್ಣುಗಳು–ಈ ಲೋಕ ಮೋಹಕವಾದ ಮೂರ್ತಿಯ ಸುತ್ತ ಶಂಖ, ಚಕ್ರ ಮೊದಲಾದ ಆಯುಧಗಳು ಪುರುಷಾ ಕಾರದಿಂದ ಆತನನ್ನು ಸೇವಿಸುತ್ತಿವೆ. ಶ್ರೀಕೃಷ್ಣನು ತನ್ನ ಬಲಗಾಲನ್ನು ಎಡತೊಡೆಯ ಮೇಲಿಟ್ಟು ಮೌನದಿಂದ ಕುಳಿತಿದ್ದಾನೆ. ಆಗ ಜರೆಯೆಂಬ ಬೇಡನು ಬೇಟೆಯಾಡುತ್ತಾ ಅತ್ತಕಡೆ ಬಂದನು. ಸಮುದ್ರದಲ್ಲಿ ಬಿಸುಟ ಒನಕೆಯ ಗೊಣಸಿನ ಚೂರು ಇದ್ದುದು ಇವನ ಬಳಿಯಲ್ಲಿಯೆ. ಬೇಟೆಗೆಂದು ಹೊರಟಿದ್ದ ಅವನಿಗೆ ಗಿಡಗಳ ಮಧ್ಯದಿಂದ ಕಾಣುತ್ತಿದ್ದ ಶ್ರೀಕೃಷ್ಣನ ಬಲಪಾದ ಜಿಂಕೆಯ ಮುಖದಂತೆ ಭಾಸವಾಯಿತು. ಅವನು ತನ್ನ ಬಾಣವನ್ನು ಗುರಿಯಿಟ್ಟು ಆ ಪಾದಕ್ಕೆ ಹೊಡೆದನು. ಅನಂತರ ಅವನು ಹತ್ತಿರಕ್ಕೆ ಬಂದು ನೋಡು ತ್ತಾನೆ, ಚತುರ್ಭುಜನಾದ ದಿವ್ಯಮೂರ್ತಿ! ಆ ಬೇಡನು ಭಯದಿಂದ ಗಡಗಡ ನಡುಗುತ್ತಾ ಆತನಿಗೆ ಅಡ್ಡಬಿದ್ದು, ತನ್ನನ್ನು ಕ್ಷಮಿಸುವಂತೆ ಬೇಡಿಕೊಂಡನು. ಆಗ ಶ್ರೀಕೃಷ್ಣನು ಮುಗುಳ್​ನಗುತ್ತಾ ‘ಅಯ್ಯಾ ಜರೆ, ನೀನೇನೂ ಭಯಪಡಬೇಡ. ನನ್ನ ಸಂಕಲ್ಪಕ್ಕೆ ತಕ್ಕಂತೆ ನೀನು ಈ ಕೆಲಸ ಮಾಡಿದ್ದಿ. ಇದರಲ್ಲಿ ನಿನ್ನ ತಪ್ಪೇನೂ ಇಲ್ಲ. ನಿನಗೆ ಸ್ವರ್ಗಪದವಿಯನ್ನು ನೀಡಿದ್ದೇನೆ, ನೀನೀಗಲೇ ಹೊರಡು’ ಎಂದು ಹೇಳಿದನು. ಆತನು ಹಾಗೆ ಹೇಳುತ್ತಿರು ವಂತೆಯೆ ವಿಮಾನವೊಂದು ಕೆಳಕ್ಕಿಳಿದು ಬಂದಿತು. ಆ ಬೇಡನು ಶ್ರೀಕೃಷ್ಣನಿಗೆ ಮೂರುಸಲ ಪ್ರದಕ್ಷಿಣ ನಮಸ್ಕಾರವನ್ನು ಮಾಡಿ, ವಿಮಾನವನ್ನೇರಿ ಸ್ವರ್ಗಕ್ಕೆ ಹೋದನು.

ಬೇಡನು ಅತ್ತ ಹೋಗುತ್ತಲೆ ಇತ್ತ ಶ್ರೀಕೃಷ್ಣನ ಸಾರಥಿಯಾದ ದಾರುಕನು ತನ್ನ ಸ್ವಾಮಿಯನ್ನು ಅರಸುತ್ತಾ ಅಲ್ಲಿಗೆ ಬಂದನು. ಯೋಗಾಸನದಲ್ಲಿ ಕುಳಿತು ಆತ್ಮಾನಂದ ದಲ್ಲಿ ಮುಳುಗಿದ್ದ ಆತನನ್ನು ಕಾಣುತ್ತಲೆ ದಾರುಕನು ರಥದಿಂದ ಕೆಳಕ್ಕಿಳಿದು, ಕಣ್ಣುಗಳಲ್ಲಿ ಆನಂದಬಾಷ್ಪವನ್ನು ಸುರಿಸುತ್ತಾ, ಆತನಿಗೆ ಅಡ್ಡಬಿದ್ದು, ‘ದೇವದೇವ, ನಿನ್ನ ಪಾದಗಳನ್ನು ಕಾಣದೆ ನನಗೆ ಕಗ್ಗತ್ತಲೆ ಮುಸುಕಿದಂತಾಗಿತ್ತು. ನನ್ನ ಹೃದಯವು ಚಂದ್ರನಿಲ್ಲದ ರಾತ್ರಿ ಯಂತಾಗಿತ್ತು’ ಎಂದನು. ಅವನು ಮಾತನಾಡುತ್ತಿದ್ದಂತೆಯೆ, ಶ್ರೀಕೃಷ್ಣನ ರಥವು ಕುದುರೆಗಳೊಡನೆ ಆಕಾಶಕ್ಕೆ ಹಾರಿಹೋಯಿತು. ಅದರ ಹಿಂದೆಯೆ ಶಂಖ, ಚಕ್ರ ಇತ್ಯಾದಿ ಆಯುಧಗಳು ಹೊರಟುಹೋದವು. ಇದನ್ನು ಕಂಡು ಅಚ್ಚರಿಯಿಂದ ಮೂಕನಂತೆ ನಿಂತಿದ್ದ ದಾರುಕನನ್ನು ಕುರಿತು ಶ್ರೀಕೃಷ್ಣನು ‘ಅಯ್ಯಾ ಸೂತ, ನೀನೀಗಲೆ ದ್ವಾರಕಿಗೆ ಹೋಗಿ, ಯಾದವಕುಲ ನಾಶವನ್ನೂ, ಬಲರಾಮನಿರ್ಯಾಣವನ್ನೂ, ಈಗಿನ ನನ್ನ ಸ್ಥಿತಿ ಯನ್ನೂ ನನ್ನ ಬಂಧುಗಳಿಗೆಲ್ಲ ತಿಳಿಸು. ಅವರು ಈಗಿಂದೀಗಲೆ ದ್ವಾರಕಿಯನ್ನು ಬಿಟ್ಟು ಹೊರಟುಹೋಗುವಂತೆ ತಿಳಿಸು. ಇನ್ನು ಸ್ವಲ್ಪಕಾಲಕ್ಕೆ ದ್ವಾರಕಿಯನ್ನು ಸಮುದ್ರ ನುಂಗಿ ಹಾಕುತ್ತದೆ. ಆದ್ದರಿಂದ ಅಲ್ಲಿರುವವರೆಲ್ಲ ತಮ್ಮ ಪರಿವಾರ ಸಾಮಗ್ರಿಗಳೊಡನೆ ನನ್ನ ತಾಯ್ತಂದೆಗಳನ್ನೂ ಕರೆದುಕೊಂಡು, ಅರ್ಜುನನೊಡನೆ ಇಂದ್ರಪ್ರಸ್ಥಕ್ಕೆ ಹೋಗಲಿ. ನೀನು ಭಾಗವತಧರ್ಮವನ್ನು ಕೈಕೊಂಡು ಶಾಂತನಾಗಿರು. ನೀನಿನ್ನು ಹೊರಡು’ ಎಂದನು. ದಾರುಕನು ಆತನನ್ನು ಅಗಲಲಾರದೆ ಕಣ್ಣೀರು ಸುರಿಸುತ್ತಾ, ಮತ್ತೆ ಮತ್ತೆ ಅಡ್ಡಬಿದ್ದು, ಅತಿ ದುಃಖದಿಂದ ದ್ವಾರಕಿಗೆ ಹೊರಟುಹೋದನು.

ದಾರುಕನು ಹೊರಟುಹೋಗುತ್ತಿದ್ದಂತೆಯೆ ಚತುರ್ಮುಖಬ್ರಹ್ಮನೂ, ಪಾರ್ವತೀ ಸಮೇತನಾದ ಶಂಕರನೂ, ಇಂದ್ರಾದಿ ಲೋಕಪಾಲಕರೂ, ಮರೀಚಿಯೇ ಮೊದಲಾದ ಮಹರ್ಷಿಗಳೂ ಶ್ರೀಕೃಷ್ಣನ ಬಳಿಗೆ ಬಂದರು. ಅವರ ಹಿಂದೆಯೆ ಗಂಧರ್ವರು ಬಂದು ಗಾನಮಾಡಿದರು, ಅಪ್ಸರೆಯರು ಬಂದು ನರ್ತನ ಮಾಡಿದರು, ಸಿದ್ಧವಿದ್ಯಾಧರರು ಶ್ರೀಕೃಷ್ಣನ ಗುಣಕೀರ್ತನೆ ಮಾಡಿದರು. ಆಕಾಶದಿಂದ ಹೂಮಳೆ ಸುರಿಯಿತು. ನೆರದವರೆಲ್ಲ ‘ಜಯ ಜಯ ಶ್ರೀಕೃಷ್ಣಪರಮಾತ್ಮ’ ಎಂದು ಜಯಘೋಷ ಮಾಡಿದರು. ಶ್ರೀಕೃಷ್ಣನು ಒಮ್ಮೆ ಪ್ರಸನ್ನದೃಷ್ಟಿಯಿಂದ ಅವರೆಲ್ಲರನ್ನೂ ನೋಡಿದನು. ಅನಂತರ ತನ್ನ ಚಿತ್ತವನ್ನು ಆತ್ಮನಲ್ಲಿ ಲಯಗೊಳಿಸಿ, ತನ್ನ ಕಮಲನೇತ್ರಗಳನ್ನು ಮುಚ್ಚಿಕೊಂಡನು. ಆತನು ತನ್ನ ದಿವ್ಯಸುಂದರವಾದ ದೇಹವನ್ನು ಯೋಗಧಾರಣೆಯಿಂದ ಅಗ್ನಿಯಿಂದ ದಹಿಸದೆ, ಆ ಶರೀರದೊಡನೆಯೆ ವೈಕುಂಠಕ್ಕೆ ತೆರಳಿದನು. ಒಡನೆಯೆ ದೇವದುಂದುಭಿಗಳು ಮೊಳಗಿ ದವು, ಹೂಮಳೆ ಸುರಿಯಿತು. ಶ್ರೀಕೃಷ್ಣನ ಸೇವೆಗಾಗಿ ಈವರೆಗೆ ಭೂಮಿಯಲ್ಲಿ ನೆಲಸಿದ್ದ ಸತ್ಯ, ಧರ್ಮ, ಧೃತಿ, ಕೀರ್ತಿಗಳು ಆತನ ಹಿಂದೆಯೇ ವೈಕುಂಠಕ್ಕೆ ತೆರಳಿದವು. ಅಲ್ಲಿ ನೆರೆದಿದ್ದ ದೇವಾನುದೇವತೆಗಳೆಲ್ಲ ಶ್ರೀಕೃಷ್ಣನು ಹೋದಹಾದಿಯನ್ನು ಅರಿಯಲು ಪ್ರಯ ತ್ನಿಸಿ ವಿಫಲರಾದರು. ಅವರೆಲ್ಲ ಶ್ರೀಹರಿಯ ಯೋಗಪ್ರಸ್ಥಾನದ ಅಚ್ಚರಿಯನ್ನು ಕೊಂಡಾ ಡುತ್ತಾ ತಮ್ಮ ತಮ್ಮ ಲೋಕಗಳಿಗೆ ಹೊರಟುಹೋದರು.

ಅತ್ತ ದ್ವಾರಕಿಯನ್ನು ಸೇರಿದ ದಾರುಕನು ವಸುದೇವ, ಉಗ್ರಸೇನರ ಪಾದಗಳಿಗೆ ಅಡ್ಡ ಬಿದ್ದು, ಶ್ರೀಕೃಷ್ಣನ ಸಂದೇಶವನ್ನು ಅವರಿಗೆ ಅರುಹಿದನು. ಅದನ್ನು ಕೇಳಿ ಅವರು ದುಃಖದಿಂದ ಹಣೆಹಣೆ ಬಡಿದುಕೊಳ್ಳುತ್ತಾ, ದ್ವಾರಕಿಯಿಂದ ಹೊರಟು ನೇರವಾಗಿ ಪ್ರಭಾಸಕ್ಷೇತಕ್ಕೆ ಬಂದರು. ವಸುದೇವನೂ ದೇವಕೀರೋಹಿಣಿಯರೂ ಬಲರಾಮಕೃಷ್ಣ ರನ್ನು ಕಾಣದೆ, ಅವರ ಅಗಲಿಕೆಯ ದುಃಖದಿಂದ ಮೃತಿ ಹೊಂದಿದರು. ಹೆಣ್ಣುಗಳೆಲ್ಲ ತಮ್ಮ ಗಂಡಂದಿರ ದೇಹಗಳನ್ನು ಅಪ್ಪಿಕೊಂಡು ಅಗ್ನಿಪ್ರವೇಶದಿಂದ ಸಹಗಮನ ಮಾಡಿ ದರು. ಶ್ರೀಕೃಷ್ಣನ ಮಡದಿಯರು ತಮ್ಮ ಗಂಡನಲ್ಲಿ ಮನಸ್ಸನ್ನು ನಿಲ್ಲಿಸಿ ಅಗ್ನಿಯಲ್ಲಿ ಅನುಗಮನಮಾಡಿದರು. ಆ ಸಮಯದಲ್ಲಿ ಅಲ್ಲಿಯೆ ಇದ್ದ ಅರ್ಜುನನು, ತನಗೆ ಪರಮ ಪ್ರಿಯ ನಾದ ಶ್ರೀಕೃಷ್ಣನ ಅಗಲಿಕೆಯಿಂದ ಮನಸ್ಸು ಕತಕತ ಕುದಿಯುತ್ತಿದ್ದರೂ, ಆತನ ಗೀತೋಪದೇಶವನ್ನು ಜ್ಞಾಪಿಸಿಕೊಂಡು, ಶಾಂತಿಯನ್ನು ತಂದುಕೊಂಡನು; ಸತ್ತವರಿಗೆಲ್ಲ ಪ್ರೇತಕರ್ಮಗಳನ್ನು ಮಾಡಿಸಿದನು. ಮಕ್ಕಳಿಲ್ಲದೆ ಸತ್ತವರಿಗೆಲ್ಲ ತಾನೆ ಆ ಕರ್ಮಗಳನ್ನು ಮಾಡಿ ಮುಗಿಸಿದನು. ಅಷ್ಟರಲ್ಲಿ ಸಮುದ್ರವು ಉಕ್ಕಿ, ಶ್ರೀಕೃಷ್ಣನ ಅರಮನೆಯೊಂದನ್ನು ಹೊರತು ಉಳಿದ ದ್ವಾರಕಾನಗರವನ್ನೆಲ್ಲ ಕೊಚ್ಚಿಹಾಕಿತು. ಸ್ಮರಣಮಾತ್ರದಿಂದ ಸಕಲ ಪಾಪಗಳನ್ನೂ ಕಳೆಯುವ ಪರಮಾತ್ಮನಿದ್ದ ಮನೆಯಲ್ಲವೆ ಅದು? ಅದು ಕೊಚ್ಚಿಹೋಗದೆ ಉಳಿದಿರುವುದರಿಂದ ಭಗವಂತನಾದ ಶ್ರೀಕೃಷ್ಣನು ಅಲ್ಲಿ ಸನ್ನಿಹಿತನಾಗಿರುವನೆಂದೇ ತಿಳಿಯಬೇಕು.

ದ್ವಾರಕಿಯು ಸಮುದ್ರದಲ್ಲಿ ಮುಳುಗಿಹೋಗುತ್ತಲೆ ಅರ್ಜುನನು ಅಳಿದುಳಿದಿದ್ದ ಹೆಂಡಿರು ಮಕ್ಕಳನ್ನೂ ಮುಪ್ಪಿನ ಮುದುಕರನ್ನೂ ತನ್ನೊಡನೆ ಇಂದ್ರಪ್ರಸ್ಥಕ್ಕೆ ಕರೆದೊ ಯ್ದನು; ಅಲ್ಲಿ ಅನಿರುದ್ಧನ ಮಗನಾದ ವಜ್ರನಿಗೆ ಮಧುರಾನಗರಿಯ ರಾಜನಾಗಿ ಪಟ್ಟ ಕಟ್ಟಿದನು. ಇದಾದ ಕೆಲವು ದಿನಗಳಲ್ಲಿಯೆ ಪಾಂಡವರು ಪರೀಕ್ಷಿತನಿಗೆ ಪಟ್ಟ ಕಟ್ಟಿ, ತಾವು ಮಹಾಪ್ರಸ್ಥಾನವನ್ನು ಕೈಕೊಂಡರು.

