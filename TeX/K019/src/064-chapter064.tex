
\chapter{೬೪. ಮಹಾತ್ಮನಾದ ಮುಚುಕುಂದ}

ದ್ವಾರಕಿಯಿಂದ ಮಧುರೆಗೆ ಹಿಂದಿರುಗಿದ ಶ್ರೀಕೃಷ್ಣ ಮಧುರಾಪುರದ ರಕ್ಷಣೆಗೆ ಅಣ್ಣ ನನ್ನು ನೇಮಿಸಿ, ತಾನೊಬ್ಬನೆ ಬರಿಯ ಕೈಲಿ ಊರ ಹೆಬ್ಬಾಗಿಲನ್ನು ದಾಟಿ ಹೊರಗೆ ಬಂದ. ಅಲ್ಲಿಯೆ ಬೀಡುಬಿಟ್ಟುಕೊಂಡಿದ್ದ ಕಾಲಯವನನು ಆತನನ್ನು ಕಂಡು ಅಚ್ಚರಿಗೊಂಡ. ನಾರದರು ತನಗೆ ವರ್ಣಿಸಿದ ಶ್ರೀಕೃಷ್ಣ ಈತನೇ ಇರಬೇಕು ಎಂದುಕೊಂಡ, ಅವನು. ಆಗತಾನೆ ಹುಟ್ಟುತ್ತಿರುವ ಪೂರ್ಣಚಂದ್ರನಂತಹ ಮುಖ, ಮೈಯೆಲ್ಲ ಶಾಮಲವರ್ಣ, ಹೊಂಬಣ್ಣದ ಪೀತಾಂಬರವನ್ನು ಉಟ್ಟಿದ್ದಾನೆ, ಎದೆಯಲ್ಲಿ ಶ್ರೀವತ್ಸವೆಂಬ ಮಚ್ಚೆ, ತುಂಬಿಕೊಂಡಿರುವ ದುಂಡನೆಯ ನಿಡಿದಾದ ತೋಳುಗಳು, ಕಮಲದಂತಿರುವ ಕಣ್ಣು ಗಳು, ಕನ್ನಡಿಯಂತಹ ಕೆನ್ನೆಗಳು, ಕಿವಿಯಲ್ಲಿ ಹೊಳೆಯುವ ಕುಂಡಲಗಳು, ತುಟಿಗಳಲ್ಲಿ ಕಾಣಿಸುವ ಎಳೆನಗೆ–ಆತನು ಶ್ರೀಕೃಷ್ಣನೆಂಬ ವಿಚಾರದಲ್ಲಿ ಕಾಲಯವನನಿಗೆ ಎಳ್ಳಷ್ಟೂ ಸಂದೇಹವಿಲ್ಲವಾಯಿತು. ‘ಓಹೋ, ನನಗೆ ಸಮನಾದ ಇದಿರಾಳು ಈಗ ಸಿಕ್ಕ ಹಾಗಾ ಯಿತು. ಈಗ ಇವನು ಬರಿಗೈಲಿ ಬರುತ್ತಿದ್ದಾನೆ. ಆದ್ದರಿಂದ ನಾನು ಆಯುಧವನ್ನು ಹಿಡಿದು ಇವನನ್ನು ಇದಿರಿಸುವುದು ನ್ಯಾಯವಲ್ಲ. ನಾನೂ ಬರಿಗೈಲೆ ಹೋಗಿ ಅವನನ್ನು ಕೆಣಕು ತ್ತೇನೆ’ ಎಂದುಕೊಂಡು, ಅವನು ಶ್ರೀಕೃಷ್ಣನ ಬಳಿಗೆ ಓಡಿಬಂದನು. ಅವನು ಹತ್ತಿರಕ್ಕೆ ಬರುವವರೆಗೂ ಸುಮ್ಮನಿದ್ದ ಶ್ರೀಕೃಷ್ಣ, ಅವನು ಬರುತ್ತಲೆ, ಹೆದರಿದವನಂತೆ ಅಲ್ಲಿಂದ ಓಡಲಾರಂಭಿಸಿದ. ಕಾಲಯವನನು ಅವನನ್ನು ಅಟ್ಟಿಕೊಂಡು ಹೊರಟ. ಓಡಿ ಓಡಿ ಅವನಿಗೆ ಸಾಕಾಯಿತು. ‘ಎಲಾ ಕೃಷ್ಣ, ಯದುವಂಶದಲ್ಲಿ ಹುಟ್ಟಿದ ನೀನು ಜೀವಗಳ್ಳನಂತೆ ಹೀಗೆ ಓಡುವುದು ನಾಚಿಕೆಗೇಡು. ನಿಲ್ಲು’ ಎಂದು ಕೂಗಿದನು. ಆದರೇನು? ಪಾಪಿಗಳಿಗೆ ಸಿಕ್ಕದ ಪರಮಾತ್ಮನಂತೆ ಶ್ರೀಕೃಷ್ಣ ಓಡಿ ಓಡಿ, ಕೊನೆಗೊಂದು ಪರ್ವತದ ಗುಹೆಯನ್ನು ಹೊಕ್ಕನು. ಹಿಂದಟ್ಟಿ ಬಂದ ಕಾಲಯವನನೂ ಅವನ ಹಿಂದೆಯೇ ಅದನ್ನು ಹೊಕ್ಕನು.

ಗುಹೆಯಲ್ಲಿ ಕಗ್ಗತ್ತಲು ಮುಸುಕಿತ್ತು. ಕಾಲಯವನನು ಕಾಲಿಂದ ತಡವರಿಸಿಕೊಂಡು ಒಳಗೆ ಹೋಗುತ್ತಿರುವಾಗ ಯಾರೋ ಒಬ್ಬರು ಅಲ್ಲಿ ಮಲಗಿರುವುದು ಕಾಣಿಸಿತು. ‘ಅಲ್ಲಿ ಬಂದು ಇನ್ನಾರು ಮಲಗಬೇಕು? ಶ್ರೀಕೃಷ್ಣನೆ ಇಲ್ಲಿ ಏನೂ ಅರಿಯದವನಂತೆ ನಟಿಸುತ್ತಾ ಮಲಗಿರಬೇಕು’ ಎಂದುಕೊಂಡು, ಅವನು ಅಲ್ಲಿ ಮಲಗಿದ್ದವನನ್ನು ಝಾಡಿಸಿ ಕಾಲಿನಿಂದ ಒದೆದನು. ಪಾಪ, ಅಲ್ಲಿ ಮಲಗಿದ್ದವನು ಕೊಡವಿಕೊಂಡು ಮೇಲಕ್ಕೆದ್ದವನೆ ತನ್ನ ಬಳಿ ಯಲ್ಲಿ ನಿಂತಿದ್ದ ಕಾಲಯವನನ್ನು ಕೋಪದಿಂದ ದಿಟ್ಟಿಸಿ ನೋಡಿದನು. ಹಾಗೆ ನೋಡಿ ದುದೇ ತಡ, ಕಾಲಯವನನು ಸುಟ್ಟು ಬೂದಿಯಾಗಿ ಹೋದ. ಇಷ್ಟಾಗುತ್ತಲೆ, ಶ್ರೀಕೃಷ್ಣನು ತಾನು ಅಡಗಿದ್ದ ಸ್ಥಳದಿಂದ ಹೊರಬಂದು ಆ ಮಹಾ ಪುರುಷನಿಗೆ ಕಾಣಿಸಿಕೊಂಡನು. ಆತನನ್ನು ನೋಡುತ್ತಲೆ ಆ ಮಹಾಪುರುಷನ ಮನಸ್ಸಿನಲ್ಲಿ ಭಕ್ತಿ ಹುಟ್ಟಿತು. ‘ನೀನಾರು?’ ಎಂದು ಕೇಳುವುದಕ್ಕೆ ಮುಂಚೆ, ತಾನಾರೆಂಬುದನ್ನು ಆತ ಶ್ರೀಕೃಷ್ಣನಲ್ಲಿ ಹೇಳಿಕೊಂಡನು.

“ಮಹಾಭಾಗ, ನಾನು ಇಕ್ಷ್ವಾಕುವಂಶಕ್ಕೆ ಸೇರಿದವನು. ಮಾಂಧಾತೃಮಹಾರಾಜನ ಮಗ. ನನ್ನ ಹೆಸರು ಮುಚುಕುಂದ. ಈಗ ಬಹುಕಾಲದ ಹಿಂದೆ ನಾನು ರಾಜನಾಗಿದ್ದೆ. ಆಗ ಇಂದ್ರನೆ ಮೊದಲಾದ ದೇವತೆಗಳು ರಾಕ್ಷಸರ ಬಾಧೆಯಿಂದ ತಮ್ಮನ್ನು ಕಾಪಾಡಬೇಕೆಂದು ನನ್ನ ಮರೆಹೊಕ್ಕರು. ನಾನು ಹೋಗಿ ರಾಕ್ಷಸರ ಬಾಧೆಯಿಂದ ಅವರನ್ನು ಕಾಪಾಡಿದೆ. ಆಮೇಲೆ ನಾನು ದೇವಲೋಕದಲ್ಲಿಯೆ ಬಹುಕಾಲ ನಿಲ್ಲಬೇಕಾಯಿತು. ರುದ್ರನ ಮಗನಾದ ಕುಮಾರಸ್ವಾಮಿಯು ಸ್ವರ್ಗಲೋಕದ ರಕ್ಷಕನಾಗಿರಲು ಒಪ್ಪಿಕೊಂಡಮೇಲೆ ದೇವತೆಗಳು ನನ್ನನ್ನು ಬೀಳ್ಕೊಡುತ್ತಾ ‘ಮಹಾರಾಜ, ಭೂಲೋಕದಲ್ಲಿ ಈಗ ನಿನ್ನವರೆನ್ನುವರಾರೂ ಇಲ್ಲ. ನಿನ್ನ ಸಮಕಾಲದವರೆಲ್ಲ ಸತ್ತುಹೋಗಿ ಎಷ್ಟೋ ಕಾಲವಾಗಿದೆ. ಆದ್ದರಿಂದ ನಿನ್ನ ಉಪಕಾರಕ್ಕೆ ಏನು ಪ್ರತ್ಯುಪಕಾರ ಮಾಡಬೇಕೆಂಬುದೇ ನಮಗೆ ತೋಚುತ್ತಿಲ್ಲ. ನಿನಗೆ ಬೇಕಾದ ವರವನ್ನು ನೀನೆ ಕೇಳಿಕೊ ಮೋಕ್ಷವೊಂದನ್ನು ಬಿಟ್ಟು, ನೀನು ಏನು ಕೇಳಿದರೂ ನಾವು ಕೊಡಬಲ್ಲೆವು’ ಎಂದರು. ನಾನು ಅವರೊಡನೆ ‘ದೇವತೆಗಳೆ, ಸದಾ ನಿಮ್ಮ ರಕ್ಷಣೆ ಯಲ್ಲಿದ್ದ ನನಗೆ ಬಹುಕಾಲದಿಂದ ನಿದ್ರೆಯೆ ಇಲ್ಲದಂತಾಗಿದೆ. ನನಗೆ ನಿರಾತಂಕವಾಗಿ ಮಲಗಿಕೊಳ್ಳಲು ಯೋಗ್ಯವಾದ ಒಂದು ಸ್ಥಳವನ್ನು ತೋರಿಸಿ’ ಎಂದು ಬೇಡಿದೆ. ಅವರು ಈ ಪರ್ವತಗುಹೆಯನ್ನು ತೋರಿಸಿ, ನನಗೆ ನಿದ್ರಾಭಂಗ ಮಾಡಿದವರು ಉರಿದುಹೋಗ ಲೆಂದು ವರವಿತ್ತರು. ನಾನು ಬಹುಕಾಲದಿಂದ ಇಲ್ಲಿ ಮಲಗಿದ್ದವನು, ಈಗ ನಿದ್ರಾಭಂಗ ವಾಗಿ ಮೇಲಕ್ಕೆದ್ದೆ. ಇದು ನನ್ನ ಕಥೆ. ಈಗ ಹೇಳು, ನೀನಾರು? ಈ ಗುಹೆಗೇಕೆ ಬಂದೆ? ನಿನ್ನ ದೇಹಕಾಂತಿಯನ್ನೂ ಮುಖದ ತೇಜಸ್ಸನ್ನೂ ನೋಡಿದರೆ ನೀನು ಸಾಮಾನ್ಯ ಮಾನವ ನಲ್ಲವೆಂಬುದು ತಾನಾಗಿಯೆ ಗೊತ್ತಾಗುತ್ತದೆ” ಎಂದು ಪ್ರಶ್ನೆ ಮಾಡಿದನು.

ಮುಚುಕುಂದನು ತನ್ನ ಪರಿಚಯವನ್ನು ಮಾಡಿಕೊಟ್ಟಂತೆ ಶ್ರೀಕೃಷ್ಣನೂ ತನ್ನ ಆತ್ಮಕಥೆ ಯನ್ನು ಹೇಳಿ: ‘ಮಹಾರಾಜ, ನಾನಾರೆಂದು ಹೇಳಲಿ? ನನ್ನ ಹೆಸರುಗಳು ಅನಂತ, ಕಾರ್ಯ ಗಳು ಅನಂತ, ಜನ್ಮಗಳು ಅನಂತ, ಭೂಮಿಯ ಮೇಲಿನ ಧೂಳಿನ ಕಣಗಳನ್ನಾದರೂ ಎಣಿಸಬಹುದು, ನನ್ನ ನಾಮ ರೂಪ ಗುಣ ಕರ್ಮಗಳನ್ನು ಎಣಿಸುವುದು ಸಾಧ್ಯವಿಲ್ಲ. ಸರ್ವಜ್ಞರೆನಿಸಿಕೊಂಡ ಮಹರ್ಷಿಗಳು ಕೂಡ ನನ್ನ ಸ್ವರೂಪವನ್ನು ಅರ್ಥಮಾಡಿಕೊಳ್ಳ ಲಾರರು. ಆದರೂ ಈಗ ನನ್ನ ಸಧ್ಯದ ಸ್ಥಿತಿ ಇಷ್ಟು– ಚತುರ್ಮುಖ ಬ್ರಹ್ಮ ನನ್ನ ಬಳಿಗೆ ಬಂದು ಭೂಭಾರವನ್ನು ಇಳುಹುವಂತೆ ಬೇಡಿಕೊಂಡ. ನಾನು ಆ ಕಾರ್ಯವನ್ನು ಮಾಡುವು ದಕ್ಕಾಗಿ ಯಾದವಕುಲದ ವಸುದೇವನ ಮಗನಾಗಿ ಅವತರಿಸಿದ್ದೇನೆ. ಈಗ ನನ್ನ ಹೆಸರು ವಾಸುದೇವ ಎಂದು. ಈಗಾಗಲೇ ಕಂಸನೇ ಮೊದಲಾದ ಹಲವು ಪಾಪಿಗಳನ್ನು ಕೊಂದು ಹಾಕಿದ್ದೇನೆ. ಈಗತಾನೆ ಕಾಲಯವನನೆಂಬ ರಕ್ಕಸನನ್ನು ನಿನ್ನಿಂದ ಕೊಲ್ಲಿಸಿದ್ದೇನೆ. ಮುಚುಕುಂದ, ನೀನು ನಿನ್ನ ಪೂರ್ವಜನ್ಮದಲ್ಲಿ ನನ್ನನ್ನು ಬಹುವಾಗಿ ಪೂಜಿಸಿರುವೆ. ಆದ್ದರಿಂದ ನಿನ್ನನ್ನು ಅನುಗ್ರಹಿಸಬೇಕೆಂದು ನನ್ನ ಸಂಕಲ್ಪ. ನಿನಗೆ ಬೇಕಾದ ವರವನ್ನು ಕೇಳಿಕೊ’ ಎಂದನು. ಇದನ್ನು ಕೇಳಿ ಆನಂದದಿಂದ ಮೈದುಂಬಿದ ಮುಚುಕುಂದನು ಶ್ರೀಕೃಷ್ಣನ ಮುಂದೆ ಅಡ್ಡಬಿದ್ದು, ‘ಪ್ರಭು, ನಿನ್ನ ಪಾದಸೇವೆಗಿಂತಲೂ ದೊಡ್ಡದಾದ ಯಾವ ವರವನ್ನು ಬೇಡಲಿ? ಇಗೋ ನಿನಗೆ ಶರಣಾಗಿದ್ದೇನೆ. ಇನ್ನು ನನಗೆ ಸಂಸಾರಭಯ ವಿಲ್ಲದಂತೆ ನನ್ನನ್ನು ಕಾಪಾಡು’ ಎಂದು ಬೇಡಿದನು. ಶ್ರೀಕೃಷ್ಣನು ‘ತಥಾಸ್ತು’ ಎಂದು ಹೇಳಿ, ‘ಅಯ್ಯಾ ಮುಚುಕುಂದ, ನೀನು ಕ್ಷತ್ರಿಯನಾದುದರಿಂದ ಆ ಧರ್ಮಕ್ಕೆ ತಕ್ಕಂತೆ ಬೇಟೆ ಮೊದಲಾದ ಕಾರಣಗಳಿಂದ ಅನೇಕ ಪ್ರಾಣಿಗಳನ್ನು ಕೊಂದು ಕರ್ಮವನ್ನು ಸಂಪಾ ದಿಸಿಕೊಂಡಿರುವೆ. ಆದ್ದರಿಂದ ಕರ್ಮವನ್ನು ತೀರಿಸುವುದಕ್ಕಾಗಿ ನೀನು ನನ್ನಲ್ಲಿ ಮನವನ್ನು ನೆಲೆಗೊಳಿಸಿ ನಿನ್ನ ಇಷ್ಟಾನುಸಾರವಾಗಿ ಭೂಸಂಚಾರವನ್ನು ಮಾಡುತ್ತಿರು. ಮುಂದಿನ ಜನ್ಮದಲ್ಲಿ ನೀನು ಬ್ರಹ್ಮಜ್ಞಾನಿಯಾಗಿ ಹುಟ್ಟಿ ಮೋಕ್ಷವನ್ನು ಪಡೆಯುವೆ’ ಎಂದನು.

ಶ್ರೀಕೃಷ್ಣಪರಮಾತ್ಮನ ಅನುಗ್ರಹವನ್ನು ಪಡೆದ ಮುಚುಕುಂದನು ಆತನಿಗೆ ಪ್ರದಕ್ಷಿಣೆ ನಮಸ್ಕಾರ ಮಾಡಿ ಗುಹೆಯಿಂದ ಹೊರಗೆ ಬಂದನು. ಅಲ್ಲಿ ನೋಡುತ್ತಾನೆ, ಮನುಷ್ಯ, ಮೃಗ, ಗಿಡಮರಗಳೊಂದೂ ತಾನು ಹಿಂದೆ ನೋಡಿದಂತೆ ಇಲ್ಲ! ಎಲ್ಲವೂ ಕುಳ್ಳು, ಕುಬ್ಜ, ಅಲ್ಪ. ‘ಓಹೋ ಇದು ಕಲಿಯುಗ’ ಎಂದುಕೊಂಡ ಮುಚುಕುಂದ, ಅಲ್ಲಿಂದ ಹೊರಟು ಬದರಿಕಾಶ್ರಮವನ್ನು ಸೇರಿ, ಅಲ್ಲಿ ತಪೋನಿರತನಾದನು.

ಮುಚುಕುಂದನನ್ನು ಗುಹೆಯಿಂದ ಬೀಳ್ಕೊಟ್ಟ ಶ್ರೀಕೃಷ್ಣನು ಮಧುರಾನಗರಿಗೆ ಹಿಂದಿ ರುಗಿ, ಅಲ್ಲಿ ಮುತ್ತಿಗೆ ಹಾಕಿ ಕುಳಿತಿದ್ದ ಯವನ ಸೇನೆಯನ್ನು ಚಲ್ಲಾಪಿಲ್ಲಿಯಾಗುವಂತೆ ಬಡಿದು ಓಡಿಸಿದನು. ಈ ಯುದ್ಧದಲ್ಲಿ ಆತನಿಗೆ ಅಪಾರವಾದ ಧನರಾಶಿ ಕೈವಶವಾಯಿತು. ಅದೆಲ್ಲವನ್ನೂ ದ್ವಾರಕೆಗೆ ಸಾಗಿಸುತ್ತಿದ್ದನು. ಅಷ್ಟರಲ್ಲಿ ಜರಾಸಂಧನು ತನ್ನ ಇಪ್ಪತ್ತ ಮೂರು ಅಕ್ಷೋಹಿಣಿ ಸೇನೆಯೊಡನೆ ಮತ್ತೆ– ಹದಿನೆಂಟನೆಯ ಬಾರಿ–ಬಂದು ಮಧುರೆಗೆ ಮುತ್ತಿಗೆಹಾಕಿದನು. ಬಲರಾಮಕೃಷ್ಣರು ಈ ಬಾರಿ ಆತನೊಡನೆ ಯುದ್ಧ ಮಾಡಲಿಲ್ಲ; ಹೆದರಿದವರಂತೆ ನಟಿಸಿ, ತಾವು ಸಾಗಿಸುತ್ತಿದ್ದ ಧನರಾಶಿಯನ್ನೂ ಅಲ್ಲಿಯೇ ಬಿಟ್ಟು ಓಡಿಹೋದರು. ಜರಾಸಂಧನು ತನ್ನ ಸೇನೆಯೊಡನೆ ಅವರನ್ನು ಬೆನ್ನಟ್ಟಿದನು. ಬಲರಾಮ ಕೃಷ್ಣರು ಓಡಿ ಓಡಿ ಸಾಕಾಗಿ, ಪ್ರವರ್ಷವೆಂಬ ಒಂದು ಕಡಿದಾದ ಬೆಟ್ಟವನ್ನು ಹತ್ತಿ ತಲೆ ಮರೆಸಿಕೊಂಡರು. ಜರಾಸಂಧನು ಅತ್ಯಂತ ಕೋಪದಿಂದ ಆ ಬೆಟ್ಟದ ಸುತ್ತಲೂ ಕಟ್ಟಿಗೆ ಗಳನ್ನು ಒಟ್ಟಿಸಿ, ಬೆಂಕಿಯನ್ನಿಡಿಸಿದನು. ಆದರೆ ಬಲರಾಮಕೃಷ್ಣರು ಅದರಿಂದ ಪಾರಾಗಿ, ಜರಾಸಂಧನ ಕಣ್ಣಿಗೆ ಬೀಳದಂತೆ ತಪ್ಪಿಸಿಕೊಂಡು, ದ್ವಾರಕಾನಗರವನ್ನು ಸೇರಿದರು. ಜರಾಸಂಧನು ಅವರೀರ್ವರೂ ಬೆಂಕಿಯಲ್ಲಿ ಸುಟ್ಟುಹೋಗಿರಬೇಕೆಂದುಕೊಂಡು, ತನ್ನ ಸೇನೆಯೊಡನೆ ಮಗಧ ದೇಶಕ್ಕೆ ಹಿಂದಿರುಗಿದನು.

