
\chapter{೮೦. ಶಿಶುಪಾಲನ ಗೆಳೆಯ ಸಾಲ್ವ}

ಸಾಲ್ವನು ಶಿಶುಪಾಲನ ಜೀವದ ಗೆಳೆಯ. ಹಿಂದೆ ರುಕ್ಮಿಣಿಯ ಸ್ವಯಂವರಕಾಲದಲ್ಲಿ ಆತನು ಸಾಲ್ವನಿಗಾದ ಅಪಮಾನವನ್ನು ಕಂಡು ರೋಷಾವೇಶದಿಂದ ‘ಮಿತ್ರ, ಆದಷ್ಟು ಬೇಗ ನಾನು ಈ ಯಾದವರ ಪೀಳಿಗೆಯನ್ನು ನಿರ್ಮೂಲಮಾಡುತ್ತೇನೆ’ ಎಂದು ಗೆಳೆಯನಿಗೆ ಮಾತು ಕೊಟ್ಟನು. ಹೀಗೆ ಕೊಟ್ಟ ಮಾತನ್ನು ಉಳಿಸಿಕೊಳ್ಳುವುದಕ್ಕಾಗಿ ಆತನು ಶಂಕರನನ್ನು ಕುರಿತು ತಪಸ್ಸು ಮಾಡಿದನು. ಇವನ ದುರುದ್ದೇಶವನ್ನರಿತ ಶಂಕರನು ಒಂದು ವರ್ಷದ ವರೆಗೆ ಅವನಿಗೆ ಪ್ರಸನ್ನನಾಗಲಿಲ್ಲ. ಆದರೆ, ದಿನಕ್ಕೊಂದು ಹಿಡಿ ಮಣ್ಣನ್ನು ಮಾತ್ರ ತಿನ್ನುತ್ತಾ ಒಂದೇ ಮನಸ್ಸಿನಿಂದ ಧ್ಯಾನಮಾಡುತ್ತಿರುವಾಗ, ಭಕ್ತವತ್ಸಲನಾದ ಶಂಕರನು ಅನಿವಾರ್ಯವಾಗಿ ಅವನಿಗೆ ಪ್ರತ್ಯಕ್ಷನಾಗಲೇಬೇಕಾಯಿತು. ಆತನು ಆ ಭಕ್ತನಿಗೆ ಗೋಚರ ನಾಗಿ, ಬೇಕಾದ ವರವನ್ನು ಬೇಡುವಂತೆ ಹೇಳಿದನು. ಸಾಲ್ವನು ‘ಹೇ ದೇವದೇವ, ಯಾದವ ರಿಗೆ ಭಯವನ್ನು ಹುಟ್ಟಿಸುವಂತಹ ಒಂದು ವಿಮಾನವನ್ನು ನನಗೆ ಕರುಣಿಸು; ನಾನು ಬಯಸಿದ ಕಡೆಗೆ ಅದು ಚಲಿಸಬೇಕು; ದೇವ ದಾನವ ಮಾನವರೆಲ್ಲರಿಗೂ ಅದು ಅಭೇದ್ಯ ವಾಗಿರಬೇಕು’ ಎಂದು ಬೇಡಿದನು. ಶಂಕರನು ದಾನವಶಿಲ್ಪಿಯಾದ ಮಯನನ್ನು ಕರೆದು, ಅವನಿಗೊಂದು ಉಕ್ಕಿನ ವಿಮಾನವನ್ನು ಮಾಡಿಕೊಡುವಂತೆ ಅಪ್ಪಣೆ ಮಾಡಿದನು. ಅದ ರಂತೆ ಮಯನು ‘ಸೌಭ’ವೆಂಬ ಒಂದು ವಿಮಾನವನ್ನು ನಿರ್ಮಾಣಮಾಡಿಕೊಟ್ಟನು. ಅದು ಕಗ್ಗತ್ತಲಿನಿಂದ ತುಂಬಿ ಯಾರ ಕಣ್ಣಿಗೂ ಕಾಣಿಸುತ್ತಿರಲಿಲ್ಲ. ಇದರ ಸಹಾಯದಿಂದ ಸಾಲ್ವನು ಯಾದವರನ್ನು ನಿರ್ನಾಮಮಾಡಲೆಂದು ಸಮಯ ಕಾಯುತ್ತಿರಲು, ಶ್ರೀಕೃಷ್ಣನು ಶಿಶುಪಾಲನನ್ನು ಕೊಂದ ಸುದ್ದಿ ಸಾಲ್ವನಿಗೆ ಗೊತ್ತಾಯಿತು. ಒಡನೆಯೆ ಆತನು ತನ್ನ ದೊಡ್ಡ ಸೇನೆಯೊಡನೆ ತನ್ನ ವಿಮಾನವನ್ನೇರಿ ದ್ವಾರಕಿಗೆ ಬಂದನು. ಅಲ್ಲಿನ ಪಿಳ್ಳೆಯೊಂದೂ ತಪ್ಪಿಸಿಕೊಳ್ಳದಂತೆ ಅವನ ಸೈನ್ಯ ದ್ವಾರಕಿಯ ಸುತ್ತ ಮುತ್ತಿತು. ಸಾಲ್ವನು ಆಕಾಶದಲ್ಲಿ ತನ್ನ ವಿಮಾನವನ್ನು ನಿಲ್ಲಿಸಿ, ಅಲ್ಲಿಂದ ದ್ವಾರಕಿಯ ಮೇಲೆ ಗಿಡಮರಗಳ, ಕಲ್ಲುಬಂಡೆಗಳ, ಶಸ್ತ್ರಾಸ್ತ್ರಗಳ ಮಳೆಗರೆದನು. ಅವನ ದಾಳಿಯಿಂದ ಆ ಸುಂದರ ನಗರದ ಉದ್ಯಾನ ವನಗಳೆಲ್ಲ ಹಾಳಾದವು, ಮನೆಮನೆಯ ಗೋಪುರಗಳೆಲ್ಲ ಮುರಿದುಬಿದ್ದವು, ಊರಿನ ಹೆಬ್ಬಾಗಿಲು ಮುರಿದುಬಿತ್ತು. ಇದು ಸಾಲದೆಂಬಂತೆ ದೊಡ್ಡ ಸುಂಟರಗಾಳಿಯೊಂದು ಮೂಡಿ, ದ್ವಾರಕಿಯನ್ನೆಲ್ಲ ಧೂಳಿನಿಂದ ತುಂಬಿತು, ಮರಳಿನ ಮಳೆ ಸುರಿಯುವುದಕ್ಕೆ ಆರಂಭವಾಯಿತು. ಶ್ರೀಕೃಷ್ಣನು ಊರಲ್ಲಿಲ್ಲದಿರುವಾಗ ಈ ಉತ್ಪಾತಗಳೆಲ್ಲ ಕಾಣಿಸಿ ಕೊಂಡುದನ್ನು ಕಂಡು, ದ್ವಾರಕಿಯ ಜನ ಭಯದಿಂದ ನಡುಗಿಹೋದರು.

ಜನರ ಭಯ ಸಂಕಟಗಳನ್ನು ಕಂಡು ಶ್ರೀಕೃಷ್ಣನ ಮಗನಾದ ಪ್ರದ್ಯುಮ್ನನಿಗೆ ರೋಷ ಉಕ್ಕಿತು. ಆತನು ಜನರಿಗೆ ಅಭಯವಿತ್ತು, ಶತ್ರುವನ್ನು ಇದಿರಿಸುವುದಕ್ಕಾಗಿ ತನ್ನ ರಥವೇರಿ ಹೊರಟನು. ಒಡನೆಯೆ ಸಾಂಬ, ಸಾತ್ಯಕಿ ಮೊದಲಾದ ಯಾದವ ಸೇನಾಪತಿಗಳು ತಮ್ಮ ತಮ್ಮ ಸೇನೆಯೊಡನೆ ಆತನನ್ನು ಹಿಂಬಾಲಿಸಿದರು. ಯಾದವ ಸೈನ್ಯಕ್ಕೂ ಸಾಲ್ವನ ಸೈನ್ಯಕ್ಕೂ ಘೋರವಾದ ಯುದ್ಧ ಪ್ರಾರಂಭವಾಯಿತು. ಪ್ರದ್ಯುಮ್ನನು ತನ್ನ ಮಡದಿಯಾದ ಮಾಯಾ ವತಿಯಿಂದ ಮಹಾಮಾಯಾ ವಿದ್ಯೆಯನ್ನು ಕಲಿತವನಲ್ಲವೆ? ಆದ್ದರಿಂದ ಆತನು ಸಾಲ್ವನ ಮಾಯಾವಿದ್ಯೆಗಳನ್ನೆಲ್ಲ ಕ್ಷಣಮಾತ್ರದಲ್ಲಿ ನಾಶಮಾಡಿದನು. ಸೂರ್ಯ ಹುಟ್ಟುತ್ತಲೆ ಕತ್ತಲೆ ಹಾರಿ ಹೋದಂತಾಯಿತು. ಪ್ರದ್ಯುಮ್ನನು ಶತ್ರುವಿನ ಮೇಲೆ ಬಾಣಗಳ ಮಳೆಗರೆಯುತ್ತಾ ಅವನ ಉರುಬನ್ನು ತಗ್ಗಿಸಿದನು. ಆದರೆ ಈಶ್ವರನ ವರದಿಂದ ಬಂದಿರುವ ವಿಮಾನದ ಉಪಟಳವನ್ನು ನಿಲ್ಲಿಸುವುದು ಹೇಗೆ? ಅದು ಒಮ್ಮೆ ಹಲವು ವಿಮಾನಗಳಂತೆ ಕಾಣಿಸಿ ಕೊಳ್ಳುತ್ತದೆ, ನೋಡುನೋಡುತ್ತಿರುವಂತೆ ಒಂದಾಗಿ ಹೋಗುತ್ತದೆ! ಒಮ್ಮೆ ಕಾಣಿಸುತ್ತದೆ, ಮತ್ತೊಮ್ಮೆ ಮಾಯವಾಗುತ್ತದೆ! ಒಮ್ಮೆ ಆಕಾಶದಲ್ಲಿದ್ದರೆ ಮತ್ತೊಮ್ಮೆ ಸಮುದ್ರದಲ್ಲಿ, ಮರುಕ್ಷಣದಲ್ಲಿ ಬೆಟ್ಟದ ನೆತ್ತಿಯ ಮೇಲೆ! ಯಾದವಭಟರು ಅದು ಕಂಡ ಕಡೆಗೆ ಬಾಣಗಳ ಮಳೆ ಸುರಿಸುತ್ತಾರೆ. ಒಮ್ಮೊಮ್ಮೆ ಅವನ ಯೋಚನೆಗೂ ಮೀರಿದ ವೇಗದಿಂದ ಅವರ ಬಾಣಗಳು ವಿಮಾನಕ್ಕೆ ಬಡಿದು ಅದು ಮಂಕಾಗುವುದೂ ಉಂಟು, ಇದನ್ನು ಕಂಡು ಸಾಲ್ವನು ಮೆಟ್ಟಿಬೀಳುವನು. ಯಾದವವೀರರು ಜೀವದ ಹಂಗುದೊರೆದು ಯುದ್ಧಮಾಡು ತ್ತಿದ್ದರು. ಆದ್ದರಿಂದ ಸಾಲ್ವನ ಸೇನೆ ತತ್ತರಿಸಿ ಹೋಯಿತು. ಇದನ್ನು ಕಂಡು ಅದರ ಸೇನಾಪತಿಯಾದ ದ್ಯುಮಂತನೆಂಬುವನು ಪ್ರದ್ಯುಮ್ನನ ಬಳಿಗೆ ನುಗ್ಗಿಬಂದು, ತನ್ನ ಕೈಲಿದ್ದ ಉಕ್ಕಿನ ಗದೆಯಿಂದ ಆತನ ಎದೆಯ ಮೇಲೆ ಹೊಡೆದನು. ಆ ಪೆಟ್ಟಿಗೆ ಪ್ರದ್ಯು ಮ್ನನು ರಥದಲ್ಲಿ ಬಿದ್ದು ಮೂರ್ಛಿತನಾದನು. ಆತನ ಸಾರಥಿ ಉಪಾಯದಿಂದ ರಥವನ್ನು ಯುದ್ಧರಂಗದಿಂದ ದೂರವಾಗಿ ತೆಗೆದುಕೊಂಡು ಹೋದನು.

ದ್ವಾರಕಿಯಲ್ಲಿ ಪ್ರದ್ಯುಮ್ನ ಮೂರ್ಛೆಹೋಗುತ್ತಲೆ ಇಂದ್ರಪ್ರಸ್ಥದಲ್ಲಿದ್ದ ಶ್ರೀಕೃಷ್ಣನಿಗೆ ಎಡಭುಜ ಅದುರಿತು, ಎಡಗಣ್ಣು ಹಾರಿತು. ಆತನು ಒಡನೆಯೆ ಧರ್ಮರಾಯನಿಂದ ಬೀಳ್ಕೊಂಡು ದ್ವಾರಕಿಗೆ ಹೊರಟನು. ಅಷ್ಟರಲ್ಲಿ ಮೂರ್ಛೆಬಿದ್ದ ಪ್ರದ್ಯುಮ್ನನು ಎಚ್ಚೆತ್ತು ತನ್ನ ಸಾರಥಿಯನ್ನು ಅವನ ಕಾರ್ಯಕ್ಕಾಗಿ ಛೀಗುಟ್ಟುತ್ತಾ ‘ಎಲಾ, ನೀನು ಅಂಜುಬುರಕ. ನಿನ್ನಿಂದ ನನಗೆ ಅಪಕೀರ್ತಿ ಬಂದಂತಾಯಿತು. ಯಾದವರು ಎಂದಾದರೂ ಯುದ್ಧಭೂಮಿ ಯಿಂದ ಓಡಿಹೋಗುತ್ತಾರೆಯೆ? ನಾನು ಹೀಗೆ ಓಡಿಬಂದುದುನ್ನು ಕೇಳಿದರೆ ಅರಮನೆಯ ಹೆಂಗಸರೆಲ್ಲ ಆಡಿಕೊಂಡು ನಗುವುದಿಲ್ಲವೆ? ನಾಳೆ ನಮ್ಮ ತಂದೆ ಬಂದು ಕೇಳಿದರೆ ನಾನು ಏನೆಂದು ಉತ್ತರ ಕೊಡಬೇಕು? ಮೊದಲು ರಥವನ್ನು ಯುದ್ಧ ಭೂಮಿಯತ್ತ ತಿರುಗಿಸು’ ಎಂದು ಹೇಳಿ, ರಣರಂಗಕ್ಕೆ ಹಿಂದಿರುಗಿದನು. ಆತನ ಅಪ್ಪಣೆಯಂತೆ ಆತನ ಸಾರಥಿ ತನ್ನ ರಥವನ್ನು ನೇರವಾಗಿ ದ್ಯುಮಂತನ ಇದಿರಿಗೆ ಕೊಂಡೊಯ್ದು ನಿಲ್ಲಿಸಿದನು. ಪ್ರದ್ಯುಮ್ನನು ಎಂಟು ಬಾಣಗಳನ್ನು ಏಕಕಾಲದಲ್ಲಿ ಪ್ರಯೋಗಿಸಿ, ತನ್ನ ಶತ್ರುವಿನ ಕುದುರೆಗಳು, ಸಾರಥಿ, ಧ್ವಜಗಳೊಡನೆ ದ್ಯುಮಂತನ ತಲೆಯನ್ನು ಕತ್ತರಿಸಿಹಾಕಿದನು. ಆ ವೇಳೆಗೆ ಯುದ್ಧ ಪ್ರಾರಂಭವಾಗಿ ಇಪ್ಪತ್ತೇಳು ದಿನಗಳಾಗಿತ್ತು. ಸಾಲ್ವನ ಸೇನೆ ಧೂಳೆದ್ದು ಹೋಗಿತ್ತಾದರೂ ಸಾಲ್ವನನ್ನು ಗೆಲ್ಲುವುದು ದುಸ್ತರವಾಗಿತ್ತು. ಆ ವೇಳೆಗೆ ಸರಿಯಾಗಿ ಶ್ರೀಕೃಷ್ಣನು ದ್ವಾರಕಿಗೆ ಬಂದನು. ಅಲ್ಲಿನ ಸ್ಥಿತಿಗತಿಗಳನ್ನು ನೋಡುತ್ತಲೆ ಆತನು ಅರಮನೆಯನ್ನು ಪ್ರವೇಶಿಸದೆ, ನೇರವಾಗಿ ಯುದ್ಧರಂಗಕ್ಕೆ ತನ್ನ ರಥವನ್ನು ಹರಿಸಿದನು.

ತನ್ನ ಸೈನ್ಯವನ್ನೆಲ್ಲ ಕಳೆದುಕೊಂಡು ಕೋಪದಿಂದ ಹೊತ್ತಿ ಉರಿಯುತ್ತಿದ್ದ ಸಾಲ್ವನು ಶ್ರೀಕೃಷ್ಣನನ್ನು ರಣರಂಗದಲ್ಲಿ ಕಾಣುತ್ತಲೆ ಸಿಡಿಲಿನಂತೆ ಗರ್ಜಿಸಿ, ತನ್ನ ಭಯಂಕರವಾದ ಗದೆಯನ್ನು ಸಾರಥಿಯಾದ ದಾರುಕನಿಗೆ ಗುರಿಯಿಟ್ಟು ಬೀಸಿದನು. ಆದರೆ ಶ್ರೀಕೃಷ್ಣನ ಬಾಣ ಅದನ್ನು ಮಾರ್ಗಮಧ್ಯದಲ್ಲಿಯೇ ನುಚ್ಚುನೂರಾಗಿ ಮಾಡಿತು. ಇದನ್ನು ಕಂಡು ಕನಲಿದ ಸಾಲ್ವನು ಮಹಾಸ್ತ್ರವೊಂದನ್ನು ತನ್ನ ಬಿಲ್ಲಿಗೆ ಹೂಡಿ, ಶ್ರೀಕೃಷ್ಣನ ಎಡಗೈಗೆ ಅದನ್ನು ಗುರಿ ಯಿಟ್ಟು ಹೊಡೆದನು. ಅವನ ಗುರಿ ತಪ್ಪಲಿಲ್ಲ; ಅದು ಬಂದು ಬಡಿಯುತ್ತಲೆ ಶ್ರೀಕೃಷ್ಣನ ಕೈಲಿದ್ದ ಶಾರ್ಙ್ಗಧನು ಕೆಳಕ್ಕೆ ಬಿದ್ದು ಹೋಯಿತು. ಅತ್ಯಾಶ್ಚರ್ಯಕರವಾದ ಈ ಘಟನೆಯನ್ನು ಕಂಡು ಸುತ್ತಮುತ್ತಿನ ಸೈನಿಕರೆಲ್ಲ ಹಾಹಾಕಾರ ಮಾಡಿದರು. ಸಾಲ್ವನೂ ಸಿಂಹನಾದ ಮಾಡುತ್ತಾ ಶ್ರೀಕೃಷ್ಣನೊಡನೆ ‘ಎಲ ಮೂಢ, ನನ್ನ ಗೆಳೆಯನಾದ ಶಿಶುಪಾಲನ ಹೆಂಡತಿ ಯನ್ನು ಕದ್ದೋಡಿದ ಕಳ್ಳ, ನೀನು. ನಿಸ್ಸಹಾಯನಾಗಿ ಏಕಾಂಗಿಯಾಗಿದ್ದ ಆ ನನ್ನ ಗೆಳೆಯ ಶಿಶುಪಾಲನನ್ನು ಅನ್ಯಾಯವಾಗಿ ಕೊಂದ ಪಾಪಿ, ನೀನು. ಕೊಬ್ಬಿದ ಕುರಿಯನ್ನು ಮಾರಿಗೆ ಬಲಿಕೊಡುವಂತೆ ನಿನ್ನನ್ನು ಈಗ ಕತ್ತರಿಸಿ ಹಾಕುತ್ತೇನೆ’ ಎಂದನು. ಅವನ ಮಾತುಗಳನ್ನು ಕೇಳಿ ಮುಗುಳ್ನಗುತ್ತಾ ಶ್ರೀಕೃಷ್ಣನು ‘ಎಲಾ ಅವಿವೇಕಿ, ಹೆಡತಲೆಯ ಮೃತ್ಯು ಗಹಗಹಿಸಿ ನಗುತ್ತಿರುವುದನ್ನು ಕಾಣದೆ ಒಣ ಗರ್ವದ ಮಾತುಗಳನ್ನಾಡುತ್ತಿರುವೆ. ಪರಾಕ್ರಮವನ್ನು ತೋರಿಸಬೇಕಾದುದು ಕಾರ್ಯದಿಂದಲೆ ಹೊರತು ಮಾತಿನಿಂದಲ್ಲ’ ಎಂದು ಹೇಳಿ, ತನ್ನ ಗದೆ ಯಿಂದ ಅವನ ಭುಜಕ್ಕೆ ಹೊಡೆದನು. ಆ ಪೆಟ್ಟಿಗೆ ಅವನು ತತ್ತರಿಸಿ, ಬಾಯಿಂದ ನೆತ್ತರನ್ನು ಕಾರಿದನು. ಮರುಕ್ಷಣದಲ್ಲಿ ಇದ್ದಕ್ಕಿದ್ದಂತೆ ಮಾಯವಾದನು.

ಅತ್ತ ಸಾಲ್ವನು ಕಣ್ಮರೆಯಾಗುತ್ತಲೆ ಇತ್ತ ದೂತನೊಬ್ಬನು ಶ್ರೀಕೃಷ್ಣನ ಬಳಿಗೆ ಅಳುತ್ತಾ ಓಡಿಬಂದು ‘ಸ್ವಾಮಿ, ದೇವಕೀದೇವಿ ನನ್ನನ್ನು ತಮ್ಮ ಬಳಿಗೆ ಕಳುಹಿಸಿದಳು. ಕಟುಕನು ಆಕಳನ್ನು ಎಳೆದೊಯ್ಯುವಂತೆ, ಸಾಲ್ವನು ನಿನ್ನ ತಂದೆಯಾದ ವಸುದೇವನನ್ನು ಎಳೆದು ಕೊಂಡು ಹೋಗಿದ್ದಾನೆ. ನೀನು ಈಗಲೆ ಹೋಗಿ ಆತನನ್ನು ಬಿಡಿಸು’ ಎಂದನು. ಇದನ್ನು ಕೇಳಿ ಶ್ರೀಕೃಷ್ಣನಿಗೆ ಆಶ್ಚರ್ಯವಾಯಿತು. ‘ಮಹಾಬಲಶಾಲಿಯಾದ ಬಲರಾಮ ಅರಮನೆ ಯಲ್ಲಿಯೆ ಇದ್ದಾನೆ. ಈ ಸಾಲ್ವ ನಮ್ಮ ತಂದೆಯನ್ನು ಹೇಗೆ ಹಿಡಿದುಕೊಂಡು ಹೋದ?’ ಎಂದು ಆತನು ಚಿಂತಿಸುತ್ತಿರುವಾಗ, ಸಾಲ್ವನು ವಸುದೇವನನ್ನು ಹಿಡಿದುಕೊಂಡೆ ಅಲ್ಲಿ ಕಾಣಿಸಿದ. ಅವನು ಶ್ರೀಕೃಷ್ಣನು ನೋಡುತ್ತಿರುವಂತೆಯೆ ವಸುದೇವನ ತಲೆಯನ್ನು ಕತ್ತರಿಸಿ, ಆಕಾಶಕ್ಕೆ ನೆಗೆದವನೆ ತನ್ನ ವಿಮಾನವನ್ನು ಹೊಕ್ಕ. ಶ್ರೀಕೃಷ್ಣನು ಕ್ಷಣಕಾಲ ಮೋಹ ಪರವಶನಾಗಿ ಕಣ್ಣೀರನ್ನು ಕರೆದ. ಆದರೆ ಇದೇನಿದು? ತಾಯಿಯಿಂದ ತನ್ನ ಬಳಿಗೆ ಬಂದ ದೂತನೂ ಇಲ್ಲ, ಸಾಲ್ವನು ಕತ್ತರಿಸಿ ಹಾಕಿದ ತನ್ನ ತಂದೆಯ ದೇಹವೂ ಅಲ್ಲಿಲ್ಲ! ಶ್ರೀಕೃಷ್ಣ ನಿಗೆ ಅರ್ಥವಾಯಿತು, ಇದೆಲ್ಲ ಸಾಲ್ವನ ಮಾಯಾವಿದ್ಯೆಯೆಂದು. ಆತನು ಕೋಪದಿಂದ ತನ್ನ ಗದೆಯನ್ನು ವಿಮಾನದತ್ತ ಒಗೆಯಲು, ಆ ವಿಮಾನವು ಸಾವಿರ ತುಂಡುಗಳಾಗಿ ಒಡೆದು ಸಮುದ್ರದಲ್ಲಿ ಬಿತ್ತು. ಸಾಲ್ವನು ನೆಲಕ್ಕೆ ಹಾರಿ, ಗದಾಪಾಣಿಯಾಗಿ ಶ್ರೀಕೃಷ್ಣನ ಮೇಲೆ ಯುದ್ಧಕ್ಕೆ ನಿಂತನು. ಒಡನೆಯೆ ಶ್ರೀಕೃಷ್ಣನು ಸೂರ್ಯನನ್ನು ಧರಿಸಿರುವ ಪೂರ್ವ ದಿಕ್ಕಿನಂತೆ ಚಕ್ರವನ್ನು ಧರಿಸಿ, ಅದರಿಂದ ಆ ಸಾಲ್ವನ ತಲೆಯನ್ನು ಕತ್ತರಿಸಿ ಹಾಕಿದನು. ಅದನ್ನು ಕಂಡು ದೇವತೆಗಳು ಸಂತೋಷದಿಂದ ಹೂಮಳೆಗರೆದರು.

ಸಾಲ್ವನು ಸತ್ತ ಸುದ್ದಿಯನ್ನು ಕೇಳುತ್ತಲೆ ಅವನ ಗೆಳೆಯನಾದ ದಂತವಕ್ತ್ರನು ಕೋಪದಿಂದ ಹಲ್ಲುಗಳನ್ನು ಕಡಿಯುತ್ತಾ, ಹೇಗಿದ್ದವನು ಹಾಗೆಯೆ ಗದೆಯೊಡನೆ ಶ್ರೀಕೃಷ್ಣನ ಮೇಲೆ ಯುದ್ಧಕ್ಕೆ ಬಂದನು. ಅವನ ನಡಗೆಗೆ ಭೂಮಿ ನಡುಗುತ್ತಿತ್ತು. ಅವನು ಶ್ರೀಕೃಷ್ಣನನ್ನು ಕುರಿತು ‘ಎಲ, ನೀನು ನನ್ನ ಮಾವನ ಮಗನಾದ ಬಂಧು. ಆದರೇನು? ದೇಹಕ್ಕೆ ಅಂಟಿದ ರೋಗವೂ ಒಂದೆ, ನಿನ್ನ ನಂಟತನವೂ ಒಂದೆ. ನಿನ್ನನ್ನು ಕೊಂದು, ನನ್ನ ಗೆಳೆಯನ ಪುಣವನ್ನು ತೀರಿಸುತ್ತೇನೆ’ ಎಂದು ಹೇಳಿ, ತನ್ನ ಗದೆಯನ್ನು ಆತನ ತಲೆಯ ಮೇಲೆ ಅಪ್ಪಳಿಸಿದನು. ಶ್ರೀಕೃಷ್ಣನು ಆ ಹೊಡೆತವನ್ನು ಸೊಳ್ಳೆಯ ಕಡಿತದಂತೆ ಬಗೆದು, ಕೌಮೋದಕಿಯೆಂಬ ತನ್ನ ಗದೆಯಿಂದ ದಂತವಕ್ತ್ರನ ಎದೆಗೆ ಹೊಡೆದನು. ತಕ್ಷಣವೇ ಅವನು ನೆತ್ತರನ್ನು ಕಾರುತ್ತಾ ಸತ್ತು ಉರುಳಿದನು. ಒಡನೆಯೆ ಅವನ ದೇಹದಿಂದ ದಿವ್ಯ ವಾದ ಒಂದು ತೇಜಸ್ಸು ಹೊರಹೊರಟು ಶ್ರೀಕೃಷ್ಣನ ದೇಹವನ್ನು ಪ್ರವೇಶಿಸಿತು. ದಂತ ವಕ್ತ್ರನು ಸತ್ತುದನ್ನು ಕಂಡು ಅವನ ತಮ್ಮನಾದ ವಿಡೂರಥನು ಶ್ರೀಕೃಷ್ಣನನ್ನು ಕೆಣಕಿ, ಅಣ್ಣನ ಹಿಂದೆಯೇ ಪರಲೋಕಕ್ಕೆ ಪ್ರಯಾಣಮಾಡಿದನು.

