
\chapter{೭. ವಿದುರ ಮೈತ್ರೇಯರ ಸಂವಾದ–ಆದಿ ವರಾಹಾವತಾರ}

ಶುಕಮುನಿಯು ಪರೀಕ್ಷಿದ್ರಾಜನಿಗೆ ಭಾಗವತವನ್ನು ಹೇಳುತ್ತಿರುವನೆಂಬುದನ್ನೂ, ಆ ಪ್ರಸಂಗದಲ್ಲಿ ಬಂದಿರುವ ವಿದುರ ಮೈತ್ರೇಯರ ಸಂಭಾಷಣೆಯನ್ನು ನಾವು ಕೇಳುತ್ತಿದ್ದೇ ವೆಂಬುದನ್ನೂ ಮರೆಯಲಾಗದು. ಭಾಗವತ ಶಿರೋಮಣಿಯಾದ ವಿದುರನು ಜ್ಞಾನಿಯಾದ ಮೈತ್ರೇಯರನ್ನು ಕುರಿತು, ಸ್ವಾಯಂಭು ಮನುವಿನಿಂದ ಪ್ರಜಾಭಿವೃದ್ಧಿಯಾದ ವಿವರಗಳನ್ನು ತಿಳಿಯಬಯಸಿದನು. ಮೈತ್ರೇಯರು ಅದರ ವಿವರವನ್ನು ಹೇಳ ಹೊರಡುತ್ತಾರೆ–

ಸ್ವಾಯಂಭುವ ಮನುವು ತನ್ನ ಪತ್ನಿಯಾದ ಶತರೂಪೆಯೊಡನೆ ಚತುರ್ಮುಖ ಬ್ರಹ್ಮನ ಇದಿರಿನಲ್ಲಿ ಕೈಮುಗಿದು ನಿಂತು ‘ಹೇ ಸ್ವಾಮಿ, ತಂದೆಯಾದ ನಿನ್ನ ಸೇವೆ ಮಾಡುವುದು ಮಗನಾದ ನನ್ನ ಕರ್ತವ್ಯ. ಇಹದಲ್ಲಿ ಕೀರ್ತಿಯೂ ಪರದಲ್ಲಿ ಸದ್ಗತಿಯೂ ಆಗುವ ಯಾವ ಕಾರ್ಯವನ್ನು ನೀನು ನನಗೆ ಬೆಸಸುತ್ತಿ? ನಿನ್ನ ಆಜ್ಞೆಯನ್ನು ನೆರವೇರಿಸಲು ಸಾಧ್ಯವಾಗುವಂತಹ ಕಾರ್ಯವನ್ನು ಅಪ್ಪಣೆ ಕೊಡಿಸು’ ಎಂದನು. ಬ್ರಹ್ಮನ ಉದ್ದೇಶ ಪ್ರಜಾಭಿವೃದ್ಧಿ; ಆದ್ದರಿಂದ ಆತ ‘ಅಯ್ಯಾ, ಈ ಶತರೂಪೆಯಲ್ಲಿ ನಿನ್ನಂತೆಯೇ ಗುಣವಂತರಾದ ಮಕ್ಕಳನ್ನು ಪಡೆದು ಭೂಮಂಡಲವನ್ನು ಸಂರಕ್ಷಿಸು’ ಎಂದನು. ಆದರೆ ಆತನು ಎಲ್ಲಿ ನೆಲಸಬೇಕು? ಅಸಲಿಗೆ ಭೂಮಂಡಲವೇ ಇಲ್ಲ; ಸರ್ವ ಪ್ರಾಣಿಗಳಿಗೂ ನೆಲೆಯಾಗಿದ್ದ ಭೂಮಂಡಲವು ಮಹಾ ಜಲದಲ್ಲಿ ಮುಳುಗಿಹೋಗಿದೆ. ಅದನ್ನು ಮೇಲಕ್ಕೆ ಎತ್ತಿ ತರಬೇಕು. ಅದು ಬ್ರಹ್ಮನಿಂದ ಸಾಧ್ಯವಿಲ್ಲ. ಆತನು ಮುಂದೋರದೆ ಭಗವಂತನ ಧ್ಯಾನದಲ್ಲಿ ತಲ್ಲೀನನಾದನು. ಆಗ ಆತನ ನಾಸಾರಂಧ್ರದಿಂದ ಹೆಬ್ಬೆಟ್ಟಿನ ಗಾತ್ರದ ಒಂದು ಹಂದಿಯ ಮರಿಯು ಹೊರಕ್ಕೆ ನೆಗದು, ಕ್ಷಣಮಾತ್ರದಲ್ಲಿ ದೊಡ್ಡ ಬೆಟ್ಟದಂತೆ ಬೆಳೆಯಿತು. ವಜ್ರದಂತೆ ಕಠೋರವಾದ ದೇಹವುಳ್ಳ ಆ ಹಂದಿಯು ದಶದಿಕ್ಕುಗಳೂ ಮೊಳಗುವಂತೆ ಒಮ್ಮೆ ಗರ್ಜಿಸಿ, ನಡುಗಡ ಲೊಳಗೆ ಧುಮ್ಮಿಕ್ಕಿ ಮಾಯವಾಯಿತು.

ಭೂಮಿ ಸಮುದ್ರದಲ್ಲಿ ಮುಳುಗಿಹೋದುದೇ ಒಂದು ದೊಡ್ಡ ಕಥೆ–ಬ್ರಹ್ಮನ ಮಾನಸಪುತ್ರರಾದ ಸನಕನೇ ಮೊದಲಾದ ನಾಲ್ವರು ಪ್ರಜಾಭಿವೃದ್ಧಿಗೆ ವಿಮುಖರಾಗಿ ವೈರಾಗ್ಯಪರರಾಗಿದ್ದರಷ್ಟೆ! ಸದಾ ಭಗವಂತನ ಚಿಂತನೆಯಲ್ಲಿಯೇ ಮಗ್ನರಾಗಿದ್ದ ಅವರು ಒಮ್ಮೆ ಆದಿಪುರುಷನಾದ ಮಹಾವಿಷ್ಣುವನ್ನು ಕಾಣಲೆಂದು ವೈಕುಂಠಕ್ಕೆ ಹೋದರು. ನಿಷ್ಕಾಮದಿಂದ ಭಗವಂತನನ್ನು ಆರಾಧಿಸುವವರ ನೆಲೆಮನೆ ಅದು. ತೇಜೋರೂಪವಾದ ಆ ವೈಕುಂಠವನ್ನು ಸನಕಾದಿಗಳು ಯೋಗಮಹಿಮೆಯಿಂದ ಪ್ರವೇಶಿಸಿ, ದೇವದೇವನ ಅರ ಮನೆಯ ಅಂತಃಪುರದ ಬಳಿಗೆ ಬಂದರು. ಅಲ್ಲಿ ಜಯ ವಿಜಯರೆಂಬ ಇಬ್ಬರು ದ್ವಾರ ಪಾಲಕರು ನಿಂತಿದ್ದರು. ಪರಬ್ರಹ್ಮದ ಸಾಮೀಪ್ಯದಲ್ಲಿದ್ದರೂ ಅವರ ಅಹಂಕಾರ ಅಡಗಿರ ಲಾರದೆ ಹೋಯಿತೆಂದು ತೋರುತ್ತದೆ. ತಮ್ಮನ್ನು ಲಕ್ಷಿಸದೆ ಒಳಕ್ಕೆ ಪ್ರವೇಶಿಸುತ್ತಿದ್ದ ಆ ಮಹರ್ಷಿಗಳನ್ನು ಅವರು ತಮ್ಮ ಕೈಕೋಲಿನಿಂದ ಅಡ್ಡ ಹಿಡಿದು ನಿಲ್ಲಿಸಿ, ಗದರಿಸಿದರು. ಪರಮಪ್ರಿಯನಾದ ಪರಮಾತ್ಮನನ್ನು ಕಾಣಲು ಅಡ್ಡಿಯಾಗಿ ನಿಂತ ಇವರನ್ನು ಕಂಡು ಪುಷಿಗಳಿಗೆ ರೇಗಿಹೋಯಿತು. ಅವರು ‘ಅಯ್ಯಾ ದ್ವಾರಪಾಲಕರೆ, ಪವಿತ್ರವಾದ ವೈಕುಂಠ ದಲ್ಲಿದ್ದರೂ ನಿಮಗೆ ಅಜ್ಞಾನ ನೀಗಿಲ್ಲ. ಭಯಕ್ಕೆ ಭಯರೂಪನಾಗಿರುವ ಭಗವಂತನಿಗೆ ನಮ್ಮಿಂದ ಭಯವೆಂದು ಶಂಕಿಸಿದಿರಲ್ಲವೆ? ನೀವು ಇಲ್ಲಿರಲು ಯೋಗ್ಯರಲ್ಲ. ನಿಮಗೆ ಅರಿ ಷಡ್ವರ್ಗಗಳಿಂದ ಕೂಡಿರುವ ಅಧೋಲೋಕ ಪ್ರಾಪ್ತಿಯಾಗಲಿ’ ಎಂದರು. ಕೆಟ್ಟಮೇಲೆ ಬುದ್ಧಿ ಬಂದ ಜಯ ವಿಜಯರು ಆ ಪುಷಿಗಳಿಗೆ ಅಡ್ಡಬಿದ್ದು ‘ಮಹಾತ್ಮರೆ, ನಿಮ್ಮ ಶಾಪವೂ ನಮಗೆ ಅನುಗ್ರಹಕಾರಿಯೇ–ನಮ್ಮ ಧರ್ಮದ್ರೋಹದ ಪಾಪ ತಮ್ಮ ಶಾಪ ದಿಂದ ನಿವಾರಣೆಯಾಗುತ್ತದೆ. ನಾವು ನಿಮ್ಮಲ್ಲಿ ಬೇಡುವುದಿಷ್ಟೆ–ನಾವು ಎಲ್ಲಿಯೇ ಹುಟ್ಟಲಿ, ನಮ್ಮ ದೈವಧ್ಯಾನವನ್ನು ಮರೆಸುವಂತಹ ಮೋಹ ನಮಗಾಗದಿರಲಿ, ಹಾಗಾ ದರೂ ದೈವಸ್ಮರಣೆ ತಪ್ಪದಿರಲಿ. ಇಷ್ಟನ್ನು ತಾವು ಅನುಗ್ರಹಿಸಬೇಕು’ ಎಂದು ಬೇಡಿದರು. ಆ ವೇಳೆಗೆ ಪರಾತ್ಪರನಾದ ಪರಬ್ರಹ್ಮಸ್ವರೂಪಿಯೇ ಅಲ್ಲಿಗೆ ಬಂದನು. ಆತನನ್ನು ಕಂಡ ಪುಷಿಗಳು ರೋಮಾಂಚನಗೊಂಡು, ಆತನನ್ನು ವಂದಿಸಿ, ಸ್ತುತಿಸಿದರು. ಆತನು ಮಂದಹಾಸದಿಂದ ಅವರನ್ನು ನೋಡುತ್ತಾ ‘ಅಯ್ಯಾ, ನಾನು ಸದಾ ಬ್ರಾಹ್ಮಣ್ಯವೆಂಬ ದೈವವನ್ನು ಪೂಜಿಸುತ್ತಿರುವುದರಿಂದಲೆ ನನ್ನ ಪಾದಧೂಳಿಕೂಡ ಪವಿತ್ರವೆನಿಸಿದೆ, ಸಮಸ್ತ ಪಾಪಗಳನ್ನೂ ಪರಿಹರಿಸುವ ಶಕ್ತಿ ನನಗಿದೆ. ಆದ್ದರಿಂದ ತಾಪಸಿಗಳಾದ ನಿಮ್ಮಲ್ಲಿ ಅಪ ಚಾರವೆಸಗಿದ ಈ ಜಯ ವಿಜಯರು ತತ್​ಕ್ಷಣವೇ ಅದಕ್ಕೆ ತಕ್ಕ ಫಲವನ್ನು ಪಡೆದಿದ್ದಾರೆ’ ಎಂದನು. ಆಗ ಪುಷಿಗಳು ಆತನನ್ನು ಕುರಿತು ‘ಹೇ, ಸ್ವಪ್ರಕಾಶನಾದ ಭಗವಂತ! ಲೋಕಾಧ್ಯಕ್ಷನೂ ಅವಾಙ್ಮಾನಸಗೋಚರನೂ ಆದ ನೀನು ನಮ್ಮನ್ನು ಹಿರಿಯರನ್ನಾಗಿ ಮಾಡಿ ಆಡುವ ನುಡಿಗಳನ್ನು ಕಂಡರೆ ನಮಗೆ ದಿಗಿಲಾಗುತ್ತದೆ; ನೀನು ನಮ್ಮನ್ನು ಅನು ಗ್ರಹಿಸುವೆಯೊ! ನಿಗ್ರಹಿಸುವೆಯೊ! ಹೇ ಪ್ರಭು, ಈ ಜಯ ವಿಜಯರಿಗೆ ನೀನು ಯಾವ ದಂಡವನ್ನು ವಿಧಿಸುವೆಯೋ, ಅರಿಷಡ್ವರ್ಗಗಳಿಗೆ ಸಿಲುಕಿ ಅವರಿಗೆ ಶಾಪವನ್ನಿತ್ತ ನಮಗೆ ಯಾವ ದಂಡವನ್ನು ವಿಧಿಸುವೆಯೋ, ಅದೆಲ್ಲವೂ ನಮಗೆ ಸಮ್ಮತವೆ’ ಎಂದರು. ಶ್ರೀ ಮಹಾವಿಷ್ಣುವು ಆ ಮಹಾಮುನಿಗಳ ಮಾತಿಗೆ ನಸುನಕ್ಕು, ಅವರ ಶಾಪದಂತೆ ಜಯ ವಿಜಯರು ರಾಕ್ಷಸರಾಗಿ ಹುಟ್ಟಿ, ತನ್ನಲ್ಲಿ ಅತ್ಯಂತ ದ್ವೇಷದಿಂದ ಕ್ರೋಧರೂಪವಾದ ಯೋಗವನ್ನು ಪಡೆದು, ಶೀಘ್ರವಾಗಿಯೇ ತನ್ನ ಬಳಿಗೆ ಹಿಂದಿರುಗುವರೆಂದು ಹೇಳಿ, ಪುಷಿ ಗಳನ್ನು ಬೀಳ್ಕೊಟ್ಟನು.

ಸನಕಾದಿಗಳನ್ನು ಬೀಳ್ಕೊಟ್ಟಮೇಲೆ ಮಹಾವಿಷ್ಣುವು ತನ್ನ ದ್ವಾರಪಾಲಕರನ್ನು ಹತ್ತಿರಕ್ಕೆ ಕರೆದು ‘ಅಯ್ಯಾ, ಬ್ರಹ್ಮತೇಜಸ್ಸನ್ನು ಕೂಡ ನಿರಾಕರಿಸುವ ಶಕ್ತಿ ನನಗಿರುವುದಾದರೂ ಹಾಗೆ ಮಾಡುವುದು ನನಗೆ ಇಷ್ಟವಿಲ್ಲ. ಅಲ್ಲದೆ ಹಿಂದೊಮ್ಮೆ ನೀವು ರಮಾದೇವಿಯನ್ನೂ ಹೀಗೆಯೇ ತಡೆದು, ಇದೇ ಬಗೆಯ ಶಾಪಕ್ಕೆ ಒಳಗಾಗಿದ್ದೀರಿ. ಆದ್ದರಿಂದ ಈ ಶಾಪವನ್ನು ಅನುಭವಿಸುವುದು ನಿಮಗೆ ಅನಿವಾರ್ಯ. ನೀವೀಗ ರಾಕ್ಷಸರೂಪದಿಂದ ಯಮಳರಾಗಿ ಹುಟ್ಟಿ, ಬೇಗ ಹಿಂದಿರುಗಿರಿ’ ಎಂದು ಹೇಳಿ ಅವರನ್ನು ಬೀಳ್ಕೊಟ್ಟನು. ಹತಭಾಗ್ಯರಾದ ಅವರು ದಿವ್ಯಲೋಕದಿಂದ ಕೆಳಗುರುಳಿದರು.

ಜಯವಿಜಯರು ವೈಕುಂಠಚ್ಯುತರಾಗುವ ವೇಳೆಗೆ ಅವರು ನೆಲೆಸಲು ಯೋಗ್ಯವಾದ ಗರ್ಭ ಸಿದ್ಧವಾಗಿತ್ತು. ದಕ್ಷಬ್ರಹ್ಮನ ಪುತ್ರಿಯಾದ ದಿತಿಯು ಕಶ್ಯಪಬ್ರಹ್ಮನ ಧರ್ಮಪತ್ನಿ. ಆಕೆಯು ಇದ್ದಕ್ಕಿದ್ದಂತೆ ಒಂದು ಸಂಜೆ ಪುತ್ರಕಾಮಿನಿಯಾಗಿ, ಮನ್ಮಥಬಾಧೆಗೆ ಈಡಾ ದಳು. ಆ ವೇಳೆಯಲ್ಲಿ ಆಕೆಯ ಗಂಡ ಸಂಧ್ಯಾಕರ್ಮದಲ್ಲಿ ನಿರತನಾಗಿ ಅಗ್ನಿಹೋತ್ರಗೃಹ ದಲ್ಲಿ ಕುಳಿತಿದ್ದನು. ದಿತಿಯು ಅಲ್ಲಿಗೆ ಹೋಗಿ, ತನ್ನ ಮನೋರಥವನ್ನು ಆ ಕ್ಷಣವೇ ಈಡೇರಿಸುವಂತೆ ಬೇಡಿದಳು. ಪುಷಿಯು ಆಕೆಯನ್ನು ವಿಮುಖಳನ್ನಾಗಿ ಮಾಡಲು ಪರಿಪರಿ ಯಾಗಿ ಪ್ರಯತ್ನಿಸಿದರೂ ಅದು ವ್ಯರ್ಥವಾಯಿತು. ‘ಹೇ ಸುಂದರಿ, ಸಂಧ್ಯಾಕಾಲವು ಭಯಂಕರರಾದವರಿಗೂ ಭಯವನ್ನುಂಟುಮಾಡುವ ಕಾಲ. ಈ ವೇಳೆಯಲ್ಲಿ ರುದ್ರನು ವೃಷಭಾರೂಢನಾಗಿ ಜಗತ್ತನ್ನೆಲ್ಲ ಸಂಚರಿಸುತ್ತಿರುವನು. ಆತನ ಮೂರು ಕಣ್ಣುಗಳು ಜಗತ್ತಿನ ಮೂಲೆ ಮೂಲೆಯನ್ನೂ ನೋಡುತ್ತಿರುತ್ತದೆ. ಆತನಿಗೆ ಮರೆಮಾಚುವುದು ಸಾಧ್ಯ ವಿಲ್ಲ. ಆತನಿಗೆ ಸ್ವಜನ, ಪರಜನರೆಂಬ ಭೇದವಿಲ್ಲ. ಆತನ ಕಣ್ಣೆದುರಿನಲ್ಲಿ ನಿಷಿದ್ಧಕಾರ್ಯ ವನ್ನು ಆಚರಿಸುವುದು ನಾಚಿಕೆಗೇಡು ಮಾತ್ರವೇ ಅಲ್ಲ, ಭಯಂಕರ ಪರಿಣಾಮಕ್ಕೂ ಕಾರಣವಾಗುವುದು’ ಎಂದು ಆತನು ಬೋಧಿಸಿದರೂ ಕಾಮಜ್ವರದಿಂದ ಕಂಗೆಟ್ಟು ಮೋಹಮಗ್ನಳಾಗಿದ್ದ ಆ ಹೆಣ್ಣು ಆತನನ್ನು ಧಿಕ್ಕರಿಸಿ, ನಿರ್ಲಜ್ಜೆಯಿಂದ ಆತನ ಸೆರಗನ್ನು ಹಿಡಿದು ಬಲವಂತಮಾಡಿದಳು. ಬ್ರಹ್ಮರ್ಷಿಯಾದ ಕಶ್ಯಪಮುನಿಯು ಮುಂಗಾಣದವ ನಾಗಿ ಆಕೆಯ ಅಭೀಷ್ಟವನ್ನು ಆಗಲೇ ನೆರವೇರಿಸಿ, ಸ್ನಾನ ಆಚಮನ ಪ್ರಾಣಾಯಾಮದಿಂದ ಪರಿಶುದ್ಧನಾಗಿ, ಗಾಯತ್ರಿಯ ಜಪದಲ್ಲಿ ಮಗ್ನನಾದನು. ದಿತಿಗೆ ಕೃತಕಾರ್ಯಕ್ಕಾಗಿ ಪಶ್ಚಾ ತ್ತಾಪವಾಯಿತು; ಆಕೆ ಗಂಡನಲ್ಲಿ ಕ್ಷಮೆಯನ್ನು ಯಾಚಿಸಿದಳು. ಕಶ್ಯಪಮುನಿಯು ಆಕೆ ಯನ್ನು ಕುರಿತು ‘ಎಲೆ ಅಮಂಗಳೆಯಾದ ಚಂಡಿ, ಕಾಲವಲ್ಲದ ಕಾಲದಲ್ಲಿ ಕಾಮಚಾರಿಣಿ ಯಾದ ನಿನ್ನ ಗರ್ಭದಲ್ಲಿ ಇಬ್ಬರು ದುರ್ಮಾರ್ಗರು ಹುಟ್ಟುತ್ತಾರೆ. ಅವರು ಮೂರು ಲೋಕ ಗಳಿಗೂ ಕಂಟಕರಾಗುತ್ತಾರೆ. ಕಡೆಗೆ ಜಗದೀಶ್ವರನಾದ ಭಗವಂತನು ಈ ಭೂಮಿಯಲ್ಲಿ ಅವತರಿಸಿ ಅವರನ್ನು ಕೊಲ್ಲುತ್ತಾನೆ’ ಎಂದು ಹೇಳಿದನು. ಜಯ ವಿಜಯರು ಈ ದಿತಿಯ ಗರ್ಭವನ್ನು ಪ್ರವೇಶಿಸಿದರು.

ದಿತಿಯು ನೂರುವರ್ಷಗಳವರೆಗೆ ಗರ್ಭವತಿಯಾಗಿದ್ದಳು. ಆಕೆಯ ಗರ್ಭ ಬೆಳದಂತೆಲ್ಲ ಜಗತ್ತಿಗೆ ಕತ್ತಲೆ ಮುಸುಕುತ್ತಾ ಹೋಯಿತು; ಇಂದ್ರನೇ ಮೊದಲಾದ ದೇವತೆಗಳು ನಿಸ್ತೇಜ ರಾದರು; ಸೂರ್ಯಚಂದ್ರಾದಿಗಳು ಕಳೆಗುಂದಿದರು; ಹಗಲಿರುಳೆಂಬ ಭೇದ ಗೊತ್ತಾಗದೆ ಜಗತ್ತಿನ ಕರ್ಮಗಳಿಗೆ ಲೋಪವೊದಗಿತು. ದಿತಿಯು ತನ್ನ ಹೊಟ್ಟೆಯಲ್ಲಿ ಹುಟ್ಟುವ ಮಕ್ಕಳು ಇನ್ನೆಂತಹ ಪಾಪಿಗಳಾಗುವರೋ ಎಂದು ಶಂಕಿಸುತ್ತಾ, ನೂರು ವರ್ಷಗಳು ತುಂಬಿದ ಮೇಲೆ ಅವಳಿಯಾದ ಇಬ್ಬರು ಗಂಡು ಮಕ್ಕಳನ್ನು ಹೆತ್ತಳು. ಅವರು ಹುಟ್ಟು ತ್ತಲೇ ಉತ್ಪಾತಗಳಾದವು. ಭೂಮಿ ನಡುಗಿತು, ದಿಕ್ಕುಗಳು ಹೊತ್ತಿಕೊಂಡವು, ಬರಸಿಡಿಲು ಬಡಿಯಿತು, ಬಿರುಗಾಳಿಯೆದ್ದಿತು. ಆಕಾಶದಲ್ಲಿ ಮೋಡಗಳೆದ್ದು ಸೂರ್ಯ ಚಂದ್ರರ ತೇಜಸ್ಸು ಉಡುಗಿತು, ಸಮುದ್ರವು ಅಲ್ಲೋಲಕಲ್ಲೋಲವಾಯಿತು, ನದಿ ಕೆರೆ ಬಾವಿಗಳ ನೀರು ಬತ್ತಿ ದವು, ನರಿಗಳು ಊಳಿಟ್ಟವು, ಗೂಗೆಗಳು ಊರೊಳಗೆ ನುಗ್ಗಿಬಂದವು, ದೇವತಾ ಪ್ರತಿಮೆ ಗಳು ಕಣ್ಣೀರುಗರೆದವು, ಜಗತ್ತಿಗೆ ಪ್ರಳಯವೇ ಸನ್ನಿಹಿತವಾದಂತೆ ಭಾಸವಾಯಿತು. ಹೀಗೆ ಭಯಂಕರವಾದ ಉತ್ಪಾತಗಳೊಡನೆ ಹುಟ್ಟಿದ ದಿತಿಯ ಮಕ್ಕಳು ದಿನದಿನಕ್ಕೆ ಭಯಂಕರಾ ಕಾರದಿಂದ ಮಹಾಪರ್ವತಾಕಾರವಾಗಿ ಬೆಳೆಯುತ್ತಿದ್ದರು. ಕಶ್ಯಪಬ್ರಹ್ಮನು ತನ್ನ ಈ ಮಕ್ಕ ಳಿಗೆ ಹಿರಣ್ಯಾಕ್ಷ ಹಿರಣ್ಯಕಶಿಪುವೆಂದು ನಾಮಕರಣ ಮಾಡಿದನು. ಹಿರಿಯನಾದ ಹಿರಣ್ಯ ಕಶಿಪು ಬ್ರಹ್ಮನನ್ನು ಕುರಿತು ಉಗ್ರತಪವನ್ನಾಚರಿಸಿ, ಯಾರಿಂದಲೂ ತನಗೆ ಮರಣವಾಗದ ವರವನ್ನು ಪಡೆದನು. ಈ ವರಗಳ ಮಹಿಮೆಯಿಂದ ಅವನು ಮದೋನ್ಮತ್ತನಾಗಿ ಸಕಲ ಲೋಕಪಾಲರನ್ನೂ ಜಯಿಸಿ, ಮೂರು ಲೋಕಗಳಿಗೂ ಒಡೆಯನಾದನು. 

ಹಿರಣ್ಯಕಶಿಪುವಿನ ತಮ್ಮನಾದ ಹಿರಣ್ಯಾಕ್ಷನು ತನ್ನ ಅಣ್ಣನಿಗಿಂತಲೂ ಬಲಶಾಲಿಯಾಗಿ ದ್ದನು. ಅವನು ತನ್ನ ಭುಜದಂಡದಲ್ಲಿ ಗದಾದಂಡವನ್ನು ಧರಿಸಿ, ದೇವಲೋಕಕ್ಕೆ ದಾಳಿ ಯಿಟ್ಟನು. ಅವನನ್ನು ಕಾಣುತ್ತಲೆ ಗರುಡನನ್ನು ಕಂಡ ಹಾವಿನಮರಿಗಳಂತೆ ದೇವತೆ ಗಳೆಲ್ಲರೂ ಅಲ್ಲಲ್ಲಿಯೇ ಅಡಗಿಕೊಂಡರು. ಆ ಹೇಡಿಗಳೆಲ್ಲ ನಡುಗುವಂತೆ ಒಮ್ಮೆ ಗರ್ಜಿ ಸಿದ ಹಿರಣ್ಯಾಕ್ಷನು ತನ್ನ ಎದುರಾಳಿಯನ್ನು ಅರಸುತ್ತಾ, ಮದ್ದಾನೆಯಂತೆ ಸಮುದ್ರವನ್ನು ಪ್ರವೇಶಿಸಿದನು. ಅಲ್ಲಿಯೂ ಆತನಿಗೆ ಎದುರಾಳಿ ಸಿಕ್ಕಲಿಲ್ಲ. ಆತನು ಸಮುದ್ರರಾಜನಾದ ವರುಣನನ್ನು ಯುದ್ಧಕ್ಕೆ ಆಹ್ವಾನಿಸಲು ಆತನು ‘ಅಯ್ಯಾ ವೀರನೆ, ನಾನು ವೈರಾಗ್ಯಪರ ನಾಗಿದ್ದೇನೆ, ಯುದ್ಧ ನನಗೆ ಬೇಡ. ಪುರಾಣಪುರುಷೋತ್ತಮನಾದ ಪರಮಾತ್ಮನೊಬ್ಬನೇ ನಿನ್ನೊಡನೆ ಯುದ್ಧಮಾಡಬಲ್ಲವನು’ ಎಂದನು. ಹಿರಣ್ಯಾಕ್ಷನು ಆ ಪರಮಾತ್ಮನನ್ನು ಅರಸುತ್ತ ನಡೆದನು. ತನ್ನ ಕೋರೆದಾಡೆಯ ಮೇಲೆ ಭೂಮಿಯನ್ನಿಟ್ಟುಕೊಂಡು ಅದನ್ನು ಬ್ರಹ್ಮನಿಗೆ ಕೊಡುವುದಕ್ಕಾಗಿ ನೀರಿನಿಂದ ಮೇಲೇರಿಬರುತ್ತಿದ್ದ ಭಯಂಕರ ವರಾಹವು ಈ ಹಿರಣ್ಯಾಕ್ಷನ ಕಣ್ಣಿಗೆ ಬಿತ್ತು. ಅದನ್ನು ಕಂಡು ಅವನಿಗೆ ಆಶ್ಚರ್ಯವಾಯಿತು. ಅವನು ಗಂಡಗರ್ವದಿಂದ “ಹೇ ಮೂಢ ಹಂದಿ, ಈ ಭೂಮಿಯನ್ನು ಎಲ್ಲಿಗೆ ಒಯ್ಯುವೆ? ಇದು ಪಾತಾಳವಾಸಿಗಳಿಗಾಗಿ ನಿರ್ಮಿತವಾದುದು. ಇದನ್ನು ಇದ್ದಲ್ಲಿ ಇಟ್ಟು ಹಿಂದಿರುಗು. ನನ ಗೀಗ ಗೊತ್ತಾಯಿತು, ಹಂದಿಯ ರೂಪವನ್ನು ಧರಿಸಿರುವ ನೀನೇ ಆ ಮಹಾವಿಷ್ಣು ಎನ್ನು ವವನು. ಮರೆಮೋಸಗಳಿಂದ ರಾಕ್ಷಸರನ್ನು ಕೊಂದು ‘ದಾನವಾರಿ’ ಎಂಬ ಬಿರುದನ್ನು ಧರಿಸಿರುವೆಯಾ? ನನ್ನ ಈ ದೋರ್ದಂಡದಲ್ಲಿರುವ ಗದಾದಂಡದಿಂದ ನಿನ್ನ ಹೆಡೆತಲೆಯ ನ್ನೊಡೆದು, ನಿನ್ನನ್ನು ಆರಾಧಿಸುತ್ತಿರುವ ಬಿಡುಗಣ್ಣರನ್ನೂ ಗೊರವರನ್ನೂ ದಿಕ್ಕಿಲ್ಲದವರ ನ್ನಾಗಿ ಮಾಡುತ್ತೇನೆ. ಹೇಡಿಯಾದರೂ ಮಾಯಾಬಲದಿಂದ ಕೊಬ್ಬಿರುವ ನಿನ್ನನ್ನು ಕೊಂದು ದಾನವ ಸಂಹಾರಕ್ಕೆ ಪ್ರತೀಕಾರ ಮಾಡುವೆನು” ಎಂದು ಗರ್ಜಿಸಿದನು. ಆ ರಕ್ಕಸನನ್ನು ಕಂಡು ಭೂದೇವಿ ಗಡಗಡ ನುಡುಗಿದಳು.

ರಕ್ಕಸನ ದುರುಕ್ತಿಗಳನ್ನು ಕೇಳಿ ವರಾಹಮೂರ್ತಿ ಚಲವಿಚಲಿತನಾಗಲಿಲ್ಲ. ಆತನು ನೇರ ವಾಗಿ ನೀರಿನಿಂದ ಮೇಲೆಕ್ಕೆದ್ದು ಬಂದನು. ಆತನ ಈ ಔದಾಸೀನ್ಯವನ್ನು ಕಂಡು ರಕ್ಕಸನಿಗೆ ರೇಗಿಹೋಯಿತು. ಅವನು ಆತನ ಬೆನ್ನ ಹಿಂದೆಯೇ ಬಂದು “ಎಲೆ ಹೇಡಿ, ನಿನಗೆ ನಾಚಿಕೆಯಾಗುವುದಿಲ್ಲವೆ?” ಎಂದು ಆತನನ್ನು ಹೀಯಾಳಿಸಿದನು. ವರಾಹರೂಪಿಯಾದ ಭಗವಂತನು ತನ್ನ ಕೋರೆದಾಡೆಯ ಮೇಲಿದ್ದ ಭೂಮಿಯನ್ನು ಒಂದು ಪಕ್ಕದಲ್ಲಿಟ್ಟು, ಕಠೋರವಾಗಿ ತನ್ನನ್ನು ನಿಂದಿಸುತ್ತಿರುವ ಹಿರಣ್ಯಾಕ್ಷನತ್ತ ತಿರುಗಿದನು. ಕೆಂಗಣ್ಣುಗಳಿಂದ ಅವನ ಕಡೆ ನೋಡುತ್ತಾ “ಎಲಾ, ಮೃತ್ಯುವಿನ ಬಾಯಿಗೆ ತುತ್ತಾಗಿರುವ ನಿನ್ನ ಬಾಯಿಂದ ಅಮಂಗಳಕರವಾದ ಮಾತುಗಳು ಬರುವುದು ಸಹಜ. ನಿನ್ನ ನುಡಿಯ ಬಡಿವಾರವನ್ನು ನಿಲ್ಲಿಸಿ, ಹೊಡೆದಾಟಕ್ಕೆ ಸಿದ್ಧನಾಗು” ಎಂದು ಹೇಳಿದನು. ಆ ಮಾತುಗಳನ್ನು ಕೇಳಿದ ಹಿರಣ್ಯಾಕ್ಷನು ಕಾಲಿಂದ ಮೆಟ್ಟಿದ ಕಾಳಸರ್ಪದಂತೆ ಬುಸುಗುಟ್ಟುತ್ತಾ, ತನ್ನ ಗದೆಯನ್ನು ಗರಗರನೆ ತಿರುಗಿಸಿಕೊಂಡು ವರಾಹಮೂರ್ತಿಯ ಬಳಿಗೆ ಬಂದನು. 

ಹರಿ ಹಿರಣ್ಯಾಕ್ಷರಿಗೆ ಘೋರವಾದ ಯುದ್ಧ ನಡೆಯಿತು. ಹಿರಣ್ಯಾಕ್ಷನು ಬೀಸಿದ ಗದೆಯ ಪೆಟ್ಟನ್ನು ಗದಾಯುದ್ಧ ಪಾರಂಗತನಾದ ಶ್ರೀಹರಿಯು ತಪ್ಪಿಸಿಕೊಂಡು ತನ್ನ ಗದಾಘಾತ ದಿಂದ ರಕ್ಕಸನ ಗದೆಯನ್ನು ನೆಲಕ್ಕೆ ಬೀಳಿಸಿದನು. ಹಿರಣ್ಯಾಕ್ಷನು ಅದನ್ನು ಎತ್ತಿಕೊಂಡು ಮತ್ತೆ ವರಾಹಮೂರ್ತಿಯನ್ನು ಹೊಡೆದನು. ಪರಸ್ಪರ ಹೊಡೆತಗಳಿಂದ ಇಬ್ಬರ ದೇಹ ಗಳೂ ರಕ್ತಮಯವಾದವು. ಇಬ್ಬರೂ ಮಲ್ಲರು ಸಮಾನಬಲಶಾಲಿಗಳಂತೆ ಕಾಣುತ್ತಿತ್ತು. ಈ ಯುದ್ಧವನ್ನು ನೋಡುತ್ತಾ ಸನಕ ಮರೀಚಾದಿ ಪುಷಿಗಳೊಡನೆ ಅಲ್ಲಿ ನಿಂತಿದ್ದ ಬ್ರಹ್ಮನು ವರಾಹಮೂರ್ತಿಯನ್ನು ಕುರಿತು “ಹೇ ಪುರಾಣಪುರುಷ! ಗೋಬ್ರಾಹ್ಮಣರಿಗೂ, ದೇವತೆಗಳಿಗೂ ಕಂಟಕನಾಗಿರುವ ಈ ದೈತ್ಯಾಧಮನೊಡನೆ ಹಾವನ್ನಾಡಿಸುವ ಹಸುಳೆ ಯಂತೆ ಸರಸವಾಡುತ್ತಿರುವೆಯಲ್ಲ! ಸಂಜೆಯಾಗುತ್ತಲೆ ರಾಕ್ಷಸಬಲ ವೃದ್ಧಿಯಾಗುತ್ತದೆ. ಆದ್ದರಿಂದ ಬೇಗ ಇವನನ್ನು ಸಂಹರಿಸಬಾರದೆ? ಈಗ ಅಭಿಜಿತ್ತೆಂಬ ಮಂಗಳಕರವಾದ ಮುಹೂರ್ತ ಬಂದಿದೆ. ಈಗಲೆ ಇವನನ್ನು ಮುಗಿಸಿಬಿಡು” ಎಂದನು. 

ಬ್ರಹ್ಮನ ಮಾತುಗಳನ್ನು ಕೇಳಿ ಪರಬ್ರಹ್ಮನಾದ ಭಗವಂತ ಮನಸ್ಸಿನಲ್ಲಿಯೇ ನಕ್ಕ–‘ಕಾಲ ಸ್ವರೂಪನಾದ ತನಗೆ ಈ ಬ್ರಹ್ಮನೂ ಮುಹೂರ್ತವನ್ನು ಹೇಳಿಕೊಡುತ್ತಿರುವ ನಲ್ಲಾ’ ಎಂದು. ಆದರೂ ಅವನ ಮಾತನ್ನು ಪುರಸ್ಕರಿಸುವವನಂತೆ ಅವನ ಕಡೆ ತನ್ನ ಕೃಪಾಕಟಾಕ್ಷವನ್ನು ಬೀರಿ, ಒಮ್ಮೆ ಅಂತರಿಕ್ಷಕ್ಕೆ ನೆಗೆದವನೇ ತನ್ನ ಗದೆಯಿಂದ ರಕ್ಕಸನನ್ನು ಅಪ್ಪಳಿಸಿದನು. ಆದರೆ ಹಿರಣ್ಯಾಕ್ಷನು ಲಾಘವದಿಂದ ಅದನ್ನು ತಪ್ಪಿಸಿಕೊಂಡು, ತನ್ನ ಗದೆಯ ಹೊಡೆತದಿಂದ ವರಾಹಮೂರ್ತಿಯ ಗದೆಯನ್ನು ನೆಲಕ್ಕೆ ಕೆಡಹಿದನು. ಆದರೂ ಆ ರಕ್ಕಸ, ಧರ್ಮಯುದ್ಧಕ್ಕೆ ಲೋಪ ಬರದಂತೆ, ನಿರಾಯುಧನಾದ ವರಾಹನನ್ನು ಹೊಡೆಯದೆ ಕೇವಲ ಕಟುನುಡಿಗಳಿಂದ ಆತನನ್ನು ಹೀಯಾಳಿಸಿದನು. ಇದನ್ನು ಕಂಡು ಬ್ರಹ್ಮನೇ ಮೊದಲಾದ ದೇವತೆಗಳು ಹಾಹಾಕಾರ ಮಾಡಿದರು. ಆಗ ವರಾಹಮೂರ್ತಿಯು ಅವರನ್ನು ಸಮಾಧಾನಪಡಿಸಿ, ತನ್ನ ಚಕ್ರಾಯುಧವನ್ನು ನೆನೆದನು. ಒಡನೆಯೇ ಅದು ಕಿಡಿಗಳನ್ನುಗುಳುತ್ತಾ ಬಂದು ಆತನ ಕೈಯಲ್ಲಿ ನಿಂತಿತು. ಅದು ಆ ರಕ್ಕಸನ ಗದಾಪ್ರಹಾರ, ತ್ರಿಶೂಲ ಪ್ರಯೋಗ, ಮುಷ್ಟಿ ಪ್ರಹಾರಗಳನ್ನೆಲ್ಲ ವ್ಯರ್ಥಗೊಳಿಸಿತು. ಸೋಲುವ ಸಮಯ ಸನ್ನಿಹಿತವಾಗುತ್ತಲೇ ಹಿರಣ್ಯಾಕ್ಷನು ಮಾಯಾವಿದ್ಯೆಗೆ ಶರಣು ಹೋದನು. ಆತನ ಮಾಯೆ ಯಿಂದ ಧೂಳು ತುಂಬಿದ ಬಿರುಗಾಳಿಯೆದ್ದು, ಅಂತರಿಕ್ಷವನ್ನೆಲ್ಲ ಕಗ್ಗತ್ತಲು ಮುಸುಕಿತು. ದಿಕ್ಕುದಿಕ್ಕುಗಳಿಂದ ಕವಣೆಯಲ್ಲಿ ಎಸೆದಂತೆ ಕಲ್ಲುಗಳ ಮಳೆ ಸುರಿಯಿತು. ವಿಕಟಾಕಾರದ ಯಕ್ಷ ರಾಕ್ಷಸ ಪಿಶಾಚಿಗಳು ‘ಕಡಿ ಬಡಿ ಕೊಲ್ಲು’ ಎಂದು ಕೂಗುತ್ತಾ ಕೋಟಿಗಟ್ಟಲೆಯಾಗಿ ಕಾಣಿಸಿಕೊಂಡವು. ಆದರೇನು? ಸಮಸ್ತ ಮಾಯೆಗೂ ಅಧಿಪತಿಯಾದ ಮಹಾವಿಷ್ಣುವಿನ ಇದಿರಿನಲ್ಲಿ ಆ ರಕ್ಕಸನ ಮಾಯೆ ನಿಲ್ಲುವುದೆ? ಸುದರ್ಶನ ಆ ಮಾಯೆಗಳನ್ನೆಲ್ಲ ಕತ್ತರಿಸಿ ಹಾಕಿತು. ಆಗ ಹಿರಣ್ಯಾಕ್ಷನು ರಭಸದಿಂದ ನುಗ್ಗಿ ಬಂದು ವರಾಹಮೂರ್ತಿಯನ್ನು ತನ್ನ ತೋಳುಗಳಿಂದ ಹಿಚುಕಿಹಾಕಲು ಯತ್ನಿಸಿದನು. ಆದರೆ ಆ ದೇವದೇವನು ಒಮ್ಮೆ ಅವನ ಕಾಪಾಳಕ್ಕೆ ತನ್ನ ವಜ್ರಮುಷ್ಟಿಯಿಂದ ಹೊಡೆಯುತ್ತಲೆ ಅವನು ಆಪಾದ ಮಸ್ತಕವೂ ಗಡಗಡ ನಡಗುತ್ತ, ನೆಲಕ್ಕೆ ಬಿದ್ದು ಸತ್ತು ಹೋದನು. ಇದನ್ನು ಕಂಡು ಬ್ರಹ್ಮಾದಿ ದೇವತೆಗಳು ಮಹಾವಿಷ್ಣುವನ್ನು ಮಕ್ತಕಂಠದಿಂದ ಹೊಗಳಿ, ಸ್ತೋತ್ರಮಾಡಿದರು. ಆ ದೇವದೇವನು ಒಮ್ಮೆ ಅವರತ್ತ ಕೃಪಾದೃಷ್ಟಿ ಬೀರಿ ಅಂತರ್ಧಾನವಾದನು.

ಭಗವಂತನ ಕೃಪೆಯಿಂದ ಸಮುದ್ರದಲ್ಲಿ ಮುಳುಗಿದ್ದ ಭೂಮಿ ಮತ್ತೆ ಸ್ವಸ್ಥಾನವನ್ನು ಸೇರಿತು. ಸ್ವಾಯಂಭುವಮನುವು ತನ್ನ ಪತ್ನಿಯಾದ ಶತರೂಪೆಯೊಡನೆ ಭೂಮಿಯ ಮೇಲೆ ನೆಲೆಸಿ, ಪ್ರಿಯವ್ರತ, ಉತ್ತಾನಪಾದ–ಎಂಬ ಇಬ್ಬರು ಗಂಡುಮಕ್ಕಳನ್ನೂ ಆಕೂತಿ, ದೇವಹೂತಿ, ಪ್ರಸೂತಿ–ಎಂಬ ಹೆಣ್ಣುಮಕ್ಕಳನ್ನೂ ಪಡೆದು, ಅವರ ಮೂಲಕ ಪ್ರಜಾಸಂಖ್ಯೆಯ ಪ್ರಗತಿಗೆ ಕಾರಣನಾದನು. 

