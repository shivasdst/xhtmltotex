
\chapter{೧೧. ಪ್ರವೃತ್ತಿ ನಿವೃತ್ತಿ ಧರ್ಮಗಳು}

“ಅಮ್ಮ, ದುಷ್ಕರ್ಮದಿಂದ ದುರ್ಗತಿಯಾದರೆ, ಸತ್ಕರ್ಮದಿಂದ ಸದ್ಗತಿ ದೊರೆಯು ತ್ತದಷ್ಟೆ. ಈ ಸತ್ಕರ್ಮದಲ್ಲಿಯೂ ಎರಡು ವಿಧ: ಫಲಾಪೇಕ್ಷೆಯಿಂದ ಮಾಡುವ ಸಕಾಮ ಕರ್ಮ, ಫಲಾಪೇಕ್ಷೆಯಿಲ್ಲದೆ ಮಾಡುವ ನಿಷ್ಕಾಮಕರ್ಮ–ಇವನ್ನು ಕ್ರಮವಾಗಿ ಪ್ರವೃತ್ತಿ ಧರ್ಮ, ನಿವೃತ್ತಿಧರ್ಮ ಎಂದು ಕರೆಯುತ್ತಾರೆ. ಸಂಸಾರಿಗಳಾದವರು ಸಾಮಾನ್ಯವಾಗಿ ಪ್ರವೃತ್ತಿಧರ್ಮವನ್ನು ಆಚರಿಸುತ್ತಾರೆ. ಹಾಲಿನ ಆಸೆಯಿಂದ ಆಕಳನ್ನು ಮೇಯಿಸಿ, ಹಾಲನ್ನು ಪಡೆದ ಮೇಲೆ ಪುನಃ ಹಾಲನ್ನು ಪಡೆವ ಆಸೆಯಿಂದ ಆಕಳನ್ನು ಮೇಯಿಸುವ ಗೊಲ್ಲನ ಹಾಗೆ, ಈ ಸಂಸಾರಿಗಳೆಂಬುವರು. ನಿಷ್ಕಾಮನಾಗಿ ಭಗವಂತನನ್ನು ಆರಾಧಿಸುವು ದಕ್ಕೆ ಬದಲಾಗಿ ಅವರು ತಮ್ಮ ಸಂಸಾರ ಸುಗುಮವಾಗಿ ನಡೆದುಕೊಂಡು ಹೋಗುವಂತಹ ಧನಧಾನ್ಯ, ಭೋಗಭಾಗ್ಯಗಳಿಗಾಗಿ ಕಾಮ್ಯಕರ್ಮಗಳನ್ನು ಆಚರಿಸುತ್ತಾರೆ. ಯಜ್ಞಯಾಗಾದಿ ಗಳಿಂದ ದೇವತೆಗಳನ್ನೂ ಶ್ರಾದ್ಧಾದಿಗಳಿಂದ ಪಿತೃಗಳನ್ನೂ ತೃಪ್ತಿಪಡಿಸುವವನು ಈ ಪುಣ್ಯಕರ್ಮಗಳ ಫಲವಾಗಿ ಧೂಮಗತಿಯಿಂದ\footnote{೧. ಧೂಮಗತಿ ಎಂದರೆ ಧೂಮ, ರಾತ್ರಿ, ಕೃಷ್ಣಪಕ್ಷ, ದಕ್ಷಿಣಾಯನ, ಪಿತೃಲೋಕ, ಆಕಾಶಗಳ ಮೂಲಕ ಚಂದ್ರಲೋಕಕ್ಕೆ ಹೋಗಿ ಸೇರುವುದು.} ಚಂದ್ರಲೋಕವನ್ನು ಸೇರಿ, ತನ್ನ ಪುಣ್ಯವು ತೀರುವವರೆಗೆ ಅಲ್ಲಿ ಸೋಮಪಾನದ ಸುಖವನ್ನು ಅನುಭವಿಸುತ್ತಿದ್ದು, ಆ ಪುಣ್ಯ ತೀರು ತ್ತಲೆ ಅಲ್ಲಿಂದ ಉರುಳಿ ಆಕಾಶ, ವಾಯು, ಧೂಮ, ಮೇಘಗಳ ಮೂಲಕ ಭೂಲೋಕಕ್ಕೆ ಬೀಳುತ್ತಾನೆ. ಇಷ್ಟೇ ಅಲ್ಲ, ಕಾಮ್ಯಫಲಗಳನ್ನು ನೀಡುವ ಚಂದ್ರಲೋಕಾದಿ ಪುಣ್ಯ ಲೋಕ ಗಳೆಲ್ಲ ಅಸ್ಥಿರವಾದುವು. ಪ್ರಳಯಕಾಲದಲ್ಲಿ ಆ ಲೋಕಗಳೆಲ್ಲ ನಾಶವಾಗಿ ಹೋಗುತ್ತವೆ. ಆದ್ದರಿಂದ ಪ್ರವೃತ್ತಿಧರ್ಮವನ್ನು ಕೈಕೊಂಡವನು ಶಾಶ್ವತ ಸುಖವನ್ನು ಪಡೆಯಲಾರ. ಶಾಶ್ವತಸುಖ ಬೇಕೆನ್ನುವವ ನಿವೃತ್ತಿಧರ್ಮವನ್ನೇ ಕೈಕೊಳ್ಳಬೇಕು.”

“ನಿವೃತ್ತಿಧರ್ಮವನ್ನು ಆಚರಿಸುವವರು ಪರಬ್ರಹ್ಮನ ಉಪಾಸಕರು. ಅವರಿಗೆ ಅಹಂ ಕಾರ ಮಮಕಾರಗಳಿಲ್ಲ. ಅವರು ಶಮದಮಾದಿಗಳಿಂದ ಕೂಡಿದ ಶುದ್ಧಮನಸ್ಸಿನವರು. ಅರ್ಥಕಾಮಗಳನ್ನು ಕೊಡುವ ವರ್ಣಾಶ್ರಮ ಧರ್ಮಗಳಿಗಿಂತಲೂ ಸ್ವಧರ್ಮದಲ್ಲಿ ಹೆಚ್ಚು ನಿಷ್ಠೆಯುಳ್ಳವರು. ಕೇವಲ ಈಶ್ವರಾರ್ಪಣಬುದ್ಧಿಯಿಂದ ಅವರು ಕರ್ಮಗಳನ್ನು ಆಚರಿಸು ತ್ತಾರೆ. ಅಂತಹವರು ದೇಹಾಂತರದಲ್ಲಿ ಅರ್ಚಿರಾದಿಗತಿ\footnote{೧. ಧೂಮಗತಿ ಎಂದರೆ ಧೂಮ, ರಾತ್ರಿ, ಕೃಷ್ಣಪಕ್ಷ, ದಕ್ಷಿಣಾಯನ, ಪಿತೃಲೋಕ, ಆಕಾಶಗಳ ಮೂಲಕ ಚಂದ್ರಲೋಕಕ್ಕೆ ಹೋಗಿ ಸೇರುವುದು.}ಯಿಂದ ಪರಮಾತ್ಮನನ್ನು ನೇರ ವಾಗಿ ಹೋಗಿ ಸೇರುತ್ತಾರೆ. ನಿವೃತ್ತಿಧರ್ಮವನ್ನು ಆಚರಿಸುವವರು ತಮ್ಮ ಬ್ರಹ್ಮೋ ಪಾಸನ ಕರ್ಮದಲ್ಲಿ ಸಿದ್ಧಿಯನ್ನು ಪಡೆಯುವ ಮುನ್ನವೇ ಸತ್ತು ಹೋದರೆ ಅಂತಹವರು ಬ್ರಹ್ಮಲೋಕಕ್ಕೆ ಹೋಗಿ ನೆಲಸಿ, ಸದಾ ಪರಮಾತ್ಮನ ಧ್ಯಾನದಲ್ಲಿರುತ್ತಾ, ಮಹಾಪ್ರಳಯ ವಾದಾಗ ಬ್ರಹ್ಮನ ಜೊತೆಯಲ್ಲಿಯೇ ಪರಬ್ರಹ್ಮನ ಸಾನ್ನಿಧ್ಯವನ್ನು ಪಡೆಯುವರು. ತಾಯಿ, ಸಮಸ್ತ ಲೋಕಗಳನ್ನೂ ಸೃಷ್ಟಿಸುವ ಮಹಿಮಾಶಾಲಿ ಆ ಚತುರ್ಮುಖ ಬ್ರಹ್ಮನೂ ಕೂಡ, ನಿಷ್ಕಾಮಕರ್ಮಗಳನ್ನು ಮಾಡಿದವನಾದರೂ ಕೂಡ, ‘ನಾನು ನಿಷ್ಕಾಮಕರ್ಮ ಗಳನ್ನು ಮಾಡಿದವನು’ ಎಂಬ ಅಹಂಕಾರದಿಂದ ಮುಕ್ತನಾಗಲಾರದೆ ಸಗುಣಬ್ರಹ್ಮನಲ್ಲಿ ಸೇರಿಕೊಂಡಿದ್ದು, ಪ್ರಳಯಾಂತ್ಯದಲ್ಲಿ ಮತ್ತೆ ಸೃಷ್ಟಿಕರ್ತನಾಗಿ ಜನಿಸಬೇಕಾಗುತ್ತದೆ. ಎಂದ ಮೇಲೆ ಸಾಮಾನ್ಯ ಮಾನವರ ಪಾಡೇನು? ಇಂದ್ರಿಯಸುಖದ ಅಭಿಲಾಷೆಯಿಂದ ಕರ್ಮ ಗಳನ್ನಾಚರಿಸಿದವರು ಧೂಮಗತಿಯಿಂದ ಪಿತೃಲೋಕಕ್ಕೆ ಹೋಗಿ, ಪುಣ್ಯಫಲ ತೀರುತ್ತಲೆ, ಭೂಲೋಕಕ್ಕೆ ಮರಳಿ, ತಮ್ಮ ಮಕ್ಕಳಿಗೆ ಮಕ್ಕಳಾಗಿ ಹುಟ್ಟುವರು.”

“ಅಮ್ಮ, ಕಾಮ್ಯಕರ್ಮವನ್ನು ಮಾಡುವವರು ಹೀಗೆ ಶಾಶ್ವತಸುಖದಿಂದ ವಂಚಿತರಾಗು ವರಾದ್ದ ರಿಂದ ನೀನು ನಿತ್ಯಪದವಿಯನ್ನು ನೀಡುವ ಭಗವಂತನನ್ನೇ ಆಶ್ರಯಿಸು. ಆತನಲ್ಲಿ ಎರಡಿಲ್ಲದ ಭಕ್ತಿಯನ್ನಿಟ್ಟರೆ ವೈರಾಗ್ಯ ತಾನಾಗಿ ಉದಿಸುತ್ತದೆ, ಬ್ರಹ್ಮಜ್ಞಾನ ಪ್ರಾಪ್ತಿಯಾಗು ತ್ತದೆ. ಆಗ ಸುಖ ದುಃಖ, ಪ್ರೀತಿ ದ್ವೇಷ–ಎಂಬ ದ್ವಂದ್ವಗಳು ಅಳಿದು ಹೋಗುತ್ತವೆ; ಎಲ್ಲವೂ ಪರಬ್ರಹ್ಮಸ್ವರೂಪವೆಂದು ಗೋಚರವಾಗುತ್ತದೆ. ತಾಯಿ, ಸೂರ್ಯನಿಗೆ ಮೋಡ ಗಳು ಮುಸುಕಿಕೊಳ್ಳುವಂತೆ, ಅಜ್ಞಾನ ಜ್ಞಾನವನ್ನು ಮುಚ್ಚಿಬಿಟ್ಟಿದೆ. ಆ ಜ್ಞಾನ ಉದಿಸಬೇಕಾ ದರೆ ಅಜ್ಞಾನಸ್ವರೂಪವಾದ ಸಂಸಾರ ವ್ಯಾಪಾರವನ್ನು ತೊಲಗಿಸಬೇಕು. ಯೋಚಿಸಿ ನೋಡು, ಈ ಜಗತ್ತೆಲ್ಲವೂ ಪರಬ್ರಹ್ಮವಸ್ತುವೇ. ಬಹಿರ್ಮುಖವಾಗಿರುವ ನಮ್ಮ ಇಂದ್ರಿಯಗಳಿಗೆ ಮಾತ್ರ ಅದು ಪ್ರತ್ಯೇಕ ಪದಾರ್ಥವೆಂಬಂತೆ ತೋರುತ್ತದೆ. ನಾನು ಹಿಂದೆಯೇ ತಿಳಿಸಿ ಹೇಳಿದಂತೆ, ಮಹತ್ ತತ್ವ ಅಹಂಕಾರ ತತ್ವವಾಗಿ ಅದರಲ್ಲಿ ಸತ್ವ ರಜ ತಮಗಳೆಂಬ ಗುಣಗಳು ಏರ್ಪಟ್ಟು, ಪೃಥ್ವಿಯೇ ಮೊದಲಾದ ಪಂಚಭೂತಗಳೂ ಇಂದ್ರಿಯರೂಪದ ಹನ್ನೊಂದು ತತ್ವಗಳೂ, ಈ ತತ್ವಸಮುದಾಯರೂಪವಾದ ಅಂಡ ಮತ್ತು ಬ್ರಹ್ಮ–ಇವೆಲ್ಲವೂ ಭಗವಂತನಿಂದ ಉದಿಸಿ ಬಂದುವಲ್ಲವೆ? ಆದ್ದರಿಂದ ಇವೆಲ್ಲವೂ ಬ್ರಹ್ಮಾತ್ಮಕ. ವಿರಕ್ತನಾಗಿ ದೃಢಚಿತ್ತನಾದ ಯೋಗಿಗೆ ಇದು ಗೋಚರವಾಗು ತ್ತದೆ.”

“ಅಮ್ಮ, ಜಡ, ಜೀವ, ಬ್ರಹ್ಮರ ಸ್ವರೂಪವನ್ನು ಅರ್ಥಮಾಡಿಕೊಳ್ಳಬಲ್ಲ ಆತ್ಮಜ್ಞಾನ ವನ್ನು ನಿನಗೆ ವಿವರಿಸಿದ್ದೇನೆ. ಈ ಜ್ಞಾನಯೋಗದ ಜೊತೆಗೆ ಭಕ್ತಿಯೋಗವನ್ನೂ ನಿನಗೆ ಬೋಧಿಸಿದ್ದೇನೆ. ಈ ಎರಡು ಯೋಗಗಳ ಗುರಿಯೂ ಭಗವಂತನೇ. ‘ಜ್ಞಾನಯೋಗದಿಂದ ಆತ್ಮಲಾಭ, ಭಕ್ತಿಯೋಗದಿಂದ ಉಪಾಸ್ಯದೈವದ ಪ್ರತ್ಯಕ್ಷ; ಎರಡೂ ಒಂದೇ ಹೇಗಾ ದೀತು?’ ಎಂದು ಶಂಕಿಸಬೇಡ. ಹಾಲು ಕಣ್ಣಿಗೆ ಬೆಳ್ಳಗೆ ಕಾಣಿಸುತ್ತದೆ, ನಾಲಗೆಗೆ ಮಧುರ ವಾಗಿದೆ, ಮೂಗಿಗೆ ಸುವಾಸನಾಯುಕ್ತವಾಗಿದೆ, ಮುಟ್ಟಿದರೆ ಬಿಸಿಯಾಗಿಯೋ ತಣ್ಣಗೋ ಇದೆ. ಬೇರೆ ಬೇರೆ ಇಂದ್ರಿಯಗಳು ಬೇರೆ ಬೇರೆಯ ಫಲಗಳನ್ನು ಪಡೆದರೂ ಹಾಲು ಹಾಲೆ. ಹಾಗೆಯೆ, ಬೇರೆಬೇರೆ ಶಾಸ್ತ್ರಗಳಲ್ಲಿ ಆಚರಣೆಯಿಂದ ಬೇರೆಬೇರೆಯಾಗಿ ಅನುಭವವಾ ದರೂ ದೇವರು ದೇವರೇ.”

ಕಪಿಲಮುನಿಯ ಉಪದೇಶದಿಂದ ದೇವಹೂತಿಯ ಅಜ್ಞಾನ ಹಾರಿಹೋಯಿತು. ಇದಿರಿಗೆ ಕುಳಿತು ಜ್ಞಾನೋಪದೇಶಮಾಡಿದವನು ತನ್ನ ಮಗನಲ್ಲ, ಸಾಕ್ಷತ್ ಪರಮಾತ್ಮ– ಎಂಬುದು ಆಕೆಗೆ ಗೋಚರವಾಯಿತು. ಆಕೆ ಆತನನ್ನು ಕುರಿತು ‘ಹೇ ಭಗವಾನ್, ಚರಾ ಚರಾತ್ಮಕವಾದ ಬ್ರಹ್ಮಾಂಡವನ್ನೆಲ್ಲ ನಿನ್ನ ಹೊಟ್ಟೆಯಲ್ಲಿ ಇಟ್ಟುಕೊಂಡು, ನಿನ್ನ ಸಂಕಲ್ಪ ಮಾತ್ರದಿಂದಲೆ ಸೃಷ್ಟಿ ಸ್ಥಿತಿ ಲಯಗಳನ್ನು ಮಾಡುತ್ತಿರುವ ನೀನು ನನ್ನ ಹೊಟ್ಟೆಯಲ್ಲಿ ಹುಟ್ಟಿಬಂದೆಯಲ್ಲವೆ? ನಾನೆಂತಹ ಧನ್ಯಳು! ಲೋಕಕ್ಕೆ ತತ್ವಜ್ಞಾನವನ್ನು ಬೋಧಿಸಲೆಂದು ಉದಿಸಿಬಂದ ನಿನ್ನ ನಾಮಸ್ಮರಣೆಯೆ ಮಂಗಳಕರ. ನಿನ್ನನ್ನು ಕಣ್ಣಾರೆ ನೋಡುತ್ತಿರುವ ನಾನು ಪುಣ್ಯಶಾಲಿನಿ. ನಾನು ನಿನ್ನ ಪಾದಗಳಿಗೆ ಶರಣಾಗಿದ್ದೇನೆ. ನನ್ನನ್ನು ಉದ್ಧರಿಸು’ ಎಂದಳು. ಆಗ ಕಪಿಲಮುನಿಯು ತುಟಿಗಳಿಂದ ಮಧುರವಾದ ಮುಗಳ್​ನಗೆಯನ್ನು ತುಳುಕಿಸುತ್ತಾ “ಅಮ್ಮ, ನಾನು ಬೋಧಿಸಿದ ಭಕ್ತಿಮಾರ್ಗವನ್ನು ಅನುಸರಿಸು. ಅದರಿಂದ ಬಹಳ ಸುಖವಾಗಿಯೇ ಮೋಕ್ಷವನ್ನು ಸಾಧಿಸಬಹುದು. ಅನೇಕ ಮಹಾತ್ಮರು ಈ ಮಾರ್ಗ ವನ್ನೆ ಅನುಸರಿಸಿ ಉದ್ಧಾರವಾಗಿದ್ದಾರೆ” ಎಂದು ಹೇಳಿದನು.

ತಾಯಿಗೆ ಉಪದೇಶವನ್ನು ನೀಡಿದ ಮೇಲೆ ಕಪಿಲಮಹಾಮುನಿಯು ಆಕೆಯಿಂದ ಬೀಳ್ಕೊಂಡು ಕಣ್ಮರೆಯಾಗಿ ಹೋದನು. ಅನಂತರ ದೇವಹೂತಿಯು ಕರ್ದಮಪ್ರಜಾ ಪತಿಯ ಆಶ್ರಮದಲ್ಲಿಯೇ ತಪೋನಿರತಳಾದಳು. ತಮ್ಮ ಭೋಗಕ್ಕಾಗಿ ಗಂಡನು ನಿರ್ಮಿ ಸಿದ್ದ ಭೋಗ ಸಾಮಗ್ರಿಗಳೆಲ್ಲವೂ ಅಚ್ಚಳಿಯದೆ ಉಳಿದಿದ್ದವಾದರೂ ಈಗ ಆಕೆಗೆ ಅವೆಲ್ಲವೂ ಹುಲ್ಲುಕಡ್ಡಿಗಿಂತಲೂ ಕಡೆಯಾಗಿ ಕಾಣಿಸಿದವು. ಕೇವಲ ಮಗನ ಅಗಲಿಕೆ ಯೊಂದು ಮಾತ್ರ ಕೆಲಕಾಲ ಆಕೆಗೆ ಒಂದು ದೊಡ್ಡ ಕೊರಗಾಗಿತ್ತು. ಕ್ರಮಕ್ರಮವಾಗಿ ಆಕೆಯ ಅಂತರಂಗ ತಿಳಿಯಾಯಿತು, ಶುದ್ಧವಾಯಿತು, ಧ್ಯಾನಮಗ್ನವಾಯಿತು. ಕನಸಿನಲ್ಲಿ ದ್ದವರು ಎಚ್ಚರಗೊಳ್ಳುವಂತೆ ಆಕೆಯ ಚೇತನ ಮಾಯೆಯ ಜಾಲದಿಂದ ಮುಕ್ತವಾಗಿ ಬ್ರಹ್ಮಜ್ಞಾನವನ್ನು ಪಡೆಯಿತು. ಯೋಗಮಾರ್ಗವನ್ನು ಹಿಡಿದು ಆಕೆ ಸಚ್ಚಿದಾನಂದನಲ್ಲಿ ಐಕ್ಯಳಾದಳು. ಆಕೆ ಯೋಗಸಿದ್ಧಿಯನ್ನು ಪಡೆದ ಸ್ಥಳ ಸಿದ್ಧಾಶ್ರಮವೆಂಬ ಹೆಸರಿನಿಂದ ಮೂರು ಲೋಕಗಳಲ್ಲಿಯೂ ಪ್ರಸಿದ್ಧವಾಯಿತು. ಜೀವವನ್ನು ತ್ಯಜಿಸಿದ ಆಕೆಯ ದೇಹವು ನದಿಯ ರೂಪದಲ್ಲಿ ಹರಿದು, ಸಿದ್ಧಗಣಗಳಿಂದ ಸೇವಿತವಾಯಿತು.

