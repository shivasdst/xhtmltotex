
\chapter{೧೪. ಮಹಾಭಕ್ತ ಧ್ರುವಕುಮಾರ}

ಇದುವರೆಗೆ ನಾವು ಸ್ವಾಯಂಭುವಮನುವಿನ ಹೆಣ್ಣುಮಕ್ಕಳ ಸಂತತಿಯ ಕಥೆಯನ್ನು ನೋಡಿದೆವು. ಈಗ ಆತನ ಗಂಡುಮಕ್ಕಳ ಕಡೆ ಸ್ವಲ್ಪ ನೋಡೋಣ. ಸ್ವಾಯಂಭುವಿಗೆ ಪ್ರಿಯವ್ರತ, ಉತ್ತಾನಪಾದ ಎಂಬ ಇಬ್ಬರು ಗಂಡುಮಕ್ಕಳಿದ್ದರಷ್ಟೆ. ಅವರಿಬ್ಬರೂ ಭೂಮಂಡಲವನ್ನು ಆಳುತ್ತಿದ್ದರು. ಅವರಲ್ಲಿ ಕಿರಿಯವನಾದ ಉತ್ತಾನಪಾದನಿಗೆ ಸುನೀತಿ, ಸುರುಚಿ ಎಂಬ ಇಬ್ಬರು ಹೆಂಡಿರಿದ್ದರು. ಇವರಲ್ಲಿ ಕಿರಿಯ ಹೆಂಡತಿಯ ಮೇಲೆ ಹೆಚ್ಚು ಪ್ರೀತಿ ರಾಜನಿಗೆ. ಒಂದು ದಿನ ಆತ ಕಿರಿಯ ಹೆಂಡತಿಯ ಬಳಿಯಲ್ಲಿ ಸಿಂಹಾಸನವೇರಿ, ಆಕೆಯ ಮಗನಾದ ಉತ್ತಮಕುಮಾರನನ್ನು ತೊಡೆಯಮೇಲೆ ಕೂಡಿಸಿಕೊಂಡು ಮುದ್ದಾಡುತ್ತಿದ್ದ. ಆ ವೇಳೆಗೆ ಹಿರಿಯ ಹೆಂಡತಿಯ ಮಗನಾದ ಧ್ರುವಕುಮಾರ ಅಲ್ಲಿಗೆ ಬಂದ. ಅವನಿಗೂ ತಂದೆಯ ತೊಡೆಯಮೇಲೆ ಏರಬೇಕೆಂಬ ಆಶೆಯಾಯಿತು; ಏರಲೆಂದು ಹತ್ತಿರಕ್ಕೆ ಹೋದ. ಆದರೆ ತಂದೆ ಅದಕ್ಕೆ ಅವಕಾಶ ಕೊಡಲಿಲ್ಲ. ಅಷ್ಟರಲ್ಲಿ ಸುರುಚಿಯು ತನ್ನ ಸವತಿಯ ಮೇಲಿನ ಅಸೂಯೆಯಿಂದ ಆ ಮಗುವನ್ನು ಗದರಿಸುತ್ತಾ, ‘ಎಲೆ ಹುಡುಗ, ನಿನಗೆ ಸಿಂಹಾಸನವನ್ನು ಏರುವ ಯೋಗ್ಯತೆಯಿಲ್ಲ. ನೀನು ರಾಜನ ಮಗನಾದರೂ ನನ್ನ ಹೊಟ್ಟೆಯಲ್ಲಿ ಹುಟ್ಟಲಿಲ್ಲ. ಬೇರೆಯವರ ಹೊಟ್ಟೆಯಲ್ಲಿ ಹುಟ್ಟಿದರೆ ಸಿಂಹಾಸನವನ್ನು ಏರುವಂತಿಲ್ಲ ಎಂಬುದು ಹಸುಳೆಯಾದ ನಿನಗೇನು ಗೊತ್ತು? ಆದ್ದರಿಂದಲೆ ಅಸಾಧ್ಯ ವಾದುದನ್ನು ಬಯಸುತ್ತಿರುವೆ. ನಮ್ಮ ಉತ್ತಮಕುಮಾರನ ಹಾಗೆ ನೀನೂ ಸಿಂಹಾಸನ ವನ್ನು ಹತ್ತಬೇಕಾದರೆ ಭಗವಂತನನ್ನು ಕುರಿತು ತಪಸ್ಸುಮಾಡಿ ನನ್ನ ಹೊಟ್ಟೆಯಲ್ಲಿ ಹುಟ್ಟಿ ಬಾ’ ಎಂದಳು.

ತನ್ನ ಮಲತಾಯಿಯ ಮಾತುಗಳನ್ನು ಕೇಳಿ ಧ್ರುವಕುಮಾರನ ಎಳೆಯ ಮನಸ್ಸು ಮಮ್ಮಲ ಮರುಗಿತು. ಆಕೆ ಹಾಗೆ ಮಾತನಾಡುತ್ತಿದ್ದರೂ ತನ್ನನ್ನು ಸಮಾಧಾನ ಮಾಡದೆ ಉದಾಸೀನನಾಗಿರುವ ತನ್ನ ತಂದೆಯನ್ನು ಕಂಡು ಆತನಿಗೆ ದುಃಖ ಒತ್ತರಿಸಿ ಬಂತು. ಏಟು ತಿಂದ ನಾಗರಹಾವಿನಂತೆ ಬುಸುಗುಟ್ಟುತ್ತಾ, ಕಣ್ಣೀರು ಸುರಿಸಿಕೊಂಡು ತಾಯಿಯ ಬಳಿಗೆ ಬಂದ. ಸುಮತಿ ಮಗನನ್ನು ಬಾಚಿ ತಬ್ಬಿಕೊಂಡು ತನ್ನ ತೊಡೆಯೇರಿಸಿಕೊಂಡಳು. ತನ್ನ ಕಂದನನ್ನು ಆಕೆ ಸಂತಯಿಸುತ್ತಿರಲು, ಯಾರೊ ಬಂದು ಆಕೆಗೆ ಅಂದು ಸುರುಚಿಯಾಡಿದ ಕಿಡುನುಡಿಗಳನ್ನು ಆಕೆಗೆ ತಿಳಿಸಿದರು. ಅದನ್ನು ಕೇಳಿ ಆಕೆಗೆ ಬಹು ಸಂಕಟವಾಯಿತು. ಕಾಡುಕಿಚ್ಚಿಗೆ ಸಿಕ್ಕಿದ ಎಳೆಯ ಬಳ್ಳಿಯಂತೆ ಆಕೆಯು ಕಂದಿ ಹೋಗಿ, ಕಣ್ಣಿನಲ್ಲಿ ಕಂಬನಿ ಯನ್ನು ಸುರಿಸುತ್ತಾ ನಿಟ್ಟುಸಿರಿಟ್ಟಳು. ಕ್ಷಣಕಾಲ ಸುಮ್ಮನೆ ಕುಳಿತು, ಆಮೇಲೆ ‘ಮಗು, ನಿನ್ನ ಚಿಕ್ಕಮ್ಮನಾದ ಸುರುಚಿ ಆಡಿದ ಮಾತು ನಿಜ. ನಿನ್ನ ತಂದೆಗೆ ನನ್ನನ್ನು ಹೆಂಡತಿ ಎಂದು ಹೇಳಿಕೊಳ್ಳುವುದಕ್ಕೆ ಕೂಡ ನಾಚಿಕೆ. ಅಂತಹ ನನ್ನ ಹೊಟ್ಟೆಯಲ್ಲಿ ಹುಟ್ಟಿದ ನೀನು ಸಿಂಹಾಸನವನ್ನು ಏರುವುದು ಹೇಗೆ ಸಾಧ್ಯ? ನಮ್ಮ ದುರದೃಷ್ಟಕ್ಕೆ ಬೇರೆಯವರನ್ನು ದೂರಲೇಕೆ? ಅವರವರ ಪಾಪ ಪುಣ್ಯ ಅವರದು. ಆದ್ದರಿಂದ ನಿನ್ನ ಬಲತಾಯಿಯ ಮಾತಿ ಗಾಗಿ ಆಕೆಯನ್ನು ನಿಂದಿಸುವ ಅಗತ್ಯವಿಲ್ಲ. ಆಕೆಯ ಮಾತನ್ನು ಬುದ್ಧಿವಾದವೆಂದು ತಿಳಿದು, ಆಕೆಯ ಹೇಳಿದಂತೆಯೆ ಮಾಡು. ಉತ್ತಮಕುಮಾರನಂತೆ ನೀನು ಸಿಂಹಾಸನವನ್ನು ಏರಬೇಕೆಂದರೆ ಭಗವಂತನನ್ನು ಭಕ್ತಿಯಿಂದ ಆರಾಧಿಸು. ಬ್ರಹ್ಮನು ಭಗವಂತನ ಆರಾ ಧನೆಯಿಂದಲೆ ಸತ್ಯಲೋಕದ ಸ್ವಾಮಿಯಾದುದು. ನಿನ್ನ ತಾತನಾದ ಮನು ಚಕ್ರವರ್ತಿಯೂ ದೇವದೇವನನ್ನು ಆರಾಧಿಸಿ ಈ ಭೂಮಿಗೆಲ್ಲ ಸ್ವಾಮಿಯಾದ, ಉತ್ತಮಗತಿಯನ್ನು ಪಡೆದ, ಮೋಕ್ಷಹೊಂದಿದ. ನೀನು ಅವನಂತೆಯೇ ಆಚರಿಸು. ನಿನ್ನ ಆಶೆ ಈಡೇರುವುದಕ್ಕೆ ಬೇರಾವ ಉಪಾಯವೂ ಇಲ್ಲ. ಒಂದೇ ಮನಸ್ಸಿನಿಂದ ಶ್ರೀಹರಿಯನ್ನು ಧ್ಯಾನ ಮಾಡು’ ಎಂದಳು.

ತಾಯಿಯ ಬೋಧನೆ ಧ್ರುವಕುಮಾರನ ಮನಸ್ಸಿಗೆ ಒಪ್ಪಿಗೆಯಾಯಿತು. ಆತನು ತಕ್ಷಣವೇ ಮನೆಯನ್ನು ಬಿಟ್ಟು ಅಡವಿಯ ಹಾದಿಯನ್ನು ಹಿಡಿದನು. ದಾರಿಯಲ್ಲಿ ನಾರದ ಮಹರ್ಷಿಗಳು ಎದುರಾದರು. ಜ್ಞಾನಿಗಳಾದ ಅವರಿಗೆ ಬಾಲಕನ ಕಥೆಯೆಲ್ಲ ಗೊತ್ತು. ಐದು ವರ್ಷದ ಬಾಲಕನಾದರೂ ಆ ಧ್ರುವಕುಮಾರ ತನಗಾದ ಅಪಮಾನವನ್ನು ಸಹಿಸಲಾರದೆ ಸಾಹಸಕಾರ್ಯಕ್ಕೆ ಕೈಯಿಟ್ಟಿರುವುದನ್ನು ಕಂಡು ಅವರಿಗೆ ಆಶ್ಚರ್ಯ. ಅದಕ್ಕೂ ಹೆಚ್ಚಾಗಿ ಅವನ ಕ್ಷತ್ರಿಯ ತೇಜಸ್ಸನ್ನು ಕಂಡು ಮೆಚ್ಚಿಗೆ! ಅವರು ತಮ್ಮ ಅಮೃತಹಸ್ತದಿಂದ ಆ ಬಾಲಕನ ತಲೆಯನ್ನು ಸವರುತ್ತಾ ‘ಮಗು, ಆಟವಾಡಿಕೊಂಡಿರಬೇಕಾದ ಈ ಎಳೆಯ ವಯಸ್ಸಿನಲ್ಲಿ ನಿನಗೇಕೆ ತಪಸ್ಸು? ಯಾರೊ ನಿಂದಿಸಿದರೆಂದು, ನಿನಗೆ ಅವಮಾನವಾಯಿ ತೆಂದು ಅರಮನೆಯನ್ನು ಬಿಟ್ಟು ಅಡವಿಗೆ ಹೊರಡುತ್ತಾರೆಯೆ? ಈ ವಯಸ್ಸಿನಲ್ಲಿ ಮಾನ ಅವಮಾನಗಳನ್ನು ಮನಸ್ಸಿಗೆ ಹಚ್ಚಿಕೊಳ್ಳದೆ ಆಟವಾಡಿಕೊಂಡಿರಬೇಕು. ಹಾಗೆ ಮಾನಾವ ಮಾನಗಳು ನಿನಗೆ ಅರ್ಥವಾಗುವ ಹಾಗಿದ್ದರೆ ನೀನು ವಿವೇಕಿಯೆಂದಾಯಿತು. ಹಾಗೆ ನೀನು ವಿವೇಕಿಯಾಗಿದ್ದರೆ ಸುಖದುಃಖಗಳು ನಮ್ಮ ಕರ್ಮವೆಂದು ತಿಳಿದು ಶಾಂತಿಯಿಂದ ಇರ ಬೇಕಾಗಿತ್ತು. ದೇಹವೇ ಆತ್ಮವೆಂಬ ಮೋಹಕ್ಕೆ ಒಳಗಾದ ನೀನು ನಿನ್ನ ಬಲತಾಯಿ ಬೈದಳೆಂಬ ಕೋಪದಿಂದ ತಪಸ್ಸಿಗೆ ಹೊರಟುಬಿಟ್ಟೆಯಲ್ಲಾ! ಅದೇನು ಹುಡುಗಾಟವೆಂದು ತಿಳಿದೆಯಾ? ಮಹಾಯೋಗಿಗಳಿಗೂ ಭಗವಂತನ ಹಾದಿ ಗೊತ್ತಾಗುವುದಿಲ್ಲ, ಇನ್ನು ನಿನಗೆ ಸಾಧ್ಯವಾಗುತ್ತದೆಯೆ? ನೀನು ಈಗ ಸುಮ್ಮನೆ ಮನೆಗೆ ಹೋಗು. ನಿನ್ನ ಪೂರ್ವಕರ್ಮ ವೆಲ್ಲವೂ ಸಮೆದುಹೋಗುವಂತೆ ಸುಖದುಃಖಗಳನ್ನೆಲ್ಲ ಅನುಭವಿಸಿ, ಮನಸ್ಸಿನ ಸಮತೆ ಯನ್ನು ಪಡೆದು, ಮುದುಕನಾದಾಗ ತಪಸ್ಸನ್ನು ಮಾಡುವೆಯಂತೆ’ ಎಂದರು.

ನಾರದರ ನುಡಿಗಳಿಂದ ಬಾಲಕನ ಮನಸ್ಸು ಮತ್ತಷ್ಟು ದೃಢವಾಯಿತು. ಅವನು ಹೇಳಿದ– ‘ಸ್ವಾಮಿ, ನಿಮ್ಮ ಉಪದೇಶವನ್ನು ಕೇಳಿ ನನಗೆ ತುಂಬ ಸಂತೋಷವಾಯಿತು. ಆದರೇನು? ನಾನು ಕ್ಷತ್ರಿಯ. ನನ್ನ ಬಲತಾಯಿಯ ಮಾತುಗಳಿಂದ ನನ್ನ ಮನಸ್ಸು ಅಲ್ಲೋಲಕಲ್ಲೋಲವಾಗಿದೆ. ನಿಮ್ಮ ಉಪದೇಶ ಅದಕ್ಕೆ ಹಿಡಿಸುವುದಿಲ್ಲ. ನಾನೀಗ ನನ್ನ ತಂದೆಯ ಸಿಂಹಾಸನವನ್ನಲ್ಲ ಬಯಸುತ್ತಿರುವುದು; ನಮ್ಮ ವಂಶದ ಹಿರಿಯರಾರಿಗೂ ಪಡೆಯಲಾಗದ ಮತ್ತು ಮೂರು ಲೋಕಗಳಲ್ಲಿಯೂ ಅತ್ಯಂತ ಶ್ರೇಷ್ಠವಾದ ಪದವಿ ಯನ್ನು ಪಡೆಯಬೇಕೆಂಬುದು ನನ್ನ ಬಯಕೆ. ಅದು ಈಡೇರುವ ಹಾದಿಯನ್ನು ತಾವು ತೋರಿಸಬೇಕು. ತಾವು ಸಾಕ್ಷಾತ್ ಬ್ರಹ್ಮನ ಮಕ್ಕಳು, ನಿಮ್ಮ ವೀಣೆಯಿಂದ ಜಗತ್ತಿನಲ್ಲೆಲ್ಲ ದೇವರನಾಮವನ್ನು ಬಿತ್ತುತ್ತಿರುವಿರಿ. ನನ್ನನ್ನು ತಾವೇ ಉದ್ಧಾರ ಮಾಡಬೇಕು’ ಎಂದನು.

ಧ್ರುವಕುಮಾರನ ಮಾತುಗಳನ್ನು ಕೇಳಿ ನಾರದರಿಗೆ ಬಹು ಸಂತೋಷವಾಯಿತು. ಅವರು ಆತನ ತಲೆಯನ್ನು ಮತ್ತೊಮ್ಮೆ ಕರುಣೆಯಿಂದ ಸವರುತ್ತಾ ‘ಮಗು, ನಿನ್ನ ತಾಯಿ ನಿನಗೆ ಹೇಳಿದ ಬುದ್ಧಿವಾದವೆ ಒಂದು ದೊಡ್ಡ ಉಪದೇಶ. ಅದೇ ಭಕ್ತಿಯೋಗ. ಆಕೆ ಹೇಳಿ ದಂತೆ ಒಂದೇ ಮನಸ್ಸಿನಿಂದ ದೇವರನ್ನು ಧ್ಯಾನಿಸು. ಹೋಗು, ಯಮುನಾನದಿಯ ತೀರ ದಲ್ಲಿ ಮಧುವನವವಿದೆ. ಅದು ಭಗವಂತನಿಗೆ ಬಹು ಪ್ರಿಯವಾದ ಸ್ಥಳ. ಮೂರು ಹೊತ್ತೂ ಯಮುನೆಯಲ್ಲಿ ಸ್ನಾನ ಮಾಡಿ, ಮಧುವನದಲ್ಲಿ ತಪಸ್ಸು ಮಾಡು. ದರ್ಭೆಯ ಆಸನದ ಮೇಲೆ ಕುಳಿತು, ಪ್ರಾಣಾಯಾಮದಿಂದ ದೇಹ ಮನಸ್ಸುಗಳನ್ನು ಶುದ್ಧಮಾಡಿ ಕೊಂಡ ಮೇಲೆ ಭಗವಂತನ ಮೂರ್ತಿಯನ್ನು ಧ್ಯಾನಮಾಡು. ದೇವದೇವನಾದ ಆ ಭಗವಂತ ಮುದ್ದಾದ ಎಳೆಯ ಮಗುವಿನಂತಿದ್ದಾನೆ. ತೊಂಡೆಯ ಹಣ್ಣಿನಂತೆ ಕೆಂಪಾದ ಆತನ ತುಟಿಗಳಲ್ಲಿ ಸದಾ ಮುಗುಳುನಗೆ ಚೆಲ್ಲುತ್ತಿರುತ್ತದೆ. ಹೊಳೆವ ಕಣ್ಣು, ನುಣುಪಾದ ಕೆನ್ನೆ, ಎಸಳಾದ ಮೂಗು, ಅಂದವಾದ ಹುಬ್ಬುಗಳು, ಅಗಲವಾದ ಎದೆಯಲ್ಲಿ ಶ್ರೀವತ್ಸ ವೆಂಬ ಮಚ್ಚೆ, ಕೊರಳಲ್ಲಿ ತುಲಸಿಯ ಮಾಲೆ, ನಾಲ್ಕು ಕೈಗಳು ಮತ್ತು ಆ ನಾಲ್ಕರಲ್ಲಿ ಕ್ರಮವಾಗಿ ಶಂಖ, ಚಕ್ರ, ಗದೆ, ಪದ್ಮಗಳು; ತಲೆಯಲ್ಲಿ ಕಿರೀಟ, ಕಿವಿಯಲ್ಲಿ ಕುಂಡಲ ಗಳು, ಕಂಠದಲ್ಲಿ ಕೌಸ್ತುಭರತ್ನ, ಮೇಘದಂತೆ ಕಪ್ಪಾದ ದೇಹ ಆತನದು; ಆತನು ಪೀತಾಂಬರವನ್ನು ಉಟ್ಟು ಅದರ ಮೇಲೆ ರತ್ನದ ನಡುಕಟ್ಟನ್ನು ಸುತ್ತಿದ್ದಾನೆ. ಆತನ ಈ ಸುಂದರಮೂರ್ತಿಯನ್ನು ಮನಸ್ಸಿನಲ್ಲಿ ಸ್ಥಿರವಾಗಿ ನಿಲ್ಲಿಸಿ ಕೊಂಡು ಧ್ಯಾನಮಾಡು’ ಎಂದು ಹೇಳಿ ಅವನಿಗೆ ವಾಸುದೇವ ಮಂತ್ರವನ್ನು ಉಪದೇಶಿಸಿದರು; ಅಲ್ಲದೆ ಅದನ್ನು ಜಪಿಸುವ ವಿಧಿವಿಧಾನಗಳನ್ನು ವಿವರಿಸಿದರು.

ನಾರದ ಮುನಿಗಳಿಂದ ಉಪದೇಶವನ್ನು ಪಡೆದ ಧ್ರುವಕುಮಾರನು ಆತನಿಗೆ ನಮ ಸ್ಕರಿಸಿ, ಆತನಿಂದ ಅಪ್ಪಣೆಯನ್ನು ಪಡೆದು ಮಧುವನದತ್ತ ನಡೆದನು. ನಾರದ ಮಹರ್ಷಿ ಗಳು ಅಲ್ಲಿಂದ ನೇರವಾಗಿ ಉತ್ತಾನಪಾದ ರಾಜನ ಬಳಿಗೆ ಬಂದರು. ಇದ್ದಕ್ಕಿದ್ದಂತೆಯೆ ಬಂದಿಳಿದ ನಾರದ ಮಹರ್ಷಿಗಳನ್ನು ರಾಜನು ಅತ್ಯಂತ ಸಂಭ್ರಮದಿಂದ ಇದಿರು ಗೊಂಡು, ಸತ್ಕರಿಸಿ ಸಿಂಹಾಸನದ ಮೇಲೆ ಕುಳ್ಳಿರಿಸಿದನು. ರಾಜನ ಮುಖವು ಕಳೆಗೆಟ್ಟಿರುವು ದನ್ನು ಕಂಡ ಪುಷಿ ಅದಕ್ಕೆ ಕಾರಣವನ್ನು ಕೇಳಲು, ರಾಜನು ‘ಸ್ವಾಮಿ, ನಾನೇನು ಹೇಳಲಿ? ಹೆಣ್ಣಿನ ಮೋಹಕ್ಕೆ ಮರುಳಾದ ನಾನು ಐದು ವರ್ಷದ ನನ್ನ ಮಗನನ್ನು ಸ್ವಲ್ಪವೂ ಮರುಕವಿಲ್ಲದೆ ಹೊಡೆದೋಡಿಸಿದೆ. ಪಾಪ, ನನ್ನ ತೊಡೆಯನ್ನು ಏರಲೆಂದು ಬಂದ ಆ ಮುದ್ದು ಮಗನನ್ನು ಹತ್ತಿರಕ್ಕೆ ಸೇರಿಸದೆ ಅವನ ಮನಸ್ಸನ್ನು ನೋಯಿಸಿದೆ. ನನ್ನಂತಹ ನೀಚನುಂಟೆ? ಆ ಮಗ ಅಡವಿಯ ಹಾದಿಯನ್ನು ಹಿಡಿದು ಹೊರಟು ಹೋದನಂತೆ! ಅವನು ಅಲ್ಲಿ ಎಲ್ಲಿ ಅಲೆಯುತ್ತಿರುವನೊ! ಹೊತ್ತಿಗೆ ಗೊತ್ತಿಗೆ ನಿದ್ರಾಹಾರಗಳಿಲ್ಲದೆ ಆ ಮಗುವಿನ ಗತಿಯೇನಾಗಿದೆಯೊ! ಯಾವ ಕಾಡುಪ್ರಾಣಿಯ ಬಾಯಿಗೆ ಅವನು ತುತ್ತಾಗು ವನೋ!’ ಎಂದು ‘ಗೊಳೋ’ ಎಂದು ಅತ್ತನು. ಆಗ ನಾರದರು ‘ರಾಜನೇ, ನಿನ್ನ ಅಳು ವನ್ನು ನಿಲ್ಲಿಸು. ನಿನ್ನ ಮಗ ಸಾಮಾನ್ಯನಲ್ಲ. ದೇವರೇ ಆತನನ್ನು ಕಾಪಾಡುತ್ತಾನೆ. ಮೂರು ಲೋಕವೂ ಮೆಚ್ಚುವಂತೆ ತಪಸ್ಸು ಮಾಡಿ, ದೇವರನ್ನೂ ಮೆಚ್ಚಿಸಿ ಅವನು ಹಿಂದಿರುಗು ತ್ತಾನೆ’ ಎಂದು ಸಮಾಧಾನ ಮಾಡಿದರು.

ಅತ್ತ ಧ್ರುವಕುಮಾರನು ನಾರದರ ಉಪದೇಶದಂತೆ ಮಧುವನವನ್ನು ಸೇರಿ, ಯಮುನೆ ಯಲ್ಲಿ ಮಿಂದು, ವಾಸುದೇವ ಮಂತ್ರವನ್ನು ಜಪಿಸುತ್ತಾ ತಪಸ್ಸಿಗೆ ಕುಳಿತನು. ಮೊದಲ ತಿಂಗಳು ಮೂರು ದಿನಗಳಿಗೊಮ್ಮೆ ಕೈಗೆ ಸಿಕ್ಕ ಕಾಡುಹಣ್ಣುಗಳಿಂದ ಹಸಿವನ್ನು ನೀಗಿದನು. ಎರಡನೆಯ ತಿಂಗಳು ಆರು ದಿನಗೊಳಿಗೊಮ್ಮೆ ತರಗೆಲೆಗಳನ್ನೂ ಒಣ ಹುಲ್ಲನ್ನೂ ತಿನ್ನುತ್ತಿ ದ್ದನು. ಮೂರನೆಯ ತಿಂಗಳು ಒಂಬತ್ತು ದಿನಗಳಿಗೊಮ್ಮೆ ಜಲಾಹಾರ ಮಾತ್ರವಾಯಿತು. ನಾಲ್ಕನೆಯ ತಿಂಗಳು ಹನ್ನೆರಡು ದಿನಗಳಿಗೊಮ್ಮೆ ಕೇವಲ ಗಾಳಿಯನ್ನು ಮಾತ್ರ ಸೇವಿಸು ತ್ತಿದ್ದನು. ಅಲ್ಲಿಂದ ಮುಂದೆ ಆತನು ನಿರಾಹಾರಿಯಾದನು. ಆತನು ಮೋಟುಮರದಂತೆ ಒಂಟಿಕಾಲಿನಲ್ಲಿ ಅಲುಗಾಡದೆ ನಿಂತು, ಒಂದೇ ಮನಸ್ಸಿನಿಂದ ದೇವರನ್ನು ಧ್ಯಾನಿಸುತ್ತಿ ದ್ದನು. ಆತನಿಗೆ ಭಗವಂತನ ದಿವ್ಯ ಮಂಗಳ ವಿಗ್ರಹದಲ್ಲಿ ಮನಸ್ಸು ಏಕಾಗ್ರತೆಯನ್ನು ಪಡೆಯಿತು; ಹೊರಗಿನ ವ್ಯಾಪಾರದ ಅರಿವಿಲ್ಲದಂತಾಯಿತು; ಭಕ್ತಿಯೋಗ ಕೈಗೂಡಿತು. ಆತನ ತಪೋಜ್ವಾಲೆಯಿಂದ ಮೂರು ಲೋಕಗಳೂ ತಲ್ಲಣಿಸಿದುವು. ಜಗತ್ತಿನ ಪ್ರಾಣಿ ಗಳಿಗೆ ಉಸಿರುಕಟ್ಟಿಹೋದಂತಾಯಿತು. ಇಂದ್ರನೇ ಮೊದಲಾದ ಲೋಕಪಾಲಕರು ಜಗ ತ್ತನ್ನು ರಕ್ಷಿಸುವಂತೆ ಶ್ರೀಮನ್ನಾರಾಯಣನನ್ನು ಬೇಡಿಕೊಂಡರು. ಅವರ ಪ್ರಾರ್ಥನೆ ದೇವದೇವನನ್ನು ಮುಟ್ಟಿತು. ಅದರ ಜೊತೆಯಲ್ಲಿಯೆ ಧ್ರುವನ ಭಕ್ತಿ ಆತನನ್ನು ಆಕ್ರಮಿ ಸಿತು. ಆತನು ಮಧುವನಕ್ಕೆ ಅವತರಿಸಿದನು.

ಧ್ರುವಕುಮಾರನು ತನ್ನ ಧ್ಯಾನದಲ್ಲಿ ಸದಾ ಕಾಣುತ್ತಿದ್ದ ಆನಂದಮಯವಾದ ದಿವ್ಯ ತೇಜಸ್ಸು ತಟ್ಟನೆ ಮಾಯವಾಗಲು ನಿಧಿಯನ್ನು ಕಳೆದುಕೊಂಡವನಂತೆ ಗಾಬರಿಯಿಂದ ಕಣ್ಣು ತೆರೆದನು. ಧ್ಯಾನದಲ್ಲಿ ಕಾಣುತ್ತಿದ್ದ ದಿವ್ಯಮೂರ್ತಿ ಕಣ್ಣೆದುರಿಗೆ ಪ್ರತ್ಯಕ್ಷವಾಗಿ ನಿಂತಿದೆ. ಬಾಲಕನಿಗೆ ಭಯವಾಯಿತು; ಆ ಮೂರ್ತಿಯ ಮುಂದೆ ಉದ್ದಕ್ಕೆ ಅಡ್ಡಬಿದ್ದನು. ಅನಂತರ ಮೇಲಕ್ಕೆದ್ದು, ಆ ದಿವ್ಯಮೂರ್ತಿಯ ಸೌಂದರ್ಯವನ್ನು ತನ್ನ ಕಣ್ಣುಗಳಿಂದ ಕುಡಿಯುವವನಂತೆ, ಮುಖದಿಂದ ಮುತ್ತಿಡುವವನಂತೆ, ತೋಳುಗಳಿಂದ ಆಲಿಂಗಿಸಿ ಕೊಳ್ಳುವವನಂತೆ ಆತನು ಆನಂದಪರವಶನಾದನು. ಆ ದೇವದೇವನನ್ನು ಹೊಗಳಬೇಕೆನಿಸಿ ದರೂ ಆನಂದದಿಂದ ಯಾವುದೂ ತೋಚದೆ ಕೈಮುಗಿದುಕೊಂಡು ಸುಮ್ಮನೆ ನಿಂತು ಕೊಂಡನು. ಅದನ್ನು ಕಂಡ ನಾರಾಯಣನು ಓಂಕಾರರೂಪಿಯಾದ ತನ್ನ ಪಾಂಚಜನ್ಯ ವನ್ನು ಅವನ ಕೆನ್ನೆಯ ಮೇಲೆ ಸವರಿದನು. ತಕ್ಷಣವೇ ಆ ಬಾಲಕನಿಗೆ ವೇದ ವೇದಾಂತಗಳ ಜ್ಞಾನವೆಲ್ಲವೂ ಕರತಲಾಮಲಕವಾಯಿತು. ಆತನು ಭಗವಂತನನ್ನು ಮುಕ್ತಕಂಠದಿಂದ ಸ್ತೋತ್ರ ಮಾಡಿದನು:

“ಕೊನೆ ಮೊದಲಿಲ್ಲದ ಮಹಾಮಹಿಮನಾದ ಹೇ ಭಗವಂತ, ನಿನಗೆ ನಮಸ್ಕಾರ. ನೀನು ಏಕ, ಅನೇಕ; ಜಗದೀಶ್ವರನಾಗಿ ಏಕ, ಪ್ರಕೃತಿ ಮತ್ತು ಜೀವ–ಇವರ ಸಂಬಂಧವಾಗಿ ಅನೇಕ. ನಿನ್ನ ಮಾಯೆಗೆ ಸಿಕ್ಕಿ ಅನೇಕ ಇಹ ಸುಖಗಳನ್ನು ಬೇಡುವವರಿಗೆ ನೀನು ಕಲ್ಪವೃಕ್ಷ. ಮಾಯೆಯನ್ನು ನೀಗಿ ಮೋಕ್ಷವನ್ನು ಕೊಡುವವನೂ ನೀನೆ. ಈ ಮೋಕ್ಷಕ್ಕಿಂತಲೂ ಮಿಗಿಲಾದುದು ನಿನ್ನ ಸಂಕೀರ್ತನೆ. ಅದರಲ್ಲಿ ಸದಾ ನಿರತರಾಗಿರುವ ನಿನ್ನ ಭಕ್ತರ ಸಂಗ ವನ್ನು ಕೊಟ್ಟು ನನ್ನನ್ನು ಕಾಪಾಡು. ಸ್ವಾಮಿ, ನಿನ್ನ ಕೃಪೆ ನನ್ನ ಮೇಲೆ ಬೀಳುತ್ತಲೆ ಚಿತ್ರವಿಚಿತ್ರವಾಗಿ ವ್ಯಾಪಿಸಿರುವ ಈ ವಿಶ್ವವೆಲ್ಲವೂ ನೀನೆ ಎಂಬುದು ನನಗೆ ಬೆಳ್ಳಂಬೆಳಗಾ ಯಿತು. ಸಮಸ್ತ ವಸ್ತುಗಳಲ್ಲಿಯೂ ನೀನು ನೆಲಸಿದ್ದರೂ ಅವುಗಳ ದೋಷಕ್ಕೆ ಸಿಕ್ಕದೆ ಮುಕ್ತ ನಾಗಿರುವೆ. ಈ ಜಗತ್ತನ್ನು ಸೃಷ್ಟಿಸುತ್ತೀಯೆ, ಸೃಷ್ಟಿಸಿದ ಜಗತ್ತಿನ ಒಳಹೊಕ್ಕು ಅದನ್ನು ನಿಯಮಿಸುತ್ತೀಯೆ, ಅದನ್ನು ರಕ್ಷಿಸುತ್ತೀಯೆ, ಮತ್ತೆ ಅದನ್ನು ನಿನ್ನ ಗರ್ಭದಲ್ಲಿ ಅಡಗಿಸಿ ಕೊಳ್ಳುತ್ತೀಯೆ. ಈ ನಿನ್ನ ಲೀಲೆ ಯಾರಿಗೆ ಗೋಚರವಾಗಬೇಕು? ಆದ್ದರಿಂದ ನಾನು ನಿನ್ನ ಪಾದಗಳಿಗೆ ಶರಣಾಗುತ್ತೇನೆ. ಆಗತಾನೆ ಹೆತ್ತ ಕರುವನ್ನು ಆಕಳು ಸಂರಕ್ಷಿಸುವಂತೆ ನೀನು ನನ್ನನ್ನು ರಕ್ಷಿಸು.”

ಭಕ್ತನ ಸ್ತೋತ್ರದಿಂದ ಸುಪ್ರೀತನಾದ ಶ್ರೀಹರಿಯು ಅತ್ಯಂತ ಕರುಣೆಯಿಂದ ‘ಮಗು, ನಿನ್ನ ಮನಸ್ಸಿನ ಕೋರಿಕೆ ನನಗೆ ಗೊತ್ತಿದೆ. ಬಾಲಕನಾದರೂ ಮಹಾ ಸಾಹಸವನ್ನು ಎಸಗಿರುವ ನಿನಗೆ ಮತ್ತಾರಿಗೂ ದೊರೆಯದಂತಹ ಮಹಾ ಪದವಿಯನ್ನೆ ಕೊಡುತ್ತೇನೆ. ಸೂರ್ಯ, ಗ್ರಹ, ನಕ್ಷತ್ರಗಳಿಗಿಂತಲೂ ಮೇಲಿನ ಸ್ಥಾನವನ್ನು ನಿನಗೆ ಕೊಡುತ್ತೇನೆ. ಮೂರು ಲೋಕಗಳು ಪ್ರಳಯದಲ್ಲಿ ನಾಶವಾದಾಗಲೂ ನಿನ್ನ ಸ್ಥಾನವು ಸ್ಥಿರವಾಗಿರುತ್ತದೆ. ಆದ್ದ ರಿಂದಲೆ ಅದು ಧ್ರುವಪದವೆನಿಸುತ್ತದೆ. ಆದರೆ ನೀನು ನಿನ್ನ ತಂದೆಯ ರಾಜ್ಯವನ್ನು ಧರ್ಮ ದಿಂದ ಮೂವತ್ತಾರು ಸಹಸ್ರ ವರ್ಷಗಳವರೆಗೆ ಆಳುತ್ತಿದ್ದು, ಇಲ್ಲಿನ ಸುಖಭೋಗ ಗಳನ್ನೆಲ್ಲ ಅನುಭವಿಸಿದಮೇಲೆ ಆ ಪದವಿ. ಪರಮಭಾಗವತನಾದ ನಿನಗೆ ದ್ರೋಹವನ್ನು ಬಗೆದುದರಿಂದ ಸುರುಚಿಯ ಮಗ ಉತ್ತಮನು ಬೇಟೆಗೆಂದು ಅಡವಿಗೆ ಹೋಗಿ, ಅಲ್ಲಿಯೆ ಸತ್ತುಹೋಗುವನು; ಅವನ ತಾಯಿ ಅವನನ್ನು ಅರಸುತ್ತಾ ಹೋಗಿ, ಕಾಡುಕಿಚ್ಚಿಗೆ ಸಿಕ್ಕಿ ಸತ್ತುಹೋಗುವಳು’ ಎಂದು ಹೇಳಿ ಮಾಯವಾದನು. ಧ್ರುವಕುಮಾರನು ‘ಭಗವಂತನಲ್ಲಿ ಮುಕ್ತಿಯನ್ನು ಬೇಡದೆ ನಶ್ವರವಾದ ಪದವಿಗಾಗಿ ಪ್ರಾರ್ಥಿಸಿದೆನಲ್ಲಾ! ಚಕ್ರವರ್ತಿಯ ಬಳಿಗೆ ಹೋಗಿ ಬೊಗಸೆ ಭತ್ತವನ್ನು ಬೇಡಿದಂತಾಯಿತಲ್ಲ!’ ಎಂದು ಮನಸ್ಸಿನಲ್ಲಿಯೆ ಮಿಡುಕಿ ಕೊಂಡನು. ಭಗವದನುಗ್ರಹದಿಂದ ಜ್ಞಾನಿಯಾಗಿದ್ದ ಆತನಿಗೆ ಈಗ ಹಾಗೆ ಪಶ್ಚಾತ್ತಾಪ ವಾದರೂ, ಸುರುಚಿಯ ಕಟುನುಡಿಗಳಿಂದ ನೊಂದು ತಪಸ್ಸಿಗೆ ಕುಳಿತಾಗ ಆ ಭಾವನೆ ಯೆಲ್ಲಿತ್ತು?

ಧ್ರುವಕುಮಾರನು ತನ್ನ ಊರಿಗೆ ಹಿಂದಿರುಗಿದನು. ಈ ಸುದ್ದಿಯನ್ನು ಕೇಳಿದ ಉತ್ತಾನ ಪಾದರಾಯನು ಮಂತ್ರಿ ಪುರೋಹಿತರೊಡನೆ ಅತ್ಯಂತ ಸಂಭ್ರಮದಿಂದ ಅವನನ್ನು ಇದಿರುಗೊಂಡು ವೈಭವದಿಂದ ಅರಮನೆಗೆ ಕರೆದೊಯ್ದನು. ರಾಜಧಾನಿಯೆಲ್ಲವೂ ಅಲಂ ಕೃತವಾಗಿತ್ತು. ಪುರಜನರು ಆದರದಿಂದ ಆತನನ್ನು ಕಂಡು, ಮಂತ್ರಾಕ್ಷತೆ, ಹೂಹಣ್ಣು ಗಳನ್ನು ಆತನಮೇಲೆ ಎರಚಿ ಹರಸಿದರು. ಹೆಣ್ಣು ಮಕ್ಕಳು ಮಂಗಳಗೀತೆಗಳನ್ನು ಹಾಡಿ, ಆರತಿಯೆತ್ತಿದರು. ಧ್ರುವಕುಮಾರನು ಎಲ್ಲರನ್ನೂ ವಂದಿಸುತ್ತ, ಅರಮನೆಯನ್ನು ಪ್ರವೇ ಶಿಸಿ, ತಂದೆತಾಯಿಗಳಿಗೆ ನಮಸ್ಕರಿಸಿದನು. ಸುರುಚಿಯು ಆತನನ್ನು ‘ಸುಖವಾಗಿ ಬಾಳು’ ಎಂದು ಹರಸಿದಳು. ತಾಯಿಯಾದ ಸುನೀತಿ ಮಗನನ್ನು ಆಲಿಂಗಿಸಿಕೊಂಡು ಆನಂದಬಾಷ್ಪ ಗಳನ್ನು ಸುರಿಸಿದಳು. ಸುತ್ತಮುತ್ತಲಿನ ಜನ ಆಕೆಯನ್ನು ಭಾಗ್ಯಶಾಲಿಯೆಂದು ಬಾಯ್ತುಂಬ ಹೊಗಳಿದರು. ಉತ್ತಮಕುಮಾರನು ಅಣ್ಣನನ್ನು ಅಪ್ಪಿಕೊಂಡು ಆದರಿಸಿದನು. ಅರಮನೆ ಯೆಲ್ಲ ಆನಂದಮಯವಾಯಿತು. ಕೆಲಕಾಲ ಕಳೆದಮೇಲೆ ಉತ್ತಾನಪಾದನು ಮಗನಿಗೆ ಪಟ್ಟಾಭಿಷೇಕವನ್ನು ಮಾಡಿ, ತಪಸ್ಸು ಮಾಡುವುದಕ್ಕಾಗಿ ಕಾಡಿಗೆ ತೆರಳಿದನು.

ಚಕ್ರವರ್ತಿಯಾದ ಧ್ರುವನು ಶಿಶುಮಾರನೆಂಬ ಪ್ರಜಾಪತಿಯ ಮಗಳಾದ ‘ಭ್ರಮಿ’ ಯನ್ನೂ ವಾಯುವಿನ ಮಗಳಾದ ‘ಇಳಾ’ ಎಂಬುವಳನ್ನೂ ಕೈಹಿಡಿದನು. ಭ್ರಮಿಯಲ್ಲಿ ಕಲ್ಪ, ವತ್ಸಕ ಎಂಬ ಇಬ್ಬರು ಮಕ್ಕಳು ಹುಟ್ಟಿದರು. ಇಳೆಯಲ್ಲಿ ಉತ್ಕಲನೆಂಬ ಮಗ ಹುಟ್ಟಿದನು. ಈ ಮಡದಿ ಮಕ್ಕಳೊಡನೆ ಆತನು ಸಂತೋಷದಿಂದ ನಲಿಯುತ್ತಿದ್ದನು. ಆತನ ತಮ್ಮನಾದ ಉತ್ತಮಕುಮಾರನು ಒಮ್ಮೆ ಬೇಟೆಗೆಂದು ಅಡವಿಗೆ ಹೋಗಿದ್ದಾಗ ಶೂರನಾದ ಯಕ್ಷನೊಬ್ಬನ ಕೈಗೆ ಸಿಕ್ಕಿ ಸತ್ತುಹೋದನು. ಅವನು ಬಹುಕಾಲವಾದರೂ ಹಿಂದಿರುಗದಿರಲು, ಅವನ ತಾಯಿಯಾದ ಸುರುಚಿ ಮಗನನ್ನು ಹುಡುಕುತ್ತಾ ಅಡವಿಗೆ ಹೋಗಿ, ಅಲ್ಲಿ ಕಾಡುಕಿಚ್ಚಿಗೆ ಸಿಕ್ಕಿ ಸತ್ತುಹೋದಳು. ತಮ್ಮನ ಸಾವನ್ನು ಕೇಳಿ ಧ್ರುವ ಚಕ್ರವರ್ತಿಯು ಕೋಪದಿಂದ ಯಕ್ಷರ ವಾಸಸ್ಥಳವಾದ ಅಲಕಾವತಿಗೆ ಹೋಗಿ, ಅವರನ್ನು ಯುದ್ಧಕ್ಕೆ ಕರೆದನು. ಘನಘೋರವಾದ ಯುದ್ಧ ನಡೆದು ಸಹಸ್ರಾರು ಜನ ಯಕ್ಷರು ಹತ ರಾದರು. ಇನ್ನು ತಮಗೆ ಉಳಿಗಾಲವಿಲ್ಲವೆಂದು ತಿಳಿದು ಯಕ್ಷರು ಮಾಯಾಯುದ್ಧಕ್ಕೆ ಪ್ರಾರಂಭಿಸಿದರು. ಇದ್ದಕ್ಕಿದ್ದಂತೆ ಬಿರುಗಾಳಿಯೆದ್ದಿತು. ಆಕಾಶದಿಂದ ನೆತ್ತರ ಮಳೆ ಸುರಿ ಯಿತು. ತಲೆಯಿಲ್ಲದ ಮುಂಡಗಳು ತಪತಪನೆ ಕೆಳಗೆ ಬಿದ್ದವು. ಎಲ್ಲ ದಿಕ್ಕುಗಳಿಂದಲೂ ಕಲ್ಲಮಳೆ ಸುರಿಯಿತು. ಹೆಬ್ಬಾವುಗಳು, ಹುಲಿ ಸಿಂಹಗಳು ಹಿಂಡು ಹಿಂಡಾಗಿ ಅಟ್ಟಿಸಿ ಕೊಂಡು ಬಂದವು. ಧ್ರುವಚಕ್ರವರ್ತಿ ಕ್ಷಣಕಾಲ ಮುಂಗಾಣದವನಾಗಿ, ತನ್ನ ಬತ್ತಳಿಕೆ ಯಿಂದ ನಾರಾಯಣಾಸ್ತ್ರವನ್ನು ಹೊರತೆಗೆದನು. ತಕ್ಷಣವೆ ಯಕ್ಷರ ಮಾಯೆಯೆಲ್ಲ ನಾಶ ವಾಯಿತು. ಅದನ್ನು ಪ್ರಯೋಗಿಸುತ್ತಲೆ ಯಕ್ಷರು ಹಿಂಡುಹಿಂಡಾಗಿ ಹತರಾದರು.

ನಿರಪರಾಧಿಗಳಾದ ಯಕ್ಷರು ಹುಳಗಳಂತೆ ಸಾಯುತ್ತಿರಲು, ಧ್ರುವಚಕ್ರವರ್ತಿಯ ತಾತ ನಾದ ಸ್ವಾಯಂಭುವಮನುವು ಪುಷಿಗಳೊಡನೆ ಅಲ್ಲಿ ಪ್ರತ್ಯಕ್ಷನಾಗಿ, ತನ್ನ ಮೊಮ್ಮಗ ಮಾಡುತ್ತಿದ್ದ ಕಗ್ಗೊಲೆಯನ್ನು ನಿಲ್ಲಿಸಿದನು. ಆತನು ಧ್ರುವಚಕ್ರಿಯನ್ನು ಕುರಿತು ‘ಮಗು, ಯಾವನೊ ಒಬ್ಬ ಯಕ್ಷ ನಿನ್ನ ತಮ್ಮನನ್ನು ಕೊಂದನೆಂದು ಯಕ್ಷವಂಶವನ್ನೆ ಅಳಿಸಿ ಬಿಡುತ್ತಾರೆಯೆ? ಇದೇನನ್ಯಾಯ? ಕೋಪ ಪಾಪಕ್ಕೆ ಕಾರಣ. ಶ್ರೀಹರಿಯನ್ನು ಪ್ರತ್ಯಕ್ಷವಾಗಿ ಕಂಡು, ಮಹಾಮಹಿನಾದ ಜ್ಞಾನಿಯೆನಿಸಿಕೊಂಡಿರುವ ನೀನು ನಿರಪರಾಧಿಗಳಾದ ಈ ಯಕ್ಷ ರನ್ನೆಲ್ಲ ಕೊಲ್ಲಹೊರಟಿರುವುದು ಎಂತಹ ಅನ್ಯಾಯ! ಸಮಸ್ತ ಜೀವಿಗಳ ಅಂತರಂಗದಲ್ಲಿ ಭಗವಂತ ನೆಲಸಿರುವನೆಂಬುದು ನಿನಗೆ ಗೊತ್ತಿಲ್ಲವೆ? ನಿನ್ನ ಸರ್ವ ಸಮಬುದ್ಧಿ ಎಲ್ಲಿ ಹೋಯಿತು? ನಿನ್ನ ತಮ್ಮನನ್ನು ಯಕ್ಷನೊಬ್ಬನು ಕೊಂದನೆಂದು ಭಾವಿಸಿದುದೇ ಮೊದಲ ತಪ್ಪು. ಅವನನ್ನು ಯಕ್ಷ ಕೊಲ್ಲಲೂ ಇಲ್ಲ, ಅವನು ಸಾಯಲೂ ಇಲ್ಲ. ಹುಟ್ಟು ಸಾವುಗಳು ಶರೀರಕ್ಕೆ ಹೊರತು ಆತ್ಮನಿಗಲ್ಲ. ಆದ್ದರಿಂದ ನಿನ್ನ ತಮ್ಮನಿಗಾಗಿ ನೀನೀಗ ನಡೆಸುತ್ತಿರುವು ದೆಲ್ಲವೂ ಶುದ್ಧ ಅವಿವೇಕ. ಜಗತ್ತಿನ ವ್ಯಾಪಾರವೆಲ್ಲವೂ ಈಶ್ವರನ ಸಂಕಲ್ಪ. ಆ ದೇವ ದೇವನು ಕರ್ತನಾಗದೆಯೆ ಕೆಲಸಮಾಡುವನು, ಹೊಡೆಯದೆಯೆ ಕೊಲ್ಲುವನು. ಅವನ ವ್ಯಾಪಾರ ವಿಚಿತ್ರ. ತಾನೆ ಜಗತ್ತಾದರೂ ತಾನು ನಾಶವಾಗದೆ ಜಗತ್ತನ್ನು ನಾಶಮಾಡುತ್ತಾನೆ, ತಾನು ಹುಟ್ಟದೆ ಅದನ್ನು ಹುಟ್ಟಿಸುತ್ತಾನೆ. ಇದರ ಮೂಲವನ್ನು ಅರಿಯಲಾರದವರು ಜಗತ್ತಿನ ವ್ಯಾಪಾರಕ್ಕೆ ಏನೇನೋ ಕಾರಣಗಳನ್ನು ಹೇಳುತ್ತಾರೆ. ಕೆಲವರು ಕರ್ಮ\footnote{೧. ಮೀಮಾಂಸಕರು.} ಎಂದು, ಮತ್ತೆ ಕೆಲವರು ಪ್ರಕೃತಿ\footnote{೨. ನಾಸ್ತಿಕ} ಎಂದು, ಇನ್ನು ಕೆಲವರು ಗ್ರಹಾದಿ ದೈವವೆಂದು\footnote{೩. ಜೋಯಿಸರು}, ಮತ್ತೆ ಕೆಲವರು ಕಾಮ\footnote{೪. ವಾತ್ಸಾಯನ}ವೆಂದು ಹೇಳುವರು. ಆದರೆ ಇವೆಲ್ಲವೂ ಮೂಲತತ್ವದ ಒಂದೊಂದು ಶಕ್ತಿ ಮಾತ್ರವಾಗಿದೆ. ಈ ಎಲ್ಲ ಶಕ್ತಿಗಳನ್ನೂ ಒಳಗೊಂಡ ಮಹಾಮಹಿಮ ಭಗವಂತ. ಆತನ ಸಂಕಲ್ಪವೆ ಎಲ್ಲಕ್ಕೂ ಕಾರಣವಾದುದರಿಂದ ಅಹಂಕಾರ ಬೇಡ’ ಎಂದನು. 

ಅಜ್ಜನ ಮಾತುಗಳನ್ನು ಕೇಳಿ ಧ್ರುವಕುಮಾರನ ಕೋಪ ಅಡಗಿತು. ಆತನು ಯುದ್ಧ ವನ್ನು ನಿಲ್ಲಿಸಿ ಶಾಂತಿಯನ್ನು ವಹಿಸಿದನು. ತಕ್ಷಣವೇ ಯಕ್ಷರಾಜನಾದ ಕುಬೇರನು ಪ್ರಕ್ಯಕ್ಷನಾಗಿ, ಧ್ರುವನ ಅಪೇಕ್ಷೆಯಂತೆ ಸದಾ ಭಗವಂತನ ಸ್ಮರಣೆಯುಂಟಾಗುವ ವರ ವನ್ನು ಕರುಣಿಸಿದನು. ಅನಂತರ ಧ್ರುವ ಚಕ್ರವರ್ತಿಯು ರಾಜಧಾನಿಗೆ ಹಿಂದಿರುಗಿ, ಶ್ರೀಮನ್ನಾರಾಯಣನು ಹೇಳಿದ್ದಂತೆ ಮೂವತ್ತಾರು ಸಹಸ್ರ ವರ್ಷಗಳವರೆಗೆ ಧರ್ಮದಿಂದ ರಾಜ್ಯಭಾರಮಾಡುತ್ತಿದ್ದು, ಅನಂತರ ಸಿಂಹಾಸನವನ್ನು ಮಗನಿಗೆ ಕೊಟ್ಟು ವೈರಾಗ್ಯದಿಂದ ಬದರಿಕಾಶ್ರಮಕ್ಕೆ ಹೊರಟುಹೋದನು. ಅಲ್ಲಿ ಯೋಗಾಭ್ಯಾಸ ನಿರತನಾಗಿದ್ದು ಸಮಾಧಿ ಸ್ಥಿತಿಯನ್ನು ಪಡೆದನು. ಭಗವಂತನ ಆತ್ಮೀಯರಾದ ಸುನಂದ ನಂದರು ಆತನಿಗೆ ಕಾಣಿಸಿ ಕೊಂಡು, ಆತನನ್ನು ಪರಮಪದಕ್ಕೆ ಕರೆದೊಯ್ದರು. ಆತನ ಪುಣ್ಯದಿಂದ ತಾಯಿಯಾದ ಸುನೀತಿ ಆತನ ಮುಂದೆಯೇ ವಿಮಾನವೇರಿ ಸಾಗುತ್ತಿದ್ದಳು. ಧ್ರುವನೇರಿದ ವಿಮಾನ ಮೂರು ಲೋಕಗಳನ್ನೂ ದಾಟಿ, ಅವನು ಸಪ್ತಪುಷಿಮಂಡಲಕ್ಕೂ ಮೇಲೆ ಇದ್ದ ನಾಶ ರಹಿತವಾದ ವಿಷ್ಣುಪದವನ್ನು ಪಡೆದನು. ಅದೇ ಧ್ರುವಲೋಕ. ನಾರದನು ಆತನ ಮಹಿಮೆ ಯನ್ನು ಹಾಡಿ ತನ್ನ ಗಾನದಿಂದ ಮೂರು ಲೋಕವನ್ನು ತುಂಬಿದನು.

