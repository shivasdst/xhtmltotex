
\chapter{೪೪. ಅಮ್ಮ, ಇದೆಂತಹ ಮಗುವೆ!}

ಯಶೋದೆಯ ಕಂದ ತನ್ನ ಶಿಶುಲೀಲೆಯನ್ನು ನಟಿಸುತ್ತ, ಒಮ್ಮೆ ಬೋರಲಾದ. ಮಕ್ಕಳು ಬೋರಲಾದರೆ ಹೆತ್ತವರಿಗೆ ಅದೇ ಒಂದು ಮಹೋತ್ಸವ. ಇದರ ಮೇಲೆ ಅಂದು ಮಗುವಿನ ಜನ್ಮ ನಕ್ಷತ್ರವಾದ ರೋಹಿಣಿಯಿಂದ ಕೂಡಿದ ಶುಭದಿನವಾಗಿತ್ತು. ಆದ್ದರಿಂದ ಯಶೋದೆ ಗೋಕುಲದ ಮುತ್ತೈದೆಯರನ್ನೆಲ್ಲ ಅರಿಸಿನ ಕುಂಕುಮಕ್ಕೆ ಆಹ್ವಾನಿಸಿ, ಅವರಿಂದ ಮಗುವಿಗೆ ಮಂಗಳ ಸ್ನಾನ ಮಾಡಿಸಿದಮೇಲೆ ಆರತಿ ಅಕ್ಷತೆಗಳನ್ನು ನೆರ ವೇರಿಸಿದಳು. ಅಷ್ಟರಲ್ಲಿ ಮಗುವಿಗೆ ನಿದ್ದೆ ಬಂತು. ಮನೆಯ ಅಂಗಳದಲ್ಲಿ, ಪಾತ್ರೆಗಳಿಂದ ತುಂಬಿದ ಬಂಡಿಯೊಂದು ನಿಂತಿತ್ತು. ಅದರ ಕೆಳಗಿದ್ದ ತೊಟ್ಟಿಲಲ್ಲಿ ಮಗುವನ್ನು ಮಲಗಿಸಿ, ಆಕೆ ಮುತ್ತೈದೆಯರಿಗೆ ತಾಂಬೂಲ ತೆಂಗಿನಕಾಯಿಗಳನ್ನು ಕೊಡುವ ಕಾರ್ಯದಲ್ಲಿ ನಿರತಳಾದಳು. ಸ್ವಲ್ಪ ಹೊತ್ತಿನೊಳಗಾಗಿ ಮಗು ಎದ್ದು ಅಳುವುದಕ್ಕೆ ಪ್ರಾರಂಭಿಸಿತು. ಯಶೋದೆಯು ಮೇಲಿಂದಮೇಲೆ ಬರುತ್ತಿದ್ದ ಮುತ್ತೈದೆಯರನ್ನು ಆದರಿಸುತ್ತಿದ್ದುದರಿಂದ ಅದರ ಅಳುವು ಕೇಳಿಸಲಿಲ್ಲ. ಆಕೆ ತನ್ನ ಕಾರ್ಯದಲ್ಲಿ ತಾನು ಮಗ್ನಳಾಗಿದ್ದಳು. ಇತ್ತ ಮಗು ಅಳುತ್ತಳುತ್ತ ಕಾಲನ್ನು ಝಾಡಿಸಿತು. ಚಿಗುರಿನಂತೆ ಮೃದುವಾದ ಅದರ ಪುಟ್ಟಕಾಲುಗಳು ಗಾಡಿಗೆ ತಾಕುತ್ತಲೆ, ಆ ತುಂಬಿದ ಗಾಡಿ ತಲೆಕೆಳಗಾಗಿ ಮಗುಚಿಕೊಳ್ಳಬೇಕೆ! ಅದರೊಳಗಿದ್ದ ಕಂಚು ತಾಮ್ರ ಹಿತ್ತಾಳೆಯ ಪಾತ್ರೆಗಳು ನೆಲಕ್ಕೆ ಬಡಿದು ಭಯಂಕರವಾದ ಶಬ್ದವಾಯಿತು. ಅದನ್ನು ಕೇಳುತ್ತಲೆ ಮನೆಯಲ್ಲಿದ್ದ ಹೆಂಗಸರು ಗಂಡಸರೆಲ್ಲ ಹೊರಗೋಡಿಬಂದರು. ಅವರಿಗೆಲ್ಲ ಪರಮಾಶ್ಚರ್ಯ, ತುಂಬಿದ ಬಂಡಿ ತಲೆ ಕೆಳಗಾದುದು ಹೇಗೆ? ಎಂದು. ಆಗ ಆ ಗಾಡಿಯ ಹತ್ತಿರದಲ್ಲೆ ಆಟವಾಡುತ್ತಿದ್ದ ಹುಡುಗರು ‘ಈ ಮಗು ಅಳುತ್ತಾ ತನ್ನ ಕಾಲಿ ನಿಂದ ಈ ಗಾಡಿಯನ್ನು ಚಿಮ್ಮಿತು’ ಎಂದು, ಅಲ್ಲಿ ಮಲಗಿದ್ದ ಮಗುವನ್ನು ತೋರಿಸಿ ಹೇಳಿದರು. ಇದನ್ನು ಕೇಳಿ ಕೆಲವರು ‘ಅಮ್ಮ, ಇದೆಂತಹ ಮಗುವಮ್ಮ!’ಎಂದರು. ಆದರೆ ನಂದ ಯಶೋದೆಯರು ಈ ಮಾತನ್ನು ನಂಬಲಿಲ್ಲ. ಇದಾವುದೋ ಗ್ರಹಚೇಷ್ಟೆ ಎಂದು ಕೊಂಡು, ಅವರು ಬ್ರಾಹ್ಮಣರಿಂದ ಸ್ವಸ್ತಿವಾಚನವನ್ನು ಮಾಡಿಸಿ, ದಕ್ಷಿಣೆಕೊಟ್ಟು ಕಳುಹಿಸಿದರು.

ಇನ್ನೊಂದು ದಿನ ಇನ್ನೊಂದು ಸೋಜಿಗ ನಡೆಯಿತು. ಯಶೋದೆ ತನ್ನ ಕಂದನನ್ನು ತೊಡೆಯ ಮೇಲೆ ಕೂಡಿಸಿಕೊಂಡು ಆಟವಾಡಿಸುತ್ತಿದ್ದಾಳೆ. ಇದ್ದಕ್ಕಿದ್ದಂತೆ ಮಗು ಪರ್ವತ ದಷ್ಟು ಭಾರವಾದ ಹಾಗೆ ಕಾಣಿಸಿತು. ಆಕೆ ಅದರ ಭಾರವನ್ನು ಹೊರಲಾರದೆ ಅದನ್ನು ನೆಲದ ಮೇಲೆ ಮಲಗಿಸಿ, ದೇವರನ್ನು ಧ್ಯಾನಮಾಡುತ್ತ ಮನೆಗೆಲಸದಲ್ಲಿ ತೊಡಗಿದಳು. ಆಗ ಎಲ್ಲಿಂದಲೋ ಸುಂಟರಗಾಳಿಯೊಂದು ಕಾಣಿಸಿಕೊಂಡು, ಮಲಗಿದ್ದ ಮಗುವನ್ನು ಹಾರಿಸಿ ಕೊಂಡು ಹೋಯಿತು. ಆ ಗಾಳಿ ಮನೆಯಲ್ಲೆಲ್ಲ ತುಂಬಿಕೊಂಡುದರಿಂದ, ಅದರ ಧೂಳಿಗೆ ಯಾರೂ ಕಣ್ಣನ್ನೆ ಬಿಡುವಂತಿರಲಿಲ್ಲ. ಕ್ಷಣಕಾಲವಾದ ಮೇಲೆ ಗಾಳಿ ಮುಂದಕ್ಕೆ ಚಲಿಸಲು, ಯಶೋದೆ ಮಗುವೇನಾಯಿತೊ ಎಂಬ ಗಾಬರಿಯಿಂದ ಅದನ್ನು ಮಲಗಿಸಿದ್ದ ಕಡೆಗೆ ಓಡಿ ಬಂದು ನೋಡುತ್ತಾಳೆ, ಮಗುವೇ ಇಲ್ಲ! ಅವಳು ಭಯದಿಂದಲೂ ದುಃಖದಿಂದಲೂ ಗಟ್ಟಿಯಾಗಿ ಅರಚಿಕೊಂಡು ಅಳುವುದಕ್ಕೆ ಪ್ರಾರಂಭಿಸಿದಳು. ಅವಳ ಆಕ್ರಂದನವನ್ನು ಕೇಳಿ ಸುತ್ತುಮುತ್ತಿನ ಹೆಣ್ಣುಮಕ್ಕಳೆಲ್ಲ ಓಡಿಬಂದರು. ಆಡುತ್ತಿದ್ದ ಮಗು ಕಣ್ಮರೆಯಾದುದನ್ನು ಕೇಳಿ ಅವರಿಗೂ ದಿಗಿಲಾಯಿತು. ಹೆಣ್ಣುಮಕ್ಕಳು ಒಟ್ಟಿಗೆ ಸೇರಿದ ಮೇಲೆ ಕೇಳಬೇಕೆ? ಎಲ್ಲರೂ ಒಟ್ಟಿಗೆ ‘ಅಯ್ಯೋ ಕಂದ, ಎಲ್ಲಿಹೋದೆಯಪ್ಪ?’ ಎಂದು ಹಾಡಿಹಾಡಿಕೊಂಡು ಅಳುವುದಕ್ಕೆ ಪ್ರಾರಂಭಿಸಿದರು.

ಅತ್ತ ಸುಂಟರಗಾಳಿ ಮಗುವನ್ನು ಹುಲ್ಲುಕಡ್ಡಿಯಂತೆ ಮೇಲಕ್ಕೆತ್ತಿಕೊಂಡು ಆಕಾಶಕ್ಕೆ ಹಾರಿಸಿತು. ಬರಿಯ ಗಾಳಿಯಾದರೆ ಅದಕ್ಕೆ ಆ ಶಕ್ತಿಯೆಲ್ಲಿಯದು? ತೃಣಾವರ್ತನೆಂಬ ರಕ್ಕಸನು ಗಾಳಿಯ ರೂಪದಿಂದ ಕಾಣಿಸಿಕೊಂಡು ಆ ಮಗುವನ್ನು ಹಾರಿಸಿಕೊಂಡು ಹೋಗಿ ದ್ದನು. ಆದರೆ ಅದೆಂತಹ ಮಗು! ಅದರ ಭಾರವನ್ನು ಬಹಳ ಹೊತ್ತು ಹೊರುವುದು ಆ ರಕ್ಕಸನಿಗೆ ಸಾಧ್ಯವಾಗದೆ ಹೋಯಿತು. ಆ ಮಗು ತನಗಿಂತಲೂ ಹೆಚ್ಚು ಭಾರವೆನಿಸಿತು, ಅವನಿಗೆ. ಅದನ್ನು ಕೆಳಕ್ಕೆ ಎತ್ತಿಹಾಕಬೇಕೆಂದು ನೋಡಿದ. ಆದರೆ ಅದು ಅವನ ಕುತ್ತಿಗೆಯನ್ನು ಗಟ್ಟಿಯಾಗಿ ಹಿಡಿದುಕೊಂಡುಬಿಟ್ಟಿದೆ. ಅದರಿಂದ ಬಿಡಿಸಿಕೊಳ್ಳುವುದು ಅವನಿಗೆ ಸಾಧ್ಯವಾಗಲಿಲ್ಲ. ಅಷ್ಟೇ ಅಲ್ಲ, ಅದರ ಹಿಡಿತದಿಂದ ಅವನಿಗೆ ಉಸಿರಾಡುವುದೇ ಕಷ್ಟವಾಯಿತು. ಅವನಿಗೆ ಕಿರುಚಿಕೊಳ್ಳುವುದೂ ಸಾಧ್ಯವಾಗಲಿಲ್ಲ. ಅವನ ಉಸಿರು ಕಟ್ಟಿತು, ಬಲ ಕುಗ್ಗಿತು, ಕಣ್ಣುಗುಡ್ಡೆಗಳು ರೆಪ್ಪೆಯಲ್ಲಿ ಸಿಕ್ಕಿಕೊಂಡವು, ಅವನು ಸತ್ತು ನೆಲಕ್ಕೆ ಧೊಪ್ಪೆಂದು ಬಿದ್ದ. ಆ ಶಬ್ದವನ್ನು ಕೇಳಿ, ಯಶೋದೆ ತನ್ನ ಸ್ತ್ರೀಪರಿವಾರದೊಡನೆ ಹೋಗಿ ನೋಡುತ್ತಾಳೆ, ಭಯಂಕರಾಕಾರನಾದ ರಕ್ಕಸ, ಅವನ ಕತ್ತನ್ನು ಹಿಡಿದು ಕಿಲಿಕಿಲಿ ನಗು ತ್ತಿರುವ ತನ್ನ ಮುದ್ದು ಕಂದ! ಆಕೆಯ ಸುತ್ತಲಿದ್ದ ಹೆಣ್ಣುಗಳಲ್ಲಿ ಒಬ್ಬಳು ಧೈರ್ಯದಿಂದ ಮುನ್ನುಗ್ಗಿ ಹೋಗಿ, ಮಗುವನ್ನು ಎತ್ತಿಕೊಂಡುಬಂದು ಯಶೋದೆಯ ಕೈಗಿತ್ತಳು. ಎಲ್ಲಿಯೋ ಹೋಗಿದ್ದ ನಂದ ಆ ವೇಳೆಗೆ ಅಲ್ಲಿಗೆ ಬಂದ. ಘೋರರಾಕ್ಷಸನ ಕೈಯಿಂದ ಬದುಕಿಬಂದ ಮಗನನ್ನು ಕಂಡು ಆತನಿಗೆ ಅತ್ಯಾನಂದವಾಯಿತು. ‘ನಾನು ಪೂರ್ವಜನ್ಮ ದಲ್ಲಿ ಯಾವ ಪುಣ್ಯವನ್ನು ಮಾಡಿದ್ದೆನೊ! ಮಗುವನ್ನು ಮತ್ತೆ ಕಾಣುವಂತಾಯಿತು’ ಎಂದು ಕೊಂಡ. ನೆರೆದ ಹೆಣ್ಣುಗಳ ಬಾಯಿಂದ ಉದ್ಗಾರ ಹರಿದು ಬಂತು: ‘ಅಮ್ಮ, ಇದೆಂತಹ ಮಗುವೆ!’

