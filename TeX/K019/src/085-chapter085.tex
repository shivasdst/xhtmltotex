
\chapter{೮೫. ಹಿತ್ತಲ ಗಿಡ ಮದ್ದಲ್ಲ}

ಶ್ರೀಕೃಷ್ಣ, ಪಾಂಡವರು, ವಸುದೇವ ಮೊದಲಾದ ಗಂಡಸರೆಲ್ಲ ಒಂದು ಕಡೆ, ದ್ರೌಪದಿ, ಕುಂತಿ, ಶ್ರೀಕೃಷ್ಣನ ಮಡದಿಯರು ಇತ್ಯಾದಿ ಹೆಂಗಸರ ಗುಂಪೆಲ್ಲ ಇನ್ನೊಂದು ಕಡೆ ಕುಳಿತು ಲೋಕಾಭಿರಾಮವಾಗಿ ಮಾತಾಡುತ್ತಿದ್ದಾರೆ. ಇದ್ದಕ್ಕಿದ್ದಂತೆ ವ್ಯಾಸ, ಚ್ಯವನ, ಭರಧ್ವಾಜ, ಗೌತಮ, ವಸಿಷ್ಠ ಮೊದಲಾದ ಮಹರ್ಷಿಗಳ ದೊಡ್ಡ ಗುಂಪೇ ಅಲ್ಲಿಗೆ ಬಂದಿತು. ಅವರೆಲ್ಲರೂ ಸೂರ್ಯಗ್ರಹಣವೆಂದು ತೀರ್ಥಸ್ನಾನಕ್ಕಾಗಿ ಅಲ್ಲಿಗೆ ಬಂದಿದ್ದವರೆ. ಅವರು ಶ್ರೀಕೃಷ್ಣನ ದರ್ಶನ ಮಾಡಿಕೊಂಡು ಅನಂತರ ತಮ್ಮ ತಮ್ಮ ಆಶ್ರಮಗಳಿಗೆ ಹಿಂದಿರುಗಬೇಕೆಂದು ನಿಶ್ಚಯಿಸಿದ್ದರು. ಆ ಮಹರ್ಷಿಗಳನ್ನು ಕಾಣುತ್ತಲೆ ಅಲ್ಲಿದ್ದವರೆಲ್ಲ ಮೇಲಕ್ಕೆದ್ದು ಅವರಿಗೆ ದೀರ್ಘದಂಡ ನಮಸ್ಕಾರ ಮಾಡಿದರು. ಶ್ರೀಕೃಷ್ಣನೂ ಅವರಿಗೆ ನಮಸ್ಕರಿಸಿ, ಕುಶಲ ಪ್ರಶ್ನೆಯನ್ನು ಕೇಳಿದಮೇಲೆ ಅವರನ್ನು ಉಚಿತಾಸನದಲ್ಲಿ ಕುಳ್ಳಿರಿಸಿ ‘ಮಹಾನುಭಾವರೆ, ನಿಮ್ಮ ದರ್ಶನದಿಂದ ನಾವೆಲ್ಲ ಧನ್ಯರಾದೆವು. ಸ್ವಾಮಿ, ತೀರ್ಥವೆಂದು ನಾವು ನೀರಿನಲ್ಲಿ ಮುಳುಗು ಹಾಕುತ್ತೇವೆ. ದೇವರೆಂದು ಕಲ್ಲಿನ ವಿಗ್ರಹವನ್ನು ಪೂಜೆಮಾಡು ತ್ತೇವೆ. ಆದರೆ ನಿಮ್ಮ ಪಾದದರ್ಶನದಿಂದ ನಮಗೆ ಎಲ್ಲ ತೀರ್ಥಸ್ನಾನದ ಪುಣ್ಯವೂ ಬರುತ್ತದೆ, ಎಲ್ಲ ದೇವರ ಪೂಜೆಯ ಫಲವೂ ಲಭಿಸುತ್ತದೆ. ಕಲ್ಲನ್ನು ದೇವರೆಂದೂ ನೀರನ್ನು ತೀರ್ಥವೆಂದೂ ತಿಳಿದು, ನಿಮ್ಮಂತಹ ತತ್ವಜ್ಞಾನಿಗಳಲ್ಲಿ ಅವೆರಡನ್ನೂ ಕಾಣ ದಾದರೆ ಅಂತಹವನು ಕತ್ತೆಗಿಂತಲೂ ಕಡೆ’ ಎಂದನು. ಆತನ ಮಾತುಗಳನ್ನು ಕೇಳಿ ಋಷಿಗಳಿಗೆಲ್ಲ ಮಂಕು ಹಿಡಿದಂತಾಯಿತು. ತಾವೆಲ್ಲ ಬಂದುದು ದೇವದೇವನಾದ ಆತನ ದರ್ಶನಕ್ಕೆ. ಆತ ತಮ್ಮನ್ನು ಅತಿಶಯವಾಗಿ ಹೊಗಳುತ್ತಿದ್ದಾನೆ, ಇದೆಂತಹ ಆಭಾಸ!

ಋಷಿಗಳು ಆಶ್ಚರ್ಯದಿಂದ ಶ್ರೀಕೃಷ್ಣಪರಮಾತ್ಮ ಹೀಗೇಕೆ ಮಾಡಿದನೆಂದು ಸ್ವಲ್ಪ ಹೊತ್ತು ಆಲೋಚಿಸಿದಮೇಲೆ ಅವರಿಗೆ ಅರ್ಥವಾಯಿತು–ಕೇವಲ ಲೋಕದ ಜನರಿಗೆ ಮಾದರಿಯಾಗಲೆಂದು ಆತ ಹಾಗೆ ಹೇಳಿದ್ದಾನೆ, ಎಂದು. ಆಗ ಅವರು ನಗುತ್ತಾ ‘ದೇವಾಧಿ ದೇವಾ, ಶ್ರೀಕೃಷ್ಣ! ನಮ್ಮನ್ನು ಜ್ಞಾನಿಗಳೆಂದು ಕರೆಯುತ್ತಿ. ನಿಜವಾಗಿಯೂ ನಾವು ತತ್ವ ಜ್ಞಾನಿಗಳಾಗಿದ್ದರೆ ಅದು ನಿನ್ನ ಅನುಗ್ರಹ. ನಾವು ಎಷ್ಟು ತತ್ವಗಳನ್ನು ತಿಳಿದರೆ ತಾನೆ ಏನು? ನಿನ್ನ ಮಾಯೆಯಿಂದ ತಪ್ಪಿಸಿಕೊಳ್ಳುವುದು ನಮಗೆ ಅಸಾಧ್ಯ. ನೀನು ಮನುಷ್ಯನಾಗಿ ಅವತರಿಸಿ, ಮನುಷ್ಯನಂತೆ ನಟಿಸುತ್ತಿ. ನಾವೂ ಅನೇಕ ವೇಳೆ ನಿನ್ನನ್ನು ಮನುಷ್ಯನೆಂದೇ ತಿಳಿದು ಮೋಸಹೋಗುತ್ತೇವೆ. ಸ್ವಾಮಿ, ನಿನ್ನ ವ್ಯಾಪಾರ ಅತಿ ವಿಚಿತ್ರ. ಸೃಷ್ಟಿಗೆ ಮೊದಲು ನೀನೊಬ್ಬನೆ ಇದ್ದೆ. ನಾಮ, ರೂಪಗಳ ಭೇದದಿಂದ ಅನೇಕವಾಗಿರುವ ಈ ಜಗತ್ತನ್ನು ಸೃಷ್ಟಿ ಸಿದೆ, ಕೊನೆಯಲ್ಲಿ ಸಂಹಾರ ಮಾಡುತ್ತಿ. ಆದರೆ ಈ ವಿಕಾರಗಳೊಂದಕ್ಕೂ ನೀನು ಒಳ ಗಾಗುವುದಿಲ್ಲ. ವೇದಧರ್ಮವನ್ನು ಜಗತ್ತಿನಲ್ಲಿ ಸ್ಥಾಪಿಸುವುದಕ್ಕಾಗಿಯೆ ಅವತಾರ ಮಾಡಿ ರುವ ನೀನು ವೇದಗಳನ್ನು ನಂಬುವ ನಮ್ಮಂತಹ ಬ್ರಾಹ್ಮಣರನ್ನು ಗೌರವಿಸುವುದು ಕೇವಲ ಜನರ ಮಾರ್ಗದರ್ಶನಕ್ಕಾಗಿ ಎಂಬುದನ್ನು ನಾವು ಬಲ್ಲೆವು. ವೇದವೇ ನಿನ್ನ ಅಂತ ರಂಗ. ಆ ವೇದದಿಂದಲೇ ನಿನ್ನ ಸ್ವರೂಪವನ್ನು ಅರ್ಥಮಾಡಿಕೊಳ್ಳಬೇಕು. ವೇದಗಳಿಗೆ ಬ್ರಾಹ್ಮಣನೇ ಆಧಾರವಾದುದರಿಂದ, ನಿನ್ನ ಸ್ವರೂಪವನ್ನು ತಿಳಿಸುವುದಕ್ಕೂ ಆತನೆ ಶಕ್ತ. ಅದರಿಂದಲೇ ನೀನು ಬ್ರಾಹ್ಮಣನನ್ನು ಬಹುಮಾನಿಸುವೆ. ನೀನು ಬ್ರಾಹ್ಮಣರಲ್ಲಿ ಬ್ರಾಹ್ಮಣ. ಸ್ವಾಮಿ, ಮಹಾಯೋಗಿಗಳಿಗೂ ಕೇವಲ ಹೃದಯಗೋಚರನಾದ ನಿನ್ನನ್ನು ನಾವು ಪ್ರತ್ಯಕ್ಷವಾಗಿ ಕಂಡುದರಿಂದ ನಮ್ಮ ಕಣ್ಣುಗಳು ಸಫಲವಾದುವು; ನಮ್ಮ ವಿದ್ಯೆ, ತಪಸ್ಸುಗಳು ಸಾರ್ಥಕವಾದವು’ ಎಂದು ಸ್ತೋತ್ರ ಮಾಡಿದರು.

ಋಷಿಗಳು ಶ್ರೀಕೃಷ್ಣನ ಗುಣಗಾನ ಮಾಡಿದ ಮೇಲೆ ಆತನಿಂದ ಬೀಳ್ಕೊಂಡು ತಮ್ಮ ತಮ್ಮ ಆಶ್ರಮಗಳ ಹಾದಿಯನ್ನು ಹಿಡಿದರು. ಆಗ ವಸುದೇವನು ಓಡಿಹೋಗಿ, ಅವರಿಗೆಲ್ಲ ಅಡ್ಡಬಿದ್ದು ‘ಪೂಜ್ಯರಿರಾ, ಸಕಲ ಪಾಪಗಳನ್ನೂ ನಿವಾರಿಸುವ ಒಂದು ಸತ್ಕರ್ಮವನ್ನು ನನಗೆ ಉಪದೇಶ ಮಾಡಿರಿ’ ಎಂದು ಬೇಡಿಕೊಂಡನು. ಇದನ್ನು ಕೇಳಿ ಅವರೆಲ್ಲ ಪಕಪಕ ನಕ್ಕರು. ‘ಸಾಕ್ಷಾತ್ ಭಗವಂತನಾದ ಶ್ರೀಕೃಷ್ಣನೇ ಇದಿರಿಗಿರುವಾಗ, ನಾವು ಉಪದೇಶ ಕೊಡಬೇಕೆ?’ ಎಂದು ಅವರ ಅಚ್ಚರಿ. ಆಗ ನಾರದನು ಆ ಋಷಿಗಳೊಡನೆ ‘ಅಯ್ಯಾ ಬ್ರಾಹ್ಮಣರೆ, ನೀವೇಕೆ ಅಚ್ಚರಿಗೊಳ್ಳುತ್ತೀರಿ? ಈ ವಸುದೇವ ಶ್ರೀಕೃಷ್ಣನನ್ನು ಕೇವಲ ತನ್ನ ಮಗನೆಂದು ತಿಳಿದಿದ್ದಾನೆ. ಅತಿ ಪರಿಚಯ ಅನಾದರಕ್ಕೆ ಕಾರಣ. ಗಂಗೆಯ ತಟದಲ್ಲಿರು ವವರು ಪುಣ್ಯತೀರ್ಥವನ್ನು ಹುಡುಕಿಕೊಂಡು ಹೋಗುವಂತೆ, ಈ ವಸುದೇವನೂ ಉಪ ದೇಶ ಮಾಡುವವರನ್ನು ಅರಸುತ್ತಿದ್ದಾನೆ. ಹಿತ್ತಲ ಗಿಡ ಮದ್ದಲ್ಲ’ ಎಂದನು. ಆಗ ಋಷಿಗಳು ವಸುದೇವನನ್ನು ಕುರಿತು ‘ಅಯ್ಯಾ, ಮುಳ್ಳಿನಿಂದಲೇ ಮುಳ್ಳು ತೆಗೆಯುವಂತೆ, ಕರ್ಮ ದಿಂದಲೇ ಕರ್ಮವನ್ನು ನೀಗಿಕೊಳ್ಳಬೇಕು. ಯಜ್ಞನಿಯಾಮಕ ಭಗವಂತ; ಯಜ್ಞ ಮಾಡಿಯೆ ಭಗವಂತನನ್ನು ಮೆಚ್ಚಿಸಬೇಕು. ಅದರಿಂದಲೇ ಸಮಸ್ತ ಪಾಪಗಳ ಪರಿಹಾರ. ನೋಡು, ಮನುಷ್ಯ ಹುಟ್ಟುವಾಗಲೆ ಮೂರು ಋಣಗಳನ್ನು–ದೇವಋಣ, ಋಷಿಋಣ, ಪಿತೃಋಣ –ಹೊತ್ತು ಹುಟ್ಟುತ್ತಾನೆ. ಅಧ್ಯಯನದಿಂದ ಋಷಿಋಣ ನೀಗುತ್ತದೆ. ಪುತ್ರಲಾಭದಿಂದ ಪಿತೃಋಣ ತೀರುತ್ತದೆ. ಇವೆರಡನ್ನೂ ನೀನು ತೀರಿಸಿರುವೆ. ಯಜ್ಞದಿಂದ ದೇವಋಣ ವೊಂದನ್ನು ನೀನು ತೀರಿಸಬೇಕಾಗಿದೆ. ಅದನ್ನಷ್ಟು ಮಾಡಿ, ಮುಗಿಸಿದೆಯೆಂದರೆ, ನೀನು ವಿರಕ್ತನಾಗಿ ತಪೋವನಕ್ಕೆ ತೆರಳಿ, ಮೋಕ್ಷವನ್ನು ಗಳಿಸುವುದಕ್ಕೆ ಪ್ರಯತ್ನಿಸ ಬಹುದು’ ಎಂದರು. ಇದನ್ನು ಕೇಳಿದ ವಸುದೇವನು ಆ ಋಷಿಗಳಿಗೆ ಮತ್ತೆ ಅಡ್ಡಬಿದ್ದು ‘ಸ್ವಾಮಿ, ತಾವೇ ಇಲ್ಲಿ ನಿಂತು ಆ ಯಜ್ಞವನ್ನು ಸಾಂಗವಾಗಿ ನೆರವೇರಿಸಿಕೊಟ್ಟು, ಆಮೇಲೆ ಇಲ್ಲಿಂದ ತೆರಳ ಬೇಕು’ ಎಂದು ಬೇಡಿಕೊಂಡ. ಆತನ ಪ್ರಾರ್ಥನೆಯಂತೆ ಆ ಋಷಿಗಳೆಲ್ಲ ಅಲ್ಲಿಯೇ ನಿಂತು, ಪುಣ್ಯಕರವಾದ ಆ ಸ್ಯಮಂತಪಂಚಕದಲ್ಲಿಯೆ, ಆತನಿಂದ ಶಾಸ್ತ್ರೋಕ್ತವಾಗಿ ಯಜ್ಞವನ್ನು ಮಾಡಿಸಿದರು.

ವಸುದೇವನು ಕೈಕೊಂಡ ಯಜ್ಞವು ಅತ್ಯಂತ ವೈಭವದಿಂದ ನಡೆಯಿತು. ಅಪ್ಸರೆಯರು ನರ್ತಿಸಿದರು, ಗಂಧರ್ವರು ಗಾನಮಾಡಿದರು. ಶ್ರೀಕೃಷ್ಣ ಬಲರಾಮರು ಸಕಲ ಯಾದವ ರೊಡನೆ ಮಿಂದು ಮಡಿಯುಟ್ಟು, ಯಾಗಶಾಲೆಯಲ್ಲಿ ಯಜ್ಞಸಂಭಾರಗಳನ್ನು ಒದಗಿಸಲು ಸಿದ್ಧರಾಗಿ ನಿಂತಿದ್ದರು. ಮಂಗಳವಾದ್ಯಗಳು ಭೋರ್ಗರೆಯುತ್ತಿರಲು ವಸುದೇವನಿಗೂ ಆತನ ಮಡದಿಯರಿಗೂ ಬೆಣ್ಣೆಯನ್ನು ಹಚ್ಚಿ ನೀರೆರೆದರು; ಕಣ್ಣುಗಳಿಗೆ ಕಾಡಿಗೆಯ ನ್ನಿಟ್ಟರು. ಯಜ್ಞದೀಕ್ಷಿತನಾದ ವಸುದೇವನು ಕೃಷ್ಣಾಜಿನವನ್ನು ಧರಿಸಿದನು. ಆತನ ಮಡದಿ ಯರೂ ಯಜ್ಞ ಮಾಡಿಸುವ ಋಷಿಗಳೂ ಹೊಸ ಪಟ್ಟೆಯ ಮಡಿಗಳನ್ನು ಧರಿಸಿದರು. ಯಜ್ಞ ಪ್ರಾರಂಭವಾಯಿತು. ಋಷಿಗಳು ಮಂತ್ರಗಳನ್ನು ಹೇಳಿ, ವಸುದೇವನಿಂದ ಅಗ್ನಿಯಲ್ಲಿ ಹೋಮವನ್ನು ಮಾಡಿಸಿದರು. ಯಜ್ಞ ಮುಗಿದಮೇಲೆ ಅಲ್ಲಿ ನೆರೆದವರೆಲ್ಲ ಸ್ಯಮಂತ ಪಂಚಕ ತೀರ್ಥದಲ್ಲಿ ಅವಭೃತಸ್ನಾನ ಮಾಡಿದರು. ವಸುದೇವನು ಯಥೇಚ್ಛವಾಗಿ ಗೋದಾನ, ಭೂದಾನ, ಧನದಾನಗಳನ್ನು ಮಾಡಿದನು. ಆತನ ಅನ್ನದಾನಗಳಿಂದ ಸಕಲ ಜೀವಿಗಳೂ ಸಂತುಷ್ಟಿಗೊಂಡವು. ಯಜ್ಞಕ್ಕೆ ಬಂದಿದ್ದ ಬಂಧುಮಿತ್ರರು ಆತನ ಬಹುಮಾನಗಳಿಂದ ತಣಿದು ಹೋದರು. ಅನಂತರ ಋಷಿಗಳೆಲ್ಲರೂ ಶ್ರೀಕೃಷ್ಣನಿಂದ ಬೀಳ್ಕೊಂಡು ಅಲ್ಲಿಂದ ಹೊರಟು ಹೋದರು. ಎಲ್ಲರ ಬಾಯಲ್ಲಿಯೂ ಯಾಗದ ವೈಭವವೇ. ಉಳಿದವರೆಲ್ಲ ತಮ್ಮತಮ್ಮ ಊರುಗಳಿಗೆ ಹೊರಟುಹೋದರಾದರೂ ನಂದನು ಬಲರಾಮಕೃಷ್ಣರಿಂದ ಅಗಲಲಾರದೆ ಅಲ್ಲಿಯೇ ಕೆಲಕಾಲ ನಿಂತನು. ವಸು ದೇವನು ಆತನ ಕೈಹಿಡಿದುಕೊಂಡು, ‘ಅಣ್ಣ, ಸ್ನೇಹಬಂಧನ ಎಷ್ಟು ಬಲವಾದುದು! ಇದನ್ನು ಕತ್ತರಿಸುವುದು ಮಹಾಯೋಗಿಗಳಿಗೂ ಕೂಡ ಕಷ್ಟ. ನೋಡು, ನೀನು ನನಗೆ ಎಷ್ಟು ಉಪಕಾರ ಮಾಡಿದೆ! ಅದಕ್ಕೆ ಪ್ರತಿಯಾಗಿ ನಾನು ಈವರೆಗೆ ಯಾವ ಪ್ರತ್ಯುಪಕಾರವನ್ನು ಮಾಡುವುದಕ್ಕೂ ಸಾಧ್ಯವಾಗಿಲ್ಲ. ಆದರೂ ನಿನ್ನ ಸ್ನೇಹ ಅಚ್ಚಳಿಯದಿದೆ. ಇದನ್ನು ನೋಡಿ ದರೆ ಭಗವಂತನು ಕೊಟ್ಟ ಈ ಸ್ನೇಹಕ್ಕೆ ಚ್ಯುತಿಯಿಲ್ಲವೆಂದು ತೋರುತ್ತದೆ’ ಎಂದು ಹೇಳಿ, ಆನಂದಬಾಷ್ಪವನ್ನು ಸುರಿಸಿದನು.

ಇಂದು, ನಾಳೆ–ಎಂದು ಹೇಳುತ್ತಾ ವಸುದೇವನು ಮೂರು ತಿಂಗಳ ಕಾಲ ನಂದನನ್ನು ಅಲ್ಲಿಯೇ ಉಳಿಸಿಕೊಂಡು, ಆಮೇಲೆ ಅಮೂಲ್ಯವಾದ ವಸ್ತ್ರಾಭರಣಗಳ ಬಹುಮಾನ ವಿತ್ತು ಆತನನ್ನು ಬೀಳ್ಕೊಟ್ಟನು. ಆ ವೇಳೆಗೆ ಮಳೆಗಾಲ ಹತ್ತಿರವಾದುದರಿಂದ ಯಾದವ ರೆಲ್ಲರೂ ದ್ವಾರಕಿಗೆ ಹಿಂದಿರುಗಿದರು. ವಸುದೇವನಿಗೆ ತಾನು ಯಜ್ಞವನ್ನು ಮಾಡಿ ಮುಗಿಸಿ ದುದಕ್ಕಾಗಿ ಅಪಾರವಾದ ಸಂತೋಷವಾಗಿತ್ತು. ಅದರ ಜೊತೆಗೆ ಶ್ರೀಕೃಷ್ಣನು ಪರಮಾತ್ಮ ನೆಂದು ಋಷಿಗಳು ಹೇಳಿದ ಮಾತು ಮನಸ್ಸಿನಲ್ಲಿ ನಾಟಿತ್ತು. ಒಮ್ಮೆ ತನಗೆ ಬಂದು ನಮಸ್ಕರಿಸಿದ ಶ್ರೀಕೃಷ್ಣನನ್ನು ಆತನು ತಬ್ಬಿಕೊಂಡು ‘ಅಪ್ಪ, ನೀನು ಹುಟ್ಟುತ್ತಲೆ ನೀನಾ ರೆಂಬುದನ್ನು ನನಗೆ ಹೇಳಿದ್ದೆ. ಆದರೂ ನಾನು ಮೋಹಪರವಶನಾಗಿ ನಿನ್ನನ್ನು ಮಗನೆಂದು ಕಾಣುತ್ತಿದ್ದೆ; “ಹಿತ್ತಲ ಗಿಡ ಮದ್ದಲ್ಲ” ಎಂಬ ಗಾದೆಗೆ ಒಂದು ದೊಡ್ಡ ಉದಾಹರಣೆ ಯಾದೆ. ಪರಮಾತ್ಮ, ನನ್ನನ್ನು ಅನುಗ್ರಹಿಸು. ನನ್ನ ವಿಷಯಾಭಿಲಾಷೆಯನ್ನು ಕಳೆದು ವೈರಾಗ್ಯವನ್ನು ಅನುಗ್ರಹಿಸು’ ಎಂದು ಬೇಡಿಕೊಂಡನು. ಆಗ ಶ್ರೀಕೃಷ್ಣನು ಮುಗುಳ್ನ ಗುತ್ತಾ “ಅಪ್ಪ, ನನ್ನನ್ನು ‘ಪರಮಾತ್ಮ’ ಎಂದು ಕರೆಯುತ್ತೀಯಲ್ಲ! ನಿನ್ನ ಮಾತು ಒಂದು ದೃಷ್ಟಿಯಿಂದ ನಿಜ. ಚರಾಚರಾತ್ಮಕವಾದ ಈ ಜಗತ್ತೆಲ್ಲವೂ ಪರಮಾತ್ಮನೆ” ಎಂದನು. ಅದೂ ಒಂದು ಉಪದೇಶವೇ ಆಯಿತು, ವಸುದೇವನಿಗೆ. ಪರಮಾತ್ಮನಲ್ಲದ ವಸ್ತುವೇ ಜಗತ್ತಿನಲ್ಲಿಲ್ಲವೆಂದು ಆತನಿಗೆ ನಿಶ್ಚಯಜ್ಞಾನ ಹುಟ್ಟಿತು.

ಗಂಡನಾದ ವಸುದೇವನಂತೆ ದೇವಕೀದೇವಿಗೂ ತನ್ನ ಮಗ ಪರಮಾತ್ಮನೆಂಬುದು ಆತನು ಹುಟ್ಟಿದಂದೆ ಗೊತ್ತಿದ್ದರೂ ಮಾಯೆಯ ಮುಸುಕಿನಲ್ಲಿ ಅದು ಮರೆಯಾಗಿ ಹೋಗಿತ್ತು. ಒಂದು ದಿನ ಆಕೆಗೆ ಅದು ಜ್ಞಾಪಕಕ್ಕೆ ಬಂತು, ಅದರೊಡನೆ ಒಂದು ವಿಲಕ್ಷಣ ವಾದ ಆಸೆಯೂ ಹುಟ್ಟಿತು. ಶ್ರೀಕೃಷ್ಣನು ತನ್ನ ಗುರು ಸಾಂದೀಪನ ಸತ್ತ ಮಗನನ್ನು ಯಮಲೋಕದಿಂದ ಹಿಂದಕ್ಕೆ ಕರೆತಂದು, ತನ್ನ ಗುರುವಿಗೆ ಒಪ್ಪಿಸಿದನೆಂಬ ಕಥೆಯನ್ನು ಆಕೆ ಕೇಳಿದ್ದಳು; ತನ್ನ ಸತ್ತ ಮಕ್ಕಳನ್ನೂ ಹಾಗೆಯೆ ಏಕೆ ಹಿಂದಕ್ಕೆ ಕರೆತರಬಾರದು? ಹಿತ್ತಿಲಲ್ಲಿಯೆ ಮದ್ದಿನ ಗಿಡವಿರುವಾಗ ಅದನ್ನ ಬಳಸಿಕೊಳ್ಳದಿರುವುದು ಎಂತಹ ಮೂಢತನ! ಆಕೆಗೆ ತನ್ನ ಸತ್ತ ಮಕ್ಕಳನ್ನು ಮತ್ತೆ ನೋಡಬೇಕೆಂಬ ಆಶೆ ಅಡಗಿಸಿಕೊಳ್ಳಲಾರ ದಷ್ಟು ಅಧಿಕವಾಯಿತು. ಆಕೆ ಬಲರಾಮಕೃಷ್ಣರನ್ನು ತನ್ನ ಬಳಿಗೆ ಕರೆಸಿ ‘ಹೇ ಪುರಾಣ ಪುರುಷನಾದ ಬಲರಾಮ, ಪರಮಾತ್ಮನಾದ ಶ್ರೀಕೃಷ್ಣ! ನೀವು ಆದಿಪುರುಷರೆಂದು, ಪರಮೇಶ್ವರರೆಂದು ನನಗೆ ಗೊತ್ತು. ಕೇವಲ ದುಷ್ಟಶಿಕ್ಷಣ ಶಿಷ್ಟರಕ್ಷಣೆಗಾಗಿ ನೀವು ಅವತರಿಸಿದ್ದೀರಿ. ಶರಣಾಗತರಕ್ಷಕರಾದ ನಿಮ್ಮಲ್ಲಿ ನಾನು ಮರೆಹೊಕ್ಕಿದ್ದೇನೆ. ನೀವು ನಿಮ್ಮ ಗುರುವಾದ ಸಾಂದೀಪನ ಕೋರಿಕೆಯಂತೆ ಎಷ್ಟೋ ವರ್ಷಗಳ ಹಿಂದೆ ಸತ್ತ ಆತನ ಮಗ ನನ್ನು ಯಮಲೋಕದಿಂದ ತಂದು ಒಪ್ಪಿಸಿದಿರಂತೆ! ಅದರಂತೆ ನನಗೂ ನನ್ನ ಸತ್ತ ಮಕ್ಕಳನ್ನು ತಂದುಕೊಡಿ. ಪಾಪಿಯಾದ ಕಂಸ ನಿಮಗಿಂತಲೂ ಮೊದಲು ನನ್ನ ಹೊಟ್ಟೆ ಯಲ್ಲಿ ಹುಟ್ಟಿದ ಆರು ಮಕ್ಕಳನ್ನು ಕೊಂದುಹಾಕಿದ. ಅವರನ್ನು ಒಮ್ಮೆ ಕಣ್ತುಂಬ ನೋಡಬೇಕೆಂಬ ಆಶೆ ನನಗೆ. ಪರಮಯೋಗಿಗಳಾದ ನಿಮಗೆ ನನ್ನ ಆಶೆಯನ್ನು ನೆರವೇರಿಸುವುದು ಕಷ್ಟವೇನೂ ಅಲ್ಲ’ ಎಂದಳು.

ತಾಯಿಯ ಪ್ರಾರ್ಥನೆಯನ್ನು ಸಲ್ಲಿಸಬೇಕೆಂದು ನಿಶ್ಚಯಿಸಿದ ಬಲರಾಮಕೃಷ್ಣರು ತಮ್ಮ ಯೋಗ ಶಕ್ತಿಯಿಂದ ಸುತಲಲೋಕಕ್ಕೆ ಹೋದರು. ಅಲ್ಲಿ ರಾಜನಾಗಿದ್ದ ಬಲಿಚಕ್ರ ವರ್ತಿಯು ಅವರಿಬ್ಬರನ್ನೂ ಭಕ್ತಿಯಿಂದ ಇದಿರುಗೊಂಡು, ನಮಸ್ಕರಿಸಿ, ಸಕಲೋಪಚಾರ ಗಳನ್ನೂ ಮಾಡಿದಮೇಲೆ ‘ಪರಮ ಪುರುಷರೆ, ನನ್ನ ಯಾವ ಸೇವೆಯನ್ನು ಬಯಸಿ ನೀವಿಲ್ಲಿಗೆ ಬಂದಿರುವಿರಿ?’ ಎಂದು ನಮ್ರವಾಗಿ ಕೇಳಿಕೊಂಡನು. ಆಗ ಶ್ರೀಕೃಷ್ಣನು ‘ಮಹಾರಾಜ, ನನ್ನ ತಾಯಿಯ ಹೊಟ್ಟೆಯಲ್ಲಿ ನನಗಿಂತಲೂ ಮೊದಲು ಹುಟ್ಟಿದ ಆರು ಜನ ಮಕ್ಕಳನ್ನು ಕಂಸ ಕೊಂದುಹಾಕಿದ. ಅವರನ್ನು ನೆನಪಿಸಿಕೊಂಡು ನನ್ನ ತಾಯಿ ಕಣ್ಣೀರಿಡುತ್ತಿದ್ದಾಳೆ. ಆ ಮಕ್ಕಳು ನಿನ್ನ ಸಮೀಪದಲ್ಲಿದ್ದಾರೆ. ಅವರನ್ನು ನಮಗೊಪ್ಪಿಸು. ನಮ್ಮ ತಾಯಿ ಅವರನ್ನು ಕಂಡು ಆನಂದಿಸಲಿ’ ಎಂದನು. ಬಲಿಯು ಒಡನೆಯೆ ಆ ಮಕ್ಕಳನ್ನು ಶ್ರೀಕೃಷ್ಣ ನಿಗೊಪ್ಪಿಸಿ, ಆತನನ್ನು ಭಕ್ತಿಯಿಂದ ಪೂಜಿಸಿ ಬೀಳ್ಕೊಟ್ಟನು. ಬಲರಾಮಕೃಷ್ಣರು ಆ ಮಕ್ಕಳನ್ನು ಕರೆತಂದು ತಾಯಿಗೊಪ್ಪಿಸಿದರು. ಅವರನ್ನು ಕಾಣುತ್ತಲೆ ದೇವಕಿದೇವಿಗಾದ ಆನಂದಕ್ಕೆ ಪಾರವೇ ಇರಲಿಲ್ಲ. ಶಿಶುಗಳಾಗಿ ಸತ್ತ ಆ ಮಕ್ಕಳು ಶಿಶು ರೂಪದಲ್ಲಿಯೇ ಇದ್ದುದರಿಂದ, ಆಕೆ ಆ ಮಕ್ಕಳನ್ನು ಆಲಿಂಗಿಸಿದಳು, ತೊಡೆಯ ಮೇಲೆ ಮಲಗಿಸಿಕೊಂಡು ಮೊಲೆಯುಣಿಸಿದಳು, ಅವರ ತಲೆಯನ್ನು ಮೂಸಿ ನೋಡಿ ಆನಂದ ಬಾಷ್ಪಗಳನ್ನು ಸುರಿಸಿದಳು. ಭಗವಂತನಾದ ಶ್ರೀಕೃಷ್ಣನು ಕುಡಿದು ಬಿಟ್ಟ ಮೊಲೆಹಾಲನ್ನು ಕುಡಿದೊಡನೆ ಅವರ ಕರ್ಮವೆಲ್ಲ ಸವೆದುಹೋಯಿತು; ಅವರಿಗೆ ಜ್ಞಾನೋದಯವಾಯಿತು; ಪೂರ್ವಜನ್ಮದಲ್ಲಿ ತಾವು ಮರೀಚಮಹರ್ಷಿಯ ಮಕ್ಕಳಾಗಿದ್ದುದನ್ನು, ಬ್ರಹ್ಮನ ಶಾಪ ದಿಂದ ಜನ್ಮಾಂತರಗಳಾದುದನ್ನೂ ಅವರು ಅರ್ಥಮಾಡಿಕೊಂಡರು. ಒಡನೆಯೆ ಅವರು ತಾಯ್ತಂದೆಗಳಿಗೂ ಬಲರಾಮಕೃಷ್ಣರಿಗೂ ಭಕ್ತಿಯಿಂದ ನಮಸ್ಕರಿಸಿ, ಎಲ್ಲರೂ ನೋಡು ತ್ತಿರುವಂತೆಯೆ ದಿವ್ಯ ವಿಮಾನವನ್ನೇರಿ ಬ್ರಹ್ಮಲೋಕಕ್ಕೆ ಹೊರಟುಹೋದರು.

ತನ್ನ ಮಕ್ಕಳು ಅತ್ತ ಹೋಗುತ್ತಲೆ, ಇತ್ತ ದೇವಕೀದೇವಿ ಶ್ರೀಕೃಷ್ಣನ ಮಾಯಾಶಕ್ತಿ ಯನ್ನು ಕಂಡು ಅಚ್ಚರಿಯಿಂದ ಬೆರಗಾದಳು. ಪ್ರತ್ಯಕ್ಷ ಭಗವಂತನೆ ಮಗನಾಗಿದ್ದರೂ ತನಗೆ ಮುಸುಗಿರುವ ಅಜ್ಞಾನವನ್ನು ನೆನೆದು ಆಕೆಗೆ ನಾಚಿಕೆಯೂ ಆಯಿತು. ಹಿತ್ತಲ ಗಿಡ ಮದ್ದಾಗಿದ್ದರೂ, ಅದು ಮದ್ದೆಂದು ಗೊತ್ತಿದ್ದರೂ, ಅದನ್ನು ಸರಿಯಾಗಿ ಬಳಸಿಕೊಳ್ಳುವ ಬುದ್ಧಿಯಿಲ್ಲದಿದ್ದರೆ ಯಾರೇನು ಮಾಡಬೇಕು?

