
\chapter{೨೯. ಪ್ರಹ್ಲಾದಕುಮಾರ}

ದಿತಿಯ ಮಕ್ಕಳಿಬ್ಬರಲ್ಲಿ ಕಿರಿಯನಾದ ಹಿರಣ್ಯಾಕ್ಷನನ್ನು ಮಹಾವಿಷ್ಣುವು ವರಾಹಾವ ತಾರದಿಂದ ಕೊಂದು ಹಾಕಿದನಷ್ಟೆ! ಇದನ್ನು ಕೇಳಿ ಹಿರಿಯನಾದ ಹಿರಣ್ಯಕಶಿಪುವಿಗೆ ಬಹು ಸಂಕಟವಾಯಿತು. ಒಂದು ಕಡೆ ಸತ್ತ ಮಗನಿಗಾಗಿ ಗೋಳಿಡುತ್ತಿರುವ ತಾಯಿ, ದಿತಿ; ಮತ್ತೊಂದು ಕಡೆ ಸತ್ತ ಗಂಡನಿಗಾಗಿ ಕಣ್ಣೀರಿಡುತ್ತಿರುವ ನಾದಿನಿ, ವೃಷದ್ಭಾನು ಮತ್ತು ಆಕೆಯ ಮಕ್ಕಳು. ಕರುಳು ಕರಗಿಸುವ ಅವರ ಅಳುವನ್ನು ಕಂಡು ಹಿರಣ್ಯಕಶಿಪುವಿನ ಕೋಪ ಉಕ್ಕಿತು. ದೇವತೆಗಳ ಪಕ್ಷಪಾತಿಯಾದ ವಿಷ್ಣುವಿನ ಮೇಲೆ ಆತನಿಗೆ ಅಪಾರವಾದ ದ್ವೇಷ ಹುಟ್ಟಿತು. ಆತನು ಕೆರಳಿ ಕೆಂಗೆಂಡವಾದ. ಆತನ ಹುಬ್ಬು ಗಂಟಿಕ್ಕಿದುವು, ಕಣ್ಣುಗಳು ಕೆಂಡವನ್ನು ಸುರಿಸಿದುವು, ಆತನ ಕೈಲಿದ್ದ ತ್ರಿಶೂಲ ಎದ್ದು ಕುಣಿಯಿತು, ಆತನ ರೋಷ ಹೆಡೆ ಬಿಚ್ಚಿ ಬುಸುಗುಟ್ಟಿತು. ತನ್ನ ಪಡೆಯ ನಾಯಕರನ್ನು ಕುರಿತು ‘ಎಲೈ ನಮುಚಿ, ಇಲ್ವಲ, ವಿಪ್ರಚಿತ್ತ ಮೊದಲಾದ ವೀರರೆ, ಕೀಳುಹಂದಿಯ ರೂಪದಿಂದ ನನ್ನ ಅಣುಗ ದಮ್ಮನನ್ನು ಕೊಂದ ಆ ಮಹಾವಿಷ್ಣುವೆಂಬುವನನ್ನು ಕೊಂದ ಹೊರತು ನನ್ನ ಜೀವಕ್ಕೆ ಶಾಂತಿ ಇಲ್ಲ. ನಾನು ಆ ಕೆಲಸವನ್ನು ನೆರವೇರಿಸುವಷ್ಟರಲ್ಲಿ ನೀವು ಭೂಲೋಕಕ್ಕೆ ಹೋಗಿ, ಅಲ್ಲಿ ಅವನ ಭಕ್ತರಾಗಿರುವವರನ್ನೆಲ್ಲ ಸದೆಬಡೆಯಿರಿ. ಅದರಲ್ಲಿಯೂ ಯಜ್ಞಸ್ವರೂಪ ನೆಂದು ಹೇಳಿಕೊಂಡು, ವಿಷ್ಣುವನ್ನು ಆರಾಧಿಸುತ್ತಾ, ದೈತ್ಯದಾನವರನ್ನು ಕೀಳಾಗಿ ಕಾಣು ತ್ತಿರುವ ಬ್ರಾಹ್ಮಣರನ್ನು ಕಂಡಲ್ಲಿ ತುಂಡರಿಸಿರಿ. ಬ್ರಾಹ್ಮಣ, ವೇದ, ವರ್ಣಾಶ್ರಮ, ಗೋಗಣ–ಇವುಗಳು ನೆಲೆಸಿರುವ ದೇಶಗಳನ್ನು ನಿರ್ಮೂಲ ಮಾಡಿರಿ’ ಎಂದು ಗುಡುಗಿ ದನು. ಮೊದಲೇ ಚೇಳು, ಅದಕ್ಕೆ ಸರ್ವಾಧಿಕಾರ ಕೊಟ್ಟರೆ ಕೇಳಬೇಕೆ? ಕ್ರೂರಸ್ವಭಾವದ ಆ ರಕ್ಕಸರು ಭೂಲೋಕವನ್ನು ಪ್ರವೇಶಿಸಿ, ಎಲ್ಲವನ್ನೂ ಅಲ್ಲೋಲ ಕಲ್ಲೋಲವಾಗಿ ಮಾಡಿದರು. ಹೊಲ ಗದ್ದೆ ತೋಟಗಳನ್ನು ಕಿತ್ತೆಸೆದರು, ಕೆರೆ ಕಟ್ಟೆಗಳನ್ನು ಒಡೆದುಹಾಕಿ ದರು, ಕೋಟೆ ಕೊತ್ತಲುಗಳನ್ನು ಕೆಡವಿ ನೆಲಸಮ ಮಾಡಿದರು, ಮನೆಗಳನ್ನು ಸುಟ್ಟರು, ಜನರನ್ನು ಹುರಿದು ತಿಂದು ಹಾಕಿದರು.

ಇತ್ತ ಹಿರಣ್ಯಕಶಿಪು ತಮ್ಮನ ಉತ್ತರಕ್ರಿಯೆಗಳನ್ನೆಲ್ಲ ಸಾಂಗವಾಗಿ ನೆರವೇರಿಸಿದನು. ತನ್ನ ತಾಯಿ ನಾದಿನಿಯರು ಸದಾ ಸಂಕಟದಲ್ಲಿ ಮುಳುಗಿರುವುದನ್ನು ಕಂಡು ಆತನಿಗೆ ‘ಅಯ್ಯೋ’ ಎನಿಸಿತು. ಮೃದು ಮಧುರವಾದ ಮಾತುಗಳಿಂದ ಅವರನ್ನು ಸಮಾಧಾನಪಡಿ ಸುತ್ತಾ “ಅಮ್ಮ, ನನ್ನ ತಮ್ಮ ಶೂರ; ಶೂರನಿಗೆ ತಕ್ಕಂತೆ ಹೋರಾಡಿ ಸತ್ತು ವೀರಸ್ವರ್ಗ ವನ್ನು ಸೇರಿದ್ದಾನೆ. ಅವನಿಗಾಗಿ ಅಳುವುದು ಬೇಡ. ಹುಟ್ಟಿದವನು ಒಂದಲ್ಲ ಒಂದು ದಿನ ಸಾಯಲೇಬೇಕು. ಸತ್ತರೇನು? ಹುಟ್ಟು ಸಾವುಗಳು ಲಿಂಗಶರೀರಕ್ಕೆ ಹೊರತು ಆತ್ಮನಿಗಲ್ಲ. ನೀರಿನಲ್ಲಿ ಕಾಣುವ ಗಿಡಮರಗಳ ನೆರಳು ಅಲ್ಲಾಡುವುದನ್ನು ಕಂಡು ಅದನ್ನು ನಿಜವೆಂದು ನಂಬುತ್ತಾರೆಯೇ? ಪೂರ್ವದಲ್ಲಿ ಸುಯಜ್ಞನೆಂಬ ರಾಜನೊಬ್ಬನಿದ್ದನಂತೆ, ಅವನು ಯುದ್ಧ ದಲ್ಲಿ ಸತ್ತುಹೋದನಂತೆ, ಆಗ ಯಮ ಎಳೆಯಹುಡುಗನ ರೂಪದಲ್ಲಿ ಅವನಿಗೆ ಕಾಣಿಸಿ ಕೊಂಡು ‘ಅಯ್ಯೋ, ಪ್ರಕೃತಿಯಿಂದ ಬಂದ ದೇಹ ಪ್ರಕೃತಿಗೆ ಹಿಂದಿರುಗಿದರೆ ಇವರು ಅಳು ತ್ತಾರಲ್ಲ!’ ಎಂದು ಹೇಳಿಕೊಂಡು ನಕ್ಕನಂತೆ! ದೇಹ ಅನಿತ್ಯವೆಂದೂ ಆತ್ಮ ನಿತ್ಯವೆಂದೂ ಅವರಿಗೆ ಆತ ಮನದಟ್ಟು ಮಾಡಿಕೊಟ್ಟನಂತೆ. ಮರಣವು ಯಾವ ಕ್ಷಣದಲ್ಲಿಯಾಗಲಿ ಬರ ಬಹುದೆಂದು ಹೇಳಿ ಆತ ಎರಡು ಹಕ್ಕಿಗಳ ಕಥೆಯನ್ನು ಹೇಳುತ್ತಾನೆ:–ಒಂದು ಅಡವಿಯಲ್ಲಿ ಎರಡು ಹಕ್ಕಿಗಳು ಗಂಡ ಹೆಂಡಿರಾಗಿ ಸುಖದಿಂದಿದ್ದುವು. ಹಕ್ಕಿಗಳ ಯಮ ನಂತಿದ್ದ ಒಬ್ಬ ಬೇಡ ಅವುಗಳಿಗಾಗಿ ಬಲೆ ಒಡ್ಡಿದ. ಹೆಣ್ಣು ಹಕ್ಕಿ ಅದರಲ್ಲಿ ಸಿಕ್ಕಿಬಿತ್ತು. ಗಂಡು ಹಕ್ಕಿ ‘ಅಯ್ಯೋ ಅಂತಹ ಪ್ರೇಮದ ಮಡದಿಯನ್ನು ಬಿಟ್ಟು ನಾನು ಹೇಗೆ ಜೀವಿ ಸಲಿ? ಎಂದು ಅಳುತ್ತಿರಲು, ಅಲ್ಲಿಯೇ ಅಡಗಿದ್ದ ಬೇಡ ಆ ಗಂಡು ಹಕ್ಕಿಯನ್ನು ಕೊಂದು ಹಾಕಿದನಂತೆ! ಈ ಕಥೆಯನ್ನು ಕೇಳಿ ಸುಯಜ್ಞನ ಸಂಬಂಧಿಗಳು ಸಮಾಧಾನ ಹೊಂದಿದ ರಂತೆ! ನಾವು ಅವರಂತೆಯೇ ಮನಸ್ಸಿಗೆ ಸಮಾಧಾನ ಹಚ್ಚಿಕೊಳ್ಳಬೇಕು” ಎಂದು ಹೇಳಿದ.

‘ವೇದಾಂತ ಹೇಳುವುದಕ್ಕೆ, ಬದನೆಯ ಕಾಯಿ ತಿನ್ನುವುದಕ್ಕೆ’ ಎಂಬ ಗಾದೆಯಂತೆ ಹಿರಣ್ಯಕಶಿಪು ತಾಯಿ ನಾದಿನಿಯರಿಗೆ ವೇದಾಂತ ಹೇಳಿದನಾದರೂ, ಅವನ ಮನಸ್ಸು ಮಾತ್ರ ವಿಷ್ಣುದ್ವೇಷದಲ್ಲಿ ಮುಳುಗಿತ್ತು. ಅವನು ಯಾರಿಗೂ ಸೋಲದಂತಹ, ಎಂತಹವ ರನ್ನೂ ಗೆಲ್ಲುವಂತಹ ಶಕ್ತಿಯನ್ನು ಗಳಿಸಿಕೊಳ್ಳಬೇಕೆಂದು ನಿಶ್ಚಯಿಸಿದನು. ಅವನು ಮಂದರ ಪರ್ವತಕ್ಕೆ ಹೋಗಿ, ಅಲ್ಲಿ ಬಹು ಉಗ್ರವಾದ ತಪಸ್ಸನ್ನು ಕೈಕೊಂಡನು. ಆತನ ದಿಟ್ಟಿ ಆಕಾಶದಲ್ಲಿ ನೆಲೆಸಿತು; ಆತನು ಕಾಲ ಹೆಬ್ಬೆರಳ ಮೇಲೆ ನಿಂತು ತೋಳುಗಳೆರಡನ್ನೂ ಮೇಲಕ್ಕೆ ಚಾಚಿದನು. ಕಾಲ ಉರುಳಿತು, ಎಷ್ಟೋ ವರ್ಷಗಳು ಕಳೆದುಹೋದುವು. ಹಿರಣ್ಯ ಕಶಿಪು ನಿಂತ ನಿಲುವಿನಿಂದ ಚಲಿಸಲಿಲ್ಲ. ಅವನ ತಲೆಯಿಂದ ತಪಸ್ಸಿನ ಬೆಂಕಿ ಹೊರ ಹೊರಟು, ಮೂರು ಲೋಕಗಳನ್ನೂ ವ್ಯಾಪಿಸಿತು. ಅದರ ಕಾವಿನಿಂದ ನದಿಗಳೂ ಸಮುದ್ರ ಗಳೂ ಕತಕತ ಕುದಿದವು, ಭೂಮಿ ನಡುಗಿತು, ಗ್ರಹಗಳು ನೆಲಕ್ಕೆ ಉದುರಿದವು, ದಿಕ್ಕುಗಳು ಹೊತ್ತಿಕೊಂಡವು. ಅದರ ಬೇಗೆಯನ್ನು ತಡೆಯಲಾರದೆ ಸ್ವರ್ಗದಲ್ಲಿದ್ದ ದೇವತೆಗಳೆಲ್ಲರೂ ಸತ್ಯ ಲೋಕಕ್ಕೆ ಓಡಿಹೋಗಿ, ಬ್ರಹ್ಮನ ಮರೆಹೊಕ್ಕರು. ಅವರ ಮೊರೆಯನ್ನು ಕೇಳಿ ಮನಸ್ಸು ಕರಗಿದ ಬ್ರಹ್ಮದೇವನು ಭೃಗು, ದಕ್ಷ ಮೊದಲಾದ ಪರಿವಾರದೊಡನೆ ಹಿರಣ್ಯ ಕಶಿಪುವಿನ ಬಳಿಗೆ ಬಂದನು. ಅಲ್ಲಿ ನೋಡುತ್ತಾನೆ. ಆ ತಪಸ್ವಿಯ ಮೇಲೆಲ್ಲ ಹುತ್ತ ಬೆಳದು ಹೋಗಿದೆ. ತಪಸ್ಸಿನ ಉರಿಯಿಂದ ಅವನಿದ್ದ ಸ್ಥಳವನ್ನು ಕಂಡು ಹಿಡಿಯಬೇಕಾಯಿತು. ಅವನ ಚರ್ಮ, ರಕ್ತ, ಮಾಂಸ ಇತ್ಯಾದಿಗಳನ್ನೆಲ್ಲ ಇರುವೆಗಳು ತಿಂದುಹೋಗಿ, ಅವನು ಕೇವಲ ಅಸ್ಥಿಪಂಜರದಂತಿದ್ದಾನೆ. ಆದರೂ ಅವನು ಸೂರ್ಯನಂತೆ ತೊಳಗಿ ಬೆಳಗು ತ್ತಿದ್ದಾನೆ. ಅವನನ್ನು ಕಂಡು ಬ್ರಹ್ಮನಿಗೆ ಪರಮಾಶ್ಚರ್ಯವಾಯಿತು. ಆತನು ಅವನನ್ನು ಕುರಿತು ‘ಅಯ್ಯಾ, ಹಿರಣ್ಯಕಶಿಪು, ನಿನಗೆ ಮಂಗಳವಾಗಲಿ! ಏಳು, ನಿನ್ನ ತಪಸ್ಸಿಗೆ ಮೆಚ್ಚಿ ನಾನು ಪ್ರತ್ಯಕ್ಷನಾಗಿದ್ದೇನೆ. ನೀನು ಕೇಳಿದ ವರಗಳನ್ನು ನಿನಗೆ ಕೊಡುತ್ತೇನೆ. ಇಂತಹ ಘೋರ ತಪಸ್ಸನ್ನು ನಾನು ಹಿಂದೆ ಎಂದೂ ಕಂಡಿಲ್ಲ’ ಎಂದನು.

ಬ್ರಹ್ಮದೇವನು ತನ್ನ ಕಮಂಡಲುವಿನಿಂದ ನೀರನ್ನು ತೆಗೆದುಕೊಂಡು ಹಿರಣ್ಯಕಶಿಪುವಿನ ಮೇಲೆ ಪ್ರೋಕ್ಷಿಸುತ್ತಲೇ ಅವನ ದೇಹವು ಬಂಗಾರದ ಬೊಂಬೆಯಂತೆ ಆಯಿತು. ಅವನು ಹುತ್ತದಿಂದ ಹೊರಕ್ಕೆ ಎದ್ದು ಬಂದನು. ಅವನು ದೇವದೇವನಾದ ಬ್ರಹ್ಮನನ್ನು ಕಂಡು ಸಂತೋಷದಿಂದ ರೋಮಾಂಚಗೊಂಡು, ಆನಂದಬಾಷ್ಪಗಳನ್ನು ಸುರಿಸುತ್ತಾ, ಗದ್ಗದ ಸ್ವರದಿಂದ ‘ಕಗ್ಗತ್ತಲೆ ತುಂಬಿದ ಜಗತ್ತಿಗೆ ನಿನ್ನ ತೇಜಸ್ಸಿನಿಂದ ಬೆಳಕನ್ನಿತ್ತು, ಸೃಷ್ಟಿ ಸ್ಥಿತಿ ಲಯಗಳಿಗೆ ಕಾರಣಭೂತನಾದ ಹೇ ಜಗದೀಶ್ವರಾ! ನಿನಗೆ ನಮಸ್ಕಾರ. ನೀನು ಸರ್ವಜ್ಞ, ಸರ್ವಶಕ್ತ, ಅಂತರ್ಯಾಮಿ, ಆದ್ಯಂತರಹಿತ, ಪರಿಪೂರ್ಣ, ಜಗದ್ರೂಪಿ, ಮನೋವಾಕ್ಕು ಗಳಿಗೆ ಅತೀತನಾದ ಅಚಿಂತ್ಯಮಹಿಮ. ನೀನು ವರ ನೀಡುವವರಲ್ಲಿ ಅಗ್ರೇಸರ. ನಾನು ಬೇಡುವ ವರಗಳನ್ನು ನೀಡುವುದೇ ದಿಟವಾದರೆ, ನಿನ್ನ ಸೃಷ್ಟಿಯೊಳಗಿರುವ ಯಾವ ಪ್ರಾಣಿ ಯಿಂದಲೂ ನನಗೆ ಮರಣವಾಗಬಾರದು; ಮನೆಯ ಹೊರಗಾಗಲಿ, ಒಳಗಾಗಲಿ ನನಗೆ ಸಾವು ಸಮನಿಸಕೂಡದು; ಹಗಲಾಗಲಿ, ರಾತ್ರಿಯಾಗಲಿ ನಾನು ಸಾಯಕೂಡದು; ಭೂಮಿಯ ಮೇಲಾಗಲಿ, ಅಂತರಿಕ್ಷದಲ್ಲಾಗಲಿ ನನಗೆ ಮೃತ್ಯುಭಯವಿರಕೂಡದು; ಯಾವ ಆಯುಧವೂ ನನ್ನನ್ನು ಕೊಲ್ಲಲು ಶಕ್ತವಾಗಕೂಡದು; ಮನುಷ್ಯ, ಮೃಗ, ದೇವದಾನವ ರಾರಿಂದಲೂ ನನಗೆ ಮರಣಭಯ ಬರಕೂಡದು’ ಎಂದು ಬೇಡಿಕೊಂಡನು. ಆಗ ಬ್ರಹ್ಮನು ‘ಅಯ್ಯಾ, ಇಂತಹ ವರವನ್ನು ಕೊಡುವುದು ಸಾಧ್ಯವಿಲ್ಲವಾದರೂ ಆ ವರ ಗಳನ್ನು ನಿನಗೆ ಕೊಟ್ಟಿದ್ದೇನೆ’ ಎಂದು ಹೇಳಿ, ತನ್ನ ಲೋಕಕ್ಕೆ ಹಿಂದಿರುಗಿದನು.

ವರಗಳನ್ನು ಪಡೆದು ಹಿಂದಿರುಗಿದ ಹಿರಣ್ಯಕಶಿಪು, ಮೂರು ಲೋಕಗಳನ್ನೂ ಜಯಿಸಿ ತನ್ನ ವಶಪಡಿಸಿಕೊಂಡನು. ದೇವ, ಮಾನವ, ಗರುಡ, ಗಂಧರ್ವಾದಿ ಸಮಸ್ತರೂ ಅವನ ಅಧೀನರಾದರು. ಅವನು ತನ್ನ ಪರಾಕ್ರಮದಿಂದ ದಿಕ್ಪಾಲಕರನ್ನು ಕೂಡ ಸೋಲಿಸಿ, ಅವರ ಅಧಿಕಾರವನ್ನು ತನ್ನ ವಶಕ್ಕೆ ತೆಗೆದುಕೊಂಡನು. ಮೂರು ಲೋಕದ ಸಿರಿಸಂಪತ್ತಿಗೂ ನೆಲೆಯಾದ ಸ್ವರ್ಗವೇ ಅವನ ರಾಜಧಾನಿಯಾಯಿತು. ಇಂದ್ರನ ಅರಮನೆಯೆ ಅವನ ವಾಸ ಸ್ಥಾನವಾಯಿತು. ರತ್ನದ ಪೀಠಗಳೇನು, ಮುತ್ತಿನ ಜಗುಲಿಗಳೇನು, ಹಾಲಿನ ನೊರೆಯಂ ತಿರುವ ಹಾಸಿಗೆಗಳೇನು, ಜಣಜಣನೆ ಕಾಲ್ಕಡಗಗಳ ದನಿಯನ್ನು ಚೆಲ್ಲುತ್ತ ಓಡಾಡುವ ಚೆಲ್ಲೆಗಂಗಳ ಸುಂದರಿಯರೇನು–ಹಿರಣ್ಯಕಶಿಪುವಿನ ಭೋಗ ಭಾಗ್ಯಕ್ಕೆ ಎಣೆಯಿಲ್ಲ, ಆತನ ಅಪ್ಪಣೆಗೆ ಇದಿರಿಲ್ಲ. ಆತನ ಭಯದಿಂದ ಗಡಗಡ ನಡುಗುತ್ತಿದ್ದ ದೇವತೆಗಳೆಲ್ಲ ಅವನ ಸೇವಕರಾದರು. ಸುಗಂಧದಿಂದ ಕೂಡಿದ ಮದ್ಯವನ್ನು ಮನಸಾರೆ ಕುಡಿದು, ಮದವೇರಿ ಅವನು ಆಜ್ಞಾಪಿಸಿದ: ‘ಇನ್ನು ಮೇಲೆ ಯಜ್ಞಯಾಗಾದಿಗಳನ್ನು ಮಾಡುವವರು ನನ್ನನ್ನೇ ಆರಾಧಿಸಬೇಕು, ನನಗೆ ಹವಿಸ್ಸನ್ನು ಅರ್ಪಿಸಬೇಕು’ ಎಂದು. ಜನ ಅದರಂತೆಯೆ ನಡೆದುಕೊಂಡರು. ಅವನು ನೆನೆದುದೆಲ್ಲ ನನಸಾಯಿತು. ಸಮುದ್ರರಾಜರು ಅವನಿಗೆ ರತ್ನಗಳನ್ನು ಕಾಣಿಕೆಯಾಗಿ ಅರ್ಪಿಸಿದರು. ಅಂತರಿಕ್ಷವು ಕೂಡ ಅವನಿಗೆ ಚಿತ್ರವಿಚಿತ್ರವಾದ ವಸ್ತುಗಳನ್ನು ಒಪ್ಪಿಸಿ ಗೌರವಿಸಿತು. ಅವನ ಕತ್ತಿಗೆ ಎದುರಿಲ್ಲದಂತಾಯಿತು. ಅವನೆದುರು ಆಶೆ ಬಡವಾಗಿ ಹೋಯಿತು.

ಹಿರಣ್ಯಕಶಿಪುವಿನ ಏಳಿಗೆಯು ಇಂದ್ರನೆ ಮೊದಲಾದ ಲೋಕಪಾಲಕರಿಗೆ ನಿಲ್ಲಲು ನೆಲೆ ಯಿಲ್ಲದಂತೆ ಮಾಡಿತು. ಅವರೆಲ್ಲರೂ ಅನನ್ಯಗತಿಕರಾಗಿ ಮಹಾವಿಷ್ಣುವನ್ನು ಮರೆ ಹೊಕ್ಕರು. ಅವರು ‘ಆತನಿರುವ ದಿಕ್ಕಿಗೆ ನಮಸ್ಕಾರ’ ಎಂದು ಹೇಳಿಕೊಂಡು, ಬಹುಕಾಲ ಆತನನ್ನು ಧ್ಯಾನಮಾಡುತ್ತಾ ಆರಾಧಿಸಿದರು. ಆಗ ಗುಡುಗಿನಂತಹ ಗಂಭೀರ ಧ್ವನಿ ಯೊಂದು ಕೇಳಿ ಬಂದಿತು: ‘ದೇವತೆಗಳೇ, ಹೆದರಬೇಡಿ! ನಿಮಗೆಲ್ಲ ಮಂಗಳವಾಗುತ್ತದೆ. ರಾಕ್ಷಸನಾದ ಹಿರಣ್ಯಕಶಿಪುವಿನಿಂದ ನಿಮಗಾಗುತ್ತಿರುವ ಹಿಂಸೆ ನನಗೆ ಗೊತ್ತಿದೆ. ಅದನ್ನು ಬಹುಬೇಗ ನಾನು ನಿವಾರಿಸುತ್ತೇನೆ. ದೇವ ಬ್ರಾಹ್ಮಣರಲ್ಲಿ ದ್ವೇಷಿಯಾದವನು ಹೆಚ್ಚುಕಾಲ ಉಳಿಯುವುದಿಲ್ಲ. ಅವನ ಮಗನಾದ ಪ್ರಹ್ಲಾದ ಮಹಾತ್ಮ, ದೈವಭಕ್ತ. ಅವನನ್ನು ಹಿರಣ್ಯಕಶಿಪು ಹಿಂಸಿಸಹೊರಡುತ್ತಾನೆ. ಆಗ ನಾನು ಅವತರಿಸಿ ಅವನನ್ನು ಕೊಲ್ಲುತ್ತೇನೆ.’ ಈ ಧ್ವನಿಯನ್ನು ಕೇಳಿದ ಲೋಕಪಾಲಕರು, ಧ್ವನಿ ಬಂದ ದಿಕ್ಕಿಗೆ ನಮಸ್ಕರಿಸಿ, ನಿಶ್ಚಿಂತೆ ಯಿಂದ ಹಿಂದಿರುಗಿದರು. 

ಹಿರಣ್ಯಕಶಿಪುವಿಗೆ ನಾಲ್ವರು ಮಕ್ಕಳು–ಪ್ರಹ್ಲಾದ, ಅನುಹ್ಲಾದ, ಸಂಹ್ಲಾದ, ಹ್ಲಾದ– ಎಂದು. ಇವರಲ್ಲಿ ಪ್ರಹ್ಲಾದನು ವಯಸ್ಸಿನಲ್ಲಿ ಮಾತ್ರವೇ ಅಲ್ಲ, ಗುಣದಲ್ಲಿಯೂ ಹಿರಿಯ. ಹಿರಣ್ಯಕಶಿಪುವು ತನ್ನ ಮಕ್ಕಳನ್ನು ವಿದ್ಯಾಭ್ಯಾಸಕ್ಕಾಗಿ ತನ್ನ ಪರೋಹಿತರಾದ ಚಂಡ, ಅಮರ್ಕರಿಗೆ ಒಪ್ಪಿಸಿದನು. ಶುಕ್ರಾಚಾರ್ಯನ ಮಕ್ಕಳಾದ ಆ ಇಬ್ಬರೂ ಸಕಲ ವಿದ್ಯಾ ಪಾರಂಗತರು. ಅವರು ತಮ್ಮ ಇತರ ಶಿಷ್ಯರೊಡನೆ ರಾಜಕುಮಾರರಿಗೂ ದಂಡನೀತಿ ಯನ್ನು ಬೋಧಿಸುತ್ತಿದ್ದರು. ಪ್ರಹ್ಲಾದನು ಗುರುಗಳ ಬೋಧನೆಯನ್ನು ಕೇಳಿದ. ‘ಇವನು ಮಿತ್ರ, ಇವನು ಶತ್ರು’ ಎಂಬುದನ್ನು ಕಲಿಸುತ್ತಿದ್ದ ಆ ವಿದ್ಯೆ ಅವನಿಗೆ ರುಚಿಸಲಿಲ್ಲ. ಕೆಲಕಾಲದ ಮೇಲೆ ಅವನ ತಂದೆ ತನ್ನ ತೊಡೆಯಮೇಲೆ ಅವನನ್ನು ಕೂಡಿಸಿಕೊಂಡು ಬಹು ಮುದ್ದಿನಿಂದ ‘ಮಗು, ನೀನೇನೇನು ಕಲಿತೆ, ಹೇಳು’ ಎಂದು ಕೇಳಿದಾಗ, ಅವನು ‘ಅಪ್ಪ, ಈ ವಿದ್ಯೆ ನನಗೆ ಸರಿ ಬೀಳಲಿಲ್ಲ; ಕತ್ತಲೆಯ ನರಕದಂತಿರುವ ಸಂಸಾರವನ್ನು ಬಿಟ್ಟು, ಅಡವಿಗೆ ಹೋಗಿ, ಶ್ರೀಹರಿಯನ್ನು ಧ್ಯಾನ ಮಾಡುವುದು ಉತ್ತಮವೆನಿಸುತ್ತದೆ, ನನಗೆ’ ಎಂದ. ಈ ಮಾತನ್ನು ಕೇಳಿ ಹಿರಣ್ಯಕಶಿಪುವಿಗೆ ಕಸಿವಿಸಿಯಾಯಿತು. ಆತನು ಗುರುಗಳನ್ನು ಕರೆಯಕಳುಹಿಸಿ ‘ಸ್ವಾಮಿ, ಯಾರೋ ಶತ್ರುಗಳ ಕಡೆಯವರು ಮರೆಯಲ್ಲಿದ್ದುಕೊಂಡು, ನನ್ನ ಮಗನ ಎಳೆಯ ಬುದ್ಧಿಗೆ ಹೊಲೆಯನ್ನು ಮೆತ್ತುತ್ತಿದ್ದಾರೆ. ನೀವು ಸ್ವಲ್ಪ ಎಚ್ಚರವಹಿಸಿ, ನನ್ನ ಮಗನನ್ನು ನೋಡಿಕೊಳ್ಳಿ’ ಎಂದು ಹೇಳಿಕಳುಹಿಸಿದನು.

ಹಿರಣ್ಯಕಶಿಪುವಿನ ಪರಮ ವೈರಿಯಾದ ಶ್ರೀಹರಿ ತಮ್ಮ ಶಿಷ್ಯನ ಆರಾಧ್ಯದೈವವೆಂದು ತಿಳಿದಾಗ ಚಂಡ ಅಮರ್ಕರಿಗೆ ಚೇಳು ಕಡಿದಂತಾಯಿತು. ಅವರು ಪ್ರಹ್ಲಾದನನ್ನು ಹತ್ತಿರ ದಲ್ಲಿ ಕೂಡಿಸಿಕೊಂಡು ‘ಮಗು, ಸುಳ್ಳಾಡಬೇಡ, ನಿಜಹೇಳು. ಇಲ್ಲಿನ ಯಾವ ಹುಡುಗ ನಿಗೂ ಇಲ್ಲದ ಈ ಬುದ್ಧಿ ನಿನಗೆಲ್ಲಿಂದ ಬಂತು? ಯಾರಾದರೂ ಬೇರೆಯವರು ಇದನ್ನು ನಿನಗೆ ಹೇಳಿಕೊಟ್ಟರೆ? ಅಥವಾ ತಾನಾಗಿಯೇ ನಿನಗೆ ಈ ಬುದ್ಧಿ ಹುಟ್ಟಿದೆಯೊ?’ ಎಂದು ಸವಿ ನುಡಿಗಳಿಂದ ಕೇಳಿದರು. ಪ್ರಹ್ಲಾದ ಹೇಳಿದ: ‘ಗುರುಗಳೆ, ನಿಮ್ಮ ಮಾತೇ ನನಗೆ ಅರ್ಥ ವಾಗುವುದಿಲ್ಲ. ‘ನಾನು–ಬೇರೆಯವರು’ ಎಂಬ ಕಲ್ಪನೆ ಬರಿಯ ಮಾಯೆ. ಶ್ರೀಹರಿಯ ಭಕ್ತನಾಗದ ಹೊರತು ಅದು ತೊಲಗುವುದಿಲ್ಲ. ನನಗೆ ಶ್ರೀಹರಿಯ ಕರುಣೆಯಿಂದ ಅದು ತೊಲಗಿದೆ. ತನಗೆ ತಾನೇ ಹೀಗಾಗಿದೆಯೇ ಹೊರತು ಇತರರಿಂದಲ್ಲ’ ಎಂದನು. ಅವನ ಉತ್ತರವನ್ನು ಕೇಳಿ ಗುರುಗಳಿಬ್ಬರು ಕಂಗೆಟ್ಟರು. ಅವರು ಮತ್ತೇನನ್ನು ಮಾಡಲೂ ತೋಚದೆ, ಕೋಪದಿಂದ ಅವನನ್ನು ಗದರಿಸುತ್ತಾ “ಎಲಾ, ನೀನು ನಮ್ಮ ರಾಕ್ಷಸವಂಶಕ್ಕೆ ಒಂದು ಕಳಂಕ. ‘ಕುಲಕ್ಕೆ ಮೃತ್ಯು ಕೊಡಲಿಯ ಕಾವು’ ಎನ್ನುವಂತೆ ಹುಟ್ಟಿದ್ದಿ. ನಿನಗೆ ಬೆತ್ತದ ಪೆಟ್ಟೆ ಉಪಾಯ” ಎಂದು ಹೇಳಿ, ಅವನಿಗೆ ಬಲವಂತವಾಗಿ ಸಾಮ, ದಾನ, ಭೇದ, ದಂಡಗಳೆಂಬ ಪಾಠಗಳನ್ನು ಕಲಿಸಿದರು. ತಾವು ಹೇಳಿಕೊಟ್ಟುದನ್ನೆಲ್ಲ ಅವನು ಕಲಿತಿರುವ ನೆಂದು ಧೈರ್ಯವಾದ ಮೇಲೆ ಅವರು ಅವನನ್ನು ಮತ್ತೊಮ್ಮೆ ತಂದೆಯ ಬಳಿಗೆ ಕಳುಹಿಸಿದರು.

ತನ್ನ ಪ್ರಯತ್ನ ಸಫಲವಾಯಿತೆಂದು ಭಾವಿಸಿದ್ದ ತಂದೆಯು ಮಗನನ್ನು ಆದರದಿಂದ ಅಪ್ಪಿಕೊಂಡು, ತೊಡೆಯ ಮೇಲೇರಿಸಿ, ನಡುನೆತ್ತಿಯನ್ನು ಮೂಸಿನೋಡಿದ ಮೇಲೆ ಮುಗು ಳ್ನಗೆಯನ್ನು ಹೊರಚೆಲ್ಲುತ್ತಾ ‘ಮಗು, ಪ್ರಹ್ಲಾದ, ಇಲ್ಲಿಯವರೆಗೆ ನೀನು ಗುರುಗಳಿಂದ ಯಾವಯಾವ ಉತ್ತಮ ವಿದ್ಯೆಗಳನ್ನು ಕಲಿತೆಯಪ್ಪ? ಅದನ್ನು ಸ್ವಲ್ಪ ಹೇಳು, ಕೇಳೋಣ’ ಎಂದ. ಆಗ ಪ್ರಹ್ಲಾದ ‘ಅಪ್ಪ, ಶ್ರೀಹರಿಯ ನಾಮ ಸಂಕೀರ್ತನೆಯನ್ನು ಕಿವಿಯಿಂದ ಕೇಳುವುದು, ನಾಲಗೆಯಿಂದ ಸಂಕೀರ್ತನೆ ಮಾಡುವುದು, ಮನಸ್ಸಿನಲ್ಲಿ ಧ್ಯಾನಮಾಡು ವುದು, ಶ್ರೀಹರಿಯ ಪಾದಸೇವೆ ಮಾಡುವುದು, ಆತನನ್ನು ಪೂಜಿಸುವುದು, ವಂದಿಸುವುದು, ಆತನ ದಾಸನಂತಿರುವುದು, ಸ್ನೇಹಿತನಂತಿರುವುದು, ತನ್ನನ್ನೇ ಆತನಿಗೆ ಒಪ್ಪಿಸುವುದು–ಈ ಒಂಬತ್ತೂ ಒಂಬತ್ತುವಿಧವಾದ ಭಕ್ತಿಗಳು. ಇದನ್ನು ನಡೆಸುವುದೇ ಅತ್ಯುತ್ತಮವಾದ ವಿದ್ಯೆ’ ಎಂದನು. ಮಗನ ಮಾತುಗಳನ್ನು ಕೇಳಿ ಹಿರಣ್ಯಕಶಿಪುವಿನ ಅಂಗಾಂಗವೆಲ್ಲವೂ ಹೊತ್ತಿ ಉರಿದಷ್ಟು ವ್ಯಥೆಯಾಯಿತು. ಆತನು ಗುರುಗಳನ್ನು ಕರೆಸಿ ‘ಎಲೈ, ನೀಚ ಬ್ರಾಹ್ಮಣರೆ, ನನ್ನ ಮಗನಿಗೆ ಎಂತಹ ದುರ್ವಿದ್ಯೆಗಳನ್ನು ಕಲಿಸಿ ಹಾಳು ಮಾಡಿರುವಿರಲ್ಲಾ! ಶತ್ರುಗಳ ಕಡೆ ಸೇರಿಕೊಂಡ ನಿಮಗೆ ನನ್ನ ಭಯ ಸ್ವಲ್ಪವೂ ಇಲ್ಲವೆಂದು ತೋರುತ್ತದೆ’ ಎಂದು ಛೀಗುಟ್ಟಿದನು. ಇದನ್ನು ಕೇಳಿ ಅವರು ಗಡಗಡ ನಡುಗುತ್ತಾ ‘ಮಹಾರಾಜ, ದಯವಿಟ್ಟು ಶಾಂತನಾಗು. ನಾವು ಇದನ್ನು ಕಲಿಸಿದವರಲ್ಲ. ಇದು ತನಗೆ ತಾನೆ ಬಂದು ದೆಂದು ರಾಜಕುಮಾರ ಹೇಳುತ್ತಿದ್ದಾನೆ. ನಮ್ಮ ಮೇಲೆ ದ್ರೋಹವನ್ನು ಹೊರಿಸಬೇಡ’ ಎಂದು ಬೇಡಿಕೊಂಡರು. ಒಡನೆಯೆ ರಾಕ್ಷಸರಾಜನ ಕೋಪ ಮಗನತ್ತ ತಿರುಗಿತು. ಆತನು ‘ಎಲಾ ನಿರ್ಭಾಗ್ಯ, ನಿನಗೆ ಈ ದುರ್ಬುದ್ಧಿ ಎಲ್ಲಿಂದ ಬಂದಿತೊ’ ಎಂದು ಗದರಿಸಿದನು. ಆದರೆ ಅದರಿಂದ ಪ್ರಹ್ಲಾದ ಸ್ವಲ್ಪವೂ ಭಯಪಡಲಿಲ್ಲ; ಪ್ರಶಾಂತನಾಗಿ ಗುರುಗಳಿಗೆ ಹೇಳಿದ ಉತ್ತರವನ್ನೆ ಅಪ್ಪನಿಗೂ ಹೇಳಿದ. ಜೊತೆಗೆ ‘ಅಪ್ಪ, ಕುರುಡರನ್ನು ಕುರುಡರು ನಡೆಸಿದರೆ ಹಳ್ಳಕ್ಕೆ ಬೀಳುವುದೇ ಗತಿ. ವೇದವೆಂಬ ಕುಣಿಕೆಗೆ ಸಿಕ್ಕಿ, ಬ್ರಾಹ್ಮಣರೆಂಬ ಹೆಸರಿನಿಂದ ಕರ್ಮ ಮಾರ್ಗದಲ್ಲಿ ಬಿದ್ದು ಒದ್ದಾಡುತ್ತಿರುವ ಈ ಗುರುಗಳಂತಹವರು ಹರಿಯ ಸನ್ನಿಧಿಯನ್ನು ಹೇಗೆ ಸೇರಿಯಾರು? ಹರಿಯ ಅನುಗ್ರಹವಿಲ್ಲದ ಜನ ಹರಿಯನ್ನು ನಿಂದಿಸುವುದರಲ್ಲಿ ಆಶ್ಚರ್ಯವೇನೂ ಇಲ್ಲ’ ಎಂದ.

ಮಗನ ಮಾತುಗಳನ್ನು ಕೇಳುತ್ತಲೆ ಹಿರಣ್ಯಕಶಿಪುವಿಗೆ ಅಂಗಾಲಿನಿಂದ ನಡುನೆತ್ತಿಯ ವರೆಗೂ ರೋಷ ಹೊತ್ತಿತು. ಆತ ಅವನನ್ನು ತೊಡೆಯಿಂದ ಕೆಳಕ್ಕೆ ತಳ್ಳಿ ‘ಎಲಾ, ಏನು ನೋಡುತ್ತೀರಿ? ಈ ಪಾಪಿ ಮಗನನ್ನು ನನ್ನ ಕಣ್ಣೆದುರಿನಿಂದ ಎಳೆದುಕೊಂಡು ಹೋಗಿ ಕೊಂದುಹಾಕಿರಿ. ಇವನು ಹುಣ್ಣು ಹುಟ್ಟಿದಂತೆ ಹುಟ್ಟಿದ್ದಾನೆ. ಹುಣ್ಣಾದ ಅಂಗವನ್ನು ಇಡೀ ದೇಹದ ರಕ್ಷಣೆಗಾಗಿ ಕತ್ತರಿಸಿಹಾಕುವಂತೆ ಈ ನೀಚನನ್ನು ಕತ್ತರಿಸಿ. ಶರೀರದಲ್ಲಿ ಹುಟ್ಟಿದ ರೋಗ ಶರೀರವನ್ನೆ ಸುಡುವಂತೆ ಈ ಪಾಪಿ ಹುಟ್ಟಿದ ವಂಶಕ್ಕೆ ಮೃತ್ಯುವಾಗಿ ದ್ದಾನೆ. ಇವನನ್ನು ವಿಷದಿಂದಲೊ, ಬೆಂಕಿಯಿಂದಲೊ, ಶೂಲದಿಂದಲೊ ತೀರಿಸಿಬಿಡಿ’ ಎಂದು ಪಕ್ಕದಲ್ಲಿದ್ದ ರಕ್ಕಸರಿಗೆ ಅಪ್ಪಣೆ ಮಾಡಿದನು. ಒಡನೆಯೆ ಅವರು ತಮ್ಮ ಕೈಲಿದ್ದ ಶೂಲಗಳಿಂದ ಅವನನ್ನು ತಿವಿದರು. ಆದರೇನು? ಶ್ರೀಹರಿಯಲ್ಲಿ ನಟ್ಟ ಮನಸ್ಸಿನಿಂದಿದ್ದ ಆ ಬಾಲಕನ ಕೂದಲು ಕೂಡ ಕೊಂಕಲಿಲ್ಲ. ಇದನ್ನು ಕಂಡು ಹಿರಣ್ಯಕಶಿಪುವಿನ ಕೋಪ ಮತ್ತಷ್ಟು ಉಲ್ಬಣವಾಯಿತು. ಅವನು ಆನೆಗಳಿಂದ ಅವನನ್ನು ತುಳಿಸಿದನು, ಹಾವುಗಳಿಂದ ಕಚ್ಚಿಸಿದನು, ಬೆಟ್ಟದಿಂದ ಕೆಳಕ್ಕೆ ನೂಕಿಸಿದನು, ವಿಷವನ್ನು ಕುಡಿಸಿದನು, ಸಮುದ್ರದಲ್ಲಿ ಹಾಕಿಸಿದನು, ಬೆಂಕಿಯಲ್ಲಿ ಬೀಳಿಸಿದನು, ಕಡೆಗೆ ಗುಂಡಿ ತೆಗೆಸಿ ಹೂಳಿಸಿದನು. ಏನಾದರೂ ಪ್ರಹ್ಲಾದ ಸಾಯಲೇ ಇಲ್ಲ.

ಪ್ರಹ್ಲಾದನನ್ನು ಕೊಲ್ಲಬೇಕೆಂದು ತಾನು ಮಾಡಿದ ಪ್ರಯತ್ನಗಳೆಲ್ಲವೂ ವ್ಯರ್ಥವಾದು ದನ್ನು ಕಂಡು ಹಿರಣ್ಯಕಶಿಪುವಿಗೆ ಅತ್ಯಾಶ್ಚರ್ಯವಾಯಿತು. ಆತನು ತನ್ನ ಮನಸ್ಸಿನಲ್ಲಿಯೇ ‘ಎಲ ಎಲ, ನನ್ನ ಶಿಕ್ಷೆಗಳನ್ನೆಲ್ಲ ಕಡೆಹಾದು ಈ ಚೋಟುದ್ದದ ಹುಡುಗ ನಿರ್ಭಯವಾಗಿರು ವನಲ್ಲಾ! ಇವನ ಮಹಿಮೆ ಅದ್ಭುತವೇ ಸರಿ. ಇವನು ಯಾರಿಗೂ ಹೆದರುವುದೂ ಇಲ್ಲ, ಏನು ಮಾಡಿದರೂ ಸಾಯುವುದೂ ಇಲ್ಲ. ಇವನೇ ನನ್ನ ಮೃತ್ಯುವಾಗುತ್ತಾನೆಯೊ ಏನೊ!’ ಎಂದು ಚಿಂತಿಸುತ್ತಾ ತಲೆಬಾಗಿ ಕುಳಿತನು. ಆಗ ಚಂಡ ಅಮರ್ಕರು ಆತನ ಬಳಿಗೆ ಬಂದು ಗುಟ್ಟಾಗಿ ‘ಪ್ರಭು, ನೀನು ಮೂರುಲೋಕಗಳನ್ನೂ ತಲ್ಲಣಗೊಳಿಸಿದ ಮಹಾವೀರ. ನೀನು ಈ ಸಣ್ಣ ವಿಷಯಕ್ಕಾಗಿ ಚಿಂತಿಸುತ್ತಾ ಕುಳಿತುಕೊಳ್ಳಬೇಕೆ? ನೀನು ಸಮಾಧಾನ ಮಾಡಿಕೊ. ಗುರುಗಳಾದ ಶುಕ್ರಾಚಾರ್ಯರು ಬರುವವರೆಗೂ ಈ ಹುಡುಗನನ್ನು ವರುಣಪಾಶದಿಂದ ಬಿಗಿಸಿ ಸೆರೆಯಲ್ಲಿಡಿಸಿರು. ಅವರು ಬಂದ ಮೇಲೆ ಮುಂದಿನದನ್ನು ಆಲೋಚಿಸೋಣ. ಹುಡುಗ ವಯಸ್ಸು ಬೆಳೆದಂತೆ ವಿವೇಕಿಯಾದಾನು, ನೋಡೋಣ!’ ಎಂದರು. ಹಿರಣ್ಯ ಕಶಿಪು ಅವರ ಮಾತಿಗೆ ಸಮ್ಮತಿಸಿ, ‘ಅಯ್ಯಾ, ಇನ್ನು ಮುಂದೆ ಇವನಿಗೆ ಸಂಸಾರಿಗಳ ನಡ ವಳಿಕೆಗಳನ್ನೂ ರಾಜಧರ್ಮವನ್ನೂ ಬೋಧಿಸಿರಿ’ ಎಂದು ಅಪ್ಪಣೆಮಾಡಿದನು.

ರಾಜನ ಅಪ್ಪಣೆಯಂತೆ ಚಂಡಾಮರ್ಕರು ಪ್ರಹ್ಲಾದನಿಗೆ ಲೌಕಿಕ ವಿದ್ಯೆಗಳನ್ನು ಕಲಿಸ ಹೊರಟರು. ಅವನಿಗೆ ಆಸಕ್ತಿಯೇನೂ ಹುಟ್ಟಲಿಲ್ಲ. ಹೀಗೆಯೇ ಕೆಲವು ಕಾಲ ಉರುಳಿತು. ಒಂದು ದಿನ ಗುರುಗಳಿಬ್ಬರೂ ಸ್ವಂತ ಕೆಲಸಕ್ಕಾಗಿ ಮನೆಗೆ ಹೋಗಿದ್ದಾರೆ; ಹುಡುಗರೆಲ್ಲ ಸ್ವಲ್ಪ ಹೊತ್ತು ಹಾಯಾಗಿ ಆಟವಾಡಿಕೊಂಡಿರೋಣವೆಂದುಕೊಂಡು, ಪ್ರಹ್ಲಾದನನ್ನೂ ಆಟಕ್ಕೆ ಕರೆದರು. ಅವನು ಮುಗುಳ್ನಗುತ್ತಾ, ಮೃದುಮಧುರವಾಗಿ ‘ಗೆಳೆಯರೆ, ಮನುಷ್ಯ ಜನ್ಮ ಸಿಗುವುದೇ ಬಹುಕಷ್ಟ. ಇದು ಸಿಕ್ಕಿದಾಗ ಇದನ್ನು ಹಾಳುಮಾಡಿಕೊಳ್ಳಬಾರದು. ನಾವು ಸುಖವಾಗಿರಬೇಕಾದರೆ ಚಿಕ್ಕಂದಿನಿಂದಲೂ ಶ್ರೀಹರಿಯ ಭಕ್ತರಾಗಬೇಕು. ನಮ್ಮ ಉದ್ಧಾರಕ್ಕೆ ಅದೊಂದೆ ದಾರಿ. ಭಗವಂತನಿಗೆ ಸಂತೋಷವಾಗುವಂತೆ ನಡೆದುಕೊಂಡರೆ ಆತ ನಮಗೆ ಏನನ್ನು ಬೇಕಾದರೂ ಕೊಡಬಲ್ಲ. ಸಾಕ್ಷಾತ್ ನಾರದ ಮಹರ್ಷಿಯೇ ನನಗೆ ಇದನ್ನು ಬೋಧಿಸಿದ. ನಾವು ಕಲಿಯುತ್ತಿರುವ ಈ ಹಾಳು ವಿದ್ಯೆಯಿಂದ ಏನು ಪ್ರಯೋ ಜನ? ಅನ್ಯಾಯವಾಗಿ ನಮ್ಮ ಆಯುಸ್ಸು ಹಾಳಾಗುತ್ತದೆಯಷ್ಟೆ’ ಎಂದು ಹೇಳಿದ. ಹುಡುಗರಿಗೆ ಆಶ್ಚರ್ಯವಾಯಿತು. ಅವರು ಚಂಡಾಮರ್ಕರನ್ನು ಹೊರತು ಬೇರೆಯ ಗುರುಗಳನ್ನು ಕಂಡಿರಲಿಲ್ಲ. ಪ್ರಹ್ಲಾದನಿಗೆ ನಾರದರು ಯಾವಾಗ ಗುರುಗಳಾಗಿದ್ದರು? ಅವರು ಪ್ರಹ್ಲಾದನನ್ನು ಕುರಿತು ‘ರಾಜಕುಮಾರ, ನೀನೂ ನಮ್ಮ ಜೊತೆಯಲ್ಲಿಯೇ ಇರುವೆ. ನಿನಗೆ ನಾರದರು ಎಲ್ಲಿ ಸಿಕ್ಕಿದರು? ಅರಮನೆಯ ಅಂತಃಪುರಕ್ಕೆ ಬಂದು ನಿನಗೆ ಬೋಧಿಸಿದರೇನು?’ ಎಂದು ಕೇಳಿದರು. ಅವರ ಸಂದೇಹನಿವಾರಣೆಗಾಗಿ ಪ್ರಹ್ಲಾದ ಹಿಂದಿನ ಕಥೆಯನ್ನು ಹೇಳಿದ.

“ಅಯ್ಯಾ ಗೆಳೆಯರೆ, ಅದೊಂದು ಕಥೆ. ಹಿಂದೆ ನಮ್ಮ ತಂದೆ ತಪಸ್ಸಿಗಾಗಿ ಮಂದರ ಪರ್ವತಕ್ಕೆ ಹೋಗಿದ್ದ. ಇದೇ ಸಮಯವೆಂದು ದೇವತೆಗಳು ನಮ್ಮ ದಾನವರ ಮೇಲೆ ದಂಡೆತ್ತಿ ಬಂದರು. ಎರಡೂ ಕಡೆಯವರಿಗೂ ಘೋರಯುದ್ಧವಾಯಿತು. ದಾನವರು ಸೋತು ದಿಕ್ಕುದಿಕ್ಕಿಗೆ ಓಡಿ ಹೋದರು. ಗೆದ್ದ ದೇವತೆಗಳು ಅರಮನೆಯನ್ನೆಲ್ಲ ಸೂರೆ ಮಾಡಿದರು. ದೇವೇಂದ್ರ ನಮ್ಮ ತಾಯಿಯನ್ನು ಕೈಸೆರೆ ಹಿಡಿದುಕೊಂಡು ಎಳೆದೊಯ್ಯು ತ್ತಿದ್ದ. ಆಕೆ ಅಳುತ್ತಾ ಅವನ ಜೊತೆಯಲ್ಲಿ ಹೋಗುತ್ತಿರುವಾಗ ದಾರಿಯಲ್ಲಿ ನಾರದರು ಎದುರಾದರು. ಅವರು ‘ಅಯ್ಯಾ ದೇವರಾಜ, ಇದೆಂತಹ ಅನ್ಯಾಯ! ಪರಸ್ತ್ರೀಯನ್ನು, ಅದರಲ್ಲಿಯೂ ನಿರಪರಾಧಿಯಾದ ಈ ಮಹಾಪತಿವ್ರತೆಯನ್ನು, ಎಳೆದುಕೊಂಡು ಹೋಗು ತ್ತಿರುವೆಯಲ್ಲ! ಮೊದಲು ಈಕೆಯನ್ನು ಬಿಟ್ಟುಬಿಡು’ ಎಂದರು. ದೇವೇಂದ್ರ ಹೇಳಿದ: ‘ಸ್ವಾಮಿ, ಈಕೆಯ ವಿಚಾರದಲ್ಲಿ ನನಗೆ ಯಾವ ಕೆಟ್ಟಭಾವನೆಯೂ ಇಲ್ಲ. ಈಕೆ ಈಗ ಗರ್ಭಿಣಿ; ಈಕೆಯ ಹೊಟ್ಟೆಯಲ್ಲಿ ಹಿರಣ್ಯಕಶಿಪುವಿನ ಪೀಳಿಗೆ ಬೆಳೆಯುತ್ತಿದೆ. ಈಕೆಯನ್ನು ನಮ್ಮ ಮನೆಗೆ ಕರೆದೊಯ್ದು, ಈಕೆ ಹೆತ್ತೊಡನೆಯೇ ಮಗುವನ್ನು ಕೊಂದು, ಈಕೆಯನ್ನು ಬಿಟ್ಟುಬಿಡುತ್ತೇನೆ’ ಎಂದ. ಆಗ ನಾರದರು ‘ಅಯ್ಯಾ, ಈಕೆಯ ಮಗ ಮಹಾಮಹಿಮ, ದೊಡ್ಡ ದೈವಭಕ್ತ, ಅವನನ್ನು ಕೊಲ್ಲುವುದು ನಿನಗೆ ಸಾಧ್ಯವಿಲ್ಲ, ಯೋಗ್ಯವೂ ಅಲ್ಲ.’ ಅದನ್ನು ಕೇಳಿದ ದೇವೇಂದ್ರ ನನ್ನ ತಾಯಿಗೆ ನಮಸ್ಕರಿಸಿ, ಆಕೆಯನ್ನು ಬಿಟ್ಟು ಬಿಟ್ಟ. ನಾರದರು ಆಕೆಯನ್ನು ತಮ್ಮ ಆಶ್ರಮಕ್ಕೆ ಕರೆದೊಯ್ದು, ನಮ್ಮ ತಂದೆ ಹಿಂದಿರುಗುವ ವರೆಗೆ ಅಲ್ಲಿಯೇ ಇಟ್ಟುಕೊಂಡಿದ್ದರು. ಆ ಸಮಯದಲ್ಲಿ ಅವರು ನಮ್ಮ ತಾಯಿಗೆ ಬೋಧಿ ಸಿದ ಭಾಗವತಧರ್ಮವನ್ನು ಗರ್ಭದಲ್ಲಿದ್ದ ನಾನು ಅರಿತುಕೊಂಡೆ. ನಮ್ಮ ತಾಯಿ ಕಾಲ ಕ್ರಮದಲ್ಲಿ ಅದನ್ನು ಮರೆತಳು. ನನಗೆ ಮಾತ್ರ ಅದು ಅಚ್ಚಳಿಯದೆ ಹೃದಯದಲ್ಲಿ ನಿಂತಿದೆ. ಅದನ್ನು ಕೇಳಿದರೆ ನಿಮಗೂ ನನ್ನಂತೆಯೆ ಜ್ಞಾನ ಹುಟ್ಟುತ್ತದೆ’ ಎಂದ.

ಪ್ರಹ್ಲಾದನ ಮಾತುಗಳಿಂದ ಅವನ ಗೆಳೆಯರ ಕುತೂಹಲ ಕೆರಳಿತು. ‘ಭಾಗವತಧರ್ಮ ವೆಂದರೇನು? ’ ಎಂದು ಅವರು ಅವನನ್ನು ಕೇಳಿದರು. ಪ್ರಹ್ಮಾದ ಹೇಳಿದ–‘ಗೆಳೆಯರೆ, ದೇವರ ಪ್ರೀತಿಯನ್ನು ಗಳಿಸುವುದಕ್ಕಾಗಿ ಹಲವಾರು ಉಪಾಯಗಳಿವೆ. ಅವುಗಳಲ್ಲಿ ಭಾಗವತಧರ್ಮ ಬಹು ಶ್ರೇಷ್ಠವಾದುದು. ಅದು ಬಹಳ ಸುಲಭವಾದುದೂ ನಿಜ. ದೇವರಲ್ಲಿ ಆಳವಾದ ಪ್ರೀತಿಯನ್ನಿಡಬೇಕು; ನಮಗೆ ದೊರೆತುದನ್ನೆಲ್ಲ ಆತನಿಗೆ ಅರ್ಪಿಸಿ ಬಿಡಬೇಕು; ಆತನನ್ನು ಪೂಜಿಸಬೇಕು; ಆತನ ಗುಣಗಳನ್ನೂ ಮಹಿಮೆಗಳನ್ನೂ ಕೊಂಡಾಡ ಬೇಕು; ಆತನ ಕಥೆಯನ್ನು ಮನಸ್ಸಿಟ್ಟು ಕೇಳಬೇಕು; ಆತನ ಮೂರ್ತಿಯನ್ನು ಧ್ಯಾನಮಾಡ ಬೇಕು; ಆತನೆ ಎಲ್ಲ ಪ್ರಾಣಿಗಳಲ್ಲಿಯೂ ನೆಲಸಿರುವನೆಂದು ಅರಿತು, ಎಲ್ಲರನ್ನೂ ಪ್ರೀತಿಸ ಬೇಕು, ಗೌರವಿಸಬೇಕು; ಸಾಧುಜನರ ಸಹವಾಸದಲ್ಲಿದ್ದುಕೊಂಡು, ಅವರಂತೆಯೇ ಸದಾ ಶ್ರೀಹರಿಯ ಸ್ಮರಣೆ ಮಾಡಬೇಕು, ಇದರಿಂದ ಮನಸ್ಸಿಗೆ ಸಂತೋಷವಾಗುತ್ತದೆ, ಶಾಂತಿ ಬರುತ್ತದೆ, ಜ್ಞಾನಮೂಡುತ್ತದೆ. ಆದ್ದರಿಂದ ನೀವೆಲ್ಲ ಒಂದೇ ಮನಸ್ಸಿನಿಂದ ಶ್ರೀಹರಿ ಯನ್ನು ಧ್ಯಾನಮಾಡಿರಿ’ ಎಂದನು. ಮಕ್ಕಳಿಗೆಲ್ಲ ಅವನ ಬೋಧನೆ ಹಿಡಿಸಿತು. ಅವರೆ ಲ್ಲರೂ ಅವನ ಹಾದಿಯನ್ನೆ ಹಿಡಿದು ಹರಿಭಕ್ತರಾದರು.

ಮಕ್ಕಳೆಲ್ಲರೂ ತಮ್ಮ ಕೈಮೀರಿ ಹೋದುದನ್ನು ಕಂಡು ಚಂಡಾಮರ್ಕರಿಗೆ ಮುಂದೋರದಂತಾಯಿತು. ಅವರು ಹಿರಣ್ಯಕಶಿಪುವಿನ ಬಳಿಗೆ ಹೋಗಿ ನಡೆದುದನ್ನೆಲ್ಲ ನಿವೇದಿಸಿದರು. ಅದನ್ನು ಕೇಳಿ ಆತನ ಕೋಪಕ್ಕೆ ಎಲ್ಲೆ ಇಲ್ಲದಂತಾಯಿತು. ಅವನು ಪ್ರಹ್ಲಾದನನ್ನು ಹಿಡಿತರಿಸಿ ‘ಎಲೆ ನೀಚನಾದ ಮನೆಹಾಳ, ನಿನ್ನನ್ನು ಕಡಿದು ತುಂಡು ತುಂಡುಮಾಡಿದರೂ ನನ್ನ ಕೋಪ ಆರದು. ನನ್ನ ಅಪ್ಪಣೆಯನ್ನು ಮೀರುವಷ್ಟು ಧೈರ್ಯ ನಿನಗೆ ಹೇಗೆ ಬಂದಿತೊ? ಯಾರೊ ನಿನಗೆ ಈ ನೀಚಬುದ್ಧಿಯನ್ನು ಹೇಳಿಕೊಟ್ಟವರು?’ ಎಂದು ಗುಡುಗಿದ. ಅವನ ಆರ್ಭಟಕ್ಕೆ ನೆಲ ನಡುಗಿದರೂ ಪ್ರಹ್ಲಾದ ಮಾತ್ರ ನಡುಗಲಿಲ್ಲ. ಅವನು ಶಾಂತವಾಗಿ ಉತ್ತರಕೊಟ್ಟ–‘ಅಪ್ಪ, ನನ್ನ ಧೈರ್ಯಕ್ಕೆ ಮೂಲ ಶ್ರೀಹರಿ. ಆತ ಬಲಶಾಲಿಗಳೆಲ್ಲರಿಗಿಂತಲೂ ಬಲಶಾಲಿ. ಸಾಮಾನ್ಯರಾದ ನಾನು, ನೀನು ಹಾಗಿರಲಿ; ಅಖಂಡ ಬ್ರಹ್ಮಾಂಡವೇ, ಸಾಕ್ಷಾತ್ ಬ್ರಹ್ಮದೇವನೂ ಕೂಡ, ಆತನ ಬಲಕ್ಕೆ ತಲೆಬಾಗು ತ್ತದೆ. ಆತ ಲೋಕೇಶ್ವರ. ಆತನ ಮೇಲೆ ನೀನು ಕತ್ತಿಕಟ್ಟುವುದು ಯಾವ ಜಾಣತನ? ಅಪ್ಪ ನಿನಗೆ ಶ್ರೀಹರಿ ಶತ್ರುವಲ್ಲ, ನಿನ್ನ ಮನಸ್ಸೆ ನಿನಗೆ ಶತ್ರು. ನಿನ್ನ ಮನಸ್ಸನ್ನು ಶ್ರೀಹರಿ ಭಕ್ತಿಯಲ್ಲಿ ನೆಲಸುವಂತೆ ಮಾಡು.’

ಮಗನ ನುಡಿಗಳನ್ನು ಕೇಳಿ ಹಿರಣ್ಯಕಶಿಪು ಕಿಡಿಕಿಡಿಯಾದ. ‘ಎಲಾ ನಿರ್ಭಾಗ್ಯ, ನೀನು ಸಾಯಬೇಕೆಂದೇ ಬಯಸುತ್ತಿರುವೆ. ಕೇಡುಗಾಲಕ್ಕೆ ನಿನಗೆ ಈ ದುರ್ಬುದ್ಧಿ ಹುಟ್ಟಿದೆ. ಅಲ್ಲವೋ ಪಾಪಿ, ಲೋಕೇಶ್ವರ ಎಂಬ ಹೆಸರು ನನಗಲ್ಲದೆ ಬೇರೊಬ್ಬನಿಗೆ ಹೇಗೆ ಸಲ್ಲು ತ್ತದೆಯೋ? ನಾನಲ್ಲದ ಆ ನಿನ್ನ ‘ಲೋಕೇಶ್ವರ’ ಎಲ್ಲಿದ್ದಾನೆಯೋ?’ ಎಂದ. ಮಗನಿಂದ ಗುರಿತಾಗುವ ಬಾಣದಂತೆ ಉತ್ತರ ಬಂತು, ‘ಎಲ್ಲೆಲ್ಲಿಯೂ ಇದ್ದಾನೆ’ಎಂದು. ಆ ಮಾತಿನ ಮೊನಚನ್ನು ತಾಳಲಾರದೆ ಆತನು ‘ದುರಾತ್ಮಾ, ಎಲ್ಲೆಲ್ಲಿಯೂ ಇರುವುದಾದರೆ ಇದಿರಿ ಗಿರುವ ಈ ಕಂಭದಲ್ಲಿ ಅವನೇಕೆ ಕಾಣುತ್ತಿಲ್ಲ? ಈಗ ನಾನು ನಿನ್ನ ತಲೆಯನ್ನು ಕತ್ತರಿಸಿ ಹಾಕುತ್ತೇನೆ. ಆ ಹರಿಯೆಂಬುವನು ಬಂದು ನಿನ್ನನ್ನು ಬಿಡಿಸಿಕೊಳ್ಳಲಿ’ ಎಂದು ಹೇಳಿ ಹಿರಿದ ಕತ್ತಿಯೊಡನೆ ಸಿಂಹಾಸನದಿಂದ ಕೆಳಕ್ಕೆ ಹಾರಿ, ಆ ಹುಡುಗನ ಬಳಿಗೆ ಓಡಿದನು. ಆಗ ಇದ್ದ ಕ್ಕಿದ್ದಂತೆ ಬ್ರಹ್ಮಾಂಡವೇ ಒಡೆದು ಹೋಗುವಂತಹ ಭಯಂಕರವಾದ ಶಬ್ದವಾಯಿತು. ಮಹಾ ಪರಾಕ್ರಮಿಯಾದ ಹಿರಣ್ಯಕಶಿಪು ಸಹ ಆ ಶಬ್ದವನ್ನು ಕೇಳಿ ನಿಟ್ಟುಬಿದ್ದ. ಸಭೆಯ ಲ್ಲಿದ್ದ ದಾನವರೆಲ್ಲ ಗಡಗಡ ನಡುಗಿದರು. ಹೇಗಿದ್ದವರು ಹಾಗೆಯೆ ಕಣ್ಣು ಬಾಯಿಗಳನ್ನು ಬಿಟ್ಟುಕೊಂಡು, ಎಲ್ಲರೂ ಚಿತ್ರದ ಪ್ರತಿಮೆಗಳಂತಾದರು. ನೋಡು ನೋಡುತ್ತಿದ್ದಂತೆ ಸಭೆಯಲ್ಲಿದ್ದ ಕಂಭ ಇಬ್ಭಾಗವಾಯಿತು, ಶ್ರೀಹರಿಯು ನರಸಿಂಹಾವತಾರದಿಂದ ಅಲ್ಲಿ ಗೋಚರಿಸಿದನು.

ನರಸಿಂಹನ ಮೂರ್ತಿ ಅದರ ಹೆಸರಿಗೆ ತಕ್ಕಂತೆ ಮನುಷ್ಯ, ಮೃಗ ಎರಡೂ ಆಗಿತ್ತು; ಅದರ ದೇಹವೆಲ್ಲ ಮಾನವ, ತಲೆ ಮಾತ್ರ ಸಿಂಹ. ಆ ಭಯಂಕರ ಮೂರ್ತಿಯನ್ನು ಕಣ್ಣೆತ್ತಿ ನೋಡುವುದು ಕೂಡ ಸಾಧ್ಯವಿರಲಿಲ್ಲ. ಅದರ ಕಣ್ಣುಗಳು ಕಾದ ಚಿನ್ನದ ರಸದಂತೆ ಹೊಳೆಯುತ್ತಿದ್ದವು. ಅದರ ತಲೆಗೂದಲೂ, ಗಡ್ಡ ಮೀಸೆಗಳೂ ನಿಮಿರಿ ನೆಟ್ಟಗೆ ನಿಂತಿ ದ್ದವು. ಮೊನಚಾದ ಕೋರೆದಾಡೆಗಳು ಕರಕರ ಶಬ್ದಮಾಡುತ್ತಿದ್ದವು. ಹೊರ ಚಾಚಿದ ಅದರ ನಾಲಗೆ ಕತ್ತಿಯಂತೆ ಬಳಕುತ್ತಿತ್ತು, ಚೂರಿಯಂತೆ ಚೂಪಾಗಿ ಹೊಳೆಯುತ್ತಿತ್ತು. ಹುಬ್ಬು ಗಂಟಿಕ್ಕಿದಂತೆ ಮುಖ ಭಯಂಕರವಾಗಿತ್ತು. ಕಿವಿಗಳು ನಿಮಿರಿ ನೆಟ್ಟಗೆ ನಿಂತಿದ್ದವು. ಬಾಯಿ ಬೆಟ್ಟದ ಗವಿಯಂತಿತ್ತು, ಮೂಗಿನ ಹೊಳ್ಳೆಗಳು ತಲೆಕೆಳಕಾದ ಬಾವಿಯಂತಿದ್ದವು. ಆಕಾಶ ವನ್ನೆ ಮುಟ್ಟುವಷ್ಟು ಎತ್ತರವಾಗಿದ್ದ ಆ ದೇಹಕ್ಕೆ ತಕ್ಕ ದಪ್ಪ ಕುತ್ತಿಗೆ, ವಿಶಾಲವಾದ ಎದೆ. ಆ ದೇಹಕ್ಕೆ ಲೆಕ್ಕವಿಲ್ಲದಷ್ಟು ಕೈಗಳು. ಮೈಯಲ್ಲಿ ಬೆಳುದಿಂಗಳಿನಂತಹ ಬಿಳಿಯ ಕೂದಲು. ಉಗುರುಗಳಂತೂ ಭಯಂಕರವಾದ ಆಯುಧಗಳೇ ಆಗಿದ್ದವು. ಆ ವಿಗ್ರಹದ ಬಳಿಗೆ ಹೋಗಲು ಯಾರಿಗೂ ಧೈರ್ಯವಾಗಲಿಲ್ಲ. ಎಲ್ಲರೂ ಬಿಡುಗಣ್ಣುಗಳಿಂದ ಅದರತ್ತ ನೋಡುತ್ತಿದ್ದರು.

ನರಸಿಂಹನ ಮೂರ್ತಿಯನ್ನು ದಿಟ್ಟಿಸಿ ನೋಡುತ್ತಾ ಹಿರಣ್ಯಕಶಿಪು ಮನಸ್ಸಿನಲ್ಲಿಯೆ ‘ಓಹೊ! ಈಗ ತಿಳಿಯಿತು. ನನ್ನ ತಮ್ಮನನ್ನು ಹಂದಿಯಾಗಿ ಕೊಂದ ಆ ಮಹಾವಿಷ್ಣುವೆ ಈಗ ಈ ರೂಪದಿಂದ ಬಂದಿದ್ದಾನೆ. ಆಗಲಿ, ನಾನು ಇವನ ಶಕ್ತಿಯನ್ನು ತೂಗಿ ನೋಡು ತ್ತೇನೆ’ ಎಂದುಕೊಂಡವನೆ, ತನ್ನ ಗದೆಯನ್ನು ಹಿಡಿದು ನರಸಿಂಹನನ್ನು ಆಕ್ರಮಿಸಿದನು. ಪತಂಗವು ದೀಪಕ್ಕೆ ಎರಗುವಂತೆ ತನ್ನ ಮೇಲೆ ಬಂದು ಬಿದ್ದ ಆ ರಕ್ಕಸನನ್ನು ನರಸಿಂಹನು ಲೀಲಾಜಾಲವಾಗಿ, ಸರ್ಪವನ್ನು ಹಿಡಿವ ಗರುಡನಂತೆ, ಕೈಲಿ ಹಿಡಿದೆತ್ತಿದನು. ಆದರೆ ರಕ್ಕಸನು ಸರ್ಪದಷ್ಟೆ ಸುಲಭವಾಗಿ ನುಣುಚಿಕೊಂಡು ಮತ್ತೆ ಆತನ ಮೇಲೆ ಎರಗಿದನು. ಒಡನೆಯೆ ನರಸಿಂಹನು ಒಮ್ಮೆ ಹೂಂಕಾರ ಮಾಡಿ, ಹಾವು ಇಲಿಯನ್ನು ಹಿಡಿಯುವಂತೆ ಅವನನ್ನು ಹಿಡಿದುಕೊಂಡು, ಸಭೆಯ ಹೊಸ್ತಿಲ ಮೇಲೆ ಆ ರಕ್ಕಸನನ್ನು ತೊಡೆಯ ಮೇಲೆ ಹಾಕಿಕೊಂಡು, ಕುಳಿತನು. ಅನಂತರ ತನ್ನ ಹರಿತವಾದ ಉಗುರುಗಳಿಂದಲೆ ಅವನ ಎದೆ ಯನ್ನು ಸೀಳಿ, ಹೊಟ್ಟೆಯನ್ನು ಬಗೆದು, ಅವನ ಕರುಳುಗಳನ್ನು ಮಾಲೆಯಾಗಿ ಕೊರಳಲ್ಲಿ ಧರಿಸಿದನು. ಮದ್ದಾನೆಯನ್ನು ಮಡುಹಿದ ಸಿಂಹದಮೇಲೆ ನರಿಗಳು ಯುದ್ಧ ಹೂಡು ವಂತೆ, ಹಿರಣ್ಯಕಶಿಪುವಿನ ಅನುಯಾಯಿಗಳು ನರಸಿಂಹನನ್ನು ಕೆಣಕ ಹೊರಡಲು ಭಯಂಕರನಾದ ಆ ಸ್ವಾಮಿಯು ತೊಡೆಯ ಮೇಲಿದ್ದ ಹೆಣವನ್ನು ಅತ್ತ ಬಿಸುಟು, ಅವರ ನ್ನೆಲ್ಲ ಕ್ಷಣಮಾತ್ರದಲ್ಲಿ ಬಡಿದು ಹಾಕಿದನು. ಹೀಗೆ ರಕ್ಕಸರನ್ನೆಲ್ಲ ಸವರಿಹಾಕಿದ ಮೇಲೆ ಆತನು ಹಿರಣ್ಯಕಶಿಪುವಿನ ಸಿಂಹಾಸನವನ್ನೇರಿ, ಕೆಕ್ಕರುಗಣ್ಣಿನಿಂದ ಸುತ್ತಲೂ ನೋಡುತ್ತಾ ಕುಳಿತನು. ಆತನ ವಿಜಯದಿಂದ ಸಂತೋಷಗೊಂಡ ದೇವತೆಗಳು ಆತನ ಮೇಲೆ ಹೂ ಮಳೆಯನ್ನು ಕರೆದರು, ಗಂಧರ್ವರು ಗಾನವನ್ನು ಮಾಡಿದರು, ಅಪ್ಸರೆಯರು ನರ್ತನ ಮಾಡಿದರು, ದೇವಲೋಕದಲ್ಲಿ ಮಂಗಳವಾದ್ಯಗಳು ಭೋರ್ಗರೆದವು, ಮೂರು ಲೋಕವೂ ನರಸಿಂಹಸ್ವಾಮಿಯನ್ನು ಸ್ತೋತ್ರಮಾಡುತ್ತಾ ಆನಂದದಿಂದ ನಲಿಯಿತು.

ಹಿರಣ್ಯಕಶಿಪುವನ್ನು ಕೊಂದ ಮೇಲೆಯೂ ನರಸಿಂಹಸ್ವಾಮಿಯ ಕೋಪ ಶಾಂತವಾಗ ಲಿಲ್ಲ. ಬ್ರಹ್ಮರುದ್ರ ಮೊದಲಾದ ದೇವತೆಗಳೆಲ್ಲರೂ ಆತನನ್ನು ಪ್ರಶಾಂತನನ್ನಾಗಿ ಮಾಡಲು ಉಪಾಯವೇನೆಂದು ಆಲೋಚಿಸಿ, ಲೋಕಮಾತೆಯಾದ ಲಕ್ಷ್ಮೀದೇವಿಯನ್ನು ಆ ಕೆಲಸಕ್ಕಾಗಿ ನೇಮಿಸಿದರು. ನರಸಿಂಹನ ಭಯಂಕರ ಮೂರ್ತಿಯನ್ನು ಕಂಡು ಆಕೆಗೂ ಗಾಬರಿಯಾಯಿತು. ಆಗ ಬ್ರಹ್ಮನು ಪ್ರಹ್ಲಾದನನ್ನು ಕುರಿತು, ‘ಮಗು, ನಿನ್ನ ತಂದೆಯ ಮೇಲೆ ಹುಟ್ಟಿದ ಕೋಪ ಈ ಸ್ವಾಮಿಯಲ್ಲಿ ಇನ್ನೂ ಆರಿಲ್ಲ. ಆತನ ಹತ್ತಿರಕ್ಕೆ ಹೋಗಲು ಎಲ್ಲರೂ ಹೆದರುತ್ತಾರೆ. ಆತನನ್ನು ಸಮಾಧಾನ ಪಡಿಸಲು ನಿನಗೆ ಮಾತ್ರ ಸಾಧ್ಯ. ಹೋಗು, ನೀನು ಆ ಕೆಲಸವನ್ನು ಮಾಡು’ ಎಂದನು. ಮಹಾನುಭಾವನಾದ ಆ ಹುಡುಗ ‘ಅಪ್ಪಣೆ’ ಎಂದು ಹೇಳಿ, ಮೆಲ್ಲನೆ ನರಸಿಂಹಸ್ವಾಮಿಯ ಬಳಿಗೆ ಹೋಗಿ, ಆತನ ಪಾದಗಳನ್ನು ಹಿಡಿದು ಅಡ್ಡಬಿದ್ದನು. ಆತನ ಕೈ ಸೋಕುತ್ತಲೆ ದೇವದೇವನ ಕೋಪ ಇಳಿಮುಖವಾಯಿತು. ಆತನು ಕರುಣೆಯಿಂದ ಬಾಲಕನ ತಲೆಯನ್ನು ಸವರುತ್ತಾ ‘ಏಳು ಮಗು, ಏಳು’ ಎಂದು ಅವನನ್ನು ಕೈಹಿಡಿದು ಮೇಲಕ್ಕೆಬ್ಬಿಸಿದನು.

ಭಗವಂತನ ಅಮೃತಹಸ್ತ ತನ್ನ ಮೈಗೆ ತಾಕುತ್ತಲೆ ಪ್ರಹ್ಲಾದನಿಗೆ ರೋಮಾಂಚ ವಾಯಿತು; ಆತನ ಕಣ್ಣುಗಳಲ್ಲಿ ಆನಂದಬಾಷ್ಪಗಳು ಸುರಿದವು; ಆತನ ದುಃಖಗಳೆಲ್ಲವೂ ಮರೆತು ಪರಮಜ್ಞಾನ ಮೂಡಿತು; ಆತನ ಬಾಯಿಂದ ಸ್ತೋತ್ರಪ್ರವಾಹ ಹರಿದುಬಂತು– ‘ಹೇ ಭಕ್ತಜನ ಕಾಮಧೇನು! ಬ್ರಹ್ಮನೇ ಮೊದಲಾದ ದೇವತೆಗಳೂ ಮಹರ್ಷಿಗಳೂ ನಿನ್ನ ಅನಂತ ಗುಣಗಳನ್ನು ಬಣ್ಣಿಸಲಾರರು, ಎಂದಮೇಲೆ ನಾನೆಷ್ಟರವನು? ಆದರೂ ನಿನ್ನ ಸ್ವಭಾವ ಉದಾರವಾದುದು; ಕೇವಲ ಪ್ರಾಣಿ ಮಾತ್ರನಾದ ಗಜೇಂದ್ರನ ಭಕ್ತಿಗೂ ಮೆಚ್ಚಿ ಮೋಕ್ಷವನ್ನು ಕೊಟ್ಟಂತಹ ದಯಾಳು, ನೀನು. ನಿನಗೆ ದೊಡ್ಡವರು ಚಿಕ್ಕವರೆಂಬ ಭೇದ ವಿಲ್ಲ; ಮೊರೆಹೊಕ್ಕವರನ್ನೆಲ್ಲ ರಕ್ಷಿಸುವೆ. ಸ್ವಾಮಿ, ನಿನ್ನ ಅವತಾರ ಲೀಲೆಗಳೆಲ್ಲ ಜಗತ್ತಿನ ರಕ್ಷಣೆಗಾಗಿ ಹೊರತು ಶಿಕ್ಷೆಗಾಗಿ ಅಲ್ಲ. ಯಾರ ಮೇಲೆ ನೀನು ಕೋಪಗೊಂಡೆಯೋ ಅವನನ್ನು ಶಿಕ್ಷಿಸಿದಂತಾಯಿತು, ಲೋಕವೆಲ್ಲವೂ ಈಗ ಸಂತೋಷಪಡುತ್ತಿದೆ, ಇನ್ನು ನಿನ್ನ ಕೋಪವನ್ನು ಬಿಟ್ಟು ಶಾಂತನಾಗು, ನಮ್ಮನ್ನು ಕಾಪಾಡು’ ಎಂದು ಬೇಡಿಕೊಂಡನು. ಭಕ್ತನ ಈ ಪ್ರಾರ್ಥನೆಯಿಂದ ಸಂತುಷ್ಟನಾದ ನರಹರಿಯು ಅವನನ್ನು ಕುರಿತು ‘ಮಗು ಪ್ರಹ್ಲಾದ, ನಾನು ನಿನ್ನ ಭಕ್ತಿಗೆ ಮೆಚ್ಚಿದ್ದೇನೆ. ಬೇಕಾದ ವರಗಳನ್ನು ಬೇಡಿಕೊ’ ಎಂದು ಹೇಳಿದನು. ದೇವರನ್ನೆ ಪ್ರತ್ಯಕ್ಷವಾಗಿ ಕಂಡು ಬ್ರಹ್ಮಜ್ಞಾನಿಯಾಗಿದ್ದ ಆ ಬಾಲಕನಿಗೆ ಇನ್ನಾವ ವರದ ಆಶೆ? ಆತ ಹೇಳಿದ: ‘ಸ್ವಾಮಿ, ಭಕ್ತಿಯೇನೂ ವ್ಯಾಪಾರವಲ್ಲ. ನಾನು ಪ್ರತಿಫಲವನ್ನು ಬಯಸುವ ಭಕ್ತನೂ ಅಲ್ಲ, ನೀನು ಭಕ್ತನಿಂದ ಆಶೆಗಳನ್ನು ನಿರೀಕ್ಷಿಸುವ ಸ್ವಾಮಿಯೂ ಅಲ್ಲ. ಆದರೂ ನನಗೆ ಬೇಕಾದ ವರವೊಂದಿದೆ. ನನ್ನ ಮನಸ್ಸಿನಲ್ಲಿ ಯಾವ ಆಶೆಯೂ ಹುಟ್ಟದಂತಹ ಆ ವರವನ್ನು ಕರುಣಿಸು’ ಎಂದನು.

ಪ್ರಹ್ಲಾದನ ಪ್ರಾರ್ಥನೆಯನ್ನು ಕೇಳಿ ನರಹರಿಗೆ ಬಹು ಸಂತೋಷವಾಯಿತು. ಆ ಬಾಲಕ ನನ್ನು ಕುರಿತು ಆತನು ‘ಮಗು, ನಿನ್ನಂತಹ ಭಕ್ತ ಯಾವುದನ್ನೂ ಬಯಸುವುದಿಲ್ಲ, ನಿಜ. ಆದರೂ ನೀನು ದೈತ್ಯದಾನವರ ರಾಜನಾಗಿ ಒಂದು ಮನ್ವಂತರ ಕಾಲ ಇಹ ಭೋಗಗಳನ್ನು ಅನುಭವಿಸುತ್ತಾ ಇರು. ಬುವಿಯವರೂ ಬಾನವರೂ ಕೊಂಡಾಡುವಂತೆ ಬಾಳಿ ಕಡೆಗೆ ನನ್ನ ಸನ್ನಿಧಿಯನ್ನು ಸೇರುವೆ. ನಿನ್ನನ್ನೂ ನನ್ನನ್ನೂ ಸ್ಮರಿಸುವವರು ಸಕಲ ಕರ್ಮಗಳಿಂದ ಮುಕ್ತರಾಗುತ್ತಾರೆ. ನೀನು ಹುಟ್ಟಿದುದರಿಂದ ನಿನ್ನ ತಂದೆ ಉದ್ಧಾರವಾದ, ನಿನ್ನ ವಂಶದ ಇಪ್ಪತ್ತೊಂದು ತಲೆಯವರೆಗಿನ ಪಿತೃಗಳೆಲ್ಲರೂ ಉದ್ಧಾರವಾದರು. ನೀನು ನನ್ನ ಭಕ್ತರಲ್ಲಿ ಅಗ್ರಗಣ್ಯ’ ಎಂದು ಹೇಳಿ, ಅವನ ಕೈಯಿಂದ ಹಿರಣ್ಯಕಶಿಪುವಿನ ಪ್ರೇತಕಾರ್ಯಗಳನ್ನು ನಡೆಸಿದಮೇಲೆ, ಅವನನ್ನು ಸಿಂಹಾಸನದಲ್ಲಿ ಕೂಡಿಸಿ, ಪಟ್ಟಗಟ್ಟಿದನು. ಅನಂತರ ತನ್ನನ್ನು ಸ್ತುತಿಸುತ್ತಿರುವ ಬ್ರಹ್ಮನನ್ನು ಕುರಿತು ಮುಗುಳ್ನಗೆಯಿಂದ ‘ಅಯ್ಯಾ, ಬ್ರಹ್ಮ! ಕ್ರೂರ ಸ್ವಭಾವದವರಿಗೆ ರಕ್ಕಸರಿಗೆ ಮನಸ್ಸಿಗೆ ಬಂದಂತೆ ವರಗಳನ್ನು ಕೊಡಬಾರದಯ್ಯಾ!’ ಎಂದು ಹೇಳಿ ಮಾಯವಾದನು. ಬ್ರಹ್ಮನೇ ಮೊದಲಾದ ದೇವತೆಗಳೂ ಪ್ರಹ್ಲಾದನನ್ನು ಆಶೀರ್ವದಿಸಿ ತಮ್ಮ ತಮ್ಮ ಲೋಕಗಳಿಗೆ ಹೊರಟುಹೋದರು. ಇತ್ತ ಪ್ರಹ್ಲಾದನು ಶ್ರೀಹರಿಯ ಅಪ್ಪಣೆಯಂತೆ ಬಹುಕಾಲ ರಾಜ್ಯವಾಳಿ, ಕಡೆಗೆ ವಿಷ್ಣುಸಾಯುಜ್ಯವನ್ನು ಪಡೆದನು.

