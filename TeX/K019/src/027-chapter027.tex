
\chapter{೨೭. ಚಿತ್ರಕೇತು}

ವೃತ್ರಾಸುರನ ಕಥೆಯನ್ನು ಓದಿದಾಗ ಒಂದು ದೊಡ್ಡ ಸಂದೇಹ ಹುಟ್ಟುವುದು ಸ್ವಾಭಾ ವಿಕ. ಅವನು ರಜೋಗುಣ ತಮೋಗುಣಗಳಿಂದ ಹುಟ್ಟಿದ ರಾಕ್ಷಸ. ಅವನ ಹುಟ್ಟಿಗೆ ಪ್ರೇರಕವಾದ ಸಂದರ್ಭಕೂಡ ದ್ವೇಷಪೂರ್ಣವಾದುದು. ಆದರೂ ಯುದ್ಧ ಮಧ್ಯದಲ್ಲಿ ಅವನು ತೋರಿದ ಸಾತ್ವಿಕಗುಣ, ದೈವಭಕ್ತಿ, ಧರ್ಮದೃಷ್ಟಿ ಆಶ್ಚರ್ಯಕರವಾದುದು. ಮಹಾಪಾಪಿಯೆಂದೂ ಲೋಕಕಂಟಕನೆಂದೂ ಪ್ರಸಿದ್ಧನಾದ ಆ ವೃತ್ರಾಸುರನಲ್ಲಿ ಅನನ್ಯವಾದ ದೈವಭಕ್ತಿ ಹೇಗೆ ಹುಟ್ಟಿತೆಂಬ ಸಂದೇಹ ಪರೀಕ್ಷಿತನಿಗೂ ಹುಟ್ಟಿತು. ಆತ ಶುಕಮಹರ್ಷಿಗಳಲ್ಲಿ ತನ್ನ ಸಂದೇಹವನ್ನು ಹೇಳಿಕೊಂಡ. ಅವರು ಆತನ ಸಂದೇಹ ನಿವಾರಣೆಗಾಗಿ ವೃತ್ರಾಸುರನ ಪೂರ್ವಜನ್ಮದ ವೃತ್ತಾಂತವನ್ನು ತಿಳಿಸಿದರು:

ಹಿಂದೆ, ಬಹುಹಿಂದೆ, ಶೂರಸೇನ ದೇಶದಲ್ಲಿ ಚಿತ್ರಸೇನನೆಂಬ ಚಕ್ರವರ್ತಿ ರಾಜ್ಯಭಾರ ಮಾಡುತ್ತಿದ್ದ. ಆತನ ಕಾಲದಲ್ಲಿ ಭೂಮಿಯೂ ಕಾಮಧೇನುವಿನಂತೆ ಬಯಸಿದುದನ್ನೆಲ್ಲ ಬೆಳೆದು ಕೊಡುತ್ತಿತ್ತು. ಆದ್ದರಿಂದ ಆತನ ಐಶ್ವರ್ಯಕ್ಕೆ ಇತಿಮಿತಿ ಇರಲಿಲ್ಲ. ದೇವರು ಆತನಿಗೆ ಎಲ್ಲವನ್ನೂ ಕರುಣಿಸಿದ್ದನು. ಕುಲ, ರೂಪು, ವಿದ್ಯೆ, ವಯಸ್ಸು, ಐಶ್ವರ್ಯ, ಅಧಿಕಾರ –ಇವುಗಳಿಗೆ ಕೋಡು ಮೂಡಿದಂತೆ ಕೀರ್ತಿ–ಎಲ್ಲವೂ ಆತನ ಬಯಕೆಯನ್ನು ಮೀರಿ ದಷ್ಟು ಇದ್ದವು. ಆತನಿಗೆ ಇದ್ದುದು ಒಂದೇ ಕೊರತೆ; ಆತನು ಮಕ್ಕಳ ಬಯಕೆಯಿಂದ ಕೋಟಿ ಹೆಣ್ಣುಗಳನ್ನು ಮದುವೆಯಾದರೂ ಒಬ್ಬಳಿಗಾಗಲಿ ಮಕ್ಕಳಾಗಲಿಲ್ಲ. ಇದರಿಂದ ಆತನು ಸಮಸ್ತ ಭೋಗಭಾಗ್ಯಗಳ ಮಧ್ಯದಲ್ಲಿಯೂ ನಿರ್ಗತಿಕನಂತೆ ಕೊರಗುತ್ತಿದ್ದನು. ಹೀಗಿರಲು ಒಮ್ಮೆ ಪೂಜ್ಯನಾದ ಅಂಗಿರಸಮಹರ್ಷಿಯು ಆತನ ಅರಮನೆಗೆ ಬಂದ. ಚಿತ್ರ ಕೇತು ಆತನನ್ನು ಆದರದಿಂದ ಇದಿರುಗೊಂಡು, ಸತ್ಕರಿಸಿ, ಸಿಂಹಾಸನದ ಮೇಲೆ ಕೂಡಿಸಿ, ತಾನು ಆತನ ಪಕ್ಕದಲ್ಲಿ ನೆಲದ ಮೇಲೆ ಕುಳಿತುಕೊಂಡನು. ಋಷಿಗೆ ರಾಜನ ಮುಖದಲ್ಲಿ ನೆಲಸಿದ್ದ ಚಿಂತೆ ಗೋಚರವಾಯಿತು. ಆತನು ರಾಜನನ್ನು ಕುರಿತು ‘ಮಹಾರಾಜ! ನೀನೂ ನಿನ್ನ ಕುಟುಂಬವರ್ಗದವರೂ ಕುಶಲವಾಗಿರುವರಷ್ಟೆ? ನಿನ್ನ ರಾಜ್ಯಭಾರಕ್ಕೆ ಸಹಾಯಕ ವಾದ ಸೈನ್ಯ, ಭಂಡಾರ ಮೊದಲಾದುವೆಲ್ಲ ತೃಪ್ತಿಕರವಾಗಿರುವುವಲ್ಲವೆ? ನಿನ್ನ ಪ್ರಜೆ ಗಳೆಲ್ಲರೂ ನಿನ್ನ ಪ್ರೀತಿಗೆ ಪಾತ್ರರಾಗಿ, ಸುಖದಿಂದಿರುವರಲ್ಲವೆ? ನಿನಗೆ ಯಾವ ಬಗೆಯ ಕೊರತೆಯೂ ಇಲ್ಲವೆಂದು ನನ್ನ ಭಾವನೆ. ಆದರೂ ನಿನ್ನ ಮುಖದಲ್ಲಿ ಚಿಂತೆಯು ಎದ್ದು ಕಾಣುತ್ತಿದೆ. ನಿನಗಾವುದರ ಕೊರತೆ?’ ಎಂದು ಕೇಳಿದನು.

ಋಷಿಯ ಪ್ರಶ್ನೆಯಿಂದ ರಾಜನ ದುಃಖ ಒತ್ತರಿಸಿ ಬಂತು. ಅದನ್ನು ಆತ ಹತ್ತೊತ್ತಿ ಕೊಂಡು, ಅತ್ಯಂತ ವಿನಯದಿಂದ ‘ಸ್ವಾಮಿ, ಪಾಪಲೇಪವಿಲ್ಲದ ಮಹಾಜ್ಞಾನಿಗಳು ನೀವು; ನಿಮಗೆ ತಿಳಿಯದುದು ಯಾವುದಿದೆ? ಆದರೂ ನೀವು ಪ್ರಶ್ನೆ ಮಾಡುತ್ತಿರುವುದರಿಂದ ಬಿನ್ನಹ ಮಾಡಿಕೊಳ್ಳತ್ತೇನೆ. ನನ್ನ ಸಂಪತ್ತು ಅಖಂಡವಾಗಿದೆ, ಬಹುಶಃ ದೇವೇಂದ್ರನೂ ನನ್ನ ಭಾಗ್ಯವನ್ನು ಗಂಡು ಕರುಬಿಯಾನು! ಆದರೇನು ಫಲ? ಹಸಿವಿನಿಂದ ಸಾಯುತ್ತಿರು ವವನಿಗೆ ಹೂ, ಗಂಧ, ಹೆಣ್ಣುಗಳ ಸುಖ ಬೇಕಾಗುತ್ತದೆಯೆ? ಮಕ್ಕಳಿಗಾಗಿ ಹಲುಬುತ್ತಿರುವ ನನಗೆ ಈ ಸಂಪತ್ತಿನಿಂದ ತೃಪ್ತಿಯಾದೀತೆ? ಆದ್ದರಿಂದ, ಹೇ ಮಹಾತ್ಮಾ! ನಾನೂ ನನ್ನ ಪಿತೃಗಳೂ ನರಕದಿಂದ ಉದ್ಧಾರವಾಗುವಂತಹ ಪುತ್ರಸಂತಾನವಾಗುವಂತೆ ನನ್ನನ್ನು ಅನುಗ್ರಹಿಸಿ ಕಾಪಾಡು’ ಎಂದು ಬೇಡಿಕೊಂಡನು. ಸಜ್ಜನನಾದ ಆ ರಾಜನ ಮಾತುಗಳನ್ನು ಕೇಳಿ ಅಂಗಿರಸನ ಮನಸ್ಸು ಕರಗಿತು. ಆತನು ರಾಜನಿಂದ ‘ಪುತ್ರಕಾಮೇಷ್ಟಿ’ ಎಂಬ ಯಾಗವನ್ನು ಮಾಡಿಸಿ, ಅದರ ಪ್ರಸಾದವನ್ನು ರಾಜನ ಹಿರಿಯ ಹೆಂಡತಿಯಾದ ಕೃತದ್ಯುತಿಗೆ ಕೊಟ್ಟನು. ಅನಂತರ ಆತನು ರಾಜನನ್ನು ಕುರಿತು ‘ಅಯ್ಯಾ, ಚಿತ್ರಕೇತು! ನಿನ್ನ ರಾಣಿಯಲ್ಲಿ ನಿನಗೊಬ್ಬ ಮಗನೇನೋ ಹುಟ್ಟುತ್ತಾನೆ. ಆದರೆ ಅವನಿಂದ ನಿನಗೆ ಸಂತೋಷವಾಗುವಷ್ಟೇ ದುಃಖವೂ ಆಗುತ್ತದೆ’ ಎಂದು ಹೇಳಿ, ಹೊರಟುಹೊದನು.

ಯಜ್ಞದ ಪ್ರಸಾದವನ್ನು ತಿಂದ ಕೃತದ್ಯುತಿ ಗರ್ಭಿಣಿಯಾಗಿ, ಒಂಬತ್ತು ತಿಂಗಳು ತುಂಬಿದ ಮೇಲೆ ಮಗನನ್ನು ಹೆತ್ತಳು. ರಾಜನ ಸಂತೋಷಕ್ಕೆ ಪಾರವಿಲ್ಲ. ಆತನು ಕೊಡುಗೈ ಯಿಂದ ದಾನಮಾಡಿ ಬ್ರಾಹ್ಮಣರನ್ನು ತೃಪ್ತಿಪಡಿಸಿದನು. ರಾಜನು ಕಲ್ಪವೃಕ್ಷದಂತೆ ಬೇಡಿದವರಿಗೆ ಬೇಡಿದುದನ್ನಿತ್ತು ಸಂತೋಷಪಡಿಸಿದನು. ಕಡುಬಡವನಿಗೆ ಕಡವರ ದೊರೆತಂತಾಗಿತ್ತು, ಚಿತ್ರಕೇತುವಿಗೆ. ಆ ಭಾಗ್ಯವನ್ನು ನೀಡಿದ ಮಡದಿ ಕೃತದ್ಯುತಿಯು ಆತನ ಪ್ರಾಣಪ್ರಿಯೆಯಾದಳು. ಗಂಡಹೆಂಡಿರು ಮಗನನ್ನು ಮಿತಿಯಿಲ್ಲದ ಪ್ರೇಮದಿಂದ ಮುದ್ದಾಡುತ್ತಿದ್ದರು. ಇದನ್ನು ಕಂಡು ರಾಜನ ಇತರ ರಾಣಿಯರಿಗೆಲ್ಲ ಹೊಟ್ಟೆಕಿಚ್ಚು ಹುಟ್ಟಿತು. ಅದಕ್ಕೆ ಸರಿಯಾಗಿ ರಾಜನಿಗೆ ಅವರಲ್ಲಿದ್ದ ಪ್ರೇಮ ಇಳಿಮುಖವಾಯಿತು. ಅವರು ಮನಸ್ಸಿನಲ್ಲೇ ಮಕ್ಕಳಿಲ್ಲದ ನಮ್ಮ ಬಾಳು ಒಂದು ಬಾಳೆ? ಬಂಜೆಯರಾದ ನಮ್ಮನ್ನು ರಾಜನೂ ಕಣ್ಣೆತ್ತಿನೋಡುತ್ತಿಲ್ಲ. ಕೇವಲ ಹೊಟ್ಟೆಹೊರೆಯುವ ಮಟ್ಟಿಗೆ ನಾವು ರಾಣಿಯರೇ ಹೊರತು, ಉಳಿದ ವಿಚಾರಗಳಲ್ಲಿ ದಾಸಿಯರಿಗಿಂತಲೂ ಕೀಳು ಎಂದು ಸಂಕಟಪಟ್ಟರು. ಕಾಲ ಕಳೆದಂತೆ ಅವರ ಅಸೂಯೆ ದ್ವೇಷವಾಗಿ ಪರಿಣಮಿಸಿತು. ಆ ದ್ವೇಷ ಅವರ ಮನಸ್ಸನ್ನು ಕಲ್ಲಿನಷ್ಟು ಕಠೋರವಾಗಿ ಮಾಡಿತು. ಒಂದು ದಿನ ಅವರೆಲ್ಲರೂ ಗುಟ್ಟಾಗಿ ಸೇರಿ ರಾಜಕುಮಾರನಿಗೆ ವಿಷವನ್ನು ಉಣಿಸಿದರು. ಆ ಮಗು ಮೈಮೇಲೆ ಎಚ್ಚರ ತಪ್ಪಿ ಹಾಸಿಗೆಯಲ್ಲಿ ಬಿದ್ದುಕೊಂಡನು.

ಹಾಸಿಗೆಯಲ್ಲಿ ಬಿದ್ದಿದ್ದ ಮಗುವನ್ನು ಕಂಡ ಕೃತದ್ಯುತಿ, ಮಗನು ನಿದ್ದೆಮಾಡುತ್ತಿರುವ ನೆಂದು ಭಾವಿಸಿ, ತನ್ನ ಕಾರ್ಯದಲ್ಲಿ ತಾನು ನಿರತಳಾಗಿದ್ದಳು. ಎಷ್ಟು ಹೊತ್ತಾದರೂ ಮಗನು ಮೇಲಕ್ಕೇಳದಿರಲು, ಆಕೆ ದಾಸಿಯನ್ನು ಕರೆದು, ಮಗುವನ್ನು ಎಬ್ಬಿಸಿ ಕರೆತರುವಂತೆ ಅಪ್ಪಣೆ ಮಾಡಿದಳು. ದಾಸಿಯು ಬಂದು ನೋಡುತ್ತಾಳೆ–ಮಗುವಿನ ಕಣ್ಣುಗುಡ್ಡೆಗಳು ಮೇಲೆ ನಾಟಿಕೊಂಡಿವೆ, ಉಸಿರು ನಿಂತುಹೋಗಿದೆ, ಮೈ ತಣ್ಣಗಾಗಿದೆ. ಅವಳು ‘ಅಯ್ಯೋ ಕೆಟ್ಟೆ,’ ಎಂದು ಗಟ್ಟಿಯಾಗಿ ಕಿರಚಿಕೊಂಡು ಮೂರ್ಛೆಯಿಂದ ನೆಲಕ್ಕೆ ಬಿದ್ದಳು. ಅದನ್ನು ಕೇಳು ತ್ತಲೇ ಕೃತದ್ಯುತಿ ಭಯದಿಂದ ಅಲ್ಲಿಗೆ ಓಡಿ ಬಂದಳು. ಸತ್ತುಬಿದ್ದಿದ್ದ ಮಗುವನ್ನು ಕಾಣುತ್ತಲೆ ಅವಳು ಕಿಟ್ಟನೆ ಕಿರಚಿಕೊಂಡು, ಸತ್ತವಳಂತೆ ನೆಲಕ್ಕೆ ಉರುಳಿದಳು. ಅಂತಃ ಪುರದ ಜನರೆಲ್ಲರೂ ಅಲ್ಲಿ ನೆರೆದರು. ವಿಷವಿಟ್ಟವರೂ ಅವರೊಡನೆ ಸೇರಿಕೊಂಡು, ಎಲ್ಲರಿಗಿಂತಲೂ ಗಟ್ಟಿಯಾಗಿ ಅಳತೊಡಗಿದರು. ಆ ಸುದ್ದಿಯನ್ನು ಕೇಳಿ ಚಿತ್ರಕೇತುವು ಅಲ್ಲಿಗೆ ಓಡಿಬಂದನು. ಆತನೂ ದುಃಖವನ್ನು ತಡೆಯಲಾರದೆ ನೆಲದಮೇಲೆ ಬಿದ್ದು ಹೊರಳಾಡುತ್ತಾ ಗಟ್ಟಿಯಾಗಿ ಅತ್ತನು. ಮೂರ್ಛೆ ಬಿದ್ದಿದ್ದ ಕೃತದ್ಯುತಿ ಮೈತಿಳಿದು ಮೇಲ ಕ್ಕೆದ್ದು, ಕಲ್ಲುಗಳು ಕೂಡ ಕರಗುವಂತೆ ಮಗನ ಗುಣಗಳನ್ನು ವರ್ಣನೆ ಮಾಡಿಕೊಂಡು ಅತ್ತಳು; ತನ್ನ ಅದೃಷ್ಟವನ್ನು ನಿಂದಿಸಿಕೊಂಡಳು; ತನಗೆ ಆ ಸ್ಥಿತಿಯನ್ನು ತಂದ ದೈವವನ್ನು ದೂರಿದಳು. ಮೇಲಕ್ಕೆದ್ದು ತನ್ನ ಹಾಲನ್ನು ಕುಡಿಯುವಂತೆ ಮಗುವನ್ನು ಬೇಡಿಕೊಂಡಳು. ಹುಚ್ಚಿಯಂತೆ ಆಡುತ್ತಿರುವ ಆಕೆಯನ್ನು ಕಂಡು ಚಿತ್ರಕೇತುವಿನ ದುಃಖ ಮತ್ತಷ್ಟು ಹೆಚ್ಚಾ ಯಿತು. ಆತನು ತನ್ನ ಸ್ಥಾನಮಾನಗಳನ್ನು ಮರೆತು ‘ಗೋಳೋ’ ಎಂದು ಗೋಳಾಡಿದನು. ಅರಮನೆಯ ಆಳುಕಾಳುಗಳೆಲ್ಲ ಗೋಳಾಡಿದರು. ಇಡೀ ಅರಮನೆಯೇ ರೋದನದ ತೌರು ಮನೆಯಾಯಿತು. ಆ ವೇಳೆಗೆ ಸರಿಯಾಗಿ ಅಂಗಿರಸ ಋಷಿ ನಾರದರೊಡನೆ ಅಲ್ಲಿಗೆ ಬಂದನು.

ದುಃಖದಲ್ಲಿ ಮುಳುಗಿದ್ದ ಚಿತ್ರಕೇತುವನ್ನು ಅಂಗಿರಸ ನಾರದರು ತಮ್ಮ ಹಿತವಚನ ಗಳಿಂದ ಮೇಲಕ್ಕೆತ್ತಿದರು. “ಮಹಾರಾಜಾ! ಇದೇನು ಮುಂಕುಮುಚ್ಚಿಕೊಂಡಿದೆ ನಿನಗೆ? ನೀನು ಏತಕ್ಕೆ ಅಳುತ್ತಿ, ಯಾರಿಗಾಗಿ ಅಳುತ್ತಿ? ಆ ಕೂಸಿಗೂ ನಿನಗೂ ಏನು ಸಂಬಂಧ? ಪ್ರವಾಹದಲ್ಲಿ ಮರಳಿನ ಹರಳುಗಳು ಪರಸ್ಪರ ಸೇರಿ ಅಗಲುವಂತೆ ಕಾಲಪ್ರವಾಹದಲ್ಲಿ ಜೀವಗಳು ಗಂಡ–ಹೆಂಡತಿ, ತಂದೆ-ಮಗ ಎಂಬ ಸಂಬಂಧದೊಡನೆ ಕ್ಷಣಕಾಲ ಸೇರಿದ್ದು ಅಗಲುತ್ತಾರೆ. ಇದು ಈಶ್ವರಸಂಕಲ್ಪ. ಇದಕ್ಕೆ ಅತ್ತರೆ ಹೇಗೆ? ದೇಹ ಅಶಾಶ್ವತವೆಂದು ಯಾರಿಗೆ ಗೊತ್ತಿಲ್ಲ? ಈ ಮಗು ಹುಟ್ಟುವುದಕ್ಕೆ ಮುಂಚೆ ಆ ದೇಹವಿತ್ತೆ? ಈಗ ಅದು ಸತ್ತುಹೋಗಿದೆ; ಇಲ್ಲಿಂದ ಮುಂದೆ ಅದರ ದೇಹವಿರುತ್ತದೆಯೆ? ನನ್ನ, ನಿನ್ನ, ಎಲ್ಲರ ದೇಹವೂ ಅಷ್ಟೆ ಎಂದಮೇಲೆ ಅಶಾಶ್ವತವಾದ ದೇಹಕ್ಕಾಗಿ ಅಳಬೇಕಾದ್ದಿಲ್ಲ. ಇನ್ನು ಪ್ರಾಣ ಕ್ಕಾಗಿ ಅಳುತ್ತೀಯಾ? ಏಕೆ ಅಳಬೇಕು? ಜೀವಕ್ಕೆ ಸಾವೇ ಇಲ್ಲ. ಕಪ್ಪೆಯ ಚಿಪ್ಪನ್ನು ಬೆಳ್ಳಿ ಯೆಂದು ಭ್ರಾಂತಿಪಟ್ಟುಕೊಳ್ಳುವಂತೆ, ಈ ದೇಹವನ್ನೇ ಆತ್ಮವೆಂದು ಭ್ರಮಿಸುವ ಅವಿ ವೇಕಿಗಳು ಮಾತ್ರ ಅದಕ್ಕಾಗಿ ಅಳಬೇಕು, ಅಷ್ಟೆ!” ಎಂದನು. ಅದನ್ನು ಕೇಳಿದ ಚಿತ್ರಕೇತು ವಿಗೆ ಸ್ವಲ್ಪ ವಿವೇಕ ಹುಟ್ಟಿತು. ಆತನು ಅಳುವನ್ನು ನಿಲ್ಲಿಸಿದನು. ಆವರೆಗೆ ತನಗೆ ವಿವೇಕ ಹೇಳಿದವರಾರೆಂಬುದೇ ತಿಳಿದಿರಲಿಲ್ಲ. ಆತ ಅವರ ಕಡೆ ತಿರುಗಿ ‘ಮಹಾನುಭಾವರೇ, ನೀವಾರು? ಅಜ್ಞಾನದ ಕತ್ತಲೆಯಲ್ಲಿ ಮುಳುಗಿರುವ ನನಗೆ ಬೆಳಕನ್ನು ತೋರಿ, ಉದ್ಧರಿಸಿರಿ’ ಎಂದು ಬೇಡಿಕೊಂಡನು.

ರಾಜನ ಮಾತುಗಳಿಗೆ ಅಂಗಿರಸಋಷಿಯೇ ಉತ್ತರವಿತ್ತ–‘ಅಯ್ಯಾ, ಇಷ್ಟು ಬೇಗ ನನ್ನನ್ನು ಮರೆತೆಯಾ? ನಿನಗೆ ಮಗನಾಗುವಂತೆ ಯಾಗ ಮಾಡಿಸಿದವನು ನಾನೇ ಅಲ್ಲವೆ? ನನ್ನ ಹೆಸರು ಅಂಗಿರಸ. ಇಗೋ ಈತ ಬ್ರಹ್ಮನ ಮಗನಾದ ನಾರದಮಹರ್ಷಿ. ನೀನು ಮಗನಿಗಾಗಿ ಅಳುವುದನ್ನು ಕೇಳಿಯೇ ನಾವಿಲ್ಲಿಗೆ ಬಂದುದು. ನೀನು ಈಗ ಬೇಡುತ್ತಿರುವ ಜ್ಞಾನದ ಬೆಳಕನ್ನು ನೀಡುವುದಕ್ಕಾಗಿಯೇ ನಾನು ಹಿಂದೆ ಬಂದಿದ್ದುದು. ಆದರೆ ನೀನು ಆಗ ಮಗನನ್ನು ಬೇಡಿದೆ; ನಾನು ಅದನ್ನು ಕೊಟ್ಟೆ. ಈಗ ನಿನಗೆ ಮಗನಿಂದ ಬರುವ ಸುಖ ಎಷ್ಟೆಂಬುದು ಗೊತ್ತಾಯಿತಷ್ಟೆ! ಇದರಂತೆಯೇ ನಿನ್ನ ರಾಜ್ಯ, ಐಶ್ವರ್ಯ, ಹೆಂಡಿರು, ಸಕಲ ಸಂಪತ್ತು ಎಲ್ಲವೂ ಕ್ಷಣಕಾಲವಿದ್ದು ಅಗಲತಕ್ಕುವೇ. ಪ್ರಾರಂಭದಲ್ಲಿ ಎಲ್ಲವೂ ಸುಖ, ಕಡೆಯಲ್ಲಿ ಎಲ್ಲವೂ ದುಃಖ. ಇದೆಲ್ಲವೂ ಒಂದು ಕನಸು. ಆದ್ದರಿಂದ ಇದಕ್ಕಾಗಿ ಅಳು ವುದು ಬೇಡ. ಹೆಂಡಿರು ಇತ್ಯಾದಿ ಮೋಹಗಳನ್ನು ಬಿಟ್ಟು ಶಾಂತಚಿತ್ತನಾಗು’ ಎಂದನು. 

ನಾರದಮಹರ್ಷಿ ಅಂಗಿರಸನ ಮಾತುಗಳನ್ನು ಮುಂದುವರಿಸುತ್ತಾ ‘ಅಯ್ಯಾ ಮಹಾ ರಾಜ! ನಿನಗೆ ನಾನೊಂದು ಮಹಾಮಂತ್ರವನ್ನು ಉಪದೇಶಿಸುತ್ತೇನೆ. ಆ ಮಂತ್ರಜಪ ವನ್ನು ಕೈಕೊಂಡರೆ ಏಳು ದಿನಗಳೊಳಗಾಗಿ ನಿನಗೆ ಸಂಕರ್ಷಣನು ಪ್ರತ್ಯಕ್ಷನಾಗುತ್ತಾನೆ’ ಎಂದು ಹೇಳಿ, ಆ ರಾಜನ ಮೋಹವನ್ನು ಬೇರುಸಹಿತ ಕಿತ್ತುಹಾಕುವುದಕ್ಕಾಗಿ, ತನ್ನ ಯೋಗಶಕ್ತಿಯಿಂದ ಸತ್ತ ಮಗುವನ್ನು ಜೀವಂತನನ್ನಾಗಿ ಮಾಡಿದನು. ಆ ಮಗುವನ್ನು ಕುರಿತು ಆತನು ‘ನೋಡಪ್ಪಾ, ನಿನ್ನ ತಾಯಿತಂದೆಗಳು ನಿನಗಾಗಿ ಅಳುತ್ತಿದ್ದಾರೆ. ಅವರನ್ನು ಸಮಾಧಾನ ಮಾಡು. ನೀನು ಮುಂದೆ ನಿಮ್ಮ ತಂದೆಯ ಸಿಂಹಾಸನವನ್ನು ಹತ್ತಿ, ಚಕ್ರವರ್ತಿಯ ಪದವಿಯನ್ನು ಅನುಭವಿಸು’ ಎಂದನು. ಆದರೆ ಆ ಮಗು ‘ಸ್ವಾಮಿ, ನಾರದರೆ! ನಾನು ನನ್ನ ಪುಣ್ಯಪಾಪಗಳಿಗೆ ತಕ್ಕಂತೆ ಹಿಂದೆ ಎಷ್ಟು ಜನ್ಮಎತ್ತಿರುವನೋ, ಮುಂದೆಷ್ಟು ಎತ್ತಬೇಕೋ! ಒಂದೊಂದು ಜನ್ಮಕ್ಕೂ ಒಬ್ಬೊಬ್ಬ ತಾಯಿ ತಂದೆ, ಬಂಧು ಬಳಗ–ಈವರೆಗೆ ಅಂತಹರು ಎಷ್ಟು ಆಗಿ ಹೋದರೊ! ಆಯಾ ದೇಹದಲ್ಲಿರುವಾಗ ಆಯಾದೇಹಸಂಬಂಧಿಗಳ ಮಮತೆ! ನಾನೀಗ ಈ ದೇಹದಿಂದ ಕಳಚಿಕೊಂಡಿದ್ದೇನೆ. ಶುದ್ಧಜೀವನಿಗೆ ಯಾವ ಸಂಬಂಧವೂ ಇಲ್ಲ. ನನಗೇಕೆ ಮತ್ತೆ ಇವರ ಸಂಬಂಧ?’ ಎಂದು ಹೇಳಿ, ಆ ಜೀವ ದೇಹವನ್ನು ಬಿಟ್ಟು ಹೊರಟುಹೋದನು. ಇದನ್ನು ಪ್ರತ್ಯಕ್ಷವಾಗಿ ಕಂಡು ಕೇಳಿದಮೇಲೆ ರಾಜನ ಪುತ್ರಮೋಹವಷ್ಟೇ ಅಲ್ಲ, ಎಲ್ಲ ಮೋಹವೂ ಹಾರಿಹೋಯಿತು. ರಾಜಕುಮಾರನನ್ನು ಕೊಂದ ರಾಣಿಯರಿಗೂ ವಿವೇಕ ಉದಿಸಿತು. ಅವರು ಯಮುನಾ ನದಿಯ ದಡವನ್ನು ಸೇರಿ ತಮ್ಮ ಪಾಪಕ್ಕೆ ಪ್ರಾಯಶ್ಚಿತ್ತ ಮಾಡಿಕೊಂಡರು. ರಾಜನು ವೈರಾಗ್ಯಪರನಾಗಿ, ಯಮುನೆಯಲ್ಲಿ ಸ್ನಾನಮಾಡಿ, ನಾರದ ಋಷಿಯ ಬಳಿಗೆ ಬಂದು, ಮಂತ್ರವನ್ನು ಉಪದೇಶಿಸುವಂತೆ ಬೇಡಿದನು. 

ನಾರದನು ಚಿತ್ರಕೇತುವಿಗೆ ಮಂತ್ರೋಪನಿಷತ್ತೆಂಬ ವಿದ್ಯೆಯನ್ನು ಉಪದೇಶಿಸಿ, ಅಂಗಿ ರಸ್ಸನೊಡನೆ ಬ್ರಹ್ಮಲೋಕಕ್ಕೆ ಹಿಂದಿರುಗಿದನು. ಅನಂತರ ಚಿತ್ರಕೇತುವು ಕೇವಲ ನೀರನ್ನು ಕುಡಿದುಕೊಂಡಿರುತ್ತಾ, ಏಳುದಿನಗಳವರೆಗೆ ಶಾಂತಮನಸ್ಸಿನಿಂದ ಆ ಮಂತ್ರವನ್ನು ಜಪಿಸುತ್ತಿದ್ದನು. ಏಳನೆಯ ದಿನ ಮುಗಿಯುವ ವೇಳೆಗೆ ಆತನಿಗೆ ವಿದ್ಯಾಧರ ಪದವಿ ಪ್ರಾಪ್ತ ವಾಯಿತು. ಆತನೀಗ ಮನಬಂದ ಕಡೆ ಹೋಗುವ ಶಕ್ತಿಯನ್ನು ಪಡೆದವನಾಗಿದ್ದನು. ಆದ್ದರಿಂದ ಆತ ನೇರವಾಗಿ ಪಾತಾಳದಲ್ಲಿದ್ದ ಸಂಕರ್ಷಣಮೂರ್ತಿಯ ಬಳಿಗೆ ಹೋದನು. ಆ ದಿವ್ಯಮೂರ್ತಿಯನ್ನು ಕಾಣುತ್ತಲೆ ಆತನ ಪಾಪಗಳೆಲ್ಲ ನೀಗಿ, ಮನಸ್ಸು ಶುದ್ಧವಾಯಿತು. ಆತನು ತನ್ನ ಆನಂದಬಾಷ್ಪಗಳಿಂದ ಸಂಕರ್ಷಣಸ್ವಾಮಿಯ ಪಾದಗಳನ್ನು ತೊಳೆಯುತ್ತ ನಮಸ್ಕರಿಸಿದನು. ಆತನ ಮೈಯಲ್ಲಿ ಪುಳಕಗಳೆದ್ದವು, ಕಂಠ ಗದ್ಗದವಾಯಿತು. ಕೆಲಕಾಲದ ಮೇಲೆ ತನ್ನ ಮನಸ್ಸನ್ನು ಹತೋಟಿಗೆ ತಂದುಕೊಂಡು, ಆತನು ದೇವದೇವನಾದ ಸಂಕರ್ ಷಣನನ್ನು ಅನನ್ಯ ಭಕ್ತಿಯಿಂದ ಸ್ತೋತ್ರಮಾಡಿದನು. ಅದನ್ನು ಕೇಳಿ ಮೆಚ್ಚಿದ ಸ್ವಾಮಿಯು ‘ಅಯ್ಯಾ, ನಾರದ ಅಂಗಿರಸರ ಉಪದೇಶವೇ ನನ್ನ ಉಪದೇಶವೆಂದು ತಿಳಿ. ಈ ಜಗತ್ತೆಲ್ಲವೂ ನಾನೆ. ಆದ್ದರಿಂದ ಅವರು ನನ್ನಿಂದ ಬೇರೆಯೇನೂ ಅಲ್ಲ. ಅಷ್ಟೇ ಅಲ್ಲ; ಮಾತು ಮತ್ತು ಮಾತಿನಿಂದ ಸೂಚಿತವಾದ ವಸ್ತು–ಅವೆರಡೂ ನನ್ನ ಶರೀರವೇ; ನನ್ನ ಸ್ವರೂಪವನ್ನು ಅರ್ಥಮಾಡಿಕೊಳ್ಳಬೇಕಾದರೆ ಸುಷುಪ್ತಿಯಲ್ಲಿ ಮಾತ್ರ ಸಾಧ್ಯ. ಆಗ ಜೀವ ನಿಗೆ ಮನಸ್ಸಿನ ಮತ್ತು ಇಂದ್ರಿಯಗಳ ವ್ಯಾಪಾರವಿಲ್ಲ, ಕುಲ ಜಾತಿಗಳ ಭೇದವಿಲ್ಲ; ಆಗ ಅತೀಂದ್ರಿಯವಾದ ಆನಂದವನ್ನು ಅನುಭವಿಸುವ ಜೀವನೇ ಪರಬ್ರಹ್ಮನಾದ ನಾನೆಂದು ತಿಳಿ’ ಎಂದು ಹೇಳಿ, ಮನುಷ್ಯನಾಗಿಯೇ ಮೋಕ್ಷವನ್ನು ಪಡೆಯಬೇಕಾದುದುರಿಂದ, ಭಕ್ತಿ ಯೋಗದಿಂದ ಮೋಕ್ಷವನ್ನು ಗಳಿಸಿಕೊಳ್ಳುವಂತೆ ಆತನಿಗೆ ಉಪದೇಶಮಾಡಿ, ಮಾಯ ವಾದನು.

ಭಗವಂತನಾದ ಸಂಕರ್ಷಣನು ಕಣ್ಮರೆಯಾಗುತ್ತಲೆ ಚಿತ್ರಕೇತುವು ಆತ ಕಣ್ಮರೆಯಾದ ದಿಕ್ಕಿಗೆ ನಮಸ್ಕಾರಮಾಡಿ, ಆಕಾಶದಲ್ಲಿ ತನ್ನ ಮನ ಬಂದ ಕಡೆ ಸಂಚಾರ ಮಾಡುತ್ತಾ ಹೊರಟನು. ಎಷ್ಟೋ ಕಾಲದವರೆಗೆ ಆತನು ಲೋಕ ಲೋಕಾಂತರಗಳಲ್ಲೆಲ್ಲಾ ಸಂಚರಿಸು ತ್ತಿದ್ದು, ಒಮ್ಮೆ ಕೈಲಾಸದ ಬಳಿಗೆ ಬಂದನು. ಅಲ್ಲಿ ಪರಶಿವನು ಸಿದ್ಧಚಾರಣರ ಮಧ್ಯದಲ್ಲಿ ಸಿಂಹಾಸನದ ಮೇಲೆ ಮಂಡಿಸಿದ್ದನು. ಆತನ ಎಡತೊಡೆಯ ಮೇಲೆ ಪಾರ್ವತಿ ಕುಳಿತಿ ದ್ದಳು. ಶಿವನು ತನ್ನ ಎಡತೋಳನ್ನು ಆಕೆಯ ಭುಜದ ಮೇಲೆ ಹಾಕಿ, ಆಕೆಯನ್ನು ಅಪ್ಪಿ ಕೊಂಡನು. ಇದನ್ನು ಕಂಡ ಚಿತ್ರಕೇತು ಪಕಪಕ ಗಟ್ಟಿಯಾಗಿ ನಕ್ಕು, ಎಲ್ಲರಿಗೂ ಕೇಳು ವಂತೆ ಗಟ್ಟಿಯಾಗಿ ‘ಈತ ಲೋಕಗುರುವಂತೆ! ಜಗತ್ತಿಗೆಲ್ಲ ಉಪದೇಶಮಾಡುತ್ತಾನಂತೆ! ಕೇವಲ ಪಾಮರರು ಕೂಡ ತಮ್ಮ ಹೆಂಡಿರೊಡನೆ ಸರಸಸಲ್ಲಾಪ ಮಾಡುವಾಗ ಏಕಾಂತ ದಲ್ಲಿರುವುದುಂಟು. ಈ ಶಿವನು ಎಲ್ಲರ ಇದಿರಿನಲ್ಲಿಯೇ ತನ್ನ ಹೆಂಡತಿಯನ್ನು ಅಪ್ಪಿ ಕೊಂಡಿರುವನಲ್ಲಾ! ಸ್ವಲ್ಪವಾದರೂ ನಾಚಿಕೆ ಹೇಚಿಕೆಯೆಂಬುದು ಇರಬೇಡವೇ’ ಎಂದನು. ಇದನ್ನು ಕೇಳಿ ಪಾರ್ವತಿಗೆ ತಡೆಯಲಾರದಷ್ಟು ಕೋಪ ಉಕ್ಕಿತು. ಆಕೆ ಕೆಂಗಣ್ಣಿನಿಂದ ಅವನತ್ತ ನೋಡುತ್ತ ‘ಓಹೋ, ನಾವು ನಾಚಿಕೆ ಇಲ್ಲದವರು, ಇವನು ನಮ್ಮನ್ನು ದಂಡಿ ಸುವ ಅಧಿಕಾರಿ! ಅಬ್ಬಾ ಇವನ ಅಹಂಕಾರವೇ! ಬ್ರಹ್ಮಾದಿ ದೇವತೆಗಳೆಲ್ಲರೂ ಸ್ತೋತ್ರ ಮಾಡುವ ಪರಮೇಶ್ವರನನ್ನು ಈ ಕ್ಷತ್ರಿಯಾಧಮ ನಿಂದಿಸುತ್ತಿರುವನಲ್ಲ! ಈ ನೀಚನಿಗೆ ತಕ್ಕ ಶಿಕ್ಷೆಯನ್ನು ಮಾಡಬೇಕು’ ಎಂದುಕೊಂಡು ‘ಎಲ, ನೀಚ ವಿದ್ಯಾಧರ, ನಿನಗೆ ರಾಕ್ಷಸ ಜನ್ಮ ಬರಲಿ! ಮಹಾನುಭಾವರನ್ನು ನಿಂದಿಸಿದರೆ ಆಗುವ ಶಿಕ್ಷೆಯೇನೆಂಬುದನ್ನು ಅರ್ಥ ಮಾಡಿಕೊ’ ಎಂದು ಶಪಿಸಿದಳು. 

ಪಾರ್ವತಿಯ ಶಾಪವನ್ನು ಕೇಳುತ್ತಲೆ ಚಿತ್ರಕೇತು ತಾನೇರಿದ್ದ ವಿಮಾನದಿಂದ ಕೆಳಕ್ಕೆ ಇಳಿದು ಬಂದು, ಆಕೆಗೆ ನಮಸ್ಕರಿಸಿ ‘ತಾಯಿ, ನಿನ್ನ ಶಾಪವನ್ನು ನಾನು ಸಂತೋಷದಿಂದ ಸ್ವೀಕರಿಸುತ್ತೇನೆ. ಅದು ನಿನ್ನ ಅನುಗ್ರಹವೆಂದು ನನ್ನ ಭಾವನೆ. ಅಮ್ಮ, ನನ್ನ ಪ್ರಾಚೀನ ಕರ್ಮ ಇನ್ನೂ ಸ್ವಲ್ಪ ಉಳಿದಿದೆ, ಅದು ಮೋಕ್ಷಕ್ಕೆ ಅಡ್ಡಿಯಾಗಿತ್ತು, ನಿನ್ನ ಶಾಪದಿಂದ ಆ ಕರ್ಮಶೇಷ ಕಳೆದುಹೋಗುವುದಕ್ಕೆ ಸಹಾಯಕವಾಯಿತು. ಮಾಯಾಮಯವಾದ ಈ ಸಂಸಾರದಲ್ಲಿ ಶಾಪವಾದರೇನು, ಅನುಗ್ರಹವಾದರೇನು? ಸ್ವರ್ಗವಾದರೇನು, ನರಕವಾದ ರೇನು? ಸುಖವಾದರೇನು, ದುಃಖವಾದರೇನು? ಎಲ್ಲವೂ ಒಂದೆ. ಸುಖದ ಕೊನೆ ದುಃಖ, ದುಃಖದ ಕೊನೆ ಸುಖ. ಆದ್ದರಿಂದ ನಿನ್ನಲ್ಲಿ ನಾನು ಶಾಪ ವಿಮೋಚನೆಗಾಗಿ ಬೇಡುವುದಿಲ್ಲ. ಆದರೆ ನನ್ನ ಕಟುನುಡಿಗಳಿಗಾಗಿ ನನ್ನನ್ನು ಕ್ಷಮಿಸೆಂದು ಬೇಡುತ್ತೇನೆ’ ಎಂದು ಹೇಳಿ, ಪಾರ್ವತಿ ಪರಮೇಶ್ವರರನ್ನು ಸಮಾಧಾನಗೊಳಿಸಿದಮೇಲೆ ತನ್ನ ವಿಮಾನವೇರಿ ಹೊರಟು ಹೋದನು.

ಚಿತ್ರಕೇತು ಹೊರಟುಹೋಗುತ್ತಲೆ ಪರಶಿವನು ಪಾರ್ವತಿಯೊಡನೆ ‘ದೇವಿ, ಭಗವ ದ್ಭಕ್ತರ ಮಹಿಮೆಯೆಂತಹುದೆಂಬುದನ್ನು ನೀನೀಗ ಪ್ರತ್ಯಕ್ಷವಾಗಿ ಕಂಡಿರುವೆ. ಅವರು ಯಾವುದಕ್ಕೂ ಅಂಜುವುದಿಲ್ಲ. ನೋಡು ಈ ಚಿತ್ರಕೇತು ಪ್ರತಿಶಾಪವನ್ನು ಕೊಡಲು ಶಕ್ತಿ ಯುಳ್ಳವನು. ಆದರೂ ಹಾಗೆ ಮಾಡದೆ ನಿನ್ನ ಶಾಪವನ್ನು ತಲೆಯಲ್ಲಿ ಧರಿಸಿದನು. ಇದೇ ಸಾಧುಗಳ ಲಕ್ಷಣ. ನಾನು, ಬ್ರಹ್ಮ, ಸನತ್ಕುಮಾರ ಇತ್ಯಾದಿಯಾದವರೆಲ್ಲರೂ ಭಗವ ದಂಶವೇ ಆದರೂ ಆ ಜಗದೀಶ್ವರನ ಸ್ವರೂಪವನ್ನು ತಿಳಿಯಲಾರೆವು. ನಾವೆಲ್ಲ ಭಗ ವಂತನ ಮಾಯೆಗೆ ಸಿಕ್ಕಿ, ಅಹಂಕಾರದಿಂದ ನಾವೇ ಪ್ರಭುಗಳೆಂದುಕೊಳ್ಳುತ್ತಿದ್ದೇವೆ. ದೇವ ದೇವನಾದ ಪರಬ್ರಹ್ಮವಸ್ತುವಿಗೆ ಶತ್ರು ಮಿತ್ರ ಭೇದವಿಲ್ಲ. ಅವನಿಗೆ ಎಲ್ಲವೂ ಪ್ರಿಯವೇ. ಅಂತಹ ಭಗವಂತನ ಭಕ್ತ ಈ ಚಿತ್ರಕೇತು. ನಾನೂ ಭಗವಂತನ ಭಕ್ತನೇ. ಆದ್ದರಿಂದ ಭಕ್ತರ ವಿಚಾರದಲ್ಲಿ ಅಹಂಕಾರ ಸಲ್ಲದು’ ಎಂದು ಹೇಳಿ, ಆಕೆಯನ್ನು ಸಮಾಧಾನಮಾಡಿ ದನು. 

ಪಾರ್ವತಿಯಿಂದ ಶಾಪವನ್ನು ಪಡೆದ ಚಿತ್ರಕೇತುವೇ ವೃತ್ರಾಸುರನಾಗಿ ಹುಟ್ಟಿದ್ದ. ಆದ್ದರಿಂದಲೇ ಆತನು ರಾಕ್ಷಸನಾಗಿ ಹುಟ್ಟಿದಾಗಲೂ ಜ್ಞಾನಿಯಾಗಿದ್ದನು, ಭಗವದ್ಭಕ್ತನಾಗಿದ್ದನು.

