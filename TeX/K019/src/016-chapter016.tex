
\chapter{೧೬. ಪೃಥುಚಕ್ರವರ್ತಿ}

ಪೃಥುವಿನ ಪಟ್ಟಾಭಿಷೇಕವಾಗುವುದೆಂದು ಕೇಳಿ ಲೋಕವೆಲ್ಲವೂ ಸಂತೋಷದಿಂದ ಹಿಗ್ಗಿತು. ಸಕಲ ದಿಕ್ಕುಗಳಿಂದಲೂ ಜನರು ಪಟ್ಟಾಭಿಷೇಕಕ್ಕೆ ಅಗತ್ಯವಾದ ಸಾಮಾನುಗಳನ್ನು ಸಂಗ್ರಹಿಸಿಕೊಂಡು ಬಂದರು. ಮನುಷ್ಯರು ಮಾತ್ರವೇ ಅಲ್ಲ; ನದಿ, ಸಮುದ್ರ, ಪರ್ವತ, ಸರ್ಪ, ಗೋವು, ಹಕ್ಕಿ, ಭೂಮಿ, ಆಕಾಶಗಳು ಕೂಡ ಆತನ ಅಭಿಷೇಕ ಸಮಯದಲ್ಲಿ ಕಾಣಿಕೆಯನ್ನು ತಂದು ಒಪ್ಪಿಸಿದವು. ಕುಬೇರನು ಆತನಿಗೆ ಬಂಗಾರದ ಸಿಂಹಾಸನವನ್ನು ಕೊಟ್ಟನು; ವರುಣನು ಚಂದ್ರಬಿಂಬದಂತೆ ಬೆಳ್ಳಗಿರುವ, ನೀರು ಹನಿಯನ್ನು ತೊಟ್ಟಿಡು ವಷ್ಟು ತಣ್ಣಗಿರುವ ಛತ್ರಿಯನ್ನು ನೀಡಿದನು; ವಾಯುದೇವನು ಎರಡು ಚಾಮರಗಳನ್ನೂ, ಧರ್ಮಪುತ್ರನು ಪುಷ್ಪಹಾರವನ್ನೂ, ದೇವೇಂದ್ರನು ರತ್ನದ ಕಿರೀಟವನ್ನೂ, ಯಮನು ದಂಡವನ್ನೂ ಒಪ್ಪಿಸಿದರು. ಬ್ರಹ್ಮನು ವೇದಮಯವಾದ ಒಂದು ಕವಚವನ್ನು ಆಶೀರ್ ವಾದ ಮಾಡಿ ಕೊಟ್ಟನು. ಸರಸ್ವತಿಯು ಮುತ್ತಿನ ಹಾರವನ್ನು ಇತ್ತಳು. ಮಹಾವಿಷ್ಣುವು ಸುದರ್ಶನದ ಅಂಶದಿಂದ ಹುಟ್ಟಿದ ಚಕ್ರವನ್ನು ಅನುಗ್ರಹಿಸಿದನು. ಲಕ್ಷ್ಮೀದೇವಿಯು ಶಾಶ್ವತವಾದ ಐಶ್ವರ್ಯವನ್ನು ನೀಡಿದಳು. ರುದ್ರ ಅಂಬಿಕೆಯರು ಒಂದೊಂದು ಖಡ್ಗ ವನ್ನು ದಯಪಾಲಿಸಿದರು. ಚಂದ್ರನು ಕುದುರೆಗಳನ್ನೂ ಸುಂದರವಾದ ರಥವನ್ನೂ ಕೊಟ್ಟನು. ಅಗ್ನಿಯು ಬಿಲ್ಲನ್ನೂ ಸೂರ್ಯನು ಬಾಣಗಳನ್ನೂ ಕೊಟ್ಟು ಮನ್ನಿಸಿದರು. ಭೂದೇವಿ ಆತನನ್ನು ಬೇಕಾದಲ್ಲಿಗೆ ಕೊಂಡೊಯ್ಯಬಲ್ಲ ಪಾದುಕೆಗಳನ್ನು ನೀಡಿದಳು. ಹೀಗೆ ಪೃಥುಚಕ್ರವರ್ತಿಯು ಸಮಸ್ತರಿಂದಲೂ ಸನ್ಮಾನಿತನಾದನು. ಇಡೀ ಭೂಮಂಡಲವನ್ನು ದರ್ಪದಿಂದ ಆಳುತ್ತಿದ್ದ ಆತನು ‘ಆದಿರಾಜ’ನೆಂಬ ಕೀರ್ತಿಗೆ ಪಾತ್ರನಾದನು.

ಪೃಥುಚಕ್ರವರ್ತಿಯು ಸಿಂಹಾಸನವನ್ನೇರಿದಾಗ ರಾಜ್ಯದ ಸ್ಥಿತಿಯೆಲ್ಲವೂ ಅಲ್ಲೋಲ ಕಲ್ಲೋಲವಾಗಿತ್ತು. ಭೂಮಿಯಲ್ಲಿ ಬೆಳೆಯೇ ಇಲ್ಲದೆ ಜನರೆಲ್ಲ ಉಪಾವಾಸದಿಂದ ಸಾಯುವ ಸ್ಥಿತಿಯಲ್ಲಿದ್ದರು. ಅವರೆಲ್ಲರೂ ರಾಜನ ಬಳಿಗೆ ಬಂದು ‘ಸ್ವಾಮಿ, ನಾವು ಹಸಿವಿ ನಿಂದ ಸಾಯುತ್ತಿದ್ದೇವೆ. ನಮಗೆ ತಕ್ಷಣವೆ ಹೊಟ್ಟೆಗೆ ಹಿಟ್ಟು ಕೊಟ್ಟು ಕಾಪಾಡು’ ಎಂದು ಅತಿ ದೀನರಾಗಿ ಬೇಡಿಕೊಂಡರು. ಅವರ ದೀನವಾಣಿಯನ್ನು ಕೇಳಿ ರಾಜನ ಕರುಳು ಕರ ಗಿತು. ಆತನು ಇದಕ್ಕೆ ಕಾರಣವೇನಿರಬಹುದೆಂದು ಆಳವಾಗಿ ಆಲೋಚಿಸಿದನು. ನೆಟ್ಟ ಬೀಜಗಳನ್ನು ಭೂಮಿಯೇ ನುಂಗುತ್ತಿದೆ; ಇದರಿಂದ ಬೆಳೆಯಿಲ್ಲದಂತಾಗಿದೆ. ಇದು ಅರ್ಥ ವಾಗುತ್ತಲೆ ಪೃಥುಚಕ್ರಿಗೆ ತ್ರಿಪುರಾಂತಕನಾದ ರುದ್ರನಂತೆ ಕೋಪುವುಕ್ಕಿತು. ಈ ಭೂಮಿಯನ್ನು ತುಂಡುತುಂಡಾಗಿ ಕತ್ತರಿಸಿಹಾಕಬೇಕೆಂದುಕೊಂಡು ತನ್ನ ಬಿಲ್ಲಿನಲ್ಲಿ ಬಾಣ ಹೂಡಿದನು. ಇದನ್ನು ಕಂಡು ಭೂದೇವಿ ಗಡಗಡ ನಡುಗುತ್ತಾ ಒಂದು ಹಸುವಿನ ರೂಪ ವನ್ನು ಧರಿಸಿ ಓಡಿಹೋಗಲಾರಂಭಿಸಿದಳು. ರಾಜನು ಹುಲ್ಲೆಯನ್ನು ಅಟ್ಟುವ ಬೇಡನಂತೆ ಆಕೆಯ ಬೆನ್ನಟ್ಟಿದನು. ತಾನಿನ್ನು ಉಳಿಯುವುದು ಅಸಾಧ್ಯವೆನಿಸಿದಾಗ ಆಕೆಯು ಪೃಥು ರಾಜನಿಗೆ ಶರಣಾಗಿ ‘ಮಹಾನುಭಾವ, ಲೋಕವೆಲ್ಲವನ್ನೂ ಕಾಪಾಡಲೆಂದು ಅವತರಿಸಿರುವ ನೀನು ನನ್ನನ್ನೇಕೆ ಕೊಲ್ಲಹೊರಟಿರುವೆ? ಧರ್ಮವನ್ನು ಬಲ್ಲ ನೀನು ಹೆಣ್ಣಾದ ನನ್ನನ್ನು ಕೊಲ್ಲಬಹುದೆ? ನಾನು ಮಾಡಿರುವ ಅಪರಾಧವಾದರೂ ಏನು? ನಾನು ಹಡಗಿನಂತೆ ನೀರಿನ ಮೇಲೆ ನಿಂತಿದ್ದೇನೆ. ಜಗತ್ತೆಲ್ಲವೂ ನನ್ನನ್ನು ಆಶ್ರಯಿಸಿ ನಿಂತಿದೆ. ನನ್ನನ್ನು ನಾಶಮಾಡಿದ ಮೇಲೆ ಈ ಜಗತ್ತನ್ನೆಲ್ಲ ನೀರುಪಾಲು ಮಾಡುವೆಯೇನು? ನೀನು ತಾನೆ ಎಲ್ಲಿ ನಿಲ್ಲುವೆ?’ ಎಂದು ಕೇಳಿದಳು.

ಪೃಥುರಾಜನು ಭೂದೇವಿಗೆ ಉತ್ತರವಿತ್ತ–‘ಎಲೆ ಭೂಮಿ, ನೀನು ನಿರಪರಾಧಿನಿಯೆಂದು ಹೇಳುತ್ತಿರುವೆಯಲ್ಲ! ನೀನು ಅನೇಕ ಅಪರಾಧಗಳನ್ನು ಮಾಡಿರುವೆ. ನೋಡು, ರಾಜನಾದ ನನ್ನ ಅಪ್ಪಣೆಯಂತೆ ನಡೆಯದಿರುವುದು ನಿನ್ನ ಮೊದಲನೆಯ ದೊಡ್ಡ ಅಪ ರಾಧ. ಪ್ರಾಣಿಗಳ ಆಹಾರವಾಗಿ ಬ್ರಹ್ಮನು ಸೃಷ್ಟಿಸಿರುವ ಬೀಜಗಳನ್ನು ಮೊಳೆಯುವುದಕ್ಕೆ ಬಿಡದೆ ನೀನೆ ನುಂಗುತ್ತಿರುವೆಯಲ್ಲ, ಇದು ಅಪರಾಧವಲ್ಲವೇನು? ಯಜ್ಞಗಳಲ್ಲಿ ಇತರ ದೇವತೆಗಳಂತೆ ನಿನಗೂ ಹವಿಸ್ಸನ್ನು ಕೊಡುತ್ತಿರುವೆವಲ್ಲ, ಆ ಕೃತಜ್ಞತೆಯಾದರೂ ನಿನಗೆ ಬೇಡವೆ? ಯಥೇಷ್ಟವಾಗಿ ಹುಲ್ಲುತಿಂದರೂ ಹಾಲುಕೊಡದೆ, ಒದೆಯುವ ಹಸು, ನೀನು. ರಾಜನಾದ ನಾನು ನನ್ನ ಪ್ರಜೆಗಳ ಹಸಿವನ್ನು ಹಿಂಗಿಸಬೇಕಾದುದು ನನ್ನ ಪವಿತ್ರ ಕರ್ತವ್ಯ. ಆದ್ದರಿಂದ ಈಗ ನಿನ್ನನ್ನು ಸೀಳಿ ನಿನ್ನ ಮಾಂಸದಿಂದಲೆ ಈ ಪ್ರಜೆಗಳ ಹಸಿವನ್ನು ಹೋಗ ಲಾಡಿಸುತ್ತೇನೆ. ನೀನು ಹೆಣ್ಣೆಂದು ಮುಂದುಮಾಡಿಕೊಂಡು ಬದುಕಬಯಸುವೆಯಾ? ಕ್ರೂರ ಸ್ವಭಾವದವರು ಹೆಣ್ಣಾಗಲಿ ಗಂಡಾಗಲಿ ಅವರನ್ನು ಕೊಂದರೆ ಪಾಪವೇನೂ ಇಲ್ಲ. ನೀನಿಲ್ಲದಿದ್ದರೆ ಜನರೆಲ್ಲ ಆಧಾರ ತಪ್ಪುವರೆಂದು ಹೆದರಿಸುತ್ತಿರುವೆಯಲ್ಲವೆ? ಕೇಳು, ನಿನ್ನನ್ನು ತುಂಡುತುಂಡಾಗಿ ಕತ್ತರಿಸಿದಮೇಲೆಯೂ ನನ್ನ ಪ್ರಜೆಗಳನ್ನು ನನ್ನ ಯೋಗ ಶಕ್ತಿ ಯಿಂದ ಹಿಂದಿನಂತೆಯೇ ನೆಲೆಯಾಗಿ ನಿಲ್ಲಿಸುತ್ತೇನೆ.’

ಕೋಪಗೊಂಡ ಯಮನಂತೆ ಬಿಲ್ಲನ್ನು ಹಿಡಿದು ನಿಂತು, ತನ್ನನ್ನು ಗದರಿಸಿ ನುಡಿದ ಪೃಥು ಚಕ್ರವರ್ತಿಯನ್ನು ಕಂಡು ಭೂದೇವಿಯು ಬಿರುಗಾಳಿಗೆ ಸಿಕ್ಕ ಬಾಳೆಯಂತೆ ಗಡಗಡ ನಡಗುತ್ತಾ ಆತನಿಗೆ ಅಡ್ಡಬಿದ್ದು, ಅತಿ ದೈನ್ಯದಿಂದ ‘ಹೇ, ದೇವದೇವ, ನೀನು ಭಗವಂತನ ಅವತಾರವೆಂದು ನಾನು ಬಲ್ಲೆ. ಸರ್ವಜ್ಞನಾಗಿ ಸರ್ವಶಕ್ತನಾಗಿರುವ ನಿನ್ನನ್ನು ವಂಚಿಸುವು ದಕ್ಕೆ ನನ್ನಿಂದ ಸಾಧ್ಯವೆ? ನೀನು ತ್ರಿಮೂರ್ತಿಸ್ವರೂಪನು. ನನ್ನ ಸೃಷ್ಟಿಗೆ ನೀನೆ ಕಾರಣ. ನೀನೆ ನನ್ನನ್ನು ಕೊಲ್ಲಹೊರಟರೆ ನನ್ನನ್ನು ರಕ್ಷಿಸುವವರಾರು? ಪ್ರಜೆಗಳನ್ನು ರಕ್ಷಿಸುವು ದಕ್ಕಾಗಿಯೆ ನೀನೀಗ ಚಕ್ರವರ್ತಿಯಾಗಿ ಜನಿಸಿರುವೆ. ಆ ಪ್ರಜೆಗಳಿಗೆ ಭೋಗ ಭೂಮಿಯೆನಿ ಸಿರುವ ನನ್ನನ್ನು ಕೊಂದರೆ ಏನು ಗತಿ? ಹಿಂದೆ ಸಮುದ್ರದಲ್ಲಿ ಮುಳುಗಿ ಹೋಗಿದ್ದ ನನ್ನನ್ನು ವರಾಹರೂಪಿನಿಂದ ಮೇಲಕ್ಕೆತ್ತಿ ತಂದವನು ನೀನೆ ಅಲ್ಲವೆ? ನೀನೆ ಈಗ ನನ್ನನ್ನು ಕೊಲ್ಲುವನೆಂದು ಹೊರಟರೆ ನಾನೇನು ಮಾಡಲಿ? ಆದರೆ ಒಂದು ವಿಚಾರ. ಬ್ರಹ್ಮನು ಸೃಷ್ಟಿಸಿದ ಸಸ್ಯಗಳ ಬೀಜಗಳನ್ನು ನಾನು ನುಂಗಿದೆನೆಂದು ನೀನು ಇಷ್ಟು ಕೋಪಗೊಂಡಿರು ವೆಯಲ್ಲ! ನಾನು ನುಂಗದೆ ಏನು ಮಾಡಬೇಕಾಗಿತ್ತು? ರಾಜ್ಯವೆಲ್ಲ ಅನಾಯಕವಾಗಿ, ಪಾಪಿ ಗಳು ಆ ಬೀಜಗಳನ್ನೆಲ್ಲ ನುಂಗಿಹಾಕುತ್ತಿದ್ದರು; ಅವರನ್ನು ತಡೆಯುವ ಮಹಾನುಭಾವ ರಾರೂ ಇರಲಿಲ್ಲ; ಅವರ ದೆಸೆಯಿಂದ ಸಸ್ಯಗಳೆಲ್ಲ ಸಮೂಲವಾಗಿ ನಾಶವಾಗುತ್ತಿದ್ದವು. ಕಳ್ಳಕಾಕರ ದೆಸೆಯಿಂದ ದೇಶ ಅನಾಯಕವಾಗಿತ್ತು. ಆದ್ದರಿಂದ ನಾನು ಅವುಗಳನ್ನು ಸಾಧ್ಯ ವಾದಷ್ಟು ಉಳಿಸಬೇಕೆಂದುಕೊಂಡು ನುಂಗಿದೆ. ಬಹುಕಾಲದವರೆಗೆ ರಕ್ಷಕನು ಬಾರದೆ ಹೋದುದರಿಂದ ಆ ಬೀಜಗಳು ಈಗ ನನ್ನ ಹೊಟ್ಟೆಯಲ್ಲಿ ಅರಗಿಹೋಗಿವೆ. ಅವನ್ನು ಮತ್ತೆ ಪಡೆಯಬೇಕಾದರೆ ಈ ಗೋರೂಪದಲ್ಲಿರುವ ನನಗೆ ಒಂದು ಕರುವನ್ನು ಕರುಣಿಸು; ನನ್ನ ಹಾಲನ್ನು ಕರೆಯಲು ತಕ್ಕುದಾದ ಒಂದು ಪಾತ್ರೆಯನ್ನು ಸಿದ್ಧಗೊಳಿಸು; ಹಾಲನ್ನು ಕರೆಯಲು ಶಕ್ತನಾದ ಮನುಷ್ಯನನ್ನು ನೇಮಿಸು. ಹಾಲಿನ ರೂಪದಿಂದ ಆ ಬೀಜಗಳೆಲ್ಲ ಹೊರಬರುತ್ತವೆ’ ಎಂದಳು.

ಭೂದೇವಿಯ ಮಾತುಗಳನ್ನು ಕೇಳಿ ಪೃಥುರಾಜನ ಕೋಪ ಶಾಂತವಾಯಿತು. ಆತನು ಸ್ವಾಯಂಭುವಮನುವನ್ನು ಕರುವಾಗಿ ಮಾಡಿ, ತಾನೆ ಕರೆಯುವುದಕ್ಕೆ ಕುಳಿತು, ತನ್ನ ಬೊಗಸೆಯೊಳಕ್ಕೆ ಸಮಸ್ತ ಸಸ್ಯಗಳನ್ನೂ ಕರೆದನು. ಆಮೇಲೆ ಇತರರೂ ತಮ್ಮ ತಮ್ಮ ಇಷ್ಟಕ್ಕೆ ಅನುಸಾರವಾಗಿ ತಮಗೆ ಬೇಕಾದ ವಸ್ತುಗಳನ್ನೂ ಕರೆದುಕೊಂಡರು. ಮಹರ್ಷಿಗಳು ಬೃಹಸ್ಪತಿಯನ್ನು ಕರುವಾಗಿ ಮಾಡಿಕೊಂಡು, ವಾಕ್ಕು, ಕಿವಿ, ಮನಸ್ಸು ಎಂಬ ಪಾತ್ರೆಯಲ್ಲಿ ವೇದವೆಂಬ ಹಾಲನ್ನು ಕರೆದುಕೊಂಡರು. ದೇವತೆಗಳು ಇಂದ್ರನನ್ನೆ ಕರುವಾಗಿ ಮಾಡಿ ಕೊಂಡು ಸೋಮರಸವೆಂಬ ಅಮೃತವನ್ನು ಕರೆದುಕೊಂಡರು. ಹೀಗೆಯೆ ರಾಕ್ಷಸರು ಹೆಂಡ ವನ್ನು, ಗಂಧರ್ವರು ಸೌಂದರ್ಯವನ್ನು, ಸಿದ್ಧರು ಅಣಿಮಾದಿ ಸಿದ್ಧಿಗಳನ್ನು, ಹಾವುಗಳು ವಿಷವನ್ನು, ಪಶುಗಳು ಹುಲ್ಲನ್ನು, ಕ್ರೂರಮೃಗಗಳು ಮಾಂಸವನ್ನು ಕರೆದುಕೊಂಡವು. ಯಾರುಯಾರಿಗೆ ಯಾವುದು ಬೇಕೊ ಅದನ್ನು ಅವರವರು ಕರೆದುಕೊಂಡ ಮೇಲೆ, ಅತ್ಯಂತ ಸಂತೋಷಗೊಂಡ ಪೃಥುರಾಜನು ಭೂದೇವಿಯನ್ನು ತನ್ನ ಮಗಳಾಗಿ ಪರಿಗ್ರಹಿಸಿದನು. ಆತನ ಕೃಪೆಯಿಂದ ಪ್ರಜೆಗಳಿಗೆಲ್ಲ ಮನೆ ಮಠಗಳು, ಊರು ಕೇರಿಗಳು ನಿರ್ಮಾಣವಾದವು. ಜನರೆಲ್ಲ ಆತನ ಆಳ್ವಿಕೆಯಲ್ಲಿ ಸುಖಶಾಂತಿಗಳನ್ನು ಪಡೆದರು.

ಧರ್ಮದಿಂದ ರಾಜ್ಯಭಾರಮಾಡುತ್ತಿದ್ದ ಪೃಥುಚಕ್ರಿಗೆ ಅಶ್ವಮೇಧಯಾಗವನ್ನು ಮಾಡ ಬೇಕೆನ್ನಿಸಿತು. ಮನಸ್ಸಿಗೆ ಬಂದುದೇ ತಡ, ಆತನು ಬ್ರಹ್ಮಾವರ್ತವೆಂಬ ಪವಿತ್ರವಾದ ಪ್ರದೇಶದಲ್ಲಿ, ಪೂರ್ವದಿಕ್ಕಿಗೆ ಹರಿಯುತ್ತಿದ್ದ ಸರಸ್ವತೀ ನದಿಯ ದಡದಲ್ಲಿ ಯಾಗವನ್ನು ಮಾಡಿ ಮುಗಿಸಿದನು. ಒಂದಾದ ಮೇಲೆ ಒಂದರಂತೆ ಆತನು ತೊಂಬತ್ತೊಂಬತ್ತು ಯಾಗ ಗಳನ್ನು ಮಾಡಿ ಮುಗಿಸಿ, ನೂರನೆಯ ಯಾಗವನ್ನೂ ಕೈಕೊಂಡನು. ನೂರು ಯಾಗಗಳನ್ನು ಮಾಡಿದವನಿಗೆ ದೇವೇಂದ್ರಪದವಿ ತಾನೆ? ಪೃಥುವಿನ ಯಾಗಗಳು ತನ್ನ ಯಾಗಕ್ಕಿಂತಲೂ ಮೇಲಾಗಿರುವುದನ್ನು ಕಂಡ ದೇವೇಂದ್ರನಿಗೆ ಆತನಲ್ಲಿ ಅಸೂಯೆ ಹುಟ್ಟಿತು. ಅಹುದು, ಪೃಥುಚಕ್ರವರ್ತಿಯ ಯಾಗವೈಭವ ಅಸೂಯೆ ಹುಟ್ಟಿಸುವಂತಹುದೇ. ಆತನ ಯಜ್ಞಕ್ಕೆ ಯಜ್ಞಪುರುಷನಾದ ನಾರಾಯಣನೇ ಪ್ರತ್ಯಕ್ಷನಾಗಿ ಬಂದು ಭಾಗವಹಿಸುತ್ತಿದ್ದನು. ಆದ್ದ ರಿಂದ ಬ್ರಹ್ಮರುದ್ರರೂ ದೇವಾನುದೇವತೆಗಳೂ ಋಷಿಗಳೂ ತಂಡತಂಡವಾಗಿ ಬಂದು ಅಲ್ಲಿ ನೆರೆದಿರುತ್ತಿದ್ದರು. ಭೂದೇವಿಯು ಆತನಿಗೆ ಬೇಕಾದ ವಸ್ತುಗಳನ್ನೆಲ್ಲ ಯಥೇಚ್ಛೆ ಯಾಗಿ ತಂದು ಒಪ್ಪಿಸುತ್ತಿದ್ದಳು. ಹಾಲು, ಬೆಣ್ಣೆ, ತುಪ್ಪ ಮೊದಲಾದವು ಅಲ್ಲಿ ಪ್ರವಾಹ ವಾಗಿ ಹರಿಯುತ್ತಿದ್ದವು. ಮರಗಿಡಗಳು ಜೇನು ತೊಟ್ಟಿಕ್ಕುವಂತಹ ರಸಭರಿತವಾದ ಹಣ್ಣು ಗಳನ್ನು ನೀಡುತ್ತಿದ್ದವು. ಪ್ರಜೆಗಳು ಕಪ್ಪಕಾಣಿಕೆಗಳನ್ನು ಒಪ್ಪಿಸುವುದಿರಲಿ, ಸಮುದ್ರಗಳು ರತ್ನರಾಶಿಗಳನ್ನೂ ಬೆಟ್ಟಗಳು ಅನ್ನರಾಶಿಯನ್ನೂ ತಂದು ಒಪ್ಪಿಸುತ್ತಿದ್ದವು. ಇಂತಹ ವೈಭವವನ್ನು ಕಂಡು ದೇವೇಂದ್ರನ ಕಣ್ಣು ಕುಕ್ಕಿದುದು ಆಶ್ಚರ್ಯ ವೇನೂ ಅಲ್ಲ. ಅವನು ಆ ಯಜ್ಞವನ್ನು ಹಾಳುಮಾಡಬೇಕೆಂದು ಹೊಂಚುಹಾಕುತ್ತಿದ್ದು, ಯಾವುದೊ ಮಾಯೆ ಯಿಂದ ಯಾಗದ ಕುದುರೆಯನ್ನು ಹೊತ್ತುಕೊಂಡು ಹೋದನು. ಇದರಿಂದ ಯಜ್ಞಕಾರ್ಯ ಅರ್ಧದಲ್ಲಿಯೇ ನಿಲ್ಲುವಂತಾಯಿತು.

ಇಂದ್ರನು ಯಾಗದ ಕುದುರೆಯನ್ನು ಕಳ್ಳತನ ಮಾಡಿಕೊಂಡು ಹೋಗುತ್ತಿದ್ದುದು ಅಕಸ್ಮಾತ್ತಾಗಿ ಅತ್ರಿಮಹರ್ಷಿಯ ಕಣ್ಣಿಗೆ ಬಿತ್ತು. ಆತನು ಅದನ್ನು ಪೃಥುಚಕ್ರಿಗೆ ತಿಳಿಸಿ ‘ಕೊಲ್ಲು ಆ ಕಳ್ಳನನ್ನು’ ಎಂದು ಕೂಗಿಕೊಂಡನು. ಆಗ ಅಲ್ಲಿಯೇ ಇದ್ದ ಚಕ್ರವರ್ತಿಯ ಮಗನು ಆ ಕಳ್ಳನನ್ನು ಬೆನ್ನಟ್ಟಿದನು. ಇದನ್ನು ಕಂಡ ದೇವೇಂದ್ರನು ಜಟೆಯನ್ನು ಧರಿಸಿ, ವಿಭೂತಿಯನ್ನು ಧರಿಸಿದ ಮಹಾತ್ಮನಂತೆ ಕಾಣಿಸಿಕೊಳ್ಳಲು, ಅವನನ್ನು ಕೊಲ್ಲಲು ರಾಜ ಪುತ್ರ ಹಿಂಜರಿದನು. ಆದರೆ ಅತ್ರಿಮುನಿಯು ‘ಅಯ್ಯಾ, ಸಂದೇಹವೇಕೆ? ಅವನೇ ದೇವೇಂದ್ರ, ವೇಷವನ್ನು ನೋಡಿ ಮೋಸಹೋಗಬೇಡ. ಕೊಲ್ಲು, ಅವನನ್ನು’ ಎಂದನು. ಇದನ್ನು ಕೇಳಿ ಆನೆಯನ್ನು ಅಟ್ಟುವ ಸಿಂಹದಂತೆ ರಾಜಕುಮಾರನು ಬೆನ್ನಟ್ಟಿಬರಲು, ಇಂದ್ರನು ಹೆದರಿ ಕುದುರೆಯನ್ನೂ, ತಾನು ಧರಿಸಿದ್ದ ಮೋಸದ ವೇಷವನ್ನೂ ಅಲ್ಲಿಯೆ ಬಿಟ್ಟು ಮಾಯವಾದನು. ರಾಜಪುತ್ರನು ಕುದುರೆಯನ್ನು ಹಿಂದಕ್ಕೆ ತಂದು ಒಪ್ಪಿಸಿದನು. ಆತನ ಸಾಹಸವನ್ನು ಕಂಡು ಮೆಚ್ಚಿದ ಋಷಿಗಳು ಆತನಿಗೆ ‘ವಿಜಿತಾಶ್ವ’ನೆಂದು ನಾಮಕರಣ ಮಾಡಿದರು.

ಯಾಗದ ಕುದುರೆ ಹಿಂದಕ್ಕೆ ಬರುತ್ತಲೆ ಯಾಗ ಮುಂದುವರೆಯಿತು. ಆಗ ದೇವೇಂದ್ರನು ತನ್ನ ಮಾಯೆಯಿಂದ ಜಗತ್ತಿಗೆಲ್ಲ ಕಗ್ಗತ್ತಲೆ ಕವಿಯುವಂತೆ ಮಾಡಿ, ಯಾರಿಗೂ ತಿಳಿಯದಂತೆ ಬಂದು, ಮತ್ತೆ ಕುದುರೆಯನ್ನು ಕದ್ದೊಯ್ದನು. ಅತ್ರಿಮುನಿಯೇ ಮತ್ತೆ ಅದನ್ನು ಪತ್ತೆಹಚ್ಚಿ, ವಿಜಿತಾಶ್ವನನ್ನು ಅವನ ಬೆನ್ನಟ್ಟುವಂತೆ ಹೇಳಿ ಕಳುಹಿಸಿದನು. ಈ ಬಾರಿ ದೇವೇಂದ್ರನು ಕೈಯಲ್ಲಿ ಕಪಾಲ ಡಮರುಗಳನ್ನು ಧರಿಸಿದ ಕಾಪಾಲಿಕನಂತೆ ಕಾಣಿಸಿಕೊಂಡನು. ಆದರೆ ಅತ್ರಿಋಷಿಯು ಅವನ ಮೋಸವನ್ನು ವಿವರಿಸಿ ಹೇಳಲು, ವಿಜಿತಾಶ್ವನು ಅವನನ್ನು ಕೊಲ್ಲಲು ಸಿದ್ಧನಾದನು. ಪುನಃ ದೇವೇಂದ್ರನು ತನ್ನ ವೇಷವನ್ನೂ ಕುದುರೆಯನ್ನೂ ತೊರೆದು ಮಾಯವಾದನು. ಯಜ್ಞದ ಕುದುರೆ ಮತ್ತೆ ಯಾಗಮಂಟಪಕ್ಕೆ ಹಿಂದಿರುಗಿತು. ಆದರೆ ಇಂದ್ರನು ಬಿಟ್ಟುಹೋದ ಮೋಸದ ವೇಷವನ್ನು ಜಗತ್ತಿನ ಮೂಢ ಜನ ಕೈಕೊಂಡರು. ಆ ವೇಷಗಳು ಪಾಪದ ಗುರುತುಗಳಾದ್ದರಿಂದ ಆ ವೇಷವನ್ನು ಧರಿಸಿ ದವರು ಪಾಷಂಡಿ(ಪಾಪಷಂಡ)ಗಳೆನಿಸಿದರು. ಇವರು ಜನರಲ್ಲಿ ಮಹಾತ್ಮರೆಂಬ ಭ್ರಾಂತಿ ಯನ್ನು ಹುಟ್ಟಿಸಿ, ಮೋಸ ಮಾಡುವವರಾದರು.

ತನ್ನ ಯಾಗಕ್ಕೆ ದೇವೇಂದ್ರನಿಂದ ಎರಡು ಸಲ ವಿಘ್ನವಾದುದನ್ನು ಕಂಡು ಪೃಥುಚಕ್ರಿಗೆ ಆತನನ್ನು ಕೊಂದುಹಾಕಲೇಬೇಕೆಂಬಷ್ಟು ಕೋಪ ಉಕ್ಕಿತು. ಆತನು ಬಿಲ್ಲಿನಲ್ಲಿ ಬಾಣ ವನ್ನು ಹೂಡಿ ದೇವೇಂದ್ರನಿಗೆ ಗುರಿಯಿಟ್ಟನು. ಆದರೆ ಅಲ್ಲಿ ನೆರೆದಿದ್ದ ಋಷಿಗಳು ಆತನನ್ನು ತಡೆದು ‘ಮಹಾರಾಜ, ನಿನ್ನ ಕೀರ್ತಿಯ ಬೆಳಕಿನಲ್ಲಿ ಕಳೆಯನ್ನು ಕಳೆದು ಕೊಂಡಿರುವ ಈ ದೇವೇಂದ್ರ ಈಗಾಗಲೆ ಸತ್ತವನಂತೆಯೆ ಆಗಿದ್ದಾನೆ. ಅವನನ್ನು ನೀನು ಕೊಲ್ಲವುದು ಬೇಡ. ನೀನು ಯಜ್ಞದ ಪಶುವನ್ನು ಹೊರತು ಬೇರೆ ಯಾವ ಪ್ರಾಣಿಯನ್ನೂ ಕೊಲ್ಲಬಾರದು. ನಾವೇ ನಮ್ಮ ಮಂತ್ರದಿಂದ ಅವನನ್ನು ಸೆಳೆದು ಅಗ್ನಿಗೆ ಆಹುತಿಯಾಗಿ ಮಾಡುತ್ತೇವೆ’ ಎಂದು ಹೇಳಿ, ಇಂದ್ರನನ್ನು ಕೊಲ್ಲಲೆಂದು ಯಜ್ಞವನ್ನು ಪ್ರಾರಂಭಿಸಿ ದರು. ಇದನ್ನು ಅರಿತ ಬ್ರಹ್ಮದೇವನು ತಕ್ಷಣವೇ ಯಜ್ಞಮಂಟಪದಲ್ಲಿ ಪ್ರತ್ಯಕ್ಷನಾಗಿ ‘ಅಯ್ಯಾ ಋಷಿಗಳೆ, ಈ ಯಾಗವು ಭಗವಂತನ ಶರೀರ, ಇದರಂತೆ ದೇವತೆಗಳೂ ಆತನ ಶರೀರ. ಆದ್ದರಿಂದ ಇಂದ್ರನನ್ನು ಕೊಲ್ಲಬೇಕೆಂಬ ನಿಮ್ಮ ಪ್ರಯತ್ನ ಸಲ್ಲದು. ಈ ಯಾಗ ವನ್ನು ಕೆಡಿಸಹೊರಟಿರುವ ದೇವೇಂದ್ರನಿಗೆ ಸಾಕಷ್ಟು ಅಪಕೀರ್ತಿ ಬಂದಿದೆ. ತೊಂಬತ್ತೊಂ ಬತ್ತು ಯಾಗಗಳಿಂದಲೆ ಪೃಥುವು ಆತನಿಗಿಂತ ಹಿರಿಯನೆನಿಸಿದ್ದಾನೆ. ಇನ್ನು ನೂರನೆಯ ಯಾಗ ಮಾಡಬೇಕಾದ ಅವಶ್ಯಕವಿಲ್ಲ. ಈಗಾಗಲೆ ಆ ದೇವೇಂದ್ರನು ಎರಡು ಸಲ ಪಾಷಂಡ ವೇಷವನ್ನು ಧರಿಸಿ, ಜನರಲ್ಲಿ ಪಾಷಂಡಮತ ಹರಡುವುದಕ್ಕೆ ಕಾರಣನಾಗಿದ್ದಾನೆ. ಅದು ಇನ್ನೂ ಹೆಚ್ಚಾಗಿ ಹರಡುವುದು ಬೇಡ. ಪೃಥುಚಕ್ರಿ ಮೋಕ್ಷಾಪೇಕ್ಷಿ. ಅದಕ್ಕೆ ಅಡ್ಡಿಯಾದ ಈ ಯಾಗಕಾರ್ಯವನ್ನು ನಿಲ್ಲಿಸಿಬಿಡಿ’ ಎಂದನು. ಬ್ರಹ್ಮನ ಬುದ್ಧಿವಾದದಂತೆ ಪೃಥು ಚಕ್ರವರ್ತಿ ತನ್ನ ಯಾಗವನ್ನು ನಿಲ್ಲಿಸಿ, ನೆರೆದಿದ್ದ ಬ್ರಾಹ್ಮಣರಿಗೆ ಬೇಕಾದಷ್ಟು ದಾನ ದಕ್ಷಿಣೆಗಳನ್ನು ಕೊಟ್ಟು ತೃಪ್ತಿಪಡಿಸಿದನು. ಋಷಿಗಳೂ ದೇವತೆಗಳೂ ಸನ್ಮಾನವನ್ನು ಪಡೆದು ಸಂತೋಷಿತರಾದರು. ಎಲ್ಲರೂ ರಾಜನನ್ನು ಬಾಯ್ತುಂಬ ಹರಸಿ, ತಮ್ಮ ತಮ್ಮ ಸ್ಥಳಗಳಿಗೆ ಹಿಂತಿರುಗಿದರು.

ರಾಜನ ನಡವಳಿಕೆಯಿಂದ ಸುಪ್ರೀತನಾದ ನಾರಾಯಣನು ದೇವೇಂದ್ರನೊಡನೆ ಆತನ ಬಳಿಗೆ ಬಂದು, “ರಾಜೇಂದ್ರ, ಈ ದೇವೇಂದ್ರ ನಿನ್ನಲ್ಲಿ ಕ್ಷಮೆ ಬೇಡಲು ಬಂದಿದ್ದಾನೆ, ಅವನನ್ನು ಕ್ಷಮಿಸು. ದೇಹಗಳು ಬೇರೆಬೇರೆಯಾದರೂ ಆತ್ಮವೊಂದೇ ಎಂಬುದನ್ನು ನಿನಗೆ ಹೇಳಿಕೊಡಬೇಕೆ? ನಿನ್ನಂತಹವನು ಮಾಯೆಗೆ ಸಿಕ್ಕಿ ಮರುಳಾದರೇನು ಗತಿ? ಈ ಶರೀರ ಅಜ್ಞಾನದಿಂದ, ಅಜ್ಞಾನಮೂಲಕವಾಗಿ ಬಂದ ಕಾಮದಿಂದ, ಮತ್ತು ಕಾಮ ಮೂಲಕವಾದ ಕರ್ಮದಿಂದ ಬಂದುದು. ಆದ್ದರಿಂದ ಶರೀರದಲ್ಲಿ ಅಹಂಕಾರ ಮಮಕಾರ ಗಳು ಇರಬಾರದು. ಆತ್ಮಕ್ಕೂ ದೇಹಕ್ಕೂ ಇರುವ ವ್ಯತ್ಯಾಸವನ್ನು ತಿಳಿಯುವುದೇ ವಿವೇಕ. ನೋಡು, ಆತ್ಮ ಒಬ್ಬನೇ, ದೇಹಗಳು ಅನೇಕ; ಆತ್ಮ ಶುದ್ಧ, ದೇಹ ಮಲಿನ; ಆತ್ಮ ಸ್ವ ಪ್ರಕಾಶ, ದೇಹ ಪರಪ್ರಕಾಶ; ಆತ್ಮ ನಿರ್ಗುಣ, ದೇಹ ಗುಣಯುಕ್ತ; ಆತ್ಮ ಗುಣಾಶ್ರಯ, ದೇಹ ಗುಣಾಧೀನ; ಆತ್ಮ ಸರ್ವವ್ಯಾಪಕ, ದೇಹ ಪರಿಮಿತ; ಆತ್ಮ ಸರ್ವಸಾಕ್ಷಿ, ದೇಹ ಜಡ; ಆತ್ಮ ಸ್ವತಂತ್ರ, ದೇಹ ಪರತಂತ್ರ, ಈ ಆತ್ಮಸ್ವರೂಪವನ್ನು ತಿಳಿದ ಆತ್ಮಾರಾಮನು ದೇಹವನ್ನು ಧರಿಸಿದ್ದರೂ ಅದರ ಕಾಮಕ್ರೋಧಾದಿ ವಿಕಾರಗಳಿಗೆ ಒಳಗಾಗುವುದಿಲ್ಲ. ದುರಾಶೆಯನ್ನು ಬಿಟ್ಟು ಸ್ವಧರ್ಮವನ್ನು ಆಚರಿಸುತ್ತಾ ನನ್ನನ್ನು ಭಜಿಸುವವನು ಶಾಂತಿ ಯನ್ನು ಪಡೆಯುತ್ತಾನೆ. ಶಾಂತಿಯೆಂದರೇನು? ಎಲ್ಲದರಲ್ಲಿಯೂ ಸಂಪೂರ್ಣ ಉದಾ ಸೀನ. ಇದನ್ನೆ ಬ್ರಹ್ಮವೆಂದೂ ಕೈವಲ್ಯವೆಂದೂ ಕರೆಯುತ್ತಾರೆ. ಪೃಥುಚಕ್ರಿ, ನೀನು ಇದ ನ್ನೆಲ್ಲ ತಿಳಿದ ಜ್ಞಾನಿಯಾದವನು. ಆದ್ದರಿಂದ ಜಗತ್ತೆಲ್ಲ ಆತ್ಮಸ್ವರೂಪವೆಂದು ತಿಳಿದು, ನಿನಗಿಂತಲೂ ಮೇಲಾದವರು ಅಥವಾ ಕೀಳಾದವರು ಎಂಬ ಭೇದವನ್ನೆಣಿಸದೆ, ನಿನ್ನ ಪರಿವಾರದೊಡನೆ ರಾಜ್ಯಭಾರಮಾಡು. ರಾಜನಾದ ನಿನ್ನ ಸ್ವಧರ್ಮವೆಂದರೆ ಪ್ರಜೆಗಳನ್ನು ಕಾಪಾಡುವುದು. ಇದರಿಂದಲೆ ನಿನಗೆ ಶ್ರೇಯಸ್ಸು. ‘ನನ್ನಿಂದಲೆ ಈ ಲೋಕರಕ್ಷಣೆ’ ಎಂಬ ಅಹಂಕಾರಮಾತ್ರ ನಿನ್ನಲ್ಲಿ ಅಂಕುರಿಸುವುದು ಬೇಡ. ಪ್ರಜೆಗಳಲ್ಲಿ ನಾನೂ ಒಬ್ಬ–ಎಂಬ ಭಾವನೆಯಿಂದ ರಾಜ್ಯ ಭಾರಮಾಡು. ಮಹಾಮಹಿಮರಾದ ಸನಕಾದಿ ಮಹರ್ಷಿಗಳು ನಿನ್ನ ಬಾಗಿಲಿಗೆ ಬರುತ್ತಾರೆ” ಎಂದು ಹೇಳಿ, ಬೇಕಾದ ವರವನ್ನು ಕೇಳಿಕೊಳ್ಳುವಂತೆ ಆತನಿಗೆ ತಿಳಿಸಿದನು.

ದೇವದೇವನ ಅಮೃತವಾಣಿಯಿಂದ ಧನ್ಯನಾದ ಪೃಥುಚಕ್ರವರ್ತಿಯು ಆತನಿಗೆ ಭಕ್ತಿ ಯಿಂದ ಅಡ್ಡ ಬಿದ್ದನು. ದೇವೇಂದ್ರನು ತಾನು ಮಾಡಿದ ತಪ್ಪಿಗಾಗಿ ನಾಚಿ, ಚಕ್ರವರ್ತಿಯ ಕಾಲಿಗೆ ಅಡ್ಡಬೀಳಲು, ಪೃಥುಚಕ್ರವರ್ತಿಯು ಆತನನ್ನು ಮೇಲಕ್ಕೆತ್ತಿ ಆದರದಿಂದ ಆಲಿಂಗಿಸಿಕೊಂಡನು. ಅನಂತರ ಆತನು ಭಗವಂತನ ಇದಿರಿಗೆ ಭಕ್ತಿಯಿಂದ ಕೈಮುಗಿದು ನಿಂತು, ಕಣ್ಣಿನಲ್ಲಿ ಆನಂದಬಾಷ್ಪಗಳನ್ನು ಸುರಿಸುತ್ತಾ, ಗದ್ಗದ ಸ್ವರದಿಂದ ‘ಹೇ, ಮುಕ್ತಿ ದಾಯಕನಾದ ಮುಕುಂದ, ನಾನು ನಿನ್ನಲ್ಲಿ ಯಾವ ವರಗಳನ್ನು ಬೇಡಲಿ? ನಿನ್ನ ಭಕ್ತರ ಬಾಯಿಂದ ಬರುವ ನಿನ್ನ ಗುಣಾಮೃತದ ಮುಂದೆ ಮೋಕ್ಷವೂ ಕೀಳುವಸ್ತು. ಆದ್ದರಿಂದ ನಿನ್ನ ದಿವ್ಯಕಥಾಮೃತವನ್ನು ಪಾನಮಾಡುವುದಕ್ಕೆ ಅನುಕೂಲವಾಗುವಂತೆ ನನಗೆ ಹತ್ತು ಸಹಸ್ರ ಕಿವಿಗಳನ್ನು ದಯಪಾಲಿಸು’ ಎಂದನು. ಇದನ್ನು ಕೇಳಿದ ಪರಮಾತ್ಮನಿಗೆ ಅತ್ಯಂತ ಸಂತೋಷವಾಯಿತು. ‘ಅಯ್ಯಾ, ನಿನ್ನ ಭಕ್ತಿಯ ಮಹಿಮೆಯಿಂದ ನೀನು ಮಾಯೆಯನ್ನು ಕೂಡ ದಾಟಿರುವೆ. ನನಗೆ ತುಂಬ ಸಂತೋಷವಾಯಿತು. ನೀನು ರಮಾದೇವಿಯಂತೆ ನನ್ನ ಸಾಯುಜ್ಯ ಪದವಿಯನ್ನು ಪಡೆಯುವೆ. ಆದರೆ ನೀನು ಇನ್ನೂ ಕೆಲಕಾಲ ಇಲ್ಲಿಯೇ ಇದ್ದು ಪ್ರಜೆಗಳನ್ನು ಸಲಹುತ್ತಿರಬೇಕು’ ಎಂದು ಹೇಳಿ ಅಂತರ್ಧಾನವಾದನು. ಪೃಥುಚಕ್ರ ವರ್ತಿಯು ತನಗಾದ ದೇವದರ್ಶನವನ್ನೆ ಮೆಲುಕು ಹಾಕುತ್ತಾ, ತನ್ನ ರಾಜಧಾನಿಗೆ ಹಿಂದಿರುಗಿದನು.

ಪೃಥುಚಕ್ರವರ್ತಿಯು ಭಗವಂತನ ಅಪ್ಪಣೆಯನ್ನು ತಲೆಯಲ್ಲಿ ಹೊತ್ತು, ರಾಜ್ಯಭಾರ ದಲ್ಲಿ ನಿರತನಾಗಿರಲು, ದೇವದೇವನು ಸೂಚಿಸಿದ್ದಂತೆ ಸನಕಸನಂದಾದಿ ಮಹರ್ಷಿಗಳು ಒಂದು ದಿನ ಆತನ ಬಳಿಗೆ ಆಗಮಿಸಿದರು. ಸೂರ್ಯನಂತೆ ತೇಜಸ್ಸಿನಿಂದ ತೊಳಗಿ ಬೆಳಗು ತ್ತಿದ್ದ ಆ ಋಷಿಗಳು, ತಮ್ಮ ತೇಜಸ್ಸಿನಿಂದಲೆ ಲೋಕವನ್ನೆಲ್ಲ ಪಾವನಗೊಳಿಸುವವರಂತೆ ಆಕಾಶದಿಂದ ಕೆಳಗೆ ಇಳಿದು ಬಂದು ಇದಿರಿಗೆ ನಿಲ್ಲುತ್ತಲೆ, ಪೃಥುಚಕ್ರಿಯು ಕುಳಿತಿ ದ್ದೆಡೆಯಿಂದ ದಿಗ್ಗನೆ ಮೇಲಕ್ಕೆದ್ದು, ಅವರಿಗೆ ಅಡ್ಡ ಬಿದ್ದನು. ಅನಂತರ ಅವರ ಕೈಕಾಲು ಗಳನ್ನು ತೊಳೆದು, ರತ್ನಸಿಂಹಾಸನದಲ್ಲಿ ಕುಳ್ಳಿರಿಸಿ, ‘ಸ್ವಾಮಿ, ನಾನೆಂತಹ ಪುಣ್ಯಶಾಲಿ! ಯೋಗಿಗಳಿಗೆ ಕೂಡ ಸಿಕ್ಕಲಾರದ ದರ್ಶನ ತಾನಾಗಿಯೇ ಆಗುವುದೆಂದರೆ ಸಾಮಾನ್ಯವೇ? ಆತ್ಮಾನಂದದಲ್ಲಿ ಸದಾ ಸುಖಿಗಳಾದ ನೀವು ನಮ್ಮಂತಹ ಸಂಸಾರಿಗಳ ಬಳಿಗೆ ಬಂದೊಡ ನೆಯೆ ನಾವೆಲ್ಲ ಪವಿತ್ರರಾಗಿ ಹೋಗುತ್ತೇವೆ. ನನ್ನಂತೆ ಸಂಸಾರದಲ್ಲಿ ತೊಳಲುತ್ತಿರುವ ಪ್ರಾಣಿಗಳಿಗೆ ನಿಮ್ಮ ಉಪದೇಶವೇ ಉದ್ಧಾರಮಾರ್ಗ. ಆದ್ದರಿಂದ ನನಗೆ ಜ್ಞಾನೋಪದೇಶ ವನ್ನು ಮಾಡಿ’ ಎಂದು ಬೇಡಿದನು. ಆತನ ಇಷ್ಟದಂತೆ ಸನತ್ಕುಮಾರನು ಉಪದೇಶವ ನ್ನಿತ್ತ. ಅದರ ಸಾರ ಇಷ್ಟು–‘ಸಜ್ಜನರ ಸಹವಾಸದಿಂದ ಭಕ್ತಿ ಹುಟ್ಟುತ್ತದೆ. ಪರಮಾತ್ಮ ನಲ್ಲಿ ಭಕ್ತಿ, ಇತರ ವಸ್ತುಗಳಲ್ಲಿ ವಿರಕ್ತಿ–ಇದೇ ಮೋಕ್ಷಸಾಧನ. ಈ ಭಕ್ತಿ ಮತ್ತು ವೈರಾಗ್ಯಗಳ ಬಲದಿಂದ ನಮ್ಮ ಶರೀರಕ್ಕೆ ಬೀಜರೂಪವಾಗಿರುವ ಅಜ್ಞಾನವನ್ನು ಸುಟ್ಟು ಹಾಕಿಬಿಡಬಹುದು. ಭಗವಂತನಿಗೆ ನೀನು ಶರೀರವೆಂದೂ, ಒಳಗೆ ಭಗವಂತನಿರುವ ನೆಂದೂ ನೆರೆ ನಂಬಿ ‘ಪರಬ್ರಹ್ಮವೇ ನಾನು’ ಎಂದು ಭಾವಿಸು. ಮತ್ತೆ ಮತ್ತೆ ಅದನ್ನು ಪರಿಭಾವಿಸಿ ದೃಢಪಡಿಸಿಕೊ. ಸಹಜಸ್ಥಿತಿಯಲ್ಲಿ ಆತ್ಮನಿಗೆ ಪ್ರಕೃತಿಯ ಸಂಬಂಧವೇನೂ ಇಲ್ಲ. ಕರ್ಮಸಂಬಂಧವಾದ ಪ್ರಕೃತಿ ವಿಕಾರಗಳು ಆತನಲ್ಲಿ ಇರುವಂತೆ ತೋರುತ್ತದೆ. ಭಗವಂತನ ಧ್ಯಾನದಿಂದ ಕರ್ಮವಾಸನೆಯ ಗಂಟು ಬಿಚ್ಚಿಹೋಗುತ್ತದೆ. ಆದ್ದರಿಂದ ಭಗವಂತನನ್ನು ಶರಣುಹೊಂದಿ ಮುಕ್ತನಾಗಬೇಕು. ಸಂಸಾರವೆಂಬ ಮಹಾಸಮುದ್ರ ವನ್ನು ದಾಟುವುದಕ್ಕೆ ಭಗವಂತನೆಂಬ ಹಡಗು ಬೇಕೇ ಬೇಕು.’

ಸನತ್ಕುಮಾರನ ಉಪದೇಶವಾಣಿಯಿಂದ ಸಂತುಷ್ಟನಾದ ಪೃಥುಚಕ್ರಿ ತನ್ನನ್ನೂ ತನ್ನ ದಾದ ಸಮಸ್ತವನ್ನೂ ಆ ಋಷಿಗಳ ಪಾದಕ್ಕೆ ಅರ್ಪಿಸಿದನು. ಆದರೆ ನಿತ್ಯಮುಕ್ತರಾದ ಅವರಿಗೆ ಅವೇಕೆ? ಅವರು ಆತನನ್ನು ಕೊಂಡಾಡುತ್ತಾ, ಆಕಾಶಮಾರ್ಗವಾಗಿ ಹೊರಟು ಕಣ್​ಮರೆಯಾದರು. ಅಲ್ಲಿಂದ ಮುಂದೆ ಪೃಥುಚಕ್ರಿಯು ಬಹುಕಾಲ ರಾಜ್ಯಭಾರ ಮಾಡುತ್ತಾ ಇದ್ದನು. ಪ್ರಜೆಗಳ ಕಾಮಧೇನುವಂತಿದ್ದ ಆ ಮಹಾನುಭಾವನು ತಾಳ್ಮೆಯಲ್ಲಿ ಭೂಮಿಯಂತೆ, ದುಷ್ಟಶಿಕ್ಷಣದಲ್ಲಿ ಯಮನಂತೆ, ವೀರ್ಯದಲ್ಲಿ ರುದ್ರನಂತೆ, ಧೈರ್ಯದಲ್ಲಿ ಸಿಂಹದಂತೆ ಇದ್ದನು. ಸಮುದ್ರದಂತೆ ಗಂಭೀರನಾಗಿದ್ದ ಆತನು ಪ್ರಭುತ್ವದಲ್ಲಿ ಸಾಕ್ಷಾತ್ ಬ್ರಹ್ಮನಂತೆಯೂ ಬುದ್ಧಿಯಲ್ಲಿ ಬೃಹಸ್ಪತಿಯಂತೆಯೂ ಇದ್ದನಾದ್ದರಿಂದ ಆತನ ಕೀರ್ತಿ ಜಗತ್ತನ್ನೆಲ್ಲ ತುಂಬಿ ವ್ಯಾಪಿಸಿತು. ಆತನು ಹುಟ್ಟಿ ಮಾಡಬೇಕಾಗಿದ್ದ ಕಾರ್ಯಗಳನ್ನೆಲ್ಲ ಮಾಡಿ ಮುಗಿಸಿದಂತಾಯಿತು. ಆತನು ತನ್ನ ರಾಜ್ಯವನ್ನು ಮಕ್ಕಳಿಗೆ ವಹಿಸಿ, ತನ್ನ ಮಡದಿಯೊಡನೆ ತಪೋವನಕ್ಕೆ ಪ್ರಯಾಣ ಮಾಡಿದನು. ಅಲ್ಲಿ ಆತನು ಕಠೋರವಾದ ತಪಸ್ಸನ್ನು ಕೈ ಕೊಂಡನು. ಅದರ ಕಾವಿನಲ್ಲಿ ಆತನ ಕರ್ಮಗಳೆಲ್ಲ ಕರಗಿಹೋದವು. ಆತನ ಮನಸ್ಸಿನಲ್ಲಿ ಕ್ರಮೇಣ ನಿಶ್ಚಲ ಭಕ್ತಿ ನೆಲಸಿತು, ಆತ್ಮಜ್ಞಾನ ಮೂಡಿತು. ಆತನು ಯೋಗಾಭ್ಯಾಸದಿಂದ ಪ್ರಾಣವಾಯುವನ್ನು ಬ್ರಹ್ಮರಂಧ್ರದ ಮೂಲಕ ಮಹಾವಾಯುವಿನೊಡನೆ ಸೇರಿಸಿ ಬಿಟ್ಟನು. ಆತನ ಪತ್ನಿಯಾದ ಅರ್ಚಿಯು ಗಂಡನ ದೇಹವನ್ನು ಚಿತೆಯ ಮೇಲಿಟ್ಟು, ತಾನೂ ಆತನೊಡನೆ ಸಹಗಮನ ಮಾಡಿದಳು. ಸುಪ್ರೀತರಾದ ದೇವತೆಗಳು ಆಕೆಯ ಮೇಲೆ ಹೂಮಳೆಗರೆದರು, ಆಕೆಯನ್ನು ಸ್ತೋತ್ರಮಾಡಿದರು.

