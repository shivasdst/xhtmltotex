
\chapter{೯೪. ಕಲಿಯುಗದ ರಾಜವಂಶಗಳು}

ಭಾಗವತ ಕಥೆಯನ್ನು ನಿರೂಪಿಸುತ್ತಿರುವ ಶುಕಮುನಿಗಳನ್ನೂ, ಭಕ್ತಿಯಿಂದ ಅದನ್ನು ಆಲಿಸುತ್ತಾ ಕುಳಿತಿರುವ ಪರೀಕ್ಷಿದ್ರಾಜನನ್ನೂ ನಾವೀಗ ಸಂದರ್ಶಿಸೋಣ. ಶ್ರೀಕೃಷ್ಣನು ಪರಂಧಾಮಕ್ಕೆ ತೆರಳಿದುದನ್ನು ಕೇಳಿದ ಪರೀಕ್ಷಿತನು ಶುಕಮಹರ್ಷಿಗಳನ್ನು ಕುರಿತು ‘ಭಗವನ್, ಶ್ರೀಕೃಷ್ಣ ಪರಮಾತ್ಮನ ನಂತರ ಚಂದ್ರವಂಶದ ಗತಿಯೇನಾಯಿತು? ಆ ವಂಶ ಮುಂದುವರಿಯಿತೊ, ಅಲ್ಲಿಗೇ ನಿಂತು ಹೋಯಿತೊ?’ ಎಂದು ಕೇಳಿದನು. ತ್ರಿಕಾಲ ಜ್ಞಾನಿಯಾದ ಶುಕಮಹರ್ಷಿಯು, ಭವಿಷ್ಯತ್ತಿನ ತೆರೆಯನ್ನು ಓರೆಮಾಡಿ, ಬಗೆಗಣ್ಣಿನಿಂದ ನೋಡುತ್ತಾ ನುಡಿದನು–“ಮಹಾರಾಜ, ಪುರುವಂಶದ ಪರಂಪರೆಯಲ್ಲಿ ಮಾರ್ಜಾರ ನೆಂಬ ರಾಜನು ಹುಟ್ಟುವನು. ಅವನ ಇಪ್ಪತ್ತು ಜನ ಮಕ್ಕಳಲ್ಲಿ ಕಟ್ಟಕಡೆಯವನಾದ ಪುರಂಜಯನಿಂದ ಚಂದ್ರವಂಶ ಮುಂದುವರಿಯುವುದು. ಆತ ಕೆಲಕಾಲ ರಾಜ್ಯಭಾರ ಮಾಡುವಷ್ಟರಲ್ಲಿ ಸ್ವಾಮಿದ್ರೋಹಿಯಾದ ಶುನಕನೆಂಬ ಮಂತ್ರಿಯು ಆತನನ್ನು ಕೊಂದು, ಆತನ ಮಗನಾದ ಪ್ರದ್ಯೋತನೆಂಬ ಬಾಲಕನನ್ನು ಸಿಂಹಾಸನದಲ್ಲಿ ಕೂಡಿಸುತ್ತಾನೆ. ಆ ಪ್ರದ್ಯೋತನೂ ಆತನ ಪೀಳಿಗೆಯವರೂ ನೂರಮುವತ್ತೆಂಟುವರ್ಷ ರಾಜ್ಯಭಾರಮಾಡಿದ ಮೇಲೆ, ಅದೇ ವಂಶಕ್ಕೆ ಸೇರಿದ ಶಿಶುನಾಗನೆಂಬುವನು ಸಿಂಹಾಸನವನ್ನು ಆಕ್ರಮಿಸುವನು. ಈ ಶಿಶುನಾಗನೂ ಆತನ ಪೀಳಿಗೆಯವರೂ ಮೂನ್ನೂರು ಅರವತ್ತು ವರ್ಷ ರಾಜ್ಯವಾಳಿದ ಮೇಲೆ ಮಹಾನಂದನೆಂಬುವನು ಪಟ್ಟವೇರುವನು. ಈತನವರೆಗೆ ಮಾತ್ರ ಚಂದ್ರವಂಶ ತನ್ನ ಶುದ್ಧತೆಯನ್ನು ಉಳಿಸಿಕೊಳ್ಳುವುದೆಂದು ಹೇಳಬಹುದು. ಇಲ್ಲಿಗೆ ನಿನ್ನಿಂದ ಮುಂದೆ ಒಂದು ಸಾವಿರದ ಒಂದುನೂರು ಹದಿನೈದು ವರ್ಷಗಳವರೆಗೆ ಚಂದ್ರವಂಶ ಬೆಳೆದುಕೊಂಡು ಹೋಗುವುದು.

“ಮಹಾನಂದನ ನಂತರ ಆತನ ಶೂದ್ರರಾಣಿಯಲ್ಲಿ ಹುಟ್ಟಿದ ನಂದನು ರಾಜನಾಗು ವನು. ಈತನು ಮಹಾಪರಾಕ್ರಮಿ; ಎರಡನೆಯ ಪರಶುರಾಮನಂತೆ ಅನೇಕ ಕ್ಷತ್ರಿಯರ ವಂಶಗಳನ್ನು ಸಮೂಲವಾಗಿ ನಾಶಮಾಡುವನು. ಆತನೂ ಆತನ ಎಂಟು ಮಂದಿ ಮಕ್ಕಳೂ ನವನಂದರೆಂಬ ಹೆಸರಿನಿಂದ ಪ್ರಖ್ಯಾತರಾಗುವರು. ಅವರು ಕೌಟಿಲ್ಯನೆಂಬ ಬ್ರಾಹ್ಮಣನನ್ನು ಆಶ್ರಯಿಸಿ ಮೇಲೇರಿದವರು, ಆತನ ಕೋಪಕ್ಕೆ ಪಾತ್ರರಾಗಿ ಮಣ್ಣುಗೂಡಿ ಹೋಗುವರು. ಈ ಕೌಟಿಲ್ಯನು ತನ್ನ ಮಗನಂತೆ ಮಮತೆಯಿಂದಿದ್ದ ಮೌರ್ಯವಂಶದ ಚಂದ್ರಗುಪ್ತನನ್ನು ರಾಜಪದವಿಯಲ್ಲಿ ನೇಮಿಸುವನು. ಈ ವಂಶ ನೂರಮೂವತ್ತೇಳು ವರ್ಷ ರಾಜ್ಯಭಾರ ಮಾಡಿದ ಮೇಲೆ ಕಡೆಯ ರಾಜನಾದ ಬೃಹದ್ರಥನನ್ನು ಅವನ ಸೇನಾಪತಿ ಯಾದ ಪುಷ್ಯಮಿತ್ರನು ಕೊಂದು ತಾನೆ ರಾಜನಾಗುವನು. ಈ ಪುಷ್ಯಮಿತ್ರನ ಪೀಳಿಗೆಯಲ್ಲಿ ಕೊನೆಯವನು ಶುಂಗ. ಸ್ತ್ರೀಲೋಲನಾದ ಈ ಶುಂಗನನ್ನು ಅವನ ಮಂತ್ರಿಯಾದ ಕಣ್ವನು ಕೊಂದು, ತಾನೆ ದೊರೆಯಾಗಿ ಕಣ್ವವಂಶದ ಮೂಲಪುರುಷನಾಗುವನು. ಈ ವಂಶದ ಕೊನೆಯ ದೊರೆ ಸುಶರ್ಮ. ಅವರ ಸೇವಕನಾದ ಬಲೀಕನೆಂಬ ಆಂಧ್ರನು ನೀಚರಲ್ಲಿ ನೀಚನಾಗಿದ್ದವನು, ತನ್ನ ಸ್ವಾಮಿಯನ್ನು ಕೊಂದು ತಾನೇ ರಾಜನಾಗುವನು. ಕೆಲಕಾಲದ ಮೇಲೆ ಅವನು ಸತ್ತುಹೋಗಲು ಅವನ ತಮ್ಮನಾದ ಕೃಷ್ಣನೆಂಬುವನು ರಾಜನಾಗುವನು. ಇವನೂ ಇವನ ವಂಶದವರೂ ನಾನೂರ ಐವತ್ತಾರು ವರ್ಷಕಾಲ ರಾಜ್ಯಭಾರಮಾಡು ವರು. ಇವರಾದಮೇಲೆ ಆಭೀರ, ಗಾರ್ದಭ, ಕಂಕ, ಯವನ, ತುರುಷ್ಕ, ಮುರುಂಡ, ಬಾಹ್ಲಿಕ, ಆಂಧ್ರ, ಕೋಸಲ, ವೈಡೂರ, ನಿಷಧ–ಇತ್ಯಾದಿ ವಂಶದವರೆಲ್ಲ ರಾಜ್ಯಭಾರ ಮಾಡುತ್ತಾರೆ. ಬಾಹ್ಲಿಕರಲ್ಲಿ ವಿಶ್ವಸ್ಫೂರ್ಜಿಯೆಂಬುವನು ಪರಾಕ್ರಮಿಯಾಗಿರುವಷ್ಟೆ ಪತಿತನೂ ಆಗಿ ಪದ್ಮಾವತಿಯೆಂಬಲ್ಲಿ ರಾಜ್ಯಭಾರಮಾಡುವನು. ಅವನು ಅನೇಕ ರಾಜ ರನ್ನು ಕೊಂದು, ರಾಜ್ಯವನ್ನು ವಿಸ್ತರಿಸುವನು. ಇದರ ಜೊತೆಗೆ ತನ್ನ ಪ್ರಜೆಗಳನ್ನು ಬಹು ವಾಗಿ ಹಿಂಸಿಸುವನು.

“ಪರೀಕ್ಷಿದ್ರಾಜ, ಈ ರಾಜರುಗಳೆಲ್ಲ ಈ ಭೂಮಿ ತಮ್ಮದೆಂಬ ಮಮತೆಯಿಂದ ಮದಾಂಧರಾಗಿರುವರು. ಆದರೆ ಇವರಾರೂ ಭೂಮಿಯನ್ನು ತಮ್ಮ ಹಿಂದೆ ಕರೆದೊಯ್ಯು ವುದಿಲ್ಲ. ಅರಸಾದರೇನು? ಆಳಾದರೇನು? ಸತ್ತಮೇಲೆ ಎಲ್ಲರೂ ಅಮೇಧ್ಯ, ಹುಳು, ಬೂದಿ. ದೇಹವನ್ನು ಆತ್ಮವೆಂದುಕೊಳ್ಳುವರು, ಭೂಮಿ ತನ್ನದೆಂದುಕೊಳ್ಳುವರು; ದೇಹ ಭೂಮಿಗಳೆರಡನ್ನು ಬಿಟ್ಟು ಸತ್ತುಹೋಗುವರು. ಅವರನ್ನು ಕಂಡು ಭೂದೇವಿ ‘ಅಯ್ಯೋ ಬೆಪ್ಪುಗಳಿರಾ!’ ಎಂದು ಪಕಪಕನೆ ನಗುವಳು. ‘ಎಷ್ಟು ರಾಜ್ಯಗಳನ್ನು ಗೆದ್ದರೇನು? ತಮ್ಮನ್ನು ತಾವು ಗೆಲ್ಲಲಾರದವರು, ಇವರು. ಇದು ನನ್ನದು ತನ್ನದೆಂದು ಹೋರಾಡಿ ಸಾಯುತ್ತಾರೆ. ಕೊನೆಗೆ ನಾನಿದ್ದಲ್ಲಿಯೆ ನನ್ನನ್ನು ಬಿಟ್ಟು ಹೋಗುತ್ತಾರೆ. ಎಂತಹ ಶೂರರಾದರೇನು? ಮೃತ್ಯುವನ್ನು ಗೆಲ್ಲಬಲ್ಲರೆ?’–ಎಂದು ಆಕೆ ಅವರಿಗಾಗಿ ಮನದಲ್ಲಿ ಮರುಗುವಳು. ಈ ರಾಜರ ಕಥೆಗಳನ್ನೆಲ್ಲ ಕೇಳಿದಮೇಲೆ, ಈ ರಾಜ್ಯಭೋಗಗಳು ನಶ್ವರವೆಂಬ ಬುದ್ಧಿ ಯಾರಿ ಗಾದರೂ ಹುಟ್ಟಲೇಬೇಕು.

“ಮಹಾರಾಜ! ಶ್ರೀಕೃಷ್ಣನು ಈ ಜಗತ್ತನ್ನು ಬಿಟ್ಟು ಪರಂಧಾಮಕ್ಕೆ ಹೋದ ದಿನ ದಿಂದಲೆ ಈ ಕಲಿಯುಗ ಪ್ರಾರಂಭವಾಯಿತು. ಈ ಯುಗಧರ್ಮಕ್ಕೆ ತಕ್ಕಂತೆ ಧರ್ಮ, ಸತ್ಯ, ದಯೆ, ಕ್ಷಮೆ–ಇವು ಕ್ಷಯಿಸುತ್ತವೆ. ಹಣವೊಂದೆ ಜನರ ಕುಲ, ಆಚಾರ, ಧರ್ಮ ಆಗಿಹೋಗುತ್ತದೆ. ಹಣಕ್ಕೆ ಪೂಜ್ಯತೆ ಕೊಟ್ಟರೆಂದರೆ ನ್ಯಾಯ, ಧರ್ಮಗಳು ಹಣವನ್ನೆ ಅನು ಸರಿಸಬೇಕಾಗುತ್ತದೆ. ಪ್ರಾಣಿವರ್ಗ, ಧಾನ್ಯ, ಗಿಡಮರಗಳು ಅಲ್ಪಪ್ರಮಾಣವಾಗುತ್ತವೆ. ಕ್ರಯ ವಿಕ್ರಯಗಳಲ್ಲಿ ವಂಚನೆಯೆ ಪ್ರಮುಖ ಪಾತ್ರಧಾರಿಯಾಗುತ್ತದೆ. ಬಾಯಿಬಡುಕ ತನವೆ ಪಾಂಡಿತ್ಯ, ಸತ್ಯ; ಉದರಂಭರಣವೆ ಸಕಲ ಪುರುಷಾರ್ಥ; ಜನಿವಾರಧಾರಣೆಯೆ ಬ್ರಾಹ್ಮಣ್ಯ, ಉತ್ತಮ ವಸ್ತ್ರಧಾರಣೆಯೆ ಪರಮ ಪೂಜ್ಯತೆ. ವೇದೋಕ್ತವಾದ ಧರ್ಮ ನಾಶವಾಗಿ ಪಾಷಂಡಧರ್ಮ ಪ್ರಬಲಿಸುತ್ತದೆ, ವರ್ಣಗಳೆಲ್ಲ ಕಲೆತುಹೋಗಿ ಒಂದೆ ವರ್ಣ –ನೀಚವರ್ಣ–ಆಗಿಹೋಗುತ್ತದೆ. ರಾಜರು ಕಳ್ಳರಾಗಿಹೋಗುತ್ತಾರೆ. ಪ್ರಜೆಗಳು ಅವರ ಹಾದಿಯನ್ನೆ ಹಿಡಿಯುತ್ತಾರೆ. ಬಲವಂತನಾದವನು ದುರ್ಬಲನ ಪ್ರಾಣ ಹಿಂಡುತ್ತಾನೆ. ಮಾನ, ಧನ, ಪ್ರಾಣಗಳು ಅತಂತ್ರವಾಗಿ ಹೋಗುತ್ತವೆ. ಸಜ್ಜನರಾದವರು ಅಡವಿಯ ಪಾಲಾಗುತ್ತಾರೆ, ಹುಳುಗಳಂತೆ ಸಾಯುತ್ತಾರೆ. ಹೀಗೆ ದುರ್ಮಾರ್ಗವ್ರವೃತ್ತಿ ಉಲ್ಬಣ ಗೊಂಡು, ಸಜ್ಜನರು ಆರ್ತನಾದ ಮಾಡುತ್ತಿರುವಾಗ ಕಲ್ಕಿಯ ಅವತಾರವಾಗುತ್ತದೆ. ಶಂಬಲವೆಂಬ ಗ್ರಾಮದಲ್ಲಿ ವಿಷ್ಣುಯಶನೆಂಬ ಬ್ರಾಹ್ಮಣನ ಮನೆಯಲ್ಲಿ ಹುಟ್ಟುವ ಆತನು ದೇವದತ್ತವೆಂಬ ಕುದುರೆಯನ್ನೇರಿ, ಬಿರುಗಾಳಿಯಂತೆ ಭೂಮಿಯಲ್ಲಿ ತಿರುಗುತ್ತಾ ಪ್ರಜಾಪೀಡಕರಾದವರನ್ನು ಕೋಟಿಕೋಟಿ ಸಂಖ್ಯೆಯಲ್ಲಿ ಕೊಂದುಹಾಕುತ್ತಾನೆ. ಇಷ್ಟಾಗು ತ್ತಲೆ ಪ್ರಜೆಗಳಿಗೆ ನೆಮ್ಮದಿಯಾಗುತ್ತದೆ. ಆ ಭಗವಂತನ ಗಾಳಿ ಸೋಕುತ್ತಲೆ ಜನರಲ್ಲಿ ಸದ್ಬುದ್ಧಿ ಮೂಡುತ್ತಾ ಹೋಗುತ್ತದೆ. ಕಲಿಯುಗ ಹೋಗಿ ಕೃತಯುಗ ಹುಟ್ಟುತ್ತದೆ.

“ರಾಜ! ಕಲಿದೋಷದಿಂದ ತಪ್ಪಿಸಿಕೊಳ್ಳುವುದು ಅತ್ಯಂತ ಸುಲಭ–ಭಗವಂತನ ನಾಮ ಸ್ಮರಣೆ, ಕೀರ್ತನೆ, ಧ್ಯಾನ. ಭಾಗವತವನ್ನು ಆಮೂಲಾಗ್ರವಾಗಿ ಕೇಳಿರುವ ನೀನು ಕಲಿ ದೋಷಕ್ಕಾಗಿ ಸ್ವಲ್ಪವೂ ಹೆದರಬೇಕಾದುದಿಲ್ಲ. ಅಷ್ಟೇ ಅಲ್ಲ, ಪುಷಿಶಾಪದಿಂದ ಸಾಯುವ ಕಾಲವು ಹತ್ತಿರವಾಗಿದೆಯಾದರೂ ನೀನು ಹೆದರಬೇಕಾಗಿಲ್ಲ. ನಾನು ಈವರೆಗೆ ಬೋಧಿಸಿರು ವುದನ್ನೆಲ್ಲ ನೀನು ಅರ್ಥಮಾಡಿಕೊಂಡಿರುವೆಯಷ್ಟೆ? ಸಾವು ದೇಹಕ್ಕೆ ಹೊರತು ಆತ್ಮಕ್ಕಲ್ಲ. ಈ ದೇಹ ಬರುವುದಕ್ಕೆ ಮುಂಚೆಯೂ ನೀನಿದ್ದೆ, ಈ ದೇಹ ಬಿದ್ದು ಹೋದಮೇಲೂ ನೀನಿರುವೆ. ಇದನ್ನು ಮನಸ್ಸಿನಲ್ಲಿ ದೃಢಮಾಡಿಕೊ. ಈ ಮನಸ್ಸೆ ಬಂಧಮೋಕ್ಷಗಳೆರಡಕ್ಕೂ ಕಾರಣ. ಮಾಯೆ ಮನಸ್ಸನ್ನು ಕರ್ಮದತ್ತ ಎಳೆಯುತ್ತದೆ. ಮನಸ್ಸು ಪುಣ್ಯಪಾಪರೂಪವಾದ ಕರ್ಮಗಳ ಮೂಲಕ ಜೀವಾತ್ಮನಿಗೆ ದೇಹವನ್ನು ತಂದುಕೊಡುತ್ತದೆ. ಮಾಯೆ, ಮನಸ್ಸು, ದೇಹ, ಕರ್ಮ–ಇವು ಸಂಸಾರ ಬಂಧನಕ್ಕೆ ಕಾರಣಗಳು. ಮನಸ್ಸು ಪಾತ್ರೆ, ಕರ್ಮ ಎಣ್ಣೆ, ದೇಹ ಬತ್ತಿ, ದೇಹಾಭಿಮಾನ ಬೆಂಕಿ, ಇವುಗಳಿಂದಾಗುವ ಸಂಸಾರವೇ ದೀಪ. ಆದ್ದರಿಂದ ಸಂಸಾರ ಬಂಧನಕ್ಕೆ ಕಾರಣವಾದ ದೇಹಾಭಿಮಾನವನ್ನು ತೊರೆದು, ಆತ್ಮನು ಅವಿನಾಶಿ, ನಿರ್ವಿಕಾರಿ, ಸ್ವಯಂ ಪ್ರಕಾಶಿ ಎಂಬುದನ್ನು ದೃಢಪಡಿಸಿಕೊ. ಶಾಸ್ತ್ರಜ್ಞಾನದಿಂದ ವಿಮರ್ಶೆಯಿಂದ ದೇವರ ಧ್ಯಾನದಿಂದ ಇದನ್ನು ದೃಢಪಡಿಸಿಕೊಂಡರೆ ನಿನಗಾಗ ಸಾವೆಲ್ಲಿಯದು? ‘ಪರಬ್ರಹ್ಮನೆ ನಾನು, ನಾನೆ ಪರಬ್ರಹ್ಮ’ ಎಂದು ಧ್ಯಾನಿಸುತ್ತಾ ಆನಂದಮಗ್ನನಾಗಿರು. ಮೃತ್ಯುವಿಗೂ ಮೃತ್ಯುವಾದ ಪರಮಾತ್ಮನೆ ನೀನಾಗಿರುವಾಗ ನಿನಗೆ ಯಾರ ಭಯವೆಲ್ಲಿಯದು?” ಎಂದನು.

ಇಷ್ಟು ಹೇಳಿದುದಾದ ಮೇಲೆ ಶುಕ ಮಹರ್ಷಿಯು “ಅಯ್ಯಾ, ನಾನು ಹೇಳಬೇಕಾದು ದನ್ನೆಲ್ಲ ಹೇಳಿರುವೆನು. ಇನ್ನೇನಾದರೂ ಕೇಳಬೇಕೆಂದಿರುವೆಯ?” ಎಂದನು. ಪರೀಕ್ಷಿ ದ್ರಾಜನು ಮೇಲಕ್ಕೆದ್ದು, ಆ ಪುಷಿಯ ಪಾದಗಳನ್ನು ಮುಟ್ಟಿ ದೀರ್ಘದಂಡ ನಮಸ್ಕಾರ ಹಾಕಿದ ಮೇಲೆ, ಕೈಮುಗಿದುಕೊಂಡು “ಸ್ವಾಮಿ, ಭಾಗವತವನ್ನು ಆಮೂಲಾಗ್ರವಾಗಿ ತಿಳಿಸಿ ನನ್ನನ್ನು ಉದ್ಧರಿಸಿದಿರಿ. ನಾನು ಧನ್ಯನಾದೆ. ನನಗಿನ್ನು ಮೃತ್ಯುಭಯವಿಲ್ಲ. ನನಗಿನ್ನು ತಾವು ಅಪ್ಪಣೆ ಕೊಡುವುದಾದರೆ, ನಾನು ಮನಸ್ಸ,ನ್ನು ಭಗವಂತನಲ್ಲಿ ನೆಲೆಗೊಳಿಸಿ, ದೇಹಾವ ಸಾನವನ್ನು ನಿರೀಕ್ಷಿಸುತ್ತಾ ಕುಳಿತುಕೊಳ್ಳುತ್ತೇನೆ” ಎಂದನು. ಪೂಜ್ಯನಾದ ಶುಕ ಮಹರ್ಷಿಯು “ತಥಾಸ್ತು” ಎಂದು ಹೇಳಿ, ಆ ರಾಜನನ್ನು ಆಶೀರ್ವದಿಸಿ, ಅಲ್ಲಿಂದ ಹೊರಟುಹೋದನು.

