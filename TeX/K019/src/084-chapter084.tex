
\chapter{೮೪. ಅತ್ತಿಗೆ ನಾದಿನಿಯರ ಸರಸ ಸಲ್ಲಾಪ}

ನಂದಗೋಪನೂ ಆತನ ಪರಿವಾರದವರೂ ಹೊರಟುಹೋದಮೇಲೆ ಜನಸಂದಣಿ ಯೆಲ್ಲ ತಗ್ಗಿದಂತಾಗಿ, ಪಾಂಡವರು ಶ್ರೀಕೃಷ್ಣನ ಸುತ್ತಲೂ ಕುಳಿತು ಆತ್ಮೀಯವಾದ ಮಾತುಕಥೆಗೆ ಪ್ರಾರಂಭಿಸಿದರು. ಆ ಸಮಯದಲ್ಲಿ ದ್ರೌಪದಿಯು ಶ್ರೀಕೃಷ್ಣನ ಮಡದಿ ಯರ ಮಧ್ಯದಲ್ಲಿ ಕುಳಿತು ಸರಸ ಸಲ್ಲಾಪದಲ್ಲಿ ಮಗ್ನಳಾಗಿದ್ದವಳು ‘ಅತ್ತಿಗೆಯರಿರಾ, ನಮ್ಮಣ್ಣ ಸಾಕ್ಷಾತ್ ಮಹಾವಿಷ್ಣು. ಆತನನ್ನು ಕೈಹಿಡಿದ ನೀವು ಧನ್ಯರೇ ಸರಿ. ನಿಮ್ಮಲ್ಲಿ ಒಬ್ಬೊಬ್ಬರನ್ನೂ ಆತ ಮಹಾಸಾಹಸದಿಂದಲೆ ಸಂಪಾದಿಸಿದನೆಂದು ಕೇಳಿದ್ದೇನೆ. ನಿಮ್ಮ ನಿಮ್ಮ ವಿವಾಹದ ಕಥೆಯನ್ನು ಕೇಳಬೇಕೆಂದು ನನ್ನ ಬಯಕೆ. ಆ ಚೆಲುವ ಚೆನ್ನಿಗನನ್ನು ನೀವು ಕೈಹಿಡಿದ ಕಥೆಯನ್ನು ಹೇಳಿರೆ, ಕೇಳೋಣ’ ಎಂದಳು. ಹಿರಿಯ ರಾಣಿಯಾದ ರುಕ್ಮಿಣಿ ಎರಡೆ ಮಾತಿನಲ್ಲಿ ತನ್ನ ಮದುವೆಯ ಕಥೆ ಹೇಳಿದಳು–‘ಅಮ್ಮ, ಜರಾಸಂಧನೇ ಮೊದ ಲಾದ ಅನೇಕ ರಾಜರು ನನ್ನನ್ನು ಶಿಶುಪಾಲನಿಗೇ ಕೊಡಿಸಬೇಕೆಂದು ಹಟ ಹಿಡಿದು ಕುಳಿತಿ ದ್ದರು. ಆದರೆ ಶ್ರೀಕೃಷ್ಣಪರಮಾತ್ಮ ಮೇಕೆಯ ಮರಿಗಳ ಮಧ್ಯದಲ್ಲಿರುವ ಆಹಾರವನ್ನು ಸಿಂಹ ಹೊತ್ತುಕೊಂಡು ಹೋಗುವಂತೆ, ಅವರ ಮಧ್ಯದಿಂದ ನನ್ನನ್ನು ಉದ್ಧರಿಸಿ ಕರೆ ತಂದು ತನ್ನ ಚರಣದಾಸಿಯಾಗಿ ಮಾಡಿಕೊಂಡು ಅನುಗ್ರಹಿಸಿದ; ನಾನು ಧನ್ಯಳಾದೆ’ ಎಂದಳು. ಆಗ ಸತ್ಯಭಾಮೆ ‘ಪಾಂಚಾಲಿ, ನನ್ನ ಚಿಕ್ಕಪ್ಪ, ನಮ್ಮಪ್ಪನ ಸ್ಯಮಂತಕಮಣಿ ಯನ್ನು ಧರಿಸಿ, ಬೇಟೆಗೆ ಹೋಗಿ ಸತ್ತ. ತಿಳಿಯದ ಜನ ಶ್ರೀಕೃಷ್ಣನೆ ಮಣಿಗಾಗಿ ಅವನನ್ನು ಕೊಂದಿರಬೇಕೆಂದು ದೂರಿದರು. ಆಗ ಶ್ರೀಕೃಷ್ಣ ನಮ್ಮ ಚಿಕ್ಕಪ್ಪನನ್ನು ಹುಡುಕುತ್ತಾ ಹೋಗಿ, ಜಾಂಬವಂತನನ್ನು ಜಯಿಸಿ, ಆತನಲ್ಲಿದ್ದ ಸ್ಯಮಂತಕಮಣಿಯನ್ನು ತಂದು ನಮ್ಮಪ್ಪನಿಗೆ ಕೊಟ್ಟ. ಆಗ ನಮ್ಮಪ್ಪ ಸಂತೋಷದಿಂದ ಆ ಮಣಿಯೊಡನೆ ನನ್ನನ್ನೂ ಶ್ರೀಕೃಷ್ಣನಿಗೆ ಕೊಟ್ಟ’ ಎಂದಳು. ಆಗ ಜಾಂಬವತಿ ‘ಕೃಷ್ಣೆ, ನನ್ನ ಕಥೆಯನ್ನು ಸತ್ಯಭಾಮೆಯೆ ಹೇಳಿದ್ದಾಳೆ. ನಮ್ಮ ತಂದೆಯಾದ ಜಾಂಬವಂತ ಶ್ರೀರಾಮನ ಪರಮಭಕ್ತ. ಶ್ರೀಕೃಷ್ಣನೆ ರಾಮನೆಂದು ತಿಳಿಯದೆ ಆತನೊಡನೆ ಇಪ್ಪತ್ತೇಳು ದಿನ ಕಾಳಗವಾಡಿ, ಆತನೇ ರಾಮನೆಂದು ತಿಳಿಯುತ್ತಲೆ ಸ್ಯಮಂತಕಮಣಿಯೊಡನೆ ನನ್ನನ್ನು ಆತನಿಗೆ ಒಪ್ಪಿಸಿದ. ನಾನು ಭಗವಂತನ ದಾಸಿಯಾದೆ’ ಎಂದಳು. ಅನಂತರ ಕಾಳಿಂದಿ ‘ದ್ರೌಪದೀ ದೇವಿ, ನಾನು ಸೂರ್ಯನ ಮಗಳು. ನಾನು ಭಗವಂತನೆ ಪತಿಯಾಗಬೇಕೆಂದು ತಪಸ್ಸು ಮಾಡುತ್ತಿದ್ದೆ. ಕೃಪಾಳುವಾದ ಭಗ ವಂತನೆ ಅರ್ಜುನನೊಡನೆ ನಾನಿದ್ದಲ್ಲಿಗೆ ಬಂದು ನನ್ನ ಕೈಹಿಡಿದು ಉದ್ಧರಿಸಿದ’ ಎಂದಳು. ಭದ್ರೆಯು ‘ಅಗ್ನಿಪುತ್ರಿ, ನನ್ನ ಸ್ವಯಂವರಕಾಲದಲ್ಲಿ ಶ್ರೀಕೃಷ್ಣನು ನಾಯಿಗಳ ಮಧ್ಯದ ಲ್ಲಿದ್ದ ಆಹಾರವನ್ನು ಸಿಂಹ ಕೊಂಡೊಯ್ಯುವಂತೆ, ಕೀಳು ರಾಜರ ಮಧ್ಯದಿಂದ ನನ್ನನ್ನು ಕರೆತಂದು ಕೈಹಿಡಿದ. ನನ್ನ ಅಪೇಕ್ಷೆ ಫಲಿಸಿತು. ನಾನು ಕೃತಕೃತ್ಯೆ’ ಎಂದಳು. ಅನಂತರ ಸತ್ಯೆಯು ‘ಯಾಜ್ಞಸೇನಿ, ನಮ್ಮ ತಂದೆ ರಾಜರ ಸತ್ವಪರೀಕ್ಷೆಗಾಗಿ ಪಣವಾಗಿಟ್ಟಿದ್ದ ಏಳು ಗೂಳಿಗಳನ್ನು ಶ್ರೀಕೃಷ್ಣನು ತನ್ನ ಪರಾಕ್ರಮದಿಂದ, ಎಳೆಯ ಮಕ್ಕಳು ಆಡಿನ ಮರಿಯನ್ನು ಹಿಡಿದು ಕಟ್ಟುವಂತೆ, ಕಟ್ಟಿಹಾಕಿದನಮ್ಮ. ಇದನ್ನು ಸಹಿಸದೆ ಅನೇಕ ರಾಜರು ಆತನ ಮೇಲೆ ಏರಿಬಂದರು. ಆದರೇನು? ನೀರೊಳ್ಳೆಗಳು ಗರುಡನನ್ನೇನು ಮಾಡಬಲ್ಲವಮ್ಮ? ನನ್ನ ಸ್ವಾಮಿ ಅವರನ್ನೆಲ್ಲ ಬಡಿದೋಡಿಸಿ, ನನ್ನ ಕೈಹಿಡಿದು ಕೃತಾರ್ಥಳನ್ನಾಗಿ ಮಾಡಿದನಮ್ಮ’ ಎಂದಳು. ಮಿತ್ರವಿಂದೆ ‘ಅಮ್ಮ, ನನ್ನದೇನೂ ಕಥೆಯಿಲ್ಲ. ನನಗೆ ಕೃಷ್ಣಸ್ವಾಮಿಯಲ್ಲಿ ಮೊದಲಿಂದ ಮೋಹವಿತ್ತು. ಆತನು ನಮಗೆ ನೆಂಟನೂ ಆಗಬೇಕು. ನಮ್ಮಪ್ಪ ನನ್ನ ಇಷ್ಟ ದಂತೆ ಶ್ರೀಕೃಷ್ಣನನ್ನು ಕರೆಸಿ, ನನ್ನನ್ನು ಮದುವೆ ಮಾಡಿಕೊಟ್ಟ’ ಎಂದಳು.

ಶ್ರೀಕೃಷ್ಣನ ಅಷ್ಟಮಹಿಷಿಯರಲ್ಲಿ ಕಡೆಯವಳು ಲಕ್ಷಣೆ. ಆಕೆಗೆ ತನ್ನ ಮದುವೆಯ ಕಥೆಯನ್ನು ಹೇಳುವುದರಲ್ಲಿ ತಂಬ ಉತ್ಸಾಹ, ಹೆಮ್ಮೆ. ಆಕೆ ‘ದ್ರೌಪದಿ, ನನ್ನ ಮದುವೆಯ ಕಥೆ ಉಳಿದವರ ಕಥೆಯಂತಲ್ಲ. ಕೇಳು, ನಾನು ಆಗಾಗ ನಾರದರಿಂದ ಶ್ರೀಕೃಷ್ಣನ ರೂಪ, ಗುಣ, ಪರಾಕ್ರಮಗಳನ್ನು ಕೇಳಿ, ಮನಸಾ ಆತನನ್ನೆ ವರಿಸಿದ್ದೆ. ಇದನ್ನು ತಿಳಿದ ನನ್ನ ತಂದೆ ಬೃಹತ್ಸೇನನು ನನಗೆ ಶ್ರೀಕೃಷ್ಣನ ಪಾದಸೇವೆ ದೊರೆಯುವಂತೆ ಒಂದು ಉಪಾಯ ಮಾಡಿದ. ನಿನ್ನ ತಂದೆ ನಿನ್ನ ಸ್ವಯಂವರದಲ್ಲಿ ಒಂದು ಮತ್ಸ್ಯಯಂತ್ರವನ್ನು ಏರ್ಪಡಿಸಿದ್ದ ನೋಡು, ಹಾಗೆ ನನ್ನ ತಂದೆಯೂ ಒಂದು ಮತ್ಸ್ಯಯಂತ್ರವನ್ನು ಮಾಡಿಸಿದ. ಅದು ನಿನ್ನದ ಕ್ಕಿಂತ ಬಹು ಕಷ್ಟವಾದ್ದು. ನಿನ್ನ ಮತ್ಸ್ಯಯಂತ್ರವನ್ನು ಹೊರಗಿನಿಂದ ನೋಡಬಹುದಾ ಗಿತ್ತು. ಆದರೆ ನಮ್ಮಪ್ಪ ಮಾಡಿಸಿದ ಯಂತ್ರ ಹೊರಗೆ ಕಾಣಿಸುತ್ತಿರಲಿಲ್ಲ; ಕೇವಲ ಅದರ ಕೆಳಗಿಟ್ಟಿದ್ದ ನೀರಿನ ಕೊಡದಲ್ಲಿ ಮಾತ್ರ ಅದರ ನೆರಳು ಕಾಣಿಸುತ್ತಿತ್ತು. ಅಮ್ಮ ಅಮ್ಮ! ರಾಜರು ಬಂದರು ಬಂದರು; ಎಷ್ಟೋ ಜನ ಬಂದು, ಆ ಬಿಲ್ಲಿಗೆ ಹಗ್ಗವನ್ನು ಕಟ್ಟುವು ದಕ್ಕೂ ಯೋಗ್ಯತೆಯಿಲ್ಲದೆ ಹೊರಟುಹೋದರು. ಕೆಲವರಂತೂ ಆ ಪ್ರಯತ್ನದಲ್ಲಿ ಕೆಳಗೆ ಬಿದ್ದು ಮೂರ್ಛೆ ಹೋದರು. ಜರಾಸಂಧ, ಶಿಶುಪಾಲ, ಕರ್ಣ, ದುರ್ಯೋಧನ–ಇವರೆಲ್ಲ ಬಂದು ಯಂತ್ರದ ಗುರಿಯನ್ನು ಹೊಡೆಯಲು ಪ್ರಯತ್ನಿಸಿದರು. ಅವರಲ್ಲಿ ಒಬ್ಬಿಬ್ಬರು ಬಿಲ್ಲಿಗೆ ಹಗ್ಗವನ್ನು ಕಟ್ಟಿದರಾದರೂ ಮತ್ಸ್ಯಯಂತ್ರವನ್ನು ಭೇದಿಸುವುದಕ್ಕೆ ಆಗಲಿಲ್ಲ. ಆಗ ಬಂದನಮ್ಮಾ, ನಮ್ಮ ಮುಗುಳ್ನಗೆಯ ಮೋಹನಾಂಗ. ಆತ ಲೀಲಾಜಾಲವಾಗಿ ಬಿಲ್ಲನ್ನು ಎಳೆದು ಹಗ್ಗವನ್ನು ಕಟ್ಟಿದ. ಒಂದು ಸಲ ನೀರಿನಲ್ಲಿ ಯಂತ್ರದ ಪ್ರತಿಬಿಂಬ ವನ್ನು ನೋಡಿದವನೆ ಬಾಣವನ್ನು ಹೂಡಿ, ಯಂತ್ರವನ್ನು ಹೊಡೆದು ಕೆಡಹಿದ. ನಾನು ತುಂಬಿದ ಸಭೆಯಲ್ಲಿ ಆ ಮೋಹನಾಂಗನಿಗೆ ಮಾಲೆಯನ್ನು ಹಾಕಿ ಆತನನ್ನು ವರಿಸಿದೆ. ಇದನ್ನು ನೋಡಿ ಸಹಿಸಲಾರದೆ ಅಲ್ಲಿ ನೆರೆದಿದ್ದ ರಾಜರೆಲ್ಲ ಆತನ ಮೇಲೆ ಯುದ್ಧಕ್ಕೆ ನಿಂತರು. ಶ್ರೀಕೃಷ್ಣ ನಗುನಗುತ್ತಲೆ ನನ್ನನ್ನು ರಥದಲ್ಲಿ ಕುಳ್ಳಿರಿಸಿ, ತಾನೂ ಪಕ್ಕದಲ್ಲಿ ಕುಳಿತುಕೊಂಡ. ಆಗ ಆತನಿಗೆ ನಾಲ್ಕು ಕೈಗಳಾದವು. ಎರಡು ಕೈಗಳಿಂದ ನನ್ನನ್ನು ಆಲಿಂಗಿಸಿ ಕೊಂಡೆ ಆತ ಇನ್ನೆರಡು ಕೈಗಳೊಡನೆ ಅವರ ಮೇಲೆ ಬಾಣಗಳ ಮಳೆಗರೆದ. ಅಯ್ಯೋ, ಆ ಮಂಕುಗಳು ಶ್ರೀಕೃಷ್ಣನ ಇದಿರಿಗೆ ನಿಲ್ಲುವುದಕ್ಕೆ ಸಾಧ್ಯವೆ? ಬಿರುಗಾಳಿಗೆ ಸಿಕ್ಕ ತರಗಲೆ ಗಳಂತೆ ಅವರು ಹಾರಿಹೋದರು. ಆಮೇಲೆ ಶ್ರೀಕೃಷ್ಣ ನಮ್ಮ ರಾಜಧಾನಿಯಾದ ಕುಶ ಸ್ಥಲಿಗೆ ಬಂದು, ನಮ್ಮ ತಂದೆಯಿಂದ ಅನೇಕ ಉಡುಗೊರೆಗಳನ್ನು ಪಡೆದು, ನಾನು ಮತ್ತು ನನ್ನ ಬಳುವಳಿಗಳೊಡನೆ ವೈಭವದಿಂದ ದ್ವಾರಕಿಗೆ ಬಂದನು. ಅಮ್ಮ ನನ್ನ ಪುಣ್ಯವೆ ಪುಣ್ಯ’ ಎಂದಳು.

ಇನ್ನುಳಿದ ಹದಿನಾರು ಸಾವಿರ ಜನ ರಾಜಪುತ್ರಿಯರೂ ಒಂದೆ ಮಾತಿನಲ್ಲಿ ತಮ್ಮ ಕಥೆ ಹೇಳಿದರು. ‘ಅಮ್ಮ, ನಾವೆಲ್ಲ ನರಕಾಸುರನ ಬಂಧಿಗಳಾಗಿದ್ದೆವು. ನಾವೆಲ್ಲ ಆತನನ್ನೆ ಪತಿ ಯಾಗುವಂತೆ ಬೇಡಿದೆವು. ಉದಾರಿಯಾದ ಆ ಭಕ್ತವತ್ಸಲ ನಮ್ಮ ಕೈಹಿಡಿದು ಉದ್ಧಾರ ಮಾಡಿದ. ಇಲ್ಲದಿದ್ದರೆ ನಮ್ಮ ಗತಿಯೇನಾಗಬೇಕಾಗಿತ್ತು! ಅಮ್ಮ, ನಮಗಾವ ಆಸೆಯೂ ಇಲ್ಲ; ಆತನ ಪಾದಧೂಳಿ ಸಿಕ್ಕರೆ ಸಾಕು’ ಎಂದರು.

ಕೃಷ್ಣನ ಮಡದಿಯರು ಹೇಳಿದ ತಮ್ಮ ತಮ್ಮ ಮದುವೆಯ ಕಥೆಗಳನ್ನು ಕೇಳಿ, ದ್ರೌಪದಿ, ಕುಂತಿ, ಸುಭದ್ರೆ ಮೊದಲಾದ ಹೆಣ್ಣುಮಕ್ಕಳೆಲ್ಲ, ಅವರಿಗೆ ತಮ್ಮ ಗಂಡನಲ್ಲಿ ಇರುವ ಪ್ರೇಮ, ಗೌರವ, ಭಕ್ತಿಗಳನ್ನು ಕಂಡು ಅಚ್ಚರಿಗೊಂಡರು, ಆನಂದಪಟ್ಟರು. ಅವರೆಲ್ಲರೂ ಸೇರಿ ಶ್ರೀಕೃಷ್ಣನ ಗುಣಗಳನ್ನು ಹೊಗಳುತ್ತಾ, ಆತನನ್ನು ಬಂಧುವಾಗಿ ಪಡೆದ ತಾವು ಧನ್ಯರೆಂದುಕೊಂಡರು.

