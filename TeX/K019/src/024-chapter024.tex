
\chapter{೨೪. ಅಜಾಮಿಳ}

ಪರೀಕ್ಷಿದ್ರಾಜ: ಗುರುದೇವ! ಅರಿತೋ ಅರಿಯದೆಯೋ ಪಾಪ ಮಾಡಿದಮೇಲೆ ನರಕ ಶಿಕ್ಷೆಯನ್ನು ಅನುಭವಿಸಬೇಕಲ್ಲವೆ? ಅದರಿಂದ ತಪ್ಪಿಸಿಕೊಳ್ಳಲು ಉಪಾಯವೇನು?

ಶುಕಮುನಿ: ರೋಗಕ್ಕೆ ತಕ್ಕ ಚಿಕಿತ್ಸೆ ಮಾಡುವಹಾಗೆ, ಪಾಪಕ್ಕೆ ತಕ್ಕ ಪ್ರಾಯಶ್ಚಿತ್ತ ಮಾಡಿ ಕೊಳ್ಳಬೇಕು.

ಪರೀಕ್ಷಿದ್ರಾಜ: ಅಹುದು, ಆನೆಯು ನೀರಿನಲ್ಲಿ ಮುಳುಗಿ, ಪುನಃ ಮಣ್ಣನ್ನು ಮೈಮೇಲೆ ಎರಚಿಕೊಳ್ಳು ವಂತೆ, ಪ್ರಾಯಶ್ಚಿತ್ತ ಮಾಡಿಕೊಂಡು ಪುನಃ ಪಾಪ ಮಾಡಿದರೆ?

ಶುಕಮುನಿ: ಅಯ್ಯಾ! ಪ್ರಾಯಶ್ಚಿತ್ತವೆಂಬುದು ರೋಗದ ಉಪಶಮನಕ್ಕಾಗಿ ತೆಗೆದು ಕೊಳ್ಳವ ಔಷಧಿ ಇದ್ದಹಾಗೆ. ಆದರಿಂದ ರೋಗ ಪೂರ್ತಿ ವಾಸಿಯಾಗುವುದೇನೂ ಇಲ್ಲ. ಕರ್ಮದಿಂದ ಕರ್ಮಕ್ಕೆ ಪರಿಹಾರವೆಂದಿಗೂ ಆಗುವುದಿಲ್ಲ. ಪ್ರಾಯಶ್ಚಿತ್ತದಿಂದ ತಾತ್ಕಾಲಿಕ ವಾದ ಪಾಪ ನೀಗುತ್ತದೆಯೇ ಹೊರತು ಹಿಂದು ಮುಂದಿನ ಕರ್ಮ ನಿವಾರಣೆಯಾಗದು.

ಪರೀಕ್ಷಿದ್ರಾಜ: ಹಾಗಾದರೆ ಕರ್ಮ ಸಮೂಲವಾಗಿ ನಾಶವಾಗುವಂತೆ ಮಾಡುವುದು ಹೇಗೆ?

ಶುಕಮುನಿ: ಮಹಾರಾಜ! ಪಾಪವೆಂಬುದು ಒಂದು ಬಗೆಯ ರೋಗವಿದ್ದ ಹಾಗೆ. ಹಿತ ಮಿತವಾದ ಆಹಾರದಿಂದ ರೋಗಕ್ಕೆ ಆಸ್ಪದವಿಲ್ಲದಂತೆ ಮಾಡಬಹುದು; ಹಾಗೆಯೇ ನಿಯಮದಿಂದ ಬಾಳುತ್ತಿದ್ದರೆ ಪಾಪಕರ್ಮದ ಸೋಂಕು ಹತ್ತುವುದಿಲ್ಲ. ತಪಸ್ಸು, ಬ್ರಹ್ಮ ಚರ್ಯ, ಶಮ, ದಮ, ತ್ಯಾಗ, ಸತ್ಯ ಶೌಚ, ಯಮ, ನಿಯಮಗಳಿಂದ ಕೂಡಿದವರು ಮನೋ ವಾಕ್ ಕಾಯ(ಮನಸ್ಸು, ಮಾತು, ದೇಹ)ಗಳಿಂದ ಆದ ಪಾಪಗಳನ್ನೆಲ್ಲ, ಕಾಡುಕಿಚ್ಚು ಬಿದಿರ ಮೆಳೆಯನ್ನು ಸುಟ್ಟು ಹಾಕುವಂತೆ, ಧ್ವಂಸ ಮಾಡಿಬಿಡುತ್ತಾರೆ. ಇದಕ್ಕಿಂತಲೂ ಸುಲಭ ವಾದುದು ಭಕ್ತಿಯೋಗ. ಮನಸ್ಸನ್ನು ಭಗವಂತನಲ್ಲಿ ನೆಲೆಗೊಳಿಸಿ, ಭಕ್ತರನ್ನು ಸೇವಿಸುತ್ತಾ ಹೋದರೆ ಅನಾಯಾಸವಾಗಿ ಮುಕ್ತಿ ದೊರಕುತ್ತದೆ. ಪ್ರಾಯಶ್ಚಿತ್ತ ಎಂಬುದು ಹೆಂಡದ ಗಡಿಗೆಯನ್ನು ಗಂಗೆಯಲ್ಲಿ ಅದ್ದಿದಹಾಗೆ! ಅದು ಅದರಿಂದ ಪವಿತ್ರವಾಗಬಲ್ಲುದೆ? ಆದ್ದರಿಂದ ಪ್ರಾಯಶ್ಚಿತ್ತಕ್ಕಿಂತ ಭಗವದ್ಭಕ್ತಿ ಎಲ್ಲ ವಿಧದಿಂದಲೂ ಅತ್ಯಂತ ಶ್ರೇಯಸ್ಕರ. ಇದಕ್ಕೆ ದೃಷ್ಟಾಂತವಾಗಿ ಒಂದು ಇತಿಹಾಸ ಕಥೆಯನ್ನು ಹೇಳುತ್ತೇನೆ, ಕೇಳು.

ಬಹು ಹಿಂದೆ ಕನ್ಯಾಕುಬ್ಜದಲ್ಲಿ ಅಜಾಮಿಳನೆಂಬ ಒಬ್ಬ ಬ್ರಾಹ್ಮಣನಿದ್ದ. ಆತನು ವೇದ ವೇದಾಂತ ಪಾರಂಗತನಾಗಿದ್ದ. ಒಳ್ಳೆಯ ಆಚಾರಶೀಲನಾಗಿದ್ದ ಆತನು ಪರಸ್ತ್ರೀಯನ್ನು ಕಣ್ಣೆತ್ತಿಯೂ ನೋಡುತ್ತಿರಲಿಲ್ಲ. ಆತನು ದಯಾಳು, ಸತ್ಯವಂತ, ಪರೋಪಕಾರಿ, ಅಸೂಯಾರಹಿತ, ಗುರುಹಿರಿಯರನ್ನೂ, ಅತಿಥಿ ಅಭ್ಯಾಗತರನ್ನೂ ದೇವರಂತೆ ಕಾಣುವನು; ಆತನ ಮಾತು ಹಿತ, ಮಿತ; ದೈವಭಕ್ತನಾದ ಆತ ಸಾವಿತ್ರಿ ಮೊದಲಾದ ಮಹಾಮಂತ್ರ ಗಳನ್ನು ಬಲ್ಲವನಾಗಿದ್ದನು. ಈ ಮಹಾಬ್ರಾಹ್ಮಣ ಒಂದು ದಿನ ತಂದೆಯ ಅಪ್ಪಣೆಯಂತೆ ಹೂ ಹಣ್ಣುಗಳನ್ನೂ ದರ್ಭೆ ಸಮಿತ್ತುಗಳನ್ನೂ ತರುವುದಕ್ಕೆಂದು ಅಡವಿಗೆ ಹೋದ. ಅಲ್ಲಿ ತನ್ನ ಕೆಲಸವನ್ನು ಮುಗಿಸಿಕೊಂಡು ಇನ್ನೇನು ಹಿಂದಿರುಗಬೇಕು, ಅಷ್ಟರಲ್ಲಿ ಸುಂದರಿ ಯಾದ ಹೆಣ್ಣೊಬ್ಬಳು ಆತನ ಕಣ್ಣಿಗೆ ಬಿದ್ದಳು. ಅವಳು ಗಂಟಲು ಮಟ್ಟ ಹೆಂಡ ಕುಡಿದು ಮತ್ತೇರಿ ಮೈಮರೆತಿದ್ದಾಳೆ. ಉಟ್ಟ ಸೀರೆ ಬಿಚ್ಚಿಹೋಗಿದ್ದರೂ ಅವಳಿಗೆ ಅದರ ಪರಿವೆ ಇಲ್ಲ. ಅವಳು ಸ್ವಲ್ಪವೂ ನಾಚಿಕೆಯಿಲ್ಲದೆ ನಗುತ್ತಾ ಹಾಡುತ್ತಾ ಕುಣಿಯುತ್ತಿದ್ದಾಳೆ. ಅವಳ ಸಮೀಪದಲ್ಲಿ ಒಬ್ಬ ಯುವಕ; ಅವನೂ ಪ್ರಜ್ಞೆತಪ್ಪುವಷ್ಟು ಕಳ್ಳು ಕುಡಿದಿದ್ದಾನೆ. ಅವರಿ ಬ್ಬರೂ ಕಾಮನ ಕೈಗೊಂಬೆಗಳಂತೆ ವಿನೋದ ವಿಹಾರದಲ್ಲಿ ಮಗ್ನರಾಗಿದ್ದಾರೆ. ಈ ದೃಶ್ಯ ವನ್ನು ಕಂಡು ಅಜಾಮಿಳನ ವಿವೇಕ ಹಾರಿ ಹೋಯಿತು; ಆತ ಆ ಹೆಣ್ಣಿಗೆ ಮರುಳಾದ. ಆತ ಎಷ್ಟು ಪ್ರಯತ್ನಮಾಡಿದರೂ ಮನಸ್ಸು ಹತೋಟಿಗೆ ಬರಲಿಲ್ಲ.

ಅಲ್ಲಿಂದ ಮುಂದೆ ಅಜಾಮಿಳನ ಬಾಳು ಹೊಸ ಹಾದಿಯನ್ನು ಹಿಡಿಯಿತು. ಆ ಹೆಣ್ಣಿನ ಚಿಂತೆಯಲ್ಲಿ ಮುಳುಗಿದ. ಅವನಿಗೆ ಸ್ನಾನ ಸಂಧ್ಯಾದಿಗಳು ಬೇಡವಾದವು. ಆ ಹೆಣ್ಣಿನ ಮನಸ್ಸನ್ನು ಒಲಿಸುವುದಕ್ಕಾಗಿ ಅವಳಿಗೆ ಅಂದವಾದ ಬಟ್ಟೆಬರೆಗಳನ್ನೂ ಒಡವೆಗಳನ್ನೂ ಕೊಂಡೊಯ್ದು ಕೊಡುವನು. ಅವಳೊಮ್ಮೆ ಕಡೆಗಣ್ಣಿನಿಂದ ತನ್ನನ್ನು ನೋಡಿದರೆ ಅವನಿಗೆ ಹಿಡಿಸಲಾರದಷ್ಟು ಸಂತೋಷ. ಹಿರಿಯರು ಗಳಿಸಿಟ್ಟ ಹಣವೆಲ್ಲ ಅವಳಿಗಾಗಿ ಕರಗಿ ಹೋಯಿತು. ಅಗ್ನಿಸಾಕ್ಷಿಯಾಗಿ ತನ್ನನ್ನು ಕೈಹಿಡಿದ ಮಡದಿಯನ್ನು ಅವನು ಮರೆತೇಬಿಟ್ಟ. ಮನೆಯ ಆಸ್ತಿಯೆಲ್ಲ ಕರಗಿಹೋಯಿತು. ತಾನು ಒಲಿದ ಹೆಣ್ಣಿಗೂ ಅವಳ ಸಂಸಾರಕ್ಕೂ ಹೊಟ್ಟೆಗೆ ತಂದುಹಾಕುವುದಕ್ಕಾಗಿ ಸುಳ್ಳು, ಮೋಸ, ಕಳ್ಳತನಕ್ಕೂ ಇಳಿದ. ಅಂತೂ ಅವಳನ್ನು ಒಲಿಸಿಕೊಂಡು, ಅವಳಲ್ಲಿಯೇ ಸ್ವರ್ಗಸುಖವನ್ನು ಕಾಣುತ್ತಿದ್ದ. 

ಕಾಲಕ್ರಮದಲ್ಲಿ ಜೂಜು, ಕುಡಿತ, ಕೊಲೆ, ಸುಲಿಗೆಗಳು ಅಜಾಮಿಳನ ವೃತ್ತಿಯಾಗಿ ಹೋದವು. ಆ ವೃತ್ತಿಯಿಂದ ಹಣ ದೊರೆಯದ ದಿನ ಪಶು ಪಕ್ಷಿ ಮೃಗಗಳನ್ನು ಕೊಂದು ತಿನ್ನುತ್ತಿದ್ದನು. ಹೀಗೆ ಅವನ ನೀಚ ಜೀವನದೊಡನೆ ಕಾಲಚಕ್ರ ಉರುಳುತ್ತಾ ಅವನಿಗೆ ಎಂಬತ್ತೆಂಟು ವರ್ಷ ವಯಸ್ಸಾಯಿತು. ಆ ವೇಳೆಗೆ ಅವನಿಗೆ ಹತ್ತು ಮಕ್ಕಳಾದರು. ಕಡೆಯ ಮಗನ ಹೆಸರು ‘ನಾರಾಯಣ’ ಎಂದು. ಅವನಮೇಲೆ ಅಜಾಮಿಳನಿಗೆ ಅತಿ ಮಮತೆ. ಅವನ ಮುದ್ದುಮಾತುಗಳನ್ನು ಕೇಳಿ ಪುಳಕಗೊಳ್ಳುವನು, ಅವನ ಆಟ ಪಾಟಗಳನ್ನು ಕಂಡು ಆನಂದಬಾಷ್ಪಗಳನ್ನು ಸುರಿಸುವನು. ಸದಾ ಅವನು ತನ್ನ ಬಳಿಯಲ್ಲಿಯೇ ಇರಬೇಕು, ತಾನು ಊಟಮಾಡುವಾಗ ಅವನಿಗೂ ತಾನೆ ತನ್ನ ಕೈಯಾರೆ ತಿನ್ನಿಸಬೇಕು, ತಾನು ಕುಡಿಯು ವಾಗಲೆಲ್ಲ ಅವನೂ ಕುಡಿಯಬೇಕು, ತಾನು ಮಲಗುವಾಗ ಅವನೂ ಮಲಗಬೇಕು. ಅವನಿಗೆ ಜಗತ್ತಿನಲ್ಲಿ ಸರ್ವಸ್ವವೂ ಆ ಮಗನೆ. ಆ ಮಗನ ಕನವರಿಕೆಯಲ್ಲಿ ಅವನಿಗೆ ಮೃತ್ಯು ದೇವತೆ ಮುಂದೆ ಬಂದು ನಿಂತರೂ ಕಾಣುವಂತಿರಲಿಲ್ಲ. ಇದನ್ನು ಕಂಡು ನಗುತ್ತಾ ಬೆನ್ನ ಹಿಂದೆ ನಿಂತಿದ್ದ ಮೃತ್ಯು, ಒಂದು ದಿನ ಇದಿರಿಗೇ ಬಂದು ನಿಂತಿತು. ಭಯಂಕರಾಕಾರದ ಯಮದೂತರು ಕೈಲಿ ಹಗ್ಗವನ್ನು ಹಿಡಿದು, ಬಿರುಗಣ್ಣಿನಿಂದ ಅವನನ್ನು ದುರದುರನೆ ದಿಟ್ಟಿಸಿ, ಕಟಕಟ ಹಲ್ಲುಕಡಿಯುತ್ತಿರಲು ಅಜಾಮಿಳನು ಗಡಗಡ ನಡುಗುತ್ತಾ ‘ನಾರಾಯಣಾ!’ ಎಂದು ತನ್ನ ಮಗನನ್ನು ಗಟ್ಟಿಯಾಗಿ ಕೂಗಿದ.

ಅಜಾಮಿಳನ ಬಾಯಿಂದ ‘ನಾರಾಯಣ’ ಎಂಬ ಶಬ್ದ ಬಂದುದೇ ತಡ, ದೇವ ದೇವ ನಾದ ನಾರಾಯಣನ ದೂತರೂ ಅಲ್ಲಿ ಪ್ರತ್ಯಕ್ಷರಾದರು. ಯಮದೂತರು ಅಜಾಮಿಳನ ಜೀವವನ್ನು ಹೊರಗೆಳೆಯುತ್ತಿರಲು, ಅವರು ಅದನ್ನು ತಡೆದು ನಿಲ್ಲಿಸಿದರು. ಯಮ ದೂತರಿಗೆ ಅವರನ್ನು ಕಂಡು ಆಶ್ಚರ್ಯವಾಯಿತು. ಅವರು ‘ಅಯ್ಯಾ, ನೀವಾರು? ಎಲ್ಲಿಂದ ಬಂದಿರಿ? ಯಮಧರ್ಮರಾಯನ ಅಪ್ಪಣೆಯನ್ನು ಮೀರಿ ಇವನ ಜೀವನವನ್ನು ಎಕೆ ನಿಲ್ಲಿಸುತ್ತಿರುವಿರಿ? ನಿಮ್ಮನ್ನು ನೋಡಿದರೆ ಮಹಾಪುರುಷರಂತೆ ಕಾಣುತ್ತಿರುವಿರಿ. ನೀವು ಧರಿಸಿರುವ ಈ ಪೀತಾಂಬರ, ಕುಂಡಲ ಕಿರೀಟಗಳೂ ನಿಮ್ಮ ನಾಲ್ಕು ತೋಳುಗಳ ಲ್ಲಿರುವ ಬಿಲ್ಲು, ಬಾಣ, ಶಂಖ, ಚಕ್ರ–ಇತ್ಯಾದಿ ಆಯುಧಗಳೂ ದಶದಿಕ್ಕುಗಳನ್ನೂ ಬೆಳಗುತ್ತಿರುವ ನಿಮ್ಮ ತೇಜಸ್ಸೂ ನೀವು ಮಹಾನುಭಾವರೆಂದು ಸಾರಿಹೇಳುತ್ತಿವೆ. ಪಾಪಿ ಯಾದ ಇವನನ್ನು ನರಕಕ್ಕೆ ಎಳೆದೊಯ್ಯಲೆಂದು ಬಂದ ಯಮದೂತರು ನಾವು. ನಮ್ಮ ನ್ನೇಕೆ ತಡೆಯುತ್ತಿರುವಿರಿ?’ ಎಂದು ಕೇಳಿದರು. ಆಗ ವಿಷ್ಣುದೂತರು ತಮ್ಮ ತಮ್ಮ ಪರಿಚಯವನ್ನು ಮಾಡಿಕೊಟ್ಟು, ಅಜಾಮಿಳನು ಕಡೆಗಾಲದಲ್ಲಿ ನಾರಾಯಣನಾಮಸ್ಮರಣೆ ಮಾಡಿದುದರಿಂದ ನರಕ ಶಿಕ್ಷೆಗೆ ಅನರ್ಹನೆಂದು ತಿಳಿಸಿದರು. ಆಗ ಯಮದೂತರಿಗೂ ವಿಷ್ಣುದೂತರಿಗೂ ಧರ್ಮಾಧರ್ಮದ ವಿಚಾರದಲ್ಲಿ ದೀರ್ಘವಾದ ಚರ್ಚೆ ನಡೆಯಿತು.

ವಿಷ್ಣುದೂತರು: ಅಯ್ಯಾ, ಧರ್ಮಾಧರ್ಮವೆಂಬ ನಿರ್ಣಯ ಹೇಗೆ?

ಯಮದೂತರು: ವೇದವಿಹಿತವಾದುದು ಧರ್ಮ, ಅದಕ್ಕೆ ವಿರೋಧವಾದುದು ಅಧರ್ಮ. ಸಾಕ್ಷಾತ್ ಭಗವಂತನೇ ವೇದಸ್ವರೂಪವೆಂದು ಹೇಳುತ್ತಾರೆ. ಆದ್ದರಿಂದ ಅಧರ್ಮಮಾಡಿದವನನ್ನು ಶಿಕ್ಷಿಸಬೇಕೆಂದು ಭಗವಂತನ ಅಪ್ಪಣೆ.

ವಿಷ್ಣುದೂತರು: ಜಗತ್ತಿನ ಜನರು ಮಾಡುವ ಪಾಪಪುಣ್ಯಗಳನ್ನೆಲ್ಲ ನೀವು ಸದಾ ಕಣ್ಣಿಟ್ಟು ನೋಡುತ್ತಿರುವಿರೊ?

ಯಮದೂತರು: ನಾವು ನೋಡದಿದ್ದರೂ ಸೂರ್ಯ, ಅಗ್ನಿ, ಆಕಾಶ, ದೇವತೆಗಳು, ಗೋವುಗಳು, ಚಂದ್ರ, ಸಂಧ್ಯೆ, ಜಲ, ಭೂಮಿ, ಕಾಲ, ಯಮಧರ್ಮ–ಎಂಬ ಹನ್ನೊಂದು ಜನ ಸದಾ ಕಣ್ಣಿಟ್ಟು ನೋಡುತ್ತಾ, ಜೀವಿಗಳ ಪಾಪಪುಣ್ಯಗಳಿಗೆ ಸಾಕ್ಷಿಗಳಾಗಿ ಇರುವರು. ಪ್ರಾಣಿಗಳಾಗಿ ಹುಟ್ಟಿದಮೇಲೆ ಕರ್ಮ ಅನಿವಾರ್ಯ, ಅದರಂತೆ ಕರ್ಮಕ್ಕೆ ತಕ್ಕ ಫಲವೂ ಅನಿವಾರ್ಯ.

ವಿಷ್ಣುದೂತರು: ಅಯ್ಯಾ ಯಮದೂತರೆ, ಮನುಷ್ಯರು ಒಳ್ಳೆಯ ಕರ್ಮಗಳನ್ನೆ ಮಾಡುತ್ತಾ, ದೇವರ ಪ್ರೀತಿಗೆ ಪಾತ್ರರಾಗಿ ಸುಖವಾಗಿರಬಾರದೆ?

ಯಮದೂತರು: ಅಗತ್ಯವಾಗಿಯೂ ಇರಬಹುದು; ಆದರೆ ಇರುವುದಿಲ್ಲವಲ್ಲ. ಅದು ಅವರಿಗೆ ಸಾಧ್ಯವೂ ಅಲ್ಲ. ಕನಸಿನಲ್ಲಿರುವ ಮನುಷ್ಯನಿಗೆ ಸ್ವಪ್ನದ ಜಗತ್ತು ಮಾತ್ರವೇ ಗೋಚರ, ನಿಜ ಜಗತ್ತಲ್ಲ. ಹಾಗೆಯೇ ಕರ್ಮವಶನಾಗಿ ಅಜ್ಞಾನಿಯಾಗಿರುವ ಮನುಷ್ಯನಿಗೆ ತನ್ನ ಈ ಜನ್ಮದ ದೇಹಮಾತ್ರ ಗೊತ್ತೇ ಹೊರತು ಹಿಂದಿನ ಜನ್ಮಗಳ ಸ್ಮರಣೆಯಿಲ್ಲ. ಹೀಗಾಗಿ ಇಂದಿನ ದೇಹದ ಪೋಷಣೆಗೆ ಅಗತ್ಯವಾದ ಕರ್ಮಗಳನ್ನು ಮಾತ್ರ ಮಾಡುತ್ತಾನೆ.

ವಿಷ್ಣುದೂತರು: ಕರ್ಮ ಮಾಡುವುದು ದೇಹ; ಅದು ಬಿದ್ದುಹೋದ ಮೇಲೆ ಶಿಕ್ಷೆ ಇನ್ನಾರಿಗೆ?

ಯಮದೂತರು: ಅಯ್ಯಾ, ಮಹಾನುಭಾವರಾದ ನಿಮಗೆ ತಿಳಿಯದುದೇನೂ ಅಲ್ಲ; ಆದರೂ ನೀವು ಕೇಳಿದುದರಿಂದ ಹೇಳುತ್ತೇವೆ. ಪಂಚಭೂತಗಳಿಂದ ಆದ ಈ ದೇಹದಲ್ಲಿ ಐದು ಕರ್ಮೇಂ ದ್ರಿಯಗಳೂ ಇವೆಯಷ್ಟೆ. ಇವು ಮನಸ್ಸಿಗೆ ಅಧೀನ. ಕರ್ಮೇಂದ್ರಿಯಗಳ ಕಾರ್ಯಕ್ಕೂ ಜ್ಞಾನೇಂದ್ರಿಯಗಳ ತಿಳಿವಳಿಕೆಗೂ ಅದೇ ಕಾರಣ. ಲಿಂಗಶರೀರವೆಂಬ ಸೂಕ್ಷ್ಮ ದೇಹದಲ್ಲಿ ಇವೆಲ್ಲವೂ ಬೀಜರೂಪವಾಗಿವೆ. ಸ್ಥೂಲ ದೇಹ ಬಿದ್ದುಹೋದರೂ ಲಿಂಗ ದೇಹ ಜೀವನಿಗೆ ಅಂಟಿಕೊಂಡೆ ಇರುತ್ತದೆ, ಸ್ಥೂಲದೇಹದ ಕರ್ಮವಾಸನೆಯನ್ನು ಹೊತ್ತು ಮುಂದಿನ ಜನ್ಮಕ್ಕೆ ಕಾರಣವಾಗುತ್ತದೆ. ಈ ದೇಹಕ್ಕೆ ಅಂಟಿಕೊಂಡಿರುವ ಜೀವ ಅದನ್ನೇ ಆತ್ಮವೆಂದು ಭ್ರಮಿಸಿ, ಇಂದ್ರಿಯಗಳಿಗೂ ಮನಸ್ಸಿಗೂ ವಶನಾಗಿಹೋಗುತ್ತಾನೆ. ಇವುಗಳ ಸಂಬಂಧದಿಂದ ಪಾಪಪುಣ್ಯ ಕರ್ಮಗಳನ್ನು ಮಾಡುತ್ತಾನೆ; ಅದರ ಫಲವಾಗಿ ಸ್ವರ್ಗವನ್ನೊ ನರಕವನ್ನೊ ಸೇರುತ್ತಾನೆ. ಈ ಅಜಾಮಿಳ ಮಾಡಬಾರದ ಪಾಪಕರ್ಮಗಳನ್ನು ಮಾಡಿ ದ್ದಾನೆ. ಆದ್ದರಿಂದ ಇವನಿಗೆ ನರಕಶಿಕ್ಷೆ ಅನಿವಾರ್ಯ. 

ವಿಷ್ಣುದೂತರು: ಅಯ್ಯಾ, ನೀವು ಹೇಳಿದುದೆಲ್ಲವೂ ಸರಿ. ನೀವೂ ನಿಮ್ಮ ಒಡೆಯನಾದ ಯಮಧರ್ಮನೂ ಸಮದೃಷ್ಟಿಗೆ ಹೆಸರಾದವರು. ಹಿರಿಯರ ನಡತೆಯನ್ನೇ ಲೋಕದವರೆಲ್ಲ ಅನುಸರಿಸುತ್ತಾರೆ. ನಿಮ್ಮ ನಡತೆಯು ಲೋಕಕ್ಕೆ ಮಾದರಿಯಾಗಿದೆ. ಆದರೆ ನೀವು ಈಗ ಸಣ್ಣದೊಂದು ತಪ್ಪುಮಾಡುತ್ತಿರುವಿರಿ. ಈ ಅಜಾಮಿಳ ಸಾಯುವ ಕಾಲದಲ್ಲಿ ‘ನಾರಾ ಯಣ’ ಎಂದು ಕೂಗಿದ. ಆ ನಾಮಸ್ಮರಣೆಯಿಂದ ಅವನ ಪಾಪಗಳೆಲ್ಲ ಅಳಿಸಿಹೋಗಿವೆ. ಶ್ರೀಹರಿಯ ನಾಮಸ್ಮರಣೆಯಿಂದ ಪಂಚಮಹಾಪಾತಕಗಳೂ ನಾಶವಾಗುತ್ತವೆ. ಯಜ್ಞ, ಯಾಗ, ವ್ರತ, ತಪಸ್ಸುಗಳಿಗಿಂತ ನಾಮಸ್ಮರಣೆ ದೊಡ್ಡದು.

ಯಮದೂತರು: ಆದರೆ ಈ ಅಜಾಮಿಳ ಕೂಗಿದುದು ಹರಿಯನ್ನಲ್ಲ, ತನ್ನ ಮಗನನ್ನು!

ವಿಷ್ಣುದೂತರು: ಆದರೇನು? ಕೋಪದಿಂದಲೋ, ಹುಡುಗಾಟಕ್ಕಾಗಿಯೋ, ಕೊನೆಗೆ ನಾಲಗೆ ತೊದಲಿಯೋ–ದೇವರ ಹೆಸರನ್ನು ಉಚ್ಚರಿಸಿದರೆ ಸಾಕು, ಮಾಡಿದ ಪಾಪಗಳೆಲ್ಲ ಹಾರಿ ಹೋಗುತ್ತವೆ. ಔಷಧದ ಶಕ್ತಿಯನ್ನು ತಿಳಿಯದ ಮಾತ್ರಕ್ಕೆ ಅದನ್ನು ಕುಡಿದ ರೋಗಿ ಗುಣಹೊಂದುವುದಿಲ್ಲವೆ? ಈ ವಿಚಾರದಲ್ಲಿ ನಿಮಗೆ ಸಂದೇಹವಿದ್ದರೆ ನಿಮ್ಮ ಸ್ವಾಮಿ ಯಾದ ಯಮಧರ್ಮರಾಯನನ್ನೆ ಕೇಳಿ ನೋಡಿ’–ಎಂದು ಹೇಳಿ ಅವರು ಮಾಯವಾಗಿ ಹೋದರು.

ಯಮದೂತರು ವಿಷ್ಣುದೂತರ ಮಾತಿಗೆ ಮರ್ಯಾದೆಗೊಟ್ಟು ಅಜಾಮಿಳನನ್ನು ಬಿಟ್ಟು ಬಿಟ್ಟರು. ಅವರು ಯಮರಾಜನಲ್ಲಿಗೆ ಹೋಗಿ, ಮನುಷ್ಯಲೋಕದಲ್ಲಿ ನಡೆದುದನ್ನೆಲ್ಲ ಆತನಿಗೆ ವರದಿಮಾಡಿದರು. ಆತನು ಅದನ್ನು ಕೇಳಿ ಸಂತೋಷಗೊಂಡು, “ಅಯ್ಯಾ ದೂತರೆ, ಬಟ್ಟೆಯಲ್ಲಿನ ಹಾಸುಹೊಕ್ಕುಗಳಂತೆ ಈ ಜಗತ್ತನ್ನೆಲ್ಲ ವ್ಯಾಪಿಸಿರುವ ಸರ್ವೇ ಶ್ವರನು ಬ್ರಹ್ಮದೇವನಿಂದ ಒಂದು ಹುಲ್ಲುಕಡ್ಡಿಯವರೆಗೆ ಎಲ್ಲವನ್ನೂ ನಿಯಮಿಸುತ್ತಿರು ವನು. ಹಗ್ಗದಿಂದ ಹಸುಗಳನ್ನು ಕಟ್ಟುವಂತೆ ಆತನು ಜನರನ್ನು ವರ್ಣಾಶ್ರಮಧರ್ಮ ಗಳಲ್ಲಿ ಕಟ್ಟಿಟ್ಟಿದ್ದಾನೆ. ಆತನ ಮಾಯೆಯನ್ನು ಅರ್ಥಮಾಡಿಕೊಳ್ಳುವುದು ದೇವಾನುದೇವತೆ ಗಳಿಗೂ ಅಸಾಧ್ಯ. ಆತನ ದೂತರೂ ಮಹಾಮಹಿಮಾನ್ವಿತರು. ಅವರು ಪಾಪಿಯಾದ ಅಜಾ ಮಿಳನನ್ನು ಅನ್ಯಾಯವಾಗಿ ಬಿಡಿಸಿದರೆಂದು ನೀವು ಭಾವಿಸಬಹುದು. ಅದು ಸರಿಯಲ್ಲ. ಧರ್ಮಸೂಕ್ಷ್ಮವನ್ನು ಅರ್ಥಮಾಡಿಕೊಳ್ಳುವುದು ಬಹು ಕಷ್ಟ. ಸ್ವಲ್ಪ ವಿಚಾರಮಾಡಿ ನೋಡಿ. ವಿಷ್ಣುದೂತರು ಅಜಾಮಿಳನನ್ನು ನಿಮ್ಮಿಂದ ಬಿಡಿಸಬೇಕಾದರೆ, ಅದಕ್ಕೆ ಯಾವುದೋ ಒಂದು ಸೂಕ್ಷ್ಮವಾದ ಧರ್ಮವೇ ಕಾರಣವಾಗಿರಬೇಕು. ಅದೇ ಭಾಗವತ ಧರ್ಮ. ಶ್ರೀಹರಿಯ ಹೆಸರನ್ನು ಹೇಳುವುದರ ಮೂಲಕ ಭಕ್ತಿಯೋಗವನ್ನು ಪಡೆಯುವುದೇ ಭಾಗವತಧರ್ಮದ ತಿರುಳು. ಭಕ್ತಿಯೋಗ ಹಾಗಿರಲಿ, ಒಮ್ಮೆ ಭಗವಂತನ ಹೆಸರು ಹೇಳಿದರೆ ಪಾಪಗಳೆಲ್ಲ ನಿವಾರಣೆಯಾಗುತ್ತವೆ” ಎಂದು ಹೇಳಿ ಆತನು ನಾರಾಯಣಸ್ಮರಣೆ ಯಲ್ಲಿ ತಲ್ಲೀನನಾದನು.

ಇತ್ತ ಅಜಾಮಿಳನು ಯಮದೂತರಿಗೂ ವಿಷ್ಣುದೂತರಿಗೂ ನಡೆದ ಸಂಭಾಷಣೆಯನ್ನು ಕೇಳಿ, ತನ್ನ ಪಾಪಕಾರ್ಯಗಳಿಗಾಗಿ ಪಶ್ಚಾತ್ತಾಪಪಟ್ಟನು. ಅವನಲ್ಲಿ ಭಕ್ತಿ ಉದಿಸಿತು. ಅವನು ವೈರಾಗ್ಯಪರನಾಗಿ ಗಂಗಾದ್ವಾರಕ್ಕೆ ಹೋಗಿ, ನಿಶ್ಚಲವಾದ ಭಕ್ತಿಯಿಂದ ತಪೋನಿರತ ನಾದನು.

ಆತನ ಮನಸ್ಸು ಪರಮಾತ್ಮನಲ್ಲಿ ನೆಲೆಯಾಗಿ ನಿಂತಿತು. ಆಗ ಮತ್ತೊಮ್ಮೆ ವಿಷ್ಣು ದೂತರು ಆತನಿಗೆ ಕಾಣಿಸಿಕೊಂಡರು. ಅಜಾಮಿಳನು ಭಕ್ತಿಯಿಂದ ಅವರಿಗೆ ನಮಸ್ಕರಿ ಸಿದನು. ಆತನ ದೇಹ ಅಲ್ಲಿಯೇ ಬಿದ್ದುಹೋಯಿತು. ಒಡನೆಯೆ ಆತನಿಗೆ ವಿಷ್ಣುದೂತರಿಗೆ ಸಮಾನವಾದ ದಿವ್ಯರೂಪವುಂಟಾಯಿತು. ಆತನು ವಿಮಾನವನ್ನೇರಿ ವೈಕುಂಠಕ್ಕೆ ಹೋಗಿ, ವಿಷ್ಣುಸನ್ನಿಧಿಯನ್ನು ಸೇರಿದನು.

ಅಯ್ಯಾ, ಪರೀಕ್ಷಿದ್ರಾಜ! ನಿನ್ನ ಪ್ರಶ್ನೆಗೆ ಉತ್ತರ ದೊರೆಯಿತಲ್ಲವೆ? ಕರ್ಮದ ಕಟ್ಟಿ ನಿಂದ ತಪ್ಪಿಸಿಕೊಳ್ಳುವುದಕ್ಕೆ ನಾಮಸ್ಮರಣೆಗಿಂತ ಉತ್ತಮವಾದ ಉಪಾಯವಿನ್ನಿಲ್ಲ. ಈ ಅಜಾಮಿಳನ ಇತಿಹಾಸವನ್ನು ಭಕ್ತಿಯಿಂದ ಕೇಳುವವನೂ, ಹೇಳುವವನೂ, ಬರೆಯು ವವನೂ ಉತ್ತಮ ಗತಿಯನ್ನು ಹೊಂದುವನು.

