
\chapter{೪೬. ಪುಟ್ಟಬಾಯಲ್ಲಿ ಅಖಂಡ ಬ್ರಹ್ಮಾಂಡ}

ಬಲರಾಮ ಶ್ರೀಕೃಷ್ಣರು ನಂದಗೋಕುಲದ ಕಣ್ಮಣಿಗಳಾಗಿ ಬೆಳೆಯುತ್ತಾ ಅಂಬೆಗಾ ಲಿಡುವಂತಾದರು. ಅವರು ಕಾಲುಗಳಲ್ಲಿರುವ ಕಡಗ ಕಿರುಗೆಜ್ಜೆಗಳು ಘಲಿಘಲಿರೆನ್ನುವಂತೆ ಕಾಲುಗಳನ್ನು ಎಳೆದೆಳೆದು ಹಾಕುತ್ತಾ ಕಿಲಿಕಿಲಿ ನಗುವರು, ನಿಂತು ನಿಂತು ಕೇಕೆ ಹಾಕುತ್ತಾ ಮುಂದುವರಿಯುವರು, ಜನರನ್ನು ಕಂಡಾಗ ಅವರನ್ನು ಹಿಂಬಾಲಿಸುವರು, ಅವರು ತಿರುಗಿ ನೋಡುತ್ತಲೆ ಹೆದರಿ ತಾಯ ಬಳಿಗೆ ಓಡುವರು. ಆಗ ರೋಹಿಣಿ ಯಶೋದೆಯರು ಬಾಚಿ ತಬ್ಬಿಕೊಂಡು, ಮುದ್ದಿಕ್ಕುವರು, ಮೊಲೆಯುಣಿಸುವರು. ತಾಯಿಯ ಹಾಲನ್ನು ಕುಡಿ ಯುತ್ತಾ, ತಾಯ ಮುಖದತ್ತ ತಮ್ಮ ಮುಗುಳ್ನಗೆಯನ್ನು ಚೆಲ್ಲಿದಾಗ, ಆ ಮಕ್ಕಳ ಮೊಳೆ ವಲ್ಲುಗಳ ಮುದ್ದು ಮುಖವನ್ನು ನೋಡಿ ಆ ತಾಯಿಯರು ಅವರ ಅಂಗಾಂಗಗಳಿಗೆಲ್ಲ ಮುತ್ತಿನ ಮಳೆಗರೆಯುವರು. ಹಾಲು ಕುಡಿದು ತೃಪ್ತರಾದ ಮಕ್ಕಳು ಅಂಬೆಗಾಲಿಕ್ಕುತ್ತಾ ಜೋಡು ಕುದುರೆಯಂತೆ ಓಡುವರು. ಮನೆಯಿಂದ ಅಂಗಳಕ್ಕಿಳಿದು, ಕರುಗಳ ಬಾಲವನ್ನು ಹಿಡಿದು ಜೋತುಬೀಳುವರು, ಅವು ಎಳೆದತ್ತ ತಾವೂ ಹೋಗುವರು. ಈ ಕಣ್ಣ ಹಬ್ಬವನ್ನು ನೋಡುತ್ತಾ ಗೋಪಿಯರು ಮನೆಗೆಲಸವನ್ನೆಲ್ಲ ಮರೆತು, ಮನೆ ಬಾಗಿಲಲ್ಲಿ ನಗುತ್ತ ನಿಲ್ಲು ವರು. ಮುದ್ದಿನ ಮುದ್ದೆಗಳಂತಿದ್ದ ಆ ಮಕ್ಕಳ ಒಂದೊಂದು ಆಟವೂ ಮನೋಹರ. ಒಮ್ಮೊಮ್ಮೆ ಅದು ಸುಮನೋಹರ ಭಯಂಕರವಾಗುವುದು ಉಂಟು. ಆ ಮಕ್ಕಳಿಗೆ, ದನಕರುಗಳಿರಲಿ, ಜಿಂಕೆಯ ಭಯವಿಲ್ಲ; ನಾಯಿಯ ಭಯವಿಲ್ಲ, ಹಾವಿನ ಭಯವಿಲ್ಲ, ನೀರಿನ ಭಯವಿಲ್ಲ, ಬೆಂಕಿಯ ಭಯವಿಲ; ಕಂಡುದನ್ನು ಹಿಡಿದುಕೊಳ್ಳುವ ಆಶೆ, ಅವರಿಗೆ. ಮುಳ್ಳುಕಳ್ಳೆಗಳನ್ನಾಗಲಿ, ನೀರಿನ ಮಡುವನ್ನಾಗಲಿ ಲಕ್ಷಿಸದೆ ಅವರು ನುಗ್ಗಿ ಹೋಗುವರು. ಆದ್ದರಿಂದ ಸದಾ ಅವರಮೇಲೆ ಒಂದು ಕಣ್ಣಿಟ್ಟಿರುವುದು ಅನಿವಾರ್ಯವಾಯಿತು, ಅವರ ತಾಯಿಯರಿಗೆ.

ಅಂಬೆಗಾಲಿಡುತ್ತಿದ್ದ ಮಕ್ಕಳು ಎದ್ದು ನಿಲ್ಲುವಂತಾಯಿತು, ತಪ್ಪು ಹೆಜ್ಜೆಗಳನ್ನಿಡುತ್ತಾ ನಡೆಯುವಂತಾಯಿತು. ಅವರೊಡನೆ ಅವರ ತುಂಟತನವೂ ಬೆಳೆಯುತ್ತಾ ಹೋಯಿತು. ಮಕ್ಕಳಿಬ್ಬರೂ ದಟ್ಟಡಿಯಿಡುತ್ತಾ ದನಗಳ ಕೊಟ್ಟಿಗೆಗೆ ನುಗ್ಗಿ, ಅಲ್ಲಿ ಕಟ್ಟಿಹಾಕಿದ್ದ ಕರುಗಳ ಕಣ್ಣಿಗಳನ್ನು ಬಿಚ್ಚಿ ಹಾಕುವರು, ಅವು ಓಡಿಹೋಗಿ ತಾಯ ಮೊಲೆಯನ್ನು ಕುಡಿಯು ವುದನ್ನು ಕಂಡು ಚಪ್ಪಾಳೆ ತಟ್ಟುತ್ತಾ ಕುಣಿಯುವರು. ಮನೆ ಮನೆಗೂ ನುಗ್ಗಿ, ಹಾಲು ಮೊಸರು-ಬೆಣ್ಣೆಗಳನ್ನು ಕೊಳ್ಳೆಹೊಡೆಯುವರು; ತಾವು ತಿಂದು ಸಾಕಾದ ಮೇಲೆ ಉಳಿದು ದನ್ನು ಕೋತಿಗಳಿಗೆ ತಿನ್ನಿಸುವರು; ಇನ್ನೂ ಮಿಕ್ಕರೆ, ಆ ಪಾತ್ರೆಯನ್ನು ಒಡೆದುಹಾಕುವರು. ಈ ತುಂಟತನದಲ್ಲಿ ಬಲರಾಮನಿಗಿಂತ ಶ್ರೀಕೃಷ್ಣನದೇ ಮೇಲುಗೈ. ಯಾವುದಾದರೂ ಮನೆ ಯಲ್ಲಿ ಏನೂ ಸಿಕ್ಕದೆ ಹೋದರೆ, ಕೋಪದಿಂದ ಆ ಪೋರ ತೊಟ್ಟಿಲಲ್ಲಿ ಮಲಗಿರುವ ಮಕ್ಕಳನ್ನು ಚಿವುಟಿ ಅಳಿಸುವನು. ಮನೆಯ ಹೆಂಗಸರು ಮನೆಗೆಲಸದಲ್ಲಿ ತೊಡಗಿರುವಾಗ ಅವರಿಗೆ ಗೊತ್ತಾಗದಂತೆ ಅಡಕಿಲ ಕೋಣೆಗೆ ನುಗ್ಗಿ, ತನ್ನ ಒಡವೆಗಳ ಬೆಳಕಿನಲ್ಲಿ ಮೊಸರು ಬೆಣ್ಣೆಗಳಿರುವ ಸ್ಥಳವನ್ನು ಪತ್ತೆ ಹಚ್ಚಿ, ಅವನ್ನು ಧ್ವಂಸ ಮಾಡುವನು. ಅವನ ಕೈಗೆ ಸಿಕ್ಕದಂತೆ ನೆಲುವಿನ ಮೇಲೇನಾದರೂ ಇಟ್ಟಿದ್ದರೆ, ಕಾಲು ಮಣೆಯನ್ನೊ ಒರಳುಕಲ್ಲನ್ನೊ ಹಾಕಿಕೊಂಡು, ಅದರ ಮೇಲೆ ಹತ್ತಿ ಅವನ್ನೂ ಸೂರೆಮಾಡುವನು; ಹಾಗೂ ಸಿಕ್ಕದಿದ್ದರೆ ಕೋಲಿನಿಂದ ಮಡಕೆಯನ್ನು ಚುಚ್ಚಿ, ಅದರ ರಂಧ್ರಕ್ಕೆ ಬಾಯೊಡ್ಡಿ ಹಾಲನ್ನೆಲ್ಲ ಕುಡಿದು ಬಿಡುವನು. ಆಗ ಯಾರಾದರೂ ಅವನನ್ನು ಕಂಡು, ‘ಎಲಾ ಕಳ್ಳ’ ಎಂದು ಛೀಗುಟ್ಟಿದರೆ ‘ನೀನೆ ಕಳ್ಳ, ನಾನು ಈ ಮನೆಯವನು’ ಎಂದು ಎದುರುತ್ತರವಿತ್ತು ನಗಿಸುವನು. ಇವನ ಕಾಟವನ್ನು ತಾಳಲಾರದೆ ಗೋಪಿಯರು ಆಗಾಗ ಯಶೋದೆಯ ಮುಂದೆ ಹೋಗಿ ದೂರು ವರು. ಆ ಸಮಯಗಳಲ್ಲಿ ಶ್ರೀಕೃಷ್ಣನು ತಾಯಿಯ ಬಳಿಯಲ್ಲೇ ನಿಂತಿರುತ್ತಿದ್ದನಾದರೂ, ಕಣ್ಣು ಪಿಳುಕಿಸುತ್ತಾ ಏನೂ ಅರಿಯದಂತೆ ನಿಂತಿರುವ ಅವನ ಮುದ್ದು ಮುಖವನ್ನು ಕಾಣುತ್ತಲೆ, ಆ ತಾಯಿಗೆ ಕೋಪವೆಲ್ಲ ಹಾರಿಹೋಗುತ್ತಿತ್ತು. ಹೀಗಿರಲು ಒಂದು ದಿನ ಒಬ್ಬ ಗೋಪಿ, ಕಳ್ಳತನ ಮಾಡುತ್ತಿದ್ದ ಕೃಷ್ಣನನ್ನು ಅವನು ಕದ್ದ ಬೆಣ್ಣೆಯೊಡನೆ ಎತ್ತಿ ಕೊಂಡು ಯಶೋದೆಯ ಬಳಿಗೆ ಬಂದಳು. ಅದೇ ವೇಳೆಗೆ ಎಲ್ಲ ಗೋಪಿಯರ ಮನೆಗಳ ಲ್ಲಿಯೂ ಹಾಗೆಯೇ ನಡೆದಿರಬೇಕೆ? ಎಲ್ಲ ಗೋಪಿಯರು ಒಬ್ಬೊಬ್ಬ ಕೃಷ್ಣನನ್ನು ಎತ್ತಿ ಕೊಂಡು ಹೊರಟರು. ಅವರೆಲ್ಲ ಏಕ ಕಾಲಕ್ಕೆ ಯಶೋದೆಯ ಬಳಿ ಬಂದು ನೋಡುತ್ತಾರೆ, ಅಲ್ಲಿ ಶ್ರೀಕೃಷ್ಣ ತಾಯ ಮಡಿಲಲ್ಲಿ ನಲಿಯುತ್ತಾ ಕುಳಿತಿದ್ದಾನೆ! ಅವರು ಆಶ್ಚರ್ಯದಿಂದ ತಮ್ಮ ಕೊಂಕುಳಲ್ಲಿ ನೋಡಿಕೊಳ್ಳುತ್ತಾರೆ, ಮಂಗ ಮಾಯೆ! ಅವರು ನಾಚಿಕೆಯಿಂದ, ಬಂದ ದಾರಿಗೆ ಸುಂಕವಿಲ್ಲವೆಂದುಕೊಂಡು ಹಿಂದಿರುಗಿದರು.

ಇನ್ನೊಂದು ದಿನ ಮತ್ತೊಂದು ಮಹದಚ್ಚರಿ ನಡೆಯಿತು. ಬಲರಾಮ ಕೃಷ್ಣರು ತಮ್ಮ ಓರಿಗೆಯ ಮಕ್ಕಳೊಡನೆ ಸೇರಿಕೊಂಡು ಆಟವಾಡುತ್ತಾ ಇರುವಾಗ ಶ್ರೀಕೃಷ್ಣನು, ಮಕ್ಕಳು ಸಹಜವಾಗಿ ಮಾಡುವಂತೆ ಸ್ವಲ್ಪ ಮಣ್ಣನ್ನು ತಿಂದನು. ಇದನ್ನು ಕಂಡ ಬಲರಾಮನು ಯಶೋದೆಯ ಬಳಿಗೆ ಬಂದು ‘ಅಮ್ಮ, ಕೃಷ್ಣ ಮಣ್ಣು ತಿನ್ನುತ್ತಿದ್ದಾನೆ’ ಎಂದು ದೂರು ಹೇಳಿದ. ಒಡನೆಯೆ ಆಕೆ ಶ್ರೀಕೃಷ್ಣನನ್ನು ಹಿಡಿದುಕೊಂಡು ‘ಎಲಾ ತುಂಟ, ಮಣ್ಣು ತಿನ್ನು ತ್ತೀಯೇನೋ!’ ಎಂದು ಗದರಿಸಿದಳು. ಅವನು ಅಳುಮೋರೆ ಮಾಡಿಕೊಂಡು ‘ಇಲ್ಲಮ್ಮ, ಅಣ್ಣ ಸುಳ್ಳು ಹೇಳುತ್ತಾನೆ, ನನ್ನನ್ನು ಹೊಡೆಸಬೇಕು ಎಂತ. ಊ....ಊ’ ಎಂದ. ಆಗ ಯಶೋದೆ ಅವನ ನಟನೆಯನ್ನು ನೆನೆದು ಮನದಲ್ಲಿಯೆ ಹಿಗ್ಗುತ್ತಾ, ಬಾಯಲ್ಲಿ ಮಾತ್ರ ‘ಆ ಅನ್ನು ನೋಡೋಣ’ ಎಂದು ಗದರಿಸಿದಳು. ಒಡನೆಯೇ ಶ್ರೀಕೃಷ್ಣನು ತನ್ನ ಪುಟ್ಟ ಬಾಯನ್ನು ‘ಆ’ಎಂದು ಅರಳಿಸಿದನು. ಯಶೋದೆ ನೋಡುತ್ತಾಳೆ, ಅಬ್ಬ, ಆ ಪುಟ್ಟ ಬಾಯಲ್ಲಿ ಅಖಂಡ ಬ್ರಹ್ಮಾಂಡವೇ ಏಕ ಕಾಲದಲ್ಲಿ ಕಾಣಬರುತ್ತಿದೆ! ಆಕಾಶ, ದಿಕ್ಕುಗಳು, ಸೂರ್ಯ ಚಂದ್ರಾದಿ ಗ್ರಹ ನಕ್ಷತ್ರಗಳು, ನದಿ, ಬೆಟ್ಟ, ಮರ, ಮಾನವ, ದೇವತೆ ಎಲ್ಲವೂ, ಎಲ್ಲವೂ ಅಲ್ಲಿಯೇ ಇವೆ. ಆ ಚಿತ್ರವಿಚಿತ್ರವಾದ ಜಗತ್ತಿನ ಮಧ್ಯದಲ್ಲಿ ತನ್ನ ಗೋಕುಲ, ತನ್ನ ಮನೆ, ತಾನು ಬಾಯ್ದೆರೆದುಕೊಂಡು ನಿಂತಿರುವ ತನ್ನ ಕಂದ ಶ್ರೀಕೃಷ್ಣ–ಇವೂ ಅಲ್ಲಿಯೆ ಕಾಣುತ್ತಿವೆ. ಯಶೋದೆ ತನ್ನ ಕಣ್ಣನ್ನು ತಾನೆ ನಂಬಲಾರದಾದಳು. ಆಕೆ ತನ್ನಲ್ಲಿ ತಾನೆ ‘ಇದೇನು ಕನಸೋ ಅಥವಾ ಮಾಯೆಯೋ; ಇಲ್ಲ ತನಗೆ ಬುದ್ಧಿ ಭ್ರಮಣೆಯೇನಾ ದರೂ ಆಗಿದೆಯೊ? ಬಾಯೊಳಗೆ ಹೊರಗೆ ಒಂದೇ ಸಮನಾದ ಜಗತ್ತು ಕಾಣಬರುತ್ತಿದೆ ಯಲ್ಲಾ! ಮತ್ತೆ ಮತ್ತೆ ನೋಡಿದರೂ ಒಂದೇ ಸಮನಾಗಿದೆ! ಇದರ ಅರ್ಥವೇನು? ಈ ಮಗು ಮಗುವಲ್ಲ, ಸಾಕ್ಷಾತ್ ಪರಮೇಶ್ವರ!’ ಎಂದುಕೊಂಡಳು. ಆಕೆಯ ದೇಹ ಭಕ್ತಿ ಭಾವದಿಂದ ಪುಲಕಿತವಾಯಿತು. ಆಕೆಯ ಮನಸ್ಸು ಮಣಿಯಿತು, ಅಹಂಕಾರ ಅಳಿಯಿತು, ಬುದ್ಧಿ ಸಾತ್ವಿಕವಾಯಿತು. ‘ಸ್ವಾಮಿ, ಮಾಯೆಯಿಂದ ನನ್ನನ್ನು ಕಾಪಾಡು’ ಎಂಬ ನುಡಿಗಳು ಬಾಯಿಂದ ತಾವಾಗಿಯೇ ಹೊರಬಂದವು. ಆಕೆ ಕಣ್ಮುಚ್ಚಿ ಕ್ಷಣಕಾಲ ಧ್ಯಾನಮಗ್ನಳಾದಳು. ಇದನ್ನು ಕಂಡು ಪರಾತ್ಪರವಸ್ತುವಾದ ಶ್ರೀಕೃಷ್ಣನು ತನ್ನ ಮಾಯೆಯನ್ನು ಆಕೆಯ ಮೇಲೆ ಹರಡಿ ‘ಅಮ್ಮ, ನಾನು ಆಡುವುದಕ್ಕೆ ಹೋಗುತ್ತೇನಮ್ಮಾ’ ಎಂದ. ಆಕೆ ಕಣ್ದೆರೆದು, ಕೃಷ್ಣನ ಮುದ್ದು ಮುಖವನ್ನು ಕಾಣುತ್ತಲೆ ಎಲ್ಲವನ್ನೂ ಮರೆತು, ‘ಮಗು, ಮಗು’ ಎಂದು ಅವನನ್ನು ಬಾಚಿ ತಬ್ಬಿಕೊಂಡು, ಮೋಹದಿಂದ ಮುದ್ದಿನ ಮಳೆಗರೆದಳು.

ಯೋಗಿಗಳಿಗೆ ಕೂಡ ದುರ್ಲಭನಾದ ಭಗವಂತನು ನಂದ ಯಶೋದೆಯರ ಮುದ್ದು ಕಂದನಾಗಿರುವದನ್ನು ಕಂಡು ಅಚ್ಚರಿಪಡಬೇಕಾದುದಿಲ್ಲ. ನಂದನು ಹಿಂದಿನ ಜನ್ಮದಲ್ಲಿ ಅಷ್ಟವಸುಗಳಲ್ಲಿ ಮುಖ್ಯನೆನಿಸಿದ್ದ ದ್ರೋಣನೆಂಬುವನು. ಯಶೋದೆ ಅಂದು ಆತನ ಪತ್ನಿ ಯಾಗಿದ್ದ ಧರಾದೇವಿ. ಅವರಿಬ್ಬರನ್ನು ಭೂಮಿಯಲ್ಲಿ ಹುಟ್ಟುವಂತೆ ಬ್ರಹ್ಮನು ಕೇಳಿ ಕೊಂಡಾಗ, ಅವರು ‘ನಾವು ಮಾನವರಾಗಿ ಹುಟ್ಟಿದಾಗ ನಮ್ಮ ದೈವಭಕ್ತಿ ಲೋಪವಾಗ ಬಾರದು’ ಎಂದು ಹೇಳಿ, ಬ್ರಹ್ಮನಿಂದ ‘ಅಸ್ತು’ ಎನಿಸಿಕೊಂಡಿದ್ದರು. ಆ ಅನುಗ್ರಹ ದಿಂದಲೆ ಇಂದು ಅವರಿಗೆ ಈ ವೈಭವ.

