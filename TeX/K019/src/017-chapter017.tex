
\chapter{೧೭. ಪ್ರಚೇತಸರು}

ಪೃಥುಚಕ್ರವರ್ತಿಯ ನಂತರ ಆತನ ಹಿರಿಯ ಮಗನಾದ ವಿಜಿತಾಶ್ವನು ಸಿಂಹಾಸನವನ್ನು ಏರಿದನು. ಆತನಿಗೆ ತನ್ನ ತಮ್ಮಂದಿರಲ್ಲಿ ಬಹು ಪ್ರೀತಿ. ಆದ್ದರಿಂದ ನಾಲ್ಕು ಜನ ತಮ್ಮಂದಿರಿಗೂ ನಾಲ್ಕು ದಿಕ್ಕುಗಳ ರಾಜ್ಯವನ್ನು ಹಂಚಿಕೊಟ್ಟನು. ಹರ್ಯಕ್ಷನಿಗೆ ಮೂಡಣ ನಾಡುಗಳು, ಧೂಮ್ರಕೇಶನಿಗೆ ತೆಂಕಣ ಸೀಮೆಗಳು, ವೃಕನಿಗೆ ಪಡುವಣ ರಾಜ್ಯ ಗಳು, ದ್ರವಿಣನಿಗೆ ಬಡಗಲ ಭಾಗಗಳು ಸೇರಿದವು. ಇವುಗಳ ಮಧ್ಯರಾಜ್ಯವು ವಿಜಿತಾಶ್ವನ ದಾಗಿತ್ತು. ಆತನು ತಂದೆಯ ಯಜ್ಞಕಾಲದಲ್ಲಿ ಇಂದ್ರನನ್ನು ಜಯಿಸಿದಾಗ, ಆತನು ಕಣ್ಣಿಗೆ ಕಾಣದಂತೆ ಎಲ್ಲಿ ಬೇಕಾದರಲ್ಲಿಗೆ ಹೋಗಬಹುದಾದ ‘ಅಂತರ್ಧಾನ ಗತಿ’ಯನ್ನು ದೇವೇಂದ್ರನಿಂದ ಪಡೆದಿದ್ದುದರಿಂದ ಈತನಿಗೆ ‘ಅಂತರ್ಧಾನ’ನೆಂಬ ಮತ್ತೊಂದು ಹೆಸರೂ ಇತ್ತು. ಆತನಿಗೆ ಇಬ್ಬರು ಹೆಂಡತಿಯರು: ಶಿಖಂಡಿನಿ, ನಭಸ್ವತಿ. ಮೊದಲ ಹೆಂಡತಿಯಲ್ಲಿ ಪಾವಕ, ಪವಮಾನ, ಶುಚಿ ಎಂಬ ಮೂವರು ಮಕ್ಕಳಾದರು. ಇವರು ಪೂರ್ವದಲ್ಲಿ ವಸಿಷ್ಠ ಪುಷಿಯ ಶಾಪಕ್ಕೆ ಒಳಗಾಗಿದ್ದ ತ್ರೇತಾಗ್ನಿಗಳು; ಅವರು ಮನುಷ್ಯ ರಾಗಿ ಹುಟ್ಟಿ, ಶಾಪವನ್ನು ಕಳೆದುಕೊಂಡು ಮತ್ತೆ ಪೂರ್ವದ ಜನ್ಮಗಳನ್ನೆ ಪಡೆದರು. ಎರಡನೆಯ ಹೆಂಡತಿ ನಭಸ್ವತಿಯಲ್ಲಿ ಹವಿರ್ಧಾನನೆಂಬ ಮಗ ಹುಟ್ಟಿದ. ಅವನಿಗೆ ರಾಜ್ಯ ವೃತ್ತಿ ಸರಿಬೀಳಲಿಲ್ಲ. ಜನರಿಂದ ಹಿಂಸೆಮಾಡಿ ಸುಂಕ ಎತ್ತುವುದು, ತಪ್ಪು ಮಾಡಿದಾಗ ಶಿಕ್ಷಿಸುವುದು–ಇವೆಲ್ಲ ಕ್ರೂರಕಾರ್ಯವೆಂದು ಅವನ ಭಾವನೆ. ಆದ್ದರಿಂದ ಅವನು ದೊಡ್ದ ದೊಂದು ಯಾಗ ಮಾಡುವೆನೆಂದು ನೆಪಮಾಡಿಕೊಂಡು ರಾಜಪದವಿಯಿಂದ ನುಣುಚಿ ಕೊಂಡುಹೋದ; ಯಾಗ ಮಾಡಿ ವಿಷ್ಣುಲೋಕವನ್ನು ಸೇರಿದ. ಆದರೆ ಈತನ ಆರು ಜನ ಮಕ್ಕಳಲ್ಲಿ ಹಿರಿಯನಾದ ಬರ್ಹಿಷದನು ಮಹಾಭಾಗ್ಯಶಾಲಿ. ಆತನು ತಂದೆಯ ಪಾಲಿನ ರಾಜ್ಯವನ್ನು ವಹಿಸಿಕೊಂಡು, ಬಹಳ ಧರ್ಮದಿಂದ ರಾಜ್ಯಭಾರ ಮಾಡುತ್ತಿದ್ದ. ಈತನಿಗೆ ಕರ್ಮಮಾರ್ಗ ಜ್ಞಾನಮಾರ್ಗಗಳು ಕರತಲಾಮಲಕವಾಗಿದ್ದವು. ಈತನು ಯಜ್ಞದ ಮೇಲೆ ಯಜ್ಞಗಳನ್ನು ಮಾಡಿ ಭೂಮಂಡಲವನ್ನೆಲ್ಲ ಯಜ್ಞಮಯವನ್ನಾಗಿ ಮಾಡಿದನು. ಈ ಯಜ್ಞ ಗಳಲ್ಲಿ ದರ್ಭೆಗಳನ್ನು ಪೂರ್ವಾಗ್ರವಾಗಿ ಹರಡುತ್ತಿದ್ದುದರಿಂದ ಈತನಿಗೆ ಪ್ರಾಚೀನಬರ್ಹಿ ಎಂದು ಹೆಸರಾಯಿತು. ಈತನು ಬ್ರಹ್ಮನ ಅಪ್ಪಣೆಯಂತೆ ಸಮುದ್ರರಾಜನ ಮಗಳಾದ ಶತಧ್ರುತಿಯನ್ನು ಮದುವೆಯಾದನು. ಅವಳು ಪರಮ ಸುಂದರಿ. ಅವಳು ವಿವಾಹಕಾಲ ದಲ್ಲಿ ಅಲಂಕಾರಮಾಡಿಕೊಂಡು ಪತಿಯೊಡನೆ ಅಗ್ನಿಯನ್ನು ಪ್ರದಕ್ಷಿಣೆ ಮಾಡುವಾಗ, ಅಗ್ನಿದೇವ ಆಕೆಯನ್ನು ಮೋಹಿಸಿದನಂತೆ! ಆಕೆ ಕಾಲ್ಗೆಜ್ಞೆಯನ್ನು ಝಣಝಣರೆನಿಸುತ್ತಾ ನಡೆದು ಬರುತ್ತಿದ್ದರೆ ದೇವ ದಾನವ ಗಂಧರ್ವಾದಿಗಳೆಲ್ಲ ಮರುಳಾಗಿಹೋದರಂತೆ. ಇಂತಹ ದಿವ್ಯಸುಂದರಿ ಶತಧ್ರುತಿಯಲ್ಲಿ ಪ್ರಾಚೀನಬರ್ಹಿ ಹತ್ತು ಮಂದಿ ಮಕ್ಕಳನ್ನು ಪಡೆದ. ಅವರೇ ಪ್ರಚೇತಸರು.

ಪ್ರಾಚೀನಬರ್ಹಿಯು ತನ್ನ ಮಕ್ಕಳನ್ನು ಪ್ರಜಾಸೃಷ್ಟಿಯ ಕಾರ್ಯಗಳನ್ನು ಕೈಕೊಳ್ಳು ವಂತೆ ಅಪ್ಪಣೆ ಮಾಡಿದನು. ಪ್ರಚೇತಸರು ‘ತಥಾಸ್ತು’ಎಂದು ಹೇಳಿ, ಪ್ರಜಾಸೃಷ್ಟಿಯ ಕಾರ್ಯಕ್ಕೆ ತಕ್ಕ ಶಕ್ತಿಯನ್ನು ಗಳಿಸುವುದಕ್ಕಾಗಿ ತಪಸ್ಸನ್ನು ಕೈಕೊಳ್ಳಬೇಕೆಂದು ನಿಶ್ಚಯಿಸಿದರು. ಅವರು ಆ ಕಾರ್ಯಕ್ಕಾಗಿ ಉತ್ತರ ದಿಕ್ಕನ್ನು ಹಿಡಿದು ಹೋಗುತ್ತಿರುವಾಗ ಹಾದಿಯಲ್ಲಿ ಸಮುದ್ರದಂತೆ ವಿಸ್ತಾರವಾಗಿಯೂ, ಮಹಾತ್ಮರ ಮನಸ್ಸಿನಂತೆ ತಿಳಿಯಾಗಿ ಶುದ್ಧ ವಾಗಿಯೂ ಇದ್ದ ನೀರಿನಿಂದ ತುಂಬಿದ ಒಂದು ಸರೋವರವು ಕಾಣಿಸಿತು. ಅದರಲ್ಲಿ ಬಣ್ಣಬಣ್ಣದ ಕಮಲಪುಷ್ಪಗಳು ತುಂಬಿದ್ದವು; ಹಂಸ, ಸಾರಸ, ಚಕ್ರವಾಕ ಮೊದಲಾದ ಹಕ್ಕಿಗಳು ಮಧುರವಾಗಿ ಹಾಡುತ್ತಿದ್ದವು. ಈ ಸಂಗೀತವನ್ನು ಕೇಳಿ ಸಂತೋಷದಿಂದ ತಲೆದೂಗುವಂತೆ ಕಾಣಿಸುತ್ತಿದ್ದುವು, ಆ ಸರೋವರದ ದಡದಲ್ಲಿದ್ದ ಗಿಡ ಮರ ಬಳ್ಳಿಗಳು. ಸರೋವರದ ಮೇಲಿನಿಂದ ಬೀಸುವ ತಂಗಾಳಿಯು ಹೂವುಗಳ ಸುವಾಸನೆಯನ್ನು ಹೊತ್ತು ತಂದು ದಾರಿಗರನ್ನು ಉಪಚರಿಸುತ್ತಿತ್ತು. ಪ್ರಚೇತಸರು ಆ ಸರೋವರದ ಸೌಂದರ್ಯವನ್ನು ಸವಿಯುತ್ತ ನಿಂತಿರಲು ದಿವ್ಯಮನೋಹರವಾದ ಗಂಧರ್ವಗಾನವು ಕೇಳಿಬಂತು. ಅದರ ಹಿಂದೆಯೆ ಬಂಗಾರದ ಮೈಬಣ್ಣದಿಂದ ತೊಳಗಿ ಬೆಳಗುವ ಮುಕ್ಕಣ್ಣನು ನೀರಿನ ಮಧ್ಯ ದಿಂದ ಮೇಲೆದ್ದು ಬಂದನು. ಆತನ ಸುತ್ತಲೂ ದೇವ ಪರಿವಾರವು ನೆರೆದು ಸ್ತ್ರೋತ್ರ ಮಾಡುತ್ತಿತ್ತು. ಪ್ರಚೇತಸರು ಅಚ್ಚರಿಗೊಂಡು ಪರಶಿವನ ಪಾದಕ್ಕೆ ಅಡ್ಡಬಿದ್ದರು. ಆತನು ಅವರನ್ನು ಬಾಯ್ತುಂಬ ಹರಸಿ, ಅವರ ತಪಸ್ಸು ಸಫಲವಾಗುವಂತೆ ಅವರಿಗೆ ರುದ್ರಗೀತೆ ಯನ್ನು ಉಪದೇಶಿಸಿ ಅಂತರ್ಧಾನನಾದನು (ಪರಿಶಿಷ್ಟ ನೋಡಿ).

ರುದ್ರಗೀತೆಯು ‘ಯೋಗಾದೇಶ’ವೆಂದು ಪ್ರಸಿದ್ಧವಾದ ಮಹಾಕೃತಿ. ಪ್ರಚೇತಸರು ಸಮುದ್ರ ಮಧ್ಯದಲ್ಲಿ ನಿಂತು, ಭಕ್ತಿಯಿಂದ ಹತ್ತುಸಹಸ್ರ ವರ್ಷಗಳವರೆಗೆ ಅದನ್ನು ಜಪಿಸುತ್ತಾ ತಪಸ್ಸು ಮಾಡಿದರು. ಅದರ ಫಲವಾಗಿ ಮಹಾವಿಷ್ಣುವು ಅವರಿಗೆ ಪ್ರತ್ಯಕ್ಷನಾಗಿ ‘ಅಯ್ಯಾ, ನಿಮ್ಮ ತಪಸ್ಸಿಗೆ ನಾನು ಮೆಚ್ಚಿದ್ದೇನೆ. ನಿಮಗೆ ತಕ್ಕವಳಾದ ‘ವಾರ್ಕ್ಷಿ’ ಎಂಬು ವಳು ನಿಮ್ಮ ಪತ್ನಿಯಾಗುತ್ತಾಳೆ. ಆಕೆ ಸಾಮಾನ್ಯಳಲ್ಲ. ಹಿಂದೆ ಕುಡುವೆಂಬ ಮಹರ್ಷಿ ತಪಸ್ಸು ಮಾಡುತ್ತಿದ್ದಾಗ ದೇವತೆಗಳು ಆತನ ತಪಸ್ಸನ್ನು ಕೆಡಿಸುವುದಕ್ಕಾಗಿ ‘ಪ್ರಮ್ಲೋಚೆ’ ಎಂಬ ಅಪ್ಸರೆಯನ್ನು ಕಳುಹಿಸಿದರು. ಪುಷಿಯು ಆಕೆಯ ಮೋಹಕ್ಕೆ ಒಳಗಾಗಿ ಆಕೆ ಯೊಡನೆ ಸಂಸಾರಜೀವನದಲ್ಲಿ ತೊಡಗಿರಲು ಆಕೆ ಗರ್ಭಿಣಿಯಾದಳು. ಆಕೆ ತನ್ನ ಗರ್ಭ ವನ್ನು ವೃಕ್ಷಗಳಲ್ಲಿಟ್ಟು ಸ್ವರ್ಗಕ್ಕೆ ಹಿಂದಿರುಗಿದಳು. ವೃಕ್ಷಗಳು ಬೆಳೆಸಿದ್ದರಿಂದ ಅವಳಿಗೆ ‘ವಾರ್ಕ್ಷಿ’ಎಂದು ಹೆಸರಾಯಿತು. ಆಕೆ ಈಗ ಗಂಡನನ್ನು ಹುಡುಕುತ್ತಿದ್ದಾಳೆ. ತಕ್ಷಣವೇ ನೀವು ಹೋಗಿ ಆಕೆಯನ್ನು ವರಿಸಿರಿ. ನೀವೆಲ್ಲರು ಶೀಲದಲ್ಲಿ, ಸ್ವಭಾವದಲ್ಲಿ, ಪ್ರೇಮದಲ್ಲಿ ಕೇವಲ ಒಂದೇ ದೇಹದಂತೆ ವ್ಯವಹರಿಸುತ್ತಿರುವುದರಿಂದ ನಿಮ್ಮೆಲ್ಲರಿಗೂ ಅವಳೊಬ್ಬಳೇ ಪತ್ನಿಯಾಗುವುದರಲ್ಲಿ ದೋಷವೇನೂ ಇಲ್ಲ. ಆಕೆಯಲ್ಲಿ ನಿಮಗೆ ಬ್ರಹ್ಮನಂತೆ ಮಹಾ ತೇಜಸ್ವಿಯಾದ ಒಬ್ಬ ಮಗ ಹುಟ್ಟುತ್ತಾನೆ. ಅವನ ಸಂತಾನ ಮೂರು ಲೋಕವನ್ನೂ ತುಂಬುತ್ತದೆ. ಇದರಿಂದ ನಿಮ್ಮ ತಂದೆಯ ಆಶಯ ನೆರವೇರಿ, ನೀವು ಜಗತ್ತಿನಲ್ಲಿ ಕೀರ್ತಿ ಶಾಲಿಗಳಾಗುವಿರಿ’ ಎಂದನು.

ಮಹಾವಿಷ್ಣುವಿನ ವರದಿಂದ ಪ್ರಚೇತಸರ ಮನಸ್ಸು ತೃಪ್ತಿಗೊಳ್ಳಲಿಲ್ಲ. ಅವರ ಗುರಿ ಮೋಕ್ಷ. ಆದ್ದರಿಂದ ಅವರು ಶ್ರೀಹರಿಯನ್ನು ಕುರಿತು ‘ಪ್ರಭು, ನಮ್ಮನ್ನು ಸಂಸಾರಿಗಳಾಗಿ ಮಾಡುತ್ತಿರುವೆ. ಈ ಬಂಧನದಿಂದ ನಮಗೆ ಬಿಡುಗಡೆ ಎಂದು? ಹೇಗೆ? ನಾವು ಸಂಸಾರಿ ಗಳಾಗಿರುವಷ್ಟು ಕಾಲ ನಿನ್ನನ್ನು ಮರೆಯದಂತೆ ನಮಗೆ ಸದ್ಭಕ್ತರಾದವರ ಸಹವಾಸವನ್ನಾ ದರೂ ಕರುಣಿಸು’ ಎಂದು ಬೇಡಿದರು. ಶ್ರೀಹರಿಯು ‘ತಥಾಸ್ತು’ ಎಂದು ಹೇಳಿ, ‘ಅಯ್ಯಾ, ನೀವು ಹತ್ತು ಲಕ್ಷ ವರ್ಷಗಳವರೆಗೆ ಇಹಲೋಕದ ಸುಖಗಳನ್ನು ಅನುಭವಿಸಿ, ಅನಂತರ ವೈರಾಗ್ಯದಿಂದ ತಪಸ್ಸುಮಾಡಿ ನನ್ನ ಲೋಕವನ್ನು ಪಡೆಯುವಿರಿ’ ಎಂದು ಹೇಳಿ ಮಾಯವಾದನು.

ಪ್ರಚೇತಸರು ಸಮುದ್ರದಿಂದ ಹೊರಗೆ ಬಂದು ನೋಡುತ್ತಾರೆ, ಆಕಾಶವನ್ನೆಲ್ಲ ಮುಚ್ಚ ಹೊರಟಿರುವಂತೆ ಮಹಾವೃಕ್ಷಗಳು ಭೂಮಿಯಲ್ಲೆಲ್ಲ ತುಂಬಿಹೋಗಿವೆ. ಹೀಗಾದರೆ ಜನ ಗಳು ಬೆಳೆಯೆಲ್ಲಿ ಬೆಳೆಯುವುದು? ಅವರಿಗೆ ರೇಗಿಹೋಯಿತು. ಅವರ ಬಾಯಿಂದ ಬೆಂಕಿ ಗಾಳಿಗಳು ಹೊರ ಹೊರಟು, ಆ ವೃಕ್ಷಗಳೆಲ್ಲ ಸುಟ್ಟು ಬೂದಿಯಾಗತೊಡಗಿದವು. ಆಗ ಬ್ರಹ್ಮನು ಅಲ್ಲಿ ಕಾಣಿಸಿಕೊಂಡು, ಅವರನ್ನು ಸಮಾಧಾನಮಾಡಿದನು. ಆಗ ವೃಕ್ಷಗಳ ಅಭಿಮಾನದೇವತೆ ಬಂದು ‘ವಾರ್ಕ್ಷಿ’ಯನ್ನು ಪ್ರಚೇತಸರಿಗೆ ಮದುವೆ ಮಾಡಿಕೊಟ್ಟಿತು. ಅವರು ಆ ‘ವಾರ್ಕ್ಷಿ’ಯಲ್ಲಿ ಒಬ್ಬ ಮಗನನ್ನು ಪಡೆದರು. ಪೂರ್ವಜನ್ಮದಲ್ಲಿ ದಕ್ಷಪ್ರಜಾ ಪತಿಯಾಗಿದ್ದವನೇ ಈಗ ಅವರ ಮಗನಾಗಿ ಹುಟ್ಟಿರುವವನು. ಅವನು ಹುಟ್ಟುವಾಗಲೆ ತನ್ನ ತೇಜಸ್ಸಿನಿಂದ ಇತರರ ತೇಜಸ್ಸನ್ನು ಅಡಗಿಸುತ್ತಿದ್ದನಾದ್ದರಿಂದ ಅವನಿಗೆ ‘ದಕ್ಷ’ನೆಂದೇ ಹೆಸರಾಯಿತು. ಪ್ರಚೇತಸರು ಮಹಾವಿಷ್ಣುವಿನ ಅಪ್ಪಣೆಯಂತೆ ಹತ್ತು ಲಕ್ಷ ವರ್ಷ ರಾಜ್ಯಭಾರ ಮಾಡಿದಮೇಲೆ, ಮಗನಾದ ದಕ್ಷನಿಗೆ ಪಟ್ಟವನ್ನು ಕಟ್ಟಿ, ತಾವು ಪಶ್ಚಿಮ ಸಮುದ್ರ ತೀರದಲ್ಲಿದ್ದ ಜಾಬಾಲಿ ಪುಷಿಯ ಆಶ್ರಮದಲ್ಲಿ ತಪೋನಿರತರಾದರು. ಅವರು ಯೋಗಾಭ್ಯಾಸದಿಂದ ಶ್ರೀಹರಿಯನ್ನು ಮೆಚ್ಚಿಸಿ, ಆತನ ಸನ್ನಿಧಿಯನ್ನು ಹೊಂದಿದರು.

