
\chapter{೧೦. “ಸಂಸಾರೋಯಮತೀವ ವಿಚಿತ್ರಃ”}

“ಅಮ್ಮ, ಸಂಸಾರದ ಗತಿ ಬಹು ವಿಚಿತ್ರವಾದುದು. ಬಿರುಗಾಳಿಯು ಮೇಘಗಳನ್ನು ಮೇಲಿಂದ ಮೇಲೆ ಚದರಿಸುವಂತೆ, ಮಹಾಶಕ್ತನಾದ ಕಾಲಪುರುಷನು ಈ ಜಗತ್ತಿನ ಜೀವಿ ಗಳ ಮೇಲೆ ಮತ್ತೆ ಮತ್ತೆ ತನ್ನ ಪ್ರಭಾವವನ್ನು ಬೀರುತ್ತಿದ್ದರೂ, ಅವರು ಅದನ್ನು ಸ್ವಲ್ಪವೂ ಅರ್ಥಮಾಡಿಕೊಳ್ಳಲಾರರಲ್ಲಾ! ತನ್ನ ಸುಖಕ್ಕಾಗಿ ಬಹು ಕಷ್ಟಪಟ್ಟು ಸಂಪಾದಿಸಿಕೊಂಡಿರು ವೆನೆಂದು ಜೀವನು ಭಾವಿಸುವುದೆಲ್ಲವೂ ಕಾಲವಶದಿಂದಲೆ ಬಂದುದು; ಅದು ಕ್ಷಣಮಾತ್ರ ದಲ್ಲಿ ನಾಶವಾಗುವುದೂ ಕಾಲರೂಪಿಯಾದ ಭಗವಂತನಿಂದಲೆ. ಅಜ್ಞಾನಿಯಾದ ಮಾನವ ಇದನ್ನು ಅರ್ಥಮಾಡಿಕೊಳ್ಳಲಾರದೆ ಅತ್ಯಂತ ದುಃಖಕ್ಕೆ ಗುರಿಯಾಗುತ್ತಾನೆ. ತನ್ನ ದೇಹ, ಮಡದಿ ಮಕ್ಕಳು ಹಣಕಾಸು, ಆಸ್ತಿಪಾಸ್ತಿಗಳು ಸ್ಥಿರವೆಂದು ನಂಬಿರುವುದರಿಂದ ಅವು ಕೈ ಬಿಟ್ಟುಹೋದಾಗ ಸಂಕಟಪಡುತ್ತಾನೆ. ದೈವಾಧೀನವಾಗಿ ಬಂದವು ದೈವಾಧೀನವಾಗಿಯೇ ಹೋಗುತ್ತವೆಂಬುದು ಅರ್ಥವಾಗದುದೇ ಅವನ ದುಃಖಕ್ಕೆ ಕಾರಣ. ಇನ್ನೂ ವಿಚಿತ್ರವಾದು ದೆಂದರೆ, ಜೀವನು ದೇವ ಮಾನವ ತಿರ್ಯಗಾದಿ, ಯಾವ ಜನ್ಮವನ್ನೇ ಎತ್ತಲಿ, ದುಃಖ ವಲ್ಲದೆ ಸುಖವನ್ನು ಅನುಭವಿಸುವುದಿಲ್ಲ; ಆದರೂ ಅವನಿಗೆ ವೈರಾಗ್ಯ ಹುಟ್ಟುವುದಿಲ್ಲ. ಕೊನೆಗೆ ನರಕದಲ್ಲಿ ಬಿದ್ದಿರುವಾಗಲೂ, ಅಲ್ಲಿನ ಆಹಾರಾದಿಗಳಲ್ಲಿಯೇ ಸುಖಭ್ರಾಂತಿ ಯನ್ನು ಹೊಂದಿ, ದೇಹರಕ್ಷಣೆಗಾಗಿ ಪ್ರಯತ್ನಿಸುವನೆ ಹೊರತು ಅದರಲ್ಲಿ ಜುಗುಪ್ಸೆ ಯನ್ನು ಪಡೆಯುವುದಿಲ್ಲ. ‘ನನ್ನ ಹೆಂಡತಿ, ನನ್ನ ಮಕ್ಕಳು, ನನ್ನ ಮನೆ, ನನ್ನ ಭೂಮಿ, ನನ್ನ ಪಶುಗಳು, ನನ್ನ ಬಂಧುಗಳು’ ಎಂಬ ಭ್ರಾಂತಿಯಿಂದ ತನಗೆ ತಾನೆ ‘ನನಗೆ ಸಮಾನ ರಾದ ಸುಖಪುರುಷನೇ ಇಲ್ಲ’ ಎಂದು ನಲಿಯುತ್ತಾನೆ. ತನ್ನ ದೇಹ ಮತ್ತು ಪರಿವಾರದ ರಕ್ಷಣೆಗಾಗಿ ಸದಾ ಹಂಬಲಿಸುತ್ತಿರುವ ಮಾನವ ಯುಕ್ತಾಯುಕ್ತ ವಿವೇಕವನ್ನೂ ಕಳೆದು ಕೊಂಡು ಪಾಪಕಾರ್ಯಗಳನ್ನು ಸಹ ಮಾಡುವನು.”

“ತಾಯಿ, ಸಂಸಾರಕ್ಕೆ ಸಿಕ್ಕಿಬಿದ್ದ ಜೀವನು ಜನನಪೂರ್ವದಿಂದ ಹಿಡಿದು ಮರಣಾ ನಂತರದವರೆಗೂ ಅನುಭವಿಸುವ ಕ್ಲೇಶಗಳು ವರ್ಣನಾತೀತ. ಆ ಜೀವನು ತನ್ನ ಕರ್ಮ ಶೇಷಕ್ಕೆ ತಕ್ಕಂತೆ ಈ ಭೂಮಿಯ ಮೇಲೆ ಜನ್ಮವನ್ನು ತಾಳಬೇಕಾಗುತ್ತದೆ. ಹಿಂದಿನ ತನ್ನ ಕರ್ಮಕ್ಕೆ ತಕ್ಕಂತೆ ಪುಣ್ಯಲೋಕವನ್ನೂ ಪಾಪಲೋಕವನ್ನೂ ಸೇರಿ, ಕರ್ಮಫಲ ಮುಗಿಯು ತ್ತಲೆ, ಶರೀರಸಂಬಂಧವನ್ನು ಉಂಟುಮಾಡತಕ್ಕ ಕರ್ಮಶೇಷವನ್ನು ಕಳೆದುಕೊಳ್ಳುವುದ ಕ್ಕಾಗಿ, ಮತ್ತೆ ಹುಟ್ಟಲೇಬೇಕು. ಮುಕ್ತಿಯಾಗಬೇಕಾದರೆ ಮಾನವ ಜನ್ಮಧಾರಣೆ ಬೇಕೇ ಬೇಕು. ಆದ್ದರಿಂದ ಜೀವನು ಭೂಮಿಗಿಳಿದುಬಂದಮೇಲೆ ಮೊದಲು ಪುರುಷನ ವೀರ್ಯ ವನ್ನು ಸೇರಿ, ಆ ಮೂಲಕ ಸ್ತ್ರೀಯ ಗರ್ಭಕೋಶವನ್ನು ಪ್ರವೇಶಿಸುತ್ತಾನೆ. ಅಲ್ಲಿ ಆ ಜೀವ ಒಂದು ರಾತ್ರಿಯೆಲ್ಲ ದ್ರವರೂಪವಾಗಿರುತ್ತದೆ. ಐದು ದಿನಗಳ ವೇಳೆಗೆ ಅದು ಪಕ್ವವಾಗಿ ನೀರುಗುಳ್ಳೆಯಂತೆ ಗುಂಡಾಗುತ್ತದೆ; ಹತ್ತು ದಿನಗಳಲ್ಲಿ ಅದು ಬೋರೆಯ ಹಣ್ಣಿನ ಆಕಾರ ವನ್ನು ಪಡೆದು ಗಟ್ಟಿಯಾಗುತ್ತದೆ. ಇದಾದ ಕೆಲವು ದಿನಗಳಲ್ಲಿ ಅದು ಮಾಂಸದ ಮುದ್ದೆಯೋ ಅಥವಾ ಮೊಟ್ಟೆಯೋ ಆಗುತ್ತದೆ. ಆಮೇಲೆ ಒಂದು ತಿಂಗಳಿಗೆ ಆ ಮಾಂಸ ಪಿಂಡದಲ್ಲಿ ತಲೆ, ಎರಡು ತಿಂಗಳಿಗೆ ಕೈಕಾಲು, ಮೂರು ತಿಂಗಳಿಗೆ ಉಗುರು ಕೂದಲು ಮೂಳೆ ಚರ್ಮಗಳು ಮತ್ತು ಸ್ತ್ರೀಪುರುಷ ಚಿಹ್ನೆಗಳು ಮೂಡುತ್ತವೆ; ನಾಲ್ಕನೆಯ ತಿಂಗಳಿ ನಲ್ಲಿ ರಕ್ತ, ಮಾಂಸ ಇತ್ಯಾದಿ ಸಪ್ತಧಾತುಗಳು ಅದರಲ್ಲಿ ಹುಟ್ಟುತ್ತವೆ; ಐದನೆಯ ತಿಂಗ ಳಲ್ಲಿ ಹಸಿವು ಬಾಯಾರಿಕೆಗಳು ಕಾಣಿಸಿಕೊಳ್ಳುತ್ತವೆ, ಆ ಪಿಂಡಕ್ಕೆ; ಆರನೆಯ ತಿಂಗಳಲ್ಲಿ ಅದರಲ್ಲಿ ಚಲನಶಕ್ತಿ ಮೂಡುತ್ತದೆ. ಆಗ ಅದಕ್ಕಾಗುವ ಸಂಕಟ ಅಷ್ಟಿಷ್ಟಲ್ಲ. ತಾಯಿಯ ಗರ್ಭಕೋಶದಲ್ಲಿ ಬಿಗಿದುಕೊಂಡು ಬಿದ್ದಿರುವ ಆ ಶಿಶುರೂಪಿಯನ್ನು ಅಲ್ಲಿನ ಕ್ರಿಮಿಗಳು ಕಚ್ಚುವುವು; ಆ ನೋವನ್ನು ತಾಳಲಾರದೆ ಆ ಮೃದುಶರೀರ ಮತ್ತೆ ಮತ್ತೆ ಮೂರ್ಛೆಗೊಂಡು ಮೇಲೇಳುವುದು; ತಾಯಿ ತಿಂದ ಉಪ್ಪು ಹುಳಿ ಖಾರಗಳು ಆ ಮೃದುದೇಹಕ್ಕೆ ತಾಕಿ ನೋವಾಗುವುದು; ಒಂದು ಕಡೆ ಗರ್ಭಕೋಶದ ಬಿಗಿ, ಮತ್ತೊಂದು ಕಡೆ ತಾಯಿಯ ಕರುಳುಗಳ ಬಂಧನ–ಈ ಪಂಜರದ ಮಧ್ಯದಲ್ಲಿ ಇಕ್ಕಟ್ಟಿಗೆ ಸಿಕ್ಕಿ, ಕೊರಳು ಬೆನ್ನುಗಳನ್ನು ಬಾಗಿಸಿ, ತಲೆಯನ್ನು ಹೊಟ್ಟೆಯಲ್ಲಿ ಹುದುಗಿಸಿಕೊಂಡು ಮಾಂಸದ ಮುದ್ದೆಯಂತೆ ಬಿದ್ದಿರಬೇಕು; ಏಳನೆಯ ತಿಂಗಳಲ್ಲಿ ಈ ಜೀವಿಗೆ ಸುಖದುಃಖಗಳ ಜ್ಞಾನವುಂಟಾಗುತ್ತದೆ. ಆ ವೇಳೆಗೆ ಪ್ರಸೂತಿ ವಾಯುವು ಹೊಟ್ಟೆಯಲ್ಲಿರುವ ಇತರ ಕ್ರಿಮಿಗಳ ಜೊತೆಯಲ್ಲಿ ಈ ಕೋಮಲ ದೇಹವನ್ನೂ ಸದಾ ಅತ್ತಿಂದಿತ್ತ ಕದಲಿಸುತ್ತಿರುತ್ತದೆ. ಆಗ ಆ ಜೀವಿಗೆ ತಾನು ಜನ್ಮಾಂತರಗಳಲ್ಲಿ ಮಾಡಿದ ಪಾಪಗಳೆಲ್ಲ ಜ್ಞಾಪಕಕ್ಕೆ ಬರುತ್ತವೆ. ಅವುಗಳ ಫಲವಾಗಿ ತಾನು ಈ ಜನ್ಮದಲ್ಲಿ ಎಂತಹ ದುಃಖಗಳನ್ನು ಅನುಭವಿಸಬೇಕೋ ಎಂದು ಆ ಜೀವಿ ಭಯದಿಂದ ನಡುಗುತ್ತಾ, ತನ್ನನ್ನು ಉದ್ಧರಿಸುಂತೆ ಭಗವಂತನನ್ನು ಶರಣುಹೋಗು ತ್ತದೆ.”

“ಹತ್ತು ತಿಂಗಳು ಗರ್ಭವಾಸ ನರಕದುಃಖ ಮುಗಿದಮೇಲೆ ಪ್ರಸೂತಿವಾಯುವಿನ ಹೊಡೆತಕ್ಕೆ ಸಿಕ್ಕ ಜೀವಿಯು ತಲೆಕೆಳಗಾಗಿ, ಬಹು ಕಷ್ಟದಿಂದ ತಾಯಿಯ ಗರ್ಭವನ್ನು ಬಿಟ್ಟು ಹೊರಬರುತ್ತದೆ. ಒಮ್ಮೆ ಭೂಸ್ಪರ್ಶವಾಗುತ್ತಲೆ ಅದರ ಜ್ಞಾನವೆಲ್ಲ ಹಾರಿಹೋಗುತ್ತದೆ. ಅದು ನೆಲಕ್ಕೆ ಬಿದ್ದು, ಹುಳುವಿನಂತೆ ಒದ್ದಾಡುತ್ತಾ ಗಟ್ಟಿಯಾಗಿ ರೋಧಿಸುತ್ತದೆ. ಇಲ್ಲಿಂದ ಮುಂದೆ ಶೈಶವದಲ್ಲೆಲ್ಲ ರೋದನವೆ ಅದರ ಲಕ್ಷಣ. ಹೊಟ್ಟೆ ಹಸಿದರೆ ರೋದನ. ಹೊಟ್ಟೆ ನೋವಾದರೆ ರೋದನ. ಸೊಳ್ಳೆ ತಿಗಣೆಗಳು ಕಚ್ಚಿದರೆ ರೋದನ, ತನ್ನ ದುಃಖವನ್ನು ಹೇಳಿಕೊಳ್ಳಲಾರದ ರೋದನ. ಈ ರೋದನವನ್ನು ಅರ್ಥಮಾಡಿಕೊಳ್ಳಲಾರದೆ ತಾಯಿ ಹೊಟ್ಟೆನೋವಾದಾಗ ಹಾಲು ಕುಡಿಸಬಹುದು, ಹಸಿವಾದಾಗ ಹೊಟ್ಟೆನೋವೆಂದು ಬೇವಿನ ರಸ ಕುಡಿಸಬಹುದು; ಬೇಡವೆನ್ನಲು ಬಾಯಿಲ್ಲದೆ ಅದು ಅನುಭವಿಸಬೇಕು. ಈ ಶೈಶವ ವನ್ನು ಮುಗಿಸಿದ ಜೀವಿ ಬಾಲ್ಯವನ್ನು ಕ್ರೀಡೆಯಲ್ಲಿ ಕಳೆಯುತ್ತಾನೆ. ಈ ಕಾಲದಲ್ಲಿ ಅವನಿಗೆ ಅಲಭ್ಯ ವಸ್ತುಗಳಲ್ಲಿ ಆಶೆ ಹೆಚ್ಚು. ಅವು ದೊರೆಯದಿದ್ದರೆ ಕೋಪ, ಕೋಪ ನಿಷ್ಫಲ ವಾದಾಗ ಅಳು.”

“ಇನ್ನು ಯೌವನಾವಸ್ಥೆಯ ಚರ್ಯೆಗಳನ್ನು ವಿಚಾರಿಸೋಣ. ಆಗ ಮನುಷ್ಯ ಸಾಮಾನ್ಯ ವಾಗಿ ಸಂಸಾರಿಯಾಗುತ್ತಾನೆ. ಸೂಜಿಗಲ್ಲು ಕಬ್ಬಿಣವನ್ನು ಎಳೆಯುವಂತೆ ಹೆಣ್ಣಿನ ರೂಪ ಪುರುಷನ ಮನಸ್ಸನ್ನು ಸೆಳೆಯುತ್ತದೆ. ಮಹಾಜ್ಞಾನಿಗಳು ಕೂಡ ಈ ಮೋಹಪಾಶಕ್ಕೆ ಒಳಗಾಗುತ್ತಾರೆ. ಅವರಿರಲಿ, ಸೃಷ್ಟಿಕರ್ತನಾದ ಬ್ರಹ್ಮನು ಕೂಡ ತನ್ನ ಮಗಳಾದ ಸರಸ್ವತಿ ಯನ್ನೆ ಮೋಹಿಸಿ, ಅವಳನ್ನು ಬೆನ್ನಟ್ಟಿಕೊಂಡು ಹೋದನಂತೆ! ಇನ್ನು ಸಾಮಾನ್ಯರ ವಿಚಾರ ಹೇಳುವುದೇನು? ಹೆಣ್ಣು ಒಮ್ಮೆ ಹುಬ್ಬು ಹಾರಿಸಿದಳೆಂದರೆ ಶೂರರಲ್ಲಿ ಶೂರನೂ ಅವಳ ದಾಸನಾಗುತ್ತಾನೆ. ಜಾರೆಯರಾದ ಹೆಣ್ಣುಗಳ ಮಾಯಾ ವಿಲಾಸಕ್ಕೂ ಸಂಚಿನ ಇಂಚರಕ್ಕೂ ಕೋಳುಹೋಗದ ಧೀರರು ಬಹು ಅಪೂರ್ವ. ಹುಲ್ಲು ಮುಚ್ಚಿದ ಹಾಳುಬಾವಿಯಂತೆ ಅವರು ಮೃತ್ಯುಸ್ವರೂಪರು. ಅಮ್ಮಾ, ಹೆಣ್ಣಿನ ಕಾರಣದಿಂದ ಆಗುವಷ್ಟು ಅನರ್ಥಗಳು ಜಗತ್ತಿನಲ್ಲಿ ಮತ್ತಾವುದರಿಂದಲೂ ಆಗುವುದಿಲ್ಲ. ಅವಳ ಸಂತೋಷಕ್ಕಾಗಿ ಕಾಮುಕನು ಮಾಡದಿರುವ ಪಾಪಕಾರ್ಯವೇ ಇಲ್ಲ. ಸಂಸಾರಿಯಾದವನು ಹೆಣ್ಣಿನ ಮಾತಿಗೆ ಮರುಳಾಗು ವನು, ಮಗುವಿನ ಮುದ್ದು ಮಾತಿಗೆ ಹಿಗ್ಗುವನು. ಅವರಿಗಾಗಿ ಅವನು ಹಗಲಿರುಳೂ ದುಡಿದು ದಣಿಯುವನು. ಅವರ ರಕ್ಷಣೆಗಾಗಿ ಅನೇಕರನ್ನು ಹಿಂಸಿಸಿ ಹಣ ಗಳಿಸುವನು. ಹೆಂಡಿರು ಮಕ್ಕಳು ತಿಂದು ತೇಗಿದಮೇಲೆ ಉಳಿದುದನ್ನು ತಾನು ತಿನ್ನುವನು. ಇಷ್ಟು ಕಷ್ಟಪಟ್ಟು ಅವನು ಗಳಿಸುವುದೇನು? ನರಕ!”

“ಅಮ್ಮ, ಎಷ್ಟೋ ಕಷ್ಟಪಟ್ಟು ದುಡಿದು ಸಂಸಾರವನ್ನು ನಡೆಸುವ ಈ ಮನುಷ್ಯ ಮುಪ್ಪು ಬರುತ್ತಲೆ ಕೆಲಸಕ್ಕೆ ಬಾರದವನಾಗುತ್ತಾನೆ. ಸವಿನುಡಿಗಳಿಂದ ರಮಿಸಿ ಅವನನ್ನು ಕಿತ್ತು ತಿನ್ನುತ್ತಿದ್ದ ಹೆಂಡತಿ ಮಕ್ಕಳು ಈಗ ಅವನನ್ನು ಕಡೆಯ ಗೂಟಕ್ಕೆ ಕಟ್ಟುತ್ತಾರೆ. ಅವನೊಂದು ಮುದಿ ಎತ್ತಿಗೂ ಕಡೆಯಾಗುತ್ತಾನೆ. ಮುಪ್ಪಿನಿಂದ ಶಕ್ತಿ ಕುಂದುತ್ತದೆ, ಚರ್ಮ ಸುಕ್ಕುಗಟ್ಟುತ್ತದೆ, ಜೀರ್ಣಶಕ್ತಿ ಉಡುಗಿ ರೋಗ ಕಾಲಿಡುತ್ತದೆ. ಅಂತಹವನು ಬಿದ್ದಲ್ಲಿಂದ ಕದಲಲಾರದೆ ನರಳುತ್ತ, ತನ್ನ ಹೆಂಡತಿ ಮಕ್ಕಳು ಹಂಗಿಸಿ ಹಾಕುವ ಕೂಳಿಗೆ ನಾಯಿಯಂತೆ ಕಾದಿರಬೇಕಾಗುತ್ತದೆ. ಕೊನೆಕೊನೆಗೆ ಶ್ಲೇಷ್ಮ ಹುಟ್ಟಿ, ಗಂಟಲು ಕಟ್ಟಿ, ಮೇಲುಸಿರು ಬಿಡುತ್ತಿರುವಾಗಲೂ ಈ ಪಾಪಿ ಜೀವನಿಗೆ ವೈರಾಗ್ಯ ಹುಟ್ಟುವುದಿಲ್ಲ. ಈ ದೇಹವನ್ನು ಬಿಟ್ಟು ಹೋಗಲಾರದೆ ಅವನು ಮರಣ ಸಂಕಟವನ್ನು ಅನುಭವಿಸುವನು. ಆದರೇನು? ಅವನು ಸಾವನ್ನು ತಪ್ಪಿಸಿಕೊಳ್ಳಬಲ್ಲನೆ? ಯಮದೂತರು ಬಂಧಿಸಿ ಅವನನ್ನು ಎಳೆದೊಯ್ಯುವರು. ಅವನು ನರಕಕ್ಕೆ ಬಿದ್ದು ಪರಿಪರಿಯಾದ ದುಃಖವನ್ನು ಅನುಭವಿಸು ವನು. ಪಾಪದಿಂದ ಅವನು ಸಂಪಾದಿಸಿದ ಐಶ್ವರ್ಯವೂ, ಐಶ್ವರ್ಯದಿಂದ ಪೋಷಿತವಾದ ಅವನ ದೇಹವೂ ಇಲ್ಲಿಯೇ ಉಳಿಯುವುವು. ಅವನು ತನ್ನೊಡನೆ ಕೊಂಡೊಯ್ಯುವುದು ಕೇವಲ ಪಾಪದ ಬುತ್ತಿಯೊಂದನ್ನು ಮಾತ್ರ.”

“ತಾಯಿ, ಜೀವನು ನಿತ್ಯನಾದುದರಿಂದ ಅವನಿಗೆ ಜರಾಮರಣಗಳೆಂತು? ಎಂಬ ಪ್ರಶ್ನೆ ಉದಿಸುವುದು ಸಹಜ. ಅದಕ್ಕೆ ಉತ್ತರ ಇಷ್ಟೆ. ಜೀವನಿಗೆ ನಾವು ಕಾಣುವ ಸ್ಥೂಲದೇಹ ವಲ್ಲದೆ ಸೂಕ್ಷ್ಮದೇಹವೊಂದು ಇದೆ. ಸ್ಥೂಲ ಶರೀರದಲ್ಲಿ ಎದ್ದು ಕಾಣುತ್ತಿದ್ದ ಪಂಚ ಭೂತಗಳೆ ಸೂಕ್ಷ್ಮ ಶರೀರದಲ್ಲಿ ಕಣ್ಣಿಗೆ ಕಾಣದಷ್ಟು ಸೂಕ್ಷ್ಮವಾಗಿರುತ್ತವೆ. ಜೀವನು ಸ್ಥೂಲಶರೀರವನ್ನು ಬಿಟ್ಟು ಸೂಕ್ಷ್ಮಶರೀರವನ್ನು ಸೇರುವುದೇ ಮರಣ. ಆ ಸೂಕ್ಷ್ಮ ಶರೀರ ಸ್ಥೂಲದೆಶೆಯನ್ನು ಹೊಂದಿ, ನಾನೆಂಬ ಅಭಿಮಾನಕ್ಕೆ ಒಳಗಾಗುವುದೆ ಜನನ. ಆದ್ದರಿಂದ ಜನನ ಮರಣವೆಂಬುದು ಜೀವನು ಮಾಡಿಕೊಳ್ಳುವ ಸ್ಥೂಲ ಸೂಕ್ಷ್ಮದೇಹಗಳ ಬದಲಾ ವಣೆ ಮಾತ್ರ. ಆದ್ದರಿಂದ ಇದೊಂದು ವಿಚಿತ್ರವಾದ ರೂಪಾಂತರ ಮಾತ್ರ. ಅಮ್ಮ, ವಸ್ತುಸ್ಥಿತಿ ಹೀಗಿರುವುದರಿಂದ ನಾವು ಜನನ ಮರಣದ ಹೆಸರನ್ನು ಕೇಳಿ ನಡುಗಬೇಕಾದು ದಿಲ್ಲ. ಹೆಂಡಿರು ಮಕ್ಕಳ ಮರಣವನ್ನು ಕಂಡು ಕಣ್ಣೀರು ಕರೆಯುವುದು ಕೇವಲ ಅಜ್ಞಾನ. ಸುಖದುಃಖಗಳನ್ನು ಸಮನಾಗಿ ಸ್ವೀಕರಿಸುವ ಧೀರನಾಗಬೇಕು. ಜೀವನ ಸ್ವರೂಪವನ್ನು ತಿಳಿದು ಮಮಕಾರವನ್ನು ತೊರೆಯಬೇಕು. ಆತ್ಮಸ್ವರೂಪವನ್ನು ಅರಿತು, ವೈರಾಗ್ಯವನ್ನು ವಹಿಸಿ, ಮಾಯಾ ಮಯವಾದ ಈ ಪ್ರಪಂಚದ ಸೆಳೆತಕ್ಕೆ ಸಿಕ್ಕದೆ, ಭಗವಂತನ ಪಾದಾರ ವಿಂದಗಳಲ್ಲಿ ಭಕ್ತಿಯನ್ನಿಡಬೇಕು. ಇದೇ ಪರಮ ಪುರುಷಾರ್ಥದ ಹಾದಿ.”

