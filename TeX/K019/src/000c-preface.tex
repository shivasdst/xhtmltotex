
\chapter*{ಪ್ರಕಾಶಕರ ನುಡಿ}

ಭಾಗವತ ಮಹಾಪುರಾಣವು ನಮ್ಮ ಹಿಂದೂಧರ್ಮದ ಪ್ರಧಾನ ಶಾಸ್ತ್ರಗಳಲ್ಲಿ ಒಂದು. ವಿಶೇಷವಾಗಿ ವೈಷ್ಣವ ಮತಾವಲಂಬಿಗಳಿಗೆ ಇದು ಪರಮಪೂಜ್ಯ ಆಕರ ಗ್ರಂಥವಾಗಿದೆ. ಆದರೆ ಈ ಬೃಹತ್​ಗ್ರಂಥವನ್ನು ಜನಸಾಮಾನ್ಯರೆಲ್ಲರೂ ಓದಿ ಅರ್ಥಮಾಡಿಕೊಳ್ಳುವುದು ಕಷ್ಟಸಾಧ್ಯ. ಆದ್ದರಿಂದ ಕನ್ನಡಿಗರಿಗೆ ಈ ಗ್ರಂಥದ ಪ್ರಯೋಜನವನ್ನು ಪಡೆಯಲು ಅನುಕೂಲವಾಗಲೆಂಬ ಉದ್ದೇಶದಿಂದ ‘ವಚನಭಾಗವತ’ವೆಂಬ ಈ ಗ್ರಂಥವನ್ನು ಕರ್ಣಾಟಕದ ಹಿರಿಯ ಸಾಹಿತಿಗಳಾದ ಶ್ರೀ ತ. ಸು. ಶಾಮರಾಯರು ತುಂಬ ಹೃದಯಂಗಮವಾಗಿ ಬರೆದಿರುತ್ತಾರೆ. ಇದು ಬಹುಬೇಗ ಜನಪ್ರಿಯವಾಗಿದೆ. ಈ ಪುಸ್ತಕವು ಎರಡು ಮುದ್ರಣಗಳನ್ನು ಕಂಡರೂ ಪ್ರತಿಗಳೆಲ್ಲ ಅತಿಶೀಘ್ರದಲ್ಲೇ ಮುಗಿದುಹೋಗಿ ಹಲವಾರು ವರ್ಷಗಳಾಗಿದ್ದುವು. ಅನಂತರ ಸನ್ಮಾನ್ಯ ಲೇಖಕರು ಈ ಗ್ರಂಥದ ಸರ್ವಸ್ವಾಮ್ಯವನ್ನು ಶ್ರೀರಾಮಕೃಷ್ಣ ಆಶ್ರಮಕ್ಕೆ ಕೊಟ್ಟರು. ಮೊದಲು ಇದು ರಾಮಕೃಷ್ಣ ಮಠ, ಬೆಂಗಳೂರಿನಿಂದ ಪ್ರಕಟವಾಗುತ್ತಿದ್ದು ಈಗ ಶ್ರೀರಾಮಕೃಷ್ಣ ಆಶ್ರಮ, ಮೈಸೂರಿನಿಂದ ಪ್ರಕಟಿಸುತ್ತಿದ್ದೇವೆ. ಶ್ರೀಮದ್ಭಾಗವತದಲ್ಲಿ ತುಂಬಿರುವ ಜ್ಞಾನ ಭಕ್ತಿಗಳೆಂಬ ಸುಧಾರಸವು ಕನ್ನಡಿಗರ ಮನೆಮನೆಗೂ ಹರಿದು ಸಕಲರನ್ನೂ ಪಾವನಗೊಳಿಸಲಿ ಎಂದು ಹಾರೈಸುತ್ತೇವೆ.

\begin{flushright}
\textbf{ಅಧ್ಯಕ್ಷರು}\\ಶ್ರೀರಾಮಕೃಷ್ಣ ಆಶ್ರಮ\\ಮೈಸೂರು ೫೭0 0೨0
\end{flushright}

\chapter*{ಭಕ್ತ-ಭಗವಂತ-ಭಾಗವತ}

“ಶುದ್ಧ ಜ್ಞಾನ-ಶುದ್ಧ ಭಕ್ತಿ ಎರಡೂ ಒಂದೇ. ಶುದ್ಧ-ಜ್ಞಾನ ಎಲ್ಲಿಗೆ ಕರೆದೊಯ್ಯುತ್ತದೋ, ಅಲ್ಲಿಗೇ ಶುದ್ಧ-ಭಕ್ತಿಯೂ ಕರೆದೊಯ್ಯುತ್ತದೆ. ಭಕ್ತಿಮಾರ್ಗ ಬಹಳ ಸುಲಭವಾದ ಮಾರ್ಗ.”

\begin{flushright}
\textbf{ಶ್ರೀರಾಮಕೃಷ್ಣ ವಚನವೇದ, ಪೂರ್ವಾರ್ಧ, ಪು. ೨೪೮}
\end{flushright}

“ಭಗವಂತನಿಲ್ಲದೆ ಭಕ್ತನಿರಲಾರದ ಹಾಗೆ, ಭಕ್ತನಿಲ್ಲದೆ ಭಗವಂತನೂ ಇರಲಾರ. ಆಗ ಭಕ್ತ ರಸವಾಗುತ್ತಾನೆ, ಭಗವಂತ ಆಸ್ವಾದಕನಾಗುತ್ತಾನೆ. ಭಕ್ತ ತಾವರೆ ಹೂವು ಆಗುತ್ತಾನೆ, ಭಗವಂತ ತನ್ನ ಮಾಧುರ್ಯವನ್ನು ತಾನೇ ಆಸ್ವಾದಿಸಲೋಸುಗ ಎರಡಾಗಿದ್ದಾನೆ. ಈ ಕಾರಣದಿಂದಲೇ ರಾಧಾಕೃಷ್ಣಲೀಲೆ.”

\begin{flushright}
\textbf{ಪೂರ್ವಾರ್ಧ, ಪು. ೨೮೪}
\end{flushright}

“ಭಗವಂತನೇ ಒಂದು ರೂಪದಲ್ಲಿ ಭಾಗವತ ಆಗಿದ್ದಾನೆ; ಆದ್ದರಿಂದ ವೇದ ಪುರಾಣ, ತಂತ್ರ ಇವನ್ನೆಲ್ಲ ಪೂಜೆ ಮಾಡಬೇಕು. ಮತ್ತೆ ಆತ ಇನ್ನೊಂದು ರೂಪದಲ್ಲಿ ಭಕ್ತನಾಗಿದ್ದಾನೆ, ಭಕ್ತನ ಹೃದಯ ಆತನ ಬೈಠಕ್​ಖಾನೆ. ಅಲ್ಲಿಗೆ ಹೋದರೆ ಸುಲಭ ವಾಗಿ ಮನೆಯ ಯಜಮಾನನನ್ನು ನೋಡಬಹುದು. ಆದ್ದರಿಂದ ಭಕ್ತನನ್ನು ಪೂಜಿಸಿದರೆ ಭಗವಂತನನ್ನು ಪೂಜಿಸಿದಂತೆಯೇ.”

\begin{flushright}
\textbf{೨೪-ಮೇ ೧೮೮೪}
\end{flushright}

“ಆತನಿಗೆ ಹಲವಾರು ವಿಧದಿಂದ ಸೇವೆ ಸಲ್ಲಿಸಬಹುದು. ಪ್ರೇಮಿಕ ಭಕ್ತ ಆತ ನೊಡನೆ ವಿವಿಧ ರೂಪದಲ್ಲಿ ಆನಂದಪಡುತ್ತಾನೆ. ಕೆಲವು ವೇಳೆ ಯೋಚಿಸುತ್ತಾನೆ. ‘ನೀನು ತಾವರೆ, ನಾನು ದುಂಬಿ’ ಎಂದು. ಇನ್ನು ಕೆಲವು ವೇಳೆ ಭಾವಿಸುತ್ತಾನೆ, ‘ನೀನು ಸಚ್ಚಿದಾನಂದ ಸಾಗರ, ನಾನು ಅದರಲ್ಲಿ ಮೀನು’ ಎಂದು. ಮತ್ತೆ ಕೆಲವು ವೇಳೆ ಯೋಚಿಸುತ್ತಾನೆ. ‘ನಾನು ನಿನ್ನ ನರ್ತಕಿ’ ಎಂದು. ಆತನ ಮುಂದೆ ಹಾರುತ್ತಾನೆ, ನರ್ತಿಸುತ್ತಾನೆ. ಕೆಲವು ವೇಳೆ ಸಖೀಭಾವ—ದಾಸಿಭಾವ. ಕೆಲವು ವೇಳೆ ವಾತ್ಸಲ್ಯ ಭಾವ—ಯೊಶೋದೆಯಂತೆ. ಕೆಲವು ವೇಳೆ ಸತೀ ಭಾವ—ಮಧುರಭಾವ—ಗೋಪಿಯಂತೆ.”

\begin{flushright}
\textbf{ಉತ್ತರಾರ್ಧ, ಪು. ೪೨}
\end{flushright}

\chapter*{ಮುನ್ನುಡಿ}

ಮೊದಲು ಭಾಗವತ ಕಥೆಯನ್ನು ಹೇಳಿದವನು ಸಾಕ್ಷಾತ್ ಶುಕಮುನಿ. ವ್ಯಾಸರ ಮಗ ಅವನು, ಹುಟ್ಟಿದಾಗಲೇ ಬ್ರಹ್ಮಜ್ಞಾನಿಯಾಗಿದ್ದವನು. ಅವನ ಬ್ರಹ್ಮಜ್ಞಾನವನ್ನು ಕಂಡು ಅವನಪ್ಪನೂ ಬೆಕ್ಕಸ ಬೆರಗಾದ–ಒಮ್ಮೆ ಶುಕಮುನಿಯು ಕಾಡಿನಲ್ಲಿ ಹೋಗುತ್ತಿದ್ದಾಗ, ಹಾದಿಯ ಸರೋವರದಲ್ಲಿ ಸ್ನಾನ ಮಾಡುತ್ತಿದ್ದ ಸ್ತ್ರೀಯರು ಆತ ತಮ್ಮತ್ತ ನೋಡಿ ದಾಗಲೂ ನಾಚಿಕೆಗೊಳ್ಳಲಿಲ್ಲ. ಅದೇ ಅವರಪ್ಪ ವ್ಯಾಸಮುನಿ ದೂರದಿಂದ ಬರುತ್ತಿರುವು ದನ್ನು ಕಾಣುತ್ತಲೆ ಆ ಹೆಣ್ಣುಗಳೆಲ್ಲ ಬುಡುಬುಡು ಮೇಲಕ್ಕೆ ಬಂದು ತಮ್ಮ ಬಟ್ಟೆಗಳನ್ನು ಧರಿಸಿದರು. ಇದನ್ನು ಕಂಡ ವ್ಯಾಸರು ಆಶ್ಚರ್ಯದಿಂದ ‘ಯುವಕನಾದ ಶುಕನನ್ನು ಕಂಡರೆ ನಾಚಿಕೆಯಾಗಲಿಲ್ಲ. ವಯಸ್ಸಾದ ನನ್ನನ್ನ ಕಂಡರೆ ನಾಚಿಕೆಯೇ?’ ಎಂದು ಆ ಹೆಂಗಸರನ್ನು ವಿಚಾರಿಸಿದರು. ಆಗ ಆ ಹೆಂಗಸರು ಹೇಳಿದರು ‘ಶುಕ ಯುವಕ. ನಿಜ; ಆದರೆ ಅವನ ಮನಸ್ಸು ಲಿಂಗಾತೀತ. ನೀವು ವಯಸ್ಸಾದವರು, ಆದರೆ ನೀವು ಪುರುಷರೆಂಬ ಭಾವ ನಿಮ್ಮಲ್ಲಿನ್ನೂ ನಾಟಿದೆ.’ ಈ ಮಾತನ್ನು ಕೇಳಿ ವ್ಯಾಸರು ಮಗನ ಮಹತ್ತನ್ನು ಅರ್ಥಮಾಡಿ ಕೊಂಡರು. ವ್ಯಾಸರೇನು ಸಾಮಾನ್ಯರೇ? ಹದಿನೆಂಟು ಮಹಾಪುರಾಣಗಳನ್ನು ಜಗತ್ತಿಗೆ ನೀಡಿದವರು ಅವರು. ಜ್ಞಾನದಲ್ಲಿ, ವಿದ್ಯೆಯಲ್ಲಿ, ಅನುಭವದಲ್ಲಿ ಹಿಮಾಲಯ ಶಿಖರ ಸದೃಶರಾದವರು. ಆ ಮಹಾನುಭಾವರ ಮಗ ಶುಕ. ಈತನು ಮಹಾಯೋಗಿ, ಆಜನ್ಮ ಬ್ರಹ್ಮಜ್ಞಾನಿ. ಆತನು ತನ್ನ ತಂದೆಯಿಂದ ಕಲಿತ ಭಾಗವತ ಪುರಾಣವನ್ನು ಮೊಟ್ಟ ಮೊದಲಬಾರಿ ಪರೀಕ್ಷಿದ್ರಾಜನಿಗೆ ಹೇಳುವನು.

ಶುಕಮುನಿ ಪರೀಕ್ಷಿತನಿಗೆ ಭಾಗವತವನ್ನು ಏಕೆ ಹೇಳಿದ? ಅದಕ್ಕೆ ಕಾರಣ ಇದು. ಪರೀಕ್ಷಿದ್ರಾಜ ಒಮ್ಮೆ ಕಾಡಿಗೆ ಹೋದಾಗ ಅಲ್ಲಿದ್ದ ಪರ್ಣಶಾಲೆಯೊಂದನ್ನು ಪ್ರವೇಶಿಸಿದ. ಅದರಲ್ಲಿದ್ದ ಮುನಿ ಬಾಹ್ಯಪ್ರಪಂಚದ ಪರಿವೆಯೆ ಇಲ್ಲದೆ ಧ್ಯಾನಮಗ್ನನಾಗಿದ್ದ. ರಾಜನಿಗೆ ಇದು ಅರ್ಥವಾಗಲಿಲ್ಲ, ಇದು ತನ್ನ ಮೇಲಿನ ಅವಜ್ಞೆ ಎಂದು ಭಾವಿಸಿದನು. ಆಗ ಆತನು ಹತ್ತಿರದಲ್ಲಿ ಬಿದ್ದಿದ್ದ ಒಂದು ಸತ್ತ ಹಾವನ್ನು ಮುನಿಯ ಕೊರಳಲ್ಲಿ ಹಾಕಿ ಹೊರಟು ಹೋದನು. ಅವನತ್ತ ಹೋಗುತ್ತಲೆ ಇತ್ತ ಆ ಮುನಿಯ ಮಗ ಅಲ್ಲಿಗೆ ಬಂದ. ತಂದೆಯ ಕೊರಳಲ್ಲಿದ್ದ ಸತ್ತ ಹಾವನ್ನು ಕಾಣುತ್ತಲೆ ಕೋಪದಿಂದ ಕಿಡಿಕಿಡಿಯಾಗಿ, ‘ಈ ಕೃತ್ಯವನ್ನು ಮಾಡಿದವನು ಏಳು ದಿನಗಳೊಳಗಾಗಿ ಹಾವು ಕಚ್ಚಿ ಸಾಯಲಿ’ ಎಂದು ಶಪಿಸಿದನು–ಇದು ಪರೀಕ್ಷಿದ್ರಾಜನಿಗೆ ಗೊತ್ತಾಯಿತು. ಆತ ಈ ಶಾಪದಿಂದ ಪಾರಾಗುವ ಬಗೆ ಹೇಗೆಂದು ಚಿಂತಾಮಗ್ನನಾದ. ತಿಳಿದವರು ಆತನಿಗೆ ಹೇಳಿದರು–‘ಪುಷಿಯ ಶಾಪದಿಂದ ಪಾರಾಗು ವುದು ಸಾಧ್ಯವಿಲ್ಲ; ಸಾಯುವುದಕ್ಕೆ ಮುಂಚೆ ಪಾಪಗಳನ್ನೆಲ್ಲ ಕಳೆದುಕೊಂಡು ಮುಕ್ತ ನಾಗಲು ಪ್ರಯತ್ನಿಸುವುದೊಂದೆ ಈಗ ಉಳಿದಿರುವ ದಾರಿ.’ ಆದರೆ ಕೇವಲ ಏಳು ದಿನಗಳ ಅವಧಿಯಲ್ಲಿ–ಇಷ್ಟು ಅಲ್ಪಕಾಲದಲ್ಲಿ–ಪಾಪಗಳನ್ನೆಲ್ಲ ಕಳೆದುಕೊಳ್ಳುವುದು ಹೇಗೆ? ಅಜ್ಞಾನದ ಪರೆಯನ್ನು ಅಷ್ಟು ಬೇಗ ಕಳಚಿಕೊಳ್ಳುವುದಕ್ಕಾಗುತ್ತದೆಯೆ? ಈ ಪ್ರಶ್ನೆಗೆ ‘ಶುಕ ಮುನಿಯಿಂದ ಭಾಗವತವನ್ನು ಕೇಳಿದರೆ ಸಾಧ್ಯ’ ಎಂಬ ಉತ್ತರ ಸಿಕ್ಕಿತು. ಮಹಾಯೋಗಿ ಯಾದ ಶುಕ ಮುನಿಯು ಹೇಳುವ ಸಾಕ್ಷಾತ್ ಭಗವಂತನ ಲೀಲೆಯಿಂದ ತುಂಬಿ ತುಳುಕು ತ್ತಿರುವ ಭಾಗವತವನ್ನು ಭಕ್ತಿ ಶ್ರದ್ಧೆಗಳಿಂದ ಕೇಳಿದವನು ಮುಕ್ತನಾಗುತ್ತಾನೆ, ಇದರಲ್ಲಿ ಸಂಶಯವಿಲ್ಲ–ಎಂಬ ಭರವಸೆಯನ್ನು ಹೊಂದಿ, ಪರೀಕ್ಷಿತನು ಶುಕ ಮುನಿಯಿಂದ ಭಾಗವತವನ್ನು ಕೇಳುತ್ತಾನೆ. ಅದು ಮುಗಿಯುವಷ್ಟರಲ್ಲಿ, ಪುಷಿಶಾಪದ ಗಡುವನ್ನು ಅನುಸರಿಸಿ ತಕ್ಷಕನು ಪರೀಕ್ಷಿತನನ್ನು ಕಚ್ಚಿ ಕೊಲ್ಲಲೆಂದು ಬರುವನು. ಆ ವೇಳೆಗೆ ರಾಜನ ಮನಸ್ಸು ಕರ್ಮ ಕಳೇಬರವನ್ನು ಕಳಚಿಕೊಂಡು ಭಗವಂತನಲ್ಲಿ ಒಂದಾಗಲು ಸಿದ್ಧ ವಾಗಿತ್ತು. ಅಂತಹ ಅಪೂರ್ವ ಶಕ್ತಿ ಆತನಲ್ಲಿ ಉದಿಸಿದುದು ಭಾಗವತ ಶ್ರವಣ ಮನನ ಗಳಿಂದ. ಇದರಿಂದ ಭಾಗವತ ಗ್ರಂಥದ ಹಿರಿಮೆ ಗರಿಮೆಗಳೇನೆಂಬುದು ನಮಗೆ ಅರ್ಥ ವಾಗುತ್ತದೆ. ಭಾಗವತದ ಶ್ರವಣದಿಂದ ಜೀವಿಯ ವಾಸನಾಸಮುದ್ರ ಬತ್ತಿಹೋಗುತ್ತದೆ. ಈ ಜಗತ್ತನ್ನು ಬಿಟ್ಟು ಹೋಗುವಾಗ ಈ ಜೀವಿ ಹುರಿದ ಬೀಜದಂತಾಗುವನು. ಇನ್ನವನು ಮೊಳೆಯಲಾರ, ಇನ್ನೊಮ್ಮೆ ಅವನು ಪ್ರಪಂಚಕ್ಕೆ ಬರುವುದಿಲ್ಲ. ಭಗವದ್ಭಕ್ತಿಯ ಆವುಗೆ ಯಲ್ಲಿ ಬೆಂದ ಆ ಜೀವ ಮತ್ತೊಮ್ಮೆ ಸಂಸಾರ ಚಕ್ರದಮೇಲೆ ಬರುವುದಿಲ್ಲ.

ಪುರಾಣಗಳಲ್ಲಿ ಭಾಗವತದ ಸ್ಥಾನ ವಿಶಿಷ್ಟವಾದುದು. ಭಾರತವನ್ನೂ ಹದಿನೇಳು ಮಹಾ ಪುರಾಣಗಳನ್ನೂ ಬರೆದರೂ ಕೂಡ ವ್ಯಾಸರ ಮನಸ್ಸಿಗೆ ಶಾಂತಿ ಇರಲಿಲ್ಲವಂತೆ! ಒಮ್ಮೆ ನಾರದರನ್ನು ಕಂಡಾಗ, ವ್ಯಾಸರು ಇದನ್ನವರಿಗೆ ತಿಳಿಸಿದರು. ಆಗ ನಾರದರು ‘ಶಾಂತಿ ಬೇಕಾದರೆ ಭಾಗವತವನ್ನು ಬರಿ. ಆಗ ಮನಸ್ಸು ಪೂರ್ಣವಾಗುತ್ತದೆ’ ಎಂದರು. ಅದರಂತೆ ವ್ಯಾಸರು ಭಾಗವತವನ್ನು ಬರೆದು ಶಾಂತಿಸುಖವನ್ನು ಪಡೆದರು. ನಾವು ಭಾರತದಲ್ಲಿ ಶ್ರೀಕೃಷ್ಣನ ವೃತ್ತಾಂತವನ್ನು ಕಾಣುವೆವಾದರೂ ಅದು ಆತನ ಜೀವನದ ಉತ್ತರಾರ್ಧ. ಶ್ರೀಕೃಷ್ಣ ಮಗುವಾಗಿದ್ದಾಗ, ಗೋಪಿಯರೊಡನೆ ಆಟವಾಡುತ್ತಿದ್ದಾಗ ಯಾವ ಪ್ರೇಮ ವನ್ನು ವ್ಯಕ್ತಪಡಿಸಿದನೋ, ಅದರ ಮುಂದೆ ಶ್ರೀಕೃಷ್ಣನ ರಾಜಕಾರಣಪಟುತ್ವ–ಆತನ ಗೀತಾಬೋಧನೆ ಕೂಡ–ಪೇಲವವಾಗಿ ಹೋಗುತ್ತದೆ. ಶ್ರೀಕೃಷ್ಣನು ತನ್ನ ವಯಸ್ಸಿನೊಡನೆ ಬುದ್ಧಿ ಮನಸ್ಸುಗಳನ್ನೂ ಬೆಳೆಸಿಕೊಂಡು ಪರಿಣತನಾಗಲಿಲ್ಲವೆ? ಅವನ ಬಾಲಭಾಷಿತ ಕ್ಕಿಂತ ಪರಿಣತ ವಯಸ್ಸಿನ ನುಡಿಗಳು ಮೌಲ್ಯದಲ್ಲಿ ಮಿಗಿಲಾದುವಲ್ಲವೆ? ಎಂಬ ಪ್ರಶ್ನೆ ಏಳಬಹುದು. ಅಂತಹ ಪ್ರಶ್ನೆ ನಮ್ಮಂತಹವರ ವಿಚಾರದಲ್ಲಿ ನ್ಯಾಯವಾಗಿಯೂ ಇರು ತ್ತದೆ. ನಾವೆಲ್ಲ ವಿಕಾಸದ ಏಣಿಯಲ್ಲಿ ಅಂಗುಲ ಅಂಗುಲವಾಗಿ ತೆವಳಿಕೊಂಡು ಮೇಲೇ ರಲು ಪ್ರಯತ್ನಿಸುತ್ತಿದ್ದೇವೆ. ನಾವು ಬೆಳೆಯುತ್ತೇವೆ, ಬದಲಾಯಿಸುತ್ತೇವೆ. ಆದರೆ ಶ್ರೀಕೃಷ್ಣ ಈ ಜಾತಿಗೆ ಸೇರಿದವನಲ್ಲ. ಆತನು ಹುಟ್ಟುವಾಗಲೆ ಪೂರ್ಣಾತ್ಮನಾಗಿ ಬಂದ. ಇತರ ಅವತಾರಗಳೆಲ್ಲ ಕಲ, ಅಂಶ, ಅಂಶದ ಅಂಶ–ಈ ಗುಂಪಿಗೆ ಸೇರಿದುವು. ಶ್ರೀಕೃಷ್ಣ ನಾದರೊ ಸ್ವಯಂ ಭಗವಂತ, ಆತನು ಬರುವಾಗಲೆ ಪೂರ್ಣಾತ್ಮನಾಗಿ ಬಂದ. ಇತರರಂತೆ ಆತ ಸಾಧನೆಯ ಮೂಲಕ ಪೂರ್ಣತೆಯ ಶಿಖರವನ್ನು ಏರಲಿಲ್ಲ. ‘ದಾರಿ ಬೇರೆ, ಗುರಿ ಬೇರೆ’ ಎಂದು ಆತ ಹೇಳಲಿಲ್ಲ. ಅದಕ್ಕೆ ಬದಲಾಗಿ ‘ನಾನೆ ದಾರಿ, ನಾನೆ ಗುರಿ’ ಎಂದ. ತನ್ನ ಪರಬ್ರಹ್ಮಸ್ವರೂಪವನ್ನು ಆತ ಕ್ಷಣಕಾಲವಾದರೂ ಮರೆತವನಲ್ಲ. ಇಂತಹ ವ್ಯಕ್ತಿ, ಶ್ರೇಷ್ಠ ಪಾತ್ರಿಗಳು ದೊರೆತೊಡನೆಯೆ ತನ್ನ ಸರ್ವಶ್ರೇಷ್ಠವಾದ ಬೋಧನೆಯನ್ನು ಅವರಿಗೆ ನೀಡಿದ. ಗೋಪಿಯರು ಶ್ರೀಕೃಷ್ಣನನ್ನು ಪ್ರೀತಿಸಿದರು, ಆ ಪ್ರೀತಿಗೆ ತಮ್ಮ ಸರ್ವಸ್ವವನ್ನೂ ಅರ್ಪಿಸಿದರು. ಅವರ ಪ್ರೀತಿಗೆ ಸಮಾನವಾದುದು ಮತ್ತೊಂದಿಲ್ಲ, ಈ ಪ್ರಪಂಚದಲ್ಲಿ. ಶ್ರೀಕೃಷ್ಣನು ಗೀತಾಚಾರ್ಯನಾಗಿರುವುದು, ಪಾಂಚಜನ್ಯವನ್ನು ಧರಿಸಿರುವುದು ಗೋಪಿಯರಿ ಗಿಂತ ಕೆಳಮಟ್ಟದ ಭಕ್ತರಿಗಾಗಿ. ಅವನು ಗೋಪಿಯರಿಗೆ ‘ಗೋಪೀಜನವಲ್ಲಭ’, ‘ಮುರಳೀಗಾನಲೋಲ’. ಆ ಕೊಳಲ ಗಾನಕ್ಕೆ ಪರವಶರಾಗದವರಾರು? ಸ್ತ್ರೀಪುರುಷರು ಅದರಿಂದ ಆಕರ್ಷಿತರಾದರು. ಪಶುಪಕ್ಷಿಪ್ರಾಣಿಗಳು ಆಕರ್ಷಿತವಾದವು. ಯಮುನೆ ಕೂಡ ಸಾವಧಾನವಾಗಿ ಹರಿದಳು. ಇನ್ನು ಗೋಪಿಯರು, ಅಯಸ್ಕಾಂತದ ಕಡೆಗೆ ಹಾರಿಹೋಗುವ ಕಬ್ಬಿಣದ ಪುಡಿಯಂತೆ ಗಾನದ ಇಂಪಿಗೆ ಮನಸೋತು ಓಡಿ ಹೋದರೆಂಬುದು ಏನಾಶ್ಚರ್ಯ? ಶ್ರೀಕೃಷ್ಣನು ಕೌರವರಿಗೆ ತನ್ನ ವಿಶ್ವರೂಪವನ್ನೆ ತೋರಿಸಿದ; ಆದರೇನು? ಅವರು ಅದರಿಂದ ಆಕರ್ಷಿತರಾದರೆ? ಅವನ ಪರಮ ಭಕ್ತನೆನಿಸಿಕೊಂಡ ಅರ್ಜುನನ ಕೈಲಿ ಯುದ್ಧ ಮಾಡಿಸ ಬೇಕಾದರೆ, ವಿಶ್ವರೂಪದರ್ಶನದ ಮೂಲಕ, ಮುಂದೆ ಏನಾಗುವುದೆಂಬುದನ್ನೂ ತೋರಿ, ಅವನನ್ನು ಸಂಶಯದಿಂದ ಮೇಲೆತ್ತಬೇಕಾಯಿತು. ಶ್ರೀಕೃಷ್ಣ ಇದಾವುದನ್ನೂ ಪ್ರದರ್ಶಿಸ ಲಿಲ್ಲ, ಗೋಪಿಯರಿಗೆ. ಅವರಿಗೆ ಶ್ರೀಕೃಷ್ಣನ ವಿಶ್ವರೂಪ ಬೇಡ, ರಾಜವೈಭವ ಬೇಡ; ಅವರು ಸರಳನಾದ ಗೋಪಬಾಲನನ್ನು ಪ್ರೀತಿಸಿದರು. ಆ ಪ್ರೀತಿಗೆ ನಮಗೇನು ಫಲ ಎಂಬ ಚೌಕಾಶಿಯೇನೂ ಇಲ್ಲ. ಪ್ರೀತಿಸುವುದು ಅವರ ಧರ್ಮ. ‘ಪ್ರೀತಿಗಾಗಿ ಪ್ರೀತಿ’ ಎಂಬ ಪರಮಾದರ್ಶವನ್ನು ಅವರಲ್ಲಿ ಕಾಣುತ್ತೇವೆ.

ಭಾಗವತದಲ್ಲಿ ಬರುವ ಶ್ರೀಕೃಷ್ಣನ ವ್ಯಕ್ತಿತ್ವವು ಬಹುಮುಖವಾದುದು. ಅದನ್ನು ಅರ್ಥಮಾಡಿಕೊಳ್ಳುವುದು ಬಹುಕಷ್ಟ. ಶ್ರೀಕೃಷ್ಣನನ್ನು ಪೂಜಿಸುವವರಂತೆ ಅವನನ್ನು ಟೀಕಿಸುವವರಿಗೂ ಬರಗಾಲವಿಲ್ಲ. ಆತನ ವ್ಯಕ್ತಿತ್ವ ಭರತಖಂಡವನ್ನೆಲ್ಲ ವ್ಯಾಪಿಸಿದೆ. ನಮ್ಮ ಜನಜೀವನದಲ್ಲಿ ಅದು ಓತಪ್ರೋತವಾಗಿ ಸೇರಿಹೋಗಿದೆ–ಆತನ ಹಲವು ಹೆಸರು ಗಳಲ್ಲಿ ಒಂದನ್ನು ನಮ್ಮ ಮಗುವಿಗೆ ಇಡುತ್ತೇವೆ, ಅವನಂತಹ ಮಗುವಾಗಲೆಂದು ಆಶಿಸುತ್ತೇವೆ, ಅವನಂತಹ ಸ್ನೇಹಿತ ನಮಗೆ ಸಿಕ್ಕಲೆಂದು ಅಪೇಕ್ಷಿಸುತ್ತೇವೆ, ಅವನಂತಹ ಪತಿ ದೊರೆಯಲೆಂದು ನಮ್ಮ ಹೆಣ್ಣುಮಕ್ಕಳು ಬಯಸುತ್ತಾರೆ. ನಾವು ಆಚರಿಸುವ ಹಬ್ಬ ಹುಣ್ಣಿಮೆಗಳಲ್ಲಿ ಎಷ್ಟೊಂದು ಶ್ರೀಕೃಷ್ಣನಿಗೆ ಸಂಬಂಧಿಸಿವೆ! ಹಾಗೆಯೆ ನಮ್ಮ ಸಾಹಿತ್ಯ ರಾಶಿಯಲ್ಲಿ ಎಷ್ಟೊಂದು ಭಾಗ ಆತನಿಗಾಗಿ ಮೀಸಲಾಗಿದೆ! ಶಿಲ್ಪಕಲೆಗೆ ಎಷ್ಟೊಂದು ಸ್ಫೂರ್ತಿ ನೀಡಿದೆ ಆತನ ಜೀವನ! ದಕ್ಷಿಣದಲ್ಲಿ ಮಧ್ವ ರಾಮಾನುಜರು, ಉತ್ತರದಲ್ಲಿ ರಮಾನಂದ ಕಬೀರರು, ಪೂರ್ವದಲ್ಲಿ ಚೈತನ್ಯಪ್ರಭು, ಪಶ್ಚಿಮದಲ್ಲಿ ವಲ್ಲಭಾಚಾರ್ಯರು ಶ್ರೀಕೃಷ್ಣನ ಭಕ್ತರ ಸಾಲಿನಲ್ಲಿ ಅಗ್ರಗಣ್ಯರು. ಶ್ರೀಕೃಷ್ಣನನ್ನು ಚಿಂತಿಸಿ, ಧ್ಯಾನಿಸಿ, ಜಪಿಸಿ, ಪೂಜಿಸಿ ಮರ್ತ್ಯ ಅಮೃತನಾಗಿದ್ದಾನೆ; ಪಾಪಿ ಪುಣ್ಯಾತ್ಮನಾಗಿದ್ದಾನೆ; ಸಂಸಾರಿ ಭವಸಾಗರ ವನ್ನು ದಾಟಿದ್ದಾನೆ; ಮುಂದೆ ಬರುವವರಿಗೆಲ್ಲ ಅವನ ಜೀವನ ಭವಸಾಗರವನ್ನು ದಾಟಿಸುವ ಸೇತುವೆಯಂತಿದೆ. ಇದೆಲ್ಲ ಶ್ರೀಕೃಷ್ಣನ ಉಪಾಸನೆಯಿಂದ ಸಾಧ್ಯವಾಗಿರಬೇಕಾದರೆ ಅವನ ವ್ಯಕ್ತಿತ್ವದಲ್ಲಿ ಏನೋ ಒಂದು ಮಹಿಮೆ ಇದ್ದಿರಬೇಕು.

ನಮ್ಮ ಹಿಂದೂ ಧರ್ಮದ ಕೇಂದ್ರದಲ್ಲಿ ತತ್ವವಿದೆ. ಆ ತತ್ವಚಿಂತನೆಗೆ ಸಹಾಯ ಮಾಡಲು–ದಾರಿತೋರಲು–ವ್ಯಕ್ತಿಗಳಿದ್ದಾರೆ. ಆ ವ್ಯಕ್ತಿಗಳ ಜೀವನವನ್ನು ಚಿತ್ರಿಸುವ ಗ್ರಂಥಗಳಿವೆ, ಅಂತಹ ಗ್ರಂಥಗಳಲ್ಲಿ ಒಂದು ಭಾಗವತ. ಇದರಲ್ಲಿ ನಾವು ಶ್ರೀಕೃಷ್ಣನ ವ್ಯಕ್ತಿತ್ವವನ್ನು ಕಾಣುತ್ತೇವೆ. ಅನೇಕವೇಳೆ ಇಲ್ಲಿ ಬರುವ ವ್ಯಕ್ತಿ ಒಬ್ಬ ಚಾರಿತ್ರಿಕ ವ್ಯಕ್ತಿಯೇ ಎಂಬ ಸಂದೇಹ ಬರುತ್ತದೆ. ಒಬ್ಬ ಶ್ರೀಕೃಷ್ಣನಿದ್ದನೊ, ಹಲವು ಶ್ರೀಕೃಷ್ಣರಿದ್ದರೊ ಎಂಬ ಸಮಸ್ಯೆ ಮೂಡುವುದಕ್ಕೂ ಅವಕಾಶವಿದೆ. ಹರಿವಂಶ, ವಿಷ್ಣುಪುರಾಣ, ಮಹಾಭಾರತ, ಭಾಗವತ–ಇವು ಶ್ರೀಕೃಷ್ಣ ಕಥೆಗೆ ಮುಖ್ಯವಾದ ಮೂಲಗಳು; ಅವುಗಳಲ್ಲಿ ಬರುವ ಕಥೆ ಒಬ್ಬ ಕೃಷ್ಣನದೊ ಹಲವು ಕೃಷ್ಣರದೊ ಎಂಬ ಸಂದೇಹಕ್ಕೆ ಆಸ್ಪದವಿತ್ತಿದೆ. ಇದನ್ನು ಕುರಿತು ಇಲ್ಲಿ ಸ್ಥೂಲವಾಗಿ ವಿಚಾರ ಮಾಡೋಣ.

ಶ್ರೀಕೃಷ್ಣನ ಜೀವನವನ್ನು ನಾವು ಮೂರು ದೃಷ್ಟಿಗಳಿಂದ ಪರಿಶೀಲಿಸಬಹುದು–ಚಾರಿ ತ್ರಿಕದೃಷ್ಟಿ, ಆಧ್ಯಾತ್ಮಿಕದೃಷ್ಟಿ, ಸಾಹಿತ್ಯದೃಷ್ಟಿ. ಚಾರಿತ್ರಿಕದೃಷ್ಟಿಯಿಂದ ನೋಡಿದಾಗ, ಶ್ರೀಕೃಷ್ಣನೆಂಬ ಒಬ್ಬ ವ್ಯಕ್ತಿ ಹಿಂದೆ ಇದ್ದಿರಬೇಕು ಎನ್ನಿಸುತ್ತದೆ. ಇದಕ್ಕೆ ಸಾಕಷ್ಟು ಸಾಧನ ಸಾಮಗ್ರಿಗಳು ದೊರೆಯುತ್ತವೆ. ಉಪನಿಷತ್ತುಗಳಲ್ಲೆಲ್ಲ ಅತ್ಯಂತ ಹಳೆಯದಾದ ಛಾಂದೋಗ್ಯ ಉಪನಿಷತ್ತಿನಲ್ಲಿ ‘ದೇವಕೀಪುತ್ರನಾದ ಕೃಷ್ಣನು ಆಂಗೀರಸನ ಶಿಷ್ಯನಾಗಿ ದ್ದನು’ ಎಂಬರ್ಥದ ಹೇಳಿಕೆಯಿದೆ (\eng{iii}, ೧೭-೬). ಈ ಉಪನಿಷತ್ತು ಬುದ್ಧನಿಗಿಂತಲೂ ಹಿಂದಿನದೆಂದು ಹೇಳುವುದರಿಂದ ಕ್ರಿ.ಪೂ. ಆರನೆಯ ಶತಮಾನಕ್ಕೂ ಹಿಂದೆಯೆ ಶ್ರೀ ಕೃಷ್ಣನ ಹೆಸರು ಪ್ರಚಾರದಲ್ಲಿತ್ತೆಂದು ಅರ್ಥವಾಗುತ್ತದೆ. ಪಾಣಿನಿ–ಈತನೂ ಬುದ್ಧನಿ ಗಿಂತ ಪ್ರಾಚೀನನೆಂದು ಪಂಡಿತರ ಮತ–ತನ್ನ ಭಾಷ್ಯದಲ್ಲಿ ‘ಕೃಷ್ಣಾರ್ಜುನರು ಪೂಜಾ ಯೋಗ್ಯರು’ ಎಂಬರ್ಥ ಬರುವಂತೆ ಹೇಳಿದ್ದಾನೆ. ಕ್ರಿ. ಪೂ. ನಾಲ್ಕನೆಯ ಶತಮಾನಕ್ಕೆ ಸೇರಿದ ‘ನಿದ್ದೀಶ’ ಎಂಬ ಬೌದ್ಧರ ಪಾಲೀಗ್ರಂಥವೊಂದರಲ್ಲಿ ವಾಸುದೇವ ಬಲದೇವರ ಭಕ್ತರ ವಿಚಾರ ಪ್ರಸ್ತಾಪಿತವಾಗಿದೆ. ಭಾರತಕ್ಕೆ ಬಂದ ಗ್ರೀಕ್ ರಾಯಭಾರಿ ಮೆಗಾಸ್ತನೀಸನು ಸುಮಾರು ಕಿ.ಪೂ. ೩೨0 ರಲ್ಲಿ ಇಲ್ಲಿಗೆ ಬಂದವನು ‘ಶೌರಸೇನಿಯರು ಕೃಷ್ಣನನ್ನು ಪೂಜಿಸು ತ್ತಾರೆ’ ಎಂದು ಹೇಳುತ್ತಾನೆ. ಕ್ರಿ.ಪೂ. ೧೮0 ಕ್ಕೆ ಸೇರಿದ ಬಸ್ ನಗರದ ಶಾಸನದಲ್ಲಿ \textbf{ಆಲಿಯಧೋರ} (?) ಎಂಬ ಗ್ರೀಕ್ ಭಾಗವತನು ವಾಸುದೇವನನ್ನು ‘ದೇವರ ದೇವ’ ಎಂದು ಕರೆಯುತ್ತಾನೆ. ಕ್ರಿ. ಪೂ. ಒಂದನೆಯ ಶತಮಾನಕ್ಕೆ ಸೇರಿದ \textbf{ವಾನಾಹಾಟಿನ} (?) ಶಾಸನದಲ್ಲಿ ದೇವರ ಹೆಸರುಗಳನ್ನು ಹೇಳುವ ಮೊದಲ ಶ್ಲೋಕವು ವಾಸುದೇವನ ಹೆಸರನ್ನು ಒಳ ಗೊಂಡಿದೆ. ಪತಂಜಲಿಯು ತನ್ನ ಮಹಾಭಾಷ್ಯದಲ್ಲಿ ಪಾಣಿನಿಯ \eng{iv}, ೩-೯೮ನೇ ಸೂತ್ರ ವನ್ನು ವಿವರಿಸುವಾಗ ಭಗವಂತನಾದ ವಾಸುದೇವನ ಹೆಸರನ್ನು ಹೇಳುತ್ತಾನೆ. ಇದನ್ನೆಲ್ಲ ನೋಡಿದರೆ ಶ್ರೀಕೃಷ್ಣನೆಂಬ ವ್ಯಕ್ತಿಯೊಬ್ಬನು ಹಿಂದೆ ಇದ್ದಿರಬೇಕೆಂಬುದು ಸುಸ್ಪಷ್ಟವಾಗು ತ್ತದೆ. ಕಾಲ ಕಳೆದಂತೆ ಆ ವ್ಯಕ್ತಿಯ ಸುತ್ತ ಬೇಕಾದಷ್ಟು ಕಲ್ಪನೆ ಉತ್ಪ್ರೇಕ್ಷೆಗಳು ಬೆಳೆದು ಕೊಂಡು ಹೋಗಿ, ನಾವೀಗ ನಂಬುವುದಕ್ಕೆ ಅಸಾಧ್ಯವಾದಂತಹ ಒಬ್ಬ ಅತಿಮಾನವ ವ್ಯಕ್ತಿ ಯನ್ನಾಗಿ ಮಾಡಿವೆಯೆಂದು ತೋರುತ್ತದೆ. ಹೀಗೆ ಆರೋಪಮಾಡಿರುವುದನ್ನೆಲ್ಲ ತೆಗೆ ದರೂ, ಇನ್ನು ತೆಗೆಯಲು ಸಾಧ್ಯವಿಲ್ಲವೆನ್ನುವಂತಹ ಒಂದು ವ್ಯಕ್ತಿತ್ವ ನಿಲ್ಲುತ್ತದೆ. ಅರ್ಜುನನಿಗೆ ಗೀತೆಯನ್ನು ಬೋಧಿಸಿದ ವ್ಯಕ್ತಿತ್ವ ಅಂತಹುದು. ಇಲ್ಲಿ ‘ಕರ್ತವ್ಯಕ್ಕಾಗಿ ಕರ್ತವ್ಯ, (ಗೀತೆ \eng{ii}-೪೭) ಎಂಬ ಚಿರನೂತನವಾದ ಒಂದು ಭಾವನೆ ಕಂಡುಬರುತ್ತದೆ. ಇದು ಬಂದುದು ಯಾವ ವ್ಯಕ್ತಿಯಿಂದಲೋ ಆ ವ್ಯಕ್ತಿಯೇ ಶ್ರೀಕೃಷ್ಣ.

ಆಧ್ಯಾತ್ಮಿಕ ದೃಷ್ಟಿಯಿಂದ ನೋಡಿದಾಗ ಶ್ರೀಕೃಷ್ಣನು ಮಹಾಮಹಿಮನಾಗಿ ಗೋಚರಿಸು ತ್ತಾನೆ. ಆತನನ್ನು ಆದರ್ಶಮೂರ್ತಿಯಾಗಿ ಸ್ವೀಕರಿಸಿ ಲೆಕ್ಕವಿಲ್ಲದಷ್ಟು ಜನ ಸಾಧುಸಂತರು, ಭಕ್ತರು ಉದ್ಧಾರವಾಗಿದ್ದಾರೆ. ಅವರ ಜೀವನ ಮತ್ತು ಸಂದೇಶಗಳು ಇತರರಿಗೆ ಉದ್ಧಾರದ ಹಾದಿಯನ್ನು ತೋರಿವೆ. ಒಂದು ವೈಜ್ಞಾನಿಕ ಗ್ರಂಥವನ್ನು ಯಾರು ಬರೆದರು, ಯಾವಾಗ ಬರೆದರು–ಎಂಬ ಇತಿಹಾಸಕ್ಕಿಂತಲೂ ಅಲ್ಲಿ ಬರೆದಿರುವುದು ಸತ್ಯವೇ, ಅಲ್ಲವೆ ಎಂಬುದು ಮುಖ್ಯ. ಪ್ರಯೋಗಶಾಲೆಯಲ್ಲಿ ಪ್ರಯೋಗ ಮಾಡಿ ನೋಡಿ ನಾವದನ್ನು ಅರಿತುಕೊಳ್ಳಬೇಕು. ಅದು ಸತ್ಯವಾಗಿದ್ದಲ್ಲಿ ಅದು ಸ್ವೀಕಾರ ಯೋಗ್ಯ. ಅದರಂತೆಯೆ, ಆಧ್ಯಾತ್ಮಿಕ ಜಗತ್ತಿನಲ್ಲಿ ಸಾಧುಸಂತರು ಒಂದು ದೊಡ್ಡ ಪ್ರಯೋಗಶಾಲೆ. ಸತ್ಯವೆನಿಸಿ ಕೊಂಡ ಶ್ರೀಕೃಷ್ಣನ ವ್ಯಕ್ತಿತ್ವ ಸರ್ವಗ್ರಾಹ್ಯವೆಂಬುದರಲ್ಲಿ ಯಾವ ಸಂಶಯವೂ ಇಲ್ಲ.

ಇನ್ನು ಸಾಹಿತ್ಯದೃಷ್ಟಿಯಿಂದ ಶ್ರೀಕೃಷ್ಣನ ಜೀವನದ ಕಡೆ ನೋಡೋಣ. ಆತನ ಬಾಳಕಥೆಯನ್ನು ಹೇಳುವ ಪ್ರಾಚೀನ ಗ್ರಂಥಗಳಲ್ಲಿ ಮಹಾಭಾರತ, ಭಾಗವತ–ಇವೆರಡು ಬಹು ಮುಖ್ಯವಾದವು. ಭಾಗವತದಲ್ಲಿ ಶ್ರೀಕೃಷ್ಣನ ಬಾಲ್ಯ ದಿವ್ಯಸುಂದರವಾಗಿದೆ; ಭಾರತದಲ್ಲಿ ಆತನ ಜೀವನದ ಉತ್ತರಾರ್ಧ ಮನೋಹರವಾಗಿದೆ. ಸಾಹಿತ್ಯದಲ್ಲಿ ಬಹು ಭಾಗ ಕವಿಯ ಕಲ್ಪನೆ ಇರಬಹುದಾದರೂ ಆತನು ಸ್ವೀಕರಿಸಿರುವ ಶ್ರೀಕೃಷ್ಣನೆಂಬ ವ್ಯಕ್ತಿ ಜನರಲ್ಲಿ ಆಗಲೆ ಪ್ರಚಾರದಲ್ಲಿದ್ದ ಒಬ್ಬ ವ್ಯಕ್ತಿಯಾಗಿರಬೇಕು. ಹಾಗೆ ಪ್ರಚಾರದಲ್ಲಿರುವ ವ್ಯಕ್ತಿಗೆ ಆದರ್ಶದ ಒಂದು ಮೆರಗನ್ನು ಅಳವಡಿಸುವುದೇ ಕವಿಕರ್ಮ. ನಮಗೆ ಶ್ರೀಕೃಷ್ಣನ ಜೀವನ ವೃತ್ತಾಂತವನ್ನು ಅರಿತುಕೊಳ್ಳುವುದಕ್ಕೆ ಇದೇ ಏಕೈಕ ಸಾಧನ.

ಭಾಗವತದಲ್ಲಿ ಬರುವ ಶ್ರೀಕೃಷ್ಣನ ಜೀವನವನ್ನು ಮೂರು ಘಟ್ಟಗಳಾಗಿ ವಿಂಗಡಿಸ ಬಹುದು–ಗೋಪಗೋಪಿಯರ ಮಧ್ಯದಲ್ಲಿ ಕಾಣಬರುವ ಬಾಲಕ, ಮಧುರೆಯಲ್ಲಿಯೂ ದ್ವಾರಕೆಯಲ್ಲಿಯೂ ಕಾಣಬರುವ ರಾಜಕಾರಣಿ, ಕುರುಕ್ಷೇತ್ರದ ಗೀತಾಚಾರ್ಯನಾಗಿದ್ದು ಕಡೆಗೆ ಬೇಟೆಗಾರನ ಬಾಣಕ್ಕೆ ತುತ್ತಾಗಿ ಈ ಪ್ರಪಂಚವನ್ನು ತೊರೆದು ಹೋಗುವುದು. ಆತನು ಹುಟ್ಟಿದುದು ದ್ವಾಪರದ ಅಂತ್ಯದಲ್ಲಿ. ದೇವಕಿ ವಸುದೇವರ ಮಗನಾಗಿ ಹುಟ್ಟಿದ ಆತನು ತಾಯ್ತಂದೆಗಳಿಂದ ಅಗಲಿ ಗೋಕುಲದಲ್ಲಿ ಬೆಳೆಯುವನು. ಅವನನ್ನು ಕೊಲ್ಲುವು ದಕ್ಕಾಗಿ ಕಂಸನು ಪೂತನಿ, ಶಕಟಾಸುರ, ತೃಣಾವರ್ತ ಮೊದಲಾದ ರಕ್ಕಸರನ್ನು ಕಳುಹಿಸು ವನು. ಅವರನ್ನೆಲ್ಲ ಎಳೆಯ ಮಗುವಾದ ಶ್ರೀಕೃಷ್ಣನು ಕೊಂದುಹಾಕುವನು. ಆತನು ಇನ್ನೂ ಹಲವಾರು ಸಾಹಸಕೃತ್ಯಗಳನ್ನು ಮಾಡುವನು. ಇಷ್ಟೇ ಅಲ್ಲ, ಶ್ರೀಕೃಷ್ಣನು ತುಂಟತನವೇ ರೂಪುವೆತ್ತಂತೆ ಇರುವನು. ಆದರೆ ಈ ತುಂಟತನ ಸುತ್ತುಮುತ್ತಿನ ಜನರ ದ್ವೇಷಕ್ಕೆ ಕಾರಣವಾಗುವುದಿಲ್ಲ. ಅಚ್ಚುಮೆಚ್ಚಾಗುತ್ತದೆ. ಭಗವಂತ ಮಾನವನಾಗಿ ಅವತರಿ ಸಿದಾಗ ಆತ ಹೊತ್ತು ಹೆತ್ತವರಿಗೆ ಮಾತ್ರವೇ ಅಲ್ಲ, ಸುತ್ತುಮುತ್ತಿನ ಜನಕ್ಕೆಲ್ಲ ಆನಂದ ದಾಯಕನಾಗುತ್ತಾನೆ. ಅವನು ಮನೆ ಮುದ್ದಿನ ಕೂಸಲ್ಲ, ಜನ ಮುದ್ದಿನ ಕೂಸು. ಅವನಿಗೆ ಎಲ್ಲರೂ ತನ್ನವರೆ. ಮನೆಯ ಮುಂದಿನ ಸಂಪಿಗೆ ಗಿಡದಲ್ಲಿ ಹೂ ಅರಳಿದರೆ, ತಾನಿದ್ದ ಮನೆಗೆ ಮಾತ್ರ ಕಂಪು ಬೀರುವುದೇನು? ಅದಕ್ಕೆ ತನ್ನ ಮನೆ, ನೆರೆ ಮನೆ ಎಂಬ ಭೇದವಿಲ್ಲ. ಹಾಗೆಯೆ ಶ್ರೀಕೃಷ್ಣ. ಅವನು ಗೋಕುಲದಲ್ಲಿರುವವರೆಲ್ಲರ ಕಣ್ಮಣಿ. ಅವನಿಗೆ ಗೋಪಾಲ ರಲ್ಲಿ ಪ್ರೀತಿ, ಗೋಪಿಯರಲ್ಲಿ ಪ್ರೀತಿ, ಪಶುಪಕ್ಷಿಮೃಗಗಳಲ್ಲಿ ಪ್ರೀತಿ. ಅವನಂತೆ ಅವು ಗಳೆಲ್ಲರಿಗೂ ಅವನಲ್ಲಿ ಪ್ರೀತಿ. ಇಂತಹ ಒಂದು ಪ್ರೀತಿಯ ವಾತಾವರಣದಲ್ಲಿ ಆತ ಬೆಳೆದದ್ದು. ಹೀಗೆ ಬೆಳೆಯುತ್ತಾ ಆತ ಒಮ್ಮೆ ಅಖಂಡ ಬ್ರಹ್ಮಾಂಡವನ್ನು ತನ್ನ ಬಾಯಲ್ಲಿ ತೋರಿದ, ತಾಯಿ ಯಶೋಧೆಗೆ. ಆದರೆ ಆಕೆಗೆ ಬೇಕಾಗಿದ್ದುದು ಬ್ರಹ್ಮಾಂಡವನ್ನು ಧರಿಸಿದ ಭಗವಂತನಲ್ಲ, ತನ್ನ ತೋಳಲ್ಲಿ ಪ್ರೀತಿಯಿಂದ ತಕ್ಕೈಸಬಹುದಾದ ಕಂದ ಮುಕುಂದ. ಆದ್ದರಿಂದ ಆ ಬ್ರಹ್ಮಾಂಡದೊಡೆಯ ಆಕೆಯ ಮಗುವಾಗಿ ಬೆಳೆದ. ಆತ ಹಾಲು, ಮೊಸರು, ಬೆಣ್ಣೆಗಾಗಿ ಆಕೆಯ ಮುಂದೆ ಕೈಯೊಡ್ಡಿದ; ಕೊಡದಾಗ ಅದನ್ನು ಕದ್ದು ಮೆದ್ದು–ಅವನ್ನು ಮಾತ್ರವೇ ಅಲ್ಲ, ಗೋಪಾಲಕರೆಲ್ಲರ, ಗೋಪಿಯರೆಲ್ಲರ ಮನಸ್ಸನ್ನೂ ಅಪಹರಿಸಿದ. ಇನ್ನು ಮೇಲೆ ಅವರ ಮನಸ್ಸು ಅವರದಾಗಿ ಉಳಿಯಲಿಲ್ಲ. ಅದು ಆ ಚಿತ್ತಚೋರನದಾಯಿತು. ಅವನಿಗಾಗಿ ಇವರ ಜೀವನವಾಯಿತು.

ಶ್ರೀಕೃಷ್ಣ ಬೃಂದಾವನದಲ್ಲಿ ಕಳೆದ ಬಾಲ್ಯದಲ್ಲಿನ ಎರಡು ಘಟನೆಗಳು ವಿಮರ್ಶಕರ ಕಟುಟೀಕೆಗೆ ಒಳಗಾಗಿವೆ. ಅವು ವಸ್ತ್ರಾಪಹರಣ ಪ್ರಸಂಗ ಮತ್ತು ರಾಸಲೀಲೆ. ಈ ಪ್ರಸಂಗ ಗಳಲ್ಲಿ ಶ್ರೀಕೃಷ್ಣನು ಒಬ್ಬ ಸ್ತ್ರೀ ಲಂಪಟನಾದ ಕಾಮುಕನಂತೆ ಅಶ್ಲೀಲವಾಗಿ ವ್ಯವಹರಿಸುವ ನೆಂದು ಅವರ ಆರೋಪ. ಲೋಕಕ್ಕೆ ಮಾರ್ಗದರ್ಶಕನಾಗಬೇಕಾದ ಮಹಾವ್ಯಕ್ತಿ ಅಕ್ಷಮ್ಯ ಅಪರಾಧವನ್ನೆಸಗಿದವನಂತೆ ಅವರಿಗೆ ಗೋಚರಿಸಿದ್ದಾನೆ. ಅವರು ಶ್ರೀಕೃಷ್ಣನನ್ನು ಒಬ್ಬ ಯುವಕನಾಗಿ ಕಲ್ಪಿಸಿಕೊಂಡು, ತಮ್ಮ ಮನಸ್ಸಿನ ಗಲೀಜನ್ನೆಲ್ಲ ಅವನಲ್ಲಿ ಆರೋಪಿಸು ತ್ತಾರೆ. ಅವರು ತಾಳ್ಮೆಯಿಂದ ಭಾಗವತವನ್ನೋದಿದರೆ ತಮ್ಮ ತಪ್ಪನ್ನು ಅರಿತುಕೊಳ್ಳುವು ದಕ್ಕೆ ಸಾಧ್ಯ. ಶ್ರೀಕೃಷ್ಣ ಬೃಂದಾವನದಲ್ಲಿದ್ದುದು ತನ್ನ ಹನ್ನೊಂದನೆಯ ವಯಸ್ಸಿನವರೆಗೆ ಮಾತ್ರ. (ಭಾಗವತ ಸ್ಕಂಧ \eng{iii}, ೨-೨೬) ವಸ್ತ್ರಾಪಹರಣ ಪ್ರಸಂಗ ನಡೆದಾಗ ಶ್ರೀಕೃಷ್ಣನಿಗೆ ಕೇವಲ ಐದಾರು ವರ್ಷ ವಯಸ್ಸು. ಯಮುನಾ ನದಿಯಲ್ಲಿ ಸ್ನಾನಮಾಡುತ್ತಿದ್ದ ಸ್ತ್ರೀಯ ರೆಲ್ಲರೂ ಅವನಿಗಿಂತ ದೊಡ್ಡವರು. ಅವರು ಶ್ರೀಕೃಷ್ಣನನ್ನು ತಮ್ಮ ಆರಾಧ್ಯದೈವವಾಗಿ ಸ್ವೀಕರಿಸಿದ್ದರು. ಪರಮಾತ್ಮನನ್ನು ಕೂಡಬಯಸುವವರು ತಮ್ಮ ಸರ್ವಸ್ವವನ್ನೂ ಅವನಿಗೆ ಅರ್ಪಣೆ ಮಾಡಬೇಕು. ತಾನು ಗಂಡಸು ಅಥವಾ ಹೆಂಗಸು ಎಂಬ ಭಾವ ಕೂಡ ಅಲ್ಲಿ ಬರಬಾರದು. ಗೋಪಬಾಲೆಯರು ದಿಗಂಬರರಾಗಿ ಅವನೆದುರು ನಿಲ್ಲಲು ನಾಚಿದುದನ್ನು ಕಂಡು ಆತ ಹೇಳಿದ– ನನ್ನನ್ನು ಹೊಂದಿದವರ ಮನಸ್ಸು ಹುರಿದ ಬೀಜದಂತಾಗುತ್ತದೆ, ಎಂದು. ಭಗವಂತನ ಸಾಕ್ಷಾತ್ಕಾರವಾದರೆ ಮನಸ್ಸಿನ ಸಂಸ್ಕಾರಗಳೆಲ್ಲ, ವಾಸನೆಗಳೆಲ್ಲ ನಾಶವಾಗಿ ಹೋಗುತ್ತವೆ. ದೇವರು ನಮಗೆ ಬೇಕಾದರೆ ನಾವು ದೇಹಾತೀತರಾಗಬೇಕು. ಆತ್ಮದೃಷ್ಟಿ ಬರಬೇಕಾದರೆ ದೇಹದೃಷ್ಟಿ ಬಲಿಯಾಗಬೇಕು. ಇದನ್ನೇ ನಾವಿಲ್ಲಿ ನೋಡು ವುದು. ಹೆಂಗಸಿಗೆ ಎಲ್ಲಕ್ಕಿಂತ ದೊಡ್ಡದು ಮಾನ. ಆಕೆಗೆ ಮಾನದ ಮುಂದೆ ಪ್ರಾಣವೂ ತೃಣಸಮಾನ. ಆದರೆ ದೇವರು ಬೇಕೆಂದು ಬಯಸುವುದಾದರೆ ಆ ಮಾನವನ್ನೂ ಕೊಡಬೇಕು. ಬೃಂದಾವನದ ಹೆಣ್ಣುಗಳು ತಮ್ಮ ಮಾನವನ್ನೆ ಶ್ರೀಕೃಷ್ಣನ ಪದತಲದಲ್ಲಿ ಸಮರ್ಪಿಸಿದ ಒಂದು ಅಸದೃಶ ಘಟನೆ, ಇದು. ಬೈಬಲ್ಲಿನಲ್ಲಿಯೂ ಇಂತಹ ಒಂದು ಪ್ರಸಂಗ ಬರುತ್ತದೆ. ಶಿಷ್ಯನೊಬ್ಬನು ಏಸುಕ್ರಿಸ್ತನನ್ನು ಕುರಿತು ‘ನಾನು ದೇವರನ್ನು ಕಾಣುವುದೆಂತು?’ ಎಂದು ಕೇಳಿದಾಗ, ಏಸು–‘ನೀನು ನಿನ್ನ ಬಟ್ಟೆ ಬರೆಗಳನ್ನೆಲ್ಲ ಕಳಚಿ, ಲಜ್ಜಾರಹಿತನಾಗಿ ದೇವರ ಇದಿರಿಗೆ ನಿಂತಾಗ’ ಎಂದು ಹೇಳಿದನಂತೆ! ಇಲ್ಲಿ ಬಟ್ಟೆಬರೆ ಗಳೆಂದರೆ ಉಪಾಧಿಗಳು ಎಂದು ಅರ್ಥ. ಹಾಗೆಯೆ ‘ಗೋಪಿಯರ ವಸ್ತ್ರ’ವೆಂದಾಗ ಅವರ ಸೀರೆಗಳೇ ಅಲ್ಲ, ಅವರ ಉಪಾಧಿಗಳು ಕೂಡ. ಅವರನ್ನು ದೇಹವೆಂಬ ಒಂದು ಗೂಟಕ್ಕೆ ಕಟ್ಟಿ ಹಾಕಿವೆ, ಈ ಉಪಾಧಿಗಳು. ಅವುಗಳಿಂದ ಪಾರಾದರೆ ಮಾತ್ರ ಈಶ್ವರ ಸಾಕ್ಷಾತ್ಕಾರ.

ಇನ್ನು ರಾಸಲೀಲೆಯ ಪ್ರಸಂಗವನ್ನು ತೆಗೆದುಕೊಳ್ಳೋಣ. ಇದೊಂದು ಭಾಗವತದಲ್ಲಿ ಬರುವ ದಿವ್ಯ ಮನೋಹರವಾದ, ಕಾವ್ಯಮಯವಾದ ಸುಂದರದೃಶ್ಯ. ಗೋಪಿಯರಿಗೆ ಶ್ರೀಕೃಷ್ಣನ ಮೇಲಿರುವ ಅಪಾರ ಪ್ರೇಮದ ದಿಗ್ದರ್ಶನವಾಗಿದೆ, ಇಲ್ಲಿ. ಒಂದು ಸಂಜೆ ಶ್ರೀಕೃಷ್ಣನು ಯಮುನಾ ನದಿಯ ತೀರದಲ್ಲಿ ಕುಳಿತು ಕೊಳಲನ್ನು ಬಾರಿಸುತ್ತಾನೆ. ಅದರ ನಾದದ ಸ್ಪಂದನ ಬೃಂದಾವನವನ್ನು ಪ್ರವೇಶಿಸುತ್ತದೆ, ಮನೆಮನೆಯ ಒಳಕ್ಕೂ ತೂರಿ ಹೋಗುತ್ತದೆ. ಆ ಕೊಳಲಿನ ಕರೆಗೆ ಗೋಪಿಯರೆಲ್ಲ ಪರವಶರಾಗುತ್ತಾರೆ. ತಾವು ಮಾಡು ತ್ತಿದ್ದ ಕೆಲಸಗಳನ್ನು ಹಾಗೆ ಹಾಗೆಯೆ ಬಿಟ್ಟು ಅವರು ಶ್ರೀಕೃಷ್ಣಾಭಿಮುಖರಾಗುತ್ತಾರೆ. ಮಾಡುವ ಕೆಲಸವನ್ನು ಮುಗಿಸಿ ಹೋಗುವೆನೆಂದರೆ ನನಗಿನ್ನು ಭಗವಂತನ ಹಸಿವು ಸಾಕಷ್ಟಾ ಗಿಲ್ಲ ಎಂದು ಅರ್ಥ. ನಾವು ಮಾಡುವ ಕೆಲಸಗಳಿಗೆ ಕೊನೆಮೊದಲೆಂಬುದು ಇದೆಯೇನು? ಒಂದು ಕೆಲಸ ಮುಗಿದರೆ ಮತ್ತೊಂದು ಕೆಲಸ ಕಾದಿರುತ್ತದೆ. ಸಮುದ್ರಸ್ನಾನಕ್ಕೆ ಹೋದ ವನು ಅಲೆ ನಿಂತಮೇಲೆ ಸ್ನಾನಮಾಡುತ್ತೇನೆ ಎಂದಹಾಗೆ ಇದು. ಶ್ರೀರಾಮಕೃಷ್ಣ ಪರಮ ಹಂಸರು ಆಳವಾದ ಭಕ್ತಿಯುಳ್ಳವನಿಗೆ ಮೈಮೇಲೆ ಪ್ರಜ್ಞೆ ಇರುವುದಿಲ್ಲವೆಂದು ಹೇಳುತ್ತಾ ‘ಸ್ವಲ್ಪ ಕುಡಿದರೆ ಪ್ರಪಂಚಪ್ರಜ್ಞೆ ಹೋಗದು, ಕಂಠಪೂರ್ತಿ ಕುಡಿದರೆ ಜಗತ್ತನ್ನೆ ಮರೆಯು ತ್ತಾನೆ’ ಎನ್ನುತ್ತಾರೆ. ಗೋಪಿಯರು ಶ್ರೀಕೃಷ್ಣ ಪ್ರೇಮದ ಸುರೆಯನ್ನು ಮಿತಿಮೀರಿ ಪಾನಮಾಡಿದ್ದಾರೆ. ಆದ್ದರಿಂದಲೆ ಅವರು ಬಾಹ್ಯ ಜಗತ್ತನ್ನು ಮರೆತರು. ಅಡಿಗೆ ಮಾಡು ತ್ತಿದ್ದವಳು ಒಲೆಯ ಮೇಲಿನ ಪಾತ್ರೆಯನ್ನು ಹಾಗೆಯೆ ಬಿಟ್ಟು ಹೊರಟಳು, ಹಾಲು ಕರೆಯು ತ್ತಿದ್ದವಳು ಪಾತ್ರೆಯನ್ನು ಅಲ್ಲಿಯೆ ಕುಕ್ಕಿ ಹಾಗೆಯೆ ಹೊರಟಳು, ಮಗುವಿಗೆ ಹಾಲು ಕುಡಿ ಸುತ್ತಿದ್ದವಳು ಆ ಮಗುವನ್ನು ನೆಲದ ಮೇಲೆ ಮಲಗಿಸಿ ಹಾಗಿಂದ ಹಾಗೆಯೆ ಹೊರಟಳು, ಗಂಡನಿಗೆ ಬಡಿಸುತ್ತಿದ್ದವಳು ಅನ್ನದ ಪಾತ್ರೆಯನ್ನು ಕೆಳಗಿಟ್ಟು ಹಾಗೆಯೆ ಹೊರಟಳು, ಗಂಡನ ಕಾಲನ್ನು ಒತ್ತುತ್ತಿದ್ದವಳು ಅದನ್ನು ಬಿಟ್ಟು ಹೊರಟಳು. ಹಾಗೆ ಹೊರಟವರೆಲ್ಲ ಯಮುನಾ ನದಿಯ ತೀರದಲ್ಲಿ ಶ್ರೀಕೃಷ್ಣನೊಡನೆ ಆನಂದದಿಂದ ಕುಣಿದಾಡಿದರು. ಹಾಗೆ ಕುಣಿಯುವಾಗ ಅವರು ತಮ್ಮ ಒಳಗೆ ಹೊರಗೆ, ಚರಾಚರ ಜಗತ್ತಿನಲ್ಲಿ ಎಲ್ಲೆಲ್ಲಿಯೂ ಶ್ರೀಕೃಷ್ಣನನ್ನು ಕಾಣುತ್ತಿದ್ದರು. ಕೃಷ್ಣನಲ್ಲದ ವಸ್ತುವೇ ಅವರಿಗೆ ಕಾಣದು. ಸರ್ವಂ ಕೃಷ್ಣಮಯಂ ಜಗತ್. ಇದೊಂದು ಭೂಮಾನುಭವ, ಅದ್ವೈತದ ಪರಾಕಾಷ್ಠ ದೆಸೆ, ಭಗವಂತನಲ್ಲಿ ಕರಗಿ ಹೋದ ಸ್ಥಿತಿ.

ಸ್ವಾಮಿ ವಿವೇಕಾನಂದರು ಗೋಪಿಪ್ರೇಮವನ್ನು ಕುರಿತು ಹೀಗೆ ಹೇಳುತ್ತಾರೆ– ‘ಮೊದಲು ಕಾಂಚನ, ಕೀರ್ತಿ, ಯಶಸ್ಸು–ಇವುಗಳ ಮೇಲಿನ, ಮತ್ತು ಈ ಕ್ಷುದ್ರ ಮಿಥ್ಯಾ ಸಂಸಾರದ ಮೇಲಿನ ಮೋಹವನ್ನು ತ್ಯಜಿಸಿ; ಆಗ ಮಾತ್ರ ನೀವು ಗೋಪೀಪ್ರೇಮವನ್ನು ಗ್ರಹಿಸಬಲ್ಲಿರಿ. ಎಲ್ಲಿಯವರೆಗೆ ಮನಸ್ಸು ಅಶುದ್ಧವಾಗಿದೆಯೊ, ಅಲ್ಲಿಯವರೆಗೆ ಅದನ್ನು ಗ್ರಹಿಸಲು ಯತ್ನಿಸಬೇಡಿ, ಎಲ್ಲಿಯವರೆಗೆ ಮನಸ್ಸು ಪೂರ್ಣವಾಗಿ ಪರಿಶುದ್ಧವಾಗಿಲ್ಲವೊ, ಅಲ್ಲಿಯವರೆಗೆ ಅದನ್ನು ತಿಳಿಯಲು ಪ್ರಯತ್ನಿಸುವುದೆಲ್ಲ ವಿಫಲ. ಯಾರ ಹೃದಯದಲ್ಲಿ ಪ್ರತಿಕ್ಷಣವೂ ಕಾಮ, ಕೀರ್ತಿ, ಧನದಾಸೆ–ಇವುಗಳ ಬುದ್ಬುದಗಳು ಏಳುತ್ತಿವೆಯೊ, ಅವರು ಗೋಪೀಪ್ರೇಮವನ್ನು ತಿಳಿಯುವ ಮತ್ತು ವಿಮರ್ಶಿಸುವ ಸಾಹಸಕ್ಕೆ ಕೈಹಾಕುವುದೆ? ಕೃಷ್ಣಾವತಾರದ ಸಾರವೇ ಗೋಪೀಪ್ರೇಮ. ದರ್ಶನಶಾಸ್ತ್ರ ಶಿರೋಮಣಿಯಂತಿರುವ ಗೀತೆ ಕೂಡ ಈ ಪ್ರೇಮೋನ್ಮಾದಕ್ಕೆ ಸರಿದೂಗಲಾರದು. ಏಕೆಂದರೆ–ಗೀತೆಯು ಸಾಧಕನಿಗೆ ಮುಕ್ತಿಯನ್ನು ಹೇಗೆ ಪಡೆಯಬೇಕೆಂದು ಕ್ರಮಶಃ ಹೇಳುತ್ತದೆ. ಗೋಪೀಪ್ರೇಮದಲ್ಲಾ ದರೋ, ಈಶ್ವರ ರಸಾಸ್ವಾದದ ಉನ್ಮತ್ತತೆಯಿದೆ, ಅದ್ಭುತ ಪ್ರೇಮೋನ್ಮಾದವಿದೆ. ಇಲ್ಲಿ ಗುರು- ಶಿಷ್ಯ, ಶಾಸ್ತ್ರ-ಉಪದೇಶ, ಸ್ವರ್ಗ-ನರಕ ಈ ಭಾವನೆಯ ಯಾವ ಚಿಹ್ನೆಯೂ ಇಲ್ಲ. ಎಲ್ಲ ಮಾಯವಾಗಿದೆ, ಇಲ್ಲಿ ಪ್ರೇಮೋನ್ಮಾದವೊಂದೇ ಉಳಿದಿರುವುದು. ಆ ಸಮಯ ದಲ್ಲಿ ಕೃಷ್ಣನೊಬ್ಬನ ವಿನಃ ಮತ್ತಾವ ವಸ್ತುವಿನ ಸ್ಮರಣೆಯೂ ಇಲ್ಲ. ಆ ಸಮಯದಲ್ಲಿ ಸಮಸ್ತ ಪ್ರಾಣಿಗಳಲ್ಲಿಯೂ ಕೃಷ್ಣನೊಬ್ಬನೆ ಕಾಣಬರುವನು.

“ಭಾಗವತದಲ್ಲಿ ಬರುವ ಗೋಪೀಜನವಲ್ಲಭನಾದ, ಬೃಂದಾವನವಿಹಾರಿಯಾದ ಶ್ರೀ ಕೃಷ್ಣನಿಗಿಂತಲೂ ಉಚ್ಚತರ ಆದರ್ಶ ಮತ್ತೊಂದಿಲ್ಲ. ಭಾಗ್ಯವತಿಯರಾದ ಗೋಪಿಯರ ಪ್ರೇಮೋನ್ಮಾದ ನಿನಗೆ ಪ್ರಾಪ್ತವಾದರೆ, ಅವರ ಮನೋಭಾವವನ್ನು ನೀನು ಗ್ರಹಿಸಿದರೆ, ಆಗ ‘ಪ್ರೇಮ’ವೆಂದರೆ ಏನೆಂಬುದು ನಿನಗೆ ಅರ್ಥವಾಗುತ್ತದೆ. ಸಮಸ್ತ ಸಂಸಾರವೂ ನಿನ್ನ ದೃಷ್ಟಿಗೆ ಅಂತರ್ಧಾನವಾದಾಗ, ನಿನ್ನ ಹೃದಯ ಪರಿಶುದ್ಧವಾದಾಗ, ಬೇರಾವ ಸತ್ಯಾನುಸಂಧಾನದ ಲಕ್ಷ್ಯವೂ ಇಲ್ಲದಾಗ ಮಾತ್ರ ಗೋಪಿಯರ ಪ್ರೇಮೋನ್ಮಾದ ನಿನಗೆ ಅರ್ಥವಾಗುವುದು; ಗೋಪಿಯರ ಪ್ರೇಮ ಮತ್ತು ಭಕ್ತಿಯು ಅರ್ಥವಾಗುವುದು. ಅದೇ ಹೆಗ್ಗುರಿ. ಈ ಪ್ರೇಮ ದೊರೆತರೆ ನಿನಗೆ ಸಕಲವೂ ಪ್ರಾಪ್ತವಾದಂತೆ.”

ಇಲ್ಲಿಂದ ಮುಂದೆ ನಾವು ಶ್ರೀಕೃಷ್ಣನ ಜೀವನದ ಎರಡನೆಯ ಘಟ್ಟವನ್ನು ಕಾಣುತ್ತೇವೆ. ಆತ ಲೋಕಕಂಟಕನಾದ ಕಂಸನನ್ನು ಕೊಂದು ಅವನ ತಂದೆಯಾದ ಉಗ್ರಸೇನನನ್ನು ಸಿಂಹಾಸನದ ಮೇಲೆ ಕೂಡಿಸುತ್ತಾನೆ. ಇದಾದ ಮೇಲೆ ಆತನು ಹಲವಾರು ಜನ ರಾಜರನ್ನು ಸಂಹರಿಸುವನಾದರೂ ಯಾರ ರಾಜ್ಯವನ್ನೂ ಅಪಹರಿಸುವುದಿಲ್ಲ; ಆಯಾ ರಾಜರ ಉತ್ತರಾ ಧಿಕಾರಿಗಳನ್ನು ಸಿಂಹಾಸನದಲ್ಲಿ ಕೂಡಿಸುತ್ತಾನೆ. ‘ಕರ್ಮಮಾಡುವುದಕ್ಕೆ ಮಾತ್ರ ನಿನಗೆ ಅಧಿಕಾರ, ಅದರಿಂದ ಬರುವ ಫಲಕ್ಕಲ್ಲ’ ಎಂದು ತಾನು ಗೀತೆಯಲ್ಲಿ ಬೋಧಿಸಿದುದನ್ನೇ ಆತನು ತನ್ನ ಜೀವನದಲ್ಲಿ ಅಳವಡಿಸಿಕೊಂಡು, ಅದಕ್ಕೆ ಒಂದು ಉದಾಹರಣೆಯಾಗಿ ದ್ದಾನೆ. ಕಂಸನನ್ನು ಕೊಂದ ಸೇಡನ್ನು ತೀರಿಸಿಕೊಳ್ಳುವುದಕ್ಕಾಗಿ ಅವನ ಅಳಿಯನಾದ ಜರಾಸಂಧ ಶ್ರೀಕೃಷ್ಣನ ಮೇಲೆ ಮತ್ತೆ ಮತ್ತೆ ದಂಡೆತ್ತಿ ಬರುತ್ತಾನೆ. ಸಲಸಲವೂ ತಾನೆ ಸೋಲುತ್ತಿದ್ದರೂ ಅವನು ಹಟಮಾರಿಯಾಗಿ ಹದಿನೆಂಟನೆಯ ಸಲ ದಂಡೆತ್ತಿ ಬರಲು, ಶ್ರೀಕೃಷ್ಣನು ಅವನ ಕಾಟದಿಂದ ತಪ್ಪಿಸಿಕೊಳ್ಳುವುದಕ್ಕಾಗಿ ಸಮುದ್ರ ಮಧ್ಯದಲ್ಲಿರುವ ದ್ವಾರಕಿಯನ್ನು ತನ್ನ ರಾಜಧಾನಿಯಾಗಿ ಮಾಡಿಕೊಳ್ಳುತ್ತಾನೆ. ಇಲ್ಲಿಗೆ ವೈರಿಗಳ ಕಾಟ ತಪ್ಪಿ ದಂತಾಗುತ್ತದೆ.

ಶ್ರೀಕೃಷ್ಣನು ವಿದರ್ಭದೇಶದ ರಾಜಕುಮಾರಿಯಾದ ರುಕ್ಮಿಣಿಯನ್ನು ಮದುವೆಯಾಗು ತ್ತಾನೆ. ಇದೊಂದು ಸುಂದರವಾದ ಪ್ರೇಮಕಥೆ. ಭರತಖಂಡದ ಎಲ್ಲ ಸಾಹಿತ್ಯಗಳ ಲ್ಲಿಯೂ ಇದು ಪ್ರಸಿದ್ಧವಾಗಿದೆ. ಇದರಂತೆಯೆ ಜಾಂಬವತಿ, ಸತ್ಯಭಾಮೆ ಮೊದಲಾದ ಅಷ್ಟಮಹಿಷಿಯರು ಅವನ ಪಟ್ಟದ ರಾಣಿಯರೆನಿಸಿಕೊಳ್ಳುತ್ತಾರೆ. ಶ್ರೀಕೃಷ್ಣನು ನರಕಾಸುರ ನನ್ನು ಕೊಂದು, ಅವನು ಸೆರೆಯಲ್ಲಿಟ್ಟಿದ್ದ ಸಾವಿರಾರು ಸ್ತ್ರೀಯರನ್ನೂ ಬಂಧಮುಕ್ತರನ್ನಾಗಿ ಮಾಡಿ, ಅವರು ತಮ್ಮ ತಮ್ಮ ಮನೆಗಳಿಗೆ ಹಿಂದಿರುಗುವಂತೆ ಹೇಳುತ್ತಾನೆ. ಆದರೆ ಅವರೆಲ್ಲ ಅಸುರನ ಸೆರೆಯಲ್ಲಿದ್ದು ಅವನಿಂದ ಹಾಳಾದವರು. ಅವರಿಗೆ ಸಮಾಜದಲ್ಲಿ ಸ್ಥಾನವಿಲ್ಲ. ತಮ್ಮ ಗತಿಯೇನೆಂದು ಅವರು ಬೇಡಿಕೊಂಡಾಗ ಶ್ರೀಕೃಷ್ಣನು ಅವರ ಪ್ರಾರ್ಥನೆಯನ್ನು ಮನ್ನಿಸಿ, ಅವರನ್ನು ಪತ್ನಿಯರನ್ನಾಗಿ ಸ್ವೀಕರಿಸುತ್ತಾನೆ. ‘ವಸ್ತ್ರಾಪಹರಣ’, ‘ರಾಸಕ್ರೀಡೆ’ಯ ಪ್ರಸಂಗಗಳಂತೆ ಶ್ರೀಕೃಷ್ಣನ ಬಹುಪತ್ನೀತ್ವವೂ ಉಗ್ರಟೀಕೆಗೆ ವಸ್ತುವಾಗಿದೆ. ಆದರೆ ಸ್ವಲ್ಪ ಸಾವಧಾನವಾಗಿ ಆಲೋಚಿಸಿದಾಗ ಈ ಟೀಕೆ ವ್ಯರ್ಥವೆನಿಸು ತ್ತದೆ. ಶ್ರೀಕೃಷ್ಣನ ಕಾಲದತ್ತ ಸ್ವಲ್ಪ ಗಮನಿಸಿ. ಆತ ಹುಟ್ಟಿದುದೆ ಭೂಭಾರ ಹರಣಕ್ಕಾಗಿ. ದಿನ ಬೆಳಗಾದರೆ ಯುದ್ಧ, ಯುದ್ಧ. ಪರಿಸ್ಥಿತಿ ಹೀಗಿರುವಾಗ ಗಂಡಸರಿಗಿಂತ ಹೆಂಗಸರೇ ಹೆಚ್ಚಾಗಿರುವುದು ಅನಿವಾರ್ಯ. ‘ಒಂದು ಗಂಡಿಗೆ ಒಂದು ಹೆಣ್ಣು’ ಎಂಬ ನೀತಿಯನ್ನು ಅನುಸರಿಸುವುದಾದರೆ, ಅನೇಕ ಹೆಂಗಸರು ಗಂಡಂದಿರೇ ಇಲ್ಲದಂತಿರಬೇಕಾಗುತ್ತದೆ. ಈ ಸಮಸ್ಯೆ ನಿವಾರಣೆಯಾಗಬೇಕಾದರೆ ಬಹುಪತ್ನೀತ್ವ ಅನಿವಾರ್ಯ. ರಾಜರು ದೊಡ್ಡ ಶ್ರೀಮಂತರು ಹಲವು ವೇಳೆ ಅನೇಕ ಹೆಂಡಿರನ್ನು ಮದುವೆಯಾಗುವ ವಾಡಿಕೆಗೆ ಇದೇ ಮೂಲ. ಶ್ರೀಕೃಷ್ಣನು ಅಷ್ಟಮಹಿಷಿಯರನ್ನು ಹೊಂದಿದ್ದುದು ಟೀಕೆಗೆ ವಸ್ತುವಾಗಬೇಕಾದ ಅಗತ್ಯವೇನೂ ಇಲ್ಲ. ಆತ ಹದಿನಾರುಸಾವಿರ ಹೆಂಡಿರನ್ನು ಹೊಂದಿದ್ದುದು ಆತನ ಮಹೌದಾರ್ಯಕ್ಕೆ ಒಂದು ಸಾಕ್ಷಿ, ಅಷ್ಟೆ. ನರಕಾಸುರನ ಮನೆಯ ಆ ಹೆಣ್ಣುಗಳಿಗೆಲ್ಲ ‘ಹೆಂಡತಿ’ಯ ಸ್ಥಾನವನ್ನು ಆತ ಕೊಡದಿದ್ದರೆ ಅವರ ಗತಿಯೇನಾಗಬೇಕಾಗಿತ್ತು?

ಇಲ್ಲಿಂದ ಮುಂದೆ ಶ್ರೀಕೃಷ್ಣ ಮತ್ತು ಪಾಂಡವರ ಸಂಬಂಧ ಪ್ರಾರಂಭವಾಗುತ್ತದೆ. ಮಹಾಭಾರತದಲ್ಲಿ ಅವರ ಪರಸ್ಪರ ಪರಿಚಯವಾಗುವುದು ದ್ರೌಪದಿಯ ಸ್ವಯಂವರ ಕಾಲದಲ್ಲಿ. ಅಲ್ಲಿಂದ ಮುಂದೆ ಆತನು ‘ಮಮ ಪ್ರಾಣಾ ಹಿ ಪಾಂಡವಾಃ’ ಎಂಬ ಮಮತೆ ಯಿಂದ ಅವರೊಡನೆ ವ್ಯವಹರಿಸುತ್ತಾನೆ. ರಾಜಮಹಾರಾಜರು ಆತನಲ್ಲಿ ಅಪಾರವಾದ ಗೌರವವನ್ನು ತೋರುತ್ತಿದ್ದರೂ ಆತನು ಪಾಂಡವರಲ್ಲಿ ಉಚ್ಚನೀಚಗಳೆಂದು ಪರಿಗಣಿಸದೆ ನಡೆದುಕೊಳ್ಳುತ್ತಿದ್ದನು. ಪಾಂಡವ-ಕೌರವ ಯುದ್ಧ ಆರಂಭವಾದಾಗ ಆತನು ಅರ್ಜುನನ ಸಾರಥಿಯಾಗಿ ಉಭಯ ಸೇನೆಗಳ ಮಧ್ಯೆ ರಥವನ್ನು ತಂದು ನಿಲ್ಲಿಸಿ, ಕಿಂಕರ್ತವ್ಯಮೂಢ ನಾದ ಅರ್ಜುನನಿಗೆ ತನ್ನ ಉಪದೇಶವಾಣಿಯಿಂದ ಉತ್ಸಾಹವನ್ನು ತುಂಬಿದನು. ಅಧರ್ಮದ ಕಳೆಯನ್ನು ಕೀಳಲು ಆತನಿಗೆ ಅರ್ಜುನನೊಂದು ನಿಮಿತ್ತ ಮಾತ್ರ. ಆ ಶಿಷ್ಯನಿಗೆ ಶ್ರೀಕೃಷ್ಣನು ಕರ್ಮವನ್ನು ಕರ್ಮಯೋಗವಾಗಿ ಮಾಡುವ ವಿಧಾನವನ್ನು ಹೇಳಿಕೊಡುತ್ತಾನೆ; ಯಜ್ಞದೃಷ್ಟಿಯಿಂದ ಕರ್ಮಮಾಡುವುದು ಹೇಗೆಂಬುದನ್ನು ಹೇಳಿಕೊಡುತ್ತಾನೆ; ಅದನ್ನು ಪೋಷಿಸಲು ಜ್ಞಾನ, ಭಕ್ತಿ, ವೈರಾಗ್ಯಗಳನ್ನು ತರುತ್ತಾನೆ. ಇಲ್ಲಿ ಶ್ರೀಕೃಷ್ಣ ಆಚಾರ್ಯಪುರುಷ ನಾಗಿ ಕಾಣಿಸಿಕೊಳ್ಳುತ್ತಾನೆ. ಅರ್ಜುನನ ನಿಮಿತ್ತದಿಂದ ಮಹೋನ್ನತವಾದ ಗೀತಾಸಂದೇಶ ಮಾನವ ಕೋಟಿಗೆ ದೊರೆಯುತ್ತದೆ. ಇಲ್ಲಿ ಬರುವುದು ಗೌರೀಶಂಕರ ಸದೃಶವಾದ ತತ್ವ, ಇಲ್ಲಿ ಬರುವುದು ಪೂರ್ಣ ದೃಷ್ಟಿ, ಯಾವುದನ್ನೂ ತೆಗಳದೆ, ನಿರಾಕರಿಸದೆ, ಮಾನವರನ್ನು ತಾನಿರುವಲ್ಲಿಂದಲೆ ಸತ್ಯದೆಡೆಗೆ ಮೆಟ್ಟಲು ಮೆಟ್ಟಲಾಗಿ ಕೊಂಡೊಯ್ದು ಮುಟ್ಟಿಸುತ್ತದೆ.

ಶ್ರೀರಾಮಕೃಷ್ಣರು ಭಾಗವತವನ್ನು ಕುರಿತು ಹೀಗೆ ಹೇಳುತ್ತಿದ್ದರು–‘ಇದು ಜ್ಞಾನವೆಂಬ ತುಪ್ಪದಲ್ಲಿ ಹುರಿದಿದೆ; ಭಕ್ತಿಯೆಂಬ ಸಕ್ಕರೆ ಪಾಕದಲ್ಲಿ ಅದ್ದಿದೆ’ ನಿಜ. ಇನ್ನಾವ ಪುರಾಣ ದಲ್ಲಿಯೂ ಇಲ್ಲಿ ಬರುವಷ್ಟು ತತ್ವ ಬರುವುದಿಲ್ಲ. ಮಾತೆತ್ತಿದರೆ ದೊಡ್ಡ ವೇದಾಂತ ತತ್ವ ಬೋಧನೆ ಪ್ರಾರಂಭವಾಗುತ್ತದೆ. ಶ್ರೀಕೃಷ್ಣ ಹುಟ್ಟಿದ ಸೂತಿಕಾಗೃಹದಲ್ಲಿ ದೇವಕಿ ಶ್ರೀಕೃಷ್ಣ ನನ್ನು ಸ್ತುತಿಸುವುದರಲ್ಲಿಯೆ ವೇದಾಂತಸಾರವೆಲ್ಲ ಇದೆ. ಕುರುಕ್ಷೇತ್ರದಲ್ಲಿ ಭಗವದ್ಗೀತೆ ಬಂದಂತೆ, ಶ್ರೀಕೃಷ್ಣ ನಿರ್ಯಾಣಕ್ಕೆ ಸ್ವಲ್ಪ ಮುಂಚೆ ಉದ್ಧವಗೀತೆ ಉದಿಸಿದೆ. ಇದಲ್ಲದೆ ಗೋಪಿಕಾಗೀತೆ, ಭ್ರಮರಗೀತೆ, ಅಣುಗೀತೆ–ಮೊದಲಾದವೂ ಇಲ್ಲಿವೆ. ಆದರೆ ಇಲ್ಲಿನ ಜ್ಞಾನ ಭಕ್ತಿಗೆ ವಿರೋಧವಾಗಿಲ್ಲ. ಉತ್ತಮ ಜ್ಞಾನ, ಉತ್ತಮ ಭಕ್ತಿ–,ಎರಡೂ ಪರಸ್ಪರ ಪೂರಕವಾಗಿವೆ. ಹಾಗೆಯೆ ದ್ವೈತ ಅದ್ವೈತಗಳು ಕೂಡ. ಇತರ ಗ್ರಂಥಗಳಲ್ಲಿ ಬರುವ ಇವುಗಳ ಕಾದಾಟ ಇಲ್ಲಿಲ್ಲ. ಒಂದು ನಿಜವಾದರೆ ಮತ್ತೊಂದು ಸುಳ್ಳೇ ಆಗಬೇಕೆ? ಬೇರೆ ಬೇರೆ ದೃಷ್ಟಿಯಿಂದ ಎರಡೂ ಏಕೆ ನಿಜವಾಗಬಾರದು? ಇಲ್ಲಿ ಭಗವಂತನೆ ಈ ಸೃಷ್ಟಿ ಮತ್ತು ಜೀವರಿಗೆಲ್ಲ ಉಪಾದಾನಕಾರಣ, ನಿಮಿತ್ತಕಾರಣ; ಅಂದ ಮೇಲೆ ಅವನು ಪ್ರತಿಯೊಂದು ಕಣದಲ್ಲಿಯೂ ಜೀವಾಣುವಿನಲ್ಲಿಯೂ ಸ್ಪಂದಿಸುತ್ತಿರಬೇಕು.

ಭಾಗವತದಲ್ಲಿ ಜ್ಞಾನಮಾರ್ಗನಿರೂಪಣೆ ಬೇಕಾದಷ್ಟು ಇದೆ. ಆದರೂ ಇಲ್ಲಿ ಪ್ರಾಮು ಖ್ಯತೆ ಇರುವುದು ಭಕ್ತಿಗೆ. ಭಕ್ತಿಯಿಂದ ಏನನ್ನು ಬೇಕಾದರೂ ಪಡೆಯಬಹುದೆನ್ನುತ್ತದೆ ಭಾಗವತ. ಪರಾಭಕ್ತಿ ಬಂದರೆ ಜ್ಞಾನ ತಾನಾಗಿಯೆ ಬರುತ್ತದೆ. ಒಂದು ಮುಂದೆ, ಮತ್ತೊಂದು ಹಿಂದೆ. ಗೋಪಿಯರು ಶ್ರೀಕೃಷ್ಣನನ್ನು ಪ್ರೀತಿಸಿದರು. ಆ ಪ್ರೀತಿಗೆ ತಮ್ಮ ಸರ್ವಸ್ವವನ್ನೂ ಅರ್ಪಿಸಿದರು. ಆ ಮೂಲಕ ಅವನೇ ಪರಬ್ರಹ್ಮ, ಅವನಿಲ್ಲದ ಸ್ಥಳವಿಲ್ಲ ಅವನಿಲ್ಲದ ಕಾಲವಿಲ್ಲ–ಎಂಬುದನ್ನು ಅರಿತರು. ಭಗವಂತನನ್ನು ತಮ್ಮ ಪ್ರೀತಿಯ ಪಂಜರದಲ್ಲಿ ಕೂಡಿಹಾಕಿದರು. ನಿರ್ಗುಣ ನಿರಾಕಾರನಾದವನು ಗೋಪಿಯರ ಪ್ರೇಮ ಪಾಶಕ್ಕೆ ಸಿಕ್ಕಿ, ಅವರ ಹೃದಯ ಪಂಜರದಲ್ಲಿ ಶ್ರೀಕೃಷ್ಣನಂತೆ ನೆಲೆಸಿದನು. ಅವರೆಂತಹ ಮಹಾವ್ಯಕ್ತಿಗಳು ಎಂಬುದನ್ನು ಉದ್ಧವನ ಬಾಯಿಂದಲೇ ಕೇಳಬಹುದು. ಶ್ರೀಕೃಷ್ಣನು ಮಧುರೆಗೆ ಹೋದಮೇಲೆ ಗೋಪಿಯರನ್ನು ಸಮಾಧಾನಮಾಡಲೆಂದು ಉದ್ಧವನನ್ನು ಗೋಕುಲಕ್ಕೆ ಕಳುಹಿಸಿದನು. ಆಗ ಗೋಪಿಯರೆಲ್ಲ ಅವನ ಸುತ್ತ ಗುಂಪು ಕೂಡುವರು. ತಮ್ಮನ್ನು ಶ್ರೀಕೃಷ್ಣನು ನೆನೆಸಿಕೊಳ್ಳುವನೋ ಎಂದು ಅವರು ಉದ್ಧವನನ್ನು ಕೇಳಿದಾಗ ಉದ್ಧವ ಹೇಳುತ್ತಾನೆ–‘ಯೋಗಿಗಳು ಅನುಗಾಲವೂ ಧ್ಯಾನಮಾಡುವುದರಿಂದ ಕೂಡ ಯಾರನ್ನು ಪಡೆಯುವುದು ಅಸಾಧ್ಯವೋ, ಜ್ಞಾನಿಗಳು ನಿತ್ಯಾನಿತ್ಯ ವಿಮರ್ಶೆಯಿಂದ ಯಾರನ್ನು ತಿಳಿಯಲು ಅಶಕ್ತರಾಗಿರುವರೋ ಅಂತಹ ಶ್ರೀಕೃಷ್ಣನನ್ನು ನೀವು ನಿಮ್ಮ ಪ್ರೀತಿಯ ಪಂಜರದಲ್ಲಿ ಕೂಡಿ ಹಾಕಿರುವಿರಿ. ನೀವೇ ಧನ್ಯರು. ಆ ಶ್ರೀಕೃಷ್ಣ ಸದಾ ನಿಮ್ಮನ್ನು ಚಿಂತಿಸುತ್ತಿರುವನು’ ಎಂದು. ಆತನು ಗೋಪಿಯರಿಂದ ಬೀಳ್ಕೊಳ್ಳುವಾಗ ಅವರನ್ನು ಮುಕ್ತಕಂಠನಾಗಿ ಹೊಗಳುವನು. ಬೃಂದಾವನದಲ್ಲಿ ಒಂದು ತರುಲತೆಯಾಗಿ ಹುಟ್ಟುವುದು ಕೂಡ ಪುಣ್ಯ ಕರವಂತೆ! ಏಕೆಂದರೆ ಅವಕ್ಕೆ ಗೋಪಿಯರ ಪಾದಧೂಳಿ ಸೋಂಕುವ ಸುಯೋಗವಿದೆಯಂತೆ! ಮಹಾಜ್ಞಾನಿಯಾದ ಉದ್ಧವ ಗೋಪಿಯರಿಗೆ ನಮ ಸ್ಕಾರಮಾಡಿ ಹಿಂದಿರುಗುತ್ತಾನೆ. ಪ್ರೇಮದ ಮೂಲಕ ಏನನ್ನು ಸಾಧಿಸಬಹುದು ಎಂಬು ದನ್ನು ಗೋಪೀಪ್ರೇಮದ ಮೂಲಕ ಭಾಗವತದಲ್ಲಿ ಚಿತ್ರಿಸಿದ್ದಾರೆ, ವ್ಯಾಸರು.

ಭಾಗವತದಲ್ಲಿ ಅನೇಕ ಕಡೆ ಬ್ರಾಹ್ಮಣನನ್ನು ಕೊಂಡಾಡುವುದು ಕಂಡುಬರುತ್ತದೆ. ಬ್ರಾಹ್ಮಣನೆಂದರೆ ಬ್ರಹ್ಮಜ್ಞನಾದ ಪುರುಷ. ಇದು ಹುಟ್ಟಿನಿಂದ ಬಂದುದಲ್ಲ, ಗುಣ ಕರ್ಮಗಳಿಂದ ಬಂದುದು. ಇಂದಿನ ನಮ್ಮ ಸಮಾಜದಲ್ಲಿ ಅಂದಿನ ವರ್ಣಾಶ್ರಮದ ಯಾವ ನಿಯಮವೂ ಇಲ್ಲ, ಯಾವ ಶಿಸ್ತೂ ಇಲ್ಲ. ಇಂದು ಬ್ರಹ್ಮಚರ್ಯಾಶ್ರಮವಿದೆ ಎಂದು ಹೇಳಿದರೂ ಹಿಂದಿನ ಯಾವ ಶಿಸ್ತೂ ಈ ಬ್ರಹ್ಮಚಾರಿಗಳಲ್ಲಿ ಕಾಣಬರುವುದಿಲ್ಲ. ಹಾಗೆಯೆ ಗೃಹಸ್ಥಾಶ್ರಮ. ಹಿಂದೆ ಎಲ್ಲ ಆಶ್ರಮಗಳಿಗೂ ಕಲಶಪ್ರಾಯವಾಗಿತ್ತು ಗೃಹಸ್ಥಾ ಶ್ರಮ. ಅದರ ಧ್ಯೇಯಗಳೆಲ್ಲ ಇಂದು ಮಂಗಮಾಯವಾಗಿವೆ. ಯಾವ ಮಂತ್ರಗಳನ್ನು ಹೇಳುತ್ತಾ ವರನು ಮದುವೆಯಾಗುವನೊ, ಆ ಮಂತ್ರಗಳ ಅರ್ಥವೆ ಗೊತ್ತಿಲ್ಲ, ಅವನಿಗೆ. ಇನ್ನು ವಾನಪ್ರಸ್ಥವೆಂಬ ಹೆಸರು ಕೂಡ ಮಾಯವಾಗಿದೆ. ಹೆಂಡತಿ ಸತ್ತರೆ ಮತ್ತೊಬ್ಬ ಹೆಂಡತಿಯ ಬಯಕೆಯೆ ಹೊರತು ಸಂಸಾರ ವಿಮುಖತೆಯೇನೂ ಇಲ್ಲ. ಅದರಂತೆಯೆ ಸಂನ್ಯಾಸ. ಪ್ರಾಚೀನಕಾಲದಲ್ಲಿ ದ್ವಿಜರಾದ ಪ್ರತಿಯೊಬ್ಬರೂ ಕಡೆಯಲ್ಲಿ ಸಂನ್ಯಾಸ ಸ್ವೀಕಾರಮಾಡಿಯೆ ಕಾಲವಾಗುತ್ತಿದ್ದರು. ನಾವಿಂದು ‘ಆಶ್ರಮ’ ಎಂದು ಕೂಗುತ್ತೇವೆ, ಆದರೆ ಆಶ್ರಮದ ಶಿಸ್ತು ಮಾತ್ರ ಇಲ್ಲ. ಆ ಶಿಸ್ತನ್ನು ಮತ್ತೆ ನಾವು ಬಳಕೆಗೆ ತರಬೇಕು. ಸಮಾಜ ಈ ಆಶ್ರಮಧರ್ಮಗಳನ್ನು ಎಷ್ಟು ಕಟ್ಟುನಿಟ್ಟಾಗಿ ಪಾಲಿಸಿದರೆ ಅಷ್ಟು ಬಲಾಢ್ಯವಾಗುವುದು.

ಆಶ್ರಮದಂತೆ ನಮ್ಮ ವರ್ಣವಿಭಾಗವೂ ಚಲ್ಲಾಪಿಲ್ಲಿಯಾಗಿದೆ. ಶ್ರೀಕೃಷ್ಣನು ಗೀತೆ ಯಲ್ಲಿ ‘ನಾಲ್ಕು ವರ್ಣಗಳು ಗುಣಕರ್ಮಕ್ಕೆ ಅನುಸಾರವಾಗಿ ನನ್ನಿಂದ ಮಾಡಲ್ಪಟ್ಟಿತು’ ಎನ್ನುತ್ತಾನೆ. ಆತನ ಮಾತೆಂದರೆ ಈಗಿನ ಕಾಲದ ಕೆಲವು ಜಗದ್ಗುರುಗಳ ಮಾತಿನಂತಲ್ಲ. ಈಗೇನು, ಕೆಲವು ಊರುಗಳಲ್ಲೊ ಹಳ್ಳಿಗಳಲ್ಲೊ ಇರುವ ಯಾವುದೋ ಒಂದು ಕೋಮಿನ ಗುರು ಜಗದ್​ಗುರು! ಶ್ರೀಕೃಷ್ಣ ಅಂತಹವನಲ್ಲ. ಎಲ್ಲ ಕಾಲಕ್ಕೆ, ಎಲ್ಲ ದೇಶಕ್ಕೆ, ಎಲ್ಲ ಜಾತಿ ಜನಾಂಗಕ್ಕೆ ಅಗತ್ಯವಾದ ಸಂದೇಶವನ್ನು ಎಲ್ಲರಿಗೂ ಅನ್ವಯಿಸುವ ದೃಷ್ಟಿಯಿಂದ, ಗೀತೆಯಲ್ಲಿ ಬೋಧಿಸುತ್ತಾನೆ. ಈಗ ನಾಲ್ಕು ವರ್ಣಗಳ ಶಿಸ್ತನ್ನು ಯಾರೂ ಜಾತಿಯ ದೃಷ್ಟಿ ಯಿಂದ ಪಾಲಿಸುತ್ತಿಲ್ಲ. ಒಂದೇ ಜಾತಿಯಲ್ಲಿ ನಾಲ್ಕುವರ್ಣದವರೂ ಹುಟ್ಟುತ್ತಾರೆ. ಒಬ್ಬ ಬ್ರಾಹ್ಮಣನಿಗೆ ನಾಲ್ಕು ಜನ ಮಕ್ಕಳಿದ್ದರೆ, ಅವರಲ್ಲಿ ಒಬ್ಬ ವೇದಾಧ್ಯಯನಸಂಪನ್ನನಾಗ ಬಹುದು, ಅವನು ಬ್ರಾಹ್ಮಣ; ಇನ್ನೊಬ್ಬ ಸೈನ್ಯಕ್ಕೆ ಸೇರಬಹುದು, ಅವನು ಕ್ಷತ್ರಿಯ; ಮತ್ತೊಬ್ಬ ವ್ಯಾಪಾರಕ್ಕೆ ಕೈ ಹಾಕಬಹುದು, ಅವನು ವೈಶ್ಯ; ನಾಲ್ಕನೆಯವನು ಯಾವುದೋ ಒಂದು ಕಾರ್ಖಾನೆಯಲ್ಲಿ ಕೂಲಿಯಾಗಿ ಸೇರಿಕೊಳ್ಳಬಹುದು, ಅವನು ಶೂದ್ರ. ಹೀಗೆ ಒಂದೇ ಮನೆಯಲ್ಲಿ ನಾಲ್ಕು ವರ್ಣದವರೂ ಇರುವುದು ಸಾಧ್ಯ. ಅವನು ಯಾವ ವರ್ಣದ ವೃತ್ತಿಯನ್ನು ಕೈಕೊಂಡಿರಲಿ, ಅದಕ್ಕೊಂದು ನೀತಿಯಿದೆ, ಧರ್ಮವಿದೆ; ಅದನ್ನು ಪಾಲಿಸಲಿ. ಬ್ರಹ್ಮಜ್ಞಾನಿಯಾದ ಬ್ರಾಹ್ಮಣ ಭಾರತದಲ್ಲಿ ಮಾತ್ರವೇ ಅಲ್ಲ, ಎಲ್ಲ ದೇಶಗಳಲ್ಲಿಯೂ ಇದ್ದಾನೆ, ಇರಲೇಬೇಕು. ಅವನು ದೂರದರ್ಶಿ, ದ್ರಷ್ಟಾರ, ಸಮಾಜದ ಧ್ಯೇಯಗಳನ್ನೂ ಅವುಗಳ ಸಾಧನಾ ವಿಧಾನವನ್ನೂ ತಿಳಿದವನು. ಅವನಿಗೆ ಲೌಕಿಕವಾದ ಅಧಿಕಾರವಾವುದೂ ಇಲ್ಲ. ಆದರೆ ಜನ ಅವನ ಮಾತನ್ನು ಗೌರವದಿಂದ ಪಾಲಿಸುತ್ತಾರೆ. ಆತನಿಗೆ ಲವಲೇಶವೂ ಸ್ವಾರ್ಥವಿಲ್ಲ. ಅವನ ಕಣ್ಣಿಗೆ ಅಜ್ಞಾನದ ಪರೆ ಮುಸುಕಿಕೊಳ್ಳುವುದಿಲ್ಲ. ಆದ್ದರಿಂದ ವಸ್ತು ಹೇಗಿದೆಯೊ ಹಾಗೆಯೆ ಆತನಿಗೆ ಕಾಣುತ್ತದೆ. ‘ತತ್ವಜ್ಞ’ ಎಂಬ ಮಾತಿಗೆ ನಿಜವಾದ ಅರ್ಥ ಇದೆಯೆ. ಇಂತಹ ವ್ಯಕ್ತಿಯನ್ನೆ ಭಾಗವತದಲ್ಲಿ ಬ್ರಾಹ್ಮಣನೆಂದು ಕರೆಯುವುದು, ಪುರಸ್ಕರಿ ಸುವುದು.

ಭಾಗವತದಲ್ಲಿ ಭಗವಂತನ ಅವತಾರದ ಪ್ರಸಕ್ತಿ ಬಂದಿದೆ. ಸೃಷ್ಟಿಯಲ್ಲಿ ಅಧರ್ಮ ಹೆಚ್ಚಿದಾಗ, ಆ ಅಧರ್ಮದ ಕಳೆಯನ್ನು ಕಿತ್ತು, ಧರ್ಮದ ಬೆಳೆಯನ್ನು ರಕ್ಷಿಸುವುದಕ್ಕಾಗಿ ಭಗವಂತ ಮಾನವ ರೂಪಿನಿಂದ ಬಂದು, ಮಾನವ ದೇವನೆಡೆಗೆ ಹೋಗಬೇಕಾದರೆ ಏನು ಮಾಡಬೇಕು ಎನ್ನುವುದನ್ನು ಮಾಡಿ ತೋರಿಸುತ್ತಾನೆ. ಸರ್ವಾಂತರ್ಯಾಮಿಯಾದ, ಸರ್ವ ಶಕ್ತನಾದ ಭಗವಂತ ಕೇವಲ ಒಬ್ಬ ಮಿತಿಯಿಂದ ಕೂಡಿದ ಮಾನವಾಕೃತಿಯಾಗಿ ಹೇಗೆ ಬರಲು ಸಾಧ್ಯ ಎಂದು ನಾವು ಕಕ್ಕಾಬಿಕ್ಕಿಯಾಗಬೇಕಾಗಿಲ್ಲ. ಅವತಾರವೆಂದರೆ ಹಾಗಲ್ಲ. ಆತ ಇರುವಲ್ಲಿ ಖಾಲಿಯಾಗಿ, ಇಲ್ಲಿ ಬರುವುದಿಲ್ಲ, ಅವನ ಒಂದು ಅಂಶ ಮಾತ್ರ ಇಳಿದು ಬರುತ್ತದೆ. ಹಾಗೆ ಬಂದಾಗ ತನ್ನ ಹಿಂದಿನದನ್ನು ಆತ ಎಂದೂ ಮರೆಯುವುದಿಲ್ಲ. ದೊಡ್ಡ ಸಾಗರದಲ್ಲಿ ಶೈತ್ಯಾಧಿಕ್ಯದಿಂದ ಕೆಲವು ಕಡೆ ನೀರು ಘನೀಭೂತವಾಗಿ, ಮಂಜಿನ ಗಡ್ಡೆ ಯಂತೆ ಮೇಲಿದ್ದರೆ, ಇಡೀ ಸಾಗರವೆ ಹಾಗಾಗಿದೆಯೆ? ಇಲ್ಲ; ಸಾಗರದ ಯಾವುದೋ ಒಂದು ಅಂಶ ಹಾಗಾಗಿರುವುದು. ಹಾಗೆಯೆ ಅವತಾರ.

ಅವತಾರದ ಜನ್ಮ ಕರ್ಮಗಳೆರಡೂ ದಿವ್ಯವಾದುವು. ಅವತಾರ ಪುರುಷನಲ್ಲಿ ವೇದ ವೇದಾಂತದ ಸಾರವೆಲ್ಲ ಸ್ಫುರಿಸುತ್ತಿರುತ್ತದೆ. ಭಕ್ತನಾದವನು ಆತನನ್ನು ಆದರ್ಶವಾಗಿಟ್ಟು ಕೊಂಡು, ಪ್ರೀತಿಸುತ್ತಾ ಹೋದರೆ ಸಾಕು; ಅವನನ್ನು ಪ್ರೀತಿಸಿದರೆ ಕ್ರಮೇಣ ಅವನನ್ನು ತಿಳಿದುಕೊಳ್ಳುತ್ತೇವೆ. ಪ್ರೇಮವೇ ದೊಡ್ಡ ಸಾಧನೆ. ಜ್ಞಾನ ಪ್ರೇಮದ ಹಿಂದೆ ಬರುತ್ತದೆ– ಎಂಜಿನ್ ಮುಂದೆ ಹೋದರೆ ರೈಲುಗಾಡಿಗಳು ಹಿಂಬಾಲಿಸುವಂತೆ. ಅವತಾರ ಸಾಮಾನ್ಯ ಮನುಷ್ಯನಿಗೆ ಒಂದು ಊರೆಗೋಲು. ಶ್ರೀಕೃಷ್ಣನನ್ನು ಊರೆಗೋಲಾಗಿ ಮಾಡಿಕೊಂಡು ಜನ ಸಂಸಾರಯಾತ್ರೆಯನ್ನು ಸುಗಮವಾಗಿ ನಡೆಸಿದ್ದಾರೆ, ನಡೆಸುತ್ತಿದ್ದಾರೆ ಮುಂದೆಯೂ ನಡೆಸುತ್ತಾರೆ. ನಮ್ಮ ಜನಾಂಗದಲ್ಲಿ ರಾಮ, ಕೃಷ್ಣ, ಶಿವ, ಶಕ್ತಿ ಇತ್ಯಾದಿ ಹೆಸರುಗಳು ರಕ್ತಗತವಾಗಿವೆ. ಎಲ್ಲಿಯವರೆಗೆ ಹಿಂದೂ ಜನಾಂಗವಿರುವುದೊ, ಅಲ್ಲಿಯವರೆಗೆ ಈ ಹೆಸರುಗಳು ಸ್ಥಿರವಾಗಿರುತ್ತವೆ.

ಗಂಗಾ, ಯಮುನಾ, ಗೋದಾವರಿ, ಕಾವೇರಿ ಮೊದಲಾದ ಜೀವನದಿಗಳು ಭರತ ಖಂಡದ ಜನಜೀವನವನ್ನು ಹೇಗೆ ಪವಿತ್ರಗೊಳಿಸಿ, ಹೊಟ್ಟೆ ಬಟ್ಟೆಗಳನ್ನು ನೀಡುತ್ತಿವೆಯೊ, ಹಾಗೆಯೇ ರಾಮಾಯಣ ಮಹಾಭಾರತ ಭಾಗವತಗಳು ಹಿಂದೂ ಸಂಸ್ಕೃತಿಯನ್ನು ಬೆಳೆಸಿ ಕೊಂಡು ಬರುತ್ತಿವೆ. ನಮ್ಮ ಜನಾಂಗದ ತಾಯಿಬೇರು ನೆಲಸಿರುವುದೇ ಈ ಗ್ರಂಥಗಳಲ್ಲಿ. ಎಷ್ಟೋ ಚಕ್ರಾಧಿಪತ್ಯಗಳು ಮೇಲೆದ್ದು, ಮುನ್ನಡೆದು, ನಾಮಾವಶೇಷವಾದವು. ಈ ರಾಜ್ಯ ಗಳಾಗಲಿ, ರಾಜರಾಗಲಿ, ಅವರ ಧನಕನಕಗಳಾಗಲಿ ನಮ್ಮ ಆಸ್ತಿಯಲ್ಲ. ನಮ್ಮನ್ನು ಒಂದು ಜನಾಂಗವಾಗಿ ನಿಲ್ಲಿಸಿರುವ ಬಹು ದೊಡ್ಡ ಆಸ್ತಿಯೆಂದರೆ ಈ ಮಹಾಗ್ರಂಥಗಳೆ. ಇಡೀ ಭರತಖಂಡವನ್ನು ಒಟ್ಟುಗೂಡಿಸಿರುವುದು ಈ ಗ್ರಂಥಗಳೇ. ನಮ್ಮ ಜನಾಂಗದ ಮುಂದೆ ಮಹೋನ್ನತವಾದ ಒಂದು ಆದರ್ಶವನ್ನು ಇಟ್ಟಿರುವುದು ಈ ಗ್ರಂಥಗಳೆ. ಇವನ್ನು ನಮ್ಮಲ್ಲಿ ಪ್ರತಿಯೊಬ್ಬರೂ ಓದಿ ತಿಳಿದುಕೊಳ್ಳಬೇಕು, ಹಾಗೆ ತಿಳಿದುಕೊಂಡವರು ಇತರ ರಿಗೆ ಅದನ್ನು ತಿಳಿಸಬೇಕು. ಹಾಗೆ ಹೇಳಿದಾಗಲೇ ತಾನು ತಿಳಿದದ್ದು ತನಗೆ ಚೆನ್ನಾಗಿ ಅರ್ಥ ವಾಗುವುದು. ಈ ಅರ್ಥದಲ್ಲಿಯೇ ತೈತ್ತಿರೀಯೋಪನಿಷತ್ತು, ಅಧ್ಯಯನ ಪ್ರವಚನಗಳನ್ನು ನಿಲ್ಲಿಸಬೇಡವೆಂದು ಬೋಧಿಸುವುದು. ರಾಮಾಯಣ ಮಹಾಭಾರತಗಳನ್ನು ಜನರು ಅರಿತುಕೊಳ್ಳುವುದಕ್ಕೆ ಸಾಕಷ್ಟು ಸಾಹಿತ್ಯ ಸಂಪತ್ತು ಸೃಷ್ಟಿಯಾಗಿದೆ. ಆದರೆ ಭಾಗವತ ಪುರಾಣ ಅಷ್ಟು ಪ್ರಚಾರವಾಗಿಲ್ಲ. ಶ್ರೀ ಶಾಮರಾಯರು ಈ ಸೇವೆಯನ್ನು ಕೈಕೊಂಡು, ಯಶಸ್ವಿಯಾಗಿ ನೆರವೇರಿಸಿದ್ದಾರೆ. ಅವರ ಭಾಷೆ ಸರಳ ಸುಂದರವಾಗಿದೆ, ನೇರವಾಗಿದೆ. ತಾವು ಅನುಭವಿಸಿ, ಅದನ್ನು ಇತರರಿಗೆ ಹಂಚ ಹೊರಟಿದ್ದಾರೆ. ಈ ಗ್ರಂಥವನ್ನು ಓದುವವರಿಗೆಲ್ಲ ಇದು ವೇದ್ಯವಾಗುತ್ತದೆ. ಭಾಗವತವನ್ನು ಕೇಳುವುದು ಪುಣ್ಯಕಾರ್ಯ, ಹೇಳುವುದು ಪುಣ್ಯಕಾರ್ಯ, ಬರೆಯುವುದು ಪುಣ್ಯಕಾರ್ಯ. ಹಾಗೆಯೆ ಅದನ್ನು ಓದುವುದೂ ಪುಣ್ಯಕಾರ್ಯವೆ. ಶಾಮರಾಯರು ಬರೆದು ಮಾಡಿರುವ ಪುಣ್ಯಕಾರ್ಯವನ್ನು ಓದುಗರು ಓದುವ ಪುಣ್ಯಕಾರ್ಯದಿಂದ ಸಫಲಗೊಳಿಸಲೆಂದು ಹಾರೈಸುತ್ತೇನೆ:

ಗೋಪಿಯರು ಗೋಪಿಕಾಗೀತದಲ್ಲಿ ಶ್ರೀಕೃಷ್ಣನನ್ನು ಸ್ತುತಿಸಿರುವ ಒಂದು ಸಣ್ಣ ಶ್ಲೋಕದೊಡನೆ ಈ ಸಣ್ಣ ಮುನ್ನುಡಿಯ ನನ್ನ ಕೈಂಕರ್ಯವನ್ನು ಸಲ್ಲಿಸುತ್ತೇನೆ.

\begin{verse}
ತವ ಕಥಾಮೃತಂ ತಪ್ತಜೀವನಂ ಕವಿಭಿರೀಡಿತಂ ಕಲ್ಮಷಾಪಹಂ\\ಶ್ರವಣಮಂಗಳಂ ಶ್ರೀಮದಾತತಂ ಭುವಿ ಗೃಣಂತಿ ಯೇ ಭೂರಿದಾ ಜನಾಃ
\end{verse}

[ತಪ್ತ ಜೀವರಿಗೆ ನಿನ್ನ ಕಥೆ ಅಮೃತಪ್ರಾಯವಾಗಿದೆ. ಅದನ್ನು ಕವಿಗಳು ಬಹುವಿಧವಾಗಿ ಸ್ತೋತ್ರ ಮಾಡಿರುವರು. ಅದು ಕಲ್ಮಷವನ್ನು ಪರಿಹರಿಸುವುದು. ಕೇಳುವುದಕ್ಕೆ ಮಂಗಳ ಕರವಾಗಿದೆ. ಅದು ಸಂಪತ್​ಪ್ರದವಾಗಿದೆ. ಯಾರು ಇದನ್ನು ಪ್ರಚಾರ ಮಾಡುವರೊ ಅವರು ಪುಣ್ಯಶಾಲಿಗಳು.]

\begin{flushright}
\textbf{ಸೋಮನಾಥಾನಂದ}
\end{flushright}

\chapter*{ಗ್ರಂಥಕರ್ತನ ಬಿನ್ನಹ}

‘ಗೀತೆ’ಯೆಂದೊಡನೆ ‘ಭಗವದ್ಗೀತೆ’ ನಮ್ಮ ಮನಸ್ಸಿಗೆ ಬೋಧೆಯಾಗುವಂತೆ, ‘ಭಾಗ ವತ’ವೆಂದೊಡನೆಯೆ ಶ್ರೀಮದ್ಭಾಗವತವು ನಮ್ಮ ಕಣ್ಣ ಮುಂದೆ ಕಟ್ಟಿ ನಿಲ್ಲುತ್ತದೆ. ಅದು ನಿಗಮ ಕಲ್ಪತರುವಿನಿಂದ ಉದುರಿದ ದಿವ್ಯ ಫಲ! ಅದನ್ನು ನಾರದನು ವ್ಯಾಸ ಮಹರ್ಷಿ ಗಳಿಗೆ ತಂದಿತ್ತನು. ಅವರು ಅದನ್ನು ತಮ್ಮ ಆತ್ಮಸಂಭವನಾದ ಶುಕಮುನಿಗೆ ನೀಡಿದರು. ಆ ಶುಕಮುಖದಿಂದ ಅದು ಕಥಾಮೃತ ರಸರೂಪವಾಗಿ ಪರೀಕ್ಷಿತನಿಗೆ ಹರಿದು ಬಂದು, ಅನಂತರ ಭೂಮಂಡಲದಲ್ಲೆಲ್ಲ ಪ್ರವಹಿಸಿ, ಪ್ರವಹಿಸಿದೆಡೆಯನ್ನೆಲ್ಲ ಪವಿತ್ರವಾಗಿಸಿತು. ತಕ್ಷಕವಿಷಾನಲಭೀತನಾದ ಪರೀಕ್ಷಿತನು ವಿರಕ್ತಚಿತ್ತನಾಗಿ, ಪುಷಿಮುನಿಗಳೊಡನೆ ಶುಕಮುನಿಯನ್ನು ಸಂಧಿಸಿ, ತನಗೆ ಮೋಕ್ಷವನ್ನು ಕರುಣಿಸುವಂತೆ ಬೇಡಿದ. ಶುಕ ಮುನಿಯು ಮೊಟ್ಟಮೊದಲಬಾರಿಗೆ ಅದನ್ನು ಪರೀಕ್ಷಿತನಿಗೆ ಹೇಳಿದ. ಭೂಮಿ ಹದವಾ ಗಿತ್ತು; ಮಳೆ ಸುರಿಯಿತು; ಹುಲುಸಾದ ಬೀಜ ದಿವ್ಯವಾದ ಫಲ ಕೊಟ್ಟಿತು. ಶುಕಮುನಿ ಯಿಂದ ಭಾಗವತವನ್ನು ಕೇಳಿದ ಪರೀಕ್ಷಿತನು ಮೋಕ್ಷಸಾಮ್ರಾಜ್ಯದ ಅಧಿಪತಿಯಾದನು. ಪದ್ಮಪುರಾಣದ ಉತ್ತರಕಾಂಡದ ಮೊದಲನೆಯ ಅಧ್ಯಾಯದಲ್ಲಿ ಭಾಗವತದ ಮಹಿಮೆ ಯನ್ನು ಬಣ್ಣಿಸುತ್ತಾ, ಅದನ್ನು ಮೋಕ್ಷಶಾಸ್ತ್ರವೆಂದೂ, ಭಗವದ್ರೂಪವಾದುದೆಂದೂ, ಕಲಿಯುಗದಲ್ಲಿ ಇದನ್ನು ಓದಿದರೆ, ಕೇಳಿದರೆ ಮೋಕ್ಷ ಲಭಿಸುವುದೆಂದೂ ಹೇಳಿದೆ. ಶುಕಮಹರ್ಷಿ ಪರೀಕ್ಷಿತನಿಗೆ ಭಾಗವತವನ್ನು ಹೇಳಲು ಪ್ರಾರಂಭಿಸುತ್ತಲೆ ದೇವತೆಗಳು ಅಮೃತಭಾಂಡದೊಡನೆ ಆತನ ಬಳಿಗೆ ಬಂದು, ಆ ಅಮೃತಭಾಂಡವನ್ನು ಪರೀಕ್ಷಿತನಿಗೆ ಕೊಟ್ಟು, ತಮಗೆ ಭಾಗವತ ಕಥಾಮೃತವನ್ನು ನೀಡುವಂತೆ ಬೇಡಿದರಂತೆ! ಅವರು ಉತ್ತಮ ಭಕ್ತರಲ್ಲವೆಂದು ಶುಕಮುನಿ ಅವರಿಗೆ ಭಾಗವತಾಮೃತವನ್ನು ನೀಡಲಿಲ್ಲವಂತೆ! ಭಾಗ ವತವು ಕೇವಲ ಭಕ್ತಜನೈಕಲಭ್ಯ. ಅಮೃತಾಧಿಕ ಫಲಪ್ರದ! ಹಿಂದೆ ಶ್ರೀಕೃಷ್ಣ ಅರ್ಜುನನಿಗೆ ಭಗವದ್ಗೀತೆಯನ್ನು ಬೋಧಿಸಿದ; ಅದರಿಂದ ಅರ್ಜುನನ ‘ಮೋಹನಷ್ಟ’ವಾಯಿತು, ‘ಸ್ಮೃತಿಲಬ್ಧ’ವಾಯಿತು. ಆದರೆ ಭಾಗವತವನ್ನು ಕೇಳಿದ ಪರೀಕ್ಷೀತನಿಗೆ ಮುಕ್ತಿಯೇ ದೊರೆಯಿತು. ಆದ್ದರಿಂದ ಭಾಗವತದ ಮಹಿಮೆ ಅದ್ಭುತ, ಅಸದೃಶ. ಇದು ನಮ್ಮ ಜನ ರಲ್ಲಿ ಪರಂಪರೆಯಾಗಿ ಹರಿದುಬಂದಿರುವ ಭಾವನೆ.

ಭಾಗವತವು ಒಂದು ಮಹಾಪುರಾಣ. ಇಂತಹ ಮಹಾಪುರಾಣಗಳ ಸಂಖ್ಯೆ ಹದಿನೆಂಟು ಎಂದು ಹೇಳುತ್ತಾರೆ.

\begin{verse}
ಭದ್ವಯಂ ಮಧ್ವಯಂ ಚೈವ ಬ್ರತ್ರಯಂ ವ ಚತುಷ್ಟಯಂ\\ಅನಾಪಲಿಂಗಕೂಸ್ಕಾನಿ ಪುರಾಣಾನಿ ಪ್ರಚಕ್ಷತೇ ॥
\end{verse}

ಎಂಬ ಶ್ಲೋಕವು ಅವುಗಳ ಹೆಸರನ್ನು ಸೂಚಿಸುತ್ತದೆ. ಭ ಕಾರದಿಂದ ಪ್ರಾರಂಭವಾಗುವ ಹೆಸರುಳ್ಳ ಪುರಾಣಗಳು ಎರಡು (ಭವಿಷ್ಯತ್ ಪುರಾಣ, ಭಾಗವತ); ಮ ಕಾರದಿಂದ ಪ್ರಾರಂಭವಾಗುವವು ಎರಡು (ಮತ್ಸ್ಯಪುರಾಣ, ಮಾರ್ಕಾಂಡೇಯ ಪುರಾಣ): ಬ್ರಕಾರದಿಂದ ಪ್ರಾರಂಭವಾಗುವವು ಮೂರು (ಬ್ರಹ್ಮ, ಬ್ರಹ್ಮವೈವರ್ತ, ಬ್ರಹ್ಮಾಂಡ ಪುರಾಣಗಳು); ವ ಕಾರದಿಂದ ಪ್ರಾರಂಭವಾಗುವವು ನಾಲ್ಕು (ವರಾಹ, ವಾಯು, ವಾಮನ, ವಿಷ್ಣು ಪುರಾಣಗಳು); ಅ-ಅಗ್ನಿ ಪುರಾಣ; ನಾ-ನಾರದ ಪುರಾಣ; ಪ-ಪದ್ಮ ಪುರಾಣ; ಲಿಂ-ಲಿಂಗ ಪುರಾಣ; ಗ-ಗರುಡ ಪುರಾಣ; ಕೂ-ಕೂರ್ಮ ಪುರಾಣ; ಸ್ಕ-ಸ್ಕಾಂದ ಪುರಾಣ. ಈ ಹದಿನೆಂಟು ಪುರಾಣಗಳಲ್ಲಿ ಹತ್ತು ಶಿವ ಮಹಾತ್ಮ್ಯವನ್ನೂ, ನಾಲ್ಕು ವಿಷ್ಣು ಮಹಾತ್ಮ್ಯ ವನ್ನೂ, ಉಳಿದೆರಡು ಶಕ್ತಿ ಮತ್ತು ಗಣಪತಿಗಳ ಮಹಾತ್ಮ್ಯವನ್ನೂ ದಿಗ್ದರ್ಶಿಸುತ್ತವೆ. ಈ ಎಲ್ಲ ಪುರಾಣಗಳ ಸಾಲಿನಲ್ಲಿ, ಎಲ್ಲ ದೃಷ್ಟಿಯಿಂದಲೂ ಅಗ್ರಗಣ್ಯವಾಗಿ, ಅವುಗಳ ತಲೆ ವಣಿಯುವಂತಿರುವುದು ಶ್ರೀಮದ್ಭಾಗವತ ಮಹಾಪುರಾಣ.

‘ಪುರಾಣಮಿತ್ಯೇವ ನ ಸಾಧು ಸರ್ವಂ’ ಎಂಬ ಸೂಕ್ತಿಯೊಂದು ನಮ್ಮಲ್ಲಿ ಬಳಕೆಗೆ ಬಂದಿದೆಯಾದರೂ ನಮ್ಮ ಪ್ರಾಚೀನರು ‘ಪುರಾಣ’ಗಳಿಗೆ ಬೇಕಾದಷ್ಟು ಪ್ರಾಶಸ್ತ್ಯವನ್ನು ಕೊಟ್ಟಿದ್ದಾರೆ. ‘ಚರ್ತುರ್ವೇದವನ್ನು ಸಾಂಗವಾಗಿ ಅಧ್ಯಯನ ಮಾಡಿದರೂ ಪುರಾಣಗಳನ್ನು ಅರಿಯದವನು ಅವಿವೇಕಿ’ ಎನ್ನುತ್ತಾರೆ ಅವರು. ಅಂತಹವನನ್ನು ಕಂಡರೆ, ಅವನು ತನ್ನನ್ನು ಹಿಂಸೆಪಡಿಸುವನೆಂದು ವೇದಮಾತೆ ಓಡಿಹೋಗುತ್ತಾಳಂತೆ! ಬರಿಯ ವೇದ ಪಾಠಕನನ್ನು ಕಪ್ಪೆಗೆ ಹೋಲಿಸಿ ಗೇಲಿ ಮಾಡುವುದೂ ಉಂಟು. ಇತಿಹಾಸ ಪುರಾಣಗಳ ಮೂಲಕವೇ ವೇದವನ್ನು ಹಿಂಬಾಲಿಸಬೇಕಂತೆ! ಶತಕೋಟಿ ಪ್ರವಿಸ್ತರವಾಗಿ ಏಕರೂಪ ವಾಗಿದ್ದ ಪುರಾಣವನ್ನು ಮಹಾವಿಷ್ಣುವು ವ್ಯಾಸರೂಪದಿಂದ ಅವತರಿಸಿ, ಮಾನವರ ಆನು ಕೂಲ್ಯಕ್ಕಾಗಿ ಅದನ್ನು ನಾಲ್ಕು ಲಕ್ಷವಾಗಿ ಸಂಗ್ರಹಿಸಿ, ಹದಿನೆಂಟು ಪುರಾಣಗಳಾಗಿ ವಿಭಾಗಿ ಸಿದನಂತೆ. ಮೂಲದ ಶತಕೋಟಿ ಪುರಾಣ ಈಗಲೂ ಬ್ರಹ್ಮಲೋಕದಲ್ಲಿ ಇದೆಯೆಂದು ಬೃಹನ್ನಾರದೀಯ ತಿಳಿಸುತ್ತದೆ. ಇವೆಲ್ಲ ನಮಗೆ ಎಟುಕದ ಸಂಗತಿಗಳು. ಅವು ಹಾಗಿರಲಿ, ಸ್ಕಾಂದ ಪುರಾಣದ ರೇವಾಖಂಡದಲ್ಲಿ ಶ್ರುತಿ ಸ್ಮೃತಿಗಳು ವಿಪ್ರರ ಎರಡು ಕಣ್ಣೆಂದೂ, ಪುರಾಣವು ಅವರ ಮೂರನೆಯ ಕಣ್ಣೆಂದೂ–ಈ ಮೂರು ಕಣ್ಣುಗಳಿಂದಲೂ ನೋಡ ಬಲ್ಲವನು ಮಹೇಶ್ವರ ಸ್ವರೂಪನೆಂದೂ ಹೇಳಿದೆ. ಪುರಾಣಗಳನ್ನು ಅರಿತವನು ಮತ್ತೇ ನನ್ನೂ ಓದದಿದ್ದರೂ ಚಿಂತೆಯಿಲ್ಲ; ಅಲ್ಲಿ ಕಾಣಬಾರದ ಧರ್ಮವೇ ಇಲ್ಲ–ಎಂಬುದು ಅಲ್ಲಿನ ಧೋರಣೆ.

ಪುರಾಣಗಳಿಗೆ ಕೊಟ್ಟಿರುವ ಈ ಪ್ರಾಶಸ್ತ್ಯ ಕೇವಲ ನಿರರ್ಥಕವೆಂದು ಭಾವಿಸಬಾರದು. ನಮ್ಮ ವೇದಗಳಂತೆ ನಮ್ಮ ಪುರಾಣಗಳೂ ಪ್ರಾಚೀನ ಭಾರತದ ಸಂಸ್ಕೃತಿಯನ್ನು ದಿಗ್ದರ್ಶಿಸುವ ಸಾಧನಗಳೆಂದು ಹೇಳಿದರೆ ಅತಿಶಯೋಕ್ತಿಯಾಗಲಾರದು. ಏಕೆಂದರೆ ವೇದಗಳ ಸಾರವೆಲ್ಲ ಹರಿದುಬಂದು ಪುರಾಣಗಳಲ್ಲಿ ನೆಲೆಸಿದೆ. ವೇದವನ್ನು ಆಮೂಲಾಗ್ರವಾಗಿ ಪಠಿಸಿ ಅರ್ಥಮಾಡಿಕೊಳ್ಳುವುದೆಂದರೆ ಇಡೀ ಒಂದು ಜೀವಮಾನವೆಲ್ಲ ವಿನಿಯೋಗಿಸಿದರೂ ಅಸಾಧ್ಯ. ಆದ್ದರಿಂದ ಕಥೋಪನ್ಯಾಸಗಳ ಮೂಲಕ ವೇದಸಾರವನ್ನು ಅವು ಬೋಧಿಸು ತ್ತವೆ. ಆ ಕಾರಣದಿಂದಲೆ ಅವುಗಳನ್ನು ‘ಪಂಚಮ’ ವೇದವೆಂದು–ಮಹಾಭಾರತದಂತೆ– ಕರೆದು ಗೌರವಿಸುವುದೂ ಉಂಟು. ವೇದದ ವಿಷಯಗಳೇ ಸ್ಮೃತಿಗಳಲ್ಲಿ ವೇದಸ್ಮೃತಿ ಗಳೆರಡರ ವಿಷಯವೂ ಪುರಾಣಗಳಲ್ಲಿ ಅಡಕವಾಗಿದ್ದು, ಪುರಾಣಗಳು ವೇದಪುರುಷನ ಆತ್ಮವೆಂಬ ಹೊಗಳಿಕೆಗೆ ಪಾತ್ರವಾಗಿವೆ. ಪ್ರಾಚೀನ ಭಾರತದಲ್ಲಿ ಇವನ್ನು ಜನರಿಗೆ ವಿವರಿಸಿ ಹೇಳಿ ಪ್ರವಚನ ಮಾಡುವ ಪದ್ಧತಿ ಬಳಕೆಯಲ್ಲಿದ್ದಂತೆ ತೋರುತ್ತದೆ. ವಿದ್ವಾಂಸರು ಕೂಡ ಈ ಪ್ರವಚನಗಳನ್ನು ಕೇಳಲು ಹೋಗುತ್ತಿದ್ದಿರಬಹುದು. ಬಾಣ ಕವಿಯು ತನ್ನ ಗ್ರಾಮ ದಲ್ಲಿ ನಡೆಯುತ್ತಿದ್ದ ವಾಯುಪುರಾಣದ ಪ್ರವಚನವನ್ನು ಕೇಳಲು ಹೋಗಿದ್ದನೆಂದು ಹರ್ಷ ಚರಿತೆಯಿಂದ ಗೊತ್ತಾಗುತ್ತದೆ.

ಪುರಾಣಗಳ ಕಾಲದ ವಿಚಾರವಾಗಿ ಬೇಕಾದಷ್ಟು ಭಿನ್ನಾಭಿಪ್ರಾಯಗಳಿವೆ. ‘ಪ್ರಥಮಂ ಸರ್ವ ಶಾಸ್ತ್ರಾಣಾಂ ಪುರಾಣಂ ಬ್ರಹ್ಮಣಾ ಸ್ಮೃತಂ । ಅನಂತರಂ ಚ ವಕ್ತ್ರೇಭ್ಯೋ ವೇದಾ ಸ್ತಸ್ಯ ವಿನಿಃಸೃತಾಃ’ ಎಂಬ ವಾಯುಪುರಾಣ(೧-೬0)ದ ಹೇಳಿಕೆಯಂತೆ ವೇದಕ್ಕಿಂತ ಹಿಂದೆಯೇ ಒಂದು ಪುರಾಣವಿತ್ತೆಂದು ನಂಬುವುದಾದರೆ, ಪುರಾಣವು ವೇದಕ್ಕಿಂತ ಪ್ರಾಚೀನವಾದುದೆಂದಾಗುತ್ತದೆ. ವೇದಾಂತರ್ಗತವಾದ ಕಥೆಗಳು ಅದು ನಿಜವಿರಬಹು ದೆಂಬ ಭ್ರಾಂತಿಯನ್ನು ಹುಟ್ಟಿಸುತ್ತವೆ. ಆದರೆ ಪುರಾಣ ವೇದಗಳಗಿಂತ ಪ್ರಾಚೀನವೆಂಬು ದನ್ನು ಒಪ್ಪಲು ನಮ್ಮ ಮನಸ್ಸು ಒಡಂಬಡುವುದಿಲ್ಲ. ವೇದಾಂತರ್ಗತವಾದ ಕಥೇತಿಹಾಸ ಗಳು ವೇದಗಳಿಗೂ ಪೂರ್ವದಲ್ಲಿ ಪ್ರಚಾರದಲ್ಲಿದ್ದು, ಅವು ಅಂದಿನಿಂದ ಇಂದಿನವರೆಗೆ ಹರಿದುಕೊಂಡು ಬಂದಿದ್ದರೆ ಅದರಲ್ಲಿ ಆಶ್ಚರ್ಯವೇನೂ ಇಲ್ಲ. ವೇದದ ಕಥೇತಿಹಾಸಗಳೂ ಪ್ರಾರ್ಥನೆ ಮತ್ತು ಪೂಜಾವಿಧಾನಗಳೂ ಪುರಾಣಗಳಿಗೆ ಮೇಲ್ಪಂಕ್ತಿಯಾದವೆಂದು ಹೇಳಿ ದರೆ ಹೆಚ್ಚು ಸಮಂಜಸವಾಗಿರುವುದೆಂದು ತೋರುತ್ತದೆ. ವಿಷ್ಣುಪುರಾಣದಲ್ಲಿ ಪುರಾಣೋ ತ್ಪತ್ತಿಯ ಬೀಜಗಳು ವೇದಗಳ ಕೊಡುಗೆಯೆಂಬ ಅರ್ಥಬರುವಂತೆ ಹೇಳಿಕೆಯಿದೆ. ಪುಗ್ವೇದದಲ್ಲಿ ಹೇಳಿರುವ ವಿಶ್ವೋತ್ಪತ್ತಿ ನಿರೂಪಣೆಯ ಪುಕ್ಕುಗಳಿಗೂ ಪುರಾಣಗಳಲ್ಲಿ ಬರುವ ಅದರ ಪ್ರತಿಪಾದನೆಗೂ ಸಾಮರಸ್ಯ ಕಂಡುಬರುತ್ತದೆ. ಪಾಶ್ಚಾತ್ಯ ಪಂಡಿತನಾದ ಮ್ಯಾಕ್ಡೊನಲ್ ತನ್ನ \eng{History of Indian Literature} (ಪು. ೧೩೮)ನಲ್ಲಿ ವಿಶ್ವೋತ್ಪತ್ತಿಗೆ ಸಂಬಂಧಿಸಿದ ಪುಗ್ವೇದದ ಪುಕ್ಕುಗಳು ಭಾರತೀಯ ತತ್ವಶಾಸ್ತ್ರಕ್ಕೆ ಮಾತ್ರವೇ ಅಲ್ಲದೆ ಪುರಾಣಗಳಿಗೆ ಕೂಡ ಬೀಜರೂಪವಾಯಿತೆಂದು ಹೇಳುತ್ತಾನೆ. ಪುಕ್ಕುಗಳು ಮಾತ್ರವೇ ಅಲ್ಲದೆ, ಬ್ರಾಹ್ಮಣಗಳಲ್ಲಿನ ಕಥೇತಿಹಾಸಗಳು ಕೂಡ ಪುರಾಣಗಳಲ್ಲಿನ ವಿವಿಧಾಂಶಗಳಿಗೆ ಆಕರವಾದುವೆಂದು ಪಾಶ್ಚಾತ್ಯ ಪಂಡಿತ ವೇಬರ್​ನ ಅಭಿಪ್ರಾಯ.

ಪುರಾಣಗಳು ಬಹು ಪ್ರಾಚೀನವೆಂಬುದರಲ್ಲಿ ಯಾವ ಸಂದೇಹವೂ ಇಲ್ಲವಾದರೂ, ಅವು ಇಂದಿನ ಸ್ವರೂಪವನ್ನು ಎಂದು ಪಡೆದವೆಂದು ಹೇಳುವುದು ಕಷ್ಟ. ಏಕೆಂದರೆ ಇವು ಗಳನ್ನು ಹೇಳುತ್ತಿದ್ದುದು ಸಾಮಾನ್ಯವಾಗಿ ರಾಜರು ಯಜ್ಞ ಮಾಡುತ್ತಿದ್ದ ಸ್ಥಳಗಳಲ್ಲಿ. ಈ ಪುರಾಣಶ್ರವಣ ಮಾಡಿಸುವವರಿಗೆ ಸೂತರೆಂದು ಹೆಸರು. (ಇವರು ಪೃಥು ಮಹಾರಾಜನ ಯಜ್ಞ ಕುಂಡದಿಂದ ಉದಿಸಿಬಂದವರಂತೆ! ಇವರಿಗೆ ‘ಪುಷಿ’ ಸ್ಥಾನಮಾನ ಸಲ್ಲುತ್ತಿತ್ತು!) ಇವರು ತಮ್ಮ ಗುರುವಿನಿಂದ ಕೇಳಿದ ವಿಷಯಗಳನ್ನು ಮಾತ್ರವೇ ಅಲ್ಲದೆ, ಸಂದರ್ಭ ಸನ್ನಿವೇಶಗಳಿಗೆ ತಕ್ಕಂತೆ ಅನೇಕ ವಿಷಯಗಳನ್ನೂ ಸೇರಿಸುತ್ತಾ ಹೋಗಿದ್ದಾರೆ. ‘ಅನುಶು ಶ್ರುಮ ಸ್ಮೃತಃ ಇತಿ ನಃ ಶ್ರುತಮ್​’ ಎಂಬ ಹೇಳಿಕೆ ಪುರಾಣಕರ್ತರಿಂದ ಕೇಳಿಬರಬಹು ದಾದರೂ ಯಜ್ಞಕರ್ತೃವಿನ ವಂಶಾನುಚರಿತವನ್ನು ಕುರಿತು ಅನೇಕ ಹೊಸ ವಿಷಯಗಳನ್ನು ಅವರು ಸೇರಿಸದೆ ಬಿಟ್ಟಿಲ್ಲ. ಇದರಿಂದ ಪ್ರಾಚೀನ ಭಾರತದ ಇತಿಹಾಸ ನಿರ್ಮಾಣವಾದಂ ತಾಯಿತು, ನಿಜ. ಆದರೆ ಪುರಾಣಗಳ ಸ್ವರೂಪ ಸ್ಥಾಯಿಯಾಗದೆ ಬೆಳೆಯುತ್ತಾ, ಬದಲಾಯಿ ಸುತ್ತಾ, ರೂಪಾಂತರ ಹೊಂದುತ್ತಾ ಹೋಯಿತೆಂದು ತೋರುತ್ತದೆ.

ಛಾಂದೋಗ್ಯೋಪನಿಷತ್ತಿನ ಏಳನೆಯ ಭಾಗದಲ್ಲಿ ಪುರಾಣೇತಿಹಾಸಗಳನ್ನು ತಾನು ಪಠಿಸಿದುದಾಗಿ ನಾರದರು ತಿಳಿಸುತ್ತಾರೆ. ಸೂತ್ರ ವಾಙ್ಮಯದಲ್ಲಿ ಪುರಾಣಗಳಿಂದ ಆಯ್ದ ಶ್ಲೋಕಗಳು ಅಲ್ಲಲ್ಲಿಯೆ ಕಂಡುಬರುತ್ತವೆ. ಮಹಾಭಾರತ ಶಾಂತಿಪರ್ವದಲ್ಲಿ ಪುರಾಣ ಮಹಾತ್ಮೆ ಮತ್ತು ಹದಿನೆಂಟು ಪುರಾಣಗಳ ಹೆಸರುಗಳು ಕಂಡುಬರುತ್ತವೆ. ಕೌಟಿಲ್ಯನ ಅರ್ಥಶಾಸ್ತ್ರದಲ್ಲಿ ರಾಜಕುಮಾರರು ಅತ್ಯಗತ್ಯವಾಗಿ ಓದಬೇಕಾದ ಗ್ರಂಥಗಳಲ್ಲಿ ಪುರಾಣ ಗಳೂ ಸೇರಿವೆ. ಅಮರಸಿಂಹನು ತನ್ನ ಕೋಶದಲ್ಲಿ ಪುರಾಣಗಳ ಪಂಚಲಕ್ಷಣವನ್ನು ತಿಳಿಸುತ್ತಾನೆ. ಇದನ್ನು ನೋಡಿದರೆ ಪುರಾಣಗಳು ಬ್ರಾಹ್ಮಣ, ಉಪನಿಷತ್ತುಗಳ ಕಾಲದಿಂದ ಗುಪ್ತ ರಾಜರ ಕಾಲದವರೆಗೆ ರೂಪಾಂತರ ಹೊಂದುತ್ತಲೆ ಹೋಗಿರಬಹುದೆಂದು ಊಹಿಸ ಬಹುದು. ಅನೇಕ ಶತಮಾನಗಳ ಕಾಲ ಅವು ರೂಪಾಂತರ ಹೊಂದುತ್ತಾ ಹೋಗಿ, ಇಂದಿನ ರೂಪವನ್ನು ಪಡೆದಿವೆ. ಕೆಲವು ಪುರಾಣಗಳು ಉಪನಿಷತ್ತುಗಳ ಕಾಲದಲ್ಲಿ, ಕೆಲವು ಭಾರತ ಇತಿಹಾಸದ ಕಾಲದಲ್ಲಿ, ಮತ್ತೆ ಕೆಲವು ಅಲ್ಲಿಂದೀಚೆಗೂ ತಮ್ಮ ಇಂದಿನ ರೂಪವನ್ನು ಪಡೆದಿರಬಹುದೆಂದು ಊಹಿಸಬೇಕಾಗಿದೆ. ಆಯಾ ಪುರಾಣಗಳಲ್ಲಿನ ವಸ್ತು ವಿಷಯಗಳನ್ನು ಪರಿಶೀಲಿಸಿ, ಬಾಹ್ಯ ಮತ್ತು ಆಂತರಿಕವಾದ ಪ್ರಮಾಣಗಳನ್ನು ಅಳೆದು, ಸುರಿದು, ಅವುಗಳ ರಚನಾಕಾಲವನ್ನು ನಿರ್ಧರಿಸಬೇಕು.

ಶ್ರೀಮದ್ಭಾಗವತದ ಕಾಲ ಮತ್ತು ಕರ್ತೃವನ್ನು ಕುರಿತು ಸಾಕಷ್ಟು ಅಭಿಪ್ರಾಯಭೇದ ವಿದೆ. ಪುರಾಣವಾಙ್ಮಯವೆಲ್ಲವೂ ವ್ಯಾಸಕೃತವೆಂದು ಹೇಳಿ, ಅದರ ಕಾಲ, ಕರ್ತೃಗಳನ್ನು ಒಂದೇ ಮಾತಿನಲ್ಲಿ ನಿರ್ಧರಿಸುವುದು ಸಂಪ್ರದಾಯವಾದಿಗಳ ಪರಿಪಾಠಿ. ಆದರೆ ಆಧುನಿಕರು ಇದನ್ನು ಒಪ್ಪರು. ಇತರ ಪುರಾಣಗಳ ಶೈಲಿಯಿಂದ ಇದು ಭಿನ್ನವಾದುದರಿಂದ ಇದು ವ್ಯಾಸಕೃತವಲ್ಲವೆಂದು ಅವರಲ್ಲಿ ಕೆಲವರ ವಾದ. ಇದು ಅತ್ಯಂತ ಅರ್ವಾಚೀನ ವಾದುದೆಂದೂ, ಹದಿಮೂರನೆಯ ಶತಮಾನದ ಮಧ್ವಾಚಾರ್ಯರು ಇದಕ್ಕೆ ವ್ಯಾಖ್ಯಾನವನ್ನು ಬರೆದಿರುವರಾದರೂ ಹನ್ನೆರಡನೆಯ ಶತಮಾನದ ರಾಮಾನುಜಾಚಾರ್ಯರು ಇದರ ಪ್ರಸಕ್ತಿ ಯನ್ನೇ ಎತ್ತದಿರುವುದರಿಂದ, ಆ ಎರಡು ಶತಮಾನಗಳ ಮಧ್ಯಕಾಲದಲ್ಲೆಲ್ಲೊ ಇದು ರಚಿತವಾಗಿರಬೇಕೆಂದೂ, ಇದನ್ನು ಬರೆದವನು ದ್ರಾವಿಡ ದೇಶದ ಕವಿಯೊಬ್ಬನೆಂದೂ ಪಾಶ್ಚಾತ್ಯ ಪಂಡಿತರ ಅಭಿಪ್ರಾಯ. ಆದರೆ ಪೋತನ ಕವಿಯ ತೆಲುಗು ಭಾಗವತಕ್ಕೆ ಮುನ್ನುಡಿಯನ್ನು ಬರೆಯುತ್ತಾ ಅದರ ಸಂಪಾದಕರು ಕ್ರಿ.ಶ. ಎರಡು ಅಥವಾ ಮೂರನೆಯ ಶತಮಾನದಲ್ಲಿ ಈ ಗ್ರಂಥರಚನೆಯಾಗಿರಬೇಕೆಂದು ನಿರ್ಧರಿಸಿರುತ್ತಾರೆ. ಅವರ ವಾದ ವೈಖರಿ ಹೀಗಿದೆ: ಭರತ ಖಂಡದಲ್ಲಿ ಸಂಕರ್ಷಣ-ವಾಸುದೇವ ತತ್ವ ಕ್ರಿ.ಪೂ. ಆರನೆಯ ಶತಮಾನದಲ್ಲಿ ಪ್ರಾರಂಭವಾಯಿತು. ಕ್ರಿ.ಪೂ. ಎರಡನೆಯ ಶತಮಾನದಲ್ಲಿ ಹುಟ್ಟಿದ ‘ಶಿಲಪ್ಪದಿಕಾರಂ’ನಲ್ಲಿ ವಾಸುದೇವ- ಸಂಕರ್ಷಣ ತತ್ವದ ಪ್ರಸಕ್ತಿ ಕಂಡುಬರುತ್ತದೆ. ಆದರೆ ಪಾಣಿನಿ ಪತಂಜಲಿಗಳಲ್ಲಿ ವಾಸುದೇವೋಪಾಸನೆ ಮಾತ್ರ ವ್ಯಕ್ತವಾಗುತ್ತದೆ. ಭಾಗವತದಲ್ಲಿ ಬಲರಾಮ-ವಾಸುದೇವರನ್ನು ಭಗವನ್ಮೂರ್ತಿಗಳಾಗಿ ಭಾವಿಸಿರುವರಾದರೂ ಸಂಕರ್ಷಣ- ವಾಸುದೇವ ತತ್ವವನ್ನು ಯಥಾವತ್ತಾಗಿ ಹೇಳಿಲ್ಲ. ಆದ್ದರಿಂದ ವಾಸುದೇವ-ಸಂಕರ್ಷಣ ತತ್ವದಿಂದ ಸಂಕರ್ಷಣದ ವಿಸರ್ಜನೆಯಾಗಿ, ವಾಸುದೇವನನ್ನು ನಾರಾಯಣ ನಿರ್ವಿಶೇಷ ನಾಗಿ ಭಾವಿಸುವುದು ಬಳಕೆಗೆ ಬಂದ ಮೇಲೆ ಭಾಗವತ ರಚನೆಯಾಗಿರಬೇಕು. ಆದರೆ ಕ್ರಿ. ಪೂ. ಎರಡನೆಯ ಶತಮಾನಕ್ಕಿಂತ ಈಚಿನದು, ಅದು. ‘ಶಿಪ್ಪದಿಕಾರಂ’ನಲ್ಲಿ ವಿಷ್ಣು ಪುರಾಣದ ಹೆಸರು ಕೇಳಿಬರುತ್ತದೆ. ಆ ಪುರಾಣದ ಕಥೆಯನ್ನು ವಿಸ್ತರಿಸಿ ಬರೆದಿರುವ ಭಾಗವತ ಅದಕ್ಕೂ ಈಚಿನದು. ಎಂದರೆ, ಇದು ಸ್ವಲ್ಪ ಹೆಚ್ಚು ಕಡಿಮೆ ಗುಪ್ತರಾಜರ ಕಾಲದ್ದು. ಆ ರಾಜರು ವಿಷ್ಣುವಿನ ಮೂರನೆಯ ಅವತಾರವಾದ ವರಾಹವನ್ನು ಪೂಜಿಸುತ್ತಾ ‘ಪರಮಭಾಗವತ’ರೆಂದು ತಮ್ಮನ್ನು ಕರೆದುಕೊಳ್ಳುತ್ತಿದ್ದುದಾಗಿ ಇತಿಹಾಸದಿಂದ ತಿಳಿದು ಬರುತ್ತದೆ. ಅದಕ್ಕೆ ತಕ್ಕಂತೆ ಭಾಗವತದಲ್ಲಿ ವರಹಾವತಾರದ ಕಥೆ ವಿಸ್ತಾರವಾಗಿ ಪ್ರತಿ ಪಾದಿತವಾಗಿದೆ. ಗುಪ್ತರಾಜರು ಸ್ಮಾರ್ತರಾದರೂ ಭಾಗವತ ಸಿದ್ಧಾಂತಕ್ಕೆ ಮನಸೋತವ ರಾಗಿ, ಭಾಗವತವನ್ನು ಪವಿತ್ರಗ್ರಂಥವಾಗಿ ಪರಿಗಣಿಸಿರಬೇಕು. ಆಗ ಹುಟ್ಟಿದುದು ಭಾಗವತ.

ಇನ್ನು ಭಾಗವತ ಕರ್ತೃವಿನ ವಿಚಾರ. ಇದು ವ್ಯಾಸಕೃತ. ಶುಕಮುನಿಕೃತ ಎಂಬುದನ್ನು ತಳ್ಳಿಹಾಕಿ ಬೋಪದೇವನೆಂಬ ಆಸ್ಥಾನ ಪಂಡಿತನೊಬ್ಬನು ತನ್ನ ಆಶ್ರಯದಾತನಾದ ಹೇಮಾದ್ರಿಯೆಂಬ ಗೋಪಾಲಮತಾವಲಂಬಿಗಾಗಿ ಇದನ್ನು ಬರೆದನೆಂದು ಹೇಳುವ ಒಂದು ಪಂಥವೂ ಇದೆ. ‘ಶ್ರೀಮದ್ಭಾಗವತೋಧ್ಯಾಯಸಾರೋ ಹ್ಯತ್ರ ನಿರೂಪ್ಯತೇ । ವಿದುಷಾ ಬೋಪದೇವೇನ ಮಂತ್ರಿ ಹೇಮಾದ್ರಿತುಷ್ಟಯೇ’ ಎಂಬುದು ಆ ಬೋಪದೇವನ ಭಾಗವತದ ಮೊದಲ ಶ್ಲೋಕವಂತೆ! ಇದನ್ನು ಖಂಡಿಸುತ್ತಾ ನನ್ನ ಗೆಳೆಯರೊಬ್ಬರು ‘ವ್ಯಾಸಕೃತ ಬ್ರಹ್ಮಸೂತ್ರಗಳನ್ನೆಲ್ಲ ಭಾಗವತದಲ್ಲಿ ಹೆಣೆದಿದೆ: ಬ್ರಹ್ಮಸೂತ್ರಭಾಷ್ಯ ಕೂಡ ಅರ್ಥವಾಗದವರು ಭಾಗವತವನ್ನು ಅಮೂಲಾಗ್ರವಾಗಿ ಓದಿ, ಆಮೇಲೆ ಬ್ರಹ್ಮಸೂತ್ರ ವನ್ನು ಓದಲಿ; ನಿಸ್ಸಂದಿಗ್ಧ‡ವಾಗಿ ಅರ್ಥಮಾಡಿಕೊಳ್ಳುತ್ತಾರೆ’ ಎಂದು ಹೇಳಿ, ಭಾಗವತವು ನಿಸ್ಸಂದೇಹವಾಗಿ ವ್ಯಾಸಕೃತ. ಆದರೆ ಅವರು ಕೃಷ್ಣದ್ವೈಪಾಯನ ವ್ಯಾಸರಲ್ಲ; ಅವರ ನಂತರದ ಪರಂಪರೆಯಲ್ಲಿ ‘ವ್ಯಾಸ’ ಪದವಿಗೇರಿದ ಮಹಾನುಭಾವರೊಬ್ಪರು, ಎಂದರು. ನನಗೆ ಈ ವಾದ ಒಪ್ಪಿಗೆಯಾಗಿದೆ.

ಶ್ರೀಮದ್ಭಾಗವತದ ಕರ್ತೃ, ಕಾಲಗಳನ್ನು ಕುರಿತು ಸಾಹಿತ್ಯೇತಿಹಾಸಜ್ಞರು ನೀಡಿರುವ ಅಭಿಪ್ರಾಯಗಳನ್ನೆಲ್ಲ ಕ್ರೋಡೀಕರಿಸಿ ಕೊಡುವ ಕಾರ್ಯಕ್ಕೆ ನಾನು ಇಲ್ಲಿ ಪ್ರಯತ್ನಿಸುವು ದಿಲ್ಲ. ಗುರುದೇವ ಶ್ರೀರಾಮಕೃಷ್ಣ ಪರಮಹಂಸರು ತಮ್ಮ ದೃಷ್ಟಾಂತ ಕಥೆಯೊಂದ ರಲ್ಲಿ–‘ಮಾವಿನ ಮರದ ಕೆಳಗೆ ನಿಂತು ಇದನ್ನು ನೆಟ್ಟವರಾರೆಂದು, ಎಂದು ನೆಟ್ಟರೆಂದು ತಲೆಯನ್ನೇಕೆ ಕೆದಕಿಕೊಳ್ಳುತ್ತಿ? ಹಣ್ಣನ್ನು ಸವಿದು ಸಂತೋಷಪಡಬಾರದೆ?’–ಎಂಬ ಅರ್ಥ ಬರುವಂತೆ ಹೇಳುತ್ತಾರೆ. ಭಾಗವತದ ವಿಚಾರದಲ್ಲೂ ಅಷ್ಟೆ. ಲೋಕೋತ್ತರವಾದ, ‘ಅಳಿದರುಂ ಉಳಿದರುಂ ಬಟ್ಟೆದೋರಿಪ’ ಈ ದಿವ್ಯಕೃತಿಯನ್ನು ಓದಿ ಸಂತೋಷಗೊಂಡರೆ ಸಾಕು. ಇದರಲ್ಲಿ ಮಾನವನ ಉದ್ಧಾರಕ್ಕೆ ಅಗತ್ಯವಾದುದೆಲ್ಲವೂ ಇದೆ. ‘ತಸ್ಮಿನ್ ವಿಜ್ಞಾತೇ, ಸರ್ವಂ ವಿಜ್ಞಾತಂ ಭವತಿ’ ಎಂಬ ಮಾತು ವೇದಗಳಂತೆ ಭಾಗವತಕ್ಕೂ ಅನ್ವಯಿಸುತ್ತದೆ. ವೇದದಲ್ಲಿ ತಾತ್ವಿಕ ಜ್ಞಾನಕ್ಕಾಗಿ ಕುಳಿತಿರುವುದು ಭಾಗವತ ಪುರಾಣದಲ್ಲಿ ಪ್ರಾಯೋಗಿಕ ವಿಜ್ಞಾನವಾಗಿ ಕಾಣಿಸಿಕೊಂಡಿದೆ. ಈ ಕಾರಣದಿಂದಲೆ ಪುರಾಣಜ್ಞಾನವಿಲ್ಲದ ವೇದ ಪಾರಂಗತತೆ ವ್ಯರ್ಥವೆಂದು ಹೇಳುವುದು. ಭಗವಂತನ ಪೂಜೆ ಪ್ರಾರ್ಥನಾ ವಿಧಾನಗಳೂ, ಆಚಾರ ಸಂಪ್ರದಾಯಗಳೂ, ಯಾತ್ರೆ, ತೀರ್ಥಸ್ಥಳಗಳ ವರ್ಣನೆಗಳೂ, ತತ್ವ ವೇದಾಂತ ವಿಷಯಗಳೂ, ಮೋಕ್ಷಪ್ರಾಪ್ತಿಯ ಸಾಧನಗಳೂ ಇಲ್ಲಿವೆ; ಐತಿಹಾಸಿಕ ಭೌಗೋಳಿಕ ವಿಷಯಗಳೂ, ಖಗೋಳ ಜ್ಯೋತಿಶಾಸ್ತ್ರದ ವಿಷಯಗಳೂ ಇಲ್ಲಿವೆ; ಮನಸ್ಸನ್ನು ಸೂರೆ ಗೊಳ್ಳಬಲ್ಲ ಅನೇಕ ಅವತಾರ ಕಥೆಗಳೂ ಸಂತರ ಕಥೆಗಳೂ ಇಲ್ಲಿವೆ. ಮುಮುಕ್ಷುಗಳಿಗೆ, ಲೌಕಿಕ ಜ್ಞಾನಾಪೇಕ್ಷಿಗಳಿಗೆ, ಕಾವ್ಯಾರಸಾಸ್ವಾದನಪಟುಗಳಿಗೆ, ಕೇವಲ ಕಥನ ಕುತೂಹಲಿ ಗಳಿಗೆ ಕೂಡ, ಇಲ್ಲಿ ಸಾಕಷ್ಟು ಸಾಧನವಿದೆ; ಮನುಷ್ಯನ ಜೀವನ ಪಿಪಾಸೆಯನ್ನು ತೀರಿಸುವಷ್ಟು ಬಹುವಿಧವಾದ ಜ್ಞಾನಸಂಪತ್ತು ಇದರಲ್ಲಿ ಅಡಕವಾಗಿದೆ. ನೀತಿ, ಧರ್ಮ ಗಳಿಗೆ ಗಣಿಯಂತಿರುವ, ಆಸ್ತಿಕತೆಯ ಜೀವಾತ್ಮನಂತಿರುವ ಈ ಭಾಗವತವು ಮಾನವನ ಕರ್ತ ವ್ಯಾಕರ್ತವ್ಯಗಳನ್ನು ಕಥೋಪಾಖ್ಯಾನಗಳ ಮೂಲಕ ಬೋಧಿಸಿ, ಲಾಕ್ಷಣಿಕರು ಹೆಸರಿಸುವ ‘ಮಿತ್ರಸಮ್ಮಿತ’ ಎಂಬ ಮಾತನ್ನು ಸಾರ್ಥಗೊಳಿಸಿಕೊಂಡಿದೆ. ಆದ್ದರಿಂದ ಇದನ್ನು ಓದಿ ಬಾಳನ್ನು ಹಸನು ಮಾಡಿಕೊಂಡರೆ ಸಾಕು.

ಭಾಗವತವು ವಿಷ್ಣುಪಾರಮ್ಯವನ್ನು ಎತ್ತಿ ಹಿಡಿಯುವ ವೈಷ್ಣವ ಪುರಾಣವಾದರೂ ಇದರಲ್ಲಿ ಇತರ ದೇವತೆಗಳ ಪ್ರಸಕ್ತಿ ಬರದೆ ಹೋಗಿಲ್ಲ. ಹಾಗೆ ಬಂದಾಗ ಆ ದೇವತೆಗಳ ಉಪೇಕ್ಷೆಯಾಗಲಿ, ಅವಹೇಳನವಾಗಲಿ ಎಲ್ಲಿಯೂ ಕಂಡುಬರುವುದಿಲ್ಲ. ಅಷ್ಟೇ ಅಲ್ಲ, ಎಲ್ಲ ದೇವತೆಗಳೂ ಸರ್ವೇಶ್ವರನ ಬೇರೆ ಬೇರೆಯ ಅಂಗಗಳೆಂಬ–ರೂಪುಗಳೆಂಬ– ಉದಾತ್ತದೃಷ್ಟಿ ಇಲ್ಲಿ ಗೋಚರವಾಗುತ್ತದೆ. ಒಮ್ಮೊಮ್ಮೆ ವಿಷ್ಣುವಲ್ಲದ ದೇವತೆಗೂ ಸರ್ವೇಶ್ವರತ್ವ ಸಲ್ಲುತ್ತದೆ. ಈ ದೃಷ್ಟಿಯಿಂದ ಸಂಸ್ಕೃತದ ಭಕ್ತಿಯ ಗ್ರಂಥಗಳಲ್ಲಿ ಭಾಗವತವು ಶಿಖರಪ್ರಾಯವಾದುದಾಗಿದೆ.

‘ವಚನಭಾಗವತ’ವು ಶ್ರೀಸಾಮಾನ್ಯನ ಸ್ವತ್ತಾಗಬೇಕೆಂಬುದು ನನ್ನ ಹೆಗ್ಗುರಿ. ಆದುದ ರಿಂದ ನಾನು ಮೂಲವನ್ನು ಭಾಷಾಂತರಿಸುವ ಕಾರ್ಯಕ್ಕೆ ಕೈಹಾಕದೆ, ಅಲ್ಲಿ ಬರುವ ವಿಷಯಗಳನ್ನು ಆದಷ್ಟು ಸಂಗ್ರಹವಾಗಿ, ಸಾರವತ್ತಾಗಿ, ಸರಳವಾದ ತಿಳಿಗನ್ನಡದಲ್ಲಿ ಬರೆಯಲು ಪ್ರಯತ್ನಿಸಿದ್ದೇನೆ. ‘ಶ್ರೀಮದ್ಬಾಗವತ’ದ ಜೀವಜೀವಾಳದಂತಿರುವ ತಾತ್ವಿಕ ವಿಷಯ ಗಳನ್ನೂ ಭಕ್ತಿಭಾವಗಳನ್ನೂ ಮೂಲಕ್ಕೆ ಅಪಚಾರವಾಗದಂತೆ ಎಚ್ಚರಿಕೆಯಿಂದ ಸಂಗ್ರಹಿಸಿದ್ದೇನೆ. ಹಾಗೆ ಮಾಡುವಾಗ ಮೂಲದಲ್ಲಿ ಪುನರಾವರ್ತನೆಯಾಗಿರುವ ಭಾಗಗಳನ್ನು ಕೈ ಬಿಟ್ಟಿದ್ದೇನೆ. ಮೂಲದಲ್ಲಿ ಬಿಡಿಬಿಡಿಯಾಗಿರುವಂತೆ ಕಾಣುವ ಕಥಾಭಾಗಗಳಿಗೆ ಒಂದು ಸಂಬಂಧ ಸೂತ್ರವನ್ನು ಕಲ್ಪಿಸಹೊರಟು, ಕಥೆಯ ಓಟ ಸಾಕಷ್ಟು ಸರಾಗವಾಗುವಂತೆ ಮಾಡಲು ಪ್ರಯತ್ನಮಾಡಿದ್ದೇನೆ. ಮುಖ್ಯ ಕಥೆಗೆ ನೇರವಾಗಿ ಸಂಬಂಧವಿಲ್ಲದ, ಆದರೂ ಜನರು ತಿಳಿಯಬಯಸಬಹುದಾದ ಧಾರ್ಮಿಕ ಮತ್ತು ಉಪಯುಕ್ತ ವಿಷಯಗಳನ್ನು ಪರಿಶಿಷ್ಟವಾಗಿ ಕೊಟ್ಟಿದ್ದೇನೆ.

‘ವಚನಭಾಗವತ’ವನ್ನು ನಾನು ಬರೆಯಹೊರಟುದು ಈಗ ಹತ್ತು ವರ್ಷಗಳ ಕೆಳಗೆ, ಬೆಂಗಳೂರಿನ ಶ್ರೀರಾಮಕೃಷ್ಣಾಶ್ರಮದಲ್ಲಿ. ಅಲ್ಲಿ ಸುಮಾರು ಒಂದು ತಿಂಗಳ ಕಾಲ, ದಿವಂಗತ ಸ್ವಾಮಿ ಯತೀಶ್ವರಾನಂದರ ಸಾನ್ನಿಧ್ಯದಲ್ಲಿ ನಿಲ್ಲುವ ಸೌಭಾಗ್ಯ ನನ್ನದಾಗಿತ್ತು. ಆ ನನ್ನ ಗುರುದೇವರ ಪ್ರೇರಣೆಯಿಂದ ನಾನಂದು ಭಾಗವತದ ದಶಮಸ್ಕಂಧವನ್ನು ಬರೆಯಲು ಆರಂಭಿಸಿದೆ. ಅಂದಿನ ಆ ಸನ್ನಿವೇಶವನ್ನು ನೆನೆಯುತ್ತಲೆ ಸುಂದರವಾದ ಸುಖ ಸ್ವಪ್ನವೊಂದು ಕಣ್ಣ ಮುಂದೆ ಕಟ್ಟಿ ನಿಂತಂತಾಗುತ್ತದೆ. ಆಶ್ರಮದಲ್ಲಿ ಆಗ ಬ್ರಹ್ಮಚಾರಿ ಗಳಾಗಿದ್ದ ಹರಿ ಮಹಾರಾಜರು ದಿನವೂ ಬೆಳಗು ಮುಂಜಾನೆ ನನ್ನನ್ನು ಎಬ್ಬಿಸಿ, ಕಾಫಿ ಮಾಡಿಕೊಟ್ಟು, ‘ವಾಕಿಂಗ್​’ ಕರೆದುಕೊಂಡು ಹೋಗಿ ಬಂದು ‘ಇನ್ನು ಸಮಾಧಾನಚಿತ್ತ ದಿಂದ ಭಾಗವತ ಬರೆಯಿರಿ’ ಎಂದು ತಮ್ಮ ನಗುವಿನ ಬುಗ್ಗೆಯನ್ನ ತುಳುಕಿಸುತ್ತಾ ಹೇಳಿ ಹೋಗುತ್ತಿದ್ದರು. ಪೂಜ್ಯ ಶ್ರೀ ಯತೀಶ್ವರಾನಂದರು ದಿನಚರಿಯಂತೆ ಆಶ್ರಮವನ್ನು ಒಮ್ಮೆ ಸುತ್ತಿ ಬರುತ್ತಾ, ಅತಿಥಿ ಮಂದಿರದಲ್ಲಿ ಇಣಿಕಿಹಾಕಿ, ಮುಗಳ್​ನಗುತ್ತಾ ‘ಭಾಗವತದ ಬರವಣಿಗೆ ಸಾಗುತ್ತಿದೆಯೊ?’ ಎಂದು ಕೇಳಿ, ಮುಂದಕ್ಕೆ ಹೋಗುತ್ತಿದ್ದರು. ಸ್ವಾಮಿ ಶ್ರೀ ಶಾಸ್ತ್ರಾನಂದರು ಆಯಾ ದಿನ ನಾನು ಬರೆದಷ್ಟು ಭಾಗವನ್ನು, ತಮ್ಮ ಮೇಲ್ವಿಚಾರಣೆಗೊಳಪಟ್ಟ ಬಾಲಕವೃಂದಕ್ಕೆ ಆಯಾ ದಿನದ ಸಂಜೆ ಕಥೆಯ ರೂಪದಲ್ಲಿ ಹೇಳಬೇಕೆಂದು ವಿಧಿಸಿದ್ದರು. ಪುಷ್ಯಾಶ್ರಮದ ಪ್ರಶಾಂತ ವಾತಾವರಣ, ಪೂಜ್ಯ ಶ್ರೀ ಯತೀಶ್ವರಾನಂದರ ಸಾನ್ನಿಧ್ಯ, ಆಶ್ರಮವಾಸಿಗಳೆಲ್ಲರ ಅವ್ಯಾಜಪ್ರೇಮ, ಅಗಾಧವಾದ ಪ್ರೋತ್ಸಾಹ–ಈ ಹೂಬಿಸಿಲಲ್ಲಿ ಮಿಂದು ಪುನೀತನಾಗಿದ್ದ ನನ್ನ ಚೇತನ ಉತ್ಸಾಹ ಮೂರ್ತಿಯಾಗಿತ್ತು; ಭಾಗವತದ ದಶಮಸ್ಕಂಧದಲ್ಲಿ ಮುಕ್ಕಾಲುಪಾಲಿನಷ್ಟು ಬರಹ ನಿರರ್ಗಳವಾಗಿ ಸಾಗಿ ಹೋಯಿತು. ಆಮೇಲೆ ನಾನು ಆಶ್ರಮದಿಂದ ಮನೆಗೆ ಹಿಂದಿರುಗಿದೆ, ನಾನು ಆರಂಭಿಸಿದ ಕಾರ್ಯ ಅಲ್ಲಿಗೆ ನಿಂತುಹೋಯಿತು. ಇಂದು ಗುರುಕರುಣೆಯಿಂದ ಅದನ್ನು ಮುಗಿಸಿ ಹೊರತರುವ ಮುನ್ನ ಶ್ರೀರಾಮಕೃಷ್ಣಾಶ್ರಮದ ಆ ಸಂತ ತ್ರಯರನ್ನು ಭಕ್ತಿಯಿಂದ ಮನಸಾ ನಮಸ್ಕರಿಸುತ್ತೇನೆ.

\begin{flushright}
\textbf{ತ. ಸು. ಶಾಮರಾಯ}
\end{flushright}

\chapter*{ಶ್ರೀರಾಮಕೃಷ್ಣರ ದಿವ್ಯಾನುಭವ}

ಒಂದಾನೊಂದು ದಿನ ವಿಷ್ಣು ದೇವಾಲಯದ ಮುಂದೆ ಒಬ್ಬರು ಭಾಗವತವನ್ನು ಓದು ತ್ತಿದ್ದರು. ಶ್ರೀರಾಮಕೃಷ್ಣರು ಅದನ್ನು ಕೇಳುತ್ತಿದ್ದರು. ಆಗ ಇದ್ದಕ್ಕಿದ್ದಂತೆ ಅವರು ಭಾವಾವಿಷ್ಟರಾದರು. ಆ ಭಾವದಲ್ಲಿ ಅವರಿಗೆ ದೇದೀಪ್ಯಮಾನವಾಗಿ ಬೆಳಗುತ್ತಿರುವ ಶ್ರೀಕೃಷ್ಣನ ದರ್ಶನವಾಯಿತು. ಆಗ ಶ್ರೀಕೃಷ್ಣನ ಪಾದಪದ್ಮಗಳಿಂದ ಬೆಳಕಿನ ಕಿರಣ ವೊಂದು ಹೊರಹೊಮ್ಮಿತು. ಅದು ಹರಿದುಬಂದು ಶ್ರೀಮದ್ಭಾಗವತವನ್ನು ಮುಟ್ಟಿತು. ಮುಟ್ಟಿ ಅಷ್ಟಕ್ಕೇ ನಿಲ್ಲಲಿಲ್ಲ. ಅಲ್ಲಿಂದಲೂ ಹರಿದುಬಂದು ಶ್ರೀರಾಮಕೃಷ್ಣರ ಹೃದಯ ವನ್ನು ಮುಟ್ಟಿತು. ಹೀಗೆಯೇ ಕೆಲಹೊತ್ತಿನವರೆಗೆ ಆ ಜ್ಯೋತಿಕಿರಣ ಶ್ರೀಮದ್ಭಾಗವತ ವನ್ನು, ಶ್ರೀರಾಮಕೃಷ್ಣರ ಹೃದಯವನ್ನು ಮತ್ತು ಶ್ರೀಕೃಷ್ಣನ ಪಾದಗಳನ್ನು ತ್ರೀಕೋಣಾ ಕೃತಿಯಲ್ಲಿ ಮುಟ್ಟಿಕೊಂಡೇ ಇತ್ತು. ಈ ದಿವ್ಯದರ್ಶನದಿಂದ ಶ್ರೀರಾಮಕೃಷ್ಣರಿಗೆ ಒಂದು ವಿಷಯ ಸ್ವಷ್ಟವಾಗಿ ಮನವರಿಕೆಯಾಯಿತು. ಏನೆಂದರೆ ಶ್ರೀಮದ್ಭಾಗವತ, ಭಕ್ತ, ಹಾಗೂ ಭಗವಂತ–ಈ ಮೂರು ಅಂಶಗಳು ಹೊರನೋಟಕ್ಕೆ ಬೇರೆಬೇರೆಯಾಗಿ ಕಂಡರೂ, ಶ್ರೀರಾಮಕೃಷ್ಣರು ಹೇಳುತ್ತಾರೆ, ಅವು ಮೂರೂ ಒಂದೇ. ಇನ್ನೊಂದು ರೀತಿಯಿಂದ ಹೇಳುವುದಾದರೆ ಒಂದೇ ದಿವ್ಯ ಸತ್ಯದ ಮೂರು ಅಭಿವ್ಯಕ್ತಿಗಳು ಅವುಗಳು. ಶ್ರೀರಾಮ ಕೃಷ್ಣರು ಹೇಳುತ್ತಿದ್ದರು: “ಭಾಗವತ, ಭಕ್ತ, ಭಗವಾನ್​–ಈ ಮೂರೂ ಒಂದೇ ಮತ್ತು ಆ ಒಂದು ವಸ್ತುವೇ ಮೂರಾಗಿ ವ್ಯಕ್ತವಾಗಿವೆ.”

ಭಾಗವತವೆಂದರೆ ಅದು ಜ್ಞಾನವೆಂಬ ತುಪ್ಪದಲ್ಲಿ ಕರಿದು ಭಕ್ತಿಯೆಂಬ ಶರ್ಕರ ಪಾಕದಲ್ಲಿ ಅದ್ದಿದ ಸಿಹಿತಿಂಡಿಯಂತೆ.

\begin{verse}
ಭವಭಯಮಪಹಂತುಂ ಜ್ಞಾನವಿಜ್ಞಾನಸಾರಂ\\ನಿಗಮಕೃದುಪಜಹ್ರೇ ಭೃಂಗವದ್ವೇಸಾರಮ್ ।\\ಅಮೃತಮುದಧಿತಶ್ಚಾತ್ಪಾಯಯದ್ಭೃತ್ಯವರ್ಗಾನ್​\\ಪುರುಷಮೃಷಭಮಾದ್ಯಂ ಕೃಷ್ಣಸಂಜ್ಞಂ ನತೋಽಸ್ಮಿ ॥
\end{verse}

ವೇದಪುರುಷನಾದ ಭಗವಂತನು ಸಂಸಾರ ಭಯವನ್ನು ಕಳೆಯುವುದಕ್ಕಾಗಿ, ಶಾಸ್ತ್ರಜ್ಞಾನ ಮತ್ತು ಅಪರೋಕ್ಷಜ್ಞಾನಗಳ ಸಾರವನ್ನು ಭ್ರಮರದಂತೆ ಸಂಗ್ರಹಿಸಿ, ಅದರ ಒಂದು ಅಮೃತವನ್ನೂ, ಕ್ಷೀರಸಮುದ್ರಮಥನದಿಂದ ಮತ್ತೊಂದು ಅಮೃತವನ್ನೂ ತೆಗೆದು ಭಕ್ತ ಸಮೂಹಕ್ಕೆ ಕುಡಿಸಿದನು. ಅಂತಹ ಪುರಾಣ ಪುರುಷಶ್ರೇಷ್ಠನಾಗಿ, ಶ್ರೀಕೃಷ್ಣನೆಂಬ ಹೆಸರನ್ನು ಹೊಂದಿದ ಪರಮಾತ್ಮನನ್ನು ನಮಸ್ಕರಿಸುತ್ತೇನೆ.

\chapter*{ಶ್ರೀಕೃಷ್ಣ ಸ್ತವ}

\begin{verse}
ನಮಸ್ಯೇ ಪುರುಷಂ ತ್ವಾದ್ಯಮೀಶ್ವರಂ ಪ್ರಕೃತೇಃ ಪರಮ್ ।\\ಅಲಕ್ಷ್ಯಂ ಸರ್ವಭೂತಾನಾಮಂತರ್ಬಹಿರವಸ್ಥಿತಮ್ ॥
\end{verse}

ಆದ್ಯನೂ, ಈಶ್ವರನೂ, ಪ್ರಕೃತಿಗಿಂತ ಶ್ರೇಷ್ಠನೂ, ಆಗೋಚರನೂ, ಸರ್ವಭೂತಗಳ ಒಳಗೂ ಹೊರಗೂ ಇರುವವನೂ ಆದ ಪುರುಷೋತ್ತಮನಿಗೆ ನಮಿಸುತ್ತೇನೆ.

\begin{verse}
ಕೃಷ್ಣಾಯ ವಾಸುದೇವಾಯ ದೇವಕೀನಂದನಾಯ ಚ ।\\ನಂದಗೋಪ ಕುಮಾರಾಯ ಗೋವಿಂದಾಯ ನಮೋನಮಃ ॥
\end{verse}

ಕೃಷ್ಣನಿಗೆ, ವಾಸುದೇವನಿಗೆ, ದೇವಕಿನಂದನನಿಗೆ, ನಂದಗೋಪಕುಮಾರನಿಗೆ, ಗೋವಿಂದನಿಗೆ ನಮೋನಮಃ.

\begin{verse}
ನಮಃ ಪಂಕಜನಾಭಾಯ ನಮಃ ಪಂಕಜಮಾಲಿನೇ ।\\ನಮಃ ಪಂಕಜನೇತ್ರಾಯ ನಮಸ್ತೇ ಪಂಕಜಾಂಘ್ರಯೇ ॥
\end{verse}

ಪದ್ಮನಾಭನಿಗೆ ನಮಸ್ಕಾರ, ಕಮಲಮಾಲೆಗಳನ್ನು ಧರಿಸುವ ನಿನಗೆ ನಮಸ್ಕಾರ, ಪಂಕಜ ನೇತ್ರನಿಗೆ ನಮಸ್ಕಾರ, ಪಂಕಜಾಂಘ್ರಿಗೆ ನಮಸ್ಕಾರ.

\begin{verse}
ನಮೋಽಕಿಂಚನವಿತ್ತಾಯ ನಿವೃತ್ತಗುಣವೃತ್ತಯೇ ।\\ಆತ್ಮಾರಾಮಾಯ ಶಾಂತಾಯ ಕೈವಲ್ಯಪತಯೇ ನಮಃ ॥
\end{verse}

ವಿರಕ್ತರನ್ನೇ ಧನವಾಗಿ ತಿಳಿದಿರುವವನೂ, ಗುಣವೃತ್ತಿಗಳಿಂದ ಬಿಡಲ್ಪಟ್ಟವನೂ, ಆತ್ಮಾರಾಮನೂ, ಶಾಂತನೂ, ಕೈವಲ್ಯಪತಿಯೂ ಆದ ನಿನಗೆ ನಮಸ್ಕಾರ.

\begin{verse}
ತ್ವಯಿ ಮೇಽನನ್ಯವಿಷಯಾ ಮತಿರ್ಮಧುಪತೇಽಸಕೃತ್ ।\\ರತಿಮುದ್ವಹತಾದದ್ಧಾ ಗಂಗೇವೌಘಮುದನ್ವತಿ ॥
\end{verse}

ಹೇ ಮಧುಪತಿ, ಗಂಗೆಯು ಸಮುದ್ರವನ್ನು ಸೇರುವಂತೆ ನನ್ನ ಮನಸ್ಸು ಬೇರೆ ವಿಷಯಗಳಲ್ಲಿ ಪ್ರವರ್ತಿಸದೆ ನಿನ್ನಲ್ಲಿಯೇ ಯಾವಾಗಲೂ ಪ್ರೀತಿಯನ್ನು ಪಡೆಯಲಿ.

\begin{flushright}
\textbf{ಭಾಗವತ}
\end{flushright}

\chapter*{ರಮಾಪತ್ಯಷ್ಟಕಮ್​}

\begin{verse}
ಜಗದಾದಿಮನಾದಿಮಜಂ ಪುರುಜಂ ಶರದಂಬರತುಲ್ಯತನುಂ ವಿತನುಮ್ ।\\ಧೃತಕಂಜರಥಾಂಗಗದಂ ವಿಗದಂ ಪ್ರಣಮಾಮಿ ರಮಾಧಿಪತಿಂ ತಮಹಮ \versenum{॥ ೧ ॥}
\end{verse}

\begin{verse}
ಕಮಲಾನನಕಂಜರತಂ ವಿರತಂ ಹೃದಿಯೋಗಿಜನೈಃ ಕಲಿತಂ ಲಲಿತಮ್ ।\\ಕುಜನೈಃ ಸುಜನೈರಲಭಂ ಸುಲಭಂ ಪ್ರಣಮಾಮಿ ರಮಾಧಿಪತಿಂ ತಮಹಮ್ \versenum{॥ ೨ ॥}
\end{verse}

\begin{verse}
ಮುನಿವೃಂದಹೃದಿಸ್ಥಪದಂ ಸುಪದಂ ನಿಖಿಲಾಧ್ವರಭಾಗಭುಜಂ ಸುಭುಜಮ್ ।\\ಹೃತವಾಸವಮುಖ್ಯಮದಂ ವಿಮದಂ ಪ್ರಣಮಾಮಿ ರಮಾಧಿಪತಿಂ ತಮಹಮ್​ \versenum{॥ ೩ ॥}
\end{verse}

\begin{verse}
ಹೃತದಾನವದೃಪ್ತಬಲಂ ಸುಬಲಂ ಸ್ವಜನಾಸ್ತಸಮಸ್ತಮಲಂ ವಿಮಲಮ್ ।\\ಸಮಪಾಸ್ತ ಗಜೇಂದ್ರದರಂ ಸುದರಂ ಪ್ರಣಮಾಮಿ ರಮಾಧಿಪತಿಂ ತಮಹಮ್​ \versenum{॥ ೪ ॥}
\end{verse}

\begin{verse}
ಪರಿಕಲ್ಪಿತಸರ್ವಕಲಂ ವಿಕಲಂ ಸಕಲಾಗಮಗೀತಗುಣಂ ವಿಗುಣಮ್ ।\\ಭವಪಾಶ ನಿರಾಕರಣಂ ಶರಣಂ ಪ್ರಣಮಾಮಿ ರಮಾಧಿಪತಿಂ ತಮಹಮ್ \versenum{॥ ೫ ॥}
\end{verse}

\begin{verse}
ಮೃತಿಜನ್ಮಜರಾಶಮನಂ ಕಮನಂ ಶರಣಾಗತಭೀತಹರಂ ದಹರಮ್ ।\\ಪರಿತುಷ್ಟರಮಾಹೃದಯಂ ಸುದಯಂ ಪ್ರಣಮಾಮಿ ರಮಾಧಿಪತಿಂ ತಮಹಮ್ \versenum{॥ ೬ ॥}
\end{verse}

\begin{verse}
ಸಕಲಾವನಿಬಿಂಬಧರಂ ಸ್ವಧರಂ ಪರಿಪೂರಿತಸರ್ವದಿಶಂ ಸುದೃಶಮ್ ।\\ಗತಶೋಕಮಶೋಕಕರಂ ಸುಕರಂ ಪ್ರಣಮಾಮಿ ರಮಾಧಿಪತಿಂ ತಮಹಮ್ \versenum{॥ ೭ ॥}
\end{verse}

\begin{verse}
ಮಥಿತಾರ್ಣವರಾಜರಸಂ ಸರಸಂ ಗ್ರಥಿತಾಖಿಲಲೋಕಹೃದಂ ಸುಹೃದಮ್ ।\\ಪ್ರಥಿತಾದ್ಭುತಶಕ್ತಿಗಣಂ ಸುಗಣಂ ಪ್ರಣಮಾಮಿ ರಮಾಧಿಪತಿಂ ತಮಹಮ್ \versenum{॥ ೮ ॥}
\end{verse}

\begin{verse}
ಸುಖರಾಶಿಕರಂ ಭವಬಂಧಹರಂ ಪರಮಾಷ್ಟಕಮೇತದನನ್ಯಮತಿಃ ।\\ಪಠತೀಹ ತು ಯೋಽನಿಶಮೇವ ನರೋ ಲಭತೇ ಖಲು ವಿಷ್ಣುಪದಂ ಸ ಪರಮ್ \versenum{॥ ೯ ॥}
\end{verse}

\begin{flushright}
\textbf{ಸ್ವಾಮಿ ಬ್ರಹ್ಮಾನಂದ}
\end{flushright}

