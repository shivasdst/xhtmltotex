
\chapter{೭೯. ಧರ್ಮರಾಯನ ರಾಜಸೂಯ}

ಜರಾಸಂಧನ ವಧೆಯಿಂದ ಧರ್ಮರಾಯನ ರಾಜಸೂಯಯಾಗಕ್ಕಿದ್ದ ಅಡ್ಡಿಆತಂಕ ಗಳೆಲ್ಲ ನಿವಾರಣೆಯಾದಂತಾಯಿತು. ಆದ್ದರಿಂದ ಧರ್ಮರಾಯನು ಶ್ರೀಕೃಷ್ಣನನ್ನು ಕುರಿತು “ಮಹಾತ್ಮಾ, ಮೂರು ಲೋಕಕ್ಕೂ ಗುರುಗಳಾದ ಸನಕ ಸನಂದಾದಿಗಳು ಕೂಡ ನಿನಗೆ ನಮಸ್ಕರಿಸುತ್ತಾರೆ, ಲೋಕಪಾಲಕರು ನಿನ್ನ ಆಜ್ಞೆಯನ್ನು ತಮ್ಮ ಭಾಗ್ಯವೆಂದು ತಲೆಯಲ್ಲಿ ಹೊತ್ತು ನಡೆಸುತ್ತಾರೆ. ಅಂತಹ ಸರ್ವೇಶ್ವರನಾದ ನೀನು, ನನ್ನಂತಹ ಅಲ್ಪನ ಮಾತನ್ನು ಚಾಚೂ ತಪ್ಪದೆ ನಡೆಸುತ್ತಿರುವೆಯಲ್ಲಾ! ಇದಕ್ಕೇನು ಹೇಳೋಣ? ಇದು ಕೇವಲ ನಿನ್ನ ಮಾಯಾನಟನೆ. ಸೂರ್ಯನು ಹುಟ್ಟಿ ಮುಳುಗಿದರೂ ಆತನ ತೇಜಸ್ಸು ಹೆಚ್ಚಾಯಿತೆಂದಾಗಲಿ, ಕಡಮೆಯಾಯಿತೆಂದಾಗಲಿ ಹೇಳುವುದಕ್ಕೆ ಸಾಧ್ಯವೆ? ಹಾಗೆಯೆ ನೀನು ಅರಸನಾಗಲಿ ಆಳಾ ಗಲಿ ನಿನ್ನ ಮಹತ್ತು ಹೆಚ್ಚುಕಡಮೆಯಾಗುವಂತಹುದಲ್ಲ. ಅಹಂಕಾರವಿದ್ದರಲ್ಲವೆ ಹೆಚ್ಚು ಕಡಮೆಯ ಪ್ರಶ್ನೆ? ಈ ಜಗತ್ತೆ ನೀನಾಗಿರುವಾಗ ‘ನೀನು, ತಾನು’ ಎಂಬುದು ನಿನಗೆಲ್ಲಿ ಯದು? ನೀನು ಸರ್ವಭೂತಸಮ. ರಾಜಸೂಯಯಾಗವನ್ನು ಆರಂಭಿಸಿ ಸಾಂಗವಾಗಿ ನಡೆಸುವಂತೆ ನನ್ನನ್ನು ಅನುಗ್ರಹಿಸು” ಎಂದು ಬೇಡಿಕೊಂಡನು. ಆತನ ಬಾಯಿಂದ ‘ಅಸ್ತು’ ಎಂದು ಹೇಳುತ್ತಲೆ ಧರ್ಮರಾಯನು ಭರದ್ವಾಜ, ವಸಿಷ್ಠ, ಗೌತಮ, ವಾಮ ದೇವ, ವಿಶ್ವಾಮಿತ್ರ, ಪರಾಶರ, ಅಗಸ್ತ್ಯ ಮೊದಲಾದ ಮಹರ್ಷಿಗಳನ್ನೆಲ್ಲ ಕರೆಸಿ, ಯಾಗ ಕಾರ್ಯವನ್ನು ಅವರಿಗೆ ವಹಿಸಿದನು. ಭೀಷ್ಮ, ದ್ರೋಣ, ಕೃಪ, ವಿದುರ, ಧೃತರಾಷ್ಟ್ರ, ದುರ್ಯೋಧನ ಮೊದಲಾದ ಬಂಧುಬಾಂಧವರೆಲ್ಲ ಹಸ್ತಿನಾವತಿಗೆ ಆಗಮಿಸಿದರು. ಅನೇಕ ರಾಜರೂ ಅವರ ಅಧಿಕಾರಿಗಳೂ, ಬ್ರಾಹ್ಮಣ, ಕ್ಷತ್ರಿಯ, ವೈಶ್ಯ, ಶೂದ್ರರೆಂಬ ನಾಲ್ಕು ವರ್ಣದವರೂ ಅಗಾಧವಾದ ಸಂಖ್ಯೆಯಲ್ಲಿ ಅಲ್ಲಿ ಬಂದು ನೆರೆದರು. ಯಾಗಭೂಮಿ ಯನ್ನು ಚಿನ್ನದ ನೇಗಿಲಿನಿಂದ ಉತ್ತು ಶುದ್ಧಿಮಾಡಿದರು. ಧರ್ಮರಾಜನು ಶಾಸ್ತ್ರೋಕ್ತವಾಗಿ ಯಾಗದೀಕ್ಷೆಯನ್ನು ವಹಿಸಿದನು. ದೇವಾನುದೇವತೆಗಳೆಲ್ಲರೂ ಶ್ರೀಕೃಷ್ಣನ ಅನುಗ್ರಹ ದಿಂದ ನಡೆದ ಆ ಯಾಗವನ್ನು ನೋಡುವುದಕ್ಕಾಗಿ ಯಾಗಮಂಟಪಕ್ಕಿಳಿದು ಬಂದರು.

ಧರ್ಮರಾಯನು ಅಜಾತಶತ್ರು. ಆತನ ಮೇಲಿನ ಪ್ರೇಮದಿಂದ ಆತನ ಬಂಧುಗಳೆ ಲ್ಲರೂ ಯಾಗಕಾರ್ಯದಲ್ಲಿ ಭಾಗವಹಿಸಿ, ತಮ್ಮ ಪಾಲಿನ ಸೇವೆಯನ್ನು ಸಲ್ಲಿಸಿದರು. ಭೀಮನು ಅಡುಗೆಮನೆಯ ಅಧ್ಯಕ್ಷ; ದುರ್ಯೋಧನನು ಹಣಕಾಸಿನ ಅಧ್ಯಕ್ಷ; ಸಹದೇವನು ಯಾಗಕ್ಕೆ ಬಂದವರನ್ನು ಸನ್ಮಾನ ಮಾಡುವ ಕಾರ್ಯವನ್ನು ವಹಿಸಿಕೊಂಡ; ನಕುಲನು ಯಾಗಕ್ಕೆ ಬೇಕಾದ ಸಾಮಗ್ರಿಗಳನ್ನು ಒದಗಿಸುತ್ತಿದ್ದ; ಅರ್ಜುನನು ಬಂದವರಿಗೆಲ್ಲ ಗಂಧ ಪುಷ್ಪ ತಾಂಬೂಲಗಳನ್ನು ಕೊಡುವುದರಲ್ಲಿ ನಿರತನಾಗಿದ್ದ; ಶ್ರೀಕೃಷ್ಣನು ಯಾಗಕ್ಕೆ ಬಂದ ಸಾಧುಗಳ ಪಾದಗಳನ್ನು ತೊಳೆಯುವ ಕಾರ್ಯದಲ್ಲಿ ತೊಡಗಿದ್ದ; ಧೃಷ್ಟದ್ಯುಮ್ನ ಬಂದವರಿಗೆಲ್ಲ ಭಕ್ಷ್ಯಭೋಜ್ಯಗಳನ್ನು ನೀಡುತ್ತಿದ್ದ; ಕರ್ಣನು ದಾನದಕ್ಷಿಣೆಗಳನ್ನು ಕೊಡು ವವನಾಗಿದ್ದ; ಹೀಗೆಯೇ ವಿದುರ, ಸಾತ್ಯಕಿ, ಕೃತವರ್ಮ ಇತ್ಯಾದಿ ಪ್ರಮುಖರು ಒಂದೊಂದು ಕೆಲಸವನ್ನು ಕೈಕೊಂಡು ಧರ್ಮರಾಯನಲ್ಲಿ ತಮಗಿರುವ ಗೌರವಗಳನ್ನು ಸೂಚಿಸುತ್ತಿದ್ದರು. ಯಾವ ಕೊರತೆಯೂ ಇಲ್ಲದೆ ಯಾಗವು ಸಾಂಗವಾಗಿ ನೆರವೇರಿತು. ಯಾಗದಲ್ಲಿ ಭಾಗವಹಿಸಿದ ಪುತ್ವಿಕ್ಕುಗಳನ್ನೂ ಸಭಾಧ್ಯಕ್ಷರನ್ನು ಪೂಜಿಸಿ ಗೌರವಿಸಿದುದು ಆಯಿತು. ಇನ್ನು ಉಳಿದುದು ‘ಅಗ್ರಪೂಜೆ.’ ಆ ಸಭೆಯಲ್ಲಿ ಅನೇಕ ಮಹಾನುಭಾವರು ನೆರೆದಿದ್ದುದರಿಂದ ಯಾರನ್ನು ಅಗ್ರಪೂಜೆಯಿಂದ ಗೌರವಿಸಬೇಕೆಂಬುದೇ ಒಂದು ದೊಡ್ಡ ಸಮಸ್ಯೆಯಾಗಿತ್ತು. ಎಲ್ಲರೂ ಮೌನದಿಂದಿರಲು ಸಹದೇವನು ಮುಂದೆ ಬಂದು ‘ಅಣ್ಣ, ಇದರಲ್ಲಿ ಯೋಚಿಸುತ್ತಾ ಕೂಡುವಂತಹುದೇನಿದೆ? ಚರಾಚರ ಜಗತ್ತಿಗೆಲ್ಲ ಆತ್ಮಸ್ವರೂಪಿ ಯಾಗಿರುವ ಶ್ರೀಕೃಷ್ಣಪರಮಾತ್ಮನೆ ಅಗ್ರಪೂಜೆಗೆ ಅರ್ಹ. ಆತನನ್ನು ಪೂಜಿಸಿದರೆ ಲೋಕದ ಸಮಸ್ತ ಜೀವಿಗಳನ್ನೂ ಪೂಜಿಸಿದಂತಾಯಿತು. ಪೂರ್ಣಕಾಮನಾದ ಆತನನ್ನು ಪೂಜಿಸುವುದರಿಂದ ಆ ಪೂಜೆಗೆ ಅಪಾರವಾದ ಫಲವುಂಟು. ಕರ್ಮಸಾರ್ಥಕನಾಗ ಬೇಕೆನ್ನುವವನು ಆತನನ್ನೆ ಪೂಜಿಸಬೇಕು’ ಎಂದನು.

ಸಹದೇವನ ಮಾತನ್ನು ಅಲ್ಲಿದ್ದ ಬ್ರಾಹ್ಮಣರೆಲ್ಲ ಅತ್ಯಂತ ಸಂತೋಷದಿಂದ ಸ್ವೀಕರಿ ಸಿದರು. ಒಡನೆಯೆ ಧರ್ಮರಾಯನು ಬಂಗಾರದ ತಟ್ಟೆಯಲ್ಲಿ ಶ್ರೀಕೃಷ್ಣನ ಪಾದಗಳನ್ನಿಟ್ಟು ತೊಳೆದು, ಆ ಪಾದತೀರ್ಥವನ್ನು ತನ್ನ ತಲೆಯಲ್ಲಿ ಧರಿಸಿದಮೇಲೆ ತನ್ನ ಮಡದಿ, ತಮ್ಮಂ ದಿರು, ಬಂಧುವರ್ಗದವರ ತಲೆಯಮೇಲೆಲ್ಲ ಅದನ್ನು ಪ್ರೋಕ್ಷಿಸಿದನು. ಅನಂತರ ದಿವ್ಯ ವಾದ ಪೀತಾಂಬರವನ್ನು ಆತನಿಗೆ ಹೊದಿಸಿ, ಅನೇಕ ಆಭರಣಗಳನ್ನು ಕಾಣಿಕೆಯಾಗಿ ಕೊಟ್ಟನು. ಆತನ ಪಾದಪೂಜೆಯನ್ನು ಕಂಡು ಅಲ್ಲಿ ನೆರೆದಿದ್ದವರೆಲ್ಲ ಶ್ರೀಕೃಷ್ಣನಿಗೆ ‘ನಮೋ, ನಮೋ’ ಎಂದು ವಂದಿಸುತ್ತಿದ್ದರು. ಈ ಪೂಜೆ, ಈ ಆಡಂಬರ, ಈ ವಂದನೆ– ಇವುಗಳನ್ನೆಲ್ಲ ನೋಡುತ್ತಾ ಅಲ್ಲಿಯೇ ರಾಜರ ಮಧ್ಯದಲ್ಲಿ ಕುಳಿತಿದ್ದ ಶಿಶುಪಾಲನಿಗೆ ಸಹನೆ ತಪ್ಪಿತು. ಅವನು ಪ್ರಳಯರುದ್ರನ ಕೋಪವನ್ನು ವಹಿಸಿ, ತನ್ನ ಪೀಠದಿಂದ ಸಿಡಿದೆದ್ದವನೆ ಎರಡೂ ಕೈಗಳನ್ನೂ ಮೇಲಕ್ಕೆತ್ತಿ ಗುಡುಗಿದನು–‘ಎಲಾ, ಇದೇನನ್ಯಾಯ! ಎಂತಹ ಕಾಲ ಬಂತು! ಎಳೆಯ ಹುಡುಗನೊಬ್ಬ ತಿಳಿಯದೆ ಹೇಳಿದ ಮಾತ್ರಕ್ಕೆ ವಯಸ್ಸಾದವರೆಲ್ಲರೂ ಆ ಮಾತಿಗೆ ತಲೆಬಾಗುವುದೆ? ಬ್ರಹ್ಮಜ್ಞರಾದ ಮಹರ್ಷಿಗಳು ಎಷ್ಟೋ ಜನ ಈ ಸಭೆಯಲ್ಲಿ ಕುಳಿತಿದ್ದಾರೆ. ದೇವತೆಗಳೂ ಕೂಡ ಅವರನ್ನು ಪೂಜಿಸುತ್ತಾರೆ. ಅಂತಹ ಮಹಾಮಹಿಮ ರನ್ನೆಲ್ಲ ಬಿಟ್ಟು, ಹಚ್ಚ ಹಸಿಯ ಗೊಲ್ಲನಾದ ಈ ಕೃಷ್ಣನಿಗೆ ಅಗ್ರಪೂಜೆಯೆ? ಇವನಿಗೆ ವರ್ಣಾಶ್ರಮಧರ್ಮಗಳುಂಟೆ, ನೀತಿನಿಯಮಗಳುಂಟೆ? ಇವನೊಬ್ಬ ಧರ್ಮಬಾಹಿರ. ಪುಣ್ಯಕರವಾದ ಆರ್ಯಾವರ್ತವನ್ನು ಬಿಟ್ಟು ನೀರಿನ ಮಧ್ಯದಲ್ಲಿ ತಲೆಮರೆಸಿಕೊಂಡಿದ್ದಾನೆ. ಆಗಾಗ ಕಳ್ಳನಂತೆ ಬಂದು ಪ್ರಜೆಗಳನ್ನು ಹಿಂಸಿಸುತ್ತಾನೆ. ಇವನಿಗೆ ಅಗ್ರಪೂಜೆ!’

ಶಿಶುಪಾಲನ ಕಿಡುನುಡಿಗಳನ್ನು ಕೇಳಿಯೂ ಶ್ರೀಕೃಷ್ಣನ ಮುಗುಳ್ನಗೆ ಬಾಡಲಿಲ್ಲ. ನರಿಯ ಕೂಗನ್ನು ಲಕ್ಷಿಸದ ಸಿಂಹದಂತಿದ್ದ, ಆತ. ಆದರೆ ಸಭೆಯಲ್ಲಿದ್ದ ಸಜ್ಜನರು ಆ ಕೃಷ್ಣನಿಂದೆಯನ್ನು ಕೇಳಲಾರದೆ ಅಲ್ಲಿಂದ ಎದ್ದುಹೋದರು. ಅಷ್ಟರಲ್ಲಿ ಪಾಂಡವಪಕ್ಷ ಪಾತಿಗಳಾದ ಕೆಲವು ರಾಜರು ಕೋಪವನ್ನು ತಡೆಯಲಾರದೆ, ಶಿಶುಪಾಲನನ್ನು ಕೊಂದುಬಿಡಬೇಕೆಂದು ಹಿರಿದ ಕತ್ತಿಗಳೊಡನೆ ಅವನತ್ತ ನುಗ್ಗಿದರು. ಇದನ್ನು ಕಂಡು ಶಿಶು ಪಾಲನೂ ಅವರೊಡನೆ ಯುದ್ಧಕ್ಕೆ ಸಿದ್ಧನಾದನು. ಹೀಗೆ ಯಾಗಭೂಮಿಯು ಯುದ್ಧ ಭೂಮಿಯಾಗುವುದನ್ನು ಕಂಡು, ಶ್ರೀಕೃಷ್ಣನು ತನ್ನ ಪೀಠದಿಂದ ಮೇಲಕ್ಕೆದ್ದನು. ಆತನು ಶಿಶುಪಾಲನೊಡನೆ ಯುದ್ಧಕ್ಕೆ ಸಿದ್ಧರಾಗಿದ್ದ ರಾಜರನ್ನೆಲ್ಲ ತಡೆದು, ತನ್ನ ಚಕ್ರಾಯುಧ ದಿಂದ ಶಿಶುಪಾಲನ ಕುತ್ತಿಗೆಯನ್ನು ಒಂದೆ ಪೆಟ್ಟಿಗೆ ಕತ್ತರಿಸಿಹಾಕಿದನು. ಒಡನೆಯೆ ಅವನ ದೇಹದಿಂದ ದಿವ್ಯವಾದ ಒಂದು ತೇಜಸ್ಸು ಹೊರಟು ಬಂದು ಶ್ರೀಕೃಷ್ಣನಲ್ಲಿ ಪ್ರವೇಶಿ ಸಿತು. ಶಿಶುಪಾಲವಧೆಯನ್ನು ಕಂಡು ಅಲ್ಲಿ ನೆರೆದಿದ್ದ ಅವನ ಗೆಳೆಯರೆಲ್ಲ ತಲೆಮರೆಸಿ ಕೊಂಡು ಅಲ್ಲಿಂದ ಹೊರಟುಹೋದರು. ಅನಂತರ ಧರ್ಮರಾಯನೊಡನೆ ಅಲ್ಲಿ ನೆರೆದಿ ದ್ದವರೆಲ್ಲ ಅವಭೃತಸ್ನಾನಕ್ಕೆಂದು ಗಂಗಾನದಿಗೆ ಹೊರಟರು. ದಾರಿಯಲ್ಲಿ ಗಂಡುಹೆಣ್ಣು ಗಳೆಲ್ಲ ಪರಸ್ಪರ ಓಕಳಿಯನ್ನು ಎರಚುತ್ತಾ ವಿನೋದದಿಂದ ನಲಿದರು. ಎಲ್ಲರೂ ಗಂಗೆ ಯಲ್ಲಿ ಮಿಂದು ಧನ್ಯರಾದರು. ಅಲ್ಲಿಂದ ಹಿಂದಿರುಗಿದಮೇಲೆ ಧರ್ಮರಾಯನು ರಾಜಸಭೆ ಯನ್ನು ನೆರಹಿ, ಆ ಯಾಗಕ್ಕೆ ಬಂದಿದ್ದವರಿಗೆಲ್ಲ ವಸ್ತ್ರಭೂಷಣಗಳನ್ನಿತ್ತು ಸತ್ಕರಿಸಿದನು. ಬ್ರಾಹ್ಮಣರೂ ದಾನದಕ್ಷಿಣೆಗಳಿಂದ ದಣಿದರು. ಎಲ್ಲರೂ ಧರ್ಮರಾಯನನ್ನು ಬಾಯ್ತುಂಬ ಹರಸಿ ಆತನಿಂದ ಬೀಳ್ಕೊಂಡು ಹಿಂದಿರುಗಿದರು. ಶ್ರೀಕೃಷ್ಣನೂ ಹಿಂದಿರುಗ ಬೇಕೆಂದುಕೊಂಡನು. ಆದರೆ ಧರ್ಮರಾಯನು ಅದಕ್ಕೆ ಒಪ್ಪಲಿಲ್ಲ. ಆದ್ದರಿಂದ ತನ್ನ ರಾಣಿ ವಾಸದವರೊಡನೆ ಆತನು ಅಲ್ಲಿಯೆ ಉಳಿದುಕೊಂಡನು.

ರಾಜಸೂಯಯಾಗಕ್ಕೆ ಬಂದಿದ್ದವರೆಲ್ಲರೂ ಅತ್ಯಂತ ಸಂತೋಷದಿಂದ ಹಿಂದಿರುಗಿ ದರು. ಆದರೆ ದುರ್ಯೋಧನನೊಬ್ಬನು ಮಾತ್ರ ಅತ್ಯಂತ ಕೋಪದಿಂದ ಕುದಿಯುತ್ತಾ ಹಿಂದಿರುಗಬೇಕಾಯಿತು. ಆತನಿಗೆ ಧರ್ಮರಾಯನ ಐಶ್ವರ್ಯ, ಯಾಗವೈಭವಗಳನ್ನು ಕಂಡು ಹೊಟ್ಟೆಹಿಚಿಕಿಕೊಳ್ಳುವಷ್ಟು ಸಂಕಟವಾಯಿತು. ದ್ರೌಪದಿಯ ಆ ಹಾವಭಾವ ವಿಲಾಸ ವೈಭವಗಳು, ಅವಳ ರಾಣೀವಾಸದ ಸೌಭಾಗ್ಯ, ಗಂಡಂದಿರಲ್ಲಿ ಅವಳು ತೋರುತ್ತಿದ್ದ ಪ್ರೇಮ–ಇವುಗಳನ್ನು ಕಂಡರಂತೂ ಅವನಿಗೆ ಎಲ್ಲಿಲ್ಲದ ಅಸೂಯೆ. ಆ ದಿವ್ಯಸುಂದರಿ ಪಾಂಡವರ ಸ್ವತ್ತೆಂಬುದನ್ನೆ ಅವನು ಸಹಿಸಲಾರ. ಇದರಮೇಲೆ ಶ್ರೀಕೃಷ್ಣನ ಸಾವಿರಾರು ಜನ ಹೆಂಡಿರು ಅವಳ ಸುತ್ತ ಮುತ್ತಿಕೊಂಡು ಓಲೈಸುತ್ತಿದ್ದಾರೆ. ಅವರ ಮಧ್ಯದಲ್ಲಿ ಅವಳು ಕಾಲಂದಿಗೆಗಳ ಘಲಿಘಲಿಧ್ವನಿಯೊಡನೆ ಬಡನಡುವನ್ನು ಬಳುಕಿಸುತ್ತಾ ಗಂಭೀರವಾಗಿ ಹೆಜ್ಜೆ ಹಾಕುವುದು, ಅವಳ ಆ ನಗು, ಆ ಸರಸಸಲ್ಲಾಪ–ಇವು ಆತನ ಕಾಮಕ್ಕೆ ಹಾಕಿದ ಧೂಪವಾಗಿ ಆತನ ಕಳವಳ ಮತ್ತಷ್ಟು ಹೆಚ್ಚಾಯಿತು. ಹೀಗೆ ಒಳಗೊಳಗೇ ಲಾವಾರಸವನ್ನು ತುಂಬಿಕೊಂಡ ಆ ಅಗ್ನಿಪರ್ವತ ಒಮ್ಮೆ ಮಯನಿರ್ಮಿತವಾದ ಪಾಂಡವರ ಸಭಾಭವನಕ್ಕೆ ಹೋಯಿತು. ಅಲ್ಲಿ ಧರ್ಮರಾಯನು ತಮ್ಮಂದಿರೊಡನೆಯೂ ಶ್ರೀಕೃಷ್ಣನೇ ಮೊದಲಾದ ಬಂಧುಗಳೊಡನೆಯೂ ಸರಸಸಲ್ಲಾಪವಾಡುತ್ತಾ ಕುಳಿತಿದ್ದ. ರಾಣಿವಾಸದವರೂ ಅಲ್ಲಿ ನೆರೆದಿದ್ದರು. ಆ ಸಭಾಭವನದ ಶಿಲ್ಪಚಾತುರ್ಯವನ್ನು ಕಂಡು ದುರ್ಯೋಧನನು ಕಕ್ಕಾಬಿಕ್ಕಿ ಯಾದನು. ಅಲ್ಲಿನ ನೆಲ ನೀರಿನಂತೆ ಕಾಣಿಸಿತು. ಅವನು ಬಟ್ಟೆಯನ್ನೆತ್ತಿಕೊಂಡು ಮೆಲ್ಲನೆ ನೀರಿಗಿಳಿಯುವಂತೆ ಇಳಿದ. ಇದನ್ನು ಕಂಡು ಅಲ್ಲಿದ್ದ ಹೆಂಗಸರೆಲ್ಲ ಘೊಳ್ಳೆಂದು ನಕ್ಕರು. ಮಾನಧನನಾದ ದುರ್ಯೋಧನನು ನಾಚಿಕೆಯಿಂದ ತಲೆಬಾಗಿಸಿದ. ಮುಂದೆ ನಾಲ್ಕು ಹೆಜ್ಜೆ ಬರುತ್ತಲೆ ಅಲ್ಲಿ ನೀರಿತ್ತು. ದುರ್ಯೋಧನನು ಮತ್ತೆ ನಗೆಪಾಟಲಾದೀತೆಂದು ನೆಲದ ಮೇಲೆ ನಡೆವಂತೆ ನೇರವಾಗಿ ನಡೆದು, ನೀರಿನಲ್ಲಿ ಬಿದ್ದ; ಬಟ್ಟೆಗಳೆಲ್ಲ ತೊಯ್ದವು. ಮತ್ತೊಮ್ಮೆ ಹೆಂಗಸರೆಲ್ಲ ನಕ್ಕರು. ಧರ್ಮರಾಯ ಅದನ್ನು ತಡೆದ. ಆದರೆ ಕಪಟನಾಟಕಸೂತ್ರಧಾರಿ ಯಾದ ಶ್ರೀಕೃಷ್ಣ ಕಣ್​ಸನ್ನೆಯಿಂದ ಅವರನ್ನು ಪ್ರೋತ್ಸಾಹಿಸಿದ. ಹೆಣ್ಣುಗಳು ಮತ್ತಷ್ಟು ಗಟ್ಟಿಯಾಗಿ ನಕ್ಕವು. ಮಾನಭಂಗದಿಂದ ನೊಂದ ದುರ್ಯೋಧನನು ತಕ್ಷಣವೆ ಅಲ್ಲಿಂದ ಹಿಂದಿರುಗಿ, ಯಾರಿಗೂ ಹೇಳದೆ ಕೇಳದೆ ಹಸ್ತಿನಾವತಿಗೆ ಹೊರಟುಹೋದ.

ದುರ್ಯೋಧನನು ಕೋಪಗೊಂಡು ಹೋದುದನ್ನು ಕಂಡು ಧರ್ಮರಾಯನು ಚಿಂತಾ ಕ್ರಾಂತನಾದನು. ಆದರೆ ಶ್ರೀಕೃಷ್ಣನು ತನ್ನ ಭೂಭಾರಹರಣಕಾರ್ಯಕ್ಕೆ ಒಳ್ಳೆಯ ಅಸ್ತಿಭಾರ ಬಿತ್ತೆಂದು ಮನದಲ್ಲಿಯೆ ಆನಂದಿಸಿದ.

