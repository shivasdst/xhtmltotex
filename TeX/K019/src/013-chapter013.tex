
\chapter{೧೩. ದಕ್ಷಪ್ರಜಾಪತಿಯ ಯಾಗ}

ಸ್ವಾಯುಂಭುವಮನುವಿನ ಮೂವರು ಹೆಣ್ಣು ಮಕ್ಕಳಲ್ಲಿ ಎರಡನೆಯವಳಾದ ದೇವ ಹೂತಿಯ ಕಥೆಯನ್ನು ನಾವೀಗ ವಿಸ್ತಾರವಾಗಿ ತಿಳಿದುಕೊಂಡಿದ್ದೇವೆ. ಮೊದಲನೆಯ ಮಗಳಾದ ಆಕೂತಿಯು ರುಚಿಯೆಂಬ ಮಹಾನುಭಾವನನ್ನು ಮದುವೆಯಾಗಿ ಯಜ್ಞನೆಂಬ ಮಗನನ್ನೂ ದಕ್ಷಿಣೆಯೆಂಬ ಮಗಳನ್ನೂ ಹೆತ್ತಳು. ಯಜ್ಞನು ವಿಷ್ಣುವಿನ ಅಂಶ, ದಕ್ಷಿಣೆ ಲಕ್ಷ್ಮಿಯ ಅವತಾರ; ಆದ್ದರಿಂದ ಅವರು ಅಣ್ಣ ತಂಗಿಯರಾಗಿ ಹುಟ್ಟಿದರೂ ದಂಪತಿ ಗಳಾಗಿ ತೋಷ, ಪ್ರತೋಷ ಇತ್ಯಾದಿಯಾಗಿ ಹನ್ನೆರಡು ಮಕ್ಕಳನ್ನು ಪಡೆದರು. ಸ್ವಾಯಂಭುವಿನ ಮೂರನೆಯ ಮಗಳು ಪ್ರಸೂತಿ. ಈಕೆಯನ್ನು ಚತುರ್ಮುಖ ಬ್ರಹ್ಮನ ಮಗನಾದ ದಕ್ಷಪ್ರಜಾಪತಿಗೆ ಕೊಟ್ಟು ಮದುವೆಯಾಗಿತ್ತು. ಆತನಿಗೆ ಹುಟ್ಟಿದ ಮಕ್ಕಳು ಮೊಮ್ಮಕ್ಕಳು ಮುಂತಾದ ಸಂತಾನಪರಂಪರೆ ಮೂರುಲೋಕಗಳನ್ನೂ ತುಂಬುವಷ್ಟು ವಿಸ್ತಾರವಾಯಿತು.

ದಕ್ಷ-ಪ್ರಸೂತಿಯರಿಗೆ ಹದಿನಾರು ಮಂದಿ ಹೆಣ್ಣುಮಕ್ಕಳು. ಅವರೆಲ್ಲರೂ ಪರಮ ಸುಂದರಿಯರು. ಅವರಲ್ಲಿ ಮೊದಲ ಹದಿಮೂರು ಕನ್ಯೆಯರನ್ನು ಯಮಧರ್ಮನಿಗೆ ವಿವಾಹ ಮಾಡಿಕೊಟ್ಟ, ದಕ್ಷಪ್ರಜಾಪತಿ. ಈ ಹೆಣ್ಣುಗಳೆಲ್ಲರೂ ಪುತ್ರವತಿಯರಾದರು. ಇವರಲ್ಲಿ ಹದಿಮೂರನೆಯಳಾದ ಮೂರ್ತಿ ಎಂಬುವಳಲ್ಲಿ ನರ, ನಾರಾಯಣರೆಂಬ ಅವಳಿ ಮಕ್ಕಳಾದರು. ಇವರಿಬ್ಬರೂ ಮಹಾ ಪುರುಷರು. ಇವರು ಹುಟ್ಟುತ್ತಲೆ ತಂಗಾಳಿ ಬೀಸಿತು, ದಿಕ್ಕು ನಿರ್ಮಲವಾಯಿತು, ನದಿ ಕೊಳಗಳ ನೀರು ತಿಳಿಯಾಯಿತು, ಸರ್ವಪ್ರಾಣಿಗಳೂ ಸಂತೋಷಗೊಂಡವು, ಹೂಮಳೆ ಸುರಿಯಿತು, ಗಂಧರ್ವರು ಗಾನ ಮಾಡಿದರು, ಅಪ್ಸರೆ ಯರು ನರ್ತಿಸಿದರು, ಬ್ರಹ್ಮಾದಿದೇವತೆಗಳು ಅವರನ್ನು ವಂದಿಸಿ ಸ್ತುತಿಸಿದರು. ಇದನ್ನು ಮುಗುಳ್​ನಗೆಯಿಂದ ಸ್ವೀಕರಿಸಿದ ನರನಾರಾಯಣರು ತಾಯ್ತಂದೆಗಳನ್ನು ತೊರೆದು, ಗಂಧಮಾದನ ಪರ್ವತಕ್ಕೆ ಹೋಗಿ ತಪೋಮಗ್ನರಾದರು. ದ್ವಾಪರಯುಗದಲ್ಲಿ ಕೃಷ್ಣಾರ್ ಜುನರಾಗಿ ಹುಟ್ಟಿದವರು ಇವರೆ.

ದಕ್ಷನ ಹದಿನಾಲ್ಕನೆಯ ಮಗಳು ಸ್ವಾಹಾದೇವಿ. ಆಕೆಯನ್ನು ಅಗ್ನಿದೇವನಿಗೆ ಮದುವೆ ಮಾಡಿ ಕೊಟ್ಟಿದ್ದರು. ಆಕೆಗೆ ಮೂವರು ಮಕ್ಕಳು, ನಲವತ್ತೈದು ಜನ ಮೊಮ್ಮಕ್ಕಳು. ಹದಿನೈದನೆ ಮಗಳು ಸ್ವಧಾ. ಆಕೆಯನ್ನು ಪಿತೃದೇವತೆಗಳು ವಿವಾಹವಾಗಿದ್ದರು; ಆಕೆಗೆ ಇಬ್ಬರು ಹೆಣ್ಣು ಮಕ್ಕಳು. ದಕ್ಷನ ಕಡೆಯ ಮಗಳು ಸತೀದೇವಿ. ಈಕೆ ರುದ್ರನ ಪತ್ನಿಯಾಗಿ, ಮಹಾಪತಿವ್ರತೆಯೆನಿಸಿದ್ದಳು. ಈಕೆ ಮಕ್ಕಳಾಗುವ ಮುನ್ನವೇ ದೇಹತ್ಯಾಗ ಮಾಡಿದಳು. ಹಾಗೆ ಆಕೆ ಪ್ರಾಣವನ್ನು ತ್ಯಜಿಸಿದ ಸಂದರ್ಭ ಸನ್ನಿವೇಶಗಳು ಅತ್ಯಂತ ಭಯಂಕರವಾಗಿವೆ:

ಒಂದಾನೊಂದು ಕಾಲದಲ್ಲಿ ಮರೀಚಿ ಮೊದಲಾದ ಮಹರ್ಷಿಗಳೆಲ್ಲರೂ ಸೇರಿ ‘ಸತ್ರಯಾಗ’ ಎಂಬ ಒಂದು ಯಾಗವನ್ನು ಮಾಡುlತ್ತಿದ್ದರು. ಇದಕ್ಕೆ ಬ್ರಹ್ಮನೇ ಮೊದ ಲಾದ ಸಕಲ ದೇವತೆಗಳೂ ಪುಷಿಗಳೂ ಶಿಷ್ಯರೊಡಗೂಡಿದ ಮುನಿಗಳೂ ಬಂದು ಸೇರಿ ದ್ದರು. ಆ ಸಮಯದಲ್ಲಿ ಮಹಾತ್ಮನಾದ ದಕ್ಷಪ್ರಜಾಪತಿಯು ಅಲ್ಲಿಗೆ ಬಂದನು. ಸೂರ್ಯ ನಂತೆ ತೇಜಸ್ಸನ್ನು ಚೆಲ್ಲುತ್ತಾ ಬಂದ ಆತನನ್ನು ಕಾಣುತ್ತಲೆ ಅಲ್ಲಿದ್ದವರೆಲ್ಲರೂ ದಿಗ್ಗನೆ ಎದ್ದು ನಿಂತರು. ಕೇವಲ ಚತುರ್ಮುಖಬ್ರಹ್ಮ ಮತ್ತು ರುದ್ರ ಇವರು ಮಾತ್ರ ಕುಳಿತೇ ಇದ್ದರು. ಬ್ರಹ್ಮನು ಆ ಸಭೆಗೆ ಅಧಿಪತಿಯಾಗಿದ್ದುದರಿಂದ ದಕ್ಷನು ನೇರವಾಗಿ ಆತನ ಬಳಿಗೆ ಹೋಗಿ ನಮಸ್ಕರಿಸಿ, ಆತನು ತೋರಿದ ಪೀಠದಲ್ಲಿ ಕುಳಿತುಕೊಂಡನು. ವಯಸ್ಸಿ ನಲ್ಲಿ ಎಲ್ಲರಿಗಿಂತಲೂ ದೊಡ್ಡವನಾದ ಬ್ರಹ್ಮನು ತಾನು ಬಂದಾಗ ಮೇಲಕ್ಕೇಳದಿದ್ದು ದೇನೋ ಸಹಜ; ಆದರೆ ತನಗಿಂತಲೂ ಕಿರಿಯನಾಗಿ, ತನಗೆ ಅಳಿಯನಾಗಿರುವ ರುದ್ರನು ಹಾಗೆ ಮಾಡದುದು ದಕ್ಷನಿಗೆ ಸಹಿಸಲಿಲ್ಲ. ತನ್ನ ಇದಿರಿನಲ್ಲಿಯೇ ಕಾಲು ಚಾಚಿಕೊಂಡು ಕುಳಿತಿದ್ದ ಅವನನ್ನು ಕಂಡು ದಕ್ಷನ ಕೋಪ ಕೆರಳಿತು. ಕಣ್ಣುಗಳಿಂದಲೆ ಅವನನ್ನು ಸುಟ್ಟುಬಿಡುವಂತೆ ದುರುಗುಟ್ಟಿಕೊಂಡು ನೋಡುತ್ತಾ ಅಲ್ಲಿ ನೆರೆದಿದ್ದವರನ್ನು ಕುರಿತು ‘ಎಲೈ ದೇವತೆಗಳೆ, ಪುಷಿಗಳೆ, ನಾನೀಗ ಹೇಳುವ ಮಾತು ದ್ವೇಷ ಅಥವಾ ಅಜ್ಞಾನದಿಂದ ಆಡುತ್ತಿರುವುದಲ್ಲ; ಸಾಧುಗಳಾಗಿರುವವರ ನಡವಳಿಕೆಯನ್ನು ಕುರಿತು ನಾನು ಮಾತನಾಡು ತ್ತಿದ್ದೇನೆ. ಈ ಮೂರ್ಖನತ್ತ ನೋಡಿರಿ! ಇವನು ಲೋಕಪಾಲಕನಂತೆ! ತನ್ನ ನಡವಳಿಕೆ ಯಿಂದ ಆ ಹೆಸರಿಗೆ ಅವಮಾನಮಾಡುತ್ತಿದ್ದಾನೆ. ನಾನು ಇವನಿಗೆ ನನ್ನ ಮಗಳನ್ನು ಕೊಟ್ಟು ಮದುವೆ ಮಾಡಿದ್ದೇನೆ; ಎಂದಮೇಲೆ ಇವನು ನನಗೆ ಶಿಷ್ಯ ಸಮಾನನಲ್ಲವೆ? ನಾನು ಬಂದರೆ ಇವನು ಹೀಗೆಯೇ ನಡೆದುಕೊಳ್ಳುವುದು? ಇವನಿಗೆ ಎಷ್ಟು ಅಹಂಕಾರ! ಕಪಿಯಂತೆ ಕಣ್ಣುಳ್ಳ ಈ ಕುರೂಪಿಗೆ ಜಿಂಕೆಯ ಕಣ್ಣಿನ ದಿವ್ಯ ಸುಂದರಿಯನ್ನು ಕೊಟ್ಟೆನಲ್ಲ! ಶ್ವಪಚನಿಗೆ ಶ್ರುತಿಯನ್ನು ಉಪದೇಶಿಸಿದಂತಾಯಿತು. ಈ ಪಾಪಿಗೆ ಭೂತ ಪ್ರೇತ ಪಿಶಾಚಿಗಳೆ ಸಂಗಡಿ ಗರು, ಇವನ ವಾಸ ಸ್ಮಶಾನ, ಚಿತೆಯ ಬೂದಿಯೆ ಇವನ ಗಂಧ, ಹೆಣಗಳ ಮೇಲೆ ಹಾಕಿದ ಹೂವೆ ಇವನ ಹೂಮಾಲೆ, ಎಲಬುಗಳೆ ಆಭರಣ, ಮರುಳನಾಗಿ ಮರುಳುಗಳೊಡನೆ ಓಡಾಡುವ ಇವನಿಗೆ ಶಿವನೆಂಬ ಹೆಸರೊಂದು ಕೇಡು. ನನ್ನ ಮುದ್ದು ಮಗಳನ್ನು ಈ ಹುಚ್ಚನ ಕೈಲಿಟ್ಟು ಮೋಸಹೋದೆನಲ್ಲಾ!’ ಎಂದು ಕಠೋರವಾಗಿ ನಿಂದಿಸಿದನು. ಆದ ರೇನು? ರುದ್ರನು ಪ್ರಶಾಂತಮೂರ್ತಿಯಾಗಿ, ಮೌನಧಾರಿಯಾಗಿ ಕುಳಿತಿದ್ದನು. ಈ ಅಲಕ್ಷ್ಯವನ್ನು ಕಂಡು ದಕ್ಷನ ಕೋಪ ಪ್ರಕೋಪಕ್ಕೇರಿತು. ಆತನು ‘ಯಜ್ಞದಲ್ಲಿ ಕೊಡುವ ಹವಿಸ್ಸಿನಲ್ಲಿ ಈ ರುದ್ರನಿಗೆ ಭಾಗವಿಲ್ಲದೆ ಹೋಗಲಿ’ ಎಂದು ಶಪಿಸಿ, ಅಲ್ಲಿ ನಿಲ್ಲಲಾರದೆ ಹೊರಟು ಹೋದನು.

ದಕ್ಷಪ್ರಜಾಪತಿಯ ಕಟುನುಡಿಗಳನ್ನು ಶಾಂತಮೂರ್ತಿಯಾದ ಶಂಕರನು ಕಿವುಡು ಗೇಳಿದನಾದರೂ, ಆತನ ಕಿಂಕರನಾದ ನಂದೀಶ್ವರನು ಕಣ್ಣುಗಳಲ್ಲಿ ಕಿಡಿಗಳನ್ನುದುರಿಸುತ್ತಾ ‘ಎಲೈ ಸಭಿಕರೆ, ಈ ದಕ್ಷನ ಅಹಂಕಾರವೆಷ್ಟು! ಇವನ ದೇಹಾಭಿಮಾನವೆಷ್ಟು! ದೇಹವೇ ತಾನೆಂದು ಭಾವಿಸಿರುವ ಈ ಅವಿವೇಕಿ, ತನ್ನ ಈ ದೇಹಕ್ಕೆ ಗೌರವ ಸಲ್ಲಲಿಲ್ಲವೆಂದು ಶಪಿಸಿದನಲ್ಲಾ! ಈ ಪಾಪಿ ದೇಹಾಭಿಮಾನದಿಂದ ಸಂಸಾರಬಂಧನದಲ್ಲಿಯೇ ನರಳು ತ್ತಿರಲಿ. ಈ ಪಾಪಿಗೆ ಕುರಿಯ ತಲೆಯುಂಟಾಗಲಿ’ ಎಂದು ಶಾಪವಿತ್ತನು. ಆತನ ಕೋಪ ಅಲ್ಲಿಗೆ ಶಾಂತವಾಗಲಿಲ್ಲ. ಅಲ್ಲಿದ್ದ ಬ್ರಾಹ್ಮಣ ಮುನಿಗಳನ್ನು ಕರ್ಮಠರಾದ ಪಾಶಂಡಿ ಗಳಾಗಲೆಂದು ಹೇಳಿ, ‘ಇವರು ವರ್ಣಾಶ್ರಮಧರ್ಮಗಳನ್ನು ಬಿಟ್ಟು, ದುರಾಚಾರರಾಗಲಿ, ಹೊಟ್ಟೆಗಾಗಿ ವಿದ್ಯಾರ್ಜನೆ ಮಾಡಲಿ, ಈ ಲೋಕದಲ್ಲಿ ತಿರುಕರಾಗಿ ಬಾಳಲಿ, ನಶ್ವರವಾದ ಸ್ವರ್ಗಸುಖಕ್ಕಾಗಿ ಕರ್ಮಮಾಡುವವರಾಗಲಿ’ ಎಂದು ಶಪಿಸಿದನು. ಇದನ್ನು ಕೇಳಿ ಭೃಗುಮುನಿಗೆ ಕೋಪಬಂತು. ಆತ ಶೈವರನ್ನು ಪಾಷಂಡಿಗಳಾಗಿ ಕಾಪಾಲಿಗಳಾಗಲೆಂದು ಶಪಿಸಿದ. ಈ ಶಾಪ ಪ್ರತಿಶಾಪಗಳನ್ನು ಕಂಡು ರುದ್ರನಿಗೆ ಅಸಹ್ಯವಾಯಿತು. ಆತನು ತನ್ನ ಪರಿವಾರದೊಡನೆ ಅಲ್ಲಿಂದ ಹೊರಟುಹೋದನು. ಅಲ್ಲಿ ಉಳಿದವರು ಯಾಗವನ್ನು ಮಾಡಿ ಮುಗಿಸಿದರು.

ಅಲ್ಲಿಂದ ಮುಂದೆ ಅಳಿಯ, ಮಾವನಲ್ಲಿ ದ್ವೇಷ ಬೆಳೆಯುತ್ತಲೆ ಹೋಯಿತು. ಹೀಗಿರು ವಲ್ಲಿ ಚತುರ್ಮುಖಬ್ರಹ್ಮನು ದಕ್ಷಪ್ರಜಾಪತಿಯನ್ನು ಎಲ್ಲ ಪ್ರಜಾಧಿಪತಿಗಳ ದೊರೆ ಯಾಗಿ ಪಟ್ಟ ಕಟ್ಟಿದನು. ಇದರಿಂದ ದಕ್ಷನ ಗರ್ವ ಮತ್ತಷ್ಟು ತಲೆಗೇರಿತು. ಅವನು ವಾಜಪೇಯವೆಂಬ ಯಾಗವನ್ನು ಮಾಡಿ ಬೃಹಸ್ಪತಿವನವೆಂಬ ಮಹಾಯಾಗವನ್ನು ಕೈ ಕೊಂಡನು. ಈ ಯಾಗಕ್ಕೆ ಲೋಕಲೋಕದವರಿಗೆಲ್ಲ ಆಹ್ವಾನ ಹೋಯಿತಾದರೂ ರುದ್ರ ನಿಗೆ ಮಾತ್ರ ಹೋಗಲಿಲ್ಲ. ಎಲ್ಲ ದೇವತೆಗಳೂ, ಪುಷಿ ಮುನಿಗಳೂ, ಪಿತೃಗಳೂ ಅತ್ಯಂತ ಸಡಗರದಿಂದ ದಿವ್ಯವಿಮಾನಗಳಲ್ಲಿ ಕುಳಿತು ಪ್ರಯಾಣ ಹೊರಟರು. ದೇವತಾಸ್ತ್ರೀಯರು ಹೊಸ ಬಟ್ಟೆಗಳನ್ನುಟ್ಟು, ಆಭರಣಗಳನ್ನು ತೊಟ್ಟು, ಕೈಲಾಸವನ್ನು ಹಾದು ಹೋಗುತ್ತಾ, ದಕ್ಷಯಜ್ಞದ ವೈಭವವನ್ನು ಬಾಯಲ್ಲಿ ನೀರೂರುವಂತೆ ವರ್ಣಿಸುತ್ತಿದ್ದರು. ಮನೆಯ ಮುಂದುಗಡೆಯಲ್ಲಿ ಕುಳಿತಿದ್ದ ಸತೀದೇವಿಗೆ ಇದು ಕೇಳಿಸಿತು. ಗಂಡಹೆಂಡಿರು ಸರಸ ಸಲ್ಲಾಪಮಾಡುತ್ತಾ ಹೋಗುತ್ತಿದ್ದುದೂ ಕಾಣಿಸಿತು. ತಾನೂ ತನ್ನ ತಂದೆಯ ಯಾಗಕ್ಕೆ ಗಂಡನೊಡನೆ ಹೋಗಬೇಕೆನ್ನಿಸಿತು. ಆಕೆ ಗಂಡನ ಬಳಿಗೆ ಬಂದು ಅತ್ಯಂತ ವಿನಯದಿಂದ ‘ಸ್ವಾಮಿ, ನಿಮ್ಮ ಮಾವ ದೊಡ್ಡ ಯಾಗವನ್ನು ಮಾಡುವನಂತೆ, ಎಲ್ಲರೂ ಅದಕ್ಕೆ ಹೋಗು ತ್ತಿದ್ದಾರೆ. ನನ್ನ ಅಕ್ಕಂದಿರೆಲ್ಲ ತಮ್ಮ ಗಂಡಂದಿರೊಡನೆ ಅಲ್ಲಿಗೆ ಬಂದಿರುತ್ತಾರೆ. ನಾವೂ ಹೋಗೋಣವೇ? ನಾನು ತೌರುಮನೆಯನ್ನು ಬಿಟ್ಟು ಬಹುದಿನಗಳಾಯಿತು. ತಾಯಿ ತಂದೆ ಗಳನ್ನು ಕಂಡು ಉಡುಗೊರೆಯನ್ನು ಪಡೆಯಬೇಕೆಂದು ಆಶೆಯಾಗಿದೆ. ನನ್ನ ಬಂಧುಬಳಗ ದವರನ್ನೆಲ್ಲ ನೋಡುವುದಕ್ಕೆ ಇದೊಂದು ಒಳ್ಳೆಯ ಅವಕಾಶ. ಆ ಯಾಗ ಬಹು ವೈಭವ ದಿಂದ ನಡೆಯುವುದಂತೆ! ನೋಡಿ, ನಮ್ಮ ನೆರೆಹೊರೆಯ ಹೆಣ್ಣುಗಳೆಲ್ಲ ಅಲಂಕರಿಸಿ ಕೊಂಡು, ತಮ್ಮ ಗಂಡಂದಿರೊಡನೆ ಸರಸಸಲ್ಲಾಪ ಮಾಡುತ್ತಾ, ಹಂಸದಂತೆ ಬಿಳಿದಾದ ವಿಮಾನಗಳಲ್ಲಿ ಕುಳಿತು ಪ್ರಯಾಣ ಹೊರಟಿದ್ದಾರೆ. ಅವರೇ ಅಷ್ಟು ಸಡಗರ ಪಡುತ್ತಿರು ವಾಗ ನನ್ನ ಮನಸ್ಸು ಹೇಗಿರಬಹುದು! ದಯವಿಟ್ಟು ನನ್ನ ಬೇಡಿಕೆಯನ್ನು ಈಡೇರಿಸಿಕೊಡಿ’ ಎಂದು ಕೇಳಿಕೊಂಡಳು.

ಮಡದಿಯ ಮಾತುಗಳನ್ನು ಕೇಳಿ ರುದ್ರನಿಗೆ ‘ಅಯ್ಯೊ’ ಎನಿಸಿತು; ‘ಕರೆಯದೆ ಹೋಗು ವುದು ಸರಿಯಲ್ಲ’ ಎಂದು ಹೇಳಿ ನೋಡಿದ. ಸತಿ ಅದಕ್ಕೆ ತಕ್ಕ ಉತ್ತರವಿತ್ತಳು: ‘ಮಿತ್ರ, ಗುರು, ಗಂಡ, ತೌರು–ಈ ಮನೆಗಳಿಗೆ ಕರೆಯದಿದ್ದರೂ ಹೋಗಬೇಕಾದುದು ಅತ್ಯಗತ್ಯ’. ಈ ಮಾತನ್ನು ರುದ್ರ ಅಲ್ಲಗಳೆಯಲಿಲ್ಲ. ಆದರೆ ತುಂಬಿದ ಸಭೆಯಲ್ಲಿ ಆಕೆಯ ತಂದೆ ತನ್ನನ್ನು ಅತ್ಯಂತ ಕೆಟ್ಟಮಾತುಗಳಿಂದ ನಿಂದಿಸಿದುದನ್ನು ಆಕೆಗೆ ಜ್ಞಾಪಿಸಿದನು. ‘ನೋಡು, ಅಂತಹ ಕಿಡಿನುಡಿಗಳಿಂದ ನೋಯಿಸಿದ ಬಂಧುಗಳ ಮನೆಯಲ್ಲಿ ಕಾಲಿಡಬಾರದು. ನಿನ್ನ ತಂದೆ ತುಂಬ ಅಹಂಕಾರಿ, ಆತನಿಗೆ ತಾರತಮ್ಯ ಜ್ಞಾನವಿಲ್ಲ. ಶತ್ರುಗಳ ಬಾಣದ ಪೆಟ್ಟಿ ಗಿಂತಲೂ ಹೆಚ್ಚಾಗಿ ನನ್ನ ನೋಯಿಸಿದೆ, ನಿನ್ನ ತಂದೆಯ ಮಾತುಗಳು. ನಾನು ಅವನ್ನು ಎಂದಿಗೂ ಮರೆಯಲಾರೆ. ಇದೂ ಅಲ್ಲದೆ ಆತ ದ್ವೇಷಸಾಧನೆಯನ್ನು ಮುಂದುವರಿಸು ತ್ತಿದ್ದಾನೆ. ಆದ್ದರಿಂದಲೇ ನಮಗೆ ಕರೆ ಬಂದಿಲ್ಲ. ನೀನು ಆತನ ಅತ್ಯಂತ ಮುದ್ದು ಮಗಳೆಂಬುದು ನಿಜವಾದರೂ, ಆತ ನನ್ನನ್ನು ದ್ವೇಷಿಸುತ್ತಿರುವುದರಿಂದ ನಿನ್ನನ್ನೂ ಆದರಿಸುವುದಿಲ್ಲ. ನನ್ನ ಪರಮ ದ್ವೇಷಿಯಾದ ಅವನನ್ನಾಗಲೀ, ಅವನ ಕಡೆಯವರನ್ನಾಗಲಿ ನೀನು ಕಣ್ಣೆತ್ತಿಕೂಡ ನೋಡಬಾರದು. ಸತಿ, ನನ್ನ ಮಾತನ್ನೂ ಮೀರಿ ನೀನು ಹೋಗುವು ದಾದರೆ ನಿನಗೆ ಮಂಗಳವಾಗುವುದಿಲ್ಲ’ ಎಂದು ಆಕೆಗೆ ಬುದ್ಧಿ ಹೇಳಿದನು.

ಹೋಗಬೇಡವೆಂದು ಮಡದಿಗೆ ಬುದ್ಧಿ ಹೇಳಿದನಾದರೂ ರುದ್ರನ ಮನಸ್ಸು ತೂಗು ಯ್ಯಾಲೆಯಂತಾಗುತಿತ್ತು. ಸತಿಗೆ ತೌರುಮನೆಯ ಗೀಳು ಎಷ್ಟು ಬಲವಾಗಿತ್ತೆಂದರೆ, ಆಕೆಗೆ ಆಶಾಭಂಗವಾದರೆ ಪ್ರಾಣತ್ಯಾಗ ಮಾಡಬಹುದು. ಕಳುಹಿಸಿದರಂತೂ ಅಪಮಾನದಿಂದ ಆಕೆ ದೇಹತ್ಯಾಗ ಮಾಡುವುದು ಶತಸ್ಸಿದ್ಧ. ಮುಂದೋರದೆ ಆತನು ಮೌನಕ್ಕೆ ಶರಣಾದನು. ಸತೀದೇವಿಗೆ ತಡೆಯಲಾರದೆ ಕೋಪ ಬಂತು. ತನ್ನ ದೇಹದಲ್ಲಿಯೇ ಅರ್ಧಭಾಗವನ್ನು ಕೊಟ್ಟಿರುವ ಆ ಪ್ರೇಮಮಯನಾದ ಪತಿಯು ತನ್ನ ಸಣ್ಣ ಬೇಡಿಕೆಯನ್ನು ಈಡೇರಿಸಲಿಲ್ಲ ವಲ್ಲ–ಎಂದು ಆಕೆಯ ಕುದಿ, ಆಕೆ ಕಣ್ಣಲ್ಲಿ ನೀರನ್ನು ಸುರಿಸುತ್ತ, ಕಡುಗೋಪದಿಂದ ಮೈಯೆಲ್ಲ ನಡುಗುತ್ತಿರಲು, ಕೆಂಗಣ್ಣಿನಿಂದ ಗಂಡನನ್ನು ಒಮ್ಮೆ ಕೆಕ್ಕರಿಸಿ ನೋಡಿ, ಮನೆಯನ್ನು ಬಿಟ್ಟು ಹೊರಟಳು. ಶಂಕರನ ಮಡದಿಯಾದರೂ ಹೆಣ್ಣಿನ ಅವಿವೇಕ ಆಕೆ ಯನ್ನು ಬಿಡಲಿಲ್ಲ. ಹೀಗೆ ಒಂಟಿಗಳಾಗಿ ತೌರಿಗೆ ಹೊರಟ ಆಕೆಯನ್ನು ರುದ್ರನ ಬಳಿಯಿದ್ದ ಪ್ರಮಥರು ನೋಡಿ, ಮುಂದೇನು ಅವಿವೇಕ ನಡೆಯುವುದೋ ಎಂಬ ಭಯದಿಂದ ಆಕೆ ಯನ್ನು ಹಿಂಬಾಲಿಸಿದರು. ನಂದಿಯನ್ನು ಮುಂದಿಟ್ಟುಕೊಂಡು ರುದ್ರನ ಪರಿವಾರ ವೆಲ್ಲವೂ ಆಕೆಯ ರಕ್ಷಣೆಗಾಗಿ ಆಕೆಯ ಹಿಂದೆಯೆ ನಡೆದರು, ನೋಡಿದವರಿಗೆ ಕೈಲಾಸದಲ್ಲಿ ನಡೆದ ಸಂಗತಿ ಗೊತ್ತಾಗದಿರುವಂತೆ, ಅವರು ಕಳಶಕನ್ನಡಿಗಳನ್ನು ಹಿಡಿದು ಮಂಗಳವಾದ್ಯ ಗಳನ್ನು ಬಾರಿಸಿದರು. ಆಕೆಯನ್ನು ಶಿವನ ವಾಹನವಾದ ಎತ್ತಿನ ಮೇಲಿ ಕೂಡಿಸಿ, ಸರ್ವಾ ಲಂಕಾರಭೂಷಿತೆಯಾಗಿ ಮಾಡಿದರು. ಆಕೆಯ ಮೇಲೆ ಬಿಳಿ ಕೊಡೆಯನ್ನೆತ್ತಿ, ಚಾಮರ ದಿಂದ ಬೀಸುತ್ತಾ, ಸಕಲ ಮರ್ಯಾದೆಗಳೊಡನೆ ಆಕೆಯನ್ನು ಯಾಗ ಮಂಟಪಕ್ಕೆ ಕರೆ ದೊಯ್ದರು.

ಸತೀದೇವಿ ಅತ್ಯಂತ ಸಡಗರದಿಂದ ಯಾಗಮಂಟಪವನ್ನು ಪ್ರವೇಶಿಸಿದಳು. ಆದರೆ ನೆರೆದ ಸಹಸ್ರಾರು ಜನರಲ್ಲಿ ಯಾರೂ ಆಕೆಯನ್ನು ಗಮನಿಸಲಿಲ್ಲ. ಸ್ವತಃ ದಕ್ಷನೇ ಆಕೆ ಯನ್ನು ಕಂಡರೂ ಕಾಣದವನಂತೆ ಮುಖವನ್ನು ತಿರುಗಿಸಿಕೊಂಡನು. ಆತನ ಭಯದಿಂದ ಉಳಿದವರೂ ಆದರಿಸಲಿಲ್ಲ. ಕೇವಲ ಆಕೆಯ ತಾಯಿ ಮತ್ತು ಅಕ್ಕಂದಿರು ಮಾತ್ರ ಆಕೆ ಯನ್ನು ಪ್ರೀತಿಯಿಂದ ಅಪ್ಪಿಕೊಂಡರು. ಅತ್ಯಂತ ಮುದ್ದುಮಗಳೆನಿಸಿಕೊಂಡಿದ್ದ ತನ್ನನ್ನು ತಂದೆ ಕಣ್ಣೆತ್ತಿಯೂ ನೋಡದಿರುವುದನ್ನು ಕಂಡು ಆಕೆಗೆ ಸಂಕಟವಾಯಿತು. ತಾಯಿ ಕುಳಿತುಕೊಳ್ಳುವಂತೆ ಹೇಳಿ ಪೀಠವನ್ನು ತೋರಿಸಿದರೂ ಆಕೆ ಕುಳಿತುಕೊಳ್ಳಲಿಲ್ಲ. ತನ್ನ ಗಂಡನಿಗೆ ಹವಿಸ್ಸನ್ನು ಕೊಡದ ಆ ಯಾಗ, ಲೋಕೇಶ್ವರನಾದ ಆತನನ್ನು ನಿಂದಿಸುತ್ತಿರುವ ಆ ತಂದೆ, ತನ್ನನ್ನು ಅವಮಾನಗೊಳಿಸಿದ ಆ ಸಭೆ–ಇವುಗಳನ್ನು ಸುಟ್ಟುಬಿಡಬೇಕೆನ್ನು ವಷ್ಟು ಕೋಪ ಉಕ್ಕಿತು, ಆಕೆಯಲ್ಲಿ. ಕಣ್ಣುಗಳಲ್ಲಿ ಕಿಡಿಗಾರುತ್ತಾ ತಂದೆಯನ್ನು ನೋಡಿ ‘ಎಲವೊ ತಂದೆಯೆನಿಸಿಕೊಂಡ ಮದಾಂಧ, ನನ್ನ ಪತಿಯಾದ ಶಂಕರನು ಜಗತ್ತಿನ ಜೀವ ಗಳಿಗೆಲ್ಲ ಆತ್ಮಸ್ವರೂಪಿಯಾಗಿರುವನು. ರಾಗದ್ವೇಷಗಳಿಲ್ಲದ ಆ ಪುಣ್ಯಪುರುಷನನ್ನು ನೀನು ದ್ವೇಷಿಸುವೆಯಲ್ಲವೆ? ಜಡದೇಹವೇ ಆತ್ಮವೆಂದು ಭಾವಿಸಿರುವ ಬುದ್ಧಿಹೀನನಾದ ನೀನು ಹಾಗೆ ಮಾಡುವುದು ಸಹಜವೇ. ‘ಶಿವ’ ಎಂಬ ಎರಡಕ್ಷರಗಳನ್ನು ಉಚ್ಚರಿಸಿದರೆ ಸಾಕು, ಪಾಪರಾಶಿಗಳು ಸುಟ್ಟು ಹೋಗುತ್ತವೆ; ಅಂತಹ ಮಹಾತ್ಮನು ಪಾಪಿಯಾದ ನಿನಗೆ ಶತ್ರು. ಬ್ರಹ್ಮಾನಂದವೆಂಬ ಮಕರಂದವನ್ನು ಬಯಸುವವರು ಆತನ ಪಾದಪದ್ಮಗಳನ್ನು ಆಶ್ರಯಿಸುತ್ತಾರೆ; ನೀನು ಆತನಿಗೆ ದ್ರೋಹಿ! ಆತನ ಪಾದದಿಂದ ಜಾರಿಬಿದ್ದ ಹೂವನ್ನು ಬ್ರಹ್ಮಾದಿ ದೇವತೆಗಳು ಕಣ್ಣಿಗೊತ್ತಿಕೊಂಡು ತಲೆಯಲ್ಲಿ ಧರಿಸುತ್ತಾರೆ; ಆತ ನಿನಗೆ ಅಮಂಗಳ! ಆತನನ್ನು ನಿಂದಿಸಿದವನ ನಾಲಗೆಯನ್ನು ಕತ್ತರಿಸಬೇಕು, ಹಾಗೆ ಮಾಡಲು ಶಕ್ತಿಯಿಲ್ಲದಿದ್ದರೆ ಪ್ರಾಣತ್ಯಾಗ ಮಾಡಬೇಕು. ನಾನು ಈಗ ಆ ಕಾರ್ಯವನ್ನು ಮಾಡುತ್ತೇನೆ. ಶಿವದ್ವೇಷಿಯಾದ ನಿನ್ನಿಂದ ಬಂದ ಈ ದೇಹ ನನಗೆ ಅಸಹ್ಯವಾಗಿದೆ. ‘ದಾಕ್ಷಾಯಿಣಿ’ ಎಂಬ ಹೆಸರು ಇಂದು ಅಳಿದು ಹೋಗಲಿ’ ಎಂದು ಗುಡುಗಿದಳು.

ಮಹಾಪತಿವ್ರತೆಯಾದ ಸತೀದೇವಿಯು ನುಡಿದಂತೆ ನಡೆದಳು. ಉತ್ತರ ಮುಖವಾಗಿ ಕುಳಿತು ಪ್ರಾಣಾಯಾಮವನ್ನು ಕೈಕೊಂಡಳು. ‘ಅನಿಲಾಗ್ನಿಧಾರಣೆ’ ಎಂಬ ಧಾರಣೆಯಿಂದ ಆಕೆಯ ಶರೀರವು ಅಗ್ನಿಜ್ವಾಲೆಯಿಂದ ತುಂಬಿತು. ಈ ಯೋಗಾಗ್ನಿಯಿಂದ ಆಕೆಯ ದೇಹ ಸುಟ್ಟು ಬೂದಿಯಾಯಿತು. ಇದನ್ನು ಕಂಡು ಭೂಮಿ ಆಕಾಶಗಳೆರಡೂ ಹಾಹಾಕಾರದಿಂದ ತುಂಬಿಹೋದವು. ಯಾರ ಬಾಯಲ್ಲಿ ಕೇಳಲಿ ‘ಅಯ್ಯೋ, ಏನನ್ಯಾಯ, ಈ ಪಾಪಿ ದಕ್ಷ ಬ್ರಹ್ಮ ಮಗಳನ್ನು ಅವಮಾನಮಾಡಿ, ಆಕೆಯ ಸಾವಿಗೆ ಕಾರಣನಾದನಲ್ಲಾ! ಶಿವದ್ವೇಷಿ ಯಾದ ಈ ನೀಚ ಇಹದಲ್ಲಿ ಅಪಕೀರ್ತಿಯನ್ನೂ ಪರದಲ್ಲಿ ನರಕವನ್ನೂ ಗಳಿಸಿಕೊಂಡ ನಲ್ಲಾ!’ ಎಂಬ ಮಾತೇ! ಈ ಮಾತುಗಳನ್ನು ಕೇಳುತ್ತಿದ್ದ ರುದ್ರನ ಪರಿವಾರದವರು– ಮೊದಲೆ ಸತಿಯ ಸಾವಿನಿಂದ ಸಂಕಟಪಡುತ್ತಿದ್ದವರು–ತಮ್ಮ ದುಃಖದ ಬೆಂಕಿಗೆ ಸಹಾನುಭೂತಿಯ ಗಾಳಿ ಬೀಸಿದಂತಾಗಿ, ರೋಷಭೀಷಣರಾದರು. ಆ ಯಜ್ಞದಲ್ಲಿ ಭಾಗ ವಹಿಸಿದವರನ್ನೆಲ್ಲ ಕೊಂದುಬಿಡಬೇಕೆಂದು ಅವರು ತಮ್ಮ ಆಯುಧಗಳೊಡನೆ ನುಗ್ಗಿ ಬಂದರು. ಇದನ್ನು ಕಂಡ ಭೃಗುಮುನಿಯು ಅಭಿಚಾರಮಂತ್ರದಿಂದ ಹೋಮವನ್ನು ಮಾಡಲು ಯಜ್ಞಕುಂಡದಿಂದ ‘ಪುಭು’ ಗಳೆಂಬ ದೇವತೆಗಳು ಹುಟ್ಟಿ ಬಂದು ಪ್ರಮಥರ ನ್ನೆಲ್ಲಾ ದಿಕ್ಕಾಪಾಲಾಗುವಂತೆ ಬಡಿದು ಓಡಿಸಿದರು.

ಅತ್ತ ರುದ್ರನು ದಕ್ಷಯಜ್ಞದಲ್ಲಿ ನಡೆದ ಸಂಗತಿಯನ್ನೆಲ್ಲ ನಾರದಮಹರ್ಷಿಯಿಂದ ಕೇಳಿ, ಕೋಪದಿಂದ ಕುದಿಯುತ್ತ, ಸಿಡಿಲಿನ ಉಂಡೆಯಂತಿದ್ದ ತನ್ನ ಜಟೆಯಿಂದ ಒಂದು ತಿರಿಯನ್ನು ಕಿತ್ತು ನೆಲಕ್ಕೆ ಅಪ್ಪಳಿಸಿದನು. ತಕ್ಷಣವೇ ಬೆಟ್ಟದಂತೆ ಎತ್ತರವಾದ ದೇಹವುಳ್ಳ ವೀರಭದ್ರನು ಪುಟಚಂಡಿನಂತೆ ನೆಗೆದು ಮೇಲೆದ್ದನು. ಕಾಲಮೇಘದಂತೆ ಕಪ್ಪಾದ ಬಣ್ಣ, ಸಾವಿರ ಆಯುಧಗಳನ್ನು ಹಿಡಿದಿರುವ ಸಾವಿರ ತೋಳುಗಳು, ಥಳಥಳಿಸುವ ಕೋರೆದಾಡೆ ಗಳು, ಏಕಕಾಲದಲ್ಲಿ ಮೂವರು ಸೂರ್ಯರು ಮೂಡಿದಂತೆ ನಿಗಿನಿಗಿ ಹೊಳೆಯುತ್ತಿರುವ ಮೂರು ಕಣ್ಣುಗಳು, ಕೆಂಡದಂತೆ ಕೆಂಪಾದ ಜಟೆ, ಕೊರಳಲ್ಲಿ ತಲೆಬುರುಡೆಗಳ ಹಾರ– ಇಂತಹ ಭಯಂಕರಾಕಾರದ ವೀರಭದ್ರನು ರುದ್ರನ ಇದಿರು ಕೈಮುಗಿದು ನಿಂತು ‘ಹೇ ಸ್ವಾಮಿ, ಏನಪ್ಪಣೆ?’ ಎಂದು ಗುಡುಗಿದನು. ರುದ್ರನು ಆತನೊಡನೆ ‘ಹೇ ವೀರಭದ್ರ, ನಿನ್ನನ್ನು ಈ ಪ್ರಮಥರ ಸೇನಾಪತಿಯನ್ನಾಗಿ ಮಾಡಿದ್ದೇನೆ. ನೀನು ಈಗಲೇ ಹೋಗಿ ಆ ದಕ್ಷನನ್ನೂ ಅವನ ಯಾಗವನ್ನೂ ಧ್ವಂಸಮಾಡಿಬರಬೇಕು’ ಎಂದು ಅಪ್ಪಣೆ ಮಾಡಿದನು. ವೀರ ಭದ್ರನು ‘ತಥಾಸ್ತು’ ಎಂದು ಹೇಳಿ, ಶಿವನಿಗೆ ನಮಸ್ಕರಿಸಿ, ತನ್ನ ಸೇನೆಯೊಡನೆ ದಕ್ಷನ ಯಾಗಶಾಲೆಗೆ ಧಾವಿಸಿದನು.

ಭಯಂಕರವಾದ ಪ್ರಮಥರ ಸೇನೆಯು ಯಾಗಮಂಟಪವನ್ನು ಪ್ರವೇಶಿಸಿ ಅಲ್ಲೋಲ ಕಲ್ಲೋಲವನ್ನುಂಟುಮಾಡಿತು. ಕೆಲವರು ಸಭಾಮಂಟಪಕ್ಕೆ ಬೆಂಕಿ ಹೊತ್ತಿಸಿದರು, ಕೆಲವರು ಅಡಿಗೆಯ ಮನೆಯನ್ನು ನೆಲಸಮ ಮಾಡಿದರು, ಕೆಲವರು ಅಲ್ಲಿದ್ದ ಯಜ್ಞದ ಸಾಮಾನುಗಳನ್ನೆಲ್ಲ ದಿಕ್ಕುದಿಕ್ಕಿಗೆ ಬಿಸಾಡಿದರು, ಕೆಲವರು ಯಾಗದ ಅಗ್ನಿಯನ್ನೇ ಆರಿಸಿ ದರು, ಇನ್ನು ಕೆಲವರು ಅಲ್ಲಿದ್ದ ಪುಷಿಗಳನ್ನೂ ಅವರ ಪತ್ನಿಯರನ್ನೂ ಗದರಿಸಿ ಗಾಬರಿ ಗೊಳಿಸಿದರು; ಮಣಿಮಂತನೆಂಬ ಪ್ರಮಥನು ಭೃಗು ಪುಷಿಯನ್ನು ಬಲವಾಗಿ ಹಿಡಿದು ನಿಲ್ಲಿಸಿದನು; ವೀರಭದ್ರನು ಆ ಪುಷಿಯ ಗಡ್ಡಮೀಸೆಗಳನ್ನು ಕಿತ್ತುಹಾಕಿದನು; ಭಗನೆಂಬ ಪುಷಿಯನ್ನು ನೆಲಕ್ಕೆ ಕೆಡಹಿ ಆತನ ಕಣ್ಣುಗಳನ್ನು ಕಿತ್ತನು. ಅನಂತರ ದಕ್ಷನನ್ನು ನೆಲಕ್ಕೆ ಮೆಟ್ಟಿ, ಅವನ ಎದೆಯ ಮೇಲೆ ನಿಂತು ಕತ್ತನ್ನು ಕತ್ತರಿಸಹೊರಟನು. ಆದರೆ ಕತ್ತಿಯ ಏಟಿಗೆ ಅದು ತುಂಡಾಗಲಿಲ್ಲ. ಇದರಿಂದ ಕೆರಳಿದ ವೀರಭದ್ರನು ಅವನ ಕತ್ತನ್ನು ಹಿಚಿಕಿ, ಅವನ ಪ್ರಾಣ ಹೋಗುತ್ತಲೆ ಅವನ ತಲೆಯನ್ನು ಕಿತ್ತೆಸೆದನು. ಆ ತಲೆಯನ್ನು ದಕ್ಷಿಣಾಗ್ನಿ ಯಲ್ಲಿ ಹೋಮ ಮಾಡಿದಮೇಲೆ ಆತನ ರೋಷ ಶಾಂತವಾಯಿತು. ಆತನು ತನ್ನ ಪರಿವಾರ ದೊಡನೆ ಕೈಲಾಸಕ್ಕೆ ಹಿಂದಿರುಗಿದನು.

ದಕ್ಷನ ಯಾಗಶಾಲೆಯಲ್ಲಿ ಪ್ರಮಥರ ಕೈಗೆ ಸಿಕ್ಕದೆ ತಪ್ಪಿಸಿಕೊಂಡ ದೇವತೆಗಳೂ ಪುಷಿ ಗಳೂ ಬ್ರಹ್ಮನ ಬಳಿಗೆ ಓಡಿಹೋಗಿ, ತಮ್ಮ ಗೋಳನ್ನು ಹೇಳಿಕೊಂಡರು. ಆತನು ಅವ ರನ್ನು ಕುರಿತು ‘ಅಯ್ಯಾ, ತಪ್ಪು ನಿಮ್ಮದು. ಯಾಗದಲ್ಲಿ ರುದ್ರನಿಗೆ ಹೋಮಭಾಗವಿಲ್ಲ ವೆಂದರೆ ಏನರ್ಥ? ಅವಿವೇಕಿಯಾದ ದಕ್ಷಬ್ರಹ್ಮ ಆತನನ್ನು ನಿಂದಿಸಿ ಕೆರಳಿಸಿದ; ಈಗ ಲಂತೂ ಹೆಂಡತಿಯ ಸಾವಿನಿಂದ ಆತನು ರೋಷಾವೇಶಗೊಂಡಿದ್ದಾನೆ. ಆತನ ಸ್ವರೂಪ ವಾಗಲಿ ಪರಾಕ್ರಮವಾಗಲಿ ನಮಗಾರಿಗೂ ಅರ್ಥವಾಗುವಂತಹುದಲ್ಲ. ಆತನನ್ನು ಸಮಾ ಧಾನಪಡಿಸಬೇಕಾದರೆ ಆತನಿಗೆ ಶರಣಾಗತರಾಗಬೇಕು. ಆತನು ಕೋಪಗೊಂಡರೆ ಜಗತ್ತು ಉಳಿಯುವುದಿಲ್ಲ. ಆದರೆ ಆತನು ಅಷ್ಟೇ ಕರುಣಾಮಯಿ. ಶರಣಾದವರಿಗೆ ಬಹುಬೇಗ ಒಲಿಯುತ್ತಾನೆ. ನಡೆಯಿರಿ, ನಿಮ್ಮೊಡನೆ ನಾನೂ ಬರುತ್ತೇನೆ. ಎಲ್ಲರೂ ಕೈಲಾಸಕ್ಕೆ ಹೋಗೋಣ’ ಎಂದು ಹೇಳಿ ಅವರೊಡನೆ ಕೈಲಾಸಕ್ಕೆ ಬಂದನು.

ಕೈಲಾಸದ ಮಹತ್ತನ್ನು ಯಾರು ಬಣ್ಣಿಸಬಲ್ಲರು? ಸಿದ್ಧ ವಿದ್ಯಾಧರರೂ, ಕಿನ್ನರ ಕಿಂಪುರುಷರೂ, ದೇವಗಂಧರ್ವರೂ ಸದಾ ಕಿಕ್ಕಿರಿದು ತುಂಬಿರುವ ಆ ಪರ್ವತದ ರತ್ನ ಮಯ ಶಿಖರಗಳು, ಸ್ಫಟಿಕದಂತೆ ಶುಭ್ರವಾದ ನೀರಿನಿಂದ ಕೂಡಿದ ನದಿ ಸರೋವರಗಳು, ಬಗೆಬಗೆಯ ಫಲವೃಕ್ಷಗಳು, ಹೂ ಗಿಡಗಳು, ಬಗೆಬಗೆಯ ಪ್ರಾಣಿಗಳು, ಚಿತ್ರವಿಚಿತ್ರವಾದ ಹಕ್ಕಿಗಳು ಮತ್ತು ಅವುಗಳ ಅವ್ಯಕ್ತಮಧುರವಾದ ಗಾನ–ಈ ಸೌಭಾಗ್ಯವನ್ನು ಸವಿಯುತ್ತಾ ಬ್ರಹ್ಮನು ತನ್ನ ಹಿಂಬಾಲಕರೊಡನೆ ಕೈಲಾಸಪರ್ವತದಲ್ಲಿ ನಡೆದು ಬರುತ್ತಿರಲು ಕುಬೇರನ ಪಟ್ಟಣವಾದ ಅಲಕಾಪುರಿಯೂ ಅದರ ಬಳಿಯಿದ್ದ ಸೌಗಂಧಿಕಾವನವೂ ಕಾಣಿಸಿದವು. ಆ ವನದ ಪಕ್ಕದಲ್ಲಿಯೆ ನಂದೆ ಅಲಕನಂದೆಗಳು ಪ್ರವಹಿಸುತ್ತಿವೆ. ಅಲ್ಲಿ ಅಪ್ಸರೆಯರು ತಮ್ಮ ರಮಣರೊಡನೆ ಸದಾ ಜಲಕೇಳಿಯಾಡುತ್ತಿರುವುದರಿಂದ ನೀರೆಲ್ಲ ಅರಿಶಿನದ ಬಣ್ಣ. ಅಲ್ಲಿಂದ ಸ್ವಲ್ಪ ಮುಂದೆ ಬಂದರೆ ಮತ್ತೊಂದು ಸುಂದರವಾದ ಉದ್ಯಾನವನ. ಅದರಲ್ಲಿ ದೊಡ್ಡದೊಂದು ಆಲದಮರ. ಅದರ ಎತ್ತರ ನೂರು ಯೋಜನ, ಎಪ್ಪತ್ತೈದು ಯೋಜನ ದಷ್ಟು ಸುತ್ತಲೂ ಹರಡಿದೆ. ಆ ಮರದಡಿಯಲ್ಲಿ ಕೋಪವನ್ನು ನೀಗಿದ ಯಮನಂತೆ ಶಾಂತಮೂರ್ತಿಯಾದ ಪರಶಿವನು ಕುಳಿತಿದ್ದನು. ಸಂಜೆಯ ಮೋಡದಂತೆ ಕೆಂಪಾದ ಆತನ ದೇಹ ವಿಭೂತಿಯಿಂದ ತುಂಬಿತ್ತು. ತಲೆಯಲ್ಲಿ ಚಂದ್ರಕಳೆಯಿಂದೊಪ್ಪುವ ಜಟೆ ವಿರಾಜಮಾನವಾಗಿತ್ತು. ಮೈಮೇಲೆ ಕೃಷ್ಣಾಜಿನವನ್ನು ಧರಿಸಿದ್ದನು. ಬಲ ತೊಡೆಯಮೇಲೆ ಎಡಗಾಲನ್ನಿಟ್ಟು, ಎಡ ತೊಡೆಯಮೇಲೆ ಎಡಗೈಯನ್ನೂರಿ, ಬಲಗೈಯಲ್ಲಿ ಜಪಸರವನ್ನು ಹಿಡಿದು ಯೋಗಮುದ್ರೆಯಲ್ಲಿ ಕುಳಿತಿದ್ದ ಆತನು ನಾರದನಿಗೆ ತತ್ವೋಪದೇಶ ಮಾಡುತ್ತಿದ್ದನು. ಸುತ್ತಲೂ ಮಹರ್ಷಿಗಳು ನೆರೆದಿದ್ದರು. ಯಕ್ಷರಾಜನಾದ ಕುಬೇರನೂ, ಸುನಂದಾದಿ ಪುಷಿಗಳೂ ಆತನ ಸೇವೆಯಲ್ಲಿ ಮಗ್ನರಾಗಿದ್ದರು. ಆತನನ್ನು ಕಾಣುತ್ತಲೆ ಬ್ರಹ್ಮನ ಅನುಯಾಯಿಗಳಾಗಿ ಬಂದಿದ್ದವರೆಲ್ಲರೂ ಆತನಿಗೆ ಅಡ್ಡಬಿದ್ದರು.

ತನಗೆ ನಮಸ್ಕರಿಸಿದವರತ್ತ ಪರಶಿವನು ತಿರುಗುತ್ತಲೆ ಚತುರ್ಮುಖಬ್ರಹ್ಮನು ಆತನ ಕಣ್ಣಿಗೆ ಬಿದ್ದನು. ಒಡನೆಯೆ ಆತ ದಿಗ್ಗನೆದ್ದು ಆತನಿಗೆ ನಮಸ್ಕರಿಸಿದನು. ಆಗ ಬ್ರಹ್ಮನು ನಸುನಗುತ್ತಾ‘ಹೇ ಲೋಕೇಶ್ವರಾ, ನೀನೇ ಪರಬ್ರಹ್ಮವಸ್ತು. ಈ ಸೃಷ್ಟಿ ಸ್ಥಿತಿ ಲಯಗಳಿಗೆ ನೀನೇ ಕಾರಣನು. ಸಕಲ ಪುರುಷಾರ್ಥಗಳನ್ನೂ ಕೊಡುವ, ವೇದಗಳನ್ನು ರಕ್ಷಿಸುವ ಹೊಣೆ ನಿನ್ನದೆ ಅಲ್ಲವೆ? ಲೋಕಕ್ಷೇಮಕ್ಕಾಗಿ ಯಜ್ಞಯಾಗಾದಿಗಳನ್ನು ಮಾಡಬೇಕೆಂದು ವಿಧಿಸಿ ದವನು ನೀನೆ. ನಿನ್ನ ಪ್ರೇರಣೆಯಿಲ್ಲದೆ ಯಾವುದೂ ನಡೆಯುವುದಿಲ್ಲ. ಮತಿಗೆಟ್ಟ ಪಾಮರರು ಅವಿವೇಕದಿಂದ ಅಪರಾಧ ಮಾಡಿದರೂ ಅವರನ್ನು ಸರ್ವಜ್ಞನಾದ ನೀನು ಕ್ಷಮಿಸಬೇಕೆ ಹೊರತು ಅವರ ಅವಿವೇಕವನ್ನು ಮನಸ್ಸಿಗೆ ತಂದುಕೊಳ್ಳಬಾರದು. ನಿನಗೆ ನ್ಯಾಯವಾಗಿ ಸಲ್ಲಬೇಕಾದ ಹವಿಸ್ಸಿನ ಭಾಗವನ್ನು ಕೊಡದೆ ಹೋದುದು ದಕ್ಷನ ಅಕ್ಷಮ್ಯ ಅಪರಾಧ. ಆದ್ದರಿಂದಲೆ ಯಾಗ ಅರ್ಧಕ್ಕೆ ನಿಂತುಹೋಯಿತು. ಅದನ್ನು ಪೂರ್ಣ ಗೊಳಿಸುವ ಭಾರ ನಿನ್ನದು ಅಷ್ಟೇ ಅಲ್ಲ, ದಕ್ಷನು ಮತ್ತೆ ಬದುಕಬೇಕು. ಪ್ರಮಥರಿಂದ ಪೆಟ್ಟು ತಿಂದು ಅಂಗಾಂಗಗಳನ್ನು ಕಳೆದುಕೊಂಡವರು ಮತ್ತೆ ಮೊದಲಿನಂತಾಗಬೇಕು’ ಎಂದು ಬೇಡಿಕೊಂಡನು.

ಪರಶಿವನು ಬ್ರಹ್ಮನ ಬೇಡಿಕೆಗೆ ಸಮ್ಮತಿಯನ್ನೀಯುತ್ತಾ ‘ಹೇ ಬ್ರಹ್ಮದೇವ, ಮಾಯೆ ಯಿಂದ ಮರುಳಾದ ಮಾನವರ ಅಪರಾಧಕ್ಕಾಗಿ ನನಗೆ ಕೋಪವೇನೂ ಇಲ್ಲ. ಆದರೆ ತಪ್ಪು ಮಾಡಿದವರು ಶಿಕ್ಷೆಗೆ ಒಳಗಾಗುವರೆಂಬುದನ್ನು ಲೋಕಕ್ಕೆ ತಿಳಿಸುವುದಕ್ಕಾಗಿ ನಾನು ಈ ಕಾರ್ಯವನ್ನು ಮಾಡಬೇಕಾಯಿತು. ನಿನ್ನ ಅಪೇಕ್ಷೆಯಂತೆ ದಕ್ಷನನ್ನು ಕ್ಷಮಿಸುತ್ತೇನೆ. ಆದರೆ ಅವನ ತಲೆ ಸುಟ್ಟು ಹೋಗಿರುವುದರಿಂದ ಅವನಿಗೆ ಕುರಿಯ ತಲೆಯುಂಟಾಗಲಿ. ಉಳಿದವ ರಿಗೂ ಅವರ ಅಂಗಾಂಗಗಳ ಕಾರ್ಯ ಸಲೀಸಾಗಿ ನಡೆಯುವಂತಾಗಲಿ’ ಎಂದು ಹೇಳಿದನು. ಇದನ್ನು ಕೇಳಿ ಅಲ್ಲಿ ನೆರೆದಿದ್ದವರೆಲ್ಲ ಪರಶಿವನನ್ನು ಕೊಂಡಾಡಿದರು. ಅವರೆಲ್ಲರ ಅಪೇಕ್ಷೆ ಯಂತೆ ಶಿವನು ಅವರೊಡನೆ ದಕ್ಷನ ಯಾಗಶಾಲೆಗೆ ಹೋದನು. ಮತ್ತೆ ಜೀವವನ್ನು ಪಡೆದ ದಕ್ಷನು ತಾನು ಮಾಡಿದ ಅಪಚಾರಕ್ಕಾಗಿ ಪಶ್ಚಾತ್ತಾಪ ಪಟ್ಟು, ಶರತ್ಕಾಲದ ನದಿಯಂತೆ ನಿರ್ಮಲ ಮನಸ್ಸಿನವನಾಗಿ, ಆತನ ಕ್ಷಮೆಯನ್ನು ಯಾಚಿಸಿದನು. ತನ್ನ ಮಗಳನ್ನು ನೆನೆದು ಆತನ ಕಣ್ಣು ಹನಿಗೂಡಿತು. ಆದರೆ ಶಿವನ ಸಾನ್ನಿಧ್ಯದಿಂದ ಮನಸ್ಸು ಶಾಂತಗೊಂಡಿತು.

ಅರ್ಧಕ್ಕೆ ನಿಂತಿದ್ದ ಯಜ್ಞ ಮತ್ತೆ ಪ್ರಾರಂಭವಾಯಿತು. ಮತ್ತೆ ಯಾವ ಅಡ್ಡಿ ಆತಂಕ ಗಳೂ ಬಾರದಿರಲೆಂದು ಪ್ರಾರ್ಥಿಸಿ ವಿಷ್ಣುವಿಗೆ ಹೋಮವನ್ನು ಅರ್ಪಿಸಿದರು. ತರುವಾಯು ದಕ್ಷನು ಒಂದೇ ಮನಸ್ಸಿನಿಂದ ಆ ದೇವದೇವನನ್ನು ಧ್ಯಾನಿಸಿದನು. ತಕ್ಷಣವೇ ವಿಷ್ಣು ಅಲ್ಲಿ ಪ್ರತ್ಯಕ್ಷನಾದನು. ಒಡನೆಯೆ ಅಲ್ಲಿ ನೆರೆದಿದ್ದವರೆಲ್ಲ ದಿಗ್ಗನೆ ಮೇಲೆದ್ದು ಆತನಿಗೆ ಅಡ್ಡ ಬಿದ್ದರು. ಯಾಗದ ಯಜಮಾನನಾದ ದಕ್ಷನು ಆತನನ್ನು ಪೂಜಿಸಿ ‘ಹೇ ಭಗವಂತ, ನೀನು ನಿನ್ನ ನಿಜಸ್ವರೂಪದಲ್ಲಿ ಸದಾ ರಮಿಸತಕ್ಕವನು, ಮಾಯಾತೀತನು, ಜ್ಞಾನಮಯನು, ಸರ್ವತಂತ್ರಸ್ವತಂತ್ರನು. ಆದರೂ ಆಗಾಗ ನೀನೇ ಇಷ್ಟಪಟ್ಟು, ನಿನ್ನ ಮಾಯೆಯನ್ನು ನೀನೇ ಸ್ವೀಕರಿಸಿ, ರಾಮ, ಕೃಷ್ಣ ಎಂಬ ಮಾನವ ರೂಪವನ್ನು ಪಡೆಯುವೆ; ಆಗ ಮಾಯೆಗೆ ಒಳಗಾದವನಂತೆ–ಕರ್ಮಬಂಧಕ್ಕೆ ಕಟ್ಟುಬಿದ್ದವನಂತೆ–ಕಾಣುವೆ. ಇಂತಹ ಲೀಲೆ ಯನ್ನು ತೋರುವ ಸರ್ವಶಕ್ತನಾದ ನಿನಗೆ ನಮಸ್ಕಾರ’ ಎಂದು ಅಡ್ಡಬಿದ್ದನು. ಅಲ್ಲಿನ ಪುತ್ವಿಕ್ಕುಗಳು ‘ಸ್ವಾಮಿ, ನಮಗೆ ಇಂತಹ ಕರ್ಮಕ್ಕೆ ಇಂತಹ ದೇವತೆ–ಎಂಬುದು ಮಾತ್ರ ಗೊತ್ತು; ನಿನ್ನ ಸ್ವರೂಪ ಗೊತ್ತಿಲ್ಲ’ ಎಂದು ಹೇಳಿ ಅಡ್ಡಬಿದ್ದರು. ಸಭೆಯಲ್ಲಿ ನೆರೆದಿ ದ್ದವರು ‘ಪ್ರಭು, ನಾವು ಸಂಸಾರದ ಹಾದಿಹೋಕರು. ಎಷ್ಟು ನಡೆದರೂ ಕುಳಿತುಕೊಳ್ಳಲು ನೆರಳಿಲ್ಲ, ಕಾಮಕ್ರೋಧವೆಂಬ ಬೆಟ್ಟಗಳನ್ನು ದಾಟಬೇಕಾಗಿದೆ, ಯಮನೆಂಬ ಹೆಬ್ಬುಲಿ ಹಾದಿಯಲ್ಲಿ ಹೊಂಚುಹಾಕಿಕೊಂಡಿದೆ; ಸುಖದುಃಖಗಳೆಂಬ ಹಳ್ಳಕೊಳ್ಳಗಳು, ದುಃಖ ವೆಂಬ ಕಾಡುಗಿಚ್ಚು–ಇವುಗಳನ್ನೆಲ್ಲ ಸರಾಗವಾಗಿ ದಾಟಿ ಹೋಗುವಂತಹ ವೈರಾಗ್ಯವನ್ನು ದಯಪಾಲಿಸು’ ಎಂದು ಹೇಳಿ ನಮಸ್ಕರಿಸಿದರು. ಹೀಗೆಯೆ ಅಲ್ಲಿ ನೆರೆದಿದ್ದ ದೇವೇಂದ್ರನೆ ಮೊದಲಾದವರೆಲ್ಲರೂ ಶ್ರೀಹರಿಯನ್ನು ಮುಕ್ತಕಂಠರಾಗಿ ಹೊಗಳಿದರು.

ಯಜ್ಞ ಕಾರ್ಯವು ಶ್ರೀಹರಿಯ ಕೃಪೆಯಿಂದ ಸಾಂಗವಾಗಿ ನೆರವೇರಿತು. ಆಮೇಲೆ ಆ ದೇವದೇವನು ದಕ್ಷಬ್ರಹ್ಮನನ್ನು ತನ್ನ ಬಳಿಗೆ ಕರೆದು ‘ಅಯ್ಯಾ ದಕ್ಷಬ್ರಹ್ಮ! ನಾನು, ಬ್ರಹ್ಮ, ರುದ್ರ ಬೇರೆಬೇರೆಯಲ್ಲ. ನಾನೇ ಬ್ರಹ್ಮನಾಗಿ ಲೋಕವನ್ನು ಸೃಷ್ಟಿಸುತ್ತೇನೆ; ನಾನೆ ರಕ್ಷಿಸುತ್ತೇನೆ; ನಾನೇ ರುದ್ರನಾಗಿ ಸಂಹರಿಸುತ್ತೇನೆ. ನಮ್ಮ ಮೂವರಲ್ಲಿ ಭೇದವೆಣಿಸಿದರೆ ಅಶಾಂತಿ; ಒಂದೆ ಎಂದು ತಿಳಿದರೆ ಶಾಂತಿ. ನನಗೆ ನಾನೆ ಪ್ರಭು, ಎಲ್ಲ ಜೀವಿಗಳಲ್ಲಿ ನೆಲಸಿರುವವನು ನಾನೆ; ಆದರೆ ಈ ಜೀವಿಗಳ ನಾಮಭೇದ ರೂಪಭೇದಗಳಾಗಲಿ, ಸುಖ ದುಃಖವಾಗಲಿ ನನಗೆ ಇಲ್ಲ. ಜಗತ್ತಿನ ಚೇತನ ಅಚೇತನಗಳೆರಡೂ ನಾನೆ. ಸ್ಥೂಲವಾಗಿ ಕಾಣುವ ಜಗತ್ತೂ ನಾನೆ, ಅದಕ್ಕೆ ಕಾರಣವಾದ ಸೂಕ್ಷ್ಮ ಚೇತನವೂ ನಾನೆ, ನನ್ನನ್ನು ಬಿಟ್ಟು ಬೇರೆ ವಸ್ತುವೇ ಇಲ್ಲ. ದೇಹದ ತಲೆ, ಕೈ, ಕಾಲು ಇತ್ಯಾದಿಗಳು ಹೇಗೆ ಆ ದೇಹದಿಂದ ಬೇರೆಯಲ್ಲವೊ ಹಾಗೆಯೆ ಜಗತ್ತಿನ ಜೀವಿಗಳು ಎಲ್ಲವೂ ನಾನಲ್ಲದೆ ಬೇರೆಯಲ್ಲ’ ಎಂದು ಹೇಳಿ ಅಂತರ್ಧಾನವಾದನು. ಬಂದಿದ್ದವರೆಲ್ಲರೂ ‘ಧರ್ಮ ಬುದ್ಧಿಯುಂಟಾಗಲಿ’ ಎಂದು ದಕ್ಷನನ್ನು ಹರಸಿ ಹೊರಟುಹೋದರು.

ದೇಹವನ್ನು ತ್ಯಜಿಸಿಹೋಗಿದ್ದ ಸತೀದೇವಿಯು ಹಿಮವಂತನ ಪತ್ನಿಯಾದ ಮೇನೆಯಲ್ಲಿ ಪಾರ್ವತಿಯೆಂಬ ಹೆಸರಿನಿಂದ ಹುಟ್ಟಿ ಮತ್ತೆ ಪರಶಿವನನ್ನು ವರಿಸಿದಳು.

