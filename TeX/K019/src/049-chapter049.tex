
\chapter{೪೯. ಬ್ರಹ್ಮನ ಹಮ್ಮು ಮುರಿಯಿತು}

ಅಘಾಸುರನನ್ನು ಕೊಂದಮೆಲೆ ಶ್ರೀಕೃಷ್ಣನು ಗೆಳೆಯರನ್ನೆಲ್ಲ ಕೂಡಿಕೊಂಡು ಅಲ್ಲಿಯೇ ಸಮೀಪದಲ್ಲಿದ್ದ ಒಂದು ಸರೋವರದ ಬಳಿಗೆ ಬಂದನು. ಆ ಸರೋವರದ ಸುತ್ತಲೂ ಸೊಗಸಾದ ಸಣ್ಣ ಮರಳು, ಅದರ ಸುತ್ತ ಎತ್ತರವಾಗಿ ಬೆಳೆದು ದಟ್ಟನಾಗಿ ನೆರಳನ್ನು ಚೆಲ್ಲಿದ್ದ ಗಿಡಮರ ಬಳ್ಳಿಗಳು; ರಮ್ಯವಾದ ಆ ಸ್ಥಳದಲ್ಲಿ ವನಭೋಜನ ಮಾಡಬೇಕೆನಿಸಿತು, ಆತನಿಗೆ. ಆಗಾಗಲೆ ಊಟದ ಹೊತ್ತಾಗಿದ್ದುದರಿಂದ, ಹಸಿದ ಗೋಪಾಲರೂ ಅದಕ್ಕೆ ಸಮ್ಮತಿಸಿದರು. ತಮ್ಮ ತುರುಮಂದೆಗೆ ಅಲ್ಲಿಯೆ ನೀರುಕುಡಿಸಿ, ಅವನ್ನು ಮೇಯುವುದಕ್ಕೆ ಬಿಟ್ಟು ಗೋಪಾಲರೆಲ್ಲ ಶ್ರೀಕೃಷ್ಣನ ಸುತ್ತಲೂ ಕುಳಿತು ತಮ್ಮ ಬುತ್ತಿಯ ಗಂಟುಗಳನ್ನು ಬಿಚ್ಚಿದರು. ಊಟ ಮಾಡುತ್ತಾ ಒಬ್ಬೊಬ್ಬರೂ ತಮ್ಮ ಬುತ್ತಿಯ ರುಚಿಯನ್ನು ಅತಿಶಯ ವಾಗಿ ಹೊಗಳಿಕೊಂಡು, ಇತರರನ್ನು ಹಾಸ್ಯ ಮಾಡಿ ನಗುತ್ತಿದ್ದರು. ಶ್ರೀಕೃಷ್ಣನೂ ತನ್ನ ಎಡಗೈಲಿ ಮೊಸರನ್ನದ ಉಂಡೆಯನ್ನಿಟ್ಟುಕೊಂಡು, ಬೆರಳಸಂದಿನ ಉಪ್ಪಿನ ಕಾಯನ್ನು ಆಗಾಗ ನಂಚಿಕೊಳ್ಳುತ್ತಾ ತನ್ನ ನಗೆನುಡಿಗಳಿಂದ ಎಲ್ಲರನ್ನೂ ನಗಿಸುತ್ತಿದ್ದ. ಭಗವಂತನ ಈ ಲೀಲೆಯನ್ನು ಕಂಡು ಸ್ವರ್ಗದ ದೇವತೆಗಳೆಲ್ಲ ಸಂತೋಷಪಡುತ್ತಿದ್ದರೆ, ಚತುರ್ಮುಖ ಬ್ರಹ್ಮನಿಗೆ ಶ್ರೀಕೃಷ್ಣನ ಮಹಿಮೆಯನ್ನು ಪರೀಕ್ಷಿಸಬೇಕೆಂಬ ದುರ್ಬುದ್ಧಿ ಹುಟ್ಟಬೇಕೆ! ಗೋಪಾಲರೆಲ್ಲ ಊಟದಲ್ಲಿ ಮೈಮರೆತಿರುವಾಗ, ಅವರ ತುರುಮಂದೆ ಸೊಂಪಾದ ಹುಲ್ಲನ್ನು ಹುಡುಕುತ್ತಾ ಕಾಡಿನಲ್ಲಿ ಬಹುದೂರ ಹೊರಟು ಹೋಯಿತು. ನಾಲ್ಮೊಗದ ಬ್ರಹ್ಮ ಆ ತುರುಮಂದೆಯನ್ನು ತನ್ನ ಮಾಯಾ ಶಕ್ತಿಯಿಂದ ಕಣ್ಮರೆಯಾಗಿ ಮಾಡಿದ. ಊಟವನ್ನು ಮುಗಿಸಿದ ಗೋಪಾಲರು ತಮ್ಮ ದನಗಳನ್ನು ಹುಡುಕಿ ನೋಡುತ್ತಾರೆ, ಅವು ಎಲ್ಲಿಯೂ ಕಣ್ಣಿಗೆ ಬೀಳಲಿಲ್ಲ. ಅವರಿಗೆ ಭಯವಾಯಿತು. ಒಬ್ಬಿಬ್ಬರ ಕಣ್ಣಿನಲ್ಲಿ ನೀರೂ ಸುರಿಯಿತು. ಶ್ರೀಕೃಷ್ಣನು ಅವರನ್ನು ಸಮಾಧಾನ ಮಾಡುತ್ತಾ ‘ಗೆಳೆಯರೆ ನೀವು ಭಯ ಪಡಬೇಡಿ. ನಾನು ಹೋಗಿ ಎಲ್ಲಿದ್ದರೂ ಆ ತುರುಗಳನ್ನು ಹೊಡೆದುಕೊಂಡು ಬರುತ್ತೇನೆ’ ಎಂದು ಹೇಳಿ ಅಡವಿಯನ್ನು ಹೊಕ್ಕನು.

ಶ್ರೀಕೃಷ್ಣನು ಅತ್ತ ಹೋಗುತ್ತಲೆ, ಇತ್ತ ಬ್ರಹ್ಮನು ಅಲ್ಲಿದ್ದ ಗೋಪಾಲರನ್ನೆಲ್ಲ ಮಂಗ ಮಾಯವಾಗಿ ಮಾಡಿದ. ಶ್ರೀಕೃಷ್ಣನು ಅಡವಿಯನ್ನೆಲ್ಲ ತಿರುಗಿ, ತುರುಮಂದೆಯನ್ನು ಕಾಣದೆ ತಾನಿದ್ದ ತಾಣಕ್ಕೆ ಹಿಂದಿರುಗಿ ಬಂದು ನೋಡುತ್ತಾನೆ, ತನ್ನ ಗೆಳೆಯರೊಬ್ಬರೂ ಅಲ್ಲಿಲ್ಲ. ಬ್ರಹ್ಮನು ಅವರೆಲ್ಲರನ್ನೂ ತನ್ನ ಮಾಯೆಯಿಂದ ಮರೆ ಮಾಡಿದ್ದ. ಶ್ರೀಕೃಷ್ಣ ಅವರನ್ನು ಹುಡುಕಿಕೊಂಡು ಮತ್ತೆ ಆ ಕಾಡನ್ನೆಲ್ಲ ಅಲೆದುದಾಯಿತು. ಎಲ್ಲೆಲ್ಲಿ ಹುಡುಕಿ ದರೂ ಗೋವುಗಳೂ ಇಲ್ಲ, ಗೋಪಾಲರೂ ಇಲ್ಲ. ಆಗ ಶ್ರೀಕೃಷ್ಣನಿಗೆ ತನ್ನ ಮಾನವಾತೀತ ಶಕ್ತಿಯನ್ನು ಬಳಸದೆ ಬೇರೆ ಮಾರ್ಗವಿಲ್ಲವಾಯಿತು. ಆತನು ಒಮ್ಮೆ ಕಣ್ಮುಚ್ಚಿ ತೆರೆಯು ವಷ್ಟರಲ್ಲಿ ಕಳ್ಳನಾರೆಂಬುದನ್ನು ಅರ್ಥಮಾಡಿಕೊಂಡನು. ಆದರೆ ಈಗ ಆ ಕಳ್ಳನಿಗಿಂತಲೂ ಗೋಕುಲದ ಜನರು ಆತನಿಗೆ ಮುಖ್ಯವಾಗಿತ್ತು. ಗೋಗಳನ್ನೂ ಗೋಪಬಾಲರನ್ನೂ ಕಾಣದೆ ಗೋಕುಲದವರು ಗೋಳಾಡುವಂತಾದರೆ ಏನು ಗತಿ? ತಕ್ಷಣವೆ ಸರ್ವಾತ್ಮಕನಾದ ಶ್ರೀಕೃಷ್ಣನು ತಾನೆ ತುರುಮಂದೆಯಾದನು, ತಾನೆ ಗೋಪಾಲಕರಾದನು; ಹೀಗೆ, ತಾನೆ ಮೇಯುವವನೂ ಮೇಯಿಸುವವನೂ ಆಗಿ, ಸಂಜೆಯಾಗುತ್ತಿದ್ದಂತೆಯೇ ಗೋಕುಲಕ್ಕೆ ಹಿಂದಿರುಗಿದನು. ಗುಣ, ರೂಪ, ಶೀಲ, ವಯಸ್ಸು, ವೇಷ, ಭೂಷಣ, ಹೆಸರು–ಎಲ್ಲದ ರಲ್ಲಿ ಯಾವ ಭೇದವೂ ಇಲ್ಲದೆ ‘ಸರ್ವಂ ವಿಷ್ಣುಮಯಂ ಜಗತ್​’ ಎಂಬ ಸೂಕ್ತಿ ಅಕ್ಷರಶಃ ನಿಜವಾದಂತಾಯಿತು. ಬೃಂದಾವನದ ಗೋಕುಲದಲ್ಲಿ ಯಾವ ವ್ಯತ್ಯಾಸವೂ ಇಲ್ಲದೆ ದಿನ ದಿನದಂತೆ ಜೀವನ ವ್ಯಾಪಾರ ಮುಂದುವರಿಯುತ್ತಾ ಹೋಯಿತು. ವ್ಯತ್ಯಾಸವಿಷ್ಟೆ–ಪ್ರತಿ ಯೊಬ್ಬ ತಾಯಿತಂದೆಗೂ ತಮ್ಮ ಮಗ ಮೊದಲಿಗಿಂತಲೂ ಮೋಹಕನಾಗಿದ್ದಾನೆ. ಅದಕ್ಕೆ ಕಾರಣವೂ ಸ್ಪಷ್ಟವೇ–ಅವರಿಗೆಲ್ಲ ಶ್ರೀಕೃಷ್ಣನಲ್ಲಿ ಅಪಾರ ಪ್ರೇಮವಿತ್ತು; ಈಗ ಅದರ ಜೊತೆಗೆ ಪಿತೃವಾತ್ಸಲ್ಯವೂ ಸೇರಿದೆ.

 ಬೃಂದಾವನದ ಜೀವನವು ಎಂದಿನಂತೆ ಸಾಗಿತ್ತು. ಆದರೆ ಬ್ರಹ್ಮಲೋಕದಲ್ಲಿ ಇದು ತಲೆಕೆಳಗಾಗಿ ಹೋಗಿತ್ತು. ನಾಲ್ಕು ಮುಖದ ಬ್ರಹ್ಮ ನಾಲ್ಕು ಕಡೆಗೂ ನೋಡಬಲ್ಲ ನಾದರೂ, ಮೇಲೆ ತನ್ನ ಲೋಕದಲ್ಲಿ ಏನಾಗುವುದೆಂಬುದನ್ನು ಕಾಣಲಾರದವನಾಗಿದ್ದ. ಆತ ಗೋವುಗಳನ್ನೂ ಗೋಪಾಲಕರನ್ನೂ ಮಂಗಮಾಯ ಮಾಡಿ ತನ್ನ ಲೋಕಕ್ಕೆ ಹಿಂದಿರು ಗಿದ; ಆದರೆ ಆ ವೇಳೆಗೆ ಅವನಂತೆಯೆ ಇದ್ದ ಮತ್ತೊಬ್ಬ ಬ್ರಹ್ಮ ಅವನ ಮನೆಯಲ್ಲಿ ಸೇರಿ ಕೊಂಡಿದ್ದ. ಬಾಗಿಲು ಕಾಯುವವರು ನಿಜವಾದ ಬ್ರಹ್ಮನನ್ನು ಬಾಗಿಲ ಬಳಿ ತಡೆದು ‘ಅಯ್ಯಾ, ನೀನಾರು? ಬ್ರಹ್ಮನಂತೆ ವೇಷ ಹಾಕಿಕೊಂಡು ಬಂದಿರುವೆಯಲ್ಲ! ನಮ್ಮ ಸ್ವಾಮಿ ಒಳಗೆ ಸಿಂಹಾಸನದಲ್ಲಿ ಕುಳಿತಿದ್ದಾನೆ; ನಿನ್ನ ಮೋಸ ನಮ್ಮಲ್ಲಿ ನಡೆಯುವುದಿಲ್ಲ’ ಎಂದು ಗದರಿಸಿದರು. ಹೆಚ್ಚು ಮಾತನಾಡಿದರೆ ಕತ್ತು ಹಿಡಿದು ನೂಕುವುದಕ್ಕೂ ಅವರು ಸಿದ್ಧ. ಬ್ರಹ್ಮ ತನ್ನ ಆಳುಗಳಿಂದಲೆ ಅವಮಾನಿತನಾಗಿ ತಾನು ಹೊರಟ ಸ್ಥಳಕ್ಕೆ ಹಿಂದಿರು ಗಿದ. ಅಲ್ಲಿ ನೋಡುತ್ತಾನೆ–ಶ್ರೀಕೃಷ್ಣ, ಬಲರಾಮ, ಗೋಪಾಲರು, ಅವರ ಮಂದೆ– ಎಲ್ಲವೂ ಇದ್ದಂತೆಯೆ ಇದೆ! ಬ್ರಹ್ಮನಿಗೆ ಆಶ್ಚರ್ಯ, ಆಶ್ಚರ್ಯಕ್ಕಿಂತಲೂ ಹೆಚ್ಚಾಗಿ ಭಯ! ತಾನು ಮಾಯೆಯಿಂದ ಮರೆಮಾಡಿದ್ದ ತುರುಗಳೂ, ಗೋವಳರೂ ಹಾಗೆಯೇ ಇವೆ. ಅವು ಗಳೇ ಇಲ್ಲಿಯೂ ಕಾಣುತ್ತಿವೆ. ಇವು ಎಲ್ಲಿಂದ ಬಂದವು? ಹೇಗೆ ಬಂದವು? ಎಷ್ಟು ಯೋಚಿಸಿದರೂ ಅವನಿಗೆ ಸಮಸ್ಯೆ ಅರ್ಥವಾಗಲಿಲ್ಲ. ಬ್ರಹ್ಮನಾದರೇನು? ಭಗವಂತನ ಮಾಯೆಯನ್ನು ಭೇದಿಸುವ ಶಕ್ತಿ ಅವನಿಗೂ ಇಲ್ಲ. ಮೋಸಮಾಡಲೆಂದು ಬಂದ ಬ್ರಹ್ಮ; ಆ ಬ್ರಹ್ಮ ತಾನೆ ಮೋಸ ಹೋಗಿದ್ದ. ಆಗ ಶ್ರೀಕೃಷ್ಣ ಅವನ ಮೇಲಿನ ಕರುಣೆಯಿಂದ, ಅಲ್ಲಿ ಚರಾಚರವಸ್ತುವೆಲ್ಲ ತನ್ನ ರೂಪದಿಂದಲೆ ಕಾಣುವಂತೆ ಬ್ರಹ್ಮನಿಗೆ ಅನುಗ್ರಹಿಸಿದ. ಈ ವಿಶ್ವರೂಪವನ್ನು ಕಂಡ ಮೇಲೆ ಬ್ರಹ್ಮನ ಕಣ್ಣು ತೆರೆಯಿತು. ಅವನು ಶ್ರೀಕೃಷ್ಣನನ್ನು ಅತ್ಯಂತ ಭಕ್ತಿಯಿಂದ ಸ್ತೋತ್ರಮಾಡಿದನು–

‘ಹೇ ಶ್ರೀಕೃಷ್ಣ, ಪರಮ ಪುರುಷ, ಮಹಾಮಹಿಮ, ನಿತ್ಯಾನಂದಮೂರ್ತಿ, ನಿನ್ನ ಮಹಿಮೆ ನಿನಗೊಬ್ಬನಿಗೇ ವೇದ್ಯ. ನಿನ್ನ ಸ್ವರೂಪವನ್ನು ಸಂಪೂರ್ಣವಾಗಿ ಯಾರೂ ತಿಳಿಯ ಲಾರರು. ನೀನು ಸರ್ವತಂತ್ರಸ್ವತಂತ್ರ. ನೀನು ನಿತ್ಯ, ನಿರಂಜನ, ನಿತ್ಯಾನಂದ, ನಿರ್ವಿಕಾರ, ಪರಿಪೂರ್ಣ, ಸರ್ವಾಂತರ್ಯಾಮಿ. ಸರ್ವಜೀವಿಗಳ ಅಂತರಾತ್ಮನಾದ ನಿನ್ನನ್ನು ಹೊರಗೆಲ್ಲ ಹುಡುಕುವುದೆಂದರೆ, ಕೊಂಕಳಲ್ಲಿ ಕೂಸನ್ನಿಟ್ಟುಕೊಂಡು ಕೇರಿಕೇರಿಯೆಲ್ಲ ಅಲೆದಂತೆ! ಸ್ವಾಮಿ, ಈ ವಿವೇಕಜ್ಞಾನ ಹುಟ್ಟುವುದಕ್ಕೂ ನಿನ್ನ ಅನುಗ್ರಹ ಅಗತ್ಯ. ನಿನ್ನ ಈ ಅನು ಗ್ರಹವಾದಾಗಲೆ ನಿನ್ನ ಮಹಿಮೆಯ ಸ್ವರೂಪವನ್ನು ಅರ್ಥಮಾಡಿಕೊಳ್ಳುವುದು ಸಾಧ್ಯ. ದೇವದೇವ, ನಿನ್ನ ಈ ಅನುಗ್ರಹವಾಯಿತೆಂದರೆ ಮನೆ ಮಂದಿ ಮಕ್ಕಳೆಂಬ ಮೋಹಪಾಶ ಗಳು ಕೂಡ ಮುಕ್ತಿಪ್ರದವಾಗುತ್ತವೆ. ಇದಕ್ಕೆ ಈ ನಂದಗೋಕುಲದ ಗೋಪಾಲರೇ ಸಾಕ್ಷಿ. ಅಮ್ಮ! ಅವರ ಪುಣ್ಯವೇ ಪುಣ್ಯ. ನಿನ್ನ ಅನುಗ್ರಹವಾಗುವವರೆಗೆ ಮಾತ್ರ ಗೃಹ ಕಾರಾಗೃಹ; ಮೋಹ ಮಹಾ ಸಂಕೋಲೆ. ನಿನ್ನ ಸಂಬಂಧವಾದೊಡನೆಯೆ ಇವೇ ಮೋಕ್ಷಸಾಧನಗಳು. ಹೇ ಸಕಲ ಜಗನ್ನಾಯಕ, ಸರ್ವಸಾಕ್ಷಿ, ಶ್ರೀಕೃಷ್ಣ, ನಿನಗೆ ನಮೋ ನಮೋ. ಸಜ್ಜನ ಸಮುದ್ರದ ಪೂರ್ಣಚಂದ್ರ! ಅಧರ್ಮದ ಅಂಧಕಾರಕ್ಕೆ ಸೂರ್ಯನಂತಿರುವ ಶ್ರೀಕೃಷ್ಣ ಪರಮಾತ್ಮಾ! ನಿನಗೆ ನಮೋ ನಮೋ! ನಿನ್ನ ಪಾದಗಳೆಂಬ ತೆಪ್ಪಗಳನ್ನು ಆಶ್ರಯಿಸಿದವರು ಸಂಸಾರಸಾಗರವನ್ನು ಸುಲಭವಾಗಿ ದಾಟಬಲ್ಲರು. ಪರಮಪದವೇ ನಿನ್ನ ವಿಹಾರಸ್ಥಾನ, ನನ್ನನ್ನು ಕಾಪಾಡು.’

ಬ್ರಹ್ಮನ ಸ್ತೋತ್ರದಿಂದ ಸಂತುಷ್ಟನಾದ ಪರಬ್ರಹ್ಮ ಆತನಿಗೆ ಮತ್ತೆ ಗೋಚರನಾಗಿ ತನ್ನ ಕಾರ್ಯದಲ್ಲಿ ನಿರಾತಂಕವಾಗಿ ನಿರತನಾಗುವಂತೆ ಹೇಳಿ, ಆತನನ್ನು ಬೀಳ್ಕೊಟ್ಟನು. ಇತ್ತ ಗೋಗಳೂ ಗೋಪಾಲಕರೂ ಬ್ರಹ್ಮಮಾಯೆಯಿಂದ ಬಿಡುಗಡೆ ಹೊಂದಿ, ಮತ್ತೆ ಸೇರಿ ದರು. ಆ ವೇಳೆಗೆ ಒಂದು ವರ್ಷದಷ್ಟು ಕಾಲ ಕಳೆದುಹೋಗಿತ್ತಾದರೂ ಶ್ರೀಕೃಷ್ಣಮಾಯೆ ಯಿಂದ ಅದು ಅವರಿಗೆ ಅರ್ಥವಾಗಲೆ ಇಲ್ಲ. ಅವರು ಮೊದಲು ಕುಳಿತಲ್ಲಿಯೇ ಕುಳಿತು ಕೃಷ್ಣನನ್ನು ನಿರೀಕ್ಷಿಸುತ್ತಿದ್ದರು. ಆ ವೇಳೆಗೆ ಕೃಷ್ಣನು ತುರುಗಳೊಡನೆ ಹಿಂದಿರುಗಿದ. ಸಂಜೆ ಯಾದುದರಿಂದ ಎಲ್ಲರೂ ಊರಿಗೆ ಹಿಂದಿರುಗಿದರು. ಗೋಪಾಲರು ತಮ್ಮ ತಾಯ್ತಂದೆ ಗಳೊಡನೆ ಶ್ರೀಕೃಷ್ಣನು ಹೆಬ್ಬಾವನ್ನು ಕೊಂದ ಕಥೆಯನ್ನು ಅಂದೆ ನಡೆದುದೆಂಬಂತೆ ವರ್ಣಿಸಿ ಹೇಳಿದರು. ಈ ಮಧ್ಯೆ ಬ್ರಹ್ಮನ ಹಮ್ಮು ಮುರಿದು ಮುಕ್ಕಾಗಿದ್ದುದನ್ನು ಅವರೇನು ಬಲ್ಲರು?

