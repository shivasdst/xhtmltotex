
\chapter{೩೮. ಶಕುಂತಲೆ}

ಯಯಾತಿಯ ಮಗ ಪೂರು ಮಹಾ ಪುಣ್ಯಶಾಲಿ. ಆತನ ವಂಶದಲ್ಲಿ ಅನೇಕ ರಾಜಪುಷಿ ಗಳೂ ಬ್ರಹ್ಮಪುಷಿಗಳೂ ಹುಟ್ಟಿ ಪ್ರಖ್ಯಾತರಾದರು. ಅವರಲ್ಲಿ ದುಶ್ಯಂತನೆಂಬುವನು ಬಹು ಮುಖ್ಯನಾದವನು. ಈತನು ಭರತಖಂಡದ ಅಖಂಡ ಚಕ್ರವರ್ತಿಯಾಗಿದ್ದು, ಬಾನವರಲ್ಲಿಯೂ ಮಾನವರಲ್ಲಿಯೂ ತನಗೆ ಸಮಾನರಾದ ಶಕ್ತಿವಂತರಿಲ್ಲವೆಂದು ಕೀರ್ತಿ ಯನ್ನು ಪಡೆದಿದ್ದನು. ಒಮ್ಮೆ ಈತನು ಬೇಟೆಯಾಡುತ್ತಾ ಆಡುತ್ತಾ ಕಣ್ವಪುಷಿಗಳ ಆಶ್ರಮಕ್ಕೆ ಹೋದನು. ಅಲ್ಲಿ ಮೋಹಿನಿಯಂತೆ ಮನೋಹರಳಾದ ರೂಪಸಿಯೊಬ್ಬಳು ಆತನ ಕಣ್ಣಿಗೆ ಬಿದ್ದಳು. ಮೊದಲ ನೋಟದಲ್ಲಿಯೇ ಆತನಿಗೆ ಅವಳಲ್ಲಿ ಮೋಹ ಹುಟ್ಟಿತು. ಆತನು ಆಕೆಯ ಬಳಿಗೆ ಹೋದ. ಆತ ನಗುನಗುತ್ತಾ ಮೃದುವಾದ ದನಿಯಲ್ಲಿ ‘ಎಲೆ, ಕಮಲವರ್ಣದ ಹೆಣ್ಣೆ, ನೀನು ಯಾರ ಮಗಳು? ಎಲ್ಲಿಂದ ಬಂದೆ? ಜನಸಂಚಾರವಿಲ್ಲದ ಈ ಅಡವಿಗೆ ನೀನೊಬ್ಬಳೇ ಏಕೆ ಬಂದೆ? ಹೇ ಬಡನಡುವಿನ ಬಿಂಕಗಾರ್ತಿ, ನೀನು ಪುಷಿಗಳ ಆಶ್ರಮದಲ್ಲಿದ್ದರೂ ಬ್ರಾಹ್ಮಣರ ಹುಡುಗಿಯಂತೂ ಅಲ್ಲ. ನೀನು ರಾಜಕುಮಾರಿಯೇ ಇರಬೇಕು. ಇಲ್ಲದಿದ್ದರೆ ನನ್ನ ಮನಸ್ಸು ಎಂದೆಂದಿಗೂ ನಿನ್ನ ಕಡೆ ಹರಿಯುತ್ತಿರಲಿಲ್ಲ. ಪೂರು ವಂಶದ ರಾಜರ ಮನಸ್ಸು ಎಂದೆಂದಿಗೂ ಅಧರ್ಮಕ್ಕೆ ಹೋಗುವುದಿಲ್ಲ’ ಎಂದು ಹೇಳಿದ. ಅವನ ಸ್ಪಷ್ಟವಾದ ಮಾತುಗಳಿಗೆ ಅಷ್ಟೇ ಸ್ಪಷ್ಟವಾಗಿ ಆ ಹೆಣ್ಣು ಉತ್ತರಕೊಟ್ಟಳು. ‘ಅಯ್ಯಾ ವೀರ, ನಿನ್ನ ಊಹೆ ನಿಜ. ನಾನು ವಿಶ್ವಾಮಿತ್ರಮಹರ್ಷಿಯಿಂದ ಮೇನಕೆಯಲ್ಲಿ ಹುಟ್ಟಿದವಳು. ನನ್ನ ತಾಯಿ ಮಗುವಾಗಿದ್ದ ನನ್ನನ್ನು ಇಲ್ಲಿ ಬಿಟ್ಟು ಹೋದಳಂತೆ. ನನ್ನ ಕಥೆಯೆಲ್ಲ ನನಗಿಂತ ಚೆನ್ನಾಗಿ ಕಣ್ವ ಮಹರ್ಷಿಗಳಿಗೆ ಗೊತ್ತು. ಅದು ಹಾಗಿರಲಿ. ಈಗ ನನ್ನಿಂದ ನಿನಗೇನಾಗಬೇಕು? ಬಾ, ಇಲ್ಲಿ ಕುಳಿತುಕೊ. ಇಲ್ಲಿ ಬೇಕಾದಷ್ಟು ನೀವಾರಧಾನ್ಯ ವಿದೆ. ಊಟಕ್ಕೇನೂ ಅಭಾವವಿಲ್ಲ, ಬೇಕಾದರೆ ಹೊಟ್ಟೆತುಂಬ ಊಟಮಾಡು. ನಿನಗೆ ಒಪ್ಪಿಗೆಯಾದರೆ ನೀನು ಇಲ್ಲಿಯೇ ಉಳಿದುಕೊಳ್ಳಬಹುದು’ ಎಂದಳು.

ಶಕುಂತಲೆಯ ಮಾತಿನ ಧಾಟಿಯನ್ನು ನೋಡಿದ ದುಷ್ಯಂತನಿಗೆ ಆ ಹೆಣ್ಣೂ ತನ್ನಂತೆಯೇ ಮೋಹ ಪರವಶಳಾಗಿರಬೇಕೆನ್ನಿಸಿತು. ಆದ್ದರಿಂದ ಧೈರ್ಯವಾಗಿ ಮಾತನ್ನು ಮುಂದುವರಿಸಿದ–‘ಸುಂದರಿ, ಮಹಾನುಭಾವನಾದ ವಿಶ್ವಾಮಿತ್ರನ ಮಗಳಿಗೆ ತಕ್ಕಂತೆಯೆ ನಿನ್ನ ನಡವಳಿಕೆ ವಿನಯ ವಿಶ್ವಾಸಗಳಿಂದ ಕೂಡಿದೆ. ತುಂಬ ಸಂತೋಷ. ನಿನ್ನ ಸವಿನುಡಿ ಗಳಿಂದಲೆ ನನ್ನ ಹೊಟ್ಟೆ ತುಂಬಿತು. ಈಗ ನಿನ್ನ ವಿಚಾರವಾಗಿಯೆ ಒಂದು ಮಾತನ್ನು ಹೇಳಬೇಕೆನ್ನಿಸುತ್ತದೆ. ರಾಜಕನ್ಯೆಯರು ತಮಗೆ ಇಷ್ಟಬಂದವನನ್ನು ಗಂಡನನ್ನಾಗಿ ಸ್ವೀಕರಿಸಬಹುದು ಎಂದು ಹೇಳುತ್ತಾರೆ. ನೀನೂ ರಾಜಕನ್ಯೆಯಾದುದರಿಂದ ನಿನಗೆ ಇಷ್ಟ ವಾದ ಪಕ್ಷದಲ್ಲಿ ನನ್ನನ್ನು ಸ್ವೀಕರಿಸಬಹುದಲ್ಲ!’ ಎಂದನು. ಅವಳಿಗೆ ಬೇಕಾಗಿದ್ದುದೂ ಅದೆ. ಆಕೆ ‘ಅಗತ್ಯವಾಗಿ ಆಗಬಹುದು’ ಎಂದಳು. ಅವರಿಬ್ಬರೂ ಗಾಂಧರ್ವವಿವಾಹದಿಂದ ಆ ಕ್ಷಣವೇ ಸತಿಪತಿಗಳಾದರು. ದುಷ್ಯಂತನು ಒಂದು ರಾತ್ರಿ ಆಶ್ರಮದಲ್ಲಿ ತಂಗಿದ್ದು ಮರುದಿನ ರಾಜಧಾನಿಗೆ ಹಿಂದಿರುಗಿದನು. ಇದಾದ ಕೆಲವು ದಿನಗಳಲ್ಲಿಯೆ ಶಕುಂತಲೆ ಯಲ್ಲಿ ಗರ್ಭಚಿಹ್ನೆಗಳು ಕಾಣಿಸಿ ಕೊಂಡವು. ನವಮಾಸ ತುಂಬುತ್ತಲೆ ಆಕೆ ಸುಂದರನಾದ ಒಬ್ಬ ಸುಕುಮಾರನನ್ನು ಹೆತ್ತಳು. ಅವನು ಶುಕ್ಲಪಕ್ಷದ ಚಂದ್ರನಂತೆ ದಿನದಿನಕ್ಕೂ ಬೆಳೆದು ಮಹಾತೇಜಸ್ವಿಯಾದನು. ಆ ಎಳೆಯ ಬಾಲಕನ ಆಟವೆಂದರೆ ಸಿಂಹಗಳನ್ನು ಎಳೆತಂದು ಕಟ್ಟಿಹಾಕುವುದು, ಅದನ್ನು ಎಳೆದಾಡುವುದು, ಹೆದರಿಸಿ ಓಡಿಸುವುದು. ಕೆಲಕಾಲದ ಮೇಲೆ ಶಕುಂತಲೆ ಅದ್ವಿತೀಯನಾದ ಈ ಬಾಲಕನನ್ನು ಕರೆದುಕೊಂಡು ಗಂಡನ ಬಳಿಗೆ ಹೋದಳು. ಆದರೆ ಬಹುಕಾಲ ಕಳೆದುಹೋಗಿದ್ದುದರಿಂದ ದುಷ್ಯಂತನಿಗೆ ಅವಳ ಗುರುತೇ ಮರೆತುಹೋಗಿತ್ತು. ಅವರನ್ನು ಸ್ವೀಕರಿಸಲು ಆತ ಒಪ್ಪಲಿಲ್ಲ. ಆಗ ಅಲ್ಲಿದ್ದವರೆಲ್ಲ ಕೇಳು ವಂತೆ ಆಕಾಶದಿಂದ ಒಂದು ದನಿ ಕೇಳಿಬಂತು–‘ಮಹಾರಾಜ, ಈ ಬಾಲಕನು ನಿನ್ನ ಮಗನೇ ನಿಜ; ಸಂದೇಹ ಬೇಡ. ಈ ಶಕುಂತಳೆಯೂ ಅಷ್ಟೆ, ಮಹಾಪತಿವ್ರತೆಯಾದ ಈಕೆ ನಿನ್ನ ಮಡದಿ; ಈಕೆಗೆ ಅವಮಾನಮಾಡಬೇಡ; ಈಕೆಯನ್ನು ಸ್ವೀಕರಿಸು. ಈ ಮಗ ಸಾಮಾನ್ಯನೆಂದು ಭಾವಿಸಬೇಡ. ಈತನು ಈಶ್ವರಾಂಶ, ಸತ್ಯಸಂಧ, ನಿನ್ನ ವಂಶೋ ದ್ಧಾರಕ. ಇವನನ್ನು ಭರಿಸು.’ ಆಕಾಶವಾಣಿ ‘ಭರಿಸು’ ಎಂದುದರಿಂದ ಈತನ ಹೆಸರು ಭರತ ಎಂದಾಯಿತು. ದುಷ್ಯಂತನು ಆ ಮಡದಿ ಮಕ್ಕಳನ್ನು ಸಂತೋಷದಿಂದ ಸ್ವೀಕರಿಸಿದನು.

ದುಷ್ಯಂತನ ಕಾಲಾನಂತರ ಭರತನು ಪಟ್ಟಾಭಿಷಿಕ್ತನಾಗಿ ‘ನ ಭೂತೋ ನ ಭವಿಷ್ಯತಿ’ ಎನ್ನುವಂತೆ ವೈಭವದಿಂದ ರಾಜ್ಯಭಾರ ಮಾಡಿದನು. ಗಂಗಾ ಯಮುನಾ ನದಿಗಳ ಉದ್ದಕ್ಕೂ ಅನೇಕ ಸ್ಥಳಗಳಲ್ಲಿ ಇನ್ನೂರೈವತ್ತು ಅಶ್ವಮೇಧಯಾಗಗಳನ್ನು ಮಾಡಿದನು. ಆತನ ಯಾಗದಲ್ಲಿ ಬ್ರಾಹ್ಮಣರು ತಮ್ಮ ಮನದಣಿಯುವಂತೆ ಗೋವುಗಳನ್ನೂ ದಕ್ಷಿಣೆ ಯನ್ನೂ ಪಡೆದರು. ಮುಷ್ಕಾರವೆಂಬ ಒಂದು ಕ್ಷೇತ್ರದಲ್ಲಿಯೆ ಆತನು ಮೂರು ಸಾವಿರದ ಮೂನ್ನೂರು ಅಶ್ವಮೇಧಯಾಗಗಳನ್ನು ಮಾಡಿ, ಚಿನ್ನದಿಂದ ಅಲಂಕರಿಸಿದ ಹದಿನಾಲ್ಕು ಲಕ್ಷ ಆನೆಗಳನ್ನು ದಾನ ಮಾಡಿದನಂತೆ! ಆತನಂತೆ ಇತರರು ಮಾಡ ಹೊರಡುವುದೆಂದರೆ ತೋಳುಗಳಿಂದ ಸ್ವರ್ಗವನ್ನು ಆಲಂಗಿಸುವೆನೆಂಬ ಎಗ್ಗನಂತಾದಾರು. ಆತನು ತನ್ನ ದಿಗ್ವಿಜಯದಲ್ಲಿ ಧರ್ಮದ್ವೇಷಿಗಳಾದ ಕಿರಾತ, ಯವನ, ಹೂಣ ಮೊದಲಾದ ರಾಜರನ್ನು ನಿರ್ನಾಮಮಾಡಿದನು. ಹಿಂದೆ ದಾನವರು ಪಾತಾಳಕ್ಕೆ ಹೊತ್ತುಕೊಂಡು ಹೋಗಿದ್ದ ದೇವತಾ ಸ್ತ್ರೀಯರನ್ನೆಲ್ಲ ಆತ ಮತ್ತೆ ಕರೆತಂದು ದೇವತೆಗಳಿಗೆ ಒಪ್ಪಿಸಿದನು. ಆತನ ಕಾಲದಲ್ಲಿ ಭೂಮಿ ಕೇಳಿದುದನ್ನು ಬೆಳೆಯುತ್ತಿತ್ತು, ಆಕಾಶ ಕೇಳಿದಾಗ ಮಳೆ ಸುರಿಸುತ್ತಿತ್ತು. ಆತನು ಇಪ್ಪತ್ತೇಳು ಸಹಸ್ರವರ್ಷ ಚಕ್ರವರ್ತಿಪದವಿಯನ್ನು ಅನುಭವಿಸಿ, ಕಡೆಗೆ ತಪಸ್ಸಿನಿಂದ ಮುಕ್ತಿಯನ್ನು ಸಾಧಿಸಿದನು. ಭರತನಿಗೆ ಮೂವರು ಪತ್ನಿಯರಲ್ಲಿ ಮೂವರು ಮಕ್ಕಳಾಗಿದ್ದರಾದರೂ ಅವರ ತಾಯಿಯರ ಅವಿವೇಕದಿಂದ ಅವರು ಸತ್ತುಹೋದರು. ಆದ್ದರಿಂದ ವಂಶವನ್ನು ಬೆಳೆಸುವುದಕ್ಕೋಸ್ಕರ ಮರುತ್ತು ಗಳು ತಮ್ಮ ಬಳಿಯಿದ್ದ ಭರದ್ವಾಜನೆಂಬುವನನ್ನು ಆತನಿಗೆ ಮಗನಾಗಿ ಒಪ್ಪಿಸಿದರು.

