
\chapter{೨೩. ಜಡಭರತ ಋಷಿ}

ಋಷಭದೇವನ ಮಗನಾದ ಭರತನು ತಂದೆಯಂತೆಯೆ ದೊಡ್ಡ ದೈವಭಕ್ತ. ಆತನು ಇಹವನ್ನು ದೂರದೆ, ಪರವನ್ನು ಕೈಬಿಡದೆ ಧರ್ಮದಿಂದ ರಾಜ್ಯಭಾರಮಾಡುತ್ತಾ, ‘ಪಂಚಜನಿ’ ಎಂಬ ಸುಂದರಿ ಯನ್ನು ಮದುವೆಯಾಗಿ, ಸರ್ವವಿಧದಲ್ಲಿಯೂ ತನಗೆ ಸಮಾನರಾದ ಐವರು ಮಕ್ಕಳನ್ನು ಪಡೆದನು. ತನ್ನ ಹಿರಿಯರಂತೆ ಆತನೂ ಅನೇಕ ಯಜ್ಞಯಾಗಾದಿಗಳನ್ನು ಮಾಡಿದನು. ಕೇವಲ ದೇವರ ಪೂಜೆಯೆಂಬ ಭಾವನೆಯಿಂದ ಅವುಗಳನ್ನು ಮಾಡಿದನೇ ಹೊರತು, ಆತನಿಗೆ ಯಾವ ಭೋಗ ಭಾಗ್ಯಗಳ ಆಸೆಯೂ ಇರಲಿಲ್ಲ. ಯಜ್ಞದಲ್ಲಿ ಬೇರೆಬೇರೆ ದೇವತೆಗಳನ್ನು ಕುರಿತು ಹೇಳುವ ಮಂತ್ರ ಗಳನ್ನೆಲ್ಲ ಆತನು ಸರ್ವೇಶ್ವರನ ಮಂತ್ರವೆಂದೇ ಭಾವಿಸುತ್ತಿದ್ದ. ಹೀಗೆ ಆತನು ಹತ್ತು ಸಹಸ್ರವರ್ಷ ಗಳವರೆಗೆ ಧರ್ಮದಿಂದ ಬಾಳುತ್ತಿದ್ದು ತನ್ನ ಕರ್ಮಗಳನ್ನೆಲ್ಲ ಸವೆಸಿದನು. ತಾನು ಮಾಡುವ ಪ್ರತಿಯೊಂದು ಕೆಲಸವೂ ಭಗವಂತನ ಪೂಜೆಯೆಂದು ಭಾವಿಸಿದ ಆತನ ಮನಸ್ಸು ಪರಿ ಶುದ್ಧವಾಯಿತು. ಆತನ ಭಕ್ತಿ ಬೆಳೆದು ಭಕ್ತಿಯೋಗವಾಯಿತು. ಆತನು ತನ್ನ ರಾಜ್ಯವನ್ನು ಮಕ್ಕಳಿಗೆ ಹಂಚಿಕೊಟ್ಟು, ಪುಲಹ ಋಷಿಯ ಆಶ್ರಮಕ್ಕೆ ಹೊರಟುಹೋದನು.

ಪುಲಹಾಶ್ರಮವು ಬಹು ಪವಿತ್ರವಾದುದು. ಅಲ್ಲಿ ಹರಿಯುವ ಗಂಡಕೀ ನದಿಯಲ್ಲಿ ಚಕ್ರಾಂಕಿತವಾದ ಸಾಲಿಗ್ರಾಮಗಳು ತುಂಬಿರುವುದರಿಂದ ಅದಕ್ಕೆ ಹರಿಕ್ಷೇತ್ರವೆಂಬ ಹೆಸರು ಬಂದಿದೆ. ಅಲ್ಲಿ ನೆಲಸಿರುವವರು ಯಾವ ರೂಪದಿಂದ ಧ್ಯಾನಿಸಿದರೆ ಆ ರೂಪದಿಂದ ಭಗವಂತ ಅವರಿಗೆ ಪ್ರತ್ಯಕ್ಷನಾಗುತ್ತಾನೆ. ಭರತನು ಗಂಡಕೀನದಿಯ ತೀರದಲ್ಲಿ ತನ್ನ ಆಶ್ರಮವನ್ನು ಮಾಡಿಕೊಂಡು, ತಪೋನಿರತನಾದನು. ಮೂರುಹೊತ್ತು ನದಿಯಲ್ಲಿ ಸ್ನಾನ, ಹೂ ತುಳಸಿಗಳಿಂದ ದೇವರ ಪೂಜೆ, ಗೆಡ್ಡೆಗೆಣಸುಗಳ ಆಹಾರ, ಸದಾ ದೇವರ ಧ್ಯಾನ–ಇದು ಆತನ ದಿನಚರಿ. ದಿನದಿನಕ್ಕೂ ಆತನ ಮನಸ್ಸು ಏಕಾಗ್ರವಾಗುತ್ತ ಹೋಯಿತು. ಆತನ ಅಂತರಂಗ ಆನಂದದಿಂದ ತುಂಬಿ, ಕಣ್ಣುಗಳಿಂದ ಆನಂದಬಾಷ್ಪ ಸುರಿಯುತ್ತಿತ್ತು; ಮೈ ರೋಮಾಂಚಗೊಳ್ಳುತ್ತಿತ್ತು. ಆತನು ದೈವಸಾಕ್ಷಾತ್ಕಾರ ಮಾಡಿ ಕೊಳ್ಳುವ ದಿನ ಹತ್ತಿರವಾಗುತ್ತಿತ್ತು.

ಭರತಮುನಿಯು ಪ್ರತಿದಿನವೂ, ಬೆಳಗ್ಗೆ ಸೂರ್ಯ ಮೂಡುತ್ತಲೆ ಸೂರ್ಯನ ಇದಿರಿನಲ್ಲಿ ನಿಂತು ಆ ಸೂರ್ಯಮಂಡಲದ ಮಧ್ಯದಲ್ಲಿ ಬಂಗಾರದ ಬಣ್ಣದಿಂದ ನೆಲಸಿರುವ ಭಗವಂತ ನನ್ನು ಕುರಿತು “ಪರೋರಜಃ ಸವಿತುರ್ಜಾತವೇದೋ ದೇವಸ್ಯ ಭರ್ಗೋ ಮನಸೇದಂ ಜಜಾನ । ಸುರೇತ ಸಾದಃ ಪುನರಾವಿಶ್ಯ ಚಷ್ಟೇ ಹಂಸಂ ಗೃಧ್ರಾಣಂ ನೃಷದ್ರಿಂಗಿ ರಾ ಮಿಮಃ”\footnote{೧. ತನ್ನ ಸಂಕಲ್ಪ ಮಾತ್ರದಿಂದಲೆ ಈ ಜಗತ್ತನ್ನು ಸೃಷ್ಟಿಸಿ, ಅಂತರ್ಯಾಮಿಯಾಗಿ ಅದರಲ್ಲಿ ವ್ಯಾಪಿಸಿ, ವಿಷಯ ಸುಖಗಳನ್ನು ಬಯಸುವ ಜೀವರನ್ನು ಆಯಾ ಕಾರ್ಯಗಳಲ್ಲಿ ನೇಮಿಸಿ, ತಾನು ಕೇವಲ ಸಾಕ್ಷೀಭೂತನಾಗಿ ನೋಡುತ್ತ, ಶುದ್ಧ ಸತ್ವರೂಪನಾಗಿ, ಆ ಜೀವಿಗಳಿಗೆ ಕರ್ಮಫಲವನ್ನು ಕೊಡುತ್ತಾ ಇರುವ ಸೂರ್ಯ ಸ್ವರೂಪವಾದ ತೇಜಸ್ಸಿಗೆ ಶರಣಾಗುತ್ತೇನೆ.} ಎಂಬ ವೇದದ ಪುಕ್ಕಿನಿಂದ ಸ್ತೋತ್ರಮಾಡುತ್ತಿದ್ದನು. ಒಮ್ಮೆ ಆತನು ಹೀಗೆ ಸೋತ್ರಮಾಡಿ ಓಂಕಾರವನ್ನು ಜಪಿಸುತ್ತಾ ಕುಳಿತಿರುವಾಗ, ಒಂದು ಹೆಣ್ಣು ಜಿಂಕೆ ನೀರು ಕುಡಿಯಲೆಂದು ಬಂದು ನದಿಗೆ ಇಳಿಯಿತು. ಅದು ನೀರು ಕುಡಿಯುತ್ತಿರುವಾಗ ಹತ್ತಿರ ದಲ್ಲಿಯೇ ಎಲ್ಲಿಯೋ ಒಂದು ಸಿಂಹವು ಭಯಂಕರವಾಗಿ ಗರ್ಜಿಸಿದುದು ಕೇಳಿಬಂದಿತು. ಸಹಜವಾಗಿಯೇ ಅಂಜುಗುಳಿಯಾದ ಜಿಂಕೆ, ಸಿಂಹದ ಗರ್ಜನೆಯಿಂದ ಮತ್ತಷ್ಟು ಭಯ ಗೊಂಡು, ನೀರನ್ನೂ ಕುಡಿಯದೆ ತಟ್ಟನೆ ಮೇಲಕ್ಕೆ ಹಾರಿ ನದಿಯಿಂದ ಆಚೆಗೆ ಓಡಿ ಹೋಗಲು ಯತ್ನಿಸಿತು. ಪೂರ್ಣ ಗರ್ಭಿಣಿಯಾಗಿದ್ದ ಆ ಜಿಂಕೆ, ಹೆದರಿಕೆಯಿಂದ ಹಾರಿದ ರಭಸಕ್ಕೆ ಗರ್ಭಸ್ರಾವವಾಗಿ ಅದರ ಮರಿ ಹೊಳೆಯಲ್ಲಿ ಬಿತ್ತು. ತಾಯಿ ಜಿಂಕೆ ದಡವೇರಿ ಓಡುತ್ತಾ ಸತ್ತುಬಿತ್ತು; ಮರಿ ನೀರಿನಲ್ಲಿ ಕೊಚ್ಚಿ ಹೋಗುತ್ತಿತ್ತು.

ಜಪದಲ್ಲಿ ಕುಳಿತಿದ್ದ ಭರತಋಷಿ ಹೊಳೆಯಲ್ಲಿ ತೇಲಿಕೊಂಡು ಹೋಗುತ್ತಿದ್ದ ಜಿಂಕೆಯ ಮರಿಯನ್ನು ಕಂಡು, ಕರುಣೆಯಿಂದ ಅದನ್ನು ನೀರಿನಿಂದೆತ್ತಿ ತನ್ನ ಆಶ್ರಮಕ್ಕೆ ಕೊಂಡೊ ಯ್ದನು. ಆತನ ಈ ದಯಾಗುಣವೇ ಆತನ ದುರ್ಗತಿಗೆ ಕಾರಣವಾಯಿತು. ವೈರಾಗ್ಯದಿಂದ ಎಲ್ಲವನ್ನೂ ತ್ಯಜಿಸಿದ್ದ ಆತನಿಗೆ, ತಾನು ಬದುಕಿಸಿದ ಜಿಂಕೆಯ ಮರಿಯಲ್ಲಿ ಅಪಾರವಾದ ಮಮತೆ ಹುಟ್ಟಿತು. ದಿನವೂ ಕಾಲಕಾಲಕ್ಕೆ ಸರಿಯಾಗಿ ಅದಕ್ಕೆ ಹುಲ್ಲು ನೀರುಗಳನ್ನು ಕೊಡು ವುದು, ಕಾಡುಮೃಗಗಳ ಕಾಟವಿಲ್ಲದಂತೆ ಅದನ್ನು ಕಾಪಾಡುವುದು, ಹೊಟ್ಟೆಯ ಮಗನಂತೆ ಆಗಾಗ ಅದನ್ನು ಎತ್ತಿ ಮುದ್ದಾಡುವುದು, ಮೈ ತುರಿಸುವುದು, ಅದಕ್ಕೆ ಯಾವುದು ಹಿತ ವೆಂದು ಯೋಚಿಸುವುದು–ಹೀಗಾಗಿ ಆತನ ಸ್ನಾನ, ಪೂಜೆ, ಧ್ಯಾನಗಳು ಹಿಂದೆ ಬಿದ್ದು, ಕ್ರಮೇಣ ನಿಂತೇ ಹೋದವು. ಆತನಿಗೆ ಸದಾ ಜಿಂಕೆಯದೇ ಧ್ಯಾನ: “ಅಯ್ಯೋ, ಪಾಪ, ಈ ಜಿಂಕೆಗೆ ನಾನಲ್ಲದೆ ಇನ್ನಾರಿದ್ದಾರೆ? ಅದಕ್ಕೆ ನಾನೇ ದಿಕ್ಕು; ಅದು ನನ್ನನ್ನೇ ನಂಬಿಕೊಂಡಿದೆ; ನಾನೆಂದರೆ ಅದಕ್ಕೆ ಪ್ರಾಣ; ಆದ್ದರಿಂದ ಅದರ ರಕ್ಷಣೆ ನನ್ನ ಧರ್ಮ” ಹೀಗೆಂದುಕೊಂಡ, ಆತ. ಕುಳಿತಿರಲಿ, ಮಲಗಿರಲಿ, ಓಡಾಡುತ್ತಿರಲಿ ಆತನಿಗೆ ಅದರದೇ ಯೋಚನೆ. ತಾನು ಹೂವನ್ನೊ ದರ್ಭೆಯನ್ನೂ ತರಲು ಹೋದಾಗ, ಯಾವ ಮೃಗದಿಂದ ಅದಕ್ಕೆ ಯಾವ ಹಿಂಸೆಯಾಗುವುದೋ ಎಂದು ಹೆದರಿ, ಅದನ್ನೂ ತನ್ನ ಜೊತೆಯಲ್ಲಿಯೇ ಕರೆದುಕೊಂಡು ಹೋಗುವನು. ಅದು ನಡೆಯಲಾರದೆ ನಿಂತರೆ ಅದನ್ನು ಹೆಗಲಿನ ಮೇಲೆ ಹೊತ್ತುಕೊಂಡು ಹೋಗುವನು. ಅದನ್ನು ತೊಡೆಯ ಮೇಲಿಟ್ಟುಕೊಂಡು ತಟ್ಟುವನು, ಎದೆಗವಚಿಕೊಂಡು ಮುದ್ದಾಡುವನು. ಅದು ಕ್ಷಣ ಕಾಲ ಕಣ್ಮರೆಯಾದರೆ, ಅದಕ್ಕೆ ಯಾವ ಅಪಾಯ ವಾಗಿದೆಯೋ ಎಂಬ ಭಯದಿಂದ ಆತ ತಲ್ಲಣಿಸುವನು. ‘ಅಯ್ಯೋ, ದಿಕ್ಕಿಲ್ಲದ ಆ ಜಿಂಕೆಯಮರಿ ಏನಾಯಿತೊ! ನಾನು ಸಾಧುವೆಂದು ನನ್ನ ಬಳಿ ಸೇರಿದ್ದುದನ್ನು ನಾನು ನನ್ನ ಮೈಮರೆವಿನಿಂದ ದೂರ ಮಾಡಿದೆನಲ್ಲಾ! ದೇವರ ದಯದಿಂದ ಅದು ಸಾಯದೆ, ಇಲ್ಲಿಯೇ ಹತ್ತಿರದಲ್ಲಿ ಎಲ್ಲಿಯಾದರೂ ಮೇಯುತ್ತಿದ್ದರೆ ಸಾಕಲ್ಲಾ! ಇನ್ನೇನು ಗತಿ, ಲೋಕಕ್ಕೆ ರಕ್ಷಕ ನಾದ ಸೂರ್ಯನು ಮುಳುಗುತ್ತಿದ್ದಾನೆ, ಆ ಮುದ್ದುಮರಿ ಯಾವ ಮೃಗದ ಬಾಯಿಗೆ ತುತ್ತಾಗುತ್ತದೊ! ಅದರ ತಾಯಿ ಅದನ್ನು ನನಗೊಪ್ಪಿಸಿ ಸತ್ತುಹೋಯಿತು. ಅಂದಿನಿಂದ ಅದು ನನ್ನನ್ನೇ ನಂಬಿಕೊಂಡಿತ್ತು. ಪಾಪ ಅದಕ್ಕೆ ನಾನೆಂದರೆ ಜೀವ. ನಾನು ತಪಸ್ಸು ಮಾಡುವವನಂತೆ ಕಣ್ಣುಮುಚ್ಚಿ ಕುಳಿತರೆ ಅದು ಮೆಲ್ಲಮೆಲ್ಲನೆ ನನ್ನ ಬಳಿಗೆ ಬಂದು, ತನ್ನ ಮೃದುವಾದ ಕೊಂಬಿನಿಂದ ನನ್ನನ್ನು ಸವರುತ್ತಿತ್ತು. ಒಮ್ಮೊಮ್ಮೆ ಅದು ಹವಿಸ್ಸನ್ನು ಎಂಜಲುಮಾಡಿದಾಗ, ನಾನು ಸ್ವಲ್ಪ ಗದರುತ್ತಲೆ, ಋಷಿಕುಮಾರನಂತೆ ನಮ್ರತೆಯಿಂದ ಮೌನವಾಗಿ ನಿಲ್ಲುತ್ತಿತ್ತು” ಎಂದು ಆತ ಹಲುಬಿ ಹಂಬಲಿಸುತ್ತಿದ್ದನು. ಅದನ್ನು ಹುಡು ಕುತ್ತಾ ಪರ್ಣಶಾಲೆಯಿಂದ ಹೊರಗೆ ಬಂದು ‘ಆಹಾ, ನನ್ನ ಮುದ್ದುಮರಿಯ ಹೆಜ್ಜೆಯ ಗುರುತುಗಳಿಂದ ಈ ಭೂದೇವಿ ಪಾವನಳಾದಳು. ಇದು ಯಜ್ಞಕ್ಕೆ ಯೋಗ್ಯವಾದ ಭೂಮಿ ಯಾಯಿತು’ ಎಂದುಕೊಳ್ಳುತ್ತಿದ್ದನು. ಆತನು ಭ್ರಾಂತನಾಗಿ ಆಕಾಶದತ್ತ ನೋಡುತ್ತಾ ‘ಓಹೋ, ದಯಾಳುವಾದ ಚಂದ್ರನು ತಬ್ಬಲಿಯಾದ ನನ್ನ ಮರಿಯನ್ನು ತನ್ನಲ್ಲಿಟ್ಟು ಕೊಂಡಿದ್ದಾನೆ’ ಎನ್ನುತ್ತಿದ್ದನು.

ಹೀಗೆ ಹುಲ್ಲೆಯ ಮರಿಯೆ ಆತನಿಗೆ ಪ್ರಾರಬ್ಧಕರ್ಮವಾಗಿ ಪರಿಣಮಿಸಿತು. ಆತನ ಭಕ್ತಿ ಯೋಗ ಬಯಲಾಯಿತು, ಆತನ ಮನಸ್ಸು ಸದಾ ಜಿಂಕೆಯಲ್ಲಿಯೇ ಆಸಕ್ತವಾಯಿತು. ಹೀಗಿರಲು, ಇಲಿಯ ಬಿಲವನ್ನು ಹೊಗುವ ಹಾವಿನಂತೆ ಮೃತ್ಯುವು ಆತನ ದೇಹವನ್ನು ಪ್ರವೇಶಿಸಿತು. ಆತನು ಸಾಯುವ ಕಾಲದಲ್ಲಿ ಆ ಜಿಂಕೆಯ ಮರಿಯು ಹೊಟ್ಟೆಯಲ್ಲಿ ಹುಟ್ಟಿದ ಮಗನಂತೆ ಆತನ ಪಕ್ಕದಲ್ಲಿಯೇ ಕುಳಿತು ಕಣ್ಣೀರಿಡುತ್ತಿತ್ತು. ಅದನ್ನು ನೋಡುತ್ತಾ, ಅದರಲ್ಲಿ ನಟ್ಟ ಮನಸ್ಸುಳ್ಳವನಾಗಿ ಆತ ಪ್ರಾಣಬಿಟ್ಟ. ಇದರ ಫಲವಾಗಿ ಆತ ಮರುಜನ್ಮದಲ್ಲಿ ಜಿಂಕೆಯಾಗಿ ಹುಟ್ಟಬೇಕಾಯಿತು. ಆದರೂ ಆತನು ಪೂರ್ವಜನ್ಮದಲ್ಲಿ ಮಾಡಿದ್ದ ದೇವರ ಪೂಜೆಯ ಫಲದಿಂದ, ಆತನಿಗೆ ಈ ಜನ್ಮ ಬಂದುದರ ಕಾರಣ ಗೊತ್ತಾ ಯಿತು. ಆಗ ಮನಸ್ಸಿನಲ್ಲಿ ‘ಅಯ್ಯೋ ನಾನೆಂತಹ ಅವಿವೇಕಿ! ವೈರಾಗ್ಯದಿಂದ ಎಲ್ಲವನ್ನೂ ತೊರೆದು ದೇವರ ಧ್ಯಾನದಲ್ಲಿ ಮುಳುಗಿದ್ದವನು, ಜಿಂಕೆಯ ಮರಿಗೆ ಮರುಳಾಗಿ ಹಾಳಾದೆ!’ ಎಂದು ಪಶ್ಚಾತ್ತಾಪಪಟ್ಟು, ಒಡನೆಯೇ ತನ್ನ ಮಂದೆಯಿಂದ ಅಗಲಿ ಪುಲಹಾಶ್ರಮಕ್ಕೆ ಹೊರಟುಬಂದಿತು. ಅಲ್ಲಿ ತರಗೆಲೆಗಳನ್ನು ತಿಂದುಕೊಂಡು ಜೀವಿಸುತ್ತಾ, ತನ್ನ ಕಡೆಗಾಲ ಸಮೀಪಿಸುತ್ತಲೆ ತನ್ನ ದೇಹದ ಅರ್ಧಭಾಗವನ್ನು ಗಂಡಕೀನದಿಯಲ್ಲಿ ಮುಳುಗಿಸಿ ಕೊಂಡು, ಕಡೆಯುಸಿರನ್ನು ಎಳೆಯಿತು. ಈ ಜನ್ಮದಲ್ಲಿ ತನ್ನ ಪ್ರಾರಬ್ಧಕರ್ಮವನ್ನು ಮುಗಿ ಸಿದ್ದ ಆ ಜಿಂಕೆ ಮರುಜನ್ಮದಲ್ಲಿ ಬ್ರಾಹ್ಮಣಕುಮಾರನಾಗಿ ಜನಿಸಿತು.

ಭರತನ ತಂದೆ ಅಂಗೀರಸನಿಗೆ ಇಬ್ಬರು ಹೆಂಡಿರು. ಹಿರಿಯ ಹೆಂಡತಿಯ ಹೊಟ್ಟೆಯಲ್ಲಿ ಒಂಬತ್ತು ಜನ ಮಕ್ಕಳು. ಕಿರಿಯ ಹೆಂಡತಿಯಲ್ಲಿ ಭರತ ಮತ್ತು ಅವನ ತಂಗಿ. ಹಿರಿಯ ಹೆಂಡತಿಯ ಮಕ್ಕಳೆಲ್ಲ ಬುದ್ಧಿವಂತರು, ವಿದ್ಯಾವಂತರು. ಆದರೆ ಭರತನಿಗೆ ವಿದ್ಯೆಯ ಕಡೆ ಒಲವಿರಲಿಲ್ಲ. ಆತನಿಗೆ ಪೂರ್ವಜನ್ಮ ಸ್ಮರಣೆಯಿತ್ತು. ಆದ್ದರಿಂದ ಪುನಃ ಸಂಸಾರದ ಮಧ್ಯೆ ಸಿಕ್ಕಿಕೊಂಡು, ಹುಟ್ಟುಸಾವುಗಳ ಬವಣೆಯನ್ನು ಅನುಭವಿಸುವುದೇಕೆಂದು, ಆತ ಸದಾ ಒಂಟಿಯಾಗಿ ಇರುತ್ತಿದ್ದ. ತಾನು ಕಿವುಡನಂತೆ, ಹುಚ್ಚನಂತೆ ಆತ ನಟಿಸುತ್ತಿದ್ದ. ಇದನ್ನು ನಿಜವೆಂದೇ ಭಾವಿಸಿ, ಅವನ ತಂದೆ ಅವನಿಗಾಗಿ ಮನಸ್ಸಿನಲ್ಲಿ ಮರುಗುತ್ತಿದ್ದ. ಮುಂದೆ ಅವನಿಗೆ ಮದುವೆಯಾಗದಿದ್ದರೂ ಚಿಂತೆಯಿಲ್ಲ, ಬ್ರಾಹ್ಮಣ ಕರ್ಮಗಳನ್ನಾ ದರೂ ಮಾಡಿಕೊಂಡಿರಲಿ–ಎಂದು ಯೋಚಿಸಿ, ತಕ್ಕ ವಯಸ್ಸಿನಲ್ಲಿಯೇ ಅವನಿಗೆ ಮುಂಜಿ ಯನ್ನು ಮಾಡಿದ. ಆದರೇನು? ವಟುವಾದ ಭರತನಿಗೆ ಮಂತ್ರಗಳನ್ನು ಹೇಳಲು ಬಾಯೇ ತಿರುಗುತ್ತಿರಲಿಲ್ಲ. ನಾಲ್ಕು ತಿಂಗಳು ಒಂದೇ ಸಮನಾಗಿ ಸಂತೆ ಹೇಳಿಸಿದರೂ ಅವನ ಬಾಯಿಂದ ಗಾಯತ್ರೀಮಂತ್ರ ಹೇಳಿಸುವುದು ಅಸಾಧ್ಯವಾಯಿತು. ನಿದ್ರೆಹೋದವರನ್ನು ಎಬ್ಬಿಸಬಹುದು, ನಿದ್ರಿಸುತ್ತಿರುವಂತೆ ನಟಿಸುವವರನ್ನು ಎಬ್ಬಿಸಲು ಸಾಧ್ಯವೆ? ಪೂರ್ವ ಜನ್ಮದ ಸ್ಮರಣೆಯಿದ್ದ ಅವನಿಗೆ ಎಲ್ಲ ಕರ್ಮಗಳೂ ಕರತಲಾಮಲಕ, ಜೊತೆಗೆ ಅವು ವ್ಯರ್ಥವೆಂಬುದೂ ಸುಸ್ಪಷ್ಟ. ಆದ್ದರಿಂದ ಅವನ ತಂದೆಯ ಪ್ರಯತ್ನವೆಲ್ಲ ನೀರಿನಲ್ಲಿ ಹೋಮಮಾಡಿದಂತಾಯಿತು. ಆ ತಂದೆಯ ಕಷ್ಟವನ್ನು ಕಂಡು ಯಮನಿಗೂ ಕರುಣೆ ಹುಟ್ಟಿರಬೇಕು. ಆತ ಅವನ ಜೀವವನ್ನು ಕೊಂಡೊಯ್ದು, ಅವನ ಕಷ್ಟವನ್ನು ನೀಗಿದ. ಕಿರಿಯ ಹೆಂಡತಿ ತನ್ನ ಮಕ್ಕಳನ್ನು ಹಿರಿಯಳಿಗೊಪ್ಪಿಸಿ, ತಾನು ಗಂಡನೊಡನೆ ಸಹಗಮನ ಮಾಡಿದಳು. ಹೀಗೆ ಭರತ ಬಾಲ್ಯದಲ್ಲಿಯೇ ತಬ್ಬಲಿಯಾದ. ಅಪ್ಪನಿಗೇ ಅರ್ಥವಾಗದ ಭರತ ಅಣ್ಣಂದಿರಿಗೆ ಅರ್ಥವಾಗುತ್ತಾನೆಯೇ? ಅವರು ಅವನನ್ನು ‘ಎರಡುಕಾಲಿನ ಪಶು’ ಎಂದುಕೊಂಡರು, ಅವನ ಹಾದಿಗೆ ಅವನನ್ನು ಬಿಟ್ಟರು. ಅವನಿಗೆ ಬೇಕಾಗಿದ್ದುದೂ ಅದೇ. ಸದಾ ಆತ್ಮಾನಂದದಲ್ಲಿ ಮುಳುಗಿದ್ದ ಅವನಿಗೆ ಚಳಿ ಬಿಸಿಲುಗಳು ಗಮನವಿಲ್ಲ; ಹುಚ್ಚನಂತೆ, ಬೆಪ್ಪನಂತೆ, ಬರಿಮೈಲಿ ತಿರುಗಾಡುತ್ತಿದ್ದ; ಬರಿಯ ನೆಲದ ಮೇಲೆ ಮಲಗು ತ್ತಿದ್ದ; ಮೃಷ್ಟಾನ್ನವೋ, ಕದನ್ನವೋ–ಯಾರು ಏನನ್ನು ಕೊಟ್ಟರೆ ಅದನ್ನು ತಿನ್ನುತ್ತಿದ್ದ. ಸೊಂಟದ ಮೇಲೊಂದು ಹರಕು ಬಟ್ಟೆ, ಎಣ್ಣೆಗಾಣದೆ ಜಡೆಗಟ್ಟಿದ ತಲೆ, ಕೊಳೆತುಂಬಿ ಕಪ್ಪಾದ ಜನಿವಾರ–ಇವು ಜನರ ಹಾಸ್ಯಕ್ಕೆ ವಸ್ತುಗಳಾಗಿದ್ದವು. ಜನ ಅವನನ್ನು ‘ಹುಚ್ಚು ಹಾರುವ’, ‘ಜಡಭರತ’ ಎಂದು ಗೇಲಿ ಮಾಡುವರು. ಸಾಣೆಯಿಕ್ಕದ ರತ್ನ ಕೋತಿಗಳ ಕೈಯ ಮಾಣಿಕ್ಯವಾಗಿತ್ತು. ಮೈ ಕೈ ತುಂಬಿಕೊಂಡು ಗೂಳಿಯಂತೆ ತಿರುಗುತ್ತಿದ್ದ ಆ ಒರಟನನ್ನು ಜನ ಬಿಟ್ಟಿಯ ಕೆಲಸಕ್ಕೆ ಹಿಡಿದು ದುಡಿಸಿಕೊಳ್ಳುವರು. ಯಾರು ಯಾವ ಕೆಲಸವನ್ನು ಹೇಳಿ ದರೂ ಆತ ತುಟಿಪಿಟಕ್ಕೆನ್ನದೆ ಅದನ್ನು ಮಾಡಿ ಮುಗಿಸುವನು; ಅವರು ‘ಪಾಪ’ ಎಂದು ಕೊಟ್ಟಿದನ್ನು ತಿಂದು ತೃಪ್ತಿ ಪಡುವನು.

ಕಂಡಕಂಡವರಿಗೆಲ್ಲ ದುಡಿಯುತ್ತಿರುವ ‘ಜಡಭರತ’ನನ್ನು ಕಂಡು ಅವನ ಅಣ್ಣಂದಿರ ಬಾಯಲ್ಲಿ ನೀರೂರಿತು. ಅವರು ಅವನನ್ನು ತಮ್ಮ ಗದ್ದೆಯ ಕೆಲಸಕ್ಕೆ ನೇಮಿಸಿದರು. ಅವರು ಕೊಟ್ಟ ಕಾಳುಕಡ್ಡಿಯನ್ನೊ, ಅನ್ನದ ಸೀಕನ್ನೊ–ಕೊನೆಗೆ ಹಿಂಡಿತೌಡುಗಳನ್ನೊ– ತಿಂದುಕೊಂಡು, ಅವನೂ ತನಗೆ ತೋಚಿದಂತೆ ಕೆಲಸಮಾಡಿಕೊಂಡಿದ್ದನು. ಹೀಗಿರಲು ಒಂದು ರಾತ್ರಿ ಕೆಲವು ಕಳ್ಳರು ಬಂದು ಅವನನ್ನು ಹಿಡಿದುಕೊಂಡರು. ಅವರ ಒಡೆಯ ಮಹಾಕಾಳಿಗೆ ಹರಸಿಕೊಂಡು ಒಬ್ಬ ಮಗನನ್ನು ಪಡೆದಿದ್ದ; ಹರಕೆಗೆ ಒಂದು ನರಬಲಿ ಕೊಡಬೇಕಾಗಿತ್ತು; ಅದಕ್ಕೆ ತಕ್ಕ ಒಬ್ಬ ಮನುಷ್ಯನನ್ನು ಅವರು ಹುಡುಕಿಕೊಂಡು ಹೊರಟಿ ದ್ದರು; ಜಡಭರತನನ್ನು ಕಾಣುತ್ತಲೆ ಅವರಿಗೆ ತುಂಬ ಆನಂದವಾಯಿತು. ದಷ್ಟಪುಷ್ಟ ನಾಗಿದ್ದ ಅವನ ಬಲಿಯಿಂದ ತಮ್ಮ ದೇವತೆ ತೃಪ್ತಳಾಗುವಳೆಂದು ಅವರು ಕುಣಿದಾಡಿ ದರು. ಹಗ್ಗದಿಂದ ಕಟ್ಟಿ, ಅವನನ್ನು ಕಾಳಿಯ ಗುಡಿಗೆ ಕೊಂಡೊಯ್ದರು. ಅಲ್ಲಿ ಅವನ ತಲೆಯ ಮೇಲೆ ಒಂದು ಕೊಡ ನೀರು ಸುರಿದು, ಹೊಸಬಟ್ಟೆಯನ್ನು ಉಡಿಸಿದರು; ಅವನ ಮೈಗೆಲ್ಲ ಗಂಧವನ್ನು ಬಳಿದರು; ಕೊರಳಿಗೆ ಕೆಂಪು ಹೂವಿನ ಸರ ಹಾಕಿದರು; ಅವನಿಗೆ ಹೊಟ್ಟೆತುಂಬ ಮೃಷ್ಟಾನ್ನ ಹಾಕಿದರು; ಪೂಜಾ ಸಾಮಗ್ರಿಗಳನ್ನೆಲ್ಲಾ ಸಿದ್ಧಮಾಡಿಕೊಂಡು, ತಮಟೆ ಭೇರಿಗಳ ವಾದ್ಯಗಳೊಡನೆ ಹಾಡನ್ನು ಹಾಡುತ್ತಾ ಅವನನ್ನು ಎಳೆತಂದು ಕಾಳಿಯ ವಿಗ್ರಹದೆದುರಿಗೆ ನಿಲ್ಲಿಸಿದರು. ಆ ಕಳ್ಳರಲ್ಲಿಯೇ ಒಬ್ಬ ಪುರೋಹಿತನಾದ. ಅವನು ಕತ್ತಿ ಯನ್ನು ಕಾಳಿಯ ಮುಂದಿಟ್ಟು ಮಂತ್ರಗಳನ್ನು ಹೇಳಿದ. ಅನಂತರ ಆ ಕತ್ತಿಯನ್ನು ಕೈಗೆತ್ತಿಕೊಂಡು ‘ಜೈ ಕಾಳಿ’ ಎಂದು ಹೇಳಿ ಆ ಕತ್ತಿಯನ್ನು ಮೇಲಕ್ಕೆತ್ತಿದ. ಆಗ ಇದ್ದಕ್ಕಿ ದ್ದಂತೆ ಸಿಡಿಲು ಬಡಿದಂತಹ ಶಬ್ದವಾಯಿತು. ಕಾಳಿಕಾದೇವಿಯ ವಿಗ್ರಹ ಒಡೆದು ಸಿಡಿ ಯಿತು. ಅದರೊಳಗಿನಿಂದ ಘೋರರೂಪದ ಭದ್ರಕಾಳಿ ಮೇಲೆದ್ದು ನಿಂತಳು. ಕೋಪ ದಿಂದ ಆಕೆಯ ಕಣ್ಣುಗಳು ಕೆಂಡದುಂಡೆಯಂತಿದ್ದವು, ಹುಬ್ಬುಗಳು ಗಂಟಿಕ್ಕಿದ್ದುವು, ಆಕೆಯ ಕೋರೆದಾಡೆಗಳು ಕರಕರನೆಂದು ಒಂದರೊಡನೊಂದು ಮಸೆಯುತ್ತಿದ್ದವು; ಜಗತ್ತು ಹಿರಿದು ಹೋಗುವಂತೆ ಆಕೆ ಒಮ್ಮೆ ಗರ್ಜಿಸಿದಳು. ಮರುನಿಮಿಷವೇ ಆಕೆ ದಾಬು ಗಾಲನ್ನಿಟ್ಟು, ಕೊಲೆಗಡುಕನ ಕೈಲಿದ್ದ ಕತ್ತಿಯನ್ನು ಕಿತ್ತುಕೊಂಡು, ಅಲ್ಲಿದ್ದ ಕಳ್ಳರ ಕತ್ತುಗಳನ್ನೆಲ್ಲ ಕತ್ತರಿಸಿಹಾಕಿದಳು. ಆಕೆಯೂ ಆಕೆಯ ಪರಿವಾರದ ಭೂತಗಳೂ ಆ ಕಳ್ಳರ ನೆತ್ತರನ್ನು ಕುಡಿದು, ಅವರ ರುಂಡಗಳಿಂದ ಚಂಡಾಟವಾಡಿದರು. ಇಷ್ಟೆಲ್ಲ ಹಗರಣ ವಾದರೂ ಜಡಭರತನು ಏನೂ ನಡೆದೇ ಇಲ್ಲವೆಂಬಂತೆ ತಣ್ಣಗೆ ತೆಪ್ಪಗಿದ್ದನು. ಅಲ್ಲಿಂದ ನೇರವಾಗಿ ತನ್ನ ಹೊಲಕ್ಕೆ ಹಿಂದಿರುಗಿ, ತನ್ನ ದಿನಚರಿಯ ಕಾರ್ಯದಲ್ಲಿ ಮಗ್ನನಾದನು.

ಒಂದು ದಿನ ಜಡಭರತಮುನಿಯು ಇಕ್ಷುಮತೀನದಿಯ ತೀರದಲ್ಲಿ ಧ್ಯಾನಪರನಾಗಿ ಕುಳಿತಿದ್ದನು. ಸಿಂಧು ಸೌವೀರ ದೇಶಗಳ ರಾಜನಾದ ರಹೂಗಣನು ಕಪಿಲಮುನಿಯಿಂದ ತತ್ವೋಪದೇಶವನ್ನು ಪಡೆಯಬೇಕೆಂದು ಪಲ್ಲಕ್ಕಿಯಲ್ಲಿ ಕುಳಿತು ಅದೇ ಮಾರ್ಗವಾಗಿ ಬರು ತ್ತಿದ್ದನು. ಆದಷ್ಟು ಬೇಗ ಹೋಗಬೇಕೆಂದು ಆತನ ಆಸೆ. ಆದ್ದರಿಂದ ಬೋಯಿಗಳ ನಾಯಕನು ಇನ್ನಷ್ಟು ಜನರನ್ನು ಪಲ್ಲಕ್ಕಿ ಹೊರಲು ನೇಮಿಸಿಕೊಳ್ಳಬೇಕೆಂದು ಯೋಚಿಸು ತ್ತಿದ್ದನು. ಅವನ ಅದೃಷ್ಟಕ್ಕೆ ನದಿಯ ದಡದಲ್ಲಿ ಕುಳಿತಿದ್ದ ಜಡಭರತ ಕಾಣಿಸಿದ. ಆತನ ದೃಢಕಾಯವನ್ನು ಕಂಡು ಅವನಿಗೆ ಬಹಳ ಸಂತೋಷವಾಯಿತು. ಆತನನ್ನು ಎಳೆದು ತಂದು ಆತನ ಹೆಗಲ ಮೇಲೆ ಪಲ್ಲಕ್ಕಿಯನ್ನು ಹೇರಿದನು. ಈ ಉದ್ಯೋಗ ಭರತನಿಗೆ ಹೊಸದು; ಇತರರೊಡನೆ ಹೊಂದಿಕೊಂಡು ಹೆಜ್ಜೆಯಿಡುವುದು ಆತನಿಗೆ ಸಾಧ್ಯವಿಲ್ಲ. ಇದರಿಂದ ಪಲ್ಲಕ್ಕಿ ಕುಲುಕಾಡಲು ಪ್ರಾರಂಭವಾಯಿತು. ಒಳಗಿದ್ದ ರಾಜನಿಗೆ ಸಿಟ್ಟು ಬಂತು. ಆತನು ಬೋಯಿಗಳನ್ನು ಗದರಿಸಿದ. ಅವರು ತಪ್ಪನ್ನು ಜಡಭರತನ ಮೇಲೆ ಹಾಕಿದರು: ‘ಮಹಾ ಸ್ವಾಮಿ, ಈಗತಾನೆ ಹೆಗಲು ಕೊಟ್ಟಿದ್ದರೂ ಇವನು ಬಹಳ ಸೋತವನಂತೆ ಹೆಜ್ಜೆ ಹಾಕು ತ್ತಿದ್ದಾನೆ. ಆದ್ದರಿಂದಲೇ ಈ ಗೋಳು’ ಎಂದರು. ರಾಜ ಬಗ್ಗಿ ನೋಡಿದ. ಹೊಸಬ ಒಳ್ಳೆ ಧಾಂಡಿಗನಾಗಿದ್ದ. ಆದ್ದರಿಂದ ಆತ, ‘ಏನಪ್ಪಾ, ಬಹು ದೂರದಿಂದ ಹೊತ್ತು ಹೊತ್ತು ದಣಿದಿರುವೆಯಲ್ಲವೆ? ಅಯ್ಯೋ ಪಾಪ, ಬಡವಾಗಿ ಕಡ್ಡಿಯಾಗಿದ್ದೀಯೆ, ಮುಪ್ಪು ಬೇರೆ. ಹೇಗೆ ಹೊತ್ತೀಯೆ?’ ಎಂದು ಆತನನ್ನು ಮೂದಲಿಸಿದ. ಆದರೆ ಭರತಮುನಿ ಆ ಮಾತು ಗಳನ್ನು ಕಿವಿಗೆ ಹಾಕಿಕೊಳ್ಳಲೇ ಇಲ್ಲ. ತನ್ನ ಪಾಡಿಗೆ ತಾನು ಎಂದಿನಂತೆ ಮುಂದುವರೆ ಯುತ್ತಿದ್ದ. ಪಲ್ಲಕ್ಕಿಯ ಕುಲುಕಾಟ ಇನ್ನಷ್ಟು ಜಾಸ್ತಿಯಾಯಿತು. ರಾಜನು ಕೆರಳಿ ಕೆಂಗೆಂಡ ವಾಗಿ ‘ಏನೋ ಠೊಣಪ, ನಾನು ಹೇಳಿದ ಮಾತು ಅರ್ಥವಾಯಿತೋ ಇಲ್ಲವೋ? ನಿನ್ನ ನೆತ್ತಿಗೇರಿರುವ ಪಿತ್ಥವನ್ನು ಇಳಿಸಬೇಕೆಂದು ತೋರುತ್ತದೆ. ನಾನು, ಕೋಪಬಂದರೆ ಯಮಶಿಕ್ಷೆ ಕೊಡುತ್ತೇನೆ, ಜೋಕೆ’ ಎಂದನು.

ಪರಮಹಂಸನಾದ ಜಡಭರತಮುನಿಗೆ ರಾಜನ ಮಾತುಗಳನ್ನು ಕೇಳಿ ಕೋಪ ಬರಲಿಲ್ಲ; ಅಜ್ಞಾನಿಯಾದ ಅವನಲ್ಲಿ ಕನಿಕರ ಹುಟ್ಟಿತು. ಆತನು ಮುಗುಳ್ನಗುತ್ತಾ, ಮೃದುಮಧುರ ವಾಗಿ ‘ಮಹಾರಾಜ, ನಿನ್ನ ಮೂದಲೆ, ಕೋಪ, ಬೆದರಿಕೆ ಇವೆಲ್ಲ ವ್ಯರ್ಥ. ನಾನು ನಿನ್ನ ಪಲ್ಲಕ್ಕಿಯನ್ನು ಹೊತ್ತವನೂ ಅಲ್ಲ, ನಡೆದವನೂ ಅಲ್ಲ, ಬಳಲಿದವನೂ ಅಲ್ಲ. ಅವೆಲ್ಲ ದೇಹಕ್ಕೆ ಸಂಬಂಧಪಟ್ಟುವೇ ಹೊರತು ‘ನಾನು’ ಎಂಬ ಆತ್ಮನಿಗಲ್ಲ. ನೀನು ರಾಜನೆಂದು ತಿಳಿದುಕೊಂಡಿರುವುದೂ, ನನ್ನನ್ನು ಶಿಕ್ಷಿಸುವೆನೆಂದು ಹೇಳುವುದೂ ಶುದ್ಧ ಭ್ರಾಂತಿ. ರಾಜನಾರು, ಆಳಾರು? ಇದು ಎರಡು ದಿನದ ನಾಟಕ. ಇಂದು ಹೀಗಿರುವುದು ನಾಳೆ ಅದಲು ಬದಲಾಗಬಹುದು. ಆತ್ಮನ ಸ್ವರೂಪವನ್ನು ಮಾತ್ರ ಬದಲಾಯಿಸುವುದು ಯಾರಿಂದಲೂ ಸಾಧ್ಯವಿಲ್ಲ ಎಂದಮೇಲೆ ನೀನು ಯಾರನ್ನು ಶಿಕ್ಷಿಸುವೆ? ಈ ದೇಹವನ್ನು ತಾನೆ? ಹುಚ್ಚ ನಂತೆ, ಕಿವುಡನಂತೆ, ಮೂಗನಂತೆ ವ್ಯವಹರಿಸುತ್ತಾ, ಬ್ರಹ್ಮಾನಂದವನ್ನು ಅನುಭವಿಸುತ್ತಿರುವ ನಾನು ನಿನ್ನ ಶಿಕ್ಷೆಗೆ ಹೆದರುತ್ತೇನೆಯೆ?’ ಎಂದು ಹೇಳಿದನು.

ಜಡಭರತಮುನಿಯ ಬಾಯಿ ಮಾತನಾಡುತ್ತಿತ್ತು. ಆದರೆ ಕಾಲುಗಳು ಪಲ್ಲಕ್ಕಿಯನ್ನು ಹೊತ್ತು ಮುಂದುವರಿಯುತ್ತಿದ್ದವು. ಅಷ್ಟರಲ್ಲಿ ರಹೂಗಣನು ಪಲ್ಲಕ್ಕಿಯಿಂದ ಕೆಳಕ್ಕೆ ಧುಮುಕಿದನು. ಆತ ಹೊರಟಿದ್ದುದೇ ಆತ್ಮತತ್ವವನ್ನು ತಿಳಿಯಬೇಕೆಂದು. ಜಢಭರತ ಮುನಿಯ ಮಾತುಗಳು ತನ್ನ ಅಜ್ಞಾನವನ್ನು ಹೋಗಲಾಡಿಸುವ ಜ್ಞಾನಜ್ಯೋತಿಯಂತಿದ್ದವು. ರಾಜನು ಭಯದಿಂದ ನಡುಗುತ್ತಾ ಭರತನ ಪಾದಗಳ ಮುಂದೆ ಅಡ್ಡಬಿದ್ದನು, ಅನಂತರ ಕೈ ಜೋಡಿಸಿಕೊಂಡು ‘ಮಹಾತ್ಮಾ, ನೀನು ಯಾರು? ಯಾರ ಮಗ? ಯಾರ ಶಿಷ್ಯ? ಎಲ್ಲಿಂದ ಇಲ್ಲಿಗೆ ಬಂದೆ? ಏಕೆ ಬಂದೆ? ಬಿನಿನ್ನ ಜನಿವಾರದಿಂದ ನೀನು ಬ್ರಾಹ್ಮಣ ನೆಂಬುದು ನಿಜ. ಆದರೆ ಈ ವೇಶಾಂತರದಿಂದ ನೀನೇಕೆ ಸಂಚರಿಸುತ್ತಿರುವೆ? ನಾನು ಇಂದ್ರನ ವಜ್ರಾಯುಧಕ್ಕಾಗಲಿ, ಈಶ್ವರನ ತ್ರಿಶೂಲಕ್ಕಾಗಲಿ, ಯಮನ ದಂಡಕ್ಕಾಗಲಿ ಹೆದರುವುದಿಲ್ಲ; ಆದರೆ ಬ್ರಹ್ಮಜ್ಞಾನಿಯಾದ ಬ್ರಾಹ್ಮಣನಿಗೆ ನಾನು ಹೆದರುತ್ತೇನೆ. ನೀನು ಬ್ರಹ್ಮಜ್ಞನಾದರೂ ಹುಚ್ಚನಂತೆ ಒಬ್ಬೊಂಟಿಗನಾಗಿ ಅಲೆಡಾಡುತ್ತಿರುವೆ! ನಾನು ಅರಸಿ ಕೊಂಡು ಹೊರಟಿರುವ ಕಪಿಲಮುನಿಯೇ ನೀನಾಗಿರಬಹುದು? ಲೋಕದ ಜನರನ್ನು ಪರೀಕ್ಷಿಸುವುದಕ್ಕಾಗಿ ನೀನು ಈ ವೇಷದಿಂದ ಸಂಚರಿಸುತ್ತಿರಬಹುದೆ? ಸಂಸಾರಿಯಾಗಿ, ಅಜ್ಞಾನದಲ್ಲಿ ಮುಳುಗಿರುವ ನನಗೆ ಯೋಗಿಗಳ ಸ್ವರೂಪವನ್ನು ಅರಿಯುವುದು ಹೇಗೆ ಸಾಧ್ಯ? ನೀನು ಯಾರೆ ಆಗಿರು; ನನ್ನ ಸಂದೇಹಗಳನ್ನು ಹೋಗಲಾಡಿಸು’ ಎಂದು ಬೇಡಿಕೊಂಡನು.

ರಹೂಗಣನು ಭರತಮಹಾಮುನಿಯ ಮುಂದೆ ತನ್ನ ಸಂದೇಹಗಳನ್ನು ಮಂಡಿಸಿದನು: “ಸ್ವಾಮಿನ್, ನನ್ನ ಪಲ್ಲಕ್ಕಿಯ ಭಾರದಿಂದ ನೀವು ಆಯಾಸಗೊಳ್ಳಲಿಲ್ಲವೆಂದು ಹೇಳಿದಿರಿ; ಅದು ಹೇಗೆ ಸಾಧ್ಯ? ಆತ್ಮನಿಗೆ ಸ್ವಭಾವವಾಗಿ ಸುಖದುಃಖಗಳಿಲ್ಲವೆಂದಾದರೂ ದೇಹವನ್ನು ಧರಿಸಿದಾಗ ಸುಖದುಃಖಗಳು ಇರಲೇಬೇಕಲ್ಲವೆ? ಪಾಯಸಮಾಡುವಾಗ ಮೊದಲು ಪಾತ್ರೆ ಸುಡುತ್ತದೆ, ಆಮೇಲೆ ಹಾಲು ಕಾಯುತ್ತದೆ, ಅನಂತರ ಅದರಲ್ಲಿಯ ಅಕ್ಕಿಯ ಹೊರ ಭಾಗವೂ ಒಳಭಾಗವೂ ಬೇಯುತ್ತದೆ. ಅದರಂತೆ ಮೊದಲು ದೇಹಕ್ಕೆ, ಆಮೇಲೆ ಇಂದ್ರಿಯಗಳಿಗೆ, ಅನಂತರ ಪ್ರಾಣ ಮನಸ್ಸುಗಳಿಗೆ ಸುಖದುಃಖಗಳು ವ್ಯಾಪಿಸಿ, ಕೊನೆಗೆ ಆತ್ಮಕ್ಕೂ ವ್ಯಾಪಿಸುತ್ತದೆಯಲ್ಲವೆ? ಇದು ನನ್ನ ಮೊದಲ ಪ್ರಶ್ನೆ. ಎರಡನೆಯದಾಗಿ, ನೀವು ‘ರಾಜ-ಆಳು ಎಂಬ ವ್ಯತ್ಯಾಸವಿಲ್ಲ’ ಎಂದು ಹೇಳುವುದು ನನಗೆ ಅರ್ಥವಾಗುತ್ತಿಲ್ಲ. ಅವರ ಸಂಬಂಧ ನಿತ್ಯವಲ್ಲದಿರಬಹುದು. ಆದರೆ ರಾಜ್ಯವಾಳುವವನು ದುಷ್ಟರನ್ನು ಶಿಕ್ಷಿಸಿ, ಶಿಷ್ಟರನ್ನು ಕಾಪಾಡಬೇಕು. ಇದು ಭಗವಂತನ ಕಟ್ಟಳೆಯೆಂದು ನಾನು ಭಾವಿಸುತ್ತೇನೆ. ಎಂದ ಮೇಲೆ ಸ್ವಾಮಿ-ಸೇವಕ ಎಂಬ ವ್ಯತ್ಯಾಸ, ಶಿಕ್ಷೆ ಮಾಡುವವನು-ಮಾಡಿಸಿಕೊಳ್ಳು ವವನು–ಎಂಬ ವ್ಯತ್ಯಾಸ ಇರಲೇಬೇಕಲ್ಲವೆ?”

ಭರತಮುನಿಯು ರಹೂಗಣನ ಪ್ರಶ್ನೆಗಳಿಗೆ ಉತ್ತರಕೊಡುತ್ತಾ “ಅಯ್ಯಾ, ನೀನು ನಿನಗೆ ತಿಳಿಯದುದನ್ನೂ ತಿಳಿದವನಂತೆ ಮಾತಾಡುತ್ತಿ. ನಿನ್ನ ಹೇಳಿಕೆ ಅಜ್ಞಾನಿಗಳಿಗೆ ಒಪ್ಪಿಗೆಯಾಗ ಬಹುದು, ಬ್ರಹ್ಮಜ್ಞರಿಗಲ್ಲ. ನೀನು ಹೇಳಿದ ಪಾತ್ರೆ, ಹಾಲು, ಅಕ್ಕಿಗಳು ಒಂದೇ ಜಾತಿಗೆ– ಕಾವು ವ್ಯಾಪಿಸುವ ಜಾತಿಗೆ ಸೇರಿದವು; ಆದ್ದರಿಂದ ಒಂದರಿಂದ ಒಂದಕ್ಕೆ ಬಿಸಿ ವ್ಯಾಪಿಸುತ್ತಾ ಹೋಗುವುದು ಸಹಜ, ಆದರೆ ದೇಹ, ಇಂದ್ರಿಯ, ಮನಸ್ಸು ಇವುಗಳ ಜಾತಿಗೆ ಸೇರಿದು ದಲ್ಲ, ಆತ್ಮ. ಆದ್ದರಿಂದ ಅವುಗಳಿಗೆ ಆದ ನೋವು ಆತ್ಮಕ್ಕೆ ವ್ಯಾಪಿಸುವುದಿಲ್ಲ. ನೋವು ವ್ಯಾಪಿಸುವಂತಿದ್ದರೆ ಅದು ಆತ್ಮವೇ ಅಲ್ಲ. ದುಃಖಕ್ಕೆ ಮೂಲ ಕಾರಣ ಅಭಿಮಾನ. ಅದನ್ನೇ ‘ಮಾಯೆ’ಯೆನ್ನುತ್ತಾರೆ. ಮನಸ್ಸು ವಿಷಯವಸ್ತುಗಳ ಮೇಲಿನ ಅಭಿಮಾನ ಅಥವಾ ಮಾಯೆಯಿಂದ ಅವುಗಳತ್ತ ಒಲಿದಾಗ ಅದಕ್ಕೆ ನೋವು, ನಲಿವುಗಳ ಭಾವನೆ ಬರುತ್ತದೆ. ಮನಸ್ಸು ವಿಷಯಗಳ ಕಡೆ ಒಲಿಯದಿದ್ದಾಗ ನೋವು ನಲಿವುಗಳೊಂದೂ ಇಲ್ಲ. ನಿಗಿ ನಿಗಿ ಹೊಳೆಯುವ ಕೆಂಡಕ್ಕೆ ಹೊಗೆಯೆಲ್ಲಿ ಬಂತು? ಎಣ್ಣೆ ಬತ್ತಿಗಳು ಸೇರಿದರೆ ಮಾತ್ರ ಹೊಗೆ. ಹಾಗೆಯೆ ವಿಷಯಗುಣಗಳಲ್ಲಿ ಅನುರಕ್ತಿ ಬಂದಾಗ ಮಾತ್ರ ಸುಖದುಃಖಗಳು; ಆತ್ಮಾನುರಕ್ತಿಯಾದಾಗ ಸುಖವೂ ಇಲ್ಲ, ದುಃಖವೂ ಇಲ್ಲ.

“ಇನ್ನು ನಿನ್ನ ಎರಡನೆಯ ಪ್ರಶ್ನೆಯನ್ನು ತೆಗೆದುಕೊ. ನೀನು ರಾಜಧರ್ಮವನ್ನು ಕುರಿತು ಬಡಿವಾರವನ್ನು ಕೊಚ್ಚುತ್ತಿರುವೆಯಲ್ಲಾ, ನಿನ್ನದು ಎಂತಹ ರಾಜಧರ್ಮ? ಭಾರವನ್ನು ಹೊರಲಾರದೆ ಬಳಲಿ ಬೆಂಡಾಗಿರುವ ಈ ಬೆಸ್ತರನ್ನು ಬಿಟ್ಟಿ ಹಿಡಿದು, ಬಲವಂತದಿಂದ ಅವರನ್ನು ಗೋಳಾಡಿಸುತ್ತಿರುವೆಯಲ್ಲಾ! ಇದು ಯಾವ ಬಗೆಯ ಧರ್ಮ? ರಾಜನೆಂಬ ಅಹಂಕಾರ ನಿನ್ನಲ್ಲಿ ತುಂಬಿ ತುಳುಕುತ್ತಿದೆ. ಈ ದೇಹವೇ ನೀನೆಂದು ತಿಳಿದು, ಆ ದೇಹಾತ್ಮ ಭಾವನೆಯಿಂದಲೆ ನೀನು ರಾಜತ್ವವನ್ನು ನಿನ್ನಲ್ಲಿ ಆರೋಪಿಸಿಕೊಂಡಿರುವೆ. ಹೀಗೆ ನೀನು ರಾಜನೆಂದು ತಿಳಿಯುವುದೂ, ರಾಜನ ಕರ್ತವ್ಯವನ್ನು ಮಾಡುತ್ತೇನೆಂದು ತಿಳಿಯುವುದೂ ಆತ್ಮಸ್ವರೂಪವನ್ನು ಅರಿಯದ ಅಜ್ಞಾನಿಯ ಲಕ್ಷಣ. ಅಯ್ಯಾ ರಾಜ, ಆತ್ಮಜ್ಞಾನ ಸುಲಭ ವಲ್ಲ, ಮಹಾತ್ಮರ ಸೇವೆಯನ್ನು ಮಾಡಿ ಅದನ್ನು ಪಡೆಯಬೇಕು. ನೋಡು, ನಾನು ಪೂರ್ವ ಜನ್ಮದಲ್ಲಿ ಭರತನೆಂಬ ರಾಜನಾಗಿದ್ದೆ. ರಾಜಪದವಿಯನ್ನು ತ್ಯಜಿಸಿ, ದೇವರ ಧ್ಯಾನ ದಲ್ಲಿಯೇ ಮುಳುಗಿದ್ದ ನನಗೆ ಜಿಂಕೆಯ ಮರಿಯೊಂದರಲ್ಲಿ ಮೋಹ ಹುಟ್ಟಿ, ದೈವಭಕ್ತಿ ಕೈಬಿಟ್ಟಿತು; ಮರಣಕಾಲದಲ್ಲಿ ಅದನ್ನೇ ಚಿಂತಿಸುತ್ತಿದ್ದು, ಮುಂದಿನ ಜನ್ಮದಲ್ಲಿ ಜಿಂಕೆ ಯಾಗಿ ಹುಟ್ಟಿದೆ. ಆದರೆ ಪೂರ್ವಜನ್ಮದ ದೈವಭಕ್ತಿಯ ಫಲವಾಗಿ ಆ ಜಿಂಕೆಯ ಜನ್ಮ ದಲ್ಲಿಯೂ ನನಗೆ ಹಿಂದಿನ ಜನ್ಮದ ಸ್ಮರಣೆಯಿತ್ತು. ನಾನು ದೈವಭಾವನೆಯಲ್ಲಿದ್ದು ಸತ್ತು, ಈಗ ಬ್ರಾಹ್ಮಣ ಜನ್ಮವನ್ನು ಪಡೆದಿದ್ದೇನೆ. ಹಿಂದಿನ ಜನ್ಮದ ಜ್ಞಾನವಿರುವುದರಿಂದ ನಾನು ಈ ಜನ್ಮದಲ್ಲಿ ಯಾರ ಸಹವಾಸವನ್ನೂ ಮಾಡದೆ ಹುಚ್ಚನಂತಿದ್ದೇನೆ” ಎಂದು ಹೇಳಿದನು.

ಜಡಭರತಮಹಾಮುನಿಯ ಮಾತುಗಳಿಂದ ರಹೂಗಣನ ಅಜ್ಞಾನ ನೀಗಿತು. ಸಂಸಾರವು ದಟ್ಟ ವಾದ ಕಾಡಿನಂತೆ ಭಯಂಕರವಾದುದೆಂಬುದನ್ನು ಆ ಮಹಾತ್ಮನಿಂದ ವಿಸ್ತಾರವಾಗಿ ಕೇಳಿ ರಾಜನ ಕಣ್ಣುಗಳು ತೆರೆದವು. ಆತನ ದೇಹಾತ್ಮಾಭಿಮಾನ ಹಾರಿಹೋಯಿತು. ಆತನು ಪರಮಹಂಸನಾದ ಜಡಭರತನಿಗೆ ನಮಸ್ಕರಿಸಿ, ತನ್ನ ಪಯಣವನ್ನು ಮುಂದುವರಿಸಿದನು.

