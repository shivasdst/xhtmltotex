
\chapter{೭೩. ಆ ತಪ್ಪಿಗೆ ಈ ಶಿಕ್ಷೆ}

ಶ್ರೀಕೃಷ್ಣನ ಮಕ್ಕಳಾದ ಪ್ರದ್ಯುಮ್ನ, ಸಾಂಬ, ಗದ, ಚಾರುಭಾನು ಮೊದಲಾದವರು ಒಮ್ಮೆ ವನ ವಿಹಾರಕ್ಕೆಂದು ಊರ ಮುಂದಿನ ಉದ್ಯಾನವನಕ್ಕೆ ಹೋದರು. ಅಲ್ಲಿ ಮಧ್ಯಾಹ್ನದವರೆಗೂ ವಿನೋದದಿಂದ ವಿಹರಿಸುತ್ತಿದ್ದು, ಬಾಯಾರಿಕೆಯಾದುದರಿಂದ ನೀರನ್ನು ಹುಡುಕುತ್ತಾ ಹೊರಟರು. ಹಾದಿಯಲ್ಲಿ ಅವರಿಗೆ ಒಂದು ಹಾಳು ಬಾವಿ ಕಾಣಿ ಸಿತು. ಅದರಲ್ಲಿ ಇಣಿಕಿ ನೋಡಲು ಅದರೊಳಗೆ ಒಂದು ದೊಡ್ಡ ಓತಿಕೇತ ಕಾಣಿಸಿತು. ಅದರ ಗಾತ್ರವನ್ನು ಕಂಡು ಅವರಿಗೆ ಆಶ್ಚರ್ಯವಾಯಿತು. ಆ ಹಾಳು ಬಾವಿಯ ತಳದಲ್ಲಿ ಅದಕ್ಕೆ ಅತ್ತಿತ್ತ ಸರಿದಾಡುವುದಕ್ಕೂ ಅವಕಾಶವಿರಲಿಲ್ಲವಾದ್ದರಿಂದ ಅದನ್ನು ಕಂಡು ಕನಿಕರವೂ ಹುಟ್ಟಿತು. ಅದನ್ನು ಅಲ್ಲಿಂದ ಎತ್ತಿ ಹೊರಗೆ ಬಿಡೋಣವೆಂದುಕೊಂಡು, ಅವರು ದಪ್ಪನಾದ ಒಂದು ಹಗ್ಗವನ್ನು ಅದಕ್ಕೆ ಕಟ್ಟಿ ಮೇಲಕ್ಕೆಳೆದರು. ಆದರೆ, ಜಪ್ಪಯ್ಯ ಎಂದರೂ ಅದನ್ನು ಕದಲಿಸುವುದಕ್ಕಾಗಲಿಲ್ಲ. ಅವರೆಲ್ಲ ಮನೆಗೆ ಬಂದು ಶ್ರೀಕೃಷ್ಣನ ಮುಂದೆ ಆ ಸಮಾಚಾರವನ್ನು ಹೇಳಿದರು. ಆತ ಅವರೊಡನೆ ಆ ಹಾಳು ಬಾವಿಗೆ ಬಂದು, ತನ್ನ ಎಡಗೈಯಿಂದ ಅದನ್ನು ಲೀಲಾಜಾಲವಾಗಿ ಮೇಲಕ್ಕೆತ್ತಿ, ಹೊರಗೆ ಬಿಟ್ಟನು. ತಕ್ಷಣವೆ ಓತಿಕೇತ ಮಾಯಾವಾಗಿ ಅಲ್ಲೊಬ್ಬ ಮಹಾಪುರುಷ ಕಾಣಿಸಿಕೊಂಡ. ಒಳ್ಳೆ ಪುಟಕ್ಕೆ ಹಾಕಿದ ಚಿನ್ನದಂತೆ ಹೊಳೆಯುವ ದೇಹ, ಮೈಮೇಲೆ ದಿವ್ಯವಾದ ಪೀತಾಂಬರ, ಕೊರಳಲ್ಲಿ ಪುಷ್ಪ ಹಾರ, ತಲೆಯಮೇಲೆ ರತ್ನದ ಕಿರೀಟ; ನೋಡಿದರೆ ಆತನೊಬ್ಬ ದೇವತೆಯಂತೆ ಕಾಣುತ್ತಿ ದ್ದಾನೆ. ಆತ ಶ್ರೀಕೃಷ್ಣನಿಗೆ ಅಡ್ಡಬಿದ್ದು, ಕೈಜೋಡಿಸಿ, ನಿಂತು ಕೊಂಡನು. ಶ್ರೀಕೃಷ್ಣನು ಆಶ್ಚರ್ಯಗೊಂಡವನಂತೆ ‘ಅಯ್ಯಾ, ನೀನಾರು?’ ಎಂದು ಕೇಳಿದನು. ಆತ ನಮ್ರನಾಗಿ ತನ್ನ ಕಥೆಯನ್ನು ಹೇಳಿಕೊಂಡನು.

“ದೇವ ದೇವ, ನಿನಗೆ ತಿಳಿಯದುದು ಯಾವುದಿದೆ? ಆದರೂ ನಿನ್ನ ಅಪ್ಪಣೆಯಂತೆ ನನ್ನ ಕಥೆಯನ್ನು ಹೇಳುತ್ತೇನೆ. ನಾನು ಇಕ್ಷ್ವಾಕು ಮಹಾರಾಜನ ಮಗ. ನನ್ನ ಹಸರು ನೃಗ. ಭೂಮಿಯಲ್ಲಿ ಎಷ್ಟು ಮಣ್ಣಿನ ಕಣಗಳಿವೆಯೋ, ಆಕಾಶದಲ್ಲಿ ಎಷ್ಟು ನಕ್ಷತ್ರಗಳಿವೆಯೋ, ಮಳೆಯಲ್ಲಿ ಎಷ್ಟು ಹನಿಗಳಿವೆಯೋ ಅಷ್ಟು ಗೋವುಗಳನ್ನು ನಾನು ದಾನ ಮಾಡಿದ್ದೇನೆ. ಅವು ಅಂತಹ ಇಂತಹ ಗೋವುಗಳಲ್ಲ; ಶುಭ ಲಕ್ಷಣಗಳುಳ್ಳ ಪ್ರಾಯದ ಗೋವುಗಳು, ಬೇಕಾದಷ್ಟು ಹಾಲನ್ನು ಕರೆಯುವ ಗೋವುಗಳು. ಅವುಗಳ ಕೋಡುಗಳಿಗೆ ಬಂಗಾರದ ಗೊಣಸುಗಳನ್ನು ಹಾಕಿ, ಕಾಲುಗಳಿಗೆ ಬೆಳ್ಳಿಯ ಗೆಜ್ಜೆಗಳನ್ನು ಕಟ್ಟಿ, ಮೈಮೇಲೆ ಉತ್ತಮ ವಸ್ತ್ರಗಳನ್ನು ಹೊದಿಸಿ, ಕರುಗಳೊಡನೆ ದಾನಮಾಡಿದ್ದೇನೆ. ಆ ದಾನವನ್ನು ತೆಗೆದು ಕೊಂಡವರೂ ಸಾಮಾನ್ಯರಲ್ಲ; ವೇದವೇದಾಂತಗಳನ್ನು ಓದಿ, ತಪಸ್ಸನ್ನು ಮಾಡುತ್ತಾ, ಸಾಧುಸಜ್ಜನರಾಗಿ, ಬಡವರಾಗಿದ್ದ ಬ್ರಾಹ್ಮಣರನ್ನೆ ಹುಡುಕಿ ಅವರಿಗೆ ದಾನಮಾಡಿದ್ದೇನೆ. ಇಷ್ಟೇ ಅಲ್ಲ; ಯಾರು ಬಂದು ಏನನ್ನು ಕೇಳಿದರೂ, ಭೂಮಿ, ಮನೆ, ಆನೆ, ಕುದುರೆ, ಕನ್ನೆ–ಏನನ್ನೆ ಬಯಸಿದರೂ ಅದನ್ನು ದಾನಮಾಡಿದ್ದೇನೆ. ಅನೇಕ ಯಾಗಗಳನ್ನು ಮಾಡಿ ದ್ದೇನೆ. ಕೆರೆ, ಬಾವಿ, ಸತ್ರಗಳನ್ನು ಕಟ್ಟಿಸಿದ್ದೇನೆ. ಇಷ್ಟಾದರೂ ನನಗೆ ಗೊತ್ತಿಲ್ಲದಂತೆ ಒಂದು ತಪ್ಪು ನಡೆದುಹೋಯಿತು. ನಾನು ಯಾರೋ ಒಬ್ಬ ಬ್ರಾಹ್ಮಣನಿಗೆ ದಾನಮಾಡಿದ್ದ ಗೋವು ತಪ್ಪಿಸಿಕೊಂಡು ಬಂದು ನನ್ನ ಮಂದೆಯಲ್ಲಿ ಸೇರಿಕೊಂಡಿತ್ತು. ನಾನು ಅರಿಯದೆ ಅದನ್ನು ಮತ್ತೊಬ್ಬ ಬ್ರಾಹ್ಮಣನಿಗೆ ದಾನಮಾಡಿದೆ. ಆತ ಅದನ್ನು ತನ್ನ ಮನೆಗೆ ಕೊಂಡೊಯ್ಯುತ್ತಿರುವಾಗ, ಹಿಂದೆ ಅದನ್ನು ದಾನವಾಗಿ ಪಡೆದಿದ್ದಾತ ಅಕಸ್ಮಾತ್ತಾಗಿ ಅದನ್ನು ಕಂಡು, ಅದನ್ನು ತನ್ನದೆಂದು ಗುರುತು ಹಿಡಿದ. ಆಗತಾನೆ ನನ್ನಿಂದ ದಾನವಾಗಿ ಪಡೆದಿದ್ದಾತ ಅದನ್ನು ತನ್ನದೆಂದ. ಇಬ್ಬರೂ ‘ನನ್ನದು, ತನ್ನದು’ ಎಂದು ಸಾಧಿಸುತ್ತಾ ಅದನ್ನು ನನ್ನ ಬಳಿಗೆ ತಂದರು. ಅದಕ್ಕೆ ಬದಲಾಗಿ ಬೇರೆಯ ಹಸುವನ್ನು ಕೊಡುವೆನೆಂದರೆ ಇಬ್ಬರಲ್ಲಿ ಒಬ್ಬರೂ ಒಪ್ಪರು. ‘ಅಯ್ಯಾ, ನೀವು ಕೇಳಿದುದನ್ನು ಕೊಡುತ್ತೇನೆ, ಒಂದು ಲಕ್ಷ ಗೋವುಗಳನ್ನು ಕೊಡುತ್ತೇನೆ’ ಎಂದರೂ ಅವರು ಅದಕ್ಕೆ ಒಪ್ಪಲಿಲ್ಲ. ಅವರಿಬ್ಬರೂ ಹಿಡಿದ ಹಟ ಬಿಡದೆ, ಆ ಗೋವನ್ನು ನನ್ನಲ್ಲಿಯೇ ಬಿಟ್ಟು ಹೊರಟುಹೋದರು. ಇದಾದ ಕೆಲವು ದಿನಗಳಮೇಲೆ ನಾನು ಸತ್ತುಹೋದೆ. ಯಮಲೋಕಕ್ಕೆ ಹೋದ ನನ್ನನ್ನು ಕುರಿತು ಯಮನು ‘ಅಯ್ಯಾ, ನಿನ್ನ ಪುಣ್ಯ ಅಪಾರ; ಆದರೆ, ದಾನ ಕೊಟ್ಟುದನ್ನು ಮತ್ತೆ ತೆಗೆದು ಕೊಂಡ ಸಣ್ಣ ಪಾಪವೂ ನಿನ್ನ ಜೀವಕ್ಕೆ ಅಂಟಿಕೊಂಡಿದೆ. ನೀನು ಯಾವ ಫಲವನ್ನು ಮೊದಲು ಅನುಭವಿಸುವೆ? ಪಾಪವೊ, ಪುಣ್ಯವೊ?’ ಎಂದನು. ನಾನು ‘ಪಾಪಫಲವೇ ಮೊದಲು ತೀರಿಹೋಗಲಿ’ ಎಂದೆ. ಯಮ ‘ಹಾಗೆ ಆಗಲಿ’ ಎಂದ. ಆತ ಹಾಗೆ ಹೇಳುತ್ತಿ ದ್ದಂತೆ ನಾನು ಕೆಳಕ್ಕೆ ಬಿದ್ದೆ. ಹಾಗೆ ಬೀಳುವಷ್ಟರಲ್ಲಿ ನಾನು ಓತಿಕೇತನಾಗಿದ್ದೆ. ನಾನು ಮಾಡಿದ ಆ ತಪ್ಪಿಗೆ, ನನಗೆ ಈ ಶಿಕ್ಷೆಯಾಯಿತು. ಇಗೋ, ನಿನ್ನ ಹಸ್ತ ಸೋಕುತ್ತಲೆ ನನ್ನ ಪಾಪ ಪರಿಹಾರವಾಯಿತು. ನಾನಿನ್ನು ಪುಣ್ಯವನ್ನು ಅನುಭವಿಸುವುದಕ್ಕಾಗಿ ಸ್ವರ್ಗಕ್ಕೆ ಹೋಗುತ್ತೇನೆ. ನಿನ್ನನ್ನು ಮರೆಯದೆ ಧ್ಯಾನಮಾಡುವಂತಹ ಬುದ್ಧಿಯನ್ನು ದಯಪಾಲಿಸು’ ಎಂದು ಬೇಡಿದನು. ಆತ ಹಾಗೆ ಹೇಳುತ್ತಿರುವಾಗಲೆ ಆತನಿಗಾಗಿ ದೇವಲೋಕದಿಂದ ವಿಮಾನವೊಂದು ಇಳಿದು ಬಂತು. ಆತನು ಶ್ರೀಕೃಷ್ಣನ ಅಪ್ಪಣೆಯನ್ನು ಪಡೆದು, ಅದನ್ನೇರಿ ಹೊರಟು ಹೋದನು.

ನೃಗ ಮಹಾರಾಜನು ಅತ್ತ ಹೋಗುತ್ತಲೆ, ಇತ್ತ ಶ್ರೀಕೃಷ್ಣ ತನ್ನ ಸುತ್ತಲಿದ್ದವರನ್ನು ಕುರಿತು ‘ನೋಡಿದಿರೋ, ದತ್ತಾಪಹಾರದ ಪಾಪಫಲವನ್ನು! ಅದರಲ್ಲಿಯೂ ಸಜ್ಜನನಾಗಿ ರುವ ಬ್ರಹ್ಮಜ್ಞನ ಸ್ವತ್ತನ್ನು ಎತ್ತಿಹಾಕಿದರೆ ಅವನ ವಂಶವೆ ನಿರ್ವಂಶವಾಗುತ್ತದೆ’ ಎಂದು ಹೇಳಿದನು.

