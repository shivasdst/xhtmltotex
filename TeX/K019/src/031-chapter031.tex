
\chapter{೩೧. ಅಮೃತ ಹುಟ್ಟಿತು}

ತಾಮಸಮನ್ವಂತರದಲ್ಲಿ ನಡೆದ ಗಜೇಂದ್ರಮೋಕ್ಷವನ್ನು ಹೇಳಿದ ಮೇಲೆ ಶುಕ ಮುನಿಯು ಐದನೆಯದಾದ ರೈವತಮನ್ವಂತರದಲ್ಲಿ ಶ್ರೀಹರಿಯು ವರಾಹರೂಪದಿಂದ ಹಿರಣ್ಯಾಕ್ಷನನ್ನು ಕೊಂದ ಕಥೆಯನ್ನು ಪರೀಕ್ಷಿತನಿಗೆ ಜ್ಞಾಪಿಸಿ, ಆರನೆಯ ಮನ್ವಂತರದ ಹರಿಲೀಲೆಯನ್ನು ಬಣ್ಣಿಸಹೊರಟನು. “ಮಹಾರಾಜ, ಭೂಮಿಯಲ್ಲಿರುವ ಮಣ್ಣಿನ ಕಣ ಗಳನ್ನಾದರೂ ಎಣಿಸಬಹುದು, ಭಗವಂತನ ಕಲ್ಯಾಣ ಗುಣಗಳನ್ನು ಮಾತ್ರ ಸಮಗ್ರವಾಗಿ ತಿಳಿಸಲು ಸಾಧ್ಯವಿಲ್ಲ. ಪ್ರತಿಯೊಂದು ಮನ್ವಂತರದಲ್ಲಿಯೂ ಆತನು ಎಷ್ಟೋ ಬಾರಿ ಅವತರಿಸಿ ದಿವ್ಯಲೀಲೆಗಳನ್ನು ಮೆರೆದಿದ್ದಾನೆ. ಆರನೆಯದಾದ ಚಾಕ್ಷುಷ ಮನ್ವಂತರದಲ್ಲಿ ಆತ ನಡೆಸಿದ ಲೀಲೆಯಂತೂ ಅದ್ಭುತಗಳಲ್ಲಿ ಅದ್ಭುತ. ಈ ಮನ್ವಂತರದಲ್ಲಿ ಆತನು ವೈರಾಜನ ಮಡದಿ ಸಂಭೂತಿಯಲ್ಲಿ ಅಜಿತನೆಂಬ ಹೆಸರಿನಿಂದ ಜನಿಸಿ, ಹಾಲಿನ ಸಮುದ್ರ ವನ್ನು ಕಡೆದು, ದೇವತೆಗಳಿಗೆ ಅಮೃತವನ್ನು ದೊರಕಿಸಿಕೊಟ್ಟನು. ಅಮೃತದಂತೆ ರುಚಿ ಕರವಾಗಿಯೂ ಶ್ರೇಯಸ್ಕರವಾಗಿಯೂ ಇರುವ ಆ ಕಥೆಯನ್ನು ವಿಸ್ತಾರವಾಗಿ ಹೇಳುತ್ತೇನೆ, ಕೇಳು. 

“ಸ್ವರ್ಗದಲ್ಲಿ ರಾಜನಾಗಿದ್ದ ದೇವೇಂದ್ರನು ಒಂದು ದಿನ ತನ್ನ ರಾಜಧಾನಿಯ ಬೀದಿ ಗಳನ್ನು ನೋಡುವುದಕ್ಕಾಗಿ ಐರಾವತವನ್ನು ಏರಿ ಹೊರಟನು. ಆತನಿಗೆ ಇದಿರಾಗಿ ಬರು ತ್ತಿದ್ದ ದುರ್ವಾಸ ಪುಷಿಯು ಇಂದ್ರನನ್ನು ಕಾಣುತ್ತಲೆ ನಿಂತು, ಆತನನ್ನು ಆಶೀರ್ವದಿಸಿ, ತನ್ನ ಕೊರಳಲ್ಲಿದ್ದ ಹೂವಿನ ಹಾರವನ್ನು ಆತನಿಗೆ ಪ್ರಸಾದವಾಗಿ ಕೊಟ್ಟನು. ಅಧಿಕಾರ ಐಶ್ವರ್ಯಗಳ ಮದದಿಂದ ತುಂಬಿ ತುಳುಕುತ್ತಿದ್ದ ದೇವೇಂದ್ರ ಆ ಪ್ರಸಾದವನ್ನು ಬೇಕಾಬಿಟ್ಟಿಯಾಗಿ ತೆಗೆದುಕೊಂಡು, ಅದನ್ನು ಆನೆಯ ನೆತ್ತಿಯ ಮೇಲಕ್ಕೆ ಎಸೆದನು. ಆನೆ ಅದನ್ನು ಸೊಂಡಲಿನಿಂದ ತೆಗದುಕೊಂಡು ಕಾಲಲ್ಲಿ ಹೊಸಕಿ ಹಾಕಿತು. ತನ್ನ ಕಣ್ಣೆದುರಿನಲ್ಲಿ ಆದ ಈ ಅಪಚಾರವನ್ನು ಕಂಡು ಮುನಿಗೆ ಕೋಪ ಉಕ್ಕಿತು. ಆತನು ‘ನಿನ್ನ ಅಧಿಕಾರ ಐಶ್ವರ್ಯ ಹಾಳಾಗಿಹೋಗಲಿ’ ಎಂದು ಶಾಪ ಕೊಟ್ಟ. ಪುಷಿಶಾಪಕ್ಕೆ ತಡೆಯುಂಟೆ? ಇಂದ್ರನ ಅಧಿಕಾರ ರಾಕ್ಷಸರ ಪಾಲಾಯಿತು. ಅವನ ಐಶ್ವರ್ಯ ಹಾರಿಹೋಯಿತು. ಅವನು ಅಗ್ನಿ, ವರುಣ ಮೊದಲಾದ ದೇವತೆಗಳೊಡನೆ ಆಲೋಚಿಸಿ ಯಾವ ಉಪಾಯವೂ ತೋಚದೆ, ತನ್ನ ಅನುಯಾಯಿಗಳೊಡನೆ ಬ್ರಹ್ಮನ ಬಳಿಗೆ ಹೋದನು. ನಾಲ್ಕು ಮೊಗದ ಬ್ರಹ್ಮನು ದೇವೆತಗಳ ಕಥೆಯನ್ನೆಲ್ಲ ಕೇಳಿ ‘ಅಯ್ಯಾ ದೇವತೆಗಳೆ, ನಾನು, ನೀವು, ಲೋಕ ದಲ್ಲಿ ಎಲ್ಲರೂ ಯಾವನ ಮಕ್ಕಳೋ ಆ ತಂದೆಯಾದ ಭಗವಂತನನ್ನು ಮರೆ ಹೋಗೋಣ. ಆತನಿಗೆ ದೇವತೆಗಳೆಂದರೆ ಪಂಚಪ್ರಾಣ. ತನ್ನ ಭಕ್ತರಾದ ನಿಮ್ಮನ್ನು ಆತ ಖಂಡಿತವಾಗಿಯೂ ಉದ್ಧಾರಮಾಡುತ್ತಾನೆ’ ಎಂದು ಹೇಳಿ, ಅವರೆಲ್ಲ ರನ್ನೂ ವೈಕುಂಠಕ್ಕೆ ಕರೆದೊಯ್ದನು. ಅಲ್ಲಿ ಹೋಗಿ ನೋಡಿದರೆ ಭಗವಂತನು ಕಣ್ಣಿಗೆ ಕಾಣಿಸಲೊಲ್ಲ. ಅದರೆ ಬ್ರಹ್ಮನು ಇದರಿಂದ ಎದೆಗುಂದಲಿಲ್ಲ. ಅತ್ಯಂತ ಭಕ್ತಿಯಿಂದ ಭಗವಂತನನ್ನು ಸ್ತೋತ್ರ ಮಾಡಿದನು–

‘ಹೇ ಸ್ವಾಮಿ, ನೀನು ಅನಿರ್ವಚನೀಯ, ಆದ್ಯಂತರಹಿತ, ನಿರ್ವಿಕಾರ, ಸರ್ವಾಂತರ್ ಯಾಮಿ. ನೀನು ಜಗತ್ಕಾರಣ, ಜಗತ್ತೇ ನೀನು. ಸಕಲ ಸಸ್ಯಗಳನ್ನು ಬೆಳಸಿ ಪೋಷಿಸುವ ಚಂದ್ರನೇ ನಿನ್ನ ಮನಸ್ಸು, ಯಜ್ಞದಲ್ಲಿ ಹವಿಸ್ಸನ್ನು ಕೊಳ್ಳುವ ಯಜ್ಞೇಶ್ವರ ನಿನ್ನ ಬಾಯಿ, ಸೂರ್ಯ ನಿನ್ನ ಕಣ್ಣು, ವಾಯು ನಿನ್ನಪ್ರಾಣ, ದಿಕ್ಕುಗಳು ನಿನ್ನ ಕಿವಿ, ಆಕಾಶ ನಿನ್ನ ಹೊಕ್ಕಳು, ಇಂದ್ರ ನಿನ್ನ ಬಲ, ದೇವತೆಗಳು ನಿನ್ನ ಅನುಗ್ರಹ, ರುದ್ರ ನಿನ್ನ ಕೋಪ, ಬ್ರಹ್ಮ ನಿನ್ನ ಬುದ್ಧಿ, ಪಿತೃದೇವೆತೆಗಳು ನಿನ್ನ ನೆರಳು, ಅಪ್ಸರೆಯರು ನಿನ್ನ ವಿಹಾರ, ಕಾಲ ನಿನ್ನ ರೆಪ್ಪೆಗಳು. ದೇವದೇವ! ವೇದಗಳೂ ವೇದವಿದರಾದ ಬ್ರಾಹ್ಮಣರೂ ನಿನ್ನ ಮುಖ, ಕ್ಷತ್ರಿಯರು ನಿನ್ನ ತೋಳುಗಳು, ವೈಶ್ಯರು ನಿನ್ನ ತೊಡೆಗಳು, ಶೂದ್ರರು ನಿನ್ನ ಪಾದ. ಸರ್ವತಂತ್ರ ಸ್ವತಂತ್ರನಾದ ಪರಮಾತ್ಮ! ನಿನ್ನ ಮುಖಕಮಲವನ್ನು ಕಾಣಲು ತವಕಿಸುತ್ತಿರುವ ನಮಗೆ ಪ್ರಸನ್ನನಾಗಿ ಪ್ರತ್ಯಕ್ಷನಾಗು. ನಿನಗೆ ಇದೇನೂ ಹೊಸದಲ್ಲ. ಆಗಾಗ ನೀನು ಬೇರೆ ಬೇರೆ ರೂಪಗಳಿಂದ ಅವತರಿಸುತ್ತಲೆ ಇರುವೆ.’

ಬ್ರಹ್ಮನ ಸ್ತೋತ್ರ ಮುಗಿಯುತ್ತಿದ್ದಂತೆ ಕೋಟಿ ಸೂರ್ಯರು ಏಕಕಾಲದಲ್ಲಿ ಮೂಡಿ ದಂತೆ, ಬೆಳಕಿನ ಹೊಳೆ ತುಂಬುನೆರೆಯಾಗಿ ಹರಿದು ಬಂತು. ಆ ಬೆಳಕಿನಿಂದ ಕಣ್ಣು ಕೋರಯಿಸುತ್ತಿರಲು, ದೇವತೆಗಳಿಗೆ ಭೂಮಿ ಆಕಾಶ ದಿಕ್ಕುಗಳೊಂದೂ ಕಾಣದಂತಾ ದುವು. ಆ ಬೆಳಕಿಗೆ ಬೆಳಕಾಗಿ ನೀಲಮೇಘಶ್ಯಾಮನಾದ ಶ್ರೀಹರಿ ಅಲ್ಲಿ ಮೂಡಿದ. ಆ ದಿವ್ಯ ಮಂಗಳ ವಿಗ್ರಹಕ್ಕೆ ಚತುರ್ಮುಖಬ್ರಹ್ಮನೇ ಮೊದಲಾದ ಸಕಲ ದೇವೆತೆಗಳೂ ಭಕ್ತಿ ಯಿಂದ ಅಡ್ಡಬಿದ್ದು ‘ಸ್ವಾಮಿ, ಕಾಡುಕಿಚ್ಚಿನಿಂದ ನೊಂದ ಕಾಡಾನೆಗಳಿಗೆ ಗಂಗೆ ಕಾಣಿಸಿ ದಂತೆ ನೀನು ನಮಗೆ ಗೋಚರನಾದೆ. ನೀನು ಸರ್ವಾಂತರ್ಯಾಮಿ, ಸರ್ವಸಾಕ್ಷಿ; ಆದ್ದ ರಿಂದ ನಮ್ಮ ಕಷ್ಟಗಳು ಏನೆಂಬುದನ್ನು ನಿನಗೆ ಹೇಳಬೇಕಾದುದೇ ಇಲ್ಲ. ಅವುಗಳಿಂದ ಉದ್ಧರಿಸೆಂದು ಮಾತ್ರ ನಿನ್ನನ್ನು ಬೇಡಿಕೊಳ್ಳುತ್ತೇವೆ. ಬೆಂಕಿಯಿಂದ ಕಿಡಿಗಳು ಹಾರುವಂತೆ ನಾವೆಲ್ಲ ನಿನ್ನಿಂದಲೆ ಹುಟ್ಟಿಬಂದವರು. ನಮಗೂ ಜಗತ್ತಿಗೂ ಮಂಗಳವಾಗುವಂತಹ ಉಪಾಯವನ್ನು ನಮಗೆ ಉಪದೇಶಿಸು’ ಎಂದು ಬೇಡಿಕೊಂಡರು.

ಭಕ್ತಿಯಿಂದ ಕೈ ಜೋಡಿಸಿಕೊಂಡು, ತನ್ನ ಉತ್ತರಕ್ಕಾಗಿ ಕಾಯುತ್ತ ನಿಂತಿರುವ ದೇವತೆ ಗಳತ್ತ ಮುಗುಳ್ನಗೆಯ ಧೈರ್ಯವನ್ನು ಎರಚುತ್ತಾ, ಶ್ರೀಹರಿಯು ‘ಅಯ್ಯಾ ದೇವತೆಗಳೆ, ನೀವೀಗಲೆ ಹೋಗಿ ದೈತ್ಯ ದಾನವರೊಡನೆ ಸಂಧಿಯನ್ನು ಮಾಡಿಕೊಳ್ಳಿ. ‘ಹಗೆಗಳೊಡನೆ ಸಂಧಿಯೆ?’ ಎಂಬ ಹಮ್ಮು ಬೇಡ. ಹಾವಾಡಿಗನ ಬುಟ್ಟಿಯಿಂದ ತಪ್ಪಿಸಿಕೊಳ್ಳಬೇಕೆಂಬ ಹಾವು ಇಲಿಯ ಸ್ನೇಹವನ್ನು ಮಾಡಿಕೊಂಡು, ಅದರಿಂದ ಬುಟ್ಟಿಯಲ್ಲಿ ರಂಧ್ರವನ್ನು ಕೊರೆಸಬೇಕು. ಕಾರ್ಯಸಾಧಕರಾದವರು ಹಗೆಗಳೊಡನೆ ಗೆಳತನ ಮಾಡಿಕೊಂಡೆ ಅವರನ್ನು ಗೆಲ್ಲಬೇಕು; ಇದರಲ್ಲಿ ಅವಮಾನವೇನೂ ಇಲ್ಲ. ಸಂಧಿ ಮಾಡಿಕೊಳ್ಳುವಾಗ ಅವರು ಹೇಳಿ ದುದಕ್ಕೆಲ್ಲ ಒಪ್ಪಿಕೊಳ್ಳಿ, ನೀವು. ಕಾಲವಶದಿಂದ ಆ ದೈತ್ಯದಾನವರು ನಿಮ್ಮ ಗೆಳೆಯರಾಗು ತ್ತಾರೆ. ಆಮೇಲೆ ನೀವು ಅವರೊಡನೆ ಸೇರಿಕೊಂಡು, ಸಕಲ ಮೂಲಿಕೆಗಳನ್ನೂ ಹಾಲಿನ ಸಮುದ್ರದಲ್ಲಿ ಹಾಕಿ, ಅದನ್ನು ಚೆನ್ನಾಗಿ ಕಡೆಯಿರಿ. ಮಂದರ ಪರ್ವತವು ನಿಮ್ಮ ಕಡೆ ಗೋಲು, ವಾಸುಕಿಯೇ ಹಗ್ಗ. ಸಮುದ್ರವನ್ನು ಕಡೆಯುವ ಕೆಲಸ ಅಷ್ಟು ಸುಲಭವೇನೂ ಅಲ್ಲ. ಆದ್ದರಿಂದ ಆ ಕಾರ್ಯದಲ್ಲಿ ನಾನೂ ನಿಮಗೆ ನೆರವಾಗುತ್ತೇನೆ. ಸಮುದ್ರವನ್ನು ಕಡೆಯುತ್ತಾ ಹೋದಂತೆ ಅದರಲ್ಲಿ ಅನೇಕ ದಿವ್ಯ ವಸ್ತುಗಳು ಹುಟ್ಟುತ್ತವೆ; ನೀವು ಅವೊಂ ದನ್ನೂ ಬಯಸಬೇಡಿ. ಆ ಕಾಲದಲ್ಲಿ ನಿಮ್ಮ ಹಟ, ಕೋಪಗಳೆಲ್ಲ ನಿಮ್ಮ ಹತೋಟಿಯ ಲ್ಲಿರಬೇಕು. ಕಟ್ಟಕಡೆಯಲ್ಲಿ ಅಮೃತ ಹುಟ್ಟುತ್ತದೆ. ಅದನ್ನು ಕುಡಿದವರಿಗೆ ಮುಪ್ಪು ಸಾವು ಗಳಿಲ್ಲ, ಅದು ನಿಮ್ಮದಾಗಬೇಕು; ಆಗುತ್ತದೆ, ನಾನು ಆಗುವ ಹಾಗೆ ಮಾಡುತ್ತೇನೆ’ ಎಂದು ಸಮಾಧಾನ ಹೇಳಿ, ಮಾಯವಾದನು.

ಶ್ರೀಹರಿಯ ಅಪ್ಪಣೆಯಂತೆ ಇಂದ್ರಾದಿ ದೇವತೆಗಳೆಲ್ಲರೂ ದಾನವಚಕ್ರವರ್ತಿಯಾದ ಬಲಿಯ ಬಳಿಗೆ ಹೋದರು. ಆಗ ಆತ ತನ್ನ ರಾಜಸಭೆಯಲ್ಲಿದ್ದ. ಆಯುಧಗಳಿಲ್ಲದೆ ತಮ್ಮ ಬಳಿಗೆ ಬಂದ ಅವರನ್ನು ದಾನವರು ನಿಷ್ಕರುಣೆಯಿಂದ ಕೊಲ್ಲಹೊರಟರು. ಆದರೆ ಬಲಿ ಚಕ್ರವರ್ತಿ ಅವರನ್ನು ತಡೆದು, ದೇವೇಂದ್ರನನ್ನು ತನ್ನ ಹತ್ತಿರಕ್ಕೆ ಬರಮಾಡಿಕೊಂಡನು. ಅಮೃತಕ್ಕಾಗಿ ತಾವೆಲ್ಲ ಒಂದಾಗಿ ಸಮುದ್ರವನ್ನು ಕಡೆಯಬೇಕೆಂಬ ಇಂದ್ರನ ಬೇಡಿಕೆ ಯನ್ನು ಬಲಿ ಒಡನೆಯೆ ಒಪ್ಪಿಕೊಂಡನು. ಅಮೃತದ ಮಹಿಮೆಯನ್ನು ಕೇಳಿದೊಡನೆಯೆ ರಕ್ಕಸರ ಉತ್ಸಾಹಕ್ಕೆ ಕಳೆಯೇರಿತು. ಅವರು ಹಾಗಿಂದ ಹಾಗೆಯೆ ಕುಣಿಕುಣಿಯುತ್ತಾ ಮಂದರ ಪರ್ವತಕ್ಕೆ ಹಾರಿದರು. ದಾನವರೂ ದೇವತೆಗಳೂ ಸೇರಿ ಆ ಪರ್ವತವನ್ನು ಕಿತ್ತು ಕೊಂಡು, ಬ್ರಹ್ಮಾಂಡ ಬಿರಿಯುವಂತೆ ಅಬ್ಬರಿಸುತ್ತಾ ಹಾಲ್ಗಡಲಿಗೆ ಹೊರಟರು. ಆದರೆ ಅರ್ಧಹಾದಿ ಹೋಗುವಷ್ಟರಲ್ಲಿ ಅವರ ಶಕ್ತಿ ಕುಂದಿಹೋಯಿತು. ಅವರು ಅದನ್ನು ಅಲ್ಲಿಯೇ ಎತ್ತಿಹಾಕಿದರು. ಆಗ ಶ್ರೀಹರಿಯು ಅಲ್ಲಿ ಪ್ರತ್ಯಕ್ಷನಾಗಿ, ತನ್ನ ಒಂದು ಕೈ ಯಿಂದ ಆ ಪರ್ವತವನ್ನು ಲೀಲಾಜಾಲವಾಗಿ ಎತ್ತಿ, ಗರುಡನ ಮೇಲಿಟ್ಟುಕೊಂಡು ಬಂದು, ಹಾಲ್ಗಡಲ ಮಧ್ಯದಲ್ಲಿ ಅದನ್ನು ಸ್ಥಾಪಿಸಿದನು. 

ಸಮುದ್ರಮಥನಕ್ಕೆ ಕಡಗೋಲೇನೊ ಸಿದ್ಧವಾಯಿತು. ಇನ್ನು ಹಗ್ಗಬೇಕಲ್ಲ! ದೇವ ದಾನವರು ಸರ್ಪರಾಜನಾದ ವಾಸುಕಿಯ ಬಳಿಗೆ ಹೋಗಿ, ತಮ್ಮ ಉದ್ದೇಶವನ್ನು ಆತನಿಗೆ ತಿಳಿಸಿ, ಹಗ್ಗವಾಗುವಂತೆ ಆತನನ್ನು ಬೇಡಿದರು. ಮೊದಲು ಆತನು ಇದಕ್ಕೆ ಒಪ್ಪದೆ ಹೋದರೂ, ಅಮೃತದ ಮಹಿಮೆಯನ್ನೂ, ತನಗೂ ಅದರಲ್ಲಿ ಒಂದು ಪಾಲು ದೊರೆಯು ವುದೆಂಬುದನ್ನೂ ಕೇಳಿ ಸಂತೋಷದಿಂದ ಹಗ್ಗವಾಗುವುದಕ್ಕೆ ಒಪ್ಪಿದನು. ಮಂದರ ಪರ್ವತಕ್ಕೆ ಆತನನ್ನು ಹಗ್ಗದಂತೆ ಸುತ್ತಿದುದೂ ಆಯಿತು. ಶ್ರೀಹರಿಯೆ ಸಾಕ್ಷಾತ್ತಾಗಿ ಆತನ ತಲೆಯ ಬಳಿ ಹಿಡಿದು ಕಡೆಯುವುದಕ್ಕೆ ನಿಂತನು; ದೇವತೆಗಳೆಲ್ಲರೂ ಆತನ ಪಕ್ಕದಲ್ಲಿ ನಿಂತರು. ರಾಕ್ಷಸರು ಇದಕ್ಕೆ ಒಪ್ಪಲಿಲ್ಲ. ತಾವು ದೇವತೆಗಳಿಗಿಂತಲೂ ಹಿರಿಯರಾದುದ ರಿಂದ ಬಾಲದ ಕಡೆ ಹಿಡಿಯುವುದು ತಮಗೆ ಅವಮಾನಕರವೆಂದರು. ಅದನ್ನು ಕೇಳಿ ಶ್ರೀಹರಿಯು ನಗುತ್ತಾ, ತನ್ನ ಅನುಯಾಯಿಗಳಾದ ದೇವತೆಗಳೊಡನೆ ಬಾಲದ ಕಡೆಯನ್ನು ಹಿಡಿದನು. ದಾನವರು ತಲೆಯ ಕಡೆ ಹಿಡಿದುಕೊಂಡರು. ಎರಡು ಕಡೆಯವರೂ ಮಹೋ ತ್ಸಾಹದಿಂದ ಸಮುದ್ರವನ್ನು ಕಡೆಯಲು ಪ್ರಾರಂಭಿಸಿದರು. ಇದ್ದಕ್ಕಿದ್ದಂತೆ ಆ ಕ[ಡಗೋಲು ನಿಲ್ಲುವುದಕ್ಕೆ ಆಧಾರವಿಲ್ಲದೆ ಸಮುದ್ರದಲ್ಲಿ ಮುಳುಗಿಹೋಯಿತು. ಅದನ್ನು ಮೇಲಕ್ಕೆತ್ತು ವುದು ಹೇಗೆ? ಅದು ಮತ್ತೆ ಮುಳುಗದಂತೆ ಹಿಡಿದುಕೊಳ್ಳುವುದು ಹೇಗೆ? ಎರಡು ಕಡೆಯ ವರೂ ಮುಂದೋರದೆ ಕಣ್ಣುಕಣ್ಣು ಬಿಡುತ್ತಿರಲು, ಶ್ರೀಹರಿಯು ಕೂರ್ಮಾವತಾರವನ್ನು ಧರಿಸಿದನು. ಭಯಂಕರಾಕಾರದ ಆ ಆಮೆ ಸಮುದ್ರ ಮಧ್ಯದಲ್ಲಿ ಒಂದು ದೊಡ್ಡ ದ್ವೀಪ ದಂತೆ ನಿಂತು, ಮುಳುಗಿದ್ದ ಪರ್ವತವನ್ನು ಒಂದು ಲಕ್ಷಯೋಜನ ವಿಸ್ತಾರವಾದ ತನ್ನ ಬೆನ್ನಮೇಲೆ ಧರಿಸಿತು. ಇದರ ಜೊತೆಗೆ ಶ್ರೀಹರಿಯು ಸಹಸ್ರ ತೋಳುಗಳ ಮಹಾಪುರುಷ ನಾಗಿ ಆಕಾಶದಲ್ಲಿ ನೆಲಸಿ, ತನ್ನ ಒಂದು ತೋಳಿನಿಂದ ಆ ಪರ್ವತವನ್ನು ಮತ್ತೆ ಕೆಳ ಕ್ಕುರುಳದಂತೆ ಹಿಡಿದುಕೊಂಡನು. ಇದರಿಂದ ದೇವದಾನವರು ಕಡೆಯುವ ಕೆಲಸವನ್ನು ನಿರಾತಂಕವಾಗಿ ನಡೆಸಲು ಸಾಧ್ಯವಾಯಿತು.

ಅಮೃತದ ಆಸೆಯಿಂದ ವಾಸುಕಿ ಹಗ್ಗವಾಗಲು ಒಪ್ಪಿದನಾದರೂ ಸ್ವಲ್ಪ ಕಾಲ ಕಡೆಯು ವಷ್ಟರಲ್ಲಿ ಆತನಿಗೆ ತಡೆಯದಷ್ಟು ಆಯಾಸವಾಗಿ ನಿಟ್ಟಸಿರು ಹೊರಹೊಮ್ಮಿತು. ಆತನ ಸಾವಿರ ಬಾಯಿಗಳಿಂದ ಹೊರಟ ಆ ನಿಟ್ಟುಸಿರು ವಿಷದ ಬೆಂಕಿಯಾಗಿ ಕಡೆಯುತ್ತಿದ್ದವ ರನ್ನೆಲ್ಲ ತಳಮಳಗೊಳಿಸಿತು. ತಲೆಯ ಕಡೆಯನ್ನು ಆಯ್ದುಕೊಂಡ ರಾಕ್ಷಸರಂತೂ ಮೃತ್ಯು ಮುಖರಾದರು. ಆಗ ಶ್ರೀಹರಿಯು ಮಳೆಗರೆದು, ತಂಗಾಳಿ ಬೀಸುವಂತೆ ಮಾಡಿ, ಅವರ ಬೇಗೆಯನ್ನು ಹೋಗಲಾಡಿಸಿದನು. ಅಷ್ಟೇ ಅಲ್ಲ, ಎಷ್ಟು ಹೊತ್ತು ಕಡೆದರೂ ಯಾವ ವಸ್ತುವೂ ಹುಟ್ಟದಿರಲು, ತಾನೆ ಕಡೆಯುವ ಕಾರ್ಯಕ್ಕೂ ಕೈ ಹಾಕಿದನು. ಆಗ ಮೊಟ್ಟ ಮೊದಲು ಹಾಲಾಹಲ ಹುಟ್ಟಿತು. ಆ ಮಹಾ ವಿಷದ ಬೇಗೆಯನ್ನು ಸಹಿಸಲಾರದೆ ಸಮುದ್ರ ದಲ್ಲಿದ್ದ ಪ್ರಾಣಿಗಳೆಲ್ಲ ಲಿಬಿಲಿಬಿ ಒದ್ದಾಡಿಹೋದವು. ಕ್ಷಣಮಾತ್ರದಲ್ಲಿ ಅದರ ಬೇಗೆ ಮೇಲೆ, ಕೆಳಗೆ ಎಲ್ಲೆಡೆಯಲ್ಲಿಯೂ ಹಬ್ಬಿ ಜಗತ್ತನ್ನೆಲ್ಲ ವ್ಯಾಪಿಸಿತು. ಮೂರು ಲೋಕದ ಪ್ರಜೆಗಳೂ ಮುಂಗಾಣದವರಾಗಿ ಜಗದ್ರಕ್ಷಕನಾದ ಸದಾಶಿವನ ಬಳಿಗೆ ಓಡಿಹೋದರು. ಕೈಲಾಸದಲ್ಲಿ ಭವಾನಿಯೊಡನೆ ನೆಲೆಸಿದ್ದ ಪರಶಿವನು ಆಗ ಲೋಕಕಲ್ಯಾಣಕ್ಕಾಗಿ ತಪೋ ನಿರತನಾಗಿದ್ದನು. “ ಪರಮೇಶ್ವರ ಕಾಪಾಡು, ಕಾಪಾಡು! ಉಪನಿಷತ್ತುಗಳು ಕೊಂಡಾಡುವ ಪರಬ್ರಹ್ಮನೆಂಬುವನು ನೀನೆ; ಬ್ರಹ್ಮ ವಿಷ್ಣು ರುದ್ರರೆಂಬ ಮೂರು ಅವತಾರಗಳಿಂದ ಸೃಷ್ಟಿ ಸ್ಥಿತಿ ಲಯಗಳನ್ನು ನೆರವೇರಿಸುತ್ತಿರುವ ಪರಾತ್ಪರ ಸ್ವರೂಪನೇ ನೀನು. ನೀನೆ ಸಕಲ ದೇವಾತ್ಮಕ. ಅಗ್ನಿ ನಿನ್ನ ಮುಖ, ಭೂಮಿ ನಿನ್ನ ಪಾದ, ಕಾಲ ನಿನ್ನ ನಡಿಗೆ, ದಿಕ್ಕು ನಿನ್ನ ಕಿವಿ, ವರುಣ ನಿನ್ನ ನಾಲಗೆ, ಅಂತರಿಕ್ಷ ನಿನ್ನ ಹೊಕ್ಕಳು, ವಾಯು ನಿನ್ನ ಉಚ್ಛ್ವಾಸ ನಿಶ್ವಾಸ, ಸೂರ್ಯ ಕಣ್ಣು, ಚಂದ್ರ ಮನಸ್ಸು, ಸ್ವರ್ಗ ತಲೆ, ಸಮುದ್ರ ಹೊಟ್ಟೆ, ಗಿಡ ಬಳ್ಳಿ ಮೈಗೂದಲು, ಧರ್ಮವೇ ಹೃದಯ, ಸದ್ಯೋಜಾತಾದಿ ಐದು ಮಂತ್ರಗಳೆ ನಿನ್ನ ಐದು ಮುಖಗಳು. ಸ್ವಾಮಿ, ನಾವಿಂದು ವಿಷದ ಬಾಧೆಯಿಂದ ಬೆಂದುಹೋಗುತ್ತಿದ್ದೇವೆ. ಮನ್ಮಥನನ್ನು ಸುಟ್ಟ, ದಕ್ಷನ ಯಾಗವನ್ನು ಧ್ವಂಸಮಾಡಿದ, ತ್ರಿಪುರನನ್ನು ಸಂಹರಿಸಿದ, ಯಮನನ್ನು ಶಿಕ್ಷಿಸಿದ ನಿನಗೆ ವಿಷಬಾಧೆಯನ್ನು ನೀಗುವುದು ಮಹಾಕಾರ್ಯವೇನೂ ಅಲ್ಲ. ದೇವದೇವ, ಜ್ಞಾನಿಗಳ ಹೃದಯದಲ್ಲಿ ಸದಾ ನೆಲೆಸಿ, ಶ್ಮಶಾನವಾಸಿಯೆಂಬಂತೆ ಲೀಲೆಯನ್ನು ತೋರುತ್ತಿರುವ ನೀನು ನಮ್ಮನ್ನು ಕಾಪಾಡು” ಎಂದು ದೇವಮಾನವರೆಲ್ಲರೂ ಮೊರೆ ಯಿಡುತ್ತಿರುವುದನ್ನು ಕೇಳಿ, ಕರುಣಾಮಯನಾದ ಮಹಾದೇವನು ಕನಿಕರದಿಂದ ಭವಾನಿ ಯೊಡನೆ ‘ಸರ್ವಮಂಗಳೆ, ಈ ನನ್ನ ಭಕ್ತರ ಮೊರೆಯನ್ನು ಕೇಳಿ ನನ್ನ ಕರುಳು ಕರಗಿತು. ನಾನು ಆ ಘೋರ ವಿಷವನ್ನು ಕುಡಿದುಬಿಡುತ್ತೇನೆ’ ಎಂದನು. ಆತನ ಮಹಿಮೆಯನ್ನು ಅರಿತ ಪಾರ್ವತಿ ‘ತಥಾಸ್ತು’ ಎಂದಳು. ಪರಶಿವನು ಲೋಕಕ್ಕೆಲ್ಲ ಹಬ್ಬಿದ್ದ ವಿಷವನ್ನು ತನ್ನ ಅಂಗೈಗೆ ಸೆಳೆದುಕೊಂಡು ಅದನ್ನು ಕುಡಿದೇಬಿಟ್ಟನು. ಅದರಿಂದ ಆತನ ಕೂದಲು ಕೂಡ ಕೊಂಕಲಿಲ್ಲವಾದರೂ, ಆತನ ಕಂಠವು ಕಪ್ಪಗಾಗಿ ಆತನಿಗೆ ‘ನೀಲಕಂಠ’ನೆಂದು ಹೆಸರಾ ಯಿತು. ಆದರೇನು? ಆ ಪುರಾಣ ಪುರುಷನಿಗೆ ಅದೂ ಒಂದು ಅಲಂಕಾರವಾದಂತಾಯಿತು. ಮಹಾದೇವನ ಮಹಾಕಾರ್ಯವನ್ನು ಕಂಡು ಬ್ರಹ್ಮ ವಿಷ್ಣು ಮೊದಲಾದ ದೇವತೆಗಳೆಲ್ಲರೂ ಆತನನ್ನು ಸ್ತುತಿಸಿದರು. ಆತನು ವಿಷವನ್ನು ಕುಡಿಯುವಾಗ ಕೈಯಿಂದ ಜಾರಿದ ಒಂದೆರಡು ತೊಟ್ಟುಗಳೆ ಹಾವು, ಚೇಳು ಮೊದಲಾದ ವಿಷಜಂತುಗಳಾದವು. 

ವಿಷದ ಬೇಗೆ ಹರಕರುಣದಿಂದ ನಿವಾರಣೆಯಾಗುತ್ತಲೆ ದೈತ್ಯದಾನವರು ತಮ್ಮ ಕೆಲಸ ವನ್ನು ಮುಂದುವರಿಸಿದರು. ಆಗ ಸಮುದ್ರದಿಂದ ಕಾಮಧೇನು ಹುಟ್ಟಿಬಂದಿತು. ಅದು ಯಜ್ಞಕ್ಕೆ ಬೇಕಾದ ಹವಿಸ್ಸನ್ನು ಒದಗಿಸಬಲ್ಲುದಾದುದರಿಂದ ವೇದನಿಷ್ಠರಾದ ಪುಷಿಗಳು ಅದನ್ನು ತೆಗೆದುಕೊಂಡರು. ಆದಾದ ಬಳಿಕ ಹಾಲಿನ ಕೆನೆಯಂತೆ ಬಿಳಿದಾದ ‘ಉಚ್ಚೈ ಶ್ರವಸ್ಸು’ ಎಂಬ ಕುದುರೆ ಎದ್ದುಬಂದಿತು. ಅದರ ಅಂದಚೆಂದಗಳನ್ನು ಕಂಡು ದೇವೇಂದ್ರನಿಗೆ ಅದರ ಮೇಲೆ ಆಸೆ ಹುಟ್ಟಿತು. ಆದರೂ ಬಲೀಂದ್ರನು ಅದನ್ನು ಅಪೇಕ್ಷೆ ಪಟ್ಟುದರಿಂದ ಅದು ಆತನಿಗೆ ಸೇರಿತು. ಅದರ ಹಿಂದೆಯೇ ಐರಾವತವೆಂಬ ಆನೆ ಹೊರ ಹೊರಟಿತು. ಹಿಮವತ್ಪರ್ವತದಂತೆ ಬೆಳ್ಳಗೆ ಬೆಳಗುತ್ತಿದ್ದ ನಾಲ್ಕು ದಂತಗಳ ಆ ಆನೆಯ ಹಿಂದೆಯೇ ಎಂಟು ದಿಗ್ಗಜಗಳೂ ಅವುಗಳ ರಾಣಿಯರೂ ಉದಿಸಿಬಂದರು; ಕಮಲದಂತೆ ಕೆಂಪಾದ ‘ಕೌಸ್ತುಭ’ ರತ್ನ ಹುಟ್ಟಿತು. ಶ್ರೀಹರಿಯು ಆ ರತ್ನವನ್ನು ತೆಗೆದುಕೊಂಡು ತನ್ನ ಎದೆಯಲ್ಲಿ ಧರಿಸಿದನು. ಆಮೇಲೆ ಕೇಳಿದವರಿಗೆ ಕೇಳಿದುದನ್ನು ಕೊಡುವ ಪಾರಿಜಾತವೇ ಮೊದಲಾದ ಕಲ್ಪವೃಕ್ಷಗಳೂ, ಲೋಕವನ್ನೆ ಮರುಳುಗೊಳಿಸುವ ಮೋಹಕ ರೂಪಿನ ಅಪ್ಸರೆ ಯರೂ ಹುಟ್ಟಿದರು. ಅವರ ಹಿಂದೆಯೆ ಮಹಾಲಕ್ಷ್ಮಿ ಮೂಡಿ ಬಂದಳು. ಮಿಂಚಿನ ಬಳ್ಳಿಯಂತೆ ದಶದಿಕ್ಕುಗಳನ್ನೂ ಬೆಳಗುತ್ತಾ ಹೊರಬಂದ ಆಕೆಯ ರೂಪ ಲಾವಣ್ಯವನ್ನು ಬಣ್ಣಿಸಲು ಮಾತಿಗೆ ಶಕ್ತಿ ಸಾಲದು. ಆಕೆಯ ಬಣ್ಣ, ರೂಪ, ಯೌವನಗಳನ್ನು ಕಂಡು, ದೇವದಾನವರೆಲ್ಲರೂ ಮೈಮರೆತು, ಮರುಳಾಗಿ ‘ನನಗೆ–ತನಗೆ’ ಎಂದು ತಹತಹ ಪಟ್ಟರು. ದೇವೇಂದ್ರನು ಆಕೆ ಕುಳಿತುಕೊಳ್ಳಲೆಂದು ಒಂದು ಸಿಂಹಾಸನವನ್ನು ತಂದು ಹಾಕಿದನು. ದೇವತೆಗಳೂ ಪುಷಿಗಳೂ ಆಕೆಗೆ ಕಾಣಿಕೆಯನ್ನು ತಂದಿತ್ತರು. ಬ್ರಹ್ಮನು ಆಕೆಗೆ ಒಂದು ಲೀಲಾಕಮಲವನ್ನು ಕೊಟ್ಟನು, ಸರಸ್ವತಿ ಮುತ್ತಿನ ಹಾರವನ್ನು ಸಮರ್ಪಿಸಿದಳು. ಬಂಗಾರದ ಬಳ್ಳಿಯಂತಿದ್ದ ಲಕ್ಷ್ಮೀದೇವಿಯು ತನಗೊಪ್ಪಿಸಿದ ಮರ್ಯಾದೆಗಳನ್ನೆಲ್ಲ ಸ್ವೀಕರಿಸುತ್ತ, ಮುಗುಳ್ನಗೆಯೊಡನೆ ಬೆರೆತ ಮಧುರವಾದ ನೋಟದಿಂದ ಒಮ್ಮೆ ಎಲ್ಲ ರನ್ನೂ ನೋಡಿದಳು. ಅನಂತರ ಶ್ರೀಹರಿಯ ಕೊರಳಲ್ಲಿ ಕಮಲದ ಹಾರವನ್ನು ಹಾಕಿ, ಆತನ ಹೃದಯದಲ್ಲಿ ನೆಲಸಿದಳು. ಆ ಶುಭಮುಹೂರ್ತದಲ್ಲಿ ಹೂಮಳೆಗರೆಯಿತು, ಗಂಧರ್ವರು ಗಾನಮಾಡಿದರು, ಅಪ್ಸರೆಯರು ನರ್ತಿಸಿ ದರು, ಲೋಕವೆಲ್ಲ ಆನಂದದಿಂದ ತುಂಬಿ ನಲಿಯಿತು. 

ಲಕ್ಷ್ಮೀದೇವಿಯು ಹುಟ್ಟಿದೊಡನೆಯೆ ಸಮುದ್ರವನ್ನು ಕಡೆಯುವ ಕೆಲಸ ನಿಂತು ಹೋಗಿತ್ತು. ಆಕೆಯ ಸ್ವಯಂವರ ನೆರವೇರುತ್ತಲೆ ಆ ಕೆಲಸ ಮುಂದುವರೆಯಿತು. ಆಗ ತಾವರೆಯಂತೆ ಕೆಂಪಾದ ಕಣ್ಣುಗಳಿಂದ ಕೂಡಿ ಲೋಕಮೋಹಕಳಾದ ಸುರಾದೇವಿ ಹುಟ್ಟಿ ದಳು. ಶ್ರೀಹರಿಯು ಆ ಮೋಹಕ ಮೂರ್ತಿಯನ್ನು ದಾನವರಿಗೆ ದಾನಮಾಡಿದನು. ಆಕೆಯ ಹಿಂದೆಯೆ ಆಜಾನುಬಾಹುವಾದ ಮಹಾ ಪುರುಷನೊಬ್ಬನು, ಸರ್ವಾಲಂಕಾರ ಭೂಷಿತನಾಗಿ ಸಮುದ್ರಮಧ್ಯದಿಂದ ಮೇಲಕ್ಕೆದ್ದನು. ಆತನೇ ಆಯುರ್ವೇದಾಚಾರ್ಯನಾದ ಧನ್ವಂತರಿ. ಆತನು ಕೈಗಳಲ್ಲಿ ಅಮೃತಕಲಶವನ್ನು ಹಿಡಿದು ಭೂಮಿಯ ಮೇಲೆ ಕಾಲಿಕ್ಕುತ್ತಿದ್ದಂತೆಯೆ ದೈತ್ಯದಾನವರು ಆತನನ್ನು ಮುತ್ತಿಕೊಂಡು ‘ಸಮುದ್ರದಲ್ಲಿ ಹುಟ್ಟಿ ಬಂದ ವಸ್ತುಗಳೆಲ್ಲ ನಮಗೇ ಸೇರಬೇಕು’ಎಂದು ಕೂಗಾಡುತ್ತಾ, ಆತನ ಕೈಲಿದ್ದ ಅಮೃತ ಕಲಶವನ್ನು ಕಿತ್ತು ಕೊಂಡರು. ಇದನ್ನು ಕಂಡು ದೇವತೆಗಳು ಮುಂದೋರದವರಾಗಿ ಕಳವಳದಿಂದ ಶ್ರೀಹರಿ ಯನ್ನು ಮೊರೆಹೊಕ್ಕರು. ಭಕ್ತರಕ್ಷಕನಾದ ಭಗವಂತ ‘ಅಯ್ಯಾ, ಭಯಪಡಬೇಡಿ! ನಾನು ಉಪಾಯದಿಂದ ಅಮೃತವನ್ನು ನಿಮಗೆ ದಕ್ಕುವಂತೆ ಮಾಡುತ್ತೇನೆ’ ಎಂದು ಹೇಳಿ ಅವರನ್ನು ಸಮಾಧಾನಪಡಿಸಿದನು.

ಶ್ರೀಹರಿಯ ಮಾಯೆಯನ್ನು ಗೆಲ್ಲಬಲ್ಲವರಾರು? ಅಮೃತ ದೈತ್ಯದಾನವರ ಕೈಗೇನೊ ಸಿಕ್ಕಿತು. ಆದರೆ ಅದನ್ನು ಕುಡಿಯುವ ವಿಚಾರದಲ್ಲಿ ನಾನು ಮೊದಲು, ತಾನು ಮೊದಲೆಂದು ಅವರಲ್ಲಿ ಜಗಳ ಹತ್ತಿತು. ಅವರಲ್ಲೆ ಕೆಲವರು ‘ಇದರಲ್ಲಿ ದೇವತೆಗಳಿಗೂ ಸಮಾನವಾದ ಭಾಗವಿದೆ’ ಎಂದು ವಾದಿಸಿದರೆ, ಮತ್ತೆ ಕೆಲವರು ‘ಬಲಾ ಚ ಪೃಥಿವೀ’–ಶಕ್ತರಾದವರಿಗೇ ಸರ್ವಸ್ವವೂ–ಎಂದು ಕೂಗಿಕೊಂಡು, ಅಮೃತದ ಪಾತ್ರೆಯನ್ನು ಮತ್ತೊಬ್ಬನಿಂದ ಕಿತ್ತು ಕೊಳ್ಳುತ್ತಿದ್ದರು. ಅವನಿಗಿಂತಲೂ ಬಲವಾದವನು ಅವನಿಂದ ಕಿತ್ತುಕೊಳ್ಳುತ್ತಿದ್ದನು. ಅವರು ಹೀಗೆ ಕಿತ್ತಾಡುತ್ತಿರುವಾಗ, ಶ್ರೀಹರಿಯು ಮೋಹಿನಿಯ ಆಕಾರವನ್ನು ತಾಳಿ ಅಲ್ಲಿ ಪ್ರಕಟನಾದನು. ಮಾಯಾಮಯದ ಆ ರೂಪು ಕಿತ್ತಾಡುತ್ತಿದ್ದ ದೈತ್ಯದಾನವರ ಮನಸ್ಸನ್ನೆಲ್ಲ ತನ್ನ ಕಡೆ ಸೆಳೆಯಿತು. ಅದೇನು ರೂಪು ಅದು! ಒಂದೊಂದು ಅಂಗದಲ್ಲಿಯೂ ಸೌಂದರ್ಯ ತುಂಬಿ ತುಳುಕುತ್ತಿತ್ತು. ಕನ್ನಡಿಯಂತೆ ಹೊಳೆಯುತ್ತಿರುವ ಆ ಕೆನ್ನೆಗಳೇನು, ಯವ್ವನವೇ ಹೋಗಿ ನೆಲಸಿದಂತಿರುವ ಆ ತುಂಬು ಎದೆಗಳೇನು, ಅವುಗಳ ಭಾರವನ್ನು ಹೊರಲಾರದೆ ಬಳುಕುತ್ತಿರುವ ಆ ಬಡನಡುವೇನು, ಸುನೀಲವಾದ ಆ ಕೇಶರಾಶಿಯೇನು! ದೈತ್ಯ ದಾನವರು ಆಕೆಯನ್ನು ಕಂಡು ಹುಚ್ಚರಾಗಿಹೋದರು. ತನ್ನ ಮುಖದ ಸುವಾಸನೆಗೆ ಮುತ್ತ ಬರುತ್ತಿರುವ ದುಂಬಿಗಳನ್ನು ಕಂಡು ಆಕೆಯ ಚೆಲ್ಲೆಗಂಗಳು ಬೆದರಿದಂತೆ ಚಲಿಸುತ್ತಲೆ ದಾನವರ ಹೃದಯಗಳು ಚಲಿಸಿದುವು; ಆಕೆಯು ಸೀರೆಯ ಸೆರಗನ್ನು ಜಾರಿಸಿ, ನಾಚಿಕೆಯಿಂದ ಅದನ್ನು ಸೆಳೆದುಕೊಳ್ಳುತ್ತಲೆ ಅವರ ಜೀವಗಳೂ ಸೆಳೆತಕ್ಕೆ ಒಳಗಾದವು. ಆಕೆಯು ನಿರಿಯನ್ನು ಚಿಮ್ಮುತ್ತಾ ಹೆಜ್ಜೆಯಿಡುತ್ತಲೆ ಅವರ ಹೃದಯಗಳಲ್ಲಿ ಕಾಮಬಾಣ ಗಳು ಆಳವಾಗಿ ನಾಟಿದವು. ಆಕೆಯ ಹುಬ್ಬು ಹಾರಿತೆಂದರೆ ಇವರ ಜೀವವೇ ಹಾರಿಹೋಗು ತ್ತಿತ್ತು. ಅವರು ಅರಿಯದಂತೆಯೆ ಅವರ ಬಾಯಿಂದ ‘ಅಹೋ ರೂಪಂ!’ ಎಂಬ ಉದ್ಗಾರ ಹೊರಟಿತು.

ತನ್ನ ಸುಳಿವಿನಿಂದ ಕಾಮವನ್ನು ತುಳುಕಿಸುತ್ತಿರುವ ಮೋಹಿನಿಗೆ ಮಾರುಹೋದ ದೈತ್ಯದಾನವರು ಅಮೃತಕ್ಕಾಗಿ ಕಾದಾಡುವುದನ್ನು ನಿಲ್ಲಿಸಿ, ಆ ಹೆಣ್ಣಿನ ಬೆನ್ನು ಹತ್ತಿದರು. ಅವರು ಆಕೆಯನ್ನು ಕುರಿತು ‘ಮೋಹನಾಂಗಿ, ನೀನಾರು? ಎಲ್ಲಿಂದ ಬರುತ್ತಿರುವಿ? ನಿನ್ನನ್ನು ಇಲ್ಲಿಗೆ ಕಳುಹಿಸಿದವರು ಯಾರು? ನೀನು ಏಕೆ ಬಂದಿರುವಿ? ನಿನಗೇನು ಬೇಕು? ಇನ್ನೂ ಯಾರೂ ವಾಸನೆ ನೋಡದ ಹೂವಿನಂತೆ ಇರುವ ನೀನು ಮಾನವಳೊ, ದೇವತೆಯೊ? ನಮ್ಮ ಇಂದ್ರಿಯಗಳನ್ನೂ ಮನಸ್ಸನ್ನೂ ತಣಿಸುವುದಕ್ಕಾಗಿಯೇ ಕೃಪಾಳುವಾದ ಬ್ರಹ್ಮನು ನಿನ್ನನ್ನು ಸೃಷ್ಟಿಸಿ ಇಲ್ಲಿಗೆ ಕಳುಹಿಸಿರುವಂತೆ ತೋರುತ್ತದೆ. ಹೇ ಬಡನಡುವಿನ ಬಿಂಕಗಾತಿ, ನಮ್ಮ ಮನಸ್ಸು ನಿನಗೆ ಸೂರೆಯಾಗಿಹೋಗಿದೆ. ಇಲ್ಲಿ ನೋಡು, ಈ ಪಾತ್ರೆಯ ತುಂಬ ಅಮೃತವಿದೆ. ಇದಕ್ಕಾಗಿ ನಾವೆಲ್ಲ ಬಡಿದಾಡುತ್ತಿದ್ದೇವೆ. ನಾವು ಕಶ್ಯಪನ ಮಕ್ಕಳು. ಸೋದರರಾದ ನಾವು ಪರಸ್ಪರ ಕಿತ್ತಾಡುವುದನ್ನು ನಿಲ್ಲಿಸಿ, ಈ ಅಮೃತವನ್ನು ನಿನ್ನ ಕೈಗೆ ಕೊಡುತ್ತೇವೆ, ನೀನು ನಮಗಾರಿಗೂ ಅನ್ಯಾಯವಾಗದಂತೆ ಇದನ್ನು ನಮ್ಮೆಲ್ಲರಿಗೂ ಹಂಚಿ ಕೊಡು’ ಎಂದು ಬೇಡಿಕೊಂಡರು. ಮೋಹಿನಿಯ ಬಯಕೆ ಅದೇ ಆಗಿರುವಾಗ ಆಕೆ ಬೇಡ ವೆನ್ನುವಳೆ? ಆದರೂ ಅವರೊಡನೆ ಚೆಲ್ಲಾಟವಾಡುವವಳಂತೆ “ವೀರರೇ, ನಾನು ಹೆಣ್ಣು, ಚಂಚಲ ಬುದ್ಧಿಯುಳ್ಳವಳು. ತಿಳಿದವರು ‘ಹೆಣ್ಣನ್ನು ನಂಬಬಾರದು’ ಎನ್ನುತ್ತಾರೆ. ನೀವು ಚೆನ್ನಾಗಿ ಯೋಚಿಸಿ ನೋಡಿ’ ಎಂದು ಒಮ್ಮೆ ಕಣ್ಣನ್ನು ಕುಣಿಸಿದಳು. ಆಕೆಯ ಆ ಬಿನ್ನಾಣ, ಲಲ್ಲೆವಾತುಗಳಿಗೆ ಮನಸೋತ ದೈತ್ಯದಾನವರು ಆಕೆಗೆ ಪ್ರತ್ಯುತ್ತರವನ್ನೆ ಕೊಡದೆ, ನಗುನಗುತ್ತಾ ಅಮೃತದ ಪಾತ್ರೆಯನ್ನು ಆಕೆಯ ಕೈಲಿಟ್ಟರು. ರೊಟ್ಟಿ ಜಾರಿ ತುಪ್ಪದಮೇಲೆ ಬಿದ್ದಂತಾದುದರಿಂದ ಮೋಹಿನಿಯೂ ಕಿರುನಗೆಯೊಡನೆ ಅದನ್ನು ಕೈಲಿ ಹಿಡಿದು ‘ದೈತ್ಯ ದಾನವ ವೀರರೆ, ನಾನು ನಿಮ್ಮ ಪ್ರೀತಿಗೆ ಪ್ರತಿ ಹೇಳಲಾರೆ. ಅಗತ್ಯವಾಗಿಯೂ ಅಮೃತವನ್ನು ಹಂಚುತ್ತೇನೆ. ಆದರೆ ನಾನು ಮಾಡಿದುದನ್ನು ನೀವು ಒಪ್ಪಿಕೊಳ್ಳಬೇಕು. ಅದರಲ್ಲಿ ಸರಿ- ತಪ್ಪುಗಳನ್ನು ನೀವು ಎತ್ತಕೂಡದು. ಹಾಗಾದರೆ ಮಾತ್ರ ನಾನು ಆ ಕೆಲಸವನ್ನು ವಹಿಸಿ ಕೊಳ್ಳುತ್ತೇನೆ’ ಎಂದಳು. ಕಾಮದಿಂದ ಕುರುಡಾಗಿದ್ದ ಅವರಿಗೆ ಅವಳು ಹೇಳಿದುದೆಲ್ಲ ಒಪ್ಪಿಗೆಯೆ. ‘ಓಹೋ’ ಎಂದು ಅವರು ತಮ್ಮ ಒಪ್ಪಿಗೆಯಿತ್ತರು.

ಅಮೃತವನ್ನು ಹಂಚಲು ಮೋಹಿನಿ ಸಿದ್ಧಳಾದಳು. ಆಕೆಯ ಅಪ್ಪಣೆಯಂತೆ ದೈತ್ಯ ದಾನವರೂ ದೇವತೆಗಳೂ ಸ್ನಾನಮಾಡಿ, ಹೊಸ ಬಟ್ಟೆಗಳನ್ನುಟ್ಟು, ದೂರದೂರವಾಗಿ ದೈತ್ಯ ದಾನವರೇ ಒಂದು ಸಾಲು, ದೇವತೆಗಳೇ ಒಂದು ಸಾಲು–ಆಗಿ ಕುಳಿತುಕೊಂಡರು. ಮೋಹಿನಿಯು ಜಾರುತ್ತಿರುವ ಸೆರಗನ್ನು ಒಯ್ಯಾರದಿಂದ ಎಳೆದುಕೊಳ್ಳುತ್ತಾ, ಕಾಲಂದುಗೆ ಗಳು ಝಣಝಣರೆನುವಂತೆ ದೈತ್ಯ ದಾನವರ ಸಾಲಿಗೆ ನಡೆದು, ತನ್ನ ಉಪಚಾರದ ಸವಿ ನುಡಿಗಳಿಂದಲೆ ಅವರನ್ನು ಆನಂದಪಡಿಸಿದಳು. ಅನಂತರ ದೇವತೆಗಳ ಸಾಲಿಗೆ ನಡೆದುಬಂದು ಅಮೃತವನ್ನು ಹಂಚಿದಳು. ರಾಕ್ಷಸರು ಇದನ್ನು ಕಣ್ಣಾರೆ ಕಂಡರೂ, ಮಾತನಾಡಿದರೆ ಅವಳ ಸ್ನೇಹವೆಲ್ಲಿ ತಪ್ಪುವುದೊ ಎಂಬ ಆತಂಕ! ಹೆಂಗಸಿನೊಡನೆ ಜಗಳ ಮಾಡುವುದು ಉಂಟೆ? ಇದರ ಮೇಲೆ ಅವಳು ಮೊದಲೆ ಹೇಳಿದ್ದಾಳೆ, ಮಾಡಿದುದನ್ನೆಲ್ಲ ಒಪ್ಪಿಕೊಳ್ಳಬೇಕೆಂದು! ಹೀಗಾಗಿ ಅವರು ನುಂಗಲಾರದ ತುತ್ತನ್ನು ನುಂಗುತ್ತಾ ಮೌನವಾಗಿ ಕುಳಿತಿರಬೇಕಾಯಿತು. ಅಷ್ಟರಲ್ಲೆ ಒಂದು ಅಚಾತುರ್ಯ ನಡೆಯಿತು. ಸ್ವರ್ಭಾನು ಎಂಬ ಒಬ್ಬ ರಾಕ್ಷಸ; ಅವನು ಯಾವ ಮಾಯದಲ್ಲೋ ದೇವತೆಗಳ ಸಾಲಿಗೆ ಸೇರಿಕೊಂಡು, ಅವ ರೊಡನೆ ಅಮೃತವನ್ನು ಕುಡಿಯುತ್ತಿದ್ದ. ಅವನ ಇಕ್ಕೆಲದಲ್ಲಿ ಕುಳಿತಿದ್ದ ಸೂರ್ಯ ಚಂದ್ರರು ಇದನ್ನು ಕಂಡು, ಆ ಸಂಗತಿಯನ್ನು ಮೋಹಿನಿಗೆ ತಿಳಿಸಿದರು. ತಕ್ಷಣವೇ ಮೋಹಿನಿಯ ರೂಪ ಮಾಯವಾಯಿತು; ಶ್ರೀವಿಷ್ಣು ಪ್ರತ್ಯಕ್ಷನಾದ. ಆತನ ಕೈಯ ಚಕ್ರ, ಸ್ವರ್ಭಾನುವಿನ ತಲೆಯನ್ನು ಕತ್ತರಿಸಿಹಾಕಿತು. ಆದರೇನು? ಅಮೃತವನ್ನು ಕುಡಿದಿದ್ದ ಆ ತಲೆಗೆ ಅಮೃತತ್ವ ಪ್ರಾಪ್ತಿಯಾಗಿತ್ತು. ಬ್ರಹ್ಮನು ಆ ತಲೆಯನ್ನೆ ಒಂದು ಗ್ರಹವಾಗಿ ಮಾಡಿದನು. ಅದೇ ರಾಹು ಗ್ರಹ. ಆ ಗ್ರಹವು ಪರ್ವಕಾಲದಲ್ಲಿ ಇಂದಿಗೂ ಸೂರ್ಯ ಚಂದ್ರರ ಮೇಲಿನ ದ್ವೇಷದಿಂದ ಅವರನ್ನು ತುಡುಕುತ್ತದೆ.

ಹೀಗೆ ದೇವತೆಗಳೂ ರಾಕ್ಷಸರೂ ಅಮೃತಕ್ಕಾಗಿ ಸಮಸಮವಾಗಿ ಕಷ್ಟಪಟ್ಟರೂ ದೈವಾನು ಗ್ರಹವಿದ್ದ ದೇವತೆಗಳಿಗೆ ಅದು ದಕ್ಕಿತು; ದೈವಕೃಪೆಯಿಲ್ಲದ ರಾಕ್ಷಸರಿಗೆ ದಕ್ಕಲಿಲ್ಲ. ನಿತ್ಯತ್ವ ವನ್ನು ನೀಡುವ ಅಮೃತವನ್ನು ತನ್ನ ಭಕ್ತರಿಗೆ ನೀಡಿದಮೇಲೆ ಶ್ರೀಹರಿಯು ಗರುಡವನ್ನೇರಿ ಹೊರಟುಹೋದನು.

