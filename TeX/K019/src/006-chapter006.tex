
\chapter{೬. ವಿದುರ ಮೈತ್ರೇಯರ ಸಂಭಾಷಣೆ}

ಯಮುನಾನದಿಯ ತೀರದಲ್ಲಿ ಉದ್ಧವನನ್ನು ಬೀಳ್ಕೊಟ್ಟು ವಿದುರನು ಅಲ್ಲಿಂದ ಹರಿ ದ್ವಾರಕ್ಕೆ ಬಂದು, ಸುಜ್ಞಾನನಿಧಿಯಾದ ಮೈತ್ರೇಯನನ್ನು ಕಂಡನು. ದೀರ್ಘದಂಡ ನಮಸ್ಕಾರ ಮಾಡಿ, ಕುಶಲ ಪ್ರಶ್ನೆಯನ್ನು ಕೇಳಿದ ಬಳಿಕ ಅವನು ಆ ಆತ್ಮಜ್ಞಾನಿಯ ಹತ್ತಿರ ಕೈಮುಗಿದುಕೊಂಡು ಕುಳಿತು ‘ಪೂಜ್ಯರೆ, ಲೋಕದಲ್ಲಿರುವ ಜನರೆಲ್ಲರೂ ಸುಖವನ್ನು ಬಯಸಿಯೇ ಲೌಕಿಕ ವೈದಿಕ ಕಾರ್ಯಗಳನ್ನು ಮಾಡುವರು. ಆದರೆ ಅದರಿಂದ ಸುಖವಾಗಲಿ, ದುಃಖನಿವೃತ್ತಿಯಾಗಲಿ ದೊರೆಯುತ್ತಿಲ್ಲ. ಅಷ್ಟೇ ಅಲ್ಲ, ಮತ್ತೆ ಮತ್ತೆ ಅವರು ದುಃಖಕ್ಕೆ ಒಳಗಾಗುವರು. ಈ ದುಃಖನಿವಾರಣೆಗೂ ಉತ್ತಮ ಸುಖಸಾಧನೆಗೂ ಮಾರ್ಗವೇ ಇಲ್ಲವೇ? ದುರದೃಷ್ಟದಿಂದ ಭಕ್ತಿಹೀನರಾಗಿ, ದುಷ್ಕಾರ್ಯಗಳಲ್ಲಿ ಮುಳುಗಿರುವವರನ್ನು ಉದ್ಧ‡ರಿಸಲೆಂದೇ ನಿಮ್ಮಂತಹ ಮಹನೀಯರು ಅವತರಿಸಿ, ಲೋಕಸಂಚಾರ ಕೈಕೊಳ್ಳು ವುದು. ಭ್ರಮರವು ಹಲವು ಪುಷ್ಪಗಳಿಂದ ಸಾರವನ್ನು ಸಂಗ್ರಹಿಸುವಂತೆ ಸಕಲಶಾಸ್ತ್ರಗಳ ಸಾರವನ್ನೂ ಅರಿತಿರುವವರು ನೀವು. ಆದ್ದರಿಂದ, ಜನರೆಲ್ಲರೂ ಭಗವಂತನ ಧ್ಯಾನದಲ್ಲಿ ತತ್ಪರರಾಗಿ, ಶಾಶ್ವತವಾದ ಆನಂದವನ್ನು ಪಡೆಯಲು ತಕ್ಕ ತತ್ವಜ್ಞಾನವಾವುದೊ, ಅದನ್ನು ನನಗೆ ಬೋಧಿಸಿ. ಅಲ್ಲದೆ, ಈ ಜಗತ್ತಿನ ಸೃಷ್ಟಿಯಾದುದು ಹೇಗೆ! ಏಕನಾದ ಭಗವಂತ ಅನೇಕನಾದುದು ಹೇಗೆ? ಆತನ ಅವತಾರಗಳಾವುವು? ಇದೆಲ್ಲವನ್ನೂ ನನಗೆ ತಿಳುಹಿಸಿ. ಸ್ವಾಮಿ, ನಾನು ಹಿಂದೆ ನಿಮ್ಮ ಮಿತ್ರರಾದ ವ್ಯಾಸರಿಂದ ವರ್ಣಾಶ್ರಮಧರ್ಮಗಳನ್ನೆಲ್ಲಾ ಕೇಳಿದ್ದೇನೆ. ಈಗ ನನಗೆ ಅವುಗಳಲ್ಲಿ ಆಸಕ್ತಿಯಿಲ್ಲ. ಭಗವಂತನ ದಿವ್ಯ ಲೀಲೆಗಳಲ್ಲಿಯೇ ಈಗ ನನ್ನ ಮನಸ್ಸು’ ಎಂದು ಹೇಳಿದನು.

ವಿದುರನ ಮಾತುಗಳಿಂದ ಮೈತ್ರೇಯನಿಗೆ ಸಂತೋಷವಾಯಿತು. ಆತನನ್ನು ಕುರಿತು ಮೈತ್ರೇಯನು “ವಿದುರ, ನೀನೇನು ಸಾಮಾನ್ಯನೆ? ಸಾಕ್ಷಾತ್ ವ್ಯಾಸರ ಅನುಗ್ರಹದಿಂದ ಹುಟ್ಟಿದವನು. ಮಾಂಡವ್ಯ ಪುಷಿಯ ಶಾಪದಿಂದ ವಿಚಿತ್ರವೀರ್ಯನ ದಾಸಿಯ ಗರ್ಭದಲ್ಲಿ ಜನಿಸಿ ಬಂದ ಸಾಕ್ಷಾತ್ ಯಮನೇ ನೀನು. ಶ್ರೀಕೃಷ್ಣಪರಮಾತ್ಮನು ಈ ಲೋಕವನ್ನು ಬಿಡುವ ಮುನ್ನ, ನಿನಗೆ ಜ್ಞಾನೋಪದೇಶ ಮಾಡುವಂತೆ ನನಗೆ ಅಪ್ಪಣೆ ಮಾಡಿರುವನು. ಆದ್ದರಿಂದ ನಾನು ಅಗತ್ಯವಾಗಿಯೂ ನಿನ್ನ ಅಪೇಕ್ಷೆಯನ್ನು ಸಲ್ಲಿಸುವೆನು. ನೋಡು, ಸೃಷ್ಟಿ ಸ್ಥಿತಿ ಲಯಗಳು ಭಗವಂತನ ಲೀಲೆ. ಸಕಲ ಜೀವರಿಗೂ ಆತ್ಮನಾಗಿರುವ ಆ ಪರಮಾತ್ಮನು ಜಗತ್​ಸೃಷ್ಟಿಗಿಂತಲೂ ಮುಂಚೆ ಅದ್ವಿತೀಯನಾಗಿ ಜ್ಞಾನರೂಪದಿಂದ ಬೆಳಗುತ್ತಿದ್ದನು. ಏಕವೇ ಅನೇಕವಾಗುವ, ದೃಶ್ಯಾದೃಶ್ಯವೆಂದು ವಿಭಾಗವಾಗುವ ಶಕ್ತಿ ಆತನಲ್ಲಿ ಅಡಗಿತ್ತು. ಆ ಶಕ್ತಿಯನ್ನೆ ‘ಮಾಯೆ’ ಎನ್ನುವರು. ಸತ್ವ, ರಜ, ತಮೋಗುಣಗಳಿಂದ ಕೂಡಿದ ಈ ಮಾಯೆಯನ್ನು ಅವ್ಯಕ್ತವೆಂದೂ ಕರೆಯುವರು. ಈ ಮಾಯೆಯಿಂದಲೆ ಮನಸ್ಸು, ಬುದ್ಧಿ, ಅಹಂಕಾರ, ಇಂದ್ರಿಯಗಳು ಇತ್ಯಾದಿ ಸಮಸ್ತ ಜಗತ್ತೂ ಉದ್ಭವಿಸಿತು” ಎಂದನು.

ವಿದುರನಿಗೆ ಒಂದು ಸಂದೇಹ ಬಂತು. “ಕೇವಲ ಜ್ಞಾನರೂಪನಾಗಿ ನಿರ್ವಿಕಾರನಾಗಿರುವ ಭಗವಂತನನ್ನು ‘ಸೃಷ್ಟಿಕರ್ತ’ ಎಂದು ಕರೆಯುವುದು ಹೇಗೆ? ಏಕೆ ಆತ ಸೃಷ್ಟಿಕಾರ್ಯವನ್ನು ಕೈಕೊಳ್ಳಬೇಕು? ತನ್ನ ‘ಲೀಲೆಗಾಗಿ’ ಎಂದು ಹೇಳುವುದಾದರೆ ನಮ್ಮಂತೆ ಆತನೂ ರಾಗ ಯುಕ್ತನೆನ್ನಬೇಕಾಯಿತು. ಮಕ್ಕಳು ಕೂಡ ಆಟವಾಡಬೇಕೆಂದು ಅಪೇಕ್ಷಿಸುವರು. ನಿತ್ಯಾನಂದಸ್ವರೂಪನಾದ ಭಗವಂತನಲ್ಲಿ ಇಚ್ಛೆ ಹುಟ್ಟುವುದು ಹೇಗೆ? ಭಗವಂತನು ತನ್ನ ಮಾಯೆಯಿಂದ ಸೃಷ್ಟಿ ಸ್ಥಿತಿ ಲಯಗಳನ್ನು ನಡೆಸುವನೆಂದು ಹೇಳುವುದಾದರೆ, ಚಿದ್ರೂಪ ನಾದ ಆತನಿಗೆ ಮಾಯೆಯ ಸಂಬಂಧ ಹೇಗೆ ಘಟಿಸೀತು? ಇನ್ನೊಂದು ಪ್ರಶ್ನೆ–ಭಗ ವಂತನು ಎಲ್ಲ ದೇಹಗಳಲ್ಲಿಯೂ ನೆಲಸಿರುವನು. ಎಂದರೆ ಪ್ರತಿಯೊಬ್ಬ ಆತ್ಮನೂ ಪರ ಮಾತ್ಮನ ಸ್ವರೂಪವೇ, ಅಥವಾ ಪರಮಾತ್ಮನ ಅಂಶ ಎಂದಮೇಲೆ ಈ ಜೀವಾತ್ಮನು ಮಾಯೆಯ ಉಪಾಧಿಗೆ ಒಳಗಾಗುವುದೇಕೆ? ಮಾನವನೆಂದರೆ ಭಗವಂತನೆ ಎಂದ ಮೇಲೆ ಅವನು ದುಃಖ ಸಂಕಟಗಳಿಗೆ ಸಿಕ್ಕಿ ನರಳುವುದೇಕೆ?” ಎಂದು ಕೇಳಿದನು.

ಮೈತ್ರೇಯರು ನಸುನಗುತ್ತಾ ಉತ್ತರವಿತ್ತರು–‘ಮಿತ್ರ, ಕನಸಿನಲ್ಲಿ ತಲೆ ಕತ್ತರಿಸಿ ಹೋದಂತೆ ಕಂಡಮಾತ್ರಕ್ಕೆ ತಲೆ ಕತ್ತರಿಸಿಹೋಗಿದೆಯೆ? ಜಲದಲ್ಲಿ ಪ್ರತಿಬಿಂಬಿತವಾದ ಚಂದ್ರ ಅಲೆಗಳಿಂದ ಚಲಿಸಿದಂತೆ ಕಂಡುಬಂದ ಮಾತ್ರಕ್ಕೆ ಚಂದ್ರ ಹಾಗೆ ಚಲಿಸಿದನೆಂದು ಅರ್ಥವೆ? ಹಾಗೆಯೆ, ಜಡವಾದ ಈ ದೇಹದ ವಿಕಾರವನ್ನು ನೀನು ಜೀವನದಲ್ಲಿ ಆರೋಪಿ ಸುತ್ತಿರುವೆ. ಇದೇ ಅಜ್ಞಾನ, ಮಾಯೆ. ಭಗವಂತನ ಅನುಗ್ರಹದಿಂದಲೂ ಭಕ್ತಿಯೋಗ ದಿಂದಲೂ ಕ್ರಮೇಣ ಈ ಅಜ್ಞಾನ ತೊಲಗುವುದು. ಆಗ ಆತ್ಮನು ನಿತ್ಯಶುದ್ಧನೆಂಬುದು ಅರಿವಾಗುವುದು. ಒಂದು ಮಾತಿನಲ್ಲಿ ಹೇಳುವುದಾದರೆ, ಭಗವಂತನಲ್ಲಿ ಪ್ರೇಮ ಹುಟ್ಟಿ ದಾಗ ದುಃಖ ನಿರ್ಮೂಲವಾಗುತ್ತದೆ.’

ಮೈತ್ರೇಯರ ಉಪದೇಶವೆಂಬ ಖಡ್ಗದಿಂದ ವಿದುರನ ಸಂಶಯವೆಲ್ಲ ಕತ್ತರಿಸಿ ಹೋಯಿತು. ಆತನು ಮೈತ್ರೇಯರನ್ನು ಕುರಿತು “ಮಹರ್ಷಿ, ನಿಮ್ಮ ಕೃಪೆಯಿಂದ ನನಗೆ ಜ್ಞಾನೋದಯವಾಯಿತು. ‘ದೇವರು ಸ್ವತಂತ್ರನಾಗಿ, ಮಾನವನು ಮಾತ್ರ ಬದ್ಧನಾಗಿರುವು ದೇಕೆ?’ ಎಂಬ ಸಂದೇಹ ನನ್ನನ್ನು ಬಾಧಿಸುತ್ತಿತ್ತು. ಈಗ ಅರ್ಥವಾಯಿತು, ಮಾಯೆ ಭಗವಂತನ ಅಧೀನ, ಮಾನವ ಮಾಯೆಯ ಅಧೀನ–ಎಂಬುದು. ಮಾನವ ಬದ್ಧನಾಗಿರು ವುದರ ಕಾರಣ ಇದೇ. ಜಗತ್ತಾಗಿ ಕಾಣುವುದೆಲ್ಲವೂ ಮಿಥ್ಯ, ಅಜ್ಞಾನಜನ್ಯ; ಆ ಅಜ್ಞಾನವೇ ಮಾಯೆ. ಎಲ್ಲವೂ ಪರಮಾತ್ಮ ಸ್ವರೂಪವೇ ಹೊರತು ಸ್ವತಂತ್ರವಾದುದು ಯಾವುದೂ ಇಲ್ಲ. ನಮ್ಮ ಕಣ್ಣಿಗೆ ಕಾಣದ ಮಾತ್ರಕ್ಕೆ ಅದನ್ನು ಇಲ್ಲವೆಂದು ಹೇಳುವುದು ತಪ್ಪು. ದೇವ ಮನುಷ್ಯಾದಿ ಶರೀರಗಳಲ್ಲಿ ಜೀವನಿರುವುದು ನಮ್ಮ ಕಣ್ಣಿಗೆ ಕಾಣುವುದಿಲ್ಲವಾದರೂ, ಅದು ಇಲ್ಲವೆನ್ನಬಹುದೆ? ಸ್ವಾಮಿ, ಲೋಕದಲ್ಲಿ ಇಬ್ಬರೆ ಸುಖಿಗಳು. ಕೇವಲ ಮೂಢ ಮತ್ತು ಕೇವಲ ಜ್ಞಾನಿ. ಸಂಪೂರ್ಣ ಜ್ಞಾನಿಯೂ ಅಲ್ಲ, ಪರಮ ಮೂಢನೂ ಅಲ್ಲದ ಮನುಷ್ಯ ಪರಮ ದುಃಖಿ. ಮೂಢನ ಅಜ್ಞತೆಯೇ ಅವನ ಸುಖಕ್ಕೆ ಕಾರಣವಾಗುತ್ತದೆ. ಜ್ಞಾನಿ ಗಳಿಗೆ ಸುಖದ ಗುಟ್ಟು ಗೊತ್ತಿದೆ. ಎರಡೂ ಇಲ್ಲದ ಸಂದೇಹ ಮನಸ್ಸಿನವನಿಗೆ ನೆಮ್ಮದಿ ಯೆಂಬುದೇ ಇಲ್ಲ. ನಿಮ್ಮಂತಹ ಮಹಾತ್ಮರ ಪಾದದರ್ಶನ, ಪಾದಸೇವೆಗಳಿಂದ ಮಾತ್ರವೇ ಭಗವಂತನಲ್ಲಿ ಭಕ್ತಿಪ್ರೇಮಗಳು ಉದಿಸಿ, ಸುಖ ಶಾಂತಿಗಳು ದೊರೆಯುವುದು ಸಾಧ್ಯ” ಎಂದು ಹೇಳಿ, ಆತನಿಗೆ ಭಕ್ತಿಯಿಂದ ನಮಸ್ಕರಿಸಿದನು.

ಸಾತ್ವಿಕನಾದ ವಿದುರನಿಗೆ ಜಗತ್​ಸೃಷ್ಟಿಯ ಕಥೆಯನ್ನೆಲ್ಲಾ ಆದ್ಯಂತವಾಗಿ ಮೈತ್ರೇಯ ರಿಂದ ಕೇಳಬೇಕೆನ್ನಿಸಿತು. ಆತ ತನ್ನ ಆಶೆಯನ್ನು ಅವರಲ್ಲಿ ಹೇಳಿ ‘ಸ್ವಾಮಿ, ಸಹಸ್ರಪಾದ, ಸಹಸ್ರಬಾಹು, ಸಹಸ್ರನೇತ್ರ ಇತ್ಯಾದಿಗಳಿಂದ ಕೂಡಿದ ವಿರಾಟ್​ಪುರುಷನಿಂದ ಈ ಜಗತ್ತೆಲ್ಲಾ ಜನಿಸಿತೆಂದು ಹೇಳುತ್ತಾರೆ. ಅದನ್ನು ನನಗೆ ಸಮಗ್ರವಾಗಿ ತಿಳಿಸಿ’ ಎಂದು ಬೇಡಿಕೊಂಡ. ವಿದುರನ ಆಸಕ್ತಿಗೆ ತಕ್ಕ ಉತ್ಸಾಹ ಮೈತ್ರೇಯರದು. ಅವರು ಹೇಳಿದರು –‘ಅಯ್ಯಾ ವಿದುರ, ಸೃಷ್ಟಿಯೆಂಬುದು ಅನಾದಿ. ನಾವಿರುವ ಈ ಸೃಷ್ಟಿ ಹಿಂದಿನ ಮತ್ತು ಮುಂದಿನ ಸೃಷ್ಟಿಗಳ ಸರಪಣಿಯಲ್ಲಿ ಒಂದು ಕೊಂಡಿ ಮಾತ್ರ. ಸಚ್ಚಿದಾನಂದಶಕ್ತಿ ವ್ಯಕ್ತ ಅವ್ಯಕ್ತಗಳಿಗೆ ಬದಲಾಯಿಸುತ್ತಿರುತ್ತದೆ. ವ್ಯಕ್ತವಾದಾಗ ಸೃಷ್ಟಿ; ಅವ್ಯಕ್ತ ವಾದಾಗ ಪ್ರಳಯ. ಪ್ರಳಯದಲ್ಲಿ ಜಗತ್ತೆಲ್ಲವೂ ನೀರಿನಲ್ಲಿ ಮುಳುಗಿ ಹೋಗಿರುವಾಗ ಪರಮಾತ್ಮನೊಬ್ಬನೆ ಶೇಷಶಾಯಿಯಾಗಿ ಆತ್ಮಾನಂದದಲ್ಲಿ ಇರುವನು. ಈ ಆನಂದಸ್ಥಿತಿಯಲ್ಲಿಯೇ ಎಷ್ಟೋ ಚತುರ್ಯುಗಗಳು ಕಳೆದುಹೋಗುವುವು. ಆಗ ಆತನಿಗೆ ಕಾಲವೆಂಬ ತನ್ನ ಶಕ್ತಿಯ ಪ್ರಭಾವ ದಿಂದ ಜಗತ್​ಸೃಷ್ಟಿಯ ಪ್ರಚೋದನೆಯಾಗುವುದು. ಒಡನೆಯೆ ಅವ್ಯಕ್ತಜಗತ್ತು ಸೂರ್ಯ ನಂತೆ ಬೆಳಗುವ ಕಮಲದ ಆಕಾರವನ್ನು ತಾಳಿ ಆತನ ಹೊಕ್ಕುಳಿನಿಂದ ಹೊರಚಾಚುವುದು. ಪರಮಾತ್ಮನು ಅದನ್ನು ಪ್ರವೇಶಿಸುವನು. ಆಗ ಸ್ವಯಂಭುವಾದ ಬ್ರಹ್ಮ ಉದಯಿಸು ವನು. ಆ ಬ್ರಹ್ಮ ಕಮಲ ಮಧ್ಯದಲ್ಲಿ ನಿಂತು ನಾಲ್ಕು ದಿಕ್ಕುಗಳಿಗೂ ತನ್ನ ಕಣ್ಣನ್ನು ತಿರುಗಿಸುವನು. ಇದರಿಂದ ಆತನಿಗೆ ನಾಲ್ಕು ಮುಖಗಳಾಗುತ್ತವೆ.

ಚತುರ್ಮುಖನು ತನ್ನಲ್ಲಿ ತಾನೆ ‘ನಾನು ಯಾರು? ಈ ಕಮಲ ನೀರಿನಲ್ಲಿ ಏಕೆ ಹುಟ್ಟಿತು? ಇದಕ್ಕೆ ಆಧಾರ ಯಾವುದು?’ ಎಂದು ಚಿಂತಿಸುತ್ತಾ ಆ ಕಮಲನಾಳದ ರಂಧ್ರದೊಳಗಿನಿಂದ ಕೆಳಕ್ಕೆ ಇಳಿದು ಹೋದನು. ಎಷ್ಟು ಇಳಿದರೂ ಅದರ ನೆಲೆ ಸಿಕ್ಕಲಿಲ್ಲ. ಕಡೆಗೆ ಆತನು ಮೊದಲಿನ ಸ್ಥಳಕ್ಕೆ ಹಿಂದಿರುಗಿ, ಯೋಗಾರೂಢನಾಗಿ ಕುಳಿತು, ಬಹುಕಾಲ ತಪಸ್ಸು ಮಾಡಿ ದನು. ಆಗ ಆತನ ಹೃದಯದಲ್ಲಿ ಪರಾತ್ಪರಮೂರ್ತಿ ಕಾಣಿಸಿಕೊಂಡಿತು. ಅದು ಆದಿಶೇಷನ ಮೇಲೆ ಸಾವಿರ ಹೆಡೆಗಳ ನೆರಳಲ್ಲಿ ಮಲಗಿತ್ತು. ಸಂಜೆಯ ಮೋಡವನ್ನು ಹೊದ್ದ ಮರಕತ ಶೈಲದಂತೆ ಕಂಗೊಳಿಸುತ್ತಿತ್ತು, ಆ ಪೀತಾಂಬರ ಧಾರಿ. ಸಹಜ ಸುಂದರವಾಗಿದ್ದ ಆ ಮೂರ್ತಿ ಮೂರು ಲೋಕಗಳನ್ನೂ ವ್ಯಾಪಿಸಿತ್ತು. ಮುಖದಲ್ಲಿ ಮುಗುಳ್ನಗೆ ಚೆಲ್ಲುತ್ತಿದ್ದ ಆ ದಿವ್ಯ ಮೂರ್ತಿಗೆ ಭಕ್ತಿಯಿಂದ ನಮಸ್ಕರಿಸಿದ ಚತುರ್ಮುಖನು ‘ಹೇ ಲೋಕೇಶ್ವರ, ನಿನ್ನ ದರ್ಶನದಿಂದ ನಾನು ಧನ್ಯನಾದೆ. ಆದ್ಯಂತರಹಿತನಾದ ನೀನೆ ನಿತ್ಯ, ಸತ್ಯ ಆನಂದ. ನಿನಗಿದೊ ನನ್ನ ನಮಸ್ಕಾರ. ಅನಾದಿನಿಧನನಾದ ನಿನ್ನನ್ನು ಶರಣು ಹೊಕ್ಕಿದ್ದೇನೆ. ಪೂರ್ವಕಲ್ಪದಲ್ಲಿ ಇದ್ದಂತೆಯೇ ಮತ್ತೆ ಜಗತ್ತನ್ನು ಸೃಷ್ಟಿಸಬೇಕೆಂದಿರುವ ನನಗೆ ಜ್ಞಾನವನ್ನು ಕರುಣಿಸು’ ಎಂದು ಪ್ರಾರ್ಥಿಸಿದನು. ಆತನ ಸ್ತುತಿಯಿಂದ ಸಂತುಷ್ಟಿಗೊಂಡ ಆ ದಿವ್ಯಮೂರ್ತಿಯು ಯೋಗನಿದ್ರೆಯಿಂದ ಮೇಲೆದ್ದು, ಅಮೃತದಂತಿರುವ ಮೃದುನುಡಿ ಗಳಿಂದ ‘ಮಗು, ಕಳವಳ ಪಡಬೇಡ. ನೀನು ಕೇಳಿದ ವರಗಳನ್ನು ನಾನಾಗಲೆ ನಿನಗೆ ಕೊಟ್ಟಿದ್ದೇನೆ. ನೀನು ಮರಳಿ ತಪಸ್ಸು ಮಾಡು. ಮರೆಯಾಗಿರುವ ಎಲ್ಲ ಲೋಕಗಳೂ ಮತ್ತೆ ನಿನಗೆ ಕಾಣಿಸುತ್ತವೆ. ನನ್ನ ಅನುಗ್ರಹಕ್ಕೆ ಪಾತ್ರನಾದುದರಿಂದ ನಿನಗೆ ಸೃಷ್ಟಿ ಕಾರ್ಯದಲ್ಲಿ ಸ್ವಲ್ಪವೂ ಬೇಸರವಾಗುವುದಿಲ್ಲ. ನಿನಗೆ ಮಂಗಳವಾಗಲಿ’ ಎಂದು ಹರಸಿ ಕಣ್ಮರೆ ಯಾಯಿತು.

ದಿವ್ಯಮೂರ್ತಿಯ ಅಪ್ಪಣೆಯಂತೆ ಚತುರ್ಮುಖನು ದೇವಮಾನದ ನೂರಾರು ವರ್ಷ ಗಳವರೆಗೆ ತಪಸ್ಸು ಮಾಡಿ ಅದ್ಭುತವಾದ ಶಕ್ತಿಯನ್ನೂ ಅಖಂಡವಾದ ಜ್ಞಾನವನ್ನೂ ಪಡೆದನು. ಅನಂತರ ಒಮ್ಮೆ ಆತನು ಸುತ್ತಲೂ ನೋಡಿದನು. ಪ್ರಳಯಕಾಲದ ಬಿರುಗಾಳಿ ಬೀಸುತ್ತಿದೆ; ಅದರ ರಭಸಕ್ಕೆ ಪ್ರಳಯಜಲ ಅಲ್ಲೋಲ ಕಲ್ಲೋಲವಾಗಿ ಅಲೆಯೆದ್ದು ಬೀಳು ತ್ತಿದೆ. ಚತುರ್ಮುಖನು ಆ ಗಾಳಿ ನೀರುಗಳೆರಡನ್ನೂ ಕುಡಿದು ಬಿಟ್ಟನು. ಅನಂತರ ಆತನು ತಾನಿದ್ದ ಕಮಲವನ್ನೇ ವ್ಯಾಪಿಸಿ ಭೂ, ಸ್ವರ್ಗ, ಪಾತಾಳಗಳೆಂಬ ಮೂರು ಲೋಕಗಳನ್ನು ಅದರಲ್ಲಿಯೇ ವಿಭಾಗಿಸಿದನು. ಅನಂತರ ಭಗವಂತನು ಸೃಷ್ಟಿಯ ಸಂಕಲ್ಪವನ್ನು ಕೈ ಕೊಂಡನು. ಒಡನೆಯೇ ಅದುವರೆಗೆ ಸಮಸ್ಥಿತಿಯಲ್ಲಿದ್ದ ಸತ್ವ, ರಜ, ತಮೋಗುಣಗಳಲ್ಲಿ ವಿಷಮಸ್ಥಿತಿಯುಂಟಾಗಿ, ಮಹತ್ತತ್ವ ಜನಿಸಿತು. ಇದೇ ಸೃಷ್ಟಿಗಳಲ್ಲಿ ಮೊಟ್ಟ ಮೊದಲನೆ ಯದು. ಇದಾದ ಮೇಲೆ ಭೂತೇಂದ್ರಿಯ ದೇವತಾಸೃಷ್ಟಿಗೆ ಕಾರಣವಾದ ಅಹಂಕಾರ ಉದಿ ಸಿತು. ಇದರ ತರುವಾಯ ಶಬ್ದಾದಿತನ್ಮಾತ್ರಗಳಿಗೂ ಪೃಥಿವ್ಯಾದಿ ಭೂತಗಳಿಗೂ ಕಾರಣವಾದ ಸೂಕ್ಷ್ಮ ಭೂತಗಳ ಸೃಷ್ಟಿಯಾಯಿತು. ಇದರ ಹಿಂದೆಯೇ ಜ್ಞಾನ, ಕರ್ಮ ರೂಪವಾದ ಇಂದ್ರಿಯಗಳೂ, ಅವುಗಳನ್ನು ಅನುಸರಿಸಿ ಇಂದ್ರಿಯಾಭಿಮಾನಿಗಳಾದ ದೇವತೆಗಳೂ, ಮನಸ್ಸೂ ಸೃಷ್ಟಿಯಾದವು. ಆರನೆಯದಾಗಿ ಅವಿದ್ಯೆಯ ಸೃಷ್ಟಿಯಾಯಿತು. ಇವು ಆರೂ ಚತುರ್ಮುಖ ಬ್ರಹ್ಮನಿಗಿಂತಲೂ ಪೂರ್ವದಲ್ಲಿಯೇ ಇದ್ದವುಗಳಾದ್ದರಿಂದ ಇವುಗಳನ್ನು ಪರಬ್ರಹ್ಮನ ಸಂಕಲ್ಪದಿಂದಲೇ ಸೃಷ್ಟಿಯಾದುವುಗಳೆಂದು ತಿಳಿಯಬೇಕು. ಆದ್ದರಿಂದಲೇ ಇವನ್ನು ಪ್ರಾಕೃತಸೃಷ್ಟಿಯೆಂದು ಕರೆಯುತ್ತಾರೆ. ಇನ್ನು ಚತುರ್ಮುಖ ಬ್ರಹ್ಮನಿಂದ ಸೃಷ್ಟಿಯಾದ ವೈಕೃತಸೃಷ್ಟಿಯಲ್ಲಿ ಸ್ಥಾವರ, ತಿರ್ಯಕ್, ಮಾನವ ಎಂದು ಮೂರು ವಿಧ. ತರುಲತೆ ಗುಲ್ಮಾದಿ ಸಸ್ಯವರ್ಗಕ್ಕೆ ಸೇರಿದವುಗಳು ಸ್ಥಾವರಗಳು. ಇವು ಚಲಿಸಲಾರವು. ತಮ್ಮ ಅಂತರಂಗದಲ್ಲಿ ಶೀತೋಷ್ಣ ಸ್ಪರ್ಶಾದಿಗಳನ್ನು ಅನುಭವಿಸಬಲ್ಲವಾದರೂ ತಮ್ಮ ಚೈತನ್ಯ ವನ್ನು ಹೊರಪಡಿಸಲಾರವು. ತಿರ್ಯಕ್ ಜಾತಿಗೆ ಸೇರಿದವುಗಳು ಚಲಿಸಬಲ್ಲವಾದರೂ ಪೂರ್ವಾಪರ ಜ್ಞಾನವುಳ್ಳವುಗಳಲ್ಲ; ಕೇವಲ ಮೂಗಿನಿಂದ ತಮ್ಮ ಇಷ್ಟಾನಿಷ್ಟಗಳನ್ನು ತಿಳಿದುಕೊಳ್ಳುತ್ತವೆ; ‘ಆಹಾರ ನಿದ್ರಾಭಯಮೈಥುನ’ಗಳನ್ನು ಮಾತ್ರ ಅವು ಬಲ್ಲವು; ಇವುಗಳನ್ನು ಇಪ್ಪತ್ತೆಂಟು ಜಾತಿಗಳಾಗಿ ವಿಭಾಗಿಸುತ್ತಾರೆ.\footnote{*ಎಮ್ಮೆ, ಕರೀ ಜಿಂಕೆ, ಹಂದಿ, ಕಡವೆ, ಚುಕ್ಕೆಯ ಜಿಂಕೆ, ಕುರಿ, ಒಂಟೆ, ಸೀಳಿದ ಗೊರಸುಗಳುಳ್ಳ ಹಸು, ಆಡು ಈ ಒಂಭತ್ತು ಪ್ರಾಣಿಗಳು; ಒಂದೇ ಗೊರಸುಳ್ಳ ಕುದುರೆ, ಕತ್ತೆ, ಹೇಸರ ಕತ್ತೆ, ಗೌರಮೃಗ, ಶರಭ, ಚಮರೀಮೃಗ ಇವಾರು ಪ್ರಾಣಿಗಳು; ಐದು ಉಗುರುಗಳುಳ್ಳ ನಾಯಿ, ನರಿ, ತೋಳ, ಹುಲಿ, ಬೆಕ್ಕು, ಮೊಲ, ಮುಳ್ಳುಹಂದಿ, ಸಿಂಹ, ಕಪಿ, ಆನೆ, ಆಮೆ, ಉಡು, ಈ ಹನ್ನೆರಡು ಪ್ರಾಣಿಗಳು; ಜಲಚರಗಳೂ ಪಕ್ಷಿಗಳೂ ಈ ವರ್ಗಕ್ಕೇ ಸೇರುತ್ತವೆ.} ಇನ್ನು ಮಾನವ ವರ್ಗ. ಇವರು ರಜೋಗುಣಪ್ರಧಾನರಾಗಿ ಕರ್ಮನಿರತ ರಾಗಿರುವರು; ಕಾಮ ಕ್ರೋಧಗಳಿಂದ ತುಂಬಿದವರು; ದುಃಖವನ್ನೇ ಸುಖವೆಂದು ಭಾವಿಸಿ ದುಡಿಯುವವರು. ದೇವತೆಗಳು ಪ್ರಾಕೃತಸೃಷ್ಟಿಗೆ ಸೇರಿದವರೆಂದು ಮೇಲೆ ಹೇಳಿದೆಯಾದರೂ, ಅವರೂ ವೈಕೃತಸೃಷ್ಟಿಯಲ್ಲಿಯೇ ಸೇರಿಕೊಳ್ಳುತ್ತಾರೆ. ಇವರಲ್ಲಿ ವಿಬುಧ, ಪಿತೃ, ಅಸುರ, ಗಂಧರ್ವ, ಅಪ್ಸರ, ಯಕ್ಷ ರಾಕ್ಷಸ, ಸಿದ್ಧ ಚಾರಣ ವಿದ್ಯಾಧರ, ಕಿನ್ನರ ಕಿಂಪುರುಷ, ಭೂತ, ಪ್ರೇತ, ಪಿಶಾಚ–ಎಂದು ಎಂಟು ವಿಧ.

ನದಿಯಲ್ಲಿ ಹರಿಯುವ ನೀರಿನಲ್ಲಿ ಸುಳಿ, ನೊರೆ, ಗುಳ್ಳೆಗಳು ಹುಟ್ಟಿ ಅಲ್ಲಿಯೇ ಅಡಗುವಂತೆ ಭಗವಂತನ ಮಹಿಮೆಯಿಂದ ಈ ಸೃಷ್ಟಿಯು ತೋರಿ ಮತ್ತೆ ಆತನಲ್ಲಿಯೇ ಅಡಗುತ್ತದೆ.

ಅಯ್ಯಾ ವಿದುರ, ಇನ್ನು ಚತುರ್ಮುಖ ಬ್ರಹ್ಮನಿಂದ ಲೋಕಸೃಷ್ಟಿಯಾದ ರೀತಿಯನ್ನು ಹೇಳುತ್ತೇನೆ, ಕೇಳು. ಆತನು ಸೃಷ್ಟಿಗೆಂದು ಕೈ ಹಾಕಿ ಮೊಟ್ಟಮೊದಲಲ್ಲಿಯೇ ಅಂಧತಾ ಮಿಸ್ರ, ತಾಮಿಸ್ರ, ಮಹಾಮೋಹ, ಮೋಹ, ತಮಸ್ಸು–ಎಂಬ ಐದು ಅಜ್ಞಾನ ವೃತ್ತಿ ಗಳನ್ನು ನಿರ್ಮಿಸಿದನು. ತಾನು ಮಾಡಿದ ಈ ಅಕಾರ್ಯವನ್ನು ಕಂಡು ಅಸಹ್ಯಗೊಂಡ ಬ್ರಹ್ಮನು ತಮೋಗುಣದಿಂದ ತುಂಬಿದ ತನ್ನ ದೇಹವನ್ನು ಆ ಕ್ಷಣದಲ್ಲಿಯೇ ತ್ಯಜಿಸಿದನು. ಆ ದೇಹದಿಂದ ಯಕ್ಷ ರಾಕ್ಷಸರು ಹುಟ್ಟಿದರು. ಹುಟ್ಟುತ್ತಲೆ ಅವರಿಗೆ ತಡೆಯಲಾರದಷ್ಟು ಹಸಿವು ಬಾಯಾರಿಕೆಗಳು; ಅವರು ತಮ್ಮ ಹಸಿವನ್ನು ಅಡಗಿಸಲು ಬ್ರಹ್ಮನ ದೇಹವನ್ನೇ ಭಕ್ಷಿಸಹೊರಟರು. ಹಾಗೆ ಹೊರಟವರಲ್ಲಿ ‘ಮಾ ರಕ್ಷತ’ (ರಕ್ಷಿಸಬೇಡಿ) ಎಂದವರು ರಾಕ್ಷಸ ರಾದರು; ‘ಜಕ್ಷಧ್ವಂ’ (ತಿಂದು ಹಾಕಿರಿ) ಎಂದವರು ಯಕ್ಷರಾದರು. ಹೀಗೆ ಅವರು ತನ್ನ ದೇಹವನ್ನೇ ತಿನ್ನ ಹೊರಟುದನ್ನು ಕಂಡ ಬ್ರಹ್ಮನು ಭಯದಿಂದ ‘ಅಯ್ಯಾ, ನೀವು ನನ್ನ ಮಕ್ಕಳು; ನನ್ನನ್ನು ರಕ್ಷಿಸಬೇಕೇ ಹೊರತು ಭಕ್ಷಿಸಬಾರದು’ಎಂದು ಹೇಳಿ, ತಮೋಗುಣ ವನ್ನು ಹೋಗಲಾಡಿಸುವ ತೇಜಸ್ಸನ್ನು ಹೊರಹೊಮ್ಮಿ, ದೇವತೆಗಳನ್ನು ಸೃಷ್ಟಿಸಿದನು. ತೇಜಸ್ಸಿನಿಂದ ತೊಳಗಿ ಬೆಳಗುತ್ತಾ ಸಾತ್ವಿಕ ಸ್ವಭಾವದವರಾಗಿದ್ದ ಆ ದೇವತೆಗಳು ಬ್ರಹ್ಮನ ದೇಹದಿಂದ ಹೊರಹೊಮ್ಮುತ್ತಿದ್ದ ತೇಜಸ್ಸಿನಿಂದ ಪ್ರಭಾಶರೀರವನ್ನು ಪಡೆದರು. ಬ್ರಹ್ಮನ ಜಘನಪ್ರದೇಶದಿಂದ ಅಸುರರು ಉದಿಸಿದರು. ಅವರು ಅತ್ಯಂತ ಕಾಮುಕರು. ಅವರು ಕಾಮೋದ್ರೇಕದಿಂದ ತಂದೆಯಾದ ಬ್ರಹ್ಮನನ್ನೇ ಕಾಮಕೇಳಿಗೆ ಕರೆದರು. ಬ್ರಹ್ಮ ನಿಗೆ ಅದನ್ನು ಕಂಡು ನಗು ಬಂತು. ಆದರೆ ಅದು ನಗುವ ಕಾರ್ಯವಾಗಲಿಲ್ಲ. ಆ ಅಸುರರು ಬ್ರಹ್ಮನ ಮೇಲೆ ಬಿದ್ದು ಆತನನ್ನು ಆಕ್ರಮಿಸ ಹೊರಟರು. ಇದನ್ನು ಕಂಡು ಲಜ್ಜೆಯೊಡನೆ ಭಯವೂ ಆಯಿತು, ಬ್ರಹ್ಮನಿಗೆ. ಆತ ಶ್ರೀಹರಿಯಲ್ಲಿಗೆ ಹೋಗಿ ತನ್ನ ದುರವಸ್ಥೆಯನ್ನು ಹೇಳಿಕೊಂಡ. ಆಗ ಮಹಾ ವಿಷ್ಣುವು ‘ಅಯ್ಯಾ, ಅವರನ್ನು ಸೃಷ್ಟಿಸುವಾಗ ನೀನು ಹೊಂದಿದ್ದ ಕಾಮಸ್ವಭಾವವನ್ನು ಹೊರ ಹಾಕು, ಎಲ್ಲ ಸರಿಹೋಗುತ್ತದೆ’ ಎಂದನು. ಬ್ರಹ್ಮನು ಹಾಗೆ ಮಾಡುತ್ತಲೆ ಆ ಸ್ವಭಾವವು ಕಾಮೋದ್ರೇಕವನ್ನುಂಟುಮಾಡುವ ಸಾಯಂ ಸಂಧ್ಯೆಯಾಗಿ ಪರಿಣಮಿಸಿ, ದಿವ್ಯ ಸುಂದರಿಯ ರೂಪವನ್ನು ಪಡೆಯಿತು.

ದಿವ್ಯ ಸುಂದರಿಯಾದ ಸಂಧ್ಯಾದೇವಿಯು ತನ್ನ ಕಾಲ್ಗೆಜ್ಜೆಗಳು ಘಲಿಘಲಿರೆನ್ನುವಂತೆ ವಯ್ಯಾರದಿಂದ ನಡೆಯುತ್ತಾ, ಕಿರುಗೆಜ್ಜೆಯ ಒಡ್ಯಾಣದಿಂದ ಅಲಂಕೃತವಾದ ತನ್ನ ಚೀನಾಂಬರದ ನೆರಿಗೆಗಳನ್ನು ಚಿಮ್ಮಿಸಿದಳು. ಅಂದವಾದ ಅವಳ ದಂತಪಂಕ್ತಿಯನ್ನು ನಸುದೋರುವ ಅವಳ ಆ ಮುಗುಳ್ನಗೆ, ತುಂಬಿದ ಎದೆಯನ್ನು ಮುಚ್ಚಿರುವ ಸೀರೆಯ ಸೆರಗನ್ನು ಮೋಹಕವಾಗಿ ಮುಂದಕ್ಕೆಳೆದುಕೊಳ್ಳುವ ಆ ವಿಲಾಸ, ಹಣೆಯಮೇಲಿನ ಆ ಕಪ್ಪಾದ ಮುಂಗುರುಳು, ಆ ಎಸಳಾದ ಮೂಗು, ಆ ಮಾದಕವಾದ ಕಣ್ಣುಗಳು–ಕಾಮೋ ದ್ರೇಕದಿಂದಿದ್ದ ಅಸುರರು ಆಕೆಯನ್ನು ಕಂಡು ಹುಚ್ಚರಾದರು. ‘ಅಹೋ ರೂಪಂ! ಅಹೋ ಧೈರ್ಯಂ! ಅಹೋ ಅಸ್ಯಾ ನವಂ ವಯಃ’–ಏನು ರೂಪು, ಏನು ಧೈರ್ಯ, ಏನು ನವ ಯೌವನ–ಎಂದು ಅವರು ಬೆಕ್ಕಸಬೆರಗಾದರು. ಅವಳಲ್ಲಿ ಅಪಾರವಾದ ಅನುರಾಗವುಳ್ಳ ತಮ್ಮ ಕಡೆ ಕಣ್ಣೆತ್ತಿಯೂ ನೋಡದೆ, ಕಾಮವಿಕಾರವೇ ಇಲ್ಲದವಳಂತೆ ತನ್ನ ಹಾದಿಯನ್ನು ಹಿಡಿದು ಹೋಗುತ್ತಿದ್ದ ಅವಳನ್ನು ಆ ಅಸುರರು ಅತ್ಯಂತ ಮೃದು ಮಧುರವಾಗಿ ‘ಹೇ ಮನೋಹರೆ, ನೀನಾರು? ಆರ ಮಗಳು? ಯಾವ ಕಾರ್ಯಕ್ಕಾಗಿ ಇಲ್ಲಿಗೆ ಬಂದೆ? ನಿನ್ನ ಅಮೂಲ್ಯವಾದ ಸೌಂದರ್ಯರತ್ನವನ್ನು ಮಾರುವುದಿಲ್ಲವೇನು? ಅದನ್ನು ಕೊಳ್ಳುವಷ್ಟು ಭಾಗ್ಯ ನಮಗಿಲ್ಲವೋ ಹೇಗೆ? ಅದನ್ನು ಕೊಳ್ಳಬೇಕೆಂದಿರುವ ನಮಗೆ ಅದನ್ನು ಮಾರದೆ ಏಕೆ ಸಂಕಟಪಡಿಸುವಿ? ನೀನು ಮಾನವಳೊ ದೇವತೆಯೊ, ಯಾರಾದರೂ ಆಗಿರು; ನಿನ್ನ ದರ್ಶನವಾದುದೇ ನಮ್ಮ ಭಾಗ್ಯ. ನಿನ್ನನ್ನು ಕಂಡ ನಮ್ಮಚೇತನ ಆಟಗುಳಿಯ ಕೈಗೆ ಸಿಕ್ಕ ಪುಟ್ಟ ಚೆಂಡಿನಂತಾಗಿದೆ. ಚಪಲವಾದ ನಿನ್ನ ನಡಿಗೆ, ತೋರಮೊಲೆಗಳ ಭಾರದಿಂದ ನಡು ಗುವ ಬಡನಡು, ಕೊಂಕಾದ ಕಡೆಗಣ್ ನೋಟ, ಗುಂಗುರು ಗುಂಗುರಾದ ಮುಂಗುರುಳು –ಇವು ನಮ್ಮ ಹೃದಯವನ್ನು ಸೂರೆಗೊಂಡಿವೆ’ ಎಂದು ಹೇಳಿ ಆಕೆಯನ್ನು ಹಿಡಿದು ಕೊಂಡರು.

ಅಸರರು ಸಂಧ್ಯೆಯ ಬೆನ್ನು ಹತ್ತಿದುದನ್ನು ಕಂಡು ಬ್ರಹ್ಮನಿಗೆ ಬಿಡುಗಡೆಯ ಸಮಾ ಧಾನವಾಯಿತು. ಆತನು ಸಂಧ್ಯಾಸೃಷ್ಟಿಗೆ ಕಾರಣವಾದ ತನ್ನ ದೇಹಭಾಗವನ್ನು ಮೂಸಿ ನೋಡುತ್ತಾ, ಸಂಧ್ಯೆಯ ಸೌಂದರ್ಯಸಾರದಿಂದ ವಿಶ್ವಾವಸುವೇ ಮೊದಲಾದ ಗಂಧರ್ವ ರನ್ನೂ ಅಪ್ಸರರನ್ನೂ ಸೃಷ್ಟಿಸಿದನು. ಅನಂತರ ಬ್ರಹ್ಮನು ತನ್ನ ಆಲಸ್ಯ ಸ್ವಭಾವವನ್ನು ಅವಲಂಬಿಸಿ, ಅದರಿಂದ ಭೂತಗಳನ್ನೂ ಪಿಶಾಚಗಳನ್ನೂ ಸೃಷ್ಟಿಸಿದನು. ಅವು ತಲೆಯನ್ನು ಕೆದರಿಕೊಂಡು ಬೆತ್ತಲೆಯಾಗಿ ನಿಂತಿರಲು ಆತನು ಅಸಹ್ಯಗೊಂಡು ಕಣ್ಣುಮುಚ್ಚಿದನು. ಅವು ಆತನ ಆಲಸ್ಯರೂಪವಾದ ಆಕಳಿಕೆಯನ್ನು ಶರೀರವನ್ನಾಗಿ ಮಾಡಿಕೊಂಡವು. ಆದ್ದ ರಿಂದಲೆ ಅವುಗಳಲ್ಲಿ ನಿದ್ರೆ, ಉನ್ಮಾದ, ಆಲಸ್ಯಗಳು ಹೆಚ್ಚು. ಬ್ರಹ್ಮನು ತನ್ನ ಸೃಷ್ಟಿಕಾರ್ಯ ವನ್ನು ಮುಂದುವರಿಸಿ, ಸಾಧ್ಯರನ್ನೂ, ಪಿತೃದೇವತೆಗಳನ್ನೂ ಅದೃಶ್ಯರೂಪದಿಂದ ನಿರ್ಮಿಸಿ ದನು. ರಾಜ್ಯವನ್ನು ಬಯಸುವವರು ಸಿದ್ಧರನ್ನೂ, ಪುತ್ರರನ್ನು ಬಯಸುವವರು ಪಿತೃ ಗಳನ್ನೂ ಆರಾಧಿಸಬೇಕು. ಬ್ರಹ್ಮದೇವನು ತನ್ನ ಅದೃಶ್ಯ ಶಕ್ತಿಯನ್ನು ಅವಲಂಬಿಸಿ ಸಿದ್ಧ ವಿದ್ಯಾಧರರನ್ನೂ, ಪ್ರತಿಬಿಂಬವನ್ನು ಅವಲಂಬಿಸಿ ಕಿನ್ನರ ಕಿಂಪುರುಷರನ್ನೂ ಸೃಷ್ಟಿಸಿದನು. ಸಿದ್ಧವಿದ್ಯಾಧರರು ಅದೃಶ್ಯಶಕ್ತಿಯುಳ್ಳವರು. ಕಿನ್ನರ ಕಿಂಪುರುಷರು ಬಿಂಬ ಪ್ರತಿಬಿಂಬ ಗಳಂತೆ ಸದಾ ಜೊತೆ ಜೊತೆಯಾಗಿದ್ದು ತಮ್ಮ ಸೃಷ್ಟಿಕರ್ತನನ್ನು ಪ್ರತಿದಿನ ಪ್ರಾತಃಕಾಲ ದಲ್ಲಿ ಗಾಯನದಿಂದ ಸ್ತುತಿಸುತ್ತಾರೆ. ಬ್ರಹ್ಮನು ತನ್ನ ಸೃಷ್ಟಿಕಾರ್ಯ ಇನ್ನೂ ಸಮರ್ಪಕವಾಗ ಲಿಲ್ಲವೆಂಬ ಕ್ರೋಧದಿಂದಿದ್ದಾಗ ಆತನ ಶರೀರದಿಂದ ಉದುರಿದ ರೋಮಗಳು ಸರ್ಪ ಗಳಾದುವು. ಅವುಗಳನ್ನೇ ನಾಗಗಳೆಂದೂ ಭೋಗಿಗಳೆಂದೂ ಕರೆಯುತ್ತಾರೆ. ಕೋಪದಲ್ಲಿ ಹುಟ್ಟಿದವುಗಳಾದ್ದರಿಂದ ಅವುಗಳ ಸ್ವಭಾವ ಕ್ರೂರ.

ಇಷ್ಟಾದ ಮೇಲೂ ತಾನು ಮಾಡಿದ ಕಾರ್ಯ ಅಪೂರ್ಣವೆನಿಸಿತು, ಬ್ರಹ್ಮನಿಗೆ. ಆತನು ಪಶ್ಚಾತ್ತಾಪದಿಂದ ಮನಶ್ಶುದ್ಧಿಯನ್ನು ಪಡೆದು ಸನಕ, ಸನಂದನ, ಸನಾತನ, ಸನತ್ಕುಮಾರ ಎಂಬ ನಾಲ್ವರು ಮಾನಸಪುತ್ರರನ್ನು ನಿರ್ಮಿಸಿದನು. ಅವರು ಜಿತೇಂದ್ರಿಯರು, ಪರಮ ವಿರಕ್ತರು, ಆತ್ಮಧ್ಯಾನ ತತ್ಪರರು. ಅವರನ್ನು ಕುರಿತು ಬ್ರಹ್ಮದೇವನು ‘ಅಯ್ಯಾ, ಲೋಕದ ಬೆಳವಣಿಗೆಗಾಗಿ ನಿಮ್ಮನ್ನು ಸೃಷ್ಟಿಸಿದ್ದೇನೆ. ನೀವು ಸಂತಾನವನ್ನು ಪಡೆದು ಆ ಕಾರ್ಯವನ್ನು ಮುಂದುವರಿಸಿ’ ಎಂದನು. ಆದರೆ ಅವರು ಆತನ ಮಾತನ್ನು ತಿರಸ್ಕರಿಸಿ, ಧ್ಯಾನಮಗ್ನ ರಾದರು. ಇದರಿಂದ ಬ್ರಹ್ಮನಿಗೆ ತಡೆಯಲಾರದಷ್ಟು ಕೋಪ ಬಂದಿತು. ಅದನ್ನು ಅಡಗಿ ಸಲು ಆತನು ಎಷ್ಟು ಪ್ರಯತ್ನಿಸಿದರೂ ಸಾಧ್ಯವಾಗದೆ ಅದು ಆತನ ಹುಬ್ಬುಗಳ ಮಧ್ಯೆ ಶಿಶುರೂಪದಿಂದ ಜನಿಸಿ, ಗಟ್ಟಿಯಾಗಿ ಅಳುತ್ತಾ ‘ಹೇ ಜಗದ್ಗುರು, ನಾನು ಎಲ್ಲಿರಲಿ? ನನ್ನ ಹೆಸರೇನು?’ ಎಂದು ಕೇಳಿತು. ಆಗ ಬ್ರಹ್ಮನು ‘ಮಗು, ನೀನು ಮಗುವಿನಂತೆ ರೋದಿಸಿದು ದರಿಂದ ನಿನ್ನ ಹೆಸರು ರುದ್ರನೆಂದು ಪ್ರಸಿದ್ಧಿಯಾಗಲಿ. ನೀನು ಮನ್ಯು, ಮಹಾಕಾಲ, ಮಹಾಂತ, ಪುತಧ್ವಜ, ಶಿವ, ಉಗ್ರತೀತ, ಭವ, ಕಾಲ, ವಾಮದೇವ, ಧೃತವ್ರತ ಎಂಬ ಹನ್ನೊಂದು ಹೆಸರುಗಳಿಂದ ಹೃದಯ, ಇಂದ್ರಿಯ, ಪ್ರಾಣ, ಆಕಾಶ, ವಾಯು, ಅಗ್ನಿ, ಜಲ, ಭೂಮಿ, ಸೂರ್ಯ, ಚಂದ್ರ, ತಪಸ್ಸುಗಳೆಂಬ ಹನ್ನೊಂದು ಸ್ಥಾನಗಳಲ್ಲಿ ನೆಲಸು. ಧೀ, ವೃತ್ತಿ, ಅಶನಾ, ಉಮಾ, ನಿಯತ್, ಸರ್ಪಿ, ಇಳಾ, ಅಂಬಿಕಾ, ಇರಾವತೀ, ಸುಧಾ, ದೀಕ್ಷ–ಎಂಬ ಹನ್ನೊಂದು ಮಂದಿ ನಿನ್ನ ಪತ್ನಿಯರಾಗುವರು. ನೀನು ಸಕಲ ಪ್ರಜೆ ಗಳಿಗೂ ಒಡೆಯನಾಗಿದ್ದುಕೊಂಡು ಪ್ರಜಾಸೃಷ್ಟಿಯ ಕಾರ್ಯದಲ್ಲಿ ತತ್ಪರನಾಗು’ ಎಂದು ಹೇಳಿದನು.

ಬ್ರಹ್ಮನ ಆಜ್ಞೆಯಂತೆ ರುದ್ರನು ಬಲ, ಆಕಾರ, ಸ್ವಭಾವಗಳಲ್ಲಿ ತನ್ನಂತೆಯೇ ಇರುವ ಅನೇಕ ರುದ್ರಗಣಗಳನ್ನು ಸೃಷ್ಟಿಸಿದನು. ಅವರು ಗುಂಪುಗುಂಪಾಗಿ ಹೊರಟು ಜಗತ್ತ ನ್ನೆಲ್ಲಾ ನಾಶಮಾಡಲು ಪ್ರಾರಂಭಿಸಿದರು. ಆಗ ಬ್ರಹ್ಮನು ‘ಅಯ್ಯಾ ರುದ್ರದೇವ, ಸರ್ವ ನಾಶಮಾಡುತ್ತಿರುವ ಇಂತಹ ರುದ್ರರನ್ನು ಇನ್ನು ಸೃಷ್ಟಿಸಬೇಡ. ನೀನು ತಪಸ್ಸನ್ನಾಚರಿಸಿ, ಸಕಲ ಭೂತಗಳಿಗೂ ಸುಖಕರವಾಗುವಂತಹ ಸೃಷ್ಟಿಯಲ್ಲಿ ಉದ್ಯೋಗಿಸು’ ಎಂದನು. ಅಂತೆಯೇ ಆಗಲೆಂದು ಆತನು ತಪೋಮಗ್ನನಾಗಿರಲು, ಇತ್ತ ಬ್ರಹ್ಮನ ತೊಡೆಯಿಂದ ನಾರದನೂ, ಉಂಗುಷ್ಠದಿಂದ ದಕ್ಷಬ್ರಹ್ಮನೂ, ಪ್ರಾಣದಿಂದ ವಸಿಷ್ಠನೂ, ಚರ್ಮದಿಂದ ಭೃಗುವೂ, ಕೈಯಿಂದ ಕ್ರತುವೂ, ಹೊಕ್ಕುಳಿನಿಂದ ಪುಲಹನೂ, ಕಿವಿಗಳಿಂದ ಪುಲಸ್ತ್ಯನೂ, ಮುಖದಿಂದ ಅಂಗಿರಸ್ಸೂ, ಕಣ್ಣುಗಳಿಂದ ಅತ್ರಿಯೂ, ಮನಸ್ಸಿನಿಂದ ಮರೀಚಿಯೂ ಮೂಡಿಬಂದರು. ಅನಂತರ ಆತನ ಎದೆಯ ಬಲಭಾಗದಿಂದ ಧರ್ಮವೂ ಎಡ ಭಾಗ ದಿಂದ ಮೃತ್ಯುವೂ ಹುಟ್ಟಿದರು. ಆತನ ಹೃದಯದಿಂದ ಕಾಮ ಜನಿಸಿತು; ಹುಬ್ಬಿನಿಂದ ಕ್ರೋಧ ಅವತರಿಸಿತು; ಬಾಯಿಂದ ಸರಸ್ವತಿ ಹೊರಬಂದಳು. ಹಾಗೆಯೇ ಆತನ ಶರೀರದ ಇತರ ಭಾಗಗಳಿಂದ ಸಮುದ್ರ, ಕಲಿ ಸೃಷ್ಟಿಯಾದವು. ಆತನ ನೆರಳಿನಿಂದ ಮಹಾಮಹಿಮ ನಾದ ಕರ್ದಮ ಪ್ರಜಾಪತಿ ಉದಿಸಿ ಬಂದನು. ಹೀಗೆ ಬ್ರಹ್ಮನು ತನ್ನ ದೇಹ ಮನಸ್ಸು ಗಳಿಂದ ಜಗತ್ತನ್ನೆಲ್ಲ ಸೃಷ್ಟಿಸಿದನು.

ಸೃಷ್ಟಿಕಾರ್ಯವನ್ನು ಕೈಕೊಂಡ ಬ್ರಹ್ಮನು ತೇಜಸ್ವಿಗಳಾದ ಪುಷಿಗಳನ್ನು ಸೃಷ್ಟಿಸಿದ ನಾದರೂ ಪ್ರಜಾಭಿವೃದ್ಧಿಕಾರ್ಯ ಮುಂದುವರಿಯಲಿಲ್ಲ. ಅದು ತಾನಾಗಿ ತಾನೇ ಮುಂದು ವರಿಯುವ ಉಪಾಯವೇನೆಂದು ಆತನು ಧ್ಯಾನಾಸಕ್ತನಾಗಿ ಯೋಚಿಸುತ್ತಿರಲು, ಆತನ ಶರೀರದಿಂದ ಅದರಂತೆಯೆ ಇರುವ ಮತ್ತೊಂದು ಶರೀರವು ಹೊರಬಂದಿತು. ಹೀಗೆ ಬಂದ ಶರೀರವು ಎರಡಾಗಿ ಸ್ತ್ರೀಪುರುಷರೂಪದ ಮಿಥುನವಾಯಿತು. ಪುರುಷರೂಪು ಸ್ವಾಯಂಭು ಮನುವೆನಿಸಿತು; ಸ್ತ್ರೀರೂಪು ಶತರೂಪೆಯೆಂಬ ಹೆಸರಿನಿಂದ ಆತನ ಮಡದಿ ಯಾದಳು. ಈ ದಂಪತಿಗಳೇ ಜಗತ್ತಿನ ಪ್ರಜಾಭಿವೃದ್ಧಿಗೆ ಮೂಲಕಾರಣರಾದರು.

ಹೀಗೆ ಸೃಷ್ಟಿಕಾರ್ಯವು ತನ್ನಿಂದ ಸಮರ್ಪಕವಾಗಿ ನೆರೆವೇರಿದುದನ್ನು ಕಂಡು ಬ್ರಹ್ಮನಿಗೆ ತೃಪ್ತಿಯಾಯಿತು. ಈ ಸಂತೋಷಭಾವದಲ್ಲಿರುವಾಗ ಆತನು ಪ್ರಜಾಸೃಷ್ಟಿಕಾರ್ಯಕ್ಕೆ ಸಮರ್ಥರಾದ ಮನುಗಳನ್ನು ಸೃಷ್ಟಿಸಿ ಅವರಿಗೆ ತನ್ನಂತೆ ಪುರುಷಾಕಾರದ ದೇಹವನ್ನು ಕೊಟ್ಟನು. ಅವರು ತಮ್ಮ ಸೃಷ್ಟಿಕರ್ತನನ್ನು ಸ್ತುತಿಸುತ್ತಾ ‘ಸ್ವಾಮಿ, ನಿನ್ನ ಈ ಪುರುಷಾ ಕಾರದ ಸೃಷ್ಟಿ ಅತ್ಯಂತ ಶ್ರೇಷ್ಠವಾದುದು. ಯಜ್ಞಾದಿ ಕರ್ಮಗಳೆಲ್ಲವೂ ಈ ಪುರುಷ ದೇಹದಿಂದಲೇ ಆಚರಿಸಬೇಕಾಗಿದೆ’ ಎಂದು ಹೇಳಿದರು. ಅವರ ಮಾತುಗಳಿಂದ ಸುಪ್ರೀತ ನಾದ ಚತುರ್ಮುಖನು ತಪಸ್ಸು, ವಿದ್ಯೆ, ಯೋಗ, ಸಮಾಧಿಗಳಿಂದ ಜಿತೇಂದ್ರಿಯನಾಗಿ ಪುಷಿಗಳನ್ನು ಸೃಷ್ಟಿಸಿದನು. ಅವರಲ್ಲಿ ಒಬ್ಬೊಬ್ಬರಿಗೂ ಆತನು ತನ್ನ ಸ್ವಭಾವದಲ್ಲಿ ಒಂದೊಂದು ಅಂಶವಾದ ಸಮಾಧಿ, ಯೋಗ, ಅಣಿಮಾದಿ ಅಷ್ಟಸಿದ್ಧಿಗಳು, ತಪಸ್ಸು, ಜ್ಞಾನ, ವೈರಾಗ್ಯಗಳನ್ನು ಕೊಟ್ಟನು.

