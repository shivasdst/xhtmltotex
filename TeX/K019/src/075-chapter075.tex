
\chapter{೭೫. ಕೆಂಬೂತಿಯ ಪುಕ್ಕಗಳೆಲ್ಲ ಉದುರಿದವು}

ಬಲರಾಮನು ನಂದಗೋಕುಲಕ್ಕೆ ಹೋಗಿದ್ದಾಗ ದ್ವಾರಕಾಪುರದಲ್ಲಿ ಒಂದು ತಮಾಷೆ ನಡೆಯಿತು. ಕರೂಷದೇಶದ ಪೌಂಡ್ರಕನೆಂಬ ರಾಜ ತನ್ನ ದೂತನೊಡನೆ ಶ್ರೀಕೃಷ್ಣನಿಗೆ ಒಂದು ನಿರೂಪವನ್ನು ಕಳುಹಿಸಿದ. ‘ಎಲಾ ಕೃಷ್ಣ, ನಾನು ಲೋಕರಕ್ಷಣೆಗಾಗಿ ಅವತರಿ ಸಿರುವ ನಿಜವಾದ ವಾಸುದೇವ. ನೀನು ಆ ಹೆಸರನ್ನಿಟ್ಟುಕೊಂಡು, ಅಟ್ಟಹಾಸದಿಂದ ಮೆರೆಯುತ್ತಿರುವೆಯೆಂದು ಕೇಳಿ ನನಗೆ ಕೋಪ ಬಂದಿದೆ. ನೀನು ನಾಟಕದಲ್ಲಿ ವೇಷ ಹಾಕು ವವನಂತೆ ಶಂಖ, ಚಕ್ರ, ಗದೆ, ಪದ್ಮ ಮೊದಲಾದ ವಾಸುದೇವನ ಚಿಹ್ನೆಗಳನ್ನೆಲ್ಲ ಧರಿಸಿರು ವಿಯಂತೆ! ನೀನು ಈ ಹುಡುಗಾಟವನ್ನೆಲ್ಲ ಕಟ್ಟಿಟ್ಟು, ತಕ್ಷಣವೆ ನನಗೆ ಶರಣಾಗತನಾಗ ಬೇಕು. ಇಲ್ಲದಿದ್ದರೆ ನಾನೇ ಬಂದು ನಿನ್ನ ಮಗ್ಗುಲು ಮುರಿಯಬೇಕಾದೀತು, ಎಚ್ಚರ’ ಎಂದು ಹೇಳಿಕಳುಹಿಸಿದ. ಆ ದೂತನು ದ್ವಾರಕಿಗೆ ಬಂದು, ತುಂಬಿದ ರಾಜಸಭೆಯಲ್ಲಿ ಕುಳಿತಿದ್ದ ಶ್ರೀಕೃಷ್ಣನಿಗೆ ಆ ಸಂದೇಶವನ್ನು ತಿಳಿಸಿದ. ಅದನ್ನು ಕೇಳಿ ಉಗ್ರಸೇನಮಹಾ ರಾಜನೆ ಮೊದಲಾದ ಸಭೆಯಲ್ಲಿದವರೆಲ್ಲ ಘೊಳ್ಳೆಂದು ನಕ್ಕರು. ಶ್ರೀಕೃಷ್ಣನೂ ಮುಗು ಳ್ನಗುತ್ತಾ ಆ ದೂತನ ಮೂಲಕವೆ ಪೌಂಡ್ರಕನಿಗೆ ಉತ್ತರ ಕಳುಹಿಸಿದ–‘ಎಲ ಅವಿವೇಕಿ, ಅಹಂಕಾರದಿಂದ ಹುಚ್ಚುಹುಚ್ಚಾಗಿ ಹರಟುವ ನಿನ್ನ ಬಾಯಿಯನ್ನು ಮುಚ್ಚಿಸುವುದಕ್ಕಾಗಿ ನಾನೆ ನಿನ್ನ ಬಳಿಗೆ ಬರುತ್ತೇನೆ; ಕೊಬ್ಬಿದ ನಿನ್ನ ದೇಹವನ್ನು ತುಂಡುತುಂಡಾಗಿ ಕತ್ತರಿಸಿ ನರಿ ನಾಯಿಗಳಿಗೆ ಆಹಾರವಾಗಿ ಮಾಡುತ್ತೇನೆ’ ಎಂದು. ಆ ದೂತ ಈ ಮಾತುಗಳನ್ನು ಯಥಾ ವತ್ತಾಗಿ ತನ್ನ ರಾಜನಿಗೆ ತಿಳಿಸಿದ.

ಮರುದಿನವೆ ಶ್ರೀಕೃಷ್ಣನು ತನ್ನ ಯಾದವ ಸೇನೆಯೊಡನೆ ಕರೂಷ ದೇಶದ ಮೇಲೆ ದಂಡ ಯಾತ್ರೆ ಹೊರಟನು. ಆತನು ಕಾಶೀ ಪಟ್ಟಣದ ಬಳಿಗೆ ಹೋಗುವಷ್ಟರಲ್ಲಿ ಪೌಂಡ್ರಕನು ಆ ಸುದ್ದಿಯನ್ನು ಕೇಳಿ ತನ್ನ ಸೇನೆಯೊಡನೆ ಅಲ್ಲಿಗೆ ಬಂದನು. ಕಾಶೀರಾಜನು ಪೌಂಡ್ರಕನ ಗೆಳೆಯನಾದುದರಿಂದ ಅವನೂ ತನ್ನ ಮೂರು ಅಕ್ಷೋಹಿಣಿ ಸೇನೆಯೊಡನೆ ಬಂದು ಅವ ನನ್ನು ಸೇರಿಕೊಂಡನು. ಈ ಮಹಾಸೇನೆ ಯಾದವ ಸೇನೆಯೊಡನೆ ಯುದ್ಧವನ್ನು ಪ್ರಾರಂಭಿ ಸಿತು. ಶ್ರೀಕೃಷ್ಣನು ಶತ್ರುಸೇನೆಯನ್ನು ಭೇದಿಸಿಕೊಂಡು ಪೌಂಡ್ರಕನ ಸಮೀಪಕ್ಕೆ ಹೋದನು. ಅವನನ್ನು ಕಾಣುತ್ತಲೆ ಆತನಿಗೆ ನಗುಬಂತು. ಆ ಪೌಂಡ್ರಕ ತನ್ನೆರಡು ಕೈಗಳ ಜೊತೆಗೆ ಮತ್ತೆರಡು ಕೈಗಳನ್ನು ಜೋಡಿಸಿಕೊಂಡಿದ್ದನು. ಆ ಕೈಗಳಲ್ಲಿ ಕ್ರಮವಾಗಿ ಶಂಖು, ಚಕ್ರ, ಗದೆ, ಪದ್ಮ; ಎದೆಯಲ್ಲಿ ಶ್ರೀವತ್ಸವೆಂಬ ಮಚ್ಚೆಯನ್ನು ಬರೆದುಕೊಂಡಿದ್ದಾನೆ, ಎದೆ ಯಲ್ಲಿ ಕೌಸ್ತುಭದಂತಹ ಮಣಿಯನ್ನು ಧರಿಸಿದ್ದಾನೆ, ಕೊರಳಲ್ಲಿ ಹಾರ, ಮೈಮೇಲೆ ರೇಷ್ಮೆ ಮಕುಟಗಳು, ರಥದ ಮೇಲೆ ಗರುಡನ ಚಿತ್ರವನ್ನು ಬರೆದ ಬಾವುಟ–ವಾಸುದೇವನ ವೇಷವನ್ನು ಅಚ್ಚಳಿಯದಂತೆ ಧರಿಸಿದ್ದಾನೆ. ಮುಖ ಮಾತ್ರ ವಾಸುದೇವನದಲ್ಲ, ಪೌಂಡ್ರಕ ನದೆ. ಶ್ರೀಕೃಷ್ಣನು ನಗುವನ್ನು ತಡೆಯಲಾರದೆ ಕೇಕೆ ಹಾಕಿಕೊಂಡು ನಕ್ಕ. ಅದನ್ನು ಕಂಡು ನಕಲಿಯ ವಾಸುದೇವನಿಗೆ ರೇಗಿಹೋಯಿತು. ತಾನು ಶಾರ್ಙ್ಗವೆಂದು ಕರೆಯುತ್ತಿದ್ದ ತನ್ನ ಬಿಲ್ಲಿನಿಂದ ಬಾಣಗಳನ್ನು ಮಳೆಗರೆದ. ಆದರೇನು? ಎಷ್ಟು ಪುಕ್ಕಗಳನ್ನು ಸಿಕ್ಕಿಸಿಕೊಂಡರೂ ಕೆಂಬೂತಿ ನವಿಲಾಗಬಲ್ಲದೆ? ಶ್ರೀಕೃಷ್ಣನ ಶಾರ್ಙ್ಗದಿಂದ ಹೊರಟ ಬಾಣಗಳು ಶತ್ರುವಿನ ಬಾಣಗಳನ್ನೆಲ್ಲ ನುಂಗಿ ನೀರು ಕುಡಿದವು. ಅಷ್ಟೇ ಅಲ್ಲ, ಶತ್ರು ಸೇನೆಯನ್ನೆಲ್ಲ ಕೊಚ್ಚಿಹಾಕಿ ದವು. ರಣಭೂಮಿ ರುದ್ರಭೂಮಿಯಾಯಿತು. ಶ್ರೀಕೃಷ್ಣನ ಬಾಣವೇಗಕ್ಕೆ ಪೌಂಡ್ರಕ ತತ್ತರಿಸಿಹೋದನು. ಅಷ್ಟರಲ್ಲಿ ಶ್ರೀಕೃಷ್ಣನ ಬಾಣಗಳು ಪೌಂಡ್ರಕ ಅಳವಡಿಸಿಕೊಂಡಿದ್ದ ಹೊಸ ಕೈಗಳೆರಡನ್ನೂ ಕತ್ತರಿಸಿ ಹಾಕಿ, ಅವನ ವಾಸುದೇವನ ವೇಷವನ್ನೆಲ್ಲ ಒಂದೊಂದಾಗಿ ಕೆಳಕ್ಕೆ ಬೀಳಿಸಿದವು. ಕೆಂಬೂತಿಯ ಪುಕ್ಕಗಳೆಲ್ಲ ಉದುರಿಹೋದವು. ಇದನ್ನು ಕಂಡು ಕಾಶೀ ರಾಜನು ತನ್ನ ಗೆಳೆಯನ ಸಹಾಯಕ್ಕೆಂದು ಅಲ್ಲಿಗೆ ನುಗ್ಗಿಬಂದನು. ಇದೇ ಸಮಯವೆಂದು ಕೊಂಡ ಶ್ರೀಕೃಷ್ಣನು ತನ್ನ ಚಕ್ರವನ್ನು ಪ್ರಯೋಗಿಸಿ, ಇಬ್ಬರ ತಲೆಗಳನ್ನೂ ಏಕಕಾಲ ದಲ್ಲಿಯೆ ಕತ್ತರಿಸಿಹಾಕಿದನು. ಪೌಂಡ್ರಕನ ತಲೆ ರಣಭೂಮಿಯಲ್ಲಿ ಬಿತ್ತು; ಕಾಶಿರಾಜನ ತಲೆ ಹಾರಿಹೋಗಿ, ಅವನ ರಾಜಧಾನಿಯಲ್ಲಿದ್ದ ಅರಮನೆಯ ಮುಂದೆ ಬಿತ್ತು. ಅವರ ಸೇನೆಗಳಲ್ಲಿ ಅಳಿದುಳಿದವರು ಜೀವಭಯದಿಂದ ದಿಕ್ಕಾಪಾಲಾಗಿ ಓಡಿಹೋದರು. ವಿಜಯಿ ಯಾದ ಶ್ರೀಕೃಷ್ಣನು ಸಂಭ್ರಮದಿಂದ ಊರಿಗೆ ಹಿಂದಿರುಗಿದನು.

ಅತ್ತ, ಅರಮನೆಯ ಮುಂದೆ ಬಿದ್ದಿದ್ದ ಕಾಶೀರಾಜನ ತಲೆಯನ್ನು ಕಂಡು ಅವನ ಹೆಂಡಿರು ಮಕ್ಕಳೆಲ್ಲ ಗಳಗಳ ಅತ್ತರು. ಆ ರಾಜನ ಮಗನಾದ ಸುದಕ್ಷಿಣನೆಂಬುವನು ತನ್ನ ತಂದೆಯನ್ನು ಕೊಂದ ಶ್ರೀಕೃಷ್ಣನ ಮೇಲೆ ಸೇಡು ತೀರಿಸಿಕೊಳ್ಳಬೇಕೆಂದು, ರುದ್ರನನ್ನು ಕುರಿತು ತಪಸ್ಸು ಮಾಡಿದನು. ಆತನ ಭಕ್ತಿಗೆ ಮೆಚ್ಚಿದ ರುದ್ರದೇವನು ಪ್ರತ್ಯಕ್ಷನಾಗಿ ‘ಅಯ್ಯಾ ಸುದಕ್ಷಿಣ, ಶ್ರೀಕೃಷ್ಣನನ್ನು ಕೊಲ್ಲುವುದಕ್ಕೆ ಯಾವ ಆಯುಧಕ್ಕೂ ಸಾಧ್ಯವಿಲ್ಲ. ನೀನು ದಕ್ಷಿಣಾಗ್ನಿಯಲ್ಲಿ ಮಾರಣಹೋಮ ಮಾಡಿ, ಅದರಿಂದ ಹುಟ್ಟುವ ಭೂತದಿಂದ ಅವನನ್ನು ಕೊಲ್ಲಿಸಬೇಕು. ಆದರೆ ಆ ಕೆಲಸವನ್ನು ಮಾಡುವಾಗ ತುಂಬ ಎಚ್ಚರಿಕೆಯಿರಲಿ. ಇದನ್ನು ನಾಸ್ತಿಕನ ಮೇಲೆ ಮಾತ್ರ ಪ್ರಯೋಗಿಸಬೇಕು’ ಎಂದು ಹೇಳಿ ಮಾಯವಾದನು. ಸುದಕ್ಷಿಣನು ರುದ್ರನು ಹೇಳಿದಂತೆ ಮಂತ್ರವಿದರಾದ ಬ್ರಾಹ್ಮಣರ ಸಹಾಯದಿಂದ ಮಾರಣಹೋಮವನ್ನು ಮಾಡಿಸಿದನು. ಒಡನೆಯೆ ಅಗ್ನಿಯಿಂದ ಕರಾಳರೂಪದ ಒಂದು ಭೂತ ಹೊರಕ್ಕೆದ್ದು ಬಂದಿತು. ತಾಮ್ರದ ತಂತಿಯಂತಿರುವ ತಲೆಗೂದಲು, ಗಡ್ಡಮೀಸೆ ಗಳು; ಕೆಂಡದುಂಡೆಯಂತಿರುವ ಕಣ್ಣುಗಳು; ಗಡಾರಿಯಂತಿರುವ ಹುಬ್ಬುಗಳು, ಕೋರೆ ದಾಡೆ, ಜೋಲುನಾಲಗೆ; ತಾಳೆಯ ಮರದಂತೆ ಎತ್ತರವಾದ ಕಾಲುಗಳು. ಭಯಂಕರಾ ಕಾರದ ಈ ಭೂತ ಕೈಲಿ ತ್ರಿಶೂಲವನ್ನು ಹಿಡಿದು, ನೆಲ ನಡುಗುವಂತೆ ಹೆಜ್ಜೆ ಹಾಕುತ್ತಾ, ದ್ವಾರಕಾ ನಗರಿಯತ್ತ ನಡೆಯಿತು. ಅದರಿಂದ ಹೊರಟ ಬೆಂಕಿ ದಿಕ್ಕುಗಳನ್ನೆಲ್ಲ ಸುಟ್ಟು ಹಾಕುತ್ತಿತ್ತು. ಇದನ್ನು ಕಂಡ ದ್ವಾರಕಾನಗರದ ಜನ ಕಾಡುಕಿಚ್ಚಿಗೆ ಸಿಕ್ಕ ಹುಲ್ಲೆಯಂತೆ ಹೆದರಿ ನಡುಗುತ್ತಾ ‘ಶ್ರೀಕೃಷ್ಣ, ಕಾಪಾಡು, ಕಾಪಾಡು’ ಎಂದು ಕಿರಿಚಿಕೊಂಡು ಅರಮನೆಯ ಹತ್ತಿರಕ್ಕೆ ಓಡಿಬಂದರು. ರುಕ್ಮಿಣಿಯೊಡನೆ ಪಗಡೆಯಾಡುತ್ತಾ ಕುಳಿತಿದ್ದ ಶ್ರೀಕೃಷ್ಣನು ಅವರ ಹುಯಲನ್ನು ಕೇಳಿ ಹೊರಕ್ಕೆ ಬಂದನು. ಆತನು ಅವರನ್ನೆಲ್ಲ ಸಮಾಧಾನ ಮಾಡಿ, ಈ ಭೂತವನ್ನು ಬಡಿದೋಡಿಸುವಂತೆ ತನ್ನ ಚಕ್ರಕ್ಕೆ ಅಪ್ಪಣೆ ಮಾಡಿದನು. ಬೆಂಕಿಗೆ ಬೆಂಕಿ ಯಂತಿರುವ ಆ ಚಕ್ರ ಕೋಟಿ ಸೂರ್ಯರ ಬೆಳಕನ್ನು ಚೆಲ್ಲುತ್ತಾ ಆ ಭೂತವನ್ನು ಅಟ್ಟಿಕೊಂಡು ಬಂದಿತು. ಅದನ್ನು ಕಾಣುತ್ತಲೆ ಆ ಭೂತ ಹೆದರಿ ಕಿರುಚುತ್ತಾ ಹಿಂದಕ್ಕೆ ಓಡಿಬಂದು, ತನ್ನನ್ನು ಮೂಡಿಸಿದ ಆ ಬ್ರಾಹ್ಮಣರನ್ನೂ ಸುದಕ್ಷಿಣನನ್ನೂ ಏಕಕಾಲಕ್ಕೆ ಸುಟ್ಟುಹಾಕಿತು.

ಪೌಂಡ್ರಕನು ವಾಸುದೇವನಾದ ಕಥೆ ಬಹುಕಾಲದವರೆಗೆ ದ್ವಾರಕಿಯ ಜನರಿಗೆ ವಿನೋದದ ವಸ್ತುವಾಗಿತ್ತು. ‘ಕೆಂಬೂತಿಯ ಪುಕ್ಕಗಳನ್ನೆಲ್ಲ ತರಿದ ನಮ್ಮ ಶ್ರೀಕೃಷ್ಣ’ ಎಂದು ಹೆಣ್ಣುಮಕ್ಕಳೆಲ್ಲ ಹಾಡಿಕೊಂಡು ನಗುತ್ತಿದ್ದರು.

