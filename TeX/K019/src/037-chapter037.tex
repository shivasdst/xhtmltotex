
\chapter{೩೭. ಯಯಾತಿ}

ಪುರೂರವನ ಮೊಮ್ಮಗ ನಹುಷ. ಆತ ನೂರು ಅಶ್ವಮೇಧಯಾಗಗಳನ್ನು ಮಾಡಿ ಇಂದ್ರ ಪದವಿಯನ್ನು ಪಡೆದ. ಆದರೆ ಮಹಾಪತಿವ್ರತೆಯಾದ ಶಚಿಯನ್ನು ಮೋಹಿಸಿ ಸಪ್ತ ಪುಷಿಗಳ ಶಾಪದಿಂದ ಹೆಬ್ಬಾವಾದ. ಈ ನಹುಷನಿಗೆ ಆರುಜನ ಮಕ್ಕಳಿದ್ದರು. ಅವರಲ್ಲಿ ಹಿರಿಯನಾದ ಯತಿಯು, ತನ್ನ ಹೆಸರಿಗೆ ತಕ್ಕಂತೆ ವೈರಾಗ್ಯಪರನಾಗಿ ತಪಸ್ಸು ಮಾಡಲು ಅಡವಿಗೆ ಹೊರಟುಹೋದ. ಆದ್ದರಿಂದ ಅವನ ತಮ್ಮನಾದ ಯಯಾತಿ ರಾಜನಾದ. ಈತ ತನ್ನ ನಾಲ್ವರು ತಮ್ಮಂದಿರಿಗೂ ನಾಲ್ಕು ದಿಕ್ಕುಗಳ ರಾಜ್ಯಾಧಿಕಾರವನ್ನು ವಹಿಸಿ, ತಾನು ರಾಜಾಧಿರಾಜನಾಗಿ ಧರ್ಮದಿಂದ ರಾಜ್ಯಭಾರ ಮಾಡುತ್ತಿದ್ದ. ಈತನ ಬಾಳಕಥೆ ಅತ್ಯಂತ ಆಶ್ಚರ್ಯಕರವಾಗಿದೆ; ಮನೋಹರವಾಗಿ, ಮಾರ್ಮಿಕವಾಗಿ, ಧರ್ಮಬೋಧಕವಾಗಿದೆ.

ಕ್ಷತ್ರಿಯನಾದ ಯಯಾತಿ ಮದುವೆಯಾದುದು ಬ್ರಹ್ಮರ್ಷಿಯಾದ ಶುಕ್ರನ ಮಗಳು ದೇವಯಾನಿಯನ್ನು. ಈ ಮದುವೆ ನಡೆದುದು ವಿಲಕ್ಷಣವಾದ ಒಂದು ಸನ್ನಿವೇಶದಲ್ಲಿ. ಶುಕ್ರನು ವೃಷಪರ್ವನೆಂಬ ದಾನವರಾಜನಲ್ಲಿ ಪುರೋಹಿತನಾಗಿದ್ದನು. ಆತನ ಮಗಳಾದ ದೇವಯಾನಿಗೂ ರಾಜಪುತ್ರಿಯಾದ ಶರ್ಮಿಷ್ಠೆಗೂ ತುಂಬ ಗೆಳೆತನ. ಒಮ್ಮೆ ಅವರಿಬ್ಬರೂ ಹಲವು ಗೆಳತಿಯರೊಡನೆ ವನವಿಹಾರಕ್ಕೆ ಹೊರಟು, ಹೂದೋಟವೊಂದರ ಮಧ್ಯದಲ್ಲಿದ್ದ ಸರೋವರದಲ್ಲಿ ನೀರಾಟವಾಡುತ್ತಿದ್ದರು. ಆ ಸಮಯದಲ್ಲಿ ಶಿವ-ಶಿವೆಯರು ನಂದಿಯ ನ್ನೇರಿ ಅತ್ತಕಡೆ ಬರುತ್ತಿರುವುದು ಕಾಣಿಸಿತು. ಕೂಡಲೆ ಹೂಡುಗಿಯರೆಲ್ಲ ದುಡುದುಡು ದಡಕ್ಕೆ ಓಡಿ ಬಂದು, ತಮ್ಮತಮ್ಮ ಬಟ್ಟೆಗಳನ್ನು ಮೈಗೆ ಸುತ್ತಿಕೊಳ್ಳ ಹೊರಟರು. ಆ ಆತುರದಲ್ಲಿ ಶರ್ಮಿಷ್ಠೆ ಮರೆತು ದೇವಯಾನಿಯ ಸೀರೆಯನ್ನು ಸುತ್ತಿಕೊಂಡಳು. ಇದನ್ನು ಕಂಡು ದೇವಯಾನಿ ಮಹಾ ಕೋಪದಿಂದ ‘ಇದೇನೆ ನಿನ್ನ ಕುಚೇಷ್ಟೆ? ನಾಯಿಯು ಯಾಗದ ಹವಿಸ್ಸನ್ನು ಕಚ್ಚಿಕೊಂಡು ಹೋಗುವಂತೆ ಅಯೋಗ್ಯಳಾದ ನೀನು ನನ್ನ ಸೀರೆಯನ್ನು ಉಡುತ್ತಿರುವೆ? ವರ್ಣಗಳಲ್ಲೆಲ್ಲ ಶ್ರೇಷ್ಠವಾದ ಬ್ರಾಹ್ಮಣ ವರ್ಣಕ್ಕೆ ಸೇರಿದವಳು, ನಾನು. ಬ್ರಾಹ್ಮಣರೆಂದರೆ ಏನು ತಿಳಿದುಕೊಂಡಿದ್ದಿ? ಈಶ್ವರನ ಮುಖದಿಂದ ಬಂದವರು! ದೇವತೆ ಗಳು ಕೂಡ ಅವರಿಗೆ ನಮಸ್ಕರಿಸುತ್ತಾರೆ. ನಮ್ಮಪ್ಪ ನಿಮ್ಮಪ್ಪನ ಗುರು. ನೀನು ನನ್ನ ಸೀರೆ ಉಡುತ್ತೀಯಾ?’ ಎಂದಳು. ಇದನ್ನು ಕೇಳಿ ಶರ್ಮಿಷ್ಠೆಗೆ ಕಾಸಿದ ಸೀಸವನ್ನು ಕಿವಿಯಲ್ಲಿ ಹೊಯ್ದಂತಾಯಿತು. ಅಂಗುಷ್ಠದಿಂದ ನಡುನೆತ್ತಿಯವರೆಗೆ ಆಕೆಯ ರೋಷ ಏರಿ ಹೋಯಿತು. ಹೆಡೆ ತುಳಿದ ಹಾವಿನಂತೆ ಆಕೆ ಬುಸುಗುಟ್ಟುತ್ತಾ, ತುಟಿಗಳನ್ನು ಕಚ್ಚಿಕೊಂಡು ‘ಎಲೆ ತಿರಕಿ! ಕಾಗೆಯಂತೆ ನಮ್ಮ ಮನೆಯ ಕೂಳಿಗೆ ಕಾದು ಬಿದ್ದಿರುವ ನೀನು ನಾಲಗೆಯನ್ನು ಉದ್ದಮಾಡಿಕೊಂಡು ನಾಯಿಯಂತೆ ಬೊಗಳುತ್ತಿರುವೆಯಾ?’ ಎಂದು ತಾನು ಉಟ್ಟಿದ್ದ ಸೀರೆಯನ್ನು ಕಿತ್ತೆಸೆದು ಅವಳನ್ನು ಒಂದು ಬಾವಿಗೆ ನೂಕಿ, ಸಖಿಯರೊಡನೆ ಅಲ್ಲಿಂದ ಹೊರಟುಹೋದಳು. ಹೀಗೆ ದೇವಯಾನಿಯು ಬರಿಮೈಯಿಂದ ಹಾಳು ಬಾವಿಯಲ್ಲಿ ಬಿದ್ದಿರುವಾಗ, ಅಕಸ್ಮಾತ್ತಾಗಿ ಯಾಯಾತಿ ಅಲ್ಲಿಗೆ ಬಂದನು. ಆತ ಬೇಟೆಗೆಂದು ಹೊರಟ ವನು, ಮಾರ್ಗಾಯಾಸದಿಂದ ಬಾಯಾರಿಕೆಯಾಗಲು, ನೀರು ಕುಡಿಯಲೆಂದು ಆ ಸರೋ ವರದ ಬಳಿಗೆ ಬಂದನು. ಸಮೀಪದ ಬಾವಿಯಿಂದ ನರಳುತ್ತಿರುವ ದನಿಯನ್ನು ಕೇಳಿ, ಆತ ಅದರಲ್ಲಿ ಬಗ್ಗಿ ನೋಡುತ್ತಾನೆ, ದಿವ್ಯ ಸುಂದರಿಯಾದ ಹುಡುಗಿಯೊಬ್ಬಳು ಬೆತ್ತಲೆಯಾಗಿ ಅಳುತ್ತಾ ನಿಂತಿದ್ದಾಳೆ. ಯಯಾತಿಯು ತಾನು ಹೊದ್ದಿದ್ದ ವಸ್ತ್ರವನ್ನು ಆಕೆಗೆ ಕೊಟ್ಟು, ಧರಿಸಿದ ಮೇಲೆ ಆಕೆಯನ್ನು ಕೈಹಿಡಿದು ಮೇಲಕ್ಕೆಳೆದುಕೊಂಡನು.

ದಡವನ್ನು ಸೇರಿದ ದೇವಯಾನಿ ಯಯಾತಿಯನ್ನು ಕುರಿತು ‘ವೀರವರ, ನೀನು ನನ್ನ ಕೈಹಿಡಿದು ಮೇಲಕ್ಕೆತ್ತಿರುವೆ. ಇದು ನನ್ನ ಪಾಣಿಗ್ರಹಣವೆಂದೇ ನಾನು ಭಾವಿಸುತ್ತೇನೆ. ದೈವವೇ ನಮ್ಮನ್ನು ಹೀಗೆ ಕೂಡಿಸಿತೆಂದು ಕಾಣಿಸುತ್ತದೆ. ನಾನು ಬ್ರಾಹ್ಮಣಳಾದರೂ ಬ್ರಾಹ್ಮಣಪತಿಯನ್ನು ವರಿಸುವಂತಿಲ್ಲ. ಕಚನೆಂಬ ನನ್ನ ತಂದೆಯ ಶಿಷ್ಯ ನನಗೆ ಹಾಗೆಂದು ಶಾಪ ಕೊಟ್ಟಿದ್ದಾನೆ. ಆದ್ದರಿಂದ ಇಂದಿನಿಂದ ನೀನೆ ನನ್ನ ಪತಿ’ ಎಂದು ನುಡಿದಳು. ಸುಂದರವಾದ ಅವಳ ಮುಖದಿಂದ ಬಂದ ಆ ಮಾತುಗಳು ಪ್ರೇಮದ ಹೊಳೆಯನ್ನೆ ಹರಿಸುತ್ತಿರುವಂತೆ ಕಾಣಿಸಿತು, ಯಯಾತಿಗೆ. ಯುವಕನಾದ ಯಯಾತಿಯ ಮನಸ್ಸು ಆಕೆಯ ರೂಪ ಲಾವಣ್ಯಕ್ಕೆ ಮರುಳಾಯಿತು. ಆತ ಆಕೆಯನ್ನು ವರಿಸಲು ಒಪ್ಪಿದ. ಹೀಗೆ ಪರಸ್ಪರ ಒಪ್ಪಿಗೆಯಾದ ಮೇಲೆ ಯಯಾತಿ ತನ್ನ ರಾಜಧಾನಿಗೆ ಹಿಂದಿರುಗಿದ. ದೇವಯಾನಿ ತಂದೆಯ ಬಳಿಗೆ ಬಂದಳು.

ತಂದೆಯನ್ನು ಕಾಣುತ್ತಲೆ ದೇವಯಾನಿಗೆ ದುಃಖ ಒತ್ತರಿಸಿ ಬಂತು; ಆಕೆ ‘ಗೋಳೋ’ ಎಂದು ಗಟ್ಟಿಯಾಗಿ ಅತ್ತಳು. ಶುಕ್ರಾಚಾರ್ಯನಿಗೆ ಮಗಳೆಂದರೆ ಪಂಚಪ್ರಾಣ. ಅವಳ ಅಳು ವನ್ನು ಕಂಡು ಆತನ ಜೀವ ತಲ್ಲಣಿಸಿತು. ಮಗಳನ್ನು ಮೃದು ನುಡಿಗಳಿಂದ ರಮಿಸಿ, ಕಾರಣ ವೇನೆಂದು ಕೇಳಿದ. ಅವಳು ಅಳುತ್ತಾ ಶರ್ಮಿಷ್ಠೆ ಆಡಿದ ಮಾತುಗಳನ್ನು ಹೇಳಿದಳು. ಅದನ್ನು ಕೇಳಿ ಆತನಿಗೂ ಮನಸ್ಸಿಗೆ ಕಸಿವಿಸಿಯಾಯಿತು. ತಾನು ರಾಜನ ಪುರೋಹಿತ ನಾಗಿರುವುದಕ್ಕಿಂತಲೂ ನಾಲ್ಕು ಮನೆಗಳಲ್ಲಿ ಭಿಕ್ಷೆ ಬೇಡುವುದು ಲೇಸೆನಿಸಿತು. ಆತ ಮಗಳೊಡನೆ ತಕ್ಷಣವೇ ಆ ಊರನ್ನೆ ಬಿಟ್ಟು ಹೊರಟ. ಇದು ವೃಷಪರ್ವನಿಗೆ ಗೊತ್ತಾ ಯಿತು. ದಾನವರಿಗೆಲ್ಲ ಪ್ರಾಣಸಮಾನನಾದ ಶುಕ್ರಾಚಾರ್ಯನು ದೇವತೆಗಳ ಕಡೆ ಸೇರಿ ಕೊಂಡರೆ ಏನು ಗತಿ? ವೃಷಪರ್ವನು ಹಿಂದೆಯೇ ಓಡಿಹೋಗಿ, ನಡುದಾರಿಯಲ್ಲಿ ಸಿಕ್ಕ ಆತನಿಗೆ ಅಡ್ಡ ಬಿದ್ದು, ತನ್ನನ್ನು ಬಿಟ್ಟು ಹೋಗಬಾರದೆಂದು ಬೇಡಿಕೊಂಡ. ಶುಕ್ರಾಚಾರ್ಯನ ಮನಸ್ಸು ಆತನ ದೈನ್ಯಕ್ಕೆ ಕರಗಿತು. ಆತ ಒಂದೆ ಮಾತು ಹೇಳಿದ. ‘ಮಹಾರಾಜ, ದೇವಯಾನಿ ನನ್ನ ಜೀವ. ಅವಳಿಲ್ಲದೆ ನಾನು ಉಳಿಯಲಾರೆ. ಅವಳು ನಿಂತರೆ ನಾನು ನಿಲ್ಲುತ್ತೇನೆ.’ ಆಗ ವೃಷಪರ್ವನು ದೇವಯಾನಿಯನ್ನು ಕುರಿತು ‘ಅಮ್ಮ, ನಿನ್ನ ಇಷ್ಟ ಬಂದಂತೆ ನಾನು ನಡೆದುಕೊಳ್ಳುತ್ತೇನೆ. ನೀನು ಇಲ್ಲಿಂದ ಹೋಗಬೇಡ’ ಎಂದ. ದೇವ ಯಾನಿ ತನ್ನ ಇಷ್ಟವನ್ನು ತಿಳಿಸಿದಳು–‘ಮಹಾರಾಜ, ನಿನ್ನ ಮಗಳು ತನ್ನ ದಾಸಿಯರೊಡನೆ ಬಂದು, ನಾನೆಲ್ಲಿದ್ದರೆ ಅಲ್ಲಿ ನನಗೆ ದಾಸಿಯಾಗಿರಬೇಕು.’ ವೃಷಪರ್ವನು ನುಂಗಲಾರದ ಈ ತುತ್ತನ್ನು ಮಗಳಾದ ಶರ್ಮಿಷ್ಠೆಯ ಮುಂದಿಟ್ಟ. ಆಕೆ ಹುಡುಗಿಯಾದರೂ ಪ್ರಾಜ್ಞೆ. ಶುಕ್ರಾಚಾರ್ಯನಿಲ್ಲದೆ ದಾನವಸಂತಾನಕ್ಕೆ ಉಳಿಗಾಲವಿಲ್ಲವೆಂಬುದನ್ನು ಆಕೆ ಬಲ್ಲಳು. ಆದ್ದ ರಿಂದ ದಾನವರ ಕ್ಷೇಮಕ್ಕಾಗಿ ಆಕೆ ಬಲಿದಾನಕ್ಕೆ ಸಿದ್ಧಳಾದಳು. ಅಂದಿನಿಂದ ದೇವ ಯಾನಿಯ ದಾಸಿಯಾಗಿ ಹೋದಳು.

ಶುಕ್ರಾಚಾರ್ಯನು ಮಗಳ ಇಷ್ಟದಂತೆ ಆಕೆಯನ್ನು ಯಯಾತಿಗೆ ಧಾರೆಯೆರೆದು ಕೊಟ್ಟನು. ಶರ್ಮಿಷ್ಠೆಯನ್ನೂ ಆತನಿಗೇ ವಿವಾಹಮಾಡಿಕೊಟ್ಟನಾದರೂ ‘ನೀನು ಎಂದಿಗೂ ಅವ ಳೊಂದಿಗೆ ರಮಿಸಕೂಡದು’’ ಎಂದು ಅಳಿಯನಿಗೆ ಅಪ್ಪಣೆಮಾಡಿದನು. ಆತನು ಅದಕ್ಕೆ ಒಪ್ಪಿ ಇಬ್ಬರು ಮಡದಿಯರನ್ನೂ ಕರೆದುಕೊಂಡು ತನ್ನ ಊರಿಗೆ ಹಿಂದಿರುಗಿದನು. ಆತ ನೊಡನೆ ದೇವಯಾನಿಯು ಸ್ವರ್ಗಸುಖವನ್ನು ಸೂರೆಗಳ್ಳುತ್ತಾ ಇಬ್ಬರು ಮಕ್ಕಳ ತಾಯಿ ಯಾದಳು. ಇದನ್ನು ಕಂಡು ಶರ್ಮಿಷ್ಠೆಗೂ ಮಕ್ಕಳ ಬಯಕೆ ಹುಟ್ಟಿತು. ಒಮ್ಮೆ ಪುತುಸ್ನಾತ ಳಾದ ಆಕೆ ಯಯಾತಿಯನ್ನು ಗುಟ್ಟಾಗಿ ಕಂಡು ತನಗೂ ಸಂತಾನವನ್ನು ನೀಡುವಂತೆ ಬೇಡಿ ದಳು. ಅಮೃತಫಲವನ್ನು ಇದಿರಿಗಿಟ್ಟು ಇದನ್ನು ತಿನ್ನಬೇಡ ಎಂದು ಹೇಳಿದಂತಾಗಿತ್ತು, ಯಯಾತಿಗೆ. ಶರ್ಮಿಷ್ಠೆಯ ಮೋಹನಾಕಾರ ಆತನ ಹೃದಯವನ್ನು ಸೂರೆಗೊಂಡಿತ್ತಾ ದರೂ ಮಾವನ ಭಯದಿಂದ ಆತ ಇದುವರೆಗೆ ಸುಮ್ಮನಿದ್ದ. ಇಂದು ಆಕೆ ತಾನಾಗಿಯೇ ಬಂದು ಬೇಡುತ್ತಲೆ ಆತನ ಭಯ ಹಾರಿಹೋಯಿತು. ಆತನು ಆಕೆಯ ಅಪೇಕ್ಷೆಯನ್ನು ಸಲ್ಲಿಸಿ ಧನ್ಯನಾದೆನೆಂದುಕೊಂಡನು. ಶರ್ಮಿಷ್ಠೆಯೂ ಧನ್ಯಳಾದಳು. ದೇವಯಾನಿಗೆ ಇಬ್ಬರು ಮಕ್ಕಳಾದರೆ, ತನಗೆ ಮೂವರು ಮಕ್ಕಳು!

ಶರ್ಮಿಷ್ಠೆಗೆ ಮಕ್ಕಳಾದುದನ್ನು ಕಾಣುತ್ತಲೆ ದೇವಯಾನಿ ಕೆರಳಿದ ಸಿಂಹಿಣಿಯಾದಳು. ಗಂಡ ಕಾಲಿಗೆ ಬಿದ್ದು ಬೇಡಿಕೊಂಡರೂ ಲಕ್ಷಿಸದೆ ಅವಳು ನೇರವಾಗಿ ತಂದೆಯ ಮನೆಗೆ ಹೊರಟುಬಂದಳು. ಮುದ್ದುಮಗಳು ಹೇಳಿದ ಸುದ್ದಿಯನ್ನು ಕೇಳಿ ಶುಕ್ರಚಾರ್ಯನಿಗೂ ರೇಗಿತು. ಆತ ಅಳಿಯನನ್ನು ಕುರಿತು ‘ಎಲೆ ಕಾಮುಕ, ನಾನು ಬಾಯಿ ಬಿಟ್ಟು ಹೇಳಿದರೂ ಕೂಡ ಹೀಗೆ ಮಾಡಿದೆಯಾ? ನೀನೇನು ಮಾಡೀಯೆ? ನಿನ್ನ ಯವ್ವನದ ಕೊಬ್ಬು ಅದು. ಆ ಕೊಬ್ಬು ಇಳಿದುಹೋಗುವಂತೆ ನಿನಗೆ ಮುಪ್ಪು ಬಂದು ಬಡಿದುಕೊಳ್ಳಲಿ’ ಎಂದ. ಯಯಾತಿ ಬಹು ದೈನ್ಯದಿಂದ ಆತನನ್ನು ಕುರಿತು ‘ಸ್ವಾಮಿ, ನಿಮ್ಮ ಮಗಳಲ್ಲಿ ನನಗೆ ಅಪಾರ ಪ್ರೇಮ. ಆಕೆಯಲ್ಲಿ ಸುಖಪಡಬೇಕೆಂಬ ನನ್ನ ಆಶೆ ಇನ್ನೂ ಹಚ್ಚ ಹಸಿಯಾಗಿದೆ. ಸ್ವಾಮಿ ನನ್ನ ಸುಖಕ್ಕೆ ಕಲ್ಲುಹಾಕಬೇಡಿ’ ಎಂದು ಬೇಡಿಕೊಂಡ. ಅವನ ದೈನ್ಯವನ್ನು ಕಂಡು ಆ ರಾಕ್ಷಸ ಪುರೋಹಿತನ ಹೃದಯ ಕರಗಿತು. ಆತ ಹೇಳಿದ–‘ಮಗು, ಇಟ್ಟ ಆಣೆ, ಕೊಟ್ಟ ಶಾಪ ತಪ್ಪುವಂತಿಲ್ಲ. ಆದರೆ ನಿನ್ನ ಮುಪ್ಪನ್ನು ಇನ್ನೊಬ್ಬರಿಗೆ ಕೊಟ್ಟು ಅವರ ಯವ್ವನವನ್ನು ನೀನು ಸ್ವೀಕರಿಸಬಹುದು. ಆ ಶಕ್ತಿಯನ್ನು ನಾನು ನಿನಗೆ ಅನುಗ್ರಹಿಸುತ್ತೇನೆ.’

ಯಯಾತಿ ಮುಪ್ಪಿನ ಮುದುಕನಾಗಿ ತನ್ನ ಅರಮನೆಗೆ ಹಿಂದಿರುಗಿದ. ಆತನಿಗೆ ಕಾಮದ ದಾಹ. ಅದಕ್ಕೆ ಅಡ್ಡಿಯಾದ ಮುಪ್ಪನ್ನು ಯಾರಿಗೆ ಕೊಡುವುದು? ಆತ ತನ್ನ ಮಕ್ಕಳಲ್ಲಿ ಒಬ್ಬೊಬ್ಬರನ್ನೆ ಕರೆದು, ತನ್ನ ಮುಪ್ಪನ್ನು ತೆಗೆದುಕೊಂಡು, ಅವರ ತಾರುಣ್ಯವನ್ನು ತನಗೆ ಕೊಡುವಂತೆ ಕೇಳಿದ. ಅವರೇಕೆ ಅದಕ್ಕೆ ಒಪ್ಪುತ್ತಾರೆ? ಹಿರಿಯ ಮಗ ಹೇಳಿದ– ‘ಅಲ್ಲಪ್ಪ, ಇಷ್ಟು ದಿನ ಸುಖವನ್ನು ಅನುಭವಿಸಿದರೂ ನಿನಗೆ ತೃಪ್ತಿಯಿಲ್ಲ. ಅಂತಹ ಸುಖವನ್ನು ಇನ್ನೂ ರುಚಿಯೇ ನೋಡದ ನಾನು ನಿನಗಾಗಿ ಬಿಟ್ಟುಕೊಡಬೇಕೆ? ಸಾಧ್ಯವಿಲ್ಲ’ ಎಂದ. ಅವನ ಐದು ಜನ ಮಕ್ಕಳಲ್ಲಿ ಮೊದಲ ನಾಲ್ಕು ಜನರೂ ಇದೇ ಧಾಟಿಯಲ್ಲಿ ಮಾತನಾಡಿ ದರು. ಕಡೆಯ ಮಗನಾದ ಪೂರುವು ವಯಸ್ಸಿನಲ್ಲಿ ಚಿಕ್ಕವನಾದರೂ ಗುಣದಲ್ಲಿ ದೊಡ್ಡ ವನು. ಆತನು ತಂದೆಯನ್ನು ಕುರಿತು ‘ಅಪ್ಪ, ಈ ದೇಹ ನೀನು ಕೊಟ್ಟುದು. ಇದು ನಿನಗೇ ಸೇರಿದುದು. ನೀನು ಹೇಳುವುದಕ್ಕೆ ಮೊದಲೇ ನಿನ್ನ ಅಪೇಕ್ಷೆಯನ್ನರಿತು ಅದಕ್ಕೆ ಅನು ಸಾರವಾಗಿ ನಡೆಯುವುದು ಸತ್ಪುತ್ರನ ಲಕ್ಷಣ. ನೀನು ಹೇಳಿದ ಮೇಲೆಯೂ ಅದನ್ನು ನಡೆಸ ದಿದ್ದರೆ ಅವನು ಮಗನಲ್ಲ, ಅಮೇಧ್ಯ. ನಾನು ಅಗತ್ಯವಾಗಿಯೂ ನನ್ನ ಯವ್ವನವನ್ನು ನಿನಗೆ ಕೊಡುತ್ತೇನೆ. ನಿನ್ನ ಮುಪ್ಪನ್ನು ನನಗೆ ಕೊಡು’ ಎಂದನು.

ಮಗನಿಂದ ಯವ್ವನವನ್ನು ಪಡೆದ ಯಯಾತಿ ಬಹುಕಾಲ ಭೋಗ ಸುಖಗಳನ್ನು ಯಥೇಚ್ಛವಾಗಿ ಅನುಭವಿಸಿದ. ಆತನು ಅನೇಕ ಯಾಗಗಳನ್ನು ಮಾಡಿ ಭಗವಂತನ ಆರಾಧನೆಯನ್ನೂ ಮಾಡಿದರೂ ಆತನ ಭೋಗೇಚ್ಛೆ ಮಾತ್ರ ಸಾಯಲಿಲ್ಲ. ಸಹಸ್ರವರ್ಷ ಆತ ಇಂದ್ರಿಯಗಳಿಗೆ ಸುಖವನ್ನು ಒದಗಿಸಿದನಾದರೂ ಅವು ಮಾತ್ರ ತೃಪ್ತಿಗೊಳ್ಳಲಿಲ್ಲ. ಕಡೆಗೊಂದು ದಿನ ಆತನಿಗೆ ತಾನಾಗಿಯೇ ವಿವೇಕ ಹುಟ್ಟಿತು. ಆತನು ಬಾಳೆಲ್ಲವನ್ನೂ ಜ್ಞಾಪಿಸಿಕೊಂಡು ಪಶ್ಚಾತ್ತಾಪಪಟ್ಟನು. ತನ್ನ ಮಡದಿ ದೇವಯಾನಿಗೂ ವಿವೇಕ ಹುಟ್ಟುವುದಕ್ಕಾಗಿ ಆತನು ತನ್ನ ಬಾಳನ್ನೆ ಒಂದು ಕಥೆಯ ರೂಪದಲ್ಲಿ ವಿವರಿಸಿ ಹೇಳಿದ– ‘ಒಂದು ಕಾಡಿನಲ್ಲಿ ಒಂದು ಟಗರಿತ್ತು. ಅದು ಅಲ್ಲಲ್ಲಿ ತಿರುಗುತ್ತಿರುವಾಗ ಹಾಳು ಬಾವಿಯಲ್ಲಿ ಬಿದ್ದಿದ್ದ ಒಂದು ಮೇಕೆ ಕಾಣಿಸಿತು. ಟಗರು ತನ್ನ ಕೊಂಬಿನಿಂದ ಆ ಬಾವಿಯ ದಡದ ಮಣ್ಣನ್ನು ಅಗೆದು ದಾರಿಮಾಡಿತು. ಈ ಉಪಕಾರಕ್ಕಾಗಿ ಮೇಕೆ ಟಗರನ್ನು ವರಿಸಿತು. ಆ ಟಗರಿನ ಮೈಕಟ್ಟನ್ನು ಕಂಡು ಮತ್ತೆ ಕೆಲವು ಮೇಕೆಗಳು ಅದನ್ನೇ ವರಿಸಿದವು. ಇದನ್ನು ಕಂಡ ಮೊದಲ ಮೇಕೆ, ಆ ಟಗರನ್ನು ಬಿಟ್ಟು ತನ್ನ ಯಜಮಾನನ ಬಳಿ ಹೋಯಿತು. ಆ ಯಜಮಾನ ಟಗರಿನ ಕೊಬ್ಬನ್ನಿಳಿಸಿದ. ಆಗ ಟಗರು ಮತ್ತೆ ಆ ಬ್ರಾಹ್ಮಣನ ಅನುಗ್ರಹ ದಿಂದಲೆ ಕೊಬ್ಬಿ ಆತನ ಮೇಕೆಯೊಡನೆ ಸದಾ ಸ್ನೇಹದಿಂದಿತ್ತು. ಇದನ್ನು ನೋಡಿದರೆ ನಿನಗೇನನ್ನಿಸುತ್ತದೆ? ಬೆಂಕಿಗೆ ತುಪ್ಪ ಹಾಕಿದರೆ ಉರಿ ಹೆಚ್ಚುತ್ತದೆ. ಕಾಮವೂ ಅಷ್ಟೆ. ಭೋಗ ಸುಖ ಸಿಕ್ಕಷ್ಟೂ ಅದರ ಆಸೆ ಹೆಚ್ಚುತ್ತದೆ. ಇದು ಈಗ ಅರ್ಥವಾಗಿರುವುದರಿಂದ ನಾನಿನ್ನು ಅಡವಿಗೆ ಹೋಗಿ ತಪಸ್ಸು ಮಾಡಬೇಕೆಂದಿರುವೆನು’ ಎಂದು ಹೇಳಿದನು. ಆತನು ತನ್ನ ಕಿರಿಯ ಮಗನ ಯವ್ವನವನ್ನು ಆತನಿಗೆ ಹಿಂದಕ್ಕಿತ್ತು, ತನ್ನ ಮುಪ್ಪನ್ನು ಸ್ವೀಕರಿಸಿದನು. ಅನಂತರ ತನ್ನ ರಾಜ್ಯವನ್ನೆಲ್ಲ ಆ ಕಿರಿಯ ಮಗನಿಗೆ ಕೊಟ್ಟು, ಉಳಿದವರು ಆತನ ಅಧೀನದ ಲ್ಲಿರುವಂತೆ ಕಟ್ಟುಮಾಡಿ, ಆತನು ತಪಸ್ಸು ಮಾಡುವುದಕ್ಕಾಗಿ ಅಡವಿಗೆ ಹೊರಟು ಹೋದನು. ದೇವಯಾನಿಯು ಆತನು ಹೇಳಿದ ಮೇಕೆಯ ಕಥೆ ತನ್ನದೇ ಎಂದು ಅರ್ಥ ಮಾಡಿಕೊಂಡು ತಾನೂ ಗಂಡನಂತೆ ಸರ್ವಸಂಗಪರಿತ್ಯಾಗ ಮಾಡಿ ದೇವರ ಧ್ಯಾನದಲ್ಲಿ ನಿರತಳಾದಳು.

