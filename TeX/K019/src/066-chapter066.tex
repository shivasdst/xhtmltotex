
\chapter{೬೬. ಆಹಾ! ಈ ಸುಂದರಾಂಗನಾರು?}

ಪರಸ್ಪರ ಅನುರಾಗದಿಂದ ಗಂಡ-ಹೆಂಡಿರಾಗಿದ್ದ ಶ್ರೀಕೃಷ್ಣರುಕ್ಮಿಣಿಯರು ಸಂಸಾರಸಾರ ಸರ್ವಸ್ವನ್ನು ಸವಿಯುತ್ತಾ ಕೆಲಕಾಲ ಕಳೆಯಲು ಅವರ ಪ್ರೇಮವೇ ಆಕಾರವನ್ನು ತಾಳಿದಂತೆ ಸರ್ವಾಂಗ ಸುಂದರನಾದ ಮಗನೊಬ್ಬನು ಹುಟ್ಟಿದನು. ಆತನ ಹೆಸರು ‘ಪ್ರದ್ಯುಮ್ನ.’ ರೂಪದಲ್ಲಿ ಮಾತ್ರವೇ ಅಲ್ಲ, ಅದೃಷ್ಟದಲ್ಲಿಯೂ ಆ ಮಗ ಅಪ್ಪನನ್ನೆ ಹೋಲುತ್ತಿದ್ದ. ಅವನು ಹುಟ್ಟಿ ಹತ್ತು ದಿನಗಳಾಗುವುದಕ್ಕೂ ಮುಂಚೆಯೇ ಅವನು ಹೆತ್ತವರಿಂದ ಅಗಲಿ ಹೋಗಬೇಕಾಯಿತು. ಶಂಬರನೆಂಬ ಒಬ್ಬ ರಾಕ್ಷಸ ಆ ಮಗುವಿನಿಂದಲೆ ತನಗೆ ಮರಣ ವೆಂದು ನಾರದರಿಂದ ತಿಳಿದು, ತನ್ನ ಮಾಯಾವಿದ್ಯೆಯಿಂದ ಅದನ್ನು ಹಾರಿಸಿಕೊಂಡು ಹೋದ. ಅವನು ಆ ಮಗುವನ್ನು ಸಮುದ್ರ ಮಧ್ಯದಲ್ಲಿ ಬಿಸುಟು, ತನ್ನ ಶತ್ರು ನಿಶ್ಶೇಷ ವಾಯ್ತೆಂದು ನಿಶ್ಚಿಂತನಾದ. ಆದರೆ ಕೊಲ್ಲುವವನಿಗಿಂತ ಕಾಯುವವನು ದೊಡ್ಡವನು. ಸಮುದ್ರ ಮಧ್ಯದಲ್ಲಿ ಬಿದ್ದ ಮಗು ಸಾಯಲಿಲ್ಲ; ದೊಡ್ಡದೊಂದು ಮೀನು ಆ ಮಗುವನ್ನು ಅನಾಮತ್ತಾಗಿ ನುಂಗಿತು. ಅದೇ ಮೀನು ದೈವಯೋಗದಿಂದ ಬೆಸ್ತನ ಬಲೆಗೆ ಬಿತ್ತು. ಅಷ್ಟು ದೊಡ್ಡ ಮೀನು ಅರಸರಿಗೆ ತಕ್ಕುದೆಂದು ಬೆಸ್ತನು ಆ ಮೀನನ್ನು ಶಂಬಾಸುರನಿಗೆ ಒಪ್ಪಿಸಿ ದನು. ಆತ ಅದನ್ನು ತನ್ನ ಅಡುಗೆಯವರ ಕೈಗೆ ಕೊಟ್ಟ. ಅಡುಗೆಯವನು ಅದನ್ನು ಸೀಳಿ ದಾಗ ಅದರ ಹೊಟ್ಟೆಯಲ್ಲಿ ಮಗುವಿತ್ತು. ಅವರು ಆಶ್ಚರ್ಯದಿಂದ ಆ ಮಗುವನ್ನು ಅಡಿಗೆ ಮನೆಯ ಒಡತಿಯಾದ ಮಾಯಾದೇವಿಯ ಕೈಗಿತ್ತರು. ಆಕೆ ಆ ಮಗುವನ್ನು ಕಂಡು ಆಶ್ಯರ್ಯದಿಂದ ‘ಆಹಾ, ಈ ಸುಂದರಾಂಗನಾರು?’ ಎಂದುಕೊಂಡಳು.

ಮಾಯಾದೇವಿ ಎಳೆಯ ಮಗುವನ್ನು ಕಂಡು ‘ಈ ಸುಂದರಾಂಗನಾರು?’ ಎಂದು ಕೊಂಡುದು ಆಶ್ಚರ್ಯವೇನೂ ಅಲ್ಲ. ಆಕೆ ಮನ್ಮಥನ ಮಡದಿಯಾದ ರತಿ. ಗಂಡ ಸತ್ತ ಮೇಲೆ ಅಳುತ್ತಾ ಅಡವಿಯಲ್ಲಿ ತಿರುಗುತ್ತಿದ್ದ ಆಕೆಯನ್ನು ಅಕಸ್ಮಾತಾಗಿ ಕಂಡ ಶಂಬರ ಆಕೆಯ ರೂಪಕ್ಕೆ ಮರುಳಾಗಿ, ಆಕೆಯನ್ನು ಎಳೆತಂದು ತನ್ನ ಅಡಿಗೆಮನೆಯ ಒಡತಿ ಯನ್ನಾಗಿ ಮಾಡಿದ್ದ. ಆಕೆ ಆ ಮಗುವಿನಲ್ಲಿ ತನ್ನ ಗಂಡನ ರೂಪು ಅಚ್ಚಳಿಯದೆ ಇರುವು ದನ್ನು ಕಂಡು, ಆಶ್ಚರ್ಯದಿಂದ ಮೈಮರೆತು ಆ ಮಾತನ್ನಾಡಿದ್ದಳು. ಆಕೆಯ ಬಾಯಿಂದ ಆ ಮಾತು ಬರುವಷ್ಟರಲ್ಲಿ ನಾರದರು ಅಲ್ಲಿ ಕಾಣಿಸಿಕೊಂಡು, ಆ ಮಗುವೇ ಮನ್ಮಥ ನೆಂದೂ ಶ್ರೀಕೃಷ್ಣನ ಮಗನಾಗಿ ಹುಟ್ಟಿರುವ ಆ ಮಗು ಮುಂದೆ ಶಂಬರನನ್ನು ಕೊಂದು ರತಿಯನ್ನು ಮದುವೆಯಾಗುವನೆಂದೂ ತಿಳಿಸಿದನು. ಇದನ್ನು ಕೇಳಿದ ಮಾಯಾದೇವಿಯು ಅತ್ಯಂತ ರಹಸ್ಯವಾಗಿ ಆ ಮಗುವನ್ನು ಪೋಷಿಸಿ ಬೆಳೆಸುತ್ತಿದ್ದಳು. ಆ ಮಗು ಶುಕ್ಲಪಕ್ಷದ ಚಂದ್ರನಂತೆ ದಿನದಿನಕ್ಕೂ ಬೆಳೆದು ಪ್ರಾಯದವನಾದನು. ಸಾಕ್ಷಾತ್ ಮನ್ಮಥನಾದ ಆತನ ರೂಪಸೌಂದರ್ಯವನ್ನು ಮಾತುಗಳಿಂದ ಬಣ್ಣಿಸಲು ಸಾಧ್ಯವೆ? ತಂದೆಯಂತೆಯೇ ನೀಲ ಮೇಘಶ್ಯಾಮವಾದ ಶರೀರ, ನೀಳವಾದ ತೋಳುಗಳು, ಕಮಲದಂತಹ ಕಣ್ಣು, ಕಪ್ಪಗೆ ಗುಂಗುರು ಗುಂಗುರಾಗಿರುವ ಮುಂಗೂದಲು, ಮುಖದಲ್ಲಿ ಮೋಹಕವಾದ ಮಂದ ಹಾಸ. ಇತ್ತೀಚೆಗೆ ಕೃಷ್ಣನನ್ನು ಕಾಣದಿದ್ದವರು ಆತನನ್ನು ಕೃಷ್ಣನೆಂದೆ ಭ್ರಮಿಸಿಯಾರು! ಈ ಲೋಕಮೋಹಕ ಮೂರ್ತಿಯನ್ನು ಕಂಡು ಮಾಯಾವತಿಯ ಮೋಹ ಕಟ್ಟುಹರಿದ ಪಂಜಿ ನಂತಾಯಿತು. ಆಕೆ ಆತನನ್ನು ಆಲಂಗಿಸಹೋದಳು. ಪ್ರದ್ಯುಮ್ನನು ಹಾವು ಮೆಟ್ಟಿದವ ನಂತೆ ಹೆದರಿ, ಹಿಂದಕ್ಕೆ ಹಾರಿ ‘ಅಮ್ಮ, ಇದೇನಿದು ಅನ್ಯಾಯ? ಹೆತ್ತ ತಾಯಾದ ನೀನು ಇದ್ದಕ್ಕಿದ್ದಂತೆ ನನ್ನಲ್ಲಿ ಕಾಮಾತುರಳಾಗಿ ಹೋಗುವುದೇ?’ ಎಂದನು. ಮಾಯಾವತಿಯು ತನ್ನ ಮತ್ತು ಆತನ ಕಥೆಯನ್ನೆಲ್ಲ ಆದ್ಯಂತವಾಗಿ ಆತನಿಗೆ ತಿಳಿಸಿದಳು. ‘ಪ್ರಾಣಪ್ರಿಯ, ನಿನ್ನ ತಾಯಿಯಾದ ರುಕ್ಮಿಣಿ ಕರುವನ್ನು ಕಳೆದುಕೊಂಡ ಆಕಳಿನಂತೆ ಈಗಲೂ ಸಂಕಟ ದಿಂದ ಕುದಿಯುತ್ತಿರುವಳು. ನೀನು ಪಾಪಿಯಾದ ಈ ಶಂಬರನನ್ನು ಕೊಂದುಹಾಕು. ಆಮೇಲೆ ನಾವಿಬ್ಬರೂ ದ್ವಾರಕೆಗೆ ಹೋಗಿ ಶ್ರೀಕೃಷ್ಣ ರುಕ್ಮಿಣಿಯರನ್ನು ಸಂತೋಷ ಪಡಿಸೋಣ’ ಎಂದು ಹೇಳಿ ಶಂಬರನನ್ನು ಕೊಲ್ಲುವ ‘ಮಹಾಮಾಯೆ’ ಎಂಬ ವಿದ್ಯೆ ಯನ್ನು ಆತನಿಗೆ ಉಪದೇಶಿಸಿದಳು.

ಸಹಜ ಶೂರನಾದ ಪ್ರದ್ಯುಮ್ನನು ‘ಮಹಾಮಾಯೆ’ಯ ಪ್ರಭಾವದಿಂದ ಅದ್ವಿತೀಯ ಧೈರ್ಯವನ್ನು ಪಡೆದು, ಶಂಬರಾಸುರನನ್ನು ಯುದ್ಧಕ್ಕೆ ಕರೆದನು. ಹೆಡೆಮೆಟ್ಟಿದ ಹಾವಿನಂತೆ ರೋಷಗೊಂಡ ಶಂಬರನು ತನ್ನ ಗದೆಯನ್ನು ಎತ್ತಿ ಪ್ರದ್ಯುಮ್ನನ ತಲೆಯ ಮೇಲೆ ಹೊಡೆಯಹೋದನು. ಅದರೆ ಪ್ರದ್ಯುಮ್ನ ಅದನ್ನು ತನ್ನ ಕೈಲಿದ್ದ ಗದೆಯಿಂದ ಒಡೆದು ಪುಡಿಮಾಡಿದನು. ಇವನ ಶಕ್ತಿಯನ್ನು ಕಂಡು ಅಚ್ಚರಿಗೊಂಡ ಶಂಬರನು ರಾಕ್ಷಸ ಮಾಯೆ ಯನ್ನು ಕೈಕೊಂಡು, ಅವನ ಮೆಲೆ ಕಲ್ಲುಗಳ ಮಳೆಯನ್ನು ಕರೆಯಲು ಪ್ರಾರಂಭಿಸಿದನು. ಆದರೆ ಪ್ರದ್ಯುಮ್ನನ ‘ಮಹಾಮಾಯಾ’ ಶಕ್ತಿಯ ಮುಂದೆ ರಕ್ಕಸನ ಆಟವೇನೂ ನಡೆಯ ಲಿಲ್ಲ. ಅವನು ಹಿರಿದ ಕತ್ತಿಯೊಡನೆ ಪ್ರದ್ಯುಮ್ನನ ಮೇಲೆ ಎರಗಿದನು. ಆದರೆ ಪ್ರದ್ಯು ಮ್ನನು ಅದರಿಂದ ತಪ್ಪಿಸಿಕೊಂಡು, ತನ್ನ ಕೈಯ ಕತ್ತಿಯಿಂದ ರಕ್ಕಸನ ಕತ್ತನ್ನು ಕತ್ತರಿಸಿ ಹಾಕಿದನು. ಒಡನೆಯೇ ಮಾಯಾದೇವಿಯು ತನ್ನ ಗಂಡನೊಡನೆ, ಮೋಡದೊಡನೆ ಚಲಿ ಸುವ ಮಿಂಚಿನಂತೆ, ಆಕಾಶಮಾರ್ಗವಾಗಿಯೆ ದ್ವಾರಕಾಪುರಕ್ಕೆ ಬಂದು, ಶ್ರೀಕೃಷ್ಣನ ಅಂತಃ ಪುರವನ್ನು ಪ್ರವೇಶಿಸಿದಳು.

ಇದ್ದಕ್ಕಿದ್ದಂತೆ ಅಂತಃಪುರವನ್ನು ಪ್ರವೇಶಿಸಿದ ಪ್ರದ್ಯುಮ್ನನನ್ನು ಕಂಡು ಅಲ್ಲಿದ್ದ ಹೆಣ್ಣುಗಳೆಲ್ಲ ಶ್ರೀಕೃಷ್ಣನೆ ಬಂದನೆಂಬ ಭ್ರಾಂತಿಯಿಂದ ಅಲ್ಲಲ್ಲೆ ಅಡಗಿಕೊಂಡರು. ಆದರೆ ಮರುಕ್ಷಣದಲ್ಲಿಯೆ ಅವರಿಗೆ ತಮ್ಮ ತಪ್ಪು ಅರಿವಾಯಿತು. ಈತನ ಎದೆಯ ಮೆಲೆ ಆ ಶ್ರೀವತ್ಸವೆಂಬ ಮಚ್ಚೆಯಿಲ್ಲ. ವಯಸ್ಸಿನಲ್ಲಿಯೂ ಈತ ಶ್ರೀಕೃಷ್ಣನಿಗಿಂತ ಕಿರಿಯ. ಅಲ್ಲದೆ ಅವನೊಡನೆ ಸುಂದರಿಯಾದ ಹೆಣ್ಣೊಬ್ಬಳು ಬೇರೆ ಬಂದಿದ್ದಾಳೆ. ತಮ್ಮ ತಪ್ಪಿಗಾಗಿ ತಾವೆ ನಗುತ್ತಾ ಅವರು ಅಡಗಿದ್ದ ಸ್ಥಳದಿಂದ ಹೊರಗೆ ಬಂದರು. ಆ ವೇಳೆಗೆ ರುಕ್ಮಿಣೀ ದೇವಿ ಅಲ್ಲಿಗೆ ಬಂದಳು. ಮಗನನ್ನು ನೋಡುತ್ತಲೆ ಆಕೆಯ ಬಾಯಿಂದ ‘ಆಹಾ, ಈ ಸುಂದರಾಂಗನಾರು? ಈತನನ್ನು ಹೆತ್ತ ಭಾಗ್ಯವತಿ ಯಾರು?’ ಎಂಬ ಮಾತುಗಳು ಹೊರ ಬಂದವು. ಆಕೆಗೆ ಅವನಲ್ಲಿ ಪುತ್ರಪ್ರೇಮ ಉಕ್ಕಿತು. ಆಕೆಯ ಮೊಲೆಗಳಲ್ಲಿ ಹಾಲು ಜಲಿಸಿತು. ನೋಡುತ್ತ ನೋಡುತ್ತ ಆತನು ತನ್ನ ಮಗನೇ ಇರಬೇಕೆನ್ನಿಸಿತು ಆಕೆಗೆ. ರೂಪದಲ್ಲಿ ತನ್ನ ಗಂಡನನ್ನು ಅಚ್ಚಳಿಯದೆ ಹೋಲುತ್ತಿದ್ದಾನೆ. ಕಳೆದುಹೋದ ತನ್ನ ಮಗ ಬದುಕಿದ್ದರೆ ಆತನಿಗೆ ಅಷ್ಟೆ ವಯಸ್ಸಾಗಿರಬೇಕಾಗಿತ್ತು. ಅಲ್ಲದೆ ಮನಸ್ಸಿನಲ್ಲಿ ಇವನ ಮೇಲೆ ಎಂತಹ ವಾತ್ಸಲ್ಯ ಹುಟ್ಟುತ್ತಾ ಇದೆ! ರುಕ್ಮಿಣಿಯ ಆಲೋಚನೆಗೆ ಬೆಂಬಲ ನೀಡುವಂತೆ ಆಕೆಯ ಎಡಭುಜ ಅದುರಿತು. ಅಷ್ಟರಲ್ಲಿ ಶ್ರೀಕೃಷ್ಣನೂ ಅಲ್ಲಿಗೆ ಬಂದ. ಆತನೂ ತನಗೇನೂ ತಿಳಿಯದವನಂತೆ ‘ಅಹಾ, ಈ ಸುಂದರಾಂಗನಾರು?’ ಎಂದುಕೊಂಡು ನಿಂತಲ್ಲಿ ನಿಂತ. ಆ ಹೊತ್ತಿಗೆ ಸರಿಯಾಗಿ ನಾರದರು ಅಲ್ಲಿ ಅವತರಿಸಿದರು. ಅವರು ಪ್ರದ್ಯುಮ್ನ ಮಾಯಾದೇವಿಯರ ಕಥೆಯನ್ನು ಆದ್ಯಂತವಾಗಿ ತಿಳಿಸಿ, ಅಲ್ಲಿದ್ದವರನ್ನೆಲ್ಲ ಅಚ್ಚರಿಗೊಳಿಸಿದರು.

ರುಕ್ಮಿಣೀದೇವಿ ಮತ್ತೆ ಮಗನನ್ನು ಹೆತ್ತವಳಂತೆ ಆ ಮಗನನ್ನು ತಬ್ಬಿಕೊಂಡು, ಸಂತೋಷದಿಂದ ತಲೆಯನ್ನು ಮೂಸಿ ನೋಡಿದಳು. ಶ್ರೀಕೃಷ್ಣನು ಕೈಗೆ ಬಂದ ಮಗ, ಸುಂದರಳಾದ ಸೊಸೆ–ಇವರನ್ನು ಕಣ್ತುಂಬ ಕಂಡು ಹೆಮ್ಮೆಯಿಂದ ಹಿಗ್ಗಿದ.

