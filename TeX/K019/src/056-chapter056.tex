
\chapter{೫೬. ವರುಣಲೋಕದಿಂದ ವೈಕುಂಠಕ್ಕೆ}

ಶ್ರೀಕೃಷ್ಣನ ಸಾನ್ನಿಧ್ಯ ನಂದನನ್ನು ಮಹಾಭಾಗವತನನ್ನಾಗಿ ಮಾಡಿತ್ತು. ಏಕಾದಶಿಯ ಉಪವಾಸ, ದ್ವಾದಶಿಯ ಪಾರಣೆಗಳನ್ನು ಆತನು ಕ್ರಮ ತಪ್ಪದೆ ಆಚರಿಸುತ್ತಿದ್ದನು. ಒಮ್ಮೆ ಆತನು ಏಕಾದಶಿಯ ಉಪವಾಸವನ್ನು ಸಾಂಗವಾಗಿ ನಡೆಸಿ, ಮರುದಿನ ದ್ವಾದಶಿ ಬಹು ಸ್ವಲ್ಪ ಕಾಲ ಮಾತ್ರ ಇದ್ದುದರಿಂದ, ಅಷ್ಟರೊಳಗೆ ಪಾರಣೆಯನ್ನು ಮುಗಿಸಿಬಿಡಬೇಕೆಂಬ ಆತುರದಲ್ಲಿ, ಬೆಳಗಿನ ಝಾವಕ್ಕೆ ಮೊದಲೆ ಎದ್ದು, ಯಮುನಾನದಿಯಲ್ಲಿ ಸ್ನಾನಕ್ಕಿಳಿದನು. ರಾಕ್ಷಸರು ಸಂಚಾರ ಮಾಡುವ ಆ ವೇಳೆಯಲ್ಲಿ ಆತ ನೀರಿಗಿಳಿದುದನ್ನು ಕಂಡು, ವರುಣ ದೇವನ ಸೇವಕನಾದ ರಾಕ್ಷಸನೊಬ್ಬನು ಆತನನ್ನು ಹಿಡಿದು ವರುಣನ ಬಳಿಗೆ ಎಳೆದೊ ಯ್ದನು. ನದಿಗೆ ಹೋಗಿದ್ದ ನಂದನು ಎಷ್ಟು ಹೊತ್ತಾದರೂ ಹಿಂದಿರುಗದುದನ್ನು ಕಂಡು ಗೋಪಾಲಕುಲದವರಿಗೆಲ್ಲ ದಿಗಿಲಾಯಿತು. ಅವರು ಬಲರಾಮ ಕೃಷ್ಣರಿಗೆ ಆ ಸುದ್ದಿಯನ್ನು ಮುಟ್ಟಿಸಿದರು. ಅವರ ಕಳವಳವನ್ನು ಕಂಡು, ಸರ್ವಜ್ಞನಾದ ಶ್ರೀಕೃಷ್ಣನು ‘ಅಯ್ಯಾ, ಭಯಪಡಬೇಡಿ. ನಾನು ಹೋಗಿ, ತಂದೆಯನ್ನು ಎಲ್ಲಿದ್ದರೂ ಹುಡುಕಿ ತರುತ್ತೇನೆ’ ಎಂದು ಸಮಾಧಾನ ಹೇಳಿ, ನೇರವಾಗಿ ವರುಣಲೋಕಕ್ಕೆ ಹೋದನು. ಆತನನ್ನು ತನ್ನ ಲೋಕದಲ್ಲಿ ಕಾಣುತ್ತಲೆ ವರುಣನಿಗೆ ಪರಮಾನಂದವಾಯಿತು. ಆತನು ಆ ದೇವದೇವನನ್ನು ನಮಸ್ಕರಿಸಿ ‘ಪ್ರಭು, ಅವಿವೇಕಿಯಾದ ನನ್ನ ಆಳು ತಿಳಿಯದೆ ತಪ್ಪುಮಾಡಿದ್ದಾನೆ, ಕ್ಷಮಿಸು. ಆದರೆ ಅವನ ತಪ್ಪೆ ನನಗೆ ಅನುಗ್ರಹವಾಗಿ ಪರಿಣಮಿಸಿತು. ನೀನು ನನ್ನ ಲೋಕಕ್ಕೆ ಬಂದು ನನಗೆ ದರ್ಶನವಿತ್ತೆ. ನಾನು ಧನ್ಯ’ ಎಂದು ಸ್ತೋತ್ರ ಮಾಡಿದನು. ಅನಂತರ ತನ್ನ ಆಳು ಕರೆ ತಂದಿದ್ದ ನಂದನನ್ನು ಆತನು ಶ್ರೀಕೃಷ್ಣನ ವಶಕ್ಕೊಪ್ಪಿಸಿದನು. ವರುಣನ ನಡೆನುಡಿಗಳನ್ನು ಕಂಡು ಮೆಚ್ಚಿದ ಶ್ರೀಕೃಷ್ಣನು ಆತನನ್ನು ಆಶೀರ್ವದಿಸಿ, ತಂದೆಯೊಡನೆ ಗೋಕುಲಕ್ಕೆ ಹಿಂದಿರುಗಿದನು.

ವರುಣಲೋಕದಿಂದ ಹಿಂದಿರುಗಿದ ನಂದನು ತನಗಾದ ಅನುಭವವನ್ನೆಲ್ಲ ಕೇಳಿದವರ ಬಾಯಲ್ಲಿ ನೀರೂರುವಂತೆ ತನ್ನ ಬಂಧುಗಳಿಗೆಲ್ಲ ಬಣ್ಣಿಸಿ ಹೇಳಿದನು. ತಾನು ಎಂದೂ ಕಂಡು ಕೇಳಿಲ್ಲದಂತಹ ವರುಣನ ಐಶ್ವರ್ಯವನ್ನು, ಅವನ ಅಧಿಕಾರ ವೈಭವಗಳನ್ನು ಕಂಡು ಆತ ದಂಗಾಗಿಹೋಗಿದ್ದನು. ಅದಕ್ಕೂ ಹೆಚ್ಚಾಗಿ ಆ ದಿಕ್ಪಾಲಕನು ತನ್ನ ಮಗನಲ್ಲಿ ತೋರಿದ ಭಯಭಕ್ತಿಗಳು ಆತನನ್ನು ಆಶ್ಚರ್ಯಚಕಿತನನ್ನಾಗಿ ಮಾಡಿದ್ದವು. ಅದನ್ನು ಕೇಳಿ ಗೋಪಾಲರಿ ಗೆಲ್ಲ ಎಷ್ಟು ಹೆಮ್ಮೆ! ಅವರು ‘ನಮ್ಮ ಕೃಷ್ಣ ಸಾಕ್ಷಾತ್ ಪರಮೇಶ್ವರನೇ ನಿಜ. ಆದರೆ ಆತನ ನಿಜಸ್ವರೂಪವನ್ನು ಕಾಣುವ ಶಕ್ತಿ ಮಾತ್ರ ನಮಗಿಲ್ಲ’ ಎಂದುಕೊಂಡರು. ಅವರ ಆಶೆ ಶ್ರೀಕೃಷ್ಣನಿಗೆ ಅರ್ಥವಾಯಿತು. ಅವರ ಅಪೇಕ್ಷೆಯನ್ನು ಸಲ್ಲಿಸಬೇಕೆಂದು ಆತ ನಿಶ್ಚಯಿ ಸಿದ. ಒಡನೆಯೆ ಆತನು ಅವರನ್ನೆಲ್ಲ ಯುಮುನಾದಿಯಲ್ಲಿದ್ದ ‘ಬ್ರಹ್ಮಕುಂಡ’ವೆಂಬ ಮಡುವಿಗೆ ಕರೆದೊಯ್ದು, ಅದರಲ್ಲಿ ಮುಳುಗಿ ಬರುವಂತೆ ಹೇಳಿದನು. ಅವರು ಹಾಗೆ ಮಾಡಿ ಬರುತ್ತಲೆ ಅವರ ಕಣ್ಣಿಗೆ ವೈಕುಂಠ ಕಾಣಿಸಿತು. ಅದನ್ನು ಕಾಣುತ್ತಲೆ ಅವರ ಮನಸ್ಸು ತೃಪ್ತಿಯನ್ನು ಪಡೆಯಿತು. ಶ್ರೀಕೃಷ್ಣನು ಪರದೈವವೆಂಬ ಅವರ ನಂಬಿಕೆ ಸ್ಥಿರವಾಯಿತು. ಅವರು ಧನ್ಯರಾದರು.

