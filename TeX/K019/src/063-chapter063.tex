
\chapter{೬೩. ದ್ವಾರಕಾಪುರ ನಿರ್ಮಾಣ}

ಅಕ್ರೂರನು ಹಸ್ತಿನಾವತಿಯಿಂದ ಹಿಂದಿರುಗುವಷ್ಟರಲ್ಲಿ ಮಧುರಾನಗರಿಗೆ ಭಯಂಕರ ವಾದ ವಿಪತ್ತು ಪ್ರಾಪ್ತವಾಗಿತ್ತು. ಕಂಸನು ಸಾಯುತ್ತಲೆ ಅವನ ಮಡದಿಯರಾದ ಅಸ್ತಿ, ಪ್ರಾಸ್ತಿಯರು ತಮ್ಮ ತಂದೆಯಾದ ಜರಾಸಂಧನ ಮನೆಗೆ ಹೋಗಿ, ಅಲ್ಲಿಯೆ ನೆಲಸಿದರು. ಜರಾಸಂಧನೆಂದರೆ, ಸಾಮಾನ್ಯನಲ್ಲ. ಆತನು ಮಗಧ ದೇಶದ ದೊರೆ. ಆತನ ಬಳಿ ಬಹು ದೊಡ್ಡ ಸೇನೆಯಿತ್ತು. ಆನೇಕ ರಾಜರು ಆತನ ಗೆಳೆಯರಾಗಿದ್ದರು. ಆತನು ಮಹಾಶೂರ ನೆಂದು ಪ್ರಖ್ಯಾತನಾಗಿದ್ದನು. ತನ್ನ ಹೆಣ್ಣುಮಕ್ಕಳಿಬ್ಬರೂ ವಿಧವೆಯರಾಗಿ, ದುಃಖದಿಂದ ನಿಟ್ಟಸಿರುಬಿಡುವುದನ್ನು ಕಂಡು ಆತನ ಕೋಪ ಹೊತ್ತಿ ಧಗಧಗ ಉರಿಯಿತು. ತನ್ನ ಅಳಿಯ ನನ್ನು ಕೊಂದ ಶ್ರೀಕೃಷ್ಣನನ್ನೂ ಅವನ ಕಡೆಯ ಯಾದವರನ್ನೂ ಹೆಸರಿಲ್ಲದಂತೆ ಮಾಡ ಬೇಕೆಂದು ಆತನು ಇಪ್ಪತ್ತುಮೂರು ಅಕ್ಷೋಹಿಣೀ ಸೇನೆಯೊಡನೆ ಮಧುರೆಯ ಮೇಲೆ ದಂಡೆತ್ತಿ ಬಂದನು. ಮೇರೆ ಮೀರಿದ ಸಮುದ್ರದಂತೆ ಅಪಾರವಾಗಿದ್ದ ಆ ಸೇನೆಯನ್ನು ಕಂಡು ಮಧುರೆಯ ಜನರೆಲ್ಲ ಭಯದಿಂದ ತತ್ತರಿಸಿದರು. ಇದನ್ನು ಕಂಡು ಶ್ರೀಕೃಷ್ಣನು ತನ್ನ ಅವತಾರದ ಕಾರ್ಯ ಪ್ರಾರಂಭವಾಯಿತೆಂದುಕೊಂಡನು. ಪಾಪಿಕ್ಷತ್ರಿಯರಿಂದ ಕೂಡಿರುವ ಈ ಮಹಾಸೈನ್ಯವನ್ನು ನಾಶಮಾಡಿದರೆ ಭೂಭಾರ ಎಷ್ಟೋ ಕಡಿಮೆಯಾಗುವು ದೆಂದು ಆತನ ಯೋಚನೆ. ಆತನು ಹಾಗೆ ಯೋಚಿಸುತ್ತಿದ್ದಂತೆಯೆ, ಥಳಥಳ ಹೊಳೆಯು ತ್ತಿದ್ದ ಎರಡು ದಿವ್ಯ ರಥಗಳು ಆಕಾಶದಿಂದ ಕೆಳಗಿಳಿದು ಬಂದವು. ಕುದುರೆಗಳು, ಸಾರಥಿ ಗಳು, ಆಯುಧಗಳು–ಎಲ್ಲವೂ ಸಜ್ಜಾಗಿದ್ದ ಆ ರಥಗಳನ್ನು ತನ್ನ ಅಣ್ಣ ಬಲರಾಮನಿಗೆ ತೋರಿಸುತ್ತಾ, ಶ್ರೀಕೃಷ್ಣನು ‘ಅಣ್ಣ, ಈ ಎರಡು ರಥಗಳೂ ನಮಗಾಗಿಯೆ ಇಳಿದು ಬಂದಿವೆ. ಇವುಗಳಲ್ಲಿ ಒಂದು ನಿನ್ನದು. ಅದರಲ್ಲಿಯೆ ನಿನಗೆ ಬೇಕಾದ ಆಯುಧಗಳೆಲ್ಲ ಅಣಿಯಾಗಿವೆ. ನೀನು ಯುದ್ದಕ್ಕೆ ಸಿದ್ಧನಾಗಿ ಆ ರಥವನ್ನೇರು. ನಮ್ಮ ಅವತಾರ ಸಾರ್ಥಕ ವಾಗುವಂತೆ, ಜರಾಸಂಧನ ಸೈನ್ಯವನ್ನೆಲ್ಲ ತರಿದುಹಾಕಿ ಈ ಯಾದವರ ಭಯವನ್ನು ಹೋಗಲಾಡಿಸು’ ಎಂದನು.

ಬಲರಾಮ ಕೃಷ್ಣರು ತಮ್ಮ ದಿವ್ಯರಥಗಳನ್ನೇರಿ, ಅಪಾರವಾದ ಯಾದವ ಸೇನೆಯೊಡನೆ ಮಧುರಾ ಪಟ್ಟಣದಿಂದ ಹೊರಟರು. ದಾರುಕನೆಂಬ ಸಾರಥಿ ಶ್ರೀಕೃಷ್ಣನ ರಥವನ್ನು ಎಲ್ಲಕ್ಕೂ ಮುಂದೆ ನಡೆಸಿದನು. ಜರಾಸಂಧನ ಸೇನೆ ಕಣ್ಣಿಗೆ ಬೀಳುತ್ತಲೆ ಶ್ರೀಕೃಷ್ಣನು ಪಾಂಚಜನ್ಯವೆಂಬ ತನ್ನ ಶಂಖವನ್ನು ತೆಗೆದು, ಶತ್ರುಗಳ ಎದೆ ನಡುಗುವಂತೆ ಗಟ್ಟಿಯಾಗಿ ಊದಿದನು. ಇದನ್ನು ಕೇಳುತ್ತಲೆ ಜರಾಸಂಧನು ತನ್ನ ರಥವನ್ನು ಅವನ ಕಡೆ ನುಗ್ಗಿಸಿ ‘ಎಲಾ ಕೃಷ್ಣ, ಸ್ವಂತ ಸೋದರಮಾವನನ್ನೆ ಕೊಂದ ಪರಮ ಪಾಪಿ! ದನ ಕಾಯುವ ಹಚ್ಚ ಹಸಿಯ ಗೊಲ್ಲ! ನಿನ್ನನ್ನು ಸೀಳಿಹಾಕುವುದೇನೂ ಕಷ್ಟವಲ್ಲ. ಆದರೆ ನೀನಿನ್ನೂ ಹುಡುಗ; ನಿನ್ನೊಡನೆ ಯುದ್ಧ ಮಾಡುವುದು ನನ್ನಂತಹ ಪರಾಕ್ರಮಿಗಳಿಗೆ ನಾಚಿಕೆಗೇಡು. ನೀನು ಅಲ್ಲಿ ಬಿದ್ದಿರು’ ಎಂದು ಹೇಳಿ, ಬಲರಾಮನ ಕಡೆ ತಿರುಗಿಕೊಂಡು ‘ಎಲ ಬಲರಾಮ, ನಿನಗೆ ಧೈರ್ಯವಿದ್ದರೆ ನನ್ನೊಡನೆ ಯುದ್ಧಕ್ಕೆ ನಿಲ್ಲು. ಈಗಲೆ ನಿನ್ನನ್ನು ಕೊಂದು ವೀರಸ್ವರ್ಗಕ್ಕೆ ಕಳುಹಿಸುತ್ತೇನೆ’ ಎಂದ. ದುರಹಂಕಾರಿಯಾದ ಜರಾಸಂಧನ ಮಾತುಗಳಿಗೆ ಉತ್ತರವಾಗಿ ಶ್ರೀಕೃಷ್ಣನು ‘ಎಲೆ ಮೂಢ, ಮಾತಿನಿಂದ ಶೂರರಾಗುವುದಿಲ್ಲ; ಕಾರ್ಯದಲ್ಲಿ ಅದನ್ನು ತೋರಿಸಬೇಕು. ನಿನ್ನ ಮಾತುಗಳೆಲ್ಲ ಶುದ್ಧ ತಲೆಹರಟೆ, ತಲೆ ಹೊಯಿಸಿಕೊಳ್ಳುವ ಮುನ್ನ ಕುರಿ ಅರಚುವಂತೆ ಅರಚುತ್ತಿದ್ದಿ. ಶಕ್ತಿಯಿದ್ದರೆ ಬಾ ಯುದ್ಧಕ್ಕೆ’ ಎಂದ. ಮೊನಚಾದ ಆತನ ಮಾತುಗಳು ತಾಗಿ ಜರಾಸಂಧನ ಕೋಪ ಕೆರಳಿತು. ಆತನು ಬಲರಾಮ ಕೃಷ್ಣರ ಮೇಲೆ ಬಾಣಗಳ ಮಳೆ ಗರೆದನು, ಅವರ ರಥಗಳೆರಡೂ ಆ ಬಾಣಗಳಿಂದ ಮುಚ್ಚಿಹೋದವು. ಯಾದವ ಸೈನ್ಯ ಇದನ್ನು ಕಂಡು ಭಯದಿಂದ ನಡುಗಿತು. ಅಷ್ಟರಲ್ಲಿ ಮೋಡವನ್ನು ಭೇದಿಸಿ ಮೂಡುವ ಸೂರ್ಯನಂತೆ, ಶ್ರೀಕೃಷ್ಣನು ಶಾರ್ಙ್ಗವೆಂಬ ತನ್ನ ಬಿಲ್ಲಿನೊಡನೆ ಕಾಣಿಸಿಕೊಂಡು, ಸಿಡಿಲಿ ನಂತಹ ಬಾಣಗಳನ್ನು ಸುರಿಸಹತ್ತಿದನು. ಅವುಗಳ ಹೊಡೆತಕ್ಕೆ ಸಿಕ್ಕಿ ಶತ್ರುಸೇನೆಯ ಆನೆ ಗಳು ರುಳಿದವು. ಕುದುರೆಗಳ ಕತ್ತು ಕತ್ತರಿಸಿಹೋದುವು. ರಥಗಳು ಕುದುರೆ ಸಾರಥಿ ಗಳೊಡನೆ ನೆಲಕ್ಕುರುಳಿದುವು. ಆತನ ಬಾಣಗಳಿಂದ ಸತ್ತವರ ನೆತ್ತರು ಹೊಳೆಯಾಗಿ ಹರಿಯಿತು.

ಶ್ರೀಕೃಷ್ಣನು ಕಾಲಯಮನಂತೆ ಸೈನ್ಯವನ್ನು ಸಂಹರಿಸುತ್ತಿರಲು ಬಲರಾಮನು ಪ್ರಳಯ ಭೈರವನಂತೆ ತನ್ನ ಮುಸಲಾಯುಧದೊಡನೆ ಶತ್ರುಗಳನ್ನೆಲ್ಲ ಬಡಿದುಹಾಕಿದನು. ಆ ವೀರ ರಿಬ್ಬರ ಅಬ್ಬರವನ್ನು ಕಂಡು ಯಾದವ ಸೇನೆ ಹುರಿಗೊಂಡಿತು, ಶತ್ರುಸೇನೆ ತತ್ತರಿಸಿತು. ಅಪಾರವಾದ ತನ್ನ ಸೇನೆಗೆ ಒದಗಿದ ಈ ದುಸ್ಥಿತಿಯನ್ನು ಕಂಡು, ಜರಾಸಂಧನಿಗೆ ತಡೆಯ ಲಾರದಷ್ಟು ಕೋಪ ಬಂದಿತು. ಆತನು ಬಲರಾಮನ ಬಳಿಗೆ ತನ್ನ ರಥವನ್ನು ನುಗ್ಗಿಸಿ ಕೊಂಡು ಹೋದನು. ಆದರೆ ಬಲರಾಮನ ಕೈಲಿದ್ದ ಮುಸಲಾಯುಧದ ಹೊಡೆತಕ್ಕೆ ಅವನ ರಥ ಚೂರುಚೂರಾಯಿತು, ಅವನ ಸಾರಥಿ ಒಂದು ಮಾಂಸದ ಮುದ್ದೆಯಾದ. ತತ್ತರಿಸಿ ಹೋದ ಜರಾಸಂಧ ಎಚ್ಚರಗೊಳ್ಳುವಷ್ಟರಲ್ಲಿ ಬಲರಾಮನು ಸಿಂಹದಂತೆ ಅವನ ಮೇಲೆರಗಿ, ಅವನನ್ನು ಸೆರೆಹಿಡಿದನು. ಮರುಕ್ಷಣದಲ್ಲಿ ಆತನು ಜರಾಸಂಧನನ್ನು ಹಗ್ಗ ದಿಂದ ಬಿಗಿದು, ಶ್ರೀಕೃಷ್ಣನ ಬಳಿಗೆ ಎಳೆತಂದು ನಿಲ್ಲಿಸಿದನು. ಶ್ರೀಕೃಷ್ಣ ಮನಸ್ಸಿನಲ್ಲಿಯೆ ಲೆಕ್ಕ ಹಾಕಿದ–‘ಈಗಲೆ ಇವನನ್ನು ಕೊಂದರೆ ಭೂಭಾರವನ್ನು ಪೂರ್ತಿಯಾಗಿ ತಗ್ಗಿಸಿದಂ ತಾಗುವುದಿಲ್ಲ. ಈಗ ಇವನನ್ನು ಬಿಟ್ಟರೆ ಮತ್ತೆ ಮತ್ತೆ, ದಂಡೆತ್ತಿ ಬರುತ್ತಾನೆ. ಆಗ ಸೈನ್ಯವನ್ನೆಲ್ಲ ಕೊಂದುಹಾಕುತ್ತಾ ಹೋದರೆ ಭೂಭಾರ ತಗ್ಗುತ್ತದೆ’ ಹೀಗೆಂದುಕೊಂಡು, ಆತನು ‘ಅಣ್ಣ, ಈಗ ಸಧ್ಯಕ್ಕೆ ಇವನನ್ನು ಬಿಟ್ಟುಬಿಡು’ ಎಂದನು. ಅದರಂತೆಯೆ ಆತ ಜರಾಸಂಧನಿಗೆ ಜೀವದಾನವನ್ನು ನೀಡಿ ಅವನನ್ನು ಬಿಟ್ಟು ಬಿಟ್ಟನು.

ಬಲರಾಮಕೃಷ್ಣರಿಂದ ಸೋತು ಸೈನ್ಯವನ್ನೆಲ್ಲ ನೀಗಿಕೊಂಡ ಜರಾಸಂಧನಿಗೆ ಬಾಳು ಭಾರವೆನಿಸಿತು. ಆತನು ನಾಚಿಕೆಯಿಂದ ಯಾರಿಗೂ ಮುಖವನ್ನು ತೋರಿಸಲಾರದೆ, ತಪಸ್ಸು ಮಾಡಲೆಂದು ಅಡವಿಯ ಹಾದಿಯನ್ನು ಹಿಡಿದನು. ಇದನ್ನು ಕೇಳಿದ ಆತನ ಮಿತ್ರರಾಜರು ನಡುಹಾದಿಯಲ್ಲಿಯೆ ಆತನನ್ನು ತಡೆದು, ‘ಮಹಾರಾಜ, ಸೊಳ್ಳೆಗಳಂತಿರುವ ಈ ಯಾದವ ರೆಲ್ಲಿ, ನೀನೆಲ್ಲಿ? ಯಾವುದೋ ದುರದೃಷ್ಟದಿಂದ ನಿನಗೆ ಸೋಲಾದ ಮಾತ್ರಕ್ಕೆ ನೀನು ಇಷ್ಟು ಮನಸ್ಸಿಗೆ ಹಚ್ಚಿಕೊಳ್ಳುವುದೆ? ನಾವೆಲ್ಲ ನಿನ್ನ ಬೆಂಬಲಕ್ಕಿರುವಾಗ ನಿನಗೇನು ಯೋಚನೆ? ಬಾ, ಮತ್ತೊಮ್ಮೆ ಸೈನ್ಯದೊಡನೆ ಬಂದು ಈ ಮಧುರೆಯನ್ನು ನೆಲಸಮ ಮಾಡುವಿ ಯಂತೆ!’ ಎಂದು ಸಮಾಧಾನ ಮಾಡಿ ಆತನನ್ನು ಹಿಂದಕ್ಕೆ ಕಳುಹಿಸಿದರು. ಇತ್ತ ಬಲರಾಮ ಕೃಷ್ಣರು ತಮ್ಮಸೇನೆಯೊಡನೆ ಮಧುರಾಪುರಿಗೆ ಹಿಂದಿರುಗಿದರು. ಅಲ್ಲಿನ ನಾಗರಿಕರು ಪಟ್ಟಣವನ್ನು ತಳಿರು ತೋರಣಗಳಿಂದ ಸಿಂಗರಿಸಿ, ಮಂಗಳವಾದ್ಯಗಳೊಡನೆ ಅವರನ್ನು ಅರಮನೆಗೆ ಕರೆದೊಯ್ದರು. ಶ್ರೀಕೃಷ್ಣನು ಯುದ್ಧದಲ್ಲಿ ತಾನು ಸಂಪಾದಿಸಿ ತಂದಿದ್ದ ಐಶ್ವರ್ಯವನ್ನೆಲ್ಲ ಮಹಾರಾಜನಾದ ಉಗ್ರಸೇನನಿಗೆ ಒಪ್ಪಿಸಿ, ಮನ್ನಣೆಯನ್ನು ಪಡೆದನು.

ಯುದ್ಧದಲ್ಲಿ ಸೋತು ಹಿಂದಿರುಗಿದ ಜರಾಸಂಧನಿಗೆ ಹಗಲಿರುಳೂ ಅದೇ ಯೋಚನೆ ಯಾಯಿತು. ಆತನು ಮಹಾಶೂರರಿಂದ ಕೂಡಿದ ಬಹು ದೊಡ್ಡ ಸೇನೆಯನ್ನು ಮತ್ತೆ ಸಿದ್ಧ ಪಡಿಸಿಕೊಂಡನು; ತನ್ನ ಗೆಳೆಯರಾದ ರಾಜರುಗಳನ್ನೆಲ್ಲ ಯುದ್ಧಕ್ಕೆ ಸಿದ್ಧರಾಗುವಂತೆ ಬೇಡಿ ಕೊಂಡನು. ಎರಡು ತಿಂಗಳು ಕಳೆಯುವಷ್ಟರಲ್ಲಿ ಮಗಧಸೇನೆ ಮತ್ತೆ ಮಧುರೆಯನ್ನು ಮುತ್ತಿತು. ಆದರೆ ಶ್ರೀಕೃಷ್ಣನ ಪರಾಕ್ರಮಕ್ಕೆ ಅದೂ ಆಹುತಿಯಾಯಿತು. ಆದರೇನು? ಹಟ ಮಾರಿಯಾದ ಜರಾಸಂಧನು ಮತ್ತೆ ಮತ್ತೆ ದಂಡೆತ್ತಿ ಬರುತ್ತಲೇ ಇದ್ದನು. ಹದಿನೇಳು ಸಲ ಸೋತರೂ ಬಿಡದೆ ಆತ ಹದಿನೆಂಟನೆಯ ಸಲ ದಂಡೆತ್ತಿಬರಲು ಹವಣಿಸುತ್ತಿದ್ದನು. ಅಷ್ಟರಲ್ಲಿ ಕಾಲಯವನನೆಂಬ ಯವನರಾಜನು ಮಧುರೆಯ ಮೇಲೆ ದಂಡೆತ್ತಿ ಬಂದನು. ಅವನಿಗೆ ಸಮಾನನಾದ ಶೂರರೇ ಲೋಕದಲ್ಲಿ ಇಲ್ಲವೆಂಬ ಗರ್ವ. ಕೋತಿಗೆ ಹೆಂಡ ಕುಡಿಸಿ ದಂತೆ, ನಾರದರು ಆತನಿಗೆ ಶ್ರೀಕೃಷ್ಣನ ಸಮಾಚಾರವನ್ನು ತಿಳಿಸಿದರು. ‘ನನಗೆ ಸಮಾನ ಬಲನೆಂದರೆ ಅವನೇ ಸೈ’ ಎಂದುಕೊಂಡ ಆತ ಮೂರುಕೋಟಿ ಸೈನ್ಯದೊಡನೆ ಮಧುರೆಗೆ ಬಂದು ಮುತ್ತಿಗೆ ಹಾಕಿದ. ಇದೇ ಸಮಯದಲ್ಲಿ ಜರಾಸಂಧನೂ ದಂಡೆತ್ತಿ ಬರುತ್ತಿರುವ ನೆಂಬ ಸುದ್ದಿ ಬಂದಿತು. ಆಗ ಶ್ರೀಕೃಷ್ಣನು ಬಲರಾಮನೊಡನೆ ‘ಅಣ್ಣ, ಏಕಕಾಲದಲ್ಲಿ ಎರಡು ದೊಡ್ಡ ಸೈನ್ಯಗಳು ನಮ್ಮನ್ನು ಮುತ್ತಿದರೆ ನಮ್ಮ ಗತಿಯೇನು? ನಾವು ಕಾಲ ಯವನನೊಡನೆ ಯುದ್ಧಕ್ಕೆ ತೊಡಗುತ್ತಲೆ, ಅತ್ತ ಜರಾಸಂಧನು ಮಧುರಾಪುರಕ್ಕೆ ನುಗ್ಗಿ ಯಾದವರನ್ನೆಲ್ಲ ಧ್ವಂಸಮಾಡುವನು. ಆದ್ದರಿಂದ ನಾವು ಈಗ ಯುದ್ಧಕ್ಕೆ ತೊಡಗುವ ಮುಂಚೆ ಮಧುರೆಯಲ್ಲಿರುವ ಯಾದವರನ್ನೆಲ್ಲ ಸುರಕ್ಷಿತವಾದ ಸ್ಥಳಕ್ಕೆ ಸಾಗಿಸಬೇಕು’ ಎಂದನು. ಬಲರಾಮನು ಅದಕ್ಕೆ ಒಪ್ಪಿದನು.

ಬಲರಾಮಕೃಷ್ಣರು ತಮ್ಮವರ ರಕ್ಷಣೆಗಾಗಿ ದುರ್ಗಮವಾದ ಒಂದು ಜಲಮಾರ್ಗ ವನ್ನು ನಿರ್ಮಿಸಬೇಕೆಂದುಕೊಂಡರು. ಇದಕ್ಕಾಗಿ ಸಮುದ್ರ ಮಧ್ಯದಲ್ಲಿದ್ದ ಒಂದು ದ್ವೀಪ ದಲ್ಲಿ ಹನ್ನೆರಡು ಯೋಜನ ವಿಸ್ತಾರವಾದ ಒಂದು ಪಟ್ಟಣವನ್ನು ಕಟ್ಟಿಸಲು ಅವರು ನಿಶ್ಚಯಿಸಿದರು. ವಿಶ್ವಕರ್ಮನೆಂಬ ಶಿಲ್ಪಿ ಅವರ ಇಷ್ಟದಂತೆ ಆ ಊರನ್ನು ನಿರ್ಮಾಣ ಮಾಡಿದನು. ದೊಡ್ಡ ರಾಜಬೀದಿಗಳ ಇಕ್ಕೆಲದಲ್ಲಿಯೂ ಮುಗಿಲನ್ನು ಮುಟ್ಟುವಂತಹ ಮಹಡಿಯ ಮನೆಗಳು. ಒಂದೊಂದು ಮನೆಗೂ ಸ್ಫಟಿಕದ ಮಾಳಿಗೆ, ಚಿನ್ನದ ಕಲಶ; ಮನೆಯ ಮುಂದೆ ವಿಸ್ತಾರವಾದ ಅಂಗಳ, ಅಲ್ಲಲ್ಲಿಯೆ ದೇವರ ಗುಡಿಗಳು, ಅನ್ನ ಸತ್ರಗಳು, ಆನೆ ಕುದುರೆಗಳಿಗಾಗಿ ಲೋಹದಿಂದಲೆ ನಿರ್ಮಿತವಾದ ಲಾಯಗಳು, ಎಲ್ಲೆಲ್ಲಿ ನೋಡಿದರೂ ಸುಂದರವಾದ ಉದ್ಯಾನವನಗಳು. ದೇವೇಂದ್ರನು ಸುಧರ್ಮೆಯೆಂಬ ತನ್ನ ಸಭಾಸ್ಥಾನವನ್ನೂ ಪಾರಿಜಾತವೃಕ್ಷವನ್ನೂ ಶ್ರೀಕೃಷ್ಣನಿಗೆ ಕೈಗಾಣಿಕೆಯಾಗಿ ಅಲ್ಲಿಗೆ ಕಳುಹಿಸಿ ಕೊಟ್ಟನು. ವರುಣನು ಒಂದು ಕಿವಿ ಮಾತ್ರ ಕಪ್ಪಾಗಿರುವ ಒಂದು ಸಹಸ್ರ ಬಿಳಿಯ ಕುದುರೆಗಳನ್ನು ಆತನಿಗೊಪ್ಪಿಸಿದನು. ಕುಬೇರನು ನವನಿಧಿಗಳಿಂದ ಕೂಡಿದ ತನ್ನ ಬೊಕ್ಕಸವನ್ನು ಆತನಿಗೆ ಒಪ್ಪಿಸಿದನು. ಉಳಿದ ದೇವತೆಗಳೂ ತಮ್ಮ ಭಕ್ತಿ ಕಾಣಿಕೆಯನ್ನು ಭಗವಂತನಿಗೆ ಅರ್ಪಿಸಿದರು. ಶ್ರೀಕೃಷ್ಣನು ತನ್ನ ಯೋಗ ಶಕ್ತಿಯಿಂದ ಮಧುರೆಯಲ್ಲಿದ್ದ ಯಾದವರ ನ್ನೆಲ್ಲಾ ಅಲ್ಲಿಗೆ ಕರೆದೊಯ್ದನು. ಇದೇ ದ್ವಾರಕಾಪುರ. ತನ್ನವರೆಲ್ಲ ಸುಖವಾಗಿ ಅಲ್ಲಿ ನೆಲಸುವಂತೆ ಏರ್ಪಡಿಸಿದವನೆ ಶ್ರೀಕೃಷ್ಣನು ಮಧುರೆಗೆ ಹಿಂದಿರುಗಿದನು.

