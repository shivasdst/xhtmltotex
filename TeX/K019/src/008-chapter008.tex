
\chapter{೮. ವಿದುರ ಮೈತ್ರೇಯರ ಸಂವಾದ–ಕರ್ದಮ ದೇವಹೂತಿ}

ಅಯ್ಯಾ ವಿದುರ, ಈಗ ಸ್ವಾಯಂಭುವಮನುವಿನ ಮಕ್ಕಳಿಂದ ಈ ಭೂಲೋಕದ ಪ್ರಜಾವೃದ್ಧಿಯಾದ ಕಥೆಯನ್ನು ಮುಂದುವರಿಸುತ್ತೇನೆ ಕೇಳು. ಆ ಸ್ವಾಯಂಭುವಿನ ಪುತ್ರರಾದ ಪ್ರಿಯವ್ರತ, ಉತ್ತಾನಪಾದರು ಸಪ್ತದ್ವೀಪಗಳಿಂದ ಕೂಡಿದ ಈ ಭೂಮಂಡಲ ವನ್ನು ಧರ್ಮದಿಂದ ಪರಿಪಾಲಿಸುತ್ತಿದ್ದರು. ಸ್ವಾಯಂಭುವಿನ ಪುತ್ರಿಯರಲ್ಲಿ ಹಿರಿಯಳಾದ ದೇವಹೂತಿಯನ್ನು ಕರ್ದಮ ಪ್ರಜಾಪತಿಗೆ ವಿವಾಹಮಾಡಿಕೊಡಬೇಕೆಂದು ತಂದೆಯ ಅಪೇಕ್ಷೆ. ಆದರೆ ಕರ್ದಮನು ಬ್ರಹ್ಮನ ಆಜ್ಞೆಯಂತೆ ಪ್ರಜಾಸೃಷ್ಟಿಯ ಕಾರ್ಯವನ್ನು ಕೈ ಕೊಳ್ಳುವ ಮುನ್ನ ಸರಸ್ವತೀ ನದೀತೀರದಲ್ಲಿ ತಪಸ್ಸನ್ನು ಕೈಕೊಂಡಿದ್ದನು. ಆತನು ಹತ್ತು ಸಹಸ್ರ ವರ್ಷಗಳವರೆಗೆ ತಪವನ್ನಾಚರಿಸಿದ ಮೇಲೆ ಮಹಾವಿಷ್ಣು ಶಬ್ದಬ್ರಹ್ಮದ ರೂಪಿ ನಿಂದ ಆತನಿಗೆ ದರ್ಶನವಿತ್ತನು. ಆ ದಿವ್ಯದರ್ಶನದಿಂದ ಆನಂದಗೊಂಡ ಕರ್ದಮನು ಭಕ್ತಿಭರಿತನಾಗಿ ಆತನಿಗೆ ಅಡ್ಡ ಬಿದ್ದು “ಹೇ ದೇವದೇವ, ನಿನ್ನ ದರ್ಶನದಿಂದ ನನ್ನ ತಪಸ್ಸು ಸಿದ್ಧಿಸಿತು. ಕಣ್ಣು ಸಾರ್ಥಕವಾದುವು. ನಾನು ಧನ್ಯನಾದೆ. ಸ್ವಾಮಿ ನನ್ನ ತಂದೆಯಾದ ಬ್ರಹ್ಮದೇವನು ಸಂತಾನವನ್ನು ಪಡೆದು ಪ್ರಜಾವೃದ್ಧಿಯನ್ನು ನಡೆಸುವಂತೆ ನನಗೆ ಆಜ್ಞಾಪಿಸಿದ್ದಾನೆ. ಅದು ನಿನ್ನ ಅಪ್ಪಣೆಯೆಂದೇ ನನ್ನ ಭಾವನೆ. ನಾನು ಕಾಮಸುಖವನ್ನು ಬಯಸುತ್ತಿಲ್ಲವಾದರೂ ನಿನ್ನ ಅಪ್ಪಣೆಯನ್ನು ಪಾಲಿಸುವುದಕ್ಕಾಗಿ ಸಂಸಾರಿಯಾಗ ಬೇಕಾಗಿದೆ. ನಾನೀಗ ನಿನ್ನ ಮರೆಹೊಕ್ಕಿದ್ದೇನೆ; ನನ್ನ ಭಕ್ತಿಗೆ ಲೋಪವಾಗದಂತೆ, ನನ್ನ ಆಯುಸ್ಸು ವ್ಯರ್ಥವಾಗದಂತೆ ನನ್ನ ಇಷ್ಟಾರ್ಥವನ್ನು ಸಲ್ಲಿಸು” ಎಂದು ಬೇಡಿಕೊಂಡನು. ಭಗವಂತನು ತನ್ನ ತುಟಿಗಳಿಂದ ಮುಗುಳ್ನಗೆಯ ಬೆಳುದಿಂಗಳನ್ನು ಹೊರಚೆಲ್ಲುತ್ತಾ, “ಕರ್ದಮಾ, ನೀನು ಹೆಂಡತಿಯನ್ನು ಬಯಸಿ ತಪಸ್ಸು ಮಾಡಿದೆ. ನಾನು ನಿನ್ನ ತಪಸ್ಸಿಗೆ ಮೆಚ್ಚಿದ್ದೇನೆ. ನೋಡು, ನಿನಗೆ ತಕ್ಕವಳಾದ ಹೆಣ್ಣೊಬ್ಬಳು ನಿನಗಾಗಿ ಸಿದ್ಧವಾಗಿದ್ದಾಳೆ. ಸಪ್ತಸಮುದ್ರದ ಮಧ್ಯ ದಲ್ಲಿರುವ ಭೂಮಂಡಲದ ದೊರೆ ಸ್ವಾಯಂಭುವ, ಆತನ ರಾಣಿ ಶತರೂಪೆ, ಆ ದಂಪತಿಗಳ ಮಗಳಾದ ದೇವಹೂತಿ ಲೋಕೋತ್ತರಳಾದ ಸುಂದರಿ. ಅವಳಿ ಗೀಗ ಮದುವೆಯ ವಯಸ್ಸು. ಸ್ವಾಯಂಭುವನು ನಿನ್ನ ಗುಣಶೀಲಗಳನ್ನು ಕೇಳಿ ನಿನಗೆ ಅವಳನ್ನು ಧಾರೆಯೆರೆದು ಕೊಡಬೇಕೆಂದು ಬಯಸಿ, ತನ್ನ ಮಗಳೊಡನೆ ನಾಳೆ ನಿನ್ನ ಆಶ್ರಮಕ್ಕೆ ಬರುತ್ತಾನೆ. ನೀನು ಆ ಕನ್ಯೆಯನ್ನು ವಿವಾಹವಾಗು. ಅವಳು ನಿನ್ನನ್ನು ಕಾಮ ಸುಖದಿಂದ ತಣಿಸಿ ಒಂಬತ್ತು ಮಕ್ಕಳ ತಾಯಿಯಾಗುತ್ತಾಳೆ. ಆ ಒಂಬತ್ತೂ ಹೆಣ್ಣು ಮಕ್ಕಳೆ. ಅವರು ಮುಂದೆ ಮರೀಚಿಯೇ ಮೊದಲಾದ ಮಹರ್ಷಿಗಳನ್ನು ಮದುವೆಯಾಗಿ, ಅನೇಕ ಮಕ್ಕಳನ್ನು ಹೆತ್ತು, ಮಾನವಸಂತಾನವನ್ನು ಬೆಳಸುತ್ತಾರೆ. ನಾನೂ ಸಹ ನನ್ನ ಒಂದು ಅಂಶದಿಂದ ನಿನ್ನ ಮಗನಾಗಿ ಹುಟ್ಟಿ ಲೋಕಕಲ್ಯಾಣವನ್ನು ಮಾಡುತ್ತೇನೆ” ಎಂದು ಹೇಳಿ ಮಾಯವಾದನು.

ಭಗವಂತನು ಹೇಳಿದಂತೆ ಮರುದಿನ ಬೆಳಗ್ಗೆ ಸ್ವಾಯಂಭುವಮನು ಚಿನ್ನದ ರಥದಲ್ಲಿ ತನ್ನ ಮಡದಿಯನ್ನೂ ಮಗಳನ್ನೂ ಕೂರಿಸಿಕೊಂಡು ಕರ್ದಮನ ಆಶ್ರಮಕ್ಕೆ ಬಂದನು. ಆ ಪುಷ್ಯಾಶ್ರಮ ಬಹು ಸುಂದರವಾಗಿತ್ತು. ಕರ್ದಮನ ಭಕ್ತಿಯನ್ನು ಕಂಡ ಭಗವಂತ ಆನಂದಬಾಷ್ಪವನ್ನು ಸುರಿಸಲು, ಅದರ ಒಂದು ಬಿಂಬ ಆತನ ಆಶ್ರಮದಲ್ಲಿ ಬಿತ್ತಂತೆ! ಅದೇ ಅಲ್ಲಿದ್ದ ಬಿಂದುಸರೋವರ. ಅದರ ನೀರು ಅಮೃತದಂತೆ ಮಧುರ, ಗಂಗೆಯಂತೆ ಪಾಪಹರ. ಅನೇಕ ಪುಷಿಗಳು ತಮ್ಮ ನಿತ್ಯಕರ್ಮಕ್ಕೆ ಈ ಸರೋವರವನ್ನೇ ಆಶ್ರಯಿಸಿ ದ್ದರು. ಅದರ ಸುತ್ತಲೂ ಫಲಪುಷ್ಪಭರಿತವಾದ ಮರಗಿಡಬಳ್ಳಿಗಳು. ಇನಿದನಿಯಲ್ಲಿ ಹಾಡುವ ಕೋಗಿಲೆ, ಗರಿಗೆದರಿ ಕುಣಿವ ನವಿಲು, Ïು|0ಕರಿಸುವ ದುಂಬಿ, ಬಣ್ಣ ಬಣ್ಣದ ಹಕ್ಕಿಗಳು ಅಲ್ಲಿ ತುಂಬಿದ್ದವು. ಮೊಲ್ಲೆ ಮಲ್ಲಿಗೆ ಪಾರಿಜಾತಾದಿ ಪುಷ್ಪಗಳ ಸುವಾಸನೆ ಯಿಂದ ತುಂಬಿದ ಮಂದಾನಿಲವು ಇಡೀ ಆಶ್ರಮವನ್ನು ತುಂಬಿತ್ತು. ಸ್ವಾಯಂಭುವಮನು ಈ ಸೊಬಗನ್ನು ಸವಿಯುತ್ತಾ ಕರ್ದಮನ ಎಲೆಮನೆಯನ್ನು ಪ್ರವೇಶಿಸಿದನು.

ಕರ್ದಮಪ್ರಜಾಪತಿಯು ಆಗತಾನೆ ಹೋಮಕಾರ್ಯವನ್ನು ಮುಗಿಸಿಕೊಂಡು ಬಂದು ಸುಖಾಸೀನನಾಗಿದ್ದನು. ಆತನ ದೇಹ ತಪಸ್ಸಿನಿಂದ ಕೃಶವಾಗಿದ್ದರೂ ಮುಖದಲ್ಲಿ ದಿವ್ಯ ಕಾಂತಿ ಮಿಂಚುತ್ತಿತ್ತು. ಉನ್ನತವಾದ ಆಕೃತಿ, ಹೊಳೆಯುವ ಕಣ್ಣುಗಳು, ಎತ್ತಿಕಟ್ಟಿದ ಜಟೆ, ಉಟ್ಟ ನಾರುಮಡಿ–ಇವುಗಳಿಂದ ಆತನು ದೇಹಸಂಸ್ಕಾರವಿಲ್ಲದುದರಿಂದ ಸಾಣೆಯಿಕ್ಕದ ಮಹಾರತ್ನದಂತೆ ಕಂಗೊಳಿಸುತ್ತಿದ್ದನು. ಸ್ವಾಯಂಭುವಮನು ಆತನನ್ನು ಕಂಡು ಭಕ್ತಿ ಯಿಂದ ನಮಸ್ಕರಿಸಿದನು. ಕರ್ದಮನು ಆತನನ್ನು ಆಶೀರ್ವದಿಸಿ, ತನ್ನ ಆಶ್ರಮಕ್ಕೆ ಆಗಮಿಸಿದ್ದ ಆ ಅತಿಥಿಯನ್ನು ಉಚಿತ ರೀತಿಯಲ್ಲಿ ಸತ್ಕರಿಸಿ ಕುಳ್ಳಿರಿಸಿದ ಮೇಲೆ, ಭಗವಂತನ ಅಪ್ಪಣೆಯನ್ನು ಸ್ಮರಿಸಿಕೊಂಡು “ಅಯ್ಯಾ ಚಕ್ರವರ್ತಿ, ಪರಬ್ರಹ್ಮನ ರಕ್ಷಣ ಶಕ್ತಿಯೇ ನೀನು. ದುಷ್ಟಶಿಕ್ಷಣ ಶಿಷ್ಟರಕ್ಷಣೆಗಾಗಿ ಭೂಮಂಡಲವನ್ನೆಲ್ಲ ಸುತ್ತುತ್ತಿರುವ ನಿನ್ನಿಂದ ಧರ್ಮ ಸ್ಥಿರವಾಗಿ ನಿಂತಿದೆ. ಧರ್ಮರಕ್ಷಣೆಯಲ್ಲಿಯೇ ಸದಾ ನಿರತನಾಗಿರುವ ನೀನು ಈಗ ನಮ್ಮ ಬಳಿಗೆ ಬಂದಿರುವುದಾದರೂ ಯಾವುದೋ ಒಂದು ಲೋಕಕ್ಷೇಮ ಕರವಾದ ಕಾರ್ಯಕ್ಕೇ ಇರಬೇಕು. ನಿನ್ನ ಉದ್ದೇಶವೇನೆಂಬುದನ್ನು ತಿಳಿಸಿದರೆ ನಾನು ಅಗತ್ಯ ವಾಗಿಯೂ ಅದನ್ನು ನಿರ್ವಂಚನೆಯಿಂದ ನಡೆಸಿಕೊಡುತ್ತೇನೆ” ಎಂದನು.

ಕರ್ದಮನ ಮಾತುಗಳಿಂದ ಸಂತಸಗೊಂಡ ಸ್ವಾಯಂಭುವನು “ಹೇ ಮುನೀಂದ್ರ, ಪ್ರಜೆಗಳನ್ನು ರಕ್ಷಿಸುವುದಕ್ಕಾಗಿ ಚತುರ್ಮುಖ ಬ್ರಹ್ಮನು ನಮ್ಮನ್ನು ಸೃಷ್ಟಿಸಿದುದು ನಿಜ. ಆದರೆ ಜ್ಞಾನ, ಕರ್ಮ, ತಪಸ್ಸುಗಳಲ್ಲಿ ನಿರತರಾಗಿರುವ ನಿಮ್ಮನ್ನಾದರೊ ವೇದಸ್ವರೂಪಿಯಾದ ತನ್ನನ್ನು ರಕ್ಷಿಸುವುದಕ್ಕಾಗಿಯೇ ಸೃಷ್ಟಿಸಿದ್ದಾನೆ. ಆದ್ದರಿಂದ ನಾವು ನಮಗೆ ವಿಹಿತವಾಗಿರುವ ಕಾರ್ಯಗಳನ್ನು ನಡೆಸುವುದರಿಂದ ನಾವು ಭಗವಂತನ ಕೃಪೆಗೆ ಪಾತ್ರರಾಗುವುದು. ಮುನೀಂದ್ರ, ಮಹಾತಪಸ್ವಿಯಾದ ನಿನ್ನ ದರ್ಶನವೇ ಪುಣ್ಯಕರ. ಮಧುರವಾದ ನಿನ್ನ ಮಾತುಗಳಿಂದ ನಾನು ಧನ್ಯನಾದೆ. ಈಗ ನಾನು ನಿನ್ನ ಬಳಿಗೆ ಬಂದಿರು ವುದು ಒಂದು ಕಾರ್ಯೋದ್ದೇಶದಿಂದಲೇ. ಇಗೋ, ಇವಳು ನನ್ನ ಮಮತೆಯ ಮಗಳು. ಇವಳ ಹೆಸರು ದೇವಹೂತಿ. ಇವಳಿಗೆ ತಕ್ಕ ವರನಿಗಾಗಿ ನನ್ನ ಮನಸ್ಸು ಕಳವಳಗೊಂಡಿದೆ. ತನ್ನ ರೂಪಗುಣ ಶೀಲಗಳಿಗೆ ತಕ್ಕ ಗಂಡನನ್ನು ಆಕೆ ಬಯಸುತ್ತಿದ್ದಾಳೆ. ಈಗ ಕೆಲವು ದಿನ ಗಳ ಕೆಳಗೆ ಈಕೆ ದೇವಪುಷಿ ನಾರದರಿಂದ ನಿನ್ನ ವಿದ್ಯೆ, ರೂಪು, ಗುಣ, ಸದಾಚಾರಗಳನ್ನು ಕೇಳಿ ಮನಸಾ ನಿನ್ನನ್ನು ವರಿಸಿದ್ದಾಳೆ. ಆದ್ದರಿಂದ ಈಕೆಯನ್ನು ನಿನ್ನ ಬಳಿಗೆ ಕರೆತಂದಿದ್ದೇನೆ. ಈಕೆ ನಿನ್ನ ಗೃಹಸ್ಥಧರ್ಮದಲ್ಲಿ ಸರ್ವಾನುಕೂಲೆಯಾಗಿರುವಳು. ನೀನೂ ಮದುವೆ ಮಾಡಿಕೊಳ್ಳಬೇಕೆಂಬ ಪ್ರಯತ್ನದಲ್ಲಿರುವುದಾಗಿ ಕೇಳಿದೆ. ಇಗೋ ನಿನಗೆ ಅನುರೂಪಳಾ ಗಿರುವ ಈ ಕನ್ಯೆಯನ್ನು ಕೈಹಿಡಿದು ನನ್ನನ್ನು ಸಂತೋಷಗೊಳಿಸು,” ಎಂದು ಬೇಡಿ ಕೊಂಡನು. 

ಕರ್ದಮಪ್ರಜಾಪತಿಯು ಸ್ವಾಯಂಭುವನ ಪ್ರಾರ್ಥನೆಯನ್ನು ಸಂತೋಷದಿಂದ ಒಪ್ಪಿ ಕೊಳ್ಳುತ್ತಾ, “ಹೇ ಚಕ್ರವರ್ತಿ, ನಿನ್ನ ಮಗಳನ್ನು ಮದುವೆಯಾಗಲು ಅಂತಹ ಒತ್ತಾಯ ವೇನೂ ಅಗತ್ಯವಿಲ್ಲ. ಅಲಂಕಾರಗಳಿಗೆ ಅಲಂಕಾರದಂತಿರುವ ಈ ಸಹಜ ಸುಂದರಿಯನ್ನು ಕಂಡು ಯಾರು ತಾನೆ ಮೋಹಗೊಳ್ಳುವುದಿಲ್ಲ? ಒಮ್ಮೆ ಈಕೆ ಅರಮನೆಯ ಉಪ್ಪರಿಗೆಯ ಮೇಲೆ ಸಖೀಜನರೊಡನೆ ಚೆಂಡಾಟದಲ್ಲಿ ತೊಡಗಿದ್ದಾಗ, ಆಕಾಶದಲ್ಲಿ ಸಂಚರಿಸುತ್ತಿದ್ದ ವಿಶ್ವಾವಸುವೆಂಬ ಗಂಧರ್ವರಾಜನು ಆಕೆಯ ಕಡೆಗಣ್ ನೋಟದ ಲಾವಣ್ಯಕ್ಕೆ ಮೋಹ ಗೊಂಡು ಮೂರ್ಛೆ ಹೋದನಂತೆ! ಚಕ್ರವರ್ತಿಯ ಮಗಳಾಗಿರುವ ಈ ಸುಂದರಾಂಗಿ ತಾನಾಗಿಯೇ ಒಲಿದು ಬಂದಿರುವಾಗ ಬೇಡವೆಂದು ಹೇಳುವಷ್ಟು ಅರಸಿಕನೇನೂ ನಾನಲ್ಲ. ನಾನು ಅತ್ಯಂತ ಆನಂದದಿಂದ ಈಕೆಯನ್ನು ವರಿಸುತ್ತೇನೆ. ಆದರೆ ನನ್ನದೊಂದು ನಿಬಂಧನೆಯುಂಟು. ನಾನು ಕೇವಲ ಸಂತಾನಾಪೇಕ್ಷೆಯಿಂದ ವಿವಾಹವಾಗುವವನೇ ಹೊರತು ಕಾಮಿಯಾಗಿ ಅಲ್ಲ. ಆದ್ದರಿಂದ ಈಕೆಯಲ್ಲಿ ಮಕ್ಕಳಾಗುವವರೆಗೆ ಮಾತ್ರ ನಾನು ಗೃಹಸ್ಥಧರ್ಮವನ್ನು ಸ್ವೀಕರಿಸಿದ್ದು, ಆಮೇಲೆ ಸಂನ್ಯಾಸಿಯಾಗತಕ್ಕವನು. ಇದಕ್ಕೆ ಒಪ್ಪುವು ದಾದರೆ ಅಗತ್ಯವಾಗಿಯೂ ವಿವಾಹ ನೆರವೇರಲಿ” ಎಂದನು. ಆತನ ಮಾತುಗಳನ್ನು ಕೇಳುತ್ತಿದ್ದ ದೇವಹೂತಿಗೆ ಆನಂದವಾಯಿತು. ಆತನ ಮೇಲಿನ ಅನುರಾಗ ಸ್ಥಿರವಾಯಿತು. ಆಕೆಯ ಇಂಗಿತವನ್ನು ಅರ್ಥ ಮಾಡಿಕೊಂಡ ಸ್ವಾಯಂಭುವನು, ಸಂತುಷ್ಟನಾಗಿ, ಸೃಷ್ಟಿ ಯಲ್ಲಿಯೇ ಮೊಟ್ಟಮೊದಲನೆಯದಾದ ವೇದೋಕ್ತವಿಧಿಯ ವಿವಾಹವನ್ನು ನೆರವೇರಿಸಿ ದನು. ರಾಜರಾಣಿಯರು ಮಗಳು ಅಳಿಯನಿಗೆ ಬೇಕಾದಷ್ಟು ಬಳುವಳಿಗಳನ್ನಿತ್ತು ಸಂತೋಷಪಡಿಸಿದರು. ಮಗಳಿಗೆ ತಕ್ಕ ವರನು ದೊರೆತುದಕ್ಕಾಗಿ ಚಕ್ರವರ್ತಿಗೆ ಪರಮ ಸಂತೋಷವಾಯಿತಾದರೂ ಆಕೆಯನ್ನು ಅಗಲುವಾಗ ಕಣ್ಣಿನಲ್ಲಿ ನೀರು ತುಂಬಿತು. ಆತನು ಮಗಳನ್ನು ಅಪ್ಪಿಕೊಂಡು ಗದ್ಗದಸ್ವರದಿಂದ ‘ಅಮ್ಮಾ, ಮಗು’ ಎಂದು ಮಾತ್ರ ಹೇಳಿ ಮುಂದೆ ಮಾತನಾಡಲಾರದೆ ತನ್ನ ಕಣ್ಣೀರಿನಿಂದ ಆಕೆಯ ನೆತ್ತಿಯನ್ನು ತೋಯಿಸಿದನು. ಕೆಲಕಾಲವಾದಮೇಲೆ ಆತನು ಮನಸ್ಸನ್ನು ಗಟ್ಟಿಮಾಡಿಕೊಂಡು, ಮಗಳು ಅಳಿಯನಿಂದ ಬೀಳ್ಕೊಂಡು, ತನ್ನ ಪತ್ನಿಯೊಡನೆ ರಾಜಧಾನಿಗೆ ಹಿಂದಿರುಗಿದನು.

ನೂತನ ದಂಪತಿಗಳಾದ ಕರ್ದಮ ದೇವಹೂತಿಯರು ಪರಸ್ಪರ ಅನುರಾಗದಿಂದ ಸುಖ ಭೋಗಗಳನ್ನು ಸೂರೆಗೊಳ್ಳುತ್ತಿದ್ದರು. ದೇವಹೂತಿ ತುಂಬ ಜಾಣೆ; ಗಂಡನ ಅಭಿಪ್ರಾಯ ವನ್ನು ಆತನ ಮುಖಭಾವದಿಂದಲೇ ಅರಿತುಕೊಂಡು ಆತನ ಸೇವೆಯಲ್ಲಿ ತಲ್ಲೀನಳಾಗಿರು ವಳು. ಆಕೆಯ ಮಾತು ಮೃದು ಮಧುರ. ಗಂಡನೇ ದೇವರೆಂಬ ಆಕೆಯ ಭಾವ ಹೆಜ್ಜೆ ಹೆಜ್ಜೆಗೂ ಒಡೆದು ಕಾಣುತ್ತಿತ್ತು. ಚಕ್ರವರ್ತಿಯ ಮೋಹದ ಮಗಳಾಗಿ ಸುಕುಮಾರಿಯಾಗಿ ದ್ದರೂ ದಾಸಿಯಂತೆ ದುಡಿಯುತ್ತಿರುವ ಆಕೆಯನ್ನು ಕಂಡು ಕರ್ದಮನಿಗೆ ಎಲ್ಲೆಯಿಲ್ಲದ ಆದರ, ಗೌರವ, ಪ್ರೇಮ. ಆತನು ಮಡದಿಯೊಡನೆ “ದೇವಿ, ನನ್ನ ಸೇವೆಯಲ್ಲಿ ನಿನ್ನ ದೇಹ ಕಂದಿ ಕುಂದಿದುದನ್ನು ನೀನು ಗಮನಿಸುತ್ತಿಲ್ಲವಲ್ಲಾ; ನಿನ್ನಂತಹ ಮಡದಿಯನ್ನು ಪಡೆದ ನಾನೆಷ್ಟು ಧನ್ಯ; ರಮಣಿ, ನಾನು ತಪಸ್ಸಿನಿಂದ ಭಗವಂತನ ಅನುಗ್ರಹಕ್ಕೆ ಪಾತ್ರನಾಗಿದ್ದೇನೆ. ಅನೇಕ ದಿವ್ಯಶಕ್ತಿಗಳು ನನ್ನ ವಶವರ್ತಿಯಾಗಿವೆ. ಎರಡರಿಯದ ಪತಿಭಕ್ತಿಯಿಂದ ಈಗ ಅವುಗಳನ್ನೆಲ್ಲ ನೀನೂ ಪಡೆದಿರುವೆ. ಇಗೋ, ನಿನಗೀಗ ದಿವ್ಯದೃಷ್ಟಿಯನ್ನು ಕೊಡುತ್ತೇನೆ, ನಿತ್ಯಸುಖಗಳಿಂದ ಕೂಡಿದ ಎಂತಹ ಭೋಗ ಭಾಗ್ಯಗಳು ನಮಗಾಗಿ ಕಾದಿವೆಯೋ ನೋಡು. ಇಂತಹ ದಿವ್ಯಶಕ್ತಿ ನಮ್ಮಲ್ಲಿರುವಾಗ ಅಲ್ಪವಾದ ಸುಖಕ್ಕೇಕೆ ನಾಲಿಗೆಯನ್ನು ಚಪ್ಪರಿಸಬೇಕು? ಚಕ್ರವರ್ತಿಗಳಿಗೂ ಎಟುಕಲಾರದ ಈ ಭೋಗಭಾಗ್ಯಗಳು ನಿನ್ನ ಪಾತಿ ವ್ರತ್ಯದ ಪ್ರಭಾವದಿಂದ ನಿನಗೆ ಲಭಿಸಿವೆ. ಇವುಗಳನ್ನು ತೃಪ್ತಿಯಾಗುವಂತೆ ಅನುಭವಿಸು” ಎಂದನು. 

ಗಂಡನ ಯೋಗಶಕ್ತಿ ತಪಃಪ್ರಭಾವಗಳಿಗೆ ಬೆರಗಾದ ದೇವಹೂತಿ, ತನ್ನ ಮನದ ಅಪೇಕ್ಷೆಯನ್ನು ಆತನಲ್ಲಿ ಹೇಳಿಕೊಳ್ಳಲು ನಸು ನಾಚಿದಳು. ಆದರೆ ಆತನಲ್ಲಿ ಹೇಳಿಕೊಳ್ಳದೆ ಅದು ನೆರವೇರುವುದೆಂತು? ಆದ್ದರಿಂದ ಆಕೆ ಕಣ್ಣಲ್ಲಿ ಕುಡಿನೋಟವನ್ನೂ, ತುಟಿಯಲ್ಲಿ ಮಂದಹಾಸವನ್ನೂ ತನ್ನ ನೆರವಿಗೆ ತೆಗೆದುಕೊಂಡು “ಸ್ವಾಮಿ, ಯೋಗಶಕ್ತಿಯ ಮಹಾ ಮಹಿಮರಾದ ನೀವು ಪತಿಯಾಗಿರುವಾಗ ನನಗೆ ದುರ್ಲಭವಾದ ಕೋರಿಕೆಯೇನುಂಟು? ನೀವು ಕೊಟ್ಟ ಮಾತಲ್ಲದೆ ನನಗೆ ಬೇರೆ ಕೋರಿಕೆಯೂ ಇಲ್ಲ. ಪತಿಯಿಂದ ಸಂತಾನವನ್ನು ಪಡೆಯುವುದಕ್ಕಿಂತಲೂ ಹೆಚ್ಚಿನ ಸೌಭಾಗ್ಯ ತಾನೆ ಹೆಣ್ಣುಗಳಿಗೆ ಏನಿದೆ? ಸುಂದರಾಂಗರಾದ ನಿಮ್ಮ ಅಂಗಸಂಗದಿಂದ ನಾನು ಧನ್ಯಳಾಗಬೇಕೆಂಬುದು ನನ್ನ ಬಯಕೆ. ಆದ್ದರಿಂದ ನಮ್ಮಿಬ್ಬರ ಕಾಮಕೇಳಿಗೆ ಅಗತ್ಯವಾದ ಭೋಗಮಂದಿರವನ್ನೂ ಭೋಗಸಾಮಗ್ರಿಗಳನ್ನೂ ಸೃಷ್ಟಿಮಾಡಬೇಕೆಂದು ಬೇಡುತ್ತೇನೆ” ಎಂದಳು.

ಮಡದಿಯ ಪ್ರಾರ್ಥನೆಯಂತೆ ಕರ್ದಮನು ತನ್ನ ಯೋಗಶಕ್ತಿಯಿಂದ ಆಕಾಶಮಾರ್ಗ ದಲ್ಲಿ ಮನಬಂದಂತೆ ಸಂಚರಿಸಬಲ್ಲ ವಿಮಾನವೊಂದನ್ನು ನಿರ್ಮಿಸಿದನು. ಆ ವಿಮಾನ ದಲ್ಲಿ ಎಲ್ಲಿ ನೋಡಿದರೂ ಭೋಗ ಸಾಮಗ್ರಿಗಳೇ. ಭ್ರಮರÏು|0ಕಾರದೊಡನೆ ತೊನೆದು ತೂಗುತ್ತಿರುವ ಹೂಮಾಲೆಗಳು; ಬಗೆಬಗೆಯ ಬಣ್ಣದ ಚಿತ್ರವಿಚಿತ್ರವಾದ ರೇಷ್ಮೆಯ ಬಟ್ಟೆಗಳು; ದಿವ್ಯ ಸುಂದರವಾಗಿ ಅಲಂಕರಿಸಿರುವ ಮಲಗುವ ಮನೆಗಳು; ಅವುಗಳಲ್ಲಿ ಚಿನ್ನದ ಮಂಚಗಳು, ಅವುಗಳ ಮೇಲೆ ಮೆತ್ತನೆಯ ಹಂಸತೂಲಿಕಾತಲ್ಪಗಳು, ಸುತ್ತಲೂ ಸುಂದರವಾದ ಆಸನಗಳು, ತಣ್ಣನೆಯ ಗಾಳಿಯನ್ನು ಒದಗಿಸುತ್ತಿರುವ ಬೀಸಣಿಗೆಗಳು, ಮನೋಹರವಾದ ಚಿತ್ರಪಟಗಳು, ಬಗೆಬಗೆಯ ಮೃಗಚರ್ಮಗಳು, ಬಗೆಬಗೆಯ ಭಂಗಿಯ ಪ್ರತಿಮೆಗಳು. ಅಲ್ಲಿನ ಐಶ್ವರ್ಯವಂತೂ ವರ್ಣನಾತೀತ. ನೆಲೆಗಟ್ಟೆಲ್ಲ ಮರಕತ ರತ್ನ, ಹವಳದ ಜಗುಲಿಗಳು, ರತ್ನದ ಬಾಗಿಲುಗಳು, ವಜ್ರದ ಗೋಡೆಯಲ್ಲಿ ಪದ್ಮರಾಗದ ಚಿತ್ರಕಲೆ, ನೀಲಮಣಿಯ ಗೋಪುರದಮೇಲೆ ಚಿನ್ನದ ಕಲಶ; ಒಂದು ಕಡೆ ಕ್ರೀಡೋದ್ಯಾನ; ಮತ್ತೊಂದು ಕಡೆ ವಿಶ್ರಾಂತಿಗೃಹ–ಹೆಚ್ಚೇನು, ಮನಸ್ಸು ಬಯಸಿದುದೆಲ್ಲ ಅಲ್ಲಿ ಆ ಕ್ಷಣಕ್ಕೆ ಸಿದ್ಧವಾಗುತ್ತಿತ್ತು. ಇದನ್ನು ಕಂಡು ಅದರ ಸೃಷ್ಟಿಕರ್ತನಾದ ಕರ್ದಮನಿಗೇ ಆಶ್ಚರ್ಯವಾಯಿತು. ಅನಂತರ ಆತನು ತನ್ನ ಮಡದಿಯನ್ನು ಬಿಂದು ಸರೋವರದಲ್ಲಿ ಮಿಂದು ವಿಮಾನವನ್ನು ಏರುವಂತೆ ತಿಳಿಸಿದನು.

ಗಂಡನ ಅಪ್ಪಣೆಯಂತೆ ದೇವಹೂತಿ ಬಿಂದು ಸರೋವರಕ್ಕೆ ಇಳಿದು ಮುಳುಗು ಹಾಕಿ ದಳು. ಆದರೆ ಇದೇನಾಶ್ಚರ್ಯ; ಜಲಮಧ್ಯದಲ್ಲಿ ಒಂದು ಸುಂದರವಾದ ಅರಮನೆ; ಅದರಲ್ಲಿ ಸುಂದರಿಯರಾದ ಸಾವಿರ ಕನ್ಯೆಯರು; ಅವರು ದೇವಹೂತಿಯ ಮುಂದೆ ಕೈಮುಗಿದು ನಿಂತು “ಅಮ್ಮ, ನಾವು ನಿನ್ನ ಸೇವಕಿಯರು, ಏನಪ್ಪಣೆ?” ಎಂದು ಹೇಳಿ, ಆಕೆಯನ್ನು ಅರಮನೆಗೆ ಕರೆದೊಯ್ದು ಅಭ್ಯಂಜನ ಸ್ನಾನವನ್ನು ಮಾಡಿಸಿ, ದಿವ್ಯ ವಸ್ತ್ರಾಭರಣಗಳಿಂದ ಅಲಂಕರಿಸಿದರು. ಅವರು ಹಿಡಿದ ಕನ್ನಡಿಯಲ್ಲಿ ತನ್ನ ಪ್ರತಿಬಿಂಬವನ್ನು ನೋಡಿಕೊಂಡ ದೇವಹೂತಿ, ತನ್ನ ಸೌಂದರ್ಯಕ್ಕೆ ತಾನೇ ಆಶ್ಚರ್ಯದಿಂದ ಹಿಗ್ಗುತ್ತಾ ತನ್ನ ಗಂಡನು ಈ ಸೌಂದರ್ಯವನ್ನು ಕಾಣಬೇಕೆಂದು ಬಯಸಿದಳು. ಮನಸ್ಸಿನಲ್ಲಿ ಆ ಭಾವನೆ ಮಿಂಚಿದುದೇ ತಡ, ಆಕೆಯು ಆ ಕನ್ನೆಯರೊಡನೆ ಕರ್ದಮನ ಇದಿರಿನಲ್ಲಿ ನಿಂತಿದ್ದಳು. ಗಂಡ ಹೆಂಡಿರು ಕೈಹಿಡಿದು ವಿಮಾನವನ್ನೇರಿದರು. ಸೇವಕಿಯರೂ ಅವರನ್ನು ಹಿಂಬಾಲಿಸಿದರು. ಎಲ್ಲರೂ ಏರುತ್ತಲೇ ವಿಮಾನವು ಚಲಿಸಿತು. ಕರ್ದಮನು ಅಷ್ಟದಿಕ್ಪಾಲಕರಿಗೆ ಕ್ರೀಡಾಸ್ಥಾನವಾಗಿರುವ ಮೇರು ಪರ್ವತದ ಗುಹೆಗಳಲ್ಲಿ, ಮಾನಸ ಸರೋವರದಲ್ಲಿ, ದೇವೋದ್ಯಾನಗಳಲ್ಲಿ ಮಡದಿ ಯೊಡನೆ ವಿಹರಿಸಿದನು. ಆತನ ಅಪೇಕ್ಷೆಯಂತೆ ವಿಮಾನವು ಎಲ್ಲ ಲೋಕಗಳಲ್ಲಿಯೂ ಸಂಚರಿಸಿತು. ಅಲ್ಲಲ್ಲಿನ ಎಲ್ಲ ರಮಣೀಯ ಸ್ಥಳಗಳನ್ನೂ ಆತನು ತನ್ನ ಮಡದಿಗೆ ತೋರಿ ಸಿದನು. ಜಗತ್ತಿನ ಎಲ್ಲ ಆಶ್ಚರ್ಯಗಳನ್ನೂ ಆತನು ಆಕೆಗೆ ಪರಿಚಯ ಮಾಡಿಕೊಟ್ಟನು. ಅನೇಕ ವರ್ಷಗಳು ಆ ದಂಪತಿಗಳಿಗೆ ಒಂದು ಕ್ಷಣದಂತೆ ಕಳೆದು ಹೋಯಿತು. ಇಷ್ಟರಲ್ಲಿ ದೇವಹೂತಿಯ ಬಯಕೆ ಸಿದ್ಧಿಸಿತು. ಆಕೆಯು ಸುಂದರಿಯರಾದ ಒಂಬತ್ತು ಜನ ಹೆಣ್ಣು ಮಕ್ಕಳ ತಾಯಿಯಾದಳು.

ಸಂತಾನಕ್ಕಾಗಿ ಸಂಸಾರಿಯಾಗಿದ್ದ ಕರ್ದಮನು, ಮದುವೆಯ ಮುನ್ನವೇ ತಿಳಿಸಿದ್ದಂತೆ, ಇನ್ನು ಸಂನ್ಯಾಸವನ್ನು ಸ್ವೀಕರಿಸುವನೆಂದು ದೇವಹೂತಿಗೆ ಗೊತ್ತು. ಆಕೆಯು ಆತನ ಅಗಲಿಕೆಯನ್ನು ತಾಳಲಾರದೆ ಒಳಗೊಳಗೆ ಕೊರಗುತ್ತಿದ್ದಳು. ಒಮ್ಮೆ ಆಕೆ ತನ್ನ ಸಂಕಟ ವನ್ನು ನುಂಗಿಕೊಂಡು, ಆತನ ಸಮೀಪದಲ್ಲಿ ತಲೆಬಾಗಿ ನಿಂತು ನೆಲವನ್ನು ಕಾಲಿನಿಂದ ಕೆರೆಯುತ್ತಾ ಮೃದುಮಧುರವಾದ ದನಿಯಿಂದ “ಸ್ವಾಮಿ, ನೀವು ಮಾತು ಕೊಟ್ಟಂತೆ ನನಗೆ ಸಂತಾನವನ್ನು ಕರುಣಿಸಿದಿರಿ. ನಿಮಗೆ ಶರಣಾಗತಳಾಗಿರುವ ನನ್ನ ಇನ್ನೊಂದು ಕೋರಿಕೆ ಯನ್ನೂ ನಡೆಸಿಕೊಡಬೇಕು. ನಿಮ್ಮ ಪುತ್ರಿಯರು ಇನ್ನೂ ಎಳೆಯ ಕೂಸುಗಳು. ಇವರು ಬೆಳೆದು ವಯಸ್ಸಿಗೆ ಬಂದಾಗ ಅನುರೂಪರಾದ ಪತಿಗಳನ್ನು ಹುಡುಕಿ ವಿವಾಹ ಮಾಡು ವುದು ನಿಮ್ಮ ಹೊಣೆಯಲ್ಲವೆ? ನೀವು ಈಗಲೇ ವಿರಕ್ತರಾಗಿ ಸಂಸಾರವನ್ನು ತ್ಯಜಿಸಿದರೆ ನಾನು ತುಂಬ ದುಃಖಕ್ಕೆ ಒಳಗಾಗುತ್ತೇನೆ. ಇಷ್ಟೇ ಅಲ್ಲ, ವಿಚಾರಿಸಬೇಕಾದ ಮತ್ತೊಂದು ವಿಷಯವೂ ಉಂಟು. ಜ್ಞಾನನಿಧಿಯಾದ ನಿಮ್ಮಲ್ಲಿ ನಾನು ಬಯಸಿದುದು ಕೇವಲ ಭೋಗ ಭಾಗ್ಯವನ್ನು, ಆತ್ಮಜ್ಞಾನವನ್ನಲ್ಲ. ತಿಳಿದೋ ತಿಳಿಯದೆಯೋ ಮಹಾನುಭಾವರ ಸಹ ವಾಸವು ಸದ್ಗತಿಗೆ ಕಾರಣವಾಗಲೇ ಬೇಕಲ್ಲವೆ? ಆದ್ದರಿಂದ ನಿಮ್ಮ ಸಹವಾಸವನ್ನು ಮಾಡಿದ ನಾನು ಉದ್ಧಾರವಾಗಿಯೇ ಆಗುತ್ತೇನೆ. ಹಾಗೆ ಉದ್ಧಾರವಾಗುವ ಮಾರ್ಗವನ್ನು ಬೋಧಿಸ ಬಲ್ಲ ಒಬ್ಬ ಸತ್ಪುತ್ರನನ್ನು ನನಗೆ ಕರುಣಿಸಿ. ನನಗೆ ಇನ್ನಾವ ಆಸೆಯೂ ಇಲ್ಲ” ಎಂದು ಬೇಡಿಕೊಂಡಳು.

ಮಡದಿಯ ನುಡಿಗಳನ್ನು ಕೇಳಿ ಕರ್ದಮನ ಮನಸ್ಸು ಕರಗಿತು. ತನಗೆ ಪ್ರತ್ಯಕ್ಷ ವಾದಂದು ಮಹಾವಿಷ್ಣುವು ತಾನೆ ತನ್ನ ಅಂಶದಿಂದ ಮಗನಾಗಿ ಜನಿಸುವೆನೆಂದು ಹೇಳಿ ದ್ದುದೂ ಆತನಿಗೆ ಈಗ ಜ್ಞಾಪಕವಾಯಿತು. ಆತನು ಆಕೆಯನ್ನು ಕುರಿತು, “ಸತೀಮಣಿ, ನೀನೇನೂ ಚಿಂತಿಸಬೇಡ. ಭಗವಂತನೇ ನಿನ್ನ ಮಗನಾಗಿ ಜನಿಸುತ್ತಾನೆ. ಇನ್ನು ಮುಂದೆ ನೀನು ಮನಸ್ಸನ್ನೂ, ಇಂದ್ರಿಯಗಳನ್ನೂ ನಿಗ್ರಹಿಸಿ ಭಗವಂತನನ್ನು ಆರಾಧಿಸು. ದಾನಧರ್ಮಗಳನ್ನು ಮಾಡುತ್ತಾ ಪಾತಿವ್ರತ್ಯದ ತಪಸ್ಸನ್ನಾಚರಿಸು. ಇದರಿಂದ ಸುಪ್ರೀತ ನಾದ ಭಗವಂತನು ನಿನ್ನ ಹೊಟ್ಟೆಯಲ್ಲಿ ಹುಟ್ಟಿ, ನನ್ನ ಕೀರ್ತಿಯನ್ನು ಹೆಚ್ಚಿಸುತ್ತಾನೆ, ನಿನ್ನ ಅಜ್ಞಾನವನ್ನೂ ಹೋಗಲಾಡಿಸುತ್ತಾನೆ” ಎಂದು ಹೇಳಿದನು. ಇದನ್ನು ಕೇಳಿ ದೇವಹೂತಿ ಅತ್ಯಂತ ಸಂತೋಷದಿಂದ, ಪತಿಯ ಉಪದೇಶದಂತೆ ದೇವದೇವನನ್ನು ಭಕ್ತಿಯಿಂದ ಆರಾಧಿಸುತ್ತಿದ್ದಳು. ಹೀಗೆ ಕೆಲವು ಕಾಲವಾದಮೇಲೆ ಆಕೆ ಗರ್ಭವತಿಯಾಗಿ ಗಂಡುಮಗ ನನ್ನು ಹೆತ್ತಳು. ಒಡನೆಯೇ ದೇವದುಂದುಭಿಗಳು ಮೊಳಗಿದವು, ಪುಷ್ಪವೃಷ್ಟಿಯಾಯಿತು, ಅಪ್ಸರೆಯರು ನರ್ತನ ಮಾಡಿದರು, ದಶದಿಕ್ಕುಗಳೂ ಶುಭ್ರವಾಗಿ ಬೆಳಗಿದವು. ಸತ್ಯಲೋಕ ದಿಂದ ಚತುರ್ಮುಖಬ್ರಹ್ಮನು ಅಲ್ಲಿಗೆ ಇಳಿತಂದು ಕರ್ದಮ ದೇವಹೂತಿಯರನ್ನು ಅಭಿ ನಂದಿಸಿ, ಮಗುವಿಗೆ ಕಪಿಲನೆಂದು ನಾಮಕರಣ ಮಾಡಿದನು.

ಪತಿವ್ರತೆಯಾದ ತನ್ನ ಮಡದಿಗೆ ಮಾತುಕೊಟ್ಟಂತೆ ಆಕೆಗೆ ಗಂಡುಮಗನನ್ನು ಕರುಣಿ ಸಿದ ಕರ್ದಮಪ್ರಜಾಪತಿಯು ತನ್ನ ಹೆಣ್ಣು ಮಕ್ಕಳಿಗೆ ಅನುರೂಪರಾದ ಗಂಡಂದಿರೊಡನೆ ವಿವಾಹವನ್ನೂ ನೆರವೇರಿಸಿದನು. ಮರೀಚಿ ಮಹರ್ಷಿ ಕಲಾದೇವಿಯನ್ನೂ, ಅತ್ರಿಮುನಿ ಅನಸೂಯೆಯನ್ನೂ, ಅಂಗಿರಸ್ಸು, ಶ್ರದ್ಧಾದೇವಿಯನ್ನೂ, ಪುಲಸ್ತ್ಯನು ಹವಿರ್ಭೂದೇವಿ ಯನ್ನೂ, ಪುಲಹನು ಗತಿಯನ್ನೂ. ಕ್ರತು ಕ್ರಿಯಾದೇವಿಯನ್ನೂ, ಭೃಗು ಖ್ಯಾತಿಯನ್ನೂ, ವಸಿಷ್ಠನು ಅರುಂಧತಿಯನ್ನೂ, ಅಥರ್ವನು ಶಾಂತಿದೇವಿಯನ್ನೂ, ಕೈಹಿಡಿದು ಮಡದಿಯ ರಾಗಿ ಮಾಡಿಕೊಂಡರು. ಹೀಗೆ ತನ್ನ ಕರ್ತವ್ಯವನ್ನು ಸಾಂಗವಾಗಿ ನೆರವೇರಿಸಿದ ಕರ್ದಮನು ಅವತಾರಪುರುಷನಾದ ಮಗನಿಂದಲೂ ಮಡದಿಯಿಂದಲೂ ಬೀಳ್ಕೊಂಡು, ವಿರಕ್ತನಾಗಿ, ತಪಸ್ಸನ್ನಾಚರಿಸುತ್ತಾ ದಿವ್ಯಜ್ಞಾನಿಯಾಗಿ ಮೋಕ್ಷವನ್ನು ಪಡೆದನು.

