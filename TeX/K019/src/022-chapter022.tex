
\chapter{೨೨. ಋಷಭದೇವ}

ಋಷಭದೇವನು ಮಹಾವಿಷ್ಣುವಿನ ಅವತಾರವಾದರೂ ಆತನು ಕೇವಲ ಮಾನವಮಾತ್ರ ನಂತೆ ವ್ಯವಹರಿಸುತ್ತಿದ್ದನು. ಮಹಾನುಭಾವರ ನಡವಳಿಕೆಯನ್ನು ಸಾಮಾನ್ಯ ಜನರು ಅನು ಕರಿಸುವರಾದ್ದರಿಂದ ಆತನು ಇತರರಿಗೆ ಮೇಲ್ಪಂಕ್ತಿಯಾಗುವಂತೆ ನಡೆದುಕೊಳ್ಳುತ್ತಿದ್ದನು. ತಾನು ಸರ್ವಜ್ಞನಾದರೂ ಆತನು ಗುರುಗಳಲ್ಲಿ ವಿದ್ಯಾಭ್ಯಾಸ ಮಾಡಿದನು. ತನಗೆ ಎಲ್ಲವೂ ತಿಳಿದಿದ್ದರೂ ಆಗಾಗ ಬ್ರಾಹ್ಮಣರಿಂದ ಧರ್ಮಸೂಕ್ಷ್ಮಗಳನ್ನು ಕೇಳಿ ತಿಳಿದುಕೊಳ್ಳುವವ ನಂತೆ ನಟಿಸುತ್ತಿದ್ದನು. ಆತನು ಯಾವ ಪುರುಷಾರ್ಥವನ್ನೂ ಬಯಸುತ್ತಿರಲಿಲ್ಲವಾದರೂ ಅನೇಕ ಯಜ್ಞ ಯಾಗಾದಿಗಳನ್ನು ಮಾಡಿ ಮುಗಿಸಿದನು. ಸದಾ ಆತನು ಆತ್ಮಾನಂದದಲ್ಲಿ ಮುಳುಗಿರಬಲ್ಲವನಾದರೂ ದೇವೇಂದ್ರನು ಕಳುಹಿಸಿದ ಜಯಂತಿಯೆಂಬ ಹೆಣ್ಣನ್ನು ಮದುವೆಯಾಗಿ, ನೂರು ಮಕ್ಕಳ ತಂದೆಯಾದನು. ಈ ಮಕ್ಕಳಲ್ಲಿ ಎಲ್ಲರಿಗೂ ಹಿರಿಯ ನಾದವನೇ ಮಹಾನುಭಾವನಾದ ಭರತ. ಈ ಪುಣ್ಯಪುರುಷನು ಆಳಿದ ಭೂಭಾಗವನ್ನೆ ನಾವೀಗ ‘ಭಾರತವರ್ಷ’ ಎಂದು ಕರೆಯುತ್ತಿರುವುದು. ಈತನ ತೊಂಬತ್ತೊಂಬತ್ತು ಜನ ತಮ್ಮಂದಿರಲ್ಲಿ ಮೊದಲ ಒಂಬತ್ತು ಜನ\footnote{೧. ಕುಶಾವರ್ತ, ಇಲಾವರ್ತ, ಬ್ರಹ್ಮಾವರ್ತ, ಮಲಯ, ಕೇತು, ಭದ್ರಸೇನ, ಇಂದ್ರಸ್ಪೃಕ್, ವಿದರ್ಭ, ಕೀಕಟ.} ಉಳಿದವರಲ್ಲಿ ಪ್ರತಿ ಹತ್ತು ಜನಕ್ಕೆ ಒಬ್ಬನಂತೆ ಪ್ರಮುಖರಾದರು. ಆ ತೊಂಬತ್ತು ಜನರಲ್ಲಿ ಒಂಬತ್ತು ಜನ\footnote{೨. ಕವಿ, ಹರಿ, ಅಂತರಿಕ್ಷ, ಪ್ರಬುದ್ಧ, ಪಿಪ್ಪಲಾಯನ, ಆವಿರ್ಹೋತ್ರ, ದ್ರುಮಿಲ, ಚಮಸ, ಕರಭಾಜನ.} ಮಹಾಭಾಗವತರಾಗಿ, ಲೋಕದಲ್ಲೆಲ್ಲ ಭಾಗವತ ಧರ್ಮವನ್ನು ಪ್ರಚಾರ ಮಾಡಿದರು\footnote{೩. ಇವರ ಕಥೆ ಮುಂದೆ ವಸುದೇವ–ನಾರದರ ಸಂವಾದದಲ್ಲಿ ಬರುತ್ತದೆ.}. ಉಳಿದ ಎಂಬತ್ತೊಂದು ಜನ ಯಜ್ಞಯಾಗಾದಿಗಳನ್ನು ಮಾಡಿಕೊಂಡು ಶುದ್ಧಬ್ರಾಹ್ಮಣರಾದರು.

ಋಷಭದೇವನು ಒಮ್ಮೆ ಭೂಯಾತ್ರೆಯನ್ನು ಕೈಕೊಂಡು ಬ್ರಹ್ಮಾವರ್ತದೇಶಕ್ಕೆ ಹೋದನು. ಅಲ್ಲಿ ಅನೇಕ ಬ್ರಹ್ಮರ್ಷಿಗಳು ಸಭೆ ಸೇರಿದ್ದರು. ಅವರಲ್ಲಿ ಋಷಭದೇವನ ಮಕ್ಕಳೂ ಇದ್ದರು. ಅವರನ್ನು ಕುರಿತು ಹೇಳುವವನಂತೆ ಆತನು ಆ ಬ್ರಹ್ಮಸಭೆಗೆ ಜ್ಞಾನೋಪದೇಶಮಾಡಿದನು: ‘ಅಯ್ಯಾ! ಜನ್ಮಗಳಲ್ಲೆಲ್ಲ ಮನುಷ್ಯಜನ್ಮವೇ ಅತ್ಯಂತ ಶ್ರೇಷ್ಠ. ಬಹು ಜನ್ಮಗಳ ಪುಣ್ಯಫಲದಿಂದ ಈ ಜನ್ಮ ಲಭಿಸುತ್ತದೆ. ಆಹಾರ, ನಿದ್ರೆ, ಭಯ, ಭೋಗಗಳು ಎಲ್ಲ ಪ್ರಾಣಿಗಳಿಗೂ ಸಮಾನ. ಆದರೆ ಇತರ ಎಲ್ಲ ಪ್ರಾಣಿಗಳಿಗಿಂತ ಮನುಷ್ಯಪ್ರಾಣಿ ಹೆಚ್ಚು ಆಳವಾಗಿ ಆಲೋಚಿಸಬಲ್ಲ. ಈ ಆಲೋಚನಾಶಕ್ತಿಯನ್ನು ಮನುಷ್ಯ ತನ್ನ ಉದ್ಧಾರಕ್ಕಾಗಿ ಬಳಸಿಕೊಳ್ಳಬೇಕು. ‘ನಾನು ಇಲ್ಲಿಗೆ ಏಕೆ ಬಂದೆ? ಹೇಗೆ ಬಂದೆ? ಈ ಹುಟ್ಟು ಸಾವುಗಳ ಕೋಟಲೆಯಿಂದ ಬಿಡುಗಡೆಹೊಂದಿ, ನಾನು ನಿತ್ಯ ಸುಖ ವನ್ನು ಪಡೆಯುವುದು ಹೇಗೆ?’ ಎಂಬುದನ್ನು ಗಾಢವಾಗಿ ಆಲೋಚಿಸಬೇಕು. ಹುಟ್ಟು ಸಾವುಗಳೆಂಬ ಸಂಸಾರ ಬಂಧನಕ್ಕೂ ಮೋಕ್ಷಕ್ಕೂ ನಮ್ಮ ಮನಸ್ಸೇ ಮುಖ್ಯ ಕಾರಣ. ವಿಷಯಸುಖದ ಆಶೆಯೇ ಬಂಧನಕ್ಕೆ ಕಾರಣ. ಈ ಆಶೆಯನ್ನು, ಮೋಹವನ್ನು ‘ಹೃದಯ ಗ್ರಂಥಿ’ ಎಂದು ಕರೆಯುತ್ತಾರೆ. ಇದೇ ಕರ್ಮ ವಾಸನೆಯ ತಾಯಿಬೇರು. ಇದರಿಂದಲೆ ಅಹಂಕಾರ ಮಮಕಾರ ಎಂಬ ‘ಅವಿದ್ಯೆ’ ಅಂಕುರಿಸುವುದು. ಹೃದಯಗ್ರಂಥಿ ಎಂಬ ಗಂಟು ಬಿಚ್ಚುವತನಕ ಸಂಸಾರಬಂಧನ ತಪ್ಪುವುದಿಲ್ಲ. ಸುಖದುಃಖಗಳನ್ನು ಅನುಭವಿಸು ವುದರಿಂದ ಕರ್ಮ ಸವೆಯುತ್ತದೆ. ಆದರೆ ಅದು ಬೇರುಸಹಿತ ನಾಶವಾಗುವುದಿಲ್ಲ. ಅದರ ಬೇರು ಸ್ವಲ್ಪ ಉಳಿದರೂ ಸಾಕು, ಅವಿದ್ಯೆ ತಲೆ ಹಾಕುತ್ತದೆ. ಮತ್ತೆ ಹುಟ್ಟಬೇಕಾಗುತ್ತದೆ. ಇದರಿಂದ ತಪ್ಪಿಸಿಕೊಳ್ಳುವುದಕ್ಕೆ ಇರುವುದು ಒಂದೇ ಉಪಾಯ–ಬೇರನ್ನು ಕಿತ್ತುಹಾಕು ವುದು. ಈ ದೇಹವೇ ಆತ್ಮವೆಂಬ ಭ್ರಾಂತಿಯಿಂದಲೆ ಇಂದ್ರಿಯಸುಖಕ್ಕೆ ವಶವಾಗುವುದು. ಆ ಭ್ರಾಂತಿಯನ್ನು ದೂರ ಮಾಡಬೇಕು. ಇದು ಬಾಯಿಂದ ಹೇಳುವಷ್ಟು ಸುಲಭವಲ್ಲ. ದೇವರ ಅನುಗ್ರಹದಿಂದಲೆ ಇದು ಸಾಧ್ಯ. ಆದ್ದರಿಂದ ಒಂದೇ ಮನಸ್ಸಿನಿಂದ ದೇವರನ್ನು ಕುರಿತು ಧ್ಯಾನಮಾಡಬೇಕು. ಸದಾ ಭಗವದ್ಭಕ್ತರ ಸಹವಾಸದಿಂದ ಭಕ್ತಿಯು ದೃಢವಾಗಿ ನೆಲಸಿ ಇಷ್ಟಾರ್ಥ ಸಿದ್ಧಿಯಾಗುತ್ತದೆ.’

ಋಷಭದೇವನು ಜಗತ್ತಿನಲ್ಲೆಲ್ಲ ಧರ್ಮವನ್ನು ಸ್ಥಾಪಿಸಿದುದಾದಮೇಲೆ, ಹಿರಿಯ ಮಗ ನಾದ ಭರತನಿಗೆ ಪಟ್ಟಗಟ್ಟಿ, ತಾನು ಅವಧೂತನಾದನು. ಆತನು ಬಟ್ಟೆಬರೆಗಳನ್ನೆಲ್ಲ ಕಿತ್ತು ಹಾಕಿ, ಬರಿಮೈಯಲ್ಲಿ ಅರಮನೆಯಿಂದ ಹೊರಟನು. ತಲೆಯನ್ನು ಕೆದರಿಕೊಂಡು, ಕಾಲೆಳೆ ದತ್ತ ಹೋಗುತ್ತಾ ಮೂಗನಂತೆ ಯಾರೊಡನೆಯೂ ಮಾತನಾಡದೆ, ಕಿವುಡನಂತೆ ಯಾರ ಮಾತನ್ನು ಕೇಳದೆ, ತನಗೆ ತಾನೆ ನಗುತ್ತಾ ಮಾತನಾಡುತ್ತಾ ಹೋಗುತ್ತಿದ್ದ ಆತನನ್ನು ಕಂಡವರೆಲ್ಲರೂ ಆತನನ್ನು ಹುಚ್ಚನೆಂದು ತಿಳಿಕೊಂಡಿದ್ದರು. ಕೆಲವರು ಆತನನ್ನು ಅಣಕಿಸಿ ದರೆ ಮತ್ತೆ ಕೆಲವರು ಆತನನ್ನು ಹೆದರಿಸಿದರು. ಕೆಲವು ನೀಚರು ಆತನ ಮೇಲೆ ಉಗುಳಿದರೆ ಮತ್ತೆ ಕೆಲವು ಕ್ರೂರಿಗಳು ಆತನ ಮೇಲೆ ಕಲ್ಲು, ಮಣ್ಣುಗಳನ್ನು ಎರಚಿದರು; ಹಲವರು ಆತನನ್ನು ಹಾಸ್ಯಮಾಡಿ ನಕ್ಕರು, ಆದರೆ ಋಷಭದೇವನಿಗೆ ಅತ್ತಕಡೆ ಗಮನವೆ ಇರಲಿಲ್ಲ. ಆತನ ಮನಸ್ಸು ಸದಾ ಬ್ರಹ್ಮಾನಂದವನ್ನು ಅನುಭವಿಸುತ್ತಿತ್ತು. ಆತನು ಜಗತ್ತೆಲ್ಲವನ್ನೂ ಅಲೆದಾಡಿ, ಕೊನೆಗೆ ಒಂದು ಸ್ಥಳದಲ್ಲಿ ‘ಅಜಗರವ್ರತ’ವೆಂಬ ವ್ರತವನ್ನು ಕೈಕೊಂಡು ಹೆಬ್ಬಾವಿನಂತೆ ಅಲುಗಾಡದೆ ಮಲಗಿಬಿಟ್ಟನು. ಮಲಗಿದ್ದಲ್ಲಿಯೇ ಯಾರಾದರೂ ತಂದು ಬಾಯಲ್ಲಿ ಹಾಕಿದರೆ ಊಟಮಾಡುವನು. ಮಲಗಿದ್ದಲ್ಲಿಯೇ ಜಲಮಲಗಳು. ಅಲ್ಲಿಯೇ ನಿದ್ರೆ. ಆತನು ಎಲ್ಲಕ್ಕೂ ಅತೀತನಾಗಿದ್ದನು. ಆತನಿಗೆ ಜಗತ್ತಿನಲ್ಲಿ ಎಲ್ಲವೂ ಬ್ರಹ್ಮ ಸ್ವರೂಪವಾಗಿತ್ತು.

ಋಷಭಯೋಗಿಗೆ ಆಕಾಶಗಮನ, ಪರಕಾಯಪ್ರವೇಶ, ಅಂತರ್ಧಾನ, ದೂರಶ್ರವಣ ಮೊದಲಾದ ಅನೇಕ ಸಿದ್ಧಿಗಳು ವಶವಾದುವು. ಆದರೆ ಆತನು ಅವನ್ನು ಆದರಿಸಲಿಲ್ಲ. ಆತನಿಗೆ ಗೊತ್ತು–ಈ ಮನಸ್ಸು ಬಹಳ ಚಂಚಲ; ನಂಬಿದ ಗಂಡನನ್ನು ತನ್ನ ಜಾರನ ಕೈಲಿ ಕೊಲ್ಲಿಸುವುದಕ್ಕೂ ಹಿಂಜರಿಯದ ಜಾರಿಣಿಯಂತೆ, ಅದು. ಸ್ವಲ್ಪ ಅವಕಾಶ ಸಿಕ್ಕರೂ ಸಾಕು, ಅದು ಯೋಗಿಯಾದವನನ್ನು ಕೂಡ ಸಂಸಾರದಲ್ಲಿ ಮುಳುಗಿಸಿ ಬಿಡುತ್ತದೆ. ಆದ್ದ ರಿಂದಲೇ ಋಷಭಯೋಗಿಯು ಆ ಸಿದ್ಧಿಗಳನ್ನು ಕಾಲಿನಿಂದೊದ್ದು, ಮತ್ತೆ ಭೂಸಂಚಾರಕ್ಕೆ ಹೊರಟನು. ಆತನು ದಕ್ಷಿಣಕರ್ಣಾಟಕದ ಕುಟಕಾಚಲಕ್ಕೆ ಬಂದು ಅಡವಿಯಲ್ಲಿ ಸಂಚರಿ ಸುತ್ತಿರುವಾಗ ದೊಡ್ಡದೊಂದು ಕಾಡುಗಿಚ್ಚು ಕಾಣಿಸಿಕೊಂಡಿತು. ಋಷಭಯೋಗಿಯ ದೇಹ ಅದಕ್ಕೆ ಸಿಕ್ಕಿ ದಗ್ಧವಾಗಿಹೋಯಿತು. ಜಗತ್ತು ಆತನನ್ನು ಮುಕ್ತಕಂಠವಾಗಿ ಹೊಗಳಿ ಸ್ತೋತ್ರಮಾಡಿತು:

\begin{verse}
ಅಹೋ ನು ವಂಶೋ ಯಶಸಾವದಾತಃ\\ಪ್ರೈಯ್ಯವ್ರತೋ ಯತ್ರ ಪುಮಾನ್ ಪುರಾಣಃ ॥\\ಕೃತಾವತಾರಃ ಪುರುಷಃ ಸ ಆದ್ಯ-\\ಶ್ಚಚಾರ ಧರ್ಮಂ ಯದಕರ್ಮ ಹೇತುಂ ॥
\end{verse}

ಪುರಾಣಪುರುಷನಾದ ಭಗವಂತನು ಋಷಭದೇವನ ರೂಪದಿಂದ ಯಾವ ವಂಶದಲ್ಲಿ ಹುಟ್ಟಿ ಮೋಕ್ಷಕ್ಕೆ ಸಾಧಕವಾದ ನಿವೃತ್ತಿಧರ್ಮವನ್ನು ಆಚರಿಸಿ, ಲೋಕವನ್ನು ಅನುಗ್ರಹಿಸಿ ದನೋ, ಆ ಪ್ರಖ್ಯಾತವಾದ ಪ್ರಿಯವ್ರತರಾಜನ ವಂಶವು ಅತಿ ಪವಿತ್ರವಾದುದು.

