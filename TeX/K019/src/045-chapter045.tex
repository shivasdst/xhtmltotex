
\chapter{೪೫. ನಂದನ ಕಂದನಿಗೆ ನಾಮಕರಣ}

ವಸುದೇವನು ಮಧುರಾಪುರಿಯಲ್ಲಿದ್ದರೂ ಅವನ ಮನಸ್ಸೆಲ್ಲ ಗೋಕುಲದಲ್ಲಿಯೆ. ಅವನ ವಂಶವನ್ನು ಬೆಳಗುವ ಬೆಳಸುವ ಮಕ್ಕಳಿಬ್ಬರು ಅಲ್ಲಿ ಬೆಳೆಯುತ್ತಿದ್ದಾರೆ. ಅವರು ಯಾದವ ವಂಶಕ್ಕೆ ಸೇರಿದವರಾದರೂ, ಕ್ಷತ್ರಿಯರ ಧರ್ಮಕರ್ಮಗಳೊಂದಕ್ಕೂ ಗತಿ ಯಿಲ್ಲದೆ, ಗೋವಳರ ಮಧ್ಯೆ ಗೋವಳರಾಗಿ ಬೆಳೆಯುತ್ತಿದ್ದಾರೆ. ಇದಕ್ಕೇನು ಮಾಡು ವುದು? ಆತನು ಸುದೀರ್ಘವಾಗಿ ಆಲೋಚನೆ ಮಾಡಿ, ಯಾದವರ ಕುಲಪುರೋಹಿತನಾದ ಗರ್ಗರಮುನಿಯಲ್ಲಿ ತನ್ನ ಸಂಕಟವನ್ನು ತೋಡಿ ಕೊಂಡನು. ಆ ಮುನಿಯು ‘ಕ್ಷತ್ರಿಯರಿಗೆ ಅಗತ್ಯವಾದ ಕರ್ಮಗಳಲ್ಲಿ ಮೊದಲನೆಯದಾದ ನಾಮಕರಣ ಈಗ ಆಗಬೇಕು. ನಾನು ಹೋಗಿ, ಆ ಕಾರ್ಯವನ್ನು ರಹಸ್ಯವಾಗಿ ನೆರವೇರಿಸಿ ಬರುತ್ತೇನೆ’ ಎಂದು ಹೇಳಿ, ನಂದ ಗೋಕುಲಕ್ಕೆ ಬಂದನು. ಮಹಾತಪಸ್ವಿಯಾದ ಆತನನ್ನು ಕಾಣುತ್ತಲೆ ನಂದನು ಭಕ್ತಿಯಿಂದ ಆತನನ್ನು ತನ್ನ ಮನೆಗೆ ಕರೆದೊಯ್ದು ಉಪಚರಿಸಿ ‘ಸ್ವಾಮಿ, ನಿಮ್ಮ ಪಾದಧೂಳಿಯಿಂದ ನನ್ನ ಮನೆ ಪಾವನವಾಯಿತು. ನನ್ನ ಪಾಲಿನ ದೇವರೆ ಬಂದಂತೆ ನೀವು ಬಂದಿದ್ದೀರಿ. ನನ್ನ ಮಗನಿಗೆ ಜಾತಕರ್ಮ, ನಾಮಕರಣ–ಒಂದೂ ಆಗಿಲ್ಲ. ನಮ್ಮ ರೋಹಿಣಿಯ ಮಗನಿಗೂ ಅಷ್ಟೆ. ನೀವು ಆ ಕರ್ಮಗಳನ್ನು ನಿಮ್ಮ ಕೈಯಾರೆ ನಡೆಸಿಕೊಟ್ಟರೆ ಮಕ್ಕಳಿಗೆ ಶ್ರೇಯಸ್ಸಾಗು ತ್ತದೆ. ನಾನು ಧನ್ಯನಾಗುತ್ತೇನೆ’ ಎಂದು ಕೇಳಿಕೊಂಡನು.

ಹುಡುಕುತ್ತಿದ್ದ ಬಳ್ಳಿ ಕಾಲಿಗೆ ಸುತ್ತಿಕೊಂಡಂತಾಯಿತು, ಗರ್ಗನಿಗೆ. ಆದರೂ ಆತ ಅದನ್ನು ಹೊರದೋರದೆ, ‘ಅಯ್ಯಾ, ನಾನು ಯಾದವರ ಕುಲಪುರೋಹಿತ. ನಾನು ಬಂದು ನಿನ್ನ ಮಕ್ಕಳಿಗೆ ಜಾತಕರ್ಮಾದಿಗಳನ್ನು ಮಾಡಿಸಿದೆನೆಂದು ಗೊತ್ತಾದರೆ ಕಂಸರಾಜನಿಗೆ ಸಂದೇಹ ಬರಬಹುದು. ನಿನಗೂ ವಸುದೇವನಿಗೂ ಇರುವ ಗೆಳೆತನ ಲೋಕಕ್ಕೆಲ್ಲ ಗೊತ್ತಿದೆ. ಅವನ ಮಡದಿಯಾದ ದೇವಕಿಗೆ ಎಂಟನೆಯ ಗರ್ಭದಲ್ಲಿ ಮಗನೆ ಹುಟ್ಟುತ್ತಾ ನೆಂದು ಆಕಾಶವಾಣಿ ತಿಳಿಸಿತ್ತು. ಹಾಗಾಗಲಿಲ್ಲವಾದರೂ ಅವಳ ಗರ್ಭದಲ್ಲಿ ಹುಟ್ಟಿದ ಮಹಾಮಾಯೆ ‘ನಿನ್ನ ಶತ್ರು ಬೇರೆಕಡೆ ಬೆಳೆಯತ್ತಿದ್ದಾನೆ’ ಎಂದು ಬೇರೆ ಹೇಳಿಹೋಯಿತು. ಸಂದರ್ಭ ಹೀಗಿರುವುದರಿಂದ, ವಸುದೇವ ತನ್ನ ಮಗನನ್ನು ನಿನ್ನ ಮನೆಯಲ್ಲಿ ಬಚ್ಚಿಟ್ಟಿರ ಬೇಕೆಂಬ ಸಂದೇಹ ಹುಟ್ಟುವುದಕ್ಕೆ ಅವಕಾಶವಾಗುತ್ತದೆ. ಹಾಗೇನಾದರೂ ಆದರೆ ನಿನ್ನ ಮಗನ ಗತಿಯೇನು?’ ಎಂದು ನಂದನ ಮನಸ್ಸಿನಲ್ಲಿ ಭಯವನ್ನು ಬಿತ್ತಿದನು. ಆಗ ನಂದನು ‘ಸ್ವಾಮಿ, ನಿಮ್ಮ ಮಾತನ್ನು ನಾನು ಒಪ್ಪಿದೆ. ಆದರೆ ಅಗತ್ಯವಾಗಿ ನಡೆಯಬೇಕಾದ ಕರ್ಮವನ್ನು ಕೈಬಿಡುವುದಕ್ಕಾಗುತ್ತದೆಯೇ? ಇದನ್ನು ಅತ್ಯಂತ ಗುಟ್ಟಾಗಿ ನಡೆಸೋಣ. ಗೋಕುಲದಲ್ಲಿ ಕೂಡ ಯಾರೊಬ್ಬರಿಗೂ ಗೊತ್ತಾಗದಂತೆ ನಾನು ನೋಡಿಕೊಳ್ಳುತ್ತೇನೆ’ ಎಂದು ಹೇಳಿದನು. ಗರ್ಗನು ಅದಕ್ಕೆ ಒಪ್ಪಿ, ಕರ್ಮಗಳನ್ನೆಲ್ಲ ಸಾಂಗವಾಗಿ ನೆರವೇರಿಸಿ ದನು. ಅನಂತರ ಆತನು ನಂದನೊಡನೆ ‘ಅಯ್ಯಾ, ಈ ರೋಹಿಣಿಯ ಮಗ ಲೋಕವನ್ನೆಲ್ಲ ಸಂತೋಷಪಡಿಸುವವನು, ಮಹಾಬಲಶಾಲಿ; ಆದ್ದರಿಂದ ಇವನಿಗೆ ‘ಬಲರಾಮ’ನೆಂದು ಹೆಸರಿರಲಿ. ನಿನ್ನ ಮಗ ಬಣ್ಣದಲ್ಲಿ ಕಪ್ಪಾಗಿರುವುದರಿಂದ ‘ಕೃಷ್ಣ’ ಎಂದು ಕರೆಯಿರಿ. ಇವನು ಹಿಂದೆ ಒಂದಾನೊಂದು ಕಾಲದಲ್ಲಿ ವಸುದೇವನ ಮಗನಾಗಿದ್ದವನಾದ್ದರಿಂದ ‘ವಾಸು ದೇವ’ನೆಂಬ ಹೆಸರೂ ಇರಲಿ. ಅಯ್ಯಾ, ಈ ನಿನ್ನ ಮಗನ ಗುಣಕರ್ಮಗಳಿಗೆ ತಕ್ಕಂತೆ ಹೆಸರು ಕೊಡುತ್ತಾಹೋದರೆ ಅದಕ್ಕೆ ಕೊನೆಮೊದಲೇ ಇಲ್ಲ. ಈತನ ಮಹಿಮೆಯನ್ನು ಅರ್ಥಮಾಡಿಕೊಳ್ಳುವುದು ನಿನಗೂ ಸಾಧ್ಯವಿಲ್ಲ, ನನಗೂ ಸಾಧ್ಯವಿಲ್ಲ. ಈತ ಸಾಕ್ಷಾತ್ ಪರಮೇಶ್ವರನೆಂದೆ ತಿಳಿ. ಈತನಿಂದ ನಿಮಗೆಲ್ಲ ಸುಖಸಂತೋಷಗಳು ದೊರೆಯುವುದು ಮಾತ್ರವೇ ಅಲ್ಲ, ಈತನಿಂದ ನಿಮ್ಮ ಕುಲ ಕೋಟಿಗಳೆಲ್ಲ ಉದ್ಧಾರವಾಗುತ್ತವೆ. ಈತನನ್ನು ಕಣ್ಣುಪಾಪೆಯಂತೆ ಎಚ್ಚರದಿಂದ ಕಾಪಾಡು’ ಎಂದು ಹೇಳಿದನು. ಹೀಗೆ ಗರ್ಗನು ಶ್ರೀಕೃಷ್ಣನ ಮಹಿಮೆಯನ್ನು ಸೂಕ್ಷ್ಮವಾಗಿ ತಿಳಿಸಿ, ಮಧುರೆಗೆ ಹಿಂದಿರುಗಿದನು. ನಂದನು ತನ್ನ ಮಗನ ಗುಣಕೀರ್ತನೆಯನ್ನು ಕೇಳಿ, ಆನಂದದಿಂದ ಹಿರಿಹಿರಿ ಹಿಗ್ಗುತ್ತಾ ತಾನು ಧನ್ಯನೆಂದುಕೊಂಡನು.

