
\chapter{೭೬. ಬಲರಾಮನ ಸಾಹಸ}

ಚೆಲುವ ಚನ್ನಿಗನಾದ ಬಲರಾಮನು ಕೊರಳಲ್ಲಿ ಕಮಲದ ಹಾರವನ್ನು ಧರಿಸಿ, ಗೆಳತಿಯ ರಾದ ಗೋಪಿಯರೊಡನೆ ರೈವತಗಿರಿಯಮೇಲೆ ವಿಹರಿಸುತ್ತಿದ್ದ. ವಾರುಣಿಯನ್ನು ಕುಡಿದು ಮೈಮರೆತಿದ್ದ ಆ ಗಂಡು ಹೆಣ್ಣುಗಳು ಪ್ರಕೃತಿಯ ಏಕಾಂತವನ್ನು ಭಂಗಿಸಿ, ಉಚ್ಚಕಂಠ ದಿಂದ ಹಾಡುತ್ತಿದ್ದರು. ಅವರ ಸಂಗೀತ ಅಲೆಅಲೆಯಾಗಿ ಮೇಲೆದ್ದು, ಗಾಳಿಯನ್ನೇರಿ, ದಿಕ್ಕುದಿಕ್ಕಿಗೆ ಹರಿದುಹೋಗುತ್ತಿತ್ತು. ಆ ಪರ್ವತ ಪ್ರಾಂತದಲ್ಲಿ ವಿಶ್ರಾಂತಿಗಾಗಿ ಮಲಗಿದ್ದ ದ್ವಿವಿದನೆಂಬ ವಾನರನನ್ನು ಈ ಸಂಗೀತ ಬಡಿದೆಬ್ಬಿಸಿತು. ಅವನು ಒಂದೆ ನೆಗೆತಕ್ಕೆ ಆ ಪರ್ವತದ ನೆತ್ತಿಗೆ ಹಾರಿ ನೋಡುತ್ತಾನೆ, ಹೆಣ್ಣಾನೆಗಳ ಮಧ್ಯದ ಸಲಗನಂತೆ ಬಲರಾಮನು ಗೋಪಿಯರ ಮಧ್ಯದಲ್ಲಿ ವಿಹರಿಸುತ್ತಿದ್ದಾನೆ! ದ್ವಿವಿದನೆಂದರೆ ಸಾಮಾನ್ಯನಾದ ಕಪಿಯಲ್ಲ. ಅವನು ತ್ರೇತಾಯುಗದಲ್ಲಿ ಸುಗ್ರೀವನ ಮಂತ್ರಿಯಾಗಿದ್ದವನು! ನರಕಾಸುರನೂ ಅವನೂ ಜೀವದ ಗೆಳೆಯರು. ಶ್ರೀಕೃಷ್ಣನು ನರಕಾಸುರನನ್ನು ಕೊಂದಮೇಲೆ ದ್ವಿವಿದನು ಶ್ರೀಕೃಷ್ಣನ ಪರಮವೈರಿಯಾದನು. ಅವನು ಶ್ರೀಕೃಷ್ಣನ ಪ್ರಜೆಗಳಿಗೆಲ್ಲ ಬಗೆಬಗೆಯ ಕಷ್ಟಗಳನ್ನು ಕೊಡುತ್ತಿದ್ದನು. ಹತ್ತುಸಾವಿರ ಆನೆಗಳ ಬಲವುಳ್ಳ ಈ ವಾನರ ಬೆಟ್ಟಗಳನ್ನು ಎತ್ತಿತಂದು ಮನೆಗಳ ಮೇಲೆ ಹಾಕಿ, ಅವುಗಳನ್ನು ನೆಲಸಮ ಮಾಡುತ್ತಿದ್ದನು; ಊರುಗಳನ್ನು ಕೊಳ್ಳೆ ಹೊಡೆದು ಸುಟ್ಟುಹಾಕುತ್ತಿದ್ದನು; ಜನಗಳನ್ನು ಎತ್ತಿಕೊಂಡು ಹೋಗಿ ಬೆಟ್ಟದ ಗವಿಗಳಲ್ಲಿ ಕೂಡಿಹಾಕಿ, ಪರ್ವತ ಶಿಖರಗಳಿಂದ ಈ ಗವಿಯನ್ನು ಮುಚ್ಚಿಹಾಕುತ್ತಿದ್ದನು; ಪುಷಿಗಳ ಆಶ್ರಮಗಳಿಗೆ ನುಗ್ಗಿ, ಅಲ್ಲಿನ ಗಿಡಮರಗಳನ್ನೆಲ್ಲ ಕಿತ್ತುಹಾಕಿ, ಅವರ ಯಾಗದ ಬೆಂಕಿ ಯನ್ನು ಆರಿಸುತ್ತಿದ್ದನು; ಸಮುದ್ರಮಧ್ಯದಲ್ಲಿ ನಿಂತು ತನ್ನ ತೋಳುಗಳಿಂದ ನೀರನ್ನೆರಚಿ, ದಡದಲ್ಲಿರುವ ಊರುಗಳೆಲ್ಲ ಕೊಚ್ಚಿಹೋಗುವಂತೆ ಮಾಡುತ್ತಿದ್ದನು; ಹೆಣ್ಣುಮಕ್ಕಳನ್ನು ಹಿಡಿದು, ಅವರ ಮಾನಭಂಗಮಾಡಿ, ಕಚ್ಚಿಹಾಕುತ್ತಿದ್ದನು. ಇಂತಹ ಈ ಕೆಟ್ಟ ವಾನರ ಬಲರಾಮನನ್ನು ಕಾಣುತ್ತಲೆ ಹತ್ತಿರದಲ್ಲಿದ್ದ ಒಂದು ಮರವನ್ನೇರಿ, ಕೊಂಬೆಗಳನ್ನು ಅಲ್ಲಾಡಿಸುತ್ತಾ, ಗೋಪಿಯರ ಸಂಗೀತಕ್ಕೆ ಸವಾಲೆಂಬುವಂತೆ ಗಟ್ಟಿಯಾಗಿ ಕಿರಿಚಲು ಆರಂಭಿಸಿದನು. ಹಾಸ್ಯಪ್ರಿಯರಾದ ಗೋಪಿಯರು ಅದನ್ನು ಕೇಳಿ ಗಟ್ಟಿಯಾಗಿ ನಕ್ಕರು. ಇದನ್ನು ಕಂಡು ಉತ್ಸಾಹಗೊಂಡ ಆ ಕೋತಿ ಹಲ್ಲು ಕಿರಿದು, ಹುಬ್ಬು ಕುಣಿಸಿ, ಪೃಷ್ಠವನ್ನು ತೋರಿಸುತ್ತಾ ಕೊಂಬೆಯಿಂದ ಕೊಂಬೆಗೆ ಹಾರಾಡಿತು.

ಕೋತಿಯ ಚೇಷ್ಟೆಗಳನ್ನು ಕಂಡು ಅಸಹ್ಯಗೊಂಡ ಬಲರಾಮನು ಅದನ್ನು ಕಲ್ಲಿನಿಂದ ಹೊಡೆದನು. ಅದರಿಂದ ತಪ್ಪಿಸಿಕೊಂಡ ದ್ವಿವಿದ ಬಲರಾಮನ ಸಮೀಪದಲ್ಲಿದ್ದ ಮದ್ಯ ಪಾತ್ರೆಯನ್ನು ಹೊತ್ತುಕೊಂಡು ಹೋಗಿ, ಅದನ್ನು ಒಡೆದುಹಾಕಿದನು. ಗೋಪಿಯರ ಸೀರೆ ಗಳನ್ನು ಹಿಡಿದು ಹರಿದುಹಾಕಿದನು. ಇದನ್ನು ಕಂಡು ಬಲರಾಮನು ಕಿಡಿಕಿಡಿಯಾದನು. ಬಹುಕಾಲದಿಂದಲೂ ಅವನನ್ನು ಹಿಡಿದು ಶಿಕ್ಷಿಸಬೇಕೆಂದು ಬಲರಾಮಕೃಷ್ಣರು ಹೊಂಚು ಹಾಕಿದ್ದರು. ಈಗ ಅನಾಯಾಸವಾಗಿ ತನ್ನ ಕೈಗೆ ಸಿಕ್ಕಿರುವುದರಿಂದ ಅವನನ್ನು ತೀರಿಸಿ ಬಿಡಬೇಕೆಂದು ಬಲರಾಮನು ನಿಶ್ಚಯಿಸಿದನು. ಆತನು ತನ್ನ ಆಯುಧವಾದ ನೇಗಿಲನ್ನು ಕೈಗೆ ತೆಗೆದುಕೊಳ್ಳುವಷ್ಟರಲ್ಲಿ ದ್ವಿವಿದನು ದೊಡ್ಡದೊಂದು ಮರವನ್ನು ಕಿತ್ತು ತಂದು ಬಲರಾಮನ ತಲೆಯಮೇಲೆ ಅಪ್ಪಳಿಸಿದನು. ಬಲರಾಮನು ತನ್ನ ಎಡಗೈಯಿಂದ ಅದನ್ನು ನಿವಾರಿಸಿ ಬಲಗೈಯಿಂದ ಅವನನ್ನು ಗುದ್ದಿದನು. ಆ ಪೆಟ್ಟಿಗೆ ಅವನ ತಲೆಯೊಡೆದು ಮೈ ಯೆಲ್ಲ ನೆತ್ತರಿಂದ ನೆನೆದು ಹೋದರೂ ಆ ದ್ವಿವಿದ ಹಟವನ್ನು ಬಿಡದೆ, ಮತ್ತೊಂದು ದೊಡ್ಡ ಮರವನ್ನು ಕಿತ್ತು, ಅದರಿಂದ ಬಲರಾಮನನ್ನು ಹೊಡೆದನು. ಅದರ ಹಿಂದೆಯೆ ಕಲ್ಲಿನ ಮಳೆಯನ್ನೂ ಕರೆದನು. ಇಷ್ಟು ಸಾಲದೆಂದು ಹಾರಿ ಅವನ ಎದೆಯ ಮೇಲೆ ಗುದ್ದಿದನು. ಇದರಿಂದ ಆತನ ಸೈರಣೆ ತಪ್ಪಿತು. ತನ್ನ ನೇಗಿಲಿನಿಂದ ಅವನನ್ನು ಹತ್ತಿರಕ್ಕೆ ಎಳೆದುಕೊಂಡು, ತನ್ನ ಶಕ್ತಿಯನ್ನೆಲ್ಲ ಬಿಟ್ಟು ಅವನ ಪಕ್ಕೆಗೆ ಗುದ್ದಿದನು, ಒಡನೆಯೆ ದ್ವಿವಿದ ನೆತ್ತರನ್ನು ಕಾರುತ್ತಾ ನೆಲಕ್ಕೆ ಬಿದ್ದು ಸತ್ತುಹೋದನು. ಇದನ್ನು ಕಂಡು ದೇವಲೋಕದವ ರೆಲ್ಲ ‘ಅಬ್ಬಾ, ಬಲರಾಮನ ಸಾಹಸವೇ!’ ಎಂದು ಉದ್ಗಾರಮಾಡಿದರು.

ಬಲರಾಮನ ಎಣೆಯಿಲ್ಲದ ಸಾಹಸಕ್ಕೆ ಮತ್ತೊಂದು ದೃಷ್ಟಾಂತವೆಂದರೆ ಆತನು ಇಡೀ ಹಸ್ತಿನಾವತೀಪಟ್ಟಣವನ್ನೆ ತನ್ನ ನೇಗಿಲಿನಿಂದ ಎಳೆದು ಗಂಗೆಗೆ ನೂಕಿದುದು. ಅದು ನಡೆದ ಸಂದರ್ಭ ಹೀಗೆ. ದುರ್ಯೋಧನನ ಮಗಳಾದ ಲಕ್ಷಣೆಗೆ ವಿವಾಹ ಮಾಡಬೇಕೆಂದು ಸ್ವಯಂವರಕ್ಕೆ ಏರ್ಪಡಿಸಿದ್ದರು. ದೇಶದೇಶಗಳಿಂದ ಅನೇಕ ರಾಜರು ಬಂದು ಸ್ವಯಂವರ ಮಂಟಪದಲ್ಲಿ ನೆರೆದಿದ್ದರು. ಆಗ ಜಾಂಬವತಿಯ ಮಗನಾದ ಸಾಂಬನು ಸ್ವಯಂವರ ಮಂಟಪಕ್ಕೆ ನುಗ್ಗಿ ಅವಳನ್ನು ಹೊತ್ತುಕೊಂಡು ಹೋದನು. ಇದನ್ನು ಕಂಡು ಕೌರವರಿ ಗೆಲ್ಲ ಕೋಪಬಂದಿತು. ಕರ್ಣ, ಶಲ್ಯ ಮೊದಲಾದ ಆರು ಜನ ಮಹಾರಥರು ಅವನ ಬೆನ್ನಟ್ಟಿಹೋದರು. ಸಾಂಬನೇನು ಸಾಮಾನ್ಯನೇ? ಅವನು ಶ್ರೀಕೃಷ್ಣನ ಮಗ; ಸಿಂಹದ ಮರಿಯಂತೆ ಧೈರ್ಯದಿಂದ ಆ ಆರು ಜನರನ್ನೂ ಇದಿರಿಸಿ, ಅವರ ರಥ, ಕುದುರೆ, ಸಾರಥಿ ಗಳನ್ನು ಹೊಡೆದುಹಾಕಿದನು. ಹುಡುಗನಾದರೂ ಮಹಾಪರಾಕ್ರಮಿಯಾದ ಆ ಸಾಂಬ ನನ್ನು ಸೋಲಿಸಲು ಸಾಧ್ಯವಿಲ್ಲದೆ, ಆ ಆರು ಜನರೂ ಒಟ್ಟಾಗಿ ಸೇರಿ, ಅವರಲ್ಲಿ ನಾಲ್ಕು ಜನ ಅವನ ರಥದ ನಾಲ್ಕು ಕುದುರೆಗಳನ್ನು ಕೊಂದರು; ಇನ್ನೊಬ್ಬ ಅವನ ಸಾರಥಿಯನ್ನು ಹೊಡೆದ; ಮತ್ತೊಬ್ಬ ಅವನ ಬಿಲ್ಲನ್ನು ಕತ್ತರಿಸಿದ. ಅನಂತರ ಆರು ಜನರೂ ಸೇರಿಕೊಂಡು ಬಹು ಕಷ್ಟದಿಂದ ಅವನನ್ನು ರಥದಿಂದ ಕೆಳಕ್ಕೆಳೆದು, ಕೈಕಾಲು ಕಟ್ಟಿ, ಉತ್ತರೆಯೊಡನೆ ಅವನನ್ನು ಹಸ್ತಿನಾವತಿಗೆ ಎಳೆದೊಯ್ದು ಸೆರೆಯಲ್ಲಿ ಇಟ್ಟರು. ಈ ಸುದ್ದಿ ನಾರದರಿಂದ ಶ್ರೀಕೃಷ್ಣನಿಗೆ ಗೊತ್ತಾಯಿತು. ಆತ ಕೋಪದಿಂದ ಕೌರವರ ಮೇಲೆ ಯುದ್ಧಕ್ಕೆ ಹೊರಡಲು ಸಿದ್ಧನಾದ. ಆದರೆ ಬಲರಾಮನು ಆತನನ್ನು ತಡೆದು, ತಾನು ಸಾಂಬನನ್ನು ಬಿಡಿಸಿಕೊಂಡು ಬರುವುದಾಗಿ ಹೇಳಿ, ಕೆಲವು ಕುಲವೃದ್ಧರನ್ನೂ ಜ್ಞಾನವೃದ್ಧರನ್ನೂ ಕರೆದುಕೊಂಡು ಹಸ್ತಿನಾ ವತಿಗೆ ಹೊರಟನು.

ಪ್ರಯಾಣ ಹೊರಟ ಬಲರಾಮನು ಹಸ್ತಿನಾವತಿಯ ಹೊರಗಿರುವ ಉದ್ಯಾನವನದಲ್ಲಿ ಇಳಿದುಕೊಂಡು, ತಾನು ಬಂದಿರುವ ಸುದ್ದಿಯನ್ನು ಧೃತರಾಷ್ಟ್ರನಿಗೆ ತಿಳಿಸುವಂತೆ ಉದ್ಧವ ನನ್ನು ಕಳುಹಿಸಿದನು. ಆತನು ಊರೊಳಗೆ ಹೋಗಿ ಧೃತರಾಷ್ಟ್ರ, ಭೀಷ್ಮ, ದುರ್ಯೋಧನ ಮೊದಲಾದ ಮುಖ್ಯರಿಗೆ ಆ ಸುದ್ದಿಯನ್ನು ತಿಳಿಸಿದ. ಅವರೆಲ್ಲರಿಗೂ ಬಲರಾಮನೆಂದರೆ ಪಂಚಪ್ರಾಣ. ಆದ್ದರಿಂದ ಅವರೆಲ್ಲರೂ ಆತನನ್ನು ಊರಿಗೆ ಕರೆತರಬೇಕೆಂದು ಅತ್ಯಂತ ಸಂಭ್ರಮದಿಂದ ಹೊರಟುಬಂದರು. ಪರಸ್ಪರ ಕುಶಲಪ್ರಶ್ನೆಗಳಾದ ಮೇಲೆ ಬಲರಾಮನು ಅವರನ್ನು ಕುರಿತು ‘ಅಯ್ಯಾ, ಮಹಾವೀರರೆ, ನೀವು ಹಲವಾರು ಜನ ಸೇರಿಕೊಂಡು ನಮ್ಮ ಹುಡುಗನನ್ನು ಮೋಸದಿಂದ ಸೆರೆ ಹಿಡಿದಿರುವಿರಂತೆ. ನಮ್ಮ ಮಹಾರಾಜನಾದ ಉಗ್ರ ಸೇನನು ಬಂಧುಗಳಾದ ನಿಮ್ಮೊಡನೆ ಯುದ್ಧಮಾಡಲು ಇಷ್ಟವಿಲ್ಲದೆ, ತಕ್ಷಣವೇ ಆ ಹುಡುಗನನ್ನು ಬಿಟ್ಟು ಕಳುಹಿಸುವಂತೆ ನಿಮಗೆ ಹೇಳಿಕಳುಹಿಸಿದ್ದಾನೆ. ಇದಕ್ಕೆ ತಪ್ಪಿದರೆ ಪರಸ್ಪರ ದ್ವೇಷಕ್ಕೆ ಕಾರಣವಾಗುತ್ತದೆ’ ಎಂದನು. ಇದನ್ನು ಕೇಳಿ ಕೌರವರಿಗೆಲ್ಲ ಕೆಂಡ ದಂತಹ ಕೋಪಬಂತು. ಅವರು ತಮ್ಮಲ್ಲಿಯೇ ‘ಅಬ್ಬ, ಈ ಯಾದವರ ಗರ್ವವೇ! ಕಾಲಿನ ಲ್ಲಿರಬೇಕಾದ ಪಾದರಕ್ಷೆ ಕಿರೀಟವಾಗುತ್ತೇನೆನ್ನುವುದಲ್ಲ! ಇವರು ಎಷ್ಟು ಮಾತ್ರದವರು? ನೆರೆಹೊರೆಯ ಸಂಬಂಧದಿಂದ ಇವರನ್ನು ಬಂಧುಗಳೆಂದು ನಾವು ಸಲಿಗೆಯನ್ನಿತ್ತುದೆ ತಪ್ಪಾಯಿತಲ್ಲ! ನಾವು ಉದಾಸೀನರಾಗಿದ್ದುದರಿಂದ ಇವರಿಗೆ ಛತ್ರ, ಚಾಮರ, ಸಿಂಹಾಸನ ಇತ್ಯಾದಿ ಬಿರುದು ಬಾವಲಿಗಳು, ರಾಜಭೋಗ! ಇವರಿಗೆ ನಾವು ಮಾಡಿದ ಉಪಕಾರವೆಲ್ಲ ಹಾವಿಗೆ ಹಾಲೆರೆದಂತಾಯಿತು! ನಮ್ಮಿಂದ ಮುಂದೆ ಬಂದು, ನಮ್ಮ ಮೇಲೆಯೆ ಅಧಿಕಾರ ಮಾಡುವ ಇವರ ಸೊಕ್ಕನ್ನು ಮುರಿಯಬೇಕು. ಸಿಂಹದ ಬಾಯಲ್ಲಿರುವ ಆಹಾರವನ್ನು ಮೇಕೆಯ ಮರಿ ಕಿತ್ತುಕೊಳ್ಳುತ್ತದೆಯೆ? ಭೀಷ್ಮ ದ್ರೋಣರ ರಕ್ಷಣೆಯಲ್ಲಿರುವ ಸಾಂಬ ನನ್ನು ಇವರು ಬಿಡಿಸಿಕೊಂಡು ಹೋಗುತ್ತಾರೆಯೆ? ಸಾಕು ಇವರ ಬಡಿವಾರ. ಅಹಂಕಾರ ದಿಂದ ಪೊಳ್ಳು ಮಾತಾಡುವ ಆ ಉಗ್ರಸೇನನನ್ನು ಸಿಂಹಾಸನದಿಂದ ಕಿತ್ತುಹಾಕೋಣ’ ಎಂದುಕೊಂಡರು.

ಕೌರವರು ತಮ್ಮತಮ್ಮಲ್ಲಿಯೆ ಮಾಡಿಕೊಂಡ ನಿಶ್ಚಯವನ್ನು ಬಲರಾಮನಿಗೆ ಸ್ಪಷ್ಟ ವಾಗಿಯೆ ತಿಳಿಸಿದರು. ಅದನ್ನು ಕೇಳುತ್ತಲೆ ಬಲರಾಮನಿಗೆ ಅಂಗಾಲಿನಿಂದ ನಡುನೆತ್ತಿಯ ವರೆಗೂ ಕೋಪ ಉಕ್ಕಿತು; ಆತನ ಕಣ್ಣುಗಳಿಂದ ಕಿಡಿಗಳುದುರಿದವು. ರೋಷಭೀಷಣನಾದ ಆತನು ರುದ್ರನಂತೆ ಅಟ್ಟಹಾಸದಿಂದ ಗಹಗಹಿಸಿ ನುಗುತ್ತಾ, ‘ಎಲಾ, ತೊಂಡುದನಗಳಿಗೆ ದಂಡಪ್ರಹಾರವೇ ಬೇಕು. ಕೋಪಗೊಂಡಿದ್ದ ಶ್ರೀಕೃಷ್ಣನನ್ನು ಸಮಾಧಾನಮಾಡಿ, ಎರಡು ವಂಶಗಳಿಗೂ ಒಳ್ಳೆಯದಾಗಲೆಂದು ನಾನು ಬಂದರೆ, ನನ್ನ ಮುಖಕ್ಕೆ ಮಸಿ ಬಳಿಯುವುದೆ? ಅಬ್ಬ, ನಿಮ್ಮ ಅಹಂಕಾರವೆ! ಮೂರು ಲೋಕಗಳಿಗೂ ಸ್ವಾಮಿಯಾಗಿ, ಮಹಾಲಕ್ಷ್ಮಿ ಯಿಂದ ಪೂಜೆಗೊಳ್ಳುವ ಶ್ರೀಕೃಷ್ಣನನ್ನು ನೀವೇನೆಂದು ತಿಳಿದಿರುವಿರಿ! ದೇವೇಂದ್ರನೆ ಬಂದು ಅಡ್ಡಬೀಳುವ ಉಗ್ರಸೇನನನ್ನು ಸಿಂಹಾಸನದಿಂದ ಉರುಳಿಸುವಿರಾ? ನೀವೆಲ್ಲ ಕಿರೀಟಗಳು, ನಾವು ಪಾದುಕೆಗಳು; ಎಲಾ, ಅಧಿಕಾರ ಐಶ್ವರ್ಯಗಳ ಸೊಕ್ಕು ತಲೆಗೇರಿ, ಹೆಂಡ ಗುಡುಕರಂತೆ ಹರಟುವ ಕೌರವ ಕುನ್ನಿಗಳೆ, ಈಗಲೆ ನಿಮ್ಮ ಹುಟ್ಟನ್ನು ಅಡಗಿಸಿಬಿಡುತ್ತೇನೆ’ ಎಂದು ಅಬ್ಬರಿಸಿದನು. ಆತನು ವಚನಶೂರನೇನೂ ಅಲ್ಲ, ನುಡಿದಂತೆ ನಡೆಯುವವನು. ತನ್ನ ಆಯುಧವಾದ ನೇಗಿಲನ್ನು ಹಸ್ತಿನಾವತಿಯ ದಕ್ಷಿಣದ ಕೊನೆಗೆ ನಾಟಿ ಬಲವಾಗಿ ಎಳೆದನು. ಒಡನೆಯೆ ಆ ನಗರವೆಲ್ಲ ಕಿತ್ತುಬಂದು, ಗಂಗಾಪ್ರವಾಹದಲ್ಲಿ ತೆಪ್ಪದಂತೆ ತೇಲುವುದಕ್ಕೆ ಆರಂಭವಾಯಿತು. ಆ ನದಿಯ ವೇಗಕ್ಕೆ ಆ ಪಟ್ಟಣವೆಲ್ಲಿ ಮುಳುಗಿಹೋಗು ವುದೋ ಎಂದು ಕೌರವರೆಲ್ಲ ನಡುಗಿದರು. ಅವರು ಬಲರಾಮನಿಗೆ ಶರಣಾಗತರಾಗಿ, ಅತ್ಯಂತ ಭಕ್ತಿಯಿಂದ ‘ಪ್ರಭು, ನೀನು ಸಾಮಾನ್ಯನಲ್ಲ; ಸಾಕ್ಷಾತ್ ಆದಿಶೇಷ. ಈ ಭೂಮಿ ಯನ್ನೆಲ್ಲ ತಲೆಯಲ್ಲಿ ಹೊತ್ತಿರುವ ನಿನಗೆ ನಾವಾಗಲಿ, ಈ ಹಸ್ತಿನಾವತಿಯಾಗಲಿ ಒಂದು ಲೆಕ್ಕವೆ? ನಮ್ಮ ತಪ್ಪುಗಳನ್ನು ಮನ್ನಿಸಿ, ನಮ್ಮನ್ನು ರಕ್ಷಿಸು’ ಎಂದು ಬೇಡಿಕೊಂಡರು. ಅವರ ಪ್ರಾರ್ಥನೆಯಿಂದ ಶಾಂತನಾದ ಬಲರಾಮನು ಅವರಿಗೆ ಅಭಯವನ್ನು ನೀಡಿ, ನಗರಿಯನ್ನು ಅದರ ಸ್ವಸ್ಥಾನಕ್ಕೆ ಎಳೆದು ನಿಲ್ಲಿಸಿದನು. ದುರ್ಯೋಧನನು ತನ್ನ ಮಗಳಿಗೆ ಅಮೂಲ್ಯವಾದ ಅನೇಕ ಬಳುವಳಿಗಳನ್ನಿತ್ತು ಆಕೆಯನ್ನೂ ಸಾಂಬನನ್ನೂ ಬಲರಾಮನಿಗೆ ತಂದೊಪ್ಪಿಸಿದನು. ಅನಂತರ ಬಲರಾಮನು ಮಗ ಸೊಸೆಯರೊಡನೆ ದ್ವಾರಕಿಗೆ ಬಂದು ತಾನು ನಡೆಸಿದುದೆಲ್ಲವನ್ನೂ ರಾಜಸಭೆಗೆ ತಿಳಿಸಿದನು. ಅವರೆಲ್ಲರೂ ಒಕ್ಕೊರಲಿನಿಂದ ಬಲರಾಮನ ಸಾಹಸವನ್ನು ಮೆಚ್ಚಿ, ಆತನನ್ನು ಕೊಂಡಾಡಿದರು.

