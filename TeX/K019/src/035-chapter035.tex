
\chapter{೩೫. ವೈವಸ್ವತನ ಪೀಳಿಗೆ}

\section{(೧) ಸುದ್ಯುಮ್ನ}

ರಾಜರ್ಷಿಯಾದ ಸತ್ಯವ್ರತನು ಶ್ರೀಹರಿಯ ಕೃಪೆಯಿಂದ ವೈವಸ್ವತ ಮನುವಾದುದನ್ನು ಕೇಳಿ, ಪರೀಕ್ಷಿತನ ಕುತೂಹಲ ಕೆರಳಿತು; ಆ ಮನುವಿನ ವಂಶದವರ ಕಥೆಯನ್ನು ಕೇಳ ಬಯಸಿದ. ಶುಕಮುನಿಯು ಅವನ ಕುತೂಹಲದಷ್ಟೆ ಉತ್ಸಾಹದಿಂದ ಅವರ ಕಥೆಯನ್ನು ಹೇಳತೊಡಗಿದ.

ಪ್ರಳಯಕಾಲದಲ್ಲಿ ಪರಮೇಶ್ವರನು ಜಗತ್ತನ್ನೆಲ್ಲ ತನ್ನಲ್ಲಿ ಅಡಗಿಸಿಕೊಂಡು ತಾನೊಬ್ಬನೇ ತನ್ನ ಲೀಲೆಯಲ್ಲಿ ಮಗ್ನನಾಗಿರಲು, ಅವನ ಹೊಕ್ಕಳಿನಿಂದ ಬಂಗಾರದ ಕಮಲವೊಂದು ಹುಟ್ಟಿ ಅರಳಿತು. ಅದರಲ್ಲಿ ಬ್ರಹ್ಮನು ಹುಟ್ಟಿದ. ಆ ಬ್ರಹ್ಮನ ಮನಸ್ಸಿ ನಿಂದ ಮರೀಚಿ ಹುಟ್ಟಿದ. ಆ ಮರೀಚಿಗೆ ಕಶ್ಯಪನೆಂಬ ಮಗನಾದ. ಆ ಕಶ್ಯಪನು ದಕ್ಷನ ಮಗಳಾದ ಅದಿತಿಯನ್ನು ಮದುವೆಯಾಗಿ ಅವಳಲ್ಲಿ ವಿವಸ್ವತನೆಂಬ ಮಗನನ್ನು ಪಡೆದ. ಇವನ ಮಗನೇ ವೈವಸ್ವತಮನು. ರಾಜನಾದ ಈತನಿಗೆ ಬಹುಕಾಲವಾದರೂ ಮಕ್ಕಳಾಗ ಲಿಲ್ಲ. ಆದ್ದರಿಂದ ಆತನು ತನ್ನ ಗುರುವಾದ ವಸಿಷ್ಠರಲ್ಲಿ ತನ್ನ ದುಃಖವನ್ನು ತೋಡಿ ಕೊಂಡನು. ವಸಿಷ್ಠರು ಪುತ್ರಸಂತಾನಕ್ಕಾಗಿ ಆತನಿಂದ ‘ಮಿತ್ರಾವರುಣ’ವೆಂಬ ಒಂದು ಯಾಗವನ್ನು ಮಾಡಿಸಿದರು. ವೈವಸ್ವತನಿಗೆ ಮಗನಾಗಬೇಕೆಂಬ ಆಸೆಯಾದರೆ ಆತನ ಮಡದಿಯಾದ ಶ್ರದ್ಧೆಗೆ ಮಗಳಾಗಬೇಕೆಂಬ ಬಯಕೆ. ಆಕೆ ಹೋಮ ಮಾಡುವವನಲ್ಲಿ ತನ್ನ ಬಯಕೆಯನ್ನು ಹೇಳಿಕೊಂಡಳು. ಆತನು ಹೋಮ ಮಾಡುವಾಗ ಹೆಣ್ಣುಮಗಳಾಗ ಲೆಂದು ಮಂತ್ರ ಹೇಳುತ್ತಾ ಹೋಮಮಾಡಿದ. ಇದರ ಫಲವಾಗಿ ಶ್ರದ್ಧೆಯ ಹೊಟ್ಟೆಯಲ್ಲಿ ಇಳೆಯೆಂಬ ಮಗಳು ಹುಟ್ಟಿದಳು. ಇದನ್ನು ಕಂಡು ವೈವಸ್ವತನಿಗೆ ನಿರಾಶೆಯಾಯಿತು. ಆತ ಗುರುಗಳಾದ ವಸಿಷ್ಠರನ್ನು ಕುರಿತು ‘ಏನು ಸ್ವಾಮಿ, ನೀವು ಮಂತ್ರಗಳನ್ನು ತಿಳಿದವರಲ್ಲಿ ಅಗ್ರಗಣ್ಯರು, ಮಹಾತಪಸ್ವಿಗಳು ಆದರೂ ಗಂಡಾಗಲೆಂದು ನೀವು ಮಾಡಿದ ಯಜ್ಞದಿಂದ ಹೆಣ್ಣೇಕೆ ಹುಟ್ಟಿತು? ಎಂದು ಪ್ರಶ್ನಿಸಿದ. ವಸಿಷ್ಠರು ಕ್ಷಣಕಾಲ ಕಣ್ಣು ಮುಚ್ಚಿ ಧ್ಯಾನ ಮಗ್ನರಾಗುತ್ತಲೆ ಅವರಿಗೆ ಹೆಣ್ಣು ಹುಟ್ಟಿದುದರ ಕಾರಣ ಗೊತ್ತಾಯಿತು. ಅವರು ವೈವಸ್ವತನಿಗೆ ಅದನ್ನು ತಿಳಿಸಿ, ‘ಅಯ್ಯಾ, ಚಿಂತೆಮಾಡಬೇಡ, ನನ್ನ ತಪಸ್ಸಿನ ಬಲದಿಂದ ಈ ಹೆಣ್ಣನ್ನೆ ಗಂಡಾಗಿ ಮಾಡುತ್ತೇನೆ’ ಎಂದು ಹೇಳಿ, ಆ ಹೆಣ್ಣನ್ನು ಗಂಡಾಗಿ ಮಾಡುವಂತೆ ಭಗವಂತನನ್ನು ಪ್ರಾರ್ಥಿಸಿದರು. ಅದರ ಫಲವಾಗಿ ಇಳೆಯು ಗಂಡಾದಳು. ಆ ಗಂಡಿನ ಹೆಸರು ಸುದ್ಯುಮ್ನ.

ಸುದ್ಯುಮ್ನನು ಬೆಳೆದು ದೊಡ್ಡವನಾದ. ಅವನಿಗೆ ಬೇಟೆಯಲ್ಲಿ ತುಂಬ ಆಸಕ್ತಿ. ಒಮ್ಮೆ ಬಿಲ್ಲು ಬಾಣಗಳನ್ನು ಧರಿಸಿ, ತನ್ನ ಪರಿವಾರದೊಡನೆ ಬೇಟೆಗಾಗಿ ಕಾಡಿಗೆ ಹೋದ. ಅಲ್ಲಿ ಆತನು ಮೃಗಗಳನ್ನು ಅಟ್ಟಿಕೊಂಡು ಉತ್ತರದಿಕ್ಕಿನಲ್ಲಿ ಬಹುದೂರ ಹೋದ ಮೇಲೆ, ಮೇರು ಪರ್ವತದ ತಪ್ಪಲಿನಲ್ಲಿದ್ದ ಒಂದು ಉದ್ಯಾನವನ ಕಣ್ಣಿಗೆ ಬಿತ್ತು. ಸುಂದರವಾದ ಆ ವನದಲ್ಲಿ ಅವನು ಕಾಲಿಡುತ್ತಿದ್ದಂತೆಯೇ ಅವನು ಹೆಣ್ಣಾಗಿ ಹೋದ! ಅವನ ಪರಿವಾರ ದವರೂ–ಕುದುರೆ ಕೂಡ–ಹೆಣ್ಣಾದರು. ಅದಕ್ಕೆ ಕಾರಣವಿಲ್ಲದೆ ಇಲ್ಲ. ಅದು ಶಿವ-ಪಾರ್ವತಿಯರ ವಿಹಾರಸ್ಥಾನ. ಒಮ್ಮೆ ಅವರು ಗುಟ್ಟಾಗಿ ವಿಹರಿಸುತ್ತಿರುವಾಗ, ಹಲವು ಪುಷಿಗಳು ಶಿವನನ್ನು ನೋಡಲೆಂದು ಆ ವನಕ್ಕೆ ಹೋದರು. ಆಗ ಪಾರ್ವತಿ ಬೆತ್ತಲೆಯಾಗಿ ಶಿವನ ತೊಡೆಯ ಮೇಲೆ ಕುಳಿತಿದ್ದಳು. ಪುಷಿಗಳನ್ನು ಕಂಡು ಆಕೆಗೆ ನಾಚಿಕೆಯಾಯಿತು, ಅದಕ್ಕೂ ಹೆಚ್ಚಾಗಿ ಅವಮಾನವಾದಂತಾಯಿತು. ಆಕೆ ತಲೆ ತಗ್ಗಿಸಿದಳು. ಆ ಕ್ಷಣವೇ ಅಲ್ಲಿಂದ ಪುಷಿಗಳೇನೋ ಹೊರಟುಹೋದರು. ಆದರೆ ಪಾರ್ವತಿಯ ಮನಸ್ಸಿಗೆ ಮಾತ್ರ ತುಂಬ ಅಸಮಾಧಾನವಾಗಿತ್ತು. ಆಕೆಗೆ ಸಮಾಧಾನವಾಗಲೆಂದು ಶಿವನು ಅಂದು ಶಾಪವಿತ್ತ–‘ಈ ವನವನ್ನು ಹೊಕ್ಕವರು ಹೆಣ್ಣಾಗಲಿ’ ಎಂದು. ಪಾಪ! ಸುದ್ಯುಮ್ನ ಇಂದು ಆ ಶಾಪಕ್ಕೆ ಬಲಿಯಾದ.

ಹೆಣ್ಣಾದ ಸುದ್ಯುಮ್ನ, ಹೆಣ್ಣಾದ ತನ್ನ ಪರಿವಾರದೊಡನೆ ಆ ವನದಲ್ಲಿ ಸುತ್ತುತ್ತ ಸುತ್ತುತ್ತ ಆ ವನವನ್ನು ಬಿಟ್ಟು ಅದರ ಸಮೀಪದಲ್ಲಿದ್ದ ಬುಧನ ಉದ್ಯಾನವನಕ್ಕೆ ಬಂದನು. ಚಂದ್ರನ ಮಗನಾದ ಬುಧನು ಆ ಮೋಹನಾಂಗಿಯನ್ನು ಕಂಡು ಮೋಹಿಸಿದ. ಅವಳೂ ಅವನ ಲಾವಣ್ಯಕ್ಕೆ ಬೆರಗಾಗಿ ಅವನನ್ನು ವರಿಸಿದಳು. ಅವರಿಬ್ಬರ ಪ್ರೇಮದ ಫಲವಾಗಿ ಪುರೂರವಸ್ ಎಂಬ ಮಗ ಹುಟ್ಟಿದ. ಆ ಮಗನನ್ನು ಹೆತ್ತಮೇಲೆ ಆ ಹೆಣ್ಣಿಗೆ ಹೆಣ್ಣುಜನ್ಮ ಸಾಕೆನಿಸಿತು. ಆಕೆ ಗುರುಗಳಾದ ವಸಿಷ್ಠರನ್ನು ನೆನೆದಳು. ‘ನೆನೆದವರ ಮನದಲ್ಲಿ’ ಎಂಬಂತೆ ವಸಿಷ್ಠರು ಆ ಕ್ಷಣವೇ ಅಲ್ಲಿ ಕಾಣಿಸಿಕೊಂಡರು. ಸುದ್ಯುಮ್ನನ ದುರವಸ್ಥೆಯನ್ನು ಕಂಡು ಅವರ ಕರುಳು ಕರಗಿತು. ಅವರು ಪರಶಿವನನ್ನು ಧ್ಯಾನ ಮಾಡಿದರು. ಆಗ ಶಿವನು ಪ್ರತ್ಯಕ್ಷನಾಗಿ ‘ಅಯ್ಯಾ, ಮಹರ್ಷಿ, ನನ್ನ ನುಡಿ ಸಟೆಯಾಗುವುದು ಸಾಧ್ಯವಿಲ್ಲ. ಆದರೂ ನಿನ್ನ ಶಿಷ್ಯನು ಇನ್ನು ಮುಂದೆ ಒಂದು ತಿಂಗಳು ಹೆಣ್ಣು, ಒಂದು ತಿಂಗಳು ಗಂಡು ಆಗಿ ತನ್ನ ರಾಜ್ಯಭಾರವನ್ನು ನಡೆಸಲಿ’ ಎಂದನು. ಸುದ್ಯುಮ್ನನು ಅದೇ ಕ್ರಮದಲ್ಲಿಯೇ ರಾಜ್ಯಭಾರ ಮಾಡುತ್ತಿದ್ದನು. ಹೆಣ್ಣಾಗಿ ಪುರೂರವನನ್ನು ಪಡೆದಂತೆ ಆತನು ಗಂಡಾಗಿದ್ದಾಗ ಆತನಿಗೆ ಉತ್ಕಲ, ಗಯ, ವಿಮಲ–ಎಂಬ ಮೂವರು ಮಕ್ಕಳಾದರು. ಸುದ್ಯುಮ್ನನು ಬಹುಕಾಲ ರಾಜ್ಯಭಾರ ಮಾಡುತ್ತಿದ್ದು, ಮುಪ್ಪು ಬಂದೊಡನೆಯೆ ಹಿರಿಯ ಮಗನಾದ ಪುರೂರವನಿಗೆ ಪಟ್ಟಗಟ್ಟಿ, ತಾನು ತಪಸ್ಸಿಗೆ ಹೊರಟುಹೋದನು.


\section{(೨) ಪೃಷಧ್ರ}

ತನ್ನ ಕಣ್ಣೆದುರಿನಲ್ಲಿಯೇ ತನ್ನ ಮಗ ಸುದ್ಯುಮ್ನ ಮುದುಕನಾಗಿ ತಪಸ್ಸಿಗೆ ಹೋದು ದನ್ನು ಕಂಡು ಮನನೊಂದು ವೈವಸ್ವತಮನು ಯಮುನಾ ತೀರಕ್ಕೆ ಹೋಗಿ, ಮಕ್ಕಳಾಗ ಬೇಕೆಂಬ ಬಯಕೆಯಿಂದ ಶ್ರೀಹರಿಯನ್ನು ಕುರಿತು ನೂರುವರ್ಷದ ಕಾಲ ತಪಸ್ಸು ಮಾಡಿದ. ಆತನಿಗೆ ಶ್ರೀಹರಿಯ ಕೃಪೆಯಿಂದ ಇಕ್ಷ್ವಾಕು ಮೊದಲಾದ ಹತ್ತು ಜನ ಮಕ್ಕಳು ಹುಟ್ಟಿದರು. ಅವರೆಲ್ಲರೂ ತಂದೆಯಷ್ಟೆ ಮಹಾನುಭಾವರು. ಮನುವು ಈ ಮಕ್ಕಳಲ್ಲಿ ಒಬ್ಬೊಬ್ಬನನ್ನೂ ಒಂದೊಂದು ಉದ್ಯೋಗದಲ್ಲಿ ನಿಯೋಜಿಸಿದನು. ಅವರಲ್ಲಿ ಪೃಷಧ್ರ ಎನ್ನುವನು ತಂದೆಯ ಅಪ್ಪಣೆಯಂತೆ ಗೋರಕ್ಷಕನಾದ. ಆತನು ಹಗಲಿರುಳೂ ಕತ್ತಿಯನ್ನು ಕೈಲಿ ಹಿಡಿದು, ಬಹು ಎಚ್ಚರದಿಂದ ಗೋಗಳನ್ನು ಕಾಪಾಡುತ್ತಿದ್ದನು. ಹೀಗಿರಲು, ಒಂದು ರಾತ್ರಿ ದಟ್ಟವಾಗಿ ಮೋಡ ಮುಸುಕಿ, ಭಯಂಕರವಾದ ಮಳೆ ಸುರಿಯಲು ಪ್ರಾರಂಭ ವಾಯಿತು. ಕಗ್ಗತ್ತಲ ಆ ಕಾಳರಾತ್ರಿಯಲ್ಲಿ ಒಂದು ಹುಲಿ ದನದ ಕೊಟ್ಟಿಗೆಗೆ ನುಗ್ಗಿತು. ಅಲ್ಲಿದ್ದ ಗೋವುಗಳೆಲ್ಲ ಭಯದಿಂದ ಅತ್ತಿತ್ತ ಓಡಲಾರಂಭಿಸಿದವು. ಅಷ್ಟರಲ್ಲಿ ಭಯಂಕರವಾದ ಆ ಹುಲಿ ಒಂದು ಒಳ್ಳೆಯ ಆಕಳನ್ನು ಹಿಡಿದೇಬಿಟ್ಟಿತು. ಒಡನೆಯೆ ಅದು ಭಯದಿಂದ ಗಟ್ಟಿಯಾಗಿ ಕೂಗಿಕೊಂಡಿತು. ಅದನ್ನು ಕೇಳಿದ ಪೃಷಧ್ರನು ಹಿರಿದ ಕತ್ತಿ ಯೊಡನೆ ಕೊಟ್ಟಿಗೆಯೊಳಗೆ ನುಗ್ಗಿದನು. ಕಗ್ಗತ್ತಲಿನಲ್ಲಿ ಅವನಿಗೆ ಹುಲಿ ಕಾಣದಿರಲು, ಕೂಗು ಬಂದ ಕಡೆ ಕತ್ತಿಯನ್ನು ಬೀಸಿದನು, ಅದು ಹುಲಿಗೆ ಬದಲಾಗಿ ಒಂದು ಆಕಳಿಗೆ ತಗಲಿ, ಆ ಆಕಳು ಎರಡು ತುಂಡಾಗಿ ಬಿತ್ತು. ಕತ್ತಿಯ ಮೊನೆ ಹುಲಿಯ ಕಿವಿಗೆ ಮಾತ್ರ ತಾಕಿ, ಅದರ ಕಿವಿ ಕತ್ತರಿಸಿಹೋಯಿತು. ಭಯಗೊಂಡ ಹುಲಿ ನೆತ್ತರು ಸೋರುತ್ತಿರುವ ಕಿವಿಯೊಡನೆ ಅಡವಿಯೊಳಕ್ಕೆ ಓಡಿ ಹೋಯಿತು. ಪೃಷಧ್ರನು ಹುಲಿಯೇ ಸತ್ತಿತೆಂದು ಸಮಾಧಾನ ಮಾಡಿಕೊಂಡು ಅಲ್ಲಿಂದ ಹಿಂದಿರುಗಿದನು.

ಮರುದಿನ ಬೆಳಗ್ಗೆ ಪೃಷಧ್ರ ಹಸುಗಳ ಕೊಟ್ಟಿಗೆಗೆ ಹೋಗಿ ನೋಡಿದಾಗ, ಹುಲಿಗೆ ಬದಲಾಗಿ ಸೊಗಸಾದ ಆಕಳೊಂದು ಸತ್ತು ಬಿದ್ದಿರುವುದನ್ನು ಕಂಡು, ಅವನ ಜೀವ ತತ್ತರಿ ಸಿತು. ಗೋವಧೆ ಪಂಚಮಹಾಪಾತಕಗಳಲ್ಲಿ ಒಂದು. ಪೃಷಧ್ರ ದುಃಖದಿಂದ ಮುಂಗಾಣ ದವನಾಗಿ, ಗುರುಗಳಾದ ವಸಿಷ್ಠರ ಬಳಿಗೆ ಓಡಿದನು. ಅವರು ನಡೆದ ಕಥೆಯನ್ನೆಲ್ಲ ಕೇಳಿ ತುಂಬ ಕೋಪಗೊಂಡರು. ತಿಳಿಯದೆ ನಡೆದುಹೋದ ಕಾರ್ಯವಾದರೂ ಅವರಿಗೆ ಪೃಷಧ್ರ ನಲ್ಲಿ ಮರುಕ ಹುಟ್ಟಲಿಲ್ಲ. ಅವರು ಕೋಪದಿಂದ ಕುರುಡರಾಗಿ ಹೋಗಿದ್ದರೆಂದು ತೋರು ತ್ತದೆ. ‘ಪಾಪಿ, ನೀನು ಕ್ಷತ್ರಿಯನಾಗಿರಲು ಯೋಗ್ಯನಲ್ಲ; ಶೂದ್ರನಾಗಿ ಹೋಗು’ ಎಂದು ಅವರು ಶಪಿಸಿದರು. ಪೃಷಧ್ರ ಅಧರ್ಮಕ್ಕೆ ಹೆದರುವನೇ ಹೊರತು ಶಾಪಕ್ಕಲ್ಲ. ಅವನು ತಾಳ್ಮೆಯಿಂದ ಆ ಶಾಪವನ್ನು ಸ್ವೀಕರಿಸಿದನು. ಅನಂತರ ಆತನು ಮನೆಯನ್ನು ಬಿಟ್ಟು ಹೊರಟು, ತಪಸ್ಸಿನಿಂದ ಇಂದ್ರಿಯಗಳನ್ನು ಜಯಿಸಿ ಭಗವಂತನ ಆರಾಧನೆಯಲ್ಲಿ ತೊಡಗಿದನು. ಬಹು ಬೇಗ ಆತನಿಗೆ ನಿಷ್ಕಾಮಭಕ್ತಿ ಸಿದ್ಧಿಸಿತು. ಆತನು ಜಗತ್ತನ್ನೆಲ್ಲ ಸರ್ವ ಸಮಭಾವದಿಂದ ಕಾಣುತ್ತ, ರಾಗಭೋಗಗಳಲ್ಲಿ ವಿರಕ್ತನಾದನು. ಅನಂತರ ಆತನು ಸನ್ಯಾಸಿಯಾಗಿ ದೇಶಸಂಚಾರ ಮಾಡುತ್ತಾ, ಹೊತ್ತಿಗೆ ಸಿಕ್ಕಿದುದನ್ನು ತಿಂದು ಕಾಲ ಹಾಕು ತ್ತಿದ್ದನು. ಸದಾ ದೈವಚಿಂತನೆಯಲ್ಲಿದ್ದ ಆತನು ಜ್ಞಾನಿಯಾಗಿ ಬ್ರಹ್ಮಾನಂದವನ್ನು ಸವಿ ಯುತ್ತಾ ತನ್ನ ಆಯುಸ್ಸನ್ನು ಸವೆಸಿದನು. ಕಡೆಗಾಲದಲ್ಲಿ ಆತನು ಕಾಡುಗಿಚ್ಚೊಂದನ್ನು ಕಂಡು, ಅದರಲ್ಲಿ ಪ್ರವೇಶಿಸಿ, ಸಾಯುಜ್ಯ ಪದವಿಯನ್ನು ಪಡೆದನು.


\section{(೩) ಸತೀ ಸುಕನ್ಯೆ}

ವೈವಸ್ವತ ಮನುವಿನ ಪೀಳಿಗೆಯಲ್ಲಿ ಹುಟ್ಟಿದ ಒಬ್ಬೊಬ್ಬನೂ ಮಹಾಪುರುಷನೆ. ಆತನ ಆ ಮಕ್ಕಳಲ್ಲಿ ಒಬ್ಬನಾದ ಶರ್ಯಾತಿಯು ಕ್ಷತ್ರಿಯನಾದರೂ ಬ್ರಹ್ಮಜ್ಞಾನಿಯಾಗಿದ್ದನು. ಒಮ್ಮೆ ಆತನು ಅಂಗಿರಸಮಹಾಮುನಿಯ ಯಾಗದಲ್ಲಿ ಎರಡನೆಯ ದಿವಸದ ಕರ್ಮಗಳ ನ್ನೆಲ್ಲ ಹೀಗೆಯೇ ಮಾಡಬೇಕೆಂದು ಪುರೋಹಿತರಿಗೇ ಉಪದೇಶಿಸಿದ್ದನಂತೆ! ಈ ಪುಣ್ಯ ಪುರುಷನಿಗೆ ಸುಕನ್ಯೆ ಎಂಬ ದಿವ್ಯ ಸುಂದರಿಯಾದ ಮಗಳೊಬ್ಬಳಿದ್ದಳು. ಆಕೆಯಲ್ಲಿ ಪಂಚಪ್ರಾಣ, ಶರ್ಯಾತಿ ರಾಜನಿಗೆ. ಒಂದು ದಿನ ಆತನು ಪರಿವಾರದೊಡನೆ ವನ ವಿಹಾರಕ್ಕೆ ಹೊರಟು, ಈ ಮುದ್ದು ಮಗಳನ್ನೂ ಜೊತೆಯಲ್ಲಿ ಕರೆದೊಯ್ದ. ಅವರೆಲ್ಲರೂ ಚ್ಯವನಪುಷಿಯ ಆಶ್ರಮಕ್ಕೆ ಹೋದರು. ಅಲ್ಲಿ ಸುಕನ್ಯೆ ತನ್ನ ಸಖಿಯರೊಡನೆ ವನದ ಸೊಬಗನ್ನು ಸವಿಯುತ್ತಾ ವಿಹರಿಸುತ್ತಿರಲು, ದೊಡ್ಡದೊಂದು ಹುತ್ತ ಅವಳ ಕಣ್ಣಿಗೆ ಬಿತ್ತು. ಅವಳು ಅದರ ಹತ್ತಿರಕ್ಕೆ ಹೋಗಿ, ಅದರಲ್ಲಿದ್ದ ರಂಧ್ರದಲ್ಲಿ ಇಣಿಕಿ ನೋಡಿದಳು. ಒಳಗೆ ವಜ್ರದಂತೆ ನಿಗಿನಿಗಿ ಹೊಳೆಯುತ್ತಿರುವ ಎರಡು ವಸ್ತುಗಳು ಕಾಣಿಸಿದುವು. ಹುಡುಗುತನದ ಕುತೂಹಲದಿಂದ ಅವಳು ಒಂದು ಮುಳ್ಳನ್ನು ತೆಗೆದುಕೊಂಡು ಆ ವಸ್ತು ಗಳಿಗೆ ಚುಚ್ಚಿದಳು. ಒಡನೆಯೆ ಅವುಗಳಿಂದ ಧಾರೆಧಾರೆಯಾಗಿ ನೆತ್ತರು ಹರಿಯಲು ಪ್ರಾರಂಭವಾಯಿತು. ಇದನ್ನು ಕಂಡು ಹುಡುಗಿಗೆ ತುಂಬ ಭಯವಾಯಿತು. ಅವಳು ಬಿಡಾರಕ್ಕೆ ಓಡಿದಳು.

ಅತ್ತ ಹುತ್ತದಿಂದ ನೆತ್ತರು ಹರಿಯಲು ಪ್ರಾರಂಭವಾಗುತ್ತಲೆ, ಇತ್ತ ಅರಸನ ಪರಿವಾರ ದವರಿಗೆಲ್ಲ ಮಲಮೂತ್ರಗಳು ಕಟ್ಟಿಹೋದವು. ಅವರೆಲ್ಲ ಸಂಕಟದಿಂದ ಲಿಬಿಲಿಬಿ ಒದ್ದಾ ಡಲು ಪ್ರಾರಂಭಿಸಿದರು. ಇದನ್ನು ಕಂಡು ರಾಜನಿಗೆ ಆಶ್ಚರ್ಯವಾಯಿತು. ಇದ್ದಕ್ಕಿ ದ್ದಂತೆಯೆ ಹೀಗಾಗಬೇಕಾಗಿದ್ದರೆ ಆಶ್ರಮದಲ್ಲಿದ್ದ ಚ್ಯವನಮಹರ್ಷಿಗೆ ಏನೋ ಅಪಚಾರ ನಡೆದಿರಬೇಕೆಂದು ಆತನಿಗೆ ಭಯವೂ ಹುಟ್ಟಿತು. ಆತನು ತನ್ನ ಪರಿವಾರದವರನ್ನೆಲ್ಲ ಕರೆದು ವಿಚಾರಿಸಿದನು. ಯಾರೂ ಯಾವ ತಪ್ಪನ್ನೂ ಮಾಡಿದಂತೆ ಕಂಡುಬರಲಿಲ್ಲ. ಆ ವೇಳೆಗೆ ಅಲ್ಲಿಗೆ ಬಂದ ಸುಕನ್ಯೆ ಭಯದಿಂದ ಗಡಗಡ ನಡುಗುತ್ತಾ ನಡೆದುದನ್ನು ನಡೆದಂತೆ ಆತನಿಗೆ ತಿಳಿಸಿದಳು. ಒಡನೆಯೆ ರಾಜನು ಮಗಳೊಡನೆ ಹುತ್ತದ ಬಳಿಗೆ ಹೋದನು. ಅಲ್ಲಿ ಆಗಿದ್ದ ಅನರ್ಥವನ್ನು ಕಂಡು ರಾಜನು ನಡುಗಿಹೋದನು. ಆ ಹುತ್ತದಲ್ಲಿ ಸ್ವತಃ ಚ್ಯವನ ಪುಷಿಯೇ ಕುಳಿತು ತಪಸ್ಸು ಮಾಡುತ್ತಿದ್ದಾನೆ! ಸುಕನ್ಯೆ ಮುಳ್ಳಿನಿಂದ ಚುಚ್ಚಿರುವುದು ಆತನ ಕಣ್ಣುಗಳನ್ನು! ಶರ್ಯಾತಿಯು ಆ ಮಹಾಮುನಿಯ ಮುಂದೆ ಅಡ್ಡ ಬಿದ್ದು, ತನ್ನ ಮಗಳ ತಪ್ಪನ್ನು ಕ್ಷಮಿಸುವಂತೆ ಬೇಡಿಕೊಂಡನು. ಕ್ಷಮಿಸುವುದೇನೋ ಸರಿ, ಕಣ್ಣಿಲ್ಲದ ಆ ಪುಷಿ ಯನ್ನು ಮುಂದೆ ನೋಡಿಕೊಳ್ಳುವವರು ಯಾರು? ದಯಾಮಯನಾದ ರಾಜನು ಆ ಕಾರ್ಯಕ್ಕೆ ತನ್ನ ಕಣ್ಮಣಿಯಾದ ಮಗಳನ್ನೆ ನೇಮಿಸಿದನು. ಮುಪ್ಪಿನ ಮುದುಕನಾದ ಚ್ಯವನಪುಷಿಗೆ ತನ್ನ ಮಗಳನ್ನು ಕೊಟ್ಟು ವಿವಾಹಮಾಡಿ, ಆಕೆಯನ್ನು ಆತನ ಸೇವೆಯಲ್ಲಿ ನಿಲ್ಲಿಸಿದನು. ಆಗ ಪರಿವಾರಕ್ಕೆ ಒದಗಿದ್ದ ಸಂಕಟ ನಿವಾರಣೆಯಾಯಿತು. ರಾಜನು ಅಳಿಯನಿಂದಲೂ ಮಗಳಿಂದಲೂ ಬೀಳ್ಕೊಂಡು ರಾಜಧಾನಿಗೆ ಹಿಂದಿರುಗಿದನು.

ಸುಕನ್ಯೆ ತನಗೆ ಅನ್ಯಾಯವಾಯಿತೆಂದು ಮರುಗಲಿಲ್ಲ. ಪಾಲಿಗೆ ಬಂದುದು ಪಂಚಾ ಮೃತವೆಂದುಕೊಂಡು ಮುದಿಗಂಡನ ಸೇವೆಯಲ್ಲಿ ತತ್ಪರಳಾದಳು. ಮುನಿಗಳಿಗೆ ಮೂಗಿನ ಮೇಲೆಯೇ ಕೋಪ. ಇದನ್ನರಿತ ಸುಕನ್ಯೆ ಮೈಯಲ್ಲ ಕಣ್ಣಾಗಿ ನಡೆದುಕೊಳ್ಳುತ್ತ ಆತನನ್ನು ಆನಂದಪಡಿಸುತ್ತಿದ್ದಳು. ಹೀಗೆ ಕೆಲವು ಕಾಲ ಕಳೆಯಲು ಒಂದು ದಿನ ದೇವವೈದ್ಯರಾದ ಅಶ್ವಿನೀದೇವತೆಗಳು ಚ್ಯವನಮಹರ್ಷಿಯ ದರ್ಶನಕ್ಕಾಗಿ ಆತನ ಬಳಿಗೆ ಬಂದರು. ಪುಷಿಯು ಆ ಅತಿಥಿಗಳನ್ನು ಅತ್ಯಂತ ಆದರದಿಂದ ಕಂಡು ಸತ್ಕರಿಸಿದನು. ಇದರಿಂದ ಅವರು ಸಂತೋಷಗೊಂಡಿರುವಾಗ ಪುಷಿಯು ‘ಅಯ್ಯಾ ದೇವತೆಗಳೆ, ಮುದುಕನಾದ ನನ್ನನ್ನು ಯುವಕನಾಗಿ ಮಾಡಬಾರದೆ? ನನಗೆ ಈಗತಾನೆ ಮದುವೆಯಾಗಿದೆ; ಹುಡುಗಿ ಯಾದ ಆ ಹೆಂಡತಿಯ ಕಣ್ಣಿಗೆ ಸಂತೋಷವಾಗುವಂತಹ ರೂಪವನ್ನು ನೀವು ನನಗೆ ದಯಪಾಲಿಸುವಿರಾದರೆ, ಇನ್ನು ಮುಂದೆ ಯಾಗಗಳಲ್ಲೆಲ್ಲ ನಿಮಗೆ ಸೋಮಭಾಗವನ್ನು ಕೊಡಿಸುತ್ತೇನೆ’ ಎಂದ. ಸೋಮಭಾಗದ ಹೆಸರನ್ನು ಕೇಳಿ ಅಶ್ವಿನೀದೇವತೆಗಳ ಬಾಯಲ್ಲಿ ನೀರೂರಿತು. ಪಾಪ, ದೇವವೈದ್ಯರಾದರೂ ಅವರಿಗೆ ಸೋಮವನ್ನು ಕುಡಿಯುವ ಭಾಗ್ಯ ಆವರೆಗೆ ದೊರೆತಿರಲಿಲ್ಲ. ಅವರು ಚ್ಯವನಪುಷಿಯ ಮಾತಿಗೆ ತಕ್ಷಣವೇ ಒಪ್ಪಿಕೊಂಡರು. ಅವರು ಪುಷಿಯನ್ನು ಕೈಹಿಡಿದು ಕರೆತಂದು ಒಂದು ಸಿದ್ಧರಸದ ಮಡುವಿನಲ್ಲಿ ಆತನೊಡನೆ ಹಾರಿಕೊಂಡರು. ಮರು ನಿಮಿಷದಲ್ಲಿ ಮೂರು ದಿವ್ಯ ಸುಂದರ ವಿಗ್ರಹಗಳು ಅದರಿಂದ ಹೊರಗೆ ಬಂದವು.

ಸುಕನ್ಯೆಗೆ ಆ ಮೂವರಲ್ಲಿ ತನ್ನ ಗಂಡ ಯಾರೆಂದು ಹೇಗೆ ಗೊತ್ತಾಗಬೇಕು? ಮೂವರ ರೂಪಸಂಪತ್ತು ಒಂದೇ ಬಗೆಯಾಗಿದೆ. ಮೂವರೂ ಧರಿಸಿರುವ ವಸ್ತ್ರಾಭರಣಗಳೂ ಒಂದೇ ರೀತಿಯಾಗಿವೆ. ಮೂವರ ಕೊರಳಲ್ಲಿಯೂ ಒಂದೇ ತರಹದ ಹೂಮಾಲೆಗಳು. ನರೆತ ಕೂದಲು, ಸುಕ್ಕುಬಿದ್ದ ತೊಗಲು, ಉಬ್ಬಿದ ನರಗಳಿದ್ದ ವಿಕಾರರೂಪಿನ ತನ್ನ ಗಂಡ ಈ ಸಮಾನ ರೂಪಿನ ಸುಂದರ ಯುವಕರಲ್ಲಿ ಯಾರೆಂದು ಹೇಗೆ ಕಂಡು ಹಿಡಿಯುವುದು? ಸುಕನ್ಯೆ ಕ್ಷಣಕಾಲ ಮುಂದೋರದೆ ನಿಂತಿದ್ದು, ಅನಂತರ ಅಶ್ವಿನೀ ದೇವತೆಗಳನ್ನೆ ಧ್ಯಾನ ಮಾಡುತ್ತಾ, ತನಗೆ ಪತಿಭಿಕ್ಷೆಯನ್ನು ನೀಡುವಂತೆ ಬೇಡಿಕೊಂಡಳು. ಅವರು ಆಕೆಯ ಪ್ರಾರ್ಥನೆಗೆ ಮೆಚ್ಚಿ ಸ್ವಸ್ವರೂಪವನ್ನು ತಾಳಿದರು. ಯುವಕನಾದ ಚ್ಯವನನು ಮುಗು ಳ್ನಗುತ್ತಾ ಮಡದಿಯ ಕೈಹಿಡಿದನು. ಅಶ್ವಿನೀದೇವತೆಗಳು ಆ ದಂಪತಿಗಳಿಂದ ಬೀಳ್ಕೊಂಡು ಸ್ವಸ್ಥಾನಕ್ಕೆ ಹಿಂದಿರುಗಿದರು.

ಚ್ಯವನ ಸುಕನ್ಯೆಯರು ಜಕ್ಕವಕ್ಕಿಗಳಂತೆ ಸಂಸಾರಸುಖದ ಸಾರವನ್ನು ಹೀರುತ್ತಾ ಯುಗ ವೊಂದು ಕ್ಷಣವಾಗಿ ಕಳೆಯುತ್ತಿರಲು, ಒಂದು ದಿನ ಶರ್ಯಾತಿ ಮಹಾರಾಜನು ತಾನು ಮಾಡಬೇಕೆಂದುಕೊಂಡಿದ್ದ ಒಂದು ಯಾಗಕ್ಕೆ ಅಳಿಯನನ್ನೂ ಮಗಳನ್ನೂ ಕರೆದು ಕೊಂಡು ಹೋಗಬೇಕೆಂದು ಚ್ಯವನಪುಷಿಯ ಆಶ್ರಮಕ್ಕೆ ಬಂದನು. ಅಲ್ಲಿ ಬಂದು ನೋಡಿದರೆ ತನ್ನ ಮಗಳು ಸುಂದರನಾದ ತರುಣನೊಬ್ಬನೊಡನೆ ಸರಸ ಸಲ್ಲಾಪದಲ್ಲಿ ತೊಡಗಿದ್ದಾಳೆ. ಆತನಿಗೆ ರೇಗಿಹೋಯಿತು. ತನಗೆ ಬಂದು ನಮಸ್ಕರಿಸಿದ ಮಗಳನ್ನು ಹರಸುವುದಕ್ಕೆ ಬದಲಾಗಿ ‘ಅಯ್ಯೋ ನೀಚಳೆ, ಎಂತಹ ಅನ್ಯಾಯ ಮಾಡಿದೆ? ಲೋಕಪೂಜ್ಯ ನಾದ ಮಹರ್ಷಿ, ನಿನ್ನ ಗಂಡ! ಅಂತಹ ಮಹಾನುಭಾವನಿಗೆ ಮೋಸಮಾಡಿದೆಯಲ್ಲಾ, ನೀನು! ಆತ ಮುದುಕನೆಂದು ಹಾದಿಹೋಕನೊಬ್ಬನನ್ನು ಕಟ್ಟಕೊಂಡೆಯಾ? ಆತ ಕುರೂಪಿ ಯೆಂದು ಸುಂದರ ವಿಟಪುರುಷನಿಗೆ ದೇಹವನ್ನು ಮಾರಿಕೊಂಡೆಯಾ? ಅಯ್ಯೋ ಪಾಪಿ ಹೆಣ್ಣೆ, ನನ್ನ ವಂಶಕ್ಕೆ ಎಂತಹ ಅಪಕೀರ್ತಿಯನ್ನು ತಂದೆ! ಹುಟ್ಟಿದ, ಸೇರಿದ ವಂಶ ಗಳೆರಡನ್ನೂ ನರಕದಲ್ಲಿ ಅದ್ದಿದೆಯಲ್ಲಾ!’ ಎಂದು ಕಿರಿಚಿಕೊಂಡನು. ಅಪ್ಪನ ಈ ಮಾತು ಗಳನ್ನು ಕೇಳಿ ಮಗಳಿಗೆ ನಗು ಬಂತು. ಆಕೆ ಮುಗುಳ್ನಗುತ್ತಾ ‘ಅಪ್ಪಾ, ಇವರೇ ನಿನ್ನ ಅಳಿಯಂದಿರು. ಅಶ್ವಿನೀ ದೇವತೆಗಳ ಅನುಗ್ರಹದಿಂದ ಇವರು ಸುಂದರ ಯುವಕರಾಗಿ ದ್ದಾರೆ. ಈ ನನ್ನ ಪತಿದೇವರನ್ನು ವಿಟರೆಂದು ಕರೆಯುವುದೆ?’ ಎಂದಳು. ಆಕೆಯಿಂದ ನಡೆದ ವೃತ್ತಾಂತವನ್ನೆಲ್ಲ ಕೇಳಿ ಶರ್ಯಾತಿಗೆ ಅತ್ಯಂತ ಸಂತೋಷವಾಯಿತು; ಮಗಳನ್ನು ಬಾಚಿ ತಬ್ಬಿಕೊಂಡು ಮುದ್ದಾಡಿದನು.

ಚ್ಯವನ ಸುಕನ್ಯೆಯರು ಶರ್ಯಾತಿರಾಜನ ಯಾಗಕ್ಕೆ ಹೋದರು. ಚ್ಯವನ ಮಹರ್ಷಿಯೆ ಪುರೋಹಿತನಾಗಿ, ಆ ಯಾಗವನ್ನು ಸಾಂಗವಾಗಿ ನೆರವೇರಿಸಿ, ಅಶ್ವಿನೀ ದೇವತೆಗಳಿಗೆ ಹಿಂದೆಂದೂ ಇಲ್ಲದ ಸೋಮಭಾಗವನ್ನು ಸಲ್ಲಿಸಿದನು. ಇದರಿಂದ ಇಂದ್ರನಿಗೆ ಕೋಪ ಬಂತು; ಚ್ಯವನಮಹರ್ಷಿಯನ್ನು ಕೊಂದುಬಿಡಬೇಕೆಂದು ತನ್ನ ವಜ್ರಾಯುಧವನ್ನು ಕೈಗೆ ಎತ್ತಿಕೊಂಡನು. ಆದರೆ ಆ ಮಹರ್ಷಿಯ ಪ್ರಭಾವದಿಂದ ಅವನ ಕೈ ಮಿಸುಕಾಡದಂತಾ ಯಿತು. ಇದನ್ನು ಕಂಡು, ದೇವತೆಗಳೆಲ್ಲ ಸೇರಿ, ಇನ್ನು ಮುಂದೆ ಅಶ್ವಿನೀ ದೇವತೆಗಳಿಗೆ ಸೋಮಭಾಗವನ್ನು ಕೊಡಬಹುದೆಂದು ತೀರ್ಮಾನಿಸಿದರು. ಇದರಿಂದ ಇಂದ್ರನ ಕೈ ಸರಿ ಹೋಯಿತು. ಚ್ಯವನಪುಷಿಯ ಕೀರ್ತಿ ಹಬ್ಬಿ ಹರಡಿತು. ಆತನು ಸುಕನ್ಯೆಯೊಡನೆ ಇಹ ದಲ್ಲಿ ಪರಮ ಸುಖಿಯಾಗಿದ್ದು, ತನ್ನ ಸತ್ಕರ್ಮಗಳಿಂದ ಸದ್ಗತಿಯನ್ನೂ ಪಡೆದನು.


\section{(೪) ನಾಭಾಗ}

ವೈವಸ್ವತಮನು ಹರಿಕೃಪೆಯಿಂದ ಪಡೆದ ಹತ್ತು ಮಕ್ಕಳಲ್ಲಿ ಎಂಟನೆಯವನು ನಭಗ. ಈ ನಭಗರಾಜನ ಹಲವು ಮಕ್ಕಳಲ್ಲಿ ಕಿರಿಯನಾದವನು ನಾಭಾಗ. ಅವನಿಗೆ ವಿದ್ಯೆಯ ಗೀಳು ಹೆಚ್ಚು. ಬಹು ಕಾಲದವರೆಗೆ ವಿದ್ಯಾಭ್ಯಾಸ ಮಾಡುತ್ತಾ, ಗುರುಕುಲದಲ್ಲಿಯೇ ನಿಂತುಬಿಟ್ಟ. ಅವನ ಅಣ್ಣಂದಿರು ತಮ್ಮ ತಂದೆಯು ತಪಸ್ಸಿಗೆ ಹೋಗುತ್ತಲೆ, ನಾಭಾಗನನ್ನು ಬಿಟ್ಟು ತಮ್ಮ ತಮ್ಮಲ್ಲಿಯೆ ರಾಜ್ಯವನ್ನು ಹಂಚಿಕೊಂಡರು. ಇದಾದ ಬಹುಕಾಲದ ಮೇಲೆ ನಾಭಾಗ ಗುರುಕುಲದಿಂದ ಹಿಂದಿರುಗಿ, ತನ್ನ ಭಾಗದ ರಾಜ್ಯವನ್ನು ಕೊಡುವಂತೆ ಅವರನ್ನು ಕೇಳಿದ. ಅವರು ‘ತಮ್ಮ, ನಿನಗೆ ಪಾಲು ಕೊಡಬೇಕೆಂಬುದು ನಮಗೆ ಮರತೇಹೋಯಿತು. ಈಗ ಮಾಡುವುದೇನು? ನಮ್ಮ ತಂದೆಯನ್ನೆ ನೀನು ನಿನ್ನ ಪಾಲಿಗೆ ತೆಗೆದುಕೊ’ ಎಂದರು. ಅವನು ಹಾಗೆಯೇ ಆಗಲೆಂದು ತಂದೆಯ ಬಳಿಗೆ ಹೋಗಿ, ಅದನ್ನು ಆತನಿಗೆ ತಿಳಿಸಿದ. ಅದನ್ನು ಕೇಳಿ ನಭಗನು ‘ಮಗು, ಅವರು ನಿನಗೆ ಮೋಸಮಾಡಿದ್ದಾರೆ. ಅವರ ಮಾತನ್ನು ನಂಬಬೇಡ. ಹೋಗಲಿ, ನಿನ್ನ ಪಾಲಿಗೆ ಅವರು ನನ್ನನ್ನು ಕೊಟ್ಟಿರುವುದರಿಂದ ನಿನಗೆ ಹೊಟ್ಟೆ ಹೊರೆಯುವ ಉಪಾಯವನ್ನು ನಾನು ನಿನಗೆ ಹೇಳಿಕೊಡುತ್ತೇನೆ. ಇಲ್ಲಿಗೆ ಹತ್ತಿರ ದಲ್ಲಿಯೇ ಅಂಗಿರಸರೆಂಬ ಪುಷಿಗಳು ಹನ್ನೆರಡು ದಿನಗಳು ನಡೆಯಬೇಕಾದ ಸತ್ರಯಾಗ ವೊಂದನ್ನು ಪ್ರಾರಂಭಿಸಿದ್ದಾರೆ. ಅವರು ಬಹು ದೊಡ್ಡ ಪಂಡಿತರಾದರೂ ಆರನೆಯ ದಿನದ ಕರ್ಮವನ್ನು ಹೇಗೆ ಮಾಡಬೇಕೆಂದು ಅವರಿಗೆ ಗೊತ್ತಿಲ್ಲ. ಅಂದು ಅವರು ಮುಂದೋರದೆ ಮರುಳರಂತೆ ಕುಳಿತುಕೊಳ್ಳುತ್ತಾರೆ. ಆಗ ನೀನು ಅಲ್ಲಿಗೆ ಹೋಗಿ, ಅವರಿಗೆ ಅತ್ಯಗತ್ಯವಾಗಿ ಬೇಕಾಗುವ ಎರಡು ಮಂತ್ರಗಳನ್ನು ಹೇಳಿಕೊಡು. ಅವರು ಯಾಗದ ಕೊನೆಯಲ್ಲಿ ಸ್ವರ್ಗಕ್ಕೆ ಹೋಗುವ ಮುನ್ನ ತಮ್ಮ ಯಾಗದಲ್ಲಿ ಉಳಿದ ಅಪಾರವಾದ ಧನವನ್ನು ನಿನಗೆ ಕೊಟ್ಟು ಹೋಗುತ್ತಾರೆ’ ಎಂದು ಹೇಳಿದ. ನಾಭಾಗನು ಹಾಗೆಯೇ ಮಾಡಿದನು. ಅದರಿಂದ, ತಂದೆ ಹೇಳಿದಂತೆಯೆ ಅವನಿಗೆ ಅಪಾರವಾದ ಧನ ದೊರೆಯಿತು.

ನಾಭಾಗನು ಯಾಗಶಾಲೆಯಿಂದ ಹಣದೊಡನೆ ಹೊರ ಹೊರಡಬೇಕೆನ್ನುವಷ್ಟರಲ್ಲಿಯೇ ಕಪ್ಪು ದೇಹದ ಮಹಾಪುರುಷನೊಬ್ಬ ಅಲ್ಲಿ ಕಾಣಿಸಿಕೊಂಡು ‘ನಿಲ್ಲು, ಯಾಗದಲ್ಲಿ ಉಳಿದು ದೆಲ್ಲವೂ ನನಗೆ ಸೇರಬೇಕು’ ಎಂದು ಆತ ಗರ್ಜಿಸಿದ. ಯಜ್ಞಮಾಡಿದವನೇ ಅದನ್ನು ತನಗೆ ಕೊಟ್ಟಿರುವನೆಂದು ನಾಭಾಗ ತಿಳಿಸಿದರೂ ಅವನು ಒಪ್ಪಲಿಲ್ಲ. ಕೊನೆಗೆ ಆ ಮನುಷ್ಯ ‘ಅಯ್ಯಾ ನಾಭಾಗ, ಈ ವಿಚಾರವಾಗಿ ನಾವು ಕಿತ್ತಾಡುವುದು ಬೇಡ. ನಿಮ್ಮ ತಂದೆಯ ಹತ್ತಿರವೇ ಹೋಗಿ ಕೇಳೋಣ. ಆತನ ತೀರ್ಮಾನದಂತೆ ನಾವಿಬ್ಬರೂ ನಡೆದುಕೊಳ್ಳೋಣ’ ಎಂದು ತಿಳಿಸಿದ. ಇಬ್ಬರೂ ನಭಗನ ಬಳಿಗೆ ಬಂದರು. ಆತನು ಅವರ ವಿವಾದವನ್ನು ಕೇಳಿ, ಕಡೆಯಲ್ಲಿ ಮಗನೊಡನೆ ‘ಮಗು, ನಿನ್ನ ಎದುರಾಳಿ ಸಾಕ್ಷಾತ್ ರುದ್ರದೇವರು. ಹಿಂದೆ ದಕ್ಷನು ಯಾಗ ಮಾಡಿದಾಗ, ಯಾಗದಲ್ಲಿ ಉಳಿದುದೆಲ್ಲ ರುದ್ರನಿಗೆ ಸಲ್ಲತಕ್ಕುದೆಂದು ಪುಷಿಗಳು ತೀರ್ಮಾನಿಸಿದರು. ಆದ್ದರಿಂದ ಈಗ ನಿನಗೆ ದೊರೆತ ಹಣವೆಲ್ಲ ಆತನಿಗೆ ಸಲ್ಲ ತಕ್ಕುದೆ’ ಎಂದನು. ಈ ಮಾತನ್ನು ಕೇಳುತ್ತಲೆ ನಾಭಾಗನು ರುದ್ರನಿಗೆ ಅಡ್ಡಬಿದ್ದು, ಹಣ ವನ್ನೆಲ್ಲ ಒಪ್ಪಿಸಿ, ತನ್ನನ್ನು ಕ್ಷಮಿಸುವಂತೆ ಬೇಡಿಕೊಂಡನು. ಇದನ್ನು ಕಂಡು ರುದ್ರನಿಗೆ ಅತ್ಯಂತ ಸಂತೋಷವಾಯಿತು. ಆತನು ‘ನಾಭಾಗ, ನಿನ್ನ ಸತ್ಯಕ್ಕೂ ನಿಮ್ಮಪ್ಪನ ಧರ್ಮ ಬುದ್ಧಿಗೂ ನಾನು ಮೆಚ್ಚಿದ್ದೇನೆ. ಈ ಹಣವನ್ನು ನೀನೇ ತೆಗೆದುಕೊ. ಇದರ ಜೊತೆಗೆ ನಿನಗೆ ನಾನು ಬ್ರಹ್ಮಜ್ಞಾನವನ್ನೂ ಉಪದೇಶಿಸುತ್ತೇನೆ. ನೀನು ಎಂತಹ ಕಷ್ಟವೇ ಬರಲಿ, ಸತ್ಯ ವನ್ನು ಬಿಡಬೇಡ. ಸತ್ಯ ನನಗೆ ಅತ್ಯಂತ ಪ್ರಿಯ’ ಎಂದು ಹೇಳಿ, ಅವನಿಗೆ ಉಪದೇಶವನ್ನು ಕೊಟ್ಟು ಮಾಯವಾದನು.


\section{(೫) ಅಂಬರೀಷ}

ಬ್ರಹ್ಮಜ್ಞನಾದ ನಾಭಾಗನ ಮಗ ಅಂಬರೀಷ. ಈತನು ಸಪ್ತದ್ವೀಪಗಳಿಂದ ಕೂಡಿದ ಅಖಂಡ ಭೂಮಂಡಲಕ್ಕೆ ಚಕ್ರವರ್ತಿಯಾಗಿದ್ದ. ಕೊನೆಮೊದಲಿಲ್ಲದಷ್ಟು ಅಧಿಕಾರ ಐಶ್ವರ್ಯ ವೈಭವಗಳಿಗೆ ಆತನು ಅಧಿಪತಿಯಾಗಿದ್ದರೂ ಆತನಿಗೆ ಅವುಗಳಲ್ಲಿ ಸ್ವಲ್ಪವೂ ಆಸಕ್ತಿ ಇರಲಿಲ್ಲ. ‘ಲೌಕಿಕ ಭೋಗ ಭಾಗ್ಯಗಳೆಲ್ಲ ಅಶಾಶ್ವತ, ಈಶ್ವರಭಕ್ತಿಯೊಂದೆ ಶಾಶ್ವತ’ ಎಂಬುದು ಆತನ ಭಾವನೆ. ಕುಲಗುರುಗಳಾದ ವಸಿಷ್ಠರ ಆಲೋಚನೆಗೆ ಅನುಸಾರ ವಾಗಿ ಆತನು ಧರ್ಮದಿಂದ ರಾಜ್ಯಭಾರವನ್ನು ನಡೆಸುತ್ತಿರಲು, ಆತನ ಪ್ರಜೆಗಳು ಸ್ವರ್ಗಸುಖವನ್ನು ಕೂಡ ಹುಲ್ಲಿಗೆ ಸಮಾನವೆಂದು ಭಾವಿಸುವಷ್ಟು ಸುಖ ಸಂತೋಷ ಗಳಲ್ಲಿ ನಲಿಯುತ್ತಿದ್ದರು. ರಾಜನು ಆಗಾಗ ವಸಿಷ್ಠ, ಗೌತಮ, ಅಸಿತ ಮೊದಲಾದ ಮಹರ್ಷಿಗಳ ಸಹಾಯದಿಂದ ಅಶ್ವಮೇಧಯಾಗವನ್ನು ಮಾಡುತ್ತಿದ್ದನು. ಆ ಕಾಲದಲ್ಲಿ ತಮ್ಮ ಭಾಗದ ಹವಿಸ್ಸನ್ನು ಪಡೆಯಲು ದೇವತೆಗಳೆ ಸ್ವತಃ ಅಲ್ಲಿಗೆ ಬರುತ್ತಿದ್ದರು. ಆ ಕಾಲ ದಲ್ಲಿ ಯಾಗವನ್ನು ನಡೆಸುವವರೂ ಅದನ್ನು ನೋಡಲು ನೆರೆದವರೂ ತಮ್ಮ ಉಡಿಗೆ ತೊಡಿಗೆಗಳಿಂದ ಆ ದೇವತೆಗಳನ್ನೂ ನಾಚಿಸುವಷ್ಟು ವೈಭವದಿಂದಿರುತ್ತಿದ್ದರು.

ಕಾಲ ಮುಂದುವರಿಯುತ್ತಾ ಹೋದಂತೆ ರಾಜನ ಮನಸ್ಸು ಭೋಗಗಳಿಂದ ಹಿಮ್ಮುಖ ವಾಗುತ್ತಾ ಹೊರಟಿತು. ಆತನು ಭಗವಂತನ ಧ್ಯಾನ ಅಥವಾ ಭಕ್ತರ ಸಂಗದಲ್ಲಿ ಕಾಲ ವನ್ನೆಲ್ಲ ಕಳೆಯುತ್ತಿದ್ದನು. ಆತನ ಕಣ್ಣುಗಳು ದೇವರ ಗುಡಿಯನ್ನೊ ಮೂರ್ತಿಯನ್ನೊ ನೋಡಲು ಬಯಸುತ್ತಿದ್ದವು; ಕಿವಿಗಳಿಗೆ ಹರಿಕೀರ್ತನೆಯಲ್ಲಿ ಆಸೆ; ಮೂಗು ಹರಿಚರಣದ ತುಳಸಿಯ ವಾಸನೆಯನ್ನು ಬೇಡುವುದು; ನಾಲಿಗೆಗೆ ಹರಿನೈವೇದ್ಯದ ಚಪಲ; ಆತನ ಕೈಗಳಿಗೆ ದೇವರ ಗುಡಿಯನ್ನು ಸಾರಿಸಬೇಕೆಂಬ, ಹರಿ ಭಕ್ತರ ಪಾದಗಳನ್ನು ಒತ್ತಬೇಕೆಂಬ ಆಶೆ; ಪಾದಗಳಿಗೆ ತೀರ್ಥಯಾತ್ರೆಯ ಹಾದಿಯನ್ನು ಹಿಡಿಯಬೇಕೆಂಬ ಗೀಳು; ತಲೆಗೆ ದೇವರ ಪಾದದಲ್ಲಿ ಉರುಳಾಡಬೇಕೆಂಬ ಭ್ರಮೆ; ಹೆಚ್ಚೇನು, ಆತನ ದೇಹ ಇಂದ್ರಿಯ ಮನಸ್ಸು ಗಳು ಪರಮೇಶ್ವರನಿಗೆ ಮೀಸಲಾಗಿ ಹೋಗಿದ್ದವು. ದೇವೇಂದ್ರನ ವೈಭವವನ್ನು ಹೀಯಾಳಿ ಸುವ ಅರಮನೆ, ದಿವ್ಯಸುಂದರಿಯರಾದ ಮಡದಿಯರು, ಭಂಡಾರ, ಚತುರಂಗಸೈನ್ಯ, ದಾಸದಾಸಿಯರು–ಎಲ್ಲದರಲ್ಲಿಯೂ ಆತನಿಗೆ ಜಿಹಾಸೆ. ಸದಾ ಆತನು ಹರಿಯ ಧ್ಯಾನ ದಲ್ಲಿ ಮಗ್ನನಾದನು. ಆತನ ಈ ದೃಢ ಭಕ್ತಿಯನ್ನು ಕಂಡು ಮೆಚ್ಚಿದ ಶ್ರೀಹರಿಯು ತನ್ನ ಚಕ್ರಾಯುಧವನ್ನೆ ಆತನಿಗೆ ಅನುಗ್ರಹಿಸಿ ಕೊಟ್ಟನು.

ಸಾಮ್ರಾಜ್ಯಸುಖದಿಂದ ವಿಮುಖನಾದ ಅಂಬರೀಷನು ತನ್ನ ಧರ್ಮಪತ್ನಿಯೊಡನೆ ದ್ವಾದಶಿಯ ವ್ರತವನ್ನು ಕೈಕೊಂಡು ಒಂದು ವರ್ಷದವರೆಗೂ ಅದನ್ನು ಆಚರಿಸಿದನು. ವ್ರತದ ಕೊನೆಯಲ್ಲಿ ಕಾರ್ತೀಕಮಾಸದ ದ್ವಾದಶಿ ಬರುವುದಕ್ಕೆ ಮುಂಚೆ, ಮೂರುದಿನಗಳು ಆತ ಉಪವಾಸವಿದ್ದು, ನಾಲ್ಕನೆಯ (ದ್ವಾದಶಿಯ) ದಿನ ಬೆಳಿಗ್ಗೆ ಯಮುನಾ ನದಿಯಲ್ಲಿ ಸ್ನಾನಮಾಡಿ, ಶ್ರೀಹರಿಯನ್ನು ಭಕ್ತಿಯಿಂದ ಪೂಜೆಮಾಡಿದನು. ಅನಂತರ ಆತ ಆಗತಾನೆ ಕರುಹಾಕಿ ಬೇಕಾದಷ್ಟು ಹಾಲನ್ನು ಕೊಡಬಲ್ಲ ಅರವತ್ತುಕೋಟಿ ಗೋವುಗಳನ್ನು, ಬಂಗಾರದ ಕೋಡಣಗಳಿಂದಲೂ ಬೆಳ್ಳಿಯ ಗಗ್ಗರಗಳಿಂದಲೂ ಅಲಂಕರಿಸಿ, ಕರು ಗಳೊಡನೆ ದಾನಮಾಡಿದನು. ಬ್ರಾಹ್ಮಣರಿಗೂ ಬಡಬಗ್ಗರಿಗೂ ರಸಕವಳವನ್ನಿತ್ತು, ಕೈ ತುಂಬ ದಕ್ಷಿಣೆ ಕೊಟ್ಟು ತೃಪ್ತಿಪಡಿಸಿದಮೇಲೆ, ಅಂಬರೀಷನು ತಾನೂ ಪಾರಣೆ ಮಾಡು ವುದಕ್ಕೆ ಸಿದ್ಧನಾದನು. ಅಷ್ಟರಲ್ಲಿ ಮಹಾ ತಪಸ್ವಿಯಾದ ದುರ್ವಾಸಮುನಿಯು ಅಲ್ಲಿಗೆ ಆಗಮಿಸಿದನು. ಪರಮೇಶ್ವರನೇ ಪ್ರತ್ಯಕ್ಷವಾದಂತೆ ಭಾವಿಸಿದ ಅಂಬರೀಷನು ಅತ್ಯಂತ ಆದರದಿಂದ ಆತನನ್ನು ಇದಿರುಗೊಂಡು, ಭಕ್ತಿಯಿಂದ ನಮಸ್ಕರಿಸಿ ಊಟಕ್ಕೆ ಏಳುವಂತೆ ಬೇಡಿದನು. ಪುಷಿಯು ಹಾಗೆಯೇ ಆಗಲೆಂದು ಹೇಳಿ, ನಿತ್ಯಕರ್ಮಗಳನ್ನು ಮಾಡಿಕೊಂಡು ಬರುವುದಕ್ಕಾಗಿ ಯಮುನಾ ನದಿಗೆ ಹೋದನು. ಅಲ್ಲಿ ಆತ ನೀರಿನಲ್ಲಿ ಮುಳುಗಿ ಧ್ಯಾನ ಮಾಡುತ್ತಾ ಸಮಾಧಿಸ್ಥನಾದನು. ಎಷ್ಟು ಹೊತ್ತಾದರೂ ಆತನು ಮೇಲಕ್ಕೇಳಲಿಲ್ಲ. ಇತ್ತ ಅಂಬರೀಷನು ದ್ವಾದಶಿಯ ಮುಹೂರ್ತ ಕಳೆಯುವುದರೊಳಗಾಗಿ ಊಟವನ್ನು ಮುಗಿಸ ಬೇಕೆಂದು ತಹತಹ ಪಡುತ್ತಿದ್ದನು. ಅತಿಥಿಯನ್ನು ಬಿಟ್ಟು ಊಟಮಾಡುವುದು ದೋಷ, ದ್ವಾದಶಿಯ ತಿಥಿ ಹೋಗುವುದರೊಳಗಾಗಿ ಊಟಮಾಡದಿದ್ದರೂ ದೋಷ. ಆತನು ಮುಂದೋರದೆ ಶಾಸ್ತ್ರಜ್ಞರನ್ನು ವಿಚಾರಿಸಲು, ಅವರು ‘ನೀರನ್ನು ಕುಡಿದರೆ ಪಾರಣೆಯ ಫಲ ಬರುತ್ತದೆ’ ಎಂದು ತಿಳಿಸಿದರು. ಅಂಬರೀಷನು ಅದರಂತೆಯೇ ಸ್ವಲ್ಪ ನೀರನ್ನು ಕುಡಿದು, ದುರ್ವಾಸನು ಹಿಂದಿರುಗುವುದನ್ನೆ ಕಾಯುತ್ತಾ ಕುಳಿತನು.

ಇದಾದ ಸ್ವಲ್ಪ ಹೊತ್ತಿನ ಮೇಲೆ ದುರ್ವಾಸನು ಸಮಾಧಿಯಿಂದ ಎಚ್ಚೆತ್ತು, ತನ್ನ ನಿತ್ಯ ಕರ್ಮಗಳೆಲ್ಲ ಮುಗಿಯುತ್ತಲೆ ಅಂಬರೀಷನ ಬಳಿಗೆ ಹಿಂದಿರುಗಿದನು. ಅವನ ಮುಖವನ್ನು ನೋಡುತ್ತಲೆ, ಅವನು ನೀರು ಕುಡಿದು ದ್ವಾದಶಿಯ ವ್ರತವನ್ನು ಮುಗಿಸಿರುವನೆಂಬುದು ಆತನ ಜ್ಞಾನದೃಷ್ಟಿಗೆ ಗೋಚರವಾಯಿತು. ದುರ್ವಾಸನೆಂದರೆ ಶೀಘ್ರ ಕೋಪಕ್ಕೆ ತವರ್ ಮನೆ. ಆತನಿಗೆ ರಾಜನ ಚರ್ಯೆಯಿಂದ ತಡೆಯಲಾರದಷ್ಟು ಕೋಪ ಬಂದಿತು. ಭಕ್ತಿ ಯಿಂದ ಕೈಕಟ್ಟಿಕೊಂಡು ತನ್ನ ಇದಿರಿಗೆ ನಿಂತಿರುವ ರಾಜನನ್ನು ಕೆಕ್ಕರುಗಣ್ಣಿನಿಂದ ಬಿರಿಬಿರಿ ನೋಡುತ್ತಾ ‘ಎಲಾ, ನಿನಗೆಷ್ಟು ಸೊಕ್ಕು? ಅತಿಥಿಯಾದ ನನ್ನನ್ನು ಊಟಕ್ಕೆಂದು ಕರೆದು, ನನಗೆ ಊಟಕ್ಕಿಡುವ ಮುನ್ನವೆ ನೀನು ಪಾರಣೆಯನ್ನು ಮಾಡಿ ಮುಗಿಸಿದೆಯಾ! ಇದೆಂತಹ ದುರಾಚಾರ! ಈ ಪಾಪಕ್ಕೆ ತಕ್ಕ ಫಲವನ್ನು ಈಗಲೆ ಅನುಭವಿಸಬೇಕು’ ಎಂದು ಹೇಳಿ, ತನ್ನ ಜಟೆಯಿಂದ ಒಂದು ಕೂದಲನ್ನು ಕಿತ್ತು ನೆಲಕ್ಕೆ ಬಡಿದನು. ಒಡನೆಯೆ ಅಲ್ಲಿ ಭಯಂಕರ ರೂಪಿನ ಮಾರಿಯೊಂದು ಕಾಣಿಸಿತು. ಅದು ತನ್ನ ದೇಹದಿಂದ ಸುತ್ತಲೂ ಉರಿಯನ್ನು ಚೆಲ್ಲುತ್ತಾ, ಹಿರಿದ ಕತ್ತಿಯೊಡನೆ ಅಂಬರೀಷನ ಬಳಿಗೆ ಹಾರಿ ಬಂದಿತು. ದೈವಭಕ್ತನಾದ ಅಂಬರೀಷನು ಸ್ವಲ್ಪವೂ ಭಯಗೊಳ್ಳದೆ ಧ್ಯಾನಸ್ಥಿಮಿತಮೂರ್ತಿಯಾಗಿ ನಿಂತಿದ್ದನು. ಅಷ್ಟರಲ್ಲಿ ವಿಷ್ಣುಚಕ್ರವು ರೊಯ್ಯೆಂದು ಅಲ್ಲಿಗೆ ಹಾರಿ ಬಂದು, ಹೆಬ್ಬಾವನ್ನು ಸುಟ್ಟು ಹಾಕುವ ಕಾಡುಕಿಚ್ಚಿನಂತೆ ಆ ಮಾರಿಯನ್ನು ಧ್ವಂಸಮಾಡಿತು. ಮರು ನಿಮಿಷದಲ್ಲಿ ಅದು ದುರ್ವಾಸನತ್ತ ತಿರುಗಿತು. ಅದರ ರಭಸವನ್ನು ಕಂಡು ಆ ಪುಷಿಯು ಭಯದಿಂದ ನಡುಗುತ್ತಾ ಕಾಲಿಗೆ ಬುದ್ಧಿ ಹೇಳಿದನು.

ಜೀವಭಯದಿಂದ ಓಟಕಿತ್ತ ದುರ್ವಾಸನನ್ನು ವಿಷ್ಣುಚಕ್ರ ಬೆನ್ನಟ್ಟಿತು. ಪುಷಿಯು ಅದರಿಂದ ತಪ್ಪಿಸಿಕೊಳ್ಳುವುದಕ್ಕಾಗಿ ಮೇರು ಪರ್ವತದ ಗುಹೆಗಳನ್ನು ಹೊಕ್ಕನು, ಚಕ್ರ ರೊಯ್ಯೆಂದು ಶಬ್ದ ಮಾಡುತ್ತಾ ಅಲ್ಲಿಗೆ ನುಗ್ಗಿತು; ಪುಷಿ ಸ್ವರ್ಗಕ್ಕೆ ಓಡಿದ, ಅದು ಹಿಂದೆಯೇ ಬೆನ್ನಟ್ಟಿ ಬಂತು; ಲೋಕ ಲೋಕಗಳನ್ನೆಲ್ಲ ಹೊಕ್ಕ, ಆದರೇನು? ಬೆನ್ನಟ್ಟಿದ ಭೇತಾಳದಂತೆ ಚಕ್ರ ಹಿಂದೆಯೇ ನುಗ್ಗಿ ಬಂತು. ಪುಷಿಯು ತನಗಿನ್ನು ಉಳಿಗಾಲವಿಲ್ಲ ವೆಂದು ಸತ್ಯಲೋಕಕ್ಕೆ ಓಡಿಹೋಗಿ ಬ್ರಹ್ಮನನ್ನು ಮರೆಹೊಕ್ಕನು. ಆದರೆ ಬ್ರಹ್ಮನು ‘ಅಯ್ಯಾ ಮಹರ್ಷಿ, ಭಗವಂತನ ಚಕ್ರದಿಂದ ನಿನ್ನನ್ನು ಕಾಪಾಡುವುದು ನನ್ನ ಕೈಲಿ ಸಾಧ್ಯ ವಿಲ್ಲ’ ಎಂದು ಕೈ ಅಲ್ಲಾಡಿಸಿದ. ಅಷ್ಟರಲ್ಲಿ ಚಕ್ರವು ಬೆನ್ನಟ್ಟಿ ಬರುತ್ತಿರುವುದನ್ನು ದೂರದಿಂದ ಕಂಡ ದುರ್ವಾಸಪುಷಿ, ಅಲ್ಲಿಂದಲೂ ಓಟ ಕಿತ್ತು ಕೈಲಾಸಕ್ಕೆ ಬಂದ. ಪರಶಿವನು ‘ಅಯ್ಯಾ ಮಹರ್ಷಿ, ನಾವಾರೂ ನಿನ್ನ ಸಹಾಯಕ್ಕೆ ಬರಲಾರೆವು. ನೀನು ಶ್ರೀಹರಿಯ ಬಳಿಗೇ ಓಡಿಹೋಗು’ ಎಂದ. ಪುಷಿಯು ಉಸಿರು ತಿರುಗಿಸಿಕೊಳ್ಳುವುದಕ್ಕೂ ಅವಕಾಶವಿಲ್ಲದೆ ವೈಕುಂಠಕ್ಕೆ ಓಡಿಬಂದ. ‘ಸ್ವಾಮಿ, ನನ್ನನ್ನು ಕಾಪಾಡು’ ಎಂದು ಬೇಡು ತ್ತಿರುವ ಆ ಪುಷಿಯನ್ನು ಕುರಿತು ಶ್ರೀಹರಿಯು ‘ಅಯ್ಯಾ ಮುನೀಂದ್ರ! ನಾನು ಸ್ವತಂತ್ರ ನಲ್ಲ, ಕೇವಲ ಭಕ್ತರ ಅಧೀನ. ಪತಿವ್ರತೆಯಾದವಳು ಗಂಡನನ್ನು ಅಧೀನದಲ್ಲಿ ಇಟ್ಟು ಕೊಂಡಿರುವ ಹಾಗೆ ಭಗವದ್ಭಕ್ತರು ನನ್ನನ್ನು ವಶಪಡಿಸಿಕೊಂಡಿದ್ದಾರೆ. ಆದ್ದರಿಂದ ನಿನ್ನ ಸಂಕಟವನ್ನು ನಾನು ತಪ್ಪಿಸಲಾರೆ. ಅದನ್ನು ತಪ್ಪಿಸಬೇಕಾದರೆ ನೀನು ಅಂಬರೀಷನನ್ನೆ ಮರೆಹೋಗಬೇಕು. ಅಯ್ಯಾ ಮಹರ್ಷಿ, ತಪಸ್ಸು ಮತ್ತು ವಿದ್ಯೆಗಳು ಸಜ್ಜನರಿಗೆ ಶ್ರೇಯ ಸ್ಕರವಾದುವು. ಅವನ್ನು ದುರುಪಯೋಗಪಡಿಸಿದರೆ ಉಪಯೋಗಿಸುವವನಿಗೇ ಮೃತ್ಯು ವಾಗುತ್ತವೆ’ ಎಂದನು.

ವಿಷ್ಣುವಿನ ನುಡಿಯ ಕಿಡಿಗಿಂತಲೂ ಚಕ್ರವು ಸುರಿಸುತ್ತಿರುವ ಬೆಂಕಿಯ ಕಿಡಿಗಳು ಹೆಚ್ಚು ಭಯಂಕರವಾಗಿದ್ದುದರಿಂದ, ದುರ್ವಾಸನು ಅಲ್ಲಿ ತಳುವದೆ ಅಂಬರೀಷನ ಬಳಿಗೆ ಓಡಿ ಬಂದು ಅವನ ಪಾದಗಳನ್ನು ಭದ್ರವಾಗಿ ಹಿಡಿದುಕೊಂಡನು. ಅಂಬರೀಷನು ತಕ್ಷಣವೇ ಪಾದಗಳನ್ನು ಬಿಡಿಸಿಕೊಂಡು ಭಕ್ತಿಯಿಂದ ಸ್ತೋತ್ರಮಾಡಿದನು–‘ಸಹಸ್ರ ಧಾರೆಗಳಿಂದ ಕೂಡಿದ ಹೇ ಸುದರ್ಶನ! ನಿನಗೆ ನಮಸ್ಕಾರ. ನೀನೆ ಬೆಂಕಿ, ನೀನೆ ಸೂರ್ಯ, ನೀನೆ ಚಂದ್ರ! ಮಹಾವಿಷ್ಣುವಿನ ಕೈಯನ್ನು ಅಲಂಕರಿಸುವ ನೀನು ಶರಣಾದವರನ್ನು ರಕ್ಷಿಸಬೇಕಲ್ಲವೆ? ಅಧರ್ಮ ಮಾಡುವ ರಾಕ್ಷಸರಿಗೆ ನೀನು ಧೂಮಕೇತು! ಆದರೆ ಧರ್ಮಪರರಾದವರ ಸಂರಕ್ಷಕನೂ ನೀನೇ ಅಲ್ಲವೆ? ಒಂದು ಪಕ್ಷ ಅಪರಾಧ ಮಾಡಿದ್ದರೂ, ಶರಣು ಹೊಕ್ಕ ಮೇಲೆ ಅವನನ್ನು ಕ್ಷಮಿಸಲೇಬೇಕು. ಶರಣಾಗತನಾದ ಈ ಪುಷಿಯನ್ನು ಕಾಪಾಡಿ ನನ್ನ ವಂಶವನ್ನು ಉಳಿಸು. ಇಲ್ಲದಿದ್ದಲ್ಲಿ ಬ್ರಹ್ಮಹತ್ಯಾದೋಷದಿಂದ ನನ್ನ ವಂಶವೇ ನಾಶ ವಾದೀತು! ನಾನು ಇದುವರೆಗೆ ಸಂಪಾದಿಸಿರುವ ಪುಣ್ಯವೆಲ್ಲವೂ ಸೇರಿ ಈ ಪುಷಿಯನ್ನು ಸಂರಕ್ಷಿಸಲಿ!’ಎಂದನು. ಆತನ ಬಾಯಿಂದ ಈ ಮಾತುಗಳು ಹೊರಬೀಳುವಷ್ಟರಲ್ಲಿ ಚಕ್ರವು ಶಾಂತವಾಯಿತು. ದುರ್ವಾಸನು ಮೇಲಕ್ಕೆದ್ದು ಅಂಬರೀಷನನ್ನು ಬಾಯ್ತುಂಬ ಹೊಗಳಿ ಹರಸಿದನು. ಮಹಾರಾಜನು ಆತನ ಮುಂದೆ ಅಡ್ಡಬಿದ್ದು ‘ಮಹಾನುಭಾವ, ಈಗ ನಡೆದುದನ್ನು ಮರೆತುಬಿಡೋಣ. ನೀನು ನನ್ನ ಆತಿಥ್ಯವನ್ನು ಸ್ವೀಕರಿಸಿ ಊಟ ಮಾಡು’ ಎಂದು ಪ್ರಾರ್ಥಿಸಿದನು. ಪುಷಿಯು ರಾಜನಿತ್ತ ರುಚಿಕರವಾದ ಆಹಾರವನ್ನು ಹೊಟ್ಟೆ ತುಂಬ ಊಟಮಾಡಿ, ಆತನನ್ನು ಹೊಗಳಿ ಹರಸುತ್ತಾ ಅಲ್ಲಿಂದ ಬ್ರಹ್ಮಲೋಕಕ್ಕೆ ಹಿಂದಿರುಗಿದನು.

ಇಷ್ಟೆಲ್ಲ ನಡೆಯುವುದಕ್ಕೆ ಒಂದು ವರ್ಷ ಹಿಡಿಯಿತು. ಅಂಬರೀಷನು ಅಲ್ಲಿಯವರೆಗೆ ಉಪವಾಸವಿದ್ದನು. ದುರ್ವಾಸ ಪುಷಿಯನ್ನು ಬೀಳ್ಕೊಟ್ಟಮೇಲೆ ಆತನು ಊಟಮಾಡಿ ದನು. ಇದಾದ ಕೆಲವು ದಿನಗಳಮೇಲೆ ಆತನು ತನ್ನ ರಾಜ್ಯವನ್ನು ಮಕ್ಕಳಿಗೊಪ್ಪಿಸಿ, ತಾನು ತಪೋವನಕ್ಕೆ ಹೋಗಿ ಭಗವಂತನ ಧ್ಯಾನದಲ್ಲಿ ತತ್ಪರನಾದನು.


\section{(೬) ವಿಕುಕ್ಷಿಯ ಮಗ ಪುರಂಜಯ}

ವೈವಸ್ವತಮನುವಿನ ಒಂಬತ್ತನೆಯ ಮಗ ಇಕ್ಷ್ವಾಕು. ಒಮ್ಮೆ ಮನುವು ಸೀನಲು, ಆತನ ಮೂಗಿನ ಹೊಳ್ಳೆಯಿಂದ ಈತ ಹುಟ್ಟಿಬಂದನಂತೆ! ಆದ್ದರಿಂದಲೆ ಈತನಿಗೆ ಆ ಹೆಸರು. ಈತನಿಗೆ ನೂರುಮಂದಿ ಮಕ್ಕಳು. ಅವರೆಲ್ಲರೂ ಆರ್ಯಾವರ್ತದ ಬೇರೆ ಬೇರೆ ಭಾಗ ಗಳಲ್ಲಿ ರಾಜರಾದರು. ಇವರಲ್ಲಿ ಹಿರಿಯನಾದ ವಿಕುಕ್ಷಿ ಒಮ್ಮೆ ತಂದೆಯು ಮಾಡುತ್ತಿದ್ದ ‘ಅಷ್ಟಕ’ ಎಂಬ ಶ್ರಾದ್ಧಕ್ಕಾಗಿ ಶುದ್ಧವಾದ ಮಾಂಸವನ್ನು ತರಲೆಂದು ಅಡವಿಗೆ ಹೋದ. ಅಲ್ಲಿ ಶ್ರಾದ್ಧಕ್ಕೆ ಯೋಗ್ಯವಾದ ಅನೇಕ ಮೃಗಗಳನ್ನು ಆತ ಬೇಟೆಯಾಡಿ ಕೊಂದ. ಆ ವೇಳೆಗೆ ಅವನಿಗೆ ತುಂಬ ಹಸಿವಾಯಿತು; ತಾನು ಕೊಂದ ಮೃಗಗಳಲ್ಲಿ ಒಂದು ಮೊಲವನ್ನು ತಿಂದುಹಾಕಿದ. ಉಳಿದುದನ್ನೆಲ್ಲ ಆತ ಕೊಂಡು ಬಂದು ತಂದೆಗೆ ಒಪ್ಪಿಸಿದ. ಶ್ರಾದ್ಧಕಾಲ ದಲ್ಲಿ ಆ ಮಾಂಸವನ್ನೆಲ್ಲ ಮಂತ್ರದ ನೀರಿನಿಂದ ಶುದ್ಧಿಮಾಡಬೇಕೆಂದು ವಸಿಷ್ಠರಿಗೆ ಒಪ್ಪಿಸಲು, ಅವರು ‘ಅಯ್ಯಾ, ಇದು ತಿಂದು ಬಿಟ್ಟ ಮಾಂಸ, ಇದು ಶ್ರಾದ್ಧಕ್ಕೆ ಯೋಗ್ಯ ವಲ್ಲ’ ಎಂದರು. ವಿಚಾರ ಮಾಡಿದಾಗ ಅದು ನಿಜವೆಂದು ಗೊತ್ತಾಯಿತು. ಇಕ್ಷ್ವಾಕು ತುಂಬ ಕೋಪಗೊಂಡು, ಮಗನನ್ನು ರಾಜ್ಯದಿಂದ ಓಡಿಸಿಬಿಟ್ಟ. ರಾಜ್ಯಭ್ರಷ್ಟನಾದ ಮಗ ದೇಶ ದೇಶಗಳಲ್ಲಿ ಅಲೆಯುತ್ತಿದ್ದು, ಇಕ್ಷ್ವಾಕು ಸತ್ತಮೇಲೆ ತನ್ನ ರಾಜ್ಯಕ್ಕೆ ಹಿಂದಿರುಗಿ ರಾಜನಾದ. ಇವನು ‘ಶಶಾದ’ ಎಂದರೆ ಮೊಲವನ್ನು ತಿಂದುದರಿಂದ ಇವನಿಗೆ ‘ಶಶ’ ಎಂದು ಹೆಸರಾ ಯಿತು. ಇವನು ಅನೇಕ ಯಾಗಗಳನ್ನು ಮಾಡಿ ಮಹಾಮಹಿಮನೆನಿಸಿದನು.

ವಿಕುಕ್ಷಿಯ ಮಗ ಪುರಂಜಯ. ಇವನು ತುಂಬ ಕೀರ್ತಿಶಾಲಿ. ಒಮ್ಮೆ ದೇವತೆಗಳಿಗೂ ರಾಕ್ಷಸರಿಗೂ ಯುದ್ಧ ನಡೆದು, ದೇವತೆಗಳೆಲ್ಲ ಸೋತು ಜೀವಭಯದಿಂದ ದಿಕ್ಕಾಪಾಲಾಗ ಬೇಕಾಯಿತು. ಆಗ ಮಹಾವಿಷ್ಣು ಈ ಪುರಂಜಯನನ್ನು ದೇವತೆಗಳ ಸಹಾಯಕ್ಕಾಗಿ ನೇಮಿ ಸಿದನು. ಆಗ ಪುರಂಜಯನು ಕವಚವನ್ನು ತೊಟ್ಟು, ಬಿಲ್ಲು ಬಾಣಗಳೊಡನೆ ಸಜ್ಜಾಗಿ ಯುದ್ಧಕ್ಕೆ ಹೊರಟನು. ಆಗ ಸಾಕ್ಷಾತ್ ಇಂದ್ರನೇ ಎತ್ತಿನ ರೂಪದಿಂದ ಅವನ ವಾಹನ ವಾದನು. ಪುರಂಜಯನು ದೇವತೆಗಳ ಸೈನ್ಯದೊಡನೆ ದಂಡೆತ್ತಿಹೋಗಿ ರಾಕ್ಷಸರ ರಾಜ ಧಾನಿಯನ್ನು ಮುತ್ತಿದನು. ಆಗ ರಾಕ್ಷಸರಿಗೂ ಆತನಿಗೂ ಭಯಂಕರವಾದ ಯುದ್ಧ ವಾಯಿತು. ಆ ಯುದ್ಧದಲ್ಲಿ ರಾಕ್ಷಸರ ಕಗ್ಗೊಲೆಯಾಗಿಹೋಯಿತು. ಸಾಯದೆ ಉಳಿದವ ರೆಲ್ಲರೂ ಪಾತಾಳಕ್ಕೆ ಹೋಗಿ ತಲೆ ಮರೆಸಿಕೊಂಡರು. ರಾಕ್ಷಸರ ರಾಜಧಾನಿಯನ್ನು ಗೆದ್ದುದ ರಿಂದ ಈತನಿಗೆ ಪುರಂಜಯನೆಂಬ ಹೆಸರು ಅನ್ವರ್ಥವಾಯಿತು. ದೇವೇಂದ್ರ ಈತನಿಗೆ ವಾಹನವಾದುದರಿಂದ ‘ಇಂದ್ರವಾಹನ’ಎಂಬ ಬಿರುದು ಬಂದಿತು. ಎತ್ತಿನ ಭುಜದ ಮೇಲೆ ಕುಳಿ ತಿದ್ದುದರಿಂದ ಈತನಿಗೆ ‘ಕಕುತ್ಸ್​ó್ಥ’ ಎಂಬ ಹೆಸರಾಯಿತು. ಈತನಿಂದಲೇ ಪ್ರಸಿದ್ಧವಾದ ಕಾಕುತ್ಸó್ಥ ವಂಶ ಪ್ರಾರಂಭವಾಯಿತು. ಶ್ರೀರಾಮಚಂದ್ರನು ಈ ವಂಶಕ್ಕೆ ಸೇರಿದವನು.


\section{(೭) ಮಾಂಧಾತ ಮಹಾರಾಜ}

ಪುರಂಜಯನ ತರುವಾಯ ಹಲವು ತಲೆಗಳು ಕಳೆದಮೇಲೆ ‘ಯುವನಾಶ್ವ’ನೆಂಬ ಮಹಾ ಪುರುಷನೊಬ್ಬನು ಆ ವಂಶದಲ್ಲಿ ಹುಟ್ಟಿದ. ಆತನು ನೂರು ಜನ ಹೆಂಡಿರನ್ನು ಮದುವೆ ಯಾಗಿ ಹಲವುಕಾಲ ಸಂಸಾರಜೀವನವನ್ನು ನಡೆಸಿದರೂ ಮಕ್ಕಳಾಗಲಿಲ್ಲ. ಇದರಿಂದ ಆತನು ತುಂಬ ದುಃಖಿತನಾಗಿ, ತನ್ನ ಹೆಂಡಿರೊಡನೆ ವೈರಾಗ್ಯದಿಂದ ಅಡವಿಗೆ ಹೋದ. ಆ ಅಡವಿಯಲ್ಲಿದ್ದ ಕೆಲವು ಮಹರ್ಷಿಗಳು ಈತನ ವೈರಾಗ್ಯಕ್ಕೆ ಕಾರಣವೇನೆಂಬುದನ್ನು ತಿಳಿದು, ಕರುಣೆಯಿಂದ ಅವನ ಕೈಲಿ ‘ಪುತ್ರಕಾಮೇಷ್ಟಿ’ ಎಂಬ ಯಾಗವನ್ನು ಮಾಡಿಸಿದರು. ಈ ಯಾಗ ಮಾಡುತ್ತಿರುವಾಗ ಒಂದು ದಿನ ರಾತ್ರಿ ಯುವನಾಶ್ವನಿಗೆ ತುಂಬ ಬಾಯಾರಿಕೆ ಯಾಯಿತು. ಆತನು ನೀರನ್ನು ಹುಡುಕುತ್ತಾ ಯಜ್ಞಶಾಲೆಗೆ ಹೋದ. ಅಲ್ಲಿ ಇದ್ದವರೆಲ್ಲರೂ ಗಾಢನಿದ್ರೆಯಲ್ಲಿ ಮುಳುಗಿದ್ದರು. ಅವರ ಸಮೀಪದಲ್ಲಿ ಒಂದು ತಂಬಿಗೆ ತುಂಬ ನೀರಿತ್ತು. ರಾಜನು ಅದನ್ನು ತೆಗೆದುಕೊಂಡು ಹೊಟ್ಟೆತುಂಬ ನೀರು ಕುಡಿದ. ಮರುದಿನ ಬೆಳಗಾದ ಮೇಲೆ ಬರಿದಾಗಿದ್ದ ಆ ತಂಬಿಗೆಯನ್ನು ಕಂಡು ಪುಷಿಗಳು ಗಾಬರಿಯಾದರು. ಆ ನೀರು ಮಂತ್ರದ ನೀರು; ಮಕ್ಕಳು ಹುಟ್ಟುವುದಕ್ಕಾಗಿ ರಾಣಿಯರಿಗೆ ಕೊಡಬೇಕೆಂದು ಇಟ್ಟಿದುದು. ಅವರು ರಾಜನ ಬಳಿಗೆ ಬಂದು ‘ಮಹಾರಾಜ, ಆ ಮಂತ್ರದ ನೀರನ್ನು ಯಾರು ಕುಡಿದರೆಂಬುದನ್ನು ಮೊದಲು ಕಂಡುಹಿಡಿ’ ಎಂದರು. ರಾಜನು ಹೊಸದಾಗಿ ಕಂಡುಹಿಡಿ ಯುವುದೇನು? ತಾನೇ ಕುಡಿದನೆಂದು ತಿಳಿಸಿದ. ಪುಷಿಗಳು ತಮ್ಮ ಮನಸ್ಸಿನಲ್ಲಿಯೇ ‘ಮನುಷ್ಯಪ್ರಯತ್ನಕ್ಕೆ ಏನು ಬೆಲೆ? ದೈವಬಲವೇ ಬಲ’ ಎಂದುಕೊಂಡು ತತ್ಕಾಲಕ್ಕೆ ಸುಮ್ಮನಾದರು. 

ಮಂತ್ರದ ನೀರು ತನ್ನ ಕೆಲಸವನ್ನು ತಾನು ಮಾಡಿತು. ರಾಜನು ಗರ್ಭಧರಿಸಿದ. ಒಂಬತ್ತು ತಿಂಗಳು ತುಂಬುತ್ತಲೆ ಆತನ ಹೊಟ್ಟೆಯನ್ನು ಸೀಳಿಕೊಂಡು ಒಂದು ಮಗು ಹೊರಕ್ಕೆ ಬಂತು. ಹುಟ್ಟುತ್ತಲೆ ಅದಕ್ಕೆ ಹೊಟ್ಟೆಯ ಹಸಿವು, ಕಿರಿಚಿಕೊಳ್ಳುವುದಕ್ಕೆ ಪ್ರಾರಂಭಿಸಿತು. ಅದಕ್ಕೆ ಮೊಲೆಹಾಲು ಎಲ್ಲಿಂದ ತರುವುದು? ಪುಷಿಗಳೆಲ್ಲ ಸೇರಿ ಮುಂದೇನು ಮಾಡಬೇಕೆಂದು ಯೋಚಿಸುತ್ತಿರಲು, ಇಂದ್ರನು ಮಗುವಿನ ಬಳಿ ಪ್ರತ್ಯಕ್ಷ ನಾಗಿ ‘ಮಗು ಅಳಬೇಡ, ಮಾಂಧಾತಾ (ನನ್ನನ್ನು ಕುಡಿ)’ ಎಂದು ಹೇಳಿ ತನ್ನ ತೋರು ಬೆರಳನ್ನು ಅದರ ಬಾಯಲ್ಲಿಟ್ಟ; ಆ ಬೆರಳಿನಿಂದ ಅಮೃತ ತೊಟ್ಟಿಕ್ಕುತ್ತಿತ್ತು. ಮಗುವು ಅದನ್ನು ಕುಡಿದು ‘ಮಾಂಧಾತಾ’ ಎಂಬ ಹೆಸರನ್ನೆ ಪಡೆಯಿತು. ಆ ಮಗುವನ್ನು ಹೊತ್ತು ಹೆತ್ತ ಯುವನಾಶ್ವನು ಹೊಟ್ಟೆ ಸೀಳಿಹೋಗಿದ್ದರೂ ಪುಷಿಗಳ ಆಶೀರ್ವಾದದಿಂದಲೂ ದೇವರ ಕೃಪೆಯಿಂದಲೂ ಸಾಯದೆ ಉಳಿದುಕೊಂಡು, ಆತ ಅಲ್ಲಿಯೇ ದೊಡ್ಡ ತಪಸ್ಸನ್ನು ಕೈಗೊಂಡು ಮುಕ್ತಿಯನ್ನು ಪಡೆದ. 

ಮಾಂಧಾತನು ಬೆಳೆದು ದೊಡ್ಡವನಾಗಿ ಎಣೆಯಿಲ್ಲದ ಪರಾಕ್ರಮಿ ಎನಿಸಿದನು. ರಾವಣ ನಂತಹ ಮಹಾ ಪರಾಕ್ರಮಿ ಕೂಡ ಅವನಿಗೆ ಸೋತು ಶರಣಾದನು. ಇದನ್ನು ಕಂಡು ದೇವೇಂದ್ರನು ಆತನಿಗೆ ‘ತ್ರಸದ್ದಸ್ಯು’ ಎಂದು ಬಿರುದನ್ನು ಕೊಟ್ಟನು. ಆತನು ಜಗತ್ತೆಲ್ಲ ವನ್ನೂ ಜಯಿಸಿ, ಅಖಂಡ ಭೂಮಂಡಲಕ್ಕೆ ಚಕ್ರವರ್ತಿಯಾದನು. ಆತನು ಅನೇಕ ಯಾಗಗಳನ್ನು ಮಾಡಿ ಮಾನವರನ್ನೂ ಬಾನವರನ್ನೂ ತಣಿಸಿದನು. ಆತನು ಬಹು ದೊಡ್ಡ ಭಕ್ತನಾಗಿ ಭಗವಂತನ ಕೃಪೆಗೂ ಪಾತ್ರನಾಗಿದ್ದನು. ಈ ಮಾಂಧಾತ ಚಕ್ರವರ್ತಿಯು ಶತಬಿಂದುವಿನ ಮಗಳಾದ ಬಿಂದುಮತಿಯೆಂಬ ಸುಂದರಿಯನ್ನು ಮದುವೆಯಾಗಿ ಮೂವರು ಗಂಡುಮಕ್ಕಳನ್ನೂ ಐವತ್ತು ಜನ ಹೆಣ್ಣುಮಕ್ಕಳನ್ನೂ ಪಡೆದನು. ಈ ಹೆಣ್ಣು ಗಳೆಲ್ಲರೂ ಸೌಭರಿಯೆಂಬ ಮಹರ್ಷಿಯನ್ನು ಮದುವೆಯಾದರು. ಅವರು ಆ ಪುಷಿ ಯನ್ನು ವರಿಸಿದುದೇ ಒಂದು ಸುಂದರ ಕಥೆ. ಮಹಾತಪಸ್ವಿಯಾದ ಸೌಭರಿಮಹರ್ಷಿ ಒಮ್ಮೆ ಯಮುನಾ ನದಿಯಲ್ಲಿ ಮುಳುಗಿ ಕುಳಿತು ತಪಸ್ಸು ಮಾಡುತ್ತಿದ್ದ. ಆಗ ಎರಡು ಮೀನುಗಳು ಕಾಮಕೇಳಿಯಲ್ಲಿದ್ದುದು ಅಕಸ್ಮಾತ್ತಾಗಿ ಆತನ ಕಣ್ಣಿಗೆ ಬಿತ್ತು. ಅದರಿಂದ ಆತನ ಮನಸ್ಸು ಚಂಚಲವಾಯಿತು. ತಾನು ಹೆಣ್ಣಿನ ಸುಖವನ್ನು ಪಡೆಯಬೇಕೆಂಬ ಆಶೆ ಮೊಳೆತು ತಡೆಯಲಾರದಷ್ಟು ಉಲ್ಬಣಿಸಿತು. ಆತನು ನೀರಿನಿಂದೆದ್ದು ಬಂದು, ನೇರವಾಗಿ ಮಾಂಧಾತೃವಿನ ಹತ್ತಿರಹೋಗಿ ‘ಅಯ್ಯಾ ಚಕ್ರವರ್ತಿ, ನಿನಗೆ ಐವತ್ತು ಹೆಣ್ಣುಮಕ್ಕಳಿದ್ದಾ ರಲ್ಲ, ಒಬ್ಬಳನ್ನು ನನಗೆ ಮದುವೆಮಾಡಿಕೊಡು’ ಎಂದು ಕೇಳಿದ. ಕುರೂಪಿಯಾಗಿದ್ದ ಈ ಮುಪ್ಪಿನ ಮುದುಕ ಹೆಣ್ಣಿಗೆ ಕಣ್ಣುಹಾಕಿದ್ದನ್ನು ಕಂಡು ಮಾಂಧಾತನಿಗೆ ಆಶ್ಚರ್ಯ ವಾಯಿತು, ಅಸಹ್ಯವೂ ಆಯಿತು. ಆದರೆ ಮಾಡುವುದೇನು? ಸೌಭರಿಯಂತಹ ಮಹಾ ಮಹಿಮನಾದ ಮಹರ್ಷಿ ಬಾಗಿಲಿಗೆ ಬಂದು ಬೇಡುವಾಗ ಇಲ್ಲವೆಂದು ಹೇಳಲಾಗು ತ್ತದೆಯೆ? ಆತ ಉಪಾಯವಾಗಿ ‘ಸ್ವಾಮಿ, ನನ್ನ ಮಕ್ಕಳನ್ನೆಲ್ಲ ನಿಮ್ಮ ಮುಂದೆ ನಿಲ್ಲಿಸು ತ್ತೇನೆ. ಅವರಲ್ಲಿ ನಿಮ್ಮನ್ನು ಯಾವಳು ವರಿಸುತ್ತಾಳೊ ಅವಳನ್ನು ವಿವಾಹ ಮಾಡಿಕೊಡು ತ್ತೇನೆ’ ಎಂದ. ಪುಷಿ ಅದಕ್ಕೆ ಒಪ್ಪಿದ.

ತನ್ನನ್ನು ವರಿಸುವವಳನ್ನು ತಾನು ಸ್ವೀಕರಿಸುವುದಾಗಿ ಸೌಭರಿ ಮಹರ್ಷಿ ಹೇಳಿದ ನಾದರೂ, ತನ್ನನ್ನು ಯಾರು ವರಿಸಿಯಾರು ಎಂಬ ಆತಂಕ ಆತನ ಮನಸ್ಸಿನಲ್ಲಿ ಹುಟ್ಟಿತು. ಕ್ಷಣಕಾಲ ಆತ ಹಾಗೆಯೇ ಆಲೋಚಿಸಿದ. ತನ್ನ ತಪಶ್ಶಕ್ತಿಯೇನು ಸಾಮಾನ್ಯವಾದುದೆ? ತನ್ನ ಯೋಗಶಕ್ತಿಯ ಬಲದಿಂದ ತಾನು ತರುಣನಾಗಲಾರನೆ? ಅಪ್ಸರಸಿಯರು ಕೂಡ ತನ್ನನ್ನು ಕಂಡು ಮೋಹಿಸುವಂತಹ ರೂಪನ್ನು ಧರಿಸಿದರಾಯಿತು. ಹೀಗೆಂದು ನೆನೆದ ಸೌಭರಿಪುಷಿಯು ರಾಜನು ಕಳುಹಿಸಿದ ದೂತರೊಡನೆ ರಾಜಕುಮಾರಿಯರಿದ್ದ ಅಂತಃ ಪುರಕ್ಕೆ ಹೋದನು. ಹೋಗು ಹೋಗುತ್ತಿರುವಂತಯೆ ಆತನು ಮನ್ಮಥನಂತಹ ಸುಂದರ ಯುವಕನಾದನು. ರಾಜಕುಮಾರಿಯರು ಆತನನ್ನು ಕಾಣುತ್ತಲೆ ಮೋಹಪರವಶರಾದರು. ಚಿಕ್ಕಂದಿನಿಂದ ಅವರು ಪರಸ್ಪರ ಅಕ್ಕರೆಯುಳ್ಳವರಾಗಿದ್ದರೂ ಯುವಕನನ್ನು ವರಿಸುವ ವಿಚಾರದಲ್ಲಿ ಮಾತ್ರ ಅವರ ಸ್ನೇಹ ಹಾರಿಹೋಯಿತು. ಅವರಲ್ಲಿ ಪ್ರತಿಯೊಬ್ಬರೂ ‘ನಾನೆ ಈತನನ್ನು ವರಿಸುತ್ತೇನೆ’ ಎಂದು ಹೇಳಿ, ಪರಸ್ಪರ ಕಿತ್ತಾಡುವುದಕ್ಕೆ ಪ್ರಾರಂಭಿಸಿದರು. ಇದನ್ನು ಕಂಡ ಸೌಭರಿ ಮಂದಹಾಸವನ್ನು ಬೀರುತ್ತಾ, ತನ್ನನ್ನು ಮೆಚ್ಚಿದ ಆ ಐವತ್ತು ಮಂದಿಯನ್ನೂ ಮದುವೆಯಾಗುವುದಾಗಿ ಹೇಳಿದನು. ರಾಜಕುಮಾರಿಯರು ಇದಕ್ಕೆ ಒಪ್ಪಿ ದರು. ಮಾಂಧಾತನು ತನ್ನ ಐವತ್ತು ಮಕ್ಕಳನ್ನೂ ಆತನಿಗೆ ಧಾರೆಯೆರೆದುಕೊಟ್ಟನು.

ಸೌಭರಿಯ ತಪೋಮಹಿಮೆಯಿಂದ ಐವತ್ತು ಅರಮನೆಗಳು ಸಿದ್ಧವಾದವು. ಒಂದೊಂದರ ಸುತ್ತಲೂ ಸುಂದರವಾದ ಉದ್ಯಾನವನ, ಅದರ ಮಧ್ಯದಲ್ಲಿ ಕಮಲದ ಸರೋವರ, ಅದರಲ್ಲಿ ನಾನಾ ಬಗೆಯ ಹಕ್ಕಿಗಳು, ಅವುಗಳ ಇಂಪಾದ ಗಾನ, ದುಂಬಿಗಳ Ïು|0ಕಾರ. ನಂದನವನವನ್ನು ಹೀಯಾಳಿಸುವ ಈ ಉದ್ಯಾನವನದ ಮಧ್ಯೆ ಇದ್ದ ಅರ ಮನೆ ಎಲ್ಲ ಬಗೆಯ ಭೋಗಸಾಮಗ್ರಿಗಳಿಂದ ಸಜ್ಜಾಗಿತ್ತು. ಅಲ್ಲಿನ ಚಿತ್ರವಿಚಿತ್ರವಾದ ವಸ್ತುಗಳು, ಹೂಮಾಲೆಗಳು, ಆಭರಣಗಳು, ಸುವಾಸನ ದ್ರವ್ಯಗಳು, ಸುಪ್ಪತ್ತಿಗೆಗಳಿಂದ ಕೂಡಿದ ಮಂಚಗಳು, ಆಸನಗಳು, ಬಯಸಿ ಬಯಸಿ ಬಾಯ್ಬಿಡುವಂತಹ ಆಹಾರ ಪಾನೀಯಗಳು, ಹೇಳಿದ ಕೆಲಸವನ್ನು ಒಡನೆಯೆ ನಡೆಸುವುದ್ಕಾಗಿ ಕೈ ಕಟ್ಟಿ ಸಿದ್ಧರಾಗಿ ನಿಂತಿರುವ ದಾಸದಾಸಿಯುರು! ಸೌಭರಿಯು ತನ್ನ ಮಡದಿಯರಲ್ಲಿ ಒಬ್ಬೊಬ್ಬರನ್ನು ಒಂದೊಂದು ಅರಮನೆಯ ಒಡತಿಯನ್ನಾಗಿ ಮಾಡಿದನು. ತಾನು ಐವತ್ತು ಶರೀರಗಳನ್ನು ಧರಿಸಿ, ತನ್ನ ಮಡದಿಯರೊಡನೆ ಪ್ರತ್ಯೇಕ ಪ್ರತ್ಯೇಕವಾಗಿ ವಿಹರಿಸುತ್ತಾ ಅವರನ್ನು ಸಂತೋಷದ ಶಿಖರಕ್ಕೆ ಏರಿಸಿದನು! ಆತನ ಸಿರಿಯನ್ನೂ ಭೋಗಭಾಗ್ಯಗಳನ್ನೂ ಕಂಡು ಮಾಂಧಾತ ಆಶ್ಚರ್ಯಪಡುತ್ತಿದ್ದನು. ಇಡೀ ಭೂಮಂಡಲಕ್ಕೆ ಚಕ್ರವರ್ತಿಯಾಗಿದ್ದರೂ ಸೌಭರಿಯ ಭಾಗ್ಯಕ್ಕೆ ಅದು ಸರಿಬಾರದೆಂದು ವಿನಮ್ರನಾಗುತ್ತಿದ್ದನು. ಸೌಭರಿಯ ಇಂದ್ರಿಯಗಳು ಭೋಗಸುಖವನ್ನು ಕುಡಿಕುಡಿದು ತೇಗಿದುವು. ಆದರೆ ಇಂದ್ರಿಯಸುಖಕ್ಕೆ ತೃಪ್ತಿಯೆಂಬುದುಂಟೆ? ತುಪ್ಪವನ್ನು ಸುರಿದಷ್ಟೂ ಬೆಂಕಿ ಪ್ರಜ್ವಲಿಸುವಂತೆ, ಭೋಗಿಸಿ ದಷ್ಟೂ ಭೋಗದ ಆಶೆ ಹೆಚ್ಚುತ್ತದೆ. ಸೌಭರಿ ಬಹುಕಾಲ ಈ ಸುಖವನ್ನು ಅನುಭವಿಸಿದರೂ ತೃಪ್ತನಾಗಲಿಲ್ಲ. ಆತನ ದಾಹ ದಿನದಿನಕ್ಕೆ ಹೆಚ್ಚುತ್ತಿತ್ತು. ವಿಚಾರಪರನಾದ ಆತ ಒಮ್ಮೆ ಇದನ್ನು ಗಮನಿಸಿ, ತನ್ನ ಅವಿವೇಕಕ್ಕಾಗಿ ತಾನೆ ಪಶ್ಚಾತ್ತಾಪಪಟ್ಟ. ‘ಹೇಗಿದ್ದವನು ನಾನು ಹೇಗಾಗಿಬಿಟ್ಟೆ! ಕೇವಲ ಎರಡು ಮೀನುಗಳ ರತಿಸುಖವನ್ನು ಕಂಡು ನನ್ನ ಬಾಳೇ ಹಾಳಾಗಿ ಹೋಯಿತಲ್ಲ! ಸಹವಾಸದೋಷ ಎಷ್ಟು ಭಯಂಕರವಾದುದು? ಮೋಕ್ಷವನ್ನು ಗಳಿಸ ಬೇಕೆನ್ನುವವನು ಸಂಸಾರಿಗಳನ್ನು ಮರೆತಾದರೂ ಕಣ್ಣೆತ್ತಿ ನೋಡಬಾರದು’ ಎಂದು ಕೊಂಡ. ಒಡನೆಯೆ ಆತನು ತನ್ನ ಐವತ್ತು ಜನ ಹೆಂಡಿರನ್ನೂ, ಆ ವೇಳೆಗೆ ಹುಟ್ಟಿದ್ದ ಅರವತ್ತು ಸಾವಿರ ಮಕ್ಕಳನ್ನೂ ಬಿಟ್ಟು ತಪೋವನಕ್ಕೆ ಹೋದನು. ಪತಿವ್ರತೆಯರಾದ ಪತ್ನಿಯರೂ ಆತನನ್ನು ಹಿಂಬಾಲಿಸಿದರು. ಅವರೆಲ್ಲರೂ ಭಗವದಾರಾಧನೆಯಿಂದ ಮುಕ್ತಿ ಯನ್ನು ಪಡೆದರು.

ಮಾಂಧಾತನು ಬಹುಕಾಲ ಧರ್ಮದಿಂದ ರಾಜ್ಯಭಾರ ಮಾಡುತ್ತಿದ್ದು ಕೊನೆಗೆ ಪುರುಕುತ್ಸ, ಅಂಬರೀಶ, ಮುಚುಕುಂದ–ಎಂಬ ತನ್ನ ಮೂವರು ಮಕ್ಕಳಿಗೆ ರಾಜ್ಯವನ್ನು ವಹಿಸಿ, ತಪಸ್ಸಿನಿಂದ ಸದ್ಗತಿಯನ್ನು ಪಡೆದನು.


\section{(೮) ತ್ರಿಶಂಕುವಿನ ಮಗ ಹರಿಶ್ಚಂದ್ರ}

ಮಾಂಧಾತನ ಹಿರಿಯಮಗನಾದ ಪುರುಕುತ್ಸನ ಪೀಳಿಗೆಯಲ್ಲಿ ಸತ್ಯವ್ರತನೆಂಬ ಮಹಾ ಪುರುಷ ಹುಟ್ಟಿದ. ಆತನಿಗೆ ತ್ರಿಶಂಕು\footnote{೧. ಶಂಕು ಎಂದರೆ ಗೂಟ. ಗೂಟದಂತೆ ಸ್ಥಿರವಾದ ಮೂರು ದೋಷಗಳುಳ್ಳವನಾದ್ದರಿಂದ ಇವನು ತ್ರಿಶಂಕು. [ಪಿಶುಶ್ಚಾಪರಿತೋಷೇಣ ಗುರೋರ್ದೋಗ್ಥ್ರಿವಧೇನಚ ಆಪ್ರೋಕ್ಷಿತೋಪ ಯೋಗಾಚ್ಚತ್ರಿವಿಧಸ್ತೇ ವ್ಯತಿಕ್ರಮಃ]}ವೆಂದು ಹೆಸರಾಯಿತು. ಈತನು ತನ್ನ ಕುಲ ಗುರುಗಳಾದ ವಸಿಷ್ಠರ ಶಾಪದಿಂದ ಚಂಡಾಲನಾದನು. ಆದರೂ ವಿಶ್ವಾಮಿತ್ರರನ್ನು ಆಶ್ರಯಿಸಿ, ಅವರ ತಪೋಮಹಿಮೆಯಿಂದ ಸಶರೀರನಾಗಿಯೆ ಸ್ವರ್ಗಕ್ಕೆ ಹೋದನು. ಆದರೆ ದೇವತೆಗಳು ಚಂಡಾಲನಾದ ಆತನನ್ನು ಸ್ವರ್ಗಕ್ಕೆ ಸೇರಿಸದೆ ಕೆಳಕ್ಕೆ ನೂಕಿದರು. ಆತ ತಲೆ ಕೆಳಗಾಗಿ ನೆಲಕ್ಕೆ ಬೀಳುತ್ತಾ ‘ರಕ್ಷಿಸು, ರಕ್ಷಿಸು’ ಎಂದು ವಿಶ್ವಾಮಿತ್ರರನ್ನು ಬೇಡಿದ. ಅವರು ಆತನ ಮೇಲಿನ ಕರುಣೆಯಿಂದ ಆತನಿಗಾಗಿಯೇ ಬೇರೊಂದು ಸ್ವರ್ಗವನ್ನು ನಿರ್ಮಿಸಿ, ಅಲ್ಲಿಗೆ ಆತನನ್ನೇ ಇಂದ್ರನನ್ನಾಗಿ ಮಾಡಿದರು. ಈಗಲೂ ಆತನು ನಕ್ಷತ್ರ ರೂಪಿನಿಂದ ಆಕಾಶದಲ್ಲಿ ಕಾಣಬರುತ್ತಾನೆ. ಈ ತ್ರಿಶಂಕುವಿನ ಮಗನೆ ಪುರಾಣಪುರುಷನಾದ ಹರಿ ಶ್ಚಂದ್ರ. ಈ ಹರಿಶ್ಚಂದ್ರನ ಕಾರಣದಿಂದ ವಸಿಷ್ಠ ವಿಶ್ವಾಮಿತ್ರರಿಬ್ಬರೂ ಪರಸ್ಪರ ಶಾಪ ದಿಂದ ಹಕ್ಕಿಗಳಾಗಿ\footnote{೧. ಒಮ್ಮೆ ದೇವೇಂದ್ರನ ಸಭೆಯಲ್ಲಿ ಹರಿಶ್ಚಂದ್ರನ ಸತ್ಯವನ್ನು ಭಂಗಮಾಡುವುದಾಗಿ ವಿಶ್ವಾಮಿತ್ರರು ಪ್ರತಿಜ್ಞೆ ಮಾಡಿದರು. ಆ ಪ್ರತಿಜ್ಞೆಯನ್ನು ನೆರವೇರಿಸುವುದಕ್ಕಾಗಿ ಆತ ಹರಿಶ್ಚಂದ್ರನಿಂದ ಯಾಗ ದಕ್ಷಿಣೆಯನ್ನು ಬೇಡುವ ನೆಪದಿಂದ ಅವನ ಸರ್ವಸ್ವವನ್ನೂ ಸುಲಿದುಕೊಂಡು ಹಿಂಸೆಪಡಿಸಿದರು. ಇದನ್ನು ಸಹಿಸಲಾರದೆ ವಸಿಷ್ಠರು ‘ನೀರುಕೋಳಿಯಾಗು’ ಎಂದು ವಿಶ್ವಾಮಿತ್ರರನ್ನು ಶಪಿಸಿದರು. ವಿಶ್ವಾಮಿತ್ರರು ರೇಗಿ ‘ನೀನು ಕೊಕ್ಕರೆಯಾಗು’ ಎಂದು ವಸಿಷ್ಠರನ್ನು ಶಪಿಸಿದರು.} ಬಹುಕಾಲ ಹೋರಾಡುತ್ತಿದ್ದರು. ಅವರಿಬ್ಬರ ಜಗಳದ ಕಥೆ ಹಾಗಿರಲಿ, ಮಹಾರಾಜನಾಗಿ ಧರ್ಮದಿಂದ ರಾಜ್ಯಭಾರ ಮಾಡುತ್ತಿದ್ದ ಹರಿಶ್ಚಂದ್ರನಿಗೆ ಬಹುಕಾಲವಾದರೂ ಮಕ್ಕಳಾಗಲಿಲ್ಲ. ಆತನು ವರುಣದೇವನನ್ನು ಆರಾಧಿಸಿ ಆತನು ಪ್ರತ್ಯಕ್ಷನಾಗಲು ‘ಸ್ವಾಮಿ, ನನಗೊಬ್ಬ ಮಗನಾಗುವಂತೆ ಅನುಗ್ರಹಿಸು. ಮಕ್ಕಳಿಲ್ಲದವ ನೆಂಬ ನಿಂದೆ ಹೋದರೆ ಸಾಕು. ನಿನ್ನ ಅನುಗ್ರಹದಿಂದ ಹುಟ್ಟಿದ ಮಗನನ್ನು ನರಪಶು ವನ್ನಾಗಿ ಮಾಡಿ, ನಿನ್ನ ಹೆಸರಿನಲ್ಲಿಯೇ ಯಜ್ಞದಲ್ಲಿ ಆಹುತಿಕೊಡುತ್ತೇನೆ’ ಎಂದು ಬೇಡಿ ಕೊಂಡ. ವರುಣದೇವನು ‘ತಥಾಸ್ತು’ಎಂದು ಹೇಳಿ ಮಾಯವಾದನು.

ವರುಣದೇವನ ವರದಿಂದ ಹರಿಶ್ಚಂದ್ರನಿಗೆ ಒಬ್ಬ ಮಗ ಹುಟ್ಟಿದ. ಅವನ ಹೆಸರು ರೋಹಿತ. ಅವನು ಹುಟ್ಟುತ್ತಲೆ ವರುಣನು ಹರಿಶ್ಚಂದ್ರನ ಬಳಿಗೆ ಬಂದು ‘ಎಲ್ಲಯ್ಯ, ಯಾಗ ಮಾಡಿ ಮಗನನ್ನು ಬಲಿಕೊಡು ಮತ್ತೆ’ ಎಂದ. ಹರಿಶ್ಚಂದ್ರನಿಗೆ ಮಗನನ್ನು–ಅಪ ರೂಪಕ್ಕೆ ಹುಟ್ಟಿದ ವಂಶೋದ್ಧಾರಕನನ್ನು–ಬಲಿಕೊಡುವ ಮನಸ್ಸಾಗಲಿಲ್ಲ; ವರುಣ ನೊಡನೆ ‘ಸ್ವಾಮಿ, ಹುಟ್ಟಿ ಹತ್ತು ದಿನಗಳವರೆಗೆ ಬಲಿಕೊಡಬಾರದು’ ಎಂದ. ವರುಣ ಒಳ್ಳೆಯ ಜಿಗುಟು ಆಸಾಮಿ. ಹತ್ತು ದಿನಗಳಾಗುತ್ತಲೆ ಮತ್ತೆ ಬಂದು ‘ಎಲ್ಲಯ್ಯ, ಬಲಿ?’ ಎಂದ. ಹರಿಶ್ಚಂದ್ರ ‘ಸ್ವಾಮಿ, ಹಲ್ಲು ಹುಟ್ಟುವವರೆಗೆ ಬಲಿ ಕೊಡಬಾರದು’ ಎಂದ. ಹಲ್ಲು ಹುಟ್ಟುತ್ತಲೆ ವರುಣ ಬಂದು ಕೇಳಿದಾಗ ‘ಹುಟ್ಟಿದ ಹಲ್ಲುಬಿದ್ದ ಮೇಲೆ’ ಎಂದ; ಬಿದ್ದಮೇಲೆ ಬಂದು ಕೇಳಲು ‘ಪುನಃ ಹಲ್ಲು ಹುಟ್ಟಿದ ಮೇಲೆ’. ಆಮೇಲೆ ಬಂದು ಕೇಳಿದರೆ ‘ಅಯ್ಯೋ ನನ್ನ ಮಗ ಕ್ಷತ್ರಿಯ; ಅವನಿಗೆ ಯುದ್ಧ ಮಾಡುವ ಯೋಗ್ಯತೆ ಬಂದಮೇಲೆ ಬಲಿಕೊಡಬೇಕು. ಅದಕ್ಕೆ ಮೊದಲೆ ಕೊಟ್ಟೇನು ಪ್ರಯೋಜನ?’ ಎಂದ. ಹೀಗೆ ಕಾಲವನ್ನು ತಳ್ಳುತ್ತಿರಲು ರೋಹಿತನಿಗೆ ಯೌವನ ಬಂತು. ತನ್ನ ತಂದೆಯು ವರುಣದೇವನಿಗೆ ಹರಕೆ ಹೊತ್ತಿರುವುದನ್ನು ಕೇಳಿ ತಿಳಿದ ಆ ಯುವಕ ಪ್ರಾಣಭಯದಿಂದ ಯಾರಿಗೂ ತಿಳಿಯದಂತೆ ತನ್ನ ಬಿಲ್ಲುಬಾಣಗಳೊಡನೆ ಅರಣ್ಯಕ್ಕೆ ಓಡಿಹೋದ. ಇದನ್ನು ಕಂಡು ವರುಣನಿಗೆ ಕೆಂಡ ದಂತಹ ಕೋಪ ಬಂತು. ಆತನು ಹರಿಶ್ಚಂದ್ರನಿಗೆ ಜಲೋದರ ರೋಗ ಬಂದು, ಅವನ ಹೊಟ್ಟೆಯ ತುಂಬ ನೀರು ತುಂಬಿಕೊಳ್ಳುವಂತೆ ಮಾಡಿದ.

ತಂದೆಗೆ ಹೊಟ್ಟೆ ಊದಿಕೊಂಡು, ಆತನು ಅದರಿಂದ ಸಂಕಟಪಟುತ್ತಿರುವ ಸುದ್ದಿ ರೋಹಿತಾಶ್ವನಿಗೆ ಗೊತ್ತಾಯಿತು. ಆತನು ತಂದೆಯನ್ನು ನೋಡಬೇಕೆಂದು ರಾಜಧಾನಿಗೆ ಹೊರಟ. ದಾರಿಯಲ್ಲಿ ದೇವೇಂದ್ರನು ಮುದುಕನ ರೂಪದಲ್ಲಿ ಕಾಣಿಸಿಕೊಂಡು, ‘ಮಗು, ನೀನು ರಾಜಧಾನಿಗೆ ಹೋಗಬೇಡ, ನಿಮ್ಮ ತಂದೆ ಸುಳ್ಳು ಹೇಳಿ ಈ ರೋಗಕ್ಕೆ ತುತ್ತಾಗಿ ದ್ದಾನೆ. ಆ ಪಾಪ ಹೋಗಬೇಕಾದರೆ ನೀನು ತೀರ್ಥಯಾತ್ರೆ ಮಾಡಬೇಕು’ ಎಂದನು. ರೋಹಿತನು ಹಾಗೆಯೇ ಆಗಲೆಂದು ಪುಣ್ಯಕ್ಷೇತ್ರಗಳ ಯಾತ್ರೆಯನ್ನು ಕೈಕೊಂಡನು. ಒಂದು ವರ್ಷಕಾಲ ಆತನು ದೇಶದೇಶಗಳಲ್ಲೆಲ್ಲ ಅಲೆದು ರಾಜಧಾನಿಗೆ ಹಿಂದಿರುಗಿದನು. ಆದರೆ ದಾರಿಯಲ್ಲಿ ಹಿಂದೆ ಕಾಣಿಸಿದ್ದ ಮುದುಕನೆ ಮತ್ತೆ ಕಾಣಿಸಿಕೊಂಡು, ಮೊದಲಿ ನಂತೆಯೆ ಹೇಳಿ, ಅವನನ್ನು ಹಿಂದಕ್ಕೆ ಕಳಿಸಿದನು. ಮುಂದಿನ ವರ್ಷವೂ ಹಾಗೆಯೇ ಆಯಿತು. ಹೀಗೆಯೇ ಐದು ವರ್ಷಗಳು ಕಳೆದವು. ಆರನೆಯ ವರ್ಷ ಆತನು ಯಾತ್ರೆ ಯಿಂದ ಹಿಂದಿರುಗುವಾಗ ಶುನಶ್ಶೇಫನೆಂಬ ಬ್ರಾಹ್ಮಣ ಬಾಲಕನೊಬ್ಬನನ್ನು ಕ್ರಯಕ್ಕೆ ಕೊಂಡುಕೊಂಡು ಬಂದು, ತನ್ನ ತಂದೆಗೆ ಒಪ್ಪಿಸಿ, ತನಗೆ ಬದಲಾಗಿ ಆ ಬ್ರಾಹ್ಮಣ ಬಾಲಕನನ್ನು ವರುಣನಿಗೆ ಯಜ್ಞಮಾಡುವಂತೆ ತಿಳಿಸಿದನು. ಹರಿಶ್ಚಂದ್ರನು ಅದರಂತೆ ನಡೆಸಿ, ತನ್ನ ರೋಗದಿಂದ ಮುಕ್ತನಾದ. ಆ ಯಜ್ಞದಿಂದ ಸಂತೋಷಗೊಂಡ ದೇವೇಂದ್ರನು ಹರಿಶ್ಚಂದ್ರ ಮಹಾರಾಜನಿಗೆ ದಿವ್ಯವಾದ ಒಂದು ರಥವನ್ನು ಬಹುಮಾನವಾಗಿ ಕೊಟ್ಟನು.

ಹರಿಶ್ಚಂದ್ರ ಬಹುಕಾಲ ಧರ್ಮದಿಂದ ರಾಜ್ಯಭಾರ ಮಾಡುತ್ತಿದ್ದು, ಕಡೆಗೆ ವಿಶ್ವಾಮಿತ್ರ ಮಹರ್ಷಿಗಳಿಂದ ಆತ್ಮಜ್ಞಾನವನ್ನು ಪಡೆದನು. ಅನಂತರ ಮಗನಿಗೆ ರಾಜ್ಯಭಾರವನ್ನು ವಹಿಸಿ, ಆತ್ಮವನ್ನು ಅನುಸಂಧಾನಮಾಡುತ್ತಾ ಆನಂದಮಯನಾದನು.


\section{(೯) ಸಗರ ಚಕ್ರವರ್ತಿ}

ಹರಿಶ್ಚಂದ್ರನ ತರುವಾಯ ರೋಹಿತನು ರಾಜನಾದ. ಆತನ ತರುವಾಯ ಆತನ ಮಗ ಚಂಪನು ರಾಜ್ಯಭಾರಮಾಡುತ್ತಾ ಚಂಪಾಪುರವನ್ನು ಕಟ್ಟಿಸಿ, ತನ್ನ ಹೆಸರನ್ನು ಸಾರ್ಥಕಪಡಿಸಿಕೊಂಡ. ಇವನ ತರುವಾಯ ಕೆಲವು ತಲೆಗಳಾದ ಮೇಲೆ ಬಾಹುಕನೆಂಬು ವನು ಪಟ್ಟಕ್ಕೆ ಬಂದ. ಈತ ಹಗೆಗಳಿಗೆ ರಾಜ್ಯವನ್ನು ಸೋತು, ಮಡದಿಯರೊಡನೆ ಅಡವಿಯ ಪಾಲಾದ. ಕೆಲಕಾಲವಾದ ಮೇಲೆ ಅವನು ಮೃತ್ಯುವಿಗೆ ತುತ್ತಾಗಲು, ಅವನ ಹೆಂಡತಿಯೂ ಅವನೊಡನೆ ಸಹಗಮನಕ್ಕೆ ಸಿದ್ಧಳಾದಳು. ಆಗ ಆಕೆ ಗರ್ಭಿಣಿ. ಇದನ್ನು ತಿಳಿದ ಔರ್ವನೆಂಬ ಮಹರ್ಷಿ ಆಕೆಯನ್ನು ಸಹಗಮನ ಮಾಡದಂತೆ ತಡೆದ. ಆಕೆ ಗರ್ಭಿಣಿ ಯೆಂಬುದನ್ನು ಕೇಳಿ, ಆಕೆಯ ಸವತಿಯರು ಹೊಟ್ಟೆಕಿಚ್ಚಿನಿಂದ ಆಕೆಗೆ ವಿಷವಿಟ್ಟರು. ಆದರೂ ಔರ್ವಪುಷಿಯ ಅನುಗ್ರಹದಿಂದ ಆಕೆ ಸಾಯದೆ ಉಳಿದುಕೊಂಡು, ಒಬ್ಬ ಮಗ ನನ್ನು ಹೆತ್ತಳು. ಆ ಕೂಸಿನೊಡನೆ ವಿಷವೂ ಹೊರಕ್ಕೆ ಬಂತು. ಇದರಿಂದ ಆ ಕೂಸಿಗೆ ‘ಸಗರ’ ಎಂದು ಹೆಸರಾಯಿತು. ಔರ್ವಮಹರ್ಷಿಯೆ ಆ ಮಗುವನ್ನು ಬೆಳೆಸಿ, ವಿದ್ಯಾವಂತ ನನ್ನಾಗಿ ಮಾಡಿದನು. ವಯಸ್ಸಿನೊಡನೆ ಕ್ಷಾತ್ರತೇಜಸ್ಸೂ ಬೆಳೆಯಿತು. ಆತನು ಹಗೆಗಳನ್ನು ಗೆದ್ದು ತನ್ನ ರಾಜ್ಯವನ್ನು ಮತ್ತೆ ಸಂಪಾದಿಸಿದುದಲ್ಲದೆ ದಿಗ್ವಿಜಯಕ್ಕೆ ಹೊರಟು ತಾಳಜಂಘ, ಯವನ, ಶಕ, ಬರ್ಬರ ಮೊದಲಾದ ದುಷ್ಟರನ್ನು ಜಯಿಸಿ, ತನ್ನ ರಾಜ್ಯವನ್ನು ವಿಸ್ತರಿಸಿದನು.

ತನ್ನ ಶಿಷ್ಯನ ಪರಾಕ್ರಮಕ್ಕೆ ಮೆಚ್ಚಿದ ಔರ್ವಮಹರ್ಷಿಯು ಅಶ್ವಮೇಧಯಾಗವನ್ನು ಕೈಗೊಳ್ಳುವಂತೆ ಸಗರನಿಗೆ ಉಪದೇಶಿಸಿದನು. ಸಗರನು ಅತ್ಯಂತ ಸಂತೋಷದಿಂದ ಅದನ್ನು ಕೈಗೊಂಡು ವೈಭವದಿಂದ ನೆರವೇರಿಸಿದನು. ಆತನಿಗೆ ನೂರು ಅಶ್ವಮೇಧಯಾಗ ಗಳನ್ನು ಮಾಡಬೇಕೆನಿಸಿತು. ಔರ್ವ ಮಹರ್ಷಿಯೂ ಸಂತೋಷದಿಂದ ಅದಕ್ಕೆ ಒಪ್ಪಿಗೆ ಯಿತ್ತನು. ಸಗರನು ಒಂದಾದಮೇಲೊಂದರಂತೆ ತೊಂಬತ್ತೊಂಬತ್ತು ಯಾಗಗಳನ್ನು ನಿರ್ವಿಘ್ನವಾಗಿ ನೆರವೇರಿಸಿ, ನೂರನೆಯ ಯಾಗವನ್ನು ಕೈ ಕೊಂಡನು. ಅದೊಂದನ್ನು ಮಾಡಿ ಮುಗಿಸಿದರೆ ಆತನಿಗೆ ಇಂದ್ರಪದವಿ ದೊರಕುತ್ತಿತ್ತು. ಇದನ್ನು ಕಂಡು ದೇವೇಂದ್ರ ನಿಗೆ ಭಯಹತ್ತಿತು. ತನ್ನ ಪದವಿಗೆ ಮೋಸವಾಗದಿರಬೇಕಾದರೆ ಈ ಯಾಗವನ್ನು ಕುಲಗೆಡಿಸ ಬೇಕೆಂದು ಆತನು ನಿಶ್ಚಯಿಸಿದ. ಆದ್ದರಿಂದ ಆತನು ಯಾಗದ ಕುದುರೆಯನ್ನು ಕದ್ದೊಯ್ದು ಪಾತಾಳದಲ್ಲಿ ತಪಸ್ಸು ಮಾಡುತ್ತಾ ಕುಳಿತಿದ್ದ ಕಪಿಲಮಹರ್ಷಿಗಳ ಹಿಂದೆ ಬಚ್ಚಿಟ್ಟನು. ಸಗರನು ತನ್ನ ಕಿರಿಯ ಹೆಂಡತಿ, ಸುಮತಿಯ ಅರವತ್ತು ಸಾವಿರ ಮಕ್ಕಳನ್ನು ಕರೆದು, ಯಜ್ಞದ ಕುದುರೆಯನ್ನು ಹುಡುಕಿತರುವಂತೆ ನೇಮಿಸಿದನು. ಅವರು ಭೂಮಂಡಲವನ್ನೆಲ್ಲ ಹುಡುಕಿ, ಎಲ್ಲಿಯೂ ಆ ಕುದುರೆ ಕಾಣಿಸದಿರಲು, ನೆಲವನ್ನು ತೋಡಿ ಪಾತಾಳವನ್ನು ಪ್ರವೇಶಿಸಿದರು. ಅಲ್ಲಿ ಹುಡುಕುತ್ತಾ ಹೊರಟಾಗ ಕಪಿಲಪುಷಿಗಳ ಹಿಂದೆ ಕುದುರೆ ಕಾಣಿಸಿತು. ಅವರಿಗೆ ಆ ಪುಷಿಯೇ ಕುದುರೆಯ ಕಳ್ಳನೆಂದು ಭ್ರಮೆ. ಆದ್ದರಿಂದ ‘ಕಳ್ಳ ಸಿಕ್ಕ, ಅವನನ್ನು ಬಡಿಯಿರಿ, ಕಡಿಯಿರಿ, ಕಳ್ಳಕೊರಮ ಕಣ್ಣುಮುಚ್ಚಿ ಕುಳಿತಿದ್ದಾನೆ’ ಎಂದು ಗರ್ಜಿಸಿದರು. ಅರವತ್ತು ಸಾವಿರಜನದ ಕೂಗಿಗೆ ಯಾವ ತಪಸ್ಸು ನಿಂತೀತು? ಪುಷಿಯು ಸಮಾಧಿಯಿಂದೆದ್ದು ಕಣ್ಣುತೆರೆದನು. ಒಡನೆಯೆ ಅವರೆಲ್ಲ ಸುಟ್ಟು ಬೂದಿಯಾಗಿ ಹೋದರು. ಕಪಿಲಮಹರ್ಷಿ ಶಾಂತಮೂರ್ತಿ. ಆತನೇನೂ ಅವರನ್ನು ಸುಡ ಬೇಕೆಂದು ಬಗೆಯಲಿಲ್ಲ. ಆ ಸಗರನ ಮಕ್ಕಳ ಅವಿವೇಕವೇ ಅವರನ್ನು ಸುಟ್ಟುಹಾಕಿತು, ಅಷ್ಟೆ.

ಕುದುರೆ ಬರಲಿಲ್ಲ, ಮಕ್ಕಳೂ ಹಿಂದಿರುಗಲಿಲ್ಲ, ಯಜ್ಞ ಅರ್ಧಕ್ಕೆ ನಿಂತಿದೆ. ಸಗರ ಚಕ್ರವರ್ತಿ ಮುಂದೇನು ಮಾಡಬೇಕು? ಆತನ ಹಿರಿಯ ಹೆಂಡಿತಿ ಕೇಶಿನಿಗೆ ಒಬ್ಬನೇ ಮಗ ಅಸಮಂಜ. ಆತನು ಪೂರ್ವ ಜನ್ಮದಲ್ಲಿ ತಪಸ್ಸು ಕೆಟ್ಟು ಹುಟ್ಟಿದವನು. ಆತನಿಗೆ ಆ ಜನ್ಮದ ಸ್ಮರಣೆಯಿದೆ. ಆದ್ದರಿಂದ ಆತ ಜನರಿಂದ ಸದಾ ದೂರವಾಗಿರುತ್ತಿದ್ದನು. ನೋಡಿ ದವರಿಗೆ ಹುಚ್ಚನಂತೆ ಕಾಣಿಸುತ್ತಿದ್ದನು. ಮಕ್ಕಳು ಯಾರಾದರೂ ಸಿಕ್ಕರೆ ಅವರನ್ನು ಸರಯೂನದಿಯಲ್ಲಿ ಎತ್ತಿಹಾಕಿ ಬಿಡುತ್ತಿದ್ದನು. ಇದರಿಂದ ಕುಪಿತನಾದ ಸಗರಚಕ್ರವರ್ತಿ ಅವನು ಮಗನೆಂಬ ಮೋಹವನ್ನೂ ತೊರೆದು ಅವನನ್ನು ದೇಶದಿಂದ ಓಡಿಸಿಬಿಟ್ಟಿದ್ದನು. ಅವನಿಗೆ ಬೇಕಾಗಿದ್ದದೂ ಅಷ್ಟೆ. ತನ್ನ ಯೋಗಶಕ್ತಿಯಿಂದ ಸರಯೂ ನದಿಗೆ ತಳ್ಳಿದ್ದ ಮಕ್ಕಳನ್ನೆಲ್ಲ ಬದುಕಿಸಿಕೊಟ್ಟು ಅಡವಿಗೆ ಹೊರಟುಹೋಗಿದ್ದನು. ಅವನಿಗೆ ಅಂಶುಮಂತ ನೆಂಬ ಒಬ್ಬ ಮಗನಿದ್ದ. ಅವನಿಗಿನ್ನೂ ಚಿಕ್ಕವಯಸ್ಸು. ಆದರೂ ಈಗ ಕುದುರೆಯನ್ನೂ ಚಿಕ್ಕಪ್ಪಂದಿರನ್ನೂ ಹುಡುಕಿಕೊಂಡು ಬರಲು ಅವನೇ ಹೋಗಬೇಕಾಯಿತು. ಅವನು ತನ್ನ ಚಿಕ್ಕಪ್ಪಂದಿರು ಅಗೆದು ಮಾಡಿದ್ದ ಹಾದಿಯಿಂದಲೆ ಪಾತಾಳಕ್ಕೆ ಹೋದನು. ಅಲ್ಲಿ ತಪಸ್ಸು ಮಾಡುತ್ತಿರುವ ಕಪಿಲಪುಷಿ, ಎದುರಿಗೆ ಬೂದಿಯ ರಾಶಿ, ಪಕ್ಕದಲ್ಲಿ ಕುದುರೆ ಕಾಣಿಸಿದವು. ಆತ ನೇರವಾಗಿ ಪುಷಿಯ ಬಳಿಗೆ ಹೋಗಿ, ಭಕ್ತಿಯಿಂದ ಅವರಿಗೆ ಅಡ್ಡಬಿದ್ದ. ಅವರು ಆತನನ್ನು ಅಕ್ಕರೆಯಿಂದ ನೋಡಿ, ‘ಮಗು, ಇಲ್ಲಿರುವ ಬೂದಿಯ ರಾಶಿ ನಿನ್ನ ಚಿಕ್ಕಪ್ಪಂ ದಿರದು. ದೇವಗಂಗೆಯನ್ನು ಇವರ ಮೇಲೆ ಹರಿಸಿದ ಹೊರತು ಇವರಿಗೆ ಸದ್ಗತಿಯಿಲ್ಲ. ಅದು ನಿನ್ನಿಂದ ಆಗದ ಕೆಲಸ. ಈಗ ನೀನು ಈ ಯಜ್ಞದ ಕುದುರೆಯನ್ನು ಕೊಂಡೊಯ್ದು ತಾತನಿಗೆ ಒಪ್ಪಿಸು’ ಎಂದರು. ಅಂಶುಮಂತನು ಅವರ ಅಪ್ಪಣೆಯಂತೆ ಕುದುರೆಯನ್ನು ಸಗರನಿಗೆ ಒಪ್ಪಿಸಿದ. ಯಾಗವೇನೋ ಮುಗಿಯಿತು. ಆದರೆ ಸಗರನ ಮನಸ್ಸು ಮಕ್ಕಳ ಸಾವಿ ನಿಂದ ಕಲಕಿಹೋಗಿತ್ತು. ಆತನು ಅಂಶುಮಂತನಿಗೆ ಪಟ್ಟಕಟ್ಟಿ, ತಪಸ್ಸಿಗೆ ಹೊರಟುಹೋದ.


\section{(೧೦) ಭಗೀರಥ}

ಅಂಶುಮಂತನು ರಾಜನಾದರೂ ಆತನಿಗೆ ರಾಜ್ಯಭೋಗಗಳೊಂದೂ ಬೇಕಾಗಿಲ್ಲ. ತನ್ನ ಚಿಕ್ಕಪ್ಪಂದಿರಿಗೆ ಸದ್ಗತಿಯನ್ನು ಕಾಣಿಸುವುದಕ್ಕೆ ತನಗೆ ಸಾಧ್ಯವಿಲ್ಲದುದಕ್ಕಾಗಿ ಆತನಿಗೆ ಹಗಲಿರುಳೂ ಚಿಂತೆ. ಆದರೆ ಬರಿಯ ಚಿಂತೆಯಿಂದ ಕಾರ್ಯಸಾಧನೆಯಾಗುತ್ತದೆಯೆ? ಆತನು ಆ ಚಿಂತೆಯಲ್ಲಿಯೇ ಕೊರಗಿ ಸತ್ತುಹೋದ. ಆತನ ಮಗ ದಿಲೀಪ. ಆತನೂ ತನ್ನ ಜೀವಿತಕಾಲವೆಲ್ಲ ‘ದೇವಗಂಗೆಯನ್ನು ತರಬೇಕು’ ಎಂದು ಹೇಳಿಕೊಳ್ಳುತ್ತಾ ಮುಗಿಸಿದ. ಈ ದಿಲೀಪನ ಮಗನೇ ಮಹಾನುಭಾವನಾದ ಭಗೀರಥ. ಈತನು ತನ್ನ ಹಿರಿಯರಂತೆ ಚಿಂತಿಸುತ್ತಾ ಕುಳಿತುಕೊಳ್ಳಲಿಲ್ಲ. ಹಿಡಿದ ಕೆಲಸವನ್ನು ಬಿಡದೆ ಸಾಧಿಸಬೇಕೆಂಬ ಛಲ ಈತನದು. ಈತನು ದೇವಗಂಗೆಯನ್ನು ಕುರಿತು ಘೋರವಾದ ತಪಸ್ಸನ್ನು ಕೈಕೊಂಡನು. ಆ ತಪಸ್ಸಿಗೆ ಮೆಚ್ಚಿ ದೇವಗಂಗೆ ಆತನಿಗೆ ಪ್ರತ್ಯಕ್ಷಳಾದಳು. ‘ಮಗು, ನಿನಗೇನು ಬೇಕು?’ ಎಂದು ಆಕೆ ಕೇಳುತ್ತಲೆ, ಭಗೀರಥ ‘ಅಮ್ಮ, ನಮ್ಮ ಚಿಕ್ಕಪ್ಪಂದಿರನ್ನು ಉದ್ಧರಿಸು’ ಎಂದು ಅಡ್ಡಬಿದ್ದ. ಅದಕ್ಕೆ ಆಕೆ ಸಿದ್ಧಳಾಗಿದ್ದಳು. ಆದರೆ ಆಕೆ ಆಕಾಶದಿಂದ ಪಾತಾಳಕ್ಕೆ ಧುಮ್ಮಿಕ್ಕು ವುದೆಂದರೆ ಸುಲಭ ಸಾಧ್ಯವೆ? ‘ಮಗು, ಕೆಳಕ್ಕೆ ನಾನು ಧುಮ್ಮಿಕ್ಕುವಾಗ ಅದರ ರಭಸವನ್ನು ತಡೆಯುವರಾರು? ಇಡೀ ಭೂಮಂಡಲವೇ ಕೊಚ್ಚಿ ಹೋಗುವುದಲ್ಲವೆ? ಅಲ್ಲದೆ ನಾನು ಭೂಮಿಗೆ ಬರುತ್ತಲೆ ಮಹಾ ಪಾಪಿಗಳೆಲ್ಲ ನನ್ನಲ್ಲಿ ತಮ್ಮಪಾಪಗಳನ್ನೆಲ್ಲ ಕಳೆದುಕೊಂಡರೆ, ಆ ಪಾಪಗಳನ್ನು ನಾನೆಲ್ಲಿ ತೊಳೆದುಕೊಳ್ಳಲಿ?’ ಎಂದಳು. ಭಗೀರಥನು ಆಕೆಯ ಎರಡ ನೆಯ ಸಂದೇಹವನ್ನು ಒಡನೆಯೆ ಹೋಗಲಾಡಿಸಿದ. ‘ತಾಯಿ, ಪಾಪಿಗಳು ಸ್ನಾನಮಾಡು ವಂತೆ ಜ್ಞಾನಿಗಳಾದ ಮಹರ್ಷಿಗಳೂ ನಿನ್ನಲ್ಲಿ ಸ್ನಾನ ಮಾಡುವರು. ಭಗವಂತ ಅಂತಹ ಮಹಾನುಭಾವರಲ್ಲಿ ನೆಲಸಿರುತ್ತಾನೆ. ಆತನ ಸಂಪರ್ಕವಾದಮೇಲೆ ಪಾಪವೆಲ್ಲಿಯದು?’ ಎಂದನು. ದೇವಗಂಗೆಯೂ ಅದನ್ನು ಒಪ್ಪಿದಳು.

ಇನ್ನು ದೇವಗಂಗೆ ಕೆಳಕ್ಕಿಳಿದು ಬರುವಾಗ ಆಕೆಯ ರಭಸವನ್ನು ತಡೆಯಬಲ್ಲ ಮಹಾನು ಭಾವನೊಬ್ಬ ಬೇಕು. ಅಂತಹ ಮಹಾನುಭಾವನಾರು? ಸಾಕ್ಷಾತ್ ಪರಶಿವನೊಬ್ಬನೆ ಅದಕ್ಕೆ ಸಮರ್ಥ. ದೇವಗಂಗೆಯೂ ಅದನ್ನು ಒಪ್ಪಿದಳು. ಆಕೆ ‘ಭಗೀರಥ, ಮೊದಲು ಪರಶಿವ ನನ್ನು ಆ ಕಾರ್ಯಕ್ಕೆ ಒಪ್ಪಿಸು, ಆಮೇಲೆ ನಾನು ಇಳಿದು ಬರುತ್ತೇನೆ’ ಎಂದು ಹೇಳಿ ತನ್ನ ಸ್ಥಾನಕ್ಕೆ ಹಿಂದಿರುಗಿದಳು. ಭಗೀರಥನು ಶಿವನನ್ನು ಕುರಿತು ತಪಸ್ಸು ಮಾಡಿದನು. ಕೆಲ ಕಾಲದ ಮೇಲೆ ಆತನು ಪ್ರತ್ಯಕ್ಷನಾಗಿ, ಭಗೀರಥನ ಅಪೇಕ್ಷೆಯಂತೆ ನಡೆಯುವುದಾಗಿ ತಿಳಿಸಿದನು. ಅನಂತರ ದೇವಗಂಗೆ ಆಕಾಶದಿಂದ ಧುಮ್ಮಿಕ್ಕಿದಳು. ಶಿವನು ಆಕೆಯನ್ನು ಜಟೆಯಲ್ಲಿ ಧರಿಸಿದನು. ಆಕೆ ಅಲ್ಲಿಯೇ ಅಡಗಿಹೋದಳು. ಭಗೀರಥನು ಮತ್ತೆ ಪರಶಿವನನ್ನು ಬೇಡಿ ದೇವಗಂಗೆಯನ್ನು ಪಡೆದ. ಆಕೆ ಭಗೀರಥನ ರಥದ ಹಿಂದೆ ಪ್ರವಹಿ ಸುತ್ತಾ ಹೊರಟಳು. ಮುಟ್ಟಿದ ಎಡೆಯನ್ನೆಲ್ಲ ಪವಿತ್ರಗೊಳಿಸುತ್ತಾ ಆಕೆ ಸಗರಪುತ್ರರ ಬೂದಿಯ ರಾಶಿಯ ಮೇಲೆ ಹರಿದುಹೋದಳು. ಅವರಿಗೆಲ್ಲ ಸದ್ಗತಿ ದೊರೆಯಿತು. ಭಗೀರಥನು ತನ್ನ ಮನೋರಥವನ್ನು ಪಡೆದು ಧನ್ಯನಾದನು.


\section{(೧೧) ಕಲ್ಮಾಷಪಾದ}

ಭಗೀರಥನ ಪೀಳಿಗೆಯಲ್ಲಿ ಅನೇಕ ಮಹಾಪುರುಷರು ಹುಟ್ಟಿ ಲೋಕಪ್ರಸಿದ್ಧರಾದರು. ನಳ ಚಕ್ರವರ್ತಿಗೆ ಗೆಳೆಯನಾಗಿ, ಅವನಿಗೆ ಅಕ್ಷವಿದ್ಯೆಯನ್ನು ಕಲಿಸಿ ಅವನಿಂದ ಅಶ್ವ ವಿದ್ಯೆಯನ್ನು ಕಲಿತ ಪುತುಪರ್ಣನು ಈ ಪೀಳಿಗೆಯವನೇ. ಈತನ ಮರಿಮಗನೇ ಕಲ್ಮಾಷ ಪಾದರಾಯ. ಈತನ ನಿಜವಾದ ಹೆಸರು ಮಿತ್ರಸಹ ಎಂದು. ಈತನು ಮಡದಿಯಾದ ಮದಯಂತಿಯೊಡನೆ ಧರ್ಮದಿಂದ ರಾಜ್ಯಭಾರ ಮಾಡುತ್ತಾ ಮಹಾಶೂರನೆಂದು ವಿಖ್ಯಾತನಾಗಿದ್ದನು. ಒಮ್ಮೆ ಈತನು ಬೇಟೆಗೆ ಹೋಗಿದ್ದಾಗ ಸೋದರರಾದ ಇಬ್ಬರು ರಾಕ್ಷಸರು ಈತನನ್ನು ಇದಿರಿಸಿದರು. ಮಿತ್ರಸಹನು ಅವರೊಡನೆ ಯುದ್ಧ ಮಾಡಿ ಒಬ್ಬ ನನ್ನು ಕೊಂದುಹಾಕಿದನು. ಇನ್ನೊಬ್ಬನು ಓಡಿಹೋಗಲು ‘ಅವನೇನು ಮಹಾ’ ಎಂದು ಕೊಂಡು, ಉದಾಸೀನದಿಂದ ಅವನನ್ನು ಕೊಲ್ಲದೆ ಬಿಟ್ಟುಬಿಟ್ಟನು. ಹೀಗೆ ಉಳಿದುಕೊಂಡ ರಾಕ್ಷಸನು ತನ್ನ ಅಣ್ಣನನ್ನು ಕೊಂದವನ ಮೇಲೆ ಹಗೆ ತೀರಿಸಿಕೊಳ್ಳಬೇಕೆಂದು, ಅಡಿಗೆ ಯವನ ವೇಷದಿಂದ ಮಿತ್ರಸಹನ ಅರಮನೆಯಲ್ಲಿ ಸೇರಿಕೊಂಡ. ಹೀಗಿರಲು ಒಂದು ದಿನ ಕುಲಗುರುಗಳಾದ ವಸಿಷ್ಠ ಮಹರ್ಷಿಗಳು ರಾಜನ ಮನೆಗೆ ಊಟಕ್ಕೆ ಬಂದರು. ಇಂತಹ ಸಮಯವನ್ನೆ ಕಾಯುತ್ತಿದ್ದ ಆ ರಕ್ಕಸ ನರಮಾಂಸವನ್ನು ಬೇಯಿಸಿ ಅವರಿಗೆ ಬಡಿಸಿದ. ಅವನು ಬಡಿಸುತ್ತಿರುವಾಗಲೆ ವಸಿಷ್ಠರಿಗೆ ಅದು ನರಮಾಂಸವೆಂದು ಗೊತ್ತಾಗಿ ಹೋಯಿತು. ಅವರು ಹಿಂದು ಮುಂದು ಆಲೋಚಿಸದೆ ಮಿತ್ರಸಹ ರಾಜನನ್ನು ರಾಕ್ಷಸನಾಗೆಂದು ಶಪಿಸಿದರು. ನಿರಪರಾಧಿಯಾದ ತನ್ನನ್ನು ಶಪಿಸಿದುದಕ್ಕಾಗಿ ರಾಜನಿಗೆ ರೇಗಿ ಹೋಯಿತು. ಆತನು ವಸಿಷ್ಠರಿಗೆ ಪ್ರತಿಶಾಪವನ್ನು ಕೊಡಬೇಕೆಂದು ನೀರನ್ನು ತೆಗೆದುಕೊಂಡು ಅಭಿಮಂತ್ರಿಸಿದ. ಅದನ್ನು ಕಂಡು, ರಾಣಿಯಾದ ಮದಯಂತಿ ಆತನ ಕೈಗಳನ್ನು ಭದ್ರವಾಗಿ ಹಿಡಿದುಕೊಂಡು ‘ಸ್ವಾಮಿ, ಅವರು ಕುಲಗುರುಗಳು. ಅವರು ತಪ್ಪುಮಾಡಿದರೂ ಅವರ ಮೇಲೆ ಕೋಪಗಳ್ಳುವುದು ತರವಲ್ಲ. ಈ ಶಾಪದ ನೀರನ್ನು ಕೆಳಕ್ಕೆ ಚೆಲ್ಲಿರಿ’ ಎಂದಳು. ಆದರೆ ಅದನ್ನು ಎಲ್ಲಿ ಚೆಲ್ಲಬೇಕು? ಎಲ್ಲಿ ಚೆಲ್ಲಿದರೂ ಅಲ್ಲಿಯ ಜೀವಿ ಗಳಿಗೆ ಅಪಾಯ! ಆದ್ದರಿಂದ ಮಿತ್ರಸಹನು ಆ ಮಂತ್ರಜಲವನ್ನು ತನ್ನ ಪಾದಗಳ ಮೇಲೆಯೇ ಸುರಿದುಕೊಂಡನು. ಒಡನೆಯೇ ಆ ಪಾದಗಳೆರಡೂ ಇದ್ದಲಿನಂತೆ ಕಪ್ಪಾದವು. ಅಂದಿನಿಂದ ಆತನಿಗೆ ‘ಕಲ್ಮಾಷಪಾದ’ನೆಂದು ಹೆಸರಾಯಿತು. ಇದನ್ನು ಕಂಡು ವಸಿಷ್ಠರ ಕಣ್ಣು ತೆರೆಯಿತು. ತಪ್ಪು ರಾಜನದಲ್ಲ, ಅಡಿಗೆಯವನ ವೇಷದಲ್ಲಿದ್ದ ರಾಕ್ಷಸನದು ಎಂಬುದೂ ಗೊತ್ತಾಯಿತು. ಅವರು ಪಶ್ಚಾತ್ತಾಪದಿಂದ, ಹನ್ನೆರಡು ವರ್ಷಗಳಾಗುತ್ತಲೆ ರಾಜನ ಶಾಪ ವಿಮೋಚನೆಯಾಗಲೆಂದು ಹರಸಿದರು.

ವಸಿಷ್ಠರ ಶಾಪಕ್ಕೆ ಅನುಸಾರವಾಗಿ ಕಲ್ಮಾಷಪಾದರಾಯನು ರಾಕ್ಷಸನಾಗಿ ಅಡವಿಯನ್ನು ಸೇರಿದ. ಅಲ್ಲಿ ಕೈಗೆ ಸಿಕ್ಕ ಮೃಗಗಳನ್ನು ತಿನ್ನುತ್ತ ಕಾಲ ಕಳೆಯುತ್ತಿರಲು, ಒಂದು ದಿನ ವನ ವಿಹಾರಕ್ಕೆಂದು ಬಂದಿದ್ದ ಬ್ರಾಹ್ಮಣ ದಂಪತಿಗಳು ಅವನ ಕಣ್ಣಿಗೆ ಬಿದ್ದರು. ಪುತ್ರಾರ್ಥಿ ಯಾದ ತನ್ನ ಮಡದಿಯೊಡನೆ ರತಿಕ್ರೀಡೆಗೆಂದು ಆ ಬ್ರಾಹ್ಮಣ ಸಿದ್ಧನಾಗುತ್ತಿರುವಾಗ ಕಲ್ಮಾಷಪಾದರಾಕ್ಷಸ ಅವನ ಪಾಲಿನ ಮೃತ್ಯುವಿನಂತೆ ಅವನನ್ನು ಹಿಡಿದ. ಆಗ ಅವನ ಮಡದಿ ಬಹು ದೈನ್ಯದಿಂದ ಆ ರಕ್ಕಸನನ್ನು ಕುರಿತು ‘ಮಹಾನುಭಾವ, ನೀನು ಇಕ್ಷ್ವಾಕು ವಂಶದ ಚಕ್ರವರ್ತಿ, ರಾಕ್ಷಸನಲ್ಲ. ಕೇವಲ ಶಾಪದಿಂದ ರಾಕ್ಷಸನಾಗಿರುವೆ. ಮಹಾ ಪರಾಕ್ರಮಿಯಾದ ನೀನು ಈ ಬಡಬ್ರಾಹ್ಮಣನನ್ನು ಏಕೆ ಕೊಲ್ಲುತ್ತಿ? ಮಹಾ ಪತಿವ್ರತೆ ಯಾದ ಮದಯಂತಿದೇವಿಯ ಪತಿ ನೀನು; ಆಕೆಯ ಹೆಸರಿನಲ್ಲಿ ನಾನು ಪತಿಭಿಕ್ಷೆಯನ್ನು ಬೇಡುತ್ತೇನೆ. ದಯವಿಟ್ಟು ನನ್ನ ಗಂಡನನ್ನು ನನಗೆ ಕೊಡು. ನಾನು ಮಕ್ಕಳಾಗಬೇಕೆಂಬ ಬಯಕೆಯುಳ್ಳವಳಾಗಿರುವೆನು. ನನ್ನನ್ನು ನಿರಾಶಳನ್ನಾಗಿ ಮಾಡಬೇಡ. ರಾಜನಾದ ನೀನು ನಿನ್ನ ಪ್ರಜೆಗಳಾಗಿರುವ ನಮ್ಮನ್ನು ಮಕ್ಕಳಂತೆ ಕಾಪಾಡಬೇಕು. ನಿನಗೆ ತಿಳಿಯದ ಧರ್ಮ ಯಾವುದಿದೆ? ವೇದಗಳನ್ನು ಓದಿ ಪರಮ ಸಾಧುವಾಗಿರುವ ಈ ಬ್ರಾಹ್ಮಣನನ್ನು ಕೊಲ್ಲು ವುದು ಮಹಾಪಾಪ. ಇವನನ್ನು ಕೊಲ್ಲಬೇಡ. ನೀನು ಹಸಿವನ್ನು ತಡೆಯಲಾರೆಯಾದರೆ ಮೊದಲು ನನ್ನನ್ನಾದರೂ ತಿನ್ನು. ಆತನನ್ನು ಅಗಲಿ ನಾನು ಬದುಕಿರಲಾರೆ’ ಎಂದು ಬೇಡಿ ಕೊಂಡಳು. ಆದರೆ ಆಕೆಯ ಪ್ರಾರ್ಥನೆಯೆಲ್ಲ ಗೋರ್ಕಲ್ಲ ಮೇಲೆ ಮಳೆಗರೆದಂತಾಯಿತು. ರಾಕ್ಷಸನಾಗಿದ್ದ ಕಲ್ಮಾಷಪಾದನು, ಕ್ರೂರವಾದ ಹುಲಿ ಹಸುವಿನ ಮೇಲೆ ಹಾರುವಂತೆ, ಆ ಬ್ರಾಹ್ಮಣನ ಮೇಲೆ ಬಿದ್ದು, ಅವನನನ್ನು ಮುರಿದು ತಿಂದು ಹಾಕಿದನು. ಇದನ್ನು ಕಂಡು ಆ ಹೆಣ್ಣು ಕಲ್ಲು ಕರಗುವಂತೆ ಗೋಳಾಡುತ್ತಾ ‘ಪಾಪಿ, ಕಾಮಕೇಳಿಗೆ ನಾನು ಬಯಸುತ್ತಿದ್ದ ನನ್ನ ಗಂಡನನ್ನು ಅನ್ಯಾಯವಾಗಿ ತಿಂದುಹಾಕಿದೆಯಲ್ಲಾ, ನಿನಗೂ ಹಾಗೆಯೆ ಆಗಲಿ. ನೀನು ಹೆಂಡತಿಯೊಡನೆ ರಮಿಸಿದೊಡನೆ ಸತ್ತು ಹೋಗು’ ಎಂದು ಶಪಿಸಿದಳು. ಅನಂತರ ಆ ರಕ್ಕಸನು ತಿಂದು ಉಳಿದಿದ್ದ ಗಂಡನ ಎಲುಬುಗಳನ್ನೆಲ್ಲ ಕೂಡಿಸಿಕೊಂಡು, ಅಗ್ನಿಪ್ರವೇಶ ಮಾಡಿದಳು.

ವಸಿಷ್ಠರು ಕೊಟ್ಟ ಶಾಪದ ಅವಧಿ ಮುಗಿಯಿತು. ಕಲ್ಮಾಷಪಾದನು ಶಾಪದಿಂದ ಬಿಡುಗಡೆಹೊಂದಿ ರಾಜಧಾನಿಗೆ ಹಿಂದಿರುಗಿದನು. ಬಹುಕಾಲ ಅಗಲಿದ್ದ ಮಡದಿಯನ್ನು ಕಂಡು ಆತನಿಗೆ ಕಾಮಾತುರವಾಯಿತು; ಮಡದಿಯನ್ನು ರತಿಸುಖಕ್ಕೆ ಕರೆದನು. ಆದರೆ ಮದ ಯಂತಿ ಆತನ ಪ್ರಯತ್ನವನ್ನು ನಿವಾರಿಸುತ್ತಾ, ಬ್ರಾಹ್ಮಣನ ಹೆಂಡತಿ ಕೊಟ್ಟಿದ್ದ ಶಾಪವನ್ನು ಜ್ಞಾಪಿಸಿದಳು. ಅಂದಿನಿಂದ ಕಲ್ಮಾಷಪಾದನು ಸ್ತ್ರೀಸಂಗವನ್ನು ಸಂಪೂರ್ಣವಾಗಿ ತ್ಯಜಿಸಿ ದನು. ಇದರಿಂದ ಆತನ ಸಂತತಿಯೆ ನಿಂತು ಹೋಗುವಂತಾಗಲು, ವಸಿಷ್ಠರ ಅನುಗ್ರಹ ದಿಂದ ಮದಯಂತಿಯು ಗರ್ಭಿಣಿಯಾದಳು. ಆದರೆ ವರ್ಷಗಳು ಏಳು ಕಳೆದರೂ ಹೆರಿಗೆ ಯಾಗಲಿಲ್ಲ. ಆಕೆಯ ವ್ಯಥೆಯನ್ನು ನೋಡಲಾರದೆ ಕಲ್ಮಾಷಪಾದನು ಒಂದು ಚೂಪಾದ ಕಲ್ಲಿನಿಂದ ಆಕೆಯ ಗರ್ಭವನ್ನು ಸೀಳಿದನು. ಒಡನೆಯೆ ಒಬ್ಬ ಮಗ ಹುಟ್ಟಿದನು. ಕಲ್ಲಿನ (ಅಶ್ಮ) ಗಾಯದಿಂದ ಹುಟ್ಟಿದವನಾದುದರಿಂದ ಅವನಿಗೆ ಅಶ್ಮಕನೆಂದು ಹೆಸರಾಯಿತು. ಅವನು ವಯಸ್ಕನಾದ ಮೇಲೆ ಕಲ್ಮಾಷಪಾದನು ಅವನಿಗೆ ರಾಜ್ಯವನ್ನು ವಹಿಸಿ ತಪೋನಿರತನಾದನು.


\section{(೧೨) ಶ್ರೀರಾಮಚಂದ್ರ}

ಕಲ್ಮಾಷಪಾದರಾಜನ ಮಗ ಅಶ್ಮಕನು ತನ್ನ ಮಗನಿಗೆ ರಾಜ್ಯವನ್ನು ಕೊಟ್ಟು ಅರಣ್ಯಕ್ಕೆ ಹೋಗಲು, ಪರುಶುರಾಮನು ಕ್ಷತ್ರಿಯವಂಶವನ್ನೇ ನಿರ್ಮೂಲಮಾಡುವುದಕ್ಕಾಗಿ ಗಂಡು ಗೊಡಲಿಯನ್ನು ಹಿಡಿದು ಬಂದನು. ಆಗ ಇನ್ನೂ ಬಾಲಕನಾಗಿದ್ದ ಅಶ್ಮಕ ಪುತ್ರನನ್ನು ಅಂತಃಪುರದವರು ತಮ್ಮ ಮಧ್ಯದಲ್ಲಿ ಬಚ್ಚಿಟ್ಟುಕೊಂಡು ಕಾಪಾಡಿದರು. ಇದರಿಂದ ಆ ಬಾಲಕನಿಗೆ ‘ನಾರೀಕವಚ’ ಎಂದು ಹೆಸರಾಯಿತು. ಪರಶುರಾಮನಿಂದ ಕ್ಷತ್ರಿಯ ವಂಶ ವೆಲ್ಲವೂ ನಾಶವಾಗಿಹೋಗಲು ಅದನ್ನು ಅಳಿಯದಂತೆ ಉಳಿಸಲು ಈತನೇ ಕಾರಣನಾದುದ ರಿಂದ ಈತನಿಗೆ ‘ಮೂಲಕ’ ಎಂಬ ಹೆಸರೂ ಉಂಟು. ಇವನ ಮರಿಮಗ ಖಟ್ವಾಂಗ. ಆತನು ಚಕ್ರವರ್ತಿಯಾಗಿ ಜಗತ್ತನೆಲ್ಲ ತನ್ನ ಕೊಡೆಯ ನೆರಳಲ್ಲಿ ಸುರಕ್ಷಿತವಾಗಿಟ್ಟಿದ್ದನು. ಅಷ್ಟೇ ಅಲ್ಲ, ದೇವತೆಗಳಿಗೆ ಕಷ್ಟ ಬಂದಾಗಲೆಲ್ಲ ಈತ ಅವರ ಸಹಾಯಕ್ಕೆ ಹೋಗುತ್ತಿ ದ್ದನು. ಒಮ್ಮೆ ದೇವದಾನವ ಯುದ್ಧದಲ್ಲಿ ಈತನು ರಕ್ಕಸರನ್ನೆಲ್ಲ ಒಕ್ಕಲಿಕ್ಕಿ ಓಡಿಸಿದನು. ಸಂತೋಷಗೊಂಡ ದೇವತೆಗಳು ಬೇಕಾದ ವರವನ್ನು ಕೊಡುವುದಾಗಿ ಆತನಿಗೆ ತಿಳಿಸಿದರು. ಖಟ್ವಾಂಗನು ಅವರನ್ನು ಕುರಿತು ‘ನನಗೆ ಇನ್ನೂ ಎಷ್ಟು ಆಯುಷ್ಯವಿದೆ?’ ಎಂದು ಕೇಳಿದ. ಅವರು ‘ಇನ್ನು ಎರಡು ಗಳಿಗೆಗಳು ಮಾತ್ರ’ ಎಂದರು. ಒಡನೆಯೆ ಖಟ್ವಾಂಗನು ತನ್ನ ರಾಜಧಾನಿಗೆ ಹಿಂದಿರುಗಿ, ತನಗಿದ್ದ ಎರಡುಗಳಿಗೆಗಳ ಅವಧಿಯಲ್ಲಿಯೇ ಆತ್ಮಾನು ಸಂಧಾನದಿಂದ ಪರಮಪುರುಷನಾದ ವಾಸುದೇವನನ್ನು ಮೆಚ್ಚಿಸಿ ಸದ್ಗತಿಯನ್ನು ಹೊಂದಿ ದನು. ಮಹಾನುಭಾವನಾದ ಖಟ್ವಾಂಗನ ಮಗ ದೀರ್ಘಬಾಹು, ಆತನ ಮಗ ರಘು, ರಘುವಿನ ಮಗ ಅಜ, ಅಜನ ಮಗ ದಶರಥ. ಈ ಪುಣ್ಯಶಾಲಿಯಾದ ದಶರಥನ ಮಗನೆ ಪುರಾಣಪುರುಷನಾದ ಶ್ರೀರಾಮಚಂದ್ರ. ವಾಲ್ಮೀಕಿ ಮಹರ್ಷಿ ಅದ್ವಿತೀಯವಾಗಿ ವರ್ಣಿಸಿ ರುವ ಈ ಪುರುಷೋತ್ತಮನ ಕಥೆ ಸಾಮಾನ್ಯವಾಗಿ ಎಲ್ಲರಿಗೂ ಗೊತ್ತಿರತಕ್ಕುದೆ. ಆ ಪುಣ್ಯಶ್ಲೋಕನ ಹೆಸರು ಲೋಕವನ್ನೆಲ್ಲ ಸಂರಕ್ಷಿಸಲಿ!

ವೇದಸ್ವರೂಪನಾದ ಶ್ರೀಹರಿಯು ದೇವತೆಗಳ ಮೊರೆಯನ್ನು ಕೇಳಿ, ಅವರನ್ನು ರಕ್ಷಿಸು ವುದಕ್ಕಾಗಿ ತನ್ನ ಬೇರೆ ಬೇರೆ ಅಂಶಗಳಿಂದ ಶ್ರೀರಾಮ, ಲಕ್ಷ್ಮಣ, ಭರತ, ಶತ್ರುಘ್ನರೆಂಬ ನಾಲ್ವರು ಮಕ್ಕಳಾಗಿ ಹುಟ್ಟಿದನು. ಶ್ರೀರಾಮನು ಎಳೆಯ ಬಾಲಕನಾಗಿರುವಾಗಲೆ ತಮ್ಮ ನಾದ ಲಕ್ಷ್ಮಣನೊಡನೆ ವಿಶ್ವಾಮಿತ್ರರ ಆಶ್ರಮಕ್ಕೆ ಹೋಗಿ ಸುಬಾಹು, ಮಾರೀಚ ಮೊದ ಲಾದ ಅನೇಕ ರಾಕ್ಷಸರನ್ನು ಕೊಂದುಹಾಕಿದನು. ಅಲ್ಲಿಂದ ಆತನು ಜನಕರಾಜನ ರಾಜ ಧಾನಿಯಾದ ಮಿಥಿಲೆಗೆ ಹೋದನು. ಮುನ್ನೂರು ಮಂದಿ ಪರಾಕ್ರಮಿಗಳು ರಥದಲ್ಲಿ ಎಳೆದುಕೊಂಡು ಬಂದ ಶಿವಧನುಸ್ಸನ್ನು, ಹೆದೆಯೇರಿಸುವ ನೆಪದಲ್ಲಿ, ಆನೆಯು ಕಬ್ಬಿನ ಜಲ್ಲೆಯನ್ನು ಮುರಿಯುವಂತೆ ಲೀಲಾಜಾಲವಾಗಿ ಮುರಿದು, ಸುರ ಸುಂದರಿಯಾದ ಸೀತೆ ಯನ್ನು ಮದುವೆಮಾಡಿಕೊಂಡನು. ಆಕೆ ಮತ್ತಾರೂ ಅಲ್ಲ, ಸಾಕ್ಷಾತ್ ಲಕ್ಷ್ಮಿ. ಶ್ರೀರಾಮ ಚಂದ್ರನು ಆಕೆಯನ್ನು ತನ್ನೊಡನೆ ಅಯೋಧ್ಯೆಗೆ ಕರೆತರುತ್ತಿರಲು, ಕ್ಷತ್ರಿಯ ವಂಶಕ್ಕೆ ಮೃತ್ಯುವಿನಂತಿದ್ದ ಪರಶುರಾಮನು ಆತನನ್ನು ಇದಿರಿಸಿದನು. ಶ್ರೀರಾಮನು ಆತನ ಗರ್ವ ವನ್ನು ನಾಶಮಾಡಿ ಊರಿಗೆ ಹಿಂದಿರುಗಿದನು. ದಶರಥನು ತನ್ನ ಕಣ್ಮಣಿಯಂತಿದ್ದ ಈ ಶ್ರೀರಾಮನಿಗೆ ಪಟ್ಟಾಭಿಷೇಕ ಮಾಡಬೇಕೆಂದಿರಲು, ಆತನ ಕಿರಿಯ ಹೆಂಡತಿಯಾದ ಕೈಕೇಯಿ, ಅದಕ್ಕೆ ಅಡ್ಡಿಯನ್ನು ತಂದೊಡ್ಡಿದಳು. ಹಿಂದೆ ದಶರಥ ತನಗೆ ಕೊಟ್ಟಿದ್ದ ಎರಡು ವರಗಳನ್ನು ಮುಂದೆ ಮಾಡಿ, ಅವುಗಳಲ್ಲಿ ಒಂದರಿಂದ ಶ್ರೀರಾಮನನ್ನು ಅಡವಿಗೆ ಅಟ್ಟ ಬೇಕೆಂದೂ, ಮತ್ತೊಂದರಿಂದ ತನ್ನ ಮಗ ಭರತನಿಗೆ ಪಟ್ಟಗಟ್ಟಬೇಕೆಂದೂ ಬೇಡಿದಳು. ಸತ್ಯಸಂಧನಾದ ದಶರಥನು ಮಾತಿಗೆ ಕಟ್ಟುಬಿದ್ದು ಸಂಕಟ ಪಡುತ್ತಿರಲು, ಇದನ್ನು ಕೇಳಿದ ಶ್ರೀರಾಮನು ಸೀತಾಲಕ್ಷ್ಮಣರೊಡನೆ ತಾನಾಗಿಯೇ ಕಾಡಿಗೆ ಹೊರಟುಹೋದನು.

ವನವಾಸಕ್ಕೆ ಹೋದ ಶ್ರೀರಾಮನು ಪರಿಪರಿಯ ಕಷ್ಟಗಳನ್ನು ಅನುಭವಿಸಬೇಕಾಯಿತು. ಆತನು ದಂಡಕಾರಣ್ಯದಲ್ಲಿದ್ದಾಗ ರಾವಣನ ತಂಗಿಯಾದ ಶೂರ್ಪಣಖಿ ಆತನನ್ನು ಮೋಹಿಸಿದಳು. ಅವಳ ನಿರ್ಬಂಧವನ್ನು ಸಹಿಸಲಾರದೆ ಅವಳ ಕಿವಿ ಮೂಗುಗಳನ್ನು ಕೊಯ್ದು ಓಡಿಸಬೇಕಾಯಿತು. ಇದನ್ನು ಕೇಳಿ ಅವಳ ಬಂಧುಗಳಾದ ಖರದೂಷಣ ಮೊದ ಲಾದ ಹದಿನಾಲ್ಕು ಸಹಸ್ರಜನ ರಕ್ಕಸರು ಏಕಕಾಲದಲ್ಲಿ ಶ್ರೀರಾಮನ ಮೇಲೆ ಬಿದ್ದರು. ಅಷ್ಟು ಜನರನ್ನೂ ಆತನೊಬ್ಬನೆ ಕ್ಷಣಮಾತ್ರದಲ್ಲಿ ಕೊಂದುಹಾಕಿದನು. ಇದನ್ನು ಶೂರ್ಪ ಣಖಿಯು ರಾವಣನಿಗೆ ತಿಳಿಸಿ, ಸೀತೆಯ ಸೌಂದರ್ಯವನ್ನು ಬಾಯಲ್ಲಿ ನೀರುಕ್ಕುವಂತೆ ವರ್ಣಿಸಿದಳು. ಕಾಮುಕನಾದ ರಾವಣನು ಆಕೆಯನ್ನು ಕದ್ದು ತರವುದಕ್ಕಾಗಿ ಮಾರೀಚನ ಸಹಾಯವನ್ನು ಕೋರಿದನು. ಆ ಮಾರೀಚ ಮಾಯೆಯ ಜಿಂಕೆಯಾಗಿ ಸುಳಿದು ಸೀತಾರಾಮ ರನ್ನು ಮರುಳುಗೊಳಿಸಿದ. ಸೀತೆಯ ಅಪೇಕ್ಷೆಯಂತೆ ಶ್ರೀರಾಮ ಆ ಮೃಗವನ್ನು ಹಿಡಿದು ತರಲೆಂದು ಹೊರಟ. ಅದು ಅವನನ್ನು ಬಹುದೂರದವರೆಗೆ ಸೆಳೆದುಕೊಂಡು ಹೋಯಿತು. ಕೊನೆಗೆ ಅದು ರಾಕ್ಷಸನ ಮಾಯೆಯೆಂದು ತಿಳಿದ ಶ್ರೀರಾಮ ಅದನ್ನು ಬಾಣ ದಿಂದ ಹೊಡೆದು ಕೊಂದ. ಅದು ಸಾಯುವಾಗ ‘ಲಕ್ಷ್ಮಣಾ, ಸೀತೇ’ ಎಂದು ಕೂಗಿ ಸತ್ತಿತು. ಶ್ರೀರಾಮನ ದನಿಯಲ್ಲಿ ಅದು ಕೂಗಿದುದರಿಂದ ಲಕ್ಷ್ಮಣ ಆ ದನಿ ಬಂದ ದಿಕ್ಕಿಗೆ ಹೊರಟ. ಆ ಸಮಯವನ್ನೇ ಕಾದಿದ್ದ ರಾವಣ ಸನ್ಯಾಸಿವೇಷದಿಂದ ಬಂದು ಸೀತೆಯನ್ನು ಕದ್ದೊಯ್ದ. ರಾಮಲಕ್ಷ್ಮಣರು ಹಿಂದಿರುಗಿ ಸೀತೆಯನ್ನು ಕಾಣದೆ ಕಳವಳಿಸಿದರು. ಸ್ತ್ರೀ ಮೋಹ ಎಷ್ಟು ಭಯಂಕರವೆಂಬುದನ್ನು ನಟಿಸಿ ತೋರಿಸುವವನಂತೆ ಶ್ರೀರಾಮನು ಅತ್ತು ಅತ್ತು ಸಣ್ಣಗಾಗಿಹೋದ. ಆತ ಲಕ್ಷ್ಮಣನೊಡನೆ ಸೀತೆಯನ್ನು ಹುಡುಕುತ್ತಾ ಹೊರಡಲು ಹಾದಿಯಲ್ಲಿ ಸಾಯುತ್ತ ಬಿದ್ದಿದ್ದ ಜಟಾಯು ಕಾಣಿಸಿದ. ಆತ ಸೀತೆಯನ್ನು ಬಿಡಿಸುವು ದಕ್ಕಾಗಿ ರಾವಣನೊಡನೆ ಹೋರಾಡಿ ನೆಲಕ್ಕೆ ಉರುಳಿದ್ದ. ತನಗಾಗಿ ಜೀವವನ್ನು ತೆತ್ತ ಆ ಹಕ್ಕಿಗೆ ಶ್ರೀರಾಮನು ಉತ್ತರಕ್ರಿಯೆಗಳನ್ನು ಮಾಡಿ ಸದ್ಗತಿಯನ್ನು ಕಾಣಿಸಿದ. ಅಲ್ಲಿಂದ ಮುಂದೆ ಕಬಂಧನೆಂಬ ರಾಕ್ಷಸನನ್ನು ಕೊಂದು, ಪುಷ್ಯಮೂಕಪರ್ವತಕ್ಕೆ ಹೋದನು. ಅಲ್ಲಿ ಸುಗ್ರೀವನೆಂಬ ಕಪಿರಾಜನ ಸ್ನೇಹವಾಯಿತು. ಆ ಸುಗ್ರೀವನನ್ನು ಹಿಂಸಿಸುತ್ತಿದ್ದ ಅವನ ಅಣ್ಣನಾದ ವಾಲಿಯನ್ನು ಕೊಂದು ಶ್ರೀರಾಮನು ಸುಗ್ರೀವನಿಗೆ ಪಟ್ಟಾಭಿಷೇಕ ಮಾಡಿದನು. ಆ ಕಪಿರಾಜನು ಆಂಜನೇಯಾದಿ ಕಪಿಗಳಿಂದ ಸೀತೆಯಿರುವ ಸ್ಥಳವನ್ನು ಕಂಡುಹಿಡಿದನು. ಶ್ರೀರಾಮನು ಕಪಿಸೇನೆಯೊಡನೆ ರಾವಣಾಸುರನ ಲಂಕೆಯ ಮೇಲೆ ದಂಡೆತ್ತಿ ಹೊರಟನು. ಹಾದಿಯಲ್ಲಿ ಸಮುದ್ರ ಅಡ್ಡವಾಗಿತ್ತು. ತನಗೆ ಹಾದಿಯನ್ನು ಕೊಡು ವಂತೆ ಶ್ರೀರಾಮ ಸಮುದ್ರರಾಜನನ್ನು ಬೇಡಿದ. ಮೂರು ದಿನದವರೆಗೆ ಬೇಡಿದರೂ ಅವನು ಮೌನವಾಗಿರಲು ಶ್ರೀರಾಮನಿಗೆ ರೇಗಿಹೋಯಿತು. ಆತ ಒಮ್ಮೆ ಕೆಂಗಣ್ಣಿನಿಂದ ಸಮುದ್ರದತ್ತ ನೋಡುತ್ತಲೆ ಅದು ಕುದಿದುಹೋಯಿತು. ಸಮುದ್ರರಾಜನು ಮನುಷ್ಯ ರೂಪದಿಂದ ಓಡಿಬಂದು, ಶ್ರೀರಾಮನ ಪಾದಗಳಿಗೆ ಅಡ್ಡಬಿದ್ದು ‘ಸ್ವಾಮಿ, ನೀನು ಯಾರೆಂದು ತಿಳಿಯದೆ ನಾನು ಮಾಡಿದ ಅಪರಾಧವನ್ನು ಕ್ಷಮಿಸು. ನೀನು ಆದಿಪುರುಷ ನೆಂಬುದು ನನಗೀಗ ಗೊತ್ತಾಗಿದೆ. ನೀನು ನನ್ನ ನೀರಿನಮೇಲೆ ಅಗತ್ಯವಾಗಿಯೂ ಸೇತುವೆಯನ್ನು ಕಟ್ಟಿ, ನಿನ್ನ ಮನಬಂದಂತೆ ಸಂಚರಿಸು. ಆ ಸೇತುವೆ ನಿನ್ನ ಕೀರ್ತಿಗೆ ಸ್ಮಾರಕ ವಾಗಿರಲಿ’ ಎಂದನು. ಅನಂತರ ಶ್ರೀರಾಮನ ಅಪ್ಪಣೆಯಂತೆ ಸೇತುವೆ ನಿರ್ಮಾಣ ವಾಯಿತು. ಕಪಿಸೇನೆ ಲಂಕೆಯನ್ನು ಪ್ರವೇಶಿಸಿ ಕೈಗೆ ಸಿಕ್ಕುದನ್ನು ಧ್ವಂಸಮಾಡುತ್ತಾ ಹೋಯಿತು. ರಾವಣನು ತನ್ನ ಸೈನ್ಯವನ್ನು ಕಪಿಸೇನೆಯ ಮೇಲೆ ಯುದ್ಧಕ್ಕೆ ಕಳುಹಿಸಿದನು. ಘನ ಘೋರವಾದ ಯುದ್ಧ ನಡೆದು ಕುಂಭ, ನಿಕುಂಭ, ದೇವಾಂತಕ, ನರಾಂತಕ, ಇಂದ್ರಜಿತ್ತು, ಪ್ರಹಸ್ತ, ಕುಂಭಕರ್ಣ ಮೊದಲಾದ ರಾಕ್ಷಸವೀರರೆಲ್ಲ ಸತ್ತುಹೋದರು. ಕಡೆಗೆ ರಾವಣನೆ ಶ್ರೀರಾಮನೊಡನೆ ಯುದ್ಧಕ್ಕೆ ನಿಂತನು. ಶ್ರೀರಾಮನು ದೇವೇಂದ್ರನು ಕಳುಹಿಸಿದ್ದ ದಿವ್ಯ ರಥವನ್ನೇರಿ ಅವನೊಡನೆ ಯುದ್ಧ ಮಾಡುತ್ತಾ ‘ಎಲಾ ನೀಚ, ಜನ ರಿಲ್ಲದ ಮನೆಯೊಳಗೆ ನುಗ್ಗಿ ಆಹಾರವನ್ನು ಕದ್ದೊಯ್ಯುವ ನಾಯಿಯಂತೆ, ನಾನಿಲ್ಲದಾಗ ನನ್ನ ಆಶ್ರಮದಿಂದ ನನ್ನ ಮಡದಿಯನ್ನು ಕದ್ದೊಯ್ದೆಯಲ್ಲವೆ? ನೀನು ನಾಚಿಕೆಯಿಲ್ಲದೆ ಹೇಗೆ ತಾನೆ ನನ್ನೆದುರಿಗೆ ನಿಂತಿರುವೆ? ನಾನೆ ನಿನ್ನ ಮೃತ್ಯು. ಇಗೋ ನಿನ್ನ ಪಾಪಕ್ಕೆ ತಕ್ಕ ಫಲ’ ಎಂದು ಹೇಳಿ, ತನ್ನ ಅಮೋಘವಾದ ಬಾಣದಿಂದ ಅವನನ್ನು ಕೊಂದುಹಾಕಿದನು.

ರಾವಣನ ಸಂಹಾರವಾದಮೇಲೆ ಶ್ರೀರಾಮನು ವಿಭೀಷಣನನ್ನು ಲಂಕೆಯ ರಾಜನನ್ನಾಗಿ ಮಾಡಿ, ಪುಷ್ಪಕವಿಮಾನದಲ್ಲಿ ಸೀತಾದೇವಿಯೊಡನೆ ಕಪಿವೀರರನ್ನೂ ರಾಕ್ಷಸವೀರರನ್ನೂ ಕರೆದುಕೊಂಡು ಅಯೋಧ್ಯೆಗೆ ಹಿಂದಿರುಗಿದನು. ದಾರಿಯಲ್ಲಿ ದೇವಾನುದೇವತೆಗಳೆಲ್ಲ ಆತನಿಗೆ ಕಾಣಿಕೆಗಳನ್ನು ಒಪ್ಪಿಸಿ, ಆತನ ಕೀರ್ತಿಯನ್ನು ಹಾಡಿ ಹರಸಿದರು. ಅವುಗಳನ್ನು ಕೈಕೊಳ್ಳುತ್ತಾ ಶ್ರೀರಾಮನು ಅಯೋಧ್ಯೆಯ ಬಳಿಯಿದ್ದ ನಂದೀಗ್ರಾಮಕ್ಕೆ ಬಂದು, ಅಲ್ಲಿ ತನ್ನನ್ನೆ ನಿರೀಕ್ಷಿಸುತ್ತಾ ಪುಷಿಯಂತೆ ಬಾಳನ್ನು ನಡಸುತ್ತಿದ್ದ ಭರತನನ್ನು ಕಂಡು ಅವನನ್ನು ಬಾಚಿ ತಬ್ಬಿಕೊಂಡನು. ಅನಂತರ ಆತನು ಸಕಲ ಪರಿವಾರದೊಡನೆ ಅಯೋಧ್ಯೆಯನ್ನು ಸೇರಿ, ವೈಭವದಿಂದ ಪಟ್ಟಾಭಿಷಿಕ್ತನಾದನು. ರಾಮರಾಜ್ಯವು ಸರ್ವಸುಖಕ್ಕೂ ಗಾದೆಯ ಮಾತಾಯಿತು. ಅದು ತ್ರೇತಾಯುಗವಾದರೂ ಶ್ರೀರಾಮನ ಪ್ರಭಾವದಿಂದ ಕೃತಯುಗ ವೆಂಬಷ್ಟು ಧರ್ಮದಿಂದ ತುಂಬಿಹೋಯಿತು. ಶ್ರೀರಾಮನು ಏಕಪತ್ನೀವ್ರತಸ್ಥನಾಗಿದ್ದು ಕೊಂಡು, ಅನೇಕ ಯಜ್ಞಯಾಗಾದಿಗಳಿಂದ ಬುವಿ ಬಾನುಗಳೆರಡರಲ್ಲಿಯೂ ಸುಖ ಸಂತೋಷಗಳನ್ನು ಅನುಭವಿಸುತ್ತಾ ಹತ್ತು ಸಹಸ್ರ ವರ್ಷಗಳವರೆಗೆ ನೆಮ್ಮದಿಯಿಂದ ರಾಜ್ಯಭಾರಮಾಡುತ್ತಿದ್ದನು. ಇದೊಂದು ದೊಡ್ಡ ಅಚ್ಚರಿ. ಸರ್ವ ಲೋಕೇಶ್ವರನಾದ ಆತನು ಯಾಗ ಮಾಡುವುದರ ಮೂಲಕ ತನ್ನನ್ನು ತಾನೆ ಪೂಜಿಸಿಕೊಳ್ಳುತ್ತಿದ್ದನು. 

ದೇವದೇವನು ಎಲ್ಲರ ಮನಸ್ಸಿನಲ್ಲಿಯೂ ನೆಲೆಸಿರುವವನಾದರೂ ಮಾನವನಾಗಿ ಹುಟ್ಟಿ ದಾಗ ಮಾನವನಂತೆಯೆ ವ್ಯಹರಿಸಬೇಕಷ್ಟೆ! ತನ್ನ ರಾಜ್ಯಭಾರದ ವಿಷಯವಾಗಿ ಜನರ ಅಭಿ ಪ್ರಾಯವೇನೆಂಬುದನ್ನು ತಿಳಿಯುವುದಕ್ಕಾಗಿ ಆತನು ವೇಷವನ್ನು ಮರೆಸಿಕೊಂಡು ಊರಿನಲ್ಲೆಲ್ಲ ಸಂಚರಿಸುತ್ತಿದ್ದನು. ಒಮ್ಮೆ ಹಾಗೆ ಸಂಚರಿಸುತ್ತಿರುವಾಗ ಒಂದಾನೊಂದು ಮನೆಯಲ್ಲಿ ದೊಡ್ಡ ಗದ್ದಲವಾಗುತ್ತಿತ್ತು. ಒಳಗಿನಿಂದ ‘ಎಲೆ ಬಜಾರಿ, ಕಂಡವರ ಮನೆ ಯಲ್ಲಿ ಇದ್ದು ಬಂದ ನಿನ್ನನ್ನು ಮತ್ತೆ ನನ್ನ ಹತ್ತಿರ ಸೇರಿಸಿಕೊಳ್ಳುವುದಕ್ಕೆ ನಾನೇನೂ ರಾಮನಂತೆ ಹೆಣ್ಣಿಗನಲ್ಲ. ನಾಯಿಮುಟ್ಟಿದ ಗಡಿಗೆಯನ್ನು ಮತ್ತೆ ಮನೆಯಲ್ಲಿ ಇಟ್ಟು ಕೊಳ್ಳಲೆ?’ ಎಂದು ಕೂಗುತ್ತಿರುವ ಗಂಡಸಿನ ದನಿ ಕೇಳಿಸಿತು. ಸಿಡಿಲನಂತಹ ಈ ನುಡಿಗಳಿಂದ ಆತನ ಮನಸ್ಸು ಕಲಕಿಹೊಯಿತು. ಆತನು ಗರ್ಭಿಣಿಯಾದ ತನ್ನ ಹೆಂಡತಿ ಯನ್ನು ಯಾರಿಗೂ ತಿಳಿಯದಂತೆ ಅಡವಿಗೆ ಅಟ್ಟಿದನು. ಅಲ್ಲಿ ಭಾಗ್ಯವಶದಿಂದ ಆಕೆ ವಾಲ್ಮೀಕಿ ಮಹರ್ಷಿಗಳ ಕಣ್ಣಿಗೆ ಬಿದ್ದಳು. ಆತ ಆಕೆಯನ್ನು ತನ್ನ ಆಶ್ರಮಕ್ಕೆ ಕರೆದೊಯ್ದ. ಅಲ್ಲಿ ಸೀತೆ ಲವ ಕುಶರೆಂಬ ಅವಳಿ ಮಕ್ಕಳನ್ನು ಹೆತ್ತಳು. ಅದೇ ವೇಳೆಗೆ ಲಕ್ಷ್ಮಣನಿಗೆ ಅಂಗದ, ಚಂದ್ರಕೇತುವೆಂಬ ಮಕ್ಕಳೂ, ಭರತನಿಗೆ ದಕ್ಷ, ಪುಷ್ಕಲ ಎಂಬ ಮಕ್ಕಳೂ, ಶತ್ರುಘ್ನನಿಗೆ ಸುಬಾಹು, ಶ್ರುತಸೇನರೆಂಬ ಮಕ್ಕಳೂ ಹುಟ್ಟಿದರು. ಈ ಸಮಸ್ತ ಬಂಧು ಗಳೊಡನೆ ಶ್ರೀರಾಮನು ರಾಜ್ಯಭಾರ ಮಾಡುತ್ತಿರುವಾಗ ಆತನ ಸೋದರರು ಎಲ್ಲೆಲ್ಲಿಯೂ ದಿಗ್ವಿಜಯಗಳನ್ನು ಪಡೆದು, ಅನೇಕ ಕಪ್ಪ ಕಾಣಿಕೆಗಳನ್ನು ತಂದು ಅಣ್ಣನ ಪಾದಗಳಿಗೆ ಒಪ್ಪಿಸಿದರು. ಇದನ್ನೆಲ್ಲ ಕೇಳಿದ ಸೀತಾದೇವಿಗೆ ಗಂಡನ ಅಗಲಿಕೆ ಅಸಹ್ಯವಾಯಿತು. ಆಕೆ ತನ್ನ ಮಕ್ಕಳನ್ನು ವಾಲ್ಮೀಕಿಗೆ ಒಪ್ಪಿಸಿ, ಶ್ರೀರಾಮನನ್ನು ಸ್ಮರಣೆ ಮಾಡುತ್ತಾ ರಸಾತಲವನ್ನು ಪ್ರವೇಶಿಸಿದಳು. ಈ ಸುದ್ದಿಯು ಶ್ರೀರಾಮನಿಗೆ ಮುಟ್ಟಲು ಆತನು ಮಮ್ಮಲ ಮರುಗಿ ದನು. ಆತನು ತನ್ನ ಉಳಿದ ಆಯುಸ್ಸನ್ನು ಬ್ರಹ್ಮಚರ್ಯದಲ್ಲಿಯೆ ಕಳೆದು ಕೊನೆಗೆ ತನ್ನ ಸ್ವಸ್ಥಾನಕ್ಕೆ ಹಿಂದಿರುಗಿದನು. ಆ ಮಹಾಪುರುಷನ ಚರಿತ್ರೆಯನ್ನು ಕೇಳಿದರೆ ಸಾಕು, ಪಾಪಕರ್ಮವೆಲ್ಲ ಸವೆದು ಹೋಗುತ್ತದೆ, ಶಾಂತಿ ನೆಲಸುತ್ತದೆ, ಸಮಸ್ತ ಸನ್ಮಂಗಳಗಳೂ ಲಭಿಸುತ್ತವೆ. ಹತ್ತು ಸಹಸ್ರ ವರ್ಷ ಆತನು ಮಡದಿಯೊಡನೆಯೂ ಸೋದರರೇ ಆದಿ ಬಂಧುಗಳೊಡನೆಯೂ ರಾಜ್ಯಭಾರ ಮಾಡುತ್ತಾ ನಡೆದುಕೊಂಡ ರೀತಿ ಲೋಕಕ್ಕೆ ಒಂದು ಆದರ್ಶವನ್ನು ನೀಡಿದೆ.


\section{(೧೩) ನಿಮಿ}

ಇಕ್ಷ್ವಾಕುವಿನ ನೂರು ಜನ ಮಕ್ಕಳಲ್ಲಿ ಹಿರಿಯನಾದವನು ವಿಕುಕ್ಷಿ. ಅವನ ಮತ್ತು ಅವನ ವಂಶದವರ ಕಥೆಯನ್ನು ಈವರೆಗೆ ಕೇಳಿದ್ದೀರಿ. ಈಗ ಇಕ್ಷ್ವಾಕುವಿನ ಎರಡನೆಯ ಮಗನಾದ ನಿಮಿಯ ಕಥೆಯನ್ನು ಕೇಳಿ. ಈತ ಒಮ್ಮೆ ಒಂದು ಯಾಗವನ್ನು ಮಾಡಬೇಕೆಂದು ನಿಶ್ಚಯಿಸಿ, ಕುಲಗುರುಗಳಾದ ವಸಿಷ್ಠರಲ್ಲಿ ಅದನ್ನು ತಿಳಿಸಿದ. ಅವರು ‘ಮಹಾರಾಜ, ನಾನು ಈಗಾಗಲೆ ದೇವೇಂದ್ರನ ಯಾಗದಲ್ಲಿ ಪುರೋಹಿತನಾಗುವುದಾಗಿ ಒಪ್ಪಿಕೊಂಡಿ ದ್ದೇನೆ. ಆದ್ದರಿಂದ ಆ ಯಾಗವನ್ನು ಮುಗಿಸಿ ಬಂದಮೇಲೆ ನಿನ್ನ ಯಾಗವನ್ನು ಮಾಡಿಸು ತ್ತೇನೆ’ ಎಂದರು. ನಿಮಿಯು ಯಾವ ಉತ್ತರವನ್ನೂ ಕೊಡದೆ ಮೌನವಾಗಿದ್ದು, ಅವರು ಅತ್ತ ಹೋಗುತ್ತಲೆ ಇತ್ತ ಆತನು–‘ಈ ಬಾಳು ಕ್ಷಣಿಕ, ಇಂದು ಇದ್ದವರು ನಾಳೆ ಇರು ವರೊ ಇಲ್ಲವೊ! ಆದ್ದರಿಂದ ಇಂದು ಮಾಡುವ ಕೆಲಸವನ್ನು ನಾಳೆಗೆ ಮುಂದೂಡುವುದು ಬೇಡ’ ಎಂದು ಯೋಚಿಸಿದನು. ತನಗೆ ಅದು ಸರಿಯೆಂದು ತೋರಿದುದರಿಂದ ಆತನು ಬೇರೆಯ ಮಹರ್ಷಿಗಳನ್ನು ಕರೆಸಿ, ತನ್ನ ಯಾಗವನ್ನು ಪ್ರಾರಂಭಿಸಿದನು. ಅದು ನಡೆಯು ತ್ತಿರುವಾಗ ವಸಿಷ್ಠರು ಇಂದ್ರನ ಯಾಗವನ್ನು ಮುಗಿಸಿ, ನೇರವಾಗಿ ನಿಮಿಯ ಬಳಿಗೆ ಬಂದರು. ತಮ್ಮ ಶಿಷ್ಯನ ಅಪಚಾರವನ್ನು ಕಂಡು ಅವರಿಗೆ ತುಂಬ ಸಿಟ್ಟು ಬಂತು. ಅವರು ‘ಎಲ ದುರಹಂಕಾರಿ, ನಿನ್ನ ದೇಹ ಬಿದ್ದುಹೋಗಲಿ’ ಎಂದು ಶಪಿಸಿದರು. ನಿಮಿಗೂ ರೇಗಿತು; ‘ದುಡ್ಡಿನಾಶೆಯಿಂದ ಧರ್ಮವನ್ನು ಗಮನಿಸದ ನಿನ್ನ ದೇಹವೂ ಬಿದ್ದುಹೋಗಲಿ’ ಎಂದು ಪ್ರತಿಶಾಪವನ್ನು ಕೊಟ್ಟ. ಇಬ್ಬರೂ ಏಕ ಕಾಲದಲ್ಲಿ ಸತ್ತು ಬಿದ್ದರು. 

ಸತ್ತ ವಸಿಷ್ಠರು ಮಿತ್ರಾವರುಣರ ಮಗನಾಗಿ ಊರ್ವಶಿಯಲ್ಲಿ ಹುಟ್ಟಿದರು. ಯಾಗ ಭೂಮಿಯಲ್ಲಿ ಬಿದ್ದಿದ್ದ ನಿಮಿಯ ದೇಹವನ್ನು ಯಾಗಕಾರ್ಯದಲ್ಲಿ ತೊಡಗಿದ್ದ ಮಹರ್ಷಿ ಗಳು ಸುಗಂಧದಿಂದ ಕೂಡಿದ ಎಣ್ಣೆಯಲ್ಲಿ ಇಟ್ಟು ಕಾಪಾಡುತ್ತಾ, ಯಾಗವನ್ನು ಮಾಡಿ ಮುಗಿಸಿದರು. ಆ ಯಾಗದ ಕೊನೆಯಲ್ಲಿ ಅವರು ಯಾಗಕ್ಕಾಗಿ ಬಂದಿದ್ದ ದೇವತೆಗಳನ್ನು ಕುರಿತು ‘ಮಹಾನುಭಾವರಿರಾ, ನಮ್ಮ ಯಾಗ ಕಾರ್ಯದಿಂದ ನೀವು ತೃಪ್ತರಾಗಿದ್ದರೆ, ಬೇಡಿದ ವರವನ್ನು ನೀಡಲು ನಿಮಗೆ ಶಕ್ತಿಯಿದ್ದಲ್ಲಿ ಈ ನಿಮಿಯ ದೇಹವನ್ನು ಬದುಕಿಸಿಕೊಡಿ’ ಎಂದು ಕೇಳಿದರು. ದೇವತೆಗಳು ಅದಕ್ಕೆ ಒಪ್ಪಿದರು. ಆದರೆ ಸೂಕ್ಷ್ಮ ದೇಹದಲ್ಲಿದ್ದ ನಿಮಿಯು ಅವರನ್ನು ಕುರಿತು ‘ಮಹಾತ್ಮರೆ, ಮತ್ತೆ ಈ ದೇಹದ ಸಂಬಂಧ ನನಗೆ ಬೇಡ. ಇದು ಎಂದಿದ್ದರೂ ಬಿದ್ದುಹೋಗುವುದೇ. ಇದನ್ನು ತಿಳಿದೇ ಜ್ಞಾನಿಗಳಾದವರು ಈ ದೇಹದ ಮೇಲಿನ ಆಶೆಯನ್ನು ತೊರೆದು ಪರಮಾತ್ಮನಲ್ಲಿ ಮನಸ್ಸನ್ನು ನೆಲೆಗೊಳಿಸುತ್ತಾರೆ’ ಎಂದನು. ಆಗ ದೇವತೆಗಳು ಮಹರ್ಷಿಗಳನ್ನು ಕುರಿತು ‘ನಾವು ಆತನ ಇಷ್ಟದಂತೆಯೇ ನಡೆದುಕೊಳ್ಳೋಣ. ಈತನಿಗೆ ಇನ್ನು ಸ್ಥೂಲದೇಹ ಬೇಡ. ಸೂಕ್ಷ್ಮದೇಹದಿಂದಲೆ ಪ್ರಾಣಿ ಗಳ ಕಣ್ಣುಗಳಲ್ಲಿ ನೆಲೆಸಿ, ಅವುಗಳನ್ನು ಮುಚ್ಚುವ ಮತ್ತು ತೆರೆಯುವ ಕಾರ್ಯವನ್ನು ನಿರ್ವಹಿ ಸಲಿ’ ಎಂದರು. ಅದರಂತೆ ಅಂದಿನಿಂದ ಆತ ಪ್ರಾಣಿಗಳ ಎವೆಯಲ್ಲಿ ಸೂಕ್ಷ್ಮದೇಹದಿಂದ ನೆಲೆಸಿದ್ದಾನೆ.

ನಿಮಿಯು ಮಕ್ಕಳಿಲ್ಲದೆ ಸತ್ತುದರಿಂದ ರಾಜ್ಯ ಅನಾಯಕವಾಗುವಂತಾಯಿತು. ಆಗ ಪುಷಿಗಳೆಲ್ಲ ಸೇರಿ ಆಲೋಚಿಸಿ, ನಿಮಿಯ ಪೀಳಿಗೆಯನ್ನು ಪಡೆಯುವುದಕ್ಕಾಗಿ ಆತನ ದೇಹವನ್ನು ಮಂತ್ರಪೂರ್ವಕವಾಗಿ ಕಡೆದರು. ಆಗ ಆ ದೇಹದಿಂದ ಒಂದು ಮಗು ಹುಟ್ಟಿಬಂತು. ಮಥನ(ಕಡೆಯುವುದು)ದಿಂದ ಹುಟ್ಟಿದುದರಿಂದ ಆ ಮಗುವಿಗೆ ಮಿಥಿಲನೆಂದು ಹೆಸರಾಯಿತು. ಪುರುಷದೇಹದಿಂದ ಹುಟ್ಟಿದುದರಿಂದ ಆ ಮಗುವಿಗೆ ಜನಕನೆಂದೂ, ಸತ್ತ ದೇಹದಿಂದ ಹುಟ್ಟಿದವನಾದುದರಿಂದ ವಿದೇಹನೆಂದೂ ಆತನಿಗೆ ಹೆಸರುಗಳಾದವು. ಮಿಥಿಲನು ಬೆಳೆದು ದೊಡ್ಡವನಾಗಿ ಹೊಸ ರಾಜಧಾನಿಯೊಂದನ್ನು ನಿರ್ಮಿಸಿದನು. ಅದರ ಹೆಸರು ಮಿಥಿಲಾನಗರಿಯೆಂದಾಯಿತು. ಜನಕನೂ ಆತನ ವಂಶ ದವರೂ ಯಾಜ್ಞವಲ್ಕ್ಯರೇ ಮೊದಲಾದ ಮಹರ್ಷಿಗಳ ಅನುಗ್ರಹಕ್ಕೆ ಪಾತ್ರರಾಗಿ ಬ್ರಹ್ಮಜ್ಞ ರಾಗಿದ್ದವರು. ಸಂಸಾರದಲ್ಲಿ ಇದ್ದುಕೊಂಡೇ ಅವರು ಮೋಕ್ಷವನ್ನು ಸಾಧಿಸಿದರು.


\section{(೧೪) ಪುರೂರವಸ್}

ಸುದ್ಯುಮ್ನನು ಹೆಣ್ಣಾಗಿ, ಬುಧನಿಂದ ಪುರೂರವನೆಂಬ ಮಗನನ್ನು ಪಡೆದನೆಂದು ಹಿಂದೆಯೇ ತಿಳಿಸಿದೆಯಷ್ಟೆ. ಬುಧನು ಚಂದ್ರನ ಮಗ. ಆದ್ದರಿಂದ ಪುರೂರವನು ಚಂದ್ರವಂಶಕ್ಕೆ ಸೇರಿದವ ನಾದಂತಾಯಿತು. ಈ ಕಾರಣದಿಂದ ನಾವೀಗ ಚಂದ್ರವಂಶದ ಇತಿಹಾಸವನ್ನು ಸ್ಥೂಲವಾಗಿ ವಿಚಾರಿಸೋಣ. ಬ್ರಹ್ಮನ ಮಗ ಅತ್ರಿಮುನಿ, ಆ ಮುನಿಯ ಕಣ್ಣುಗಳಿಂದ ಅಮೃತಮಯವಾದ ಶರೀರವುಳ್ಳ ಚಂದ್ರ ಹುಟ್ಟಿದನು. ಈತ ಮಹಾ ಪರಾಕ್ರಮಿ; ಮೂರು ಲೋಕಗಳನ್ನೂ ಜಯಿಸಿ ರಾಜಸೂಯ ಯಾಗವನ್ನು ಮಾಡಿದನು. ತನ್ನ ಕತ್ತಿಗೆ ಇದುರಿಲ್ಲವೆಂಬ ಅಹಂಕಾರದಿಂದ ಅವನು ಬೃಹಸ್ಪತಿಯ ಮಡದಿಯಾದ ತಾರೆಯನ್ನು ಬಲಾತ್ಕಾರದಿಂದ ಹೊತ್ತುಕೊಂಡು ಹೋದನು. ಬೃಹಸ್ಪತಿಯು ಬಹಳ ದೈನ್ಯ ದಿಂದ ತನ್ನ ಮಡದಿಯನ್ನು ಹಿಂದಕ್ಕೆ ಕೊಡುವಂತೆ ಅನೇಕಬಾರಿ ಕೇಳಿದರೂ ಚಂದ್ರನು ಹಾಗೆ ಮಾಡಲಿಲ್ಲ. ಇದರಿಂದ ಗುರು (ಬೃಹಸ್ಪತಿ) ಚಂದ್ರರಿಗೆ ಪರಸ್ಪರ ದ್ವೇಷ ಹುಟ್ಟಿತು. ದೇವತೆಗಳೂ ರುದ್ರನೂ ಗುರುವಿನ ಪಕ್ಷ ವಹಿಸಿದರು; ಶುಕ್ರನು ದಾನವರೊಡನೆ ಚಂದ್ರನ ಪಕ್ಷವನ್ನು ವಹಿಸಿದನು. ದೇವ ದಾನವರಲ್ಲಿ ಭಯಂಕರವಾದ ಯುದ್ಧ ನಡೆಯಿತು. ಎರಡು ಕಡೆಯೂ ಅನೇಕರು ಹತರಾದರು. ಈ ಅನರ್ಥವನ್ನು ಕಂಡು ಬೃಹಸ್ಪತಿಯ ತಂದೆಯಾದ ಅಂಗಿರಸ್ಸು ಬ್ರಹ್ಮನ ಬಳಿಗೆ ಹೋಗಿ ಅದನ್ನು ತಿಳಿಸಿದನು. ಆಗ ಬ್ರಹ್ಮನು ಯುದ್ಧರಂಗಕ್ಕೆ ಬಂದು ಚಂದ್ರನನ್ನು ಛೀಗುಟ್ಟಿ, ತಾರೆಯನ್ನು ಆಕೆಯ ಗಂಡನಿಗೆ ಹಿಂದಕ್ಕೆ ಕೊಡಿಸಿದನು.

 ತಾರೆ ಗಂಡನ ಮನೆಗೆ ಹಿಂದಿರುಗಿದಾಗ ಗರ್ಭಿಣಿಯಾಗಿದ್ದಳು. ಇದನ್ನು ಕಂಡು ಕನಲಿ ಕೆಂಗೆಂಡವಾದ ಬೃಹಸ್ಪತಿ ‘ಎಲೆ ನೀತಿಗೆಟ್ಟವಳೆ, ಪರಪುರುಷನಿಂದ ಪಡೆದ ಈ ಗರ್ಭವನ್ನು ತಕ್ಷಣವೇ ಹೊರಹಾಕು. ಇಲ್ಲದಿದ್ದರೆ ನಿನ್ನನ್ನು ಸುಟ್ಟು ಬೂದಿಯಾಗಿ ಮಾಡುತ್ತೇನೆ,’ ಎಂದು ಗದರಿಸಿದನು. ಕ್ಷಣಕಾಲವಾದಮೇಲೆ ಆತನು ಸ್ವಲ್ಪ ಶಾಂತನಾಗಿ ‘ನೋಡೋಣ, ನಿನ್ನ ಗರ್ಭದಿಂದ ಮಗನು ಹುಟ್ಟುವುದಾದರೆ ನಿನ್ನನ್ನು ಕ್ಷಮಿಸಿದರೂ ಕ್ಷಮಿಸುತ್ತೇನೆ’ ಎಂದನು. ತಾರೆ ನಾಚಿಕೆಯಿಂದ ಸತ್ತವಳಂತಾಗಿ ತನ್ನ ಗರ್ಭದಲ್ಲಿದ್ದ ಮಗುವನ್ನು ಕಳಚಿ ಹಾಕಿದಳು. ಆ ಮಗುವು ಚಂದ್ರನಂತೆ ತೊಳಗಿ ಬೆಳಗುತ್ತಿತ್ತು. ಅದನ್ನು ಕಂಡ ಬೃಹ ಸ್ಪತಿಯು ಅತ್ಯಂತ ಆದರದಿಂದ ಆ ಮಗುವನ್ನು ತನ್ನದಾಗಿ ಸ್ವೀಕರಿಸಿದನು. ತಾರೆಗೆ ಮಗುವಾದುದನ್ನು ಕೇಳಿದ ಚಂದ್ರನು ಅಲ್ಲಿಗೆ ಬಂದು, ಅದು ತನ್ನ ಮಗುವಾದುದರಿಂದ ತನಗೇ ಸಲ್ಲಬೇಕೆಂದು ಹಟ ಹಿಡಿದನು. ಗುರು ಅದಕ್ಕೆ ಒಪ್ಪಲಿಲ್ಲ. ಮತ್ತೆ ಅವರಿಬ್ಬರೂ ಕಾದಾಡುವಂತಾಯಿತು. ಸುತ್ತಮುತ್ತಲಿದ್ದ ಜನ ತಾರೆಯನ್ನೆ ಕೇಳಿದರು, ‘ಅದಾರ ಮಗು?’ ಎಂದು. ಅವಳು ನಾಚಿಕೆಯಿಂದ ಬಾಯಿ ಬಿಡಲಿಲ್ಲ. ಆಗತಾನೆ ಹುಟ್ಟಿದ ಮಗುವಿಗೆ ಆಕೆಯ ನಡತೆ ಸರಿಬೀಳಲಿಲ್ಲ. ಅದು ‘ಏ ನಡತೆಗೆಟ್ಟವಳೆ, ನಡೆದುದನ್ನು ನಡೆದಂತೆ ತಿಳಿಸದೆ ಏಕೆ ಸುಮ್ಮನಿರುವೆ?’ಎಂದು ಗದರಿಸಿತು. ಹೆತ್ತ ಮಗನೆ ಹೀಗೆ ಮಾತಾಡುವುದನ್ನು ಕಂಡು ಬ್ರಹ್ಮನಿಗೆ ‘ಅಯ್ಯೋ’ ಎನಿಸಿತು. ಆತನು ತಾರೆಯನ್ನು ಓರೆಯಾಗಿ ಕರೆದು ‘ಮಗಳೆ, ನಿಜ ವನ್ನು ಹೇಳು’ ಎಂದು ಕೇಳಿದನು. ಆಕೆ ‘ಚಂದ್ರನೇ ಈ ಮಗುವಿನ ತಂದೆ’ ಎಂದು ಹೇಳಿದಳು. ಇದರಿಂದ ಮಗು ಚಂದ್ರನಿಗೆ ಸೇರಿತು. ಮಗುವಿನ ಗಂಭೀರ ಬುದ್ಧಿಯನ್ನು ಕಂಡು ಬ್ರಹ್ಮನೇ ಆ ಮಗುವಿಗೆ ‘ಬುಧ’ ಎಂದು ನಾಮಕರಣ ಮಾಡಿದನು. ಈ ಬುಧನು ಚಂದ್ರನ ಮುದ್ದು ಮಗನಾಗಿ ಬೆಳೆದು, ಹೆಣ್ಣು ಸುದ್ಯುಮ್ನನಿಂದ ಪುರೂರವನನ್ನು ಪಡೆದನು.

ಪುರೂರವನು ಬೆಳೆದು ದೊಡ್ಡವನಾಗುತ್ತಲೆ ತನ್ನ ಪರಾಕ್ರಮದಿಂದ ಭೂಮಂಡಲ ವನ್ನೆಲ್ಲ ಜಯಿಸಿ, ಚಕ್ರವರ್ತಿಯಾದನು. ಆತನ ಕೀರ್ತಿ ಮೂರು ಲೋಕಗಳನ್ನೂ ತುಂಬಿತು. ಒಮ್ಮೆ ದೇವಪುಷಿಯಾದ ನಾರದನು ದೇವೇಂದ್ರನ ಸಭೆಯಲ್ಲಿ ಪುರೂರವನ ಗುಣ, ಶೀಲ, ರೂಪು, ಪರಾಕ್ರಮಗಳನ್ನು ಬಾಯ್ತುಂಬ ಹೊಗಳುತ್ತಿರಲು, ಅದನ್ನು ಕೇಳಿದ ಊರ್ವಶಿಗೆ ಆತನಲ್ಲಿ ಮೋಹ ಹುಟ್ಟಿತು. ಆಕೆ ವಿರಹತಾಪವನ್ನು ತಾಳಲಾರದೆ ನೇರವಾಗಿ ಭೂಮಿಗಿಳಿದು ಬಂದು ಆತನಿಗೆ ಕಾಣಿಸಿಕೊಂಡಳು. ಪುರೂರವನು ಮನ್ಮಥನಂತೆ ಸುಂದರನಾದರೆ ಊರ್ವಶಿ ಕೋಟಿ ರತಿಯರಷ್ಟು ಸುಂದರಳು. ಅವಳು ಪುರೂರವನನ್ನು ಮೋಹಿಸಿದರೆ ಪುರೂರವ ಅವಳನ್ನು ಮೋಹಿಸಿ ಮರುಳನಾದ. ಆತ ಅವಳನ್ನು ಕುರಿತು ‘ಸುಂದರಿ, ನೀನಾರು? ಎಲ್ಲಿಂದ ಬಂದೆ? ನನ್ನಿಂದೇನಾಗಬೇಕು? ಬಾ, ಕುಳಿತುಕೊ. ನಾವಿ ಬ್ಬರೂ ಪರಸ್ಪರ ಪ್ರೀತಿಯಿಂದ ಒಂದಾಗೋಣ. ಕಡೆಯವರೆಗೂ ನಾವು ಜಕ್ಕೆವಕ್ಕಿಗಳಂತೆ ಜೊತೆಯಲ್ಲಿ ಬಾಳೋಣ’ ಎಂದನು. ಬೆಡಗುಗಾತಿಯಾದ ಊರ್ವಶಿ ಆತನನ್ನು ಕುರಿತು ‘ಸುಂದರಾಂಗ, ನಿನ್ನನ್ನು ಕಂಡು ಮೋಹಿಸದಿರುವುದು ಯಾವ ಹೆಣ್ಣಿಗೆ ಸಾಧ್ಯ? ನಿನ್ನ ವಿಶಾಲವಾದ ಎದೆಯಲ್ಲಿ ಬಂದು ನೆಲಸುವ ಭಾಗ್ಯವನ್ನು ಯಾವಳು ತಾನೆ ಬೇಡವೆಂ ದಾಳು? ನಿನ್ನನ್ನು ನೋಡುವ ನನ್ನ ಕಣ್ಣುಗಳು ಧನ್ಯವಾದವು. ನಾನಿನ್ನು ನಿನ್ನವಳೆ. ಆದರೆ ನನ್ನ ಕೆಲವು ಸಣ್ಣ ಬೇಡಿಕೆಗಳನ್ನು ನೀನು ಈಡೇರಿಸಬೇಕು. ಇದೋ, ಈ ನನ್ನ ಎರಡು ಮುದ್ದಿನ ಕುರಿಮರಿಗಳನ್ನು ನಾನು ಮಕ್ಕಳಂತೆ ಬೆಳೆಸಿದ್ದೇನೆ. ಇವುಗಳ ರಕ್ಷಣೆ ಇನ್ನು ಮುಂದೆ ನಿನ್ನದು. ನಾನು ತುಪ್ಪವನ್ನು ಹೊರತು ಮತ್ತಾವ ಆಹಾರವನ್ನೂ ತೆಗೆದುಕೊಳ್ಳು ವುದಿಲ್ಲ. ಇನ್ನೂ ಒಂದು ವಿಚಾರ. ರತಿಕ್ರೀಡೆಯ ಕಾಲದಲ್ಲಿ ಹೊರತು ಬೇರೆ ಯಾವಾ ಗಲೂ ನೀನು ಬರಿಯ ಮೈಯಿಂದ ನನ್ನ ಕಣ್ಣಿಗೆ ಬೀಳಕೂಡದು. ಈ ನಿಯಮಗಳನ್ನು ನೀನು ಒಪ್ಪುವುದಾದರೆ ನಾನು ಧನ್ಯಳು’ ಎಂದು ಹೇಳಿದಳು. ಪುರೂರವನು ಒಡನೆಯೆ ಅವಕ್ಕೆ ಒಪ್ಪಿದನು.

ಊರ್ವಶಿ ಪುರೂರವರು ಎಣೆಯಿಲ್ಲದ ಸುಖವನ್ನು ಸೂರೆಗೊಳ್ಳುತ್ತಾ ಇಂದ್ರನ ನಂದನವನವೇ ಮೊದಲಾದ ದೇವೋದ್ಯಾನಗಳಲ್ಲಿ ಮನಬಂದಂತೆ ವಿಹರಿಸಿದರು. ವರ್ಷಗಳು ಕ್ಷಣಗಳಂತೆ ಕಳೆದು ಹೋದವು. ಅವರು ಹೀಗೆ ಮೈಮರೆತು ವಿಹರಿಸುತ್ತಿರಲು, ಅತ್ತ ದೇವೇಂದ್ರನ ಸಭೆ ಊರ್ವಶಿಯಿಲ್ಲದೆ ಕಳೆಗೆಟ್ಟಿತು. ಆಕೆಯನ್ನು ಹುಡುಕಿ ಕರೆತರು ವಂತೆ ಆತನು ಇಬ್ಬರು ಗಂಧರ್ವರನ್ನು ಭೂಲೋಕಕ್ಕೆ ಅಟ್ಟಿದನು. ಅವರಿಬ್ಬರೂ ನಟ್ಟಿರು ಳಲ್ಲಿ ಪುರೂರವನ ಅರಮನೆಯನ್ನು ಪ್ರವೇಶಿಸಿ, ಊರ್ವಶಿಯ ಕುರಿಮರಿಗಳೆರಡನ್ನೂ ಕದ್ದೊಯ್ದರು. ಅವು ಭಯದಿಂದ ‘ಮ್ಯಾ’ ಎಂದು ಅರಚಿಕೊಂಡವು. ಒಡನೆಯೆ ಊರ್ವಶಿ ‘ಅಯ್ಯೋ ಇನ್ನೇನು ಗತಿ? ನನ್ನ ಮರಿಗಳನ್ನು ಯಾರೋ ಕದ್ದೊಯ್ಯುತ್ತಿದ್ದಾರೆ! ಅಯ್ಯೋ, ಹೇಡಿಯಾದ ಈ ಪುರೂರವನನ್ನು ನಂಬಿ ನಾನು ಕೆಟ್ಟೆ. ಅಯ್ಯೋ, ಹೆಂಗಸಿನಂತೆ ಹೇಡಿ ಯಾಗಿ ಮಲಗಿರುವ ಪುರೂರವ, ನಿನ್ನ ಷಂಡತನಕ್ಕೆ ಬೆಂಕಿ ಹಾಕಲಿ, ಮೇಲಕ್ಕೇಳು’ ಎಂದು ಕಿರಿಚಿಕೊಂಡಳು. ತನ್ನ ಪೌರುಷವನ್ನು ಪ್ರಶ್ನಿಸುವ ಈ ಚುಚ್ಚು ನುಡಿಗಳಿಂದ ಪುರೂರವನ ಕೋಪ ಕೆರಳಿತು. ಆತನು ಹಿಂದು ಮುಂದು ನೋಡದೆ ಮೇಲೆಕ್ಕೆದ್ದು, ಬರಿಮೈಲಿರುವು ದನ್ನೂ ಗಮನಿಸದೆ, ಹಿರಿದ ಕತ್ತಿಯೊಡನೆ ಆ ಗಂಧರ್ವರನ್ನು ಬೆನ್ನಟ್ಟಿದನು. ಆತನನ್ನು ಕಾಣುತ್ತಲೆ ಅವರು ಕುರಿಮರಿಗಳನ್ನು ಅಲ್ಲಿಯೆ ಬಿಟ್ಟು, ಮಿಂಚಿನಂತೆ ಮಾಯವಾಗಿ ಹೋದರು. ಪುರೂರವನು ಆ ಕುರಿಮರಿಗಳನ್ನು ಹಿಂದಕ್ಕೆ ಕರೆತಂದನು. ಆದರೆ ಆ ಕ್ಷಣದಲ್ಲಿಯೆ ಆತ ಆ ಸುರಸುಂದರಿಯನ್ನು ಕಳೆದುಕೊಂಡನು. ಬರಿಯ ಮೈಲಿ ಕಾಣಿಸಿ ಕೊಂಡುದರಿಂದ ಆಕೆ ಮಾಯವಾಗಿಹೋದಳು.

ಮೇಕೆಗಳೊಡನೆ ಮನೆಯೊಳಗೆ ಬಂದ ಪುರೂರವನಿಗೆ ಮಂಚದ ಮೇಲಿನ ಹಾಸಿಗೆ ಬರಿ ದಾಗಿರುವುದನ್ನು ಕಂಡು ಜೀವ ಝಲ್ಲೆಂದಿತು. ಆತ ಊರ್ವಶಿಯನ್ನು ಮನೆಯಲ್ಲೆಲ್ಲ ಹುಡುಕಿದ. ಕಾಣಿಸಲಿಲ್ಲ. ಅವಳ ಅಗಲಿಕೆಯನ್ನು ಸಹಿಸಲಾರದೆ ಆತ ‘ಊರ್ವಶೀ, ಊರ್ವಶೀ’ ಎಂದು ಕೂಗುತ್ತಾ ದೇಶದೇಶಗಳನ್ನೆಲ್ಲ ಅಲೆದ. ಕಡೆಗೆ ಆತನು ಕುರುಕ್ಷೇತ್ರದ ಸಮೀಪದಲ್ಲಿ ಅಲೆಯುತ್ತಿರುವಾಗ, ಐವರು ಸಖಿಯರೊಡನೆ ಅವಳು ವಿಹರಿಸುತ್ತಿರುವುದು ಕಣ್ಣಿಗೆ ಬಿತ್ತು. ಆತ ತಡೆಯಲಾರದ ಸಂತೋಷದಿಂದ ಅವಳ ಬಳಿಗೆ ಓಡಿಹೋಗಿ, ‘ಮನೋಹರಿ, ನನ್ನನ್ನು ತೊರೆದು ಎಲ್ಲಿಗೆ ಹೋದೆ? ನಿನ್ನೂಡನೆ ಗುಟ್ಟಾಗಿ ನಡಸಬೇಕಾದ ಸರಸ ಸಲ್ಲಾಪ ಇನ್ನೂ ಮುಗಿದೇ ಇರಲಿಲ್ಲ. ಅಷ್ಟರಲ್ಲಿ ನನ್ನನ್ನು ಅಗಲಿ ಬಂದಿರುವೆಯಲ್ಲಾ! ನಿನ್ನಿಂದ ಅಗಲಿ ನಾನು ಬದುಕುತ್ತೇನೆಯೆ? ನೀನು ತಕ್ಷಣ ನನ್ನನ್ನು ಅಪ್ಪಿಕೊಳ್ಳ ದಿದ್ದರೆ ನಾನು ಸತ್ತೇ ಹೋಗುತ್ತೇನೆ. ಆಮೇಲೆ ಈ ದೇಹ ನಾಯಿನರಿಗಳ ಪಾಲಾಗಿಹೋಗು ತ್ತದೆ’ ಎಂದು ಅಂಗಲಾಚಿದನು. ಆಗ ಊರ್ವಶಿ ‘ಮಹಾರಾಜ, ನೀನು ಧೀರ, ಧೈರ್ಯವನ್ನು ಕಳೆದುಕೊಳ್ಳಬೇಡ. ಸ್ವಲ್ಪ ಯೋಚಿಸು, ನೀನಾಗಲೆ ನಾಯಿನರಿಗಳ ಪಾಲಾಗಿಹೋಗಿ ದ್ದೀಯೆ. ಹೆಣ್ಣೆಂದರೆ ನಾಯಿ ನರಿಗಳಿಗಿಂತ ಮೇಲೆಂದು ತಿಳಿಯುವೆಯೇನು? ಹೆಣ್ಣೆಂದರೆ ಮೋಸ! ಅವರು ತಮ್ಮ ಸಾಧನೆಗಾಗಿ ಗಂಡನಿಗೂ ಸೋದರರಿಗೂ ಅನ್ಯಾಯಮಾಡಲು ಹೇಸುವುದಿಲ್ಲ. ಅವರು ಮಹಾ ಕ್ರೂರಿಗಳು. ಆ ವಿಚಾರ ಹಾಗಿರಲಿ. ನಾನೀಗ ಗರ್ಭಿಣಿ. ಇನ್ನು ಒಂದು ವರ್ಷವಾದಮೇಲೆ ನಾನು ನಿನ್ನ ಬಳಿಗೆ ಬರುತ್ತೇನೆ. ಅಲ್ಲಿಯವರೆಗೆ ಶಾಂತಿ ಯಿಂದಿರು’ ಎಂದು ಹೇಳಿದಳು. ಆಕೆಯ ಮಾತಿನಂತೆ ಆತ ಶಾಂತನಾಗಿ ರಾಜಧಾನಿಗೆ ಹಿಂದಿರುಗಿದ.

ಮಾತುಕೊಟ್ಟಿದ್ದಂತೆ ಒಂದು ವರ್ಷವಾದ ಮೇಲೆ ಊರ್ವಶಿ ತನ್ನ ಮಗನೊಡನೆ ಪುರೂರವನ ಬಳಿಗೆ ಬಂದಳು. ಆದರೆ ಆಕೆ ಒಂದು ರಾತ್ರಿಯ ಮಟ್ಟಿಗೆ ನಿಲ್ಲುವವ ಳಾಗಿದ್ದಳು. ರಾಜನು ಆ ರಾತ್ರಿಯನ್ನು ಅತ್ಯಾನಂದದಿಂದ ಕಳೆದನಾದರೂ, ಬೆಳಕು ಹರಿ ಯುತ್ತಾ ಹೋದಂತೆ ಆತನ ಆತಂಕ ಹೆಚ್ಚುತ್ತಾ ಹೋಯಿತು. ಅತಿ ದೀನನಾಗಿದ್ದ ಆತನನ್ನು ಕಂಡು ಊರ್ವಶಿ ‘ಮಹಾರಾಜ, ನೀನು ಗಂಧರ್ವರನ್ನು ಸ್ತುತಿಸಿ ಮೆಚ್ಚಿಸಿದರೆ ಅವರು ನನ್ನನ್ನು ನಿನಗೆ ಒಪ್ಪಿಸುವರು’ ಎಂದು ಹೇಳಿ, ತನ್ನ ಲೋಕಕ್ಕೆ ಹಿಂದಿರುಗಿದಳು. ಅದರಂತೆಯೆ ಪುರೂರವನು ಗಂಧರ್ವರನ್ನು ಭಕ್ತಿಯಿಂದ ಸ್ತೋತ್ರ ಮಾಡಿದನು. ಅವರು ಅದಕ್ಕೆ ಮೆಚ್ಚಿ ಒಂದು ಅಗ್ನಿಪಾತ್ರೆಯನ್ನು ನೀಡಿದರು. ಅದನ್ನೆ ಊರ್ವಶಿಯೆಂದು ಭ್ರಮಿಸಿ, ಪುರೂರವನು ಅದರೊಡನೆ ಅಡವಿಯಲ್ಲಿ ಸುತ್ತಾಡುತ್ತಿದ್ದನು. ಹೀಗೆ ಸುತ್ತಾಡು ತ್ತಿರುವಾಗ ಒಮ್ಮೆ ಅದನ್ನು ಅಲ್ಲಿಯೇ ಇಟ್ಟು ಊರಿಗೆ ಹಿಂದಿರುಗಿದನು. ಆತನಿಗೆ ಊರ್ವಶಿಯ ಗೀಳು ಹಿಡಿಯಿತು. ಹಗಲು ರಾತ್ರಿ ಅವಳನ್ನೆ ಚಿಂತಿಸುತ್ತಾ ಅನೇಕ ವರ್ಷಗಳು ಕಳೆದುಹೋದವು. ಕಡೆಗೆ ಒಂದು ದಿನ ಕೃತಯುಗ ಕಳೆದು ತ್ರೇತಿ ಪ್ರಾರಂಭವಾಯಿತು. ತಕ್ಷಣವೇ ಆತನಿಗೆ ಮೂರು ವೇದಗಳು ಗೋಚರವಾದುವು. ಪುರೂರವನು ಅಡವಿಗೆ ಓಡಿಹೋಗಿ, ಅಗ್ನಿಪಾತ್ರೆಯನ್ನು ಇಟ್ಟಿದ್ದ ಸ್ಥಳವನ್ನು ನೋಡಿದನು. ಅಲ್ಲಿ ಒಂದು ದೊಡ್ಡ ಬನ್ನಿಯ ಮರದೊಳಗೆ ಒಂದು ಅರಳಿಯ ಮರ ಚಿಗುರಿರುವುದು ಕಾಣಿಸಿತು. ಬನ್ನಿಯಲ್ಲಿ ಹುಟ್ಟಿದ ಅರಳಿಯ ಕೊನೆಗಳನ್ನು ಪರಸ್ಪರ ಉಜ್ಜಿದರೆ ಬೆಂಕಿ ಹುಟ್ಟುವುದೆಂಬ ವೇದವಾಕ್ಯ ಆತನಿಗೆ ಜ್ಞಾಪಕಕ್ಕೆ ಬಂತು. ಎರಡು ಅರಳಿಯ ಕೊನೆಗಳನ್ನು ತೆಗೆದುಕೊಂಡು, ‘ಕೆಳಗಿನದೇ ಊರ್ವಶಿ, ಮೇಲಿನದೇ ನಾನು, ಇವುಗಳನ್ನು ಉಜ್ಜಿದರೆ ಹುಟ್ಟುವ ಬೆಂಕಿಯೇ ನಮ್ಮ ಮಗು’ ಎಂದು ಮಂತ್ರಪೂರ್ವಕವಾಗಿ ಧ್ಯಾನಮಾಡುತ್ತಾ ಉಜ್ಜಿದನು. ಅದರಿಂದ ಬೆಂಕಿ ಹುಟ್ಟಿತು. ಅನಂತರ, ತನಗೆ ಊರ್ವಶಿಯ ಲೋಕ ಪ್ರಾಪ್ತವಾಗಲೆಂದು ಪ್ರಾರ್ಥಿಸುತ್ತಾ, ಅಗ್ನಿಯಮೂಲಕ ಶ್ರೀಹರಿಯನ್ನು ಪೂಜಿಸುತ್ತಿದ್ದನು. ಕೊನೆಯಲ್ಲಿ ಆತನು ಗಂಧರ್ವ ಲೋಕವನ್ನು ಸೇರಿ, ತನ್ನ ಇಷ್ಟಾರ್ಥವನ್ನು ಪಡೆದನು.

