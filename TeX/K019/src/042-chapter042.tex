
\chapter{೪೨. ಮಹಾಮಾಯೆ}

ವಸುದೇವನು ಸೆರೆಮನೆಗೆ ಹಿಂದಿರುಗುತ್ತಲೆ ಅದರ ಬಾಗಿಲುಗಳು ಮೊದಲಿನಂತೆ ಭದ್ರ ವಾಗಿ ಮುಚ್ಚಿಕೊಂಡುವು. ಇಷ್ಟಾಗುವುದನ್ನೇ ಕಾದಿತ್ತೋ ಎನ್ನುವಂತೆ ದೇವಕಿಯ ಪಕ್ಕದ ಲ್ಲಿದ್ದ ಕೂಸು ಗಟ್ಟಿಯಾಗಿ ಅಳುವುದಕ್ಕೆ ಮೊದಲುಮಾಡಿತು. ಅದರ ಕೂಗು ನಿದ್ರೆ ಮಾಡು ತ್ತಿದ್ದ ಕಾವಲುಗಾರರನ್ನು ಬಡಿದೆಬ್ಬಿಸಿತು. ಅವರಲ್ಲಿ ಕೆಲವರು ನಿಟ್ಟೋಟದಿಂದ ಕಂಸ ರಾಜನ ಬಳಿಗೆ ಹೋಗಿ ‘ಮಹಾಸ್ವಾಮಿ, ದೇವಕೀದೇವಿಗೆ ಮಗು ಹುಟ್ಟಿದೆ’ ಎಂದು ಅರಿಕೆ ಮಾಡಿದರು. ಹಾಸಿಗೆಯಲ್ಲಿ ಮಲಗಿದ್ದ ಕಂಸನು ಆ ಮಾತನ್ನು ಕೇಳುತ್ತಲೇ ಮೆಟ್ಟಿಬಿದ್ದು ಮೇಲಕ್ಕೆದ್ದನು. ಆತನು ಬಿಚ್ಚಿದ ಕೂದಲನ್ನು ಕೂಡ ಗಂಟು ಹಾಕಿಕೊಳ್ಳದೆ ಹೇಗಿದ್ದವನು ಹಾಗೆಯೇ ಹಿರಿದ ಕತ್ತಿಯೊಡನೆ ಸೆರಮನೆಗೆ ಓಡಿಬಂದನು. ಅವನು ಒಳಗೆ ನುಗ್ಗಿ ಬಂದ ರಭಸಕ್ಕೆ ದೇವಕೀದೇವಿ ಗಡಗಡ ನಡುಗಿಹೋದಳು. ಆಕೆ ಆತನನ್ನು ಕುರಿತು ‘ಅಣ್ಣ, ಇದು ಹೆಣ್ಣು, ನಿನ್ನ ಸೋದರಸೊಸೆ, ಇದರಿಂದ ನಿನಗಾವ ಭಯವಣ್ಣಾ? ಇದನ್ನು ಕೊಲ್ಲಬೇಡ, ಅಯ್ಯೋ ದೈವವೇ, ನನ್ನ ಮಕ್ಕಳಿಗೆ, ಎಲ್ಲರನ್ನೂ ಬಿಟ್ಟು ನೀನೆ ಮೃತ್ಯುವಾಗಬೇಕೆ? ಒಂದಲ್ಲ, ಎರಡಲ್ಲ; ಆರು ಮಕ್ಕಳು ನಿನ್ನ ಕೋಪಕ್ಕೆ ಆಹುತಿಯಾಗಿ ಹೋದರು. ಹೋದುದು ಹೋಗಲಿ, ಈ ಮಗುವನ್ನಾದರೂ ಉಳಿಸಣ್ಣ. ನಿನ್ನ ಕಾಲು ಹಿಡಿದು ಬೇಡಿ ಕೊಳ್ಳುತ್ತೇನಣ್ಣ’ ಎಂದು ದೀನಳಾಗಿ ಬೇಡಿಕೊಂಡಳು. ಆದರೆ ಕಂಸನ ಮನಸ್ಸು ಜೀವ ಭಯದಿಂದ ಕಲ್ಲಾಗಿ ಹೋಗಿತ್ತು. ಅವನು ತಂಗಿಯ ಮಾತನ್ನು ಕಿವಿಗೆ ಹಾಕಿಕೊಳ್ಳಲೇ ಇಲ್ಲ. ಅವಳು ಎದೆಗೆ ಅವುಚಿಕೊಂಡಿದ್ದ ಕೂಸನ್ನು ಅವನು, ಹೊರೆಯಿಂದ ಕಟ್ಟಿಗೆಯನ್ನು ಕೀಳುವಂತೆ ಕಿತ್ತು, ಅಲ್ಲಿಯೇ ಇದ್ದ ಒಂದು ಬಂಡೆಯ ಮೇಲೆ ಅಗಸನು ಬಟ್ಟೆಯೊಗೆಯು ವಂತೆ ಅದರ ಕಾಲುಗಳನ್ನು ಹಿಡಿದುಕೊಂಡು ಅಪ್ಪಳಿಸಿದನು. ಆದರೆ, ಅದು ಅಷ್ಟರ ಲ್ಲಿಯೇ ಅವನ ಕೈಯಿಂದ ನುಣುಚಿಕೊಂಡು, ಪುಟಚಂಡಿನಂತೆ ಆಕಾಶಕ್ಕೆ ನೆಗೆಯಿತು. ಕಂಸನು ಆಶ್ಚರ್ಯದಿಂದ ತಲೆಯೆತ್ತಿ ನೋಡುತ್ತಾನೆ, ಆ ಕೂಸು ಭಯಂಕರವಾದ ಆಕಾರ ತಾಳಿದೆ. ಆ ಆಕೃತಿಗೆ ತಕ್ಕಂತೆ ಅದಕ್ಕೆ ಎಂಟು ತೋಳುಗಳು. ಒಂದೊಂದು ಕೈಲಿ ಒಂದೊಂದು ಬಗೆಯ ಆಯುಧ–ಬಿಲ್ಲು, ಬಾಣ, ಚಕ್ರ, ಗದೆ ಇತ್ಯಾದಿಗಳು. ಆಕೆಯ ಮೈಮೇಲೆ ದಿವ್ಯವಾದ ವಸ್ತ್ರ ಒಡವೆಗಳು; ಸಿದ್ಧ, ಚಾರಣ, ವಿದ್ಯಾಧರ ಮೊದಲಾದ ದೇವತೆ ಗಳು ಆಕೆಯ ಸುತ್ತಲೂ ಕೈಕಟ್ಟಿ ನಿಂತಿದ್ದಾರೆ. ಆ ಮಹಾದೇವಿ ಕಂಸನನ್ನು ಕುರಿತು ‘ಎಲ ಮೂಢ! ನನ್ನನ್ನು ನೀನು ಕೊಲ್ಲಬಲ್ಲೆಯಾ? ನನ್ನನ್ನು ಕೊಂದರೆ ತಾನೆ ನಿನಗೆ ಬರುವ ಭಾಗ್ಯವೇನು? ನಾನೇನೂ ನಿನ್ನನ್ನು ಕೊಲ್ಲುವವಳಲ್ಲ. ನಿನ್ನ ನಿಜವಾದ ಮೃತ್ಯು ಬೇರೊಂದು ಕಡೆ ಬೆಳೆಯುತ್ತಿದ್ದಾನೆ. ಬಡಪಾಯಿಗಳನ್ನೇಕೆ ಅನ್ಯಾಯವಾಗಿ ಹಿಂಸಿಸುತ್ತಿರುವೆ?’ ಎಂದು ಹೇಳಿ ಮಾಯವಾದಳು.

ಮಹಾಮಾಯೆಯ ಮಾತುಗಳು ಕಂಸನ ಕಣ್ಣು ತೆರೆಸಿದವು. ತಾನು ಅದುವರೆಗೆ ನಡೆಸಿದ ಶಿಶುಹತ್ಯೆಯನ್ನು ನೆನೆದು ಆತನ ಮನಸ್ಸು ಪಶ್ಚಾತ್ತಾಪದಿಂದ ಮಮ್ಮಲ ಮರುಗಿತು. ಒಡನೆಯೇ ವಸುದೇವ, ದೇವಕಿದೇವಿಯವರ ಬಳಿಗೆ ಓಡಿಹೋಗಿ, ಅವರ ಸಂಕಲೆಗಳನ್ನು ಕಿತ್ತೆಸೆದನು. ಅನಂತರ ಆತನು ಅವರನ್ನು ಕುರಿತು ‘ಅಯ್ಯೋ ಭಾವ, ನಿಮಗೆ ನಾನೆಷ್ಟು ಹಿಂಸೆ ಮಾಡಿದೆ! ಮಗು ದೇವಕಿ, ನಿನ್ನ ಮಕ್ಕಳನ್ನೆಲ್ಲಾ ಅನ್ಯಾಯವಾಗಿ ಕೊಂದ ಈ ರಕ್ಕಸ ನನ್ನು ಕ್ಷಮಿಸು ತಾಯಿ. ಹಾಳು ಆ ಆಕಾಶವಾಣಿಯ ಮಾತು ನಿಜವೆಂದು ಭ್ರಮಿಸಿ ಮಾಡ ಬಾರದ ಕೆಲಸಗಳನ್ನು ಮಾಡಿಬಿಟ್ಟೆ. ಲೋಕದಲ್ಲಿ ಮನುಷ್ಯರು ಸುಳ್ಳು ಹೇಳುವುದಂತೂ ಸರಿ, ದೈವವೂ ಸುಳ್ಳು ಹೇಳುವುದೆಂದು ಯಾರಿಗೆ ಗೊತ್ತು? ಆ ಹಾಳು ದೈವದ ಮಾತನ್ನು ನಂಬಿ, ಪ್ರಾಣಭಯದಿಂದ, ನಿಮ್ಮ ಮಕ್ಕಳನ್ನೆಲ್ಲಾ ಕೊಂದುಹಾಕಿದೆ. ಅದು ನನ್ನ ದುರ ದೃಷ್ಟ, ನಿಮ್ಮ ಕರ್ಮ. ನಡೆದುಹೋದುದನ್ನು ಮರೆತು ಬಿಡೋಣ. ಇಗೋ ನಿಮ್ಮ ಕಾಲು ಹಿಡಿದು ಬೇಡಿಕೊಳ್ಳುತ್ತೇನೆ; ನನ್ನ ತಪ್ಪನ್ನು ಹೊಟ್ಟೆಯಲ್ಲಿ ಹಾಕಿಕೊಂಡು ಕ್ಷಮಿಸಿಬಿಡಿ’ ಎಂದು ಬೇಡಿಕೊಂಡನು.

ಅಣ್ಣನ ಮಾತುಗಳನ್ನು ಕೇಳಿ ದೇವಕಿಯ ಹೆಂಗರುಳು ಕರಗಿಹೋಯಿತು. ಆಕೆ ತನ್ನ ಕಣ್ಣೀರಿನಿಂದ ಅಣ್ಣನ ಅನ್ಯಾಯವನ್ನೆಲ್ಲ ತೊಳೆದು ಹಾಕಿದಳು. ಆದರೆ ವಸುದೇವನ ಗಂಡು ಮನಸ್ಸಿಗೆ ಕಂಸನ ಮಾತುಗಳು ಕೇವಲ ಮೊಸಳೆಯ ಕಣ್ಣೀರೆನಿಸಿತು. ಆತನು ಪಕಪಕ ನಗುತ್ತಾ ‘ಅಯ್ಯಾ, ರಾಜಕುಮಾರ, ನೀನು ಹೇಳಿದ ಮಾತು ಸರಿಯಪ್ಪ. ಇದೆಲ್ಲ ನಮ್ಮ ಕರ್ಮ. ಪಾಪ, ನೀನೇನು ಮಾಡುತ್ತಿ? ಇದರಮೇಲೆ ಕೊಲ್ಲುವವನಾರು, ಕೊಲ್ಲಿಸಿ ಕೊಳ್ಳುವವನಾರು? ಕೇವಲ ಅಜ್ಞಾನದಿಂದ, ದೇಹವೇ ನಾನೆಂದು ತಿಳಿದು ನಾವು ಸಂಕಟ ಪಡುತ್ತೇವೆ. ದೇಹದ ಮೇಲೆ ಮಮತೆಯನ್ನು ಇಟ್ಟವರಿಗೆ ‘ತಾನು, ಇತರರು’ ಎಂಬ ಭೇದ ಬುದ್ಧಿ ಹುಟ್ಟುತ್ತದೆಯೇ ಹೊರತು, ಎಲ್ಲರಿಗೂ ಮೇಲೆ ಈಶ್ವರನೊಬ್ಬನಿರುವನೆಂಬ ಭಾವನೆಯೆ ಬರುವುದಿಲ್ಲ ನೋಡು. ಇದಕ್ಕೆ ಬೇರೆ ಉದಾಹರಣೆಯೇ ಬೇಡ. ನೀನೆ ದೊಡ್ಡ ನಿದರ್ಶನ’ ಎಂದನು. ಅಶಾಂತಮನಸ್ಸಿನಿಂದಿದ್ದ ಕಂಸನಿಗೆ ಆತನ ಮಾತಿನ ಕೊಂಕು ಅರ್ಥವಾಗಲಿಲ್ಲ. ಆತನು ಅವರ ಅಪ್ಪಣೆಯನ್ನು ಪಡೆದು ಮನೆಗೆ ಹಿಂದಿರುಗಿದನು.

ಕಂಸರಾಜನ ಮನಸ್ಸಿಗೆ ಶಾಂತಿಯಿಲ್ಲದಂತಾಯಿತು. ಆತನು ಆ ದಿನವನ್ನು ಹೇಗೋ ಕಳೆದು, ಮರುದಿನ ಬೆಳಗಾಗುತ್ತಲೇ ತನ್ನ ಮಂತ್ರಿಗಳನ್ನು ಕರೆಸಿ, ಮಹಾಮಾಯೆಯ ಮಾತುಗಳನ್ನೆಲ್ಲ ಅವರಿಗೆ ತಿಳಿಸಿದನು. ಕಂಸನ ಮಂತ್ರಿಗಳೆಂದರೆ ಕೆಡುಕಿನ ಬುಗ್ಗೆಗಳು. ಅವರು ಆತನನ್ನು ಕುರಿತು ‘ಮಹಾರಾಜ, ಆ ಮಾಯೆಯೆಂಬ ಹೆಂಗಸಿನ ಮಾತಿಗೆ ನೀನೇಕೆ ಬೆಲೆ ಕೊಡುತ್ತಿ? ಒಂದು ಪಕ್ಷ ಅವಳು ಹೇಳಿದ ಮಾತು ನಿಜವಾಗಿದ್ದಲ್ಲಿ ನಿನ್ನ ಶತ್ರು ಶಿಶು ವಾಗಿ ಹುಟ್ಟಿ ಇಲ್ಲಿಯೇ ಎಲ್ಲಿಯೊ ಬೆಳೆಯುತ್ತಿರುವನೆಂದಾಯಿತು. ನಾವೀಗ ನಮ್ಮ ರಾಜ್ಯದ ಎಲ್ಲ ಊರುಗಳಲ್ಲಿಯೂ ಇರುವ ಎಳೆಯ ಮಕ್ಕಳನ್ನೆಲ್ಲ ಹುಡುಕಿ, ಕೊಂದುಹಾಕಿ ಬಿಡೋಣ. ಅಲ್ಲಿಗೆ ನಿನ್ನ ಮುಖ್ಯ ಶತ್ರುವಿನ ಕಥೆ ಮುಗಿಯಿತು. ಇನ್ನು ಸದಾ ನಮ್ಮ ಶತ್ರು ಗಳಾಗಿರುವ ಆ ದೇವತೆಗಳು; ಅವರು ಕೇವಲ ಮಾತಿನ ಶೂರರು. ಕಾಳಗವಿಲ್ಲದಾಗ ಅವರು ಹಮ್ಮೀರರು; ನಿನ್ನ ಬಾಣಗಳು ಕಿಡಿಗೆದರುತ್ತಾ ಹಾರಿಬರುವಾಗ ಅವರು ಜೀವಭಯದಿಂದ ಬಿಟ್ಟ ಮಂಡೆಯೊಡನೆ ಓಡಿಹೋಗುತ್ತಾರೆ, ಅಥವಾ ಮಂಡಿಯೂರಿ ಜೀವದಾನ ಬೇಡು ತ್ತಾರೆ. ಆ ದೇವತೆಗಳಿಗೆ ಬೆಂಬಲನಾಗಿರುವ ಶ್ರೀಹರಿಯೆಂಬುವನು ನಿನ್ನ ಭಯದಿಂದ ಎಲ್ಲಿಯೋ ಗೊತ್ತಾಗದಂತೆ ಅಡಗಿಕೊಂಡಿದ್ದಾನೆ; ಬ್ರಹ್ಮನು ಬ್ರಾಹ್ಮಣ ವೇಷವನ್ನು ಧರಿಸಿ ವೇದವನ್ನು ಓದುತ್ತಾ ಕುಳಿತಿದ್ದಾನೆ; ರುದ್ರನೆಂಬುವನು ಮಂದರಪರ್ವತದ ಕಾಡು ಗಳಲ್ಲಿ ಅಲೆದಾಡುತ್ತ ತಲೆಮರೆಸಿಕೊಂಡಿದ್ದಾನೆ. ನಾವು ಆದಷ್ಟು ಬೇಗ ಇವರನ್ನೆಲ್ಲ ಬಲಿ ಹಾಕಬೇಕು. ರೋಗ ಬಂದಾಗ ಒಡನೆ ಚಿಕಿತ್ಸೆ, ಶತ್ರು ಕಂಡಾಗ ಒಡನೆ ಸಂಹಾರ–ಇದು ಅತ್ಯಗತ್ಯವಾದ ಕಾರ್ಯ. ಪ್ರಭು, ಈ ದೇವತೆಗಳು ಇಷ್ಟು ಹೆಚ್ಚಿಕೊಳ್ಳುವುದಕ್ಕೆ ಮುಖ್ಯ ಕಾರಣವೆಂದರೆ ಪುಷಿಗಳು, ಬ್ರಾಹ್ಮಣರು, ಮತ್ತು ಗೋಗಳು. ಪುಷಿಗಳ ತಪಸ್ಸು, ಬ್ರಾಹ್ಮಣರ ಯಜ್ಞ, ಆ ಯಜ್ಞಕ್ಕೆ ಹವಿಸ್ಸನ್ನು ಒದಗಿಸುವ ಗೋಗಣ–ಇವುಗಳನ್ನು ನಿರ್ಮೂಲಮಾಡಿದರೆ ದೇವತೆಗಳೆಲ್ಲ ಸತ್ತಂತೆಯೆ. ಮಹಾರಾಜ, ನಮಗೆ ಅಪ್ಪಣೆಯನ್ನು ಕೊಡು, ಶತ್ರುಗಳ ಹುಟ್ಟಡಗಿಸಿಬಿಡುತ್ತೇವೆ’ ಎಂದರು. ಪಾಪ, ಸ್ವಬುದ್ಧಿಯಿಲ್ಲದ ಕಂಸ ನಿಗೆ ಅವರ ಮಾತು ಸರಿಯೆಂದೇ ತೋರಿತು! ‘ನಿಮ್ಮ ಇಷ್ಟದಂತೆಯೇ ಆಗಲಿ’ ಎಂದು ಅವರಿಗೆ ಅನುಮತಿಯನ್ನಿತ್ತನು. ಹೆಡತಲೆಯ ಮೃತ್ಯುವನ್ನು ಕಾಣದೆ, ಅವರು ಸಾಧು ಸಜ್ಜನರ ಹಿಂಸಾಕಾರ್ಯವನ್ನು ರಭಸದಿಂದ ಸಾಗಿಸಹೊರಟರು. ಇದನ್ನು ಕಂಡು ಮಹಾ ಮಾಯೆ ಪಕಪಕ ನಕ್ಕಳು.

