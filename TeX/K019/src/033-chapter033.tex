
\chapter{೩೩. ಬಲಿಚಕ್ರವರ್ತಿ}

ಅಮೃತದ ಮೇಲಿನ ಆಶೆಯಿಂದ ರಾಕ್ಷಸರೂ ದೇವತೆಗಳಷ್ಟೇ ಕಷ್ಟಪಟ್ಟು ದುಡಿದ ರಾದರೂ ಶ್ರೀಹರಿಯ ಮಾಯೆಯಿಂದ ಅದು ದೇವತೆಗಳಿಗೆ ಮಾತ್ರ ದಕ್ಕಿತು. ಆಶಾಭಂಗ ವನ್ನು ಹೊಂದಿದ ರಾಕ್ಷಸರು ಅತ್ಯಂತ ಕ್ರೋಧದಿಂದ ದೇವತೆಗಳ ಮೇಲೆ ಯುದ್ಧವನ್ನು ಹೂಡಿದರು. ಆದರೇನು? ಅಮೃತವನ್ನು ಕುಡಿದು ಅಮರರಾಗಿದ್ದ ದೇವತೆಗಳು ಸ್ವಲ್ಪವೂ ಅಳುಕದೆ ಅವರೊಡನೆ ಹೋರಾಟಕ್ಕೆ ನಿಂತರು. ಇಬ್ಬಣದವರಿಗೂ ಭಯಂಕರವಾದ ಕಾಳಗ ನಡೆಯಿತು. ಚತುರಂಗಬಲದ ಕಾಲ್ತುಳಿತದಿಂದ ಎದ್ದ ಧೂಳು ನಾಲ್ಕು ದಿಕ್ಕು ಗಳನ್ನೂ ಮುಚ್ಚಿಹಾಕಿತು. ಆದರೆ ಪ್ರವಾಹದಂತೆ ಹರಿದುಬಂದ ನೆತ್ತರಿನಲ್ಲಿ ನೆನೆದು ಆ ಧೂಳೆಲ್ಲ ಕ್ಷಣಮಾತ್ರದಲ್ಲಿಯೆ ಕೆಸರಾಗಿಹೋಯಿತು. ಆ ಯುದ್ಧರಂಗದಲ್ಲಿ ಎಲ್ಲಿ ನೋಡಿದರೂ ಸತ್ತುಬಿದ್ದಿರುವ ಯೋಧರು, ಆನೆ, ಕುದುರೆಗಳು; ಮುರಿದು ಬಿದ್ದಿರುವ ರಥ, ಧ್ವಜ, ಆಯುಧಗಳು; ಕತ್ತರಿಸಿ ಬಿದ್ದಿರುವ ರುಂಡ, ಮುಂಡ, ಕೈ, ಕಾಲುಗಳು. ಕೆಲವು ತಲೆಯಿಲ್ಲದ ಅಟ್ಟೆಗಳು ಸೆಳೆದ ಕತ್ತಿಯೊಡನೆ ಶತ್ರುಗಳನ್ನು ಅಟ್ಟಿಕೊಂಡು ಹೋಗು ತ್ತಿದ್ದರೆ, ಕೆಲವು ಮುಂಡವಿಲ್ಲದ ರುಂಡಗಳು ಬಿರಿಬಿರಿ ಕಣ್ಣುಬಿಡುತ್ತಾ ತುಟಿಗಳನ್ನು ರೋಷದಿಂದ ಕಚ್ಚುತ್ತಿದ್ದವು. ಮೃತ್ಯುವಿನ ನರ್ತನಶಾಲೆಯಂತೆ, ಪಿಶಾಚಿಗಳ ಅಡಿಗೆಯ ಮನೆಯಂತೆ ಆ ರಣಭೂಮಿ ಭಯಂಕರವಾಗಿತ್ತು.

ನಿರ್ಭಯವಾಗಿ ತಮ್ಮೊಡನೆ ನಿಂತು ಸಮಸಮವಾಗಿ ಯುದ್ಧ ಮಾಡುತ್ತಿರುವ ದೇವತೆ ಗಳನ್ನು ಕಂಡು, ರಾಕ್ಷಸರಾಜನಾದ ಬಲೀಂದ್ರನಿಗೆ ಕೋಪ ಕೆರಳಿತು. ಆತನು ಮಯನಿಂದ ನಿರ್ಮಿತವಾಗಿ, ತನಗೆ ತಾನೆ ತನ್ನ ಮನದಂತೆ ನಡೆಯಬಲ್ಲ ‘ವೈಹಾಯಸ’ವೆಂಬ ಮಾಯಾವಿಮಾನವನ್ನೇರಿ, ಇದಿರಿಗೆ ಸಿಕ್ಕ ಶತ್ರುಗಳನ್ನೆಲ್ಲಾ ಕತ್ತರಿಸುತ್ತಾ, ನೇರವಾಗಿ ದೇವೇಂದ್ರನ ಬಳಿಗೆ ಹೋದನು. ಹಾಗೆ ಹೋದವನೆ, ಇದಿರಾಳಿಗೆ ಉಸಿರು ತಿರುಗಿಸಿ ಕೊಳ್ಳುವುದಕ್ಕೂ ಅವಕಾಶವಿಲ್ಲದಂತೆ ಬಾಣಗಳ ಮಳೆಯನ್ನು ಸುರಿಸಿದನು. ಆದರೆ ದೇವೇಂದ್ರನೇನು ಸಾಮಾನ್ಯನೆ? ಆತ ಬಲೀಂದ್ರನ ಬಾಣಗಳನ್ನೆಲ್ಲ ಮಾರ್ಗಮಧ್ಯದಲ್ಲೆ ಕತ್ತರಿಸಿ ಹಾಕಿದನು. ಬಲೀಂದ್ರನು ಬಿಟ್ಟ ಶಕ್ತಿ, ತೋಮರ, ಭಲ್ಲೆ, ಶೂಲ ಮೊದಲಾದ ಆಯುಧಗಳಿಗೂ ಇದೇ ಗತಿಯಾಯಿತು. ಇದನ್ನು ಕಂಡು ಬಲಿಗೆ ದಿಕ್ಕು ತೋಚದಂತಾ ಯಿತು. ಆತನು ರಾಕ್ಷಸಮಾಯೆಯನ್ನು ಕೈಕೊಂಡು ಮಾಯವಾಗಿ ಹೋದನು. ಮರು ನಿಮಿಷದಲ್ಲಿಯೇ ದೇವಸೇನೆಯ ಮೇಲ್ಭಾಗದಲ್ಲಿ ದೊಡ್ಡದೊಂದು ಪರ್ವತ ಕಾಣಿಸಿ ಕೊಂಡಿತು. ಅದರಿಂದ ಉಳಿಯಂತೆ ಹರಿತವಾದ ಹೆಬ್ಬಂಡೆಗಳು ಸೇನೆಯ ಮೇಲೆ ಸುರಿ ಯುವುದಕ್ಕೆ ಪ್ರಾರಂಭವಾಯಿತು. ಅದರ ಹಿಂದೆಯೇ ಹಾವು ಚೇಳುಗಳು, ಹುಲಿ ಸಿಂಹಗಳೂ ಹಾರಿ ಬಂದು ದೇವತೆಗಳನ್ನು ಧ್ವಂಸಮಾಡಲು ಪ್ರಾರಂಭಿಸಿದವು; ಗುಡುಗು ಸಿಡಿಲುಗಳೊಡನೆ ಕೂಡಿದ ಬೆಂಕಿಯ ಮಳೆ ಸುರಿಯಿತು; ಸಮುದ್ರ ಮೇರೆದಪ್ಪಿ ಸೈನ್ಯದ ಮೇಲೆ ಹರಿದುಬಂದಿತು. ರಾಕ್ಷಸರೊಬ್ಬರೂ ಕಣ್ಣಿಗೆ ಬೀಳದೆ ಬರಿಯ ಉತ್ಪಾತವೊಂದನ್ನೆ ಕಾಣುತ್ತಿದ್ದ ದೇವತೆಗಳಿಗೆ ದಿಕ್ಕು ತೋಚದೆ ‘ದೇವರೆ, ನೀನೆ ಗತಿ’ ಎಂದು ಧ್ಯಾನಮಾಡು ವಂತಾಯಿತು. ನೆನೆದವರ ಮನದಲ್ಲೆ ನೆಲಸಿರುವ ಶ್ರೀಹರಿ, ದೇವತೆಗಳು ಪ್ರಾರ್ಥಿಸು ತ್ತಿದ್ದಂತೆಯೆ ಶಂಖ ಚಕ್ರ ಗದಾ ಪದ್ಮಧಾರಿಯಾಗಿ ಅಲ್ಲಿ ಪ್ರತ್ಯಕ್ಷನಾದನು. ಎಚ್ಚರ ವಾದಾಗ ಕನಸಿಗೆ ಎಡೆಯೆಲ್ಲಿ? ಶ್ರೀಹರಿ ಮೈದೋರಿದಾಗ ಮಾಯೆಗೆ ತಾವೆಲ್ಲಿದೆ? ರಾಕ್ಷಸರ ಮಾಯೆ ಮಾಯವಾಯಿತು. ರಾಕ್ಷಸರು ಎಲ್ಲಿದ್ದವರಲ್ಲೆ ಕಾಣಿಸಿಕೊಂಡರು. ಅವರಲ್ಲಿ ಕಾಲನೇಮಿಯೆಂಬುವನು ಶ್ರೀಹರಿಯ ಮೇಲೆ ಶೂಲವನ್ನು ಎಸೆಯಲು, ಶ್ರೀಹರಿಯು ಆಟದ ಚೆಂಡನ್ನು ಹಿಡಿಯುವಂತೆ ಅದನ್ನು ಹಿಡಿದು, ಅದರಿಂದಲೆ ಆ ರಕ್ಕಸನನ್ನು ಕೊಂದುಹಾಕಿದನು. ಅವನಂತೆಯೇ ಮಾಲಿ ಸುಮಾಲಿಗಳೂ ಸತ್ತುಬಿದ್ದರು. ಶ್ರೀಹರಿಯು ಮಾಯವಾಗಿ ಹೋದನು.

ಶ್ರೀಹರಿಯ ಕೃಪೆಯಿಂದ ರಾಕ್ಷಸಮಾಯೆ ಹಾರಿಹೋಗುತ್ತಲೆ ದೇವೇಂದ್ರನು ಬಲಿ ರಾಜನೊಡನೆ ತನ್ನ ಯುದ್ಧವನ್ನು ಮುಂದುವರಿಸಿದನು. ಆತನು ತನ್ನ ಆಯುಧಗಳಲ್ಲೆಲ್ಲ ಅತ್ಯಂತ ಶ್ರೇಷ್ಠವಾದ ವಜ್ರಾಯುಧವನ್ನು ಕೈಗೆ ತೆಗೆದುಕೊಂಡು, ‘ಮೂರ್ಖ, ಮಕ್ಕಳ ಕಣ್ಣು ಕಟ್ಟಿ ಅವರಲ್ಲಿರುವ ವಸ್ತುಗಳನ್ನು ಲಪಟಾಯಿಸುವ ಕಳ್ಳನಂತೆ ನಿನ್ನ ಮಾಯಾ ವಿದ್ಯೆಯಿಂದ ನಮ್ಮ ಸತ್ವವನ್ನು ಹೀರುವುದಕ್ಕೆ ಪ್ರಯತ್ನಿಸುವಿಯಾ? ಎಲ್ಲಿ, ಈ ವಜ್ರದ ಮುಂದೆ ನಿನ್ನ ಮಾಯಾವಿದ್ಯೆಯನ್ನು ತೋರಿಸು, ನೋಡೋಣ’ ಎಂದು ಹರಿತವಾದ ಎಂಟು ಅಲಗುಗಳುಳ್ಳ ಆ ವಜ್ರಾಯುಧದಿಂದ ಅವನ ತಲೆಯ ಮೇಲೆ ಹೊಡೆದನು. ಆ ವಜ್ರವೆಂದರೆ ಏನು ಸಾಮಾನ್ಯವೆ? ಯಾವ ಅಸ್ತ್ರ ಶಸ್ತ್ರಗಳಿಂದಲೂ ಕೂದಲು ಕೊಂಕದಿದ್ದ ವೃತ್ರಾಸುರನನ್ನು ಅದು ಕತ್ತರಿಸಿಹಾಕಿತ್ತು. ದಧೀಚಿ ಮಹರ್ಷಿಯ ತಪೋಫಲವನ್ನು ಮೈಗೂಡಿಸಿಕೊಂಡು ಶತ್ರುವಿಗೆ ಮೃತ್ಯುವಿನಂತೆ ಮೂಡಿದ್ದ ಆ ವಜ್ರಾಯುಧ ಬಲಿ ಚಕ್ರವರ್ತಿಯನ್ನು ಕೊಲ್ಲುವುದಿರಲಿ, ಆತನ ಕುತ್ತಿಗೆಯ ಮೇಲೆ ಗಾಯವನ್ನು ಮಾಡು ವುದಕ್ಕೆ ಕೂಡ ಶಕ್ತವಾಗಲಿಲ್ಲ. ಇದನ್ನು ಕಂಡು ದೇವೇಂದ್ರ ಕಂಗಾಲಾಗಿ ಹೋದನು. ಅಷ್ಟರಲ್ಲಿ ಆಕಾಶದಿಂದ ಒಂದು ದನಿ ಕೇಳಿಬಂತು–‘ಅಯ್ಯಾ ದೇವೇಂದ್ರ, ಈ ಬಲೀಂದ್ರನಿಗೆ ಗಟ್ಟಿಯಾದ ವಸ್ತುಗಳಿಂದಾಗಲಿ, ನೀರಾದ ವಸ್ತುಗಳಿಂದಾಗಲಿ ಮರಣ ವಾಗದೆಂಬ ವರವಿದೆ. ಆದ್ದರಿಂದ ಇವನನ್ನು ಕೊಲ್ಲುವುದಕ್ಕೆ ಬೇರೊಂದು ಉಪಾಯ ವನ್ನು ಯೋಚಿಸು.’ ಇದನ್ನು ಕೇಳುತ್ತಲೆ ದೇವೇಂದ್ರನು ಕ್ಷಣಕಾಲ ಯೋಚನಾಮಗ್ನ ನಾಗಿದ್ದು, ಗಟ್ಟಿಯೂ ನೀರೂ ಅಲ್ಲದ ಸಮುದ್ರದ ನೊರೆಯನ್ನು ತೆಗೆದುಕೊಂಡು ಅದರಿಂದ ಅವನನ್ನು ಹೊಡೆದನು. ತಕ್ಷಣವೇ ಬಲೀಂದ್ರನು ಸತ್ತು ನೆಲಕ್ಕುರುಳಿದನು.

ಬಲೀಂದ್ರನು ಸತ್ತುದನ್ನು ಕಂಡು ದೇವತೆಗಳ ಧೈರ್ಯ ಇಮ್ಮಡಿಯಾಯಿತು. ಅವರು ರಾಕ್ಷಸರನ್ನು ಹುಳುಗಳಂತೆ ಹೊಸಕಿಹಾಕುವುದಕ್ಕೆ ಪ್ರಾರಂಭಿಸಿದರು. ಆ ವೇಳೆಗೆ ನಾರದರು ಅಲ್ಲಿ ಕಾಣಿಸಿಕೊಂಡು ‘ಅಯ್ಯಾ ದೇವತೆಗಳೆ, ಬ್ರಹ್ಮದೇವರು ನನ್ನನ್ನು ನಿಮ್ಮ ಬಳಿಗೆ ಕಳುಹಿಸಿದ್ದಾನೆ. ನೀವು ಶ್ರೀಹರಿಯ ಕರುಣೆಯಿಂದ ಅಮೃತವನ್ನು ಪಡೆದುದಾಯಿತು. ಮೂರುಲೋಕದ ಐಶ್ವರ್ಯವೂ ನಿಮ್ಮದಾಗಿದೆ, ಎಂದಮೇಲೆ ನೀವು ಮತ್ತಾವುದಕ್ಕಾಗಿ ಯುದ್ಧಮಾಡಬೇಕು? ಇದು ಸರಿಯಲ್ಲ. ನೀವು ತಕ್ಷಣವೇ ಯುದ್ಧವನ್ನು ನಿಲ್ಲಿಸಿರಿ’ ಎಂದು ಹೇಳಿದನು. ಒಡನೆಯೆ ದೇವತೆಗಳು ಯುದ್ಧವನ್ನು ನಿಲ್ಲಿಸಿ, ಸ್ವರ್ಗಕ್ಕೆ ಹಿಂದಿರುಗಿದರು. ಅನಂತರ ರಾಕ್ಷಸರು ನಾರದಮುನಿಯ ಅಪ್ಪಣೆಯಂತೆ ಬಲೀಂದ್ರನ ದೇಹವನ್ನು ಹೊತ್ತುಕೊಂಡು ತಮ್ಮ ಸ್ವಸ್ಥಾನವಾದ ಅಸ್ತಗಿರಿಗೆ ಹಿಂದಿರುಗಿದರು. ಅಲ್ಲಿ ರಾಕ್ಷಸರ ಪುರೋಹಿತರಾದ ಶುಕ್ರಾಚಾರ್ಯರು ಸತ್ತ ಬಲೀಂದ್ರನನ್ನು ತಮ್ಮ ಸಂಜೀವಿನೀ ವಿದ್ಯೆಯಿಂದ ಮತ್ತೆ ಬದುಕಿಸಿದರು. ಅವರ ಉಪಚಾರದಿಂದ ಆ ರಾಕ್ಷಸ ರಾಜನಿಗೆ ಮತ್ತೆ ದೇಹಶಕ್ತಿ ಜ್ಞಾಪಕಶಕ್ತಿಗಳು ಎಂದಿನಂತೆ ಬಂದವು. ಮಹಾಜ್ಞಾನಿಯಾದ ಬಲೀಂದ್ರನು ‘ಸೋಲು ಗೆಲುವು, ಸುಖ ದುಃಖ–ಇವೆಲ್ಲ ಪೂರ್ವಜನ್ಮದ ಕರ್ಮ’ ಎಂದುಕೊಂಡು ಸಮಾಧಾನ ವನ್ನು ವಹಿಸಿದನು.

ತನಗೆ ಪ್ರಾಣದಾನ ಮಾಡಿದ ಶುಕ್ರಾಚಾರ್ಯರಲ್ಲಿ ಬಲೀಂದ್ರನಿಗೆ ಬಹುವಾಗಿ ಭಕ್ತಿ ಗೌರವಗಳು ಹುಟ್ಟಿದವು. ಅವರನ್ನು ಆತ ದಾನ ದಕ್ಷಿಣೆಗಳಿಂದ ತೃಪ್ತಿಪಡಿಸಿದನು. ಇದರಿಂದ ಸಂತಸಗೊಂಡ ಶುಕ್ರಾಚಾರ್ಯರು ಬಲೀಂದ್ರನಿಂದ ‘ವಿಶ್ವಜಿತ್​’ಯಾಗವನ್ನು ಮಾಡಿಸಿದರು. ಅದರ ಫಲವಾಗಿ ಯಜ್ಞ ಕುಂಡದಿಂದ ಒಂದು ಚಿನ್ನದ ರಥ ಮೂಡಿಬಂತು. ಅದಕ್ಕೆ ದೇವೇಂದ್ರನ ರಥದ ಕುದುರೆಗಳಂತೆ ಹಸುರಾದ ಕುದುರೆಗಳನ್ನು ಹೂಡಿತ್ತು; ಸಿಂಹದ ಬಾವುಟವಿತ್ತು. ಈ ರಥದ ಜೊತೆಗೆ ಬಿಲ್ಲು, ಬಾಣಗಳಿಂದ ತುಂಬಿದ ಬತ್ತಳಿಕೆ, ವಜ್ರದ ಕವಚ–ಇವೂ ಬಂದುವು. ಯಾಗದ ಕೊನೆಯಲ್ಲಿ ಬ್ರಹ್ಮನು ಪ್ರತ್ಯಕ್ಷನಾಗಿ, ಬಾಡದಿರುವ ಕಮಲಗಳ ಮಾಲೆಯೊಂದನ್ನು ರಾಕ್ಷಸರಾಜನ ಕೊರಳಲ್ಲಿ ಹಾಕಿದನು. ಶುಕ್ರಾಚಾರ್ಯರು ಶಿಷ್ಯನಿಗೆ ಒಂದು ದಿವ್ಯವಾದ ಶಂಖವನ್ನು ಬಹುಮಾನವಾಗಿ ಕೊಟ್ಟರು. ಇವುಗಳನ್ನೆಲ್ಲ ಪಡೆದ ಬಲೀಂದ್ರನು ಅಜೇಯನಾದನು. ಆತನು ಭಕ್ತಿಯಿಂದ ಗುರು ಗಳಾದ ಶುಕ್ರಾಚಾರ್ಯರಿಗೆ ನಮಸ್ಕರಿಸಲು ‘ವಿಜಯೀಭವ’ ಎಂದು ಅವರು ಆಶೀರ್ವದಿಸಿ ದರು. ರಾಕ್ಷಸವಂಶಕ್ಕೆ ಹಿರಿಯನೆನಿಸಿದ್ದ ಪ್ರಹ್ಲಾದನೂ ತನಗೆ ನಮಸ್ಕರಿಸಿದ ಬಲೀಂದ್ರ ನನ್ನು ‘ಜಯಶಾಲಿಯಾಗು’ ಎಂದು ಹರಸಿದನು. 

ಹೀಗೆ ಗುರುಹಿರಿಯರ ಹರಕೆಯನ್ನು ಹೊತ್ತ ಬಲೀಂದ್ರನು ಸರ್ವಾಲಂಕಾರಭೂಷಿತ ನಾಗಿ, ಯಾಗದಲ್ಲಿ ದೊರೆತ ರಥವನ್ನೇರಿ, ಬ್ರಹ್ಮನಿತ್ತ ಹಾರವನ್ನು ಧರಿಸಿಕೊಂಡು, ವರವಾಗಿ ಬಂದಿದ್ದ ಬಿಲ್ಲು ಬಾಣಗಳೊಡನೆ ಸಜ್ಜಾಗಿ, ವಿಜಯಯಾತ್ರೆ ಹೊರಟನು. ನಿಗಿನಿಗಿ ಹೊಳೆಯುವ ಅಗ್ನಿಯಂತೆ ತೊಳಗಿ ಬೆಳಗುತ್ತಿದ್ದ ಆತನು ತನ್ನ ದಿಟ್ಟಿಯಿಂದಲೇ ಆಕಾಶವನ್ನು ಕುಡಿಯುವವನಂತೆ, ದಿಕ್ಕುಗಳನ್ನು ಸುಡುವವನಂತೆ ರೋಷ ತಾಮ್ರಾಕ್ಷ ನಾಗಿದ್ದನು. ಆತನು ತನ್ನಂತೆಯೆ ಶೂರರಾದ ರಾಕ್ಷಸಸೈನ್ಯದೊಡನೆ ಸ್ವರ್ಗದಮೇಲೆ ದಂಡೆತ್ತಿಹೋದನು. ಆ ಸ್ವರ್ಗದ ಸೌಂದರ್ಯ ಐಶ್ವರ್ಯಗಳನ್ನು ಕೇಳಬೇಕೆ? ಅದರ ಸುತ್ತಲೂ ಪಾರಿಜಾತ ಕಲ್ಪವೃಕ್ಷಾದಿ ಫಲಪುಷ್ಪಭರಿತವಾದ ಗಿಡ ಬಳ್ಳಿಗಳು; ಅವುಗಳಲ್ಲಿ ಕಣ್ಣಿಗೆ ಇಂಪನ್ನೂ ಮನಸ್ಸಿಗೆ ತಂಪನ್ನೂ ಕೊಡುವ ಬಗಬಗೆಯ ಬಣ್ಣದ ಹಕ್ಕಿಗಳು; ಕಿವಿಗೆ ಅಮೃತವನ್ನು ಸುರಿಯುವ ಅವುಗಳ ಮಂಜುಳಗಾನ; ಅಲ್ಲಲ್ಲಿ ಕ್ರೀಡಾ ಪರ್ವತಗಳು; ಜಲಕೇಳಿಗಾಗಿ ನಿರ್ಮಿಸಿದ ತಿಳಿನೀರಿನ ಸರೋವರಗಳು. ಇವುಗಳ ಆನಂದವನ್ನು ಸವಿ ಯುತ್ತಾ ಮುಂದುವರಿದರೆ ಕೋಟೆಯ ಕಂದಕ ಕಾಣಿಸುವುದು. ಅದರಲ್ಲಿ ದೇವಗಂಗೆ ತುಂಬಿದೆ. ಅದರಾಚೆ ಕೋಟೆ. ಮುಗಿಲೆತ್ತರಕ್ಕೆ ಬೆಳೆದುನಿಂತ ಆ ರನ್ನದ ಕೋಟೆಗೆ ಚಿನ್ನದ ಚೌಕಟ್ಟನ್ನು ಹಾಕಿರುವ ಸ್ಫಟಿಕದ ಬಾಗಿಲುಗಳು. ಅವುಗಳ ಮೂಲಕ ಒಳಗೆ ಪ್ರವೇಶಿಸಿದರೆ ಕಣ್ಣನ್ನು ಕೋರೈಸುವ ಉಪ್ಪರಿಗೆಗಳು; ವಿಸ್ತಾರವಾದ ರಾಜ ಬೀದಿಗಳು; ಅವು ಸೇರುವ ಕಡೆ ವಜ್ರ ಹವಳ ಇತ್ಯಾದಿಗಳಿಂದ ಮಾಡಿದ ಸುಖಾಸನಗಳು. ಬೀದಿಗಳಲ್ಲಿ ಓಡಾಡುವ ಹೆಣ್ಣು ಗಳೆಲ್ಲ ಅಪ್ಸರೆಯರೇ. ನಿತ್ಯ ಯೌವನೆಯರಾದ ಆ ಸುರನಾರಿಯರು ಬಳ್ಳಿ ಮಿಂಚಿನಂತೆ ನಡೆದಾಡುತ್ತಿರುವಾಗ ಅವರ ಮುಡಿಯಲ್ಲಿದ್ದ ಪಾರಿಜಾತದ ಹೂ ಕೆಳಕ್ಕುದುರಿ, ಅದರ ಸುವಾಸನೆಯಿಂದ ಅಲ್ಲಲ್ಲಿ ಬೀಸುವ ಗಾಳಿ ಸುಗಂಧಯುಕ್ತವಾಗಿರುವುದು. ಅಲ್ಲಲ್ಲಿಯೆ ಮನೆಯೊಳಗಿನಿಂದ ಗಾಳಿಯಲ್ಲಿ ತೇಲಿಬರುತ್ತಿರುವ ಮಧುರಗಾನವು, ಆ ಮನೆಯ ಚಿನ್ನದ ಕಿಟಕಿಯಿಂದ ಹೊರಕ್ಕೆದ್ದು ಬರುವ ಧೂಪಧೂಮದೊಡನೆ ಸೇರಿ ಮುನಿಜನರ ಮಾರಕಾಸ್ತ್ರ ದಂತಿರುವುದು. ಇಂತಹ ನಿತ್ಯ ಭೋಗದ ಸ್ವರ್ಣಸ್ವಪ್ನದಂತಿದ್ದ ಸ್ವರ್ಗಲೋಕಕ್ಕೆ ಬಲೀಂದ್ರನ ರಾಕ್ಷಸ ಸೈನ್ಯ ನುಗ್ಗಿ ಬಂದಿತು.

ತನ್ನ ಸೈನ್ಯವೆಲ್ಲವೂ ಸ್ವರ್ಗವನ್ನು ಪ್ರವೇಶಿಸುತ್ತಲೆ ರಾಕ್ಷಸರಾಜನಾದ ಬಲೀಂದ್ರನು ಅಪ್ಸರಸಿಯರ ಗರ್ಭ ನಿರ್ಭೇದವಾಗುವಂತೆ ತನ್ನ ಶಂಖವನ್ನು ಊದಿದನು. ಇದನ್ನು ಕೇಳಿ ಸ್ವತಃ ದೇವೇಂದ್ರನಿಗೆ ತಲ್ಲಣ ಹುಟ್ಟಿತು. ಆತನು ಕುಲಗುರವಾದ ಬೃಹಸ್ಪತಿಯ ಬಳಿಗೆ ಓಡಿಹೋಗಿ ‘ಸ್ವಾಮಿ, ನನ್ನ ಹಳೆಯ ಶತ್ರು ಬಲಿ, ದಂಡೆತ್ತಿಬಂದಿದ್ದಾನೆ. ಆತನ ಶಂಖ ಧ್ವನಿಯನ್ನು ಕೇಳಿಯೇ ನನಗೆ ನಡುಕ ಹುಟ್ಟಿದೆ. ಸತ್ತವನು ಮತ್ತೆ ಇಷ್ಟು ಪ್ರಬಲನಾದುದು ಹೇಗೊ ನಾನರಿಯೆ. ಮುಂದೆ ನನ್ನ ಗತಿಯೇನು?’ ಎಂದು ಕೇಳಿದ. ಬೃಹಸ್ಪತಿ ಹೇಳಿದರು ‘ಅಯ್ಯಾ, ಶುಕ್ರಾಚಾರ್ಯರು ತಮ್ಮ ತಪಸ್ಸನ್ನು ಇವನಲ್ಲಿ ತುಂಬಿ ಕಳುಹಿಸಿದ್ದಾರೆ. ಶ್ರೀಹರಿ ಯೊಬ್ಬನ ಹೊರತು ಮತ್ತಾರೂ ಈಗ ಇವನನ್ನು ಇದಿರಿಸಲಾರರು. ಬ್ರಹ್ಮತೇಜಸ್ಸಿನಿಂದ ಬಂದ ಇವನ ಶಕ್ತಿ ಮತ್ತೆ ಬ್ರಹ್ಮತೇಜಸ್ಸಿನಿಂದಲೇ ನಾಶವಾಗಬೇಕು. ಹಾಗಾಗುವವರೆಗೆ ತಲೆಮರೆಸಿಕೊಳ್ಳುವುದೊಂದೆ ನಿನಗಿರುವ ಉಪಾಯ.] ಈ ಮಾತನ್ನು ಕೇಳುತ್ತಲೆ ದೇವೇಂದ್ರನು ತನ್ನ ಅನುಯಾಯಿಗಳೊಡನೆ ಸ್ವರ್ಗದಿಂದ ಓಡಿಹೋದನು. ಇದರಿಂದ ಬಲಿಯು ನಿರಾತಂಕವಾಗಿಯೇ ಸ್ವರ್ಗವನ್ನು ಸ್ವಾಧೀನಮಾಡಿಕೊಂಡು ಮೂರುಲೋಕ ಗಳಿಗೂ ಅಧಿಪತಿಯಾದನು. ಅನಂತರ ಆತನು ಶುಕ್ರಾಚಾರ್ಯ ಮತ್ತು ಆತನ ವಂಶ ದವರಾದ ಬ್ರಾಹ್ಮಣರ ಸಹಾಯದಿಂದ ನೂರು ಯಾಗಗಳನ್ನು ಮಾಡಿದನು. ಆತನ ಕೀರ್ತಿ ಮೂರು ಲೋಕಗಳಲ್ಲಿಯೂ ಹಬ್ಬಿ ಹರಡಿತು.

ದೇವತೆಗಳೆಲ್ಲರೂ ಬಲೀಂದ್ರನ ಭಯದಿಂದ ತಲೆಮರೆಸಿಕೊಂಡುದನ್ನು ಕಂಡು, ದೇವತೆಗಳ ತಾಯಿಯಾದ ಅದಿತಿಗೆ ಸಂಕಟವಾಯಿತು. ಆಕೆಯ ಗಂಡನಾದ ಕಶ್ಯಪ ಪುಷಿಯು ಆ ಸಮಯದಲ್ಲಿ ತಪೋಮಗ್ನನಾಗಿದ್ದನು. ಆತನು ಬಹುಕಾಲದ ಮೇಲೆ ತಪಸ್ಸಿ ನಿಂದ ಎಚ್ಚೆತ್ತು, ಮಡದಿಯ ಬಾಡಿದ ಮುಖವನ್ನು ನೋಡುತ್ತಲೆ ‘ಎಲೆ ಮಂಗಳಾಂಗಿ, ನಿನ್ನ ಮುಖ ಅದೇಕಿಷ್ಟು ಬಾಡಿಹೋಗಿದೆ? ನಾನು ತಪಸ್ಸಿಗೆ ಕುಳಿತಾಗ ನೀನು ಗೃಹಸ್ಥಾಶ್ರಮ ಕಾರ್ಯದಲ್ಲಿ ಎಲ್ಲಿಯಾದರೂ ತಪ್ಪುಮಾಡಿದೆಯಾ? ಅತಿಥಿಗಳಿಗೆ ಮಾಡುವ ಸತ್ಕಾರ್ಯದಲ್ಲಿ ಏನಾದರೂ ಲೋಪವಾಯಿತೆ? ನಿನ್ನ ಮಕ್ಕಳೆಲ್ಲ ನೆಮ್ಮದಿಯಾಗಿರುವರೆ? ನಿನ್ನನ್ನು ನೋಡಿದರೆ ನಿನ್ನ ಮನಸ್ಸು ತುಂಬ ಕಳವಳಕ್ಕೆ ಈಡಾಗಿರುವಂತೆ ಕಾಣಿಸುತ್ತಿದೆಯಲ್ಲ! ಅದಕ್ಕೆ ಏನು ಕಾರಣ?’ ಎಂದು ಕೇಳಿದನು. ಅದಿತಿದೇವಿಯು ಪ್ರತ್ಯುತ್ತರವೀಯುತ್ತಾ ‘ಸ್ವಾಮಿ, ನಾನು ಏನು ಹೇಳಲಿ? ನಿನ್ನಂತಹ ಮಹಾನುಭಾವನನ್ನು ಕೈಹಿಡಿದ ನಾನು ಯಾವ ಲೋಪದೋಷಗಳಿಗೂ ಕಾರಣಳಾಗಲಾರೆ. ನೀನು ಊಹಿಸಿರುವಂತೆ ಮಕ್ಕಳ ನೆಮ್ಮದಿ ಗಾಗಿಯೇ ನಾನು ಕುದಿಯುತ್ತಿದ್ದೇನೆ. ದೈತ್ಯದಾನವರಂತೆ ನಾನೂ ನನ್ನ ಮಕ್ಕಳೂ ನೀನೇ ದಿಕ್ಕೆಂದು ನಂಬಿಕೊಂಡಿದ್ದೇವೆ. ನನ್ನ ಮಕ್ಕಳ ಅಧಿಕಾರ ಸಂಪತ್ತುಗಳನ್ನು ನನ್ನ ಸವತಿಯ ಮಕ್ಕಳಾದ ದೈತ್ಯದಾನವರು ಕಿತ್ತುಕೊಂಡಿದ್ದಾರೆ. ಪ್ರಬಲರಾದ ಅವರನ್ನು ಇದಿರಿಸಲಾರದೆ ಸಜ್ಜನರಾದ ನನ್ನ ಮಕ್ಕಳು ತಲೆಮರೆಸಿಕೊಂಡು ದೇಶಾಂತರ ಹೋಗಿದ್ದಾರೆ. ನಮಗೆ ಈಗ ದಿಕ್ಕಾರು? ನೀನೆ ಅಲ್ಲವೆ? ನನ್ನನ್ನೂ ನನ್ನ ಮಕ್ಕಳನ್ನೂ ಕಾಪಾಡು; ನಮ್ಮ ರಾಜ್ಯವನ್ನು ನಮಗೆ ಕೊಡಿಸಿಕೊಡು’ ಎಂದು ಬೇಡಿಕೊಂಡಳು.

ಹೆಂಡತಿಯ ಮಾತುಗಳನ್ನು ಕೇಳಿ ಕಶ್ಯಪಪುಷಿಗೆ ನಗುಬಂತು. ‘ಅಯ್ಯೋ ಮಾಯಾ ಶಕ್ತಿಯೆ! ಯಾರು ತಂದೆ? ಯಾರು ಮಗ?’ ಎಂದು ಮನಸ್ಸಿನಲ್ಲಿ ಹೇಳಿಕೊಂಡು “ದೇವಿ, ನಿನ್ನ ಮೋಹಕ್ಕೆ ನಾನೇನು ಹೇಳಲಿ? ಶಿಕ್ಷೆ ರಕ್ಷೆಗೆ ನಾನಾರು? ಭಗವಂತನನ್ನು ಮರೆಹೋಗು, ಆತ ನಿನ್ನ ಕೋರಿಕೆಗಳನ್ನೆಲ್ಲ ನೆರವೇರಿಸುತ್ತಾನೆ” ಎಂದು ಹೇಳಿ ಆಕೆಗೆ ಪಯೋವ್ರತವನ್ನು ಉಪದೇಶಿಸಿದನು. ಫಾಲ್ಗುಣಮಾಸದ ಶುಕ್ಲಪಕ್ಷದಲ್ಲಿ ಹನ್ನೆರಡು ದಿನಗಳವರೆಗೆ ನಡೆಸ ಬೇಕಾದ ವ್ರತ, ಅದು. ಆ ಕಾಲದಲ್ಲಿ ಶ್ರೀಹರಿಯನ್ನು ಭಕ್ತಿಯಿಂದ ಪೂಜಿಸಿ, ಹಾಲನ್ನು ಮಾತ್ರ ಆಹಾರವಾಗಿ ಸೇವಿಸಬೇಕು. ವ್ರತದ ಹಿಂದಿನ ದಿನ ಬರುವ ಅಮಾವಾಸ್ಯೆಯಂದು ‘ತ್ವಂ ದೇವ್ಯಾದಿವರಾಹೇಣ ರಸಾಯಾ ಸ್ಥಾನ ಮಿಚ್ಛತಾ ಉದ್ಧೃತಾನಿ ನಮಸ್ತುಭ್ಯಂ ಪಾಪ್ಮಾನಂ ಮೇ ಪ್ರಣಾಶಯ’–ಹೇ ಭೂದೇವಿ, ನೀನು ನೆಲೆಯನ್ನು ಬೇಡಿದಾಗ ಶ್ರೀಹರಿಯು ವರಾಹರೂಪದಿಂದ ಬಂದು ಪಾತಾಳದಿಂದ ನಿನ್ನನ್ನು ಉದ್ಧರಿಸಿದ. ಆದ್ದ ರಿಂದ ಭಗವಂತನ ಪೂಜೆಗಾಗಿ ಸಿದ್ಧವಾಗಿರುವ ನನ್ನ ಪಾಪಗಳನ್ನು ಪರಿಹರಿಸು–ಎಂಬ ಮಂತ್ರವನ್ನು ಜಪಮಾಡುತ್ತಾ ಸ್ನಾನ ಮಾಡಬೇಕು. ಅನಂತರ ಶ್ರೀಹರಿಯನ್ನು ಪೂಜಿಸಿ, ಕೈಕೊಂಡ ವ್ರತವನ್ನು ಮುಂದಿನ ಹನ್ನೆರಡು ದಿನಗಳೂ ನಡೆಸಿ, ಅದರ ಕಡೆಯಲ್ಲಿ ಅತಿಥಿ ಅಭಾಗ್ಯತರನ್ನೂ ಪುರೋಹಿತರನ್ನೂ ದಾನ ದಕ್ಷಿಣೆಗಳಿಂದ ತೃಪ್ತಿಪಡಿಸಬೇಕು’ ಎಂದು ಹೇಳಿದನು. ಸತೀಮಣಿಯಾದ ಅದಿತಿಯು ಗಂಡನ ಅನುಜ್ಞೆಯನ್ನು ಪಡೆದು ಆ ವ್ರತವನ್ನು ಕೈಕೊಂಡಳು. ಆಕೆಯ ಭಕ್ತಿ ಪೂಜೆಗಳಿಂದ ಸಂತೋಷಗೊಂಡ ಶ್ರೀಹರಿಯು ಆಕೆಗೆ ಪ್ರತ್ಯಕ್ಷನಾಗಿ ‘ಅಮ್ಮ, ದೇವಮಾತೆ, ನಿನ್ನ ಬಯಕೆಯೇನೆಂಬುದು ನನಗೆ ಗೊತ್ತು. ನಾನು ನಿನ್ನ ಗಂಡನ ತಪಸ್ಸಿನಲ್ಲಿ ನೆಲೆಸಿ, ನಿನ್ನ ಹೊಟ್ಟೆಯಲ್ಲಿ ಮಗನಾಗಿ ಹುಟ್ಟುತ್ತೇನೆ. ಆಗ ನಿನ್ನ ಮಕ್ಕಳನ್ನು ಕಾಪಾಡುತ್ತೇನೆ. ನೀನು ಈ ಸಂಗತಿಯನ್ನು ರಹಸ್ಯವಾಗಿಟ್ಟುಕೊ’ ಎಂದು ಹೇಳಿ ಮಾಯವಾದನು.

ಶ್ರೀಹರಿಯು ಮಾತುಕೊಟ್ಟಿದ್ದಂತೆ ಅದಿತಿದೇವಿಯ ಮಗನಾಗಿ ಅವತರಿಸಿದನು. ಆತನ ದೇಹ ಕಾಂತಿಯಿಂದ ಕಶ್ಯಪನ ಮನೆಯೆಲ್ಲ ತೊಳಗಿ ಬೆಳಗಿತು; ಹೊಳೆಕೊಳಗಳು ನಿರ್ಮಲ ವಾದವು. ಜನರ ಮನಸ್ಸಿನಲ್ಲಿ ಆನಂದ ಉಕ್ಕಿತು, ಗಿಡಗಳು ಫಲಪುಷ್ಪಭರಿತವಾದವು. ಸಿದ್ಧ ವಿದ್ಯಾಧರಾದಿಗಳು ಗಾನಮಾಡುತ್ತಾ ಕಶ್ಯಪಪುಷಿಯ ಆಶ್ರಮದಲ್ಲಿ ಹೂಮಳೆಗರೆದರು. ಅದಿತಿದೇವಿಗೆ ತನ್ನ ಮಗನನ್ನು ಕಂಡು ಭಕ್ತಿಯುದಿಸಿತು. ಆಕೆ ಸರ್ವಜ್ಞನಾದ ತನ್ನ ಗಂಡನೊಡನೆ ಆ ಮಗುವನ್ನು ಸಂತೋಷ ಸಂಭ್ರಮಗಳಿಂದ ನೋಡುತ್ತಿರುವಂತೆಯೆ ಆ ಮಗು ಪುಟ್ಟ ಬಾಲಕನ ಆಕಾರವನ್ನು ತಳೆಯಿತು. ಕಶ್ಯಪ ಪುಷಿಯು ಈ ಮಗುವಿಗೆ ಜಾತಕರ್ಮದ ಶಾಸ್ತ್ರವನ್ನು ಮಾಡಿ ಮುಗಿಸಿ ವಾಮನನೆಂದು ನಾಮಕರಣ ಮಾಡಿದನು. ಕೆಲಕಾಲದಮೇಲೆ ಆತನಿಗೆ ಉಪನಯನವನ್ನು ನೆರವೇರಿಸಬೇಕೆಂದಿರಲು ಸೂರ್ಯನೇ ಕೆಳ ಕ್ಕಿಳಿದು ಬಂದು ಅವನಿಗೆ ಗಾಯತ್ರಿಯನ್ನು ಉಪದೇಶಿಸಿದನು; ಆ ಕರ್ಮಕ್ಕೆ ಅಗತ್ಯವಾದ ಜನಿವಾರವನ್ನು ಬೃಹಸ್ಪತಿ ತಂದು ಒದಗಿಸಿದನು. ಆ ಕಾಲದಲ್ಲಿ ಭೂದೇವಿಯು ಕೃಷ್ಣಾಜಿನ ವನ್ನೂ, ಚಂದ್ರನು ದಂಡವನ್ನೂ, ಬ್ರಹ್ಮನು ಕಮಂಡಲವನ್ನೂ, ಕುಬೇರನು ಭಿಕ್ಷಾಪಾತ್ರೆ ಯನ್ನೂ ನೂತನ ವಟುವಿಗೆ ಬಹುಮಾನವಾಗಿ ಕೊಟ್ಟು ಗೌರವಿಸಿದರು. ಸಾಕ್ಷಾತ್ ಅನ್ನ ಪೂರ್ಣೆಯೇ ಆತನಿಗೆ ಭಿಕ್ಷವಿಕ್ಕಿದಳು. ಬ್ರಹ್ಮಚಾರಿಯಾದ ವಾಮನಮೂರ್ತಿಯು ಬ್ರಹ್ಮ ತೇಜಸ್ಸಿನಿಂದ ತೊಳಗಿ ಬೆಳಗುತ್ತಾ ಬ್ರಹ್ಮಪುಷಿಗಳಿಗಿಂತ ಹೆಚ್ಚು ಕಾಂತಿಯುಕ್ತನಾಗಿ ದ್ದನು.

ವಾಮನಾವತಾರಿಯಾದ ಶ್ರೀಹರಿಯು ಕಶ್ಯಪಾಶ್ರಮದಲ್ಲಿ ಉದಿತೋದಿತನಾಗು ತ್ತಿರಲು, ಮೂರು ಲೋಕಗಳ ಸ್ವಾಮಿಯಾಗಿದ್ದ ಬಲಿಚಕ್ರವರ್ತಿಯು ಶುಕ್ರಾಚಾರ್ಯರ ನೇತೃತ್ವದಲ್ಲಿ ಅಶ್ವಮೇಧಯಾಗವನ್ನು ಕೈಕೊಂಡನು. ಇದನ್ನು ಕೇಳಿದ ವಾಮನನು ಅದನ್ನು ನೋಡಲೆಂದು ಹೊರಟು, ನರ್ಮದಾ ನದಿಯ ತೀರದಲ್ಲಿದ್ದ ಯಾಗಶಾಲೆಯನ್ನು ಸೇರಿದನು. ಒಂದು ಕೈಲಿ ಕೊಡೆ, ಮತ್ತೊಂದು ಕೈಲಿ ಕಮಂಡಲು, ಕೊರಳಲ್ಲಿ ಜನಿವಾರ, ಸೊಂಟದಲ್ಲಿ ಮೌಂಜಿ, ಮೈಮೇಲೆ ಉತ್ತರೀಯ–ಇವುಗಳನ್ನು ಧರಿಸಿ ಮೂಡಿದ ಸೂರ್ಯ ನಂತೆ ತೇಜಸ್ವಿಯಾದ ಆ ವಾಮನಮೂರ್ತಿಯನ್ನು ಕಾಣುತ್ತಲೆ ಅಲ್ಲಿದ್ದವರೆಲ್ಲ ಎದ್ದು ನಿಂತು ಆತನನ್ನು ಬರಮಾಡಿಕೊಂಡರು. ಬಲಿಚಕ್ರವರ್ತಿಗೆ ಆತನಲ್ಲಿ ಅಪಾರವಾದ ಭಕ್ತಿ ಗೌರವಗಳು ಮೂಡಿದವು. ಆತನು ಆ ವಟುವಿಗೆ ನಮಸ್ಕರಿಸಿ, ಆತನನ್ನು ಒಂದು ಮಣೆಯಮೇಲೆ ಕುಳ್ಳಿರಿಸಿದನು. ಅನಂತರ ಆತನ ಪಾದವನ್ನು ತೊಳೆದು, ಆ ಪಾದ ತೀರ್ಥವನ್ನು ತನ್ನ ತಲೆಯಲ್ಲಿ ಧರಿಸಿದ ಮೇಲೆ ‘ಸ್ವಾಮಿ, ತಪಸ್ಸೇ ಆಕಾರವನ್ನು ತಾಳಿ ಬಂದಂತಿರುವ ನಿಮ್ಮನ್ನು ಕಂಡು ಧನ್ಯನಾದೆ, ನನ್ನ ವಂಶ ಪವಿತ್ರವಾಯಿತು, ನನ್ನ ಯಾಗ ಸಫಲವಾದಂತಾಯಿತು. ನಿಮ್ಮನ್ನು ನೋಡಿದರೆ ಯಾವುದೋ ಯಾಚನೆಗಾಗಿ ನನ್ನ ಬಳಿಗೆ ಬಂದಂತೆ ಕಾಣಿಸುತ್ತದೆ. ತಮ್ಮ ಬಯಕೆ ಯಾವುದಿದ್ದರೂ ಅದನ್ನು ಪೂರೈಸಲು ನಾನು ಸಿದ್ಧನಾಗಿದ್ದೇನೆ. ಅದನ್ನು ಸ್ವೀಕರಿಸಿ ನನ್ನನ್ನು ಉದ್ಧರಿಸಬೇಕು’ ಎಂದು ಕೇಳಿಕೊಂಡ.

ಬಲೀಂದ್ರನ ಮಾತುಗಳನ್ನು ಕೇಳಿ ವಾಮನ ಮೂರ್ತಿಗೆ ಬಹು ಸಂತೋಷವಾಯಿತು. ಆತನು ‘ಅಯ್ಯಾ ಬಲೀಂದ್ರ, ಶುಕ್ರನಂತಹ ಗುರು, ಪ್ರಹ್ಲಾದನಂತಹ ಹಿರಿಯ–ಇವರ ಸಹವಾಸದಲ್ಲಿರುವ ನೀನು ಧರ್ಮಪರನಾಗಿರುವುದು ಸಹಜವಾಗಿಯೇ ಇದೆ. ಕಲಿತನ, ದಾನಗುಣಗಳು ನಿಮ್ಮ ವಂಶದ ಹುಟ್ಟುಗುಣ. ಇದನ್ನು ತಿಳಿದೇ ನಾನು ನಿನ್ನಲ್ಲಿಗೆ ಯಾಚನೆಗಾಗಿ ಬಂದಿದ್ದೇನೆ. ನನಗೆ ಅತಿಯಾಸೆಯೇನೂ ಇಲ್ಲ. ನನ್ನ ಹೆಜ್ಜೆಯಲ್ಲಿ ಮೂರು ಹೆಜ್ಜೆಗಳಾಗುವಷ್ಟು ಭೂಮಿಯನ್ನು ನನಗೆ ದಯಪಾಲಿಸು. ಅಷ್ಟು ಸಾಕು’ ಎಂದನು. ಇದನ್ನು ಕೇಳಿದ ಬಲಿಯು ‘ಅಯ್ಯೋ ಬ್ರಾಹ್ಮಣ, ಎಂತಹ ಸಣ್ಣ ಬೇಡಿಕೆ ನಿನ್ನದು! ಮೂರು ಲೋಕಕ್ಕೂ ಒಡೆಯನಾದ ನನ್ನಲ್ಲಿ, ಹಲವು ದ್ವೀಪಗಳನ್ನೇ ಕೊಡೆಂದರೂ ಸಲ್ಲುತ್ತಿತ್ತು. ನನ್ನಲ್ಲಿ ದಾನವನ್ನು ಪಡೆದವನು ಮತ್ತೊಬ್ಬನ ಬಳಿ ಕೈಯೊಡ್ಡುವುದೆ ಬೇಡ. ಆದ್ದರಿಂದ ನಿನಗೆ ಫಲವತ್ತಾದ ಒಂದು ದೊಡ್ಡ ಜಮೀನನ್ನು ಕೊಡುತ್ತೇನೆ ತೆಗೆದುಕೊ’ ಎಂದನು. ಆದರೆ ವಾಮನನು ಅದಕ್ಕೆ ಒಪ್ಪಲಿಲ್ಲ. ‘ಅಯ್ಯಾ ಆಸೆಗೆ ಕೊನೆಯೆಲ್ಲಿ? ನನಗೆ ದುರಾಸೆ ಯಿಲ್ಲ. ನಾನು ಕೇಳಿದಷ್ಟು ಕೊಟ್ಟರೆ ಸಾಕು, ನಾನು ತೃಪ್ತ’ ಎಂದ. ಆಗ ಬಲೀಂದ್ರನು ‘ಹಾಗೆಯೆ ಆಗಲಿ. ನಿನಗೆ ಬೇಕಾದಷ್ಟನ್ನೆ ತೆಗೆದುಕೊ’ ಎಂದು ಹೇಳಿ, ದಾನಧಾರೆಯನ್ನು ಕೊಡುವುದಕ್ಕಾಗಿ ಕೈಗೆ ಪಾತ್ರೆಯನ್ನೆತ್ತಿಕೊಂಡನು. ಇದನ್ನು ಕಂಡು ಬಳಿಯಲ್ಲಿಯೇ ಇದ್ದ ಶುಕ್ರಾಚಾರ್ಯರು “ಮಹಾರಾಜ, ನೀನು ಮೋಸಹೋಗುತ್ತಿರುವೆ. ಈ ವಾಮನ ವಿಷ್ಣುವಿನ ಅವತಾರ; ದೇವತೆಗಳ ರಕ್ಷಣೆಗಾಗಿ ಅದಿತಿಯ ಮಗನಾಗಿ ಹುಟ್ಟಿಬಂದಿದ್ದಾನೆ. ಇವನಿಂದ ನಿನಗೆ ಕೇಡು ತಪ್ಪದು. ಆದ್ದರಿಂದ ನಿನ್ನ ದಾನಕಾರ್ಯವನ್ನು ನಿಲ್ಲಿಸು. ‘ಕೊಟ್ಟಮಾತು ತಪ್ಪುವುದೆಂತು?’ ಎಂಬ ಶಂಕೆ ಕೂಡ ಬೇಡ. ಪ್ರಾಣರಕ್ಷಣೆಗಾಗಿ ಸುಳ್ಳು ಹೇಳುವುದು ಅಧರ್ಮವಾಗದು” ಎಂದು ಬೋಧಿಸಿದನು. ಆದರೆ ಬಲಿರಾಜನಿಗೆ ಆತನ ಬೋಧೆ ಹಿಡಿಸ ಲಿಲ್ಲ. ‘ಸುಳ್ಳಿಗಿಂತ ದೊಡ್ಡ ಪಾಪವಿಲ್ಲ’ ಎಂದು ಹೇಳಿ, ಆತನು ವಾಮನನಿಗೆ ದಾನಧಾರೆ ಯನ್ನು ಎರೆದನು. ಆತನ ಮಡದಿಯಾದ ವಿಂಧ್ಯಾವಳಿಯು ಬಂಗಾರದ ಕಲಶದಿಂದ ನೀರೆರೆಯಲು, ಆತನು ವಾಮನನ ಪಾದಗಳನ್ನು ತೊಳೆದು ಆತನಿಗೆ ಮತ್ತೊಮ್ಮೆ ನಮಸ್ಕರಿಸಿದನು.

ದಾನಧಾರೆಯು ಕೈಯಲ್ಲಿ ಬೀಳುತ್ತಲೆ ವಾಮನನು ಬೆಳೆಯುವುದಕ್ಕೆ ಪ್ರಾರಂಭಿಸಿದನು. ಆ ಬೆಳವಣಿಗೆಗೆ ಕೊನೆಯೇ ಇಲ್ಲ. ಪಾತಾಳ ಆತನ ಅಂಗಾಲಿನಲ್ಲಿ, ಭೂಮಿ ಹೆಜ್ಜೆಯಲ್ಲಿ, ವಾಯುಮಂಡಲ ತೊಡೆಯಲ್ಲಿ, ನಾಭಿಯಲ್ಲಿ ಅಂತರಿಕ್ಷ, ಎದೆಯಲ್ಲಿ ನಕ್ಷತ್ರಮಂಡಲ– ಶಿರಸ್ಸು ಎಲ್ಲಿಯೋ–ಸಕಲ ಲೋಕಗಳನ್ನೂ ಆವರಿಸಿಕೊಂಡು ಬೆಳೆದ ಆ ಭಯಂಕರಾಕೃತಿ ಯನ್ನು ಕಂಡು ದೇವದಾನವರು ಭಯದಿಂದ ಭ್ರಾಂತರಾಗಿಹೋದರು. ಇಂತು ಬೃಹದಾ ಕಾರವನ್ನು ತಾಳಿ, ಒಂದು ಪಾದದಿಂದ ಬಲಿಯು ಆಳುತ್ತಿದ್ದ ಸಕಲ ಭೂಮಂಡಲವನ್ನೂ ಮತ್ತೊಂದರಿಂದ ನಭೋಮಂಡಲವನ್ನೂ ಅಳೆದು, ಮೂರನೆಯ ಹೆಜ್ಜೆಯನ್ನಿಡಲು ತಾವೆಲ್ಲಿಯೆಂದು ಆತನು ಬಲಿಯನ್ನು ಕೇಳಿದನು. ಆತನು ಮುಂದೋರದೆ ಸುಮ್ಮನಿರಲು, ತ್ರಿವಿಕ್ರಮನಾದ ವಾಮನನು ‘ಅಯ್ಯಾ, ನೀನು ಕೊಟ್ಟ ಮಾತಿಗೆ ತಪ್ಪಿದರೆ ನರಕಕ್ಕೆ ಹೋಗುವೆ. ಲೋಕೇಶ್ವರನೆಂಬ ಅಹಂಕಾರದಿಂದ ನೀನು ಮಾತು ಕೊಟ್ಟು ಈಗ ತಪ್ಪುವೆಯಾ?’ ಎಂದು ಗರ್ಜಿಸಿದನು. ಹೀಗೆ ಮೋಸದಿಂದ ತನಗೆ ಮಹತ್ತಾದ ಅಪಕಾರ ವನ್ನು ಮಾಡಿದ್ದರೂ ಬಲಿಯು ಸ್ವಲ್ಪವೂ ಅಶಾಂತನಾಗದೆ ‘ಅಯ್ಯಾ, ನೀನು ನನಗೆ ಮೋಸ ಮಾಡಿದ್ದಿ. ಆದರೂ ನನ್ನನ್ನೇ ಸುಳ್ಳನೆಂದು ಹಂಗಿಸುತ್ತಿದ್ದಿ. ನಾನು ಎಂದಿಗೂ ಸುಳ್ಳಾಡು ವುದಿಲ್ಲ. ಇಗೋ ನೋಡು, ಈ ನನ್ನ ತಲೆಯ ಮೇಲೆ ನಿನ್ನ ಮೂರನೆಯ ಹೆಜ್ಜೆಯನ್ನಿಡು. ನಾನು ರಾಜ್ಯಕೋಶಗಳು ಹೋದವೆಂದಾಗಲಿ, ನನಗೆ ಅನ್ಯಾಯವಾಯಿತೆಂದಾಗಲಿ ವ್ಯಥೆಪಡುವುದಿಲ್ಲ; ಸುಳ್ಳನೆಂದು ಕರೆದರೆ ಮಾತ್ರ ಸಂಕಟವಾಗುತ್ತದೆ. ನಾನು ಮಹಾಭಕ್ತ ನಾದ ಪ್ರಹ್ಲಾದನ ಮೊಮ್ಮಗ. ನಾನೂ ಆತನಂತೆ ದೈವಭಕ್ತ. ದೈವಕಾರ್ಯದಲ್ಲಿ, ಧರ್ಮ ಕಾರ್ಯದಲ್ಲಿ ಜೀವವನ್ನು ಬಲಿದಾನ ಮಾಡಲು ನಾನೇನೂ ಹೆದರುವವನಲ್ಲ’ ಎಂದನು.

ಬಲೀಂದ್ರನು ಮಾತು ಮುಗಿಸುವ ವೇಳೆಗೆ ಸರಿಯಾಗಿ ಪ್ರಹ್ಲಾದನೇ ಅಲ್ಲಿಗೆ ಬಂದನು. ಆತನು ಮೊಮ್ಮಗನಿಗೆ ಒದಗಿದ ದುಃಸ್ಥಿತಿಯನ್ನು ಕಂಡು ಮರುಗಲಿಲ್ಲ. ಆತನು ತ್ರಿವಿ ಕ್ರಮರೂಪಿಯಾದ ಶ್ರೀಹರಿಯನ್ನು ಭಕ್ತಿಯಿಂದ ನಮಸ್ಕರಿಸಿ ‘ದೇವದೇವ, ನೀನೇ ಕೊಟ್ಟ ತ್ರಿಲೋಕಸಾಮ್ರಾಜ್ಯವನ್ನು ನೀನೇ ಕಿತ್ತುಕೊಳ್ಳುತ್ತಿರುವೆ. ಇದಕ್ಕಾಗಿ ಅಳುವುದೇಕೆ? ಸಂಪತ್ತು ಮನುಷ್ಯನ ಅಹಂಕಾರವನ್ನು ಹೆಚ್ಚಿಸಿ, ಅವನನ್ನು ಅಧೋಗತಿಗೆ ಎಳೆಯುತ್ತದೆ. ಅಂತಹ ಸಂಪತ್ತನ್ನು ನೀನಾಗಿಯೇ ನಿವಾರಿಸುತ್ತಿರುವುದು ಒಂದು ದೊಡ್ಡ ಅನುಗ್ರಹ’ ಎಂದನು. ವಿಂಧ್ಯಾವಳಿಯೂ ಶ್ರೀಹರಿಗೆ ಅಡ್ಡಬಿದ್ದು ‘ಸ್ವಾಮಿ, ನಿಮ್ಮ ಪಾದವನ್ನು ತೊಳೆದು ಮೂರು ಲೋಕಗಳನ್ನೂ ನಿಮಗೆ ಅರ್ಪಿಸಿರುವೆವಲ್ಲವೆ? ಅಂತಹ ನಮಗೆ ದುಃಖ ಹೇಗೆ ಬಂದೀತು? ನಮ್ಮನ್ನು ಉದ್ಧರಿಸಿ ಕಾಪಾಡು’ ಎಂದು ಹೇಳಿದಳು. ಆಗ ಶ್ರೀಹರಿಯು ಗುಡಿಗಿನಂತಹ ಗಂಭೀರ ಧ್ವನಿಯಿಂದ ಅಲ್ಲಿದ್ದವರನ್ನೆಲ್ಲ ಕುರಿತು ‘ಅಯ್ಯಾ, ನಾನು ಯಾರನ್ನು ಉದ್ಧಾರಮಾಡಬೇಕೆಂದಿರುವೆನೊ ಅವನ ಸರ್ವಸ್ವವನ್ನೂ ಮೊದಲು ಕಿತ್ತು ಕೊಳ್ಳುತ್ತೇನೆ. ನನ್ನ ಅನುಗ್ರಹಕ್ಕೆ ಪಾತ್ರನಾದನೆಂದರೆ ಅವನಲ್ಲಿ ಕುಲ, ರೂಪ, ವಿದ್ಯೆ ಅಧಿಕಾರ, ಧನ ಮೊದಲಾದವುಗಳ ಮದವೊಂದೂ ಅಂಕುರಿಸದು. ಈ ಬಲೀಂದ್ರನು ನಾನು ಒಡ್ಡಿದ ಪರೀಕ್ಷೆಯಲ್ಲಿ ಗೆದ್ದು ನನ್ನ ಪರಮಭಕ್ತನಾಗಿದ್ದಾನೆ. ಯಾರೂ ಗೆಲ್ಲಲಾಗದ ಮಾಯೆಯನ್ನು ಈತ ಗೆದ್ದು ಮಹಾನುಭಾವನಾಗಿದ್ದಾನೆ. ಸತ್ಯವನ್ನು ವ್ರತವಾಗಿ ಪಡೆ ದಿರುವ ಈತನಿಗೆ ನಾನು ಸಾಲೋಕ್ಯಪದವಿಯನ್ನು ಕೊಟ್ಟಿದ್ದೇನೆ. ಮುಂದೆ ಸಾವರ್ಣಿ ಮನ್ವಂತರದಲ್ಲಿ ಈತನು ದೇವೇಂದ್ರನಾಗುತ್ತಾನೆ. ಅಲ್ಲಿಯವರೆಗೆ ಈತನು ಸುತಲ ಲೋಕದ ಪ್ರಭುವಾಗಿರುವನು’ ಎಂದು ಹೇಳಿದನು. ಅನಂತರ ಆತನು ಬಲಿಯಕಡೆ ನೋಡುತ್ತಾ ‘ಅಯ್ಯಾ, ಬಲೀಂದ್ರ, ನೀನು ಇಂದಿನಿಂದ ಸುತಲ ಲೋಕದ ಸ್ವಾಮಿ, ನಿನ್ನ ಬಂಧು ಬಾಂಧವರೊಡನೆ ನೀನು ಅಲ್ಲಿ ನೆಲಸು. ನಾನು ಅಲ್ಲಿ ನಿನಗೆ ಸದಾ ದರ್ಶನವನ್ನು ಕೊಡುತ್ತಿರುವೆನು’ ಎಂದು ಹೇಳಿದನು. ಬಲಿಯು ಶ್ರೀಹರಿಗೆ ನಮಸ್ಕರಿಸಿ, ಆತನ ಅಪ್ಪಣೆ ಯಂತೆ ಸುತಲಲೋಕಕ್ಕೆ ಹೊರಟುಹೋದನು. ದೇವೇಂದ್ರನು ಮತ್ತೆ ಮೂರುಲೋಕಗಳ ಸಾಮ್ರಾಜ್ಯವನ್ನು ಪಡೆದು ಸುಖವಾಗಿದ್ದನು.

