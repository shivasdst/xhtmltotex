
\chapter{೮೩. ಗೋಪಿಯರ ಕಣ್ಣು ತೆರೆಯಿತು}

ನಂದಗೋಕುಲದಿಂದ ಹೊರಟುಬಂದ ಶ್ರೀಕೃಷ್ಣ ಬಹುಕಾಲದ ಮೇಲೆ ಮತ್ತೆ ಅವ ರನ್ನು ಕಾಣುವ ಒಂದು ಸನ್ನಿವೇಶ ಬಂತು. ಒಮ್ಮೆ, ಪ್ರಪಂಚಕ್ಕೆ ಪ್ರಳಯವನ್ನು ಸೂಚಿಸು ವಂತಹ ಭಯಂಕರವಾದ ಒಂದು ಸೂರ್ಯಗ್ರಹಣ ಬಂತು. ಅಂದು ದೇಶದೇಶಗಳ ಜನರೆಲ್ಲರೂ ತರ್ಪಣವೆ ಮೊದಲಾದ ಪುಣ್ಯ ಕರ್ಮಗಳನ್ನು ಆಚರಿಸುವುದಕ್ಕಾಗಿ, ಕುರು ಕ್ಷೇತ್ರದಲ್ಲಿದ್ದ ಸ್ಯಮಂತಪಂಚಕವೆಂಬ ಪುಣ್ಯತೀರ್ಥಕ್ಕೆ ಬಂದರು. ಹಿಂದೆ ಪರಶು ರಾಮನು ಜಗತ್ತಿನ ಕ್ಷತ್ರಿಯರನ್ನೆಲ್ಲ ಕೊಂದು, ಅವರ ನೆತ್ತರಿನಿಂದ ಐದು ಮಡುಗಳನ್ನು ತುಂಬಿ, ತನ್ನ ಪಾಪನಿವಾರಣೆಗಾಗಿ ಮಹಾಯಾಗವೊಂದನ್ನು ಮಾಡಿದ ಪುಣ್ಯಸ್ಥಳವದು. ಪರ್ವಕಾಲದಲ್ಲಿ ಅಲ್ಲಿ ಸ್ನಾನಮಾಡಿದರೆ ಸಕಲ ಪಾಪಗಳೂ ಪರಿಹಾರವಾಗುತ್ತವೆ. ಆದ್ದ ರಿಂದಲೆ ಬಲರಾಮಕೃಷ್ಣರು ಕೂಡ ತಮ್ಮ ಬಂಧುಬಾಂಧವರನ್ನೆಲ್ಲ ಕರೆದುಕೊಂಡು ಆ ದಿನ ಅಲ್ಲಿಗೆ ಹೋದರು. ಹಸ್ತಿನಾವತಿಯಿಂದ ಭೀಷ್ಮ, ದ್ರೋಣ, ಧೃತರಾಷ್ಟ್ರ, ದುರ್ಯೋ ಧನ ಮೊದಲಾದವರೆಲ್ಲ ತಮ್ಮ ಬಂಧುಬಾಂಧವರೊಡನೆ ಅಲ್ಲಿಗೆ ಬಂದಿದ್ದರು. ತನ್ನ ಮಕ್ಕಳ ಜೊತೆಯಲ್ಲಿ ಕುಂತಿಯೂ ಬಂದಿದ್ದಳು. ನಂದಗೋಕುಲದಿಂದ ನಂದಯಶೋದೆ ಯರೊಡನೆ ಅಲ್ಲಿನ ಗೋಪಗೋಪಿಯರೆಲ್ಲ ಹೊರಟು ಬಂದಿದ್ದರು. ಇನ್ನೂ ಅನೇಕ ದೇಶದ ರಾಜರೂ ಅಧಿಕಾರಿಗಳೂ ಸಾಮಾನ್ಯಜನರೂ ಲೆಕ್ಕವಿಲ್ಲದಷ್ಟು ಜನ ಅಲ್ಲಿ ಬಂದು ನೆರೆದಿದ್ದರು. ಗ್ರಹಣ ಹಿಡಿಯುತ್ತಲೆ ಅಲ್ಲಿ ನೆರೆದವರೆಲ್ಲ ಪುಣ್ಯತೀರ್ಥದಲ್ಲಿ ಮಿಂದು, ಪಿತೃಗಳಿಗೆ ತರ್ಪಣವಿತ್ತು, ಬೇಕಾದಷ್ಟು ದಾನಧರ್ಮಗಳನ್ನು ಮಾಡಿದರು. ಗ್ರಹಣ ಬಿಟ್ಟ ಮೇಲೆ ಮತ್ತೆ ಸ್ನಾನ, ಸಂಧ್ಯೆ, ಜಪ, ತಪಗಳನ್ನು ಮುಗಿಸಿ, ಊಟಮಾಡಿ, ಅಲ್ಲಲ್ಲೆ ಮರಗಳ ಕೆಳಗೆ ವಿಶ್ರಾಂತಿಯನ್ನು ಪಡೆದರು. 

ಎಲ್ಲ ಕಡೆಯಿಂದಲೂ ಜನ ಬಂದು ಸೇರಿದ್ದುದರಿಂದ ಅವರಿಗೆ ತಮ್ಮ ಬಂಧು ಗಳನ್ನೂ, ಮಿತ್ರರನ್ನೂ ಪರಸ್ಪರ ನೋಡುವುದಕ್ಕೆ ಒಳ್ಳೆಯ ಅವಕಾಶ ಸಿಕ್ಕಿತು. ಕುಂತೀದೇವಿ ತನ್ನ ತೌರಿನವರನ್ನು ಕಂಡು ಎಷ್ಟೋ ವರ್ಷಗಳಾಗಿತ್ತು. ಆಕೆ ಅವರನ್ನು ಹುಡುಕಿಕೊಂಡು ಹೊರಟು, ವಸುದೇವನನ್ನು ಕಾಣುತ್ತಲೆ ಆನಂದಬಾಷ್ಪಗಳನ್ನು ಸುರಿಸುತ್ತಾ, ಅವನಿಗೆ ಅಡ್ಡಬಿದ್ದು ‘ಅಣ್ಣ, ನೀನು ನನ್ನನ್ನು ಮರೆತೇಬಿಟ್ಟೆಯಲ್ಲವೆ? ನಾನೂ ನನ್ನ ಮಕ್ಕಳೂ ಪಡಬಾರದ ಕಷ್ಟಗಳನ್ನೆಲ್ಲ ಪಟ್ಟೆವು. ಒಂದು ಸಾರಿಯಾದರೂ ನೀನು ನಮ್ಮನ್ನು ಕಣ್ಣೆತ್ತಿ ನೋಡಿದೆಯಾ? ಅಪ್ಪ, ನಿನ್ನದೇನು ತಪ್ಪು? ದೈವವೇ ನಮಗೆ ಪ್ರತಿಕೂಲವಾಗಿರುವಾಗ ಬಂಧುಗಳು ಉದಾಸೀನರಾಗುವುದು ಏನಾಶ್ಚರ್ಯ?’ ಎಂದು ಆಕ್ಷೇಪಿಸಿದಳು. ತಂಗಿಯ ಮಾತುಗಳನ್ನು ಕೇಳಿ ವಸುದೇವನಿಗೂ ಕಣ್ಣಲ್ಲಿ ನೀರು ಬಂತು. ಆತ ‘ಅಮ್ಮ, ನಿನ್ನ ದುಃಖ ನನಗೆ ಅರ್ಥವಾಗುತ್ತದೆ. ಆದರೆ ನಾನು ತಾನೆ ನಿನಗೆ ಏನು ಮಾಡುವಹಾಗಿದ್ದೆ? ಕಂಸನ ಭಯದಿಂದ ಜೀವವನ್ನು ಕೈಲಿ ಹಿಡಿದುಕೊಂಡು, ಎಲ್ಲಿಯೋ ಅಜ್ಞಾತರಂತಿದ್ದ ನಾವು ಮತ್ತೊಬ್ಬರಿಗೆ ಸಹಾಯ ಮಾಡುವ ಸ್ಥಿತಿಯಲ್ಲಿದ್ದೆವೆ? ಏನೊ ಈಗ ಈ ಮಗ ಶ್ರೀಕೃಷ್ಣನ ದಯೆಯಿಂದ ನಾವೂ ಒಬ್ಬ ಮನುಷ್ಯರೆಂದು ತಲೆಯೆತ್ತಿಕೊಂಡು ತಿರುಗಾಡುತ್ತಿದ್ದೇವೆ. ನೋಡಮ್ಮ, ನಾವೆಲ್ಲ ದೈವದ ಕೈಗೊಂಬೆಗಳು, ಅವನು ಕುಣಿಸಿದಂತೆ ಕುಣಿದಾಡಬೇಕು’ ಎಂದು ಹೇಳಿ, ಆಕೆಯನ್ನು ಸಮಾಧಾನಮಾಡಿದ.

ಯಾತ್ರೆಗಾಗಿ ಬಂದಿದ್ದವರೆಲ್ಲ ತಮ್ಮ ಬಂಧುಗಳನ್ನು ಕಾಣುವುದರ ಜೊತೆಗೆ ಶ್ರೀಕೃಷ್ಣ ನನ್ನು ಕಾಣಬೇಕೆಂದು ತವಕಪಡುತ್ತಿದ್ದರು. ಪಂಚಪಾಂಡವರು ದ್ರೌಪದಿಯೊಡನೆ ಬಂದು, ಆತನ ದರ್ಶನಮಾಡಿ ಆನಂದಿಸಿದರು. ಕೌರವರು ಕೂಡ ತಮ್ಮ ಮಡದಿ ಮಕ್ಕ ಳೊಡನೆ ಬಂದು, ಆತನನ್ನು ಕಂಡು ಗೌರವಿಸಿದರು. ಅನೇಕ ರಾಜರು ಧರ್ಮರಾಯನನ್ನು ಬೇಡಿ, ಆತನ ಸಹಾಯದಿಂದ ಶ್ರೀಕೃಷ್ಣನ ಸಂದರ್ಶನ ಪರಿಚಯಗಳನ್ನು ಮಾಡಿ ಕೊಂಡರು. ಅವರೆಲ್ಲ ಉಗ್ರಸೇನ ಮಹಾರಾಜನನ್ನು ಕಂಡು ‘ಅಯ್ಯಾ ನಿನ್ನ ಪುಣ್ಯವೇ ಪುಣ್ಯ; ಯೋಗಿಗಳಿಗೂ ಅಗೋಚರನಾದ ಶ್ರೀಕೃಷ್ಣ ನಿನ್ನ ವಶವರ್ತಿ’ ಎಂದು ಆತನನ್ನು ಕೊಂಡಾಡಿದರು. ತಮ್ಮನ್ನು ಕಾಣಲು ಬಂದವರನ್ನೆಲ್ಲ ಬಲರಾಮಕೃಷ್ಣರು ಅತ್ಯಂತ ವಿನಯದಿಂದ ಕಂಡು, ತಮ್ಮ ನಗುಮುಖಗಳಿಂದಲೂ ಮೃದುನುಡಿಗಳಿಂದಲೂ ಸತ್ಕರಿಸಿ ಗೌರವಿಸಿದರು. ನಂದಗೋಪ ಯಶೋದೆಯರೂ ಅವರ ಪರಿವಾರವೂ ಈ ಯಾತ್ರೆಯನ್ನು ಕೈಗೊಂಡಿದ್ದುದು ಗ್ರಹಣಕ್ಕಿಂತ ಹೆಚ್ಚಾಗಿ ಬಲರಾಮ ಕೃಷ್ಣರನ್ನು ಕಾಣಬೇಕೆಂಬ ಕಾರಣ ದಿಂದ. ಅವರು ಯಾದವರು ಎಲ್ಲಿರುವರೆಂದು ಹುಡುಕುತ್ತಾ ಬರಲು, ದೂರದಿಂದ ಅವ ರನ್ನು ಕಂಡು ವಸುದೇವನು ಓಡಿಹೋಗಿ, ನಂದನನ್ನು ಆಲಿಂಗಿಸಿಕೊಂಡು, ಆತನನ್ನೂ ಯಶೋದೆಯನ್ನೂ ತಾವಿದ್ದ ಸ್ಥಳಕ್ಕೆ ಕರೆತಂದನು. ಈ ಸಾಕುತಂದೆತಾಯಿಯರನ್ನು ಕಂಡ ಬಲರಾಮಕೃಷ್ಣರು ಆನಂದದಿಂದ ಕಣ್ಣೀರುತುಂಬಿ, ಗಂಟಲು ಕಟ್ಟಲು, ಮೂಕರಂತೆ ಮೌನವಾಗಿ ಅವರಿಗೆ ನಮಸ್ಕರಿಸಿ ನಿಂತರು. ನಂದ ಯಶೋದೆಯರು ಅವರನ್ನು ಬಾಚಿ ತಬ್ಬಿಕೊಂಡು ತಮ್ಮ ತೊಡೆಗಳ ಮೇಲೆ ಕೂಡಿಸಿಕೊಂಡರು. ಇದನ್ನು ಕಂಡು ರೋಹಿಣಿ ಮತ್ತು ದೇವಕಿಯರು ಯಶೋದೆಯೊಡನೆ ‘ಅಮ್ಮ, ನೀನೂ ನಿನ್ನ ಗಂಡನೂ ಮಾಡಿದ ಉಪಕಾರವನ್ನು ನಾವು ಜನ್ಮಗಳೆತ್ತಿ ತೀರಿಸಿದರೂ ಸವೆಯದು. ಆ ಉಪಕಾರಕ್ಕೆ ದೇವೇಂದ್ರನ ಐಶ್ವರ್ಯ ಕೂಡ ಸಮವಾಗುವುದಿಲ್ಲ. ತಾಯಿ, ಎಳೆಯ ಕಂದಮ್ಮಗಳಾಗಿದ್ದ ನಮ್ಮ ಮಕ್ಕಳನ್ನು ನಮಗಿಂತ ಹೆಚ್ಚು ಪ್ರೀತಿಯಿಂದ ನೀವು ಸಾಕಿ ಸಲಹಿದಿರಿ. ಅವರಿಗೆ ನಾಮಕರಣ ಮಾಡಿದಿರಿ, ಬೆಳೆಸಿದಿರಿ, ಮುಂದಕ್ಕೆ ತಂದಿರಿ. ಅವರು ಇವತ್ತು ಇಷ್ಟು ದೊಡ್ಡವರೆನಿಸಿಕೊಂಡುದಕ್ಕೆ ನೀವೆ ಕಾರಣ. ರೆಪ್ಪೆಗಳು ಕಣ್ಣನ್ನು ಕಾಯುವ ಹಾಗೆ ಅವರನ್ನು ಜಾಗರೂಕತೆಯಿಂದ ರಕ್ಷಿಸಿ, ಅವರನ್ನು ಉದ್ಧಾರಮಾಡಿದಿರಿ. ಅಮ್ಮ, ಸಜ್ಜನರಾದ ನಿಮ್ಮನ್ನು ಇದಿರಿಗೆ ಹೊಗಳಬಾರದು. ನಿಮ್ಮ ಮನಸ್ಸು ಚೊಕ್ಕ ಬಂಗಾರವಮ್ಮ’ ಎಂದರು.

ನಂದನ ಜೊತೆಗೆ ಬಂದಿದ್ದ ಗೋಪಿಯರು ಬಹುಕಾಲದ ಮೇಲೆ ಕಾಣಿಸಿದ ಶ್ರೀಕೃಷ್ಣ ನನ್ನು ನೆಟ್ಟದಿಟ್ಟಿಯಿಂದ ನೋಡುವುದಕ್ಕೆ ಅಡ್ಡಿಯಾಗಿದ್ದ ರೆಪ್ಪೆಗಳನ್ನು ಬಯ್ಯುತ್ತಾ, ಆತನ ಮೂರ್ತಿಯನ್ನು ತಮ್ಮ ಹೃದಯದಲ್ಲಿ ಸೆರೆಹಿಡಿದು ಅಲ್ಲಿಯೆ ಆಲಿಂಗಿಸಿಕೊಂಡರು. ಅವರನ್ನು ಕಾಣುತ್ತಲೆ ಶ್ರೀಕೃಷ್ಣನು ಮುಗುಳ್ನಗೆಯ ಅಮೃತವನ್ನು ತುಟಿಗಳಲ್ಲಿ ತುಂಬಿ ತೊಟ್ಟಿಡಿಸುತ್ತಾ ‘ಎಲೆ ಗೆಳತಿಯರೆ, ನಿಮ್ಮನ್ನು ಕಂಡು ಎಷ್ಟು ದಿನಗಳಾಗಿ ಹೋದವು! ಬೇಗ ಬರುತ್ತೇನೆಂದು ನಿಮಗೆ ಮಾತು ಕೊಟ್ಟೆನಾದರೂ ನಾನು ಗೋಕುಲಕ್ಕೆ ಬರಲಾಗಲೇ ಇಲ್ಲ. ನೀವು “ಇವನೆಂತಹ ಕೃತಘ್ನ!” ಎಂದುಕೊಂಡಿರೊ ಏನೋ! ಏನು ಮಾಡುವುದು? ನನಗೆ ಸಹಸ್ರ ತಾಪತ್ರಯಗಳು ಗಂಟುಬಿದ್ದವು. ಸಂಕಟದಲ್ಲಿ ಸಿಕ್ಕಿಬಿದ್ದಿದ್ದ ನನ್ನ ತಾಯಿ ತಂದೆಗಳನ್ನು, ಬಂಧುಬಳಗವನ್ನು ಕಂಸನಿಂದ ಬಿಡಿಸಬೇಕಾಯಿತು; ವಿಶ್ವಕುಟುಂಬಿ ಯಾದ ನನಗೆ ಒಬ್ಬರೆ, ಇಬ್ಬರೆ? ನೂರಾರು ಜನ ಶತ್ರುಗಳು! ಅವರನ್ನೆಲ್ಲ ಹೊಡೆದುಹಾಕ ಬೇಕಾಯಿತು. ಲೋಕದಲ್ಲಿ ಜನ ಪರಸ್ಪರ ಸೇರುವುದೂ ಅಗಲುವುದೂ ದೈವಸಂಕಲ್ಪ. ಗಾಳಿಗೆ ಸಿಕ್ಕ ಮೋಡಗಳಿದ್ದಂತೆ ನಮ್ಮ ಬಾಳು. ಗಾಳಿಯ ವೇಗಕ್ಕೆ ಒಮ್ಮೆ ಸೇರುತ್ತವೆ, ಮತ್ತೊಮ್ಮೆ ಅಗಲುತ್ತವೆ, ಅವು; ನಾವೂ ಹಾಗೆಯೆ. ದೈವಯೋಗದಿಂದ ಕಲೆಯುತ್ತೇವೆ, ಅಗಲುತ್ತೇವೆ, ಇದರಲ್ಲಿ ನಮ್ಮ ಪ್ರಯತ್ನವೇನೂ ನಡೆಯುವುದಿಲ್ಲ. ಆದರೆ ನಮ್ಮ ಅಗಲಿಕೆಯಿಂದಾಗುವ ದುಃಖವನ್ನು ತಡೆಯುವುದಕ್ಕೆ ಒಂದು ಉಪಾಯವಿದೆ. ನೀವು ನನ್ನಲ್ಲಿ ದೃಢವಾದ ಭಕ್ತಿಯನ್ನಿಡಿ; ಇದರಿಂದ ನನ್ನ ನಿಮ್ಮ ಸಂಬಂಧ ಶಾಶ್ವತವಾಗುತ್ತದೆ. ನಾನು ಯಾರೆಂದು ತಿಳಿದಿರುವಿರಿ? ಎಲ್ಲ ಜೀವಗಳ ಒಳಗೂ ಹೊರಗೂ ನಾನು ವ್ಯಾಪಿಸಿ ದ್ದೇನೆ ಅಷ್ಟೇ ಅಲ್ಲ, ಇಡೀ ಬ್ರಹ್ಮಾಂಡದ ಒಳಗೂ ಹೊರಗೂ ವ್ಯಾಪಿಸಿದ್ದೇನೆ. ಆದ್ದರಿಂದ ಈ ಜಗತ್ತಿನ ಚೇತನಾಚೇತನ ವಸ್ತುಗಳೆಲ್ಲವೂ ನಾನೆ. ಹೀಗೆ ನಾನೇ ಸರ್ವವ್ಯಾಪಿಯಾಗಿ ದ್ದರೂ, ಸರ್ವಾತ್ಮನಾಗಿದ್ದರೂ, ಅವುಗಳಿಂದ ಬೇರೆಯಾಗಿಯೂ ಅವುಗಳನ್ನೆಲ್ಲ ನೋಡು ವವನಾಗಿಯೂ ಇದ್ದೇನೆ. ಈ ತತ್ವವನ್ನು ನೀವು ಚಿಂತನೆಯಿಂದ ಅರ್ಥಮಾಡಿಕೊಳ್ಳಿ’ ಎಂದನು.

ಶ್ರೀಕೃಷ್ಣನ ಉಪದೇಶದಿಂದ ಗೋಪಿಯರ ಕಣ್ಣು ತೆರೆಯಿತು. ಅವರು ಶ್ರೀಕೃಷ್ಣನನ್ನು ಕುರಿತು ‘ಹೇ ಪದ್ಮನಾಭ! ಯೋಗಿಗಳಿಗೂ ಅಗೋಚರವಾದ, ಸಂಸಾರಸಾಗರದಲ್ಲಿ ಮುಳುಗಿಹೋಗುವವರನ್ನು ಉದ್ಧರಿಸತಕ್ಕುದಾದ ನಿನ್ನ ಪಾದಾರವಿಂದವು ಸದಾ ಸಾಕ್ಷಾ ತ್ತಾಗಿ ನಮಗೆ ಕಾಣುವಂತಹ ವರವನ್ನು ನಮಗೆ ಅನುಗ್ರಹಿಸು’ ಎಂದು ಬೇಡಿಕೊಂಡರು. ಶ್ರೀಕೃಷ್ಣನು ‘ತಥಾಸ್ತು’ ಎಂದು ಹೇಳಿ ಅವರನ್ನು ಗೋಕುಲಕ್ಕೆ ಬೀಳ್ಕೊಟ್ಟನು.

