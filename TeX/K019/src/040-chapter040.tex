
\chapter{೪೦. ಶ್ರೀ ಕೃಷ್ಣಾವತಾರದ ಹಿನ್ನೆಲೆ}

ರಂತಿದೇವನ ಕಥೆಯನ್ನು ಕೇಳಿ ಪುಳಕಿತನಾದ ಪರೀಕ್ಷಿದ್ರಾಜನು ಪರಮಜ್ಞಾನಿಯಾದ ಶುಕ ಮಹರ್ಷಿಯನ್ನು ಕುರಿತು “ಸ್ವಾಮಿ, ಭಕ್ತರ ಕಥೆ ಭಗವಂತನ ಕಥೆಯಷ್ಟೆ ಪಾವನ ವಾದುದು. ಅವರಿಗೆ ಪರಮಾತ್ಮನಲ್ಲಿ ಅದೆಷ್ಟು ನಂಬಿಕೆ! ನಮ್ಮ ಹಿರಿಯರಾದ ಪಾಂಡ ವರು ಭಗವಂತನಾದ ಶ್ರೀಕೃಷ್ಣ ಪರಮಾತ್ಮನನ್ನು ಎರಡರಿಯದ ಭಕ್ತಿಯಿಂದ ಆರಾಧಿಸಿ ಉದ್ಧಾರವಾದರಂತೆ! ಆ ದೇವಮಾನವ ಮಹಾಭಾರತ ಯುದ್ಧದಲ್ಲಿ ನನ್ನ ತಾತನಿಗೆ ಸಾರಥಿ ಯಾಗಿದ್ದನಂತೆ! ಕೌರವಸೇನೆಯ ಮಹಾ ಸಾಗರದಲ್ಲಿ ತಿಮಿ, ತಿಮಿಂಗಿಲಗಳಂತಿದ್ದ ಭೀಷ್ಮ, ದ್ರೋಣ ಮೊದಲಾದ ಅತಿರಥ ಮಹಾರಥರನ್ನು ನಮ್ಮ ತಾತ ಗೆದ್ದು ದಡ ಸೇರಿ ದುದು ಆ ಪುಣ್ಯಪುರುಷನ ಕೃಪೆಯಿಂದಲೆ–ಎಂದು ಕೇಳಿದ್ದೇನೆ. ಅಶ್ವತ್ಥಾಮನು ಪಾಂಡ ವರ ವಂಶವನ್ನು ನಿರ್ಮೂಲ ಮಾಡಬೇಕೆಂದು ನನ್ನ ತಾಯಿಯ ಗರ್ಭಕ್ಕೆ ಬ್ರಹ್ಮಾಸ್ತ್ರವನ್ನು ಪ್ರಯೋಗಿಸಿದಾಗ, ಆಕೆ ‘ಕೃಷ್ಣ, ಕಾಪಾಡು!’ ಎಂದು ಬೇಡುತ್ತಲೆ, ಸರ್ವಾಂತರ್ಯಾಮಿ ಯಾದ ಆತ ನನ್ನ ತಾಯಿಯ ಗರ್ಭವನ್ನು ಹೊಕ್ಕು ನನ್ನನ್ನು ಕಾಪಾಡಿದನಂತೆ! ಆತನು ಸಾಕ್ಷಾತ್ ಪರಮಾತ್ಮನೆಂದೇ ಹೇಳುವುದನ್ನು ಕೇಳಿದ್ದೇನೆ. ಆತನ ಪುಣ್ಯಕಥೆಯನ್ನು ಆದ್ಯಂತವಾಗಿ ಕೇಳಬೇಕೆಂಬ ಆಸೆ, ನನಗೆ. ಆತ ಎಲ್ಲಿ ಹುಟ್ಟಿದ? ಏಕೆ ಹುಟ್ಟಿದ? ಹೇಗೆ ಹುಟ್ಟಿದ? ಆತನು ಬೆಳೆದುದೆಲ್ಲಿ? ಆತನು ನಡೆಸಿದ ಮಹಾಕಾರ್ಯಗಳಾವುವು? ಎಲ್ಲವನ್ನೂ ವಿವರವಿವರವಾಗಿ ಕೇಳಬೇಕೆಂದು ನನ್ನ ಬಯಕೆ. ನಿಮ್ಮ ಬಾಯಿಂದ ಬರುವ ಕಥಾಮೃತ ನನ್ನ ಹಸಿವು ಬಾಯಾರಿಕೆಗಳನ್ನು ಹಿಂಗಿಸುತ್ತದೆ, ನಿದ್ರಾಲಸ್ಯಗಳನ್ನು ತೊಲಗಿಸುತ್ತದೆ; ನನ್ನ ಚೇತನಕ್ಕೆ ಉತ್ಸಾಹವನ್ನು ತುಂಬಿ, ಆತ್ಮವನ್ನು ಉದ್ಧಾರ ಮಾಡುತ್ತದೆ. ನೀವು ಆತನ ಕಥೆ ಯನ್ನು ನನಗೆ ತಿಳಿಸುವ ಕೃಪೆಮಾಡಿ” ಎಂದು ಬೇಡಿಕೊಂಡನು. 

ಪರೀಕ್ಷಿತನ ಆಸಕ್ತಿಯನ್ನು ಕಂಡು ಬ್ರಹ್ಮಜ್ಞಾನಿಯಾದ ಶುಕಮಹರ್ಷಿಗೆ ಪರಮಾನಂದ ವಾಯಿತು. ಆತನು ಕ್ಷಣಮಾತ್ರ ಕಣ್ಮುಚ್ಚಿ, ಶ್ರೀಕೃಷ್ಣನ ಧ್ಯಾನ ಮಾಡಿದ ಮೇಲೆ ಹೇಳಿದ ‘ಅಯ್ಯಾ, ಶ್ರೀಕೃಷ್ಣನ ಚರಿತ್ರೆ ಅಮೃತಕ್ಕಿಂತಲೂ ರುಚಿಕರ. ಅಮೃತದಿಂದ ನಾಲಿಗೆಗೆ ಮಾತ್ರ ರುಚಿ; ಕೃಷ್ಣಚರಿತಾಮೃತದಿಂದ ನಾಲಗೆ ಕಿವಿಗಳೆರಡೂ ತಣಿಯುತ್ತವೆ. ಅದನ್ನು ಕೇಳಬೇಕೆಂಬ ಬಯಕೆ ನಿನ್ನಲ್ಲಿ ಹುಟ್ಟಿದುದೆ ನಿನ್ನ ಪುಣ್ಯ. ದೇವಗಂಗೆಯಂತೆ ಪರಮ ಪಾವನವಾದ ಆ ಶ್ರೀಕೃಷ್ಣನ ಕಥೆ ಹೇಳುವವರನ್ನೂ ಕೇಳುವವರನ್ನೂ ಜೊತೆಜೊತೆ ಯಾಗಿಯೇ ಉದ್ಧಾರಮಾಡುತ್ತದೆ. ಸಂಸಾರವೆಂಬ ರೋಗಕ್ಕೆ ಸಿದ್ಧೌಷಧದಂತಿರುವ ಆ ಕಥೆಯನ್ನು ಹೇಳುತ್ತೇನೆ ಕೇಳು–ಹೀಗೆಂದು ನುಡಿದ ಶುಕಮಹರ್ಷಿಯು ಶ್ರೀಕೃಷ್ಣನ ದಿವ್ಯ ಚರಿತ್ರೆಯನ್ನು ಆದ್ಯಂತವಾಗಿ ಹೇಳಿದನು.

“ಈಗ ಬಹುಕಾಲದ ಹಿಂದೆ ದುಷ್ಟರಾದ ರಾಕ್ಷಸರು ಕ್ಷತ್ರಿಯರಾಗಿ ಹುಟ್ಟಿ ಲೋಕ ಕಂಟಕರಾಗಿದ್ದರು. ಆ ಪಾಪಿಗಳ ಭಾರವನ್ನು ಹೊರಲಾರದೆ ಭೂದೇವಿಯು ಬ್ರಹ್ಮನ ಬಳಿಗೆ ಹೋಗಿ, ಆತನಲ್ಲಿ ತನ್ನ ದುಃಖವನ್ನು ತೋಡಿಕೊಂಡಳು. ಆಕೆಯನ್ನು ಕಂಡು ಬ್ರಹ್ಮನ ಮನಸ್ಸು ಕರಗಿತು. ಆದರೆ–ಆತನೇನೂ ಮಾಡುವಂತಿರಲಿಲ್ಲ. ಆದ್ದರಿಂದ ಆತನು ಆಕೆಯೊಡನೆ ಪರಶಿವನನ್ನೂ ಇಂದ್ರಾದಿ ದೇವತೆಗಳನ್ನೂ ಕರೆದುಕೊಂಡು ಮಹಾ ವಿಷ್ಣು ನೆಲಸಿರುವ ಕ್ಷೀರಸಮುದ್ರಕ್ಕೆ ಹೋದನು. ಅಲ್ಲಿ ಆತನು ಭಕ್ತಿಯಿಂದ ಭಗವಂತ ನನ್ನು ಧ್ಯಾನಮಾಡುತ್ತಾ ನಿಂತಿರಲು, ಆತನಿಗೆ ಮಾತ್ರ ಕೇಳುವಂತೆ ಒಂದು ಅಶರೀರವಾಣಿ ಯಾಯಿತು. ಅದನ್ನು ಕೇಳಿ ಸಂತೋಷದಿಂದ ಆತನು ತನ್ನ ಜೊತೆಯಲ್ಲಿ ಬಂದವರನ್ನು ಕುರಿತು ‘ಅಯ್ಯಾ, ದೇವತೆಗಳೆ, ನನ್ನ ಧ್ಯಾನಕ್ಕೆ ಮೆಚ್ಚಿದ ಶ್ರೀಹರಿ ಅಭಯವಾಣಿಯನ್ನು ನೀಡಿದ್ದಾನೆ. ಭೂದೇವಿಗೆ ಬಂದಿರುವ ಕಷ್ಟ ನಾವು ಹೇಳುವ ಮೊದಲೆ ಆತನಿಗೆ ಗೊತ್ತಾ ಗಿದೆ. ಅದನ್ನು ಹೋಗಲಾಡಿಸುವ ಉಪಾಯವನ್ನೂ ಆತನಾಗಲೆ ಯೋಚಿಸಿದ್ದಾನೆ. ಆತನು ಯಾದವವಂಶದಲ್ಲಿ ಮಾನವರೂಪಿನಿಂದ ಹುಟ್ಟಿ ಭೂಭಾರವನ್ನು ಇಳಿಸುವವನಾಗಿದ್ದಾನೆ. ಅಷ್ಟರಲ್ಲಿ ನೀವೆಲ್ಲರೂ ಆ ವಂಶದಲ್ಲಿ ಮಾನವರಾಗಿ ಹುಟ್ಟಿರಬೇಕು. ಪುಷಿಗಳೆಲ್ಲರೂ ಗೋವುಗಳ ರೂಪದಿಂದ ನಂದಗೋಕುಲದಲ್ಲಿ ಹುಟ್ಟಲಿ. ಆದಿಶೇಷನು ಭಗವಂತನ ಅಣ್ಣನಾಗಿ ಹುಟ್ಟುತ್ತಾನೆ. ಶ್ರೀಹರಿಯ ಮಾಯೆಯೂ ಹೆಣ್ಣಿನ ರೂಪದಿಂದ ಅಲ್ಲಿಯೆ ಹುಟ್ಟುತ್ತಾಳೆ’ ಎಂದನು. ಇದನ್ನು ಕೇಳಿ ಭೂದೇವಿಗೂ ಆಕೆಯ ಜೊತೆಯಲ್ಲಿ ಬಂದಿದ್ದವ ರಿಗೂ ಪರಮ ಸಂತೋಷವಾಯಿತು. ಅವರು ತಮ್ಮ ಮುಂದಿನ ಕಾರ್ಯವನ್ನು ನಿರ್ವಹಿ ಸುವುದಕ್ಕಾಗಿ ಸಡಗರದಿಂದ ಹಿಂದಿರುಗಿದರು.”

