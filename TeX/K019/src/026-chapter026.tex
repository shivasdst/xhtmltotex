
\chapter{೨೬. ವೃತ್ರಾಸುರ}

ಕಶ್ಯಪಮಹರ್ಷಿಯನ್ನು ಮದುವೆಯಾದ ಪ್ರಾಚೇತಸಕನ್ನೆಯರಲ್ಲಿ ಅದಿತಿದೇವಿ ಅತ್ಯಂತ ಪುಣ್ಯವಂತೆ. ಆಕೆಯ ಹೊಟ್ಟೆಯಲ್ಲಿ ಹುಟ್ಟಿದ ಹನ್ನೆರಡು ಜನ ಆದಿತ್ಯರು ಮಹಾನುಭಾವರೆಂದು ಪ್ರಸಿದ್ಧರಾದರು. ಅವರಲ್ಲಿ ತ್ವಷ್ಟೃ ಎಂಬಾತನಿಗೆ ದೈತ್ಯರ ತಂಗಿ ಯಾದ ‘ರಚನೆ’ ಎಂಬುವಳನ್ನು ಕೊಟ್ಟು ವಿವಾಹವಾಗಿತ್ತು. ಆಕೆ ಸನ್ನಿವೇಶ, ವಿಶ್ವರೂಪ ಎಂಬ ಇಬ್ಬರು ಮಕ್ಕಳನ್ನು ಹೆತ್ತಳು. ಅವರಿಬ್ಬರೂ ವೇದವೇದಾಂತಗಳಲ್ಲಿ ಪಾರಂಗತ ರಾಗಿದ್ದರು. ವಿಶ್ವರೂಪನಂತೂ ಕೆಲಕಾಲ ದೇವತೆಗಳಿಗೆಲ್ಲ ಪುರೋಹಿತನೂ ಕೂಡ ಆಗಿದ್ದ. ಅವನ ತಾಯಿ ದೈತ್ಯಳಾಗಿದ್ದರೂ ವಿಶ್ವರೂಪನ ಪಾಂಡಿತ್ಯ ಆತನಿಗೆ ಆ ಸ್ಥಾನವನ್ನು ಗಳಿಸಿಕೊಟ್ಟಿತು.

ವಿಶ್ವರೂಪನು ದೇವಪುರೋಹಿತನಾದ ಕಥೆ ಬಹು ಸ್ವಾರಸ್ಯವಾಗಿದೆ. ದೇವರಾಜನಾದ ಇಂದ್ರನು ಮೂರು ಲೋಕಗಳ ಸ್ವಾಮಿಯಾಗಿ ಎಣೆಯಿಲ್ಲದ ಐಶ್ವರ್ಯ ಅಧಿಕಾರಗಳನ್ನು ಪಡೆದವನಾಗಿದ್ದನು. ಇದರಿಂದ ಸ್ವಾಭಾವಿಕವಾಗಿಯೇ ಅವನಲ್ಲಿ ತನಗೆ ಸಮಾನರಿಲ್ಲ ವೆಂಬ ಅಹಂಕಾರ ಉದಿಸಿತು. ಒಮ್ಮೆ ಆತ ತನ್ನ ಸಭೆಯಲ್ಲಿ ಮಡದಿಯಾದ ಶಚಿಯೊಡನೆ ಸಿಂಹಾಸನದಲ್ಲಿ ಅಟ್ಟಹಾಸದಿಂದ ಕುಳಿತಿದ್ದಾನೆ. ಆತನ ಸುತ್ತಲೂ ಮಹರ್ಷಿಗಳೂ ದೇವಾನುದೇವತೆಗಳೂ ನೆರೆದಿದ್ದಾರೆ. ಒಂದು ಕಡೆ ದಿವ್ಯ ಕಂಠದಿಂದ ವಿದ್ಯಾಧರರು ಆತನ ಕೀರ್ತಿಯನ್ನು ಹಾಡುತ್ತಿದ್ದಾರೆ. ಸುತ್ತಲೂ ಮುತ್ತಿಕೊಂಡಿರುವ ಅಪ್ಸರೆಯರು ಆತನಿಗೆ ಬೆಳ್ಗೊಡೆ ಹಿಡಿಯುವುದೇನು, ಚಾಮರ ಬೀಸುವುದೇನು–ಇತ್ಯಾದಿ ಸೇವೆಯಲ್ಲಿ ನಿರತ ರಾಗಿದ್ದಾರೆ. ಅಲ್ಲಿದ್ದವರೆಲ್ಲರೂ ಆತನ ಕುರುಣೆಗಾಗಿ, ಕಡೆಗಣ್​ನೋಟಕ್ಕಾಗಿ ಬಯಸಿ ಬಾಯ್ಬಿಡುತ್ತಿದ್ದಾರೆ. ಅಂತಹ ಸಮಯದಲ್ಲಿ ದೇವಗುರುವಾದ ಬೃಹಸ್ಪತಿ ಸಭೆಯನ್ನು ಪ್ರವೇಶಿಸಿದ. ಗರ್ವದಿಂದ ಕುರುಡನಾಗಿದ್ದ ದೇವೇಂದ್ರ ಆತನನ್ನು ನೋಡಿದರೂ ಮೇಲ ಕ್ಕೇಳಲಿಲ್ಲ, ನಮಸ್ಕರಿಸಲಿಲ್ಲ. ಬೃಹಸ್ಪತಿಗೆ ಅದು ಸರಿಬೀಳಲಿಲ್ಲ. ಆತ ಸಭೆಯಿಂದ ಹೊರಕ್ಕೆ ಹೊರಟುಹೋದನು. ಆಗ ದೇವೇಂದ್ರನ ಕಣ್ಣು ತೆರೆಯಿತು. ಸತ್ವಗುಣಪ್ರಧಾನ ರಾದ ದೇವತೆಗಳ ರಾಜನಾದರೂ ತಾನು ರಾಕ್ಷಸನಂತೆ ನಡೆದುಕೊಂಡುದಕ್ಕಾಗಿ ಪಶ್ಚಾತ್ತಾಪ ಪಟ್ಟನು. ಕಲ್ಲಿನ ದೋಣಿಯಲ್ಲಿ ಕುಳಿತು ಸಮುದ್ರವನ್ನು ದಾಟುವುದಕ್ಕಾದೀತೆ? ತನ್ನ ಅಹಂಕಾರದಿಂದ ದುರ್ಗತಿ ತಪ್ಪದೆಂದುಕೊಂಡ ದೇವೇಂದ್ರನು ಬೃಹಸ್ಪತಿಯಲ್ಲಿ ಮನ್ನಣೆ ಯನ್ನು ಬೇಡಿಕೊಳ್ಳಬೇಕೆಂದು ಸಭೆಯಿಂದ ಹೊರಗೆ ಬಂದನು. ಆದರೆ ಆತನನ್ನು ಕಾಣು ತ್ತಲೆ ಬೃಹಸ್ಪತಿಯು ಮಾಯವಾದನು.

ದೇವತೆಗಳ ಗುರುವು ತಲೆಮರೆಸಿಕೊಂಡು ಹೋಗಿರುವನೆಂಬ ಸುದ್ದಿ ಅಸುರರಿಗೆ ಗೊತ್ತಾಯಿತು. ಅವರ ಗುರುವಾದ ಶುಕ್ರಾಚಾರ್ಯನು ದೇವತೆಗಳ ಮೇಲೆ ದಂಡೆತ್ತಿ ಹೋಗುವುದಕ್ಕೆ ಇದೇ ಸಮಯವೆಂದು ಹೇಳಿದನು. ಅವರು ತಕ್ಷಣವೆ ಬಗೆಬಗೆಯ ಆಯುಧಗಳೊಡನೆ ಸ್ವರ್ಗಕ್ಕೆ ನುಗ್ಗಿ ದೇವತೆಗಳ ಮೇಲೆ ಬಿದ್ದರು. ಅವರ ಪೆಟ್ಟುಗಳನ್ನು ಸಹಿಸಲಾರದೆ, ದೇವತೆಗಳು ಸೋತು, ದಿಕ್ಕಾಪಾಲಾಗಿ ಓಡಿಹೋದರು. ಅವರೆಲ್ಲರೂ ದೇವೇಂದ್ರನೊಡನೆ ಬ್ರಹ್ಮನ ಬಳಿಗೆ ಹೋಗಿ, ತಮಗಾದ ಸಂಕಟವನ್ನು ಆತನಲ್ಲಿ ದೂರಿ ಕೊಂಡರು. ಆತನು ‘ಗುರುದೂಷಣೆಯೇ ನಿಮ್ಮ ದುರವಸ್ಥೆಗೆ ಕಾರಣ. ಈಗ ರಾಕ್ಷಸರನ್ನು ಜಯಿಸಬೇಕಾದರೆ ನಿಮಗೆ ಒಬ್ಬ ಗುರು ಅತ್ಯಗತ್ಯ. ಬೃಹಸ್ಪತಿ ಕಣ್ಮರೆಯಾಗಿರುವುದರಿಂದ ನೀವು ಬೇರೊಬ್ಬ ಗುರುವನ್ನು ನೇಮಿಸಿಕೊಳ್ಳಬೇಕು. ತ್ವಷ್ಟೃವಿನ ಮಗನಾದ ವಿಶ್ವರೂಪನು ಮಹಾತಪಸ್ವಿ, ಜಿತೇಂದ್ರಿಯ; ಬೃಹಸ್ಪತಿಯು ಮತ್ತೆ ಹಿಂದಿರುಗುವವರೆಗೆ ಆತನು ನಿಮ್ಮ ಗುರುವಾಗಿರಲಿ. ರಾಕ್ಷಸಿಯ ಮಗನಾದ ಆತ ರಾಕ್ಷಸರಿಗೆ ಪಕ್ಷಪಾತಿಯಾಗಿರುತ್ತಾನೆ. ನೀವು ಅದನ್ನು ಸದ್ಯಕ್ಕೆ ಸಹಿಸಿಕೊಳ್ಳಲೇಬೇಕು’ ಎಂದು ಹೇಳಿದನು.

ಬ್ರಹ್ಮನ ಉಪದೇಶದಂತೆ ದೇವತೆಗಳು ವಿಶ್ವರೂಪನ ಬಳಿಗೆ ಹೋಗಿ, ತಮಗೆ ಪುರೋ ಹಿತನಾಗಬೇಕೆಂದು ಆತನನ್ನು ಬೇಡಿಕೊಂಡರು. ಪುರೋಹಿತವೃತ್ತಿ ನೀಚಕಾರ್ಯ–ಬ್ರಹ್ಮ ತೇಜಸ್ಸನ್ನು ಹಾಳು ಮಾಡುತ್ತದೆ–ಎಂದು ವಿಶ್ವರೂಪನಿಗೆ ಗೊತ್ತಿದ್ದರೂ ಹಿರಿಯರಾದ ವರು ಬಂದು ಬೇಡುತ್ತಿರುವಾಗ ಒಲ್ಲೆನೆಂದು ಹೇಳಲಾರದೆ, ಅದನ್ನು ಸ್ವೀಕರಿಸಿದನು. ಅನಂತರ ಆತನು ‘ನಾರಾಯಣಕವಚ’ವೆಂಬ ಮಂತ್ರವಿದ್ಯೆಯ ಮಹಿಮೆಯಿಂದ ದೇವತೆ ಗಳ ಸಂಪತ್ತನ್ನೆಲ್ಲ ಅವರಿಗೆ ಮತ್ತೆ ಸಂಪಾದಿಸಿಕೊಟ್ಟನು. ದೇವೇಂದ್ರನು ತನ್ನ ಹೊಸ ಗುರುವಿನಿಂದ ನಾರಾಯಣಕವಚದ ಉಪದೇಶವನ್ನು ಪಡೆದು, ಶತ್ರುಗಳನ್ನು ಜಯಿಸಿ ದನು, ಮತ್ತೆ ವೈಭವದಿಂದ ಮೂರುಲೋಕಗಳ ಸ್ವಾಮಿಯಾದನು.

ಪುರೋಹಿತನಾದ ವಿಶ್ವರೂಪನ ತಂದೆ ದೇವಕುಲದವನು. ತಾಯಿ ದಾನವ ಕುಲದ ವಳು. ಆದ್ದರಿಂದ ದೇವಪುರೋಹಿತನಾದರೂ ಆತನಿಗೆ ದಾನವರಲ್ಲಿ ಮಮತೆ. ಆತನು ಯಜ್ಞಯಾಗಾದಿಗಳನ್ನು ಮಾಡುವಾಗ ಗಟ್ಟಿಯಾಗಿ ದೇವತೆಗಳ ಹೆಸರನ್ನು ಕೂಗಿ ಹೇಳಿ, ಅವರಿಗೆ ಹವಿಸ್ಸನ್ನು ಕೊಡುವನು. ಆದರೆ ಒಳಗೊಳಗೆ ಯಾರಿಗೂ ಗೊತ್ತಾಗದಂತೆ ತನ್ನ ತಾಯಿಯವರ ಕಡೆಗೂ ಹವಿಸ್ಸನ್ನು ಸಾಗಿಸುವನು. ಕೆಲವುಕಾಲದ ಮೇಲೆ ದೇವೇಂದ್ರನಿಗೆ ಈ ಮೋಸ ಗೊತ್ತಾಯಿತು. ಆತನು ಅತ್ಯಂತ ಕೋಪಗೊಂಡವನಾಗಿ, ಹಿಂದು ಮುಂದು ನೋಡದೆ ವಿಶ್ವರೂಪನ ತಲೆಗಳನ್ನು ಕತ್ತರಿಸಿ ಹಾಕಿದನು. (ಆತನಿಗೆ ಮೂರು ತಲೆಗಳಿ ದ್ದವಂತೆ! ಸೋಮರಸ ಕುಡಿವುದೊಂದು, ಸುರೆಯನ್ನು ಕುಡಿವುದೊಂದು, ಊಟ ಮಾಡು ವುದೊಂದು–ಮೂರು ಮುಖಗಳು, ಮೂರು ತಲೆಗಳು.) ಇದರಿಂದ ಆತನಿಗೆ ಬ್ರಹ್ಮ ಹತ್ಯಾದೋಷ ಬಂದಿತು. ಆ ದೋಷವನ್ನೇನೊ ಆತನು ಭೂಮಿ, ನೀರು, ಮರ, ಹೆಣ್ಣು– ಇವರಿಗೆ ಹಂಚಿ ಕಳೆದುಕೊಂಡನು\footnote{೧. ಭೂಮಿ ವಹಿಸಿದ ಬ್ರಹ್ಮಹತ್ಯಾಭಾಗವೇ ಚೌಳು ನೆಲ; ತನ್ನಲ್ಲಿ ಬಿದ್ದ ಹಳ್ಳಗಳು ತಾವಾಗಿಯೇ ಮುಚ್ಚಿಕೊಳ್ಳುವ ವರದೊಡನೆ ಭೂಮಿ ಶಾಪಭಾಗವನ್ನು ಸ್ವೀಕರಿಸಿತು. ಎಷ್ಟು ಕತ್ತರಿಸಿದರೂ ಮತ್ತೆ ಚಿಗುರುವ ವರದೊಡನೆ ಮರಗಳು ಶಾಪವನ್ನು ವಹಿಸಿಕೊಂಡವು; ಅದೇ ಮರದ ಅಂಟು ಅಥವಾ ಗೋಂದು. ಸ್ತ್ರೀಯರು ಪ್ರಸವಕಾಲದವರೆಗೆ ಸಂಭೋಗ ಸಾಮರ್ಥ್ಯದ ವರವನ್ನು ಪಡೆದು, ಶಾಪಭಾಗ ಸ್ವೀಕರಿಸಿದರು; ಆ ಶಾಪವೇ ರಜಸ್ವಲೆಯರಾಗುವುದಕ್ಕೆ ಕಾರಣ. ನೀರು ಎಲ್ಲ ಕಡೆಗೂ ಹರಡಿಕೊಳ್ಳುವ ವರವನ್ನು ಪಡೆದು ಶಾಪವನ್ನು ಪರಿಗ್ರಹಿಸಿತು; ಅದೇ ನೀರಿನಲ್ಲಿ ಕಾಣುವ ನೊರೆ.}. ಆದರೆ ಇಂದ್ರನಿಂದ ತನ್ನ ಮಗನು ಸತ್ತುದನ್ನು ಕೇಳಿದ ತ್ವಷ್ಟೃವು ರೋಷಾವೇಶದಿಂದ ಒಂದು ಯಾಗವನ್ನು ಕೈಕೊಂಡು ‘ಓ ಇಂದ್ರ ಶತ್ರು, ಬೇಗ ಬೆಳೆ. ಬೆಳೆದು ಬಾ. ನಿನ್ನ ಶತ್ರುವನ್ನು ಕೊಲ್ಲು’ ಎಂದು ಹೇಳಿ ಹೋಮಮಾಡಿ ದನು. ಒಡನೆಯೇ ಹೋಮಕುಂಡದಿಂದ ಪ್ರಳಯಕಾಲದ ಮೃತ್ಯುವಿನಂತಿದ್ದ ಭಯಂಕರ ಮೂರ್ತಿಯೊಂದು ಹೊರಕ್ಕೆ ಹಾರಿಬಂದಿತು. ಅವನೇ ವೃತ್ರಾಸುರ.

ವೃತ್ರನು ದಿನಕ್ಕೊಂದು ಬಾಣವೇಗದಷ್ಟು ಪ್ರಮಾಣದಲ್ಲಿ ಬೆಳೆಯುತ್ತಾ ಹೋದನು. ಅವನ ಮೈಬಣ್ಣ ಕಾಡುಗಿಚ್ಚಿನಂತಿತ್ತು. ಅವನ ಗಡ್ಡ ಮೀಸೆಗಳು ಕಾದ ತಾಮ್ರದ ತಂತಿ ಯಂತಿದ್ದವು. ಅವನ ಕಣ್ಣುಗಳು ನಡು ಹಗಲಿನ ಸೂರ್ಯನಂತಿದ್ದವು. ಅವನ ನಡೆಗೆ ಭೂಮಿ ನಡುಗುವುದು. ಆಕಾಶವನ್ನು ಭೇದಿಸುವಂತಿರುವ ಅವನ ತಲೆ, ನಕ್ಷತ್ರಗಳನ್ನು ನೆಕ್ಕುತ್ತಿರು ವಂತಿರುವ ಅವನ ನಾಲಗೆ, ಮೂರು ಲೋಕಗಳನ್ನೂ ಏಕಕಾಲದಲ್ಲಿ ನುಂಗುವಂತಹ ಅವನ ಬಾಯಿ, ನೋಡಿದವರು ನಡುಗುವಂತಹ ಆ ಕೋರೆ ದಾಡೆ–ಇಂತಹ ಭಯಂಕರ ರೂಪಿನ ಆ ರಕ್ಕಸ ಆಕಳಿಸಲೆಂದು ಬಾಯ್ತೆರೆದಾಗ ಜಗತ್ತಿನ ಜೀವರಾಶಿಯೆಲ್ಲ ಭಯದಿಂದ ದಿಕ್ಕಾಪಾಲಾಗಿ ಓಡಿಹೋಯಿತು. ದೇವತೆಗಳು ಇವನನ್ನು ಕೊಲ್ಲಬೇಕೆಂದು ಅವನ ಮೇಲೆ ಬಾಣಗಳ ಮಳೆಗರೆದರು. ಆದರೆ ಅವೆಲ್ಲ ಆತನಿಗೆ ಆಪೋಶನಕ್ಕಾದವು. ಇದನ್ನು ಕಂಡು ದೇವತೆಗಳು ಜೀವಭಯದಿಂದ ತಲ್ಲಣಿಸುತ್ತಾ ನಾರಾಯಣನನ್ನು ಮರೆ ಹೊಕ್ಕರು. “ಹೇ ಸ್ವಾಮಿ, ಸರ್ವೇಶ್ವರ! ನಮಗೆ ಈಗ ನೀನಲ್ಲದೆ ಬೇರೆ ದಿಕ್ಕಿಲ್ಲ. ಭಯಂಕರನಾದ ಮೃತ್ಯು ವಿಗೆ ಕೂಡ ಭಯಂಕರನಾದ ನೀನಲ್ಲದೆ ಈ ವೃತ್ರನಿಂದ ನಮ್ಮನ್ನು ರಕ್ಷಿಸುವವರಿನ್ನಾರು? ಪ್ರಳಯಕಾಲದಲ್ಲಿ ಭೂಮಿಯು ಸಮುದ್ರದಲ್ಲಿ ಮುಳುಗಿಹೋಗುತ್ತಿದ್ದಾಗ ಮತ್ಸ್ಯರೂಪ ದಿಂದ ಅದನ್ನು ರಕ್ಷಿಸಿದ ನೀನಲ್ಲದೆ ನಮಗಾರು ದಿಕ್ಕು? ನಿನ್ನನ್ನು ಬಿಟ್ಟು ಬೇರೊಬ್ಬರನ್ನು ಆಶ್ರಯಿಸುವುದೆಂದರೆ ನಾಯಿಯ ಬಾಲವನ್ನು ಹಿಡಿದು ಸಮುದ್ರವನ್ನು ದಾಟುವೆನೆಂಬ ಎಗ್ಗನಂತಾಗುತ್ತದೆ. ಒಂದೊಂದು ಯುಗದಲ್ಲಿ ಒಂದೊಂದು ಬಗೆಯ ಅವತಾರವನ್ನೆತ್ತಿ ನಮ್ಮನ್ನು ರಕ್ಷಿಸಿರುವೆ. ಈಗಲೂ ನೀನೆ ನಮಗೆ ಶರಣು” ಎಂದು ಪ್ರಾರ್ಥಿಸಿದರು.

ದೇವತೆಗಳ ಪ್ರಾರ್ಥನೆ ದೇವದೇವನನ್ನು ಮುಟ್ಟಿತು. ಆತನು ಅವರ ಭಾವನೆಗೆ ತಕ್ಕಂತೆ ಶಂಖ ಚಕ್ರ ಗದಾಧಾರಿಯಾಗಿ ಪ್ರತ್ಯಕ್ಷನಾದನು. ಒಡನೆಯೆ ದೇವತೆಗಳೆಲ್ಲ ದಿಗ್ಗನೆದ್ದು, ಆತನಿಗೆ ಅಡ್ಡಬಿದ್ದು, ಮೇಲೆದ್ದು ಕೈಜೋಡಿಸಿಕೊಂಡು, ಅತ್ಯಂತ ಭಕ್ತಿಯಿಂದ ‘ಹೇ ಭಗವಂತ, ವಾಸುದೇವ, ಆದಿಪುರುಷ, ಮಹಾಪುರುಷ, ಮಹಾನುಭಾವ, ಕರುಣಾಕರ, ಜಗದಾಧಾರ, ಲೋಕನಾಥ, ಭಕ್ತವತ್ಸಲ, ಲಕ್ಷ್ಮೀ ರಮಣ, ನಿನಗೆ ನಮಸ್ಕಾರ! ಪರಮ ಹಂಸರಾದ ಯೋಗಿಗಳು ಯೋಗಮಾರ್ಗದಿಂದ ನಿನ್ನನ್ನು ಪ್ರತ್ಯಕ್ಷ ಮಾಡಿಕೊಳ್ಳುತ್ತಾರೆ. ಅವರಿಗೆ ಆನಂದಮಯ ರೂಪನಾಗಿ ನೀನು ಗೋಚರಿಸುತ್ತಿ. ನೀನು ನಿರಾಶ್ರಯ, ನಿರ್ಗುಣ, ನಿರ್ವಿಕಾರ; ಆದಅರೆ ಈ ಜಗತ್ತನ್ನು ಸೃಷ್ಟಿಸಿ, ಬೆಳಸಿ, ಲಯಗೊಳಿಸುತ್ತಿ. ಏಕೆ ನೀನು ಈ ಜಗನ್ನಾಟಕವನ್ನು ಆಡುತ್ತೀಯೆಂಬುದು ನಮಗೆ ಅರ್ಥವಾಗುವುದಿಲ್ಲ. ಜೀವ ರೂಪದಿಂದ ದೇಹಧಾರಿಯಾಗುವ ನೀನು ಸುಖದುಃಖಗಳನ್ನು ಅನುಭವಿಸುತ್ತೀಯೋ, ಆತ್ಮಾನಂದದಲ್ಲಿ ಮುಳುಗಿ, ಕೇವಲ ಸಾಕ್ಷಿಭೂತನಾಗಿರುವೆಯೊ! ಸ್ವಾಮಿ, ನಮಗೆ ನೀನು ಅರ್ಥವಾಗುವುದಿಲ್ಲ. ನಿನ್ನ ಮಹಿಮೆಯೆಂಬ ಅಮೃತದ ಒಂದು ತೊಟ್ಟನ್ನು ಕುಡಿದರೆ ಸಾಕು, ಆತನು ಆನಂದಮಯನಾಗಿ ಹೋಗುತ್ತಾನಂತೆ! ಹೇ ಲೋಕಾಧಾರ, ಲೋಕ ಸ್ವರೂಪ, ಲೋಕರಕ್ಷಕ ನಾವು ನಿನ್ನಲ್ಲಿ ಏನನ್ನು ಬೇಡೋಣ? ದೇವತೆಗಳು, ದಾನವರು, ಮಾನವರು, ಎಲ್ಲ ಪ್ರಾಣಿಗಳೂ ನಿನ್ನ ವಿಭೂತಿಗಳೇ ಎಂದ ಮೇಲೆ ದಾನವನಾದ ವೃತ್ರ ನನ್ನು ಕೊಲ್ಲೆಂದು ನಾವು ಹೇಗೆ ಪ್ರಾರ್ಥಿಸೋಣ? ನೀನು ಸೂಕ್ತ ತೋರಿದಂತೆ ಮಾಡು–ಎಂದು ಮಾತ್ರ ನಾವು ಕೇಳಿಕೊಳ್ಳಬಹುದು. ನಾವು ನಿನ್ನವರು; ನಮ್ಮನ್ನು ಸಂಕಟದಿಂದ ಉದ್ಧರಿಸು. ಸರ್ವಜ್ಞನಾದ ನಿನಗೆ ನಾವು ಹೇಳಿಕೊಳ್ಳುವುದೇನು ಬಂತು? ನಮ್ಮ ಮನಸ್ಸಿನಲ್ಲಿ ಏನಿದೆಯೆಂಬುದು ಕೂಡ ನಿನಗೆ ಗೊತ್ತಿಲ್ಲದಿಲ್ಲ. ನಮ್ಮ ಅಪೇಕ್ಷೆ ಯನ್ನು ಸಲ್ಲಿಸು’ ಎಂದು ಬೇಡಿಕೊಂಡರು. 

ದೇವತೆಗಳ ಸ್ತೋತ್ರದಿಂದ ಸುಪ್ರೀತನಾದ ಶ್ರೀಹರಿಯು ತನ್ನ ತುಟಿಗಳಿಂದ ಹೊರ ಚೆಲ್ಲುತ್ತಿರುವ ಆನಂದ ಬಿಂದುಗಳಂತೆ ಇರುವ ಮಧುರವಾದ ನುಡಿಗಳಿಂದ ಅವರನ್ನು ಸಂತೈಸಿದನು: ‘ಎಲೈ ದೇವತೆಗಳೆ, ನನ್ನ ನಿಜವಾದ ಭಕ್ತನು ನನ್ನನ್ನು ಹೊರತು ಮತ್ತೇ ನನ್ನೂ ಬಯಸುವುದಿಲ್ಲ. ಅಲ್ಪವಾದ ಸುಖವನ್ನು ಬಯಸಿದರೆ ನಾನದನ್ನು ಕೊಡಲೂ ಬಾರದು. ರೋಗಿಯಾದವನು ಚಪಲದಿಂದ ಅಪಥ್ಯದ ವಸ್ತುಗಳನ್ನು ಅಪೇಕ್ಷಿಸಿದರೆ ವೈದ್ಯ ನಾದವನು ಅದನ್ನು ಒಪ್ಪುತ್ತಾನೆಯೆ? ಎಂದಿಗೂ ಇಲ್ಲ. ಆದರೂ ನೀವೀಗ ಬೇಡುತ್ತಿರುವ ಅಪೇಕ್ಷೆಯನ್ನು ನಾನು ನೀಡುತ್ತಿದ್ದೇನೆ. ಅಯ್ಯಾ ದೇವೇಂದ್ರ! ನೀನು ದಧೀಚಿ ಮಹರ್ಷಿ ಗಳ ಬಳಿಗೆ ಹೋಗಿ ಆತನ ಶರೀರವನ್ನು ನಿನಗೆ ಒಪ್ಪಿಸುವಂತೆ ಮಾಡಿಕೊ. ಆತನ ಶರೀರ ಜ್ಞಾನ, ತಪಸ್ಸು, ನಿಯಮಗಳಿಂದ ಬಹಳ ದೃಢವಾದದು. ಮಹನೀಯನಾದ ಆತ ಕೇಳಿದು ದುದನ್ನು ಇಲ್ಲವೆನ್ನುವುದಿಲ್ಲ. ಆತನ ದೇಹದಲ್ಲಿರುವ ಮೂಳೆಗಳಿಂದ ವಿಶ್ವಕರ್ಮನು ವಜ್ರಾಯುಧವನ್ನು ಮಾಡಿಕೊಡುತ್ತಾನೆ. ಅದರಿಂದ ನೀನು ವೃತ್ರಾಸುರನನ್ನು ಕೊಲ್ಲು ವುದು ಸಾಧ್ಯವಾಗುತ್ತದೆ. ನಿಮಗೆಲ್ಲರಿಗೂ ಮಂಗಳವಾಗಲಿ’ ಎಂದು ಹೇಳಿ, ಮಾಯವಾದನು.

ಮಹಾವಿಷ್ಣುವಿನ ಅಪ್ಪಣೆಯಂತೆ ದೇವತೆಗಳೆಲ್ಲರೂ ನೇರವಾಗಿ ದಧೀಚಿಋಷಿಯ ಬಳಿಗೆ ಹೋಗಿ ಆತನ ದೇಹವನ್ನು ತಮಗೆ ಒಪ್ಪಿಸುವಂತೆ ಕೇಳಿಕೊಂಡರು. ಮಹಾತ್ಯಾಗಿ ಯಾದ ದಧೀಚಿಗೂ ಅವರ ಪ್ರಾರ್ಥನೆಯನ್ನು ಕೇಳಿ ನಗುಬಂತು. ‘ಅಯ್ಯಾ, ಮರಣ ಸಂಕಟವೆಂದರೆ ಏನೆಂಬುದು ನಿಮಗೇನು ಗೊತ್ತು? ಪ್ರಾಣಿಗೆ ತನ್ನ ದೇಹಕ್ಕಿಂತಲೂ ಪ್ರಿಯಕರವಾದ ವಸ್ತು ಬೇರೊಂದಿಲ್ಲ. ಸಾಕ್ಷಾತ್ ದೇವರೇ ಬಂದು ಕೇಳಿದರೂ, ದೇಹ ವನ್ನು ಯಾರೂ ದಾನಮಾಡುವುದಿಲ್ಲ’ ಎಂದನು ಆತ. ಸ್ವಾರ್ಥದಲ್ಲಿ ಮುಳುಗಿದ್ದ ದೇವತೆ ಗಳು ಹೇಳಿದರು: ‘ಸ್ವಾಮಿ, ಕೇಳುವವನ ಆಶೆಗೆ ಹೇಗೆ ಕೊನೆಯಿಲ್ಲವೋ, ಹಾಗೆಯೇ ದಾನಿ ಯಾದವನ ಔದಾರ್ಯಕ್ಕೂ ಕೊನೆಯಿಲ್ಲ. ಸ್ವಪ್ರಯೋಜನಕ್ಕೆ ಬಾಯಿಬಿಡುವವನು ಪರರ ಸಂಕಟಗಳನ್ನು ಗಮನಿಸುವುದಿಲ್ಲ. ಹಾಗೆಯೇ ಪರೋಪಕಾರಿಯಾದವನು ತನ್ನ ಸಂಕಟ ವನ್ನು ಗಮನಿಸದೆ, ಪರರ ಹಿತವನ್ನು ಬಯಸುತ್ತಾನೆ’ ಎಂದರು. ಅವರ ಬಿಚ್ಚುನುಡಿಗಳನ್ನು ಕೇಳಿ ದಧೀಚಿ ಪುಷಿಗೆ ಆನಂದವಾಯಿತು. ಆತ ಹೇಳಿದ: ‘ಅಯ್ಯಾ ಈ ದೇಹ ಎಂದಿದ್ದರೂ ಬಿದ್ದುಹೋಗುವುದೇ. ಅದು ಇಂದೇ ಬಿದ್ದು ಹೋದರೇನು ನಷ್ಟ? ಅಗತ್ಯವಾಗಿಯೂ ಅದನ್ನು ಪರೀಕ್ಷಿಸುವುದಕ್ಕಾಗಿ ಮೊದಲು ಹಾಗೆ ಹೇಳಿದೆ ಅಷ್ಟೆ. ನಿಮಗೆ ಈ ದೇಹದಿಂದ ಉಪಯೋಗವಾಗುವ ಹಾಗಿದ್ದರೆ, ಇಗೋ ಈ ಕ್ಷಣ ತೆಗೆದುಕೊಳ್ಳಿ’ ಎಂದು ಹೇಳಿ, ಸಮಾಧಿಯೋಗದಿಂದ ತನ್ನ ದೇಹವನ್ನು ತ್ಯಜಿಸಿದನು. ಆತನ ಜೀವಾತ್ಮನು ಪರಮಾತ್ಮ ನಲ್ಲಿ ಐಕ್ಯನಾದನು. ಆತನ ಜಡದೇಹ ನೆಲಕ್ಕುರುಳಿತು.

ದಧೀಚಿಯ ದೇಹದ ಮೂಳೆಗಳಿಂದ ವಿಶ್ವಕರ್ಮನು ವಜ್ರಾಯುಧವನ್ನು ನಿರ್ಮಿಸಿ ಕೊಟ್ಟನು. ಅದನ್ನು ಕೈಲಿ ಹಿಡಿದ ದೇವೇಂದ್ರನಲ್ಲಿ ಆ ಮಹರ್ಷಿಯ ತೇಜಸ್ಸು ನೆಲೆ ಗೊಂಡಿತು. ಆತನು ಐರಾವತವನ್ನು ಏರಿ ವೃತ್ರಾಸುರನ ಮೇಲೆ ಯುದ್ಧಕ್ಕೆ ಹೊರಟನು. ಆತನೊಡನೆ ಏಕಾದಶರುದ್ರರು, ಅಷ್ಟವಸುಗಳು, ದ್ವಾದಶಾದಿತ್ಯರು–ಇತ್ಯಾದಿ ದೇವಸೈನ್ಯ ವೆಲ್ಲವೂ ಹೊರಟಿತು. ಇದನ್ನು ಕೇಳಿ ವೃತ್ರಾಸುರನು ದೈತ್ಯ ದಾನವ ಯಕ್ಷರಾಕ್ಷಸರ ಸೈನ್ಯ ದೊಡನೆ ಅವನಿಗೆ ಇದಿರಾದನು. ಅದು ಕೃತ ತ್ರೇತೆಗಳ ಸಂಧಿಕಾಲ. ನರ್ಮದಾ ನದಿಯ ತೀರದಲ್ಲಿ ಎರಡು ಪಡೆಗಳಿಗೂ ಘೋರವಾದ ಯುದ್ಧ ನಡೆಯಿತು. ವೃತ್ರನ ಕಡೆಯವರು ಬಿಟ್ಟ ಬಾಣಗಳನ್ನೆಲ್ಲ ಇಂದ್ರನ ಕಡೆಯವರು ಮಧ್ಯಮಾರ್ಗದಲ್ಲಿಯೇ ಕತ್ತರಿಸಿ ಹಾಕಿ ದರು. ಇದರಿಂದ ಕೋಪಗೊಂಡ ರಾಕ್ಷಸರು ಗಿಡಮರಗಳನ್ನೂ ಬೆಟ್ಟಗುಡ್ಡಗಳನ್ನೂ ಕಿತ್ತು ತಂದು ದೇವತೆಗಳಮೇಲೆ ಎತ್ತಿಹಾಕಿದರು. ಆದರೇನು? ನೀಚರಾದವರು ಸತ್ಪುರುಷರ ಮೇಲೆ ಬಳಸುವ ಕೆಟ್ಟ ಬಯ್ಗಳಂತೆ ಅವು ವ್ಯರ್ಥವಾದವು. ದೇವದ್ವೇಷಿಗಳಾದ ರಾಕ್ಷಸರು ತಮ್ಮ ಕೈಕೆಳಗಾದುದನ್ನು ಕಂಡು ಭಯದಿಂದ ದಿಕ್ಕಾಪಾಲಾಗಿ ಚದರಿ ಓಡಿಹೋದರು. ಧೀರನಾದ ವೃತ್ರನು ಮಾತ್ರ ಯುದ್ಧರಂಗದಲ್ಲಿ ಸ್ಥಿರವಾಗಿ ನಿಂತು ‘ಅಯ್ಯಾ, ರಾಕ್ಷಸ ವೀರರೇ, ಹುಟ್ಟಿದವರು ಸಾಯಲೇಬೇಕು. ಆದರೆ ಸಾಯುವುದರಲ್ಲಿಯೂ ಯೋಗ್ಯರೀತಿ ಯಲ್ಲಿ ಸಾಯೋಣ. ಯೋಗ್ಯಮಾರ್ಗದಲ್ಲಿ ಶ್ವಾಸಧಾರಣೆ ಮಾಡಿ ಸಾಯುವುದು ಮತ್ತು ಯುದ್ಧದಲ್ಲಿ ಶತ್ರುವಿನೊಡನೆ ಹೋರಾಡುತ್ತಾ ಸಾಯುವುದು–ಇವು ಶ್ರೇಷ್ಠವಾದ ಸಾವು ಗಳು’ ಎಂದು ಕೂಗಿ ಹೇಳಿದ. ಜೀವಗಳ್ಳರಾದ ರಾಕ್ಷಸರಿಗೆ ಅವನ ಬುದ್ಧಿವಾದ ರುಚಿಸಲಿಲ್ಲ. ಅವರು ಕಾಲಿಗೆ ಬುದ್ಧಿ ಹೇಳಿದರು.

ರಾಕ್ಷಸರು ಓಡಿಹೋಗುತ್ತಿದ್ದಾರೆ, ದೇವತೆಗಳು ಅವರನ್ನು ಬೆನ್ನಟ್ಟಿಕೊಂಡು ಹೋಗಿ ಕೊಲ್ಲುತ್ತಿದ್ದಾರೆ. ರಾಕ್ಷಸರ ಹೇಡಿತನ, ದೇವತೆಗಳ ಉತ್ಸಾಹ–ಇವನ್ನು ಕಂಡು ವೃತ್ರಾ ಸುರನಿಗೆ ತಡೆಯಲಾರದಷ್ಟು ಕೋಪ ಉಕ್ಕಿತು. ಆತನು ‘ಹೇ ಕೀಳು ದೇವತೆಗಳೆ! ಹೇಡಿ ಗಳಾಗಿ ಓಡುತ್ತಿರುವ ಆ ಬಡರಕ್ಕಸರನ್ನು ಕೊಲ್ಲವ ನಿಮ್ಮ ಪರಾಕ್ರಮವನ್ನು ಏನೆಂದು ಕೊಂಡಾಡಲಿ! ನಿಮಗೆ ಧೈರ್ಯವಿದ್ದರೆ ನನ್ನನ್ನು ಇದಿರಿಸಿ’ ಎಂದು ಗುಡುಗಿದನು. ಅವನ ಅಬ್ಬರಕ್ಕೆ ದೇವತೆಗಳು ಸಿಡಿಲು ಹೊಡೆದ ಬೆಟ್ಟಗಳಂತೆ ನೆಲಕ್ಕೆ ಬಿದ್ದರು. ವೃತ್ರಾಸುರನು ಭೂಮಿ ನಡುಗುವಂತೆ ಅವರತ್ತ ನಡೆದು ಬಂದು, ಕಾಲಿನಿಂದ ತಿಕ್ಕಿ ಅವರನ್ನು ಕೊಲ್ಲ ಹೊರಟನು. ಇದನ್ನು ಕಂಡ ದೇವೇಂದ್ರನು ಅವನತ್ತ ಧಾವಿಸಿ ತನ್ನ ಕೈಲಿದ್ದ ಗದೆಯನ್ನು ಅವನತ್ತ ಬೀಸಿದನು. ಅದನ್ನು ವೃತ್ರನು ಲೀಲಾಜಾಲವಾಗಿ ತನ್ನ ಎಡಗೈಯಲ್ಲಿ ಹಿಡಿದು; ಅದರಿಂದಲೇ ಇಂದ್ರನು ಹತ್ತಿಕೊಂಡಿದ್ದ ಐರಾವತದ ನೆತ್ತಿಯ ಮೇಲೆ ಅಪ್ಪಳಿಸಿದನು. ಆ ಹೊಡೆತಕ್ಕೆ ಇಂದ್ರನ ಆನೆ ತತ್ತರಿಸಿ ಹೋಯಿತು; ತಲೆಯಿಂದ ನೆತ್ತರು ಸುರಿಸುತ್ತಾ ಅದು ಏಳು ಮಾರು ದೂರ ಹಿಂದಕ್ಕೆ ಓಡಿಹೋಯಿತು. ಇದನ್ನು ಕಂಡು ಉತ್ಸಾಹಗೊಂಡ ವೃತ್ರನು “ಎಲಾ ದೇವೇಂದ್ರ! ಗುರುಹತ್ಯೆ, ಬ್ರಹ್ಮಹತ್ಯೆಯನ್ನು ಮಾಡಿದ ಮಹಾಪಾಪಿ, ನೀನು. ನನ್ನ ಅದೃಷ್ಟದಿಂದ ನೀನು ನನಗೆ ಇದಿರಾಗಿರುವೆ. ನಿನ್ನನ್ನು ಕೊಂದು ನನ್ನ ಅಣ್ಣ ನಾದ ವಿಶ್ವರೂಪನ ಪುಣವನ್ನು ತೀರಿಸಿಕೊಳ್ಳುತ್ತೇನೆ. ನೀಚ, ನಿನಗೆ ನಾರಾಯಣಕವಚ ವನ್ನು ಬೋಧಿಸಿದ ಗುರುವನ್ನೂ ಕೊಂದೆಯಲ್ಲಾ, ನಿನಗೆ ನಾಚಿಕೆಯಾಗುವುದಿಲ್ಲವೆ? ನಿನಗೆ ದಯೆ ಧರ್ಮಗಳ ವಾಸನೆಯಾದರೂ ಇದೆಯೆ? ಲೋಕದ ಜನ ನಿಂದಿಸುವರೆಂಬ ಭಯ ವಾದರೂ ಬೇಡವೆ? ನೀನು ಪಿಶಾಚಿಗಳಿಗಿಂತ ಕೀಳು. ನಿನ್ನನ್ನು ನನ್ನ ಶೂಲದಿಂದ ತುಂಡು ತುಂಡಾಗಿ ಕತ್ತರಿಸಿ, ನಾಯಿ ಹದ್ದುಗಳಿಗೆ ಆಹಾರವಾಗಿ ಮಾಡುತ್ತೇನೆ. ಪಾಪಿ, ನಿನ್ನ ಹಾಗೆ ನಾನೇನೂ ಜೀವಗಳ್ಳನಲ್ಲ. ನೀನು ನಿನ್ನ ವಜ್ರಾಯುಧವನ್ನು ಬಳಸಿ ನನ್ನನ್ನು ಕೊಂದರೂ, ನಾನು ನನ್ನ ಗುರುವೂ ಸ್ವಾಮಿಯೂ ಆದ ಸಂಕರ್ಷಣನನ್ನು ಒಂದೇ ಮನಸ್ಸಿನಿಂದ ಧ್ಯಾನಮಾಡುತ್ತಾ ಯೋಗಿಗಳ ಸದ್ಗತಿಯನ್ನು ಪಡೆಯುತ್ತೇನೆ” ಎಂದು ಹೇಳಿದನು. 

ದೇವೇಂದ್ರನ ಕರ್ತವ್ಯವನ್ನು ವೃತ್ರನೇ ಸೂಚಿಸಿದನು. ಅದನ್ನು ಗ್ರಹಿಸಿದ ದೇವೇಂದ್ರನು ತನ್ನ ವಜ್ರಾಯುಧದಿಂದ ವೃತ್ರಾಸುರನ ಶೂಲವನ್ನೂ, ಅದನ್ನು ಹಿಡಿ ಯುವ ಅವನ ಬಲದೋಳನ್ನೂ ಕತ್ತರಿಸಿ ಹಾಕಿದನು. ಆದರೇನು? ವೃತ್ರನು ಅದರಿಂದ ಸ್ವಲ್ಪವೂ ಕೊಂಕಲಿಲ್ಲ, ಕೊರಗಲಿಲ್ಲ. ನಿಂತಲ್ಲಿಯೇ ನಿಂತು, ತನ್ನ ಬಳಿಗೆ ಬಂದ ಇಂದ್ರ ಮತ್ತು ಅವನ ಆನೆಯ ಕಪಾಳಕ್ಕೆ ಎಡಗೈಯಿಂದ ಬಲವಾದ ಒಂದು ಪೆಟ್ಟು ಕೊಟ್ಟನು. ಆನೆ ತತ್ತರಿಸಿತು, ಇಂದ್ರನ ಕೈಲಿದ್ದ ವಜ್ರಾಯುಧ ಕೈಜಾರಿ ಕೆಳಗೆ ಬಿತ್ತು. ಅವನು ಮಂಕು ಬಡಿದವನಂತೆ ಸುಮ್ಮನೆ ನಿಂತು ಬಿಟ್ಟನು. ಆಗ ವೃತ್ರನೇ ಅವನನ್ನು ಕುರಿತು ‘ಹೇ ಮಂಕೆ, ಏಕೆ ಸುಮ್ಮನೆ ನಿಂತೆ? ತೆಗೆದುಕೊ ವಜ್ರಾಯುಧವನ್ನು. ಹೊಡಿ ನನ್ನನ್ನು ಮತ್ತೊಮ್ಮೆ. ಯುದ್ಧದಲ್ಲಿ ಸೋಲು ಗೆಲವು ಒಂದು ಲೆಕ್ಕವಲ್ಲ. ಭಗವಂತನೊಬ್ಬನ ಹೊರತು ಉಳಿದ ಯಾರಿಗೂ ಜಯವೆಂಬುದು ಸ್ಥಿರವಲ್ಲ. ಎಲ್ಲವೂ ಅವನ ಕೈಲಿದೆ. ‘ನಾನು ಗೆದ್ದೆ, ನಾನು ಸೋತೆ’ ಎಂದುಕೊಳ್ಳುವುದು ಮೂಢತನ. ಚತುರ್ಮುಖಬ್ರಹ್ಮನು ಕೂಡ ಭಗವಂತನ ಅನುಗ್ರಹವಿಲ್ಲದೆ ಸೃಷ್ಟಿಕಾರ್ಯವನ್ನು ನಡೆಸಲಾರ ಎಂದ ಮೇಲೆ ನಾನು, ನೀನು ಜಯ ಗಳಿಸುತ್ತೇನೆಂದು ಅಹಾಂಕಾರಪಡುವುದು ಶುದ್ಧ ನಗೆಪಾಟಲು. ಅವನು ಆಡಿಸಿದಂತೆ ನಾವು ಆಡೋಣ. ತಗೋ ನಿನ್ನ ಆಯುಧವನ್ನು ಕೈಗೆ. ಯುದ್ಧವೆಂಬುದು ಒಂದು ಜೂಜು. ನಮ್ಮ ಪ್ರಾಣಗಳೇ ನಾವು ಒಡ್ಡಿರುವ ಪಣ. ಯಾರು ಗೆಲ್ಲುವರೋ ಯಾರು ಸೋಲು ವರೋ, ಯಾರಿಗೆ ಗೊತ್ತು? ದೇವರು ನಿನಗೆ ಸಹಾಯಕನಾಗಿರುವುದರಿಂದ ನೀನೇ ಗೆಲ್ಲ ಬಹುದು. ಹೆದರದೆ ಯುದ್ಧಮಾಡು’ ಎಂದನು. ವೃತ್ರನ ಮಾತುಗಳನ್ನು ದೇವೇಂದ್ರನು ಮನಸಾ ಮೆಚ್ಚಿ, ಆತನ ಮಾತಿನಂತೆ ತನ್ನ ವಜ್ರಾಯುಧವನ್ನು ಮತ್ತೆ ಕೈಗೆತ್ತಿಕೊಂಡನು. ಆತನು ತನ್ನ ಶತ್ರುವನ್ನು ಬಾಯ್ತುಂಬ ಹೊಗಳುತ್ತಾ, ‘ಹೇ ದಾನವೇಂದ್ರ, ನೀನು ನಿಜ ವಾಗಿಯೂ ಮಾಯೆಯನ್ನು ಗೆದ್ದ ಮಹಾತ್ಮಾ! ನಿನ್ನ ದೈವಭಕ್ತಿ ಆಶ್ಚರ್ಯಕರವಾದುದು. ನಿನ್ನ ಇಷ್ಟದಂತೆ ಇಗೋ ಯುದ್ಧಕ್ಕೆ ಸಿದ್ಧನಾಗಿ ನಿಂತಿದ್ದೇನೆ’ ಎಂದು ಹೇಳಿದನು. ಆಗ ವೃತ್ರನು ಒಂದು ಉಕ್ಕಿನ ಪರಿಘವನ್ನು ಇಂದ್ರನ ಮೇಲೆ ಪ್ರಯೋಗಿಸಿದನು. ಇಂದ್ರನು ಅದನ್ನೂ, ವೃತ್ರನ ಎಡ ತೋಳನ್ನೂ ಏಕಕಾಲದಲ್ಲಿ ತನ್ನ ವಜ್ರಾಯುಧದಿಂದ ಕತ್ತರಿಸಿಹಾಕಿದನು. ರೆಕ್ಕೆ ಕತ್ತರಿಸಿದ ಪರ್ವತದಂತೆ ಆದ ವೃತ್ರಾಸುರನು ಭೂಮಿ ಆಕಾಶಗಳಷ್ಟು ಅಗಲವಾದ ಬಾಯನ್ನು ತೆರೆದು, ಐರಾವತದೊಡನೆ ಆ ದೇವೇಂದ್ರನನ್ನು ಒಂದೇ ತುತ್ತಿಗೆ ನುಂಗಿಹಾಕಿ ದನು. ಇದನ್ನು ಕಂಡ ದೇವತೆಗಳೆಲ್ಲ ಹಾಹಾಕಾರ ಮಾಡಿದರು. ಆದರೆ ‘ನಾರಾಯಣಕವಚ’ ಮಂತ್ರದ ಸಿದ್ಧಿಯನ್ನು ಪಡೆದಿದ್ದುದರಿಂದ ದೇವೇಂದ್ರನಿಗೆ ಮೃತ್ಯುವೇನೂ ಬರಲಿಲ್ಲ. ಅವನು ವಜ್ರಾಯುಧದಿಂದ ವೃತ್ರನ ಹೊಟ್ಟೆಯನ್ನು ಬಗೆದುಕೊಂಡು ಹೊರಗೆ ಬಂದನು. ಅನಂತರ ವಜ್ರದಿಂದ ಅವನ ತಲೆಯನ್ನೂ ಕತ್ತರಿಸಿ ಹಾಕಿದನು. ಒಡನೆಯೇ ಸ್ವರ್ಗದಿಂದ ಹೂಮಳೆ ಸುರಿಯಿತು. ವೃತ್ರನ ದೇಹ ನೆಲಕ್ಕೆ ಬೀಳುತ್ತಲೆ ಅದರಿಂದ ಒಂದು ಬೆಳಕು ಹೊರಕ್ಕೆ ಹೊರಟು ವೈಕುಂಠವನ್ನು ಪ್ರವೇಶಿಸಿತು.

ವೃತ್ರನು ಸತ್ತುದನ್ನು ಕಂಡು ದೇವತೆಗಳೆಲ್ಲ ನಲಿದು ನರ್ತಿಸುತ್ತಾ ಸ್ವಸ್ಥಾನಗಳಿಗೆ ತೆರಳಿ ದರು. ಆದರೆ ದೇವೇಂದ್ರನಿಗೆ ಮಾತ್ರ ಬ್ರಹ್ಮಹತ್ಯಾದೋಷವು ಬೆನ್ನಟ್ಟಿ ಬಂದಿತು. ಕ್ಷಯ ಯೋಗದಿಂದ ಕೂಡಿದ ಮುದುಕಿಯ ರೂಪದಿಂದ ಕಾಣಿಸಿಕೊಂಡ ಆ ಪಾಪವು, ಬಿಳಿ ಕೂದಲುಗಳನ್ನು ಬಿರಿ ಹೊಯ್ದುಕೊಂಡು, ತನ್ನ ಉಸಿರಿನಿಂದ ದಿಕ್ಕುದಿಕ್ಕಿಗೂ ದುರ್ವಾಸನೆ ಯನ್ನು ಚೆಲ್ಲುತ್ತಾ ‘ನಿಲ್ಲು, ನಿಲ್ಲು’ ಎಂದು ಕೂಗಿಕೊಂಡು ದೇವೇಂದ್ರನ ಬಳಿಗೆ ಓಡಿ ಬರುತ್ತಿತ್ತು. ಅದರಿಂದ ತಪ್ಪಿಸಿಕೊಳ್ಳುವುದಕ್ಕಾಗಿ ಆತನು ದಿಕ್ಕುದಿಕ್ಕುಗಳಲ್ಲಿಯೂ ಓಡಿ ಓಡಿ ದಣಿದು, ಕೊನೆಗೆ ಈಶಾನ್ಯ ದಿಕ್ಕಿನಲ್ಲಿ ಕಣ್ಣಿಗೆ ಬಿದ್ದ ಮಾನಸ ಸರೋವರವನ್ನು ಹೊಕ್ಕು ಅಲ್ಲಿಯೇ ಅಡಗಿಕೊಂಡಿದ್ದನು. ಪಾಪ, ಅಲ್ಲಿಗೆ ಅವನ ದೂತನಾದ ಅಗ್ನಿಯು ಪ್ರವೇಶಿಸುವಂತಿಲ್ಲ; ಆದ್ದರಿಂದ ಹೊಟ್ಟೆಗೆ ಹಿಟ್ಟಿಲ್ಲದೆ ಸಹಸ್ರವರ್ಷಗಳವರೆಗೆ ಅವನು ಅಲ್ಲಿಯೇ ಅಡಗಿಕೊಂಡು ಕುಳಿತಿರಬೇಕಾಯಿತು. ಆಗ ಸ್ವರ್ಗಲೋಕವು ರಾಜನಿಲ್ಲದೆ ಅನಾಯಕವಾಗುವ ಸಂದರ್ಭ ಬಂದುದರಿಂದ ವಿದ್ಯೆ, ಯೋಗ, ತಪಸ್ಸುಗಳ ಬಲವನ್ನು ಪಡೆದಿದ್ದ ನಹುಷನು ಇಂದ್ರನ ಗದ್ದುಗೆಯನ್ನೇರಿದನು. ಇಂದ್ರಪದವಿಯ ಅಧಿಕಾರ, ವೈಭವ, ಭೋಗಭಾಗ್ಯಗಳನ್ನು ಕಂಡು ಅವನ ತಲೆ ತಿರುಗಿತು. ಅವನು ಇಂದ್ರನ ಮಡದಿ ಯಾದ ಶಚೀದೇವಿಯ ಮೇಲೆ ಮೋಹಗೊಂಡು, ಅದರ ಫಲವಾಗಿ ಸರ್ಪಜನ್ಮವನ್ನು ಪಡೆದು ಭೂಮಿಗೆ ಬಿದ್ದು ಹೋದನು.

ಅತ್ತ ಮಾನಸ ಸರೋವರದಲ್ಲಿ ತಲೆಮರೆಸಿಕೊಂಡಿದ್ದ ದೇವೇಂದ್ರನು ಪರಮಾತ್ಮ ನನ್ನು ಧ್ಯಾನ ಮಾಡಿ, ತನ್ನ ಪಾಪವನ್ನು ಕಳೆದುಕೊಂಡನು. ಈಶಾನ್ಯ ದಿಕ್ಕಿಗೆ ಒಡೆಯನಾದ ರುದ್ರನೂ ಲಕ್ಷ್ಮೀ ದೇವಿಯೂ ಆತನನ್ನು ಬ್ರಹ್ಮಹತ್ಯೆಯಿಂದ ಆವರೆಗೆ ರಕ್ಷಿಸಿದ್ದರು. ಪಾಪವನ್ನು ಕಳೆದುಕೊಂಡ ದೇವೇಂದ್ರ ನಿರ್ಭಯವಾಗಿ ತನ್ನ ಲೋಕಕ್ಕೆ ಹಿಂದಿರುಗಿದ. ಪುಷಿಗಳು ಆತನಿಂದ ಅಶ್ವಮೇಧ ಯಾಗವನ್ನು ಮಾಡಿಸಿದರು. ಆ ಯಾಗದ ಪುಣ್ಯಫಲ ದಿಂದ ಬೆಟ್ಟದಂತಿದ್ದ ಆತನ ಪಾಪ ಮಂಜಿನಂತೆ ಕರಗಿಹೋಯಿತು. ಆತನು ಸ್ವರ್ಗಕ್ಕೆ ಅಧಿಪತಿಯಾಗಿ ಸುಖವಾಗಿದ್ದನು.

