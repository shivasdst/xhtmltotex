
\chapter{೫೯. ಶ್ರೀಕೃಷ್ಣ ಸಂಹಾರಕ್ಕೆ ಮಹಾಯೋಜನೆ}

ಗೂಳಿಯಾಗಿ ಬಂದ ಅರಿಷ್ಟ ಕೃಷ್ಣನಿಂದ ಸತ್ತನೆಂಬ ಸುದ್ದಿ ಜನಜನಿತವಾಗಿ ಕಂಸನಿಗೂ ಗೊತ್ತಾಯಿತು. ತನ್ನ ಕಡೆಯ ರಕ್ಕಸರು ಒಬ್ಬೊಬ್ಬರಾಗಿ ಎಳೆಯ ಬಾಲಕನಾದ ಶ್ರೀಕೃಷ್ಣನಿಗೆ ಆಹುತಿಯಾಗುತ್ತಿರುವುದನ್ನು ಕಂಡು ಆತನಿಗೆ ಆಶ್ಚರ್ಯವಾಯಿತು, ಭಯವೂ ಆಯಿತು. ಆತನು ಚಿಂತೆಯಲ್ಲಿ ಮುಳುಗಿ ಕುಳಿತಿರಲು, ಹೇಳಿ ಕಳುಹಿಸಿದಂತೆ ನಾರದ ಮಹರ್ಷಿಗಳು ಆತನ ಬಳಿಗೆ ಬಂದರು. ಅವರು ಕಂಸನನ್ನು ಕುರಿತು ‘ಅಯ್ಯಾ, ಆ ಕೃಷ್ಣ ಯಾರೆಂದು ತಿಳಿದಿದ್ದಿ? ಅವನೇ ದೇವಕಿಯ ಎಂಟನೆಯ ಮಗ. ನೀನು ಕೊಲ್ಲುವುದಕ್ಕೆ ಹೊರಟ ಆ ಹೆಣ್ಣುಮಗು ದೇವಕಿಯದಲ್ಲ, ಯಶೋದೆಯದು. ದೇವಕಿಯು ಮಗುವನ್ನು ಹೆತ್ತೊಡ ನೆಯೆ ವಸುದೇವ ಅದನ್ನು ಗುಟ್ಟಾಗಿ ನಂದಗೋಕುಲಕ್ಕೆ ಸಾಗಿಸಿದ. ಅಲ್ಲಿಂದ ಯಶೋದೆ ಹೆತ್ತ ಮಗುವನ್ನು ಇಲ್ಲಿಗೆ ಹೊತ್ತು ತಂದ. ಅವನ ಇನ್ನೊಬ್ಬ ಹೆಂಡತಿ ರೋಹಿಣಿಯ ಮಗನೆ ಬಲರಾಮನೆಂಬುವನು. ನಂದಗೋಕುಲದಲ್ಲಿ ಗುಟ್ಟಾಗಿ ಬೆಳೆಯುತ್ತಿರುವ ಆ ಮಕ್ಕಳಿಬ್ಬರೆ ನಿನ್ನ ಕಡೆಯ ರಕ್ಕಸರನ್ನೆಲ್ಲ ಕೊಂದುಹಾಕಿದವರು. ಇಗೋ, ಇರುವ ವಿಚಾರ ವನ್ನು ನಿನಗೆ ತಿಳಿಸಿದ್ದೇನೆ. ಇನ್ನು ನಿನ್ನ ಕ್ಷೇಮಕ್ಕೆ ಏನು ಬೇಕೋ ಅದನ್ನು ಮಾಡುವುದು ನಿನ್ನ ಹೊಣೆ’ ಎಂದು ಹೇಳಿದರು. ಆ ಮಾತುಗಳನ್ನು ಕೇಳಿ ಕಂಸನ ರೋಷ ಹೊತ್ತಿ ಧಗಧಗ ಉರಿಯಿತು. ಅವನು ವಸುದೇವನನ್ನು ತಕ್ಷಣವೆ ಕತ್ತರಿಸಿಹಾಕಬೇಕೆಂದುಕೊಂಡು ಹಿರಿದ ಕತ್ತಿಯೊಡನೆ ಮೇಲಕ್ಕೆದ್ದನು. ಆದರೆ ನಾರದರು ಅವನನ್ನು ಕುರಿತು ‘ಅಯ್ಯಾ, ವಸುದೇವ ನನ್ನು ಕೊಂದರೆ ಕೃಷ್ಣಬಲರಾಮರು ತಲೆ ತಪ್ಪಿಸಿಕೊಳ್ಳುತ್ತಾರೆ. ಆದ್ದರಿಂದ ಆ ಕೆಲಸ ಮಾಡಬೇಡ’ ಎಂದು ಹೇಳಿ, ಉಪಾಯದಿಂದ ಅದನ್ನು ತಪ್ಪಿಸಿದನು. ಕಂಸನು ತನ್ನ ಕೋಪವನ್ನು ತಡೆದುಕೊಳ್ಳಲಾರದೆ, ತಕ್ಷಣವೆ ವಸುದೇವ ದೇವಕಿಯರನ್ನು ಅಲ್ಲಿಗೆ ಹಿಡಿ ತರಿಸಿ ಸಂಕೋಲೆಗಳೊಡನೆ ಅವರನ್ನು ಸೆರೆಯಲ್ಲಿಟ್ಟನು.

ತಾವು ಬಂದ ಕೆಲಸ ನೆರವೇರುತ್ತಲೆ ನಾರದರು ಕಂಸನಿಂದ ಬೀಳ್ಕೊಂಡು ಲೋಕ ಸಂಚಾರಕ್ಕೆ ಹೊರಟರು. ಅವರು ಅತ್ತ ಹೋಗುತ್ತಲೆ ಇತ್ತ ಕಂಸನು ಕೇಶಿಯೆಂಬ ತನ್ನ ಗೆಳೆಯನನ್ನು ಕರೆಸಿ, ಗೋಕುಲದಲ್ಲಿರುವ ರಾಮಕೃಷ್ಣರನ್ನು ತಕ್ಷಣವೆ ಕೊಂದುಬರುವಂತೆ ಅವನನ್ನು ನೇಮಿಸಿದನು. ಅನಂತರ ತನ್ನ ಗೆಳೆಯರನ್ನೆಲ್ಲ ಕರೆಸಿ, ಬಲರಾಮಕೃಷ್ಣರನ್ನು ಕೊಲ್ಲುವುದಕ್ಕಾಗಿ ದೊಡ್ಡ ಯೋಜನೆಯೊಂದನ್ನು ಗೊತ್ತುಮಾಡಿದನು. ಮೊಟ್ಟಮೊದಲ ನೆಯದಾಗಿ ಮುಷ್ಟಿಕ, ಚಾಣೂರ–ಎಂಬ ತನ್ನ ಜಟ್ಟಿಗಳಿಬ್ಬರನ್ನು ಕುರಿತು ‘ಅಯ್ಯಾ, ವೀರರೆ, ಗೋಕುಲದಲ್ಲಿರುವ ಬಲರಾಮಕೃಷ್ಣರು ನನ್ನ ಪಾಲಿಗೆ ಮೃತ್ಯುವೆಂದು ಗೊತ್ತಾ ಗಿದೆ. ಅವರಿಬ್ಬರನ್ನೂ ಹೆಸರಿಲ್ಲದಂತೆ ಮಾಡಿಬಿಡಬೇಕು. ಅದಕ್ಕಾಗಿ ಅವರಿಬ್ಬರನ್ನೂ ಇಲ್ಲಿಗೆ ಉಪಾಯವಾಗಿ ಕರೆಸುತ್ತೇನೆ. ನೀವು ಅವರೊಡನೆ ಮಲ್ಲಯುದ್ಧಮಾಡುವ ನೆಪ ದಿಂದ ಅವರನ್ನು ಕೊಂದುಹಾಕಬೇಕು. ಅದಕ್ಕಾಗಿ ನೀವು ಈಗಿಂದೀಗಲೆ ಜಟ್ಟಿ ಕಾಳಗದ ಕಣವನ್ನು ವಿಸ್ತಾರವಾಗಿ ರಚಿಸಿ, ಅದರ ಸುತ್ತಲೂ ಸಹಸ್ರಾರು ಜನ ಕೂತು ನೋಡುವುದಕ್ಕೆ ಅನುಕೂಲಿಸುವಂತೆ ಪೀಠಗಳನ್ನು ಸಿದ್ಧಪಡಿಸಿರಿ. ರಾಜಧಾನಿಯಲ್ಲಿರುವವರು ಮಾತ್ರವೇ ಅಲ್ಲದೆ ದೇಶದ ಜನರೆಲ್ಲರೂ ಅದನ್ನು ನೋಡಿ ಸಂತೋಷಪಡಲಿ’ ಎಂದನು. ಅನಂತರ ಆನೆಯ ಮಾವಟಿಗರ ಮುಖ್ಯಸ್ಥನನ್ನು ಕರೆದು ‘ಎಲಾ, ಮಾಹಾಮಾತ್ರಾ, ನಾನು ಬಲರಾಮ ಕೃಷ್ಣರನ್ನು ಇಲ್ಲಿಗೆ ಕರೆಸಿದಾಗ, ನೀನು ಕುವಲಯವೆಂಬ ನಮ್ಮ ಮದ್ದಾನೆಯನ್ನು ಜಟ್ಟಿಯ ಕಾಳಗ ನಡೆಯುವ ಕಣದ ಬಳಿಯಲ್ಲಿಯೆ ನಿಲ್ಲಿಸಿಕೊಂಡಿದ್ದು, ರಾಮಕೃಷ್ಣರು ಅಲ್ಲಿಗೆ ಬರುತ್ತಲೆ ಅದರ ಕೈಯಿಂದ ಅವರನ್ನು ಕೊಲ್ಲಿಸಬೇಕು’ ಎಂದನು. ಅನಂತರ ಆತನು ಮಂತ್ರಿಗಳತ್ತ ತಿರುಗಿ “ಗೆಳೆಯರೆ ಆ ರಾಮಕೃಷ್ಣರು ಸತ್ತಮೇಲೆ ನನ್ನ ಮನಸ್ಸಿಗೆ ಸಮಾಧಾನ. ನೀವು ‘ಧನುರ್ಯಾಗ’ವೆಂಬ ಮಹೋತ್ಸವಕ್ಕೆ ಸಕಲ ಸಿದ್ಧತೆಗಳನ್ನೂ ಮಾಡಿರಿ. ಆ ಉತ್ಸವದ ನೆಪದಿಂದಲೆ ಆ ಹುಡುಗರಿಬ್ಬರನ್ನೂ ಇಲ್ಲಿಗೆ ಹಿಡಿತರಿಸುತ್ತೇನೆ” ಎಂದನು.

ಕಂಸನ ಮಹಾಯೋಜನೆಯ ಮುಖ್ಯ ಅಂಗವೆಂದರೆ ಬಲರಾಮಕೃಷ್ಣರನ್ನು ಉಪಾಯ ದಿಂದ ಮಧುರೆಗೆ ಹಿಡಿತರಿಸುವುದು. ಈ ಕಾರ್ಯಕ್ಕೆ ಆತನು ಅಕ್ರೂರನೆಂಬ ಯಾದವನನ್ನು ನೇಮಿಸಿದನು. ಆತನನ್ನು ಕುರಿತು ಕಂಸನು ‘ಮಿತ್ರ, ಅಕ್ರೂರ, ನನಗೆ ನಿನ್ನಂತಹ ಆಪ್ತರು ಬೇರೊಬ್ಬರಿಲ್ಲ, ದೇವೇಂದ್ರನು ತನ್ನ ಕಾರ್ಯಸಾಧನೆಗೆ ಮಹಾವಿಷ್ಣುವನ್ನು ಆಶ್ರಯಿಸು ವಂತೆ ನಾನು ನನ್ನ ಕಾರ್ಯಸಾಧನೆಗೆ ನಿನ್ನನ್ನು ಆಶ್ರಯಿಸುತ್ತಿದ್ದೇನೆ. ನೀನು ಈಗಿಂದೀಗಲೆ ನಂದಗೋಕುಲಕ್ಕೆ ಹೋಗಬೇಕು. ಅದಕ್ಕಾಗಿ ನನ್ನ ರಥವನ್ನು ನಿನ್ನ ಮನೆಯ ಮುಂದೆ ತರಿಸಿ ನಿಲ್ಲಿಸಿದ್ದೇನೆ. ನೀನು ಆ ರಥದಲ್ಲಿ ಕುಳಿತು ಹೋಗಿ, ಅದೇ ರಥದಲ್ಲಿಯೆ ಬಲರಾಮ ಕೃಷ್ಣರನ್ನು ಕೂಡಿಸಿಕೊಂಡು ಬರಬೇಕು. ಈಗ ನಂದಗೋಪನು ಕಪ್ಪಕಾಣಿಕೆಗಳೊಡನೆ ಹೇಗೂ ಇಲ್ಲಿಗೆ ಬರುವವನಾಗಿದ್ದಾನೆ. ಅವನ ಜೊತೆಯಲ್ಲಿ ಆ ಹುಡುಗರನ್ನೂ ಉಪಾಯ ವಾಗಿ ಇಲ್ಲಿಗೆ ಬರುವಂತೆ ಮಾಡಬೇಕು. ನಮ್ಮ ಧನುರ್ಯಾಗದ ಉತ್ಸವವನ್ನು ಬಾಯಲ್ಲಿ ನೀರೂರುವಂತೆ ಬಣ್ಣಿಸಿ, ಆ ಉತ್ಸವಕ್ಕೆ ನನ್ನ ಪರವಾಗಿ ಆಹ್ವಾನಿಸು. ಅವರಿಬ್ಬರೂ ನನಗೆ ಮೃತ್ಯುಸ್ವರೂಪರಾದವರು. ಅವರು ಇಲ್ಲಿಗೆ ಬರುತ್ತಲೆ ನನ್ನ ಜಟ್ಟಿಗಳಿಂದ ಅವರನ್ನು ಕೊಲ್ಲಿಸುವುದಕ್ಕೆ ಹವಣಿಸಿದ್ದೇನೆ. ಅವರಿಬ್ಬರೂ ಹತರಾಗಿಹೋದಮೇಲೆ ಉಳಿದ ವಸು ದೇವನೇ ಮೊದಲಾದ ಯಾದವರೆಲ್ಲರನ್ನೂ–ಕೊನೆಗೆ ನನ್ನ ತಂದೆ ಉಗ್ರಸೇನ, ಚಿಕ್ಕಪ್ಪ ದೇವಕ ಇವರನ್ನು ಕೂಡ–ಕೊಂದು ನನ್ನ ರಾಜ್ಯವನ್ನು ನಿಷ್ಕಂಟಕವಾಗಿ ಮಾಡುತ್ತೇನೆ. ಆಮೇಲೆ ಜರಾಸಂಧನೇ ಮೊದಲಾದ ಗುರುಹಿರಿಯರೊಡನೆಯೂ, ಶಂಬರ, ನರಕ, ಬಾಣ ಮೊದಲಾದ ಗೆಳೆಯರೊಡನೆಯೂ ಈ ಅಖಂಡ ಭೂಮಂಡಲದ ಚಕ್ರವರ್ತಿಯಾಗಿ ಮೆರೆ ಯುತ್ತೇನೆ. ಮಿತ್ರ, ನಾನು ಈಗ ಹೇಳಿದುದೆಲ್ಲ ಗುಟ್ಟಾಗಿರಲಿ, ನೀನು ತಕ್ಷಣವೇ ಇಲ್ಲಿಂದ ಹೊರಡು’ ಎಂದನು. ಪರಮಸಾಧುವಾದ ಅಕ್ರೂರನಿಗೆ ಕಂಸನ ಮಾತುಗಳು ಹಿಡಿಸಲಿಲ್ಲ. ಆದರೆ, ಕಂಸನ ನುಡಿಗೆ ಎದುರುತ್ತರ ಕೊಡುವ ಗಂಡುತನ ಆತನಿಗಿರಲಿಲ್ಲ. ‘ಆಗಲಿ’ ಎಂದು ಹೇಳಿ, ಆತನಿಂದ ಬೀಳ್ಕೊಂಡು ಅಲ್ಲಿಂದ ಹೊರಟನು.

ಅತ್ತ ಬಲರಾಮಕೃಷ್ಣರನ್ನು ಕೊಲ್ಲಲೆಂದು ಆಜ್ಞಪ್ತರಾದ ಕೇಶಿರಕ್ಕಸನು ಭಯಂಕರಾ ಕಾರದ ಒಂದು ಕುದುರೆಯಾಗಿ ನಂದಗೋಕುಲವನ್ನು ಹೊಕ್ಕನು. ಅವನ ಕಾಲ್ನಡಿಗೆಗೆ ಭೂಮಿ ಅದುರುತ್ತಿತ್ತು, ಅವನು ಕೆನೆದನೆಂದರೆ ದಿಕ್ಕುಗಳು ಕಿವುಡಾಗುತ್ತಿದ್ದವು; ಬಂಡಿಯ ಗಾಲಿಯಂತಿರುವ ಅವನ ಕಣ್ಣುಗಳೊ! ಗುಹೆಯಂತಿರುವ ಅವನ ಬಾಯೊ! ಪರ್ವತ ದಂತಿರುವ ಅವನ ದೇಹದಿಂದ ಶಿಖರದಂತೆ ಮೇಲೆದ್ದಿರುವ ಅವನ ಕಿವಿಗಳೊ! ಈ ಘೋರ ರೂಪಿನ ರಕ್ಕಸ ಗೋಕುಲವನ್ನು ಪ್ರವೇಶಿಸುತ್ತಿದ್ದಂತೆ, ಅವನು ಅರಸಹೊರಟಿದ್ದ ಶ್ರೀಕೃಷ್ಣನೇ ಅವನಿಗೆ ಇದಿರಾದನು. ಒಡನೆಯೆ ಆ ರಕ್ಕಸ ಭೂಮಿ ಬಿರಿಯುವಂತೆ ಒಮ್ಮೆ ಗರ್ಜಿಸಿ, ಬಿರುಗಾಳಿಯಂತೆ ನುಗ್ಗಿಬಂದವನೆ ತನ್ನ ಹಿಂಗಾಲುಗಳನ್ನೆತ್ತಿ ಶ್ರೀಕೃಷ್ಣನನ್ನು ಬಡಿದಪ್ಪಳಿಸುವುದಕ್ಕೆ ಹೊರಟನು. ಶ್ರೀಕೃಷ್ಣನು ಛಂಗನೆ ಹಾರಿ ಅದರಿಂದ ತಪ್ಪಿಸಿಕೊಂಡ ವನೆ ಆ ರಕ್ಕಸನ ಹಿಂಗಾಲುಗಳೆರಡನ್ನೂ ಹಿಡಿದು ಗಿರಿಗಿರಿ ತಿರುಗಿಸಿ, ನೂರು ಮಾರು ದೂರಕ್ಕೆ ಎಸೆದನು. ಕೆಳಕ್ಕೆ ಬಿದ್ದು ಮೈಯೆಲ್ಲ ಜಜ್ಜಿಹೋದ ಆ ರಕ್ಕಸ ಮೆಲ್ಲಗೆ ಮೇಲಕ್ಕೆದ್ದು, ಶ್ರೀಕೃಷ್ಣನನ್ನು ನುಂಗುವವನಂತೆ ತನ್ನ ಬಾಯನ್ನು ತೆರೆದುಕೊಂಡು, ಮತ್ತೆ ಆತನ ಮೇಲೆ ನುಗ್ಗಿ ಬಂದನು. ಈಬಾರಿ ಶ್ರೀಕೃಷ್ಣನು ತನ್ನ ತೋಳನ್ನು ಅವನ ಬಾಯಲ್ಲಿ ತುರುಕಿದನು. ಹುತ್ತವನ್ನು ಹೊಕ್ಕ ಕಾಳಸರ್ಪದಂತೆ ಅವನ ಬಾಯನ್ನು ಹೊಕ್ಕ ಆ ತೋಳು, ಅಲ್ಲಿ ಉಬ್ಬುತ್ತಾ ಹೋಯಿತು, ರಕ್ಕಸನಿಗೆ ಉಸಿರಾಡಲು ಅವಕಾಶವಿಲ್ಲದಂತಾಯಿತು, ಮೈಯೆಲ್ಲ ಬೆವತುಹೋಯಿತು, ಕಣ್ಣುಗುಡ್ಡೆಗಳು ಬೆಳ್ಳಗಾದವು, ಅವನು ಸತ್ತು ನೆಲಕ್ಕುರುಳಿದ.

ಕೇಶಿರಕ್ಕಸ ಸತ್ತು ನೆಲಕ್ಕುರುಳುತ್ತಿದ್ದಂತೆ ನಾರದರ ಸವಾರಿ ಅಲ್ಲಿಗೆ ಚಿತ್ತೈಸಿತು. ಅವರು ಶ್ರೀಕೃಷ್ಣನೊಡನೆ ಗುಟ್ಟಾಗಿ ‘ಹೇ ಜಗನ್ನಾಯಕ, ಸರ್ವಾಧಾರ, ಭಕ್ತರಕ್ಷಕ, ವಾಸುದೇವ! ದುಷ್ಟಶಿಕ್ಷಣ ಶಿಷ್ಟರಕ್ಷಣೆಗಾಗಿ ಅವತರಿಸಿರುವ ನೀನು ಈ ಕೇಶಿರಕ್ಕಸನನ್ನು ಕೊಂದುದನ್ನು ಕಂಡು ತುಂಬ ಸಂತೋಷವಾಯಿತು. ನಾಳೆ ನಿನ್ನನ್ನು ಮಧುರೆಗೆ ಕರೆದೊಯ್ಯುವುದಕ್ಕಾಗಿ ಅಕ್ರೂರನು ಇಲ್ಲಿಗೆ ಬರುತ್ತಾನೆ. ಇದೇ ನೆಪವಾಗಿ ಕಂಸವಧೆ ನಡೆಯಬೇಕಾಗಿದೆ. ಇಂತಹ ಕಾರ್ಯವನ್ನು ಇನ್ನೂ ಎಷ್ಟೋ ನೀನು ಮಾಡುವವನಾಗಿರುವೆ. ಹೀಗೆ ಮಾನವ ರೂಪಿನಿಂದ ಲೀಲೆಯನ್ನು ನಡೆಸುತ್ತಿರುವ ನಿನಗೆ ನಮೋ ನಮೋ’ ಎಂದು ಹೇಳಿ, ಆತನಿಂದ ಅಪ್ಪಣೆ ಯನ್ನು ಪಡೆದು ಹೊರಟುಹೋದರು.

ನಾರದರು ಅತ್ತ ಹೋಗುತ್ತಲೆ, ಇತ್ತ ಶ್ರೀಕೃಷ್ಣನು ಗೊಲ್ಲಬಾಲಕರೊಡನೆ ದನಗಳನ್ನು ಅಟ್ಟಿಕೊಂಡು ಆಡವಿಗೆ ಹೊರಟನು. ಅಲ್ಲಿ ಕಾಲ ಕಳೆಯುವುದಕ್ಕಾಗಿ ಅವರೆಲ್ಲ ಕಳ್ಳರಾಟ ಆಡುತ್ತಿದ್ದರು. ಅವರಲ್ಲಿ ಕೆಲವರು ಕುರಿಗಳಾಗುವುದು, ಮತ್ತೆ ಕೆಲವರು ಕಳ್ಳರಾಗಿ ಅವರನ್ನು ಹೊತ್ತುಕೊಂಡು ಹೋಗುವುದು, ಇನ್ನು ಕೆಲವರು ಕುರುಬರಾಗಿ, ಆ ಕಳ್ಳರನ್ನು ಹಿಡಿದು ಕುರಿಗಳನ್ನು ಬಿಡಿಸಿಕೊಂಡು ಬರುವುದು–ಹೀಗೆ ಅವರು ಮೈಮರೆತು ಆಡುತ್ತಿರು ವಾಗ ವ್ಯೋಮನೆಂಬ ಒಬ್ಬ ರಕ್ಕಸ ಗೊಲ್ಲ ಬಾಲಕನ ವೇಷದಿಂದ ಅವರ ಮಧ್ಯೆ ಸೇರಿ ಕೊಂಡು, ಕಳ್ಳನ ವೇಷದಿಂದ ಕುರಿಗಳಿಗೆ ಬದಲಾಗಿ ಬಾಲಕರನ್ನು ಕದಿಯಹೊರಟನು. ಹೀಗೆ ಕದ್ದ ಹುಡುಗರನ್ನೆಲ್ಲ ಅವನು ಒಂದು ಪರ್ವತದ ಗುಹೆಯಲ್ಲಿ ಬಚ್ಚಿಟ್ಟು ಅದರ ಬಾಯಿಗೆ ಒಂದು ದೊಡ್ಡ ಬಂಡೆಯನ್ನು ಮುಚ್ಚಿಟ್ಟು ಬರುತ್ತಿದ್ದನು. ಸ್ವಲ್ಪ ಹೊತ್ತಿನೊಳ ಗಾಗಿ ಕೇವಲ ನಾಲ್ಕೈದು ಬಾಲಕರು ಮಾತ್ರ ಉಳಿದು, ಉಳಿದವರೆಲ್ಲ ಗವಿಯ ಪಾಲಾದರು. ಶ್ರೀಕೃಷ್ಣನು ಇದನ್ನು ಗಮನಿಸಿ, ಅವನನ್ನು ಕೈಹಿಡಿದು ನಿಲ್ಲಿಸಲು, ಆ ರಕ್ಕಸ ಸ್ವಸ್ವರೂಪ ವನ್ನು ತಳೆದು ಶ್ರೀಕೃಷ್ಣನ ಮೇಲೆ ಬಿದ್ದನು. ಆದರೆ ಶ್ರೀಕೃಷ್ಣನು ಕ್ಷಣಮಾತ್ರದಲ್ಲಿ ಆ ರಕ್ಕಸನನ್ನು ಕೊಂದು, ಗುಹೆಯಲ್ಲಿದ್ದ ಗೊಲ್ಲರ ಹುಡುಗರನ್ನೆಲ್ಲ ಹಿಂದಕ್ಕೆ ಕರೆತಂದನು.

ಮರುದಿನ ಸಂಜೆ ಅಕ್ರೂರ ಗೋಕುಲಕ್ಕೆ ಬಂದ. ತಕ್ಷಣವೇ ಹೊರಡುವಂತೆ ಕಂಸನು ಹೇಳಿದ್ದರೂ ಆತನು ಮಧುರೆಯನ್ನು ಬಿಟ್ಟು ಹೊರಟುದು ಮರುದಿನವೇ. ಶ್ರೀಕೃಷ್ಣನ ಅದ್ಭುತ ಚರಿತ್ರೆಯನ್ನು ಸಾಕಷ್ಟು ಕೇಳಿದ್ದ, ಶ್ರೀಕೃಷ್ಣನೆಂದರೆ ಸಾಕ್ಷಾತ್​ಪರಮೇಶ್ವರನೇ ಎಂದು ಅರ್ಥಮಾಡಿಕೊಂಡಿದ್ದ. ಕಂಸನ ಭಯದಿಂದ ಇಂದಿನವರೆಗೆ ನೋಡಲಾಗದಿದ್ದ ಶ್ರೀಕೃಷ್ಣಪರಮಾತ್ಮನನ್ನು ಇಂದು ತಾನು ಕಣ್ಣಾರೆ ಕಾಣುವುದು ಮಾತ್ರವೇ ಅಲ್ಲ, ಆತ ನೊಡನೆ ಮಾತನಾಡಬಹುದು, ಆತನ ಪಾದಗಳನ್ನು ಮುಟ್ಟಿ ನಮಸ್ಕರಿಸಬಹುದು. “ಯೋಗಿಗಳ ಮನಸ್ಸಿಗೆ ಕೂಡ ಅಗೋಚರನಾದ ಆ ಪರಮಪುರುಷನು ಇಂದು ಗೊಲ್ಲರ ಮಧ್ಯದಲ್ಲಿ ಗೊಲ್ಲನಾಗಿ ವ್ಯವಹರಿಸುತ್ತಿರುವನಲ್ಲಾ! ಏನು ಅವನ ಲೀಲೆ! ನಾನು ಪೂರ್ವಜನ್ಮದಲ್ಲಿ ಏನು ಪುಣ್ಯವನ್ನು ಮಾಡಿದ್ದೆನೊ, ನನಗಿಂದು ಆತನ ದರ್ಶನಭಾಗ್ಯ ದೊರೆಯುತ್ತದೆ. ನಾನು ಗೋಕುಲಕ್ಕೆ ಹೋದಕೊಡಲೆ ಮೊದಲು ಆ ಶ್ರೀಕೃಷ್ಣನನ್ನು ಕಣ್ತುಂಬ, ಕಣ್ದಣಿಯೆ ನೋಡಿಬಿಡುತ್ತೇನೆ. ಆಮೇಲೆ ಆತನ ಪಾದಗಳನ್ನು ಮುಟ್ಟಿ ನಮಸ್ಕರಿಸುತ್ತೇನೆ. ಆಗ ಶ್ರೀಕೃಷ್ಣ ನನ್ನ ತಲೆಯಮೇಲೆ ತನ್ನ ಅಮೃತಹಸ್ತವನ್ನು ಇಡುತ್ತಾನೆ. ನಾನು ಕೈ ಜೋಡಿಸಿ ಇದಿರಿಗೆ ನಿಂತುಕೊಳ್ಳುತ್ತೇನೆ. ಆತ ಮುಗುಳ್ನಗೆಯೊಡನೆ ನನ್ನ ಕಡೆ ನೋಡು ತ್ತಾನೆ. ನಾನು ಆತನಿಗೆ ಬಂಧುವಾದುದರಿಂದ ನನ್ನನ್ನು ಬರಸೆಳೆದು ಅಪ್ಪಿಕೊಂಡರೂ ಅಪ್ಪಿ ಕೊಳ್ಳಬಹುದು. ‘ಅಕ್ರೂರಾ, ಎಲ್ಲರೂ ಆರೋಗ್ಯವೆ?’ ಎಂದು ಕುಶಲ ಪ್ರಶ್ನೆಯನ್ನೂ ಮಾಡಬಹುದು. ನಾನು ಕಂಸನ ಕಡೆಯವನಾದರೂ ಸರ್ವಾಂತರ್ಯಾಮಿಯಾದ ಆತನಿಗೆ ನನ್ನ ಮನಸ್ಸು ಎಂತಹದೆಂಬುದು ಗೊತ್ತಾಗದೆ ಇರುತ್ತದೆಯೆ? ಆತನ ಭಕ್ತನಾದ ನನ್ನನ್ನು ಪ್ರೀತಿಯಿಂದಲೆ ಕಾಣುತ್ತಾನೆ. ಆತನನ್ನು ಕಂಡಮೇಲೆ ಬಲರಾಮನನ್ನು ಕಂಡು ನಮಸ್ಕರಿ ಸುತ್ತೇನೆ. ಆತನೂ ನನ್ನನ್ನು ಆದರಿಸಿ, ಕಂಸನ ವಿಚಾರವಾಗಿ, ಪ್ರಶ್ನಿಸುತ್ತಾನೆ.” ಹೀಗೆಂದು ಅಕ್ರೂರನು ಬೀದಿಯುದ್ದಕ್ಕೂ ಮನಸ್ಸಿನಲ್ಲಿಯೆ ಸ್ವರ್ಗಸುಖವನ್ನು ಸವಿಯುತ್ತಾ ಗೋಕುಲವನ್ನು ಬಂದು ಮುಟ್ಟಿ ದನು.

ಅಕ್ರೂರನು ಗೋಕುಲವನ್ನು ಸೇರಿದಾಗ ಸೂರ್ಯನು ಆಗತಾನೆ ಮುಳುಗುತ್ತಿದ್ದ. ಆತನ ಹೃದಯವೆಲ್ಲವೂ ಶ್ರೀಕೃಷ್ಣಮಯವಾಗಿತ್ತು. ಭಕ್ತಿಭಾವದಿಂದ ತುಂಬಿ ತುಳುಕುತ್ತಿದ್ದ ಆತನ ಕಣ್ಣಿಗೆ ಹಾದಿಯಲ್ಲಿ ಬಿದ್ದಿದ್ದ ಹೆಜ್ಜೆಯ ಗುರುತೊಂದು ಶ್ರೀಕೃಷ್ಣನದಿರಬೇಕೆನಿಸಿತು. ಆ ಹೆಜ್ಜೆಯಲ್ಲಿ ಪದ್ಮ, ಅಂಕುಶ ಇತ್ಯಾದಿ ರೇಖೆಗಳು ಮೂಡಿದ್ದುವು. ಇದನ್ನು ಕಾಣುತ್ತಲೆ ಆತನು ಆನಂದಬಾಷ್ಪಗಳನ್ನು ಸುರಿಸುತ್ತಾ ರಥದಿಂದ ಕೆಳಗೆ ಧುಮ್ಮಿಕ್ಕಿ, ಆ ಹೆಜ್ಜೆಯ ಧೂಳಿಯನ್ನು ಹಣೆಯಲ್ಲಿ ಧರಿಸಿದನು. ಅನಂತರ ರಥವನ್ನು ಮುಂದಕ್ಕೆ ನಡೆಸುತ್ತಾ ದಾರಿ ಯಲ್ಲಿ ಕಂಡವರನ್ನು ‘ಶ್ರೀಕೃಷ್ಣನ ಮನೆ ಯಾವುದು?’ ಎಂದು ಕೇಳಿಕೊಂಡು ಗೋಕುಲ ವನ್ನು ಪ್ರವೇಶಿಸಿದನು. ಅದರ ಮುಂಭಾಗದಲ್ಲಿಯೇ ಗೋಗಳ ಮಂದೆ, ಆ ಮಂದೆಯಲ್ಲಿ ಬಲರಾಮಕೃಷ್ಣರು ಆಗ ಹಾಲು ಕರೆಯುತ್ತಿದ್ದರು. ನೋಡುತ್ತಲೆ ಇವರು ಇಂತಹವರೆಂದು ಸುಲಭವಾಗಿ ಗುರುತಿಸಬಹುದಾಗಿತ್ತು. ಕೃಷ್ಣ ಕಪ್ಪು, ಬಲರಾಮ ಬಿಳುಪು; ಕೃಷ್ಣ ಉಟ್ಟ ಬಟ್ಟೆ ಹಳದಿ, ರಾಮನದು ನೀಲಿ; ಅವರಿಬ್ಬರೂ ಆಗತಾನೆ ಸ್ನಾನಮಾಡಿ, ಮಡಿಯುಟ್ಟು, ಗಂಧವನ್ನು ಮೈಗೆ ಬಳಿದುಕೊಂಡು ಮಂದೆಯಲ್ಲಿ ಓಡಾಡುತ್ತಿದ್ದರು. ಅವರ ಮುಖಗಳು ತೇಜಸ್ಸಿನಿಂದ ತುಂಬಿ ತುಳುಕುತ್ತಿದ್ದವು. ಸಾಮಾನ್ಯ ಗೊಲ್ಲರಿಗೆ ಆ ತೇಜಸ್ಸು ಎಲ್ಲಿಯದು? ಅವರನ್ನು ಕಾಣುತ್ತಲೆ ಅಕ್ರೂರನು ರಥದಿಂದ ಕೆಳಕ್ಕೆ ಧುಮ್ಮಿಕಿ, ನಿಂತಲ್ಲಿಂದಲೆ ಅವರಿಗೆ ಅಡ್ಡಬಿದ್ದನು. ಅವನ ಮೈ ರೋಮಾಂಚನಗೊಂಡಿತು, ಕಣ್ಣುಗಳು ಆನಂದಬಾಷ್ಪಗಳನ್ನು ಸುರಿಸಿದವು, ಕಂಠ ಗದ್ಗದವಾಗಿ ಮಾತು ಹೊರಡದಂತಾಯಿತು. ಶ್ರೀಕೃಷ್ಣನು ಬಲರಾಮ ನೊಡನೆ ಅವನಿದ್ದಲ್ಲಿಗೆ ಓಡಿಬಂದು, ಅವ ನನ್ನು ಕೈಹಿಡಿದು ತನ್ನ ಮನೆಗೆ ಕರೆತಂದನು. ಅಲ್ಲಿ ನಂದನು ಆತನನ್ನು ಪರಿಪರಿಯಾಗಿ ಉಪಚರಿಸಿ, ಊಟವಾದಮೇಲೆ ಆತನ ಕ್ಷೇಮ ಲಾಭವನ್ನು ವಿಚಾರಿಸುತ್ತಾ ‘ಮಿತ್ರ ಅಕ್ರೂರ, ಆ ಕಂಸನ ಹತ್ತಿರ ಕಟುಕನ ಬಳಿಯ ಕುರಿ ಗಳಂತೆ ನೀವು ಹೇಗೆ ಬದುಕಿರುವಿರೊ!’ ಎಂದು ಅವನ ಮೇಲೆ ಮರುಕವನ್ನು ತೋರಿದನು.

ಅಕ್ರೂರನು ಗೋಕುಲಕ್ಕೆ ಬರುವ ಹಾದಿಯಲ್ಲಿ ಕಂಡ ಹೊಂಗನಸೆಲ್ಲವೂ ನನಸಾ ಯಿತು. ಆತನು ಊಟಮಾಡಿ ಮಂಚದಮೇಲೆ ಮಲಗಿರಲು ಶ್ರೀಕೃಷ್ಣನು ಬಂದು ಆತನ ಪಕ್ಕದಲ್ಲಿ ಕುಳಿತುಕೊಂಡನು. ಅನಂತರ ‘ಅಪ್ಪ ಅಕ್ರೂರ, ಪ್ರಯಾಣ ಸುಖಕರವಾಗಿತ್ತೆ? ನಿನ್ನ ಬಂಧುಬಳಗದವರೆಲ್ಲ ಸುಖವಾಗಿರುವರೆ? ಹೀಗೆಂದು ಕೇಳುವುದಕ್ಕೂ ನನಗೆ ಸಂಕೋಚ. ನಮ್ಮ ಸೋದರಮಾವನೆಂದು ಹೇಳುವ ಆ ಕಂಸನು ರಾಜನಾಗಿರುವಾಗ ನಮ್ಮ ಬಂಧುಗಳಾದವರಿಗೆ ಸುಖವೆಲ್ಲಿ ಬಂತು? ಅಯ್ಯೋ ಪಾಪ, ನನ್ನ ದೆಸೆಯಿಂದ ನನ್ನ ಅಪ್ಪ ಅಮ್ಮನಿಗೆ ಎಂತಹ ಸಂಕಟ ಬಂತು! ಮಕ್ಕಳನ್ನೆಲ್ಲ ಕಳೆದುಕೊಂಡರು, ಸೆರೆಮನೆಗೆ ತುತ್ತಾದರು. ಹೋಗಲಿ, ಅದೆಲ್ಲ ನೆನೆದು ಏನು ಫಲ? ಈಗ ನೀನು ಇಲ್ಲಿಗೆ ಬಂದೆಯಲ್ಲಾ, ನನಗೆ ಅದೇ ಎಷ್ಟೋ ಸಂತೋಷ. ಬಹುದಿನಗಳಿಂದ ನಿನ್ನನ್ನು ನೋಡಬೇಕೆಂದಿದ್ದ ಆಸೆ ಇಂದು ಈಡೇರಿದಂತಾಯಿತು. ಅದು ಹಾಗಿರಲಿ, ಈಗ ಇದ್ದಕ್ಕಿದ್ದಂತೆಯೇ ನೀನು ಬಂದೆ ಯಲ್ಲ, ಏನು ಕಾರಣ?’ ಎಂದು ಕೇಳಿದನು. ಅಕ್ರೂರನು ತಾನು ಬಂದ ಕಾರಣವನ್ನು ಸ್ವಲ್ಪವೂ ಮರೆಮಾಚದೆ “ನಾರದರು ಬಂದು ಶ್ರೀಕೃಷ್ಣ ಬಲರಾಮರು ವಸುದೇವನ ಮಕ್ಕ ಳೆಂದು ಹೇಳಿದರು, ಆ ಮಕ್ಕಳನ್ನು ಕೊಲ್ಲಬೇಕೆಂದು ಕಂಸನ ಮಹಾಯೋಜನೆ”–ಎಂದು ಎಲ್ಲವನ್ನೂ ವಿವರ ವಿವರವಾಗಿ ತಿಳಿಸಿದನು. ಇದನ್ನು ಕೇಳಿ ಶ್ರೀಕೃಷ್ಣನು ಕಂಸನ ಅವಿವೇಕ ಕ್ಕಾಗಿ ನಕ್ಕನು. ಆತನು ನಂದಗೋಪನೊಡನೆ ಉಳಿದಾವುದನ್ನೂ ತಿಳಿಸದೆ, ‘ಕಂಸನು ನಮ್ಮನ್ನು ಧನುರ್ಯಾಗಕ್ಕೆ ಆಹ್ವಾನಿಸಿದ್ದಾನೆ’ ಎಂದು ಮಾತ್ರ ತಿಳಿಸಿದನು. ‘ಬೆಳ್ಳಗಿದ್ದುದೆಲ್ಲ ಹಾಲು’ ಎನ್ನುವಂತಹ ತಿಳಿಮನಸ್ಸಿನವನು, ನಂದ. ಕಂಸ ಮಹಾರಾಜ ಉತ್ಸವಕ್ಕೆ ಆಹ್ವಾನಿ ಸಿರುವನೆಂಬ ಹಿಗ್ಗಿನಿಂದ ಆತನು ಗೋಕುಲದವರೆಲ್ಲ ಬಂಡಿಗಳನ್ನು ಕಟ್ಟಿಕೊಂಡು, ತಮ್ಮಲ್ಲಿ ಸಿಕ್ಕುವ ಹಾಲು ಮೊಸರು ಬೆಣ್ಣೆ ತುಪ್ಪಗಳ ಕೈಗಾಣಿಕೆಗಳೊಡನೆ ಮಧುರೆಗೆ ಹೊರಡುವಂತೆ ಅಪ್ಪಣೆ ಮಾಡಿದನು.

ಅಕ್ರೂರನು ಬಲರಾಮ ಕೃಷ್ಣರನ್ನು ಮಧುರೆಗೆ ಕರೆದೊಯ್ಯುವನೆಂಬುದನ್ನು ಕೇಳಿ ನಂದ ಗೋಕುಲದ ಗೋಪಿಯರಿಗೆಲ್ಲ ಸಿಡಿಲುಬಡಿದಂತಾಯಿತು. ಅವರ ಮುಖ ಕುಂದಿತು, ಮಾತು ನಿಂತಿತು, ನಿಟ್ಟುಸಿರು ಹೊರಹೊರಟಿತು. ಆತನ ಮುದ್ದು ಮುಖ, ಮಂದಹಾಸ ಮೃದುನುಡಿ, ಮನೋಹರ ಲೀಲೆಗಳನ್ನು ನೆನೆದು ಕೆಲವರು ತಡೆಯಲಾರದ ಸಂಕಟದಿಂದ ಮೂರ್ಛೆಹೋದರು; ಮತ್ತೆ ಕೆಲವರು ಅವನನ್ನೆ ಧ್ಯಾನಿಸುತ್ತ ಯೋಗಿ ಗಳಂತೆ ಕಣ್ಮುಚ್ಚಿ ಕುಳಿತರು. ಒಬ್ಬಳು ತನಗೆ ತಾನೆ ‘ಆ ಕಪ್ಪಾದ ಮುಂಗುರುಳು, ಕನ್ನಡಿ ಯಂತಹ ಕೆನ್ನೆ, ಎಸಳಾದ ಮೂಗು, ತಾಪಹಾರಕವಾದ ಆ ಮುಗುಳ್ನಗೆ. ಅದನ್ನು ಮತ್ತೆ ಕಾಣುವುದು ಯಾವಾಗ? ಅಯ್ಯೋ ವಿಧಿ! ಸುಖವನ್ನು ತೋರಿಸಿ, ದುಃಖವನ್ನು ನೀಡಿ ದೆಯಾ?’ ಎಂದು ಅತ್ತಳು. ಮತ್ತೊಬ್ಬ ಮುಗ್ಧೆ ‘ಅಯ್ಯೋ ಅಕ್ರೂರ, ನೀನು ಅಕ್ರೂರ ನಲ್ಲ, ಕ್ರೂರ. ಕಂಸನ ದೂತನಾದ ನೀನು ಅಕ್ರೂರ ಹೇಗಾದೀಯೆ? ನಮ್ಮ ಪ್ರಾಣ ಕ್ಕಿಂತಲೂ ಪ್ರಿಯತಮನಾದ ಶ್ರೀಕೃಷ್ಣನನ್ನು ಇಲ್ಲಿಂದ ಕರೆದೊಯ್ಯುತ್ತಿರುವ ನೀನು ನಮ್ಮ ಪಾಲಿನ ಯಮ’ ಎಂದು ಆಕ್ರೋಶ ಮಾಡಿದಳು. ಮತ್ತೊಬ್ಬ ಹೆಣ್ಣು ‘ಅಮ್ಮ, ಶ್ರೀಕೃಷ್ಣ ಮಧುರೆಗೆ ಹೋದಮೇಲೆ ನಮ್ಮನ್ನೆಲ್ಲಿ ನೆನೆಯುತ್ತಾನೆ? ಪಟ್ಟಣದ ಆ ಬೆಡಗುಗಾತಿ ಯರನ್ನು ಕಂಡ ಮೇಲೆ ಈ ಕಾಡುಹೆಣ್ಣುಗಳನ್ನು ಮರೆಯುವುದು ಸಹಜ. ಮುಖ್ಯವಾಗಿ ಮಧುರೆಯ ಅದೃಷ್ಟವೇ ಅದೃಷ್ಟ’ ಎಂದು ನಿಟ್ಟುಸಿರು ಬಿಟ್ಟಳು. ಮತ್ತೊಬ್ಬ ಗೋಪಿ ‘ಅಯ್ಯೋ, ಹೊರಟು ನಿಂತಿರುವ ಶ್ರೀಕೃಷ್ಣನನ್ನು ಯಾರೂ ತಡೆಯುವವರೆ ಇಲ್ಲವೆ?’ ಎಂದು ರೇಗಿದಳು. ಮತ್ತೊಬ್ಬಳು ‘ದೇವರ ದಯದಿಂದ ಏನಾದರೂ ವಿಘ್ನ ಬಂದು ಈ ಪ್ರಯಾಣ ನಿಂತು ಹೋಗಬಾರದೆ!’ ಎಂದು ಹಾರೈಸಿದಳು. ಅವರಲ್ಲಿ ಒಬ್ಬಳಂತೂ ಗಟ್ಟಿ ಯಾಗಿ ‘ಬನ್ನಿರೆ, ಏನು ನೋಡುತ್ತೀರಿ; ನಾವೆಲ್ಲರೂ ಒಟ್ಟಾಗಿ ಹೋಗಿ ಅವನನ್ನು ತಡೆದು ನಿಲ್ಲಿಸೋಣ! ಭಯ ನಾಚಿಕೆಗಳಿಂದ ಹಿಂಜರಿದರೆ ಕೆಲಸ ಕೆಟ್ಟು ಹೋಗುತ್ತದೆ. ಯಾರು ಏನು ಬೇಕಾದರೂ ಮಾಡಿಕೊಳ್ಳಲಿ, ನಾವು ಬೇರೆ ಕೃಷ್ಣನನ್ನು ಅಗಲಿರಲಾರೆವು’ ಎಂದು ಗುಡುಗಿದಳು. ಅವಳ ಮಾತು ಸರಿಯೆಂದು ತೋರಿತು, ಉಳಿದವರಿಗೆಲ್ಲ. ಅವರೆಲ್ಲರೂ ಶ್ರೀಕೃಷ್ಣನ ರಥದ ಸುತ್ತ ಮುತ್ತಿಕೊಂಡರು.

ಗೋಪಿಯರು ತನ್ನ ಅಗಲಿಕೆಗಾಗಿ ಪರಿತಪಿಸುತ್ತಿರುವುದನ್ನು ಶ್ರೀಕೃಷ್ಣ ಅರ್ಥಮಾಡಿ ಕೊಂಡ. ಆತನು ಕೊಳಲಿನ ಇನಿದನಿಯಲ್ಲಿ ‘ನಾನು ಬಹುಬೇಗ ಹಿಂದಿರುಗಿ ಬರುತ್ತೇನೆ’ ಎಂಬ ಒಂದೇ ಮಾತಿನಿಂದಲೆ ಅವರನ್ನು ಸಮಾಧಾನ ಪಡಿಸಿದ. ಇದನ್ನು ಕೇಳುತ್ತಲೆ ಅವರು ಶ್ರೀಕೃಷ್ಣನನ್ನು ಬೀಳ್ಕೊಡಲು ಸಿದ್ಧರಾದರು. ಆಕ್ರೂರನು ತನ್ನ ಪರಿಚಯದವರಿ ಗೆಲ್ಲ ಹೋಗಿಬರುವುದಾಗಿ ಹೇಳಿ ಬಲರಾಮ ಕೃಷ್ಣರೊಡನೆ ತನ್ನ ರಥವನ್ನೇರಿದನು. ನಂದನೇ ಮೊದಲಾದ ಗೋಪಾಲರು ತಾವು ಸಂಗ್ರಹಿಸಿದ ಹಾಲು ಮೊಸರು ಬೆಣ್ಣೆ ಗಳೊಡನೆ ತಮ್ಮ ತಮ್ಮ ಗಾಡಿಗಳನ್ನು ಹತ್ತಿಕೊಂಡರು. ಗೋಪಿಯರೆಲ್ಲ ತಮ್ಮ ಕಣ್ಣೀರಿ ನೊಡನೆ ‘ಶ್ರೀಕೃಷ್ಣ, ಗೋವಿಂದ, ಮಾಧವ, ಮುರಳೀಧರ’ ಎಂದು ಕೂಗುತ್ತಿದ್ದಂತೆಯೆ ರಥ ಚಲಿಸಿತು, ಅದರ ಹಿಂದೆ ಗಾಡಿಗಳು ಹೊರಟವು. ಶ್ರೀಕೃಷ್ಣನ ರಥ ಕಣ್ಮರೆಯಾಗಿ, ಅದರಿಂದೆದ್ದ ದೂಳು ಅಡಗುವವರೆಗೂ ಗೋಪಿಯರು ನಟ್ಟ ದೃಷ್ಟಿಯಿಂದ ಶ್ರೀಕೃಷ್ಣನು ಹೋದ ದಿಕ್ಕನ್ನೆ ನೋಡುತ್ತ ನಿಂತಿದ್ದರು. ಅನಂತರ ಅವರು ನಿರಾಶರಾಗಿ, ನಡೆವೆಣ ಗಳಂತೆ ತಮ್ಮ ತಮ್ಮ ಮನೆಗೆ ಹಿಂದಿರುಗಿದರು. ದೇಹ ಮನೆಗೆ ಬಂದರೂ ಅವರ ಮನಸ್ಸು ಮಾತ್ರ ಶ್ರೀಕೃಷ್ಣನನ್ನು ಹಿಂಬಾಲಿಸಿ ಹೋಗಿತ್ತು.

ರಥದಲ್ಲಿ ಪ್ರಯಾಣ ಹೊರಟ ಬಲರಾಮಕೃಷ್ಣರು ಬಹುಬೇಗ ಉಳಿದವರನ್ನು ಹಿಂದೆ ಹಾಕಿ, ಯಮುನಾ ನದಿಯ ತೀರವನ್ನು ಸೇರಿದರು. ಆ ನದಿಯನ್ನು ಕಾಣುತ್ತಲೆ ಅವರು ರಥದಿಂದಿಳಿದು, ಅದರಲ್ಲಿ ಸ್ನಾನಪಾನಗಳನ್ನು ಮಾಡಿ ಅಲ್ಲಿದ್ದ ಮರದಡಿಯಲ್ಲಿ ಕುಳಿತು ಕೊಂಡರು. ಅವರ ಸ್ನಾನಾನಂತರ ಅಕ್ರೂರನು ಸ್ನಾನಕ್ಕಿಳಿದನು. ಆತ ಒಂದು ಮುಳುಗು ಹಾಕುತ್ತಲೆ ಬಲರಾಮಕೃಷ್ಣರು ನೀರಿನಲ್ಲಿರುವಂತೆ ಕಾಣಿಸಿದರು. ಅಷ್ಟು ಬೇಗ ಅವರು ಅಲ್ಲಿ ಹೇಗೆ ಬಂದರೆಂದು ಆಶ್ಚರ್ಯದಿಂದ ಆತನು ನೀರಿನಿಂದ ಎದ್ದು ನೋಡುತ್ತಾನೆ, ಅವರಿಬ್ಬರೂ ಮರದಡಿಯಲ್ಲಿ ಮಾತನಾಡುತ್ತಾ ಕುಳಿತಿದ್ದಾರೆ! ಅಚ್ಚರಿಗೊಂಡ ಅಕ್ರೂರ ಮೈಯುಜ್ಜಿಕೊಂಡು ಮತ್ತೆ ನೀರಿನಲ್ಲಿ ಮುಳುಗಿದ. ಈ ಬಾರಿ ಅಚ್ಚರಿಯ ಮೇಲೆ ಅಚ್ಚರಿ–ಸಾಕ್ಷಾತ್ ಆದಿಶೇಷನೇ ತನ್ನ ಸಾವಿರ ಹೆಡೆಗಳೊಡನೆ ಅಲ್ಲಿ ನೆಲಸಿದ್ದಾನೆ; ತಾವರೆಯ ದಂಟಿನಂತೆ ಶುಭ್ರಧವಳವಾದ ಆತನ ದೇಹದ ಮೇಲೆ ನೀಲಮೇಘ ಶ್ಯಾಮ ನಾದ ಶ್ರೀಹರಿ ಮಲಗಿದ್ದಾನೆ! ಆತನ ನಾಲ್ಕು ಭುಜಗಳು, ಪ್ರಸನ್ನವಾದ ಮುಖ, ಕಿರುನಗೆ ಯನ್ನು ಚೆಲ್ಲುತ್ತಿರುವ ವಿಸ್ತಾರವಾದ ಕಣ್ಣುಗಳು, ಅಂದವಾದ ಹುಬ್ಬು, ಉದ್ದವಾದ ಮೂಗು, ನುಣುಪಾದ ಕೆನ್ನೆ, ಕೆಂಪಾದ ತುಟಿ, ವಿಸ್ತಾರವಾದ ಎದೆ. ಆ ಎದೆಯ ಮೇಲೆ ಮಹಾಲಕ್ಷ್ಮಿ, ಚಿಗುರೆಲೆಯಂತೆ ತೆಳುವಾದ ಹೊಟ್ಟೆ, ಆನೆಯ ಸೊಂಡಲಿನಂತೆ ದುಂಡಾದ ತೊಡೆಗಳು, ತಿದ್ದಿ ನಯಮಾಡಿದಂತಿರುವ ಮೊಣಕಾಲು, ಕಮಲದಂತಿರುವ ಪಾದದಲ್ಲಿ ರೇಕುಗಳಂತೆ ಇರುವ ಬೆರಳುಗಳು, ಕೆಂಪಾದ ಉಗುರುಗಳು–ಈ ನಯನ ಮನೋಹರ ಮೂರ್ತಿ ತಲೆಯಲ್ಲಿ ನವರತ್ನದ ಕಿರೀಟವನ್ನು ಧರಿಸಿದ್ದಾನೆ; ಕೈನಲ್ಲಿ ಶಂಖ, ಚಕ್ರ, ಗದಾ, ಪದ್ಮಗಳು; ಕಿವಿಯಲ್ಲಿ ಕುಂಡಲಗಳು; ಕತ್ತಿನಲ್ಲಿ ಮುತ್ತಿನ ಹಾರ; ಸೊಂಟದಲ್ಲಿ ಪೀತಾಂಬರದ ಮೇಲೆ ಕಟಿಸೂತ್ರ. ಆತನ ಸುತ್ತ ನಂದ ಸುನಂದಾದಿ ಪರಿಚಾರಕರು, ಬ್ರಹ್ಮ ರುದ್ರರೇ ಮೊದಲಾದ ದೇವತೆಗಳು, ಪ್ರಹ್ಲಾದ ನಾರದಾದಿ ಭಕ್ತರು, ಸನಕ ಸನಂ ದಾದಿ ಮಹರ್ಷಿಗಳು!

ಈ ದೃಶ್ಯವನ್ನು ಕಾಣುತ್ತಾ ಅಕ್ರೂರ ಬಿಡುಗಣ್ಣನಾದ. ಆತನ ಮೈ ಪುಳಕಿತವಾಯಿತು, ಕಣ್ಣು ಆನಂದಬಾಷ್ಪಗಳನ್ನು ಸುರಿಸಿತು, ಬಾಯಿ ಗದ್ಗದ ದನಿಯಿಂದ ‘ದೇವದೇವ ಲೋಕ ವನ್ನೆಲ್ಲ ಸೃಷ್ಟಿ ಸುವ ಬ್ರಹ್ಮನನ್ನು ಸೃಷ್ಟಿಸುವವ ನೀನು. ನೀನು ಅನಾದಿ, ಅನಂತ. ನಿನ್ನ ಸ್ವರೂಪವನ್ನು ಯಾರು ಬಲ್ಲರು? ಯೋಗಿಗಳು ನಿನ್ನನ್ನು ಸಚ್ಚಿದಾನಂದಸ್ವರೂಪನೆನ್ನು ವರು; ನಿಷ್ಠರಾದ ಸಾಧುಗಳು ಸೂರ್ಯಮಂಡಲದ ಮಧ್ಯದಲ್ಲಿ ನೀನಿರುವೆಯೆಂದೊ, ಇಂದ್ರಾದಿ ದೇವತೆಗಳಲ್ಲಿರುವೆಯೆಂದೊ, ಶರೀರದ ಅಂಗಗಳಲ್ಲಿರುವೆಯೆಂದೊ ಭಾವಿಸಿ ಪೂಜಿಸುವರು, ಕರ್ಮಠರಾದ ಬ್ರಾಹ್ಮಣರು ಯಾಗಗಳಲ್ಲಿ ಇಂದ್ರನೇ ಮೊದಲಾದ ದೇವತೆಗಳ ಹೆಸರಿನಲ್ಲಿ ಮಾಡುವುದು ನಿನ್ನ ಆರಾಧನೆಯನ್ನೇ; ಪಂಚರಾತ್ರಾಗಮದಂತೆ ವಿಷ್ಣುರೂಪವಾಗಿಯೋ ಶಿವರೂಪವಾಗಿಯೋ ಮಾಡುವ ಪೂಜೆಯೂ ನಿನ್ನ ಪೂಜೆಯೇ. ಬೇರೆ ಬೇರೆ ಹರಿವ ನದಿಗಳೆಲ್ಲ ಸಮುದ್ರವನ್ನೇ ಸೇರುವಂತೆ ಬೇರೆ ಬೇರೆ ದೇವತೆಗಳ ಹೆಸರಿ ನಿಂದ ಮಾಡುವ ಪೂಜೆಗಳೆಲ್ಲ ನಿನ್ನನ್ನೇ ಸೇರುತ್ತವೆ. ಬ್ರಹ್ಮನಿಂದ ಹುಲ್ಲುಕಡ್ಡಿಯವರೆಗೆ ಜಗತ್ತಿನ ಚರಾಚರವಸ್ತುಗಳೆಲ್ಲ ನಿನ್ನ ಅಧೀನ. ಸರ್ವಕ್ಕೂ ನೀನೆ ನಿಯಾಮಕ. ಅಗ್ನಿ ನಿನ್ನ ಮುಖ, ಸೂರ್ಯ ನಿನ್ನ ಕಣ್ಣು, ಭೂಮಿ ನಿನ್ನ ಪಾದ, ಆಕಾಶ ನಿನ್ನ ನಾಭಿ, ದಿಕ್ಕುಗಳು ನಿನ್ನ ಕಿವಿ, ಸ್ವರ್ಗ ನಿನ್ನ ತಲೆ, ಇಂದ್ರಾದಿ ದೇವತೆಗಳು ನಿನ್ನ ತೋಳುಗಳು, ಸಮುದ್ರ ನಿನ್ನ ಹೊಟ್ಟೆ, ವಾಯು ನಿನ್ನ ಪ್ರಾಣ, ಗಿಡಮರಗಳು ನಿನ್ನ ಮೈಕೂದಲು, ಮೋಡ ನಿನ್ನ ತಲೆಗೂದಲು, ಹಗಲಿರುಳು ನಿನ್ನ ರೆಪ್ಪೆಗಳು! ಸಕಲ ಬ್ರಹ್ಮಾಂಡಗಳೂ ಅತ್ತಿಯ ಹಣ್ಣಿನಲ್ಲಿ ಇರುವ ಹುಳುಗಳಂತೆ ನಿನ್ನಲ್ಲಿ ಅಡಕವಾಗಿವೆ. ಇಂತಹ ಮಹಾಮಹಿಮನಾದ ನೀನು ಲೀಲೆ ಯಾಗಿ ಅವತಾರಗಳನ್ನೆತ್ತುವೆ. ನಿನ್ನ ಸ್ವರೂಪವು ತಿಳಿಯಲು ಅಶಕ್ಯವಾದುದರಿಂದ, ಈ ಅವತಾರಗಳ ಮೂಲಕ ನಿನ್ನ ಗುಣರೂಪಗಳನ್ನು ಅರಿತು, ನಿನ್ನ ಗುಣಗಾನದಿಂದ ನಾವು ಉದ್ಧಾರವಾಗುತ್ತೇವೆ. ಈಗ ಕೃಷ್ಣರೂಪಿಯಾಗಿ ನಮಗೆ ಕಾಣಿಸಿಕೊಳ್ಳುತ್ತಿರುವ ಹೇ ಪರಮಾತ್ಮ, ನಿನಗೆ ನಮೊ ನಮೊ! ಕಾಮಕ್ಕೂ ಕರ್ಮಕ್ಕೂ ಸಿಕ್ಕಿ ತೊಳಲುತ್ತಿರುವ ನನ್ನನ್ನು ಉದ್ಧರಿಸು. ನಾನು ನಿನ್ನನ್ನು ಮರೆಹೊಕ್ಕಿದ್ದೇನೆ’ ಎಂದು ಬೇಡಿಕೊಂಡನು. ಆಕ್ರೂರನು ಹೀಗೆ ಸ್ತುತಿಸುತ್ತಿರುವಂತೆಯೇ ಆ ದಿವ್ಯ ದೃಶ್ಯ ಕಣ್ಮರೆಯಾಯಿತು. ಆತನು ನೀರಿನಿಂದೆದ್ದು ಬಂದು ತನ್ನ ರಥವನ್ನೇರಿದನು. ಆ ವೇಳೆಗಾಗಲೆ ಬಲರಾಮಕೃಷ್ಣರು ಅದರಲ್ಲಿ ಮಂಡಿಸಿ ದ್ದರು.

ಅಕ್ರೂರನು ರಥವನ್ನೇರುತ್ತಿದ್ದಂತೆಯೆ ಶ್ರೀಕೃಷ್ಣನು ಆತನ ಮುಖದತ್ತ ದಿಟ್ಟಿಸಿ ಮುಗುಳ್ನಗುತ್ತಾ ‘ಏನು ಅಕ್ರೂರ, ನಿನ್ನ ಮುಖದಮೇಲೆ ಅಚ್ಚರಿಯು ಅಚ್ಚೊತ್ತಿದಂತಿದೆ. ನೆಲ, ಜಲ, ಆಕಾಶದಲ್ಲಿ ಎಲ್ಲಿಯಾದರೂ ಏನಾದರೂ ಆಶ್ಚರ್ಯವನ್ನು ಕಂಡೆಯೇನು?’ ಎಂದು ಕೇಳಿದನು. ಅದಕ್ಕೆ ಪಾಪ, ಅಕ್ರೂರ ಏನು ಉತ್ತರ ಕೊಡಬೇಕು? ‘ಪ್ರಭು, ಎಲ್ಲ ಆಶ್ಚರ್ಯಗಳಿಗೂ ನಿಧಿಯಾಗಿರುವ ನೀನು ನನ್ನ ಕಣ್ಣೆದುರಿಗೆ ಕುಳಿತಿರುವಾಗ, ಅದನ್ನು ಬೇರೆ ಎಲ್ಲಿ ಹುಡುಕಲಿ?’ ಎಂದು ಮಾತ್ರ ಹೇಳಿ, ರಥವನ್ನು ನಡೆಸುತ್ತಾ ಸಂಜೆಯವೇಳೆಗೆ ಮಧುರಾಪುರಿಗೆ ಬಂದನು. ಆ ವೇಳೆಗೆ ನಂದಾದಿಗಳೆಲ್ಲ ಆ ಊರಿನ ಹೊರಗಿದ್ದ ಉದ್ಯಾನ ವನದಲ್ಲಿ ಇಳಿದುಕೊಂಡು, ಅವರ ಬರವನ್ನೆ ಇದಿರು ನೋಡುತ್ತಿದ್ದರು. ಶ್ರೀಕೃಷ್ಣನು ಅಲ್ಲಿ ರಥವನ್ನಿಳಿದು ‘ಅಯ್ಯಾ, ಅಕ್ರೂರ, ನಾವು ಈ ರಾತ್ರಿ ಇಲ್ಲಿಯೆ ಇದ್ದು, ನಾಳೆಯ ಬೆಳಗ್ಗೆ ಮಧುರಾ ಪಟ್ಟಣವನ್ನು ಪ್ರವೇಶಿಸುತ್ತೇವೆ’ ಎಂದನು. ಅಕ್ರೂರನಿಗೆ ಅದು ಒಪ್ಪಿಗೆಯಾಗ ಲಿಲ್ಲ. ಎಲ್ಲ ಗೋಪಾಲರೊಡನೆ ಬಲರಾಮಕೃಷ್ಣರು ಅಂದು ತನ್ನ ಅತಿಥಿಗಳಾಗಿ ತನ್ನ ಮನೆಯಲ್ಲಿ ಇಳಿದುಕೊಳ್ಳಬೇಕೆಂದು ಆತ ಬೇಡಿಕೊಂಡನು. ಆದರೆ ಶ್ರೀಕೃಷ್ಣನು ಅದನ್ನು ನಿರಾಕರಿಸಿ, ‘ಮಿತ್ರ, ನಾವು ಮೊದಲು ಯದುವಂಶದ್ರೋಹಿಯಾದ ಕಂಸನನ್ನು ಕೊಂದು ಯಾದವರಾದ ನಿಮಗೆಲ್ಲ ಸಂತೋಷವನ್ನುಂಟುಮಾಡಿದ ಮೇಲೆ ನಾನು, ಅಣ್ಣ ಬಲರಾಮ ನಿಮ್ಮ ಮನೆಗೆ ಬರುತ್ತೇವೆ’ ಎಂದನು. ಅಕ್ರೂರನು ಆತನ ಮಾತನ್ನು ಮೀರಲಾರದೆ ತಾನೊಬ್ಬನೆ ಮಧುರೆಯನ್ನು ಪ್ರವೇಶಿಸಿದನು. ಮೊದಲು ಆತನು ಕಂಸನನ್ನು ಕಂಡು, ಬಲರಾಮಕೃಷ್ಣರನ್ನು ಕರೆತಂದಿರುವುದಾಗಿ ತಿಳಿಸಿ ಅನಂತರ ತನ್ನ ಮನೆಗೆ ಹೋದನು.

ಮರುದಿನ ಮಧ್ಯಾಹ್ನ ಶ್ರೀಕೃಷ್ಣನು ಬಲರಾಮನೊಡನೆ ಇತರ ಗೋಪಾಲಬಾಲಕರನ್ನೂ ಕರೆದುಕೊಂಡು ಮಧುರೆಯನ್ನು ಪ್ರವೇಶಿಸಿದನು. ಮೊದಲೆ ಅದು ರಾಜಧಾನಿ, ಸುಂದರ ವಾದ ಪಟ್ಟಣ, ಈಗ ಅದು ಧನುರ್ಯಾಗಕ್ಕಾಗಿ ಸಿಂಗರಿಸಬೇಕೆಂದು ರಾಜನ ಅಪ್ಪಣೆ ಯಾಗಿದೆ. ಎಂದಮೇಲೆ ಅದರ ವೈಭವವನ್ನು ಕೇಳಬೇಕೆ! ಸ್ಫಟಿಕದ ಗೋಪುರವುಳ್ಳ ಬಂಗಾರದ ಹೆಬ್ಬಾಗಿಲನ್ನು ದಾಟಿ ಪಟ್ಟಣದ ಒಳಕ್ಕೆ ಹೋಗುತ್ತಲೆ ಮೋಡವನ್ನು ಮುಟ್ಟು ವಂತಹ ಉಪ್ಪರಿಗೆಗಳು ಕಣ್ಣಿಗೆ ಬೀಳುತ್ತವೆ. ಬೀದಿಗಳಿಗೆಲ್ಲ ಪನ್ನೀರನ್ನು ಚೆಲ್ಲಿ ಹೂ ಗಳನ್ನು ಎರಚಿದ್ದಾರೆ. ಮನೆಗಳೆಲ್ಲ ಸುಣ್ಣ ಬಣ್ಣಗಳಿಂದ ಅಲಂಕೃತವಾಗಿವೆ. ಮನೆ ಮನೆಯ ಬಾಗಿಲಲ್ಲಿಯೂ ಕಾರಣೆ ಸಾರಣೆ ತೋರಣಗಳು; ಗಂಧ ಹೂಗಳಿಂದ ಅಲಂಕೃತ ವಾದ ಜಲಕುಂಭಗಳು; ಕೆಲವು ಮನೆಗಳ ಮುಂದೆ ಚಪ್ಪರ ಹಾಕಿ, ಬಾಳೆಯಕಂಬಗಳಿಂ ದಲೂ ಹೊಂಬಾಳೆಗಳಿಂದಲೂ ಬಾವುಟಗಳಿಂದಲೂ ಅಲಂಕರಿಸಿದ್ದಾರೆ. ಹಾದಿಯ ಲ್ಲಂತೂ ಹೆಜ್ಜೆ ಹೆಜ್ಜೆಗೆ ಚಿಗುರಿನ ತೋರಣಗಳು, ಹೂಮಾಲೆಗಳು. ಕಂಸರಾಜನು ಧನುರ್ಯಾಗಕ್ಕಾಗಿ ಮಾಡಿಸಿದ್ದ ಈ ಅಲಂಕಾರ ಕೃಷ್ಣ ಬಲರಾಮರ ಪುರಪ್ರವೇಶಕ್ಕಾಗಿ ಮಾಡಿಸಿದ ಅಲಂಕಾರವಾದಂತಾಯಿತು. ಅವರಿಬ್ಬರೂ ತಮ್ಮ ಗೆಳೆಯರೊಡನೆ ಬೀದಿ ಯಲ್ಲಿ ಬರುತ್ತಲೆ ಮನೆ ಮನೆ ಗಂಡು ಹೆಣ್ಣುಗಳೂ ಬೀದಿಗೆ ಓಡಿಬಂದು ಅವರನ್ನು ಕಂಡು ಆನಂದಿಸಿದರು. ಶ್ರೀಕೃಷ್ಣನ ಮಹಿಮೆಯನ್ನು ಕೇಳಿದ್ದ ಆ ಜನ ಆತನನ್ನು ಕಣ್ಣಾರೆ ಕಾಣುವ ಸಮಯ ಸಿಕ್ಕಾಗ ಬಿಟ್ಟಾರೆಯೆ? ಹೆಣ್ಣುಮಕ್ಕಳಂತೂ ತಾವು ಮಾಡುತ್ತಿದ್ದ ಮನೆಗೆಲಸವನ್ನು ಅರ್ಧಕ್ಕೆ ಬಿಟ್ಟು ಹಾಗೆಯೆ ಓಡಿಬಂದರು; ಊಟಮಾಡುತ್ತಿದ್ದವರು ಅರ್ಧಕ್ಕೆ ನಿಲ್ಲಿಸಿ ಎಂಜಲ ಕೈಯಿಂದ ಹಾಗೆ ಹೊರಗೆ ಬಂದರು, ಮಕ್ಕಳಿಗೆ ಮೊಲೆ ಕೊಡುತ್ತಿದ್ದವರು ಅರ್ಧಕ್ಕೆ ಅದನ್ನು ನಿಲ್ಲಿಸಿ ಹೊರಗೋಡಿಬಂದರು. ವಸ್ತ್ರ ಒಡವೆಗಳನ್ನು ಧರಿಸುವುದಕ್ಕೂ ಅವರಿಗೆ ಪುರುಸತ್ತಿಲ್ಲ. ಒಂದೆ ಓಲೆ ತೊಟ್ಟವಳು, ಒಂದೆ ಕಾಲಂದುಗೆ ಇಟ್ಟವಳು, ಒಂದೆ ಕಣ್ಣಿಗೆ ಕಾಡಿಗೆ ಹಚ್ಚಿದವಳು, ಅರ್ಧ ಸೀರೆಯುಟ್ಟವಳು ಹಾಗೆ ಹಾಗೆಯೆ ಬೀದಿಗೆ ಬಂದರು; ತಮ್ಮ ಕಣ್ಣುಗಳ ಮೂಲಕ ಅವರು ಅವನ ರೂಪವನ್ನು ಹೃದಯಕ್ಕೆ ಸೆಳೆದುಕೊಂಡು, ಆಲಿಂಗನ ಸುಖವನ್ನು ಅನುಭವಿಸಿದರು. ಕೆಲವು ಹೆಣ್ಣುಗಳು ಉಪ್ಪರಿಗೆಯನ್ನೇರಿ ಆತನ ಮೇಲೆ ಹೂಗಳ ಮಳೆಗರೆದರು. ಅವರೆಲ್ಲರ ಮನಸ್ಸಿನಲ್ಲೂ ಒಂದೆ ಭಾವನೆ: ‘ಈ ಮೋಹ ನಾಂಗನನ್ನು ಸದಾ ಕಾಣುತ್ತಿರುವ ಗೋಪಿಯರೆ ಪುಣ್ಯವಂತರು!’

ಹೀಗೆ ಬಲರಾಮಕೃಷ್ಣರು ಗೆಳೆಯರೊಡನೆ ರಾಜಮಾರ್ಗದಲ್ಲಿ ಬರುತ್ತಿರುವಾಗ, ರಾಜರ ಮನೆಯ ಬಟ್ಟೆಗಳನ್ನು ಮಡಿಮಾಡಿಕೊಂಡು ಬರುತ್ತಿದ್ದ ಅರಮನೆಯ ಅಗಸನು ಇದಿರಾ ದನು. ಶ್ರೀಕೃಷ್ಣನು ಅವನನ್ನು ಕುರಿತು ‘ಅಯ್ಯಾ, ನಾವು ರಾಜರನ್ನು ಕಾಣಲು ಹೋಗ ಬೇಕಾಗಿದೆ. ಅದಕ್ಕೆ ತಕ್ಕ ಒಳ್ಳೆಯ ಕೆಲವು ಮಡಿಬಟ್ಟೆಗಳನ್ನು ನಮಗೆ ಕೊಡುತ್ತೀಯಾ? ಅರ ಮನೆಯಿಂದ ಹಿಂದಿರುಗಿದೊಡನೆ ಅವುಗಳನ್ನು ನಿನಗೆ ಒಪ್ಪಿಸುತ್ತೇವೆ. ಈ ಉಪಕಾರಕ್ಕೆ ತಕ್ಕ ಬಹುಮಾನವನ್ನು ಕೊಡುತ್ತೇವೆ’ ಎಂದ. ಅವನು ಅಗಸನಾದರೂ ಕಂಸರಾಜನ ಅಗಸ. ಅದೇನು ಸಣ್ಣ ಪದವಿಯೆ? ಅವನು ಅತ್ಯಂತ ಕೋಪದಿಂದ ‘ಎಲೊ ಹಳ್ಳಿ ಗಮಾರ, ನಿನಗೇನು ಕೊಬ್ಬಾ? ದಿನವೂ ನೀನು ಇಂತಹ ಬಟ್ಟೆಯನ್ನೆ ಉಡುತ್ತೀಯಲ್ಲವೆ? ನೀನು ನೋಡು, ನಿನ್ನ ಅಹಂಕಾರ ನೋಡು. ನಿನ್ನ ಮಾತು ಯಾರಾದರೂ ರಾಜಭಟರ ಕಿವಿಗೆ ಬಿದ್ದರೆ ಎಲುಬು ಮುದಿದಾವು, ಜೋಕೆ’ ಎಂದ. ಇದನ್ನು ಕೇಳಿ ಶ್ರೀಕೃಷ್ಣನಿಗೆ ಕೋಪ ಉಕ್ಕಿತು. ಒಮ್ಮೆ ಆತ ಅವನ ತಲೆಯ ಮೇಲೆ ಗುದ್ದುತ್ತಲೆ ಅವನು ನೆತ್ತರು ಕಾರುತ್ತಾ ನೆಲದಮೇಲೆ ಬಿದ್ದು ಸತ್ತ. ಅವನ ಹಿಂದೆ ಬರುತ್ತಿದ್ದ ಅವನ ಆಳುಗಳು ತಾವು ಹೊತ್ತಿದ್ದ ಮಡಿಬಟ್ಟೆಗಳನ್ನೆಲ್ಲ ಅಲ್ಲಿಯೆ ಬಿಸುಟು ಓಡಿಹೋದರು. ಬಲರಾಮಕೃಷ್ಣರೂ ಅವರ ಗೆಳೆಯರೂ ಅವುಗಳಲ್ಲಿ ತಮಗೆ ಬೇಕಾದುವನ್ನು ಆಯ್ದು ಧರಿಸಿದರು. ಉಳಿದುವನ್ನು ಅಲ್ಲಿಯೆ ಬಿಟ್ಟು ಅವರು ತಮ್ಮ ದಾರಿ ಹಿಡಿದರು.

ಶ್ರೀಕೃಷ್ಣನ ತಂಡ ಇನ್ನೆರಡು ಹೆಜ್ಜೆ ಮುಂದೆ ಹೋಗುವಷ್ಟರಲ್ಲಿ ನೇಗೆಯವನೊಬ್ಬನು ಅವರಿಗೆ ಇದಿರಾದನು. ಆವನು ತನ್ನ ಹೊಸಬಟ್ಟೆಗಳನ್ನು ಮಾರುವುದಕ್ಕಾಗಿ ಹೊರಟಿ ದ್ದನು. ಬಲರಾಮಕೃಷ್ಣರ ಮೋಹಕ ರೂಪವನ್ನು ಕಂಡು ಅವನಿಗೆ ಅವರಲ್ಲಿ ಅಪಾರವಾದ ಮಮತೆ ಹುಟ್ಟಿತು. ಅವನು ತಾನಾಗಿಯೇ ತನ್ನಲ್ಲಿದ್ದ ಹೊಸ ರುಮಾಲುಗಳನ್ನು ಅವರ ತಲೆಗೆ ಸುತ್ತಿ ಅಲಂಕರಿಸಿದನು. ಇದರಿಂದ ಸುಪ್ರೀತನಾದ ಶ್ರೀಕೃಷ್ಣನು ಅವನೊಡನೆ ‘ಅಯ್ಯಾ, ದೇವರು ನಿನಗೆ ಒಳ್ಳೆಯದು ಮಾಡಲಿ’ ಎಂದು ಹರಸಿ, ತನ್ನ ಗೆಳೆಯರೊಡನೆ ಮುಂದಕ್ಕೆ ಹೊರಟನು. ಅವರು ಇನ್ನೆರಡು ಹೆಜ್ಜೆ ಹೋಗುವಷ್ಟರಲ್ಲಿ ಒಬ್ಬ ಹೂ ಮಾರು ವವನ ಮನೆ ಸಿಕ್ಕಿತು. ಅವರೆಲ್ಲರೂ ಶ್ರೀಕೃಷ್ಣನನ್ನು ಮುಂದೆ ಮಾಡಿಕೊಂಡು ಅವನ ಮನೆಯನ್ನು ಪ್ರವೇಶಿಸಿದರು. ಸುದಾಮನೆಂಬ ಆ ಹೂವಾಡಿಗ ಶ್ರೀಕೃಷ್ಣನ ಕೀರ್ತಿಯನ್ನು ಸಾಕಷ್ಟು ಕೇಳಿದ್ದ. ಆತ ತನ್ನ ಮನೆಗೆ ತಾನಾಗಿಯೇ ಬಂದುದನ್ನು ಕಂಡು ಆತನಿಗೆ ಬಲು ಸಂತೋಷವಾಯಿತು. ಅವನು ಶ್ರೀಕೃಷ್ಣನಿಗೆ ಭಕ್ತಿಯಿಂದ ನಮಸ್ಕರಿಸಿ, ತನ್ನ ಹೂಗಳಲ್ಲಿ ಅತ್ಯುತ್ತಮವಾದುದನ್ನು ಆಯ್ದು ಕಟ್ಟಿದ ಹೂಮಾಲೆಯನ್ನು ಆತನ ಕೊರಳಿಗೆ ಹಾಕಿದನು. ಇತರರಿಗೂ ಸೊಗಸಾದ ಹೂಮಾಲೆಗಳು ದೊರೆತವು. ಅವರೆಲ್ಲರೂ ಅವನನ್ನು ಬಾಯಿ ತುಂಬ ಹರಸಿ, ಅಲ್ಲಿಂದ ಮುಂದಕ್ಕೆ ಹೊರಟರು.

ಹೊಸಬಟ್ಟೆಗಳನ್ನುಟ್ಟು, ಹೂ ಮುಡಿದು ಸಿಂಗಾರವಾಗಿದ್ದ ಶ್ರೀಕೃಷ್ಣನ ತಂಡ ಅರಸು ಮಕ್ಕಳಂತೆ ರಾಜಬೀದಿಯಲ್ಲಿ ಮೆರೆಯುತ್ತಾ ಬರುತ್ತಿರುವಾಗ, ಕೈಲಿ ಗಂಧದ ಬಟ್ಟಲನ್ನು ಹಿಡಿದ ಹುಡುಗಿಯೊಬ್ಬಳು ಇದಿರಾದಳು. ಅವಳು ಬಹು ಸುಂದರಿಯಾದರೂ ಬೆನ್ನು ಬಗ್ಗಿದ ಗೂನಿಯಾಗಿದ್ದಳು. ಶ್ರೀಕೃಷ್ಣನು ಮಂದಹಾಸದಿಂದ ಅವಳನ್ನು ಕುರಿತು ‘ಎಲೆ ಚೆಲುವೆ, ಯಾರೆ ನೀನು? ಈ ಗಂಧವನ್ನು ಯಾರಿಗಾಗಿ ತೆಗೆದುಕೊಂಡು ಹೋಗುತ್ತಿ? ಇದ ರಲ್ಲಿ ನಮಗೂ ಸ್ವಲ್ಪ ಕೊಡುತ್ತೀಯಾ?’ ಎಂದು ಕೇಳಿದ. ಅವಳು ‘ಎಲೆ ಚೆಲುವ, ನಾನು ಕಂಸರಾಜನ ದಾಸಿಯಾದ ತ್ರಿವಕ್ರೆ. ಈ ಗಂಧವನ್ನು ಆತನಿಗಾಗಿ ಕೊಂಡೊಯ್ಯುತ್ತಿದ್ದೇನೆ. ಆದರೇನು? ನಿನ್ನಂತಹ ಸುಂದರ ಪುರುಷನನ್ನು ನಾನೆಲ್ಲಿಯೂ ಕಂಡಿಲ್ಲ. ಇದಕ್ಕೆ ನಿನಗಿಂತ ಅರ್ಹರಾದವರು ಬೇರೆ ಯಾರೂ ಇಲ್ಲ. ತಗೋ, ನೀನೂ ಮತ್ತು ನಿನ್ನ ಜೊತೆಯ ಆ ಸುಂದರಾಂಗನೂ ನಿಮಗೆ ಬೇಕಾದಷ್ಟು ಗಂಧವನ್ನು ತೆಗೆದುಕೊಳ್ಳಿರಿ’ ಎಂದಳು. ಶ್ರೀಕೃಷ್ಣ ಬಲರಾಮರು ರಾಜಯೋಗ್ಯವಾದ ಆ ಗಂಧವನ್ನು ಮೈಗೆಲ್ಲ ಹಚ್ಚಿಕೊಂಡರು. ಅವಳ ಸುಗುಣವನ್ನು ಕಂಡು ಶ್ರೀಕೃಷ್ಣನಿಗೆ ಬಹು ಸಂತೋಷವಾಯಿತು. ತಕ್ಷಣವೇ ಆಕೆಯನ್ನು ಅನುಗ್ರಹಿಸಬೇಕೆಂದುಕೊಂಡ ಆತ. ತನ್ನ ಇದಿರಿಗೆ ನಿಂತಿದ್ದ ಆ ಹುಡುಗಿಯ ಪಾದ ಗಳೆರಡನ್ನೂ ಆತ ತನ್ನ ಪಾದಗಳಲ್ಲಿ ಮೆಟ್ಟಿಕೊಂಡು, ಅವಳ ಗಲ್ಲವನ್ನು ಹಿಡಿದು ಮೇಲ ಕ್ಕೆತ್ತಿದನು. ಹಾಗೆ ಎತ್ತಿದೊಡನೆಯೆ ಅವಳ ಗೂನು ಮಾಯವಾಯಿತು. ಅವಳೊಬ್ಬ ಅಪೂರ್ವ ಸುಂದರಿಯಾಗಿ ನಿಂತಳು. ಅವಳು ಶ್ರೀಕೃಷ್ಣನ ಉತ್ತರೀಯವನ್ನು ಹಿಡಿದು ಕೊಂಡು ‘ಹೇ ಮೋಹನಾಂಗ, ನಿನ್ನ ಗುಣ ರೂಪಗಳಿಗೆ ನಾನು ಮಾರುಹೋಗಿದ್ದೇನೆ. ನಿನ್ನನ್ನು ಬಿಟ್ಟು ನಾನು ಅಗಲಲಾರೆ. ಬಾ, ನಮ್ಮ ಮನೆಗೆ ಹೋಗೋಣ. ಮೀಸಲಾಗಿರುವ ಈ ಮೈಯನ್ನು ನಿನಗೊಪ್ಪಿಸಿ, ನಾನು ಧನ್ಯಳಾಗುತ್ತೇನೆ’ ಎಂದಳು. ಶ್ರೀಕೃಷ್ಣ ತನ್ನ ಗೆಳೆಯರ ಮುಖದತ್ತ ನೋಡಿ ನಗುತ್ತಾ ‘ಸುಂದರಿ, ನಾವು ಬೇರೆ ಊರಿನವರು, ಈ ಊರಿನಲ್ಲಿ ಮನೆ ಮಠಗಳೊಂದೂ ಇಲ್ಲ. ಆದ್ದರಿಂದ ನಿನ್ನ ಆಗ್ರಹ ನಮಗೆ ಅನುಗ್ರಹವೇ ಸರಿ. ನಾವು ಅಗತ್ಯವಾಗಿಯೂ ನಿನ್ನ ಮನೆಗೆ ಬರುತ್ತೇವೆ. ಆದರೆ ನಾವು ಇಲ್ಲಿಗೆ ಬಂದಿರುವ ಕೆಲಸವನ್ನು ಮುಗಿಸಿಕೊಂಡು ಆಮೇಲೆ ಬರುತ್ತೇವೆ’ ಎಂದು ಸಕ್ಕರೆಯಂತೆ ಸವಿಯಾದ ಮಾತುಗಳಿಂದ ಅವಳನ್ನು ಬೀಳ್ಕೊಟ್ಟು ಮುಂದೆ ಹೊರಟನು.

ಶ್ರೀಕೃಷ್ಣನು ತನ್ನ ಪರಿವಾರದೊಡನೆ ಮುಂದೆ ನಡೆಯುತ್ತಾ, ಧನುರ್ಯಾಗದ ಬಿಲ್ಲು ಎಲ್ಲಿರುವುದೆಂದು ಇದಿರಿಗೆ ಸಿಕ್ಕಿದವರನ್ನು ಕೇಳಿಕೊಂಡು ಧನುಶ್ಶಾಲೆಗೆ ಬಂದನು. ಕಂಸ ನಿಂದ ಪೂಜೆಗೊಂಡ ಆ ಮಹಾಧನುವಿನ ಸುತ್ತ ಅನೇಕ ರಾಜಭಟರು ಕಾವಲು ಕಾಯುತ್ತಿ ದ್ದರು. ಶ್ರೀಕೃಷ್ಣನು ನೇರವಾಗಿ ಅದರ ಬಳಿಗೆ ಹೋಗಿ ಅದನ್ನು ಕೈಗೆತ್ತಿಕೊಂಡನು. ಇದನ್ನು ಕಂಡು ಸುತ್ತಲಿದ್ದ ಭಟರು ಕೋಪದಿಂದ ಗರ್ಜಿಸುತ್ತಾ ಅವನನ್ನು ಹಿಡಿದುಕೊಳ್ಳಲು ಬರು ವಷ್ಟರಲ್ಲಿ ಶ್ರೀಕೃಷ್ಣನು ಅದನ್ನು ಹೆದಯೇರಿಸುವ ನೆಪದಿಂದ ಮುರಿದೇಬಿಟ್ಟನು. ಮದ್ದಾನೆಯ ಕೈಗೆ ಸಿಕ್ಕ ಕಬ್ಬಿನ ಜಲ್ಲೆಯಂತೆ ಅದು ಎರಡು ತುಂಡಾಗುತ್ತಲೆ ಕಾವಲಿದ್ದ ವರು ‘ಈ ಪುಂಡನನ್ನು ಹಿಡಿಯಿರಿ, ಬಡಿಯಿರಿ, ಕಡಿಯಿರಿ’ ಎಂದು ಕೂಗಿಕೊಂಡು ಅವ ನನ್ನು ಮುತ್ತಿದರು. ಒಡನೆಯೆ ಬಲರಾಮಕೃಷ್ಣರು ಮುರಿದ ಬಿಲ್ಲಿನ ಒಂದೊಂದು ತುಂಡನ್ನು ಕೈಗೆ ತೆಗೆದುಕೊಂಡು, ಆ ಕಾವಲುಗಾರರನ್ನು ಸದೆಬಡಿದರು. ಅವರ ಸಹಾಯ ಕ್ಕೆಂದು ಬಂದವರಿಗೂ ಅದೇ ಗತಿಯಾಯಿತು. ಹೀಗೆ ಮೇಲೆ ಬಿದ್ದವರನ್ನೆಲ್ಲ ಹುಚ್ಚು ನಾಯಿಗಳನ್ನು ಬಡಿದೋಡಿಸುವಂತೆ ಓಡಿಸಿ, ಅವರಿಬ್ಬರೂ ತಮ್ಮ ಗೆಳೆಯರೊಡನೆ ಮುಧುರಾಪುರಿಯ ಸಿಂಗಾರವನ್ನು ನೋಡಹೊರಟರು. ಇವರ ಧೈರ್ಯ ಸಾಹಸಗಳನ್ನೂ ತೇಜಸ್ಸನ್ನೂ ಕಂಡ ಜನ ಅವರಾರೋ ದೇವತೆಗಳೆ ಹೊರತು ಮನುಷ್ಯಮಾತ್ರದವ ರಲ್ಲವೆಂದುಕೊಂಡರು. ಶ್ರೀಕೃಷ್ಣನು ತನ್ನವರೊಡನೆ ಸಂಜೆಯವರೆಗೆ ಊರಲ್ಲೆಲ್ಲ ನಿರ್ಭಯವಾಗಿ ತಿರುಗಾಡುತ್ತಿದ್ದು, ಕತ್ತಲಾಗುತ್ತಲೆ ತಮ್ಮ ಗಾಡಿಗಳನ್ನು ನಿಲ್ಲಿಸಿದ್ದ ಸ್ಥಳಕ್ಕೆ ಹಿಂದಿರುಗಿದನು.

ಬಿಡಾರಕ್ಕೆ ಹಿಂದಿರುಗಿದ ಶ್ರೀಕೃಷ್ಣನು ಅಣ್ಣನೊಡನೆ ಹಾಲು-ಅನ್ನವನುಂಡು, ಸ್ವಸ್ಥ ವಾಗಿ ಮಲಗಿ ನಿದ್ದೆ ಹೋದನು. ಆದರೆ ಕಂಸರಾಜನಿಗೆ ಆ ರಾತ್ರಿ ನಿದ್ರೆ ಬರಲಿಲ್ಲ. ಪೂಜೆ ಮಾಡಿ ಇಟ್ಟಿದ್ದ ತನ್ನ ಮಹಾ ಧನುಸ್ಸನ್ನು ಶ್ರೀಕೃಷ್ಣನು ಮುರಿದನೆಂಬುದನ್ನು ಕೇಳಿದಾಗಿ ನಿಂದ ಆತನ ಶಾಂತಿ ಹಾರಿ ಹೋಗಿತ್ತು. ಬಲರಾಮಕೃಷ್ಣರು ತನ್ನ ಕಾವಲುಗಾರರನ್ನೆಲ್ಲ ಬಡಿದುಹಾಕಿದುದನ್ನು ನೆನೆದು ಆತನಿಗೆ ಭಯವಾಯಿತು. ಆತ ಸ್ವಲ್ಪ ಕಣ್ಣು ಮುಚ್ಚುತ್ತಲೆ ಭಯಂಕರವಾದ ಕನಸುಗಳು–ಮೈಗೆಲ್ಲ ಎಣ್ಣೆ ಬಳಿದುಕೊಂಡು, ಕೆಂಪು ಹೂಮಾಲೆ ಧರಿಸಿ, ಕತ್ತೆಯೇರಿ ದಕ್ಷಿಣಕ್ಕೆ ಪ್ರಯಾಣ ಹೊರಟಂತೆ; ಹೆಣವನ್ನು ಅಪ್ಪಿಕೊಂಡಿರುವಂತೆ; ತಾವರೆಯ ದಂಟುಗಳನ್ನು ಕಡಿದು ತಿನ್ನುತ್ತಿರುವಂತೆ! ಈ ಕೆಟ್ಟ ಕನಸುಗಳಿಂದ ಭಯ ಗೊಂಡು ಕಣ್ಣು ತೆರೆದರೆ ಇನ್ನೂ ಭಯಂಕರವಾದ ನಿಮಿತ್ತಗಳು–ನೆರಳಿನಲ್ಲಿ ಅವನ ತಲೆಯೇ ಕಾಣಿಸುತ್ತಿಲ್ಲ; ನೆರಳಿನ ತುಂಬ ತೂತುಗಳು. ಅಲ್ಲಿದ್ದ ಒಂದು ದೀಪ ಎರಡಾಗಿ ಕಾಣಿಸುತ್ತದೆ; ಹಸಿರು ಮರ ಬಂಗಾರದ ಬಣ್ಣವಾಗಿ ಕಾಣಿಸುತ್ತದೆ. ಏನಿದೆಲ್ಲ? ತನ್ನ ಸಾವನ್ನು ಸೂಚಿಸುವ ಈ ನಿಮಿತ್ತಗಳನ್ನು ಕಂಡು ಆತನ ಜೀವ ತತ್ತರಿಸಿತು. ಬಹು ಕಷ್ಟ ದಿಂದ ಆತನು ಆ ರಾತ್ರಿಯನ್ನು ಕಳೆದು, ಬೆಳಗಾಗುತ್ತಲೆ ತನ್ನ ಆಪ್ತರನ್ನು ಕರೆಸಿ, ಆ ದಿನವೇ ಜಟ್ಟಿಯ ಕಾಳಗವನ್ನು ಏರ್ಪಡಿಸುವಂತೆ ಸೂಚಿಸಿದನು.

ಕಂಸರಾಜನ ಅಪ್ಪಣೆಯಂತೆ ಜಟ್ಟಿಯ ಕಣವು ಸಿದ್ಧವಾಯಿತು. ಸುತ್ತಲೂ ಸಹಸ್ರಾರು ಜನರು ಕುಳಿತು ನೊಡುವುದಕ್ಕೆ ಅನುಕೂಲವಾಗುವಂತೆ ಆಸನಗಳನ್ನು ಸಿದ್ಧಪಡಿಸಿದರು. ಕಣದ ಸುತ್ತಲೂ ಎಳೆಯ ಚಿಗುರಿನಿಂದ ತೋರಣವನ್ನು ಕಟ್ಟಿ ಬಾವುಟಗಳನ್ನು ಅಲಂಕರಿಸಿ ದರು. ಊರ ಜನರೆಲ್ಲರೂ ತಮ್ಮ ಯೋಗ್ಯತೆಗೆ ತಕ್ಕ ಸ್ಥಳಗಳಲ್ಲಿ ಬಂದು ಕುಳಿತರು. ನಂದನೇ ಮೊದಲಾದ ಗೋಪಾಲರೂ ಬಂದು ತಮಗಾಗಿ ಪ್ರತ್ಯೇಕವಾಗಿ ತೆರವು ಮಾಡಿದ ಸ್ಥಳದಲ್ಲಿ ಕುಳಿತುಕೊಂಡರು. ಕಂಸನು ಮಂತ್ರಿಗಳೊಡನೆ ಅಟ್ಟಹಾಸದಿಂದ ಕಣವನ್ನು ಪ್ರವೇಶಿಸಿ, ಅಲಂಕೃತವಾದ ರತ್ನಸಿಂಹಾಸನದಲ್ಲಿ ಮಂಡಿಸಿದನು. ಒಡನೆಯೆ ಮಂಗಳ ವಾದ್ಯಗಳು ಭೋರ್ಗರೆದವು. ಇದನ್ನು ಕೇಳುತ್ತಲೆ ಬಲರಾಮಕೃಷ್ಣರು ಕುಸ್ತಿಯ ಕಣದ ಬಳಿಗೆ ಬಂದರು. ಆದರೆ ಅದರ ಬಾಗಿಲಲ್ಲಿ ಭಯಂಕರಾಕಾರದ ಆನೆಯೊಂದು ನಿಂತಿತ್ತು. ಅದನ್ನು ಕಂಡು, ಶ್ರೀಕೃಷ್ಣನು ಅದರ ಮಾವಟಿಗನೊಡನೆ ‘ಅಯ್ಯಾ, ನಾವು ಒಳಕ್ಕೆ ಹೋಗ ಬೇಕು. ನಿನ್ನ ಆನೆಯನ್ನು ಸ್ವಲ್ಪ ಅತ್ತಕಡೆ ಕೊಂಡುಹೋಗು. ಇಲ್ಲದಿದ್ದರೆ ನಿನಗೂ ನಿನ್ನ ಆನೆಗೂ ಗ್ರಹಚಾರ ಕಡಿಮೆಯಾದೀತು!’ ಎಂದು ಹೇಳಿದ. ಬಲರಾಮಕೃಷ್ಣರನ್ನು ಕೊಲ್ಲಿಸ ಲೆಂದೇ ಆನೆಯನ್ನು ಕರೆತಂದಿದ್ದ ಆ ಮಾವಟಿಗ ಕೃಷ್ಣನ ಮಾತನ್ನು ಕೇಳುತ್ತಲೆ, ಕೋಪ ದಿಂದ ತನ್ನ ಆನೆಯನ್ನು ಆತನ ಮೇಲೆ ನುಗ್ಗಿಸಿದ. ಬೆಟ್ಟದಂತಿದ್ದ ಆ ಆನೆ ಬಾಲಕನಾದ ಶ್ರೀಕೃಷ್ಣನ ಮೇಲೆ ನುಗ್ಗಿ ಬಂದು, ತನ್ನ ಸೊಂಡಿಲಿನಿಂದ ಆತನನ್ನು ಹಿಡಿಯಿತು. ಒಡನೆಯೆ ಆತ ಅದರಿಂದ ಬಿಡಿಸಿಕೊಂಡು, ತನ್ನ ಮುಷ್ಠಿಯಿಂದ ಅದರ ಹಣೆಗೆ ಗುದ್ದಿದವನೆ ಅದರ ಹೊಟ್ಟೆಯ ಕೆಳಗೆ ಅವಿತುಕೊಂಡ. ಅದು ಆತನನ್ನು ಹುಡುಕಿಕೊಂಡು ಗರ ಗರ ತಿರುಗು ತ್ತಿರಲು, ಶ್ರೀಕೃಷ್ಣ ಉಪಾಯವಾಗಿ ಅದರ ಬಾಲವನ್ನು ಹಿಡಿದುಕೊಂಡು, ಗರುಡನು ಹಾವನ್ನು ಎಳೆದಾಡುವಂತೆ, ಆ ಮದ್ದಾನೆಯನ್ನು ಇಪ್ಪತ್ತೈದು ಮಾರಿನಷ್ಟು ದೂರ ದರದರ ಹಿಂದಕ್ಕೆ ಎಳೆದುಕೊಂಡು ಹೋದ. ಅದು ಎಡಕ್ಕೆ ತಿರುಗಿದರೆ ತಾನು ಬಲಕ್ಕೆ, ಬಲಕ್ಕೆ ತಿರುಗಿದರೆ ಎಡಕ್ಕೆ–ಹೀಗೆ ಮಾಡುತ್ತಾ ಆತನು ಅದನ್ನು ಗಿರಿಗಿರಿ ತಿರುಗಿಸಿ ಅದನ್ನು ಬಳಲಿ ಸಿದ. ಅದನ್ನು ಗುದ್ದುವುದು, ಓಡುವುದು–ಹೀಗೆ ಬಹಳ ಹೊತ್ತು ಅದರೊಡನೆ ಆಟ ವಾಡುತ್ತಿದ್ದು ಅದರ ಶಕ್ತಿ ಕುಂದಿಸಿದ. ಅನಂತರ ತನ್ನ ಮೆಲೆ ಏರಿ ಬರುತ್ತಿರುವ ಆನೆಗೆ ಇದಿರಾಗಿ ನಿಂತು, ಅದರ ಸೊಂಡಿಲನ್ನು ತಿರುಚಿ ನೆಲಕ್ಕೆ ಬೀಳಿಸಿದ, ಒಡನೆಯೆ ಸಿಂಹದಂತೆ ಅದರ ಮೇಲೆ ಬಿದ್ದು, ಅದರ ನೆತ್ತಿಯನ್ನು ಗುದ್ದಿ, ಕೊಂದು ಹಾಕಿದ. ಅದು ಸಾಯುತ್ತಲೆ ಆತನು ಅದರ ಕೊಂಬುಗಳೆರಡನ್ನೂ ಕಿತ್ತು, ಒಂದನ್ನು ಬಲರಾಮನಿಗೆ ಕೊಟ್ಟು, ಮತ್ತೊಂದರಿಂದ ಮಾವಟಿಗನನ್ನು ಕೊಂದು ಹಾಕಿದ. ಅದಾದ ಮೇಲೆ ಅಣ್ಣತಮ್ಮಂದಿರು ಆನೆ ಕೊಂಬುಗಳನ್ನು ಭುಜದಮೇಲಿಟ್ಟುಕೊಂಡು ಕುಸ್ತಿಯ ಕಣವನ್ನು ಪ್ರವೇಶಿಸಿದರು.

ಬಲರಾಮ ಕೃಷ್ಣರು ಒಳಕ್ಕೆ ಬರುತ್ತಿದ್ದಂತೆಯೆ ಕಣದಲ್ಲಿ ಕುಳಿತವರ ಕಣ್ಣುಗಳೆಲ್ಲ ಅವ ರತ್ತ ತಿರುಗಿದವು. ಅಲ್ಲಿದ್ದ ಜಟ್ಟಿಗಳಿಗೆ ಅವರು ಸಿಡಿಲಿನಂತೆ ರುದ್ರಭಯಾನಕರಾಗಿದ್ದಾರೆ. ಅಲ್ಲಿದ್ದ ಹೆಣ್ಣುಗಳಿಗೆ ಶೃಂಗಾರರಸವನ್ನು ಚೆಲ್ಲುತ್ತಿರುವ ಮನ್ಮಥನಂತಿದ್ದಾರೆ; ಅಲ್ಲಿನ ಗೋಪಾಲರಿಗೆ ಅವರು ಬಂಧುವಿನಂತೆ ಪ್ರೇಮರಸವನ್ನು ಉಕ್ಕಿಸುತ್ತಿದ್ದರೆ, ಅಲ್ಲಿನ ಕ್ಷತ್ರಿಯರಿಗೆ ತಮ್ಮನ್ನು ನಿಯಮಿಸುವ ಪ್ರಭುವಿನಂತೆ ವೀರರಸವನ್ನು ಕಾಣಿಸಿದರು; ನಂದ ನಿಗೆ ಮುದ್ದು ಮಗುವಿನಂತೆ ಕರುಣರಸವನ್ನು ಕೋಡಿ ಹರಿಸುತ್ತಿದ್ದ ಆ ಬಲರಾಮಕೃಷ್ಣರೇ ಕಂಸನಿಗೆ ಮೃತ್ಯುಭಯವನ್ನು ಹುಟ್ಟಿಸುವಂತಿದ್ದರು. ಅಜ್ಞರಿಗೆ ಅವರು ಗೊಲ್ಲರಂತೆ ಹಾಸ್ಯಕ್ಕೆ ವಸ್ತುವಾಗಿದ್ದರೆ ಜ್ಞಾನಿಗಳಿಗೆ ಪರಮಾತ್ಮ ಸ್ವರೂಪಿಯೆಂಬ ಶಾಂತಿರಸವನ್ನು ನೀಡುತ್ತಿದ್ದರು. ಅಷ್ಟರಲ್ಲಿ ಕಂಸನ ಕಡೆಯವರಾರೋ ಆತನ ಬಳಿಗೆ ಹೋಗಿ, ಆನೆಯೂ ಅದರ ಮಾವಟಿಗನೂ ಸತ್ತ ಸುದ್ದಿಯನ್ನು ಕಂಸನ ಕಿವಿಯಲ್ಲಿ ಉಸುರಿದರು. ಅದನ್ನು ಕೇಳಿ ಆತನ ಹೃದಯ ತರಗೆಲೆಯಾಯಾತು. ಆತನು ತನ್ನ ಬಳಿಯಲ್ಲಿ ನಿಂತಿದ್ದ ಚಾಣೂರನಿಗೆ ಕಣ್ಣಸನ್ನೆ ಮಾಡಿದನು. ಒಡನೆಯೇ ಆ ಜಟ್ಟಿ ಕೃಷ್ಣನ ಬಳಿಗೆ ಬಂದು ‘ಅಯ್ಯಾ, ನೀನೇನೊ ಕೃಷ್ಣನೆಂಬುವನು? ನೀನು ಜಟ್ಟಿಯ ಕಾಳಗದಲ್ಲಿ ಬಹಳ ನಿಪುಣನಂತೆ. ನೀನೂ ಈ ಬಲ ರಾಮನೂ ಈವರೆಗೆ ಅನೇಕ ಶೂರರನ್ನು ಕೊಂದುಹಾಕಿರುವಿರಂತೆ, ಅನೇಕ ಅಸಾಧ್ಯ ಕಾರ್ಯ ಗಳನ್ನೆಲ್ಲ ಮಾಡಿರುವಿರಂತೆ. ನಮ್ಮ ಕಂಸ ಮಹಾರಾಜರು ನಿಮ್ಮ ಶಕ್ತಿಯನ್ನು ನೋಡ ಬೇಕೆಂದೇ ನಿಮ್ಮನ್ನು ಇಲ್ಲಿಗೆ ಕರೆಸಿರುವುದು. ಈಗ ನಾವೂ ನೀವೂ ಸೇರಿ ಮಹಾರಾಜನ ಕಣ್ಣಿಗೂ ಮನಸ್ಸಿಗೂ ಹಬ್ಬವಾಗುವಂತೆ ಕುಸ್ತಿಯನ್ನು ಮಾಡೋಣ’ ಎಂದನು. ಆಗ ಶ್ರೀಕೃಷ್ಣನು ಅವನನ್ನು ಕುರಿತು ‘ಅಯ್ಯಾ, ನಾವು ಕಂಸರಾಜನ ಪ್ರಜೆಗಳೆ. ರಾಜನ ಮನಸ್ಸಿಗೆ ಸಂತೋಷವಾಗುವಂತೆ ನಡೆದುಕೊಳ್ಳುವುದು ನಮ್ಮ ಪವಿತ್ರ ಕರ್ತವ್ಯ. ಆದರೆ ನಾವಿನ್ನೂ ಹುಡುಗರು. ನಮಗೆ ಸಮಾನರಾದ ಹುಡುಗರಮೇಲಾದರೆ ಕುಸ್ತಿ ಮಾಡ ಬಹುದು. ಇಲ್ಲಿ ಕುಳಿತಿರುವವರಿಗೆಲ್ಲ ನ್ಯಾಯವೆಂದು ತೋರಿದಂತೆ ನಾವು ನಡೆದುಕೊಳ್ಳು ತ್ತೇವೆ’ ಎಂದ. ಚಾಣೂರ ಅವನ ಮಾತನ್ನು ಒಪ್ಪದೆ ‘ಅಯ್ಯಾ ವಯಸ್ಸಿಗೂ ಶಕ್ತಿಗೂ ಏನು ಸಂಬಂಧ? ನೀನು ಚಿಕ್ಕ ವಯಸ್ಸಿನವನಾದರೂ ಮಹಾಪರಾಕ್ರಮಿ ಎಂಬುದಕ್ಕೆ ನಮ್ಮ ಮದ್ದಾನೆಯನ್ನು ಕೊಂದುದೇ ಸಾಕ್ಷಿ. ಆದ್ದರಿಂದ ನನ್ನೊಡನೆ ಕುಸ್ತಿಗೆ ನಿಲ್ಲು, ಬಲರಾಮ ಮುಷ್ಠಿಕನೊಡನೆ ಕಾಳಗ ಮಾಡಲಿ’ ಎಂದ. ಶ್ರೀಕೃಷ್ಣ ಅದಕ್ಕೆ ‘ಬೇರೆ ಮಾರ್ಗವಿಲ್ಲ, ಹಾಗೆಯೆ ಆಗಲಿ’ ಎಂದ.

ಶ್ರೀಕೃಷ್ಣಬಲರಾಮರು ಬಿಗಿದ ಕಾಚಗಳೊಡನೆ ಕುಸ್ತಿಗೆ ಸಿದ್ಧರಾಗಿ ಕಣವನ್ನು ಪ್ರವೇಶಿಸಿದರು. ಚಾಣೂರ ಮುಷ್ಠಿಕರೂ ಸಿದ್ಧರಾಗಿ ನಿಂತರು. ಅವರನ್ನು ಕಣದಲ್ಲಿ ಕಾಣು ತ್ತಲೆ ನೆರೆದ ಗುಂಪಿನಲ್ಲಿ ಗುಜುಗುಜು ಪ್ರಾರಂಭವಾಯಿತು. ಗುಂಪಿನ ಒಂದು ಕಡೆಯಲ್ಲಿ ‘ಅಗೋ ಅಲ್ಲಿ ಮೇಘದಂತೆ ಕಪ್ಪಗೆ ಕಾಣುತ್ತಾನಲ್ಲ, ಅವನೇ ಶ್ರೀಕೃಷ್ಣ. ಎಳೆಯ ಕೂಸಾ ಗಿದ್ದಾಗಲೆ ಪೂತನಿಯನ್ನು ಕೊಂದನಂತೆ! ಅವನೇ ಗೋವರ್ಧನ ಪರ್ವತವನ್ನು ಬೆರಳಲ್ಲಿ ಎತ್ತಿದನಂತೆ! ಕಾಳಿಯನ್ನು ಕೊಂದವನಂತೆ! ಕೇಶಿ ಧೇನುಕ ಮೊದಲಾದ ಅನೇಕ ರಕ್ಕಸರನ್ನು ಕೊಂದುಹಾಕಿದನಂತೆ! ಗೋಕುಲದ ಗೋಪಿಯರನ್ನೆಲ್ಲ ಗೊಂಬೆಗಳಂತೆ ಆಡಿಸುತ್ತಾ ನಂತೆ! ಈತನನ್ನು ಸಾಕ್ಷಾತ್ ನಾರಾಯಣನೆಂದೆ ಹೇಳುತ್ತಾರೆ. ಅಗೋ ಅವನ ಪಕ್ಕದಲ್ಲಿ ನಿಂತಿದ್ದಾನಲ್ಲ, ಅವನೇ ಬಲರಾಮ. ಅವನು ಪ್ರಲಂಬ, ವತ್ಸ, ಧೇನುಕ ಮೊದಲಾದವರ ನ್ನೆಲ್ಲ ಬಲಿಹಾಕಿದನಂತೆ!’ ಎಂದು ಪರಸ್ಪರ ಮಾತನಾಡುತ್ತಿದ್ದರು. ಮತ್ತೊಂದು ಕಡೆ ಯಲ್ಲಿ ‘ಅಯ್ಯೋ ಪಾಪ, ಇನ್ನೂ ಎಳೆಯ ಕಂದಮ್ಮಗಳು, ಇವು. ಇದು ಏನನ್ಯಾಯ! ಕೋಣಗಳಂತೆ ಕೊಬ್ಬಿ ಬೆಳೆದಿರುವ ಈ ಜಟ್ಟಿಗಳೆಲ್ಲಿ, ಕೆನ್ನೆ ಹಿಂಡಿದರೆ ಇನ್ನೂ ಹಾಲು ತೊಟ್ಟಿಕ್ಕುವ ಈ ಎಳೆಯ ಮಕ್ಕಳೆಲ್ಲಿ? ಈ ಜೋಡಿಗಳಿಗೆ ಕುಸ್ತಿಯನ್ನು ಏರ್ಪಡಿಸಿರುವ ಈ ಕಂಸನ ಮನೆ ಹಾಳಾಗ’ ಎಂದು ಜನ ಪರಿತಪಿಸುತ್ತಿದ್ದರು. ಹೆಂಗಸರ ಗುಂಪಿನಲ್ಲಂತೂ ಶ್ರೀಕೃಷ್ಣಬಲರಾಮರ ಸೌಂದರ್ಯ ವರ್ಣನೆ ತಡೆಯಿಲ್ಲದೆ ನಡೆದಿತ್ತು. ‘ಶ್ರೀಕೃಷ್ಣನ ವೇಣುಗಾನವನ್ನು ಕೇಳುವ ಗೋಪಿಯರು ಏನು ಧನ್ಯರೋ!’ ಎಂಬುದು ಅವರ ಪಲ್ಲವಿ, ಅನುಪಲ್ಲವಿ. ನಂದಾದಿಗೋಪಾಲರಿಗೆ ಶ್ರೀಕೃಷ್ಣನ ಮಹತ್ತು ತಿಳಿದಿದ್ದರೂ, ಅವರಿಗೂ ಕೂಡ ಮನಸ್ಸಿಗೆ ಏನೋ ಆತಂಕ, ಅವರು ಚಡಪಡಿಸುತ್ತಿದ್ದರು.

ಕುಸ್ತಿ ಪ್ರಾರಂಭವಾಯಿತು. ಕೃಷ್ಣ-ಚಾಣೂರ, ಬಲರಾಮ-ಮುಷ್ಠಿಕ ಈ ಎರಡು ಜೋಡಿಗಳೂ ಕುಸ್ತಿಯ ಕಲೆಯಲ್ಲಿ ನುರಿತ ಜಟ್ಟಿಗಳೇ. ಸ್ವಲ್ಪ ಹೊತ್ತು ಈ ಜೋಡಿಗಳು ಸಮಸಮವಾಗಿಯೇ ಹೋರಾಡಿದವು. ಈ ಹೋರಾಟವನ್ನು ಹೆಚ್ಚು ಬೆಳೆಸಬಾರದೆನಿಸಿತು ಶ್ರೀಕೃಷ್ಣನಿಗೆ; ತನ್ನ ಒಂದು ಗುದ್ದಿನಿಂದ ಆತ ಚಾಣೂರನ ಅರ್ಧಬಲವನ್ನು ಕುಗ್ಗಿಸಿದನು. ಶಕ್ತಿ ಕುಗ್ಗುತ್ತಲೆ ಕೋಪ ಹೆಚ್ಚಿತು, ಚಾಣೂರನಿಗೆ. ಅವನು ಗಿಡಗನಂತೆ ಹಾರಿಬಂದು ಶ್ರೀಕೃಷ್ಣನ ಎದೆಯ ಮೆಲೆ ತನ್ನೆರಡು ಕೈಗಳಿಂದಲೂ ಬಲವಾಗಿ ಗುದ್ದಿದ. ಆದರೆ ಅದು ಮದ್ದಾನೆಯನ್ನು ಹೂಮಾಲೆಯಿಂದ ಹೊಡೆದಂತಾಯಿತು, ಅಷ್ಟೆ. ಶ್ರೀಕೃಷ್ಣನು ನಗುತ್ತಾ ಅವನ ತೋಳುಗಳೆರಡನ್ನು ಹಿಡಿದು, ಅವನನ್ನು ಗಿರಿಗಿರಿ ತಿರುಗಿಸಿದವನೇ ನೆಲಕ್ಕೆ ಅಪ್ಪಳಿಸಿ, ಕಾಲಿನಿಂದ ತುಳಿದ. ಅಲ್ಲಿಗೆ ಚಾಣೂರನ ಕತೆ ಮುಗಿಯಿತು. ಅತ್ತ ಬಲರಾಮನು ಮುಷ್ಠಿಕನ ತಲೆಯನ್ನು ತನ್ನ ವಜ್ರಮುಷ್ಠಿಯಿಂದ ಒಂದು ಸಲ ಗುದ್ದುತ್ತಲೆ, ಆ ರಕ್ಕಸ ನೆತ್ತರನ್ನು ಕಾರುತ್ತಾ ಬಿರುಗಾಳಿಗೆ ಬಿದ್ದ ಹೆಮ್ಮರದಂತೆ ನೆಲಕ್ಕುರುಳಿದ. ಇದನ್ನು ಕಂಡು ಮುಷ್ಠಿಕನ ಶಿಷ್ಯನಾದ ಕೂಟನೆಂಬ ಜಟ್ಟಿ ಅಬ್ಬರಿಸುತ್ತಾ ಹಾರಿಬಂದು ಬಲರಾಮನ ಮೇಲೆ ಬಿದ್ದ. ಬಲರಾಮ ನಗುತ್ತಾ ತನ್ನ ಎಡಗೈಯಿಂದ ಗುದ್ದಿ ಅವನನ್ನು ಸಾಯಿಸಿದ. ಅದೇ ಕಾಲದಲ್ಲಿ ಮೇಲೆ ಬಂದು ಬಿದ್ದ ಶಲ, ತೋಸಲಕ–ಎಂಬ ಇಬ್ಬರು ಜಟ್ಟಿಗಳನ್ನು ಶ್ರೀಕೃಷ್ಣ ತನ್ನೆರಡು ಕೈಗಳಿಂದಲೂ ಹಿಡಿದು ಗಿರಿಗಿರಿ ತಿರುಗಿಸಿದವನೇ ನೆಲಕ್ಕೆ ಬಡಿದು, ಕಾಲಿನಿಂದ ಹಿಸಿಕಿಹಾಕಿದ. ಇದನ್ನು ಕಂಡು ಅಲ್ಲಿದ್ದ ಉಳಿದ ಜಟ್ಟಿಗಳೆಲ್ಲ ಪ್ರಾಣಭಯ ದಿಂದ ಓಡಿಹೋದರು. ಗೆದ್ದ ಶ್ರೀಕೃಷ್ಣಬಲರಾಮರ ಸುತ್ತ ಗೋಪಾಲ ಬಾಲಕರು ಬಂದು ನೆರೆದು, ಓಡಿಹೋಗುವ ಜಟ್ಟಿಗಳನ್ನು ನೋಡಿ, ಚಪ್ಪಾಳೆ ತಟ್ಟುತ್ತಾ ಕುಣಿದಾಡಿದರು.

ತನ್ನವರಿಗಾದ ದುರ್ಗತಿಯನ್ನು ಕಂಡು ಕಂಸನ ಕೋಪ ಧಗಧಗ ಉರಿವ ದಳ್ಳುರಿ ಯಂತಾಯಿತು. ಆತನು ತನ್ನ ದೂತರೊಡನೆ ಸಿಡಿಲಿನಂತಹ ಕಠೋರ ಧ್ವನಿಯಿಂದ ‘ಎಲೆ ಭಟರೆ, ಈ ಹುಡುಗರಿಬ್ಬರನ್ನೂ ಈಗಿಂದೀಗಲೇ ನಮ್ಮೂರಿನಿಂದ ಒದ್ದು ಓಡಿಸಿರಿ. ಈ ಗೊಲ್ಲರ ಆಸ್ತಿಪಾಸ್ತಿಗಳೇನಿದ್ದರೂ ಅವನ್ನು ಕಿತ್ತುಕೊಂಡು ಅವರೆನ್ನೆಲ್ಲ ಹೊಡೆದಟ್ಟಿರಿ. ಈ ನಂದನನ್ನು ಎಳೆದುಕೊಂಡುಹೋಗಿ ಸೆರೆಯಿಲ್ಲಿಡಿರಿ. ಆ ನೀಚನಾದ ವಸುದೇವನನ್ನು ಈ ತಕ್ಷಣವೇ ಕತ್ತರಿಸಿ ಹಾಕಿರಿ. ಇವರಿಗೆಲ್ಲ ಆಶ್ರಯವಿತ್ತಿರುವ ನನ್ನ ತಂದೆ ಉಗ್ರಸೇನ ನನ್ನೂ ಅವನ ಪರಿವಾರದವರನ್ನೂ ದಯಾದಾಕ್ಷಿಣ್ಯವಿಲ್ಲದೆ ಕೊಂದುಹಾಕಿರಿ’ ಎಂದು ಅಪ್ಪಣೆ ಮಾಡಿದನು. ಇದನ್ನು ಕೇಳುತ್ತಲೆ ಶ್ರೀಕೃಷ್ಣನು ತಾನು ನಿಂತ ಸ್ಥಳದಿಂದ ಒಂದೇ ನೆಗೆತಕ್ಕೆ ಹಾರಿ, ಸಿಂಹಾಸನದ ಬಳಿಗೆ ಹೋದವನೆ ಕಂಸನ ಮೇಲೆ ಬಿದ್ದನು. ಕಂಸನು ಇದಕ್ಕೆ ಸಿದ್ಧನಾಗಿಯೇ ಇದ್ದನು. ಅವನು ತನ್ನ ಕತ್ತಿಯನ್ನು ಒರೆಯಿಂದ ಹಿರಿದು ತನ್ನ ಬೇಟೆಯ ಮೇಲೆ ಎರಗುವ ಗಿಡುಗನಂತೆ ಶ್ರೀಕೃಷ್ಣನ ಮೇಲೆ ಎರಗಿದನು. ಶ್ರೀಕೃಷ್ಣನು ಅವನ ಹೊಡೆತದಿಂದ ತಪ್ಪಿಸಿಕೊಂಡು, ಗರುಡನು ಹಾವಿನ ಹೆಡೆಯನ್ನು ಹಿಡಿಯುವಂತೆ ಕಂಸನ ತಲೆಯನ್ನು ಹಿಡಿದು, ಅವನ ಕಿರೀಟವನ್ನು ಅಲ್ಲಿಂದ ಕಿತ್ತು ಬಿಸುಟನು. ಅನಂತರ ಅವನ ತಲೆಗೂದಲನ್ನು ಹಿಡಿದೆಳೆದು, ಅವನನ್ನು ಸಿಂಹಾಸನದಿಂದ ಕೆಳಕ್ಕುರುಳಿಸಿದನು. ಸುತ್ತಮುತ್ತಿನ ಜನರೆಲ್ಲ ಹಾಹಾಕಾರ ಮಾಡುತ್ತಿರಲು, ಶ್ರೀಕೃಷ್ಣನು ಕೆಳಗೆ ಬಿದ್ದ ಕಂಸನನ್ನು ನೆಲಕ್ಕೆ ಕುಕ್ಕಿ ಕೊಂದುಹಾಕಿದನು. ಅಷ್ಟಕ್ಕೆ ಬಿಡದೆ ಆತನು ಹೆಣದ ಕಾಲನ್ನು ಹಿಡಿದು ನೆಲದ ಮೇಲೆಲ್ಲ ಎಳೆದಾಡಿದನು. ಹೀಗೆ ಶ್ರೀಕೃಷ್ಣನಿಂದ ತಮ್ಮ ಅಣ್ಣನಾದ ಕಂಸನು ಹತನಾದು ದನ್ನು ಕಂಡು ಅಲ್ಲಿಯೇ ಇದ್ದ ಕಂಕ, ನ್ಯಗ್ರೋಧಕ ಮೊದಲಾದ ಅವನ ಎಂಟು ಮಂದಿ ಸಹೋದರರೂ ಶ್ರೀಕೃಷ್ಣನ ಮೇಲೆ ಏರಿಬಂದರು. ಆದರೆ ಬಲರಾಮನು ಅಲ್ಲಿಯೇ ಇದ್ದ ಒಂದು ಗದೆಯನ್ನು ತೆಗೆದುಕೊಂಡು, ಒಂದೆ ಏಟಿಗೆ ಆ ಎಂಟು ಮಂದಿಯನ್ನೂ ಕೊಂದುಹಾಕಿದನು.

ಕಂಸರಾಜನು ಸತ್ತ ಸುದ್ದಿಯನ್ನು ಕೇಳಿ ಅರಮನೆಯಲ್ಲಿದ್ದ ಆತನ ರಾಣಿಯರೆಲ್ಲ ‘ಹೋ’ ಎಂದಳುತ್ತಾ ಅಲ್ಲಿಗೆ ಓಡಿಬಂದರು. ಅವರು ಕಂಸನ ಹೆಣದ ಮೇಲೆ ಬಿದ್ದು ಹೊರಳಾಡುತ್ತಾ ‘ಅಯ್ಯೋ ಸ್ವಾಮಿ, ನಮ್ಮನ್ನು ತೊರೆದು ಹೋದೆಯಾ? ಇನ್ನು ನಮ್ಮ ಗತಿಯೇನು? ನಮ್ಮಂತೆ ಈ ಮಧುರೆಯೂ ಅನಾಥವಾಯಿತು. ಅಯ್ಯೋ, ಅನ್ಯಾಯವಾಗಿ ನೀನು ಈ ಪರಮ ಪುರುಷನಾದ ಶ್ರೀಕೃಷ್ಣನಲ್ಲಿ ದ್ವೇಷವನ್ನು ಬಿತ್ತಿ ಬೆಳಸಿದೆ. ನಿನ್ನ ಕರ್ಮವೆ ನಿನಗೆ ಮೃತ್ಯುವಾಯಿತು’ ಎಂದು ಅತ್ತರು. ಶ್ರೀಕೃಷ್ಣನು ಅವರನ್ನೆಲ್ಲಾ ಸಮಾಧಾನ ಪಡಿಸಿ, ಅಲ್ಲಿ ಸತ್ತು ಬಿದ್ದಿದ್ದ ಕಂಸನಿಗೂ ಅವನ ಸೋದರರಿಗೂ ಅಂತ್ಯಸಂಸ್ಕಾರಗಳನ್ನು ಮಾಡಿಸಿ ದನು. ಅನಂತರ ಆತನು ಬಲರಾಮನೊಡನೆ ನೇರವಾಗಿ ತನ್ನ ತಾಯಿ ತಂದೆಗಳಿದ್ದ ಸೆರೆ ಮನೆಗೆ ಬಂದು, ಅವರ ಕೈಕಾಲುಗಳಿಗೆ ಹಾಕಿದ್ದ ಸಂಕಲೆಗಳನ್ನು ಕಿತ್ತೆಸೆದು, ಅವರ ಪಾದಕ್ಕೆ ನಮಸ್ಕರಿಸಿದನು. ಆ ತಾಯ್ತಂದೆಗಳು ಮಕ್ಕಳಿಬ್ಬರನ್ನೂ ಬಾಚಿ ತಬ್ಬಿಕೊಂಡು, ಆನಂದ ಬಾಷ್ಪಗಳನ್ನು ಸುರಿಸಿದರು. ಶ್ರೀಕೃಷ್ಣನು ಅವರನ್ನು ಕುರಿತು ‘ಅಮ್ಮ, ಅಪ್ಪ! ನಮ್ಮ ಅದೃಷ್ಟ ಕೆಟ್ಟುದು. ಪ್ರೇಮಮಯರಾದ ನಿಮ್ಮಂತಹ ತಾಯ್ತಂದೆಗಳಿದ್ದರೂ ನಾವು ತಬ್ಬಲಿ ಗಳಂತೆ ಎಲ್ಲಿಯೋ ನಮ್ಮ ಶೈಶವಬಾಲ್ಯಗಳನ್ನು ಕಳೆಯಬೇಕಾಯಿತು. ಜನ್ಮವನ್ನು ಕೊಟ್ಟ ತಾಯ್ತಂದೆಗಳ ಪುಣವನ್ನು ನೂರು ವರ್ಷ ಸೇವೆ ಮಾಡಿದರೂ ತೀರಿಸುವುದಕ್ಕೆ ಸಾಧ್ಯ ವಿಲ್ಲ. ಮುಪ್ಪಿನಲ್ಲಿ ತಂದೆತಾಯಿಗಳ ಸೇವೆ ಮಾಡದ ಪಾಪಿಯನ್ನು ಯಮದೂತರು ಎಳೆ ದೊಯ್ದು, ತನ್ನ ಮಾಂಸವನ್ನೇ ತಾನು ತಿನ್ನುವಂತೆ ಮಾಡುತ್ತಾರಂತೆ! ಮುಪ್ಪಿನ ತಾಯ್ತಂದೆಗಳು, ಪತಿವ್ರತೆಯಾದ ಮಡದಿ, ಎಳೆಯ ಕೂಸಾಗಿರುವ ಮಗ, ಗುರುವಾದ ಬ್ರಾಹ್ಮಣ, ಮರೆಹೊಕ್ಕ ಸಜ್ಜನ–ಇವರನ್ನು ಕಾಪಾಡದ ಮನುಷ್ಯ ಬದುಕಿದ್ದರೂ ಸತ್ತಂತೆಯೆ. ನೋಡಿ, ಕಂಸನ ಬಾಧೆಯಿಂದ ಸಂಕಟಪಡುತ್ತಿದ್ದ ನಿಮ್ಮನ್ನು ಇದುವರೆಗೆ ನಾವು ಕಾಪಾಡಲಾರದೆ ಹೋದೆವು. ನಮ್ಮಂತಹ ಮಕ್ಕಳಿದ್ದು ತಾನೆ ಏನು ಪ್ರಯೋಜನ? ನಮ್ಮ ತಪ್ಪನ್ನು ಮನ್ನಿಸಿ’ ಎಂದನು. ಆ ಮಾತುಗಳನ್ನು ಕೇಳಿ ವಸುದೇವ ದೇವಕಿಯರು ಆತನನ್ನು ಬಾಚಿ, ತಬ್ಬಿಕೊಂಡು, ಆನಂದಬಾಷ್ಪಗಳನ್ನು ಸುರಿಸುತ್ತಾ ‘ಕೃಷ್ಣ, ಕೃಷ್ಣ’ಎಂದು ಗದ್ಗದಕಂಠದಿಂದ ನುಡಿದರು. 

ತಾಯ್ತಂದೆಗಳನ್ನು ಸಮಾಧಾನ ಪಡಿಸಿದಮೇಲೆ ಶ್ರೀಕೃಷ್ಣನು ತನ್ನ ತಾತನಾದ ಉಗ್ರ ಸೇನನ ಬಳಿಗೆ ಬಂದು, ಆತನನ್ನು ಸಿಂಹಾಸನದ ಮೇಲೆ ಕೂಡಿಸಿ, ‘ಅಜ್ಜ, ನೀವಿಂದು ಮತ್ತೆ ರಾಜರಾಗುವಿರಿ. ಇಗೋ ನಾನು ನಿಮ್ಮ ಸೇವಕ. ನಾನು ಸೇವಕನಾಗಿರುವಾಗ ದೇವತೆ ಗಳು ಕೂಡ ನಿಮಗೆ ವಿಧೇಯರಾಗಿ ಕಪ್ಪಕಾಣಿಕೆಗಳನ್ನು ತಂದೊಪ್ಪಿಸುತ್ತಾರೆ’ ಎಂದು ಹೇಳಿ ದನು. ಅನಂತರ ಆತನು ಕಂಸನ ಭಯದಿಂದ ಇದುವರೆಗೆ ತಲೆಮರೆಸಿಕೊಂಡಿದ್ದವರನ್ನೆಲ್ಲ ಕಂಡು, ಅವರಿಗೆ ಧನಕನಕಾದಿಗಳನ್ನು ಕೊಟ್ಟು, ಸಮಾಧಾನ ಹೇಳಿ, ಅವರೆಲ್ಲ ನೆಮ್ಮದಿ ಯಿಂದ ಬಾಳುವಂತೆ ಮಾಡಿದನು. ಮಧುರಾಪುರಿಯ ಜನರೆಲ್ಲರೂ ಸುಖಶಾಂತಿಗಳಿಂದ ಸಂತುಷ್ಟರಾಗಿರುವರೆಂಬುದನ್ನು ನಿಶ್ಚಯ ಮಾಡಿಕೊಂಡ ಮೇಲೆ ಶ್ರೀಕೃಷ್ಣನು ಬಲರಾಮ ನೊಡನೆ ನಂದಗೋಪನ ಬಳಿಗೆ ಹೋದನು. ಆ ಸಾಕುತಂದೆಯನ್ನು ಕಾಣುತ್ತಲೆ ಶ್ರೀ ಕೃಷ್ಣನು ಭಕ್ತಿಯಿಂದ ಆತನಿಗೆ ನಮಸ್ಕರಿಸಿ ‘ಅಪ್ಪ, ನೀವು ನಮ್ಮನ್ನು ಇದುವರೆಗೆ ಪುತ್ರ ವಾತ್ಸಲ್ಯದಿಂದ ಕಾಪಾಡಿದಿರಿ. ನಮ್ಮ ತಾಯ್ತಂದೆಗಳು ನಮ್ಮನ್ನು ಕಾಪಾಡಲಾರದೆ ಕೈ ಬಿಟ್ಟಾಗ, ನೀವು ನಮ್ಮನ್ನು ಪೋಷಿಸಿ ಬೆಳೆಸಿದಿರಿ. ನೀವೆ ನಿಜವಾಗಿ ನಮ್ಮ ತಾಯ್ತಂದೆಗಳು. ಅಪ್ಪ, ನೀವು ಈಗ ಗೋಕುಲಕ್ಕೆ ತೆರಳಿ, ನಾವು ಇನ್ನು ಕೆಲವು ದಿನಗಳು ಇಲ್ಲಿಯೇ ಇದ್ದು, ನಮ್ಮ ಬಂಧು ಬಳಗದವರನ್ನೆಲ್ಲ ಕಂಡು, ಮಾತಾಡಿಸಿಕೊಂಡು ಬರುತ್ತೇವೆ’ ಎಂದು ಹೇಳಿದನು. ಆತನಿಗೂ ಇತರ ಗೋಪಾಲರಿಗೂ ಬೇಕಾದಷ್ಟು ಉಡುಗೊರೆಗಳನ್ನು ಕೊಟ್ಟು ಮನ್ನಿಸಿದ ಮೇಲೆ ಶ್ರೀಕೃಷ್ಣನು ಆತನನ್ನು ಬೀಳ್ಕೊಟ್ಟು ಕಳುಹಿಸಿದನು.

