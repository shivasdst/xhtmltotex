
\chapter{೪. ಭಗವಂತನು ಬ್ರಹ್ಮನಿಗೆ ಉಪದೇಶಿಸಿದ ಭಾಗವತ}

ಶುಕಮುನಿಯು ಪರೀಕ್ಷಿದ್ರಾಜನನ್ನು ಕುರಿತು ‘ಅಯ್ಯಾ ಮಹಾರಾಜ, ಚತುರ್ಮುಖ ಬ್ರಹ್ಮನು ತನಗೆ ನೆಲೆವನೆಯಾದ ಭಗವಂತನ ನಾಭೀಕಮಲದಲ್ಲಿ ಕುಳಿತು ಜಗತ್ತನ್ನು ಸೃಷ್ಟಿಸುವುದು ಹೇಗೆಂದು ಆಳವಾಗಿ ಚಿಂತಿಸುತ್ತಿರಲು, ಮಹಾಜಲದ ಮಧ್ಯದಿಂದ ‘ತಪ’ ಎಂಬ ಶಬ್ದವು ಎರಡು ಸಲ ಕೇಳಿ ಬಂತು. ಬ್ರಹ್ಮನು ದಿಗ್ಗನೆ ಮೇಲಕ್ಕೆದ್ದು ಸುತ್ತಲೂ ನೋಡಿದನು; ಯಾರೂ ಕಾಣಬರಲಿಲ್ಲ. ಆದರೂ ಆ ಮಾತಿಗೆ ಬೆಲೆಕೊಟ್ಟು ತಪಸ್ಸಿಗೆ ಆರಂಭಿಸಿದನು. ದೇವಮಾನದ ಒಂದು ಸಹಸ್ರವರ್ಷಗಳವರೆಗೂ ಆತನು ಘೋರವಾದ ತಪಸ್ಸನ್ನು ಆಚರಿಸಲು ಭಗವಂತನು ಆತನ ತಪಸ್ಸಿಗೆ ಮೆಚ್ಚಿ, ಆತನಿಗೆ ತನ್ನ ವಾಸಸ್ಥಾನ ವಾದ ವೈಕುಂಠವನ್ನು ದರ್ಶನಮಾಡಿಸಿದನು. ಆ ಲೋಕದ ಸೊಬಗು ವರ್ಣನಾತೀತ. ಶುದ್ಧ ಸತ್ವಗುಣವು ತುಂಬಿತುಳುಕುತ್ತಿರುವ ಆ ಲೋಕದಲ್ಲಿ ಮಾಯೆಯ ಸುಳಿವೇ ಇಲ್ಲ; ಇನ್ನು ಕಾಮಕ್ರೋಧಾದಿಗಳಾಗಲಿ ಸಾವಾಗಲಿ ಎಲ್ಲಿಯದು? ಅಲ್ಲಿ ಎಲ್ಲಿ ನೋಡಿದರೂ ಭಗವದ್ಭಕ್ತರು; ಕೋಮಲ ಶ್ಯಾಮಲವರ್ಣದ ಮನೋಹರಾಕಾರರಾಗಿ, ಚತುರ್ಭುಜರಾದ ತೇಜಸ್ವಿಗಳಾಗಿ, ರತ್ನಖಚಿತಗಳಾದ ಸ್ವರ್ಣಾಭರಣಗಳಿಂದ ಅಲಂಕೃತರಾಗಿ, ಕಮಲನೇತ್ರ ರಾಗಿ ಕಂಗೊಳಿಸುತ್ತಿರುವರು. ಎಲ್ಲೆಲ್ಲಿಯೂ ಯೌವನದಿಂದ ಕೂಡಿದ ಹೆಣ್ಣು ಗಂಡು ಗಳು ವಿಮಾನಗಳಲ್ಲಿ ಸಂಚರಿಸುತ್ತಾ ಮಿಂಚಿನ ಬಳ್ಳಿಗಳಿಂದ ತುಂಬಿದ ಮೇಘಗಳನ್ನು ಜ್ಞಾಪಕಕ್ಕೆ ತರುವರು. ಸಾಕ್ಷಾತ್ ಲಕ್ಷ್ಮೀದೇವಿಯೇ ಅಲ್ಲಿ ಭಗವಂತನ ಪಾದಸೇವೆಯಲ್ಲಿ ತೊಡಗಿರುವಳು. ಭಕ್ತಪಾಲಕನೂ, ಯಜ್ಞನಾಯಕನೂ, ಜಗದೀಶ್ವರನೂ ಆದ ಭಗ ವಂತನು ನೋಟಕರ ಕಣ್ಣುಗಳಿಗೆ ಹಬ್ಬವನ್ನುಂಟುಮಾಡುತ್ತಾ ಪ್ರಸನ್ನವಾದ ಕಿರುನಗ್ ೆಯಿಂದಲೂ, ನಸುಗೆಂಪಾದ ಕಟಾಕ್ಷದಿಂದಲೂ, ಸುಂದರವಾದ ಮುಖಕಮಲದಿಂದಲೂ ಕೂಡಿ ರತ್ನ ಸಿಂಹಾಸನದಲ್ಲಿ ಕುಳಿತಿರುವನು. ಸ್ವಾನಂದದಲ್ಲಿ ಮಗ್ನನಾಗಿದ್ದ ಆ ಪೀತಾಂ ಬರಧಾರಿಯನ್ನು ಕಾಣುತ್ತಲೆ ಬ್ರಹ್ಮನು ರೋಮಾಂಚಗೊಂಡು, ಆನಂದಬಾಷ್ಪಗಳನ್ನು ಸುರಿಸುತ್ತಾ ಆತನ ಕಾಲಿಗೆರಗಿದನು.

ಬ್ರಹ್ಮನ ಭಕ್ತಿಯಿಂದ ಸಂತುಷ್ಟನಾದ ಭಗವಂತನು ಮುಗುಳ್ನಗೆಯೊಡನೆ ಆತನ ಕೈಗಳನ್ನು ಹಿಡಿದುಕೊಂಡು “ಅಯ್ಯಾ ಕಮಲಾಸನ, ನಿನ್ನ ತಪಸ್ಸಿಗೆ ನಾನು ಮೆಚ್ಚಿದೆ. ಶ್ರೇಯಸ್ಸಿನ ತುತ್ತತುದಿಯೇ ನನ್ನ ದರ್ಶನ. ಕೇಳು, ನಿನಗೇನು ಬೇಕು. ‘ತಪ’ ಎಂಬ ಎರಡು ಅಕ್ಷರಗಳಿಂದ ನಿನಗೆ ಸೂಚನೆ ಕೊಟ್ಟವನು ನಾನೆ. ನಿನ್ನ ತಪಸ್ಸಿನ ಫಲವೇ ಈ ವೈಕುಂಠದರ್ಶನ. ನಾನು ಸೃಷ್ಟಿ ಸ್ಥಿತಿ ಲಯಗಳನ್ನು ನಡೆಸುವುದು ತಪಸ್ಸಿನಿಂದಲೆ. ನನ್ನ ಅಂತರಂಗ ಬಹಿರಂಗ ಶಕ್ತಿ ತಪಸ್ಸೆ. ನಾನೆ ತಪಸ್ಸು” ಎಂದು ಹೇಳಿದನು. ಬ್ರಹ್ಮನು ಕೈಮುಗಿದುಕೊಂಡು ಅತ್ಯಂತ ವಿನಯದಿಂದ ಆತನೊಡನೆ “ತಂದೆ, ನೀನು ಸರ್ವೇಶ್ವರ, ಸರ್ವಾಂತರ್ಯಾಯಾಮಿ. ಸೃಷ್ಟಿ ಸ್ಥಿತಿ ಲಯಗಳನ್ನು ಬೇರೆಬೇರೆ ರೂಪಗಳನ್ನು ತಳೆದು ನಡೆಸುತ್ತಿರುವವನು ನೀನೇ. ಜೇಡರ ಹುಳು ತನ್ನ ತಂತುವಿನಿಂದ ಗೂಡನ್ನು ಕಟ್ಟಿ ಅದರಲ್ಲಿ ನೆಲಸುವಂತೆ, ನೀನು ಸೃಷ್ಟಿಸಿರುವ ಈ ಜಗತ್ತಿನಲ್ಲಿ ನೀನೇ ಲೀಲಾಮೂರ್ತಿಯಾಗಿರುವೆ. ಸ್ವಾಮಿ ನನಗೆ ಸೃಷ್ಟಿಶಕ್ತಿಯನ್ನು ನೀಡು; ನಿನ್ನ ಆಜ್ಞೆಯಂತೆ ಕಾರ್ಯಗಳನ್ನು ನಿರ್ವಹಿಸುತ್ತೇನೆ. ಇದರಲ್ಲಿ ತೊಡಗಿರುವಾಗ ‘ನಾನೇ ಲೋಕಕರ್ತ’ ಎಂಬ ಅಹಂಕಾರ ನನ್ನಲ್ಲಿ ತಲೆದೋರ ದಂತೆ ಅನುಗ್ರಹಿಸು. ನಿನ್ನ ಸೇವಾರೂಪವಾದ ಸೃಷ್ಟಿಕಾರ್ಯವನ್ನು ನಡೆಸುವಂತೆ ನನಗೆ ವರಕೊಡು” ಎಂದು ಬೇಡಿಕೊಂಡನು. ಆಗ ಪರಮಪುರುಷನು ಬ್ರಹ್ಮನನ್ನು ಕುರಿತು ‘ಮಗು, ನನ್ನ ರೂಪ ಗುಣ ಕರ್ಮಗಳ ನಿಜತತ್ವವೆಲ್ಲವೂ ನಿನಗೆ ಬೆಳ್ಳಂಬೆಳಕಾಗಲಿ. ನೋಡು, ನಾನು ಆದ್ಯಂತಗಳಿಲ್ಲದ, ಅದ್ವಿತೀಯನಾದ, ಪರಿಪೂರ್ಣ. ನಾನು ಪ್ರಪಂಚ ರೂಪನಾಗುವುದಕ್ಕೆ ನಿಮಿತ್ತವಾದುದು ನನ್ನ ಮಾಯೆ. ಮಿಥ್ಯಾರೂಪವಾದ ದೇಹಾದಿಗಳು ಸತ್ಯನಾದ ಪರಮಾತ್ಮನಂತೆ ಗೋಚರವಾಗುವುದು ಈ ಮಾಯೆಯ ಪ್ರಭಾವದಿಂದಲೇ. ಆದರೆ ನೀನು ಯಾವ ಕಲ್ಪದಲ್ಲಿಯೇ ಆಗಲಿ, ಎಂತಹ ವಿಚಿತ್ರ ಸೃಷ್ಟಿಯಲ್ಲಿಯೇ ಆಗಲಿ ಮಾಯೆಗೆ ಅಧೀನನಾಗುವುದಿಲ್ಲ’ ಎಂದು ಹೇಳಿ ಮಾಯವಾದನು. ಬ್ರಹ್ಮನು ಆ ಪರ ಮಾತ್ಮನನ್ನು ನಮಿಸಿ ಜಗತ್ತನ್ನು ಸೃಷ್ಟಿಸಿದನು.

ಜಗತ್ತನ್ನು ಸೃಷ್ಟಿಸಿದ ಬ್ಪಹ್ಮನು ತನ್ನ ಸೃಷ್ಟಿಯೆಲ್ಲವೂ ನೆಮ್ಮದಿಯಿಂದಿರುವುದಕ್ಕಾಗಿ ಭಾಗವತ ಧರ್ಮವನ್ನು ಪ್ರಚಾರಗೊಳಿಸಬೇಕೆಂದು ಬಗೆದು, ಯಮ ನಿಯಮಗಳನ್ನು ಅವಲಂಬಿಸಿದನು. ಆಗ ನಾರದನು ಆತನ ಬಳಿಗೆ ಬಂದು, ತನ್ನ ಸೇವೆಯಿಂದ ಆತನನ್ನು ಮೆಚ್ಚಿಸಿ, ತನಗೆ ಭಾಗವತವನ್ನು ಉಪದೇಶಿಸಬೇಕೆಂದು ಬೇಡಿಕೊಂಡನು. ಬ್ರಹ್ಮನು ದಶಲಕ್ಷಣಗಳಿಂದ ಕೂಡಿದ ಆ ಭಾಗವತವನ್ನು ಭಗವಂತನಿಂದ ತಾನು ಕೇಳಿದ್ದಂತೆಯೇ ನಾರದನಿಗೆ ಉಪದೇಶಿಸಿದನು. ನಾರದನು ಅದನ್ನು ವ್ಯಾಸರಿಗೆ ಉಪದೇಶಿಸಿದನು. ಅವರು ಅದನ್ನು ಶುಕಮುನಿಗೆ ಹೇಳಿಕೊಟ್ಟರು. ಆತನು ಪರೀಕ್ಷಿದ್ರಾಜನಿಗೆ ಉಪದೇಶಕೊಟ್ಟನು.

ಭಾಗವತಕ್ಕೆ ಸರ್ಗ, ಪ್ರತಿಸರ್ಗ, ಸ್ಥಾನ, ಪೋಷಣ, ಊತಿ, ಮನ್ವಂತರ, ಈಶ್ವರಕಥೆ, ನಿರೋಧ, ಮುಕ್ತಿ, ಆಶ್ರಯ ಎಂಬ ಹತ್ತು ಲಕ್ಷಣಗಳು. ಪಂಚಭೂತಗಳು, ಪಂಚತ ನ್ಮಾತ್ರಗಳು, ಜ್ಞಾನೇಂದ್ರಿಯ ಕರ್ಮೇಂದ್ರಿಯಗಳು, ಬುದ್ಧಿಯೆಂಬ ಮಹತ್ತತ್ವ–ಇವು ಗಳ ಸೃಷ್ಟಿಯೇ ‘ಸರ್ಗ’; ದೇವ ಮನುಷ್ಯಾದಿ ಭೂತಸೃಷ್ಟಿಯೇ ‘ಪ್ರತಿಸರ್ಗ’; ಶ್ರೀಮ ನ್ನಾರಾಯಣನು ದುಷ್ಟರನ್ನು ದಮನಮಾಡಿ ಲೋಕ ರಕ್ಷಣೆ ಮಾಡಿದುದೇ ‘ಸ್ಥಾನ’; ಭಗವಂತನು ಅನೇಕ ಅವತಾರಗಳನ್ನೆತ್ತಿ ಭಕ್ತರಕ್ಷಣೆ ಮಾಡಿದುದೇ ‘ಪೋಷಣ’; ಜನ್ಮಾಂ ತರ ಕರ್ಮಗಳೇ ‘ಊತಿ’; ರಕ್ಷಣಕಾರ್ಯದಲ್ಲಿ ನಿಯುಕ್ತರಾದ ಮಹಾತ್ಮರ ಧರ್ಮಗಳ ವರ್ಣನೆಯೇ ‘ಮನ್ವಂತರ’; ಬೇರೆಬೇರೆ ಅವತಾರಗಳಲ್ಲಿ ಭಗವಂತನು ನಡೆಸಿದ ಅದ್ಭುತಕಾರ್ಯಗಳೇ ‘ಈಶಾನುಕಥನ’; ಜೀವಾತ್ಮನು ಕರ್ಮರೂಪದ ಶಕ್ತಿಯೊಡಗೂಡಿ ಸೂಕ್ಷ್ಮಪ್ರಕೃತಿಯಲ್ಲಿರುವುದು ‘ನಿರೋಧ’; ಜೀವನು ದೈವಾನುಗ್ರಹಕ್ಕೆ ಪಾತ್ರನಾಗಿ ಬ್ರಹ್ಮೀಭಾವವನ್ನು ಪಡೆಯುವುದೇ ‘ಮುಕ್ತಿ’; ಸೃಷ್ಟಿ ಸ್ಥಿತಿ ಲಯಗಳಿಗೆ ಕಾರಣನಾದ ಪರ ಮಾತ್ಮನೆ ‘ಆಶ್ರಯ’. ಈ ಪರಮಾತ್ಮನು ಜೀವನಲ್ಲಿ ನಿಯಾಮಕನಾಗಿ ‘ಆಧ್ಯಾತ್ಮಿಕ’ ನೆಂದೂ, ಇಂದ್ರಿಯಾಭಿಮಾನಿಗಳಾದ ದೇವತೆಗಳಲ್ಲಿದ್ದುಕೊಂಡು ‘ಆಧಿದೈವಿಕ’ನೆಂದೂ, ಶಬ್ದಾದಿ ವಿಷಯಗಳಲ್ಲಿ ಅಂತರ್ಯಾಮಿಯಾಗಿ ‘ಆಧಿಭೌತಿಕ’ನೆಂದೂ ಕರೆಸಿಕೊಳ್ಳುವನು; ಆದ್ದರಿಂದ ಆತನು ಜ್ಞಾತೃ, ಜ್ಞೇಯ, ಜ್ಞಾನಗಳೆಂಬ ತ್ರಿಪುಟೀ ಸ್ವರೂಪನು. ಆ ಭಗವಂತನೇ ಹೇಳಿದುದು ‘ಭಾಗವತ’.

