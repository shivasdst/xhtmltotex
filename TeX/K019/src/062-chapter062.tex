
\chapter{೬೨. ಶ್ರೀಕೃಷ್ಣನ ಕೃಪೆ ಹಸ್ತಿನಾವತಿಯತ್ತ}

ಉದ್ಧವನು ನಂದಗೋಕುಲದಿಂದ ಹಿಂದಿರುಗಿದ ಮೇಲೆ, ಶ್ರೀಕೃಷ್ಣನು ಮಧುರೆಯಲ್ಲಿ ತಾನು ಮಾಡಬೇಕಾಗಿದ್ದ ಕಾರ್ಯಗಳತ್ತ ಗಮನವನ್ನು ಹರಿಸಿದನು. ಕಂಸನನ್ನು ಕೊಲ್ಲುವ ಮುನ್ನಾದಿನ ಆತನು ‘ತ್ರಿವಕ್ರೆ’ ಎಂಬ ಗೂನಿಯಿಂದ ಗಂಧವನ್ನು ಸ್ವೀಕರಿಸಿ, ಅವಳ ಮನೆಗೆ ಬರುವುದಾಗಿ ಮಾತು ಕೊಟ್ಟಿದ್ದನಷ್ಟೆ! ಶ್ರೀಕೃಷ್ಣನು ತನ್ನ ಮಾತಿನಂತೆ ಆ ಭಕ್ತೆಯ ಮನೆಗೆ ಹೋದನು. ಅವಳು ದಿನದಿನವೂ ಶ್ರೀಕೃಷ್ಣನ ನಿರೀಕ್ಷಣೆಯಲ್ಲಿ ಕ್ಷಣವೊಂದು ಯುಗವಾಗಿ ಕಾಲವನ್ನು ಕಳೆಯುತ್ತಿದ್ದಳು. ಆದ್ದರಿಂದ ಆತನು ಮನೆಗೆ ಬರುತ್ತಲೆ ಆಕೆ ಸಡಗರದಿಂದ ಆತನನ್ನು ಕೈಹಿಡಿದು ಕರೆದೊಯ್ದು ಪೀಠದ ಮೇಲೆ ಕುಳ್ಳಿರಿಸಿ ಪೂಜಿಸಿದಳು. ಅವಳ ಅಪೇಕ್ಷೆಯೇನೆಂಬುದನ್ನು ಅರಿತ ಶ್ರೀಕೃಷ್ಣನು ಕೇವಲ ಕಾಮುಕನಂತೆ ಅವಳ ಮಲಗುವ ಮನೆಗೆ ಹೋಗಿ, ಅಲ್ಲಿ ಮನೋಹರವಾಗಿ ಅಲಂಕೃತವಾಗಿದ್ದ ಮಂಚದ ಮೇಲೆ ಕುಳಿತನು. ತ್ರಿವಕ್ರೆ ದಿವ್ಯಾಲಂಕಾರಭೂಷಿತೆಯಾಗಿ ಆತನ ಬಳಿಗೆ ಹೋದಳು. ಶ್ರೀಕೃಷ್ಣನು ಅವಳನ್ನು ಕೈಹಿಡಿದು ತನ್ನ ಪಕ್ಕದಲ್ಲಿ ಕುಳ್ಳಿರಿಸಿಕೊಂಡು ಅವಳ ಭುಜದ ಮೇಲೆ ಕೈಹಾಕಿದನು. ಈ ಸಲಿಗೆಯಿಂದ ಅವಳು ಧೈರ್ಯಗೊಂಡು, ಆಕೆ ಆತನನ್ನು ಅಪ್ಪಿಕೊಂಡಳು. ಅವಳ ಬಹುದಿನ ಗಳ ಆಶೆ ನೆರವೇರಿತು.

ತ್ರಿವಕ್ರೆಯ ಆಶೆಯನ್ನು ನೆರವೇರಿಸಿದ ಮೇಲೆ ಶ್ರೀಕೃಷ್ಣನು ಅಕ್ರೂರನಿಗೆ ಕೊಟ್ಟ ಮಾತನ್ನು ಉಳಿಸಿಕೊಳ್ಳುವುದಕ್ಕಾಗಿ ಉದ್ಧವ ಬಲರಾಮನೊಡನೆ ಆತನ ಮನೆಗೆ ಹೋದನು. ಅದನ್ನು ಕಂಡು ಅಕ್ರೂರನಿಗೆ ಪರಮ ಸಂತೋಷವಾಯಿತು. ಆತನು ಆ ಮೂವರಿಗೂ ಕಾಲನ್ನು ತೊಳೆದು ಮಣೆಯ ಮೇಲೆ ಕುಳ್ಳಿರಿಸಿ ನಮಸ್ಕರಿಸಿದನು. ಅನಂತರ ಶ್ರೀಕೃಷ್ಣನ ಪಾದಗಳನ್ನು ತನ್ನ ತೊಡೆಯ ಮೇಲಿಟ್ಟು ಕೊಂಡು ಒತ್ತುತ್ತಾ ‘ಸ್ವಾಮಿ, ಪಾಪಿಯಾದ ಕಂಸನನ್ನು ಕೊಂದು ನೀನು ನಮ್ಮನ್ನೆಲ್ಲಾ ಬದುಕಿಸಿದೆ. ಇದು ನಿನಗೇನೂ ದೊಡ್ಡ ಕೆಲಸವಲ್ಲ, ಏಕೆಂದರೆ ನೀನು ಸಾಕ್ಷಾತ್ ಭಗವಂತ. ನೀನು ಹುಟ್ಟಿರುವುದೇ ಭೂಮಿಯ ಭಾರವನ್ನು ಹೋಗಲಾಡಿಸುವುದಕ್ಕಾಗಿ. ಇನ್ನೂ ಎಷ್ಟು ಜನ ಪಾಪಿಗಳನ್ನು ಕೊಲ್ಲ ಬೇಕಾಗಿದೆಯೋ! ಪಾವನಕ್ಕೆ ಪಾವನವಾಗಿರುವ ನಿನ್ನ ಪಾದಗಳ ಧೂಳಿ ನನ್ನ ಮನೆಯಲ್ಲಿ ಬಿದ್ದುದರಿಂದ ನಾನು ಧನ್ಯನಾದೆ, ನನ್ನ ಕುಲಕೋಟಿಗಳು ಉದ್ಧಾರವಾದವು. ನೀನು ನಿನ್ನ ಭಕ್ತರಿಗೆ ಬೇಡಿದುದನ್ನು ಕೊಡತಕ್ಕವನು. ಸ್ವಾಮಿ, ಇಗೋ ನಿನಗೆ ಅಡ್ಡಬಿದ್ದು ಬೇಡಿಕೊಳ್ಳುತ್ತಿದ್ದೇನೆ, ನಿನ್ನ ಮಾಯೆಯಿಂದ ಗಂಟುಬಿದ್ದಿರುವ ನನ್ನ ಸಂಸಾರಮೋಹವನ್ನು ಹೋಗಲಾಡಿಸಿ, ನನ್ನನ್ನು ಉದ್ಧರಿಸು’ ಎಂದು ಬೇಡಿದನು.

ಅಕ್ರೂರನ ಪೂಜೆಯಿಂದ ಸಂತಸಗೊಂಡ ಶ್ರೀಕೃಷ್ಣನು ಮಂದಹಾಸದೊಡನೆ ಆತ ನನ್ನು ಕುರಿತು ‘ಚಿಕ್ಕಪ್ಪ, ನೀನು ನಮಗೆ ಹಿರಿಯ, ಬುದ್ಧಿ ಹೇಳುವ ಗುರು; ನಿನಗೆ ಮಕ್ಕಳಂ ತಿರುವ ನಮ್ಮನ್ನು ನೀನು ಲಾಲನೆ ಪಾಲನೆ ಮಾಡಬೇಕೆ ಹೊರತು ಹೀಗೆಲ್ಲ ಪೂಜಿಸು ವುದೂ, ಗೌರವಿಸುವುದೂ ಅಗತ್ಯವಿಲ್ಲ. ನಿನ್ನಂತಹ ಭಗವದ್ಭಕ್ತರ ಸೇವೆಯಿಂದಲೇ ಜನ ಉದ್ಧಾರವಾಗಬೇಕು. ಸ್ವಾರ್ಥಪರರಾದ ದೇವತೆಗಳಿಗಿಂತಲೂ ಪರೋಪಕಾರಿಗಳಾದ ನಿಮ್ಮಂತಹ ಸಾಧುಗಳೇ ದೊಡ್ಡವರು. ನಿಮ್ಮಂತಹ ಮಹಾನುಭಾವರು ನಮ್ಮವರಾಗಿರು ವುದೇ ನಮಗೊಂದು ಭಾಗ್ಯ. ಚಿಕ್ಕಪ್ಪ, ಈಗ ನಿಮ್ಮಿಂದ ನನಗೆ ದೊಡ್ಡದೊಂದು ಉಪಕಾರವಾಗಬೇಕಾಗಿದೆ. ನಮ್ಮ ಸೋದರತ್ತೆಯಾದ ಕುಂತಿದೇವಿಯ ಗಂಡ ಪಾಂಡು ರಾಜನು ಸತ್ತುಹೋದನಂತೆ. ಆತನ ಅಣ್ಣನಾದ ಧೃತರಾಷ್ಟ್ರ ಕುಂತಿಯನ್ನೂ ಆಕೆಯ ಮಕ್ಕಳಾದ ಪಂಚಪಾಂಡವರನ್ನೂ ಹಸ್ತಿನಾವತಿ ಪಟ್ಟಣಕ್ಕೆ ಕರೆದೊಯ್ದನಂತೆ. ಆ ಧೃತರಾಷ್ಟ್ರ ಮುದುಕ, ಅದರ ಮೇಲೆ ಕಣ್ಣಿಲ್ಲದ ಕಬೋಜಿ. ದುಷ್ಟರಾದ ಧುರ್ಯೋಧನನೇ ಮೊದಲಾದ ತನ್ನ ಮಕ್ಕಳನ್ನು ತನ್ನ ಹತೋಟಿಯಲ್ಲಿಟ್ಟುಕೊಳ್ಳುವುದು ಅವನಿಗೆ ಸಾಧ್ಯ ವಾಗುತ್ತದೆಯೋ ಇಲ್ಲವೊ! ಆ ಮಕ್ಕಳ ಮಾತನ್ನು ಕೇಳಿಕೊಂಡು ಅವನು ಪಾಂಡವರಿಗೆ ಕೇಡು ಬಗೆದರೆ ಏನು ಗತಿ? ಆದ್ದರಿಂದ ನೀನು ಈಗಲೆ ಹಸ್ತಿನಾವತಿಗೆಹೋಗಿ, ಪಾಂಡವರ ಸ್ಥಿತಿಗತಿಗಳನ್ನು ತಿಳಿದುಕೊಂಡು ಬರಬೇಕು. ಅದನ್ನು ತಿಳಿದ ಮೇಲೆ ಪಾಂಡವರು ನೆಮ್ಮದಿಯಿಂದಿರುವುದಕ್ಕೆ ತಕ್ಕ ಪ್ರಯತ್ನಗಳನ್ನು ಮಾಡೋಣ’ ಎಂದನು.

ಶ್ರೀಕೃಷ್ಣನ ಅಪ್ಪಣೆಯಂತೆ ಅಕ್ರೂರನು ಹಸ್ತಿನಾವತಿಗೆ ಹೋದನು. ಅಲ್ಲಿ ಆತನು ಕುರು ವಂಶಕ್ಕೆ ಹಿರಿಯನಾದ ಭೀಷ್ಮನನ್ನೂ ರಾಜನಾದ ಧೃತರಾಷ್ಟ್ರನನ್ನೂ ವಿದುರ ಕುಂತಿಯ ರನ್ನೂ ಕಂಡು ನಮಸ್ಕರಿಸಿ, ಅವರ ಯೋಗಕ್ಷೇಮವನ್ನು ವಿಚಾರಿಸಿದ ಮೇಲೆ ತನಗೆ ಪರಿಚಿತರಾದ ಬಾಹ್ಲೀಕ, ಸೋಮ ದತ್ತ, ದ್ರೋಣ, ಕೃಪ, ಕರ್ಣ, ದುರ್ಯೋಧನಾದಿ ಗಳನ್ನೂ ತನಗೆ ಪರಮಮಿತ್ರರಾದ ಪಾಂಡವರನ್ನೂ ಕಂಡು ಅವರ ಯೋಗಕ್ಷೆಮವನ್ನೆಲ್ಲ ವಿಚಾರಿಸಿದನು. ತಾನು ಬಂದ ಕಾರ್ಯವನ್ನು ಸಾಧಿಸುವುದಕ್ಕಾಗಿ ಆತನು ಹಸ್ತಿನಾವತಿ ಯಲ್ಲಿಯೇ ಕೆಲಕಾಲ ನಿಲ್ಲಬೇಕಾಯಿತು. ಕ್ರಮಕ್ರಮವಾಗಿ ಆತನಿಗೆ ಅಲ್ಲಿಯ ಸುದ್ದಿಯೆಲ್ಲ ಗೊತ್ತಾಯಿತು. ಪಾಂಡವರು ಸಜ್ಜನರು, ವಿನಯಶೀಲರು, ಮಹಾ ಪರಾಕ್ರಮಶಾಲಿಗಳು. ತೇಜಸ್ವಿಗಳಾದ ಅವರನ್ನು ಕಂಡರೆ ಪ್ರಜೆಗಳಿಗೆಲ್ಲ ತುಂಬ ಭಕ್ತಿ, ವಿಶ್ವಾಸ, ಆದರ, ಗೌರವ. ಇದನ್ನು ನೋಡಿ ದುರ್ಯೋಧನನೇ ಮೊದಲಾದ ಧೃತರಾಷ್ಟ್ರನ ಮಕ್ಕಳಿಗೆಲ್ಲ ಪಾಂಡವರಲ್ಲಿ ತುಂಬ ಹೊಟ್ಟೆ ಕಿಚ್ಚು. ಒಂದು ದಿನ ಕುಂತಿ ವಿದುರರು ಅಕ್ರೂರನ ಮುಂದೆ ಕೌರವರು ಮಾಡಿದ ಕಿಡಿಗೇಡಿತನವನ್ನೆಲ್ಲ–ಪಾಂಡವರಿಗೆ ವಿಷದ ಅನ್ನವನ್ನು ಊಟ ಮಾಡಿಸಿದುದು, ಹಾವುಗಳಿಂದ ಕಚ್ಚಿಸಿದುದು, ಭೀಮನನ್ನು ಕೈಕಾಲು ಕಟ್ಟಿ ಮಡುವಿಗೆ ಎಸೆದುದು–ಇತ್ಯಾದಿ ಗಳಬಿನ್ನೆಲ್ಲ ವಿವರವಾಗಿ ವರ್ಣಿಸಿ ಹೇಳಿದರು. ಕುಂತಿಯಂತೂ ಗಳಗಳ ಅಳುತ್ತಾ ‘ಅಪ್ಪಾ ಅಕ್ರೂರ, ನಮ್ಮ ತೌರುಮನೆಯವರು ಯಾರೂ ನನ್ನನ್ನು ನೆನೆಸಿಕೊಳ್ಳುವಂತೆಯೇ ಕಾಣು ತ್ತಿಲ್ಲ. ನಮ್ಮ ಕಷ್ಟದಲ್ಲಿ ಒಬ್ಬರೂ ಇತ್ತ ತಲೆ ಹಾಕಿಲ್ಲ. ನನ್ನ ಅಣ್ಣನ ಮಕ್ಕಳಾದ ಬಲರಾಮ ಶ್ರೀಕೃಷ್ಣರಾದರೂ ತಮ್ಮ ಸೋದರತ್ತೆಯನ್ನೂ ಅವಳ ಮಕ್ಕಳನ್ನೂ ಕಣ್ಣೆತ್ತಿ ನೋಡಬಾರದೆ? ನಾವೆಲ್ಲ ಇಲ್ಲಿ ತೋಳಗಳ ಮಧ್ಯೆ ಸಿಕ್ಕಿಬಿದ್ದಿರುವ ಜಿಂಕೆಯ ಮರಿಗಳಂತಾಗಿದ್ದೇವೆ. ಗಂಡನಿಲ್ಲದೆ ಅನಾಥೆಯಾಗಿರುವ ನನ್ನನ್ನೂ ತಬ್ಬಲಿಗಳಾದ ನನ್ನ ಮಕ್ಕಳನ್ನೂ ಶ್ರೀಕೃಷ್ಣನಾದರೂ ವಿಚಾರಿಸಬೇಡವೆ? ಎಂದು ಗೋಳಾಡಿದಳು. ವಿದುರ ಅಕ್ರೂರರಿಬ್ಬರೂ ಆಕೆಯನ್ನು ‘ನಿನ್ನ ಮಕ್ಕಳು ಮಹಾ ಶೂರರು, ದೈವಾಂಶಸಂಭೂತರು; ಅವರಿಗೆ ಯಾವ ಕೇಡೂ ಆಗುವುದಿಲ್ಲ’ ಎಂದು ಸಮಾಧಾನ ಮಾಡಿದರು.

ಬಂದ ಕೆಲಸ ಮುಗಿದಂತಾದುದರಿಂದ ಅಕ್ರೂರನು ಹಸ್ತಿನಾವತಿಯಿಂದ ಮಧುರೆಗೆ ಹಿಂದಿರುಗಿದನು. ಹಾಗೆ ಹಿಂದಿರುಗುವ ಮುನ್ನ ಆತನು ಧೃತರಾಷ್ಟ್ರನನ್ನು ಕಂಡು ‘ಮಹಾರಾಜ, ಪಾಂಡುವು ಅರಣ್ಯಕ್ಕೆ ಹೋದುದರಿಂದ ನೀನು ಕುರುವಂಶದ ಮಹಾರಾಜನಾಗಿ ರುವೆ. ನಿನ್ನ ಹೊಣೆ ಹೊರೆಗಳು ಮಹತ್ತಾದುವು. ನೀನು ಪ್ರಜೆಗಳಿಗೆಲ್ಲ ಸಂತೋಷವಾಗುವಂತೆ ಧರ್ಮದಿಂದ ರಾಜ್ಯಭಾರ ಮಾಡಬೇಕು. ನಿನ್ನವರೆನ್ನುವವರನ್ನೆಲ್ಲ ಪಕ್ಷಪಾತವಿಲ್ಲದೆ ಸಮದೃಷ್ಟಿಯಿಂದ ಕಾಣಬೇಕು. ಇಲ್ಲದಿದ್ದರೆ ಇಹಪರಗಳೆರಡೂ ಹಾಳಾಗುತ್ತವೆ. ನಿನ್ನ ಮಕ್ಕಳನ್ನೂ ಪಾಂಡವರನ್ನೂ ಸಮಭಾವದಿಂದ ನೋಡಬೇಕು. ಧರ್ಮದೃಷ್ಟಿ ನಿನ್ನಿಂದ ಮರೆಯಾಗದಿರಲಿ. ಹುಟ್ಟುತ್ತಾ ಬಂದವನು ಒಬ್ಬನೇ, ಸಾಯುತ್ತಾ ಹೋಗುವವನು ಒಬ್ಬನೇ. ಉಳಿದ ಮಡದಿ ಮಕ್ಕಳೆಂಬ ಸಂಸಾರವೆಲ್ಲ ಮಧ್ಯೆ ಬಂದದು. ಇವರಾರೂ ನಾವು ಹೋಗುವಾಗ ಜೊತೆಯಲ್ಲಿ ಬರುವುದಿಲ್ಲ. ನಮ್ಮ ಜೊತೆಯಲ್ಲಿ ಬರುವುದು ಕೇವಲ ಪಾಪ ಪುಣ್ಯಗಳು ಮಾತ್ರ. ಆದ್ದರಿಂದ ನಾವು ಭೂಮಿಯಲ್ಲಿ ಬದುಕಿರುವಷ್ಟು ಕಾಲವು ನಮ್ಮ ಧರ್ಮಕ್ಕೆ ಲೋಪವಾಗದಂತೆ ನೋಡಿಕೊಳ್ಳಬೇಕು’ ಎಂದನು. ಆ ಮಾತುಗಳನ್ನು ಧೃತರಾಷ್ಟ್ರನು ತುಂಬ ಮೆಚ್ಚಿಕೊಂಡು ‘ಅಯ್ಯಾ, ನಿನ್ನ ಮಾತು ಸತ್ಯ, ಸತ್ಯವಾಗಿರುವಷ್ಟೆ ಪ್ರಿಯ’ ಎಂದು ಹೊಗಳಿದನು. ಅಕ್ರೂರನು ಆತನಿಂದ ಬೀಳ್ಕೊಂಡು ಮಧುರಾನಗರಿಗೆ ಹೊರಟು ಬಂದನು.

