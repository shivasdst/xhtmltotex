
\chapter{೫೪. ಗಂಡಿಗಿಂತ ಹೆಣ್ಣೆ ವಾಸಿ}

ಬಾಲಕನಾದ ಶ್ರೀಕೃಷ್ಣನ ಮಹತ್ತು ಸುತ್ತಮುತ್ತಿನ ಜಗತ್ತಿಗೆಲ್ಲ ವ್ಯಾಪಿಸಿತು. ಹಲವಾರು ಜನ ಆತನನ್ನು ಅವತಾರ ಪುರುಷನೆಂದು ಅರ್ಥಮಾಡಿಕೊಂಡರು. ಆದರೆ ಅವರಿಗೆಲ್ಲ ಕಂಸನ ಭಯ. ಅವನ ವೈರಿ ಶ್ರೀಕೃಷ್ಣ ಎಂದು ಎಲ್ಲರಿಗೂ ಗೊತ್ತು. ಆದ್ದರಿಂದ ಅವರು ಬಹಿರಂಗವಾಗಿ ಶ್ರೀಕೃಷ್ಣನೊಡನೆ ಯಾವ ವ್ಯವಹಾರವನ್ನೂ ನಡೆಸರು. ಹೀಗಿರುವಾಗ ಒಂದು ದಿನ ಬಲರಾಮ ಕೃಷ್ಣರು ಗೋಪಾಲಬಾಲರೊಡನೆ ಗೋವುಗಳನ್ನು ಮೇಯಿಸುತ್ತಾ ಗೋಕುಲದಿಂದ ಬಹುದೂರ ಹೋದರು. ಅಂದು ಅವರು ಒಬ್ಬರೂ ಬುತ್ತಿಯನ್ನು ತಂದಿರಲಿಲ್ಲ. ಬಿಸಿಲೇರುತ್ತಾ ಹೋದಂತೆ ಅವರ ಹೊಟ್ಟೆ ಚುರುಕಾಗುತ್ತಾ ಹೊರಟಿತು. ಗೊಲ್ಲರ ಹುಡುಗರೆಲ್ಲ ಶ್ರೀಕೃಷ್ಣನ ಸುತ್ತ ನಿಂತು ‘ಅಯ್ಯಾ, ನಮಗೆ ಬಹಳ ಹಸಿವು’ ಎಂದರು. ಶ್ರೀಕೃಷ್ಣನು ಸ್ವಲ್ಪ ಯೋಚಿಸುವವನಂತೆ ಮಾಡಿ ‘ಎಲ, ಒಂದು ಕೆಲಸಮಾಡಿ. ಇಲ್ಲಿಗೆ ಬಹು ಹತ್ತಿರದಲ್ಲಿಯೆ ಇರುವ ಈ ಕಾಡಿನ ಅಂಚಿನಲ್ಲಿ ಒಂದು ಅಗ್ರಹಾರವಿದೆ. ಅಲ್ಲಿನ ಬ್ರಾಹ್ಮಣರು ಈಗ ಒಂದು ಯಾಗವನ್ನು ಮಾಡುತ್ತಿದ್ದಾರೆ. ಯಾಗ ನಡೆಯುವಲ್ಲಿ ಅನ್ನಕ್ಕೇನು ಕೊರತೆ? ನೀವು ಹೋಗಿ ನನ್ನ ಮತ್ತು ಅಣ್ಣನ ಹೆಸರನ್ನು ಹೇಳಿ, ಸ್ವಲ್ಪ ಅನ್ನವನ್ನು ಕೊಡುವಂತೆ ಕೇಳಿ, ಅವರು ಖಂಡಿತವಾಗಿಯೂ ನೀವು ಕೇಳಿದಷ್ಟು ಅಹಾರ ವನ್ನು ಕೊಡುತ್ತಾರೆ. ನೀವು ತಿಂದು ನಮಗೂ ಸ್ವಲ್ಪ ತನ್ನಿ’ ಎಂದ. ಆತನ ಅಪ್ಪಣೆಯಂತೆ ಗೊಲ್ಲರ ಹುಡುಗರೆಲ್ಲ ಆ ಬ್ರಾಹ್ಮಣರ ಬಳಿಗೆ ಹೋಗಿ ಅನ್ನವನ್ನು ಬೇಡಿದರು. ಆದರೆ ಕರ್ಮಠರಾದ ಆ ಬ್ರಾಹ್ಮಣರು ಅವರ ಮಾತನ್ನು ಕಿವಿಗೆ ಹಾಕಿಕೊಳ್ಳಲಿಲ್ಲ. ಅವರಿಂದ ಯಾವ ಉತ್ತರವೂ ಬಾರದುದನ್ನು ಕಂಡು, ಗೊಲ್ಲರ ಹುಡುಗರು ನಿರಾಶೆಯಿಂದ ಕೃಷ್ಣನ ಬಳಿಗೆ ಹಿಂದಿರುಗಿದರು.

ಮುಖ ಒಣಗಿಸಿಕೊಂಡು, ಬರಿಗೈಲಿ ಬಂದ ಆ ಗೊಲ್ಲ ಬಾಲಕರಿಂದ ನಡೆದ ಸಂಗತಿ ಯನ್ನೆಲ್ಲ ಕೇಳಿ, ಶ್ರೀಕೃಷ್ಣ ನಕ್ಕ. ಯಾರ ಪ್ರೀತಿಗಾಗಿ ಅವರು ಯಾಗಮಾಡುವರೋ, ಅವನಿಗೆ, ಅಲ್ಲಿ ಒಂದು ತುತ್ತು ಅನ್ನ ಸಿಕ್ಕಲಿಲ್ಲವೆಂದರೆ! ಆತ ಗೊಲ್ಲರ ಹುಡುಗರೊಡನೆ ‘ಅಯ್ಯಾ, ತಿರುಪೆಗೆ ಹೊರಟಮೇಲೆ ನಾಚಿಗೆ ಪಟ್ಟರೆ ಹೇಗೆ? ಈಗ ನೀವು ಆ ಬ್ರಾಹ್ಮಣರಿಗೆ ಕಾಣದಂತೆ ಅವರ ಹೆಂಡಿರ ಹತ್ತಿರ ಹೋಗಿ, ನಮ್ಮ ಹೆಸರನ್ನು ಹೇಳಿ. ಅವರಿಗೆ ನಮ್ಮಲ್ಲಿ ತುಂಬ ಪ್ರೇಮ. ಅವರು ನೀವು ಕೇಳಿದಷ್ಟು ಆಹಾರ ಕೊಡುತ್ತಾರೆ’ ಎಂದ. ಪಾಪ, ಆ ಹುಡುಗರು ಶ್ರೀಕೃಷ್ಣ ಹೇಳಿದಂತೆಯೆ ಹೆಂಗಸರ ಹತ್ತಿರ ಹೋಗಿ ‘ತಾಯಿಯರೆ, ದನ ಗಳನ್ನು ಕಾಯುವುದಕ್ಕಾಗಿ ಅಡವಿಗೆ ಬಂದಿರುವ ಬಲರಾಮ ಕೃಷ್ಣರು ನಮ್ಮನ್ನು ನಿಮ್ಮ ಬಳಿಗೆ ಕಳುಹಿಸಿದ್ದಾರೆ. ನಮಗೆಲ್ಲ ತುಂಬ ಹmಸಿವು. ದಯಮಾಡಿ ಅನ್ನವಿಕ್ಕಿರಿ’ ಎಂದು ಕೇಳಿದರು. ಜನಜನಿತವಾಗಿದ್ದ ಶ್ರೀಕೃಷ್ಣನ ಕತೆಗಳನ್ನು ಕೇಳಿ, ಸದಾ ಆತನನ್ನೆ ಧ್ಯಾನಿಸುತ್ತಿದ್ದ ಆ ಹೆಂಗಸರು ಆತನ ಹೆಸರನ್ನು ಕೇಳುತ್ತಲೆ ಸಡಗರದಿಂದ ಮೇಲೆದ್ದು, ಮನೆಯಲ್ಲಿದ್ದ ಬಗೆಬಗೆಯ ಭಕ್ಷ್ಯಗಳನ್ನೆಲ್ಲ ಗಂಟುಕಟ್ಟಿಕೊಂಡು, ತಲೆಯಲ್ಲಿ ಹೊತ್ತು, ಶ್ರೀಕೃಷ್ಣನ ಬಳಿಗೆ ಹೊರಟರು. ಹೆಣ್ಣುಗಳ ಈ ಕಾರ್ಯಕ್ರಮವನ್ನು ಕಂಡು ಗಂಡಸರಿಗೆಲ್ಲ ರೇಗಿ ಹೋಯಿತು. ಅವರು ಆ ಹೆಣ್ಣುಗಳನ್ನು ತಡೆಯಲು ಪ್ರಯತ್ನಿಸಿದರು. ಆದರೆ ಅವರ ಕೈ ಸಾಗಲಿಲ್ಲ. ಹೆಂಗಸರೆಲ್ಲ ಆಹಾರವನ್ನು ಹೊತ್ತು ಶ್ರೀಕೃಷ್ಣನ ಬಳಿಗೆ ಬಂದರು. ಇದನ್ನು ಕಂಡ ಗೊಲ್ಲರ ಹುಡುಗರು ‘ಗಂಡಸಿಗಿಂತ ಹೆಂಗಸೆ ವಾಸಿ’ ಎಂದುಕೊಂಡರು.

ಬ್ರಾಹ್ಮಣರ ಹೆಂಗಸರು ಶ್ರೀಕೃಷ್ಣನ ಮೋಹನಾಕಾರವನ್ನು ಕಂಡು ಆನಂದಪರವಶ ರಾದರು. ಅಲ್ಲಿನವರೆಗೆ ಅವರು ಆತನ ಗುಣಗಳನ್ನು ಮಾತ್ರ ಕೇಳಿದ್ದರು. ಇಂದು ಆತನ ಪ್ರತ್ಯಕ್ಷದರ್ಶನವಾಗುತ್ತಲೆ ಅವರಿಗೆ ತಾವು ಧನ್ಯರಾದೆವೆನ್ನಿಸಿತು. ಶ್ರೀಕೃಷ್ಣನು ಮುಗು ಳ್ನಗೆಯಿಂದ ಅವರನ್ನು ಆದರಿಸಿ ‘ಅಮ್ಮ, ನನ್ನನ್ನು ನೋಡಲೆಂದು ನೀವೆಲ್ಲ ಬಂದಿರ ಲ್ಲವೆ? ತುಂಬ ಸಂತೋಷ. ಯಾವ ಫಲವನ್ನೂ ಬಯಸದೆ ನನ್ನ ಮೇಲೆ ಪ್ರೇಮದೋರು ವವರನ್ನು ಕಂಡರೆ ನನಗೆ ಬಲು ಸಂತೋಷ. ನಾನು ನಿಮ್ಮ ಪ್ರೇಮಕ್ಕೆ ಮಾರುಹೋಗಿ ದ್ದೇನೆ. ನೀವಿನ್ನು ಬೇಗ ಇಲ್ಲಿಂದ ಹಿಂದಿರುಗಿರಿ. ನಿಮ್ಮ ಗಂಡಂದಿರು ನಿಮಗಾಗಿ ಕಾಯು ತ್ತಿರುತ್ತಾರೆ. ನಿಮ್ಮ ಸಹಾಯದಿಂದಲೆ ಅವರು ಯಾಗವನ್ನು ಮುಗಿಸಬೇಕು’ ಎಂದನು. ಆದರೆ ಆ ಹೆಂಗಸರು ಅಷ್ಟು ಬೇಗ ಹಿಂದಿರುಗಲು ಸಿದ್ಧವಾಗಿರುವಂತೆ ಕಾಣಲಿಲ್ಲ. ಅವರು ‘ಶ್ರೀಕೃಷ್ಣ, ಪ್ರಭು, ನಾವು ನಿನ್ನನ್ನು ನಂಬಿ ಬಂದಿದ್ದೇವೆ. ನೀನು ನಂಬಿದವರನ್ನು ಕೈ ಬಿಡುವವನಲ್ಲ. ನಮಗೆ ಮನೆಗೆ ಹಿಂದಿರುಗಬೇಕೆಂಬ ಆಶೆ ಕೂಡ ಇಲ್ಲ. ನಾವು ಮಾತು ಮೀರಿ ಬಂದುದರಿಂದ ನಮ್ಮ ಪತಿಗಳು ಮತ್ತೆ ನಮಗೆ ಆಶ್ರಯ ಕೊಡುತ್ತಾರೊ, ಇಲ್ಲವೊ! ಲೋಕಕ್ಕೇ ಆಶ್ರಯನಾಗಿರುವ ನೀನೆ ನಮಗೂ ಆಶ್ರಯವನ್ನು ಕೊಡು’ ಎಂದು ಬೇಡಿದರು. ಆದರೆ ಶ್ರೀಕೃಷ್ಣನು ಅವರ ಮಾತನ್ನು ಒಪ್ಪದೆ ‘ತಾಯಿಯರಿರ, ನಿಮಗೆ ಸ್ವಲ್ಪವೂ ಭಯ ಬೇಡ. ನಿಮ್ಮ ಬಂಧುಗಳು ಮೊದಲಿಗಿಂತ ಹೆಚ್ಚು ಅಕ್ಕರೆಯಿಂದ ನಿಮ್ಮನ್ನು ಕಾಣುತ್ತಾರೆ. ನೀವು ಮನೆಗೆ ಹಿಂದಿರುಗಿ ನನ್ನ ಧ್ಯಾನವನ್ನು ಮಾಡುತ್ತಿರಿ. ನಿಮಗೆ ಮುಕ್ತಿ ದೊರೆಯುತ್ತದೆ’ ಎಂದು ಸಮಾಧಾನ ಮಾಡಿ, ಅವರನ್ನು ಹಿಂದಕ್ಕೆ ಕಳುಹಿಸಿದನು. ಅವರು ಹಿಂದಿರುಗುತ್ತಲೆ ಆತನು ತನ್ನ ಗೆಳೆಯರೊಡನೆ ‘ಗಂಡಿಗಿಂತ ಹೆಣ್ಣೆ ವಾಸಿ’ ಎಂದ. ಎಲ್ಲರೂ ಹೊಟ್ಟೆ ತುಂಬ ಉಂಡು, ಹೊಟ್ಟೆ ತುಂಬ ನಕ್ಕರು.

ಶ್ರೀಕೃಷ್ಣನಿಂದ ಬೀಳ್ಕೊಂಡು ಹೊರಟ ಬ್ರಾಹ್ಮಣ ಮುತ್ತೈದೆಯರು ಮನೆಯನ್ನು ಸೇರುವಷ್ಟರಲ್ಲಿ ಅವರ ಮನೆಯ ಗಂಡಸರಿಗಲ್ಲ ಪಶ್ಚಾತ್ತಾಪ ಹುಟ್ಟಿತ್ತು. ಅವರು ‘ಅಯ್ಯೋ, ಭಗವಂತನಾದ ಶ್ರೀಕೃಷ್ಣನೇ ಹಿಡಿ ಅನ್ನಕ್ಕಾಗಿ ಹೇಳಿಕಳುಹಿಸಿರುವಾಗ ನಾವು ನೀಡದೆ ಹೋದುದು ಎಂತಹ ಅಪರಾಧ! ನಮಗಿಂತ ನಮ್ಮ ಹೆಂಗಸರು ಎಷ್ಟೋ ವಾಸಿ. ಅವರಿಗಿರುವ ಭಕ್ತಿ ನಮಗಿಲ್ಲವಾಯಿತು. ನಾವು ಶ್ರೇಷ್ಠವಾದ ಬ್ರಾಹ್ಮಣಜನ್ಮದಲ್ಲಿ ಹುಟ್ಟಿ ದವರು, ವೇದವೇದಾಂತಗಳನ್ನು ಓದಿದವರು, ಯಜ್ಞಯಾಗಾದಿಗಳನ್ನು ಆಚರಿಸುವವರು, ಇಡೀ ಸಮಾಜವನ್ನೆ ಸನ್ಮಾರ್ಗದಲ್ಲಿ ನಡೆಸುವ ಹೊಣೆ ಹೊತ್ತವರು; ನಮ್ಮ ಹಣೆಯಬರಹ ಹೆಣ್ಣಿಗಿಂತ ಕೀಳಾಯಿತು. ಭಗವಂತಾ, ಶ್ರೀಕೃಷ್ಣ, ನಾವು ನಿನ್ನ ಮಾಯೆಗೆ ಸಿಕ್ಕಿ, ಮರುಳಾಗಿ, ನಿನ್ನ ಮಹಿಮೆಯನ್ನರಿಯಲಾರದೆ ಕರ್ಮಮಾರ್ಗದಲ್ಲಿ ಬಿದ್ದು ಒದ್ದಾಡುತ್ತಿದ್ದೇವೆ. ನಮ್ಮನ್ನು ಉದ್ಧರಿಸಿ ಕಾಪಾಡು’ ಎಂದು ಬೇಡಿದರು. ಅವರು ತಕ್ಷಣವೇ ಓಡಿಹೋಗಿ, ಶ್ರೀಕೃಷ್ಣನಲ್ಲಿ ಕ್ಷಮೆಯನ್ನು ಬೇಡಿಕೊಳ್ಳಬೇಕೆಂದುಕೊಂಡರು. ಆದರೆ ರಾಜನಾದ ಕಂಸನ ಭಯ, ಅವರಿಗೆ. ಆದ್ದರಿಂದ ಹೋಗಲಿಲ್ಲ. ಪರಿಸ್ಥಿತಿ ಹೀಗಿರುವಾಗ, ಬ್ರಾಹ್ಮಣ ಮುತ್ತೈದೆ ಯರು ಮನೆಯನ್ನು ಸೇರಿದಾಗ, ಅವರಿಗೆ ಕಾದಿದ್ದುದು ಗಂಡಂದಿರ ಕೋಪವಲ್ಲ, ‘ಗಂಡಿ ಗಿಂತ ಹೆಣ್ಣೆ ವಾಸಿ’ ಎಂಬ ಪ್ರಶಸ್ತಿ.

