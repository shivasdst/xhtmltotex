
\chapter{೭೨. ಬಾಣನ ಮಗಳು ಉಷೆ}

ಬಲಿಚಕ್ರವರ್ತಿಯ ನೂರು ಮಂದಿ ಮಕ್ಕಳಲ್ಲಿ ಬಾಣ ಹಿರಿಯ. ಮಹಾ ಶಿವಭಕ್ತನಾದ ಈತ ಶೋಣಿತಪುರದಲ್ಲಿ ರಾಜನಾಗಿದ್ದ. ಸಾವಿರ ತೋಳುಗಳುಳ್ಳ ಈತನ ಶಕ್ತಿಗೆ ಅಂಜಿ, ದೇವತೆಗಳೆಲ್ಲ ಆತನ ಸೇವಕರಾಗಿದ್ದರು. ಒಮ್ಮೆ ಈ ಬಾಣಾಸುರನು ರುದ್ರದೇವನ ತಾಂಡವನೃತ್ಯದಲ್ಲಿ ತನ್ನ ಸಹಸ್ರ ತೋಳುಗಳಿಂದಲೂ ವಾದ್ಯಗಳನ್ನು ನುಡಿಸಿ, ಆತನನ್ನು ಸಂತೋಷಗೊಳಿಸಿದನು. ಆಗ ಶಿವನು ‘ಬೇಕಾದ ವರವನ್ನು ಕೇಳಿಕೊ’ ಎಂದು ಹೇಳಲು ತನ್ನ ಪಟ್ಟಣದ ರಕ್ಷಕನಾಗಿದ್ದು ತನ್ನನ್ನು ಕಾಪಾಡಬೇಕೆಂದು ಕೇಳಿಕೊಂಡನು. ರುದ್ರನು ‘ತಥಾಸ್ತು’ ಎಂದು ಹೇಳಿ ಅಂದಿನಿಂದ ಅವನ ಊರನ್ನು ಕಾಪಾಡುತ್ತಿದ್ದನು. ಅಲ್ಲಿಂದ ಮುಂದೆ ಬಾಣಾಸುರನ ಕತ್ತಿಗೆ ಇದಿರಿಲ್ಲದಂತಾಯಿತು. ಇದರಿಂದ ಅವನಿಗೆ ಕೊನೆಯಿಲ್ಲದ ಗರ್ವವೂ ಬೆಳೆಯಿತು. ಅವನು ತನ್ನ ಸಹಸ್ರ ತೋಳುಗಳ ತೀಟೆಯನ್ನು ತೀರಿಸಿಕೊಳ್ಳುವು ದಕ್ಕಾಗಿ ಬೆಟ್ಟಗಳನ್ನು ಗುದ್ದಿ ಪುಡಿಮಾಡಿದನು, ದಿಗ್ಗಜಗಳನ್ನು ಬಡಿದು ಓಡಿಸಿದನು. ತನ್ನ ಶಕ್ತಿಯನ್ನು ತೋರಿಸುವುದಕ್ಕೆ ಅವಕಾಶವೇ ಇಲ್ಲದಂತಾಯಿತು. ಅವನು ತನ್ನ ಆರಾಧ್ಯ ದೈವವಾದ ಈಶ್ವರನನ್ನು ಕುರಿತು ‘ಸ್ವಾಮಿ, ನಿನ್ನೊಬ್ಬನ ಹೊರತು, ಮೂರು ಲೋಕಗಳಲ್ಲಿ ನನ್ನ ಸಮಾನರಾದ ಬಲಶಾಲಿಗಳೇ ಇಲ್ಲದಂತಾಗಿದೆ. ನನ್ನ ಈ ಸಹಸ್ರ ತೋಳುಗಳು ಕೇವಲ ಭಾರವಾಗಿವೆಯೇ ಹೊರತು ಇವಕ್ಕೆ ಕೆಲಸವೇ ಇಲ್ಲ; ಏನು ಮಾಡಲಿ’ ಎಂದ. ಅವನ ಗರ್ವವನ್ನು ಕಂಡ ಈಶ್ವರನಿಗೆ ಕೋಪ ಬಂತು. ‘ಎಲಾ ಮೂಢ, ನನಗೆ ಸಮಾನ ನಾದ ಬಲಶಾಲಿಯು ಹುಟ್ಟಿ ನಿನ್ನ ಕೊಬ್ಬನ್ನು ಮುರಿಯುತ್ತಾನೆ. ನಿನ್ನ ರಥದ ಧ್ವಜವು ಮುರಿದುಬಿದ್ದೊಡನೆಯೇ ಅಂತಹ ಶತ್ರು ಬರುವನೆಂದು ತಿಳಿದುಕೊ’ ಎಂದನು. ಬಾಣಾ ಸುರನು ಸಂತೋಷದಿಂದ ಮನೆಗೆ ಹಿಂದಿರುಗಿ, ತನ್ನ ರಥದ ಧ್ವಜವು ಯಾವಾಗ ಮುರಿಯುವುದೋ ಎಂದು ಆತುರದಿಂದ ಕಾಯುತ್ತಿದ್ದನು.

ಬಾಣಾಸುರನ ಮಗಳಾದ ಉಷೆ ತ್ರಿಲೋಕಸುಂದರಿ. ಅವಳು ಒಂದು ದಿನ ಬೆಳಗ್ಗೆ ಏಳುವ ಮುನ್ನ ಒಂದು ಕನಸನ್ನು ಕಂಡು ‘ಹಾ ಕಾಂತ ಎಲ್ಲಿರುವೆ?’ ಎಂದು ಕನವರಿಸಿ ಕೊಳ್ಳುತ್ತಾ ಮೇಲಕ್ಕೆದ್ದಳು. ಆಕೆ ಸುತ್ತಲೂ ನೋಡುತ್ತಾಳೆ, ತನ್ನ ಸಖಿಯರು ಕುಳಿತಿದ್ದಾರೆ. ಆಕೆಗೆ ನಾಚಿಕೆಯಾಯಿತು. ತಲೆ ಬಾಗಿಸಿಕೊಂಡು ಹಾಸಿಗೆಯಮೇಲೆ ಹಾಗೆಯೇ ಕುಳಿತಳು. ಅವಳ ಸಖಿಯರಲ್ಲಿ ಅತ್ಯಂತ ಪ್ರೀತಿಪಾತ್ರಳಾದ ಚಿತ್ರಲೇಖೆ ಎಂಬುವಳು ಅವಳ ಬಳಿಗೆ ಬಂದು ‘ಉಷಾ, ಯಾವನೆ ಅವನು ನಿನ್ನ ಕಾಂತ? ಈಗತಾನೆ ಕೂಗುತ್ತಿದ್ದೆಯಲ್ಲ!’ ಎಂದು ಕೇಳಿದಳು. ಉಷೆ ತಾನು ಕಂಡ ಕನಸನ್ನು ತಿಳಿಸಿದಳು–‘ನೋಡೆ ಚಿತ್ರ, ನನಗೆ ಎಚ್ಚರ ವಾಗುವುದಕ್ಕೆ ಮುಂಚೆ ಒಂದು ಕನಸು ಬಿದ್ದಿತು. ಅದರಲ್ಲಿ ಚೆಲುವ ಚೆನ್ನಿಗನೊಬ್ಬ ನನ್ನ ಕಣ್ಣಿಗೆ ಕಾಣಿಸಿದ. ಒಳ್ಳೆ ಶ್ಯಾಮಲ ವರ್ಣ, ಕಮಲದಂತಹ ಕಣ್ಣುಗಳು, ಪೀತಾಂಬರವನ್ನು ಧರಿಸಿದ್ದ; ಅವನ ಉದ್ದವಾದ ತೋಳುಗಳು, ಎತ್ತರವಾದ ಆಕಾರ, ನಗು ಮುಖಗಳು ಎಂತಹ ಹೆಣ್ಣಿನ ಮನಸ್ಸನ್ನೂ ಸೂರೆಗೊಳ್ಳುತ್ತವೆ. ಆತನು ನನ್ನ ಹತ್ತಿರಕ್ಕೆ ಬಂದು ಒಮ್ಮೆ ನನ್ನ ತುಟಿಗಳನ್ನು ಕಚ್ಚಿ, ಅಮೃತವನ್ನು ಸುರಿಸಿ ಹೊರಟೇಹೋದ’ ಎಂದು ಹೇಳಿ, ಕಣ್ತುಂಬ ನೀರು ತುಂಬಿ ನಿಟ್ಟುಸಿರುಬಿಟ್ಟಳು. ಚಿತ್ರಲೇಖೆ ಅವಳ ಕಣ್ಣೀರನ್ನೊರಸಿ, ತಲೆ ಯನ್ನು ಸವರುತ್ತಾ ‘ಉಷಾ, ನೀನೇನೂ ಚಿಂತೆ ಮಾಡಬೇಡ. ಅವನು ಮೂರೂ ಲೋಕ ದಲ್ಲಿ ಎಲ್ಲಿದ್ದರೂ ನಿನಗೆ ತಂದುಕೊಡುತ್ತೇನೆ. ಅವನಾರೆಂಬುದು ಗೊತ್ತಾಗಬೇಕಷ್ಟೆ’ ಎಂದಳು.

ಉಷೆಯ ಕನಸಿನ ಚೆಲುವ ಯಾರೆಂದು ಹೇಗೆ ಕಂಡುಹಿಡಿಯುವುದು? ಚಿತ್ರಲೇಖೆಗೆ ಒಂದು ಉಪಾಯ ಹೊಳೆಯಿತು. ಅವಳು ಹೆಸರಿಗೆ ತಕ್ಕಂತೆ ಚಿತ್ರಗಳನ್ನು ಬರೆಯುವುದ ರಲ್ಲಿ ಬಹು ಜಾಣೆ. ಅವಳು ಲೋಕದಲ್ಲಿ ಅತ್ಯಂತ ಚೆಲುವರೆನಿಸಿಕೊಂಡವರ ಚಿತ್ರಗಳ ನ್ನೆಲ್ಲ ಬರೆಬರೆದು ಉಷೆಗೆ ತೋರಿಸುತ್ತಾ ಹೋದಳು. ವಸುದೇವ, ಬಲರಾಮ, ಶ್ರೀಕೃಷ್ಣ, ಪ್ರದ್ಯುಮ್ನ–ಅಹುದು ಅದೇ ರೂಪ, ಆದರೂ ಅವನಲ್ಲ–ಅನಿರುದ್ಧ! ಉಷೆ ಅನಿರುದ್ಧನ ಚಿತ್ರವನ್ನು ಕಂಡು ನಾಚಿಕೆಯಿಂದ ತಲೆತಗ್ಗಿಸಿ ಕೊಂಡು, ತುಟಿಯಲ್ಲಿ ಮಂದಹಾಸವನ್ನು ಚೆಲ್ಲುತ್ತಾ ‘ಚಿತ್ರಾ, ಇವನೇ ಅವನು’ ಎಂದಳು. ಚಿತ್ರಲೇಖೆ ಚಿತ್ರಕಲೆಯಲ್ಲಿ ಮಾತ್ರವೇ ಅಲ್ಲ, ಮಾಯಾವಿದ್ಯೆಯಲ್ಲಿಯೂ ನಿಪುಣಳು. ಆಕೆ ಆ ರಾತ್ರಿ ಆಕಾಶ ಮಾರ್ಗದಿಂದ ದ್ವಾರಕಾಪುರವನ್ನು ಹೊಕ್ಕು, ಅಲ್ಲಿ ಅರಮನೆಯೊಳಗೆ ಮೈಮರೆತು ಮಲಗಿದ್ದ ಅನಿರುದ್ಧ ನನ್ನು ಮಂಚಸಹಿತವಾಗಿ ಎತ್ತಿಕೊಂಡು ಬಂದು ಉಷೆಯ ಮಲಗುವ ಮನೆಯಲ್ಲಿಟ್ಟಳು. ಗಂಡುಪಿಳ್ಳೆಯ ಪ್ರವೇಶಕ್ಕೂ ಅವಕಾಶವಿಲ್ಲದ ಆ ಅಂತಃಪುರದಲ್ಲಿ ಉಷೆಯು ಅನಿರುದ್ಧ ನೊಡನೆ ತನ್ನ ಇಷ್ಟಾರ್ಥವನ್ನು ಪಡೆದು ಕೃತಕೃತ್ಯಳಾದಳು. ಆ ಹೆಣ್ಣಿನ ರೂಪ, ಅವಳ ಆದರೋಪಚಾರಗಳು, ಅವಳ ನಡೆ ನುಡಿಗಳಿಂದ ಸಂತಸಗೊಂಡ ಅನಿರುದ್ಧನಿಗೆ ದಿನ ಕಳೆದಂತೆ ಅವಳಲ್ಲಿ ಅನುರಾಗ ಹೆಚ್ಚುತ್ತಾ ಹೋಯಿತು. ತಾನು ಕನ್ಯೆಯೆಂಬುದನ್ನೆ ಮರೆತು ಆ ಹೆಣ್ಣು ಅವನೊಡನೆ ರಮಿಸುತ್ತಿದ್ದಳು. ತಾನು ಬೇರೆಯವರ ಮನೆಯಲ್ಲಿ ಇರುವೆನೆಂಬು ದನ್ನು ಮರೆತು ಅನಿರುದ್ಧನು ಆ ಹೆಣ್ಣಿನಲ್ಲಿ ಸಮರಸವಾಗಿ ಬೆರೆತನು. ಗೆಡೆವಕ್ಕಿಗಳಂತೆ ನಲಿಯುತ್ತಿದ್ದ ಅವರಿಗೆ ದಿನಗಳಮೇಲೆ ದಿನಗಳು ಉರುಳಿ ಹೋದರೂ ಅತ್ತ ಗಮನವೇ ಇಲ್ಲ.

ಜನರ ಕಣ್ಣಿಗೆ ಮಣ್ಣೆರಚಬಹುದು, ಪ್ರಕೃತಿಯ ವ್ಯಾಪಾರವನ್ನು ವಂಚಿಸುವುದಕ್ಕೆ ಸಾಧ್ಯವೆ? ಆಕೆ ಗಂಡಿನೊಡನೆ ರಮಿಸುತ್ತಿರುವಳೆಂಬುದನ್ನು ಮರೆಮಾಡುವುದಕ್ಕೆ ಸಾಧ್ಯ ವಿಲ್ಲದಂತಹ ಗುರುತುಗಳು ಆಕೆಯಲ್ಲಿ ಕಾಣಿಸಿದವು. ಇದನ್ನು ಕಂಡ ಬಾಗಿಲು ಕಾಯು ವವರಿಗೆ ಜೀವವೇ ಹಾರಿಹೋದಂತಾಯಿತು. ತಮ್ಮ ತಲೆಯನ್ನು ಉಳಿಸಿಕೊಳ್ಳುವುದಕ್ಕಾಗಿ ಅವರು ಬಾಣಾಸುರನ ಬಳಿಗೆ ಹೋಗಿ ‘ಮಹಾಪ್ರಭು, ರಾಜಕುಮಾರಿಯಾದ ಉಷಾ ದೇವಿಯು ಕನ್ಯಾವ್ರತವನ್ನು ಮೀರಿ ನಡೆಯುವಂತೆ ಕಾಣುತ್ತಿದೆ. ನಾವು ಹಗಲಿರುಳೂ ಕಣ್ಣಲ್ಲಿ ಎಣ್ಣೆ ಹಾಕಿಕೊಂಡು ಕಾಯುತ್ತಿದ್ದೇವೆ. ಯಾವ ಗಂಡೂ ಒಳಗೆ ಹೋಗಲು ಸಾಧ್ಯವೇ ಇಲ್ಲ. ಆದರೂ ಆಕೆ ಯಾವನೋ ಒಬ್ಬ ಪುರುಷನೊಡನೆ ರಮಿಸುವಂತೆ ತೋರು ತ್ತಿದೆ’ ಎಂದರು. ಈ ಮಾತನ್ನು ಕೇಳುತ್ತಲೆ ಆತನಿಗೆ ಬರಸಿಡಿಲು ಬಡಿದಂತಾಯಿತು. ಆತನು ಹೇಗಿದ್ದವನು ಹಾಗೆಯೆ ಉಷೆಯ ಅಂತಃಪುರಕ್ಕೆ ಓಡಿಬಂದನು. ಒಳಹೊಕ್ಕು ನೋಡುತ್ತಾನೆ–ದಿವ್ಯಸುಂದರ ಮೂರ್ತಿಯೊಬ್ಬ ತನ್ನ ಮಗಳೊಡನೆ ಪಗಡೆಯಾಡುತ್ತಾ ಕುಳಿತಿದ್ದಾನೆ. ಬಾಣನಿಗೆ ಅಚ್ಚರಿಯಾಯಿತು–‘ಇವನಾರು? ಎಲ್ಲಿಂದ ಬಂದ? ಹೇಗೆ ಬಂದ?’ ಕ್ಷಣಕಾಲ ಆತ ಏನೂ ತೋಚದೆ ಮೂಢನಂತೆ ನಿಂತ! ತನ್ನ ಮಾನ ಮರ್ಯಾದೆ ಮಣ್ಣುಮುಕ್ಕಿಹೋದುದನ್ನು ಕಂಡು ಆತನಿಗೆ ತಡೆಯಲಾರದಷ್ಟು ಕೋಪ ಬಂತು. ‘ಕತ್ತ ರಿಸಿಹಾಕಿರೊ ಇವನನ್ನು’ ಎಂದು ಬಾಗಿಲು ಕಾಯುವವರಿಗೆ ಅಪ್ಪಣೆ ಮಾಡಿದನು. ಆದರೆ ಅನಿರುದ್ಧನೇನು ಸಾಮಾನ್ಯನೆ? ಅಲ್ಲಿಯೆ ಇದ್ದ ಗದೆಯನ್ನು ತೆಗೆದುಕೊಂಡು, ತನ್ನ ಮೇಲೆ ಬಿದ್ದವರನ್ನಲ್ಲ ಬಡಿದುಹಾಕಿದನು. ಇದನ್ನು ಕಂಡು ಬಾಣಾಸುರನು ಕೋಪ ಗೊಂಡು ನಾಗಪಾಶದಿಂದ ಅವನನ್ನು ಕಟ್ಟಿ ಸೆರೆಮನೆಗೆ ನೂಕಿದನು. ಉಷೆ ಕಣ್ಣೀರನ್ನು ಸುರಿಸುತ್ತಾ ಹಾಸಿಗೆ ಹಿಡಿದಳು.

ಅತ್ತ ದ್ವಾರಕಿಯಲ್ಲಿ ಅನಿರುದ್ಧನನ್ನು ಕಾಣದೆ ಅರಮನೆಯವರೆಲ್ಲ ಕಳವಳಗೊಂಡರು. ಅವನು ಮನೆಬಿಟ್ಟು ಹೋಗಿ ಆಗಲೆ ನಾಲ್ಕು ತಿಂಗಳುಗಳಾದವು. ಎಲ್ಲಿ ಹುಡುಕಿದರೂ ಅವನ ಸುಳಿವೇ ಇಲ್ಲ. ಮುಂದೇನು ಮಾಡಬೇಕೆಂದು ತೋಚದೆ ಅವರು ಮರುಗುತ್ತಿರು ವಾಗ ತ್ರಿಲೋಕಸಂಚಾರಿಗಳಾದ ನಾರದರು ದ್ವಾರಕೆಗೆ ಆಗಮಿಸಿದರು. ಅವರಿಗೆ ತಿಳಿಯದ ವಿಷಯವೇನಿದೆ? ಅವರು ಅನಿರುದ್ಧ-ಉಷೆಯರ ಪ್ರಣಯ ವೃತ್ತಾಂತವನ್ನೂ, ಈಗ ಅನಿ ರುದ್ಧ ಬಾಣಾಸುರನ ಸೆರೆಯಲ್ಲಿರುವುದನ್ನೂ ಆದ್ಯಂತವಾಗಿ ಬಣ್ಣಿಸಿ ಹೇಳಿದರು. ಒಡನೆಯೆ ಶ್ರೀಕೃಷ್ಣನು ಹನ್ನೆರಡು ಅಕ್ಷೋಹಿಣಿ ಯಾದವ ಸೇನೆಯೊಡನೆ ಅಣ್ಣನಾದ ಬಲ ರಾಮನನ್ನು ಕರೆದುಕೊಂಡು, ಶೋಣಿತಪುರಕ್ಕೆ ಹೋಗಿ, ಅದಕ್ಕೆ ಮುತ್ತಿಗೆ ಹಾಕಿದನು. ಆತನ ಸೇನೆ ಶತ್ರುವಿನ ಕೋಟೆ ಕೊತ್ತಲಗಳನ್ನು ಕೆಡವಿ ಊರಿನ ಉದ್ಯಾನವನವನ್ನು ಹಾಳು ಮಾಡಿತು. ಆ ವೇಳೆಗೆ ಸರಿಯಾಗಿ ಬಾಣನ ರಥದ ಬಾವುಟ ಮುರಿದುಬಿತ್ತು. ಇದನ್ನು ಕಂಡು ಬಾಣಾಸುರನು ತನ್ನ ಇದಿರಾಳಿ ಸಿಕ್ಕನೆಂಬ ಸಂತೋಷದಿಂದ ದೊಡ್ಡ ಸೇನೆಯೊಡನೆ ಯುದ್ಧಕ್ಕೆ ಹೊರಟನು. ಶೋಣಿತಪುರದ ರಕ್ಷಣೆಯ ಹೊಣೆ ಹೊತ್ತ ರುದ್ರನು ಸುಮ್ಮನಿರು ವುದಕ್ಕಾಗುತ್ತದೆಯೆ? ಆತನು ತನ್ನ ಪ್ರಮಥರ ಸೇನೆಯೊಡನೆ ಬಾಣಾಸುರನ ಬೆಂಗಾವಲಾಗಿ ಹೊರಟನು. ಎರಡೂ ಸೇನೆಗೂ ಕೈಗೆ ಕೈ ಹತ್ತಿತು. ಸಾಕ್ಷಾತ್ ರುದ್ರದೇವನೇ ಶ್ರೀಕೃಷ್ಣನಿಗೆ ಎದುರಾದನು. ಅವನ ಮಗ ಷಣ್ಮುಖನು ಶ್ರೀಕೃಷ್ಣನ ಮಗನಾದ ಪ್ರದ್ಯುಮ್ನನನ್ನು ಇದಿರಿ ಸಿದನು. ಪ್ರಮಥರಲ್ಲಿ ಪ್ರಮುಖರಾದವರು ಯಾದವರಲ್ಲಿ ಮುಖ್ಯರಾದವರನ್ನು ಎದುರಿಸಿ ದರು. ಬ್ರಹ್ಮಾದಿ ದೇವತೆಗಳೆಲ್ಲರೂ ಈ ಯುದ್ಧವನ್ನು ನೋಡಬೇಕೆಂದು ಆಕಾಶದಲ್ಲಿ ನೆರೆದರು. ಬ್ರಹ್ಮಾಸ್ತ್ರ ನಾರಾಯಣಾಸ್ತ್ರ, ಪಾಶುಪತಾಸ್ತ್ರ, ಆಗ್ನೇಯಾಸ್ತ್ರ, ವಾರುಣಾಸ್ತ್ರ– ಅಸ್ತ್ರಶಸ್ತ್ರಗಳ ಮಾರ್ಮೊಳಗಿನ ವಾತಾವರಣವೆಲ್ಲ ಕ್ಷೋಭೆಗೊಂಡಿತು. ಕೊನೆಗೆ ಶ್ರೀ ಕೃಷ್ಣನು ಜೃಂಭಣಾಸ್ತ್ರವನ್ನು ಬಿಟ್ಟು ರುದ್ರನು ತಟಸ್ಥನಾಗುವಂತೆ ಮಾಡಿದನು. ಷಣ್ಮುಖನು ಪ್ರದ್ಯುಮ್ನನಿಂದ ಪೆಟ್ಟುತಿಂದು ಪಲಾಯನ ಮಾಡಿದನು. ಬಲರಾಮನು ಬಾಣನ ಮಂತ್ರಿಗಳಾದ ಕುಂಭಾಂಡ, ಕೂಪಕರ್ಣರನ್ನು ಕೊಂದುಹಾಕಿದನು. ಇದನ್ನು ಕಂಡು ಬಾಣನ ಸೇನೆ ಚೆಲ್ಲಾಪಿಲ್ಲಿಯಾಗಿ ಚದರಿ ಓಡಿತು.

ರುದ್ರನು ಸೋತು ತನ್ನ ಸೈನ್ಯವೆಲ್ಲ ಪಲಾಯನ ಮಾಡುವುದನ್ನು ಕಂಡ ಬಾಣಾಸುರನು ತನ್ನ ಎದುರಾಳಿಯಾದ ಸಾತ್ಯಕಿಯನ್ನು ಬಿಟ್ಟು, ಶ್ರೀಕೃಷ್ಣನ ಕಡೆ ತನ್ನ ರಥವನ್ನು ಹರಿಸಿದನು. ಆತನ ಸಾವಿರ ಕೈಗಳಲ್ಲಿ ಐನೂರು ಬಿಲ್ಲನ್ನು ಹಿಡಿದು, ಉಳಿದ ಐನೂರು ಕೈಗಳು ಬಾಣಗಳ ಮಳೆಯನ್ನು ಸುರಿಸಿದವು. ಆದರೆ ಶ್ರೀಕೃಷ್ಣನು ಬಿಟ್ಟ ಬಾಣಗಳು ಒಮ್ಮೆಯೆ ಅವಷ್ಟನ್ನೂ ಕತ್ತರಿಸಿದುದಲ್ಲದೆ ಅವನ ಸಾರಥಿ, ಕುದುರೆ, ರಥ, ಬಾವುಟಗಳನ್ನೂ, ಅವನ ಕೈಲಿ ಹಿಡಿದಿದ್ದ ಬಿಲ್ಲುಗಳನ್ನೂ ಕತ್ತರಿಸಿಹಾಕಿದವು. ವಿಜಯಿಯಾದ ಶ್ರೀಕೃಷ್ಣನು ಧರೆ ಬಿರಿಯುವಂತೆ ಶಂಖವನ್ನು ಊದಿದನು. ಆ ವೇಳೆಗೆ ಸರಿಯಾಗಿ ಬಾಣನ ತಾಯಿಯಾದ ಕೋಟರೆಯೆಂಬುವಳು ತನ್ನ ಮಗನನ್ನು ಉಳಿಸುವುದಕ್ಕಾಗಿ ಬೆತ್ತಲೆಯಾಗಿ ಓಡಿಬಂದು ಶ್ರೀಕೃಷ್ಣನ ಮುಂದೆ ನಿಂತಳು. ಆ ಸ್ಥಿತಿಯಲ್ಲಿ ಅವಳನ್ನು ನೋಡಲಾರದೆ ಶ್ರೀಕೃಷ್ಣನು ಕಣ್ಣು ಮುಚ್ಚಿಕೊಳ್ಳಲು, ನಿರಾಯುಧನಾಗಿದ್ದ ಬಾಣನು ತಲೆ ತಪ್ಪಿಸಿಕೊಂಡು ಊರೊಳಕ್ಕೆ ಓಡಿ ಹೋದನು. ಇದನ್ನು ಕಂಡು ರುದ್ರನು ಮಾಹೇಶ್ವರ ಜ್ವರವನ್ನು ಸೃಷ್ಟಿ ಮಾಡಿ ಶ್ರೀಕೃಷ್ಣನ ಮೇಲೆ ಪ್ರಯೋಗಿಸಿದನು. ಆದರೆ ವೈಷ್ಣವ ಜ್ವರವು ಅದನ್ನು ಇದಿರಿಸಿ ಅಡಗಿಸಿತು. ಆ ವೇಳೆಗೆ ಬಾಣಾಸುರನು ಹೊಸ ರಥವನ್ನೇರಿ, ನಾನಾ ಬಗೆಯ ಆಯುಧ ಗಳನ್ನು ಹಿಡಿದು ಮತ್ತೆ ಯುದ್ಧಕ್ಕೆ ಬಂದನು. ಶ್ರೀಕೃಷ್ಣನು ಅವನು ಬಿಟ್ಟ ಬಾಣಗಳನ್ನು ಕತ್ತರಿಸಿದುದಲ್ಲದೆ, ತನ್ನ ಚಕ್ರದಿಂದ ಅವನ ತೋಳುಗಳನ್ನು ಮರದೆ ರೆಂಬೆಗಳನ್ನು ಕತ್ತರಿಸುವಂತೆ ಕತ್ತರಿಸುತ್ತಾ ಹೋದನು.

ತನ್ನ ಭಕ್ತನ ದುರವಸ್ಥೆಯನ್ನು ಕಂಡು, ಭಕ್ತವತ್ಸಲನಾದ ರುದ್ರನಿಗೆ ಕರುಣೆ ಹುಟ್ಟಿತು. ಆತನು ಶ್ರೀಕೃಷ್ಣನನ್ನು ಕುರಿತು “ಹೇ ಶ್ರೀಕೃಷ್ಣ, ವೇದಗಳು ಹೊಗಳುವ ‘ಸ್ವಯಂ ಪ್ರಕಾಶ ವುಳ್ಳ ಪರಬ್ರಹ್ಮ’ ಎಂಬುವನೆ ನೀನು! ಆಕಾಶದಂತೆ ನಿರ್ಲೇಪನಾದ ನಿನ್ನ ಸ್ವರೂಪವನ್ನು ಮಹಾಯೋಗಿಗಳು ಮಾತ್ರ ಅರಿಯಬಲ್ಲರು. ಪ್ರಭು, ಭೂಭಾರಹರಣಕ್ಕಾಗಿ ನೀನು ಯಾದವ ವಂಶದಲ್ಲಿ ಹುಟ್ಟಿರುವೆ. ಮಾನವನಾಗಿ ಹುಟ್ಟಿರುವಾಗಲೂ ನಿನ್ನ ಸರ್ವಜ್ಞತ್ವಾದಿ ಗುಣಗಳಿಗೆ ಚ್ಯುತಿಯಿಲ್ಲ. ಈ ಪ್ರಪಂಚವನ್ನು ಸೃಷ್ಟಿಸಿದವನೂ ನೀನೆ. ನಿನ್ನ ಮಾಯೆಗೆ ಸಿಕ್ಕಿ ಮರುಳಾದವರು ಮನೆ, ಮಡದಿ, ಮಕ್ಕಳು ಎಂದು ದುಃಖಸಮುದ್ರದಲ್ಲಿ ಮುಳುಗಿ ತೇಲುತ್ತಾರೆ. ದಯಾಮಯನಾದ ನೀನು ಮಾಯೆಯನ್ನು ದಾಟಲೆಂದು ಮನುಷ್ಯದೇಹ ವನ್ನು ಕೊಡುವೆ. ಆಗ ನಿನ್ನ ಆರಾಧನೆ ಮಾಡಿ ಉದ್ಧಾರವಾಗುವುದಕ್ಕೆ ಬದಲಾಗಿ ಇಂದ್ರಿಯ ಸುಖಕ್ಕೆ ಜೋತುಬಿದ್ದರೆ ಅವನು ಆತ್ಮವಂಚಕನಿದ್ದಂತೆ. ವಿಷವನ್ನು ಅಮೃತವೆಂದು ಭಾವಿಸುವ ಆ ಮೂಢನಿಗೇನು ಮಾಡುವುದು? ಸ್ವಾಮಿ, ನೀನು ಸೃಷ್ಟಿ ಸ್ಥಿತಿ ಲಯಗಳಿಗೆ ಕಾರಣನು. ನಾನಾಗಲಿ, ಬ್ರಹ್ಮಇಂದ್ರಾದಿ ದೇವತೆಗಳಾಗಲಿ, ಪುಷಿಮುನಿಗಳಾಗಲಿ–ನಾವೆ ಲ್ಲರೂ ನಿನ್ನನ್ನು ಭೋಗ ಮೋಕ್ಷಗಳೆರಡಕ್ಕೂ ದಿಕ್ಕೆಂದು ಪೂಜಿಸುತ್ತೇವೆ. ಸ್ವಾಮಿ, ಈ ಬಾಣಾಸುರನು ಮೊದಲಿನಿಂದಲೂ ನನ್ನ ಭಕ್ತ. ಇವನನ್ನು ಕಾಪಾಡುವುದಾಗಿ ನಾನಿವನಿಗೆ ವರವಿತ್ತಿದ್ದೇನೆ. ನನ್ನ ಮಾತು ಉಳಿಯುವಂತೆ ನೀನಿವನಿಗೆ ಪ್ರಾಣದಾನ ಮಾಡಬೇಕು” ಎಂದು ಬೇಡಿಕೊಂಡನು. ಆಗ ಶ್ರೀಕೃಷ್ಣನು “ಹೇ ಶಂಕರ, ನಿನ್ನ ಇಷ್ಟವೇ ನನ್ನ ಇಷ್ಟ. ನನಗೂ ಇವನನ್ನು ಕೊಲ್ಲಬೇಕೆಂಬ ಉದ್ದೇಶವಿಲ್ಲ. ಇವನು ನನ್ನ ಪರಮ ಭಕ್ತನಾದ ಪ್ರಹ್ಲಾದನ ಮಗ. ಹಿಂದೆ ನಾನು ಪ್ರಹ್ಲಾದನಿಗೆ ‘ನಿನ್ನ ವಂಶದವರನ್ನು ಕೊಲ್ಲುವುದಿಲ್ಲ’ ಎಂದು ಮಾತುಕೊಟ್ಟಿದ್ದೇನೆ. ಆದ್ದರಿಂದ ಇವನ ಅಹಂಕಾರ ಅಡಗುವಂತೆ ತೋಳು ಗಳನ್ನು ಮಾತ್ರ ಕತ್ತರಿಸಿಹಾಕಿದ್ದೇನೆ. ಇವನ ಸೈನ್ಯವೆಲ್ಲ ಸತ್ತುದರಿಂದ ಭೂಭಾರ ತಗ್ಗಿ ದಂತಾಯಿತು. ಈಗ ಇವನಿಗೆ ನಾಲ್ಕು ತೋಳುಗಳನ್ನು ಮಾತ್ರ ಉಳಿಸಿದ್ದೇನೆ. ಮುಂದೆ ಅವನು ಜರಾಮರಣಗಳಿಲ್ಲದ ನಿನ್ನ ಪ್ರಮಥರಲ್ಲಿ ಒಬ್ಬನಾಗುವನು” ಎಂದು ಅಭಯ ವನ್ನು ಕೊಟ್ಟನು.

ಶ್ರೀಕೃಷ್ಣನಿಂದ ಅಭಯವನ್ನು ಪಡೆದ ಬಾಣಾಸುರನು ಸೆರೆಯಲ್ಲಿಟ್ಟಿದ್ದ ಅನಿರುದ್ಧ ನನ್ನೂ ತನ್ನ ಮಗಳಾದ ಉಷೆಯನ್ನೂ ರಥದಲ್ಲಿ ಕುಳ್ಳಿರಿಸಿ ಕರೆತಂದು ಶ್ರೀಕೃಷ್ಣನಿಗೆ ಒಪ್ಪಿಸಿದನು. ಅಲ್ಲಿಯೆ, ಆ ಕ್ಷಣದಲ್ಲಿಯೆ ಅವರಿಬ್ಬರಿಗೂ ವಿವಾಹ ನಡೆಯಿತು. ನೂತನ ವಧೂವರರನ್ನು ಮುಂದಿಟ್ಟುಕೊಂಡು, ತನ್ನ ಸೈನ್ಯದೊಡನೆ ಶ್ರೀಕೃಷ್ಣನು ಸಂಭ್ರಮದಿಂದ ದ್ವಾರಕಿಗೆ ಹಿಂದಿರುಗಿದನು.

