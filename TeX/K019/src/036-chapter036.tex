
\chapter{೩೬. ಜಮದಗ್ನಿಯ ಮಗ ಪರಶುರಾಮ}

ಪುರೂರವನಿಗೆ ಊರ್ವಶಿಯಿಂದ ಆರುಮಂದಿ ಮಕ್ಕಳು ಹುಟ್ಟಿದರು. ಅವರಲ್ಲಿ ವಿಜಯನೆಂಬುವನ ಪೀಳಿಗೆ ಲೋಕೋತ್ತರವಾದ ಕೀರ್ತಿಗೆ ಕಾರಣವಾಯಿತು. ಈ ಪೀಳಿಗೆ ಯಲ್ಲಿ ಹುಟ್ಟಿದ ಜಹ್ನುವೆಂಬ ಮಹಾನುಭಾವನು ಮಹಾತಪಸ್ವಿಯಾಗಿ, ಭಗೀರಥನು ತಂದ ದೇವಗಂಗೆಯನ್ನು ಆಪೋಶನವಾಗಿ ತೆಗೆದುಕೊಂಡನು. ವಿಶ್ವಾಮಿತ್ರ ಮಹರ್ಷಿಯ ತಂದೆಯಾದ ಗಾಧಿರಾಜನು ಹುಟ್ಟಿದುದೂ ಈ ವಂಶದಲ್ಲಿಯೇ. ಈ ಗಾಧಿರಾಜನಿಗೆ ಸತ್ಯವತಿಯೆಂಬ ಸುಂದರಿಯಾದ ಮಗಳಿದ್ದಳು. ಈ ಮಗಳನ್ನು ತನಗೆ ಮದುವೆ ಮಾಡಿ ಕೊಡಬೇಕೆಂದು ಪುಚೀಕನೆಂಬ ಬ್ರಾಹ್ಮಣನೊಬ್ಬ ಬೇಡಿದ. ಆ ಬಡಬ್ರಾಹ್ಮಣನಿಗೆ ಮಗಳನ್ನು ಕೊಡಲು ಮನಸ್ಸಿಲ್ಲದೆ ಗಾಧಿಯು ‘ಅಯ್ಯಾ, ನನ್ನ ಮಗಳು ಅಷ್ಟು ಸುಲಭ ವಾಗಿ ದಕ್ಕುವವಳಲ್ಲ. ದೇಹವೆಲ್ಲ ಚಂದ್ರನಂತೆ ಬೆಳ್ಳಗಿರಬೇಕು, ಕಿವಿ ಮಾತ್ರ ಕಪ್ಪಾಗಿರ ಬೇಕು; ಅಂತಹ ಒಂದು ಸಾವಿರ ಕುದುರೆಗಳನ್ನು ತಂದುಕೊಟ್ಟವನು ನನ್ನ ಮಗಳನ್ನು ಕೈ ಹಿಡಿಯಬೇಕು’ ಎಂದ. ಪುಚೀಕನು ಧನದಲ್ಲಿ ಬಡವನಾದರೂ ವಿದ್ಯೆಯಲ್ಲಿ ಶ್ರೀಮಂತ. ಆತನು ವರುಣನನ್ನು ಪ್ರಾರ್ಥಿಸಿ, ಗಾಧಿರಾಜ ಹೇಳಿದಂತಹ ಸಾವಿರ ಕುದುರೆಗಳನ್ನು ಪಡೆದು ಆತನಿಗಿತ್ತನು. ಇದರಿಂದ ಸತ್ಯವತಿ ಪುಚೀಕನ ಮಡದಿಯಾದಳು. ಆಕೆಯು ಗಂಡನ ಮನೆಗೆ ಹೊರಟಾಗ ಆಕೆಯ ತಾಯಿಯೂ ಜೊತೆಯಲ್ಲಿ ಹೊರಟಳು. ಅವರಿ ಬ್ಬರೂ ಪುಚೀಕನ ಮುಂದೆ ನಿಂತು, ತಮ್ಮಿಬ್ಬರಿಗೂ ಸತ್ಪುತ್ರರಾಗುವಂತೆ ಅನುಗ್ರಹಿಸ ಬೇಕೆಂದು ಬೇಡಿಕೊಂಡರು. ಪುಚೀಕನು ಹಾಗೆಯೇ ಆಗಲೆಂದು ಹೇಳಿ, ಕ್ಷತ್ರಿಯನು ಹುಟ್ಟತಕ್ಕ ಮಂತ್ರದಿಂದ ಅತ್ತೆಗೂ, ಬ್ರಾಹ್ಮಣನು ಹುಟ್ಟತಕ್ಕ ಮಂತ್ರದಿಂದ ಹೆಂಡತಿಗೂ ಹವಿಸ್ಸನ್ನು ಬೇರೆ ಬೇರೆಯಾಗಿ ತಯಾರುಮಾಡಿಸಿಟ್ಟು, ತಾನು ಸ್ನಾನಕ್ಕೆಂದು ನದಿಗೆ ಹೋದನು. ಸತ್ಯವತಿಗಾಗಿ ಮಾಡಿಸಿರುವ ಹವಿಸ್ಸು ತನ್ನದಕ್ಕಿಂತಲೂ ಶ್ರೇಷ್ಠವಾದು ದಿರಬೇಕೆಂದು ಅನಿಸಿತು, ಅವಳ ತಾಯಿಗೆ. ಅವಳು ಮಗಳೊಡನೆ ‘ನಿನ್ನದನ್ನು ನನಗೆ ಕೊಡು, ನನ್ನದನ್ನು ನೀನು ತೆಗೆದುಕೊ’ ಎಂದಳು. ಮಗಳು ಹಾಗೆಯೆ ಆಗಲೆಂದಳು. ಇಬ್ಬರೂ ಪರಸ್ಪರ ಬದಲಾಯಿಸಿಕೊಂಡು, ಅವುಗಳನ್ನು ತಿಂದುಹಾಕಿದರು. ಗಂಡ ಮನೆಗೆ ಹಿಂತಿರುಗುತ್ತಲೆ ಸತ್ಯವತಿ ಅದನ್ನು ಆತನಿಗೆ ತಿಳಿಸಿದಳು. ಪಾಪ, ಪುಚೀಕ ಏನು ಮಾಡಬೇಕು? ‘ನಿಮ್ಮ ಹಣೆಯ ಬರಹ. ನಿನ್ನ ತಾಯಿಯ ಹೊಟ್ಟೆಯಲ್ಲಿ ಬ್ರಹ್ಮಜ್ಞಾನಿ ಹುಟ್ಟುತ್ತಾನೆ, ನಿನ್ನ ಹೊಟ್ಟೆಯಲ್ಲಿ ಕ್ರೂರನಾದ ಕ್ಷತ್ರಿಯನು ಹುಟ್ಟುತ್ತಾನೆ’ ಎಂದನು. ಸತ್ಯವತಿ ಗಾಬರಿಯಾದಳು. ಆಕೆ ಗಂಡನ ಕಾಲನ್ನು ಹಿಡಿದುಕೊಂಡು ‘ಸ್ವಾಮಿ, ನನ್ನ ತಪ್ಪನ್ನು ಕ್ಷಮಿಸಿ. ನನ್ನ ಹೊಟ್ಟೆಯಲ್ಲಿ ಹುಟ್ಟುವ ಮಗ ಕ್ರೂರಿಯಾಗುವುದು ಬೇಡ’ ಎಂದು ಬೇಡಿದಳು. ಪುಚೀಕನು ಶಾಂತನಾಗಿ ‘ಹಾಗೆಯೇ ಆಗಲಿ. ನಿನ್ನ ಮಗ ಶಾಂತನಾ ದರೂ ನಿನ್ನ ಮಗನ ಮಗ ಭಯಂಕರನಾಗುತ್ತಾನೆ’ ಎಂದನು. ಅದರಂತೆ ಆಕೆಯ ಗರ್ಭ ದಲ್ಲಿ ಮಹಾತಪಸ್ವಿಯಾದ ಜಮದಗ್ನಿ ಪುಷಿ ಹುಟ್ಟಿದನು. ಆತನನ್ನು ಕಂಡ ತಾಯಿ ಸಂತೋಷದಿಂದ ನದಿಯಾಗಿ ಹರಿದು ಹೋದಳು. ಅದೇ ಕೌಶಕೀ ನದಿ.

ಜಮದಗ್ನಿಯು ಪ್ರಾಪ್ತವಯಸ್ಕನಾಗುತ್ತಲೆ ರೇಣುಕೆ ಎಂಬ ಸುಂದರ ಕನ್ಯೆಯನ್ನು ಮದುವೆ ಮಾಡಿಕೊಂಡು ಆಕೆಯಲ್ಲಿ ಹಲವು ಮಕ್ಕಳನ್ನು ಪಡೆದನು. ಅವರಲ್ಲಿ ಕಡೆ ಯವನೇ ಪ್ರಸಿದ್ಧನಾದ ಪರಶುರಾಮ. ಈತನನ್ನು ಭಗವಂತನ ಅವತಾರವೆಂದೇ ಕರೆಯು ತ್ತಾರೆ. ಈತನು ಲೋಕದಲ್ಲಿ ಕ್ಷತ್ರಿಯ ಜಾತಿಯೇ ಇಲ್ಲದಂತೆ ಮಾಡಬೇಕೆಂದು ಬಗೆದು, ಇಪ್ಪತ್ತೊಂದು ಸಲ ಭೂಮಿಯನ್ನು ಸುತ್ತಿ ಸುತ್ತಿ, ಕಂಡ ಕಂಡ ಕ್ಷತ್ರಿಯರನ್ನೆಲ್ಲ ಕೊಂದು ಹಾಕಿದನು. ಇದಕ್ಕೆ ಮೂಲಕಾರಣ ಹೈಹಯ ದೇಶದ ರಾಜನಾದ ಕಾರ್ತವೀರ್ಯಾರ್ಜುನ. ಈ ರಾಜನು ದತ್ತಾತ್ರೇಯನ ಆರಾಧನೆಯಿಂದ ಸಹಸ್ರ ತೋಳುಗಳನ್ನು ಪಡೆದು, ಎಣೆ ಯಿಲ್ಲದ ಪರಾಕ್ರಮಿಯೆನಿಸಿದ್ದನು. ತನ್ನ ಆರಾಧ್ಯದೈವದ ವರಪ್ರಸಾದದಿಂದ ಆತನಿಗೆ ಎಲ್ಲ ಲೋಕಗಳಲ್ಲಿಯೂ ಸಂಚರಿಸಬಲ್ಲ ಶಕ್ತಿ ಇತ್ತು. ಅಹಂಕಾರದಿಂದ ಬೀಗಿ ಬಿರಿಯು ತ್ತಿದ್ದ ಈ ಮನುಷ್ಯ ಒಮ್ಮೆ ತನ್ನ ತೋಳುಗಳಿಂದ ರೇವಾನದಿಯ ನೀರಿಗೆ ಅಡ್ಡಗಟ್ಟೆಹಾಕಿ, ಅದರಲ್ಲಿ ಮಡದಿಯರೊಡನೆ ನೀರಾಟವಾಡುತ್ತಿದ್ದನು. ಅದೇ ಸಮಯದಲ್ಲಿ ಲಂಕೆಯ ರಾಜನಾದ ರಾವಣನು ಕಾರ್ತವೀರ್ಯಾರ್ಜುನನ ಮೇಲೆ ದಂಡೆತ್ತಿ ಹೊರಟು, ರೇವಾನದಿಯ ತೀರದಲ್ಲಿಯೇ ಬೀಡುಬಿಟ್ಟಿದ್ದನು. ಕಾರ್ತವೀರ್ಯನು ನೀರನ್ನು ಅಡ್ಡಗಟ್ಟಿದುದರಿಂದ ನೀರು ಉಕ್ಕಿ, ಅದರಲ್ಲಿ ರಾವಣನ ಪಾಳೆಯವೆಲ್ಲ ಕೊಚ್ಚಿಹೋಯಿತು. ಇದನ್ನು ಕಂಡು ಆ ರಾಕ್ಷಸ ರಾಜನಿಗೆ ರೇಗಿಹೋಯಿತು. ಆತನು ಕಾರ್ತವೀರ್ಯನನ್ನು ಶಿಕ್ಷಿಸಬೇಕೆಂದು ಅವನ ಬಳಿಗೆ ಓಡಿ ಬಂದನು. ಆದರೆ ಕಾರ್ತವೀರ್ಯನು ತನ್ನ ಜಲಕ್ರೀಡೆಗೆ ಅಡ್ಡಿಯಾದ ಆ ರಕ್ಕಸನನ್ನು ಹೆಣ್ಣುಗಳೆಲ್ಲ ಕೈತಟ್ಟಿ ನಗುತ್ತಿರುವಾಗ, ಹಗ್ಗದಿಂದ ಕಟ್ಟಿ ಎಳೆದುಕೊಂಡು ಹೋಗಿ, ತನ್ನ ಊರಿನಲ್ಲಿ ಕೆಲಕಾಲ ಸೆರೆಯಲ್ಲಿಟ್ಟು ಆಮೇಲೆ ಬಿಟ್ಟುಬಿಟ್ಟನು. ಇಂತಹ ಪಟುಪರಾಕ್ರಮಿ, ಕಾರ್ತವೀರ್ಯ. ಆತ ಒಮ್ಮೆ ಜಮದಗ್ನಿಯ ಆಶ್ರಮಕ್ಕೆ ಬಂದನು. ಪುಷಿಯು ಕಾಮಧೇನು ವಿನ ಸಹಾಯದಿಂದ ಆತನಿಗೂ ಆತನ ಪರಿವಾರಕ್ಕೂ ದಿವ್ಯವಾದ ಔತಣವನ್ನು ಮಾಡಿ ದನು. ಆ ರಾಜನು ಇದನ್ನು ಕಂಡು ಸಂತೋಷಪಡುವುದಕ್ಕೆ ಬದಲಾಗಿ ‘ಎಲಾ, ಈ ಆಕಳಿನ ದೆಸೆಯಿಂದ ಈ ಬಡ ಬ್ರಾಹ್ಮಣನಿಗೆ ಇಷ್ಟೊಂದು ವೈಭವವೆ’ ಎಂದು ಹೊಟ್ಟೆಕಿಚ್ಚುಪಟ್ಟು, ‘ಆ ಆಕಳನ್ನು ರಾಜಧಾನಿಗೆ ಹಿಡಿದುಕೊಂಡು ಹೋಗಿ’ ಎಂದು ತನ್ನ ಸೇನೆಗೆ ಅಪ್ಪಣೆ ಮಾಡಿದನು. ಅವರು ಅದನ್ನು ಎಳೆದುಕೊಂಡು ಹೋದರು.

ಹೊರಗೆಲ್ಲೊ ಕಾರ್ಯಾರ್ಥವಾಗಿ ಹೋಗಿದ್ದ ಪರಶುರಾಮ ಆಶ್ರಮಕ್ಕೆ ಹಿಂದಿರುಗುತ್ತಲೆ ನಡೆದ ಸಂಗತಿಯೆಲ್ಲ ಆತನಿಗೆ ಗೊತ್ತಾಯಿತು. ಆತನು ರೋಷಭೀಷಣನಾಗಿ, ಹೊಡೆತ ತಿಂದ ಕರಿಯ ನಾಗರದಂತೆ ಬುಸುಗುಟ್ಟುತ್ತಾ ತನ್ನ ಗಂಡುಗೊಡಲಿಯನ್ನು ಕೈಗೆ ತೆಗೆದು ಕೊಂಡು, ಆನೆಯನ್ನು ಬೆನ್ನಟ್ಟುವ ಸಿಂಹದಂತೆ ಕಾರ್ತವೀರ್ಯನ ಬೆನ್ನಟ್ಟಿದನು. ತಲೆಯಲ್ಲಿ ಜಟೆ, ಮೈಮೇಲೆ ಕೃಷ್ಣಾಜಿನ, ಕೈಲಿ ಮಾತ್ರ ಗಂಡುಗೊಡಲಿ. ವಿಕಟವೇಷದ ಈ ಬ್ರಾಹ್ಮಣ ನನ್ನು ಕಾಣುತ್ತಲೆ ಕಾರ್ತವೀರ್ಯನು ಅವನನ್ನು ಕೊಂದುಹಾಕುವಂತೆ ಸೈನ್ಯಕ್ಕೆ ಹೇಳಿ, ತಾನು ಅರಮನೆಯನ್ನು ಹೊಕ್ಕನು. ಆ ರಾಜನ ಹದಿನೇಳು ಅಕ್ಷೋಹಿಣಿಯ ಚದುರಂಗಸೇನೆ ಕ್ಷಣಮಾತ್ರದಲ್ಲಿ ಪರಶರಾಮನ ಕೊಡಲಿಗೆ ಸಿಕ್ಕಿ ನಾಶವಾಗಿ ಹೋಯಿತು. ರಣಭೂಮಿ ನೆತ್ತರ ಹೊಳೆಯಾಯಿತು. ಇದನ್ನು ಕಂಡು ರೋಷದಿಂದ ಕಾರ್ತವೀರ್ಯಾರ್ಜುನನು ತಾನೆ ಪರಶುರಾಮನೊಡನೆ ಯುದ್ಧಕ್ಕೆ ನಿಂತನು. ಅವನ ಐನೂರು ಕೈಗಳಲ್ಲಿ ಐನೂರು ಬಿಲ್ಲು ಗಳು, ಇನ್ನುಳಿದ ಐನೂರು ಕೈಗಳಿಂದ ಏಕಕಾಲಕ್ಕೆ ಐನೂರು ಬಾಣಗಳ ಮಳೆಯನ್ನೆ ಸುರಿಸಿ ದನು. ಆದರೇನು? ರಾಮನ ಒಂದು ಬಾಣ ಅವೆಲ್ಲವನ್ನೂ ನುಂಗಿ ನೀರುಕುಡಿಯಿತು. ಅನಂತರ ಆತ ತನ್ನ ಕೊಡಲಿಯಿಂದ ಅವನ ಕೈಗಳನ್ನೆಲ್ಲ ಕತ್ತರಿಸಿದ, ನಂತರ ತಲೆಯನ್ನೂ ಕತ್ತರಿಸಿದ. ಇದನ್ನು ಕಂಡು ಕಾರ್ತವೀರ್ಯನ ಮಕ್ಕಳಿಗೆ ಜೀವಭಯ ಹುಟ್ಟಿತು. ಅವರು ಓಡಿಹೋಗಿ, ಕಾಮಧೇನುವನ್ನು ತಂದು ಆತನಿಗೆ ಒಪ್ಪಿಸಿದರು. ಪರಶುರಾಮನು ಅದನ್ನು ತಂದೆಗೊಪ್ಪಿಸಿ, ತಾನು ಮಾಡಿದ ಕಾರ್ಯವನ್ನೆಲ್ಲ ಆತನಿಗೆ ವರದಿ ಒಪ್ಪಿಸಿದ. ಮಗ ಮಾಡಿದ ಕಾರ್ಯವನ್ನು ಜಮದಗ್ನಿ ಮೆಚ್ಚಲಿಲ್ಲ, ಒಪ್ಪಲಿಲ್ಲ. ‘ರಾಮ, ನೀನು ಮಾಡಿದುದು ತುಂಬ ತಪ್ಪು ಕೆಲಸ. ಬ್ರಾಹ್ಮಣನಿಗೆ ಶಾಂತಿಯೆ ಸರ್ವಸ್ವ, ಅದೇ ಬ್ರಹ್ಮತೇಜಸ್ಸು! ಪಟ್ಟಾಭಿಷಿಕ್ತ ನಾದ ರಾಜನನ್ನು ಕೊಲ್ಲುವುದು ಬ್ರಹ್ಮಹತ್ಯೆಗಿಂತ ದೊಡ್ಡ ಪಾಪ. ಈ ಪಾಪವನ್ನು ಕಳೆದುಕೊಳ್ಳುವುದಕ್ಕಾಗಿ ನೀನು ತೀರ್ಥಯಾತ್ರೆಯನ್ನು ಮಾಡಿಕೊಂಡು ಬಾ’ ಎಂದು ಹೇಳಿದನು. ಪರಶುರಾಮನು ‘ಹಸಾದ’ ಎಂದು ಹೇಳಿ, ತಂದೆಗೆ ನಮಸ್ಕರಿಸಿ, ಒಂದು ವರ್ಷಕಾಲ ತೀರ್ಥಯಾತ್ರೆ ಮಾಡಿ ಹಿಂದಿರುಗಿದನು.

ಪರಶುರಾಮನು ತೀರ್ಥಯಾತ್ರೆಯಿಂದ ಹಿಂದಿರುಗಿದ ಕೆಲವು ದಿನಗಳ ಮೇಲೆ ಮತ್ತೊಂದು ಭಯಂಕರ ಘಟನೆ ನಡೆಯಿತು. ಆತನ ತಾಯಿ ರೇಣುಕೆ ಎಂದಿನಂತೆ ನೀರು ತರಲೆಂದು ಗಂಗೆಗೆ ಹೋದಳು. ಅಲ್ಲಿ ಚಿತ್ರರಥನೆಂಬ ಗಂಧರ್ವ ಅಪ್ಸರೆಯರೊಡನೆ ನೀರಾಟವಾಡುತ್ತಿದ್ದ. ಆ ಗಂಧರ್ವನ ಸೌಂದರ್ಯವನ್ನು ಕಂಡು ಮರುಳಾದ ರೇಣುಕೆ ಬಹಳ ಹೊತ್ತಿನವರೆಗೆ ಅವನನ್ನೇ ನೋಡುತ್ತಾ ನಿಂತುಬಿಟ್ಟಳು. ಅದು ಹೋಮದ ಸಮಯ. ತಾನು ತಡಮಾಡಿದುದಕ್ಕಾಗಿ ಗಂಡನು ಕೋಪಿಸುವನೆಂಬ ಭಯದಿಂದಲೆ ಆಕೆ ಬೇಗಬೇಗ ಹಿಂದಿರುಗಿದಳು. ಆಕೆ ನೀರಿನ ಕೊಡವನ್ನು ಗಂಡನ ಮುಂದಿಟ್ಟು, ಭಕ್ತಿಯಿಂದ ಕೈಮುಗಿದುಕೊಂಡು ನಿಂತಳು. ಆದರೇನು? ಮಹಾತಪಸ್ವಿಯಾದ ಜಮದಗ್ನಿಗೆ ಜ್ಞಾನದೃಷ್ಟಿ ಯಿಂದ ಎಲ್ಲವೂ ಗೋಚರವಾಗಿತ್ತು. ಆತನು ಆಕೆಯತ್ತ ಕಣ್ಣೆತ್ತಿ ಕೂಡ ನೋಡದೆ, ಮಕ್ಕಳನ್ನು ಕರೆದು ಅವಳನ್ನು ಕತ್ತರಿಸಿ ಹಾಕುವಂತೆ ಹೇಳಿದನು. ಆದರೆ ತಾಯಿಯನ್ನು ಯಾರು ಕೈಯಾರೆ ಕೊಂದಾರು? ಅವರು ಆತನ ಮಾತನ್ನು ನಡೆಸಲಿಲ್ಲ. ಆತನ ಕೋಪ ಪ್ರಕೋಪಕ್ಕೆ ಹೋಯಿತು. ಆತನು ಪರಶುರಾಮನನ್ನು ಕೂಗಿ ಕರೆದು, ‘ಈ ನಿನ್ನ ತಾಯಿ ಯನ್ನೂ ಅಣ್ಣಂದಿರನ್ನೂ ಕತ್ತರಿಸಿಹಾಕು’ ಎಂದನು. ಪರಶುರಾಮನು ‘ಹಸಾದ’ ಎಂದು ಹೇಳಿ, ತನ್ನ ಗಂಡುಗೊಡಲಿಯಿಂದ ಒಂದೇ ಏಟಿಗೆ ಅವರನ್ನೆಲ್ಲ ಕತ್ತರಿಸಿಹಾಕಿದ. ಇದನ್ನು ಕಂಡು ಜಮದಗ್ನಿಗೆ ತುಂಬ ಸಂತೋಷವಾಯಿತು. ‘ಮಗು, ನಿನಗೆ ಬೇಕಾದ ವರವನ್ನು ಕೇಳು, ಕೊಡುತ್ತೇನೆ’ ಎಂದ. ಪರಶುರಾಮನಿಗೆ ತನ್ನ ತಂದೆಯ ಶಕ್ತಿ ಎಷ್ಟೆಂಬುದು ಗೊತ್ತು. ಆತ ‘ಅಪ್ಪ, ಈ ನನ್ನ ತಾಯಿಯನ್ನೂ ಅಣ್ಣಂದಿರನ್ನೂ ಬದುಕಿಸು. ನಾನು ಇವರನ್ನು ಕೊಂದೆನೆಂಬುದು ಇವರಿಗೆ ಜ್ಞಾಪಕವಿರಬಾರದು’ ಎಂದ. ಜಮದಗ್ನಿ ‘ತಥಾಸ್ತು’ ಎಂದ. ಸತ್ತವರೆಲ್ಲ ನಿದ್ದೆಯಿಂದ ಮೇಲಕ್ಕೇಳುವಂತೆ ಎದ್ದು ನಿಂತರು.

ಪರಶುರಾಮನು ಅವತರಿಸಿದ್ದುದೇ ದುಷ್ಟರಾದ ಕ್ಷತ್ರಿಯರನ್ನು ನಿರ್ಮೂಲ ಮಾಡುವು ದಕ್ಕೆ. ಅದಕ್ಕೆ ತಕ್ಕ ಸನ್ನಿವೇಶವೂ ಒದಗಿ ಬಂತು. ಕಾರ್ತವೀರ್ಯಾರ್ಜುನನ ಮಕ್ಕಳು ತಮ್ಮ ತಂದೆಯ ಮರಣವನ್ನು ಮತ್ತೆ ಮತ್ತೆ ನೆನೆದುಕೊಂಡು ಸಂಕಟಪಡುತ್ತಿದ್ದರು. ಒಂದು ದಿನ ಪರಶುರಾಮನೂ ಆತನ ಸೋದರರೂ ಆಶ್ರಮದಲ್ಲಿಲ್ಲದ ಸಮಯವನ್ನು ನೋಡಿ ಕೊಂಡು, ಅವರು ಜಮದಗ್ನಿಯ ಎಲೆವನೆಗೆ ನುಗ್ಗಿ, ರೇಣುಕೆಯ ದೈನ್ಯಕ್ಕೂ ಕಣ್ಣೀರಿಗೂ ಕರಗದೆ, ಧ್ಯಾನಮಗ್ನನಾಗಿದ್ದ ಆತನ ಕತ್ತನ್ನು ಕರಕರ ಕತ್ತರಿಸಿ, ತಮ್ಮೊಡನೆ ತೆಗೆದುಕೊಂಡು ಹೋದರು. ರೇಣುಕೆ ಗಟ್ಟಿಯಾಗಿ ಅಳುತ್ತಾ ‘ಅಯ್ಯೋ, ಮಗೂ, ರಾಮಾ, ಎಲ್ಲಿದ್ದಿ, ಬೇಗ ಬಾ’ ಎಂದು ಗಟ್ಟಿಯಾಗಿ ಕೂಗಿಕೊಂಡಳು. ಆ ಕೂಗು ದೂರದಲ್ಲಿದ್ದ ರಾಮನ ಕಿವಿಗೆ ಬಿತ್ತು. ಆತ ಬಿಲ್ಲಿನಿಂದ ಹಾರಿದ ಬಾಣದಂತೆ ಒಂದೆ ಉಸಿರಿಗೆ ಆಶ್ರಮಕ್ಕೆ ಓಡಿಬಂದು ನೋಡುತ್ತಾನೆ, ಮಹಾ ಅನರ್ಥ ನಡೆದುಹೋಗಿದೆ! ಆತ ಕೋಡಿಗಟ್ಟಿ ಹರಿ ಯುತ್ತಿರುವ ಕಣ್ಣೀರನ್ನು ಒರೆಸಿಕಳ್ಳುತ್ತಾ, ಗದ್ಗದನಾಗಿ, ‘ಅಪ್ಪ, ಮಹಾನುಭಾವನಾದ ನೀನು ಹೋಗಿ ನಾವು ಅನಾಥರಾದೆವು. ನಿನ್ನಂತಹ ಸಾಧುಪುರುಷನಿಗೆ ಇಂತಹ ಮರಣವೆ?’ ಎಂದು ಹೇಳುವಷ್ಟರಲ್ಲಿ ಆತನ ಹುಬ್ಬು ಗಂಟಿಕ್ಕಿತು, ಹಲ್ಲುಗಳು ಕಟಕಟ ಶಬ್ದಮಾಡಿ ತುಟಿಯನ್ನು ಕಚ್ಚಿದವು. ಆತನು ಕೆಂಗಣ್ಣಿನಿಂದ ಅಣ್ಣಂದಿರ ಕಡೆ ನೋಡಿ ‘ನೀವು ಈ ದೇಹವನ್ನು ಎಚ್ಚರದಿಂದ ನೋಡಿಕೊಳ್ಳುತ್ತಿರಿ’ ಎಂದು ಹೇಳಿ ತನ್ನ ಗಂಡು ಗೊಡಲಿಯೊಡನೆ ಹೊರಕ್ಕೋಡಿದನು. ಆತನು ನೇರವಾಗಿ ಮಾಹಿಷ್ಮತೀ ನಗರಕ್ಕೆ ನುಗ್ಗಿ, ಅಲ್ಲಿದ್ದ ಕಾರ್ತವೀರ್ಯಾರ್ಜುನನ ಪೀಳಿಗೆಯವರನ್ನು ಒಂದು ಪಿಳ್ಳೆಯೂ ಉಳಿಯದಂತೆ ಕೊಚ್ಚಿ ಹಾಕಿದನು. ಅವರ ಹೆಣಗಳ ಬಣಬೆಯನ್ನು ಬೆಟ್ಟದಂತೆ ರಾಶಿಹಾಕಿ, ಅವರ ನೆತ್ತ ರನ್ನು ಹೊಳೆಯಾಗಿ ಹರಿಸಿದನು. ಆದರೂ ಆತನ ಕೋಪ ಶಾಂತವಾಗಲಿಲ್ಲ. ಆತನು ಬಿರುಗಾಳಿಯಂತೆ ಭೂಮಂಡಲದ ಮೂಲೆ ಮೂಲೆಗಳನ್ನೂ ಹೊಕ್ಕು ಕ್ಷತ್ರಿಯರ ಹುಟ್ಟಡ ಗಿಸಿದನು. ಒಮ್ಮೆ ಇಮ್ಮೆಯಲ್ಲ, ಇಪ್ಪತ್ತೊಂದು ಬಾರಿ ಜಗತ್ತನ್ನೆಲ್ಲ ತಿರುಗಿ, ತಾನು ಕೊಂದ ಕ್ಷತ್ರಿಯರ ರಕ್ತದಿಂದ ಶಮಂತಪಂಚಕವೆಂಬ ಐದು ಮಡುಗಳನ್ನು ತುಂಬಿದನು. ಆ ನೆತ್ತರಿಂದಲೆ ತಂದೆಗೆ ತರ್ಪಣ ಕೊಟ್ಟ ಮೇಲೆ ಆತನ ರೋಷ ಅಡಗಿತು. ಅನಂತರ ಮಾಹಿಷ್ಮತಿಯಲ್ಲಿದ್ದ ತಂದೆಯ ತಲೆಯನ್ನು ಹಿಂದಕ್ಕೆ ತಂದು, ತಂದೆಯ ದೇಹಕ್ಕೆ ಅದನ್ನು ಸೇರಿಸಿ, ಉತ್ತರಕ್ರಿಯಾದಿಗಳನ್ನು ನಡೆಸಿದನು. ಜಮದಗ್ನಿಯು ಸದ್ಗತಿಯನ್ನು ಪಡೆದು ಸಪ್ತಪುಷಿ ಮಂಡಲದಲ್ಲಿ ಒಬ್ಬನಾಗಿ ಇಂದಿಗೂ ಆಕಾಶದಲ್ಲಿ ಬೆಳಗುತ್ತಿರುವನು. ಪರುಶುರಾಮನು ಅನೇಕ ಯಾಗಗಳನ್ನು ಮಾಡಿ, ಮೇಘದ ಮರೆಯಿಂದ ಬಂದ ಸೂರ್ಯ ನಂತೆ ಸಮಸ್ತ ಪಾಪಗಳಿಂದಲೂ ಮುಕ್ತನಾಗಿ, ಮುಂದಿನ ಮನ್ವಂತರದಲ್ಲಿ ತಾನೂ ಸಪ್ತ ಪುಷಿಗಳಲ್ಲಿ ಒಬ್ಬನಾಗಬೇಕಾದುದುರಿಂದ ಮಹೇಂದ್ರಪರ್ವತದಲ್ಲಿ ತಪಸ್ಸು ಮಾಡು ತ್ತಿರುವನು.

ಇದಿಷ್ಟೂ ಸತ್ಯವತಿಯ ಮೊಮ್ಮಗನ ಪ್ರತಾಪವಾಯಿತು. ಆಕೆಯ ತಾಯಿ ಮಗಳ ಪಾಲಿನ ಹವಿಸ್ಸನ್ನು ತಿಂದಳಲ್ಲಾ, ಆಕೆಯ ಹೊಟ್ಟೆಯಲ್ಲಿಯೇ ವಿಶ್ವಾಮಿತ್ರ ಹುಟ್ಟಿದುದು. ಇವನು ಕ್ಷತ್ರಿಯನಾಗಿ ಹುಟ್ಟಿದರೂ ಬ್ರಹ್ಮರ್ಷಿಯಾದುದಕ್ಕೆ ಆ ಹವಿಸ್ಸೆ ಕಾರಣ. ರೋಹಿತನು ಶುನಶ್ಶೇಫನನ್ನು ಯಾಗಪಶುವಾಗಿ ಕೊಂಡೊಯ್ಯುತ್ತಿದ್ದಾಗ, ಈತ ತಪಸ್ಸು ಮಾಡುತ್ತಿದ್ದ. ಶುನಶ್ಯೇಫನು ಓಡಿಬಂದು ತನಗೆ ಶರಣಾಗಲು, ಈತನು ಅವನನ್ನು ತನ್ನ ಮಗನಾಗಿ ಸ್ವೀಕರಿಸಿ, ತನಗಿದ್ದ ನೂರು ಮಕ್ಕಳಿಗೆ ಅವನನ್ನು ಹಿರಿಯನನ್ನಾಗಿ ಮಾಡಿದ. ಅವನಿಗೆ ಅಲ್ಲಿಂದ ಮುಂದೆ ದೇವರಾತನೆಂದು ಹೆಸರಾಯಿತು. ಈತನನ್ನು ಹಿರಿಯನನ್ನಾಗಿ ಸ್ವೀಕರಿಸಲು ವಿಶ್ವಾಮಿತ್ರನ ಮೊದಲ ಐವತ್ತು ಮಕ್ಕಳು ಒಪ್ಪಲಿಲ್ಲ. ಆಗ ವಿಶ್ವಾಮಿತ್ರ ಕೋಪದಿಂದ ಅವರನ್ನೆಲ್ಲ ಮ್ಲೇಚ್ಛರಾಗುವಂತೆ ಶಪಿಸಿದ. ಉಳಿದ ಮಕ್ಕಳೂ ದೇವ ರಾತನೂ ವಿಶ್ವಾಮಿತ್ರನ ವಂಶ ಎಲ್ಲೆಲ್ಲಿಯೂ ಹಬ್ಬಿ ಹರಡುವಂತೆ ಮಾಡಿದರು.

