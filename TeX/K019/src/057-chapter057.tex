
\chapter{೫೭. ರಾಸಕ್ರೀಡೆ}

ಭಗವಂತನ ಅನುಗ್ರಹ ಶರತ್ಕಾಲದ ಬೆಳದಿಂಗಳಾಗಿ ಬಾನಿನಿಂದ ಬುವಿಗೆ ಅವತರಿ ಸಿತು. ಒಡನೆಯೆ ಬೃಂದಾವನದ ಸುಂದರವನವು ತನ್ನ ಎರಡೂ ಕೈಗಳಿಂದ ಅದನ್ನು ಬಾಚಿ ತಬ್ಬಿಕೊಂಡಿತು. ಈ ಸಮ್ಮಿಳನವನ್ನು ಕಂಡು ಸಂತೋಷಿಸಿದ ಪ್ರಕೃತಿದೇವಿ ಅರೆ ಬಿರಿದ ಮಲ್ಲಿಗೆಗಳ ಕಂಪನ್ನು ಅವರಿಗೆ ಕುಡಿಸಿ, ತನ್ನ ಭ್ರಮರ ಕಂಠದಿಂದ ಮಂಗಳ ಗೀತವನ್ನು ಹಾಡಿ ಅವರನ್ನು ಹರಸಿದಳು. ಈ ಮಂಗಳ ಮುಹೂರ್ತದಲ್ಲಿ ಭೂದೇವಿಯ ಇಷ್ಟಾರ್ಥ ಸಿದ್ಧಿಯಂತೆ ಅಲ್ಲಿ ಪ್ರತ್ಯಕ್ಷನಾದ ಶ್ರೀಕೃಷ್ಣ ಪರಮಾತ್ಮ. ಒಡನೆಯೆ ಬೆಳದಿಂಗಳ ದೇಹ ಕಾಂತಿ ಉಜ್ವಲವಾಯಿತು, ಬೃಂದಾವನದ ಪ್ರಣಯೋತ್ಸಾಹ ಮೈ ದುಂಬಿತು. ಈ ರಸ ನಿಮಿಷಕ್ಕೆ ಚೇತನವನ್ನು ತುಂಬುವುದಕ್ಕಾಗಿ ಶ್ರೀಕೃಷ್ಣನು ತನ್ನ ಚಿಗುರುಗೈಗಳಲ್ಲಿದ್ದ ಕೊಳಲನ್ನು ತೊಂಡೆಯ ತುಟಿಗೆ ತಾಕಿಸಿದನು. ಮರುಕ್ಷಣದಲ್ಲಿ ಗಾನಾಮೃತದ ತುಂಬು ನೆರೆ ಹರಿದು ಬಂದು ಇಡೀ ಬ್ರಹ್ಮಾಂಡವನ್ನೆ ತುಂಬಿತು. ಅಂಬರದಲ್ಲಿದ್ದ ಚಂದ್ರಬಿಂಬ ನಕ್ಷತ್ರ ಸಮೂಹದೊಡನೆ ಅದನ್ನು ಕೇಳುತ್ತಾ ಮೈಮರೆತು ಕುಳಿತನು. ಗೋವರ್ಧನ ಪರ್ವತವು ತನ್ನ ಸಮಸ್ತ ಜೀವರಾಶಿಯೊಡನೆ ಮೈಯೆಲ್ಲ ಕಿವಿಯಾಗಿ ಆಲಿಸುತ್ತಾ ಕುಳಿತಿತು. ಯಮುನಾ ತರಂಗಿಣಿ ತನ್ನ ಗಂಡನ ಮನೆಗೆ ಹೋಗುವುದನ್ನು ಮರೆತು, ಗಾನಸುಧೆಯನ್ನು ಸವಿಯುತ್ತ ನಿಂತುಬಿಟ್ಟಳು. ಬೃಂದಾವನದ ಗಿಡ ಮರ ಬಳ್ಳಿಗಳು ಕೇವಲ ಚಿತ್ರಗಳಾಗಿ ಹೋದವು. ಗಾಳಿಯೂ ತನ್ನ ಕರ್ತವ್ಯವನ್ನು ಮರೆತು ಶ್ರೀಕೃಷ್ಣನ ಸಮೀಪದಲ್ಲಿಯೇ ಸಪ್ಪುಳಿಲ್ಲದೆ ಸುಳಿದಾಡುತ್ತಿದ್ದನು.

ಹೀಗೆ ಶ್ರೀಕೃಷ್ಣನ ಕೊಳಲುಗಾನದಿಂದ ಇಡೀ ಬ್ರಹ್ಮಾಂಡವೆ ಶಾಂತಿಯ ಸಾಗರವಾಗಿ ಹೋಗಿರುವಾಗ ಗೋಕುಲದ ಗೋಪಿಯರ ಮನಸ್ಸು ಮಾತ್ರ ಅಲ್ಲೋಲಕಲ್ಲೋಲವಾದ ಕಡಲಂತಾಯಿತು. ತೆರೆತೆರೆಯಾಗಿ ಹರಿದು ಬಂದ ಆ ಗಾನ ಕಿವಿಗೆ ಬೀಳುತ್ತಿದ್ದಂತೆ ಅವರು ನಿಟ್ಟುಬಿದ್ದರು. ಅವರು ಮಾಡುತ್ತಿದ್ದ ಮನೆಗೆಲಸ ಅಲ್ಲಿಗೆ ನಿಂತುಹೋಯಿತು. ಹಾಲು ಕರೆಯುತ್ತಿದ್ದವರು ಪಾತ್ರೆಯನ್ನು ಅಲ್ಲಿಯೆ ಇಟ್ಟು ಮೇಲಕ್ಕೆದ್ದರು; ಒಲೆಯಮೇಲೆ ಹಾಲಿ ಟ್ಟವಳು, ಅದು ಉಕ್ಕುತ್ತಿರಲು ಅದನ್ನು ಮನಸ್ಸಿಗೆ ಹಚ್ಚಿಕೊಳ್ಳದೆ ಹಾಗೆಯೆ ಹೊರಟಳು; ಊಟಕ್ಕೆ ಕುಳಿತಿದ್ದವಳು ತುತ್ತನ್ನು ನೆಲಕ್ಕೆ ಹಾಕಿ ಹಾಗೆಯೆ ಮೇಲೆದ್ದಳು; ಮೈ ತೊಳೆಯು ತ್ತಿದ್ದವಳು ಅರ್ಧಕ್ಕೆ ಅದನ್ನು ನಿಲ್ಲಿಸಿ ಹಾಗೆಯೆ ಹೊರಟಳು. ಕೊಳಲ ಗಾನದಿಂದ ಪರವಶ ರಾದ ಅವರಿಗೆ ಉಚಿತ ಅನುಚಿತದ ಪ್ರಶ್ನೆಯೆ ಏಳುವಂತಿಲ್ಲ. ಅವರಿಗೀಗ ತಾಯ್ತಂದೆಗಳ, ಗಂಡಂದಿರ, ದೇವರ–ಯಾವ ಭಯವೂ ಇಲ್ಲ. ಅವರೆಲ್ಲ ದುಡುದುಡು ಓಡಿ ಬಂದರು, ಶ್ರೀಕೃಷ್ಣನ ಬಳಿಗೆ. ಹೀಗೆ ಬಂದು ತನ್ನ ಸುತ್ತಲೂ ನಿಂತ ಗೋಪಿಯರನ್ನು ಕಂಡು ಆ ಕಪಟನಾಟಕ ಸೂತ್ರಧಾರಿಯು ‘ಎಲೆ ಬಡನಡುವಿನ ಬೆಡಗುಗಾತಿಯರೆ, ಇಷ್ಟು ಹೊತ್ತಿ ನಲ್ಲಿ ಇಲ್ಲಿಗೇಕೆ ಬಂದಿರಿ? ಗಂಡಂದಿರಿರುವ ಗರತಿಯರು ಹೀಗೆ ಮಾಡುತ್ತಾರೆಯೆ? ಗಂಡನ ಸೇವೆ, ಮಕ್ಕಳ ರಕ್ಷಣೆ–ಇವು ಗೃಹಿಣಿಯರ ಮುಖ್ಯಧರ್ಮ. ಗಂಡನಾದವನು ನಡತೆಗೆಟ್ಟವನಾಗಿರಲಿ, ಕುರೂಪಿಯಾಗಿರಲಿ, ಮುಪ್ಪಿನಮುದುಕನಾಗಿರಲಿ, ರೋಗಿಯಾಗಿ ಇರಲಿ, ದರಿದ್ರನಾಗಿರಲಿ ಆತನೆ ಪರದೈವವೆಂದು ಭಾವಿಸಬೇಕಾದುದು ಒಳ್ಳೆಯ ಹೆಣ್ಣಿನ ಲಕ್ಷಣ. ಒಳ್ಳೆಯ ಮನೆತನದ ಹೆಣ್ಣು ಪರಪುರುಷನನ್ನು ಕಣ್ಣೆತ್ತಿಯೂ ನೋಡಬಾರದು. ನಡತೆಗೆಟ್ಟ ಹೆಣ್ಣು ಲೋಕದ ನಿಂದೆಗೂ ಹಾಸ್ಯಕ್ಕೂ ಗುರಿಯಾಗುವುದಲ್ಲದೆ ಸತ್ತಮೇಲೆ ನರಕಕ್ಕೂ ಗುರಿಯಾಗುತ್ತಾಳೆ. ಆದ್ದರಿಂದ ನೀವು ಬಂದ ಹಾದಿಯಿಂದಲೆ ಹಿಂತಿರುಗಿ ಹೋಗಿ. ನಾನು ಹಿಂದೆಯೆ ನಿಮಗೊಮ್ಮೆ ತಿಳಿಸಿದ್ದೇನೆ–ನನ್ನ ಪ್ರೇಮವೆಂದರೆ ಇಂದ್ರಿಯ ಸುಖಾನುಭವವಲ್ಲವೆಂದು. ಒಂದು ಪಕ್ಷ ನೀವು ನನ್ನನ್ನು ಭಗವಂತನೆಂಬ ಭಾವನೆಯಿಂದ ಆದರಿಸಲು ಬಂದಿದ್ದರೂ ನಿಮ್ಮ ಕಾರ್ಯ ಸರಿಯಲ್ಲ. ನನ್ನ ಧ್ಯಾನ ಕೀರ್ತನೆಗಳಿಂದ ಬರುವ ಭಕ್ತಿ ನನ್ನ ದೇಹ ಸಂಬಂಧದಿಂದ ಹುಟ್ಟಲಾರದು. ನೀವು ನಿಮ್ಮ ಮನೆಯಲ್ಲಿದ್ದುಕೊಂಡೆ ನನ್ನ ಉಪಾಸನೆಯನ್ನು ನಡೆಸಿರಿ’ ಎಂದನು.

ಶ್ರೀಕೃಷ್ಣನ ನುಡಿಗಳನ್ನು ಕೇಳಿ ಗೋಪಿಯರ ಆಶಾಗೋಪುರ ಮುರಿದು ಬಿತ್ತು. ಅವರು ನಿಟ್ಟುಸಿರು ಬಿಟ್ಟರು, ತುಟಿಗಳು ಒಣಗಿದವು, ಕಣ್ಣೀರು ಸುರಿದವು; ಅವರು ಉತ್ತರ ಕೊಡಲು ತೋಚದೆ ನಾಚಿಕೆಯಿಂದ ತಲೆಯನ್ನು ಬಾಗಿಸಿ ಉಂಗುಷ್ಠದಿಂದ ನೆಲವನ್ನು ಕೆರೆ ಯುತ್ತ ನಿಂತುಕೊಂಡರು. ಸ್ವಲ್ಪಕಾಲವಾದ ಮೇಲೆ ಅವರಲ್ಲೊಬ್ಬಳು ಮೌನವನ್ನು ಮುರಿದು ‘ಪ್ರಭು ಶ್ರೀಕೃಷ್ಣ, ನಾವು ಎಲ್ಲವನ್ನೂ ತೊರೆದು ನಿನ್ನ ಪಾದವನ್ನು ಆಶ್ರಯಿಸಿ ದ್ದೇವೆ. ನೀನು ನಮ್ಮನ್ನು ಸ್ವೀಕರಿಸಬೇಕು. ಹೆಣ್ಣಿನ ಧರ್ಮವನ್ನು ಕುರಿತು ನೀನು ನೀಡಿದ ಬುದ್ಧಿವಾದ ನಮಗೆ ಈಗ ಹಿಡಿಸುವಂತಿಲ್ಲ. ನೀನೆ ನಮ್ಮ ಪತಿ, ಬಂಧು, ಮಿತ್ರ, ಮನೆ, ಮಗು–ಎಲ್ಲವೂ ಎಂದು ನಂಬಿ ಬಂದಿರುವವರು, ನಾವು. ಮೋಕ್ಷವನ್ನು ಬೇಡಿ ಭಜಿಸುವವರನ್ನು ಭಗವಂತ ಸ್ವೀಕರಿಸುವಂತೆ ನೀನು ನಮ್ಮನ್ನು ಸ್ವೀಕರಿಸು. ಹೇ ಆನಂದ ಮೂರ್ತಿ, ನಿರಾತಂಕವಾಗಿ ಮನೆಗೆಲಸದಲ್ಲಿ ಮುಳುಗಿದ್ದ ನಮ್ಮ ಮನಸ್ಸನ್ನು ಅಲ್ಲಿಂದ ಸೆಳೆದು ಸೂರೆಗೊಂಡ ನೀನು ಈಗ ಮನೆಗೆ ಹಿಂದಿರುಗಿರೆಂದರೆ ಹೇಗೆ? ನಮ್ಮ ಮನಸ್ಸು ನಿನ್ನಲ್ಲಿ ನೆಲಸಿರುವಾಗ ಈ ದೇಹ ಅಲ್ಲಿ ಹೋಗಿ ಮಾಡುವುದಾದರೂ ಏನು? ನಮ್ಮ ಕಾಲು ಗಳು ನಿನ್ನ ಹತ್ತಿರವೆ ನಾಟಿಹೋಗಿವೆ, ಅವನ್ನು ಕಿತ್ತುಕೊಂಡು ಹೋಗುವುದು ಸಾಧ್ಯವೇ ಇಲ್ಲ. ಹೇ ಜೀವೇಶ! ನಿನ್ನ ಕೊಳಲಗಾನವನ್ನು ಕೇಳಿ ನಮ್ಮ ಹೃದಯದಲ್ಲಿ ಹೊತ್ತಿಕೊಂಡಿ ರುವ ವಿರಹಾಗ್ನಿಯು ನಿನ್ನ ಅಧರಾಮೃತದಿಂದ ಮಾತ್ರ ಆರುವುದಕ್ಕೆ ಸಾಧ್ಯ. ಹಾಗಾಗ ದಿದ್ದರೆ ನಾವು ನಿನ್ನನ್ನೆ ನೆನೆಯುತ್ತ ಸತ್ತುಹೋಗುತ್ತೇವೆ. ಹೇ ನಾಥ, ಮುಂಗುರುಳುಗಳಿಂದ ಅಲಂಕೃತವಾದ ಆ ನಿನ್ನ ಮುಖ, ಅಮೃತವನ್ನು ತೊಟ್ಟಿಕ್ಕುತ್ತಿರುವ ಆ ತುಟಿಗಳು, ಲಕ್ಷ್ಮಿ ನೆಲಸಿರುವ ಆ ವಿಶಾಲವಾದ ಎದೆ–ಇವುಗಳನ್ನು ನಮ್ಮ ಕಣ್ಣಾರೆ ಕಂಡು, ನಿನ್ನ ಕೊಳಲ ಗಾನವನ್ನು ಕಿವಿಯಾರೆ ಕೇಳಿ ನಿನ್ನನ್ನು ಮೋಹಿಸದಿರುವುದು ಎಂತು ಸಾಧ್ಯ? ನಿನ್ನರೂಪ ಗಾನಗಳಿಗೆ ಪಶುಪಕ್ಷಿಗಳು ಕೂಡ ಮೈಮರೆತು ಮೋಹಗೊಳ್ಳುತ್ತಿರುವಾಗ ಹೆಣ್ಣು ಹೆಂಗಸ ರಾದ ನಮ್ಮ ಪಾಡೇನು? ಆದ್ದರಿಂದ ವಿರಹಾಗ್ನಿಯಿಂದ ಬೆಂದುಹೋಗುತ್ತಿರುವ ಈ ನಿನ್ನ ದಾಸಿಯರನ್ನು ನಿನ್ನ ಅಮೃತಹಸ್ತದಿಂದ ಮುಟ್ಟಿ ಬದುಕಿಸು’ ಎಂದು ಬೇಡಿದಳು.

ಯೋಗಿಗಳಲ್ಲಿ ಯೋಗಿಯಾದ ಶ್ರೀಕೃಷ್ಣನಲ್ಲಿ ಕಾಮಕ್ಕೆ ಎಡೆಯಲ್ಲಿ? ಆದರೆ ಸರ್ವ ಸಂಗವನ್ನೂ ತ್ಯಜಿಸಿ ತನ್ನನ್ನೆ ಬಯಸುತ್ತಿರುವ ಆ ಗೋಪಿಯರನ್ನು ಕಂಡು ಆತನ ಹೃದಯ ಕರಗಿತು. ಆತನು ಅವರೊಡನೆ ವಿಹರಿಸ ಹೊರಟನು. ಮಲ್ಲಿಗೆಯ ಮೊಗ್ಗುಗಳಂತಿರುವ ತನ್ನ ಹಲ್ಲುಗಳನ್ನು ಕಾಣುವಂತೆ ಆತನೊಮ್ಮೆ ನಗುತ್ತಲೆ ಆ ಗೋಪಿಯರ ಮೈ ಛಳಿಯೆಲ್ಲ ಬಿಟ್ಟುಹೋಯಿತು. ಅವರೆಲ್ಲರೂ ಚಂದ್ರನನ್ನು ಸುತ್ತುವರಿದ ನಕ್ಷತ್ರಗಳಂತೆ ಆತನ ಸುತ್ತಲೂ ನಿಂತು ಪ್ರಣಯಗೀತೆಗಳನ್ನು ಹಾಡುವುದಕ್ಕೆ ಪ್ರಾರಂಭಿಸಿದರು. ಅವರ ಗಾನಕ್ಕೆ ತಕ್ಕಂತೆ ಆತ ಕೊಳಲನ್ನು ಬಾರಿಸುತ್ತಾ ಹೋದ. ಅವರೆಲ್ಲ ಹಾಡುತ್ತ ಹಾಡುತ್ತಾ ಯಮುನಾನದಿ ಮರಳತೀರವನ್ನು ಸೇರಿದರು. ಅಲ್ಲಿ ಶ್ರೀಕೃಷ್ಣನು ತನ್ನ ಸುತ್ತಲೂ ನಿಂತ ಆ ಹೆಣ್ಣುಗಳಲ್ಲಿ ಒಬ್ಬಳ ಭುಜದಮೇಲೆ ಕೈಯಿಟ್ಟ, ಮತ್ತೊಬ್ಬಳ ಮುಂಗುರುಳನ್ನು ಸವರಿದ, ಇನ್ನೊಬ್ಬಳ ತೋಳೊಳಗೆ ತನ್ನ ತೋಳನ್ನು ಹೆಣೆದು ಎದೆಗೊರಗಿಸಿಕೊಂಡ, ಬೇರೊಬ್ಬಳನ್ನು ಆಲಂಗಿಸಿಕೊಂಡ. ಹೀಗೆ ಒಬ್ಬೊಬ್ಬಳನ್ನೂ ಶೃಂಗಾರ ಚೇಷ್ಟೆಗಳಿಂದ ಕೀಟಲೆಮಾಡಿ ಅವರ ಕಾಮವನ್ನು ಕೆರಳಿಸಿದ. ಶ್ರೀಕೃಷ್ಣನ ಈ ಸಲಿಗೆಯನ್ನು ಕಂಡು ಗೋಪಿಯರೆಲ್ಲ ತಮ್ಮ ಸಮಾನರಾದ ಸುಂದರಿಯರೆ ಜಗತ್ತಿನಲ್ಲಿಲ್ಲವೆಂದು ಅಹಂಕಾರ ಕ್ಕೊಳಗಾದರು!

ಗೋಪಿಯರ ಹೃದಯದಲ್ಲಿ ಅಹಂಕಾರ ಮೊಳೆಯುತ್ತಲೆ ಶ್ರೀಕೃಷ್ಣನು ಅವರ ಮಧ್ಯ ದಿಂದ ಮಾಯವಾಗಿ ಹೋದನು. ಇದ್ದಕ್ಕಿದ್ದಂತೆ ಆತ ಕಣ್ಮರೆಯಾಗಲು ಗೋಪಿಯರು ಮುಂಗಾಣದೆ ಮಂಕಾಗಿ ಹೋದರು. ಅವರು ಕಣ್ಣಿಗೆ ಬಿದ್ದ ಮರಗಿಡಗಳನ್ನೂ ಪಶುಪಕ್ಷಿ ಗಳನ್ನೂ ಪ್ರಶ್ನೆಮಾಡುತ್ತಾ ಬೃಂದಾವನದಲ್ಲೆಲ್ಲಾ ಅಲೆದಾಡಿದರು. ‘ಎಲೆ ಅರಳಿಮರವೆ, ಶ್ರೀಕೃಷ್ಣನನ್ನು ಕಂಡೆಯಾ? ಎಲೆ ಜಾಜಿ, ಎಲೆ ಮಲ್ಲಿಗೆ, ಶ್ರೀಕೃಷ್ಣ ಎತ್ತಕಡೆ ಹೋದ? ಎಲೆ ಮಾವಿನಮರವೆ, ಹಲಸಿನಮರವೆ, ನೇರಳೆಮರವೆ, ನೀವೆಲ್ಲ ಸದಾ ಇಲ್ಲಿಯೆ ಇರುವ ವರು; ಶ್ರೀಕೃಷ್ಣನ ಪರಿಚಯವುಳ್ಳವರು; ಪರೋಪಕಾರಕ್ಕಾಗಿಯೆ ಬಾಳನ್ನು ಸವೆಸುವವರು; ಆತನೆಲ್ಲಿರುವನೆಂಬುದನ್ನು ತಿಳಿಸಿ ನಮ್ಮ ಸಂಕಟವನ್ನು ಹೋಗಲಾಡಿಸಿ. ಭೂದೇವಿಯ ಪುಳಕದಂತೆ ಇರುವ ಹೇ ಗರುಕೆಯೆ, ಶ್ರೀಕೃಷ್ಣನ ಪದತಲವನ್ನು ಸೋಕುವ ನೀನೆ ಪುಣ್ಯ ಶಾಲಿ; ಆತ ಎತ್ತ ಹೋದ? ಏ ಜಿಂಕೆ, ನಿನ್ನ ಕಣ್ಣುಗಳನ್ನು ನೋಡಿದರೆ ಈಗತಾನೆ ಶ್ರೀಕೃಷ್ಣ ನನ್ನು ನೀನು ಕಂಡಿರಬೇಕೆನಿಸುತ್ತದೆ; ಎಲ್ಲಿ ಆತ?’ ಎಲ್ಲಿ ಕೇಳಿದರೂ ಆತನ ಸುಳಿವಿಲ್ಲ. ಗೋಪಿಯರಿಗೆ ಹುಚ್ಚು ಹಿಡಿದಂತಾಯಿತು. ಅವರೊಬ್ಬರೊಬ್ಬರೂ ತಾವೇ ಶ್ರೀಕೃಷ್ಣ ನೆಂದು ಭಾವಿಸಿಕೊಂಡು ಆತನ ಲೀಲೆಗಳನ್ನು ಅನುಕರಿಸಿದರು. ತಾನು ಕೃಷ್ಣನೆಂದು ಹೇಳಿ ಕೊಂಡು ಪಕ್ಕದಲ್ಲಿರುವವಳನ್ನು ಪೂತನಿಯೆಂದು ಕರೆದು, ಅವಳ ಮೊಲೆಗುಡಿಯ ಹೊರ ಟಳು; ಮತ್ತೊಬ್ಬಳು ತನ್ನ ಪಕ್ಕದವಳನ್ನು ವತ್ಸಾಸುರನೆಂದು ಹೇಳಿ ಎತ್ತಿಹಾಕಿದಳು; ಇನ್ನೊಬ್ಬಳು ಶ್ರೀಕೃಷ್ಣನಂತೆ ಕೊಳಲೂದುವುದನ್ನು ನಟಿಸಿದಳು; ಮಗುದೊಬ್ಬಳು ಪಕ್ಕ ದವಳನ್ನು ಆಲಿಂಗಿಸಿಕೊಂಡು ಮುತ್ತಿಕ್ಕಿದಳು; ಇನ್ನೂ ಒಬ್ಬಳು ಅವನಂತೆ ನಡೆದಳು. ಅವರ ಹುಚ್ಚು ಇನ್ನೂ ಯಾವ ಹಾದಿಯನ್ನು ಹಿಡಿಯುತ್ತಿತ್ತೊ! ಅಷ್ಟರಲ್ಲಿಯೆ ಶ್ರೀಕೃಷ್ಣನ ಹೆಜ್ಜೆಯ ಗುರುತು ಕಾಣಿಸಿತು. ಹೆಣ್ಣುಗಳೆಲ್ಲ ಓಡಿಬಂದು ಆ ಹೆಜ್ಜೆಯ ಗುರುತನ್ನೇ ಅನು ಸರಿಸಿಕೊಂಡು ಹೋದರು. ಇದೇನು? ಅವನ ಹೆಜ್ಜೆಯ ಜೊತೆಯಲ್ಲಿ ಹೆಣ್ಣೊಬ್ಬಳ ಹೆಜ್ಜೆಯೂ ಹೋಗಿದೆ. ಅದನ್ನು ಕಂಡು ಅವರು ಕಣ್ಣೀರುಗರೆಯುತ್ತಾ ‘ಅಯ್ಯೋ, ನಮ್ಮ ಕೃಷ್ಣ ಮತ್ತಾವಳ ಜೊತೆಯಲ್ಲೊ ವಿಹರಿಸುತ್ತಾ ಹೊರಟಿದ್ದಾನೆ. ಅವಳ ಪುಣ್ಯವೇ ಪುಣ್ಯ. ಪೂರ್ವಜನ್ಮದಲ್ಲಿ ಅವಳೇನು ಪುಣ್ಯ ಮಾಡಿದ್ದಳೊ! ಅಲ್ಲಲ್ಲಿ ಅವಳ ಹೆಜ್ಜೆ ಕಾಣುತ್ತಿಲ್ಲ; ಶ್ರೀಕೃಷ್ಣ ಅವಳನ್ನು ಎತ್ತಿಕೊಂಡು ನಡೆದಿರಬೇಕು. ಇಲ್ಲಿ ಕೃಷ್ಣನು ಮೆಟ್ಟಿಂಗಾಲಿನಿಂದ ನಿಂತ ಗುರುತಿದೆ, ಗಿಡದಿಂದ ಹೂಗಳನ್ನು ಕಿತ್ತು ಅವಳ ತುರುಬಿಗೆ ಮುಡಿಸಿರಬೇಕು’ ಎಂದು ಒಳಗೊಳಗೆ ಕುದಿಯುತ್ತಿದ್ದರು.

ಅತ್ತ ಶ್ರೀಕೃಷ್ಣನು ಆ ಹೆಣ್ಣುಗಳಿಗೆಲ್ಲ ಪಾಠ ಕಲಿಸಲೆಂದು ನಿಜವಾಗಿಯೂ ಮತ್ತೊಬ್ಬ ಹೆಣ್ಣಿನೊಡನೆ ವಿಹರಿಸುತ್ತಿದ್ದ. ಆ ಭಾಗ್ಯವನ್ನು ಪಡೆದ ಆ ಹೆಣ್ಣಿಗೂ ಅಹಂಕಾರ ಹುಟ್ಟ ಬೇಕೆ? ಅವಳು ತನ್ನ ಸಮಾನರಾದ ರೂಪವತಿಯರಿಲ್ಲವೆಂದೂ, ಶ್ರೀಕೃಷ್ಣ ತನ್ನ ಮೇಲಿನ ಮೋಹದಿಂದ ತನ್ನ ಕೈಗೊಂಬೆಯಾಗಿರುನೆಂದೂ ಅವಳ ಭ್ರಮೆ. ಅವಳು ಶ್ರೀಕೃಷ್ಣ ನೊಡನೆ ‘ನಡೆದು ನಡೆದು ನನ್ನ ಕಾಲು ಸೋತಿವೆ. ನನ್ನನ್ನು ಎತ್ತಿಕೊಂಡು ಹೋಗು’ ಎಂದಳು. ಶ್ರೀಕೃಷ್ಣನು ನಗುತ್ತಾ ‘ಅಯ್ಯೋ ಪಾಪ; ಬಾ, ನನ್ನ ಭುಜದ ಮೇಲೆ ಕೂತುಕೊ’ ಎಂದು ಬಾಗಿ ನಿಂತ. ಆ ಹೆಣ್ಣು ತನ್ನ ಕಾಲೆತ್ತಿ ಕುಳಿತುಕೊಳ್ಳುವುದಕ್ಕೆ ಹೋಗಿ ಆಧಾರ ತಪ್ಪಿದ ಬಳ್ಳಿಯಂತೆ ನೆಲಕ್ಕುರುಳಿದಳು. ಶ್ರೀಕೃಷ್ಣ ಮಾಯವಾಗಿಹೋಗಿದ್ದ. ಕೆಟ್ಟಮೇಲೆ ಆ ಹೆಣ್ಣಿಗೆ ಬುದ್ಧಿ ಬಂತು. ಅವಳು ‘ಹಾ ನಾಥಾ, ಹೇ ಸ್ವಾಮಿ, ಎಲ್ಲಿ ಹೋದೆ?’ ಎಂದು ಅಳುತ್ತಾ ಆತನನ್ನು ಅರಸ ಹೊರಟಳು. ಆದರೆ ಆಕೆಗೆ ಸಿಕ್ಕುದು ತನ್ನಂತೆಯೆ ಶ್ರೀಕೃಷ್ಣನನ್ನು ಅರಸುತ್ತಿದ್ದ ಗೋಪಿಯರ ತಂಡ. ಪರಸ್ಪರ ತಮ್ಮ ದುಃಖಗಳನ್ನು ತೋಡಿಕೊಂಡು ಅವರೆಲ್ಲ ಒಟ್ಟಿಗೆ ಶ್ರೀಕೃಷ್ಣನನ್ನು ಹುಡುಕಹೊರಟರು. ಆ ವೇಳೆಗೆ ಚಂದ್ರನು ಮುಳುಗುವ ಹೊತ್ತಾಯಿತು. ಬೆಳದಿಂಗಳು ತಗ್ಗಿತು. ಕತ್ತಲು ನುಗ್ಗಿತು. ಮುಂದೆ ಕಾಣದಂತೆ ಹಳುವಿ ನಲ್ಲಿ ಸಿಕ್ಕಿಬಿದ್ದಾಗಲೂ ಅವರಿಗೆ ಕೃಷ್ಣನ ಹುಚ್ಚು ಬಿಡಲಿಲ್ಲ, ಮನೆಯ ಯೋಚನೆ ಕಾಡ ಲಿಲ್ಲ. ಅವರೆಲ್ಲ ‘ಕೃಷ್ಣಾ, ಕೃಷ್ಣಾ!’ ಎಂದು ಕೂಗುತ್ತಾ ತಾವು ಹೊರಟ ಯುಮುನಾತೀರಕ್ಕೆ ಬಂದರು.

ಯಮುನಾನದಿಯ ಮಳಲಮೇಲೆ ಕುಳಿತ ಗೋಪಿಯರು ಅವನ ಗುಣಗಾನಗಳನ್ನು ಹಾಡತೊಡಗಿದರು. ‘ಹೇ ಶ್ರೀಕೃಷ್ಣ, ನೀನು ಹುಟ್ಟಿ ಗೋಕುಲ ಸನಾಥವಾಯಿತು! ಲಕ್ಷ್ಮಿ ಸದಾ ಇಲ್ಲಿ ನೆಲಸುವಂತಾಯಿತು! ನಾವು ನಿನ್ನಲ್ಲಿಯೆ ಪ್ರಾಣವಿಟ್ಟುಕೊಂಡಿರುವವರು. ನಮಗೆ ದರ್ಶನವಿತ್ತು ಕಾಪಾಡು. ನಿನ್ನನ್ನು ಅಕೃತ್ರಿಮ ಪ್ರೇಮದಿಂದ ಆಶ್ರಯಿಸಿರುವ ನಮ್ಮನ್ನು ಕಾಪಾಡದಿದ್ದರೆ ನಿನಗೆ ಹೆಂಗೊಲೆಯ ಪಾಪ ಬರುತ್ತದೆ! ನೀನು ಕಾಳಿಂದಿ ಮಡು ವಿನ ವಿಷದಿಂದ ನಮ್ಮನ್ನು ಕಾಪಾಡಿದೆ, ಅಘಾಸುರ ಅರಿಷ್ಟರಿಂದ ಕಾಪಾಡಿದೆ. ಈಗ ಮಾತ್ರ ಈ ಉದಾಸೀನವೇಕೆ? ನಿನ್ನ ಗುಟ್ಟೆಲ್ಲವೂ ನಮಗೆ ಗೊತ್ತಿದೆ, ಬ್ರಹ್ಮನ ಪ್ರಾರ್ಥನೆಯಂತೆ ಲೋಕರಕ್ಷಣೆಗಾಗಿ ಅವತರಿಸಿರುವ ನೀನು ಯಶೋದೆಯ ಮಗನಂತೆ ಲೀಲೆಯನ್ನು ತೋರುತ್ತಿರುವೆ. ಸ್ವಾಮಿ, ನಾವು ನಿನ್ನಲ್ಲಿ ಬೇರಾವುದನ್ನೂ ಬಯಸುತ್ತಿಲ್ಲ, ನಿನ್ನ ಪಾದ ಗಳನ್ನೆ ನಂಬಿರುವ ನಮ್ಮ ತಲೆಯಮೇಲೆ ನಿನ್ನ ಕೈಯಿಡು. ದೇವದೇವ, ನಿನ್ನ ಮಂದ ಹಾಸಕ್ಕೆ ಮರುಳಾಗಿ ನಾವು ನಿನ್ನ ದಾಸಿಯರಾದೆವು. ಆ ಮಂದಹಾಸವನ್ನು ಚೆಲ್ಲುವ ನಿನ್ನ ಮುಖಕಮಲವನ್ನು ನಮಗೆ ತೋರಿಸು. ಹೇ ಕಮಲನಯನ! ನಿನ್ನ ಮುದ್ದು ಮಾತುಗಳಿಗೆ ಮಾರುಹೋಗಿರುವ ನಮಗೆ ನಿನ್ನ ಆ ತುಟಿಗಳಿಂದ ತೊಟ್ಟಿಡುವ ಅಮೃತವನ್ನು ಕುಡಿಸಿ ಬದುಕಿಸು. ಸ್ವಾಮಿ! ನಿನ್ನ ಕಥೆ ಸಂಸಾರದಲ್ಲಿ ನೊಂದು ಬೆಂದವರಿಗೆ ಜೀವದಾನ ಮಾಡು ತ್ತದೆ. ಅದು ಕಿವಿಗೆ ಹಿತ, ಹೃದಯಕ್ಕೆ ತಂಪು. ಅಮೃತಕ್ಕಿಂತಲೂ ರುಚಿಯಾಗಿರುವ ಆ ಕಥೆ ಯನ್ನು ಕೇಳಬೇಕಾದರೆ ಪೂರ್ವಜನ್ಮದಲ್ಲಿ ಬಹಳ ಪುಣ್ಯವನ್ನು ಮಾಡಿರಬೇಕು. ಅಯ್ಯೋ ಶ್ರೀಕೃಷ್ಣ! ನಿನ್ನ ಮುಗುಳ್​ನಗೆ, ಪ್ರೇಮಮಯವಾದ ನೋಟ, ಮನೋಹರವಾದ ಆಟ ಪಾಟಗಳಿಂದ ವಂಚಿತರಾಗಿ ನಾವು ಹೇಗೆ ಬದುಕೋಣ? ಹೇ ಮನೋಹರ! ನೀನು ತುರು ಮಂದೆಯನ್ನು ಮೇಯಿಸ ಹೊರಟಾಗ ನಿನ್ನ ಮೆಲ್ಲಡಿಗೆಲ್ಲಿ ಹುಲ್ಲು ಚುಚ್ಚಿ ನೋವಾಗು ವುದೋ ಎಂದು ನಾವು ಮಿಡುಕುತ್ತಿದ್ದೆವಲ್ಲ! ಆಗ ನಿನ್ನ ಅಗಲಿಕೆಯಿಂದ ಕ್ಷಣವನ್ನು ಯುಗದಂತೆ ಕಳೆದು, ಸಂಜೆ ನಿನ್ನ ಸುಳಿಗುರುಳಿನ ಮುದ್ದು ಮುಖವನ್ನು ಕಾಣುತ್ತಲೆ ಬಿಡು ಗಣ್ಣರಾಗಿ ನಿನ್ನನ್ನು ನೋಡುತ್ತಾ, ನೋಟಕ್ಕೆ ಅಡ್ಡಿ ಮಾಡುವ ರೆಪ್ಪೆಗಳನ್ನು ನಿಂದಿಸು ತ್ತಿದ್ದೆವಲ್ಲಾ! ಈ ನಮ್ಮ ಪ್ರೇಮವನ್ನು ನೀನು ಕಾಣೆಯೇನು? ಪತಿ ಪುತ್ರರನ್ನೂ ಬಂಧು ಬಾಂಧವರನ್ನೂ ತೊರೆದು, ನಿನ್ನ ಕೊಳಲಗಾನಕ್ಕೆ ಮೋಹಿತರಾಗಿ ಓಡಿಬಂದೆವಲ್ಲಾ, ಇಂತಹ ಅಪರಾತ್ರಿಯಲ್ಲಿ ಹೀಗೆ ಓಡಿಬಂದ ಹೆಣ್ಣುಗಳನ್ನು ಯಾವ ಗಂಡು ತಾನೆ ಹೀಗೆ ನಿರಾಕರಿಸಿಯಾನು? ಮನೋಹರ! ನೀನು ಹುಟ್ಟಿ ಗೋಕುಲವನ್ನು ಪಾವನಗೊಳಿಸಿದೆ, ಗೋಪಾಲರನ್ನೆಲ್ಲ ಸಂತಸಗೊಳಿಸಿದೆ. ನಮಗೆ ಮಾತ್ರ ಏಕೆ ಇಂತಹ ಸಂಕಟವನ್ನುಂಟು ಮಾಡುತ್ತಿರುವೆ? ಹೇ ಪ್ರಾಣಪ್ರಿಯ, ನಮ್ಮ ಗತಿ ಏನಾದರೂ ಆಗಲಿ, ನಮ್ಮ ಪ್ರಾಣ ಸ್ವರೂಪಿಯಾದ ನೀನು ಅಡವಿಯಲ್ಲಿ ಎಲ್ಲಿ ತಿರುಗಾಡುತ್ತಿರುವೆಯೋ ಎಂದು ನಮ್ಮ ಸಂಕಟ. ಆ ಚಿಗುರಿನಂತಿರುವ ನಿನ್ನ ಪಾದಗಳು ಕಠಿಣವಾದ ನಮ್ಮ ಮಾಲೆಗಳನ್ನು ತಾಕಿದರೆ ಎಲ್ಲಿ ನೋಯುತ್ತವೋ ಎಂದು ಆತಂಕಗೊಳ್ಳುತ್ತಿದ್ದೆವಲ್ಲಾ! ಕಲ್ಲುಮುಳ್ಳಗಳು ತಗಲಿ ಅದೆಷ್ಟು ನೋಯುವುದೋ ಎಂದು ನಮ್ಮ ಜೀವ ತುಡಿದುಕೊಳ್ಳುತ್ತಿದೆ’ ಎಂದು ಹಂಬಲಿ ಸಿದರು. ಇಷ್ಟೇ ಅಲ್ಲ ಅವರು ಗಟ್ಟಿಯಾಗಿ ಅಳತೊಡಗಿದರು.

ಹೆಣ್ಣಿನ ಕಣ್ಣೀರು ಶ್ರೀಕೃಷ್ಣನ ಹೃದಯವನ್ನು ಕರಗಿಸಿತು. ಆತನು ಮಂದಹಾಸವನ್ನು ಬೀರುತ್ತಾ, ಮನ್ಮಥನಿಗೂ ಮನ್ಮಥನಂತಿರುವ ಮೋಹನಾಕಾರದಿಂದ ಅವರ ಮಧ್ಯದಲ್ಲಿ ಪ್ರತ್ಯಕ್ಷನಾದನು. ಆತನನ್ನು ಕಾಣುತ್ತಲೆ ಗೋಪಿಯರಿಗೆ ಹೋದ ಪ್ರಾಣ ಹಿಂದಕ್ಕೆ ಬಂದಂತಾಯಿತು. ಅವರ ಕಣ್ಣುಗಳು ಕಾಂತಿಯನ್ನು ಮುಕ್ಕುಳಿಸಿದುವು. ಅವರ ಸಂತೋಷ ಸಂಭ್ರಮಗಳಿಗೆ ಗರಿಮೂಡಿದಂತಾಯಿತು. ಒಬ್ಬಳು ತನ್ನೆರಡು ಕೈಗಳಿಂದಲೂ ಅವನ ಕೈಗಳನ್ನು ಭದ್ರವಾಗಿ ಹಿಡಿದುಕೊಂಡಳು, ಮತ್ತೊಬ್ಬಳು ಅವನ ತೋಳನ್ನು ಹಿಡಿದು ತನ್ನ ಭುಜದ ಮೇಲಿಟ್ಟುಕೊಂಡಳು. ಒಬ್ಬಳು ಆತನ ತಂಬುಲಕ್ಕೆ ಕೈಯೊಡ್ಡಿದರೆ, ಮತ್ತೊಬ್ಬಳು ಆತನ ಪಾದಗಳನ್ನು ಎತ್ತಿ ತನ್ನ ಎದೆಗೆ ಒತ್ತಿಕೊಂಡಳು. ಶ್ರೀಕೃಷ್ಣನು ತಮ್ಮನ್ನು ವಂಚಿಸಿ ಹೋದನೆಂಬ ಪ್ರಣಯಕೋಪದಿಂದ ಒಬ್ಬಳು ತನ್ನ ಹುಬ್ಬುಗಳನ್ನು ಗಂಟು ಹಾಕಿ ಕೊಂಡು ತುಟಿಯನ್ನು ಕಚ್ಚುತ್ತಿದ್ದರೆ, ಮತ್ತೊಬ್ಬಳು ನೆಟ್ಟ ದಿಟ್ಟಿಯಿಂದ ಅವನ ಮುಖ ವನ್ನೆ ನೋಡುತ್ತಾ, ಮೈಮರೆತು ನಿಂತಳು. ಮಗುದೊಬ್ಬಳು ಮನಸ್ಸಿನಲ್ಲಿಯೆ ಆ ಮೋಹನ ರೂಪವನ್ನು ಕಲ್ಪಿಸಿಕೊಂಡು ಆಲಿಂಗನ ಸುಖವನ್ನು ಅನುಭವಿಸುತ್ತಿದ್ದಳು. ಭಗವಂತನ ದರ್ಶನವಾಗುತ್ತಲೆ ತಾಪತ್ರಯಗಳು ಹಾರಿಹೋಗುವಂತೆ, ಶ್ರೀಕೃಷ್ಣನ ದರ್ಶನದಿಂದ ಅವರ ವಿರಹತಾಪ ದೂರವಾಯಿತು. ತಾವು ಹೊದ್ದಿದ್ದ ಮೇಲ್ಮುಸುಕನ್ನು ಹಾಸಿ ಅವರು ಆತನಿಗಾಗಿ ಒಂದು ಆಸನವನ್ನು ಸಿದ್ಧಪಡಿಸಿದರು. ಯೋಗಿಗಳ ಹೃದಯಪೀಠದಲ್ಲಿ ಮಂಡಿಸುವ ಶ್ರೀಕೃಷ್ಣನು ಅವರ ಸೀರೆಯ ಸೆರಗಿನ ಮೇಲೆಯೆ ಕುಳಿತನು. ಗೋಪಿಯರು ಆತನ ಸುತ್ತಲೂ ಕುಳಿತು, ತಮ್ಮ ಮುಗುಳ್ನಗೆಯಿಂದಲೂ ಓರೆನೋಟದಿಂದಲೂ ಆತನ ಮನವನ್ನು ಸೆಳೆಯುತ್ತಾ, ಒಬ್ಬೊಬ್ಬರೂ ಸ್ಪರ್ಧೆಯಿಂದ ಆತನ ಪಾದವನ್ನು ತಮ್ಮ ತೊಡೆಯಮೇಲಿಟ್ಟುಕೊಂಡು ಒತ್ತುತ್ತಿದ್ದರು.

ಗೋಪಿಯರ ಉಪಚಾರದಿಂದ ಸಂತುಷ್ಟನಾದ ಶ್ರೀಕೃಷ್ಣನು, ಅವರನ್ನು ಕುರಿತು ‘ಸಖಿಯರೆ, ನಾನು ಒಮ್ಮೆ ನನ್ನನ್ನು ಭಜಿಸಿದವರಿಗೂ ಉದಾಸೀನನಾಗಿರುವಂತೆ ಕಾಣ ಬಹುದು. ಆದರೆ ನಾನು ಹಾಗೆ ಮಾಡುವುದು ಕೇವಲ ಉಪಕಾರ ಬುದ್ಧಿಯಿಂದಲೆ. ಅದಕ್ಕೆ ನೀವೆ ಉತ್ತಮ ನಿದರ್ಶನ. ನೋಡಿ, ನೀವು ಪತಿಪುತ್ರಾದಿ ಸಮಸ್ತವನ್ನೂ ತೊರೆದು ನನ್ನ ಬಳಿಗೆ ಬಂದಿರಿ. ಆದರೂ ನಾನು ಕಣ್ಮರೆಯಾದೆ. ಇದರಿಂದ ನೀವು ಎಡೆಬಿಡದೆ ನನ್ನ ಧ್ಯಾನದಲ್ಲಿ ಮಗ್ನರಾದಿರಿ. ನಿಮ್ಮ ಧ್ಯಾನ ನಿಶ್ಚಲವಾಗಲೆಂದೆ ನಾನು ಕಣ್ಮರೆಯಾದುದು. ಧ್ಯಾನ ನಿಶ್ಚಲವಾದಾಗ ನಾನು ಅತ್ಯುತ್ತಮ ಫಲವನ್ನು ಕೊಡುತ್ತೇನೆ. ಗೆಳತಿಯರೆ, ನಿಮ್ಮ ಶ್ರೇಯಸ್ಸಿಗಾಗಿಯೆ ನಾನು ಕಣ್ಮರೆಯಾಗಿದ್ದುದರಿಂದ ನೀವು ನನ್ನ ಮೇಲೆ ಕೋಪ ಮಾಡಿ ಕೊಳ್ಳಬೇಡಿ. ಇಗೋ ನಿಮಗೆ ಈ ವರವನ್ನು ನೀಡುತ್ತಿದ್ದೇನೆ. ನೀವು ಮನೆ ಮಠಗಳನ್ನು ತೊರೆದು ನನ್ನನ್ನು ಒಂದೇ ಮನಸ್ಸಿನಿಂದ ಭಜಿಸಿದುದಕ್ಕಾಗಿ, ಇನ್ನು ಮುಂದೆ ನಿಮಗೆ ಎಂದೆಂದಿಗೂ ಸಂಸಾರ ಬಂಧನವಿಲ್ಲದಂತಾಗಲಿ. ನಿಮ್ಮ ಆತ್ಮಗಳು ಪರಮಾತ್ಮನಲ್ಲಿ ಒಂದಾಗಿಹೋಗಲಿ’ ಎಂದನು.

ಒಮ್ಮೆ ಶ್ರೀಕೃಷ್ಣನನ್ನು ಮುಟ್ಟುತ್ತಲೆ ಗೋಪಿಯರ ವಿರಹವೇದನೆ ಹಾರಿಹೋಯಿತು. ಅವರು ಶ್ರೀಕೃಷ್ಣನ ಸುತ್ತ ಗುಂಡಾಗಿ ನಿಂತು ರಾಸಕ್ರೀಡೆಯ ನರ್ತನವನ್ನು ಆರಂಭಿಸಿದರು. ಭಗವಂತನಾದ ಶ್ರೀಕೃಷ್ಣನು ತನ್ನ ಮಾಯಾಶಕ್ತಿಯಿಂದ ಹಲವು ಶ್ರೀಕೃಷ್ಣರ ದೇಹಗಳನ್ನು ತಾಳಿ ಇಬ್ಬಿಬ್ಬರು ಗೋಪಿಯರ ಮಧ್ಯೆ ಒಬ್ಬೊಬ್ಬ ಶ್ರೀಕೃಷ್ಣನಾಗಿ ಕಾಣಿಸಿಕೊಂಡನು. ಗೋಪಿಯರು ಸರಪಣಿಯ ಕೊಂಡಿಯಂತೆ ಪರಸ್ಪರ ಕೈಹಿಡಿದು ನಿಂತರು; ಶ್ರೀಕೃಷ್ಣನು ತನ್ನ ಎಡಬಲದ ಗೋಪಿಯರ ಕೊರಳನ್ನು ತನ್ನ ತೋಳುಗಳಿಂದ ಆಲಿಂಗಿಸಿಕೊಂಡನು. ಈ ದಿವ್ಯದರ್ಶನದಿಂದ ಧನ್ಯರಾಗುವುದಕ್ಕೆ ಆಕಾಶದಲ್ಲಿ ನೆರೆದ ದೇವತೆಗಳು ಶ್ರೀಕೃಷ್ಣನ ಮೇಲೆ ಹೂವಿನ ಮಳೆಗರೆದರು. ಒಬ್ಬೊಬ್ಬ ಗೋಪಿಯೂ ‘ನಾನೆ ಧನ್ಯೆ; ಶ್ರೀಕೃಷ್ಣನ ಪ್ರೇಮ ನನ್ನ ಮೇಲೆಯೆ ಹೆಚ್ಚು’ ಎಂಬ ಭಾವನೆಯಿಂದ ಉಬ್ಬಿ, ಕೈಬಳೆ ಕಾಲಂದುಗೆಗಳ ಝಣಝಣ ನಾದದೊಡನೆ ಕುಣಿಯುತ್ತ, ಕೃಷ್ಣನತ್ತ ಕಣ್ಣು ಹಾರಿಸುವಳು, ಮುಗುಳ್​ನಗು ವಳು, ಮಧುರ ಧ್ವನಿಯಿಂದ ಹಾಡುವಳು. ಗೋಪಿಯರ ಹಾಡು ದಶದಿಕ್ಕುಗಳನ್ನೂ ತುಂಬಿತು. ಶ್ರೀಕೃಷ್ಣನು ‘ಭೇಷ್​’ ಎಂದು ಹಾಡುವವರನ್ನು ಪ್ರೋತ್ಸಾಹಿಸುವನು; ಅವರು ಮೈದುಂಬಿ ಮತ್ತಷ್ಟು ಗಟ್ಟಿಯಾಗಿ ಹಾಡುವರು. ಶ್ರೀಕೃಷ್ಣನು ಅವರೊಡನೆ ತಾನೂ ಹಾಡಿ ಅವರನ್ನು ಸಂತೋಷಪಡಿಸುವನು. ಮಧ್ಯೆಮಧ್ಯೆ ಆ ಹೆಣ್ಣುಗಳಲ್ಲೊಬ್ಬಳು ಮೋಹಾತಿ ಶಯದಿಂದ ಆತನ ಕೈಯನ್ನು ತನ್ನ ಎದೆಯ ಮೇಲಿಟ್ಟುಕೊಳ್ಳುವಳು; ಮತ್ತೊಬ್ಬಳು ಅವನ ತೋಳನ್ನು ಮುತ್ತಿಟ್ಟುಕೊಳ್ಳುವಳು; ಇನ್ನೊಬ್ಬಳು ಅವನ ಎದೆಯ ಮೇಲೆ ತಲೆಯಿಟ್ಟು ಆನಂದಿಸುವಳು; ಮಗುದೊಬ್ಬಳು ತನ್ನ ಕೆನ್ನೆಯನ್ನು ಅವನ ಕೆನ್ನೆಗೆ ಸೋಕಿಸಿ, ಅವನ ಬಾಯ್ದಂಬುಲವನ್ನು ತನ್ನ ತುಟಿಗಳಿಂದ ಹೀರಿ ಆನಂದ ಪಡುವಳು.

ಗೋಪಿಯರು ಕುಣಿದರು, ಕುಣಿದರು, ದಣಿವಿಲ್ಲದೆ ಕುಣಿದರು. ಆವರು ಮುಡಿದ ಹೂ ಉದುರಿದವು, ಮುಂಗುರುಳು ಚದರಿದವು, ಬೆವರು ತೊಟ್ಟಿಕ್ಕಿತು, ಮುಡಿ ಬಿಚ್ಚಿತು, ಉಡಿಗೆ ಸಡಲಿತು. ಆದರೇನು? ಶ್ರೀಕೃಷ್ಣನ ಕೈಹಿಡಿದು ಮೈಮರೆತ ಆ ಹೆಣ್ಣುಗಳು ರಾಸಕ್ರೀಡೆಯಲ್ಲಿ ತಲ್ಲೀನರಾಗಿ ಕುಣಿದರು. ಶ್ರೀಕೃಷ್ಣನೂ ಅವರಂತೆಯೇ ನಗುತ್ತ, ಕೆಲೆಯುತ್ತ, ಚಿಕ್ಕ ಮಗು ಕನ್ನಡಿಯಲ್ಲಿ ಪ್ರತಿಬಿಂಬವನ್ನು ಕಂಡು ಕುಣಿಯುವಂತೆ, ಕುಣಿದಾಡಿದ. ಈ ದಿವ್ಯ ಲೀಲೆ ಯನ್ನು ಕಂಡು ದೇವತೆಗಳು ಮೈಮರೆತರು, ನಕ್ಷತ್ರಸಹಿತನಾದ ಚಂದ್ರ ಮೈಮರೆತ, ಪ್ರಕೃತಿ ಮೈಮರೆಯಿತು, ಕಾಲಪುರುಷ ಕೂಡ ಮೈಮರೆತ. ಈ ಲೀಲೆಯಲ್ಲಿ ಮೈಮರೆಯ ದಿದ್ದವನೆಂದರೆ ಲೀಲಾಮಾನುಷವಿಗ್ರಹವಾದ ಆ ಶ್ರೀಕೃಷ್ಣನೊಬ್ಬನೆ. ಆತನು ತನ್ನ ಹುಟ್ಟು ಹಿಗ್ಗಲಿಸಿ, ಒಬ್ಬೊಬ್ಬ ಗೋಪಿಗೆ ಒಬ್ಬೊಬ್ಬ ಕೃಷ್ಣನಾದ. ರಾಸಕ್ರೀಡೆಯಿಂದ ಬಳಲಿದ ಆ ಹೆಣ್ಣುಗಳನ್ನು ಮೃದುನುಡಿಗಳಿಂದ ರಮಿಸುತ್ತಾ ಅವರ ಮುಖದ ಬೆವರನ್ನು ಒರಸಿದ. ಅವರ ಬಳಲಿಕೆಯನ್ನು ಹೋಗಲಾಡಿಸುವುದಕ್ಕಾಗಿ ಆತನು ಅವರನ್ನು ಯಮುನಾ ನದಿಯ ನೀರಿಗೆ ಎಳೆದೊಯ್ದು, ಅಲ್ಲಿ ಅವರೊಡನೆ ನೀರಾಟಕ್ಕೆ ಆರಂಭಿಸಿದ. ಅವರು ಆತನ ಮೇಲೆ ನೀರೆರಚಿದರು, ಆತ ಅವರ ಮೇಲೆ ಎರಚಿದ; ತಾನು ನಕ್ಕ, ಅವರನ್ನೂ ನಗಿಸಿದ; ಆತನ ಲೀಲೆಯನ್ನು ಕಂಡ ಗೋಪಿಯರೆಲ್ಲ ಸಂತೋಷದಿಂದ ಹಿರಿಹಿರಿ ಹಿಗ್ಗಿಹೋದರು. ನೀರಾಟ ಮುಗಿದ ಮೇಲೆ ವನವಿಹಾರ. ಈಗ ಹಲವು ಕೃಷ್ಣರು ಹೋಗಿ ಒಬ್ಬನೇ ಕೃಷ್ಣ ಇದ್ದಾನೆ. ಹೆಣ್ಣಾನೆಗಳ ತಂಡದೊಡನೆ ಹೊರಟ ಸಲಗನಂತೆ ಆತನು ಗೋಪಿಯರನ್ನೆಲ್ಲ ಮನೋಹರವಾದ ಒಂದು ಉಪವನಕ್ಕೆ ಕರೆ ತಂದನು. ಅರಳಿದ ಹೂಗಳ ಸುವಾಸನೆಯನ್ನು ಹೊತ್ತು ತಂಗಾಳಿ ಬೆಳುದಿಂಗಳೊಡನೆ ಚೆಲ್ಲಾಟವಾಡುತ್ತಿರುವ ಆ ವನದಲ್ಲಿ ಶ್ರೀಕೃಷ್ಣನು ಗೋಪಿಯರೊಡನೆ ವಿಹರಿಸಿ, ಅವರನ್ನು ಧನ್ಯರನ್ನಾಗಿ ಮಾಡಿದನು.

ಇಲ್ಲಿ ಒಂದು ಪ್ರಶ್ನೆ ಏಳುತ್ತದೆ. ಧರ್ಮಸ್ಥಾಪನೆಗಾಗಿಯೆ ಅವತರಿಸಿದ ಭಗವಂತನು ಜಗತ್ತಿಗೆ ಮಾದರಿಯಾಗುವುದಕ್ಕೆ ಬದಲಾಗಿ ಪರಸ್ತ್ರೀಯರೊಡನೆ ಹೀಗೆ ವಿಹರಿಸುವುದು ಸರಿಯೆ? ಅದಕ್ಕೆ ಉತ್ತರ ಇಷ್ಟೆ–ಧರ್ಮಾಧರ್ಮದ ಪ್ರಶ್ನೆ ಅಹಂಕಾರ ಮಮಕಾರಗಳುಳ್ಳ ಸಂಸಾರಿಗಳಿಗೆ ಹೊರತು ಬ್ರಹ್ಮಜ್ಞಾನಿಗಳಿಗಲ್ಲ. ಬೆಂಕಿಯು ಎಲ್ಲವನ್ನೂ ತಿಂದು ತೇಗಿ ದರೂ ಅದಕ್ಕೆ ಪಾಪಲೇಪವಿಲ್ಲ. ಹಾಗೆಯೇ ಜ್ಞಾನಿಗಳೂ. ಅವರು ಹೀಗೆ ಮಾಡುವರೆಂದು ಸಾಮಾನ್ಯರೂ ಹಾಗೆ ಮಾಡಹೊರಡುವುದು ಶುದ್ಧ ತಪ್ಪು. ಪರಮೇಶ್ವರನು ಹಾಲಾಹಲ ವನ್ನು ಕುಡಿದನೆಂದು ಸಾಮಾನ್ಯನು ಕುಡಿಯುವುದಕ್ಕೆ ಸಾಧ್ಯವೆ? ಮಹಾಪುರುಷರು ಆಡಿದ ಮಾತು ಪ್ರಮಾಣವೇ ಹೊರತು ಅವರು ಮಾಡಿದ ಕಾರ್ಯ ನಮಗೆ ಪ್ರಮಾಣವಲ್ಲ. ಇದರ ರಹಸ್ಯ ಇಷ್ಟೆ–ಮಹಾಪುರುಷರಲ್ಲಿ ನಾನು, ನನ್ನದು ಎಂಬ ಅಹಂಕಾರ ಮಮಕಾರಗಳಾ ಗಲಿ ಅವುಗಳಿಂದ ಹುಟ್ಟುವ ಕಾಮ ಕ್ರೋದಾಧಿಗಳಾಗಲಿ ಇರುವುದಿಲ್ಲ. ಅವರು ಮಾಡುವ ಸತ್ಕರ್ಮಕ್ಕೆ ಪುಣ್ಯವಿಲ್ಲ, ದುಷ್ಕರ್ಮಕ್ಕೆ ಪಾರವಿಲ್ಲ. ಅವರು ಸದಾ ಬ್ರಹ್ಮಾನಂದದಲ್ಲಿ ತಲ್ಲೀನರಾಗಿರುವರು. ಮಹಾಪುರುಷರ ವಿಚಾರದಲ್ಲಿಯೇ ಈ ಮಾತನ್ನು ಹೇಳುವಾಗ, ಸರ್ವೇಶ್ವರನೆನಿಸಿಕೊಂಡ ಶ್ರೀಕೃಷ್ಣನಿಗೆ ಪಾಪಪುಣ್ಯಗಳ, ಧರ್ಮಾಧರ್ಮಗಳ ಪ್ರಶ್ನೆಯೆಲ್ಲಿ ಯದು? ಆ ಗೋಪಿಯರಿಗೂ ಗೋಪಾಲರಿಗೂ ಅಂತರಾತ್ಮವಾಗಿರುವ ಆತನು ಗೋಪಿಯ ರನ್ನು ಮುಟ್ಟಿದರೆ ದೋಷವೆ? ಕರ್ಮಮಾಡಿಸುವವನು, ಆ ಕರ್ಮಕ್ಕೆ ಫಲನೀಡುವವನು, ಅದಕ್ಕೆ ಸಾಕ್ಷಿಯಾಗಿರುವವನು ಎಲ್ಲವೂ ಆತನೆ. ಇನ್ನೊಂದು ದೊಡ್ಡ ಆಶ್ಚರ್ಯವೆಂದರೆ, ಗೋಪಿಯರೆಲ್ಲ ಶ್ರೀಕೃಷ್ಣನೊಡನೆ ಬೃಂದಾವನದಲ್ಲಿ ವಿಹರಿಸುತ್ತಿರುವಾಗ, ಅವರು ತಮ್ಮ ಪಕ್ಕದಲ್ಲಿಯೇ ಇದ್ದಂತೆ ಅವರ ಗಂಡಂದಿರ ಅನುಭವ. ಶ್ರೀಕೃಷ್ಣನೆ ಆಯಾ ರೂಪದಲ್ಲಿ ಅವರ ಪಕ್ಕದಲ್ಲಿದ್ದನೆಂದು ಹೇಳಬಹುದು. ಎಂದಮೇಲೆ ಶ್ರೀಕೃಷ್ಣನ ಮೇಲೆ ಪರಸ್ತ್ರೀ ಗಮನದ ಆರೋಪ ಶುದ್ಧ ಅವಿವೇಕ ಅಷ್ಟೆ.

ಗೋಪಿಯರು ಬೆಳಗಿನ ಜಾವದ ಹೊತ್ತಿಗೆ ಶ್ರೀಕೃಷ್ಣನಿಂದ ಬೀಳ್ಕೊಂಡು ತಮ್ಮತಮ್ಮ ಮನೆಗಳಿಗೆ ಹಿಂದಿರುಗಿದರು. ಅವರ ರಾಸಕ್ರೀಡೆಯ ಕಥೆಯನ್ನು ಕೇಳಿದವರು ಪಾಪ ವಿಮುಕ್ತರಾಗಿ ಧನ್ಯರಾಗುತ್ತಾರೆ!

