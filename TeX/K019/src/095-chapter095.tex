
\chapter{೯೫. ಸಹೇಂದ್ರಾಯ ತಕ್ಷಕಾಯ ಸ್ವಾಹಾ}

ಶುಕಮಹರ್ಷಿ ಅತ್ತ ಹೋಗುತ್ತಲೆ, ಇತ್ತ ಪರೀಕ್ಷಿದ್ರಾಜನು ಗಂಗಾತೀರಕ್ಕೆ ಬಂದು, ಪೂರ್ವಾಗ್ರವಾಗಿರುವ ದರ್ಭಾಸನದಮೇಲೆ ಉತ್ತರಾಭಿಮುಖವಾಗಿ ಕುಳಿತು ಧ್ಯಾನಮಗ್ನ ನಾಗಿದ್ದನು. ಆತನ ಚಿತ್ತ ಪರಬ್ರಹ್ಮನಲ್ಲಿ ಲೀನವಾಗಿ, ಆತನು ಸಮಾಧಿಸ್ಥನಾದನು. ಮರು ಕ್ಷಣದಲ್ಲಿ ಆತನು ಮುಕ್ತಾವಸ್ಥೆಯನ್ನೂ ಪಡೆದನು. ಅಂದು ಪುಷಿಶಾಪದ ಕಡೆಯದಿನ ವಾದ್ದರಿಂದ, ಪುಷಿಶಾಪವನ್ನು ಪುತವಾಗಿ ಮಾಡುವುದಕ್ಕಾಗಿ ಸಾಕ್ಷಾತ್ ತಕ್ಷಕನೆ ಪರೀಕ್ಷಿ ದ್ರಾಜನನ್ನು ಕಚ್ಚಿ ಕೊಲ್ಲುವುದಕ್ಕಾಗಿ ಹೊರಟನು. ಅದೇ ಸಮಯದಲ್ಲಿಯೆ ಕಾಶ್ಯಪನೆಂಬ ವಿಷವೈದ್ಯನು ಸರ್ಪದಷ್ಟನಾಗಿ ಸಾಯುವ ರಾಜನನ್ನು ಬದುಕಿಸಿ, ಆತನಿಂದ ಅಪಾರವಾದ ಹಣವನ್ನು ಗಳಿಸಬೇಕೆಂಬ ಪ್ರತ್ಯಾಶೆಯಿಂದ ಪರೀಕ್ಷಿತನ ಬಳಿಗೆ ಹೊರಟ. ಕೊಲ್ಲು ವವನೂ ಬದುಕಿಸುವವನೂ ಇಬ್ಬರೂ ಹಾದಿಯಲ್ಲಿ ಸಂಧಿಸಿದರು. ಪರಸ್ಪರ ಪರಿಚಯ ವಾದ ಮೇಲೆ ತಮ್ಮ ತಮ್ಮ ಶಕ್ತಿಪರೀಕ್ಷೆಯನ್ನು ನಡೆಸಲು ಅವರು ನಿರ್ಧರಿಸಿದರು. ತಕ್ಷಕನು ಹಚ್ಚ ಹಸುರಾಗಿದ್ದ ಒಂದು ದೊಡ್ಡ ಮರವನ್ನು ಕಚ್ಚಲು, ಅದು ಸುಟ್ಟು ಬೂದಿ ಯಾಯಿತು. ಒಡನೆಯೆ ಕಾಶ್ಯಪನ ತನ್ನ ಮಂತ್ರಶಕ್ತಿಯನ್ನು ಪ್ರಯೋಗಿಸಿ, ಆ ಮರ ಮತ್ತೆ ಮೊದಲಿದ್ದಂತೆ ಹಚ್ಚಹಸುರಾಗಿ ನಿಲ್ಲುವಹಾಗೆ ಮಾಡಿದನು. ಆತನ ಶಕ್ತಿಯನ್ನು ಕಂಡು ಬೆರಗಾದ ತಕ್ಷಕನು ‘ಅಯ್ಯಾ, ಬ್ರಾಹ್ಮಣಶಾಪ ಸುಳ್ಳಾಗಿಹೋದರೆ ಗತಿಯೇನು? ನಿನಗೆ ಬೇಕಾದಷ್ಟು ಹಣವನ್ನು ಕೊಡುತ್ತೇನೆ, ಹೊರಟುಹೋಗು’ ಎಂದು ಹೇಳಿ, ಅವನ ಆಶೆ ಹಿಂಗುವಷ್ಟು ಹಣವನ್ನು ತೆತ್ತು ಅವನನ್ನು ಹಿಂದಕ್ಕೆ ಕಳುಹಿಸಿದನು. ಅನಂತರ ತಕ್ಷಕನು ಬ್ರಾಹ್ಮಣವೇಷವನ್ನು ಧರಿಸಿ, ಪರೀಕ್ಷಿತನ ಬಳಿಗೆ ಹೋಗಿ, ಆತನನ್ನು ಕಚ್ಚಿದನು. ಒಡನೆಯೆ ಆ ರಾಜನ ದೇಹ ಸುಟ್ಟು ಬೂದಿಯಾಗಿಹೋಯಿತು. ಆದರೆ ಆ ವೇಳೆಗೆ ರಾಜನು ಮುಕ್ತಾವಸ್ಥೆಯಲ್ಲಿದ್ದುದರಿಂದ, ನಿರ್ಜೀವವಾದ ದೇಹಮಾತ್ರ ನಾಶವಾದಂತಾಯಿತು. ಪರೀಕ್ಷಿದ್ರಾಜನು ಸಾಧಿಸಿದ ಯೋಗಸಿದ್ಧಿಯನ್ನು ಕಂಡು ದೇವಾನುದೇವತೆಗಳೆಲ್ಲ ಆತ ನನ್ನು ಮುಕ್ತಕಂಠರಾಗಿ ಹೊಗಳಿ ಹೂಮಳೆಯನ್ನು ಕರೆದರು, ಗಂಧರ್ವರು ಗಾನಮಾಡಿ ದರು, ದೇವದುಂದುಭಿಗಳು ಮೊಳಗಿದವು.

ಪರೀಕ್ಷಿದ್ರಾಜನ ಮಗನಾದ ಜನಮೇಜಯ ಮಹಾರಾಜನಿಗೆ ತಕ್ಷಕನು ಕಚ್ಚಿ ತನ್ನ ತಂದೆ ಸತ್ತನೆಂಬ ಸುದ್ದಿ ಮುಟ್ಟಿತು. ಅದನ್ನು ಕೇಳಿ ದುಃಖ ಕ್ರೋಧಗಳಿಂದ ಮೈಮರೆತ ಜನಮೇ ಜಯನು ಹಾವುಗಳ ಜಾತಿಯೆ ಭೂಮಿಯ ಮೇಲಿರದಂತೆ ಮಾಡುವೆನೆಂದು ನಿಶ್ಚಯಿಸಿ, ಬ್ರಾಹ್ಮಣರ ಸಹಾಯದಿಂದ ಘೋರವಾದ ಸರ್ಪಯಾಗವನ್ನು ಕೈಕೊಂಡನು. ಮಂತ್ರ ವನ್ನು ಹೇಳಿ ಆಹ್ವಾನಮಾಡಿದೊಡನೆಯೆ, ಹಾಗೆ ಆಹ್ವಾನಿಸಿದ ಮಹಾಸರ್ಪಗಳು ಬಂದು, ಅಗ್ನಿಯಲ್ಲಿ ಬಿದ್ದು ದಗ್ಧವಾಗಿ ಹೋಗುತ್ತಿದ್ದವು. ಇದನ್ನು ಕಂಡು ತಕ್ಷಕನು ಪ್ರಾಣಭಯ ದಿಂದ ಓಡಿಹೋಗಿ ದೇವೇಂದ್ರನ ಮೊರೆಹೊಕ್ಕನು. ಬ್ರಾಹ್ಮಣರು ‘ತಕ್ಷಕಾಯ ಸ್ವಾಹಾ’ ಎಂದರು. ತಕ್ಷಕ ಬರಲಿಲ್ಲ. ಆಗ ಜನಮೇಜಯನು ‘ಬ್ರಾಹ್ಮಣರೆ, ಸರ್ಪಾಧಮನಾದ ತಕ್ಷಕನೇಕಿನ್ನೂ ಹೋಮಾಗ್ನಿಯಲ್ಲಿ ಬಿದ್ದು ದಗ್ಧನಾಗಲಿಲ್ಲ? ಅವನಲ್ಲವೆ ನನ್ನ ತಂದೆ ಯನ್ನು ಕಚ್ಚಿದವನು? ಅವನು ಸುಟ್ಟು ಬೂದಿಯಾಗದ ಹೊರತು ನನ್ನ ಜೀವಕ್ಕೆ ಶಾಂತಿ ಯಿಲ್ಲ’ ಎಂದ. ಬ್ರಾಹ್ಮಣರು ಹೇಳಿದರು–‘ಮಹಾರಾಜ, ಆ ತಕ್ಷಕನು ದೇವೇಂದ್ರನನ್ನು ಮೊರೆಹೊಕ್ಕಿದ್ದಾನೆ. ಆತನ ರಕ್ಷಣೆಯಲ್ಲಿರುವುದರಿಂದಲೆ ಅವನು ಬರುತ್ತಿಲ್ಲ.’ ಇದನ್ನು ಕೇಳಿ ಕಿಡಿಕಿಡಿಯಾದ ಜನಮೇ ಜಯನು ‘ಹಾಗಾದರೆ, ಆ ತಕ್ಷಕನು ದೇವೇಂದ್ರಸಹಿತನಾಗಿ ಬಂದು ಈ ಅಗ್ನಿಕುಂಡದಲ್ಲಿ ಬೀಳುವಂತೆ ಮಾಡಬಾರದೇಕೆ?’ ಎಂದು ಗುಡುಗಿದನು. ಬ್ರಾಹ್ಮಣರು ಹಾಗೆ ಆಗಲೆಂದು, ‘ಸಹೇಂದ್ರಾಯ ತಕ್ಷಕಾಯ ಸ್ವಾಹಾ’–ಎಂದು ಹೋಮ ಮಾಡಿದರು. ತಕ್ಷಣವೇ ದೇವೇಂದ್ರನು ಸಿಂಹಾಸನ ಸಹಿತನಾಗಿ ತಕ್ಷಕನೊಡನೆ ಕೆಳಕ್ಕೆ ಉರುಳಿ ಬೀಳಲು ಪ್ರಾರಂಭವಾಯಿತು. ಇದನ್ನು ಕಂಡು ದೇವ ಗುರುವಾದ ಬೃಹ ಸ್ಪತಿಯು ಜನಮೇಜಯನ ಬಳಿಯಲ್ಲಿ ಕಾಣಿಸಿಕೊಂಡು ‘ಮಹಾರಾಜ, ಈ ತಕ್ಷಕ ಅಮೃತ ಪಾನ ಮಾಡಿದವನಾದುದರಿಂದ ಇವನಿಗೆ ಜರಾಮರಣಗಳಿಲ್ಲ. ನೀನು ಅವನ ಮೇಲಿನ ಕ್ರೋಧವನ್ನು ಬಿಟ್ಟುಬಿಡು. ಹುಟ್ಟು ಸಾವುಗಳು ಅವನವನ ಕರ್ಮಾಧೀನ. ಆದ್ದರಿಂದ ನಿನ್ನ ತಂದೆಯ ಸಾವಿಗೆ ಈ ತಕ್ಷಕನೆ ಕಾರಣನೆಂದೇಕೆ ಭ್ರಮಿಸುತ್ತಿರುವೆ? ನೀನು ಆರಂಭಿಸಿರುವ ಈ ಭಯಂಕರ ಯಾಗವನ್ನು ಇಲ್ಲಿಗೆ ನಿಲ್ಲಿಸು. ನೀನು ಮಾಡುತ್ತಿರುವುದು ಅಭಿಚಾರಿಕ(ಮಾಟ) ಕರ್ಮ, ಧರ್ಮ ಸಮ್ಮತವಾದ ಯಾಗವಲ್ಲ. ಅನ್ಯಾಯವಾಗಿ ನಿರಪರಾಧಿಗಳಾದ ಸರ್ಪಗಳನ್ನು ಏಕೆ ಕೊಲ್ಲುತ್ತಿ? ಇದು ಪಾಪಕರ’ ಎಂದನು. ಜನಮೇ ಜಯನು ದೇವಗುರುವಿನ ಅಪ್ಪಣೆಯಂತೆ ಆ ಯಾಗವನ್ನು ಅಲ್ಲಿಗೆ ನಿಲ್ಲಿಸಿದನು. 

