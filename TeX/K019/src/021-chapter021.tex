
\chapter{೨೧. ನಾಭಿರಾಜ}

ಅಗ್ನೀಧ್ರನ ತರುವಾಯ ಅವನ ಒಂಬತ್ತು ಜನ ಮಕ್ಕಳು ಜಂಬೂದ್ವೀಪದ ಬೇರೆ ಬೇರೆ ಭಾಗಗಳಲ್ಲಿ ರಾಜ್ಯವಾಳುತ್ತಿದ್ದರು. ಅವರವರು ಆಳುತ್ತಿದ್ದ ರಾಜ್ಯಗಳಿಗೆ ಅವರವರ ಹೆಸರೇ ನಿಂತಿತು. ಅವರು ಮೇರುವಿನ ಕುಮಾರಿಯರನ್ನು ಅನುಕ್ರಮವಾಗಿ ಮದುವೆಯಾದರು\footnote{೧. ಮೇರುದೇವಿ, ಪ್ರತಿರೂಪೆ, ಉದಗ್ರದಂಷ್ಟ್ರಿ, ಲತೆ, ರಮ್ಮೆ, ಶಾಮೆ, ನಾರಿ, ಭದ್ರೆ, ದೇವವೀತಿ.}. ಹಿರಿಯಮಗನಾದ ನಾಭಿರಾಜನು ತನ್ನ ಮಡದಿ ಮೇರುದೇವಿಯೊಡನೆ ಸುಖಭೋಗ ಗಳನ್ನು ಅನುಭವಿಸುತ್ತಾ ಹಲವುಕಾಲ ರಾಜ್ಯಭಾರಮಾಡಿದ ಮೇಲೆ, ಮಕ್ಕಳನ್ನು ಪಡೆಯ ಬೇಕೆಂದು ಆಶೆಯಿಂದ ಯಜ್ಞವನ್ನು ಕೈಕೊಂಡು, ದೇವದೇವನಾದ ಮಹಾವಿಷ್ಣುವನ್ನು ಭಕ್ತಿಯಿಂದ ಪೂಜಿಸಿದನು. ಆತನ ಭಕ್ತಿಗೆ ಭಗವಂತನು ಪ್ರತ್ಯಕ್ಷನಾದನು. ನಾಭಿರಾಜನು ಆತನ ಮುಂದೆ ಅಡ್ಡ ಬಿದ್ದು, ಆತನನ್ನು ಪರಿಪರಿಯಾಗಿ ಸ್ತುತಿಸಿದ ಮೇಲೆ ‘ನಿನಗೆ ಸಮಾನ ನಾದ ಮಗನೊಬ್ಬನನ್ನು ನನಗೆ ಕರುಣಿಸು’ ಎಂದು ವರವನ್ನು ಬೇಡಿದನು. ಭಗವಂತನು ರಾಣಿಯಾದ ಮೇರುದೇವಿಯ ಹೊಟ್ಟೆಯಲ್ಲಿ ತಾನೆ ತನ್ನ ಅಂಶದಿಂದ ಹುಟ್ಟುವುದಾಗಿ ಹೇಳಿ, ಅಂತರ್ಧಾನನಾದನು.

ಕೆಲವು ಕಾಲದ ಮೇಲೆ, ಭಗವಂತನು ತಾನು ಕೊಟ್ಟ ಮಾತಿನಂತೆ ನಾಭಿರಾಜನ ಮಗನಾಗಿ ಹುಟ್ಟಿದನು. ಹುಟ್ಟುವಾಗಲೆ ಆತನಲ್ಲಿ ಭಗವಂತನ ಲಕ್ಷಣಗಳೆಲ್ಲ ಕಾಣಬರು ತ್ತಿದ್ದವು. ಮಗು ಬೆಳೆಯುತ್ತಾ ಹೋದಂತೆ ಆತನಲ್ಲಿ ಸಮಬುದ್ಧಿ, ವೈರಾಗ್ಯಗಳೂ ಬೆಳೆ ಯುತ್ತಾ ಹೋದವು, ಆತನ ಸದ್ಗುಣಗಳನ್ನು ಕಂಡು, ಪ್ರಜೆಗಳೆಲ್ಲ ಆತನೆ ರಾಜ ನಾಗಲೆಂದು ಹಾರೈಸುತ್ತಿದ್ದರು. ಆತನ ದೇಹಶಕ್ತಿ, ವಿರಕ್ತಿ, ಕೀರ್ತಿಗಳಿಗೆ ತಕ್ಕಂತೆ ಆತನಿಗೆ ಋಷಭನೆಂದು ನಾಮಕರಣವಾಯಿತು. ಆತನ ಕೀರ್ತಿಯನ್ನು ಕಂಡು ಅಸೂಯೆಯಿಂದ ದೇವೇಂದ್ರನು ಆತನ ದೇಶದಲ್ಲಿ ಮಳೆ ಬರುವುದನ್ನೆ ನಿಲ್ಲಿಸಿದನಂತೆ! ಋಷಭದೇವನು ಅವನ ಮರುಳಾಟಕ್ಕೆ ನಕ್ಕು, ತನ್ನ ಯೋಗಪ್ರಭಾವದಿಂದ ಮಳೆಯನ್ನು ಸುರಿಸಿದನಂತೆ! ಆದರೆ ನಾಭಿರಾಜನ ಪುತ್ರಮೋಹ ಋಷಭನ ಮಹತ್ತನ್ನು ಅರಿತುಕೊಳ್ಳಲಾರದೆ ಹೋಗಿತ್ತು. ಅವನನ್ನು ತನ್ನ ಮಗನೆಂಬ ಮಮತೆಯಿಂದ ಗದ್ಗದ ಕಂಠದಿಂದ ‘ಅಪ್ಪ, ಮಗು’ ಎಂದು ಆಗಾಗ ಕೂಗಿ ಸಂತೋಷಗೊಳ್ಳುತ್ತಿದ್ದ, ಆತ.

ಜನರೆಲ್ಲರೂ ಋಷಭನೇ ರಾಜನಾಗಲೆಂದು ಬಯಸುತ್ತಿದ್ದುದು ನಾಭಿರಾಜನಿಗೆ ಅರ್ಥ ವಾಯಿತು. ಆತನು ಮಗನಿಗೆ ಪಟ್ಟ ಕಟ್ಟಿ, ಆತನ ರಕ್ಷಣೆಯನ್ನು ಸಾಧುಗಳಾದ ಬ್ರಾಹ್ಮಣರಿ ಗೊಪ್ಪಿಸಿ, ತಾನು ತನ್ನ ಮಡದಿಯೊಡನೆ ಬದರಿಕಾಶ್ರಮಕ್ಕೆ ಹೋದನು. ಅಲ್ಲಿ ಭಗವಂತ ನನ್ನು ಭಕ್ತಿಯೋಗದಿಂದ ಆರಾಧಿಸಿ, ಮುಕ್ತಿಪದವಿಯನ್ನು ಪಡೆದನು.

