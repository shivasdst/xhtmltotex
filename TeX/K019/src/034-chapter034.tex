
\chapter{೩೪. ಶ್ರೀಹರಿಯು ಮೀನಾದನೇಕೆ?}

ಬಲೀಂದ್ರನ ಕಥೆಯನ್ನು ಕೇಳಿ ಸಂತೋಷಗೊಂಡು ಪರೀಕ್ಷಿದ್ರಾಜನು ಶುಕಮುನಿ ಯನ್ನು ಕುರಿತು ‘ಸ್ವಾಮಿ, ಶ್ರೀಹರಿಯು ಹಲವು ಅವತಾರಗಳನ್ನು ಎತ್ತಿದನೆಂದು ಹೇಳು ತ್ತಾರೆ. ಆ ಅವತಾರಗಳ ಸಾಲಿನಲ್ಲಿ ಆತನು ಮೀನಾಗಿ ಹುಟ್ಟಿದುದೂ ಒಂದಂತೆ! ಪ್ರಾಣಿ ಗಳಲ್ಲೆಲ್ಲ ಅತ್ಯಂತ ಕೀಳಾದ ಮೀನಿನ ಅವತಾರವನ್ನು ಆತ ಏಕೆ ಎತ್ತಿದ?’ ಎಂದು ಪ್ರಶ್ನಿಸಿದನು. ಆಗ ಶುಕಮುನಿಯು ಗಹಗಹಿಸಿ ನಗುತ್ತಾ ‘ಅಯ್ಯಾ, ನಿನಗೆ ಕೀಳಾಗಿದ್ದ ಮಾತ್ರಕ್ಕೆ ಭಗವಂತನಿಗೂ ಅದು ಕೀಳೆಂದು ಹೇಳಬಹುದೆ? ಜ್ಞಾನಿಗಳ ದೃಷ್ಟಿಯಲ್ಲಿ ನೀನೂ ಒಂದೆ, ಮೀನೂ ಒಂದೆ. ಸಾಧುಸಜ್ಜನರಿಗೆ ಸಂಕಟವೊದಗಿದಾಗ ಧರ್ಮಕ್ಕೆ ಲೋಪ ಬಂದಾಗ ಭಗವಂತನು ಆಯಾ ಸಂದರ್ಭಕ್ಕೆ ಅಗತ್ಯವಾದಂತಹ ರೂಪವನ್ನು ಧರಿಸಿ ಅವತರಿಸುತ್ತಾನೆ. ಅದರಂತೆ, ಆತನು ಅಗತ್ಯವೆಂದು ತೋರಿಬಂದಾಗ ಮತ್ಸ್ಯಾವತಾರ ವನ್ನೂ ತಾಳಿದನು. ಅದರ ಕಥೆಯನ್ನು ಹೇಳುತ್ತೇನೆ ಕೇಳು. 

ಹಿಂದೆ, ಬಹು ಹಿಂದೆ, ಈ ಕಲ್ಪಕ್ಕೂ ಹಿಂದೆ, ಹಿಂದಿನ ಕಲ್ಪ ಕೊನೆಯಾಗುವ ಮುನ್ನ, ಬ್ರಹ್ಮನ ಒಂದು ಹಗಲು ಮುಗಿದು ರಾತ್ರಿಯಾದಾಗ, ಮಹಾಪ್ರಳಯವೊಂದು ಸಂಭವಿ ಸಿತು. ಬ್ರಹ್ಮನು ಗಾಢನಿದ್ರೆಯಲ್ಲಿ ಮುಳುಗಿಹೋದನು. ಮಲಗಿದ್ದ ಬ್ರಹ್ಮನು ಉಸಿ ರಾಡುತ್ತಿರುವಾಗ ಆತನ ಉಸಿರಿನಿಂದ ಹೊರಬಂದ ವೇದಗಳನ್ನು ಹಯಗ್ರೀವನೆಂಬ ರಾಕ್ಷಸನು ಹೊತ್ತುಕೊಂಡು ಹೋದನು. ಆ ರಕ್ಕಸನನ್ನು ಕೊಂದು ವೇದಗಳನ್ನು ಉದ್ಧರಿಸುವುದಕ್ಕಾಗಿ ಶ್ರೀಹರಿಯು ಮೀನಿನ ಅವತಾರವನ್ನು ಎತ್ತಬೇಕಾಯಿತು. ಆ ವೇಳೆಗೆ ಸರಿಯಾಗಿ ಸತ್ಯವ್ರತನೆಂಬ ರಾಜಪುಷಿಯೊಬ್ಬನು ಶ್ರೀಹರಿಯನ್ನು ಕುರಿತು ತಪಸ್ಸು ಮಾಡುತ್ತಿದ್ದನು. ಆತನು ನೀರಿನಿಂದ ತರ್ಪಣ ಕೊಡುತ್ತಿದ್ದಾಗ ಆತನ ಬೊಗಸೆಯಲ್ಲಿದ್ದ ನೀರಿನಲ್ಲಿ ಸಣ್ಣದೊಂದು ಮೀನು ಕಾಣಿಸಿತು. ಆತನು ಅದನ್ನು ನೀರೊಳಕ್ಕೆಸೆದನು. ಅದು ರಾಜನನ್ನು ಕುರಿತು ‘ಅಯ್ಯಾ, ಸಣ್ಣ ಮೀನನ್ನು ಕಂಡರೆ ದೊಡ್ಡಮೀನು ನುಂಗಿಹಾಕುತ್ತದೆ. ಇದು ನಿನಗೆ ಗೊತ್ತಿಲ್ಲದ ವಿಷಯವೇನೂ ಅಲ್ಲ. ಆದ್ದರಿಂದ ನನ್ನ ಜೀವಭಯ ನೀಗುವಂತೆ ನನ್ನನ್ನು ಈ ಜಲರಾಶಿಯಿಂದ ಹೊರಕ್ಕೊಯ್ದು ಕಾಪಾಡು, ಎಂದು ಬೇಡಿ ಕೊಂಡಿತು. ಕರುಣಾಳುವಾದ ಆ ರಾಜಪುಷಿ ಅದನ್ನು ತನ್ನ ಕಲಶದಲ್ಲಿ ಇಟ್ಟುಕೊಂಡು ತನ್ನ ಆಶ್ರಮಕ್ಕೆ ಕೊಂಡೊಯ್ದನು.

ಸತ್ಯವ್ರತನು ಕೊಂಡೊಯ್ದ ಮೀನು ಸಾಮಾನ್ಯವಾದ ಮೀನಲ್ಲ. ಮರುದಿನ ಬೆಳ ಗಾಗುವ ಹೊತ್ತಿಗೆ ಅದು ರಾಜನ ಕಲಶದ ತುಂಬ ಬೆಳೆದಿತ್ತು. ಇನ್ನು ಅದರೊಳಗಿರುವುದು ಅದಕ್ಕೆ ಸಾಧ್ಯವಿರಲಿಲ್ಲ. ಆದ್ದರಿಂದ ‘ಪುಣ್ಯಾತ್ಮ, ನಾನು ಸುಖವಾಗಿ ವಾಸಮಾಡಲು ಸಾಧ್ಯವಾಗುವಂತಹ ವಿಸ್ತಾರವಾದ ಒಂದು ನೀರಿನ ಸ್ಥಳದಲ್ಲಿ ನನ್ನನ್ನು ಇಡು’ ಎಂದು ಸತ್ಯವ್ರತನನ್ನು ಬೇಡಿಕೊಂಡಿತು. ಆತನು ಅದನ್ನು ಎತ್ತಿ ದೊಡ್ಡದೊಂದು ನೀರಿನ ಹಂಡೆಯಲ್ಲಿ ಇಟ್ಟ. ಅದು ಕ್ಷಣಮಾತ್ರದಲ್ಲಿ ಹಂಡೆಯ ತುಂಬ ಬೆಳೆಯಿತು. ಅದರ ಅಪೇಕ್ಷೆಯಂತೆ ಅದನ್ನು ಒಂದು ಬಾವಿಯಲ್ಲಿ ಇಟ್ಟುದಾಯಿತು. ಮರುನಿಮಿಷ ಅದು ಸಾಲದೆ, ಆ ಮೀನನ್ನು ನದಿಯಲ್ಲಿ, ಅನಂತರ ಒಂದು ಸರೋವರದಲ್ಲಿ ಇಟ್ಟು, ಅದರ ಗಾತ್ರಕ್ಕೆ ಆ ನದಿ, ಸರೋವರಗಳೂ ಸಾಲದೆ ಹೋದುದರಿಂದ ಅದನ್ನು ಸಮುದ್ರಕ್ಕೆ ಒಯ್ದು ಮುಟ್ಟಿಸಿದುದಾಯಿತು. ಅದು ಅಲ್ಲಿಯೂ ಕೊನೆಮೊದಲಿಲ್ಲದೆ ಬೆಳೆಯುತ್ತಾ ಇತ್ತು. ಅದನ್ನು ಕಂಡು ಸತ್ಯವ್ರತನು ಅತ್ಯಂತ ಆಶ್ಚರ್ಯದಿಂದ ‘ಹೇ ಮೀನಿನ ರೂಪದಲ್ಲಿರುವ ಮಹಾನುಭಾವ, ನೀನು ಯಾರು? ಹಿಂದೆ ಎಂದೂ ಕಂಡು ಕೇಳದ ನಿನ್ನ ಪವಾಡವನ್ನು ನೋಡಿ ನಾನು ಮರುಳಾಗಿದ್ದೇನೆ. ಸೃಷ್ಟಿ ಸ್ಥಿತಿ ಲಯಗಳಿಗೆ ಕಾರಣನಾದ ಭಗವಂತನೇ ನೀನೆಂದು ನನಗನಿಸುತ್ತಿದೆ. ಮೀನಿನ ರೂಪದಿಂದ ಕಾಣಿಸಿಕೊಂಡು ನನ್ನನ್ನು ಉದ್ಧರಿಸಹೊರಟಿರುವೆಯೋ ಏನೊ!’ ಎಂದನು.

ಸತ್ಯವ್ರತನ ನುಡಿಗಳಿಂದ ಸಂತಸಗೊಂಡ ಆ ಮೀನು ‘ಅಯ್ಯಾ, ಭಕ್ತ, ಇಂದಿಗೆ ಏಳು ದಿನಕ್ಕೆ ಮೂರು ಲೋಕಗಳೂ ಪ್ರಳಯದ ನೀರಿನಲ್ಲಿ ಮುಳುಗಿಹೋಗುತ್ತವೆ. ಆಗ ಒಂದು ದೊಡ್ಡ ಹಡಗು ನಿನ್ನ ಬಳಿಗೆ ಬರುತ್ತದೆ. ನೀನು ಜಗತ್ತಿನಲ್ಲಿರುವ ಎಲ್ಲ ಪ್ರಾಣಿವರ್ಗ ಗಳನ್ನೂ ಸಸ್ಯಗಳ ಬೀಜಗಳನ್ನೂ ಸಿದ್ಧವಾಗಿ ಇಟ್ಟುಕೊಂಡಿದ್ದು, ಆ ಹಡಗು ನಿನ್ನ ಸಮೀಪಕ್ಕೆ ಬರುತ್ತಲೆ ಅವನ್ನೆಲ್ಲ ಅದರಲ್ಲಿ ತುಂಬಿ, ಸಪ್ತಪುಷಿಗಳೊಡನೆ ನೀನೂ ಅದರಲ್ಲಿ ಕುಳಿತು ಪ್ರಯಾಣಮಾಡು. ಆಗ ಕಗ್ಗತ್ತಲೆ ಕವಿದಿದ್ದರೂ ಸಪ್ತಪುಷಿಗಳ ದೇಹಕಾಂತಿಯಿಂದ ನೀನು ನಿರಾತಂಕವಾಗಿ ಸಂಚರಿಸಬಹುದು. ಪ್ರಳಯದ ಬಿರುಗಾಳಿಗೆ ಸಿಕ್ಕಿ ಹಡಗು ತಲೆಕೆಳಕಾಗುವಂತಾಗುತ್ತದೆ. ಆದರೂ ನೀನು ಭಯಪಡಬೇಡ. ಆ ವೇಳೆಗೆ ಸರಿಯಾಗಿ ನಾನು ಅಲ್ಲಿ ಕಾಣಿಸಿಕೊಳ್ಳುತ್ತೇನೆ. ಆಗ ನೀನು ಹಡಗನ್ನು ಒಂದು ಹೆಬ್ಬಾವಿನಿಂದ ನನ್ನ ಮುಖದ ಮೇಲಿನ ಕೊಂಬಿಗೆ ಕಟ್ಟಿಹಾಕು. ಹಾಗೆ ಮಾಡಿದರೆ ಹಡಗು ಅಲುಗಾಡದೆ ನಿಲ್ಲುತ್ತದೆ. ನಾನು ಅದನ್ನು ಬ್ರಹ್ಮನು ನಿದ್ದೆಯಿಂದ ಮೇಲಕ್ಕೇಳುವವರೆಗೆ ಸಮುದ್ರದಲ್ಲಿ ಸುತ್ತಿಸುತ್ತಿರುತ್ತೇನೆ. ಆ ಸಮಯದಲ್ಲಿ ನಾನು ನನ್ನ ಸ್ವರೂಪವನ್ನೂ ಮಹಿಮೆಯನ್ನೂ ನಿನಗೆ ವಿಶದವಾಗಿ ಉಪದೇಶಿಸುತ್ತೇನೆ’ ಎಂದು ಹೇಳಿ, ಆ ಮೀನು ಮಾಯವಾಗಿಹೋಯಿತು.

ಮೀನುರೂಪಿಯಾದ ಭಗವಂತನು ಹೇಳಿದ ಕಾಲವನ್ನು ನಿರೀಕ್ಷಿಸುತ್ತಾ ಸತ್ಯವ್ರತನು ಆ ಭಗವಂತನ ಧ್ಯಾನದಲ್ಲಿಯೇ ಮಗ್ನನಾಗಿ ಕುಳಿತಿರಲು, ಏಳು ದಿನಗಳಿಗೆ ಸರಿಯಾಗಿ ಭಯಂಕರವಾದ ಮಳೆ ಬಂದು, ಜಗತ್ತೆಲ್ಲವೂ ಒಂದು ದೊಡ್ಡ ಸಮುದ್ರವಾಗಿ ಹೋಯಿತು. ಅಷ್ಟರಲ್ಲಿ ದೊಡ್ಡ ದೊಂದು ಹಡಗು ಸತ್ಯವ್ರತನ ಬಳಿಗೆ ಬಂದಿತು. ಆತನು ತಾನು ಸಂಗ್ರಹಿಸಿಟ್ಟುಕೊಂಡಿದ್ದ ಸಸ್ಯ ಬೀಜಗಳನ್ನು ಹಡಗಿಗೆ ತುಂಬಿ, ಸಪ್ತರ್ಷಿಗಳೊಡನೆ ತಾನೂ ಹಡಗನ್ನೇರಿದನು. ಆ ವೇಳೆಗೆ ಸರಿಯಾಗಿ ಮೀನಿನಾಕಾರದ ಜಗದೀಶ್ವರನು ಅಲ್ಲಿ ಕಾಣಿಸಿಕೊಂಡನು. ಆ ಮಹಾಮೀನಿಗೆ ಖಡ್ಗಮೃಗದಂತೆ ಮೂತಿಯಮೇಲೆ ದೊಡ್ಡ ದೊಂದು ಕೊಂಬಿತ್ತು. ಸತ್ಯವ್ರತನು ಅದಕ್ಕೆ ಹಡಗನ್ನು ಭದ್ರವಾಗಿ ಕಟ್ಟಿ, ನಿರಾತಂಕವಾದ ಮನಸ್ಸಿನಿಂದ ಭಗವಂತನನ್ನು ಪರಿಪರಿಯಾಗಿ ಸ್ತೋತ್ರಮಾಡಿದನು. ಸುಪ್ರೀತನಾದ ಭಗವಂತನು ಆತನಿಗೆ ಆತ್ಮತತ್ವವನ್ನು ವಿವರಿಸಿ ಹೇಳಿದನು. ಬ್ರಹ್ಮನು ನಿದ್ರೆ ತಿಳಿದು ಮೇಲಕ್ಕೆ ಏಳುವವರೆಗೂ ಸತ್ಯವ್ರತನು ಮಹಾಮತ್ಸ್ಯದ ಸಹಾಯದಿಂದ ಸಮುದ್ರದಲ್ಲಿ ವಿಹರಿಸುತ್ತಲೆ ಇದ್ದನು. ಪ್ರಳಯದ ಕೊನೆಯಲ್ಲಿ ಬ್ರಹ್ಮನು ನಿದ್ರೆಯಿಂದ ಮೇಲಕ್ಕೆದ್ದನು. ಒಡನೆಯೆ ಮತ್ಸ್ಯಮೂರ್ತಿಯು ಹಯಗ್ರೀವ ರಾಕ್ಷಸನನ್ನು ಕೊಂದು ವೇದಗಳನ್ನು ಬ್ರಹ್ಮ ನಿಗೆ ಕೊಟ್ಟನು. ಆ ದೇವದೇವನ ಅನುಗ್ರಹದಿಂದ ಸತ್ಯವ್ರತನು ಆ ಕಲ್ಪದ ಮನುವಾದನು.

