
\chapter{೯. ಕಪಿಲಮುನಿ}

ಕರ್ದಮ ಪ್ರಜಾಪತಿಯು ವಿರಕ್ತನಾಗಿ ಹೊರಟುಹೋದ ಮೇಲೆ ದೇವಹೂತಿಗೂ ವೈರಾಗ್ಯ ಉದಿಸಿತು. ಆಕೆ ಪುರಾಣಪುರುಷನ ಅವತಾರಿಯಾದ ತನ್ನ ಮಗನ ಬಳಿಗೆ ಬಂದು “ಮಗು, ನೀನು ಪುತ್ರರೂಪಿಯಾದ ಪರಮಾತ್ಮನೆಂದು ನಾನು ಬಲ್ಲೆ. ಅಜ್ಞಾನದ ಕತ್ತಲೆ ಯಲ್ಲಿ ಬಿದ್ದುಹೋದವರನ್ನು ಉದ್ಧರಿಸುವುದಕ್ಕಾಗಿಯೇ ನೀನು ನನ್ನ ಗರ್ಭದಿಂದ ಅವತರಿಸಿರುವೆ. ನಾನು ಈ ತನಕ ಕೇವಲ ಇಂದ್ರಿಯಸುಖಗಳಲ್ಲಿಯೇ ಮಗ್ನಳಾಗಿದ್ದೆ. ‘ನಾನು, ನನ್ನದು’ ಎಂಬ ಅಹಂಕಾರ ಮಮಕಾರಗಳು ನನ್ನನ್ನು ಇನ್ನೂ ಬಾಧಿಸುತ್ತಿವೆ. ಈ ಮಾಯೆಗೆ ಮೂಲಕಾರಣನಾದ ನೀನೆ ಇವುಗಳನ್ನು ನಿವಾರಿಸಬೇಕು. ನಾನು ನಿನ್ನ ಮರೆ ಹೊಕ್ಕಿದ್ದೇನೆ. ಶರಣಾಗತರಾದ ಭಕ್ತರನ್ನು ಸಂಸಾರದಿಂದ ಉದ್ಧರಿಸಬಲ್ಲ ನೀನು ನನಗೆ ಜ್ಞಾನೋಪದೇಶವನ್ನು ಮಾಡಿ ಮೋಕ್ಷಮಾರ್ಗವನ್ನು ತೋರಿಸು” ಎಂದು ಬೇಡಿದಳು.

ತಾಯಿಯ ಪ್ರಾರ್ಥನೆಯನ್ನು ಕೇಳಿ ಕಪಿಲಮುನಿಗೆ ಪರಮಾನಂದವಾಯಿತು. ಆತನು ಆಕೆಯೊಡನೆ ಹೇಳಿದ “ಅಮ್ಮ ಸಂಸಾರಬಂಧನ, ಮೋಕ್ಷ ಇವೆರಡಕ್ಕೂ ನಮ್ಮ ಮನಸ್ಸೇ ಕಾರಣ. ಮನಸ್ಸು ವಿಷಯಸುಖಗಳಿಗೆ ಆಶೆಪಟ್ಟರೆ ಬಂಧನ, ಆಶೆಯನ್ನು ತೊರೆದರೆ ಮೋಕ್ಷ. ‘ಈ ದೇಹ ನಾನು’ ಎಂದು ವ್ಯವಹರಿಸುವುದು ಅಹಂಕಾರ, ‘ಈ ವಸ್ತುಗಳು ನನ್ನವು’ ಎನ್ನುವುದು ಮಮಕಾರ; ಈ ಅಹಂಕಾರ ಮಮಕಾರಗಳಿಂದ ನಾವು ಕಾಮ ಕ್ರೋಧಾದಿ ಅರಿಷಡ್ವರ್ಗಗಳಿಗೆ ಒಳಗಾಗುತ್ತೇವೆ. ಅವುಗಳಿಗೆ ಒಳಗಾಗಿ ನಾವು ಮಾಡುವ ಕಾರ್ಯಗಳಿಂದ ಪಾಪ-ಪುಣ್ಯಗಳು ನಮ್ಮಮನಸ್ಸಿಗೆ ಮೆತ್ತಿಕೊಳ್ಳುತ್ತವೆ. ನಾವು ನಮ್ಮ ಮನಸ್ಸನ್ನು ಆ ಅರಿಷಡ್ವರ್ಗಗಳಿಗೆ ವಶವಾಗದಂತೆ ನೋಡಿಕೊಂಡರೆ ಮನಸ್ಸು ಪರಿಶುದ್ಧ ವಾಗುತ್ತದೆ. ಮನಸ್ಸು ಪ್ರಶಾಂತವಾದ ಸಮುದ್ರದಂತಾಗಬೇಕು, ಅದರಲ್ಲಿ ಸುಖ ದುಃಖ ಗಳ ಅಲೆಗಳು ಏಳಬಾರದು. ಆ ಪ್ರಶಾಂತವಾದ ಮನಸ್ಸಿನಲ್ಲಿ ಭಕ್ತಿ, ಜ್ಞಾನ, ವೈರಾಗ್ಯಗಳು ನೆಲಸಿದರೆ ಆತ್ಮ ಸಾಕ್ಷಾತ್ಕಾರವಾಗುತ್ತದೆ. ಅದೇ ಮೋಕ್ಷ. 

“ತಾಯಿ, ಆತ್ಮಸ್ವರೂಪ ಅರ್ಥವಾಗುವುದು ಸುಲಭವಲ್ಲ. ಅದು ಇಂದ್ರಿಯಗೋಚರ ವಾದುದಲ್ಲ. ಶರೀರ, ಇಂದ್ರಿಯ, ಮನಸ್ಸು, ಪ್ರಾಣ ಮೊದಲಾದವುಗಳಿಂದ ಸ್ವರೂಪ ಸ್ವಭಾವಗಳಲ್ಲಿ ಭಿನ್ನವಾದುದು, ಅದು. ಅದಕ್ಕೆ ಸತ್ವ, ರಜ ತಮವೆಂಬ ಗುಣಗಳ ಸಂಬಂಧವಿಲ್ಲ; ದೇವ ಮನುಷ್ಯ ಇತ್ಯಾದಿ ಆಕಾರಭೇದವಿಲ್ಲ; ಅವೇನಿದ್ದರೂ ಕೇವಲ ಪ್ರಕೃತಿವಿಕಾರವಾದ ಶರೀರಕ್ಕೆ ಸೇರಿದವು. ಆತ್ಮ ಸ್ವಪ್ರಕಾಶಿ, ಸೂಕ್ಷ್ಮಾತಿಸೂಕ್ಷ್ಮ. ಅದನ್ನು ಕತ್ತಿಯಿಂದ ಕಡಿಯಲಾಗುವುದಿಲ್ಲ, ಬೆಂಕಿಯಿಂದ ಸುಡಲಾಗುವುದಿಲ್ಲ, ನೀರಿನಿಂದ ನೆನೆಸ ಲಾಗುವುದಿಲ್ಲ. ಅದನ್ನು ಯಾವ ವಿಕಾರಕ್ಕೂ ಒಳಪಡಿಸುವುದು ಸಾಧ್ಯವಿಲ್ಲ. ಅದು ಸಕಲ ಜೀವಿಗಳ ಅಂತರಂಗದಲ್ಲಿಯೂ ಇದೆ. ಹೀಗೆ ಸರ್ವಾತ್ಮಕವಾದುದರಿಂದಲೇ ಅದು ಪರ ಮಾತ್ಮ. ಅದರ ಸಾಕ್ಷಾತ್ಕಾರಕ್ಕೆ ಅನನ್ಯ ಭಕ್ತಿಯೇ ಅತ್ಯಂತ ಸುಲಭಮಾರ್ಗ.”

“ಅಮ್ಮ, ಭಕ್ತಿಭಾವ ಹುಟ್ಟಿ ಬೆಳೆಯುವುದಕ್ಕೆ ಸಾಧುಸಜ್ಜನರ ಸಂಗವೇ ಮುಖ್ಯ ಸಾಧನ. ಮೋಕ್ಷದ ಹೆಬ್ಬಾಗಿಲನ್ನು ಹಾರಹೊಡೆಯುವಂತೆ ಮಾಡಬಲ್ಲ ಈ ಸಾಧುಸತ್ಪುರುಷರ ಲಕ್ಷಣವನ್ನು ಹೇಳುತ್ತೇನೆ, ಕೇಳು. ತಮಗೆ ಇತರರು ಕೇಡನ್ನು ಮಾಡಿದರೂ ಅವರು ಅದನ್ನು ಸಹಿಸಿಕೊಂಡು, ಉಪಕಾರವನ್ನೇ ಮಾಡುತ್ತಾರೆ. ಅವರು ಸರ್ವಪ್ರಾಣಿಗಳ ಲ್ಲಿಯೂ ಸಮಾನವಾದ ಪ್ರೇಮವುಳ್ಳವರು. ಯಾರನ್ನೂ ಅವರು ದ್ವೇಷಿಸರು, ಯಾರ ಲ್ಲಿಯೂ ಅಸೂಯೆಪಡರು. ಅವರು ಶಮದಮಾದಿಗಳಿಂದ ಕೂಡಿದ ಸದಾಚಾರಿಗಳು, ದುರಾಶೆಯನ್ನು ತೊರೆದ ದೈವಭಕ್ತರು. ಅವರು ಒಂದೇ ಮನಸ್ಸಿನಿಂದ ಪರಮೇಶ್ವರನನ್ನು ಆರಾಧಿಸುತ್ತಾರೆ, ಧ್ಯಾನಿಸುತ್ತಾರೆ; ಇದರ ಫಲವಾಗಿ ಅವರು ಈಶ್ವರನ ಕೃಪೆಗೆ ಪಾತ್ರ ರಾಗುತ್ತಾರೆ. ಅವರಿಗೆ ಸಂಸಾರತಾಪತ್ರಯದ ಬಾಧೆಯಿಲ್ಲ. ಅಂತಹ ಸಾಧುಗಳ ಸಹವಾಸ ವನ್ನು ಸಂಪಾದಿಸುವುದೇ ಮೋಕ್ಷಾಪೇಕ್ಷಿಯ ಮೊದಲ ಕೆಲಸ. ಏಕೆಂದರೆ ಅವರು ಸದಾ ಭಗವಂತನ ಸಂಕೀರ್ತನ ಮಾಡುತ್ತಿರುವುದರಿಂದ ಅವರ ಸಂಗದಿಂದ ಕಿವಿಗೆ ಇಂಪೂ ಹೃದಯಕ್ಕೆ ತಂಪೂ ತಾನೇ ತಾನಾಗಿ ದೊರೆಯುತ್ತದೆ. ಅಷ್ಟೇ ಅಲ್ಲ, ಅವರು ಹೇಳುವ ಭಗವಂತನ ಅದ್ಭುತ ಚರಿತ್ರೆಯನ್ನು ಕೇಳಿ ಕೇಳಿ ಭಕ್ತಿಯೂ ತಾನಾಗಿ ಹುಟ್ಟುತ್ತದೆ. ಒಮ್ಮೆ ಭಕ್ತಿ ಮೂಡಿತೆಂದರೆ ಭೋಗದ ಆಸೆ ತಗ್ಗುತ್ತದೆ. ಕ್ರಮಕ್ರಮವಾಗಿ ಇಹಭೋಗದಷ್ಟೇ ಸ್ವರ್ಗಸುಖವೂ ನಶ್ವರವೆನಿಸುತ್ತದೆ. ಯಾವುದು ನಿತ್ಯ, ಯಾವುದು ಅನಿತ್ಯ ಎಂಬ ವಿವೇಕ ಹುಟ್ಟುತ್ತದೆ. ಅನಿತ್ಯದಲ್ಲಿ ಅಸಹ್ಯ ತೋರಿ ಮನಸ್ಸು ಮೋಕ್ಷದತ್ತ ಹರಿಯುತ್ತದೆ. ಆಗ ಮನುಷ್ಯ ಪ್ರತಿಫಲವನ್ನು ಬಯಸದೆ ಕರ್ಮವನ್ನು ಮಾಡುತ್ತಾನೆ. ಹೀಗೆ ವೈರಾಗ್ಯದಿಂದ ಮೂಡಿದ ಜ್ಞಾನ, ನಿಷ್ಕಾಮದಿಂದ ಮಾಡಿದ ಕರ್ಮ–ಇವುಗಳಿಂದ ಪರಮಪ್ರೇಮರೂಪ ವಾದ ಭಕ್ತಿಯೋಗವನ್ನು ಮೈಗೂಡಿಸಿಕೊಂಡವನು ಈ ಜನ್ಮದಲ್ಲಿಯೇ ಪರಮಾತ್ಮನ ಸಾಕ್ಷಾತ್ಕಾರವನ್ನು ಪಡೆಯಬಹುದು.”

ಮಗನ ಉಪದೇಶವಾಣಿಯಿಂದ ದೇವಹೂತಿಗೆ ಪರಮಾನಂದವಾಯಿತು. ಆತನಿಂದ ಭಕ್ತಿಯೋಗದ ಸ್ವರೂಪವನ್ನು ಇನ್ನೂ ಸ್ಪಷ್ಟವಾಗಿ ಅರಿಯಲೆಂದು ಆಕೆ, “ಹೇ ಮಹಾತ್ಮಾ, ನಾನು ಎಷ್ಟಾದರೂ ಹೆಂಗಸು; ವೇದಾಂತ ವಿಚಾರಗಳನ್ನು ಬೇಗ ಅರ್ಥಮಾಡಿಕೊಳ್ಳಲಾರೆ. ಅದೇನೋ ಭಕ್ತಿಯೋಗವೆಂದು ಹೇಳುತ್ತಿರುವೆಯಲ್ಲಾ, ಹಾಗೆಂದರೇನು? ಗುರಿಯಿಟ್ಟ ಬಾಣದಂತೆ ನೇರವಾಗಿ ಪರಮಾತ್ಮನಲ್ಲಿಗೆ ಹೋಗಿ ಸೇರುವಂತಹ ಭಕ್ತಿಯ ಸ್ವರೂಪವೆಂತ ಹುದು? ಮೂಢಳಾದ ನನಗೆ ಅರ್ಥವಾಗುವಂತೆ ವಿವರಿಸಿ ಹೇಳು” ಎಂದು ಕೇಳಿಕೊಂಡಳು. ಕಪಿಲಮುನಿ ಅದನ್ನು ಆಕೆಗೆ ವಿವರಿಸಿ ಹೇಳಿದ:

“ಅಮ್ಮ, ದೇವರು ಕೊಟ್ಟಿರುವ ಕಣ್ಣು, ಕಿವಿ, ಮೂಗು, ನಾಲಗೆ, ಚರ್ಮ ಎಂಬ ಐದು ಜ್ಞಾನೇಂದ್ರಿಯಗಳನ್ನೂ ವಾಕ್ಕು, ಕೈ, ಕಾಲು, ಜಲದ್ವಾರ, ಮಲದ್ವಾರ ಎಂಬ ಐದು ಕರ್ಮೇಂದ್ರಿಯಗಳನ್ನೂ ಫಲಾಪೇಕ್ಷೆಯಿಲ್ಲದೆ ಕೇವಲ ಭಗವಂತನ ಆರಾಧನೆಗಾಗಿಯೇ ಬಳಸಬೇಕೆಂಬ ಮನೋವೃತ್ತಿಯೇ ಭಕ್ತಿ. ತಿಂದ ಅನ್ನವನ್ನು ಜಠರಾಗ್ನಿ ಜೀರ್ಣಮಾಡು ವಂತೆ ಈ ಭಕ್ತಿಯೂ ಅನಾದಿಯಾಗಿ ಅನುಸರಿಸಿ ಬಂದಿರುವ ಪಾಪಕರ್ಮವನ್ನು ನಾಶ ಮಾಡುತ್ತದೆ. ಭಕ್ತಿಯ ರುಚಿಯನ್ನು ಕಂಡವರು ಮುಕ್ತಿಯನ್ನು ಕೂಡ ಬಯಸುವುದಿಲ್ಲ. ಆದರೆ ಅವರ ಭಕ್ತಿಯೇ ಅವರಿಗೆ ಮುಕ್ತಿಮಾರ್ಗವನ್ನು ತಾನಾಗಿ ತಂದುಕೊಡುತ್ತದೆ. ಅವರು ಅರ್ಚಿರಾದಿ ಗತಿಯಿಂದ ಮುಕ್ತಿಗೆ ಸಾಗುತ್ತಿರುವಾಗ ಬ್ರಹ್ಮ, ಇಂದ್ರ ಇತ್ಯಾದಿ ದೇವತೆಗಳು ತಮ್ಮ ಲೋಕಗಳ ಸರ್ವಾಧಿಕಾರಗಳನ್ನು ನೀಡಿದರೂ ಭಕ್ತರು ಅವುಗಳನ್ನು ಕಣ್ಣೆತ್ತಿಯೂ ನೋಡುವುದಿಲ್ಲ. ಯಾರಿಗೆ ಭಗವಂತನು ತನ್ನ ಸ್ವಂತ ದೇಹದಂತೆ, ಪುತ್ರ ನಂತೆ, ಮಿತ್ರನಂತೆ, ಗುರುವಿನಂತೆ, ಆಪ್ತನಂತೆ, ಇಷ್ಟದೈವದಂತೆ ಆತ್ಮೀಯನಾಗು ತ್ತಾನೆಯೋ ಅಂತಹ ಭಕ್ತನಿಗೆ ಸಂಸಾರಭಯವಿಲ್ಲ. ಸರ್ವಜೀವಿಗಳ ಅಂತರಾತ್ಮನಾಗಿ, ಪ್ರಕೃತಿ ಪುರುಷರ ನಿಯಾಮಕನಾಗಿ, ಮಹಾಮಹಿಮನಾಗಿರುವ ಭಗವಂತನ ಕೃಪೆಯಿಲ್ಲದೆ ಸಂಸಾರಬಂಧನ ತಪ್ಪದು. ಎಲ್ಲ ದೇವತೆಗಳೂ ಆ ದೇವದೇವನಿಗೆ ವಿಧೇಯರು. ಆ ಪರಮಪುರುಷನ ಭಯದಿಂದಲೇ ಗಾಳಿ ಬೀಸುತ್ತದೆ, ಸೂರ್ಯ ಬೆಳಗುತ್ತಾನೆ, ಇಂದ್ರ ಮಳೆ ಗರೆಯುತ್ತಾನೆ, ಬೆಂಕಿ ಉರಿಯುತ್ತದೆ, ಮೃತ್ಯು ತನ್ನ ಕರ್ತವ್ಯವನ್ನು ನಿರ್ವಹಿಸುತ್ತದೆ. ಇದನ್ನು ಅರ್ಥಮಾಡಿಕೊಂಡ ಯೋಗಿಗಳು ಜ್ಞಾನವೈರಾಗ್ಯಗಳಿಂದ ಕೂಡಿದ ಭಕ್ತಿಯಿಂದ ಆ ಪುರಾಣಪುರುಷನ ಪಾದಗಳನ್ನು ಆಶ್ರಯಿಸುತ್ತಾರೆ, ತಮ್ಮ ಮನಸ್ಸನ್ನು ಆತನಲ್ಲಿ ಸ್ಥಿರವಾಗಿ ನಿಲ್ಲಿಸುತ್ತಾರೆ. ಇದೇ ಭಕ್ತಿಯೋಗ, ಮುಕ್ತಿಮಾರ್ಗ.”

“ಅಮ್ಮ, ಮೋಕ್ಷ, ಬೇಕೆನ್ನುವವನು ಮೊದಲು ಪ್ರಕೃತಿಗುಣಗಳಾದ ಸತ್ವ ರಜ ತಮ ಗಳಿಂದ ಮುಕ್ತನಾಗಬೇಕು. ಪ್ರಕೃತಿ, ಪುರುಷ, ಕಾಲ–ಎಂಬ ಮೂರು ತತ್ವಗಳ ಸ್ವರೂಪ ಅರ್ಥವಾಗದ ಹೊರತು ಇದು ಸಾಧ್ಯವಿಲ್ಲ. ಆದ್ದರಿಂದ ಅವುಗಳ ಸ್ವರೂಪವನ್ನು ನಿನಗೀಗ ವಿವರಿಸುತ್ತೇನೆ. ‘ಪ್ರಕೃತಿ’ಯೆಂಬುದು ನಮ್ಮ ಕಣ್ಣಿಗೆ ಕಾಣುತ್ತಿರುವ ಈ ಜಗತ್ತಿನ ಸೃಷ್ಟಿಗೆ ಮೂಲಭೂತವಾದ ಪ್ರಧಾನವಸ್ತು. ಇದು ನಿತ್ಯವಾದುದು. ಪ್ರಳಯಕಾಲದಲ್ಲಿ ಇದು ನಿರಾ ಕಾರವಾಗಿರುತ್ತದೆ; ಆದ್ದರಿಂದ ಆಗ ಅದನ್ನು ‘ಅವ್ಯಕ್ತ’ ಎಂದು ಕರೆಯುತ್ತಾರೆ. ಅದರ ಗುಣಗಳಾದ ಸತ್ವ ರಜ ತಮಗಳು ಸಮಸ್ಥಿತಿಯಲ್ಲಿರುವಷ್ಟು ಕಾಲವೂ ಅದು ಅವ್ಯಕ್ತ ಸ್ಥಿತಿಯಲ್ಲಿಯೇ ಇರುತ್ತದೆ. ಆ ಸಮಸ್ಥಿತಿ ತಪ್ಪಿ, ಆ ಗುಣಗಳಲ್ಲಿ ಏರುಪೇರುಗಳುಂಟಾಗು ತ್ತಲೆ ಸೃಷ್ಟಿಕಾರ್ಯ ಪ್ರಾರಂಭವಾಗುತ್ತದೆ. ಆ ಏರುಪೇರನ್ನು ಮಾಡುವವನೇ ‘ಪುರುಷ’. ಪ್ರಕೃತಿಯು ಆತನ ಶರೀರ ಮಾತ್ರ. ಅವ್ಯಕ್ತವನ್ನು ವ್ಯಕ್ತವಾಗಿ ಮಾಡಲೆಂದು ಸಂಕಲ್ಪಿಸಿದ ಆತನು ಪ್ರಾರಂಭದಲ್ಲಿ ಪ್ರಕೃತಿಯಲ್ಲಿನ ತಮೋಗುಣವನ್ನೆಲ್ಲ ಅಡಗಿಸಿ, ಸತ್ವಗುಣ ತಲೆ ಯೆತ್ತುವಂತೆ ಮಾಡುತ್ತಾನೆ. ನಿರ್ಮಲವಾಗಿ, ಪ್ರಕಾಶಮಾನವಾಗಿರುವ ಈ ಸತ್ವಗುಣ ಮಹತ್ ತತ್ವವಾಗಿ ಪರಿಣಮಿಸುತ್ತದೆ. ಇದೇ ಚಿತ್ತ. ಸ್ವಚ್ಛವಾದ ನೀರಿನಲ್ಲಿ ಪ್ರತಿಬಿಂಬ ಕಾಣಬರುವಂತೆ, ವಿಕಾರವಾವುದೂ ಇಲ್ಲದ ಚಿತ್ತಕ್ಕೆ ಪರಮಾತ್ಮನ ರೂಪವನ್ನು ಗ್ರಹಿಸ ಬಲ್ಲ ಶಕ್ತಿಯಿರುತ್ತದೆ. ಈ ಚಿತ್ತವು ಪರಮಾತ್ಮನ ಸಂಕಲ್ಪದಿಂದ ವಿಕಾರ ಹೊಂದಿ ಅಹಂಕಾರ ತತ್ವವು ಹುಟ್ಟುತ್ತದೆ. ಇದು ರಾಜಸಗುಣಪ್ರಧಾನವಾಗಿ ಕ್ರಿಯಾಶಕ್ತಿ ಯುಳ್ಳುದು. ಇದರಲ್ಲಿ ಸಾತ್ವಿಕ ರಾಜಸ ತಾಮಸ ಎಂದು ಮೂರುವಿಧ. ಸಾತ್ವಿಕಾಹಂಕಾರ ದಿಂದ ಬುದ್ಧಿ ಮತ್ತು ಇಂದ್ರಿಯಗಳೂ, ತಾಮಸಾಹಂಕಾರದಿಂದ ಪಂಚಭೂತಗಳೂ ಹುಟ್ಟುತ್ತವೆ. ಮನಸ್ಸೆಂಬುದು ಸಂಕಲ್ಪ ವಿಕಲ್ಪಗಳನ್ನು ಹೊಂದಿ ಶಬ್ದಾದಿ ವಿಷಯಗಳಲ್ಲಿ ಆಸಕ್ತಿಯುಳ್ಳುದಾಗಿರುತ್ತದೆ. ಬುದ್ಧಿಯೆಂಬುದು ಇಂದ್ರಿಯಗಳಿಗೆ ಸಹಾಯಕವಾದುದು; ವಸ್ತುಜ್ಞಾನ, ಸಂಶಯ, ಅನ್ಯಥಾಜ್ಞಾನ, ವಿಪರೀತಜ್ಞಾನ, ನಿಶ್ಚಯ, ಸ್ಮರಣ, ಜ್ಞಾಪಕ– ಇವೆಲ್ಲ ಬುದ್ಧಿಯ ವ್ಯಾಪಾರಗಳೆ.”

“ಭೂಮಿ ನೀರು ಬೆಂಕಿ ಗಾಳಿ ಆಕಾಶ–ಎಂಬ ಪಂಚಭೂತಗಳೂ, ಅವುಗಳ ಗುಣ ರೂಪವಾದ ಗಂಧ ರಸ ರೂಪ ಸ್ಪರ್ಶ ಶಬ್ದವೆಂಬ ಪಂಚ ತನ್ಮಾತ್ರಗಳೂ, ಕಣ್ಣು ಕಿವಿ ಮೂಗು ನಾಲಗೆ ಚರ್ಮ–ಎಂಬ ಐದು ಜ್ಞಾನೇಂದ್ರಿಯಗಳೂ, ವಾಕ್ಕು ಕೈ ಕಾಲು ಜಲ ದ್ವಾರ ಮಲದ್ವಾರ ಎಂಬ ಐದು ಕರ್ಮೇಂದ್ರಿಯಗಳೂ, ಮನಸ್ಸು ಬುದ್ಧಿ ಅಹಂಕಾರ ಚಿತ್ತ–ಎಂಬ ನಾಲ್ಕು ಅಂತಃಕರಣಗಳೂ ಸೇರಿ ಇಪ್ಪತ್ತುನಾಲ್ಕು ಪ್ರಕೃತಿ ತತ್ವಗಳಾಗಿವೆ. ಅನಂತವಾದ ಕಾಲವನ್ನೂ ಇವುಗಳ ಸಾಲಿನಲ್ಲಿಯೇ ಸೇರಿಸಿ ಇಪ್ಪತ್ತೈದೆಂದು ಹೇಳುವು ದುಂಟು. ಆದರೆ ಇದು ಪ್ರತ್ಯೇಕವಾದ ತತ್ವವಲ್ಲ, ಪರಮಪುರುಷನ ದಿವ್ಯಶಕ್ತಿ. ಈ ಶಕ್ತಿ ಯಿಂದಲೇ ಸೃಷ್ಟಿ ಸ್ಥಿತಿ ಲಯಗಳೆಂಬ ಸಂಸಾರಕ್ಕೆ ಪ್ರೇರಣೆಯಾಗುವುದು. ಸರ್ವಶಕ್ತನಾದ ಪರಮಾತ್ಮನು ಜೀವಿಗಳ ಅಂತರಂಗದಲ್ಲಿ ಜೀವಾತ್ಮನಾಗಿಯೂ ಬಹಿರಂಗದಲ್ಲಿ ಕಾಲ ರೂಪನಾಗಿಯೂ ವ್ಯಾಪಿಸಿದ್ದಾನೆ.”

“ತಾಯಿ, ನೆಲ ನೀರು ಇತ್ಯಾದಿ ಸ್ಥೂಲರೂಪದಲ್ಲಿರುವ ಪ್ರಕೃತಿಯ ಸ್ವರೂಪವನ್ನು ತಿಳಿದುಕೊಳ್ಳುವುದು ನಮಗೆ ಸಾಧ್ಯ. ಆದರೆ ಪುರುಷನು ಇಂತಹ ಸ್ಥೂಲರೂಪವಾಗಿ ಪರಿ ಣಮಿಸುವುದಿಲ್ಲವಾದ್ದರಿಂದ ಆತನ ಸ್ವರೂಪವನ್ನು ಯುಕ್ತಿಯಿಂದ ಅರ್ಥಮಾಡಿಕೊಳ್ಳ ಬೇಕು. ನೋಡು, ಈ ದೇಹ ಪಂಚಭೂತಗಳಿಂದ ಆದುದು. ಜಡವಾದ ಈ ಪಂಚಭೂತ ಗಳಿಗೆ ಸುಖದುಃಖಾದಿ ಜ್ಞಾನವಿಕಾರಗಳನ್ನು ಹೊಂದಿರುವುದು ಸಾಧ್ಯವಿಲ್ಲ, ಆದ್ದರಿಂದ ಈ ಸ್ಥೂಲದೇಹದಿಂದ ಭಿನ್ನವಾದ ಯಾವುದೋ ಒಂದು ಶಕ್ತಿ ಸುಖದುಃಖ, ರಾಗದ್ವೇಷಾದಿ ಗಳನ್ನು ಅನುಭವಿಸುತ್ತದೆ. ದೇಹದ ಇಂದ್ರಿಯಗಳೆಲ್ಲ ತಮ್ಮ ತಮ್ಮ ಸ್ಥಾನಗಳಲ್ಲಿಯೇ ನೆಲಸಿದ್ದರೂ ಈ ಶಕ್ತಿ ಒಳಗಿದ್ದ ಹೊರತು ತಮ್ಮ ಕಾರ್ಯಗಳನ್ನು ಮಾಡಲು ಶಕ್ತವಾಗುವು ದಿಲ್ಲ. ಈ ಶಕ್ತಿಯನ್ನೇ ಪುರುಷ, ಜೀವ, ಆತ್ಮ, ಜೀವಾತ್ಮ, ಕ್ಷೇತ್ರಜ್ಞ; ಪ್ರತ್ಯಗಾತ್ಮ– ಇತ್ಯಾದಿ ಹೆಸರುಗಳಿಂದ ಕರೆಯುತ್ತೇವೆ. ಶರೀರದ ಮೂಲಕ ಸುಖದುಃಖಗಳನ್ನು ಅನು ಭವಿಸುವುದರಿಂದ ಈ ಜೀವನನ್ನು ‘ಭೋಕ್ತೃ’ ಎನ್ನುತ್ತಾರೆ. ದೀಪದಂತೆ ಸ್ವಯಂಪ್ರಕಾಶ ವುಳ್ಳ ಜೀವನು ಬ್ರಹ್ಮನಿಂದ ಹಿಡಿದು ಸಣ್ಣಕೀಟದವರೆಗೆ ಎಲ್ಲವನ್ನೂ ವ್ಯಾಪಿಸಿರುವನು. ಪುಣ್ಯ ಪಾಪರೂಪವಾದ ಕರ್ಮಗಳ ವ್ಯತ್ಯಾಸದಿಂದ ದೇವ, ಮನುಷ್ಯ ಇತ್ಯಾದಿ ವ್ಯತ್ಯಾಸ ಗಳನ್ನು ಜೀವನು ತನ್ನಲ್ಲಿ ಆರೋಪಿಸಿಕೊಳ್ಳುತ್ತಾನೆಯೇ ಹೊರತು ಆಯಾ ದೇಹಗಳೇನೂ ಅವನಲ್ಲ. ಹುಟ್ಟು ಸಾವು ಮೊದಲಾದ ವಿಕಾರಗಳೇನಿದ್ದರೂ ಆಯಾ ದೇಹಗಳಿಗೆ ಹೊರತು ಜೀವನಿಗಲ್ಲ. ಈ ಜೀವ ಒಮ್ಮೊಮ್ಮೆ ಪ್ರಕೃತಿಯ ಸಂಬಂಧವನ್ನು ಪಡೆಯುತ್ತಾನೆ. ಪ್ರಕೃತಿಯು ಜ್ಞಾನವನ್ನು ಮರೆಸುವ ಸ್ವಭಾವವುಳ್ಳುದಾದುದರಿಂದ, ಅದರ ಸಂಬಂಧ ದಿಂದ ಜೀವನಿಗೆ ಮೋಹ ಮುಸುಕಿ, ಸ್ವಸ್ವರೂಪವನ್ನು ಮರೆತು ದೇಹವೇ ತಾನೆಂದು ಭ್ರಮಿಸುತ್ತಾನೆ. ಅಷ್ಟೇ ಅಲ್ಲ, ಆ ಪ್ರಕೃತಿಗುಣಗಳಿಂದ ನಡೆವ ಕರ್ಮಗಳನ್ನೆಲ್ಲ ತಾನೆ ಮಾಡುವುದಾಗಿ ಭ್ರಮಿಸುವನು. ಹೀಗೆ ಕರ್ತೃತ್ವವನ್ನು ತನ್ನಲ್ಲಿ ಆರೋಪಿಸಿಕೊಳ್ಳುವುದ ರಿಂದಲೆ ಸರ್ವತಂತ್ರ ಸ್ವತಂತ್ರನಾದ ಜೀವಾತ್ಮನಿಗೆ ಸಂಸಾರಬಂಧನ ಪ್ರಾಪ್ತವಾಗು ವುದು.”

“ಅಮ್ಮ, ಆತ್ಮನಿಗೆ ಬಂದು ಮುಸುಕಿದ ಈ ಸಂಸಾರಬಂಧನದಿಂದ ಪಾರಾಗುವುದು ಹೇಗೆ? ಅದಕ್ಕೆ ಮೊದಲು ಉಪಾಯವೆಂದರೆ ಮನಸ್ಸನ್ನು ವಿಷಯಸುಖಗಳ ಕಡೆ ಹರಿಯ ದಂತೆ ತಡೆಯುವುದು. ಸ್ಥಿರವಾದ ವೈರಾಗ್ಯ ಮತ್ತು ಭಕ್ತಿಯೋಗದಿಂದ ಮನಸ್ಸು ಸ್ವಾಧೀನಕ್ಕೆ ಬರುತ್ತದೆ. ಮಾತಿನಲ್ಲಿ ಮತ್ತು ಊಟದಲ್ಲಿ ಮಿತ, ಏಕಾಂತವಾಸ, ಸರ್ವ ಸಮಾನತೆ, ವಿರೋಧಿಗಳಲ್ಲಿಯೂ ಮೈತ್ರಿ, ನಿರ್ಭಯ, ಭಗವಂತನಲ್ಲಿಯೇ ನಟ್ಟ ಮನಸ್ಸು –ಇವುಗಳಿಂದ ಸ್ವಸ್ವರೂಪಜ್ಞಾನವುಂಟಾಗುತ್ತದೆ. ಆತ್ಮನ ಸ್ವರೂಪ ಆನಂದ, ಜ್ಞಾನ. ಆತ್ಮಧರ್ಮವಾದ ಜ್ಞಾನಂದಿಂದಲೇ ಆತ್ಮಸ್ವರೂಪದ ದರ್ಶನ ವಾಗಬೇಕು. ಈ ದೇಹವೆಂಬುದು ಪಂಚಭೂತಗಳು, ಇಂದ್ರಿಯ ಮತ್ತು ಮನಸ್ಸುಗಳ ಪರಿಣಾಮ ವಷ್ಟೆ? ಆತ್ಮ ಎಂಬುದು ಇದರಿಂದ ವಿಲಕ್ಷಣವಾದುದು ಎಂದು ಯಾವ ಜ್ಞಾನದಿಂದ ಅರ್ಥವಾಗುತ್ತ ದೆಯೋ ಅದೇ ಜ್ಞಾನದಿಂದ ಆತ್ಮಸ್ವರೂಪವನ್ನು ಅರ್ಥಮಾಡಿಕೊಳ್ಳಬೇಕು. ಆ ದೇಹವು ಆತ್ಮವೆಂಬ ಭ್ರಾಂತಿಯೂ, ಇದು ಆತ್ಮವಲ್ಲವೆಂಬ ಸತ್ಯಸ್ವರೂಪವೂ ಗೋಚರವಾಗುವುದು ಆತ್ಮನ ಗುಣವಾದ ಜ್ಞಾನದಿಂದಲೆ. ಎಚ್ಚರ, ನಿದ್ರೆ, ಕನಸು–ಎಂಬ ಮೂರು ಸ್ಥಿತಿಗಳಲ್ಲಿಯೂ ಆತ್ಮನಿಗೆ ‘ನಾನು’ ಎಂಬ ಜ್ಞಾನ ಇದ್ದೇ ಇರುತ್ತದೆ. ಆದ್ದರಿಂದಲೇ ‘ನಾನು ಚೆನ್ನಾಗಿ ನಿದ್ದೆಹೋದೆ,‘ ‘ನಾನು ಕನಸು ಕಂಡೆ’ ಎಂದು ಹೇಳುವುದು. ಯೋಚಿಸಿ ನೋಡು, ಆತ್ಮನು ದೇಹವನ್ನು ಬಿಡುತ್ತಲೆ ಈ ದೇಹ ಜಡವಾಗಿ ಹೋಗುತ್ತದೆ, ಆದ್ದರಿಂದ ದೇಹ ಆತ್ಮಗಳು ಬೇರೆಬೇರೆ ಎಂಬುದು ಸಪ್ರಮಾಣ; ಈ ಜಡ ದೇಹದಿಂದ ಆತ್ಮ ಶಾಶ್ವತವಾದ ಬಿಡುಗಡೆಯನ್ನು ಪಡೆಯುವುದು ಸುಲಭವಲ್ಲದಿದ್ದರೂ ಅಸಾಧ್ಯ ವೇನಲ್ಲ. ಕಟ್ಟಿಗೆಯಲ್ಲಿರುವ ಬೆಂಕೆ ಕಟ್ಟಿಗೆಯನ್ನೇ ಸುಡುವಂತೆ ದೇಹದೊಳಗಿರುವ ಆತ್ಮದೇಹನಾಶಕ್ಕೆ ಕಾರಣವಾಗುತ್ತದೆ.”

“ಪ್ರಕೃತಿಪುರುಷರ ಸ್ವರೂಪಜ್ಞಾನದಿಂದ ಜೀವನು ಸಂಸಾರಬಂಧನದಿಂದ ಮುಕ್ತನಾಗು ತ್ತಾನೆ. ಈ ಜ್ಞಾನವು ಭಗವಂತನ ಅನುಗ್ರಹದಿಂದಲೇ ಲಭಿಸಬೇಕು. ದೇವದೇವನ ಅನು ಗ್ರಹಕ್ಕೆ ಭಕ್ತಿಯೇ ಮುಖ್ಯ ಸಾಧನ. ಈ ಭಕ್ತಿಯಲ್ಲಿ ಸಾತ್ವಿಕ, ರಾಜಸ, ತಾಮಸವೆಂಬ ಮೂರು ವಿಧ. ತನ್ನ ಶತ್ರು ಸಾಯಲೆಂದೊ, ಇತರರನ್ನು ವಂಚಿಸಲೆಂದೊ, ಅಥವಾ ತನ್ನ ಸ್ಪರ್ಧಿಯನ್ನು ಸೋಲಿಸಲೆಂದೊ ದೇವರನ್ನು ಆರಾಧಿಸುವುದು ತಾಮಸಭಕ್ತಿ. ಇಂದ್ರಿಯ ಸುಖವನ್ನೊ, ಕೀರ್ತಿಯನ್ನೊ, ಸಂಪತ್ತನ್ನೊ ಬಯಸಿ ಪೂಜಿಸುವುದು ರಾಜಸಭಕ್ತಿ; ಕರ್ಮ ಗಳು ನಾಶವಾಗಿ ಮೋಕ್ಷವು ಲಭಿಸಲೆಂದು ಪೂಜಿಸುವುದು ಸಾತ್ವಿಕಭಕ್ತಿ. ಈ ಮೂರ ರಲ್ಲಿಯೂ ಫಲಾಪೇಕ್ಷೆ ಇರುವುದರಿಂದ ಇದನ್ನು ‘ಸಗುಣ ಭಕ್ತಿ’ ಎನ್ನುತ್ತಾರೆ. ಆದರೆ ಯಾವ ಫಲಾಪೇಕ್ಷೆಯೂ ಇಲ್ಲದೆ, ಸರ್ವಾತ್ಮನಾದ ಭಗವಂತನಲ್ಲಿ ಸದಾ ಮನಸ್ಸನ್ನಿಟ್ಟಿರು ವುದೇ ‘ನಿರ್ಗುಣ ಭಕ್ತಿ’, ಈ ನಿರ್ಗುಣಭಕ್ತಿಯುಳ್ಳವನು ಸಾಲೋಕ್ಯ, ಸಾಮೀಪ್ಯ, ಸಾಯುಜ್ಯ ಎಂಬ ಮೋಕ್ಷಗಳನ್ನು ಕೂಡ ಬಯಸುವುದಿಲ್ಲ. ಆತನ ಮನಸ್ಸು ನಿರ್ವಿಕಾರ ವಾಗಿರುತ್ತದೆ. ಈ ಚಿತ್ತಶುದ್ಧಿ ಹೊಂದಿದವನು ಎಲ್ಲದರಲ್ಲಿಯೂ ಆತ್ಮನನ್ನು ಕಾಣು ತ್ತಾನೆ. ಹಾಗೆ ಕಾಣುವುದರಿಂದ ಆತನು ಎಲ್ಲವನ್ನೂ ತನ್ನಂತೆಯೇ ಕಾಣುತ್ತಾನೆ. ಈತ ‘ಸಮದರ್ಶಿ.’”

“ಅಮ್ಮ, ಭಗವಂತನಲ್ಲಿ ಮನಸ್ಸನ್ನು ಸ್ಥಿರವಾಗಿ ನಿಲ್ಲಿಸುವುದೇ ಭಕ್ತಿಯೋಗವೆಂದು ಹೇಳಿದೆನಷ್ಟೆ. ಹಾಗೆ ಸ್ಥಿರವಾಗಿ ನಿಲ್ಲಿಸಲು ಅನುಸರಿಸಬೇಕಾದ ಸಾಧನವಿಧಾನವನ್ನು ನಿನಗೀಗ ತಿಳಿಸುತ್ತೇನೆ, ಚಿತ್ತ ಕೊಟ್ಟು ಕೇಳು.”

“ಭಕ್ತಿಯೋಗವನ್ನು ಕೈಕೊಳ್ಳುವವನು ತನ್ನ ವರ್ಣ ಮತ್ತು ಆಶ್ರಮಕ್ಕೆ ಉಚಿತವಾದ ಕರ್ಮಗಳನ್ನು ಫಲಾಪೇಕ್ಷೆಯಿಲ್ಲದೆ ಆಚರಿಸಬೇಕು. ಯಾವುದಕ್ಕೂ ಆಶೆಪಡದೆ ದೊರೆತಷ್ಟ ರಿಂದಲೆ ತೃಪ್ತನಾಗಬೇಕು. ಸಾಧುಸಂತರನ್ನು ಪೂಜಿಸಬೇಕು; ಧರ್ಮ, ಅರ್ಥ, ಕಾಮ ಗಳೆಂಬ ಪುರುಷಾರ್ಥಗಳನ್ನು ಬಯಸದೆ ಕೇವಲ ಮೋಕ್ಷದಲ್ಲಿ ಆಸಕ್ತಿಯುಳ್ಳವನಾಗ ಬೇಕು; ಹಿತಮಿತವಾದ ಸಾತ್ವಿಕಾಹಾರವನ್ನು ಸೇವಿಸ ಬೇಕು; ಏಕಾಂತವಾಸವನ್ನು ಕೈಕೊಳ್ಳ ಬೇಕು. ಭಕ್ತಿಯೋಗವನ್ನು ಅಷ್ಟಾಂಗಯೋಗವೆಂದು ಕರೆಯುತ್ತಾರೆ. ಯಮ, ನಿಯಮ, ಆಸನ, ಪ್ರಾಣಾಯಾಮ, ಪ್ರತ್ಯಾಹಾರ, ಧ್ಯಾನ, ಧಾರಣ, ಸಮಾಧಿ ಎಂಬ ಎಂಟು ಅಂಗ ಗಳನ್ನು ಒಳಗೊಂಡುದುದರಿಂದ ಅದಕ್ಕೆ ಅಷ್ಟಾಂಗಯೋಗವೆಂದು ಹೆಸರು. ಇವುಗಳಲ್ಲಿ ಮೊದಲ ಎರಡು–ಯಮ, ನಿಯಮ–ನೀತಿಗೆ ಅಥವಾ ಚಾರಿತ್ರಶುದ್ಧಿಗೆ ಸಂಬಂಧಿಸಿ ದುವು. ಕಾಯಾ ವಾಚಾ ಮನಸಾ ಯಾರಿಗೂ ಹಿಂಸಿಸದಿರುವುದು (ಅಹಿಂಸೆ), ನಡೆ ನುಡಿ ಗಳಲ್ಲಿ ಸತ್ಯವಂತನಾಗಿರುವುದು (ಸತ್ಯ), ಇತರರ ವಸ್ತುಗಳನ್ನು ಕದಿಯದಿರುವುದು (ಅಸ್ತೇಯ), ಜಿತೇಂದ್ರಿಯನಾಗಿರುವುದು (ಬ್ರಹ್ಮಚರ್ಯ), ಯಾರಿಂದಲೂ ಏನನ್ನೂ ದಾನ ವಾಗಿ ಸ್ವೀಕರಿಸದಿರುವುದು (ಅಪರಿಗ್ರಹ)–ಈ ಐದೂ “ಯಮ” ಎನಿಸುಕೊಳ್ಳುತ್ತವೆ; ಇದರಂತೆ ನಿಯಮದಲ್ಲಿಯೂ ಐದು ವಿಧಿ ವಿಧಾನಗಳಿವೆ–ದೇಹಮನಸ್ಸುಗಳೆರಡೂ ಶುದ್ಧವಾಗಿರುವುದು (ಶೌಚ), ತಾನಾಗಿ ದೊರೆತಷ್ಟರಿಂದ ಸಂತೋಷ (ತೃಪ್ತಿ), ದೇವರ ಪೂಜೆ ಜಪ(ತಪಸ್ಸು)ಗಳು, ಗುರುಮುಖದಿಂದ ಬಂದ ಅಧ್ಯಾತ್ಮ ವಿಷಯಗಳನ್ನು ಮೇಲಿಂದ ಮೇಲೆ ಓದಿ ಮನನಮಾಡಿಕೊಳ್ಳುವುದು (ಸ್ವಾಧ್ಯಾಯ), ಈಶ್ವರನಲ್ಲಿ ಭಕ್ತಿ ಯಿಡುವುದು (ಪ್ರಣಿಧಾನ); ಅಷ್ಟಾಂಗಗಳಲ್ಲಿ ಮೂರನೆಯದಾದ ‘ಆಸನ’ವೆಂದರೆ ಒಂದೆಡೆಯಲ್ಲಿಯೇ ಬಹಳ ಹೊತ್ತು ಕುಳಿತಿರುವುದು; ನಮ್ಮ ಉಸಿರಾಟವನ್ನು ಒಂದು ಕ್ರಮಕ್ಕೆ ತರುವುದು ‘ಪ್ರಾಣಾಯಾಮ’; ಇಂದ್ರಿಯಗಳನ್ನೂ ಮನಸ್ಸನ್ನೂ ಹೊರಮುಖ ವಾಗದಂತೆ ತಡೆದು ಹೃದಯದಲ್ಲಿ ಅಡಗಿಸುವುದು ‘ಪ್ರತ್ಯಾಹಾರ’; ದೇಹದ ಯಾವುದಾ ದರೂ ಒಂದು ಭಾಗದ ಮೇಲೆ ಮನಸ್ಸನ್ನು ನೆಲೆಯಾಗಿ ನಿಲ್ಲಿಸುವುದು ‘ಧಾರಣ’; ಪರಮಾತ್ಮನ ಯಾವುದಾದರೂ ಒಂದು ಆಕಾರ ಅಥವಾ ಗುಣದಲ್ಲಿ ಮನಸ್ಸು ಏಕಾಗ್ರ ವಾಗುವಂತೆ ಮಾಡುವುದು ‘ಧ್ಯಾನ’; ಹೀಗೆ ಧ್ಯಾನಿಸುತ್ತಾ ಮೈಮರೆತು ತಲ್ಲೀನನಾಗುವುದು ‘ಸಮಾಧಿ’.”

“ತಾಯಿ, ಮೇಲೆ ಹೇಳಿದ ಅಷ್ಟಾಂಗಗಳಲ್ಲಿ ಪ್ರಾಣಾಯಾಮವು ಬಹಳ ಮುಖ್ಯ ವಾದುದು. ನಮ್ಮ ಪ್ರಾಣವಾಯುವನ್ನು ವಶಮಾಡಿಕೊಂಡರೆ ಮನಸ್ಸೂ ನಮ್ಮ ಅಧೀನಕ್ಕೆ ಬರುತ್ತದೆ. ಇದು ಕಷ್ಟ ಸಾಧ್ಯವಾದ ಕೆಲಸ. ಪರಿಶುದ್ಧವಾದ ಸ್ಥಳದಲ್ಲಿ ಒಂದು ಎಡೆ ದರ್ಭೆಗಳನ್ನು ಹರಡಿ, ಅದರ ಮೇಲೆ ಕೃಷ್ಣಾಜಿನವನ್ನೂ ಅದರ ಮೇಲೆ ಒಂದು ಶುಭ್ರ ವಸ್ತ್ರವನ್ನೂ ಹಾಸಿ, ಅದರ ಮೇಲೆ ಪದ್ಮಾಸನದಲ್ಲಿ\footnote{೧. ‘ಸ್ಥಿರಸುಖಮಾಸನಂ’ ಎಂದು ಹೇಳುತ್ತಾರೆ. ಯಾವ ವಿಧದಲ್ಲಿ ಕುಳಿತುಕೊಂಡರೆ ನಮಗೆ ನಿದ್ರೆ, ಆಕಳಿಕೆ, ಸೋಮಾರಿತನ, ದೇಹಾಲಸ್ಯ ಉಂಟಾಗುವುದಿಲ್ಲವೋ ಆ ವಿಧದಲ್ಲಿ ನಾವು ಕುಳಿತುಕೊಳ್ಳಬಹುದು.} ಕುಳಿತು ಓಂಕಾರವನ್ನು ಅಭ್ಯಾಸ ಮಾಡುತ್ತಾ ಹೋಗಬೇಕು. ಇದರ ಜೊತೆಗೆ ಪ್ರಾಣಾಯಾಮದಿಂದ\footnote{೨. ಇದು ರೇಚಕ, ಪೂರಕ ಮತ್ತು ಕುಂಭಕ ಎಂದು ಮೂರುವಿಧ. ಉಸಿರನ್ನು ಹೊರಕ್ಕೆ ಬಿಡುವುದು ರೇಚಕ; ಉಸಿರನ್ನು ಒಳಕ್ಕೆಳೆದುಕೊಳ್ಳುವುದು ಪೂರಕ; ಒಳಕ್ಕೆ ತೆಗೆದುಕೊಂಡ ಗಾಳಿಯನ್ನು ದೇಹದಲ್ಲಿಯೆ ನಿರ್ದಿಷ್ಟ ಕಾಲದವರೆಗೆ ನಿಲ್ಲಿಸುವುದು ಕುಂಭಕ. ಇದರ ವಿಧಿವಿಧಾನಗಳನ್ನು ಗುರುಮುಖೇನ ಅರಿತುಕೊಳ್ಳಬೇಕು.} ಪ್ರಾಣವಾಯುವಿನ ಮಾರ್ಗವನ್ನು ಶುದ್ಧಿಗೊಳಿಸಬೇಕು. ಇದರಿಂದ ಮನಸ್ಸು ಪುಟಕ್ಕೆ ಹಾಕಿದ ಚಿನ್ನದಂತೆ ಶುದ್ಧವಾಗುತ್ತದೆ, ಸ್ವಾಧೀನಕ್ಕೆ ಬರುತ್ತದೆ. ಆಗ ಯೋಗಾಭ್ಯಾಸನಿರತನಾದವನು ತನ್ನ ದೃಷ್ಟಿಯನ್ನು ಮೂಗಿನ ತುದಿಯಲ್ಲಿ ನಿಲ್ಲಿಸಿ, ಭಗವಂತನ ದಿವ್ಯಮಂಗಳವಿಗ್ರಹವನ್ನು ಧ್ಯಾನಿಸಬೇಕು. ಆ ಕಾಲದಲ್ಲಿ ತನ್ನ ಆರಾಧ್ಯದೈವವು ಸಕಲಾಲಂಕಾರಭೂಷಿತವಾಗಿ ತನ್ನ ಕಣ್ಣ ಮುಂದೆ ನಿಂತಿರುವಂತೆಯೊ, ಕುಳಿತಿರುವಂತೆಯೊ, ಮಲಗಿರುವಂತೆಯೊ, ಅಥವಾ ತನ್ನ ಹೃದಯದಲ್ಲಿ ನೆಲೆಸಿರುವಂತೆಯೊ ಭಾವಿಸಬಹುದು. ಹೀಗೆ ನಿಶ್ಚಲವಾದ ನಂಬುಗೆ ಯಿಂದ ಧ್ಯಾನಿಸುತ್ತಾ ಬಂದರೆ ಮನಸ್ಸಿನ ಚಾಂಚಲ್ಯ ಹೋಗುತ್ತದೆ. ಒಮ್ಮೆ ಭಗವಂತನ ಅಖಂಡಮೂರ್ತಿಯನ್ನು ಧ್ಯಾನಿಸಿದಮೇಲೆ ಕಾಲಿನಿಂದ ತಲೆಯವರೆಗೆ ಆತನ ಒಂದೊಂದು ಅವಯವದ ಸೌಂದರ್ಯವನ್ನು ಸವಿಯುತ್ತಾ ಧ್ಯಾನತತ್ಪರನಾಗಬೇಕು. ಬಾಹ್ಯ ಜಗತ್ತನ್ನು ಮರೆತು ಆ ಧ್ಯಾನದಲ್ಲಿ ತಲ್ಲೀನನಾಗುವುದನ್ನೇ ‘ಸಮಾಧಿ’ ಎನ್ನುವುದು. ಈ ಸಮಾಧಿಸ್ಥಿತಿಯಲ್ಲಿ ಯೋಗಿಯಾದವನು ಆನಂದಮಯನಾಗಿ ಹೋಗುತ್ತಾನೆ. ಆತನ ದೇಹ ಪುಳಕಿತವಾಗುತ್ತದೆ, ಕಣ್ಣುಗಳಿಂದ ಆನಂದಬಾಷ್ಪ ಸುರಿಯುತ್ತದೆ, ಮೀನನ್ನು ಹಿಡಿ ಯುವ ಗಾಳದಂತೆ ದೇವರನ್ನು ಕಾಣಲು ಬಳಸುತ್ತಿದ್ದ ಮನಸ್ಸನ್ನೂ ದೂರಮಾಡುತ್ತಾನೆ, ಅನಿರ್ವಚನೀಯವಾದ ಆತ್ಮಸುಖವನ್ನು ಪಡೆಯುತ್ತಾನೆ. ಆಗ ಅವನು ಜೀವನ್ಮುಕ್ತನೆನಿಸು ತ್ತಾನೆ. ಆತನಿಗೆ ಸುಖದುಃಖಗಳಿಗೆ ಕಾರಣವಾದ ದೇಹಾಭಿಮಾನ ಅಳಿಸಿಹೋಗುತ್ತದೆ. ಮದ್ಯವನ್ನು ಕುಡಿದವನು ಉಟ್ಟ ಬಟ್ಟೆಯನ್ನು ಅರಿಯದಿರುವಂತೆ ಯೋಗಿಯು ತನ್ನ ಶರೀರವನ್ನು ಮರೆತುಬಿಡುತ್ತಾನೆ. ಹೆಂಡಿರು ಮಕ್ಕಳು ತನ್ನವರಾದರೂ ತನ್ನಿಂದ ಹೇಗೆ ಪ್ರತ್ಯೇಕವೋ ಹಾಗೆಯೇ ದೇಹವೂ ಆತ್ಮನಿಂದ ಪ್ರತ್ಯೇಕವೆಂಬುದು ಮನದಟ್ಟಾಗುತ್ತದೆ. ಆತನು ಮತ್ತೆ ಸಂಸಾರಬಂಧನಕ್ಕೆ ಸಿಕ್ಕಿ ಬೀಳುವುದಿಲ್ಲ. ಜ್ಯೋತಿಯಲ್ಲಿ ಜ್ಯೋತಿ ಒಂದಾಗುವಂತೆ ಆತನ ಆತ್ಮವು ಪರಮಾತ್ಮನಲ್ಲಿ ಒಂದಾಗಿಹೋಗುತ್ತದೆ.”

