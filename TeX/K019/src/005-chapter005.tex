
\chapter{೫. ವಿದುರ ಉದ್ಧವರ ಸಮಾಗಮ}

ಪರೀಕ್ಷಿದ್ರಾಜನು ಶುಕಮುನಿಯನ್ನು ಕುರಿತು ‘ಮಹಾನುಭಾವ, ವಿದುರನು ತೀರ್ಥ ಯಾತ್ರೆ ಹೋದುದೇಕೆ? ಆತನು ಆ ಕಾಲದಲ್ಲಿ ಮೈತ್ರೇಯರನ್ನು ಕಂಡನೆಂದು ಹೇಳಿದೆ ಯಲ್ಲವೆ? ಶ್ರೀಕೃಷ್ಣಭಕ್ತನಾದ ವಿದುರನೂ ಮಹಾತಪಸ್ವಿಗಳಾದ ಮೈತ್ರೇಯರೂ ಸೇರಿ ದಾಗ ನಡೆದ ಸಂಭಾಷಣೆ ಭಕ್ತಿಭರಿತವಾದುದಾಗಿಯೇ ಇರಬೇಕು. ನನ್ನಲ್ಲಿ ಅನುಗ್ರಹ ವಿಟ್ಟು ಅದನ್ನು ತಿಳಿಸು’ ಎಂದು ಕೇಳಿಕೊಂಡನು. ಶುಕಮುನಿಯು ಆತನಿಗೆ ಉತ್ತರ ಕೊಡುತ್ತಾ ‘ರಾಜನೇ, ವಿದುರನು ತೀರ್ಥಯಾತ್ರೆ ಹೊರಟುದಕ್ಕೆ ಕಾರಣವೇನೆಂಬುದನ್ನು ಮೊದಲು ಕೇಳು. ಕೌರವರಾಜನಾದ ಧೃತರಾಷ್ಟ್ರನು ದುಷ್ಟರಾದ ತನ್ನ ಮಕ್ಕಳನ್ನು ಶಿಕ್ಷಿಸಲಾರದೆ, ದಿಕ್ಕಿಲ್ಲದ ಪಾಂಡವರು ಪರಿಪರಿಯ ಸಂಕಟಗಳಿಗೆ ಒಳಗಾಗುವಂತೆ ಮಾಡಿ ದನು. ಪಾಪ, ಆ ಪಾಂಡವರನ್ನು ಅರಗಿನ ಮನೆಯಲ್ಲಿ ಸೇರಿಸಿ ಸುಡಿಸುವುದಕ್ಕೆ ಅವಕಾಶ ಕೊಟ್ಟನು; ಸ್ವಂತ ಸೊಸೆಯಾದ ದ್ರೌಪದಿಯನ್ನು ದುರುಳನಾದ ದುಶ್ಶಾಸನನು ತಲೆ ಗೂದಲು ಹಿಡಿದು ಎಳೆಯುತ್ತಾ, ಗಳಗಳ ಅಳುತ್ತಿದ್ದ ಆ ಹೆಣ್ಣನ್ನು ತುಂಬಿದ ಸಭೆಯಲ್ಲಿ ಅವಮಾನಪಡಿಸಿದರೂ ಆ ಕುರುಡದೊರೆ ತನ್ನ ಮಕ್ಕಳನ್ನು ದಂಡಿಸಲಿಲ್ಲ. ಮೋಸದ ಜೂಜಿನಲ್ಲಿ ಸರ್ವಸ್ವವನ್ನೂ ಕಳೆದುಕೊಂಡು, ಹನ್ನೆರಡು ವರುಷ ವನವಾಸ, ಒಂದು ವರ್ಷ ಅಜ್ಞಾತವಾಸವನ್ನು ಅನುಭವಿಸಿದ ಧರ್ಮರಾಜನು, ಅಲ್ಲಿಂದ ಹಿಂದಿರುಗಿದ ಮೇಲೆ ತನ್ನ ಭಾಗದ ರಾಜ್ಯವನ್ನು ಕೊಡಿಸಿಕೊಡುವಂತೆ ಬೇಡಿಕೊಂಡರೂ ಮದಾಂಧನಾದ ಧೃತರಾಷ್ಟ್ರನು ಕಿವುಡುಗೇಳಿ ಧಿಕ್ಕರಿಸಿದನು. ನೊಂದ ಧರ್ಮರಾಯನು ಶ್ರೀಕೃಷ್ಣನನ್ನು ಕೌರವರ ಬಳಿಗೆ ಸಂಧಿಗಾಗಿ ಕಳುಹಿಸಿದನು. ಆಗ ಶ್ರೀಕೃಷ್ಣನು ಆಡಿದ ಮಾತುಗಳು ಎಲ್ಲರಿಗೂ ಮೆಚ್ಚಿಗೆಯಾದರೂ ದುಷ್ಟನಾದ ಧೃತರಾಷ್ಟನಿಗೆ ಮಾತ್ರ ಹಿಡಿಸಲಿಲ್ಲ. ಅವನು ವಿದುರನನ್ನು ಕರೆಸಿ, ಅವನೊಡನೆ ಮಂತ್ರಾಲೋಚನೆ ಮಾಡಿದನು. ಆಗ ವಿದುರನು ‘ಮಹಾರಾಜ, ನಿನ್ನ ಮಕ್ಕಳು ಮಾಡಿದ ಅಕ್ಷಮ್ಯ ಅಪರಾಧಗಳನ್ನೆಲ್ಲಾ ಕ್ಷಮಿಸಿ, ಧರ್ಮ ರಾಯನು ಅರ್ಧರಾಜ್ಯಕ್ಕಾಗಿ ನಿನ್ನನ್ನು ಬೇಡುತ್ತಿರುವನು. ಆತನಿಗೆ ಅರ್ಧರಾಜ್ಯವನ್ನು ಕೊಟ್ಟುಬಿಡು, ನಿನ್ನ ಮಕ್ಕಳ ಹೆಸರನ್ನು ಕೇಳುತ್ತಲೆ ಭೀಮನು ಕೆರಳಿದ ಕಾಳಸರ್ಪದಂತೆ ಕ್ರೋಧದಿಂದ ಬುಸುಗುಟ್ಟುವನು. ಅವನಿಂದ ನಿನ್ನ ಮಕ್ಕಳಿಗೆ ಉಳಿಗಾಲವಿಲ್ಲ. ನೋಡು, ಶ್ರೀಕೃಷ್ಣನು ಪಾಂಡವರ ಪಕ್ಷ ವಹಿಸಿದ್ದಾನೆ. ಆದ್ದರಿಂದ ದೇವತೆಗಳೂ ಬ್ರಾಹ್ಮಣರೂ ಅವರ ಪರವಾಗಿರುವರು. ಶ್ರೀಕೃಷ್ಣನು ಲೋಕದಲ್ಲಿರುವ ರಾಜರುಗಳನ್ನೆಲ್ಲಾ ಗೆದ್ದು ಧರ್ಮರಾಯನ ವಶವರ್ತಿಯಾಗಿ ಮಾಡಿರುವನು. ಧರ್ಮರಾಯನೂ ಸಾಮಾನ್ಯನಲ್ಲ. ಯುಧಿಷ್ಠಿರನೆಂಬ ಆತನ ಹೆಸರೇ ಸೂಚಿಸುವಂತೆ ಯುದ್ಧದಲ್ಲಿ ಆತನು ಸುಸ್ಥಿರನಾದವನು. ನಿನ್ನ ಮಗನಾದರೊ ಶ್ರೀಕೃಷ್ಣನ ದ್ವೇಷಿ. ಇದರ ಫಲವಾಗಿ ನಿನ್ನ ಸಕಲ ಐಶ್ವರ್ಯವೂ ವಿನಾಶಹೊಂದುತ್ತದೆ. ಈಗಲೂ ಕಾಲಮೀರಿಲ್ಲ. ನಿನ್ನ ಕುಲಕ್ಕೆ ಅಪಕೀರ್ತಿ ಬರದಂತೆ ನಿನ್ನ ಆ ನೀಚ ಮಗನನ್ನು ಪರಿತ್ಯಜಿಸು’ ಎಂದು ಹೇಳಿದನು.

ವಿದುರನು ಧೃತರಾಷ್ಟ್ರನಿಗೆ ವಿವೇಕ ಹೇಳುತ್ತಿದ್ದಾಗ ದುರ್ಯೋಧನನೂ ಅಲ್ಲಿಯೇ ಇದ್ದನು. ವಿದುರನ ನುಡಿಗಳಿಂದ ಅವನ ರೋಷ ಹೊತ್ತಿತು. ತನ್ನ ಗೆಳೆಯರಾದ ಕರ್ಣ ಶಕುನಿ ದುಶ್ಶಾಸನರೊಡನೆ ಅವನು ಕಟಕಟ ಹಲ್ಲು ಕಡಿಯುತ್ತಾ ‘ಅಯ್ಯೊ, ಈ ಪಾಪಿ ನಮ್ಮ ಅನ್ನವನ್ನು ತಿಂದು ನಮಗೇ ಎರಡು ಬಗೆಯುತ್ತಿರುವನಲ್ಲಾ! ಶತ್ರುಗಳಿಗೆ ಸಹಾಯಕನಾಗಿ ತಿಂದಮನೆಗೆ ದ್ರೋಹ ಬಗೆಯುತ್ತಿರವ ಈ ತೊತ್ತಿನ ಮಗನನ್ನು ಇಲ್ಲಿಗೆ ಕರೆದವರಾರು? ಈ ನೀಚನನ್ನು ಈಗಲೇ ನಮ್ಮ ಪಟ್ಟಣದಿಂದ ಹೊಡೆದು ಓಡಿಸಿರಿ’ ಎಂದು ಅಬ್ಬರಿಸಿದನು. ತನ್ನ ಅಣ್ಣನ ಇದಿರಿನಲ್ಲಿಯೇ ತನಗಾದ ಈ ಅವಮಾನವನ್ನು ಸಹಿಸ ಲಾರದೆ ವಿದುರನು ‘ಈ ಪಾಪಿಗಳಿಂದ ತಳ್ಳಿಸಿಕೊಂಡು ಹೋಗುವುದೇಕೆ? ನಾನೇ ಹೊರಟು ಹೋಗುತ್ತೇನೆ’ ಎಂದುಕೊಂಡು, ಕೈಲಿದ್ದ ಧನುಸ್ಸನ್ನು ಆ ಮನೆಯ ಬಾಗಿಲಿನಲ್ಲಿಯೇ ಬಿಸುಟು, ಕೌರವರ ಭಾಗ್ಯವೇ ಹೊರಟುಹೋದಂತೆ ಹಸ್ತಿನಾವತಿಯಿಂದ ಹೊರಟು ಹೋದನು. ಹಾಗೆ ಹೊರಟವನು ಭಗವಂತನ ಸಾನ್ನಿಧ್ಯದಿಂದ ಪವಿತ್ರವಾಗಿದ್ದ ಪುಣ್ಯ ಕ್ಷೇತ್ರಗಳಲ್ಲಿ ಸಂಚರಿಸುತ್ತಿದ್ದನು. ಅಲ್ಲಲ್ಲಿ ಸಿಕ್ಕ ಪುಣ್ಯನದಿಗಳಲ್ಲಿ ಸ್ನಾನಮಾಡುತ್ತಾ, ದೇವತಾವಿಗ್ರಹಗಳಿಗೆ ನಮಸ್ಕರಿಸುತ್ತಾ, ಆತನು ಏಕಾಂಗಿಯಾಗಿ, ಅವಧೂತವೇಷದಿಂದ ಸುತ್ತುತ್ತಿದ್ದನು. ನಾರುಬಟ್ಟೆ ಕೃಷ್ಣಾಜಿನಗಳನ್ನು ಧರಿಸಿ, ವ್ರತ ಉಪವಾಸಗಳನ್ನು ಕೈಕೊಂಡು, ಕೃಶನಾಗಿ ಹೋಗಿದ್ದ ಆತನನ್ನು ಗುರುತು ಹಿಡಿಯುವುದು ಕೂಡ ಕಷ್ಟ ವಾಗಿತ್ತು. ಹೀಗೆ ಆತನು ಭರತಖಂಡವನ್ನೆಲ್ಲಾ ಸುತ್ತಿಕೊಂಡು ಪ್ರಭಾಸತೀರ್ಥವನ್ನು ಸೇರಿದನು. ಆತನು ಅಲ್ಲಿರುವಾಗ, ಬಿದಿರುಮೆಳೆಗಳು ಒಂದಕ್ಕೊಂದು ಉಜ್ಜಿ, ಅದರ ಬೆಂಕಿಯಿಂದ ವನವೆಲ್ಲವೂ ದಹಿಸುವಂತೆ ಕೌರವರು ಪರಸ್ಪರ ಹೊಡೆದಾಡಿ ಸತ್ತರೆಂಬ ಸುದ್ದಿ ಬಂದಿತು. ಅಲ್ಲದೆ ಧರ್ಮರಾಯನು ಚಕ್ರವರ್ತಿಪದವಿಯನ್ನು ವಹಿಸಿಕೊಂಡು ಅತ್ಯಂತ ವೈಭವದಿಂದ ರಾಜ್ಯಭಾರ ಮಾಡುತ್ತಾ ಅಜಾತ ಶತ್ರುವೆಂಬ ಕೀರ್ತಿಗೆ ಪಾತ್ರನಾಗಿರುವನೆಂಬ ಸುದ್ದಿಯೂ ಗೊತ್ತಾಯಿತು. ಕೌರವರ ದುರಂತವನ್ನು ಕೇಳಿ ವಿದುರ ನಿಗೆ ವ್ಯಥೆಯಾಯಿತಾದರೂ ‘ಚಿಂತಿಸಿ ಫಲವಿಲ್ಲ’ ಎಂದುಕೊಂಡು ತನ್ನ ಪ್ರಯಾಣವನ್ನು ಮುಂದುವರಿಸಿದನು. ಅಲ್ಲಿಂದ ಆತನು ಪಶ್ಚಿಮಕ್ಕೆ ಹೊರಟು ಸರಸ್ವತೀ ನದೀತೀರವನ್ನು ಸೇರಿ, ಅಲ್ಲಿ ತ್ರಿತ, ಉಶನ ಮೊದಲಾದ ಹೆಸರುಳ್ಳ ಹನ್ನೊಂದು ಪುಣ್ಯತೀರ್ಥಗಳಲ್ಲಿ ಸ್ನಾನಮಾಡಿದನು. ಅಲ್ಲಿ ಶಂಖಚಕ್ರಾದಿ ಅಂಕಿತಗಳಿಂದ ಕೂಡಿದ ಗೋಪುರಗಳುಳ್ಳ ಅನೇಕ ದೇವಾಲಯಗಳಿದ್ದುವು. ಪುಷಿಗಳೂ ದೇವತೆಗಳೂ ಪ್ರತಿಷ್ಠಿಸಿದ್ದ ಅಲ್ಲಿನ ವಿಷ್ಣು ಮೂರ್ತಿಗಳಿಗೆ ಭಕ್ತಿಯಿಂದ ನಮಿಸಿ ಆತನು ಮುಂದಕ್ಕೆ ಪ್ರಯಾಣ ಮಾಡಿದನು. ಸೌವೀರ, ಸೌರಾಷ್ಟ್ರ ಮೊದಲಾದ ದೇಶಗಳನ್ನು ದಾಟಿ ಆತನು ಯಮುನಾನದಿಯ ತೀರವನ್ನು ಸೇರಿ ದನು.

ವಿದುರನು ಯಮುನಾನದೀ ತೀರದಲ್ಲಿರುವಾಗ ಶ್ರೀಕೃಷ್ಣನ ಭಕ್ತನಾದ ಉದ್ಧವನು ಅಕಸ್ಮಾತ್ತಾಗಿ ಅಲ್ಲಿಗೆ ಬಂದನು. ಅವನನ್ನು ಕಾಣುತ್ತಲೆ ವಿದುರನಿಗೆ ಅತ್ಯಾನಂದ ವಾಯಿತು. ಆತನನ್ನು ಗಾಢಾಲಿಂಗನ ಮಾಡಿಕೊಂಡು, ಆತನೊಡನೆ ‘ಅಯ್ಯಾ, ಬ್ರಹ್ಮನ ಪ್ರಾರ್ಥನೆಯಂತೆ ಲೋಕೋದ್ಧಾರಕ್ಕಾಗಿ ಅವತರಿಸಿದ ಶ್ರೀಕೃಷ್ಣ ಬಲರಾಮರೂ ಅವರ ಪತ್ನಿಪುತ್ರಾದಿ ಪರಿವಾರದವರೂ ಆರೋಗ್ಯವಾಗಿರುವರೇ? ಪಂಚಪಾಂಡವರು ದ್ರೌಪದಿ ಕುಂತಿಯರ ಸಹಿತವಾಗಿ ಸುಖದಿಂದಿರುವರೇ? ಆ ಕುರುಡ ದೊರೆ ಧೃತರಾಷ್ಟ್ರನ ಗತಿ ಯೇನಾಗಿದೆ? ಮಕ್ಕಳ ಮೇಲಿನ ಮೋಹದಿಂದ ಆತ ಪಾಂಡವರಿಗೆ ಬಹಳ ದ್ರೋಹವನ್ನು ಮಾಡಿದ. ಆತನ ಕ್ಷೇಮಕ್ಕಾಗಿಯೇ ಬುದ್ಧಿ ಹೇಳಲು ಹೊರಟ ನನ್ನನ್ನು ಮನೆಯಿಂದ ಓಡಿಸಿದ. ಹೋಗಲಿ, ಅದು ಒಳ್ಳೆಯದೇ ಆಯಿತು. ಆ ದುಷ್ಟ ಸಹವಾಸದಿಂದ ಹೊರಟು ಬರುವುದಕ್ಕೆ ಒಳ್ಳೆಯ ನೆಪ ಸಿಕ್ಕಿದಂತಾಯಿತು. ದುಷ್ಟನಿಗ್ರಹ ಶಿಷ್ಟಪರಿಪಾಲನೆಗಾಗಿ ಜನಿಸಿದ ಶ್ರೀಕೃಷ್ಣನು ಕೌರವರನ್ನು ಧ್ವಂಸಮಾಡಿ ಪಾಂಡವರನ್ನು ಉದ್ಧರಿಸಿದ. ಈಗ ಶ್ರೀಕೃಷ್ಣನು ಏನು ಮಾಡುತ್ತಿರುವನು? ಆತನ ಸಮಾಚಾರವನ್ನು ತಿಳಿಸು’ ಎಂದು ಕೇಳಿದನು.

ವಿದುರನ ಮಾತುಗಳನ್ನು ಮೌನವಾಗಿ ಕುಳಿತು ಕೇಳುತ್ತಿದ್ದ ಉದ್ಧವನು ದೊಡ್ಡ ದೊಂದು ನಿಟ್ಟುಸಿರನ್ನು ಬಿಟ್ಟನು. ಆತನಿಗೆ ಶ್ರೀಕೃಷ್ಣನಲ್ಲಿ ಅಪಾರವಾದ ಭಕ್ತಿ. ಐದು ವರ್ಷದ ಬಾಲಕನಾಗಿರುವಾಗಲೇ ಆತನು ಶ್ರೀಕೃಷ್ಣನನ್ನು ಪೂಜಿಸುವ ಆಟದಲ್ಲಿ ತೊಡಗಿರುತ್ತಿದ್ದನು. ಆಗ ಅವನನ್ನು ಊಟಕ್ಕೆ ಎಬ್ಬಿಸುವುದು ಕೂಡ ಕಷ್ಟವಾಗುತ್ತಿತ್ತು. ಹೀಗೆ ಚಿಕ್ಕಂದಿನಿಂದಲೇ ಶ್ರೀಕೃಷ್ಣನ ಭಕ್ತನಾಗಿದ್ದ ಉದ್ಧವ ಶ್ರೀಕೃಷ್ಣನ ಹೆಸರನ್ನು ಕೇಳುತ್ತಲೆ ಸಂತೋಷದಿಂದ ರೋಮಾಂಚನಗೊಂಡರೂ ಆತನ ಅಗಲಿಕೆಯನ್ನು ನೆನೆದು ದುಃಖದಿಂದ ಕಣ್ಣೀರುಗರೆಯುತ್ತಾ ‘ವಿದುರ, ಶ್ರೀಕೃಷ್ಣನೆಂಬ ಸೂರ್ಯ ಮುಳುಗಿ ಹೋದ. ಕಾಲವೆಂಬ ಮಹಾಸರ್ಪ ಯಾದವರನ್ನೆಲ್ಲಾ ನುಂಗಿತು. ನಾನು ಏನು ಕ್ಷೇಮಸಮಾಚಾರ ವನ್ನು ನಿನಗೆ ತಿಳಿಸಲಿ? ಈಗ ಸಮಸ್ತ ಭೂಮಂಡಲವೇ ನಿರ್ಭಾಗ್ಯವಾಯಿತು. ನೋಡು, ಈ ಯಾದವರೆಲ್ಲಾ ಸದಾ ಶ್ರೀಕೃಷ್ಣನ ಜೊತೆಯಲ್ಲಿಯೇ ಇರುತ್ತಿದ್ದರಾದರೂ ಆತನು ಸರ್ವೇಶ್ವರನೆಂದು ಅರ್ಥಮಾಡಿಕೊಳ್ಳಲಾರದೆ ಹೋದರು. ಮೀನುಗಳು ಸಮುದ್ರದಲ್ಲಿನ ಹಡಗುಗಳನ್ನು ನೋಡಿ, ಅವೂ ತಮ್ಮಂತೆಯೇ ಒಂದು ಮೀನೆಂದು ಭಾವಿಸುವಂತೆ ಈ ಯಾದವರು ಶ್ರೀಕೃಷ್ಣ ಪರಾಮಾತ್ಮನನ್ನು ತಮ್ಮಂತೆಯೇ ಒಬ್ಬ ಮನುಷ್ಯನೆಂದು ಭಾವಿಸಿ ದರು. ವಿದುರ, ಈ ಯಾದವರು ಅತ್ಯಂತ ಬುದ್ಧಿವಂತರು, ಪರೇಂಗಿತಜ್ಞರು. ಆದರೇನು? ಭಗವಂತನ ಮಾಯೆಗೆ ಸಿಕ್ಕಿ ಆತನನ್ನು ಬಂಧುವೆಂದು ಭ್ರಮಿಸಿದರು; ಶಿಶುಪಾಲನೇ ಮೊದಲಾದವರು ಆತನನ್ನು ತಮ್ಮ ಶತ್ರುವೆಂದು ಬಗೆದರು. ಆತನು ಯಾರ ಶತ್ರು ಮಿತ್ರನೂ ಅಲ್ಲದ ಪರಮಾತ್ಮನೆಂದು ಆತನಲ್ಲಿಯೇ ನೆಟ್ಟ ಮನಸ್ಸುಳ್ಳ ಕೆಲವು ಮಹಾತ್ಮರು ಮಾತ್ರ ಅರ್ಥಮಾಡಿಕೊಂಡಿದ್ದರು. ಮಹಾತ್ಮನಾದ ಶ್ರೀಕೃಷ್ಣನು ಯೋಗಿ ಗಳಿಗೆ ಮಾತ್ರ ಗೋಚರಿಸಬಲ್ಲ ತನ್ನ ದಿವ್ಯರೂಪವನ್ನು ಈ ಪ್ರಾಕೃತರಿಗೆಲ್ಲಾ ಕೆಲವು ಕಾಲ ತೋರಿಸುತ್ತಿದ್ದು ಈಗ ಅಂತರ್ಧಾನನಾದ.

ವಿದುರ, ಏನು ಆಶ್ಚರ್ಯವಿದು! ಮನುಷ್ಯಲೀಲೆಗೆ ತಕ್ಕುದಾಗಿದ್ದು, ಆಭರಣಗಳಿಗೇ ಆಭರಣದಂತಿದ್ದ ಆ ಶ್ರೀಕೃಷ್ಣನ ದಿವ್ಯಾಕೃತಿ ಈಗ ಕಣ್ಮರೆಯಾಯಿತಲ್ಲಾ! ಧರ್ಮರಾಯನ ರಾಜಸೂಯ ಯಾಗದಲ್ಲಿ ನೆರೆದಿದ್ದ ಮೂರು ಲೋಕದ ಜನರೂ ಆ ದಿವ್ಯಮಂಗಳ ವಿಗ್ರಹ ವನ್ನು ಕಂಡು ಬ್ರಹ್ಮನ ಕಲಾಕೌಶಲದ ಸಾರಸರ್ವಸ್ವವೆಂದು ಅದನ್ನು ಹೊಗಳಿದರಲ್ಲವೆ? ಗೋಕುಲದ ಗೋಪಿಯರೆಲ್ಲರೂ ಆತನ ಸೌಂದರ್ಯಕ್ಕೆ ಮರುಳಾಗಿ ತಮ್ಮ ಮನೆಮಠ ಗಳನ್ನು ಕೂಡ ಮರೆತರು! ಅಂತಹ ಸುಂದರಮೂರ್ತಿ ಇಂದು ಕಣ್ಮರೆಯಾಗಿಹೋಯಿತು. ದುಷ್ಟರ ಬಾಧೆಗೊಳಗಾಗಿದ್ದ ಸಜ್ಜನರನ್ನು ಸಂರಕ್ಷಿಸುವುದಕ್ಕಾಗಿ ಜನ್ಮರಹಿತನಾದ ಆ ಪರಮೇಶ್ವರ ಮನುಷ್ಯಾವತಾರವನ್ನು ತಾಳಿ ಬಂದ ಸಾಮಾನ್ಯ ಮನುಷ್ಯನಂತೆ ನಟಿಸುತ್ತಾ ಆತ ತೋರಿದ ಲೀಲೆಗಳು ಎಷ್ಟು ಸುಂದರ! ವಸುದೇವನ ಮಗನಾಗಿ ಹುಟ್ಟಿ, ಕಂಸನಿಗೆ ಹೆದರಿದವನಂತೆ ನಂದಗೋಕುಲದಲ್ಲಿ ನಿಂತ; ಜರಾಸಂಧಾದಿಗಳಿಗೆ ಹೆದರಿದವನಂತೆ ಸಮುದ್ರ ಮಧ್ಯದಲ್ಲಿ ನೆಲಸಿದ; ಒಮ್ಮೆ ಹುಬ್ಬು ಗಂಟುಹಾಕಿದ ಮಾತ್ರದಿಂದಲೆ ಲೋಕ ಕಂಟಕರೆಲ್ಲರೂ ನಿರ್ಮೂಲವಾಗುವಂತಿದ್ದರೂ ಆತನ ಆ ನಟನೆ ಏನು ಚಂದ! ಅಂತಹ ಮಹಾನುಭಾವ ಕಂಸನನ್ನು ಕೊಂದಮೇಲೆ ವಸುದೇವ ದೇವಕಿಯರ ಪಾದಕ್ಕೆ ಅಡ್ಡಬಿದ್ದು ‘ಅಪ್ಪ, ಅಮ್ಮ, ಈ ಪಾಪಿ ಕಂಸನಿಗೆ ಹೆದರಿ ಇಷ್ಟು ದಿನಗಳೂ ನಿಮ್ಮ ಪಾದಸೇವೆಯನ್ನು ಮಾಡಲಿಲ್ಲ; ಈ ಅಪಚಾರವನ್ನು ಮನ್ನಿಸಿರಿ’ ಎಂದು ಬೇಡಿದನಲ್ಲಾ! ನೋಡು, ನೀಚರಾದ ಶಿಶುಪಾಲಾದಿಗಳು ಶತ್ರುಗಳಾದರೂ ಆತನಿಂದ ಸತ್ತು ಯೋಗಿಗಳಿಗೂ ದುರ್ಲಭವಾದ ಮೋಕ್ಷವನ್ನು ಪಡೆದರು. ಕುರುಕ್ಷೇತ್ರದ ಯುದ್ಧದಲ್ಲಿ ಅರ್ಜುನನ ಬಾಣಕ್ಕೆ ಆಹುತಿ ಯಾದ ರಾಜರು ಶ್ರೀಕೃಷ್ಣನ ಮುಖವನ್ನು ನೋಡುತ್ತಾ ಸತ್ತು ಸದ್ಗತಿಯನ್ನು ಪಡೆದರು. ಸಮಸ್ತ ಸೃಷ್ಟಿಗೂ ಕಾರಣಭೂತನಾಗಿ ಸತ್ವರಜಸ್ತಮೋಗುಣಗಳಿಗೆ ಒಡೆಯನಾದ ಆತನ ಪಾದಕಮಲವನ್ನು ಸಾಕ್ಷಾತ್ ಲೋಕಪಾಲಕರೆಲ್ಲರೂ ಬಂದು ಪೂಜಿಸುವರು. ಇಂತಹ ಮಹಾಮಹಿಮನು ಉಗ್ರಸೇನನಿಗೆ ಯಾದವರಾಜ್ಯದ ಪಟ್ಟವನ್ನು, ಕಟ್ಟಿ ತಾನು ಆತನ ಸೇವಕನಂತೆ ಸಿಂಹಾಸನದ ಮುಂದೆ ‘ಭೊ, ಮಹಾರಾಜ, ಏನಪ್ಪಣೆ?’ ಎಂದು ಓಲೈಸು ತ್ತಿದ್ದನು. ಶಿಶುವಾಗಿದ್ದಾಗಲೇ ಆ ಪೂತನಿಯನ್ನು ಕೊಂದುದೊ, ಶಕಟ ಧೇನುಕರನ್ನು ಬಲಿ ಹಾಕಿದುದೊ, ಕಾಳೀಯನನ್ನು ಮರ್ದಿಸಿದುದೊ, ಗೋವರ್ಧನಪರ್ವತವನ್ನು ಎತ್ತಿ ಹಿಡಿ ದುದೊ, ವೇಣುಗಾನದಿಂದ ಗೋಪಿಯರನ್ನು ಮೋಹಗೊಳಿಸಿದುದೊ–ಒಂದೆ, ಎರಡೆ ಆತನ ಲೀಲೆಗಳು? ವಿದುರ, ಎಳೆತನವನ್ನು ಗೋಕುಲದಲ್ಲಿ ಕಳೆದ ಮೇಲೆ ಮಧುರೆಗೆ ಬಂದ ಶ್ರೀಕೃಷ್ಣನು ಕಂಸನನ್ನು ಕೊಂದು ಕಾಶಿಗೆ ಹೋದನು. ಅಲ್ಲಿ ಸಾಂದೀಪಿನಿಯೆಂಬ ಉಪಾ ಧ್ಯಾಯನಲ್ಲಿ ವಿದ್ಯಾಭ್ಯಾಸ ಮಾಡಿದನು. ಅನಂತರ ರುಕ್ಮಿಣಿಯನ್ನು ಮದುವೆಯಾದನು. ಸತ್ಯಭಾಮೆಯನ್ನು ಸಂತೋಷಪಡಿಸುವವನಂತೆ, ಇಂದ್ರನ ಗರ್ವವನ್ನು ಮುರಿದು ಪಾರಿಜಾತವೃಕ್ಷವನ್ನು ಭೂಮಿಗೆ ತಂದನು. ನರಕಾಸುರನನ್ನು ಕೊಂದನು. ಅರವತ್ತು ಸಹಸ್ರ ಪತ್ನಿಯರೊಡನೆ ಅರವತ್ತು ಸಹಸ್ರ ಕೃಷ್ಣರೂಪಿಯಾಗಿ ವಿಹರಿಸಿದನು. ಪ್ರಕೃತಿ ತತ್ವದ ಬೆಳವಣಿಗೆಯನ್ನು ಜಗತ್ತಿಗೆ ತೋರುವಂತೆ ಹಲವು ಮಕ್ಕಳನ್ನು ಪಡೆದನು. ಜರಾ ಸಂಧಾದಿ ಹಲವು ಪಾಪಿಗಳನ್ನು ತಾನೆ ಕೊಂದನು; ಶಂಬರ ಮೊದಲಾದವರನ್ನು ಬಲ ರಾಮಾದಿಗಳಿಂದಲೂ, ಕೌರವರನ್ನು ಪಾಂಡವರಿಂದಲೂ ಕೊಲ್ಲಿಸಿ ಭೂಭಾರವಿಳುಹಿ ದನು.

ಇಷ್ಟಾದರೂ ಶ್ರೀಕೃಷ್ಣನ ಅವತಾರಕಾರ್ಯ ಮುಗಿಯಲಿಲ್ಲ. ಮದಾಂಧರಾದ ಯಾದವ ರನ್ನು ಕೊಲ್ಲಿಸುವುದೊಂದು ಉಳಿದಿತ್ತು. ಅದಕ್ಕೂ ತಕ್ಕ ಉಪಾಯವೊಂದನ್ನು ಆತ ಆಲೋಚಿಸಿದನು. ಒಮ್ಮೆ ದ್ವಾರಕೆಯಲ್ಲಿ ಯಾದವರು ಪುಷಿಗಳೊಡನೆ ಕುಚೇಷ್ಟೆ ಮಾಡಲು, ಅವರು ಕೋಪಗೊಂಡು ಯಾದವರೆಲ್ಲ ನಾಶವಾಗುವಂತೆ ಶಾಪ ಕೊಟ್ಟರು. ಇದರ ಫಲವಾಗಿ, ಒಂದು ದಿನ ಯಾದವರು ಪ್ರಭಾಸಕ್ಷೇತ್ರಕ್ಕೆ ಹೋಗಿ, ಪಿತೃಯಾಗಾದಿ ಗಳನ್ನು ಮಾಡಿ, ಅದರ ಅಂತ್ಯದಲ್ಲಿ ಮದ್ಯಪಾನ ಮಾಡಿದುದರಿಂದ ಪ್ರಜ್ಞೆತಪ್ಪಿ, ಪರಸ್ಪರ ಹೊಡೆದಾಡಿ ಎಲ್ಲರೂ ಸತ್ತುಹೋದರು. ಶ್ರೀಕೃಷ್ಣನು ಭಕ್ತನಾದ ನನ್ನನ್ನು ಉಳಿಸ ಬೇಕೆಂದು ಬಗೆದು, ಬದರಿಕಾಶ್ರಮಕ್ಕೆ ಹೋಗುವಂತೆ ನನಗೆ ಅಪ್ಪಣೆ ಮಾಡಿದನು. ಆ ಬಳಿಕ ಆತನು ಸರಸ್ವತೀ ನದಿಯಲ್ಲಿ ಆಚಮನ ಮಾಡಿ, ಅಲ್ಲಿಯೇ ಇದ್ದ ಒಂದು ಅರಳಿಯ ಮರದ ಬುಡದಲ್ಲಿ ಕುಳಿತನು. ನಾನು ಆತನನ್ನು ಅಗಲಿಹೋಗಲಾರದೆ ಆತನ ಬಳಿ ಯಲ್ಲಿಯೇ ನಿಂತಿದ್ದೆನು. ಅಷ್ಟು ಹೊತ್ತಿಗೆ ಭಗವದ್ಭಕ್ತನಾದ ಮೈತ್ರೇಯ ಪುಷಿಯು ಲೋಕಸಂಚಾರ ಮಾಡುತ್ತಾ ಅಲ್ಲಿಗೆ ಬಂದನು. ಶ್ರೀಕೃಷ್ಣನು ಆತನೊಡನೆ ಮಾತನಾಡದೆ, ಮುಗುಳ್ನಗೆಯನ್ನು ಚೆಲ್ಲುವ ತನ್ನ ಮುಖದಿಂದ ನನ್ನೊಡನೆ ‘ಉದ್ಧವ, ನೀನು ಹಿಂದೆ ವಸುವಾಗಿದ್ದಾಗ, ನನ್ನ ಅನುಗ್ರಹದಿಂದ ನನ್ನನ್ನು ಪೂಜಿಸಿರುವೆ. ಆದ್ದರಿಂದಲೇ ಇತರರಿಗೆ ದುರ್ಲಭವಾದ ವರವನ್ನು ನಾನು ನಿನಗೆ ಕರುಣಿಸುವೆನು. ನಾನು ಲೋಕವನ್ನು ಬಿಟ್ಟು ವೈಕುಂಠಕ್ಕೆ ತೆರಳಲಿರುವಾಗ ನೀನು ಏಕಾಂತಭಕ್ತಿಯಿಂದ ನನ್ನ ದರ್ಶನವನ್ನು ತೆಗೆದು ಕೊಂಡಿರುವೆ. ನಿನಗೆ ಪುನರ್ಜನ್ಮವಿಲ್ಲದ ಶಾಶ್ವತಪದವಿ ಪ್ರಾಪ್ತವಾಗಲಿ. ಅಲ್ಲದೆ ನಾನು ಹಿಂದೆ ಚತುರ್ಮುಖ ಬ್ರಹ್ಮನಿಗೆ ಉಪದೇಶಿಸಿದ ಭಾಗವತವನ್ನು ನಿನಗೂ ಉಪದೇಶಿಸು ವೆನು’ ಎಂದು ಹೇಳಿದನು. ಆತನ ನುಡಿಗಳಿಂದ ನನಗೆ ಸಂತೋಷದಿಂದ ರೋಮಾಂಚ ವಾಯಿತು; ಕಣ್ಣುಗಳಿಂದ ಆನಂದಬಾಷ್ಪಗಳು ಸುರಿದವು; ಗದ್ಗದ ಸ್ವರದಿಂದ ಆತನನ್ನು ಕುರಿತು ‘ಪರಮೇಶ್ವರ, ನಿನ್ನ ಪಾದಸೇವೆಯೊಂದಲ್ಲದೆ ಬೇರಾವುದೂ ನನಗೆ ಬೇಡ. ಸರ್ವಜ್ಞನಾದ ನೀನು ಅನೇಕವೇಳೆ ನನ್ನ ಸಲಹೆಗಳನ್ನು ಕೇಳಿರುವೆ. ನಿನ್ನ ಪ್ರೇಮ ಅಪಾರ. ಬ್ರಹ್ಮನಿಗೆ ಮಾಡಿದ ಜ್ಞಾನೋಪದೇಶಕ್ಕೆ ನಾನೂ ಯೋಗ್ಯನಾದರೆ ಅಗತ್ಯವಾಗಿಯೂ ಅನುಗ್ರಹಮಾಡು’ ಎಂದು ಕೇಳಿಕೊಂಡೆನು. ಆತನು ನನಗೆ ತತ್ವೋಪದೇಶಮಾಡಿದನು. ಅನಂತರ ಆತನನ್ನು ಅಗಲುವುದು ಅತ್ಯಂತ ಸಂಕಟಕರವಾಗಿದ್ದರೂ, ಆತನ ಅಪ್ಪಣೆ ಯನ್ನು ಮೀರಲಾರದೆ ಅಲ್ಲಿಂದ ಹೊರಟು ಬಂದೆನು’ ಎಂದು ಹೇಳಿದನು.

ಯಾದವರೆಲ್ಲರೂ ನಾಶವಾದುದನ್ನು ಕೇಳಿ ವಿದುರನಿಗೆ ಸಂಕಟವಾಯಿತು. ಆದರೂ ಜ್ಞಾನಿಯಾದ ಆತನು ಮನಸ್ಸನ್ನು ಸಮಾಧಾನ ಮಾಡಿಕೊಂಡು ಉದ್ಧವನೊಡನೆ ‘ಮಿತ್ರ, ಶ್ರೀಕೃಷ್ಣಪರಮಾತ್ಮನು ನಿನಗೆ ಉಪದೇಶಿಸಿದ ಜ್ಞಾನಾಮೃತವನ್ನು ನಾನೂ ಸವಿಯಬೇಕೆಂ ದಿರುವೆನು. ಅದನ್ನು ನನಗೂ ಅನುಗ್ರಹಿಸು’ ಎಂದು ಕೇಳಿಕೊಂಡನು. ಆದರೆ ಉದ್ಧವನು ಆತನೊಡನೆ ‘ಅಯ್ಯಾ, ಶ್ರೀಕೃಷ್ಣ ಪರಮಾತ್ಮನು ದೇಹತ್ಯಾಗ ಮಾಡುವ ಮುನ್ನ ಜ್ಞಾನ ವನ್ನೆಲ್ಲಾ ನಿನಗೆ ಉಪದೇಶಿಸುವಂತೆ ಮೈತ್ರೇಯನಿಗೆ ಅಪ್ಪಣೆಮಾಡಿರುವನು. ಆದ್ದರಿಂದ ನೀನು ಆತನಿಂದ ಉಪದೇಶವನ್ನು ಪಡೆ’ ಎಂದು ಹೇಳಿದನು. ಗೆಳೆಯರಿಬ್ಬರೂ ಆ ರಾತ್ರಿ ಯಮುನಾನದಿಯ ಮರಳದಿಣ್ಣೆಯ ಮೇಲೆ ಕುಳಿತು ಪರಸ್ಪರ ಮಾತನಾಡುತ್ತಾ, ಆ ರಾತ್ರಿ ಯನ್ನು ಕಳೆದರು. ಮರುದಿನ ಬೆಳಗ್ಗೆ ಉದ್ಧವನು ಬದರಿಕಾಶ್ರಮಕ್ಕೆ ಹೊರಟುಹೋದನು. ಇತ್ತ ವಿದುರನು ಶ್ರೀಕೃಷ್ಣನು ದೇಹತ್ಯಾಗ ಮಾಡಿದ್ದನ್ನು ನೆನೆದು ಬಹುವಾಗಿ ದುಃಖಿಸುತ್ತಾ ಅಲ್ಲಿಂದ ಹೊರಟು ಮೈತ್ರೇಯರು ವಾಸಿಸುತ್ತಿರುವ ಗಂಗಾತೀರದ ಕಡೆ ತೆರಳಿದನು.

