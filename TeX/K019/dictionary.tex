\sethyphenation{kannada}{
ನೇ
ಮೇ
ಅ
ಅ-ಅಗ್ನಿ
ಅಂಕಿ-ತ-ಗಳಿಂದ
ಅಂಕು-ರಿ-ಸದು
ಅಂಕು-ರಿ-ಸು-ವುದು
ಅಂಕುಶ
ಅಂಗ
ಅಂಗ-ಗ-ಳ-ಲ್ಲಿ-ರು-ವೆ-ಯೆಂದೊ
ಅಂಗ-ಗ-ಳು-ಳ್ಳ-ವನೆ
ಅಂಗ-ಗ-ಳೆಂ-ಬ-ರೂ-ಪು-ಗ-ಳೆಂಬ
ಅಂಗ-ಡಿ-ಗಳ
ಅಂಗ-ಡಿ-ಗಳು
ಅಂಗದ
ಅಂಗ-ದ-ಲ್ಲಿಯೂ
ಅಂಗ-ನೆಂ-ಬು-ವನು
ಅಂಗ-ನ್ಯಾಸ
ಅಂಗ-ಭೂ-ತ-ರಾದ
ಅಂಗ-ರ-ಕ್ಷ-ಕರು
ಅಂಗ-ರ-ಕ್ಷ-ಕರೂ
ಅಂಗ-ರಾ-ಜನ
ಅಂಗ-ರಾ-ಜನು
ಅಂಗ-ಲಾ-ಚಿ-ದನು
ಅಂಗ-ಲಾ-ಚು-ವುದನ್ನು
ಅಂಗಳ
ಅಂಗ-ಳ-ಕ್ಕಿ-ಳಿದು
ಅಂಗ-ಳಕ್ಕೆ
ಅಂಗ-ಳ-ಗ-ಳೆಲ್ಲ
ಅಂಗ-ಳ-ದಲ್ಲಿ
ಅಂಗ-ವನ್ನು
ಅಂಗ-ವ-ಸ್ತ್ರ-ವನ್ನು
ಅಂಗ-ವೆಂದರೆ
ಅಂಗ-ಸಂ-ಗ-ದಿಂದ
ಅಂಗಾಂ-ಗ-ಗಳ
ಅಂಗಾಂ-ಗ-ಗಳನ್ನು
ಅಂಗಾಂ-ಗ-ಗ-ಳಿ-ಗೆಲ್ಲ
ಅಂಗಾಂ-ಗ-ವೆ-ಲ್ಲವೂ
ಅಂಗಾ-ರಕ
ಅಂಗಾ-ಲಿ-ನಲ್ಲಿ
ಅಂಗಾ-ಲಿ-ನಿಂದ
ಅಂಗಾಲು
ಅಂಗಿ
ಅಂಗಿ-ರಸ
ಅಂಗಿ-ರ-ಸ-ಋ-ಷಿಯೇ
ಅಂಗಿ-ರ-ಸನ
ಅಂಗಿ-ರ-ಸ-ನೆಂಬ
ಅಂಗಿ-ರ-ಸ-ಮ-ಹ-ರ್ಷಿಯು
ಅಂಗಿ-ರ-ಸ-ಮ-ಹಾ-ಮು-ನಿಯ
ಅಂಗಿ-ರ-ಸರ
ಅಂಗಿ-ರ-ಸ-ರೆಂಬ
ಅಂಗಿ-ರಸ್ಸು
ಅಂಗಿ-ರಸ್ಸೂ
ಅಂಗೀ-ರ-ಸ-ನಿಗೆ
ಅಂಗುಲ
ಅಂಗು-ಲ-ವಾಗಿ
ಅಂಗುಷ್ಠ
ಅಂಗು-ಷ್ಠದ
ಅಂಗು-ಷ್ಠ-ದಿಂದ
ಅಂಗು-ಷ್ಠಾ-ಭ್ಯಾಂ
ಅಂಗೈ
ಅಂಗೈಗೆ
ಅಂಚಿ-ನಲ್ಲಿ
ಅಂಚು-ಳ್ಳ-ದಾಗಿ
ಅಂಜಿ
ಅಂಜು-ಗು-ಳಿ-ಯಾದ
ಅಂಜು-ಬು-ರಕ
ಅಂಜು-ವು-ದಿಲ್ಲ
ಅಂಟಿ
ಅಂಟಿ-ಕೊಂ-ಡಿದೆ
ಅಂಟಿ-ಕೊಂ-ಡಿ-ರುವ
ಅಂಟಿ-ಕೊಂ-ಡಿ-ರು-ವು-ದ-ರಿಂದ
ಅಂಟಿ-ಕೊಂಡೆ
ಅಂಟಿದ
ಅಂಟಿ-ಸಿ-ಕೊಂಡು
ಅಂಡ
ಅಂತ
ಅಂತಃ
ಅಂತಃ-ಕ-ರಣ
ಅಂತಃ-ಕ-ರ-ಣ-ಗಳೂ
ಅಂತಃ-ಪುರ
ಅಂತಃ-ಪು-ರಕ್ಕೆ
ಅಂತಃ-ಪು-ರದ
ಅಂತಃ-ಪು-ರ-ದಲ್ಲಿ
ಅಂತಃ-ಪು-ರ-ದ-ವ-ರಿ-ಗೆಲ್ಲ
ಅಂತಃ-ಪು-ರ-ದ-ವರು
ಅಂತಃ-ಪು-ರ-ದಿಂದ
ಅಂತಃ-ಪು-ರ-ವನ್ನು
ಅಂತ-ರಂಗ
ಅಂತ-ರಂ-ಗದ
ಅಂತ-ರಂ-ಗ-ದಲ್ಲಿ
ಅಂತ-ರಂ-ಗ-ದ-ಲ್ಲಿಯೂ
ಅಂತ-ರಾ-ತ್ಮ-ನಾಗಿ
ಅಂತ-ರಾ-ತ್ಮ-ನಾದ
ಅಂತ-ರಾ-ತ್ಮ-ವಾ-ಗಿ-ರುವ
ಅಂತ-ರಿಕ್ಷ
ಅಂತ-ರಿ-ಕ್ಷಕ್ಕೆ
ಅಂತ-ರಿ-ಕ್ಷ-ದ-ಲ್ಲಾ-ಗಲಿ
ಅಂತ-ರಿ-ಕ್ಷ-ದಿಂದ
ಅಂತ-ರಿ-ಕ್ಷ-ವ-ನ್ನೆಲ್ಲ
ಅಂತ-ರಿ-ಕ್ಷವು
ಅಂತ-ರಿ-ಕ್ಷವೂ
ಅಂತ-ರ್ಧಾನ
ಅಂತ-ರ್ಧಾ-ನ-ನಾದ
ಅಂತ-ರ್ಧಾ-ನ-ನಾ-ದನು
ಅಂತ-ರ್ಧಾ-ನ-ನೆಂಬ
ಅಂತ-ರ್ಧಾ-ನ-ವಾ-ದನು
ಅಂತ-ರ್ಧಾ-ನ-ವಾ-ದಾಗ
ಅಂತ-ರ್ಬ-ಹಿ-ರ್ಭ-ಗ-ವಾ-ನ್ನಾ-ರ-ಸಿಂಹಃ
ಅಂತ-ರ್ಬ-ಹಿಶ್ಚ
ಅಂತ-ರ್ಯಾಮಿ
ಅಂತ-ರ್ಯಾ-ಮಿ-ಯಾಗಿ
ಅಂತ-ರ್ಯಾ-ಮೀ-ಶ್ವ-ರ-ಸ್ಸಾ-ಕ್ಷಾ-ತ್ಪಾತು
ಅಂತಹ
ಅಂತ-ಹನು
ಅಂತ-ಹ-ರಲ್ಲಿ
ಅಂತ-ಹರು
ಅಂತ-ಹರೂ
ಅಂತ-ಹ-ವ-ನನ್ನು
ಅಂತ-ಹ-ವ-ನಲ್ಲ
ಅಂತ-ಹ-ವನು
ಅಂತ-ಹ-ವರು
ಅಂತ-ಹ-ವರೇ
ಅಂತ-ಹುದು
ಅಂತೂ
ಅಂತೆಯೇ
ಅಂತ್ಯ-ಕಾಲ
ಅಂತ್ಯ-ದಲ್ಲಿ
ಅಂತ್ಯ-ಸಂ-ಸ್ಕಾ-ರ-ಗಳನ್ನು
ಅಂದ
ಅಂದ-ಗೊ-ಳಿ-ಸಿ-ದರು
ಅಂದ-ಚೆಂ-ದ-ಗಳನ್ನು
ಅಂದ-ವಾದ
ಅಂದಿ
ಅಂದಿನ
ಅಂದಿ-ನಿಂದ
ಅಂದು
ಅಂದೆ
ಅಂದೇ
ಅಂಧಃ
ಅಂಧ-ಕ-ವೃಷ್ಣಿ
ಅಂಧ-ಕಾ-ರಕ್ಕೆ
ಅಂಧತಾ
ಅಂಬ-ರ-ದ-ಲ್ಲಿದ್ದ
ಅಂಬ-ರೀಶ
ಅಂಬ-ರೀಷ
ಅಂಬ-ರೀ-ಷನ
ಅಂಬ-ರೀ-ಷ-ನನ್ನು
ಅಂಬ-ರೀ-ಷ-ನನ್ನೆ
ಅಂಬ-ರೀ-ಷನು
ಅಂಬಾ
ಅಂಬಿಕಾ
ಅಂಬಿ-ಕಾ-ಗು-ಡಿ-ಗೆಂದು
ಅಂಬಿ-ಕಾ-ವ-ನ-ವೆಂಬ
ಅಂಬಿಕೆ
ಅಂಬಿ-ಕೆಗೆ
ಅಂಬಿ-ಕೆಯ
ಅಂಬಿ-ಕೆ-ಯರು
ಅಂಬಿ-ನಂತೆ
ಅಂಬೆಗಾ
ಅಂಬೆ-ಗಾ-ಲಿ-ಕ್ಕುತ್ತಾ
ಅಂಬೆ-ಗಾ-ಲಿ-ಡು-ತ್ತಿದ್ದ
ಅಂಬೆ-ಗಾ-ಲಿ-ಡುವ
ಅಂಶ
ಅಂಶಈ
ಅಂಶ-ಗಳಿಂದ
ಅಂಶ-ಗಳು
ಅಂಶದ
ಅಂಶ-ದಿಂದ
ಅಂಶ-ರೂ-ಪ-ಗಳು
ಅಂಶ-ವಲ್ಲ
ಅಂಶ-ವಾದ
ಅಂಶ-ವೆಂ-ಬು-ದನ್ನು
ಅಂಶಾ-ವ-ತಾ-ರ-ಗಳು
ಅಂಶು-ಮಂತ
ಅಂಶು-ಮಂ-ತ-ನಿಗೆ
ಅಂಶು-ಮಂ-ತನು
ಅಕ-ರ್ತೃ-ವಾ-ಗಿ-ರುವ
ಅಕ-ಸ್ಮಾ-ತಾಗಿ
ಅಕ-ಸ್ಮಾ-ತ್ತಾಗಿ
ಅಕಾರಣ-ವಾಗಿ
ಅಕಾ-ರ್ಯ-ವನ್ನು
ಅಕೂ-ಪಾ-ರಾಯ
ಅಕೃ-ತ್ರಿಮ
ಅಕೆ
ಅಕ್ಕ
ಅಕ್ಕಂ-ದಿರು
ಅಕ್ಕಂ-ದಿ-ರೆಲ್ಲ
ಅಕ್ಕ-ರೆ-ಯಂತೆ
ಅಕ್ಕ-ರೆ-ಯಿಂದ
ಅಕ್ಕ-ರೆ-ಯಿಂ-ದ-ವ-ನನ್ನು
ಅಕ್ಕ-ರೆ-ಯು-ಳ್ಳ-ವ-ರಾ-ಗಿ-ದ್ದರೂ
ಅಕ್ಕಿ
ಅಕ್ಕಿ-ಗಳು
ಅಕ್ಕಿಯ
ಅಕ್ರೂರ
ಅಕ್ರೂ-ರ-ಭೋಜ
ಅಕ್ರೂ-ರನ
ಅಕ್ರೂ-ರ-ನಂತೂ
ಅಕ್ರೂ-ರ-ನಿಗೆ
ಅಕ್ರೂ-ರನು
ಅಕ್ರೂ-ರ-ನೆಂಬ
ಅಕ್ರೂ-ರ-ರಿ-ಬ್ಬರೂ
ಅಕ್ರೂರಾ
ಅಕ್ಷ-ತೆ-ಗಳನ್ನು
ಅಕ್ಷ-ತೆ-ಗಳಿಂದ
ಅಕ್ಷಮ್ಯ
ಅಕ್ಷ-ರ-ಗಳಿಂದ
ಅಕ್ಷ-ರಶಃ
ಅಕ್ಷ-ವಿ-ದ್ಯೆ-ಯನ್ನು
ಅಕ್ಷೋ-ಹಿಣಿ
ಅಕ್ಷೋ-ಹಿ-ಣಿಯ
ಅಕ್ಷೋ-ಹಿಣೀ
ಅಖಂಡ
ಅಖಂ-ಡ-ಮೂ-ರ್ತಿ-ಯನ್ನು
ಅಖಂ-ಡ-ವಾ-ಗಿದೆ
ಅಖಂ-ಡ-ವಾದ
ಅಖಿ-ಲಾಂ-ಡ-ಕೋಟಿ
ಅಗ-ಣಿತ
ಅಗತ್ಯ
ಅಗತ್ಯ-ಗಳನ್ನು
ಅಗತ್ಯ-ವಾಗಿ
ಅಗತ್ಯ-ವಾ-ಗಿಯೂ
ಅಗತ್ಯ-ವಾದ
ಅಗತ್ಯ-ವಾ-ದಂ-ತಹ
ಅಗತ್ಯ-ವಾ-ದರೆ
ಅಗತ್ಯ-ವಾ-ದ-ಷ್ಟನ್ನು
ಅಗತ್ಯ-ವಾ-ದು-ದೆ-ಲ್ಲವೂ
ಅಗತ್ಯ-ವಿಲ್ಲ
ಅಗತ್ಯ-ವೆಂದು
ಅಗತ್ಯ-ವೆಂ-ಬು-ದನ್ನು
ಅಗತ್ಯ-ವೇನೂ
ಅಗರು
ಅಗ-ರು-ಬ-ತ್ತಿ-ಗಳು
ಅಗ-ಲ-ತ-ಕ್ಕುವೇ
ಅಗ-ಲ-ಲಾ-ರದೆ
ಅಗ-ಲ-ಲಾರೆ
ಅಗ-ಲ-ವಾದ
ಅಗಲಿ
ಅಗ-ಲಿಕೆ
ಅಗ-ಲಿ-ಕೆ-ಗಾಗಿ
ಅಗ-ಲಿ-ಕೆಯ
ಅಗ-ಲಿ-ಕೆ-ಯನ್ನು
ಅಗ-ಲಿ-ಕೆ-ಯಿಂದ
ಅಗ-ಲಿ-ಕೆ-ಯಿಂ-ದಾ-ಗುವ
ಅಗ-ಲಿ-ಕೆ-ಯೆಂ-ಬುದೇ
ಅಗ-ಲಿ-ದ-ಮೇಲೆ
ಅಗ-ಲಿದ್ದ
ಅಗ-ಲಿ-ರ-ಲಾ-ರದೆ
ಅಗ-ಲಿ-ರ-ಲಾ-ರೆವು
ಅಗ-ಲಿ-ಸು-ವ-ವನೂ
ಅಗ-ಲಿ-ಹೋ-ಗ-ಲಾ-ರದೆ
ಅಗ-ಲು-ತ್ತವೆ
ಅಗ-ಲು-ತ್ತಾರೆ
ಅಗ-ಲು-ತ್ತೇವೆ
ಅಗ-ಲು-ವಂತೆ
ಅಗ-ಲು-ವಾಗ
ಅಗ-ಲು-ವುದು
ಅಗ-ಲು-ವುದೂ
ಅಗ-ಳಿ-ಗಳು
ಅಗಸ
ಅಗ-ಸ-ನಾ-ದರೂ
ಅಗ-ಸನು
ಅಗಸ್ತ್ಯ
ಅಗ-ಸ್ತ್ಯ-ನಿಗೆ
ಅಗ-ಸ್ತ್ಯನು
ಅಗ-ಸ್ತ್ಯ-ಮ-ಹ-ರ್ಷಿ-ಗ-ಳಿಗೆ
ಅಗಾಧ
ಅಗಾ-ಧ-ವಾದ
ಅಗೆದು
ಅಗೋ
ಅಗೋ-ಚರ
ಅಗೋ-ಚ-ರ-ನಾದ
ಅಗೋ-ಚ-ರ-ವಾದ
ಅಗ್ನಿ
ಅಗ್ನಿ-ಕುಂ-ಡ-ದಲ್ಲಿ
ಅಗ್ನಿ-ಗ-ಳಿಗೆ
ಅಗ್ನಿಗೆ
ಅಗ್ನಿ-ಜ್ವಾಲೆ
ಅಗ್ನಿ-ಜ್ವಾ-ಲೆ-ಯಿಂದ
ಅಗ್ನಿ-ದೇವ
ಅಗ್ನಿ-ದೇ-ವ-ನಿಗೆ
ಅಗ್ನಿ-ಪ-ರ್ವತ
ಅಗ್ನಿ-ಪಾ-ತ್ರೆ-ಯನ್ನು
ಅಗ್ನಿ-ಪುತ್ರಿ
ಅಗ್ನಿ-ಪು-ರಾಣ
ಅಗ್ನಿ-ಪು-ರುಷ
ಅಗ್ನಿ-ಪು-ರು-ಷನು
ಅಗ್ನಿ-ಪ್ರ-ವೇಶ
ಅಗ್ನಿ-ಪ್ರ-ವೇ-ಶಕ್ಕೆ
ಅಗ್ನಿ-ಪ್ರ-ವೇ-ಶ-ದಿಂದ
ಅಗ್ನಿ-ಯಂತೆ
ಅಗ್ನಿ-ಯನ್ನು
ಅಗ್ನಿ-ಯನ್ನೇ
ಅಗ್ನಿ-ಯ-ಮೂ-ಲಕ
ಅಗ್ನಿ-ಯಲ್ಲಿ
ಅಗ್ನಿ-ಯಿಂದ
ಅಗ್ನಿಯು
ಅಗ್ನಿ-ವ-ಲ-ಯ-ದಂ-ತಿದೆ
ಅಗ್ನಿ-ಸಂ-ಬಂ-ಧ-ದಿಂದ
ಅಗ್ನಿ-ಸಾ-ಕ್ಷಿ-ಯಾಗಿ
ಅಗ್ನಿ-ಹೋ-ತ್ರ-ಗೃಹ
ಅಗ್ನಿ-ಹೋತ್ರಾ
ಅಗ್ನೀಧ್ರ
ಅಗ್ನೀ-ಧ್ರನ
ಅಗ್ನೀ-ಧ್ರನು
ಅಗ್ರ-ಗಣ್ಯ
ಅಗ್ರ-ಗ-ಣ್ಯ-ನಾ-ಗಿ-ರುವೆ
ಅಗ್ರ-ಗ-ಣ್ಯ-ನಾದ
ಅಗ್ರ-ಗ-ಣ್ಯ-ರಾದ
ಅಗ್ರ-ಗ-ಣ್ಯರು
ಅಗ್ರ-ಗ-ಣ್ಯ-ವಾಗಿ
ಅಗ್ರ-ಪೂಜೆ
ಅಗ್ರ-ಪೂ-ಜೆಗೆ
ಅಗ್ರ-ಪೂ-ಜೆ-ಯನ್ನು
ಅಗ್ರ-ಪೂ-ಜೆ-ಯಿಂದ
ಅಗ್ರ-ಪೂ-ಜೆಯೆ
ಅಗ್ರ-ಹಾ-ರಕ್ಕೆ
ಅಗ್ರ-ಹಾ-ರ-ವನ್ನು
ಅಗ್ರ-ಹಾ-ರ-ವಿದೆ
ಅಗ್ರೇ-ಸರ
ಅಘ
ಅಘಾ-ಸುರ
ಅಘಾ-ಸು-ರ-ನನ್ನು
ಅಘಾ-ಸು-ರ-ನೆಂಬ
ಅಚ-ರವು
ಅಚ-ಲ-ವಾ-ದುದು
ಅಚಾ-ತುರ್ಯ
ಅಚಿಂ-ತ್ಯ-ಮ-ಹಿಮ
ಅಚೇ-ತ-ನ-ಗ-ಳೆ-ರಡೂ
ಅಚೇ-ತ-ನ-ವಾ-ಗಿದ್ದ
ಅಚ್ಚರಿ
ಅಚ್ಚರಿ-ಗೊಂಡ
ಅಚ್ಚರಿ-ಗೊಂ-ಡರು
ಅಚ್ಚರಿ-ಗೊಂಡು
ಅಚ್ಚರಿ-ಗೊ-ಳಿ-ಸಿ-ದರು
ಅಚ್ಚರಿ-ಗೊ-ಳ್ಳು-ತ್ತೀರಿ
ಅಚ್ಚರಿ-ಪ-ಡ-ಬೇ-ಕಾ-ದು-ದಿಲ್ಲ
ಅಚ್ಚರಿಯ
ಅಚ್ಚರಿ-ಯನ್ನು
ಅಚ್ಚರಿ-ಯಾ-ಯಿ-ತು-ಇ-ವ-ನಾರು
ಅಚ್ಚರಿ-ಯಿಂದ
ಅಚ್ಚರಿಯು
ಅಚ್ಚರಿ-ಯೆಂ-ದರೆ
ಅಚ್ಚರಿ-ಸಾ-ಕ್ಷಾತ್
ಅಚ್ಚ-ಳಿ-ಯ-ದಂತೆ
ಅಚ್ಚ-ಳಿ-ಯ-ದಿದೆ
ಅಚ್ಚ-ಳಿ-ಯದೆ
ಅಚ್ಚು
ಅಚ್ಚು-ಮೆ-ಚ್ಚಾ-ಗು-ತ್ತದೆ
ಅಚ್ಚೊ-ತ್ತಿ-ದಂ-ತಿದೆ
ಅಚ್ಯುತ
ಅಚ್ಯು-ತ-ಚ-ರಿ-ತ್ರೆ-ಯೆಂಬ
ಅಜ
ಅಜ-ಗ-ರ-ವೃತ್ತಿ
ಅಜ-ಗ-ರ-ವ್ರ-ತ-ವೆಂಬ
ಅಜನ
ಅಜಾ
ಅಜಾತ
ಅಜಾ-ತ-ಶತ್ರು
ಅಜಾ-ಮಿಳ
ಅಜಾ-ಮಿ-ಳನ
ಅಜಾ-ಮಿ-ಳ-ನನ್ನು
ಅಜಾ-ಮಿ-ಳ-ನಿಗೆ
ಅಜಾ-ಮಿ-ಳನು
ಅಜಾ-ಮಿ-ಳ-ನೆಂಬ
ಅಜಿ-ತ-ನೆಂಬ
ಅಜೇ-ಯ-ನಾ-ದನು
ಅಜೇ-ಯ-ರಾ-ಗಿ-ಬಿ-ಟ್ಟಿ-ದ್ದೀರಿ
ಅಜ್ಜ
ಅಜ್ಜನ
ಅಜ್ಞ
ಅಜ್ಞ-ತೆಯೇ
ಅಜ್ಞ-ನಂತೆ
ಅಜ್ಞ-ರಲ್ಲಿ
ಅಜ್ಞ-ರಿ-ಗಾಗಿ
ಅಜ್ಞ-ರಿಗೆ
ಅಜ್ಞಾ-ತ-ರಂ-ತಿದ್ದ
ಅಜ್ಞಾ-ತ-ವಾ-ಸ-ವನ್ನು
ಅಜ್ಞಾನ
ಅಜ್ಞಾನ-ಜನ್ಯ
ಅಜ್ಞಾನದ
ಅಜ್ಞಾನ-ದಲ್ಲಿ
ಅಜ್ಞಾನ-ದಿಂದ
ಅಜ್ಞಾನ-ದೊ-ಡನೆ
ಅಜ್ಞಾನ-ಮೂ-ಲ-ಕ-ವಾಗಿ
ಅಜ್ಞಾನ-ವನ್ನು
ಅಜ್ಞಾನ-ವನ್ನೂ
ಅಜ್ಞಾನವೇ
ಅಜ್ಞಾನ-ಸ್ವ-ರೂ-ಪ-ವಾದ
ಅಜ್ಞಾನಿ-ಗಳ
ಅಜ್ಞಾನಿ-ಗ-ಳಿಗೆ
ಅಜ್ಞಾನಿ-ಗಳು
ಅಜ್ಞಾನಿ-ಗಳೇ
ಅಜ್ಞಾನಿಯ
ಅಜ್ಞಾನಿ-ಯಂತೆ
ಅಜ್ಞಾನಿ-ಯಾ-ಗಿ-ರುವ
ಅಜ್ಞಾನಿ-ಯಾದ
ಅಟ್ಟ
ಅಟ್ಟ-ಹಾ-ಸ-ದಿಂದ
ಅಟ್ಟ-ಹಾ-ಸ-ದಿಂ-ದಲೇ
ಅಟ್ಟಿ
ಅಟ್ಟಿ-ಕೊಂಡು
ಅಟ್ಟಿ-ದನು
ಅಟ್ಟಿಸಿ
ಅಟ್ಟಿ-ಸಿ-ಕೊಂಡು
ಅಟ್ಟುವ
ಅಟ್ಟೆ-ಗಳು
ಅಡ-ಕ-ವಾ-ಗಿದೆ
ಅಡ-ಕ-ವಾ-ಗಿದ್ದು
ಅಡ-ಕ-ವಾ-ಗಿವೆ
ಅಡ-ಕಿಲ
ಅಡಗಿ
ಅಡಗಿ-ಕೊಂ-ಡರು
ಅಡಗಿ-ಕೊಂ-ಡಿ-ದ್ದನು
ಅಡಗಿ-ಕೊಂ-ಡಿ-ದ್ದಾನೆ
ಅಡಗಿ-ಕೊಂಡು
ಅಡಗಿತು
ಅಡಗಿತ್ತು
ಅಡಗಿದ್ದ
ಅಡಗಿರ
ಅಡಗಿ-ಸಲು
ಅಡಗಿಸಿ
ಅಡಗಿ-ಸಿ-ಕೊಂಡು
ಅಡಗಿ-ಸಿ-ಕೊ-ಳ್ಳ-ಲಾರ
ಅಡಗಿ-ಸಿ-ಕೊ-ಳ್ಳು-ತ್ತ-ದೆ-ಯಲ್ಲ
ಅಡಗಿ-ಸಿ-ಕೊ-ಳ್ಳು-ವುದು
ಅಡಗಿ-ಸಿತು
ಅಡಗಿ-ಸಿ-ಬಿ-ಡು-ತ್ತೇನೆ
ಅಡಗಿ-ಸು-ತ್ತಿ-ದ್ದ-ನಾ-ದ್ದ-ರಿಂದ
ಅಡಗಿ-ಸುವ
ಅಡಗಿ-ಸು-ವುದು
ಅಡಗಿ-ಹೋ-ದಳು
ಅಡ-ಗು-ತ್ತದೆ
ಅಡ-ಗು-ವಂತೆ
ಅಡ-ಗು-ವ-ವ-ರೆಗೂ
ಅಡ-ವಿಗೆ
ಅಡ-ವಿಯ
ಅಡ-ವಿ-ಯನ್ನು
ಅಡ-ವಿ-ಯ-ನ್ನೆಲ್ಲ
ಅಡ-ವಿ-ಯಲ್ಲಿ
ಅಡ-ವಿ-ಯ-ಲ್ಲಿದ್ದ
ಅಡ-ವಿ-ಯ-ಲ್ಲಿ-ದ್ದರೂ
ಅಡ-ವಿ-ಯ-ಲ್ಲಿ-ದ್ದ-ರೇನು
ಅಡ-ವಿ-ಯಿಂದ
ಅಡ-ವಿ-ಯೊ-ಳಕ್ಕೆ
ಅಡಿಗೆ
ಅಡಿ-ಗೆ-ಮ-ನೆಯ
ಅಡಿ-ಗೆಯ
ಅಡಿ-ಗೆ-ಯ-ವನ
ಅಡು-ಗೆ-ಮ-ನೆಯ
ಅಡು-ಗೆ-ಯ-ವನು
ಅಡು-ಗೆ-ಯ-ವರ
ಅಡೆ-ಕೆಯ
ಅಡ್ಡ
ಅಡ್ಡ-ಗ-ಟ್ಟಿ-ದು-ದ-ರಿಂದ
ಅಡ್ಡ-ಗ-ಟ್ಟೆ-ಹಾಕಿ
ಅಡ್ಡ-ಬಿದ್ದ
ಅಡ್ಡ-ಬಿ-ದ್ದನು
ಅಡ್ಡ-ಬಿ-ದ್ದರು
ಅಡ್ಡ-ಬಿ-ದ್ದಳು
ಅಡ್ಡ-ಬಿದ್ದು
ಅಡ್ಡ-ಬೀ-ಳ-ಬೇಡ
ಅಡ್ಡ-ಬೀ-ಳಲು
ಅಡ್ಡ-ಬೀ-ಳುವ
ಅಡ್ಡ-ವಾ-ಗಿತ್ತು
ಅಡ್ಡ-ಹೆ-ಸ-ರಿ-ಟ್ಟಿ-ದ್ದರು
ಅಡ್ಡಾ-ಡು-ತ್ತಿ-ರು-ವಾಗ
ಅಡ್ಡಿ
ಅಡ್ಡಿ-ಆ-ತಂಕ
ಅಡ್ಡಿ-ಗಳು
ಅಡ್ಡಿ-ಯನ್ನು
ಅಡ್ಡಿ-ಯಾಗಿ
ಅಡ್ಡಿ-ಯಾ-ಗಿತ್ತು
ಅಡ್ಡಿ-ಯಾ-ಗಿದ್ದ
ಅಡ್ಡಿ-ಯಾದ
ಅಡ್ಡಿಯೂ
ಅಣ-ಕಿಸಿ
ಅಣಿ
ಅಣಿ-ಮಾದಿ
ಅಣಿ-ಯಾಗಿ
ಅಣಿ-ಯಾ-ಗಿದ್ದ
ಅಣಿ-ಯಾ-ಗಿವೆ
ಅಣುಗ
ಅಣು-ಗೀ-ತೆ-ಮೊ-ದ-ಲಾ-ದವೂ
ಅಣು-ಮಾ-ತ್ರ-ದಂತೆ
ಅಣ್ಣ
ಅಣ್ಣಂ-ದಿರ
ಅಣ್ಣಂ-ದಿ-ರಂತೆ
ಅಣ್ಣಂ-ದಿ-ರಂ-ತೆಯೆ
ಅಣ್ಣಂ-ದಿ-ರನ್ನೂ
ಅಣ್ಣಂ-ದಿ-ರಲ್ಲಿ
ಅಣ್ಣಂ-ದಿ-ರಿಗೆ
ಅಣ್ಣಂ-ದಿರು
ಅಣ್ಣ-ಗ-ಳಾರು
ಅಣ್ಣ-ತ-ಮ್ಮಂ-ದಿರು
ಅಣ್ಣನ
ಅಣ್ಣ-ನನ್ನು
ಅಣ್ಣ-ನನ್ನೆ
ಅಣ್ಣ-ನಾಗಿ
ಅಣ್ಣ-ನಾದ
ಅಣ್ಣ-ನಿ-ಗಾದ
ಅಣ್ಣ-ನಿ-ಗಿಂ-ತಲೂ
ಅಣ್ಣ-ನಿಗೆ
ಅಣ್ಣ-ನೊ-ಡನೆ
ಅತಂ-ತ್ರ-ವಾಗಿ
ಅತಳ
ಅತಿ
ಅತಿ-ಕ್ರ-ಮಿಸಿ
ಅತಿಥಿ
ಅತಿ-ಥಿ-ಗಳನ್ನು
ಅತಿ-ಥಿ-ಗ-ಳಾಗಿ
ಅತಿ-ಥಿ-ಗ-ಳಿಗೆ
ಅತಿ-ಥಿ-ಗ-ಳೆ-ಲ್ಲ-ರನ್ನೂ
ಅತಿ-ಥಿ-ಯನ್ನು
ಅತಿ-ಥಿ-ಯಾ-ಗಿಯೆ
ಅತಿ-ಥಿ-ಯಾದ
ಅತಿ-ಥಿ-ಯೆಂ-ದರೆ
ಅತಿ-ಥಿ-ಸ-ತ್ಕಾ-ರ-ದಿಂದ
ಅತಿ-ಮಾನವ
ಅತಿ-ಯಾ-ಸೆ-ಯೇನೂ
ಅತಿ-ರಥ
ಅತಿ-ವೃಷ್ಟಿ
ಅತಿ-ಶಯ
ಅತಿ-ಶ-ಯ-ವಾಗಿ
ಅತಿ-ಶ-ಯ-ವಾದ
ಅತಿ-ಶ-ಯೋ-ಕ್ತಿ-ಯಾ-ಗ-ಲಾ-ರದು
ಅತಿ-ಶೀ-ಘ್ರ-ದಲ್ಲೇ
ಅತೀಂ-ದ್ರಿ-ಯ-ವಾದ
ಅತೀತ
ಅತೀ-ತ-ನಾ-ಗಿ-ದ್ದನು
ಅತೀ-ತ-ನಾದ
ಅತೀ-ತ-ನೆಂಬು
ಅತೃ-ಪ್ತಿಯೂ
ಅತ್ತ
ಅತ್ತ-ಕಡೆ
ಅತ್ತನು
ಅತ್ತರು
ಅತ್ತರೆ
ಅತ್ತಳು
ಅತ್ತವು
ಅತ್ತ-ಹೋ-ಗು-ತ್ತಲೇ
ಅತ್ತಿಂ-ದಿತ್ತ
ಅತ್ತಿಗೆ
ಅತ್ತಿ-ಗೆ-ಯ-ರಿರಾ
ಅತ್ತಿತು
ಅತ್ತಿತ್ತ
ಅತ್ತಿಯ
ಅತ್ತು
ಅತ್ತೆಗೂ
ಅತ್ತೆಯ
ಅತ್ಯಂತ
ಅತ್ಯ-ಗತ್ಯ
ಅತ್ಯ-ಗ-ತ್ಯ-ವಾಗಿ
ಅತ್ಯ-ಗ-ತ್ಯ-ವಾದ
ಅತ್ಯ-ಗ-ತ್ಯ-ವಾ-ದಷ್ಟು
ಅತ್ಯ-ಗ-ತ್ಯ-ವೆಂ-ದನು
ಅತ್ಯ-ಲ್ಪ-ವಾ-ಗಿ-ರುವ
ಅತ್ಯಾ-ನಂದ
ಅತ್ಯಾ-ನಂ-ದ-ಗೊಂಡ
ಅತ್ಯಾ-ನಂ-ದ-ದಿಂದ
ಅತ್ಯಾ-ನಂ-ದ-ವಾ-ಯಿತು
ಅತ್ಯಾ-ಶ್ಚ-ರ್ಯ-ಕ-ರ-ವಾದ
ಅತ್ಯಾ-ಶ್ಚ-ರ್ಯ-ವಾ-ಯಿತು
ಅತ್ಯು-ತ್ಕಂ-ಠ-ಶ್ಯ-ಬ-ಲ-ಹೃ-ದ-ಯೋ-ಸ್ಮ-ದ್ವಿಧೋ
ಅತ್ಯು-ತ್ತಮ
ಅತ್ಯು-ತ್ತ-ಮ-ವಾದ
ಅತ್ಯು-ತ್ತ-ಮ-ವಾ-ದು-ದನ್ನು
ಅತ್ರಿ
ಅತ್ರಿ-ಅ-ನ-ಸೂ-ಯೆ-ಯರ
ಅತ್ರಿ-ಋ-ಷಿಯು
ಅತ್ರಿ-ಮ-ಹರ್ಷಿ
ಅತ್ರಿ-ಮ-ಹ-ರ್ಷಿಯ
ಅತ್ರಿ-ಮುನಿ
ಅತ್ರಿ-ಮು-ನಿಯ
ಅತ್ರಿ-ಮು-ನಿಯು
ಅತ್ರಿ-ಮು-ನಿಯೇ
ಅತ್ರಿಯೂ
ಅಥ
ಅಥ-ರ್ವನು
ಅಥವಾ
ಅದ
ಅದ-ಕ್ಕಾಗಿ
ಅದ-ಕ್ಕಾ-ಗಿಯೇ
ಅದ-ಕ್ಕಾ-ಗುವ
ಅದ-ಕ್ಕಿಂ-ತಲೂ
ಅದಕ್ಕೂ
ಅದಕ್ಕೆ
ಅದ-ಕ್ಕೊಂದು
ಅದ-ಕ್ಕೊಪ್ಪಿ
ಅದನ್ನ
ಅದ-ನ್ನಷ್ಟು
ಅದನ್ನು
ಅದನ್ನೂ
ಅದನ್ನೆ
ಅದ-ನ್ನೆಲ್ಲ
ಅದನ್ನೇ
ಅದ-ನ್ನೇರಿ
ಅದರ
ಅದ-ರಂ-ತಿ-ರ-ಬೇಕು
ಅದ-ರಂತೆ
ಅದ-ರಂ-ತೆಯೆ
ಅದ-ರಂ-ತೆಯೇ
ಅದ-ರತ್ತ
ಅದ-ರದೇ
ಅದ-ರಲ್ಲಿ
ಅದ-ರ-ಲ್ಲಿದ್ದ
ಅದ-ರ-ಲ್ಲಿಯ
ಅದ-ರ-ಲ್ಲಿಯೂ
ಅದ-ರ-ಲ್ಲಿಯೆ
ಅದ-ರ-ಲ್ಲಿಯೇ
ಅದ-ರಷ್ಟೆ
ಅದ-ರಷ್ಟೇ
ಅದ-ರಾಚೆ
ಅದ-ರಿಂದ
ಅದ-ರಿಂ-ದಲೆ
ಅದ-ರಿಂ-ದಲೇ
ಅದ-ರಿಂ-ದೆದ್ದ
ಅದರೆ
ಅದ-ರೊ-ಡನೆ
ಅದ-ರೊ-ಳ-ಗಿದ್ದ
ಅದ-ರೊ-ಳ-ಗಿ-ನಿಂದ
ಅದ-ರೊ-ಳ-ಗಿ-ರು-ವುದು
ಅದ-ರೊ-ಳಗೆ
ಅದಲು
ಅದ-ಲ್ಲದೆ
ಅದಾದ
ಅದಾರ
ಅದಿ-ತಿಗೆ
ಅದಿ-ತಿ-ದೇವಿ
ಅದಿ-ತಿ-ದೇ-ವಿಗೆ
ಅದಿ-ತಿ-ದೇ-ವಿಯ
ಅದಿ-ತಿ-ದೇ-ವಿಯು
ಅದಿ-ತಿಯ
ಅದಿ-ತಿ-ಯನ್ನು
ಅದಿ-ತಿ-ಯ-ರಾಗಿ
ಅದಿ-ತಿ-ಯರೆ
ಅದಿ-ತಿಯು
ಅದಿದ್ದ
ಅದಿ-ರಲಿ
ಅದು
ಅದು-ರಿತು
ಅದು-ರು-ತ್ತಿತ್ತು
ಅದು-ವ-ರೆಗೆ
ಅದೂ
ಅದೃಶ್ಯ
ಅದೃ-ಶ್ಯ-ರೂ-ಪ-ದಿಂದ
ಅದೃ-ಶ್ಯ-ವಾಗಿ
ಅದೃ-ಶ್ಯ-ಶ-ಕ್ತಿ-ಯು-ಳ್ಳ-ವರು
ಅದೃಷ್ಟ
ಅದೃ-ಷ್ಟಕ್ಕೆ
ಅದೃ-ಷ್ಟ-ದ-ಲ್ಲಿಯೂ
ಅದೃ-ಷ್ಟ-ದಿಂದ
ಅದೃ-ಷ್ಟ-ದಿಂ-ದಲೂ
ಅದೃ-ಷ್ಟ-ವಂ-ತರೇ
ಅದೃ-ಷ್ಟ-ವನ್ನು
ಅದೃ-ಷ್ಟ-ವ-ಶ-ದಿಂದ
ಅದೃ-ಷ್ಟವೇ
ಅದೆ
ಅದೆಂ-ತಹ
ಅದೆಲ್ಲ
ಅದೆ-ಲ್ಲ-ವನ್ನೂ
ಅದೆ-ಲ್ಲವೂ
ಅದೆಷ್ಟು
ಅದೆಷ್ಟೋ
ಅದೇ
ಅದೇ-ಕಿಷ್ಟು
ಅದೇ-ನಾ-ದರೂ
ಅದೇನು
ಅದೇ-ನೆಂದು
ಅದೇನೋ
ಅದೊಂ-ದನ್ನು
ಅದೊಂದು
ಅದೊಂದೆ
ಅದ್ದಿದ
ಅದ್ದಿ-ದ-ಹಾಗೆ
ಅದ್ದಿದೆ
ಅದ್ದಿ-ದೆ-ಯಲ್ಲಾ
ಅದ್ಭುತ
ಅದ್ಭು-ತ-ಕಾ-ರ್ಯ-ಗಳನ್ನು
ಅದ್ಭು-ತ-ಕಾ-ರ್ಯ-ಗ-ಳಿ-ಗಾಗಿ
ಅದ್ಭು-ತ-ಕಾ-ರ್ಯ-ಗಳೇ
ಅದ್ಭು-ತ-ಗಳಲ್ಲಿ
ಅದ್ಭು-ತ-ಗಳು
ಅದ್ಭು-ತ-ಲೀ-ಲೆ-ಯನ್ನು
ಅದ್ಭು-ತ-ವಾದ
ಅದ್ಭು-ತವೇ
ಅದ್ವಿ-ತೀಯ
ಅದ್ವಿ-ತೀ-ಯ-ನಾಗಿ
ಅದ್ವಿ-ತೀ-ಯ-ನಾದ
ಅದ್ವಿ-ತೀ-ಯನೂ
ಅದ್ವಿ-ತೀ-ಯ-ವಾಗಿ
ಅದ್ವಿ-ತೀ-ಯ-ವಾ-ದುದು
ಅದ್ವೈ-ತ-ಗಳು
ಅದ್ವೈ-ತದ
ಅಧ-ರಾ-ಮೃ-ತ-ದಿಂದ
ಅಧ-ರಾ-ಮೃ-ತ-ವನ್ನು
ಅಧರ್ಮ
ಅಧ-ರ್ಮಕ್ಕೂ
ಅಧ-ರ್ಮಕ್ಕೆ
ಅಧ-ರ್ಮ-ಗ-ಳ-ಲ್ಲದೆ
ಅಧ-ರ್ಮ-ಗಳು
ಅಧ-ರ್ಮ-ಗಳೂ
ಅಧ-ರ್ಮದ
ಅಧ-ರ್ಮನು
ಅಧ-ರ್ಮ-ಬುದ್ಧಿ
ಅಧ-ರ್ಮ-ಮಾ-ಡಿ-ದ-ವ-ನನ್ನು
ಅಧ-ರ್ಮ-ವಾ-ಗದು
ಅಧಿ-ಕ-ನಾ-ದ-ವನು
ಅಧಿ-ಕ-ವಾ-ಯಿತು
ಅಧಿ-ಕಾರ
ಅಧಿ-ಕಾ-ರ-ಗಳನ್ನು
ಅಧಿ-ಕಾ-ರ-ವನ್ನು
ಅಧಿ-ಕಾ-ರ-ವಾ-ವುದೂ
ಅಧಿ-ಕಾ-ರ-ಸೂ-ಚ-ಕ-ವಾದ
ಅಧಿ-ಕಾರಿ
ಅಧಿ-ಕಾ-ರಿ-ಗಳೂ
ಅಧಿ-ದೇ-ವ-ತೆ-ಗಳೂ
ಅಧಿ-ಪತಿ
ಅಧಿ-ಪ-ತಿ-ಮ-ಗೃ-ಹಾ-ಣಾ-ಮ-ಗ್ರತೋ
ಅಧಿ-ಪ-ತಿ-ಯಾಗಿ
ಅಧಿ-ಪ-ತಿ-ಯಾ-ಗಿ-ದ್ದರೂ
ಅಧಿ-ಪ-ತಿ-ಯಾ-ಗಿ-ದ್ದು-ದ-ರಿಂದ
ಅಧಿ-ಪ-ತಿ-ಯಾದ
ಅಧಿ-ಪ-ತಿ-ಯಾ-ದನು
ಅಧೀನ
ಅಧೀ-ನ-ಎಂ-ಬುದು
ಅಧೀ-ನಕ್ಕೆ
ಅಧೀ-ನದ
ಅಧೀ-ನ-ದಲ್ಲಿ
ಅಧೀ-ನ-ನಾ-ಗು-ವು-ದಿಲ್ಲ
ಅಧೀ-ನ-ರಾ-ದರು
ಅಧೀ-ನ-ವಾ-ದುದು
ಅಧೀ-ನ-ವೆ-ನಿ-ಸಿ-ರು-ವುದೋ
ಅಧೀ-ರ-ನಾಗಿ
ಅಧೋ-ಗ-ತಿಗೆ
ಅಧೋ-ಲೋಕ
ಅಧೋ-ಲೋ-ಕ-ಗಳೂ
ಅಧ್ಯಕ್ಷ
ಅಧ್ಯ-ಕ್ಷರು
ಅಧ್ಯ-ಯನ
ಅಧ್ಯ-ಯ-ನ-ಕ್ಕೆಂದು
ಅಧ್ಯ-ಯ-ನ-ದಿಂದ
ಅಧ್ಯಾತ್ಮ
ಅಧ್ಯಾ-ಯ-ದಲ್ಲಿ
ಅನಂತ
ಅನಂ-ತ-ನಿ-ದ್ದಾನೆ
ಅನಂ-ತರ
ಅನಂ-ತರಂ
ಅನಂ-ತ-ರ-ದಲ್ಲಿ
ಅನಂ-ತ-ವಾ-ಗಿವೆ
ಅನಂ-ತ-ವಾದ
ಅನಕ
ಅನನ್ಯ
ಅನ-ನ್ಯ-ಗ-ತಿ-ಕ-ರಾಗಿ
ಅನ-ನ್ಯ-ವಾದ
ಅನ-ಪೇ-ಕ್ಷಿ-ತ-ವಾಗಿ
ಅನರ್ಥ
ಅನ-ರ್ಥ-ಗಳು
ಅನ-ರ್ಥ-ವನ್ನು
ಅನ-ರ್ಹ-ನೆಂದು
ಅನ-ವ-ರ-ತವೂ
ಅನ-ಸೂ-ಯೆ-ಯನ್ನೂ
ಅನ-ಸೂ-ಯೆ-ಯೊ-ಡನೆ
ಅನಾ-ಥ-ನ-ನ್ನಾಗಿ
ಅನಾ-ಥ-ರಾ-ದೆವು
ಅನಾ-ಥ-ವಾ-ಯಿತು
ಅನಾ-ಥೆ-ಯಾ-ಗಿ-ರುವ
ಅನಾ-ದ-ರಕ್ಕೆ
ಅನಾದಿ
ಅನಾ-ದಿ-ಕಾ-ಲ-ದಿಂದ
ಅನಾ-ದಿ-ನಿ-ಧ-ನ-ನಾದ
ಅನಾ-ದಿ-ಯಾಗಿ
ಅನಾ-ದಿ-ಯಾದ
ಅನಾ-ಪ-ಲಿಂ-ಗ-ಕೂ-ಸ್ಕಾನಿ
ಅನಾ-ಮ-ತ್ತಾಗಿ
ಅನಾ-ಮ-ರೂ-ಪ-ಶ್ಚಿ-ನ್ಮಾ-ತ್ರ-ಸ್ಸೋ-ವ್ಯಾ-ನ್ನ-ಸ್ಸ-ದ-ಸ-ತ್ಪರಃ
ಅನಾ-ಮಿ-ಕಾ-ಭ್ಯಾಂ
ಅನಾ-ಯ-ಕ-ವಾಗಿ
ಅನಾ-ಯ-ಕ-ವಾ-ಗಿತ್ತು
ಅನಾ-ಯ-ಕ-ವಾ-ಗುವ
ಅನಾ-ಯ-ಕ-ವಾ-ಗು-ವಂ-ತಾ-ಯಿತು
ಅನಾ-ಯ-ಕ-ವಾ-ಗು-ವು-ದಲ್ಲ
ಅನಾ-ಯ-ಕ-ವಾ-ಯಿತು
ಅನಾ-ಯಾ-ಸ-ವಾಗಿ
ಅನಾ-ವೃಷ್ಟಿ
ಅನಾ-ಸಕ್ತ
ಅನಾ-ಹುತ
ಅನಾ-ಹು-ತ-ಗ-ಳಾ-ಗು-ವುದನ್ನು
ಅನಾ-ಹು-ತ-ವನ್ನು
ಅನಿ
ಅನಿತ್ಯ
ಅನಿ-ತ್ಯ-ದಲ್ಲಿ
ಅನಿ-ತ್ಯ-ವೆಂದೂ
ಅನಿ-ತ್ಯ-ವೆಂ-ಬುದು
ಅನಿ-ರುದ್ಧ
ಅನಿ-ರು-ದ್ಧ-ಉ-ಷೆ-ಯರ
ಅನಿ-ರು-ದ್ಧ-ಸಾ-ತ್ಯಕಿ
ಅನಿ-ರು-ದ್ಧನ
ಅನಿ-ರು-ದ್ಧ-ನನ್ನು
ಅನಿ-ರು-ದ್ಧ-ನಿಗೆ
ಅನಿ-ರು-ದ್ಧನು
ಅನಿ-ರು-ದ್ಧ-ನೇನು
ಅನಿ-ರ್ವ-ಚ-ನೀಯ
ಅನಿ-ರ್ವ-ಚ-ನೀ-ಯ-ವಾದ
ಅನಿಲ
ಅನಿ-ಲಾ-ಗ್ನಿ-ಧಾ-ರಣೆ
ಅನಿ-ವಾರ್ಯ
ಅನಿ-ವಾ-ರ್ಯ-ವಾಗಿ
ಅನಿ-ವಾ-ರ್ಯ-ವಾ-ಯಿತು
ಅನಿ-ಸಿತು
ಅನಿ-ಸಿ-ತು-ನನ್ನ
ಅನೀಕ
ಅನು
ಅನು-ಕ-ರಿ-ಸಿ-ದರು
ಅನು-ಕೂ-ಲ-ಗಳನ್ನು
ಅನು-ಕೂ-ಲ-ದಾಂ-ಪ-ತ್ಯಕ್ಕೆ
ಅನು-ಕೂ-ಲ-ಳಾ-ಗಿ-ರು-ವಳೊ
ಅನು-ಕೂ-ಲ-ವಾ-ಗ-ಲೆಂಬ
ಅನು-ಕೂ-ಲ-ವಾ-ಗು-ವಂತೆ
ಅನು-ಕೂ-ಲ-ವಾದ
ಅನು-ಕೂ-ಲ-ವಾ-ದಂ-ತಹ
ಅನು-ಕೂ-ಲಿ-ಸು-ವಂತೆ
ಅನು-ಕ್ರ-ಮ-ವಾಗಿ
ಅನು-ಗ-ಮ-ನ-ಮಾ-ಡಿ-ದರು
ಅನು-ಗಾ-ಲವೂ
ಅನು-ಗ್ರಹ
ಅನು-ಗ್ರ-ಹ-ಕಾ-ರಿ-ಯೇ-ನಮ್ಮ
ಅನು-ಗ್ರ-ಹಕ್ಕೆ
ಅನು-ಗ್ರ-ಹ-ದಿಂದ
ಅನು-ಗ್ರ-ಹ-ದಿಂ-ದಲೂ
ಅನು-ಗ್ರ-ಹ-ದಿಂ-ದಲೆ
ಅನು-ಗ್ರ-ಹ-ದಿಂ-ದಲೇ
ಅನು-ಗ್ರ-ಹ-ಮಾಡು
ಅನು-ಗ್ರ-ಹ-ವನ್ನು
ಅನು-ಗ್ರ-ಹ-ವನ್ನೆ
ಅನು-ಗ್ರ-ಹ-ವಾ-ಗ-ಬೇಕು
ಅನು-ಗ್ರ-ಹ-ವಾಗಿ
ಅನು-ಗ್ರ-ಹ-ವಾ-ಗು-ವ-ವ-ರೆಗೆ
ಅನು-ಗ್ರ-ಹ-ವಾ-ದ-ರೇನು
ಅನು-ಗ್ರ-ಹ-ವಾ-ಯಿ-ತೆಂ-ದರೆ
ಅನು-ಗ್ರ-ಹ-ವಿ-ಲ್ಲದ
ಅನು-ಗ್ರ-ಹ-ವಿ-ಲ್ಲದೆ
ಅನು-ಗ್ರ-ಹವೆ
ಅನು-ಗ್ರ-ಹ-ವೆಂದು
ಅನು-ಗ್ರ-ಹ-ವೆ-ನಿ-ಸು-ತ್ತಿದೆ
ಅನು-ಗ್ರ-ಹವೇ
ಅನು-ಗ್ರ-ಹಿಸ
ಅನು-ಗ್ರ-ಹಿ-ಸ-ಬೇಕು
ಅನು-ಗ್ರ-ಹಿ-ಸ-ಬೇ-ಕೆಂದು
ಅನು-ಗ್ರ-ಹಿ-ಸ-ಬೇ-ಕೆಂ-ದು-ಕೊಂಡ
ಅನು-ಗ್ರ-ಹಿ-ಸಲಿ
ಅನು-ಗ್ರ-ಹಿ-ಸ-ಲೆಂದೆ
ಅನು-ಗ್ರ-ಹಿಸಿ
ಅನು-ಗ್ರ-ಹಿ-ಸಿದ
ಅನು-ಗ್ರ-ಹಿ-ಸಿ-ದನು
ಅನು-ಗ್ರ-ಹಿ-ಸಿ-ದರು
ಅನು-ಗ್ರ-ಹಿ-ಸಿ-ರುವ
ಅನು-ಗ್ರ-ಹಿ-ಸಿ-ರುವೆ
ಅನು-ಗ್ರ-ಹಿಸು
ಅನು-ಗ್ರ-ಹಿ-ಸುತ್ತಾ
ಅನು-ಗ್ರ-ಹಿ-ಸು-ತ್ತೇನೆ
ಅನು-ಗ್ರ-ಹಿ-ಸುವ
ಅನು-ಗ್ರ-ಹಿ-ಸು-ವು-ದ-ಕ್ಕಾ-ಗಿಯೆ
ಅನು-ಚಿ-ತದ
ಅನು-ಜ್ಞೆ-ಯನ್ನು
ಅನು-ಪ-ಲ್ಲವಿ
ಅನು-ಭವ
ಅನು-ಭ-ವ-ದಲ್ಲಿ
ಅನು-ಭ-ವ-ವನ್ನು
ಅನು-ಭ-ವ-ವ-ನ್ನೆಲ್ಲ
ಅನು-ಭ-ವವಾ
ಅನು-ಭವಿ
ಅನು-ಭ-ವಿ-ಸ-ಬ-ಲ್ಲ-ವಾ-ದರೂ
ಅನು-ಭ-ವಿ-ಸ-ಬೇ-ಕ-ಲ್ಲವೆ
ಅನು-ಭ-ವಿ-ಸ-ಬೇ-ಕಾಗಿ
ಅನು-ಭ-ವಿ-ಸ-ಬೇ-ಕಾ-ಗು-ತ್ತದೆ
ಅನು-ಭ-ವಿ-ಸ-ಬೇ-ಕಾ-ಯಿತು
ಅನು-ಭ-ವಿ-ಸ-ಬೇಕು
ಅನು-ಭ-ವಿ-ಸ-ಬೇಕೋ
ಅನು-ಭ-ವಿ-ಸ-ಲೇ-ಬೇಕು
ಅನು-ಭ-ವಿಸಿ
ಅನು-ಭ-ವಿ-ಸಿತು
ಅನು-ಭ-ವಿ-ಸಿದ
ಅನು-ಭ-ವಿ-ಸಿ-ದ-ಮೇಲೆ
ಅನು-ಭ-ವಿ-ಸಿ-ದರು
ಅನು-ಭ-ವಿ-ಸಿ-ದರೂ
ಅನು-ಭ-ವಿ-ಸಿಯೇ
ಅನು-ಭ-ವಿಸು
ಅನು-ಭ-ವಿ-ಸು-ತ್ತದೆ
ಅನು-ಭ-ವಿ-ಸುತ್ತಾ
ಅನು-ಭ-ವಿ-ಸು-ತ್ತಿತ್ತು
ಅನು-ಭ-ವಿ-ಸು-ತ್ತಿ-ದ್ದಳು
ಅನು-ಭ-ವಿ-ಸು-ತ್ತಿದ್ದು
ಅನು-ಭ-ವಿ-ಸು-ತ್ತಿ-ರುವ
ಅನು-ಭ-ವಿ-ಸು-ತ್ತೀಯೋ
ಅನು-ಭ-ವಿ-ಸುವ
ಅನು-ಭ-ವಿ-ಸು-ವನು
ಅನು-ಭ-ವಿ-ಸು-ವು-ದ-ಕ್ಕಾಗಿ
ಅನು-ಭ-ವಿ-ಸು-ವು-ದಿಲ್ಲ
ಅನು-ಭ-ವಿ-ಸು-ವುದು
ಅನು-ಭ-ವಿ-ಸು-ವು-ದೇ-ಕೆಂದು
ಅನು-ಭ-ವಿ-ಸುವೆ
ಅನು-ಮ-ತಿ-ಯ-ನ್ನಿ-ತ್ತನು
ಅನು-ಮಾನ
ಅನು-ಮಿಂದ
ಅನು-ಮೋ-ದಿ-ಸಿ-ದರು
ಅನು-ಯಾ-ಯಿ-ಗ-ಳಾಗಿ
ಅನು-ಯಾ-ಯಿ-ಗ-ಳಾದ
ಅನು-ಯಾ-ಯಿ-ಗಳು
ಅನು-ಯಾ-ಯಿ-ಗಳೂ
ಅನು-ಯಾ-ಯಿ-ಗ-ಳೊ-ಡನೆ
ಅನು-ರಕ್ತಿ
ಅನು-ರಾಗ
ಅನು-ರಾ-ಗ-ದಿಂದ
ಅನು-ರಾ-ಗ-ವುಳ್ಳ
ಅನು-ರಾ-ಗ-ವು-ಳ್ಳ-ವಳು
ಅನು-ರಾ-ಗ-ವೇನು
ಅನು-ರೂ-ಪ-ನ-ಲ್ಲದ
ಅನು-ರೂ-ಪ-ನಾದ
ಅನು-ರೂ-ಪ-ರಾದ
ಅನು-ರೂ-ಪ-ರಾ-ದ-ವರು
ಅನು-ರೂ-ಪ-ಳ-ಲ್ಲ-ವೆಂ-ಬು-ದನ್ನು
ಅನು-ರೂ-ಪಳಾ
ಅನು-ರೂ-ಪ-ವಾದ
ಅನು-ವಿಂದ
ಅನು-ವ್ರ-ತ-ಎಂದು
ಅನುಶು
ಅನು-ಸಂ-ಧಾನ
ಅನು-ಸಂ-ಧಾ-ನ-ಮಾ-ಡುತ್ತಾ
ಅನು-ಸ-ರಿ-ಸ-ಬೇ-ಕಾದ
ಅನು-ಸ-ರಿಸಿ
ಅನು-ಸ-ರಿ-ಸಿ-ಕೊಂಡು
ಅನು-ಸ-ರಿ-ಸಿತು
ಅನು-ಸ-ರಿ-ಸಿದ
ಅನು-ಸ-ರಿ-ಸಿ-ದರು
ಅನು-ಸ-ರಿ-ಸಿ-ದಳು
ಅನು-ಸ-ರಿ-ಸಿರಿ
ಅನು-ಸ-ರಿಸು
ಅನು-ಸ-ರಿ-ಸು-ತ್ತವೆ
ಅನು-ಸ-ರಿ-ಸು-ತ್ತಾರೆ
ಅನು-ಸ-ರಿ-ಸುವ
ಅನು-ಸ-ರಿ-ಸು-ವು-ದಾ-ದರೆ
ಅನು-ಸಾರ
ಅನು-ಸಾ-ರ-ವಾಗಿ
ಅನು-ಸಾರಿ
ಅನು-ಸಾ-ರಿ-ಯಾದ
ಅನು-ಸೂ-ಯಾ-ದೇವಿ
ಅನು-ಸೂ-ಯೆಯ
ಅನು-ಸೂ-ಯೆ-ಯಲ್ಲಿ
ಅನು-ಹ್ಲಾದ
ಅನೇಕ
ಅನೇ-ಕ-ನಾ-ದುದು
ಅನೇ-ಕ-ಬಾರಿ
ಅನೇ-ಕ-ರನ್ನು
ಅನೇ-ಕರು
ಅನೇ-ಕ-ವಾ-ಗಿ-ರುವ
ಅನೇ-ಕ-ವಾ-ಗುವ
ಅನೇ-ಕ-ವೇಳೆ
ಅನ್ನ
ಅನ್ನ-ಕ್ಕಾಗಿ
ಅನ್ನ-ಕ್ಕೇನು
ಅನ್ನದ
ಅನ್ನ-ದಾ-ನ-ಗಳಿಂದ
ಅನ್ನ-ದಾ-ನ-ವನ್ನೂ
ಅನ್ನ-ಬ-ಟ್ಟೆ-ಗ-ಳಿ-ಲ್ಲದೆ
ಅನ್ನ-ಭಿ-ಕ್ಷೆ-ಗಾಗಿ
ಅನ್ನ-ಮಯ
ಅನ್ನ-ರಾ-ಶಿ-ಯನ್ನೂ
ಅನ್ನ-ವ-ನ್ನಿ-ಕ್ಕಿರಿ
ಅನ್ನ-ವನ್ನು
ಅನ್ನ-ವಿ-ಕ್ಕದೆ
ಅನ್ನ-ವಿ-ಕ್ಕಿ-ದ-ವ-ನನ್ನೆ
ಅನ್ನ-ವಿ-ಕ್ಕಿರಿ
ಅನ್ನು
ಅನ್ಯ-ಥಾ-ಜ್ಞಾನ
ಅನ್ಯಾಯ
ಅನ್ಯಾ-ಯಕ್ಕೆ
ಅನ್ಯಾ-ಯ-ಗಳನ್ನು
ಅನ್ಯಾ-ಯ-ಗಳು
ಅನ್ಯಾ-ಯದ
ಅನ್ಯಾ-ಯ-ಮಾ-ಡಲು
ಅನ್ಯಾ-ಯ-ವ-ನ್ನೆಲ್ಲ
ಅನ್ಯಾ-ಯ-ವಾ-ಗ-ದಂತೆ
ಅನ್ಯಾ-ಯ-ವಾಗಿ
ಅನ್ಯಾ-ಯ-ವಾ-ಯಿ-ತೆಂ-ದಾ-ಗಲಿ
ಅನ್ಯಾ-ಯ-ವಾ-ಯಿ-ತೆಂದು
ಅನ್ವ-ಯಿ-ಸು-ತ್ತದೆ
ಅನ್ವ-ಯಿ-ಸುವ
ಅನ್ವ-ರ್ಥ-ವಾ-ಯಿತು
ಅಪ
ಅಪ-ಕಾರ
ಅಪ-ಕೀರ್ತಿ
ಅಪ-ಕೀ-ರ್ತಿ-ಯನ್ನು
ಅಪ-ಕೀ-ರ್ತಿ-ಯನ್ನೂ
ಅಪ-ಚಾರ
ಅಪ-ಚಾ-ರ-ಕ್ಕಾಗಿ
ಅಪ-ಚಾ-ರ-ವನ್ನು
ಅಪ-ಚಾ-ರ-ವನ್ನೇ
ಅಪ-ಚಾ-ರ-ವಾ-ಗ-ದಂತೆ
ಅಪ-ಜಯ
ಅಪ-ಥ್ಯದ
ಅಪ-ಮಾನ
ಅಪ-ಮಾ-ನ-ದಿಂದ
ಅಪ-ಮಾ-ನ-ಮಾ-ಡಿದ
ಅಪ-ಮಾ-ನ-ವನ್ನು
ಅಪ-ಮಾ-ನಿ-ತ-ನಾಗಿ
ಅಪರ
ಅಪ-ರಾ-ಜಿತ
ಅಪ-ರಾತ್ರಿ
ಅಪ-ರಾ-ತ್ರಿ-ಯಲ್ಲಿ
ಅಪ-ರಾಧ
ಅಪ-ರಾ-ಧ-ಕ್ಕಾಗಿ
ಅಪ-ರಾ-ಧ-ಗಳನ್ನು
ಅಪ-ರಾ-ಧ-ಗಳನ್ನೆಲ್ಲಾ
ಅಪ-ರಾ-ಧ-ವನ್ನು
ಅಪ-ರಾ-ಧ-ವ-ನ್ನೆ-ಸ-ಗಿ-ದ-ವ-ನಂತೆ
ಅಪ-ರಾ-ಧ-ವ-ಲ್ಲ-ವೇನು
ಅಪ-ರಾ-ಧ-ವಾ-ದರೂ
ಅಪ-ರಾ-ಧಿ-ಯಾ-ದ-ವನು
ಅಪ-ರಾ-ಹ್ನ-ದ-ಲ್ಲಿಯೂ
ಅಪ-ರಿ-ಗ್ರ-ಹಈ
ಅಪ-ರಿ-ಚ್ಛಿ-ನ್ನನೂ
ಅಪ-ರಿ-ಮಿ-ತ-ವಾದ
ಅಪ-ರೋ-ಕ್ಷ-ಜ್ಞಾ-ನ-ಗಳ
ಅಪ-ವಾ-ದ-ವನ್ನು
ಅಪ-ಶ-ಕು-ನ-ಗಳೂ
ಅಪ-ಹ-ರಿಸಿ
ಅಪ-ಹ-ರಿ-ಸಿದ
ಅಪ-ಹ-ರಿ-ಸು-ತ್ತಿ-ದ್ದರು
ಅಪ-ಹ-ರಿ-ಸು-ವು-ದಿಲ್ಲ
ಅಪ-ಹಾಸ್ಯ
ಅಪ-ಹಾ-ಸ್ಯ-ವನ್ನು
ಅಪಾಯ
ಅಪಾ-ಯ-ಗಳನ್ನು
ಅಪಾ-ಯ-ಗಳಿಂದ
ಅಪಾ-ಯ-ವಿ-ಲ್ಲ-ದಂತೆ
ಅಪಾರ
ಅಪಾ-ರ-ಮ-ಹಿಮ
ಅಪಾ-ರ-ವಾ-ಗಿದ್ದ
ಅಪಾ-ರ-ವಾದ
ಅಪಾ-ಶ್ರಯ
ಅಪಾ-ಶ್ರ-ಯ-ಎಂಬ
ಅಪಿ
ಅಪು-ತ್ರ-ವಂ-ತ-ನಾ-ದನು
ಅಪೂ-ರ್ಣ-ವೆ-ನಿ-ಸಿತು
ಅಪೂರ್ವ
ಅಪೂ-ರ್ವ-ವೆ-ನ್ನು-ವಂತೆ
ಅಪೇ-ಕ್ಷಿ-ಸಿ-ದರೆ
ಅಪೇ-ಕ್ಷಿ-ಸು-ತ್ತೇವೆ
ಅಪೇ-ಕ್ಷಿ-ಸು-ವರು
ಅಪೇ-ಕ್ಷಿ-ಸು-ವ-ವ-ರ-ಲ್ಲ-ವಾ-ದ್ದ-ರಿಂದ
ಅಪೇಕ್ಷೆ
ಅಪೇ-ಕ್ಷೆ-ಯಂತೆ
ಅಪೇ-ಕ್ಷೆ-ಯ-ನ್ನ-ರಿತು
ಅಪೇ-ಕ್ಷೆ-ಯನ್ನು
ಅಪೇ-ಕ್ಷೆ-ಯಿಂ-ದಲೂ
ಅಪೇ-ಕ್ಷೆ-ಯಿತ್ತು
ಅಪೇ-ಕ್ಷೆಯೆ
ಅಪೇ-ಕ್ಷೆಯೇ
ಅಪೇ-ಕ್ಷೆ-ಯೇ-ನೆಂ-ಬು-ದನ್ನು
ಅಪೋ-ಹ-ವಿಲ್ಲ
ಅಪ್ಪ
ಅಪ್ಪಣೆ
ಅಪ್ಪ-ಣೆ-ಕೊಡು
ಅಪ್ಪ-ಣೆಗೆ
ಅಪ್ಪ-ಣೆ-ಮಾಡಿ
ಅಪ್ಪ-ಣೆ-ಮಾ-ಡಿ-ದನು
ಅಪ್ಪ-ಣೆ-ಮಾ-ಡಿ-ರು-ವನು
ಅಪ್ಪ-ಣೆ-ಯಂತೆ
ಅಪ್ಪ-ಣೆ-ಯನ್ನು
ಅಪ್ಪ-ಣೆ-ಯಿ-ಲ್ಲದೆ
ಅಪ್ಪ-ಣೆ-ಯೆಂದೇ
ಅಪ್ಪನ
ಅಪ್ಪ-ನನ್ನೆ
ಅಪ್ಪ-ನಿಗೂ
ಅಪ್ಪ-ನಿಗೇ
ಅಪ್ಪ-ಳಿಸ
ಅಪ್ಪ-ಳಿ-ಸಲು
ಅಪ್ಪ-ಳಿಸಿ
ಅಪ್ಪ-ಳಿ-ಸಿ-ದನು
ಅಪ್ಪ-ಳಿ-ಸು-ತ್ತಲೆ
ಅಪ್ಪ-ಳಿ-ಸು-ತ್ತಿವೆ
ಅಪ್ಪಾ
ಅಪ್ಪಿ
ಅಪ್ಪಿ-ಕೊಂ-ಡನು
ಅಪ್ಪಿ-ಕೊಂ-ಡರು
ಅಪ್ಪಿ-ಕೊಂ-ಡರೂ
ಅಪ್ಪಿ-ಕೊಂ-ಡಳು
ಅಪ್ಪಿ-ಕೊಂ-ಡಿ-ರು-ವಂತೆ
ಅಪ್ಪಿ-ಕೊಂಡು
ಅಪ್ಪಿ-ಕೊಳ್ಳ
ಅಪ್ಪು
ಅಪ್ರಾ-ಪ್ಯ-ಮು-ಷ್ಟ-ಹೃ-ದ-ಯಾಃ
ಅಪ್ಸರ
ಅಪ್ಸ-ರ-ರನ್ನೂ
ಅಪ್ಸ-ರ-ಸಿ-ಯರ
ಅಪ್ಸ-ರ-ಸಿ-ಯರು
ಅಪ್ಸ-ರ-ಸ್ತ್ರೀ-ಯರು
ಅಪ್ಸರೆ
ಅಪ್ಸ-ರೆ-ಯನ್ನು
ಅಪ್ಸ-ರೆ-ಯರ
ಅಪ್ಸ-ರೆ-ಯರು
ಅಪ್ಸ-ರೆ-ಯರೇ
ಅಪ್ಸ-ರೆ-ಯ-ರೊ-ಡನೆ
ಅಬ್ಬ
ಅಬ್ಬ-ರಕ್ಕೆ
ಅಬ್ಬ-ರ-ವನ್ನು
ಅಬ್ಬರಿ
ಅಬ್ಬ-ರಿಸಿ
ಅಬ್ಬ-ರಿ-ಸಿ-ದನು
ಅಬ್ಬ-ರಿ-ಸುತ್ತಾ
ಅಬ್ಬಾ
ಅಭಯ
ಅಭ-ಯ-ಮ-ಭ-ಯ-ಮಾ-ತ್ಮನಿ
ಅಭ-ಯ-ವನ್ನು
ಅಭ-ಯ-ವಾ-ಣಿ-ಯನ್ನು
ಅಭ-ಯ-ವಿತ್ತು
ಅಭಾ-ಗ್ಯ-ತ-ರನ್ನೂ
ಅಭಾ-ವ-ವಿಲ್ಲ
ಅಭಿ
ಅಭಿ-ಚಾ-ರ-ಮಂ-ತ್ರ-ದಿಂದ
ಅಭಿ-ಚಾ-ರವೇ
ಅಭಿ-ಚಾ-ರಿ-ಕ-ಮಾಟ
ಅಭಿ-ಜಿತ್
ಅಭಿ-ಜಿ-ತ್ತೆಂಬ
ಅಭಿ-ನಂ-ದಿ-ಸಿ-ದರು
ಅಭಿ-ಪ್ರಾಯ
ಅಭಿ-ಪ್ರಾ-ಯ-ಗಳನ್ನೆಲ್ಲ
ಅಭಿ-ಪ್ರಾ-ಯ-ದಂತೆ
ಅಭಿ-ಪ್ರಾ-ಯ-ಭೇದ
ಅಭಿ-ಪ್ರಾ-ಯ-ವನ್ನು
ಅಭಿ-ಪ್ರಾ-ಯವೆ
ಅಭಿ-ಮಂ-ತ್ರಿ-ಸಿದ
ಅಭಿ-ಮ-ನ್ಯು-ವಿನ
ಅಭಿ-ಮಾನ
ಅಭಿ-ಮಾ-ನಕ್ಕೆ
ಅಭಿ-ಮಾ-ನ-ದಿಂದ
ಅಭಿ-ಮಾ-ನ-ದೇ-ವತೆ
ಅಭಿ-ಯುಕ್ತ
ಅಭಿ-ರುಚಿ
ಅಭಿ-ಲಾ-ಷೆ-ಯಿಂದ
ಅಭಿ-ವ್ಯ-ಕ್ತಿ-ಗಳು
ಅಭಿ-ಷೇಕ
ಅಭಿ-ಷೇ-ಕ-ಮಾ-ಡು-ತ್ತೇನೆ
ಅಭೀ-ಷ್ಟ-ವನ್ನು
ಅಭೇದ್ಯ
ಅಭ್ಯಂ-ಜನ
ಅಭ್ಯಾ-ಗ-ತ-ರನ್ನೂ
ಅಭ್ಯಾಸ
ಅಮಂ-ಗಳ
ಅಮಂ-ಗ-ಳ-ಕ-ರ-ವಾದ
ಅಮಂ-ಗ-ಳವೂ
ಅಮಂ-ಗ-ಳೆ-ಯಾದ
ಅಮ-ರ-ರಾ-ಗಿದ್ದ
ಅಮ-ರ-ಸಿಂ-ಹನು
ಅಮರಾ
ಅಮ-ರಾ-ವತಿ
ಅಮ-ರ್ಕ-ರಿಗೆ
ಅಮ-ರ್ಕರು
ಅಮಾ-ವಾ-ಸ್ಯೆಯ
ಅಮಾ-ವಾ-ಸ್ಯೆ-ಯಂದು
ಅಮೂ-ಲಾ-ಗ್ರ-ವಾಗಿ
ಅಮೂ-ಲ್ಯ-ವಾದ
ಅಮೃತ
ಅಮೃ-ತ-ಕ-ಲ-ಶ-ವನ್ನು
ಅಮೃ-ತ-ಕಿ-ವಿಗೆ
ಅಮೃ-ತ-ಕ್ಕಾಗಿ
ಅಮೃ-ತ-ಕ್ಕಿಂ-ತಲೂ
ಅಮೃ-ತತ್ವ
ಅಮೃ-ತದ
ಅಮೃ-ತದಂ
ಅಮೃ-ತ-ದಂ-ತಿದ್ದ
ಅಮೃ-ತ-ದಂ-ತಿ-ರುವ
ಅಮೃ-ತ-ದಂತೆ
ಅಮೃ-ತ-ದಿಂದ
ಅಮೃ-ತ-ದಿಂ-ದಲೆ
ಅಮೃ-ತ-ನಾ-ಗಿ-ದ್ದಾನೆ
ಅಮೃ-ತ-ಪಾನ
ಅಮೃ-ತ-ಪ್ರಾ-ಯ-ವಾ-ಗಿದೆ
ಅಮೃ-ತ-ಫ-ಲ-ವನ್ನು
ಅಮೃ-ತ-ಬಿಂ-ದು-ಗಳನ್ನು
ಅಮೃ-ತ-ಭಾಂ-ಡ-ದೊ-ಡನೆ
ಅಮೃ-ತ-ಭಾಂ-ಡ-ವನ್ನು
ಅಮೃ-ತ-ಮಯ
ಅಮೃ-ತ-ಮ-ಯ-ವಾದ
ಅಮೃ-ತ-ಮು-ದ-ಧಿ-ತ-ಶ್ಚಾ-ತ್ಪಾ-ಯ-ಯ-ದ್ಭೃ-ತ್ಯ-ವ-ರ್ಗಾನ್
ಅಮೃ-ತ-ವನ್ನು
ಅಮೃ-ತ-ವನ್ನೂ
ಅಮೃ-ತ-ವಾ-ಗ-ಬೇಕು
ಅಮೃ-ತ-ವಾ-ಣಿ-ಯಿಂದ
ಅಮೃ-ತ-ವಿದೆ
ಅಮೃ-ತ-ವೆಂದು
ಅಮೃ-ತ-ವೆಲ್ಲ
ಅಮೃ-ತಸ್ಯ
ಅಮೃ-ತ-ಹಸ್ತ
ಅಮೃ-ತ-ಹ-ಸ್ತ-ದಿಂದ
ಅಮೃ-ತ-ಹ-ಸ್ತ-ವನ್ನು
ಅಮೃ-ತಾ-ದಿ-ಗಳು
ಅಮೃ-ತಾ-ಧಿಕ
ಅಮೇಧ್ಯ
ಅಮೋ-ಘ-ವಾದ
ಅಮ್ಮ
ಅಮ್ಮ-ನಿಗೆ
ಅಮ್ಮಾ
ಅಯ-ಸ್ಕಾಂ-ತದ
ಅಯೋ-ಗ್ಯ-ಳಾದ
ಅಯೋ-ಧ್ಯಾ-ಪ-ಟ್ಟ-ಣ-ದಲ್ಲಿ
ಅಯೋ-ಧ್ಯೆಗೆ
ಅಯೋ-ಧ್ಯೆಯ
ಅಯೋ-ಧ್ಯೆ-ಯನ್ನು
ಅಯ್ಯಯ್ಯೊ
ಅಯ್ಯಾ
ಅಯ್ಯೊ
ಅಯ್ಯೋ
ಅರ
ಅರ-ಕೆ-ಯಾ-ಗದ
ಅರಕ್ಕೆ
ಅರ-ಗಿನ
ಅರ-ಗಿಳಿ
ಅರ-ಗಿ-ಹೋ-ಗಿವೆ
ಅರಚಿ
ಅರ-ಚಿ-ಕೊಂ-ಡರು
ಅರ-ಚಿ-ಕೊಂ-ಡವು
ಅರ-ಚಿ-ಕೊಂಡು
ಅರ-ಚು-ತ್ತಿದ್ದಿ
ಅರ-ಚು-ವಂತೆ
ಅರ-ಣ್ಯಕ್ಕೆ
ಅರ-ಣ್ಯ-ದಲ್ಲಿ
ಅರ-ಣ್ಯ-ದ-ಲ್ಲಿಯೂ
ಅರ-ಣ್ಯ-ಮಾ-ರ್ಗ-ವನ್ನು
ಅರ-ಮನೆ
ಅರ-ಮ-ನೆ-ಗಳನ್ನು
ಅರ-ಮ-ನೆ-ಗಳು
ಅರ-ಮ-ನೆಗೆ
ಅರ-ಮ-ನೆಯ
ಅರ-ಮ-ನೆ-ಯನ್ನು
ಅರ-ಮ-ನೆ-ಯನ್ನೆ
ಅರ-ಮ-ನೆ-ಯ-ನ್ನೆಲ್ಲ
ಅರ-ಮ-ನೆ-ಯಲ್ಲಿ
ಅರ-ಮ-ನೆ-ಯ-ಲ್ಲಿದ್ದ
ಅರ-ಮ-ನೆ-ಯ-ಲ್ಲಿಯೆ
ಅರ-ಮ-ನೆ-ಯ-ಲ್ಲಿ-ರುವ
ಅರ-ಮ-ನೆ-ಯ-ವರೆಲ್ಲ
ಅರ-ಮ-ನೆ-ಯ-ವರೆ-ಲ್ಲರೂ
ಅರ-ಮ-ನೆ-ಯಿಂದ
ಅರ-ಮ-ನೆಯೆ
ಅರ-ಮ-ನೆಯೇ
ಅರ-ಮ-ನೆ-ಯೊಂ-ದನ್ನು
ಅರ-ಮ-ನೆ-ಯೊ-ಳಕ್ಕೆ
ಅರ-ಮ-ನೆ-ಯೊ-ಳಗೆ
ಅರಳಿ
ಅರ-ಳಿತು
ಅರ-ಳಿದ
ಅರ-ಳಿ-ದರೆ
ಅರ-ಳಿ-ದವು
ಅರ-ಳಿ-ಮ-ರವೆ
ಅರ-ಳಿಯ
ಅರ-ಳಿ-ಸಿ-ಕೊಂಡು
ಅರ-ಳಿ-ಸಿ-ದನು
ಅರಳು
ಅರ-ಳು-ತ್ತಿ-ರುವ
ಅರ-ವತ್ತು
ಅರ-ವ-ತ್ತು-ಕೋಟಿ
ಅರಸ
ಅರ-ಸನ
ಅರ-ಸ-ನಾ-ಗಲಿ
ಅರ-ಸ-ನಾಗಿ
ಅರ-ಸ-ನಾ-ಗಿದ್ದ
ಅರ-ಸ-ರಿಗೆ
ಅರ-ಸರು
ಅರ-ಸ-ಹೊ-ರ-ಟಿದ್ದ
ಅರ-ಸಾ-ದ-ರೇನು
ಅರಸಿ
ಅರ-ಸಿ-ಕ-ನೇನೂ
ಅರ-ಸಿ-ಬಂ-ದಿ-ದ್ದೇನೆ
ಅರಸು
ಅರ-ಸುತ್ತ
ಅರ-ಸುತ್ತಾ
ಅರ-ಸು-ತ್ತಿದ್ದ
ಅರ-ಸು-ತ್ತಿ-ದ್ದಾನೆ
ಅರ-ಸುವ
ಅರ-ಸು-ವಂತೆ
ಅರ-ಸು-ವು-ದ-ಕ್ಕಾಗಿ
ಅರಾ-ಜ-ಕ-ವಾದ
ಅರಿ
ಅರಿಕೆ
ಅರಿ-ಜಿತ್ತು
ಅರಿತ
ಅರಿ-ತರು
ಅರಿ-ತರೆ
ಅರಿ-ತ-ವನಾ
ಅರಿ-ತ-ವ-ನಾಗಿ
ಅರಿ-ತ-ವನು
ಅರಿ-ತಿದ್ದ
ಅರಿ-ತಿ-ರು-ವ-ವರು
ಅರಿತು
ಅರಿ-ತು-ಕೊಂಡು
ಅರಿ-ತು-ಕೊಂಡೆ
ಅರಿ-ತು-ಕೊಂ-ಡೆನು
ಅರಿ-ತು-ಕೊ-ಳ್ಳ-ಬೇಕು
ಅರಿ-ತು-ಕೊ-ಳ್ಳ-ಲಾ-ರದೆ
ಅರಿ-ತು-ಕೊ-ಳ್ಳುವು
ಅರಿ-ತು-ಕೊ-ಳ್ಳು-ವು-ದಕ್ಕೆ
ಅರಿ-ತು-ಕೊ-ಳ್ಳು-ವುದು
ಅರಿತೆ
ಅರಿತೋ
ಅರಿ-ಯದ
ಅರಿ-ಯ-ದಂತೆ
ಅರಿ-ಯ-ದಂ-ತೆಯೆ
ಅರಿ-ಯ-ದ-ವ-ನಂತೆ
ಅರಿ-ಯ-ದ-ವನು
ಅರಿ-ಯ-ದಿ-ರು-ವಂತೆ
ಅರಿ-ಯದೆ
ಅರಿ-ಯ-ದೆಯೋ
ಅರಿ-ಯ-ಬ-ಲ್ಲರು
ಅರಿ-ಯ-ಲಾ-ರ-ದ-ವರು
ಅರಿ-ಯ-ಲಾ-ರರು
ಅರಿ-ಯಲು
ಅರಿ-ಯ-ಲೆಂದು
ಅರಿ-ಯಳು
ಅರಿ-ಯು-ವುದು
ಅರಿ-ವಾ-ಗು-ತ್ತದೆ
ಅರಿ-ವಾ-ಗು-ವುದು
ಅರಿ-ವಾ-ಯಿತು
ಅರಿ-ವಿ-ಲ್ಲ-ದಂ-ತಾ-ಯಿತು
ಅರಿ-ವಿ-ಲ್ಲ-ದಂ-ತೆಯೇ
ಅರಿವು
ಅರಿವೇ
ಅರಿ-ಶಿ-ನದ
ಅರಿ-ಷ-ಡ್ವ-ರ್ಗ-ಗ-ಳಿಗೆ
ಅರಿಷ್ಟ
ಅರಿ-ಷ್ಟ-ನಿಗೆ
ಅರಿ-ಷ್ಟನು
ಅರಿ-ಷ್ಟ-ನೆಂಬ
ಅರಿ-ಷ್ಟ-ರಿಂದ
ಅರಿ-ಸಿನ
ಅರುಂ-ಧ-ತಿ-ಯನ್ನೂ
ಅರು-ವತ್ತು
ಅರು-ವ-ತ್ತು-ನಾಲ್ಕು
ಅರು-ಹಿ-ದನು
ಅರೆ
ಅರೆ-ಕಾ-ಸಿನ
ಅರೆದು
ಅರೆ-ಯ-ರ-ಳಿ-ರುವ
ಅರೆಸಿ
ಅರ್ಕ
ಅರ್ಚಿ
ಅರ್ಚಿಯು
ಅರ್ಚಿ-ರಾದಿ
ಅರ್ಚಿ-ರಾ-ದಿ-ಗತಿ
ಅರ್ಚಿ-ರಾ-ದಿ-ಗ-ತಿಯೇ
ಅರ್ಜುನ
ಅರ್ಜು-ನನ
ಅರ್ಜು-ನ-ನಂತೆ
ಅರ್ಜು-ನ-ನನ್ನು
ಅರ್ಜು-ನ-ನಿಗೆ
ಅರ್ಜು-ನನು
ಅರ್ಜು-ನನೂ
ಅರ್ಜು-ನ-ನೆಂದು
ಅರ್ಜು-ನ-ನೊಂದು
ಅರ್ಜು-ನ-ನೊ-ಡನೆ
ಅರ್ಜು-ನ-ನೊ-ಡ-ನೆ-ಸ್ವಾಮಿ
ಅರ್ಜು-ನನ್ನೂ
ಅರ್ಜು-ನರು
ಅರ್ಥ
ಅರ್ಥ-ಕಾ-ಮ-ಗಳನ್ನು
ಅರ್ಥ-ದ-ಲ್ಲಿಯೇ
ಅರ್ಥ-ಬ-ರು-ವಂತೆ
ಅರ್ಥ-ಮಾಡಿ
ಅರ್ಥ-ಮಾ-ಡಿ-ಕೊಂಡ
ಅರ್ಥ-ಮಾ-ಡಿ-ಕೊಂ-ಡನು
ಅರ್ಥ-ಮಾ-ಡಿ-ಕೊಂ-ಡರು
ಅರ್ಥ-ಮಾ-ಡಿ-ಕೊಂ-ಡಿದ್ದ
ಅರ್ಥ-ಮಾ-ಡಿ-ಕೊಂ-ಡಿ-ದ್ದರು
ಅರ್ಥ-ಮಾ-ಡಿ-ಕೊಂ-ಡಿ-ರು-ವೆ-ಯಷ್ಟೆ
ಅರ್ಥ-ಮಾ-ಡಿ-ಕೊಂಡೆ
ಅರ್ಥ-ಮಾ-ಡಿ-ಕೊಂ-ಡೆನು
ಅರ್ಥ-ಮಾ-ಡಿ-ಕೊಳ್ಳ
ಅರ್ಥ-ಮಾ-ಡಿ-ಕೊ-ಳ್ಳ-ಬಲ್ಲ
ಅರ್ಥ-ಮಾ-ಡಿ-ಕೊ-ಳ್ಳ-ಬ-ಹುದು
ಅರ್ಥ-ಮಾ-ಡಿ-ಕೊ-ಳ್ಳ-ಬೇ-ಕಾ-ದರೆ
ಅರ್ಥ-ಮಾ-ಡಿ-ಕೊ-ಳ್ಳ-ಬೇಕು
ಅರ್ಥ-ಮಾ-ಡಿ-ಕೊ-ಳ್ಳ-ಲಾ-ರದೆ
ಅರ್ಥ-ಮಾ-ಡಿ-ಕೊ-ಳ್ಳ-ಲಾ-ರ-ರಲ್ಲಾ
ಅರ್ಥ-ಮಾ-ಡಿ-ಕೊ-ಳ್ಳ-ಲಾರೆ
ಅರ್ಥ-ಮಾ-ಡಿ-ಕೊ-ಳ್ಳ-ಲೆಂದು
ಅರ್ಥ-ಮಾ-ಡಿ-ಕೊಳ್ಳಿ
ಅರ್ಥ-ಮಾ-ಡಿ-ಕೊ-ಳ್ಳು-ತ್ತಾರೆ
ಅರ್ಥ-ಮಾ-ಡಿ-ಕೊ-ಳ್ಳು-ವುದು
ಅರ್ಥ-ಮಾ-ಡಿ-ಕೊ-ಳ್ಳು-ವು-ದೆಂ-ದರೆ
ಅರ್ಥ-ವ-ತ್ತಾ-ಗಿದೆ
ಅರ್ಥ-ವನ್ನು
ಅರ್ಥ-ವಾ-ಗದ
ಅರ್ಥ-ವಾ-ಗ-ದ-ವರು
ಅರ್ಥ-ವಾ-ಗ-ದುದೇ
ಅರ್ಥ-ವಾ-ಗ-ಲಿಲ್ಲ
ಅರ್ಥ-ವಾ-ಗಲೆ
ಅರ್ಥ-ವಾ-ಗಿ-ರು-ವು-ದ-ರಿಂದ
ಅರ್ಥ-ವಾ-ಗುತ್ತ
ಅರ್ಥ-ವಾ-ಗು-ತ್ತದೆ
ಅರ್ಥ-ವಾ-ಗು-ತ್ತಾ-ನೆಯೇ
ಅರ್ಥ-ವಾ-ಗು-ತ್ತಿಲ್ಲ
ಅರ್ಥ-ವಾ-ಗುವ
ಅರ್ಥ-ವಾ-ಗು-ವಂ-ತ-ಹು-ದಲ್ಲ
ಅರ್ಥ-ವಾ-ಗು-ವಂತೆ
ಅರ್ಥ-ವಾ-ಗು-ವು-ದಿಲ್ಲ
ಅರ್ಥ-ವಾ-ಗು-ವುದು
ಅರ್ಥ-ವಾ-ಯಿತು
ಅರ್ಥ-ವಾ-ಯಿ-ತು-ಕೇ-ವಲ
ಅರ್ಥ-ವಾ-ಯಿತೊ
ಅರ್ಥ-ವಾ-ಯಿತೋ
ಅರ್ಥ-ವಿದೆ
ಅರ್ಥ-ವಿ-ಲ್ಲದೆ
ಅರ್ಥವೆ
ಅರ್ಥ-ವೇನು
ಅರ್ಥ-ಶಾ-ಸ್ತ್ರ-ದಲ್ಲಿ
ಅರ್ಧ
ಅರ್ಧಕ್ಕೆ
ಅರ್ಧ-ಚಂ-ದ್ರಾ-ಕಾ-ರ-ವಾಗಿ
ಅರ್ಧ-ದ-ಲ್ಲಿಯೇ
ಅರ್ಧ-ಬ-ಲ-ವನ್ನು
ಅರ್ಧ-ಭಾ-ಗಕ್ಕೆ
ಅರ್ಧ-ಭಾ-ಗ-ವನ್ನು
ಅರ್ಧ-ರಾಜ್ಯ
ಅರ್ಧ-ರಾ-ಜ್ಯ-ಕ್ಕಾಗಿ
ಅರ್ಧ-ರಾ-ಜ್ಯ-ವನ್ನು
ಅರ್ಧ-ರಾ-ತ್ರಿ-ಯಲ್ಲಿ
ಅರ್ಧ-ವನ್ನು
ಅರ್ಧ-ಹಾದಿ
ಅರ್ಧಾ-ಸ-ನ-ವನ್ನು
ಅರ್ಪಣೆ
ಅರ್ಪಿ-ತ-ವಾ-ದಾಗ
ಅರ್ಪಿ-ಸ-ಬೇಕು
ಅರ್ಪಿಸಿ
ಅರ್ಪಿ-ಸಿದ
ಅರ್ಪಿ-ಸಿ-ದನು
ಅರ್ಪಿ-ಸಿ-ದರು
ಅರ್ಪಿ-ಸಿ-ದ್ದೇನೆ
ಅರ್ಪಿ-ಸಿ-ರು-ವೆ-ವ-ಲ್ಲವೆ
ಅರ್ಪಿ-ಸು-ವ-ನಾ-ದ್ದ-ರಿಂದ
ಅರ್ವಾ-ಚೀನ
ಅರ್ಹ
ಅರ್ಹ-ರಾ-ದ-ವರು
ಅಲಂ
ಅಲಂ-ಕ-ರಿಸಿ
ಅಲಂ-ಕ-ರಿ-ಸಿ-ಕೊಂಡು
ಅಲಂ-ಕ-ರಿ-ಸಿತು
ಅಲಂ-ಕ-ರಿ-ಸಿದ
ಅಲಂ-ಕ-ರಿ-ಸಿ-ದನು
ಅಲಂ-ಕ-ರಿ-ಸಿ-ದರು
ಅಲಂ-ಕ-ರಿ-ಸಿ-ದ್ದಾರೆ
ಅಲಂ-ಕ-ರಿ-ಸಿ-ರುವ
ಅಲಂ-ಕ-ರಿ-ಸುವ
ಅಲಂ-ಕಾರ
ಅಲಂ-ಕಾ-ರ-ಗ-ಳಿಗೆ
ಅಲಂ-ಕಾ-ರ-ದಂ-ತಿ-ರುವ
ಅಲಂ-ಕಾ-ರ-ಮಾ-ಡಿ-ಕೊಂಡು
ಅಲಂ-ಕಾ-ರ-ವ-ನ್ನೆಲ್ಲ
ಅಲಂ-ಕಾ-ರ-ವಾ-ದಂ-ತಾ-ಯಿತು
ಅಲಂ-ಕಾ-ರಾದಿ
ಅಲಂ-ಕೃತ
ಅಲಂ-ಕೃ-ತ-ನಾಗಿ
ಅಲಂ-ಕೃ-ತ-ರಾಗಿ
ಅಲಂ-ಕೃ-ತ-ರಾ-ಗಿ-ದ್ದಾರೆ
ಅಲಂ-ಕೃ-ತ-ವಾ-ಗಿದ್ದ
ಅಲಂ-ಕೃ-ತ-ವಾ-ಗಿ-ದ್ದವು
ಅಲಂ-ಕೃ-ತ-ವಾ-ಗಿವೆ
ಅಲಂ-ಕೃ-ತ-ವಾದ
ಅಲಂ-ಕೃ-ತ-ವಾ-ಯಿತು
ಅಲ-ಕ-ನಂ-ದೆ-ಗಳು
ಅಲ-ಕಾ-ಪು-ರಿಯೂ
ಅಲ-ಕಾ-ವ-ತಿಗೆ
ಅಲಕ್ಷ್ಯಂ
ಅಲ-ಕ್ಷ್ಯ-ವನ್ನು
ಅಲ-ಗು-ಗಳಿಂದ
ಅಲ-ಗು-ಗ-ಳುಳ್ಳ
ಅಲ-ಗು-ತ್ತಿ-ರುವ
ಅಲ-ಬ್ಧ-ನಿ-ದ್ರೋ-ಽಧಿ-ಗ-ತಾ-ತ್ಮಜಾ
ಅಲಭ್ಯ
ಅಲ-ರ್ಕ-ನೆಂಬ
ಅಲು-ಗಾ-ಡದೆ
ಅಲೆ
ಅಲೆ-ಅ-ಲೆ-ಯಾಗಿ
ಅಲೆ-ಗಳಿಂದ
ಅಲೆ-ಗಳು
ಅಲೆ-ಡಾ-ಡು-ತ್ತಿ-ರುವೆ
ಅಲೆದ
ಅಲೆ-ದಂತೆ
ಅಲೆ-ದಾಡ
ಅಲೆ-ದಾಡಿ
ಅಲೆ-ದಾ-ಡಿ-ದರು
ಅಲೆ-ದಾ-ಡುತ್ತ
ಅಲೆದು
ಅಲೆ-ದು-ದಾ-ಯಿತು
ಅಲೆ-ಯ-ಬೇ-ಕಾ-ಗು-ತ್ತದೆ
ಅಲೆಯು
ಅಲೆ-ಯುತ್ತಾ
ಅಲೆ-ಯು-ತ್ತಿದ್ದ
ಅಲೆ-ಯು-ತ್ತಿದ್ದು
ಅಲೆ-ಯು-ತ್ತಿ-ರಲು
ಅಲೆ-ಯು-ತ್ತಿ-ರು-ವನೊ
ಅಲೆ-ಯು-ತ್ತಿ-ರು-ವಾಗ
ಅಲೆ-ಯು-ವು-ದೇಕೆ
ಅಲೆ-ಯೆದ್ದು
ಅಲೋ-ಕ-ದಲ್ಲಿ
ಅಲೋ-ಕ-ವೆಂದು
ಅಲ್ಪ
ಅಲ್ಪ-ಕಾ-ಲ-ದ-ಲ್ಲಿ-ಪಾ-ಪ-ಗಳನ್ನೆಲ್ಲ
ಅಲ್ಪ-ತೃಪ್ತ
ಅಲ್ಪನ
ಅಲ್ಪ-ಪು-ರಾ-ಣಕ್ಕೆ
ಅಲ್ಪ-ಪು-ರಾ-ಣ-ಗಳು
ಅಲ್ಪ-ಪ್ರ-ಮಾ-ಣ-ವಾ-ಗು-ತ್ತವೆ
ಅಲ್ಪ-ಬು-ದ್ಧಿಯ
ಅಲ್ಪ-ವಾದ
ಅಲ್ಪ-ಸ್ವ-ಲ್ಪ-ವಿ-ದ್ದರೆ
ಅಲ್ಪಾಯು
ಅಲ್ಪಾ-ಯು-ಸ್ಸು-ಇ-ವು-ಗಳ
ಅಲ್ಲ
ಅಲ್ಲ-ಗ-ಳೆ-ಯ-ಲಿಲ್ಲ
ಅಲ್ಲದ
ಅಲ್ಲದೆ
ಅಲ್ಲಪ್ಪ
ಅಲ್ಲಯ್ಯ
ಅಲ್ಲಲ್ಲಿ
ಅಲ್ಲ-ಲ್ಲಿನ
ಅಲ್ಲ-ಲ್ಲಿಯೆ
ಅಲ್ಲ-ಲ್ಲಿಯೇ
ಅಲ್ಲ-ಲ್ಲಿ-ರುವ
ಅಲ್ಲಲ್ಲೆ
ಅಲ್ಲವೆ
ಅಲ್ಲವೋ
ಅಲ್ಲಾ-ಡಿಸಿ
ಅಲ್ಲಾ-ಡಿ-ಸಿದ
ಅಲ್ಲಾ-ಡಿ-ಸುತ್ತಾ
ಅಲ್ಲಾ-ಡು-ವುದನ್ನು
ಅಲ್ಲಾ-ಡು-ವು-ದಿಲ್ಲ
ಅಲ್ಲಿ
ಅಲ್ಲಿಂದ
ಅಲ್ಲಿಂ-ದಲೂ
ಅಲ್ಲಿಂ-ದೀ-ಚೆಗೂ
ಅಲ್ಲಿ-ಗಿ-ಳಿದು
ಅಲ್ಲಿಗೂ
ಅಲ್ಲಿಗೆ
ಅಲ್ಲಿಗೇ
ಅಲ್ಲಿಟ್ಟು
ಅಲ್ಲಿತ್ತು
ಅಲ್ಲಿದ್ದ
ಅಲ್ಲಿ-ದ್ದ-ವ-ರ-ನ್ನೆಲ್ಲ
ಅಲ್ಲಿ-ದ್ದ-ವರೆಲ್ಲ
ಅಲ್ಲಿ-ದ್ದ-ವರೆ-ಲ್ಲರೂ
ಅಲ್ಲಿನ
ಅಲ್ಲಿ-ನ-ವ-ರಿಂದ
ಅಲ್ಲಿ-ನ-ವ-ರಿಗೆ
ಅಲ್ಲಿ-ನ-ವ-ರೆಗೆ
ಅಲ್ಲಿ-ನ-ವರೆ-ಲ್ಲರೂ
ಅಲ್ಲಿಯ
ಅಲ್ಲಿ-ಯ-ವ-ರೆಗೆ
ಅಲ್ಲಿಯೂ
ಅಲ್ಲಿಯೆ
ಅಲ್ಲಿಯೇ
ಅಲ್ಲಿ-ರ-ಲಿಲ್ಲ
ಅಲ್ಲಿ-ರುವ
ಅಲ್ಲಿ-ರು-ವ-ವರೆಲ್ಲ
ಅಲ್ಲಿ-ರು-ವಾಗ
ಅಲ್ಲಿಲ್ಲ
ಅಲ್ಲೆ
ಅಲ್ಲೊಂದು
ಅಲ್ಲೊಬ್ಬ
ಅಲ್ಲೋಲ
ಅಲ್ಲೋ-ಲ-ಕ-ಲ್ಲೋಲ
ಅಲ್ಲೋ-ಲ-ಕ-ಲ್ಲೋ-ಲ-ವಾ-ಗಿದೆ
ಅಲ್ಲೋ-ಲ-ಕ-ಲ್ಲೋ-ಲ-ವಾದ
ಅಲ್ಲೋ-ಲ-ಕ-ಲ್ಲೋ-ಲ-ವಾ-ಯಿತು
ಅಳ-ತೆ-ಯಲ್ಲಿ
ಅಳ-ತೊ-ಡ-ಗಿ-ದರು
ಅಳ-ಬೇ-ಕಾ-ದ್ದಿಲ್ಲ
ಅಳ-ಬೇಕು
ಅಳ-ಬೇಡ
ಅಳ-ಬೇ-ಡ-ಎಂದು
ಅಳ-ವ-ಡಿ-ಸಿ-ಕೊಂ-ಡಿದ್ದ
ಅಳ-ವ-ಡಿ-ಸಿ-ಕೊಂ-ಡಿ-ದ್ದನು
ಅಳ-ವ-ಡಿ-ಸಿ-ಕೊಂಡು
ಅಳ-ವ-ಡಿ-ಸು-ವುದೇ
ಅಳಿ-ದ-ರುಂ
ಅಳಿದು
ಅಳಿ-ದು-ಳಿ-ದ-ವರು
ಅಳಿ-ದು-ಳಿ-ದಿದ್ದ
ಅಳಿಯ
ಅಳಿ-ಯಂ-ದಿರು
ಅಳಿ-ಯ-ದಂತೆ
ಅಳಿ-ಯ-ನನ್ನು
ಅಳಿ-ಯ-ನನ್ನೂ
ಅಳಿ-ಯ-ನಾ-ಗಿ-ರುವ
ಅಳಿ-ಯ-ನಾದ
ಅಳಿ-ಯ-ನಿಂದ
ಅಳಿ-ಯ-ನಿಂ-ದಲೂ
ಅಳಿ-ಯ-ನಿಗೆ
ಅಳಿ-ಯಿತು
ಅಳಿಸಿ
ಅಳಿ-ಸಿ-ಹೋಗಿ
ಅಳಿ-ಸಿ-ಹೋ-ಗಿವೆ
ಅಳಿ-ಸಿ-ಹೋ-ಗು-ತ್ತದೆ
ಅಳಿ-ಸು-ವನು
ಅಳಿ-ಸುವೆ
ಅಳು
ಅಳು-ಕದೆ
ಅಳುಕು
ಅಳು-ತ್ತ-ಳುತ್ತ
ಅಳುತ್ತಾ
ಅಳು-ತ್ತಾನೆ
ಅಳುತ್ತಿ
ಅಳು-ತ್ತಿದ್ದ
ಅಳು-ತ್ತಿ-ದ್ದಾರೆ
ಅಳು-ತ್ತಿ-ರಲು
ಅಳು-ತ್ತಿ-ರು-ವಂತೆ
ಅಳು-ತ್ತಿ-ರು-ವೆಯೋ
ಅಳು-ತ್ತೀಯಾ
ಅಳು-ಮು-ಖ-ದಿಂದ
ಅಳು-ಮೋರೆ
ಅಳು-ಮೋ-ರೆ-ಯನ್ನು
ಅಳು-ವನು
ಅಳು-ವನ್ನು
ಅಳುವು
ಅಳು-ವು-ದಕ್ಕೆ
ಅಳು-ವುದನ್ನು
ಅಳು-ವು-ದಿಲ್ಲ
ಅಳು-ವುದು
ಅಳು-ವು-ದೇಕೆ
ಅಳುವೆ
ಅಳು-ವೆ-ಯ-ಲ್ಲವೆ
ಅಳೆದ
ಅಳೆದು
ಅವ
ಅವಂ-ತಿ-ದೇ-ಶದ
ಅವಂ-ತೀ-ಪು-ರ-ದ-ಲ್ಲಿದ್ದ
ಅವ-ಅ-ಅ-ರ-ನ್ನೆಲ್ಲ
ಅವ-ಕಾಶ
ಅವ-ಕಾ-ಶ-ವನ್ನು
ಅವ-ಕಾ-ಶ-ವಾ-ಗು-ತ್ತದೆ
ಅವ-ಕಾ-ಶ-ವಿತ್ತು
ಅವ-ಕಾ-ಶ-ವಿದೆ
ಅವ-ಕಾ-ಶ-ವಿ-ರ-ಲಿ-ಲ್ಲ-ವಾ-ದ್ದ-ರಿಂದ
ಅವ-ಕಾ-ಶ-ವಿಲ್ಲ
ಅವ-ಕಾ-ಶ-ವಿ-ಲ್ಲದ
ಅವ-ಕಾ-ಶ-ವಿ-ಲ್ಲ-ದಂ-ತಾ-ಯಿತು
ಅವ-ಕಾ-ಶ-ವಿ-ಲ್ಲ-ದಂತೆ
ಅವ-ಕಾ-ಶ-ವಿ-ಲ್ಲದೆ
ಅವ-ಕಾ-ಶವೇ
ಅವ-ಕ್ಕಿಂ-ತಲೂ
ಅವಕ್ಕೆ
ಅವಜ್ಞೆ
ಅವ-ತ-ರ-ಣ-ವನ್ನು
ಅವ-ತರಿ
ಅವ-ತ-ರಿಸಿ
ಅವ-ತ-ರಿ-ಸಿತು
ಅವ-ತ-ರಿ-ಸಿದ
ಅವ-ತ-ರಿ-ಸಿ-ದನು
ಅವ-ತ-ರಿ-ಸಿ-ದರು
ಅವ-ತ-ರಿ-ಸಿ-ದಾಗ
ಅವ-ತ-ರಿ-ಸಿ-ದ್ದೀರಿ
ಅವ-ತ-ರಿ-ಸಿ-ದ್ದುದೇ
ಅವ-ತ-ರಿ-ಸಿ-ದ್ದೇನೆ
ಅವ-ತ-ರಿ-ಸಿ-ಬಂದು
ಅವ-ತ-ರಿ-ಸಿ-ರುವ
ಅವ-ತ-ರಿ-ಸಿ-ರು-ವಿರಿ
ಅವ-ತ-ರಿ-ಸಿ-ರು-ವುದು
ಅವ-ತ-ರಿ-ಸಿ-ರುವೆ
ಅವ-ತ-ರಿ-ಸು-ತ್ತಲೆ
ಅವ-ತ-ರಿ-ಸು-ತ್ತಾನೆ
ಅವ-ತ-ರಿ-ಸು-ತ್ತಿ-ರುವೆ
ಅವ-ತ-ರಿ-ಸುವ
ಅವ-ತಾರ
ಅವ-ತಾ-ರ-ಕಾರ್ಯ
ಅವ-ತಾ-ರ-ಕಾ-ರ್ಯದ
ಅವ-ತಾ-ರ-ಗಳ
ಅವ-ತಾ-ರ-ಗಳನ್ನು
ಅವ-ತಾ-ರ-ಗಳನ್ನೂ
ಅವ-ತಾ-ರ-ಗ-ಳ-ನ್ನೆತ್ತಿ
ಅವ-ತಾ-ರ-ಗ-ಳ-ನ್ನೆ-ತ್ತುವೆ
ಅವ-ತಾ-ರ-ಗಳಲ್ಲಿ
ಅವ-ತಾ-ರ-ಗ-ಳಾ-ವುವು
ಅವ-ತಾ-ರ-ಗಳಿಂದ
ಅವ-ತಾ-ರ-ಗ-ಳಿಗೂ
ಅವ-ತಾ-ರ-ಗಳು
ಅವ-ತಾ-ರ-ಗ-ಳೆಲ್ಲ
ಅವ-ತಾ-ರದ
ಅವ-ತಾ-ರ-ದಲ್ಲಿ
ಅವ-ತಾ-ರ-ಪು-ರು-ಷ-ನಾದ
ಅವ-ತಾ-ರ-ಪು-ರು-ಷರು
ಅವ-ತಾ-ರ-ವಂತೆ
ಅವ-ತಾ-ರ-ವನ್ನು
ಅವ-ತಾ-ರ-ವ-ನ್ನೆತ್ತಿ
ಅವ-ತಾ-ರ-ವಾ-ಗು-ತ್ತದೆ
ಅವ-ತಾ-ರ-ವಾದ
ಅವ-ತಾ-ರ-ವಾ-ದರೂ
ಅವ-ತಾ-ರ-ವಾ-ಯಿತು
ಅವ-ತಾ-ರವೆ
ಅವ-ತಾ-ರ-ವೆಂದರೆ
ಅವ-ತಾ-ರ-ವೆಂದು
ಅವ-ತಾ-ರ-ವೆಂದೇ
ಅವ-ತಾ-ರ-ವೆ-ತ್ತುವ
ಅವ-ತಾ-ರ-ವೆ-ತ್ತು-ವು-ದೇಕೆ
ಅವ-ತಾ-ರ-ವೆ-ನಿ-ಸಿ-ಕೊಂ-ಡಿ-ರುವ
ಅವ-ತಾ-ರವೇ
ಅವ-ತಾ-ರಿ-ಯಾದ
ಅವಧಿ
ಅವ-ಧಿ-ಯ-ಲ್ಲಿ-ಇಷ್ಟು
ಅವ-ಧಿ-ಯ-ಲ್ಲಿಯೇ
ಅವ-ಧಿ-ಯಿದೆ
ಅವ-ಧೂ-ತ-ನಾ-ದನು
ಅವ-ಧೂ-ತ-ವೇ-ಷ-ದಿಂದ
ಅವನ
ಅವ-ನಂ-ತಹ
ಅವ-ನಂತೆ
ಅವ-ನಂ-ತೆಯೆ
ಅವ-ನಂ-ತೆಯೇ
ಅವ-ನತ್ತ
ಅವ-ನ-ನನ್ನು
ಅವ-ನ-ನ್ನಾ-ಗಲೀ
ಅವ-ನನ್ನು
ಅವ-ನನ್ನೂ
ಅವ-ನನ್ನೆ
ಅವ-ನನ್ನೇ
ಅವ-ನ-ಪ್ಪನೂ
ಅವ-ನ-ಮೇಲೆ
ಅವ-ನಲ್ಲ
ಅವ-ನ-ಲ್ಲ-ಅ-ನಿ-ರುದ್ಧ
ಅವ-ನ-ಲ್ಲವೆ
ಅವ-ನಲ್ಲಿ
ಅವ-ನ-ಲ್ಲಿದ್ದ
ಅವ-ನ-ವನ
ಅವ-ನಾ-ರೆಂ-ಬುದು
ಅವ-ನಾರೋ
ಅವ-ನಾ-ವನೊ
ಅವ-ನಿಂದ
ಅವ-ನಿ-ಗಾ-ಗ-ಲಿಲ್ಲ
ಅವ-ನಿ-ಗಾಗಿ
ಅವ-ನಿ-ಗಿಂತ
ಅವ-ನಿ-ಗಿಂ-ತಲೂ
ಅವ-ನಿ-ಗಿನ್ನೂ
ಅವ-ನಿಗೂ
ಅವ-ನಿಗೆ
ಅವ-ನಿ-ಗೇನು
ಅವ-ನಿ-ಗೊಂದು
ಅವ-ನಿದ್ದ
ಅವ-ನಿ-ದ್ದ-ಲ್ಲಿಗೆ
ಅವ-ನಿ-ಲ್ಲದ
ಅವ-ನೀಗ
ಅವನು
ಅವನೂ
ಅವನೆ
ಅವ-ನೆ-ದುರು
ಅವ-ನೆಲ್ಲಿ
ಅವನೇ
ಅವ-ನೇಕೆ
ಅವ-ನೇನು
ಅವ-ನೊಂದು
ಅವ-ನೊ-ಡನೆ
ಅವನ್ನು
ಅವನ್ನೂ
ಅವ-ನ್ನೆಲ್ಲ
ಅವ-ಭೃ-ತ-ಸ್ನಾನ
ಅವ-ಭೃ-ತ-ಸ್ನಾ-ನ-ಕ್ಕೆಂದು
ಅವ-ಮಾನ
ಅವ-ಮಾ-ನ-ಕ-ರ-ವೆಂ-ದರು
ಅವ-ಮಾ-ನ-ಗಳನ್ನು
ಅವ-ಮಾ-ನ-ಗೊ-ಳಿ-ಸಿದ
ಅವ-ಮಾ-ನ-ದಿಂದ
ಅವ-ಮಾ-ನ-ಪ-ಡಿ-ಸಿ-ದರೂ
ಅವ-ಮಾ-ನ-ಮಾ-ಡ-ಬೇಡ
ಅವ-ಮಾ-ನ-ಮಾಡಿ
ಅವ-ಮಾ-ನ-ಮಾ-ಡಿ-ದುದು
ಅವ-ಮಾ-ನ-ಮಾ-ಡು-ತ್ತಿ-ದ್ದಾನೆ
ಅವ-ಮಾ-ನ-ಮಾ-ಡು-ತ್ತಿ-ರು-ವ-ನೆಂದು
ಅವ-ಮಾ-ನ-ವನ್ನು
ಅವ-ಮಾ-ನ-ವಾ-ಗು-ವಂತೆ
ಅವ-ಮಾ-ನ-ವಾ-ದಂ-ತಾ-ಯಿತು
ಅವ-ಮಾ-ನ-ವಾಯಿ
ಅವ-ಮಾ-ನ-ವೇನೂ
ಅವ-ಮಾ-ನಿ-ತ-ನಾಗಿ
ಅವ-ಯವ
ಅವ-ಯ-ವ-ಗಳನ್ನೂ
ಅವ-ಯ-ವ-ಗ-ಳಾದ
ಅವ-ಯ-ವ-ಗಳಿಂದ
ಅವ-ಯ-ವ-ಗಳು
ಅವ-ಯ-ವ-ಗ-ಳು-ಳ್ಳ-ವ-ನಾಗಿ
ಅವ-ಯ-ವದ
ಅವ-ಯ-ವ-ನಾಗಿ
ಅವರ
ಅವ-ರಂ-ತೆಯೇ
ಅವ-ರತ್ತ
ಅವ-ರ-ದಾಗಿ
ಅವ-ರದು
ಅವ-ರನ್ನು
ಅವ-ರನ್ನೂ
ಅವ-ರ-ನ್ನೆಲ್ಲ
ಅವ-ರ-ನ್ನೆಲ್ಲಾ
ಅವ-ರಪ್ಪ
ಅವ-ರ-ಮೇಲೆ
ಅವ-ರಲ್ಲಿ
ಅವ-ರ-ಲ್ಲಿದ್ದ
ಅವ-ರ-ಲ್ಲಿಯೇ
ಅವ-ರ-ಲ್ಲಿ-ರುವ
ಅವ-ರಲ್ಲೆ
ಅವ-ರ-ಲ್ಲೊ-ಬ್ಬಳು
ಅವ-ರ-ವರ
ಅವ-ರ-ವರು
ಅವ-ರಾರೋ
ಅವರಿ
ಅವ-ರಿಂದ
ಅವ-ರಿ-ಗಾಗಿ
ಅವ-ರಿ-ಗಾದ
ಅವ-ರಿ-ಗಿ-ರುವ
ಅವ-ರಿ-ಗೀಗ
ಅವ-ರಿಗೂ
ಅವ-ರಿಗೆ
ಅವ-ರಿ-ಗೆಲ್ಲ
ಅವ-ರಿತ್ತ
ಅವ-ರಿ-ದು-ರಿಗೆ
ಅವ-ರಿ-ದ್ದೆಡೆ
ಅವ-ರಿ-ಬ್ಬರ
ಅವ-ರಿ-ಬ್ಬ-ರನ್ನು
ಅವ-ರಿ-ಬ್ಬ-ರನ್ನೂ
ಅವ-ರಿ-ಬ್ಬ-ರಲ್ಲಿ
ಅವ-ರಿ-ಬ್ಬ-ರಿಗೂ
ಅವ-ರಿ-ಬ್ಬರೂ
ಅವ-ರಿ-ರಲಿ
ಅವ-ರಿ-ವ-ರಿ-ರಲಿ
ಅವ-ರಿ-ವರೆ-ನ್ನದೆ
ಅವ-ರೀಗ
ಅವ-ರೀ-ರ್ವ-ರನ್ನೂ
ಅವ-ರೀ-ರ್ವರೂ
ಅವರು
ಅವ-ರು-ಗ-ಳಿ-ಗೆಲ್ಲ
ಅವರೂ
ಅವರೆ
ಅವ-ರೆಂ-ತಹ
ಅವರೆ-ನ್ನೆಲ್ಲ
ಅವರೆಲ್ಲ
ಅವರೆ-ಲ್ಲರ
ಅವರೆ-ಲ್ಲ-ರನ್ನೂ
ಅವರೆ-ಲ್ಲ-ರಿಂದ
ಅವರೆ-ಲ್ಲ-ರಿಗೂ
ಅವರೆ-ಲ್ಲರೂ
ಅವರೇ
ಅವ-ರೇಕೆ
ಅವ-ರೇನು
ಅವ-ರೊ-ಡನೆ
ಅವ-ರೊ-ಡ್ಡಿದ
ಅವ-ರೊ-ಬ್ಬ-ಬ್ಬ-ರಿಗೂ
ಅವ-ರೊ-ಬ್ಬ-ರೊ-ಬ್ಬರೂ
ಅವ-ರೊಮ್ಮೆ
ಅವ-ಲಂ-ಬಿಸಿ
ಅವ-ಲಂ-ಬಿ-ಸಿ-ದನು
ಅವ-ಲಕ್ಕಿ
ಅವ-ಲ-ಕ್ಕಿ-ಯನ್ನು
ಅವಳ
ಅವ-ಳಂ-ತಹ
ಅವ-ಳನ್ನು
ಅವ-ಳನ್ನೆ
ಅವ-ಳನ್ನೇ
ಅವ-ಳಲ್ಲಿ
ಅವ-ಳ-ಲ್ಲಿಯೇ
ಅವ-ಳಾ-ರೆಂದು
ಅವಳಿ
ಅವ-ಳಿಂದ
ಅವ-ಳಿ-ಗಾಗಿ
ಅವ-ಳಿಗೆ
ಅವ-ಳಿ-ಗೆಷ್ಟು
ಅವ-ಳಿ-ಯಾದ
ಅವ-ಳಿ-ಲ್ಲದೆ
ಅವಳು
ಅವಳೂ
ಅವ-ಳೇನು
ಅವ-ಳೊಬ್ಬ
ಅವ-ಳೊ-ಬ್ಬಳೇ
ಅವ-ಳೊಮ್ಮೆ
ಅವ-ಶ್ಯ-ಕ-ವಿಲ್ಲ
ಅವ-ಷ್ಟನ್ನೂ
ಅವ-ಸ್ಥೆ-ಗಳನ್ನೂ
ಅವ-ಸ್ಥೆ-ಗ-ಳಿಗೆ
ಅವ-ಸ್ಥೆ-ಗಳೂ
ಅವ-ಸ್ಥೆಯೂ
ಅವ-ಹೇ-ಳ-ನ-ವಾ-ಗಲಿ
ಅವಾ-ಙ್ಮಾ-ನ-ಸ-ಗೋ-ಚ-ರ-ನಾದ
ಅವಾ-ಙ್ಮಾ-ನ-ಸ-ಗೋ-ಚ-ರನೂ
ಅವಿ
ಅವಿ-ಜ್ಞಾತ
ಅವಿ-ಜ್ಞಾ-ತನು
ಅವಿ-ತಿ-ಟ್ಟಳು
ಅವಿ-ತು-ಕೊಂಡ
ಅವಿ-ತು-ಕೊಂಡಿ
ಅವಿದ್ಯೆ
ಅವಿ-ದ್ಯೆಯ
ಅವಿ-ನಾಶಿ
ಅವಿ-ವೇಕ
ಅವಿ-ವೇ-ಕ-ಕ್ಕಾಗಿ
ಅವಿ-ವೇ-ಕದ
ಅವಿ-ವೇ-ಕ-ದಿಂದ
ಅವಿ-ವೇ-ಕ-ವನ್ನು
ಅವಿ-ವೇ-ಕವೂ
ಅವಿ-ವೇ-ಕವೆ
ಅವಿ-ವೇ-ಕವೇ
ಅವಿ-ವೇಕಿ
ಅವಿ-ವೇ-ಕಿ-ಯಾ-ಗ-ಲಿಲ್ಲ
ಅವಿ-ವೇ-ಕಿ-ಯಾದ
ಅವು
ಅವುಗ
ಅವು-ಗಳ
ಅವು-ಗ-ಳಂತೆ
ಅವು-ಗ-ಳತ್ತ
ಅವು-ಗಳನ್ನು
ಅವು-ಗಳನ್ನೆಲ್ಲ
ಅವು-ಗ-ಳನ್ನೇ
ಅವು-ಗಳಲ್ಲಿ
ಅವು-ಗಳಿಂದ
ಅವು-ಗ-ಳಿಂ-ದಲೆ
ಅವು-ಗ-ಳಿ-ಗಾಗಿ
ಅವು-ಗ-ಳಿಗೆ
ಅವು-ಗ-ಳಿದ್ದ
ಅವು-ಗಳು
ಅವು-ಚಿ-ಕೊಂ-ಡಿದ್ದ
ಅವೂ
ಅವೆ-ರ-ಡನ್ನೂ
ಅವೆಲ್ಲ
ಅವೆ-ಲ್ಲ-ವನ್ನೂ
ಅವೆ-ಲ್ಲವೂ
ಅವೇಕೆ
ಅವೇ-ನಿ-ದ್ದರೂ
ಅವೊಂ
ಅವ್ಯಕ್ತ
ಅವ್ಯ-ಕ್ತ-ಗ-ಳಿಗೆ
ಅವ್ಯ-ಕ್ತ-ಜ-ಗತ್ತು
ಅವ್ಯ-ಕ್ತ-ದಿಂದ
ಅವ್ಯ-ಕ್ತ-ಮ-ಧು-ರ-ವಾದ
ಅವ್ಯ-ಕ್ತ-ವನ್ನು
ಅವ್ಯ-ಕ್ತ-ವೆಂದೂ
ಅವ್ಯಾ-ಜ-ಪ್ರೇಮ
ಅಶ-ಕ್ತ-ರಾ-ಗಿ-ರು-ವರೋ
ಅಶ-ಕ್ಯ-ವಾ-ದು-ದ-ರಿಂದ
ಅಶನಾ
ಅಶ-ರೀ-ರ-ವಾಣಿ
ಅಶಾಂ-ತ-ನಾ-ಗದೆ
ಅಶಾಂ-ತ-ನಾಗಿ
ಅಶಾಂ-ತ-ಮ-ನ-ಸ್ಸಿ-ನಿಂ-ದಿದ್ದ
ಅಶಾಂತಿ
ಅಶಾ-ಶ್ವತ
ಅಶಾ-ಶ್ವ-ತ-ಗ-ಳಾದ
ಅಶಾ-ಶ್ವ-ತ-ವಾದ
ಅಶಾ-ಶ್ವ-ತ-ವೆಂದು
ಅಶು-ದ್ಧ-ವಾ-ಗಿ-ದೆಯೊ
ಅಶು-ಭ-ಗ್ರ-ಹ-ವಾ-ದರೂ
ಅಶ್ಮ
ಅಶ್ಮಕ
ಅಶ್ಮ-ಕನು
ಅಶ್ಮ-ಕ-ನೆಂದು
ಅಶ್ಲೀ-ಲ-ವಾಗಿ
ಅಶ್ವ
ಅಶ್ವ-ತ್ಥಾಮ
ಅಶ್ವ-ತ್ಥಾ-ಮನ
ಅಶ್ವ-ತ್ಥಾ-ಮ-ನನ್ನು
ಅಶ್ವ-ತ್ಥಾ-ಮನು
ಅಶ್ವ-ಮೇಧ
ಅಶ್ವ-ಮೇ-ಧ-ಯಾಗ
ಅಶ್ವ-ಮೇ-ಧ-ಯಾ-ಗ-ಗಳನ್ನು
ಅಶ್ವ-ಮೇ-ಧ-ಯಾ-ಗ-ವನ್ನು
ಅಶ್ವ-ಸೇನ
ಅಶ್ವಿನೀ
ಅಶ್ವಿ-ನೀ-ದೇ-ವ-ತೆ-ಗಳ
ಅಶ್ವಿ-ನೀ-ದೇ-ವ-ತೆ-ಗಳು
ಅಷ್ಟಕ
ಅಷ್ಟಕ್ಕೆ
ಅಷ್ಟಕ್ಕೇ
ಅಷ್ಟ-ದಿ-ಕ್ಪಾ-ಲ-ಕ-ರಿಗೆ
ಅಷ್ಟ-ಮ-ಹಿ-ಷಿ-ಯ-ರನ್ನು
ಅಷ್ಟ-ಮ-ಹಿ-ಷಿ-ಯ-ರಲ್ಲಿ
ಅಷ್ಟ-ಮ-ಹಿ-ಷಿ-ಯರು
ಅಷ್ಟರ
ಅಷ್ಟ-ರಲ್ಲಿ
ಅಷ್ಟ-ರ-ಲ್ಲಿಯೆ
ಅಷ್ಟ-ರ-ಲ್ಲಿಯೇ
ಅಷ್ಟ-ರಲ್ಲೆ
ಅಷ್ಟ-ರೊಳ
ಅಷ್ಟ-ರೊ-ಳಗೆ
ಅಷ್ಟ-ವ-ಸು-ಗಳಲ್ಲಿ
ಅಷ್ಟ-ವ-ಸು-ಗಳು
ಅಷ್ಟ-ಸಿ-ದ್ಧಿ-ಗಳು
ಅಷ್ಟ-ಸಿ-ದ್ಧಿ-ಗ-ಳೊ-ಡನೆ
ಅಷ್ಟಾಂ-ಗ-ಗಳಲ್ಲಿ
ಅಷ್ಟಾಂ-ಗ-ಯೋ-ಗ-ವೆಂದು
ಅಷ್ಟಾ-ಕ್ಷರ
ಅಷ್ಟಾ-ಗು-ತ್ತಲೆ
ಅಷ್ಟಾ-ದರೆ
ಅಷ್ಟಿ-ಷ್ಟಲ್ಲ
ಅಷ್ಟು
ಅಷ್ಟೆ
ಅಷ್ಟೇ
ಅಷ್ಟೇಕೆ
ಅಸಂ-ಗ-ವ-ಮಾ-ತ್ತ-ವೇ-ಣುಃ
ಅಸ-ಡ್ಡೆ-ಯಿಂದ
ಅಸ-ತ್ಯ-ಎಂದು
ಅಸ-ದೃಶ
ಅಸ-ಮಂಜ
ಅಸ-ಮಾ-ಧಾ-ನ-ವಾ-ಗಿತ್ತು
ಅಸ-ರರು
ಅಸ-ಲಿಗೆ
ಅಸಹ್ಯ
ಅಸ-ಹ್ಯ-ಗೊಂಡ
ಅಸ-ಹ್ಯ-ಗೊಂಡು
ಅಸ-ಹ್ಯ-ವಾ-ಗಿದೆ
ಅಸ-ಹ್ಯ-ವಾ-ಯಿತು
ಅಸ-ಹ್ಯವೂ
ಅಸಾ-ಧಾ-ರಣ
ಅಸಾಧ್ಯ
ಅಸಾ-ಧ್ಯ-ರಾದ
ಅಸಾ-ಧ್ಯ-ವಾ-ಗಿತ್ತು
ಅಸಾ-ಧ್ಯ-ವಾದ
ಅಸಾ-ಧ್ಯ-ವಾ-ದಂ-ತಹ
ಅಸಾ-ಧ್ಯ-ವಾ-ದುದೂ
ಅಸಾ-ಧ್ಯ-ವಾ-ಯಿತು
ಅಸಾ-ಧ್ಯ-ವೆ-ನಿ-ಸಿ-ದಾಗ
ಅಸಾ-ಧ್ಯವೋ
ಅಸಿಕ್ನಿ
ಅಸಿ-ಕ್ನಿ-ಯನ್ನು
ಅಸಿ-ಕ್ನಿ-ಯಲ್ಲಿ
ಅಸಿತ
ಅಸುರ
ಅಸು-ರನ
ಅಸು-ರ-ರಿಗೆ
ಅಸು-ರರು
ಅಸೂ-ಯಾ-ರ-ಹಿತ
ಅಸೂಯೆ
ಅಸೂ-ಯೆ-ಪ-ಡರು
ಅಸೂ-ಯೆ-ಯಿಂದ
ಅಸ್ತ-ಗಿ-ರಿಗೆ
ಅಸ್ತಿ
ಅಸ್ತಿ-ಭಾರ
ಅಸ್ತು
ಅಸ್ತೇಯ
ಅಸ್ತ್ರ
ಅಸ್ತ್ರಕ್ಕೆ
ಅಸ್ತ್ರ-ಗಳ
ಅಸ್ತ್ರವೂ
ಅಸ್ತ್ರ-ಶ-ಸ್ತ್ರ-ಗಳ
ಅಸ್ತ್ರಾಯ
ಅಸ್ತ್ವಿ-ತಿ-ತನ್ನ
ಅಸ್ಥಿ-ಪಂ-ಜ-ರ-ದಂ-ತಿ-ದ್ದಾನೆ
ಅಸ್ಥಿರ
ಅಸ್ಥಿ-ರ-ವಾ-ದುವು
ಅಸ್ಥಿ-ರ-ವೆ-ನ್ನು-ವು-ದಾ-ದರೆ
ಅಸ್ಯಾ
ಅಹಂ
ಅಹಂ-ಕಾರ
ಅಹಂ-ಕಾ-ರ-ದಿಂದ
ಅಹಂ-ಕಾ-ರ-ಮಾತ್ರ
ಅಹಂ-ಕಾ-ರ-ವನ್ನು
ಅಹಂ-ಕಾ-ರ-ವಿ-ದ್ದ-ರ-ಲ್ಲವೆ
ಅಹಂ-ಕಾ-ರವೂ
ಅಹಂ-ಕಾ-ರವೆ
ಅಹಂ-ಕಾ-ರ-ವೆಷ್ಟು
ಅಹಂ-ಕಾ-ರವೇ
ಅಹಂ-ಕಾರಿ
ಅಹಂ-ಕಾ-ರಿ-ಯಾ-ಗ-ದಂತೆ
ಅಹಹ
ಅಹಾ
ಅಹಾಂ-ಕಾ-ರ-ಪ-ಡು-ವುದು
ಅಹಾರ
ಅಹಿಂಸೆ
ಅಹಿತ
ಅಹು-ಕ-ನೆಂ-ಬು-ವ-ನಿಗೆ
ಅಹುದು
ಅಹೋ
ಅಹೋ-ರಾ-ತ್ರಿ-ಗಳು
ಆ
ಆಂಗೀ-ರ-ಸನ
ಆಂಜ-ನೇ-ಯನು
ಆಂಜ-ನೇ-ಯಾದಿ
ಆಂತ-ರಿ-ಕ-ವಾದ
ಆಂಧ್ರ
ಆಂಧ್ರನು
ಆಎಂದು
ಆಕರ
ಆಕ-ರ-ವಾ-ದು-ವೆಂದು
ಆಕ-ರ್ಷ-ಕ-ವಾ-ಗಿದೆ
ಆಕ-ರ್ಷಿ-ತ-ರಾ-ದರು
ಆಕ-ರ್ಷಿ-ತ-ರಾ-ದರೆ
ಆಕ-ರ್ಷಿ-ತ-ವಾ-ದವು
ಆಕಳ
ಆಕ-ಳನ್ನು
ಆಕ-ಳಿ-ಕೆ-ಯನ್ನು
ಆಕ-ಳಿಗೆ
ಆಕ-ಳಿನ
ಆಕ-ಳಿ-ನಂತೆ
ಆಕ-ಳಿ-ಸ-ಲೆಂದು
ಆಕಳು
ಆಕ-ಳು-ಗ-ಳೆಲ್ಲ
ಆಕ-ಳೊಂದು
ಆಕಾರ
ಆಕಾ-ರ-ಗಳೂ
ಆಕಾ-ರ-ದ-ಲ್ಲಿದ್ದ
ಆಕಾ-ರ-ಭೇ-ದ-ವಿಲ್ಲ
ಆಕಾ-ರ-ವನ್ನು
ಆಕಾಶ
ಆಕಾ-ಶ-ಎಂಬ
ಆಕಾ-ಶಕ್ಕೂ
ಆಕಾ-ಶಕ್ಕೆ
ಆಕಾ-ಶ-ಗ-ಮನ
ಆಕಾ-ಶ-ಗ-ಳಷ್ಟು
ಆಕಾ-ಶ-ಗಳು
ಆಕಾ-ಶ-ಗ-ಳೆ-ರಡೂ
ಆಕಾ-ಶದ
ಆಕಾ-ಶ-ದಂತೆ
ಆಕಾ-ಶ-ದತ್ತ
ಆಕಾ-ಶ-ದಲ್ಲಿ
ಆಕಾ-ಶ-ದ-ಲ್ಲಿ-ದ್ದರೆ
ಆಕಾ-ಶ-ದಿಂದ
ಆಕಾ-ಶ-ಮಾರ್ಗ
ಆಕಾ-ಶ-ಮಾ-ರ್ಗ-ದಲ್ಲಿ
ಆಕಾ-ಶ-ಮಾ-ರ್ಗ-ವಾಗಿ
ಆಕಾ-ಶ-ಮಾ-ರ್ಗ-ವಾ-ಗಿಯೆ
ಆಕಾ-ಶ-ವನ್ನು
ಆಕಾ-ಶ-ವ-ನ್ನೆಲ್ಲ
ಆಕಾ-ಶ-ವಾಣಿ
ಆಕಾ-ಶ-ವಾ-ಣಿಯ
ಆಕಾ-ಶ-ವಾ-ಣಿ-ಯೊಂದು
ಆಕಾ-ಶವು
ಆಕಾ-ಶಾದಿ
ಆಕೂತಿ
ಆಕೂ-ತಿಯು
ಆಕೂ-ತೀ-ನಾಂ
ಆಕೃತಿ
ಆಕೃ-ತಿಗೆ
ಆಕೃ-ತಿ-ಯನ್ನು
ಆಕೃ-ತಿ-ಯುಳ್ಳ
ಆಕೃ-ತಿಯೆ
ಆಕೆ
ಆಕೆಗೂ
ಆಕೆಗೆ
ಆಕೆ-ಗೆಲ್ಲಿ
ಆಕೆಯ
ಆಕೆ-ಯತ್ತ
ಆಕೆ-ಯನ್ನು
ಆಕೆ-ಯನ್ನೂ
ಆಕೆ-ಯಲ್ಲಿ
ಆಕೆ-ಯಿಂದ
ಆಕೆಯು
ಆಕೆಯೂ
ಆಕೆ-ಯೊ-ಡನೆ
ಆಕ್ರಂ-ದ-ನ-ವನ್ನು
ಆಕ್ರಮಿ
ಆಕ್ರ-ಮಿಸ
ಆಕ್ರ-ಮಿ-ಸಿ-ದನು
ಆಕ್ರ-ಮಿ-ಸು-ತ್ತಾನೆ
ಆಕ್ರ-ಮಿ-ಸು-ವಂತೆ
ಆಕ್ರ-ಮಿ-ಸು-ವನು
ಆಕ್ರೂ-ರನು
ಆಕ್ರೋಶ
ಆಕ್ಷೇ-ಪಿ-ಸಿ-ದಳು
ಆಗ
ಆಗ-ತಾನೆ
ಆಗದ
ಆಗ-ಬ-ಹುದು
ಆಗ-ಬೇಕು
ಆಗ-ಬೇಕೆ
ಆಗ-ಮ-ನ-ದಿಂದ
ಆಗ-ಮ-ನ-ವನ್ನು
ಆಗ-ಮಾ-ಡಿ-ಸುವ
ಆಗ-ಮಿ-ಸಿ-ದನು
ಆಗ-ಮಿ-ಸಿ-ದರು
ಆಗ-ಮಿ-ಸಿದ್ದ
ಆಗಲಿ
ಆಗ-ಲಿಲ್ಲ
ಆಗಲೂ
ಆಗಲೆ
ಆಗ-ಲೆಂ-ದಳು
ಆಗ-ಲೆಂದು
ಆಗಲೇ
ಆಗಾಗ
ಆಗಾ-ಗಲೆ
ಆಗಿ
ಆಗಿತ್ತು
ಆಗಿದೆ
ಆಗಿ-ದೆಯೊ
ಆಗಿದ್ದ
ಆಗಿ-ದ್ದವು
ಆಗಿ-ದ್ದಾನೆ
ಆಗಿರು
ಆಗಿ-ರುವ
ಆಗಿ-ರು-ವಾಗ
ಆಗಿಲ್ಲ
ಆಗಿವೆ
ಆಗಿ-ಹೋ-ಗು-ತ್ತದೆ
ಆಗಿ-ಹೋ-ಯಿತು
ಆಗು-ತ್ತದೆ
ಆಗು-ತ್ತಾನೆ
ಆಗು-ತ್ತಾ-ರಂತೆ
ಆಗು-ತ್ತಾರೆ
ಆಗು-ತ್ತೇನೆ
ಆಗುವ
ಆಗು-ವಷ್ಟು
ಆಗು-ವು-ದಿಲ್ಲ
ಆಗು-ವುದು
ಆಗು-ವು-ದೆಂ-ದರೆ
ಆಗೋ-ಚ-ರನೂ
ಆಗ್ನೇ-ಯಾಸ್ತ್ರ
ಆಗ್ರಹ
ಆಗ್ರ-ಹವೆ
ಆಚ-ಮನ
ಆಚ-ರ-ಣೆ-ಯಿಂದ
ಆಚ-ರಿ-ಸ-ಬೇ-ಕಾ-ಗಿದೆ
ಆಚ-ರಿ-ಸ-ಬೇಕು
ಆಚ-ರಿ-ಸ-ಬೇ-ಕೆಂದು
ಆಚ-ರಿ-ಸಲು
ಆಚ-ರಿಸಿ
ಆಚ-ರಿ-ಸಿ-ದನು
ಆಚ-ರಿ-ಸಿ-ದರು
ಆಚ-ರಿ-ಸಿ-ದರೆ
ಆಚ-ರಿ-ಸಿದು
ಆಚ-ರಿಸು
ಆಚ-ರಿ-ಸುತ್ತಾ
ಆಚ-ರಿ-ಸು-ತ್ತಾರೆ
ಆಚ-ರಿ-ಸು-ತ್ತಿ-ದ್ದನು
ಆಚ-ರಿ-ಸು-ತ್ತಿ-ರುವ
ಆಚ-ರಿ-ಸುವ
ಆಚ-ರಿ-ಸು-ವನು
ಆಚ-ರಿ-ಸು-ವ-ವರು
ಆಚ-ರಿ-ಸುವು
ಆಚ-ರಿ-ಸು-ವು-ದ-ಕ್ಕಾಗಿ
ಆಚ-ರಿ-ಸು-ವುದು
ಆಚಾರ
ಆಚಾ-ರ-ಶೀ-ಲ-ನಾ-ಗಿದ್ದ
ಆಚಾ-ರ್ಯ-ಪು-ರುಷ
ಆಚೆ
ಆಚೆಗೆ
ಆಜನ್ಮ
ಆಜಾ-ನು-ಬಾ-ಹು-ವಾದ
ಆಜ್ಞ-ಪ್ತ-ರಾದ
ಆಜ್ಞಾ-ಚ-ಕ್ರಕ್ಕೆ
ಆಜ್ಞಾ-ಚ-ಕ್ರ-ದ-ಲ್ಲಿಯೇ
ಆಜ್ಞಾ-ಪಿ-ಸಿದ
ಆಜ್ಞಾ-ಪಿ-ಸಿ-ದ್ದಾನೆ
ಆಜ್ಞೆ-ಯಂತೆ
ಆಜ್ಞೆ-ಯನ್ನು
ಆಟ
ಆಟಕ್ಕೆ
ಆಟ-ಗಳಲ್ಲಿ
ಆಟ-ಗಳಿಂದ
ಆಟ-ಗು-ಳಿಯ
ಆಟದ
ಆಟ-ದಲ್ಲಿ
ಆಟ-ದ-ಲ್ಲಿಯೂ
ಆಟ-ಪಾ-ಟ-ಗಳನ್ನೆಲ್ಲ
ಆಟ-ವನ್ನು
ಆಟ-ವಾ-ಡ-ಬೇ-ಕೆಂದು
ಆಟ-ವಾ-ಡಲು
ಆಟ-ವಾ-ಡಿ-ಕೊಂ-ಡಿ-ರ-ಬೇ-ಕಾದ
ಆಟ-ವಾ-ಡಿ-ಕೊಂ-ಡಿ-ರ-ಬೇಕು
ಆಟ-ವಾ-ಡಿ-ಕೊಂ-ಡಿ-ರೋ-ಣ-ವೆಂ-ದು-ಕೊಂಡು
ಆಟ-ವಾ-ಡಿ-ಸು-ತ್ತಿ-ದ್ದಾಳೆ
ಆಟ-ವಾಡು
ಆಟ-ವಾ-ಡುತ್ತಾ
ಆಟ-ವಾ-ಡು-ತ್ತಿದ್ದ
ಆಟ-ವಾ-ಡು-ತ್ತಿ-ದ್ದ-ವರು
ಆಟ-ವಾ-ಡು-ತ್ತಿ-ದ್ದಾಗ
ಆಟ-ವಾ-ಡು-ತ್ತಿ-ರು-ವಾಗ
ಆಟ-ವಾ-ಡುವ
ಆಟ-ವಾ-ಡು-ವ-ವನೂ
ಆಟ-ವಾ-ಡು-ವಾಗ
ಆಟ-ವಾ-ಡು-ವಿ-ಯಂತೆ
ಆಟ-ವಾ-ಡು-ವು-ದ-ಕ್ಕಾಗಿ
ಆಟ-ವಾ-ಡು-ವು-ದೆಂ-ದರೆ
ಆಟವೂ
ಆಟ-ವೆಂದರೆ
ಆಟ-ವೇನೂ
ಆಟ-ಹೀಗೆ
ಆಡಂ-ಬರ
ಆಡ-ದಂತೆ
ಆಡ-ವಿಗೆ
ಆಡಿ
ಆಡಿಕೊ
ಆಡಿ-ಕೊಂಡು
ಆಡಿದ
ಆಡಿನ
ಆಡಿ-ಸಿದ
ಆಡಿ-ಸಿ-ದಂತೆ
ಆಡಿ-ಸುತ್ತಾ
ಆಡುತ್ತಾ
ಆಡು-ತ್ತಾನೆ
ಆಡು-ತ್ತಿದೆ
ಆಡು-ತ್ತಿದ್ದ
ಆಡು-ತ್ತಿ-ದ್ದರು
ಆಡು-ತ್ತಿರು
ಆಡು-ತ್ತಿ-ರುವ
ಆಡು-ತ್ತಿ-ರು-ವು-ದಲ್ಲ
ಆಡು-ತ್ತೀ-ಯೆಂ-ಬುದು
ಆಡುವ
ಆಡು-ವ-ಷ್ಟ-ರಲ್ಲಿ
ಆಡು-ವು-ದಕ್ಕೆ
ಆಡು-ವುದು
ಆಡೋಣ
ಆಣೆ
ಆತ
ಆತಂಕ
ಆತಂ-ಕ-ಗೊಂ-ಡಿದ್ದ
ಆತಂ-ಕ-ಗೊ-ಳ್ಳು-ತ್ತಿ-ದ್ದೆ-ವಲ್ಲಾ
ಆತಂ-ಕವೂ
ಆತನ
ಆತ-ನಂತೆ
ಆತ-ನಂ-ತೆಯೇ
ಆತ-ನದು
ಆತ-ನನ್ನು
ಆತ-ನನ್ನೂ
ಆತ-ನನ್ನೆ
ಆತ-ನನ್ನೇ
ಆತ-ನ-ಮೇಲೆ
ಆತ-ನಲ್ಲಿ
ಆತ-ನ-ಲ್ಲಿಗೆ
ಆತ-ನ-ಲ್ಲಿದ್ದ
ಆತ-ನ-ಲ್ಲಿಯೇ
ಆತ-ನಾ-ಗಲೆ
ಆತ-ನಿಂದ
ಆತ-ನಿಂ-ದಲೆ
ಆತ-ನಿಂ-ದಲೇ
ಆತ-ನಿ-ಗಾಗಿ
ಆತ-ನಿ-ಗಾ-ಗಿಯೇ
ಆತ-ನಿ-ಗಿಂತ
ಆತ-ನಿ-ಗಿ-ತ್ತನು
ಆತ-ನಿ-ಗಿ-ರ-ಲಿಲ್ಲ
ಆತ-ನಿ-ಗಿ-ರು-ವು-ದಿಲ್ಲ
ಆತ-ನಿಗೂ
ಆತ-ನಿಗೆ
ಆತ-ನಿಗೇ
ಆತ-ನಿ-ಗೊಂದು
ಆತ-ನಿ-ಗೊ-ಪ್ಪಿಸಿ
ಆತ-ನಿ-ಗೊ-ಪ್ಪಿ-ಸಿ-ದನು
ಆತ-ನಿ-ರುವ
ಆತ-ನೀಗ
ಆತನು
ಆತ-ನುಈ
ಆತನೂ
ಆತನೆ
ಆತ-ನೆ-ಲ್ಲಿ-ರು-ವ-ನೆಂ-ಬು-ದನ್ನು
ಆತನೇ
ಆತ-ನೇನು
ಆತ-ನೇನೂ
ಆತ-ನೊ-ಡನೆ
ಆತ-ನೊಬ್ಬ
ಆತ-ನೊ-ಬ್ಬನೆ
ಆತ-ನೊ-ಬ್ಬನೇ
ಆತ-ನೊಮ್ಮೆ
ಆತಿ-ಥ್ಯ-ವನ್ನು
ಆತುರ
ಆತು-ರ-ದಲ್ಲಿ
ಆತು-ರ-ದಿಂದ
ಆತು-ರ-ಪ-ಡುವ
ಆತ್ಮ
ಆತ್ಮ-ಕಥೆ
ಆತ್ಮ-ಕ್ಕಲ್ಲ
ಆತ್ಮಕ್ಕೂ
ಆತ್ಮಕ್ಕೆ
ಆತ್ಮ-ಗಳು
ಆತ್ಮ-ಜ್ಞಾನ
ಆತ್ಮ-ಜ್ಞಾ-ನ-ದಲ್ಲಿ
ಆತ್ಮ-ಜ್ಞಾ-ನ-ವ-ನ್ನಲ್ಲ
ಆತ್ಮ-ಜ್ಞಾ-ನ-ವನ್ನು
ಆತ್ಮ-ಜ್ಞಾ-ನಿ-ಗ-ಳಿಗೆ
ಆತ್ಮ-ಜ್ಞಾ-ನಿಯ
ಆತ್ಮ-ಜ್ಞಾ-ನಿ-ಯಾ-ಗಿದ್ದ
ಆತ್ಮ-ತ-ತ್ವ-ವನ್ನು
ಆತ್ಮ-ತ-ತ್ವ-ವೆಲ್ಲ
ಆತ್ಮ-ದೃಷ್ಟಿ
ಆತ್ಮ-ದೇ-ಹ-ನಾ-ಶಕ್ಕೆ
ಆತ್ಮ-ಧ-ರ್ಮ-ವಾದ
ಆತ್ಮ-ಧ್ಯಾನ
ಆತ್ಮನ
ಆತ್ಮ-ನನ್ನು
ಆತ್ಮ-ನಲ್ಲಿ
ಆತ್ಮ-ನಾ-ಗಿ-ರುವ
ಆತ್ಮನಿ
ಆತ್ಮ-ನಿಂದ
ಆತ್ಮ-ನಿ-ಗಲ್ಲ
ಆತ್ಮ-ನಿಗೆ
ಆತ್ಮನು
ಆತ್ಮನೂ
ಆತ್ಮ-ರೂ-ಪ-ನಾಗಿ
ಆತ್ಮ-ರೂ-ಪಿ-ನಿಂದ
ಆತ್ಮ-ಲಾಭ
ಆತ್ಮ-ವಂ-ಚ-ಕ-ನಿ-ದ್ದಂತೆ
ಆತ್ಮ-ವನ್ನು
ಆತ್ಮ-ವನ್ನೇ
ಆತ್ಮ-ವ-ಲ್ಲ-ವೆಂಬ
ಆತ್ಮ-ವಿ-ದ್ಯೆ-ಯನ್ನು
ಆತ್ಮವು
ಆತ್ಮ-ವೆಂದು
ಆತ್ಮ-ವೆಂ-ದು-ಕೊ-ಳ್ಳು-ವರು
ಆತ್ಮ-ವೆಂಬ
ಆತ್ಮವೇ
ಆತ್ಮ-ವೊಂದೇ
ಆತ್ಮ-ಸಂ-ಭ-ವ-ನಾದ
ಆತ್ಮ-ಸಾ-ಕ್ಷಾ-ತ್ಕಾ-ರ-ವಾ-ಗು-ತ್ತದೆ
ಆತ್ಮ-ಸು-ಖ-ವನ್ನು
ಆತ್ಮ-ಸ್ವ-ರೂಪ
ಆತ್ಮ-ಸ್ವ-ರೂ-ಪದ
ಆತ್ಮ-ಸ್ವ-ರೂ-ಪ-ವನ್ನು
ಆತ್ಮ-ಸ್ವ-ರೂ-ಪ-ವಾ-ಗಿಯೂ
ಆತ್ಮ-ಸ್ವ-ರೂ-ಪ-ವೆಂದು
ಆತ್ಮ-ಸ್ವ-ರೂಪಿ
ಆತ್ಮ-ಸ್ವ-ರೂ-ಪಿ-ಯಾ-ಗಿ-ರು-ವನು
ಆತ್ಮ-ಹ-ತ್ಯೆ-ಮಾ-ಡಿ-ಕೊ-ಳ್ಳು-ತ್ತೇನೆ
ಆತ್ಮಾ
ಆತ್ಮಾನಂ
ಆತ್ಮಾ-ನಂದ
ಆತ್ಮಾ-ನಂ-ದ-ದಲ್ಲಿ
ಆತ್ಮಾ-ನಂ-ದ-ದಿಂದ
ಆತ್ಮಾ-ನಂ-ದ-ವನ್ನು
ಆತ್ಮಾ-ನಂ-ದಾ-ನು-ಭ-ವ-ದಿಂ-ದಲೆ
ಆತ್ಮಾ-ನಂ-ದಾ-ನು-ಭೂ-ತ್ಯೈವ
ಆತ್ಮಾನು
ಆತ್ಮಾ-ನು-ರ-ಕ್ತಿ-ಯಾ-ದಾಗ
ಆತ್ಮಾ-ರಾ-ಧಿ-ಪ-ತಯೇ
ಆತ್ಮಾ-ರಾಮ
ಆತ್ಮಾ-ರಾ-ಮ-ನಾ-ಗಿ-ರು-ವನು
ಆತ್ಮಾ-ರಾ-ಮ-ನಾದ
ಆತ್ಮಾ-ರಾ-ಮನು
ಆತ್ಮಾ-ರಾ-ಮನೂ
ಆತ್ಮಾ-ರಾ-ಮಾಯ
ಆತ್ಮೀ-ಯ-ನಾಗು
ಆತ್ಮೀ-ಯ-ರಾದ
ಆತ್ಮೀ-ಯ-ವಾದ
ಆತ್ಯಂ-ತಿಕ
ಆದ
ಆದ-ಅರೆ
ಆದರ
ಆದ-ರ-ದಿಂದ
ಆದ-ರ-ವು-ಳ್ಳ-ವನು
ಆದ-ರಾ-ತಿಥ್ಯ
ಆದ-ರಿಂದ
ಆದ-ರಿಂ-ದಲೆ
ಆದ-ರಿ-ಸ-ಲಿಲ್ಲ
ಆದ-ರಿ-ಸಲು
ಆದ-ರಿಸಿ
ಆದ-ರಿ-ಸಿ-ದನು
ಆದ-ರಿ-ಸಿ-ದೆವು
ಆದ-ರಿ-ಸು-ತ್ತಿ-ದ್ದು-ದ-ರಿಂದ
ಆದ-ರಿ-ಸು-ವು-ದಿಲ್ಲ
ಆದರೂ
ಆದರೆ
ಆದ-ರೆ-ಆ-ತ-ನೇನೂ
ಆದ-ರೇನು
ಆದ-ರೋ-ಪ-ಚಾರ
ಆದ-ರೋ-ಪ-ಚಾ-ರ-ಗಳನ್ನು
ಆದ-ರೋ-ಪ-ಚಾ-ರ-ಗಳು
ಆದರ್ಶ
ಆದ-ರ್ಶದ
ಆದ-ರ್ಶ-ಮೂ-ರ್ತಿ-ಯಾಗಿ
ಆದ-ರ್ಶ-ವನ್ನು
ಆದ-ರ್ಶ-ವಾ-ಗಿಟ್ಟು
ಆದಷ್ಟು
ಆದಾದ
ಆದಿ
ಆದಿ-ತ್ಯರು
ಆದಿ-ಪು-ರುಷ
ಆದಿ-ಪು-ರು-ಷ-ನಾದ
ಆದಿ-ಪು-ರು-ಷ-ನಿಗೆ
ಆದಿ-ಪು-ರು-ಷ-ರೆಂದು
ಆದಿ-ಮ-ಧ್ಯಾಂ-ತ-ರ-ಹಿ-ತನೂ
ಆದಿ-ರಾ-ಜ-ನೆಂಬ
ಆದಿ-ಶೇಷ
ಆದಿ-ಶೇ-ಷನ
ಆದಿ-ಶೇ-ಷನು
ಆದಿ-ಶೇ-ಷನೇ
ಆದುದ
ಆದುದು
ಆದೇ-ಶ-ದಂತೆ
ಆದ್ದ
ಆದ್ದ-ರಿಂದ
ಆದ್ದ-ರಿಂ-ದಲೆ
ಆದ್ದ-ರಿಂ-ದಲೇ
ಆದ್ಯ
ಆದ್ಯಂ-ತ-ಗ-ಳಿ-ಲ್ಲದ
ಆದ್ಯಂ-ತ-ರ-ಹಿತ
ಆದ್ಯಂ-ತ-ರ-ಹಿ-ತ-ನಾದ
ಆದ್ಯಂ-ತ-ವಾಗಿ
ಆದ್ಯನೂ
ಆಧಾರ
ಆಧಾ-ರ-ದಿಂ-ದಲೊ
ಆಧಾ-ರ-ವಾದ
ಆಧಾ-ರ-ವಾ-ದು-ದ-ರಿಂದ
ಆಧಾ-ರ-ವಿ-ಲ್ಲದೆ
ಆಧಾ-ರಿ-ಎಂದು
ಆಧಿ-ದೈ-ವಿ-ಕ-ನೆಂದೂ
ಆಧಿ-ಭೌ-ತಿ-ಕ-ನೆಂದೂ
ಆಧು-ನಿ-ಕರು
ಆಧ್ಯಾ-ತ್ಮಿಕ
ಆಧ್ಯಾ-ತ್ಮಿ-ಕ-ದೃಷ್ಟಿ
ಆನಂದ
ಆನಂ-ದಕ್ಕೆ
ಆನಂ-ದ-ಗೊಂಡ
ಆನಂ-ದ-ಘ-ನ-ವಾದ
ಆನಂ-ದದ
ಆನಂ-ದ-ದಿಂದ
ಆನಂ-ದ-ಪ-ಟ್ಟರು
ಆನಂ-ದ-ಪ-ಡಿ-ಸಿ-ದರು
ಆನಂ-ದ-ಪ-ಡಿ-ಸಿ-ದಳು
ಆನಂ-ದ-ಪ-ಡಿ-ಸು-ತ್ತಿ-ದ್ದಳು
ಆನಂ-ದ-ಪ-ಡು-ತ್ತಾನೆ
ಆನಂ-ದ-ಪ-ಡು-ತ್ತಿವೆ
ಆನಂ-ದ-ಪ-ಡು-ವರು
ಆನಂ-ದ-ಪ-ರ-ವಶ
ಆನಂ-ದ-ಪ-ರ-ವ-ಶ-ನಾ-ದನು
ಆನಂ-ದ-ಪ-ರ-ವ-ಶ-ರಾ-ದರು
ಆನಂ-ದ-ಬಾಷ್ಪ
ಆನಂ-ದ-ಬಾ-ಷ್ಪ-ಗಳನ್ನು
ಆನಂ-ದ-ಬಾ-ಷ್ಪ-ಗಳಿಂದ
ಆನಂ-ದ-ಬಾ-ಷ್ಪ-ಗಳು
ಆನಂ-ದ-ಬಾ-ಷ್ಪ-ವನ್ನು
ಆನಂ-ದ-ಮ-ಗ್ನ-ನಾ-ಗಿರು
ಆನಂ-ದ-ಮಯ
ಆನಂ-ದ-ಮ-ಯ-ನಾಗಿ
ಆನಂ-ದ-ಮ-ಯ-ನಾ-ದನು
ಆನಂ-ದ-ಮ-ಯ-ವಾದ
ಆನಂ-ದ-ಮ-ಯ-ವಾ-ಯಿತು
ಆನಂ-ದ-ರೂ-ಪಿ-ಯಾದ
ಆನಂ-ದ-ವನ್ನು
ಆನಂ-ದ-ವಾ-ಗಿ-ರುವ
ಆನಂ-ದ-ವಾ-ಗಿ-ರುವಿ
ಆನಂ-ದ-ವಾ-ಯಿತು
ಆನಂ-ದ-ಸ್ಥಿ-ತಿ-ಯ-ಲ್ಲಿಯೇ
ಆನಂ-ದಾಂ-ಶ-ಗಳೇ
ಆನಂ-ದಿ-ಸಲಿ
ಆನಂ-ದಿ-ಸಿದ
ಆನಂ-ದಿ-ಸಿ-ದರು
ಆನಂ-ದಿ-ಸು-ವನು
ಆನಂ-ದಿ-ಸು-ವಳು
ಆನು
ಆನೆ
ಆನೆ-ಗಳ
ಆನೆ-ಗಳನ್ನು
ಆನೆ-ಗಳನ್ನೆಲ್ಲ
ಆನೆ-ಗಳಿಂದ
ಆನೆ-ಗ-ಳಿಗೂ
ಆನೆ-ಗ-ಳು-ಅ-ವೆಲ್ಲ
ಆನೆ-ಗ-ಳೆಲ್ಲ
ಆನೆಗೂ
ಆನೆಗೆ
ಆನೆಯ
ಆನೆ-ಯಂತೆ
ಆನೆ-ಯನ್ನು
ಆನೆ-ಯಾಗಿ
ಆನೆಯು
ಆನೆಯೂ
ಆನೆ-ಯೊಂದು
ಆನೇಕ
ಆಪಃ
ಆಪ-ತ್ತು-ಗ-ಳ-ಲ್ಲಿಯೂ
ಆಪಾದ
ಆಪೋ-ಶ-ನ-ಕ್ಕಾ-ದವು
ಆಪೋ-ಶ-ನ-ವಾಗಿ
ಆಪ್ತ
ಆಪ್ತ-ನಂತೆ
ಆಪ್ತ-ರನ್ನು
ಆಪ್ತರು
ಆಭ-ರಣ
ಆಭ-ರ-ಣ-ಗಳನ್ನು
ಆಭ-ರ-ಣ-ಗ-ಳಿಗೇ
ಆಭ-ರ-ಣ-ಗಳು
ಆಭ-ರ-ಣ-ದಂ-ತಿದ್ದ
ಆಭಾಸ
ಆಭೀರ
ಆಮ
ಆಮೂ-ಲಾ-ಗ್ರ-ವಾಗಿ
ಆಮೆ
ಆಮೇಲೆ
ಆಯಸ್ಸು
ಆಯಾ
ಆಯಾ-ದೇ-ಹ-ಸಂ-ಬಂ-ಧಿ-ಗಳ
ಆಯಾಸ
ಆಯಾ-ಸ-ಗೊಂ-ಡಂತೆ
ಆಯಾ-ಸ-ಗೊ-ಳ್ಳ-ಲಿ-ಲ್ಲ-ವೆಂದು
ಆಯಾ-ಸ-ದಿಂದ
ಆಯಾ-ಸ-ಮಾಡಿ
ಆಯಾ-ಸ-ವಾಗಿ
ಆಯಾ-ಸ-ವಾ-ಗಿ-ದ್ದು-ದ-ರಿಂದ
ಆಯಿ-ತಲ್ಲ
ಆಯಿತು
ಆಯು
ಆಯುಧ
ಆಯು-ಧಕ್ಕೂ
ಆಯು-ಧ-ಗಳನ್ನು
ಆಯು-ಧ-ಗ-ಳ-ಲ್ಲೆಲ್ಲ
ಆಯು-ಧ-ಗ-ಳಿಗೂ
ಆಯು-ಧ-ಗ-ಳಿ-ಲ್ಲದೆ
ಆಯು-ಧ-ಗಳು
ಆಯು-ಧ-ಗ-ಳು-ಎ-ಲ್ಲವೂ
ಆಯು-ಧ-ಗಳೂ
ಆಯು-ಧ-ಗ-ಳೆಲ್ಲ
ಆಯು-ಧ-ಗಳೇ
ಆಯು-ಧ-ಗ-ಳೊ-ಡನೆ
ಆಯು-ಧ-ಬಿಲ್ಲು
ಆಯು-ಧ-ವನ್ನು
ಆಯು-ಧ-ವಾ-ಗು-ತ್ತದೆ
ಆಯು-ಧ-ವಾದ
ಆಯು-ಧವೂ
ಆಯು-ರ್ವೇ-ದಾ-ಚಾ-ರ್ಯ-ನಾದ
ಆಯು-ಷ್ಯ-ದಲ್ಲಿ
ಆಯು-ಷ್ಯ-ವಿದೆ
ಆಯು-ಸ್ಸನ್ನು
ಆಯುಸ್ಸು
ಆಯ್ದ
ಆಯ್ದು
ಆಯ್ದು-ಕೊಂಡ
ಆಯ್ದು-ಕೊಂಡು
ಆರ
ಆರಂ-ಭ-ದಲ್ಲಿ
ಆರಂ-ಭ-ವಾ-ದಾಗ
ಆರಂ-ಭ-ವಾ-ಯಿತು
ಆರಂಭಿ
ಆರಂ-ಭಿಸಿ
ಆರಂ-ಭಿ-ಸಿದ
ಆರಂ-ಭಿ-ಸಿ-ದನು
ಆರಂ-ಭಿ-ಸಿ-ದರು
ಆರಂ-ಭಿ-ಸಿದೆ
ಆರಂ-ಭಿ-ಸಿ-ರುವ
ಆರತಿ
ಆರ-ತಿ-ಯೆ-ತ್ತಿ-ದರು
ಆರದು
ಆರ-ನೆಯ
ಆರ-ನೆ-ಯ-ದಾಗಿ
ಆರ-ನೆ-ಯ-ದಾದ
ಆರನ್ನೂ
ಆರಾ
ಆರಾ-ಧ-ಕ-ರಾ-ಗಿ-ದ್ದಾರೆ
ಆರಾ-ಧ-ನಾ-ರೂ-ಪ-ವಾದ
ಆರಾ-ಧನೆ
ಆರಾ-ಧ-ನೆ-ಗಾಗಿ
ಆರಾ-ಧ-ನೆ-ಗಾ-ಗಿಯೇ
ಆರಾ-ಧ-ನೆಗೆ
ಆರಾ-ಧ-ನೆ-ಯನ್ನೂ
ಆರಾ-ಧ-ನೆ-ಯನ್ನೇ
ಆರಾ-ಧ-ನೆ-ಯಲ್ಲಿ
ಆರಾ-ಧ-ನೆ-ಯಿಂದ
ಆರಾಧಿ
ಆರಾ-ಧಿ-ಸ-ತೊ-ಡ-ಗಿದ
ಆರಾ-ಧಿ-ಸ-ಬೇಕು
ಆರಾ-ಧಿ-ಸ-ಬೇ-ಕೆಂ-ದು-ಕೊಂ-ಡಿ-ದ್ದೇನೆ
ಆರಾ-ಧಿ-ಸ-ಹೊ-ರ-ಟರು
ಆರಾ-ಧಿಸಿ
ಆರಾ-ಧಿ-ಸಿ-ದರು
ಆರಾ-ಧಿ-ಸಿ-ರುವೆ
ಆರಾ-ಧಿಸು
ಆರಾ-ಧಿ-ಸುತ್ತಾ
ಆರಾ-ಧಿ-ಸು-ತ್ತಾರೆ
ಆರಾ-ಧಿ-ಸು-ತ್ತಿ-ದ್ದನು
ಆರಾ-ಧಿ-ಸು-ತ್ತಿ-ದ್ದಳು
ಆರಾ-ಧಿ-ಸು-ತ್ತಿ-ರುವ
ಆರಾ-ಧಿ-ಸು-ತ್ತಿ-ರು-ವಾಗ
ಆರಾ-ಧಿ-ಸುವ
ಆರಾ-ಧಿ-ಸು-ವನು
ಆರಾ-ಧಿ-ಸು-ವ-ವರ
ಆರಾ-ಧಿ-ಸುವು
ಆರಾ-ಧಿ-ಸು-ವುದು
ಆರಾಧ್ಯ
ಆರಾ-ಧ್ಯ-ದೈವ
ಆರಾ-ಧ್ಯ-ದೈ-ವದ
ಆರಾ-ಧ್ಯ-ದೈ-ವ-ವಾಗಿ
ಆರಾ-ಧ್ಯ-ದೈ-ವವು
ಆರಾ-ಧ್ಯ-ದೈ-ವ-ವೆಂದು
ಆರಿತು
ಆರಿಲ್ಲ
ಆರಿಸಿ
ಆರಿ-ಸು-ತ್ತಿ-ದ್ದನು
ಆರು
ಆರು-ಜನ
ಆರು-ಮಂದಿ
ಆರು-ವು-ದಕ್ಕೆ
ಆರೂ
ಆರೋ-ಗ್ಯ-ವಂ-ತ-ನಾದ
ಆರೋ-ಗ್ಯ-ವಾ-ಗಿ-ರು-ವರೇ
ಆರೋ-ಗ್ಯವೆ
ಆರೋಪ
ಆರೋ-ಪ-ಮಾ-ಡಿ-ರು-ವು-ದ-ನ್ನೆಲ್ಲ
ಆರೋಪಿ
ಆರೋ-ಪಿ-ಸಿ-ಕೊಂ-ಡಿ-ರುವೆ
ಆರೋ-ಪಿ-ಸಿ-ಕೊ-ಳ್ಳು-ತ್ತಾ-ನೆಯೇ
ಆರೋ-ಪಿ-ಸಿ-ಕೊ-ಳ್ಳು-ವುದ
ಆರೋ-ಪಿ-ಸಿ-ದನು
ಆರೋ-ಪಿಸು
ಆರ್ತ-ನಾದ
ಆರ್ತ-ನಾ-ದಕ್ಕೆ
ಆರ್ಭ-ಟಕ್ಕೆ
ಆರ್ಭ-ಟಿ-ಸು-ವರು
ಆರ್ಯ-ಪು-ತ್ರ-ನಾದ
ಆರ್ಯ-ಮ-ನೆಂ-ಬು-ವನು
ಆರ್ಯ-ಲ-ಕ್ಷ-ಣ-ಶೀ-ಲ-ವ್ರ-ತಾಯ
ಆರ್ಯಾ-ವ-ರ್ತದ
ಆರ್ಯಾ-ವ-ರ್ತ-ವನ್ನು
ಆಲಂ-ಗಿ-ಸ-ಹೋ-ದಳು
ಆಲಂ-ಗಿ-ಸಿ-ಕೊಂಡ
ಆಲಂ-ಗಿ-ಸಿ-ದರು
ಆಲಂ-ಗಿ-ಸು-ವೆ-ನೆಂಬ
ಆಲದ
ಆಲ-ದ-ಮರ
ಆಲ-ದ-ಮ-ರದ
ಆಲ-ದ-ಮ-ರವೂ
ಆಲಸ್ಯ
ಆಲ-ಸ್ಯ-ಗಳು
ಆಲ-ಸ್ಯ-ರೂ-ಪ-ವಾದ
ಆಲಿಂ-ಗನ
ಆಲಿಂ-ಗ-ನಕ್ಕೆ
ಆಲಿಂ-ಗ-ನದ
ಆಲಿಂ-ಗ-ನ-ದಿಂದ
ಆಲಿಂ-ಗಿಸಿ
ಆಲಿಂ-ಗಿ-ಸಿ-ಕೊಂ-ಡನು
ಆಲಿಂ-ಗಿ-ಸಿ-ಕೊಂ-ಡರು
ಆಲಿಂ-ಗಿ-ಸಿ-ಕೊಂ-ಡಾಗ
ಆಲಿಂ-ಗಿ-ಸಿ-ಕೊಂಡು
ಆಲಿಂ-ಗಿ-ಸಿ-ದ-ನಲ್ಲಾ
ಆಲಿಂ-ಗಿ-ಸಿ-ದಳು
ಆಲಿಂ-ಗಿ-ಸುತ್ತಿ
ಆಲಿಂ-ಗಿ-ಸು-ವು-ದಕ್ಕೆ
ಆಲಿ-ಕ-ಲ್ಲಿನ
ಆಲಿ-ಯ-ಧೋರ
ಆಲಿ-ಸುತ್ತಾ
ಆಲೆ-ದಾ-ಡು-ತ್ತಿ-ದ್ದರು
ಆಲೋ-ಚ-ನಾ-ಶ-ಕ್ತಿ-ಯನ್ನು
ಆಲೋ-ಚನೆ
ಆಲೋ-ಚ-ನೆಗೆ
ಆಲೋ-ಚಿ-ಸದೆ
ಆಲೋ-ಚಿ-ಸ-ಬಲ್ಲ
ಆಲೋ-ಚಿ-ಸ-ಬೇಕು
ಆಲೋ-ಚಿಸಿ
ಆಲೋ-ಚಿ-ಸಿದ
ಆಲೋ-ಚಿ-ಸಿ-ದನು
ಆಲೋ-ಚಿ-ಸಿ-ದ-ಮೇಲೆ
ಆಲೋ-ಚಿ-ಸಿ-ದರು
ಆಲೋ-ಚಿ-ಸಿ-ದಾಗ
ಆಲೋ-ಚಿ-ಸೋಣ
ಆಳ-ಕ್ಕಿ-ಳಿದು
ಆಳ-ದಲ್ಲಿ
ಆಳ-ವಾಗಿ
ಆಳ-ವಾದ
ಆಳಾ
ಆಳಾ-ದ-ರೇನು
ಆಳಾರು
ಆಳಿದ
ಆಳು
ಆಳು-ಕಾಳು-ಗ-ಳೆಲ್ಲ
ಆಳು-ಗ-ಳಿಂ-ದಲೆ
ಆಳು-ಗಳು
ಆಳು-ತ್ತಾರೆ
ಆಳು-ತ್ತಿದ್ದ
ಆಳು-ತ್ತಿ-ದ್ದರು
ಆಳು-ತ್ತಿದ್ದು
ಆಳ್ವಿ-ಕೆ-ಯಲ್ಲಿ
ಆವನು
ಆವ-ರ-ಣ-ಗಳನ್ನು
ಆವ-ರ-ಣ-ಗ-ಳೆಂ-ದರೆ
ಆವ-ರ-ಣ-ಗಳೇ
ಆವ-ರಿ-ಸಿ-ಕೊಂಡು
ಆವ-ರಿ-ಸಿ-ರು-ವು-ದ-ರಿಂದ
ಆವರು
ಆವ-ರೆಗೆ
ಆವರ್ತ
ಆವಿ-ರಾ-ವಿ-ರ್ಭವ
ಆವಿ-ರ್ಭ-ವಿ-ಸಿ-ದನು
ಆವಿ-ರ್ಹೋತ
ಆವುಗೆ
ಆಶಯ
ಆಶಾ-ಗೋ-ಪುರ
ಆಶಾ-ಭಂಗ
ಆಶಾ-ಭಂ-ಗ-ವಾದ
ಆಶಾ-ಭಂ-ಗ-ವಾ-ದರೆ
ಆಶಾ-ಭಂ-ಗ-ವಾ-ಯಿತು
ಆಶಾ-ಶಾಃ
ಆಶಿ-ಸು-ತ್ತೇವೆ
ಆಶೀರ್
ಆಶೀ-ರ್ವ-ದಿಸಿ
ಆಶೀ-ರ್ವ-ದಿ-ಸಿದ
ಆಶೀ-ರ್ವ-ದಿ-ಸಿ-ದನು
ಆಶೀ-ರ್ವ-ದಿ-ಸಿರಿ
ಆಶೀ-ರ್ವಾದ
ಆಶೀ-ರ್ವಾ-ದ-ದಿಂ-ದಲೂ
ಆಶೀ-ರ್ವಾ-ದ-ಮಾ-ಡಿ-ದರು
ಆಶೀ-ರ್ವಾ-ದ-ವನ್ನು
ಆಶೆ
ಆಶೆ-ಗಳನ್ನು
ಆಶೆಗೆ
ಆಶೆ-ಪ-ಟ್ಟರೆ
ಆಶೆ-ಪ-ಡದೆ
ಆಶೆ-ಯನ್ನು
ಆಶೆ-ಯಷ್ಟು
ಆಶೆ-ಯಾ-ಗಿದೆ
ಆಶೆ-ಯಾ-ಯಿತು
ಆಶೆ-ಯಿಂದ
ಆಶೆಯೂ
ಆಶೆಯೇ
ಆಶೆ-ಹು-ಟ್ಟು-ವುದು
ಆಶ್ಚರ್ಯ
ಆಶ್ಚ-ರ್ಯ-ಕ-ರ-ವಾ-ಗಿದೆ
ಆಶ್ಚ-ರ್ಯ-ಕ-ರ-ವಾ-ದುದು
ಆಶ್ಚ-ರ್ಯ-ಕ್ಕಿಂ-ತಲೂ
ಆಶ್ಚ-ರ್ಯ-ಗಳನ್ನೂ
ಆಶ್ಚ-ರ್ಯ-ಗ-ಳಿಗೂ
ಆಶ್ಚ-ರ್ಯ-ಗೊಂ-ಡರು
ಆಶ್ಚ-ರ್ಯ-ಗೊಂ-ಡ-ವ-ನಂತೆ
ಆಶ್ಚ-ರ್ಯ-ಚ-ಕಿ-ತ-ನ-ನ್ನಾಗಿ
ಆಶ್ಚ-ರ್ಯ-ದಿಂದ
ಆಶ್ಚ-ರ್ಯ-ಪ-ಡು-ತ್ತಿ-ದ್ದನು
ಆಶ್ಚ-ರ್ಯ-ಪ-ಡು-ತ್ತಿ-ದ್ದರು
ಆಶ್ಚ-ರ್ಯ-ವನ್ನು
ಆಶ್ಚ-ರ್ಯ-ವಾ-ಯಿತು
ಆಶ್ಚ-ರ್ಯ-ವಿದು
ಆಶ್ಚ-ರ್ಯ-ವೆಂದರೆ
ಆಶ್ಚ-ರ್ಯ-ವೇನೂ
ಆಶ್ಯ-ರ್ಯ-ದಿಂದ
ಆಶ್ರಮ
ಆಶ್ರ-ಮ-ಎಂಬ
ಆಶ್ರ-ಮಕ್ಕೆ
ಆಶ್ರ-ಮ-ಗಳ
ಆಶ್ರ-ಮ-ಗ-ಳ-ಲ್ಲೆಲ್ಲ
ಆಶ್ರ-ಮ-ಗ-ಳಿಗೂ
ಆಶ್ರ-ಮ-ಗ-ಳಿಗೆ
ಆಶ್ರ-ಮದ
ಆಶ್ರ-ಮ-ದಂತೆ
ಆಶ್ರ-ಮ-ದಲ್ಲಿ
ಆಶ್ರ-ಮ-ದ-ಲ್ಲಿದ್ದ
ಆಶ್ರ-ಮ-ದ-ಲ್ಲಿ-ದ್ದರೂ
ಆಶ್ರ-ಮ-ದ-ಲ್ಲಿಯೇ
ಆಶ್ರ-ಮ-ದ-ಲ್ಲಿ-ಲ್ಲದ
ಆಶ್ರ-ಮ-ದಲ್ಲೆಲ್ಲ
ಆಶ್ರ-ಮ-ದಿಂದ
ಆಶ್ರ-ಮ-ಧ-ರ್ಮ-ಗಳನ್ನು
ಆಶ್ರ-ಮ-ವನ್ನು
ಆಶ್ರ-ಮ-ವಾ-ಸಿ-ಗ-ಳೆ-ಲ್ಲರ
ಆಶ್ರ-ಮವೂ
ಆಶ್ರ-ಮ-ವೆಲ್ಲ
ಆಶ್ರಯ
ಆಶ್ರ-ಯ-ದ-ಲ್ಲಿ-ರು-ವು-ದ-ರಿಂದ
ಆಶ್ರ-ಯ-ದಾ-ತ-ನಾದ
ಆಶ್ರ-ಯ-ನಾ-ಗಿ-ರುವ
ಆಶ್ರ-ಯನು
ಆಶ್ರ-ಯ-ವನ್ನು
ಆಶ್ರ-ಯ-ವ-ಲ್ಲವೆ
ಆಶ್ರ-ಯ-ವಿ-ತ್ತಿ-ರುವ
ಆಶ್ರ-ಯಿ-ಸ-ಬೇಕು
ಆಶ್ರ-ಯಿ-ಸ-ಬೇ-ಕೆಂದು
ಆಶ್ರ-ಯಿಸಿ
ಆಶ್ರ-ಯಿ-ಸಿದ
ಆಶ್ರ-ಯಿ-ಸಿ-ದ-ವರು
ಆಶ್ರ-ಯಿ-ಸಿ-ರುವ
ಆಶ್ರ-ಯಿ-ಸಿಲ್ಲ
ಆಶ್ರ-ಯಿಸು
ಆಶ್ರ-ಯಿ-ಸು-ತ್ತಾರೆ
ಆಶ್ರ-ಯಿ-ಸು-ತ್ತಿ-ದ್ದೇನೆ
ಆಶ್ರ-ಯಿ-ಸು-ವಂತೆ
ಆಶ್ರ-ಯಿ-ಸು-ವ-ವ-ರಿಗೆ
ಆಶ್ರ-ಯಿ-ಸು-ವು-ದೆಂ-ದರೆ
ಆಸ-ಕ್ತ-ರಾಗಿ
ಆಸ-ಕ್ತ-ರಾ-ಗಿ-ರು-ವರು
ಆಸ-ಕ್ತ-ವಾ-ಯಿತು
ಆಸಕ್ತಿ
ಆಸ-ಕ್ತಿಗೆ
ಆಸ-ಕ್ತಿ-ಯನ್ನು
ಆಸ-ಕ್ತಿ-ಯಿಲ್ಲ
ಆಸ-ಕ್ತಿ-ಯು-ಳ್ಳ-ವ-ನಾಗ
ಆಸ-ಕ್ತಿ-ಯು-ಳ್ಳು-ದಾ-ಗಿ-ರು-ತ್ತದೆ
ಆಸ-ಕ್ತಿಯೇ
ಆಸ-ಕ್ತಿ-ಯೇನೂ
ಆಸನ
ಆಸ-ನ-ಗಳನ್ನು
ಆಸ-ನ-ಗಳು
ಆಸ-ನದ
ಆಸ-ನ-ದಲ್ಲಿ
ಆಸ-ನ-ವನ್ನು
ಆಸ-ನ-ವೆಂದರೆ
ಆಸಾಮಿ
ಆಸೆ
ಆಸೆ-ಗಳನ್ನೂ
ಆಸೆಗೆ
ಆಸೆ-ಪಟ್ಟ
ಆಸೆ-ಪ-ಟ್ಟ-ವ-ನಲ್ಲ
ಆಸೆ-ಪ-ಡು-ವಂ-ತೆಯೇ
ಆಸೆ-ಯಾ-ಗಿತ್ತು
ಆಸೆ-ಯಾ-ದರೆ
ಆಸೆ-ಯಿಂದ
ಆಸೆಯೂ
ಆಸ್ತಿ-ಕ-ತೆಯ
ಆಸ್ತಿ-ಗಳನ್ನು
ಆಸ್ತಿ-ಪಾ-ಸ್ತಿ-ಗಳು
ಆಸ್ತಿ-ಪಾ-ಸ್ತಿ-ಗ-ಳೇ-ನಿ-ದ್ದರೂ
ಆಸ್ತಿ-ಯಲ್ಲ
ಆಸ್ತಿ-ಯೆಂ-ದರೆ
ಆಸ್ತಿ-ಯೆಲ್ಲ
ಆಸ್ಥಾನ
ಆಸ್ಥಾ-ನ-ವನ್ನು
ಆಸ್ಪ-ದ-ವಿ-ತ್ತಿದೆ
ಆಸ್ಪ-ದ-ವಿ-ಲ್ಲ-ದಂತೆ
ಆಸ್ವಾ-ದ-ಕ-ನಾ-ಗು-ತ್ತಾನೆ
ಆಸ್ವಾ-ದಿ-ಸ-ಲೋ-ಸುಗ
ಆಹಾ
ಆಹಾ-ಇದು
ಆಹಾರ
ಆಹಾ-ರ-ಕ್ಕೆಂದು
ಆಹಾ-ರ-ದಲ್ಲಿ
ಆಹಾ-ರ-ದಿಂದ
ಆಹಾ-ರ-ವನ್ನು
ಆಹಾ-ರ-ವನ್ನೂ
ಆಹಾ-ರ-ವ-ನ್ನೆಲ್ಲ
ಆಹಾ-ರ-ವಾಗಿ
ಆಹಾ-ರ-ವಾ-ವುದು
ಆಹಾ-ರ-ವಿ-ಲ್ಲ-ದೆಯೇ
ಆಹಾ-ರಾ-ದಿ-ಗ-ಳ-ಲ್ಲಿಯೇ
ಆಹುತಿ
ಆಹು-ತಿ-ಕೊ-ಡು-ತ್ತೇನೆ
ಆಹು-ತಿ-ಗೊಟ್ಟು
ಆಹು-ತಿ-ಯಾಗಿ
ಆಹು-ತಿ-ಯಾ-ಗಿ-ದ್ದರು
ಆಹು-ತಿ-ಯಾ-ಗಿ-ಹೋ-ಯಿತು
ಆಹು-ತಿ-ಯಾ-ಗು-ತ್ತಿ-ರು-ವುದನ್ನು
ಆಹು-ತಿ-ಯಾ-ಗು-ವುದು
ಆಹು-ತಿ-ಯಾ-ದನು
ಆಹು-ತಿ-ಯಾ-ದರು
ಆಹು-ತಿ-ಯಾ-ಯಿತು
ಆಹ್ವಾನ
ಆಹ್ವಾ-ನ-ಗಳು
ಆಹ್ವಾ-ನ-ಮಾ-ಡಿ-ದೊ-ಡ-ನೆಯೆ
ಆಹ್ವಾನಿ
ಆಹ್ವಾ-ನಿ-ಸಲು
ಆಹ್ವಾ-ನಿಸಿ
ಆಹ್ವಾ-ನಿ-ಸಿದ
ಆಹ್ವಾ-ನಿ-ಸಿ-ದ್ದಾನೆ
ಆಹ್ವಾ-ನಿಸು
ಆಹ್ವಾ-ನಿ-ಸು-ತ್ತವೆ
ಇಂಗಿ-ತ-ವನ್ನು
ಇಂಗು-ತ್ತದೆ
ಇಂಚ-ರ-ಕ್ಕಿಂ-ತಲೂ
ಇಂಚ-ರಕ್ಕೂ
ಇಂತಹ
ಇಂತ-ಹ-ವನು
ಇಂತ-ಹ-ವ-ರೆಂದು
ಇಂತ-ಹು-ದನ್ನು
ಇಂತು
ಇಂದಿಗೂ
ಇಂದಿಗೆ
ಇಂದಿನ
ಇಂದಿ-ನ-ವ-ರೆಗೆ
ಇಂದಿ-ನಿಂದ
ಇಂದು
ಇಂದೇ
ಇಂದೊ
ಇಂದ್ರ
ಇಂದ್ರ-ಜಿತ್ತು
ಇಂದ್ರ-ದ್ಯುಮ್ನ
ಇಂದ್ರ-ದ್ಯು-ಮ್ನ-ನಿಗೆ
ಇಂದ್ರನ
ಇಂದ್ರ-ನ-ನ್ನಾಗಿ
ಇಂದ್ರ-ನನ್ನು
ಇಂದ್ರ-ನನ್ನೆ
ಇಂದ್ರ-ನಿಂದ
ಇಂದ್ರ-ನಿಗೆ
ಇಂದ್ರ-ನೀ-ಲ-ಮಣಿ
ಇಂದ್ರನು
ಇಂದ್ರನೆ
ಇಂದ್ರನೇ
ಇಂದ್ರ-ಪ-ದವಿ
ಇಂದ್ರ-ಪ-ದ-ವಿಯ
ಇಂದ್ರ-ಪ್ರ-ಸ್ಥಕ್ಕೆ
ಇಂದ್ರ-ಪ್ರ-ಸ್ಥದ
ಇಂದ್ರ-ಪ್ರ-ಸ್ಥ-ದಲ್ಲಿ
ಇಂದ್ರ-ಪ್ರ-ಸ್ಥ-ದ-ಲ್ಲಿದ್ದ
ಇಂದ್ರ-ಪ್ರ-ಸ್ಥ-ಪುರ
ಇಂದ್ರ-ಪ್ರ-ಸ್ಥ-ಪು-ರ-ದಲ್ಲಿ
ಇಂದ್ರ-ಪ್ರ-ಸ್ಥ-ವನ್ನು
ಇಂದ್ರ-ಯಾ-ಗವು
ಇಂದ್ರ-ವಾ-ಹ-ನ-ಎಂಬ
ಇಂದ್ರಾದಿ
ಇಂದ್ರಾ-ದಿ-ಗಳನ್ನು
ಇಂದ್ರಾ-ದಿ-ಗಳು
ಇಂದ್ರಿಯ
ಇಂದ್ರಿ-ಯಕ್ಕೆ
ಇಂದ್ರಿ-ಯ-ಗಳ
ಇಂದ್ರಿ-ಯ-ಗಳನ್ನು
ಇಂದ್ರಿ-ಯ-ಗಳನ್ನೂ
ಇಂದ್ರಿ-ಯ-ಗಳಿಂದ
ಇಂದ್ರಿ-ಯ-ಗ-ಳಿಗೂ
ಇಂದ್ರಿ-ಯ-ಗ-ಳಿಗೆ
ಇಂದ್ರಿ-ಯ-ಗಳು
ಇಂದ್ರಿ-ಯ-ಗಳೂ
ಇಂದ್ರಿ-ಯ-ಗ-ಳೆಲ್ಲ
ಇಂದ್ರಿ-ಯ-ಗೋ-ಚರ
ಇಂದ್ರಿ-ಯ-ಜಯ
ಇಂದ್ರಿ-ಯ-ಜ್ಞಾ-ನ-ವಿ-ರು-ವು-ದಿ-ಲ್ಲ-ವಾ-ದರೂ
ಇಂದ್ರಿ-ಯ-ನಿ-ಗ್ರ-ಹ-ಮಾಡಿ
ಇಂದ್ರಿ-ಯ-ಪ್ರೇ-ರಕ
ಇಂದ್ರಿ-ಯ-ರೂ-ಪದ
ಇಂದ್ರಿ-ಯ-ಶಕ್ತಿ
ಇಂದ್ರಿ-ಯ-ಸು-ಖಕ್ಕೆ
ಇಂದ್ರಿ-ಯ-ಸು-ಖ-ಗಳ
ಇಂದ್ರಿ-ಯ-ಸು-ಖ-ಗ-ಳ-ಲ್ಲಿಯೇ
ಇಂದ್ರಿ-ಯ-ಸು-ಖದ
ಇಂದ್ರಿ-ಯ-ಸು-ಖ-ವಲ್ಲ
ಇಂದ್ರಿ-ಯಾಧಿ
ಇಂದ್ರಿ-ಯಾ-ಭಿ-ಮಾ-ನಿ-ಗ-ಳಾದ
ಇಂಪನ್ನೂ
ಇಂಪಾಗಿ
ಇಂಪಾದ
ಇಂಪಿಗೆ
ಇಂಪು
ಇಂಪೂ
ಇಕ್ಕ-ಟ್ಟಿಗೆ
ಇಕ್ಕ-ಡೆಯ
ಇಕ್ಕಿ-ದುದು
ಇಕ್ಕೆ-ಲ-ಗ-ಳ-ಲ್ಲಿಯೂ
ಇಕ್ಕೆ-ಲ-ದಲ್ಲಿ
ಇಕ್ಕೆ-ಲ-ದ-ಲ್ಲಿಯೂ
ಇಕ್ಷು-ಮ-ತೀ-ನ-ದಿಯ
ಇಕ್ಷು-ಸ-ಮು-ದ್ರ-ವಿದೆ
ಇಕ್ಷ್ವಾಕು
ಇಕ್ಷ್ವಾ-ಕು-ವಂ-ಶಕ್ಕೆ
ಇಕ್ಷ್ವಾ-ಕು-ವಿನ
ಇಕ್ಷ್ವಾ-ಕು-ವಿ-ನಂತೆ
ಇಗೋ
ಇಚ್ಛಾ-ಮಾ-ತ್ರ-ದಿಂದ
ಇಚ್ಛಿ-ಸಿ-ದನು
ಇಚ್ಛೆ
ಇಟ್ಟ
ಇಟ್ಟರು
ಇಟ್ಟ-ವ-ರಿಗೆ
ಇಟ್ಟ-ವಳು
ಇಟ್ಟಿ-ದುದು
ಇಟ್ಟಿದ್ದ
ಇಟ್ಟಿ-ದ್ದರೆ
ಇಟ್ಟಿ-ರ-ಬೇಕು
ಇಟ್ಟಿ-ರು-ವುದು
ಇಟ್ಟು
ಇಟ್ಟು-ಕೊಂ-ಡಿ-ದ್ದರು
ಇಟ್ಟು-ಕೊಂ-ಡಿದ್ದು
ಇಟ್ಟು-ಕೊಂಡು
ಇಟ್ಟು-ಕೊ-ಳ್ಳು-ತ್ತೇನೆ
ಇಟ್ಟು-ದಾ-ಯಿತು
ಇಡೀ
ಇಡು
ಇಡು-ತ್ತಾನೆ
ಇಡು-ತ್ತಿ-ದ್ದೇನೆ
ಇಡು-ತ್ತಿ-ರುವ
ಇಡು-ತ್ತೇನೋ
ಇಡು-ತ್ತೇವೆ
ಇಡು-ವು-ದಕ್ಕೆ
ಇಣಿಕಿ
ಇಣಿ-ಕಿ-ಹಾಕಿ
ಇತರ
ಇತ-ರರ
ಇತ-ರ-ರಂತೆ
ಇತ-ರ-ರ-ನ್ನಾ-ದರೂ
ಇತ-ರ-ರನ್ನು
ಇತ-ರ-ರಿಂ-ದಲ್ಲ
ಇತ-ರ-ರಿಗೂ
ಇತ-ರ-ರಿಗೆ
ಇತ-ರರು
ಇತ-ರರೂ
ಇತ-ರ-ರೊ-ಡನೆ
ಇತಿ
ಇತಿ-ಮಿತಿ
ಇತಿ-ಹಾಸ
ಇತಿ-ಹಾ-ಸ-ಕ್ಕಿಂ-ತಲೂ
ಇತಿ-ಹಾ-ಸದ
ಇತಿ-ಹಾ-ಸ-ದಿಂದ
ಇತಿ-ಹಾ-ಸ-ವನ್ನು
ಇತಿ-ಹಾ-ಸ-ವನ್ನೂ
ಇತ್ತ
ಇತ್ತಂ-ಡ-ದಲ್ಲಿ
ಇತ್ತ-ಕಡೆ
ಇತ್ತರೆ
ಇತ್ತಳು
ಇತ್ತೀ-ಚೆಗೆ
ಇತ್ತು
ಇತ್ತೆ-ಯಲ್ಲಾ
ಇತ್ಯಾದಿ
ಇತ್ಯಾ-ದಿ-ಗಳನ್ನೆಲ್ಲ
ಇತ್ಯಾ-ದಿ-ಗಳಿಂದ
ಇತ್ಯಾ-ದಿ-ಗಳು
ಇತ್ಯಾ-ದಿ-ಯಾಗಿ
ಇತ್ಯಾ-ದಿ-ಯಾ-ದ-ವರೆ-ಲ್ಲರೂ
ಇದ
ಇದ-ಕ್ಕಾಗಿ
ಇದ-ಕ್ಕಾ-ಗಿಯೇ
ಇದ-ಕ್ಕಿಂ-ತಲೂ
ಇದಕ್ಕೂ
ಇದಕ್ಕೆ
ಇದ-ಕ್ಕೇನು
ಇದ-ನ್ನ-ರಿತ
ಇದ-ನ್ನ-ವ-ರಿಗೆ
ಇದನ್ನು
ಇದನ್ನೆ
ಇದ-ನ್ನೆಲ್ಲ
ಇದನ್ನೇ
ಇದರ
ಇದ-ರಂತೆ
ಇದ-ರಂ-ತೆಯೆ
ಇದ-ರಂ-ತೆಯೇ
ಇದ-ರ-ಮೇಲೆ
ಇದ-ರಲ್ಲಿ
ಇದ-ರ-ಲ್ಲಿಯೇ
ಇದ-ರಿಂದ
ಇದ-ರಿಂ-ದಲೆ
ಇದ-ಲ್ಲದೆ
ಇದಾದ
ಇದಾ-ವು-ದನ್ನೂ
ಇದಾ-ವುದೋ
ಇದಿರಾ
ಇದಿ-ರಾಗಿ
ಇದಿ-ರಾ-ಗಿ-ರುವೆ
ಇದಿ-ರಾ-ದನು
ಇದಿ-ರಾ-ದರು
ಇದಿ-ರಾ-ದಳು
ಇದಿ-ರಾ-ಯಿತು
ಇದಿ-ರಾಳಿ
ಇದಿ-ರಾ-ಳಿಗೆ
ಇದಿ-ರಾಳು
ಇದಿರಿ
ಇದಿ-ರಿ-ಗಿಟ್ಟು
ಇದಿ-ರಿ-ಗಿದ್ದ
ಇದಿ-ರಿ-ಗಿ-ರು-ವಾಗ
ಇದಿ-ರಿಗೆ
ಇದಿ-ರಿಗೇ
ಇದಿ-ರಿನ
ಇದಿ-ರಿ-ನಲ್ಲಿ
ಇದಿ-ರಿ-ನ-ಲ್ಲಿಯೇ
ಇದಿ-ರಿಲ್ಲ
ಇದಿ-ರಿ-ಲ್ಲ-ದಂ-ತಾ-ಯಿತು
ಇದಿ-ರಿ-ಸ-ಲಾ-ರದೆ
ಇದಿ-ರಿ-ಸ-ಲಾ-ರರು
ಇದಿ-ರಿ-ಸಲು
ಇದಿ-ರಿಸಿ
ಇದಿ-ರಿ-ಸಿತು
ಇದಿ-ರಿ-ಸಿದ
ಇದಿ-ರಿ-ಸಿ-ದನು
ಇದಿ-ರಿ-ಸಿ-ದರು
ಇದಿ-ರಿ-ಸು-ವು-ದ-ಕ್ಕಾಗಿ
ಇದಿ-ರಿ-ಸು-ವುದು
ಇದಿರು
ಇದಿ-ರು-ಗೊಂ-ಡನು
ಇದಿ-ರು-ಗೊಂಡು
ಇದಿ-ರು-ನೋಡು
ಇದಿರೇ
ಇದಿಷ್ಟು
ಇದಿಷ್ಟೂ
ಇದು
ಇದು-ರಿ-ಲ್ಲ-ವೆಂಬ
ಇದು-ವ-ರೆಗೂ
ಇದು-ವ-ರೆಗೆ
ಇದೂ
ಇದೆ
ಇದೆಂ-ತಹ
ಇದೆಯೆ
ಇದೆ-ಯೆಂದು
ಇದೆ-ಯೇನು
ಇದೆಲ್ಲ
ಇದೆ-ಲ್ಲ-ವನ್ನೂ
ಇದೆ-ಲ್ಲವೂ
ಇದೇ
ಇದೇ-ನ-ನ್ಯಾಯ
ಇದೇ-ನಾ-ಶ್ಚರ್ಯ
ಇದೇ-ನಿದು
ಇದೇನು
ಇದೇನೂ
ಇದೇನೆ
ಇದೊಂ-ದನ್ನು
ಇದೊಂದು
ಇದೋ
ಇದ್ದ
ಇದ್ದಂತೆ
ಇದ್ದಂ-ತೆಯೆ
ಇದ್ದಂ-ತೆಯೇ
ಇದ್ದ-ಕಿ-ದ್ದಂ-ತೆಯೆ
ಇದ್ದಕ್ಕಿ
ಇದ್ದ-ಕ್ಕಿ-ದ್ದಂತೆ
ಇದ್ದ-ಕ್ಕಿ-ದ್ದಂ-ತೆಯೆ
ಇದ್ದ-ಕ್ಕಿ-ದ್ದಂ-ತೆಯೇ
ಇದ್ದ-ನಾ-ದ್ದ-ರಿಂದ
ಇದ್ದನು
ಇದ್ದ-ರ-ಲ್ಲವೆ
ಇದ್ದ-ರಾ-ದರೂ
ಇದ್ದರು
ಇದ್ದರೂ
ಇದ್ದರೆ
ಇದ್ದ-ಲಿ-ನಂತೆ
ಇದ್ದಲ್ಲಿ
ಇದ್ದ-ವರು
ಇದ್ದ-ವರೆ-ಲ್ಲರೂ
ಇದ್ದವು
ಇದ್ದ-ವು-ಗ-ಳಾ-ದ್ದ-ರಿಂದ
ಇದ್ದ-ಹಾಗೆ
ಇದ್ದಾನೆ
ಇದ್ದಾ-ನೆ-ಎಂದು
ಇದ್ದಿ-ರ-ಬೇಕು
ಇದ್ದಿ-ರ-ಬೇ-ಕೆಂ-ಬುದು
ಇದ್ದು
ಇದ್ದು-ಕೊಂಡೇ
ಇದ್ದು-ದನ್ನೆ
ಇದ್ದು-ದ-ರಿಂದ
ಇದ್ದುದು
ಇದ್ದೂ
ಇದ್ದೆ
ಇದ್ದೇ
ಇದ್ದೇನೆ
ಇಧ್ಮ-ಜಿ-ಹ್ವನು
ಇಧ್ಮ-ಧ-ರ-ರೆಂಬ
ಇನಿ-ದನಿ
ಇನಿ-ದ-ನಿ-ಯಲ್ಲಿ
ಇನಿ-ದಾದ
ಇನ್ನ-ವನು
ಇನ್ನಷ್ಟು
ಇನ್ನಾರ
ಇನ್ನಾ-ರಿಗೆ
ಇನ್ನಾ-ರಿ-ದ್ದಾರೆ
ಇನ್ನಾರು
ಇನ್ನಾವ
ಇನ್ನಿಲ್ಲ
ಇನ್ನು
ಇನ್ನು-ಮುಂದೆ
ಇನ್ನು-ಳಿದ
ಇನ್ನೂ
ಇನ್ನೂರು
ಇನ್ನೂ-ರೈ-ವತ್ತು
ಇನ್ನೆಂ-ತಹ
ಇನ್ನೆ-ರಡು
ಇನ್ನೇ-ನಾ-ಗ-ಬೇಕು
ಇನ್ನೇ-ನಾ-ದರೂ
ಇನ್ನೇನು
ಇನ್ನೊ
ಇನ್ನೊಂದು
ಇನ್ನೊಬ್ಬ
ಇನ್ನೊ-ಬ್ಬನು
ಇನ್ನೊ-ಬ್ಬ-ರಿಗೆ
ಇನ್ನೊ-ಬ್ಬಳ
ಇನ್ನೊ-ಬ್ಬ-ಳಿಲ್ಲ
ಇನ್ನೊ-ಬ್ಬಳು
ಇನ್ನೊಮ್ಮೆ
ಇಪ್ಪತ್ತ
ಇಪ್ಪ-ತ್ತ-ರಲ್ಲಿ
ಇಪ್ಪತ್ತು
ಇಪ್ಪ-ತ್ತು-ನಾಲ್ಕು
ಇಪ್ಪ-ತ್ತು-ಮೂರು
ಇಪ್ಪ-ತ್ತೆಂಟು
ಇಪ್ಪ-ತ್ತೇಳು
ಇಪ್ಪ-ತ್ತೈದು
ಇಪ್ಪ-ತ್ತೈ-ದು-ಕೋಟಿ
ಇಪ್ಪ-ತ್ತೈ-ದೆಂದು
ಇಪ್ಪ-ತ್ತೊಂದು
ಇಬ್ಬ-ಣ-ದ-ವ-ರಿಗೂ
ಇಬ್ಬರ
ಇಬ್ಬ-ರಲ್ಲಿ
ಇಬ್ಬ-ರಿ-ದ್ದರೆ
ಇಬ್ಬರು
ಇಬ್ಬರೂ
ಇಬ್ಬರೆ
ಇಬ್ಬಿ-ಬ್ಬ-ರನ್ನು
ಇಬ್ಬಿ-ಬ್ಬರು
ಇಬ್ಭಾ-ಗ-ವಾಗಿ
ಇಬ್ಭಾ-ಗ-ವಾ-ಯಿತು
ಇಮ್ಮ-ಡಿ-ಯಾ-ಯಿತು
ಇಮ್ಮೆ-ಯಲ್ಲ
ಇರ
ಇರ-ಬ-ಹು-ದಾ-ದರೂ
ಇರ-ಬ-ಹುದು
ಇರ-ಬಾ-ರದು
ಇರ-ಬೇಕು
ಇರ-ಬೇ-ಕೆಂದು
ಇರ-ಬೇ-ಕೆ-ನ್ನಿ-ಸಿತು
ಇರ-ಬೇ-ಡವೇ
ಇರ-ಲಾರ
ಇರಲಿ
ಇರ-ಲಿಲ್ಲ
ಇರ-ಲಿ-ಲ್ಲ-ವಂತೆ
ಇರಲು
ಇರ-ಲೇ-ಬೇ-ಕ-ಲ್ಲವೆ
ಇರ-ಲೇ-ಬೇಕು
ಇರಾ-ವ-ತಿ-ಯನ್ನು
ಇರಾ-ವತೀ
ಇರಿ
ಇರಿದು
ಇರಿ-ಸಿ-ದನು
ಇರು
ಇರು-ತ್ತದೆ
ಇರು-ತ್ತ-ದೆಯೆ
ಇರು-ತ್ತವೆ
ಇರುತ್ತಿ
ಇರು-ತ್ತಿದ್ದ
ಇರು-ತ್ತಿ-ದ್ದನೆ
ಇರು-ತ್ತಿ-ದ್ದ-ರಾ-ದರೂ
ಇರು-ತ್ತಿ-ದ್ದವು
ಇರು-ತ್ತೇನೆ
ಇರುಳು
ಇರುವ
ಇರು-ವಂತೆ
ಇರು-ವನು
ಇರು-ವ-ನೆಂ-ಬು-ದನ್ನು
ಇರು-ವರು
ಇರು-ವ-ರೇನು
ಇರು-ವಲ್ಲಿ
ಇರು-ವ-ವನೂ
ಇರು-ವ-ವ-ರಾರೂ
ಇರು-ವ-ವಳು
ಇರು-ವಾಗ
ಇರು-ವಾ-ಗಲೂ
ಇರು-ವಿಯೊ
ಇರುವು
ಇರು-ವುದನ್ನು
ಇರು-ವು-ದ-ರಿಂದ
ಇರು-ವು-ದಾ-ದರೆ
ಇರು-ವು-ದಿಲ್ಲ
ಇರು-ವು-ದಿ-ಲ್ಲ-ವಂತೆ
ಇರು-ವು-ದಿ-ಲ್ಲ-ವಲ್ಲ
ಇರು-ವು-ದಿ-ಲ್ಲ-ವೆಂದು
ಇರು-ವುದು
ಇರು-ವು-ದೇನು
ಇರು-ವು-ದೊಂದೇ
ಇರುವೆ
ಇರು-ವೆ-ಗಳು
ಇರು-ವೆ-ನೆಂಬು
ಇಲಾ-ವೃತ
ಇಲಾ-ವೃತ್
ಇಲಿ
ಇಲಿಯ
ಇಲಿ-ಯನ್ನು
ಇಲ್ಲ
ಇಲ್ಲ-ಎಂ-ಬುದು
ಇಲ್ಲದ
ಇಲ್ಲ-ದಂ-ತಾ-ಗಿದೆ
ಇಲ್ಲ-ದಂ-ತಾ-ಯಿತು
ಇಲ್ಲ-ದಂ-ತಿ-ರ-ಬೇ-ಕಾ-ಗು-ತ್ತದೆ
ಇಲ್ಲ-ದಂತೆ
ಇಲ್ಲ-ದ-ವರು
ಇಲ್ಲ-ದ-ವ-ಳಂತೆ
ಇಲ್ಲ-ದಾಗ
ಇಲ್ಲ-ದಿ-ದ್ದರೆ
ಇಲ್ಲ-ದಿ-ದ್ದಲ್ಲಿ
ಇಲ್ಲ-ದಿರು
ಇಲ್ಲದೆ
ಇಲ್ಲಮ್ಮ
ಇಲ್ಲ-ವಣ್ಣ
ಇಲ್ಲ-ವಲ್ಲ
ಇಲ್ಲ-ವಾ-ಗು-ವವು
ಇಲ್ಲ-ವಾ-ದರೂ
ಇಲ್ಲವೆ
ಇಲ್ಲ-ವೆಂದು
ಇಲ್ಲ-ವೆಂಬ
ಇಲ್ಲ-ವೆಂ-ಬಂತೆ
ಇಲ್ಲ-ವೆ-ನ್ನ-ಬ-ಹುದೆ
ಇಲ್ಲ-ವೆ-ನ್ನ-ಬಾ-ರ-ದೆಂ-ದು-ಕೊಂ-ಡನು
ಇಲ್ಲ-ವೆ-ನ್ನು-ವು-ದಿಲ್ಲ
ಇಲ್ಲವೇ
ಇಲ್ಲ-ವೇನು
ಇಲ್ಲವೊ
ಇಲ್ಲವೋ
ಇಲ್ಲಿ
ಇಲ್ಲಿಂದ
ಇಲ್ಲಿಗೆ
ಇಲ್ಲಿ-ಗೇಕೆ
ಇಲ್ಲಿನ
ಇಲ್ಲಿ-ನ-ವ-ರಲ್ಲಿ
ಇಲ್ಲಿ-ನ-ವರು
ಇಲ್ಲಿ-ಯ-ವ-ರೆಗೆ
ಇಲ್ಲಿಯೂ
ಇಲ್ಲಿಯೆ
ಇಲ್ಲಿಯೇ
ಇಲ್ಲಿ-ರಲು
ಇಲ್ಲಿ-ರುವ
ಇಲ್ಲಿ-ರು-ವುದು
ಇಲ್ಲಿಲ್ಲ
ಇಲ್ಲಿವೆ
ಇಲ್ಲೇಕೆ
ಇಲ್ವಲ
ಇಳಾ
ಇಳಿ-ತಂದು
ಇಳಿದ
ಇಳಿ-ದರೂ
ಇಳಿ-ದಿದ್ದ
ಇಳಿ-ದಿ-ರಲು
ಇಳಿ-ದಿ-ರುವ
ಇಳಿದು
ಇಳಿ-ದು-ಕೊಂಡು
ಇಳಿ-ದು-ಕೊ-ಳ್ಳ-ಬೇ-ಕೆಂದು
ಇಳಿ-ದು-ಹೋ-ಗು-ವಂತೆ
ಇಳಿ-ಮು-ಖ-ವಾ-ಯಿತು
ಇಳಿ-ಯಿತು
ಇಳಿ-ಸ-ಬೇ-ಕೆಂದು
ಇಳಿಸಿ
ಇಳಿ-ಸಿ-ದನು
ಇಳಿ-ಸಿ-ದು-ದಾ-ಯಿತು
ಇಳಿ-ಸು-ವಂತೆ
ಇಳಿ-ಸು-ವ-ವ-ನಾ-ಗಿ-ದ್ದಾನೆ
ಇಳಿ-ಸು-ವು-ದಕ್ಕೆ
ಇಳು-ಹು-ವಂತೆ
ಇಳೆ-ಇ-ರು-ವಲ್ಕ
ಇಳೆಗೆ
ಇಳೆ-ತೆ-ಗೆದು
ಇಳೆ-ಯಲ್ಲಿ
ಇಳೆಯು
ಇಳೆ-ಯೆಂಬ
ಇವ
ಇವಕ್ಕೆ
ಇವತ್ತು
ಇವನ
ಇವ-ನನ್ನು
ಇವ-ನಲ್ಲಿ
ಇವ-ನಿಂದ
ಇವ-ನಿಗೆ
ಇವ-ನಿದ್ದ
ಇವನು
ಇವನೂ
ಇವ-ನೆಂ-ತಹ
ಇವನೇ
ಇವ-ನೊ-ಡ-ನೆಯೂ
ಇವ-ನೊಬ್ಬ
ಇವನ್ನು
ಇವ-ನ್ನೆಲ್ಲ
ಇವರ
ಇವ-ರನ್ನು
ಇವ-ರ-ನ್ನೆಲ್ಲ
ಇವ-ರ-ಪ್ಪ-ನಾದ
ಇವ-ರಲ್ಲಿ
ಇವ-ರಾ-ದ-ಮೇಲೆ
ಇವ-ರಾರೂ
ಇವ-ರಿಗೆ
ಇವ-ರಿ-ಗೆಲ್ಲ
ಇವ-ರಿ-ಬ್ಬರೂ
ಇವ-ರಿ-ವ-ರಲ್ಲಿ
ಇವರು
ಇವರೆ
ಇವರೆಲ್ಲ
ಇವರೆ-ಲ್ಲರೂ
ಇವರೇ
ಇವಳ
ಇವ-ಳನ್ನು
ಇವ-ಳಿಗೆ
ಇವಳು
ಇವ-ಳೇನೂ
ಇವಾ-ವನ್ನೂ
ಇವು
ಇವು-ಗಳ
ಇವು-ಗಳನ್ನು
ಇವು-ಗಳನ್ನೆಲ್ಲ
ಇವು-ಗಳಲ್ಲಿ
ಇವು-ಗಳಿಂದ
ಇವು-ಗ-ಳಿಂ-ದಾ-ಗುವ
ಇವು-ಗ-ಳಿಗೆ
ಇವೆ
ಇವೆ-ಯಷ್ಟೆ
ಇವೆ-ರ-ಡಕ್ಕೂ
ಇವೆ-ರ-ಡನ್ನೂ
ಇವೆ-ರ-ಡರ
ಇವೆ-ರ-ಡ-ರಲ್ಲಿ
ಇವೆಲ್ಲ
ಇವೆ-ಲ್ಲವೂ
ಇವೇ
ಇಷು-ಮಂತ
ಇಷ್ಟ
ಇಷ್ಟಕ್ಕೆ
ಇಷ್ಟ-ದಂತೆ
ಇಷ್ಟ-ದಂ-ತೆಯೇ
ಇಷ್ಟ-ದೇ-ವ-ತೆ-ಯನ್ನು
ಇಷ್ಟ-ದೈ-ವ-ದಂತೆ
ಇಷ್ಟನ್ನು
ಇಷ್ಟ-ಪಟ್ಟು
ಇಷ್ಟ-ಬಂ-ದ-ವ-ನನ್ನು
ಇಷ್ಟ-ಭ-ರ್ತುಃ
ಇಷ್ಟ-ಮಿ-ತ್ರ-ರನ್ನೂ
ಇಷ್ಟರ
ಇಷ್ಟ-ರಲ್ಲಿ
ಇಷ್ಟ-ರ-ಲ್ಲಿಯೆ
ಇಷ್ಟ-ರಲ್ಲೆ
ಇಷ್ಟ-ವನ್ನು
ಇಷ್ಟ-ವಿ-ರ-ಲಿಲ್ಲ
ಇಷ್ಟ-ವಿಲ್ಲ
ಇಷ್ಟ-ವಿ-ಲ್ಲದೆ
ಇಷ್ಟವೇ
ಇಷ್ಟಾಗಿ
ಇಷ್ಟಾಗು
ಇಷ್ಟಾ-ಗು-ತ್ತಲೆ
ಇಷ್ಟಾ-ಗು-ವು-ದನ್ನೇ
ಇಷ್ಟಾದ
ಇಷ್ಟಾ-ದರೂ
ಇಷ್ಟಾ-ನಿ-ಷ್ಟ-ಗಳನ್ನು
ಇಷ್ಟಾ-ನಿ-ಷ್ಟ-ಗಳಲ್ಲಿ
ಇಷ್ಟಾ-ನು-ಸಾ-ರ-ವಾಗಿ
ಇಷ್ಟಾ-ಯಿ-ತೆಂ-ದರೆ
ಇಷ್ಟಾರ್ಥ
ಇಷ್ಟಾ-ರ್ಥ-ವನ್ನು
ಇಷ್ಟಾ-ರ್ಥ-ಸಿ-ದ್ಧಿ-ಯನ್ನು
ಇಷ್ಟು
ಇಷ್ಟು-ಸ-ಜ್ಜ-ನರ
ಇಷ್ಟೆ
ಇಷ್ಟೆ-ಧ-ರ್ಮಾ-ಧ-ರ್ಮದ
ಇಷ್ಟೆ-ಮ-ಹಾ-ಪು-ರು-ಷ-ರಲ್ಲಿ
ಇಷ್ಟೆಲ್ಲ
ಇಷ್ಟೇ
ಇಷ್ಟೊಂದು
ಇಹ
ಇಹ-ದಲ್ಲಿ
ಇಹ-ಪರ
ಇಹ-ಪ-ರ-ಗಳು
ಇಹ-ಪ-ರ-ಗ-ಳೆ-ರಡೂ
ಇಹ-ಭೋ-ಗ-ದಷ್ಟೇ
ಇಹ-ಲೋಕ
ಇಹ-ಲೋ-ಕದ
ಇಹ-ವನ್ನು
ಈ
ಈಕೆ
ಈಕೆಗೆ
ಈಕೆಯ
ಈಕೆ-ಯನ್ನು
ಈಕೆ-ಯಲ್ಲಿ
ಈಕ್ಷಣ
ಈಗ
ಈಗ-ತಾನೆ
ಈಗ-ಲಂತೂ
ಈಗ-ಲಾ-ದರೂ
ಈಗಲೂ
ಈಗಲೆ
ಈಗಲೇ
ಈಗಾ-ಗಲೆ
ಈಗಾ-ಗಲೇ
ಈಗಿಂ-ದೀ-ಗಲೆ
ಈಗಿಂ-ದೀ-ಗಲೇ
ಈಗಿನ
ಈಗೇನು
ಈಚಿ-ನದು
ಈಚೆಗೆ
ಈಡಾ
ಈಡಾ-ಗಿ-ರು-ವಂತೆ
ಈಡಾ-ಗು-ವು-ದಿಲ್ಲ
ಈಡು
ಈಡೇ-ರಿ-ದಂ-ತಾ-ಯಿತು
ಈಡೇ-ರಿ-ಸ-ಬೇಕು
ಈಡೇ-ರಿ-ಸ-ಲಿಲ್ಲ
ಈಡೇ-ರಿ-ಸ-ಲೆಂದು
ಈಡೇ-ರಿಸಿ
ಈಡೇ-ರಿ-ಸಿ-ಕೊಡಿ
ಈಡೇ-ರಿ-ಸು-ವಂತೆ
ಈಡೇ-ರಿ-ಸು-ವನೊ
ಈಡೇ-ರುವ
ಈಡೇ-ರು-ವು-ದಕ್ಕೆ
ಈತ
ಈತನ
ಈತ-ನದು
ಈತ-ನನ್ನು
ಈತ-ನ-ವ-ರೆಗೆ
ಈತ-ನಿಂದ
ಈತ-ನಿಂ-ದಲೇ
ಈತ-ನಿಂ-ದಾಚೆ
ಈತ-ನಿಗೆ
ಈತನು
ಈತನೆ
ಈತ-ನೆಲ್ಲಿ
ಈತನೇ
ಈಬಾರಿ
ಈವ-ರೆಗೆ
ಈಶಃ
ಈಶಾ-ನು-ಕ-ಥನ
ಈಶಾನ್ಯ
ಈಶಾ-ನ್ಯ-ದಿ-ಕ್ಕಿಗೆ
ಈಶೋ-ಽಸಿ-ಧರೋ
ಈಶ್ವರ
ಈಶ್ವ-ರ-ಕಥೆ
ಈಶ್ವ-ರನ
ಈಶ್ವ-ರ-ನನ್ನು
ಈಶ್ವ-ರ-ನಲ್ಲಿ
ಈಶ್ವ-ರ-ನಿಗೆ
ಈಶ್ವ-ರನು
ಈಶ್ವ-ರನೂ
ಈಶ್ವ-ರ-ನೊ-ಬ್ಬ-ನಿ-ರು-ವ-ನೆಂಬ
ಈಶ್ವ-ರ-ಭ-ಕ್ತಿ-ಯೊಂದೆ
ಈಶ್ವ-ರ-ರಲ್ಲಿ
ಈಶ್ವ-ರ-ಸಂ-ಕಲ್ಪ
ಈಶ್ವ-ರ-ಸ್ವ-ರೂ-ಪ-ವನ್ನು
ಈಶ್ವ-ರಾಂಶ
ಈಶ್ವ-ರಾ-ಧಿ-ಷ್ಠಿ-ತ-ವಾದ
ಈಶ್ವ-ರಾ-ರ್ಪ-ಣ-ಬು-ದ್ಧಿ-ಯಿಂದ
ಈಶ್ವ-ರಾ-ರ್ಪಿ-ತ-ವಾಗಿ
ಉಂಗು-ಷ್ಠ-ದಿಂದ
ಉಂಟಾ-ಗ-ಬೇಕು
ಉಂಟಾಗು
ಉಂಟಾ-ಗು-ತ್ತವೆ
ಉಂಟಾ-ದುವು
ಉಂಟು
ಉಂಟು-ಮಾ-ಡ-ತಕ್ಕ
ಉಂಟೆ
ಉಂಡು
ಉಂಡೆ-ಯಂ-ತಿದ್ದ
ಉಂಡೆ-ಯ-ನ್ನಿ-ಟ್ಟು-ಕೊಂಡು
ಉಕ್ಕಿ
ಉಕ್ಕಿ-ತಾ-ದರೂ
ಉಕ್ಕಿತು
ಉಕ್ಕಿನ
ಉಕ್ಕಿ-ನಂ-ತಿ-ರುವ
ಉಕ್ಕಿ-ಸು-ತ್ತಿ-ದ್ದರೆ
ಉಕ್ಕು
ಉಕ್ಕು-ತ್ತಿ-ರಲು
ಉಕ್ಕು-ವು-ದಿಲ್ಲ
ಉಕ್ತಂ
ಉಗುರು
ಉಗು-ರು-ಗ-ಳಂತೂ
ಉಗು-ರು-ಗ-ಳಿಂ-ದಲೆ
ಉಗು-ರು-ಗಳು
ಉಗು-ರು-ಗ-ಳುಈ
ಉಗು-ಳಿ-ದರೆ
ಉಗು-ಳುವ
ಉಗ್ರ
ಉಗ್ರ-ಟೀ-ಕೆಗೆ
ಉಗ್ರ-ತ-ಪ-ವ-ನ್ನಾ-ಚ-ರಿಸಿ
ಉಗ್ರ-ತೀತ
ಉಗ್ರ-ವಾದ
ಉಗ್ರ-ಶ್ರ-ವನು
ಉಗ್ರ-ಸೇನ
ಉಗ್ರ-ಸೇ-ನ-ನನ್ನು
ಉಗ್ರ-ಸೇ-ನ-ನಿಗೆ
ಉಗ್ರ-ಸೇ-ನನೇ
ಉಗ್ರ-ಸೇ-ನ-ಮಹಾ
ಉಗ್ರ-ಸೇ-ನರ
ಉಗ್ರಾಣ
ಉಚಿತ
ಉಚಿ-ತ-ವಾದ
ಉಚಿ-ತಾ-ನು-ಚಿ-ತದ
ಉಚಿ-ತಾ-ಸ-ನ-ದಲ್ಲಿ
ಉಚಿ-ತಾ-ಸ-ನ-ವ-ನ್ನಿತ್ತು
ಉಚ್ಚ-ಕಂಠ
ಉಚ್ಚ-ತರ
ಉಚ್ಚ-ನೀ-ಚ-ಗ-ಳೆಂದು
ಉಚ್ಚ-ರಿ-ಸಿ-ದರೆ
ಉಚ್ಚ-ರಿ-ಸುತ್ತಾ
ಉಚ್ಚೈ
ಉಚ್ಛ್ವಾಸ
ಉಜ್ಜಿ
ಉಜ್ಜಿ-ದನು
ಉಜ್ಜಿ-ದರೆ
ಉಜ್ವ-ಲ-ವಾ-ಯಿತು
ಉಟ್ಟ
ಉಟ್ಟ-ಸೀರೆ
ಉಟ್ಟಿದ್ದ
ಉಟ್ಟಿ-ದ್ದಾನೆ
ಉಟ್ಟಿ-ದ್ದಾಳೆ
ಉಟ್ಟು
ಉಡಲು
ಉಡಿಗೆ
ಉಡಿ-ಸಿ-ದರು
ಉಡುಗಿ
ಉಡು-ಗಿತು
ಉಡು-ಗೊ-ರೆ-ಗಳನ್ನು
ಉಡು-ಗೊ-ರೆ-ಯನ್ನು
ಉಡು-ತ್ತಿ-ರುವೆ
ಉಡು-ತ್ತೀ-ಯ-ಲ್ಲವೆ
ಉಡು-ತ್ತೀಯಾ
ಉಣಿ-ಸಿ-ದರು
ಉತಾ-ರ್ಧ-ರಾತ್ರೇ
ಉತ್ಕಲ
ಉತ್ಕ-ಲನು
ಉತ್ಕ-ಲ-ನೆಂಬ
ಉತ್ತಮ
ಉತ್ತ-ಮ-ಕು-ಮಾ-ರನ
ಉತ್ತ-ಮ-ಕು-ಮಾ-ರ-ನಂತೆ
ಉತ್ತ-ಮ-ಕು-ಮಾ-ರ-ನನ್ನು
ಉತ್ತ-ಮ-ಕು-ಮಾ-ರನು
ಉತ್ತ-ಮ-ಗ-ತಿ-ಯನ್ನು
ಉತ್ತ-ಮ-ನಾದ
ಉತ್ತ-ಮನು
ಉತ್ತ-ಮ-ರಾದ
ಉತ್ತ-ಮ-ವಾದ
ಉತ್ತ-ಮ-ವೆ-ನಿ-ಸು-ತ್ತದೆ
ಉತ್ತ-ಮ-ಶ್ಲೋ-ಕಾಯ
ಉತ್ತರ
ಉತ್ತ-ರ-ಕಾಂ-ಡದ
ಉತ್ತ-ರ-ಕೊ-ಟ್ಟ-ಅಪ್ಪ
ಉತ್ತ-ರ-ಕೊ-ಟ್ಟಳು
ಉತ್ತ-ರ-ಕೊಡು
ಉತ್ತ-ರ-ಕೊ-ಡುತ್ತಾ
ಉತ್ತ-ರ-ಕೊ-ಡುವ
ಉತ್ತ-ರ-ಕ್ಕಾಗಿ
ಉತ್ತ-ರಕ್ಕೆ
ಉತ್ತ-ರ-ಕ್ರಿ-ಯಾ-ದಿ-ಗಳನ್ನು
ಉತ್ತ-ರ-ಕ್ರಿ-ಯೆ-ಗಳನ್ನು
ಉತ್ತ-ರ-ಕ್ರಿ-ಯೆ-ಗಳನ್ನೆಲ್ಲ
ಉತ್ತ-ರ-ಗಳಲ್ಲಿ
ಉತ್ತ-ರ-ಗೋ-ಗ್ರ-ಹ-ಣ-ದಲ್ಲಿ
ಉತ್ತ-ರ-ದಲ್ಲಿ
ಉತ್ತ-ರ-ದಿ-ಕ್ಕಿಗೆ
ಉತ್ತ-ರ-ದಿ-ಕ್ಕಿ-ನಲ್ಲಿ
ಉತ್ತ-ರನ
ಉತ್ತ-ರ-ಪಾ-ರ್ಶ್ವ-ದಲ್ಲಿ
ಉತ್ತ-ರ-ಭಾ-ರ-ತದ
ಉತ್ತ-ರ-ವನ್ನು
ಉತ್ತ-ರ-ವನ್ನೂ
ಉತ್ತ-ರ-ವನ್ನೆ
ಉತ್ತ-ರ-ವಾಗಿ
ಉತ್ತ-ರ-ವಿತ್ತ
ಉತ್ತ-ರ-ವಿ-ತ್ತ-ಅಯ್ಯಾ
ಉತ್ತ-ರ-ವಿ-ತ್ತ-ಎಲೆ
ಉತ್ತ-ರ-ವಿ-ತ್ತ-ರು-ಮಿತ್ರ
ಉತ್ತ-ರ-ವಿ-ತ್ತಳು
ಉತ್ತ-ರವೂ
ಉತ್ತರಾ
ಉತ್ತ-ರಾ-ಭಿ-ಮು-ಖ-ವಾಗಿ
ಉತ್ತ-ರಾ-ಯಣ
ಉತ್ತ-ರಾರ್ಧ
ಉತ್ತ-ರೀ-ಯ-ಇ-ವು-ಗಳನ್ನು
ಉತ್ತ-ರೀ-ಯ-ವನ್ನು
ಉತ್ತ-ರೀ-ಯ-ವನ್ನೆ
ಉತ್ತರೆ
ಉತ್ತ-ರೆಗೆ
ಉತ್ತ-ರೆಯ
ಉತ್ತ-ರೆ-ಯೊ-ಡನೆ
ಉತ್ತಾನ
ಉತ್ತಾ-ನ-ಪಾದ
ಉತ್ತಾ-ನ-ಪಾ-ದ-ಎಂಬ
ಉತ್ತಾ-ನ-ಪಾ-ದ-ನಿಗೆ
ಉತ್ತಾ-ನ-ಪಾ-ದನು
ಉತ್ತಾ-ನ-ಪಾ-ದರು
ಉತ್ತು
ಉತ್ಪಾತ
ಉತ್ಪಾ-ತ-ಗಳನ್ನು
ಉತ್ಪಾ-ತ-ಗಳಾ
ಉತ್ಪಾ-ತ-ಗ-ಳಾ-ದವು
ಉತ್ಪಾ-ತ-ಗಳು
ಉತ್ಪಾ-ತ-ಗ-ಳೆಲ್ಲ
ಉತ್ಪಾ-ತ-ಗ-ಳೊ-ಡನೆ
ಉತ್ಪಾ-ತ-ವೊಂ-ದನ್ನೆ
ಉತ್ಪ್ರೇ-ಕ್ಷೆ-ಗಳು
ಉತ್ಸವ
ಉತ್ಸ-ವಕ್ಕೆ
ಉತ್ಸ-ವದ
ಉತ್ಸ-ವ-ವನ್ನು
ಉತ್ಸಾಹ
ಉತ್ಸಾ-ಹ-ಇ-ವನ್ನು
ಉತ್ಸಾ-ಹಕ್ಕೆ
ಉತ್ಸಾ-ಹ-ಗೊಂಡ
ಉತ್ಸಾ-ಹ-ದಿಂದ
ಉತ್ಸಾ-ಹ-ವನ್ನು
ಉತ್ಸಾ-ಹ-ವೆಲ್ಲ
ಉದ-ನ್ವನ್
ಉದ-ಯ-ರಾ-ಗ-ವನ್ನು
ಉದ-ಯಿ-ಸಿ-ದನು
ಉದ-ಯಿ-ಸಿ-ದರು
ಉದ-ಯಿಸು
ಉದ-ರಂ-ಭ-ರ-ಣವೆ
ಉದ-ರಾಯ
ಉದಾ
ಉದಾ-ತ್ತ-ದೃಷ್ಟಿ
ಉದಾ-ತ್ತ-ಶ-ಕ್ತಿಃ
ಉದಾ-ರ-ವಾ-ದುದು
ಉದಾರಿ
ಉದಾ-ರಿ-ಯಾದ
ಉದಾ-ಸೀನ
ಉದಾ-ಸೀ-ನ-ದಿಂದ
ಉದಾ-ಸೀ-ನ-ನಾ-ಗಿ-ರುವ
ಉದಾ-ಸೀ-ನ-ನಾ-ಗಿ-ರು-ವಂತೆ
ಉದಾ-ಸೀ-ನ-ರಾ-ಗಿ-ದ್ದು-ದ-ರಿಂದ
ಉದಾ-ಸೀ-ನ-ರಾ-ಗು-ವುದು
ಉದಾ-ಸೀ-ನ-ವೇಕೆ
ಉದಾ-ಹ-ರಣೆ
ಉದಾ-ಹ-ರ-ಣೆ-ಯಾಗಿ
ಉದಾ-ಹ-ರ-ಣೆ-ಯಿಂದ
ಉದಾ-ಹ-ರ-ಣೆಯೇ
ಉದಿ
ಉದಿ-ತೋ-ದಿ-ತ-ನಾಗು
ಉದಿ-ಸ-ಬೇಕಾ
ಉದಿಸಿ
ಉದಿ-ಸಿತು
ಉದಿ-ಸಿ-ದರು
ಉದಿ-ಸಿ-ದವು
ಉದಿ-ಸಿ-ದುದು
ಉದಿ-ಸಿದೆ
ಉದಿ-ಸಿ-ಬಂದ
ಉದಿ-ಸಿ-ಬಂ-ದರು
ಉದಿ-ಸಿ-ಬಂ-ದ-ವ-ರಂತೆ
ಉದಿ-ಸು-ತ್ತದೆ
ಉದಿ-ಸು-ವುದು
ಉದು-ರಿದ
ಉದು-ರಿ-ದವು
ಉದು-ರಿವೆ
ಉದು-ರಿ-ಸಿ-ದನು
ಉದು-ರಿ-ಹೋ-ದವು
ಉದು-ರು-ತ್ತದೆ
ಉದ್ಗಾರ
ಉದ್ಗಾ-ರ-ಮಾ-ಡಿ-ದರು
ಉದ್ದ
ಉದ್ದಕ್ಕೂ
ಉದ್ದಕ್ಕೆ
ಉದ್ದ-ಮಾ-ಡಿ-ಕೊಂಡು
ಉದ್ದ-ವಾದ
ಉದ್ದೇಶ
ಉದ್ದೇ-ಶ-ದಿಂದ
ಉದ್ದೇ-ಶ-ದಿಂ-ದಲೆ
ಉದ್ದೇ-ಶ-ವನ್ನು
ಉದ್ದೇ-ಶ-ವಿಲ್ಲ
ಉದ್ದೇ-ಶ-ವೇ-ನೆಂ-ಬು-ದನ್ನು
ಉದ್ದೇ-ಶ-ವೇ-ನೆಂ-ಬುದು
ಉದ್ದೇ-ಶಿಸಿ
ಉದ್ಧ
ಉದ್ಧರಿ
ಉದ್ಧ-ರಿ-ಸ-ತ-ಕ್ಕು-ದಾದ
ಉದ್ಧ-ರಿ-ಸ-ಬಲ್ಲ
ಉದ್ಧ-ರಿ-ಸ-ಬೇಕು
ಉದ್ಧ-ರಿ-ಸ-ಲೆಂದೇ
ಉದ್ಧ-ರಿ-ಸ-ಹೊ-ರ-ಟಿ-ರು-ವೆಯೋ
ಉದ್ಧ-ರಿಸಿ
ಉದ್ಧ-ರಿ-ಸಿದ
ಉದ್ಧ-ರಿ-ಸಿ-ದಿರಿ
ಉದ್ಧ-ರಿ-ಸಿರಿ
ಉದ್ಧ-ರಿಸು
ಉದ್ಧ-ರಿ-ಸುಂತೆ
ಉದ್ಧ-ರಿ-ಸು-ವಂತೆ
ಉದ್ಧ-ರಿ-ಸು-ವು-ದ-ಕ್ಕಾಗಿ
ಉದ್ಧ-ರಿ-ಸು-ವು-ದ-ಕ್ಕಾ-ಗಿಯೇ
ಉದ್ಧ-ರಿ-ಸೆಂದು
ಉದ್ಧವ
ಉದ್ಧ-ವ-ಗೀತೆ
ಉದ್ಧ-ವನ
ಉದ್ಧ-ವ-ನನ್ನು
ಉದ್ಧ-ವ-ನಿಗೆ
ಉದ್ಧ-ವನು
ಉದ್ಧ-ವ-ನೊ-ಡನೆ
ಉದ್ಧ-ವರ
ಉದ್ಧವಾ
ಉದ್ಧಾರ
ಉದ್ಧಾ-ರ-ಕ್ಕಾಗಿ
ಉದ್ಧಾ-ರ-ಕ್ಕಾ-ಗಿಯೆ
ಉದ್ಧಾ-ರಕ್ಕೆ
ಉದ್ಧಾ-ರದ
ಉದ್ಧಾ-ರ-ಮಾ-ಡ-ಬೇ-ಕೆಂ-ದಿ-ರು-ವೆನೊ
ಉದ್ಧಾ-ರ-ಮಾ-ಡಿ-ದಿರಿ
ಉದ್ಧಾ-ರ-ಮಾ-ಡು-ತ್ತದೆ
ಉದ್ಧಾ-ರ-ಮಾ-ಡು-ತ್ತಾನೆ
ಉದ್ಧಾ-ರ-ಮಾರ್ಗ
ಉದ್ಧಾ-ರ-ವನ್ನು
ಉದ್ಧಾ-ರ-ವಾ-ಗ-ಬೇಕು
ಉದ್ಧಾ-ರ-ವಾ-ಗಲಿ
ಉದ್ಧಾ-ರ-ವಾ-ಗಲು
ಉದ್ಧಾ-ರ-ವಾಗಿ
ಉದ್ಧಾ-ರ-ವಾ-ಗಿ-ದ್ದಾರೆ
ಉದ್ಧಾ-ರ-ವಾ-ಗಿಯೇ
ಉದ್ಧಾ-ರ-ವಾ-ಗು-ತ್ತವೆ
ಉದ್ಧಾ-ರ-ವಾ-ಗು-ತ್ತಾರೆ
ಉದ್ಧಾ-ರ-ವಾ-ಗು-ತ್ತೇವೆ
ಉದ್ಧಾ-ರ-ವಾ-ಗುವ
ಉದ್ಧಾ-ರ-ವಾ-ಗು-ವಂ-ತಹ
ಉದ್ಧಾ-ರ-ವಾ-ಗು-ವು-ದಕ್ಕೆ
ಉದ್ಧಾ-ರ-ವಾ-ಗು-ವುದು
ಉದ್ಧಾ-ರ-ವಾದ
ಉದ್ಧಾ-ರ-ವಾ-ದ-ರಂತೆ
ಉದ್ಧಾ-ರ-ವಾ-ದರು
ಉದ್ಧಾ-ರ-ವಾ-ದವು
ಉದ್ಧಾ-ರ-ವೆಂತು
ಉದ್ಧೃ-ತಾನಿ
ಉದ್ಭ-ವಿ-ಸಿತು
ಉದ್ಯಾನ
ಉದ್ಯಾ-ನಕ್ಕೆ
ಉದ್ಯಾ-ನದ
ಉದ್ಯಾ-ನ-ವನ
ಉದ್ಯಾ-ನ-ವ-ನಕ್ಕೆ
ಉದ್ಯಾ-ನ-ವ-ನ-ಗಳು
ಉದ್ಯಾ-ನ-ವ-ನದ
ಉದ್ಯಾ-ನ-ವ-ನ-ದಲ್ಲಿ
ಉದ್ಯಾ-ನ-ವ-ನ-ವನ್ನು
ಉದ್ಯಾ-ನ-ವ-ನ-ವಿದೆ
ಉದ್ಯೋಗ
ಉದ್ಯೋ-ಗ-ದಲ್ಲಿ
ಉದ್ಯೋ-ಗಿಸು
ಉನ್ನ-ತ-ವಾದ
ಉನ್ಮ-ತ್ತ-ತೆ-ಯಿದೆ
ಉನ್ಮಾದ
ಉಪ
ಉಪ-ಕಾರ
ಉಪ-ಕಾ-ರ-ಕ್ಕಾಗಿ
ಉಪ-ಕಾ-ರಕ್ಕೆ
ಉಪ-ಕಾ-ರ-ವನ್ನು
ಉಪ-ಕಾ-ರ-ವನ್ನೇ
ಉಪ-ಕಾ-ರ-ವಾ-ಗ-ಬೇ-ಕಾ-ಗಿದೆ
ಉಪ-ಕಾ-ರ-ವೆಲ್ಲ
ಉಪ-ಚರಿ
ಉಪ-ಚ-ರಿಸಿ
ಉಪ-ಚ-ರಿ-ಸಿದ
ಉಪ-ಚ-ರಿ-ಸಿ-ದ-ನಲ್ಲಾ
ಉಪ-ಚ-ರಿ-ಸಿ-ದನು
ಉಪ-ಚ-ರಿ-ಸು-ತ್ತಿತ್ತು
ಉಪ-ಚ-ರಿ-ಸು-ವನು
ಉಪ-ಚಾರ
ಉಪ-ಚಾ-ರ-ಗ-ಳಾ-ಗಲಿ
ಉಪ-ಚಾ-ರ-ಗಳಿಂದ
ಉಪ-ಚಾ-ರ-ಗ-ಳೆಲ್ಲ
ಉಪ-ಚಾ-ರದ
ಉಪ-ಚಾ-ರ-ದಿಂದ
ಉಪ-ಟ-ಳ-ವನ್ನು
ಉಪ-ದೇ-ವಾ-ಕ-ಲ್ಪ-ವರ್ಷ
ಉಪ-ದೇಶ
ಉಪ-ದೇ-ಶ-ಕೊ-ಟ್ಟನು
ಉಪ-ದೇ-ಶ-ದಂತೆ
ಉಪ-ದೇ-ಶ-ದಿಂದ
ಉಪ-ದೇ-ಶ-ಮಾಡಿ
ಉಪ-ದೇ-ಶ-ಮಾ-ಡಿ-ದನು
ಉಪ-ದೇ-ಶ-ಮಾ-ಡು-ತ್ತಾ-ನಂತೆ
ಉಪ-ದೇ-ಶವ
ಉಪ-ದೇ-ಶ-ವನ್ನು
ಉಪ-ದೇ-ಶ-ವಾ-ಣಿ-ಯಿಂದ
ಉಪ-ದೇ-ಶ-ವೆಂದು
ಉಪ-ದೇ-ಶ-ವೆಂಬ
ಉಪ-ದೇ-ಶವೇ
ಉಪ-ದೇ-ಶಿ-ಸ-ಬೇ-ಕೆಂದು
ಉಪ-ದೇ-ಶಿಸಿ
ಉಪ-ದೇ-ಶಿ-ಸಿದ
ಉಪ-ದೇ-ಶಿ-ಸಿ-ದಂ-ತಾ-ಯಿತು
ಉಪ-ದೇ-ಶಿ-ಸಿ-ದನು
ಉಪ-ದೇ-ಶಿ-ಸಿ-ದರು
ಉಪ-ದೇ-ಶಿ-ಸಿ-ದಳು
ಉಪ-ದೇ-ಶಿ-ಸಿ-ದ್ದ-ನಂತೆ
ಉಪ-ದೇ-ಶಿಸು
ಉಪ-ದೇ-ಶಿ-ಸು-ತ್ತೇನೆ
ಉಪ-ದೇ-ಶಿ-ಸು-ವಂತೆ
ಉಪ-ದೇ-ಶಿ-ಸು-ವ-ವನೂ
ಉಪ-ದೇ-ಶಿ-ಸು-ವೆನು
ಉಪ-ದ್ರ-ವ-ಗಳನ್ನು
ಉಪ-ದ್ರ-ವ-ಗಳೂ
ಉಪ-ನಂದ
ಉಪ-ನ-ಯನ
ಉಪ-ನ-ಯ-ನದ
ಉಪ-ನ-ಯ-ನ-ವನ್ನು
ಉಪ-ನಿ-ಷ-ತ್ತಿ-ನಲ್ಲಿ
ಉಪ-ನಿ-ಷತ್ತು
ಉಪ-ನಿ-ಷ-ತ್ತು-ಗಳ
ಉಪ-ನಿ-ಷ-ತ್ತು-ಗ-ಳ-ಲ್ಲೆಲ್ಲ
ಉಪ-ನಿ-ಷ-ತ್ತು-ಗಳು
ಉಪ-ಪಾಂ-ಡ-ವ-ರನ್ನು
ಉಪ-ಮಂ-ತ್ರಿನ್
ಉಪ-ಯುಕ್ತ
ಉಪ-ಯೋ-ಗ-ವಾ-ಗುವ
ಉಪ-ಯೋ-ಗಿ-ಸು-ವ-ವ-ನಿಗೇ
ಉಪ-ವನ
ಉಪ-ವ-ನಕ್ಕೆ
ಉಪ-ವ-ನವು
ಉಪ-ವಾಸ
ಉಪ-ವಾ-ಸ-ಗಳನ್ನು
ಉಪ-ವಾ-ಸ-ವನ್ನು
ಉಪ-ವಾ-ಸ-ವಿ-ದ್ದನು
ಉಪ-ವಾ-ಸ-ವಿದ್ದು
ಉಪ-ವಾ-ಸ-ವಿ-ರು-ವನು
ಉಪ-ವಾ-ಸ-ವ್ರತ
ಉಪ-ಶ-ಮ-ನ-ಕ್ಕಾಗಿ
ಉಪ-ಶ-ಮ-ಶೀ-ಲಾಯೋ
ಉಪ-ಶಿ-ಕ್ಷಿ-ತಾ-ತ್ಮನೇ
ಉಪ-ಸಂ-ಹಾರ
ಉಪ-ಹಾ-ರ-ವಿತ್ತು
ಉಪಾ
ಉಪಾ-ದಾ-ನ-ಕಾ-ರಣ
ಉಪಾ-ಧಿ-ಗಳು
ಉಪಾ-ಧಿಗೆ
ಉಪಾಯ
ಉಪಾ-ಯ-ಗ-ಳಿವೆ
ಉಪಾ-ಯ-ಗ-ಳಿ-ವೆ-ಯಾ-ದರೂ
ಉಪಾ-ಯ-ದಿಂದ
ಉಪಾ-ಯ-ಬೇ-ರನ್ನು
ಉಪಾ-ಯ-ವನ್ನು
ಉಪಾ-ಯ-ವನ್ನೂ
ಉಪಾ-ಯ-ವಾಗಿ
ಉಪಾ-ಯ-ವಿದೆ
ಉಪಾ-ಯ-ವಿ-ನ್ನಿಲ್ಲ
ಉಪಾ-ಯವೂ
ಉಪಾ-ಯ-ವೆಂದರೆ
ಉಪಾ-ಯ-ವೇನು
ಉಪಾ-ಯ-ವೇ-ನೆಂದು
ಉಪಾ-ಯ-ವೊಂ-ದನ್ನು
ಉಪಾ-ವಾ-ಸ-ದಿಂದ
ಉಪಾ-ಸ-ಕರು
ಉಪಾ-ಸ-ನಾ-ರೂ-ಪ-ವಾದ
ಉಪಾ-ಸನೆ
ಉಪಾ-ಸ-ನೆ-ಗಳ
ಉಪಾ-ಸ-ನೆ-ಯನ್ನು
ಉಪಾ-ಸ-ನೆ-ಯಿಂದ
ಉಪಾ-ಸಿತ
ಉಪಾಸ್ತೇ
ಉಪಾ-ಸ್ಯ-ದೈ-ವದ
ಉಪೇಕ್ಷೆ
ಉಪೇ-ಕ್ಷೆ-ಯಾ-ಗಲಿ
ಉಪ್ಪ-ರಿ-ಗೆ-ಗ-ಳಿಂ-ದಲೂ
ಉಪ್ಪ-ರಿ-ಗೆ-ಗಳು
ಉಪ್ಪ-ರಿ-ಗೆಯ
ಉಪ್ಪ-ರಿ-ಗೆ-ಯ-ನ್ನೇರಿ
ಉಪ್ಪಿಗೆ
ಉಪ್ಪಿನ
ಉಪ್ಪು
ಉಪ್ಪು-ನೀ-ರಿನ
ಉಬ್ಬಿ
ಉಬ್ಬಿದ
ಉಬ್ಬುತ್ತಾ
ಉಭ
ಉಭಯ
ಉಮಾ
ಉಮಾ-ಮ-ಹೇ-ಶ್ವ-ರ-ನನ್ನು
ಉರಸೇ
ಉರಿ
ಉರಿ-ದಷ್ಟು
ಉರಿದು
ಉರಿ-ದು-ಹೋಗ
ಉರಿಯ
ಉರಿ-ಯನ್ನು
ಉರಿ-ಯಿಂದ
ಉರಿ-ಯಿತು
ಉರಿ-ಯು-ತ್ತದೆ
ಉರಿ-ಯು-ತ್ತಿದ್ದ
ಉರಿ-ಯು-ತ್ತಿ-ರು-ತ್ತದೆ
ಉರಿ-ಯು-ತ್ತಿ-ರುವ
ಉರಿ-ಯು-ತ್ತಿವೆ
ಉರಿವ
ಉರು
ಉರು-ಬನ್ನು
ಉರು-ಳಾ-ಡ-ಬೇ-ಕೆಂಬ
ಉರುಳಿ
ಉರು-ಳಿತು
ಉರು-ಳಿ-ದನು
ಉರು-ಳಿ-ದಳು
ಉರು-ಳಿದ್ದ
ಉರು-ಳಿ-ಸ-ಬೇ-ಕೆಂದೆ
ಉರು-ಳಿ-ಸು-ವಿರಾ
ಉರು-ಳಿ-ಹೋ-ದವು
ಉರು-ಳುತ್ತಾ
ಉರು-ಳುವ
ಉಲ್ಬಣ
ಉಲ್ಬ-ಣ-ವಾ-ಯಿತು
ಉಲ್ಬ-ಣಿ-ಸಿತು
ಉಲ್ಲಾಸ
ಉಳಿ
ಉಳಿ-ಗಾ-ಲ-ವಿಲ್ಲ
ಉಳಿ-ಗಾ-ಲ-ವಿ-ಲ್ಲ-ವೆಂದು
ಉಳಿ-ಗಾ-ಲ-ವಿ-ಲ್ಲ-ವೆಂ-ದು-ಕೊಂಡ
ಉಳಿ-ಗಾ-ಲ-ವಿ-ಲ್ಲ-ವೆಂ-ಬು-ದನ್ನು
ಉಳಿ-ಗಾ-ಲ-ವುಂಟೆ
ಉಳಿದ
ಉಳಿ-ದದ್ದು
ಉಳಿ-ದ-ರುಂ
ಉಳಿ-ದರೂ
ಉಳಿ-ದವ
ಉಳಿ-ದ-ವರ
ಉಳಿ-ದ-ವ-ರನ್ನು
ಉಳಿ-ದ-ವ-ರಲ್ಲಿ
ಉಳಿ-ದ-ವ-ರಿ-ಗಾಗಿ
ಉಳಿ-ದ-ವ-ರಿಗೂ
ಉಳಿ-ದ-ವ-ರಿ-ಗೆಲ್ಲ
ಉಳಿ-ದ-ವರು
ಉಳಿ-ದ-ವರೂ
ಉಳಿ-ದ-ವರೆಲ್ಲ
ಉಳಿ-ದಾ-ವು-ದನ್ನೂ
ಉಳಿ-ದಿತ್ತು
ಉಳಿ-ದಿದೆ
ಉಳಿ-ದಿದ್ದ
ಉಳಿ-ದಿ-ದ್ದ-ವಾ-ದರೂ
ಉಳಿ-ದಿ-ಬ್ಬರು
ಉಳಿ-ದಿ-ರುವ
ಉಳಿ-ದಿ-ರು-ವು-ದ-ರಿಂದ
ಉಳಿ-ದಿ-ರು-ವುದು
ಉಳಿದು
ಉಳಿ-ದು-ಕೊಂಡ
ಉಳಿ-ದು-ಕೊಂ-ಡನು
ಉಳಿ-ದು-ಕೊಂಡು
ಉಳಿ-ದು-ಕೊ-ಳ್ಳ-ಬ-ಹುದು
ಉಳಿ-ದು-ದನ್ನು
ಉಳಿ-ದು-ದ-ನ್ನೆಲ್ಲ
ಉಳಿ-ದುದು
ಉಳಿ-ದು-ದೆಲ್ಲ
ಉಳಿ-ದು-ವನ್ನು
ಉಳಿ-ದೆ-ರಡು
ಉಳಿ-ದೆಲ್ಲ
ಉಳಿ-ಯಂತೆ
ಉಳಿ-ಯ-ದಂತೆ
ಉಳಿ-ಯ-ಬೇಕು
ಉಳಿ-ಯ-ಲಾರೆ
ಉಳಿ-ಯ-ಲಿಲ್ಲ
ಉಳಿ-ಯು-ವಂತೆ
ಉಳಿ-ಯು-ವು-ದಿಲ್ಲ
ಉಳಿ-ಯು-ವುದು
ಉಳಿ-ಯು-ವುವು
ಉಳಿಸ
ಉಳಿ-ಸಣ್ಣ
ಉಳಿ-ಸ-ಬೇ-ಕೆಂ-ದು-ಕೊಂಡು
ಉಳಿ-ಸ-ಲಾ-ರದೆ
ಉಳಿ-ಸಲು
ಉಳಿ-ಸಿ-ಕೊಂಡು
ಉಳಿ-ಸಿ-ಕೊ-ಳ್ಳು-ವು-ದ-ಕ್ಕಾಗಿ
ಉಳಿ-ಸಿ-ಕೊ-ಳ್ಳು-ವು-ದೆಂದು
ಉಳಿ-ಸಿ-ದ್ದೇನೆ
ಉಳಿಸು
ಉಳಿ-ಸು-ತ್ತೇನೆ
ಉಳಿ-ಸು-ವು-ದ-ಕ್ಕಾಗಿ
ಉಳ್ಳ
ಉಶನ
ಉಷಾ
ಉಷೆ
ಉಷೆಗೆ
ಉಷೆಯ
ಉಷೆ-ಯನ್ನೂ
ಉಷೆಯು
ಉಷ್ಣ
ಉಸಿ
ಉಸಿ-ರನ್ನು
ಉಸಿ-ರ-ನ್ನೆ-ಳೆದು
ಉಸಿ-ರಾ-ಟ-ದಿಂದ
ಉಸಿ-ರಾ-ಟ-ವನ್ನು
ಉಸಿ-ರಾ-ಡಲು
ಉಸಿ-ರಾ-ಡುವ
ಉಸಿ-ರಾ-ಡು-ವುದನ್ನು
ಉಸಿ-ರಾ-ಡು-ವುದೇ
ಉಸಿ-ರಿಗೆ
ಉಸಿ-ರಿ-ನಿಂದ
ಉಸಿರು
ಉಸಿ-ರು-ಕ-ಟ್ಟಿ-ಹೋ-ದಂ-ತಾ-ಯಿತು
ಉಸು-ರಿ-ದರು
ಉಹು
ಊಊ
ಊಟ
ಊಟ-ಕ್ಕಿ-ಡುವ
ಊಟಕ್ಕೆ
ಊಟ-ಕ್ಕೆಂದು
ಊಟ-ಕ್ಕೇನೂ
ಊಟದ
ಊಟ-ದಲ್ಲಿ
ಊಟ-ಮಾ-ಡ-ದಿ-ದ್ದರೂ
ಊಟ-ಮಾಡಿ
ಊಟ-ಮಾ-ಡಿ-ಕೊಂಡು
ಊಟ-ಮಾ-ಡಿ-ಸಿರಿ
ಊಟ-ಮಾಡು
ಊಟ-ಮಾ-ಡು-ತ್ತಿ-ದ್ದ-ವರು
ಊಟ-ಮಾ-ಡು-ವನು
ಊಟ-ಮಾ-ಡು-ವಾಗ
ಊಟ-ಮಾ-ಡು-ವಾ-ಗ-ಜೊ-ತೆ-ಯಲ್ಲಿ
ಊಟ-ಮಾ-ಡು-ವುದು
ಊಟ-ವನ್ನು
ಊಟ-ವಾ-ಗು-ತ್ತಲೆ
ಊಟ-ವಾ-ದ-ಮೇಲೆ
ಊತಿ
ಊದಿ-ಕೊಂಡು
ಊದಿ-ದನು
ಊರ
ಊರ-ಜ-ನ-ಕ್ಕೆಲ್ಲ
ಊರನ್ನು
ಊರನ್ನೆ
ಊರ-ನ್ನೆಲ್ಲಾ
ಊರ-ಲ್ಲಿದ್ದ
ಊರ-ಲ್ಲಿ-ಲ್ಲ-ದಿ-ರು-ವಾಗ
ಊರ-ಲ್ಲೆಲ್ಲ
ಊರಿಗೆ
ಊರಿ-ಗೊಂದು
ಊರಿನ
ಊರಿ-ನಲ್ಲಿ
ಊರಿ-ನ-ಲ್ಲೆಲ್ಲ
ಊರಿ-ನ-ವ-ರಿಗೆ
ಊರಿ-ನ-ವರು
ಊರು
ಊರು-ಗಳನ್ನು
ಊರು-ಗ-ಳ-ಲ್ಲಿಯೂ
ಊರು-ಗ-ಳಲ್ಲೊ
ಊರು-ಗ-ಳಿಗೆ
ಊರು-ಗ-ಳೆಲ್ಲ
ಊರು-ಗೋಲು
ಊರು-ಬಾ-ಗಿ-ಲನ್ನು
ಊರು-ಭ್ಯಾ-ನ್ನಮಃ
ಊರೂರು
ಊರೆ-ಗೋ-ಲಾಗಿ
ಊರೆ-ಗೋಲು
ಊರೊ-ಳಕ್ಕೆ
ಊರೊ-ಳಗೆ
ಊರ್ಜ-ಸ್ವ-ತಿ-ಯನ್ನು
ಊರ್ಧ್ವ
ಊರ್ಧ್ವಗ
ಊರ್ಧ್ವಾ-ಯನ
ಊರ್ವಶಿ
ಊರ್ವ-ಶಿಗೆ
ಊರ್ವ-ಶಿಯ
ಊರ್ವ-ಶಿ-ಯನ್ನು
ಊರ್ವ-ಶಿ-ಯಲ್ಲಿ
ಊರ್ವ-ಶಿ-ಯಿಂದ
ಊರ್ವ-ಶಿ-ಯಿ-ಲ್ಲದೆ
ಊರ್ವ-ಶಿ-ಯೆಂದು
ಊರ್ವಶೀ
ಊಳಿ-ಗ-ವನ್ನು
ಊಳಿ-ಟ್ಟವು
ಊಹಿಸ
ಊಹಿ-ಸ-ಬೇ-ಕಾ-ಗಿದೆ
ಊಹಿ-ಸಿರು
ಊಹಿ-ಸಿ-ರು-ವಂತೆ
ಊಹೆ
ಋಣ-ಗ-ಳ-ನ್ನು-ದೇ-ವ-ಋಣ
ಋಷ-ಭ-ದೇವ
ಋಷ-ಭ-ದೇ-ವನ
ಋಷ-ಭ-ದೇ-ವ-ನಿಗೆ
ಋಷ-ಭ-ದೇ-ವನು
ಋಷ-ಭನ
ಋಷ-ಭ-ನೆಂದು
ಋಷ-ಭನೇ
ಋಷ-ಭ-ಯೋ-ಗಿಗೆ
ಋಷ-ಭ-ಯೋ-ಗಿಯ
ಋಷ-ಭ-ಯೋ-ಗಿಯು
ಋಷ-ಭಾಯ
ಋಷಿ
ಋಷಿ-ಋಣ
ಋಷಿ-ಕು-ಮಾ-ರ-ನಂತೆ
ಋಷಿ-ಗಳ
ಋಷಿ-ಗ-ಳಾಗಿ
ಋಷಿ-ಗ-ಳಿಗೆ
ಋಷಿ-ಗ-ಳಿ-ಗೆಲ್ಲ
ಋಷಿ-ಗಳು
ಋಷಿ-ಗಳೂ
ಋಷಿ-ಗಳೆ
ಋಷಿ-ಗ-ಳೆಲ್ಲ
ಋಷಿ-ಗ-ಳೆ-ಲ್ಲರ
ಋಷಿ-ಗ-ಳೆ-ಲ್ಲರೂ
ಋಷಿ-ಗ-ಳೊ-ಡನೆ
ಋಷಿಗೆ
ಋಷಿ-ತಿ-ಲಕ
ಋಷಿ-ಪುಂ-ಗ-ವನೂ
ಋಷಿಯ
ಋಷಿ-ಯನ್ನು
ಋಷಿಯು
ಎಂಜಲ
ಎಂಜ-ಲು-ಮಾ-ಡಿ-ದಾಗ
ಎಂಜಿನ್
ಎಂಟ-ನೂ-ರ-ರ-ಷ್ಟಿತ್ತು
ಎಂಟ-ನೆಯ
ಎಂಟ-ನೆ-ಯದು
ಎಂಟ-ನೆ-ಯ-ವನು
ಎಂಟು
ಎಂತ
ಎಂತ-ವ-ಹ-ರಿಗೂ
ಎಂತಹ
ಎಂತ-ಹ-ದೆಂ-ಬುದು
ಎಂತ-ಹವ
ಎಂತ-ಹ-ವ-ರಿಗೂ
ಎಂತ-ಹುದು
ಎಂತು
ಎಂತೆಂ-ತಹ
ಎಂತೋ
ಎಂಥ
ಎಂದ
ಎಂದನು
ಎಂದ-ಮೇಲೆ
ಎಂದ-ರಂತೆ
ಎಂದರು
ಎಂದರೂ
ಎಂದರೆ
ಎಂದಳು
ಎಂದ-ಳುತ್ತಾ
ಎಂದ-ವರು
ಎಂದ-ಹಾಗೆ
ಎಂದಾ-ದರೂ
ಎಂದಾ-ಯಿತು
ಎಂದಿ
ಎಂದಿಗೂ
ಎಂದಿ-ದ್ದರೂ
ಎಂದಿ-ನಂತೆ
ಎಂದು
ಎಂದು-ಕೊಂಡ
ಎಂದು-ಕೊಂ-ಡ-ನಾ-ದರೂ
ಎಂದು-ಕೊಂ-ಡನು
ಎಂದು-ಕೊಂ-ಡರು
ಎಂದು-ಕೊಂ-ಡಳು
ಎಂದು-ಕೊಂ-ಡ-ವನೆ
ಎಂದು-ಕೊಂ-ಡಾಗ
ಎಂದು-ಕೊಂ-ಡಿರೊ
ಎಂದು-ಕೊಂಡು
ಎಂದು-ಕೊಂಡೆ
ಎಂದು-ಕೊ-ಳ್ಳುತ್ತಾ
ಎಂದು-ಕೊ-ಳ್ಳು-ತ್ತಿ-ದ್ದನು
ಎಂದು-ಕೊ-ಳ್ಳು-ವುದು
ಎಂದು-ದ-ರಿಂದ
ಎಂದೂ
ಎಂದೆ
ಎಂದೆಂ-ದಿಗೂ
ಎಂದೆಂದೂ
ಎಂದೇ
ಎಂಬ
ಎಂಬಂತೆ
ಎಂಬ-ತ್ತೆಂಟು
ಎಂಬ-ತ್ತೊಂದು
ಎಂಬರ್ಥ
ಎಂಬ-ರ್ಥದ
ಎಂಬಾ-ತ-ನಿಗೆ
ಎಂಬು
ಎಂಬು-ದಕ್ಕೆ
ಎಂಬು-ದನ್ನು
ಎಂಬು-ದಾಗಿ
ಎಂಬುದು
ಎಂಬುದೂ
ಎಂಬು-ವನೆ
ಎಂಬು-ವ-ಳನ್ನು
ಎಂಬು-ವ-ಳನ್ನೂ
ಎಂಬು-ವ-ಳಲ್ಲಿ
ಎಂಬು-ವಳು
ಎಂಬುವು
ಎಕೆ
ಎಗ್ಗ-ನಂ-ತಾ-ಗು-ತ್ತದೆ
ಎಗ್ಗ-ನಂ-ತಾ-ದಾರು
ಎಗ್ಗ-ನಂತೆ
ಎಚ್ಚರ
ಎಚ್ಚ-ರ-ಗೊ-ಳ್ಳು-ವಂತೆ
ಎಚ್ಚ-ರ-ಗೊ-ಳ್ಳು-ವ-ಷ್ಟ-ರಲ್ಲಿ
ಎಚ್ಚ-ರ-ದಪ್ಪಿ
ಎಚ್ಚ-ರ-ದಿಂದ
ಎಚ್ಚ-ರ-ದಿಂ-ದಿ-ದ್ದ-ವನು
ಎಚ್ಚ-ರ-ದಿಂ-ದಿ-ರ-ಬೇಕು
ಎಚ್ಚ-ರ-ದಿಂ-ದಿ-ರ-ಬೇ-ಕೆಂದು
ಎಚ್ಚ-ರ-ವ-ಹಿಸಿ
ಎಚ್ಚ-ರ-ವಾ-ದರೂ
ಎಚ್ಚ-ರ-ವಿ-ದ್ದಾಗ
ಎಚ್ಚ-ರಿಕೆ
ಎಚ್ಚ-ರಿ-ಕೆ-ಯಿಂದ
ಎಚ್ಚ-ರಿ-ಕೆ-ಯಿ-ರಲಿ
ಎಚ್ಚ-ರಿಸಿ
ಎಚ್ಚೆ
ಎಚ್ಚೆತ್ತು
ಎಟು-ಕದ
ಎಟು-ಕ-ಲಾ-ರದ
ಎಡ
ಎಡಕ್ಕೆ
ಎಡ-ಕ್ಕೆ-ಹೀಗೆ
ಎಡ-ಗಣ್ಣು
ಎಡ-ಗ-ಣ್ಣು-ಗಳು
ಎಡ-ಗಾ-ಲ-ನ್ನಿಟ್ಟು
ಎಡ-ಗೈಗೆ
ಎಡ-ಗೈ-ಯ-ನ್ನೂರಿ
ಎಡ-ಗೈ-ಯಲ್ಲಿ
ಎಡ-ಗೈ-ಯಿಂದ
ಎಡ-ಗೈಲಿ
ಎಡ-ತೊ-ಡೆಯ
ಎಡ-ತೋ-ಳನ್ನು
ಎಡ-ಬ-ಲದ
ಎಡ-ಭುಜ
ಎಡ-ಭು-ಜ-ಗಳು
ಎಡ-ವು-ತ್ತೇ-ನೆಂಬ
ಎಡೆ
ಎಡೆ-ಬಿ-ಡದೆ
ಎಡೆ-ಯ-ನ್ನೆಲ್ಲ
ಎಡೆ-ಯಲ್ಲಿ
ಎಡೆ-ಯೆಲ್ಲಿ
ಎಣಿಸ
ಎಣಿ-ಸ-ಬ-ಹುದು
ಎಣಿ-ಸು-ವುದು
ಎಣೆ
ಎಣೆ-ಯಿಲ್ಲ
ಎಣೆ-ಯಿ-ಲ್ಲದ
ಎಣೆ-ಯಿ-ಲ್ಲ-ದ-ವ-ಳೆಂದು
ಎಣ್ಣೆ
ಎಣ್ಣೆ-ಗಾ-ಣದೆ
ಎಣ್ಣೆಯ
ಎಣ್ಣೆ-ಯಲ್ಲಿ
ಎಣ್ಣೆ-ಯೊತ್ತಿ
ಎತ್ತ
ಎತ್ತ-ಕಡೆ
ಎತ್ತ-ಕೂ-ಡದು
ಎತ್ತ-ದಿ-ರು-ವು-ದ-ರಿಂದ
ಎತ್ತನ್ನು
ಎತ್ತ-ಬೇ-ಕಾ-ಯಿತು
ಎತ್ತ-ಬೇಕೋ
ಎತ್ತರ
ಎತ್ತ-ರ-ದಲ್ಲಿ
ಎತ್ತ-ರ-ದ-ವ-ರೆಗೆ
ಎತ್ತ-ರ-ವಾಗಿ
ಎತ್ತ-ರ-ವಾ-ಗಿದೆ
ಎತ್ತ-ರ-ವಾ-ಗಿದ್ದ
ಎತ್ತ-ರ-ವಾ-ಗಿ-ರು-ವುದ
ಎತ್ತ-ರ-ವಾದ
ಎತ್ತಲಿ
ಎತ್ತಿ
ಎತ್ತಿ-ಕ-ಟ್ಟಿದ
ಎತ್ತಿ-ಕೊಂ-ಡನು
ಎತ್ತಿ-ಕೊಂ-ಡರು
ಎತ್ತಿ-ಕೊಂ-ಡ-ವನೆ
ಎತ್ತಿ-ಕೊಂಡು
ಎತ್ತಿ-ಕೊಂ-ಡು-ಬಂದು
ಎತ್ತಿ-ಕೊ-ಳ್ಳ-ಬೇ-ಕೆಂದು
ಎತ್ತಿಗೂ
ಎತ್ತಿ-ತಂದು
ಎತ್ತಿದ
ಎತ್ತಿ-ದ-ನಂತೆ
ಎತ್ತಿ-ದನು
ಎತ್ತಿ-ದ-ನೆಂದು
ಎತ್ತಿ-ದು-ದಕ್ಕೆ
ಎತ್ತಿ-ದೊ-ಡ-ನೆಯೆ
ಎತ್ತಿ-ದ್ದೆ-ಯಂತೆ
ಎತ್ತಿನ
ಎತ್ತಿ-ಹಾ-ಕ-ಬೇ-ಕೆಂದು
ಎತ್ತಿ-ಹಾಕಿ
ಎತ್ತಿ-ಹಾ-ಕಿ-ದರು
ಎತ್ತಿ-ಹಾ-ಕಿ-ದರೆ
ಎತ್ತಿ-ಹಾ-ಕಿ-ದಳು
ಎತ್ತುತ್ತಿ
ಎತ್ತುವ
ಎತ್ತು-ವುದು
ಎದು-ರಾ-ದನು
ಎದು-ರಾ-ದರು
ಎದು-ರಾಳಿ
ಎದು-ರಾ-ಳಿ-ಯನ್ನು
ಎದು-ರಾ-ಳಿ-ಯಾದ
ಎದು-ರಿಗೆ
ಎದು-ರಿ-ಲ್ಲ-ದಂ-ತಾ-ಯಿತು
ಎದು-ರಿಸಿ
ಎದುರು
ಎದು-ರುಂಟೆ
ಎದು-ರು-ತ್ತರ
ಎದು-ರು-ತ್ತ-ರ-ವಿತ್ತು
ಎದೆ
ಎದೆ-ಇ-ವು-ಗಳನ್ನು
ಎದೆ-ಗ-ಪ್ಪಿ-ಕೊಂಡು
ಎದೆ-ಗ-ಳೇನು
ಎದೆ-ಗ-ವ-ಚಿ-ಕೊಂಡು
ಎದೆ-ಗುಂ-ದ-ಲಿಲ್ಲ
ಎದೆಗೆ
ಎದೆ-ಗೊ-ರ-ಗಿ-ಸಿ-ಕೊಂಡ
ಎದೆಯ
ಎದೆ-ಯನ್ನು
ಎದೆ-ಯ-ಮೇಲೆ
ಎದೆ-ಯಲ್ಲಿ
ಎದ್ದ
ಎದ್ದ-ವರೆ
ಎದ್ದಿತು
ಎದ್ದು
ಎದ್ದು-ಬಂ-ದಿತು
ಎದ್ದು-ಹೋ-ದರು
ಎನಿ-ಸಿ-ಕೊಂ-ಡಿ-ದ್ದರು
ಎನಿ-ಸಿತು
ಎನಿ-ಸಿ-ದನು
ಎನಿ-ಸಿ-ದರೆ
ಎನಿ-ಸು-ಕೊ-ಳ್ಳು-ತ್ತವೆ
ಎನ್ನಿ-ಸು-ತ್ತದೆ
ಎನ್ನು
ಎನ್ನು-ತ್ತಾನೆ
ಎನ್ನು-ತ್ತಾರೆ
ಎನ್ನು-ತ್ತಿ-ದ್ದನು
ಎನ್ನು-ವಂ-ತಹ
ಎನ್ನು-ವಂ-ತಾ-ಯಿತು
ಎನ್ನು-ವಂತೆ
ಎನ್ನು-ವನು
ಎನ್ನು-ವರು
ಎನ್ನು-ವುದನ್ನು
ಎನ್ನು-ವು-ದಿಲ್ಲ
ಎನ್ನು-ವುದು
ಎಪ್ಪ-ತ್ತೈದು
ಎಬ್ಬಿ-ಸ-ಬ-ಹುದು
ಎಬ್ಬಿ-ಸಲು
ಎಬ್ಬಿಸಿ
ಎಬ್ಬಿ-ಸು-ವುದು
ಎರಗಿ
ಎರ-ಗಿದ
ಎರ-ಗಿ-ದನು
ಎರ-ಗುವ
ಎರ-ಗು-ವಂತೆ
ಎರಚಿ
ಎರ-ಚಿ-ಕೊಳ್ಳು
ಎರ-ಚಿದ
ಎರ-ಚಿ-ದ-ನೆಂದು
ಎರ-ಚಿ-ದರು
ಎರ-ಚಿ-ದ್ದಾರೆ
ಎರ-ಚುತ್ತಾ
ಎರಡ
ಎರ-ಡ-ಕ್ಷ-ರ-ಗಳನ್ನು
ಎರ-ಡ-ನೆಯ
ಎರ-ಡ-ನೆ-ಯ-ದಾಗಿ
ಎರ-ಡ-ನೆ-ಯದು
ಎರ-ಡ-ನೆ-ಯ-ವ-ಳಾದ
ಎರ-ಡ-ರಷ್ಟು
ಎರ-ಡ-ರಿ-ಯದ
ಎರ-ಡಲ್ಲ
ಎರ-ಡಾಗಿ
ಎರ-ಡಾ-ಗಿ-ದ್ದಾನೆ
ಎರ-ಡಿ-ಲ್ಲದ
ಎರಡು
ಎರ-ಡು-ಕಾ-ಲಿನ
ಎರ-ಡು-ಗ-ಳಿ-ಗೆ-ಗಳ
ಎರ-ಡು-ಯಮ
ಎರಡೂ
ಎರಡೆ
ಎರ-ಡೆ-ರಡು
ಎರೆ-ದನು
ಎಲ
ಎಲ-ಬು-ಗಳೆ
ಎಲವೊ
ಎಲಾ
ಎಲುಬು
ಎಲು-ಬು-ಗಳನ್ನೆಲ್ಲ
ಎಲು-ಬು-ಗಳು
ಎಲೆ
ಎಲೆ-ಮ-ನೆ-ಯನ್ನು
ಎಲೆಯ
ಎಲೆ-ವ-ನೆಗೆ
ಎಲೈ
ಎಲೊ
ಎಲೋ
ಎಲ್ಲ
ಎಲ್ಲ-ಕ್ಕಿಂತ
ಎಲ್ಲಕ್ಕೂ
ಎಲ್ಲ-ದ-ರ-ಲ್ಲಿಯೂ
ಎಲ್ಲಯ್ಯ
ಎಲ್ಲರ
ಎಲ್ಲ-ರನ್ನೂ
ಎಲ್ಲ-ರ-ಲ್ಲಿಯೂ
ಎಲ್ಲ-ರಿಂ-ದಲೂ
ಎಲ್ಲ-ರಿ-ಗಿಂತ
ಎಲ್ಲ-ರಿ-ಗಿಂ-ತಲೂ
ಎಲ್ಲ-ರಿಗೂ
ಎಲ್ಲರೂ
ಎಲ್ಲ-ರೆ-ದು-ರಿಗೆ
ಎಲ್ಲ-ರೊ-ಡನೆ
ಎಲ್ಲ-ವನ್ನೂ
ಎಲ್ಲವೂ
ಎಲ್ಲಿ
ಎಲ್ಲಿಂದ
ಎಲ್ಲಿಂ-ದಲೋ
ಎಲ್ಲಿಗೆ
ಎಲ್ಲಿ-ದ್ದರೂ
ಎಲ್ಲಿ-ದ್ದ-ವ-ರ-ಲ್ಲಿಯೆ
ಎಲ್ಲಿ-ದ್ದ-ವ-ರಲ್ಲೆ
ಎಲ್ಲಿ-ದ್ದಾ-ನೆಯೋ
ಎಲ್ಲಿದ್ದಿ
ಎಲ್ಲಿ-ನೋ-ಡಿ-ದರೂ
ಎಲ್ಲಿಯ
ಎಲ್ಲಿ-ಯದು
ಎಲ್ಲಿ-ಯ-ವ-ರೆಗೆ
ಎಲ್ಲಿ-ಯಾ-ದರೂ
ಎಲ್ಲಿಯೂ
ಎಲ್ಲಿಯೇ
ಎಲ್ಲಿಯೊ
ಎಲ್ಲಿಯೋ
ಎಲ್ಲಿ-ಯೋ-ಸ-ಕಲ
ಎಲ್ಲಿ-ರಲಿ
ಎಲ್ಲಿ-ರು-ವ-ನೆಂದು
ಎಲ್ಲಿ-ರು-ವ-ರೆಂದು
ಎಲ್ಲಿ-ರು-ವು-ದೆಂದು
ಎಲ್ಲಿ-ರುವೆ
ಎಲ್ಲಿ-ರು-ವೆ-ನೆಂ-ಬು-ದರ
ಎಲ್ಲಿ-ಲ್ಲದ
ಎಲ್ಲಿ-ಹೋ-ದೆ-ಯಪ್ಪ
ಎಲ್ಲೆ
ಎಲ್ಲೆ-ಡೆ-ಯ-ಲ್ಲಿಯೂ
ಎಲ್ಲೆ-ಯಿ-ಲ್ಲದ
ಎಲ್ಲೆಲ್ಲಿ
ಎಲ್ಲೆ-ಲ್ಲಿಯೂ
ಎಲ್ಲೆ-ಲ್ಲಿಯೋ
ಎಲ್ಲೋ
ಎಳ-ಸಾದ
ಎಳ-ಸಿ-ದ-ವರು
ಎಳೆ
ಎಳೆ-ತಂ-ದಳು
ಎಳೆ-ತಂ-ದವು
ಎಳೆ-ತಂದು
ಎಳೆ-ತ-ನ-ವನ್ನು
ಎಳೆದ
ಎಳೆ-ದತ್ತ
ಎಳೆ-ದನು
ಎಳೆ-ದಾ-ಡಿ-ದನು
ಎಳೆ-ದಾ-ಡು-ವಂತೆ
ಎಳೆ-ದಾ-ಡು-ವುದು
ಎಳೆದು
ಎಳೆ-ದು-ಕೊಂ-ಡಂ-ತಾ-ಯಿತು
ಎಳೆ-ದು-ಕೊಂ-ಡಿತು
ಎಳೆ-ದು-ಕೊಂಡು
ಎಳೆ-ದು-ಕೊಂ-ಡು-ಹೋಗಿ
ಎಳೆ-ದು-ಕೊಂ-ಡು-ಹೋ-ದನು
ಎಳೆ-ದು-ಕೊ-ಳ್ಳುತ್ತಾ
ಎಳೆ-ದೆ-ಳೆದು
ಎಳೆದೊ
ಎಳೆ-ದೊಯ್ದು
ಎಳೆ-ದೊ-ಯ್ಯ-ಲೆಂದು
ಎಳೆ-ದೊಯ್ಯು
ಎಳೆ-ದೊ-ಯ್ಯು-ವಂತೆ
ಎಳೆ-ದೊ-ಯ್ಯು-ವರು
ಎಳೆ-ನ-ಗೆ-ಆ-ತನು
ಎಳೆ-ನಾ-ಗ-ರ-ದಂತೆ
ಎಳೆಯ
ಎಳೆ-ಯ-ಬೇ-ಕಾ-ದರೆ
ಎಳೆ-ಯ-ಹು-ಡು-ಗನ
ಎಳೆ-ಯಿತು
ಎಳೆಯು
ಎಳೆ-ಯು-ತ್ತದೆ
ಎಳೆ-ಯುತ್ತಾ
ಎಳೆ-ಯು-ವಂತೆ
ಎಳೆ-ಯು-ವು-ದೆಂ-ದ-ರೇನು
ಎಳ್ಳಷ್ಟೂ
ಎವೆ-ಯಲ್ಲಿ
ಎಷ್ಟಾ-ದರೂ
ಎಷ್ಟಿದೆ
ಎಷ್ಟು
ಎಷ್ಟು-ಹೊತ್ತು
ಎಷ್ಟೆಂ-ಬುದು
ಎಷ್ಟೊ
ಎಷ್ಟೊಂದು
ಎಷ್ಟೋ
ಎಸ-ಗಿ-ರುವ
ಎಸ-ಳಾದ
ಎಸೆದ
ಎಸೆ-ದಂ-ತಾ-ಯಿತು
ಎಸೆ-ದಂತೆ
ಎಸೆ-ದನು
ಎಸೆ-ದಳು
ಎಸೆ-ದು-ದು-ಇ-ತ್ಯಾದಿ
ಎಸೆ-ಯಲು
ಎಸೆ-ಯು-ವರು
ಏ
ಏಕ
ಏಕ-ಕಾಲ
ಏಕ-ಕಾ-ಲಕ್ಕೆ
ಏಕ-ಕಾ-ಲ-ದಲ್ಲಿ
ಏಕ-ಕಾ-ಲ-ದ-ಲ್ಲಿಯೆ
ಏಕ-ನಾದ
ಏಕ-ಪ-ತ್ನೀ-ವ್ರ-ತ-ಸ್ಥ-ನಾ-ಗಿದ್ದು
ಏಕ-ರೂಪ
ಏಕಲ
ಏಕ-ವಾ-ಗಿದ್ದ
ಏಕವೇ
ಏಕಾಂ-ಗಿ-ಯಾಗಿ
ಏಕಾಂ-ಗಿ-ಯಾ-ಗಿದ್ದ
ಏಕಾಂತ
ಏಕಾಂ-ತ-ದಲ್ಲಿ
ಏಕಾಂ-ತ-ಭಕ್ತ
ಏಕಾಂ-ತ-ಭ-ಕ್ತಿ-ಯಿಂದ
ಏಕಾಂ-ತ-ಭಕ್ತೆ
ಏಕಾಂ-ತ-ಮ-ದ್ವಯಂ
ಏಕಾಂ-ತ-ವನ್ನು
ಏಕಾಂ-ತ-ವಾಸ
ಏಕಾಂ-ತ-ವಾ-ಸ-ವನ್ನು
ಏಕಾಕಿ
ಏಕಾ-ಕಿ-ಯಾ-ಗಿ-ರು-ತ್ತೇನೆ
ಏಕಾಗ್ರ
ಏಕಾ-ಗ್ರತೆ
ಏಕಾ-ಗ್ರ-ತೆ-ಯನ್ನು
ಏಕಾ-ಗ್ರ-ಮ-ನ-ಸ್ಕನೂ
ಏಕಾ-ಗ್ರ-ಮ-ನ-ಸ್ಸಿ-ನಿಂದ
ಏಕಾ-ಗ್ರ-ವಾ-ಗ-ಬೇಕು
ಏಕಾ-ಗ್ರ-ವಾ-ಗಿ-ರ-ಬೇಕು
ಏಕಾ-ಗ್ರ-ವಾ-ಗುತ್ತ
ಏಕಾ-ದಶ
ಏಕಾ-ದ-ಶ-ರು-ದ್ರರು
ಏಕಾ-ದ-ಶಿಯ
ಏಕಾ-ದ-ಶೇಂ-ದ್ರಿಯ
ಏಕಿ-ರ-ಬೇಕು
ಏಕೆ
ಏಕೆಂ-ದರೆ
ಏಕೆಂ-ದ-ರೆ-ಗೀ-ತೆಯು
ಏಕೈಕ
ಏಕೋ
ಏಕೋ-ಽವತು
ಏಟಿಗೆ
ಏಟು
ಏಣಿ-ಯಲ್ಲಿ
ಏತಕ್ಕೆ
ಏನನ್ನು
ಏನನ್ನೂ
ಏನ-ನ್ಯಾಯ
ಏನಪ್ಪ
ಏನ-ಪ್ಪಣೆ
ಏನಪ್ಪಾ
ಏನಯ್ಯ
ಏನಯ್ಯಾ
ಏನರ್ಥ
ಏನಾ
ಏನಾ-ಗಿ-ದೆಯೊ
ಏನಾ-ಗು-ವು-ದೆಂ-ಬು-ದನ್ನು
ಏನಾ-ಗು-ವು-ದೆಂ-ಬು-ದನ್ನೂ
ಏನಾ-ದರೂ
ಏನಾ-ದ-ರೇನು
ಏನಾ-ಯಿತೊ
ಏನಾ-ಶ್ಚರ್ಯ
ಏನಿದೆ
ಏನಿ-ದೆ-ಯೆಂದು
ಏನಿ-ದೆ-ಯೆಂ-ಬುದು
ಏನಿ-ದೆಲ್ಲ
ಏನು
ಏನು-ಮಾ-ಡ-ಬೇಕು
ಏನು-ಮಾ-ಡಲಿ
ಏನೂ
ಏನೆಂ-ದರೂ
ಏನೆಂ-ದರೆ
ಏನೆಂದು
ಏನೆಂ-ದು-ಕೊಂ-ಡಿ-ರು-ವರೋ
ಏನೆಂ-ಬು-ದನ್ನು
ಏನೆಂ-ಬುದು
ಏನೇನು
ಏನೇನೋ
ಏನೊ
ಏನೋ
ಏರ-ಬೇ-ಕೆಂ-ದರೆ
ಏರ-ಬೇ-ಕೆಂಬ
ಏರ-ಲಿಲ್ಲ
ಏರ-ಲೆಂದು
ಏರಿ
ಏರಿ-ದ-ನಾ-ದರೂ
ಏರಿ-ದನು
ಏರಿ-ದ-ಮೇಲೆ
ಏರಿ-ದ-ವನೆ
ಏರಿ-ದಾಗ
ಏರಿ-ಬಂ-ದರು
ಏರಿ-ಬಂದು
ಏರಿ-ಸಿದ
ಏರಿ-ಸಿ-ದನು
ಏರಿ-ಹೋ-ದನು
ಏರು-ತ್ತಲೇ
ಏರು-ಪೇ-ರನ್ನು
ಏರು-ಪೇ-ರು-ಗ-ಳುಂ-ಟಾಗು
ಏರುವ
ಏರು-ವಂ-ತಿಲ್ಲ
ಏರು-ವಂತೆ
ಏರು-ವುದು
ಏರ್ಪಟ್ಟು
ಏರ್ಪ-ಡಿ-ಸಿ-ದನು
ಏರ್ಪ-ಡಿ-ಸಿ-ದರು
ಏರ್ಪ-ಡಿ-ಸಿ-ದ-ವನೆ
ಏರ್ಪ-ಡಿ-ಸಿದ್ದ
ಏರ್ಪ-ಡಿ-ಸಿ-ದ್ದರು
ಏರ್ಪ-ಡಿ-ಸಿ-ರುವ
ಏರ್ಪ-ಡಿ-ಸು-ವಂತೆ
ಏಳ-ನೆಯ
ಏಳ-ನೆ-ಯ-ಸಾರಿ
ಏಳ-ಬ-ಹುದು
ಏಳ-ಬಾ-ರದು
ಏಳಿ-ಗೆಗೆ
ಏಳಿ-ಗೆಯು
ಏಳು
ಏಳು-ತ್ತದೆ
ಏಳು-ತ್ತಿ-ದ್ದಾಳೆ
ಏಳು-ತ್ತಿ-ವೆಯೊ
ಏಳು-ದಿ-ನ-ಗ-ಳ-ವ-ರೆಗೆ
ಏಳುವ
ಏಳು-ವಂ-ತಿಲ್ಲ
ಏಳು-ವಂತೆ
ಏಳು-ವ-ರ್ಷ-ಗಳನ್ನು
ಏಳು-ವ-ವ-ರೆಗೂ
ಏಳೂ-ವರೆ
ಏವ
ಏಸು-ಕ್ರಿ-ಸ್ತ-ನನ್ನು
ಏಸು-ನೀನು
ಐಕ್ಯ-ನಾ-ದನು
ಐಕ್ಯ-ಳಾ-ದಳು
ಐಕ್ಯ-ವಾ-ಯಿತು
ಐಕ್ಯವೇ
ಐತಿ-ಹಾ-ಸಿಕ
ಐದ-ನೆಯ
ಐದ-ನೆ-ಯ-ದಾದ
ಐದ-ನೆ-ಯದು
ಐದಾರು
ಐದು
ಐದೂ
ಐನೂರು
ಐರಾ-ವ-ತದ
ಐರಾ-ವ-ತ-ದೊ-ಡನೆ
ಐರಾ-ವ-ತ-ವನ್ನು
ಐರಾ-ವ-ತ-ವೆಂಬ
ಐವ-ತ್ತಾರು
ಐವತ್ತು
ಐವ-ತ್ತೊಂದು
ಐವರು
ಐಶ್ವರ್ಯ
ಐಶ್ವ-ರ್ಯಕ್ಕೆ
ಐಶ್ವ-ರ್ಯ-ಗಳ
ಐಶ್ವ-ರ್ಯ-ಗಳನ್ನು
ಐಶ್ವ-ರ್ಯ-ಗಳಿಂದ
ಐಶ್ವ-ರ್ಯ-ದಿಂದ
ಐಶ್ವ-ರ್ಯ-ವಂತೂ
ಐಶ್ವ-ರ್ಯ-ವ-ನ್ನಿತ್ತೆ
ಐಶ್ವ-ರ್ಯ-ವನ್ನು
ಐಶ್ವ-ರ್ಯ-ವ-ನ್ನೆಲ್ಲ
ಐಶ್ವ-ರ್ಯವೂ
ಒಂಟಿ
ಒಂಟಿ-ಕಾ-ಲಿ-ನಲ್ಲಿ
ಒಂಟಿ-ಗ-ಳಾಗಿ
ಒಂಟಿ-ಗಾ-ಲಲ್ಲಿ
ಒಂಟಿ-ಯಾಗಿ
ಒಂಟಿ-ಯಾ-ಗಿ-ರ-ಬೇ-ಕೆಂದೂ
ಒಂದಂತೆ
ಒಂದ-ಕ್ಕಿಂತ
ಒಂದಕ್ಕೆ
ಒಂದ-ಕ್ಕೊಂದು
ಒಂದ-ನೆಯ
ಒಂದನ್ನು
ಒಂದರ
ಒಂದ-ರಂತೆ
ಒಂದ-ರಿಂದ
ಒಂದ-ರೊ-ಡ-ನೊಂದು
ಒಂದಲ್ಲ
ಒಂದಾ-ಗಲು
ಒಂದಾಗಿ
ಒಂದಾ-ಗಿದ್ದ
ಒಂದಾ-ಗಿ-ಹೋ-ಗಲಿ
ಒಂದಾ-ಗಿ-ಹೋ-ಗು-ತ್ತದೆ
ಒಂದಾ-ಗಿ-ಹೋದ
ಒಂದಾ-ಗಿ-ಹೋ-ದವು
ಒಂದಾ-ಗು-ವಂತೆ
ಒಂದಾ-ಗೋಣ
ಒಂದಾದ
ಒಂದಾ-ದ-ಮೇ-ಲೊಂ-ದ-ರಂತೆ
ಒಂದಾ-ದರೂ
ಒಂದಾ-ನೊಂದು
ಒಂದಾ-ಯಿತು
ಒಂದು
ಒಂದು-ನೂರು
ಒಂದೂ
ಒಂದೆ
ಒಂದೆಡೆ
ಒಂದೆ-ಡೆ-ಯ-ಲ್ಲಿಯೇ
ಒಂದೆ-ರಡು
ಒಂದೇ
ಒಂದೊಂ-ದರ
ಒಂದೊಂ-ದಾಗಿ
ಒಂದೊಂದು
ಒಂದೊಂದೂ
ಒಂದೊಂದೇ
ಒಂದೊ-ದನ್ನು
ಒಂಬ
ಒಂಬ-ತ್ತ-ನೆಯ
ಒಂಬತ್ತು
ಒಂಬ-ತ್ತು-ವಿ-ಧ-ವಾದ
ಒಂಬತ್ತೂ
ಒಕ್ಕ-ಲಿಕ್ಕಿ
ಒಕ್ಕೊ-ರ-ಲಿ-ನಿಂದ
ಒಗ-ಟನ್ನೆ
ಒಗೆದ
ಒಗೆದು
ಒಗೆ-ಯಲು
ಒಟ್ಟಾಗಿ
ಒಟ್ಟಾ-ಗಿ-ಯಾ-ದರೂ
ಒಟ್ಟಿಗೆ
ಒಟ್ಟಿಸಿ
ಒಟ್ಟು-ಗೂ-ಡಿ-ಸಿ-ಕೊಂಡು
ಒಟ್ಟು-ಗೂ-ಡಿ-ಸಿ-ರು-ವುದು
ಒಡಂ-ಬ-ಡು-ವು-ದಿಲ್ಲ
ಒಡತಿ
ಒಡ-ತಿ-ಯ-ನ್ನಾಗಿ
ಒಡ-ತಿ-ಯಾದ
ಒಡ-ನಾ-ಡಿ-ಗ-ಳಾದ
ಒಡನೆ
ಒಡ-ನೆಯೆ
ಒಡ-ನೆಯೇ
ಒಡ-ವೆ-ಗಳ
ಒಡ-ವೆ-ಗ-ಳ-ನ್ನಿಟ್ಟು
ಒಡ-ವೆ-ಗಳನ್ನು
ಒಡ-ವೆ-ಗಳನ್ನೂ
ಒಡ-ವೆ-ಗಳು
ಒಡೆ-ತನ
ಒಡೆದ
ಒಡೆದು
ಒಡೆ-ದು-ಹಾಕಿ
ಒಡೆ-ದು-ಹಾ-ಕಿ-ದನು
ಒಡೆ-ದು-ಹಾ-ಕು-ವರು
ಒಡೆ-ದು-ಹೋ-ಗಿದೆ
ಒಡೆ-ದು-ಹೋ-ಯಿತು
ಒಡೆಯ
ಒಡೆ-ಯನ
ಒಡೆ-ಯ-ನಾಗಿ
ಒಡೆ-ಯ-ನಾ-ಗಿ-ದ್ದು-ಕೊಂಡು
ಒಡೆ-ಯ-ನಾ-ಗಿ-ರುವ
ಒಡೆ-ಯ-ನಾದ
ಒಡೆ-ಯ-ನಾ-ದನು
ಒಡೆ-ಯ-ನಿಗೆ
ಒಡೆ-ಯನೂ
ಒಡೆ-ಯ-ರ-ನ್ನಾಗಿ
ಒಡ್ಡಿ
ಒಡ್ಡಿದ
ಒಡ್ಡಿ-ದ-ವನು
ಒಡ್ಡಿ-ರುವ
ಒಡ್ಯಾಣ
ಒಡ್ಯಾ-ಣ-ದಿಂದ
ಒಣ
ಒಣಗಿ
ಒಣ-ಗಿ-ದವು
ಒಣ-ಗಿ-ಸಿ-ಕೊಂಡು
ಒತ್ತ-ಬೇ-ಕೆಂಬ
ಒತ್ತ-ರಿಸಿ
ಒತ್ತಾಯ
ಒತ್ತಾಸೆ
ಒತ್ತಿ
ಒತ್ತಿ-ಕೊಂ-ಡಳು
ಒತ್ತುತ್ತಾ
ಒತ್ತು-ತ್ತಿ-ದ್ದರು
ಒತ್ತು-ತ್ತಿ-ದ್ದ-ವಳು
ಒದಗಿ
ಒದ-ಗಿದ
ಒದ-ಗಿದ್ದ
ಒದ-ಗಿ-ರುವ
ಒದ-ಗಿ-ಸ-ಬ-ಲ್ಲು-ದಾ-ದು-ದ-ರಿಂದ
ಒದ-ಗಿ-ಸಲು
ಒದ-ಗಿ-ಸಿ-ಕೊ-ಡು-ತ್ತೇನೆ
ಒದ-ಗಿ-ಸಿ-ದ-ನಾ-ದರೂ
ಒದ-ಗಿ-ಸಿ-ದನು
ಒದ-ಗಿ-ಸು-ತ್ತಾನೆ
ಒದ-ಗಿ-ಸು-ತ್ತಿದ್ದ
ಒದ-ಗಿ-ಸು-ತ್ತಿ-ರುವ
ಒದ-ಗಿ-ಸುವ
ಒದ-ಗಿ-ಸು-ವು-ದಾ-ಗಿಯೂ
ಒದ-ರುತ್ತಾ
ಒದೆ-ದನು
ಒದೆ-ಯುವ
ಒದ್ದಾ
ಒದ್ದಾ-ಡಿ-ಹೋ-ದವು
ಒದ್ದಾ-ಡುತ್ತಾ
ಒದ್ದಾ-ಡು-ತ್ತಿ-ದ್ದೇವೆ
ಒದ್ದಾ-ಡು-ತ್ತಿ-ರಲು
ಒದ್ದಾ-ಡು-ತ್ತಿ-ರುವ
ಒದ್ದು
ಒನಕೆ
ಒನ-ಕೆಯ
ಒನ-ಕೆ-ಯನ್ನು
ಒನ-ಕೆ-ಯನ್ನೂ
ಒನ-ಕೆ-ಯಿಂದ
ಒನ-ಕೆ-ಯಿತ್ತು
ಒನ-ಕೆಯೆ
ಒನ-ಕೆ-ಯೊಂ-ದನ್ನು
ಒನ-ಕೆ-ಯೊ-ಡನೆ
ಒಪ್ಪಂದ
ಒಪ್ಪಂ-ದ-ಮಾ-ಡಿ-ಕೊಂ-ಡರು
ಒಪ್ಪದೆ
ಒಪ್ಪರು
ಒಪ್ಪ-ಲಿಲ್ಲ
ಒಪ್ಪಲು
ಒಪ್ಪಿ
ಒಪ್ಪಿ-ಕೊಂ-ಡನು
ಒಪ್ಪಿ-ಕೊಂ-ಡ-ಮೇಲೆ
ಒಪ್ಪಿ-ಕೊಂ-ಡರು
ಒಪ್ಪಿ-ಕೊಂ-ಡಳು
ಒಪ್ಪಿ-ಕೊಂಡಿ
ಒಪ್ಪಿ-ಕೊಂಡು
ಒಪ್ಪಿ-ಕೊ-ಳ್ಳ-ಬೇಕು
ಒಪ್ಪಿ-ಕೊ-ಳ್ಳ-ಬೇ-ಕೆಂದು
ಒಪ್ಪಿ-ಕೊಳ್ಳಿ
ಒಪ್ಪಿಗೆ
ಒಪ್ಪಿ-ಗೆ-ಯಾಗ
ಒಪ್ಪಿ-ಗೆ-ಯಾ-ಗ-ದಿ-ದ್ದರೂ
ಒಪ್ಪಿ-ಗೆ-ಯಾ-ಗ-ಲಿಲ್ಲ
ಒಪ್ಪಿ-ಗೆ-ಯಾ-ಗಿತ್ತು
ಒಪ್ಪಿ-ಗೆ-ಯಾ-ಗಿದೆ
ಒಪ್ಪಿ-ಗೆ-ಯಾದ
ಒಪ್ಪಿ-ಗೆ-ಯಾ-ದರೆ
ಒಪ್ಪಿ-ಗೆ-ಯಾ-ಯಿತು
ಒಪ್ಪಿ-ಗೆ-ಯಿ-ತ್ತರು
ಒಪ್ಪಿ-ಗೆಯೆ
ಒಪ್ಪಿದ
ಒಪ್ಪಿ-ದ-ನಾ-ದರೂ
ಒಪ್ಪಿ-ದನು
ಒಪ್ಪಿ-ದರು
ಒಪ್ಪಿ-ದಳು
ಒಪ್ಪಿದೆ
ಒಪ್ಪಿದ್ದ
ಒಪ್ಪಿ-ಸ-ಬೇಕು
ಒಪ್ಪಿ-ಸಲು
ಒಪ್ಪಿಸಿ
ಒಪ್ಪಿ-ಸಿದ
ಒಪ್ಪಿ-ಸಿ-ದನು
ಒಪ್ಪಿ-ಸಿ-ದ-ನೆಂಬ
ಒಪ್ಪಿ-ಸಿ-ದರು
ಒಪ್ಪಿ-ಸಿ-ದವು
ಒಪ್ಪಿ-ಸಿ-ದಿ-ರಂತೆ
ಒಪ್ಪಿ-ಸಿ-ಬಿ-ಡು-ವಷ್ಟು
ಒಪ್ಪಿಸು
ಒಪ್ಪಿ-ಸು-ತ್ತಿ-ದ್ದಳು
ಒಪ್ಪಿ-ಸು-ತ್ತಿ-ದ್ದವು
ಒಪ್ಪಿ-ಸು-ತ್ತಿವೆ
ಒಪ್ಪಿ-ಸು-ತ್ತೇನೆ
ಒಪ್ಪಿ-ಸು-ತ್ತೇವೆ
ಒಪ್ಪಿ-ಸು-ವಂತೆ
ಒಪ್ಪಿ-ಸು-ವರು
ಒಪ್ಪಿ-ಸು-ವು-ದಿ-ರಲಿ
ಒಪ್ಪಿ-ಸು-ವು-ದಿ-ಲ್ಲ-ವೆಂದು
ಒಪ್ಪಿ-ಸು-ವು-ದುಈ
ಒಪ್ಪು-ತ್ತಾ-ನೆಯೆ
ಒಪ್ಪು-ತ್ತಾರೆ
ಒಪ್ಪುವು
ಒಪ್ಪು-ವು-ದಾ-ದರೆ
ಒಬ್ಬ
ಒಬ್ಬ-ಎಂಬ
ಒಬ್ಬ-ನಂತೆ
ಒಬ್ಬ-ನಾ-ಗ-ಬೇ-ಕಾ-ದು-ದು-ರಿಂದ
ಒಬ್ಬ-ನಾಗಿ
ಒಬ್ಬ-ನಾ-ಗು-ವನು
ಒಬ್ಬ-ನಾದ
ಒಬ್ಬನು
ಒಬ್ಬನೇ
ಒಬ್ಬ-ನೊ-ಡನೆ
ಒಬ್ಬರ
ಒಬ್ಬ-ರ-ನ್ನೊ-ಬ್ಬರು
ಒಬ್ಬ-ರಲ್ಲ
ಒಬ್ಬ-ರಿಗೆ
ಒಬ್ಬರು
ಒಬ್ಬರೂ
ಒಬ್ಬರೆ
ಒಬ್ಬಳ
ಒಬ್ಬ-ಳಂತೂ
ಒಬ್ಬ-ಳನ್ನು
ಒಬ್ಬ-ಳಿ-ಗಾ-ಗಲಿ
ಒಬ್ಬಳು
ಒಬ್ಬಿ-ಬ್ಬರ
ಒಬ್ಬಿ-ಬ್ಬರು
ಒಬ್ಬೊಂ-ಟಿ-ಗ-ನಾಗಿ
ಒಬ್ಬೊಬ್ಬ
ಒಬ್ಬೊ-ಬ್ಬ-ನನ್ನೂ
ಒಬ್ಬೊ-ಬ್ಬನೂ
ಒಬ್ಬೊ-ಬ್ಬ-ರನ್ನು
ಒಬ್ಬೊ-ಬ್ಬ-ರನ್ನೂ
ಒಬ್ಬೊ-ಬ್ಬ-ರನ್ನೆ
ಒಬ್ಬೊ-ಬ್ಬ-ರಾಗಿ
ಒಬ್ಬೊ-ಬ್ಬ-ರಿಗೂ
ಒಬ್ಬೊ-ಬ್ಬರೂ
ಒಬ್ಬೊ-ಬ್ಬ-ಳನ್ನೂ
ಒಬ್ಬೊ-ಬ್ಬ-ಳಿಗೂ
ಒಮ್ಮೆ
ಒಮ್ಮೆಗೆ
ಒಮ್ಮೆ-ಯಾ-ದರೂ
ಒಮ್ಮೆಯೆ
ಒಮ್ಮೊಮ್ಮೆ
ಒಯ್ದು
ಒಯ್ಯನೆ
ಒಯ್ಯಾ-ರ-ದಿಂದ
ಒಯ್ಯುವೆ
ಒರ-ಟ-ನನ್ನು
ಒರ-ಟ-ನಾದ
ಒರ-ಳನ್ನು
ಒರ-ಳಿಗೆ
ಒರಳು
ಒರ-ಳು-ಕ-ಲ್ಲನ್ನೂ
ಒರ-ಳು-ಕ-ಲ್ಲನ್ನೊ
ಒರ-ಸಿದ
ಒರೆ
ಒರೆ-ಗ-ಲ್ಲಿ-ನಂತೆ
ಒರೆ-ಗ-ಳ-ಚಿದ
ಒರೆ-ಯಲ್ಲಿ
ಒರೆ-ಯ-ಲ್ಲಿದ್ದ
ಒರೆ-ಯಿಂದ
ಒರೆ-ಸಿ-ಕ-ಳ್ಳುತ್ತಾ
ಒರೆ-ಸಿ-ದನು
ಒಲ-ವಿ-ರ-ಲಿಲ್ಲ
ಒಲಿದ
ಒಲಿ-ದಾಗ
ಒಲಿದು
ಒಲಿ-ಯ-ದಿ-ದ್ದಾಗ
ಒಲಿಯು
ಒಲಿ-ಯು-ತ್ತಾನೆ
ಒಲಿ-ಸಿ-ಕೊಂಡು
ಒಲಿ-ಸಿ-ಕೊ-ಳ್ಳುವ
ಒಲಿ-ಸು-ವು-ದ-ಕ್ಕಾಗಿ
ಒಲೆಯ
ಒಲೆ-ಯ-ಮೇಲೆ
ಒಲ್ಲೆ-ನೆಂದು
ಒಳ
ಒಳ-ಕೊಂಡ
ಒಳಕ್ಕೂ
ಒಳಕ್ಕೆ
ಒಳ-ಗಡೆ
ಒಳ-ಗಣ್ಣು
ಒಳ-ಗಾ-ಗ-ಬಾ-ರದು
ಒಳ-ಗಾ-ಗಲಿ
ಒಳ-ಗಾಗಿ
ಒಳ-ಗಾ-ಗಿದ್ದ
ಒಳ-ಗಾ-ಗಿ-ದ್ದೀರಿ
ಒಳ-ಗಾ-ಗಿವೆ
ಒಳ-ಗಾ-ಗು-ತ್ತಾರೆ
ಒಳ-ಗಾ-ಗು-ತ್ತೇನೆ
ಒಳ-ಗಾ-ಗು-ತ್ತೇವೆ
ಒಳ-ಗಾ-ಗು-ವಂತೆ
ಒಳ-ಗಾ-ಗು-ವರು
ಒಳ-ಗಾ-ಗು-ವ-ರೆಂ-ಬು-ದನ್ನು
ಒಳ-ಗಾ-ಗು-ವು-ದಿಲ್ಲ
ಒಳ-ಗಾ-ಗು-ವುದೆ
ಒಳ-ಗಾ-ಗು-ವು-ದೇಕೆ
ಒಳ-ಗಾದ
ಒಳ-ಗಾ-ದರೂ
ಒಳ-ಗಾ-ದ-ವ-ನಂ-ತೆ-ಕ-ರ್ಮ-ಬಂ-ಧಕ್ಕೆ
ಒಳ-ಗಾ-ದ-ವರ
ಒಳ-ಗಾ-ದ-ವರು
ಒಳ-ಗಾ-ದವು
ಒಳ-ಗಾ-ಯಿತು
ಒಳ-ಗಿದ್ದ
ಒಳ-ಗಿ-ನಿಂದ
ಒಳಗೂ
ಒಳಗೆ
ಒಳ-ಗೊಂಡ
ಒಳ-ಗೊಂ-ಡು-ದು-ದ-ರಿಂದ
ಒಳ-ಗೊ-ಳಗೆ
ಒಳ-ಗೊ-ಳಗೇ
ಒಳ-ಪ-ಡಿ-ಸು-ವುದು
ಒಳ-ಭಾ-ಗವೂ
ಒಳ-ಮ-ನೆ-ಯಲ್ಲಿ
ಒಳ-ಹೊಕ್ಕು
ಒಳ-ಹೊ-ರ-ಗಿನ
ಒಳ್ಳೆ
ಒಳ್ಳೆಯ
ಒಳ್ಳೆ-ಯ-ದಾ-ಗ-ಲೆಂದು
ಒಳ್ಳೆ-ಯದು
ಒಳ್ಳೆ-ಯದೆ
ಒಳ್ಳೆ-ಯದೇ
ಓ
ಓಂ
ಓಂಕಾ-ರ-ದಲ್ಲಿ
ಓಂಕಾ-ರ-ರೂ-ಪಿ-ಯಾದ
ಓಂಕಾ-ರ-ವನ್ನು
ಓಕ-ಳಿ-ಯನ್ನು
ಓಜಸೇ
ಓಟ
ಓಟ-ಕಿತ್ತ
ಓಡ
ಓಡ-ತೊ-ಡ-ಗಿದ
ಓಡ-ತೊ-ಡ-ಗಿ-ದಳು
ಓಡ-ಬೇ-ಕಾ-ಯಿತು
ಓಡ-ಬೇಕು
ಓಡ-ಲಾ-ರಂ-ಭಿ-ಸಿದ
ಓಡ-ಲಾ-ರಂ-ಭಿ-ಸಿ-ದವು
ಓಡಲು
ಓಡಾ-ಡಿ-ದಳು
ಓಡಾ-ಡುತ್ತಾ
ಓಡಾ-ಡು-ತ್ತಿ-ದ್ದರು
ಓಡಾ-ಡು-ತ್ತಿ-ದ್ದಾರೆ
ಓಡಾ-ಡು-ತ್ತಿ-ರಲಿ
ಓಡಾ-ಡುವ
ಓಡಾ-ಡು-ವಾಗ
ಓಡಿ
ಓಡಿತು
ಓಡಿದ
ಓಡಿ-ದನು
ಓಡಿ-ದರು
ಓಡಿ-ದರೂ
ಓಡಿ-ದಳು
ಓಡಿ-ದು-ದಾ-ಯಿತು
ಓಡಿ-ಬಂದ
ಓಡಿ-ಬಂ-ದನು
ಓಡಿ-ಬಂ-ದರು
ಓಡಿ-ಬಂದು
ಓಡಿ-ಬಂ-ದು-ದುನ್ನು
ಓಡಿ-ಬಂದೆ
ಓಡಿ-ಬಂ-ದೆ-ವಲ್ಲಾ
ಓಡಿ-ಸ-ಬೇ-ಕಾ-ಯಿತು
ಓಡಿಸಿ
ಓಡಿ-ಸಿದ
ಓಡಿ-ಸಿ-ದನು
ಓಡಿ-ಸಿ-ದರು
ಓಡಿ-ಸಿ-ದರೂ
ಓಡಿ-ಸಿ-ಬಿಟ್ಟ
ಓಡಿ-ಸಿ-ಬಿ-ಟ್ಟಿ-ದ್ದನು
ಓಡಿ-ಸಿರಿ
ಓಡಿ-ಸು-ವುದು
ಓಡಿ-ಹೋ-ಗ-ಲಾ-ರಂ-ಭಿ-ಸಿ-ದಳು
ಓಡಿ-ಹೋ-ಗಲು
ಓಡಿ-ಹೋಗಿ
ಓಡಿ-ಹೋ-ಗಿ-ರು-ವಾಗ
ಓಡಿ-ಹೋಗು
ಓಡಿ-ಹೋ-ಗುತ್ತಾ
ಓಡಿ-ಹೋ-ಗು-ತ್ತಾರೆ
ಓಡಿ-ಹೋ-ಗು-ತ್ತಾ-ರೆಯೆ
ಓಡಿ-ಹೋ-ಗು-ತ್ತಾ-ಳಂತೆ
ಓಡಿ-ಹೋ-ಗು-ತ್ತಿ-ದ್ದಾರೆ
ಓಡಿ-ಹೋ-ಗು-ತ್ತಿ-ರು-ವಾಗ
ಓಡಿ-ಹೋ-ಗುವ
ಓಡಿ-ಹೋದ
ಓಡಿ-ಹೋ-ದನು
ಓಡಿ-ಹೋ-ದ-ನೆಂದು
ಓಡಿ-ಹೋ-ದರು
ಓಡಿ-ಹೋ-ಯಿತು
ಓಡು-ತ್ತದೆ
ಓಡುತ್ತಾ
ಓಡು-ತ್ತಿ-ದ್ದಾಳೆ
ಓಡು-ತ್ತಿ-ರುವ
ಓಡು-ತ್ತಿ-ರು-ವುದನ್ನು
ಓಡು-ವರು
ಓಡು-ವು-ದಕ್ಕೆ
ಓಡು-ವುದು
ಓಡು-ವು-ದು-ಹೀಗೆ
ಓತ-ಪ್ರೋ-ತ-ವಾಗಿ
ಓತಿ-ಕೇತ
ಓತಿ-ಕೇ-ತ-ನಾ-ಗಿದ್ದೆ
ಓದ-ದಿ-ದ್ದರೂ
ಓದ-ಬೇ-ಕಾದ
ಓದಲಿ
ಓದಿ
ಓದಿದ
ಓದಿ-ದರೆ
ಓದಿ-ದ-ವರು
ಓದಿ-ದಾಗ
ಓದು
ಓದು-ಗರು
ಓದುತ್ತಾ
ಓದುವ
ಓದು-ವಂತೆ
ಓದು-ವ-ವ-ರಿ-ಗೆಲ್ಲ
ಓದು-ವುದೂ
ಓರಿ-ಗೆಯ
ಓರೆ-ಗ-ಣ್ನೋಟ
ಓರೆ-ನೋ-ಟ-ದಿಂ-ದಲೂ
ಓರೆ-ಮಾಡಿ
ಓರೆ-ಯಾಗಿ
ಓಲೆ
ಓಲೆ-ಗ-ಳನ್ನೆ
ಓಲೆ-ಗಳು
ಓಲೈಸು
ಓಲೈ-ಸು-ತ್ತಿ-ದ್ದಾರೆ
ಓಹೊ
ಓಹೋ
ಔತ-ಣ-ವನ್ನು
ಔದಾ-ರ್ಯಕ್ಕೂ
ಔದಾ-ರ್ಯ-ವನ್ನೂ
ಔದಾ-ಸೀ-ನ್ಯ-ವನ್ನು
ಔರ್ವ
ಔರ್ವ-ನೆಂಬ
ಔರ್ವ-ಪು-ಷಿಯ
ಔರ್ವ-ಮ-ಹ-ರ್ಷಿಯು
ಔರ್ವ-ಮ-ಹ-ರ್ಷಿಯೆ
ಔಷಧ
ಔಷ-ಧದ
ಔಷಧಿ
ಔಷ-ಧಿ-ಗಳೂ
ಔಷ-ಧಿ-ಗಳೇ
ಕಂಕ
ಕಂಕೆ-ಬಕ
ಕಂಗಾ-ಲಾ-ಗ-ಲಿಲ್ಲ
ಕಂಗಾ-ಲಾಗಿ
ಕಂಗಾ-ಲಾ-ಗಿದ್ದ
ಕಂಗಾ-ಲಾದ
ಕಂಗೆ-ಟ್ಟರು
ಕಂಗೆಟ್ಟು
ಕಂಗೊ-ಳಿ-ಸುತ್ತಾ
ಕಂಗೊ-ಳಿ-ಸು-ತ್ತಿತ್ತು
ಕಂಗೊ-ಳಿ-ಸು-ತ್ತಿ-ದ್ದನು
ಕಂಗೊ-ಳಿ-ಸು-ತ್ತಿ-ರು-ವರು
ಕಂಚು
ಕಂಟ-ಕ-ನಾ-ಗಿ-ರುವ
ಕಂಟ-ಕ-ರಾ-ಗಿ-ದ್ದರು
ಕಂಟ-ಕ-ರಾ-ಗು-ತ್ತಾರೆ
ಕಂಟ-ಕ-ರೆ-ಲ್ಲರೂ
ಕಂಟ-ಕ-ವೇನೋ
ಕಂಠ
ಕಂಠದ
ಕಂಠ-ದಲ್ಲಿ
ಕಂಠ-ದಿಂದ
ಕಂಠ-ಧ್ವನಿ
ಕಂಠ-ಪೂರ್ತಿ
ಕಂಠವು
ಕಂಡ
ಕಂಡಂ-ತಾ-ಯಿತು
ಕಂಡ-ಕಂ-ಡ-ವ-ರಿ-ಗೆಲ್ಲ
ಕಂಡನು
ಕಂಡ-ನೆಂದು
ಕಂಡ-ಮಾ-ತ್ರಕ್ಕೆ
ಕಂಡ-ಮೇಲೆ
ಕಂಡ-ರಂತೂ
ಕಂಡ-ರಾ-ಗದು
ಕಂಡ-ರಿ-ಯದ
ಕಂಡರೂ
ಕಂಡರೆ
ಕಂಡಲ್ಲಿ
ಕಂಡ-ವರ
ಕಂಡ-ವ-ರನ್ನು
ಕಂಡ-ವ-ರಿಲ್ಲ
ಕಂಡ-ವರು
ಕಂಡ-ವರೆ-ಲ್ಲರೂ
ಕಂಡಷ್ಟು
ಕಂಡ-ಹಾಗೆ
ಕಂಡಾಗ
ಕಂಡಾ-ಗಲೂ
ಕಂಡಾ-ಗ-ಲೆಲ್ಲ
ಕಂಡಿದ್ದ
ಕಂಡಿ-ದ್ದೀರಾ
ಕಂಡಿ-ದ್ದು-ದ-ರಿಂದ
ಕಂಡಿ-ರ-ಬೇ-ಕೆ-ನಿ-ಸು-ತ್ತದೆ
ಕಂಡಿ-ರ-ಲಿಲ್ಲ
ಕಂಡಿ-ರುವೆ
ಕಂಡಿಲ್ಲ
ಕಂಡು
ಕಂಡು-ಕೊ-ಳ್ಳ-ಲಾ-ರವು
ಕಂಡು-ದನ್ನು
ಕಂಡು-ದ-ರಿಂದ
ಕಂಡು-ದೆಲ್ಲ
ಕಂಡು-ದೆ-ಲ್ಲವೂ
ಕಂಡು-ಬಂದ
ಕಂಡು-ಬ-ರ-ಲಿಲ್ಲ
ಕಂಡು-ಬ-ರು-ತ್ತದೆ
ಕಂಡು-ಬ-ರು-ತ್ತವೆ
ಕಂಡು-ಬ-ರು-ತ್ತೇನೆ
ಕಂಡು-ಬ-ರುವು
ಕಂಡು-ಬ-ರು-ವು-ದಿಲ್ಲ
ಕಂಡು-ಹಿಡಿ
ಕಂಡು-ಹಿ-ಡಿ-ದನು
ಕಂಡು-ಹಿ-ಡಿದು
ಕಂಡು-ಹಿ-ಡಿ-ಯು-ವುದು
ಕಂಡೆ
ಕಂಡೆಯಾ
ಕಂಡೆ-ಯೇನು
ಕಂಡೇ
ಕಂಡೊ-ಡನೆ
ಕಂಡೊ-ಡ-ನೆಯೇ
ಕಂದ
ಕಂದಕ
ಕಂದ-ನನ್ನು
ಕಂದ-ನಾ-ಗಿ-ರು-ವ-ದನ್ನು
ಕಂದ-ನಾದ
ಕಂದ-ನಿಗೆ
ಕಂದ-ಮ್ಮ-ಗ-ಳಾ-ಗಿದ್ದ
ಕಂದ-ಮ್ಮ-ಗಳು
ಕಂದಿ
ಕಂಪನ್ನು
ಕಂಪ-ಯನ್
ಕಂಪು
ಕಂಪುಈ
ಕಂಬ
ಕಂಬ-ಗಳಿಂದ
ಕಂಬ-ಗ-ಳೆಲ್ಲ
ಕಂಬನಿ
ಕಂಬ-ಳಿ-ಯನ್ನು
ಕಂಭ
ಕಂಭ-ದಲ್ಲಿ
ಕಂಸ
ಕಂಸ-ಕು-ಮಾರ
ಕಂಸನ
ಕಂಸ-ನನ್ನು
ಕಂಸ-ನಿಂದ
ಕಂಸ-ನಿ-ಗಂತೂ
ಕಂಸ-ನಿಗೂ
ಕಂಸ-ನಿಗೆ
ಕಂಸನು
ಕಂಸ-ನೆಲ್ಲಿ
ಕಂಸನೇ
ಕಂಸ-ರಾ-ಜನ
ಕಂಸ-ರಾ-ಜ-ನಿಗೆ
ಕಂಸ-ರಾ-ಜನು
ಕಂಸ-ವ-ತಿ-ಸು-ವೀರ
ಕಂಸ-ವಧೆ
ಕಂಸೆ-ಚಿ-ತ್ರ-ಕೇತು
ಕಂಸೆಯೇ
ಕಕು-ತ್ಸ್್ಥ
ಕಕ್ಕಾ-ಬಿಕ್ಕಿ
ಕಕ್ಕಾ-ಬಿ-ಕ್ಕಿ-ಯಾ-ಗ-ಬೇ-ಕಾ-ಗಿಲ್ಲ
ಕಕ್ಕಾ-ಬಿ-ಕ್ಕಿ-ಯಾಗಿ
ಕಕ್ಕಿ
ಕಕ್ಷಂ
ಕಗ್ಗಂಟು
ಕಗ್ಗ-ತ್ತಲ
ಕಗ್ಗ-ತ್ತ-ಲಿ-ನಲ್ಲಿ
ಕಗ್ಗ-ತ್ತ-ಲಿ-ನಿಂದ
ಕಗ್ಗ-ತ್ತಲು
ಕಗ್ಗ-ತ್ತಲೆ
ಕಗ್ಗ-ತ್ತ-ಲೆಗೆ
ಕಗ್ಗ-ತ್ತ-ಲೆಯ
ಕಗ್ಗ-ತ್ತ-ಲೆಯೆ
ಕಗ್ಗಲ್ಲು
ಕಗ್ಗಾ-ಡಿ-ನಲ್ಲಿ
ಕಗ್ಗೊ-ಲೆ-ಯನ್ನು
ಕಗ್ಗೊ-ಲೆ-ಯಾ-ಗಿ-ಹೋ-ಯಿತು
ಕಚ-ನೆಂಬ
ಕಚ್ಚ-ಬಾ-ರ-ದೆಂ-ಬು-ದಾ-ಗಿಯೂ
ಕಚ್ಚಲು
ಕಚ್ಚಿ
ಕಚ್ಚಿ-ಕೊಂಡು
ಕಚ್ಚಿ-ಕೊಂಡೇ
ಕಚ್ಚಿತು
ಕಚ್ಚಿ-ದ-ಜಿತಃ
ಕಚ್ಚಿ-ದನು
ಕಚ್ಚಿ-ದರೆ
ಕಚ್ಚಿ-ದ-ವನು
ಕಚ್ಚಿ-ದವು
ಕಚ್ಚಿ-ದ್ಗಾ-ಢ-ನಿ-ರ್ಭಿನ್ನ
ಕಚ್ಚಿ-ನ್ಮು-ಕುಂದ
ಕಚ್ಚಿ-ಸಿ-ದನು
ಕಚ್ಚಿ-ಸಿ-ದುದು
ಕಚ್ಚಿ-ಹಾ-ಕು-ತ್ತಿ-ದ್ದನು
ಕಚ್ಚು-ತ್ತಿ-ದ್ದರೆ
ಕಚ್ಚು-ತ್ತಿ-ದ್ದವು
ಕಚ್ಚು-ವುವು
ಕಟ-ಕಟ
ಕಟಾ-ಕ್ಷ-ದಿಂ-ದಲೂ
ಕಟಿ-ಸೂತ್ರ
ಕಟು-ಕನ
ಕಟು-ಕನು
ಕಟು-ಟೀ-ಕೆಗೆ
ಕಟು-ನು-ಡಿ-ಗಳನ್ನು
ಕಟು-ನು-ಡಿ-ಗಳಿಂದ
ಕಟು-ನು-ಡಿ-ಗ-ಳಿ-ಗಾಗಿ
ಕಟು-ವಾಗಿ
ಕಟ್ಟ-ಕ-ಡೆ-ಯಲ್ಲಿ
ಕಟ್ಟ-ಕ-ಡೆ-ಯ-ವ-ನಾದ
ಕಟ್ಟ-ಕೊಂ-ಡೆಯಾ
ಕಟ್ಟಲು
ಕಟ್ಟ-ಳೆ-ಯೆಂದು
ಕಟ್ಟಿ
ಕಟ್ಟಿ-ಕೊಂ-ಡಿತು
ಕಟ್ಟಿ-ಕೊಂ-ಡಿತ್ತು
ಕಟ್ಟಿ-ಕೊಂ-ಡಿದ್ದ
ಕಟ್ಟಿ-ಕೊಂಡು
ಕಟ್ಟಿ-ಕೊಳ್ಳಿ
ಕಟ್ಟಿಗೆ
ಕಟ್ಟಿ-ಗೆ-ಯನ್ನು
ಕಟ್ಟಿ-ಗೆ-ಯನ್ನೇ
ಕಟ್ಟಿ-ಗೆ-ಯ-ಲ್ಲಿ-ರುವ
ಕಟ್ಟಿ-ಟ್ಟಿ-ದ್ದಾನೆ
ಕಟ್ಟಿಟ್ಟು
ಕಟ್ಟಿತು
ಕಟ್ಟಿದ
ಕಟ್ಟಿ-ದನು
ಕಟ್ಟಿ-ದನೊ
ಕಟ್ಟಿ-ದ-ರಾ-ದರೂ
ಕಟ್ಟಿ-ದರೊ
ಕಟ್ಟಿ-ದಳು
ಕಟ್ಟಿದ್ದ
ಕಟ್ಟಿ-ದ್ದರು
ಕಟ್ಟಿ-ನಿಂದ
ಕಟ್ಟಿ-ಸಲು
ಕಟ್ಟಿಸಿ
ಕಟ್ಟಿ-ಸಿ-ಕೊ-ಟ್ಟನು
ಕಟ್ಟಿ-ಸಿ-ದ್ದೇನೆ
ಕಟ್ಟಿ-ಹಾಕಿ
ಕಟ್ಟಿ-ಹಾ-ಕಿದ
ಕಟ್ಟಿ-ಹಾ-ಕಿ-ದ-ನಮ್ಮ
ಕಟ್ಟಿ-ಹಾ-ಕಿದ್ದ
ಕಟ್ಟಿ-ಹಾಕು
ಕಟ್ಟಿ-ಹಾ-ಕು-ತ್ತೇನೆ
ಕಟ್ಟಿ-ಹಾ-ಕು-ವುದು
ಕಟ್ಟಿ-ಹೋ-ದವು
ಕಟ್ಟು
ಕಟ್ಟು-ಗಳನ್ನು
ಕಟ್ಟು-ಗಳನ್ನೆಲ್ಲ
ಕಟ್ಟು-ಗಳು
ಕಟ್ಟು-ತ್ತಾರೆ
ಕಟ್ಟು-ನಿ-ಟ್ಟಾಗಿ
ಕಟ್ಟು-ಬಿ-ದ್ದ-ವ-ನಂ-ತೆ-ಕಾ-ಣುವೆ
ಕಟ್ಟು-ಬಿದ್ದು
ಕಟ್ಟು-ಮಾಡಿ
ಕಟ್ಟು-ವಂತೆ
ಕಟ್ಟುವು
ಕಟ್ಟು-ಹ-ರಿದ
ಕಟ್ಟೆ
ಕಟ್ಟೆ-ಗಳನ್ನು
ಕಠಿ-ಣ-ವಾದ
ಕಠಿ-ನ-ವಾದ
ಕಠೋರ
ಕಠೋ-ರ-ವಾಗಿ
ಕಠೋ-ರ-ವಾದ
ಕಡಗ
ಕಡ-ಗೋಲು
ಕಡ-ಗೋ-ಲೇನೊ
ಕಡ-ಮೆಯ
ಕಡ-ಮೆ-ಯಾ-ಯಿತು
ಕಡ-ಮೆ-ಯಾ-ಯಿ-ತೆಂ-ದಾ-ಗಲಿ
ಕಡಲ
ಕಡ-ಲಂ-ತಾ-ಯಿತು
ಕಡ-ವರ
ಕಡವೆ
ಕಡಿ
ಕಡಿ-ಕ-ಡಿ-ದು-ಹಾಕು
ಕಡಿ-ತ-ದಂತೆ
ಕಡಿ-ದಂ-ತಾ-ಯಿತು
ಕಡಿ-ದಾದ
ಕಡಿದು
ಕಡಿಮೆ
ಕಡಿ-ಮೆ-ಯಾ-ಗಿತ್ತು
ಕಡಿ-ಮೆ-ಯಾ-ಗುವು
ಕಡಿ-ಮೆ-ಯಾ-ದೀತು
ಕಡಿ-ಯ-ಲಾ-ಗು-ವು-ದಿಲ್ಲ
ಕಡಿ-ಯಿರಿ
ಕಡಿ-ಯುತ್ತಾ
ಕಡು-ಗೋ-ಪ-ದಿಂದ
ಕಡು-ಬ-ಡ-ವ-ನಿಗೆ
ಕಡು-ಬು-ಗಳು
ಕಡೆ
ಕಡೆ-ಗಣ್
ಕಡೆ-ಗ-ಣ್ಣಿನ
ಕಡೆ-ಗ-ಣ್ಣಿ-ನಿಂದ
ಕಡೆ-ಗ-ಣ್ದೃಷ್ಟಿ
ಕಡೆ-ಗ-ಣ್ನೋಟ
ಕಡೆ-ಗ-ಣ್ನೋ-ಟ-ಕ್ಕಾಗಿ
ಕಡೆ-ಗಾಲ
ಕಡೆ-ಗಾ-ಲ-ದಲ್ಲಿ
ಕಡೆಗೂ
ಕಡೆಗೆ
ಕಡೆ-ಗೊಂದು
ಕಡೆ-ಗೋ-ಲನ್ನು
ಕಡೆ-ಗೋ-ಲನ್ನೇ
ಕಡೆ-ದರು
ಕಡೆ-ದರೂ
ಕಡೆದು
ಕಡೆಯ
ಕಡೆ-ಯ-ದಾಗಿ
ಕಡೆ-ಯ-ದಿನ
ಕಡೆ-ಯನ್ನು
ಕಡೆ-ಯ-ಬೇ-ಕಾ-ಯಿತು
ಕಡೆ-ಯ-ಬೇ-ಕೆಂಬ
ಕಡೆ-ಯಲು
ಕಡೆ-ಯಲ್ಲಿ
ಕಡೆ-ಯ-ವ-ನಾ-ದರೂ
ಕಡೆ-ಯ-ವನು
ಕಡೆ-ಯ-ವನೇ
ಕಡೆ-ಯ-ವ-ರ-ನ್ನಾ-ಗಲಿ
ಕಡೆ-ಯ-ವ-ರನ್ನೂ
ಕಡೆ-ಯ-ವ-ರಾರೋ
ಕಡೆ-ಯ-ವ-ರಿಗೂ
ಕಡೆ-ಯ-ವರು
ಕಡೆ-ಯ-ವರೂ
ಕಡೆ-ಯ-ವ-ರೆಗೂ
ಕಡೆ-ಯ-ವಳು
ಕಡೆ-ಯಾಗಿ
ಕಡೆ-ಯಾ-ಗು-ತ್ತಾನೆ
ಕಡೆ-ಯಿಂದ
ಕಡೆ-ಯಿಂ-ದಲೂ
ಕಡೆ-ಯಿರಿ
ಕಡೆಯು
ಕಡೆ-ಯುತ್ತಾ
ಕಡೆ-ಯು-ತ್ತಿ-ದ್ದವ
ಕಡೆ-ಯುವ
ಕಡೆ-ಯು-ವು-ದಕ್ಕೆ
ಕಡೆ-ಯು-ಸಿ-ರನ್ನು
ಕಡೆಯೂ
ಕಡೆಯೆ
ಕಡೆ-ಹಾದು
ಕಡ್ಡಿ-ಯಂ-ತಿದ್ದ
ಕಡ್ಡಿ-ಯನ್ನು
ಕಡ್ಡಿ-ಯಾ-ಗಿ-ದ್ದೀಯೆ
ಕಣ
ಕಣ-ಗ-ಳ-ನ್ನಾ-ದರೂ
ಕಣ-ಗಳನ್ನು
ಕಣ-ಗ-ಳಿ-ವೆಯೋ
ಕಣ-ಜದ
ಕಣದ
ಕಣ-ದಲ್ಲಿ
ಕಣ-ದ-ಲ್ಲಿಯೂ
ಕಣಯ್ಯ
ಕಣ-ವನ್ನು
ಕಣವು
ಕಣ್ಣ
ಕಣ್ಣನ್ನು
ಕಣ್ಣನ್ನೆ
ಕಣ್ಣಲ್ಲಿ
ಕಣ್ಣ-ಸನ್ನೆ
ಕಣ್ಣ-ಹ-ಬ್ಬ-ವನ್ನು
ಕಣ್ಣಾಗಿ
ಕಣ್ಣಾರ
ಕಣ್ಣಾರೆ
ಕಣ್ಣಿಂ-ದಲೆ
ಕಣ್ಣಿ-ಗಳನ್ನು
ಕಣ್ಣಿಗೂ
ಕಣ್ಣಿಗೆ
ಕಣ್ಣಿ-ಗೊ-ತ್ತಿ-ಕೊಂಡು
ಕಣ್ಣಿ-ಟ್ಟಿ-ರು-ವುದು
ಕಣ್ಣಿಟ್ಟು
ಕಣ್ಣಿ-ದ್ದರೂ
ಕಣ್ಣಿನ
ಕಣ್ಣಿ-ನಂತೆ
ಕಣ್ಣಿ-ನಲ್ಲಿ
ಕಣ್ಣಿ-ನ-ಲ್ಲಿದ್ದ
ಕಣ್ಣಿ-ನಿಂದ
ಕಣ್ಣಿ-ಲ್ಲದ
ಕಣ್ಣೀ-ರನ್ನು
ಕಣ್ಣೀ-ರನ್ನೆ
ಕಣ್ಣೀ-ರ-ನ್ನೊ-ರಸಿ
ಕಣ್ಣೀರಿ
ಕಣ್ಣೀ-ರಿಗೂ
ಕಣ್ಣೀ-ರಿ-ಡು-ತ್ತಿತ್ತು
ಕಣ್ಣೀ-ರಿ-ಡು-ತ್ತಿ-ದ್ದಾಳೆ
ಕಣ್ಣೀ-ರಿ-ಡು-ತ್ತಿ-ರುವ
ಕಣ್ಣೀ-ರಿ-ನಿಂದ
ಕಣ್ಣೀರು
ಕಣ್ಣೀ-ರು-ಗರೆ
ಕಣ್ಣೀ-ರು-ಗ-ರೆ-ದನು
ಕಣ್ಣೀ-ರು-ಗ-ರೆ-ದಳು
ಕಣ್ಣೀ-ರು-ಗ-ರೆ-ದವು
ಕಣ್ಣೀ-ರು-ಗ-ರೆ-ಯುತ್ತಾ
ಕಣ್ಣೀ-ರು-ತುಂಬಿ
ಕಣ್ಣೀ-ರೆ-ನಿ-ಸಿತು
ಕಣ್ಣೀ-ರ್ಗ-ರೆ-ಯುತ್ತಾ
ಕಣ್ಣು
ಕಣ್ಣು-ಕಣ್ಣು
ಕಣ್ಣು-ಗಳ
ಕಣ್ಣು-ಗಳನ್ನು
ಕಣ್ಣು-ಗ-ಳನ್ನೆ
ಕಣ್ಣು-ಗಳಲ್ಲಿ
ಕಣ್ಣು-ಗಳಿಂದ
ಕಣ್ಣು-ಗ-ಳಿಂ-ದಲೂ
ಕಣ್ಣು-ಗ-ಳಿಂ-ದಲೆ
ಕಣ್ಣು-ಗ-ಳಿಗೆ
ಕಣ್ಣು-ಗಳು
ಕಣ್ಣು-ಗ-ಳುಈ
ಕಣ್ಣು-ಗ-ಳು-ಕಾಮೋ
ಕಣ್ಣು-ಗ-ಳೆಲ್ಲ
ಕಣ್ಣು-ಗಳೊ
ಕಣ್ಣು-ಗ-ಳ್ಳುಳ್ಳ
ಕಣ್ಣು-ಗು-ಡ್ಡೆ-ಗಳು
ಕಣ್ಣು-ತೆ-ರೆ-ದನು
ಕಣ್ಣು-ಪಾ-ಪೆ-ಯಂತೆ
ಕಣ್ಣು-ಬಿ-ಡುತ್ತಾ
ಕಣ್ಣು-ಮು-ಚ್ಚಾಲೆ
ಕಣ್ಣು-ಮುಚ್ಚಿ
ಕಣ್ಣು-ಮು-ಚ್ಚಿ-ದನು
ಕಣ್ಣುಳ್ಳ
ಕಣ್ಣು-ಹಾ-ಕಿ-ದ್ದನ್ನು
ಕಣ್ಣೆ
ಕಣ್ಣೆಂದೂ
ಕಣ್ಣೆಂ-ದೂಈ
ಕಣ್ಣೆತ್ತಿ
ಕಣ್ಣೆ-ತ್ತಿ-ಕೂಡ
ಕಣ್ಣೆ-ತ್ತಿ-ನೋ-ಡು-ತ್ತಿಲ್ಲ
ಕಣ್ಣೆ-ತ್ತಿಯೂ
ಕಣ್ಣೆ-ದು-ರಿಗೆ
ಕಣ್ಣೆ-ದು-ರಿ-ನಲ್ಲಿ
ಕಣ್ಣೆ-ದು-ರಿ-ನ-ಲ್ಲಿಯೇ
ಕಣ್ಣೆ-ದು-ರಿ-ನಿಂದ
ಕಣ್ತುಂಬ
ಕಣ್ದ-ಣಿಯೆ
ಕಣ್ದಿ-ಟ್ಟಿಯ
ಕಣ್ದೆ-ರೆದು
ಕಣ್ಮಣಿ
ಕಣ್ಮ-ಣಿ-ಗ-ಳಾಗಿ
ಕಣ್ಮ-ಣಿ-ಯಂ-ತಿದ್ದ
ಕಣ್ಮ-ಣಿ-ಯನ್ನು
ಕಣ್ಮ-ಣಿ-ಯಾದ
ಕಣ್ಮ-ಣಿ-ಯಾ-ಯಿತು
ಕಣ್ಮ-ನ-ಗ-ಳಿಗೆ
ಕಣ್ಮರೆ
ಕಣ್ಮ-ರೆ-ಯಾ-ಗಲು
ಕಣ್ಮ-ರೆ-ಯಾಗಿ
ಕಣ್ಮ-ರೆ-ಯಾ-ಗಿ-ದ್ದು-ದ-ರಿಂದ
ಕಣ್ಮ-ರೆ-ಯಾ-ಗಿ-ರು-ವು-ದ-ರಿಂದ
ಕಣ್ಮ-ರೆ-ಯಾ-ಗಿ-ಹೋ-ಯಿತು
ಕಣ್ಮ-ರೆ-ಯಾ-ಗು-ತ್ತಲೆ
ಕಣ್ಮ-ರೆ-ಯಾದ
ಕಣ್ಮ-ರೆ-ಯಾ-ದಂ-ತಾ-ಯಿತು
ಕಣ್ಮ-ರೆ-ಯಾ-ದನು
ಕಣ್ಮ-ರೆ-ಯಾ-ದರು
ಕಣ್ಮ-ರೆ-ಯಾ-ದರೆ
ಕಣ್ಮ-ರೆ-ಯಾ-ದು-ದನ್ನು
ಕಣ್ಮ-ರೆ-ಯಾ-ದುದು
ಕಣ್ಮ-ರೆ-ಯಾದೆ
ಕಣ್ಮ-ರೆ-ಯಾ-ಯಿ-ತಲ್ಲಾ
ಕಣ್ಮ-ರೆ-ಯಾ-ಯಿತು
ಕಣ್ಮುಚ್ಚಿ
ಕಣ್ಮು-ಚ್ಚಿ-ಕೊಂಡೆ
ಕಣ್ಮು-ಚ್ಚಿ-ದನು
ಕಣ್ವ
ಕಣ್ವನು
ಕಣ್ವ-ಪು-ಷಿ-ಗಳ
ಕಣ್ವ-ವಂ-ಶದ
ಕಣ್ಸ-ನ್ನೆ-ಯಿಂದ
ಕತ-ಕತ
ಕತೆ
ಕತೆ-ಗಳನ್ನು
ಕತ್ತ
ಕತ್ತನ್ನು
ಕತ್ತ-ರಿ-ಸ-ಬೇಕು
ಕತ್ತ-ರಿ-ಸ-ಲೆಂದು
ಕತ್ತ-ರಿ-ಸ-ಹೊ-ರ-ಟನು
ಕತ್ತ-ರಿಸಿ
ಕತ್ತ-ರಿ-ಸಿದ
ಕತ್ತ-ರಿ-ಸಿ-ದ-ಮೇ-ಲೆಯೂ
ಕತ್ತ-ರಿ-ಸಿ-ದರೂ
ಕತ್ತ-ರಿ-ಸಿ-ದು-ದ-ಲ್ಲದೆ
ಕತ್ತ-ರಿ-ಸಿ-ರುವೆ
ಕತ್ತ-ರಿ-ಸಿ-ಹಾ-ಕ-ಬೇ-ಕೆಂ-ದು-ಕೊಂಡು
ಕತ್ತ-ರಿ-ಸಿ-ಹಾಕಿ
ಕತ್ತ-ರಿ-ಸಿ-ಹಾ-ಕಿತು
ಕತ್ತ-ರಿ-ಸಿ-ಹಾ-ಕಿತ್ತು
ಕತ್ತ-ರಿ-ಸಿ-ಹಾ-ಕಿದ
ಕತ್ತ-ರಿ-ಸಿ-ಹಾ-ಕಿ-ದನು
ಕತ್ತ-ರಿ-ಸಿ-ಹಾ-ಕಿ-ದನೂ
ಕತ್ತ-ರಿ-ಸಿ-ಹಾ-ಕಿ-ದಳು
ಕತ್ತ-ರಿ-ಸಿ-ಹಾ-ಕಿ-ದವು
ಕತ್ತ-ರಿ-ಸಿ-ಹಾ-ಕಿ-ದ್ದೇನೆ
ಕತ್ತ-ರಿ-ಸಿ-ಹಾಕು
ಕತ್ತ-ರಿ-ಸಿ-ಹಾ-ಕು-ತ್ತಿದ್ದ
ಕತ್ತ-ರಿ-ಸಿ-ಹಾ-ಕು-ವಂತೆ
ಕತ್ತ-ರಿ-ಸಿ-ಹಾ-ಕು-ವುದು
ಕತ್ತ-ರಿ-ಸಿ-ಹೋ-ಗಿ-ದೆಯೆ
ಕತ್ತ-ರಿ-ಸಿ-ಹೋ-ದುವು
ಕತ್ತ-ರಿ-ಸಿ-ಹೋ-ಯಿತು
ಕತ್ತ-ರಿ-ಸುತ್ತಾ
ಕತ್ತ-ರಿ-ಸು-ವಂತೆ
ಕತ್ತ-ರಿ-ಸು-ವುದು
ಕತ್ತ-ಲಾ-ಗು-ತ್ತಲೆ
ಕತ್ತಲು
ಕತ್ತಲೆ
ಕತ್ತ-ಲೆ-ಗಿಂ-ತಲೂ
ಕತ್ತ-ಲೆಯ
ಕತ್ತ-ಲೆ-ಯನ್ನು
ಕತ್ತ-ಲೆ-ಯಲ್ಲಿ
ಕತ್ತಿ
ಕತ್ತಿ-ಕ-ಟ್ಟು-ವುದು
ಕತ್ತಿ-ಗ-ಳೊ-ಡನೆ
ಕತ್ತಿಗೆ
ಕತ್ತಿ-ನಲ್ಲಿ
ಕತ್ತಿಯ
ಕತ್ತಿ-ಯಂ-ತಿದ್ದ
ಕತ್ತಿ-ಯಂತೆ
ಕತ್ತಿ-ಯನ್ನು
ಕತ್ತಿ-ಯಿಂದ
ಕತ್ತಿ-ಯಿಂ-ದಲೆ
ಕತ್ತಿ-ಯೊ-ಡನೆ
ಕತ್ತು
ಕತ್ತು-ಗಳನ್ನೆಲ್ಲ
ಕತ್ತೆ-ಗಿಂ-ತಲೂ
ಕತ್ತೆಯ
ಕತ್ತೆ-ಯಂ-ತೆಯೆ
ಕತ್ತೆ-ಯವೆ
ಕತ್ತೆ-ಯೇರಿ
ಕಥಂ
ಕಥನ
ಕಥ-ನಾ-ಮೃ-ತ-ವನ್ನು
ಕಥ-ಮಿ-ಹಾ-ಸ್ಮಾನ್
ಕಥಾ
ಕಥಾಂ
ಕಥಾ-ಭಾ-ಗ-ಗ-ಳಿಗೆ
ಕಥಾ-ಮೃತ
ಕಥಾ-ಮೃತಂ
ಕಥಾ-ಮೃ-ತ-ವನ್ನು
ಕಥಾ-ಶ್ರ-ವಣ
ಕಥೆ
ಕಥೆ-ಎ-ರಡೂ
ಕಥೆ-ಗಳನ್ನು
ಕಥೆ-ಗಳನ್ನೆಲ್ಲ
ಕಥೆ-ಗಳು
ಕಥೆ-ಗಳೂ
ಕಥೆಗೆ
ಕಥೆ-ಬ್ರ-ಹ್ಮನ
ಕಥೆಯ
ಕಥೆ-ಯಂ-ತಲ್ಲ
ಕಥೆ-ಯನ್ನು
ಕಥೆ-ಯ-ನ್ನೆಲ್ಲ
ಕಥೆ-ಯ-ನ್ನೆಲ್ಲಾ
ಕಥೆ-ಯಲ್ಲಿ
ಕಥೆ-ಯಷ್ಟೆ
ಕಥೆ-ಯಿಲ್ಲ
ಕಥೆ-ಯೆಲ್ಲ
ಕಥೆ-ಯೊಂದ
ಕಥೇ-ತಿ-ಹಾಸ
ಕಥೇ-ತಿ-ಹಾ-ಸ-ಗಳು
ಕಥೇ-ತಿ-ಹಾ-ಸ-ಗಳೂ
ಕಥೋ-ಪ-ನ್ಯಾ-ಸ-ಗಳ
ಕಥೋ-ಪಾ-ಖ್ಯಾ-ನ-ಗಳ
ಕದಂಬ
ಕದಡಿ
ಕದ-ಡಿ-ಸಿ-ದನು
ಕದ-ನ್ನವೊ
ಕದ-ನ್ನ-ವೋ-ಯಾರು
ಕದ-ಲದೆ
ಕದ-ಲ-ಲಾ-ರದೆ
ಕದ-ಲಿ-ಸು-ತ್ತಿ-ರು-ತ್ತದೆ
ಕದ-ಲಿ-ಸು-ವು-ದ-ಕ್ಕಾ-ಗ-ಲಿಲ್ಲ
ಕದಿ-ಯ-ದಿ-ರು-ವುದು
ಕದಿ-ಯ-ಹೊ-ರ-ಟನು
ಕದ್ದ
ಕದ್ದು
ಕದ್ದೊ
ಕದ್ದೊಯ್ದ
ಕದ್ದೊ-ಯ್ದನು
ಕದ್ದೊ-ಯ್ದ-ನೆಂಬ
ಕದ್ದೊ-ಯ್ದರು
ಕದ್ದೊಯ್ದು
ಕದ್ದೊ-ಯ್ದೆ-ಯ-ಲ್ಲವೆ
ಕದ್ದೊ-ಯ್ಯು-ತ್ತಿ-ದ್ದಾರೆ
ಕದ್ದೊ-ಯ್ಯುವ
ಕದ್ದೋ-ಡಿದ
ಕದ್ರು-ವಿನ
ಕನಲಿ
ಕನ-ಲಿದ
ಕನ-ವ-ರಿ-ಕೆ-ಯಲ್ಲಿ
ಕನ-ವ-ರಿಸಿ
ಕನ-ಸನ್ನು
ಕನ-ಸಾ-ಯಿತು
ಕನ-ಸಿಗೆ
ಕನ-ಸಿನ
ಕನ-ಸಿ-ನಲ್ಲಿ
ಕನ-ಸಿ-ನ-ಲ್ಲಿ-ರುವ
ಕನಸು
ಕನ-ಸು-ಎಂಬ
ಕನ-ಸು-ಗಳಿಂದ
ಕನ-ಸು-ಗ-ಳು-ಮೈ-ಗೆಲ್ಲ
ಕನಸೋ
ಕನಿ-ಕರ
ಕನಿ-ಕ-ರ-ದಿಂದ
ಕನಿ-ಕ-ರ-ವನ್ನೂ
ಕನಿ-ಕ-ರವೂ
ಕನಿ-ಷ್ಠಿ-ಕಾ-ಭ್ಯಾಂ
ಕನ್ನಡಿ
ಕನ್ನ-ಡಿ-ಗರ
ಕನ್ನ-ಡಿ-ಗ-ರಿಗೆ
ಕನ್ನ-ಡಿ-ಯಂ-ತಹ
ಕನ್ನ-ಡಿ-ಯಂತೆ
ಕನ್ನ-ಡಿ-ಯಲ್ಲಿ
ಕನ್ನೆ-ಏ-ನನ್ನೆ
ಕನ್ನೆ-ಯ-ರೊ-ಡನೆ
ಕನ್ನೆ-ಯಿಂದ
ಕನ್ನೈ-ದಿ-ಲೆ-ಗಳು
ಕನ್ಯಾ-ಕು-ಬ್ಜ-ದಲ್ಲಿ
ಕನ್ಯಾ-ಕು-ಮಾ-ರಿಗೆ
ಕನ್ಯಾ-ಮ-ಣಿ-ಯೊ-ಬ್ಬಳು
ಕನ್ಯಾ-ರ-ತ್ನ-ವನ್ನು
ಕನ್ಯಾ-ರ-ತ್ನವೇ
ಕನ್ಯಾ-ವ್ರ-ತ-ವನ್ನು
ಕನ್ಯೆಯ
ಕನ್ಯೆ-ಯನ್ನು
ಕನ್ಯೆ-ಯ-ರನ್ನು
ಕನ್ಯೆ-ಯರು
ಕನ್ಯೆ-ಯೆಂ-ಬು-ದನ್ನೆ
ಕಪಟ
ಕಪ-ಟ-ಗಾ-ನಕ್ಕೆ
ಕಪ-ಟ-ತ-ನಕ್ಕೆ
ಕಪ-ಟದ
ಕಪ-ಟ-ನಾ-ಟಕ
ಕಪ-ಟ-ನಾ-ಟ-ಕ-ಸೂ-ತ್ರ-ಧಾರಿ
ಕಪ-ಟ-ವಿ-ಲ್ಲದೆ
ಕಪ-ಟಿ-ಯಾದ
ಕಪಾಲ
ಕಪಾ-ಳಕ್ಕೆ
ಕಪಿ
ಕಪಿ-ಗಳ
ಕಪಿ-ಗಳಿಂದ
ಕಪಿ-ಯಂತೆ
ಕಪಿ-ಯಲ್ಲ
ಕಪಿ-ರಾ-ಜನ
ಕಪಿ-ರಾ-ಜನು
ಕಪಿಲ
ಕಪಿಲಃ
ಕಪಿ-ಲ-ನೆಂದು
ಕಪಿ-ಲ-ಪುಷಿ
ಕಪಿ-ಲ-ಪು-ಷಿ-ಗಳ
ಕಪಿ-ಲ-ಮ-ಹರ್ಷಿ
ಕಪಿ-ಲ-ಮ-ಹ-ರ್ಷಿ-ಗಳ
ಕಪಿ-ಲ-ಮ-ಹಾ-ಮು-ನಿಯು
ಕಪಿ-ಲ-ಮುನಿ
ಕಪಿ-ಲ-ಮು-ನಿಗೆ
ಕಪಿ-ಲ-ಮು-ನಿಯ
ಕಪಿ-ಲ-ಮು-ನಿ-ಯಿಂದ
ಕಪಿ-ಲ-ಮು-ನಿಯು
ಕಪಿ-ಲ-ಮು-ನಿಯೇ
ಕಪಿಲೆ
ಕಪಿ-ವೀ-ರ-ರನ್ನೂ
ಕಪಿ-ಸೇನೆ
ಕಪಿ-ಸೇ-ನೆಯ
ಕಪಿ-ಸೇ-ನೆ-ಯೊ-ಡನೆ
ಕಪೀಂದ್ರಂ
ಕಪ್ಪ
ಕಪ್ಪ-ಕಾ-ಣಿಕೆ
ಕಪ್ಪ-ಕಾ-ಣಿ-ಕೆ-ಗಳನ್ನು
ಕಪ್ಪ-ಕಾ-ಣಿ-ಕೆ-ಗ-ಳೊ-ಡನೆ
ಕಪ್ಪ-ಗಾಗಿ
ಕಪ್ಪಗೆ
ಕಪ್ಪಾ-ಗಿರ
ಕಪ್ಪಾ-ಗಿ-ರುವ
ಕಪ್ಪಾ-ಗಿ-ರು-ವು-ದ-ರಿಂದ
ಕಪ್ಪಾದ
ಕಪ್ಪಾ-ದವು
ಕಪ್ಪು
ಕಪ್ಪೆ
ಕಪ್ಪೆಗೆ
ಕಪ್ಪೆಯ
ಕಬಂ-ಧ-ನೆಂಬ
ಕಬ-ರೇಣ
ಕಬೀ-ರರು
ಕಬೋಜಿ
ಕಬೋ-ಜಿ-ಯಾ-ಗಿ-ರುವ
ಕಬ್ಬಿ-ಣದ
ಕಬ್ಬಿ-ಣ-ದಿಂದ
ಕಬ್ಬಿ-ಣ-ವನ್ನು
ಕಬ್ಬಿನ
ಕಮಂ-ಡ-ಲ-ವನ್ನು
ಕಮಂ-ಡ-ಲ-ವನ್ನೂ
ಕಮಂ-ಡಲು
ಕಮಂ-ಡ-ಲು-ವಿ-ನಿಂದ
ಕಮನಂ
ಕಮರಿ
ಕಮಲ
ಕಮ-ಲ-ಗಳ
ಕಮ-ಲ-ಗಳನ್ನು
ಕಮ-ಲದ
ಕಮ-ಲ-ದಂ-ತಹ
ಕಮ-ಲ-ದಂ-ತಿ-ರುವ
ಕಮ-ಲ-ದಂತೆ
ಕಮ-ಲ-ದಲ್ಲಿ
ಕಮ-ಲ-ನ-ಯನ
ಕಮ-ಲ-ನಾ-ಳದ
ಕಮ-ಲ-ನೇತ್ರ
ಕಮ-ಲ-ನೇ-ತ್ರ-ಗಳನ್ನು
ಕಮ-ಲ-ಪು-ಷ್ಪ-ಗಳ
ಕಮ-ಲ-ಪು-ಷ್ಪ-ಗಳು
ಕಮ-ಲ-ಮಾ-ಲೆ-ಗಳನ್ನು
ಕಮ-ಲ-ವನ್ನೇ
ಕಮ-ಲ-ವ-ರ್ಣದ
ಕಮ-ಲ-ವಿದೆ
ಕಮ-ಲ-ವೊಂದು
ಕಮ-ಲಾ-ನ-ನ-ಕಂ-ಜ-ರತಂ
ಕಮ-ಲಾ-ಸನ
ಕರ
ಕರ-ಕ-ಮಲ
ಕರ-ಕ-ಮ-ಲ-ದಿಂದ
ಕರ-ಕರ
ಕರ-ಕ-ರ-ನೆಂದು
ಕರ-ಗ-ತ-ವಾ-ಗು-ತ್ತದೆ
ಕರ-ಗದೆ
ಕರಗಿ
ಕರ-ಗಿತು
ಕರ-ಗಿದ
ಕರ-ಗಿ-ಸಿತು
ಕರ-ಗಿ-ಸುವ
ಕರ-ಗಿ-ಹೋ-ದವು
ಕರ-ಗಿ-ಹೋ-ಯಿತು
ಕರ-ಗು-ತ್ತಿದೆ
ಕರ-ಗು-ವಂತೆ
ಕರ-ಗು-ವು-ದಂತೆ
ಕರಡಿ
ಕರ-ಡಿಯ
ಕರ-ಡಿ-ಯಲ್ಲ
ಕರ-ಡಿ-ಯೊಂದು
ಕರ-ತ-ಲಾ-ಮ-ಲಕ
ಕರ-ತ-ಲಾ-ಮ-ಲ-ಕ-ವಾ-ಗಿ-ದ್ದವು
ಕರ-ತ-ಲಾ-ಮ-ಲ-ಕ-ವಾ-ಯಿತು
ಕರ-ನ್ಯಾಸಃ
ಕರ-ನ್ಯಾ-ಸ-ಗಳನ್ನು
ಕರ-ಭಾ-ಜನ
ಕರ-ವಂತೆ
ಕರ-ವಾ-ಗಿದೆ
ಕರ-ವಾ-ಗಿಯೂ
ಕರ-ವಾಣಿ
ಕರ-ವಾದ
ಕರಾ-ಳ-ರೂ-ಪದ
ಕರಿ-ಕಾ-ಗಿ-ದ್ದವು
ಕರಿದು
ಕರಿಯ
ಕರಿ-ಸಿದ
ಕರಿ-ಸು-ವ-ರಾ-ದ್ದ-ರಿಂದ
ಕರು
ಕರು-ಗಳ
ಕರು-ಗಳನ್ನು
ಕರು-ಗಳೂ
ಕರು-ಗ-ಳೊ-ಡನೆ
ಕರುಣಂ
ಕರು-ಣ-ರ-ಸ-ವನ್ನು
ಕರು-ಣಾ-ಕರ
ಕರು-ಣಾ-ಮ-ಯ-ನಾದ
ಕರು-ಣಾ-ಮಯಿ
ಕರು-ಣಾ-ಳು-ವಾದ
ಕರುಣಿ
ಕರು-ಣಿ-ಸ-ಲೇ-ಬೇ-ಕೆಂ-ದಿ-ದ್ದರೆ
ಕರು-ಣಿ-ಸ-ವಂತೆ
ಕರು-ಣಿಸಿ
ಕರು-ಣಿ-ಸಿ-ದನು
ಕರು-ಣಿ-ಸಿ-ದಿರಿ
ಕರು-ಣಿ-ಸಿ-ದ್ದನು
ಕರು-ಣಿ-ಸಿ-ರ-ಬೇಕು
ಕರು-ಣಿ-ಸಿ-ರು-ವುದೇ
ಕರು-ಣಿಸು
ಕರು-ಣಿ-ಸು-ವಂತೆ
ಕರು-ಣಿ-ಸು-ವೆನು
ಕರುಣೆ
ಕರು-ಣೆ-ಯಿಂದ
ಕರು-ಣೆಯೇ
ಕರು-ಬಿ-ಯಾನು
ಕರುಳು
ಕರು-ಳು-ಗಳ
ಕರು-ಳು-ಗಳನ್ನು
ಕರು-ವನ್ನು
ಕರು-ವಲ್ಲ
ಕರು-ವಾಗಿ
ಕರುವೂ
ಕರು-ಹಾಕಿ
ಕರೂಷ
ಕರೂ-ಷ-ದೇ-ಶದ
ಕರೆ
ಕರೆಗೆ
ಕರೆ-ತಂ-ದ-ನಲ್ಲಾ
ಕರೆ-ತಂ-ದನು
ಕರೆ-ತಂ-ದಳು
ಕರೆ-ತಂ-ದಿದ್ದ
ಕರೆ-ತಂ-ದಿ-ದ್ದಾರೆ
ಕರೆ-ತಂ-ದಿ-ದ್ದೇನೆ
ಕರೆ-ತಂ-ದಿ-ರು-ವು-ದಾಗಿ
ಕರೆ-ತಂದು
ಕರೆ-ತ-ರ-ಬಾ-ರದು
ಕರೆ-ತ-ರ-ಬೇ-ಕೆಂದು
ಕರೆ-ತ-ರಲಿ
ಕರೆ-ತ-ರಿಸ
ಕರೆ-ತ-ರಿ-ಸಿ-ದುದು
ಕರೆ-ತರು
ಕರೆ-ತ-ರು-ತ್ತಾರೆ
ಕರೆ-ತ-ರು-ತ್ತಿ-ರಲು
ಕರೆ-ತ-ರು-ತ್ತೇನೆ
ಕರೆ-ತ-ರು-ವಂತೆ
ಕರೆದ
ಕರೆ-ದನು
ಕರೆ-ದರು
ಕರೆ-ದರೆ
ಕರೆ-ದಳು
ಕರೆ-ದ-ವ-ರಾರು
ಕರೆದು
ಕರೆ-ದು-ಕೊಂಡ
ಕರೆ-ದು-ಕೊಂ-ಡರು
ಕರೆ-ದು-ಕೊಂ-ಡವು
ಕರೆ-ದು-ಕೊಂಡು
ಕರೆ-ದು-ಕೊಂ-ಡು-ಹೋಗಿ
ಕರೆ-ದು-ಕೊ-ಳ್ಳು-ತ್ತಿ-ದ್ದು-ದಾಗಿ
ಕರೆದೊ
ಕರೆ-ದೊಯ್ದ
ಕರೆ-ದೊ-ಯ್ದ-ನಂತೆ
ಕರೆ-ದೊ-ಯ್ದನು
ಕರೆ-ದೊ-ಯ್ದರು
ಕರೆ-ದೊ-ಯ್ದಳು
ಕರೆ-ದೊಯ್ದು
ಕರೆ-ದೊಯ್ಯಿ
ಕರೆ-ದೊಯ್ಯು
ಕರೆ-ದೊ-ಯ್ಯು-ತ್ತದೆ
ಕರೆ-ದೊ-ಯ್ಯು-ತ್ತದೋ
ಕರೆ-ದೊ-ಯ್ಯು-ತ್ತಿ-ರುವ
ಕರೆ-ದೊ-ಯ್ಯು-ವ-ನೆಂ-ಬು-ದನ್ನು
ಕರೆ-ದೊ-ಯ್ಯು-ವು-ದ-ಕ್ಕಾಗಿ
ಕರೆ-ಯ-ಕ-ಳು-ಹಿಸಿ
ಕರೆ-ಯ-ದಿ-ದ್ದರೂ
ಕರೆ-ಯದೆ
ಕರೆ-ಯಲು
ಕರೆ-ಯಿರಿ
ಕರೆಯು
ಕರೆ-ಯು-ತ್ತಾನೆ
ಕರೆ-ಯು-ತ್ತಾರೆ
ಕರೆ-ಯುತ್ತಿ
ಕರೆ-ಯು-ತ್ತಿದ್ದ
ಕರೆ-ಯು-ತ್ತಿ-ದ್ದರು
ಕರೆ-ಯು-ತ್ತಿ-ದ್ದ-ವರು
ಕರೆ-ಯು-ತ್ತಿ-ರಲು
ಕರೆ-ಯು-ತ್ತಿ-ರು-ವುದು
ಕರೆ-ಯು-ತ್ತೀ-ಯಲ್ಲ
ಕರೆ-ಯು-ತ್ತೇವೆ
ಕರೆ-ಯುವ
ಕರೆ-ಯು-ವರು
ಕರೆ-ಯು-ವು-ದಕ್ಕೆ
ಕರೆ-ಯು-ವುದು
ಕರೆ-ಯು-ವುದೆ
ಕರೆಸಿ
ಕರೆ-ಸಿ-ಕೊ-ಳ್ಳು-ವನು
ಕರೆ-ಸಿ-ದನು
ಕರೆ-ಸಿ-ದಾಗ
ಕರೆ-ಸಿ-ರು-ವುದು
ಕರೆ-ಸು-ತ್ತೇನೆ
ಕರ್ಕ-ಶ-ವಾಗಿ
ಕರ್ಣ
ಕರ್ಣನು
ಕರ್ಣಾ-ಟ-ಕದ
ಕರ್ಣಿ-ಕಾ-ಪುಷ್ಪ
ಕರ್ಣಿ-ಕೆ-ಪು-ತ-ಧಾಮ
ಕರ್ತ
ಕರ್ತ-ನಾ-ಗ-ದೆಯೆ
ಕರ್ತವ್ಯ
ಕರ್ತ-ವ್ಯ-ಕ್ಕಾಗಿ
ಕರ್ತ-ವ್ಯ-ವನ್ನು
ಕರ್ತ-ವ್ಯ-ವೇ-ನೆಂಬು
ಕರ್ತೃ
ಕರ್ತೃ-ಗಳನ್ನು
ಕರ್ತೃ-ತ್ವ-ವನ್ನು
ಕರ್ತೃ-ವನ್ನು
ಕರ್ತೃ-ವಿನ
ಕರ್ದಮ
ಕರ್ದ-ಮನ
ಕರ್ದ-ಮ-ನಿಗೆ
ಕರ್ದ-ಮ-ನಿಗೇ
ಕರ್ದ-ಮನು
ಕರ್ದ-ಮ-ಪ್ರಜಾ
ಕರ್ದ-ಮ-ಪ್ರ-ಜಾ-ಪ-ತಿಯು
ಕರ್ದಮಾ
ಕರ್ಮ
ಕರ್ಮ-ಇವು
ಕರ್ಮ-ಇ-ವು-ಗಳಿಂದ
ಕರ್ಮ-ಕಾಂ-ಡ-ವನ್ನು
ಕರ್ಮಕ್ಕೂ
ಕರ್ಮಕ್ಕೆ
ಕರ್ಮ-ಕ್ಷ-ಯ-ದಿಂದ
ಕರ್ಮ-ಕ್ಷೇತ್ರ
ಕರ್ಮ-ಕ್ಷೇ-ತ್ರ-ವಾದ
ಕರ್ಮ-ಗಳ
ಕರ್ಮ-ಗ-ಳನ್ನಾ
ಕರ್ಮ-ಗಳನ್ನು
ಕರ್ಮ-ಗಳನ್ನೂ
ಕರ್ಮ-ಗ-ಳನ್ನೆ
ಕರ್ಮ-ಗಳನ್ನೆಲ್ಲ
ಕರ್ಮ-ಗಳಲ್ಲಿ
ಕರ್ಮ-ಗಳಿಂದ
ಕರ್ಮ-ಗ-ಳಿಂ-ದಲೂ
ಕರ್ಮ-ಗ-ಳಿಗೆ
ಕರ್ಮ-ಗಳು
ಕರ್ಮ-ಗಳೂ
ಕರ್ಮ-ಗ-ಳೆ-ರಡೂ
ಕರ್ಮ-ಗ-ಳೆಲ್ಲ
ಕರ್ಮ-ಗ-ಳೆ-ಲ್ಲವೂ
ಕರ್ಮ-ಗಳೇ
ಕರ್ಮ-ಠ-ರಾದ
ಕರ್ಮದ
ಕರ್ಮ-ದತ್ತ
ಕರ್ಮ-ದಲ್ಲಿ
ಕರ್ಮ-ದಿಂದ
ಕರ್ಮ-ದಿಂ-ದಲೊ
ಕರ್ಮ-ನಾ-ಶಕ್ಕೆ
ಕರ್ಮ-ನಾ-ಶವೂ
ಕರ್ಮ-ನಿ-ರತ
ಕರ್ಮ-ಪ್ರ-ತಿ-ಪಾ-ದ-ಕ-ವೆಂ-ದು-ವೇದ
ಕರ್ಮ-ಫಲ
ಕರ್ಮ-ಫ-ಲ-ಗಳನ್ನು
ಕರ್ಮ-ಫ-ಲ-ವನ್ನು
ಕರ್ಮ-ಬಂ-ಧಕ್ಕೆ
ಕರ್ಮ-ಬಂ-ಧ-ವನ್ನು
ಕರ್ಮ-ಬಂ-ಧ-ವಾ-ಗ-ದಂತೆ
ಕರ್ಮ-ಬಂ-ಧಾತ್
ಕರ್ಮ-ಮಾ-ಡಿದ
ಕರ್ಮ-ಮಾ-ಡಿ-ಸು-ವ-ವನು
ಕರ್ಮ-ಮಾ-ಡು-ವ-ವ-ರಾ-ಗಲಿ
ಕರ್ಮ-ಮಾ-ಡು-ವು-ದಕ್ಕೆ
ಕರ್ಮ-ಮಾ-ಡು-ವುದು
ಕರ್ಮ-ಮಾರ್ಗ
ಕರ್ಮ-ಮಾ-ರ್ಗಕ್ಕೆ
ಕರ್ಮ-ಮಾ-ರ್ಗ-ದಲ್ಲಿ
ಕರ್ಮ-ಯೋ-ಗಜಂ
ಕರ್ಮ-ಯೋ-ಗ-ವಾಗಿ
ಕರ್ಮ-ರೂ-ಪದ
ಕರ್ಮ-ವನ್ನು
ಕರ್ಮ-ವ-ಶ-ನಾಗಿ
ಕರ್ಮ-ವಾ-ದರೆ
ಕರ್ಮ-ವಾ-ಸ-ನೆ-ಗಳನ್ನು
ಕರ್ಮ-ವಾ-ಸ-ನೆಯ
ಕರ್ಮ-ವಾ-ಸ-ನೆ-ಯನ್ನು
ಕರ್ಮ-ವಾ-ಸ-ನೆ-ಯಿಂದ
ಕರ್ಮ-ವಿ-ಮೋ-ಚ-ನೆ-ಯಾದ
ಕರ್ಮವು
ಕರ್ಮವೆ
ಕರ್ಮ-ವೆಂದು
ಕರ್ಮ-ವೆಲ್ಲ
ಕರ್ಮವೇ
ಕರ್ಮ-ವೊಂದೆ
ಕರ್ಮ-ಶು-ಕ್ಲಾಯ
ಕರ್ಮ-ಶೇಷ
ಕರ್ಮ-ಶೇ-ಷ-ವನ್ನು
ಕರ್ಮ-ಸಂ-ಬಂ-ಧ-ವಾದ
ಕರ್ಮ-ಸಾ-ರ್ಥ-ಕ-ನಾಗ
ಕರ್ಮಸು
ಕರ್ಮಾಂ-ತ-ರ-ಗಳನ್ನು
ಕರ್ಮಾ-ಚ-ರ-ಣೆಗೆ
ಕರ್ಮಾ-ಧೀನ
ಕರ್ಮಾ-ಶ-ಯಾನ್
ಕರ್ಮೇಂ
ಕರ್ಮೇಂ-ದ್ರಿಯ
ಕರ್ಮೇಂ-ದ್ರಿ-ಯ-ಗಳ
ಕರ್ಮೇಂ-ದ್ರಿ-ಯ-ಗಳನ್ನೂ
ಕರ್ಮೇಂ-ದ್ರಿ-ಯ-ಗಳು
ಕರ್ಮೇಂ-ದ್ರಿ-ಯ-ಗಳೂ
ಕರ್ಶಿತಾ
ಕರ್ಶಿ-ತಾಶ್ಚ
ಕಲ
ಕಲ-ಕಲ
ಕಲ-ಕಿ-ಹೊ-ಯಿತು
ಕಲ-ಕಿ-ಹೋ-ಗಿತ್ತು
ಕಲ-ರ-ವ-ದಿಂದ
ಕಲಶ
ಕಲ-ಶದ
ಕಲ-ಶ-ದಂ-ತಿ-ರುವ
ಕಲ-ಶ-ದಲ್ಲಿ
ಕಲ-ಶ-ದಿಂದ
ಕಲ-ಶ-ಪ್ರಾ-ಯ-ವಾ-ಗಿತ್ತು
ಕಲ-ಶ-ವನ್ನು
ಕಲ-ಹ-ದಿಂದ
ಕಲಾ-ಕೌ-ಶ-ಲದ
ಕಲಾ-ಕೌ-ಶ-ಲ್ಯ-ವ-ನ್ನೆಲ್ಲ
ಕಲಾ-ಚ-ತು-ರ-ನಾದ
ಕಲಾ-ದೇ-ವಿ-ಯನ್ನೂ
ಕಲಾ-ನೈ-ಪುಣ್ಯ
ಕಲಾ-ರೂ-ಪ-ಗಳು
ಕಲಿ
ಕಲಿಗೆ
ಕಲಿತ
ಕಲಿತಂ
ಕಲಿ-ತಂತೆ
ಕಲಿ-ತನ
ಕಲಿ-ತನು
ಕಲಿ-ತ-ವ-ನ-ಲ್ಲವೆ
ಕಲಿ-ತಿ-ರುವ
ಕಲಿತು
ಕಲಿ-ತು-ಕೊಂ-ಡನು
ಕಲಿ-ತು-ಕೊಂ-ಡೆನು
ಕಲಿತೆ
ಕಲಿ-ತೆ-ಯಪ್ಪ
ಕಲಿ-ದೋ-ಷ-ದಿಂದ
ಕಲಿ-ಪು-ರುಷ
ಕಲಿ-ಪು-ರು-ಷನು
ಕಲಿ-ಬಾ-ಧೆ-ಯನ್ನು
ಕಲಿಯ
ಕಲಿ-ಯನ್ನು
ಕಲಿ-ಯ-ಬ-ಹು-ದ-ಲ್ಲವೆ
ಕಲಿ-ಯ-ಬ-ಹುದು
ಕಲಿ-ಯ-ಬೇಕು
ಕಲಿಯು
ಕಲಿ-ಯುಗ
ಕಲಿ-ಯು-ಗಕ್ಕೆ
ಕಲಿ-ಯು-ಗದ
ಕಲಿ-ಯು-ಗ-ದಲ್ಲಿ
ಕಲಿ-ಯು-ತ್ತಿ-ರುವ
ಕಲಿ-ರೂ-ಪ-ನಾದ
ಕಲಿಸ
ಕಲಿ-ಸ-ಲೆಂದು
ಕಲಿಸಿ
ಕಲಿ-ಸಿ-ದರು
ಕಲಿ-ಸಿ-ದ-ವ-ರಲ್ಲ
ಕಲಿ-ಸು-ತ್ತಿದ್ದ
ಕಲೆತ
ಕಲೆತು
ಕಲೆ-ತು-ಹೋಗಿ
ಕಲೆ-ಯಲ್ಲಿ
ಕಲೆ-ಯು-ತ್ತೇವೆ
ಕಲೆ-ಸಿದ
ಕಲೇಃ
ಕಲ್ಕಿ
ಕಲ್ಕಿಯ
ಕಲ್ಕಿಯು
ಕಲ್ಕೀ
ಕಲ್ಪ
ಕಲ್ಪಕ್ಕೂ
ಕಲ್ಪ-ತ-ರು-ವಿ-ನಿಂದ
ಕಲ್ಪದ
ಕಲ್ಪ-ದಲ್ಲಿ
ಕಲ್ಪ-ದ-ಲ್ಲಿದ್ದ
ಕಲ್ಪ-ದ-ಲ್ಲಿಯೇ
ಕಲ್ಪನೆ
ಕಲ್ಪ-ಯಂ-ತೀ-ಷ್ಟ-ಮಿ-ಷ್ಟಾಃ
ಕಲ್ಪ-ಯಿ-ತ್ವೇ-ದ-ಮ-ಬ್ರ-ವೀತ್
ಕಲ್ಪ-ವೃಕ್ಷ
ಕಲ್ಪ-ವೃ-ಕ್ಷ-ಗಳೂ
ಕಲ್ಪ-ವೃ-ಕ್ಷ-ದಂತೆ
ಕಲ್ಪ-ವೃ-ಕ್ಷಾದಿ
ಕಲ್ಪಾಂ-ತ-ದ-ವರೆ-ಗಿ-ರಲಿ
ಕಲ್ಪಿ-ತ-ವಾ-ಗಿರು
ಕಲ್ಪಿ-ತ-ವಾ-ಗು-ತ್ತದೆ
ಕಲ್ಪಿ-ತ-ವಾದ
ಕಲ್ಪಿ-ಸ-ಲ್ಪ-ಟ್ಟಿದೆ
ಕಲ್ಪಿ-ಸ-ಹೊ-ರಟು
ಕಲ್ಪಿಸಿ
ಕಲ್ಪಿ-ಸಿ-ಕೊಂಡು
ಕಲ್ಪಿ-ಸಿ-ದರು
ಕಲ್ಮ-ಷ-ವನ್ನು
ಕಲ್ಮ-ಷಾ-ಪಹಂ
ಕಲ್ಮಾಷ
ಕಲ್ಮಾ-ಷ-ಪಾದ
ಕಲ್ಮಾ-ಷ-ಪಾ-ದನು
ಕಲ್ಮಾ-ಷ-ಪಾ-ದ-ನೆಂದು
ಕಲ್ಮಾ-ಷ-ಪಾ-ದ-ರಾ-ಕ್ಷಸ
ಕಲ್ಮಾ-ಷ-ಪಾ-ದ-ರಾ-ಜನ
ಕಲ್ಮಾ-ಷ-ಪಾ-ದ-ರಾ-ಯನು
ಕಲ್ಯಾಣ
ಕಲ್ಯಾ-ಣ-ಗುಣ
ಕಲ್ಯಾ-ಣ-ಗು-ಣ-ಗಳನ್ನು
ಕಲ್ಯಾ-ಣ-ಗು-ಣ-ಪ-ರಿ-ಪೂ-ರ್ಣ-ನಾದ
ಕಲ್ಯಾ-ಣ-ವಾ-ಗು-ತ್ತದೆ
ಕಲ್ಲನ್ನು
ಕಲ್ಲ-ನ್ನೆತ್ತಿ
ಕಲ್ಲ-ಮಳೆ
ಕಲ್ಲ-ಮೇಲೆ
ಕಲ್ಲಾಗಿ
ಕಲ್ಲಿನ
ಕಲ್ಲಿ-ನಷ್ಟು
ಕಲ್ಲಿ-ನಿಂದ
ಕಲ್ಲು
ಕಲ್ಲು-ಕೂಡ
ಕಲ್ಲು-ಗಳ
ಕಲ್ಲು-ಗಳು
ಕಲ್ಲು-ಬಂ-ಡೆ-ಗಳ
ಕಲ್ಲು-ಬಂ-ಡೆಯ
ಕಲ್ಲು-ಮು-ಳ್ಳ-ಗಳು
ಕಲ್ಲು-ಹಾ-ಕ-ಬೇಡಿ
ಕಲ್ಲೋ-ಲ-ವ-ನ್ನುಂ-ಟು-ಮಾ-ಡಿತು
ಕಲ್ಲೋ-ಲ-ವಾಗಿ
ಕಲ್ಲೋ-ಲ-ವಾ-ಗಿತ್ತು
ಕಳಂಕ
ಕಳಚಿ
ಕಳ-ಚಿ-ಕೊಂ-ಡಿ-ದ್ದೇನೆ
ಕಳ-ಚಿ-ಕೊಂಡು
ಕಳ-ಚಿ-ಕೊ-ಳ್ಳು-ವು-ದ-ಕ್ಕಾ-ಗು-ತ್ತ-ದೆಯೆ
ಕಳ-ವಳ
ಕಳ-ವ-ಳಕ್ಕೆ
ಕಳ-ವ-ಳ-ಗೊಂಡ
ಕಳ-ವ-ಳ-ಗೊಂ-ಡರು
ಕಳ-ವ-ಳ-ಗೊಂ-ಡಿದೆ
ಕಳ-ವ-ಳ-ಗೊಂ-ಡುದು
ಕಳ-ವ-ಳ-ಗೊ-ಳ್ಳು-ತ್ತಿ-ದ್ದರು
ಕಳ-ವ-ಳ-ದಿಂದ
ಕಳ-ವ-ಳ-ವನ್ನು
ಕಳ-ವ-ಳ-ವಾ-ಯಿತು
ಕಳ-ವ-ಳಿ-ಸಿ-ದರು
ಕಳ-ಶ-ಕ-ನ್ನ-ಡಿ-ಗಳನ್ನು
ಕಳಿಂಗ
ಕಳಿಂ-ಗನು
ಕಳಿಂ-ಗ-ರಾ-ಜನೂ
ಕಳಿಂ-ಗರು
ಕಳಿ-ಸಿ-ದನು
ಕಳಿ-ಸಿದ್ದ
ಕಳಿ-ಸಿ-ದ್ದಾರೆ
ಕಳಿ-ಸು-ತ್ತಾ-ರೆಯೆ
ಕಳು
ಕಳುಹಿ
ಕಳು-ಹಿಸಿ
ಕಳು-ಹಿ-ಸಿ-ಕೊ-ಟ್ಟನು
ಕಳು-ಹಿ-ಸಿದ
ಕಳು-ಹಿ-ಸಿ-ದಂತೆ
ಕಳು-ಹಿ-ಸಿ-ದ-ಎಲ
ಕಳು-ಹಿ-ಸಿ-ದನು
ಕಳು-ಹಿ-ಸಿ-ದನೊ
ಕಳು-ಹಿ-ಸಿ-ದ-ರಂತೂ
ಕಳು-ಹಿ-ಸಿ-ದರು
ಕಳು-ಹಿ-ಸಿ-ದಳು
ಕಳು-ಹಿ-ಸಿ-ದ-ವರು
ಕಳು-ಹಿ-ಸಿದ್ದ
ಕಳು-ಹಿ-ಸಿ-ದ್ದಾನೆ
ಕಳು-ಹಿ-ಸಿ-ದ್ದಾರೆ
ಕಳು-ಹಿ-ಸಿ-ದ್ದಾಳೆ
ಕಳು-ಹಿ-ಸಿ-ರ-ಬೇಕು
ಕಳು-ಹಿ-ಸಿ-ರು-ವಂತೆ
ಕಳು-ಹಿಸು
ಕಳು-ಹಿ-ಸು-ತ್ತೇನೆ
ಕಳು-ಹಿ-ಸು-ವಂತೆ
ಕಳು-ಹಿ-ಸೋಣ
ಕಳೆ-ಗಳ
ಕಳೆ-ಗಳು
ಕಳೆ-ಗುಂದಿ
ಕಳೆ-ಗುಂ-ದಿ-ದರು
ಕಳೆ-ಗೆ-ಟ್ಟಿತು
ಕಳೆ-ಗೆ-ಟ್ಟಿ-ರುವು
ಕಳೆದ
ಕಳೆ-ದಂತೆ
ಕಳೆ-ದ-ನಾ-ದರೂ
ಕಳೆ-ದನು
ಕಳೆ-ದ-ಮೇಲೆ
ಕಳೆ-ದರು
ಕಳೆ-ದರೂ
ಕಳೆ-ದವು
ಕಳೆದು
ಕಳೆ-ದು-ಕೊಂಡ
ಕಳೆ-ದು-ಕೊಂ-ಡನು
ಕಳೆ-ದು-ಕೊಂ-ಡರು
ಕಳೆ-ದು-ಕೊಂ-ಡರೆ
ಕಳೆ-ದು-ಕೊಂ-ಡ-ವ-ನಂತೆ
ಕಳೆ-ದು-ಕೊಂ-ಡ-ವರು
ಕಳೆ-ದು-ಕೊಂಡು
ಕಳೆ-ದು-ಕೊ-ಳ್ಳ-ಬೇಡ
ಕಳೆ-ದು-ಕೊ-ಳ್ಳು-ತ್ತ-ದೆಯೆ
ಕಳೆ-ದು-ಕೊ-ಳ್ಳು-ವುದ
ಕಳೆ-ದು-ಕೊ-ಳ್ಳು-ವು-ದ-ಕ್ಕಾಗಿ
ಕಳೆ-ದು-ಕೊ-ಳ್ಳು-ವುದು
ಕಳೆ-ದು-ಹೋಗಿ
ಕಳೆ-ದು-ಹೋ-ಗಿ-ತ್ತಾ-ದರೂ
ಕಳೆ-ದು-ಹೋ-ಗಿ-ದ್ದು-ದ-ರಿಂದ
ಕಳೆ-ದು-ಹೋ-ಗು-ವ-ವ-ರೆಗೂ
ಕಳೆ-ದು-ಹೋ-ಗು-ವು-ದಕ್ಕೆ
ಕಳೆ-ದು-ಹೋ-ಗು-ವುವು
ಕಳೆ-ದು-ಹೋದ
ಕಳೆ-ದು-ಹೋ-ದವು
ಕಳೆ-ದು-ಹೋ-ದುವು
ಕಳೆ-ದು-ಹೋ-ಯಿತು
ಕಳೆ-ಯನ್ನು
ಕಳೆ-ಯ-ಬಲ್ಲ
ಕಳೆ-ಯ-ಬೇ-ಕಾ-ಯಿತು
ಕಳೆ-ಯ-ಲಿಲ್ಲ
ಕಳೆ-ಯಲು
ಕಳೆಯು
ಕಳೆ-ಯು-ತ್ತಲೆ
ಕಳೆ-ಯು-ತ್ತಾನೆ
ಕಳೆ-ಯು-ತ್ತಿ-ದ್ದನು
ಕಳೆ-ಯು-ತ್ತಿ-ದ್ದಳು
ಕಳೆ-ಯು-ತ್ತಿ-ರಲು
ಕಳೆ-ಯುವ
ಕಳೆ-ಯು-ವ-ಷ್ಟ-ರಲ್ಲಿ
ಕಳೆ-ಯು-ವು-ದ-ಕ್ಕಾಗಿ
ಕಳೆ-ಯು-ವು-ದ-ರೊ-ಳ-ಗಾಗಿ
ಕಳೆ-ಯೇ-ರಿತು
ಕಳೇ-ಬ-ರ-ವನ್ನು
ಕಳ್ಳ
ಕಳ್ಳ-ಕಾ-ಕರ
ಕಳ್ಳ-ಕೊ-ರಮ
ಕಳ್ಳ-ತನ
ಕಳ್ಳ-ತ-ನಕ್ಕೂ
ಕಳ್ಳನ
ಕಳ್ಳ-ನಂತೆ
ಕಳ್ಳ-ನನ್ನು
ಕಳ್ಳ-ನಾ-ರೆಂ-ಬು-ದನ್ನು
ಕಳ್ಳ-ನಾ-ರೆಂ-ಬುದು
ಕಳ್ಳ-ನಿ-ಗಿಂ-ತಲೂ
ಕಳ್ಳ-ನೆಂದು
ಕಳ್ಳರ
ಕಳ್ಳ-ರನ್ನು
ಕಳ್ಳ-ರ-ಲ್ಲಿಯೇ
ಕಳ್ಳ-ರಾಗಿ
ಕಳ್ಳ-ರಾ-ಗಿ-ಹೋ-ಗು-ತ್ತಾರೆ
ಕಳ್ಳ-ರಾಟ
ಕಳ್ಳರು
ಕಳ್ಳು
ಕವಚ
ಕವ-ಚ-ಇ-ವು-ಗಳನ್ನು
ಕವ-ಚ-ಇವೂ
ಕವ-ಚ-ವನ್ನು
ಕವ-ಣೆ-ಯಲ್ಲಿ
ಕವ-ರಿ-ಕೊಂಡು
ಕವಳ
ಕವಿ
ಕವಿ-ಕರ್ಮ
ಕವಿ-ಗಳು
ಕವಿ-ದಿದೆ
ಕವಿ-ದಿ-ದ್ದರೂ
ಕವಿ-ಭಿ-ರೀ-ಡಿತಂ
ಕವಿಯ
ಕವಿಯು
ಕವಿ-ಯು-ವಂತೆ
ಕವಿ-ಯೊ-ಬ್ಬ-ನೆಂದೂ
ಕಶಿಪು
ಕಶಿ-ಪು-ವಿನ
ಕಶ್ಯಪ
ಕಶ್ಯ-ಪ-ಋಷಿ
ಕಶ್ಯ-ಪ-ಋ-ಷಿಯ
ಕಶ್ಯ-ಪ-ಋ-ಷಿಯು
ಕಶ್ಯ-ಪನ
ಕಶ್ಯ-ಪ-ನಿಗೂ
ಕಶ್ಯ-ಪನು
ಕಶ್ಯ-ಪ-ನೆಂಬ
ಕಶ್ಯ-ಪ-ಪು-ಷಿಗೆ
ಕಶ್ಯ-ಪ-ಪು-ಷಿಯ
ಕಶ್ಯ-ಪ-ಬ್ರ-ಹ್ಮನ
ಕಶ್ಯ-ಪ-ಬ್ರ-ಹ್ಮನು
ಕಶ್ಯ-ಪ-ಮ-ಹ-ರ್ಷಿ-ಯನ್ನು
ಕಶ್ಯ-ಪ-ಮು-ನಿಯು
ಕಶ್ಯ-ಪಾ-ಶ್ರ-ಮ-ದಲ್ಲಿ
ಕಷ್ಟ
ಕಷ್ಟ-ಗಳನ್ನು
ಕಷ್ಟ-ಗಳನ್ನೆಲ್ಲ
ಕಷ್ಟ-ಗಳು
ಕಷ್ಟ-ಗಳೂ
ಕಷ್ಟ-ದಲ್ಲಿ
ಕಷ್ಟ-ದಿಂದ
ಕಷ್ಟ-ಪ-ಟ್ಟರೂ
ಕಷ್ಟ-ಪಟ್ಟು
ಕಷ್ಟ-ವ-ನ್ನಾ-ದರೂ
ಕಷ್ಟ-ವನ್ನು
ಕಷ್ಟ-ವಲ್ಲ
ಕಷ್ಟ-ವಾ-ಗು-ತ್ತದೆ
ಕಷ್ಟ-ವಾ-ಗು-ತ್ತಿತ್ತು
ಕಷ್ಟ-ವಾದ್ದು
ಕಷ್ಟ-ವಾ-ಯಿತು
ಕಷ್ಟವೇ
ಕಷ್ಟ-ವೇನೂ
ಕಷ್ಟ-ಸಾಧ್ಯ
ಕಷ್ಟ-ಸಾ-ಧ್ಯ-ವಾ-ದುದು
ಕಸಿ-ವಿ-ಸಿ-ಯಾದಂ
ಕಸಿ-ವಿ-ಸಿ-ಯಾ-ಯಿತು
ಕಹ-ಳೆ-ಗಳು
ಕಾ
ಕಾಂಚನ
ಕಾಂಚಿ
ಕಾಂತ
ಕಾಂತಾಯ
ಕಾಂತಿ
ಕಾಂತಿ-ಯಂ-ತಿದೆ
ಕಾಂತಿ-ಯನ್ನು
ಕಾಂತಿ-ಯಿಂದ
ಕಾಂತಿ-ಯು-ಕ್ತ-ನಾಗಿ
ಕಾಂತಿ-ಯು-ಕ್ತ-ನಾ-ದನು
ಕಾಂತಿ-ಹೀ-ನ-ನಾ-ದನು
ಕಾಕರ
ಕಾಕುತ್ಸ್ಥ
ಕಾಗೆ
ಕಾಗೆ-ಯಂತೆ
ಕಾಚ-ಗ-ಳೊ-ಡನೆ
ಕಾಚಿನ್
ಕಾಟ
ಕಾಟಕ್ಕೆ
ಕಾಟ-ದಿಂದ
ಕಾಟ-ವನ್ನು
ಕಾಟ-ವಿಲ್ಲ
ಕಾಟ-ವಿ-ಲ್ಲ-ದಂತೆ
ಕಾಡ
ಕಾಡನ್ನು
ಕಾಡ-ನ್ನೆಲ್ಲ
ಕಾಡಾ-ನೆ-ಗ-ಳಿಗೆ
ಕಾಡಾ-ನೆ-ಯಂತೆ
ಕಾಡಿ-ಗ-ಟ್ಟಿ-ದು-ದುಈ
ಕಾಡಿಗೆ
ಕಾಡಿ-ಗೆಯ
ಕಾಡಿನ
ಕಾಡಿ-ನಂತೆ
ಕಾಡಿ-ನಲ್ಲಿ
ಕಾಡಿ-ನ-ಲ್ಲಿ-ರುವ
ಕಾಡು
ಕಾಡು-ಕಿ-ಚ್ಚನ್ನು
ಕಾಡು-ಕಿಚ್ಚಿ
ಕಾಡು-ಕಿ-ಚ್ಚಿಗೆ
ಕಾಡು-ಕಿ-ಚ್ಚಿ-ನಂತೆ
ಕಾಡು-ಕಿ-ಚ್ಚಿ-ನಿಂದ
ಕಾಡು-ಕಿಚ್ಚು
ಕಾಡು-ಗಳನ್ನೂ
ಕಾಡು-ಗಿ-ಚ್ಚಿ-ನಂ-ತಿತ್ತು
ಕಾಡು-ಗಿ-ಚ್ಚಿ-ನಲ್ಲಿ
ಕಾಡು-ಗಿಚ್ಚು
ಕಾಡು-ಗಿ-ಚ್ಚು-ಇ-ವು-ಗಳನ್ನೆಲ್ಲ
ಕಾಡು-ಗಿ-ಚ್ಚೊಂ-ದನ್ನು
ಕಾಡು-ತ್ತದೆ
ಕಾಡುತ್ತಾ
ಕಾಡು-ಪ್ರಾ-ಣಿಯ
ಕಾಡು-ಬೆ-ಟ್ಟ-ಗಳನ್ನೂ
ಕಾಡು-ಮೃ-ಗ-ಗಳ
ಕಾಡು-ಹಣ್ಣು
ಕಾಡು-ಹ-ಣ್ಣು-ಗಳಿಂದ
ಕಾಡು-ಹ-ರಟೆ
ಕಾಡು-ಹೆ-ಣ್ಣು-ಗಳನ್ನು
ಕಾಣ
ಕಾಣದ
ಕಾಣ-ದಂತಾ
ಕಾಣ-ದಂ-ತಾ-ಗಿ-ರುವ
ಕಾಣ-ದಂತೆ
ಕಾಣ-ದ-ವ-ನಂತೆ
ಕಾಣ-ದಷ್ಟು
ಕಾಣ-ದಾ-ಯಿತು
ಕಾಣ-ದಿ-ದ್ದ-ವರು
ಕಾಣ-ದಿ-ರಲು
ಕಾಣ-ದಿ-ರು-ವಾಗ
ಕಾಣದು
ಕಾಣ-ದು-ದ-ಕ್ಕಾಗಿ
ಕಾಣದೆ
ಕಾಣ-ಬ-ರ-ಲಿಲ್ಲ
ಕಾಣ-ಬರು
ಕಾಣ-ಬ-ರು-ತ್ತವೆ
ಕಾಣ-ಬ-ರು-ತ್ತಾನೆ
ಕಾಣ-ಬ-ರು-ತ್ತಿದೆ
ಕಾಣ-ಬ-ರು-ತ್ತಿ-ದ್ದಳು
ಕಾಣ-ಬ-ರುವ
ಕಾಣ-ಬ-ರು-ವಂತೆ
ಕಾಣ-ಬ-ರು-ವನು
ಕಾಣ-ಬ-ರು-ವು-ದಿಲ್ಲ
ಕಾಣ-ಬಾ-ರದ
ಕಾಣ-ಬೇಕು
ಕಾಣ-ಬೇ-ಕೆಂದು
ಕಾಣ-ಬೇ-ಕೆಂಬ
ಕಾಣ-ಲಾ-ರ-ದ-ವ-ನಾ-ಗಿದ್ದ
ಕಾಣ-ಲಿಲ್ಲ
ಕಾಣಲು
ಕಾಣ-ಲೆಂದು
ಕಾಣಿ
ಕಾಣಿಕೆ
ಕಾಣಿ-ಕೆ-ಗಳನ್ನು
ಕಾಣಿ-ಕೆ-ಗಳನ್ನೂ
ಕಾಣಿ-ಕೆ-ಯನ್ನು
ಕಾಣಿ-ಕೆ-ಯಾಗಿ
ಕಾಣಿ-ಸ-ದಿ-ರಲು
ಕಾಣಿ-ಸ-ಲಿಲ್ಲ
ಕಾಣಿ-ಸಲು
ಕಾಣಿ-ಸ-ಲೊಲ್ಲ
ಕಾಣಿಸಿ
ಕಾಣಿ-ಸಿ-ಕೊಂಡ
ಕಾಣಿ-ಸಿ-ಕೊಂ-ಡನು
ಕಾಣಿ-ಸಿ-ಕೊಂ-ಡರು
ಕಾಣಿ-ಸಿ-ಕೊಂ-ಡಳು
ಕಾಣಿ-ಸಿ-ಕೊಂ-ಡವು
ಕಾಣಿ-ಸಿ-ಕೊಂ-ಡಿತು
ಕಾಣಿ-ಸಿ-ಕೊಂ-ಡಿದೆ
ಕಾಣಿ-ಸಿ-ಕೊಂ-ಡಿವೆ
ಕಾಣಿ-ಸಿ-ಕೊಂಡು
ಕಾಣಿ-ಸಿ-ಕೊ-ಳ್ಳಲು
ಕಾಣಿ-ಸಿ-ಕೊ-ಳ್ಳು-ತ್ತವೆ
ಕಾಣಿ-ಸಿ-ಕೊ-ಳ್ಳು-ತ್ತಾನೆ
ಕಾಣಿ-ಸಿ-ಕೊ-ಳ್ಳು-ತ್ತಿ-ರುವ
ಕಾಣಿ-ಸಿ-ಕೊ-ಳ್ಳು-ತ್ತೇನೆ
ಕಾಣಿ-ಸಿತು
ಕಾಣಿ-ಸಿದ
ಕಾಣಿ-ಸಿ-ದರು
ಕಾಣಿ-ಸಿ-ದರೂ
ಕಾಣಿ-ಸಿ-ದಳು
ಕಾಣಿ-ಸಿ-ದವು
ಕಾಣಿ-ಸಿ-ದುದು
ಕಾಣಿ-ಸಿ-ದು-ದೆಲ್ಲ
ಕಾಣಿ-ಸಿ-ದುವು
ಕಾಣಿ-ಸಿದ್ದ
ಕಾಣಿ-ಸಿ-ರುವೆ
ಕಾಣಿ-ಸು-ತ್ತದೆ
ಕಾಣಿ-ಸು-ತ್ತವೆ
ಕಾಣಿ-ಸು-ತ್ತಿತ್ತು
ಕಾಣಿ-ಸು-ತ್ತಿದೆ
ಕಾಣಿ-ಸು-ತ್ತಿ-ದೆ-ಯಲ್ಲ
ಕಾಣಿ-ಸು-ತ್ತಿ-ದ್ದನು
ಕಾಣಿ-ಸು-ತ್ತಿ-ದ್ದುವು
ಕಾಣಿ-ಸು-ತ್ತಿ-ರ-ಲಿಲ್ಲ
ಕಾಣಿ-ಸು-ತ್ತಿ-ರುವ
ಕಾಣಿ-ಸು-ತ್ತಿಲ್ಲ
ಕಾಣಿ-ಸುವ
ಕಾಣಿ-ಸು-ವಂ-ತಿಲ್ಲ
ಕಾಣಿ-ಸು-ವು-ದಕ್ಕೆ
ಕಾಣಿ-ಸು-ವುದು
ಕಾಣು
ಕಾಣುತ್ತ
ಕಾಣು-ತ್ತದೆ
ಕಾಣು-ತ್ತಲೆ
ಕಾಣು-ತ್ತಲೇ
ಕಾಣು-ತ್ತವೆ
ಕಾಣುತ್ತಾ
ಕಾಣು-ತ್ತಾ-ನಲ್ಲ
ಕಾಣು-ತ್ತಾನೆ
ಕಾಣು-ತ್ತಾರೆ
ಕಾಣುತ್ತಿ
ಕಾಣು-ತ್ತಿತ್ತು
ಕಾಣು-ತ್ತಿದೆ
ಕಾಣು-ತ್ತಿ-ದೆ-ಯಲ್ಲಾ
ಕಾಣು-ತ್ತಿದ್ದ
ಕಾಣು-ತ್ತಿ-ದ್ದರು
ಕಾಣು-ತ್ತಿ-ದ್ದರೂ
ಕಾಣು-ತ್ತಿ-ದ್ದಾರೆ
ಕಾಣು-ತ್ತಿದ್ದೆ
ಕಾಣು-ತ್ತಿ-ರುವ
ಕಾಣು-ತ್ತಿ-ರು-ವಿರಿ
ಕಾಣು-ತ್ತಿಲ್ಲ
ಕಾಣು-ತ್ತಿವೆ
ಕಾಣು-ತ್ತೀ-ಯಲ್ಲ
ಕಾಣು-ತ್ತೇವೆ
ಕಾಣುವ
ಕಾಣು-ವಂ-ತಹ
ಕಾಣು-ವಂ-ತಾ-ಯಿತು
ಕಾಣು-ವಂ-ತಿ-ರ-ಲಿಲ್ಲ
ಕಾಣು-ವಂತೆ
ಕಾಣು-ವನು
ಕಾಣು-ವ-ವ-ನಂತೆ
ಕಾಣು-ವು-ದ-ಕ್ಕಾಗಿ
ಕಾಣು-ವು-ದಕ್ಕೆ
ಕಾಣು-ವು-ದರ
ಕಾಣು-ವು-ದ-ರಿಂದ
ಕಾಣು-ವು-ದಿಲ್ಲ
ಕಾಣು-ವು-ದಿ-ಲ್ಲ-ವಾ-ದರೂ
ಕಾಣು-ವುದು
ಕಾಣು-ವು-ದೆಂತು
ಕಾಣು-ವು-ದೆ-ಲ್ಲವೂ
ಕಾಣು-ವುದೇ
ಕಾಣು-ವೆ-ವಾ-ದರೂ
ಕಾಣೆ-ಯೇನು
ಕಾತ-ರ-ರಾ-ಗಿ-ದ್ದರು
ಕಾತ-ರಿ-ಸು-ತ್ತಿ-ದ್ದವು
ಕಾತ್ಯಾ-ಯಿನೀ
ಕಾದ
ಕಾದಳು
ಕಾದಾಟ
ಕಾದಾ-ಡು-ವಂ-ತಾ-ಯಿತು
ಕಾದಾ-ಡು-ವುದನ್ನು
ಕಾದಿತ್ತೋ
ಕಾದಿ-ದೆ-ಯೆಂದು
ಕಾದಿದ್ದ
ಕಾದಿ-ದ್ದ-ವರೆಲ್ಲ
ಕಾದಿ-ದ್ದಾರೆ
ಕಾದಿ-ದ್ದಾಳೆ
ಕಾದಿದ್ದು
ಕಾದಿ-ದ್ದುದು
ಕಾದಿ-ರ-ಬೇ-ಕಾ-ಗು-ತ್ತದೆ
ಕಾದಿ-ರು-ತ್ತದೆ
ಕಾದಿ-ರುವ
ಕಾದಿವೆ
ಕಾದಿ-ವೆಯೋ
ಕಾದು
ಕಾದು-ಕೊಂ-ಡಿ-ದ್ದಾರೆ
ಕಾದು-ತ್ತಿದೆ
ಕಾದು-ತ್ತಿ-ದ್ದರು
ಕಾದು-ವು-ದ-ಕ್ಕಾ-ಗಿಯೆ
ಕಾನ-ನದ
ಕಾನ-ನ-ದತ್ತ
ಕಾಪಾ
ಕಾಪಾ-ಡದ
ಕಾಪಾ-ಡ-ದಿ-ದ್ದರೆ
ಕಾಪಾ-ಡ-ಬೇಕು
ಕಾಪಾ-ಡ-ಬೇ-ಕೆಂದು
ಕಾಪಾ-ಡ-ಲಾ-ರದೆ
ಕಾಪಾ-ಡ-ಲಾ-ರನೆ
ಕಾಪಾ-ಡಲಿ
ಕಾಪಾ-ಡ-ಲೆಂದು
ಕಾಪಾಡಿ
ಕಾಪಾ-ಡಿ-ಕೊಂ-ಡಿ-ದ್ದಳು
ಕಾಪಾ-ಡಿದ
ಕಾಪಾ-ಡಿ-ದ-ನಂತೆ
ಕಾಪಾ-ಡಿ-ದನು
ಕಾಪಾ-ಡಿ-ದರು
ಕಾಪಾ-ಡಿ-ದಿರಿ
ಕಾಪಾ-ಡಿದೆ
ಕಾಪಾಡು
ಕಾಪಾ-ಡುತ್ತಾ
ಕಾಪಾ-ಡು-ತ್ತಾನೆ
ಕಾಪಾ-ಡು-ತ್ತಿ-ದ್ದನು
ಕಾಪಾ-ಡು-ತ್ತಿ-ರಪ್ಪ
ಕಾಪಾ-ಡು-ತ್ತಿ-ರು-ವನು
ಕಾಪಾ-ಡು-ತ್ತಿ-ರು-ವ-ವನು
ಕಾಪಾ-ಡು-ತ್ತೇನೆ
ಕಾಪಾ-ಡುವ
ಕಾಪಾ-ಡು-ವು-ದಾಗಿ
ಕಾಪಾ-ಡು-ವುದು
ಕಾಪಾ-ಲಿ-ಕ-ನಂತೆ
ಕಾಪಾ-ಲಿ-ಗ-ಳಾ-ಗ-ಲೆಂದು
ಕಾಪಾ-ಳಕ್ಕೆ
ಕಾಫಿ
ಕಾಮ
ಕಾಮ-ಕೇ-ಳಿಗೆ
ಕಾಮ-ಕೇ-ಳಿ-ಯ-ಲ್ಲಿ-ದ್ದುದು
ಕಾಮ-ಕೋಟಿ
ಕಾಮಕ್ಕೂ
ಕಾಮಕ್ಕೆ
ಕಾಮ-ಕ್ರೋ-ಧ-ವೆಂಬ
ಕಾಮ-ಕ್ರೋ-ಧಾದಿ
ಕಾಮ-ಕ್ರೋ-ಧಾ-ದಿ-ಗ-ಳಾ-ಗಲಿ
ಕಾಮ-ಗಳನ್ನು
ಕಾಮ-ಚಾ-ರಿಣಿ
ಕಾಮ-ಜ್ವ-ರ-ದಿಂದ
ಕಾಮ-ತೃ-ಪ್ತಿ-ಯಾ-ಗು-ತ್ತಲೆ
ಕಾಮದ
ಕಾಮದಂ
ಕಾಮ-ದಿಂದ
ಕಾಮ-ದೇ-ವಾತ್
ಕಾಮ-ಧೇನು
ಕಾಮ-ಧೇ-ನು-ವಂ-ತಿದ್ದ
ಕಾಮ-ಧೇ-ನು-ವನ್ನು
ಕಾಮ-ಧೇ-ನು-ವಿ-ನಂತೆ
ಕಾಮ-ಧೇ-ನುವು
ಕಾಮ-ಧೇ-ನುವೂ
ಕಾಮನ
ಕಾಮ-ಪ-ರಾ-ಯ-ಣ-ರಾಗಿ
ಕಾಮ-ಬಾಣ
ಕಾಮ-ಬಾಧೆ
ಕಾಮ-ಭೋ-ಗ-ವನ್ನು
ಕಾಮ-ಯಸೇ
ಕಾಮ-ಯಾ-ನಾಂ
ಕಾಮ-ಲೋ-ಭ-ಗ-ಳೆಂಬ
ಕಾಮ-ವನ್ನು
ಕಾಮ-ವಿ-ಕಾರ
ಕಾಮ-ವಿ-ಕಾ-ರಕ್ಕೆ
ಕಾಮ-ವಿ-ಕಾ-ರ-ವಿ-ಲ್ಲ-ದಂ-ತೆಯೂ
ಕಾಮ-ವಿ-ಕಾ-ರವೇ
ಕಾಮವೂ
ಕಾಮ-ವೆಂ-ಬುದು
ಕಾಮ-ವೆ-ಲ್ಲಿ-ಯದು
ಕಾಮ-ಸು-ಖ-ವನ್ನು
ಕಾಮ-ಸ್ವ-ಭಾ-ವ-ವನ್ನು
ಕಾಮಾಂ-ಧ-ನಾ-ಗಿದ್ದ
ಕಾಮಾ-ತು-ರ-ಳಾಗಿ
ಕಾಮಾ-ತು-ರ-ವಾ-ಯಿತು
ಕಾಮಾ-ಭಿ-ಲಾಷೆ
ಕಾಮಾಯ
ಕಾಮಿನಿ
ಕಾಮಿ-ಯ-ಲ್ಲ-ದಿ-ದ್ದರೂ
ಕಾಮಿ-ಯಾಗಿ
ಕಾಮಿ-ಸಿ-ಬಂದ
ಕಾಮುಕ
ಕಾಮು-ಕ-ನಂತೆ
ಕಾಮು-ಕ-ನಾದ
ಕಾಮು-ಕನು
ಕಾಮು-ಕರು
ಕಾಮು-ಕರೂ
ಕಾಮೋ-ದ್ರೇ-ಕ-ದಿಂದ
ಕಾಮೋ-ದ್ರೇ-ಕ-ವ-ನ್ನುಂ-ಟು-ಮಾ-ಡುವ
ಕಾಮೋ-ದ್ರೇ-ಕ-ವಾಗಿ
ಕಾಮ್ಯ
ಕಾಮ್ಯ-ಕ-ರ್ಮ-ಗಳನ್ನು
ಕಾಮ್ಯ-ಕ-ರ್ಮ-ಗ-ಳ-ಲ್ಲಿಯೇ
ಕಾಮ್ಯ-ಕ-ರ್ಮ-ವನ್ನು
ಕಾಮ್ಯ-ಫ-ಲ-ಕ್ಕಲ್ಲ
ಕಾಮ್ಯ-ಫ-ಲ-ಗಳನ್ನು
ಕಾಯ
ಕಾಯನ್ನು
ಕಾಯ-ಮ-ನಸ್ಸು
ಕಾಯಾ
ಕಾಯಿ
ಕಾಯು
ಕಾಯುತ್ತ
ಕಾಯು-ತ್ತದೆ
ಕಾಯುತ್ತಾ
ಕಾಯುತ್ತಿ
ಕಾಯು-ತ್ತಿದ್ದ
ಕಾಯು-ತ್ತಿ-ದ್ದನು
ಕಾಯು-ತ್ತಿ-ದ್ದ-ವರೆಲ್ಲ
ಕಾಯು-ತ್ತಿದ್ದೆ
ಕಾಯು-ತ್ತಿ-ದ್ದೇವೆ
ಕಾಯು-ತ್ತಿ-ರಲು
ಕಾಯು-ತ್ತಿ-ರು-ತ್ತದೆ
ಕಾಯುವ
ಕಾಯು-ವ-ವ-ನಾ-ಗಿದ್ದ
ಕಾಯು-ವ-ವನು
ಕಾಯು-ವ-ವ-ರಿಗೆ
ಕಾಯು-ವ-ವರು
ಕಾಯು-ವು-ದ-ಕ್ಕಾಗಿ
ಕಾಯು-ವುದು
ಕಾರ
ಕಾರಣ
ಕಾರ-ಣ-ಎಂಬ
ಕಾರ-ಣ-ಗಳನ್ನು
ಕಾರ-ಣ-ಗಳನ್ನೂ
ಕಾರ-ಣ-ಗಳಿಂದ
ಕಾರ-ಣ-ಗಳು
ಕಾರ-ಣ-ದಿಂದ
ಕಾರ-ಣ-ದಿಂ-ದಲೆ
ಕಾರ-ಣ-ದಿಂ-ದಲೇ
ಕಾರ-ಣ-ನಾಗಿ
ಕಾರ-ಣ-ನಾ-ಗಿ-ದ್ದರೂ
ಕಾರ-ಣ-ನಾ-ಗಿ-ದ್ದಾನೆ
ಕಾರ-ಣ-ನಾ-ಗಿ-ರುವೆ
ಕಾರ-ಣ-ನಾದ
ಕಾರ-ಣ-ನಾ-ದ-ನಲ್ಲಾ
ಕಾರ-ಣ-ನಾ-ದನು
ಕಾರ-ಣ-ನಾ-ದ-ವನು
ಕಾರ-ಣ-ನಾ-ದುದ
ಕಾರ-ಣನು
ಕಾರ-ಣ-ನೆಂ-ದೇಕೆ
ಕಾರ-ಣ-ಭೂ-ತ-ನಾಗಿ
ಕಾರ-ಣ-ಭೂ-ತ-ನಾದ
ಕಾರ-ಣ-ರಲ್ಲ
ಕಾರ-ಣ-ಳಾ-ಗ-ಲಾರೆ
ಕಾರ-ಣ-ವನ್ನು
ಕಾರ-ಣ-ವ-ಲ್ಲದೆ
ಕಾರ-ಣ-ವಾ-ಗಲೇ
ಕಾರ-ಣ-ವಾ-ಗಿ-ರ-ಬೇಕು
ಕಾರ-ಣ-ವಾ-ಗು-ತ್ತದೆ
ಕಾರ-ಣ-ವಾ-ಗುವ
ಕಾರ-ಣ-ವಾ-ಗು-ವು-ದಿಲ್ಲ
ಕಾರ-ಣ-ವಾ-ಗು-ವುದು
ಕಾರ-ಣ-ವಾದ
ಕಾರ-ಣ-ವಾ-ದಂ-ತಾ-ಗು-ತ್ತದೆ
ಕಾರ-ಣ-ವಾ-ದರೂ
ಕಾರ-ಣ-ವಾ-ದರೆ
ಕಾರ-ಣ-ವಾ-ದು-ದ-ರಿಂದ
ಕಾರ-ಣ-ವಾ-ದುದು
ಕಾರ-ಣ-ವಾ-ಯಿತು
ಕಾರ-ಣ-ವಿ-ಲ್ಲದೆ
ಕಾರ-ಣವೂ
ಕಾರ-ಣ-ವೆಂದರೆ
ಕಾರ-ಣ-ವೇ-ನಿ-ರ-ಬ-ಹು-ದೆಂದು
ಕಾರ-ಣ-ವೇನು
ಕಾರ-ಣ-ವೇ-ನೆಂ-ಬು-ದನ್ನು
ಕಾರ-ಣ-ವೇ-ನೆಂ-ಬುದು
ಕಾರಣೆ
ಕಾರದ
ಕಾರ-ದಿಂದ
ಕಾರ-ವನ್ನು
ಕಾರ-ವಾ-ಗಿ-ರು-ತ್ತದೆ
ಕಾರಾ-ಗೃಹ
ಕಾರಿ-ದನು
ಕಾರಿ-ದವು
ಕಾರುತ್ತಾ
ಕಾರ್ಖಾ-ನೆ-ಯಲ್ಲಿ
ಕಾರ್ತ-ವೀರ್ಯ
ಕಾರ್ತ-ವೀ-ರ್ಯನ
ಕಾರ್ತ-ವೀ-ರ್ಯ-ನನ್ನು
ಕಾರ್ತ-ವೀ-ರ್ಯನು
ಕಾರ್ತ-ವೀ-ರ್ಯಾರ್
ಕಾರ್ತ-ವೀ-ರ್ಯಾ-ರ್ಜುನ
ಕಾರ್ತ-ವೀ-ರ್ಯಾ-ರ್ಜು-ನನ
ಕಾರ್ತ-ವೀ-ರ್ಯಾ-ರ್ಜು-ನನು
ಕಾರ್ತೀ-ಕ-ಮಾ-ಸದ
ಕಾರ್ಯ
ಕಾರ್ಯ-ಕ-ಲಾ-ಪ-ಗಳನ್ನೂ
ಕಾರ್ಯ-ಕಾರಣ-ರೂ-ಪ-ವಾದ
ಕಾರ್ಯ-ಕ್ಕಾಗಿ
ಕಾರ್ಯಕ್ಕೂ
ಕಾರ್ಯಕ್ಕೆ
ಕಾರ್ಯಕ್ಕೇ
ಕಾರ್ಯ-ಕ್ರಮ
ಕಾರ್ಯ-ಕ್ರ-ಮ-ವನ್ನು
ಕಾರ್ಯ-ಗ-ಳತ್ತ
ಕಾರ್ಯ-ಗಳನ್ನು
ಕಾರ್ಯ-ಗಳನ್ನೆಲ್ಲ
ಕಾರ್ಯ-ಗಳಲ್ಲಿ
ಕಾರ್ಯ-ಗಳಿಂದ
ಕಾರ್ಯ-ಗಳು
ಕಾರ್ಯ-ದಲ್ಲಿ
ಕಾರ್ಯ-ದಿಂದ
ಕಾರ್ಯ-ದಿಂ-ದಲೆ
ಕಾರ್ಯ-ರೂ-ಪ-ವನ್ನೂ
ಕಾರ್ಯ-ವನ್ನು
ಕಾರ್ಯ-ವನ್ನೂ
ಕಾರ್ಯ-ವ-ನ್ನೆಲ್ಲ
ಕಾರ್ಯ-ವಾ-ಗ-ಲಿಲ್ಲ
ಕಾರ್ಯ-ವಾ-ಗು-ತ್ತಲೆ
ಕಾರ್ಯ-ವಾ-ದರೂ
ಕಾರ್ಯವು
ಕಾರ್ಯ-ವೇ-ನಾ-ದರೂ
ಕಾರ್ಯ-ಸಾ-ಧ-ಕ-ರಾ-ದ-ವರು
ಕಾರ್ಯ-ಸಾ-ಧ-ನೆಗೆ
ಕಾರ್ಯ-ಸಾ-ಧ-ನೆ-ಯಾ-ಗು-ತ್ತ-ದೆಯೆ
ಕಾರ್ಯಾ-ರ್ಥ-ವಾಗಿ
ಕಾರ್ಯೋ-ದ್ದೇ-ಶ-ದಿಂ-ದಲೇ
ಕಾಲ
ಕಾಲಂ-ದಿ-ಗೆ-ಗಳ
ಕಾಲಂ-ದುಗೆ
ಕಾಲಂ-ದು-ಗೆ-ಗಳ
ಕಾಲಂ-ದು-ಗೆ-ಗಳು
ಕಾಲಂ-ದು-ಗೆಯ
ಕಾಲ-ಎಂಬ
ಕಾಲ-ಕಾ-ಲಕ್ಕೆ
ಕಾಲಕ್ಕೆ
ಕಾಲ-ಕ್ರ-ಮ-ದಲ್ಲಿ
ಕಾಲ-ಗಳನ್ನು
ಕಾಲ-ಚಕ್ರ
ಕಾಲ-ಚ-ಕ್ರಕ್ಕೆ
ಕಾಲ-ಚ-ಕ್ರ-ಹೀ-ಗೆಂ-ದು-ಕೊಂಡು
ಕಾಲದ
ಕಾಲ-ದತ್ತ
ಕಾಲ-ದ-ಮೇಲೆ
ಕಾಲ-ದಲ್ಲಿ
ಕಾಲ-ದ-ಲ್ಲಿಯೂ
ಕಾಲ-ದ-ವ-ರೆಗೆ
ಕಾಲ-ದಿಂದ
ಕಾಲ-ದೋಷ
ಕಾಲದ್ದು
ಕಾಲ-ನೇ-ಮಿ-ಯೆಂ-ಬು-ವನು
ಕಾಲನ್ನು
ಕಾಲನ್ನೂ
ಕಾಲ-ನ್ನೆತ್ತಿ
ಕಾಲ-ಪು-ರುಷ
ಕಾಲ-ಪು-ರು-ಷನು
ಕಾಲ-ಪ್ರ-ವಾ-ಹ-ದಲ್ಲಿ
ಕಾಲ-ಮ-ಲಾ-ತ್ಪ್ರ-ಪಾತು
ಕಾಲ-ಮೀ-ರಿಲ್ಲ
ಕಾಲ-ಮೂರ್ತಿ
ಕಾಲ-ಮೇ-ಘ-ದಂತೆ
ಕಾಲ-ಯ-ಮ-ನಂತೆ
ಕಾಲ-ಯ-ವ-ನ-ನಿಗೆ
ಕಾಲ-ಯ-ವ-ನನು
ಕಾಲ-ಯ-ವ-ನನೂ
ಕಾಲ-ಯ-ವ-ನ-ನೆಂಬ
ಕಾಲ-ಯ-ವ-ನನ್ನು
ಕಾಲ-ರೂ-ಪ-ದಿಂದ
ಕಾಲ-ರೂ-ಪಿ-ಯಾದ
ಕಾಲಲ್ಲಿ
ಕಾಲ-ವನ್ನು
ಕಾಲ-ವನ್ನೂ
ಕಾಲ-ವ-ಲ್ಲದ
ಕಾಲ-ವ-ಶ-ದಿಂದ
ಕಾಲ-ವ-ಶ-ದಿಂ-ದಲೆ
ಕಾಲ-ವಾ-ಗಿದೆ
ಕಾಲ-ವಾ-ಗು-ತ್ತಲೆ
ಕಾಲ-ವಾ-ಗು-ತ್ತಿ-ದ್ದರು
ಕಾಲ-ವಾ-ದ-ಮೇಲೆ
ಕಾಲ-ವಿ-ಭಾ-ಗ-ಗಳು
ಕಾಲ-ವಿ-ಲ್ಲ-ಎಂ-ಬು-ದನ್ನು
ಕಾಲವು
ಕಾಲವೂ
ಕಾಲ-ವೆಂಬ
ಕಾಲವೇ
ಕಾಲ-ವೇನೂ
ಕಾಲ-ಸ್ವ-ರೂ-ಪಿಯೂ
ಕಾಲಾ-ನಂ-ತರ
ಕಾಲಾ-ಳು-ಗಳು
ಕಾಲಾ-ವ-ಕಾ-ಶ-ವಿತ್ತು
ಕಾಲಾಽಯ
ಕಾಲಿ
ಕಾಲಿಂದ
ಕಾಲಿ-ಕ್ಕು-ತ್ತಿ-ದ್ದಂ-ತೆಯೆ
ಕಾಲಿಗೆ
ಕಾಲಿ-ಗೆ-ರ-ಗಿ-ದನು
ಕಾಲಿ-ಟ್ಟರು
ಕಾಲಿ-ಡ-ಬಾ-ರದು
ಕಾಲಿ-ಡು-ತ್ತದೆ
ಕಾಲಿ-ಡು-ತ್ತಿ-ದಂ-ತೆಯೇ
ಕಾಲಿ-ಡು-ತ್ತಿ-ದ್ದಂತೆ
ಕಾಲಿ-ಡು-ತ್ತಿ-ದ್ದಂ-ತೆಯೇ
ಕಾಲಿನ
ಕಾಲಿ-ನಿಂದ
ಕಾಲಿ-ನಿಂ-ದಲೂ
ಕಾಲಿ-ನಿಂ-ದೊದ್ದು
ಕಾಲು
ಕಾಲು-ಗಳನ್ನು
ಕಾಲು-ಗಳಲ್ಲಿ
ಕಾಲು-ಗ-ಳ-ಲ್ಲಿ-ರುವ
ಕಾಲು-ಗ-ಳಿಗೆ
ಕಾಲು-ಗಳು
ಕಾಲು-ವೆ-ಗಳು
ಕಾಲೆತ್ತಿ
ಕಾಲೆ-ರ-ಡನ್ನೂ
ಕಾಲೆಳೆ
ಕಾಲೊ-ತ್ತ-ಬೇಕು
ಕಾಲೊತ್ತಿ
ಕಾಲೊ-ತ್ತುತ್ತ
ಕಾಲ್ಕ-ಡ-ಗ-ಗಳ
ಕಾಲ್ಕೆ-ರೆದು
ಕಾಲ್ಗ-ಡ-ಗ-ಹೀಗೆ
ಕಾಲ್ಗೆ-ಜ್ಜೆ-ಗಳನ್ನು
ಕಾಲ್ಗೆ-ಜ್ಜೆ-ಗಳು
ಕಾಲ್ಗೆ-ಜ್ಞೆ-ಯನ್ನು
ಕಾಲ್ತು-ಳಿ-ತಕ್ಕೆ
ಕಾಲ್ತು-ಳಿ-ತ-ದಿಂದ
ಕಾಲ್ತೊ-ಳೆದು
ಕಾಲ್ನ-ಡಿ-ಗೆಗೆ
ಕಾಲ್ನ-ಡಿ-ಗೆ-ಯ-ಲ್ಲಿಯೆ
ಕಾಲ್ಮುಟ್ಟಿ
ಕಾಳಗ
ಕಾಳ-ಗದ
ಕಾಳ-ಗ-ದಲ್ಲಿ
ಕಾಳ-ಗ-ವನ್ನು
ಕಾಳ-ಗ-ವಾಡಿ
ಕಾಳ-ಗ-ವಿ-ಲ್ಲ-ದಾಗ
ಕಾಳ-ರಾ-ತ್ರಿ-ಯಲ್ಲಿ
ಕಾಳ-ಸ-ರ್ಪ-ದಂತೆ
ಕಾಳಿ
ಕಾಳಿಂದಿ
ಕಾಳಿಂ-ದಿ-ಕವಿ
ಕಾಳಿಂ-ದಿಯ
ಕಾಳಿಂ-ದಿ-ಯೊ-ಡನೆ
ಕಾಳಿ-ಕಾ-ದೇ-ವಿಯ
ಕಾಳಿಯ
ಕಾಳಿ-ಯನ
ಕಾಳಿ-ಯ-ನನ್ನು
ಕಾಳಿ-ಯ-ನ-ಮೇಲೆ
ಕಾಳಿ-ಯ-ನಿಗೆ
ಕಾಳಿ-ಯನು
ಕಾಳಿ-ಯನೂ
ಕಾಳಿ-ಯ-ನೆಂಬ
ಕಾಳಿ-ಯ-ನೇನೂ
ಕಾಳಿ-ಯನ್ನು
ಕಾಳೀ-ಯ-ನನ್ನು
ಕಾಳು-ಕ-ಡ್ಡಿ-ಯನ್ನೊ
ಕಾವ-ಲಾಗಿ
ಕಾವ-ಲಿದ್ದ
ಕಾವಲು
ಕಾವ-ಲು-ಗಾ-ರ-ರನ್ನು
ಕಾವ-ಲು-ಗಾ-ರ-ರ-ನ್ನೆಲ್ಲ
ಕಾವ-ಲೇನೂ
ಕಾವಿ-ನಲ್ಲಿ
ಕಾವಿ-ನ-ಲ್ಲಿದ್ದ
ಕಾವಿ-ನಿಂದ
ಕಾವು
ಕಾವೇರಿ
ಕಾವ್ಯ-ಮ-ಯ-ವಾದ
ಕಾವ್ಯಾ-ರ-ಸಾ-ಸ್ವಾ-ದ-ನ-ಪ-ಟು-ಗ-ಳಿಗೆ
ಕಾಶಿ
ಕಾಶಿಗೆ
ಕಾಶಿ-ರಾ-ಜನ
ಕಾಶೀ
ಕಾಶೀ-ರಾ-ಜನ
ಕಾಶೀ-ರಾ-ಜನು
ಕಾಶ್ಯ-ಪನ
ಕಾಶ್ಯ-ಪ-ನೆಂಬ
ಕಾಸಿದ
ಕಾಸ್ತ್ರಿ-ಯ-ಸ್ತ-ದ್ದು-ರಾ-ಪಾಃ
ಕಿಂ
ಕಿಂಕ-ರ-ನಾದ
ಕಿಂಕ-ರೀ-ಣಾಂ
ಕಿಂಕ-ರ್ತ-ವ್ಯ-ಮೂಢ
ಕಿಂಪು-ರುಷ
ಕಿಂಪು-ರು-ಷ-ರನ್ನೂ
ಕಿಂಪು-ರು-ಷರು
ಕಿಂಪು-ರು-ಷರೂ
ಕಿಂಪು-ರು-ಷ-ರೊ-ಡನೆ
ಕಿಂಪು-ರು-ಷ-ವರ್ಷ
ಕಿಂಪು-ರು-ಷ-ವ-ರ್ಷ-ದಲ್ಲಿ
ಕಿಂಪು-ರು-ಷಾ-ದಿ-ಗ-ಳಿಗೆ
ಕಿಕ್ಕಿ-ರಿದು
ಕಿಚ್ಚು
ಕಿಚ್ಚೂ
ಕಿಟ-ಕಿ-ಯಿಂದ
ಕಿಟ್ಟನೆ
ಕಿಡಿ
ಕಿಡಿ-ಕಿ-ಡಿ-ಯಾಗಿ
ಕಿಡಿ-ಕಿ-ಡಿ-ಯಾದ
ಕಿಡಿ-ಕಿ-ಡಿ-ಯಾ-ದನು
ಕಿಡಿ-ಗ-ರೆವ
ಕಿಡಿ-ಗಳನ್ನು
ಕಿಡಿ-ಗ-ಳ-ನ್ನು-ಗು-ಳುತ್ತಾ
ಕಿಡಿ-ಗ-ಳ-ನ್ನು-ದು-ರಿ-ಸುತ್ತಾ
ಕಿಡಿ-ಗಳು
ಕಿಡಿ-ಗ-ಳು-ದು-ರಿ-ದವು
ಕಿಡಿ-ಗಾ-ರುತ್ತಾ
ಕಿಡಿ-ಗಿಂ-ತಲೂ
ಕಿಡಿ-ಗೆ-ದ-ರುತ್ತಾ
ಕಿಡಿ-ಗೇ-ಡಿ-ಗಳು
ಕಿಡಿ-ಗೇ-ಡಿ-ತನ
ಕಿಡಿ-ಗೇ-ಡಿ-ತ-ನ-ವ-ನ್ನೆ-ಲ್ಲ-ಪಾಂ-ಡ-ವ-ರಿಗೆ
ಕಿಡಿ-ನುಡಿ
ಕಿಡಿ-ನು-ಡಿ-ಗಳಿಂದ
ಕಿಡು-ನು-ಡಿ-ಗಳನ್ನು
ಕಿತ-ವ-ಬಂಧೋ
ಕಿತು
ಕಿತ್ತನು
ಕಿತ್ತಾ-ಡು-ತ್ತಿದ್ದ
ಕಿತ್ತಾ-ಡು-ತ್ತಿ-ರು-ವಾಗ
ಕಿತ್ತಾ-ಡು-ವು-ದಕ್ಕೆ
ಕಿತ್ತಾ-ಡು-ವುದನ್ನು
ಕಿತ್ತಾ-ಡು-ವುದು
ಕಿತ್ತು
ಕಿತ್ತು-ಕೊಂ-ಡಿತು
ಕಿತ್ತು-ಕೊಂ-ಡಿ-ದ್ದಾರೆ
ಕಿತ್ತು-ಕೊಂಡು
ಕಿತ್ತು-ಕೊ-ಳ್ಳಲು
ಕಿತ್ತು-ಕೊ-ಳ್ಳು-ತ್ತ-ದೆಯೆ
ಕಿತ್ತು-ಕೊ-ಳ್ಳು-ತ್ತಿ-ದ್ದನು
ಕಿತ್ತು-ಕೊ-ಳ್ಳು-ತ್ತಿ-ರುವೆ
ಕಿತ್ತು-ಕೊ-ಳ್ಳು-ವು-ದಕ್ಕೆ
ಕಿತ್ತು-ಬಂದು
ಕಿತ್ತು-ಹಾಕಿ
ಕಿತ್ತು-ಹಾ-ಕಿ-ದನು
ಕಿತ್ತು-ಹಾಕು
ಕಿತ್ತು-ಹಾ-ಕು-ವು-ದ-ಕ್ಕಾಗಿ
ಕಿತ್ತು-ಹಾ-ಕೋಣ
ಕಿತ್ತೆತ್ತಿ
ಕಿತ್ತೆ-ಸೆ-ದನು
ಕಿತ್ತೆ-ಸೆ-ದರು
ಕಿತ್ತೆ-ಸೆದು
ಕಿತ್ತೆ-ಸೆ-ಯಲು
ಕಿತ್ತೆ-ಸೆ-ಯು-ವಂತೆ
ಕಿನ್ನರ
ಕಿನ್ನು
ಕಿಪೂ
ಕಿಮದ್ಯ
ಕಿಮ-ನು-ರುಂಧೇ
ಕಿಮಿಹ
ಕಿರ-ಚಿ-ಕೊಂಡು
ಕಿರಣ
ಕಿರ-ಣ-ಗಳಿಂದ
ಕಿರಾತ
ಕಿರಾ-ತ-ರೂ-ಪಿ-ನಿಂದ
ಕಿರಿ-ಕಿ-ರಿಯೇ
ಕಿರಿ-ಚಲು
ಕಿರಿ-ಚಿ-ಕೊಂಡ
ಕಿರಿ-ಚಿ-ಕೊಂ-ಡನು
ಕಿರಿ-ಚಿ-ಕೊಂ-ಡಳು
ಕಿರಿ-ಚಿ-ಕೊಂ-ಡವು
ಕಿರಿ-ಚಿ-ಕೊಂ-ಡಿತು
ಕಿರಿ-ಚಿ-ಕೊಂಡು
ಕಿರಿ-ಚಿ-ಕೊ-ಳ್ಳು-ವು-ದಕ್ಕೆ
ಕಿರಿದು
ಕಿರಿಯ
ಕಿರಿ-ಯ-ನಾಗಿ
ಕಿರಿ-ಯ-ನಾದ
ಕಿರಿ-ಯ-ನಾ-ದ-ವನು
ಕಿರಿ-ಯ-ರಾದ
ಕಿರಿ-ಯ-ರೆಲ್ಲ
ಕಿರಿ-ಯ-ವ-ನಾದ
ಕಿರೀಟ
ಕಿರೀ-ಟ-ಒಂ-ದೊಂದೂ
ಕಿರೀ-ಟ-ಕುಂ-ಡ-ಲ-ಗಳು
ಕಿರೀ-ಟ-ಗಳು
ಕಿರೀ-ಟ-ಗಳೂ
ಕಿರೀ-ಟ-ವನ್ನು
ಕಿರೀ-ಟ-ವನ್ನೂ
ಕಿರೀ-ಟ-ವಾ-ಗು-ತ್ತೇ-ನೆ-ನ್ನು-ವು-ದಲ್ಲ
ಕಿರು-ಕು-ಳ-ಗ-ಳೆ-ಲ್ಲವೂ
ಕಿರು-ಗೆ-ಜ್ಜೆ-ಗಳು
ಕಿರು-ಗೆ-ಜ್ಜೆಯ
ಕಿರು-ಚಿ-ಕೊ-ಳ್ಳು-ವುದೂ
ಕಿರು-ಚುತ್ತಾ
ಕಿರು-ನಗೆ
ಕಿರು-ನ-ಗೆ-ಗಳು
ಕಿರು-ನ-ಗೆ-ಯನ್ನು
ಕಿರು-ನ-ಗೆ-ಯೊ-ಡನೆ
ಕಿರು-ನಗ್
ಕಿಲಿ-ಕಿಲಿ
ಕಿವಿ
ಕಿವಿ-ಗಳನ್ನು
ಕಿವಿ-ಗಳಿಂದ
ಕಿವಿ-ಗ-ಳಿಗೆ
ಕಿವಿ-ಗಳು
ಕಿವಿ-ಗಳೆ
ಕಿವಿ-ಗ-ಳೆ-ರಡೂ
ಕಿವಿ-ಗಳೊ
ಕಿವಿಗೂ
ಕಿವಿಗೆ
ಕಿವಿಯ
ಕಿವಿ-ಯಲ್ಲಿ
ಕಿವಿ-ಯಾಗಿ
ಕಿವಿ-ಯಾರೆ
ಕಿವಿ-ಯಿಂದ
ಕಿವಿ-ಯಿ-ದ್ದರೂ
ಕಿವಿ-ಯೊ-ಡನೆ
ಕಿವಿ-ಯೋ-ಲೆ-ಗಳನ್ನು
ಕಿವಿ-ಯೋ-ಲೆ-ಗಳೆ
ಕಿವು-ಡ-ನಂತೆ
ಕಿವು-ಡಾ-ಗು-ತ್ತಿ-ದ್ದವು
ಕಿವುಡು
ಕಿವು-ಡು-ಗೇಳಿ
ಕೀಟ-ಲೆ-ಮಾಡಿ
ಕೀರ್ತನ
ಕೀರ್ತನೆ
ಕೀರ್ತ-ನೆ-ಗಳಿಂದ
ಕೀರ್ತ-ನೆಗೂ
ಕೀರ್ತಿ
ಕೀರ್ತಿ-ಎ-ಲ್ಲವೂ
ಕೀರ್ತಿ-ಗ-ಳಿಗೆ
ಕೀರ್ತಿ-ಗಳು
ಕೀರ್ತಿಗೆ
ಕೀರ್ತಿ-ಮಂತ
ಕೀರ್ತಿಯ
ಕೀರ್ತಿ-ಯನ್ನು
ಕೀರ್ತಿ-ಯನ್ನೊ
ಕೀರ್ತಿಯೂ
ಕೀರ್ತಿ-ಲಕ್ಷ್ಮಿ
ಕೀರ್ತಿ-ವಂತ
ಕೀರ್ತಿ-ವಂ-ತ-ನೆಂದು
ಕೀರ್ತಿ-ಶಾಲಿ
ಕೀರ್ತಿ-ಶಾ-ಲಿ-ಗ-ಳಾಗಿ
ಕೀರ್ತಿ-ಶಾ-ಲಿ-ಗ-ಳಾ-ಗಿರಿ
ಕೀರ್ತಿ-ಶಾ-ಲಿ-ಗ-ಳಾ-ಗು-ವರು
ಕೀರ್ತಿ-ಶಾ-ಲಿ-ಯಾ-ಗು-ವನು
ಕೀರ್ತಿ-ಶಾ-ಲಿ-ಯಾ-ದನು
ಕೀರ್ತಿ-ಸಿ-ದರೆ
ಕೀರ್ತಿ-ಸು-ತ್ತಾರೆ
ಕೀಲಕಂ
ಕೀಲು-ಗ-ಳೆಲ್ಲ
ಕೀಳಲು
ಕೀಳಾಗಿ
ಕೀಳಾ-ಗಿದ್ದ
ಕೀಳಾದ
ಕೀಳಾ-ದ-ವರು
ಕೀಳಾ-ಯಿತು
ಕೀಳು
ಕೀಳು-ವಂತೆ
ಕೀಳು-ವಸ್ತು
ಕೀಳು-ಹಂ-ದಿಯ
ಕೀಳೆಂದು
ಕುಂಕು-ಮಕ್ಕೆ
ಕುಂಕು-ಮ-ಗ-ಳಿಂ-ದಲೂ
ಕುಂಕು-ಮದ
ಕುಂಟುತ್ತಾ
ಕುಂಡ-ದಿಂದ
ಕುಂಡಲ
ಕುಂಡ-ಲ-ಗಳು
ಕುಂಡಿನ
ಕುಂಡಿ-ನ-ಪು-ರ-ದಲ್ಲಿ
ಕುಂತಿ
ಕುಂತಿ-ದೇ-ವಿಯ
ಕುಂತಿ-ದೇ-ವಿಯೂ
ಕುಂತಿಯ
ಕುಂತಿ-ಯಂತೂ
ಕುಂತಿ-ಯನ್ನೂ
ಕುಂತಿ-ಯರ
ಕುಂತಿಯೂ
ಕುಂತೀ
ಕುಂತೀ-ದೇವಿ
ಕುಂತೀ-ರಾ-ಜನ
ಕುಂದಿತು
ಕುಂದಿ-ದು-ದನ್ನು
ಕುಂದಿ-ಸಿದ
ಕುಂದಿ-ಹೋ-ಯಿತು
ಕುಂದು-ತ್ತದೆ
ಕುಂದುತ್ತಾ
ಕುಂದು-ತ್ತಾ-ಬಂತು
ಕುಂಭ
ಕುಂಭ-ಕರ್ಣ
ಕುಂಭಾಂಡ
ಕುಕ್ಕಿ
ಕುಕ್ಕಿ-ದುದು
ಕುಗ್ಗ-ಬಾ-ರದು
ಕುಗ್ಗಿತು
ಕುಗ್ಗಿ-ಸಿ-ದನು
ಕುಗ್ಗು-ತ್ತಲೆ
ಕುಗ್ಗುತ್ತಾ
ಕುಗ್ಗು-ತ್ತಿದೆ
ಕುಚ
ಕುಚ-ಕುಂ-ಕು-ಮ-ವನ್ನು
ಕುಚ-ವಿ-ಲು-ಳಿ-ತ-ಮಾ-ಲಾ-ಕುಂ-ಕು-ಮ-ಶ್ಮ-ಶ್ರು-ಭಿರ್ನಃ
ಕುಚೇಲ
ಕುಚೇ-ಲ-ನನ್ನು
ಕುಚೇ-ಲ-ನಿಗೆ
ಕುಚೇ-ಲನು
ಕುಚೇಷ್ಟೆ
ಕುಚೇ-ಷ್ಟೆ-ಯನ್ನು
ಕುಚೇ-ಷ್ಟೆ-ಯಿಂದ
ಕುಚ್ಚು-ಗ-ಳುಳ್ಳ
ಕುಜ-ನೈಃ
ಕುಟ-ಕಾ-ಚ-ಲಕ್ಕೆ
ಕುಟುಂಬ
ಕುಟುಂ-ಬದ
ಕುಟುಂ-ಬ-ದಲ್ಲಿ
ಕುಟುಂ-ಬ-ಲೀ-ಲೆ-ಯನ್ನು
ಕುಟುಂ-ಬ-ವ-ರ್ಗ-ದ-ವರ
ಕುಟುಂ-ಬ-ವ-ರ್ಗ-ದ-ವರೂ
ಕುಟುಂ-ಬಿಯು
ಕುಟ್ಟು-ವು-ದಕ್ಕೆ
ಕುಟ್ಮ-ಲೋ-ಪ-ಲಾ-ಲಿತ
ಕುಡಿ
ಕುಡಿ-ಕು-ಡಿದು
ಕುಡಿತ
ಕುಡಿದ
ಕುಡಿ-ದ-ನೆಂದು
ಕುಡಿ-ದರೆ
ಕುಡಿ-ದ-ರೆಂ-ಬು-ದನ್ನು
ಕುಡಿ-ದ-ವನು
ಕುಡಿ-ದ-ವ-ರಿಗೆ
ಕುಡಿ-ದವು
ಕುಡಿ-ದಿದ್ದ
ಕುಡಿ-ದಿ-ದ್ದಾನೆ
ಕುಡಿದು
ಕುಡಿ-ದು-ಕೊಂ-ಡಿ-ರುತ್ತಾ
ಕುಡಿ-ದು-ಬಿ-ಟ್ಟನು
ಕುಡಿ-ದು-ಬಿ-ಡು-ತ್ತೇನೆ
ಕುಡಿ-ದೇ-ಬಿ-ಟ್ಟನು
ಕುಡಿ-ದೊ-ಡನೆ
ಕುಡಿ-ನೋ-ಟ-ಇ-ವು-ಗಳನ್ನು
ಕುಡಿ-ನೋ-ಟ-ದಿಂದ
ಕುಡಿ-ನೋ-ಟ-ವನ್ನೂ
ಕುಡಿ-ಯದೆ
ಕುಡಿ-ಯನ್ನು
ಕುಡಿ-ಯ-ಬೇಕು
ಕುಡಿ-ಯಲು
ಕುಡಿ-ಯ-ಲೆಂದು
ಕುಡಿಯು
ಕುಡಿ-ಯು-ತ್ತಿದ್ದ
ಕುಡಿ-ಯು-ತ್ತಿ-ದ್ದರು
ಕುಡಿ-ಯು-ತ್ತಿ-ರು-ವಾಗ
ಕುಡಿ-ಯುವ
ಕುಡಿ-ಯು-ವಂತೆ
ಕುಡಿ-ಯು-ವನು
ಕುಡಿ-ಯು-ವ-ವ-ನಂತೆ
ಕುಡಿ-ಯು-ವಾಗ
ಕುಡಿ-ಯು-ವು-ದ-ಕ್ಕಾಗಿ
ಕುಡಿ-ಯು-ವು-ದಕ್ಕೆ
ಕುಡಿ-ವು-ದೊಂದು
ಕುಡಿ-ಸ-ಬ-ಹುದು
ಕುಡಿಸಿ
ಕುಡಿ-ಸಿ-ದನು
ಕುಡಿ-ಸು-ವಂತೆ
ಕುಡು-ವೆಂಬ
ಕುಣಿ
ಕುಣಿ-ಕು-ಣಿ-ಯುತ್ತಾ
ಕುಣಿ-ಕೆಗೆ
ಕುಣಿ-ದಂತೆ
ಕುಣಿ-ದರು
ಕುಣಿ-ದಾ-ಡ-ಬೇಕು
ಕುಣಿ-ದಾಡಿ
ಕುಣಿ-ದಾ-ಡಿದ
ಕುಣಿ-ದಾ-ಡಿ-ದನು
ಕುಣಿ-ದಾ-ಡಿ-ದರು
ಕುಣಿ-ದಾ-ಡುತ್ತಾ
ಕುಣಿ-ಯಲು
ಕುಣಿ-ಯಿತು
ಕುಣಿ-ಯುತ್ತ
ಕುಣಿ-ಯು-ತ್ತ-ವಂತೆ
ಕುಣಿ-ಯು-ತ್ತವೆ
ಕುಣಿ-ಯುತ್ತಾ
ಕುಣಿ-ಯು-ತ್ತಾನೆ
ಕುಣಿ-ಯು-ತ್ತಿ-ದ್ದ-ವರೆ
ಕುಣಿ-ಯು-ತ್ತಿ-ದ್ದಾಳೆ
ಕುಣಿ-ಯು-ತ್ತಿ-ರು-ವಾಗ
ಕುಣಿ-ಯು-ತ್ತಿವೆ
ಕುಣಿ-ಯು-ವಂತೆ
ಕುಣಿ-ಯು-ವರು
ಕುಣಿ-ಯು-ವಾಗ
ಕುಣಿವ
ಕುಣಿಸಿ
ಕುಣಿ-ಸಿ-ದಂತೆ
ಕುಣಿ-ಸಿ-ದರು
ಕುಣಿ-ಸಿ-ದಳು
ಕುಣಿ-ಸುವ
ಕುತ-ಕುತ
ಕುತೂ-ಹಲ
ಕುತೂ-ಹ-ಲ-ದಷ್ಟೆ
ಕುತೂ-ಹ-ಲ-ದಿಂದ
ಕುತೂ-ಹಲಿ
ಕುತ್ತಾ
ಕುತ್ತಿಗೆ
ಕುತ್ತಿ-ಗೆಯ
ಕುತ್ತಿ-ಗೆ-ಯನ್ನು
ಕುತ್ತಿ-ಗೆ-ಯಲ್ಲಿ
ಕುದಿ
ಕುದಿ-ದವು
ಕುದಿ-ದು-ಹೋ-ಯಿತು
ಕುದಿ-ಯಿತು
ಕುದಿ-ಯುತ್ತ
ಕುದಿ-ಯುತ್ತಾ
ಕುದಿ-ಯು-ತ್ತಿತ್ತು
ಕುದಿ-ಯು-ತ್ತಿ-ದ್ದರು
ಕುದಿ-ಯು-ತ್ತಿ-ದ್ದರೂ
ಕುದಿ-ಯು-ತ್ತಿ-ದ್ದೇನೆ
ಕುದಿ-ಯು-ತ್ತಿ-ರು-ವಳು
ಕುದಿ-ಯು-ವಂತೆ
ಕುದುರೆ
ಕುದು-ರೆ-ಗಳ
ಕುದು-ರೆ-ಗ-ಳಂತೆ
ಕುದು-ರೆ-ಗಳನ್ನು
ಕುದು-ರೆ-ಗಳನ್ನೂ
ಕುದು-ರೆ-ಗ-ಳಿ-ಗಾಗಿ
ಕುದು-ರೆ-ಗ-ಳಿ-ದ್ದಂತೆ
ಕುದು-ರೆ-ಗಳು
ಕುದು-ರೆ-ಗ-ಳೊ-ಡನೆ
ಕುದು-ರೆ-ಚೆಂ-ಡಿನ
ಕುದು-ರೆಯ
ಕುದು-ರೆ-ಯಂತೆ
ಕುದು-ರೆ-ಯನ್ನು
ಕುದು-ರೆ-ಯನ್ನೂ
ಕುದು-ರೆ-ಯ-ನ್ನೇರಿ
ಕುದು-ರೆ-ಯಾಗಿ
ಕುನ್ನಿ-ಗಳೆ
ಕುಪಿ-ತ-ನಾದ
ಕುಪ್ಪುಸ
ಕುಬೇ-ರನ
ಕುಬೇ-ರನು
ಕುಬೇ-ರನೂ
ಕುಬೇ-ರರ
ಕುಬ್ಜ
ಕುಮಾರ
ಕುಮಾ-ರನ
ಕುಮಾ-ರ-ಸ್ವಾ-ಮಿಯು
ಕುಮಾ-ರಾಯ
ಕುಮಾ-ರಿ-ಯ-ರನ್ನು
ಕುರರ
ಕುರ-ರ-ಪ-ಕ್ಷಿ-ಯನ್ನು
ಕುರರಿ
ಕುರಿ
ಕುರಿ-ಗಳನ್ನು
ಕುರಿ-ಗ-ಳಾ-ಗು-ವುದು
ಕುರಿ-ಗ-ಳಿಗೆ
ಕುರಿತು
ಕುರಿ-ತು-ಮಹಾ
ಕುರಿ-ಮ-ರಿ-ಗಳನ್ನು
ಕುರಿ-ಮ-ರಿ-ಗ-ಳೆ-ರ-ಡನ್ನೂ
ಕುರಿಯ
ಕುರಿ-ಯನ್ನು
ಕುರು
ಕುರು-ಕ್ಷೇ-ತ್ರದ
ಕುರು-ಕ್ಷೇ-ತ್ರ-ದಲ್ಲಿ
ಕುರುಡ
ಕುರು-ಡ-ದೊರೆ
ಕುರು-ಡ-ನಂತೆ
ಕುರು-ಡ-ನಾ-ಗಿದ್ದ
ಕುರು-ಡ-ನಾ-ಗು-ವನು
ಕುರು-ಡ-ರನ್ನು
ಕುರು-ಡ-ರಾಗಿ
ಕುರು-ಡ-ರಾ-ಜನ
ಕುರು-ಡರು
ಕುರು-ಡಾ-ಗಿದ್ದ
ಕುರು-ಣೆ-ಗಾಗಿ
ಕುರು-ಬ-ರಾಗಿ
ಕುರು-ವಂ-ಶದ
ಕುರು-ವ-ರ್ಷ-ದಲ್ಲಿ
ಕುರು-ವ-ರ್ಷ-ವೆಂಬ
ಕುರು-ವಿನ
ಕುರು-ಸೇ-ನೆ-ಯನ್ನು
ಕುರೂಪಿ
ಕುರೂ-ಪಿಗೆ
ಕುರೂ-ಪಿ-ಯಾ-ಗಿದ್ದ
ಕುರೂ-ಪಿ-ಯಾ-ಗಿ-ರಲಿ
ಕುರೂ-ಷ-ದೇ-ಶದ
ಕುರ್ಮ-ಸ್ತ-ವಾಂ-ಗ-ಪ-ಯ-ಸೋ-ಪ-ಸೇ-ಚನಂ
ಕುರ್ಯಾ-ದ್ವಿ-ನೇ-ಶ್ವರಂ
ಕುಲ
ಕುಲ-ಕ-ಎಂಬ
ಕುಲ-ಕೋ-ಟಿ-ಗಳು
ಕುಲಕ್ಕೆ
ಕುಲ-ಗು-ರ-ವಾದ
ಕುಲ-ಗು-ರು-ಗ-ಳಾದ
ಕುಲ-ಗು-ರು-ಗಳು
ಕುಲ-ಗೆ-ಡಿಸ
ಕುಲ-ಗೋ-ತ್ರ-ಗ-ಳೊಂದೂ
ಕುಲದ
ಕುಲ-ದ-ವನು
ಕುಲ-ದೇ-ವ-ತೆಯ
ಕುಲ-ದೇ-ವ-ತೆ-ಯಾದ
ಕುಲ-ಪ-ರ್ವ-ತ-ಗಳಲ್ಲಿ
ಕುಲ-ಪ-ರ್ವ-ತ-ಗ-ಳ-ಲ್ಲೆಲ್ಲ
ಕುಲ-ಪು-ರೋ-ಹಿತ
ಕುಲ-ಪು-ರೋ-ಹಿ-ತ-ನಾದ
ಕುಲ-ಪು-ರೋ-ಹಿ-ತ-ರಾದ
ಕುಲ-ವನ್ನು
ಕುಲವೂ
ಕುಲ-ವೃ-ದ್ಧ-ರನ್ನೂ
ಕುಲವೆ
ಕುಲು-ಕಾಟ
ಕುಲು-ಕಾ-ಡಲು
ಕುಳಿ
ಕುಳಿ-ಕ-ರು-ತ-ಮಿ-ವಾ-ಜ್ಞಾಃ
ಕುಳಿತ
ಕುಳಿ-ತನು
ಕುಳಿ-ತರು
ಕುಳಿ-ತರೆ
ಕುಳಿ-ತ-ಲ್ಲಿಯೇ
ಕುಳಿ-ತಳು
ಕುಳಿ-ತ-ವರ
ಕುಳಿ-ತಾಗ
ಕುಳಿತಿ
ಕುಳಿ-ತಿತು
ಕುಳಿ-ತಿದ್ದ
ಕುಳಿ-ತಿ-ದ್ದನು
ಕುಳಿ-ತಿ-ದ್ದರೂ
ಕುಳಿ-ತಿ-ದ್ದಳು
ಕುಳಿ-ತಿ-ದ್ದ-ವಳು
ಕುಳಿ-ತಿ-ದ್ದಾನೆ
ಕುಳಿ-ತಿ-ದ್ದಾರೆ
ಕುಳಿ-ತಿ-ದ್ದು-ದ-ರಿಂದ
ಕುಳಿ-ತಿ-ರ-ಬೇ-ಕಾ-ಯಿತು
ಕುಳಿ-ತಿ-ರಲಿ
ಕುಳಿ-ತಿ-ರಲು
ಕುಳಿ-ತಿ-ರುವ
ಕುಳಿ-ತಿ-ರು-ವಂತೆ
ಕುಳಿ-ತಿ-ರು-ವಂ-ತೆಯೊ
ಕುಳಿ-ತಿ-ರು-ವನು
ಕುಳಿ-ತಿ-ರು-ವ-ವ-ರಿ-ಗೆಲ್ಲ
ಕುಳಿ-ತಿ-ರು-ವಾಗ
ಕುಳಿ-ತಿ-ರು-ವು-ದನ್ನೂ
ಕುಳಿ-ತಿ-ರು-ವುದು
ಕುಳಿ-ತಿವೆ
ಕುಳಿತು
ಕುಳಿ-ತುಕೊ
ಕುಳಿ-ತು-ಕೊಂಡ
ಕುಳಿ-ತು-ಕೊಂ-ಡನು
ಕುಳಿ-ತು-ಕೊಂ-ಡರು
ಕುಳಿ-ತು-ಕೊ-ಳ್ಳ-ಬೇಕೆ
ಕುಳಿ-ತು-ಕೊ-ಳ್ಳ-ಲಿಲ್ಲ
ಕುಳಿ-ತು-ಕೊ-ಳ್ಳಲು
ಕುಳಿ-ತು-ಕೊ-ಳ್ಳ-ಲೆಂದು
ಕುಳಿ-ತು-ಕೊಳ್ಳು
ಕುಳಿ-ತು-ಕೊ-ಳ್ಳು-ತ್ತಾರೆ
ಕುಳಿ-ತು-ಕೊ-ಳ್ಳು-ತ್ತೇನೆ
ಕುಳಿ-ತು-ಕೊ-ಳ್ಳು-ವಂತೆ
ಕುಳಿ-ತು-ಕೊ-ಳ್ಳು-ವು-ದಕ್ಕೆ
ಕುಳಿ-ತೆನು
ಕುಳಿತೇ
ಕುಳ್ಳ-ದೇಹ
ಕುಳ್ಳಿ-ರಿಸಿ
ಕುಳ್ಳಿ-ರಿ-ಸಿ-ಕೊಂಡು
ಕುಳ್ಳಿ-ರಿ-ಸಿದ
ಕುಳ್ಳಿ-ರಿ-ಸಿ-ದನು
ಕುಳ್ಳಿರು
ಕುಳ್ಳು
ಕುವ-ಲ-ಯ-ವೆಂಬ
ಕುಶ
ಕುಶ-ದ್ವೀ-ಪದ
ಕುಶ-ದ್ವೀ-ಪವೂ
ಕುಶ-ರೆಂಬ
ಕುಶಲ
ಕುಶ-ಲ-ಪ್ರ-ಶ್ನೆ-ಗ-ಳಾದ
ಕುಶ-ಲ-ಮತಿ
ಕುಶ-ಲ-ವನ್ನು
ಕುಶ-ಲ-ವಾ-ಗಿ-ರು-ವ-ರಷ್ಟೆ
ಕುಸಿದು
ಕುಸ್ತಿ
ಕುಸ್ತಿಗೆ
ಕುಸ್ತಿಯ
ಕುಸ್ತಿ-ಯನ್ನು
ಕೂ-ಕೂರ್ಮ
ಕೂಗನ್ನು
ಕೂಗಾ-ಡುತ್ತಾ
ಕೂಗಿ
ಕೂಗಿ-ಕೊಂಡ
ಕೂಗಿ-ಕೊಂ-ಡನು
ಕೂಗಿ-ಕೊಂ-ಡರು
ಕೂಗಿ-ಕೊಂ-ಡಳು
ಕೂಗಿ-ಕೊಂ-ಡವು
ಕೂಗಿ-ಕೊಂ-ಡಿತು
ಕೂಗಿ-ಕೊಂಡು
ಕೂಗಿಗೆ
ಕೂಗಿದ
ಕೂಗಿ-ದನು
ಕೂಗಿ-ದು-ದ-ರಿಂದ
ಕೂಗಿ-ದುದು
ಕೂಗು
ಕೂಗುತ್ತಾ
ಕೂಗು-ತ್ತಿ-ದ್ದಂ-ತೆಯೆ
ಕೂಗು-ತ್ತಿ-ದ್ದೆ-ಯಲ್ಲ
ಕೂಗು-ತ್ತಿ-ರುವ
ಕೂಗು-ತ್ತೇವೆ
ಕೂಟ-ನೆಂಬ
ಕೂಡ
ಕೂಡ-ಇಲ್ಲ
ಕೂಡ-ಕೊಂದು
ಕೂಡದು
ಕೂಡ-ದೂ-ರ-ವಾ-ಗಿ-ರ-ಬೇ-ಕೆ-ನಿ-ಸು-ತ್ತದೆ
ಕೂಡ-ಪ-ರ-ಮೇ-ಶ್ವ-ರ-ನೆಂದು
ಕೂಡ-ಪೇ-ಲ-ವ-ವಾಗಿ
ಕೂಡ-ಬ-ಯ-ಸು-ವ-ವರು
ಕೂಡಲೆ
ಕೂಡಲೇ
ಕೂಡ-ಶ್ರೀ-ಕೃ-ಷ್ಣ-ನನ್ನು
ಕೂಡ-ಹೆ-ಣ್ಣಾ-ದರು
ಕೂಡಿ
ಕೂಡಿ-ಕೊಂಡು
ಕೂಡಿಟ್ಟ
ಕೂಡಿ-ಟ್ಟು-ಕೊಂ-ಡಿ-ದ್ದಾಗ
ಕೂಡಿ-ಡು-ವುದೇ
ಕೂಡಿದ
ಕೂಡಿ-ದ-ವ-ನಾ-ಗಿಯೂ
ಕೂಡಿ-ದ-ವರು
ಕೂಡಿದೆ
ಕೂಡಿ-ರುವ
ಕೂಡಿ-ಸ-ಬೇ-ಕೆಂದು
ಕೂಡಿಸಿ
ಕೂಡಿ-ಸಿಕೊ
ಕೂಡಿ-ಸಿ-ಕೊಂ-ಡರು
ಕೂಡಿ-ಸಿ-ಕೊಂಡು
ಕೂಡಿ-ಸಿ-ತೆಂದು
ಕೂಡಿ-ಸಿ-ದ್ದೆವು
ಕೂಡಿ-ಸು-ತ್ತಾನೆ
ಕೂಡಿ-ಹಾಕಿ
ಕೂಡಿ-ಹಾ-ಕಿ-ದರು
ಕೂಡು-ವಂ-ತ-ಹು-ದೇ-ನಿದೆ
ಕೂಡು-ವನು
ಕೂಡು-ವರು
ಕೂತರು
ಕೂತರೆ
ಕೂತ-ಲ್ಲಿಂದ
ಕೂತ-ವರು
ಕೂತಿದ್ದ
ಕೂತು
ಕೂತುಕೊ
ಕೂದ
ಕೂದಲ
ಕೂದ-ಲನ್ನು
ಕೂದ-ಲನ್ನೂ
ಕೂದಲು
ಕೂದ-ಲು-ಇಂ-ತಿ-ರುವ
ಕೂದ-ಲು-ಗಳನ್ನು
ಕೂಪ-ಕ-ರ್ಣ-ರನ್ನು
ಕೂರಿ-ಸಿ-ಕೊಂಡು
ಕೂರ್ತ-ಮ-ಡ-ದಿ-ಯಾದ
ಕೂರ್ಮ
ಕೂರ್ಮನು
ಕೂರ್ಮ-ಪು-ರಾಣ
ಕೂರ್ಮಾ-ವ-ತಾ-ರ-ವನ್ನು
ಕೂರ್ಮೋ
ಕೂಲಿ-ಯಾಗಿ
ಕೂಲ್ಯ-ಕ್ಕಾಗಿ
ಕೂಳಿಗೆ
ಕೂಶ್ಮಾಂಡ
ಕೂಸ-ನ್ನಿ-ಟ್ಟು-ಕೊಂಡು
ಕೂಸನ್ನು
ಕೂಸಲ್ಲ
ಕೂಸಾ
ಕೂಸಾ-ಗಿ-ರುವ
ಕೂಸಿಗೂ
ಕೂಸಿಗೆ
ಕೂಸಿನ
ಕೂಸಿ-ನೊ-ಡನೆ
ಕೂಸು
ಕೂಸು-ಗಳು
ಕೃತ
ಕೃತಂ
ಕೃತಕ
ಕೃತ-ಕಾ-ರ್ಯ-ಕ್ಕಾಗಿ
ಕೃತ-ಕೃತ್ಯ
ಕೃತ-ಕೃ-ತ್ಯ-ಳಾ-ದಳು
ಕೃತ-ಕೃತ್ಯೆ
ಕೃತಘ್ನ
ಕೃತ-ಜ್ಞ-ತೆ-ಯಾ-ದರೂ
ಕೃತ-ದ್ಯುತಿ
ಕೃತ-ದ್ಯು-ತಿಗೆ
ಕೃತ-ದ್ಯು-ತಿಯು
ಕೃತ-ಯುಗ
ಕೃತ-ಯು-ಗ-ದಲ್ಲಿ
ಕೃತ-ಯು-ಗ-ದ-ವರು
ಕೃತ-ವರ್ಮ
ಕೃತ-ವ-ರ್ಮ-ರಿಗೆ
ಕೃತ-ವಾ-ಗಿತ್ತು
ಕೃತಾ-ರ್ಥ-ನ-ನ್ನಾಗಿ
ಕೃತಾ-ರ್ಥ-ಳ-ನ್ನಾಗಿ
ಕೃತಾ-ವ-ತಾರಃ
ಕೃತಿ
ಕೃತಿ-ಯಲ್ಲಿ
ಕೃತ್ತಿ-ಕೆಯೇ
ಕೃತ್ಯ-ವನ್ನು
ಕೃಪ
ಕೃಪಣ
ಕೃಪಾ-ಕ-ಟಾ-ಕ್ಷ-ವನ್ನು
ಕೃಪಾ-ಚಾ-ರ್ಯನು
ಕೃಪಾ-ದೃಷ್ಟಿ
ಕೃಪಾಳು
ಕೃಪಾ-ಳು-ವಾದ
ಕೃಪೆ
ಕೃಪೆಗೂ
ಕೃಪೆಗೆ
ಕೃಪೆ-ದೋರಿ
ಕೃಪೆ-ಮಾಡ
ಕೃಪೆ-ಮಾಡಿ
ಕೃಪೆ-ಯಿಂದ
ಕೃಪೆ-ಯಿಂ-ದಲೂ
ಕೃಪೆ-ಯಿಂ-ದ-ಲೆ-ಎಂದು
ಕೃಪೆ-ಯಿ-ಲ್ಲದೆ
ಕೃಪೆ-ಯೊಂದೆ
ಕೃಶ-ನಾಗಿ
ಕೃಶ-ವಾ-ಗಿ-ದ್ದರೂ
ಕೃಶಾ-ಶ್ವ-ರಿಗೂ
ಕೃಷ್ಟ-ಸಂ-ಗಮಂ
ಕೃಷ್ಣ
ಕೃಷ್ಣ-ಚಾ-ಣೂರ
ಕೃಷ್ಣ-ಎಂದು
ಕೃಷ್ಣ-ಗ-ತಿ-ಯ-ನ್ನು-ಎಂ-ದರೆ
ಕೃಷ್ಣ-ಚ-ರಿ-ತಾ-ಮೃ-ತ-ದಿಂದ
ಕೃಷ್ಣ-ಚ-ರಿತ್ರೆ
ಕೃಷ್ಣ-ಜಾಂ-ಬ-ವಂ-ತ-ರಿ-ಬ್ಬರೂ
ಕೃಷ್ಣ-ದ್ವೈ-ಪಾ-ಯನ
ಕೃಷ್ಣನ
ಕೃಷ್ಣ-ನತ್ತ
ಕೃಷ್ಣ-ನದೊ
ಕೃಷ್ಣ-ನನ್ನು
ಕೃಷ್ಣ-ನ-ಲ್ಲದ
ಕೃಷ್ಣ-ನಾದ
ಕೃಷ್ಣ-ನಿಂದ
ಕೃಷ್ಣ-ನಿಂ-ದೆ-ಯನ್ನು
ಕೃಷ್ಣ-ನಿ-ಗಿಂ-ತಲೂ
ಕೃಷ್ಣ-ನಿಗೆ
ಕೃಷ್ಣ-ನಿ-ರು-ವ-ವ-ರೆಗೆ
ಕೃಷ್ಣನು
ಕೃಷ್ಣನೂ
ಕೃಷ್ಣ-ನೆಂದು
ಕೃಷ್ಣ-ನೆಂದೆ
ಕೃಷ್ಣ-ನೆಂ-ಬು-ವನು
ಕೃಷ್ಣನೇ
ಕೃಷ್ಣ-ನೊ-ಡನೆ
ಕೃಷ್ಣ-ನೊ-ಬ್ಬನ
ಕೃಷ್ಣ-ನೊ-ಬ್ಬನೆ
ಕೃಷ್ಣ-ಪ-ಕ್ಷ-ಗಳಿಂದ
ಕೃಷ್ಣ-ಪ-ಕ್ಷ-ದಲ್ಲಿ
ಕೃಷ್ಣ-ಪೂ-ರಿತೋ
ಕೃಷ್ಣ-ಬ-ಲ-ರಾ-ಮರು
ಕೃಷ್ಣ-ಮಂತ್ರಿ
ಕೃಷ್ಣ-ಮಯಂ
ಕೃಷ್ಣ-ಮೂ-ರ್ತಿ-ಯೊ-ಡನೆ
ಕೃಷ್ಣರ
ಕೃಷ್ಣ-ರದೊ
ಕೃಷ್ಣ-ರನ್ನು
ಕೃಷ್ಣ-ರಲ್ಲ
ಕೃಷ್ಣ-ರಾ-ದರು
ಕೃಷ್ಣ-ರಿಗೆ
ಕೃಷ್ಣ-ರಿ-ಬ್ಬರೂ
ಕೃಷ್ಣರು
ಕೃಷ್ಣ-ರು-ಕ್ಮಿ-ಣಿ-ಯರ
ಕೃಷ್ಣ-ರೂ-ಪಿ-ಯಾಗಿ
ಕೃಷ್ಣ-ರೆಲ್ಲಿ
ಕೃಷ್ಣ-ರೊ-ಡನೆ
ಕೃಷ್ಣ-ಶು-ಕ್ಲ-ಯೋಃ
ಕೃಷ್ಣ-ಸಂಜ್ಞಂ
ಕೃಷ್ಣ-ಸ್ವಾ-ಮಿ-ಯಲ್ಲಿ
ಕೃಷ್ಣಾ
ಕೃಷ್ಣಾ-ಜಿನ
ಕೃಷ್ಣಾ-ಜಿ-ನ-ಗಳನ್ನು
ಕೃಷ್ಣಾ-ಜಿ-ನ-ವನ್ನು
ಕೃಷ್ಣಾ-ಜಿ-ನ-ವನ್ನೂ
ಕೃಷ್ಣಾಯ
ಕೃಷ್ಣಾರ್
ಕೃಷ್ಣಾ-ರ್ಜು-ನ-ರನ್ನು
ಕೃಷ್ಣಾ-ರ್ಜು-ನರು
ಕೃಷ್ಣಾ-ರ್ಪಣ
ಕೃಷ್ಣಾ-ವ-ತ-ರಣ
ಕೃಷ್ಣಾ-ವ-ತಾ-ರದ
ಕೃಷ್ಣಾ-ವ-ತಾ-ರ-ವೆತ್ತಿ
ಕೃಷ್ಣೆ
ಕೆಂಗ-ಣ್ಣಿ-ನಿಂದ
ಕೆಂಗ-ಣ್ಣು-ಗಳಿಂದ
ಕೆಂಗೂ-ದಲು
ಕೆಂಗೆಂಡ
ಕೆಂಗೆಂ-ಡ-ನಾ-ದನು
ಕೆಂಗೆಂ-ಡ-ವಾದ
ಕೆಂಡ
ಕೆಂಡಕ್ಕೆ
ಕೆಂಡ-ದಂ-ತಹ
ಕೆಂಡ-ದಂ-ತಿ-ರುವ
ಕೆಂಡ-ದಂತೆ
ಕೆಂಡ-ದುಂ-ಡೆ-ಯಂ-ತಿ-ದ್ದವು
ಕೆಂಡ-ದುಂ-ಡೆ-ಯಂ-ತಿ-ರುವ
ಕೆಂಡ-ವನ್ನು
ಕೆಂಡುಂ-ಡೆ-ಯಂ-ತಹ
ಕೆಂಪ-ಗಾ-ಯಿತು
ಕೆಂಪನ್ನು
ಕೆಂಪಾದ
ಕೆಂಪು
ಕೆಂಬೂತಿ
ಕೆಂಬೂ-ತಿಯ
ಕೆಕ್ಕ-ರ-ಗ-ಣ್ಣಿನ
ಕೆಕ್ಕ-ರಿಸಿ
ಕೆಕ್ಕ-ರಿ-ಸಿ-ಕೊಂಡು
ಕೆಕ್ಕ-ರು-ಗ-ಣ್ಣಿ-ನಿಂದ
ಕೆಟ್ಟ
ಕೆಟ್ಟ-ಭಾ-ವ-ನೆಯೂ
ಕೆಟ್ಟ-ಮಾ-ತು-ಗಳಿಂದ
ಕೆಟ್ಟ-ಮೇಲೆ
ಕೆಟ್ಟ-ವಾ-ಸನೆ
ಕೆಟ್ಟು
ಕೆಟ್ಟು-ದಿಲ್ಲ
ಕೆಟ್ಟುದು
ಕೆಟ್ಟು-ಹೋ-ಗು-ವರು
ಕೆಟ್ಟೆ
ಕೆಡ-ದಂತೆ
ಕೆಡ-ವಲು
ಕೆಡವಿ
ಕೆಡ-ವಿ-ದನು
ಕೆಡಹಿ
ಕೆಡ-ಹಿದ
ಕೆಡ-ಹಿ-ದನು
ಕೆಡಿ-ಸ-ಬೇ-ಕೆಂದು
ಕೆಡಿ-ಸ-ಹೊ-ರ-ಟಿ-ರುವ
ಕೆಡಿ-ಸಿ-ದರು
ಕೆಡಿ-ಸು-ವು-ದ-ಕ್ಕಾಗಿ
ಕೆಡು-ಕಿನ
ಕೆಣಕ
ಕೆಣಕಿ
ಕೆಣ-ಕಿದ
ಕೆಣ-ಕಿ-ದುದು
ಕೆಣಕು
ಕೆದ-ಕಿ-ಕೊ-ಳ್ಳುತ್ತಿ
ಕೆದ-ರಿ-ಕೊಂಡು
ಕೆದ-ರಿದ
ಕೆನೆ-ದ-ನೆಂ-ದರೆ
ಕೆನೆ-ಯಂತೆ
ಕೆನ್ನೆ
ಕೆನ್ನೆ-ಗಳು
ಕೆನ್ನೆ-ಗ-ಳೇನು
ಕೆನ್ನೆ-ಗಳೊ
ಕೆನ್ನೆಗೆ
ಕೆನ್ನೆಯ
ಕೆನ್ನೆ-ಯನ್ನು
ಕೆರಳಿ
ಕೆರ-ಳಿತು
ಕೆರ-ಳಿದ
ಕೆರ-ಳಿ-ದನು
ಕೆರ-ಳಿ-ಸಿದ
ಕೆರೆ
ಕೆರೆ-ಯುತ್ತಾ
ಕೆರೆ-ಯು-ತ್ತಿತ್ತು
ಕೆಲ
ಕೆಲ-ಕಾಲ
ಕೆಲ-ಕಾ-ಲದ
ಕೆಲ-ಕಾ-ಲ-ದ-ಮೇಲೆ
ಕೆಲ-ಕಾ-ಲ-ವಾದ
ಕೆಲ-ಕಾ-ಲ-ವಾ-ದ-ಮೇಲೆ
ಕೆಲ-ದಿ-ನ-ಗಳ
ಕೆಲ-ವರ
ಕೆಲ-ವ-ರಂತೂ
ಕೆಲ-ವ-ರನ್ನು
ಕೆಲ-ವ-ರಿಗೆ
ಕೆಲ-ವರು
ಕೆಲವು
ಕೆಲ-ವು-ಕಾ-ಲದ
ಕೆಲಸ
ಕೆಲ-ಸ-ಕ್ಕಾಗಿ
ಕೆಲ-ಸಕ್ಕೆ
ಕೆಲ-ಸ-ಗಳನ್ನು
ಕೆಲ-ಸ-ಗಳಲ್ಲಿ
ಕೆಲ-ಸ-ಗ-ಳ-ಲ್ಲಿಯೂ
ಕೆಲ-ಸ-ಗ-ಳಿಗೆ
ಕೆಲ-ಸ-ಗಳೂ
ಕೆಲ-ಸ-ದಲ್ಲಿ
ಕೆಲ-ಸ-ಮಾಡಿ
ಕೆಲ-ಸ-ಮಾ-ಡಿ-ಕೊಂ-ಡಿ-ದ್ದನು
ಕೆಲ-ಸ-ಮಾ-ಡು-ವನು
ಕೆಲ-ಸ-ವನ್ನು
ಕೆಲ-ಸ-ವಲ್ಲ
ಕೆಲ-ಸವೂ
ಕೆಲ-ಸವೆ
ಕೆಲ-ಸ-ವೆಂದರೆ
ಕೆಲ-ಸವೇ
ಕೆಲ-ಹೊ-ತ್ತಿ-ನ-ವ-ರೆಗೆ
ಕೆಲ-ಹೊತ್ತು
ಕೆಲೆ-ಯುತ್ತ
ಕೆಳ
ಕೆಳ-ಕ್ಕಿ-ಳಿದ
ಕೆಳ-ಕ್ಕಿ-ಳಿದು
ಕೆಳ-ಕ್ಕು-ದುರಿ
ಕೆಳ-ಕ್ಕು-ರು-ಳಿತು
ಕೆಳ-ಕ್ಕು-ರು-ಳಿ-ಸಿ-ದನು
ಕೆಳಕ್ಕೆ
ಕೆಳ-ಕ್ಕೆ-ಳೆದು
ಕೆಳ-ಗಾಗಿ
ಕೆಳ-ಗಾ-ದುದು
ಕೆಳ-ಗಿ-ಟ್ಟಿದ್ದ
ಕೆಳ-ಗಿಟ್ಟು
ಕೆಳ-ಗಿದ್ದ
ಕೆಳ-ಗಿನ
ಕೆಳ-ಗಿ-ನದೇ
ಕೆಳ-ಗಿ-ರುವ
ಕೆಳ-ಗಿ-ಳಿದು
ಕೆಳ-ಗಿ-ಳಿಸಿ
ಕೆಳ-ಗು-ರು-ಳಿ-ದರು
ಕೆಳ-ಗು-ರು-ಳು-ವಂತೆ
ಕೆಳಗೆ
ಕೆಳ-ಗೆಲ್ಲ
ಕೆಳ-ಗೊಂ-ದಿವೆ
ಕೆಳ-ಮ-ಟ್ಟದ
ಕೆಳ-ಲೋಕ
ಕೆಸ-ರಾ-ಗಿ-ಹೋ-ಯಿತು
ಕೇಂದ್ರ-ದಲ್ಲಿ
ಕೇಂದ್ರ-ದ-ಲ್ಲಿದೆ
ಕೇಂದ್ರ-ದ-ಲ್ಲಿ-ರು-ವಂತೆ
ಕೇಂದ್ರ-ಸ್ಥಾ-ನ-ದ-ಲ್ಲಿ-ದ್ದಾನೆ
ಕೇಕಯ
ಕೇಕ-ಯ-ರಾ-ಜನ
ಕೇಕೆ
ಕೇಡನ್ನು
ಕೇಡಿಗ
ಕೇಡು
ಕೇಡು-ಗಾ-ಲಕ್ಕೆ
ಕೇಡೂ
ಕೇತು
ಕೇತು-ಗಳಿಂದ
ಕೇತು-ಮೂ-ಲ-ವರ್ಷ
ಕೇತು-ಮೂ-ಲ-ವ-ರ್ಷ-ದಲ್ಲಿ
ಕೇದಾರ
ಕೇರಿ-ಕೇ-ರಿ-ಯೆಲ್ಲ
ಕೇರಿ-ಗಳು
ಕೇಳ
ಕೇಳದ
ಕೇಳ-ದ-ವ-ನಂತೆ
ಕೇಳದೆ
ಕೇಳ-ದೆಯೆ
ಕೇಳ-ಬ-ಹುದು
ಕೇಳ-ಬಾ-ರ-ದು-ದನ್ನು
ಕೇಳ-ಬೇ-ಕಾ-ದರೆ
ಕೇಳ-ಬೇಕು
ಕೇಳ-ಬೇಕೆ
ಕೇಳ-ಬೇ-ಕೆಂ-ದಿ-ರು-ವೆಯ
ಕೇಳ-ಬೇ-ಕೆಂದು
ಕೇಳ-ಬೇ-ಕೆಂದೆ
ಕೇಳ-ಬೇ-ಕೆಂಬ
ಕೇಳ-ಬೇ-ಕೆ-ನ್ನಿ-ಸಿತು
ಕೇಳ-ಲಾ-ರದೆ
ಕೇಳಲಿ
ಕೇಳ-ಲಿ-ಲ್ಲ-ವಲ್ಲಾ
ಕೇಳಲು
ಕೇಳಲೂ
ಕೇಳಿ
ಕೇಳಿಕೊ
ಕೇಳಿ-ಕೊಂಡ
ಕೇಳಿ-ಕೊಂ-ಡನು
ಕೇಳಿ-ಕೊಂ-ಡರು
ಕೇಳಿ-ಕೊಂ-ಡರೆ
ಕೇಳಿ-ಕೊಂ-ಡಳು
ಕೇಳಿ-ಕೊಂಡು
ಕೇಳಿ-ಕೊಂ-ಡೆನು
ಕೇಳಿ-ಕೊ-ಳ್ಳ-ಬ-ಹುದು
ಕೇಳಿ-ಕೊ-ಳ್ಳಲು
ಕೇಳಿ-ಕೊ-ಳ್ಳು-ವಂತೆ
ಕೇಳಿತು
ಕೇಳಿದ
ಕೇಳಿ-ದನು
ಕೇಳಿ-ದ-ಮೇಲೆ
ಕೇಳಿ-ದರು
ಕೇಳಿ-ದರೂ
ಕೇಳಿ-ದರೆ
ಕೇಳಿ-ದಳು
ಕೇಳಿ-ದ-ವನು
ಕೇಳಿ-ದ-ವರ
ಕೇಳಿ-ದ-ವ-ರಿಗೆ
ಕೇಳಿ-ದ-ವರು
ಕೇಳಿ-ದವು
ಕೇಳಿ-ದ-ಷ್ಟನ್ನು
ಕೇಳಿ-ದಷ್ಟು
ಕೇಳಿ-ದಾಗ
ಕೇಳಿ-ದಾಗಿ
ಕೇಳಿ-ದಿ-ರ-ಲ್ಲವೇ
ಕೇಳಿದು
ಕೇಳಿ-ದು-ದನ್ನು
ಕೇಳಿ-ದು-ದ-ರಿಂದ
ಕೇಳಿ-ದು-ದೆ-ಲ್ಲ-ವನ್ನೂ
ಕೇಳಿ-ದುದೇ
ಕೇಳಿದೆ
ಕೇಳಿ-ದೊ-ಡ-ನೆಯೆ
ಕೇಳಿದ್ದ
ಕೇಳಿ-ದ್ದಂ-ತೆಯೇ
ಕೇಳಿ-ದ್ದರು
ಕೇಳಿ-ದ್ದರೂ
ಕೇಳಿ-ದ್ದಳು
ಕೇಳಿ-ದ್ದೀಯೆ
ಕೇಳಿ-ದ್ದೀರಿ
ಕೇಳಿ-ದ್ದೇನೆ
ಕೇಳಿ-ದ್ದೇವೆ
ಕೇಳಿ-ಬಂತು
ಕೇಳಿ-ಬಂ-ತು-ಅಯ್ಯಾ
ಕೇಳಿ-ಬಂ-ತು-ಮಗು
ಕೇಳಿ-ಬಂ-ತು-ಮ-ಹಾ-ರಾಜ
ಕೇಳಿ-ಬಂ-ದಿತು
ಕೇಳಿ-ಬ-ರ-ಬಹು
ಕೇಳಿ-ಬ-ರು-ತ್ತದೆ
ಕೇಳಿಯೂ
ಕೇಳಿಯೇ
ಕೇಳಿ-ಯೇ-ಬಿಟ್ಟ
ಕೇಳಿರಿ
ಕೇಳಿರು
ಕೇಳಿ-ರುವ
ಕೇಳಿ-ರುವೆ
ಕೇಳಿ-ರು-ವೆ-ಯಲ್ಲ
ಕೇಳಿ-ಲ್ಲ-ದಂ-ತಹ
ಕೇಳಿ-ಲ್ಲವೆ
ಕೇಳಿ-ಸ-ಲಿಲ್ಲ
ಕೇಳಿ-ಸಿತು
ಕೇಳಿ-ಸು-ತ್ತಲೆ
ಕೇಳು
ಕೇಳು-ತ್ತಲೆ
ಕೇಳು-ತ್ತಲೇ
ಕೇಳುತ್ತಾ
ಕೇಳು-ತ್ತಾನೆ
ಕೇಳು-ತ್ತಿದ್ದ
ಕೇಳು-ತ್ತಿ-ದ್ದರು
ಕೇಳು-ತ್ತಿ-ದ್ದರೆ
ಕೇಳು-ತ್ತಿ-ದ್ದಾನೆ
ಕೇಳು-ತ್ತಿದ್ದಿ
ಕೇಳು-ತ್ತಿದ್ದೇ
ಕೇಳು-ತ್ತಿ-ರುವ
ಕೇಳು-ತ್ತೇವೆ
ಕೇಳುವ
ಕೇಳು-ವಂ-ತಿಲ್ಲ
ಕೇಳು-ವಂತೆ
ಕೇಳು-ವ-ಎಲ್ಲ
ಕೇಳು-ವ-ವನ
ಕೇಳು-ವ-ವನೂ
ಕೇಳು-ವ-ವ-ರನ್ನೂ
ಕೇಳು-ವ-ವರು
ಕೇಳು-ವ-ವರೂ
ಕೇಳು-ವಾಗ
ಕೇಳು-ವು-ದಕ್ಕೂ
ಕೇಳು-ವು-ದಕ್ಕೆ
ಕೇಳು-ವು-ದ-ರಲ್ಲಿ
ಕೇಳು-ವು-ದ-ರಿಂದ
ಕೇಳು-ವು-ದಿಲ್ಲ
ಕೇಳು-ವುದು
ಕೇಳು-ಹೀ-ಗೆಂದು
ಕೇಳೋಣ
ಕೇವಲ
ಕೇಶ-ಗಳು
ಕೇಶ-ರಾ-ಶಿ-ಯೇನು
ಕೇಶ-ವನು
ಕೇಶವೋ
ಕೇಶಿ
ಕೇಶಿ-ನಿಗೆ
ಕೇಶಿ-ಯೆಂಬ
ಕೇಶಿ-ರ-ಕ್ಕಸ
ಕೇಶಿ-ರ-ಕ್ಕ-ಸ-ನನ್ನು
ಕೇಶಿ-ರ-ಕ್ಕ-ಸನು
ಕೈ
ಕೈಂಕ-ರ್ಯ-ವನ್ನು
ಕೈಕಟ್ಟಿ
ಕೈಕ-ಟ್ಟಿ-ಕೊಂಡು
ಕೈಕಾಲು
ಕೈಕಾ-ಲು-ಗಳನ್ನು
ಕೈಕಾ-ಲು-ಗ-ಳಿಗೆ
ಕೈಕೆ-ಳ-ಗಾ-ದು-ದನ್ನು
ಕೈಕೇಯಿ
ಕೈಕೊಂಡ
ಕೈಕೊಂ-ಡನು
ಕೈಕೊಂ-ಡರು
ಕೈಕೊಂ-ಡರೆ
ಕೈಕೊಂ-ಡಳು
ಕೈಕೊಂ-ಡ-ವನು
ಕೈಕೊಂ-ಡಿ-ದ್ದನು
ಕೈಕೊಂ-ಡಿ-ರಲಿ
ಕೈಕೊಂ-ಡಿ-ರುವೆ
ಕೈಕೊಂಡು
ಕೈಕೊಳ್ಳ
ಕೈಕೊ-ಳ್ಳ-ಬೇಕು
ಕೈಕೊ-ಳ್ಳ-ಬೇ-ಕೆಂದು
ಕೈಕೊ-ಳ್ಳ-ಲೆಂದು
ಕೈಕೊಳ್ಳು
ಕೈಕೊ-ಳ್ಳುತ್ತಾ
ಕೈಕೊ-ಳ್ಳುವ
ಕೈಕೊ-ಳ್ಳು-ವ-ವನು
ಕೈಕೋ-ಲಿ-ನಿಂದ
ಕೈಗ-ಡಗ
ಕೈಗ-ನ್ನ-ಡಿ-ಯಂ-ತಿ-ರುವ
ಕೈಗಳ
ಕೈಗಳನ್ನು
ಕೈಗಳನ್ನೂ
ಕೈಗಳನ್ನೆಲ್ಲ
ಕೈಗಳಲ್ಲಿ
ಕೈಗ-ಳ-ಲ್ಲಿಯೂ
ಕೈಗ-ಳಾ-ದವು
ಕೈಗಳಿಂದ
ಕೈಗ-ಳಿಂ-ದಲೂ
ಕೈಗ-ಳಿಗೆ
ಕೈಗಳು
ಕೈಗಳೆ
ಕೈಗ-ಳೆ-ರ-ಡನ್ನೂ
ಕೈಗ-ಳೊ-ಡನೆ
ಕೈಗಾ-ಣಿಕೆ
ಕೈಗಾ-ಣಿ-ಕೆ-ಗಳ
ಕೈಗಾ-ಣಿ-ಕೆ-ಗಳನ್ನು
ಕೈಗಾ-ಣಿ-ಕೆ-ಗ-ಳೊ-ಡನೆ
ಕೈಗಾ-ಣಿ-ಕೆ-ಯಾಗಿ
ಕೈಗಾ-ಣಿ-ಕೆ-ಯೊ-ಡನೆ
ಕೈಗಿ-ತ್ತನು
ಕೈಗಿ-ತ್ತರು
ಕೈಗಿ-ತ್ತಳು
ಕೈಗೂ-ಡಿತು
ಕೈಗೆ
ಕೈಗೆತ್ತಿ
ಕೈಗೆ-ತ್ತಿ-ಕೊಂ-ಡನು
ಕೈಗೆ-ತ್ತಿ-ಕೊಂಡು
ಕೈಗೇ
ಕೈಗೇನೊ
ಕೈಗೊಂ-ಡ-ನೆಂದು
ಕೈಗೊಂ-ಡಿ-ದ್ದುದು
ಕೈಗೊಂಡು
ಕೈಗೊಂ-ಬೆ-ಗ-ಳಂತೆ
ಕೈಗೊಂ-ಬೆ-ಗ-ಳಾಗಿ
ಕೈಗೊಂ-ಬೆ-ಗಳು
ಕೈಗೊಂ-ಬೆ-ಯಾ-ಗಿ-ದ್ದನು
ಕೈಗೊಂ-ಬೆ-ಯಾ-ಗಿ-ರು-ನೆಂದೂ
ಕೈಗೊಂ-ಬೆ-ಯಾದ
ಕೈಗೊಂ-ಬೆ-ಯಾದೆ
ಕೈಗೊ-ಳ್ಳು-ವಂತೆ
ಕೈಜಾರಿ
ಕೈಜೋ-ಡಿಸಿ
ಕೈಜೋ-ಡಿ-ಸಿ-ಕೊಂಡು
ಕೈತಟ್ಟಿ
ಕೈತೋ-ಟ-ದಲ್ಲಿ
ಕೈನಲ್ಲಿ
ಕೈಬಳೆ
ಕೈಬ-ಳೆ-ಗಳನ್ನು
ಕೈಬಿ-ಟ್ಟಿತು
ಕೈಬಿ-ಡದೆ
ಕೈಬಿ-ಡು-ತ್ತೇನೆ
ಕೈಬಿ-ಡು-ವು-ದ-ಕ್ಕಾ-ಗು-ತ್ತ-ದೆಯೇ
ಕೈಬೊ-ಗಸೆ
ಕೈಮಿಂ-ಚಿ-ದು-ದನ್ನು
ಕೈಮಿಂ-ಚಿ-ಹೋ-ಗಿತ್ತು
ಕೈಮೀರಿ
ಕೈಮು-ಗಿದ
ಕೈಮು-ಗಿದು
ಕೈಮು-ಗಿ-ದು-ಕೊಂಡು
ಕೈಯ
ಕೈಯನ್ನು
ಕೈಯನ್ನೆ
ಕೈಯಲ್ಲಿ
ಕೈಯಾರೆ
ಕೈಯಿಂದ
ಕೈಯಿಂ-ದಲೇ
ಕೈಯಿಟ್ಟ
ಕೈಯಿ-ಟ್ಟಿ-ರು-ವುದನ್ನು
ಕೈಯಿ-ಟ್ಟುಕೊ
ಕೈಯಿಡು
ಕೈಯೆತ್ತಿ
ಕೈಯೊ-ಡ್ಡಿದ
ಕೈಯೊ-ಡ್ಡಿ-ದರೆ
ಕೈಯೊ-ಡ್ಡು-ವುದೆ
ಕೈಲಾ-ಗದ
ಕೈಲಾ-ಸಕ್ಕೆ
ಕೈಲಾ-ಸದ
ಕೈಲಾ-ಸ-ದಲ್ಲಿ
ಕೈಲಾ-ಸ-ಪ-ರ್ವ-ತ-ದಲ್ಲಿ
ಕೈಲಾ-ಸ-ವನ್ನು
ಕೈಲಿ
ಕೈಲಿ-ಟ್ಟರು
ಕೈಲಿಟ್ಟು
ಕೈಲಿ-ಡ-ಬೇ-ಕೆಂದು
ಕೈಲಿದೆ
ಕೈಲಿದ್ದ
ಕೈಲಿ-ರುವ
ಕೈವ-ಲ್ಯ-ಪ-ತಯೇ
ಕೈವ-ಲ್ಯ-ಪ-ತಿಯೂ
ಕೈವ-ಲ್ಯ-ವೆಂದೂ
ಕೈವ-ಶ-ವಾ-ಯಿತು
ಕೈಸೆರೆ
ಕೈಸೇ-ರು-ತ್ತಿ-ರು-ವಾಗ
ಕೈಸೋಕಿ
ಕೈಹಾ-ಕದೆ
ಕೈಹಾ-ಕಲು
ಕೈಹಾ-ಕಿದ
ಕೈಹಾ-ಕಿ-ದನು
ಕೈಹಾ-ಕು-ವುದೆ
ಕೈಹಿ-ಡಿದ
ಕೈಹಿ-ಡಿ-ದನು
ಕೈಹಿ-ಡಿ-ದಿ-ದ್ದರೆ
ಕೈಹಿ-ಡಿದು
ಕೈಹಿ-ಡಿ-ದು-ಕೊಂಡು
ಕೈಹಿ-ಡಿ-ಯ-ಬೇಕು
ಕೈಹಿ-ಡಿ-ಯ-ಬೇ-ಕೆಂಬ
ಕೈಹಿ-ಡಿವ
ಕೊಂಕ-ದಿದ್ದ
ಕೊಂಕ-ದಿ-ರು-ವುದನ್ನು
ಕೊಂಕ-ಲಿಲ್ಲ
ಕೊಂಕ-ಲಿ-ಲ್ಲ-ವಾ-ದರೂ
ಕೊಂಕ-ಳಲ್ಲಿ
ಕೊಂಕ-ಳ-ಲ್ಲಿದ್ದ
ಕೊಂಕಾದ
ಕೊಂಕು
ಕೊಂಕು-ಳಲ್ಲಿ
ಕೊಂಡ
ಕೊಂಡನು
ಕೊಂಡರು
ಕೊಂಡರೆ
ಕೊಂಡಳು
ಕೊಂಡ-ವರೂ
ಕೊಂಡವು
ಕೊಂಡಾ
ಕೊಂಡಾಗ
ಕೊಂಡಾಡ
ಕೊಂಡಾ-ಡಲಿ
ಕೊಂಡಾ-ಡಿ-ದರು
ಕೊಂಡಾ-ಡುತ್ತಾ
ಕೊಂಡಾ-ಡುವ
ಕೊಂಡಾ-ಡು-ವಂತೆ
ಕೊಂಡಾ-ಡು-ವುದು
ಕೊಂಡಾ-ರೆಂಬ
ಕೊಂಡಿ
ಕೊಂಡಿತು
ಕೊಂಡಿ-ತೆಂ-ದ-ಮೇಲೆ
ಕೊಂಡಿದ್ದ
ಕೊಂಡಿ-ದ್ದಳು
ಕೊಂಡಿ-ದ್ದಾನೆ
ಕೊಂಡಿದ್ದು
ಕೊಂಡಿ-ಯಂತೆ
ಕೊಂಡಿ-ರುವ
ಕೊಂಡಿ-ರು-ವ-ನಲ್ಲಾ
ಕೊಂಡಿ-ರುವೆ
ಕೊಂಡು
ಕೊಂಡು-ಕೊಂಡು
ಕೊಂಡು-ದನ್ನು
ಕೊಂಡು-ದ-ರಿಂದ
ಕೊಂಡುದು
ಕೊಂಡು-ಹೋಗು
ಕೊಂಡು-ಹೋದ
ಕೊಂಡೆ
ಕೊಂಡೆಯಾ
ಕೊಂಡೊ
ಕೊಂಡೊಯ್ದ
ಕೊಂಡೊ-ಯ್ದನು
ಕೊಂಡೊ-ಯ್ದರು
ಕೊಂಡೊಯ್ದು
ಕೊಂಡೊ-ಯ್ಯ-ಬಲ್ಲ
ಕೊಂಡೊಯ್ಯಿ
ಕೊಂಡೊ-ಯ್ಯುತ್ತಿ
ಕೊಂಡೊ-ಯ್ಯು-ತ್ತಿ-ದ್ದಾಗ
ಕೊಂಡೊ-ಯ್ಯು-ತ್ತಿ-ದ್ದೇನೆ
ಕೊಂಡೊ-ಯ್ಯು-ತ್ತಿ-ರು-ವಾಗ
ಕೊಂಡೊ-ಯ್ಯು-ವಂತೆ
ಕೊಂಡೊ-ಯ್ಯು-ವುದು
ಕೊಂದ
ಕೊಂದ-ನಂತೆ
ಕೊಂದನು
ಕೊಂದ-ನೆಂದು
ಕೊಂದ-ಮೆಲೆ
ಕೊಂದ-ಮೇಲೆ
ಕೊಂದರು
ಕೊಂದರೂ
ಕೊಂದರೆ
ಕೊಂದ-ವನ
ಕೊಂದ-ವ-ನಂತೆ
ಕೊಂದಾರು
ಕೊಂದಿತು
ಕೊಂದಿ-ರ-ಬೇ-ಕೆಂದು
ಕೊಂದು
ಕೊಂದು-ದನ್ನು
ಕೊಂದು-ದನ್ನೂ
ಕೊಂದುದು
ಕೊಂದುದೇ
ಕೊಂದುದೊ
ಕೊಂದು-ಬ-ರು-ವಂತೆ
ಕೊಂದು-ಬಿ-ಡ-ಬೇ-ಕೆಂದು
ಕೊಂದು-ಹಾ-ಕ-ಬೇಕು
ಕೊಂದು-ಹಾ-ಕ-ಲೇ-ಬೇ-ಕೆಂ-ಬಷ್ಟು
ಕೊಂದು-ಹಾಕಿ
ಕೊಂದು-ಹಾ-ಕಿದ
ಕೊಂದು-ಹಾ-ಕಿ-ದ-ನಂತೆ
ಕೊಂದು-ಹಾ-ಕಿ-ದ-ನಲ್ಲಾ
ಕೊಂದು-ಹಾ-ಕಿ-ದನು
ಕೊಂದು-ಹಾ-ಕಿ-ದ-ವರು
ಕೊಂದು-ಹಾ-ಕಿದೆ
ಕೊಂದು-ಹಾ-ಕಿ-ಬಿ-ಡಲೆ
ಕೊಂದು-ಹಾ-ಕಿ-ರ-ಬೇ-ಕೆಂ-ದು-ಕೊಂಡ
ಕೊಂದು-ಹಾ-ಕಿರಿ
ಕೊಂದು-ಹಾ-ಕಿ-ರು-ವಿ-ರಂತೆ
ಕೊಂದು-ಹಾಕು
ಕೊಂದು-ಹಾ-ಕುತ್ತಾ
ಕೊಂದು-ಹಾ-ಕು-ತ್ತಾನೆ
ಕೊಂದು-ಹಾ-ಕು-ತ್ತಿ-ದ್ದರು
ಕೊಂದು-ಹಾ-ಕು-ವಂತೆ
ಕೊಂದು-ಹಾ-ಕು-ವನು
ಕೊಂದು-ಹಾ-ಕು-ವು-ದ-ಕ್ಕಾಗಿ
ಕೊಂದು-ಹಾ-ಕು-ವು-ದೆಂ-ದರೆ
ಕೊಂದೆ-ನೆಂ-ಬುದು
ಕೊಂದೆ-ಯಲ್ಲಾ
ಕೊಂಬಿಗೆ
ಕೊಂಬಿತ್ತು
ಕೊಂಬಿನ
ಕೊಂಬಿ-ನಿಂದ
ಕೊಂಬು
ಕೊಂಬು-ಗಳ
ಕೊಂಬು-ಗಳನ್ನು
ಕೊಂಬು-ಗ-ಳ-ಲ್ಲಿಯೇ
ಕೊಂಬು-ಗಳಿಂದ
ಕೊಂಬು-ಗಳು
ಕೊಂಬು-ಗ-ಳೆ-ರ-ಡನ್ನೂ
ಕೊಂಬೆ
ಕೊಂಬೆ-ಗಳನ್ನು
ಕೊಂಬೆಗೆ
ಕೊಂಬೆ-ಯನ್ನು
ಕೊಂಬೆ-ಯಲ್ಲಿ
ಕೊಂಬೆ-ಯಿಂದ
ಕೊಕ್ಕು-ಗಳನ್ನು
ಕೊಕ್ಕು-ಗಳಿಂದ
ಕೊಚ್ಚಿ
ಕೊಚ್ಚಿ-ಕೊಂಡು
ಕೊಚ್ಚಿ-ಹಾಕಿ
ಕೊಚ್ಚಿ-ಹಾ-ಕಿತು
ಕೊಚ್ಚಿ-ಹೋ-ಗದೆ
ಕೊಚ್ಚಿ-ಹೋ-ಗು-ವಂತೆ
ಕೊಚ್ಚಿ-ಹೋ-ಯಿತು
ಕೊಚ್ಚು-ತ್ತಿ-ರು-ವೆ-ಯಲ್ಲಾ
ಕೊಟ್ಟ
ಕೊಟ್ಟಂ-ತಹ
ಕೊಟ್ಟಂ-ತಾ-ಯಿತು
ಕೊಟ್ಟಂತೆ
ಕೊಟ್ಟನು
ಕೊಟ್ಟ-ಮಾತು
ಕೊಟ್ಟರು
ಕೊಟ್ಟರೂ
ಕೊಟ್ಟರೆ
ಕೊಟ್ಟ-ರೆಂ-ದರೆ
ಕೊಟ್ಟಳು
ಕೊಟ್ಟ-ಳು-ಮ-ಹಾ-ರಾಜ
ಕೊಟ್ಟ-ವನು
ಕೊಟ್ಟ-ವರು
ಕೊಟ್ಟಾಗ
ಕೊಟ್ಟಿ-ಗೆಗೆ
ಕೊಟ್ಟಿ-ಗೆ-ಯೊ-ಳಗೆ
ಕೊಟ್ಟಿತು
ಕೊಟ್ಟಿ-ದನ್ನು
ಕೊಟ್ಟಿದ್ದ
ಕೊಟ್ಟಿ-ದ್ದ-ನಷ್ಟೆ
ಕೊಟ್ಟಿ-ದ್ದರು
ಕೊಟ್ಟಿ-ದ್ದರೂ
ಕೊಟ್ಟಿ-ದ್ದಾನೆ
ಕೊಟ್ಟಿ-ದ್ದಾರೆ
ಕೊಟ್ಟಿ-ದ್ದೆವು
ಕೊಟ್ಟಿ-ದ್ದೇನೆ
ಕೊಟ್ಟಿರ
ಕೊಟ್ಟಿ-ರುವ
ಕೊಟ್ಟಿ-ರು-ವ-ನೆಂದು
ಕೊಟ್ಟಿ-ರು-ವು-ದ-ರಿಂದ
ಕೊಟ್ಟಿ-ರು-ವೆನು
ಕೊಟ್ಟೀಯ
ಕೊಟ್ಟು
ಕೊಟ್ಟು-ದ-ಕ್ಕಾಗಿ
ಕೊಟ್ಟು-ದನ್ನು
ಕೊಟ್ಟುದು
ಕೊಟ್ಟು-ಬಿಡು
ಕೊಟ್ಟು-ಹೋದ
ಕೊಟ್ಟೆ
ಕೊಟ್ಟೆ-ನಲ್ಲ
ಕೊಟ್ಟೆ-ನಾ-ದರೂ
ಕೊಟ್ಟೇನು
ಕೊಡ
ಕೊಡ-ತ-ಕ್ಕ-ವನು
ಕೊಡದ
ಕೊಡ-ದಲ್ಲಿ
ಕೊಡ-ದಾಗ
ಕೊಡ-ದಿ-ದ್ದರೆ
ಕೊಡದೆ
ಕೊಡ-ಬಲ್ಲ
ಕೊಡ-ಬ-ಲ್ಲೆವು
ಕೊಡ-ಬ-ಹು-ದೆಂದು
ಕೊಡ-ಬಾ-ರ-ದಯ್ಯಾ
ಕೊಡ-ಬಾ-ರದು
ಕೊಡ-ಬೇ-ಕಾ-ಗಿತ್ತು
ಕೊಡ-ಬೇಕು
ಕೊಡ-ಬೇಕೆ
ಕೊಡ-ಬೇ-ಕೆಂ-ದಿ-ರುವ
ಕೊಡ-ಬೇ-ಕೆಂದು
ಕೊಡ-ಬೇ-ಕೆಂ-ಬುದು
ಕೊಡ-ಲಾ-ರದೆ
ಕೊಡ-ಲಾ-ರೆನೆ
ಕೊಡಲಿ
ಕೊಡ-ಲಿ-ಎಂದು
ಕೊಡ-ಲಿಗೆ
ಕೊಡ-ಲಿಯ
ಕೊಡ-ಲಿ-ಯಿಂದ
ಕೊಡ-ಲಿಲ್ಲ
ಕೊಡಲು
ಕೊಡಲೂ
ಕೊಡ-ವನ್ನು
ಕೊಡ-ವಿ-ಕೊಂಡು
ಕೊಡಿ
ಕೊಡಿ-ಸ-ಬೇ-ಕೆಂದು
ಕೊಡಿ-ಸ-ಬೇ-ಕೆಂಬ
ಕೊಡಿಸಿ
ಕೊಡಿ-ಸಿ-ಕೊಡು
ಕೊಡಿ-ಸಿ-ಕೊ-ಡು-ವಂತೆ
ಕೊಡಿ-ಸಿ-ದನು
ಕೊಡಿ-ಸಿ-ದು-ದಾ-ಯಿತು
ಕೊಡಿಸು
ಕೊಡಿ-ಸು-ತ್ತೇನೆ
ಕೊಡು
ಕೊಡು-ಎಂಬ
ಕೊಡು-ಗೆ-ಯೆಂಬ
ಕೊಡುಗೈ
ಕೊಡು-ಗೈಯ
ಕೊಡುತ್ತಾ
ಕೊಡು-ತ್ತಾರೆ
ಕೊಡು-ತ್ತಾರೊ
ಕೊಡು-ತ್ತಾ-ಹೋ-ದರೆ
ಕೊಡುತ್ತಿ
ಕೊಡು-ತ್ತಿತ್ತು
ಕೊಡು-ತ್ತಿ-ದ್ದನು
ಕೊಡು-ತ್ತಿ-ದ್ದ-ವರು
ಕೊಡು-ತ್ತಿ-ದ್ದಾಗ
ಕೊಡು-ತ್ತಿ-ರು-ವೆನು
ಕೊಡು-ತ್ತಿ-ರು-ವೆ-ವಲ್ಲ
ಕೊಡು-ತ್ತೀಯಾ
ಕೊಡು-ತ್ತೀಯೋ
ಕೊಡು-ತ್ತೀರಾ
ಕೊಡು-ತ್ತೇನೆ
ಕೊಡು-ತ್ತೇವೆ
ಕೊಡುವ
ಕೊಡು-ವಂ-ತಿ-ರ-ಲಿಲ್ಲ
ಕೊಡು-ವಂತೆ
ಕೊಡು-ವನು
ಕೊಡು-ವ-ವನು
ಕೊಡು-ವ-ವನೂ
ಕೊಡು-ವ-ವೆಂದು
ಕೊಡುವು
ಕೊಡು-ವು-ದ-ಕ್ಕಾಗಿ
ಕೊಡು-ವು-ದ-ರಲ್ಲಿ
ಕೊಡು-ವು-ದಾಗಿ
ಕೊಡು-ವು-ದಾ-ದರೆ
ಕೊಡು-ವು-ದಿಲ್ಲ
ಕೊಡು-ವುದು
ಕೊಡುವೆ
ಕೊಡು-ವೆ-ನೆಂ-ದರೆ
ಕೊಡೆ
ಕೊಡೆಂ-ದರೂ
ಕೊಡೆಯ
ಕೊಡೆ-ಯನ್ನೆ
ಕೊಡೆ-ಯ-ನ್ನೆತ್ತಿ
ಕೊಡೆ-ಯಾಗಿ
ಕೊಡೆ-ಯಾ-ಯಿತು
ಕೊತ್ತ-ಲ-ಗಳನ್ನು
ಕೊತ್ತ-ಲು-ಗಳನ್ನು
ಕೊನೆ
ಕೊನೆ-ಕೊ-ನೆಗೆ
ಕೊನೆ-ಗ-ದನ್ನು
ಕೊನೆ-ಗಳನ್ನು
ಕೊನೆ-ಗಾ-ಣಿಸು
ಕೊನೆ-ಗಾ-ಲ-ದಲ್ಲಿ
ಕೊನೆಗೆ
ಕೊನೆ-ಗೊಂದು
ಕೊನೆ-ಮೊ-ದ-ಲಿಲ್ಲ
ಕೊನೆ-ಮೊ-ದ-ಲಿ-ಲ್ಲ-ದಷ್ಟು
ಕೊನೆ-ಮೊ-ದ-ಲಿ-ಲ್ಲದೆ
ಕೊನೆ-ಮೊ-ದ-ಲೆಂ-ಬುದು
ಕೊನೆ-ಮೊ-ದಲೇ
ಕೊನೆಯ
ಕೊನೆ-ಯನ್ನು
ಕೊನೆ-ಯಲ್ಲಿ
ಕೊನೆ-ಯ-ವನು
ಕೊನೆ-ಯಾ-ಗುವ
ಕೊನೆ-ಯಾ-ದು-ವೆಂದು
ಕೊನೆ-ಯಾ-ಯಿತು
ಕೊನೆ-ಯಿಲ್ಲ
ಕೊನೆ-ಯಿ-ಲ್ಲದ
ಕೊನೆ-ಯಿ-ಲ್ಲವೋ
ಕೊನೆ-ಯೆ-ರ-ಡನ್ನು
ಕೊನೆ-ಯೆಲ್ಲಿ
ಕೊನೆಯೇ
ಕೊಪ್ಪ-ರಿ-ಗೆ-ಯ-ಲ್ಲಿ-ರಿಸಿ
ಕೊಬ್ಬ-ನ್ನಿ-ಳಿ-ಸಿದ
ಕೊಬ್ಬನ್ನು
ಕೊಬ್ಬಾ
ಕೊಬ್ಬಿ
ಕೊಬ್ಬಿದ
ಕೊಬ್ಬಿ-ದರೆ
ಕೊಬ್ಬಿ-ರುವ
ಕೊಬ್ಬಿ-ಹೋ-ಗಿ-ರುವ
ಕೊಬ್ಬು
ಕೊಯ್ದು
ಕೊರ-ಗ-ಲಿಲ್ಲ
ಕೊರ-ಗಾ-ಗಿತ್ತು
ಕೊರಗಿ
ಕೊರ-ಗು-ತ್ತಿ-ದ್ದನು
ಕೊರ-ಗು-ತ್ತಿ-ದ್ದಳು
ಕೊರತೆ
ಕೊರ-ತೆ-ಯಿ-ದ್ದೀತು
ಕೊರ-ತೆಯೂ
ಕೊರ-ತೆಯೆ
ಕೊರಳ
ಕೊರ-ಳನ್ನು
ಕೊರ-ಳಲ್ಲಿ
ಕೊರ-ಳ-ಲ್ಲಿದ್ದ
ಕೊರ-ಳ-ಲ್ಲಿಯೂ
ಕೊರ-ಳ-ಹಾರ
ಕೊರ-ಳಿಗೆ
ಕೊರಳು
ಕೊರೆ-ಸ-ಬೇಕು
ಕೊಲೆ
ಕೊಲೆ-ಗ-ಡು-ಕನ
ಕೊಲೆ-ಮಾ-ಡು-ವು-ದಕ್ಕೆ
ಕೊಲೆಯ
ಕೊಲೆ-ಯನ್ನು
ಕೊಲ್ಲ
ಕೊಲ್ಲದೆ
ಕೊಲ್ಲ-ಬ-ಲ್ಲೆಯಾ
ಕೊಲ್ಲ-ಬ-ಹುದೆ
ಕೊಲ್ಲ-ಬಾ-ರದು
ಕೊಲ್ಲ-ಬೇಕು
ಕೊಲ್ಲ-ಬೇ-ಕೆಂದು
ಕೊಲ್ಲ-ಬೇ-ಕೆಂದೆ
ಕೊಲ್ಲ-ಬೇ-ಕೆಂಬ
ಕೊಲ್ಲ-ಬೇ-ಕೆಂ-ಬುದೇ
ಕೊಲ್ಲ-ಬೇಡ
ಕೊಲ್ಲ-ಲಾರೆ
ಕೊಲ್ಲಲು
ಕೊಲ್ಲಲೂ
ಕೊಲ್ಲ-ಲೆಂದು
ಕೊಲ್ಲ-ಲೆ-ಳ-ಸಿ-ದರು
ಕೊಲ್ಲವ
ಕೊಲ್ಲ-ವುದು
ಕೊಲ್ಲ-ಹೊ-ರ-ಟರು
ಕೊಲ್ಲ-ಹೊ-ರ-ಟರೆ
ಕೊಲ್ಲ-ಹೊ-ರ-ಟಿ-ರು-ವುದು
ಕೊಲ್ಲ-ಹೊ-ರ-ಟಿ-ರುವೆ
ಕೊಲ್ಲಿಸ
ಕೊಲ್ಲಿ-ಸ-ಬೇಕು
ಕೊಲ್ಲಿ-ಸಲೆ
ಕೊಲ್ಲಿಸಿ
ಕೊಲ್ಲಿ-ಸಿದ
ಕೊಲ್ಲಿ-ಸಿ-ದ್ದೇನೆ
ಕೊಲ್ಲಿ-ಸು-ತ್ತೇನೆ
ಕೊಲ್ಲಿ-ಸುವ
ಕೊಲ್ಲಿ-ಸು-ವು-ದಕ್ಕೂ
ಕೊಲ್ಲಿ-ಸು-ವು-ದಕ್ಕೆ
ಕೊಲ್ಲಿ-ಸು-ವು-ದೊಂದು
ಕೊಲ್ಲು
ಕೊಲ್ಲುತ್ತಾ
ಕೊಲ್ಲು-ತ್ತಾನೆ
ಕೊಲ್ಲುತ್ತಿ
ಕೊಲ್ಲು-ತ್ತಿ-ದ್ದನು
ಕೊಲ್ಲು-ತ್ತಿ-ದ್ದಾರೆ
ಕೊಲ್ಲು-ತ್ತೇನೆ
ಕೊಲ್ಲುವ
ಕೊಲ್ಲು-ವಂ-ತಹ
ಕೊಲ್ಲು-ವನು
ಕೊಲ್ಲು-ವ-ನೆಂದು
ಕೊಲ್ಲು-ವ-ವ-ನಾ-ಗಿ-ದ್ದಾನೆ
ಕೊಲ್ಲು-ವ-ವ-ನಾರು
ಕೊಲ್ಲು-ವ-ವ-ನಿ-ಗಿಂತ
ಕೊಲ್ಲು-ವ-ವರು
ಕೊಲ್ಲು-ವ-ವ-ಳಲ್ಲ
ಕೊಲ್ಲುವು
ಕೊಲ್ಲು-ವು-ದ-ಕ್ಕಾಗಿ
ಕೊಲ್ಲು-ವು-ದ-ಕ್ಕಾ-ಗಿಯೆ
ಕೊಲ್ಲು-ವು-ದಕ್ಕೆ
ಕೊಲ್ಲು-ವು-ದಿ-ರಲಿ
ಕೊಲ್ಲು-ವು-ದಿಲ್ಲ
ಕೊಲ್ಲು-ವುದು
ಕೊಲ್ಲು-ವು-ದೆಂ-ದರೆ
ಕೊಲ್ಲು-ವೆ-ನೆಂಬ
ಕೊಲ್ಲೆಂದು
ಕೊಳ
ಕೊಳ-ಗಳ
ಕೊಳ-ದಲ್ಲಿ
ಕೊಳ-ದಿಂದ
ಕೊಳಲ
ಕೊಳ-ಲ-ಗಾನ
ಕೊಳ-ಲ-ಗಾ-ನಕ್ಕೆ
ಕೊಳ-ಲ-ಗಾ-ನದ
ಕೊಳ-ಲ-ಗಾ-ನ-ವನ್ನು
ಕೊಳ-ಲ-ದನಿ
ಕೊಳ-ಲ-ದ-ನಿಗೆ
ಕೊಳ-ಲನ್ನು
ಕೊಳ-ಲಿನ
ಕೊಳಲು
ಕೊಳ-ಲು-ಗಾ-ನ-ದಿಂದ
ಕೊಳ-ಲೂ-ದುತ್ತಾ
ಕೊಳ-ಲೂ-ದು-ವರು
ಕೊಳ-ಲೂ-ದು-ವುದನ್ನು
ಕೊಳೆ-ತುಂಬಿ
ಕೊಳೆ-ಯ-ಬೇ-ಕಾ-ಗು-ತ್ತದೆ
ಕೊಳ್ಳ-ಬ-ಹುದು
ಕೊಳ್ಳ-ಬೇಕು
ಕೊಳ್ಳ-ಬೇ-ಕೆಂ-ದಿ-ರುವ
ಕೊಳ್ಳ-ಬೇ-ಕೆಂದು
ಕೊಳ್ಳ-ಬೇಡಿ
ಕೊಳ್ಳಲು
ಕೊಳ್ಳಲೆ
ಕೊಳ್ಳ-ಲೇ-ಬೇಕು
ಕೊಳ್ಳವ
ಕೊಳ್ಳಿ-ಯಿಂದ
ಕೊಳ್ಳು-ತ್ತದೆ
ಕೊಳ್ಳುತ್ತಾ
ಕೊಳ್ಳು-ತ್ತಾನೆ
ಕೊಳ್ಳು-ತ್ತಿ-ದ್ದರು
ಕೊಳ್ಳು-ತ್ತಿ-ದ್ದವು
ಕೊಳ್ಳು-ತ್ತೀಯೆ
ಕೊಳ್ಳು-ತ್ತೇ-ನಣ್ಣ
ಕೊಳ್ಳು-ತ್ತೇನೆ
ಕೊಳ್ಳುವ
ಕೊಳ್ಳು-ವಂತೆ
ಕೊಳ್ಳು-ವ-ವ-ನಂತೆ
ಕೊಳ್ಳು-ವ-ವ-ನಾರು
ಕೊಳ್ಳು-ವಷ್ಟು
ಕೊಳ್ಳು-ವು-ದ-ಕ್ಕಾಗಿ
ಕೊಳ್ಳು-ವು-ದಕ್ಕೂ
ಕೊಳ್ಳು-ವು-ದಕ್ಕೆ
ಕೊಳ್ಳೆ
ಕೊಳ್ಳೆ-ಹೊ-ಡೆ-ಯುವ
ಕೊಳ್ಳೆ-ಹೊ-ಡೆ-ಯು-ವರು
ಕೋ
ಕೋಕಿಲ
ಕೋಗಿಲೆ
ಕೋಗಿ-ಲೆಗೆ
ಕೋಟ-ರೆ-ಯೆಂ-ಬು-ವಳು
ಕೋಟ-ಲೆ-ಯಂತೂ
ಕೋಟ-ಲೆ-ಯಿಂದ
ಕೋಟಿ
ಕೋಟಿ-ಕೋಟಿ
ಕೋಟಿ-ಗ-ಟ್ಟ-ಲೆ-ಯಾಗಿ
ಕೋಟಿ-ಗ-ಳೆಲ್ಲ
ಕೋಟಿಗೆ
ಕೋಟೆ
ಕೋಟೆ-ಗಳನ್ನೂ
ಕೋಟೆಗೆ
ಕೋಟೆಯ
ಕೋಟೆ-ಯಂತೆ
ಕೋಟೆ-ಯನ್ನು
ಕೋಟೆ-ಯಲ್ಲಿ
ಕೋಟೆ-ಯಿಂದ
ಕೋಡ-ಣ-ಗ-ಳಿಂ-ದಲೂ
ಕೋಡಿ
ಕೋಡಿ-ಗಟ್ಟಿ
ಕೋಡು
ಕೋಡು-ಗ-ಳಿಗೆ
ಕೋಣ-ಗ-ಳಂತೆ
ಕೋಣೆಗೆ
ಕೋತಿ
ಕೋತಿ-ಗಳ
ಕೋತಿ-ಗ-ಳಿಗೆ
ಕೋತಿಗೆ
ಕೋತಿಯ
ಕೋಪ
ಕೋಪಕ್ಕೆ
ಕೋಪ-ಗ-ಳೆಲ್ಲ
ಕೋಪ-ಗ-ಳ್ಳು-ವುದು
ಕೋಪ-ಗೊಂಡ
ಕೋಪ-ಗೊಂ-ಡರು
ಕೋಪ-ಗೊಂ-ಡರೆ
ಕೋಪ-ಗೊಂ-ಡ-ವ-ನಾಗಿ
ಕೋಪ-ಗೊಂ-ಡಿದ್ದ
ಕೋಪ-ಗೊಂ-ಡಿರು
ಕೋಪ-ಗೊಂಡು
ಕೋಪ-ಗೊಂ-ಡೆಯೋ
ಕೋಪ-ದಲ್ಲಿ
ಕೋಪ-ದಿಂದ
ಕೋಪ-ದಿಂ-ದಲೋ
ಕೋಪ-ಬಂತು
ಕೋಪ-ಬಂ-ದರೆ
ಕೋಪ-ಬಂ-ದಿತು
ಕೋಪ-ವನ್ನು
ಕೋಪ-ವಲ್ಲ
ಕೋಪವೂ
ಕೋಪ-ವೆಲ್ಲ
ಕೋಪ-ವೇನೂ
ಕೋಪಿ-ಸು-ವ-ನೆಂಬ
ಕೋಪು-ವು-ಕ್ಕಿತು
ಕೋಮಲ
ಕೋಮಿನ
ಕೋರ-ಯಿ-ಸು-ತ್ತಿ-ರಲು
ಕೋರಿಕೆ
ಕೋರಿ-ಕೆ-ಗಳನ್ನೆಲ್ಲ
ಕೋರಿ-ಕೆ-ಗ-ಳೆಲ್ಲ
ಕೋರಿ-ಕೆ-ಯಂತೆ
ಕೋರಿ-ಕೆಯೂ
ಕೋರಿ-ಕೆ-ಯೇನು
ಕೋರಿ-ಕೆ-ಯೇ-ನುಂಟು
ಕೋರಿ-ದನು
ಕೋರೆ
ಕೋರೆ-ದಾಡೆ
ಕೋರೆ-ದಾ-ಡೆ-ಗಳಿಂದ
ಕೋರೆ-ದಾ-ಡೆ-ಗಳು
ಕೋರೆ-ದಾ-ಡೆ-ಗ-ಳುಳ್ಳ
ಕೋರೆ-ದಾ-ಡೆಯ
ಕೋರೆ-ಹಲ್ಲು
ಕೋರೈ-ಸು-ತ್ತಿ-ರಲು
ಕೋರೈ-ಸುವ
ಕೋಲಾ-ಟ-ವಾ-ಡು-ವರು
ಕೋಲಿ-ನಿಂದ
ಕೋಲಿ-ನಿಂ-ದಲೂ
ಕೋಳಿ
ಕೋಳಿ-ಯಿ-ಲ್ಲದೆ
ಕೋಳು-ಹೋ-ಗದ
ಕೋವಿದ
ಕೋಶ-ದಲ್ಲಿ
ಕೋಸಲ
ಕೋಸ-ಲ-ದೇ-ಶದ
ಕೌಟಿ-ಲ್ಯನ
ಕೌಟಿ-ಲ್ಯನು
ಕೌಟಿ-ಲ್ಯ-ನೆಂಬ
ಕೌಪೀ-ನ-ವಾಗಿ
ಕೌಮಾ-ರಾ-ವ-ತಾ-ರ-ವೆಂದು
ಕೌಮೋ-ದ-ಕಿ-ಯೆಂಬ
ಕೌರವ
ಕೌರ-ವ-ನನ್ನು
ಕೌರ-ವ-ನಿಗೆ
ಕೌರ-ವನು
ಕೌರ-ವರ
ಕೌರ-ವ-ರನ್ನು
ಕೌರ-ವ-ರಾ-ಜ-ನಾದ
ಕೌರ-ವರಿ
ಕೌರ-ವ-ರಿಗೆ
ಕೌರ-ವ-ರಿ-ಗೆಲ್ಲ
ಕೌರ-ವರು
ಕೌರ-ವರೆಲ್ಲ
ಕೌರ-ವ-ಸೇ-ನೆಯ
ಕೌಶಕೀ
ಕೌಸ-ಲ್ಯೆ-ಕೇಶಿ
ಕೌಸ್ತುಭ
ಕೌಸ್ತು-ಭ-ದಂ-ತಹ
ಕೌಸ್ತು-ಭ-ಮಣಿ
ಕೌಸ್ತು-ಭ-ರತ್ನ
ಕ್ಕಾಗಿ
ಕ್ಕಾಗಿಯೆ
ಕ್ಕಿಂತ
ಕ್ಕಿಂತಲೂ
ಕ್ಕಿದ್ದಂತೆ
ಕ್ಕಿಳಿದು
ಕ್ಕುರು-ಳ-ದಂತೆ
ಕ್ಕೆ
ಕ್ಕೆಂದು
ಕ್ಕೆತ್ತಿ-ದನು
ಕ್ಕೆದ್ದು
ಕ್ಕೇಳ-ಲಿಲ್ಲ
ಕ್ಕೊಳ-ಗಾ-ದರು
ಕ್ಕೋಡಿ
ಕ್ರತು
ಕ್ರತುವೂ
ಕ್ರಮ
ಕ್ರಮಕ್ಕೆ
ಕ್ರಮ-ಕ್ರ-ಮ-ವಾಗಿ
ಕ್ರಮ-ಗಳನ್ನು
ಕ್ರಮ-ದಲ್ಲಿ
ಕ್ರಮ-ದ-ಲ್ಲಿಯೇ
ಕ್ರಮ-ಪೂ-ಜೆ-ಯನ್ನು
ಕ್ರಮ-ರೂ-ಪಿ-ಯಾದ
ಕ್ರಮ-ವಾಗಿ
ಕ್ರಮ-ವಿದೆ
ಕ್ರಮಶಃ
ಕ್ರಮೇಣ
ಕ್ರಯ
ಕ್ರಯಕ್ಕೆ
ಕ್ರಾಂತ-ನಾ-ದನು
ಕ್ರಿ
ಕ್ರಿಪೂ
ಕ್ರಿಮಿ-ಗಳ
ಕ್ರಿಮಿ-ಗಳು
ಕ್ರಿಯಾ-ದೇ-ವಿ-ಯನ್ನೂ
ಕ್ರಿಯಾ-ಶಕ್ತಿ
ಕ್ರಿಯಾ-ಶ-ಕ್ತಿ-ಗಳನ್ನು
ಕ್ರಿಯೆ
ಕ್ರಿಶ
ಕ್ರೀಡಾ
ಕ್ರೀಡಾ-ಸ್ಥಾ-ನ-ವಾ-ಗಿ-ರುವ
ಕ್ರೀಡೆ-ಯಲ್ಲಿ
ಕ್ರೀಡೋ-ದ್ಯಾನ
ಕ್ರೂರ
ಕ್ರೂರ-ಕಾ-ರ್ಯ-ವನ್ನು
ಕ್ರೂರ-ಕಾ-ರ್ಯ-ವೆಂದು
ಕ್ರೂರ-ದೃ-ಷ್ಟಿಯ
ಕ್ರೂರ-ನಾದ
ಕ್ರೂರ-ಮೃ-ಗ-ಗಳು
ಕ್ರೂರ-ರಾ-ಕ್ಷಸ
ಕ್ರೂರ-ವಾದ
ಕ್ರೂರ-ಸ್ವ-ಭಾ-ವದ
ಕ್ರೂರಿ
ಕ್ರೂರಿ-ಗಳು
ಕ್ರೂರಿ-ಯನ್ನು
ಕ್ರೂರಿ-ಯಾ-ಗು-ವುದು
ಕ್ರೋಡೀ-ಕ-ರಿಸಿ
ಕ್ರೋದಾ-ಧಿ-ಗ-ಳಾ-ಗಲಿ
ಕ್ರೋಧ
ಕ್ರೋಧ-ಗಳಿಂದ
ಕ್ರೋಧ-ದಿಂದ
ಕ್ರೋಧ-ದಿಂ-ದಿ-ದ್ದಾಗ
ಕ್ರೋಧ-ರೂ-ಪ-ವಾದ
ಕ್ರೋಧ-ವನ್ನು
ಕ್ರೋಧ-ವ-ಶಾ-ದ-ಹೀಂದ್ರಃ
ಕ್ರೋಧಾದಿ
ಕ್ರೋಷ್ಟು-ವಿನ
ಕ್ರೌಂಚ
ಕ್ರೌಂಚ-ದ್ವೀ-ಪ-ವಿದೆ
ಕ್ರೌಂಚ-ವೆಂಬ
ಕ್ಲೇಶ-ಗಳು
ಕ್ವಚಿ-ದಪಿ
ಕ್ಷಣ
ಕ್ಷಣ-ಕಾಲ
ಕ್ಷಣ-ಕಾ-ಲ-ವಾದ
ಕ್ಷಣ-ಕಾ-ಲ-ವಾ-ದ-ಮೇಲೆ
ಕ್ಷಣ-ಕಾ-ಲ-ವಾ-ದರೂ
ಕ್ಷಣ-ಕಾ-ಲ-ವಿದ್ದು
ಕ್ಷಣಕ್ಕೆ
ಕ್ಷಣ-ಗ-ಳಂತೆ
ಕ್ಷಣ-ದಂತೆ
ಕ್ಷಣ-ದಲ್ಲಿ
ಕ್ಷಣ-ದ-ಲ್ಲಿ-ಯಾ-ಗಲಿ
ಕ್ಷಣ-ದ-ಲ್ಲಿಯೆ
ಕ್ಷಣ-ದ-ಲ್ಲಿಯೇ
ಕ್ಷಣ-ಮಾತ್ರ
ಕ್ಷಣ-ಮಾ-ತ್ರ-ದಲ್ಲಿ
ಕ್ಷಣ-ಮಾ-ತ್ರ-ದ-ಲ್ಲಿಯೆ
ಕ್ಷಣ-ಮಾ-ತ್ರವೂ
ಕ್ಷಣ-ವನ್ನು
ಕ್ಷಣ-ವಾಗಿ
ಕ್ಷಣವೆ
ಕ್ಷಣ-ವೆಂ-ಬಂತೆ
ಕ್ಷಣವೇ
ಕ್ಷಣ-ವೊಂದು
ಕ್ಷಣಾ-ರ್ಧ-ದಲ್ಲಿ
ಕ್ಷಣಿಕ
ಕ್ಷತ್ರಿಯ
ಕ್ಷತ್ರಿ-ಯ-ನಾಗಿ
ಕ್ಷತ್ರಿ-ಯ-ನಾ-ಗಿ-ರಲು
ಕ್ಷತ್ರಿ-ಯ-ನಾದ
ಕ್ಷತ್ರಿ-ಯ-ನಾ-ದರೂ
ಕ್ಷತ್ರಿ-ಯ-ನಾ-ದು-ದ-ರಿಂದ
ಕ್ಷತ್ರಿ-ಯನು
ಕ್ಷತ್ರಿ-ಯನೂ
ಕ್ಷತ್ರಿ-ಯರ
ಕ್ಷತ್ರಿ-ಯ-ರನ್ನು
ಕ್ಷತ್ರಿ-ಯ-ರ-ನ್ನೆಲ್ಲ
ಕ್ಷತ್ರಿ-ಯ-ರಾಗಿ
ಕ್ಷತ್ರಿ-ಯ-ರಿಗೆ
ಕ್ಷತ್ರಿ-ಯರು
ಕ್ಷತ್ರಿ-ಯರೂ
ಕ್ಷತ್ರಿ-ಯರೇ
ಕ್ಷತ್ರಿ-ಯ-ವಂ-ಶ-ವನ್ನೇ
ಕ್ಷತ್ರಿ-ಯಾ-ಧಮ
ಕ್ಷಮಿ-ಸ-ಬೇಕು
ಕ್ಷಮಿ-ಸ-ಬೇಕೆ
ಕ್ಷಮಿ-ಸ-ಲಿಲ್ಲ
ಕ್ಷಮಿ-ಸ-ಲೇ-ಬೇಕು
ಕ್ಷಮಿಸಿ
ಕ್ಷಮಿ-ಸಿ-ದರೂ
ಕ್ಷಮಿ-ಸಿ-ಬಿಡಿ
ಕ್ಷಮಿಸು
ಕ್ಷಮಿ-ಸು-ತ್ತಿದ್ದ
ಕ್ಷಮಿ-ಸು-ತ್ತೇನೆ
ಕ್ಷಮಿ-ಸು-ವಂತೆ
ಕ್ಷಮಿ-ಸು-ವು-ದೇನೋ
ಕ್ಷಮಿ-ಸೆಂದು
ಕ್ಷಮೆ
ಕ್ಷಮೆ-ಇವು
ಕ್ಷಮೆ-ಯನ್ನು
ಕ್ಷಯ
ಕ್ಷಯ-ರೋ-ಗ-ದಿಂದ
ಕ್ಷಯ-ರೋ-ಗಿ-ಯಾಗಿ
ಕ್ಷಯ-ರೋ-ಗಿ-ಯಾ-ಗಿದ್ದ
ಕ್ಷಯಿ-ಸು-ತ್ತವೆ
ಕ್ಷಾತ್ರ
ಕ್ಷಾತ್ರ-ತೇ-ಜಸ್ಸೂ
ಕ್ಷಾಮೆ
ಕ್ಷಾಮೆಯ
ಕ್ಷಿಣೋಷಿ
ಕ್ಷಿತಿ-ಧರ
ಕ್ಷೀಣ-ಸ್ತಮೋ
ಕ್ಷೀಣೇ
ಕ್ಷೀರ-ಸ-ಮು-ದ್ರಕ್ಕೆ
ಕ್ಷೀರ-ಸ-ಮು-ದ್ರ-ದಿಂದ
ಕ್ಷೀರ-ಸ-ಮು-ದ್ರ-ಮ-ಥ-ನ-ದಿಂದ
ಕ್ಷುತ್
ಕ್ಷುದ್ರ
ಕ್ಷುಪಿ-ತ-ಕು-ಚ-ರು-ಜಸ್ತೇ
ಕ್ಷೇತ್ರಕ್ಕೆ
ಕ್ಷೇತ್ರ-ಗಳಲ್ಲಿ
ಕ್ಷೇತ್ರ-ಗಳು
ಕ್ಷೇತ್ರಜ್ಞ
ಕ್ಷೇತ್ರದ
ಕ್ಷೇತ್ರ-ದ-ಲ್ಲಿದ್ದ
ಕ್ಷೇತ್ರ-ದ-ಲ್ಲಿಯೆ
ಕ್ಷೇಮ
ಕ್ಷೇಮ-ಕ್ಕಾಗಿ
ಕ್ಷೇಮ-ಕ್ಕಾ-ಗಿಯೇ
ಕ್ಷೇಮಕ್ಕೆ
ಕ್ಷೇಮ-ವಾ-ಗಿ-ದ್ದಾ-ನೆಯೊ
ಕ್ಷೇಮ-ವಾ-ಗಿ-ರು-ವೆ-ನೆಂದು
ಕ್ಷೇಮವೆ
ಕ್ಷೇಮ-ಸ-ಮಾ-ಚಾರ
ಕ್ಷೇಮ-ಸ-ಮಾ-ಚಾ-ರ-ವನ್ನು
ಕ್ಷೋಭೆ-ಗೊಂ-ಡಿತು
ಕ್ಷೌದ್ರಾ-ಲಾ-ಪಯ
ಖಂಡದ
ಖಂಡ-ದಲ್ಲಿ
ಖಂಡಿ-ತ-ವಾ-ಗಿಯೂ
ಖಂಡಿ-ಸುತ್ತಾ
ಖಗೋಳ
ಖಚಿ-ತ-ವಾದ
ಖಟ್ವಾಂಗ
ಖಟ್ವಾಂ-ಗನ
ಖಟ್ವಾಂ-ಗನು
ಖಟ್ವಾಂ-ಗ-ನೆಂಬ
ಖಡ್ಗ
ಖಡ್ಗ-ದಿಂದ
ಖಡ್ಗ-ಧಾ-ರಿ-ಯಾದ
ಖಡ್ಗ-ಮೃಗ
ಖಡ್ಗ-ಮೃ-ಗ-ದಂತೆ
ಖಡ್ಗವೇ
ಖರ-ದೂ-ಷಣ
ಖಲು
ಖಾಂಡ-ವ-ವನ
ಖಾಂಡ-ವ-ವ-ನ-ವನ್ನು
ಖಾರ-ಗಳು
ಖಾಲಿ-ಯಾಗಿ
ಖೇಽವತು
ಖ್ಯತೆ
ಖ್ಯಾತಿ-ಯನ್ನೂ
ಗ-ಗ-ರುಡ
ಗಂ
ಗಂಗಾ
ಗಂಗಾ-ತೀ-ರಕ್ಕೆ
ಗಂಗಾ-ತೀ-ರದ
ಗಂಗಾ-ತೀ-ರ-ದ-ಲ್ಲಿ-ರುವ
ಗಂಗಾದಿ
ಗಂಗಾ-ದ್ವಾ-ರಕ್ಕೆ
ಗಂಗಾ-ನ-ದಿಗೆ
ಗಂಗಾ-ನ-ದಿ-ಯಲ್ಲಿ
ಗಂಗಾ-ಪ್ರ-ವಾ-ಹ-ದಲ್ಲಿ
ಗಂಗೆ
ಗಂಗೆಗೆ
ಗಂಗೆಯ
ಗಂಗೆ-ಯಂತೆ
ಗಂಗೆ-ಯಲ್ಲಿ
ಗಂಗೆಯು
ಗಂಗೇ-ವೌ-ಘ-ಮು-ದ-ನ್ವತಿ
ಗಂಟನ್ನು
ಗಂಟಲು
ಗಂಟಿ-ಕ್ಕಿತು
ಗಂಟಿ-ಕ್ಕಿ-ದಂತೆ
ಗಂಟಿ-ಕ್ಕಿ-ದುವು
ಗಂಟಿ-ಕ್ಕಿ-ದ್ದುವು
ಗಂಟು
ಗಂಟು-ಕ-ಟ್ಟಿ-ಕೊಂಡು
ಗಂಟು-ಗಳನ್ನು
ಗಂಟು-ಬಿ-ದ್ದವು
ಗಂಟು-ಬಿ-ದ್ದಿ-ರುವ
ಗಂಟು-ಬೀ-ಳು-ವುದೊ
ಗಂಟು-ಹಾ-ಕಿದ
ಗಂಟೆ-ಗಳು
ಗಂಡ
ಗಂಡ-ಹೆಂ-ಡಿ-ರಾ-ಗಿದ್ದ
ಗಂಡಂ-ದಿರ
ಗಂಡಂ-ದಿ-ರಲ್ಲಿ
ಗಂಡಂ-ದಿ-ರಿ-ರುವ
ಗಂಡಂ-ದಿರು
ಗಂಡಂ-ದಿರೇ
ಗಂಡಂ-ದಿ-ರೊ-ಡನೆ
ಗಂಡಕೀ
ಗಂಡ-ಕೀ-ನ-ದಿಯ
ಗಂಡ-ಕೀ-ನ-ದಿ-ಯಲ್ಲಿ
ಗಂಡ-ಗ-ರ್ವ-ದಿಂದ
ಗಂಡ-ಗ-ರ್ವ-ವನ್ನು
ಗಂಡ-ಗ-ರ್ವ-ವೆಲ್ಲ
ಗಂಡನ
ಗಂಡ-ನಂತೆ
ಗಂಡ-ನ-ನ್ನಾಗಿ
ಗಂಡ-ನನ್ನು
ಗಂಡ-ನಲ್ಲಿ
ಗಂಡ-ನಾಗ
ಗಂಡ-ನಾ-ಗು-ವಂತೆ
ಗಂಡ-ನಾದ
ಗಂಡ-ನಾ-ದ-ವನು
ಗಂಡ-ನಿ-ಗಾಗಿ
ಗಂಡ-ನಿಗೂ
ಗಂಡ-ನಿಗೆ
ಗಂಡ-ನಿ-ಲ್ಲದೆ
ಗಂಡನು
ಗಂಡನೂ
ಗಂಡನೇ
ಗಂಡ-ನೊ-ಡನೆ
ಗಂಡ-ಸ-ರಂತೆ
ಗಂಡ-ಸ-ರಿ-ಗಲ್ಲ
ಗಂಡ-ಸ-ರಿ-ಗಿಂತ
ಗಂಡ-ಸ-ರಿ-ಗೆಲ್ಲ
ಗಂಡ-ಸ-ರೆಲ್ಲ
ಗಂಡ-ಸಿ-ಗಿಂತ
ಗಂಡ-ಸಿನ
ಗಂಡಸು
ಗಂಡ-ಹೆಂ-ಡತಿ
ಗಂಡ-ಹೆಂ-ಡಿರು
ಗಂಡಾಂ-ತ-ರ-ವೊಂದು
ಗಂಡಾ-ಗಲಿ
ಗಂಡಾ-ಗ-ಲೆಂದು
ಗಂಡಾಗಿ
ಗಂಡಾ-ಗಿ-ದ್ದಾಗ
ಗಂಡಾ-ದಳು
ಗಂಡಾನೆ
ಗಂಡಿ
ಗಂಡಿ-ಗಿಂತ
ಗಂಡಿಗೆ
ಗಂಡಿನ
ಗಂಡಿ-ನೊ-ಡನೆ
ಗಂಡು
ಗಂಡು-ಕ-ರಡಿ
ಗಂಡು-ಗಳು
ಗಂಡು-ಗೊ-ಡಲಿ
ಗಂಡು-ಗೊ-ಡ-ಲಿ-ಯನ್ನು
ಗಂಡು-ಗೊ-ಡ-ಲಿ-ಯಿಂದ
ಗಂಡು-ತನ
ಗಂಡು-ಪಿ-ಳ್ಳೆಯ
ಗಂಡು-ಮ-ಕ್ಕಳ
ಗಂಡು-ಮ-ಕ್ಕ-ಳನ್ನೂ
ಗಂಡು-ಮ-ಕ್ಕ-ಳಿ-ದ್ದ-ರಷ್ಟೆ
ಗಂಡು-ಮ-ಕ್ಕಳು
ಗಂಡು-ಮ-ಕ್ಕಳೂ
ಗಂಡು-ಮ-ಕ್ಕ-ಳೊ-ಡ-ನೆಯೂ
ಗಂಡು-ಮಗ
ಗಂಡು-ಮ-ಗ-ನನ್ನು
ಗಂಡು-ಮಗು
ಗಂಡು-ಹೆಣ್ಣು
ಗಂಡೂ
ಗಂಡೆಂದು
ಗಂಡೊ
ಗಂತೂ
ಗಂಧ
ಗಂಧದ
ಗಂಧ-ಮಾ-ದನ
ಗಂಧರ್ವ
ಗಂಧ-ರ್ವ-ಗಾ-ನವು
ಗಂಧ-ರ್ವನ
ಗಂಧ-ರ್ವರ
ಗಂಧ-ರ್ವ-ರನ್ನು
ಗಂಧ-ರ್ವ-ರಾಜ
ಗಂಧ-ರ್ವ-ರಾ-ಜನು
ಗಂಧ-ರ್ವ-ರಾ-ಜ-ನೊಬ್ಬ
ಗಂಧ-ರ್ವರು
ಗಂಧ-ರ್ವ-ಲೋ-ಕ-ವೆಂ-ಬಂತೆ
ಗಂಧ-ರ್ವಾದಿ
ಗಂಧ-ರ್ವಾ-ದಿ-ಗ-ಳೆಲ್ಲ
ಗಂಧ-ರ್ವಾ-ದಿ-ಗಳೇ
ಗಂಧ-ವನ್ನು
ಗಂಭಿರ
ಗಂಭೀರ
ಗಂಭೀ-ರ-ನಾ-ಗಿದ್ದ
ಗಂಭೀ-ರನು
ಗಂಭೀ-ರ-ವಾಗಿ
ಗಂಭೀ-ರ-ವಾ-ಣಿ-ಯಿಂದ
ಗಂಭೀ-ರ-ವಾದ
ಗಗ್ಗ-ರ-ಗ-ಳಿಂ-ದಲೂ
ಗಜ
ಗಜ-ರಾ-ಜನ
ಗಜ-ರಾ-ಜ-ನನ್ನು
ಗಜ-ರಾ-ಜನು
ಗಜ-ರಾ-ಜ-ನೆ-ನಿ-ಸಿ-ಕೊಂಡು
ಗಜೇಂ-ದ್ರ-ದರಂ
ಗಜೇಂ-ದ್ರನ
ಗಜೇಂ-ದ್ರನು
ಗಜೇಂ-ದ್ರ-ಮೋಕ್ಷ
ಗಜೇಂ-ದ್ರ-ಮೋ-ಕ್ಷ-ವನ್ನು
ಗಟ್ಟಿ
ಗಟ್ಟಿ-ಮಾ-ಡಿ-ಕೊಂಡು
ಗಟ್ಟಿ-ಯಂತೆ
ಗಟ್ಟಿ-ಯಾಗಿ
ಗಟ್ಟಿ-ಯಾ-ಗು-ತ್ತದೆ
ಗಟ್ಟಿ-ಯಾದ
ಗಟ್ಟಿಯೂ
ಗಡ-ಗಡ
ಗಡಾ-ರಿ-ಯಂ-ತಿ-ರುವ
ಗಡಿಗೆ
ಗಡಿ-ಗೆ-ಯನ್ನು
ಗಡಿ-ಗೆ-ಯ-ಲ್ಲಿನ
ಗಡು-ವನ್ನು
ಗಡೆಯೂ
ಗಡ್ಡ
ಗಡ್ಡ-ಮೀಸೆ
ಗಡ್ಡ-ಮೀ-ಸೆ-ಗಳನ್ನು
ಗಡ್ಡೆ
ಗಣನೆ
ಗಣ-ಪ-ತಿ-ಗಳ
ಗಣ-ಪ-ತಿಯ
ಗಣಾತ್
ಗಣಾ-ತ್ಪ್ರ-ಮಾ-ದಾತ್
ಗಣಿ-ಯಂ-ತಿ-ರುವ
ಗಣಿ-ಯಾದ
ಗಣೇಭ್ಯೋ
ಗಣ್
ಗಣ್ಣ-ರಾಗಿ
ಗತ-ಶೋ-ಕ-ಮ-ಶೋ-ಕ-ಕರಂ
ಗತಾ
ಗತಿ
ಗತಿ-ಯನ್ನು
ಗತಿ-ಯನ್ನೂ
ಗತಿ-ಯನ್ನೆ
ಗತಿ-ಯಾ-ಯಿತು
ಗತಿ-ಯಿಂದ
ಗತಿ-ಯಿ-ರ-ಲಿಲ್ಲ
ಗತಿ-ಯಿ-ಲ್ಲದೆ
ಗತಿ-ಯೆಂದು
ಗತಿ-ಯೇ-ನಾ-ಗ-ಬೇ-ಕಾ-ಗಿತ್ತು
ಗತಿ-ಯೇ-ನಾ-ಗಿ-ದೆಯೊ
ಗತಿ-ಯೇ-ನಾ-ಗು-ತ್ತದೊ
ಗತಿ-ಯೇ-ನಾ-ಯಿತು
ಗತಿ-ಯೇನು
ಗತಿ-ಯೇ-ನು-ಎಂದು
ಗತಿ-ಯೇ-ನೆಂದು
ಗತೋ
ಗತ್ಯಂ-ತ-ರ-ವಿಲ್ಲ
ಗತ್ಯಂ-ತ-ರ-ವಿ-ಲ್ಲದೆ
ಗದ
ಗದಯಾ
ಗದ-ರಿಸಿ
ಗದ-ರಿ-ಸಿತು
ಗದ-ರಿ-ಸಿದ
ಗದ-ರಿ-ಸಿ-ದನು
ಗದ-ರಿ-ಸಿ-ದರು
ಗದ-ರಿ-ಸಿ-ದಳು
ಗದ-ರಿ-ಸುತ್ತಾ
ಗದ-ರು-ತ್ತಲೆ
ಗದ-ಸಾ-ರಣ
ಗದಾ
ಗದಾ-ಘಾತ
ಗದಾ-ದಂ-ಡ-ದಿಂದ
ಗದಾ-ದಂ-ಡ-ವನ್ನು
ಗದಾ-ಧಾರಿ
ಗದಾ-ಧಾ-ರಿ-ಯಾಗಿ
ಗದಾ-ಪಾ-ಣಿ-ಯಾಗಿ
ಗದಾ-ಪ್ರ-ಹಾರ
ಗದಾ-ಯುದ್ಧ
ಗದಾ-ಯು-ದ್ಧ-ದಲ್ಲಿ
ಗದಾ-ಯು-ದ್ಧ-ವನ್ನು
ಗದಾ-ಯು-ಧವೇ
ಗದಿ-ತಾನಿ
ಗದೆ
ಗದೆ-ಗಳ
ಗದೆಯ
ಗದೆ-ಯನ್ನು
ಗದೆ-ಯಿಂದ
ಗದೆಯೇ
ಗದೆ-ಯೊ-ಡನೆ
ಗದೇ-ಽಶನಿ
ಗದ್ಗದ
ಗದ್ಗ-ದ-ಕಂ-ಠ-ದಿಂದ
ಗದ್ಗ-ದ-ನಾಗಿ
ಗದ್ಗ-ದ-ವಾಗಿ
ಗದ್ಗ-ದ-ವಾ-ಯಿತು
ಗದ್ಗ-ದ-ಸ್ವ-ರ-ದಿಂದ
ಗದ್ದ-ಲ-ವಾ-ಗು-ತ್ತಿತ್ತು
ಗದ್ದು-ಗೆ-ಯ-ನ್ನೇ-ರಿ-ದನು
ಗದ್ದೆ
ಗದ್ದೆಯ
ಗಮನ
ಗಮ-ನದ
ಗಮ-ನ-ವನ್ನು
ಗಮ-ನ-ವಿಲ್ಲ
ಗಮ-ನವೆ
ಗಮ-ನವೇ
ಗಮ-ನಿ-ಸದ
ಗಮ-ನಿ-ಸ-ದಷ್ಟು
ಗಮ-ನಿ-ಸದೆ
ಗಮ-ನಿ-ಸ-ಲಿಲ್ಲ
ಗಮ-ನಿಸಿ
ಗಮ-ನಿ-ಸು-ತ್ತಿ-ರ-ಲಿಲ್ಲ
ಗಮ-ನಿ-ಸು-ತ್ತಿ-ಲ್ಲ-ವಲ್ಲಾ
ಗಮ-ನಿ-ಸು-ವು-ದಿಲ್ಲ
ಗಮಾರ
ಗಯ
ಗಯೆ
ಗಯ್ಯಾಳಿ
ಗರ
ಗರ-ಗರ
ಗರ-ಗ-ರನೆ
ಗರ-ತಿ-ಯರು
ಗರಿ
ಗರಿ-ಕೆ-ಯಿಂದ
ಗರಿ-ಗೆ-ದರಿ
ಗರಿ-ಮೂ-ಡಿ-ದಂ-ತಾ-ಯಿತು
ಗರಿ-ಮೆ-ಗ-ಳಿಗೆ
ಗರಿ-ಮೆ-ಗ-ಳೇ-ನೆಂ-ಬುದು
ಗರಿ-ಸ-ಹಿ-ತ-ವಾದ
ಗರು
ಗರು-ಕೆಯೆ
ಗರುಡ
ಗರು-ಡನ
ಗರು-ಡ-ನಂತೆ
ಗರು-ಡ-ನನ್ನು
ಗರು-ಡ-ನ-ನ್ನೇನು
ಗರು-ಡ-ನ-ನ್ನೇರಿ
ಗರು-ಡ-ನಿಂದ
ಗರು-ಡ-ನಿಗೆ
ಗರು-ಡನು
ಗರು-ಡನೂ
ಗರು-ಡ-ನೊಂ-ದಿಗೆ
ಗರು-ಡ-ಪು-ರಾಣ
ಗರು-ಡ-ವ-ನ್ನೇರಿ
ಗರು-ಡ-ವಾ-ಹ-ನ-ನಾಗಿ
ಗರು-ಡ-ಶೇ-ಷರು
ಗರುಡೋ
ಗರೆ-ದನು
ಗರೆ-ಯು-ತ್ತಾನೆ
ಗರ್ಗ-ನಿಗೆ
ಗರ್ಗನು
ಗರ್ಗ-ಮು-ನಿ-ಗಳು
ಗರ್ಗರ
ಗರ್ಗ-ರ-ಮು-ನಿ-ಯಲ್ಲಿ
ಗರ್ಜ-ನೆ-ಯಿಂದ
ಗರ್ಜಿ
ಗರ್ಜಿಸಿ
ಗರ್ಜಿ-ಸಿದ
ಗರ್ಜಿ-ಸಿ-ದನು
ಗರ್ಜಿ-ಸಿ-ದರು
ಗರ್ಜಿ-ಸಿ-ದಳು
ಗರ್ಜಿ-ಸಿ-ದುದು
ಗರ್ಜಿ-ಸುತ್ತಾ
ಗರ್ದ-ಭಾ-ಸುರ
ಗರ್ಭ
ಗರ್ಭ-ಕೋ-ಶದ
ಗರ್ಭ-ಕೋ-ಶ-ದಲ್ಲಿ
ಗರ್ಭ-ಕೋ-ಶ-ವನ್ನು
ಗರ್ಭಕ್ಕೆ
ಗರ್ಭ-ಗಳು
ಗರ್ಭ-ಚಿ-ಹ್ನೆ-ಗಳು
ಗರ್ಭ-ದಲ್ಲಿ
ಗರ್ಭ-ದ-ಲ್ಲಿದ್ದ
ಗರ್ಭ-ದ-ಲ್ಲಿ-ರು-ವಾಗ
ಗರ್ಭ-ದಿಂದ
ಗರ್ಭ-ಧ-ರಿ-ಸಿದ
ಗರ್ಭ-ನಾ-ಶ-ವಾ-ಗ-ಲಿಲ್ಲ
ಗರ್ಭ-ವ-ತಿ-ಯಾಗಿ
ಗರ್ಭ-ವ-ತಿ-ಯಾ-ಗಿ-ದ್ದಳು
ಗರ್ಭ-ವನ್ನು
ಗರ್ಭ-ವಾಸ
ಗರ್ಭ-ಸ್ರಾ-ವ-ವಾಗಿ
ಗರ್ಭ-ಸ್ರಾ-ವ-ವಾ-ಯಿ-ತೆಂದು
ಗರ್ಭಾಃ
ಗರ್ಭಿಣಿ
ಗರ್ಭಿ-ಣಿ-ಯರ
ಗರ್ಭಿ-ಣಿ-ಯಾಗಿ
ಗರ್ಭಿ-ಣಿ-ಯಾ-ಗಿದ್ದ
ಗರ್ಭಿ-ಣಿ-ಯಾ-ಗಿ-ದ್ದಳು
ಗರ್ಭಿ-ಣಿ-ಯಾ-ಗಿ-ದ್ದಾಳೆ
ಗರ್ಭಿ-ಣಿ-ಯಾದ
ಗರ್ಭಿ-ಣಿ-ಯಾ-ದಳು
ಗರ್ವ
ಗರ್ವದ
ಗರ್ವ-ದಿಂದ
ಗರ್ವ-ವನ್ನು
ಗರ್ವವೂ
ಗರ್ವವೇ
ಗಲಿ
ಗಲೀ-ಜ-ನ್ನೆಲ್ಲ
ಗಲೂ
ಗಲ್ಲ
ಗಲ್ಲ-ವನ್ನು
ಗಳ
ಗಳಂತೆ
ಗಳ-ಗಳ
ಗಳತ್ತ
ಗಳ-ನ್ನಾ-ಚ-ರಿ-ಸಿ-ದ-ವರು
ಗಳ-ನ್ನಾಡಿ
ಗಳ-ನ್ನಾ-ಡಿ-ದರೂ
ಗಳ-ನ್ನಾ-ದರೂ
ಗಳನ್ನು
ಗಳನ್ನೂ
ಗಳನ್ನೆ
ಗಳ-ನ್ನೆಲ್ಲ
ಗಳ-ನ್ನೆಲ್ಲಾ
ಗಳ-ಬಿ-ನ್ನೆಲ್ಲ
ಗಳಲ್ಲ
ಗಳಲ್ಲಿ
ಗಳ-ಲ್ಲಿಯೇ
ಗಳ-ಲ್ಲೆಲ್ಲ
ಗಳ-ವ-ರೆಗೆ
ಗಳಾ-ಗಲಿ
ಗಳಾ-ಗ-ಲೆಂದು
ಗಳಾಗಿ
ಗಳಾ-ಗಿದ್ದ
ಗಳಾ-ಗಿ-ರುವ
ಗಳಾ-ಗಿ-ರು-ವಷ್ಟು
ಗಳಾ-ಗು-ತ್ತ-ವಂತೆ
ಗಳಾದ
ಗಳಾ-ದುವು
ಗಳಿಂದ
ಗಳಿಂ-ದಲೆ
ಗಳಿಗೂ
ಗಳಿಗೆ
ಗಳಿ-ಗೆ-ಗ-ಳಂತೆ
ಗಳಿ-ಗೆ-ಗ-ಳ-ಲ್ಲಿಯೇ
ಗಳಿ-ಗೆ-ಗಳು
ಗಳಿ-ಗೆಲ್ಲ
ಗಳಿಲ್ಲ
ಗಳಿಸ
ಗಳಿ-ಸ-ಬೇ-ಕೆಂಬ
ಗಳಿ-ಸಿ-ಕೊಂಡ
ಗಳಿ-ಸಿ-ಕೊಂ-ಡನು
ಗಳಿ-ಸಿ-ಕೊ-ಟ್ಟಿತು
ಗಳಿ-ಸಿ-ಕೊ-ಳ್ಳ-ಬೇ-ಕೆಂದು
ಗಳಿ-ಸಿ-ಕೊ-ಳ್ಳು-ತ್ತಿ-ರುವೆ
ಗಳಿ-ಸಿ-ಕೊ-ಳ್ಳು-ವಂತೆ
ಗಳಿ-ಸಿಟ್ಟ
ಗಳಿ-ಸು-ತ್ತೇ-ನೆಂದು
ಗಳಿ-ಸು-ವನು
ಗಳಿ-ಸು-ವು-ದ-ಕ್ಕಾಗಿ
ಗಳಿ-ಸು-ವು-ದಕ್ಕೆ
ಗಳಿ-ಸು-ವು-ದೇನು
ಗಳು
ಗಳೂ
ಗಳೆ
ಗಳೆಂ-ದರೆ
ಗಳೆಂಬ
ಗಳೆ-ರ-ಡನ್ನೂ
ಗಳೆ-ರ-ಡರ
ಗಳೆ-ರಡೂ
ಗಳೆಲ್ಲ
ಗಳೆ-ಲ್ಲ-ರಿಗೂ
ಗಳೆ-ಲ್ಲರೂ
ಗಳೆ-ಲ್ಲವೂ
ಗಳೇ
ಗಳೇಕೆ
ಗಳೊ-ಡನೆ
ಗವಾ-ಕ್ಷೆ-ಯಿಂದ
ಗವಿ
ಗವಿ-ಗಳಲ್ಲಿ
ಗವಿ-ಗಳು
ಗವಿಯ
ಗವಿ-ಯಂ-ತಿತ್ತು
ಗವಿ-ಯನ್ನು
ಗವಿ-ಯೆಂದು
ಗಹ-ಗ-ಹಿಸಿ
ಗಾಂಡೀ-ವ-ಧ-ನು-ಸ್ಸಿ-ನಲ್ಲಿ
ಗಾಂಡೀ-ವ-ವೆಂಬ
ಗಾಂಧ-ರ್ವ-ವಿ-ವಾ-ಹ-ದಿಂದ
ಗಾಂಧಾರಿ
ಗಾಂಧಾ-ರಿ-ಯರು
ಗಾಂಧಾ-ರಿಯೂ
ಗಾಗಿ
ಗಾಗಿಯೇ
ಗಾಗುವ
ಗಾಗು-ವು-ದಿಲ್ಲ
ಗಾಡಿ
ಗಾಡಿ-ಗಳನ್ನು
ಗಾಡಿ-ಗಳಲ್ಲಿ
ಗಾಡಿ-ಗಳು
ಗಾಡಿಗೆ
ಗಾಡಿಯ
ಗಾಡಿ-ಯನ್ನು
ಗಾಢ-ನಿದ್ರೆ
ಗಾಢ-ನಿ-ದ್ರೆ-ಯಲ್ಲಿ
ಗಾಢ-ವಾಗಿ
ಗಾಢಾ-ಲಿಂ-ಗನ
ಗಾತ್ರಕ್ಕೆ
ಗಾತ್ರದ
ಗಾತ್ರ-ವಂತ
ಗಾತ್ರ-ವನ್ನು
ಗಾದರೂ
ಗಾದೆ
ಗಾದೆಗೆ
ಗಾದೆಯ
ಗಾದೆ-ಯಂತೆ
ಗಾದೆಯೆ
ಗಾಧಿಯು
ಗಾಧಿ-ರಾಜ
ಗಾಧಿ-ರಾ-ಜ-ನಿಗೆ
ಗಾಧಿ-ರಾ-ಜನು
ಗಾನ
ಗಾನಈ
ಗಾನಕ್ಕೆ
ಗಾನ-ಗ-ಳಿಗೆ
ಗಾನದ
ಗಾನ-ದಂ-ತಿದೆ
ಗಾನ-ದಿಂದ
ಗಾನ-ಮಾಡಿ
ಗಾನ-ಮಾ-ಡಿ-ದರು
ಗಾನ-ಮಾ-ಡಿ-ದವು
ಗಾನ-ಮಾಡು
ಗಾನ-ಮಾ-ಡುತ್ತ
ಗಾನ-ಮಾ-ಡುತ್ತಾ
ಗಾನ-ಮಾ-ಡು-ತ್ತಿ-ರಲು
ಗಾನ-ವನ್ನು
ಗಾನ-ಸು-ಧೆ-ಯನ್ನು
ಗಾನಾ-ಮೃ-ತದ
ಗಾಬರಿ
ಗಾಬ-ರಿ-ಯಾಗಿ
ಗಾಬ-ರಿ-ಯಾ-ದರು
ಗಾಬ-ರಿ-ಯಾ-ದಳು
ಗಾಬ-ರಿ-ಯಾ-ಯಿತು
ಗಾಬ-ರಿ-ಯಿಂದ
ಗಾಯ
ಗಾಯ-ಗ-ಳೆಲ್ಲ
ಗಾಯ-ತ್ರಿಯ
ಗಾಯ-ತ್ರಿ-ಯನ್ನು
ಗಾಯ-ತ್ರೀ-ಮಂತ್ರ
ಗಾಯ-ದಿಂದ
ಗಾಯನ
ಗಾಯ-ನ-ದಿಂದ
ಗಾಯ-ನ-ದೊ-ಡನೆ
ಗಾಯ-ವನ್ನು
ಗಾಯಸಿ
ಗಾರ್ದಭ
ಗಾಲ
ಗಾಲಕ್ಕೆ
ಗಾಲ-ನ್ನಿಟ್ಟು
ಗಾಲಿ
ಗಾಲಿ-ಗಳ
ಗಾಲಿ-ಯಂ-ತಿ-ರುವ
ಗಾಳಕ್ಕೆ
ಗಾಳ-ದಂತೆ
ಗಾಳಿ
ಗಾಳಿ-ಗಳು
ಗಾಳಿಗೆ
ಗಾಳಿಯ
ಗಾಳಿ-ಯನ್ನು
ಗಾಳಿ-ಯ-ನ್ನೇರಿ
ಗಾಳಿ-ಯಲ್ಲಿ
ಗಾಳಿ-ಯಾ-ದರೆ
ಗಾಳಿ-ಯಿಂದ
ಗಾಳಿ-ಯಿ-ಲ್ಲದ
ಗಾಳಿಯೂ
ಗಾಳಿ-ಯೊ-ಡನೆ
ಗಾವು-ದ-ಗ-ಳಷ್ಟು
ಗಿಂತ
ಗಿಂತಲೂ
ಗಿಡ
ಗಿಡ-ಗ-ಗ-ಳಂತೆ
ಗಿಡ-ಗ-ನಂತೆ
ಗಿಡ-ಗಳ
ಗಿಡ-ಗಳಲ್ಲಿ
ಗಿಡ-ಗಳು
ಗಿಡ-ಗಳೂ
ಗಿಡ-ದಲ್ಲಿ
ಗಿಡ-ದಿಂದ
ಗಿಡ-ಬ-ಳ್ಳಿ-ಗಳ
ಗಿಡ-ಬ-ಳ್ಳಿ-ಗಳು
ಗಿಡ-ಮರ
ಗಿಡ-ಮ-ರ-ಗಳ
ಗಿಡ-ಮ-ರ-ಗಳನ್ನೂ
ಗಿಡ-ಮ-ರ-ಗಳನ್ನೆಲ್ಲ
ಗಿಡ-ಮ-ರ-ಗಳು
ಗಿಡ-ಮ-ರ-ಗ-ಳೊಂದೂ
ಗಿಡ-ಮ-ರ-ಬ-ಳ್ಳಿ-ಗ-ಳೆಲ್ಲ
ಗಿಡ-ವಿ-ರು-ವಾಗ
ಗಿಡು-ಗ-ನಂತೆ
ಗಿಣ್ಣು-ಗ-ಳ-ಲ್ಲಿಯೂ
ಗಿತು
ಗಿತ್ತು
ಗಿದ
ಗಿದೆ
ಗಿದ್ದ
ಗಿದ್ದಾ-ಗಲೆ
ಗಿರಾ
ಗಿರಿ
ಗಿರಿ-ಗಿರಿ
ಗಿರಿಜೆ
ಗಿರಿ-ಜೆ-ಯೊ-ಡನೆ
ಗಿರಿ-ಯನ್ನು
ಗಿರಿ-ವ್ರ-ಜಕ್ಕೆ
ಗಿರಿ-ವ್ರ-ಜ-ವೆಂಬ
ಗಿರುವ
ಗಿಲ್ಲ
ಗಿಳಿ
ಗಿಳಿಗೆ
ಗಿಳಿ-ದುದು
ಗಿಸಿ-ದನು
ಗೀಗ
ಗೀತ-ವನ್ನು
ಗೀತಾ
ಗೀತಾ-ಚಾ-ರ್ಯ-ನಾ-ಗಿದ್ದು
ಗೀತಾ-ಚಾ-ರ್ಯ-ನಾ-ಗಿ-ರು-ವುದು
ಗೀತಾ-ಬೋ-ಧನೆ
ಗೀತಾ-ಸಂ-ದೇಶ
ಗೀತೆ
ಗೀತೆ-ಗಳನ್ನು
ಗೀತೆ-ಯನ್ನು
ಗೀತೆ-ಯಲ್ಲಿ
ಗೀತೆಯು
ಗೀತೆ-ಯೆಂ-ದೊ-ಡನೆ
ಗೀತೋ-ಪ-ದೇ-ಶ-ವನ್ನು
ಗೀಯ-ತಾಂ
ಗೀರು-ಪ-ಲ-ಕ್ಷ್ಯ-ಸೇನಃ
ಗೀಳು
ಗೀಸ್ತ್ರ
ಗುಂಗು-ರಾ-ಗಿ-ರುವ
ಗುಂಗು-ರಾದ
ಗುಂಗುರು
ಗುಂಡಾಗಿ
ಗುಂಡಾ-ಗು-ತ್ತದೆ
ಗುಂಡಿ
ಗುಂಡಿಗೆ
ಗುಂಡಿ-ನಂ-ತಿ-ರುವ
ಗುಂಡು-ಕ-ಲ್ಲಿ-ನಿಂದ
ಗುಂಡು-ಗಳನ್ನು
ಗುಂಪನ್ನು
ಗುಂಪಿಗೆ
ಗುಂಪಿನ
ಗುಂಪಿ-ನ-ಲ್ಲಂತೂ
ಗುಂಪಿ-ನಲ್ಲಿ
ಗುಂಪು
ಗುಂಪು-ಗ-ಳಾಗಿ
ಗುಂಪು-ಗುಂ-ಪಾಗಿ
ಗುಂಪೆಲ್ಲ
ಗುಂಪೇ
ಗುಗ್ಗು-ಗಳು
ಗುಜು-ಗುಜು
ಗುಟು-ರು-ಹಾ-ಕುತ್ತಾ
ಗುಟ್ಟಾಗಿ
ಗುಟ್ಟಾ-ಗಿ-ರಲಿ
ಗುಟ್ಟು
ಗುಟ್ಟೆ-ಲ್ಲವೂ
ಗುಡಾ-ರ-ಗ-ಳಿ-ಗಾಗಿ
ಗುಡಿ
ಗುಡಿ-ಗಳು
ಗುಡಿ-ಗಿ-ನಂ-ತಹ
ಗುಡಿಗೆ
ಗುಡಿಯ
ಗುಡಿ-ಯನ್ನು
ಗುಡಿ-ಯನ್ನೊ
ಗುಡಿ-ಸ-ಲಿದ್ದ
ಗುಡು-ಕ-ರಂತೆ
ಗುಡು-ಗಾ-ಡಿಸಿ
ಗುಡುಗಿ
ಗುಡು-ಗಿದ
ಗುಡು-ಗಿ-ದನು
ಗುಡು-ಗಿ-ದ-ನು-ಎಲಾ
ಗುಡು-ಗಿ-ದಳು
ಗುಡು-ಗಿನ
ಗುಡು-ಗಿ-ನಂ-ತಹ
ಗುಡುಗು
ಗುಣ
ಗುಣ-ಕ-ರ್ಮಕ್ಕೆ
ಗುಣ-ಕ-ರ್ಮ-ಗ-ಳಿಗೆ
ಗುಣ-ಕೀ-ರ್ತನೆ
ಗುಣ-ಕೀ-ರ್ತ-ನೆ-ಯನ್ನು
ಗುಣಕ್ಕೆ
ಗುಣ-ಗಳ
ಗುಣ-ಗಳನ್ನು
ಗುಣ-ಗಳನ್ನೂ
ಗುಣ-ಗಳಲ್ಲಿ
ಗುಣ-ಗ-ಳಾದ
ಗುಣ-ಗಳಿಂದ
ಗುಣ-ಗ-ಳಿಗೆ
ಗುಣ-ಗಳು
ಗುಣ-ಗ-ಳೆಂಬ
ಗುಣ-ಗ-ಳೆಲ್ಲ
ಗುಣ-ಗಾನ
ಗುಣ-ಗಾ-ನ-ಗಳನ್ನು
ಗುಣ-ಗಾ-ನ-ದಿಂದ
ಗುಣ-ಗಾ-ನ-ಮಾಡು
ಗುಣ-ದಲ್ಲಿ
ಗುಣ-ದ-ಲ್ಲಿಯೂ
ಗುಣ-ದೋ-ಷ-ಗ-ಳಿಗೆ
ಗುಣ-ಯುಕ್ತ
ಗುಣ-ರೂಪ
ಗುಣ-ರೂ-ಪ-ಗಳ
ಗುಣ-ರೂ-ಪ-ಗಳನ್ನು
ಗುಣ-ರೂ-ಪದ
ಗುಣ-ವಂ-ತ-ರಾದ
ಗುಣ-ವನ್ನು
ಗುಣ-ವಾದ
ಗುಣ-ವಿ-ಕಾ-ರ-ಗ-ಳಿಗೆ
ಗುಣ-ವೃ-ತ್ತಿ-ಗಳಿಂದ
ಗುಣ-ಶೀ-ಲ-ಗಳನ್ನು
ಗುಣ-ಹೊಂ-ದು-ವು-ದಿ-ಲ್ಲವೆ
ಗುಣಾ-ಧೀನ
ಗುಣಾ-ಮೃ-ತದ
ಗುಣಾ-ಶ್ರಯ
ಗುತ್ತಾ
ಗುತ್ತಿತ್ತು
ಗುದ್ದಿ
ಗುದ್ದಿದ
ಗುದ್ದಿ-ದನು
ಗುದ್ದಿ-ದ-ನೆಂ-ದರೆ
ಗುದ್ದಿ-ದ-ವನೆ
ಗುದ್ದಿ-ನಿಂದ
ಗುದ್ದು-ತ್ತಲೆ
ಗುದ್ದು-ವುದು
ಗುಪ್ತ
ಗುಪ್ತ-ಬೋಧಃ
ಗುಪ್ತ-ರಾ-ಜರ
ಗುಪ್ತ-ರಾ-ಜರು
ಗುರಾ-ಣಿ-ಗಳ
ಗುರಿ
ಗುರಿ-ತಾ-ಗುವ
ಗುರಿ-ಯನ್ನು
ಗುರಿ-ಯಾ-ಗಲಿ
ಗುರಿ-ಯಾ-ಗು-ತ್ತಾನೆ
ಗುರಿ-ಯಾ-ಗು-ತ್ತಾಳೆ
ಗುರಿ-ಯಾ-ಗು-ವು-ದ-ಲ್ಲದೆ
ಗುರಿ-ಯಾದ
ಗುರಿ-ಯಾ-ಯಿತು
ಗುರಿ-ಯಿಂದ
ಗುರಿ-ಯಿಟ್ಟ
ಗುರಿ-ಯಿ-ಟ್ಟನು
ಗುರಿ-ಯಿಟ್ಟು
ಗುರಿ-ಯಿ-ಡು-ವುದನ್ನು
ಗುರಿಯೂ
ಗುರು
ಗುರು
ಗುರು-ಕ-ರು-ಣೆ-ಯಿಂದ
ಗುರು-ಕು-ಲ-ದ-ಲ್ಲಿ-ದ್ದಾಗ
ಗುರು-ಕು-ಲ-ದ-ಲ್ಲಿಯೇ
ಗುರು-ಕು-ಲ-ದಿಂದ
ಗುರು-ಕು-ಲ-ವಾಸ
ಗುರು-ಗಳ
ಗುರು-ಗ-ಳಂ-ತ-ಹ-ವರು
ಗುರು-ಗಳನ್ನು
ಗುರು-ಗಳಲ್ಲಿ
ಗುರು-ಗ-ಳಾ-ಗಿ-ದ್ದರು
ಗುರು-ಗ-ಳಾದ
ಗುರು-ಗ-ಳಾರು
ಗುರು-ಗಳಿಂದ
ಗುರು-ಗ-ಳಿಗೆ
ಗುರು-ಗ-ಳಿ-ಬ್ಬರು
ಗುರು-ಗ-ಳಿ-ಬ್ಬರೂ
ಗುರು-ಗಳು
ಗುರು-ಗಳೆ
ಗುರು-ಜ-ನರ
ಗುರು-ತನ್ನು
ಗುರು-ತನ್ನೇ
ಗುರು-ತಿದೆ
ಗುರು-ತಿ-ಸ-ಬ-ಹು-ದಾ-ಗಿತ್ತು
ಗುರುತು
ಗುರು-ತು-ಗ-ಳಾ-ದ್ದ-ರಿಂದ
ಗುರು-ತು-ಗಳಿಂದ
ಗುರು-ತು-ಗಳು
ಗುರುತೂ
ಗುರುತೇ
ಗುರು-ತೊಂದು
ಗುರು-ದ-ಕ್ಷಿಣೆ
ಗುರು-ದ-ಕ್ಷಿ-ಣೆ-ಯನ್ನು
ಗುರು-ದ-ಕ್ಷಿ-ಣೆ-ಯೇ-ನೆಂದು
ಗುರು-ದೂ-ಷ-ಣೆಯೇ
ಗುರು-ದೇವ
ಗುರು-ದೇ-ವರ
ಗುರು-ಪುತ್ರ
ಗುರು-ಪು-ತ್ರ-ನನ್ನು
ಗುರು-ಪು-ತ್ರ-ನಿ-ರ-ಲಿಲ್ಲ
ಗುರು-ಭ-ಕ್ತಿ-ಯಿಂದ
ಗುರು-ಮು-ಖ-ದಿಂದ
ಗುರು-ವಂ-ದ-ನೆ-ಗೆಂದು
ಗುರು-ವನ್ನು
ಗುರು-ವನ್ನೂ
ಗುರು-ವಾ-ಗಿ-ರಲಿ
ಗುರು-ವಾ-ಗು-ತ್ತದೆ
ಗುರು-ವಾದ
ಗುರು-ವಾ-ಯಿತು
ಗುರು-ವಾರು
ಗುರು-ವಿಗೆ
ಗುರು-ವಿ-ಗೊ-ಪ್ಪಿಸಿ
ಗುರು-ವಿನ
ಗುರು-ವಿ-ನಂತೆ
ಗುರು-ವಿ-ನಿಂದ
ಗುರುವು
ಗುರುವೂ
ಗುರು-ವೆಂದರೆ
ಗುರುವೇ
ಗುರು-ಸೇ-ವೆ-ಯನ್ನು
ಗುರು-ಸ್ವ-ರೂ-ಪ-ವನ್ನು
ಗುರು-ಹತ್ಯೆ
ಗುರು-ಹಿ-ರಿ-ಯರ
ಗುರು-ಹಿ-ರಿ-ಯ-ರನ್ನೂ
ಗುರು-ಹಿ-ರಿ-ಯ-ರೊ-ಡ-ನೆಯೂ
ಗುಲ್ಮಾದಿ
ಗುಳ್ಳೆ-ಗಳು
ಗುವ
ಗುಸು-ಗು-ಟ್ಟು-ತ್ತಿ-ದ್ದರು
ಗುಹೆ
ಗುಹೆ-ಗಳನ್ನು
ಗುಹೆ-ಗಳಲ್ಲಿ
ಗುಹೆ-ಗೇಕೆ
ಗುಹೆಯ
ಗುಹೆ-ಯಂ-ತಿ-ರುವ
ಗುಹೆ-ಯನ್ನು
ಗುಹೆ-ಯಲ್ಲಿ
ಗುಹೆ-ಯ-ಲ್ಲಿದ್ದ
ಗುಹೆ-ಯಿಂದ
ಗುಹೆ-ಯೊ-ಳಗೆ
ಗೂಗೆ-ಗಳು
ಗೂಟಕ್ಕೆ
ಗೂಡನ್ನು
ಗೂಡಿಗೆ
ಗೂಡಿನ
ಗೂಡು
ಗೂಢ-ಕಥೆ
ಗೂಢಾ-ರ್ಥ-ವನ್ನು
ಗೂದಲು
ಗೂನಿ-ಯಾ-ಗಿ-ದ್ದಳು
ಗೂನಿ-ಯಿಂದ
ಗೂನು
ಗೂಳಿ
ಗೂಳಿ-ಗಳ
ಗೂಳಿ-ಗಳನ್ನು
ಗೂಳಿ-ಗ-ಳೊ-ಡನೆ
ಗೂಳಿಯ
ಗೂಳಿ-ಯಂತೆ
ಗೂಳಿ-ಯಾಗಿ
ಗೂಳಿ-ಯಾ-ಗಿ-ರುವ
ಗೃಣಂತಿ
ಗೃಣೀತೇ
ಗೃಧ್ರ
ಗೃಧ್ರಾಣಂ
ಗೃಹ
ಗೃಹ-ಕು-ಟುಂಬಂ
ಗೃಹ-ಸ್ಥ-ಧ-ರ್ಮ-ದಲ್ಲಿ
ಗೃಹ-ಸ್ಥ-ಧ-ರ್ಮ-ವನ್ನು
ಗೃಹ-ಸ್ಥ-ನಾಗಿ
ಗೃಹಸ್ಥಾ
ಗೃಹ-ಸ್ಥಾ-ಶ್ರಮ
ಗೃಹ-ಸ್ಥಾ-ಶ್ರ-ಮ-ವನ್ನೇ
ಗೃಹ-ಸ್ಥಾ-ಶ್ರ-ಮವು
ಗೃಹ-ಸ್ಥಾ-ಶ್ರ-ಮ-ವೆಂ-ಬುದು
ಗೃಹಿ-ಣಿ-ಯರ
ಗೃಹೀತ
ಗೃಹೇ-ಭ್ಯೋ-ಽಭೂ-ತ್ಕೇ-ತುಭ್ಯೋ
ಗೆಂದು
ಗೆಜ್ಜೆ-ಗಳನ್ನು
ಗೆಡೆ-ವ-ಕ್ಕಿ-ಗ-ಳಂತೆ
ಗೆಡ್ಡೆ-ಗೆ-ಣ-ಸು-ಗಳ
ಗೆದ್ದ
ಗೆದ್ದ-ರೇನು
ಗೆದ್ದ-ವನು
ಗೆದ್ದ-ವ-ರನ್ನು
ಗೆದ್ದು
ಗೆದ್ದುದ
ಗೆದ್ದು-ದಾಗಿ
ಗೆದ್ದುದು
ಗೆದ್ದೆ
ಗೆಲವು
ಗೆಲ-ಸ-ಕ್ಕೆಂದು
ಗೆಲುವು
ಗೆಲ್ಲ
ಗೆಲ್ಲದ
ಗೆಲ್ಲ-ಬ-ಲ್ಲ-ರೆ-ಎಂದು
ಗೆಲ್ಲ-ಬ-ಲ್ಲ-ವ-ರಾರು
ಗೆಲ್ಲ-ಬೇಕು
ಗೆಲ್ಲ-ಬೇ-ಕೆಂ
ಗೆಲ್ಲ-ರಿ-ಗಿಂ-ತಲೂ
ಗೆಲ್ಲ-ಲಾ-ಗದ
ಗೆಲ್ಲ-ಲಾ-ರ-ದ-ವರು
ಗೆಲ್ಲಲು
ಗೆಲ್ಲು-ತ್ತೇವೆ
ಗೆಲ್ಲುವ
ಗೆಲ್ಲು-ವಂ-ತಹ
ಗೆಲ್ಲು-ವರೋ
ಗೆಲ್ಲು-ವು-ದಿಲ್ಲ
ಗೆಲ್ಲು-ವುದು
ಗೆಳ-ತನ
ಗೆಳತಿ
ಗೆಳ-ತಿಯ
ಗೆಳ-ತಿ-ಯ-ರಿಗೂ
ಗೆಳ-ತಿ-ಯರೂ
ಗೆಳ-ತಿ-ಯರೆ
ಗೆಳ-ತಿ-ಯ-ರೊ-ಡನೆ
ಗೆಳೆ-ತನ
ಗೆಳೆ-ತ-ನ-ವಿತ್ತು
ಗೆಳೆಯ
ಗೆಳೆ-ಯ-ಎಂಬ
ಗೆಳೆ-ಯನ
ಗೆಳೆ-ಯ-ನನ್ನು
ಗೆಳೆ-ಯ-ನಾಗಿ
ಗೆಳೆ-ಯ-ನಾ-ಗಿದ್ದ
ಗೆಳೆ-ಯ-ನಾ-ಗಿದ್ದೆ
ಗೆಳೆ-ಯ-ನಾದ
ಗೆಳೆ-ಯ-ನಾ-ದು-ದ-ರಿಂದ
ಗೆಳೆ-ಯ-ನಿಗೆ
ಗೆಳೆ-ಯ-ನೆಂದೇ
ಗೆಳೆ-ಯನೇ
ಗೆಳೆ-ಯರ
ಗೆಳೆ-ಯ-ರನ್ನು
ಗೆಳೆ-ಯ-ರ-ನ್ನೆಲ್ಲ
ಗೆಳೆ-ಯ-ರಾ-ಗಿ-ದ್ದರು
ಗೆಳೆ-ಯ-ರಾಗು
ಗೆಳೆ-ಯ-ರಾದ
ಗೆಳೆ-ಯ-ರಿ-ಬ್ಬರು
ಗೆಳೆ-ಯ-ರಿ-ಬ್ಬರೂ
ಗೆಳೆ-ಯರು
ಗೆಳೆ-ಯರೂ
ಗೆಳೆ-ಯರೆ
ಗೆಳೆ-ಯ-ರೆಲ್ಲ
ಗೆಳೆ-ಯ-ರೊ-ಡನೆ
ಗೆಳೆ-ಯ-ರೊ-ಡ-ನೆಯೂ
ಗೆಳೆ-ಯ-ರೊ-ಬ್ಬರು
ಗೆಳೆ-ಯ-ರೊ-ಬ್ಬರೂ
ಗೇಲಿ
ಗೇಲಿ-ಗೆ-ಬ್ಬಿ-ಸ-ಬೇ-ಕೆಂಬ
ಗೇಲಿ-ಮಾ-ಡಿ-ದರು
ಗೇಲಿ-ಮಾ-ಡುತ್ತಾ
ಗೇಳಿ-ದ-ನಾ-ದರೂ
ಗೊಂಡ
ಗೊಂಡನು
ಗೊಂಡಿತು
ಗೊಂಡಿದೆ
ಗೊಂಡಿದ್ದ
ಗೊಂಡು
ಗೊಂಬೆ-ಗ-ಳಂತೆ
ಗೊಡ-ಲಿ-ಯನ್ನು
ಗೊಡ-ಲಿ-ಯೊ-ಡನೆ
ಗೊಣ-ಗದೆ
ಗೊಣ-ಗಿ-ಕೊಳ್ಳು
ಗೊಣ-ಸನ್ನೂ
ಗೊಣ-ಸಿನ
ಗೊಣಸು
ಗೊಣ-ಸು-ಗಳನ್ನು
ಗೊತ್ತಾ
ಗೊತ್ತಾ-ಗ-ದಂತೆ
ಗೊತ್ತಾ-ಗ-ದಿ-ರು-ವಂತೆ
ಗೊತ್ತಾ-ಗದೆ
ಗೊತ್ತಾ-ಗ-ಬೇ-ಕಷ್ಟೆ
ಗೊತ್ತಾ-ಗ-ಬೇಕು
ಗೊತ್ತಾ-ಗ-ಲಿಲ್ಲ
ಗೊತ್ತಾಗಿ
ಗೊತ್ತಾ-ಗಿದೆ
ಗೊತ್ತಾ-ಗಿ-ರ-ಬೇಕು
ಗೊತ್ತಾ-ಗು-ತ್ತದೆ
ಗೊತ್ತಾ-ಗು-ವು-ದಿಲ್ಲ
ಗೊತ್ತಾದ
ಗೊತ್ತಾ-ದ-ಮೇಲೆ
ಗೊತ್ತಾ-ದರೆ
ಗೊತ್ತಾ-ಯಿ-ತಷ್ಟೆ
ಗೊತ್ತಾ-ಯಿತು
ಗೊತ್ತಿಗೆ
ಗೊತ್ತಿತ್ತು
ಗೊತ್ತಿದೆ
ಗೊತ್ತಿ-ದ್ದರೂ
ಗೊತ್ತಿ-ರ-ತ-ಕ್ಕುದೆ
ಗೊತ್ತಿ-ರು-ವುದು
ಗೊತ್ತಿಲ್ಲ
ಗೊತ್ತಿ-ಲ್ಲದ
ಗೊತ್ತಿ-ಲ್ಲ-ದಂತೆ
ಗೊತ್ತಿ-ಲ್ಲ-ದಿಲ್ಲ
ಗೊತ್ತಿ-ಲ್ಲವೆ
ಗೊತ್ತಿ-ಲ್ಲ-ವೆಂದು
ಗೊತ್ತು
ಗೊತ್ತುಈ
ಗೊತ್ತು-ಮಾ-ಡಿ-ದನು
ಗೊತ್ತೇ
ಗೊತ್ತೊ
ಗೊನೆ
ಗೊಪ್ಪಿಸಿ
ಗೊರ-ವ-ರನ್ನೂ
ಗೊರ-ಸಿನ
ಗೊರ-ಸಿ-ನಿಂದ
ಗೊಲ್ಲ
ಗೊಲ್ಲನ
ಗೊಲ್ಲ-ನಾಗಿ
ಗೊಲ್ಲ-ನಾದ
ಗೊಲ್ಲ-ನಿಗೆ
ಗೊಲ್ಲ-ಬಾ-ಲ-ಕ-ರೊ-ಡನೆ
ಗೊಲ್ಲರ
ಗೊಲ್ಲ-ರಂತೆ
ಗೊಲ್ಲ-ರ-ವಲ್ಲ
ಗೊಲ್ಲ-ರಿಗೆ
ಗೊಲ್ಲರು
ಗೊಲ್ಲರೂ
ಗೊಲ್ಲ-ರೆಲ್ಲಿ
ಗೊಳಗೇ
ಗೊಳಿ-ಸಿ-ದರು
ಗೊಳಿ-ಸಿ-ರಿ-ಎಂದು
ಗೊಳಿ-ಸುವ
ಗೊಳೋ
ಗೊಳ್ಳ-ಬಲ್ಲ
ಗೊಳ್ಳು-ತ್ತವೆ
ಗೊಳ್ಳುತ್ತಾ
ಗೋಕ-ರ್ಣ-ಕ್ಷೇ-ತ್ರ-ದಲ್ಲಿ
ಗೋಕುಲ
ಗೋಕು-ಲಕ್ಕೂ
ಗೋಕು-ಲಕ್ಕೆ
ಗೋಕು-ಲ-ಕ್ಕೆಲ್ಲ
ಗೋಕು-ಲದ
ಗೋಕು-ಲ-ದಲ್ಲಿ
ಗೋಕು-ಲ-ದ-ಲ್ಲಿಯೆ
ಗೋಕು-ಲ-ದ-ಲ್ಲಿಯೇ
ಗೋಕು-ಲ-ದ-ಲ್ಲಿ-ರುವ
ಗೋಕು-ಲ-ದ-ಲ್ಲಿ-ರು-ವ-ವರೆ-ಲ್ಲರ
ಗೋಕು-ಲ-ದ-ವರು
ಗೋಕು-ಲ-ದ-ವರೆಲ್ಲ
ಗೋಕು-ಲ-ದಿಂದ
ಗೋಕು-ಲ-ವನ್ನು
ಗೋಕು-ಲ-ವೆಲ್ಲ
ಗೋಗ-ಣ-ಇ-ವು-ಗಳನ್ನು
ಗೋಗ-ಣ-ಇ-ವು-ಗಳು
ಗೋಗ-ಣ-ದೊ-ಡನೆ
ಗೋಗಳ
ಗೋಗಳನ್ನು
ಗೋಗಳನ್ನೂ
ಗೋಗ-ಳಿ-ಗೆಲ್ಲ
ಗೋಗಳು
ಗೋಗಳೂ
ಗೋಚರ
ಗೋಚ-ರ-ನಾಗಿ
ಗೋಚ-ರ-ನಾ-ಗು-ವನು
ಗೋಚ-ರ-ನಾದೆ
ಗೋಚ-ರ-ವಾ-ಗ-ದಂತೆ
ಗೋಚ-ರ-ವಾ-ಗದು
ಗೋಚ-ರ-ವಾ-ಗ-ಬೇಕು
ಗೋಚ-ರ-ವಾ-ಗಿತ್ತು
ಗೋಚ-ರ-ವಾಗು
ಗೋಚ-ರ-ವಾ-ಗು-ತ್ತದೆ
ಗೋಚ-ರ-ವಾ-ಗುವ
ಗೋಚ-ರ-ವಾ-ಗು-ವುದು
ಗೋಚ-ರ-ವಾ-ದಂತಾ
ಗೋಚ-ರ-ವಾ-ದಂ-ತಾ-ಯಿತು
ಗೋಚ-ರ-ವಾ-ದುವು
ಗೋಚ-ರ-ವಾ-ಯಿತು
ಗೋಚ-ರಿ-ಸ-ಬಲ್ಲ
ಗೋಚ-ರಿ-ಸ-ಲಿಲ್ಲ
ಗೋಚ-ರಿ-ಸಲು
ಗೋಚ-ರಿ-ಸಿತು
ಗೋಚ-ರಿ-ಸಿ-ದನು
ಗೋಚ-ರಿ-ಸಿ-ದರು
ಗೋಚ-ರಿ-ಸಿ-ದ್ದಾನೆ
ಗೋಚ-ರಿಸು
ಗೋಚ-ರಿ-ಸು-ತ್ತದೆ
ಗೋಚ-ರಿ-ಸುತ್ತಾ
ಗೋಚ-ರಿ-ಸು-ತ್ತಾನೆ
ಗೋಚ-ರಿ-ಸುತ್ತಿ
ಗೋಚ-ರಿ-ಸು-ವನು
ಗೋಚ-ರಿ-ಸು-ವು-ದೆ-ಲ್ಲವೂ
ಗೋಡೆ-ಗಳ
ಗೋಡೆ-ಗ-ಳೆಲ್ಲ
ಗೋಡೆ-ಯಲ್ಲಿ
ಗೋದಾನ
ಗೋದಾ-ನವೇ
ಗೋದಾ-ವರಿ
ಗೋದಿಯ
ಗೋಪ
ಗೋಪ-ಗೋ-ಪಿ-ಯ-ರೆಲ್ಲ
ಗೋಪ-ನನ್ನು
ಗೋಪ-ಬಾ-ಲ-ನನ್ನು
ಗೋಪ-ಬಾ-ಲ-ರನ್ನೂ
ಗೋಪ-ಬಾ-ಲೆ-ಯರು
ಗೋಪಾ-ದದ
ಗೋಪಾನ್
ಗೋಪಾಲ
ಗೋಪಾ-ಲಕ
ಗೋಪಾ-ಲ-ಕರ
ಗೋಪಾ-ಲ-ಕ-ರನ್ನೂ
ಗೋಪಾ-ಲ-ಕ-ರಾ-ದನು
ಗೋಪಾ-ಲ-ಕ-ರಿ-ಗೆಲ್ಲ
ಗೋಪಾ-ಲ-ಕರು
ಗೋಪಾ-ಲ-ಕರೂ
ಗೋಪಾ-ಲ-ಕ-ರೆಲ್ಲ
ಗೋಪಾ-ಲ-ಕ-ರೆ-ಲ್ಲರ
ಗೋಪಾ-ಲ-ಕು-ಲ-ದ-ವ-ರಿ-ಗೆಲ್ಲ
ಗೋಪಾ-ಲ-ಬಾ-ಲ-ಕ-ರನ್ನೂ
ಗೋಪಾ-ಲ-ಬಾ-ಲ-ಕ-ರೆಲ್ಲ
ಗೋಪಾ-ಲ-ಬಾ-ಲ-ಕ-ರೊ-ಡನೆ
ಗೋಪಾ-ಲ-ಬಾ-ಲರು
ಗೋಪಾ-ಲ-ಬಾ-ಲ-ರೆಲ್ಲ
ಗೋಪಾ-ಲ-ಬಾ-ಲ-ರೊ-ಡನೆ
ಗೋಪಾ-ಲ-ಮ-ತಾ-ವ-ಲಂ-ಬಿ-ಗಾಗಿ
ಗೋಪಾ-ಲರ
ಗೋಪಾ-ಲ-ರನ್ನೂ
ಗೋಪಾ-ಲ-ರ-ನ್ನೆಲ್ಲ
ಗೋಪಾ-ಲರಿ
ಗೋಪಾ-ಲ-ರಿಗೂ
ಗೋಪಾ-ಲ-ರಿಗೆ
ಗೋಪಾ-ಲ-ರಿ-ಗೆಲ್ಲ
ಗೋಪಾ-ಲರು
ಗೋಪಾ-ಲ-ರು-ಹೀಗೆ
ಗೋಪಾ-ಲರೂ
ಗೋಪಾ-ಲ-ರೆಲ್ಲ
ಗೋಪಾ-ಲರೇ
ಗೋಪಾ-ಲ-ರೊ-ಡನೆ
ಗೋಪಿ
ಗೋಪಿ-ಕಾ-ಕ-ನ್ಯೆ-ಯ-ರಂತೆ
ಗೋಪಿ-ಕಾ-ಗೀ-ತ-ದಲ್ಲಿ
ಗೋಪಿ-ಕಾ-ಗೀತೆ
ಗೋಪಿಗೆ
ಗೋಪಿ-ಪ್ರೇ-ಮ-ವನ್ನು
ಗೋಪಿಯ
ಗೋಪಿ-ಯರ
ಗೋಪಿ-ಯ-ರಂತೂ
ಗೋಪಿ-ಯ-ರನ್ನು
ಗೋಪಿ-ಯ-ರನ್ನೂ
ಗೋಪಿ-ಯ-ರ-ನ್ನೆಲ್ಲ
ಗೋಪಿ-ಯ-ರಲ್ಲಿ
ಗೋಪಿ-ಯರಿ
ಗೋಪಿ-ಯ-ರಿಂದ
ಗೋಪಿ-ಯ-ರಿಗೂ
ಗೋಪಿ-ಯ-ರಿಗೆ
ಗೋಪಿ-ಯ-ರಿ-ಗೆಲ್ಲ
ಗೋಪಿ-ಯರು
ಗೋಪಿ-ಯರೆ
ಗೋಪಿ-ಯ-ರೆಲ್ಲ
ಗೋಪಿ-ಯ-ರೆ-ಲ್ಲರ
ಗೋಪಿ-ಯ-ರೆ-ಲ್ಲರೂ
ಗೋಪಿ-ಯ-ರೊ-ಡನೆ
ಗೋಪಿಯೂ
ಗೋಪಿ-ಯೊ-ಬ್ಬಳು
ಗೋಪೀ-ಜ-ನ-ವ-ಲ್ಲಭ
ಗೋಪೀ-ಜ-ನ-ವ-ಲ್ಲ-ಭ-ನಾದ
ಗೋಪೀ-ಪ್ರೇಮ
ಗೋಪೀ-ಪ್ರೇ-ಮದ
ಗೋಪೀ-ಪ್ರೇ-ಮ-ದಲ್ಲಾ
ಗೋಪೀ-ಪ್ರೇ-ಮ-ವನ್ನು
ಗೋಪು-ರ-ಗ-ಳುಳ್ಳ
ಗೋಪು-ರ-ಗ-ಳೆಲ್ಲ
ಗೋಪು-ರ-ಗ-ಳೊ-ಡನೆ
ಗೋಪು-ರ-ದ-ಮೇಲೆ
ಗೋಪು-ರ-ವುಳ್ಳ
ಗೋಬ್ರಾ-ಹ್ಮಣ
ಗೋಬ್ರಾ-ಹ್ಮ-ಣ-ರನ್ನೂ
ಗೋಬ್ರಾ-ಹ್ಮ-ಣ-ರಿಗೂ
ಗೋರ-ಕ್ಷ-ಕ-ನಾದ
ಗೋರೂ-ಪ-ದ-ಲ್ಲಿದ್ದ
ಗೋರೂ-ಪ-ದ-ಲ್ಲಿ-ರುವ
ಗೋರೂ-ಪನ್ನು
ಗೋರ್ಕಲ್ಲ
ಗೋಲು
ಗೋಳನ್ನು
ಗೋಳಾ-ಡಿ-ದನು
ಗೋಳಾ-ಡಿ-ದರು
ಗೋಳಾ-ಡಿ-ದಳು
ಗೋಳಾ-ಡಿ-ಸು-ತ್ತಿ-ರು-ವೆ-ಯಲ್ಲಾ
ಗೋಳಾ-ಡುತ್ತಾ
ಗೋಳಾ-ಡು-ತ್ತಿ-ದ್ದರು
ಗೋಳಾ-ಡು-ತ್ತಿ-ರು-ವೆ-ಯಲ್ಲಾ
ಗೋಳಾ-ಡು-ವಂ-ತಾ-ದರೆ
ಗೋಳಿ-ಡು-ತ್ತಿ-ರುವ
ಗೋಳು
ಗೋಳೋ
ಗೋವಧೆ
ಗೋವನ್ನು
ಗೋವ-ರ್ಧನ
ಗೋವ-ರ್ಧ-ನ-ಪ-ರ್ವತ
ಗೋವ-ರ್ಧ-ನ-ಪ-ರ್ವ-ತ-ವನ್ನು
ಗೋವ-ರ್ಧ-ನ-ಪೂ-ಜೆ-ಯಾಗಿ
ಗೋವ-ರ್ಧ-ನ-ವನ್ನು
ಗೋವ-ಳರ
ಗೋವ-ಳ-ರಾಗಿ
ಗೋವ-ಳರೂ
ಗೋವಿಂದ
ಗೋವಿಂ-ದ-ನಿಗೆ
ಗೋವಿಂ-ದನು
ಗೋವಿಂ-ದಾ-ಪಾಂ-ಗ-ನಿ-ರ್ಭಿನ್ನೇ
ಗೋವಿಂ-ದಾಯ
ಗೋವು
ಗೋವು-ಗಳ
ಗೋವು-ಗಳನ್ನು
ಗೋವು-ಗಳನ್ನೂ
ಗೋವು-ಗ-ಳಲ್ಲ
ಗೋವು-ಗಳಿಂದ
ಗೋವು-ಗಳು
ಗೋವು-ಗಳೂ
ಗೋವು-ಗ-ಳೆಲ್ಲ
ಗೌತಮ
ಗೌರವ
ಗೌರ-ವ-ಗಳನ್ನು
ಗೌರ-ವ-ಗ-ಳಿಗೆ
ಗೌರ-ವ-ಗಳು
ಗೌರ-ವ-ದಿಂದ
ಗೌರ-ವ-ವನ್ನು
ಗೌರ-ವಿ-ಸ-ಬೇಕು
ಗೌರ-ವಿ-ಸ-ಬೇ-ಕೆಂ-ಬುದೇ
ಗೌರ-ವಿ-ಸಿತು
ಗೌರ-ವಿ-ಸಿ-ದನು
ಗೌರ-ವಿ-ಸಿ-ದರು
ಗೌರ-ವಿ-ಸಿ-ದುದು
ಗೌರ-ವಿ-ಸು-ವುದು
ಗೌರ-ವಿ-ಸು-ವುದೂ
ಗೌರಿ-ದೇ-ವಿಯ
ಗೌರಿಯ
ಗೌರಿ-ಯನ್ನು
ಗೌರೀ-ಶಂ-ಕರ
ಗ್ರಂಥ
ಗ್ರಂಥ-ಕ-ರ್ತನ
ಗ್ರಂಥ-ಗಳಲ್ಲಿ
ಗ್ರಂಥ-ಗ-ಳಿವೆ
ಗ್ರಂಥ-ಗಳೆ
ಗ್ರಂಥ-ಗಳೇ
ಗ್ರಂಥದ
ಗ್ರಂಥ-ರ-ಚ-ನೆ-ಯಾ-ಗಿ-ರ-ಬೇ-ಕೆಂದು
ಗ್ರಂಥ-ವನ್ನು
ಗ್ರಂಥ-ವಾ-ಗಿದೆ
ಗ್ರಂಥ-ವಿದೆ
ಗ್ರಂಥಿ
ಗ್ರಥಿ-ತಾ-ಖಿ-ಲ-ಲೋ-ಕ-ಹೃದಂ
ಗ್ರಸ
ಗ್ರಹ
ಗ್ರಹಕ್ಕೆ
ಗ್ರಹ-ಗಳನ್ನು
ಗ್ರಹ-ಗಳು
ಗ್ರಹ-ಚಾರ
ಗ್ರಹ-ಚೇಷ್ಟೆ
ಗ್ರಹಣ
ಗ್ರಹ-ಣ-ಕ್ಕಿಂತ
ಗ್ರಹ-ಣ-ಗ-ಳಿಗೆ
ಗ್ರಹ-ದಿಂದ
ಗ್ರಹ-ವಾಗಿ
ಗ್ರಹ-ವಾ-ದಾ-ಗಲೆ
ಗ್ರಹ-ವಿದ್ದ
ಗ್ರಹವು
ಗ್ರಹಾದಿ
ಗ್ರಹಿಸ
ಗ್ರಹಿ-ಸ-ಬ-ಲ್ಲಿರಿ
ಗ್ರಹಿ-ಸ-ಬೇಕು
ಗ್ರಹಿ-ಸಲು
ಗ್ರಹಿಸಿ
ಗ್ರಹಿ-ಸಿದ
ಗ್ರಹಿ-ಸಿ-ದರೆ
ಗ್ರಹಿ-ಸಿ-ಬಿ-ಡು-ತ್ತಿ-ದ್ದರು
ಗ್ರಹಿಸು
ಗ್ರಹಿ-ಸು-ವಂತೆ
ಗ್ರಹಿ-ಸು-ವೆಯೊ
ಗ್ರಾಮ
ಗ್ರಾಮ-ದಲ್ಲಿ
ಗ್ರೀಕ್
ಗ್ರೀಷ್ಮ
ಗ್ರೀಷ್ಮ-ಋತು
ಗ್ರೀಷ್ಮವೂ
ಘಟನೆ
ಘಟ-ನೆ-ಗಳು
ಘಟ-ನೆ-ಯನ್ನು
ಘಟ-ನೆ-ಯೊಂದು
ಘಟಿ-ಸೀತು
ಘಟ್ಟ-ಗ-ಳಾಗಿ
ಘಟ್ಟ-ವನ್ನು
ಘನ
ಘನ-ಘೋ-ರ-ವಾದ
ಘನೀ-ಭೂ-ತ-ವಾಗಿ
ಘಲಿ-ಘ-ಲಿ-ಧ್ವ-ನಿ-ಯೊ-ಡನೆ
ಘಲಿ-ಘ-ಲಿ-ರೆ-ನಿ-ಸುತ್ತಾ
ಘಲಿ-ಘ-ಲಿ-ರೆ-ನ್ನು-ವಂತೆ
ಘಲ್ಘಲ್
ಘಳಿಗೆ
ಘಳಿ-ಗೆ-ಯಂತೆ
ಘೀಳಿ-ಟ್ಟವು
ಘೇಂಕ-ರಿಸಿ
ಘೊಳ್ಳೆಂದು
ಘೋರ
ಘೋರ-ಯು-ದ್ಧ-ವಾ-ಯಿತು
ಘೋರ-ರಾ-ಕ್ಷ-ಸನ
ಘೋರ-ರೂ-ಪದ
ಘೋರ-ವಾದ
ಘ್ನನ್
ಚ
ಚಂಚಲ
ಚಂಚ-ಲ-ದೃಷ್ಟಿ
ಚಂಚ-ಲ-ವಾದ
ಚಂಚ-ಲ-ವಾ-ಯಿತು
ಚಂಡ
ಚಂಡ-ವೇ-ಗ-ನೆಂ-ದರೆ
ಚಂಡ-ವೇ-ಗ-ನೆಂಬ
ಚಂಡಾ-ಟ-ವಾ-ಡಿ-ದರು
ಚಂಡಾ-ಮ-ರ್ಕ-ರನ್ನು
ಚಂಡಾ-ಮ-ರ್ಕ-ರಿಗೆ
ಚಂಡಾ-ಮ-ರ್ಕರು
ಚಂಡಾ-ಲ-ನತ್ತ
ಚಂಡಾ-ಲ-ನಲ್ಲ
ಚಂಡಾ-ಲ-ನಾದ
ಚಂಡಾ-ಲ-ನಾ-ದನು
ಚಂಡಿ
ಚಂಡಿ-ನತ್ತ
ಚಂದ
ಚಂದನ
ಚಂದಿ-ರನು
ಚಂದ್ರ
ಚಂದ್ರ-ಕ-ಳೆ-ಯನ್ನು
ಚಂದ್ರ-ಕ-ಳೆ-ಯಿಂ-ದೊ-ಪ್ಪುವ
ಚಂದ್ರ-ಕೇ-ತು-ವೆಂಬ
ಚಂದ್ರ-ಗು-ಪ್ತ-ನನ್ನು
ಚಂದ್ರನ
ಚಂದ್ರ-ನಂತೆ
ಚಂದ್ರ-ನನ್ನು
ಚಂದ್ರ-ನಿಗೂ
ಚಂದ್ರ-ನಿಗೆ
ಚಂದ್ರ-ನಿ-ಲ್ಲದ
ಚಂದ್ರನು
ಚಂದ್ರನೂ
ಚಂದ್ರನೇ
ಚಂದ್ರ-ನೊ-ಡನೆ
ಚಂದ್ರ-ಬಿಂಬ
ಚಂದ್ರ-ಬಿಂ-ಬ-ದಂತೆ
ಚಂದ್ರ-ಮಂ-ಡ-ಲ-ವಿದೆ
ಚಂದ್ರರ
ಚಂದ್ರ-ರನ್ನು
ಚಂದ್ರ-ರಿಗೆ
ಚಂದ್ರರು
ಚಂದ್ರ-ಲೋ-ಕ-ವನ್ನು
ಚಂದ್ರ-ಲೋ-ಕಾದಿ
ಚಂದ್ರ-ವಂಶ
ಚಂದ್ರ-ವಂ-ಶಕ್ಕೆ
ಚಂದ್ರ-ವಂ-ಶದ
ಚಂದ್ರಾ-ಕಾ-ರದ
ಚಂದ್ರಾದಿ
ಚಂದ್ರಿಕೆ
ಚಂದ್ರೋ-ದ-ಯ-ವಾ-ಗು-ತ್ತಲೆ
ಚಂಪನು
ಚಂಪಾ-ಪು-ರ-ವನ್ನು
ಚಕ್ರ
ಚಕ್ರಂ
ಚಕ್ರ-ಇ-ತ್ಯಾದಿ
ಚಕ್ರಕ್ಕೆ
ಚಕ್ರ-ಗಳಲ್ಲಿ
ಚಕ್ರ-ಗ-ಳಿಂ
ಚಕ್ರ-ಗ-ಳು-ಸತ್ವ
ಚಕ್ರ-ದಂ-ತಹ
ಚಕ್ರ-ದಂತೆ
ಚಕ್ರ-ದ-ಮೇಲೆ
ಚಕ್ರ-ದಿಂದ
ಚಕ್ರ-ಪಾಣಿ
ಚಕ್ರ-ರೇ-ಖೆ-ಯನ್ನೂ
ಚಕ್ರ-ವನ್ನು
ಚಕ್ರ-ವನ್ನೂ
ಚಕ್ರ-ವನ್ನೇ
ಚಕ್ರ-ವರ್ತಿ
ಚಕ್ರ-ವ-ರ್ತಿ-ಗ-ಳಾ-ದರು
ಚಕ್ರ-ವ-ರ್ತಿ-ಗ-ಳಿಗೂ
ಚಕ್ರ-ವ-ರ್ತಿಗೆ
ಚಕ್ರ-ವ-ರ್ತಿ-ಪ-ದ-ವಿ-ಯನ್ನು
ಚಕ್ರ-ವ-ರ್ತಿ-ಪ-ದ-ವಿ-ಯಲ್ಲಾ
ಚಕ್ರ-ವ-ರ್ತಿಯ
ಚಕ್ರ-ವ-ರ್ತಿ-ಯಂತೆ
ಚಕ್ರ-ವ-ರ್ತಿ-ಯ-ನ್ನಾಗಿ
ಚಕ್ರ-ವ-ರ್ತಿ-ಯನ್ನು
ಚಕ್ರ-ವ-ರ್ತಿ-ಯಾಗಿ
ಚಕ್ರ-ವ-ರ್ತಿ-ಯಾ-ಗಿದ್ದ
ಚಕ್ರ-ವ-ರ್ತಿ-ಯಾ-ಗಿ-ದ್ದರೂ
ಚಕ್ರ-ವ-ರ್ತಿ-ಯಾ-ಗಿದ್ದು
ಚಕ್ರ-ವ-ರ್ತಿ-ಯಾದ
ಚಕ್ರ-ವ-ರ್ತಿ-ಯಾ-ದನು
ಚಕ್ರ-ವ-ರ್ತಿಯು
ಚಕ್ರ-ವ-ರ್ತಿಯೂ
ಚಕ್ರ-ವಾಕ
ಚಕ್ರ-ವಾ-ಕಕ್ಕೆ
ಚಕ್ರ-ವಾಕಿ
ಚಕ್ರವು
ಚಕ್ರ-ವೆಂದರೆ
ಚಕ್ರ-ವೆಂದೂ
ಚಕ್ರವೇ
ಚಕ್ರ-ವೊಂದು
ಚಕ್ರಾಂ-ಕಿ-ತ-ವಾದ
ಚಕ್ರಾ-ಧಿ-ಪ-ತ್ಯ-ಗಳು
ಚಕ್ರಾ-ಯುಧ
ಚಕ್ರಾ-ಯು-ಧ-ದಿಂದ
ಚಕ್ರಾ-ಯು-ಧ-ವನ್ನು
ಚಕ್ರಾ-ಯು-ಧ-ವನ್ನೆ
ಚಕ್ರಾ-ಯು-ಧವು
ಚಕ್ಷುಷಾ
ಚಕ್ಷು-ಷಾಂ
ಚಕ್ಷೂಂಷಿ
ಚಟಾ-ಕಿ-ಯಿಂದ
ಚಡ-ಪ-ಡಿ-ಸು-ತ್ತಿ-ದ್ದರು
ಚಡ-ಪ-ಡಿ-ಸು-ತ್ತಿ-ರಲು
ಚಡ್ಡಿ-ಯನ್ನು
ಚತು-ರಂ-ಗ-ಬ-ಲದ
ಚತು-ರಂ-ಗ-ಸೇ-ನೆ-ಯೊ-ಡನೆ
ಚತು-ರಂ-ಗ-ಸೈನ್ಯ
ಚತು-ರನು
ಚತು-ರ್ಭು-ಜ-ಗಳಿಂದ
ಚತು-ರ್ಭು-ಜ-ನಾದ
ಚತು-ರ್ಭು-ಜ-ರಾದ
ಚತು-ರ್ಮುಖ
ಚತು-ರ್ಮು-ಖನು
ಚತು-ರ್ಮು-ಖ-ಬ್ರಹ್ಮ
ಚತು-ರ್ಮು-ಖ-ಬ್ರ-ಹ್ಮನು
ಚತು-ರ್ಮು-ಖ-ಬ್ರ-ಹ್ಮನೂ
ಚತು-ರ್ಮು-ಖ-ಬ್ರ-ಹ್ಮನೇ
ಚತು-ರ್ಯು-ಗ-ಗಳು
ಚತು-ಷ್ಟಯಂ
ಚದರಿ
ಚದ-ರಿ-ದವು
ಚದ-ರಿ-ಸು-ವಂತೆ
ಚದು-ರಂ-ಗ-ಸೇನೆ
ಚದು-ರ-ನಾ-ಗಿ-ದ್ದಾನೆ
ಚದು-ರಿ-ದವು
ಚನ್ನಿ-ಗ-ನಾದ
ಚಪಲ
ಚಪ-ಲ-ಗ-ಳಿ-ಗಿಂತ
ಚಪ-ಲ-ದಿಂದ
ಚಪ-ಲ-ವಾದ
ಚಪ್ಪ-ಟೆ-ಯಾದ
ಚಪ್ಪರ
ಚಪ್ಪ-ರಿ-ಸ-ಬೇಕು
ಚಪ್ಪಾಳೆ
ಚಮ-ಚ-ವನ್ನು
ಚಮ-ತ್ಕಾ-ರದ
ಚಮಸ
ಚರ
ಚರಂತಿ
ಚರ-ಜೀ-ವಿಗೆ
ಚರ-ಣ-ದಾ-ಸಿ-ಯಾಗಿ
ಚರ-ಣ-ರಜ
ಚರ-ಣಾರ
ಚರಾ
ಚರಾ-ಚರ
ಚರಾ-ಚ-ರ-ವ-ಸ್ತು-ಗ-ಳೆಲ್ಲ
ಚರಾ-ಚ-ರ-ವ-ಸ್ತು-ವೆಲ್ಲ
ಚರಾ-ಚ-ರಾ-ತ್ಮ-ಕ-ವಾದ
ಚರಾ-ತ್ಮ-ಕ-ವಾದ
ಚರಿ-ತೆ-ಯಿಂದ
ಚರಿತ್ರೆ
ಚರಿ-ತ್ರೆ-ಯನ್ನು
ಚರಿಸಿ
ಚರ್ಚೆ
ಚರ್ತು-ರ್ವೇ-ದ-ವನ್ನು
ಚರ್ಮ
ಚರ್ಮ-ಎಂಬ
ಚರ್ಮ-ಗಳು
ಚರ್ಮ-ದಿಂದ
ಚರ್ಮನ್
ಚರ್ಮವೇ
ಚರ್ಯ
ಚರ್ಯೆ-ಗಳನ್ನು
ಚರ್ಯೆ-ಯನ್ನು
ಚರ್ಯೆ-ಯಿಂದ
ಚಲ-ನ-ಶಕ್ತಿ
ಚಲ-ವಿ-ಚ-ಲಿ-ತ-ನಾ-ಗ-ಲಿಲ್ಲ
ಚಲಸಿ
ಚಲಿ
ಚಲಿ-ಸ-ಬ-ಲ್ಲ-ವಾ-ದರೂ
ಚಲಿ-ಸ-ಬೇಕು
ಚಲಿ-ಸ-ಲಾ-ರವು
ಚಲಿ-ಸ-ಲಿಲ್ಲ
ಚಲಿ-ಸಲು
ಚಲಿ-ಸಿತು
ಚಲಿ-ಸಿ-ದಂತೆ
ಚಲಿ-ಸಿ-ದ-ನೆಂದು
ಚಲಿ-ಸಿ-ದವು
ಚಲಿ-ಸಿ-ದುವು
ಚಲಿ-ಸು-ತ್ತಲೆ
ಚಲಿ-ಸು-ತ್ತಿದೆ
ಚಲಿ-ಸುವ
ಚಲಿ-ಸು-ವಂ-ತಿಲ್ಲ
ಚಲ್ಲಾ-ಪಿ-ಲ್ಲಿ-ಯಾಗಿ
ಚಲ್ಲಾ-ಪಿ-ಲ್ಲಿ-ಯಾ-ಗಿದೆ
ಚಲ್ಲಾ-ಪಿ-ಲ್ಲಿ-ಯಾ-ಗು-ವಂತೆ
ಚಳಿ
ಚಳಿ-ಯನ್ನು
ಚಳಿ-ಯಲ್ಲಿ
ಚಷ್ಟೇ
ಚಾಂಚಲ್ಯ
ಚಾಕ್ಷುಷ
ಚಾಚಿ-ಕೊಂಡು
ಚಾಚಿದ
ಚಾಚಿ-ದನು
ಚಾಚೂ
ಚಾಟು-ಕಾರೈ
ಚಾಣೂರ
ಚಾಣೂ-ರ-ಎಂಬ
ಚಾಣೂ-ರನ
ಚಾಣೂ-ರ-ನಿಗೆ
ಚಾತು-ರ್ಮಾ-ಸ್ಯದ
ಚಾಮರ
ಚಾಮ-ರ-ಗಳನ್ನು
ಚಾಮ-ರ-ಗಳನ್ನೂ
ಚಾರಣ
ಚಾರ-ವೆ-ಸ-ಗಿದ
ಚಾರಿ-ತ್ರ-ಶು-ದ್ಧಿಗೆ
ಚಾರಿ-ತ್ರಿಕ
ಚಾರಿ-ತ್ರಿ-ಕ-ದೃ-ಷ್ಟಿ-ಯಿಂದ
ಚಾರು
ಚಾರು-ದೇವ
ಚಾರು-ಧೇಷ್ಣ
ಚಾರು-ಭಧ್ರ
ಚಾರು-ಭಾನು
ಚಾರು-ಮಂತ
ಚಾರು-ಮತಿ
ಚಾಽಧಿ-ಪ-ತಯೇ
ಚಿಂತ-ನೆ-ಯ-ಲ್ಲಿಯೇ
ಚಿಂತ-ನೆ-ಯಿಂದ
ಚಿಂತ-ಯಸೇ
ಚಿಂತಾ
ಚಿಂತಾ-ಕ್ರಾಂ-ತ-ರಾ-ದರು
ಚಿಂತಾ-ಮ-ಗ್ನ-ನಾದ
ಚಿಂತಿ-ಸ-ಬೇಡ
ಚಿಂತಿಸಿ
ಚಿಂತಿ-ಸಿ-ದನು
ಚಿಂತಿ-ಸುತ್ತಾ
ಚಿಂತಿ-ಸು-ತ್ತಿದ್ದ
ಚಿಂತಿ-ಸು-ತ್ತಿದ್ದು
ಚಿಂತಿ-ಸು-ತ್ತಿ-ರಲು
ಚಿಂತಿ-ಸು-ತ್ತಿ-ರು-ವನು
ಚಿಂತಿ-ಸು-ತ್ತಿ-ರು-ವಾಗ
ಚಿಂತೆ
ಚಿಂತೆಗೆ
ಚಿಂತೆ-ಬೇಡ
ಚಿಂತೆ-ಮಾ-ಡ-ಬೇಡ
ಚಿಂತೆ-ಯನ್ನು
ಚಿಂತೆ-ಯಲ್ಲಿ
ಚಿಂತೆ-ಯ-ಲ್ಲಿಯೂ
ಚಿಂತೆ-ಯ-ಲ್ಲಿಯೇ
ಚಿಂತೆ-ಯಲ್ಲೆ
ಚಿಂತೆ-ಯಿಂದ
ಚಿಂತೆ-ಯಿಂ-ದಲೆ
ಚಿಂತೆ-ಯಿಲ್ಲ
ಚಿಂತೆಯು
ಚಿಂತೆ-ಯೂ-ಊ-ಟದ
ಚಿಂತೆಯೇ
ಚಿಂದಿ-ಗಳನ್ನು
ಚಿಂದಿಯ
ಚಿಕಿತ್ಸೆ
ಚಿಕ್ಕ
ಚಿಕ್ಕಂ-ದಿ-ನಲ್ಲಿ
ಚಿಕ್ಕಂ-ದಿ-ನ-ಲ್ಲಿಯೇ
ಚಿಕ್ಕಂ-ದಿ-ನಿಂದ
ಚಿಕ್ಕಂ-ದಿ-ನಿಂ-ದಲೂ
ಚಿಕ್ಕಂ-ದಿ-ನಿಂ-ದಲೇ
ಚಿಕ್ಕಪ್ಪ
ಚಿಕ್ಕಪ್ಪಂ
ಚಿಕ್ಕ-ಪ್ಪಂ-ದಿ-ರನ್ನು
ಚಿಕ್ಕ-ಪ್ಪಂ-ದಿ-ರನ್ನೂ
ಚಿಕ್ಕ-ಪ್ಪಂ-ದಿ-ರಿಗೆ
ಚಿಕ್ಕ-ಪ್ಪಂ-ದಿರು
ಚಿಕ್ಕ-ಪ್ಪ-ನನ್ನು
ಚಿಕ್ಕ-ಮ್ಮನ
ಚಿಕ್ಕ-ಮ್ಮ-ನಾದ
ಚಿಕ್ಕ-ಮ್ಮ-ನಿಗೆ
ಚಿಕ್ಕ-ವ-ನಾ-ದರೂ
ಚಿಕ್ಕ-ವ-ಯಸ್ಸು
ಚಿಕ್ಕ-ವ-ರೆಂಬ
ಚಿಗು-ರಿಗೆ
ಚಿಗು-ರಿನ
ಚಿಗು-ರಿ-ನಂ-ತಿ-ರುವ
ಚಿಗು-ರಿ-ನಂತೆ
ಚಿಗು-ರಿ-ನಿಂದ
ಚಿಗು-ರಿ-ರು-ವುದು
ಚಿಗು-ರು-ಗೈ-ಗ-ಳ-ಲ್ಲಿದ್ದ
ಚಿಗು-ರೆ-ಲೆ-ಯಂತೆ
ಚಿಗು-ರೊ-ಡೆದು
ಚಿತೆಯ
ಚಿತ್ತ
ಚಿತ್ತ-ಎಂಬ
ಚಿತ್ತಕ್ಕೆ
ಚಿತ್ತ-ಚಾಂ-ಚ-ಲ್ಯ-ವಿ-ಲ್ಲ-ದ-ವನ
ಚಿತ್ತ-ಚೋ-ರ-ನ-ದಾ-ಯಿತು
ಚಿತ್ತ-ದಿಂದ
ಚಿತ್ತ-ವನ್ನು
ಚಿತ್ತವು
ಚಿತ್ತ-ಶುದ್ಧಿ
ಚಿತ್ತೀ-ನಾಂ
ಚಿತ್ತೈ-ಸಿತು
ಚಿತ್ರ
ಚಿತ್ರ-ಕಲೆ
ಚಿತ್ರ-ಕ-ಲೆ-ಯಲ್ಲಿ
ಚಿತ್ರ-ಕೇತು
ಚಿತ್ರ-ಕೇ-ತು-ವನ್ನು
ಚಿತ್ರ-ಕೇ-ತು-ವಿಗೆ
ಚಿತ್ರ-ಕೇ-ತು-ವಿನ
ಚಿತ್ರ-ಕೇ-ತುವು
ಚಿತ್ರ-ಕೇ-ತುವೇ
ಚಿತ್ರ-ಗಳ
ಚಿತ್ರ-ಗಳನ್ನು
ಚಿತ್ರ-ಗ-ಳಾಗಿ
ಚಿತ್ರಗು
ಚಿತ್ರದ
ಚಿತ್ರ-ಪ-ಟ-ಗಳು
ಚಿತ್ರ-ಪ್ರ-ತಿ-ಮೆ-ಗ-ಳಂತೆ
ಚಿತ್ರ-ರ-ಥ-ನೆಂಬ
ಚಿತ್ರ-ಲೇಖೆ
ಚಿತ್ರ-ಲೇ-ಖೆಗೆ
ಚಿತ್ರ-ವನ್ನು
ಚಿತ್ರ-ವಿ-ಚಿ-ತ್ರ-ವಾಗಿ
ಚಿತ್ರ-ವಿ-ಚಿ-ತ್ರ-ವಾ-ಗಿಯೂ
ಚಿತ್ರ-ವಿ-ಚಿ-ತ್ರ-ವಾದ
ಚಿತ್ರವೂ
ಚಿತ್ರ-ಸೇ-ನ-ನೆಂಬ
ಚಿತ್ರಾ
ಚಿತ್ರಿ-ಸಿ-ದ್ದಾರೆ
ಚಿತ್ರಿ-ಸುವ
ಚಿತ್ಸ್ವ-ರೂ-ಪ-ನಾಗಿ
ಚಿದ್ರೂಪ
ಚಿನ್ನ
ಚಿನ್ನದ
ಚಿನ್ನ-ದಂತೆ
ಚಿನ್ನ-ದಿಂದ
ಚಿನ್ನ-ವನ್ನು
ಚಿಪ್ಪನ್ನು
ಚಿಮ್ಮಿತು
ಚಿಮ್ಮಿ-ಸಿ-ದಳು
ಚಿಮ್ಮುತ್ತಾ
ಚಿರಂ-ಜೀ-ವಿ-ಯಾಗಿ
ಚಿರಂ-ಜೀ-ವಿ-ಯಾದ
ಚಿರಂ-ಜೀ-ವಿ-ಯಾ-ದ-ನಂತೆ
ಚಿರ-ಕಾಲ
ಚಿರ-ನೂ-ತ-ನ-ವಾದ
ಚಿರ-ಪ-ರಿ-ಚಿ-ತ-ನಾದ
ಚಿರಾ-ಯು-ವಾ-ದರೂ
ಚಿವುಟಿ
ಚಿಹ್ನೆ-ಗಳನ್ನೆಲ್ಲ
ಚಿಹ್ನೆ-ಗಳು
ಚಿಹ್ನೆ-ಯನ್ನು
ಚಿಹ್ನೆಯೂ
ಚೀನಾಂ-ಬ-ರದ
ಚೀಪು-ತ್ತಿದೆ
ಚೀಮಾರಿ
ಚೀರಿ
ಚೀರಿ-ದರು
ಚುಚ್ಚಿ
ಚುಚ್ಚಿ-ದಳು
ಚುಚ್ಚಿ-ರು-ವುದು
ಚುಚ್ಚು
ಚುರು-ಕಾ-ಗುತ್ತಾ
ಚೂಪಾಗಿ
ಚೂಪಾದ
ಚೂರನ್ನು
ಚೂರಿ-ಯಂತೆ
ಚೂರು
ಚೂರು-ಚೂ-ರಾ-ಯಿತು
ಚೂರ್ಣ-ಯಾ-ಽರೀನ್
ಚೆಂಡನ್ನು
ಚೆಂಡಾ-ಟ-ದಲ್ಲಿ
ಚೆಂಡಿನ
ಚೆಂಡಿ-ನಂ-ತಾ-ಗಿದೆ
ಚೆಂಡು
ಚೆಂಬು
ಚೆದ-ರಿ-ಹೋ-ದವು
ಚೆನ್ನಾಗಿ
ಚೆನ್ನಾ-ಗಿ-ರು-ತ್ತಿತ್ತು
ಚೆನ್ನಿ-ಗ-ನನ್ನು
ಚೆನ್ನಿ-ಗ-ನೊಬ್ಬ
ಚೆಲುವ
ಚೆಲು-ವನ್ನು
ಚೆಲು-ವರೆ-ನಿ-ಸಿ-ಕೊಂ-ಡ-ವರ
ಚೆಲು-ವಿ-ಗಾಗಿ
ಚೆಲು-ವಿ-ನಿಂದ
ಚೆಲುವು
ಚೆಲು-ವು-ಳ್ಳ-ವರು
ಚೆಲುವೆ
ಚೆಲ್ಲ-ಬೇಕು
ಚೆಲ್ಲಾ-ಟ-ವಾ-ಡು-ತ್ತಿ-ರುವ
ಚೆಲ್ಲಾ-ಟ-ವಾ-ಡು-ವ-ವ-ಳಂತೆ
ಚೆಲ್ಲಾ-ಪಿ-ಲ್ಲಿ-ಯಾಗಿ
ಚೆಲ್ಲಿ
ಚೆಲ್ಲಿ-ದಂ-ತಹ
ಚೆಲ್ಲಿ-ದರೂ
ಚೆಲ್ಲಿ-ದಾಗ
ಚೆಲ್ಲಿದ್ದ
ಚೆಲ್ಲಿರಿ
ಚೆಲ್ಲುತ್ತ
ಚೆಲ್ಲು-ತ್ತದೆ
ಚೆಲ್ಲುತ್ತಾ
ಚೆಲ್ಲು-ತ್ತಿದೆ
ಚೆಲ್ಲು-ತ್ತಿದ್ದ
ಚೆಲ್ಲು-ತ್ತಿ-ದ್ದರೂ
ಚೆಲ್ಲು-ತ್ತಿ-ರು-ತ್ತದೆ
ಚೆಲ್ಲು-ತ್ತಿ-ರುವ
ಚೆಲ್ಲು-ತ್ತಿವೆ
ಚೆಲ್ಲುವ
ಚೆಲ್ಲೆ-ಗಂ-ಗಳ
ಚೆಲ್ಲೆ-ಗಂ-ಗಳು
ಚೇತನ
ಚೇತ-ನಕ್ಕೆ
ಚೇತ-ನ-ಗೊಳಿ
ಚೇತ-ನ-ವನ್ನು
ಚೇತ-ನವೂ
ಚೇತನಾ
ಚೇತ-ನಾ-ಚೇ-ತನ
ಚೇತ-ಸಾಂ
ಚೇತಾಃ
ಚೇದಿ-ದೇ-ಶದ
ಚೇದಿ-ರಾ-ಜ-ನಾದ
ಚೇಳಿಗೆ
ಚೇಳು
ಚೇಳು-ಗಳು
ಚೇಷ್ಟೆ-ಗಳನ್ನು
ಚೇಷ್ಟೆ-ಗಳಿಂದ
ಚೈತನ್ಯ
ಚೈತ-ನ್ಯ-ಪ್ರಭು
ಚೈತ್ರ
ಚೈವ
ಚೊಕ್ಕ
ಚೋಟು-ದ್ದದ
ಚೋರ-ವೃ-ತ್ತಿಗೆ
ಚೌಕ-ಟ್ಟನ್ನು
ಚೌಕಾ-ಶಿ-ಯೇನೂ
ಚ್ಯವನ
ಚ್ಯವ-ನನು
ಚ್ಯವ-ನ-ಪು-ಷಿಗೆ
ಚ್ಯವ-ನ-ಪು-ಷಿಯ
ಚ್ಯವ-ನ-ಮ-ಹ-ರ್ಷಿಗೆ
ಚ್ಯವ-ನ-ಮ-ಹ-ರ್ಷಿಯ
ಚ್ಯವ-ನ-ಮ-ಹ-ರ್ಷಿ-ಯನ್ನು
ಚ್ಯುತಿ-ಯಿಲ್ಲ
ಚ್ಯುತಿ-ಯಿ-ಲ್ಲ-ವೆಂದು
ಛಂಗನೆ
ಛಂದೋ-ಮ-ಯಾಽನ್ನ
ಛತ್ರ
ಛತ್ರಿ-ಯನ್ನು
ಛಲ
ಛಳಿ-ಯೆಲ್ಲ
ಛಾಂದೋಗ್ಯ
ಛಾಂದೋ-ಗ್ಯೋ-ಪ-ನಿ-ಷ-ತ್ತಿನ
ಛಾದಯ
ಛಿ
ಛಿಂದಿ
ಛೀ
ಛೀಗುಟ್ಟಿ
ಛೀಗು-ಟ್ಟಿದ
ಛೀಗು-ಟ್ಟಿ-ದನು
ಛೀಗು-ಟ್ಟಿ-ದರೆ
ಛೀಗು-ಟ್ಟುತ್ತಾ
ಛೆ
ಜಂಗಮ
ಜಂಜ-ಡ-ವ-ನ್ನೆಲ್ಲ
ಜಂಬೂ-ದ್ವೀಪ
ಜಂಬೂ-ದ್ವೀ-ಪದ
ಜಂಬೂ-ದ್ವೀ-ಪ-ದಲ್ಲಿ
ಜಂಬೂ-ದ್ವೀ-ಪ-ವನ್ನು
ಜಕ್ಕ-ವಕ್ಕಿ
ಜಕ್ಕ-ವ-ಕ್ಕಿ-ಗ-ಳಂತೆ
ಜಕ್ಕೆ-ವ-ಕ್ಕಿ-ಗ-ಳಂತೆ
ಜಕ್ಷಧ್ವಂ
ಜಗ
ಜಗತಿ
ಜಗತ್
ಜಗ-ತ್ಕಾ-ರಣ
ಜಗತ್ತ
ಜಗ-ತ್ತ-ನೆಲ್ಲ
ಜಗ-ತ್ತನ್ನು
ಜಗ-ತ್ತನ್ನೆ
ಜಗ-ತ್ತ-ನ್ನೆಲ್ಲ
ಜಗ-ತ್ತ-ನ್ನೆಲ್ಲಾ
ಜಗ-ತ್ತಲ್ಲ
ಜಗ-ತ್ತಾಗಿ
ಜಗ-ತ್ತಾ-ದರೂ
ಜಗ-ತ್ತಿಗೂ
ಜಗ-ತ್ತಿಗೆ
ಜಗ-ತ್ತಿ-ಗೆಲ್ಲ
ಜಗ-ತ್ತಿಗೇ
ಜಗ-ತ್ತಿನ
ಜಗ-ತ್ತಿ-ನಲ್ಲಿ
ಜಗ-ತ್ತಿ-ನ-ಲ್ಲಿ-ರುವ
ಜಗ-ತ್ತಿ-ನ-ಲ್ಲಿ-ಲ್ಲ-ವೆಂದು
ಜಗ-ತ್ತಿ-ನ-ಲ್ಲೆಲ್ಲ
ಜಗತ್ತು
ಜಗತ್ತೂ
ಜಗತ್ತೆ
ಜಗ-ತ್ತೆಲ್ಲ
ಜಗ-ತ್ತೆ-ಲ್ಲ-ವನ್ನೂ
ಜಗ-ತ್ತೆ-ಲ್ಲವೂ
ಜಗ-ತ್ತೆಲ್ಲಾ
ಜಗತ್ತೇ
ಜಗ-ತ್ಸೃ-ಷ್ಟಿ-ಗಿಂ-ತಲೂ
ಜಗ-ತ್ಸೃ-ಷ್ಟಿಯ
ಜಗ-ದಾ-ದಿ-ಮ-ನಾ-ದಿ-ಮಜಂ
ಜಗ-ದಾ-ಧಾರ
ಜಗ-ದೀ-ಶ್ವ-ರನ
ಜಗ-ದೀ-ಶ್ವ-ರ-ನಾಗಿ
ಜಗ-ದೀ-ಶ್ವ-ರ-ನಾದ
ಜಗ-ದೀ-ಶ್ವ-ರನು
ಜಗ-ದೀ-ಶ್ವ-ರನೂ
ಜಗ-ದೀ-ಶ್ವರಾ
ಜಗ-ದ್ಗುರು
ಜಗ-ದ್ಗು-ರು-ಗಳ
ಜಗ-ದ್ರ-ಕ್ಷ-ಕ-ನಾದ
ಜಗ-ದ್ರೂಪಿ
ಜಗ-ನ್ನಾ-ಟ-ಕ-ವನ್ನು
ಜಗ-ನ್ನಾ-ಯಕ
ಜಗಳ
ಜಗ-ಳಕ್ಕೆ
ಜಗ-ಳದ
ಜಗ-ಳ-ವಾ-ಡುತ್ತಾ
ಜಗು-ಲಿ-ಗಳು
ಜಗು-ಲಿ-ಗ-ಳೇನು
ಜಗ್ಗಿತು
ಜಘ-ನ-ಪ್ರ-ದೇ-ಶ-ದಿಂದ
ಜಜಾನ
ಜಜ್ಜಿ
ಜಜ್ಜಿ-ಹೋದ
ಜಟಾಯು
ಜಟೆ
ಜಟೆ-ಯನ್ನು
ಜಟೆ-ಯಲ್ಲಿ
ಜಟೆ-ಯಿಂದ
ಜಟ್ಟಿ
ಜಟ್ಟಿ-ಗಳನ್ನು
ಜಟ್ಟಿ-ಗಳಿಂದ
ಜಟ್ಟಿ-ಗ-ಳಿಗೆ
ಜಟ್ಟಿ-ಗ-ಳಿ-ಬ್ಬ-ರನ್ನು
ಜಟ್ಟಿ-ಗ-ಳೆಲ್ಲ
ಜಟ್ಟಿ-ಗ-ಳೆಲ್ಲಿ
ಜಟ್ಟಿ-ಗಳೇ
ಜಟ್ಟಿಯ
ಜಠರ
ಜಠ-ರಾಗ್ನಿ
ಜಡ
ಜಡ-ದೇಹ
ಜಡ-ದೇ-ಹವೇ
ಜಡ-ನಂತೆ
ಜಡ-ಬು-ದ್ಧಿ-ಯ-ವರು
ಜಡ-ಭ-ರತ
ಜಡ-ಭ-ರ-ತನ
ಜಡ-ಭ-ರ-ತ-ನನ್ನು
ಜಡ-ಭ-ರ-ತ-ನಿಗೆ
ಜಡ-ಭ-ರ-ತನು
ಜಡ-ಭ-ರ-ತ-ಮ-ಹಾ-ಮು-ನಿಯ
ಜಡ-ಭ-ರ-ತ-ಮು-ನಿಗೆ
ಜಡ-ಭ-ರ-ತ-ಮು-ನಿಯ
ಜಡ-ಭ-ರ-ತ-ಮು-ನಿಯು
ಜಡ-ರಾದ
ಜಡ-ವಾಗಿ
ಜಡ-ವಾದ
ಜಡೆ-ಗ-ಟ್ಟಿದ
ಜಡ್ಡು-ಗ-ಟ್ಟಿದ
ಜಢ-ಭ-ರತ
ಜಣ-ಜ-ಣನೆ
ಜನ
ಜನ-ಕನೂ
ಜನ-ಕ-ನೆಂದೂ
ಜನ-ಕ-ರಾ-ಜನ
ಜನಕ್ಕೆ
ಜನ-ಕ್ಕೆಲ್ಲ
ಜನ-ಗಳನ್ನು
ಜನ-ಜ-ನಿ-ತ-ವಾಗಿ
ಜನ-ಜ-ನಿ-ತ-ವಾ-ಗಿದ್ದ
ಜನ-ಜೀ-ವ-ನ-ದಲ್ಲಿ
ಜನ-ಜೀ-ವ-ನ-ವನ್ನು
ಜನನ
ಜನನಂ
ಜನ-ನ-ದೊ-ಡನೆ
ಜನ-ನ-ಪೂ-ರ್ವ-ದಿಂದ
ಜನ-ಪ್ರಿಯಃ
ಜನ-ಪ್ರಿ-ಯ-ವಾ-ಗಿದೆ
ಜನ-ಮ-ನ-ಗಳು
ಜನ-ಮ-ನ-ವನ್ನು
ಜನಮೇ
ಜನ-ಮೇ-ಜಯ
ಜನ-ಮೇ-ಜ-ಯನ
ಜನ-ಮೇ-ಜ-ಯ-ನಿಗೆ
ಜನ-ಮೇ-ಜ-ಯನು
ಜನ-ಮೇ-ಜ-ಯನೇ
ಜನರ
ಜನ-ರನ್ನು
ಜನ-ರನ್ನೂ
ಜನ-ರಲ್ಲಿ
ಜನ-ರಾರೂ
ಜನ-ರಿಂದ
ಜನ-ರಿಗೂ
ಜನ-ರಿಗೆ
ಜನ-ರಿ-ಗೆಲ್ಲ
ಜನರು
ಜನರೂ
ಜನ-ರೆಲ್ಲ
ಜನ-ರೆ-ಲ್ಲರೂ
ಜನ-ರೊ-ಡನೆ
ಜನ-ಸಂ-ಚಾ-ರ-ವಿ-ಲ್ಲದ
ಜನ-ಸಂ-ದಣಿ
ಜನ-ಸಾ-ಮಾ-ನ್ಯ-ರೆ-ಲ್ಲರೂ
ಜನ-ಸೇ-ವ್ಯನೂ
ಜನಾಂ-ಗಕ್ಕೆ
ಜನಾಂ-ಗದ
ಜನಾಂ-ಗ-ದಲ್ಲಿ
ಜನಾಂ-ಗ-ವಾಗಿ
ಜನಾಂ-ಗ-ವಿ-ರು-ವುದೊ
ಜನಾಃ
ಜನಾ-ಪ-ವಾದ
ಜನಾ-ರ್ದನಃ
ಜನಾ-ರ್ದ-ನನು
ಜನಿ-ವಾರ
ಜನಿ-ವಾ-ರ-ಇವು
ಜನಿ-ವಾ-ರ-ದಿಂದ
ಜನಿ-ವಾ-ರ-ಧಾ-ರ-ಣೆಯೆ
ಜನಿ-ವಾ-ರ-ವನ್ನು
ಜನಿ-ಸ-ಬೇ-ಕಾ-ಗು-ತ್ತದೆ
ಜನಿಸಿ
ಜನಿ-ಸಿತು
ಜನಿ-ಸಿ-ತೆಂದು
ಜನಿ-ಸಿದ
ಜನಿ-ಸಿ-ದರು
ಜನಿ-ಸಿ-ದ-ವರು
ಜನಿ-ಸಿ-ದವು
ಜನಿ-ಸಿ-ದೆವು
ಜನಿ-ಸಿ-ರುವ
ಜನಿ-ಸಿ-ರುವೆ
ಜನಿ-ಸು-ತ್ತಾನೆ
ಜನಿ-ಸು-ತ್ತಾರೆ
ಜನಿ-ಸು-ವೆ-ನೆಂದು
ಜನೋ-ಽಚ-ರ್ಯೆತ್
ಜನ್ಮ
ಜನ್ಮ-ಎ-ತ್ತಿ-ರು-ವನೋ
ಜನ್ಮಕ್ಕೂ
ಜನ್ಮಕ್ಕೆ
ಜನ್ಮ-ಗಳ
ಜನ್ಮ-ಗಳನ್ನು
ಜನ್ಮ-ಗ-ಳನ್ನೆ
ಜನ್ಮ-ಗ-ಳ-ಲ್ಲೆಲ್ಲ
ಜನ್ಮ-ಗಳು
ಜನ್ಮ-ಗ-ಳೆತ್ತಿ
ಜನ್ಮ-ಜ-ನ್ಮ-ದ-ಲ್ಲಿಯೂ
ಜನ್ಮ-ಜ-ನ್ಮಾಂ-ತ-ರ-ಗ-ಳ-ಲ್ಲಿ-ಯಾ-ದರೂ
ಜನ್ಮ-ಜ-ನ್ಮಾಂ-ತ-ರ-ಗಳಿಂದ
ಜನ್ಮದ
ಜನ್ಮ-ದಂತೆ
ಜನ್ಮ-ದಲ್ಲಿ
ಜನ್ಮ-ದ-ಲ್ಲಿಯೂ
ಜನ್ಮ-ದ-ಲ್ಲಿಯೇ
ಜನ್ಮ-ಧಾ-ರಣೆ
ಜನ್ಮ-ನ-ಕ್ಷತ್ರ
ಜನ್ಮ-ರ-ಹಿ-ತ-ನಾಗಿ
ಜನ್ಮ-ರ-ಹಿ-ತ-ನಾದ
ಜನ್ಮ-ರಾಶಿ
ಜನ್ಮ-ವನ್ನು
ಜನ್ಮ-ವ-ನ್ನೆಲ್ಲ
ಜನ್ಮ-ವನ್ನೇ
ಜನ್ಮ-ವೆ-ತ್ತ-ಬೇಕು
ಜನ್ಮ-ವೆ-ತ್ತ-ಬೇ-ಕೆಂದು
ಜನ್ಮ-ವೆ-ತ್ತಿದ
ಜನ್ಮ-ವೆ-ತ್ತಿಯೇ
ಜನ್ಮಾಂ
ಜನ್ಮಾಂ-ತ-ರ-ಗಳಲ್ಲಿ
ಜನ್ಮಾಂ-ತ-ರ-ಗ-ಳಾ-ದು-ದನ್ನೂ
ಜಪ
ಜಪ-ತ-ಪ-ಸ್ಸು-ಗಳು
ಜಪ-ದಲ್ಲಿ
ಜಪ-ಮಾ-ಡುತ್ತಾ
ಜಪ-ಸ-ರ-ಗಳನ್ನು
ಜಪ-ಸ-ರ-ವನ್ನು
ಜಪಿ-ಸಲಿ
ಜಪಿಸಿ
ಜಪಿ-ಸಿತು
ಜಪಿ-ಸಿ-ದರೆ
ಜಪಿ-ಸುತ್ತ
ಜಪಿ-ಸುತ್ತಾ
ಜಪಿ-ಸು-ತ್ತಾರೆ
ಜಪಿ-ಸು-ತ್ತಿ-ದ್ದನು
ಜಪಿ-ಸುವ
ಜಪಿ-ಸು-ವನು
ಜಪ್ಪಯ್ಯ
ಜಮ-ದಗ್ನಿ
ಜಮ-ದ-ಗ್ನಿಗೆ
ಜಮ-ದ-ಗ್ನಿಯ
ಜಮ-ದ-ಗ್ನಿಯು
ಜಮೀ-ನನ್ನು
ಜಯ
ಜಯಂ-ತಿ-ಯೆಂಬ
ಜಯ-ಕಾರ
ಜಯ-ಘೋಷ
ಜಯ-ದ್ರ-ಥ-ನನ್ನು
ಜಯನು
ಜಯ-ವಿ-ಜ-ಯರು
ಜಯ-ವೆಂ-ಬುದು
ಜಯ-ಶಾ-ಲಿ-ಯಾಗು
ಜಯ-ಸೇ-ನ-ವಿಂದ
ಜಯಿ-ಸ-ಬೇ-ಕಾ-ದರೆ
ಜಯಿ-ಸ-ಲಾ-ರದ
ಜಯಿಸಿ
ಜಯಿ-ಸಿ-ದರೆ
ಜಯಿ-ಸಿ-ದಾಗ
ಜಯಿ-ಸು-ವ-ವರು
ಜಯ್
ಜರ-ತಾ-ರಿಯ
ಜರಾ
ಜರಾ-ಮ-ರ-ಣ-ಗ-ಳಿಲ್ಲ
ಜರಾ-ಮ-ರ-ಣ-ಗ-ಳಿ-ಲ್ಲದ
ಜರಾ-ಮ-ರ-ಣ-ಗ-ಳೆಂತು
ಜರಾ-ಸಂಧ
ಜರಾ-ಸಂ-ಧನ
ಜರಾ-ಸಂ-ಧ-ನನ್ನು
ಜರಾ-ಸಂ-ಧ-ನಿಗೆ
ಜರಾ-ಸಂ-ಧನು
ಜರಾ-ಸಂ-ಧನೂ
ಜರಾ-ಸಂ-ಧನೆ
ಜರಾ-ಸಂ-ಧ-ನೆಂ-ದರೆ
ಜರಾ-ಸಂ-ಧ-ನೆಂಬ
ಜರಾ-ಸಂ-ಧನೇ
ಜರಾ-ಸಂ-ಧ-ನೊ-ಬ್ಬನು
ಜರಾ-ಸಂ-ಧರ
ಜರಾ-ಸಂ-ಧಾ-ದಿ-ಗ-ಳಿಗೆ
ಜರೆ
ಜರೆ-ಯೆಂಬ
ಜಲ
ಜಲ-ಕುಂ-ಭ-ಗಳು
ಜಲ-ಕೇ-ಳಿ-ಗಾಗಿ
ಜಲ-ಕೇ-ಳಿ-ಯಾ-ಡು-ತ್ತಿ-ರು-ವು-ದ-ರಿಂದ
ಜಲ-ಕ್ರೀ-ಡೆಗೆ
ಜಲ-ಕ್ರೀ-ಡೆ-ಯಾ-ಡ-ಬೇ-ಕೆ-ನಿ-ಸಿತು
ಜಲ-ಕ್ರೀ-ಡೆ-ಯಾ-ಡಿ-ದನು
ಜಲ-ಕ್ರೀ-ಡೆ-ಯಾ-ಡು-ತ್ತಿ-ದ್ದ-ರಂತೆ
ಜಲ-ಚ-ರ-ಗಳು
ಜಲ-ಜಂ-ತು-ಗ-ಳಿಂ-ದಲೂ
ಜಲ-ದಲ್ಲಿ
ಜಲ-ದಿಂದ
ಜಲ-ದೇ-ವ-ತೆ-ಗಳೇ
ಜಲ-ದೇ-ವ-ತೆಯ
ಜಲ-ದ್ವಾರ
ಜಲ-ಪ್ರ-ಳಯ
ಜಲ-ಪ್ರ-ಳ-ಯ-ದಲ್ಲಿ
ಜಲ-ಬಾ-ಧೆ-ಯನ್ನು
ಜಲ-ಬಿಂ-ದು-ವಿ-ನಂ-ತಿದ್ದು
ಜಲ-ಮ-ಧ್ಯ-ದಲ್ಲಿ
ಜಲ-ಮ-ಲ-ಗಳು
ಜಲ-ಮಾರ್ಗ
ಜಲ-ರಾ-ಶಿಯ
ಜಲ-ರಾ-ಶಿ-ಯಿಂದ
ಜಲಾ-ಹಾರ
ಜಲಿ-ಸಿತು
ಜಲೇಷು
ಜಲೋ-ದರ
ಜಲ್ಲೆ-ಯಂತೆ
ಜಲ್ಲೆ-ಯನ್ನು
ಜಹ್ನು-ವೆಂಬ
ಜಾಂಬ-ವಂತ
ಜಾಂಬ-ವಂ-ತನ
ಜಾಂಬ-ವಂ-ತ-ನನ್ನು
ಜಾಂಬ-ವತಿ
ಜಾಂಬ-ವ-ತಿಯ
ಜಾಂಬ-ವ-ತಿ-ಯನ್ನೂ
ಜಾಂಬ-ವ-ತಿ-ಯೊ-ಡನೆ
ಜಾಂಬ-ವ-ತಿ-ಸಾಂಬ
ಜಾಂಬು-ವಂ-ತ-ನಿಗೆ
ಜಾಗ-ರೂ-ಕ-ತೆ-ಯಿಂದ
ಜಾಗ್ರತ್
ಜಾಗ್ರ-ದ-ವ-ಸ್ಥೆ-ಯಲ್ಲಿ
ಜಾಜಿ
ಜಾಜಿಯ
ಜಾಜಿ-ಯಷ್ಟು
ಜಾಣ-ತನ
ಜಾಣೆ
ಜಾಣೆ-ಯಾದ
ಜಾತ-ಕರ್ಮ
ಜಾತ-ಕ-ರ್ಮದ
ಜಾತ-ಕ-ರ್ಮಾ-ದಿ-ಗಳನ್ನು
ಜಾತ-ಕ-ವನ್ನು
ಜಾತಿ
ಜಾತಿ-ಗಳ
ಜಾತಿ-ಗ-ಳಾಗಿ
ಜಾತಿಗೆ
ಜಾತಿಯ
ಜಾತಿ-ಯಲ್ಲಿ
ಜಾತಿಯೆ
ಜಾತಿಯೇ
ಜಾತ್ರೆ-ಯನ್ನು
ಜಾನು-ಭ್ಯಾ-ನ್ನಮಃ
ಜಾಬಾಲಿ
ಜಾಯತೇ
ಜಾರನ
ಜಾರನು
ಜಾರಿ
ಜಾರಿ-ಣಿ-ಯಂತೆ
ಜಾರಿದ
ಜಾರಿ-ಬಿದ್ದ
ಜಾರಿಸಿ
ಜಾರು-ತ್ತಿದೆ
ಜಾರು-ತ್ತಿ-ರುವ
ಜಾರು-ವು-ದ-ನ್ನಾ-ಗಲಿ
ಜಾರೆ-ಯ-ರಾದ
ಜಾಲ-ದಿಂದ
ಜಾಲರಿ
ಜಾಲ-ರಿ-ಗಳಿಂದ
ಜಾವ
ಜಾವದ
ಜಾಸ್ತಿ-ಯಾ-ಯಿತು
ಜಿಂಕೆ
ಜಿಂಕೆ-ಗ-ಳಂತೆ
ಜಿಂಕೆ-ಗಳು
ಜಿಂಕೆ-ಗಳೂ
ಜಿಂಕೆ-ಗ-ಳೆಲ್ಲ
ಜಿಂಕೆಗೆ
ಜಿಂಕೆಯ
ಜಿಂಕೆ-ಯದೇ
ಜಿಂಕೆ-ಯ-ಮರಿ
ಜಿಂಕೆ-ಯ-ಲ್ಲಿಯೇ
ಜಿಂಕೆ-ಯಾಗಿ
ಜಿಗಣೆ
ಜಿಗುಟು
ಜಿತ-ನಿಗೆ
ಜಿತೇಂ-ದ್ರಿಯ
ಜಿತೇಂ-ದ್ರಿ-ಯ-ನಾಗಿ
ಜಿತೇಂ-ದ್ರಿ-ಯ-ನಾ-ಗಿ-ರು-ವುದು
ಜಿತೇಂ-ದ್ರಿ-ಯ-ನಾದ
ಜಿತೇಂ-ದ್ರಿ-ಯರು
ಜಿಹಾಸೆ
ಜಿಹಾ-ಸೆ-ಗ-ಳಿ-ಲ್ಲದೆ
ಜಿಹ್ಮ-ವ್ಯಾ-ಹೃತಂ
ಜೀರ್ಣ-ಮಾಡು
ಜೀರ್ಣ-ಶಕ್ತಿ
ಜೀರ್ಣಿ-ಸು-ವು-ದಿಲ್ಲ
ಜೀವ
ಜೀವಂ-ತ-ನ-ನ್ನಾಗಿ
ಜೀವ-ಇ-ವರ
ಜೀವ-ಕ್ಕಿಂ-ತಲೂ
ಜೀವಕ್ಕೆ
ಜೀವ-ಗಳ
ಜೀವ-ಗಳು
ಜೀವ-ಗಳೂ
ಜೀವ-ಗ-ಳ್ಳ-ನಂತೆ
ಜೀವ-ಗ-ಳ್ಳ-ನಲ್ಲ
ಜೀವ-ಗ-ಳ್ಳ-ನಾದ
ಜೀವ-ಗ-ಳ್ಳ-ರಾದ
ಜೀವ-ಚೈ-ತ-ನ್ಯ-ವನ್ನು
ಜೀವ-ಜೀ-ವಾ-ಳ-ದಂ-ತಿ-ರುವ
ಜೀವದ
ಜೀವ-ದಾನ
ಜೀವ-ದಾ-ನ-ಮಾ-ಡಿ-ದನು
ಜೀವ-ದಾ-ನ-ವನ್ನು
ಜೀವ-ಧಾ-ರ-ಣೆಯೇ
ಜೀವನ
ಜೀವ-ನದ
ಜೀವ-ನ-ದಲ್ಲಿ
ಜೀವ-ನ-ದ-ಲ್ಲಿ-ಯಾ-ಗಲಿ
ಜೀವ-ನ-ದಿ-ಗಳು
ಜೀವ-ನ-ದೊ-ಡನೆ
ಜೀವ-ನನ್ನು
ಜೀವ-ನಲ್ಲಿ
ಜೀವ-ನ-ವನ್ನು
ಜೀವ-ನ-ವಾ-ಯಿತು
ಜೀವ-ನವು
ಜೀವ-ನಾ-ಧಾ-ರಕ್ಕೆ
ಜೀವ-ನಿ-ಗಲ್ಲ
ಜೀವ-ನಿಗೆ
ಜೀವ-ನಿ-ರು-ವುದು
ಜೀವನು
ಜೀವನೇ
ಜೀವ-ನೋ-ಪಾಯ
ಜೀವ-ನ್ಮು-ಕ್ತ-ನೆ-ನಿಸು
ಜೀವ-ಭಯ
ಜೀವ-ಭ-ಯ-ದಿಂದ
ಜೀವ-ಮಾ-ನ-ವೆಲ್ಲ
ಜೀವರ
ಜೀವ-ರಾ-ಶಿ-ಅ-ಖಂಡ
ಜೀವ-ರಾ-ಶಿ-ಗಳ
ಜೀವ-ರಾ-ಶಿ-ಗಳೂ
ಜೀವ-ರಾ-ಶಿ-ಯೆಲ್ಲ
ಜೀವ-ರಾ-ಶಿ-ಯೊ-ಡನೆ
ಜೀವ-ರಿಗೂ
ಜೀವ-ರಿಗೆ
ಜೀವ-ರಿ-ಗೆಲ್ಲ
ಜೀವರು
ಜೀವ-ವನ್ನು
ಜೀವವೇ
ಜೀವ-ವ್ಯಾ-ಪಾ-ರ-ಗ-ಳಿಗೆ
ಜೀವ-ಸ-ರ್ವ-ಸ್ವ-ವಾಗಿ
ಜೀವಾ-ಣು-ವಿ-ನ-ಲ್ಲಿಯೂ
ಜೀವಾತ್ಮ
ಜೀವಾ-ತ್ಮನ
ಜೀವಾ-ತ್ಮ-ನಂ-ತಿ-ರುವ
ಜೀವಾ-ತ್ಮ-ನನ್ನು
ಜೀವಾ-ತ್ಮ-ನಲ್ಲಿ
ಜೀವಾ-ತ್ಮ-ನಾ-ಗಿಯೂ
ಜೀವಾ-ತ್ಮ-ನಿಗೆ
ಜೀವಾ-ತ್ಮನು
ಜೀವಾ-ತ್ಮರ
ಜೀವಿ
ಜೀವಿ-ಗಳ
ಜೀವಿ-ಗಳನ್ನೂ
ಜೀವಿ-ಗಳಲ್ಲಿ
ಜೀವಿ-ಗ-ಳ-ಲ್ಲಿಯೂ
ಜೀವಿ-ಗ-ಳಿಗೆ
ಜೀವಿ-ಗಳು
ಜೀವಿ-ಗಳೂ
ಜೀವಿಗೆ
ಜೀವಿ-ತ-ಕಾ-ಲ-ದಲ್ಲಿ
ಜೀವಿ-ತ-ಕಾ-ಲ-ವೆಲ್ಲ
ಜೀವಿಯ
ಜೀವಿಯು
ಜೀವಿಸ
ಜೀವಿ-ಸ-ಬ-ಲ್ಲೆನೆ
ಜೀವಿ-ಸ-ಲಾ-ರೆ-ವೆಂ-ದು-ಕೊಂಡು
ಜೀವಿ-ಸುತ್ತಾ
ಜೀವಿ-ಸು-ತ್ತಿ-ದ್ದರು
ಜೀವೇಶ
ಜುಗುಪ್ಸೆ
ಜುನನ
ಜುನ-ರಾಗಿ
ಜೂಜಾ-ಟಕ್ಕೆ
ಜೂಜಿ-ನಲ್ಲಿ
ಜೂಜು
ಜೃಂಭ-ಣಾ-ಸ್ತ್ರ-ವನ್ನು
ಜೇಂಕಾ-ರದ
ಜೇಡರ
ಜೇಡ-ರ-ಹು-ಳು-ವನ್ನು
ಜೇನನ್ನು
ಜೇನು
ಜೇನು-ತು-ಪ್ಪ-ಗ-ಳೊ-ಡನೆ
ಜೇನು-ಹು-ಳ-ದಂತೆ
ಜೇನು-ಹು-ಳು-ವನ್ನು
ಜೈ
ಜೊಂಡು
ಜೊಂಡು-ಹು-ಲ್ಲಾಗಿ
ಜೊತೆ
ಜೊತೆ-ಗೂ-ಡಿಸು
ಜೊತೆಗೆ
ಜೊತೆ-ಜೊತೆ
ಜೊತೆ-ಜೊ-ತೆ-ಯಾ-ಗಿಯೆ
ಜೊತೆಯ
ಜೊತೆ-ಯಲ್ಲಿ
ಜೊತೆ-ಯ-ಲ್ಲಿ-ದ್ದ-ವನು
ಜೊತೆ-ಯ-ಲ್ಲಿಯೆ
ಜೊತೆ-ಯ-ಲ್ಲಿಯೇ
ಜೊತೆ-ಯ-ಲ್ಲಿ-ರು-ತ್ತಿದ್ದ
ಜೊತೆ-ಯಲ್ಲೊ
ಜೊತೆ-ಯ-ವರೆಲ್ಲ
ಜೊತೆ-ಯಾ-ಗಿದ್ದು
ಜೋಕಾಲಿ
ಜೋಕೆ
ಜೋಗುಳ
ಜೋಡಿ
ಜೋಡಿ-ಗ-ಳಿಗೆ
ಜೋಡಿ-ಗಳು
ಜೋಡಿ-ಗಳೂ
ಜೋಡಿ-ಯಾಗಿ
ಜೋಡಿ-ಯಾ-ಗು-ವು-ದೆಂದು
ಜೋಡಿಯು
ಜೋಡಿಸಿ
ಜೋಡಿ-ಸಿ-ಕೊಂ-ಡಿ-ದ್ದನು
ಜೋಡಿ-ಸಿ-ಕೊಂಡು
ಜೋಡಿ-ಸಿ-ದರೂ
ಜೋಡು
ಜೋಡು-ಹ-ಕ್ಕಿ-ಗಳು
ಜೋತು-ಬಿ-ದ್ದರೆ
ಜೋತು-ಬೀ-ಳು-ವರು
ಜೋಲು-ನಾ-ಲಗೆ
ಜ್ಞಾತೃ
ಜ್ಞಾನ
ಜ್ಞಾನ-ಶುದ್ಧ
ಜ್ಞಾನಂ-ದಿಂ-ದಲೇ
ಜ್ಞಾನ-ಕ್ಕಾಗಿ
ಜ್ಞಾನ-ಗ-ಳೆಂಬ
ಜ್ಞಾನ-ಜ್ಯೋ-ತಿ-ಯಂ-ತಿ-ದ್ದವು
ಜ್ಞಾನದ
ಜ್ಞಾನ-ದಲ್ಲಿ
ಜ್ಞಾನ-ದಾಹ
ಜ್ಞಾನ-ದಾ-ಹ-ವನ್ನು
ಜ್ಞಾನ-ದಿಂದ
ಜ್ಞಾನ-ದಿಂ-ದಲೆ
ಜ್ಞಾನ-ದೃಷ್ಟಿ
ಜ್ಞಾನ-ದೃ-ಷ್ಟಿಗೆ
ಜ್ಞಾನ-ನಿ-ಧಿ-ಯಾದ
ಜ್ಞಾನ-ಬೋ-ಧೆ-ಯಾ-ಗು-ತ್ತದೆ
ಜ್ಞಾನ-ಮ-ಯನು
ಜ್ಞಾನ-ಮಾ-ರ್ಗ-ಗಳ
ಜ್ಞಾನ-ಮಾ-ರ್ಗ-ಗಳು
ಜ್ಞಾನ-ಮಾ-ರ್ಗ-ನಿ-ರೂ-ಪಣೆ
ಜ್ಞಾನ-ಮೂ-ಡು-ತ್ತದೆ
ಜ್ಞಾನ-ಯೋ-ಗದ
ಜ್ಞಾನ-ಯೋ-ಗ-ದಿಂದ
ಜ್ಞಾನ-ಯೋ-ಗ-ವನ್ನು
ಜ್ಞಾನ-ರೂ-ಪ-ದಿಂದ
ಜ್ಞಾನ-ರೂ-ಪ-ನಾಗಿ
ಜ್ಞಾನ-ವನ್ನು
ಜ್ಞಾನ-ವನ್ನೂ
ಜ್ಞಾನ-ವಿ-ಕಾ-ರ-ಗಳನ್ನು
ಜ್ಞಾನ-ವಿ-ಜ್ಞಾ-ನ-ಸಾರಂ
ಜ್ಞಾನ-ವಿ-ರು-ವು-ದ-ರಿಂದ
ಜ್ಞಾನ-ವಿಲ್ಲ
ಜ್ಞಾನವು
ಜ್ಞಾನ-ವುಂ-ಟಾ-ಗು-ತ್ತದೆ
ಜ್ಞಾನ-ವು-ಳ್ಳ-ವು-ಗ-ಳಲ್ಲ
ಜ್ಞಾನವೂ
ಜ್ಞಾನ-ವೃ-ದ್ಧ-ರನ್ನೂ
ಜ್ಞಾನ-ವೆಂಬ
ಜ್ಞಾನ-ವೆಲ್ಲ
ಜ್ಞಾನ-ವೆ-ಲ್ಲವೂ
ಜ್ಞಾನವೇ
ಜ್ಞಾನ-ವೈ-ರಾ-ಗ್ಯ-ಗಳಿಂದ
ಜ್ಞಾನ-ಶಾ-ಸ್ತ್ರ-ವನ್ನು
ಜ್ಞಾನ-ಸಂ-ಪತ್ತು
ಜ್ಞಾನ-ಸ್ವ-ರೂ-ಪ-ನಾ-ಗಿಯೂ
ಜ್ಞಾನಾ-ಪೇ-ಕ್ಷಿ-ಗ-ಳಿಗೆ
ಜ್ಞಾನಾ-ಮೃ-ತ-ವನ್ನು
ಜ್ಞಾನಿ
ಜ್ಞಾನಿ-ಗಳ
ಜ್ಞಾನಿ-ಗ-ಳಾ-ಗಿ-ದ್ದರೆ
ಜ್ಞಾನಿ-ಗ-ಳಾದ
ಜ್ಞಾನಿ-ಗ-ಳಾ-ದ-ವರು
ಜ್ಞಾನಿ-ಗ-ಳಿಗೆ
ಜ್ಞಾನಿ-ಗಳು
ಜ್ಞಾನಿ-ಗಳೂ
ಜ್ಞಾನಿ-ಗ-ಳೆಂದು
ಜ್ಞಾನಿಗೆ
ಜ್ಞಾನಿ-ಯಾಗಿ
ಜ್ಞಾನಿ-ಯಾ-ಗಿದ್ದ
ಜ್ಞಾನಿ-ಯಾ-ಗಿ-ದ್ದನು
ಜ್ಞಾನಿ-ಯಾದ
ಜ್ಞಾನಿ-ಯಾ-ದ-ವನು
ಜ್ಞಾನಿಯೂ
ಜ್ಞಾನಿ-ಯೆ-ನಿ-ಸಿ-ಕೊಂ-ಡಿ-ರುವ
ಜ್ಞಾನೇಂ
ಜ್ಞಾನೇಂ-ದ್ರಿಯ
ಜ್ಞಾನೇಂ-ದ್ರಿ-ಯ-ಗಳ
ಜ್ಞಾನೇಂ-ದ್ರಿ-ಯ-ಗಳನ್ನೂ
ಜ್ಞಾನೇಂ-ದ್ರಿ-ಯ-ಗಳು
ಜ್ಞಾನೇಂ-ದ್ರಿ-ಯ-ಗಳೂ
ಜ್ಞಾನೋ-ದ-ಯ-ವಾ-ಯಿತು
ಜ್ಞಾನೋ-ಪ-ದೇಶ
ಜ್ಞಾನೋ-ಪ-ದೇ-ಶಕ್ಕೆ
ಜ್ಞಾನೋ-ಪ-ದೇ-ಶ-ಮಾ-ಡಿ-ದನು
ಜ್ಞಾನೋ-ಪ-ದೇ-ಶ-ಮಾ-ಡಿ-ದ-ವನು
ಜ್ಞಾನೋ-ಪ-ದೇ-ಶ-ವನ್ನು
ಜ್ಞಾನೋ-ಪ-ದೇ-ಶ-ವೆಂಬ
ಜ್ಞಾಪಕ
ಜ್ಞಾಪ-ಕಕ್ಕೆ
ಜ್ಞಾಪ-ಕ-ದ-ಲ್ಲಿ-ಟ್ಟು-ಕೊಂ-ಡಿರು
ಜ್ಞಾಪ-ಕ-ವಾ-ಯಿತು
ಜ್ಞಾಪ-ಕ-ವಿ-ದೆಯೆ
ಜ್ಞಾಪ-ಕ-ವಿ-ರ-ಬಾ-ರದು
ಜ್ಞಾಪ-ಕ-ವಿ-ರು-ತ್ತದೆ
ಜ್ಞಾಪ-ಕ-ಶ-ಕ್ತಿ-ಗಳು
ಜ್ಞಾಪಿಸಿ
ಜ್ಞಾಪಿ-ಸಿಕೊ
ಜ್ಞಾಪಿ-ಸಿ-ಕೊಂಡು
ಜ್ಞಾಪಿ-ಸಿ-ಕೊ-ಳ್ಳು-ತ್ತಾ-ನೆಯೊ
ಜ್ಞಾಪಿ-ಸಿ-ಕೊ-ಳ್ಳು-ತ್ತಾನೋ
ಜ್ಞಾಪಿ-ಸಿ-ದನು
ಜ್ಞಾಪಿ-ಸಿ-ದಳು
ಜ್ಞಾಪಿ-ಸು-ವು-ದ-ಕ್ಕಾ-ಗಿಯೇ
ಜ್ಞೇಯ
ಜ್ಯೋತಿ
ಜ್ಯೋತಿ-ಕಿ-ರಣ
ಜ್ಯೋತಿ-ಯಲ್ಲಿ
ಜ್ಯೋತಿರ್ಲೋ
ಜ್ಯೋತಿ-ಶಾ-ಸ್ತ್ರದ
ಜ್ವರ
ಜ್ವರ-ವನ್ನು
ಜ್ವರವು
ಝಗ
ಝಗಿ-ಸುವ
ಝಣಕ್
ಝಣ-ಝಣ
ಝಣ-ಝ-ಣ-ರೆ-ನಿ-ಸುತ್ತಾ
ಝಣ-ಝ-ಣ-ರೆ-ನು-ವಂತೆ
ಝಲ್ಲೆಂ-ದಿತು
ಝಾಡಿಸಿ
ಝಾಡಿ-ಸಿತು
ಝಾಡಿ-ಸು-ತ್ತಿ-ರುವ
ಝಾವಕ್ಕೆ
ಝಾವ-ದ-ಲ್ಲಿಯೂ
ಟಗ-ರನ್ನು
ಟಗ-ರಿತ್ತು
ಟಗ-ರಿನ
ಟಗರು
ಟಳು
ಟಿತು
ಟೀಕಿ-ಸು-ವ-ವ-ರಿಗೂ
ಟೀಕೆ
ಟೀಕೆಗೆ
ಟ್ಟವಳು
ಠೇಂಕಾರ
ಠೊಣಪ
ಡಂಗುರ
ಡಮ-ರು-ಗಳನ್ನು
ಡಲಿ-ಎಂಬ
ಡಲು
ಡಾಬನ್ನು
ಡುತ್ತಾ
ಡುವ
ಡೇಯನು
ಣಂ
ಣಖಿ-ಯನ್ನು
ಣಖಿಯು
ಣಮಿ-ಸು-ವು-ದಿ-ಲ್ಲ-ವಾ-ದ್ದ-ರಿಂದ
ಣಾಂ
ತ
ತಂ
ತಂಗಾಳಿ
ತಂಗಾ-ಳಿಯು
ತಂಗಿ
ತಂಗಿ-ಗಾಗಿ
ತಂಗಿದ್ದು
ತಂಗಿಯ
ತಂಗಿ-ಯನ್ನು
ತಂಗಿ-ಯ-ನ್ನು-ಕೊ-ಲ್ಲು-ವು-ದೆಂದ
ತಂಗಿ-ಯ-ರಾಗಿ
ತಂಗಿ-ಯಾದ
ತಂಗಿ-ಯಿ-ದ್ದಳು
ತಂಡ
ತಂಡ-ತಂ-ಡ-ವಾಗಿ
ತಂಡ-ದೊ-ಡನೆ
ತಂತಿ
ತಂತಿ-ಯಂ-ತಿ-ರುವ
ತಂತಿ-ಯಂತೆ
ತಂತು-ವಿ-ನಿಂದ
ತಂತ್ರ
ತಂತ್ರ-ಸ್ವ-ತಂ-ತ್ರ-ನಾದ
ತಂದ
ತಂದಂತೆ
ತಂದನು
ತಂದರು
ತಂದ-ವನು
ತಂದಿಟ್ಟ
ತಂದಿ-ಟ್ಟು-ಹೋದ
ತಂದಿ-ತ್ತನು
ತಂದಿ-ತ್ತರು
ತಂದಿದ್ದ
ತಂದಿ-ದ್ದರೂ
ತಂದಿ-ರ-ಲಿಲ್ಲ
ತಂದಿರಿ
ತಂದಿ-ರುವ
ತಂದು
ತಂದು-ಕೊಂ-ಡನು
ತಂದು-ಕೊಂ-ಡಳು
ತಂದು-ಕೊಂಡು
ತಂದು-ಕೊಂ-ಡು-ದ-ಕ್ಕಾಗಿ
ತಂದು-ಕೊ-ಟ್ಟ-ವನು
ತಂದು-ಕೊಟ್ಟು
ತಂದು-ಕೊ-ಡ-ಬೇ-ಕೆಂ-ಬು-ದನ್ನು
ತಂದು-ಕೊಡಿ
ತಂದು-ಕೊ-ಡು-ತ್ತದೆ
ತಂದು-ಕೊ-ಡುತ್ತಾ
ತಂದು-ಕೊ-ಡು-ತ್ತೇನೆ
ತಂದು-ಕೊ-ಳ್ಳ-ಬಾ-ರದು
ತಂದು-ಕೊ-ಳ್ಳು-ತ್ತದೆ
ತಂದು-ದನ್ನು
ತಂದು-ಹಾ-ಕು-ವು-ದ-ಕ್ಕಾಗಿ
ತಂದೆ
ತಂದೆ-ಮಗ
ತಂದೆ-ಗಳನ್ನು
ತಂದೆ-ಗ-ಳಿದ್ದ
ತಂದೆ-ಗಾದ
ತಂದೆಗೆ
ತಂದೆ-ಗೊ-ಪ್ಪಿಸಿ
ತಂದೆ-ತಾ-ಯಿ-ಗಳ
ತಂದೆ-ತಾ-ಯಿ-ಗ-ಳಾರೊ
ತಂದೆ-ತಾ-ಯಿ-ಗ-ಳಿಗೆ
ತಂದೆಯ
ತಂದೆ-ಯಂ-ತೆಯೆ
ತಂದೆ-ಯಂ-ತೆಯೇ
ತಂದೆ-ಯನ್ನು
ತಂದೆ-ಯನ್ನೆ
ತಂದೆ-ಯಲ್ಲಿ
ತಂದೆ-ಯಷ್ಟೆ
ತಂದೆ-ಯಾದ
ತಂದೆ-ಯಾ-ದನು
ತಂದೆ-ಯಿಂದ
ತಂದೆಯು
ತಂದೆಯೂ
ತಂದೆಯೆ
ತಂದೆ-ಯೆ-ನಿ-ಸಿ-ಕೊಂಡ
ತಂದೆ-ಯೊ-ಡನೆ
ತಂದೊ-ಡ್ಡಿ-ದಳು
ತಂದೊ-ಪ್ಪಿ-ಸಿ-ದನು
ತಂದೊ-ಪ್ಪಿ-ಸು-ತ್ತಾರೆ
ತಂಪನ್ನು
ತಂಪನ್ನೂ
ತಂಪಾ-ಗಲಿ
ತಂಪಾಗಿ
ತಂಪಾದ
ತಂಪಾ-ಯಿತು
ತಂಪು
ತಂಪು-ಬೆ-ಳಕು
ತಂಪೂ
ತಂಬ
ತಂಬಿಗೆ
ತಂಬಿ-ಗೆ-ಯನ್ನು
ತಂಬು-ಲಕ್ಕೆ
ತಕ-ತಕ
ತಕ್ಕ
ತಕ್ಕಂತೆ
ತಕ್ಕಂ-ತೆಯೆ
ತಕ್ಕ-ರೀ-ತಿ-ಯಲ್ಲಿ
ತಕ್ಕ-ವರು
ತಕ್ಕ-ವ-ಳಾದ
ತಕ್ಕ-ವ-ಳಾ-ದೇನು
ತಕ್ಕ-ವಳೆ
ತಕ್ಕು-ದಾ-ಗಿದ್ದು
ತಕ್ಕು-ದಾದ
ತಕ್ಕುದು
ತಕ್ಕುದೆ
ತಕ್ಕು-ದೆಂದು
ತಕ್ಕೈ-ಸ-ಬ-ಹು-ದಾದ
ತಕ್ಷಕ
ತಕ್ಷ-ಕನು
ತಕ್ಷ-ಕನೆ
ತಕ್ಷ-ಕ-ನೇ-ಕಿನ್ನೂ
ತಕ್ಷ-ಕ-ನೊ-ಡನೆ
ತಕ್ಷ-ಕ-ವಿ-ಷಾ-ನ-ಲ-ಭೀ-ತ-ನಾದ
ತಕ್ಷ-ಕಾಯ
ತಕ್ಷಣ
ತಕ್ಷ-ಣವೆ
ತಕ್ಷ-ಣವೇ
ತಗಲಿ
ತಗುಲಿ
ತಗೋ
ತಗ್ಗದೆ
ತಗ್ಗ-ಲಿಲ್ಲ
ತಗ್ಗಿ
ತಗ್ಗಿತು
ತಗ್ಗಿ-ದಂ-ತಾಗಿ
ತಗ್ಗಿಸಿ
ತಗ್ಗಿ-ಸಿತು
ತಗ್ಗಿ-ಸಿದಂ
ತಗ್ಗಿ-ಸಿ-ದನು
ತಗ್ಗಿ-ಸಿ-ದಳು
ತಗ್ಗಿ-ಸುವ
ತಗ್ಗು-ತ್ತದೆ
ತಗ್ಗು-ತ್ತಿದೆ
ತಟ-ದಲ್ಲಿ
ತಟ-ದ-ಲ್ಲಿರು
ತಟ-ಸ್ಥ-ನಾ-ಗು-ವಂತೆ
ತಟ್ಟನೆ
ತಟ್ಟಿ
ತಟ್ಟಿ-ಕೊಂಡು
ತಟ್ಟಿ-ದನು
ತಟ್ಟು-ತ್ತದೆ
ತಟ್ಟುತ್ತಾ
ತಟ್ಟು-ವನು
ತಟ್ಟೆ
ತಟ್ಟೆಯ
ತಟ್ಟೆ-ಯಲ್ಲಿ
ತಟ್ಟೆ-ಯೊ-ಡನೆ
ತಡ
ತಡ-ಮಾ-ಡಿ-ದು-ದ-ಕ್ಕಾಗಿ
ತಡ-ವ-ರಿ-ಸಿ-ಕೊಂಡು
ತಡ-ವಾ-ದೀ-ತೆಂದು
ತಡಿಗೆ
ತಡಿ-ಯಲ್ಲಿ
ತಡೆ
ತಡೆದ
ತಡೆ-ದರು
ತಡೆ-ದರೂ
ತಡೆ-ದಳು
ತಡೆದು
ತಡೆ-ದು-ಕೊಂಡು
ತಡೆ-ದು-ಕೊ-ಳ್ಳ-ಲಾ-ರದೆ
ತಡೆಯ
ತಡೆ-ಯ-ದಷ್ಟು
ತಡೆ-ಯ-ಬಲ್ಲ
ತಡೆ-ಯ-ಬೇ-ಕಾ-ದುದು
ತಡೆ-ಯ-ಬೇಕು
ತಡೆ-ಯ-ಲಾ-ರದ
ತಡೆ-ಯ-ಲಾ-ರ-ದಷ್ಟು
ತಡೆ-ಯ-ಲಾ-ರ-ದಾ-ದು-ದ-ರಿಂದ
ತಡೆ-ಯ-ಲಾ-ರದೆ
ತಡೆ-ಯ-ಲಾ-ರೆ-ಯಾ-ದರೆ
ತಡೆ-ಯಲು
ತಡೆ-ಯ-ಹೋದ
ತಡೆ-ಯಿ-ಲ್ಲದೆ
ತಡೆ-ಯುಂಟೆ
ತಡೆ-ಯು-ತ್ತಿ-ರು-ವಿರಿ
ತಡೆ-ಯುವ
ತಡೆ-ಯು-ವ-ರಾರು
ತಡೆ-ಯು-ವ-ವರೆ
ತಡೆ-ಯು-ವ-ಷ್ಟ-ರಲ್ಲಿ
ತಡೆ-ಯು-ವು-ದಕ್ಕೆ
ತಡೆ-ಯು-ವುದು
ತಣಿದು
ತಣಿ-ದು-ಹೋ-ಗಿದೆ
ತಣಿ-ಯು-ತ್ತವೆ
ತಣಿಸಿ
ತಣಿ-ಸಿದ
ತಣಿ-ಸಿ-ದನು
ತಣಿ-ಸುತ್ತಾ
ತಣಿ-ಸು-ವು-ದ-ಕ್ಕಾ-ಗಿಯೇ
ತಣ್ಣ-ಗಾ-ಗಿದೆ
ತಣ್ಣ-ಗಿ-ರುವ
ತಣ್ಣಗೆ
ತಣ್ಣಗೋ
ತಣ್ಣ-ನೆಯ
ತತ್
ತತ್ಕಾ-ಲಕ್ಕೆ
ತತ್ಕ್ಷ-ಣವೇ
ತತ್ತರಿ
ತತ್ತ-ರಿಸಿ
ತತ್ತ-ರಿ-ಸಿತು
ತತ್ತ-ರಿ-ಸಿ-ದರು
ತತ್ತ-ರಿ-ಸಿ-ಹೋ-ದನು
ತತ್ತ-ರಿಸು
ತತ್ತ-ರಿ-ಸು-ತ್ತಿ-ದ್ದಾನೆ
ತತ್ಪರ
ತತ್ಪ-ರ-ನಾಗು
ತತ್ಪ-ರ-ನಾ-ಗು-ವ-ವನು
ತತ್ಪ-ರ-ನಾ-ದನು
ತತ್ಪ-ರ-ರಾಗಿ
ತತ್ಪ-ರ-ರಾ-ಗಿ-ಬೇಕು
ತತ್ಪ-ರರು
ತತ್ಪ-ರ-ಳಾಗಿ
ತತ್ಪ-ರ-ಳಾ-ದಳು
ತತ್ಪ-ರ-ವಾ-ಗಿದೆ
ತತ್ಪಾ-ದ-ಪದ್ಮಂ
ತತ್ಪ್ರ-ಸಂಗಃ
ತತ್ವ
ತತ್ವ-ಗಳ
ತತ್ವ-ಗಳನ್ನು
ತತ್ವ-ಗ-ಳಾ-ಗಿವೆ
ತತ್ವ-ಗಳೂ
ತತ್ವ-ಚಿಂ-ತ-ನೆಗೆ
ತತ್ವಜ್ಞ
ತತ್ವ-ಜ್ಞಾನ
ತತ್ವ-ಜ್ಞಾ-ನ-ವನ್ನು
ತತ್ವ-ಜ್ಞಾ-ನ-ವಾ-ವುದೊ
ತತ್ವ-ಜ್ಞಾ-ನಿ-ಗಳಲ್ಲಿ
ತತ್ವದ
ತತ್ವ-ದಿಂದ
ತತ್ವ-ರೂ-ಪಿ-ಯಾಗಿ
ತತ್ವ-ವನ್ನು
ತತ್ವ-ವಲ್ಲ
ತತ್ವ-ವಾಗಿ
ತತ್ವ-ವಿದೆ
ತತ್ವವು
ತತ್ವವೂ
ತತ್ವ-ವೊಂ-ದರ
ತತ್ವ-ವೊಂದು
ತತ್ವ-ಶಾ-ಸ್ತ್ರಕ್ಕೆ
ತತ್ವ-ಸ-ಮು-ದಾ-ಯ-ರೂ-ಪ-ವಾದ
ತತ್ವೋ-ಪ-ದೇಶ
ತತ್ವೋ-ಪ-ದೇ-ಶ-ಮಾ-ಡಿ-ದನು
ತತ್ವೋ-ಪ-ದೇ-ಶ-ವನ್ನು
ತಥಾಸ್ತು
ತಥಾ-ಸ್ತು-ಎಂದು
ತದ-ಲ-ಮ-ಸಿ-ತ-ಸ-ಖ್ಯೈ-ರ್ದು-ಸ್ತ್ಯ-ಜ-ಸ್ತ-ತ್ಕ-ಥಾರ್ಥಃ
ತದೇ-ಕ-ಧ್ಯಾ-ನ-ದಿಂದ
ತದ್ದ್ರ-ಷ್ಟ್ರ-ಪ-ದೇ-ಶ-ಮೇತಿ
ತನಕ
ತನ-ಗಾಗಿ
ತನ-ಗಾದ
ತನ-ಗಿಂ-ತಲೂ
ತನ-ಗಿದ್ದ
ತನ-ಗಿನ್ನು
ತನಗೂ
ತನಗೆ
ತನ-ಗೆಂ-ತಹ
ತನಗೇ
ತನ-ಗೇನೂ
ತನ-ಗೊ-ಪ್ಪಿ-ಸಿದ
ತನವೆ
ತನೆಗೆ
ತನ್ನ
ತನ್ನಂತೆ
ತನ್ನಂ-ತೆಯೆ
ತನ್ನಂ-ತೆಯೇ
ತನ್ನ-ದ-ಕ್ಕಿಂ-ತಲೂ
ತನ್ನ-ದಾಗಿ
ತನ್ನದು
ತನ್ನದೆ
ತನ್ನ-ದೆಂದ
ತನ್ನ-ದೆಂದು
ತನ್ನ-ದೆಂ-ದು-ಕೊ-ಳ್ಳು-ವರು
ತನ್ನದೇ
ತನ್ನನ್ನು
ತನ್ನನ್ನೂ
ತನ್ನನ್ನೆ
ತನ್ನನ್ನೇ
ತನ್ನಲ್ಲಿ
ತನ್ನ-ಲ್ಲಿಗೆ
ತನ್ನ-ಲ್ಲಿಟ್ಟು
ತನ್ನ-ಲ್ಲಿದ್ದ
ತನ್ನ-ಲ್ಲಿಯೆ
ತನ್ನ-ಲ್ಲಿಯೇ
ತನ್ನ-ವ-ರಾ-ದರೂ
ತನ್ನ-ವ-ರಿಂದ
ತನ್ನ-ವ-ರಿ-ಗಾದ
ತನ್ನ-ವರೆ
ತನ್ನ-ವರೆಲ್ಲ
ತನ್ನ-ವ-ರೊ-ಡನೆ
ತನ್ನಿ
ತನ್ನಿಂದ
ತನ್ನಿಂ-ದಾದ
ತನ್ನಿ-ಬ್ಬರು
ತನ್ನೂ-ರಿಗೆ
ತನ್ನೆ-ದು-ರಿಗೆ
ತನ್ನೆ-ರಡು
ತನ್ನೊ-ಡನೆ
ತನ್ನೊ-ಳಕ್ಕೆ
ತನ್ನೊ-ಳಗೆ
ತನ್ಮಾ-ತ್ರ-ಗಳೂ
ತಪ
ತಪಃ-ಪ್ರ-ಭಾ-ವ-ಗ-ಳಿಗೆ
ತಪ-ಗಳನ್ನು
ತಪ-ತ-ಪನೆ
ತಪ-ವ-ನ್ನಾ-ಚ-ರಿ-ಸಿದ
ತಪ-ಶ್ಶಕ್ತಿ-ಯೇನು
ತಪ-ಸ್ವಿಯ
ತಪ-ಸ್ವಿ-ಯಾದ
ತಪ-ಸ್ಸನ್ನಾ
ತಪ-ಸ್ಸ-ನ್ನಾ-ಚ-ರಿಸಿ
ತಪ-ಸ್ಸ-ನ್ನಾ-ಚ-ರಿಸು
ತಪ-ಸ್ಸ-ನ್ನಾ-ಚ-ರಿ-ಸುತ್ತಾ
ತಪ-ಸ್ಸನ್ನು
ತಪಸ್ಸಿ
ತಪ-ಸ್ಸಿ-ಗಾಗಿ
ತಪ-ಸ್ಸಿಗೆ
ತಪ-ಸ್ಸಿ-ಗೆಂದು
ತಪ-ಸ್ಸಿ-ದ್ಧಿ-ಯನ್ನು
ತಪ-ಸ್ಸಿನ
ತಪ-ಸ್ಸಿ-ನಲ್ಲಿ
ತಪ-ಸ್ಸಿ-ನಿಂದ
ತಪ-ಸ್ಸಿ-ನಿಂ-ದಲೆ
ತಪಸ್ಸು
ತಪ-ಸ್ಸು-ಗಳ
ತಪ-ಸ್ಸು-ಗಳಲ್ಲಿ
ತಪ-ಸ್ಸು-ಗ-ಳಿ-ಗಿಂತ
ತಪ-ಸ್ಸು-ಗಳು
ತಪ-ಸ್ಸು-ಗ-ಳೆಂಬ
ತಪ-ಸ್ಸು-ಮಾಡಿ
ತಪ-ಸ್ಸು-ಮಾ-ಡಿ-ದನು
ತಪ-ಸ್ಸು-ಮಾ-ಡು-ತ್ತಿ-ದ್ದಾನೆ
ತಪಸ್ಸೆ
ತಪ-ಸ್ಸೆಲ್ಲ
ತಪಸ್ಸೇ
ತಪೋ
ತಪೋ-ಜ್ವಾಲೆ
ತಪೋ-ಜ್ವಾ-ಲೆ-ಯಿಂದ
ತಪೋ-ನಿ-ರತ
ತಪೋ-ನಿ-ರ-ತ-ನಾಗಿ
ತಪೋ-ನಿ-ರ-ತ-ನಾದ
ತಪೋ-ನಿ-ರ-ತ-ನಾ-ದನು
ತಪೋ-ನಿ-ರ-ತ-ರಾ-ದರು
ತಪೋ-ನಿ-ರ-ತ-ಳಾ-ದಳು
ತಪೋ-ಫ-ಲ-ವನ್ನು
ತಪೋ-ಮ-ಗ್ನ-ನಾಗಿ
ತಪೋ-ಮ-ಗ್ನ-ನಾ-ಗಿ-ದ್ದನು
ತಪೋ-ಮ-ಗ್ನ-ನಾ-ಗಿ-ರಲು
ತಪೋ-ಮ-ಗ್ನ-ರಾ-ದರು
ತಪೋ-ಮ-ಹಿ-ಮೆ-ಯಿಂದ
ತಪೋ-ವ-ನಕ್ಕೆ
ತಪ್ತ
ತಪ್ತ-ಜೀ-ವನಂ
ತಪ್ಪ-ದಿ-ರಲಿ
ತಪ್ಪದು
ತಪ್ಪದೆ
ತಪ್ಪ-ದೆಂ-ದು-ಕೊಂಡ
ತಪ್ಪನ್ನು
ತಪ್ಪನ್ನೂ
ತಪ್ಪ-ಲಲ್ಲಿ
ತಪ್ಪ-ಲಾರೆ
ತಪ್ಪ-ಲಿ-ನ-ಲ್ಲಿದ್ದ
ತಪ್ಪ-ಲಿ-ನ-ಲ್ಲಿ-ರುವ
ತಪ್ಪ-ಲಿಲ್ಲ
ತಪ್ಪಾ-ಯಿ-ತಲ್ಲ
ತಪ್ಪಿ
ತಪ್ಪಿ-ಗಾಗಿ
ತಪ್ಪಿಗೆ
ತಪ್ಪಿತು
ತಪ್ಪಿದ
ತಪ್ಪಿ-ದರೆ
ತಪ್ಪಿ-ದ-ವನು
ತಪ್ಪಿದು
ತಪ್ಪಿ-ದು-ದಲ್ಲ
ತಪ್ಪಿ-ದುದು
ತಪ್ಪಿ-ಸ-ಬ-ಲ್ಲೆ-ಯಾ-ದರೂ
ತಪ್ಪಿ-ಸ-ಬೇ-ಕಾ-ದರೆ
ತಪ್ಪಿ-ಸ-ಲಾರೆ
ತಪ್ಪಿಸಿ
ತಪ್ಪಿ-ಸಿ-ಕೊಂಡ
ತಪ್ಪಿ-ಸಿ-ಕೊಂಡು
ತಪ್ಪಿ-ಸಿ-ಕೊ-ಳ್ಳ-ದಂತೆ
ತಪ್ಪಿ-ಸಿ-ಕೊ-ಳ್ಳ-ಬ-ಲ್ಲನೆ
ತಪ್ಪಿ-ಸಿ-ಕೊ-ಳ್ಳ-ಬೇ-ಕಾ-ದರೆ
ತಪ್ಪಿ-ಸಿ-ಕೊ-ಳ್ಳ-ಬೇ-ಕೆಂಬ
ತಪ್ಪಿ-ಸಿ-ಕೊ-ಳ್ಳಲು
ತಪ್ಪಿ-ಸಿ-ಕೊ-ಳ್ಳು-ತ್ತಾರೆ
ತಪ್ಪಿ-ಸಿ-ಕೊ-ಳ್ಳು-ವುದ
ತಪ್ಪಿ-ಸಿ-ಕೊ-ಳ್ಳು-ವು-ದ-ಕ್ಕಾಗಿ
ತಪ್ಪಿ-ಸಿ-ಕೊ-ಳ್ಳು-ವು-ದಕ್ಕೆ
ತಪ್ಪಿ-ಸಿ-ಕೊ-ಳ್ಳು-ವುದು
ತಪ್ಪಿ-ಸಿ-ದನು
ತಪ್ಪಿ-ಸಿ-ದಳು
ತಪ್ಪಿ-ಸೋ-ಣ-ವೆಂದು
ತಪ್ಪಿ-ಹೋ-ಯಿತು
ತಪ್ಪು
ತಪ್ಪು-ಗಳನ್ನು
ತಪ್ಪು-ತ್ತದೆ
ತಪ್ಪು-ತ್ತ-ವೆ-ಯಂತೆ
ತಪ್ಪು-ಮಾ-ಡಿ-ದರೂ
ತಪ್ಪು-ಮಾ-ಡಿ-ದೆಯಾ
ತಪ್ಪು-ಮಾ-ಡಿ-ದ್ದಾನೆ
ತಪ್ಪು-ಮಾ-ಡು-ತ್ತಿ-ರು-ವಿರಿ
ತಪ್ಪು-ವಂ-ತಿಲ್ಲ
ತಪ್ಪು-ವ-ರೆಂದು
ತಪ್ಪು-ವು-ದಿಲ್ಲ
ತಪ್ಪು-ವು-ದೆಂತು
ತಪ್ಪು-ವುದೊ
ತಪ್ಪು-ವೆಯಾ
ತಪ್ಪೆ
ತಪ್ಪೇನೂ
ತಬ್ಬಲಿ
ತಬ್ಬ-ಲಿ-ಗ-ಳಾದ
ತಬ್ಬ-ಲಿ-ಯಾದ
ತಬ್ಬಿ-ಕೊಂ-ಡನು
ತಬ್ಬಿ-ಕೊಂ-ಡಿತು
ತಬ್ಬಿ-ಕೊಂಡು
ತಮ
ತಮ-ಎಂಬ
ತಮ-ಗಳು
ತಮ-ಗ-ಳೆಂಬ
ತಮ-ಗಾಗಿ
ತಮ-ಗಾದ
ತಮ-ಗಿ-ರುವ
ತಮಗೆ
ತಮಗೇ
ತಮಟೆ
ತಮ-ವೆಂಬ
ತಮ-ಸ್ಸಿ-ನಲ್ಲಿ
ತಮ-ಸ್ಸು-ಎಂಬ
ತಮ-ಹಮ
ತಮ-ಹಮ್
ತಮಾಷೆ
ತಮಾ-ಷೆ-ನೋ-ಡ-ಬೇ-ಕೆ-ನ್ನಿ-ಸಿತು
ತಮೋ
ತಮೋ-ಗುಣ
ತಮೋ-ಗು-ಣಕ್ಕೆ
ತಮೋ-ಗು-ಣ-ಗಳ
ತಮೋ-ಗು-ಣ-ಗಳಲ್ಲಿ
ತಮೋ-ಗು-ಣ-ಗಳಿಂದ
ತಮೋ-ಗು-ಣ-ದಿಂದ
ತಮೋ-ಗು-ಣ-ವ-ನ್ನೆಲ್ಲ
ತಮೋ-ಗು-ಣ-ವುಳ್ಳ
ತಮ್ಮ
ತಮ್ಮಂ
ತಮ್ಮಂ-ತೆಯೇ
ತಮ್ಮಂದಿ
ತಮ್ಮಂ-ದಿ-ರಂತೆ
ತಮ್ಮಂ-ದಿ-ರನ್ನೂ
ತಮ್ಮಂ-ದಿ-ರಲ್ಲಿ
ತಮ್ಮಂ-ದಿ-ರಿಗೂ
ತಮ್ಮಂ-ದಿರು
ತಮ್ಮಂ-ದಿರೂ
ತಮ್ಮಂ-ದಿ-ರೊ-ಡನೆ
ತಮ್ಮಂ-ದಿ-ರೊ-ಡ-ನೆಯೂ
ತಮ್ಮ-ತಮ್ಮ
ತಮ್ಮ-ತ-ಮ್ಮ-ಲ್ಲಿಯೆ
ತಮ್ಮತ್ತ
ತಮ್ಮ-ದೆಂಬ
ತಮ್ಮನ
ತಮ್ಮ-ನನ್ನು
ತಮ್ಮ-ನಾದ
ತಮ್ಮ-ನಿ-ಗಾಗಿ
ತಮ್ಮ-ನಿ-ಗಾ-ಗಿಯೂ
ತಮ್ಮನ್ನು
ತಮ್ಮ-ಪಾ-ಪ-ಗಳನ್ನೆಲ್ಲ
ತಮ್ಮಲ್ಲಿ
ತಮ್ಮ-ಲ್ಲಿ-ದ್ದು-ದ-ರ-ಲ್ಲಿಯೆ
ತಮ್ಮ-ಲ್ಲಿಯೆ
ತಮ್ಮ-ಲ್ಲಿಯೇ
ತಮ್ಮ-ವರ
ತಮ್ಮ-ಸೇ-ನೆ-ಯೊ-ಡನೆ
ತಮ್ಮಿ-ಬ್ಬ-ರಿಗೂ
ತಮ್ಮೊ-ಡನೆ
ತಯಾ-ರು-ಮಾ-ಡಿ-ಸಿಟ್ಟು
ತಯೇ
ತರ
ತರಂ-ಗಿಣಿ
ತರ-ಗಲೆ
ತರ-ಗೆ-ಲೆ-ಗಳನ್ನು
ತರ-ಗೆ-ಲೆ-ಗಳನ್ನೂ
ತರ-ಗೆ-ಲೆ-ಯಾ-ಯಾತು
ತರ-ಣಕ್ಕೆ
ತರದೆ
ತರ-ಬೇಕು
ತರ-ಬೇ-ಕೆಂದು
ತರಲು
ತರ-ಲೆಂದು
ತರ-ವಲ್ಲ
ತರ-ವು-ದ-ಕ್ಕಾಗಿ
ತರ-ಹದ
ತರಿದ
ತರಿ-ದು-ಹಾಕಿ
ತರಿಸಿ
ತರು-ಣ-ನಾ-ಗ-ಲಾ-ರನೆ
ತರು-ಣ-ನೊ-ಬ್ಬ-ನೊ-ಡನೆ
ತರು-ತ್ತಾನೆ
ತರು-ತ್ತಾರೆ
ತರು-ತ್ತೇನೆ
ತರು-ಲತೆ
ತರು-ಲ-ತೆ-ಯಾಗಿ
ತರು-ವಂತೆ
ತರು-ವನೊ
ತರು-ವರು
ತರು-ವಾಯ
ತರು-ವಾಯು
ತರು-ವು-ದ-ಕ್ಕೆಂದು
ತರು-ವುದು
ತರ್ಜಿ-ನೀ-ಭ್ಯಾಂ
ತರ್ಪಣ
ತರ್ಪ-ಣ-ಕಾ-ರ್ಯ-ವನ್ನು
ತರ್ಪ-ಣ-ಗಳ
ತರ್ಪ-ಣ-ವಿತ್ತು
ತರ್ಪ-ಣವೆ
ತಲೆ
ತಲೆ-ಕೆಟ್ಟ
ತಲೆ-ಕೆ-ಳ-ಕಾ-ಗು-ವಂ-ತಾ-ಗು-ತ್ತದೆ
ತಲೆ-ಕೆ-ಳ-ಕಾದ
ತಲೆ-ಕೆ-ಳ-ಗಾಗಿ
ತಲೆ-ಕೆ-ಳಗು
ತಲೆ-ಗಳನ್ನು
ತಲೆ-ಗಳನ್ನೂ
ತಲೆ-ಗ-ಳಾದ
ತಲೆ-ಗ-ಳಾ-ದ-ಮೇಲೆ
ತಲೆ-ಗಳಿ
ತಲೆ-ಗಳು
ತಲೆ-ಗ-ಳೊ-ಡನೆ
ತಲೆ-ಗೂ-ದ-ಲನ್ನು
ತಲೆ-ಗೂ-ದ-ಲನ್ನೂ
ತಲೆ-ಗೂ-ದಲು
ತಲೆ-ಗೂ-ದಲೂ
ತಲೆಗೆ
ತಲೆ-ಗೇರಿ
ತಲೆ-ಗೇ-ರಿತು
ತಲೆ-ತ-ಗ್ಗಿಸಿ
ತಲೆ-ದೂಗಿ
ತಲೆ-ದೂಗು
ತಲೆ-ದೂ-ಗುವ
ತಲೆ-ದೂ-ಗು-ವಂತೆ
ತಲೆ-ದೋರ
ತಲೆ-ಬಾಗಿ
ತಲೆ-ಬಾ-ಗಿ-ದಳು
ತಲೆ-ಬಾ-ಗಿ-ಸಿದ
ತಲೆ-ಬಾಗು
ತಲೆ-ಬಾ-ಗು-ವುದೆ
ತಲೆ-ಬು-ರು-ಡೆ-ಗಳ
ತಲೆ-ಮ-ರೆಸಿ
ತಲೆ-ಮ-ರೆ-ಸಿ-ಕೊಂ-ಡರು
ತಲೆ-ಮ-ರೆ-ಸಿ-ಕೊಂ-ಡಿದ್ದ
ತಲೆ-ಮ-ರೆ-ಸಿ-ಕೊಂ-ಡಿ-ದ್ದ-ವ-ರ-ನ್ನೆಲ್ಲ
ತಲೆ-ಮ-ರೆ-ಸಿ-ಕೊಂ-ಡಿ-ದ್ದಾನೆ
ತಲೆ-ಮ-ರೆ-ಸಿ-ಕೊಂಡು
ತಲೆ-ಮ-ರೆ-ಸಿ-ಕೊಂ-ಡು-ದನ್ನು
ತಲೆ-ಮ-ರೆ-ಸಿ-ಕೊ-ಳ್ಳು-ವು-ದೊಂದೆ
ತಲೆಯ
ತಲೆ-ಯನ್ನು
ತಲೆ-ಯನ್ನೂ
ತಲೆ-ಯನ್ನೆ
ತಲೆ-ಯ-ನ್ನೆತ್ತಿ
ತಲೆ-ಯ-ನ್ನೇಕೆ
ತಲೆ-ಯ-ಮೇ-ಲಿ-ಡು-ವನೋ
ತಲೆ-ಯ-ಮೇ-ಲಿನ
ತಲೆ-ಯ-ಮೇಲೆ
ತಲೆ-ಯ-ಮೇ-ಲೆಲ್ಲ
ತಲೆ-ಯಲ್ಲಿ
ತಲೆ-ಯ-ಲ್ಲಿದ್ದ
ತಲೆ-ಯ-ವರೆ-ಗಿನ
ತಲೆ-ಯ-ವ-ರೆಗೆ
ತಲೆ-ಯಿಂದ
ತಲೆ-ಯಿಟ್ಟು
ತಲೆ-ಯಿ-ಲ್ಲದ
ತಲೆ-ಯುಂ-ಟಾ-ಗಲಿ
ತಲೆಯೆ
ತಲೆ-ಯೆತ್ತಿ
ತಲೆ-ಯೆ-ತ್ತಿ-ಕೊಂಡು
ತಲೆಯೇ
ತಲೆ-ಯೊ-ಡೆದು
ತಲೆ-ಹ-ರಟೆ
ತಲ್ಲಣ
ತಲ್ಲ-ಣ-ಗೊಂ-ಡುವು
ತಲ್ಲ-ಣ-ಗೊ-ಳಿ-ಸಿದ
ತಲ್ಲ-ಣ-ಗೊ-ಳಿ-ಸುತ್ತಾ
ತಲ್ಲ-ಣ-ಗೊ-ಳ್ಳು-ತ್ತಿ-ರಲು
ತಲ್ಲ-ಣಿ-ಸಿತು
ತಲ್ಲ-ಣಿ-ಸಿ-ದುವು
ತಲ್ಲ-ಣಿ-ಸುತ್ತಾ
ತಲ್ಲ-ಣಿ-ಸು-ವನು
ತಲ್ಲೀ-ನ-ನಾ-ಗ-ಬೇಕು
ತಲ್ಲೀ-ನ-ನಾ-ಗು-ವು-ದನ್ನೇ
ತಲ್ಲೀ-ನ-ನಾ-ಗು-ವುದು
ತಲ್ಲೀ-ನ-ನಾ-ದನು
ತಲ್ಲೀ-ನ-ರಾಗಿ
ತಲ್ಲೀ-ನ-ರಾ-ಗಿ-ರು-ವರು
ತಲ್ಲೀ-ನ-ಳಾ-ಗಿರು
ತಳ-ದಲ್ಲಿ
ತಳ-ಮಳ
ತಳ-ಮ-ಳ-ಗೊಂ-ಡಿತು
ತಳ-ಮ-ಳ-ಗೊ-ಳಿ-ಸಿತು
ತಳ-ಮ-ಳ-ಗೊ-ಳ್ಳು-ತ್ತಿತ್ತು
ತಳ-ಮ-ಳಿ-ಸಿ-ದರು
ತಳಾ-ತಳ
ತಳಿರು
ತಳು-ವದೆ
ತಳೆದು
ತಳೆ-ಯಿತು
ತಳ್ಳಿ
ತಳ್ಳಿ-ದನು
ತಳ್ಳಿದ್ದ
ತಳ್ಳಿ-ಸಿ-ಕೊಂಡು
ತಳ್ಳಿ-ಹಾಕಿ
ತಳ್ಳು-ತ್ತಿ-ದ್ದರು
ತಳ್ಳು-ತ್ತಿ-ರಲು
ತವ
ತವ-ಕ-ಪ-ಡು-ತ್ತಿ-ದ್ದರು
ತವಕಿ
ತವ-ಕಿ-ಸು-ತ್ತಿ-ರುವ
ತವ-ರು-ಮನೆ
ತವರ್
ತಸ್ಮಾ-ದ್ಭ-ಜಾಮೋ
ತಸ್ಮಿನ್
ತಸ್ಮೈ
ತಹ-ತಹ
ತಹರ
ತಹ-ವ-ನಿಗೆ
ತಹ-ವರ
ತಾ
ತಾಂಡ-ವ-ನೃ-ತ್ಯ-ದಲ್ಲಿ
ತಾಂಡ-ವ-ವಾ-ಡು-ತ್ತಿದ್ದ
ತಾಂತ್ರಿಕ
ತಾಂಬೂಲ
ತಾಂಬೂ-ಲ-ಗಳನ್ನು
ತಾಂಬೂ-ಲದ
ತಾಂಬೂ-ಲ-ವನ್ನು
ತಾಕ-ಲಾ-ಟವು
ತಾಕಿ
ತಾಕಿ-ದರೆ
ತಾಕಿ-ಸಿ-ದನು
ತಾಕು-ತ್ತಲೆ
ತಾಗಿ
ತಾಗು-ವು-ದಿಲ್ಲ
ತಾಣಕ್ಕೆ
ತಾಣ-ವಾ-ಯಿತು
ತಾತ
ತಾತ-ನಾದ
ತಾತ-ನಿಗೆ
ತಾತ್ಕಾ-ಲಿಕ
ತಾತ್ವಿಕ
ತಾನಾಗಿ
ತಾನಾ-ಗಿಯೆ
ತಾನಾ-ಗಿಯೇ
ತಾನಾ-ದರೂ
ತಾನಾ-ಯಿತು
ತಾನಾ-ರೆಂ-ಬು-ದನ್ನು
ತಾನಿದ್ದ
ತಾನಿನ್ನು
ತಾನಿ-ರು-ವ-ಲ್ಲಿಂ-ದಲೆ
ತಾನು
ತಾನುಂಟೊ
ತಾನೂ
ತಾನೆ
ತಾನೆಂದು
ತಾನೆ-ತ್ತಿ-ಕೊಂಡು
ತಾನೇ
ತಾನೇ-ತಾ-ನಾಗಿ
ತಾನೇ-ರಿದ್ದ
ತಾನೊ-ಬ್ಬನೆ
ತಾನೊ-ಬ್ಬನೇ
ತಾಪ
ತಾಪಕ್ಕೆ
ತಾಪ-ತ್ರಯ
ತಾಪ-ತ್ರ-ಯ-ಗಳನ್ನೂ
ತಾಪ-ತ್ರ-ಯ-ಗಳು
ತಾಪ-ಸಿ-ಗ-ಳಾದ
ತಾಪ-ಹಾ-ರ-ಕ-ವಾದ
ತಾಮಸ
ತಾಮ-ಸ-ಭಕ್ತಿ
ತಾಮ-ಸ-ಮ-ನ್ವಂ-ತ-ರ-ದಲ್ಲಿ
ತಾಮ-ಸ-ವೆಂಬ
ತಾಮ-ಸಾ-ಹಂ-ಕಾ-ರ-ದಿಂದ
ತಾಮ-ಸಾ-ಹಂ-ಕಾ-ರ-ವೆಂಬ
ತಾಮಿಸ್ರ
ತಾಮ್ರ
ತಾಮ್ರದ
ತಾಮ್ರಾಕ್ಷ
ತಾಯ
ತಾಯಾ-ಗ-ಲಿ-ರು-ವ-ವಳ
ತಾಯಾದ
ತಾಯಾ-ದಳು
ತಾಯಿ
ತಾಯಿ-ಗಿಂ-ತಲೂ
ತಾಯಿಗೂ
ತಾಯಿಗೆ
ತಾಯಿ-ಗೊ-ಪ್ಪಿ-ಸಿ-ದರು
ತಾಯಿ-ತಂ-ದೆ-ಗಳು
ತಾಯಿ-ತಂ-ದೆಗೂ
ತಾಯಿತು
ತಾಯಿ-ಬೇರು
ತಾಯಿಯ
ತಾಯಿ-ಯನ್ನು
ತಾಯಿ-ಯನ್ನೂ
ತಾಯಿ-ಯರ
ತಾಯಿ-ಯ-ರಿಗೆ
ತಾಯಿ-ಯ-ರಿರ
ತಾಯಿ-ಯರು
ತಾಯಿ-ಯರೆ
ತಾಯಿ-ಯ-ವರ
ತಾಯಿ-ಯಾ-ಗು-ತ್ತಾಳೆ
ತಾಯಿ-ಯಾದ
ತಾಯಿ-ಯಾ-ದಳು
ತಾಯಿ-ಯಿಂದ
ತಾಯಿಯೂ
ತಾಯ್ತಂದೆ
ತಾಯ್ತಂ-ದೆ-ಗಳ
ತಾಯ್ತಂ-ದೆ-ಗ-ಳಂ-ತಿ-ರುವ
ತಾಯ್ತಂ-ದೆ-ಗಳನ್ನು
ತಾಯ್ತಂ-ದೆ-ಗಳನ್ನೂ
ತಾಯ್ತಂ-ದೆ-ಗ-ಳಾದ
ತಾಯ್ತಂ-ದೆ-ಗಳಿಂದ
ತಾಯ್ತಂ-ದೆ-ಗ-ಳಿಗೂ
ತಾಯ್ತಂ-ದೆ-ಗ-ಳಿ-ದ್ದರೂ
ತಾಯ್ತಂ-ದೆ-ಗಳು
ತಾರ-ತಮ್ಯ
ತಾರ-ದಿಂದ
ತಾರು-ಣ್ಯ-ವನ್ನು
ತಾರೆ
ತಾರೆಗೆ
ತಾರೆ-ಯನ್ನು
ತಾರೆ-ಯನ್ನೆ
ತಾರ್ಕ್ಷ-್ಯ-ನೆಂಬ
ತಾರ್ಕ್ಷ-್ಯೆ-ಯನ್ನು
ತಾಳ
ತಾಳ-ಜಂಘ
ತಾಳ-ಬೇ-ಕಾ-ಗು-ತ್ತದೆ
ತಾಳ-ಲಾ-ರದೆ
ತಾಳಿ
ತಾಳಿದ
ತಾಳಿ-ದಂತೆ
ತಾಳಿ-ದನು
ತಾಳಿ-ದರು
ತಾಳಿದೆ
ತಾಳಿದ್ದ
ತಾಳೆಯ
ತಾಳ್ಮೆ-ಯಲ್ಲಿ
ತಾಳ್ಮೆ-ಯಿಂದ
ತಾಳ್ಮೆ-ಯು-ಳ್ಳ-ವನು
ತಾವರೆ
ತಾವರೆ-ಗೊ-ಳ-ದಂ-ತಿದೆ
ತಾವ-ರೆಯ
ತಾವರೆ-ಯಂತೆ
ತಾವಾಗಿ
ತಾವಾ-ಗಿಯೆ
ತಾವಾ-ಗಿಯೇ
ತಾವಿದ್ದ
ತಾವು
ತಾವೂ
ತಾವೆ
ತಾವೆಲ್ಲ
ತಾವೆ-ಲ್ಲಿದೆ
ತಾವೆ-ಲ್ಲಿ-ಯೆಂದು
ತಾವೇ
ತಿಂಗ
ತಿಂಗಳ
ತಿಂಗಳನ್ನು
ತಿಂಗಳಲ್ಲಿ
ತಿಂಗ-ಳಾ-ದುವು
ತಿಂಗಳಿ
ತಿಂಗ-ಳಿಗೆ
ತಿಂಗಳು
ತಿಂಗ-ಳು-ಗ-ಳಾ-ದವು
ತಿಂಡಿ
ತಿಂದ
ತಿಂದನು
ತಿಂದ-ಮ-ನೆಗೆ
ತಿಂದರೆ
ತಿಂದ-ಳಲ್ಲಾ
ತಿಂದ-ಹಾಗೆ
ತಿಂದು
ತಿಂದು-ಕೊಂಡು
ತಿಂದು-ತಿಂದು
ತಿಂದು-ದ-ರಿಂದ
ತಿಂದು-ಹಾ-ಕಿದ
ತಿಂದು-ಹಾ-ಕಿ-ದನು
ತಿಂದು-ಹಾ-ಕಿ-ದರು
ತಿಂದು-ಹಾ-ಕಿ-ದೆ-ಯಲ್ಲಾ
ತಿಂದು-ಹೋಗಿ
ತಿಕ್ಕಿ
ತಿಗ-ಣೆ-ಗಳು
ತಿಗ್ಮ-ಧಾ-ರಾಸಿ
ತಿಗ್ಮ-ನೇ-ಮಿಃ
ತಿತ್ತಿ-ರಿ-ಪ-ಕ್ಷಿ-ಗಳ
ತಿಥಿ
ತಿದ್ದಿ
ತಿದ್ದು-ದ-ರಿಂದ
ತಿದ್ದು-ವು-ದ-ಕ್ಕಾಗಿ
ತಿನಿ-ಸನ್ನು
ತಿನಿ-ಸು-ಗಳಿಂದ
ತಿನ್ನ
ತಿನ್ನ-ದಂತೆ
ತಿನ್ನ-ಬೇ-ಕಲ್ಲ
ತಿನ್ನ-ಬೇಕು
ತಿನ್ನ-ಬೇಡ
ತಿನ್ನಿ-ಸ-ಬೇಕು
ತಿನ್ನಿ-ಸುತ್ತಾ
ತಿನ್ನಿ-ಸು-ವರು
ತಿನ್ನು
ತಿನ್ನುತ್ತ
ತಿನ್ನುತ್ತಾ
ತಿನ್ನು-ತ್ತಾರೆ
ತಿನ್ನುತ್ತಿ
ತಿನ್ನು-ತ್ತಿದ್ದ
ತಿನ್ನು-ತ್ತಿ-ದ್ದನು
ತಿನ್ನು-ತ್ತಿ-ದ್ದಾನೆ
ತಿನ್ನು-ತ್ತಿ-ರು-ವಂತೆ
ತಿನ್ನು-ತ್ತೇನೆ
ತಿನ್ನುವ
ತಿನ್ನು-ವಂತೆ
ತಿನ್ನು-ವನು
ತಿನ್ನು-ವು-ದ-ಕ್ಕಾಗಿ
ತಿನ್ನು-ವು-ದಕ್ಕೆ
ತಿನ್ನು-ವುದು
ತಿಮಿ
ತಿಮಿಂ-ಗಿ-ಲ-ಗ-ಳಂ-ತಿದ್ದ
ತಿರಕಿ
ತಿರ-ಸ್ಕ-ರಿಸಿ
ತಿರಿದು
ತಿರಿ-ಯನ್ನು
ತಿರು-ಕ-ನನ್ನು
ತಿರು-ಕ-ರಾಗಿ
ತಿರು-ಗ-ತೊ-ಡ-ಗಿ-ದನು
ತಿರು-ಗಲಿ
ತಿರು-ಗ-ಲಿಲ್ಲ
ತಿರು-ಗ-ಲೆಂದು
ತಿರು-ಗಾಟ
ತಿರು-ಗಾ-ಡು-ತ್ತಿದ್ದ
ತಿರು-ಗಾ-ಡು-ತ್ತಿದ್ದು
ತಿರು-ಗಾ-ಡು-ತ್ತಿ-ದ್ದೇವೆ
ತಿರು-ಗಾ-ಡು-ತ್ತಿ-ರು-ವೆಯೋ
ತಿರುಗಿ
ತಿರು-ಗಿ-ಕೊಂಡು
ತಿರು-ಗಿತು
ತಿರು-ಗಿ-ದನು
ತಿರು-ಗಿ-ದರೆ
ತಿರು-ಗಿ-ದವು
ತಿರು-ಗಿ-ನಿಂತು
ತಿರು-ಗಿ-ರು-ವು-ದ-ರಿಂದ
ತಿರು-ಗಿಸಿ
ತಿರು-ಗಿ-ಸಿ-ಕೊಂ-ಡನು
ತಿರು-ಗಿ-ಸಿ-ಕೊಂಡು
ತಿರು-ಗಿ-ಸಿ-ಕೊ-ಳ್ಳು-ವು-ದಕ್ಕೂ
ತಿರು-ಗಿ-ಸಿ-ದ-ವನೇ
ತಿರು-ಗಿಸು
ತಿರು-ಗಿ-ಸುತ್ತಾ
ತಿರು-ಗಿ-ಸು-ವನು
ತಿರುಗು
ತಿರು-ಗು-ತ್ತಲೆ
ತಿರು-ಗುತ್ತಾ
ತಿರು-ಗು-ತ್ತಿದ್ದ
ತಿರು-ಗು-ತ್ತಿದ್ದಿ
ತಿರು-ಗು-ತ್ತಿ-ರ-ಲಿಲ್ಲ
ತಿರು-ಗು-ತ್ತಿ-ರಲು
ತಿರು-ಗು-ತ್ತಿ-ರು-ತ್ತದೆ
ತಿರು-ಗು-ತ್ತಿ-ರುವ
ತಿರು-ಗು-ತ್ತಿ-ರು-ವಾಗ
ತಿರುಚಿ
ತಿರು-ಚಿ-ದನು
ತಿರು-ಚು-ತ್ತಲೆ
ತಿರು-ಪತಿ
ತಿರು-ಪೆಗೆ
ತಿರುಳು
ತಿರುವ
ತಿರು-ವಿ-ಹಾ-ಕುತ್ತಾ
ತಿರ್ಯಕ್
ತಿರ್ಯ-ಗಾದಿ
ತಿಳಿ
ತಿಳಿ-ಕೊಂ-ಡಿ-ದ್ದರು
ತಿಳಿ-ಗ-ನ್ನ-ಡ-ದಲ್ಲಿ
ತಿಳಿದ
ತಿಳಿ-ದದ್ದು
ತಿಳಿ-ದರೆ
ತಿಳಿ-ದ-ವ-ನಂತೆ
ತಿಳಿ-ದ-ವ-ನಾಗಿ
ತಿಳಿ-ದ-ವ-ನಾ-ದರೂ
ತಿಳಿ-ದ-ವನು
ತಿಳಿ-ದ-ವನೇ
ತಿಳಿ-ದ-ವ-ರಲ್ಲಿ
ತಿಳಿ-ದ-ವರು
ತಿಳಿ-ದಾಗ
ತಿಳಿ-ದಿತ್ತು
ತಿಳಿ-ದಿ-ದ್ದರೂ
ತಿಳಿ-ದಿ-ದ್ದಾನೆ
ತಿಳಿ-ದಿದ್ದಿ
ತಿಳಿ-ದಿ-ರ-ಲಿಲ್ಲ
ತಿಳಿ-ದಿ-ರು-ವಂತೆ
ತಿಳಿ-ದಿ-ರು-ವ-ವನೂ
ತಿಳಿ-ದಿ-ರು-ವಿರಿ
ತಿಳಿ-ದಿಲ್ಲ
ತಿಳಿದು
ತಿಳಿ-ದುಕೊ
ತಿಳಿ-ದು-ಕೊಂ-ಡರು
ತಿಳಿ-ದು-ಕೊಂ-ಡ-ವರು
ತಿಳಿ-ದು-ಕೊಂ-ಡಿದ್ದಿ
ತಿಳಿ-ದು-ಕೊಂ-ಡಿ-ದ್ದೇವೆ
ತಿಳಿ-ದು-ಕೊಂ-ಡಿ-ರು-ವುದೂ
ತಿಳಿ-ದು-ಕೊಂಡು
ತಿಳಿ-ದು-ಕೊಂ-ಡೆನು
ತಿಳಿ-ದು-ಕೊ-ಳ್ಳ-ಬೇಕು
ತಿಳಿ-ದು-ಕೊ-ಳ್ಳ-ಲಾ-ರದ
ತಿಳಿ-ದು-ಕೊ-ಳ್ಳು-ತ್ತವೆ
ತಿಳಿ-ದು-ಕೊ-ಳ್ಳು-ತ್ತೇವೆ
ತಿಳಿ-ದು-ಕೊ-ಳ್ಳು-ವವ
ತಿಳಿ-ದು-ಕೊ-ಳ್ಳು-ವುದು
ತಿಳಿ-ದು-ಬಂತು
ತಿಳಿ-ದೆಯಾ
ತಿಳಿದೇ
ತಿಳಿದೋ
ತಿಳಿ-ನೀ-ರನ್ನು
ತಿಳಿ-ನೀ-ರಿನ
ತಿಳಿ-ನೀ-ರಿ-ನಿಂದ
ತಿಳಿ-ಮ-ನ-ಸ್ಸಿ-ನ-ವನು
ತಿಳಿಯ
ತಿಳಿ-ಯದ
ತಿಳಿ-ಯ-ದಂತೆ
ತಿಳಿ-ಯ-ದ-ವ-ನಂತೆ
ತಿಳಿ-ಯ-ದು-ದನ್ನೂ
ತಿಳಿ-ಯ-ದುದು
ತಿಳಿ-ಯ-ದು-ದೇನೂ
ತಿಳಿ-ಯದೆ
ತಿಳಿ-ಯ-ದೆಯೋ
ತಿಳಿ-ಯ-ಬ-ಯ-ಸ-ಬ-ಹು-ದಾದ
ತಿಳಿ-ಯ-ಬ-ಯ-ಸಿ-ದನು
ತಿಳಿ-ಯ-ಬ-ಯ-ಸು-ವ-ವನು
ತಿಳಿ-ಯ-ಬಲ್ಲ
ತಿಳಿ-ಯ-ಬೇಕು
ತಿಳಿ-ಯ-ಬೇಕೆ
ತಿಳಿ-ಯ-ಬೇ-ಕೆಂದು
ತಿಳಿ-ಯ-ಬೇ-ಕೆಂಬ
ತಿಳಿ-ಯ-ಬೇಡಿ
ತಿಳಿ-ಯ-ಲಾ-ಗದ
ತಿಳಿ-ಯ-ಲಾ-ರದೆ
ತಿಳಿ-ಯ-ಲಾ-ರೆವು
ತಿಳಿ-ಯಲು
ತಿಳಿ-ಯ-ವಾ-ದ್ದ-ರಿಂದ
ತಿಳಿ-ಯಾಗಿ
ತಿಳಿ-ಯಾ-ಯಿತು
ತಿಳಿ-ಯಿತು
ತಿಳಿ-ಯು-ತ್ತಲೆ
ತಿಳಿ-ಯುವ
ತಿಳಿ-ಯು-ವಂ-ತಿಲ್ಲ
ತಿಳಿ-ಯು-ವು-ದ-ಕ್ಕಾಗಿ
ತಿಳಿ-ಯು-ವುದೂ
ತಿಳಿ-ಯು-ವುದೇ
ತಿಳಿ-ಯು-ವೆ-ಯೇನು
ತಿಳಿ-ಯೋಣ
ತಿಳಿ-ವ-ಳಿಕೆ
ತಿಳಿ-ವ-ಳಿ-ಕೆಗೂ
ತಿಳಿ-ಸದೆ
ತಿಳಿ-ಸ-ಬೇಕು
ತಿಳಿ-ಸಲಿ
ತಿಳಿ-ಸ-ಲಿಲ್ಲ
ತಿಳಿ-ಸಲು
ತಿಳಿಸಿ
ತಿಳಿ-ಸಿತ್ತು
ತಿಳಿ-ಸಿದ
ತಿಳಿ-ಸಿ-ದನು
ತಿಳಿ-ಸಿ-ದ-ನು-ನೋ-ಡಯ್ಯ
ತಿಳಿ-ಸಿ-ದರು
ತಿಳಿ-ಸಿ-ದರೂ
ತಿಳಿ-ಸಿ-ದರೆ
ತಿಳಿ-ಸಿ-ದಳು
ತಿಳಿ-ಸಿ-ದ-ಳು-ನೋಡೆ
ತಿಳಿ-ಸಿ-ದ-ಳು-ಮ-ಹಾ-ರಾಜ
ತಿಳಿ-ಸಿದೆ
ತಿಳಿ-ಸಿ-ದೆ-ಯಷ್ಟೆ
ತಿಳಿ-ಸಿ-ದ್ದಂತೆ
ತಿಳಿ-ಸಿ-ದ್ದರೂ
ತಿಳಿ-ಸಿ-ದ್ದೇನೆ
ತಿಳಿ-ಸಿ-ದ್ದೇ-ನೆ-ನನ್ನ
ತಿಳಿಸು
ತಿಳಿ-ಸು-ತ್ತದೆ
ತಿಳಿ-ಸು-ತ್ತಾನೆ
ತಿಳಿ-ಸು-ತ್ತಾರೆ
ತಿಳಿ-ಸು-ತ್ತೇನೆ
ತಿಳಿ-ಸುವ
ತಿಳಿ-ಸು-ವಂತೆ
ತಿಳಿ-ಸು-ವು-ದ-ಕ್ಕಾಗಿ
ತಿಳಿ-ಸು-ವು-ದ-ಕ್ಕಾ-ಗಿಯೆ
ತಿಳಿ-ಸು-ವು-ದ-ಕ್ಕಾ-ಗಿಯೇ
ತಿಳಿ-ಸು-ವು-ದಕ್ಕೂ
ತಿಳಿ-ಸು-ವುದು
ತಿಳು-ಹಿಸಿ
ತಿವಿ-ದರು
ತಿವಿ-ದಾ-ಡು-ವರು
ತಿವಿದು
ತಿವಿ-ಯು-ವು-ದಕ್ಕೆ
ತಿಷ್ಠ-ತ್ಯ-ಪ್ಯೇತಿ
ತೀಟೆ-ಯನ್ನು
ತೀತ
ತೀರ
ತೀರಕ್ಕೆ
ತೀರ-ದಲ್ಲಿ
ತೀರ-ದ-ಲ್ಲಿದ್ದ
ತೀರ-ದ-ಲ್ಲಿಯೇ
ತೀರ-ದ-ಲ್ಲಿ-ರು-ವಾಗ
ತೀರ-ಬೇಕು
ತೀರ-ಲಿಲ್ಲ
ತೀರ-ವನ್ನು
ತೀರಿ-ಸ-ಬೇ-ಕಾ-ಗಿದೆ
ತೀರಿ-ಸ-ಲಿಲ್ಲ
ತೀರಿಸಿ
ತೀರಿ-ಸಿ-ಕೊಳ್ಳ
ತೀರಿ-ಸಿ-ಕೊ-ಳ್ಳ-ಬೇ-ಕೆಂದು
ತೀರಿ-ಸಿ-ಕೊ-ಳ್ಳು-ತ್ತೇನೆ
ತೀರಿ-ಸಿ-ಕೊ-ಳ್ಳುವ
ತೀರಿ-ಸಿ-ಕೊ-ಳ್ಳುವು
ತೀರಿ-ಸಿ-ಕೊ-ಳ್ಳು-ವು-ದ-ಕ್ಕಾಗಿ
ತೀರಿ-ಸಿ-ದ-ರಾ-ಯಿತು
ತೀರಿ-ಸಿ-ದರೂ
ತೀರಿ-ಸಿ-ಬಿಡಿ
ತೀರಿ-ಸಿ-ರುವೆ
ತೀರಿ-ಸು-ತ್ತೇನೆ
ತೀರಿ-ಸು-ವಷ್ಟು
ತೀರಿ-ಸು-ವು-ದ-ಕ್ಕಾಗಿ
ತೀರಿ-ಸು-ವು-ದಕ್ಕೆ
ತೀರಿ-ಹೋ-ಗಲಿ
ತೀರಿ-ಹೋ-ದವು
ತೀರು
ತೀರು-ತ್ತದೆ
ತೀರು-ತ್ತಲೆ
ತೀರು-ವ-ವ-ರೆಗೆ
ತೀರ್ಥ
ತೀರ್ಥ-ಕುಂ-ಡ-ದಲ್ಲಿ
ತೀರ್ಥ-ಕ್ಷೇ-ತ್ರ-ವಿತ್ತು
ತೀರ್ಥ-ಗಳಲ್ಲಿ
ತೀರ್ಥ-ಗಳೂ
ತೀರ್ಥ-ದಂತೆ
ತೀರ್ಥ-ದಲ್ಲಿ
ತೀರ್ಥ-ಯಾತ್ರೆ
ತೀರ್ಥ-ಯಾ-ತ್ರೆ-ಗಾಗಿ
ತೀರ್ಥ-ಯಾ-ತ್ರೆ-ಗೆಂದು
ತೀರ್ಥ-ಯಾ-ತ್ರೆಯ
ತೀರ್ಥ-ಯಾ-ತ್ರೆ-ಯನ್ನು
ತೀರ್ಥ-ಯಾ-ತ್ರೆ-ಯಿಂದ
ತೀರ್ಥ-ವನ್ನು
ತೀರ್ಥ-ವೆಂದು
ತೀರ್ಥ-ವೆಂದೂ
ತೀರ್ಥ-ಸ್ಥ-ಳ-ಗಳ
ತೀರ್ಥ-ಸ್ನಾನ
ತೀರ್ಥ-ಸ್ನಾ-ನ-ಕ್ಕಾಗಿ
ತೀರ್ಥ-ಸ್ನಾ-ನದ
ತೀರ್ಮಾ-ನ-ದಂತೆ
ತೀರ್ಮಾ-ನಿ-ಸಿ-ದರು
ತೀವ್ರ-ವಾದ
ತೀವ್ರಾ-ಸಕ್ತಿ
ತು
ತುಂಟ
ತುಂಟ-ತನ
ತುಂಟ-ತ-ನ-ದಲ್ಲಿ
ತುಂಟ-ತ-ನವೂ
ತುಂಟ-ತ-ನವೇ
ತುಂಟ-ಹು-ಡು-ಗ-ನಂತೆ
ತುಂಡನ್ನು
ತುಂಡನ್ನೂ
ತುಂಡ-ರಿ-ಸಿರಿ
ತುಂಡಾ-ಗ-ಲಿಲ್ಲ
ತುಂಡಾಗಿ
ತುಂಡಾ-ಗು-ತ್ತಲೆ
ತುಂಡು
ತುಂಡು-ಗ-ಳಾಗಿ
ತುಂಡು-ಗಳು
ತುಂಡು-ತುಂ-ಡಾಗಿ
ತುಂಡು-ಮಾ-ಡಿ-ದರೂ
ತುಂಬ
ತುಂಬಿ
ತುಂಬಿ-ಕೊಂಡ
ತುಂಬಿ-ಕೊಂ-ಡಿತ್ತು
ತುಂಬಿ-ಕೊಂ-ಡಿ-ರುವ
ತುಂಬಿ-ಕೊಂಡು
ತುಂಬಿ-ಕೊಂ-ಡು-ದ-ರಿಂದ
ತುಂಬಿ-ಕೊ-ಳ್ಳು-ವಂತೆ
ತುಂಬಿತು
ತುಂಬಿ-ತು-ಳು-ಕು-ತ್ತಿ-ರುವ
ತುಂಬಿತ್ತು
ತುಂಬಿದ
ತುಂಬಿ-ದಂ-ತಾ-ಯಿತು
ತುಂಬಿ-ದನು
ತುಂಬಿ-ದ-ವರು
ತುಂಬಿ-ದವು
ತುಂಬಿ-ದುವು
ತುಂಬಿದೆ
ತುಂಬಿ-ದ್ದವು
ತುಂಬಿ-ರುವ
ತುಂಬಿ-ರು-ವು-ದ-ರಿಂದ
ತುಂಬಿ-ಹೋ-ಗಿವೆ
ತುಂಬಿ-ಹೋ-ಗು-ವಷ್ಟು
ತುಂಬಿ-ಹೋ-ದವು
ತುಂಬಿ-ಹೋ-ಯಿತು
ತುಂಬು
ತುಂಬು-ತ್ತದೆ
ತುಂಬು-ತ್ತಲೆ
ತುಂಬು-ತ್ತವೆ
ತುಂಬು-ನೆ-ರೆ-ಯಾಗಿ
ತುಂಬು-ವಷ್ಟು
ತುಂಬು-ವು-ದ-ಕ್ಕಾಗಿ
ತುಂಬು-ವು-ದಿ-ಲ್ಲವೆ
ತುಂಬು-ಹೊ-ಳೆ-ಯಾಗಿ
ತುಟಿ
ತುಟಿ-ಗಳನ್ನು
ತುಟಿ-ಗಳಲ್ಲಿ
ತುಟಿ-ಗಳಿಂದ
ತುಟಿ-ಗಳು
ತುಟಿಗೆ
ತುಟಿ-ಗೊ-ಯ್ದನು
ತುಟಿ-ಪಿ-ಟ-ಕ್ಕೆ-ನ್ನದೆ
ತುಟಿಯ
ತುಟಿ-ಯನ್ನು
ತುಟಿ-ಯಲ್ಲಿ
ತುಡಿ-ದು-ಕೊ-ಳ್ಳು-ತ್ತಿದೆ
ತುಡು-ಕು-ತ್ತದೆ
ತುತ್ತ-ತು-ದಿಯೇ
ತುತ್ತನ್ನು
ತುತ್ತ-ನ್ನೆತ್ತಿ
ತುತ್ತಾ
ತುತ್ತಾ-ಗಲು
ತುತ್ತಾಗಿ
ತುತ್ತಾ-ಗಿ-ರುವ
ತುತ್ತಾಗು
ತುತ್ತಾ-ಗು-ತ್ತದೊ
ತುತ್ತಾ-ಗು-ವ-ವರೇ
ತುತ್ತಾ-ದರು
ತುತ್ತಿಗೆ
ತುತ್ತು
ತುದಿ-ಯಲ್ಲಿ
ತುಪ್ಪ
ತುಪ್ಪ-ಗಳ
ತುಪ್ಪದ
ತುಪ್ಪ-ದ-ಮೇಲೆ
ತುಪ್ಪ-ದಲ್ಲಿ
ತುಪ್ಪ-ವನ್ನು
ತುರಿ-ಸು-ವುದು
ತುರು
ತುರು-ಕಿ-ದನು
ತುರು-ಗಳ
ತುರು-ಗಳನ್ನು
ತುರು-ಗ-ಳಿಗೆ
ತುರು-ಗಳು
ತುರು-ಗಳೂ
ತುರು-ಗ-ಳೊ-ಡನೆ
ತುರು-ಬನ್ನು
ತುರು-ಬಿಗೆ
ತುರು-ಬಿನ
ತುರು-ಬಿ-ನಲ್ಲಿ
ತುರು-ಬು-ಗಟ್ಟಿ
ತುರು-ಮಂದೆ
ತುರು-ಮಂ-ದೆ-ಗಳನ್ನು
ತುರು-ಮಂ-ದೆಗೆ
ತುರು-ಮಂ-ದೆ-ಯನ್ನು
ತುರು-ಮಂ-ದೆ-ಯಾ-ದನು
ತುರುಷ್ಕ
ತುಲ-ಸಿಯ
ತುಳ-ಸಿ-ಗಳಿಂದ
ತುಳ-ಸಿಯ
ತುಳಿ
ತುಳಿದ
ತುಳಿ-ಸಿ-ದನು
ತುಳು
ತುಳು-ಕಾ-ಡು-ತ್ತಿ-ರು-ತ್ತವೆ
ತುಳು-ಕಿಸು
ತುಳು-ಕಿ-ಸುತ್ತಾ
ತುಳು-ಕಿ-ಸು-ತ್ತಿದ್ದ
ತುಳು-ಕಿ-ಸು-ತ್ತಿ-ರುವ
ತುಳುಕು
ತುಳು-ಕು-ತ್ತಿತ್ತು
ತುಳು-ಕು-ತ್ತಿದೆ
ತುಳು-ಕು-ತ್ತಿದ್ದ
ತುಳು-ಕು-ತ್ತಿ-ದ್ದವು
ತುಷ್ಟಿ
ತುಸು-ದೂ-ರ-ದ-ಲ್ಲಿದ್ದ
ತೂಕ
ತೂಗಿ
ತೂಗಿ-ದವು
ತೂಗು
ತೂಗು-ತ್ತಿ-ರುವ
ತೂತು-ಗಳು
ತೂರಿ
ತೂರಿ-ಬಂ-ದರೂ
ತೃಣ-ಸ-ಮಾನ
ತೃಣಾ
ತೃಣಾ-ವರ್ತ
ತೃಣಾ-ವ-ರ್ತ-ನನ್ನು
ತೃಣಾ-ವ-ರ್ತ-ನೆಂಬ
ತೃಪ್ತ
ತೃಪ್ತ-ನಾ-ಗ-ಬಲ್ಲ
ತೃಪ್ತ-ನಾ-ಗ-ಬೇಕು
ತೃಪ್ತ-ನಾ-ಗ-ಲಿಲ್ಲ
ತೃಪ್ತ-ನಾದ
ತೃಪ್ತ-ನಾ-ದ-ವ-ನಿಗೆ
ತೃಪ್ತ-ರಾ-ಗಿ-ದ್ದರೆ
ತೃಪ್ತ-ರಾದ
ತೃಪ್ತ-ಳಾ-ಗಿ-ರು-ವಳು
ತೃಪ್ತ-ಳಾ-ಗು-ವ-ಳೆಂದು
ತೃಪ್ತ-ವಾ-ಗಿ-ರು-ವುದೇ
ತೃಪ್ತಿ
ತೃಪ್ತಿ-ಕ-ರ-ವಾಗಿ
ತೃಪ್ತಿ-ಕ-ರ-ವಾ-ಗಿ-ರು-ವು-ವ-ಲ್ಲವೆ
ತೃಪ್ತಿ-ಗಾ-ಗಲಿ
ತೃಪ್ತಿ-ಗೊ-ಳ್ಳ-ಲಿಲ್ಲ
ತೃಪ್ತಿ-ಪ-ಡಿ-ಸ-ಬೇಕು
ತೃಪ್ತಿ-ಪ-ಡಿ-ಸಿದ
ತೃಪ್ತಿ-ಪ-ಡಿ-ಸಿ-ದನು
ತೃಪ್ತಿ-ಪ-ಡಿ-ಸಿ-ದ-ಮೇಲೆ
ತೃಪ್ತಿ-ಪ-ಡಿ-ಸು-ವ-ವನು
ತೃಪ್ತಿ-ಯನ್ನು
ತೃಪ್ತಿ-ಯಾ-ಗು-ತ್ತದೆ
ತೃಪ್ತಿ-ಯಾ-ಗು-ವಂತೆ
ತೃಪ್ತಿ-ಯಾದ
ತೃಪ್ತಿ-ಯಾ-ದೀತೆ
ತೃಪ್ತಿ-ಯಾ-ಯಿತು
ತೃಪ್ತಿ-ಯಿಲ್ಲ
ತೃಪ್ತಿ-ಯೆಂ-ಬು-ದುಂಟೆ
ತೃಪ್ತಿ-ಹೊಂ-ದಿದ
ತೆಂಕಣ
ತೆಂಗಿ-ನ-ಕಾ-ಯಿ-ಗಳನ್ನು
ತೆಂದು
ತೆಗ-ದು-ಕೊಂಡು
ತೆಗ-ಳದೆ
ತೆಗೆ
ತೆಗೆದ
ತೆಗೆ-ದಿದ್ದ
ತೆಗೆದು
ತೆಗೆ-ದುಕೊ
ತೆಗೆ-ದು-ಕೊಂ-ಡನು
ತೆಗೆ-ದು-ಕೊಂ-ಡರು
ತೆಗೆ-ದು-ಕೊಂಡು
ತೆಗೆ-ದು-ಕೊ-ಳ್ಳ-ಬೇ-ಕೆಂದು
ತೆಗೆ-ದು-ಕೊಳ್ಳಿ
ತೆಗೆ-ದು-ಕೊ-ಳ್ಳಿರಿ
ತೆಗೆ-ದು-ಕೊಳ್ಳು
ತೆಗೆ-ದು-ಕೊ-ಳ್ಳು-ವ-ಷ್ಟ-ರಲ್ಲಿ
ತೆಗೆ-ದು-ಕೊ-ಳ್ಳೋಣ
ತೆಗೆ-ದುಕೋ
ತೆಗೆ-ಯಲು
ತೆಗೆ-ಯು-ವಂತೆ
ತೆಗೆಸಿ
ತೆತ್ತ
ತೆತ್ತು
ತೆಪ್ಪ-ಗಳನ್ನು
ತೆಪ್ಪ-ಗಿ-ದ್ದನು
ತೆಪ್ಪಗೆ
ತೆಪ್ಪ-ದಂತೆ
ತೆರಳ
ತೆರ-ಳ-ಬ-ಹುದು
ತೆರ-ಳ-ಬೇ-ಕೆಂದು
ತೆರ-ಳ-ಲಿ-ರು-ವಾಗ
ತೆರಳಿ
ತೆರ-ಳಿ-ದನು
ತೆರ-ಳಿ-ದ-ನು-ಎಂದು
ತೆರ-ಳಿ-ದವು
ತೆರ-ಳಿ-ದು-ದನ್ನು
ತೆರ-ಳಿ-ದು-ದನ್ನೂ
ತೆರಳು
ತೆರ-ಳು-ತ್ತೇನೆ
ತೆರ-ಳು-ವುದು
ತೆರವು
ತೆರ-ಸುವೆ
ತೆರೆ
ತೆರೆ-ತೆ-ರೆ-ಯಾಗಿ
ತೆರೆದ
ತೆರೆ-ದನು
ತೆರೆ-ದರೆ
ತೆರೆ-ದವು
ತೆರೆದು
ತೆರೆ-ದು-ಕೊಂ-ಡವು
ತೆರೆ-ದು-ಕೊಂಡು
ತೆರೆ-ಯನ್ನು
ತೆರೆ-ಯಿತು
ತೆರೆಯು
ತೆರೆ-ಯುವ
ತೆರೆ-ಯು-ವು-ದಕ್ಕೆ
ತೆರೆ-ಸಿತು
ತೆರೆ-ಸಿ-ದವು
ತೆಲುಗು
ತೆಳು-ವಾದ
ತೆವ-ಳಿ-ಕೊಂಡು
ತೇ
ತೇಗಿ
ತೇಗಿ-ದ-ಮೇಲೆ
ತೇಗಿ-ದುವು
ತೇಗುತ್ತಾ
ತೇಜ
ತೇಜ-ಸ್ವಿ-ಗ-ಳಾಗಿ
ತೇಜ-ಸ್ವಿ-ಗ-ಳಾದ
ತೇಜ-ಸ್ವಿ-ಯಾದ
ತೇಜ-ಸ್ಸನ್ನು
ತೇಜ-ಸ್ಸನ್ನೂ
ತೇಜ-ಸ್ಸಿಗೆ
ತೇಜ-ಸ್ಸಿ-ನಂತೆ
ತೇಜ-ಸ್ಸಿ-ನಿಂದ
ತೇಜ-ಸ್ಸಿ-ನಿಂ-ದಲೆ
ತೇಜ-ಸ್ಸಿ-ನಿಂ-ದಲೇ
ತೇಜಸ್ಸು
ತೇಜ-ಸ್ಸು-ಗಳನ್ನೂ
ತೇಜ-ಸ್ಸುಳ್ಳ
ತೇಜಸ್ಸೂ
ತೇಜೋ-ಬಲ
ತೇಜೋ-ಮೂ-ರ್ತಿ-ಯಾಗಿ
ತೇಜೋ-ರಾ-ಶಿ-ಯಾದ
ತೇಜೋ-ರೂ-ಪ-ವಾದ
ತೇನೈವ
ತೇಲಿ-ಕೊಂಡು
ತೇಲಿ-ಬ-ರು-ತ್ತಿ-ರುವ
ತೇಲಿ-ಸುವ
ತೇಲಿ-ಹೋ-ಗು-ತ್ತಿ-ದ್ದನು
ತೇಲಿ-ಹೋ-ಗು-ತ್ತಿ-ರು-ವಾಗ
ತೇಲಿ-ಹೋ-ದನು
ತೇಲು-ತ್ತಾರೆ
ತೇಲು-ತ್ತಿತ್ತು
ತೇಲು-ತ್ತಿದ್ದ
ತೇಲು-ತ್ತಿ-ದ್ದಾನೆ
ತೇಲು-ವು-ದಕ್ಕೆ
ತೈತ್ತಿ-ರೀ-ಯೋ-ಪ-ನಿ-ಷತ್ತು
ತೊಂಡು-ದ-ನ-ಗ-ಳಿಗೆ
ತೊಂಡೆಯ
ತೊಂಬತ್ತು
ತೊಂಬ-ತ್ತೊಂ
ತೊಂಬ-ತ್ತೊಂ-ಬತ್ತು
ತೊಗಲು
ತೊಟ್ಟ
ತೊಟ್ಟನ್ನು
ತೊಟ್ಟ-ವಳು
ತೊಟ್ಟಿ-ಕ್ಕಿತು
ತೊಟ್ಟಿ-ಕ್ಕು-ತ್ತಿತ್ತು
ತೊಟ್ಟಿ-ಕ್ಕು-ತ್ತಿ-ರುವ
ತೊಟ್ಟಿ-ಕ್ಕುವ
ತೊಟ್ಟಿ-ಕ್ಕು-ವಂ-ತಹ
ತೊಟ್ಟಿ-ಡಿ-ಸುತ್ತಾ
ತೊಟ್ಟಿಡು
ತೊಟ್ಟಿ-ಡುವ
ತೊಟ್ಟಿ-ದ್ದಾಳೆ
ತೊಟ್ಟಿ-ಲಲ್ಲಿ
ತೊಟ್ಟಿ-ಲ-ಲ್ಲಿಟ್ಟು
ತೊಟ್ಟು
ತೊಟ್ಟು-ಗಳೆ
ತೊಡಗಿ
ತೊಡ-ಗಿ-ದನು
ತೊಡ-ಗಿ-ದಳು
ತೊಡ-ಗಿ-ದವು
ತೊಡ-ಗಿದ್ದ
ತೊಡ-ಗಿ-ದ್ದಾಗ
ತೊಡ-ಗಿ-ದ್ದಾನೆ
ತೊಡ-ಗಿ-ದ್ದಾಳೆ
ತೊಡ-ಗಿ-ರಲು
ತೊಡ-ಗಿ-ರು-ತ್ತಿ-ದ್ದನು
ತೊಡ-ಗಿ-ರು-ವನು
ತೊಡ-ಗಿ-ರು-ವಳು
ತೊಡ-ಗಿ-ರು-ವಾಗ
ತೊಡ-ಗಿ-ರು-ವು-ದಾಗಿ
ತೊಡ-ಗಿ-ರು-ವೆನು
ತೊಡ-ಗು-ತ್ತಲೆ
ತೊಡ-ಗುವ
ತೊಡ-ರಿ-ಕೊ-ಳ್ಳ-ದಂತೆ
ತೊಡಿ-ಗೆ-ಗಳಿಂದ
ತೊಡಿಸಿ
ತೊಡೆ
ತೊಡೆ-ಗಳ
ತೊಡೆ-ಗಳನ್ನು
ತೊಡೆ-ಗಳಿಂದ
ತೊಡೆ-ಗಳು
ತೊಡೆ-ಗ-ಳು-ನೋಡಿ
ತೊಡೆ-ತೋಳು
ತೊಡೆ-ದನು
ತೊಡೆಯ
ತೊಡೆ-ಯನ್ನು
ತೊಡೆ-ಯ-ಮೇ-ಲಿ-ಟ್ಟು-ಕೊಂಡು
ತೊಡೆ-ಯ-ಮೇ-ಲಿದ್ದ
ತೊಡೆ-ಯ-ಮೇಲೆ
ತೊಡೆ-ಯಲ್ಲಿ
ತೊಡೆ-ಯಿಂದ
ತೊಡೆ-ಯೇ-ರಿ-ಸಿ-ಕೊಂ-ಡಳು
ತೊತ್ತಿನ
ತೊದ-ಲಿ-ಯೋ-ದೇ-ವರ
ತೊನೆ-ದಾ-ಡು-ತ್ತಿದೆ
ತೊನೆದು
ತೊಯ್ದವು
ತೊರೆ
ತೊರೆ-ಗಳ
ತೊರೆ-ಗಳು
ತೊರೆದ
ತೊರೆ-ದರೆ
ತೊರೆದು
ತೊರೆ-ಯ-ಬೇಕು
ತೊಲಗಿ
ತೊಲ-ಗಿತು
ತೊಲ-ಗಿ-ದು-ದ-ಕ್ಕಾಗಿ
ತೊಲ-ಗಿದೆ
ತೊಲ-ಗಿ-ಸ-ಬೇಕು
ತೊಲ-ಗಿಸು
ತೊಲ-ಗಿ-ಸು-ತ್ತದೆ
ತೊಲ-ಗಿ-ಹೋ-ಗಲು
ತೊಲ-ಗು-ವು-ದಿಲ್ಲ
ತೊಲ-ಗು-ವುದು
ತೊಳಗಿ
ತೊಳ-ಲು-ತ್ತಿ-ರುವ
ತೊಳೆ
ತೊಳೆ-ದನು
ತೊಳೆದು
ತೊಳೆ-ದು-ಕೊಂಡು
ತೊಳೆ-ದು-ಕೊ-ಳ್ಳಲಿ
ತೊಳೆ-ಯದೆ
ತೊಳೆಯು
ತೊಳೆ-ಯುತ್ತ
ತೊಳೆ-ಯುವ
ತೋಚ-ದಂತಾ
ತೋಚ-ದಂ-ತಾ-ಗು-ತ್ತದೆ
ತೋಚದೆ
ತೋಚಿತು
ತೋಚಿ-ದಂತೆ
ತೋಚು-ತ್ತಿಲ್ಲ
ತೋಟ-ಗಳನ್ನು
ತೋಟದ
ತೋಟ-ದಂತೆ
ತೋಟ-ದೊ-ಳಕ್ಕೆ
ತೋಟ-ವಿದೆ
ತೋಡಿ
ತೋಡಿ-ಕೊಂ-ಡಳು
ತೋಡಿ-ಕೊಂ-ಡಿ-ದ್ದಳು
ತೋಡಿ-ಕೊಂಡು
ತೋಮರ
ತೋಯಿ-ಸಿ-ದನು
ತೋರಣ
ತೋರ-ಣ-ಗಳನ್ನು
ತೋರ-ಣ-ಗಳಿಂದ
ತೋರ-ಣ-ಗಳು
ತೋರ-ಣ-ವನ್ನು
ತೋರ-ದಿ-ದ್ದರೆ
ತೋರದೆ
ತೋರ-ಬ-ಲ್ಲ-ವನು
ತೋರ-ಮೊ-ಲೆ-ಗಳ
ತೋರಿ
ತೋರಿತು
ತೋರಿದ
ತೋರಿ-ದಂ-ತಾ-ಗಿದೆ
ತೋರಿ-ದಂತೆ
ತೋರಿ-ದನು
ತೋರಿ-ದು-ದ-ರಿಂದ
ತೋರಿ-ಬಂ-ದಾಗ
ತೋರಿವೆ
ತೋರಿ-ಸ-ತಕ್ಕ
ತೋರಿ-ಸ-ಬೇ-ಕಾ-ದುದು
ತೋರಿ-ಸ-ಬೇಕು
ತೋರಿ-ಸ-ಲಾ-ರದೆ
ತೋರಿಸಿ
ತೋರಿ-ಸಿದ
ತೋರಿ-ಸಿ-ದನು
ತೋರಿ-ಸಿ-ದರು
ತೋರಿ-ಸಿ-ದರೂ
ತೋರಿ-ಸಿ-ದರೆ
ತೋರಿ-ಸಿ-ಯೇ-ಬಿ-ಟ್ಟಿತು
ತೋರಿಸು
ತೋರಿ-ಸುತ್ತಾ
ತೋರಿ-ಸು-ತ್ತಾನೆ
ತೋರಿ-ಸು-ತ್ತಿದ್ದು
ತೋರಿ-ಸು-ತ್ತೇನೆ
ತೋರಿ-ಸುವ
ತೋರಿ-ಸು-ವ-ವ-ನಂತೆ
ತೋರಿ-ಸು-ವ-ವನೂ
ತೋರಿ-ಸು-ವು-ದ-ಕ್ಕಾ-ಗಿಯೆ
ತೋರಿ-ಸು-ವು-ದಕ್ಕೆ
ತೋರು
ತೋರು-ತ್ತದೆ
ತೋರು-ತ್ತಿತ್ತು
ತೋರು-ತ್ತಿದ್ದ
ತೋರು-ತ್ತಿ-ದ್ದರೂ
ತೋರು-ತ್ತಿ-ರುವ
ತೋರು-ತ್ತಿ-ರುವೆ
ತೋರುವ
ತೋರು-ವಂತೆ
ತೋರು-ವು-ದ-ಕ್ಕಾಗಿ
ತೋರು-ವುದೇ
ತೋಳ
ತೋಳ-ಗಳ
ತೋಳನ್ನು
ತೋಳನ್ನೂ
ತೋಳಲ್ಲಿ
ತೋಳಿ-ನ-ಲ್ಲಿಯೂ
ತೋಳಿ-ನಿಂದ
ತೋಳು
ತೋಳು-ಗಳ
ತೋಳು-ಗಳನ್ನು
ತೋಳು-ಗಳಲ್ಲಿ
ತೋಳು-ಗಳಿಂದ
ತೋಳು-ಗ-ಳಿಂ-ದಲೂ
ತೋಳು-ಗ-ಳಿ-ರು-ವಾಗ
ತೋಳು-ಗಳು
ತೋಳು-ಗ-ಳುಳ್ಳ
ತೋಳು-ಗ-ಳೆ-ರ-ಡನ್ನು
ತೋಳು-ಗ-ಳೆ-ರ-ಡನ್ನೂ
ತೋಳೊ-ಳಗೆ
ತೋಳ್ಬಳೆ
ತೋಷ
ತೋಸ-ಲ-ಕ-ಎಂಬ
ತೌರಿಗೆ
ತೌರಿ-ನ-ವ-ರನ್ನು
ತೌರಿ-ನ-ವರೆ-ಲ್ಲರ
ತೌರು
ತೌರುಈ
ತೌರು-ಮ-ನೆಯ
ತೌರು-ಮ-ನೆ-ಯನ್ನು
ತೌರು-ಮ-ನೆ-ಯ-ವರು
ತೌರು-ಮ-ನೆ-ಯಾ-ಗಿ-ದ್ದವು
ತೌರು-ಮ-ನೆ-ಯಾದ
ತೌರು-ಮ-ನೆ-ಯಾ-ದರೂ
ತ್ಕಾರ್ಯ-ಗಳನ್ನೂ
ತ್ತದಷ್ಟೆ
ತ್ತದೆ
ತ್ತದೆಯೆ
ತ್ತದೆಯೋ
ತ್ತನೆಯ
ತ್ತನ್ನು
ತ್ತಲೆ
ತ್ತಲೇ
ತ್ತಳು
ತ್ತವೆ
ತ್ತಾಗಿ
ತ್ತಾನೆ
ತ್ತಾನೆಯೋ
ತ್ತಾಪ-ವಾ-ಯಿತು
ತ್ತಾರಲ್ಲ
ತ್ತಾರೆ
ತ್ತಿಗೆ-ಗಳು
ತ್ತಿಗೆಯ
ತ್ತಿತ್ತು
ತ್ತಿದೆ
ತ್ತಿದ್ದ
ತ್ತಿದ್ದಂ-ತೆಯೆ
ತ್ತಿದ್ದನು
ತ್ತಿದ್ದರು
ತ್ತಿದ್ದರೆ
ತ್ತಿದ್ದ-ವಳು
ತ್ತಿದ್ದವು
ತ್ತಿದ್ದಾನೆ
ತ್ತಿದ್ದಾರೆ
ತ್ತಿದ್ದಿ
ತ್ತಿದ್ದು
ತ್ತಿದ್ದು-ದ-ರಿಂದ
ತ್ತಿದ್ದೆ
ತ್ತಿದ್ದೆ-ವಲ್ಲಾ
ತ್ತಿದ್ದೇನೆ
ತ್ತಿದ್ದೇವೆ
ತ್ತಿರಲಿ
ತ್ತಿರಲು
ತ್ತಿರು-ತ್ತಾರೆ
ತ್ತಿರುವ
ತ್ತಿರು-ವಂತೆ
ತ್ತಿರು-ವಂ-ತೆಯೆ
ತ್ತಿರು-ವನು
ತ್ತಿರು-ವ-ರೆಂಬ
ತ್ತಿರು-ವಾಗ
ತ್ತಿರು-ವಿರಿ
ತ್ತಿರು-ವೆ-ಯಲ್ಲ
ತ್ತಿಲ್ಲ
ತ್ತಿವೆ
ತ್ತೀಯೇನೋ
ತ್ತೀರಾ
ತ್ತೇನೆ
ತ್ತೇವೆ
ತ್ತೊಂದು
ತ್ಥಾಮನು
ತ್ನಿಸಿ
ತ್ಪತ್ತಿಯ
ತ್ಯಜಿ-ಸಲು
ತ್ಯಜಿಸಿ
ತ್ಯಜಿ-ಸಿದ
ತ್ಯಜಿ-ಸಿ-ದನು
ತ್ಯಜಿ-ಸಿ-ದರೆ
ತ್ಯಜಿ-ಸಿದ್ದ
ತ್ಯಜಿ-ಸಿ-ಹೋ-ಗಿದ್ದ
ತ್ಯಜಿ-ಸಿ-ಹೋದ
ತ್ಯಾಗ
ತ್ಯಾಗ-ಮಾ-ಡಿ-ದರೆ
ತ್ಯಾಗ-ವಿಲ್ಲ
ತ್ರಯ-ರನ್ನು
ತ್ರಸ-ದ್ದಸ್ಯು
ತ್ರಿಕ-ದೃಷ್ಟಿ
ತ್ರಿಕಾಲ
ತ್ರಿಕಾ-ಲ-ಗ-ಳ-ಲ್ಲಿಯೂ
ತ್ರಿಕಾ-ಲ-ಜ್ಞಾ-ನ-ವಿ-ರಲಿ
ತ್ರಿಕಾ-ಲ-ಜ್ಞಾ-ನಿ-ಗ-ಳಾದ
ತ್ರಿಕೂ-ಟ-ಗಿ-ರಿ-ಯೆ-ಲ್ಲವೂ
ತ್ರಿಕೂ-ಟ-ಪ-ರ್ವ-ತ-ದಲ್ಲಿ
ತ್ರಿಕೂ-ಟ-ವೆಂಬ
ತ್ರಿತ
ತ್ರಿಧಾ-ಮ-ಽವತು
ತ್ರಿಪುಟೀ
ತ್ರಿಪು-ರ-ನನ್ನು
ತ್ರಿಪು-ರಾಂ-ತ-ಕ-ನಾದ
ತ್ರಿಮೂರ್ತಿ
ತ್ರಿಮೂ-ರ್ತಿ-ಗ-ಳನ್ನೆ
ತ್ರಿಮೂ-ರ್ತಿ-ಗಳಲ್ಲಿ
ತ್ರಿಮೂ-ರ್ತಿ-ಗ-ಳಾದ
ತ್ರಿಮೂ-ರ್ತಿ-ಗಳು
ತ್ರಿಮೂ-ರ್ತಿ-ಸ್ವ-ರೂಪ
ತ್ರಿಮೂ-ರ್ತಿ-ಸ್ವ-ರೂ-ಪನು
ತ್ರಿಯು-ಗಾಯ
ತ್ರಿಲೋ-ಕ-ಸಂ-ಚಾ-ರಿ-ಗ-ಳಾದ
ತ್ರಿಲೋ-ಕ-ಸಂ-ಚಾ-ರಿ-ಯಾದ
ತ್ರಿಲೋ-ಕ-ಸಾ-ಮ್ರಾ-ಜ್ಯ-ವನ್ನು
ತ್ರಿಲೋ-ಕ-ಸುಂ-ದರಿ
ತ್ರಿವಕ್ರೆ
ತ್ರಿವ-ಕ್ರೆಯ
ತ್ರಿವಿ
ತ್ರಿವಿ-ಕ್ರಮಃ
ತ್ರಿವಿ-ಕ್ರ-ಮ-ನಂತೆ
ತ್ರಿವಿ-ಕ್ರ-ಮ-ನಾದ
ತ್ರಿವಿ-ಕ್ರ-ಮನು
ತ್ರಿಶಂಕು
ತ್ರಿಶಂ-ಕು-ವಿನ
ತ್ರಿಶೂಲ
ತ್ರಿಶೂ-ಲ-ಕ್ಕಾ-ಗಲಿ
ತ್ರಿಶೂ-ಲ-ವನ್ನು
ತ್ರೀಕೋಣಾ
ತ್ರೇತಾ-ಗ್ನಿ-ಗಳು
ತ್ರೇತಾ-ಗ್ನಿಗೆ
ತ್ರೇತಾ-ಯು-ಗ-ದಲ್ಲಿ
ತ್ರೇತಾ-ಯು-ಗ-ವಾ-ದರೂ
ತ್ರೇತಿ
ತ್ರೇತೆ-ಗಳ
ತ್ವಂ
ತ್ವಯಿ
ತ್ವಷ್ಟೃ
ತ್ವಷ್ಟೃ-ವಿನ
ತ್ವಷ್ಟೃವು
ತ್ವಾಂ
ತ್ವಾಚ-ರಿ-ತ-ಮಾ-ಸ್ಮಾ-ಭಿ-ರ್ಮ-ಲ-ಯಾ-ನಲ
ತ್ವಾದ್ಯ-ಮೀ-ಶ್ವರಂ
ತ್ಸಾಹ-ದಿಂದ
ಥಳ-ಥಳ
ಥಳ-ಥ-ಳಿ-ಸುವ
ದಂ
ದಂಗಾ-ಗಿ-ಹೋ-ಗಿ-ದ್ದನು
ದಂಗು-ಬ-ಡಿ-ದು-ಹೋದ
ದಂಟಿ-ನಂತೆ
ದಂಟು-ಗಳನ್ನು
ದಂಡ
ದಂಡ-ಕ-ಮಂ-ಡ-ಲ-ಗಳು
ದಂಡ-ಕಾರಣ್ಯ-ದ-ಲ್ಲಿ-ದ್ದಾಗ
ದಂಡ-ಕ್ಕಾ-ಗಲಿ
ದಂಡ-ಗ-ಳೆಂಬ
ದಂಡ-ದಿಂದ
ದಂಡ-ನೀತಿ
ದಂಡನೆ
ದಂಡ-ನೆ-ಯಿಂದ
ದಂಡ-ಪ್ರ-ಹಾ-ರವೇ
ದಂಡ-ವನ್ನು
ದಂಡ-ವನ್ನೂ
ದಂಡಿ
ದಂಡಿ-ಸ-ಲಿಲ್ಲ
ದಂಡಿಸು
ದಂಡಿ-ಸುವ
ದಂಡೆತ್ತಿ
ದಂಡೆ-ತ್ತಿ-ಬಂ-ದಿ-ದ್ದಾನೆ
ದಂಡೆ-ತ್ತಿ-ಬ-ರಲು
ದಂಡೆ-ತ್ತಿ-ಹೋಗಿ
ದಂಡೆ-ತ್ತಿ-ಹೋ-ದನು
ದಂಡೆ-ಯನ್ನು
ದಂತ
ದಂತ-ಗಳ
ದಂತ-ಪಂ-ಕ್ತಿ-ಯನ್ನು
ದಂತ-ವಕ್ತ್ರ
ದಂತ-ವ-ಕ್ತ್ರನ
ದಂತ-ವ-ಕ್ತ್ರನು
ದಂತಹ
ದಂತಾ-ಗು-ತ್ತದೆ
ದಂತಾ-ಯಿತು
ದಂತಿತ್ತು
ದಂತಿ-ದ್ದನು
ದಂತಿ-ರುವ
ದಂತಿ-ರು-ವುದು
ದಂತೆ
ದಂತೆಯೆ
ದಂಪತಿ
ದಂಪ-ತಿ-ಗಳ
ದಂಪ-ತಿ-ಗಳನ್ನು
ದಂಪ-ತಿ-ಗ-ಳಾ-ಗಿ-ದ್ದರು
ದಂಪ-ತಿ-ಗ-ಳಾದ
ದಂಪ-ತಿ-ಗಳಿಂದ
ದಂಪ-ತಿ-ಗ-ಳಿಗೆ
ದಂಪ-ತಿ-ಗಳು
ದಂಪ-ತಿ-ಗಳೇ
ದಂಶವೇ
ದಂಷ್ಟ್ರಿಭ್ಯೋ
ದಕ್ಕ-ಲಿಲ್ಲ
ದಕ್ಕಾಗಿ
ದಕ್ಕಾ-ಗಿಯೆ
ದಕ್ಕಿ
ದಕ್ಕಿತು
ದಕ್ಕು-ವಂತೆ
ದಕ್ಕು-ವ-ವ-ನಲ್ಲ
ದಕ್ಕು-ವ-ವ-ಳಲ್ಲ
ದಕ್ಕೂ
ದಕ್ಕೆ
ದಕ್ಷ
ದಕ್ಷ-ಪ್ರ-ಸೂ-ತಿ-ಯ-ರಿಗೆ
ದಕ್ಷನ
ದಕ್ಷ-ನನ್ನು
ದಕ್ಷ-ನನ್ನೂ
ದಕ್ಷ-ನಿಗೆ
ದಕ್ಷನು
ದಕ್ಷ-ನೆಂದೇ
ದಕ್ಷನೇ
ದಕ್ಷ-ಪ್ರಜಾ
ದಕ್ಷ-ಪ್ರ-ಜಾ-ಪತಿ
ದಕ್ಷ-ಪ್ರ-ಜಾ-ಪ-ತಿಗೆ
ದಕ್ಷ-ಪ್ರ-ಜಾ-ಪ-ತಿಯ
ದಕ್ಷ-ಪ್ರ-ಜಾ-ಪ-ತಿ-ಯನ್ನು
ದಕ್ಷ-ಪ್ರ-ಜಾ-ಪ-ತಿಯು
ದಕ್ಷ-ಬ್ರಹ್ಮ
ದಕ್ಷ-ಬ್ರ-ಹ್ಮನ
ದಕ್ಷ-ಬ್ರ-ಹ್ಮ-ನನ್ನು
ದಕ್ಷ-ಬ್ರ-ಹ್ಮನೂ
ದಕ್ಷ-ಬ್ರ-ಹ್ಮನೇ
ದಕ್ಷ-ಯ-ಜ್ಞದ
ದಕ್ಷ-ಯ-ಜ್ಞ-ದಲ್ಲಿ
ದಕ್ಷಿಣ
ದಕ್ಷಿ-ಣ-ಕ-ರ್ಣಾ-ಟ-ಕದ
ದಕ್ಷಿ-ಣಕ್ಕೂ
ದಕ್ಷಿ-ಣಕ್ಕೆ
ದಕ್ಷಿ-ಣದ
ದಕ್ಷಿ-ಣ-ದಲ್ಲಿ
ದಕ್ಷಿ-ಣ-ದ-ಲ್ಲಿ-ರು-ವಾಗ
ದಕ್ಷಿ-ಣ-ದೇ-ಶದ
ದಕ್ಷಿ-ಣ-ಭಾ-ರ-ತ-ದಲ್ಲಿ
ದಕ್ಷಿ-ಣಾಗ್ನಿ
ದಕ್ಷಿ-ಣಾ-ಗ್ನಿ-ಯಲ್ಲಿ
ದಕ್ಷಿಣೆ
ದಕ್ಷಿ-ಣೆ-ಕೊಟ್ಟು
ದಕ್ಷಿ-ಣೆ-ಗಳನ್ನು
ದಕ್ಷಿ-ಣೆ-ಗಳಿಂದ
ದಕ್ಷಿ-ಣೆ-ಯೆಂಬ
ದಗ್ದೃ-ಧಿ-ಸೈ-ನ್ಯ-ಮಾಶು
ದಗ್ಧ-ನಾ-ಗ-ಲಿಲ್ಲ
ದಗ್ಧ-ವಾಗಿ
ದಗ್ಧ-ವಾ-ಗಿ-ಹೋ-ಯಿತು
ದಗ್ಧಿದಂ
ದಟ್ಟ
ದಟ್ಟ-ಡಿ-ಯಿ-ಡುತ್ತಾ
ದಟ್ಟ-ದ-ರಿ-ದ್ರ-ನಾದ
ದಟ್ಟ-ನಾಗಿ
ದಟ್ಟ-ವಾಗಿ
ದಟ್ಟ-ವಾದ
ದಡ
ದಡಕ್ಕೆ
ದಡ-ಗಳನ್ನು
ದಡ-ಗ-ಳೆಂ-ದರೆ
ದಡದ
ದಡ-ದಲ್ಲಿ
ದಡ-ದ-ಲ್ಲಿದ್ದ
ದಡ-ದ-ಲ್ಲಿ-ರುವ
ದಡ-ವನ್ನು
ದಡ-ವೇರಿ
ದಡ್ಡಿ
ದಣಿ
ದಣಿ-ದರು
ದಣಿ-ದಿ-ರು-ವೆ-ಯ-ಲ್ಲವೆ
ದಣಿದು
ದಣಿ-ಯು-ವನು
ದಣಿ-ವಿ-ಲ್ಲದೆ
ದತ್ತ
ದತ್ತ-ಸ್ತ್ವ-ಯೋ-ಗಾ-ದಥ
ದತ್ತಾ-ತ್ರೇಯ
ದತ್ತಾ-ತ್ರೇ-ಯನ
ದತ್ತಾ-ತ್ರೇ-ಯನು
ದತ್ತಾ-ತ್ರೇ-ಯ-ನೆಂಬ
ದತ್ತಾ-ತ್ರೇ-ಯಾ-ವ-ತಾರ
ದತ್ತಾ-ಪ-ಹಾ-ರದ
ದದೃ-ಶಿಮ
ದಧಾ-ನೋ-ಽಷ್ಟ-ಗು-ಣೋ-ಷ್ಟ-ಬಾ-ಹುಃ
ದಧೀಚಿ
ದಧೀ-ಚಿ-ಋ-ಷಿಯ
ದಧೀ-ಚಿಗೂ
ದಧೀ-ಚಿಯ
ದನ
ದನ-ಕ-ರು-ಗ-ಳಿ-ಗೆಲ್ಲ
ದನ-ಕ-ರು-ಗ-ಳಿ-ರಲಿ
ದನ-ಗಳ
ದನ-ಗಳನ್ನು
ದನ-ಗಳನ್ನೂ
ದನ-ಗಳು
ದನ-ಗ-ಳೆಲ್ಲ
ದನದ
ದನ-ವನ್ನು
ದನಿ
ದನಿ-ಯನ್ನು
ದನಿ-ಯಲ್ಲಿ
ದನಿ-ಯಿಂದ
ದನು
ದನೋ
ದನ್ನಾ-ಗಲಿ
ದನ್ನು
ದನ್ನೂ
ದನ್ನೆಲ್ಲ
ದಪ-ದಪ
ದಪ್ಪ
ದಪ್ಪ-ನಾದ
ದಪ್ಪ-ವಿದ್ದ
ದಮ
ದಮ-ಘೋ-ಷ-ಶಿ-ಶು-ಪಾಲ
ದಮನ
ದಮ-ನ-ಮಾಡಿ
ದಮ್ಮ-ನನ್ನು
ದಯ
ದಯ-ದಿಂದ
ದಯ-ಪಾ-ಲಿ-ಸಿ-ದರು
ದಯ-ಪಾ-ಲಿಸು
ದಯ-ಪಾ-ಲಿ-ಸು-ವಿ-ರಾ-ದರೆ
ದಯ-ಮಾಡಿ
ದಯ-ಮಾ-ಡಿಸಿ
ದಯ-ವಿಟ್ಟು
ದಯಾ-ಗು-ಣವೇ
ದಯಾ-ದಾ-ಕ್ಷಿ-ಣ್ಯ-ವಿ-ಲ್ಲದೆ
ದಯಾ-ಮಯ
ದಯಾ-ಮ-ಯ-ನಾದ
ದಯಾ-ರ-ಸ-ವನ್ನು
ದಯಾಳು
ದಯಾ-ಳು-ಗಳು
ದಯಾ-ಳು-ವಾದ
ದಯೆ
ದಯೆಗೆ
ದಯೆ-ಯಿಂದ
ದರ-ದರ
ದರನು
ದರಾ-ಽರಿ-ಚ-ರ್ಮ-ಽಸಿ-ಗ-ದೇ-ಷು-ಚಾಪ
ದರಿಂದ
ದರಿ-ದ್ರನ
ದರಿ-ದ್ರ-ನಾ-ಗಿ-ರಲಿ
ದರಿ-ದ್ರರೂ
ದರು
ದರೂ
ದರೆ
ದರೋ
ದರ್ಪ-ಣ-ರೂ-ಪ-ವಾಗಿ
ದರ್ಪ-ದಿಂದ
ದರ್ಭಾ-ಸ-ನ-ದ-ಮೇಲೆ
ದರ್ಭೆ
ದರ್ಭೆ-ಗಳನ್ನು
ದರ್ಭೆಯ
ದರ್ಭೆ-ಯನ್ನು
ದರ್ಭೆ-ಯನ್ನೂ
ದರ್ಭೆ-ಯಿಂದ
ದರ್ಶ
ದರ್ಶನ
ದರ್ಶ-ನ-ಕ್ಕಾಗಿ
ದರ್ಶ-ನಕ್ಕೂ
ದರ್ಶ-ನಕ್ಕೆ
ದರ್ಶ-ನ-ಗಳಿಂದ
ದರ್ಶ-ನ-ದಿಂದ
ದರ್ಶ-ನ-ದಿಂ-ದಲೆ
ದರ್ಶ-ನ-ಭಾಗ್ಯ
ದರ್ಶ-ನ-ಭಾ-ಗ್ಯ-ವನ್ನು
ದರ್ಶ-ನ-ಮಾಡಿ
ದರ್ಶ-ನ-ಮಾ-ಡಿ-ಸಿ-ದನು
ದರ್ಶ-ನ-ವನ್ನು
ದರ್ಶ-ನ-ವಾ-ಗು-ತ್ತಲೆ
ದರ್ಶ-ನ-ವಾ-ದಂ-ತಾ-ಯಿತು
ದರ್ಶ-ನ-ವಾ-ದು-ದ-ರಿಂದ
ದರ್ಶ-ನ-ವಾ-ದುದೇ
ದರ್ಶ-ನ-ವಾ-ಯಿತು
ದರ್ಶ-ನ-ವಿ-ತ್ತನು
ದರ್ಶ-ನ-ವಿತ್ತು
ದರ್ಶ-ನ-ವಿತ್ತೆ
ದರ್ಶ-ನವೆ
ದರ್ಶ-ನವೇ
ದರ್ಶ-ನ-ಶಾಸ್ತ್ರ
ದಲೂ
ದಲ್ಲ
ದಲ್ಲಿ
ದಲ್ಲಿದ್ದ
ದಲ್ಲಿ-ದ್ದರೂ
ದಲ್ಲಿಯೂ
ದಲ್ಲಿಯೆ
ದಲ್ಲಿಯೇ
ದಲ್ಲಿ-ರುವ
ದಲ್ಲಿ-ರು-ವಾಗ
ದಲ್ಲಿ-ರು-ವು-ದುಂಟು
ದಳು
ದಳ್ಳುರಿ
ದವ-ನಂತೆ
ದವ-ನಾಗಿ
ದವನು
ದವನೆ
ದವರ
ದವ-ರ-ನ್ನೆಲ್ಲ
ದವ-ರಾ-ಗಿದ್ದು
ದವ-ರಾದ
ದವ-ರಿಗೆ
ದವ-ರಿ-ಗೆಲ್ಲ
ದವರು
ದವರೂ
ದವ-ರೂ-ಕು-ದುರೆ
ದವರೆಲ್ಲ
ದವರೆ-ಲ್ಲ-ಹೆಂ-ಗ-ಸರು
ದವ-ಳನ್ನು
ದವು
ದಶ-ದಿ-ಕ್ಕು-ಗಳನ್ನು
ದಶ-ದಿ-ಕ್ಕು-ಗಳನ್ನೂ
ದಶ-ದಿ-ಕ್ಕು-ಗಳೂ
ದಶ-ಮ-ಸ್ಕಂ-ಧ-ದಲ್ಲಿ
ದಶ-ಮ-ಸ್ಕಂ-ಧ-ವನ್ನು
ದಶ-ರಥ
ದಶ-ರ-ಥನ
ದಶ-ರ-ಥನು
ದಶ-ಲ-ಕ್ಷ-ಣ-ಗಳಿಂದ
ದಶಾಂ
ದಷ್ಟ-ಪುಷ್ಟ
ದಷ್ಟು
ದಷ್ಟೂ
ದಹ-ನ-ಕಾ-ಲ-ದಲ್ಲಿ
ದಹ-ನ-ಶ-ಕ್ತಿ-ಯುಂ-ಟಾಗಿ
ದಹ-ನ-ಶ-ಕ್ತಿಯೂ
ದಹ-ರಮ್
ದಹಿ-ಸದೆ
ದಹಿ-ಸ-ಲಿ-ಲ್ಲ-ವಾ-ಗಿ-ದ್ದರೂ
ದಹಿಸು
ದಹಿ-ಸು-ವಂತೆ
ದಾಂಪತ್ಯ
ದಾಕ್ಷಾ-ಯಿಣಿ
ದಾಗ
ದಾಗಲೂ
ದಾಗಿ
ದಾಗಿತ್ತು
ದಾಗಿದೆ
ದಾಗಿ-ರು-ವುದನ್ನು
ದಾಚೆ
ದಾಟ-ಬ-ಲ್ಲರು
ದಾಟ-ಬೇ-ಕಾ-ಗಿದೆ
ದಾಟ-ಲೆಂದು
ದಾಟಿ
ದಾಟಿ-ದರು
ದಾಟಿ-ದರೆ
ದಾಟಿ-ದ್ದಾನೆ
ದಾಟಿ-ರುವೆ
ದಾಟಿ-ಸ-ಬ-ಹುದು
ದಾಟಿ-ಸುವ
ದಾಟಿ-ಹೋ-ಗ-ಲಾ-ರದು
ದಾಟು-ತ್ತಿ-ದ್ದಂ-ತೆಯೆ
ದಾಟು-ವು-ದ-ಕ್ಕಾ-ದೀತೆ
ದಾಟು-ವು-ದಕ್ಕೆ
ದಾಟು-ವೆ-ನೆಂಬ
ದಾಡಿ
ದಾಡಿ-ದರು
ದಾಡೆ
ದಾಡೆ-ಇಂ-ತಹ
ದಾಡೆಈ
ದಾಡೆ-ಗಳು
ದಾದ
ದಾದರೂ
ದಾದರೆ
ದಾದಿ
ದಾನ
ದಾನ-ಕಾ-ರ್ಯ-ವನ್ನು
ದಾನ-ಗಳನ್ನು
ದಾನ-ಗು-ಣ-ಗಳು
ದಾನ-ದ-ಕ್ಷಿ-ಣೆ-ಗಳನ್ನು
ದಾನ-ದ-ಕ್ಷಿ-ಣೆ-ಗಳಿಂದ
ದಾನ-ಧರ್ಮ
ದಾನ-ಧ-ರ್ಮ-ಗ-ಳ-ನ್ನಿತ್ತು
ದಾನ-ಧ-ರ್ಮ-ಗಳನ್ನು
ದಾನ-ಧ-ರ್ಮಾ-ದಿ-ಗಳನ್ನು
ದಾನ-ಧಾರೆ
ದಾನ-ಧಾ-ರೆ-ಯನ್ನು
ದಾನ-ಧಾ-ರೆಯು
ದಾನ-ಮಾಡ
ದಾನ-ಮಾಡಿ
ದಾನ-ಮಾ-ಡಿ-ದನು
ದಾನ-ಮಾ-ಡಿದೆ
ದಾನ-ಮಾ-ಡಿದ್ದ
ದಾನ-ಮಾ-ಡಿ-ದ್ದೇನೆ
ದಾನ-ಮಾ-ಡು-ವು-ದಿಲ್ಲ
ದಾನವ
ದಾನ-ವ-ಚ-ಕ್ರ-ವ-ರ್ತಿ-ಯಾದ
ದಾನ-ವ-ನಾದ
ದಾನ-ವನ್ನು
ದಾನ-ವರ
ದಾನ-ವ-ರಲ್ಲಿ
ದಾನ-ವ-ರಾ-ಜ-ನಲ್ಲಿ
ದಾನ-ವ-ರಾದ
ದಾನ-ವ-ರಿಗೆ
ದಾನ-ವ-ರಿ-ಗೆಲ್ಲ
ದಾನ-ವರು
ದಾನ-ವರೂ
ದಾನ-ವರೆಲ್ಲ
ದಾನ-ವರೇ
ದಾನ-ವ-ರೊ-ಡನೆ
ದಾನ-ವ-ಶಿ-ಲ್ಪಿ-ಯಾದ
ದಾನ-ವ-ಸಂ-ತಾ-ನಕ್ಕೆ
ದಾನ-ವಾಗಿ
ದಾನ-ವಾರಿ
ದಾನ-ವೇಂದ್ರ
ದಾನ-ವ್ರತ
ದಾನಿ
ದಾಬು
ದಾಮೋ
ದಾಮೋ-ದ-ರೋ-ಽವ್ಯಾ-ದ-ನು-ಸಂಧ್ಯಂ
ದಾಯ-ಕ-ನಾ-ಗು-ತ್ತಾನೆ
ದಾಯ-ಕ-ನಾದ
ದಾಯಾ-ದಿ-ಗ-ಳಾದ
ದಾಯಾ-ದಿ-ಮಾ-ತ್ಸ-ರ್ಯ-ದಿಂದ
ದಾರ-ವನ್ನು
ದಾರಾ
ದಾರಿ
ದಾರಿ-ಗ-ರನ್ನು
ದಾರಿಗೆ
ದಾರಿ-ತಪ್ಪಿ
ದಾರಿ-ತ-ಪ್ಪಿ-ದವು
ದಾರಿ-ತ-ಪ್ಪಿ-ದೆವು
ದಾರಿ-ತೋ-ರ-ಬೇಕು
ದಾರಿ-ಮಾ-ಡಿತು
ದಾರಿಯ
ದಾರಿ-ಯಲ್ಲಿ
ದಾರಿ-ಯೇನೂ
ದಾರಿ-ಸ್ವ-ಯೂ-ಥ-ಕ-ಲಹ
ದಾರುಕ
ದಾರು-ಕ-ನನ್ನು
ದಾರು-ಕ-ನಿಗೆ
ದಾರು-ಕನು
ದಾರು-ಕ-ನೆಂಬ
ದಾಳಿ
ದಾಳಿ-ಯಿಂದ
ದಾಳು
ದಾಸ
ದಾಸ-ದಾ-ಸಿ-ಯ-ರಿ-ದ್ದರೂ
ದಾಸ-ದಾ-ಸಿ-ಯರು
ದಾಸ-ದಾ-ಸಿ-ಯ-ರು-ಎ-ಲ್ಲ-ದ-ರ-ಲ್ಲಿಯೂ
ದಾಸ-ದಾ-ಸಿ-ಯರೂ
ದಾಸ-ದಾ-ಸಿ-ಯುರು
ದಾಸ-ನಂ-ತಿ-ರು-ವುದು
ದಾಸ-ನಾ-ಗು-ತ್ತಾನೆ
ದಾಸ-ನಿಗೆ
ದಾಸ-ರಾದ
ದಾಸಾ-ನು-ದಾ-ಸ-ನಾ-ಗಲು
ದಾಸಿಯ
ದಾಸಿ-ಯಂತೆ
ದಾಸಿ-ಯನ್ನು
ದಾಸಿ-ಯ-ರನ್ನು
ದಾಸಿ-ಯ-ರಾದ
ದಾಸಿ-ಯ-ರಾ-ದೆವು
ದಾಸಿ-ಯ-ರಿಂದ
ದಾಸಿ-ಯ-ರಿ-ಗಿಂ-ತಲೂ
ದಾಸಿ-ಯ-ರಿ-ದ್ದರೂ
ದಾಸಿ-ಯರು
ದಾಸಿ-ಯ-ರೆಂದು
ದಾಸಿ-ಯ-ರೆಲ್ಲ
ದಾಸಿ-ಯ-ರೊ-ಡನೆ
ದಾಸಿ-ಯಾಗಿ
ದಾಸಿ-ಯಾ-ಗಿ-ರ-ಬೇಕು
ದಾಸಿ-ಯಾದ
ದಾಸಿ-ಯಾದೆ
ದಾಸಿಯು
ದಾಸೀ-ಪುತ್ರ
ದಾಸ್ಯಂ
ದಾಹ
ದಾಹ-ವನ್ನು
ದಿಂದ
ದಿಂದಲೂ
ದಿಂದಲೆ
ದಿಂದಲೇ
ದಿಂದಲ್ಲ
ದಿಂದಿರು
ದಿಂಬಿನ
ದಿಂಬು-ಗಳನ್ನು
ದಿಕ್ಕನ್ನು
ದಿಕ್ಕನ್ನೆ
ದಿಕ್ಕನ್ನೇ
ದಿಕ್ಕಾ-ಪಾ-ಲಾಗ
ದಿಕ್ಕಾ-ಪಾ-ಲಾಗಿ
ದಿಕ್ಕಾ-ಪಾ-ಲಾ-ಗು-ವಂತೆ
ದಿಕ್ಕಾರು
ದಿಕ್ಕಿಗೂ
ದಿಕ್ಕಿಗೆ
ದಿಕ್ಕಿ-ನಂತೆ
ದಿಕ್ಕಿ-ನತ್ತ
ದಿಕ್ಕಿ-ನಲ್ಲಿ
ದಿಕ್ಕಿ-ನಿಂದ
ದಿಕ್ಕಿಲ್ಲ
ದಿಕ್ಕಿ-ಲ್ಲದ
ದಿಕ್ಕಿ-ಲ್ಲ-ದ-ವ-ನಾದ
ದಿಕ್ಕಿ-ಲ್ಲ-ದ-ವರ
ದಿಕ್ಕು
ದಿಕ್ಕು-ಗಳ
ದಿಕ್ಕು-ಗಳನ್ನು
ದಿಕ್ಕು-ಗಳನ್ನೆಲ್ಲ
ದಿಕ್ಕು-ಗ-ಳ-ಲ್ಲಿಯೂ
ದಿಕ್ಕು-ಗ-ಳಿಂ-ದಲೂ
ದಿಕ್ಕು-ಗ-ಳಿಗೂ
ದಿಕ್ಕು-ಗ-ಳಿಗೆ
ದಿಕ್ಕು-ಗಳು
ದಿಕ್ಕು-ಗ-ಳೆ-ಲ್ಲವೂ
ದಿಕ್ಕು-ಗ-ಳೆಲ್ಲಾ
ದಿಕ್ಕು-ಗಳೇ
ದಿಕ್ಕು-ಗ-ಳೊಂದೂ
ದಿಕ್ಕು-ದಿ-ಕ್ಕಿಗೂ
ದಿಕ್ಕು-ದಿ-ಕ್ಕಿಗೆ
ದಿಕ್ಕು-ದಿ-ಕ್ಕು-ಗ-ಳ-ಲ್ಲಿಯೂ
ದಿಕ್ಕು-ದಿ-ಕ್ಕು-ಗಳಿಂದ
ದಿಕ್ಕು-ಯಾ-ವುದೂ
ದಿಕ್ಕೆಂದು
ದಿಕ್ಕೆಂ-ದು-ಕೊಂಡು
ದಿಕ್ದಿ-ಗಂ-ತ-ಗಳನ್ನು
ದಿಕ್ಪಾ-ಲ-ಕನು
ದಿಕ್ಪಾ-ಲ-ಕ-ರನ್ನು
ದಿಕ್ಷೂ-ರ್ಧ್ವ-ಮ-ದ-ಸ್ಸ-ಮಂ-ತಾತ್
ದಿಗಂ-ಬ-ರ-ರಾಗಿ
ದಿಗಿ-ಲಾ-ಗು-ತ್ತದೆ
ದಿಗಿ-ಲಾ-ಯಿತು
ದಿಗಿಲು
ದಿಗ್ಗ-ಜ-ಗಳನ್ನು
ದಿಗ್ಗ-ಜ-ಗಳೂ
ದಿಗ್ಗನೆ
ದಿಗ್ಗ-ನೆದ್ದು
ದಿಗ್ದ-ರ್ಶ-ನ-ವಾ-ಗಿದೆ
ದಿಗ್ದ-ರ್ಶಿ-ಸು-ತ್ತವೆ
ದಿಗ್ದ-ರ್ಶಿ-ಸುವ
ದಿಗ್ಬಂಧಃ
ದಿಗ್ಬಲಿ
ದಿಗ್ಭಾ-ಗ-ದಲ್ಲಿ
ದಿಗ್ವಿ-ಜಯ
ದಿಗ್ವಿ-ಜ-ಯ-ಕ್ಕಾಗಿ
ದಿಗ್ವಿ-ಜ-ಯಕ್ಕೆ
ದಿಗ್ವಿ-ಜ-ಯ-ಗಳನ್ನು
ದಿಗ್ವಿ-ಜ-ಯ-ದಲ್ಲಿ
ದಿಟ-ವಾದ
ದಿಟ-ವಾ-ದರೆ
ದಿಟವೆ
ದಿಟ್ಟಿ
ದಿಟ್ಟಿ-ಯಿಂದ
ದಿಟ್ಟಿ-ಯಿಂ-ದಲೇ
ದಿಟ್ಟಿಸಿ
ದಿಣ್ಣೆಯ
ದಿತಿ
ದಿತಿಗೆ
ದಿತಿಯ
ದಿತಿಯು
ದಿದ್ದರೂ
ದಿದ್ದರೆ
ದಿದ್ದ-ವ-ನೆಂ-ದರೆ
ದಿದ್ದುದು
ದಿನ
ದಿನಕ್ಕೆ
ದಿನ-ಕ್ಕೊಂದು
ದಿನ-ಗಳ
ದಿನ-ಗಳನ್ನು
ದಿನ-ಗ-ಳ-ಮೇಲೆ
ದಿನ-ಗಳಲ್ಲಿ
ದಿನ-ಗ-ಳ-ಲ್ಲಿಯೆ
ದಿನ-ಗ-ಳ-ಲ್ಲಿಯೇ
ದಿನ-ಗ-ಳ-ವ-ರೆಗೆ
ದಿನ-ಗ-ಳಾಗಿ
ದಿನ-ಗ-ಳಾ-ಗಿತ್ತು
ದಿನ-ಗ-ಳಾ-ಗು-ತ್ತಲೆ
ದಿನ-ಗ-ಳಾ-ಗು-ವು-ದಕ್ಕೂ
ದಿನ-ಗ-ಳಿಗೆ
ದಿನ-ಗ-ಳಿ-ಗೊಮ್ಮೆ
ದಿನ-ಗಳು
ದಿನ-ಗಳೂ
ದಿನ-ಗಳೇ
ದಿನ-ಗ-ಳೊ-ಳ-ಗಾಗಿ
ದಿನ-ಗೊ-ಳಿ-ಗೊಮ್ಮೆ
ದಿನ-ಚರಿ
ದಿನ-ಚ-ರಿಯ
ದಿನ-ಚ-ರಿ-ಯಂತೆ
ದಿನದ
ದಿನ-ದಂತೆ
ದಿನ-ದ-ವ-ರೆಗೆ
ದಿನ-ದಿ-ನಕೂ
ದಿನ-ದಿ-ನಕ್ಕೂ
ದಿನ-ದಿ-ನಕ್ಕೆ
ದಿನ-ದಿ-ನವೂ
ದಿನ-ವನ್ನು
ದಿನ-ವನ್ನೂ
ದಿನವೂ
ದಿನ-ವೆಲ್ಲ
ದಿನವೇ
ದಿನ-ವೊಂ-ದರ
ದಿಬ್ಬಣ
ದಿಬ್ಬ-ಣದ
ದಿಬ್ಬ-ಣವು
ದಿರದು
ದಿರ-ಬೇ-ಕೆಂದು
ದಿರಲಿ
ದಿರು
ದಿರುವ
ದಿರುವೆ
ದಿರು-ವೆನು
ದಿರೆಂ-ದರೆ
ದಿಲೀಪ
ದಿಲೀ-ಪನ
ದಿಲ್ಲ
ದಿಲ್ಲ-ವೆಂದು
ದಿವಂ-ಗತ
ದಿವಂ-ಗ-ತ-ನಾದ
ದಿವಸ
ದಿವ-ಸದ
ದಿವಿ
ದಿವ್ಯ
ದಿವ್ಯಂ
ದಿವ್ಯ-ಕ-ಥಾ-ಮೃ-ತ-ವನ್ನು
ದಿವ್ಯ-ಕೃತಿ
ದಿವ್ಯ-ಕೃ-ತಿ-ಯನ್ನು
ದಿವ್ಯ-ಜ್ಞಾ-ನಿ-ಯಾಗಿ
ದಿವ್ಯ-ತೇ-ಜಸ್ಸು
ದಿವ್ಯ-ದ-ರ್ಶ-ನ-ದಿಂದ
ದಿವ್ಯ-ದೃ-ಷ್ಟಿ-ಯನ್ನು
ದಿವ್ಯ-ನಾಮ
ದಿವ್ಯ-ಪು-ರು-ಷನು
ದಿವ್ಯ-ಪು-ರು-ಷರೆ
ದಿವ್ಯ-ಮಂ-ಗಳ
ದಿವ್ಯ-ಮಂ-ಗ-ಳ-ವಿ-ಗ್ರ-ಹ-ವನ್ನು
ದಿವ್ಯ-ಮ-ನೋ-ಹ-ರ-ವಾದ
ದಿವ್ಯ-ಮೂರ್ತಿ
ದಿವ್ಯ-ಮೂ-ರ್ತಿಯ
ದಿವ್ಯ-ಮೂ-ರ್ತಿ-ಯನ್ನು
ದಿವ್ಯ-ಮೂ-ರ್ತಿಯು
ದಿವ್ಯ-ರ-ಥ-ಗ-ಳ-ನ್ನೇರಿ
ದಿವ್ಯ-ರೂ-ಪ-ವನ್ನು
ದಿವ್ಯ-ರೂ-ಪ-ವುಂ-ಟಾ-ಯಿತು
ದಿವ್ಯ-ಲೀ-ಲೆ-ಗಳನ್ನು
ದಿವ್ಯ-ಲೋ-ಕ-ದಿಂದ
ದಿವ್ಯ-ವ-ಸ್ತ್ರ-ಗಳನ್ನೂ
ದಿವ್ಯ-ವಾಗಿ
ದಿವ್ಯ-ವಾದ
ದಿವ್ಯ-ವಾ-ದುವು
ದಿವ್ಯ-ವಿ-ಮಾ-ನ-ಗಳಲ್ಲಿ
ದಿವ್ಯ-ಶಕ್ತಿ
ದಿವ್ಯ-ಶ-ಕ್ತಿ-ಗಳು
ದಿವ್ಯ-ಸುಂ-ದರ
ದಿವ್ಯ-ಸುಂ-ದ-ರ-ನಾದ
ದಿವ್ಯ-ಸುಂ-ದ-ರ-ವಾ-ಗಿದೆ
ದಿವ್ಯ-ಸುಂ-ದ-ರ-ವಾದ
ದಿವ್ಯ-ಸುಂ-ದರಿ
ದಿವ್ಯ-ಸುಂ-ದ-ರಿ-ಯ-ರಾದ
ದಿವ್ಯ-ಸುಂ-ದ-ರಿ-ಯಾದ
ದಿವ್ಯಾ-ಕಾ-ರ-ಇ-ವು-ಗಳನ್ನು
ದಿವ್ಯಾ-ಕೃತಿ
ದಿವ್ಯಾ-ನು-ಭವ
ದಿವ್ಯಾ-ಲಂ-ಕಾ-ರ-ಭೂ-ಷಿ-ತೆ-ಯಾಗಿ
ದಿವ್ಯಾ-ಸ್ತ್ರ-ವನ್ನು
ದಿಶೋ
ದಿಸಿ-ಕೊಂ-ಡಿ-ರುವೆ
ದೀಕ್ಷ-ಎಂಬ
ದೀತು
ದೀನ
ದೀನ-ನಾ-ಗಿದ್ದ
ದೀನ-ಮ-ತ್ಸೃಜ್ಯ
ದೀನ-ರಾಗಿ
ದೀನ-ಳಾಗಿ
ದೀನ-ವಾ-ಣಿ-ಯನ್ನು
ದೀನಾ
ದೀಪ
ದೀಪಕ್ಕೆ
ದೀಪ-ಗಳಿಂದ
ದೀಪದ
ದೀಪ-ದಂತೆ
ದೀಪ-ವನ್ನು
ದೀರ್ಘ-ಕಾಲ
ದೀರ್ಘ-ದಂಡ
ದೀರ್ಘ-ಬಾಹು
ದೀರ್ಘ-ವಾದ
ದೀವಿ-ಗೆ-ಗಳಿಂದ
ದೀವಿ-ಗೆ-ಗ-ಳು-ತಮ್ಮ
ದುಂಟು
ದುಂಡ-ನೆಯ
ದುಂಡಾದ
ದುಂಬಿ
ದುಂಬಿ-ಗಳ
ದುಂಬಿ-ಗ-ಳಂತೆ
ದುಂಬಿ-ಗಳನ್ನು
ದುಂಬಿ-ಗಳು
ದುಂಬಿತು
ದುಂಬಿಯ
ದುಂಬಿಯಂ
ದುಂಬಿ-ಯಂತೆ
ದುಂಬಿ-ಯೊಂ-ದನ್ನು
ದುಃಖ
ದುಃಖಂ
ದುಃಖ-ಇ-ವೆಲ್ಲ
ದುಃಖಕ್ಕೆ
ದುಃಖ-ಗಳನ್ನು
ದುಃಖ-ಗಳಿಂದ
ದುಃಖ-ಗ-ಳೆ-ಲ್ಲವೂ
ದುಃಖದ
ದುಃಖ-ದಲ್ಲಿ
ದುಃಖ-ದ-ಸ್ತ-ತ್ಪ್ರ-ಸಂಗಃ
ದುಃಖ-ದಿಂದ
ದುಃಖ-ದಿಂ-ದಲೂ
ದುಃಖ-ನಿ-ವಾ-ರ-ಣೆಗೂ
ದುಃಖ-ನಿ-ವೃತ್ತಿ
ದುಃಖ-ನಿ-ವೃ-ತ್ತಿ-ಯಾ-ಗಲಿ
ದುಃಖ-ವನ್ನು
ದುಃಖ-ವ-ನ್ನೆಲ್ಲ
ದುಃಖ-ವ-ನ್ನೆಲ್ಲಾ
ದುಃಖ-ವನ್ನೇ
ದುಃಖ-ವಾ-ಗಲಿ
ದುಃಖ-ವಾ-ದ-ರೇನು
ದುಃಖ-ವಿಲ್ಲ
ದುಃಖವೂ
ದುಃಖ-ಸ-ಮು-ದ್ರ-ದಲ್ಲಿ
ದುಃಖ-ಹೇ-ತುವೆ
ದುಃಖಿ
ದುಃಖಿ-ತ-ನಾಗಿ
ದುಃಖಿ-ತರೂ
ದುಃಖಿ-ಸುತ್ತಾ
ದುಃಖಿ-ಸು-ತ್ತಿ-ದ್ದನು
ದುಃಖಿ-ಸು-ತ್ತಿ-ರು-ವಂ-ತೆಯೆ
ದುಃಖಿ-ಸು-ತ್ತಿ-ರು-ವೆನು
ದುಃಖಿ-ಸುವೆ
ದುಃಸ್ಥಿ-ತಿ-ಯನ್ನು
ದುಗು-ಡ-ಗೊಂ-ಡರೂ
ದುಗು-ಡ-ವನ್ನು
ದುಡಿದ
ದುಡಿ-ದಿ-ದ್ದೀರಿ
ದುಡಿದು
ದುಡಿಯು
ದುಡಿ-ಯು-ತ್ತಿ-ರುವ
ದುಡಿ-ಯು-ವ-ವರು
ದುಡಿ-ಸಿ-ಕೊ-ಳ್ಳು-ವರು
ದುಡು-ದುಡು
ದುಡ್ಡಿ-ನಾ-ಶೆ-ಯಿಂದ
ದುದ-ಕ್ಕಾಗಿ
ದುದ-ಕ್ಕೆಲ್ಲ
ದುದನ್ನು
ದುದ-ರಿಂದ
ದುದಿಲ್ಲ
ದುದು
ದುದೇ
ದುದೊ
ದುರ
ದುರಂತ
ದುರಂ-ತ-ವನ್ನು
ದುರ-ತ್ಯ-ಯಾಂ
ದುರ-ದು-ರನೆ
ದುರ-ದೃ-ಷ್ಟ-ಕ್ಕಾಗಿ
ದುರ-ದೃ-ಷ್ಟಕ್ಕೆ
ದುರ-ದೃ-ಷ್ಟ-ದಿಂದ
ದುರ-ವ-ಸ್ಥೆಗೆ
ದುರ-ವ-ಸ್ಥೆ-ಯನ್ನು
ದುರ-ಹಂ-ಕಾ-ರ-ದಿಂದ
ದುರ-ಹಂ-ಕಾ-ರ-ವನ್ನು
ದುರ-ಹಂ-ಕಾರಿ
ದುರ-ಹಂ-ಕಾ-ರಿ-ಯಾದ
ದುರಾ-ಚಾರ
ದುರಾ-ಚಾ-ರ-ರಾ-ಗಲಿ
ದುರಾತ್ಮಾ
ದುರಾ-ಶೆ-ಯನ್ನು
ದುರಾಸೆ
ದುರಿ-ತ-ಗಳೂ
ದುರು
ದುರು-ಕ್ತಿ-ಗಳನ್ನು
ದುರು-ಗು-ಟ್ಟಿ-ಕೊಂಡು
ದುರು-ದುರು
ದುರು-ದ್ದೇ-ಶ-ವ-ನ್ನ-ರಿತ
ದುರು-ಪ-ಯೋ-ಗ-ಪ-ಡಿ-ಸಿ-ದರೆ
ದುರು-ಳ-ನಾದ
ದುರ್ಗಂ-ಧ-ಇ-ತ್ಯಾ-ದಿ-ಗಳಲ್ಲಿ
ದುರ್ಗತಿ
ದುರ್ಗ-ತಿಗೆ
ದುರ್ಗ-ತಿ-ಯನ್ನು
ದುರ್ಗ-ತಿ-ಯಾ-ದರೆ
ದುರ್ಗ-ಮ-ವಾದ
ದುರ್ಗಾ-ದೇ-ವಿಗೆ
ದುರ್ಗೇ-ಷ್ವ-ಟ-ವ್ಯಾ-ಜಿ-ಮು-ಖಾ-ದಿಷು
ದುರ್ಜನ
ದುರ್ಜ-ಯ-ನಾ-ಗಿ-ದ್ದ-ವನು
ದುರ್ಜ-ಯ-ರಾ-ಗಿ-ಹೋ-ಗಿ-ದ್ದಾರೆ
ದುರ್ದ-ಮ-ನೀ-ಯ-ವಾದ
ದುರ್ಬ-ಲನ
ದುರ್ಬುದ್ಧಿ
ದುರ್ಬು-ದ್ಧಿಯೂ
ದುರ್ಮದ
ದುರ್ಮ-ರ್ಷಣ
ದುರ್ಮಾ-ರ್ಗ-ರಿ-ಗೇನು
ದುರ್ಮಾ-ರ್ಗರು
ದುರ್ಮಾ-ರ್ಗ-ವ್ರ-ವೃತ್ತಿ
ದುರ್ಯೋ
ದುರ್ಯೋ-ಧನ
ದುರ್ಯೋ-ಧ-ನ-ಇ-ವರೆಲ್ಲ
ದುರ್ಯೋ-ಧ-ನನ
ದುರ್ಯೋ-ಧ-ನ-ನಿಂದ
ದುರ್ಯೋ-ಧ-ನ-ನಿಗೆ
ದುರ್ಯೋ-ಧ-ನನು
ದುರ್ಯೋ-ಧ-ನನೂ
ದುರ್ಯೋ-ಧ-ನನೇ
ದುರ್ಯೋ-ಧ-ನ-ನೊ-ಬ್ಬನು
ದುರ್ಯೋ-ಧ-ನ-ರನ್ನು
ದುರ್ಯೋ-ಧ-ನರು
ದುರ್ಯೋ-ಧ-ನಾದಿ
ದುರ್ಲಭ
ದುರ್ಲ-ಭ-ನಾದ
ದುರ್ಲ-ಭ-ವಾದ
ದುರ್ಲ-ಭ-ವಾ-ದರೂ
ದುರ್ವಾಸ
ದುರ್ವಾ-ಸ-ನತ್ತ
ದುರ್ವಾ-ಸ-ನನ್ನು
ದುರ್ವಾ-ಸನು
ದುರ್ವಾ-ಸನೂ
ದುರ್ವಾ-ಸನೆ
ದುರ್ವಾ-ಸ-ನೆಂ-ದರೆ
ದುರ್ವಾ-ಸ-ಪುಷಿ
ದುರ್ವಾ-ಸ-ಮು-ನಿಯು
ದುರ್ವಿ-ದ್ಯೆ-ಗಳನ್ನು
ದುವು
ದುಶ್ಯಂ-ತ-ನೆಂ-ಬು-ವನು
ದುಶ್ಶಾ-ಸ-ನನು
ದುಶ್ಶಾ-ಸ-ನ-ರೊ-ಡನೆ
ದುಷ್ಕ-ರ್ಮಕ್ಕೆ
ದುಷ್ಕ-ರ್ಮ-ದಿಂದ
ದುಷ್ಕಾ-ರ್ಯ-ಗಳಲ್ಲಿ
ದುಷ್ಟ
ದುಷ್ಟ-ನಾದ
ದುಷ್ಟ-ನಿ-ಗ್ರಹ
ದುಷ್ಟ-ನಿ-ಗ್ರ-ಹ-ವನ್ನು
ದುಷ್ಟ-ಮೃ-ಗ-ಗಳನ್ನು
ದುಷ್ಟ-ಮೃ-ಗ-ಗಳಿಂದ
ದುಷ್ಟರ
ದುಷ್ಟ-ರನ್ನು
ದುಷ್ಟ-ರಾದ
ದುಷ್ಟ-ರಿಗೆ
ದುಷ್ಟ-ಶಿ-ಕ್ಷಣ
ದುಷ್ಟ-ಶಿ-ಕ್ಷ-ಣ-ದಲ್ಲಿ
ದುಷ್ಯಂ-ತನ
ದುಷ್ಯಂ-ತ-ನಿಗೆ
ದುಷ್ಯಂ-ತನು
ದುಸ್ತ-ರ-ವಾ-ಗಿತ್ತು
ದುಸ್ತ್ಯ-ಜ-ದ್ಪಂ-ದ್ವ-ಪಾರ್ಶ್ವಂ
ದುಸ್ಥಿ-ತಿಗೆ
ದುಸ್ಥಿ-ತಿ-ಯನ್ನು
ದೂತ
ದೂತಂ
ದೂತನ
ದೂತ-ನ-ನ್ನಾಗಿ
ದೂತ-ನಾದ
ದೂತನು
ದೂತನೂ
ದೂತ-ನೆಂದು
ದೂತ-ನೊ-ಡನೆ
ದೂತ-ನೊ-ಬ್ಬನು
ದೂತ-ರನ್ನು
ದೂತ-ರಿಗೆ
ದೂತರು
ದೂತರೂ
ದೂತರೆ
ದೂತ-ರೊ-ಡನೆ
ದೂತ-ಸ್ತ್ವ-ಮೀ-ದೃಕ್
ದೂರ
ದೂರಕ್ಕೆ
ದೂರ-ದರ್ಶಿ
ದೂರ-ದಲ್ಲಿ
ದೂರ-ದ-ಲ್ಲಿದ್ದ
ದೂರ-ದಿಂದ
ದೂರ-ದಿಂ-ದಲೆ
ದೂರ-ದೂ-ರ-ವಾಗಿ
ದೂರದೆ
ದೂರ-ಮಾ-ಡು-ತ್ತಾನೆ
ದೂರ-ಲೇಕೆ
ದೂರ-ವಾಗಿ
ದೂರ-ವಾ-ಗಿ-ರು-ತ್ತಿ-ದ್ದನು
ದೂರ-ವಾದ
ದೂರ-ವಾ-ಯಿತು
ದೂರ-ಶ್ರ-ವಣ
ದೂರಿ
ದೂರಿ-ದರು
ದೂರಿ-ದಳು
ದೂರು
ದೂರ್ವಾ-ಕ್ಷಿ-ತಕ್ಷ
ದೂಳು
ದೃಢ
ದೃಢ-ಕಾ-ಯ-ವನ್ನು
ದೃಢ-ಚಿ-ತ್ತ-ನಾದ
ದೃಢ-ಪ-ಡಿ-ಸಿಕೊ
ದೃಢ-ಪ-ಡಿ-ಸಿ-ಕೊಂ-ಡರೆ
ದೃಢ-ಮಾ-ಡಿಕೊ
ದೃಢ-ವಾಗಿ
ದೃಢ-ವಾದ
ದೃಢ-ವಾ-ದದು
ದೃಢ-ವಾ-ಯಿತು
ದೃಢ-ವಾ-ಯಿ-ತೆಂ-ದರೆ
ದೃಶ್ಯ
ದೃಶ್ಯತೇ
ದೃಶ್ಯ-ವನ್ನು
ದೃಶ್ಯಾ-ದೃ-ಶ್ಯ-ವೆಂದು
ದೃಷ್ಟ
ದೃಷ್ಟಾಂತ
ದೃಷ್ಟಾಂ-ತ-ವಾಗಿ
ದೃಷ್ಟಾಂ-ತ-ವೆಂದರೆ
ದೃಷ್ಟಿ
ದೃಷ್ಟಿ-ಗಳಿಂದ
ದೃಷ್ಟಿಗೆ
ದೃಷ್ಟಿ-ಯನ್ನು
ದೃಷ್ಟಿ-ಯಲ್ಲಿ
ದೃಷ್ಟಿ-ಯಿಂದ
ದೃಷ್ಟಿ-ಯಿಂ-ದಲೂ
ದೃಷ್ಟ್ವ-ಧ್ಯಾ-ಯಂತೀ
ದೆಂದರೆ
ದೆಂದು
ದೆಂಬ
ದೆಯಾ
ದೆಯೋ
ದೆಲ್ಲ-ವನ್ನೂ
ದೆಲ್ಲವೂ
ದೆವ್ವಕ್ಕೆ
ದೆವ್ವ-ಬಡಿ
ದೆಸೆ
ದೆಸೆ-ಯಿಂದ
ದೆಸೆ-ಯಿಂ-ದಲೇ
ದೇಂ
ದೇಕೆ
ದೇದೀ-ಪ್ಯ-ಮಾ-ನ-ವಾಗಿ
ದೇನೂ
ದೇನೋ
ದೇವ
ದೇವ-ಋಣ
ದೇವ-ಎಂಬ
ದೇವಕ
ದೇವ-ಕ-ನ್ಯೆ-ಯರು
ದೇವಕಿ
ದೇವ-ಕಿ-ಕೀ-ರ್ತಿ-ಮಂತ
ದೇವ-ಕಿಗೆ
ದೇವ-ಕಿ-ದೇವಿ
ದೇವ-ಕಿ-ದೇ-ವಿ-ಗಾದ
ದೇವ-ಕಿ-ದೇ-ವಿ-ಯ-ವರ
ದೇವ-ಕಿ-ನಂ-ದ-ನ-ನಿಗೆ
ದೇವ-ಕಿಯ
ದೇವ-ಕಿ-ಯ-ದಲ್ಲ
ದೇವ-ಕಿ-ಯನ್ನು
ದೇವ-ಕಿ-ಯರ
ದೇವ-ಕಿ-ಯ-ರನ್ನು
ದೇವ-ಕಿ-ಯ-ರಾಗಿ
ದೇವ-ಕಿ-ಯರು
ದೇವ-ಕಿ-ಯರೂ
ದೇವ-ಕಿ-ಯಲ್ಲಿ
ದೇವ-ಕಿ-ಯಿದ್ದ
ದೇವ-ಕಿಯು
ದೇವ-ಕಿಯೇ
ದೇವ-ಕೀ-ದೇವಿ
ದೇವ-ಕೀ-ದೇ-ವಿಗೂ
ದೇವ-ಕೀ-ದೇ-ವಿಗೆ
ದೇವ-ಕೀ-ದೇ-ವಿಯ
ದೇವ-ಕೀ-ದೇ-ವಿ-ಯನ್ನು
ದೇವ-ಕೀ-ದೇ-ವಿ-ಯರ
ದೇವ-ಕೀ-ದೇ-ವಿಯೂ
ದೇವ-ಕೀ-ನಂ-ದ-ನಾಯ
ದೇವ-ಕೀ-ಪು-ತ್ರ-ನಾದ
ದೇವ-ಕೀ-ರೋ-ಹಿ-ಣಿ-ಯರೂ
ದೇವ-ಕು-ಲ-ದ-ವನು
ದೇವ-ಗಂಗೆ
ದೇವ-ಗಂ-ಗೆ-ಯಂತೆ
ದೇವ-ಗಂ-ಗೆ-ಯನ್ನು
ದೇವ-ಗಂ-ಗೆ-ಯಲ್ಲಿ
ದೇವ-ಗಂ-ಗೆ-ಯಿಂದ
ದೇವ-ಗಂ-ಗೆಯೂ
ದೇವ-ಗಂ-ಧ-ರ್ವರೂ
ದೇವ-ಗು-ರು-ವಾದ
ದೇವ-ಗು-ರು-ವಿನ
ದೇವತಾ
ದೇವ-ತಾ-ವಿ-ಗ್ರ-ಹ-ಗ-ಳಿಗೆ
ದೇವ-ತಾ-ವಿ-ಗ್ರ-ಹ-ಗ-ಳೆ-ಲ್ಲವೂ
ದೇವ-ತಾ-ಸೃ-ಷ್ಟಿಗೆ
ದೇವ-ತಾ-ಸ್ತ್ರೀ-ಯರು
ದೇವತೆ
ದೇವ-ತೆ-ಅ-ವರ
ದೇವ-ತೆ-ಎಂ-ಬುದು
ದೇವ-ತೆ-ಗಳ
ದೇವ-ತೆ-ಗ-ಳಂತೆ
ದೇವ-ತೆ-ಗಳನ್ನು
ದೇವ-ತೆ-ಗಳನ್ನೂ
ದೇವ-ತೆ-ಗ-ಳನ್ನೆ
ದೇವ-ತೆ-ಗ-ಳ-ಮೇಲೆ
ದೇವ-ತೆ-ಗಳಲ್ಲಿ
ದೇವ-ತೆ-ಗ-ಳ-ಲ್ಲಿ-ಅ-ದ-ರ-ಲ್ಲಿಯೂ
ದೇವ-ತೆ-ಗ-ಳ-ಲ್ಲಿ-ದ್ದು-ಕೊಂಡು
ದೇವ-ತೆ-ಗ-ಳ-ಲ್ಲಿ-ರು-ವೆ-ಯೆಂದೊ
ದೇವ-ತೆ-ಗ-ಳಷ್ಟೇ
ದೇವ-ತೆ-ಗ-ಳಾ-ಗಲಿ
ದೇವ-ತೆ-ಗ-ಳಾದ
ದೇವ-ತೆ-ಗ-ಳಾ-ದರು
ದೇವ-ತೆ-ಗ-ಳಾರೂ
ದೇವ-ತೆ-ಗಳಿಂದ
ದೇವ-ತೆ-ಗ-ಳಿ-ಗಿಂ-ತಲೂ
ದೇವ-ತೆ-ಗ-ಳಿಗೂ
ದೇವ-ತೆ-ಗ-ಳಿಗೆ
ದೇವ-ತೆ-ಗ-ಳಿ-ಗೆಲ್ಲ
ದೇವ-ತೆ-ಗಳು
ದೇವ-ತೆ-ಗಳೂ
ದೇವ-ತೆ-ಗಳೆ
ದೇವ-ತೆ-ಗ-ಳೆಂ-ದರೆ
ದೇವ-ತೆ-ಗ-ಳೆಲ್ಲ
ದೇವ-ತೆ-ಗ-ಳೆ-ಲ್ಲರೂ
ದೇವ-ತೆ-ಗಳೇ
ದೇವ-ತೆ-ಗ-ಳೊ-ಡನೆ
ದೇವ-ತೆಗೂ
ದೇವ-ತೆ-ಯಂತೆ
ದೇವ-ತೆ-ಯರು
ದೇವ-ತೆ-ಯಲ್ಲ
ದೇವ-ತೆಯೊ
ದೇವ-ತೆ-ಯೊಬ್ಬ
ದೇವ-ದ-ತ್ತ-ವೆಂಬ
ದೇವ-ದ-ರ್ಶ-ನ-ವನ್ನೆ
ದೇವ-ದಾ-ನವ
ದೇವ-ದಾ-ನ-ವರು
ದೇವ-ದಾ-ನ-ವರೆ-ಲ್ಲರೂ
ದೇವ-ದುಂ-ದು-ಭಿ-ಗಳು
ದೇವ-ದೇವ
ದೇವ-ದೇ-ವನ
ದೇವ-ದೇ-ವ-ನನ್ನು
ದೇವ-ದೇ-ವ-ನಾದ
ದೇವ-ದೇ-ವ-ನಿಗೆ
ದೇವ-ದೇ-ವನು
ದೇವ-ದ್ವೇ-ಷಿ-ಗ-ಳಾದ
ದೇವನ
ದೇವ-ನನ್ನು
ದೇವ-ನಾಗಿ
ದೇವ-ನಾ-ಗು-ತ್ತಾನೆ
ದೇವ-ನಾದ
ದೇವ-ನಿಂದೆ
ದೇವ-ನಿಗೆ
ದೇವನು
ದೇವ-ನೆಂಬ
ದೇವ-ನೆ-ಡೆಗೆ
ದೇವನೇ
ದೇವ-ಪು-ರೋ-ಹಿ-ತ-ನಾದ
ದೇವ-ಪು-ರೋ-ಹಿ-ತ-ನಾ-ದರೂ
ದೇವ-ಪುಷಿ
ದೇವ-ಪು-ಷಿ-ಯಾದ
ದೇವ-ಭಾಗ
ದೇವ-ಮಯ
ದೇವ-ಮ-ಯ-ವಾದ
ದೇವ-ಮಾತೆ
ದೇವ-ಮಾ-ನದ
ದೇವ-ಮಾನವ
ದೇವ-ಮಾ-ನ-ವರೆ-ಲ್ಲರೂ
ದೇವ-ಮಾ-ನ-ವಾದಿ
ದೇವ-ಯಾನಿ
ದೇವ-ಯಾ-ನಿಗೂ
ದೇವ-ಯಾ-ನಿಗೆ
ದೇವ-ಯಾ-ನಿಯ
ದೇವ-ಯಾ-ನಿ-ಯನ್ನು
ದೇವ-ಯಾ-ನಿಯು
ದೇವ-ಯಾ-ನಿ-ಯೆಂಬ
ದೇವರ
ದೇವ-ರಂತೆ
ದೇವ-ರ-ಕ್ಷಿ-ತಾ-ಗದಾ
ದೇವ-ರ-ನಾ-ಮ-ವನ್ನು
ದೇವ-ರನ್ನು
ದೇವ-ರನ್ನೂ
ದೇವ-ರನ್ನೆ
ದೇವ-ರ-ಪೂಜೆ
ದೇವ-ರ-ಪೂ-ಜೆ-ಗಳನ್ನು
ದೇವ-ರ-ಯಾವ
ದೇವ-ರಲ್ಲಿ
ದೇವ-ರಾಜ
ದೇವ-ರಾ-ಜ-ನಾದ
ದೇವ-ರಾ-ತ-ನೆಂದು
ದೇವ-ರಾರೂ
ದೇವ-ರಿಗೆ
ದೇವರು
ದೇವ-ರು-ಗಳನ್ನು
ದೇವರೆ
ದೇವ-ರೆಂದು
ದೇವ-ರೆಂದೂ
ದೇವ-ರೆಂಬ
ದೇವರೇ
ದೇವ-ರೇನೋ
ದೇವ-ರ್ಷಿ-ವರ್ಯಃ
ದೇವಲ
ದೇವ-ಲನೆ
ದೇವ-ಲೋ-ಕಕ್ಕೆ
ದೇವ-ಲೋ-ಕ-ದಲ್ಲಿ
ದೇವ-ಲೋ-ಕ-ದ-ಲ್ಲಿಯೆ
ದೇವ-ಲೋ-ಕ-ದವ
ದೇವ-ಲೋ-ಕ-ದಿಂದ
ದೇವ-ವೈ-ದ್ಯ-ರಾದ
ದೇವ-ವೈ-ದ್ಯ-ರಾ-ದರೂ
ದೇವ-ಶ್ರವ
ದೇವ-ಸೇ-ನೆಯ
ದೇವ-ಸೈನ್ಯ
ದೇವಸ್ಯ
ದೇವ-ಹೂತಿ
ದೇವ-ಹೂ-ತಿಗೂ
ದೇವ-ಹೂ-ತಿಗೆ
ದೇವ-ಹೂ-ತಿಯ
ದೇವ-ಹೂ-ತಿ-ಯನ್ನು
ದೇವ-ಹೂ-ತಿ-ಯ-ರನ್ನು
ದೇವ-ಹೂ-ತಿ-ಯರು
ದೇವ-ಹೂ-ತಿಯು
ದೇವ-ಹೇ-ಳ-ನಾತ್
ದೇವಾ
ದೇವಾಂ-ತಕ
ದೇವಾಂ-ಶ-ಸಂ-ಭೂ-ತ-ರಾದ
ದೇವಾ-ತ್ಮಕ
ದೇವಾಧಿ
ದೇವಾನಂ
ದೇವಾ-ನು-ದೇ-ವತೆ
ದೇವಾ-ನು-ದೇ-ವ-ತೆ-ಗಳೂ
ದೇವಾ-ನು-ದೇ-ವ-ತೆ-ಗ-ಳೆಲ್ಲ
ದೇವಾ-ನು-ದೇ-ವ-ತೆ-ಗ-ಳೆ-ಲ್ಲರೂ
ದೇವಾ-ಲ-ಯಕ್ಕೆ
ದೇವಾ-ಲ-ಯ-ಗ-ಳಿ-ದ್ದುವು
ದೇವಾ-ಲ-ಯ-ಗಳು
ದೇವಾ-ಲ-ಯದ
ದೇವಿ
ದೇವಿ-ಯನ್ನು
ದೇವಿಯು
ದೇವಿಯೂ
ದೇವುಕ
ದೇವು-ಕನ
ದೇವು-ಕ-ನಿಗೆ
ದೇವೆ-ತ-ಗಳ
ದೇವೆ-ತೆ-ಗಳೂ
ದೇವೇಂದ್ರ
ದೇವೇಂ-ದ್ರನ
ದೇವೇಂ-ದ್ರ-ನನ್ನು
ದೇವೇಂ-ದ್ರ-ನಲ್ಲ
ದೇವೇಂ-ದ್ರ-ನಲ್ಲಿ
ದೇವೇಂ-ದ್ರ-ನ-ಲ್ಲಿ-ದಿ-ತಿಗೆ
ದೇವೇಂ-ದ್ರ-ನಾ-ಗು-ತ್ತಾನೆ
ದೇವೇಂ-ದ್ರ-ನಿಂದ
ದೇವೇಂ-ದ್ರ-ನಿಗೂ
ದೇವೇಂ-ದ್ರ-ನಿಗೆ
ದೇವೇಂ-ದ್ರನು
ದೇವೇಂ-ದ್ರನೂ
ದೇವೇಂ-ದ್ರನೆ
ದೇವೇಂ-ದ್ರ-ನೇನು
ದೇವೇಂ-ದ್ರ-ನೊ-ಡನೆ
ದೇವೇಂ-ದ್ರ-ಪ-ದವಿ
ದೇವೇಂ-ದ್ರ-ಸ-ಹಿ-ತ-ನಾಗಿ
ದೇವೇಭ್ಯೋ
ದೇವೋ-ತ್ತಮ
ದೇವೋ-ದ್ಯಾ-ನ-ಗಳಲ್ಲಿ
ದೇವೋ-ಽಪ-ರಾಹ್ಣೇ
ದೇವ್ಯಾ-ದಿ-ವ-ರಾ-ಹೇಣ
ದೇಶ
ದೇಶಕ್ಕೆ
ದೇಶ-ಗಳ
ದೇಶ-ಗಳನ್ನು
ದೇಶ-ಗಳಲ್ಲಿ
ದೇಶ-ಗ-ಳ-ಲ್ಲಿಯೂ
ದೇಶದ
ದೇಶ-ದಲ್ಲಿ
ದೇಶ-ದಿಂದ
ದೇಶ-ದೇ-ಶ-ಗಳ
ದೇಶ-ದೇ-ಶ-ಗಳನ್ನೆಲ್ಲ
ದೇಶ-ದೇ-ಶ-ಗ-ಳ-ಲ್ಲೆಲ್ಲ
ದೇಶ-ದೇ-ಶ-ಗಳಿಂದ
ದೇಶ-ದೇ-ಶದ
ದೇಶ-ಸಂ-ಚಾರ
ದೇಶಾಂ-ತರ
ದೇಶಾಂ-ತ-ರ-ಗ-ಳಿಗೆ
ದೇಹ
ದೇಹ-ಕಾಂತಿ
ದೇಹ-ಕಾಂ-ತಿ-ಯನ್ನೂ
ದೇಹ-ಕಾಂ-ತಿ-ಯಿಂದ
ದೇಹ-ಕ್ಕಾಗಿ
ದೇಹ-ಕ್ಕಿಂತ
ದೇಹ-ಕ್ಕಿಂ-ತಲೂ
ದೇಹಕ್ಕೂ
ದೇಹಕ್ಕೆ
ದೇಹ-ಗಳನ್ನು
ದೇಹ-ಗ-ಳ-ಲ್ಲಿಯೂ
ದೇಹ-ಗಳಿಂದ
ದೇಹ-ಗ-ಳಿಗೆ
ದೇಹ-ಗಳು
ದೇಹ-ಗ-ಳೇನೂ
ದೇಹ-ತ್ಯಾಗ
ದೇಹ-ತ್ಯಾ-ಗಕ್ಕೂ
ದೇಹ-ತ್ಯಾ-ಗ-ಮಾ-ಡುವ
ದೇಹದ
ದೇಹ-ದಂತೆ
ದೇಹ-ದಲ್ಲಿ
ದೇಹ-ದ-ಲ್ಲಿದ್ದ
ದೇಹ-ದ-ಲ್ಲಿಯೆ
ದೇಹ-ದ-ಲ್ಲಿಯೇ
ದೇಹ-ದ-ಲ್ಲಿ-ರುವ
ದೇಹ-ದ-ಲ್ಲಿ-ರು-ವಾಗ
ದೇಹ-ದಿಂದ
ದೇಹ-ದಿಂ-ದಲೇ
ದೇಹ-ದಿಂ-ದಿ-ರು-ವಷ್ಟು
ದೇಹ-ದೃಷ್ಟಿ
ದೇಹ-ದೊ-ಡನೆ
ದೇಹ-ದೊ-ಳ-ಗಿ-ರುವ
ದೇಹ-ಧಾ-ರ-ಣೆಗೆ
ದೇಹ-ಧಾ-ರಿ-ಯಾ-ಗುವ
ದೇಹ-ಭಾ-ಗ-ವನ್ನು
ದೇಹ-ಮಾತ್ರ
ದೇಹ-ರ-ಕ್ಷ-ಣೆ-ಗಾಗಿ
ದೇಹ-ವನ್ನು
ದೇಹ-ವನ್ನೂ
ದೇಹ-ವನ್ನೆ
ದೇಹ-ವ-ನ್ನೆಲ್ಲ
ದೇಹ-ವನ್ನೇ
ದೇಹ-ವಲ್ಲ
ದೇಹ-ವಾ-ಗಲಿ
ದೇಹ-ವಿತ್ತೆ
ದೇಹ-ವಿ-ರು-ತ್ತ-ದೆಯೆ
ದೇಹ-ವಿ-ರು-ವ-ವ-ರೆಗೆ
ದೇಹವು
ದೇಹ-ವುಳ್ಳ
ದೇಹವೂ
ದೇಹ-ವೆಂದು
ದೇಹ-ವೆಂಬ
ದೇಹ-ವೆಂ-ಬುದು
ದೇಹ-ವೆಲ್ಲ
ದೇಹವೇ
ದೇಹ-ಶಕ್ತಿ
ದೇಹ-ಶುದ್ಧಿ
ದೇಹ-ಸಂ-ಬಂ-ಧ-ವನ್ನು
ದೇಹ-ಸಂ-ಸ್ಕಾ-ರ-ವಿ-ಲ್ಲ-ದು-ದ-ರಿಂದ
ದೇಹಾಂ-ತ-ರ-ದಲ್ಲಿ
ದೇಹಾ-ತೀ-ತ-ರಾ-ಗ-ಬೇಕು
ದೇಹಾತ್ಮ
ದೇಹಾ-ತ್ಮಾ-ಭಿ-ಮಾನ
ದೇಹಾ-ದಿ-ಗಳು
ದೇಹಾ-ಭಿ-ಮಾನ
ದೇಹಾ-ಭಿ-ಮಾ-ನ-ದಿಂದ
ದೇಹಾ-ಭಿ-ಮಾ-ನ-ವನ್ನು
ದೇಹಾ-ಭಿ-ಮಾ-ನ-ವೆಷ್ಟು
ದೇಹಾವ
ದೇಹಾ-ವ-ಸಾ-ನದ
ದೇಹೇಂ-ದ್ರಿಯ
ದೈತ್ಯ
ದೈತ್ಯ-ದಾ-ನ-ವರ
ದೈತ್ಯ-ದಾ-ನ-ವ-ರಂತೆ
ದೈತ್ಯ-ದಾ-ನ-ವ-ರನ್ನು
ದೈತ್ಯ-ದಾ-ನ-ವ-ರಿ-ಗೆಲ್ಲ
ದೈತ್ಯ-ದಾ-ನ-ವರು
ದೈತ್ಯರ
ದೈತ್ಯರೂ
ದೈತ್ಯ-ಳಾ-ಗಿ-ದ್ದರೂ
ದೈತ್ಯಾ-ಧ-ಮ-ನೊ-ಡನೆ
ದೈನ್ಯ
ದೈನ್ಯಕ್ಕೂ
ದೈನ್ಯಕ್ಕೆ
ದೈನ್ಯ-ದಿಂದ
ದೈನ್ಯ-ವನ್ನು
ದೈವ-ಕಾ-ರ್ಯ-ದಲ್ಲಿ
ದೈವ-ಕೃ-ಪೆ-ಯಿ-ಲ್ಲದ
ದೈವಕ್ಕೆ
ದೈವ-ಚಿಂ-ತನೆ
ದೈವ-ಚಿಂ-ತ-ನೆ-ಯ-ಲ್ಲಿದ್ದ
ದೈವದ
ದೈವ-ಧ್ಯಾ-ನ-ವನ್ನು
ದೈವ-ಬ-ಲವೇ
ದೈವ-ಭಕ್ತ
ದೈವ-ಭ-ಕ್ತ-ನಾದ
ದೈವ-ಭ-ಕ್ತ-ರಾದ
ದೈವ-ಭ-ಕ್ತರು
ದೈವ-ಭಕ್ತಿ
ದೈವ-ಭ-ಕ್ತಿಯ
ದೈವ-ಭ-ಕ್ತಿ-ಯುಕ್ತ
ದೈವ-ಭಾ-ವ-ನೆ-ಯ-ಲ್ಲಿದ್ದು
ದೈವ-ಯೋ-ಗ-ದಿಂದ
ದೈವ-ವನ್ನು
ದೈವ-ವಾದ
ದೈವವೂ
ದೈವ-ವೆಂದು
ದೈವವೇ
ದೈವ-ಸಂ-ಕಲ್ಪ
ದೈವ-ಸಾ-ಕ್ಷಾ-ತ್ಕಾರ
ದೈವ-ಸಾ-ಕ್ಷಾ-ತ್ಕಾ-ರಕ್ಕೆ
ದೈವ-ಸ್ಮ-ರಣೆ
ದೈವಾಂ-ಶ-ಸಂ-ಭೂ-ತರು
ದೈವಾ-ಧೀ-ನ-ವಾಗಿ
ದೈವಾ-ಧೀ-ನ-ವಾ-ಗಿಯೇ
ದೈವಾನು
ದೈವಾ-ನು-ಗ್ರ-ಹಕ್ಕೆ
ದೊಂದು
ದೊಡನೆ
ದೊಡ್ಡ
ದೊಡ್ಡ-ದಾದ
ದೊಡ್ಡದು
ದೊಡ್ಡ-ದೇಹ
ದೊಡ್ಡ-ದೊಂದು
ದೊಡ್ಡ-ಪ್ಪ-ನಾದ
ದೊಡ್ಡ-ಮೀನು
ದೊಡ್ಡ-ವ-ನಾಗಿ
ದೊಡ್ಡ-ವ-ನಾ-ಗು-ತ್ತಲೆ
ದೊಡ್ಡ-ವ-ನಾದ
ದೊಡ್ಡ-ವ-ನಾ-ದನು
ದೊಡ್ಡ-ವನು
ದೊಡ್ಡ-ವರ
ದೊಡ್ಡ-ವ-ರಾ-ದ-ಮೇಲೆ
ದೊಡ್ಡ-ವ-ರಾ-ದರು
ದೊಡ್ಡ-ವ-ರಾ-ದಿರಿ
ದೊಡ್ಡ-ವರು
ದೊಡ್ಡ-ವರೆ-ನಿ-ಸಿ-ಕೊಂ-ಡು-ದಕ್ಕೆ
ದೊಡ್ಡವು
ದೊಡ್ದ
ದೊಣ್ಣೆ-ಯಿಂದ
ದೊನ್ನೆ-ಯಂ-ತಿ-ರುವ
ದೊಪ್ಪನೆ
ದೊಯ್ದರು
ದೊಯ್ದು
ದೊಯ್ಯ-ದಿ-ದ್ದರೆ
ದೊರ-ಕಿ-ಸಿ-ಕೊ-ಟ್ಟನು
ದೊರ-ಕು-ತ್ತದೆ
ದೊರ-ಕು-ತ್ತಿತ್ತು
ದೊರೆ
ದೊರೆ-ಗಳು
ದೊರೆತ
ದೊರೆ-ತಂ-ತಾ-ಗಿತ್ತು
ದೊರೆ-ತಂ-ತಾ-ಗು-ತ್ತದೆ
ದೊರೆ-ತಂ-ತಾ-ಯಿತು
ದೊರೆ-ತರೆ
ದೊರೆ-ತವು
ದೊರೆ-ತಷ್ಟ
ದೊರೆ-ತ-ಷ್ಟ-ರಿಂದ
ದೊರೆ-ತಷ್ಟೆ
ದೊರೆ-ತಿ-ರ-ಲಿಲ್ಲ
ದೊರೆ-ತು-ದ-ಕ್ಕಾಗಿ
ದೊರೆ-ತು-ದ-ನ್ನೆಲ್ಲ
ದೊರೆತೆ
ದೊರೆ-ತೊ-ಡ-ನೆಯೆ
ದೊರೆ-ಯದ
ದೊರೆ-ಯ-ದಂ-ತಹ
ದೊರೆ-ಯ-ದಂ-ತಾ-ಯಿತು
ದೊರೆ-ಯ-ದಿ-ದ್ದರೆ
ದೊರೆ-ಯ-ಬೇ-ಕಾ-ದರೆ
ದೊರೆ-ಯ-ಲೆಂದು
ದೊರೆ-ಯಾಗಿ
ದೊರೆ-ಯಿ-ತ-ಲ್ಲವೆ
ದೊರೆ-ಯಿತು
ದೊರೆ-ಯಿ-ತೆಂದು
ದೊರೆಯು
ದೊರೆ-ಯು-ತ್ತದೆ
ದೊರೆ-ಯು-ತ್ತವೆ
ದೊರೆ-ಯು-ತ್ತಿಲ್ಲ
ದೊರೆ-ಯು-ವಂತೆ
ದೊರೆ-ಯು-ವುದು
ದೊರೆ-ಯು-ವುದೋ
ದೊರೆವ
ದೊರೆ-ವುದು
ದೋಣಿ-ಯಲ್ಲಿ
ದೋರ್ದಂ-ಡ-ದ-ಲ್ಲಿ-ರುವ
ದೋಷ
ದೋಷ-ಕ್ಕಾಗಿ
ದೋಷಕ್ಕೆ
ದೋಷ-ಗ-ಳಿಗೂ
ದೋಷ-ಗ-ಳೆ-ಷ್ಟಿ-ದ್ದರೂ
ದೋಷ-ವನ್ನು
ದೋಷ-ವ-ನ್ನೇನೊ
ದೋಷ-ವಿತ್ತು
ದೋಷ-ವಿದೆ
ದೋಷ-ವಿ-ರ-ಬೇಕು
ದೋಷವೂ
ದೋಷವೆ
ದೋಷ-ವೇನೂ
ದೋಷೇ
ದೌತ್ಯೈ-ರ್ಮು-ಕುಂ-ದಾತ್
ದ್ದಂತೆ
ದ್ದಂತೆಯೆ
ದ್ದನು
ದ್ದನೆ
ದ್ದನ್ನು
ದ್ದರು
ದ್ದರೂ
ದ್ದರೆ
ದ್ದಳು
ದ್ದವಂತೆ
ದ್ದವನೂ
ದ್ದವರು
ದ್ದವರೆಲ್ಲ
ದ್ದವು
ದ್ದಾಗ
ದ್ದಾನೆ
ದ್ದಾರೆ
ದ್ದೀಯೆ
ದ್ದುದ-ರಿಂದ
ದ್ದುದೂ
ದ್ದೆಡೆ-ಯಿಂದ
ದ್ದೆವು
ದ್ದೇನೆ
ದ್ದೇವೆ
ದ್ಧಾರಕ
ದ್ಧ್ಯೇಯಂ-ಷ-ಟ್ಚ-ಕ್ರಿ-ಭಿ-ರ್ಯು-ತಮ್
ದ್ಪೀಪಕ್ಕೆ
ದ್ಭಕ್ತರ
ದ್ಭಕ್ತ-ರ-ಲ್ಲಿಯೂ
ದ್ಯುಮಂ-ತನ
ದ್ಯುಮಂ-ತ-ನೆಂ-ಬು-ವನು
ದ್ಯುಮ್ನ
ದ್ರಮಿಳ
ದ್ರವ-ರೂ-ಪ-ವಾ-ಗಿ-ರು-ತ್ತದೆ
ದ್ರವಿಣ
ದ್ರವಿ-ಣ-ನಿಗೆ
ದ್ರವ್ಯ-ಗಳು
ದ್ರಷ್ಟಾರ
ದ್ರಾಜ-ನನ್ನು
ದ್ರಾಜನು
ದ್ರಾವಿಡ
ದ್ರಿಯ
ದ್ರಿಯ-ಗಳು
ದ್ರಿಯ-ಗಳೂ
ದ್ರಿಯ-ಗಳೇ
ದ್ರೇಕ-ದಿಂ-ದಿದ್ದ
ದ್ರೋಣ
ದ್ರೋಣ-ನೆಂ-ಬು-ವನು
ದ್ರೋಣರ
ದ್ರೋಹ
ದ್ರೋಹ-ಮಾಡಿ
ದ್ರೋಹ-ವನ್ನು
ದ್ರೋಹಿ
ದ್ರೋಹಿಯು
ದ್ರೌಪದಿ
ದ್ರೌಪ-ದಿಯ
ದ್ರೌಪ-ದಿ-ಯನ್ನು
ದ್ರೌಪ-ದಿಯು
ದ್ರೌಪ-ದಿಯೂ
ದ್ರೌಪ-ದಿಯೇ
ದ್ರೌಪ-ದಿ-ಯೊ-ಡನೆ
ದ್ರೌಪದೀ
ದ್ವಂದ್ವ-ಗ-ಳಿ-ಲ್ಲ-ದಂತೆ
ದ್ವಂದ್ವ-ಗಳು
ದ್ವಂದ್ವ-ಯುದ್ಧ
ದ್ವಂದ್ವ-ಯು-ದ್ಧಕ್ಕೆ
ದ್ವಂದ್ವ-ಯು-ದ್ಧ-ವಾ-ಗದ
ದ್ವಂದ್ವಾ-ದ್ಭ-ಯಾ-ದೃ-ಷ-ಭೋ-ನಿ-ರ್ಜಿ-ತಾತ್ಮಾ
ದ್ವಾದ-ಶಾ-ಕ್ಷ-ರ-ಗಳಿಂದ
ದ್ವಾದ-ಶಾ-ಕ್ಷರೀ
ದ್ವಾದ-ಶಾ-ದಿ-ತ್ಯ-ರು-ಇ-ತ್ಯಾದಿ
ದ್ವಾದಶಿ
ದ್ವಾದ-ಶಿಯ
ದ್ವಾಪ-ರದ
ದ್ವಾಪ-ರ-ಯು-ಗದ
ದ್ವಾಪ-ರ-ಯು-ಗ-ದಲ್ಲಿ
ದ್ವಾರ
ದ್ವಾರಕಾ
ದ್ವಾರ-ಕಾ-ನ-ಗ-ರಕ್ಕೆ
ದ್ವಾರ-ಕಾ-ನ-ಗ-ರದ
ದ್ವಾರ-ಕಾ-ನ-ಗ-ರ-ವನ್ನು
ದ್ವಾರ-ಕಾ-ನ-ಗ-ರ-ವ-ನ್ನೆಲ್ಲ
ದ್ವಾರ-ಕಾ-ನ-ಗ-ರಿಗೆ
ದ್ವಾರ-ಕಾ-ನ-ಗ-ರಿಯು
ದ್ವಾರ-ಕಾ-ಪ-ಟ್ಟ-ಣದ
ದ್ವಾರ-ಕಾ-ಪುರ
ದ್ವಾರ-ಕಾ-ಪು-ರಕ್ಕೆ
ದ್ವಾರ-ಕಾ-ಪು-ರದ
ದ್ವಾರ-ಕಾ-ಪು-ರ-ದಲ್ಲಿ
ದ್ವಾರ-ಕಾ-ಪು-ರ-ವನ್ನು
ದ್ವಾರಕಿ
ದ್ವಾರ-ಕಿಗೆ
ದ್ವಾರ-ಕಿಯ
ದ್ವಾರ-ಕಿ-ಯನ್ನು
ದ್ವಾರ-ಕಿ-ಯ-ನ್ನೆಲ್ಲ
ದ್ವಾರ-ಕಿ-ಯಲ್ಲಿ
ದ್ವಾರ-ಕಿ-ಯಿಂದ
ದ್ವಾರ-ಕಿಯು
ದ್ವಾರಕೆ
ದ್ವಾರ-ಕೆಗೆ
ದ್ವಾರ-ಕೆ-ಯನ್ನು
ದ್ವಾರ-ಕೆ-ಯಲ್ಲಿ
ದ್ವಾರ-ಕೆ-ಯ-ಲ್ಲಿದ್ದ
ದ್ವಾರ-ಕೆ-ಯ-ಲ್ಲಿಯೂ
ದ್ವಾರಕ್ಕೆ
ದ್ವಾರ-ಗಳು
ದ್ವಾರ-ಗ-ಳೆ-ಲ್ಲವೂ
ದ್ವಾರ-ಪಾ-ಲ-ಕ-ರನ್ನು
ದ್ವಾರ-ಪಾ-ಲ-ಕರು
ದ್ವಾರ-ಪಾ-ಲ-ಕರೆ
ದ್ವಿಜ-ರಾದ
ದ್ವಿದ-ಳ-ಧಾ-ನ್ಯದ
ದ್ವಿವಿದ
ದ್ವಿವಿ-ದನು
ದ್ವಿವಿ-ದ-ನೆಂ-ದರೆ
ದ್ವಿವಿ-ದ-ನೆಂಬ
ದ್ವಿವಿಧ
ದ್ವಿಷಾ-ಮ-ಘೋ-ನಾಂ
ದ್ವೀಪ
ದ್ವೀಪಕ್ಕೆ
ದ್ವೀಪ-ಗಳ
ದ್ವೀಪ-ಗ-ಳನ್ನೇ
ದ್ವೀಪ-ಗ-ಳಿಗೂ
ದ್ವೀಪ-ಗಳು
ದ್ವೀಪದ
ದ್ವೀಪ-ದಂ-ತಿದ್ದ
ದ್ವೀಪ-ವನ್ನು
ದ್ವೀಪ-ವಿದೆ
ದ್ವೀಪೇ
ದ್ವೇಷ
ದ್ವೇಷ-ಎಂಬ
ದ್ವೇಷಕ್ಕೆ
ದ್ವೇಷ-ದಿಂದ
ದ್ವೇಷ-ಪೂ-ರ್ಣ-ವಾ-ದುದು
ದ್ವೇಷ-ವನ್ನು
ದ್ವೇಷ-ವಾಗಿ
ದ್ವೇಷ-ಸಾ-ಧ-ನೆ-ಯನ್ನು
ದ್ವೇಷಿ
ದ್ವೇಷಿ-ಗಳು
ದ್ವೇಷಿ-ಯಾದ
ದ್ವೇಷಿ-ಯಾ-ದ-ವನು
ದ್ವೇಷಿ-ಸರು
ದ್ವೇಷಿ-ಸು-ತ್ತಾನೆ
ದ್ವೇಷಿ-ಸು-ತ್ತಿ-ರು-ವು-ದ-ರಿಂದ
ದ್ವೇಷಿ-ಸು-ವೆ-ಯ-ಲ್ಲವೆ
ದ್ವೈತ
ದ್ವೈಪಾ-ಯನೋ
ಧಅ-ರಿಸಿ
ಧಗ-ಧಗ
ಧತ್ತೇ
ಧನ
ಧನ-ಕ-ನ-ಕ-ಗ-ಳಾ-ಗಲಿ
ಧನ-ಕ-ನ-ಕಾ-ದಿ-ಗಳನ್ನು
ಧನ-ದಲ್ಲಿ
ಧನ-ದಾ-ನ-ಗಳನ್ನು
ಧನ-ದಾ-ಸೆ-ಇ-ವು-ಗಳ
ಧನ-ದಿಂದ
ಧನ-ಧಾನ್ಯ
ಧನ-ಧಾ-ನ್ಯ-ಗಳೂ
ಧನ-ಮದ
ಧನ-ರಾಶಿ
ಧನ-ರಾ-ಶಿ-ಯನ್ನೂ
ಧನ-ವನ್ನು
ಧನ-ವಾಗಿ
ಧನ-ವೇ-ನನ್ನೂ
ಧನು-ರ್ಧಾ-ರಿ-ಯಾದ
ಧನು-ರ್ಯಾ-ಗ-ಕ್ಕಾಗಿ
ಧನು-ರ್ಯಾ-ಗಕ್ಕೆ
ಧನು-ರ್ಯಾ-ಗದ
ಧನು-ರ್ಯಾ-ಗ-ವೆಂಬ
ಧನು-ರ್ವೇದ
ಧನು-ಶ್ಶಾ-ಲೆಗೆ
ಧನು-ಸ್ಸನ್ನು
ಧನೆ-ಯನ್ನು
ಧನೆ-ಯಿಂ-ದಲೆ
ಧನ್ಯ
ಧನ್ಯ-ತೆ-ಯನ್ನು
ಧನ್ಯ-ನಾ-ಗು-ತ್ತಾನೆ
ಧನ್ಯ-ನಾ-ಗು-ತ್ತೇನೆ
ಧನ್ಯ-ನಾದ
ಧನ್ಯ-ನಾ-ದನು
ಧನ್ಯ-ನಾದೆ
ಧನ್ಯ-ನಾ-ದೆ-ನೆಂ-ದು-ಕೊಂ-ಡನು
ಧನ್ಯ-ನೆಂ-ದು-ಕೊಂ-ಡನು
ಧನ್ಯ-ರ-ನ್ನಾಗಿ
ಧನ್ಯ-ರಾ-ಗು-ತ್ತಾರೆ
ಧನ್ಯ-ರಾ-ಗು-ವು-ದಕ್ಕೆ
ಧನ್ಯ-ರಾದ
ಧನ್ಯ-ರಾ-ದರು
ಧನ್ಯ-ರಾ-ದೆವು
ಧನ್ಯ-ರಾ-ದೆ-ವೆ-ನ್ನಿ-ಸಿತು
ಧನ್ಯರು
ಧನ್ಯ-ರೆಂ-ದು-ಕೊಂ-ಡರು
ಧನ್ಯರೇ
ಧನ್ಯರೋ
ಧನ್ಯ-ಳಾ-ಗ-ಬೇ-ಕೆಂ-ಬುದು
ಧನ್ಯ-ಳಾ-ಗು-ತ್ತೇನೆ
ಧನ್ಯ-ಳಾ-ಗು-ವೆನು
ಧನ್ಯ-ಳಾ-ದಳು
ಧನ್ಯ-ಳಾದೆ
ಧನ್ಯಳು
ಧನ್ಯ-ವಾ-ದವು
ಧನ್ಯ-ವಾ-ಯಿತು
ಧನ್ಯೆ
ಧನ್ವಂ-ತರಿ
ಧನ್ವಂ-ತ-ರಿ-ಯಾಗಿ
ಧನ್ವಂ-ತ-ರಿಯು
ಧನ್ವಂ-ತ-ರಿ-ರ್ಭ-ಗ-ವಾನ್
ಧಮರ
ಧರರು
ಧರಾ-ದೇವಿ
ಧರಾ-ಪತಿ
ಧರಿಸ
ಧರಿ-ಸ-ಬಲ್ಲೆ
ಧರಿ-ಸ-ಬೇ-ಕೆಂದು
ಧರಿ-ಸ-ಬೇಕೋ
ಧರಿಸಿ
ಧರಿ-ಸಿ-ಕೊಂಡು
ಧರಿ-ಸಿತು
ಧರಿ-ಸಿದ
ಧರಿ-ಸಿ-ದನು
ಧರಿ-ಸಿ-ದ-ಮೇಲೆ
ಧರಿ-ಸಿ-ದ-ರಾ-ಯಿತು
ಧರಿ-ಸಿ-ದರು
ಧರಿ-ಸಿ-ದಾಗ
ಧರಿ-ಸಿದ್ದ
ಧರಿ-ಸಿ-ದ್ದನು
ಧರಿ-ಸಿ-ದ್ದರೂ
ಧರಿ-ಸಿ-ದ್ದಾನೆ
ಧರಿ-ಸಿರು
ಧರಿ-ಸಿ-ರುವ
ಧರಿ-ಸಿ-ರು-ವ-ನೆಂ-ಬುದು
ಧರಿ-ಸಿ-ರು-ವುದು
ಧರಿ-ಸಿ-ರು-ವೆಯಾ
ಧರಿ-ಸು-ತ್ತಾರೆ
ಧರಿ-ಸು-ತ್ತಿ-ದ್ದು-ದ-ರಿಂದ
ಧರಿ-ಸುವ
ಧರಿ-ಸು-ವು-ದಕ್ಕೂ
ಧರಿ-ಸು-ವು-ದಕ್ಕೆ
ಧರೆ
ಧರೇಂದ್ರ
ಧರ್ಮ
ಧರ್ಮಂ
ಧರ್ಮ-ಎಂದು
ಧರ್ಮ-ಕಂ-ಟ-ಕರು
ಧರ್ಮ-ಕ-ರ್ಮ-ಗ-ಳೊಂ-ದಕ್ಕೂ
ಧರ್ಮಕ್ಕೆ
ಧರ್ಮ-ಗಳ
ಧರ್ಮ-ಗಳನ್ನು
ಧರ್ಮ-ಗಳನ್ನೂ
ಧರ್ಮ-ಗ-ಳಿ-ಗಿಂ-ತಲೂ
ಧರ್ಮ-ಗ-ಳಿಗೂ
ಧರ್ಮ-ಗಳು
ಧರ್ಮ-ಗ್ರಂ-ಥ-ಗಳ
ಧರ್ಮದ
ಧರ್ಮ-ದಲ್ಲಿ
ಧರ್ಮ-ದಿಂದ
ಧರ್ಮ-ದೃಷ್ಟಿ
ಧರ್ಮ-ದ್ರೋ-ಹದ
ಧರ್ಮ-ದ್ವೇ-ಷಿ-ಗ-ಳಾದ
ಧರ್ಮನ
ಧರ್ಮ-ಪತ್ನಿ
ಧರ್ಮ-ಪ-ತ್ನಿ-ಯೊ-ಡನೆ
ಧರ್ಮ-ಪರ
ಧರ್ಮ-ಪ-ರ-ನಾ-ಗಿ-ರು-ವುದು
ಧರ್ಮ-ಪ-ರ-ನಾ-ದರೆ
ಧರ್ಮ-ಪ-ರ-ರಾ-ದ-ವರ
ಧರ್ಮ-ಪು-ತ್ರನು
ಧರ್ಮ-ಪು-ರು-ಷನು
ಧರ್ಮ-ಪು-ರು-ಷ-ನೊ-ಡನೆ
ಧರ್ಮ-ಬಾ-ಹಿರ
ಧರ್ಮ-ಬೋ-ಧ-ಕ-ವಾ-ಗಿದೆ
ಧರ್ಮ-ಮಾ-ರ್ಗ-ವನ್ನು
ಧರ್ಮ-ಯು-ದ್ಧಕ್ಕೆ
ಧರ್ಮ-ರ-ಕ್ಷ-ಕ-ನಾದ
ಧರ್ಮ-ರ-ಕ್ಷ-ಣೆ-ಗಾಗಿ
ಧರ್ಮ-ರ-ಕ್ಷ-ಣೆ-ಯ-ಲ್ಲಿಯೇ
ಧರ್ಮ-ರಾಜ
ಧರ್ಮ-ರಾ-ಜನ
ಧರ್ಮ-ರಾ-ಜನು
ಧರ್ಮ-ರಾಯ
ಧರ್ಮ-ರಾ-ಯನ
ಧರ್ಮ-ರಾ-ಯ-ನನ್ನು
ಧರ್ಮ-ರಾ-ಯ-ನಲ್ಲಿ
ಧರ್ಮ-ರಾ-ಯ-ನಿಂದ
ಧರ್ಮ-ರಾ-ಯ-ನಿಗೂ
ಧರ್ಮ-ರಾ-ಯ-ನಿಗೆ
ಧರ್ಮ-ರಾ-ಯನು
ಧರ್ಮ-ರಾ-ಯನೂ
ಧರ್ಮ-ರಾ-ಯ-ನೊ-ಡನೆ
ಧರ್ಮ-ರಾ-ಯ-ನೊ-ಡ-ನೆ-ಮ-ಹಾ-ರಾಜ
ಧರ್ಮ-ವನ್ನು
ಧರ್ಮ-ವನ್ನೂ
ಧರ್ಮ-ವಿದೆ
ಧರ್ಮವೂ
ಧರ್ಮವೇ
ಧರ್ಮ-ಶಾಸ್ತ್ರ
ಧರ್ಮ-ಶಾ-ಸ್ತ್ರ-ಗಳನ್ನು
ಧರ್ಮ-ಸಾಂ-ಕರ್ಯ
ಧರ್ಮ-ಸೂ-ಕ್ಷ್ಮ-ಗಳನ್ನು
ಧರ್ಮ-ಸೂ-ಕ್ಷ್ಮ-ವನ್ನು
ಧರ್ಮ-ಸ್ಥಾ-ಪ-ನೆ-ಗಾ-ಗಿಯೆ
ಧರ್ಮಾ
ಧರ್ಮಾ-ಧ-ರ್ಮ-ಗಳ
ಧರ್ಮಾ-ಧ-ರ್ಮದ
ಧರ್ಮಾ-ಧ-ರ್ಮ-ವೆಂಬ
ಧರ್ಮಾ-ವ-ನಾ-ಯೋ-ರು-ಕೃ-ತಾ-ವ-ತಾರಃ
ಧರ್ಮಿಷ್ಠ
ಧವ-ಳ-ವಾದ
ಧಾಂಡಿ-ಗ-ನಾ-ಗಿದ್ದ
ಧಾಟಿ-ಯನ್ನು
ಧಾಟಿ-ಯಲ್ಲಿ
ಧಾತು-ಗಳು
ಧಾನ-ಗೊಂ-ಡನು
ಧಾನ-ಪ-ಡಿ-ಸ-ಬೇ-ಕಾ-ದರೆ
ಧಾನ-ವಾ-ಯಿತು
ಧಾನಿ-ಯನ್ನು
ಧಾನಿ-ಯಾದ
ಧಾನ್ಯ
ಧಾಮಾ-ಪ-ರ-ರಾತ್ರ
ಧಾರಣ
ಧಾರ-ಣಾ-ಯೋ-ಗದ
ಧಾರ-ಣೆಗೆ
ಧಾರ-ಣೆ-ಯಿಂದ
ಧಾರ-ಣೆ-ಯೆಂದು
ಧಾರಿ
ಧಾರಿ-ಯಾದ
ಧಾರೆ
ಧಾರೆ-ಗಳಿಂದ
ಧಾರೆ-ಧಾ-ರೆ-ಯಾಗಿ
ಧಾರೆ-ಯೆ-ರೆದು
ಧಾರೆ-ಯೆ-ರೆ-ದು-ಕೊ-ಟ್ಟನು
ಧಾರ್ಮಿಕ
ಧಾರ್ಮಿ-ಕ-ರಂತೆ
ಧಾವಿಸಿ
ಧಾವಿ-ಸಿ-ದನು
ಧಾಸ್ಯ-ತ್ಕದಾ
ಧಿಕಾ-ರಿ-ಗಳನ್ನು
ಧಿಕ್ಕ-ರಿಸಿ
ಧಿಕ್ಕ-ರಿ-ಸಿ-ದನು
ಧಿಕ್ಕಾರ
ಧಿಯಾಂ-ಪತಿ
ಧಿಯೋಮೀ
ಧೀ
ಧೀಮಹಿ
ಧೀರ
ಧೀರ-ಗಂ-ಭೀ-ರ-ವಾ-ಣಿ-ಯಿಂದ
ಧೀರ-ನಾ-ಗ-ಬೇಕು
ಧೀರ-ನಾದ
ಧೀರರು
ಧೀರ್ಘಾ-ಯುಷಿ
ಧುಮು-ಕಿ-ದನು
ಧುಮ್ಮಿಕಿ
ಧುಮ್ಮಿಕ್ಕಿ
ಧುಮ್ಮಿ-ಕ್ಕಿ-ದಳು
ಧುಮ್ಮಿ-ಕ್ಕಿ-ದ-ವನೆ
ಧುಮ್ಮಿಕ್ಕು
ಧುಮ್ಮಿ-ಕ್ಕು-ವಾಗ
ಧುಮ್ಮು-ಕ್ಕಿ-ದನು
ಧುರ್ಯೋ-ಧ-ನನೇ
ಧೂಪ
ಧೂಪ-ಗಳು
ಧೂಪ-ಧೂ-ಮ-ದೊ-ಡನೆ
ಧೂಪ-ವನ್ನು
ಧೂಪ-ವಾಗಿ
ಧೂಮ
ಧೂಮ-ಕೇತು
ಧೂಮ-ಗ-ತಿ-ಯಿಂದ
ಧೂಮಾ-ದಿ-ಗ-ತಿ-ಎಂಬ
ಧೂಮಾ-ದಿ-ಗ-ತಿ-ಯನ್ನು
ಧೂಮ್ರ-ಕೇ-ಶ-ನಿಗೆ
ಧೂಳನ್ನು
ಧೂಳಿ
ಧೂಳಿಗೆ
ಧೂಳಿನ
ಧೂಳಿ-ನಿಂದ
ಧೂಳಿ-ಯನ್ನು
ಧೂಳಿ-ಯಲ್ಲಿ
ಧೂಳೀ-ಪಟ
ಧೂಳು
ಧೂಳೆದ್ದು
ಧೂಳೆಲ್ಲ
ಧೃತ-ಕಂ-ಜ-ರ-ಥಾಂ-ಗ-ಗದಂ
ಧೃತ-ದೇ-ವಾ-ತ್ರಿ-ಪೃಷ್ಠ
ಧೃತ-ರಾ-ಷ್ಟ-ನಿಗೆ
ಧೃತ-ರಾಷ್ಟ್ರ
ಧೃತ-ರಾ-ಷ್ಟ್ರನ
ಧೃತ-ರಾ-ಷ್ಟ್ರ-ನನ್ನು
ಧೃತ-ರಾ-ಷ್ಟ್ರ-ನನ್ನೂ
ಧೃತ-ರಾ-ಷ್ಟ್ರ-ನಿಗೆ
ಧೃತ-ರಾ-ಷ್ಟ್ರನು
ಧೃತ-ವ್ರತ
ಧೃತಿ
ಧೃಷ್ಟ-ಕೇತು
ಧೃಷ್ಟ-ದ್ಯುಮ್ನ
ಧೇನುಕ
ಧೇನು-ಕ-ನೆಂಬ
ಧೇನು-ಕ-ರನ್ನು
ಧೇನು-ಕಾ-ಸುರ
ಧೇನು-ಕಾ-ಸು-ರನ
ಧೇನು-ಕಾ-ಸು-ರನು
ಧೈರ್ಯ
ಧೈರ್ಯಂ
ಧೈರ್ಯಕ್ಕೆ
ಧೈರ್ಯ-ಗೊಂಡು
ಧೈರ್ಯ-ದಲ್ಲಿ
ಧೈರ್ಯ-ದಿಂದ
ಧೈರ್ಯ-ಬಂತು
ಧೈರ್ಯ-ವನ್ನು
ಧೈರ್ಯ-ವಾ-ಗ-ಲಿಲ್ಲ
ಧೈರ್ಯ-ವಾಗಿ
ಧೈರ್ಯ-ವಾದ
ಧೈರ್ಯ-ವಿ-ದ್ದರೆ
ಧೈರ್ಯವೂ
ಧೈರ್ಯ-ಶಾಲಿ
ಧೊಪ್ಪೆಂದು
ಧೋರಣೆ
ಧ್ಯಾನ
ಧ್ಯಾನ-ಇದು
ಧ್ಯಾನಕ್ಕೂ
ಧ್ಯಾನಕ್ಕೆ
ಧ್ಯಾನ-ಗಳು
ಧ್ಯಾನ-ತ-ತ್ಪ-ರ-ನಾ-ಗ-ಬೇಕು
ಧ್ಯಾನ-ದಲ್ಲಿ
ಧ್ಯಾನ-ದ-ಲ್ಲಿದ್ದ
ಧ್ಯಾನ-ದ-ಲ್ಲಿಯೇ
ಧ್ಯಾನ-ದ-ಲ್ಲಿ-ರುತ್ತಾ
ಧ್ಯಾನ-ದಿಂದ
ಧ್ಯಾನ-ಪ-ರ-ನಾಗಿ
ಧ್ಯಾನ-ಮಗ್ನ
ಧ್ಯಾನ-ಮ-ಗ್ನ-ನಾಗಿ
ಧ್ಯಾನ-ಮ-ಗ್ನ-ನಾ-ಗಿದ್ದ
ಧ್ಯಾನ-ಮ-ಗ್ನ-ಳಾ-ದಳು
ಧ್ಯಾನ-ಮ-ಗ್ನ-ವಾ-ಯಿತು
ಧ್ಯಾನ-ಮಾಡ
ಧ್ಯಾನ-ಮಾ-ಡ-ಬೇಕು
ಧ್ಯಾನ-ಮಾ-ಡಲಿ
ಧ್ಯಾನ-ಮಾಡಿ
ಧ್ಯಾನ-ಮಾ-ಡಿ-ದ-ನುಹೇ
ಧ್ಯಾನ-ಮಾ-ಡಿರಿ
ಧ್ಯಾನ-ಮಾಡು
ಧ್ಯಾನ-ಮಾ-ಡುತ್ತ
ಧ್ಯಾನ-ಮಾ-ಡುತ್ತಾ
ಧ್ಯಾನ-ಮಾ-ಡು-ತ್ತಾರೆ
ಧ್ಯಾನ-ಮಾ-ಡು-ತ್ತಿ-ರು-ವಂತೆ
ಧ್ಯಾನ-ಮಾ-ಡು-ತ್ತಿ-ರು-ವಾಗ
ಧ್ಯಾನ-ಮಾ-ಡುವ
ಧ್ಯಾನ-ಮಾ-ಡು-ವಂ-ತಹ
ಧ್ಯಾನ-ಮಾ-ಡು-ವು-ದ-ರಿಂದ
ಧ್ಯಾನಮ್
ಧ್ಯಾನ-ವನ್ನು
ಧ್ಯಾನ-ಸ್ತಿ-ಮಿತ
ಧ್ಯಾನ-ಸ್ಥಿ-ಮಿ-ತ-ಮೂ-ರ್ತಿ-ಯಾಗಿ
ಧ್ಯಾನಾ-ಸ-ಕ್ತ-ನಾಗಿ
ಧ್ಯಾನಿ
ಧ್ಯಾನಿ-ಸ-ಬೇಕು
ಧ್ಯಾನಿಸಿ
ಧ್ಯಾನಿ-ಸಿ-ದನು
ಧ್ಯಾನಿ-ಸಿ-ದ-ನುಹೇ
ಧ್ಯಾನಿ-ಸಿ-ದ-ಮೇಲೆ
ಧ್ಯಾನಿ-ಸಿ-ದರೆ
ಧ್ಯಾನಿಸು
ಧ್ಯಾನಿ-ಸುತ್ತ
ಧ್ಯಾನಿ-ಸುತ್ತಾ
ಧ್ಯಾನಿ-ಸು-ತ್ತಾರೆ
ಧ್ಯಾನಿ-ಸುತ್ತಿ
ಧ್ಯಾನಿ-ಸು-ತ್ತಿದ್ದ
ಧ್ಯಾನಿ-ಸು-ತ್ತಿ-ರಲು
ಧ್ಯಾನಿ-ಸು-ತ್ತಿ-ರುವ
ಧ್ಯಾನಿ-ಸುವ
ಧ್ಯಾನಿ-ಸು-ವಾಗ
ಧ್ಯಾನಿ-ಸು-ವೆ-ಯೇನು
ಧ್ಯಾಯತಿ
ಧ್ಯಾಯ-ನಲ್ಲಿ
ಧ್ಯಾಯೇ
ಧ್ಯೇಯ-ಗಳನ್ನೂ
ಧ್ಯೇಯ-ಗ-ಳೆಲ್ಲ
ಧ್ರುವ
ಧ್ರುವ-ಕು-ಮಾರ
ಧ್ರುವ-ಕು-ಮಾ-ರನ
ಧ್ರುವ-ಕು-ಮಾ-ರನು
ಧ್ರುವ-ಚ-ಕ್ರ-ವರ್ತಿ
ಧ್ರುವ-ಚ-ಕ್ರ-ವ-ರ್ತಿಯ
ಧ್ರುವ-ಚ-ಕ್ರ-ವ-ರ್ತಿ-ಯ-ನಂ-ತರ
ಧ್ರುವ-ಚ-ಕ್ರಿ-ಯನ್ನು
ಧ್ರುವನ
ಧ್ರುವ-ನ-ಕ್ಷ-ತ್ರ-ಕ್ಕಿಂ-ತಲೂ
ಧ್ರುವನು
ಧ್ರುವ-ನೇ-ರಿದ
ಧ್ರುವ-ಪ-ದ-ವೆ-ನಿ-ಸು-ತ್ತದೆ
ಧ್ರುವ-ಮಂ-ಡ-ಲ-ವನ್ನು
ಧ್ರುವ-ಲೋಕ
ಧ್ವಂಸ
ಧ್ವಂಸ-ಮಾ-ಡಲು
ಧ್ವಂಸ-ಮಾಡಿ
ಧ್ವಂಸ-ಮಾ-ಡಿತು
ಧ್ವಂಸ-ಮಾ-ಡಿದ
ಧ್ವಂಸ-ಮಾ-ಡಿ-ಬ-ರ-ಬೇಕು
ಧ್ವಂಸ-ಮಾ-ಡಿ-ಸಿದ
ಧ್ವಂಸ-ಮಾ-ಡುತ್ತಾ
ಧ್ವಂಸ-ಮಾ-ಡು-ವನು
ಧ್ವಜ
ಧ್ವಜ-ಗಳು
ಧ್ವಜ-ಗ-ಳೊ-ಡನೆ
ಧ್ವಜ-ಪ-ತಾ-ಕೆ-ಗಳು
ಧ್ವಜವು
ಧ್ವನಿ
ಧ್ವನಿ-ಇ-ವು-ಗಳನ್ನು
ಧ್ವನಿ-ಯನ್ನು
ಧ್ವನಿ-ಯಿಂದ
ಧ್ವನಿಯೂ
ನ
ನಂ
ನಂಚಿ-ಕೊ-ಳ್ಳುತ್ತಾ
ನಂಟ-ತ-ನವೂ
ನಂತರ
ನಂತ-ರದ
ನಂತ-ರ-ದ-ವ-ರೆಗೂ
ನಂತಹ
ನಂತಾ-ಯಿತು
ನಂತಿದ್ದ
ನಂತೆ
ನಂತೆಯೆ
ನಂತೆಯೇ
ನಂದ
ನಂದ-ಇ-ಬ್ಬರೂ
ನಂದ-ಕ-ಖ-ಡ್ಗವೇ
ನಂದ-ಗೋ-ಕುಲ
ನಂದ-ಗೋ-ಕು-ಲಕ್ಕೆ
ನಂದ-ಗೋ-ಕು-ಲದ
ನಂದ-ಗೋ-ಕು-ಲ-ದಲ್ಲಿ
ನಂದ-ಗೋ-ಕು-ಲ-ದಿಂದ
ನಂದ-ಗೋ-ಕು-ಲ-ವನ್ನು
ನಂದ-ಗೋಪ
ನಂದ-ಗೋ-ಪ-ಕು-ಮಾ-ರ-ನಿಗೆ
ನಂದ-ಗೋ-ಪನ
ನಂದ-ಗೋ-ಪನು
ನಂದ-ಗೋ-ಪನೂ
ನಂದ-ಗೋ-ಪ-ನೊ-ಡನೆ
ನಂದನ
ನಂದ-ನನ್ನು
ನಂದ-ನ-ವ-ನ-ವನ್ನು
ನಂದ-ನ-ವ-ನವೇ
ನಂದ-ನಿಗೆ
ನಂದ-ನಿ-ಲ್ಲ-ದಿ-ರು-ವಾಗ
ನಂದನು
ನಂದನೂ
ನಂದನೇ
ನಂದ-ನೊ-ಡನೆ
ನಂದ-ಯ-ಶೋದೆ
ನಂದ-ರಾ-ಜನ
ನಂದ-ರಾ-ಜ-ನಿಗೆ
ನಂದರು
ನಂದಾ-ದಿ-ಗ-ಳೆಲ್ಲ
ನಂದಾ-ದಿ-ಗೋ-ಪಾ-ಲ-ರಿಗೆ
ನಂದಿಯ
ನಂದಿ-ಯನ್ನು
ನಂದಿ-ಯ-ನ್ನೇರಿ
ನಂದಿಸಿ
ನಂದೀ-ಗ್ರಾ-ಮಕ್ಕೆ
ನಂದೀ-ಶ್ವ-ರನು
ನಂದೆ
ನಂಬ-ದಷ್ಟು
ನಂಬ-ಬಾ-ರದು
ನಂಬ-ಬೇಡ
ನಂಬ-ಲಾ-ರ-ದಾ-ದಳು
ನಂಬ-ಲಿಲ್ಲ
ನಂಬಿ
ನಂಬಿಕೆ
ನಂಬಿ-ಕೆ-ಯಿಲ್ಲ
ನಂಬಿ-ಕೊಂ-ಡಿತ್ತು
ನಂಬಿ-ಕೊಂ-ಡಿದೆ
ನಂಬಿ-ಕೊಂ-ಡಿ-ದ್ದೇವೆ
ನಂಬಿದ
ನಂಬಿ-ದ-ವ-ರನ್ನು
ನಂಬಿ-ದ-ವ-ರಿಗೆ
ನಂಬಿ-ದೆವು
ನಂಬಿ-ದ್ದೇನೆ
ನಂಬಿ-ರುವ
ನಂಬಿ-ರು-ವು-ದ-ರಿಂದ
ನಂಬುಗೆ
ನಂಬು-ತ್ತಾ-ರೆಯೆ
ನಂಬು-ತ್ತಾ-ರೆಯೇ
ನಂಬುವ
ನಂಬು-ವು-ದಕ್ಕೆ
ನಂಬು-ವು-ದಾ-ದರೆ
ನಃ
ನಕ-ಲಿಯ
ನಕುಲ
ನಕು-ಲನು
ನಕ್ಕ
ನಕ್ಕ-ಕಾಲ
ನಕ್ಕ-ನಂತೆ
ನಕ್ಕನು
ನಕ್ಕರು
ನಕ್ಕರೆ
ನಕ್ಕಳು
ನಕ್ಕವು
ನಕ್ಕು
ನಕ್ಕೆ
ನಕ್ಷತ್ರ
ನಕ್ಷ-ತ್ರ-ಕಾಂ-ತಿಯೂ
ನಕ್ಷ-ತ್ರ-ಗಳ
ನಕ್ಷ-ತ್ರ-ಗ-ಳಂತೆ
ನಕ್ಷ-ತ್ರ-ಗಳನ್ನು
ನಕ್ಷ-ತ್ರ-ಗ-ಳ-ನ್ನೊಳ
ನಕ್ಷ-ತ್ರ-ಗ-ಳಿ-ಗಿಂ-ತಲೂ
ನಕ್ಷ-ತ್ರ-ಗ-ಳಿ-ವೆಯೋ
ನಕ್ಷ-ತ್ರ-ಗಳು
ನಕ್ಷ-ತ್ರ-ಗಳೂ
ನಕ್ಷ-ತ್ರ-ಗ-ಳೆಲ್ಲ
ನಕ್ಷ-ತ್ರ-ಗ-ಳೆ-ಲ್ಲವೂ
ನಕ್ಷ-ತ್ರ-ಮಂ-ಡಲ
ನಕ್ಷ-ತ್ರ-ಲೋಕ
ನಕ್ಷ-ತ್ರ-ವಾದ
ನಕ್ಷ-ತ್ರವು
ನಕ್ಷ-ತ್ರ-ಸ-ಹಿ-ತ-ನಾದ
ನಖ-ಕ್ಷ-ತಿಗೆ
ನಗರ
ನಗ-ರ-ಕ್ಕಿದ್ದ
ನಗ-ರಕ್ಕೆ
ನಗ-ರದ
ನಗ-ರ-ದಲ್ಲಿ
ನಗ-ರ-ವಾ-ವುದೋ
ನಗ-ರ-ವೆಲ್ಲ
ನಗ-ರಿಗೆ
ನಗ-ರಿ-ಯತ್ತ
ನಗ-ರಿ-ಯನ್ನು
ನಗಾ-ರಿ-ಯನ್ನು
ನಗಿಸಿ
ನಗಿ-ಸಿದ
ನಗಿ-ಸು-ತ್ತಿದ್ದ
ನಗಿ-ಸು-ವನು
ನಗು
ನಗುತ್ತ
ನಗು-ತ್ತಲೆ
ನಗುತ್ತಾ
ನಗು-ತ್ತಾನೆ
ನಗು-ತ್ತಿ-ದ್ದರು
ನಗು-ತ್ತಿ-ದ್ದಾರೆ
ನಗು-ತ್ತಿ-ರು-ವಾಗ
ನಗು-ತ್ತಿ-ರು-ವುದನ್ನು
ನಗು-ನ-ಗು-ತ್ತಲೆ
ನಗು-ನ-ಗುತ್ತಾ
ನಗು-ಬಂತು
ನಗು-ಮು-ಖ-ಗ-ಳಿಂ-ದಲೂ
ನಗುವ
ನಗು-ವಂ-ತಿ-ದ್ದವು
ನಗು-ವನು
ನಗು-ವನ್ನು
ನಗು-ವರು
ನಗು-ವಳು
ನಗು-ವಿನ
ನಗು-ವು-ದಿ-ಲ್ಲವೆ
ನಗೆ
ನಗೆಗೂ
ನಗೆ-ಚೆ-ಲ್ಲ-ಹೊ-ರಟ
ನಗೆ-ನು-ಡಿ-ಗಳಿಂದ
ನಗೆ-ಪಾ-ಟ-ಲಾ-ದರೂ
ನಗೆ-ಪಾ-ಟ-ಲಾ-ದೀ-ತೆಂದು
ನಗೆ-ಪಾ-ಟ-ಲಿಗೆ
ನಗೆ-ಪಾ-ಟಲು
ನಗೆ-ಬಂತು
ನಗೆ-ಮೊ-ಗ-ದಿಂದ
ನಗೆ-ಯಿಂದ
ನಗ್ನ-ಜಿತ್
ನಗ್ನ-ಜಿ-ತ್ತನು
ನಗ್ನ-ನಾಗಿ
ನಟನೆ
ನಟ-ನೆ-ಗಾಗಿ
ನಟ-ನೆ-ಯನ್ನು
ನಟ-ನೆ-ಯಲ್ಲಿ
ನಟಿಸಿ
ನಟಿ-ಸಿ-ದಳು
ನಟಿ-ಸುತ್ತ
ನಟಿ-ಸುತ್ತಾ
ನಟಿ-ಸುತ್ತಿ
ನಟಿ-ಸು-ತ್ತಿದ್ದ
ನಟಿ-ಸು-ತ್ತಿ-ದ್ದನು
ನಟಿ-ಸು-ವ-ವ-ರನ್ನು
ನಟಿ-ಸು-ವು-ದ-ಕ್ಕಾಗಿ
ನಟಿ-ಸು-ವೆ-ಯಾ-ದರೂ
ನಟ್ಟ
ನಟ್ಟಿರು
ನಟ್ಟಿ-ರು-ಳಿ-ನಲ್ಲಿ
ನಡ
ನಡ-ಗುತ್ತ
ನಡ-ಗುತ್ತಾ
ನಡಗೆ
ನಡ-ಗೆಗೆ
ನಡ-ಗೆ-ಯ-ಲ್ಲಿಯೂ
ನಡತೆ
ನಡ-ತೆ-ಗೆಟ್ಟ
ನಡ-ತೆ-ಗೆ-ಟ್ಟ-ವ-ನಾ-ಗಿ-ರಲಿ
ನಡ-ತೆ-ಗೆ-ಟ್ಟ-ವಳೆ
ನಡ-ತೆ-ಯನ್ನು
ನಡ-ತೆ-ಯನ್ನೇ
ನಡ-ತೆಯು
ನಡ-ವ-ಳಿಕೆ
ನಡ-ವ-ಳಿ-ಕೆ-ಯನ್ನು
ನಡ-ವ-ಳಿ-ಕೆ-ಯಿಂದ
ನಡ-ಸ-ಬೇ-ಕಾದ
ನಡ-ಸು-ತ್ತಿದ್ದ
ನಡ-ಸು-ವೆನು
ನಡಿ
ನಡಿಗೆ
ನಡಿ-ಗೆ-ಯಿಂ-ದಲೆ
ನಡು
ನಡುಕ
ನಡು-ಕ-ಟ್ಟನ್ನು
ನಡು-ಕಟ್ಟು
ನಡು-ಗಡ
ನಡು-ಗ-ಡ್ಡೆ-ಗಳೇ
ನಡು-ಗ-ಬೇ-ಕಾದು
ನಡು-ಗ-ಲಿಲ್ಲ
ನಡುಗಿ
ನಡು-ಗಿತು
ನಡು-ಗಿ-ದರು
ನಡು-ಗಿ-ದರೂ
ನಡು-ಗಿಸಿ
ನಡು-ಗಿ-ಸಿದ
ನಡು-ಗಿ-ಹೋ-ಗು-ತ್ತಿತ್ತು
ನಡು-ಗಿ-ಹೋ-ಗು-ತ್ತಿ-ದ್ದರು
ನಡು-ಗಿ-ಹೋ-ದನು
ನಡು-ಗಿ-ಹೋ-ದರು
ನಡು-ಗಿ-ಹೋ-ದಳು
ನಡು-ಗುತ್ತ
ನಡು-ಗುತ್ತಾ
ನಡು-ಗು-ತ್ತಿತ್ತು
ನಡು-ಗು-ತ್ತಿದ್ದ
ನಡು-ಗು-ತ್ತಿ-ದ್ದರು
ನಡು-ಗು-ತ್ತಿ-ರಲು
ನಡು-ಗು-ತ್ತಿ-ರುವ
ನಡು-ಗು-ವಂ-ತಹ
ನಡು-ಗು-ವಂ-ತಿದ್ದ
ನಡು-ಗು-ವಂತೆ
ನಡು-ಗು-ವುದು
ನಡುಗೆ
ನಡು-ದಾರಿ-ಯಲ್ಲಿ
ನಡು-ನೆ-ತ್ತಿಯ
ನಡು-ನೆ-ತ್ತಿ-ಯನ್ನು
ನಡು-ನೆ-ತ್ತಿ-ಯ-ವ-ರೆಗೆ
ನಡು-ನೆ-ತ್ತಿ-ಯಿಂದ
ನಡು-ರಾತ್ರಿ
ನಡು-ವಿನ
ನಡು-ವಿ-ನಲ್ಲಿ
ನಡು-ಹಾ-ದಿ-ಯ-ಲ್ಲಿಯೆ
ನಡೆ
ನಡೆಗೆ
ನಡೆದ
ನಡೆ-ದಂತೆ
ನಡೆ-ದನು
ನಡೆ-ದರು
ನಡೆ-ದರೂ
ನಡೆ-ದಳು
ನಡೆ-ದ-ವನೂ
ನಡೆ-ದ-ವರು
ನಡೆ-ದವು
ನಡೆ-ದಾಗ
ನಡೆ-ದಾ-ಡು-ತ್ತಿ-ರು-ವಾಗ
ನಡೆ-ದಿತ್ತು
ನಡೆ-ದಿ-ದ್ದವು
ನಡೆ-ದಿ-ದ್ದು-ವೆಂ-ಬು-ದನ್ನು
ನಡೆ-ದಿ-ರ-ಬೇಕು
ನಡೆ-ದಿ-ರ-ಬೇಕೆ
ನಡೆ-ದಿ-ರ-ಬೇ-ಕೆಂದು
ನಡೆದು
ನಡೆ-ದುಕೊ
ನಡೆ-ದು-ಕೊಂಡ
ನಡೆ-ದು-ಕೊಂ-ಡನು
ನಡೆ-ದು-ಕೊಂ-ಡರು
ನಡೆ-ದು-ಕೊಂ-ಡರೆ
ನಡೆ-ದು-ಕೊಂಡು
ನಡೆ-ದು-ಕೊಂ-ಡು-ದ-ಕ್ಕಾಗಿ
ನಡೆ-ದು-ಕೊ-ಳ್ಳ-ಬೇ-ಕಾದ
ನಡೆ-ದು-ಕೊಳ್ಳು
ನಡೆ-ದು-ಕೊ-ಳ್ಳುತ್ತ
ನಡೆ-ದು-ಕೊ-ಳ್ಳು-ತ್ತಿ-ದ್ದನು
ನಡೆ-ದು-ಕೊ-ಳ್ಳು-ತ್ತೇನೆ
ನಡೆ-ದು-ಕೊ-ಳ್ಳು-ವು-ದಾಗಿ
ನಡೆ-ದು-ಕೊ-ಳ್ಳು-ವುದು
ನಡೆ-ದು-ಕೊ-ಳ್ಳೋಣ
ನಡೆ-ದು-ದನ್ನು
ನಡೆ-ದು-ದ-ನ್ನೆಲ್ಲ
ನಡೆ-ದುದು
ನಡೆ-ದು-ದೆಂ-ಬಂತೆ
ನಡೆ-ದು-ಬಂದು
ನಡೆ-ದು-ಹೋ-ಗಿದೆ
ನಡೆ-ದು-ಹೋದ
ನಡೆ-ದು-ಹೋ-ದು-ದನ್ನು
ನಡೆ-ದು-ಹೋ-ಯಿತು
ನಡೆದೇ
ನಡೆ-ನು-ಡಿ-ಗಳನ್ನು
ನಡೆಯ
ನಡೆ-ಯ-ದಿ-ರು-ವುದು
ನಡೆ-ಯ-ಬಲ್ಲ
ನಡೆ-ಯ-ಬ-ಹು-ದಾದ
ನಡೆ-ಯ-ಬ-ಹುದು
ನಡೆ-ಯ-ಬೇಕಾ
ನಡೆ-ಯ-ಬೇ-ಕಾ-ಗಿದೆ
ನಡೆ-ಯ-ಬೇ-ಕಾದ
ನಡೆ-ಯ-ಬೇಕೇ
ನಡೆ-ಯ-ಲಾ-ರದೆ
ನಡೆ-ಯಿತು
ನಡೆ-ಯಿ-ತೆಂದು
ನಡೆ-ಯಿರಿ
ನಡೆಯು
ನಡೆ-ಯು-ತ್ತ-ವೆಯೊ
ನಡೆ-ಯುತ್ತಾ
ನಡೆ-ಯು-ತ್ತಾ-ಇದೆ
ನಡೆ-ಯು-ತ್ತಿತ್ತು
ನಡೆ-ಯು-ತ್ತಿದ್ದ
ನಡೆ-ಯು-ತ್ತಿ-ದ್ದರೆ
ನಡೆ-ಯು-ತ್ತಿ-ರ-ಲಿ-ಲ್ಲ-ವಾ-ದ್ದ-ರಿಂದ
ನಡೆ-ಯು-ತ್ತಿ-ರುವ
ನಡೆ-ಯುವ
ನಡೆ-ಯು-ವಂ-ತಾ-ಗಲಿ
ನಡೆ-ಯು-ವಂ-ತಾ-ಯಿತು
ನಡೆ-ಯು-ವಂತೆ
ನಡೆ-ಯು-ವಲ್ಲಿ
ನಡೆ-ಯು-ವ-ವನು
ನಡೆ-ಯುವು
ನಡೆ-ಯು-ವು-ದಂತೆ
ನಡೆ-ಯು-ವು-ದಕ್ಕೂ
ನಡೆ-ಯು-ವು-ದಕ್ಕೆ
ನಡೆ-ಯು-ವು-ದಾಗಿ
ನಡೆ-ಯು-ವು-ದಿಲ್ಲ
ನಡೆ-ಯು-ವುದು
ನಡೆ-ಯು-ವು-ದೆಲ್ಲ
ನಡೆ-ಯು-ವುದೋ
ನಡೆವ
ನಡೆ-ವಂತೆ
ನಡೆ-ವ-ಳಿ-ಕೆ-ಯಿಂ-ದಲೂ
ನಡೆ-ವೆಣ
ನಡೆಸ
ನಡೆ-ಸ-ಕೂ-ಡದು
ನಡೆ-ಸ-ಬೇಕು
ನಡೆ-ಸರು
ನಡೆ-ಸ-ಲಾರ
ನಡೆ-ಸಲಿ
ನಡೆ-ಸ-ಲಿಲ್ಲ
ನಡೆ-ಸಲು
ನಡೆಸಿ
ನಡೆ-ಸಿ-ಕೊ-ಟ್ಟರೆ
ನಡೆ-ಸಿ-ಕೊ-ಡ-ಬೇಕು
ನಡೆ-ಸಿ-ಕೊ-ಡು-ತ್ತೇನೆ
ನಡೆ-ಸಿ-ಕೊ-ಡು-ವು-ದಾಗಿ
ನಡೆ-ಸಿದ
ನಡೆ-ಸಿ-ದನು
ನಡೆ-ಸಿ-ದ-ಮೇಲೆ
ನಡೆ-ಸಿ-ದರು
ನಡೆ-ಸಿ-ದರೂ
ನಡೆ-ಸಿ-ದರೆ
ನಡೆ-ಸಿ-ದು-ದೆ-ಲ್ಲ-ವನ್ನೂ
ನಡೆ-ಸಿ-ದ್ದಾರೆ
ನಡೆ-ಸಿರಿ
ನಡೆ-ಸುತ್ತಾ
ನಡೆ-ಸು-ತ್ತಾರೆ
ನಡೆ-ಸು-ತ್ತಿದ್ದ
ನಡೆ-ಸು-ತ್ತಿ-ದ್ದರು
ನಡೆ-ಸು-ತ್ತಿ-ದ್ದರೂ
ನಡೆ-ಸು-ತ್ತಿ-ದ್ದಾರೆ
ನಡೆ-ಸು-ತ್ತಿ-ರಲು
ನಡೆ-ಸು-ತ್ತಿ-ರುವ
ನಡೆ-ಸು-ತ್ತಿ-ರು-ವ-ವನು
ನಡೆ-ಸು-ತ್ತಿ-ರುವು
ನಡೆ-ಸು-ತ್ತಿ-ರು-ವೆ-ನಲ್ಲಾ
ನಡೆ-ಸು-ತ್ತಿ-ರು-ವೆ-ಯಲ್ಲಾ
ನಡೆ-ಸುವ
ನಡೆ-ಸು-ವಂತೆ
ನಡೆ-ಸು-ವ-ನೆಂದು
ನಡೆ-ಸು-ವ-ವರೂ
ನಡೆ-ಸು-ವ-ಷ್ಟ-ರಲ್ಲಿ
ನಡೆ-ಸುವು
ನಡೆ-ಸು-ವು-ದ-ರಿಂದ
ನಡೆ-ಸು-ವುದು
ನಡೆ-ಸು-ವುದೇ
ನಡೆ-ಸು-ವು-ದ್ಕಾಗಿ
ನಡೆ-ಸೋಣ
ನತೋಽಸ್ಮಿ
ನತ್ತ
ನತ್ತ-ಮ-ದೃಷ್ಟ
ನದಿ
ನದಿ-ಗಳ
ನದಿ-ಗಳಲ್ಲಿ
ನದಿ-ಗ-ಳಿವೆ
ನದಿ-ಗಳು
ನದಿ-ಗಳೂ
ನದಿ-ಗ-ಳೆಲ್ಲ
ನದಿಗೆ
ನದಿಯ
ನದಿ-ಯಂತೆ
ನದಿ-ಯನ್ನು
ನದಿ-ಯಲ್ಲಿ
ನದಿ-ಯಾಗಿ
ನದಿ-ಯಿಂದ
ನದಿಯು
ನದಿ-ಯೊಂದು
ನದೀ-ತೀ-ರ-ದಲ್ಲಿ
ನದೀ-ತೀ-ರ-ವನ್ನು
ನದೆ
ನನ
ನನ-ಗ-ನಿ-ಸಿ-ತು-ಎಲಾ
ನನ-ಗ-ನಿ-ಸು-ತ್ತಿದೆ
ನನ-ಗ-ನ್ನಿ-ಸು-ತ್ತದೆ
ನನ-ಗ-ಲ್ಲದೆ
ನನ-ಗಾ-ಗಲಿ
ನನ-ಗಾಗಿ
ನನ-ಗಿಂತ
ನನ-ಗಿಂ-ತಲೂ
ನನ-ಗಿಂದು
ನನ-ಗಿದೆ
ನನ-ಗಿನ್ನು
ನನ-ಗಿ-ನ್ನೇನು
ನನ-ಗಿ-ರು-ವು-ದಾ-ದರೂ
ನನ-ಗೀಗ
ನನಗೂ
ನನಗೆ
ನನ-ಗೆ-ತ-ನಗೆ
ನನ-ಗೆಲ್ಲಿ
ನನಗೇ
ನನ-ಗೇಕೆ
ನನ-ಗೇನು
ನನ-ಗೊಂದು
ನನ-ಗೊ-ಪ್ಪಿಸಿ
ನನ-ಗೊ-ಪ್ಪಿಸು
ನನ-ಗೊಬ್ಬ
ನನಸಾ
ನನ-ಸಾ-ಗ-ಬೇ-ಕಾ-ದರೆ
ನನ-ಸಾ-ಯಿತು
ನನ್ನ
ನನ್ನಂ
ನನ್ನಂ-ತಹ
ನನ್ನಂತೆ
ನನ್ನಂ-ತೆಯೆ
ನನ್ನಣ್ಣ
ನನ್ನತ್ತ
ನನ್ನ-ದನ್ನು
ನನ್ನ-ದಲ್ಲ
ನನ್ನ-ದಾ-ಗಿತ್ತು
ನನ್ನದು
ನನ್ನ-ದೇನೂ
ನನ್ನ-ದೊಂದು
ನನ್ನನ್ನ
ನನ್ನ-ನ್ನಾ-ದರೂ
ನನ್ನನ್ನು
ನನ್ನನ್ನೂ
ನನ್ನನ್ನೆ
ನನ್ನನ್ನೇ
ನನ್ನ-ನ್ನೇಕೆ
ನನ್ನಲ್ಲಿ
ನನ್ನ-ಲ್ಲಿ-ದೆ-ಯೆಂದು
ನನ್ನ-ಲ್ಲಿಯೇ
ನನ್ನವು
ನನ್ನಾಗಿ
ನನ್ನಾ-ದರೂ
ನನ್ನಿಂದ
ನನ್ನಿಂ-ದಲೆ
ನನ್ನಿಂ-ದೇ-ನಾ-ಗ-ಬೇಕು
ನನ್ನು
ನನ್ನೂ
ನನ್ನೆ
ನನ್ನೆ-ದು-ರಿಗೆ
ನನ್ನೆಲ್ಲ
ನನ್ನೊ-ಡನೆ
ನಭಗ
ನಭ-ಗನ
ನಭ-ಗನು
ನಭ-ಗ-ರಾ-ಜನ
ನಭ-ಸ್ವತಿ
ನಭ-ಸ್ವ-ತಿ-ಯಲ್ಲಿ
ನಭೋ-ಮಂ-ಡ-ಲ-ವನ್ನೂ
ನಮ
ನಮಃ
ನಮಃ-ಕ-ರ್ಮ-ಫಲ
ನಮಃ-ಧರ್ಮ
ನಮಃ-ಷ-ಡ್ಗುಣ
ನಮ-ಗಾ-ಗ-ದಿ-ರಲಿ
ನಮ-ಗಾಗಿ
ನಮ-ಗಾ-ಗಿಯೆ
ನಮ-ಗಾ-ರಿಗೂ
ನಮ-ಗಾರು
ನಮ-ಗಾವ
ನಮ-ಗಿಂತ
ನಮ-ಗಿಂದು
ನಮ-ಗಿಲ್ಲ
ನಮ-ಗಿ-ಲ್ಲ-ವಲ್ಲ
ನಮ-ಗಿ-ಲ್ಲ-ವಾ-ಯಿತು
ನಮ-ಗಿ-ಲ್ಲವೋ
ನಮ-ಗುಂ-ಟಾ-ಗ-ಬ-ಹು-ದಾದ
ನಮಗೂ
ನಮ-ಗೂ-ಗೊ-ತ್ತಿದೆ
ನಮಗೆ
ನಮ-ಗೆಲ್ಲ
ನಮಗೇ
ನಮ-ಗೇನು
ನಮ-ಗೊಂದು
ನಮ-ಗೊ-ಪ್ಪಿಸು
ನಮಸ್ಕ
ನಮ-ಸ್ಕರಿ
ನಮ-ಸ್ಕ-ರಿ-ಸ-ಬ-ಹುದು
ನಮ-ಸ್ಕ-ರಿ-ಸ-ಲಿಲ್ಲ
ನಮ-ಸ್ಕ-ರಿ-ಸಲು
ನಮ-ಸ್ಕ-ರಿಸಿ
ನಮ-ಸ್ಕ-ರಿ-ಸಿ-ಕೊ-ಳ್ಳ-ಬ-ಹುದು
ನಮ-ಸ್ಕ-ರಿ-ಸಿದ
ನಮ-ಸ್ಕ-ರಿ-ಸಿ-ದನು
ನಮ-ಸ್ಕ-ರಿ-ಸಿ-ದರು
ನಮ-ಸ್ಕ-ರಿ-ಸಿ-ದಳು
ನಮ-ಸ್ಕ-ರಿ-ಸಿ-ದ-ವ-ರತ್ತ
ನಮ-ಸ್ಕ-ರಿಸು
ನಮ-ಸ್ಕ-ರಿ-ಸುತ್ತಾ
ನಮ-ಸ್ಕ-ರಿ-ಸು-ತ್ತಾನೆ
ನಮ-ಸ್ಕ-ರಿ-ಸು-ತ್ತಾರೆ
ನಮ-ಸ್ಕ-ರಿ-ಸು-ತ್ತೇನೆ
ನಮ-ಸ್ಕ-ರಿ-ಸುವ
ನಮ-ಸ್ಕ-ರಿ-ಸು-ವು-ದಕ್ಕೆ
ನಮ-ಸ್ಕಾರ
ನಮ-ಸ್ಕಾ-ರ-ಇ-ತ್ಯಾ-ದಿ-ಯಾಗಿ
ನಮ-ಸ್ಕಾ-ರ-ಎಂಬ
ನಮ-ಸ್ಕಾ-ರ-ಗಳನ್ನು
ನಮ-ಸ್ಕಾ-ರ-ಗೊಂಡು
ನಮ-ಸ್ಕಾ-ರ-ಮಾ-ಡದ
ನಮ-ಸ್ಕಾ-ರ-ಮಾಡಿ
ನಮ-ಸ್ಕಾ-ರ-ವನ್ನು
ನಮ-ಸ್ಕಾ-ರ-ವನ್ನೂ
ನಮ-ಸ್ತುಭ್ಯಂ
ನಮಸ್ತೇ
ನಮ-ಸ್ತೇಜ
ನಮ-ಸ್ತೇ-ಽನಂತ
ನಮಸ್ಯೇ
ನಮ-ಸ್ಸಂ-ಕ-ರ್ಷ-ಣಾಯ
ನಮಿಸಿ
ನಮಿ-ಸು-ತ್ತೇನೆ
ನಮುಚಿ
ನಮೊ
ನಮೋ
ನಮೋ-ನಮಃ
ನಮೋ-ನಮೋ
ನಮೋ-ಽಕಿಂ-ಚ-ನ-ವಿ-ತ್ತಾಯ
ನಮೋ-ಽವ-ಸ್ಥಾ-ನಾಯ
ನಮ್ಮ
ನಮ್ಮಂ
ನಮ್ಮಂ-ತಹ
ನಮ್ಮಂ-ತ-ಹ-ವರ
ನಮ್ಮಂತೆ
ನಮ್ಮಂ-ತೆಯೆ
ನಮ್ಮಂ-ತೆಯೇ
ನಮ್ಮ-ಚೇ-ತನ
ನಮ್ಮಣ್ಣ
ನಮ್ಮನ್ನು
ನಮ್ಮ-ನ್ನೆಲ್ಲಾ
ನಮ್ಮ-ನ್ನೆಲ್ಲಿ
ನಮ್ಮ-ನ್ನೇಕೆ
ನಮ್ಮಪ್ಪ
ನಮ್ಮ-ಪ್ಪನ
ನಮ್ಮ-ಪ್ಪ-ನಿಗೆ
ನಮ್ಮ-ಮ-ನ-ಸ್ಸಿಗೆ
ನಮ್ಮಲ್ಲಿ
ನಮ್ಮ-ಲ್ಲಿ-ರು-ವಾಗ
ನಮ್ಮ-ವ-ರಾ-ಗಿರು
ನಮ್ಮಿಂದ
ನಮ್ಮಿ-ಬ್ಬರ
ನಮ್ಮಿ-ಬ್ಬ-ರಿಗೂ
ನಮ್ಮೂ-ರಿ-ನಿಂದ
ನಮ್ಮೆ-ಲ್ಲರ
ನಮ್ಮೆ-ಲ್ಲ-ರಿಗೂ
ನಮ್ರ-ತೆ-ಯಿಂದ
ನಮ್ರ-ನಾಗಿ
ನಮ್ರ-ವಾಗಿ
ನಯ
ನಯನ
ನಯ-ಮಾ-ಡಿ-ದಂ-ತಿ-ರುವ
ನಯಸಿ
ನರ
ನರ
ನರಃ
ನರಕ
ನರ-ಕಕ್ಕೂ
ನರ-ಕಕ್ಕೆ
ನರ-ಕ-ಗ-ಳಿಗೂ
ನರ-ಕ-ದಂ-ತಿ-ರುವ
ನರ-ಕ-ದಲ್ಲಿ
ನರ-ಕ-ದಿಂದ
ನರ-ಕ-ದುಃಖ
ನರ-ಕನ
ನರ-ಕ-ವನ್ನೂ
ನರ-ಕ-ವನ್ನೊ
ನರ-ಕ-ವಾದ
ನರ-ಕ-ವಿದೆ
ನರ-ಕ-ಶಿಕ್ಷೆ
ನರ-ಕ-ಸ-ಮಾ-ನ-ವಾಗಿ
ನರಕಾ
ನರ-ಕಾ-ಸುರ
ನರ-ಕಾ-ಸು-ರನ
ನರ-ಕಾ-ಸು-ರ-ನನ್ನು
ನರ-ಕಾ-ಸು-ರ-ನಿಂದ
ನರ-ಕಾ-ಸು-ರ-ನಿಗೆ
ನರ-ಕಾ-ಸು-ರನು
ನರ-ಕಾ-ಸು-ರನೂ
ನರ-ಗ-ಳಿದ್ದ
ನರ-ಗಳು
ನರ-ನದು
ನರ-ನಾ-ರಾ-ಯ-ಣರು
ನರ-ನಾ-ರಾ-ಯ-ಣಾಯ
ನರ-ನೊ-ಡನೆ
ನರ-ಪಶು
ನರ-ಬಲಿ
ನರ-ಬಾ-ಧೆ-ಯಾ-ಗ-ದಂ-ತೆಯೂ
ನರ-ಮಾಂ-ಸ-ವನ್ನು
ನರ-ಮಾಂ-ಸ-ವೆಂದು
ನರಳು
ನರ-ಳುತ್ತ
ನರ-ಳು-ತ್ತಿದ್ದ
ನರ-ಳು-ತ್ತಿ-ದ್ದೇವೆ
ನರ-ಳು-ತ್ತಿ-ರುವ
ನರ-ಳು-ವು-ದೇಕೆ
ನರಶ್ಚ
ನರ-ಸಿಂಹ
ನರ-ಸಿಂ-ಹನ
ನರ-ಸಿಂ-ಹ-ನನ್ನು
ನರ-ಸಿಂ-ಹನು
ನರ-ಸಿಂ-ಹ-ಮೂ-ರ್ತಿಯು
ನರ-ಸಿಂ-ಹ-ರೂ-ಪ-ದಿಂ-ದಲೂ
ನರ-ಸಿಂ-ಹ-ಸ್ವಾ-ಮಿಯ
ನರ-ಸಿಂ-ಹ-ಸ್ವಾ-ಮಿ-ಯನ್ನು
ನರ-ಸಿಂ-ಹಾ-ವ-ತಾ-ರ-ದಿಂದ
ನರ-ಹ-ರಿಗೆ
ನರ-ಹ-ರಿಯು
ನರಾ
ನರಾಂ-ತಕ
ನರಿ
ನರಿ-ಗಳ
ನರಿ-ಗ-ಳಿ-ಗಿಂತ
ನರಿ-ಗಳು
ನರಿಯ
ನರೆತ
ನರೋ
ನರ್ತಕಿ
ನರ್ತನ
ನರ್ತ-ನ-ಗಳನ್ನು
ನರ್ತ-ನ-ವನ್ನು
ನರ್ತ-ನ-ಶಾ-ಲೆ-ಯಂತೆ
ನರ್ತಿಸಿ
ನರ್ತಿ-ಸಿ-ದಂ-ತಾ-ಯಿತು
ನರ್ತಿ-ಸಿ-ದರು
ನರ್ತಿ-ಸು-ತ್ತವೆ
ನರ್ತಿ-ಸುತ್ತಾ
ನರ್ತಿ-ಸು-ತ್ತಾನೆ
ನರ್ತಿ-ಸು-ತ್ತಿತ್ತು
ನರ್ತಿ-ಸು-ತ್ತಿ-ರುವ
ನರ್ಮದಾ
ನರ್ಮ-ದೆ-ಯಲ್ಲಿ
ನಲ-ವ-ತ್ತೆಂಟು
ನಲ-ವ-ತ್ತೈದು
ನಲ-ವ-ತ್ತೊಂ
ನಲ-ವ-ತ್ತೊಂ-ಬ-ತ್ತ-ನೆಯ
ನಲ-ವ-ತ್ತೊಂ-ಬತ್ತು
ನಲಿ-ದರು
ನಲಿ-ದಾ-ಡಿತು
ನಲಿ-ದಾ-ಡು-ತ್ತಿ-ದ್ದರು
ನಲಿದು
ನಲಿ-ಯಿತು
ನಲಿ-ಯುತ್ತಾ
ನಲಿ-ಯು-ತ್ತಾನೆ
ನಲಿ-ಯು-ತ್ತಿದ್ದ
ನಲಿ-ಯು-ತ್ತಿ-ದ್ದನು
ನಲಿ-ಯು-ತ್ತಿ-ದ್ದರು
ನಲಿ-ಯು-ತ್ತಿ-ರುವ
ನಲಿ-ವರು
ನಲಿ-ವು-ಗಳ
ನಲಿ-ವು-ಗ-ಳೊಂದೂ
ನಲ್ಲ
ನಲ್ಲದೆ
ನಲ್ಲ-ವೆಂ-ಬುದು
ನಲ್ಲಾ
ನಲ್ಲಿ
ನಳ
ನಳ-ಕೂ-ಬರ
ನಳಿ-ನ-ನ-ಯನ
ನವ
ನವಂ
ನವ-ದ-ಸ್ಯು-ದಾರ
ನವ-ನಂ-ದ-ರೆಂಬ
ನವ-ನಿ-ಧಿ-ಗಳಿಂದ
ನವ-ಬ್ರ-ಹ್ಮರೂ
ನವ-ಮಾಸ
ನವ-ರ-ತ್ನದ
ನವಿ-ಲ-ಕಣ್ಣು
ನವಿ-ಲಾ-ಗ-ಬ-ಲ್ಲದೆ
ನವಿ-ಲಿ-ನ-ಲ್ಲಿ-ರು-ವಂತೆ
ನವಿಲು
ನವಿ-ಲು-ಗ-ರಿ-ಎ-ಲ್ಲವೂ
ನವಿ-ಲು-ಗ-ರಿ-ಗ-ಳಿಂ-ದಲೂ
ನವಿ-ಲು-ಗಳನ್ನು
ನವಿ-ಲು-ಗಳು
ನವಿ-ಲು-ಗ-ಳೆಲ್ಲ
ನವೆ-ಯನ್ನು
ನಶ್ಚ-ಲ-ಸೌ-ಹೃದಃ
ನಶ್ವ-ರ-ವಾದ
ನಶ್ವ-ರ-ವೆಂದು
ನಶ್ವ-ರ-ವೆಂಬ
ನಶ್ವ-ರ-ವೆ-ನಿ-ಸು-ತ್ತದೆ
ನಷ್ಟ
ನಸು
ನಸು-ಗೆಂ-ಪಾದ
ನಸು-ದೋ-ರುವ
ನಸು-ನಕ್ಕು
ನಸು-ನ-ಗುತ್ತಾ
ನಸು-ನ-ಗು-ತ್ತಾಹೇ
ನಸ್ಸ್ಮರಂ
ನಹುಷ
ನಹು-ಷ-ನಿಗೆ
ನಹು-ಷನು
ನಾ-ನಾ-ರದ
ನಾಂ
ನಾಂಗ-ನನ್ನು
ನಾಗ-ಕ-ನ್ನಿ-ಕೆ-ಯರೂ
ನಾಗ-ಗ-ಳೆಂದೂ
ನಾಗ-ಪಾ-ಶ-ದಿಂದ
ನಾಗ-ಬೇ-ಕೆಂದು
ನಾಗ-ರ-ದಂತೆ
ನಾಗ-ರ-ಹಾ-ವಿ-ನಂತೆ
ನಾಗ-ರಾ-ಜನ
ನಾಗ-ರಿ-ಕರು
ನಾಗ-ರಿ-ಕ-ರೆ-ಲ್ಲರೂ
ನಾಗರೂ
ನಾಗಲು
ನಾಗ-ಲೆಂದು
ನಾಗಾ-ಲೋ-ಟ-ದಿಂದ
ನಾಗಿ
ನಾಗಿದ್ದ
ನಾಗಿ-ದ್ದನು
ನಾಗಿ-ದ್ದರೆ
ನಾಗಿದ್ದು
ನಾಗಿ-ದ್ದೇನೆ
ನಾಗಿಯೂ
ನಾಗಿಯೇ
ನಾಗಿ-ರು-ವನು
ನಾಗಿ-ರು-ವು-ದ-ಕ್ಕಿಂ-ತಲೂ
ನಾಗಿ-ರುವೆ
ನಾಗು
ನಾಗು-ವಂತೆ
ನಾಗುವೆ
ನಾಗ್ನ-ಜಿ-ತಿ-ಭಾನು
ನಾಚಿ
ನಾಚಿಕೆ
ನಾಚಿ-ಕೆ-ಗಳಿಂದ
ನಾಚಿ-ಕೆ-ಗೇಡು
ನಾಚಿ-ಕೆ-ಗೊ-ಳ್ಳ-ಲಿಲ್ಲ
ನಾಚಿ-ಕೆ-ಯನ್ನು
ನಾಚಿ-ಕೆ-ಯಾ-ಗ-ಲಿಲ್ಲ
ನಾಚಿ-ಕೆ-ಯಾ-ಗು-ವು-ದಿ-ಲ್ಲವೆ
ನಾಚಿ-ಕೆ-ಯಾ-ಯಿತು
ನಾಚಿ-ಕೆ-ಯಿಂದ
ನಾಚಿ-ಕೆ-ಯಿ-ಲ್ಲದೆ
ನಾಚಿ-ಕೆಯೂ
ನಾಚಿ-ಕೆಯೇ
ನಾಚಿಗೆ
ನಾಚಿ-ದನು
ನಾಚಿ-ದಳು
ನಾಚಿ-ದು-ದನ್ನು
ನಾಚಿ-ಸು-ವಷ್ಟು
ನಾಟಕ
ನಾಟ-ಕ-ದಲ್ಲಿ
ನಾಟಿ
ನಾಟಿ-ಕೊಂ-ಡಿವೆ
ನಾಟಿತ್ತು
ನಾಟಿದ
ನಾಟಿ-ದವು
ನಾಟಿದೆ
ನಾಟಿ-ಹೋ-ಗಿವೆ
ನಾಡಿ-ಗಳು
ನಾಡು-ಗಳು
ನಾಥ
ನಾಥಾ
ನಾದ
ನಾದಂ-ತಾ-ಯಿತು
ನಾದದ
ನಾದ-ದೊ-ಡನೆ
ನಾದನು
ನಾದರೂ
ನಾದರೆ
ನಾದರೊ
ನಾದ-ವನು
ನಾದ-ವನೇ
ನಾದಿನಿ
ನಾದಿ-ನಿ-ಯರ
ನಾದಿ-ನಿ-ಯ-ರಿಗೆ
ನಾದಿ-ನಿ-ಯರು
ನಾದೆ-ನೆಂ-ದು-ಕೊಂ-ಡನು
ನಾನಂದು
ನಾನ-ದನ್ನು
ನಾನ-ರಿಯೆ
ನಾನಲ್ಲ
ನಾನ-ಲ್ಲದ
ನಾನ-ಲ್ಲದೆ
ನಾನ-ಲ್ಲಿಗೆ
ನಾನಾ
ನಾನಾಗ
ನಾನಾ-ಗಲಿ
ನಾನಾ-ಗಲೆ
ನಾನಾಗಿ
ನಾನಾರು
ನಾನಾ-ರೂಪ
ನಾನಾ-ರೆಂದು
ನಾನಾವ
ನಾನಿಂದು
ನಾನಿ-ದ್ದ-ಲ್ಲಿಗೆ
ನಾನಿ-ದ್ದ-ಲ್ಲಿಯೆ
ನಾನಿನ್ನು
ನಾನಿ-ಲ್ಲ-ದಾಗ
ನಾನಿ-ವ-ನಿಗೆ
ನಾನೀಗ
ನಾನು
ನಾನು-ಎಂಬ
ನಾನು-ಬೇ-ರೆ-ಯ-ವರು
ನಾನೂ
ನಾನೂರ
ನಾನೂರು
ನಾನೆ
ನಾನೆಂ-ತಹ
ನಾನೆಂ-ದರೆ
ನಾನೆಂದು
ನಾನೆಂದೆ
ನಾನೆಂಬ
ನಾನೆಲ್ಲಿ
ನಾನೆ-ಲ್ಲಿ-ದ್ದರೆ
ನಾನೆ-ಲ್ಲಿಯೂ
ನಾನೆ-ಷ್ಟ-ರ-ವನು
ನಾನೆಷ್ಟು
ನಾನೇ
ನಾನೇ-ನಯ್ಯ
ನಾನೇನು
ನಾನೇನೂ
ನಾನೊಂದು
ನಾನೊಬ್ಬ
ನಾನೊಮ್ಮೆ
ನಾನೋ-ರ್ವನೆ
ನಾಪಿ-ವಾಚಾ
ನಾಭಾಗ
ನಾಭಾ-ಗನ
ನಾಭಾ-ಗ-ನನ್ನು
ನಾಭಾ-ಗನು
ನಾಭಿ
ನಾಭಿ-ಮೇ-ರು-ದೇ-ವಿ-ಯರ
ನಾಭಿ-ಯಲ್ಲಿ
ನಾಭಿ-ಯ-ಲ್ಲಿ-ರುವ
ನಾಭಿ-ರಾಜ
ನಾಭಿ-ರಾ-ಜನ
ನಾಭಿ-ರಾ-ಜ-ನಿಗೆ
ನಾಭಿ-ರಾ-ಜನು
ನಾಭೀ-ಕ-ಮ-ಲ-ದಲ್ಲಿ
ನಾಭೀ-ಕ-ಮ-ಲ-ದಿಂದ
ನಾಮ
ನಾಮ-ಇ-ವು-ಗಳ
ನಾಮ-ಕ-ರಣ
ನಾಮ-ಕ-ರ-ಣ-ಒಂದೂ
ನಾಮ-ಕ-ರ-ಣ-ವಾ-ಯಿತು
ನಾಮ-ಗಳನ್ನು
ನಾಮ-ಭೇದ
ನಾಮ-ರೂ-ಪ-ರ-ಹಿ-ತ-ನಾಗಿ
ನಾಮ-ವನ್ನು
ನಾಮ-ಸಂ-ಕೀ-ರ್ತ-ನೆ-ಯನ್ನು
ನಾಮ-ಸ್ಮ-ರಣೆ
ನಾಮ-ಸ್ಮ-ರ-ಣೆ-ಗಿಂತ
ನಾಮ-ಸ್ಮ-ರ-ಣೆ-ಯಿಂದ
ನಾಮ-ಸ್ಮ-ರ-ಣೆಯೆ
ನಾಮಾ-ಮೃ-ತ-ವನ್ನು
ನಾಮಾ-ವ-ಳಿ-ಗಳನ್ನು
ನಾಮಾ-ವ-ಶೇ-ಷ-ವಾಗಿ
ನಾಮಾ-ವ-ಶೇ-ಷ-ವಾ-ದವು
ನಾಯಕ
ನಾಯ-ಕನು
ನಾಯ-ಕ-ರ-ತ್ನ-ದಂ-ತಿದೆ
ನಾಯ-ಕ-ರನ್ನು
ನಾಯ-ಕ-ರಾ-ದರು
ನಾಯಾ-ನಿ-ಮಿ-ಷಾಂ
ನಾಯಿ
ನಾಯಿ-ಗಳ
ನಾಯಿ-ಗಳನ್ನು
ನಾಯಿ-ಗಳನ್ನೂ
ನಾಯಿ-ಗ-ಳಿಗೂ
ನಾಯಿ-ಗ-ಳಿಗೆ
ನಾಯಿ-ಗ-ಳೊ-ಡನೆ
ನಾಯಿ-ನ-ರಿ-ಗಳ
ನಾಯಿ-ಮು-ಟ್ಟಿದ
ನಾಯಿಯ
ನಾಯಿ-ಯಂತೆ
ನಾಯಿಯು
ನಾರದ
ನಾರ-ದನ
ನಾರ-ದ-ನಿಗೆ
ನಾರ-ದನು
ನಾರ-ದನೂ
ನಾರ-ದ-ಪು-ರಾಣ
ನಾರ-ದ-ಮ-ಹರ್ಷಿ
ನಾರ-ದ-ಮ-ಹ-ರ್ಷಿ-ಯಿಂದ
ನಾರ-ದ-ಮು-ನಿಯ
ನಾರ-ದರ
ನಾರ-ದ-ರನ್ನು
ನಾರ-ದ-ರಿಂದ
ನಾರ-ದ-ರಿಗೆ
ನಾರ-ದರು
ನಾರ-ದರೆ
ನಾರ-ದ-ರೊ-ಡನೆ
ನಾರ-ದಾದಿ
ನಾರ-ಸಿಂಹ
ನಾರ-ಸಿಂ-ಹಾಯ
ನಾರಾ
ನಾರಾ-ಯಣ
ನಾರಾ-ಯಣಃ
ನಾರಾ-ಯ-ಣ-ಕ-ವಚ
ನಾರಾ-ಯ-ಣ-ಕ-ವ-ಚದ
ನಾರಾ-ಯ-ಣ-ಕ-ವ-ಚ-ವೆಂಬ
ನಾರಾ-ಯ-ಣನ
ನಾರಾ-ಯ-ಣ-ನನ್ನು
ನಾರಾ-ಯ-ಣ-ನಾ-ಮ-ಸ್ಮ-ರಣೆ
ನಾರಾ-ಯ-ಣನು
ನಾರಾ-ಯ-ಣನೆ
ನಾರಾ-ಯ-ಣ-ನೆಂದೆ
ನಾರಾ-ಯ-ಣನೇ
ನಾರಾ-ಯ-ಣ-ನೊ-ಡನೆ
ನಾರಾ-ಯ-ಣ-ಮು-ನಿಗೆ
ನಾರಾ-ಯ-ಣರು
ನಾರಾ-ಯ-ಣ-ರೂ-ಪಿ-ನಿಂದ
ನಾರಾ-ಯ-ಣ-ರೆಂಬ
ನಾರಾ-ಯ-ಣ-ಸ್ಮ-ರಣೆ
ನಾರಾ-ಯಣಾ
ನಾರಾ-ಯ-ಣಾ-ಯ-ಅ-ಷ್ಟಾ-ಕ್ಷರೀ
ನಾರಾ-ಯ-ಣಾ-ವ-ತಾರ
ನಾರಾ-ಯ-ಣಾಸ್ತ್ರ
ನಾರಾ-ಯ-ಣಾ-ಸ್ತ್ರ-ವನ್ನು
ನಾರಾ-ಯಣಿ
ನಾರೀ-ಕ-ವಚ
ನಾರು
ನಾರು-ಬಟ್ಟೆ
ನಾರು-ಮ-ಡಿ-ಇ-ವು-ಗಳಿಂದ
ನಾರು-ಮ-ಡಿ-ಯ-ನ್ನುಟ್ಟು
ನಾರು-ಮ-ಡಿ-ಯುಟ್ಟು
ನಾಲಗೆ
ನಾಲ-ಗೆಗೆ
ನಾಲ-ಗೆಯ
ನಾಲ-ಗೆ-ಯಂ-ತಿ-ರುವ
ನಾಲ-ಗೆ-ಯನ್ನು
ನಾಲ-ಗೆ-ಯಿಂದ
ನಾಲಿ-ಗೆಗೆ
ನಾಲಿ-ಗೆ-ಯನ್ನು
ನಾಲ್ಕ-ನೆಯ
ನಾಲ್ಕ-ನೆ-ಯ-ವನು
ನಾಲ್ಕ-ರಲ್ಲಿ
ನಾಲ್ಕ-ರ-ಲ್ಲಿಯೂ
ನಾಲ್ಕಾರು
ನಾಲ್ಕು
ನಾಲ್ಕು-ವ-ರ್ಣ-ದವ
ನಾಲ್ಕು-ವ-ರ್ಣ-ದ-ವ-ರಿ-ದ್ದಾರೆ
ನಾಲ್ಕು-ವ-ರ್ಣ-ದ-ವರೂ
ನಾಲ್ಕೈದು
ನಾಲ್ಮೊ-ಗದ
ನಾಲ್ವರು
ನಾಳೆ
ನಾಳೆ-ಎಂದು
ನಾಳೆಗೆ
ನಾಳೆಯ
ನಾವಂತೂ
ನಾವ-ದನ್ನು
ನಾವನೂ
ನಾವ-ರಿ-ಯೆವು
ನಾವ-ಲ್ಲಿಗೆ
ನಾವಾ-ಗಲಿ
ನಾವಾ-ದರೂ
ನಾವಾರೂ
ನಾವಿ
ನಾವಿಂದು
ನಾವಿ-ಕನ
ನಾವಿ-ಕ-ನಾಗಿ
ನಾವಿನ್ನೂ
ನಾವಿ-ಬ್ಬರೂ
ನಾವಿ-ರುವ
ನಾವಿಲ್ಲಿ
ನಾವಿ-ಲ್ಲಿಗೆ
ನಾವೀಗ
ನಾವು
ನಾವೂ
ನಾವೆ
ನಾವೆ-ಯಂತೆ
ನಾವೆ-ಯಲ್ಲಿ
ನಾವೆಲ್ಲ
ನಾವೆ-ಲ್ಲರೂ
ನಾವೇ
ನಾವೇಕೆ
ನಾವೇನು
ನಾಶ
ನಾಶ-ಕ್ಕಾಗಿ
ನಾಶಕ್ಕೆ
ನಾಶ-ಮಾ-ಡ-ಬೇ-ಕಾ-ಗಿದೆ
ನಾಶ-ಮಾ-ಡಲು
ನಾಶ-ಮಾಡಿ
ನಾಶ-ಮಾ-ಡಿದ
ನಾಶ-ಮಾ-ಡಿ-ದನು
ನಾಶ-ಮಾ-ಡಿ-ದರೆ
ನಾಶ-ಮಾ-ಡು-ತ್ತದೆ
ನಾಶ-ಮಾ-ಡು-ತ್ತಾನೆ
ನಾಶ-ಮಾ-ಡು-ತ್ತಾರೆ
ನಾಶ-ಮಾ-ಡು-ತ್ತಿ-ರುವ
ನಾಶ-ಮಾ-ಡು-ವಂ-ತಾ-ಗ-ಬೇ-ಕಾ-ಗಿದೆ
ನಾಶ-ಮಾ-ಡು-ವನು
ನಾಶ-ಮು-ಪ-ದ್ರ-ವಾಃ
ನಾಶ-ರ-ಹಿ-ತನೂ
ನಾಶ-ವನ್ನೂ
ನಾಶ-ವಾಗ
ನಾಶ-ವಾ-ಗದೆ
ನಾಶ-ವಾ-ಗ-ಬೇಕು
ನಾಶ-ವಾ-ಗಲಿ
ನಾಶ-ವಾಗಿ
ನಾಶ-ವಾ-ಗಿ-ಹೋ-ಗಲು
ನಾಶ-ವಾ-ಗಿ-ಹೋಗು
ನಾಶ-ವಾ-ಗು-ತ್ತದೆ
ನಾಶ-ವಾ-ಗು-ತ್ತವೆ
ನಾಶ-ವಾ-ಗು-ತ್ತಿ-ದ್ದವು
ನಾಶ-ವಾ-ಗು-ವಂತೆ
ನಾಶ-ವಾ-ಗು-ವುದನ್ನು
ನಾಶ-ವಾ-ಗು-ವು-ದಿಲ್ಲ
ನಾಶ-ವಾ-ಗು-ವುದು
ನಾಶ-ವಾ-ಗು-ವುದೂ
ನಾಶ-ವಾ-ದಂ-ತಾ-ಯಿತು
ನಾಶ-ವಾ-ದಾ-ಗಲೂ
ನಾಶ-ವಾ-ದು-ದನ್ನು
ನಾಶ-ವೆ-ಲ್ಲಿ-ಯದು
ನಾಸಾ-ರಂ-ಧ್ರ-ದಿಂದ
ನಾಸಿ-ಕ-ಪುಟ
ನಾಸ್ತಿ-ಕತೆ
ನಾಸ್ತಿ-ಕನ
ನಿಂಗೆ
ನಿಂತ
ನಿಂತಂ-ತಾ-ಗು-ತ್ತದೆ
ನಿಂತಂ-ತಾ-ಯಿತು
ನಿಂತ-ಕಡೆ
ನಿಂತನು
ನಿಂತ-ನೆಂ-ದರೆ
ನಿಂತ-ಮೇಲೆ
ನಿಂತರು
ನಿಂತರೂ
ನಿಂತರೆ
ನಿಂತಲ್ಲಿ
ನಿಂತ-ಲ್ಲಿಂ-ದಲೆ
ನಿಂತ-ಲ್ಲಿಯೇ
ನಿಂತಳು
ನಿಂತ-ವನೆ
ನಿಂತವು
ನಿಂತಾಗ
ನಿಂತಿ
ನಿಂತಿತು
ನಿಂತಿತ್ತು
ನಿಂತಿದೆ
ನಿಂತಿದ್ದ
ನಿಂತಿ-ದ್ದನು
ನಿಂತಿ-ದ್ದರು
ನಿಂತಿ-ದ್ದರೆ
ನಿಂತಿ-ದ್ದಳು
ನಿಂತಿ-ದ್ದವು
ನಿಂತಿ-ದ್ದಾ-ನಲ್ಲ
ನಿಂತಿ-ದ್ದಾರೆ
ನಿಂತಿ-ದ್ದಾಳೆ
ನಿಂತಿದ್ದು
ನಿಂತಿ-ದ್ದೆನು
ನಿಂತಿ-ದ್ದೆವು
ನಿಂತಿ-ದ್ದೇನೆ
ನಿಂತಿ-ರ-ಬೇಕು
ನಿಂತಿ-ರಲು
ನಿಂತಿ-ರು-ತ್ತ-ವಂತೆ
ನಿಂತಿ-ರು-ತ್ತಿ-ದ್ದ-ನಾ-ದರೂ
ನಿಂತಿ-ರುವ
ನಿಂತಿ-ರು-ವಂ-ತೆಯೊ
ನಿಂತಿ-ರು-ವು-ದಾಗಿ
ನಿಂತಿ-ರುವೆ
ನಿಂತಿವೆ
ನಿಂತೀತು
ನಿಂತು
ನಿಂತು-ಕೊಂ-ಡನು
ನಿಂತು-ಕೊಂ-ಡರು
ನಿಂತು-ಕೊಂಡು
ನಿಂತು-ಕೊ-ಳ್ಳು-ತ್ತೇನೆ
ನಿಂತು-ಬಿಟ್ಟ
ನಿಂತು-ಬಿ-ಟ್ಟಳು
ನಿಂತು-ಹೋ-ಗಿದೆ
ನಿಂತು-ಹೋ-ಯಿತು
ನಿಂತೆ
ನಿಂತೇ
ನಿಂದ
ನಿಂದಲೆ
ನಿಂದಿ-ಸ-ಬೇ-ಕಾ-ಗಿಲ್ಲ
ನಿಂದಿಸಿ
ನಿಂದಿ-ಸಿ-ಕೊಂ-ಡಳು
ನಿಂದಿ-ಸಿದ
ನಿಂದಿ-ಸಿ-ದನು
ನಿಂದಿ-ಸಿ-ದರೆ
ನಿಂದಿ-ಸಿ-ದ-ರೆಂದು
ನಿಂದಿ-ಸಿ-ದ-ವನ
ನಿಂದಿ-ಸಿ-ದು-ದನ್ನು
ನಿಂದಿಸು
ನಿಂದಿ-ಸು-ತ್ತಾರೆ
ನಿಂದಿ-ಸು-ತ್ತಿರು
ನಿಂದಿ-ಸು-ತ್ತಿ-ರುವ
ನಿಂದಿ-ಸು-ತ್ತಿ-ರು-ವ-ನಲ್ಲ
ನಿಂದಿ-ಸುವ
ನಿಂದಿ-ಸು-ವ-ರೆಂಬ
ನಿಂದಿ-ಸು-ವು-ದ-ರಲ್ಲಿ
ನಿಂದೆ
ನಿಂದೆಗೂ
ನಿಕ-ರ್ಷ-ಣಾಯ
ನಿಕುಂಭ
ನಿಖಿ-ಲಾ-ಧ್ವ-ರ-ಭಾ-ಗ-ಭುಜಂ
ನಿಗಮ
ನಿಗ-ಮ-ಕೃ-ದು-ಪ-ಜಹ್ರೇ
ನಿಗಿ
ನಿಗಿಂತ
ನಿಗಿ-ನಿಗಿ
ನಿಗಿಲ್ಲ
ನಿಗೂ
ನಿಗೆ
ನಿಗೊ-ಪ್ಪಿಸಿ
ನಿಗ್ರ-ಹಿ-ಸ-ಬೇ-ಕಲ್ಲಾ
ನಿಗ್ರ-ಹಿಸಿ
ನಿಗ್ರ-ಹಿ-ಸಿದ
ನಿಗ್ರ-ಹಿ-ಸಿ-ದ-ವ-ನಾ-ಗಿಯೂ
ನಿಗ್ರ-ಹಿ-ಸು-ವೆಯೊ
ನಿಜ
ನಿಜ-ತ-ತ್ವ-ವೆ-ಲ್ಲವೂ
ನಿಜ-ದೀ-ಧಿ-ತಿ-ಭಿಃ
ನಿಜ-ವಾ-ಗ-ಬಾ-ರದು
ನಿಜ-ವಾಗಿ
ನಿಜ-ವಾ-ಗಿ-ದ್ದಲ್ಲಿ
ನಿಜ-ವಾ-ಗಿಯೂ
ನಿಜ-ವಾದ
ನಿಜ-ವಾ-ದಂ-ತಾ-ಯಿತು
ನಿಜ-ವಾ-ದರೂ
ನಿಜ-ವಾ-ದರೆ
ನಿಜ-ವಿ-ರ-ಬಹು
ನಿಜ-ವೆಂದು
ನಿಜ-ವೆಂದೇ
ನಿಜ-ಸ್ವ-ರೂ-ಪ-ದಲ್ಲಿ
ನಿಜ-ಸ್ವ-ರೂ-ಪ-ವನ್ನು
ನಿಜ-ಸ್ವ-ರೂ-ಪ-ವೇ-ನೆಂ-ಬು-ದನ್ನು
ನಿಜ-ಹೇಳು
ನಿಟ್ಟ-ಸಿರು
ನಿಟ್ಟ-ಸಿ-ರು-ಬಿ-ಡು-ವುದನ್ನು
ನಿಟ್ಟು-ಬಿದ್ದ
ನಿಟ್ಟು-ಬಿ-ದ್ದರು
ನಿಟ್ಟು-ಬಿದ್ದು
ನಿಟ್ಟು-ಸಿ-ರನ್ನು
ನಿಟ್ಟು-ಸಿ-ರಿ-ಟ್ಟಳು
ನಿಟ್ಟು-ಸಿರು
ನಿಟ್ಟು-ಸಿ-ರು-ಬಿ-ಟ್ಟಳು
ನಿಟ್ಟೋ-ಟ-ದಿಂದ
ನಿಡಿ-ದಾದ
ನಿತ್ಯ
ನಿತ್ಯ-ಕ-ರ್ಮಕ್ಕೆ
ನಿತ್ಯ-ಕ-ರ್ಮ-ಗಳನ್ನು
ನಿತ್ಯ-ತೃ-ಪ್ತ-ನಾದ
ನಿತ್ಯತ್ವ
ನಿತ್ಯದ
ನಿತ್ಯ-ನಾ-ದು-ದ-ರಿಂದ
ನಿತ್ಯ-ನಿಧಿ
ನಿತ್ಯ-ನಿ-ಯ-ಮ-ಗಳನ್ನು
ನಿತ್ಯ-ಪ-ದ-ವಿ-ಯನ್ನು
ನಿತ್ಯ-ಮು-ಕ್ತ-ರಾದ
ನಿತ್ಯ-ವಲ್ಲ
ನಿತ್ಯ-ವ-ಲ್ಲ-ದಿ-ರ-ಬ-ಹುದು
ನಿತ್ಯ-ವಾ-ದುದು
ನಿತ್ಯವೂ
ನಿತ್ಯ-ವೆಂದು
ನಿತ್ಯ-ವೆಂದೂ
ನಿತ್ಯ-ಶು-ದ್ಧ-ನಾದ
ನಿತ್ಯ-ಶು-ದ್ಧ-ನೆಂ-ಬುದು
ನಿತ್ಯ-ಸಾ-ನಿ-ಧ್ಯವು
ನಿತ್ಯ-ಸು-ಖ-ಗಳಿಂದ
ನಿತ್ಯಾ-ನಂದ
ನಿತ್ಯಾ-ನಂ-ದ-ಮೂರ್ತಿ
ನಿತ್ಯಾ-ನಂ-ದ-ಸ್ವ-ರೂ-ಪ-ನಾದ
ನಿತ್ಯಾ-ನಿತ್ಯ
ನಿದ-ರ್ಶನ
ನಿದ-ರ್ಶ-ನಕ್ಕೆ
ನಿದ್ದೀಶ
ನಿದ್ದೆ
ನಿದ್ದೆ-ಮಾ-ಡು-ತ್ತಿ-ರುವ
ನಿದ್ದೆಯ
ನಿದ್ದೆ-ಯಿಂದ
ನಿದ್ದೆ-ಹೋದೆ
ನಿದ್ದೆ-ಹೋ-ಯಿತು
ನಿದ್ರಾ
ನಿದ್ರಾ-ಭಂಗ
ನಿದ್ರಾ-ಭ-ಯ-ಮೈ-ಥು-ನ-ಗಳನ್ನು
ನಿದ್ರಾ-ಲ-ಸ್ಯ-ಗಳನ್ನು
ನಿದ್ರಾ-ಹಾ-ರ-ಗ-ಳಿ-ಲ್ಲದೆ
ನಿದ್ರಿಸು
ನಿದ್ರಿ-ಸು-ತ್ತಿದ್ದ
ನಿದ್ರಿ-ಸು-ತ್ತಿ-ದ್ದಾನೆ
ನಿದ್ರಿ-ಸು-ತ್ತಿ-ರು-ವಂತೆ
ನಿದ್ರೆ
ನಿದ್ರೆಈ
ನಿದ್ರೆ-ಎ-ಲ್ಲ-ದ-ರ-ಲ್ಲಿಯೂ
ನಿದ್ರೆ-ಗ-ಳ-ನ್ನಾ-ಗಲಿ
ನಿದ್ರೆ-ಗ-ಳ-ಲ್ಲಿಯೂ
ನಿದ್ರೆ-ಯಿಂದ
ನಿದ್ರೆ-ಯಿ-ಲ್ಲದೆ
ನಿದ್ರೆಯೆ
ನಿದ್ರೆ-ಹೋ-ಗಿದ್ದ
ನಿದ್ರೆ-ಹೋ-ಗುವ
ನಿದ್ರೆ-ಹೋ-ದ-ವ-ರನ್ನು
ನಿಧಾ-ನ-ವಾಗಿ
ನಿಧಿ
ನಿಧಿ-ಯನ್ನು
ನಿಧಿ-ಯಾ-ಗಿ-ರುವ
ನಿಧಿಯೂ
ನಿನ-ಗಾಗ
ನಿನ-ಗಾಗಿ
ನಿನ-ಗಾ-ಗಿಯೂ
ನಿನ-ಗಾರು
ನಿನ-ಗಾವ
ನಿನ-ಗಾ-ವು-ದರ
ನಿನ-ಗಿಂತ
ನಿನ-ಗಿಂ-ತಲೂ
ನಿನ-ಗಿದೆ
ನಿನ-ಗಿದೊ
ನಿನ-ಗಿನ್ನೂ
ನಿನ-ಗಿ-ರುವ
ನಿನ-ಗೀಗ
ನಿನಗೂ
ನಿನಗೆ
ನಿನ-ಗೆಲ್ಲಿ
ನಿನ-ಗೆ-ಲ್ಲಿಂದ
ನಿನ-ಗೆಷ್ಟು
ನಿನಗೇ
ನಿನ-ಗೇಕೆ
ನಿನ-ಗೇ-ನ-ನ್ನಿ-ಸು-ತ್ತದೆ
ನಿನ-ಗೇ-ನಾ-ಗ-ಬೇಕು
ನಿನ-ಗೇನು
ನಿನ-ಗೇನೂ
ನಿನ-ಗೊಂದು
ನಿನ-ಗೊ-ಪ್ಪಿಸಿ
ನಿನ-ಗೊಬ್ಬ
ನಿನ-ಗೊ-ಬ್ಬ-ನಿಗೇ
ನಿನ್ನ
ನಿನ್ನಂ-ತಹ
ನಿನ್ನಂ-ತ-ಹ-ವನು
ನಿನ್ನಂ-ತೆಯೇ
ನಿನ್ನಂಥ
ನಿನ್ನದ
ನಿನ್ನ-ದನ್ನು
ನಿನ್ನದು
ನಿನ್ನದೆ
ನಿನ್ನ-ದೇನು
ನಿನ್ನ-ದೊಂದು
ನಿನ್ನನ್ನು
ನಿನ್ನನ್ನೂ
ನಿನ್ನನ್ನೆ
ನಿನ್ನ-ನ್ನೆಲ್ಲಿ
ನಿನ್ನನ್ನೇ
ನಿನ್ನ-ನ್ನೇನು
ನಿನ್ನ-ಪ್ರಾಣ
ನಿನ್ನ-ರೂಪ
ನಿನ್ನಲ್ಲಿ
ನಿನ್ನ-ಲ್ಲಿಗೆ
ನಿನ್ನ-ಲ್ಲಿದ್ದ
ನಿನ್ನ-ಲ್ಲಿಯೆ
ನಿನ್ನ-ಲ್ಲಿಯೇ
ನಿನ್ನ-ವ-ರಿಂದ
ನಿನ್ನ-ವರು
ನಿನ್ನ-ವರೆ-ನ್ನು-ವ-ರಾರೂ
ನಿನ್ನ-ವರೆ-ನ್ನು-ವ-ವ-ರ-ನ್ನೆಲ್ಲ
ನಿನ್ನ-ವಳೆ
ನಿನ್ನಿಂದ
ನಿನ್ನಿಂ-ದಲೂ
ನಿನ್ನಿಂ-ದಲೆ
ನಿನ್ನೂ-ಡನೆ
ನಿನ್ನೊ-ಡನೆ
ನಿನ್ನೊ-ಬ್ಬನ
ನಿಪು-ಣ-ನಂತೆ
ನಿಪು-ಣಳು
ನಿಬಂ-ಧ-ನೆ-ಯುಂಟು
ನಿಮ
ನಿಮ-ಗ-ದನ್ನು
ನಿಮ-ಗಾ-ಗಲಿ
ನಿಮ-ಗಾಗಿ
ನಿಮ-ಗಾ-ಗು-ತ್ತಿ-ರುವ
ನಿಮ-ಗಿಂ-ತಲೂ
ನಿಮ-ಗಿ-ರು-ವಂ-ತಹ
ನಿಮಗೀ
ನಿಮಗೂ
ನಿಮಗೆ
ನಿಮ-ಗೆಲ್ಲ
ನಿಮ-ಗೆ-ಲ್ಲ-ರಿಗೂ
ನಿಮ-ಗೇನು
ನಿಮ-ಗೊಮ್ಮೆ
ನಿಮಿ
ನಿಮಿಗೂ
ನಿಮಿತ್ತ
ನಿಮಿ-ತ್ತ-ಕಾ-ರಣ
ನಿಮಿ-ತ್ತ-ಗಳನ್ನು
ನಿಮಿ-ತ್ತ-ಗ-ಳು-ನೆ-ರ-ಳಿ-ನಲ್ಲಿ
ನಿಮಿ-ತ್ತ-ದಿಂದ
ನಿಮಿ-ತ್ತ-ವಾ-ದುದು
ನಿಮಿಯ
ನಿಮಿಯು
ನಿಮಿರಿ
ನಿಮಿಷ
ನಿಮಿ-ಷಕ್ಕೆ
ನಿಮಿ-ಷ-ದಲ್ಲಿ
ನಿಮಿ-ಷ-ದ-ಲ್ಲಿಯೇ
ನಿಮಿ-ಷ-ವನ್ನೇ
ನಿಮೀ-ಲ-ಯಸಿ
ನಿಮ್ಮ
ನಿಮ್ಮಂ
ನಿಮ್ಮಂ-ತಹ
ನಿಮ್ಮ-ಜ್ಜ-ನಾದ
ನಿಮ್ಮ-ದಾ-ಗ-ಬೇಕು
ನಿಮ್ಮ-ದಾ-ಗಿದೆ
ನಿಮ್ಮದು
ನಿಮ್ಮ-ನ್ನಾ-ದರೊ
ನಿಮ್ಮನ್ನು
ನಿಮ್ಮ-ಪ್ಪನ
ನಿಮ್ಮ-ಪ್ಪ-ನಾ-ಗಲೀ
ನಿಮ್ಮಲ್ಲಿ
ನಿಮ್ಮ-ಲ್ಲಿನ್ನೂ
ನಿಮ್ಮ-ಲ್ಲಿಯೇ
ನಿಮ್ಮ-ವರೆ-ಲ್ಲರ
ನಿಮ್ಮಿಂದ
ನಿಮ್ಮಿ-ಬ್ಬ-ರನ್ನೂ
ನಿಮ್ಮಿ-ಬ್ಬ-ರಲ್ಲಿ
ನಿಮ್ಮೆ-ಲ್ಲ-ರನ್ನೂ
ನಿಮ್ಮೆ-ಲ್ಲ-ರಿಗೂ
ನಿಮ್ಮೊ-ಡನೆ
ನಿಯಂ-ತ್ರಿ-ಸುವ
ನಿಯತ್
ನಿಯಮ
ನಿಯ-ಮ-ಗಳನ್ನು
ನಿಯ-ಮ-ಗಳಿಂದ
ನಿಯ-ಮ-ಗ-ಳಿಂ-ದಲೂ
ನಿಯ-ಮ-ದಂತೆ
ನಿಯ-ಮ-ದ-ಲ್ಲಿಯೂ
ನಿಯ-ಮ-ದಿಂದ
ನಿಯ-ಮ-ನೀ-ತಿಗೆ
ನಿಯ-ಮ-ವನ್ನು
ನಿಯ-ಮವೂ
ನಿಯ-ಮಿ-ಸು-ತ್ತಿರು
ನಿಯ-ಮಿ-ಸು-ತ್ತೀಯೆ
ನಿಯ-ಮಿ-ಸುವ
ನಿಯಾ-ಮಕ
ನಿಯಾ-ಮ-ಕ-ನಾಗಿ
ನಿಯಾ-ಮ-ಕ-ನಾ-ಗಿ-ರುವ
ನಿಯಾ-ಮ-ಕ-ನಾದ
ನಿಯಾ-ಮ-ಕ-ನಾ-ದ-ವನೇ
ನಿಯು-ಕ್ತ-ರಾದ
ನಿಯೋ-ಜಿ-ಸಿ-ದನು
ನಿರಂ-ಕು-ಶ-ಪ್ರ-ಭು-ವಾ-ಗಿದ್ದ
ನಿರಂ-ಜನ
ನಿರಂ-ತ-ರ-ವಾ-ಗಿ-ರಲಿ
ನಿರಂ-ತ-ರ-ವಾದ
ನಿರಂ-ತ-ರವೂ
ನಿರತ
ನಿರ-ತ-ನಾಗಿ
ನಿರ-ತ-ನಾ-ಗಿದ್ದ
ನಿರ-ತ-ನಾ-ಗಿ-ದ್ದನು
ನಿರ-ತ-ನಾ-ಗಿದ್ದು
ನಿರ-ತ-ನಾ-ಗಿ-ರಲು
ನಿರ-ತ-ನಾ-ಗಿ-ರುವ
ನಿರ-ತ-ನಾ-ಗು-ವಂತೆ
ನಿರ-ತ-ನಾದ
ನಿರ-ತ-ನಾ-ದನು
ನಿರ-ತ-ರಾ-ಗಿ-ದ್ದಾರೆ
ನಿರ-ತ-ರಾ-ಗಿ-ರುವ
ನಿರ-ತ-ಳಾ-ಗಿ-ದ್ದಳು
ನಿರ-ತ-ಳಾ-ದಳು
ನಿರ-ತಿ-ಶ-ಯ-ಮ-ಹಿ-ಮನೂ
ನಿರ-ತಿ-ಶ-ಯ-ವಾದ
ನಿರ-ಪ-ರಾ-ಧಿ-ಗ-ಳಾದ
ನಿರ-ಪ-ರಾ-ಧಿ-ನಿ-ಯೆಂದು
ನಿರ-ಪ-ರಾ-ಧಿ-ಯಾದ
ನಿರ-ಪೇಕ್ಷ
ನಿರ-ಯಾ-ದ-ಶೇ-ಷಾತ್
ನಿರ-ಯೌ-ಪಮ್ಯಂ
ನಿರ-ರ್ಗ-ಳ-ವಾಗಿ
ನಿರ-ರ್ಥ-ಕ-ವೆಂದು
ನಿರ-ಶ-ನ-ವ್ರ-ತ-ದಿಂದ
ನಿರಾ
ನಿರಾ-ಕ-ರಣಂ
ನಿರಾ-ಕ-ರಿ-ಸದೆ
ನಿರಾ-ಕ-ರಿಸಿ
ನಿರಾ-ಕ-ರಿ-ಸಿ-ಯಾನು
ನಿರಾ-ಕ-ರಿ-ಸುವ
ನಿರಾ-ಕ-ರಿ-ಸು-ವನೊ
ನಿರಾ-ಕಾರ
ನಿರಾ-ಕಾ-ರ-ದಲ್ಲಿ
ನಿರಾ-ಕಾ-ರ-ನಾ-ದ-ವನು
ನಿರಾ-ತಂ-ಕ-ವಾಗಿ
ನಿರಾ-ತಂ-ಕ-ವಾ-ಗಿಯೇ
ನಿರಾ-ತಂ-ಕ-ವಾ-ಗಿ-ರುವ
ನಿರಾ-ತಂ-ಕ-ವಾದ
ನಿರಾ-ಯು-ಧ-ನಾ-ಗಿದ್ದ
ನಿರಾ-ಯು-ಧ-ನಾದ
ನಿರಾ-ಶ-ರಾ-ಗ-ಬೇ-ಕಾ-ಗಿಲ್ಲ
ನಿರಾ-ಶ-ರಾಗಿ
ನಿರಾ-ಶ-ಳ-ನ್ನಾಗಿ
ನಿರಾ-ಶಾಃ
ನಿರಾ-ಶೆ-ಯಾ-ಗುವ
ನಿರಾ-ಶೆ-ಯಾ-ಯಿತು
ನಿರಾ-ಶೆ-ಯಿಂದ
ನಿರಾ-ಶೆ-ಯಿಂ-ದಿ-ರು-ವುದೇ
ನಿರಾ-ಶ್ರಯ
ನಿರಾ-ಹಾ-ರಿ-ಯಾ-ದನು
ನಿರಿ-ಯನ್ನು
ನಿರೀ-ಕ್ಷ-ಣೆ-ಯಲ್ಲಿ
ನಿರೀ-ಕ್ಷಿ-ಸ-ಬೇಕೇ
ನಿರೀ-ಕ್ಷಿ-ಸುತ್ತಾ
ನಿರೀ-ಕ್ಷಿ-ಸು-ತ್ತಿ-ದ್ದರು
ನಿರೀ-ಕ್ಷಿ-ಸು-ತ್ತಿ-ದ್ದರೆ
ನಿರೀ-ಕ್ಷಿ-ಸು-ತ್ತಿವೆ
ನಿರೀ-ಕ್ಷಿ-ಸುವ
ನಿರೂ-ಪಣೆ
ನಿರೂ-ಪ-ಣೆಯ
ನಿರೂ-ಪ-ವನ್ನು
ನಿರೂ-ಪಿ-ತ-ವಾ-ಗಿವೆ
ನಿರೂ-ಪಿ-ಸಿ-ದನು
ನಿರೂ-ಪಿ-ಸಿದೆ
ನಿರೂ-ಪಿ-ಸು-ತ್ತಿ-ರುವ
ನಿರೂ-ಪಿ-ಸುವ
ನಿರೂ-ಪ್ಯತೇ
ನಿರೋಧ
ನಿರ್ಗ-ತಿ-ಕ-ನಂತೆ
ನಿರ್ಗುಣ
ನಿರ್ಗು-ಣ-ಬ್ರ-ಹ್ಮ-ನನ್ನು
ನಿರ್ಗು-ಣ-ಭ-ಕ್ತಿ-ಯು-ಳ್ಳ-ವನು
ನಿರ್ಜೀ-ವ-ವಾದ
ನಿರ್ಣಯ
ನಿರ್ಣ-ಯ-ಸ್ಥಾ-ನ-ವಾ-ಗಿಯೂ
ನಿರ್ದ-ಯರೂ
ನಿರ್ದ-ಯ-ವಾಗಿ
ನಿರ್ದಾ-ಕ್ಷಿಣ್ಯ
ನಿರ್ದಿಷ್ಟ
ನಿರ್ಧ-ರ-ವಾ-ಯಿತು
ನಿರ್ಧ-ರಿ-ಸ-ಬೇಕು
ನಿರ್ಧ-ರಿ-ಸ-ಬೇ-ಕೆಂದು
ನಿರ್ಧ-ರಿಸಿ
ನಿರ್ಧ-ರಿ-ಸಿ-ದರು
ನಿರ್ಧ-ರಿ-ಸಿ-ರು-ತ್ತಾರೆ
ನಿರ್ಧ-ರಿ-ಸು-ವುದು
ನಿರ್ನಾ-ಮ-ಮಾ-ಡ-ಲೆಂದು
ನಿರ್ನಾ-ಮ-ಮಾ-ಡಿ-ದನು
ನಿರ್ನಾ-ಮ-ವಾ-ಯಿತು
ನಿರ್ಬಂ-ಧ-ವನ್ನು
ನಿರ್ಭಯ
ನಿರ್ಭ-ಯ-ನಾ-ಗ-ಬೇ-ಕಾ-ದರೆ
ನಿರ್ಭ-ಯ-ವಾಗಿ
ನಿರ್ಭ-ಯ-ವಾ-ಗಿರು
ನಿರ್ಭ-ಯ-ವಾ-ಗಿ-ರು-ತ್ತಿ-ದ್ದವು
ನಿರ್ಭಾಗ್ಯ
ನಿರ್ಭಾ-ಗ್ಯರ
ನಿರ್ಭಾ-ಗ್ಯ-ವಾ-ಯಿತು
ನಿರ್ಭೇ-ದ-ವಾ-ಗು-ವಂತೆ
ನಿರ್ಮಲ
ನಿರ್ಮ-ಲ-ಜ-ಲದ
ನಿರ್ಮ-ಲ-ವಾಗಿ
ನಿರ್ಮ-ಲ-ವಾದ
ನಿರ್ಮ-ಲ-ವಾ-ಯಿತು
ನಿರ್ಮಾಣ
ನಿರ್ಮಾ-ಣ-ಮಾ-ಡಿ-ಕೊ-ಟ್ಟನು
ನಿರ್ಮಾ-ಣ-ವಾಗಿ
ನಿರ್ಮಾ-ಣ-ವಾದಂ
ನಿರ್ಮಾ-ಣ-ವಾ-ದವು
ನಿರ್ಮಿ
ನಿರ್ಮಿ-ತ-ವಾಗಿ
ನಿರ್ಮಿ-ತ-ವಾದ
ನಿರ್ಮಿ-ತ-ವಾ-ದುದು
ನಿರ್ಮಿ-ಸ-ಬೇ-ಕೆಂ-ದು-ಕೊಂ-ಡರು
ನಿರ್ಮಿಸಿ
ನಿರ್ಮಿ-ಸಿ-ಕೊ-ಟ್ಟನು
ನಿರ್ಮಿ-ಸಿ-ಕೊ-ಟ್ಟಿ-ದ್ದಾನೆ
ನಿರ್ಮಿ-ಸಿದ
ನಿರ್ಮಿ-ಸಿ-ದನು
ನಿರ್ಮೂಲ
ನಿರ್ಮೂ-ಲ-ಮಾ-ಡಿ-ದರೆ
ನಿರ್ಮೂ-ಲ-ಮಾ-ಡು-ತ್ತೇನೆ
ನಿರ್ಮೂ-ಲ-ಮಾ-ಡುವ
ನಿರ್ಮೂ-ಲ-ಮಾ-ಡು-ವನು
ನಿರ್ಮೂ-ಲ-ಮಾ-ಡು-ವು-ದ-ಕ್ಕಾಗಿ
ನಿರ್ಮೂ-ಲ-ವಾ-ಗು-ತ್ತದೆ
ನಿರ್ಮೂ-ಲ-ವಾ-ಗು-ವಂ-ತಿ-ದ್ದರೂ
ನಿರ್ಯಾ-ಣಕ್ಕೆ
ನಿರ್ಲ-ಜ್ಜೆ-ಯಿಂದ
ನಿರ್ಲೇ-ಪ-ನಾದ
ನಿರ್ವಂ-ಚ-ನೆ-ಯಿಂದ
ನಿರ್ವಂ-ಶ-ವಾಗಿ
ನಿರ್ವಂ-ಶ-ವಾ-ಗು-ತ್ತದೆ
ನಿರ್ವಹಿ
ನಿರ್ವ-ಹಿ-ಸ-ಬಲ್ಲೆ
ನಿರ್ವ-ಹಿಸಿ
ನಿರ್ವ-ಹಿ-ಸು-ತ್ತದೆ
ನಿರ್ವ-ಹಿ-ಸು-ತ್ತೇನೆ
ನಿರ್ವಿಂ-ಧ್ಯೆ-ಯೆಂಬ
ನಿರ್ವಿ-ಕಾರ
ನಿರ್ವಿ-ಕಾ-ರ-ನಾ-ಗಿ-ರುವ
ನಿರ್ವಿ-ಕಾ-ರ-ನಾ-ಗಿ-ರು-ವನು
ನಿರ್ವಿ-ಕಾ-ರ-ನಾದ
ನಿರ್ವಿ-ಕಾರಿ
ನಿರ್ವಿ-ಘ್ನ-ವಾಗಿ
ನಿರ್ವಿ-ಶೇಷ
ನಿಲು-ವಿ-ನಿಂದ
ನಿಲ್ಲ-ಕೂ-ಡದು
ನಿಲ್ಲದೆ
ನಿಲ್ಲ-ಬ-ಲ್ಲ-ವ-ರಾರು
ನಿಲ್ಲ-ಬೇ-ಕಾ-ಯಿತು
ನಿಲ್ಲ-ಬೇಕು
ನಿಲ್ಲ-ಬೇ-ಕೆಂದು
ನಿಲ್ಲ-ಬೇಡ
ನಿಲ್ಲ-ಲಾ-ರದೆ
ನಿಲ್ಲ-ಲಿಲ್ಲ
ನಿಲ್ಲಲು
ನಿಲ್ಲಿ-ಸ-ಬೇಕು
ನಿಲ್ಲಿ-ಸ-ಬೇ-ಡ-ವೆಂದು
ನಿಲ್ಲಿ-ಸಲು
ನಿಲ್ಲಿಸಿ
ನಿಲ್ಲಿ-ಸಿ-ಕೊಂ-ಡಿದ್ದು
ನಿಲ್ಲಿ-ಸಿದ
ನಿಲ್ಲಿ-ಸಿ-ದ-ನಂತೆ
ನಿಲ್ಲಿ-ಸಿ-ದನು
ನಿಲ್ಲಿ-ಸಿ-ದರು
ನಿಲ್ಲಿ-ಸಿದ್ದ
ನಿಲ್ಲಿ-ಸಿ-ದ್ದೇನೆ
ನಿಲ್ಲಿ-ಸಿ-ಬಿಡಿ
ನಿಲ್ಲಿ-ಸಿರಿ
ನಿಲ್ಲಿ-ಸಿ-ರುವ
ನಿಲ್ಲಿ-ಸಿರೆ
ನಿಲ್ಲಿಸು
ನಿಲ್ಲಿ-ಸು-ತ್ತಾರೆ
ನಿಲ್ಲಿ-ಸು-ತ್ತಿ-ರು-ವಿರಿ
ನಿಲ್ಲಿ-ಸು-ತ್ತೇನೆ
ನಿಲ್ಲಿ-ಸು-ವು-ದ-ಕ್ಕಾಗಿ
ನಿಲ್ಲಿ-ಸು-ವುದು
ನಿಲ್ಲಿ-ಸು-ವುದೇ
ನಿಲ್ಲಿ-ಸೋಣ
ನಿಲ್ಲು
ನಿಲ್ಲು-ತ್ತದೆ
ನಿಲ್ಲು-ತ್ತ-ದೆಯೆ
ನಿಲ್ಲು-ತ್ತಲೆ
ನಿಲ್ಲು-ತ್ತ-ವಂತೆ
ನಿಲ್ಲು-ತ್ತವೆ
ನಿಲ್ಲುತ್ತಿ
ನಿಲ್ಲು-ತ್ತಿತ್ತು
ನಿಲ್ಲು-ತ್ತೇನೆ
ನಿಲ್ಲುವ
ನಿಲ್ಲು-ವಂ-ತಾ-ಯಿತು
ನಿಲ್ಲು-ವನು
ನಿಲ್ಲು-ವವ
ನಿಲ್ಲು-ವ-ವನೇ
ನಿಲ್ಲು-ವ-ಹಾಗೆ
ನಿಲ್ಲು-ವು-ದಕ್ಕೆ
ನಿಲ್ಲು-ವು-ದಿಲ್ಲ
ನಿಲ್ಲು-ವುದೂ
ನಿಲ್ಲು-ವುದೆ
ನಿಲ್ಲು-ವುದೇ
ನಿಲ್ಲುವೆ
ನಿವಾ-ರ-ಣೆ-ಗಾಗಿ
ನಿವಾ-ರ-ಣೆ-ಯಾ-ಗದು
ನಿವಾ-ರ-ಣೆ-ಯಾ-ಗ-ಬೇ-ಕಾ-ದರೆ
ನಿವಾ-ರ-ಣೆ-ಯಾಗು
ನಿವಾ-ರ-ಣೆ-ಯಾ-ಗು-ತ್ತದೆ
ನಿವಾ-ರ-ಣೆ-ಯಾ-ಗು-ತ್ತಲೆ
ನಿವಾ-ರ-ಣೆ-ಯಾ-ಗು-ತ್ತವೆ
ನಿವಾ-ರ-ಣೆ-ಯಾ-ಗು-ವುದೊ
ನಿವಾ-ರ-ಣೆ-ಯಾ-ದಂ-ತಾ-ಯಿತು
ನಿವಾ-ರ-ಣೆ-ಯಾ-ಯಿತು
ನಿವಾ-ರಿ-ಸ-ಬೇ-ಕಾ-ಯಿತು
ನಿವಾ-ರಿ-ಸ-ಬೇಕು
ನಿವಾ-ರಿಸಿ
ನಿವಾ-ರಿ-ಸಿರಿ
ನಿವಾ-ರಿ-ಸುತ್ತಾ
ನಿವಾ-ರಿ-ಸು-ತ್ತಿ-ರು-ವುದು
ನಿವಾ-ರಿ-ಸು-ತ್ತೇನೆ
ನಿವಾ-ರಿ-ಸುವ
ನಿವಾ-ರಿ-ಸು-ವು-ದ-ಕ್ಕಾ-ಗಿಯೆ
ನಿವೃ-ತ್ತ-ಗು-ಣ-ವೃ-ತ್ತಯೇ
ನಿವೃ-ತ್ತ-ದ್ವೈ-ತ-ದೃ-ಷ್ಟಯೇ
ನಿವೃ-ತ್ತ-ನಾ-ದಂ-ತಾ-ಯಿತು
ನಿವೃತ್ತಿ
ನಿವೃ-ತ್ತಿ-ಧರ್ಮ
ನಿವೃ-ತ್ತಿ-ಧ-ರ್ಮ-ವನ್ನು
ನಿವೃ-ತ್ತಿ-ಧ-ರ್ಮ-ವನ್ನೇ
ನಿವೇ-ದಿಸಿ
ನಿವೇ-ದಿ-ಸಿದ
ನಿವೇ-ದಿ-ಸಿ-ದರು
ನಿಶ್ಚಯ
ನಿಶ್ಚ-ಯ-ಗ-ಳಿಗೂ
ನಿಶ್ಚ-ಯ-ಜ್ಞಾನ
ನಿಶ್ಚ-ಯ-ಮಾ-ಡಿ-ಕೊಂಡು
ನಿಶ್ಚ-ಯ-ವನ್ನು
ನಿಶ್ಚ-ಯ-ವಾ-ಗು-ತ್ತಲೆ
ನಿಶ್ಚ-ಯ-ವಾದ
ನಿಶ್ಚಯಿ
ನಿಶ್ಚ-ಯಿಸಿ
ನಿಶ್ಚ-ಯಿ-ಸಿದ
ನಿಶ್ಚ-ಯಿ-ಸಿ-ದನು
ನಿಶ್ಚ-ಯಿ-ಸಿ-ದರು
ನಿಶ್ಚ-ಯಿ-ಸಿ-ದು-ದ-ರಿಂದ
ನಿಶ್ಚ-ಯಿ-ಸಿ-ದ್ದರು
ನಿಶ್ಚಲ
ನಿಶ್ಚ-ಲ-ಚಿ-ತ್ತ-ನಾಗಿ
ನಿಶ್ಚ-ಲ-ವಾ-ಗ-ಲೆಂದೆ
ನಿಶ್ಚ-ಲ-ವಾಗಿ
ನಿಶ್ಚ-ಲ-ವಾದ
ನಿಶ್ಚ-ಲ-ವಾ-ದಾಗ
ನಿಶ್ಚಿಂ-ತ-ನಾಗಿ
ನಿಶ್ಚಿಂ-ತ-ನಾದ
ನಿಶ್ಚಿಂ-ತ-ಳಾಗಿ
ನಿಶ್ಚಿಂತೆ
ನಿಶ್ಚಿಂ-ತೆ-ಯಾ-ಗಿ-ರು-ವುದು
ನಿಶ್ಚಿಂ-ತೆ-ಯಿಂದ
ನಿಶ್ಚಿಂ-ತೆ-ಯಿಂ-ದಿ-ರ-ಬ-ಹುದು
ನಿಶ್ವಾಸ
ನಿಶ್ಶ-ಬ್ದ-ವಾಗಿ
ನಿಶ್ಶೇಷ
ನಿಶ್ಶೇ-ಷ-ರಾ-ಗದ
ನಿಷ-ಧ-ಇ-ತ್ಯಾದಿ
ನಿಷಿ-ದ್ಧ-ಕಾರ್ಯ
ನಿಷೀದ
ನಿಷ್ಕಂ-ಟ-ಕ-ವಾಗಿ
ನಿಷ್ಕ-ರು-ಣೆ-ಯಿಂದ
ನಿಷ್ಕಾ-ಮ-ಕರ್ಮ
ನಿಷ್ಕಾ-ಮ-ಕ-ರ್ಮ-ಇ-ವನ್ನು
ನಿಷ್ಕಾ-ಮ-ಕ-ರ್ಮ-ಗಳನ್ನು
ನಿಷ್ಕಾ-ಮ-ದಿಂದ
ನಿಷ್ಕಾ-ಮ-ನಾಗಿ
ನಿಷ್ಕಾ-ಮ-ಭಕ್ತಿ
ನಿಷ್ಕೃ-ತಿ-ಯೆಲ್ಲಿ
ನಿಷ್ಠ-ರಾದ
ನಿಷ್ಠೆ-ಯು-ಳ್ಳ-ವರು
ನಿಷ್ಪ-ನಸೇ
ನಿಷ್ಪಾಂ-ಡ-ವಾ-ಸ್ತ್ರ-ವನ್ನು
ನಿಷ್ಪಿಂಡಿ
ನಿಷ್ಪಿಂ-ಡ್ಯ-ಜಿ-ತ-ಪ್ತಿ-ಯಾಸಿ
ನಿಷ್ಫಲ
ನಿಸ್ತೇಜ
ನಿಸ್ಸಂ-ದಿ-ಗ್ಧ-ವಾಗಿ
ನಿಸ್ಸಂ-ದೇ-ಹ-ವಾಗಿ
ನಿಸ್ಸಂ-ಶ-ಯ-ವಾ-ಗಿಯೂ
ನಿಸ್ಸ-ಹಾ-ಯ-ನಾಗಿ
ನೀಗಿ
ನೀಗಿ-ಕೊಂಡ
ನೀಗಿ-ಕೊಂ-ಡಿತು
ನೀಗಿ-ಕೊ-ಳ್ಳ-ಬೇಕು
ನೀಗಿತು
ನೀಗಿದ
ನೀಗಿ-ದನು
ನೀಗಿಲ್ಲ
ನೀಗು-ತ್ತದೆ
ನೀಗು-ತ್ತ-ದೆಯೇ
ನೀಗುವ
ನೀಗು-ವಂತೆ
ನೀಗು-ವುದು
ನೀಚ
ನೀಚ-ಕಾ-ರ್ಯ-ಬ್ರಹ್ಮ
ನೀಚ-ಕಾ-ರ್ಯ-ವನ್ನು
ನೀಚ-ಗ್ರಹ
ನೀಚ-ನನ್ನು
ನೀಚ-ನಾ-ಗಿ-ದ್ದ-ವನು
ನೀಚ-ನಾದ
ನೀಚ-ನಿಗೆ
ನೀಚ-ನುಂಟೆ
ನೀಚ-ಬು-ದ್ಧಿ-ಯನ್ನು
ನೀಚ-ಬು-ದ್ಧಿ-ಯ-ವ-ನಾದ
ನೀಚ-ರಲ್ಲಿ
ನೀಚ-ರಾದ
ನೀಚ-ರಾ-ದ-ವರು
ನೀಚರು
ನೀಚಳೆ
ನೀಚ-ವ-ರ್ಣ-ಆ-ಗಿ-ಹೋ-ಗು-ತ್ತದೆ
ನೀಡ-ತ-ಕ್ಕುದು
ನೀಡದೆ
ನೀಡ-ಬೇಕು
ನೀಡ-ಬೇ-ಕೆಂದು
ನೀಡ-ಲಿ-ಲ್ಲ-ವಂತೆ
ನೀಡಲು
ನೀಡಿ
ನೀಡಿದ
ನೀಡಿ-ದನು
ನೀಡಿ-ದ-ಮೇಲೆ
ನೀಡಿ-ದರು
ನೀಡಿ-ದರೂ
ನೀಡಿ-ದಳು
ನೀಡಿ-ದ-ವರು
ನೀಡಿದೆ
ನೀಡಿ-ದ್ದಾನೆ
ನೀಡಿ-ದ್ದೇನೆ
ನೀಡಿ-ರುವ
ನೀಡು
ನೀಡು-ತ್ತದೆ
ನೀಡು-ತ್ತಿತ್ತು
ನೀಡು-ತ್ತಿದ್ದ
ನೀಡು-ತ್ತಿ-ದ್ದರು
ನೀಡು-ತ್ತಿ-ದ್ದವು
ನೀಡು-ತ್ತಿ-ದ್ದೇನೆ
ನೀಡು-ತ್ತಿ-ರುವ
ನೀಡು-ತ್ತಿ-ವೆಯೊ
ನೀಡುವ
ನೀಡು-ವಂತೆ
ನೀಡು-ವ-ವ-ರಲ್ಲಿ
ನೀಡು-ವು-ದ-ಕ್ಕಾ-ಗಿಯೇ
ನೀಡು-ವುದೇ
ನೀತಿ
ನೀತಿ-ಗೆ-ಟ್ಟ-ವಳೆ
ನೀತಿ-ನಿ-ಯ-ಮ-ಗ-ಳುಂಟೆ
ನೀತಿ-ಯನ್ನು
ನೀತಿ-ಯಿದೆ
ನೀನ-ಲ್ಲದೆ
ನೀನಾ
ನೀನಾ-ಗಲೀ
ನೀನಾ-ಗಲೆ
ನೀನಾ-ಗಿಯೇ
ನೀನಾ-ಗಿ-ರ-ಬ-ಹುದು
ನೀನಾ-ಗಿ-ರು-ವಾಗ
ನೀನಾ-ರನ್ನು
ನೀನಾರು
ನೀನಾ-ರೆಂ-ಬು-ದನ್ನು
ನೀನಿದ್ದೆ
ನೀನಿನ್ನು
ನೀನಿನ್ನೂ
ನೀನಿ-ರುವ
ನೀನಿ-ರುವೆ
ನೀನಿ-ರು-ವೆ-ಯೆಂದೊ
ನೀನಿ-ಲ್ಲ-ದಿ-ದ್ದರೆ
ನೀನಿಲ್ಲಿ
ನೀನಿ-ವ-ನಿಗೆ
ನೀನಿಷ್ಟು
ನೀನೀಗ
ನೀನೀ-ಗಲೆ
ನೀನೀ-ಗಲೇ
ನೀನು
ನೀನುಟ್ಟ
ನೀನೂ
ನೀನೆ
ನೀನೆಂ-ತಹ
ನೀನೆಂದು
ನೀನೆಲ್ಲಿ
ನೀನೇ
ನೀನೇಕೆ
ನೀನೇನು
ನೀನೇನೂ
ನೀನೇ-ನೇನು
ನೀನೇನೊ
ನೀನೊಬ್ಬ
ನೀನೊ-ಬ್ಬನೆ
ನೀನೊ-ಬ್ಬಳೆ
ನೀನೊ-ಬ್ಬಳೇ
ನೀನೋ
ನೀರನ್ನು
ನೀರನ್ನೂ
ನೀರ-ನ್ನೆ-ರಚಿ
ನೀರಾಟ
ನೀರಾ-ಟ-ಕ್ಕಿ-ಳಿ-ದಿ-ದ್ದರು
ನೀರಾ-ಟಕ್ಕೆ
ನೀರಾ-ಟ-ವಾಡಿ
ನೀರಾ-ಟ-ವಾ-ಡುತ್ತಾ
ನೀರಾ-ಟ-ವಾ-ಡು-ತ್ತಿದ್ದ
ನೀರಾ-ಟ-ವಾ-ಡು-ತ್ತಿ-ದ್ದನು
ನೀರಾ-ಟ-ವಾ-ಡು-ತ್ತಿ-ದ್ದರು
ನೀರಾ-ಟ-ವಾ-ಡುವ
ನೀರಾದ
ನೀರಿ
ನೀರಿ-ಗಿಳಿ-ದು-ದನ್ನು
ನೀರಿ-ಗಿಳಿ-ಯು-ವಂತೆ
ನೀರಿಗೆ
ನೀರಿತ್ತು
ನೀರಿನ
ನೀರಿ-ನಂ-ತಾ-ಗು-ವು-ದೆಂದು
ನೀರಿ-ನಂತೆ
ನೀರಿ-ನ-ಮೇಲೆ
ನೀರಿ-ನಲ್ಲಿ
ನೀರಿ-ನ-ಲ್ಲಿ-ರು-ವಂತೆ
ನೀರಿ-ನಿಂದ
ನೀರಿ-ನಿಂ-ದೆತ್ತಿ
ನೀರಿ-ನಿಂ-ದೆದ್ದು
ನೀರು
ನೀರುಂಡ
ನೀರು-ಏನು
ನೀರು-ಕು-ಡಿ-ಯಿತು
ನೀರು-ಕು-ಡಿಸಿ
ನೀರು-ಕ್ಕು-ವಂತೆ
ನೀರು-ಗಳನ್ನು
ನೀರು-ಗ-ಳೆ-ರ-ಡನ್ನೂ
ನೀರು-ಗು-ಳ್ಳೆ-ಯಂತೆ
ನೀರು-ಪಾಲು
ನೀರು-ಹೂ-ಗಳಿಂದ
ನೀರೂ
ನೀರೂ-ರಿತು
ನೀರೂ-ರು-ತ್ತದೆ
ನೀರೂ-ರು-ವಂತೆ
ನೀರೆಂದು
ನೀರೆಂದೂ
ನೀರೆ-ರ-ಚಿ-ದರು
ನೀರೆ-ರೆ-ದರು
ನೀರೆ-ರೆ-ದು-ಕೊ-ಳ್ಳು-ತ್ತಿ-ದ್ದಾನೆ
ನೀರೆ-ರೆ-ಯ-ಬೇಕು
ನೀರೆ-ರೆ-ಯಲು
ನೀರೆಲ್ಲ
ನೀರೆ-ಲ್ಲವೂ
ನೀರೊ-ಣಗಿ
ನೀರೊಳ
ನೀರೊ-ಳಕ್ಕೆ
ನೀರೊ-ಳ-ಕ್ಕೆ-ಸೆ-ದನು
ನೀರೊ-ಳ್ಳೆ-ಗಳು
ನೀಲ
ನೀಲ-ಕಂ-ಠ-ನೆಂದು
ನೀಲ-ಮ-ಣಿಯ
ನೀಲ-ಮೇಘ
ನೀಲ-ಮೇ-ಘ-ಶ್ಯಾ-ಮ-ನಾಗಿ
ನೀಲ-ಮೇ-ಘ-ಶ್ಯಾ-ಮ-ನಾದ
ನೀಲಿ
ನೀಳ-ವಾದ
ನೀವಾ-ರ-ಧಾನ್ಯ
ನೀವಾರು
ನೀವಿಂದು
ನೀವಿ-ಕೊ-ಳ್ಳುತ್ತಾ
ನೀವಿನ್ನು
ನೀವಿ-ಬ್ಬರೂ
ನೀವಿ-ರು-ವು-ದಕ್ಕೆ
ನೀವಿ-ಲ್ಲಿಗೆ
ನೀವೀಗ
ನೀವೀ-ಗಲೆ
ನೀವು
ನೀವೂ
ನೀವೆ
ನೀವೆಲ್ಲ
ನೀವೆ-ಲ್ಲರು
ನೀವೆ-ಲ್ಲರೂ
ನೀವೇ
ನೀವೇಕೆ
ನೀವೇ-ನೆಂದು
ನೀವೊಮ್ಮೆ
ನೀಶೀಥ
ನು
ನುಂಗ
ನುಂಗದೆ
ನುಂಗ-ಲಾ-ರದ
ನುಂಗ-ಲೆಂದೆ
ನುಂಗಿ
ನುಂಗಿ-ಕೊಂಡು
ನುಂಗಿತು
ನುಂಗಿ-ದ-ವನು
ನುಂಗಿದೆ
ನುಂಗಿ-ದೆ-ನೆಂದು
ನುಂಗಿ-ಹಾಕಿ
ನುಂಗಿ-ಹಾ-ಕಿದ್ದಿ
ನುಂಗಿ-ಹಾ-ಕು-ತ್ತದೆ
ನುಂಗಿ-ಹಾ-ಕು-ತ್ತಿ-ದ್ದರು
ನುಂಗುತ್ತಾ
ನುಂಗು-ತ್ತಿದೆ
ನುಂಗು-ತ್ತಿ-ದೆ-ಯಲ್ಲಾ
ನುಂಗು-ತ್ತಿ-ರು-ವೆ-ಯಲ್ಲ
ನುಂಗುವ
ನುಂಗು-ವಂ-ತಹ
ನುಂಗು-ವ-ವ-ನಂತೆ
ನುಂಗು-ವು-ದಕ್ಕೆ
ನುಂಗು-ವು-ದ-ಕ್ಕೆಂ-ಬಂತೆ
ನುಗುತ್ತಾ
ನುಗ್ಗ-ದಂತೆ
ನುಗ್ಗಿ
ನುಗ್ಗಿತು
ನುಗ್ಗಿ-ದನು
ನುಗ್ಗಿ-ದರು
ನುಗ್ಗಿ-ಬಂ-ದನು
ನುಗ್ಗಿ-ಬಂ-ದ-ವನೆ
ನುಗ್ಗಿ-ಬಂ-ದವು
ನುಗ್ಗಿ-ಬಂದು
ನುಗ್ಗಿಸಿ
ನುಗ್ಗಿ-ಸಿದ
ನುಗ್ಗಿ-ಹೋ-ದನು
ನುಚ್ಚು-ನೂ-ರಾಗಿ
ನುಜ್ಜು-ಗು-ಜ್ಜಾ-ದವು
ನುಡಿ
ನುಡಿ-ಗಳನ್ನು
ನುಡಿ-ಗಳಲ್ಲಿ
ನುಡಿ-ಗಳಿಂದ
ನುಡಿ-ಗ-ಳಿಂ-ದಲೆ
ನುಡಿ-ಗಳು
ನುಡಿಗೆ
ನುಡಿದ
ನುಡಿ-ದಂತೆ
ನುಡಿ-ದ-ನು-ಮ-ಹಾ-ರಾಜ
ನುಡಿ-ದರು
ನುಡಿ-ದಳು
ನುಡಿ-ದ-ವರು
ನುಡಿಯ
ನುಡಿ-ಯು-ತ್ತಿ-ರುವೆ
ನುಡಿಸಿ
ನುಡಿ-ಸುತ್ತ
ನುಡು-ಗಿ-ದಳು
ನುಣುಚಿ
ನುಣು-ಚಿ-ಕೊಂಡು
ನುಣು-ಪಾದ
ನುರಿತ
ನೂಕಿ
ನೂಕಿ-ದನು
ನೂಕಿ-ದರು
ನೂಕಿ-ದುದು
ನೂಕಿ-ಸಿ-ದನು
ನೂಕು-ತ್ತಿ-ರಲು
ನೂಕು-ವು-ದಕ್ಕೂ
ನೂತನ
ನೂನಂ
ನೂರ
ನೂರ-ನೆಯ
ನೂರ-ಮು-ವ-ತ್ತೆಂ-ಟು-ವರ್ಷ
ನೂರ-ಮೂ-ವ-ತ್ತೇಳು
ನೂರ-ರ-ಷ್ಟಾ-ದವು
ನೂರಾರು
ನೂರಿ-ಪ್ಪ-ತ್ತೈ-ದು-ವ-ರ್ಷ-ಗಳು
ನೂರು
ನೂರು-ಕಾಲ
ನೂರು-ಮಂದಿ
ನೂರು-ವ-ರ್ಷ-ಗ-ಳ-ವ-ರೆಗೆ
ನೂರು-ವ-ರ್ಷದ
ನೃಗ
ನೃತ್ಯ-ಗ-ಳಿಗೆ
ನೃಭ್ಯ
ನೃಷ-ದ್ರಿಂಗಿ
ನೃಸಿಂ-ಹಾ-ವ-ತಾ-ರ್ಅ
ನೆಂಟನೂ
ನೆಂಟ-ಸ್ತನ
ನೆಂತು
ನೆಂದರೆ
ನೆಂದು
ನೆಂದೂ
ನೆಂಬ
ನೆಂಬುದು
ನೆಂಬುದೇ
ನೆಕ್ಕು-ತ್ತಿರು
ನೆಗದು
ನೆಗೆ-ತಕ್ಕೆ
ನೆಗೆ-ದ-ವನೆ
ನೆಗೆ-ದ-ವನೇ
ನೆಗೆದು
ನೆಗೆ-ಯಿತು
ನೆಗೆ-ವರು
ನೆಟ್ಟ
ನೆಟ್ಟ-ಗಿದ್ದ
ನೆಟ್ಟಗೆ
ನೆಟ್ಟ-ದಿ-ಟ್ಟಿ-ಯಿಂದ
ನೆಟ್ಟ-ಮ-ನ-ಸ್ಸು-ಳ್ಳ-ವ-ನಾಗಿ
ನೆಟ್ಟ-ರೆಂದು
ನೆಟ್ಟ-ವನು
ನೆಟ್ಟ-ವ-ರಾ-ರೆಂದು
ನೆಟ್ಟಿ-ರುವ
ನೆತ್ತ
ನೆತ್ತರ
ನೆತ್ತ-ರನ್ನು
ನೆತ್ತ-ರಿಂದ
ನೆತ್ತ-ರಿಂ-ದಲೆ
ನೆತ್ತ-ರಿನ
ನೆತ್ತ-ರಿ-ನಲ್ಲಿ
ನೆತ್ತ-ರಿ-ನಿಂದ
ನೆತ್ತರು
ನೆತ್ತಿಗೆ
ನೆತ್ತಿ-ಗೇ-ರಿದೆ
ನೆತ್ತಿ-ಗೇ-ರಿ-ರುವ
ನೆತ್ತಿಯ
ನೆತ್ತಿ-ಯನ್ನು
ನೆನ-ಪನ್ನು
ನೆನ-ಪಾ-ಯಿತು
ನೆನ-ಪಿ-ದೆಯೋ
ನೆನ-ಪಿ-ನಲ್ಲಿ
ನೆನ-ಪಿ-ಸಿ-ಕೊಂಡು
ನೆನಪು
ನೆನ-ಸಲಿ
ನೆನ-ಸಿ-ಕೊ-ಳ್ಳು-ತ್ತಾನೊ
ನೆನಿ-ಸಿ-ಕೊ-ಳ್ಳು-ವನು
ನೆನಿ-ಸಿ-ರುವ
ನೆನೆ
ನೆನೆದ
ನೆನೆ-ದನು
ನೆನೆ-ದರೆ
ನೆನೆ-ದಳು
ನೆನೆ-ದ-ವರ
ನೆನೆದು
ನೆನೆ-ದು-ಕೊಂಡು
ನೆನೆ-ದು-ದೆಲ್ಲ
ನೆನೆದೆ
ನೆನೆ-ದೊ-ಡ-ನೆಯೇ
ನೆನೆ-ನೆ-ನೆದು
ನೆನೆ-ಯದೆ
ನೆನೆ-ಯುತ್ತ
ನೆನೆ-ಯು-ತ್ತಲೆ
ನೆನೆ-ಯು-ತ್ತಾನೆ
ನೆನೆ-ಯು-ತ್ತಾರೆ
ನೆನೆಸ
ನೆನೆ-ಸಿ-ಕೊಂ-ಡರೆ
ನೆನೆ-ಸಿ-ಕೊ-ಳ್ಳು-ತ್ತೇವೆ
ನೆನೆ-ಸಿ-ಕೊ-ಳ್ಳು-ವಂ-ತೆಯೇ
ನೆನೆ-ಸಿ-ಕೊ-ಳ್ಳು-ವನೋ
ನೆಪ
ನೆಪ-ದಲ್ಲಿ
ನೆಪ-ದಿಂದ
ನೆಪ-ದಿಂ-ದಲೆ
ನೆಪ-ಮಾ-ಡಿ-ಕೊಂಡು
ನೆಪ-ವಾಗಿ
ನೆಮ್ಮದಿ
ನೆಮ್ಮ-ದಿ-ಯಾಗಿ
ನೆಮ್ಮ-ದಿ-ಯಾ-ಗಿ-ರು-ವರೆ
ನೆಮ್ಮ-ದಿ-ಯಾ-ಗು-ತ್ತದೆ
ನೆಮ್ಮ-ದಿ-ಯಿಂದ
ನೆಮ್ಮ-ದಿ-ಯಿಂ-ದಿ-ರು-ವು-ದ-ಕ್ಕಾಗಿ
ನೆಮ್ಮ-ದಿ-ಯಿಂ-ದಿ-ರು-ವು-ದಕ್ಕೆ
ನೆಮ್ಮ-ದಿ-ಯಿಲ್ಲ
ನೆಯ
ನೆಯ-ದಾಗಿ
ನೆಯೆ
ನೆಯೇ
ನೆರ
ನೆರ-ದ-ವರೆಲ್ಲ
ನೆರ-ಳನ್ನು
ನೆರ-ಳಲ್ಲಿ
ನೆರ-ಳಿನ
ನೆರ-ಳಿ-ನಿಂದ
ನೆರ-ಳಿಲ್ಲ
ನೆರಳು
ನೆರ-ವಾ-ಗು-ತ್ತೇನೆ
ನೆರ-ವಿಗೆ
ನೆರ-ವಿ-ಲ್ಲದೆ
ನೆರ-ವೇ-ರದೆ
ನೆರ-ವೇ-ರ-ಬೇಕು
ನೆರ-ವೇ-ರಲಿ
ನೆರ-ವೇರಿ
ನೆರ-ವೇ-ರಿತು
ನೆರ-ವೇ-ರಿ-ದಂ-ತಾ-ಯಿತು
ನೆರ-ವೇ-ರಿ-ಸ-ಬೇ-ಕೆಂ-ದಿ-ರಲು
ನೆರ-ವೇ-ರಿ-ಸಲು
ನೆರ-ವೇ-ರಿಸಿ
ನೆರ-ವೇ-ರಿ-ಸಿ-ಕೊಟ್ಟು
ನೆರ-ವೇ-ರಿ-ಸಿದ
ನೆರ-ವೇ-ರಿ-ಸಿ-ದನು
ನೆರ-ವೇ-ರಿ-ಸಿ-ದಳು
ನೆರ-ವೇ-ರಿ-ಸಿ-ದ್ದಾರೆ
ನೆರ-ವೇ-ರಿ-ಸು-ತ್ತಾನೆ
ನೆರ-ವೇ-ರಿ-ಸು-ತ್ತಿ-ರುವ
ನೆರ-ವೇ-ರಿ-ಸು-ವ-ಷ್ಟ-ರಲ್ಲಿ
ನೆರ-ವೇ-ರಿ-ಸು-ವುದು
ನೆರ-ವೇ-ರು-ತ್ತಲೆ
ನೆರ-ವೇ-ರು-ತ್ತಿತ್ತು
ನೆರ-ವೇ-ರು-ತ್ತಿದೆ
ನೆರ-ವೇ-ರು-ವಂತೆ
ನೆರ-ವೇ-ರು-ವು-ದೆಂತು
ನೆರಹಿ
ನೆರಿ-ಗೆ-ಗಳನ್ನು
ನೆರಿ-ಗೆ-ಯನ್ನು
ನೆರೆ
ನೆರೆದ
ನೆರೆ-ದರು
ನೆರೆ-ದ-ವರೂ
ನೆರೆ-ದ-ವರೆಲ್ಲ
ನೆರೆದಿ
ನೆರೆ-ದಿದ್ದ
ನೆರೆ-ದಿ-ದ್ದರು
ನೆರೆ-ದಿ-ದ್ದ-ವ-ರನ್ನು
ನೆರೆ-ದಿ-ದ್ದ-ವ-ರಿ-ಗೆಲ್ಲ
ನೆರೆ-ದಿ-ದ್ದ-ವರು
ನೆರೆ-ದಿ-ದ್ದ-ವರೆಲ್ಲ
ನೆರೆ-ದಿ-ದ್ದಾರೆ
ನೆರೆ-ದಿ-ದ್ದು-ದ-ರಿಂದ
ನೆರೆ-ದಿ-ರು-ತ್ತಿ-ದ್ದರು
ನೆರೆದು
ನೆರೆ-ವೇ-ರಿ-ದು-ದನ್ನು
ನೆರೆ-ಹೊ-ರೆಯ
ನೆಲ
ನೆಲ-ಕ್ಕು-ರು-ಳಿತು
ನೆಲ-ಕ್ಕು-ರು-ಳಿದ
ನೆಲ-ಕ್ಕು-ರು-ಳಿ-ದನು
ನೆಲ-ಕ್ಕು-ರು-ಳಿ-ದರು
ನೆಲ-ಕ್ಕು-ರು-ಳಿ-ದಳು
ನೆಲ-ಕ್ಕು-ರು-ಳಿ-ದುವು
ನೆಲ-ಕ್ಕು-ರು-ಳಿಸಿ
ನೆಲ-ಕ್ಕು-ರು-ಳು-ತ್ತಲೆ
ನೆಲ-ಕ್ಕು-ರು-ಳು-ತ್ತಿ-ದ್ದಂತೆ
ನೆಲಕ್ಕೆ
ನೆಲ-ಕ್ಕೊ-ರ-ಗಿದ
ನೆಲ-ಕ್ಕೊ-ರ-ಗಿ-ದರು
ನೆಲದ
ನೆಲ-ದ-ಮೇಲೆ
ನೆಲ-ವನ್ನು
ನೆಲವೆ
ನೆಲ-ವೆಂದೂ
ನೆಲ-ವೆಲ್ಲ
ನೆಲ-ಸ-ಬೇಕು
ನೆಲ-ಸಮ
ನೆಲ-ಸ-ಮ-ವಾ-ಗು-ವಂತೆ
ನೆಲ-ಸಲಿ
ನೆಲಸಿ
ನೆಲ-ಸಿತು
ನೆಲ-ಸಿದ
ನೆಲ-ಸಿ-ದಂ-ತಿ-ರುವ
ನೆಲ-ಸಿ-ದನು
ನೆಲ-ಸಿ-ದರು
ನೆಲ-ಸಿ-ದರೆ
ನೆಲ-ಸಿ-ದಳು
ನೆಲ-ಸಿದ್ದ
ನೆಲ-ಸಿ-ದ್ದರು
ನೆಲ-ಸಿ-ದ್ದರೂ
ನೆಲ-ಸಿ-ದ್ದವು
ನೆಲ-ಸಿ-ದ್ದಾನೆ
ನೆಲ-ಸಿದ್ದು
ನೆಲ-ಸಿ-ರ-ಬಾ-ರದು
ನೆಲ-ಸಿ-ರು-ತ್ತಾನೆ
ನೆಲ-ಸಿ-ರುವ
ನೆಲ-ಸಿ-ರು-ವನು
ನೆಲ-ಸಿ-ರು-ವ-ನೆಂದು
ನೆಲ-ಸಿ-ರು-ವ-ನೆಂ-ಬುದು
ನೆಲ-ಸಿ-ರು-ವ-ವನು
ನೆಲ-ಸಿ-ರು-ವ-ವರು
ನೆಲ-ಸಿ-ರು-ವಾಗ
ನೆಲ-ಸಿ-ರು-ವುದೇ
ನೆಲಸು
ನೆಲ-ಸು-ತ್ತದೆ
ನೆಲ-ಸು-ತ್ತಲೆ
ನೆಲ-ಸುವ
ನೆಲ-ಸು-ವಂ-ತಾ-ಯಿತು
ನೆಲ-ಸು-ವಂತೆ
ನೆಲ-ಸು-ವು-ದಕ್ಕೆ
ನೆಲು-ವಿನ
ನೆಲೆ
ನೆಲೆ-ಗ-ಟ್ಟೆಲ್ಲ
ನೆಲೆ-ಗಳನ್ನು
ನೆಲೆ-ಗೊ-ಳಿ-ಸ-ಬೇಕು
ನೆಲೆ-ಗೊ-ಳಿಸಿ
ನೆಲೆ-ಗೊ-ಳಿಸು
ನೆಲೆ-ಗೊ-ಳಿ-ಸು-ತ್ತಾರೆ
ನೆಲೆ-ಗೊ-ಳ್ಳು-ತ್ತದೆ
ನೆಲೆ-ಮನೆ
ನೆಲೆ-ಮ-ನೆ-ಯಾದ
ನೆಲೆ-ಯನ್ನು
ನೆಲೆ-ಯಲ್ಲಿ
ನೆಲೆ-ಯಾಗಿ
ನೆಲೆ-ಯಾ-ಗಿದ್ದ
ನೆಲೆ-ಯಾ-ಗಿ-ರು-ವಂತೆ
ನೆಲೆ-ಯಾದ
ನೆಲೆ-ಯಿಂ-ದಲೆ
ನೆಲೆ-ಯು-ಳ್ಳ-ವನೂ
ನೆಲೆ-ವ-ನೆಯಾ
ನೆಲೆ-ವ-ನೆ-ಯಾದ
ನೆಲೆ-ಸಲು
ನೆಲೆಸಿ
ನೆಲೆ-ಸಿತು
ನೆಲೆ-ಸಿದ
ನೆಲೆ-ಸಿ-ದನು
ನೆಲೆ-ಸಿದೆ
ನೆಲೆ-ಸಿದ್ದ
ನೆಲೆ-ಸಿ-ದ್ದಾನೆ
ನೆಲೆ-ಸಿ-ದ್ದೇನೆ
ನೆಲೆ-ಸಿರು
ನೆಲೆ-ಸಿ-ರುವ
ನೆಲೆ-ಸಿ-ರು-ವಂ-ತೆಯೊ
ನೆಲೆ-ಸಿ-ರು-ವನು
ನೆಲೆ-ಸಿ-ರು-ವಳು
ನೆಲೆ-ಸಿ-ರು-ವ-ವ-ನಾ-ದರೂ
ನೆಲೆ-ಸಿ-ರುವೆ
ನೆಲೆ-ಸಿ-ರು-ವೆಯೊ
ನೇಗಿ
ನೇಗಿ-ಲನ್ನು
ನೇಗಿ-ಲಿ-ನಂ-ತಹ
ನೇಗಿ-ಲಿ-ನಂತೆ
ನೇಗಿ-ಲಿ-ನಿಂದ
ನೇಗೆ-ಯ-ವ-ನೊ-ಬ್ಬನು
ನೇತಾ-ಡು-ತ್ತಿವೆ
ನೇತೃ-ತ್ವ-ದಲ್ಲಿ
ನೇತ್ರ-ತ್ರ-ಯಾಯ
ನೇತ್ರ-ನಿಗೆ
ನೇತ್ರೇ
ನೇಮಿ
ನೇಮಿ-ನಿ-ಮ್ನೈ-ರ-ಕ-ರೋ-ಚ್ಛಾ-ಯಾಂ
ನೇಮಿಸಿ
ನೇಮಿ-ಸಿ-ಕೊ-ಳ್ಳ-ಬೇಕು
ನೇಮಿ-ಸಿ-ಕೊ-ಳ್ಳ-ಬೇ-ಕೆಂದು
ನೇಮಿ-ಸಿ-ದನು
ನೇಮಿ-ಸಿ-ದರು
ನೇಮಿಸು
ನೇಮಿ-ಸು-ವನು
ನೇರ
ನೇರ-ಳೆ-ಮ-ರವೆ
ನೇರ-ವಾಗಿ
ನೇರ-ವಾ-ಗಿದೆ
ನೇವ-ರಿ-ಸಿ-ದನು
ನೈಮಿ-ತ್ತಿಕ
ನೈಮಿ-ಶಾ-ರ-ಣ್ಯಕ್ಕೆ
ನೈಮಿ-ಶಾ-ರ-ಣ್ಯದ
ನೈಮಿ-ಶಾ-ರ-ಣ್ಯ-ವೆಂ-ಬುದು
ನೈವಾ-ಽನ್ಯ-ದಾ-ಲೋ-ಹ-ಮಿವ
ನೈವೇ-ದ್ಯ-ಕ್ಕಿ-ಟ್ಟಿ-ರುವ
ನೊಂದ
ನೊಂದು
ನೊಗ
ನೊಗ-ಗಳು
ನೊಡನೆ
ನೊಡು-ವು-ದಕ್ಕೆ
ನೊದೆ
ನೊಬ್ಬ-ನನ್ನು
ನೊರೆ
ನೊರೆಯಂ
ನೊರೆ-ಯಂ-ತಿ-ರುವ
ನೊರೆ-ಯಂತೆ
ನೊರೆ-ಯನ್ನು
ನೊಳಕ್ಕೆ
ನೋಟ
ನೋಟ-ಕರ
ನೋಟಕ್ಕೆ
ನೋಟ-ಗಳನ್ನು
ನೋಟ-ಗ-ಳಿಗೆ
ನೋಟದ
ನೋಟ-ದ-ಲ್ಲಿಯೇ
ನೋಟ-ದಿಂದ
ನೋಟ-ವನ್ನು
ನೋಟ-ವೇನು
ನೋಡ
ನೋಡದ
ನೋಡ-ದಿ-ದ್ದರೂ
ನೋಡ-ದಿರು
ನೋಡ-ದಿ-ರು-ವುದನ್ನು
ನೋಡದೆ
ನೋಡಪ್ಪಾ
ನೋಡ-ಬಂ-ದ-ವ-ರೆಂದು
ನೋಡ-ಬಲ್ಲ
ನೋಡ-ಬ-ಹುದಾ
ನೋಡ-ಬ-ಹುದು
ನೋಡ-ಬಾ-ರದು
ನೋಡ-ಬಾ-ರದೆ
ನೋಡ-ಬೇಕು
ನೋಡ-ಬೇ-ಕೆಂ-ದಿದ್ದ
ನೋಡ-ಬೇ-ಕೆಂದು
ನೋಡ-ಬೇ-ಕೆಂಬ
ನೋಡಮ್ಮ
ನೋಡ-ಲಾ-ಗ-ದಿದ್ದ
ನೋಡ-ಲಾ-ರದೆ
ನೋಡ-ಲಿಲ್ಲ
ನೋಡಲು
ನೋಡ-ಲೆಂದು
ನೋಡ-ಹೊ-ರ-ಟರು
ನೋಡಿ
ನೋಡಿ-ಎಂದು
ನೋಡಿ-ಕೊಂಡ
ನೋಡಿ-ಕೊಂ-ಡರೆ
ನೋಡಿ-ಕೊಂಡು
ನೋಡಿ-ಕೊ-ಳ್ಳ-ಬೇಕು
ನೋಡಿ-ಕೊಳ್ಳಿ
ನೋಡಿ-ಕೊ-ಳ್ಳು-ತ್ತಾರೆ
ನೋಡಿ-ಕೊ-ಳ್ಳು-ತ್ತಿರಿ
ನೋಡಿ-ಕೊ-ಳ್ಳು-ತ್ತೇನೆ
ನೋಡಿ-ಕೊ-ಳ್ಳು-ವ-ವರು
ನೋಡಿತು
ನೋಡಿದ
ನೋಡಿ-ದಂತೆ
ನೋಡಿ-ದನು
ನೋಡಿ-ದರು
ನೋಡಿ-ದರೂ
ನೋಡಿ-ದರೆ
ನೋಡಿ-ದಳು
ನೋಡಿ-ದ-ವನೆ
ನೋಡಿ-ದ-ವ-ರಿಗೆ
ನೋಡಿ-ದ-ವರು
ನೋಡಿ-ದ-ವರೆಲ್ಲ
ನೋಡಿ-ದಾಗ
ನೋಡಿ-ದಾ-ಗ-ಲೆಲ್ಲ
ನೋಡಿ-ದಿರಾ
ನೋಡಿ-ದಿರೋ
ನೋಡಿದೆ
ನೋಡಿ-ದೆಯಾ
ನೋಡಿ-ದೆವು
ನೋಡಿ-ದ್ದೀನಿ
ನೋಡಿ-ದ್ದೇನೆ
ನೋಡಿ-ಬ-ರ-ಬೇ-ಕೆಂ-ದು-ಕೊಂ-ಡನು
ನೋಡಿ-ಬಿ-ಡು-ತ್ತೇನೆ
ನೋಡಿರಿ
ನೋಡು
ನೋಡುತ್ತ
ನೋಡು-ತ್ತಲೆ
ನೋಡುತ್ತಾ
ನೋಡು-ತ್ತಾನೆ
ನೋಡು-ತ್ತಾ-ನೆ-ತಾನು
ನೋಡು-ತ್ತಾ-ನೆ-ದಿ-ವ್ಯ-ಸುಂ-ದರ
ನೋಡು-ತ್ತಾ-ನೆ-ಶ್ರೀ-ಕೃಷ್ಣ
ನೋಡು-ತ್ತಾ-ರಂತೆ
ನೋಡು-ತ್ತಾರೆ
ನೋಡು-ತ್ತಾ-ರೆ-ಗೋ-ಪಾ-ಲ-ಬಾ-ಲ-ರೆಲ್ಲ
ನೋಡು-ತ್ತಾಳೆ
ನೋಡು-ತ್ತಾ-ಳೆ-ಮ-ಗು-ವಿನ
ನೋಡುತ್ತಿ
ನೋಡು-ತ್ತಿದ್ದ
ನೋಡು-ತ್ತಿ-ದ್ದಂತೆ
ನೋಡು-ತ್ತಿ-ದ್ದರು
ನೋಡು-ತ್ತಿ-ರ-ಲಿಲ್ಲ
ನೋಡು-ತ್ತಿ-ರಲು
ನೋಡು-ತ್ತಿ-ರು-ತ್ತದೆ
ನೋಡು-ತ್ತಿ-ರುವ
ನೋಡು-ತ್ತಿ-ರು-ವಂ-ತೆಯೆ
ನೋಡು-ತ್ತಿ-ರು-ವಂ-ತೆಯೇ
ನೋಡು-ತ್ತಿ-ರು-ವಿರೊ
ನೋಡು-ತ್ತೀರಿ
ನೋಡು-ತ್ತೇವೆ
ನೋಡು-ನೋ-ಡು-ತ್ತಿ-ದ್ದಂತೆ
ನೋಡು-ನೋ-ಡು-ತ್ತಿ-ರು-ವಂತೆ
ನೋಡುವ
ನೋಡು-ವ-ವರ
ನೋಡು-ವು-ದ-ಕ್ಕಾಗಿ
ನೋಡು-ವು-ದಕ್ಕೆ
ನೋಡು-ವು-ದಿಲ್ಲ
ನೋಡು-ವುದು
ನೋಡೋಣ
ನೋಡೋ-ಣ-ವೆಂದು
ನೋಯ-ದ್ವಶೇ
ನೋಯಿ-ಸಿದ
ನೋಯಿ-ಸಿ-ದನು
ನೋಯಿ-ಸಿದೆ
ನೋಯಿ-ಸು-ತ್ತಿ-ರು-ವೆ-ಯಲ್ಲಾ
ನೋಯು-ತ್ತವೋ
ನೋಯು-ವುದೋ
ನೋವನ್ನು
ನೋವ-ನ್ನೆಲ್ಲ
ನೋವಾ-ಗ-ದಂತೆ
ನೋವಾಗು
ನೋವಾ-ಗು-ವುದು
ನೋವಾ-ದರೆ
ನೋವು
ನ್ನವೊ
ನ್ನಾಗಿ
ನ್ನಾಚ-ರಿಸಿ
ನ್ನಾರಾ-ಯಣಃ
ನ್ನಾರಾ-ಯ-ಣನು
ನ್ನಿಟ್ಟರು
ನ್ನಿತ್ತ
ನ್ನೆಲ್ಲ
ನ್ನೆಲ್ಲಾ
ನ್ನೇಕೆ
ನ್ನೇರಿ
ನ್ನೊಡೆದು
ನ್ಮಾತ್ರ-ಗಳು
ನ್ಯಗ್ರೋ-ಧಕ
ನ್ಯಸ್ತ-ಶ-ಕ್ತ್ಯೂ-ರ್ಮಯೇ
ನ್ಯಸ್ತಾಂ-ಘ್ರಿ-ಪದ್ಮಃ
ನ್ಯಾಯ
ನ್ಯಾಯ-ವಲ್ಲ
ನ್ಯಾಯ-ವಾಗಿ
ನ್ಯಾಯ-ವಾ-ಗಿಯೂ
ನ್ಯಾಯ-ವೆಂದು
ನ್ಯಾಯ-ವೆ-ನಿ-ಸಿತು
ನ್ಯಾಸ
ಪ-ಪದ್ಮ
ಪಂಕಜ
ಪಂಕ-ಜ-ನಾ-ಭಾಯ
ಪಂಕ-ಜ-ನೇ-ತ್ರಾಯ
ಪಂಕ-ಜ-ಮಾ-ಲಿನೇ
ಪಂಕ-ಜಾಂ-ಘ್ರಯೇ
ಪಂಕ-ಜಾಂ-ಘ್ರಿಗೆ
ಪಂಚ
ಪಂಚಕ
ಪಂಚ-ಜನ
ಪಂಚ-ಜ-ನ-ನನ್ನು
ಪಂಚ-ಜ-ನ-ನೆಂಬ
ಪಂಚ-ಜನಿ
ಪಂಚತ
ಪಂಚ-ಪಾಂ-ಡ-ವ-ರನ್ನೂ
ಪಂಚ-ಪಾಂ-ಡ-ವ-ರಲ್ಲಿ
ಪಂಚ-ಪಾಂ-ಡ-ವರು
ಪಂಚ-ಪಾಂ-ಡ-ವರೂ
ಪಂಚ-ಪ್ರ-ಸ್ಥ-ವೆಂಬ
ಪಂಚ-ಪ್ರಾಣ
ಪಂಚ-ಪ್ರಾ-ಣ-ಗಳೇ
ಪಂಚ-ಪ್ರಾ-ಣ-ದಂ-ತಿದ್ದ
ಪಂಚ-ಭೂತ
ಪಂಚ-ಭೂ-ತ-ಗಳಿಂದ
ಪಂಚ-ಭೂ-ತ-ಗ-ಳಿಂ-ದಾದ
ಪಂಚ-ಭೂ-ತ-ಗ-ಳಿಗೂ
ಪಂಚ-ಭೂ-ತ-ಗಳು
ಪಂಚ-ಭೂ-ತ-ಗಳೂ
ಪಂಚ-ಭೂ-ತ-ಗ-ಳೆಂಬ
ಪಂಚಮ
ಪಂಚ-ಮ-ಹಾ-ಪಾ-ತ-ಕ-ಗಳಲ್ಲಿ
ಪಂಚ-ಮ-ಹಾ-ಪಾ-ತ-ಕ-ಗಳೂ
ಪಂಚ-ರಾ-ತ್ರ-ದಲ್ಲಿ
ಪಂಚ-ರಾ-ತ್ರ-ವಿ-ಧಿ-ಯಿಂದ
ಪಂಚ-ರಾ-ತ್ರಾ-ಗ-ಮ-ದಂತೆ
ಪಂಚ-ರಾ-ತ್ರಾ-ಗ-ಮ-ವನ್ನು
ಪಂಚ-ಲ-ಕ್ಷ-ಣ-ವನ್ನು
ಪಂಚ-ವಿ-ಷ-ಯ-ಗ-ಳೆಂಬ
ಪಂಚಾ
ಪಂಚೇಂ-ದ್ರಿ-ಯ-ಗ-ಳಿಂ-ದಲೂ
ಪಂಚೇಂ-ದ್ರಿ-ಯ-ಗಳು
ಪಂಜ-ರದ
ಪಂಜ-ರ-ದಲ್ಲಿ
ಪಂಜಿ
ಪಂಡಿತ
ಪಂಡಿ-ತ-ನಾದ
ಪಂಡಿ-ತ-ನೆಂಬ
ಪಂಡಿ-ತ-ನೊ-ಬ್ಬನು
ಪಂಡಿ-ತರ
ಪಂಡಿ-ತ-ರಾ-ದರೂ
ಪಂಥವೂ
ಪಕ-ಪಕ
ಪಕ-ಪ-ಕನೆ
ಪಕ್ಕ
ಪಕ್ಕದ
ಪಕ್ಕ-ದಲ್ಲಿ
ಪಕ್ಕ-ದ-ಲ್ಲಿಟ್ಟು
ಪಕ್ಕ-ದ-ಲ್ಲಿದ್ದ
ಪಕ್ಕ-ದ-ಲ್ಲಿ-ದ್ದ-ನೆಂದು
ಪಕ್ಕ-ದ-ಲ್ಲಿನ
ಪಕ್ಕ-ದ-ಲ್ಲಿಯೆ
ಪಕ್ಕ-ದ-ಲ್ಲಿಯೇ
ಪಕ್ಕ-ದ-ಲ್ಲಿ-ರು-ವ-ವ-ಳನ್ನು
ಪಕ್ಕ-ದ-ವ-ಳನ್ನು
ಪಕ್ಕೆಗೆ
ಪಕ್ವ-ವಾಗಿ
ಪಕ್ಷ
ಪಕ್ಷ-ದಲ್ಲಿ
ಪಕ್ಷ-ದವ
ಪಕ್ಷ-ಪಾ-ತ-ವಿ-ಲ್ಲದೆ
ಪಕ್ಷ-ಪಾ-ತಿ-ಯಾ-ಗಿ-ರು-ತ್ತಾನೆ
ಪಕ್ಷ-ಪಾ-ತಿ-ಯಾದ
ಪಕ್ಷ-ವನ್ನು
ಪಕ್ಷಿ
ಪಕ್ಷಿ-ಗಳು
ಪಕ್ಷಿ-ದಂ-ಪ-ತಿ-ಗಳು
ಪಕ್ಷೇ
ಪಗಡೆ
ಪಗ-ಡೆ-ಯಾಟ
ಪಗ-ಡೆ-ಯಾ-ಟ-ದಲ್ಲಿ
ಪಗ-ಡೆ-ಯಾ-ಡುತ್ತಾ
ಪಗ-ಡೆ-ಯೆಂ-ದರೆ
ಪಟು-ಪ-ರಾ-ಕ್ರಮಿ
ಪಟ್ಟ
ಪಟ್ಟ-ಕಟ್ಟಿ
ಪಟ್ಟ-ಕ-ಟ್ಟಿ-ದುದು
ಪಟ್ಟಕ್ಕೆ
ಪಟ್ಟ-ಕ್ಕೇ-ರಿ-ಸಿ-ದರು
ಪಟ್ಟ-ಗ-ಟ್ಟ-ಬೇ-ಕೆಂದೂ
ಪಟ್ಟ-ಗಟ್ಟಿ
ಪಟ್ಟ-ಗ-ಟ್ಟಿ-ದನು
ಪಟ್ಟಣ
ಪಟ್ಟ-ಣಕ್ಕೆ
ಪಟ್ಟ-ಣ-ಗಳನ್ನೂ
ಪಟ್ಟ-ಣ-ಗಳಲ್ಲಿ
ಪಟ್ಟ-ಣ-ಗ-ಳಿವೆ
ಪಟ್ಟ-ಣದ
ಪಟ್ಟ-ಣ-ದಲ್ಲಿ
ಪಟ್ಟ-ಣ-ದಿಂದ
ಪಟ್ಟ-ಣ-ವನ್ನು
ಪಟ್ಟ-ಣ-ವಾದ
ಪಟ್ಟ-ಣ-ವೆಲ್ಲಿ
ಪಟ್ಟ-ಣವೇ
ಪಟ್ಟ-ಣಿ-ಗರು
ಪಟ್ಟದ
ಪಟ್ಟನು
ಪಟ್ಟರು
ಪಟ್ಟರೆ
ಪಟ್ಟ-ವನ್ನು
ಪಟ್ಟ-ವೇ-ರು-ವನು
ಪಟ್ಟಾಭಿ
ಪಟ್ಟಾ-ಭಿ-ಷಿಕ್ತ
ಪಟ್ಟಾ-ಭಿ-ಷಿ-ಕ್ತ-ನಾಗಿ
ಪಟ್ಟಾ-ಭಿ-ಷಿ-ಕ್ತ-ನಾ-ದನು
ಪಟ್ಟಾ-ಭಿ-ಷೇಕ
ಪಟ್ಟಾ-ಭಿ-ಷೇ-ಕಕ್ಕೆ
ಪಟ್ಟಾ-ಭಿ-ಷೇ-ಕ-ವನ್ನು
ಪಟ್ಟಾ-ಭಿ-ಷೇ-ಕ-ವಾ-ಗುವ
ಪಟ್ಟಾ-ಭಿ-ಷೇ-ಕ-ವಾ-ಗು-ವು-ದೆಂದು
ಪಟ್ಟು
ಪಟ್ಟು-ದ-ರಿಂದ
ಪಟ್ಟೆ
ಪಟ್ಟೆಯ
ಪಟ್ಟೆ-ವ-ಸ್ತ್ರ-ಗಳನ್ನು
ಪಟ್ಟೆವು
ಪಠ-ತೀಹ
ಪಠಿಸಿ
ಪಠಿ-ಸಿ-ದು-ದಾಗಿ
ಪಡದೆ
ಪಡ-ಬಾ-ರದ
ಪಡ-ಬೇ-ಕಾ-ದು-ದಿಲ್ಲ
ಪಡ-ಬೇಡ
ಪಡ-ಬೇಡಿ
ಪಡ-ಲಿ-ಲ್ಲ-ವಂತೆ
ಪಡಿ-ನೆ-ರ-ಳಂತೆ
ಪಡಿ-ಸಲು
ಪಡಿಸಿ
ಪಡಿ-ಸಿ-ಕೊಂ-ಡನು
ಪಡಿ-ಸಿ-ಕೊ-ಳ್ಳ-ಹೊ-ರ-ಟರು
ಪಡಿ-ಸಿದ
ಪಡಿ-ಸಿ-ದನು
ಪಡಿ-ಸಿ-ದ-ಮೇಲೆ
ಪಡಿ-ಸೋಣ
ಪಡು-ತ್ತದೆ
ಪಡು-ತ್ತಿ-ದ್ದನು
ಪಡು-ತ್ತಿ-ರಲು
ಪಡು-ತ್ತಿರು
ಪಡು-ತ್ತೇವೆ
ಪಡು-ವಣ
ಪಡು-ವನು
ಪಡು-ವಳು
ಪಡೆ
ಪಡೆ-ಗ-ಳಿಗೂ
ಪಡೆದ
ಪಡೆ-ದಂತೆ
ಪಡೆ-ದ-ನಾ-ದರೂ
ಪಡೆ-ದನು
ಪಡೆ-ದ-ನೆಂದು
ಪಡೆ-ದ-ನೆಂ-ಬುದು
ಪಡೆ-ದನೋ
ಪಡೆ-ದ-ಮೇಲೆ
ಪಡೆ-ದ-ರಷ್ಟೆ
ಪಡೆ-ದರು
ಪಡೆ-ದರೂ
ಪಡೆ-ದ-ಳೆಂ-ಬು-ದಕ್ಕೆ
ಪಡೆ-ದವ
ಪಡೆ-ದ-ವ-ನಾ-ಗಿ-ದ್ದನು
ಪಡೆ-ದ-ವನು
ಪಡೆ-ದ-ವನೆ
ಪಡೆ-ದ-ವರು
ಪಡೆ-ದವು
ಪಡೆ-ದ-ವೆಂದು
ಪಡೆ-ದಿದ್ದ
ಪಡೆ-ದಿ-ದ್ದನು
ಪಡೆ-ದಿ-ದ್ದಳು
ಪಡೆ-ದಿ-ದ್ದ-ವನು
ಪಡೆ-ದಿ-ದ್ದಾತ
ಪಡೆ-ದಿ-ದ್ದಾನೆ
ಪಡೆ-ದಿ-ದ್ದಾರೆ
ಪಡೆ-ದಿ-ದ್ದು-ದ-ರಿಂದ
ಪಡೆ-ದಿ-ದ್ದೇನೆ
ಪಡೆ-ದಿ-ರ-ಬ-ಹು-ದೆಂದು
ಪಡೆ-ದಿ-ರುವ
ಪಡೆ-ದಿ-ರುವೆ
ಪಡೆ-ದಿವೆ
ಪಡೆದು
ಪಡೆ-ದು-ದಾ-ಯಿತು
ಪಡೆ-ದುದು
ಪಡೆ-ದೆವು
ಪಡೆಯ
ಪಡೆ-ಯ-ಬ-ಹು-ದಾ-ದರೂ
ಪಡೆ-ಯ-ಬ-ಹುದು
ಪಡೆ-ಯ-ಬ-ಹು-ದೆ-ನ್ನು-ತ್ತದೆ
ಪಡೆ-ಯ-ಬೇ-ಕಾ-ದರೆ
ಪಡೆ-ಯ-ಬೇ-ಕಾ-ದು-ದು-ರಿಂದ
ಪಡೆ-ಯ-ಬೇಕು
ಪಡೆ-ಯ-ಬೇ-ಕೆಂದು
ಪಡೆ-ಯ-ಬೇ-ಕೆಂಬ
ಪಡೆ-ಯ-ಬೇ-ಕೆಂ-ಬುದು
ಪಡೆ-ಯ-ಲಾ-ಗದ
ಪಡೆ-ಯ-ಲಾರ
ಪಡೆ-ಯಲಿ
ಪಡೆ-ಯ-ಲಿಲ್ಲ
ಪಡೆ-ಯಲು
ಪಡೆ-ಯ-ಲೆಂ-ಬುದೆ
ಪಡೆ-ಯ-ಹೊ-ರ-ಟ-ವ-ನಿಗೆ
ಪಡೆ-ಯಿತು
ಪಡೆ-ಯು-ತ್ತಾನೆ
ಪಡೆ-ಯು-ತ್ತಾರೆ
ಪಡೆ-ಯು-ತ್ತಿ-ದ್ದಾನೆ
ಪಡೆ-ಯು-ತ್ತೇನೆ
ಪಡೆ-ಯುವ
ಪಡೆ-ಯು-ವಂ-ತೆಯೆ
ಪಡೆ-ಯು-ವನು
ಪಡೆ-ಯು-ವರು
ಪಡೆ-ಯು-ವ-ವ-ನಾಗು
ಪಡೆ-ಯು-ವಿರಿ
ಪಡೆ-ಯು-ವು-ದ-ಕ್ಕಾಗಿ
ಪಡೆ-ಯು-ವು-ದ-ಕ್ಕಿಂ-ತಲೂ
ಪಡೆ-ಯು-ವು-ದಕ್ಕೆ
ಪಡೆ-ಯು-ವು-ದ-ರ-ಲ್ಲಿಯೇ
ಪಡೆ-ಯು-ವು-ದಿಲ್ಲ
ಪಡೆ-ಯು-ವುದು
ಪಡೆ-ಯು-ವುದೇ
ಪಡೆ-ಯುವೆ
ಪಡೆವ
ಪಣ
ಪಣ-ವಾಗಿ
ಪಣ-ವಾ-ಗಿ-ಟ್ಟಿದ್ದ
ಪತಂಗ
ಪತಂ-ಗ-ವನ್ನು
ಪತಂ-ಗವು
ಪತಂ-ಜ-ಲಿ-ಗಳಲ್ಲಿ
ಪತಂ-ಜ-ಲಿಯು
ಪತ-ಗೇಂ-ದ್ರ-ಪೃಷ್ಠೇ
ಪತಯೇ
ಪತಾ-ಕಿ-ಗಳು
ಪತಿ
ಪತಿ-ಗಳನ್ನು
ಪತಿ-ಗಳು
ಪತಿ-ತನೂ
ಪತಿ-ದೇ-ವ-ರನ್ನು
ಪತಿ-ಪು-ತ್ರ-ರನ್ನೂ
ಪತಿ-ಪು-ತ್ರಾದಿ
ಪತಿ-ಪು-ತ್ರಾ-ದಿ-ಗಳನ್ನೂ
ಪತಿ-ಭ-ಕ್ತಿ-ಯಿಂದ
ಪತಿ-ಭಿ-ಕ್ಷೆ-ಯನ್ನು
ಪತಿಯ
ಪತಿ-ಯನ್ನೂ
ಪತಿ-ಯಾ-ಗ-ಬೇ-ಕೆಂದು
ಪತಿ-ಯಾ-ಗಲಿ
ಪತಿ-ಯಾಗಿ
ಪತಿ-ಯಾ-ಗಿ-ದ್ದ-ವನೇ
ಪತಿ-ಯಾ-ಗಿ-ರು-ವಾಗ
ಪತಿ-ಯಾ-ಗು-ವಂತೆ
ಪತಿ-ಯಾದ
ಪತಿ-ಯಿಂದ
ಪತಿಯು
ಪತಿ-ಯೆಂದು
ಪತಿ-ಯೊ-ಡನೆ
ಪತಿ-ವ್ರತೆ
ಪತಿ-ವ್ರ-ತೆ-ಯ-ರಾದ
ಪತಿ-ವ್ರ-ತೆ-ಯ-ರಿಲ್ಲ
ಪತಿ-ವ್ರ-ತೆ-ಯಾದ
ಪತಿ-ವ್ರ-ತೆ-ಯಾ-ದ-ವಳು
ಪತಿ-ಸೇ-ವೆ-ಯನ್ನು
ಪತ್ತೆ
ಪತ್ತೆ-ಹಚ್ಚಿ
ಪತ್ತೆ-ಹಚ್ಚು
ಪತ್ನ
ಪತ್ನಿ
ಪತ್ನಿ-ಪು-ತ್ರಾದಿ
ಪತ್ನಿ-ಯ-ರ-ನ್ನಾಗಿ
ಪತ್ನಿ-ಯ-ರನ್ನೂ
ಪತ್ನಿ-ಯ-ರಲ್ಲಿ
ಪತ್ನಿ-ಯ-ರಾ-ಗು-ವರು
ಪತ್ನಿ-ಯ-ರಾದ
ಪತ್ನಿ-ಯರೂ
ಪತ್ನಿ-ಯ-ರೊ-ಡನೆ
ಪತ್ನಿ-ಯಾಗಿ
ಪತ್ನಿ-ಯಾ-ಗು-ತ್ತಾಳೆ
ಪತ್ನಿ-ಯಾ-ಗು-ವು-ದ-ರಲ್ಲಿ
ಪತ್ನಿ-ಯಾದ
ಪತ್ನಿಯೂ
ಪತ್ನಿ-ಯೊ-ಡನೆ
ಪತ್ಯ-ನ್ನ-ಲೋಕಾ
ಪತ್ರ
ಪತ್ರ-ದಲ್ಲಿ
ಪತ್ರ-ವನ್ನು
ಪಥಿ
ಪದ-ತ-ಲದ
ಪದ-ತ-ಲ-ದಲ್ಲಿ
ಪದ-ತ-ಲ-ದ-ಲ್ಲಿಟ್ಟು
ಪದ-ತ-ಲ-ವನ್ನು
ಪದವಿ
ಪದ-ವಿ-ಗಾಗಿ
ಪದ-ವಿಗೆ
ಪದ-ವಿ-ಗೇ-ರಿದ
ಪದ-ವಿ-ಯನ್ನು
ಪದ-ವಿ-ಯನ್ನೆ
ಪದ-ವಿ-ಯಲ್ಲಿ
ಪದ-ವಿಯೆ
ಪದಾನಿ
ಪದಾರ್ಥ
ಪದಾ-ರ್ಥ-ಗಳನ್ನು
ಪದಾ-ರ್ಥ-ಗ-ಳನ್ನೇ
ಪದಾ-ರ್ಥ-ವೆಂ-ಬಂತೆ
ಪದ್ಧತಿ
ಪದ್ಧ-ತಿ-ಯಲ್ಲಿ
ಪದ್ಮ
ಪದ್ಮ-ಗಳು
ಪದ್ಮ-ಧಾ-ರಿ-ಯಾಗಿ
ಪದ್ಮ-ನಾಭ
ಪದ್ಮ-ನಾಭಃ
ಪದ್ಮ-ನಾ-ಭ-ನಿಗೆ
ಪದ್ಮ-ನಾ-ಭನು
ಪದ್ಮ-ಪು-ರಾಣ
ಪದ್ಮ-ಪು-ರಾ-ಣದ
ಪದ್ಮ-ರಾ-ಗದ
ಪದ್ಮ-ರೇ-ಖೆ-ಯನ್ನೂ
ಪದ್ಮಾ
ಪದ್ಮಾ-ವ-ತಿ-ಯೆಂ-ಬಲ್ಲಿ
ಪದ್ಮಾ-ಸ-ನ-ದಲ್ಲಿ
ಪನ್ನೀ-ರನ್ನು
ಪನ್ನೀ-ರಿ-ನಂತೆ
ಪಯಣ
ಪಯ-ಣದ
ಪಯ-ಣ-ವನ್ನು
ಪಯೋ
ಪಯೋ-ವ್ರ-ತ-ವನ್ನು
ಪರ
ಪರಂ-ಧಾ-ಮಕ್ಕೆ
ಪರಂ-ಪ-ರೆಯ
ಪರಂ-ಪ-ರೆ-ಯನ್ನೂ
ಪರಂ-ಪ-ರೆ-ಯಲ್ಲಿ
ಪರಂ-ಪ-ರೆ-ಯಾಗಿ
ಪರಂ-ಪ-ರೆ-ಯಿಂದ
ಪರ-ಕಾ-ಯ-ಪ್ರ-ವೇಶ
ಪರ-ಗ-ಳೆ-ರ-ಡರ
ಪರ-ಜ-ನ-ರೆಂಬ
ಪರ-ತಂತ್ರ
ಪರ-ತಾ-ಽನಾ-ತ್ಮಾಯ
ಪರ-ದಲ್ಲಿ
ಪರ-ದೆ-ಯಂ-ತಿವೆ
ಪರ-ದೈ-ವ-ವೆಂದು
ಪರ-ದೈ-ವ-ವೆಂಬ
ಪರ-ನಾಗಿ
ಪರ-ನಾದ
ಪರ-ಪು-ರು-ಷ-ನನ್ನು
ಪರ-ಪು-ರು-ಷ-ನಿಂದ
ಪರ-ಪ್ರ-ಕಾಶ
ಪರ-ಬ್ರಹ್ಮ
ಪರ-ಬ್ರ-ಹ್ಮದ
ಪರ-ಬ್ರ-ಹ್ಮನ
ಪರ-ಬ್ರ-ಹ್ಮ-ನಲ್ಲಿ
ಪರ-ಬ್ರ-ಹ್ಮ-ನಾ-ಗು-ವುದು
ಪರ-ಬ್ರ-ಹ್ಮ-ನಾದ
ಪರ-ಬ್ರ-ಹ್ಮ-ನಿಗೆ
ಪರ-ಬ್ರ-ಹ್ಮನೆ
ಪರ-ಬ್ರ-ಹ್ಮ-ನೆಂ-ಬು-ವನು
ಪರ-ಬ್ರ-ಹ್ಮನೇ
ಪರ-ಬ್ರ-ಹ್ಮ-ವಸ್ತು
ಪರ-ಬ್ರ-ಹ್ಮ-ವ-ಸ್ತು-ವಿಗೆ
ಪರ-ಬ್ರ-ಹ್ಮ-ವ-ಸ್ತುವೇ
ಪರ-ಬ್ರ-ಹ್ಮ-ವೆಂದು
ಪರ-ಬ್ರ-ಹ್ಮವೇ
ಪರ-ಬ್ರ-ಹ್ಮ-ಸ್ವ-ರೂ-ಪ-ವನ್ನು
ಪರ-ಬ್ರ-ಹ್ಮ-ಸ್ವ-ರೂ-ಪ-ವೆಂದು
ಪರ-ಬ್ರ-ಹ್ಮ-ಸ್ವ-ರೂ-ಪಿ-ಯಾದ
ಪರ-ಬ್ರ-ಹ್ಮ-ಸ್ವ-ರೂ-ಪಿಯೇ
ಪರಮ
ಪರಮಂ
ಪರ-ಮ-ಕಾ-ರು-ಣಿ-ಕ-ನಾದ
ಪರ-ಮ-ಗು-ರುವೇ
ಪರ-ಮ-ಜ್ಞಾನ
ಪರ-ಮ-ಜ್ಞಾ-ನಿ-ಯಾದ
ಪರ-ಮ-ಜ್ಞಾ-ನಿ-ಯಾ-ದರೂ
ಪರ-ಮ-ದ್ವೇಷಿ
ಪರ-ಮ-ಪ-ದಕ್ಕೆ
ಪರ-ಮ-ಪ-ದವೇ
ಪರ-ಮ-ಪಾ-ವ-ನ-ವಾ-ಗಿದೆ
ಪರ-ಮ-ಪು-ರು-ಷನ
ಪರ-ಮ-ಪು-ರು-ಷ-ನನ್ನು
ಪರ-ಮ-ಪು-ರು-ಷ-ನಾದ
ಪರ-ಮ-ಪು-ರು-ಷನು
ಪರ-ಮ-ಪು-ರು-ಷನೇ
ಪರ-ಮ-ಪು-ರು-ಷ-ರೆಂದು
ಪರ-ಮ-ಪೂಜ್ಯ
ಪರ-ಮ-ಪ್ರಿಯ
ಪರ-ಮ-ಪ್ರಿ-ಯ-ನಾದ
ಪರ-ಮ-ಪ್ರೇ-ಮ-ರೂಪ
ಪರ-ಮ-ಭಕ್ತ
ಪರ-ಮ-ಭ-ಕ್ತ-ನಾ-ಗಿ-ದ್ದಾನೆ
ಪರ-ಮ-ಭಾ-ಗ-ವ-ತ-ನಾದ
ಪರ-ಮ-ಭಾ-ಗ-ವ-ತ-ರೆಂದು
ಪರ-ಮ-ಮಿ-ತ್ರ-ರಾದ
ಪರ-ಮ-ಯೋ-ಗಿ-ಗ-ಳಾದ
ಪರ-ಮ-ಯೋ-ಗೀ-ಶ್ವರ
ಪರ-ಮ-ವೈ-ರಿ-ಯಾ-ದನು
ಪರ-ಮ-ಶಾಂ-ತ-ನಾ-ಗಿಯೂ
ಪರ-ಮ-ಸಂ-ತೋ-ಷ-ವಾ-ಯಿತು
ಪರ-ಮ-ಸಾ-ಧು-ವಾದ
ಪರ-ಮ-ಸುಖ
ಪರ-ಮ-ಸು-ಖ-ದಿಂದ
ಪರ-ಮ-ಹಂಸ
ಪರ-ಮ-ಹಂ-ಸ-ನಾದ
ಪರ-ಮ-ಹಂ-ಸ-ರಿಗೆ
ಪರ-ಮ-ಹಂ-ಸರು
ಪರ-ಮ-ಹಂ-ಸಾ-ಶ್ರ-ಮ-ದ-ಲ್ಲಿ-ರು-ವ-ವ-ರಿಗೆ
ಪರ-ಮಾತ್ಮ
ಪರ-ಮಾ-ತ್ಮನ
ಪರ-ಮಾ-ತ್ಮ-ನಂತೆ
ಪರ-ಮಾ-ತ್ಮ-ನನ್ನು
ಪರ-ಮಾ-ತ್ಮ-ನ-ಲ್ಲದ
ಪರ-ಮಾ-ತ್ಮ-ನಲ್ಲಿ
ಪರ-ಮಾ-ತ್ಮ-ನ-ಲ್ಲಿಗೆ
ಪರ-ಮಾ-ತ್ಮ-ನಾದ
ಪರ-ಮಾ-ತ್ಮ-ನಿಗೆ
ಪರ-ಮಾ-ತ್ಮ-ನಿದ್ದ
ಪರ-ಮಾ-ತ್ಮನು
ಪರ-ಮಾ-ತ್ಮನೆ
ಪರ-ಮಾ-ತ್ಮ-ನೆಂದು
ಪರ-ಮಾ-ತ್ಮ-ನೆಂದೇ
ಪರ-ಮಾ-ತ್ಮ-ನೆಂ-ಬುದು
ಪರ-ಮಾ-ತ್ಮ-ನೊ-ಬ್ಬನೆ
ಪರ-ಮಾ-ತ್ಮ-ನೊ-ಬ್ಬನೇ
ಪರ-ಮಾ-ತ್ಮ-ಪ್ರಾಪ್ತಿ
ಪರ-ಮಾತ್ಮಾ
ಪರ-ಮಾ-ದ-ರ್ಶ-ವನ್ನು
ಪರ-ಮಾ-ನಂದ
ಪರ-ಮಾ-ನಂ-ದ-ಮೂ-ರ್ತಯೇ
ಪರ-ಮಾ-ನಂ-ದ-ವಾ-ಗಿ-ರು-ತ್ತದೆ
ಪರ-ಮಾ-ನಂ-ದ-ವಾ-ಯಿತು
ಪರ-ಮಾ-ಶ್ಚರ್ಯ
ಪರ-ಮಾ-ಶ್ಚ-ರ್ಯ-ವಾ-ಯಿತು
ಪರ-ಮಾ-ಷ್ಟ-ಕ-ಮೇ-ತ-ದ-ನ-ನ್ಯ-ಮ-ತಿಃ
ಪರಮೇ
ಪರ-ಮೇ-ಶ್ವರ
ಪರ-ಮೇ-ಶ್ವ-ರನ
ಪರ-ಮೇ-ಶ್ವ-ರ-ನಂತೆ
ಪರ-ಮೇ-ಶ್ವ-ರ-ನನ್ನು
ಪರ-ಮೇ-ಶ್ವ-ರ-ನಿಗೆ
ಪರ-ಮೇ-ಶ್ವ-ರನು
ಪರ-ಮೇ-ಶ್ವ-ರನೂ
ಪರ-ಮೇ-ಶ್ವ-ರನೆ
ಪರ-ಮೇ-ಶ್ವ-ರ-ನೆಂದೆ
ಪರ-ಮೇ-ಶ್ವ-ರನೇ
ಪರ-ಮೇ-ಶ್ವ-ರ-ನೊ-ಡನೆ
ಪರ-ಮೇ-ಶ್ವ-ರ-ರನ್ನು
ಪರ-ಮೇ-ಶ್ವ-ರ-ರೆಂದು
ಪರ-ಮೇ-ಶ್ವರಾ
ಪರ-ಮೇ-ಷ್ಠಿನ್
ಪರಮ್
ಪರರ
ಪರ-ರ-ದೆಂಬ
ಪರ-ರಿ-ಗಾ-ಗಿಯೇ
ಪರ-ಲೋ-ಕಕ್ಕೆ
ಪರ-ಲೋ-ಕ-ಗ-ಳೆ-ರಡೂ
ಪರ-ಲೋ-ಕ-ವನ್ನೂ
ಪರ-ವನ್ನು
ಪರ-ವಶ
ಪರ-ವ-ಶ-ನಾಗಿ
ಪರ-ವ-ಶ-ರಾ-ಗ-ದ-ವ-ರಾರು
ಪರ-ವ-ಶ-ರಾ-ಗು-ತ್ತಾರೆ
ಪರ-ವ-ಶ-ಳಾ-ಗಿ-ರ-ಬೇ-ಕೆ-ನ್ನಿ-ಸಿತು
ಪರ-ವ-ಶ-ವಾಗಿ
ಪರ-ವಾಗಿ
ಪರ-ವಾ-ಗಿ-ರು-ವರು
ಪರ-ಶ-ರಾ-ಮನ
ಪರ-ಶಿವ
ಪರ-ಶಿ-ವನ
ಪರ-ಶಿ-ವ-ನನ್ನು
ಪರ-ಶಿ-ವ-ನನ್ನೂ
ಪರ-ಶಿ-ವ-ನಿಗೆ
ಪರ-ಶಿ-ವನು
ಪರ-ಶಿ-ವ-ನೊ-ಬ್ಬನೆ
ಪರ-ಶಿವೆ
ಪರ-ಶಿ-ವೆ-ಯೊ-ಡನೆ
ಪರಶು
ಪರ-ಶು-ರಾಮ
ಪರ-ಶು-ರಾ-ಮ-ನಂತೆ
ಪರ-ಶು-ರಾ-ಮ-ನನ್ನು
ಪರ-ಶು-ರಾ-ಮ-ನಾಗಿ
ಪರ-ಶು-ರಾ-ಮ-ನಿಂದ
ಪರ-ಶು-ರಾ-ಮ-ನಿಗೆ
ಪರ-ಶು-ರಾ-ಮನು
ಪರ-ಶು-ರಾ-ಮನೂ
ಪರ-ಶು-ರಾ-ಮ-ನೊ-ಡನೆ
ಪರಸ್ತ್ರೀ
ಪರ-ಸ್ತ್ರೀ-ಯನ್ನು
ಪರ-ಸ್ತ್ರೀ-ಯ-ರೊ-ಡನೆ
ಪರ-ಸ್ಪರ
ಪರಸ್ಯ
ಪರಾ-ಕಾಷ್ಠ
ಪರಾಕು
ಪರಾ-ಕ್ರ-ಮಕ್ಕೆ
ಪರಾ-ಕ್ರ-ಮ-ಗಳನ್ನು
ಪರಾ-ಕ್ರ-ಮ-ದಿಂದ
ಪರಾ-ಕ್ರ-ಮ-ವನ್ನು
ಪರಾ-ಕ್ರ-ಮ-ವಾ-ಗಲಿ
ಪರಾ-ಕ್ರ-ಮ-ಶಾ-ಲಿ-ಗಳು
ಪರಾ-ಕ್ರಮಿ
ಪರಾ-ಕ್ರ-ಮಿ-ಗ-ಳಿಗೆ
ಪರಾ-ಕ್ರ-ಮಿ-ಗಳು
ಪರಾ-ಕ್ರ-ಮಿ-ಯಾ-ಗಿ-ರು-ವಷ್ಟೆ
ಪರಾ-ಕ್ರ-ಮಿ-ಯಾದ
ಪರಾ-ಕ್ರ-ಮಿ-ಯೆ-ನಿ-ಸಿ-ದ್ದನು
ಪರಾಗ
ಪರಾ-ತ್ಪರ
ಪರಾ-ತ್ಪ-ರ-ನಾದ
ಪರಾ-ತ್ಪ-ರ-ಮೂರ್ತಿ
ಪರಾ-ತ್ಪ-ರ-ವ-ಸ್ತು-ವಾದ
ಪರಾ-ಭಕ್ತಿ
ಪರಾ-ಮಾ-ತ್ಮ-ನನ್ನು
ಪರಾ-ಶರ
ಪರಾ-ಶ-ರ-ಸ-ತ್ಯ-ವ-ತಿ-ಯರ
ಪರಿ
ಪರಿ-ಕ-ಲ್ಪಿ-ತ-ಸ-ರ್ವ-ಕಲಂ
ಪರಿ-ಗ-ಣಿ-ಸದೆ
ಪರಿ-ಗ-ಣಿ-ಸಿ-ರ-ಬೇಕು
ಪರಿ-ಗ್ರ-ಹಿ-ಸಿ-ದನು
ಪರಿ-ಗ್ರ-ಹಿ-ಸು-ತ್ತಿ-ದ್ದ-ವನು
ಪರಿ-ಘ-ವನ್ನು
ಪರಿ-ಚಯ
ಪರಿ-ಚ-ಯ-ಗಳನ್ನು
ಪರಿ-ಚ-ಯ-ದ-ವರಿ
ಪರಿ-ಚ-ಯ-ದ-ವರೆ-ಲ್ಲರ
ಪರಿ-ಚ-ಯ-ವನ್ನು
ಪರಿ-ಚ-ಯ-ವಾ-ಗು-ವುದು
ಪರಿ-ಚ-ಯ-ವು-ಳ್ಳ-ವರು
ಪರಿ-ಚ-ರತಿ
ಪರಿ-ಚಾ-ರ-ಕರು
ಪರಿ-ಚಿ-ತ-ರಾದ
ಪರಿ-ಚಿ-ತ-ವಾ-ದು-ದೆ-ನ್ನಿ-ಸಿತು
ಪರಿ-ಣತ
ಪರಿ-ಣ-ತ-ನಾ-ಗ-ಲಿ-ಲ್ಲವೆ
ಪರಿ-ಣಮಿ
ಪರಿ-ಣ-ಮಿಸಿ
ಪರಿ-ಣ-ಮಿ-ಸಿತು
ಪರಿ-ಣ-ಮಿ-ಸು-ತ್ತದೆ
ಪರಿ-ಣಾಮ
ಪರಿ-ಣಾ-ಮಕ್ಕೂ
ಪರಿ-ತ-ಪಿ-ಸಿತು
ಪರಿ-ತ-ಪಿ-ಸಿ-ದನು
ಪರಿ-ತ-ಪಿ-ಸು-ತ್ತಿ-ದ್ದರು
ಪರಿ-ತ-ಪಿ-ಸು-ತ್ತಿ-ದ್ದಾರೆ
ಪರಿ-ತ-ಪಿ-ಸು-ತ್ತಿ-ರು-ವುದನ್ನು
ಪರಿ-ತು-ಷ್ಟ-ರ-ಮಾ-ಹೃ-ದಯಂ
ಪರಿ-ತ್ಯ-ಜಿಸು
ಪರಿ-ತ್ಯಾಗ
ಪರಿ-ತ್ಯಾ-ಗ-ವ-ನ್ನು-ಸ-ನ್ಯಾ-ಸ-ವ-ನ್ನು-ಬೋ-ಧಿ-ಸು-ತ್ತಿ-ರುವೆ
ಪರಿ-ಪರಿ
ಪರಿ-ಪ-ರಿಯ
ಪರಿ-ಪ-ರಿ-ಯಾಗಿ
ಪರಿ-ಪ-ರಿ-ಯಾದ
ಪರಿ-ಪಾಠಿ
ಪರಿ-ಪಾ-ಲನೆ
ಪರಿ-ಪಾ-ಲಿ-ಸು-ತ್ತಿ-ದ್ದರು
ಪರಿ-ಪೂ-ರಿ-ತ-ಸ-ರ್ವ-ದಿಶಂ
ಪರಿ-ಪೂರ್ಣ
ಪರಿ-ಪೂ-ರ್ಣ-ನಾದ
ಪರಿ-ಪೂ-ರ್ಣನೂ
ಪರಿ-ಬೃ-ಥ-ನಿ-ಕರ
ಪರಿ-ಭಾ-ವಿಸಿ
ಪರಿ-ಮ-ಳ-ಯು-ಕ್ತ-ವಾ-ಗಿ-ರು-ತ್ತದೆ
ಪರಿ-ಮಿತ
ಪರಿ-ವ-ರ್ತ-ನೆ-ಯನ್ನು
ಪರಿ-ವಾರ
ಪರಿ-ವಾ-ರಕ್ಕೂ
ಪರಿ-ವಾ-ರಕ್ಕೆ
ಪರಿ-ವಾ-ರದ
ಪರಿ-ವಾ-ರ-ದ-ವ-ರನ್ನೂ
ಪರಿ-ವಾ-ರ-ದ-ವ-ರ-ನ್ನೆಲ್ಲ
ಪರಿ-ವಾ-ರ-ದ-ವರು
ಪರಿ-ವಾ-ರ-ದ-ವರೂ
ಪರಿ-ವಾ-ರ-ದೊ-ಡನೆ
ಪರಿ-ವಾ-ರ-ವನ್ನೂ
ಪರಿ-ವಾ-ರವು
ಪರಿ-ವಾ-ರವೂ
ಪರಿವೆ
ಪರಿ-ವೆಯೆ
ಪರಿ-ವೇಷ
ಪರಿ-ವ್ರಾ-ಜಕ
ಪರಿ-ಶಿಷ್ಟ
ಪರಿ-ಶಿ-ಷ್ಟ-ಗಳು
ಪರಿ-ಶಿ-ಷ್ಟ-ಭ್ರ-ಮ-ರ-ಗೀತ
ಪರಿ-ಶಿ-ಷ್ಟ-ವಾಗಿ
ಪರಿ-ಶೀ-ಲಿ-ಸ-ಬ-ಹು-ದು-ಚಾರಿ
ಪರಿ-ಶೀ-ಲಿಸಿ
ಪರಿ-ಶುದ್ಧ
ಪರಿ-ಶು-ದ್ಧ-ನಾಗಿ
ಪರಿ-ಶು-ದ್ಧ-ವಾ-ಗಿ-ಲ್ಲವೊ
ಪರಿ-ಶು-ದ್ಧ-ವಾದ
ಪರಿ-ಶು-ದ್ಧ-ವಾ-ದಾಗ
ಪರಿ-ಸ್ಥಿತಿ
ಪರಿ-ಹ-ರಿ-ಸು-ಎಂಬ
ಪರಿ-ಹ-ರಿ-ಸುತ್ತಾ
ಪರಿ-ಹ-ರಿ-ಸುವ
ಪರಿ-ಹ-ರಿ-ಸು-ವುದು
ಪರಿ-ಹಾರ
ಪರಿ-ಹಾ-ರ-ವನ್ನೂ
ಪರಿ-ಹಾ-ರ-ವಾ-ಗು-ತ್ತವೆ
ಪರಿ-ಹಾ-ರ-ವಾ-ದಂತಾ
ಪರಿ-ಹಾ-ರ-ವಾ-ಯಿತು
ಪರಿ-ಹಾ-ರ-ವೆಂ-ದಿಗೂ
ಪರೀಕ್ಷಿ
ಪರೀ-ಕ್ಷಿತ
ಪರೀ-ಕ್ಷಿ-ತ-ಕು-ಮಾ-ರನು
ಪರೀ-ಕ್ಷಿ-ತನ
ಪರೀ-ಕ್ಷಿ-ತ-ನನ್ನು
ಪರೀ-ಕ್ಷಿ-ತ-ನಿಗೂ
ಪರೀ-ಕ್ಷಿ-ತ-ನಿಗೆ
ಪರೀ-ಕ್ಷಿ-ತನು
ಪರೀ-ಕ್ಷಿ-ದ್ರಾಜ
ಪರೀ-ಕ್ಷಿ-ದ್ರಾ-ಜನ
ಪರೀ-ಕ್ಷಿ-ದ್ರಾ-ಜ-ನನ್ನು
ಪರೀ-ಕ್ಷಿ-ದ್ರಾ-ಜ-ನನ್ನೂ
ಪರೀ-ಕ್ಷಿ-ದ್ರಾ-ಜ-ನಿಗೆ
ಪರೀ-ಕ್ಷಿ-ದ್ರಾ-ಜನು
ಪರೀ-ಕ್ಷಿ-ದ್ರಾ-ಜ-ನೆಂದೇ
ಪರೀ-ಕ್ಷಿ-ಸ-ಬೇ-ಕೆಂಬ
ಪರೀ-ಕ್ಷಿಸು
ಪರೀ-ಕ್ಷಿ-ಸು-ವು-ದ-ಕ್ಕಾಗಿ
ಪರೀ-ಕ್ಷೀ-ತ-ನಿಗೆ
ಪರೀ-ಕ್ಷೆ-ಮಾಡಿ
ಪರೀ-ಕ್ಷೆ-ಯಲ್ಲಿ
ಪರು-ಶು-ರಾ-ಮನು
ಪರೆ
ಪರೆ-ಯನ್ನು
ಪರೇಂ-ಗಿ-ತ-ಜ್ಞರು
ಪರೋಕ್ಷ
ಪರೋ-ಕ್ಷ-ವಾ-ಗಿ-ರುವ
ಪರೋ-ಪ-ಕಾರ
ಪರೋ-ಪ-ಕಾ-ರ-ಕ್ಕಾಗಿ
ಪರೋ-ಪ-ಕಾ-ರ-ಕ್ಕಾ-ಗಿಯೆ
ಪರೋ-ಪ-ಕಾರಿ
ಪರೋ-ಪ-ಕಾ-ರಿ-ಗ-ಳಾದ
ಪರೋ-ಪ-ಕಾ-ರಿ-ಯಾ-ದ-ವನು
ಪರೋ-ರಜಃ
ಪರೋ-ಹಿ-ತ-ನಾ-ಗಿ-ದ್ದನು
ಪರೋ-ಹಿ-ತ-ರಾದ
ಪರ್ಣ-ಶಾ-ಲೆ-ಯಿಂದ
ಪರ್ಣ-ಶಾ-ಲೆ-ಯೊಂ-ದನ್ನು
ಪರ್ಯ-ವ-ಸಾ-ನ-ವಾ-ಗು-ತ್ತದೆ
ಪರ್ವ-ಕಾ-ಲ-ದಲ್ಲಿ
ಪರ್ವತ
ಪರ್ವ-ತಕ್ಕೆ
ಪರ್ವ-ತ-ಗಳನ್ನೂ
ಪರ್ವ-ತ-ಗ-ಳಿವೆ
ಪರ್ವ-ತ-ಗಳು
ಪರ್ವ-ತ-ಗಳೂ
ಪರ್ವ-ತ-ಗು-ಹೆ-ಯನ್ನು
ಪರ್ವ-ತದ
ಪರ್ವ-ತ-ದಂತೆ
ಪರ್ವ-ತ-ದಲ್ಲಿ
ಪರ್ವ-ತ-ರಾಜ
ಪರ್ವ-ತ-ವನ್ನು
ಪರ್ವ-ತ-ವನ್ನೆ
ಪರ್ವ-ತ-ವಿತ್ತು
ಪರ್ವ-ತ-ವಿದೆ
ಪರ್ವ-ತವು
ಪರ್ವ-ತ-ಶಿ-ಖ-ರ-ದಲ್ಲಿ
ಪಲಾ-ಯನ
ಪಲ್ಲಕ್ಕಿ
ಪಲ್ಲ-ಕ್ಕಿ-ಗಳೂ
ಪಲ್ಲ-ಕ್ಕಿಯ
ಪಲ್ಲ-ಕ್ಕಿ-ಯನ್ನು
ಪಲ್ಲ-ಕ್ಕಿ-ಯಲ್ಲಿ
ಪಲ್ಲ-ಕ್ಕಿ-ಯಿಂದ
ಪಲ್ಲವಿ
ಪಲ್ವ-ಲ-ನೆಂಬ
ಪಲ್ವ-ಲ-ರಾ-ಕ್ಷಸ
ಪವ-ಮಾನ
ಪವಾ-ಡ-ವನ್ನು
ಪವಿತ್ರ
ಪವಿ-ತ್ರ-ಗೊ-ಳಿ-ಸಲು
ಪವಿ-ತ್ರ-ಗೊ-ಳಿಸಿ
ಪವಿ-ತ್ರ-ಗೊ-ಳಿ-ಸುತ್ತಾ
ಪವಿ-ತ್ರ-ಗ್ರಂ-ಥ-ವಾಗಿ
ಪವಿ-ತ್ರ-ರಾಗಿ
ಪವಿ-ತ್ರ-ವಾ-ಗ-ಬ-ಲ್ಲುದೆ
ಪವಿ-ತ್ರ-ವಾ-ಗಿದ್ದ
ಪವಿ-ತ್ರ-ವಾ-ಗಿ-ಸಿತು
ಪವಿ-ತ್ರ-ವಾದ
ಪವಿ-ತ್ರ-ವಾ-ದುದು
ಪವಿ-ತ್ರ-ವಾ-ಯಿತು
ಪವಿ-ತ್ರ-ವಿದೆ
ಪವಿ-ತ್ರ-ವೆ-ನಿ-ಸಿದೆ
ಪವಿ-ತ್ರಾತ್ಮ
ಪವಿ-ತ್ರಾ-ತ್ಮ-ರಾಗಿ
ಪವೀ-ತ-ವನ್ನು
ಪಶು
ಪಶು-ಗ-ಳಂತೆ
ಪಶು-ಗಳು
ಪಶು-ಗ-ಳೆಲ್ಲ
ಪಶು-ಪಕ್ಷಿ
ಪಶು-ಪ-ಕ್ಷಿ-ಗಳು
ಪಶು-ಪ-ಕ್ಷಿ-ಪ್ರಾ-ಣಿ-ಗಳು
ಪಶು-ಪ-ಕ್ಷಿ-ಮೃ-ಗ-ಗಳಲ್ಲಿ
ಪಶು-ಪ್ರಾ-ಯನೇ
ಪಶು-ವನ್ನು
ಪಶ್ಚಾ
ಪಶ್ಚಾ-ತ್ತಾಪ
ಪಶ್ಚಾ-ತ್ತಾ-ಪ-ದಿಂದ
ಪಶ್ಚಾ-ತ್ತಾ-ಪ-ಪಟ್ಟ
ಪಶ್ಚಾ-ತ್ತಾ-ಪ-ಪ-ಟ್ಟನು
ಪಶ್ಚಾ-ತ್ತಾ-ಪ-ಪ-ಟ್ಟರು
ಪಶ್ಚಾ-ತ್ತಾ-ಪ-ಪಟ್ಟು
ಪಶ್ಚಾ-ತ್ತಾ-ಪ-ಪ-ಡುತ್ತಾ
ಪಶ್ಚಾ-ತ್ತಾ-ಪ-ವಾ-ಯಿತು
ಪಶ್ಚಿಮ
ಪಶ್ಚಿ-ಮಕ್ಕೂ
ಪಶ್ಚಿ-ಮಕ್ಕೆ
ಪಶ್ಚಿ-ಮ-ದಲ್ಲಿ
ಪಶ್ಚಿ-ಮ-ದಿಕ್ಕಿ
ಪಶ್ಚಿ-ಮ-ಸ-ಮು-ದ್ರ-ತೀ-ರಕ್ಕೆ
ಪಾಂಚ-ಜನ್ಯ
ಪಾಂಚ-ಜ-ನ್ಯ-ವನ್ನು
ಪಾಂಚ-ಜ-ನ್ಯ-ವೆಂಬ
ಪಾಂಚಾಲಿ
ಪಾಂಡ
ಪಾಂಡವ
ಪಾಂಡ-ವ-ಕೌ-ರವ
ಪಾಂಡ-ವ-ಪಕ್ಷ
ಪಾಂಡ-ವರ
ಪಾಂಡ-ವ-ರನ್ನು
ಪಾಂಡ-ವ-ರನ್ನೂ
ಪಾಂಡ-ವ-ರ-ನ್ನೇ-ನಾ-ದರೂ
ಪಾಂಡ-ವ-ರಲ್ಲಿ
ಪಾಂಡ-ವರಿ
ಪಾಂಡ-ವ-ರಿಂ-ದಲೂ
ಪಾಂಡ-ವ-ರಿ-ಗಾದ
ಪಾಂಡ-ವ-ರಿಗೆ
ಪಾಂಡ-ವರು
ಪಾಂಡ-ವರೂ
ಪಾಂಡ-ವ-ರೈ-ವರೂ
ಪಾಂಡ-ವಾಃ
ಪಾಂಡಿತ್ಯ
ಪಾಂಡು
ಪಾಂಡುವು
ಪಾಂಡ್ಯ-ದೇ-ಶದ
ಪಾಂತು
ಪಾಕ-ದಲ್ಲಿ
ಪಾಟ-ಗಳನ್ನು
ಪಾಟ-ಗಳಿಂದ
ಪಾಠ
ಪಾಠ-ಕ-ನನ್ನು
ಪಾಠ-ಗಳನ್ನು
ಪಾಠ-ವನ್ನು
ಪಾಡಿಗೆ
ಪಾಡೇನು
ಪಾಣಿ-ಗ್ರ-ಹ-ಣ-ವೆಂದೇ
ಪಾಣಿನಿ
ಪಾಣಿ-ನಿ-ಈ-ತನೂ
ಪಾಣಿ-ನಿಯ
ಪಾತಾಳ
ಪಾತಾ-ಳ-ಕ್ಕಿ-ಳಿ-ದಿದ್ದ
ಪಾತಾ-ಳಕ್ಕೆ
ಪಾತಾ-ಳ-ಗ-ಳಿಗೆ
ಪಾತಾ-ಳ-ಗ-ಳೆಂಬ
ಪಾತಾ-ಳ-ದಲ್ಲಿ
ಪಾತಾ-ಳ-ದ-ಲ್ಲಿದ್ದ
ಪಾತಾ-ಳ-ದಿಂದ
ಪಾತಾ-ಳ-ವನ್ನು
ಪಾತಾ-ಳ-ವಾ-ಸಿ-ಗ-ಳಿ-ಗಾಗಿ
ಪಾತಾ-ಳವೇ
ಪಾತಿ
ಪಾತಿ-ಗ-ಳಾದ
ಪಾತಿ-ವ್ರ-ತ್ಯದ
ಪಾತು
ಪಾತ್ರ
ಪಾತ್ರ-ಧಾ-ರಿ-ಯಾ-ಗು-ತ್ತದೆ
ಪಾತ್ರ-ನಾಗಿ
ಪಾತ್ರ-ನಾ-ಗಿ-ದ್ದನು
ಪಾತ್ರ-ನಾ-ಗಿ-ದ್ದೇನೆ
ಪಾತ್ರ-ನಾ-ಗಿ-ರು-ವ-ನೆಂಬ
ಪಾತ್ರ-ನಾದ
ಪಾತ್ರ-ನಾ-ದನು
ಪಾತ್ರ-ನಾ-ದ-ನೆಂ-ದರೆ
ಪಾತ್ರ-ನಾ-ದು-ದ-ರಿಂದ
ಪಾತ್ರ-ರಾ-ಗದೆ
ಪಾತ್ರ-ರಾ-ಗ-ಬೇ-ಕಾ-ಗು-ತ್ತದೆ
ಪಾತ್ರ-ರಾಗಿ
ಪಾತ್ರ-ರಾ-ಗಿ-ದ್ದೀರಿ
ಪಾತ್ರ-ರಾ-ಗು-ವುದು
ಪಾತ್ರ-ವಾ-ಗಿವೆ
ಪಾತ್ರಿ-ಗಳು
ಪಾತ್ರೆ
ಪಾತ್ರೆ-ಗಳನ್ನು
ಪಾತ್ರೆ-ಗಳಿಂದ
ಪಾತ್ರೆ-ಗಳು
ಪಾತ್ರೆಯ
ಪಾತ್ರೆ-ಯನ್ನು
ಪಾತ್ರೆ-ಯ-ನ್ನೆ-ತ್ತಿ-ಕೊಂ-ಡನು
ಪಾತ್ರೆ-ಯಲ್ಲಿ
ಪಾತ್ರೆ-ಯೊ-ಡನೆ
ಪಾತ್ವ-ಪ-ಥ್ಯಾತ್
ಪಾದ
ಪಾದಂ
ಪಾದ-ಕ-ಮಲ
ಪಾದ-ಕ-ಮ-ಲ-ಗಳನ್ನು
ಪಾದ-ಕ-ಮ-ಲ-ವನ್ನು
ಪಾದ-ಕ-ಮ-ಲ-ವ-ನ್ನೂರಿ
ಪಾದಕ್ಕೆ
ಪಾದ-ಕ್ಕೊ-ಪ್ಪಿ-ಸಿ-ದರು
ಪಾದ-ಗಳ
ಪಾದ-ಗ-ಳ-ನ್ನಿಟ್ಟು
ಪಾದ-ಗಳನ್ನು
ಪಾದ-ಗಳಲ್ಲಿ
ಪಾದ-ಗ-ಳ-ಲ್ಲಿದ್ದ
ಪಾದ-ಗಳಿಂದ
ಪಾದ-ಗ-ಳಿಗೆ
ಪಾದ-ಗಳು
ಪಾದ-ಗಳೂ
ಪಾದ-ಗ-ಳೆಂಬ
ಪಾದ-ಗ-ಳೆ-ರಡೂ
ಪಾದ-ತೀ-ರ್ಥ-ವನ್ನು
ಪಾದದ
ಪಾದ-ದ-ರ್ಶನ
ಪಾದ-ದ-ರ್ಶ-ನ-ಕ್ಕಾಗಿ
ಪಾದ-ದ-ರ್ಶ-ನ-ದಿಂದ
ಪಾದ-ದ-ರ್ಶ-ನ-ವನ್ನೆ
ಪಾದ-ದಲ್ಲಿ
ಪಾದ-ದಿಂದ
ಪಾದ-ಧೂಳಿ
ಪಾದ-ಧೂ-ಳಿ-ಕೂಡ
ಪಾದ-ಧೂ-ಳಿ-ಯಿಂದ
ಪಾದ-ಧ್ಯಾ-ನ-ದಲ್ಲಿ
ಪಾದ-ನಾ-ಗಿ-ರುವೆ
ಪಾದ-ಪ-ದ್ಮ-ಗಳನ್ನು
ಪಾದ-ಪ-ದ್ಮ-ಗಳಿಂದ
ಪಾದ-ಪೂ-ಜೆ-ಯನ್ನು
ಪಾದ-ರಕ್ಷೆ
ಪಾದ-ರಾಯ
ಪಾದ-ರಾ-ಯನು
ಪಾದ-ವಂ-ದ-ನಕ್ಕೆ
ಪಾದ-ವನ್ನು
ಪಾದ-ವು-ಳ್ಳ-ವನೆ
ಪಾದ-ಸೇವೆ
ಪಾದ-ಸೇ-ವೆ-ಗಳಿಂದ
ಪಾದ-ಸೇ-ವೆ-ಗಿಂ-ತಲೂ
ಪಾದ-ಸೇ-ವೆಯ
ಪಾದ-ಸೇ-ವೆ-ಯನ್ನು
ಪಾದ-ಸೇ-ವೆ-ಯಲ್ಲಿ
ಪಾದ-ಸೇ-ವೆಯೇ
ಪಾದ-ಸೇ-ವೆ-ಯೊಂ-ದ-ಲ್ಲದೆ
ಪಾದಾ-ಭ್ಯಾ-ನ್ನಮಃ
ಪಾದಾರ
ಪಾದಾ-ರ-ವಿಂ-ದ-ದಲ್ಲಿ
ಪಾದಾ-ರ-ವಿಂ-ದವು
ಪಾದಿ-ತ-ವಾ-ಗಿದೆ
ಪಾದು-ಕೆ-ಗಳನ್ನು
ಪಾದು-ಕೆ-ಗಳು
ಪಾದೋ-ದ-ಕ-ವನ್ನು
ಪಾನ
ಪಾನ-ಮಾ-ಡಿ-ದರೆ
ಪಾನ-ಮಾ-ಡಿ-ದ್ದಾರೆ
ಪಾನ-ಮಾ-ಡು-ವು-ದಕ್ಕೆ
ಪಾನೀ-ಯ-ಗಳು
ಪಾಪ
ಪಾಪ-ಪು-ಣ್ಯ-ಗಳು
ಪಾಪ-ಕರ
ಪಾಪ-ಕ-ರ-ವಾದ
ಪಾಪ-ಕ-ರ್ಮ-ಗಳನ್ನು
ಪಾಪ-ಕ-ರ್ಮದ
ಪಾಪ-ಕ-ರ್ಮ-ವನ್ನು
ಪಾಪ-ಕ-ರ್ಮ-ವೆಲ್ಲ
ಪಾಪ-ಕಾರ್ಯ
ಪಾಪ-ಕಾ-ರ್ಯ-ಗಳನ್ನು
ಪಾಪ-ಕಾ-ರ್ಯ-ಗ-ಳಿ-ಗಾಗಿ
ಪಾಪ-ಕಾ-ರ್ಯ-ವನ್ನು
ಪಾಪ-ಕಾ-ರ್ಯವೇ
ಪಾಪಕ್ಕೆ
ಪಾಪ-ಗಳ
ಪಾಪ-ಗಳನ್ನು
ಪಾಪ-ಗಳನ್ನೂ
ಪಾಪ-ಗಳನ್ನೆಲ್ಲ
ಪಾಪ-ಗಳನ್ನೆಲ್ಲಾ
ಪಾಪ-ಗಳಿಂದ
ಪಾಪ-ಗ-ಳಿಂ-ದಲೂ
ಪಾಪ-ಗ-ಳಿಗೆ
ಪಾಪ-ಗಳು
ಪಾಪ-ಗಳೂ
ಪಾಪ-ಗ-ಳೆಲ್ಲ
ಪಾಪ-ಗ-ಳೆ-ಲ್ಲವೂ
ಪಾಪದ
ಪಾಪ-ದಿಂದ
ಪಾಪ-ದಿಂ-ದಲೆ
ಪಾಪ-ನಾ-ಶಕ
ಪಾಪ-ನಿ-ವಾ-ರ-ಣೆ-ಗಾಗಿ
ಪಾಪ-ಪುಣ್ಯ
ಪಾಪ-ಪು-ಣ್ಯ-ಗಳ
ಪಾಪ-ಪು-ಣ್ಯ-ಗಳನ್ನೆಲ್ಲ
ಪಾಪ-ಪು-ಣ್ಯ-ಗ-ಳಿಗೆ
ಪಾಪ-ಪು-ಣ್ಯ-ವನ್ನು
ಪಾಪ-ಫ-ಲ-ವನ್ನು
ಪಾಪ-ಫ-ಲವೇ
ಪಾಪ-ರಾ-ಶಿ-ಗಳು
ಪಾಪ-ರಾ-ಶಿ-ಯನ್ನು
ಪಾಪ-ರೂ-ಪ-ವಾದ
ಪಾಪ-ಲೇಪ
ಪಾಪ-ಲೇ-ಪ-ವಿಲ್ಲ
ಪಾಪ-ಲೇ-ಪ-ವಿ-ಲ್ಲದ
ಪಾಪ-ಲೋ-ಕ-ವನ್ನೂ
ಪಾಪ-ವನ್ನು
ಪಾಪ-ವ-ಲ್ಲವೆ
ಪಾಪ-ವಿಲ್ಲ
ಪಾಪವು
ಪಾಪವೂ
ಪಾಪ-ವೆಂ-ಬುದು
ಪಾಪ-ವೆ-ಲ್ಲವೂ
ಪಾಪ-ವೆ-ಲ್ಲಿ-ಯದು
ಪಾಪ-ವೇನೂ
ಪಾಪವೊ
ಪಾಪ-ಹರ
ಪಾಪಿ
ಪಾಪಿ-ಕ್ಷ-ತ್ರಿ-ಯ-ರಿಂದ
ಪಾಪಿ-ಗಳ
ಪಾಪಿ-ಗಳನ್ನು
ಪಾಪಿ-ಗಳಲ್ಲಿ
ಪಾಪಿ-ಗ-ಳಾ-ಗು-ವರೋ
ಪಾಪಿ-ಗಳಿಂದ
ಪಾಪಿ-ಗ-ಳಿಗೆ
ಪಾಪಿ-ಗಳು
ಪಾಪಿ-ಗ-ಳೆಲ್ಲ
ಪಾಪಿಗೆ
ಪಾಪಿ-ಯನ್ನು
ಪಾಪಿ-ಯಾದ
ಪಾಪಿ-ಯೆಂದು
ಪಾಪಿಷ್ಠ
ಪಾಪ್ಮಾನಂ
ಪಾಮ-ರರು
ಪಾಯ-ಯಿತ್ವಾ
ಪಾಯಸ
ಪಾಯ-ಸ-ಗಳನ್ನು
ಪಾಯ-ಸ-ದಿಂದ
ಪಾಯ-ಸ-ಮಾ-ಡು-ವಾಗ
ಪಾಯಾ-ನ್ನೃ-ಸಿಂ-ಹೋ-ಽಸು-ರ-ಯೂ-ಥ-ಪಾ-ರಿಃ
ಪಾಯು-ದ್ಗು-ಣೇಶಃ
ಪಾರಂ-ಗತ
ಪಾರಂ-ಗ-ತತೆ
ಪಾರಂ-ಗ-ತ-ನಾ-ಗಿದ್ದ
ಪಾರಂ-ಗ-ತ-ನಾದ
ಪಾರಂ-ಗ-ತ-ನಾ-ದರೂ
ಪಾರಂ-ಗ-ತ-ರಾ-ದರು
ಪಾರಂ-ಗ-ತರು
ಪಾರಣೆ
ಪಾರ-ಣೆ-ಗಳನ್ನು
ಪಾರ-ಣೆಯ
ಪಾರ-ಣೆ-ಯನ್ನು
ಪಾರ-ತಂತ್ರ್ಯ
ಪಾರ-ವಿಲ್ಲ
ಪಾರ-ವಿ-ಲ್ಲ-ದಂ-ತಾ-ಯಿತು
ಪಾರವೇ
ಪಾರಾ-ಗ-ಬೇ-ಕೆ-ನ್ನು-ವ-ವನು
ಪಾರಾಗಿ
ಪಾರಾಗು
ಪಾರಾ-ಗುವ
ಪಾರಾ-ಗು-ವುದು
ಪಾರಾ-ದರೆ
ಪಾರಾ-ದುದು
ಪಾರಿ-ಜಾತ
ಪಾರಿ-ಜಾ-ತದ
ಪಾರಿ-ಜಾ-ತ-ವನ್ನು
ಪಾರಿ-ಜಾ-ತ-ವೃ-ಕ್ಷ-ವನ್ನು
ಪಾರಿ-ಜಾ-ತ-ವೃ-ಕ್ಷ-ವನ್ನೂ
ಪಾರಿ-ಜಾ-ತವೇ
ಪಾರಿ-ಜಾ-ತಾದಿ
ಪಾರಿ-ವಾಳ
ಪಾರಿ-ವಾ-ಳ-ದಂತೆ
ಪಾರಿ-ವಾ-ಳ-ದಿಂದ
ಪಾರಿ-ವಾ-ಳವು
ಪಾರು-ಪತ್ಯ
ಪಾರು-ಮಾ-ಡಿದ
ಪಾರ್ಥ
ಪಾರ್ವತಿ
ಪಾರ್ವ-ತಿಗೆ
ಪಾರ್ವ-ತಿಯ
ಪಾರ್ವ-ತಿ-ಯಿಂದ
ಪಾರ್ವ-ತಿಯು
ಪಾರ್ವ-ತಿ-ಯೆಂಬ
ಪಾರ್ವ-ತಿಯೊ
ಪಾರ್ವ-ತಿ-ಯೊ-ಡನೆ
ಪಾರ್ವತೀ
ಪಾರ್ವ-ತೀ-ದೇ-ವಿಗೆ
ಪಾರ್ಷ-ದ-ಭೂ-ಷ-ಣಾಃ
ಪಾರ್ಷ-ದ-ರಲ್ಲಿ
ಪಾಲ-ಕರು
ಪಾಲ-ನಂ-ತಹ
ಪಾಲನೂ
ಪಾಲನೆ
ಪಾಲಾ-ಗಿತ್ತು
ಪಾಲಾ-ಗಿ-ಹೋಗಿ
ಪಾಲಾ-ಗಿ-ಹೋಗು
ಪಾಲಾ-ಗು-ತ್ತದೆ
ಪಾಲಾ-ಗು-ತ್ತಾರೆ
ಪಾಲಾದ
ಪಾಲಾ-ದರು
ಪಾಲಾ-ದು-ದನ್ನು
ಪಾಲಾ-ಯಿತು
ಪಾಲಿಗೆ
ಪಾಲಿನ
ಪಾಲಿ-ಸಲಿ
ಪಾಲಿ-ಸಿ-ದನು
ಪಾಲಿ-ಸಿ-ದರೆ
ಪಾಲಿ-ಸುತ್ತಾ
ಪಾಲಿ-ಸು-ತ್ತಾರೆ
ಪಾಲಿ-ಸು-ತ್ತಿಲ್ಲ
ಪಾಲಿ-ಸು-ವು-ದ-ಕ್ಕಾಗಿ
ಪಾಲೀ-ಗ್ರಂ-ಥ-ವೊಂ-ದ-ರಲ್ಲಿ
ಪಾಲು
ಪಾಲು-ಗಾ-ರ-ನಾಗು
ಪಾಲೆ
ಪಾಳೆ-ಯಕ್ಕೆ
ಪಾಳೆ-ಯ-ವೆಲ್ಲ
ಪಾವಕ
ಪಾವನ
ಪಾವ-ನಕ್ಕೆ
ಪಾವ-ನ-ಗೊ-ಳಿ-ಸ-ಬೇಕು
ಪಾವ-ನ-ಗೊ-ಳಿ-ಸಲಿ
ಪಾವ-ನ-ಗೊ-ಳಿ-ಸಿದೆ
ಪಾವ-ನ-ಗೊ-ಳಿ-ಸುವ
ಪಾವ-ನ-ಗೊ-ಳಿ-ಸು-ವ-ವ-ರಂತೆ
ಪಾವ-ನ-ಮಾ-ಡು-ವು-ದ-ಕ್ಕಾ-ಗಿಯೆ
ಪಾವ-ನ-ರ-ನ್ನಾಗಿ
ಪಾವ-ನ-ಳಾ-ದಳು
ಪಾವ-ನ-ವಾ-ಗಿ-ರುವ
ಪಾವ-ನ-ವಾದ
ಪಾವ-ನ-ವಾ-ದವು
ಪಾವ-ನ-ವಾ-ಯಿತು
ಪಾಶಂ
ಪಾಶಂಡಿ
ಪಾಶ-ಇ-ವನ್ನು
ಪಾಶಕ್ಕೆ
ಪಾಶ-ದಿಂ-ದಲೂ
ಪಾಶಾತ್
ಪಾಶಾನ್
ಪಾಶು-ಪ-ತಾಸ್ತ್ರ
ಪಾಶು-ಪ-ತಾ-ಸ್ತ್ರ-ವನ್ನು
ಪಾಶ್ಚಾತ್ಯ
ಪಾಷಂಡ
ಪಾಷಂ-ಡ-ಧರ್ಮ
ಪಾಷಂ-ಡ-ಮತ
ಪಾಷಂ-ಡರ
ಪಾಷಂ-ಡಿ-ಗ-ಳಾಗಿ
ಪಾಷಂ-ಡಿ-ಪಾ-ಪ-ಷಂ-ಡ-ಗ-ಳೆ-ನಿ-ಸಿ-ದರು
ಪಾಸನ
ಪಿಂಗ-ಳೆ-ಯೆಂಬ
ಪಿಂಡ
ಪಿಂಡಕ್ಕೆ
ಪಿಂಡ-ದಲ್ಲಿ
ಪಿಂಡಾ-ರ-ಕ-ವೆಂಬ
ಪಿಡಿ-ಗಿ-ನಂತೆ
ಪಿತೃ
ಪಿತೃ-ಋಣ
ಪಿತೃ-ಗಳನ್ನೂ
ಪಿತೃ-ಗ-ಳಿಗೂ
ಪಿತೃ-ಗ-ಳಿಗೆ
ಪಿತೃ-ಗಳೂ
ಪಿತೃ-ಗ-ಳೆ-ಲ್ಲರೂ
ಪಿತೃ-ದೇ-ವ-ತೆ-ಗಳನ್ನೂ
ಪಿತೃ-ದೇ-ವ-ತೆ-ಗಳು
ಪಿತೃ-ದೇ-ವೆ-ತೆ-ಗಳು
ಪಿತೃ-ಪು-ಣ-ವನ್ನು
ಪಿತೃ-ಯಾ-ಗಾದಿ
ಪಿತೃ-ಲೋ-ಕಕ್ಕೆ
ಪಿತೃ-ಲೋ-ಕ-ವನ್ನು
ಪಿತೃ-ವಾ-ತ್ಸ-ಲ್ಯವೂ
ಪಿತೃ-ಹೃ-ದಯ
ಪಿತ್ತ-ವನ್ನು
ಪಿತ್ಥ-ವನ್ನು
ಪಿಪಾ-ಸೆ-ಯನ್ನು
ಪಿಪ್ಪ-ಲಾ-ಯನ
ಪಿಬ
ಪಿಳು-ಕಿ-ಸುತ್ತಾ
ಪಿಳ್ಳೆಯೂ
ಪಿಳ್ಳೆ-ಯೊಂದೂ
ಪಿಶಾ-ಚ-ಎಂದು
ಪಿಶಾ-ಚ-ಗಳನ್ನೂ
ಪಿಶಾ-ಚ-ವಿ-ಪ್ರ-ಗ್ರ-ಹ-ಘೋ-ರ-ದೃ-ಷ್ಟೀನ್
ಪಿಶಾಚಿ
ಪಿಶಾ-ಚಿ-ಗಳ
ಪಿಶಾ-ಚಿ-ಗ-ಳಿ-ಗಿಂತ
ಪಿಶಾ-ಚಿ-ಗಳು
ಪಿಶಾ-ಚಿ-ಗಳೆ
ಪೀಠ-ಗಳ
ಪೀಠ-ಗಳನ್ನು
ಪೀಠ-ಗ-ಳೇನು
ಪೀಠದ
ಪೀಠ-ದಲ್ಲಿ
ಪೀಠ-ದಿಂದ
ಪೀಠ-ವನ್ನು
ಪೀಠ-ಹಾ-ಕ-ಬೇಕು
ಪೀಠಾ-ಸು-ರ-ನೆಂಬ
ಪೀಡೆ
ಪೀಡೆಯು
ಪೀತಾಂ
ಪೀತಾಂ-ಬರ
ಪೀತಾಂ-ಬ-ರದ
ಪೀತಾಂ-ಬ-ರ-ವನ್ನು
ಪೀತಾಂ-ಬ-ರ-ವ-ನ್ನುಟ್ಟು
ಪೀತಾಂ-ಬ-ರ-ವ-ನ್ನು-ಡಿಸಿ
ಪೀಳಿಗೆ
ಪೀಳಿ-ಗೆಗೆ
ಪೀಳಿ-ಗೆ-ಯನ್ನು
ಪೀಳಿ-ಗೆ-ಯನ್ನೆ
ಪೀಳಿ-ಗೆ-ಯಲ್ಲಿ
ಪೀಳಿ-ಗೆ-ಯ-ವನೇ
ಪೀಳಿ-ಗೆ-ಯ-ವ-ರನ್ನು
ಪೀಳಿ-ಗೆ-ಯ-ವರೂ
ಪೀಳಿ-ಗೆ-ಯ-ವು-ದ್ವಾ-ರ-ಕಿಗೆ
ಪೀಳಿ-ಗೆ-ಯಿಂದ
ಪೀಳಿ-ಗೆ-ಯಿಂ-ದಲೂ
ಪೀಳಿ-ಗೆಯೇ
ಪು
ಪುಂಗಿ-ಯನ್ನು
ಪುಂಜಿ-ಕ-ಸ್ಥಲಿ
ಪುಂಡ-ನನ್ನು
ಪುಂಸ-ವನ
ಪುಕ್ಕ-ಗಳನ್ನು
ಪುಕ್ಕ-ಗಳನ್ನೆಲ್ಲ
ಪುಕ್ಕ-ಗ-ಳೆಲ್ಲ
ಪುಕ್ಕಿ-ನಿಂದ
ಪುಕ್ಕು-ಗ-ಳಿಗೂ
ಪುಕ್ಕು-ಗಳು
ಪುಕ್ಷ-ಪ-ರ್ವ-ತಕ್ಕೆ
ಪುಗ್ವೇ-ದದ
ಪುಗ್ವೇ-ದ-ದಲ್ಲಿ
ಪುಚೀಕ
ಪುಚೀ-ಕನ
ಪುಚೀ-ಕನು
ಪುಚೀ-ಕ-ನೆಂಬ
ಪುಜು
ಪುಟ
ಪುಟಕ್ಕೆ
ಪುಟ-ಚಂ-ಡಿ-ನಂತೆ
ಪುಟ-ಚೆಂ-ಡಿನ
ಪುಟ್ಟ
ಪುಟ್ಟ-ಕಾ-ಲು-ಗಳು
ಪುಟ್ಟ-ಬಾ-ಯಲ್ಲಿ
ಪುಡಿ-ಪುಡಿ
ಪುಡಿ-ಪು-ಡಿ-ಯಾ-ದವು
ಪುಡಿ-ಮಾ-ಡಿ-ದನು
ಪುಡಿ-ಮಾ-ಡಿಸಿ
ಪುಡಿ-ಯಂತೆ
ಪುಣ-ವನ್ನು
ಪುಣ್ಣ
ಪುಣ್ಯ
ಪುಣ್ಯ-ಪಾ-ಪ-ಕ-ರ್ಮ-ಗಳನ್ನು
ಪುಣ್ಯ-ಕ-ಥೆ-ಗಳನ್ನು
ಪುಣ್ಯ-ಕ-ಥೆ-ಯನ್ನು
ಪುಣ್ಯ-ಕರ
ಪುಣ್ಯ-ಕ-ರ-ವಾದ
ಪುಣ್ಯ-ಕ-ರ್ಮ-ಗಳ
ಪುಣ್ಯ-ಕ-ರ್ಮ-ಗಳು
ಪುಣ್ಯ-ಕಾರ್ಯ
ಪುಣ್ಯ-ಕಾ-ರ್ಯ-ದಿಂದ
ಪುಣ್ಯ-ಕಾ-ರ್ಯ-ವನ್ನು
ಪುಣ್ಯ-ಕಾ-ರ್ಯವೆ
ಪುಣ್ಯಕ್ಕೆ
ಪುಣ್ಯ-ಕ್ಷೇತ್ರ
ಪುಣ್ಯ-ಕ್ಷೇ-ತ್ರಕ್ಕೆ
ಪುಣ್ಯ-ಕ್ಷೇ-ತ್ರ-ಗಳ
ಪುಣ್ಯ-ಕ್ಷೇ-ತ್ರ-ಗ-ಳ-ಲ್ಲಿನ
ಪುಣ್ಯ-ಕ್ಷೇ-ತ್ರ-ಗ-ಳ-ಲ್ಲಿಯೂ
ಪುಣ್ಯ-ಗಳು
ಪುಣ್ಯ-ಚ-ರಿತ್ರೆ
ಪುಣ್ಯ-ತೀ-ರ್ಥಕ್ಕೆ
ಪುಣ್ಯ-ತೀ-ರ್ಥ-ಗಳಲ್ಲಿ
ಪುಣ್ಯ-ತೀ-ರ್ಥ-ದಲ್ಲಿ
ಪುಣ್ಯ-ತೀ-ರ್ಥ-ವನ್ನು
ಪುಣ್ಯದ
ಪುಣ್ಯ-ದಿಂದ
ಪುಣ್ಯ-ದಿಂ-ದಲೊ
ಪುಣ್ಯ-ನದಿ
ಪುಣ್ಯ-ನ-ದಿ-ಗಳಲ್ಲಿ
ಪುಣ್ಯ-ನ-ದಿ-ಗಳು
ಪುಣ್ಯ-ಪಾ-ಪ-ಗ-ಳಿಗೆ
ಪುಣ್ಯ-ಪಾ-ಪ-ಗಳು
ಪುಣ್ಯ-ಪಾ-ಪ-ರೂ-ಪ-ವಾದ
ಪುಣ್ಯ-ಪು-ರು-ಷನ
ಪುಣ್ಯ-ಪು-ರು-ಷ-ನನ್ನು
ಪುಣ್ಯ-ಪು-ರು-ಷ-ನಾದ
ಪುಣ್ಯ-ಪು-ರು-ಷನು
ಪುಣ್ಯ-ಫಲ
ಪುಣ್ಯ-ಫ-ಲ-ದಿಂದ
ಪುಣ್ಯ-ಮಾ-ಡಿತ್ತೊ
ಪುಣ್ಯ-ಲೋ-ಕ-ದಿಂದ
ಪುಣ್ಯ-ಲೋ-ಕ-ವನ್ನೂ
ಪುಣ್ಯ-ಲೋ-ಕವೂ
ಪುಣ್ಯ-ವಂ-ತರು
ಪುಣ್ಯ-ವಂತೆ
ಪುಣ್ಯ-ವನ್ನು
ಪುಣ್ಯ-ವಿಲ್ಲ
ಪುಣ್ಯವು
ಪುಣ್ಯವೂ
ಪುಣ್ಯವೆ
ಪುಣ್ಯ-ವೆಂಬ
ಪುಣ್ಯ-ವೆ-ಲ್ಲವೂ
ಪುಣ್ಯವೇ
ಪುಣ್ಯವೊ
ಪುಣ್ಯ-ಶಾಲಿ
ಪುಣ್ಯ-ಶಾ-ಲಿ-ಗಳು
ಪುಣ್ಯ-ಶಾ-ಲಿನಿ
ಪುಣ್ಯ-ಶಾ-ಲಿ-ಯಾದ
ಪುಣ್ಯ-ಶ್ಲೋ-ಕನ
ಪುಣ್ಯ-ಸ್ಥಳ
ಪುಣ್ಯ-ಸ್ಥ-ಳ-ವದು
ಪುಣ್ಯಾತ್ಮ
ಪುಣ್ಯಾ-ತ್ಮ-ನಾ-ಗಿ-ದ್ದಾನೆ
ಪುಣ್ಯಾ-ತ್ಮ-ನಾದ
ಪುಣ್ಯಾ-ತ್ಮನು
ಪುಣ್ಯೇ
ಪುತ
ಪುತ-ಧ್ವಜ
ಪುತ-ವಾಗಿ
ಪುತು-ಪ-ರ್ಣನು
ಪುತು-ಸ್ನಾತ
ಪುತ್ರ
ಪುತ್ರಕಾ
ಪುತ್ರ-ಕಾ-ಮಿ-ನಿ-ಯಾಗಿ
ಪುತ್ರ-ಕಾ-ಮೇಷ್ಟಿ
ಪುತ್ರ-ದುಃ-ಖ-ವನ್ನು
ಪುತ್ರ-ನನ್ನು
ಪುತ್ರ-ನಾ-ಗ-ಬೇಕು
ಪುತ್ರ-ನಾ-ಗು-ತ್ತಾನೆ
ಪುತ್ರ-ನಾದ
ಪುತ್ರ-ನಾ-ಮಸಿ
ಪುತ್ರ-ಪ್ರೇಮ
ಪುತ್ರ-ಪ್ರೇ-ಮ-ದಿಂದ
ಪುತ್ರ-ಮೋಹ
ಪುತ್ರ-ಮೋ-ಹ-ವಷ್ಟೇ
ಪುತ್ರ-ರನ್ನು
ಪುತ್ರ-ರನ್ನೂ
ಪುತ್ರ-ರಾದ
ಪುತ್ರರು
ಪುತ್ರ-ರೂ-ಪಿ-ಯಾದ
ಪುತ್ರ-ಲಾ-ಭ-ದಿಂದ
ಪುತ್ರ-ವ-ತಿ-ಯ-ರಾ-ಗ-ಲಿಲ್ಲ
ಪುತ್ರ-ವ-ತಿ-ಯ-ರಾಗಿ
ಪುತ್ರ-ವ-ತಿ-ಯ-ರಾ-ದರು
ಪುತ್ರ-ಸಂ-ತಾ-ನ-ಕ್ಕಾಗಿ
ಪುತ್ರ-ಸಂ-ತಾ-ನ-ವಾ-ಗು-ವಂತೆ
ಪುತ್ರಾ-ದಿ-ಗ-ಳ-ಲ್ಲಿನ
ಪುತ್ರಾರ್ಥಿ
ಪುತ್ರಿ-ಯ-ರಲ್ಲಿ
ಪುತ್ರಿ-ಯರು
ಪುತ್ರಿ-ಯಾದ
ಪುತ್ವಿ-ಕ್ಕು-ಗಳನ್ನೂ
ಪುತ್ವಿ-ಕ್ಕು-ಗಳು
ಪುತ್ವಿ-ಜ-ರಿಗೆ
ಪುನಂತೀ
ಪುನಂ-ತ್ವ-ಮೀ-ವ-ಘ್ನೀಃ
ಪುನಃ
ಪುನ-ರಪಿ
ಪುನ-ರಾ-ಗಾಃ
ಪುನ-ರಾ-ವ-ರ್ತ-ನೆ-ಯಾ-ಗಿ-ರುವ
ಪುನ-ರಾ-ವಿಶ್ಯ
ಪುನ-ರ್ಜ-ನ್ಮಕ್ಕೆ
ಪುನ-ರ್ಜ-ನ್ಮದ
ಪುನ-ರ್ಜ-ನ್ಮ-ವಿ-ಲ್ಲದ
ಪುನೀ-ತ-ನಾ-ಗಿದ್ದ
ಪುನೀ-ತ-ನಾದ
ಪುನೀ-ತ-ರ-ನ್ನಾಗಿ
ಪುಭು
ಪುಮಾನ್
ಪುರ
ಪುರಂ-ಜನ
ಪುರಂ-ಜ-ನನ
ಪುರಂ-ಜ-ನ-ನಾ-ಗಿ-ದ್ದಾಗ
ಪುರಂ-ಜ-ನ-ನಿಗೆ
ಪುರಂ-ಜ-ನನು
ಪುರಂ-ಜ-ನ-ನೆಂಬ
ಪುರಂ-ಜ-ನನೇ
ಪುರಂ-ಜ-ನ-ರಾಜ
ಪುರಂ-ಜಯ
ಪುರಂ-ಜ-ಯನ
ಪುರಂ-ಜ-ಯ-ನನ್ನು
ಪುರಂ-ಜ-ಯ-ನಿಂದ
ಪುರಂ-ಜ-ಯನು
ಪುರಂ-ಜ-ಯ-ನೆಂಬ
ಪುರಕ್ಕೆ
ಪುರ-ಜ-ನರ
ಪುರ-ಜ-ನರು
ಪುರ-ಜ-ನ-ರೆಲ್ಲ
ಪುರದ
ಪುರ-ಪ್ರ-ವೇ-ಶ-ಕ್ಕಾಗಿ
ಪುರ-ವನ್ನು
ಪುರ-ಸ್ಕರಿ
ಪುರ-ಸ್ಕ-ರಿ-ಸು-ವ-ವ-ನಂತೆ
ಪುರ-ಸ್ತ್ರೀ-ಯರು
ಪುರಾ
ಪುರಾಣ
ಪುರಾಣಂ
ಪುರಾಣಃ
ಪುರಾ-ಣ-ಕ-ರ್ತ-ರಿಂದ
ಪುರಾ-ಣಕ್ಕೆ
ಪುರಾ-ಣ-ಗಳ
ಪುರಾ-ಣ-ಗಳನ್ನು
ಪುರಾ-ಣ-ಗಳನ್ನೂ
ಪುರಾ-ಣ-ಗಳಲ್ಲಿ
ಪುರಾ-ಣ-ಗ-ಳ-ಲ್ಲಿನ
ಪುರಾ-ಣ-ಗ-ಳಾಗಿ
ಪುರಾ-ಣ-ಗಳಿಂದ
ಪುರಾ-ಣ-ಗ-ಳಿಗೆ
ಪುರಾ-ಣ-ಗಳು
ಪುರಾ-ಣ-ಗಳೂ
ಪುರಾ-ಣ-ಜ್ಞಾ-ನ-ವಿ-ಲ್ಲದ
ಪುರಾ-ಣದ
ಪುರಾ-ಣ-ದಲ್ಲಿ
ಪುರಾ-ಣ-ದ-ಲ್ಲಿಯೂ
ಪುರಾ-ಣ-ಪ-ಠ-ನದ
ಪುರಾ-ಣ-ಪು-ರುಷ
ಪುರಾ-ಣ-ಪು-ರು-ಷನ
ಪುರಾ-ಣ-ಪು-ರು-ಷ-ನಾದ
ಪುರಾ-ಣ-ಪು-ರು-ಷೋ-ತ್ತ-ಮ-ನಾದ
ಪುರಾ-ಣ-ಮಿ-ತ್ಯೇವ
ಪುರಾ-ಣ-ವನ್ನು
ಪುರಾ-ಣ-ವಾ-ಙ್ಮ-ಯ-ವೆ-ಲ್ಲವೂ
ಪುರಾ-ಣ-ವಾ-ದರೂ
ಪುರಾ-ಣ-ವಿ-ತ್ತೆಂದು
ಪುರಾ-ಣವು
ಪುರಾ-ಣ-ಶ್ರ-ವಣ
ಪುರಾ-ಣಾನಿ
ಪುರಾ-ಣಿಕ
ಪುರಾ-ಣಿ-ಕ-ನನ್ನು
ಪುರಾ-ಣಿ-ಕನು
ಪುರಾ-ಣೇ-ತಿ-ಹಾ-ಸ-ಗಳನ್ನು
ಪುರಾಣೋ
ಪುರು-ಕುತ್ಸ
ಪುರು-ಕು-ತ್ಸನ
ಪುರುಜಂ
ಪುರು-ಜಿತ್
ಪುರು-ಜಿತ್ತು
ಪುರು-ವಂ-ಶದ
ಪುರುಷ
ಪುರುಷಂ
ಪುರುಷಃ
ಪುರು-ಷ-ದೇ-ಹ-ದಿಂದ
ಪುರು-ಷನ
ಪುರು-ಷ-ನನ್ನು
ಪುರು-ಷ-ನಲ್ಲಿ
ಪುರು-ಷ-ನಾದ
ಪುರು-ಷ-ನಿಗೆ
ಪುರು-ಷನು
ಪುರು-ಷ-ನೆಂದು
ಪುರು-ಷ-ನೊ-ಡನೆ
ಪುರು-ಷ-ನೊ-ಬ್ಬನು
ಪುರು-ಷ-ಮೃ-ಷ-ಭ-ಮಾದ್ಯಂ
ಪುರು-ಷರ
ಪುರು-ಷ-ರಾದ
ಪುರು-ಷ-ರಿಂದ
ಪುರು-ಷರು
ಪುರು-ಷ-ರೂಪು
ಪುರು-ಷರೆ
ಪುರು-ಷ-ರೆಂಬ
ಪುರು-ಷ-ಶ್ರೇ-ಷ್ಠ-ನಾಗಿ
ಪುರುಷಾ
ಪುರು-ಷಾಂ-ಗ-ನಾಮ್
ಪುರು-ಷಾ-ಕಾ-ರದ
ಪುರು-ಷಾ-ರ್ಚ-ನಾಂ-ತ-ರಾತ್
ಪುರು-ಷಾರ್ಥ
ಪುರು-ಷಾ-ರ್ಥ-ಗಳ
ಪುರು-ಷಾ-ರ್ಥ-ಗಳನ್ನು
ಪುರು-ಷಾ-ರ್ಥ-ಗಳನ್ನೂ
ಪುರು-ಷಾ-ರ್ಥ-ಗ-ಳಿ-ಗಿಂ-ತಲೂ
ಪುರು-ಷಾ-ರ್ಥದ
ಪುರು-ಷಾ-ರ್ಥ-ವನ್ನೂ
ಪುರು-ಷಾ-ವ-ತಾರ
ಪುರು-ಷಾ-ವ-ತಾ-ರ-ವಾ-ದನು
ಪುರುಷೋ
ಪುರು-ಷೋ-ತ್ತ-ಮನ
ಪುರು-ಷೋ-ತ್ತ-ಮ-ನಾಗಿ
ಪುರು-ಷೋ-ತ್ತ-ಮ-ನಾದ
ಪುರು-ಷೋ-ತ್ತ-ಮ-ನಿಗೆ
ಪುರು-ಷೋ-ತ್ತ-ಮನು
ಪುರು-ಷೋ-ತ್ತ-ಮನೂ
ಪುರು-ಸ-ತ್ತಿಲ್ಲ
ಪುರೂ-ರವ
ಪುರೂ-ರ-ವನ
ಪುರೂ-ರ-ವ-ನನ್ನು
ಪುರೂ-ರ-ವ-ನಿಗೆ
ಪುರೂ-ರ-ವನು
ಪುರೂ-ರ-ವ-ನೆಂಬ
ಪುರೂ-ರ-ವರು
ಪುರೂ-ರ-ವಸ್
ಪುರೋ
ಪುರೋ-ಹಿತ
ಪುರೋ-ಹಿ-ತನ
ಪುರೋ-ಹಿ-ತ-ನಾಗಿ
ಪುರೋ-ಹಿ-ತ-ನಾ-ಗಿ-ದ್ದನು
ಪುರೋ-ಹಿ-ತ-ನಾ-ಗು-ವು-ದಾಗಿ
ಪುರೋ-ಹಿ-ತ-ನಾದ
ಪುರೋ-ಹಿ-ತನೂ
ಪುರೋ-ಹಿ-ತರ
ಪುರೋ-ಹಿ-ತ-ರನ್ನು
ಪುರೋ-ಹಿ-ತ-ರನ್ನೂ
ಪುರೋ-ಹಿ-ತ-ರಾದ
ಪುರೋ-ಹಿ-ತ-ರಿಗೂ
ಪುರೋ-ಹಿ-ತ-ರಿಗೇ
ಪುರೋ-ಹಿ-ತರು
ಪುರೋ-ಹಿ-ತ-ರೊ-ಡನೆ
ಪುರೋ-ಹಿ-ತ-ವೃತ್ತಿ
ಪುಲ-ಕ-ಗೊಂ-ಡರು
ಪುಲ-ಕಿತ
ಪುಲ-ಕಿ-ತ-ಳಾ-ಗು-ತ್ತಿದ್ದ
ಪುಲ-ಕಿ-ತ-ವಾ-ಯಿತು
ಪುಲ-ಸ್ತ್ಯನು
ಪುಲ-ಸ್ತ್ಯನೂ
ಪುಲಹ
ಪುಲ-ಹನು
ಪುಲ-ಹನೂ
ಪುಲ-ಹಾ-ಶ್ರ-ಮಕ್ಕೆ
ಪುಲ-ಹಾ-ಶ್ರ-ಮವು
ಪುಳ-ಕ-ಗ-ಳೆ-ದ್ದವು
ಪುಳ-ಕ-ಗೊ-ಳ್ಳು-ವನು
ಪುಳ-ಕ-ದಂತೆ
ಪುಳ-ಕಿ-ತ-ನಾದ
ಪುಳ-ಕಿ-ತ-ವಾ-ಗು-ತ್ತದೆ
ಪುಳ-ಕಿ-ತ-ವಾ-ಯಿತು
ಪುಳು-ಕಿ-ತ-ರಾ-ದರು
ಪುಷಭ
ಪುಷ-ಭನು
ಪುಷ-ಭ-ನೆಂಬ
ಪುಷ-ಭಾ-ವ-ತಾರ
ಪುಷಿ
ಪುಷಿ-ಗಳ
ಪುಷಿ-ಗಳನ್ನು
ಪುಷಿ-ಗಳನ್ನೂ
ಪುಷಿ-ಗಳನ್ನೆಲ್ಲ
ಪುಷಿ-ಗಳಲ್ಲಿ
ಪುಷಿ-ಗ-ಳಾ-ಗಿ-ದ್ದವೊ
ಪುಷಿ-ಗಳಿಂದ
ಪುಷಿ-ಗ-ಳಿಗೂ
ಪುಷಿ-ಗ-ಳಿಗೆ
ಪುಷಿ-ಗ-ಳಿ-ಗೆಲ್ಲಾ
ಪುಷಿ-ಗಳು
ಪುಷಿ-ಗಳೂ
ಪುಷಿ-ಗಳೆ
ಪುಷಿ-ಗ-ಳೆಲ್ಲ
ಪುಷಿ-ಗ-ಳೆ-ಲ್ಲರ
ಪುಷಿ-ಗ-ಳೆ-ಲ್ಲರೂ
ಪುಷಿ-ಗ-ಳೇನೋ
ಪುಷಿ-ಗ-ಳೊ-ಡನೆ
ಪುಷಿಗೆ
ಪುಷಿ-ಜೀ-ವ-ನ-ವನ್ನು
ಪುಷಿ-ದೇ-ಹವೇ
ಪುಷಿ-ಮು-ನಿ-ಗ-ಳಾ-ಗ-ಲಿ-ನಾವೆ
ಪುಷಿ-ಮು-ನಿ-ಗ-ಳೊ-ಡನೆ
ಪುಷಿಯ
ಪುಷಿ-ಯಂತೆ
ಪುಷಿ-ಯನ್ನು
ಪುಷಿ-ಯಿಂದ
ಪುಷಿಯು
ಪುಷಿಯೇ
ಪುಷಿ-ಶಾಪ
ಪುಷಿ-ಶಾ-ಪಕ್ಕೆ
ಪುಷಿ-ಶಾ-ಪದ
ಪುಷಿ-ಶಾ-ಪ-ದಿಂದ
ಪುಷಿ-ಶಾ-ಪ-ವನ್ನು
ಪುಷ್ಕರ
ಪುಷ್ಕ-ರ-ದ್ವೀ-ಪ-ವಿದೆ
ಪುಷ್ಕಲ
ಪುಷ್ಟಿ-ಗ-ಳಿಗೆ
ಪುಷ್ಪ
ಪುಷ್ಪ-ಕ-ವಿ-ಮಾ-ನ-ದಲ್ಲಿ
ಪುಷ್ಪ-ಗಳ
ಪುಷ್ಪ-ಗಳಿಂದ
ಪುಷ್ಪದ
ಪುಷ್ಪ-ಬಾ-ಣ-ವನ್ನು
ಪುಷ್ಪ-ಭದ್ರಾ
ಪುಷ್ಪ-ಭದ್ರೆ
ಪುಷ್ಪ-ಮಾ-ಲೆ-ಯನ್ನು
ಪುಷ್ಪ-ವೃ-ಷ್ಟಿ-ಯಾ-ಯಿತು
ಪುಷ್ಪ-ಹಾ-ರ-ವನ್ನೂ
ಪುಷ್ಯ-ಮಿ-ತ್ರನ
ಪುಷ್ಯ-ಮಿ-ತ್ರನು
ಪುಷ್ಯ-ಮೂ-ಕ-ಪ-ರ್ವ-ತಕ್ಕೆ
ಪುಷ್ಯ-ಶೃಂ-ಗನ
ಪುಷ್ಯಾ-ಶ್ರಮ
ಪುಷ್ಯಾ-ಶ್ರ-ಮಕ್ಕೆ
ಪುಷ್ಯಾ-ಶ್ರ-ಮದ
ಪುಷ್ಯಾ-ಶ್ರ-ಮ-ದಲ್ಲಿ
ಪುಸ-ಲಾ-ಯಿಸಿ
ಪುಸ್ತ-ಕ-ಪಾಠ
ಪುಸ್ತ-ಕವು
ಪೂ
ಪೂಜಾ
ಪೂಜಾ-ಕಾ-ರ್ಯ-ಗಳನ್ನು
ಪೂಜಾ-ದಿ-ಕಾರ್ಯ
ಪೂಜಾ-ಫಲ
ಪೂಜಾ-ಮೂ-ರ್ತಿ-ಯನ್ನು
ಪೂಜಾ-ವಸ್ತು
ಪೂಜಾ-ವಿ-ಧಾ-ನ-ಗಳೂ
ಪೂಜಾ-ವಿಧಿ
ಪೂಜಿ
ಪೂಜಿ-ಸದೆ
ಪೂಜಿ-ಸ-ಬೇಕು
ಪೂಜಿ-ಸಲಿ
ಪೂಜಿ-ಸ-ಲೆಂದು
ಪೂಜಿಸಿ
ಪೂಜಿ-ಸಿ-ಕೊ-ಳ್ಳು-ತ್ತಿ-ದ್ದನು
ಪೂಜಿ-ಸಿ-ದಂ-ತಾ-ಯಿತು
ಪೂಜಿ-ಸಿ-ದಂ-ತೆಯೇ
ಪೂಜಿ-ಸಿ-ದನು
ಪೂಜಿ-ಸಿ-ದರೂ
ಪೂಜಿ-ಸಿ-ದರೆ
ಪೂಜಿ-ಸಿ-ದಳು
ಪೂಜಿ-ಸಿದೆ
ಪೂಜಿ-ಸಿ-ರುವೆ
ಪೂಜಿಸು
ಪೂಜಿ-ಸುತ್ತಾ
ಪೂಜಿ-ಸು-ತ್ತಾರೆ
ಪೂಜಿ-ಸು-ತ್ತಿ-ದ್ದನು
ಪೂಜಿ-ಸು-ತ್ತಿ-ದ್ದರೆ
ಪೂಜಿ-ಸು-ತ್ತಿ-ರುವ
ಪೂಜಿ-ಸು-ತ್ತಿ-ರುವಿ
ಪೂಜಿ-ಸು-ತ್ತಿ-ರು-ವು-ದ-ರಿಂ-ದಲೆ
ಪೂಜಿ-ಸು-ತ್ತೇವೆ
ಪೂಜಿ-ಸುವ
ಪೂಜಿ-ಸು-ವರು
ಪೂಜಿ-ಸು-ವ-ವನು
ಪೂಜಿ-ಸು-ವ-ವ-ರಂತೆ
ಪೂಜಿ-ಸು-ವ-ವರೇ
ಪೂಜಿ-ಸು-ವಾಗ
ಪೂಜಿ-ಸು-ವು-ದ-ರಲ್ಲಿ
ಪೂಜಿ-ಸು-ವು-ದ-ರಿಂದ
ಪೂಜಿ-ಸು-ವುದು
ಪೂಜೆ
ಪೂಜೆ-ಗಳಿಂದ
ಪೂಜೆ-ಗ-ಳೆಲ್ಲ
ಪೂಜೆ-ಗಾಗಿ
ಪೂಜೆಗೂ
ಪೂಜೆಗೆ
ಪೂಜೆ-ಗೆಂದು
ಪೂಜೆ-ಗೊಂಡ
ಪೂಜೆ-ಗೊ-ಳ್ಳುವ
ಪೂಜೆ-ಮಾ-ಡದ
ಪೂಜೆ-ಮಾ-ಡಿ-ದನು
ಪೂಜೆ-ಮಾ-ಡಿ-ದ-ಮೇಲೆ
ಪೂಜೆ-ಮಾ-ಡಿರಿ
ಪೂಜೆ-ಮಾ-ಡಿಸಿ
ಪೂಜೆ-ಮಾ-ಡಿ-ಸಿ-ದರು
ಪೂಜೆ-ಮಾಡು
ಪೂಜೆಯ
ಪೂಜೆ-ಯನ್ನು
ಪೂಜೆ-ಯಲ್ಲಿ
ಪೂಜೆ-ಯಿಂದ
ಪೂಜೆಯೂ
ಪೂಜೆಯೆ
ಪೂಜೆ-ಯೆಂದು
ಪೂಜೆ-ಯೆಂಬ
ಪೂಜೆಯೇ
ಪೂಜ್ಯ
ಪೂಜ್ಯತೆ
ಪೂಜ್ಯ-ನಾದ
ಪೂಜ್ಯನೇ
ಪೂಜ್ಯ-ರಿರಾ
ಪೂಜ್ಯರು
ಪೂಜ್ಯರೆ
ಪೂತನಿ
ಪೂತ-ನಿಯ
ಪೂತ-ನಿ-ಯನ್ನು
ಪೂತ-ನಿಯು
ಪೂತ-ನಿ-ಯೆಂದು
ಪೂತ-ನಿ-ಯೆಂಬ
ಪೂರ-ಕ-ವಾ-ಗಿವೆ
ಪೂರು
ಪೂರುವು
ಪೂರೈ-ಸಲು
ಪೂರೈ-ಸಿ-ಕೊ-ಳ್ಳ-ಬೇಕು
ಪೂರ್ಣ
ಪೂರ್ಣ-ಕಾ-ಮ-ನಾದ
ಪೂರ್ಣ-ಚಂದ್ರ
ಪೂರ್ಣ-ಚಂ-ದ್ರ-ನಂ-ತಹ
ಪೂರ್ಣ-ಚಂ-ದ್ರ-ನಂತೆ
ಪೂರ್ಣ-ತೆಯ
ಪೂರ್ಣ-ಪ್ರ-ವಾ-ಹ-ದಿಂದ
ಪೂರ್ಣ-ವಾಗಿ
ಪೂರ್ಣ-ವಾ-ಗು-ತ್ತದೆ
ಪೂರ್ಣಾಂ-ಶ-ದಿಂದ
ಪೂರ್ಣಾ-ತ್ಮ-ನಾಗಿ
ಪೂರ್ಣಾ-ಯು-ಷಿ-ಗ-ಳಾಗಿ
ಪೂರ್ಣಾ-ವ-ತಾರ
ಪೂರ್ಣೆಯೇ
ಪೂರ್ತಿ
ಪೂರ್ತಿ-ಯಾಗಿ
ಪೂರ್ವ
ಪೂರ್ವ-ಕರ್ಮ
ಪೂರ್ವ-ಕ-ರ್ಮ-ಗ-ಳಿಗೆ
ಪೂರ್ವ-ಕ-ರ್ಮದ
ಪೂರ್ವ-ಕ-ಲ್ಪ-ದಲ್ಲಿ
ಪೂರ್ವ-ಕಾ-ಲ-ದಲ್ಲಿ
ಪೂರ್ವಕ್ಕೆ
ಪೂರ್ವ-ಚಿತ್ತಿ
ಪೂರ್ವ-ಚಿ-ತ್ತಿ-ಯನ್ನು
ಪೂರ್ವ-ಚಿ-ತ್ತಿಯು
ಪೂರ್ವ-ಜನ್ಮ
ಪೂರ್ವ-ಜ-ನ್ಮದ
ಪೂರ್ವ-ಜ-ನ್ಮ-ದಲ್ಲಿ
ಪೂರ್ವ-ಜ-ನ್ಮ-ವನ್ನು
ಪೂರ್ವದ
ಪೂರ್ವ-ದಲ್ಲಿ
ಪೂರ್ವ-ದ-ಲ್ಲಿಯೇ
ಪೂರ್ವ-ದಿ-ಕ್ಕಿಗೂ
ಪೂರ್ವ-ದಿ-ಕ್ಕಿಗೆ
ಪೂರ್ವ-ದಿ-ಕ್ಕಿ-ನಂತೆ
ಪೂರ್ವ-ಪು-ಣ್ಯ-ವೇ-ನಾ-ದರೂ
ಪೂರ್ವಾ-ಗ್ರ-ವಾಗಿ
ಪೂರ್ವಾ-ಗ್ರ-ವಾ-ಗಿ-ರುವ
ಪೂರ್ವಾದಿ
ಪೂರ್ವಾ-ಪರ
ಪೂರ್ವಾ-ರ್ಜಿತ
ಪೂರ್ವಾರ್ಧ
ಪೂರ್ವಾ-ಹ್ನ-ದ-ಲ್ಲಿಯೂ
ಪೃಥಾ
ಪೃಥಿ-ವೀ-ಶ-ಕ್ತ-ರಾ-ದ-ವ-ರಿಗೇ
ಪೃಥಿ-ವ್ಯಾದಿ
ಪೃಥಿ-ವ್ಯಾಧಿ
ಪೃಥು
ಪೃಥು-ಚಕ್ರ
ಪೃಥು-ಚ-ಕ್ರ-ವರ್ತಿ
ಪೃಥು-ಚ-ಕ್ರ-ವ-ರ್ತಿಯ
ಪೃಥು-ಚ-ಕ್ರ-ವ-ರ್ತಿಯು
ಪೃಥು-ಚಕ್ರಿ
ಪೃಥು-ಚ-ಕ್ರಿಗೆ
ಪೃಥು-ಚ-ಕ್ರಿಯು
ಪೃಥು-ರಾ-ಜನ
ಪೃಥು-ರಾ-ಜನು
ಪೃಥು-ವಿನ
ಪೃಥುವು
ಪೃಥು-ವೆಂಬ
ಪೃಥ್ವಿ
ಪೃಥ್ವಿಯೇ
ಪೃಶ್ನಿ-ಗ-ರ್ಭ-ನೆಂಬ
ಪೃಷಧ್ರ
ಪೃಷ-ಧ್ರನು
ಪೃಷ್ಠ-ರಾಜ
ಪೃಷ್ಠ-ವನ್ನು
ಪೆಟ್ಟನ್ನು
ಪೆಟ್ಟಿ
ಪೆಟ್ಟಿಗೆ
ಪೆಟ್ಟಿ-ನಿಂದ
ಪೆಟ್ಟು
ಪೆಟ್ಟು-ಗಳನ್ನು
ಪೆಟ್ಟು-ತಿಂದು
ಪೆಟ್ಟೆ
ಪೇಕ್ಷೆ-ಯಿ-ಲ್ಲದೆ
ಪೇಚಾ-ಡು-ತ್ತಿ-ರಲು
ಪೊದ-ರು-ಗಳನ್ನೂ
ಪೊಳ್ಳು
ಪೋತನ
ಪೋರ
ಪೋಷಣ
ಪೋಷ-ಣೆ-ಗಾಗಿ
ಪೋಷ-ಣೆಗೆ
ಪೋಷ-ಣೆಯ
ಪೋಷಿ-ತ-ವಾದ
ಪೋಷಿ-ಸಲು
ಪೋಷಿಸಿ
ಪೋಷಿ-ಸುವ
ಪೌಂಡ್ರಕ
ಪೌಂಡ್ರ-ಕನ
ಪೌಂಡ್ರ-ಕ-ನಿಗೆ
ಪೌಂಡ್ರ-ಕನು
ಪೌಂಡ್ರ-ಕ-ನೆಂಬ
ಪೌರ-ವಿ-ಸು-ಭದ್ರ
ಪೌರು-ಷ-ವನ್ನು
ಪ್ರಕ-ಟ-ಗೊ-ಳಿ-ಸಿದೆ
ಪ್ರಕ-ಟ-ನಾ-ದನು
ಪ್ರಕ-ಟ-ವಾ-ಗು-ತ್ತಿದ್ದು
ಪ್ರಕ-ಟಿ-ಸು-ತ್ತಿ-ದ್ದೇವೆ
ಪ್ರಕಾಶ
ಪ್ರಕಾ-ಶ-ಕರ
ಪ್ರಕಾ-ಶ-ಕ-ರಿಗೆ
ಪ್ರಕಾ-ಶನೂ
ಪ್ರಕಾ-ಶ-ಮಾ-ನ-ವಾ-ಗಿ-ರುವ
ಪ್ರಕಾ-ಶ-ಮಾ-ನ-ವಾದ
ಪ್ರಕಾ-ಶ-ವು-ಳ್ಳದ್ದು
ಪ್ರಕಾಶಿ
ಪ್ರಕಾ-ಶಿ-ಸುವ
ಪ್ರಕೃತಿ
ಪ್ರಕೃ-ತಿ-ಗಿಂತ
ಪ್ರಕೃ-ತಿ-ಗು-ಣ-ಗ-ಳಾದ
ಪ್ರಕೃ-ತಿ-ಗು-ಣ-ಗಳಿಂದ
ಪ್ರಕೃ-ತಿಗೆ
ಪ್ರಕೃ-ತಿ-ದೇವಿ
ಪ್ರಕೃ-ತಿ-ದೇ-ವಿಗೆ
ಪ್ರಕೃ-ತಿ-ಪು-ರು-ಷರ
ಪ್ರಕೃ-ತಿಯ
ಪ್ರಕೃ-ತಿ-ಯ-ಲ್ಲಿನ
ಪ್ರಕೃ-ತಿ-ಯಿಂದ
ಪ್ರಕೃ-ತಿಯು
ಪ್ರಕೃ-ತಿ-ಯೆಂ-ಬುದು
ಪ್ರಕೃ-ತಿಯೇ
ಪ್ರಕೃ-ತಿ-ವಿ-ಕಾ-ರ-ವಾದ
ಪ್ರಕೃ-ತಿ-ಸಂ-ಬಂಧ
ಪ್ರಕೃ-ತೇಃ
ಪ್ರಕೋ-ಪಕ್ಕೆ
ಪ್ರಕೋ-ಪಕ್ಕೇ-ರಿತು
ಪ್ರಕ್ಯ-ಕ್ಷ-ನಾಗಿ
ಪ್ರಖ-ರ-ವಾ-ಗಿತ್ತು
ಪ್ರಖ್ಯಾ-ತ-ನಾ-ಗಿ-ದ್ದನು
ಪ್ರಖ್ಯಾ-ತ-ನಾ-ದನು
ಪ್ರಖ್ಯಾ-ತ-ರಾ-ಗು-ವರು
ಪ್ರಖ್ಯಾ-ತ-ರಾ-ದರು
ಪ್ರಖ್ಯಾ-ತ-ಳಾ-ಗಿದ್ದ
ಪ್ರಖ್ಯಾ-ತ-ವಾದ
ಪ್ರಗ-ತಿಗೆ
ಪ್ರಚಂ-ಡ-ವಾದ
ಪ್ರಚ-ಕ್ಷತೇ
ಪ್ರಚ-ರಂತಿ
ಪ್ರಚಾರ
ಪ್ರಚಾ-ರ-ಗೊ-ಳಿ-ಸ-ಬೇ-ಕೆಂದು
ಪ್ರಚಾ-ರ-ದ-ಲ್ಲಿತ್ತು
ಪ್ರಚಾ-ರ-ದ-ಲ್ಲಿ-ತ್ತೆಂದು
ಪ್ರಚಾ-ರ-ದ-ಲ್ಲಿದ್ದ
ಪ್ರಚಾ-ರ-ದ-ಲ್ಲಿದ್ದು
ಪ್ರಚಾ-ರ-ದ-ಲ್ಲಿ-ರುವ
ಪ್ರಚಾ-ರ-ವಾ-ಗಿಲ್ಲ
ಪ್ರಚೇ-ತ-ಸರ
ಪ್ರಚೇ-ತ-ಸ-ರನ್ನು
ಪ್ರಚೇ-ತ-ಸ-ರಿಗೆ
ಪ್ರಚೇ-ತ-ಸರು
ಪ್ರಚೋ-ದ-ನೆ-ಯಾ-ಗು-ವುದು
ಪ್ರಜಾ-ಗ-ರ-ನೆಂ-ಬು-ವನು
ಪ್ರಜಾ-ಧಿ-ಪ-ತಿ-ಗಳ
ಪ್ರಜಾ-ಧಿ-ಪ-ತಿಗೆ
ಪ್ರಜಾ-ನಾಂ
ಪ್ರಜಾ-ಪತಿ
ಪ್ರಜಾ-ಪ-ತಿಗೆ
ಪ್ರಜಾ-ಪ-ತಿಯ
ಪ್ರಜಾ-ಪ-ತಿಯು
ಪ್ರಜಾ-ಪೀ-ಡ-ಕ-ರಾ-ದ-ವ-ರನ್ನು
ಪ್ರಜಾ-ಭಿ-ವೃದ್ಧಿ
ಪ್ರಜಾ-ಭಿ-ವೃ-ದ್ಧಿ-ಕಾರ್ಯ
ಪ್ರಜಾ-ಭಿ-ವೃ-ದ್ಧಿಗೆ
ಪ್ರಜಾ-ಭಿ-ವೃ-ದ್ಧಿ-ಯಾದ
ಪ್ರಜಾ-ರಂ-ಜ-ಕನು
ಪ್ರಜಾ-ವೃ-ದ್ಧಿ-ಯನ್ನು
ಪ್ರಜಾ-ವೃ-ದ್ಧಿ-ಯಾದ
ಪ್ರಜಾ-ಸಂ-ಖ್ಯೆಯ
ಪ್ರಜಾ-ಸೃ-ಷ್ಟಿ-ಕಾ-ರ್ಯಕ್ಕೆ
ಪ್ರಜಾ-ಸೃ-ಷ್ಟಿ-ಕಾ-ರ್ಯ-ವನ್ನು
ಪ್ರಜಾ-ಸೃ-ಷ್ಟಿಗೆ
ಪ್ರಜಾ-ಸೃ-ಷ್ಟಿಯ
ಪ್ರಜಾ-ಸೃ-ಷ್ಟಿ-ಯನ್ನು
ಪ್ರಜೆ
ಪ್ರಜೆ-ಗಳ
ಪ್ರಜೆ-ಗಳನ್ನು
ಪ್ರಜೆ-ಗಳಲ್ಲಿ
ಪ್ರಜೆ-ಗ-ಳಾ-ಗಿ-ರುವ
ಪ್ರಜೆ-ಗ-ಳಾದ
ಪ್ರಜೆ-ಗ-ಳಾ-ರಿಗೂ
ಪ್ರಜೆ-ಗ-ಳಿಗೆ
ಪ್ರಜೆ-ಗ-ಳಿ-ಗೆಲ್ಲ
ಪ್ರಜೆ-ಗಳು
ಪ್ರಜೆ-ಗಳೂ
ಪ್ರಜೆ-ಗಳೆ
ಪ್ರಜೆ-ಗ-ಳೆಲ್ಲ
ಪ್ರಜೆ-ಗ-ಳೆ-ಲ್ಲರೂ
ಪ್ರಜೆ-ಗ-ಳೊ-ಡನೆ
ಪ್ರಜ್ಞೆ
ಪ್ರಜ್ಞೆ-ತಪ್ಪಿ
ಪ್ರಜ್ಞೆ-ತ-ಪ್ಪು-ವಷ್ಟು
ಪ್ರಜ್ವ-ರನು
ಪ್ರಜ್ವ-ರ-ರೆಂಬ
ಪ್ರಜ್ವ-ಲಿ-ಸು-ವಂತೆ
ಪ್ರಣ-ಮಾಮಿ
ಪ್ರಣಯ
ಪ್ರಣ-ಯ-ಕ-ಟಾ-ಕ್ಷಕ್ಕೆ
ಪ್ರಣ-ಯ-ಕ-ಲಹ
ಪ್ರಣ-ಯ-ಕ-ಲ-ಹ-ಗಳಿಂದ
ಪ್ರಣ-ಯ-ಕೋ-ಪ-ದಿಂದ
ಪ್ರಣ-ಯ-ಗೀ-ತೆ-ಗಳನ್ನು
ಪ್ರಣ-ಯಾ-ವ-ಲೋಕಂ
ಪ್ರಣ-ಯೋ-ತ್ಸಾಹ
ಪ್ರಣಾ-ಶ-ಯಹೇ
ಪ್ರಣಿ-ಧಾನ
ಪ್ರತಪ್ತಂ
ಪ್ರತಾ-ಪ-ವನ್ನು
ಪ್ರತಾ-ಪ-ವಾ-ಯಿತು
ಪ್ರತಿ
ಪ್ರತಿ-ಕೂ-ಲ-ವಾ-ಗಿ-ರು-ವಾಗ
ಪ್ರತಿ-ಕ್ಷ-ಣವೂ
ಪ್ರತಿ-ಗ-ಳೆಲ್ಲ
ಪ್ರತಿಜ್ಞೆ
ಪ್ರತಿ-ಜ್ಞೆ-ಯನ್ನು
ಪ್ರತಿ-ಜ್ಞೆಯೂ
ಪ್ರತಿ-ದಿನ
ಪ್ರತಿ-ದಿ-ನವೂ
ಪ್ರತಿ-ದ್ವಂ-ದ್ವಿ-ಗ-ಳಾಗಿ
ಪ್ರತಿ-ಧ್ವ-ನಿ-ಗೊ-ಳ್ಳು-ವಂತೆ
ಪ್ರತಿ-ಪಾ-ದ-ನೆಗೂ
ಪ್ರತಿ-ಪಾ-ದಿ-ಸಿ-ಲ್ಲವೊ
ಪ್ರತಿ-ಪಾ-ದಿ-ಸುವ
ಪ್ರತಿ-ಫ-ಲದ
ಪ್ರತಿ-ಫ-ಲ-ವನ್ನು
ಪ್ರತಿ-ಬಿಂಬ
ಪ್ರತಿ-ಬಿಂ-ಬ-ವನ್ನು
ಪ್ರತಿ-ಬಿಂ-ಬಿ-ತ-ವಾದ
ಪ್ರತಿ-ಭಾನು
ಪ್ರತಿಮೆ
ಪ್ರತಿ-ಮೆ-ಗ-ಳಂ-ತಾ-ದರು
ಪ್ರತಿ-ಮೆ-ಗಳು
ಪ್ರತಿ-ಮೆ-ಗಳೂ
ಪ್ರತಿ-ಮೆ-ಯಲ್ಲಿ
ಪ್ರತಿ-ಯಾಗಿ
ಪ್ರತಿ-ಯು-ಗ-ದ-ಲ್ಲಿಯೂ
ಪ್ರತಿ-ಯೊಂದು
ಪ್ರತಿ-ಯೊಬ್ಬ
ಪ್ರತಿ-ಯೊ-ಬ್ಬರೂ
ಪ್ರತಿ-ವ-ರ್ಷವೂ
ಪ್ರತಿ-ಶಾಪ
ಪ್ರತಿ-ಶಾ-ಪ-ಗಳನ್ನು
ಪ್ರತಿ-ಶಾ-ಪ-ವನ್ನು
ಪ್ರತಿ-ಷ್ಠಿ-ಸಿದ್ದ
ಪ್ರತಿಷ್ಠೆ
ಪ್ರತಿ-ಷ್ಠೆಗೆ
ಪ್ರತಿ-ಸರ್ಗ
ಪ್ರತಿ-ಸಲ
ಪ್ರತೀ-ಕಾರ
ಪ್ರತೀ-ಕಾ-ರವೂ
ಪ್ರತೋಷ
ಪ್ರತ್ನಸ್ಯ
ಪ್ರತ್ಯಕ್ಷ
ಪ್ರತ್ಯ-ಕ್ಷ-ದ-ರ್ಶ-ನ-ವಾ-ಗು-ತ್ತಲೆ
ಪ್ರತ್ಯ-ಕ್ಷ-ನಾ-ಗಲು
ಪ್ರತ್ಯ-ಕ್ಷ-ನಾ-ಗ-ಲೇ-ಬೇ-ಕಾ-ಯಿತು
ಪ್ರತ್ಯ-ಕ್ಷ-ನಾಗಿ
ಪ್ರತ್ಯ-ಕ್ಷ-ನಾ-ಗಿ-ದ್ದೇನೆ
ಪ್ರತ್ಯ-ಕ್ಷ-ನಾ-ಗಿ-ರು-ವಾಗ
ಪ್ರತ್ಯ-ಕ್ಷ-ನಾಗು
ಪ್ರತ್ಯ-ಕ್ಷ-ನಾ-ಗು-ತ್ತಾನೆ
ಪ್ರತ್ಯ-ಕ್ಷ-ನಾದ
ಪ್ರತ್ಯ-ಕ್ಷ-ನಾ-ದನು
ಪ್ರತ್ಯ-ಕ್ಷ-ಪ್ರ-ಮಾಣ
ಪ್ರತ್ಯ-ಕ್ಷ-ರಾಗಿ
ಪ್ರತ್ಯ-ಕ್ಷ-ರಾ-ಗಿಯೆ
ಪ್ರತ್ಯ-ಕ್ಷ-ರಾ-ದರು
ಪ್ರತ್ಯ-ಕ್ಷ-ಳಾ-ದಳು
ಪ್ರತ್ಯ-ಕ್ಷ-ವಾ-ಗದೆ
ಪ್ರತ್ಯ-ಕ್ಷ-ವಾ-ಗ-ಲಿ-ಲ್ಲ-ವಲ್ಲಾ
ಪ್ರತ್ಯ-ಕ್ಷ-ವಾಗಿ
ಪ್ರತ್ಯ-ಕ್ಷ-ವಾ-ಗಿದೆ
ಪ್ರತ್ಯ-ಕ್ಷ-ವಾ-ಗಿಯೂ
ಪ್ರತ್ಯ-ಕ್ಷ-ವಾ-ದಂತೆ
ಪ್ರತ್ಯ-ಕ್ಷ-ವಾ-ದನು
ಪ್ರತ್ಯ-ಕ್ಷ-ವಾ-ದುವು
ಪ್ರತ್ಯ-ಗಾತ್ಮ
ಪ್ರತ್ಯಾ-ಶೆ-ಯಿಂದ
ಪ್ರತ್ಯಾ-ಹಾರ
ಪ್ರತ್ಯು-ತ್ತ-ರ-ವನ್ನೆ
ಪ್ರತ್ಯು-ತ್ತ-ರ-ವಾ-ಗಿತ್ತು
ಪ್ರತ್ಯು-ತ್ತ-ರ-ವೀ-ಯುತ್ತಾ
ಪ್ರತ್ಯು-ಪ-ಕಾರ
ಪ್ರತ್ಯು-ಪ-ಕಾ-ರ-ವನ್ನು
ಪ್ರತ್ಯೂಷ
ಪ್ರತ್ಯೇಕ
ಪ್ರತ್ಯೇ-ಕ-ವಾಗಿ
ಪ್ರತ್ಯೇ-ಕ-ವಾದ
ಪ್ರತ್ಯೇ-ಕ-ವೆಂ-ಬುದು
ಪ್ರತ್ಯೇ-ಕವೋ
ಪ್ರಥಮಂ
ಪ್ರಥ-ಮಾ-ವ-ತಾ-ರ-ರೂ-ಪಿ-ಯಾ-ಗಿ-ರುವ
ಪ್ರಥಿ-ತಾ-ದ್ಭು-ತ-ಶ-ಕ್ತಿ-ಗಣಂ
ಪ್ರದ-ಕ್ಷಿಣ
ಪ್ರದ-ಕ್ಷಿಣೆ
ಪ್ರದ-ರ್ಶಿಸ
ಪ್ರದಾ-ತನು
ಪ್ರದೇ-ಶ-ಗಳನ್ನು
ಪ್ರದೇ-ಶ-ದಲ್ಲಿ
ಪ್ರದೋ-ಷ-ದ-ಲ್ಲಿಯೂ
ಪ್ರದ್ಯು
ಪ್ರದ್ಯುಮ್ನ
ಪ್ರದ್ಯು-ಮ್ನ-ಸಾಂಬ
ಪ್ರದ್ಯು-ಮ್ನ-ಅ-ಹುದು
ಪ್ರದ್ಯು-ಮ್ನನ
ಪ್ರದ್ಯು-ಮ್ನ-ನನ್ನು
ಪ್ರದ್ಯು-ಮ್ನ-ನಿಂದ
ಪ್ರದ್ಯು-ಮ್ನ-ನಿಗೆ
ಪ್ರದ್ಯು-ಮ್ನನು
ಪ್ರದ್ಯು-ಮ್ನನೂ
ಪ್ರದ್ಯು-ಮ್ನ-ರೂ-ಪ-ದಿಂದ
ಪ್ರದ್ಯು-ಮ್ನಾ-ಯಾ-ಽನಿ-ರು-ದ್ಧಾಯ
ಪ್ರದ್ಯೋ-ತನೂ
ಪ್ರದ್ಯೋ-ತ-ನೆಂಬ
ಪ್ರಧಾನ
ಪ್ರಧಾ-ನನೂ
ಪ್ರಧಾ-ನ-ವಸ್ತು
ಪ್ರಧಾ-ನ-ವಾದ
ಪ್ರಪಂಚ
ಪ್ರಪಂ-ಚಕ್ಕೆ
ಪ್ರಪಂ-ಚ-ಕ್ಕೆಲ್ಲ
ಪ್ರಪಂ-ಚದ
ಪ್ರಪಂ-ಚ-ದಲ್ಲಿ
ಪ್ರಪಂ-ಚ-ಪ್ರಜ್ಞೆ
ಪ್ರಪಂ-ಚ-ವನ್ನು
ಪ್ರಪಂ-ಚ-ವೆಂಬ
ಪ್ರಪಂ-ಚ-ವೆ-ಲ್ಲವೂ
ಪ್ರಪಂ-ಚ-ವೊಂ-ದನ್ನು
ಪ್ರಪಂ-ಚ-ಸೃಷ್ಟಿ
ಪ್ರಪಂಚಾ
ಪ್ರಪಂ-ಚಾ-ತ್ಮಕ
ಪ್ರಬಲ
ಪ್ರಬ-ಲ-ನಾ-ದುದು
ಪ್ರಬ-ಲ-ರಾದ
ಪ್ರಬ-ಲಿ-ಸು-ತ್ತದೆ
ಪ್ರಬುದ್ಧ
ಪ್ರಭಾತೇ
ಪ್ರಭಾನು
ಪ್ರಭಾವ
ಪ್ರಭಾ-ವ-ದಿಂದ
ಪ್ರಭಾ-ವ-ದಿಂ-ದಲೇ
ಪ್ರಭಾ-ವ-ವನ್ನು
ಪ್ರಭಾ-ವ-ವೆಂದು
ಪ್ರಭಾ-ಶ-ರೀ-ರ-ವನ್ನು
ಪ್ರಭಾ-ಸ-ಕ್ಷೇ-ತಕ್ಕೆ
ಪ್ರಭಾ-ಸ-ಕ್ಷೇ-ತ್ರಕ್ಕೆ
ಪ್ರಭಾ-ಸ-ಕ್ಷೇ-ತ್ರದ
ಪ್ರಭಾ-ಸ-ಕ್ಷೇ-ತ್ರ-ದ-ಲ್ಲಿ-ರುವ
ಪ್ರಭಾ-ಸ-ತೀ-ರ್ಥಕ್ಕೆ
ಪ್ರಭಾ-ಸ-ತೀ-ರ್ಥ-ವನ್ನು
ಪ್ರಭು
ಪ್ರಭುಃ
ಪ್ರಭು-ಗ-ಳೆಂ-ದು-ಕೊ-ಳ್ಳು-ತ್ತಿ-ದ್ದೇವೆ
ಪ್ರಭು-ತ್ವ-ದಲ್ಲಿ
ಪ್ರಭು-ವಾ-ಗಿ-ರು-ವನು
ಪ್ರಭು-ವಿ-ನಂತೆ
ಪ್ರಭೋ
ಪ್ರಮ-ಥನು
ಪ್ರಮ-ಥರ
ಪ್ರಮ-ಥ-ರಲ್ಲಿ
ಪ್ರಮ-ಥ-ರಿಂದ
ಪ್ರಮ-ಥರು
ಪ್ರಮಾಣ
ಪ್ರಮಾ-ಣ-ಗಳನ್ನು
ಪ್ರಮಾ-ಣ-ದಂ-ತಿದ್ದ
ಪ್ರಮಾ-ಣ-ದಲ್ಲಿ
ಪ್ರಮಾ-ಣ-ದಿಂದ
ಪ್ರಮಾ-ಣ-ದಿಂ-ದಲೇ
ಪ್ರಮಾ-ಣ-ವಲ್ಲ
ಪ್ರಮಾ-ಣವೇ
ಪ್ರಮಾ-ದ-ಗಳಿಂದ
ಪ್ರಮುಖ
ಪ್ರಮು-ಖ-ರಾ-ದರು
ಪ್ರಮು-ಖ-ರಾ-ದ-ವರು
ಪ್ರಮು-ಖರು
ಪ್ರಮ್ಲೋಚೆ
ಪ್ರಯ
ಪ್ರಯತ್ನ
ಪ್ರಯ-ತ್ನ-ಗಳನ್ನು
ಪ್ರಯ-ತ್ನ-ಗಳೂ
ಪ್ರಯ-ತ್ನ-ಗ-ಳೆ-ಲ್ಲವೂ
ಪ್ರಯ-ತ್ನ-ದಂತೆ
ಪ್ರಯ-ತ್ನ-ದಲ್ಲಿ
ಪ್ರಯ-ತ್ನ-ದ-ಲ್ಲಿ-ರು-ವು-ದಾಗಿ
ಪ್ರಯ-ತ್ನ-ಮಾ-ಡಿ-ದರೂ
ಪ್ರಯ-ತ್ನ-ಮಾ-ಡಿ-ದ್ದೇನೆ
ಪ್ರಯ-ತ್ನ-ವನ್ನು
ಪ್ರಯ-ತ್ನವೂ
ಪ್ರಯ-ತ್ನ-ವೆಲ್ಲ
ಪ್ರಯ-ತ್ನ-ವೇನೂ
ಪ್ರಯ-ತ್ನಿಸ
ಪ್ರಯ-ತ್ನಿಸಿ
ಪ್ರಯ-ತ್ನಿ-ಸಿ-ತಾ-ದರೂ
ಪ್ರಯ-ತ್ನಿ-ಸಿತು
ಪ್ರಯ-ತ್ನಿ-ಸಿ-ದನು
ಪ್ರಯ-ತ್ನಿ-ಸಿ-ದರು
ಪ್ರಯ-ತ್ನಿ-ಸಿ-ದರೂ
ಪ್ರಯ-ತ್ನಿ-ಸಿ-ದಳು
ಪ್ರಯ-ತ್ನಿ-ಸಿ-ದ್ದೇನೆ
ಪ್ರಯ-ತ್ನಿ-ಸು-ತ್ತಿ-ದ್ದೇವೆ
ಪ್ರಯ-ತ್ನಿ-ಸು-ತ್ತಿ-ರುವು
ಪ್ರಯ-ತ್ನಿ-ಸು-ವನೆ
ಪ್ರಯ-ತ್ನಿ-ಸು-ವ-ವನು
ಪ್ರಯ-ತ್ನಿ-ಸು-ವ-ವರೇ
ಪ್ರಯ-ತ್ನಿ-ಸು-ವಿಯಾ
ಪ್ರಯ-ತ್ನಿ-ಸುವು
ಪ್ರಯ-ತ್ನಿ-ಸು-ವು-ದೆಲ್ಲ
ಪ್ರಯ-ತ್ನಿ-ಸು-ವು-ದೊಂದೆ
ಪ್ರಯ-ತ್ನಿ-ಸೋಣ
ಪ್ರಯಾಂತು
ಪ್ರಯಾಗ
ಪ್ರಯಾಣ
ಪ್ರಯಾ-ಣ-ಕಾ-ಲ-ದಲ್ಲಿ
ಪ್ರಯಾ-ಣ-ಮಾಡಿ
ಪ್ರಯಾ-ಣ-ಮಾ-ಡಿದ
ಪ್ರಯಾ-ಣ-ಮಾ-ಡಿ-ದನು
ಪ್ರಯಾ-ಣ-ಮಾಡು
ಪ್ರಯಾ-ಣ-ಮಾ-ಡುತ್ತಾ
ಪ್ರಯಾ-ಣ-ವನ್ನು
ಪ್ರಯಾ-ಣ-ಸ-ನ್ನದ್ಧ
ಪ್ರಯಾ-ಣ-ಸ-ನ್ನ-ದ್ಧ-ರಾ-ಗು-ತ್ತಿ-ರಲು
ಪ್ರಯೋ
ಪ್ರಯೋಗ
ಪ್ರಯೋ-ಗ-ಶಾಲೆ
ಪ್ರಯೋ-ಗ-ಶಾ-ಲೆ-ಯಲ್ಲಿ
ಪ್ರಯೋ-ಗಿ-ಸ-ಬೇಕು
ಪ್ರಯೋ-ಗಿ-ಸ-ಲ್ಪಟ್ಟ
ಪ್ರಯೋ-ಗಿಸಿ
ಪ್ರಯೋ-ಗಿ-ಸಿದ
ಪ್ರಯೋ-ಗಿ-ಸಿ-ದನು
ಪ್ರಯೋ-ಗಿ-ಸಿ-ದಾಗ
ಪ್ರಯೋ-ಗಿ-ಸು-ತ್ತಲೆ
ಪ್ರಯೋ-ಜನ
ಪ್ರಯೋ-ಜ-ನ-ವನ್ನು
ಪ್ರಯೋ-ಜ-ನ-ವಿಲ್ಲ
ಪ್ರಲಂಬ
ಪ್ರಲಂ-ಬನು
ಪ್ರಲಂ-ಬ-ನೆಂಬ
ಪ್ರಲಂ-ಬಾ-ಸು-ರ-ರೆಲ್ಲ
ಪ್ರಳಯ
ಪ್ರಳ-ಯ-ಕಾ-ಲದ
ಪ್ರಳ-ಯ-ಕಾ-ಲ-ದಲ್ಲಿ
ಪ್ರಳ-ಯ-ಕಾ-ಲ-ದ-ವ-ರೆಗೂ
ಪ್ರಳ-ಯಕ್ಕೆ
ಪ್ರಳ-ಯ-ಗಳ
ಪ್ರಳ-ಯ-ಜಲ
ಪ್ರಳ-ಯದ
ಪ್ರಳ-ಯ-ದಲ್ಲಿ
ಪ್ರಳ-ಯ-ರು-ದ್ರನ
ಪ್ರಳ-ಯ-ವನ್ನು
ಪ್ರಳ-ಯ-ವಾ-ಗು-ವಂ-ತಾಯ್ತು
ಪ್ರಳ-ಯವೂ
ಪ್ರಳ-ಯವೇ
ಪ್ರಳ-ಯಾಂ-ತ್ಯ-ದಲ್ಲಿ
ಪ್ರಳ-ಯಾ-ಗ್ನಿ-ಯಂತೆ
ಪ್ರವ-ಚನ
ಪ್ರವ-ಚ-ನ-ಗಳನ್ನು
ಪ್ರವ-ಚ-ನ-ವನ್ನು
ಪ್ರವ-ರ್ತಿ-ಸದೆ
ಪ್ರವ-ರ್ತಿ-ಸು-ತ್ತವೆ
ಪ್ರವ-ರ್ಷ-ವೆಂಬ
ಪ್ರವಹಿ
ಪ್ರವ-ಹಿಸಿ
ಪ್ರವ-ಹಿ-ಸಿ-ದೆ-ಡೆ-ಯ-ನ್ನೆಲ್ಲ
ಪ್ರವ-ಹಿ-ಸು-ತ್ತಿವೆ
ಪ್ರವ-ಹಿ-ಸು-ವಂತೆ
ಪ್ರವಾಹ
ಪ್ರವಾ-ಹ-ದಂತೆ
ಪ್ರವಾ-ಹ-ದಲ್ಲಿ
ಪ್ರವಾ-ಹ-ವೆಂದರೆ
ಪ್ರವಿಶ್ಯ
ಪ್ರವಿ-ಸ್ತ-ರ-ವಾಗಿ
ಪ್ರವೃತ್ತಿ
ಪ್ರವೃ-ತ್ತಿ-ಧರ್ಮ
ಪ್ರವೃ-ತ್ತಿ-ಧ-ರ್ಮ-ವನ್ನು
ಪ್ರವೇ
ಪ್ರವೇ-ಶಕ್ಕೂ
ಪ್ರವೇ-ಶಕ್ಕೆ
ಪ್ರವೇಶಿ
ಪ್ರವೇ-ಶಿ-ಸದೆ
ಪ್ರವೇ-ಶಿ-ಸಲು
ಪ್ರವೇ-ಶಿಸಿ
ಪ್ರವೇ-ಶಿ-ಸಿತು
ಪ್ರವೇ-ಶಿ-ಸಿದ
ಪ್ರವೇ-ಶಿ-ಸಿ-ದನು
ಪ್ರವೇ-ಶಿ-ಸಿ-ದರು
ಪ್ರವೇ-ಶಿ-ಸಿ-ದರೆ
ಪ್ರವೇ-ಶಿ-ಸಿ-ದಳು
ಪ್ರವೇ-ಶಿ-ಸಿ-ದ-ವರೆ-ಲ್ಲರೂ
ಪ್ರವೇ-ಶಿ-ಸಿ-ರು-ವನು
ಪ್ರವೇ-ಶಿ-ಸಿ-ರು-ವ-ನೆಂಬ
ಪ್ರವೇ-ಶಿ-ಸು-ತ್ತದೆ
ಪ್ರವೇ-ಶಿ-ಸು-ತ್ತಲೆ
ಪ್ರವೇ-ಶಿ-ಸು-ತ್ತಾನೆ
ಪ್ರವೇ-ಶಿ-ಸು-ತ್ತಿದ್ದ
ಪ್ರವೇ-ಶಿ-ಸು-ತ್ತಿ-ದ್ದಂತೆ
ಪ್ರವೇ-ಶಿ-ಸು-ತ್ತೇವೆ
ಪ್ರವೇ-ಶಿ-ಸು-ವಂ-ತಿಲ್ಲ
ಪ್ರವೇ-ಶಿ-ಸು-ವನು
ಪ್ರಶಸ್ತಿ
ಪ್ರಶಾಂತ
ಪ್ರಶಾಂ-ತ-ನ-ನ್ನಾಗಿ
ಪ್ರಶಾಂ-ತ-ನಾಗಿ
ಪ್ರಶಾಂ-ತ-ನಾ-ಗಿ-ದ್ದನು
ಪ್ರಶಾಂ-ತ-ಮೂ-ರ್ತಿ-ಯಾಗಿ
ಪ್ರಶಾಂ-ತ-ವಾದ
ಪ್ರಶ್ನಿ-ಸಿದ
ಪ್ರಶ್ನಿ-ಸಿ-ದನು
ಪ್ರಶ್ನಿ-ಸು-ತ್ತಾನೆ
ಪ್ರಶ್ನಿ-ಸುವ
ಪ್ರಶ್ನೆ
ಪ್ರಶ್ನೆ-ಗ-ಳಿಗೆ
ಪ್ರಶ್ನೆಗೆ
ಪ್ರಶ್ನೆ-ಭಗ
ಪ್ರಶ್ನೆ-ಮಾ-ಡುತ್ತಾ
ಪ್ರಶ್ನೆ-ಯನ್ನು
ಪ್ರಶ್ನೆ-ಯನ್ನೂ
ಪ್ರಶ್ನೆ-ಯಿಂದ
ಪ್ರಶ್ನೆ-ಯಿಲ್ಲ
ಪ್ರಶ್ನೆಯೆ
ಪ್ರಶ್ನೆ-ಯೆಲ್ಲಿ
ಪ್ರಶ್ನೆಯೇ
ಪ್ರಶ್ರಿತ
ಪ್ರಸಂಗ
ಪ್ರಸಂ-ಗ-ಗ-ಳಂತೆ
ಪ್ರಸಂ-ಗ-ದಲ್ಲಿ
ಪ್ರಸಂ-ಗ-ವನ್ನು
ಪ್ರಸಕ್ತಿ
ಪ್ರಸನ್ನ
ಪ್ರಸ-ನ್ನ-ದೃ-ಷ್ಟಿ-ಯಿಂದ
ಪ್ರಸ-ನ್ನ-ನಾ-ಗ-ಲಿಲ್ಲ
ಪ್ರಸ-ನ್ನ-ನಾಗಿ
ಪ್ರಸ-ನ್ನ-ವಾದ
ಪ್ರಸಾದಂ
ಪ್ರಸಾ-ದ-ವನ್ನು
ಪ್ರಸಾ-ದ-ವಾಗಿ
ಪ್ರಸಿ-ದ್ಧ-ನಾ-ಗಿ-ದ್ದಾನೆ
ಪ್ರಸಿ-ದ್ಧ-ನಾದ
ಪ್ರಸಿ-ದ್ಧ-ರಾ-ದರು
ಪ್ರಸಿ-ದ್ಧ-ವಾ-ಗಿದೆ
ಪ್ರಸಿ-ದ್ಧ-ವಾದ
ಪ್ರಸಿ-ದ್ಧ-ವಾ-ಯಿತು
ಪ್ರಸಿ-ದ್ಧಿ-ಯಾ-ಗಲಿ
ಪ್ರಸೂತಿ
ಪ್ರಸೂ-ತಿ-ಎಂಬ
ಪ್ರಸೂ-ತಿ-ವಾ-ಯು-ವಿನ
ಪ್ರಸೇನ
ಪ್ರಸೇ-ನ-ನನ್ನು
ಪ್ರಸೇ-ನ-ನೆಂ-ಬು-ವನು
ಪ್ರಸ್ತಾಪ
ಪ್ರಸ್ತಾ-ಪಿ-ತ-ವಾ-ಗಿದೆ
ಪ್ರಸ್ಥ-ಪು-ರದ
ಪ್ರಹ-ರಣ
ಪ್ರಹಸ್ತ
ಪ್ರಹಾ-ಪ-ಯನ್
ಪ್ರಹಾ-ರ-ಗಳನ್ನೆಲ್ಲ
ಪ್ರಹ್ಮಾದ
ಪ್ರಹ್ಲಾದ
ಪ್ರಹ್ಲಾ-ದ-ಕು-ಮಾರ
ಪ್ರಹ್ಲಾ-ದನ
ಪ್ರಹ್ಲಾ-ದ-ನಂ-ತಹ
ಪ್ರಹ್ಲಾ-ದ-ನಂತೆ
ಪ್ರಹ್ಲಾ-ದ-ನನ್ನು
ಪ್ರಹ್ಲಾ-ದ-ನನ್ನೂ
ಪ್ರಹ್ಲಾ-ದ-ನಿಗೆ
ಪ್ರಹ್ಲಾ-ದನು
ಪ್ರಹ್ಲಾ-ದನೂ
ಪ್ರಹ್ಲಾ-ದನೇ
ಪ್ರಾಂತ-ದಲ್ಲಿ
ಪ್ರಾಕೃ-ತ-ರಿ-ಗೆಲ್ಲಾ
ಪ್ರಾಕೃ-ತ-ಸೃ-ಷ್ಟಿಗೆ
ಪ್ರಾಕೃ-ತ-ಸೃ-ಷ್ಟಿ-ಯೆಂದು
ಪ್ರಾಕೃ-ತಿಕ
ಪ್ರಾಗ್ಜೋ-ತಿ-ಷ-ಪು-ರದ
ಪ್ರಾಗ್ಜೋ-ತಿ-ಷ-ಪು-ರ-ದಿಂದ
ಪ್ರಾಗ್ಜೋ-ತಿ-ಷ-ವೆಂಬ
ಪ್ರಾಚೀನ
ಪ್ರಾಚೀ-ನ-ಕಾ-ಲ-ದಲ್ಲಿ
ಪ್ರಾಚೀ-ನ-ನೆಂದು
ಪ್ರಾಚೀ-ನ-ಬರ್ಹಿ
ಪ್ರಾಚೀ-ನ-ಬ-ರ್ಹಿಗೆ
ಪ್ರಾಚೀ-ನ-ಬ-ರ್ಹಿ-ಯನ್ನು
ಪ್ರಾಚೀ-ನ-ಬ-ರ್ಹಿಯು
ಪ್ರಾಚೀ-ನರು
ಪ್ರಾಚೀ-ನ-ವಾ-ದು-ದೆಂ-ದಾ-ಗು-ತ್ತದೆ
ಪ್ರಾಚೀ-ನ-ವೆಂಬು
ಪ್ರಾಚೀ-ನ-ವೆಂ-ಬು-ದ-ರಲ್ಲಿ
ಪ್ರಾಚೇ-ತಸ
ಪ್ರಾಚೇ-ತ-ಸ-ಕ-ನ್ನೆ-ಯ-ರಲ್ಲಿ
ಪ್ರಾಚೇ-ತ-ಸನ
ಪ್ರಾಚೇ-ತ-ಸ-ನಾಗಿ
ಪ್ರಾಚೇ-ತ-ಸ-ನಿಗೆ
ಪ್ರಾಚೇ-ತ-ಸನು
ಪ್ರಾಚೇ-ತ-ಸ-ನೆಂಬ
ಪ್ರಾಜ್ಞೆ
ಪ್ರಾಣ
ಪ್ರಾಣ-ಇ-ವನ್ನು
ಪ್ರಾಣಕ್ಕೆ
ಪ್ರಾಣ-ಗಳು
ಪ್ರಾಣ-ಗಳೇ
ಪ್ರಾಣ-ಗ-ಳೊ-ಡನೆ
ಪ್ರಾಣ-ತ್ಯಾಗ
ಪ್ರಾಣದ
ಪ್ರಾಣ-ದಾನ
ಪ್ರಾಣ-ದಿಂದ
ಪ್ರಾಣ-ಧಾ-ರ-ಣೆಗೆ
ಪ್ರಾಣ-ಪ್ರಿಯ
ಪ್ರಾಣ-ಪ್ರಿ-ಯ-ನಾದ
ಪ್ರಾಣ-ಪ್ರಿ-ಯನು
ಪ್ರಾಣ-ಪ್ರಿ-ಯೆ-ಯಾ-ದಳು
ಪ್ರಾಣ-ಬಿಟ್ಟ
ಪ್ರಾಣ-ಭಯ
ಪ್ರಾಣ-ಭ-ಯ-ದಿಂದ
ಪ್ರಾಣ-ಭ-ಯ-ವನ್ನು
ಪ್ರಾಣ-ರ-ಕ್ಷ-ಣೆ-ಗಾಗಿ
ಪ್ರಾಣ-ವನ್ನು
ಪ್ರಾಣ-ವಾಯು
ಪ್ರಾಣ-ವಾ-ಯು-ಗಳನ್ನು
ಪ್ರಾಣ-ವಾ-ಯು-ಗಳಿಂದ
ಪ್ರಾಣ-ವಾ-ಯು-ವನ್ನು
ಪ್ರಾಣ-ವಾ-ಯು-ವನ್ನೇ
ಪ್ರಾಣ-ವಾ-ಯು-ವಿನ
ಪ್ರಾಣ-ವಾ-ಯುವೂ
ಪ್ರಾಣ-ವಿ-ಟ್ಟು-ಕೊಂ-ಡಿ-ರು-ವರು
ಪ್ರಾಣ-ವಿ-ಟ್ಟು-ಕೊಂ-ಡಿ-ರು-ವ-ವರು
ಪ್ರಾಣವೂ
ಪ್ರಾಣ-ಸಂ-ಕ-ಟ-ದಿಂದ
ಪ್ರಾಣ-ಸ-ಮಾ-ನ-ನಾದ
ಪ್ರಾಣ-ಸ್ವ-ರೂ-ಪಿ-ಯಾಗಿ
ಪ್ರಾಣ-ಹೀ-ರಲು
ಪ್ರಾಣ-ಹೋ-ಗು-ತ್ತಿದೆ
ಪ್ರಾಣಾ
ಪ್ರಾಣಾ-ದಿ-ವೃ-ತ್ತಿ-ಗಳಿಂದ
ಪ್ರಾಣಾನ್
ಪ್ರಾಣಾ-ಯಾಮ
ಪ್ರಾಣಾ-ಯಾ-ಮ-ದಿಂದ
ಪ್ರಾಣಾ-ಯಾ-ಮ-ವನ್ನು
ಪ್ರಾಣಾ-ಯಾ-ಮವು
ಪ್ರಾಣಿ
ಪ್ರಾಣಿ-ಗಳ
ಪ್ರಾಣಿ-ಗಳನ್ನು
ಪ್ರಾಣಿ-ಗಳನ್ನೂ
ಪ್ರಾಣಿ-ಗ-ಳ-ಲ್ಲಿಯೂ
ಪ್ರಾಣಿ-ಗ-ಳಾಗಿ
ಪ್ರಾಣಿ-ಗ-ಳಿ-ಗಿಂತ
ಪ್ರಾಣಿ-ಗ-ಳಿಗೂ
ಪ್ರಾಣಿ-ಗ-ಳಿಗೆ
ಪ್ರಾಣಿ-ಗಳು
ಪ್ರಾಣಿ-ಗಳೂ
ಪ್ರಾಣಿ-ಗ-ಳೆ-ರಡೂ
ಪ್ರಾಣಿ-ಗ-ಳೆಲ್ಲ
ಪ್ರಾಣಿಗೆ
ಪ್ರಾಣಿ-ಯನ್ನೂ
ಪ್ರಾಣಿಯೂ
ಪ್ರಾಣಿ-ವರ್ಗ
ಪ್ರಾಣಿ-ಹಿಂ-ಸೆಯೇ
ಪ್ರಾತಃ-ಕಾಲ
ಪ್ರಾತಃ-ಕಾ-ಲ-ದ-ಲ್ಲಿಯೂ
ಪ್ರಾತಃ-ಕೃ-ತ್ಯ-ಗ-ಳಿ-ಗೆಂದು
ಪ್ರಾತ-ರ-ವ್ಯಾತ್
ಪ್ರಾತ-ರಾ-ಹ್ನಿ-ಕ-ಗಳನ್ನು
ಪ್ರಾಪ್ತ
ಪ್ರಾಪ್ತ-ವ-ಯ-ಸ್ಕ-ನಾ-ಗು-ತ್ತಲೆ
ಪ್ರಾಪ್ತ-ವಾ-ಗಲಿ
ಪ್ರಾಪ್ತ-ವಾ-ಗ-ಲೆಂದು
ಪ್ರಾಪ್ತ-ವಾ-ಗಿತ್ತು
ಪ್ರಾಪ್ತ-ವಾಗು
ಪ್ರಾಪ್ತ-ವಾ-ದಂತೆ
ಪ್ರಾಪ್ತ-ವಾ-ದರೆ
ಪ್ರಾಪ್ತಾಃ
ಪ್ರಾಪ್ತಿ-ಯಾ-ಗಲಿ
ಪ್ರಾಪ್ತಿ-ಯಾ-ಗಿತ್ತು
ಪ್ರಾಪ್ತಿ-ಯಾಗು
ಪ್ರಾಪ್ಯ
ಪ್ರಾಮು
ಪ್ರಾಯದ
ಪ್ರಾಯ-ದ-ವ-ನಾ-ದನು
ಪ್ರಾಯ-ವೇ-ನೆಂ-ಬು-ದನ್ನು
ಪ್ರಾಯ-ಶ್ಚಿತ್ತ
ಪ್ರಾಯ-ಶ್ಚಿ-ತ್ತ-ಕ್ಕಿಂತ
ಪ್ರಾಯ-ಶ್ಚಿ-ತ್ತದ
ಪ್ರಾಯ-ಶ್ಚಿ-ತ್ತ-ದಿಂದ
ಪ್ರಾಯ-ಶ್ಚಿ-ತ್ತ-ವೆಂ-ಬುದು
ಪ್ರಾಯೋ-ಗಿಕ
ಪ್ರಾಯೋಪ
ಪ್ರಾಯೋ-ಪ-ವೇಶ
ಪ್ರಾಯೋ-ಪ-ವೇ-ಶ-ಕ್ಕಾಗಿ
ಪ್ರಾಯೋ-ಪ-ವೇ-ಶ-ಮಾಡಿ
ಪ್ರಾರಂಭ
ಪ್ರಾರಂ-ಭ-ದಲ್ಲಿ
ಪ್ರಾರಂ-ಭ-ದ-ಲ್ಲಿಯೆ
ಪ್ರಾರಂ-ಭ-ವಾಗಿ
ಪ್ರಾರಂ-ಭ-ವಾ-ಗಿದೆ
ಪ್ರಾರಂ-ಭ-ವಾ-ಗು-ತ್ತದೆ
ಪ್ರಾರಂ-ಭ-ವಾ-ಗು-ತ್ತಲೆ
ಪ್ರಾರಂ-ಭ-ವಾ-ಗುವ
ಪ್ರಾರಂ-ಭ-ವಾ-ಗು-ವವು
ಪ್ರಾರಂ-ಭ-ವಾ-ಯಿತು
ಪ್ರಾರಂ-ಭ-ವಾ-ಯಿ-ತೆಂ-ದು-ಕೊಂ-ಡನು
ಪ್ರಾರಂಭಿ
ಪ್ರಾರಂ-ಭಿಸಿ
ಪ್ರಾರಂ-ಭಿ-ಸಿತು
ಪ್ರಾರಂ-ಭಿ-ಸಿದ
ಪ್ರಾರಂ-ಭಿ-ಸಿ-ದನು
ಪ್ರಾರಂ-ಭಿ-ಸಿ-ದರು
ಪ್ರಾರಂ-ಭಿ-ಸಿ-ದಳು
ಪ್ರಾರಂ-ಭಿ-ಸಿ-ದವು
ಪ್ರಾರಂ-ಭಿ-ಸಿ-ದ್ದಾನೆ
ಪ್ರಾರಂ-ಭಿ-ಸಿ-ದ್ದಾರೆ
ಪ್ರಾರಂ-ಭಿ-ಸು-ತ್ತಲೆ
ಪ್ರಾರಂ-ಭಿ-ಸು-ವಂ-ತಿಲ್ಲ
ಪ್ರಾರಬ್ಧ
ಪ್ರಾರ-ಬ್ಧ-ಕರ್ಮ
ಪ್ರಾರ-ಬ್ಧ-ಕ-ರ್ಮ-ವನ್ನು
ಪ್ರಾರ-ಬ್ಧ-ಕ-ರ್ಮ-ವಾಗಿ
ಪ್ರಾರ್ಥನಾ
ಪ್ರಾರ್ಥನೆ
ಪ್ರಾರ್ಥ-ನೆಗೆ
ಪ್ರಾರ್ಥ-ನೆ-ಯಂತೆ
ಪ್ರಾರ್ಥ-ನೆ-ಯನ್ನು
ಪ್ರಾರ್ಥ-ನೆ-ಯಿಂದ
ಪ್ರಾರ್ಥ-ನೆ-ಯೆಲ್ಲ
ಪ್ರಾರ್ಥಿಸಿ
ಪ್ರಾರ್ಥಿ-ಸಿ-ದನು
ಪ್ರಾರ್ಥಿ-ಸಿ-ದರು
ಪ್ರಾರ್ಥಿ-ಸಿ-ದಳು
ಪ್ರಾರ್ಥಿ-ಸಿ-ದೆ-ನಲ್ಲಾ
ಪ್ರಾರ್ಥಿಸು
ಪ್ರಾರ್ಥಿ-ಸುತ್ತಾ
ಪ್ರಾರ್ಥಿ-ಸೋಣ
ಪ್ರಾಶಸ್ತ್ಯ
ಪ್ರಾಶ-ಸ್ತ್ಯ-ವನ್ನು
ಪ್ರಾಸ್ತಿ-ಯರು
ಪ್ರಾಹ್ಣ
ಪ್ರಿಯ
ಪ್ರಿಯಂ
ಪ್ರಿಯ-ಕ-ರ-ನಾದ
ಪ್ರಿಯ-ಕ-ರ-ವಾದ
ಪ್ರಿಯ-ತ-ಮ-ನಾದ
ಪ್ರಿಯ-ನಾದ
ಪ್ರಿಯ-ಪ್ರ-ಸ್ಥಾ-ಪಿತಂ
ಪ್ರಿಯ-ರ-ಮ-ಣ-ನಾದ
ಪ್ರಿಯ-ರಾ-ದ-ವರು
ಪ್ರಿಯ-ರಾವ
ಪ್ರಿಯ-ಳಾ-ದ-ವಳು
ಪ್ರಿಯ-ವ-ನ್ನುಂ-ಟು-ಮಾ-ಡುವೆ
ಪ್ರಿಯ-ವಾಗಿ
ಪ್ರಿಯ-ವಾ-ಗು-ತ್ತದೆ
ಪ್ರಿಯ-ವಾದ
ಪ್ರಿಯ-ವಾ-ದುದು
ಪ್ರಿಯವೇ
ಪ್ರಿಯ-ವ್ರತ
ಪ್ರಿಯ-ವ್ರ-ತ-ಕೃತಂ
ಪ್ರಿಯ-ವ್ರ-ತನ
ಪ್ರಿಯ-ವ್ರ-ತ-ನಿಗೆ
ಪ್ರಿಯ-ವ್ರ-ತನು
ಪ್ರಿಯ-ವ್ರ-ತ-ನೊ-ಡನೆ
ಪ್ರಿಯ-ವ್ರ-ತ-ಪು-ತ್ರನೇ
ಪ್ರಿಯ-ವ್ರ-ತ-ರಾ-ಜನ
ಪ್ರಿಯ-ವ್ರ-ತ-ರಾ-ಜನು
ಪ್ರಿಯ-ವ್ರ-ತರು
ಪ್ರಿಯ-ವ್ರತಾ
ಪ್ರಿಯ-ಸಖ
ಪ್ರಿಯಾಂ
ಪ್ರಿಯಾ-ಪ್ರಿಯ
ಪ್ರೀತ-ನಾ-ಗು-ತ್ತೇನೆ
ಪ್ರೀತಿ
ಪ್ರೀತಿ-ಗಾಗಿ
ಪ್ರೀತಿಗೆ
ಪ್ರೀತಿ-ಪಾ-ತ್ರ-ಳಾದ
ಪ್ರೀತಿಯ
ಪ್ರೀತಿ-ಯ-ನ್ನಿ-ಡ-ಬೇಕು
ಪ್ರೀತಿ-ಯನ್ನು
ಪ್ರೀತಿ-ಯಾ-ಗ-ಲೆಂದು
ಪ್ರೀತಿ-ಯಿಂದ
ಪ್ರೀತಿ-ಯಿಂ-ದಲೆ
ಪ್ರೀತಿ-ಯಿ-ದ್ದರೆ
ಪ್ರೀತಿ-ಯುಂ-ಟಾ-ಗು-ವಂತೆ
ಪ್ರೀತಿ-ಯುಳ್ಳ
ಪ್ರೀತಿಸ
ಪ್ರೀತಿ-ಸಿ-ದರು
ಪ್ರೀತಿ-ಸಿ-ದರೆ
ಪ್ರೀತಿ-ಸುತ್ತಾ
ಪ್ರೀತಿ-ಸುವ
ಪ್ರೀತಿ-ಸು-ವುದು
ಪ್ರೇಕ್ಷಕ
ಪ್ರೇತ
ಪ್ರೇತ-ಕ-ರ್ಮ-ಗಳನ್ನು
ಪ್ರೇತ-ಕಾ-ರ್ಯ-ಗಳನ್ನು
ಪ್ರೇತ-ಗಳು
ಪ್ರೇಮ
ಪ್ರೇಮ-ಇ-ವು-ಗಳನ್ನು
ಪ್ರೇಮ-ಕಥೆ
ಪ್ರೇಮಕ್ಕೆ
ಪ್ರೇಮದ
ಪ್ರೇಮ-ದಲ್ಲಿ
ಪ್ರೇಮ-ದಿಂದ
ಪ್ರೇಮ-ದಿಂ-ದಿ-ರುವ
ಪ್ರೇಮ-ದೋರು
ಪ್ರೇಮ-ಧಾರೆ
ಪ್ರೇಮ-ಪತ್ರ
ಪ್ರೇಮ-ಫಲ
ಪ್ರೇಮ-ಬದ್ಧಃ
ಪ್ರೇಮ-ಮ-ಯ-ನಾದ
ಪ್ರೇಮ-ಮ-ಯ-ರಾದ
ಪ್ರೇಮ-ಮ-ಯ-ವಾದ
ಪ್ರೇಮ-ಮಯಿ
ಪ್ರೇಮ-ರ-ಸ-ವನ್ನು
ಪ್ರೇಮ-ವನ್ನು
ಪ್ರೇಮ-ವಿತ್ತು
ಪ್ರೇಮ-ವಿ-ರು-ವಂತೆ
ಪ್ರೇಮ-ವು-ಳ್ಳ-ನಾಗಿ
ಪ್ರೇಮ-ವು-ಳ್ಳ-ವ-ನಾ-ದರೂ
ಪ್ರೇಮ-ವು-ಳ್ಳ-ವರು
ಪ್ರೇಮ-ವೆಂದರೆ
ಪ್ರೇಮವೇ
ಪ್ರೇಮಿ
ಪ್ರೇಮಿಕ
ಪ್ರೇಮೋ-ನ್ಮಾದ
ಪ್ರೇಮೋ-ನ್ಮಾ-ದಕ್ಕೆ
ಪ್ರೇಮೋ-ನ್ಮಾ-ದ-ವಿದೆ
ಪ್ರೇಮೋ-ನ್ಮಾ-ದ-ವೊಂದೇ
ಪ್ರೇಯಸಾ
ಪ್ರೇರ-ಕ-ನಾ-ದ-ವ-ನಿಗೆ
ಪ್ರೇರ-ಕ-ವಾದ
ಪ್ರೇರ-ಣೆ-ಯಾ-ಗು-ವುದು
ಪ್ರೇರ-ಣೆ-ಯಿಂದ
ಪ್ರೇರ-ಣೆ-ಯಿ-ಲ್ಲದೆ
ಪ್ರೇರಿ-ತ-ನಾಗಿ
ಪ್ರೇಷಿತಃ
ಪ್ರೈಯ್ಯ-ವ್ರತೋ
ಪ್ರೋಕ್ಷಿ-ಸಿ-ಕೊಂ-ಡನು
ಪ್ರೋಕ್ಷಿ-ಸಿ-ಕೊಂಡು
ಪ್ರೋಕ್ಷಿ-ಸಿ-ದನು
ಪ್ರೋಕ್ಷಿ-ಸು-ತ್ತಲೇ
ಪ್ರೋತ್ಸಾ-ಹಈ
ಪ್ರೋತ್ಸಾ-ಹ-ವಿ-ತ್ತುದು
ಪ್ರೋತ್ಸಾ-ಹಿ-ಸಿದ
ಪ್ರೋತ್ಸಾ-ಹಿ-ಸು-ವನು
ಪ್ರೋತ್ಸಾ-ಹಿ-ಸು-ವು-ದ-ಕ್ಕಾ-ಗಿಯೇ
ಪ್ಲಕ್ಷ-ದ್ವೀ-ಪದ
ಪ್ಲಕ್ಷ-ದ್ವೀ-ಪ-ವಿದೆ
ಫಟ್
ಫಲ
ಫಲ-ಕ್ಕಲ್ಲ
ಫಲ-ಗಳನ್ನು
ಫಲ-ತಾಂ-ಬೂ-ಲ-ಗಳನ್ನು
ಫಲ-ದಿಂದ
ಫಲ-ನೀ-ಡು-ವ-ವನು
ಫಲ-ಪು-ಷ್ಪ-ಭ-ರಿ-ತ-ವಾದ
ಫಲ-ಪು-ಷ್ಪ-ಭ-ರಿ-ತ-ವಾ-ದವು
ಫಲ-ಪ್ರದ
ಫಲ-ಪ್ರಾ-ಪ್ತಿ-ಗಳಲ್ಲಿ
ಫಲ-ವ-ತ್ತಾದ
ಫಲ-ವನ್ನು
ಫಲ-ವನ್ನೂ
ಫಲ-ವಾಗಿ
ಫಲ-ವಿಲ್ಲ
ಫಲ-ವುಂಟು
ಫಲವೂ
ಫಲ-ವೃ-ಕ್ಷ-ಗಳು
ಫಲ-ವೆಂ-ತ-ಹು-ದೆಂ-ಬು-ದನ್ನು
ಫಲವೇ
ಫಲ-ವೇನು
ಫಲ-ಶ್ರು-ತಿ-ಯನ್ನೂ
ಫಲ-ಸ-ಮೃ-ದ್ಧ-ವಾದ
ಫಲಾ
ಫಲಾ-ಪೇಕ್ಷೆ
ಫಲಾ-ಪೇ-ಕ್ಷೆ-ಯಿಂದ
ಫಲಾ-ಪೇ-ಕ್ಷೆ-ಯಿ-ಡದೆ
ಫಲಾ-ಪೇ-ಕ್ಷೆ-ಯಿ-ಲ್ಲದ
ಫಲಾ-ಪೇ-ಕ್ಷೆ-ಯಿ-ಲ್ಲದೆ
ಫಲಾ-ಪೇ-ಕ್ಷೆಯೂ
ಫಲಿ-ಸಿತು
ಫಾಲ್ಗು-ಣ-ಮಾ-ಸದ
ಬಂಗಾರ
ಬಂಗಾ-ರದ
ಬಂಗಾ-ರ-ದವು
ಬಂಗಾ-ರ-ದಿಂದ
ಬಂಗಾ-ರ-ವಮ್ಮ
ಬಂಜೆ-ಯ-ರಾದ
ಬಂಡನ್ನು
ಬಂಡಿ
ಬಂಡಿ-ಗಳನ್ನು
ಬಂಡಿಯ
ಬಂಡಿ-ಯನ್ನು
ಬಂಡಿ-ಯೊಂದು
ಬಂಡೆಯ
ಬಂಡೆ-ಯನ್ನು
ಬಂತು
ಬಂದ
ಬಂದಂತಾ
ಬಂದಂ-ತಾ-ಯಿತು
ಬಂದಂ-ತಿ-ರುವ
ಬಂದಂತೆ
ಬಂದದು
ಬಂದ-ನಮ್ಮಾ
ಬಂದ-ನ-ಲ್ಲಪ್ಪ
ಬಂದನು
ಬಂದ-ನೆಂಬ
ಬಂದನೋ
ಬಂದ-ಮೇಲೆ
ಬಂದರು
ಬಂದರೂ
ಬಂದರೆ
ಬಂದ-ರೆಂದು
ಬಂದ-ರೆಂಬ
ಬಂದಳು
ಬಂದ-ವ-ನಿಗೆ
ಬಂದ-ವನು
ಬಂದ-ವನೆ
ಬಂದ-ವ-ರನ್ನು
ಬಂದ-ವ-ರ-ನ್ನೆಲ್ಲ
ಬಂದ-ವ-ರಿಗೂ
ಬಂದ-ವ-ರಿಗೆ
ಬಂದ-ವ-ರಿ-ಗೆಲ್ಲ
ಬಂದ-ವರು
ಬಂದ-ವರೆ
ಬಂದ-ವರೆಲ್ಲ
ಬಂದವು
ಬಂದಷ್ಟು
ಬಂದಾ
ಬಂದಾಗ
ಬಂದಾ-ಗಲೂ
ಬಂದಾ-ಗ-ಲೆಲ್ಲ
ಬಂದಾ-ನೆಂಬ
ಬಂದಿ
ಬಂದಿತು
ಬಂದಿತೊ
ಬಂದಿತ್ತು
ಬಂದಿದೆ
ಬಂದಿ-ದೆ-ಯಾ-ದರೂ
ಬಂದಿದ್ದ
ಬಂದಿ-ದ್ದರು
ಬಂದಿ-ದ್ದರೂ
ಬಂದಿ-ದ್ದರೆ
ಬಂದಿ-ದ್ದಳು
ಬಂದಿ-ದ್ದವ
ಬಂದಿ-ದ್ದ-ವ-ರಿ-ಗೆಲ್ಲ
ಬಂದಿ-ದ್ದ-ವರೆ
ಬಂದಿ-ದ್ದ-ವರೆಲ್ಲ
ಬಂದಿ-ದ್ದ-ವರೆ-ಲ್ಲರೂ
ಬಂದಿ-ದ್ದಾನೆ
ಬಂದಿ-ದ್ದಾಳೆ
ಬಂದಿ-ದ್ದೀಯೆ
ಬಂದಿ-ದ್ದೀರಿ
ಬಂದಿ-ದ್ದುದು
ಬಂದಿದ್ದೆ
ಬಂದಿ-ದ್ದೇನೆ
ಬಂದಿ-ದ್ದೇವೆ
ಬಂದಿರ
ಬಂದಿ-ರ-ಬೇ-ಕೆ-ನ್ನಿ-ಸಿತು
ಬಂದಿ-ರ-ಲಿಲ್ಲ
ಬಂದಿರಿ
ಬಂದಿರು
ಬಂದಿ-ರು-ತ್ತಾರೆ
ಬಂದಿ-ರುವ
ಬಂದಿ-ರು-ವ-ನೆಂದು
ಬಂದಿ-ರು-ವರು
ಬಂದಿ-ರು-ವರೋ
ಬಂದಿ-ರು-ವ-ವರು
ಬಂದಿ-ರು-ವಾಗ
ಬಂದಿ-ರುವಿ
ಬಂದಿ-ರು-ವಿರಿ
ಬಂದಿ-ರು-ವು-ದನ್ನೇ
ಬಂದಿ-ರು-ವು-ದಾ-ದರೂ
ಬಂದಿ-ರು-ವುದು
ಬಂದಿ-ರುವೆ
ಬಂದಿ-ರು-ವೆ-ಯಲ್ಲ
ಬಂದಿ-ರು-ವೆ-ಯಲ್ಲಾ
ಬಂದಿಲ್ಲ
ಬಂದಿ-ಳಿದ
ಬಂದಿ-ಳಿ-ದರು
ಬಂದಿವೆ
ಬಂದೀತು
ಬಂದು
ಬಂದು-ದನ್ನು
ಬಂದು-ದರ
ಬಂದು-ದ-ರಿಂದ
ಬಂದು-ದಲ್ಲ
ಬಂದುದು
ಬಂದುದೇ
ಬಂದು-ವ-ಲ್ಲವೆ
ಬಂದುವು
ಬಂದೆ
ಬಂದೆನು
ಬಂದೆ-ಯಲ್ಲಾ
ಬಂದೆ-ರ-ಗ-ಬಲ್ಲ
ಬಂದೊಡ
ಬಂದೊ-ಡ-ನೆಯೆ
ಬಂಧ
ಬಂಧನ
ಬಂಧ-ನಈ
ಬಂಧ-ನಕ್ಕೂ
ಬಂಧ-ನಕ್ಕೆ
ಬಂಧ-ನ-ದಲ್ಲಿ
ಬಂಧ-ನ-ದ-ಲ್ಲಿದ್ದ
ಬಂಧ-ನ-ದಿಂದ
ಬಂಧ-ನ-ವಿ-ಲ್ಲ-ದಂ-ತಾ-ಗಲಿ
ಬಂಧ-ಮು-ಕ್ತ-ರ-ನ್ನಾಗಿ
ಬಂಧ-ಮೋ-ಕ್ಷ-ಗ-ಳೆ-ರ-ಡಕ್ಕೂ
ಬಂಧಿ-ಗ-ಳಾ-ಗಿ-ದ್ದೆವು
ಬಂಧಿ-ಸ-ಲಾ-ಗಿತ್ತು
ಬಂಧಿಸಿ
ಬಂಧು
ಬಂಧುಃ
ಬಂಧು-ಗಳ
ಬಂಧು-ಗಳನ್ನು
ಬಂಧು-ಗಳಲ್ಲಿ
ಬಂಧು-ಗ-ಳಾದ
ಬಂಧು-ಗ-ಳಾ-ದ-ವ-ರಿಗೆ
ಬಂಧು-ಗ-ಳಿಗೆ
ಬಂಧು-ಗ-ಳಿ-ಗೆಲ್ಲ
ಬಂಧು-ಗಳು
ಬಂಧು-ಗಳೆ
ಬಂಧು-ಗ-ಳೆಂದು
ಬಂಧು-ಗ-ಳೆ-ಲ್ಲರೂ
ಬಂಧು-ಗಳೇ
ಬಂಧು-ಗ-ಳೊ-ಡನೆ
ಬಂಧು-ಗ-ಳೊ-ಡ-ನೆಯೂ
ಬಂಧು-ಜ-ನ-ಪ್ರಿ-ಯನು
ಬಂಧು-ಬ-ಳಗ
ಬಂಧು-ಬ-ಳ-ಗ-ದ-ವರೆಲ್ಲ
ಬಂಧು-ಬ-ಳ-ಗ-ವನ್ನು
ಬಂಧು-ಬಾಂ-ಧ-ವ-ರ-ನ್ನೆಲ್ಲ
ಬಂಧು-ಬಾಂ-ಧ-ವ-ರೆಂಬ
ಬಂಧು-ಬಾಂ-ಧ-ವರೆಲ್ಲ
ಬಂಧು-ಬಾಂ-ಧ-ವ-ರೊ-ಡನೆ
ಬಂಧು-ಬಾಂ-ಧ-ವ-ರೊ-ಡ-ನೆಯೂ
ಬಂಧು-ಮಿ-ತ್ರರು
ಬಂಧು-ವ-ರ್ಗ-ದಲ್ಲಿ
ಬಂಧು-ವ-ರ್ಗ-ದ-ವರ
ಬಂಧು-ವಾಗಿ
ಬಂಧು-ವಾದ
ಬಂಧು-ವಾ-ದು-ದ-ರಿಂದ
ಬಂಧು-ವಿ-ನಂ-ತಿದ್ದ
ಬಂಧು-ವಿ-ನಂತೆ
ಬಂಧು-ವೆಂದು
ಬಂಧು-ಹತ್ಯಾ
ಬಂಧೂಂಶ್ಚ
ಬಕ
ಬಕ-ನೆಂಬ
ಬಕ-ಪಕ್ಷಿ
ಬಕ-ಪ-ಕ್ಷಿ-ಯ-ಲ್ಲ-ವಪ್ಪ
ಬಕ-ಪ-ಕ್ಷಿ-ಯಾಗಿ
ಬಕಾ-ಸುರ
ಬಕಾ-ಸು-ರನ
ಬಗ-ಬ-ಗೆಯ
ಬಗೆ
ಬಗೆ-ಗ-ಣ್ಣಿ-ನಿಂದ
ಬಗೆ-ದರು
ಬಗೆ-ದರೆ
ಬಗೆ-ದಿ-ದ್ದಾನೆ
ಬಗೆದು
ಬಗೆ-ದು-ಕೊಂಡು
ಬಗೆ-ದು-ದ-ರಿಂದ
ಬಗೆ-ಬ-ಗೆಯ
ಬಗೆಯ
ಬಗೆ-ಯ-ಲಿಲ್ಲ
ಬಗೆ-ಯಾ-ಗಿದೆ
ಬಗೆ-ಯಾದ
ಬಗೆ-ಯು-ತ್ತಿ-ರವ
ಬಗೆ-ಯು-ತ್ತಿ-ರು-ವ-ನಲ್ಲಾ
ಬಗ್ಗಿ
ಬಗ್ಗಿದ
ಬಗ್ಗಿ-ಸಿದ
ಬಚ್ಚಿ-ಟ್ಟನು
ಬಚ್ಚಿ-ಟ್ಟಿರ
ಬಚ್ಚಿಟ್ಟು
ಬಚ್ಚಿ-ಟ್ಟು-ಕೊಂಡು
ಬಜಾರಿ
ಬಟ್ಟ-ಲನ್ನು
ಬಟ್ಟೆ
ಬಟ್ಟೆ-ಗಳನ್ನು
ಬಟ್ಟೆ-ಗ-ಳ-ನ್ನುಟ್ಟು
ಬಟ್ಟೆ-ಗಳನ್ನೆಲ್ಲ
ಬಟ್ಟೆ-ಗಳು
ಬಟ್ಟೆ-ಗ-ಳೆಲ್ಲ
ಬಟ್ಟೆಗೆ
ಬಟ್ಟೆ-ದೋ-ರಿಪ
ಬಟ್ಟೆ-ಬರೆ
ಬಟ್ಟೆ-ಬ-ರೆ-ಗಳನ್ನೂ
ಬಟ್ಟೆ-ಬ-ರೆ-ಗಳನ್ನೆಲ್ಲ
ಬಟ್ಟೆ-ಯನ್ನು
ಬಟ್ಟೆ-ಯನ್ನೆ
ಬಟ್ಟೆ-ಯ-ನ್ನೆ-ತ್ತಿ-ಕೊಂಡು
ಬಟ್ಟೆ-ಯ-ಲ್ಲಿನ
ಬಟ್ಟೆ-ಯಿಂ-ದಲೇ
ಬಟ್ಟೆ-ಯೊ-ಗೆಯು
ಬಡ
ಬಡ-ಗಲ
ಬಡ-ತ-ನ-ಎಂಬ
ಬಡ-ತ-ನಕ್ಕೆ
ಬಡ-ತ-ನದ
ಬಡ-ತ-ನ-ವಿಲ್ಲ
ಬಡ-ನಡು
ಬಡ-ನ-ಡು-ವನ್ನು
ಬಡ-ನ-ಡು-ವಿನ
ಬಡ-ನ-ಡು-ವೇನು
ಬಡ-ಪಾ-ಯಿ-ಗ-ಳ-ನ್ನೇಕೆ
ಬಡ-ಬ-ಗ್ಗ-ರಿಗೂ
ಬಡ-ಬ್ರಾ-ಹ್ಮಣ
ಬಡ-ಬ್ರಾ-ಹ್ಮ-ಣ-ನನ್ನು
ಬಡ-ಬ್ರಾ-ಹ್ಮ-ಣ-ನಿಗೆ
ಬಡ-ಬ್ರಾ-ಹ್ಮ-ಣ-ನಿದ್ದ
ಬಡ-ರ-ಕ್ಕ-ಸ-ರನ್ನು
ಬಡವ
ಬಡ-ವ-ನಾ-ದರೂ
ಬಡ-ವ-ರಾ-ಗಿದ್ದ
ಬಡ-ವ-ರಿಗೆ
ಬಡ-ವರೇ
ಬಡ-ವಾಗಿ
ಬಡ-ವಾಗು
ಬಡ-ವಾ-ಗು-ತ್ತಿ-ರು-ವಿ-ರೇನು
ಬಡ-ವಾ-ಗು-ತ್ತಿ-ರುವೆ
ಬಡ-ವಾ-ಗು-ತ್ತಿ-ರು-ವೆಯಾ
ಬಡಿ
ಬಡಿ-ದಂ-ತಹ
ಬಡಿ-ದಂ-ತಾ-ಯಿತು
ಬಡಿ-ದನು
ಬಡಿ-ದ-ಪ್ಪ-ಳಿ-ಸು-ವು-ದಕ್ಕೆ
ಬಡಿ-ದ-ವ-ನಂತೆ
ಬಡಿ-ದಾ-ಡು-ತ್ತಿ-ದ್ದೇವೆ
ಬಡಿದು
ಬಡಿ-ದು-ಕೊಂ-ಡಿದೆ
ಬಡಿ-ದು-ಕೊ-ಳ್ಳಲಿ
ಬಡಿ-ದು-ಕೊ-ಳ್ಳುತ್ತಾ
ಬಡಿ-ದು-ಹಾ-ಕಿ-ದನು
ಬಡಿ-ದು-ಹಾ-ಕಿ-ದು-ದನ್ನು
ಬಡಿ-ದೆ-ಬ್ಬಿ-ಸಿತು
ಬಡಿ-ದೆ-ಬ್ಬಿ-ಸಿದಂ
ಬಡಿ-ದೋ-ಡಿಸಿ
ಬಡಿ-ದೋ-ಡಿ-ಸು-ವಂತೆ
ಬಡಿ-ಯಿತು
ಬಡಿ-ಯಿರಿ
ಬಡಿ-ಯು-ತ್ತಲೆ
ಬಡಿ-ವಾರ
ಬಡಿ-ವಾ-ರ-ವನ್ನು
ಬಡಿ-ಸ-ಬೇ-ಕಾ-ಗಿತ್ತು
ಬಡಿ-ಸ-ಬೇಕು
ಬಡಿ-ಸಿದ
ಬಡಿ-ಸು-ತ್ತಿ-ದ್ದ-ವಳು
ಬಡಿ-ಸು-ತ್ತಿ-ರು-ವಾ-ಗಲೆ
ಬಣ-ಬೆ-ಯನ್ನು
ಬಣ್ಣ
ಬಣ್ಣ-ಗಳಿಂದ
ಬಣ್ಣ-ಗ-ಳಿಂ-ದಲೂ
ಬಣ್ಣದ
ಬಣ್ಣ-ದಲ್ಲಿ
ಬಣ್ಣ-ದಿಂದ
ಬಣ್ಣ-ನೆಗೆ
ಬಣ್ಣ-ಬ-ಣ್ಣದ
ಬಣ್ಣ-ವನ್ನು
ಬಣ್ಣ-ವಾಗಿ
ಬಣ್ಣಿ
ಬಣ್ಣಿ-ಸದೆ
ಬಣ್ಣಿ-ಸ-ಬ-ಲ್ಲರು
ಬಣ್ಣಿ-ಸ-ಲಾ-ರರು
ಬಣ್ಣಿ-ಸಲು
ಬಣ್ಣಿ-ಸ-ಹೊ-ರ-ಟನು
ಬಣ್ಣಿಸಿ
ಬಣ್ಣಿ-ಸುತ್ತಾ
ಬಣ್ಣಿ-ಸು-ವುದು
ಬತ
ಬತ್ತ
ಬತ್ತ-ಳಿಕೆ
ಬತ್ತ-ಳಿ-ಕೆ-ಗಳನ್ನು
ಬತ್ತ-ಳಿ-ಕೆ-ಗಳು
ಬತ್ತ-ವಿತ್ತು
ಬತ್ತಿ
ಬತ್ತಿ-ಗಳು
ಬತ್ತಿತು
ಬತ್ತಿ-ಹೋ-ಗಿ-ದ್ದರು
ಬತ್ತಿ-ಹೋ-ಗು-ತ್ತದೆ
ಬತ್ತಿ-ಹೋ-ಯಿತು
ಬತ್ತು
ಬತ್ತು-ವು-ದಿಲ್ಲ
ಬದ-ನೆಯ
ಬದ-ರಿ-ಕಾ-ಶ್ರ-ಮಕ್ಕೆ
ಬದ-ರಿ-ಕಾ-ಶ್ರ-ಮ-ದಲ್ಲಿ
ಬದ-ರಿ-ಕಾ-ಶ್ರ-ಮ-ವನ್ನು
ಬದಲಾ
ಬದ-ಲಾ-ಗ-ಬ-ಹುದು
ಬದ-ಲಾಗಿ
ಬದ-ಲಾಯಿ
ಬದ-ಲಾ-ಯಿ-ಸಿ-ಕೊಂಡು
ಬದ-ಲಾ-ಯಿ-ಸಿ-ದು-ದನ್ನು
ಬದ-ಲಾ-ಯಿ-ಸು-ತ್ತಿ-ರು-ತ್ತದೆ
ಬದ-ಲಾ-ಯಿ-ಸು-ತ್ತೇವೆ
ಬದ-ಲಾ-ಯಿ-ಸು-ವುದು
ಬದ-ಲಾ-ವ-ಣೆ-ಯಾ-ಗಿದೆ
ಬದಲು
ಬದಿ-ಗೊತ್ತಿ
ಬದು-ಕ-ಬ-ಯ-ಸು-ವೆಯಾ
ಬದು-ಕ-ಬಲ್ಲ
ಬದು-ಕ-ಬೇಕು
ಬದು-ಕಿದ
ಬದು-ಕಿ-ದ್ದರೂ
ಬದು-ಕಿ-ದ್ದರೆ
ಬದು-ಕಿ-ದ್ದಾಳೆ
ಬದು-ಕಿನ
ಬದು-ಕಿ-ಬಂದ
ಬದು-ಕಿ-ರ-ಬ-ಹುದೆ
ಬದು-ಕಿ-ರ-ಲಾರೆ
ಬದು-ಕಿ-ರು-ವಷ್ಟು
ಬದು-ಕಿ-ರು-ವಿರೊ
ಬದು-ಕಿ-ರುವೆ
ಬದು-ಕಿ-ಸ-ಬಲ್ಲ
ಬದು-ಕಿಸಿ
ಬದು-ಕಿ-ಸಿ-ಕೊಟ್ಟು
ಬದು-ಕಿ-ಸಿ-ಕೊಡಿ
ಬದು-ಕಿ-ಸಿದ
ಬದು-ಕಿ-ಸಿ-ದನು
ಬದು-ಕಿ-ಸಿ-ದರು
ಬದು-ಕಿ-ಸಿದೆ
ಬದು-ಕಿಸು
ಬದು-ಕಿ-ಸು-ವ-ವನೂ
ಬದು-ಕು-ತ್ತೇ-ನೆಯೆ
ಬದು-ಕು-ವುದು
ಬದು-ಕೋಣ
ಬದ್ಧ-ನಾ-ಗಿರು
ಬದ್ಧ-ನಾ-ಗಿ-ರುವು
ಬದ್ಧಶ್ಚ
ಬನ-ದೇ-ವಿಯ
ಬನ್ನಿ
ಬನ್ನಿಯ
ಬನ್ನಿ-ಯಲ್ಲಿ
ಬನ್ನಿರೆ
ಬನ್ನಿರೊ
ಬಯಕೆ
ಬಯ-ಕೆ-ಯನ್ನು
ಬಯ-ಕೆ-ಯಿಂದ
ಬಯ-ಕೆ-ಯು-ಳ್ಳ-ವ-ಳಾ-ಗಿ-ರು-ವೆನು
ಬಯ-ಕೆಯೆ
ಬಯ-ಕೆ-ಯೇ-ನೆಂ-ಬುದು
ಬಯ-ಲಾ-ಯಿತು
ಬಯ-ಸದೆ
ಬಯ-ಸ-ಬೇಡಿ
ಬಯಸಿ
ಬಯ-ಸಿದ
ಬಯ-ಸಿ-ದರು
ಬಯ-ಸಿ-ದರೂ
ಬಯ-ಸಿ-ದರೆ
ಬಯ-ಸಿ-ದಳು
ಬಯ-ಸಿ-ದ-ವರು
ಬಯ-ಸಿ-ದು-ದ-ನ್ನೆಲ್ಲ
ಬಯ-ಸಿ-ದುದು
ಬಯ-ಸಿ-ದು-ದೆಲ್ಲ
ಬಯ-ಸಿದೆ
ಬಯ-ಸಿದ್ದ
ಬಯ-ಸಿ-ದ್ದಾನೆ
ಬಯ-ಸಿಯೇ
ಬಯಸು
ಬಯ-ಸು-ತ್ತಾನೆ
ಬಯ-ಸು-ತ್ತಾರೆ
ಬಯ-ಸು-ತ್ತಿದ್ದ
ಬಯ-ಸು-ತ್ತಿ-ದ್ದವು
ಬಯ-ಸು-ತ್ತಿ-ದ್ದಾಳೆ
ಬಯ-ಸು-ತ್ತಿ-ದ್ದುದು
ಬಯ-ಸು-ತ್ತಿರ
ಬಯ-ಸು-ತ್ತಿ-ರ-ಲಿ-ಲ್ಲ-ವಾ-ದರೂ
ಬಯ-ಸು-ತ್ತಿ-ರುವ
ಬಯ-ಸು-ತ್ತಿ-ರು-ವುದು
ಬಯ-ಸು-ತ್ತಿ-ರುವೆ
ಬಯ-ಸು-ತ್ತಿಲ್ಲ
ಬಯ-ಸು-ತ್ತಿ-ಲ್ಲ-ವಾ-ದರೂ
ಬಯ-ಸುವ
ಬಯ-ಸು-ವ-ವನು
ಬಯ-ಸು-ವ-ವರು
ಬಯ-ಸು-ವು-ದಾ-ದರೆ
ಬಯ-ಸು-ವು-ದಿಲ್ಲ
ಬಯ-ಸು-ವು-ದಿ-ಲ್ಲ-ವೆಂದು
ಬಯ-ಸು-ವುದು
ಬಯ್ಗ-ಳಂತೆ
ಬಯ್ದು
ಬಯ್ಯುತ್ತಾ
ಬರ
ಬರ-ಕೂ-ಡದು
ಬರಕ್ಕೆ
ಬರ-ಗಾ-ಲ-ವಿಲ್ಲ
ಬರ-ಡಾ-ಗಿ-ರುವ
ಬರ-ದಂತಾ
ಬರ-ದಂತೆ
ಬರ-ದಿ-ರಲು
ಬರದು
ಬರದೆ
ಬರ-ಧಾ-ರಿ-ಯನ್ನು
ಬರ-ಬ-ರುತ್ತಾ
ಬರ-ಬ-ಹು-ದಾದ
ಬರ-ಬ-ಹುದು
ಬರ-ಬಾ-ರದು
ಬರ-ಬಾ-ರದೆ
ಬರ-ಬೇ-ಕಾ-ದರೆ
ಬರ-ಬೇ-ಕಾ-ಯಿತು
ಬರ-ಬೇಕು
ಬರ-ಬೇ-ಕು-ಎ-ನ್ನಿ-ಸಿತು
ಬರ-ಬೇ-ಕೆಂದು
ಬರ-ಮಾಡಿ
ಬರ-ಮಾ-ಡಿ-ಕೊಂ-ಡನು
ಬರ-ಮಾ-ಡಿ-ಕೊಂ-ಡರು
ಬರ-ಲಾ-ಗಲೇ
ಬರ-ಲಾ-ರೆವು
ಬರಲಿ
ಬರ-ಲಿಲ್ಲ
ಬರಲು
ಬರ-ಲೆಂದು
ಬರ-ವ-ಣಿಗೆ
ಬರ-ವನ್ನೆ
ಬರ-ಸಿ-ಡಿ-ಲಿ-ನಂ-ತಹ
ಬರ-ಸಿ-ಡಿ-ಲಿ-ನಂತೆ
ಬರ-ಸಿ-ಡಿಲು
ಬರ-ಸೆ-ಳೆದು
ಬರಹ
ಬರಿ
ಬರಿ-ಗೈಲಿ
ಬರಿ-ಗೈಲೆ
ಬರಿ-ದಾ-ಗಿತ್ತು
ಬರಿ-ದಾ-ಗಿದ್ದ
ಬರಿ-ದಾ-ಯಿತು
ಬರಿ-ದು-ಮಾ-ಡಿ-ಕೊಂ-ಡೆವು
ಬರಿ-ಮೈ-ಯಲ್ಲಿ
ಬರಿ-ಮೈ-ಯ-ಲ್ಲಿ-ರು-ವಾಗ
ಬರಿ-ಮೈ-ಯಿಂದ
ಬರಿ-ಮೈಲಿ
ಬರಿ-ಮೈ-ಲಿ-ರುವು
ಬರಿಯ
ಬರು
ಬರು-ತ್ತದೆ
ಬರು-ತ್ತ-ದೆಯೆ
ಬರು-ತ್ತಲೆ
ಬರು-ತ್ತಲೇ
ಬರು-ತ್ತ-ವಂತೆ
ಬರು-ತ್ತವೆ
ಬರುತ್ತಾ
ಬರು-ತ್ತಾ-ನಂತೋ
ಬರು-ತ್ತಾನೆ
ಬರು-ತ್ತಾರೆ
ಬರುತ್ತಿ
ಬರು-ತ್ತಿತ್ತು
ಬರು-ತ್ತಿದ್ದ
ಬರು-ತ್ತಿ-ದ್ದಂ-ತೆಯೆ
ಬರು-ತ್ತಿ-ದ್ದನು
ಬರು-ತ್ತಿ-ದ್ದ-ರಾ-ದರೂ
ಬರು-ತ್ತಿ-ದ್ದರು
ಬರು-ತ್ತಿ-ದ್ದರೆ
ಬರು-ತ್ತಿ-ದ್ದಳು
ಬರು-ತ್ತಿ-ದ್ದಾನೆ
ಬರು-ತ್ತಿ-ರಲು
ಬರು-ತ್ತಿ-ರುವ
ಬರು-ತ್ತಿ-ರು-ವಾಗ
ಬರು-ತ್ತಿ-ರುವಿ
ಬರು-ತ್ತಿ-ರುವು
ಬರು-ತ್ತಿ-ರು-ವುದನ್ನು
ಬರು-ತ್ತಿ-ರು-ವುದು
ಬರು-ತ್ತಿಲ್ಲ
ಬರು-ತ್ತಿವೆ
ಬರು-ತ್ತೇನೆ
ಬರು-ತ್ತೇ-ನೆಂದು
ಬರು-ತ್ತೇವೆ
ಬರು-ತ್ತೇವೋ
ಬರುವ
ಬರು-ವಂ-ತಿದೆ
ಬರು-ವಂತೆ
ಬರು-ವನು
ಬರು-ವ-ನೆಂದು
ಬರು-ವರೋ
ಬರು-ವ-ವ-ನಾ-ಗಿ-ದ್ದಾನೆ
ಬರು-ವ-ವನೇ
ಬರು-ವ-ವ-ರಾ-ಗಿ-ದ್ದಾರೆ
ಬರು-ವ-ವ-ರಿ-ಗೆಲ್ಲ
ಬರು-ವ-ವ-ರೆಗೂ
ಬರು-ವ-ವೇ-ಳೆಗೆ
ಬರು-ವ-ಷ್ಟ-ರಲ್ಲಿ
ಬರು-ವಷ್ಟು
ಬರು-ವಾಗ
ಬರು-ವಾ-ಗಲೆ
ಬರು-ವು-ದ-ಕ್ಕಾಗಿ
ಬರು-ವು-ದಕ್ಕೆ
ಬರು-ವುದನ್ನು
ಬರು-ವು-ದನ್ನೆ
ಬರು-ವು-ದಾಗಿ
ಬರು-ವು-ದಾ-ದರೂ
ಬರು-ವು-ದಿಲ್ಲ
ಬರು-ವು-ದಿ-ಲ್ಲವೆ
ಬರು-ವುದು
ಬರು-ವು-ದು-ಹೀಗೆ
ಬರು-ವು-ದೆಂ-ದಿ-ದ್ದರೆ
ಬರು-ವೆನು
ಬರು-ವೆ-ನೆಂದು
ಬರೆ-ಗಳನ್ನೆಲ್ಲ
ಬರೆದ
ಬರೆ-ದ-ನೆಂದು
ಬರೆ-ದರು
ಬರೆ-ದ-ರು-ಎಂಬ
ಬರೆ-ದರೂ
ಬರೆ-ದ-ವನು
ಬರೆ-ದ-ವರು
ಬರೆ-ದಷ್ಟು
ಬರೆ-ದಿ-ರು-ತ್ತಾರೆ
ಬರೆ-ದಿ-ರುವ
ಬರೆ-ದಿ-ರು-ವ-ರಾ-ದರೂ
ಬರೆ-ದಿ-ರು-ವುದು
ಬರೆ-ದಿ-ಲ್ಲ-ವ-ಲ್ಲವೆ
ಬರೆದು
ಬರೆ-ದು-ಕೊಂ-ಡಿ-ದ್ದಾನೆ
ಬರೆ-ಬ-ರೆದು
ಬರೆ-ಯಲು
ಬರೆ-ಯ-ಹೊ-ರ-ಟುದು
ಬರೆ-ಯಿರಿ
ಬರೆಯು
ಬರೆ-ಯುತ್ತಾ
ಬರೆ-ಯು-ವುದ
ಬರೆ-ಯು-ವುದು
ಬರೋಣ
ಬರ್ಬರ
ಬರ್ಹಿಯ
ಬರ್ಹಿ-ಷ-ದನು
ಬರ್ಹಿ-ಷ್ಮ-ತಿ-ಯನ್ನು
ಬಲ
ಬಲಕ್ಕೆ
ಬಲ-ಗಾ-ಲನ್ನು
ಬಲ-ಗೈ-ಯಲ್ಲಿ
ಬಲ-ಗೈ-ಯಿಂದ
ಬಲ-ಗೈ-ಲಿದ್ದ
ಬಲ-ತಾಯಿ
ಬಲ-ತಾ-ಯಿಯ
ಬಲ-ದಿಂದ
ಬಲ-ದೇವ
ಬಲ-ದೇ-ವರ
ಬಲ-ದೋ-ಳನ್ನೂ
ಬಲ-ನೆಂ-ದರೆ
ಬಲ-ಪಾದ
ಬಲ-ಪಾ-ದ-ವನ್ನು
ಬಲ-ಭ-ದ್ರ-ನೆಂದೂ
ಬಲ-ಭಾ-ಗ-ದಿಂದ
ಬಲ-ರಾಂ
ಬಲ-ರಾಮ
ಬಲ-ರಾ-ಮ-ಮು-ಷ್ಠಿಕ
ಬಲ-ರಾ-ಮ-ವಾ-ಸು-ದೇ-ವ-ರನ್ನು
ಬಲ-ರಾ-ಮ-ಕೃಷ್ಣ
ಬಲ-ರಾ-ಮ-ಕೃ-ಷ್ಣರ
ಬಲ-ರಾ-ಮ-ಕೃ-ಷ್ಣ-ರನ್ನು
ಬಲ-ರಾ-ಮ-ಕೃ-ಷ್ಣ-ರಿಂದ
ಬಲ-ರಾ-ಮ-ಕೃ-ಷ್ಣ-ರಿಗೂ
ಬಲ-ರಾ-ಮ-ಕೃ-ಷ್ಣ-ರಿಗೆ
ಬಲ-ರಾ-ಮ-ಕೃ-ಷ್ಣ-ರಿ-ಬ್ಬರೂ
ಬಲ-ರಾ-ಮ-ಕೃ-ಷ್ಣರು
ಬಲ-ರಾ-ಮ-ಕೃ-ಷ್ಣರೂ
ಬಲ-ರಾ-ಮ-ಕೃ-ಷ್ಣರೆ
ಬಲ-ರಾ-ಮ-ಕೃ-ಷ್ಣರೇ
ಬಲ-ರಾ-ಮನ
ಬಲ-ರಾ-ಮ-ನನ್ನು
ಬಲ-ರಾ-ಮ-ನನ್ನೂ
ಬಲ-ರಾ-ಮ-ನಲ್ಲಿ
ಬಲ-ರಾ-ಮ-ನಾಗಿ
ಬಲ-ರಾ-ಮ-ನಾ-ಗಿಯೂ
ಬಲ-ರಾ-ಮ-ನಿ-ಗಾಗಿ
ಬಲ-ರಾ-ಮ-ನಿ-ಗಿಂತ
ಬಲ-ರಾ-ಮ-ನಿ-ಗಿಂ-ತಲೂ
ಬಲ-ರಾ-ಮ-ನಿಗೂ
ಬಲ-ರಾ-ಮ-ನಿಗೆ
ಬಲ-ರಾ-ಮ-ನಿ-ರ್ಯಾ-ಣ-ವನ್ನೂ
ಬಲ-ರಾ-ಮನು
ಬಲ-ರಾ-ಮನೂ
ಬಲ-ರಾ-ಮ-ನೆಂ-ದರೆ
ಬಲ-ರಾ-ಮ-ನೆಂದು
ಬಲ-ರಾ-ಮ-ನೆಂದೂ
ಬಲ-ರಾ-ಮ-ನೆಂ-ಬು-ವನು
ಬಲ-ರಾ-ಮ-ನೊ-ಡನೆ
ಬಲ-ರಾ-ಮರ
ಬಲ-ರಾ-ಮರು
ಬಲ-ರಾ-ಮರೂ
ಬಲ-ವಂತ
ಬಲ-ವಂ-ತ-ದಿಂದ
ಬಲ-ವಂ-ತ-ನಾ-ದ-ವನು
ಬಲ-ವಂ-ತ-ಮಾ-ಡಿ-ದಳು
ಬಲ-ವಂ-ತ-ವಾಗಿ
ಬಲ-ವ-ತಾಸಿ
ಬಲ-ವನ್ನು
ಬಲ-ವಾಗಿ
ಬಲ-ವಾ-ಗಿ-ತ್ತೆಂ-ದರೆ
ಬಲ-ವಾದ
ಬಲ-ವಾ-ದ-ವನು
ಬಲ-ವಾ-ದುದು
ಬಲ-ವುಳ್ಳ
ಬಲ-ಶಾಲಿ
ಬಲ-ಶಾ-ಲಿ-ಗಳು
ಬಲ-ಶಾ-ಲಿ-ಗ-ಳೆ-ಲ್ಲ-ರಿ-ಗಿಂ-ತಲೂ
ಬಲ-ಶಾ-ಲಿ-ಗಳೇ
ಬಲ-ಶಾ-ಲಿ-ಯಾಗಿ
ಬಲ-ಶಾ-ಲಿ-ಯಾದ
ಬಲ-ಶಾ-ಲಿಯು
ಬಲಾ
ಬಲಾ-ಢ್ಯರು
ಬಲಾ-ಢ್ಯ-ವಾ-ಗು-ವುದು
ಬಲಾ-ತ್ಕಾ-ರ-ದಿಂದ
ಬಲಾಯ
ಬಲಿ
ಬಲಿ-ಕೊ-ಡ-ಬಾ-ರದು
ಬಲಿ-ಕೊ-ಡ-ಬೇಕು
ಬಲಿ-ಕೊಡು
ಬಲಿ-ಕೊ-ಡು-ವಂತೆ
ಬಲಿಗೆ
ಬಲಿ-ಚಕ್ರ
ಬಲಿ-ಚ-ಕ್ರ-ವರ್ತಿ
ಬಲಿ-ಚ-ಕ್ರ-ವ-ರ್ತಿಗೆ
ಬಲಿ-ಚ-ಕ್ರ-ವ-ರ್ತಿಯ
ಬಲಿ-ಚ-ಕ್ರ-ವ-ರ್ತಿಯು
ಬಲಿ-ದಾನ
ಬಲಿ-ದಾ-ನಕ್ಕೆ
ಬಲಿ-ಮ-ತ್ವಾ-ವೇ-ಷ್ಟ-ಯ-ಧ್ವಾಂ-ಕ್ಷ-ವದ್ಯಃ
ಬಲಿ-ಮಪಿ
ಬಲಿಯ
ಬಲಿ-ಯ-ಕಡೆ
ಬಲಿ-ಯನ್ನು
ಬಲಿ-ಯಾ-ಗ-ಬೇಕು
ಬಲಿ-ಯಾದ
ಬಲಿ-ಯಾದೆ
ಬಲಿ-ಯಿಂದ
ಬಲಿಯು
ಬಲಿ-ರಾ-ಜ-ನಿಗೆ
ಬಲಿ-ಹಾ-ಕ-ಲೇ-ಬೇಕು
ಬಲಿ-ಹಾ-ಕಿ-ದ-ನಂತೆ
ಬಲಿ-ಹಾ-ಕು-ತ್ತೇನೆ
ಬಲಿ-ಹಾ-ಕು-ವ-ನೆಂದು
ಬಲೀಂದ್ರ
ಬಲೀಂ-ದ್ರನ
ಬಲೀಂ-ದ್ರ-ನನ್ನು
ಬಲೀಂ-ದ್ರ-ನಿಂದ
ಬಲೀಂ-ದ್ರ-ನಿಗೆ
ಬಲೀಂ-ದ್ರನು
ಬಲೀ-ಕ-ನೆಂಬ
ಬಲು
ಬಲೆ
ಬಲೆಗೆ
ಬಲೆ-ಯನ್ನು
ಬಲೆ-ಯಲ್ಲಿ
ಬಲೋ
ಬಲ್ಲ
ಬಲ್ಲರು
ಬಲ್ಲಳು
ಬಲ್ಲ-ವ-ನಾ-ಗಿ-ದ್ದನು
ಬಲ್ಲ-ವನು
ಬಲ್ಲವು
ಬಲ್ಲೆ
ಬಲ್ಲೆವು
ಬಳ-ಕು-ತ್ತಿತ್ತು
ಬಳ-ಕುವ
ಬಳ-ಕೆಗೆ
ಬಳ-ಕೆ-ಯ-ಲ್ಲಿ-ದ್ದಂತೆ
ಬಳಗ
ಬಳ-ಗ-ಈ-ವ-ರೆಗೆ
ಬಳ-ಗ-ದ-ವ-ರ-ನ್ನೆಲ್ಲ
ಬಳ-ಗ-ವನ್ನೂ
ಬಳ-ಬಳ
ಬಳ-ಬ-ಳನೆ
ಬಳಲಿ
ಬಳ-ಲಿ-ಕೆ-ಯನ್ನು
ಬಳ-ಲಿದ
ಬಳ-ಲಿ-ದ-ವನೂ
ಬಳ-ಲಿ-ಹೋದ
ಬಳ-ಸದೆ
ಬಳ-ಸ-ಬೇ-ಕೆಂಬ
ಬಳಸಿ
ಬಳ-ಸಿ-ಕೊ-ಳ್ಳ-ದಿ-ರು-ವುದು
ಬಳ-ಸಿ-ಕೊ-ಳ್ಳ-ಬೇಕು
ಬಳ-ಸಿ-ಕೊ-ಳ್ಳುವ
ಬಳ-ಸು-ತ್ತಾನೆ
ಬಳ-ಸು-ತ್ತಿದ್ದ
ಬಳ-ಸುವ
ಬಳಿ
ಬಳಿಕ
ಬಳಿಗೆ
ಬಳಿಗೇ
ಬಳಿ-ದಂ-ತಾ-ಗು-ತ್ತದೆ
ಬಳಿ-ದಂ-ತಾ-ಯಿತು
ಬಳಿ-ದರು
ಬಳಿ-ದು-ಕೊಂಡು
ಬಳಿಯ
ಬಳಿ-ಯಲ್ಲಿ
ಬಳಿ-ಯ-ಲ್ಲಿಯೆ
ಬಳಿ-ಯ-ಲ್ಲಿಯೇ
ಬಳಿ-ಯ-ಲ್ಲಿಯೋ
ಬಳಿ-ಯಲ್ಲೇ
ಬಳಿ-ಯಿದ್ದ
ಬಳಿ-ಯು-ವುದೆ
ಬಳು-ಕಿ-ಸುತ್ತಾ
ಬಳು-ಕು-ತ್ತದೆ
ಬಳು-ಕುತ್ತಾ
ಬಳು-ಕು-ತ್ತಿ-ರುವ
ಬಳು-ವ-ಳಿ-ಗ-ಳ-ನ್ನಿತ್ತು
ಬಳು-ವ-ಳಿ-ಗಳನ್ನು
ಬಳು-ವ-ಳಿ-ಗ-ಳೊ-ಡನೆ
ಬಳು-ವ-ಳಿ-ಯಾಗಿ
ಬಳು-ವ-ಳಿಯೂ
ಬಳೆ-ಗಳನ್ನು
ಬಳೆ-ಗ-ಳಿ-ದ್ದು-ದ-ರಿಂದ
ಬಳೆ-ಗಳೆ
ಬಳ್ಳಿ
ಬಳ್ಳಿ-ಗಳ
ಬಳ್ಳಿ-ಗಳಿಂದ
ಬಳ್ಳಿ-ಗಳು
ಬಳ್ಳಿ-ಗ-ಳೆಲ್ಲ
ಬಳ್ಳಿಗೆ
ಬಳ್ಳಿ-ಯಂ-ತಿದ್ದ
ಬಳ್ಳಿ-ಯಂ-ತಿ-ರುವ
ಬಳ್ಳಿ-ಯಂತೆ
ಬಳ್ಳಿ-ಯಿಂದ
ಬವ-ಣೆ-ಪ-ಡುವು
ಬವ-ಣೆ-ಯನ್ನು
ಬಸ್
ಬಹಳ
ಬಹ-ಳ-ವಾಗಿ
ಬಹವ
ಬಹಿ-ರಂಗ
ಬಹಿ-ರಂ-ಗ-ದಲ್ಲಿ
ಬಹಿ-ರಂ-ಗ-ವಾಗಿ
ಬಹಿ-ರ್ಮು-ಖ-ನಾದ
ಬಹಿ-ರ್ಮು-ಖ-ನಾ-ದ-ವನು
ಬಹಿ-ರ್ಮು-ಖ-ವಾ-ಗಿ-ರುವ
ಬಹು
ಬಹು-ಕಷ್ಟ
ಬಹು-ಕಾಲ
ಬಹು-ಕಾ-ಲದ
ಬಹು-ಕಾ-ಲ-ದ-ಮೇಲೆ
ಬಹು-ಕಾ-ಲ-ದ-ವ-ರೆಗೆ
ಬಹು-ಕಾ-ಲ-ದಿಂದ
ಬಹು-ಕಾ-ಲ-ದಿಂ-ದಲೂ
ಬಹು-ಕಾ-ಲ-ವಾ-ದರೂ
ಬಹು-ದಿನ
ಬಹು-ದಿ-ನ-ಗ-ಳಾ-ಯಿತು
ಬಹು-ದಿ-ನ-ಗಳಿಂದ
ಬಹುದು
ಬಹು-ದು-ಗೋ-ಪ-ಗೋ-ಪಿ-ಯರ
ಬಹು-ದೂರ
ಬಹು-ದೂ-ರ-ದ-ವ-ರೆಗೆ
ಬಹು-ದೂ-ರ-ಹೋದ
ಬಹು-ದೆಂದು
ಬಹು-ದೆ-ಎಂ-ದು-ಕೊಂ-ಡನು
ಬಹು-ದೊಡ್ಡ
ಬಹು-ಪ-ತ್ನೀತ್ವ
ಬಹು-ಪ-ತ್ನೀ-ತ್ವವೂ
ಬಹು-ಬು-ದ್ಧಿ-ಶಾಲಿ
ಬಹು-ಬೇಗ
ಬಹು-ಭಕ್ತಿ
ಬಹು-ಮ-ಟ್ಟಿಗೆ
ಬಹು-ಮಾನ
ಬಹು-ಮಾ-ನ-ಗ-ಳ-ನ್ನಿತ್ತು
ಬಹು-ಮಾ-ನ-ಗಳಿಂದ
ಬಹು-ಮಾ-ನ-ವನ್ನು
ಬಹು-ಮಾ-ನ-ವಾಗಿ
ಬಹು-ಮಾ-ನಿ-ಸ-ಬೇ-ಕೆಂ-ಬುದೇ
ಬಹು-ಮಾ-ನಿ-ಸುವೆ
ಬಹು-ಮು-ಖ-ವಾ-ದುದು
ಬಹು-ಳಾಶ್ವ
ಬಹು-ಳಾ-ಶ್ವ-ನಿಗೆ
ಬಹು-ಳಾ-ಶ್ವನು
ಬಹು-ವಾಗಿ
ಬಹು-ವಿ-ಧ-ವಾಗಿ
ಬಹು-ವಿ-ಧ-ವಾದ
ಬಹುಶಃ
ಬಹು-ಸ್ವಲ್ಪ
ಬಹು-ಹಿಂದೆ
ಬಹ್ವನ್ನ
ಬಾ
ಬಾಂಧ-ವ-ರನ್ನೂ
ಬಾಂಧ-ವ-ರೊ-ಡನೆ
ಬಾಗದ
ಬಾಗಿ
ಬಾಗಿಲ
ಬಾಗಿ-ಲನ್ನು
ಬಾಗಿ-ಲಲ್ಲಿ
ಬಾಗಿ-ಲ-ಲ್ಲಿಯೂ
ಬಾಗಿ-ಲಿಗೆ
ಬಾಗಿ-ಲಿನ
ಬಾಗಿ-ಲಿ-ನ-ಲ್ಲಿಯೇ
ಬಾಗಿಲು
ಬಾಗಿ-ಲು-ಗಳನ್ನು
ಬಾಗಿ-ಲು-ಗಳು
ಬಾಗಿಸಿ
ಬಾಗಿ-ಸಿ-ಕೊಂಡು
ಬಾಚಿ
ಬಾಡ-ದಿ-ರುವ
ಬಾಡ-ಲಿಲ್ಲ
ಬಾಡಿದ
ಬಾಡಿ-ಹೋ-ಗಿದೆ
ಬಾಡುವ
ಬಾಣ
ಬಾಣಕ್ಕೆ
ಬಾಣ-ಗಳ
ಬಾಣ-ಗಳನ್ನು
ಬಾಣ-ಗಳನ್ನೂ
ಬಾಣ-ಗಳನ್ನೆಲ್ಲ
ಬಾಣ-ಗಳಿಂದ
ಬಾಣ-ಗಳು
ಬಾಣ-ಗ-ಳೊ-ಡನೆ
ಬಾಣದ
ಬಾಣ-ದಂ-ತಿದೆ
ಬಾಣ-ದಂತೆ
ಬಾಣ-ದಿಂದ
ಬಾಣನ
ಬಾಣ-ನಿಗೆ
ಬಾಣನು
ಬಾಣ-ವನ್ನು
ಬಾಣ-ವೇ-ಗಕ್ಕೆ
ಬಾಣ-ವೇ-ಗ-ದಷ್ಟು
ಬಾಣ-ವೊಂದು
ಬಾಣಾ
ಬಾಣಾ-ಸು-ರನ
ಬಾಣಾ-ಸು-ರನು
ಬಾಧ-ಕವೂ
ಬಾಧಿ-ಸು-ತ್ತಿತ್ತು
ಬಾಧಿ-ಸು-ತ್ತಿವೆ
ಬಾಧೆ
ಬಾಧೆ-ಗೊ-ಳ-ಗಾ-ಗಿದ್ದ
ಬಾಧೆ-ಯಾ-ಗ-ದಂತೆ
ಬಾಧೆ-ಯಿಂದ
ಬಾಧೆ-ಯಿಲ್ಲ
ಬಾಧೆ-ಯೇನೂ
ಬಾನ-ವ-ರನ್ನೂ
ಬಾನ-ವ-ರ-ಲ್ಲಿಯೂ
ಬಾನ-ವರೂ
ಬಾನಿ-ನಿಂದ
ಬಾನು-ಗ-ಳೆ-ರ-ಡ-ರ-ಲ್ಲಿಯೂ
ಬಾಪ್ಪ
ಬಾಮ್ಮ
ಬಾಯನ್ನು
ಬಾಯಲ್ಲಿ
ಬಾಯ-ಲ್ಲಿಟ್ಟ
ಬಾಯ-ಲ್ಲಿ-ಟ್ಟಳು
ಬಾಯ-ಲ್ಲಿಯೂ
ಬಾಯ-ಲ್ಲಿ-ರುವ
ಬಾಯಾರಿ
ಬಾಯಾ-ರಿಕೆ
ಬಾಯಾ-ರಿ-ಕೆ-ಗಳನ್ನು
ಬಾಯಾ-ರಿ-ಕೆ-ಗ-ಳಿಲ್ಲ
ಬಾಯಾ-ರಿ-ಕೆ-ಗಳು
ಬಾಯಾ-ರಿ-ಕೆ-ಯನ್ನು
ಬಾಯಾ-ರಿ-ಕೆ-ಯಾ-ಗಲು
ಬಾಯಾ-ರಿ-ಕೆ-ಯಾ-ದಾಗ
ಬಾಯಾ-ರಿ-ಕೆ-ಯಾ-ದು-ದ-ರಿಂದ
ಬಾಯಾ-ರಿ-ಕೆ-ಯಾ-ಯಿತು
ಬಾಯಾ-ರಿ-ಕೆ-ಯಿಂದ
ಬಾಯಾ-ರಿದ
ಬಾಯಿ
ಬಾಯಿಂದ
ಬಾಯಿಂ-ದಲೇ
ಬಾಯಿ-ಗಳನ್ನು
ಬಾಯಿ-ಗಳನ್ನೂ
ಬಾಯಿ-ಗಳಿಂದ
ಬಾಯಿ-ಗ-ಳಿಂ-ದಲೂ
ಬಾಯಿಗೆ
ಬಾಯಿ-ಜ-ಗಳ
ಬಾಯಿ-ದ್ದರೂ
ಬಾಯಿ-ಬ-ಡುಕ
ಬಾಯಿ-ಬಿಟ್ಟು
ಬಾಯಿ-ಬಿ-ಡು-ವ-ವನು
ಬಾಯಿ-ಮಾ-ತಿ-ನಿಂದ
ಬಾಯಿ-ಮು-ಚ್ಚಿ-ಕೊಂಡು
ಬಾಯಿ-ಯನ್ನು
ಬಾಯಿ-ಲ್ಲದೆ
ಬಾಯಿ-ಹಾ-ಕಿತು
ಬಾಯೇ
ಬಾಯೊ
ಬಾಯೊಡ್ಡಿ
ಬಾಯೊ-ಳಗೆ
ಬಾಯ್ತುಂಬ
ಬಾಯ್ತೆ-ರೆ-ದಾಗ
ಬಾಯ್ದಂ-ಬು-ಲ-ವನ್ನು
ಬಾಯ್ದೆ-ರೆ-ದು-ಕೊಂಡು
ಬಾಯ್ಬಿಟ್ಟು
ಬಾಯ್ಬಿ-ಡು-ತ್ತಿ-ದ್ದಾರೆ
ಬಾಯ್ಬಿ-ಡು-ವಂ-ತಹ
ಬಾರಣ್ಣ
ಬಾರದ
ಬಾರ-ದ-ವ-ನಾ-ಗು-ತ್ತಾನೆ
ಬಾರ-ದಿ-ರ-ಲೆಂದು
ಬಾರದು
ಬಾರ-ದು-ದನ್ನು
ಬಾರದೆ
ಬಾರಿ
ಬಾರಿ-ಬಂದು
ಬಾರಿ-ಬಾ-ರಿಗೂ
ಬಾರಿಸಿ
ಬಾರಿ-ಸಿ-ದ-ನೆಂ-ದರೆ
ಬಾರಿ-ಸಿ-ದರು
ಬಾರಿ-ಸುತ್ತಾ
ಬಾರಿ-ಸು-ತ್ತಾನೆ
ಬಾರಿ-ಸು-ವು-ದ-ರಲ್ಲಿ
ಬಾರೋ
ಬಾಲಕ
ಬಾಲ-ಕನ
ಬಾಲ-ಕ-ನನ್ನು
ಬಾಲ-ಕ-ನಾ-ಗಿದ್ದ
ಬಾಲ-ಕ-ನಾ-ಗಿ-ರುವ
ಬಾಲ-ಕ-ನಾ-ಗಿ-ರು-ವಾ-ಗಲೆ
ಬಾಲ-ಕ-ನಾ-ಗಿ-ರು-ವಾ-ಗಲೇ
ಬಾಲ-ಕ-ನಾದ
ಬಾಲ-ಕ-ನಾ-ದನು
ಬಾಲ-ಕ-ನಾ-ದರೂ
ಬಾಲ-ಕ-ನಿಗೆ
ಬಾಲ-ಕನು
ಬಾಲ-ಕ-ನೊಬ್ಬ
ಬಾಲ-ಕ-ನೊ-ಬ್ಬ-ನನ್ನು
ಬಾಲ-ಕರ
ಬಾಲ-ಕ-ರನ್ನು
ಬಾಲ-ಕ-ರಾಗಿ
ಬಾಲ-ಕ-ರಾದ
ಬಾಲ-ಕ-ರಿಂದ
ಬಾಲ-ಕರು
ಬಾಲ-ಕ-ರೆಲ್ಲ
ಬಾಲ-ಕ-ವೃಂ-ದಕ್ಕೆ
ಬಾಲದ
ಬಾಲ-ದಿಂದ
ಬಾಲ-ಭಾ-ಷಿತ
ಬಾಲ-ರೂ-ಪಿ-ಯಾದ
ಬಾಲ-ಲೀ-ಲೆ-ಗಳನ್ನು
ಬಾಲ-ಲೀ-ಲೆ-ಯಿಂದ
ಬಾಲ-ವನ್ನು
ಬಾಲ-ವ-ನ್ನೆ-ತ್ತಿ-ಕೊಂಡು
ಬಾಲ-ವ-ಲ್ಲಾ-ಡಿ-ಸುತ್ತಾ
ಬಾಲ-ಸೂ-ರ್ಯ-ನಂತೆ
ಬಾಲೆ
ಬಾಲ್ಯ
ಬಾಲ್ಯ-ಜೀ-ವ-ನದ
ಬಾಲ್ಯದ
ಬಾಲ್ಯ-ದ-ಲ್ಲಿನ
ಬಾಲ್ಯ-ದ-ಲ್ಲಿಯೇ
ಬಾಲ್ಯ-ದಿಂದ
ಬಾಲ್ಯ-ವನ್ನು
ಬಾಲ್ಯ-ಸೂ-ರ್ಯ-ನಂತೆ
ಬಾಳ-ಕಥೆ
ಬಾಳ-ಕ-ಥೆ-ಯನ್ನು
ಬಾಳನ್ನು
ಬಾಳನ್ನೆ
ಬಾಳಲಿ
ಬಾಳಿ
ಬಾಳು
ಬಾಳು-ತ್ತಿ-ದ್ದರೆ
ಬಾಳು-ತ್ತಿದ್ದು
ಬಾಳು-ವಂತೆ
ಬಾಳುವು
ಬಾಳೆ
ಬಾಳೆಯ
ಬಾಳೆ-ಯಂತೆ
ಬಾಳೆ-ಯ-ಕಂ-ಬ-ಗ-ಳಿಂ
ಬಾಳೆ-ಲ್ಲ-ವನ್ನೂ
ಬಾಳೇ
ಬಾಳೋಣ
ಬಾವ-ಲಿ-ಗಳು
ಬಾವಿ
ಬಾವಿ-ಗಳ
ಬಾವಿಗೆ
ಬಾವಿಯ
ಬಾವಿ-ಯಂ-ತಿ-ದ್ದವು
ಬಾವಿ-ಯಂ-ತಿ-ರುವ
ಬಾವಿ-ಯಲ್ಲಿ
ಬಾವಿ-ಯಿಂದ
ಬಾವುಟ
ಬಾವು-ಟ-ಗಳನ್ನು
ಬಾವು-ಟ-ಗಳನ್ನೂ
ಬಾವು-ಟ-ಗ-ಳಿಂ-ದಲೂ
ಬಾವು-ಟ-ವಾ-ಸು-ದೇ-ವನ
ಬಾವು-ಟ-ವಿತ್ತು
ಬಾಷ್ಪ-ಗಳನ್ನು
ಬಾಷ್ಪ-ಧಾ-ರಾಃ
ಬಾಹುಕ
ಬಾಹು-ಕನ
ಬಾಹು-ಕ-ನೆಂಬು
ಬಾಹು-ಗಳು
ಬಾಹ್ಯ
ಬಾಹ್ಯ-ಪ್ರ-ಜ್ಞೆ-ಯಿ-ಲ್ಲದ
ಬಾಹ್ಯ-ಪ್ರ-ಪಂ-ಚದ
ಬಾಹ್ಯೇಂ
ಬಾಹ್ಲಿಕ
ಬಾಹ್ಲಿ-ಕ-ರಲ್ಲಿ
ಬಾಹ್ಲೀಕ
ಬಿಂಕ-ಗಾತಿ
ಬಿಂಕ-ಗಾರ್ತಿ
ಬಿಂದು
ಬಿಂದು-ಗ-ಳಂತೆ
ಬಿಂದು-ಮ-ತಿ-ಯೆಂಬ
ಬಿಂದು-ಸ-ರೋ-ವರ
ಬಿಂಬ
ಬಿಗಿ
ಬಿಗಿದ
ಬಿಗಿ-ದನು
ಬಿಗಿದು
ಬಿಗಿ-ದು-ಕೊಂಡು
ಬಿಗಿ-ಯಪ್ಪಿ
ಬಿಗಿ-ಯ-ಬೇ-ಕೆಂದು
ಬಿಗಿ-ಯಾಗಿ
ಬಿಗಿಸಿ
ಬಿಗಿ-ಹಿ-ಡಿ-ಯ-ಲಾ-ರದೆ
ಬಿಚ್ಚ-ಬೇಕು
ಬಿಚ್ಚಿ
ಬಿಚ್ಚಿತು
ಬಿಚ್ಚಿದ
ಬಿಚ್ಚಿ-ದರು
ಬಿಚ್ಚಿ-ಹಾ-ಕಿ-ದನು
ಬಿಚ್ಚಿ-ಹೋ-ಗಿ-ದ್ದರೂ
ಬಿಚ್ಚಿ-ಹೋ-ಗು-ತ್ತದೆ
ಬಿಚ್ಚಿ-ಹೋ-ಗು-ವು-ದಂತೆ
ಬಿಚ್ಚಿ-ಹೋ-ಯಿತು
ಬಿಚ್ಚು-ನು-ಡಿ-ಗಳನ್ನು
ಬಿಚ್ಚು-ವ-ತ-ನಕ
ಬಿಜ-ಯ-ಮಾಡಿ
ಬಿಟ್ಟ
ಬಿಟ್ಟನು
ಬಿಟ್ಟರು
ಬಿಟ್ಟರೆ
ಬಿಟ್ಟಳು
ಬಿಟ್ಟಾಗ
ಬಿಟ್ಟಾ-ರೆಯೆ
ಬಿಟ್ಟಿ
ಬಿಟ್ಟಿತು
ಬಿಟ್ಟಿ-ದ್ದೇನೆ
ಬಿಟ್ಟಿಯ
ಬಿಟ್ಟಿ-ರ-ಲಾ-ರ-ಎಂಬ
ಬಿಟ್ಟಿ-ರುವ
ಬಿಟ್ಟಿಲ್ಲ
ಬಿಟ್ಟು
ಬಿಟ್ಟು-ಕೊಂಡು
ಬಿಟ್ಟು-ಕೊ-ಟ್ಟನೊ
ಬಿಟ್ಟು-ಕೊ-ಡ-ಬೇಕೆ
ಬಿಟ್ಟು-ದನ್ನು
ಬಿಟ್ಟುದೇ
ಬಿಟ್ಟು-ನೋ-ಡು-ತ್ತಾ-ಳೆ-ನಿ-ಗಿ-ನಿಗಿ
ಬಿಟ್ಟು-ಬಿ-ಟ್ಟನು
ಬಿಟ್ಟು-ಬಿಡು
ಬಿಟ್ಟು-ಬಿ-ಡು-ತ್ತೇನೆ
ಬಿಟ್ಟು-ಬಿ-ಡು-ವು-ದೆಂ-ದರೆ
ಬಿಟ್ಟು-ಹೋ-ಗ-ಲಿಲ್ಲ
ಬಿಟ್ಟು-ಹೋದ
ಬಿಟ್ಟು-ಹೋ-ದಾಗ
ಬಿಟ್ಟು-ಹೋ-ದು-ದ-ರಿಂದ
ಬಿಟ್ಟು-ಹೋ-ಯಿತು
ಬಿಡದೆ
ಬಿಡ-ಬೇಕು
ಬಿಡ-ಬೇ-ಕೆಂದು
ಬಿಡ-ಬೇಡ
ಬಿಡಲಿ
ಬಿಡ-ಲಿಲ್ಲ
ಬಿಡ-ಲ್ಪ-ಟ್ಟ-ವನೂ
ಬಿಡಾರ
ಬಿಡಾ-ರಕ್ಕೆ
ಬಿಡಾ-ರ-ದಲ್ಲಿ
ಬಿಡಾ-ರ-ವನ್ನು
ಬಿಡಿ-ಬಿ-ಡಿ-ಯಾ-ಗಿ-ರು-ವಂತೆ
ಬಿಡಿರಿ
ಬಿಡಿ-ಸ-ಬೇ-ಕಾ-ದರೆ
ಬಿಡಿ-ಸ-ಬೇ-ಕಾ-ಯಿತು
ಬಿಡಿ-ಸ-ಬೇಕು
ಬಿಡಿಸಿ
ಬಿಡಿ-ಸಿ-ಕೊಂಡು
ಬಿಡಿ-ಸಿ-ಕೊ-ಳ್ಳಲಿ
ಬಿಡಿ-ಸಿ-ಕೊ-ಳ್ಳು-ವುದು
ಬಿಡಿ-ಸಿದ
ಬಿಡಿ-ಸಿ-ದ-ರೆಂದು
ಬಿಡಿ-ಸಿದೆ
ಬಿಡಿಸು
ಬಿಡಿ-ಸು-ವ-ವರು
ಬಿಡಿ-ಸುವು
ಬಿಡು
ಬಿಡು-ಗಡೆ
ಬಿಡು-ಗ-ಡೆಯ
ಬಿಡು-ಗ-ಡೆ-ಯನ್ನು
ಬಿಡು-ಗ-ಡೆ-ಯಿಲ್ಲ
ಬಿಡು-ಗ-ಡೆ-ಹೊಂದಿ
ಬಿಡು-ಗ-ಣ್ಣ-ನಾದ
ಬಿಡು-ಗ-ಣ್ಣ-ರನ್ನೂ
ಬಿಡು-ಗ-ಣ್ಣು-ಗಳಿಂದ
ಬಿಡು-ತ್ತದೆ
ಬಿಡು-ತ್ತಲೆ
ಬಿಡುತ್ತಾ
ಬಿಡು-ತ್ತಾ-ರೆಯೆ
ಬಿಡು-ತ್ತಿ-ದ್ದನು
ಬಿಡು-ತ್ತಿ-ರಲು
ಬಿಡು-ತ್ತಿ-ರು-ವಾ-ಗಲೂ
ಬಿಡುವ
ಬಿಡು-ವಂ-ತಿ-ರ-ಲಿಲ್ಲ
ಬಿಡು-ವನು
ಬಿಡು-ವ-ವ-ನಲ್ಲ
ಬಿಡು-ವು-ದಿಲ್ಲ
ಬಿಡು-ವುದೆ
ಬಿಡೋಣ
ಬಿಡೋ-ಣ-ವೆಂ-ದು-ಕೊಂಡು
ಬಿತ್ತಂತೆ
ಬಿತ್ತಿ
ಬಿತ್ತಿ-ದನು
ಬಿತ್ತು
ಬಿತ್ತು-ತ್ತಿ-ರು-ವಿರಿ
ಬಿತ್ತೆಂದು
ಬಿದಿರ
ಬಿದಿ-ರಿ-ನಲ್ಲಿ
ಬಿದಿರು
ಬಿದಿ-ರು-ಮೆ-ಳೆ-ಗಳು
ಬಿದಿ-ರು-ಮೆ-ಳೆ-ಯಲ್ಲಿ
ಬಿದ್ದ
ಬಿದ್ದಂ-ತಾ-ದು-ದ-ರಿಂದ
ಬಿದ್ದ-ಕಡೆ
ಬಿದ್ದನು
ಬಿದ್ದ-ಮೇಲೆ
ಬಿದ್ದರು
ಬಿದ್ದರೆ
ಬಿದ್ದ-ಲ್ಲಿಂದ
ಬಿದ್ದಳು
ಬಿದ್ದ-ವ-ರ-ನ್ನಲ್ಲ
ಬಿದ್ದ-ವ-ರ-ನ್ನೆಲ್ಲ
ಬಿದ್ದವು
ಬಿದ್ದಿ-ತಂತೆ
ಬಿದ್ದಿತು
ಬಿದ್ದಿತ್ತು
ಬಿದ್ದಿದೆ
ಬಿದ್ದಿದ್ದ
ಬಿದ್ದಿ-ರ-ಬ-ಹುದು
ಬಿದ್ದಿ-ರ-ಬೇಕು
ಬಿದ್ದಿ-ರಲಿ
ಬಿದ್ದಿರಿ
ಬಿದ್ದಿರು
ಬಿದ್ದಿ-ರು-ತ್ತದೆ
ಬಿದ್ದಿ-ರುವ
ಬಿದ್ದಿ-ರು-ವಾಗ
ಬಿದ್ದಿ-ರು-ವಾ-ಗಲೂ
ಬಿದ್ದಿ-ರುವು
ಬಿದ್ದಿ-ರು-ವುದನ್ನು
ಬಿದ್ದು
ಬಿದ್ದು-ಕೊಂ-ಡನು
ಬಿದ್ದು-ದನ್ನು
ಬಿದ್ದು-ದ-ರಿಂದ
ಬಿದ್ದು-ದ-ರಿಂ-ದಾದ
ಬಿದ್ದು-ವೆಂದು
ಬಿದ್ದು-ಹೋ-ಗಲಿ
ಬಿದ್ದು-ಹೋ-ಗು-ವುದೇ
ಬಿದ್ದು-ಹೋದ
ಬಿದ್ದು-ಹೋ-ದರೂ
ಬಿದ್ದು-ಹೋ-ದ-ವ-ರನ್ನು
ಬಿದ್ದು-ಹೋ-ಯಿತು
ಬಿದ್ದೆ
ಬಿನಿನ್ನ
ಬಿನ್ನಹ
ಬಿನ್ನಾಣ
ಬಿನ್ನಾ-ಣದ
ಬಿರಿ
ಬಿರಿದ
ಬಿರಿ-ಬಿರಿ
ಬಿರಿಯು
ಬಿರಿ-ಯುತ್ತಾ
ಬಿರಿ-ಯು-ವಂತೆ
ಬಿರು
ಬಿರು-ಗ-ಣ್ಣಿ-ನಿಂದ
ಬಿರು-ಗಾಳಿ
ಬಿರು-ಗಾ-ಳಿ-ಗಳು
ಬಿರು-ಗಾ-ಳಿಗೆ
ಬಿರು-ಗಾ-ಳಿ-ಯಂತೆ
ಬಿರು-ಗಾ-ಳಿ-ಯಿಂದ
ಬಿರು-ಗಾ-ಳಿಯು
ಬಿರು-ಗಾ-ಳಿ-ಯೆ-ದ್ದಿತು
ಬಿರು-ಗಾ-ಳಿ-ಯೆದ್ದು
ಬಿರು-ದನ್ನು
ಬಿರುದು
ಬಿರು-ಬೇ-ಸಗೆ
ಬಿರು-ಬೇ-ಸ-ಗೆಯ
ಬಿಲ-ವನ್ನು
ಬಿಲ್
ಬಿಲ್ದ-ನಿಗೆ
ಬಿಲ್ಲನ್ನು
ಬಿಲ್ಲನ್ನೂ
ಬಿಲ್ಲಲ್ಲಿ
ಬಿಲ್ಲಿ
ಬಿಲ್ಲಿಗೆ
ಬಿಲ್ಲಿನ
ಬಿಲ್ಲಿ-ನಂತೆ
ಬಿಲ್ಲಿ-ನಲ್ಲಿ
ಬಿಲ್ಲಿ-ನಿಂದ
ಬಿಲ್ಲಿ-ನೊ-ಡನೆ
ಬಿಲ್ಲು
ಬಿಲ್ಲು-ಗಳನ್ನೂ
ಬಿಲ್ಲು-ಗಾರ
ಬಿಲ್ಲು-ಬಾ-ಣ-ಗಳನ್ನು
ಬಿಲ್ಲು-ಬಾ-ಣ-ಗ-ಳೊ-ಡನೆ
ಬಿಳಿ
ಬಿಳಿ-ದಾದ
ಬಿಳಿಯ
ಬಿಳುಪು
ಬಿಸಾ-ಡಿದ
ಬಿಸಾ-ಡಿ-ದರು
ಬಿಸಿ
ಬಿಸಿ-ಯಾ-ಗಿಯೋ
ಬಿಸಿಲ
ಬಿಸಿ-ಲಿನ
ಬಿಸಿಲು
ಬಿಸಿ-ಲು-ಗ-ಳ-ನ್ನಾ-ಗಲಿ
ಬಿಸಿ-ಲು-ಗ-ಳಿಗೆ
ಬಿಸಿ-ಲು-ಗಳು
ಬಿಸಿ-ಲೇ-ರುತ್ತಾ
ಬಿಸಿ-ಲ್ಗು-ದು-ರೆಯ
ಬಿಸಿ-ಲ್ದೊ-ರೆ-ಯನ್ನು
ಬಿಸುಟ
ಬಿಸು-ಟನು
ಬಿಸುಟು
ಬೀಗಿ
ಬೀಗಿ-ಬಿ-ರಿ-ಯು-ತ್ತಿ-ರುವ
ಬೀಜ
ಬೀಜ-ಗಳನ್ನು
ಬೀಜ-ಗಳನ್ನೂ
ಬೀಜ-ಗಳನ್ನೆಲ್ಲ
ಬೀಜ-ಗಳು
ಬೀಜ-ಗ-ಳೆಲ್ಲ
ಬೀಜ-ದಂ-ತಾ-ಗು-ತ್ತದೆ
ಬೀಜ-ದಂ-ತಾ-ಗು-ವನು
ಬೀಜ-ದಿಂದ
ಬೀಜ-ಮಾತ್ರ
ಬೀಜ-ರೂ-ಪ-ವಾ-ಗಿ-ರುವ
ಬೀಜ-ರೂ-ಪ-ವಾ-ಗಿವೆ
ಬೀಜ-ರೂ-ಪ-ವಾದ
ಬೀಜ-ರೂ-ಪ-ವಾ-ದುದು
ಬೀಜ-ರೂ-ಪ-ವಾ-ಯಿ-ತೆಂದು
ಬೀಜ-ವನ್ನು
ಬೀಡು-ಬಿ-ಟ್ಟಿ-ದ್ದನು
ಬೀಡು-ಬಿ-ಟ್ಟು-ಕೊಂ-ಡಿದ್ದ
ಬೀದಿ
ಬೀದಿ-ಗಳಲ್ಲಿ
ಬೀದಿ-ಗ-ಳ-ಲ್ಲೆಲ್ಲ
ಬೀದಿ-ಗ-ಳಿ-ಗೆಲ್ಲ
ಬೀದಿ-ಗಳು
ಬೀದಿಗೆ
ಬೀದಿಯ
ಬೀದಿ-ಯಲ್ಲಿ
ಬೀದಿ-ಯು-ದ್ದಕ್ಕೂ
ಬೀರಿ
ಬೀರುತ್ತಾ
ಬೀರು-ತ್ತಿ-ದ್ದರೂ
ಬೀರು-ವು-ದೇನು
ಬೀಳ-ಕೂ-ಡದು
ಬೀಳ-ದಂತೆ
ಬೀಳದೆ
ಬೀಳ-ಲಿಲ್ಲ
ಬೀಳಲು
ಬೀಳಿ-ಸಿದ
ಬೀಳಿ-ಸಿ-ದನು
ಬೀಳಿ-ಸಿ-ದವು
ಬೀಳಿ-ಸು-ತ್ತಿ-ದ್ದನು
ಬೀಳು
ಬೀಳು-ತ್ತಲೆ
ಬೀಳು-ತ್ತವೆ
ಬೀಳುತ್ತಾ
ಬೀಳು-ತ್ತಾನೆ
ಬೀಳು-ತ್ತಾ-ರಂ-ತಲ್ಲಾ
ಬೀಳು-ತ್ತಿದ್ದ
ಬೀಳು-ತ್ತಿ-ದ್ದಂತೆ
ಬೀಳು-ತ್ತಿವೆ
ಬೀಳು-ವಂ-ತಾ-ಗುವ
ಬೀಳು-ವಂತೆ
ಬೀಳು-ವ-ಷ್ಟ-ರಲ್ಲಿ
ಬೀಳುವು
ಬೀಳು-ವು-ದಕ್ಕೆ
ಬೀಳು-ವು-ದಿಲ್ಲ
ಬೀಳು-ವುದು
ಬೀಳು-ವು-ದೆಂ-ದ-ರೇನು
ಬೀಳು-ವುದೇ
ಬೀಳ್ಕೊಂ-ಡ-ವರೆ
ಬೀಳ್ಕೊಂಡು
ಬೀಳ್ಕೊಟ್ಟ
ಬೀಳ್ಕೊ-ಟ್ಟನು
ಬೀಳ್ಕೊ-ಟ್ಟ-ಮೇಲೆ
ಬೀಳ್ಕೊ-ಟ್ಟರು
ಬೀಳ್ಕೊಟ್ಟು
ಬೀಳ್ಕೊ-ಡಲು
ಬೀಳ್ಕೊ-ಡುತ್ತಾ
ಬೀಳ್ಕೊ-ಳ್ಳ-ಲೆಂದು
ಬೀಳ್ಕೊ-ಳ್ಳುವ
ಬೀಳ್ಕೊ-ಳ್ಳು-ವಾಗ
ಬೀಸ-ಣಿಗೆ
ಬೀಸ-ಣಿ-ಗೆ-ಗಳು
ಬೀಸ-ಣಿ-ಗೆ-ಯನ್ನು
ಬೀಸಿ
ಬೀಸಿ-ಕೊಂಡು
ಬೀಸಿತು
ಬೀಸಿದ
ಬೀಸಿ-ದಂ-ತಾಗಿ
ಬೀಸಿ-ದನು
ಬೀಸು-ತ್ತದೆ
ಬೀಸುತ್ತಾ
ಬೀಸು-ತ್ತಿತ್ತು
ಬೀಸು-ತ್ತಿದೆ
ಬೀಸು-ತ್ತಿ-ದ್ದಳು
ಬೀಸು-ತ್ತಿ-ರುವ
ಬೀಸುವ
ಬೀಸು-ವಂತೆ
ಬೀಸು-ವು-ದೇ-ನು-ಇ-ತ್ಯಾದಿ
ಬುಗ್ಗೆ-ಗಳು
ಬುಗ್ಗೆ-ಯನ್ನ
ಬುಟ್ಟಿ-ಯಲ್ಲಿ
ಬುಟ್ಟಿ-ಯಿಂದ
ಬುಡಕ್ಕೆ
ಬುಡ-ದಲ್ಲಿ
ಬುಡು-ಬುಡು
ಬುತ್ತಿಯ
ಬುತ್ತಿ-ಯನ್ನು
ಬುತ್ತಿ-ಯೊಂ-ದನ್ನು
ಬುದ್ಧನಿ
ಬುದ್ಧ-ನಿ-ಗಿಂ-ತಲೂ
ಬುದ್ಧನು
ಬುದ್ಧಾ-ವ-ತಾರ
ಬುದ್ಧಿ
ಬುದ್ಧಿ-ಗ-ಳುಂ-ಟಾ-ಗು-ತ್ತವೆ
ಬುದ್ಧಿಗೂ
ಬುದ್ಧಿಗೆ
ಬುದ್ಧಿಯ
ಬುದ್ಧಿ-ಯನ್ನು
ಬುದ್ಧಿ-ಯನ್ನೂ
ಬುದ್ಧಿ-ಯಲ್ಲಿ
ಬುದ್ಧಿ-ಯಿಂದ
ಬುದ್ಧಿ-ಯಿಂ-ದಲೆ
ಬುದ್ಧಿ-ಯಿ-ಲ್ಲ-ದಿ-ದ್ದರೆ
ಬುದ್ಧಿ-ಯುಂ-ಟಾ-ಗಲಿ
ಬುದ್ಧಿ-ಯು-ಳ್ಳ-ವಳು
ಬುದ್ಧಿ-ಯೆಂಬ
ಬುದ್ಧಿ-ಯೆಂ-ಬುದು
ಬುದ್ಧಿಯೇ
ಬುದ್ಧಿ-ವಂ-ತರು
ಬುದ್ಧಿ-ವಾದ
ಬುದ್ಧಿ-ವಾ-ದಕ್ಕೆ
ಬುದ್ಧಿ-ವಾ-ದ-ಗಳನ್ನು
ಬುದ್ಧಿ-ವಾ-ದ-ದಂತೆ
ಬುದ್ಧಿ-ವಾ-ದ-ವನ್ನು
ಬುದ್ಧಿ-ವಾ-ದವೆ
ಬುದ್ಧಿ-ವಾ-ದ-ವೆಂದು
ಬುದ್ಧಿ-ಶಾ-ಲಿ-ಯೆಂದು
ಬುದ್ಧಿ-ಹೀ-ನ-ನಾದ
ಬುದ್ಧೀಂ-ದ್ರ-ಯ-ಮನಃ
ಬುದ್ಧೇ
ಬುದ್ಧೇಂ-ದ್ರಿ-ಯಾ-ಽಸವಃ
ಬುದ್ಬು-ದ-ಗಳು
ಬುಧ
ಬುಧ-ಗ್ರ-ಹ-ವಿದೆ
ಬುಧನ
ಬುಧ-ನಿಂದ
ಬುಧನು
ಬುವನು
ಬುವಿ
ಬುವಿಗೆ
ಬುವಿ-ಬಾ-ನು-ಗ-ಳೆ-ರ-ಡ-ರ-ಲ್ಲಿಯೂ
ಬುವಿ-ಯ-ವರೂ
ಬುಸು-ಗು-ಟ್ಟಿತು
ಬುಸು-ಗು-ಟ್ಟುತ್ತಾ
ಬುಸು-ಗು-ಟ್ಟು-ವನು
ಬೂದಿ
ಬೂದಿ-ಮು-ಚ್ಚಿದ
ಬೂದಿಯ
ಬೂದಿ-ಯಾ-ಗ-ತೊ-ಡ-ಗಿ-ದವು
ಬೂದಿ-ಯಾ-ಗದ
ಬೂದಿ-ಯಾಗಿ
ಬೂದಿ-ಯಾ-ಗಿ-ಹೋ-ಯಿತು
ಬೂದಿ-ಯಾ-ಯಿತು
ಬೂದಿಯೆ
ಬೂರು-ಗದ
ಬೂೃಹ್ಯಂಗ
ಬೃಂದಾ
ಬೃಂದಾ-ವ-ನ-ಇ-ವು-ಗಳ
ಬೃಂದಾ-ವ-ನದ
ಬೃಂದಾ-ವ-ನ-ದಲ್ಲಿ
ಬೃಂದಾ-ವ-ನ-ದ-ಲ್ಲಿ-ದ್ದುದು
ಬೃಂದಾ-ವ-ನ-ದಲ್ಲೆಲ್ಲಾ
ಬೃಂದಾ-ವ-ನ-ವನ್ನು
ಬೃಂದಾ-ವ-ನ-ವಿ-ಹಾ-ರಿ-ಯಾದ
ಬೃಂದಾ-ವ-ನವೆ
ಬೃಂದಾ-ವ-ನ-ವೆಂಬ
ಬೃಹ
ಬೃಹ-ತ್ಗ್ರಂ-ಥ-ವನ್ನು
ಬೃಹ-ತ್ಸೇನ
ಬೃಹ-ತ್ಸೇ-ನನು
ಬೃಹದಾ
ಬೃಹ-ದ್ಬಲ
ಬೃಹ-ದ್ಭಾನು
ಬೃಹ-ದ್ರ-ಥ-ನನ್ನು
ಬೃಹ-ನ್ನಾ-ರ-ದೀಯ
ಬೃಹ-ಸ್ಪತಿ
ಬೃಹ-ಸ್ಪ-ತಿಗೆ
ಬೃಹ-ಸ್ಪ-ತಿಯ
ಬೃಹ-ಸ್ಪ-ತಿ-ಯಂ-ತೆಯೂ
ಬೃಹ-ಸ್ಪ-ತಿ-ಯನ್ನು
ಬೃಹ-ಸ್ಪ-ತಿ-ಯಲ್ಲಿ
ಬೃಹ-ಸ್ಪ-ತಿಯು
ಬೃಹ-ಸ್ಪ-ತಿ-ವ-ನ-ವೆಂಬ
ಬೆಂಕಿ
ಬೆಂಕಿಗೆ
ಬೆಂಕಿಯ
ಬೆಂಕಿ-ಯ-ನ್ನಿ-ಡಿ-ಸಿ-ದನು
ಬೆಂಕಿ-ಯಲ್ಲಿ
ಬೆಂಕಿ-ಯಾಗಿ
ಬೆಂಕಿ-ಯಿಂದ
ಬೆಂಕಿ-ಯಿಂ-ದಲೊ
ಬೆಂಕಿಯು
ಬೆಂಕಿಯೇ
ಬೆಂಕೆ
ಬೆಂಗ-ಳೂ-ರಿನ
ಬೆಂಗ-ಳೂ-ರಿ-ನಿಂದ
ಬೆಂಗಾ-ವ-ಲಾಗಿ
ಬೆಂಡಾ-ಗಿ-ರುವ
ಬೆಂದ
ಬೆಂದ-ವ-ರಿಗೆ
ಬೆಂದು
ಬೆಂದು-ಹೋ-ಗು-ತ್ತಿ-ದ್ದೇವೆ
ಬೆಂದು-ಹೋ-ಗು-ತ್ತಿ-ರುವ
ಬೆಂದು-ಹೋ-ದ-ರೆಂದು
ಬೆಂಬ-ತ್ತಿ-ದನು
ಬೆಂಬಲ
ಬೆಂಬ-ಲ-ಕ್ಕಿ-ರು-ವಾಗ
ಬೆಂಬ-ಲ-ದಿಂದ
ಬೆಂಬ-ಲ-ನಾ-ಗಿ-ರುವ
ಬೆಕ್ಕಸ
ಬೆಕ್ಕ-ಸ-ಬೆ-ರ-ಗಾ-ದರು
ಬೆಕ್ಕಿಗೆ
ಬೆಟ್ಟ
ಬೆಟ್ಟ-ಕ್ಕೆಂಥ
ಬೆಟ್ಟ-ಗ-ಳಂತೆ
ಬೆಟ್ಟ-ಗಳನ್ನು
ಬೆಟ್ಟ-ಗಳು
ಬೆಟ್ಟ-ಗು-ಡ್ಡ-ಗಳನ್ನೂ
ಬೆಟ್ಟದ
ಬೆಟ್ಟ-ದಂ-ತಹ
ಬೆಟ್ಟ-ದಂ-ತಿದ್ದ
ಬೆಟ್ಟ-ದಂತೆ
ಬೆಟ್ಟ-ದ-ಷ್ಟಿತ್ತು
ಬೆಟ್ಟ-ದಷ್ಟು
ಬೆಟ್ಟ-ದಿಂದ
ಬೆಟ್ಟ-ವನ್ನು
ಬೆಟ್ಟ-ವ-ನ್ನೇರಿ
ಬೆಡಗು
ಬೆಡ-ಗು-ಗಾತಿ
ಬೆಡ-ಗು-ಗಾ-ತಿ-ಯರೆ
ಬೆಡ-ಗು-ಗಾ-ತಿ-ಯಾದ
ಬೆಣ್ಣೆ
ಬೆಣ್ಣೆ-ಗ-ಳಿ-ರುವ
ಬೆಣ್ಣೆ-ಗಾಗಿ
ಬೆಣ್ಣೆ-ಯಂ-ತಹ
ಬೆಣ್ಣೆ-ಯನ್ನು
ಬೆಣ್ಣೆ-ಯ-ನ್ನೆಲ್ಲ
ಬೆಣ್ಣೆ-ಯೆಲ್ಲ
ಬೆಣ್ಣೆ-ಯೊ-ಡನೆ
ಬೆತ್ತದ
ಬೆತ್ತ-ಲೆ-ಯಾಗಿ
ಬೆತ್ತ-ಲೆ-ಯಿ-ದ್ದರೂ
ಬೆದ-ರಿಕೆ
ಬೆದ-ರಿದ
ಬೆದ-ರಿ-ದಂತೆ
ಬೆನ್ನ
ಬೆನ್ನಟ್ಟಿ
ಬೆನ್ನ-ಟ್ಟಿ-ಕೊಂಡು
ಬೆನ್ನ-ಟ್ಟಿತು
ಬೆನ್ನ-ಟ್ಟಿದ
ಬೆನ್ನ-ಟ್ಟಿ-ದನು
ಬೆನ್ನ-ಟ್ಟಿ-ದರು
ಬೆನ್ನ-ಟ್ಟಿ-ಬ-ರಲು
ಬೆನ್ನ-ಟ್ಟಿ-ಹೋ-ದರು
ಬೆನ್ನ-ಟ್ಟುವ
ಬೆನ್ನ-ಟ್ಟು-ವಂತೆ
ಬೆನ್ನ-ಮೇಲೆ
ಬೆನ್ನಿಗೆ
ಬೆನ್ನು
ಬೆನ್ನು-ಗಳನ್ನು
ಬೆಪ್ಪ-ನಂತೆ
ಬೆಪ್ಪು-ಗ-ಳಿರಾ
ಬೆರ-ಗಾಗಿ
ಬೆರ-ಗಾ-ಗಿ-ದ್ದಾನೆ
ಬೆರ-ಗಾದ
ಬೆರ-ಗಾ-ದ-ಒಮ್ಮೆ
ಬೆರ-ಗಾ-ದಳು
ಬೆರ-ಳನ್ನು
ಬೆರ-ಳಲ್ಲಿ
ಬೆರ-ಳಷ್ಟು
ಬೆರ-ಳ-ಸಂ-ದಿನ
ಬೆರ-ಳಿ-ನಿಂದ
ಬೆರಳು
ಬೆರ-ಳು-ಗಳಿಂದ
ಬೆರ-ಳು-ಗಳು
ಬೆರೆತ
ಬೆರೆ-ತನು
ಬೆಲೆ
ಬೆಲೆ-ಕೊಟ್ಟು
ಬೆಲ್ಲಿ
ಬೆಳ
ಬೆಳ-ಕ-ನ್ನಿತ್ತು
ಬೆಳ-ಕನ್ನು
ಬೆಳ-ಕ-ನ್ನುಂ-ಟು-ಮಾ-ಡಿ-ದನು
ಬೆಳ-ಕಾಗಿ
ಬೆಳ-ಕಾ-ಗು-ತ್ತದೆ
ಬೆಳ-ಕಾ-ಯಿತು
ಬೆಳ-ಕಿಗೆ
ಬೆಳ-ಕಿನ
ಬೆಳ-ಕಿ-ನಲ್ಲಿ
ಬೆಳ-ಕಿ-ನಿಂದ
ಬೆಳ-ಕಿ-ರು-ವಂತೆ
ಬೆಳಕು
ಬೆಳಕೇ
ಬೆಳ-ಕೊಂದು
ಬೆಳಗ
ಬೆಳ-ಗಾ-ಗ-ದೇನು
ಬೆಳ-ಗಾ-ಗು-ತ್ತಲೆ
ಬೆಳ-ಗಾ-ಗು-ತ್ತಲೇ
ಬೆಳ-ಗಾದ
ಬೆಳ-ಗಾ-ದರೆ
ಬೆಳ-ಗಾ-ಯಿತು
ಬೆಳ-ಗಿತು
ಬೆಳ-ಗಿ-ದಳು
ಬೆಳ-ಗಿ-ದವು
ಬೆಳ-ಗಿನ
ಬೆಳ-ಗಿ-ನಿಂದ
ಬೆಳಗು
ಬೆಳ-ಗುತ್ತ
ಬೆಳ-ಗುತ್ತಾ
ಬೆಳ-ಗು-ತ್ತಾನೆ
ಬೆಳ-ಗು-ತ್ತಿತ್ತು
ಬೆಳ-ಗು-ತ್ತಿದೆ
ಬೆಳ-ಗು-ತ್ತಿದ್ದ
ಬೆಳ-ಗು-ತ್ತಿ-ದ್ದನು
ಬೆಳ-ಗು-ತ್ತಿ-ದ್ದಾರೆ
ಬೆಳ-ಗು-ತ್ತಿ-ರು-ತ್ತದೆ
ಬೆಳ-ಗು-ತ್ತಿ-ರುವ
ಬೆಳ-ಗು-ತ್ತಿ-ರು-ವನು
ಬೆಳ-ಗು-ತ್ತಿ-ರು-ವು-ದ-ರಿಂದ
ಬೆಳ-ಗುವ
ಬೆಳ-ಗು-ವನೋ
ಬೆಳಗ್ಗೆ
ಬೆಳ-ದಂ-ತೆಲ್ಲ
ಬೆಳ-ದಿಂ-ಗಳ
ಬೆಳ-ದಿಂ-ಗ-ಳಾಗಿ
ಬೆಳ-ದಿಂ-ಗಳು
ಬೆಳದು
ಬೆಳ-ವ-ಣಿ-ಗೆ-ಗಾಗಿ
ಬೆಳ-ವ-ಣಿ-ಗೆಗೆ
ಬೆಳ-ವ-ಣಿ-ಗೆ-ಯನ್ನು
ಬೆಳಸಿ
ಬೆಳ-ಸಿದೆ
ಬೆಳ-ಸು-ತ್ತಾರೆ
ಬೆಳ-ಸುವ
ಬೆಳ-ಸು-ವು-ದ-ಕ್ಕಾಗಿ
ಬೆಳಿಗ್ಗೆ
ಬೆಳು-ದಿಂ-ಗಳ
ಬೆಳು-ದಿಂ-ಗಳನ್ನು
ಬೆಳು-ದಿಂ-ಗಳಲ್ಲಿ
ಬೆಳು-ದಿಂ-ಗ-ಳಿ-ನಂ-ತಹ
ಬೆಳು-ದಿಂ-ಗಳು
ಬೆಳು-ದಿಂ-ಗ-ಳೊ-ಡನೆ
ಬೆಳೆ
ಬೆಳೆದ
ಬೆಳೆ-ದಂತೆ
ಬೆಳೆ-ದಂ-ತೆಲ್ಲ
ಬೆಳೆ-ದದ್ದು
ಬೆಳೆ-ದಿತ್ತು
ಬೆಳೆ-ದಿದ್ದ
ಬೆಳೆ-ದಿ-ದ್ದವು
ಬೆಳೆ-ದಿರು
ಬೆಳೆ-ದಿ-ರುವ
ಬೆಳೆದು
ಬೆಳೆ-ದು-ಕೊಂಡು
ಬೆಳೆ-ದು-ದೆಲ್ಲಿ
ಬೆಳೆ-ದು-ನಿಂತ
ಬೆಳೆ-ಯ-ತೊ-ಡ-ಗಿತು
ಬೆಳೆ-ಯ-ತ್ತಿ-ದ್ದಾನೆ
ಬೆಳೆ-ಯನ್ನು
ಬೆಳೆ-ಯ-ಬೇ-ಕಾ-ಯಿತು
ಬೆಳೆ-ಯ-ಬೇಕು
ಬೆಳೆ-ಯ-ಲಾ-ರಂ-ಭ-ವಾ-ಯಿತು
ಬೆಳೆ-ಯಿತು
ಬೆಳೆ-ಯಿ-ಲ್ಲ-ದಂ-ತಾ-ಗಿದೆ
ಬೆಳೆಯು
ಬೆಳೆ-ಯುತ್ತ
ಬೆಳೆ-ಯು-ತ್ತಲೆ
ಬೆಳೆ-ಯುತ್ತಾ
ಬೆಳೆ-ಯು-ತ್ತಿತ್ತು
ಬೆಳೆ-ಯು-ತ್ತಿದೆ
ಬೆಳೆ-ಯು-ತ್ತಿ-ದ್ದರು
ಬೆಳೆ-ಯು-ತ್ತಿ-ದ್ದಾನೆ
ಬೆಳೆ-ಯು-ತ್ತಿ-ದ್ದಾರೆ
ಬೆಳೆ-ಯು-ತ್ತಿ-ರುವ
ಬೆಳೆ-ಯು-ತ್ತಿ-ರು-ವ-ನೆಂ-ದಾ-ಯಿತು
ಬೆಳೆ-ಯು-ತ್ತಿ-ರು-ವುದನ್ನು
ಬೆಳೆ-ಯು-ತ್ತೇವೆ
ಬೆಳೆ-ಯು-ವನು
ಬೆಳೆ-ಯು-ವು-ದಕ್ಕೆ
ಬೆಳೆ-ಯು-ವುದು
ಬೆಳೆ-ಯೆಲ್ಲಿ
ಬೆಳೆಯೇ
ಬೆಳೆ-ಸ-ಬಾ-ರ-ದೆ-ನಿ-ಸಿತು
ಬೆಳೆಸಿ
ಬೆಳೆ-ಸಿ-ಕೊಂಡು
ಬೆಳೆ-ಸಿ-ದನು
ಬೆಳೆ-ಸಿ-ದರು
ಬೆಳೆ-ಸಿ-ದಿರಿ
ಬೆಳೆ-ಸಿದೆ
ಬೆಳೆ-ಸಿ-ದ್ದ-ರಿಂದ
ಬೆಳೆ-ಸಿ-ದ್ದೇನೆ
ಬೆಳೆ-ಸು-ತ್ತಿ-ದ್ದಳು
ಬೆಳೆ-ಸು-ವು-ದ-ಕ್ಕೋ-ಸ್ಕರ
ಬೆಳ್ಗೊಡೆ
ಬೆಳ್ಳಂ-ಬೆ-ಳ-ಕಾ-ಗಲಿ
ಬೆಳ್ಳಂ-ಬೆ-ಳಗಾ
ಬೆಳ್ಳಂ-ಬೆ-ಳಗು
ಬೆಳ್ಳ-ಗಾ-ದವು
ಬೆಳ್ಳ-ಗಿ-ದ್ದು-ದೆಲ್ಲ
ಬೆಳ್ಳ-ಗಿ-ರ-ಬೇಕು
ಬೆಳ್ಳ-ಗಿ-ರುವ
ಬೆಳ್ಳಗೆ
ಬೆಳ್ಳಿ
ಬೆಳ್ಳಿಯ
ಬೆವ-ತು-ಹೋ-ಯಿತು
ಬೆವ-ರನ್ನು
ಬೆವ-ರಿತು
ಬೆವರು
ಬೆಸ-ಸುತ್ತಿ
ಬೆಸ್ತನ
ಬೆಸ್ತನು
ಬೆಸ್ತ-ರನ್ನು
ಬೇಕಂತೆ
ಬೇಕ-ಲ್ಲವೆ
ಬೇಕಾ
ಬೇಕಾ-ಗಿತ್ತು
ಬೇಕಾ-ಗಿದೆ
ಬೇಕಾ-ಗಿ-ದೆಯೋ
ಬೇಕಾ-ಗಿ-ದ್ದದೂ
ಬೇಕಾ-ಗಿ-ದ್ದುದು
ಬೇಕಾ-ಗಿ-ದ್ದುದೂ
ಬೇಕಾ-ಗಿ-ರು-ವುದು
ಬೇಕಾ-ಗಿಲ್ಲ
ಬೇಕಾ-ಗು-ತ್ತದೆ
ಬೇಕಾ-ಗು-ತ್ತ-ದೆಯೆ
ಬೇಕಾ-ಗುವ
ಬೇಕಾದ
ಬೇಕಾ-ದ-ರ-ಲ್ಲಿಗೆ
ಬೇಕಾ-ದರೂ
ಬೇಕಾ-ದರೆ
ಬೇಕಾ-ದ-ಲ್ಲಿಗೆ
ಬೇಕಾ-ದ-ಷ್ಟನ್ನೆ
ಬೇಕಾ-ದಷ್ಟು
ಬೇಕಾ-ದು-ದನ್ನು
ಬೇಕಾ-ದು-ದಿಲ್ಲ
ಬೇಕಾ-ದು-ವನ್ನು
ಬೇಕಾ-ಬಿ-ಟ್ಟಿ-ಯಾಗಿ
ಬೇಕಾ-ಯಿತು
ಬೇಕಿಲ್ಲ
ಬೇಕು
ಬೇಕು-ಐ-ಶ್ವ-ರ್ಯಕ್ಕೆ
ಬೇಕು-ಹಾಗೆ
ಬೇಕೆ
ಬೇಕೆಂದು
ಬೇಕೆಂ-ದು-ಕೊಂ-ಡನು
ಬೇಕೆಂದೂ
ಬೇಕೆಂದೇ
ಬೇಕೆಂಬ
ಬೇಕೆ-ನ್ನಿ-ಸಿತು
ಬೇಕೆ-ನ್ನು-ವವ
ಬೇಕೆ-ನ್ನು-ವ-ವನು
ಬೇಕೇ
ಬೇಕೊ
ಬೇಕೋ
ಬೇಗ
ಬೇಗನೆ
ಬೇಗ-ಬೇಗ
ಬೇಗ-ಸುಡು
ಬೇಗೆ
ಬೇಗೆ-ಯನ್ನು
ಬೇಗೆ-ಯಿಂದ
ಬೇಟೆ
ಬೇಟೆ-ಗಾಗಿ
ಬೇಟೆ-ಗಾ-ರನ
ಬೇಟೆಗೆ
ಬೇಟೆ-ಗೆಂದು
ಬೇಟೆಯ
ಬೇಟೆ-ಯಲ್ಲಿ
ಬೇಟೆ-ಯಾಡಿ
ಬೇಟೆ-ಯಾ-ಡಿದ
ಬೇಟೆ-ಯಾ-ಡಿ-ದರು
ಬೇಟೆ-ಯಾ-ಡುತ್ತಾ
ಬೇಟೆ-ಯಾ-ಡು-ತ್ತಿ-ರು-ವಾಗ
ಬೇಟೆ-ಯಿಂದ
ಬೇಡ
ಬೇಡದೆ
ಬೇಡನ
ಬೇಡ-ನಂತೆ
ಬೇಡನು
ಬೇಡ-ನೊ-ಬ್ಬನು
ಬೇಡ-ಬೇಕು
ಬೇಡರ
ಬೇಡಲಿ
ಬೇಡಲು
ಬೇಡ-ವಾ-ದವು
ಬೇಡವೆ
ಬೇಡ-ವೆಂ
ಬೇಡ-ವೆಂದು
ಬೇಡ-ವೆ-ನ್ನಲು
ಬೇಡಿ
ಬೇಡಿಕೆ
ಬೇಡಿ-ಕೆ-ಗಳನ್ನು
ಬೇಡಿ-ಕೆಗೆ
ಬೇಡಿ-ಕೆ-ಯನ್ನು
ಬೇಡಿಕೊ
ಬೇಡಿ-ಕೊಂಡ
ಬೇಡಿ-ಕೊಂ-ಡನು
ಬೇಡಿ-ಕೊಂ-ಡರು
ಬೇಡಿ-ಕೊಂ-ಡರೂ
ಬೇಡಿ-ಕೊಂ-ಡಳು
ಬೇಡಿ-ಕೊಂ-ಡವು
ಬೇಡಿ-ಕೊಂ-ಡಾಗ
ಬೇಡಿ-ಕೊಂ-ಡಿತು
ಬೇಡಿ-ಕೊಂಡೆ
ಬೇಡಿ-ಕೊ-ಳ್ಳ-ಬೇ-ಕೆಂದು
ಬೇಡಿ-ಕೊ-ಳ್ಳ-ಬೇ-ಕೆಂ-ದು-ಕೊಂ-ಡರು
ಬೇಡಿ-ಕೊಳ್ಳು
ಬೇಡಿ-ಕೊ-ಳ್ಳು-ತ್ತಿ-ದ್ದರು
ಬೇಡಿ-ಕೊ-ಳ್ಳು-ತ್ತಿ-ದ್ದೇನೆ
ಬೇಡಿ-ಕೊ-ಳ್ಳು-ತ್ತೇನೆ
ಬೇಡಿ-ಕೊ-ಳ್ಳು-ತ್ತೇವೆ
ಬೇಡಿದ
ಬೇಡಿ-ದಂ-ತಾ-ಯಿ-ತಲ್ಲ
ಬೇಡಿ-ದ-ನಲ್ಲಾ
ಬೇಡಿ-ದನು
ಬೇಡಿ-ದ-ರಂತೆ
ಬೇಡಿ-ದರು
ಬೇಡಿ-ದರೂ
ಬೇಡಿ-ದಳು
ಬೇಡಿ-ದ-ವ-ರಿಗೆ
ಬೇಡಿ-ದಾಗ
ಬೇಡಿ-ದು-ದ-ನ್ನಿತ್ತು
ಬೇಡಿ-ದು-ದನ್ನು
ಬೇಡಿದೆ
ಬೇಡಿ-ದೆವು
ಬೇಡು
ಬೇಡು-ತ್ತಲೆ
ಬೇಡು-ತ್ತಿ-ದ್ದನು
ಬೇಡು-ತ್ತಿ-ರುವ
ಬೇಡು-ತ್ತಿ-ರು-ವನು
ಬೇಡು-ತ್ತಿ-ರು-ವಾಗ
ಬೇಡು-ತ್ತಿ-ರು-ವುದನ್ನು
ಬೇಡು-ತ್ತೇನೆ
ಬೇಡು-ತ್ತೇವೆ
ಬೇಡುವ
ಬೇಡು-ವಂತೆ
ಬೇಡು-ವ-ವ-ರಿಗೆ
ಬೇಡು-ವಾಗ
ಬೇಡು-ವು-ದಿಲ್ಲ
ಬೇಡು-ವು-ದಿ-ಷ್ಟೆ-ನಾವು
ಬೇಡು-ವುದು
ಬೇಡು-ವೆನು
ಬೇಡೆ-ನ್ನಲು
ಬೇಡೋಣ
ಬೇತಾ-ಳ-ದಂತೆ
ಬೇಯಿಸಿ
ಬೇಯು-ತ್ತದೆ
ಬೇರ-ಬೇ-ರೆ-ಯಾಗಿ
ಬೇರಾವ
ಬೇರಾ-ವು-ದನ್ನೂ
ಬೇರಾ-ವುದೂ
ಬೇರು
ಬೇರು-ಗಳ
ಬೇರು-ಸ-ಹಿತ
ಬೇರೆ
ಬೇರೆ-ಕಡೆ
ಬೇರೆ-ಬೇರೆ
ಬೇರೆ-ಬೇ-ರೆ-ಯಲ್ಲ
ಬೇರೆ-ಬೇ-ರೆ-ಯಾಗಿ
ಬೇರೆ-ಬೇ-ರೆ-ಯಾ-ದರೂ
ಬೇರೆ-ಬೇ-ರೆ-ಯೇನೂ
ಬೇರೆಯ
ಬೇರೆ-ಯಲ್ಲ
ಬೇರೆ-ಯ-ಲ್ಲವೊ
ಬೇರೆ-ಯ-ವರ
ಬೇರೆ-ಯ-ವ-ರನ್ನು
ಬೇರೆ-ಯ-ವರು
ಬೇರೆ-ಯಾಗಿ
ಬೇರೆ-ಯಾಗಿಯೂ
ಬೇರೆ-ಯೇನೂ
ಬೇರೊಂ-ದಿಲ್ಲ
ಬೇರೊಂದು
ಬೇರೊಂ-ದೆ-ಡೆಗೆ
ಬೇರೊಬ್ಬ
ಬೇರೊ-ಬ್ಬ-ನಿಗೆ
ಬೇರೊ-ಬ್ಬ-ರನ್ನು
ಬೇರೊ-ಬ್ಬ-ರಿಲ್ಲ
ಬೇರೊ-ಬ್ಬ-ಳನ್ನು
ಬೇರೊ-ಬ್ಬಳು
ಬೇಲ-ದ-ಮ-ರಕ್ಕೆ
ಬೇವಿನ
ಬೇಸ-ಗೆ-ಯಿಂದ
ಬೇಸ-ರ-ವಾ-ಗು-ವು-ದಿಲ್ಲ
ಬೇಸಿ-ಗೆಯ
ಬೈಠ-ಕ್ಖಾನೆ
ಬೈದ-ಳೆಂಬ
ಬೈದು
ಬೈಬ-ಲ್ಲಿ-ನ-ಲ್ಲಿಯೂ
ಬೊಂಬೆ-ಯಂತೆ
ಬೊಂಬೆ-ಯಿಂದ
ಬೊಕ್ಕ-ಸ-ವನ್ನು
ಬೊಕ್ಕೆ
ಬೊಗ-ಳು-ತ್ತಿ-ರು-ವೆಯಾ
ಬೊಗಸೆ
ಬೊಗ-ಸೆ-ಯ-ಲ್ಲಿದ್ದ
ಬೊಗ-ಸೆ-ಯಿಂದ
ಬೊಗ-ಸೆಯೆ
ಬೊಗ-ಸೆ-ಯೊ-ಳಕ್ಕೆ
ಬೋದಿ-ಗೆ-ಗ-ಳೆಲ್ಲ
ಬೋಧನೆ
ಬೋಧ-ನೆ-ಯನ್ನು
ಬೋಧ-ನೆ-ಯಾ-ಗು-ತ್ತದೆ
ಬೋಧಿ
ಬೋಧಿಸ
ಬೋಧಿ-ಸ-ಬೇಕು
ಬೋಧಿ-ಸ-ಲೆಂದು
ಬೋಧಿಸಿ
ಬೋಧಿ-ಸಿದ
ಬೋಧಿ-ಸಿ-ದನು
ಬೋಧಿ-ಸಿ-ದರೂ
ಬೋಧಿ-ಸಿ-ದ-ರೇನು
ಬೋಧಿ-ಸಿ-ದು-ದನ್ನೇ
ಬೋಧಿ-ಸಿದೆ
ಬೋಧಿ-ಸಿದ್ದ
ಬೋಧಿ-ಸಿ-ದ್ದೇನೆ
ಬೋಧಿ-ಸಿರಿ
ಬೋಧಿ-ಸಿರು
ಬೋಧಿಸು
ಬೋಧಿ-ಸು-ತ್ತದೆ
ಬೋಧಿ-ಸು-ತ್ತಾನೆ
ಬೋಧಿ-ಸು-ತ್ತಿ-ದ್ದರು
ಬೋಧಿ-ಸು-ವ-ವ-ರಾ-ದರು
ಬೋಧಿ-ಸು-ವುದು
ಬೋಧಿ-ಸು-ವುದೇ
ಬೋಧೆ
ಬೋಧೆಯಾ
ಬೋಧೆ-ಯಾ-ಗು-ವಂತೆ
ಬೋಧೆ-ಯಾ-ಯಿತು
ಬೋಪ-ದೇ-ವನ
ಬೋಪ-ದೇ-ವ-ನೆಂಬ
ಬೋಪ-ದೇ-ವೇನ
ಬೋಯಿ-ಗಳ
ಬೋಯಿ-ಗಳನ್ನು
ಬೋರ-ಲಾದ
ಬೋರ-ಲಾ-ದರೆ
ಬೋರೆಯ
ಬೋಳಿ-ಸಿ-ಹಾ-ಕಿ-ದನು
ಬೌದ್ಧರ
ಬ್ಪಹ್ಮನು
ಬ್ಬರು
ಬ್ಬರೂ
ಬ್ರಕಾ-ರ-ದಿಂದ
ಬ್ರತ್ರಯಂ
ಬ್ರಹ್ಮ
ಬ್ರಹ್ಮ-ನಾ-ರ-ದರ
ಬ್ರಹ್ಮ-ಇಂ-ದ್ರಾದಿ
ಬ್ರಹ್ಮ-ಇ-ವೆ-ಲ್ಲವೂ
ಬ್ರಹ್ಮ-ಕುಂ-ಡ-ವೆಂಬ
ಬ್ರಹ್ಮ-ಚರ್ಯ
ಬ್ರಹ್ಮ-ಚ-ರ್ಯ-ದ-ಲ್ಲಿಯೆ
ಬ್ರಹ್ಮ-ಚ-ರ್ಯ-ವ್ರ-ತ-ದಿಂದ
ಬ್ರಹ್ಮ-ಚ-ರ್ಯ-ವ್ರ-ತ-ವನ್ನು
ಬ್ರಹ್ಮ-ಚ-ರ್ಯಾ-ಶ್ರ-ಮ-ವಿದೆ
ಬ್ರಹ್ಮ-ಚ-ರ್ಯೆಗೆ
ಬ್ರಹ್ಮ-ಚಾರಿ
ಬ್ರಹ್ಮ-ಚಾ-ರಿ-ಗಳಲ್ಲಿ
ಬ್ರಹ್ಮ-ಚಾ-ರಿ-ಗ-ಳಾದ
ಬ್ರಹ್ಮ-ಚಾ-ರಿ-ಯಾಗಿ
ಬ್ರಹ್ಮ-ಚಾ-ರಿ-ಯಾದ
ಬ್ರಹ್ಮ-ಚಾ-ರಿಯು
ಬ್ರಹ್ಮಜ್ಞ
ಬ್ರಹ್ಮ-ಜ್ಞನ
ಬ್ರಹ್ಮ-ಜ್ಞ-ನಾದ
ಬ್ರಹ್ಮ-ಜ್ಞ-ನಾ-ದರೂ
ಬ್ರಹ್ಮ-ಜ್ಞ-ರಾದ
ಬ್ರಹ್ಮ-ಜ್ಞ-ರಿ-ಗಲ್ಲ
ಬ್ರಹ್ಮ-ಜ್ಞಾನ
ಬ್ರಹ್ಮ-ಜ್ಞಾ-ನ-ವನ್ನು
ಬ್ರಹ್ಮ-ಜ್ಞಾ-ನ-ವನ್ನೂ
ಬ್ರಹ್ಮ-ಜ್ಞಾನಿ
ಬ್ರಹ್ಮ-ಜ್ಞಾ-ನಿ-ಗಳಲ್ಲಿ
ಬ್ರಹ್ಮ-ಜ್ಞಾ-ನಿ-ಗ-ಳಿ-ಗಲ್ಲ
ಬ್ರಹ್ಮ-ಜ್ಞಾ-ನಿ-ಯಾಗಿ
ಬ್ರಹ್ಮ-ಜ್ಞಾ-ನಿ-ಯಾ-ಗಿದ್ದ
ಬ್ರಹ್ಮ-ಜ್ಞಾ-ನಿ-ಯಾ-ಗಿ-ದ್ದನು
ಬ್ರಹ್ಮ-ಜ್ಞಾ-ನಿ-ಯಾ-ಗಿ-ದ್ದ-ವನು
ಬ್ರಹ್ಮ-ಜ್ಞಾ-ನಿ-ಯಾದ
ಬ್ರಹ್ಮಣಃ
ಬ್ರಹ್ಮ-ಣ-ಸ್ಸಾ-ಕ್ಷಾ-ಜ್ಜಾತ
ಬ್ರಹ್ಮಣಾ
ಬ್ರಹ್ಮಣೇ
ಬ್ರಹ್ಮ-ಣ್ಯ-ದೇ-ವಾಯ
ಬ್ರಹ್ಮ-ತೇ-ಜ-ಸ್ಸನ್ನು
ಬ್ರಹ್ಮ-ತೇ-ಜ-ಸ್ಸಿ-ನಿಂದ
ಬ್ರಹ್ಮ-ತೇ-ಜ-ಸ್ಸಿ-ನಿಂ-ದಲೇ
ಬ್ರಹ್ಮ-ತೇ-ಜಸ್ಸು
ಬ್ರಹ್ಮ-ದೇವ
ಬ್ರಹ್ಮ-ದೇ-ವನ
ಬ್ರಹ್ಮ-ದೇ-ವ-ನಿಂದ
ಬ್ರಹ್ಮ-ದೇ-ವನು
ಬ್ರಹ್ಮ-ದೇ-ವನೂ
ಬ್ರಹ್ಮ-ದೇ-ವರು
ಬ್ರಹ್ಮ-ಧ್ಯಾನ
ಬ್ರಹ್ಮನ
ಬ್ರಹ್ಮ-ನಂತೆ
ಬ್ರಹ್ಮ-ನಂ-ತೆಯೂ
ಬ್ರಹ್ಮ-ನನ್ನು
ಬ್ರಹ್ಮ-ನನ್ನೇ
ಬ್ರಹ್ಮ-ನಲ್ಲಿ
ಬ್ರಹ್ಮ-ನಾಗಿ
ಬ್ರಹ್ಮ-ನಾ-ದ-ರೇನು
ಬ್ರಹ್ಮ-ನಿಂದ
ಬ್ರಹ್ಮ-ನಿ-ಗಿಂ-ತಲೂ
ಬ್ರಹ್ಮ-ನಿಗೂ
ಬ್ರಹ್ಮ-ನಿಗೆ
ಬ್ರಹ್ಮ-ನಿತ್ತ
ಬ್ರಹ್ಮ-ನಿ-ರ್ಮಿ-ತ-ವಾದ
ಬ್ರಹ್ಮ-ನಿಷ್ಠ
ಬ್ರಹ್ಮನು
ಬ್ರಹ್ಮನೂ
ಬ್ರಹ್ಮನೆ
ಬ್ರಹ್ಮ-ನೆ-ನಿ-ಸುವೆ
ಬ್ರಹ್ಮನೇ
ಬ್ರಹ್ಮ-ಪ-ದವಿ
ಬ್ರಹ್ಮ-ಪು-ರಾಣ
ಬ್ರಹ್ಮ-ಪು-ಷಿ-ಗಳನ್ನು
ಬ್ರಹ್ಮ-ಪು-ಷಿ-ಗ-ಳಿ-ಗಿಂತ
ಬ್ರಹ್ಮ-ಪು-ಷಿ-ಗಳೂ
ಬ್ರಹ್ಮ-ಬೀ-ಜಾ-ಕ್ಷ-ರ-ವಾದ
ಬ್ರಹ್ಮ-ಮಾ-ಯೆ-ಯಿಂದ
ಬ್ರಹ್ಮರ
ಬ್ರಹ್ಮ-ರಂ-ಧ್ರದ
ಬ್ರಹ್ಮ-ರಂ-ಧ್ರ-ವನ್ನು
ಬ್ರಹ್ಮ-ರಿಗೆ
ಬ್ರಹ್ಮ-ರುದ್ರ
ಬ್ರಹ್ಮ-ರು-ದ್ರರೂ
ಬ್ರಹ್ಮ-ರು-ದ್ರಾದಿ
ಬ್ರಹ್ಮರ್ಷಿ
ಬ್ರಹ್ಮ-ರ್ಷಿ-ಗಳು
ಬ್ರಹ್ಮ-ರ್ಷಿ-ಯಾದ
ಬ್ರಹ್ಮ-ರ್ಷಿ-ಯಾ-ದು-ದಕ್ಕೆ
ಬ್ರಹ್ಮ-ಲಿಂಗಂ
ಬ್ರಹ್ಮ-ಲೋ-ಕಕ್ಕೆ
ಬ್ರಹ್ಮ-ಲೋ-ಕ-ದಲ್ಲಿ
ಬ್ರಹ್ಮ-ಲೋ-ಕ-ವನ್ನು
ಬ್ರಹ್ಮ-ವಿ-ದ್ಯೆ-ಯಲ್ಲಿ
ಬ್ರಹ್ಮ-ವೆಂದೂ
ಬ್ರಹ್ಮ-ವೈ-ವರ್ತ
ಬ್ರಹ್ಮ-ವೈ-ವ-ರ್ತ-ಪು-ರಾಣ
ಬ್ರಹ್ಮ-ಸಭೆ
ಬ್ರಹ್ಮ-ಸ-ಭೆಗೆ
ಬ್ರಹ್ಮ-ಸೂತ್ರ
ಬ್ರಹ್ಮ-ಸೂ-ತ್ರ-ಗಳನ್ನೆಲ್ಲ
ಬ್ರಹ್ಮ-ಸೂ-ತ್ರ-ಭಾಷ್ಯ
ಬ್ರಹ್ಮ-ಸ್ವ-ರೂಪ
ಬ್ರಹ್ಮ-ಸ್ವ-ರೂ-ಪ-ವನ್ನು
ಬ್ರಹ್ಮ-ಸ್ವ-ರೂ-ಪ-ವಾ-ಗಿಯೂ
ಬ್ರಹ್ಮ-ಹ-ತ್ಯಾ-ದೋ-ಷ-ದಿಂದ
ಬ್ರಹ್ಮ-ಹ-ತ್ಯಾ-ದೋ-ಷವು
ಬ್ರಹ್ಮ-ಹ-ತ್ಯೆ-ಗಿಂತ
ಬ್ರಹ್ಮ-ಹ-ತ್ಯೆಗೆ
ಬ್ರಹ್ಮ-ಹ-ತ್ಯೆ-ಯನ್ನು
ಬ್ರಹ್ಮ-ಹ-ತ್ಯೆ-ಯಿಂದ
ಬ್ರಹ್ಮಾ
ಬ್ರಹ್ಮಾಂಡ
ಬ್ರಹ್ಮಾಂ-ಡ-ಗಳು
ಬ್ರಹ್ಮಾಂ-ಡ-ಗಳೂ
ಬ್ರಹ್ಮಾಂ-ಡದ
ಬ್ರಹ್ಮಾಂ-ಡ-ದೊ-ಡೆಯ
ಬ್ರಹ್ಮಾಂ-ಡ-ನಾ-ಯಕ
ಬ್ರಹ್ಮಾಂ-ಡ-ಪು-ರಾಣ
ಬ್ರಹ್ಮಾಂ-ಡ-ವನ್ನು
ಬ್ರಹ್ಮಾಂ-ಡ-ವನ್ನೂ
ಬ್ರಹ್ಮಾಂ-ಡ-ವನ್ನೆ
ಬ್ರಹ್ಮಾಂ-ಡ-ವ-ನ್ನೆಲ್ಲ
ಬ್ರಹ್ಮಾಂ-ಡವೆ
ಬ್ರಹ್ಮಾಂ-ಡ-ವೆಂಬ
ಬ್ರಹ್ಮಾಂ-ಡವೇ
ಬ್ರಹ್ಮಾ-ತ್ಮಕ
ಬ್ರಹ್ಮಾದಿ
ಬ್ರಹ್ಮಾ-ದಿ-ದೇ-ವ-ತೆ-ಗಳು
ಬ್ರಹ್ಮಾ-ನಂದ
ಬ್ರಹ್ಮಾ-ನಂ-ದ-ದಲ್ಲಿ
ಬ್ರಹ್ಮಾ-ನಂ-ದ-ವನ್ನು
ಬ್ರಹ್ಮಾ-ನಂ-ದ-ವೆಂಬ
ಬ್ರಹ್ಮಾ-ವ-ರ್ತ-ದೇ-ಶಕ್ಕೆ
ಬ್ರಹ್ಮಾ-ವ-ರ್ತ-ವೆಂಬ
ಬ್ರಹ್ಮಾ-ಸ-ನ-ದಲ್ಲಿ
ಬ್ರಹ್ಮಾಸ್ತ್ರ
ಬ್ರಹ್ಮಾ-ಸ್ತ್ರ-ವನ್ನು
ಬ್ರಹ್ಮಾ-ಸ್ತ್ರ-ವನ್ನೇ
ಬ್ರಹ್ಮೀ-ಭಾ-ವ-ವನ್ನು
ಬ್ರಹ್ಮೋ
ಬ್ರಾಹ್ಮಣ
ಬ್ರಾಹ್ಮ-ಣ-ಕು-ಮಾ-ರ-ನಾಗಿ
ಬ್ರಾಹ್ಮ-ಣ-ಗ-ಳ-ಲ್ಲಿನ
ಬ್ರಾಹ್ಮ-ಣ-ಜ-ನ್ಮ-ದಲ್ಲಿ
ಬ್ರಾಹ್ಮ-ಣನ
ಬ್ರಾಹ್ಮ-ಣ-ನನ್ನು
ಬ್ರಾಹ್ಮ-ಣ-ನಿಂದ
ಬ್ರಾಹ್ಮ-ಣ-ನಿಗೆ
ಬ್ರಾಹ್ಮ-ಣ-ನಿದ್ದ
ಬ್ರಾಹ್ಮ-ಣನು
ಬ್ರಾಹ್ಮ-ಣನೂ
ಬ್ರಾಹ್ಮ-ಣ-ನೆಂ-ದರೆ
ಬ್ರಾಹ್ಮ-ಣ-ನೆಂದು
ಬ್ರಾಹ್ಮ-ಣನೇ
ಬ್ರಾಹ್ಮ-ಣ-ನೊ-ಡನೆ
ಬ್ರಾಹ್ಮ-ಣ-ನೊಬ್ಬ
ಬ್ರಾಹ್ಮ-ಣ-ನೊ-ಬ್ಬ-ನಲ್ಲಿ
ಬ್ರಾಹ್ಮ-ಣ-ಪ-ತಿ-ಯನ್ನು
ಬ್ರಾಹ್ಮ-ಣ-ಪ್ರಿ-ಯ-ನಾ-ಗಿಯೂ
ಬ್ರಾಹ್ಮ-ಣರ
ಬ್ರಾಹ್ಮ-ಣ-ರಂತೆ
ಬ್ರಾಹ್ಮ-ಣ-ರನ್ನು
ಬ್ರಾಹ್ಮ-ಣ-ರನ್ನೂ
ಬ್ರಾಹ್ಮ-ಣ-ರನ್ನೆ
ಬ್ರಾಹ್ಮ-ಣ-ರಲ್ಲ
ಬ್ರಾಹ್ಮ-ಣ-ರಲ್ಲಿ
ಬ್ರಾಹ್ಮ-ಣ-ರಾದ
ಬ್ರಾಹ್ಮ-ಣ-ರಾ-ದರು
ಬ್ರಾಹ್ಮ-ಣರಿ
ಬ್ರಾಹ್ಮ-ಣ-ರಿಂದ
ಬ್ರಾಹ್ಮ-ಣ-ರಿ-ಗಾ-ಗಿಯೂ
ಬ್ರಾಹ್ಮ-ಣ-ರಿಗೂ
ಬ್ರಾಹ್ಮ-ಣ-ರಿಗೆ
ಬ್ರಾಹ್ಮ-ಣರು
ಬ್ರಾಹ್ಮ-ಣರೂ
ಬ್ರಾಹ್ಮ-ಣರೆ
ಬ್ರಾಹ್ಮ-ಣ-ರೆಂ-ದರೆ
ಬ್ರಾಹ್ಮ-ಣ-ರೆಂಬ
ಬ್ರಾಹ್ಮ-ಣ-ರೆಲ್ಲ
ಬ್ರಾಹ್ಮ-ಣರೇ
ಬ್ರಾಹ್ಮ-ಣ-ಳಾ-ದರೂ
ಬ್ರಾಹ್ಮ-ಣ-ವೇ-ಷದ
ಬ್ರಾಹ್ಮ-ಣ-ವೇ-ಷ-ದಿಂದ
ಬ್ರಾಹ್ಮ-ಣ-ವೇ-ಷ-ವನ್ನು
ಬ್ರಾಹ್ಮ-ಣ-ಶಾಪ
ಬ್ರಾಹ್ಮ-ಣಾ-ಧ-ಮ-ನನ್ನು
ಬ್ರಾಹ್ಮ-ಣೋ-ತ್ತಮ
ಬ್ರಾಹ್ಮಣ್ಯ
ಬ್ರಾಹ್ಮ-ಣ್ಯ-ವೆಂಬ
ಭ
ಭಂ
ಭಂಗ-ವನ್ನು
ಭಂಗ-ವಾ-ಗು-ವಂತೆ
ಭಂಗ-ವಾ-ದು-ದ-ಕ್ಕಾಗಿ
ಭಂಗಿಯ
ಭಂಗಿಸಿ
ಭಂಡಾರ
ಭಕ್ತ
ಭಕ್ತ-ಭ-ಗ-ವಂ-ತ-ಭಾ-ಗ-ವತ
ಭಕ್ತ-ಜನ
ಭಕ್ತ-ಜ-ನೈ-ಕ-ಲಭ್ಯ
ಭಕ್ತನ
ಭಕ್ತ-ನನ್ನು
ಭಕ್ತ-ನಾ-ಗದ
ಭಕ್ತ-ನಾಗಿ
ಭಕ್ತ-ನಾ-ಗಿದ್ದ
ಭಕ್ತ-ನಾ-ಗಿ-ದ್ದಾನೆ
ಭಕ್ತ-ನಾ-ಗಿದ್ದು
ಭಕ್ತ-ನಾ-ಗಿ-ರುವ
ಭಕ್ತ-ನಾದ
ಭಕ್ತ-ನಾ-ದ-ವನು
ಭಕ್ತ-ನಿಂದ
ಭಕ್ತ-ನಿಗೆ
ಭಕ್ತ-ನಿ-ರ-ಲಾ-ರದ
ಭಕ್ತ-ನಿ-ಲ್ಲದೆ
ಭಕ್ತನು
ಭಕ್ತನೂ
ಭಕ್ತ-ನೆ-ನಿ-ಸಿ-ಕೊಂಡ
ಭಕ್ತನೇ
ಭಕ್ತ-ಪಾಲ
ಭಕ್ತ-ಪಾ-ಲ-ಕನೂ
ಭಕ್ತ-ಪ್ರೇಮಿ
ಭಕ್ತರ
ಭಕ್ತ-ರ-ಕ್ಷಕ
ಭಕ್ತ-ರ-ಕ್ಷ-ಕ-ನಾದ
ಭಕ್ತ-ರ-ಕ್ಷಣೆ
ಭಕ್ತ-ರನ್ನು
ಭಕ್ತ-ರ-ಲ್ಲ-ವೆಂದು
ಭಕ್ತ-ರಲ್ಲಿ
ಭಕ್ತ-ರಾ-ಗ-ಬೇಕು
ಭಕ್ತ-ರಾ-ಗಿ-ರು-ವ-ವ-ರ-ನ್ನೆಲ್ಲ
ಭಕ್ತ-ರಾದ
ಭಕ್ತ-ರಿ-ಗಾಗಿ
ಭಕ್ತ-ರಿಗೂ
ಭಕ್ತ-ರಿಗೆ
ಭಕ್ತರು
ಭಕ್ತ-ವ-ತ್ಸಲ
ಭಕ್ತ-ವ-ತ್ಸ-ಲ-ನಾದ
ಭಕ್ತ-ವ-ತ್ಸ-ಲನೊ
ಭಕ್ತ-ಸ-ಮೂ-ಹ-ದೊ-ಡನೆ
ಭಕ್ತಿ
ಭಕ್ತಿ-ಇ-ವು-ಗಳನ್ನು
ಭಕ್ತಿ-ಎ-ರಡೂ
ಭಕ್ತಿ-ಗಳನ್ನು
ಭಕ್ತಿ-ಗಳು
ಭಕ್ತಿ-ಗ-ಳೆಂಬ
ಭಕ್ತಿಗೂ
ಭಕ್ತಿಗೆ
ಭಕ್ತಿ-ಪಾ-ಶ-ದಿಂದ
ಭಕ್ತಿ-ಪ್ರೇ-ಮ-ಗಳು
ಭಕ್ತಿ-ಭ-ರಿ-ತ-ನಾಗಿ
ಭಕ್ತಿ-ಭ-ರಿ-ತ-ವಾ-ದು-ದಾ-ಗಿಯೇ
ಭಕ್ತಿ-ಭಾವ
ಭಕ್ತಿ-ಭಾ-ವ-ಗಳನ್ನೂ
ಭಕ್ತಿ-ಭಾ-ವ-ದಿಂದ
ಭಕ್ತಿ-ಮಾರ್ಗ
ಭಕ್ತಿ-ಮಾ-ರ್ಗ-ವನ್ನು
ಭಕ್ತಿಯ
ಭಕ್ತಿ-ಯ-ನ್ನಿ-ಟ್ಟರೆ
ಭಕ್ತಿ-ಯ-ನ್ನಿ-ಡ-ಬೇಕು
ಭಕ್ತಿ-ಯ-ನ್ನಿಡಿ
ಭಕ್ತಿ-ಯ-ನ್ನಿ-ಡು-ವು-ದ-ರಿಂದ
ಭಕ್ತಿ-ಯನ್ನು
ಭಕ್ತಿ-ಯಲ್ಲಿ
ಭಕ್ತಿ-ಯಿಂದ
ಭಕ್ತಿಯು
ಭಕ್ತಿ-ಯು-ದಿ-ಸಿತು
ಭಕ್ತಿ-ಯು-ಳ್ಳ-ವ-ನಿಗೆ
ಭಕ್ತಿಯೂ
ಭಕ್ತಿ-ಯೆಂ-ದರೆ
ಭಕ್ತಿ-ಯೆಂಬ
ಭಕ್ತಿಯೇ
ಭಕ್ತಿ-ಯೇನೂ
ಭಕ್ತಿ-ಯೋಗ
ಭಕ್ತಿ-ಯೋ-ಗಕ್ಕೆ
ಭಕ್ತಿ-ಯೋ-ಗದ
ಭಕ್ತಿ-ಯೋ-ಗ-ದಿಂದ
ಭಕ್ತಿ-ಯೋ-ಗ-ವನ್ನು
ಭಕ್ತಿ-ಯೋ-ಗ-ವನ್ನೂ
ಭಕ್ತಿ-ಯೋ-ಗ-ವಾ-ಯಿತು
ಭಕ್ತಿ-ಯೋ-ಗ-ವೆಂದು
ಭಕ್ತಿ-ಶಾ-ಸ್ತ್ರ-ವನ್ನು
ಭಕ್ತಿ-ಹೀ-ನ-ರಾಗಿ
ಭಕ್ತೆಯ
ಭಕ್ತೋ-ತ್ತಮ
ಭಕ್ಷಿ-ಸ-ಬಾ-ರ-ದು-ಎಂದು
ಭಕ್ಷಿ-ಸ-ಹೊ-ರ-ಟರು
ಭಕ್ಷ್ಯ-ಗಳನ್ನೆಲ್ಲ
ಭಕ್ಷ್ಯ-ಭೋ-ಜ್ಯ-ಗಳನ್ನು
ಭಗ
ಭಗ-ದ-ತ್ತ-ನನ್ನು
ಭಗ-ನೆಂಬ
ಭಗವ
ಭಗ-ವಂತ
ಭಗ-ವಂ-ತಈ
ಭಗ-ವಂ-ತನ
ಭಗ-ವಂ-ತ-ನನ್ನು
ಭಗ-ವಂ-ತ-ನನ್ನೆ
ಭಗ-ವಂ-ತ-ನನ್ನೇ
ಭಗ-ವಂ-ತ-ನಲ್ಲ
ಭಗ-ವಂ-ತ-ನಲ್ಲಿ
ಭಗ-ವಂ-ತ-ನ-ಲ್ಲಿಯೂ
ಭಗ-ವಂ-ತ-ನ-ಲ್ಲಿಯೇ
ಭಗ-ವಂ-ತ-ನಾದ
ಭಗ-ವಂ-ತ-ನಾ-ದ-ವನು
ಭಗ-ವಂ-ತ-ನಿಂದ
ಭಗ-ವಂ-ತ-ನಿಂ-ದಲೆ
ಭಗ-ವಂ-ತ-ನಿಗೂ
ಭಗ-ವಂ-ತ-ನಿಗೆ
ಭಗ-ವಂ-ತ-ನಿ-ರುವ
ಭಗ-ವಂ-ತ-ನಿ-ಲ್ಲದೆ
ಭಗ-ವಂ-ತನು
ಭಗ-ವಂ-ತನೂ
ಭಗ-ವಂ-ತನೆ
ಭಗ-ವಂ-ತ-ನೆಂದೆ
ಭಗ-ವಂ-ತ-ನೆಂಬ
ಭಗ-ವಂ-ತನೇ
ಭಗ-ವಂ-ತ-ನೊ-ಬ್ಬನ
ಭಗ-ವಂತಾ
ಭಗ-ವತೇ
ಭಗ-ವತ್
ಭಗ-ವ-ತ್ಕ-ಥೆ-ಯನ್ನು
ಭಗ-ವ-ದ-ನು-ಗ್ರ-ಹ-ದಿಂದ
ಭಗ-ವ-ದ-ವ-ತಾರ
ಭಗ-ವ-ದಾ-ರಾ-ಧ-ನೆ-ಯಿಂದ
ಭಗ-ವ-ದ್ಗೀತೆ
ಭಗ-ವ-ದ್ಗೀ-ತೆ-ಯನ್ನು
ಭಗ-ವ-ದ್ಧ್ಯಾ-ನ-ತ-ತ್ಪ-ರ-ರಾ-ಗಿ-ರಲು
ಭಗ-ವ-ದ್ಭ-ಕ್ತ-ನಾ-ಗಿ-ದ್ದನು
ಭಗ-ವ-ದ್ಭ-ಕ್ತ-ನಾದ
ಭಗ-ವ-ದ್ಭ-ಕ್ತ-ನಾ-ದು-ದ-ರಿಂದ
ಭಗ-ವ-ದ್ಭ-ಕ್ತರ
ಭಗ-ವ-ದ್ಭ-ಕ್ತ-ರನ್ನು
ಭಗ-ವ-ದ್ಭ-ಕ್ತ-ರಲ್ಲಿ
ಭಗ-ವ-ದ್ಭ-ಕ್ತ-ರಾ-ದು-ದ-ರಿಂದ
ಭಗ-ವ-ದ್ಭ-ಕ್ತರು
ಭಗ-ವ-ದ್ಭಕ್ತಿ
ಭಗ-ವ-ದ್ಭ-ಕ್ತಿಯ
ಭಗ-ವ-ದ್ಭ-ಕ್ತಿ-ಯಿಂದ
ಭಗ-ವ-ದ್ರೂ-ಪ-ವಾ-ದು-ದೆಂದೂ
ಭಗ-ವನ್
ಭಗ-ವ-ನ್ನಾ-ಮ-ರೂ-ಪಾ-ಸ್ತ್ರ-ಕೀ-ರ್ತ-ನಾತ್
ಭಗ-ವ-ನ್ಮೂ-ರ್ತಿ-ಗ-ಳಾಗಿ
ಭಗ-ವಾ-ನೇವ
ಭಗ-ವಾನ್
ಭಗ-ವಾ-ನ್ಈ
ಭಗೀ-ರಥ
ಭಗೀ-ರ-ಥನ
ಭಗೀ-ರ-ಥನು
ಭಜ-ನೆ-ಯಲ್ಲಿ
ಭಜಿಸ
ಭಜಿ-ಸ-ಬಾ-ರದು
ಭಜಿ-ಸ-ಬೇಕು
ಭಜಿ-ಸಿ-ದ-ವ-ರಿಗೂ
ಭಜಿ-ಸಿ-ದು-ದ-ಕ್ಕಾಗಿ
ಭಜಿಸು
ಭಜಿ-ಸು-ವ-ವನು
ಭಜಿ-ಸು-ವ-ವ-ರನ್ನು
ಭಜಿ-ಸು-ವ-ವ-ರಿಗೆ
ಭಜಿ-ಸು-ವ-ವರು
ಭಟರು
ಭಟರೂ
ಭಟರೆ
ಭಣ್ಯ-ತಾ-ಮ-ನ್ಯ-ವಾರ್ತಾ
ಭತ್ತ-ವನ್ನು
ಭದ್ರ
ಭದ್ರ-ಕಾಳಿ
ಭದ್ರ-ಚಾರು
ಭದ್ರನು
ಭದ್ರ-ಬಾಹು
ಭದ್ರ-ವಾಗಿ
ಭದ್ರ-ಶ್ರ-ವ-ನೆಂ-ಬು-ವನು
ಭದ್ರ-ಸೇನ
ಭದ್ರಾ-ದೇ-ವಿ-ಸಂ-ಗ್ರಾ-ಮ-ಜಿತ್ತು
ಭದ್ರಾಶ್ವ
ಭದ್ರಾ-ಶ್ವ-ವರ್ಷ
ಭದ್ರೆ
ಭದ್ರೆಯು
ಭದ್ರೆ-ಯೆಂ-ಬು-ವ-ಳನ್ನು
ಭದ್ವಯಂ
ಭಯ
ಭಯಂ
ಭಯಂ-ಕರ
ಭಯಂ-ಕ-ರ-ನಾ-ಗು-ತ್ತಾನೆ
ಭಯಂ-ಕ-ರ-ನಾದ
ಭಯಂ-ಕ-ರನು
ಭಯಂ-ಕ-ರ-ರಾ-ದ-ವ-ರಿಗೂ
ಭಯಂ-ಕ-ರ-ಳಾ-ದರೆ
ಭಯಂ-ಕ-ರ-ವಾಗಿ
ಭಯಂ-ಕ-ರ-ವಾ-ಗಿತ್ತು
ಭಯಂ-ಕ-ರ-ವಾ-ಗಿದ್ದ
ಭಯಂ-ಕ-ರ-ವಾ-ಗಿ-ದ್ದು-ದ-ರಿಂದ
ಭಯಂ-ಕ-ರ-ವಾ-ಗಿವೆ
ಭಯಂ-ಕ-ರ-ವಾ-ಗು-ವುದು
ಭಯಂ-ಕ-ರ-ವಾದ
ಭಯಂ-ಕ-ರ-ವಾ-ದುದು
ಭಯಂ-ಕ-ರ-ವಾ-ದು-ದೆಂ-ಬು-ದನ್ನು
ಭಯಂ-ಕ-ರ-ವಿ-ಷ-ದಿಂದ
ಭಯಂ-ಕ-ರ-ವೆಂ-ಬು-ದನ್ನು
ಭಯಂ-ಕರಾ
ಭಯಂ-ಕ-ರಾ-ಕಾ-ರದ
ಭಯಂ-ಕ-ರಾ-ಕಾ-ರ-ದಲ್ಲಿ
ಭಯಂ-ಕ-ರಾ-ಕಾ-ರ-ನಾದ
ಭಯಂ-ಕ-ರಾ-ಕೃತಿ
ಭಯಕ್ಕೆ
ಭಯ-ಕ್ಕೆಯೇ
ಭಯ-ಗಳಿಂದ
ಭಯ-ಗೊಂಡ
ಭಯ-ಗೊ-ಳ್ಳದೆ
ಭಯ-ದಿಂದ
ಭಯ-ದಿಂ-ದಲೂ
ಭಯ-ದಿಂ-ದಲೆ
ಭಯ-ದಿಂ-ದಲೇ
ಭಯ-ದುಃಖ
ಭಯ-ನೆಂ-ಬ-ವನು
ಭಯ-ಪ-ಟ್ಟ-ವ-ನಂತೆ
ಭಯ-ಪ-ಡ-ಬೇಡ
ಭಯ-ಪ-ಡ-ಬೇಡಿ
ಭಯ-ಪ-ಡ-ಲಿಲ್ಲ
ಭಯ-ಪ-ಡು-ತ್ತ-ದೆ-ಯಂತೆ
ಭಯ-ಪ-ಡು-ವುದು
ಭಯ-ಭ-ಕ್ತಿ-ಗಳು
ಭಯ-ಭ-ಕ್ತಿ-ಯಿಂದ
ಭಯ-ರೂ-ಪ-ನಾ-ಗಿ-ರುವ
ಭಯ-ವಣ್ಣಾ
ಭಯ-ವನ್ನು
ಭಯ-ವ-ನ್ನುಂ-ಟು-ಮಾ-ಡುವ
ಭಯ-ವನ್ನೂ
ಭಯ-ವಾ-ಯಿತು
ಭಯ-ವಿಲ
ಭಯ-ವಿಲ್ಲ
ಭಯ-ವಿ-ಲ್ಲದೆ
ಭಯವೂ
ಭಯವೆ
ಭಯ-ವೆಂದು
ಭಯ-ವೆ-ಲ್ಲಿ-ಯದು
ಭಯವೇ
ಭಯ-ಹ-ತ್ತಿತು
ಭರತ
ಭರ-ತ-ಋಷಿ
ಭರ-ತ-ಖಂ-ಡದ
ಭರ-ತ-ಖಂ-ಡ-ವನ್ನು
ಭರ-ತ-ಖಂ-ಡ-ವ-ನ್ನೆಲ್ಲ
ಭರ-ತ-ಖಂ-ಡ-ವ-ನ್ನೆಲ್ಲಾ
ಭರ-ತ-ಖಂ-ಡ-ವೆಂಬ
ಭರ-ತನ
ಭರ-ತ-ನನ್ನು
ಭರ-ತ-ನಿಗೆ
ಭರ-ತನು
ಭರ-ತ-ನೆಂಬ
ಭರ-ತ-ಮ-ಹಾ-ಮು-ನಿಯ
ಭರ-ತ-ಮುನಿ
ಭರ-ತ-ಮು-ನಿಯು
ಭರ-ತಾ-ಗ್ರ-ಜ-ನಾದ
ಭರ-ದ್ವಾಜ
ಭರ-ದ್ವಾ-ಜನು
ಭರ-ದ್ವಾ-ಜ-ನೆಂ-ಬು-ವ-ನನ್ನು
ಭರ-ಧ್ವಾಜ
ಭರ-ವಸೆ
ಭರ-ವ-ಸೆ-ಯನ್ನು
ಭರ-ವ-ಸೆ-ಯಿತ್ತು
ಭರಿಸು
ಭರ್ಗೋ
ಭರ್ತ್ಯಾ-ತ್ಮ-ಕೇ-ತು-ಭಿಃ
ಭಲೆ
ಭಲ್ಲೆ
ಭವ
ಭವಂತು
ಭವತಿ
ಭವ-ನ-ಗಳಲ್ಲಿ
ಭವ-ನ-ವನ್ನು
ಭವ-ನ-ವಿತ್ತು
ಭವ-ಪಾಶ
ಭವ-ಬಂ-ಧ-ವನ್ನು
ಭವ-ಬಂ-ಧ-ಹರಂ
ಭವ-ಭ-ಯ-ಮ-ಪ-ಹಂ-ತುಂ
ಭವ-ಸಾ-ಗರ
ಭವ-ಸಾ-ಗ-ರ-ವನ್ನು
ಭವಾ-ದೃಕ್
ಭವಾನಿ
ಭವಾ-ನಿ-ಯೊ-ಡನೆ
ಭವಾನ್
ಭವಿ-ಷ್ಯತಿ
ಭವಿ-ಷ್ಯತ್
ಭವಿ-ಷ್ಯ-ತ್ತಿನ
ಭವಿ-ಷ್ಯ-ಪು-ರಾಣ
ಭವಿ-ಷ್ಯ-ವನ್ನು
ಭವಿ-ಸು-ವು-ದ-ಕ್ಕಾಗಿ
ಭವಿ-ಸು-ವು-ದ-ರಿಂದ
ಭವ್ಯ
ಭಾಗ
ಭಾಗ-ಗಳನ್ನು
ಭಾಗ-ಗಳಲ್ಲಿ
ಭಾಗ-ಗ-ಳಾಗಿ
ಭಾಗ-ಗಳಿಂದ
ಭಾಗ-ಗಳು
ಭಾಗ-ಗಳೇ
ಭಾಗದ
ಭಾಗ-ದಲ್ಲಿ
ಭಾಗ-ವತ
ಭಾಗ-ವ-ತ-ಇವು
ಭಾಗ-ವ-ತ-ಇ-ವೆ-ರಡು
ಭಾಗ-ವ-ತ-ಕಾ-ರನು
ಭಾಗ-ವ-ತಕ್ಕೂ
ಭಾಗ-ವ-ತಕ್ಕೆ
ಭಾಗ-ವ-ತ-ಗಳು
ಭಾಗ-ವ-ತದ
ಭಾಗ-ವ-ತ-ದಲ್ಲಿ
ಭಾಗ-ವ-ತ-ಧರ್ಮ
ಭಾಗ-ವ-ತ-ಧ-ರ್ಮದ
ಭಾಗ-ವ-ತ-ಧ-ರ್ಮ-ವನ್ನು
ಭಾಗ-ವ-ತ-ಧ-ರ್ಮ-ವೆಂದು
ಭಾಗ-ವ-ತ-ನಾಗಿ
ಭಾಗ-ವ-ತ-ನಾದ
ಭಾಗ-ವ-ತ-ನಾ-ದ-ವನು
ಭಾಗ-ವ-ತನು
ಭಾಗ-ವ-ತರ
ಭಾಗ-ವ-ತ-ರಲ್ಲಿ
ಭಾಗ-ವ-ತರು
ಭಾಗ-ವ-ತ-ವನ್ನು
ಭಾಗ-ವ-ತ-ವ-ನ್ನೋ-ದಿ-ದರೆ
ಭಾಗ-ವ-ತವು
ಭಾಗ-ವ-ತ-ವೆಂದರೆ
ಭಾಗ-ವ-ತ-ವೆಂಬ
ಭಾಗ-ವ-ತಾ-ಮೃ-ತ-ವನ್ನು
ಭಾಗ-ವ-ತಾ-ವ-ತ-ರಣ
ಭಾಗ-ವ-ತೋ-ತ್ತಮ
ಭಾಗ-ವ-ತೋ-ತ್ತ-ಮನು
ಭಾಗ-ವ-ತೋ-ತ್ತ-ಮ-ರಾಗಿ
ಭಾಗ-ವನ್ನು
ಭಾಗ-ವ-ಹಿಸಿ
ಭಾಗ-ವ-ಹಿ-ಸಿದ
ಭಾಗ-ವ-ಹಿ-ಸು-ತ್ತಿ-ದ್ದನು
ಭಾಗ-ವಿದೆ
ಭಾಗ-ವಿ-ಲ್ಲದೆ
ಭಾಗವೂ
ಭಾಗವೆ
ಭಾಗ್ಯ
ಭಾಗ್ಯಕ್ಕೆ
ಭಾಗ್ಯ-ಗಳ
ಭಾಗ್ಯ-ಗಳು
ಭಾಗ್ಯ-ಗ-ಳೆಲ್ಲ
ಭಾಗ್ಯ-ನಿ-ಧಿ-ಯಾ-ಗಿ-ರುವ
ಭಾಗ್ಯ-ವಂತ
ಭಾಗ್ಯ-ವತಿ
ಭಾಗ್ಯ-ವ-ತಿ-ಯ-ರಾದ
ಭಾಗ್ಯ-ವನ್ನು
ಭಾಗ್ಯ-ವ-ಶ-ದಿಂದ
ಭಾಗ್ಯವೆ
ಭಾಗ್ಯ-ವೆಂದು
ಭಾಗ್ಯವೇ
ಭಾಗ್ಯ-ವೇನು
ಭಾಗ್ಯ-ಶಾಲಿ
ಭಾಗ್ಯ-ಶಾ-ಲಿ-ಯೆಂದು
ಭಾನು
ಭಾನು-ಮಂತ
ಭಾಪು
ಭಾಮಾ-ಮ-ಣಿಯೆ
ಭಾರ
ಭಾರತ
ಭಾರ-ತಕ್ಕೆ
ಭಾರ-ತದ
ಭಾರ-ತ-ದಲ್ಲಿ
ಭಾರ-ತ-ಯುದ್ಧ
ಭಾರ-ತ-ಯು-ದ್ಧದ
ಭಾರ-ತ-ವನ್ನೂ
ಭಾರ-ತ-ವರ್ಷ
ಭಾರ-ತ-ವ-ರ್ಷ-ದಲ್ಲಿ
ಭಾರ-ತೀಯ
ಭಾರ-ದಷ್ಟು
ಭಾರ-ದಿಂದ
ಭಾರ-ಮಾಡು
ಭಾರ-ವನ್ನು
ಭಾರ-ವಾ-ಗಿ-ವೆಯೇ
ಭಾರ-ವಾದ
ಭಾರ-ವೆ-ನಿ-ಸಿತು
ಭಾವ
ಭಾವ-ದಲ್ಲಿ
ಭಾವ-ದಿಂದ
ಭಾವ-ನನ್ನು
ಭಾವ-ನಾದ
ಭಾವನೆ
ಭಾವ-ನೆಗೆ
ಭಾವ-ನೆಯ
ಭಾವ-ನೆ-ಯಿಂದ
ಭಾವ-ನೆ-ಯಿಂ-ದಲೆ
ಭಾವ-ನೆಯೆ
ಭಾವ-ನೊಬ್ಬ
ಭಾವ-ಮ-ಧು-ರ-ಭಾ-ವ-ಗೋ-ಪಿ-ಯಂತೆ
ಭಾವ-ಯೊ-ಶೋ-ದೆ-ಯಂತೆ
ಭಾವಾ
ಭಾವಾ-ನಾಂ
ಭಾವಾ-ವಿ-ಷ್ಟ-ರಾ-ದರು
ಭಾವಿ-ಸ-ಬ-ಹುದು
ಭಾವಿ-ಸ-ಬಾ-ರದು
ಭಾವಿ-ಸ-ಬೇ-ಕಾ-ದುದು
ಭಾವಿ-ಸ-ಬೇಡ
ಭಾವಿಸಿ
ಭಾವಿ-ಸಿ-ಕೊಂಡು
ಭಾವಿ-ಸಿತು
ಭಾವಿ-ಸಿದ
ಭಾವಿ-ಸಿ-ದನು
ಭಾವಿ-ಸಿ-ದರು
ಭಾವಿ-ಸಿ-ದುದೇ
ಭಾವಿ-ಸಿದ್ದ
ಭಾವಿ-ಸಿ-ದ್ದರು
ಭಾವಿ-ಸಿ-ರುವ
ಭಾವಿ-ಸಿ-ರು-ವ-ರಾ-ದರೂ
ಭಾವಿಸು
ಭಾವಿ-ಸು-ತ್ತಾನೆ
ಭಾವಿ-ಸು-ತ್ತಿದ್ದ
ಭಾವಿ-ಸು-ತ್ತೇನೆ
ಭಾವಿ-ಸುವ
ಭಾವಿ-ಸು-ವಂತೆ
ಭಾವಿ-ಸು-ವಷ್ಟು
ಭಾವಿ-ಸು-ವುದು
ಭಾವಿ-ಸು-ವು-ದೆ-ಲ್ಲವೂ
ಭಾಷಸೇ
ಭಾಷಾಂ-ತ-ರಿ-ಸುವ
ಭಾಷೆ
ಭಾಷೆ-ಯನ್ನು
ಭಾಷ್ಯ-ದಲ್ಲಿ
ಭಾಸ-ವಾ-ಯಿತು
ಭಿಕಾ-ರಿ-ಗ-ಳ-ನ್ನಾಗಿ
ಭಿಕಾ-ರಿ-ಗಳೆ
ಭಿಕ್ಷಕ್ಕೆ
ಭಿಕ್ಷ-ವಿ-ಕ್ಕಿ-ದಳು
ಭಿಕ್ಷಾ-ಪಾತ್ರೆ
ಭಿಕ್ಷು-ಕ-ರಾಗಿ
ಭಿಕ್ಷು-ಚ-ರ್ಯಾಂ
ಭಿಕ್ಷೆ
ಭಿಕ್ಷೆ-ಗಾಗಿ
ಭಿಕ್ಷೆ-ಯನ್ನು
ಭಿಕ್ಷೆ-ಯಿಂದ
ಭಿನ್ನ
ಭಿನ್ನ-ವಾದ
ಭಿನ್ನ-ವಾ-ದು-ದ-ರಿಂದ
ಭಿನ್ನ-ವಾ-ದುದು
ಭಿನ್ನಾ-ಭಿ-ಪ್ರಾ-ಯ-ಗ-ಳಿವೆ
ಭೀಮ
ಭೀಮನ
ಭೀಮ-ನನ್ನು
ಭೀಮ-ನಲ್ಲಿ
ಭೀಮ-ನಿಂದ
ಭೀಮ-ನಿಗೆ
ಭೀಮನು
ಭೀಮ-ರಿಗೆ
ಭೀಮ-ಸೇ-ನನು
ಭೀಮ-ಸ್ವ-ನೋ-ಽರೇ-ರ್ಹೃ-ದ-ಯಾನಿ
ಭೀಮಾ-ರ್ಜು-ನ-ರೊ-ಡನೆ
ಭೀಷ್ಮ
ಭೀಷ್ಮ-ಕನ
ಭೀಷ್ಮ-ಕ-ನಿಗೂ
ಭೀಷ್ಮ-ಕ-ನಿಗೆ
ಭೀಷ್ಮ-ಕನು
ಭೀಷ್ಮ-ಕ-ರಾ-ಜನು
ಭೀಷ್ಮ-ನನ್ನು
ಭೀಷ್ಮ-ನನ್ನೂ
ಭೀಷ್ಮನು
ಭುಜ
ಭುಜಕ್ಕೆ
ಭುಜ-ಗಳಲ್ಲಿ
ಭುಜ-ಗಳು
ಭುಜದ
ಭುಜ-ದಂ-ಡ-ದಲ್ಲಿ
ಭುಜ-ದ-ಮೇ-ಲಿ-ಟ್ಟು-ಕೊಂಡು
ಭುಜ-ದ-ಮೇಲೆ
ಭುಜ-ಮ-ಗ-ರು-ಸು-ಗಂ-ಧಂ-ಮೂ-ರ್ಧ್ನ್ಯ
ಭುಜ-ವನ್ನು
ಭುವ
ಭುವಃಹೇ
ಭುವಿ
ಭೂ
ಭೂತ
ಭೂತ-ಇ-ತ್ಯಾದಿ
ಭೂತ-ಗಳ
ಭೂತ-ಗಳನ್ನು
ಭೂತ-ಗಳನ್ನೂ
ಭೂತ-ಗ-ಳ-ಲ್ಲಿಯೂ
ಭೂತ-ಗಳಿಂದ
ಭೂತ-ಗ-ಳಿಗೂ
ಭೂತ-ಗಳೂ
ಭೂತ-ಗಳೆ
ಭೂತ-ಗ್ರ-ಹಾಂ-ಶ್ಚೂ-ರ್ಣಯ
ಭೂತ-ದಿಂದ
ಭೂತ-ನಿ-ರ್ವೃತ್ಯೈ
ಭೂತ-ವನ್ನು
ಭೂತ-ಸೃ-ಷ್ಟಿಯೇ
ಭೂತ-ಹಿ-ಡಿ-ದ-ವ-ನಂತೆ
ಭೂತಾನಿ
ಭೂತಿ-ರ್ವಯಂ
ಭೂತೇಂ-ದ್ರಿಯ
ಭೂತೇ-ಭ್ಯೋಂ-ಹೇಭ್ಯ
ಭೂತೋ
ಭೂದಾನ
ಭೂದೇವಿ
ಭೂದೇ-ವಿ-ಧ-ರ್ಮ-ಪು-ರು-ಷರ
ಭೂದೇ-ವಿಗೂ
ಭೂದೇ-ವಿಗೆ
ಭೂದೇ-ವಿಯ
ಭೂದೇ-ವಿ-ಯನ್ನು
ಭೂದೇ-ವಿಯು
ಭೂದೇ-ವಿಯೊ
ಭೂಭಾಗ
ಭೂಭಾ-ಗ-ವನ್ನೆ
ಭೂಭಾರ
ಭೂಭಾ-ರ-ವನ್ನು
ಭೂಭಾ-ರ-ವಿ-ಳಿ-ಯಿತು
ಭೂಭಾ-ರ-ವಿ-ಳುಹಿ
ಭೂಭಾ-ರ-ಹ-ರಣ
ಭೂಭಾ-ರ-ಹ-ರ-ಣ-ಕಾರ್ಯ
ಭೂಭಾ-ರ-ಹ-ರ-ಣ-ಕಾ-ರ್ಯಕ್ಕೆ
ಭೂಭಾ-ರ-ಹ-ರ-ಣ-ಕ್ಕಾಗಿ
ಭೂಮಂ-ಡಲ
ಭೂಮಂ-ಡ-ಲಕ್ಕೆ
ಭೂಮಂ-ಡ-ಲ-ಕ್ಕೆಲ್ಲ
ಭೂಮಂ-ಡ-ಲದ
ಭೂಮಂ-ಡ-ಲ-ದಲ್ಲೆಲ್ಲ
ಭೂಮಂ-ಡ-ಲ-ವನ್ನು
ಭೂಮಂ-ಡ-ಲ-ವನ್ನೂ
ಭೂಮಂ-ಡ-ಲ-ವ-ನ್ನೆಲ್ಲ
ಭೂಮಂ-ಡ-ಲವು
ಭೂಮಂ-ಡ-ಲವೂ
ಭೂಮಂ-ಡ-ಲವೇ
ಭೂಮಾ-ನು-ಭವ
ಭೂಮಿ
ಭೂಮಿ-ಗ-ಳೆ-ರ-ಡನ್ನು
ಭೂಮಿ-ಗಿಂತ
ಭೂಮಿ-ಗಿ-ಳಿದು
ಭೂಮಿ-ಗಿಳಿ-ದು-ಬಂ-ದ-ಮೇಲೆ
ಭೂಮಿಗೂ
ಭೂಮಿಗೆ
ಭೂಮಿ-ಗೆಲ್ಲ
ಭೂಮಿ-ತಾ-ಯಿ-ಯಿಂದ
ಭೂಮಿಯ
ಭೂಮಿ-ಯಂತೆ
ಭೂಮಿ-ಯತ್ತ
ಭೂಮಿ-ಯ-ನ್ನಿ-ಟ್ಟು-ಕೊಂಡು
ಭೂಮಿ-ಯನ್ನು
ಭೂಮಿ-ಯಲ್ಲಿ
ಭೂಮಿ-ಯ-ಲ್ಲಿ-ರುವ
ಭೂಮಿ-ಯ-ಲ್ಲೆಲ್ಲ
ಭೂಮಿ-ಯಾ-ಗು-ವುದನ್ನು
ಭೂಮಿ-ಯಿಂದ
ಭೂಮಿಯು
ಭೂಮಿಯೂ
ಭೂಮಿ-ಯೆನಿ
ಭೂಮಿಯೇ
ಭೂಮ್ನೇ
ಭೂಯಾ-ತ್ರೆ-ಯನ್ನು
ಭೂಯಾತ್ಹೇ
ಭೂಯಿಷ್ಠ
ಭೂರಿದಾ
ಭೂರ್ಭು-ವ-ಸ್ಸುವಃ
ಭೂಲೋಕ
ಭೂಲೋ-ಕಕ್ಕೆ
ಭೂಲೋ-ಕದ
ಭೂಲೋ-ಕ-ದಲ್ಲಿ
ಭೂಲೋ-ಕ-ವನ್ನು
ಭೂಷಣ
ಭೂಷ-ಣಾ-ಯು-ಧ-ಲಿಂ-ಗಾಖ್ಯಾ
ಭೂಷಿ-ತ-ನಾಗಿ
ಭೂಷಿ-ತ-ವಾದ
ಭೂಸಂ-ಚಾರ
ಭೂಸಂ-ಚಾ-ರಕ್ಕೆ
ಭೂಸಂ-ಚಾ-ರ-ವನ್ನು
ಭೂಸಂ-ಸ್ಥಾನಂ
ಭೂಸ್ಪ-ರ್ಶ-ವಾ-ಗು-ತ್ತಲೆ
ಭೃಂಗ-ವ-ದ್ವೇ-ಸಾ-ರಮ್
ಭೃಗು
ಭೃಗು-ಋಷಿ
ಭೃಗು-ಋ-ಷಿಯು
ಭೃಗು-ಮು-ನಿಗೆ
ಭೃಗು-ಮು-ನಿ-ಯನ್ನು
ಭೃಗು-ಮು-ನಿಯು
ಭೃಗುವು
ಭೃಗುವೂ
ಭೇತಾ-ಳ-ದಂತೆ
ಭೇದ
ಭೇದ-ದಿಂದ
ಭೇದ-ದೃ-ಷ್ಟಿ-ಯಿ-ಲ್ಲ-ದ-ವ-ನಾ-ಗಿಯೂ
ಭೇದ-ಬುದ್ಧಿ
ಭೇದ-ಭಾ-ವ-ನೆಯೂ
ಭೇದ-ವ-ನ್ನೆ-ಣಿ-ಸದ
ಭೇದ-ವ-ನ್ನೆ-ಣಿ-ಸದೆ
ಭೇದ-ವಿಲ್ಲ
ಭೇದ-ವಿ-ಲ್ಲ-ದಿ-ದ್ದರೂ
ಭೇದ-ವಿ-ಲ್ಲದೆ
ಭೇದವೂ
ಭೇದ-ವೆ-ಣಿ-ಸ-ಬೇಡ
ಭೇದ-ವೆ-ಣಿ-ಸಿ-ದರೆ
ಭೇದಿ-ಸ-ಹೊ-ರ-ಟಿತು
ಭೇದಿಸಿ
ಭೇದಿ-ಸಿ-ಕೊಂಡು
ಭೇದಿ-ಸಿ-ದರೆ
ಭೇದಿ-ಸುವ
ಭೇದಿ-ಸು-ವಂ-ತಿ-ರುವ
ಭೇದಿ-ಸು-ವು-ದಕ್ಕೆ
ಭೇರಿ
ಭೇರಿ-ಗಳ
ಭೇಷ್
ಭೈರ-ವ-ನಂತೆ
ಭೊ
ಭೋ
ಭೋಕ್ತೃ
ಭೋಗ
ಭೋಗ-ಕ್ಕಾಗಿ
ಭೋಗ-ಗಳನ್ನು
ಭೋಗ-ಗಳಿಂದ
ಭೋಗ-ಗಳು
ಭೋಗದ
ಭೋಗ-ಭಾಗ್ಯ
ಭೋಗ-ಭಾ-ಗ್ಯ-ಗಳ
ಭೋಗ-ಭಾ-ಗ್ಯ-ಗಳನ್ನು
ಭೋಗ-ಭಾ-ಗ್ಯ-ಗಳನ್ನೂ
ಭೋಗ-ಭಾ-ಗ್ಯ-ಗಳಿಂದ
ಭೋಗ-ಭಾ-ಗ್ಯ-ಗ-ಳಿ-ಗಾಗಿ
ಭೋಗ-ಭಾ-ಗ್ಯ-ಗಳು
ಭೋಗ-ಮಂ-ದಿ-ರ-ವನ್ನೂ
ಭೋಗ-ವ-ತಿ-ಯಂತೆ
ಭೋಗ-ವಸ್ತು
ಭೋಗ-ವ-ಸ್ತು-ಗಳನ್ನು
ಭೋಗ-ಸಾ-ಮ-ಗ್ರಿ-ಗಳನ್ನೂ
ಭೋಗ-ಸಾ-ಮ-ಗ್ರಿ-ಗಳಿಂದ
ಭೋಗ-ಸು-ಖ-ವನ್ನು
ಭೋಗಿ-ಗ-ಳೆಂದೂ
ಭೋಗಿಸಿ
ಭೋಗಿ-ಸು-ವು-ದ-ಕ್ಕಾಗಿ
ಭೋಗೇಚ್ಛೆ
ಭೋಗ್ಯ-ವ-ಸ್ತು-ಗಳನ್ನು
ಭೋಗ್ಯ-ವ-ಸ್ತು-ವೆಂದು
ಭೋಜ-ಕ-ಟ-ವೆಂಬ
ಭೋಜ-ನ-ಪಾತ್ರೆ
ಭೋಜ-ನ-ಮಾ-ಡಿ-ಸಿ-ದನು
ಭೋಜ-ನ-ವನ್ನು
ಭೋಜ-ನಾ-ನಂ-ತರ
ಭೋರಿ-ಡು-ತ್ತಿ-ರು-ವೆ-ಯಲ್ಲಾ
ಭೋರ್ಗ-ರೆ-ದವು
ಭೋರ್ಗ-ರೆ-ಯು-ತ್ತಿ-ದ್ದವು
ಭೋರ್ಗ-ರೆ-ಯು-ತ್ತಿ-ರಲು
ಭೋಸ್ಸದಾ
ಭೌಗೋ-ಳಿಕ
ಭೌತಿಕ
ಭೌಮಂ
ಭ್ಯೋ
ಭ್ರಮ-ಣೆ-ಯೇನಾ
ಭ್ರಮ-ತ್ಸ-ಮಂ-ತಾ-ದ್ಭ-ಗ-ವ-ತ್ಪ್ರ-ಯುಕ್ತಂ
ಭ್ರಮರ
ಭ್ರಮ-ರ-ಗೀತೆ
ಭ್ರಮ-ರ-ದಂತೆ
ಭ್ರಮ-ರ-ದುಂ-ಬಿ-ವೊಂ-ದನ್ನು
ಭ್ರಮ-ರನು
ಭ್ರಮ-ರನೆ
ಭ್ರಮ-ರ-ಪ-ತಿ-ಯಾದ
ಭ್ರಮ-ರವು
ಭ್ರಮ-ರವೆ
ಭ್ರಮ-ರು-ಕಾ-ರ-ದೊ-ಡನೆ
ಭ್ರಮಿ
ಭ್ರಮಿಯ
ಭ್ರಮಿ-ಯಲ್ಲಿ
ಭ್ರಮಿಸಿ
ಭ್ರಮಿ-ಸಿ-ದರು
ಭ್ರಮಿ-ಸಿ-ಯಾರು
ಭ್ರಮಿ-ಸು-ತ್ತದೆ
ಭ್ರಮಿ-ಸು-ತ್ತಾನೆ
ಭ್ರಮಿ-ಸು-ತ್ತಿ-ರುವೆ
ಭ್ರಮಿ-ಸು-ತ್ತೇವೆ
ಭ್ರಮಿ-ಸುವ
ಭ್ರಮಿ-ಸು-ವಂತೆ
ಭ್ರಮಿ-ಸು-ವನು
ಭ್ರಮಿ-ಸು-ವರು
ಭ್ರಮೆ
ಭ್ರಷ್ಟ-ನನ್ನು
ಭ್ರಾಂತ-ನಾಗಿ
ಭ್ರಾಂತ-ರಾ-ಗಿ-ಹೋ-ದರು
ಭ್ರಾಂತಿ
ಭ್ರಾಂತಿ-ಗೊಂ-ಡಿ-ರು-ವು-ದಕ್ಕೆ
ಭ್ರಾಂತಿ-ಪ-ಟ್ಟು-ಕೊ-ಳ್ಳು-ವಂತೆ
ಭ್ರಾಂತಿ-ಯನ್ನು
ಭ್ರಾಂತಿ-ಯಿಂದ
ಭ್ರಾಂತಿ-ಯಿಂ-ದಲೆ
ಭ್ರಾಂತಿಯೂ
ಭ್ರೂಭಂ-ಗ-ದಿಂ-ದಲೂ
ಭ್ರೂಭಂಗಿ
ಭ್ರೂಮ-ಧ್ಯಾಯ
ಮ
ಮಂಕ-ನಂತೆ
ಮಂಕ-ರಂತೆ
ಮಂಕಾಗಿ
ಮಂಕಾ-ಗು-ತ್ತದೆ
ಮಂಕಾ-ಗು-ವುದೂ
ಮಂಕಾದ
ಮಂಕು
ಮಂಕು-ಗಳು
ಮಂಕು-ಗ-ಳೇನು
ಮಂಕು-ಬೂದಿ
ಮಂಕು-ಬೂ-ದಿ-ಯಾ-ಗಿ-ರ-ಬೇಕು
ಮಂಕೆ
ಮಂಗ
ಮಂಗ-ಮಾಯ
ಮಂಗ-ಮಾ-ಯ-ವಾ-ಗಿವೆ
ಮಂಗಳ
ಮಂಗ-ಳ-ಕರ
ಮಂಗ-ಳ-ಕ-ರ-ವಾದ
ಮಂಗ-ಳ-ಕ-ರ-ವಾ-ದುದು
ಮಂಗ-ಳ-ಗೀತ
ಮಂಗ-ಳ-ಗೀ-ತ-ಗಳನ್ನು
ಮಂಗ-ಳ-ಗೀ-ತೆ-ಗಳನ್ನು
ಮಂಗ-ಳ-ವಾ-ಗಲಿ
ಮಂಗ-ಳ-ವಾ-ಗು-ತ್ತದೆ
ಮಂಗ-ಳ-ವಾ-ಗು-ವಂ-ತಹ
ಮಂಗ-ಳ-ವಾ-ಗು-ವು-ದಿಲ್ಲ
ಮಂಗ-ಳ-ವಾದ್ಯ
ಮಂಗ-ಳ-ವಾ-ದ್ಯ-ಗಳು
ಮಂಗ-ಳ-ವಾ-ದ್ಯ-ಗ-ಳೊ-ಡನೆ
ಮಂಗ-ಳ-ಸ್ನಾನ
ಮಂಗ-ಳ-ಸ್ನಾ-ನ-ವನ್ನು
ಮಂಗ-ಳಾಂಗಿ
ಮಂಚಕ್ಕೆ
ಮಂಚ-ಗಳ
ಮಂಚ-ಗಳು
ಮಂಚ-ಗ-ಳೆಲ್ಲ
ಮಂಚದ
ಮಂಚ-ದ-ಮೇಲೆ
ಮಂಚ-ದಿಂದ
ಮಂಚ-ಸ-ಹಿ-ತ-ವಾಗಿ
ಮಂಜಿನ
ಮಂಜಿ-ನಂತೆ
ಮಂಜು-ಳ-ಗಾನ
ಮಂಟ-ಪಕ್ಕೆ
ಮಂಟ-ಪ-ದಲ್ಲಿ
ಮಂಡ-ಲ-ಗ-ಳುಳ್ಳ
ಮಂಡ-ಲ-ದಲ್ಲಿ
ಮಂಡ-ಲ-ದ-ಲ್ಲಿನ
ಮಂಡ-ಲ-ದಿಂದ
ಮಂಡ-ಲ-ವಿದೆ
ಮಂಡಿ-ಯೂರಿ
ಮಂಡಿಸಿ
ಮಂಡಿ-ಸಿದ
ಮಂಡಿ-ಸಿ-ದನು
ಮಂಡಿ-ಸಿ-ದ್ದನು
ಮಂಡಿ-ಸಿ-ರುವ
ಮಂಡಿ-ಸುವ
ಮಂಡೆ-ಯೊ-ಡನೆ
ಮಂತ್ರ
ಮಂತ್ರ-ಗಳ
ಮಂತ್ರ-ಗಳನ್ನು
ಮಂತ್ರ-ಗಳಿಂದ
ಮಂತ್ರ-ಗಳೆ
ಮಂತ್ರ-ಗಳೇ
ಮಂತ್ರ-ಜಪ
ಮಂತ್ರ-ಜ-ಲ-ವನ್ನು
ಮಂತ್ರಜ್ಞ
ಮಂತ್ರ-ತ-ತ್ವ-ಲಿಂ-ಗಾಯ
ಮಂತ್ರದ
ಮಂತ್ರ-ದಿಂದ
ಮಂತ್ರ-ಪೂ-ರ್ವ-ಕ-ವಾಗಿ
ಮಂತ್ರ-ಮು-ದಾ-ಹ-ರೇತ್
ಮಂತ್ರ-ವನ್ನು
ಮಂತ್ರ-ವಿ-ದ-ರಾದ
ಮಂತ್ರ-ವಿ-ದ್ಯೆಯ
ಮಂತ್ರ-ವೆಂದೇ
ಮಂತ್ರ-ಶ-ಕ್ತಿ-ಯನ್ನು
ಮಂತ್ರಾ-ಕ್ಷತೆ
ಮಂತ್ರಾ-ಲೋ-ಚನೆ
ಮಂತ್ರಾ-ಲೋ-ಚ-ನೆಗೆ
ಮಂತ್ರಿ
ಮಂತ್ರಿ-ಗಳ
ಮಂತ್ರಿ-ಗ-ಳತ್ತ
ಮಂತ್ರಿ-ಗಳನ್ನು
ಮಂತ್ರಿ-ಗ-ಳಾದ
ಮಂತ್ರಿ-ಗಳೂ
ಮಂತ್ರಿ-ಗ-ಳೆಂ-ದರೆ
ಮಂತ್ರಿ-ಗ-ಳೊ-ಡನೆ
ಮಂತ್ರಿ-ಯಾ-ಗಿ-ದ್ದ-ವನು
ಮಂತ್ರಿ-ಯಾದ
ಮಂತ್ರಿಯು
ಮಂತ್ರಿಯೇ
ಮಂತ್ರೋ-ಪ-ನಿ-ಷತ್ತು
ಮಂತ್ರೋ-ಪ-ನಿ-ಷ-ತ್ತೆಂಬ
ಮಂದ
ಮಂದ-ಗ-ತಿ-ಯಿಂದ
ಮಂದ-ಗ-ಮನ
ಮಂದ-ಗ-ಮ-ನ-ದಿಂದ
ಮಂದ-ಮಂ-ದ-ವಾಗಿ
ಮಂದ-ಮಾ-ರುತ
ಮಂದರ
ಮಂದ-ರ-ಪ-ರ್ವ-ತದ
ಮಂದ-ರ-ಪ-ರ್ವ-ತ-ವನ್ನು
ಮಂದ-ವಾಗಿ
ಮಂದ-ಹಾಸ
ಮಂದ-ಹಾ-ಸ-ದಿಂದ
ಮಂದ-ಹಾ-ಸ-ದೊ-ಡನೆ
ಮಂದ-ಹಾ-ಸ-ವನ್ನು
ಮಂದ-ಹಾ-ಸ-ವನ್ನೂ
ಮಂದಾ-ನಿ-ಲವು
ಮಂದಾರ
ಮಂದಿ
ಮಂದಿ-ಯನ್ನು
ಮಂದಿ-ಯನ್ನೂ
ಮಂದಿ-ರ-ದಲ್ಲಿ
ಮಂದೆ
ಮಂದೆ-ಯನ್ನು
ಮಂದೆ-ಯಲ್ಲಿ
ಮಂದೆ-ಯ-ಲ್ಲಿದ್ದ
ಮಂದೆ-ಯಿಂದ
ಮಃ
ಮಕ-ರಂ-ದ-ವನ್ನು
ಮಕು-ಟ-ಗಳು
ಮಕ್ಕ
ಮಕ್ಕಳ
ಮಕ್ಕಳಂ
ಮಕ್ಕ-ಳಂತೆ
ಮಕ್ಕ-ಳನ್ನು
ಮಕ್ಕ-ಳನ್ನೂ
ಮಕ್ಕ-ಳ-ನ್ನೆಲ್ಲ
ಮಕ್ಕ-ಳ-ನ್ನೆಲ್ಲಾ
ಮಕ್ಕ-ಳಲ್ಲಿ
ಮಕ್ಕ-ಳಾಗ
ಮಕ್ಕ-ಳಾ-ಗ-ಬೇ-ಕೆಂಬ
ಮಕ್ಕ-ಳಾ-ಗ-ಲಿಲ್ಲ
ಮಕ್ಕ-ಳಾಗಿ
ಮಕ್ಕ-ಳಾ-ಗಿದ್ದ
ಮಕ್ಕ-ಳಾ-ಗಿ-ದ್ದ-ರಾ-ದರೂ
ಮಕ್ಕ-ಳಾ-ಗಿ-ದ್ದು-ದನ್ನು
ಮಕ್ಕ-ಳಾ-ಗಿ-ರ-ಲಿಲ್ಲ
ಮಕ್ಕ-ಳಾ-ಗುವ
ಮಕ್ಕ-ಳಾ-ಗು-ವ-ವ-ರೆಗೆ
ಮಕ್ಕ-ಳಾದ
ಮಕ್ಕ-ಳಾ-ದರು
ಮಕ್ಕ-ಳಾ-ದರೆ
ಮಕ್ಕ-ಳಾ-ದು-ದನ್ನು
ಮಕ್ಕ-ಳಾ-ದುವು
ಮಕ್ಕ-ಳಿಂದ
ಮಕ್ಕ-ಳಿ-ಗಾಗಿ
ಮಕ್ಕ-ಳಿಗೂ
ಮಕ್ಕ-ಳಿಗೆ
ಮಕ್ಕ-ಳಿ-ಗೆಲ್ಲ
ಮಕ್ಕ-ಳಿ-ಗೊ-ಪ್ಪಿಸಿ
ಮಕ್ಕ-ಳಿ-ದ್ದರು
ಮಕ್ಕ-ಳಿ-ದ್ದರೆ
ಮಕ್ಕ-ಳಿದ್ದು
ಮಕ್ಕ-ಳಿ-ಬ್ಬ-ರನ್ನೂ
ಮಕ್ಕ-ಳಿ-ಬ್ಬ-ರಲ್ಲಿ
ಮಕ್ಕ-ಳಿ-ಬ್ಬರು
ಮಕ್ಕ-ಳಿ-ಬ್ಬರೂ
ಮಕ್ಕ-ಳಿ-ಬ್ಬರೆ
ಮಕ್ಕ-ಳಿ-ಲ್ಲದ
ಮಕ್ಕ-ಳಿ-ಲ್ಲ-ದವ
ಮಕ್ಕ-ಳಿ-ಲ್ಲ-ದ-ವ-ರಿಂದ
ಮಕ್ಕ-ಳಿ-ಲ್ಲ-ದಿ-ರು-ವುದೇ
ಮಕ್ಕ-ಳಿ-ಲ್ಲದೆ
ಮಕ್ಕಳು
ಮಕ್ಕ-ಳು-ಪ್ರ-ಹ್ಲಾದ
ಮಕ್ಕಳೂ
ಮಕ್ಕಳೆ
ಮಕ್ಕ-ಳೆಂದು
ಮಕ್ಕ-ಳೆಂಬ
ಮಕ್ಕ-ಳೆಲ್ಲ
ಮಕ್ಕ-ಳೆ-ಲ್ಲರೂ
ಮಕ್ಕ-ಳೆಲ್ಲಿ
ಮಕ್ಕ-ಳೊ-ಡನೆ
ಮಕ್ಕಳೋ
ಮಕ್ತ-ಕಂ-ಠ-ದಿಂದ
ಮಗ
ಮಗ-ಘೃತ
ಮಗ-ದೊ-ಬ್ಬಳು
ಮಗ-ದೊಮ್ಮೆ
ಮಗಧ
ಮಗ-ಧ-ಸೇನೆ
ಮಗನ
ಮಗ-ನಂತೆ
ಮಗ-ನತ್ತ
ಮಗ-ನನ್ನು
ಮಗ-ನ-ನ್ನು-ಅಪ
ಮಗ-ನನ್ನೂ
ಮಗ-ನಲ್ಲ
ಮಗ-ನಲ್ಲಿ
ಮಗ-ನಾ-ಗ-ಬೇ-ಕೆಂಬ
ಮಗ-ನಾಗಿ
ಮಗ-ನಾ-ಗಿ-ದ್ದರೂ
ಮಗ-ನಾ-ಗಿ-ದ್ದ-ವ-ನಾ-ದ್ದ-ರಿಂದ
ಮಗ-ನಾ-ಗು-ವಂತೆ
ಮಗ-ನಾದ
ಮಗ-ನಾ-ದರೂ
ಮಗ-ನಾ-ದರೊ
ಮಗ-ನಿಂದ
ಮಗ-ನಿಂ-ದಲೂ
ಮಗ-ನಿ-ಗಾಗಿ
ಮಗ-ನಿಗೂ
ಮಗ-ನಿಗೆ
ಮಗ-ನಿದ್ದ
ಮಗನು
ಮಗನೆ
ಮಗ-ನೆಂದು
ಮಗ-ನೆಂಬ
ಮಗನೇ
ಮಗ-ನೇನೋ
ಮಗನೊ
ಮಗ-ನೊ-ಡನೆ
ಮಗ-ನೊ-ಬ್ಬ-ನನ್ನು
ಮಗ-ನೊ-ಬ್ಬನು
ಮಗ-ರಾ-ಯ-ನದೇ
ಮಗಳ
ಮಗ-ಳಂತೆ
ಮಗಳನ್ನು
ಮಗಳನ್ನೂ
ಮಗ-ಳನ್ನೆ
ಮಗಳಲ್ಲಿ
ಮಗ-ಳಾ-ಗ-ಬೇ-ಕೆಂಬ
ಮಗ-ಳಾಗಿ
ಮಗ-ಳಾ-ಗಿ-ರುವ
ಮಗ-ಳಾದ
ಮಗ-ಳಿಂ-ದಲೂ
ಮಗ-ಳಿ-ಗಿಂ-ತಲೂ
ಮಗ-ಳಿಗೆ
ಮಗ-ಳಿ-ಗೆಲ್ಲಿ
ಮಗ-ಳಿ-ದ್ದಳು
ಮಗ-ಳಿ-ದ್ದಾಳೆ
ಮಗಳು
ಮಗಳೆ
ಮಗ-ಳೆಂ-ದರೆ
ಮಗ-ಳೆಂ-ಬುದು
ಮಗ-ಳೊ-ಡನೆ
ಮಗ-ಳೊ-ಬ್ಬ-ಳಿ-ದ್ದಳು
ಮಗ-ವನ್ನು
ಮಗು
ಮಗು-ಎ-ಲ್ಲವೂ
ಮಗು-ಚಿ-ಕೊ-ಳ್ಳ-ಬೇಕೆ
ಮಗು-ಟ-ಗಳು
ಮಗು-ದೊಂ-ದ-ರಲ್ಲಿ
ಮಗು-ದೊಬ್ಬ
ಮಗು-ದೊ-ಬ್ಬಳು
ಮಗು-ಳ್ನ-ಗೆ-ಯನ್ನು
ಮಗು-ವ-ನ್ನಾ-ದರೂ
ಮಗು-ವ-ನ್ನಿಡು
ಮಗು-ವನ್ನು
ಮಗು-ವನ್ನೂ
ಮಗು-ವ-ನ್ನೆ-ತ್ತಿ-ಕೊಂಡು
ಮಗು-ವ-ಮ್ಮ-ಎಂ-ದರು
ಮಗು-ವಲ್ಲ
ಮಗು-ವಾ-ಗ-ಲೆಂದು
ಮಗು-ವಾಗಿ
ಮಗು-ವಾ-ಗಿದ್ದ
ಮಗು-ವಾ-ಗಿ-ದ್ದಾಗ
ಮಗು-ವಾದ
ಮಗು-ವಾ-ದು-ದನ್ನು
ಮಗು-ವಾ-ದು-ದ-ರಿಂದ
ಮಗುವಿ
ಮಗು-ವಿಗೂ
ಮಗು-ವಿಗೆ
ಮಗು-ವಿತ್ತು
ಮಗು-ವಿನ
ಮಗು-ವಿ-ನಂ-ತಿ-ದ್ದಾನೆ
ಮಗು-ವಿ-ನಂತೆ
ಮಗು-ವಿ-ನಲ್ಲಿ
ಮಗು-ವಿ-ನಿಂದ
ಮಗು-ವಿ-ನಿಂ-ದಲೆ
ಮಗು-ವಿ-ನೊ-ಡನೆ
ಮಗು-ವಿ-ರ-ಲಿಲ್ಲ
ಮಗುವು
ಮಗುವೂ
ಮಗುವೆ
ಮಗು-ವೆಂದು
ಮಗುವೇ
ಮಗು-ವೇ-ನಾ-ಯಿತೊ
ಮಗೂ
ಮಗ್ಗುಲು
ಮಗ್ನ-ನಾಗಿ
ಮಗ್ನ-ನಾ-ಗಿದ್ದ
ಮಗ್ನ-ನಾ-ಗಿ-ದ್ದನು
ಮಗ್ನ-ನಾ-ಗಿ-ದ್ದಾನೆ
ಮಗ್ನ-ನಾ-ಗಿ-ರಲು
ಮಗ್ನ-ನಾ-ಗಿ-ರುವ
ಮಗ್ನ-ನಾ-ಗು-ವನು
ಮಗ್ನ-ನಾದ
ಮಗ್ನ-ನಾ-ದನು
ಮಗ್ನ-ರಾ-ಗಿದ್ದ
ಮಗ್ನ-ರಾ-ಗಿ-ದ್ದರು
ಮಗ್ನ-ರಾ-ಗಿ-ದ್ದಾರೆ
ಮಗ್ನ-ರಾ-ಗಿ-ರು-ವಾಗ
ಮಗ್ನ-ರಾ-ಗು-ತ್ತಲೆ
ಮಗ್ನ-ರಾ-ಗು-ವರು
ಮಗ್ನ-ರಾ-ದಿರಿ
ಮಗ್ನ-ಳಾ-ಗಿ-ದ್ದಳು
ಮಗ್ನ-ಳಾ-ಗಿ-ದ್ದ-ವಳು
ಮಗ್ನ-ಳಾ-ಗಿ-ದ್ದಾಳೆ
ಮಗ್ನ-ಳಾ-ಗಿದ್ದೆ
ಮಗ್ನ-ಳಾ-ದಳು
ಮಚ್ಚೆ
ಮಚ್ಚೆ-ಯನ್ನು
ಮಚ್ಚೆ-ಯಿಲ್ಲ
ಮಟ್ಟ
ಮಟ್ಟಿಗೆ
ಮಠ
ಮಠ-ಗಳನ್ನು
ಮಠ-ಗಳು
ಮಠ-ಗ-ಳೊಂದೂ
ಮಡ-ಕೆ-ಯನ್ನು
ಮಡ-ಕೆ-ಯ-ಲ್ಲಿನ
ಮಡದಿ
ಮಡ-ದಿಗೂ
ಮಡ-ದಿಗೆ
ಮಡ-ದಿ-ಮ-ಕ್ಕಳೂ
ಮಡ-ದಿ-ಮ-ಕ್ಕ-ಳೊ-ಡನೆ
ಮಡ-ದಿಯ
ಮಡ-ದಿ-ಯನ್ನು
ಮಡ-ದಿ-ಯನ್ನೂ
ಮಡ-ದಿ-ಯನ್ನೇ
ಮಡ-ದಿ-ಯ-ರನ್ನೂ
ಮಡ-ದಿ-ಯ-ರಲ್ಲಿ
ಮಡ-ದಿ-ಯ-ರಾದ
ಮಡ-ದಿ-ಯ-ರಿಗೂ
ಮಡ-ದಿ-ಯರು
ಮಡ-ದಿ-ಯ-ರೊ-ಡನೆ
ಮಡ-ದಿ-ಯಲ್ಲಿ
ಮಡ-ದಿ-ಯಾದ
ಮಡ-ದಿ-ಯಾ-ದರೂ
ಮಡ-ದಿ-ಯಾ-ದಳು
ಮಡ-ದಿ-ಯಿಂದ
ಮಡ-ದಿ-ಯಿಂ-ದಲೂ
ಮಡ-ದಿಯು
ಮಡ-ದಿಯೂ
ಮಡ-ದಿ-ಯೊ-ಡನೆ
ಮಡ-ದಿ-ಯೊ-ಡ-ನೆಯೂ
ಮಡಿ
ಮಡಿ-ಗಳನ್ನು
ಮಡಿ-ದರು
ಮಡಿ-ದು-ದನ್ನು
ಮಡಿ-ಬ-ಟ್ಟೆ-ಗಳನ್ನು
ಮಡಿ-ಬ-ಟ್ಟೆ-ಗಳನ್ನೆಲ್ಲ
ಮಡಿ-ಮಾ-ಡಿ-ಕೊಂಡು
ಮಡಿ-ಯುಟ್ಟು
ಮಡಿ-ಲಲ್ಲಿ
ಮಡಿಸಿ
ಮಡು
ಮಡು-ಗಳನ್ನು
ಮಡು-ವ-ನ್ನಾ-ಗಲಿ
ಮಡು-ವನ್ನು
ಮಡು-ವಾ-ಗಿತ್ತು
ಮಡು-ವಾ-ಯಿತು
ಮಡು-ವಿಗೆ
ಮಡು-ವಿನ
ಮಡು-ವಿ-ನಂ-ತಾ-ಯಿತು
ಮಡು-ವಿ-ನಲ್ಲಿ
ಮಡು-ವಿ-ನ-ಲ್ಲಿದ್ದ
ಮಡು-ವಿ-ನೊ-ಳಕ್ಕೆ
ಮಡು-ವಿ-ನೊ-ಳಗೆ
ಮಡು-ವೆಂದು
ಮಡು-ಹಿದ
ಮಣಿ
ಮಣಿ-ಕ-ಟ್ಟಿ-ನಲ್ಲಿ
ಮಣಿ-ಗಾಗಿ
ಮಣಿ-ಗ್ರೀ-ವರು
ಮಣಿ-ಗ್ರೀ-ವ-ರೆಂದು
ಮಣಿ-ಪ-ರ್ವತ
ಮಣಿ-ಪೂ-ರಕ
ಮಣಿ-ಮಂ-ತ-ನೆಂಬ
ಮಣಿಯ
ಮಣಿ-ಯನ್ನು
ಮಣಿ-ಯನ್ನೂ
ಮಣಿ-ಯಿತು
ಮಣಿ-ಯು-ವಂತೆ
ಮಣಿ-ಯೊ-ಡನೆ
ಮಣೆ-ಗಳು
ಮಣೆಯ
ಮಣೆ-ಯನ್ನೊ
ಮಣೆ-ಯ-ಮೇಲೆ
ಮಣ್ಣನ್ನು
ಮಣ್ಣಿನ
ಮಣ್ಣು
ಮಣ್ಣು-ಗಳನ್ನು
ಮಣ್ಣು-ಗೂಡಿ
ಮಣ್ಣು-ಮು-ಕ್ಕಿ-ಹೋ-ದು-ದನ್ನು
ಮಣ್ಣೆ-ರ-ಚ-ಬ-ಹುದು
ಮಣ್ಣೆ-ರಚಿ
ಮಣ್ಣೆಲ್ಲ
ಮತ-ತನ್ನ
ಮತಾ-ವ-ಲಂ-ಬಿ-ಗ-ಳಿಗೆ
ಮತಿ-ಗೆಟ್ಟ
ಮತಿ-ಗೆ-ಡ-ಬಾ-ರದು
ಮತಿ-ರ್ಮ-ಧು-ಪ-ತೇ-ಽಸ-ಕೃತ್
ಮತ್ತ-ರ್ಧ-ಭಾ-ಗಕ್ಕೆ
ಮತ್ತಷ್ಟು
ಮತ್ತಾರ
ಮತ್ತಾ-ರ-ಲ್ಲಿಯೂ
ಮತ್ತಾ-ರಿಗೂ
ಮತ್ತಾರೂ
ಮತ್ತಾವ
ಮತ್ತಾ-ವ-ನಾ-ದಾನು
ಮತ್ತಾ-ವಳ
ಮತ್ತಾ-ವು-ದ-ಕ್ಕಾಗಿ
ಮತ್ತಾ-ವು-ದ-ರಿಂ-ದಲೂ
ಮತ್ತಾ-ವುದೊ
ಮತ್ತಿಯ
ಮತ್ತು
ಮತ್ತೂ
ಮತ್ತೆ
ಮತ್ತೆ-ಮತ್ತೆ
ಮತ್ತೆ-ರಡು
ಮತ್ತೆಲ್ಲಿ
ಮತ್ತೇ
ಮತ್ತೇ-ನನ್ನು
ಮತ್ತೇರಿ
ಮತ್ತೊಂ-ದಕ್ಕೆ
ಮತ್ತೊಂ-ದ-ರಿಂದ
ಮತ್ತೊಂ-ದಿಲ್ಲ
ಮತ್ತೊಂದು
ಮತ್ತೊಂ-ದೆಡೆ
ಮತ್ತೊಬ್ಬ
ಮತ್ತೊ-ಬ್ಬನ
ಮತ್ತೊ-ಬ್ಬ-ನಿಂದ
ಮತ್ತೊ-ಬ್ಬ-ನಿ-ಲ್ಲ-ದು-ದ-ರಿಂದ
ಮತ್ತೊ-ಬ್ಬ-ರಿಗೆ
ಮತ್ತೊ-ಬ್ಬ-ರಿಲ್ಲ
ಮತ್ತೊ-ಬ್ಬರು
ಮತ್ತೊ-ಬ್ಬ-ರೊ-ಡನೆ
ಮತ್ತೊ-ಬ್ಬಳ
ಮತ್ತೊ-ಬ್ಬ-ಳಿ-ಲ್ಲ-ವೆಂದೇ
ಮತ್ತೊ-ಬ್ಬಳು
ಮತ್ತೊಮ್ಮೆ
ಮತ್ಸ್ಯ
ಮತ್ಸ್ಯ-ಪು-ರಾಣ
ಮತ್ಸ್ಯ-ಮೂ-ರ್ತಿಃ
ಮತ್ಸ್ಯ-ಮೂ-ರ್ತಿಯು
ಮತ್ಸ್ಯ-ಯಂ-ತ್ರ-ವನ್ನು
ಮತ್ಸ್ಯ-ರೂಪ
ಮತ್ಸ್ಯ-ರೂ-ಪ-ನಾದ
ಮತ್ಸ್ಯ-ರೂ-ಪ-ವನ್ನು
ಮತ್ಸ್ಯಾ-ವ-ತಾರ
ಮಥ-ನ-ಕ-ಡೆ-ಯು-ವು-ದು-ದಿಂದ
ಮಥಿ-ತಾ-ರ್ಣ-ವ-ರಾ-ಜ-ರಸಂ
ಮದ
ಮದಕ್ಕೆ
ಮದ-ದಿಂದ
ಮದ-ಯಂತಿ
ಮದ-ಯಂ-ತಿ-ದೇ-ವಿಯ
ಮದ-ಯಂ-ತಿಯು
ಮದ-ಯಂ-ತಿ-ಯೊ-ಡನೆ
ಮದ-ವೇರಿ
ಮದ-ವೇ-ರಿದ
ಮದ-ವೊಂದೂ
ಮದಾಂಧ
ಮದಾಂ-ಧ-ನಾದ
ಮದಾಂ-ಧರ
ಮದಾಂ-ಧ-ರಾ-ಗಿ-ರು-ವರು
ಮದಾಂ-ಧ-ರಾದ
ಮದಾಂ-ಧ-ರಿರಾ
ಮದಿ-ರೆ-ನಂದ
ಮದಿ-ಸಿದ
ಮದು-ಮ-ಗ-ನಾ-ಗಲು
ಮದು-ಮ-ಗಳ
ಮದು-ಮ-ಗಳನ್ನು
ಮದು-ಮ-ಗ-ಳಾದ
ಮದು-ವ-ಣ-ಗಿ-ತ್ತಿ-ಯನ್ನು
ಮದು-ವ-ಣಿ-ಗನ
ಮದುವೆ
ಮದು-ವೆ-ಗೆಂದು
ಮದು-ವೆ-ಮಾ-ಡ-ಬೇ-ಕೆಂದು
ಮದು-ವೆ-ಮಾ-ಡಿ-ಕೊಂ-ಡನು
ಮದು-ವೆ-ಮಾ-ಡಿ-ಕೊ-ಟ್ಟನು
ಮದು-ವೆ-ಮಾ-ಡಿ-ಕೊಡು
ಮದು-ವೆ-ಮಾ-ಡಿ-ಕೊ-ಳ್ಳ-ಬೇ-ಕೆಂದು
ಮದು-ವೆಯ
ಮದು-ವೆ-ಯನ್ನು
ಮದು-ವೆ-ಯಾ-ಗ-ದಿ-ದ್ದರೂ
ಮದು-ವೆ-ಯಾ-ಗ-ಬೇ-ಕೆಂದು
ಮದು-ವೆ-ಯಾ-ಗಲು
ಮದು-ವೆ-ಯಾಗಿ
ಮದು-ವೆ-ಯಾ-ಗಿತ್ತು
ಮದು-ವೆ-ಯಾ-ಗಿದೆ
ಮದು-ವೆ-ಯಾ-ಗಿವೆ
ಮದು-ವೆ-ಯಾಗು
ಮದು-ವೆ-ಯಾ-ಗುವ
ಮದು-ವೆ-ಯಾ-ಗು-ವ-ನೆಂದೂ
ಮದು-ವೆ-ಯಾ-ಗು-ವನೊ
ಮದು-ವೆ-ಯಾ-ಗು-ವು-ದಾಗಿ
ಮದು-ವೆ-ಯಾದ
ಮದು-ವೆ-ಯಾ-ದನು
ಮದು-ವೆ-ಯಾ-ದ-ನೆಂಬ
ಮದು-ವೆ-ಯಾ-ದರು
ಮದು-ವೆ-ಯಾ-ದರೂ
ಮದು-ವೆ-ಯಾ-ದುದು
ಮದು-ವೆ-ಯಾ-ಯಿತೆ
ಮದೋ-ನ್ಮ-ತ್ತ-ನಾಗಿ
ಮದ್ದಲ್ಲ
ಮದ್ದಾ-ಗಿ-ದ್ದರೂ
ಮದ್ದಾನೆ
ಮದ್ದಾ-ನೆಯ
ಮದ್ದಾ-ನೆ-ಯಂತೆ
ಮದ್ದಾ-ನೆ-ಯನ್ನು
ಮದ್ದಾ-ನೆ-ಯೊಂದು
ಮದ್ದಿನ
ಮದ್ದು
ಮದ್ದೆಂದು
ಮದ್ಯ
ಮದ್ಯದ
ಮದ್ಯ-ಪಾನ
ಮದ್ಯ-ಪಾ-ನ-ದಿಂದ
ಮದ್ಯ-ವನ್ನು
ಮದ್ರ-ರಾ-ಜನ
ಮಧು-ಕರಂ
ಮಧುಪ
ಮಧು-ಪತಿ
ಮಧು-ಪ-ತಿ-ಸ್ತ-ನ್ಮಾ-ನ-ನೀ-ನಾಂ
ಮಧು-ಪು-ರ್ಯಾ-ಮಾ-ರ್ಯ-ಪು-ತ್ರೋ-ಽಧು-ನಾಸ್ತೇ
ಮಧುರ
ಮಧು-ರ-ಗಾ-ನವು
ಮಧು-ರ-ಭ-ಕ್ತಿ-ಯನ್ನು
ಮಧು-ರ-ವಾಗಿ
ಮಧು-ರ-ವಾ-ಗಿ-ರ-ಬೇಕು
ಮಧು-ರ-ವಾದ
ಮಧು-ರ-ಸ್ವ-ರ-ದಿಂದ
ಮಧುರಾ
ಮಧು-ರಾ-ನ-ಗ-ರಿಗೆ
ಮಧು-ರಾ-ನ-ಗ-ರಿಯ
ಮಧು-ರಾ-ನ-ಗ-ರಿ-ಯ-ಲ್ಲಿಯೆ
ಮಧು-ರಾ-ಪು-ರಕ್ಕೆ
ಮಧು-ರಾ-ಪು-ರದ
ಮಧು-ರಾ-ಪು-ರಿಗೆ
ಮಧು-ರಾ-ಪು-ರಿಯ
ಮಧು-ರಾ-ಪು-ರಿ-ಯನ್ನು
ಮಧು-ರಾ-ಪು-ರಿ-ಯಲ್ಲಿ
ಮಧು-ರಾ-ಪು-ರಿ-ಯ-ಲ್ಲಿ-ದ್ದರೂ
ಮಧುರೆ
ಮಧು-ರೆ-ಗಳ
ಮಧು-ರೆಗೆ
ಮಧು-ರೆಯ
ಮಧು-ರೆ-ಯನ್ನು
ಮಧು-ರೆ-ಯಲ್ಲಿ
ಮಧು-ರೆ-ಯ-ಲ್ಲಿದ್ದ
ಮಧು-ರೆ-ಯ-ಲ್ಲಿಯೂ
ಮಧು-ರೆ-ಯ-ಲ್ಲಿಯೆ
ಮಧು-ರೆ-ಯ-ಲ್ಲಿಯೇ
ಮಧು-ರೆ-ಯ-ಲ್ಲಿ-ರುವ
ಮಧು-ರೆಯೂ
ಮಧು-ವ-ನಕ್ಕೆ
ಮಧು-ವ-ನ-ದತ್ತ
ಮಧು-ವ-ನ-ದಲ್ಲಿ
ಮಧು-ವ-ನ-ವನ್ನು
ಮಧು-ವ-ನ-ವ-ವಿದೆ
ಮಧು-ವನ್ನು
ಮಧು-ಸೂ-ದ-ನನು
ಮಧು-ಹೋ-ಗ್ರ-ಧನ್ವಾ
ಮಧ್ಯ
ಮಧ್ಯ-ಕಾ-ಲ-ದಲ್ಲೆಲ್ಲೊ
ಮಧ್ಯದ
ಮಧ್ಯ-ದಲ್ಲಿ
ಮಧ್ಯ-ದ-ಲ್ಲಿ-ಆ-ತನು
ಮಧ್ಯ-ದ-ಲ್ಲಿದ್ದ
ಮಧ್ಯ-ದ-ಲ್ಲಿಯೂ
ಮಧ್ಯ-ದ-ಲ್ಲಿ-ರುವ
ಮಧ್ಯ-ದಿಂದ
ಮಧ್ಯ-ಭಾ-ಗ-ದಲ್ಲಿ
ಮಧ್ಯಮ
ಮಧ್ಯ-ಮಾ-ಭ್ಯಾಂ
ಮಧ್ಯ-ಮಾ-ರ್ಗ-ದ-ಲ್ಲಿಯೇ
ಮಧ್ಯ-ರಾ-ಜ್ಯವು
ಮಧ್ಯ-ರಾ-ತ್ರಿ-ಯಲ್ಲಿ
ಮಧ್ಯ-ರಾ-ತ್ರಿ-ಯ-ಲ್ಲಿಯೂ
ಮಧ್ಯ-ರಾ-ತ್ರಿ-ವ-ರೆಗೆ
ಮಧ್ಯಾಹ್ನ
ಮಧ್ಯಾ-ಹ್ನದ
ಮಧ್ಯಾ-ಹ್ನ-ದ-ಲ್ಲಿಯೂ
ಮಧ್ಯಾ-ಹ್ನ-ದ-ವ-ರೆಗೂ
ಮಧ್ಯೆ
ಮಧ್ಯೆ-ಮಧ್ಯೆ
ಮಧ್ವ
ಮಧ್ವಯಂ
ಮಧ್ವಾ-ಚಾ-ರ್ಯರು
ಮನ
ಮನದ
ಮನ-ದಂತೆ
ಮನ-ದ-ಟ್ಟಾ-ಗು-ತ್ತದೆ
ಮನ-ದ-ಟ್ಟಾ-ಯಿತು
ಮನ-ದಟ್ಟು
ಮನ-ದ-ಣಿ-ಯು-ವಂತೆ
ಮನ-ದನ್ನೆ
ಮನ-ದಲ್ಲಿ
ಮನ-ದ-ಲ್ಲಿಯೆ
ಮನ-ದಲ್ಲೆ
ಮನನ
ಮನ-ನ-ಮಾ-ಡಿ-ಕೊ-ಳ್ಳು-ವುದು
ಮನ-ನೊಂದು
ಮನ-ಬಂದ
ಮನ-ಬಂ-ದಂತೆ
ಮನ-ಬಂ-ದಷ್ಟು
ಮನ-ವನ್ನು
ಮನ-ವ-ರಿ-ಕೆ-ಯಾ-ಯಿತು
ಮನ-ಶ್ಶಕ್ತಿ
ಮನ-ಶ್ಶಾಂತಿ
ಮನ-ಶ್ಶುದ್ಧಿ-ಯನ್ನು
ಮನಸಾ
ಮನ-ಸಾರೆ
ಮನ-ಸೇದಂ
ಮನ-ಸೋತ
ಮನ-ಸೋ-ತವ
ಮನ-ಸೋತು
ಮನ-ಸ್ತೃ-ಪ್ತಿ-ಯಾ-ಗು-ವಷ್ಟು
ಮನ-ಸ್ಸ-ನ್ನಿ-ಟ್ಟಿರು
ಮನ-ಸ್ಸನ್ನು
ಮನ-ಸ್ಸನ್ನೂ
ಮನ-ಸ್ಸ-ನ್ನೆಲ್ಲ
ಮನ-ಸ್ಸಾ-ಗಲಿ
ಮನ-ಸ್ಸಾ-ಗ-ಲಿಲ್ಲ
ಮನಸ್ಸಿ
ಮನ-ಸ್ಸಿಗೂ
ಮನ-ಸ್ಸಿಗೆ
ಮನ-ಸ್ಸಿಟ್ಟು
ಮನ-ಸ್ಸಿನ
ಮನ-ಸ್ಸಿ-ನಂತೆ
ಮನ-ಸ್ಸಿ-ನಲ್ಲಿ
ಮನ-ಸ್ಸಿ-ನ-ಲ್ಲಿಯೂ
ಮನ-ಸ್ಸಿ-ನ-ಲ್ಲಿಯೆ
ಮನ-ಸ್ಸಿ-ನ-ಲ್ಲಿಯೇ
ಮನ-ಸ್ಸಿ-ನ-ಲ್ಲಿ-ರುವ
ಮನ-ಸ್ಸಿ-ನ-ಲ್ಲಿ-ರು-ವು-ದನ್ನೆ
ಮನ-ಸ್ಸಿ-ನಲ್ಲೂ
ಮನ-ಸ್ಸಿ-ನಲ್ಲೆ
ಮನ-ಸ್ಸಿ-ನಲ್ಲೇ
ಮನ-ಸ್ಸಿ-ನ-ವ-ನಾಗಿ
ಮನ-ಸ್ಸಿ-ನ-ವ-ನಾ-ಗಿ-ರ-ಬೇಕು
ಮನ-ಸ್ಸಿ-ನ-ವ-ನಿಗೆ
ಮನ-ಸ್ಸಿ-ನಿಂದ
ಮನ-ಸ್ಸಿ-ನಿಂ-ದಿದ್ದ
ಮನ-ಸ್ಸಿ-ಲ್ಲದೆ
ಮನಸ್ಸು
ಮನ-ಸ್ಸು-ಗಳ
ಮನ-ಸ್ಸು-ಗಳನ್ನು
ಮನ-ಸ್ಸು-ಗಳನ್ನೂ
ಮನ-ಸ್ಸು-ಗ-ಳಿಗೆ
ಮನ-ಸ್ಸು-ಗಳು
ಮನ-ಸ್ಸು-ಗ-ಳೆಲ್ಲ
ಮನ-ಸ್ಸುಳ್ಳ
ಮನ-ಸ್ಸು-ಳ್ಳ-ವ-ನಾಗಿ
ಮನಸ್ಸೂ
ಮನಸ್ಸೆ
ಮನ-ಸ್ಸೆಂ-ಬುದು
ಮನ-ಸ್ಸೆಲ್ಲ
ಮನಸ್ಸೇ
ಮನು
ಮನು-ಗಳ
ಮನು-ಗಳನ್ನು
ಮನು-ಗಳು
ಮನು-ಚ-ಕ್ರ-ವ-ರ್ತಿಯು
ಮನು-ಪು-ತ್ರ-ನಾದ
ಮನು-ಪು-ತ್ರರು
ಮನು-ವನ್ನು
ಮನು-ವಾ-ದನು
ಮನು-ವಾ-ದು-ದನ್ನು
ಮನು-ವಿಗೆ
ಮನು-ವಿನ
ಮನು-ವಿ-ನಿಂದ
ಮನುವು
ಮನು-ವೆ-ನಿ-ಸಿತು
ಮನುಷ್ಯ
ಮನು-ಷ್ಯ-ಜ-ನ್ಮವೇ
ಮನು-ಷ್ಯ-ದೇಹ
ಮನು-ಷ್ಯನ
ಮನು-ಷ್ಯ-ನಂತೆ
ಮನು-ಷ್ಯ-ನನ್ನು
ಮನು-ಷ್ಯ-ನಲ್ಲ
ಮನು-ಷ್ಯ-ನ-ಲ್ಲಪ್ಪ
ಮನು-ಷ್ಯ-ನಾಗಿ
ಮನು-ಷ್ಯ-ನಾ-ಗಿಯೇ
ಮನು-ಷ್ಯ-ನಿಗೂ
ಮನು-ಷ್ಯ-ನಿಗೆ
ಮನು-ಷ್ಯನು
ಮನು-ಷ್ಯನೂ
ಮನು-ಷ್ಯ-ನೆಂದು
ಮನು-ಷ್ಯ-ನೆಂದೇ
ಮನು-ಷ್ಯ-ಪ್ರ-ಯ-ತ್ನಕ್ಕೆ
ಮನು-ಷ್ಯ-ಪ್ರಾಣಿ
ಮನು-ಷ್ಯ-ಮಾ-ತ್ರ-ದವ
ಮನು-ಷ್ಯ-ರನ್ನು
ಮನು-ಷ್ಯ-ರಾ-ಗಿ-ರಲು
ಮನು-ಷ್ಯ-ರಾ-ದುವು
ಮನು-ಷ್ಯ-ರಿಂದ
ಮನು-ಷ್ಯ-ರಿಗೆ
ಮನು-ಷ್ಯರು
ಮನು-ಷ್ಯರೂ
ಮನು-ಷ್ಯ-ರೂ-ಪ-ದಿಂದ
ಮನು-ಷ್ಯ-ರೆಂದು
ಮನು-ಷ್ಯ-ಲೀ-ಲೆಗೆ
ಮನು-ಷ್ಯ-ಲೋ-ಕ-ದಲ್ಲಿ
ಮನು-ಷ್ಯಾದಿ
ಮನು-ಷ್ಯಾ-ವ-ತಾ-ರ-ವನ್ನು
ಮನೆ
ಮನೆ-ಕೆ-ಲ-ಸ-ಗ-ಳೊಂದೂ
ಮನೆ-ಗಳ
ಮನೆ-ಗಳನ್ನು
ಮನೆ-ಗಳಲ್ಲಿ
ಮನೆ-ಗ-ಳ-ಲ್ಲಿದ್ದ
ಮನೆ-ಗಳಿಂದ
ಮನೆ-ಗ-ಳಿಗೆ
ಮನೆ-ಗಳು
ಮನೆ-ಗ-ಳೆಲ್ಲ
ಮನೆಗೂ
ಮನೆಗೆ
ಮನೆ-ಗೆ-ಲಸ
ಮನೆ-ಗೆ-ಲ-ಸ-ದಲ್ಲಿ
ಮನೆ-ಗೆ-ಲ-ಸ-ವನ್ನು
ಮನೆ-ಗೆ-ಲ-ಸ-ವ-ನ್ನೆಲ್ಲ
ಮನೆ-ತ-ನದ
ಮನೆ-ದೇ-ವ-ತೆ-ಯಾದ
ಮನೆ-ದೇ-ವರ
ಮನೆ-ಬಿಟ್ಟು
ಮನೆ-ಮಠ
ಮನೆ-ಮ-ಠ-ಗಳನ್ನೂ
ಮನೆ-ಮ-ನೆಗೂ
ಮನೆ-ಮ-ನೆಯ
ಮನೆ-ಮ-ನೆ-ಯನ್ನೂ
ಮನೆ-ಮ-ನೆ-ಯ-ಲ್ಲಿಯೂ
ಮನೆ-ಮಾ-ಡಿ-ಕೊಂ-ಡಿ-ರುವ
ಮನೆಯ
ಮನೆ-ಯಂತೆ
ಮನೆ-ಯನ್ನು
ಮನೆ-ಯನ್ನೂ
ಮನೆ-ಯ-ಲ್ಲವೆ
ಮನೆ-ಯಲ್ಲಿ
ಮನೆ-ಯ-ಲ್ಲಿ-ಟ್ಟಳು
ಮನೆ-ಯ-ಲ್ಲಿಟ್ಟು
ಮನೆ-ಯ-ಲ್ಲಿದ್ದ
ಮನೆ-ಯ-ಲ್ಲಿ-ದ್ದರೂ
ಮನೆ-ಯ-ಲ್ಲಿ-ದ್ದ-ರೇನು
ಮನೆ-ಯ-ಲ್ಲಿ-ದ್ದು-ಕೊಂಡೆ
ಮನೆ-ಯ-ಲ್ಲಿಯೇ
ಮನೆ-ಯಲ್ಲೆ
ಮನೆ-ಯ-ಲ್ಲೆಲ್ಲ
ಮನೆ-ಯ-ವನು
ಮನೆ-ಯ-ವರೆ-ಲ್ಲರೂ
ಮನೆ-ಯಾ-ಯಿತು
ಮನೆ-ಯಿಂದ
ಮನೆಯೆ
ಮನೆ-ಯೆಲ್ಲ
ಮನೆ-ಯೊ-ಳ-ಗಿ-ನಿಂದ
ಮನೆ-ಯೊ-ಳಗೆ
ಮನೆ-ಹಾಳ
ಮನೋ
ಮನೋ-ಭಾ-ವ-ವನ್ನು
ಮನೋ-ಮ-ಯ-ನೆ-ನಿ-ಸಿ-ದ್ದಾನೆ
ಮನೋ-ರ-ಥ-ವನ್ನು
ಮನೋ-ವಾಕ್ಕು
ಮನೋ-ವೃ-ತ್ತಿಯೇ
ಮನೋ-ಹರ
ಮನೋ-ಹ-ರ-ಳಾದ
ಮನೋ-ಹ-ರ-ವಾಗಿ
ಮನೋ-ಹ-ರ-ವಾ-ಗಿತ್ತು
ಮನೋ-ಹ-ರ-ವಾ-ಗಿದೆ
ಮನೋ-ಹ-ರ-ವಾ-ಗಿಯೂ
ಮನೋ-ಹ-ರ-ವಾ-ಗಿ-ರು-ವಂತೆ
ಮನೋ-ಹ-ರ-ವಾದ
ಮನೋ-ಹ-ರಾ-ಕಾ-ರ-ರಾಗಿ
ಮನೋ-ಹರಿ
ಮನೋ-ಹರೆ
ಮನ್ನಣೆ
ಮನ್ನ-ಣೆ-ಯನ್ನು
ಮನ್ನಿಸಿ
ಮನ್ನಿ-ಸಿದ
ಮನ್ನಿ-ಸಿ-ದರು
ಮನ್ನಿ-ಸಿ-ದು-ದ-ಲ್ಲದೆ
ಮನ್ನಿ-ಸಿರಿ
ಮನ್ನಿ-ಸು-ವಂತೆ
ಮನ್ಮಥ
ಮನ್ಮ-ಥನ
ಮನ್ಮ-ಥ-ನಂ-ತಹ
ಮನ್ಮ-ಥ-ನಂ-ತಿ-ದ್ದಾರೆ
ಮನ್ಮ-ಥ-ನಂ-ತಿ-ರುವ
ಮನ್ಮ-ಥ-ನಂತೆ
ಮನ್ಮ-ಥ-ನನ್ನು
ಮನ್ಮ-ಥ-ನಾದ
ಮನ್ಮ-ಥ-ನಿಗೂ
ಮನ್ಮ-ಥನು
ಮನ್ಮ-ಥನೆ
ಮನ್ಮ-ಥ-ಬಾ-ಧೆಗೆ
ಮನ್ಮ-ಥ-ರಂತೆ
ಮನ್ಯು
ಮನ್ವಂ-ತರ
ಮನ್ವಂ-ತ-ರ-ಗಳನ್ನು
ಮನ್ವಂ-ತ-ರ-ಗ-ಳಷ್ಟು
ಮನ್ವಂ-ತ-ರದ
ಮನ್ವಂ-ತ-ರ-ದಲ್ಲಿ
ಮನ್ವಂ-ತ-ರ-ದ-ಲ್ಲಿಯೂ
ಮನ್ವಂ-ತ-ರಾ-ಧಿ-ಪ-ತಿ-ಗ-ಳಾ-ದರು
ಮಬ್ಬಿ-ನಲ್ಲಿ
ಮಬ್ಬಿ-ನ-ಲ್ಲಿದ್ದ
ಮಬ್ಬು
ಮಮ
ಮಮ-ಕಾರ
ಮಮ-ಕಾ-ರ-ಗಳಾ
ಮಮ-ಕಾ-ರ-ಗಳಿಂದ
ಮಮ-ಕಾ-ರ-ಗ-ಳಿಲ್ಲ
ಮಮ-ಕಾ-ರ-ಗಳು
ಮಮ-ಕಾ-ರ-ಗ-ಳುಳ್ಳ
ಮಮ-ಕಾ-ರ-ವನ್ನು
ಮಮತೆ
ಮಮ-ತೆಯ
ಮಮ-ತೆ-ಯನ್ನು
ಮಮ-ತೆ-ಯಿಂದ
ಮಮ-ತೆ-ಯಿಂ-ದಿದ್ದ
ಮಮ್ಮಲ
ಮಮ್ಮು
ಮಯ
ಮಯಂ
ಮಯನ
ಮಯ-ನನ್ನು
ಮಯ-ನಿಂದ
ಮಯ-ನಿ-ರ್ಮಿ-ತ-ವಾದ
ಮಯನು
ಮಯ-ನೆಂಬ
ಮಯ-ವಾದ
ಮಯಾ-ಯಾ-ಽಮೃ-ತ-ಮ-ಯಾಯ
ಮರ
ಮರ-ಕತ
ಮರಕ್ಕೆ
ಮರ-ಗಳ
ಮರ-ಗಳನ್ನು
ಮರ-ಗಳಲ್ಲಿ
ಮರ-ಗ-ಳಾ-ಗಿ-ರು-ವು-ದಕ್ಕೆ
ಮರ-ಗಳು
ಮರ-ಗಳೂ
ಮರ-ಗ-ಳೆ-ರಡು
ಮರ-ಗ-ಳೆ-ರಡೂ
ಮರ-ಗ-ಳೆಲ್ಲ
ಮರ-ಗ-ಳೆ-ಲ್ಲವೂ
ಮರ-ಗಿಡ
ಮರ-ಗಿ-ಡ-ಗಳನ್ನು
ಮರ-ಗಿ-ಡ-ಗಳನ್ನೂ
ಮರ-ಗಿ-ಡ-ಗಳು
ಮರ-ಗಿ-ಡ-ಗ-ಳೆಲ್ಲ
ಮರ-ಗಿ-ಡ-ಬ-ಳ್ಳಿ-ಗಳು
ಮರಣ
ಮರಣಂ
ಮರ-ಣ-ಕಾ-ಲ-ದಲ್ಲಿ
ಮರ-ಣದ
ಮರ-ಣ-ಭಯ
ಮರ-ಣ-ಭ-ಯ-ದಿಂದ
ಮರ-ಣ-ವನ್ನು
ಮರ-ಣ-ವಾ-ಗದ
ಮರ-ಣ-ವಾ-ಗ-ಬಾ-ರದು
ಮರ-ಣ-ವಾ-ಗ-ಲೆಂದು
ಮರ-ಣ-ವಿಲ್ಲ
ಮರ-ಣವು
ಮರ-ಣವೆ
ಮರ-ಣ-ವೆಂ-ಬುದು
ಮರ-ಣ-ಸಂ-ಕ-ಟಕ್ಕೆ
ಮರ-ಣ-ಸ-ನ್ನಿ-ಹಿ-ತ-ವಾ-ಗಿ-ರುವ
ಮರಣಾ
ಮರ-ತೇ-ಹೋ-ಯಿತು
ಮರದ
ಮರ-ದಂತೆ
ಮರ-ದ-ಡಿ-ಯಲ್ಲಿ
ಮರ-ದ-ಮೇಲೆ
ಮರ-ದಿಂದ
ಮರದೆ
ಮರ-ದೊ-ಳಗೆ
ಮರ-ಳ-ತೀ-ರ-ವನ್ನು
ಮರ-ಳ-ದಿ-ಣ್ಣೆಯ
ಮರಳಿ
ಮರ-ಳಿನ
ಮರ-ಳಿ-ನಲ್ಲಿ
ಮರಳು
ಮರ-ವನ್ನು
ಮರ-ವನ್ನೆ
ಮರ-ವ-ನ್ನೇರಿ
ಮರ-ವಾಗಿ
ಮರ-ವಿ-ರು-ವಂತೆ
ಮರವು
ಮರ-ವೊಂ-ದ-ನ್ನೇರಿ
ಮರಿ
ಮರಿ-ಗಳ
ಮರಿ-ಗ-ಳಂ-ತಾ-ಗಿ-ದ್ದೇವೆ
ಮರಿ-ಗ-ಳಂತೆ
ಮರಿ-ಗಳನ್ನು
ಮರಿ-ಗ-ಳಿ-ಗಾಗಿ
ಮರಿ-ಗ-ಳಿಗೆ
ಮರಿ-ಗಳು
ಮರಿಗೆ
ಮರಿ-ಮಗ
ಮರಿ-ಮ-ಗ-ನಾದ
ಮರಿ-ಮ-ಗನೇ
ಮರಿಯ
ಮರಿ-ಯಂತೆ
ಮರಿ-ಯನ್ನು
ಮರಿ-ಯಲ್ಲಿ
ಮರಿ-ಯಾ-ನೆ-ಗಳನ್ನೂ
ಮರಿಯು
ಮರಿಯೆ
ಮರಿ-ಯೊಂ-ದ-ರಲ್ಲಿ
ಮರೀ-ಚ-ಮ-ಹ-ರ್ಷಿಯ
ಮರೀ-ಚಾದಿ
ಮರೀಚಿ
ಮರೀ-ಚಿಗೆ
ಮರೀ-ಚಿಯೂ
ಮರೀ-ಚಿಯೇ
ಮರು
ಮರುಕ
ಮರು-ಕ-ವನ್ನು
ಮರು-ಕ-ವಿ-ಲ್ಲದೆ
ಮರು-ಕ್ಷ-ಣ-ದಲ್ಲಿ
ಮರು-ಕ್ಷ-ಣ-ದ-ಲ್ಲಿಯೆ
ಮರು-ಕ್ಷ-ಣ-ದ-ಲ್ಲಿಯೇ
ಮರು-ಗ-ಲಿಲ್ಲ
ಮರುಗಿ
ಮರು-ಗಿತು
ಮರು-ಗಿ-ದನು
ಮರು-ಗಿ-ದರು
ಮರು-ಗುತ್ತಾ
ಮರು-ಗು-ತ್ತಿತ್ತು
ಮರು-ಗು-ತ್ತಿದ್ದ
ಮರು-ಗು-ತ್ತಿರು
ಮರು-ಗು-ತ್ತಿ-ರು-ವೆಯೊ
ಮರು-ಗು-ವಳು
ಮರು-ಜ-ನ್ಮ-ದಲ್ಲಿ
ಮರುತ್ತು
ಮರು-ದಿನ
ಮರು-ದಿ-ನವೆ
ಮರು-ದಿ-ನವೇ
ಮರು-ದಿ-ವಸ
ಮರು-ನಿ-ಮಿಷ
ಮರು-ನಿ-ಮಿ-ಷ-ದಲ್ಲಿ
ಮರು-ನಿ-ಮಿ-ಷವೆ
ಮರು-ನಿ-ಮಿ-ಷವೇ
ಮರು-ಳ-ನಾಗಿ
ಮರು-ಳ-ನಾದ
ಮರು-ಳ-ನೆಂದೇ
ಮರು-ಳ-ರಂತೆ
ಮರು-ಳಾಗಿ
ಮರು-ಳಾ-ಗಿದೆ
ಮರು-ಳಾ-ಗಿ-ದ್ದ-ವ-ನಲ್ಲ
ಮರು-ಳಾ-ಗಿ-ದ್ದ-ವರೇ
ಮರು-ಳಾ-ಗಿ-ದ್ದೇನೆ
ಮರು-ಳಾ-ಗಿ-ಹೋ-ದ-ರಂತೆ
ಮರು-ಳಾಗು
ಮರು-ಳಾ-ಗುವ
ಮರು-ಳಾ-ಗು-ವ-ವ-ಳಲ್ಲ
ಮರು-ಳಾ-ಟಕ್ಕೆ
ಮರು-ಳಾದ
ಮರು-ಳಾ-ದ-ರೇನು
ಮರು-ಳಾ-ದ-ವರು
ಮರು-ಳಾದೆ
ಮರು-ಳಾ-ಯಿತು
ಮರುಳು
ಮರು-ಳು-ಗ-ಳೊ-ಡನೆ
ಮರು-ಳು-ಗೊ-ಳಿ-ಸಿದ
ಮರು-ಳು-ಗೊ-ಳಿ-ಸುವ
ಮರು-ಳು-ಮಾಡಿ
ಮರು-ಳು-ಮಾ-ಡಿ-ದನು
ಮರೆ
ಮರೆ-ಗೊ-ಳಿಸು
ಮರೆತ
ಮರೆ-ತರು
ಮರೆ-ತರೆ
ಮರೆ-ತಳು
ಮರೆ-ತ-ವ-ನಲ್ಲ
ಮರೆ-ತಾ-ದರೂ
ಮರೆ-ತಿ-ರು-ವು-ದ-ರಿಂದ
ಮರೆತು
ಮರೆ-ತು-ಬಿಟ್ಟ
ಮರೆ-ತು-ಬಿ-ಡು-ತ್ತಾನೆ
ಮರೆ-ತು-ಬಿ-ಡೋಣ
ಮರೆ-ತು-ಹೋ-ಗಿತ್ತು
ಮರೆ-ತು-ಹೋ-ಗು-ತ್ತಿತ್ತು
ಮರೆ-ತೆಯಾ
ಮರೆ-ತೇ-ಬಿಟ್ಟ
ಮರೆ-ತೇ-ಬಿ-ಟ್ಟೆ-ಯ-ಲ್ಲವೆ
ಮರೆ-ಮಾ-ಚದೆ
ಮರೆ-ಮಾ-ಚು-ವುದು
ಮರೆ-ಮಾ-ಡಿದ್ದ
ಮರೆ-ಮಾ-ಡು-ವು-ದಕ್ಕೆ
ಮರೆ-ಮೋ-ಸ-ಗಳಿಂದ
ಮರೆ-ಯ-ದಂ-ತಹ
ಮರೆ-ಯ-ದಂತೆ
ಮರೆ-ಯದೆ
ಮರೆ-ಯ-ಬೇಡ
ಮರೆ-ಯ-ಲಾ-ಗದು
ಮರೆ-ಯ-ಲಾರೆ
ಮರೆ-ಯಲ್ಲಿ
ಮರೆ-ಯ-ಲ್ಲಿ-ದ್ದು-ಕೊಂಡು
ಮರೆ-ಯಾ-ಗ-ದಿ-ರಲಿ
ಮರೆ-ಯಾಗಿ
ಮರೆ-ಯಾ-ಗಿ-ರುವ
ಮರೆ-ಯಾ-ಗು-ತ್ತಲೆ
ಮರೆ-ಯಿಂದ
ಮರೆಯು
ಮರೆ-ಯು-ವು-ದಿಲ್ಲ
ಮರೆ-ಯು-ವುದು
ಮರೆಸಿ
ಮರೆ-ಸಿ-ಕೊಂ-ಡರು
ಮರೆ-ಸಿ-ಕೊಂ-ಡಿ-ದ್ದಾಳೆ
ಮರೆ-ಸಿ-ಕೊಂ-ಡಿ-ದ್ದೇನೆ
ಮರೆ-ಸಿ-ಕೊಂಡು
ಮರೆ-ಸುವ
ಮರೆ-ಸು-ವಂ-ತಹ
ಮರೆ-ಹೊಕ್ಕ
ಮರೆ-ಹೊ-ಕ್ಕನು
ಮರೆ-ಹೊ-ಕ್ಕರು
ಮರೆ-ಹೊ-ಕ್ಕಿ-ದ್ದೇನೆ
ಮರೆ-ಹೋ-ಗ-ಬೇಕು
ಮರೆ-ಹೋಗು
ಮರ್ತ್ಯ
ಮರ್ತ್ಯ-ಲೋಕಂ
ಮರ್ದಿ-ಸಿ-ದುದೊ
ಮರ್ಯಾದೆ
ಮರ್ಯಾ-ದೆ-ಗಳನ್ನೆಲ್ಲ
ಮರ್ಯಾ-ದೆ-ಗ-ಳೊ-ಡನೆ
ಮರ್ಯಾ-ದೆ-ಗೊಟ್ಟು
ಮಲ-ಗ-ಬಾ-ರದು
ಮಲ-ಗ-ಬೇಕು
ಮಲಗಿ
ಮಲ-ಗಿ-ಕೊಂ-ಡರೆ
ಮಲ-ಗಿ-ಕೊಂ-ಡಳು
ಮಲ-ಗಿ-ಕೊ-ಳ್ಳಲು
ಮಲ-ಗಿತ್ತು
ಮಲ-ಗಿ-ದರೆ
ಮಲ-ಗಿದ್ದ
ಮಲ-ಗಿ-ದ್ದ-ಲ್ಲಿಗೆ
ಮಲ-ಗಿ-ದ್ದ-ಲ್ಲಿಯೇ
ಮಲ-ಗಿ-ದ್ದಳು
ಮಲ-ಗಿ-ದ್ದ-ವ-ನನ್ನು
ಮಲ-ಗಿ-ದ್ದ-ವನು
ಮಲ-ಗಿ-ದ್ದ-ವರೆಲ್ಲ
ಮಲ-ಗಿ-ದ್ದಾನೆ
ಮಲ-ಗಿ-ಬಿ-ಟ್ಟನು
ಮಲ-ಗಿ-ರ-ಬೇಕು
ಮಲ-ಗಿ-ರಲಿ
ಮಲ-ಗಿ-ರಲು
ಮಲ-ಗಿ-ರುವ
ಮಲ-ಗಿ-ರು-ವಂ-ತೆಯೊ
ಮಲ-ಗಿ-ರು-ವಾಗ
ಮಲ-ಗಿ-ರು-ವುದು
ಮಲ-ಗಿಸಿ
ಮಲ-ಗಿ-ಸಿ-ಕೊಂ-ಡಳು
ಮಲ-ಗಿ-ಸಿ-ಕೊಂಡು
ಮಲ-ಗಿ-ಸಿ-ದನು
ಮಲ-ಗಿ-ಸಿ-ದಳು
ಮಲ-ಗಿ-ಸಿದ್ದ
ಮಲಗು
ಮಲ-ಗು-ತ್ತಾ-ರೆಂದು
ಮಲ-ಗುವ
ಮಲ-ಗು-ವನು
ಮಲ-ಗು-ವಾಗ
ಮಲ-ಗು-ವು-ದಕ್ಕೆ
ಮಲ-ತಾ-ಯಿಯ
ಮಲ-ದ್ವಾರ
ಮಲ-ಮೂತ್ರ
ಮಲ-ಮೂ-ತ್ರ-ಗಳ
ಮಲ-ಮೂ-ತ್ರ-ಗಳನ್ನು
ಮಲ-ಮೂ-ತ್ರ-ಗಳು
ಮಲಯ
ಮಲ-ಯ-ಧ್ವ-ಜನು
ಮಲ-ಯ-ಧ್ವ-ಜ-ನೆಂಬ
ಮಲ-ಯ-ಮಾ-ರು-ತ-ವನ್ನು
ಮಲ-ಯಾ-ನಿಲ
ಮಲಿನ
ಮಲ್ಲ-ಯು-ದ್ಧ-ಮಾ-ಡುವ
ಮಲ್ಲರು
ಮಲ್ಲಿಗೆ
ಮಲ್ಲಿ-ಗೆ-ಗಳ
ಮಲ್ಲಿ-ಗೆಯ
ಮಳ-ಲ-ಮೇಲೆ
ಮಳೆ
ಮಳೆ-ಗ-ರೆದ
ಮಳೆ-ಗ-ರೆ-ದಂ-ತಾ-ದು-ದನ್ನು
ಮಳೆ-ಗ-ರೆ-ದಂ-ತಾ-ಯಿತು
ಮಳೆ-ಗ-ರೆ-ದನು
ಮಳೆ-ಗ-ರೆ-ದರು
ಮಳೆ-ಗ-ರೆ-ದಳು
ಮಳೆ-ಗ-ರೆದು
ಮಳೆ-ಗ-ರೆ-ಯುತ್ತಾ
ಮಳೆ-ಗ-ರೆ-ಯು-ವರು
ಮಳೆ-ಗ-ರೆ-ಯು-ವುದು
ಮಳೆ-ಗಾಲ
ಮಳೆ-ಗಾ-ಲ-ವಾ-ದು-ದ-ರಿಂದ
ಮಳೆಯ
ಮಳೆ-ಯನ್ನು
ಮಳೆ-ಯನ್ನೂ
ಮಳೆ-ಯನ್ನೆ
ಮಳೆ-ಯಲ್ಲಿ
ಮಳೆ-ಯಾ-ಗು-ತ್ತಿ-ತ್ತಂತೆ
ಮಸಿ
ಮಸೆ-ಯು-ತ್ತಿ-ದ್ದವು
ಮಸ್ತ-ಕವೂ
ಮಹ
ಮಹ-ಡಿ-ಮ-ನೆ-ಗಳು
ಮಹ-ಡಿಯ
ಮಹತೇ
ಮಹತ್
ಮಹ-ತ್ತತ್ವ
ಮಹ-ತ್ತ-ತ್ವ-ಇವು
ಮಹ-ತ್ತ-ತ್ವ-ಎಂಬ
ಮಹ-ತ್ತ-ತ್ವ-ದಿಂದ
ಮಹ-ತ್ತನ್ನು
ಮಹ-ತ್ತ-ರ-ವಾ-ದುದು
ಮಹ-ತ್ತಾದ
ಮಹ-ತ್ತಾ-ದುವು
ಮಹತ್ತು
ಮಹ-ತ್ತೆಲ್ಲ
ಮಹತ್ವಂ
ಮಹ-ತ್ವ-ವಿದೆ
ಮಹ-ದ-ಚ್ಚರಿ
ಮಹ-ದಾ-ಕಾ-ರ-ದಿಂದ
ಮಹ-ದಾ-ದಿ-ತ-ತ್ವ-ಗಳು
ಮಹ-ದಾ-ನಂ-ದ-ದಿಂದ
ಮಹ-ದೈ-ಶ್ವರ್ಯ
ಮಹ-ನೀ-ಯ-ನಾದ
ಮಹ-ನೀ-ಯರು
ಮಹ-ರ್ಲೋ-ಕ-ವನ್ನೂ
ಮಹರ್ಷಿ
ಮಹ-ರ್ಷಿ-ಗಳ
ಮಹ-ರ್ಷಿ-ಗಳನ್ನು
ಮಹ-ರ್ಷಿ-ಗಳನ್ನೂ
ಮಹ-ರ್ಷಿ-ಗಳನ್ನೆಲ್ಲ
ಮಹ-ರ್ಷಿ-ಗ-ಳಾದ
ಮಹ-ರ್ಷಿ-ಗಳಿಂದ
ಮಹ-ರ್ಷಿ-ಗ-ಳಿ-ಗಾ-ಗಿಯೂ
ಮಹ-ರ್ಷಿ-ಗ-ಳಿಗೆ
ಮಹ-ರ್ಷಿ-ಗಳು
ಮಹ-ರ್ಷಿ-ಗಳೂ
ಮಹ-ರ್ಷಿ-ಗಳೆ
ಮಹ-ರ್ಷಿ-ಗ-ಳೆ-ಲ್ಲರೂ
ಮಹ-ರ್ಷಿ-ಗಳೇ
ಮಹ-ರ್ಷಿ-ಗ-ಳೊ-ಡನೆ
ಮಹ-ರ್ಷಿಯ
ಮಹ-ರ್ಷಿ-ಯನ್ನು
ಮಹ-ರ್ಷಿ-ಯಾದ
ಮಹ-ರ್ಷಿಯು
ಮಹ-ರ್ಷಿಯೂ
ಮಹ-ರ್ಷಿಯೆ
ಮಹ-ರ್ಷಿಯೇ
ಮಹ-ರ್ಷಿ-ಶಾ-ಪವೂ
ಮಹಾ
ಮಹಾಂತ
ಮಹಾಂ-ತ-ಮರ್ಥಂ
ಮಹಾ-ಕಾ-ರ್ಯ-ಗ-ಳನ್ನೆ
ಮಹಾ-ಕಾ-ರ್ಯ-ಗ-ಳಾ-ವುವು
ಮಹಾ-ಕಾ-ರ್ಯ-ವನ್ನು
ಮಹಾ-ಕಾ-ರ್ಯ-ವೇನೂ
ಮಹಾ-ಕಾಲ
ಮಹಾ-ಕಾ-ಳಿಗೆ
ಮಹಾ-ಕೃತಿ
ಮಹಾ-ಕೋ-ಪ-ವುಳ್ಳ
ಮಹಾ-ಕ್ರೂರಿ
ಮಹಾ-ಗ-ರ್ವ-ದಿಂದ
ಮಹಾ-ಗ್ರಂ-ಥ-ಗಳೆ
ಮಹಾ-ಜಲ
ಮಹಾ-ಜ-ಲದ
ಮಹಾ-ಜ-ಲ-ವನ್ನು
ಮಹಾ-ಜ್ಞಾ-ನಿ-ಗಳು
ಮಹಾ-ಜ್ಞಾ-ನಿ-ಯಾದ
ಮಹಾ-ಟ್ಟ-ಹಾಸಂ
ಮಹಾ-ತ-ಪಸ್ವಿ
ಮಹಾ-ತ-ಪ-ಸ್ವಿ-ಗ-ಳಾದ
ಮಹಾ-ತ-ಪ-ಸ್ವಿ-ಗಳು
ಮಹಾ-ತ-ಪ-ಸ್ವಿ-ಯಾಗಿ
ಮಹಾ-ತ-ಪ-ಸ್ವಿ-ಯಾದ
ಮಹಾ-ತಳ
ಮಹಾ-ತೇ-ಜ-ಸ್ವಿ-ಯಾ-ಗಿ-ರುವ
ಮಹಾ-ತೇ-ಜ-ಸ್ವಿ-ಯಾ-ದನು
ಮಹಾತ್ಮ
ಮಹಾ-ತ್ಮ-ನಂತೆ
ಮಹಾ-ತ್ಮ-ನಾದ
ಮಹಾ-ತ್ಮ-ನಿಂದ
ಮಹಾ-ತ್ಮ-ನಿಗೆ
ಮಹಾ-ತ್ಮನು
ಮಹಾ-ತ್ಮರ
ಮಹಾ-ತ್ಮ-ರಾದ
ಮಹಾ-ತ್ಮರು
ಮಹಾ-ತ್ಮರೆ
ಮಹಾ-ತ್ಮ-ರೆಂಬ
ಮಹಾತ್ಮಾ
ಮಹಾತ್ಮೆ
ಮಹಾತ್ಮ್ಯ
ಮಹಾ-ತ್ಮ್ಯ-ವನ್ನೂ
ಮಹಾ-ತ್ಯಾಗಿ
ಮಹಾ-ದೇ-ವನ
ಮಹಾ-ದೇ-ವನು
ಮಹಾ-ದೇವಿ
ಮಹಾ-ಧ-ನು-ವಿನ
ಮಹಾ-ನಂ-ದನ
ಮಹಾ-ನಂ-ದ-ನೆಂ-ಬು-ವನು
ಮಹಾ-ನದಿ
ಮಹಾ-ನ-ದಿ-ಗಳೂ
ಮಹಾನು
ಮಹಾ-ನು-ಭಾವ
ಮಹಾ-ನು-ಭಾ-ವ-ನನ್ನು
ಮಹಾ-ನು-ಭಾ-ವ-ನಾ-ಗಿ-ದ್ದಾನೆ
ಮಹಾ-ನು-ಭಾ-ವ-ನಾದ
ಮಹಾ-ನು-ಭಾ-ವ-ನಾರು
ಮಹಾ-ನು-ಭಾ-ವ-ನಿಗೆ
ಮಹಾ-ನು-ಭಾ-ವನು
ಮಹಾ-ನು-ಭಾ-ವನೆ
ಮಹಾ-ನು-ಭಾ-ವ-ನೆಂದು
ಮಹಾ-ನು-ಭಾ-ವರ
ಮಹಾ-ನು-ಭಾ-ವ-ರನ್ನು
ಮಹಾ-ನು-ಭಾ-ವ-ರಲ್ಲಿ
ಮಹಾ-ನು-ಭಾ-ವ-ರಾದ
ಮಹಾ-ನು-ಭಾ-ವ-ರಿರಾ
ಮಹಾ-ನು-ಭಾ-ವರು
ಮಹಾ-ನು-ಭಾ-ವರೆ
ಮಹಾ-ನು-ಭಾ-ವ-ರೆಂದು
ಮಹಾ-ನು-ಭಾ-ವರೇ
ಮಹಾ-ನು-ಭಾ-ವ-ರೊ-ಬ್ಪರು
ಮಹಾ-ನು-ಭಾ-ವಳು
ಮಹಾ-ನು-ಭಾ-ವಾಯ
ಮಹಾ-ಪ-ತಿ-ವ್ರ-ತೆ-ಯನ್ನು
ಮಹಾ-ಪ-ತಿ-ವ್ರ-ತೆ-ಯಾದ
ಮಹಾ-ಪ-ತಿ-ವ್ರ-ತೆ-ಯೆ-ನಿ-ಸಿ-ದ್ದಳು
ಮಹಾ-ಪ-ರಾ-ಕ್ರಮಿ
ಮಹಾ-ಪ-ರಾ-ಕ್ರ-ಮಿ-ಯಾದ
ಮಹಾ-ಪ-ರಾ-ಕ್ರ-ಮಿಯೆ
ಮಹಾ-ಪ-ರ್ವತ
ಮಹಾ-ಪ-ರ್ವ-ತ-ಗಳೂ
ಮಹಾ-ಪ-ರ್ವ-ತ-ದಿಂದ
ಮಹಾ-ಪ-ರ್ವ-ತಾ-ಕಾ-ರ-ವಾಗಿ
ಮಹಾ-ಪಾಪ
ಮಹಾ-ಪಾಪಿ
ಮಹಾ-ಪಾ-ಪಿ-ಯೆಂದೂ
ಮಹಾ-ಪು-ರಾಣ
ಮಹಾ-ಪು-ರಾ-ಣ-ಗಳ
ಮಹಾ-ಪು-ರಾ-ಣ-ಗಳನ್ನು
ಮಹಾ-ಪು-ರಾ-ಣ-ಗಳು
ಮಹಾ-ಪು-ರಾ-ಣವು
ಮಹಾ-ಪು-ರಾ-ಣಾಯ
ಮಹಾ-ಪು-ರುಷ
ಮಹಾ-ಪು-ರು-ಷನ
ಮಹಾ-ಪು-ರು-ಷ-ನಾದ
ಮಹಾ-ಪು-ರು-ಷನು
ಮಹಾ-ಪು-ರು-ಷನೆ
ಮಹಾ-ಪು-ರು-ಷ-ನೆಂದು
ಮಹಾ-ಪು-ರು-ಷ-ನೊಬ್ಬ
ಮಹಾ-ಪು-ರು-ಷ-ನೊ-ಬ್ಬನು
ಮಹಾ-ಪು-ರು-ಷರ
ಮಹಾ-ಪು-ರು-ಷ-ರಂತೆ
ಮಹಾ-ಪು-ರು-ಷ-ರನ್ನು
ಮಹಾ-ಪು-ರು-ಷರು
ಮಹಾ-ಪು-ರು-ಷಾಯ
ಮಹಾ-ಪು-ರು-ಷಾ-ಯಾಽಭಿ
ಮಹಾ-ಪು-ಷಿ-ಗಳು
ಮಹಾ-ಪ್ರಭು
ಮಹಾ-ಪ್ರ-ಮಾ-ಣಕ್ಕೆ
ಮಹಾ-ಪ್ರ-ಳಯ
ಮಹಾ-ಪ್ರ-ಳ-ಯ-ದಲ್ಲಿ
ಮಹಾ-ಪ್ರ-ಳ-ಯ-ವೊಂದು
ಮಹಾ-ಪ್ರ-ಸಾ-ದ-ವೆಂದು
ಮಹಾ-ಪ್ರ-ಸ್ಥಾ-ನಕ್ಕೆ
ಮಹಾ-ಪ್ರ-ಸ್ಥಾ-ನ-ವನ್ನು
ಮಹಾ-ಪ್ರಾಜ್ಞ
ಮಹಾ-ಬ-ಲ-ಶಾಲಿ
ಮಹಾ-ಬ-ಲ-ಶಾ-ಲಿ-ಯಾದ
ಮಹಾ-ಬ್ರಾ-ಹ್ಮಣ
ಮಹಾ-ಭಕ್ತ
ಮಹಾ-ಭಾಗ
ಮಹಾ-ಭಾ-ಗ-ವ-ತ-ನ-ನ್ನಾಗಿ
ಮಹಾ-ಭಾ-ಗ-ವ-ತ-ರಾಗಿ
ಮಹಾ-ಭಾ-ಗ್ಯ-ಶಾಲಿ
ಮಹಾ-ಭಾ-ರತ
ಮಹಾ-ಭಾ-ರ-ತ-ಗಳನ್ನು
ಮಹಾ-ಭಾ-ರ-ತ-ದಲ್ಲಿ
ಮಹಾ-ಭಾ-ರ-ತ-ವನ್ನೂ
ಮಹಾ-ಭಾ-ರ-ತವೇ
ಮಹಾ-ಭಾ-ಷ್ಯ-ದಲ್ಲಿ
ಮಹಾ-ಮಂತ್ರ
ಮಹಾ-ಮಂ-ತ್ರ-ವನ್ನು
ಮಹಾ-ಮ-ತ್ಸ್ಯದ
ಮಹಾ-ಮ-ತ್ಸ್ಯಾಯ
ಮಹಾ-ಮ-ಹಿ-ನಾದ
ಮಹಾ-ಮ-ಹಿಮ
ಮಹಾ-ಮ-ಹಿ-ಮನ
ಮಹಾ-ಮ-ಹಿ-ಮ-ನಾಗಿ
ಮಹಾ-ಮ-ಹಿ-ಮ-ನಾ-ಗಿ-ರುವ
ಮಹಾ-ಮ-ಹಿ-ಮ-ನಾದ
ಮಹಾ-ಮ-ಹಿ-ಮನು
ಮಹಾ-ಮ-ಹಿ-ಮನೂ
ಮಹಾ-ಮ-ಹಿ-ಮ-ನೆ-ನಿ-ಸಿ-ದನು
ಮಹಾ-ಮ-ಹಿ-ಮ-ರಾದ
ಮಹಾ-ಮ-ಹಿ-ಮಾ-ನ್ವಿ-ತರು
ಮಹಾ-ಮಾಯಾ
ಮಹಾ-ಮಾಯೆ
ಮಹಾ-ಮಾ-ಯೆಯ
ಮಹಾ-ಮೀ-ನಿಗೆ
ಮಹಾ-ಮು-ನಿ-ಗಳ
ಮಹಾ-ಮು-ನಿಯ
ಮಹಾ-ಮೋಹ
ಮಹಾ-ಯಾ-ಗ-ವನ್ನು
ಮಹಾ-ಯಾ-ಗ-ವೊಂ-ದನ್ನು
ಮಹಾ-ಯೋಗಿ
ಮಹಾ-ಯೋ-ಗಿ-ಗ-ಳಿಗೂ
ಮಹಾ-ಯೋ-ಗಿ-ಗ-ಳಿಗೆ
ಮಹಾ-ಯೋ-ಗಿ-ಗಳು
ಮಹಾ-ಯೋ-ಗಿ-ಗ-ಳೆ-ನಿಸಿ
ಮಹಾ-ಯೋ-ಜನೆ
ಮಹಾ-ಯೋ-ಜ-ನೆ-ಎಂದು
ಮಹಾ-ಯೋ-ಜ-ನೆಯ
ಮಹಾ-ರ-ತ್ನ-ದಂತೆ
ಮಹಾ-ರ-ಥ-ರನ್ನು
ಮಹಾ-ರ-ಥರು
ಮಹಾ-ರಾಜ
ಮಹಾ-ರಾ-ಜನ
ಮಹಾ-ರಾ-ಜ-ನಂತೆ
ಮಹಾ-ರಾ-ಜ-ನನ್ನು
ಮಹಾ-ರಾ-ಜ-ನಾಗಿ
ಮಹಾ-ರಾ-ಜ-ನಾದ
ಮಹಾ-ರಾ-ಜ-ನಿಗೆ
ಮಹಾ-ರಾ-ಜನು
ಮಹಾ-ರಾ-ಜ-ನೊ-ಬ್ಬ-ನನ್ನು
ಮಹಾ-ರಾ-ಜ-ರಿಗೆ
ಮಹಾ-ರಾ-ಜರು
ಮಹಾ-ರಾಜಾ
ಮಹಾ-ರಾ-ಜಾಯ
ಮಹಾ-ಲಕ್ಷ್ಮಿ
ಮಹಾ-ಲ-ಕ್ಷ್ಮಿ-ಯಂತೆ
ಮಹಾ-ಲ-ಕ್ಷ್ಮಿ-ಯೊ-ಡನೆ
ಮಹಾ-ವಾ-ಯು-ವಿ-ನೊ-ಡನೆ
ಮಹಾ-ವಿಭೂ
ಮಹಾ-ವಿಷ್ಣು
ಮಹಾ-ವಿ-ಷ್ಣು-ವನ್ನು
ಮಹಾ-ವಿ-ಷ್ಣು-ವಿನ
ಮಹಾ-ವಿ-ಷ್ಣುವು
ಮಹಾ-ವಿ-ಷ್ಣುವೆ
ಮಹಾ-ವಿ-ಷ್ಣು-ವೆಂ-ಬು-ವ-ನನ್ನು
ಮಹಾ-ವಿ-ಷ್ಣು-ವೆ-ನಿ-ಸುವೆ
ಮಹಾ-ವೀರ
ಮಹಾ-ವೀ-ರರೆ
ಮಹಾ-ವೃಕ್ಷ
ಮಹಾ-ವೃ-ಕ್ಷ-ಗಳು
ಮಹಾ-ವ್ಯಕ್ತಿ
ಮಹಾ-ವ್ಯ-ಕ್ತಿ-ಗಳು
ಮಹಾಶ
ಮಹಾ-ಶ-ಕ್ತ-ನಾದ
ಮಹಾ-ಶ-ರೀ-ರಿ-ಯಾಗಿ
ಮಹಾ-ಶೂರ
ಮಹಾ-ಶೂ-ರ-ನೆಂದು
ಮಹಾ-ಶೂ-ರ-ರಾದ
ಮಹಾ-ಶೂ-ರ-ರಿಂದ
ಮಹಾ-ಸ-ಮುದ್ರ
ಮಹಾ-ಸ-ರ-ಸ್ಸಿ-ನಿಂದ
ಮಹಾ-ಸರ್ಪ
ಮಹಾ-ಸ-ರ್ಪ-ಗಳು
ಮಹಾ-ಸಾ-ಹ-ಸ-ದಿಂ-ದಲೆ
ಮಹಾ-ಸೇನೆ
ಮಹಾ-ಸೈ-ನ್ಯ-ವನ್ನು
ಮಹಾ-ಸ್ತ್ರ-ವೊಂ-ದನ್ನು
ಮಹಾ-ಸ್ವಾಮಿ
ಮಹಾ-ಽಧ್ವ-ರಾ-ಽವ-ಯ-ವಾಯ
ಮಹಿಮ
ಮಹಿ-ಮ-ನಾದ
ಮಹಿ-ಮ-ರಾದ
ಮಹಿ-ಮಾ-ಶಾಲಿ
ಮಹಿಮೆ
ಮಹಿ-ಮೆ-ಗಳನ್ನು
ಮಹಿ-ಮೆ-ಗಳನ್ನೂ
ಮಹಿ-ಮೆಯ
ಮಹಿ-ಮೆ-ಯ-ನ್ನ-ರಿತ
ಮಹಿ-ಮೆ-ಯ-ನ್ನ-ರಿ-ಯ-ಲಾ-ರದೆ
ಮಹಿ-ಮೆ-ಯನ್ನು
ಮಹಿ-ಮೆ-ಯನ್ನೂ
ಮಹಿ-ಮೆ-ಯಿಂದ
ಮಹಿ-ಮೆ-ಯುಳ್ಳ
ಮಹಿ-ಮೆಯೆ
ಮಹಿ-ಮೆ-ಯೆಂ-ತ-ಹು-ದೆಂ-ಬು-ದನ್ನು
ಮಹಿ-ಮೆ-ಯೆಂಬ
ಮಹಿಷೀ
ಮಹಿ-ಷೀ-ಗೀತೆ
ಮಹೇಂ-ದ್ರ-ಪ-ರ್ವ-ತ-ದಲ್ಲಿ
ಮಹೇ-ಶ್ವರ
ಮಹೇ-ಶ್ವ-ರ-ನಿಗೂ
ಮಹೇ-ಶ್ವ-ರ-ರೆಂಬ
ಮಹೋ
ಮಹೋ-ತ್ಸವ
ಮಹೋ-ತ್ಸ-ವಕ್ಕೆ
ಮಹೋ-ತ್ಸ-ವ-ವನ್ನು
ಮಹೋ-ನ್ನ-ತ-ವಾದ
ಮಹೌ-ದಾ-ರ್ಯಕ್ಕೆ
ಮಾ
ಮಾಂ
ಮಾಂಗ-ಲ್ಯ-ಭಾ-ಗ್ಯ-ವನ್ನು
ಮಾಂಡವ್ಯ
ಮಾಂಧಾತ
ಮಾಂಧಾ-ತನ
ಮಾಂಧಾ-ತ-ನಿಗೆ
ಮಾಂಧಾ-ತನು
ಮಾಂಧಾತಾ
ಮಾಂಧಾ-ತೃ-ಮ-ಹಾ-ರಾ-ಜನ
ಮಾಂಧಾ-ತೃ-ವಿನ
ಮಾಂಸ
ಮಾಂಸ-ಕ್ಕಾಗಿ
ಮಾಂಸದ
ಮಾಂಸ-ದಿಂದ
ಮಾಂಸ-ದಿಂ-ದಲೆ
ಮಾಂಸ-ವನ್ನು
ಮಾಂಸ-ವನ್ನೆ
ಮಾಂಸ-ವ-ನ್ನೆಲ್ಲ
ಮಾಂಸ-ವನ್ನೇ
ಮಾಂಸ-ವೆಲ್ಲ
ಮಾಗ-ಧರು
ಮಾಡ
ಮಾಡ-ಕೂ-ಡದು
ಮಾಡದ
ಮಾಡ-ದಂತೆ
ಮಾಡ-ದಿ-ದ್ದರೆ
ಮಾಡ-ದಿ-ರ-ಲೆಂಬ
ಮಾಡ-ದಿ-ರುವ
ಮಾಡ-ದಿ-ರು-ವುದನ್ನು
ಮಾಡ-ದುದು
ಮಾಡದೆ
ಮಾಡ-ಬಲ್ಲ
ಮಾಡ-ಬ-ಲ್ಲ-ವಮ್ಮ
ಮಾಡ-ಬ-ಹುದು
ಮಾಡ-ಬಾ-ರದ
ಮಾಡ-ಬಾ-ರ-ದಿತ್ತು
ಮಾಡ-ಬಾ-ರದು
ಮಾಡ-ಬಾ-ರ-ದು-ದನ್ನು
ಮಾಡ-ಬಾ-ರದೆ
ಮಾಡ-ಬಾ-ರ-ದೇಕೆ
ಮಾಡ-ಬೇ-ಕಾ-ಗಿತ್ತು
ಮಾಡ-ಬೇ-ಕಾ-ಗಿದ್ದ
ಮಾಡ-ಬೇ-ಕಾದ
ಮಾಡ-ಬೇ-ಕಾ-ದರೆ
ಮಾಡ-ಬೇ-ಕಾ-ದುದೇ
ಮಾಡ-ಬೇ-ಕಾ-ಯಿತು
ಮಾಡ-ಬೇಕು
ಮಾಡ-ಬೇಕೆ
ಮಾಡ-ಬೇ-ಕೆಂ-ದಿ-ರಲು
ಮಾಡ-ಬೇ-ಕೆಂ-ದಿ-ರುವ
ಮಾಡ-ಬೇ-ಕೆಂ-ದಿ-ರು-ವೆನು
ಮಾಡ-ಬೇ-ಕೆಂದು
ಮಾಡ-ಬೇ-ಕೆಂ-ದು-ಕೊಂ-ಡಿದ್ದ
ಮಾಡ-ಬೇ-ಕೆಂ-ದು-ಕೊಂಡು
ಮಾಡ-ಬೇ-ಕೆಂ-ಬುದು
ಮಾಡ-ಬೇ-ಕೆಂ-ಬುದೇ
ಮಾಡ-ಬೇ-ಕೆ-ನಿ-ಸಿತು
ಮಾಡ-ಬೇ-ಕೆ-ನ್ನಿ-ಸಿತು
ಮಾಡ-ಬೇಡ
ಮಾಡ-ಲಾ-ಗ-ಲಿಲ್ಲ
ಮಾಡ-ಲಾ-ಯಿತು
ಮಾಡ-ಲಾ-ರದ
ಮಾಡಲಿ
ಮಾಡ-ಲಿಲ್ಲ
ಮಾಡ-ಲಿ-ಲ್ಲ-ವಲ್ಲ
ಮಾಡಲು
ಮಾಡ-ಲು-ದಾರಿ-ತೋ-ರ-ಲು-ವ್ಯ-ಕ್ತಿ-ಗ-ಳಿ-ದ್ದಾರೆ
ಮಾಡಲೂ
ಮಾಡ-ಲೆಂದು
ಮಾಡ-ಲ್ಪ-ಟ್ಟಿತು
ಮಾಡ-ಹೊ-ರ-ಡು-ವುದು
ಮಾಡಿ
ಮಾಡಿಕೊ
ಮಾಡಿ-ಕೊಂಡ
ಮಾಡಿ-ಕೊಂ-ಡನು
ಮಾಡಿ-ಕೊಂ-ಡರು
ಮಾಡಿ-ಕೊಂ-ಡರೆ
ಮಾಡಿ-ಕೊಂ-ಡವು
ಮಾಡಿ-ಕೊಂ-ಡಿ-ರ-ಲಿ-ಎಂದು
ಮಾಡಿ-ಕೊಂಡು
ಮಾಡಿ-ಕೊಂಡೆ
ಮಾಡಿ-ಕೊಂ-ಡೊ-ಡ-ನೆಯೇ
ಮಾಡಿ-ಕೊಟ್ಟ
ಮಾಡಿ-ಕೊ-ಟ್ಟಂತೆ
ಮಾಡಿ-ಕೊ-ಟ್ಟ-ನಂತೆ
ಮಾಡಿ-ಕೊ-ಟ್ಟನು
ಮಾಡಿ-ಕೊ-ಟ್ಟಾಳು
ಮಾಡಿ-ಕೊ-ಟ್ಟಿತು
ಮಾಡಿ-ಕೊಟ್ಟು
ಮಾಡಿ-ಕೊಡು
ಮಾಡಿ-ಕೊ-ಡು-ತ್ತಾನೆ
ಮಾಡಿ-ಕೊ-ಡು-ವಂತೆ
ಮಾಡಿ-ಕೊ-ಳ್ಳ-ತ್ತೇನೆ
ಮಾಡಿ-ಕೊ-ಳ್ಳದೆ
ಮಾಡಿ-ಕೊ-ಳ್ಳ-ಬ-ಲ್ಲರು
ಮಾಡಿ-ಕೊ-ಳ್ಳ-ಬೇಕು
ಮಾಡಿ-ಕೊ-ಳ್ಳ-ಬೇ-ಕೆಂದು
ಮಾಡಿ-ಕೊ-ಳ್ಳ-ಬೇ-ಕೆಂಬ
ಮಾಡಿ-ಕೊ-ಳ್ಳಲಿ
ಮಾಡಿ-ಕೊ-ಳ್ಳಲು
ಮಾಡಿ-ಕೊಳ್ಳಿ
ಮಾಡಿ-ಕೊ-ಳ್ಳು-ತ್ತಾನೆ
ಮಾಡಿ-ಕೊ-ಳ್ಳು-ತ್ತಾರೆ
ಮಾಡಿ-ಕೊ-ಳ್ಳುವ
ಮಾಡಿ-ಕೊ-ಳ್ಳು-ವಾಗ
ಮಾಡಿ-ಕೊ-ಳ್ಳು-ವುದು
ಮಾಡಿತು
ಮಾಡಿತ್ತು
ಮಾಡಿದ
ಮಾಡಿ-ದಂತೆ
ಮಾಡಿ-ದ-ನಂತೆ
ಮಾಡಿ-ದ-ನಮ್ಮ
ಮಾಡಿ-ದ-ನಷ್ಟೆ
ಮಾಡಿ-ದನು
ಮಾಡಿ-ದ-ನೆಂದು
ಮಾಡಿ-ದನೇ
ಮಾಡಿ-ದನೊ
ಮಾಡಿ-ದನೋ
ಮಾಡಿ-ದ-ಮೇಲೆ
ಮಾಡಿ-ದ-ರಾ-ಯಿತು
ಮಾಡಿ-ದರು
ಮಾಡಿ-ದರೂ
ಮಾಡಿ-ದರೆ
ಮಾಡಿ-ದ-ರೇನು
ಮಾಡಿ-ದ-ಳ-ಲ್ಲಮ್ಮ
ಮಾಡಿ-ದಳು
ಮಾಡಿ-ದ-ವ-ನಾ-ದನು
ಮಾಡಿ-ದ-ವ-ನಾ-ದರೂ
ಮಾಡಿ-ದ-ವ-ನಾ-ದು-ದ-ರಿಂದ
ಮಾಡಿ-ದ-ವ-ನಿಗೆ
ಮಾಡಿ-ದ-ವನು
ಮಾಡಿ-ದ-ವ-ರ-ಲ್ಲಿಯೂ
ಮಾಡಿ-ದ-ವರು
ಮಾಡಿ-ದಾಗ
ಮಾಡಿ-ದಿರಿ
ಮಾಡಿ-ದು-ದನ್ನು
ಮಾಡಿ-ದು-ದ-ನ್ನೆಲ್ಲ
ಮಾಡಿ-ದು-ದರ
ಮಾಡಿ-ದು-ದ-ರಿಂದ
ಮಾಡಿ-ದುದು
ಮಾಡಿ-ದುದೇ
ಮಾಡಿ-ದು-ದೇಕೆ
ಮಾಡಿದೆ
ಮಾಡಿ-ದೆ-ನಲ್ಲಾ
ಮಾಡಿ-ದೆ-ಯಲ್ಲ
ಮಾಡಿ-ದೆಯಾ
ಮಾಡಿ-ದೆ-ಯೆಂ-ಬುದು
ಮಾಡಿ-ದೆಯೋ
ಮಾಡಿ-ದೊ-ಡನೆ
ಮಾಡಿದ್ದ
ಮಾಡಿ-ದ್ದನ್ನು
ಮಾಡಿ-ದ್ದರು
ಮಾಡಿ-ದ್ದರೂ
ಮಾಡಿ-ದ್ದಳು
ಮಾಡಿ-ದ್ದಳೊ
ಮಾಡಿ-ದ್ದವು
ಮಾಡಿ-ದ್ದವೊ
ಮಾಡಿ-ದ್ದಾನೆ
ಮಾಡಿ-ದ್ದಾರೊ
ಮಾಡಿದ್ದಿ
ಮಾಡಿ-ದ್ದೆನೊ
ಮಾಡಿ-ದ್ದೆವೊ
ಮಾಡಿ-ದ್ದೇನೆ
ಮಾಡಿ-ಬಿ-ಟ್ಟ-ನಲ್ಲ
ಮಾಡಿ-ಬಿಟ್ಟೆ
ಮಾಡಿ-ಬಿ-ಡ-ಬೇಕು
ಮಾಡಿ-ಬಿ-ಡು-ತ್ತದೆ
ಮಾಡಿ-ಬಿ-ಡು-ತ್ತಾರೆ
ಮಾಡಿಯೆ
ಮಾಡಿ-ರ-ಬೇಕು
ಮಾಡಿರಿ
ಮಾಡಿ-ರುವ
ಮಾಡಿ-ರು-ವನು
ಮಾಡಿ-ರು-ವರು
ಮಾಡಿ-ರು-ವಿ-ರಂತೆ
ಮಾಡಿ-ರು-ವಿ-ರಲ್ಲಾ
ಮಾಡಿ-ರುವೆ
ಮಾಡಿ-ರು-ವೆ-ವೆಂದೇ
ಮಾಡಿವೆ
ಮಾಡಿ-ವೆ-ಯೆಂದು
ಮಾಡಿಸ
ಮಾಡಿಸಿ
ಮಾಡಿ-ಸಿ-ಕೊಂಡು
ಮಾಡಿ-ಸಿದ
ಮಾಡಿ-ಸಿ-ದನು
ಮಾಡಿ-ಸಿ-ದ-ಮೇಲೆ
ಮಾಡಿ-ಸಿ-ದರು
ಮಾಡಿ-ಸಿ-ದ-ವನು
ಮಾಡಿ-ಸಿ-ದುದು
ಮಾಡಿ-ಸಿದೆ
ಮಾಡಿ-ಸಿ-ದೆ-ನೆಂದು
ಮಾಡಿ-ಸಿದ್ದ
ಮಾಡಿ-ಸಿ-ರುವ
ಮಾಡಿಸು
ಮಾಡಿ-ಸುವ
ಮಾಡಿ-ಸು-ವ-ವ-ರಿಗೆ
ಮಾಡೀಯೆ
ಮಾಡು
ಮಾಡು-ಎಂದು
ಮಾಡು-ಎಂಬ
ಮಾಡುತ್ತ
ಮಾಡು-ತ್ತದೆ
ಮಾಡು-ತ್ತ-ದೆ-ಎಂದು
ಮಾಡು-ತ್ತ-ದೆಯೆ
ಮಾಡು-ತ್ತಲೆ
ಮಾಡು-ತ್ತಲೇ
ಮಾಡು-ತ್ತ-ವೆಯೆ
ಮಾಡುತ್ತಾ
ಮಾಡು-ತ್ತಾನೆ
ಮಾಡು-ತ್ತಾ-ರಂತೆ
ಮಾಡು-ತ್ತಾರೆ
ಮಾಡು-ತ್ತಾ-ರೆಯೆ
ಮಾಡು-ತ್ತಾಳೊ
ಮಾಡುತ್ತಿ
ಮಾಡು-ತ್ತಿತ್ತು
ಮಾಡು-ತ್ತಿದ್ದ
ಮಾಡು-ತ್ತಿ-ದ್ದನು
ಮಾಡು-ತ್ತಿ-ದ್ದ-ನೆಂದು
ಮಾಡು-ತ್ತಿ-ದ್ದ-ರಿಂದ
ಮಾಡು-ತ್ತಿ-ದ್ದರು
ಮಾಡು-ತ್ತಿ-ದ್ದರೂ
ಮಾಡು-ತ್ತಿ-ದ್ದಳು
ಮಾಡು-ತ್ತಿ-ದ್ದವು
ಮಾಡು-ತ್ತಿ-ದ್ದಾಗ
ಮಾಡು-ತ್ತಿ-ದ್ದಾನೆ
ಮಾಡು-ತ್ತಿ-ದ್ದಾರೆ
ಮಾಡು-ತ್ತಿದ್ದು
ಮಾಡು-ತ್ತಿ-ದ್ದು-ದ-ರಿಂ-ದಲೆ
ಮಾಡು-ತ್ತಿ-ದ್ದುದು
ಮಾಡು-ತ್ತಿದ್ದೆ
ಮಾಡು-ತ್ತಿ-ದ್ದೇನೆ
ಮಾಡು-ತ್ತಿ-ರಲು
ಮಾಡು-ತ್ತಿರಿ
ಮಾಡು-ತ್ತಿರು
ಮಾಡು-ತ್ತಿ-ರುವ
ಮಾಡು-ತ್ತಿ-ರು-ವನು
ಮಾಡು-ತ್ತಿ-ರು-ವ-ವನು
ಮಾಡು-ತ್ತಿ-ರು-ವ-ಷ್ಟ-ರಲ್ಲಿ
ಮಾಡು-ತ್ತಿ-ರು-ವಾಗ
ಮಾಡು-ತ್ತಿ-ರು-ವು-ದ-ರಿಂದ
ಮಾಡು-ತ್ತಿ-ರು-ವುದು
ಮಾಡು-ತ್ತಿ-ರುವೆ
ಮಾಡು-ತ್ತಿ-ರು-ವೆ-ಯಲ್ಲಾ
ಮಾಡು-ತ್ತೀನಿ
ಮಾಡು-ತ್ತೀಯಾ
ಮಾಡು-ತ್ತೇನೆ
ಮಾಡು-ತ್ತೇ-ನೆಂದು
ಮಾಡು-ತ್ತೇವೆ
ಮಾಡುವ
ಮಾಡು-ವಂ-ತ-ಹ-ನಾ-ಗ-ಬೇಕು
ಮಾಡು-ವಂ-ತಿ-ರ-ಲಿಲ್ಲ
ಮಾಡು-ವಂ-ತಿಲ್ಲ
ಮಾಡು-ವಂತೆ
ಮಾಡು-ವ-ನಂತೆ
ಮಾಡು-ವನು
ಮಾಡು-ವರು
ಮಾಡು-ವ-ರೆಂದು
ಮಾಡು-ವರೊ
ಮಾಡು-ವಳು
ಮಾಡು-ವ-ವ-ನಂತೆ
ಮಾಡು-ವ-ವ-ನಲ್ಲಿ
ಮಾಡು-ವ-ವ-ನ-ಲ್ಲಿಯೇ
ಮಾಡು-ವ-ವ-ನಾ-ಗಿ-ರುವೆ
ಮಾಡು-ವ-ವನು
ಮಾಡು-ವ-ವ-ನು-ಮಾ-ಡಿ-ಸಿ-ಕೊಳ್ಳು
ಮಾಡು-ವ-ವನೂ
ಮಾಡು-ವ-ವನೇ
ಮಾಡು-ವ-ವ-ರನ್ನು
ಮಾಡು-ವ-ವ-ರ-ನ್ನೆಲ್ಲ
ಮಾಡು-ವ-ವ-ರಾ-ದರು
ಮಾಡು-ವ-ವ-ರಾ-ದು-ದ-ರಿಂದ
ಮಾಡು-ವ-ವ-ರಿ-ಗಿಂ-ತಲೂ
ಮಾಡು-ವ-ವರು
ಮಾಡು-ವ-ಷ್ಟ-ರಲ್ಲಿ
ಮಾಡು-ವ-ಹಾ-ಗಿದ್ದೆ
ಮಾಡು-ವ-ಹಾಗೆ
ಮಾಡು-ವಾಗ
ಮಾಡುವಿ
ಮಾಡುವು
ಮಾಡು-ವು-ದ-ಕ್ಕಾಗಿ
ಮಾಡು-ವು-ದ-ಕ್ಕಾ-ಗಿಯೆ
ಮಾಡು-ವು-ದಕ್ಕೂ
ಮಾಡು-ವು-ದಕ್ಕೆ
ಮಾಡು-ವುದನ್ನು
ಮಾಡು-ವು-ದರ
ಮಾಡು-ವು-ದಾಗಿ
ಮಾಡು-ವು-ದಾ-ದರೂ
ಮಾಡು-ವುದು
ಮಾಡು-ವುದೂ
ಮಾಡು-ವು-ದೆಂ-ಬು-ದನ್ನು
ಮಾಡು-ವು-ದೇನು
ಮಾಡುವೆ
ಮಾಡು-ವೆನು
ಮಾಡು-ವೆ-ನೆಂದು
ಮಾಡು-ವೆ-ಯಂತೆ
ಮಾಡು-ವೆ-ಯೇನು
ಮಾಡೋಣ
ಮಾಣಿ-ಕ್ಯ-ವಾ-ಗಿತ್ತು
ಮಾತ-ನಾ-ಡದೆ
ಮಾತ-ನಾ-ಡ-ಬ-ಹುದು
ಮಾತ-ನಾ-ಡ-ಲಾ-ರದೆ
ಮಾತ-ನಾ-ಡಲು
ಮಾತ-ನಾಡಿ
ಮಾತ-ನಾ-ಡಿ-ಕೊಂ-ಡರು
ಮಾತ-ನಾ-ಡಿ-ಕೊ-ಳ್ಳುತ್ತಾ
ಮಾತ-ನಾ-ಡಿ-ದರೆ
ಮಾತ-ನಾ-ಡಿಸಿ
ಮಾತ-ನಾ-ಡಿ-ಸುತ್ತಾ
ಮಾತ-ನಾ-ಡಿ-ಸು-ತ್ತಿ-ದ್ದು-ದನ್ನು
ಮಾತ-ನಾ-ಡಿ-ಸು-ವಂತೆ
ಮಾತ-ನಾಡು
ಮಾತ-ನಾ-ಡುತ್ತಾ
ಮಾತ-ನಾ-ಡುತ್ತಿ
ಮಾತ-ನಾ-ಡು-ತ್ತಿತ್ತು
ಮಾತ-ನಾ-ಡು-ತ್ತಿ-ದ್ದಂತೆ
ಮಾತ-ನಾ-ಡು-ತ್ತಿ-ದ್ದಂ-ತೆಯೆ
ಮಾತ-ನಾ-ಡು-ತ್ತಿ-ದ್ದರು
ಮಾತ-ನಾ-ಡು-ತ್ತಿ-ದ್ದರೂ
ಮಾತ-ನಾ-ಡು-ತ್ತಿ-ದ್ದು-ದನ್ನು
ಮಾತ-ನಾ-ಡು-ತ್ತಿ-ರು-ವಾಗ
ಮಾತ-ನಾ-ಡು-ವು-ದಿಲ್ಲ
ಮಾತ-ನ್ನಾ-ಡಿ-ದ್ದಳು
ಮಾತನ್ನು
ಮಾತನ್ನೂ
ಮಾತ-ಲ್ಲದೆ
ಮಾತಾ-ಗಿ-ರಲಿ
ಮಾತಾ-ಡ-ಬಾ-ರದು
ಮಾತಾ-ಡ-ಲಿಲ್ಲ
ಮಾತಾ-ಡಿ-ಸಿ-ಕೊಂಡು
ಮಾತಾಡು
ಮಾತಾ-ಡು-ತ್ತಲೆ
ಮಾತಾ-ಡುತ್ತಿ
ಮಾತಾ-ಡು-ತ್ತಿ-ದ್ದರು
ಮಾತಾ-ಡು-ತ್ತಿ-ದ್ದ-ವ-ರಿ-ಬ್ಬರೂ
ಮಾತಾ-ಡು-ತ್ತಿ-ದ್ದಾರೆ
ಮಾತಾ-ಡು-ತ್ತಿ-ರು-ವುದು
ಮಾತಾ-ಡುವ
ಮಾತಾ-ಡು-ವುದನ್ನು
ಮಾತಾ-ಯಿತು
ಮಾತಿ
ಮಾತಿಗೆ
ಮಾತಿ-ಗೆಲ್ಲಾ
ಮಾತಿನ
ಮಾತಿ-ನಂ-ತಲ್ಲ
ಮಾತಿ-ನಂತೆ
ಮಾತಿ-ನಲ್ಲಿ
ಮಾತಿ-ನಿಂದ
ಮಾತಿ-ನಿಂ-ದಲೆ
ಮಾತಿ-ನಿಂ-ದಲ್ಲ
ಮಾತು
ಮಾತು-ಕ-ಥೆಗೆ
ಮಾತು-ಕೊ-ಟ್ಟಂತೆ
ಮಾತು-ಕೊ-ಟ್ಟಿ-ದ್ದಂತೆ
ಮಾತು-ಕೊ-ಟ್ಟಿ-ದ್ದೇನೆ
ಮಾತು-ಗ-ಳ-ನ್ನಾ-ಡಿದೆ
ಮಾತು-ಗ-ಳ-ನ್ನಾ-ಡು-ತ್ತಿ-ರುವೆ
ಮಾತು-ಗಳನ್ನು
ಮಾತು-ಗಳನ್ನೆಲ್ಲ
ಮಾತು-ಗಳಿಂದ
ಮಾತು-ಗ-ಳಿಗೂ
ಮಾತು-ಗ-ಳಿಗೆ
ಮಾತು-ಗಳು
ಮಾತು-ಗಳೆ
ಮಾತು-ಗ-ಳೆಲ್ಲ
ಮಾತು-ಬ-ಲ್ಲ-ವ-ನಯ್ಯ
ಮಾತೂ
ಮಾತೃ-ಸ್ಥಾ-ನ-ವಾ-ದ್ದ-ರಿಂದ
ಮಾತೆಂ-ದರೆ
ಮಾತೆ-ತ್ತಿ-ದರೆ
ಮಾತೆಲ್ಲ
ಮಾತೇ
ಮಾತ್ಮ
ಮಾತ್ಮನ
ಮಾತ್ಮ-ನನ್ನು
ಮಾತ್ಮನು
ಮಾತ್ಮನೆ
ಮಾತ್ಮರ
ಮಾತ್ರ
ಮಾತ್ರಕ್ಕೆ
ಮಾತ್ರ-ದಲ್ಲಿ
ಮಾತ್ರ-ದ-ವರು
ಮಾತ್ರ-ದಿಂ-ದಲೆ
ಮಾತ್ರ-ನಾದ
ಮಾತ್ರ-ರಿಂದ
ಮಾತ್ರ-ವಾ-ಗಿದೆ
ಮಾತ್ರ-ವಾ-ಯಿತು
ಮಾತ್ರವೇ
ಮಾಥುರ
ಮಾದ-ಕ-ವಾದ
ಮಾದ-ರಿ-ಯಾ-ಗ-ಲೆಂದು
ಮಾದ-ರಿ-ಯಾ-ಗಿದೆ
ಮಾದ-ರಿ-ಯಾ-ಗು-ವಂತೆ
ಮಾದ-ರಿ-ಯಾ-ಗು-ವು-ದಕ್ಕೆ
ಮಾಧವ
ಮಾಧವೀ
ಮಾಧವೋ
ಮಾಧು-ರ್ಯ-ವನ್ನು
ಮಾಧ್ಯಂ-ದಿನೇ
ಮಾನ
ಮಾನ-ಗಳು
ಮಾನದ
ಮಾನ-ಧ-ನ-ನಾದ
ಮಾನ-ನೀ-ಯೋಸಿ
ಮಾನ-ಭಂ-ಗ-ದಿಂದ
ಮಾನ-ಭಂ-ಗ-ಮಾಡಿ
ಮಾನವ
ಮಾನ-ವ-ಜ-ನ್ಮ-ವನ್ನು
ಮಾನ-ವನ
ಮಾನ-ವ-ನಂ-ತೆಯೆ
ಮಾನ-ವ-ನಂ-ತೆಯೇ
ಮಾನ-ವ-ನಾಗಿ
ಮಾನ-ವ-ನಾ-ದರೂ
ಮಾನ-ವನು
ಮಾನ-ವನೂ
ಮಾನ-ವ-ನೆಂ-ದರೆ
ಮಾನ-ವನ್ನು
ಮಾನ-ವನ್ನೂ
ಮಾನ-ವನ್ನೆ
ಮಾನ-ವ-ಮಾತ್ರ
ಮಾನ-ವರ
ಮಾನ-ವ-ರಂತೆ
ಮಾನ-ವ-ರನ್ನು
ಮಾನ-ವ-ರನ್ನೂ
ಮಾನ-ವ-ರ-ಲ್ಲಿಯೂ
ಮಾನ-ವ-ರಾಗಿ
ಮಾನ-ವ-ರಾ-ದು-ದನ್ನು
ಮಾನ-ವ-ರಿಂದ
ಮಾನ-ವ-ರಿಗೆ
ಮಾನ-ವ-ರಿ-ಗೆಲ್ಲ
ಮಾನ-ವ-ರಿ-ಗೆ-ಲ್ಲಿ-ಯದು
ಮಾನ-ವರು
ಮಾನ-ವ-ರೂ-ಪ-ದಲ್ಲಿ
ಮಾನ-ವ-ರೂ-ಪಿ-ನಿಂದ
ಮಾನ-ವರೆ-ಲ್ಲ-ರಿಗೂ
ಮಾನ-ವಳೇ
ಮಾನ-ವಳೊ
ಮಾನ-ವ-ವಿ-ಕಾ-ರ-ಗ-ಳೆಲ್ಲ
ಮಾನ-ವ-ಸಂ-ತಾ-ನ-ವನ್ನು
ಮಾನ-ವಾ-ಕೃ-ತಿ-ಯಾಗಿ
ಮಾನ-ವಾ-ತೀತ
ಮಾನಸ
ಮಾನ-ಸ-ಪು-ತ್ರ-ರನ್ನು
ಮಾನ-ಸ-ಪು-ತ್ರ-ರಾದ
ಮಾನ-ಸ-ಸ-ರೋ-ವ-ರ-ದಲ್ಲಿ
ಮಾನ-ಸೋ-ತ್ತರ
ಮಾನಾವ
ಮಾನಾ-ವ-ಮಾ-ನದ
ಮಾನುಷಂ
ಮಾಮು-ಗ್ರ-ಧ-ರ್ಮಾ-ದ-ಖಿ-ಲಾ-ತ್ಪ್ರ-ಮಾದ
ಮಾಯ
ಮಾಯ-ದಲ್ಲೋ
ಮಾಯ-ವಾ-ಗಲು
ಮಾಯ-ವಾಗಿ
ಮಾಯ-ವಾ-ಗಿದೆ
ಮಾಯ-ವಾ-ಗಿ-ಹೋ-ಗಿದ್ದ
ಮಾಯ-ವಾ-ಗಿ-ಹೋ-ಗು-ತ್ತದೆ
ಮಾಯ-ವಾ-ಗಿ-ಹೋ-ದಳು
ಮಾಯ-ವಾ-ಗಿ-ಹೋ-ಯಿತು
ಮಾಯ-ವಾ-ಗು-ತ್ತದೆ
ಮಾಯ-ವಾ-ದನು
ಮಾಯ-ವಾ-ದಳು
ಮಾಯ-ವಾ-ಯಿತು
ಮಾಯಾ
ಮಾಯಾ-ಕ-ಲ್ಪಿತ
ಮಾಯಾ-ಕ-ಲ್ಪಿ-ತ-ಗ-ಳೆಂದು
ಮಾಯಾ-ತೀತ
ಮಾಯಾ-ತೀ-ತನು
ಮಾಯಾ-ದೇವಿ
ಮಾಯಾ-ದೇ-ವಿಯ
ಮಾಯಾ-ದೇ-ವಿ-ಯರ
ಮಾಯಾ-ದೇ-ವಿಯು
ಮಾಯಾ-ನ-ಟನೆ
ಮಾಯಾ-ನಿ-ಮಿ-ತ್ತ-ಗ-ಳಾದ
ಮಾಯಾ-ನಿ-ಯಾ-ಮ-ಕ-ನಾದ
ಮಾಯಾ-ಪ್ರ-ಭಾ-ವ-ವನ್ನು
ಮಾಯಾ-ಬ-ಲ-ದಿಂದ
ಮಾಯಾ-ಮಯ
ಮಾಯಾ-ಮ-ಯದ
ಮಾಯಾ-ಮ-ಯ-ವಾದ
ಮಾಯಾ-ಯು-ದ್ಧಕ್ಕೆ
ಮಾಯಾ-ವ-ಟು-ವಾ-ಮ-ನೋ-ಽವ್ಯಾತ್
ಮಾಯಾ-ವ-ತಿಯ
ಮಾಯಾ-ವ-ತಿ-ಯನ್ನು
ಮಾಯಾ-ವ-ತಿಯು
ಮಾಯಾ-ವಾಗಿ
ಮಾಯಾ-ವಿ-ದ್ಯೆ-ಗಳನ್ನೆಲ್ಲ
ಮಾಯಾ-ವಿ-ದ್ಯೆಗೆ
ಮಾಯಾ-ವಿ-ದ್ಯೆ-ಯನ್ನು
ಮಾಯಾ-ವಿ-ದ್ಯೆ-ಯ-ಲ್ಲಿಯೂ
ಮಾಯಾ-ವಿ-ದ್ಯೆ-ಯಿಂದ
ಮಾಯಾ-ವಿ-ದ್ಯೆ-ಯೆಂದು
ಮಾಯಾ-ವಿ-ಮಾ-ನ-ವ-ನ್ನೇರಿ
ಮಾಯಾ-ವಿ-ಯಾದ
ಮಾಯಾ-ಶಕ್ತಿ
ಮಾಯಾ-ಶ-ಕ್ತಿ-ಯಿಂದ
ಮಾಯಾ-ಸು-ರ-ನೆಂ-ಬು-ವನು
ಮಾಯೆ
ಮಾಯೆ-ಗಳನ್ನೆಲ್ಲ
ಮಾಯೆಗೂ
ಮಾಯೆಗೆ
ಮಾಯೆಯ
ಮಾಯೆ-ಯನ್ನು
ಮಾಯೆ-ಯಲ್ಲಿ
ಮಾಯೆ-ಯಿಂದ
ಮಾಯೆ-ಯಿಂ-ದಲೆ
ಮಾಯೆಯು
ಮಾಯೆಯೂ
ಮಾಯೆಯೆ
ಮಾಯೆ-ಯೆಂದು
ಮಾಯೆ-ಯೆಂಬ
ಮಾಯೆ-ಯೆ-ನ್ನು-ತ್ತಾರೆ
ಮಾಯೆ-ಯೆಲ್ಲ
ಮಾಯೆಯೇ
ಮಾಯೆಯೊ
ಮಾಯೆಯೋ
ಮಾರ-ಕಾಸ್ತ್ರ
ಮಾರ-ಣ-ಹೋಮ
ಮಾರ-ಣ-ಹೋ-ಮ-ವನ್ನು
ಮಾರದೆ
ಮಾರಿ-ಕೊಂ-ಡೆಯಾ
ಮಾರಿಗೆ
ಮಾರಿ-ನಷ್ಟು
ಮಾರಿ-ಯನ್ನು
ಮಾರಿ-ಯಾದ
ಮಾರಿ-ಯೊಂದು
ಮಾರಿಷಾ
ಮಾರೀಚ
ಮಾರೀ-ಚನ
ಮಾರು
ಮಾರುತ
ಮಾರುದ
ಮಾರು-ದ-ಅ-ಳ-ಬೇ-ಡಿ-ಎಂದು
ಮಾರು-ದ್ದದ
ಮಾರು-ವು-ದ-ಕ್ಕಾಗಿ
ಮಾರು-ವು-ದಿ-ಲ್ಲ-ವೇನು
ಮಾರು-ಹೋಗಿ
ಮಾರು-ಹೋ-ಗಿ-ದ್ದೇನೆ
ಮಾರು-ಹೋ-ಗಿ-ರುವ
ಮಾರು-ಹೋದ
ಮಾರ್ಕಂ
ಮಾರ್ಕಂ-ಡೇಯ
ಮಾರ್ಕಂ-ಡೇ-ಯ-ಋಷಿ
ಮಾರ್ಕಂ-ಡೇ-ಯ-ಋ-ಷಿಯ
ಮಾರ್ಕಂ-ಡೇ-ಯನ
ಮಾರ್ಕಂ-ಡೇ-ಯ-ನಿ-ಗಾದ
ಮಾರ್ಕಂ-ಡೇ-ಯನು
ಮಾರ್ಕಂ-ಡೇ-ಯ-ಪು-ರಾಣ
ಮಾರ್ಕಾಂ-ಡೇಯ
ಮಾರ್ಗ
ಮಾರ್ಗ-ಗಳನ್ನು
ಮಾರ್ಗ-ಗ-ಳಿಂ-ದಲೂ
ಮಾರ್ಗ-ಗ-ಳುಂಟು
ಮಾರ್ಗ-ದ-ರ್ಶ-ಕ-ನಾ-ಗ-ಬೇ-ಕಾದ
ಮಾರ್ಗ-ದ-ರ್ಶ-ನ-ಕ್ಕಾಗಿ
ಮಾರ್ಗ-ದಲ್ಲಿ
ಮಾರ್ಗ-ದ-ಲ್ಲಿಯೇ
ಮಾರ್ಗ-ದಿಂದ
ಮಾರ್ಗ-ಮ-ಧ್ಯ-ದ-ಲ್ಲಿಯೇ
ಮಾರ್ಗ-ಮ-ಧ್ಯ-ದಲ್ಲೆ
ಮಾರ್ಗ-ವನ್ನು
ಮಾರ್ಗ-ವನ್ನೆ
ಮಾರ್ಗ-ವಾಗಿ
ಮಾರ್ಗ-ವಿಲ್ಲ
ಮಾರ್ಗ-ವಿ-ಲ್ಲದೆ
ಮಾರ್ಗ-ವಿ-ಲ್ಲ-ವಾ-ಯಿತು
ಮಾರ್ಗ-ವುಂಟೆ
ಮಾರ್ಗವೇ
ಮಾರ್ಗಾ-ಯಾ-ಸ-ದಿಂದ
ಮಾರ್ಜಾರ
ಮಾರ್ತಾಂ-ಡ-ಎಂದು
ಮಾರ್ತಾಂ-ಡನ
ಮಾರ್ಮಿ-ಕ-ವಾಗಿ
ಮಾರ್ಮೊ-ಳ-ಗಿನ
ಮಾರ್ಮೊ-ಳ-ಗು-ತ್ತಿ-ರಲು
ಮಾಲಿ
ಮಾಲೆ
ಮಾಲೆ-ಗಳನ್ನು
ಮಾಲೆ-ಯನ್ನು
ಮಾಲೆ-ಯಲ್ಲಿ
ಮಾಲೆ-ಯಾಗಿ
ಮಾಲೆ-ಯೊಂ-ದನ್ನು
ಮಾಳಿಗೆ
ಮಾವ
ಮಾವ-ಟಿಗ
ಮಾವ-ಟಿ-ಗ-ನನ್ನು
ಮಾವ-ಟಿ-ಗನೂ
ಮಾವ-ಟಿ-ಗ-ನೊ-ಡನೆ
ಮಾವ-ಟಿ-ಗರ
ಮಾವನ
ಮಾವ-ನಲ್ಲಿ
ಮಾವಿ-ನ-ಮ-ರವೆ
ಮಾವು
ಮಾಸ
ಮಾಸು-ವೆಯೊ
ಮಾಹಾ-ಮಾತ್ರಾ
ಮಾಹಿ-ಷ್ಮ-ತಿ-ಯ-ಲ್ಲಿದ್ದ
ಮಾಹಿ-ಷ್ಮತೀ
ಮಾಹೇ-ಶ್ವರ
ಮಾಽಧ್ವನಿ
ಮಿಂಚನ್ನು
ಮಿಂಚಿ-ದಂ-ತಾ-ಯಿತು
ಮಿಂಚಿ-ದುದೇ
ಮಿಂಚಿನ
ಮಿಂಚಿ-ನಂ-ತಿ-ರುವ
ಮಿಂಚಿ-ನಂತೆ
ಮಿಂಚಿ-ನೊ-ಡನೆ
ಮಿಂಚಿ-ಹೋದ
ಮಿಂಚು
ಮಿಂಚು-ತ್ತಿತ್ತು
ಮಿಂದು
ಮಿಕ್ಕರೆ
ಮಿಕ್ಕು-ದನ್ನು
ಮಿಗಿ-ಲಾ-ಗಿ-ರು-ವನು
ಮಿಗಿ-ಲಾದ
ಮಿಗಿ-ಲಾ-ದುದು
ಮಿಗಿ-ಲಾ-ದು-ವ-ಲ್ಲವೆ
ಮಿಚ್ಛತಾ
ಮಿಡುಕಿ
ಮಿಡು-ಕಿ-ಕೊಂಡ
ಮಿಡು-ಕಿ-ಕೊಂ-ಡನು
ಮಿಡು-ಕಿ-ಕೊಂ-ಡರು
ಮಿಡು-ಕು-ತ್ತಿ-ದ್ದರೆ
ಮಿಡು-ಕು-ತ್ತಿ-ದ್ದೆ-ವಲ್ಲ
ಮಿತ
ಮಿತ-ವಾದ
ಮಿತಿ
ಮಿತಿ-ಮೀರಿ
ಮಿತಿ-ಯಿಂದ
ಮಿತಿ-ಯಿ-ಲ್ಲದ
ಮಿತ್ರ
ಮಿತ್ರ-ನಂತೆ
ಮಿತ್ರ-ನಾಗಿ
ಮಿತ್ರನೂ
ಮಿತ್ರ-ರನ್ನೂ
ಮಿತ್ರ-ರಾ-ಜರು
ಮಿತ್ರ-ರಾದ
ಮಿತ್ರ-ರಿಗೂ
ಮಿತ್ರ-ರೆಂಬ
ಮಿತ್ರ-ವಿಂದೆ
ಮಿತ್ರ-ವಿಂ-ದೆ-ಯನ್ನು
ಮಿತ್ರ-ವಿಂ-ದೆ-ವೃಕ
ಮಿತ್ರ-ಸ-ಮ್ಮಿತ
ಮಿತ್ರ-ಸಹ
ಮಿತ್ರ-ಸ-ಹನ
ಮಿತ್ರ-ಸ-ಹನು
ಮಿತ್ರಾ-ವ-ರು-ಣರ
ಮಿತ್ರಾ-ವ-ರು-ಣ-ವೆಂಬ
ಮಿಥಿ-ಲನು
ಮಿಥಿ-ಲ-ನೆಂದು
ಮಿಥಿ-ಲಾ-ನ-ಗ-ರಿಗೆ
ಮಿಥಿ-ಲಾ-ನ-ಗ-ರಿ-ಯನ್ನು
ಮಿಥಿ-ಲಾ-ನ-ಗ-ರಿ-ಯೆಂ-ದಾ-ಯಿತು
ಮಿಥಿ-ಲಾ-ರಾ-ಜ-ನಿಂದ
ಮಿಥಿ-ಲೆಗೆ
ಮಿಥಿ-ಲೆಯ
ಮಿಥಿ-ಲೆ-ಯನ್ನು
ಮಿಥು-ನ-ವಾ-ಯಿತು
ಮಿಥ್ಯ
ಮಿಥ್ಯಾ
ಮಿಥ್ಯಾ-ರೂ-ಪ-ವಾದ
ಮಿಮಃ
ಮಿಳ-ನನ್ನು
ಮಿಸು-ಕಾ-ಡ-ದಂತಾ
ಮಿಸ್ರ
ಮೀಟಿ
ಮೀನನ್ನು
ಮೀನಲ್ಲ
ಮೀನಾಗಿ
ಮೀನಾ-ದ-ನೇಕೆ
ಮೀನಿನ
ಮೀನಿ-ನಂ-ತಿ-ರುವ
ಮೀನಿ-ನಾ-ಕಾ-ರದ
ಮೀನು
ಮೀನು-ಗಳ
ಮೀನು-ಗಳು
ಮೀನು-ರೂ-ಪಿ-ಯಾದ
ಮೀನೂ
ಮೀನೆ
ಮೀನೆಂದು
ಮೀನೊಂ-ದನ್ನು
ಮೀನೊಂದು
ಮೀರ-ಲಾ-ರದೆ
ಮೀರಿ
ಮೀರಿದ
ಮೀರಿ-ದರೆ
ಮೀರಿ-ದುದು
ಮೀರಿಲ್ಲ
ಮೀರಿ-ಸಿತ್ತು
ಮೀರು-ವಂ-ತಿಲ್ಲ
ಮೀರು-ವಷ್ಟು
ಮೀಶ-ಪ್ರ-ಯುಕ್ತೋ
ಮೀಸ-ಲಾಗಿ
ಮೀಸ-ಲಾಗಿದೆ
ಮೀಸ-ಲಾ-ಗಿ-ರಲಿ
ಮೀಸ-ಲಾ-ಗಿ-ರುವ
ಮೀಸ-ಲಿ-ರಿ-ಸಿವೆ
ಮೀಸೆ
ಮೀಸೆ-ಗಳು
ಮೀಸೆ-ಗಳೂ
ಮೀಸೆ-ಯನ್ನೂ
ಮೀಸೆ-ಯಿಂದ
ಮುಂಕು-ಮು-ಚ್ಚಿ-ಕೊಂ-ಡಿದೆ
ಮುಂಗಾಣ
ಮುಂಗಾ-ಣ-ದವ
ಮುಂಗಾ-ಣ-ದ-ವ-ನಾಗಿ
ಮುಂಗಾ-ಣ-ದ-ವ-ರಾಗಿ
ಮುಂಗಾ-ಣದೆ
ಮುಂಗಾರ
ಮುಂಗು
ಮುಂಗು-ರು-ಳನ್ನು
ಮುಂಗು-ರುಳು
ಮುಂಗು-ರು-ಳು-ಗಳಿಂದ
ಮುಂಗು-ರು-ಳು-ನಿನ್ನ
ಮುಂಗೂ-ದಲು
ಮುಂಚೆ
ಮುಂಚೆಯೂ
ಮುಂಚೆಯೇ
ಮುಂಜಾನೆ
ಮುಂಜಿ
ಮುಂಡ
ಮುಂಡ-ಗಳು
ಮುಂಡ-ವಿ-ಲ್ಲದ
ಮುಂತಾದ
ಮುಂದಕ್ಕೆ
ಮುಂದ-ಕ್ಕೆ-ಳೆ-ದು-ಕೊ-ಳ್ಳುವ
ಮುಂದಣ
ಮುಂದಿಟ್ಟ
ಮುಂದಿಟ್ಟು
ಮುಂದಿ-ಟ್ಟು-ಕೊಂಡು
ಮುಂದಿ-ಟ್ಟು-ದನ್ನು
ಮುಂದಿನ
ಮುಂದಿ-ನ-ದನ್ನು
ಮುಂದು
ಮುಂದು-ಗ-ಡೆ-ಯಲ್ಲಿ
ಮುಂದು-ಮಾ-ಡಿ-ಕೊಂಡು
ಮುಂದು-ರು-ಳಿತು
ಮುಂದು-ವ-ರಿ-ದರೆ
ಮುಂದು-ವ-ರಿ-ಯ-ಲಿಲ್ಲ
ಮುಂದು-ವ-ರಿ-ಯಿತೊ
ಮುಂದು-ವ-ರಿ-ಯುತ್ತಾ
ಮುಂದು-ವ-ರಿ-ಯು-ತ್ತಿ-ದ್ದವು
ಮುಂದು-ವ-ರಿ-ಯು-ವಂತೆ
ಮುಂದು-ವ-ರಿ-ಯು-ವರು
ಮುಂದು-ವ-ರಿ-ಯು-ವುದು
ಮುಂದು-ವ-ರಿಸಿ
ಮುಂದು-ವ-ರಿ-ಸಿತು
ಮುಂದು-ವ-ರಿ-ಸಿ-ದನು
ಮುಂದು-ವ-ರಿ-ಸಿ-ದರು
ಮುಂದು-ವ-ರಿ-ಸಿ-ದ-ಸುಂ-ದರಿ
ಮುಂದು-ವ-ರಿಸು
ಮುಂದು-ವ-ರಿ-ಸುತ್ತಾ
ಮುಂದು-ವ-ರಿ-ಸು-ತ್ತೇನೆ
ಮುಂದು-ವರೆ
ಮುಂದು-ವರೆ-ಯಿತು
ಮುಂದೂ-ಡು-ವುದು
ಮುಂದೆ
ಮುಂದೆ-ಮಾ-ಡಿ-ಕೊಂಡು
ಮುಂದೆಯೂ
ಮುಂದೆಯೇ
ಮುಂದೆಷ್ಟು
ಮುಂದೇನು
ಮುಂದೋ-ರದ
ಮುಂದೋ-ರ-ದಂ-ತಾ-ಯಿತು
ಮುಂದೋ-ರ-ದ-ವ-ರಾಗಿ
ಮುಂದೋ-ರದೆ
ಮುಂಭಾ-ಗ-ದ-ಲ್ಲಿಯೇ
ಮುಕುಂದ
ಮುಕುಂ-ದನ
ಮುಕುಂ-ದ-ನನ್ನು
ಮುಕುಂ-ದಾ-ಪ-ಹೃ-ತಾ-ತ್ಮ-ಲಾಂ-ಭನಃ
ಮುಕ್ಕಣ್ಣ
ಮುಕ್ಕ-ಣ್ಣನು
ಮುಕ್ಕಾ-ಗಿ-ದ್ದು-ದನ್ನು
ಮುಕ್ಕಾ-ಲು-ಪಾ-ಲಿ-ನಷ್ಟು
ಮುಕ್ಕಿ-ದನು
ಮುಕ್ಕು-ಳಿ-ಸಿ-ದುವು
ಮುಕ್ತ
ಮುಕ್ತ-ಕಂ-ಠ-ದಿಂದ
ಮುಕ್ತ-ಕಂ-ಠ-ನಾಗಿ
ಮುಕ್ತ-ಕಂ-ಠ-ರಾಗಿ
ಮುಕ್ತ-ಕಂ-ಠ-ವಾಗಿ
ಮುಕ್ತ-ನಾ-ಗ-ಬೇಕು
ಮುಕ್ತ-ನಾ-ಗ-ಲಾ-ರದೆ
ಮುಕ್ತ-ನಾಗಿ
ಮುಕ್ತ-ನಾಗು
ಮುಕ್ತ-ನಾ-ಗು-ತ್ತಾನೆ
ಮುಕ್ತ-ನಾ-ಗುತ್ತಿ
ಮುಕ್ತ-ನಾದ
ಮುಕ್ತ-ನಾ-ದ-ವ-ನಿಗೂ
ಮುಕ್ತ-ರಾ-ಗು-ತ್ತಾರೆ
ಮುಕ್ತ-ವಾಗಿ
ಮುಕ್ತಾ-ವಸ್ಥೆ
ಮುಕ್ತಾ-ವ-ಸ್ಥೆ-ಯನ್ನೂ
ಮುಕ್ತಾ-ವ-ಸ್ಥೆ-ಯ-ಲ್ಲಿ-ದ್ದು-ದ-ರಿಂದ
ಮುಕ್ತಿ
ಮುಕ್ತಿಗೆ
ಮುಕ್ತಿ-ನಾ-ಯಕ
ಮುಕ್ತಿ-ಪ-ದ-ವಿಗೆ
ಮುಕ್ತಿ-ಪ-ದ-ವಿ-ಯನ್ನು
ಮುಕ್ತಿ-ಪ್ರ-ದ-ವಾ-ಗು-ತ್ತವೆ
ಮುಕ್ತಿ-ಮಾರ್ಗ
ಮುಕ್ತಿ-ಮಾ-ರ್ಗ-ವನ್ನು
ಮುಕ್ತಿ-ಮಾ-ರ್ಗ-ವನ್ನೆ
ಮುಕ್ತಿಯ
ಮುಕ್ತಿ-ಯನ್ನು
ಮುಕ್ತಿ-ಯಾ-ಗ-ಬೇ-ಕಾ-ದರೆ
ಮುಕ್ತಿಯೂ
ಮುಕ್ತಿಯೆ
ಮುಕ್ತಿ-ಯೆಂ-ದರೆ
ಮುಕ್ತಿಯೇ
ಮುಖ
ಮುಖ-ಕ-ಮ-ಲ-ದಿಂ-ದಲೂ
ಮುಖ-ಕ-ಮ-ಲ-ವನ್ನು
ಮುಖಕ್ಕೆ
ಮುಖ-ಗ-ಳಾ-ಗು-ತ್ತವೆ
ಮುಖ-ಗಳು
ಮುಖದ
ಮುಖ-ದಂತೆ
ಮುಖ-ದತ್ತ
ಮುಖ-ದ-ಮೇಲೆ
ಮುಖ-ದಲ್ಲಿ
ಮುಖ-ದಿಂದ
ಮುಖ-ಭಾ-ವ-ದಿಂ-ದಲೇ
ಮುಖ-ರಾ-ದರು
ಮುಖ-ವನ್ನು
ಮುಖ-ವಾಗಿ
ಮುಖವು
ಮುಖವೂ
ಮುಖ-ವೆಲ್ಲ
ಮುಖವೇ
ಮುಖಾಯ
ಮುಖ್ಯ
ಮುಖ್ಯ-ಕಾ-ರ್ಯ-ವೆಂದು
ಮುಖ್ಯ-ತ-ಮಾಯ
ಮುಖ್ಯ-ಧರ್ಮ
ಮುಖ್ಯ-ನಾ-ದ-ವನು
ಮುಖ್ಯ-ನೆ-ನಿ-ಸಿದ್ದ
ಮುಖ್ಯ-ರಾ-ದ-ವ-ರನ್ನು
ಮುಖ್ಯ-ರಿಗೆ
ಮುಖ್ಯ-ವಾಗಿ
ಮುಖ್ಯ-ವಾ-ಗಿತ್ತು
ಮುಖ್ಯ-ವಾದ
ಮುಖ್ಯ-ವಾ-ದವು
ಮುಖ್ಯ-ವಿ-ಹಾ-ರ-ವ-ಸ್ತು-ವಾದ
ಮುಖ್ಯವೆ
ಮುಖ್ಯ-ಸ್ಥ-ನನ್ನು
ಮುಗ-ಳ್ನ-ಗುತ್ತಾ
ಮುಗ-ಳ್ನ-ಗೆ-ಯನ್ನು
ಮುಗಿ
ಮುಗಿದ
ಮುಗಿದಂ
ಮುಗಿ-ದಂ-ತಾ-ಗಿತ್ತು
ಮುಗಿ-ದಂ-ತಾ-ದು-ದ-ರಿಂದ
ಮುಗಿ-ದ-ಮೇಲೆ
ಮುಗಿ-ದರೆ
ಮುಗಿದು
ಮುಗಿ-ದು-ಕೊಂಡು
ಮುಗಿ-ದು-ಹೋಗಿ
ಮುಗಿದೇ
ಮುಗಿ-ದೊ-ಡ-ನೆಯೇ
ಮುಗಿ-ಯ-ದೆಂ-ದು-ಕೊಂಡ
ಮುಗಿ-ಯ-ಲಿಲ್ಲ
ಮುಗಿ-ಯಿತು
ಮುಗಿ-ಯಿ-ತೆಂದು
ಮುಗಿಯು
ಮುಗಿ-ಯು-ತ್ತಲೆ
ಮುಗಿ-ಯು-ತ್ತಿ-ದ್ದಂತೆ
ಮುಗಿ-ಯುವ
ಮುಗಿ-ಯು-ವ-ವ-ರೆಗೂ
ಮುಗಿ-ಯು-ವ-ವ-ರೆಗೆ
ಮುಗಿ-ಯು-ವ-ಷ್ಟ-ರಲ್ಲಿ
ಮುಗಿ-ಲನ್ನು
ಮುಗಿ-ಲಿ-ನಂ-ತಿ-ರುವ
ಮುಗಿ-ಲೆ-ತ್ತ-ರಕ್ಕೆ
ಮುಗಿಸ
ಮುಗಿ-ಸ-ಬೇಕು
ಮುಗಿ-ಸ-ಲೇ-ಬೇಕು
ಮುಗಿಸಿ
ಮುಗಿ-ಸಿ-ಕೊಂಡು
ಮುಗಿ-ಸಿದ
ಮುಗಿ-ಸಿ-ದಂ-ತಾ-ಯಿತು
ಮುಗಿ-ಸಿ-ದನು
ಮುಗಿ-ಸಿ-ದ-ಮೇಲೆ
ಮುಗಿ-ಸಿ-ದರು
ಮುಗಿ-ಸಿ-ದರೆ
ಮುಗಿ-ಸಿ-ದೆಯಾ
ಮುಗಿ-ಸಿ-ದೆ-ಯೆಂ-ದರೆ
ಮುಗಿ-ಸಿ-ಬಿ-ಡ-ಬೇ-ಕೆಂಬ
ಮುಗಿ-ಸಿ-ಬಿಡು
ಮುಗಿ-ಸಿ-ರು-ವ-ನೆಂ-ಬುದು
ಮುಗಿ-ಸಿ-ರು-ವಿರಿ
ಮುಗಿ-ಸಿ-ರುವೆ
ಮುಗಿ-ಸು-ತ್ತಿ-ದ್ದಂ-ತೆಯೆ
ಮುಗಿ-ಸುವ
ಮುಗಿ-ಸು-ವನು
ಮುಗು
ಮುಗುಳು
ಮುಗು-ಳು-ನಗೆ
ಮುಗುಳ್
ಮುಗುಳ್ನ
ಮುಗು-ಳ್ನಗು
ಮುಗು-ಳ್ನ-ಗುತ್ತಾ
ಮುಗು-ಳ್ನ-ಗು-ವನ್ನು
ಮುಗು-ಳ್ನಗೆ
ಮುಗು-ಳ್ನ-ಗೆಯ
ಮುಗು-ಳ್ನ-ಗೆ-ಯನ್ನು
ಮುಗು-ಳ್ನ-ಗೆ-ಯಿಂದ
ಮುಗು-ಳ್ನ-ಗೆ-ಯಿಂ-ದಲೂ
ಮುಗು-ಳ್ನ-ಗೆ-ಯೊ-ಡನೆ
ಮುಗು-ವಿನ
ಮುಗ್ಧ-ನಾಗಿ
ಮುಗ್ಧ-ನಾದ
ಮುಗ್ಧ-ರಾದ
ಮುಗ್ಧ-ವಾಗಿ
ಮುಗ್ಧೆ
ಮುಗ್ಧೆ-ಯ-ರಾದ
ಮುಚು-ಕುಂದ
ಮುಚು-ಕುಂ-ದ-ಎಂಬ
ಮುಚು-ಕುಂ-ದ-ನನ್ನು
ಮುಚು-ಕುಂ-ದನು
ಮುಚ್ಚ
ಮುಚ್ಚ-ಲಾ-ರದೆ
ಮುಚ್ಚಿ
ಮುಚ್ಚಿ-ಕೊಂ-ಡನು
ಮುಚ್ಚಿ-ಕೊಂಡು
ಮುಚ್ಚಿ-ಕೊಂ-ಡುವು
ಮುಚ್ಚಿ-ಕೊ-ಳ್ಳಲು
ಮುಚ್ಚಿ-ಕೊ-ಳ್ಳಿರಿ
ಮುಚ್ಚಿ-ಕೊಳ್ಳು
ಮುಚ್ಚಿಟ್ಟ
ಮುಚ್ಚಿಟ್ಟು
ಮುಚ್ಚಿದ
ಮುಚ್ಚಿ-ದ್ದರೆ
ಮುಚ್ಚಿ-ಬಿ-ಟ್ಟಿದೆ
ಮುಚ್ಚಿ-ರುವ
ಮುಚ್ಚಿ-ಸು-ವು-ದ-ಕ್ಕಾಗಿ
ಮುಚ್ಚಿ-ಹಾ-ಕಿತು
ಮುಚ್ಚಿ-ಹಾಕು
ಮುಚ್ಚಿ-ಹಾ-ಕು-ತ್ತಿ-ದ್ದನು
ಮುಚ್ಚಿ-ಹೋ-ಗು-ವವು
ಮುಚ್ಚಿ-ಹೋ-ದವು
ಮುಚ್ಚು-ತ್ತಲೆ
ಮುಚ್ಚುವ
ಮುಟ್ಟ-ಬೇ-ಕೆನ್ನು
ಮುಟ್ಟ-ಬೇಡ
ಮುಟ್ಟಲು
ಮುಟ್ಟಿ
ಮುಟ್ಟಿ-ಕೊಂಡೇ
ಮುಟ್ಟಿತು
ಮುಟ್ಟಿದ
ಮುಟ್ಟಿ-ದಂ-ತಾಗು
ಮುಟ್ಟಿ-ದರೆ
ಮುಟ್ಟಿ-ದ-ರೆಲ್ಲಿ
ಮುಟ್ಟಿ-ದು-ದನ್ನು
ಮುಟ್ಟಿ-ಸಾಟ
ಮುಟ್ಟಿ-ಸಿ-ದರು
ಮುಟ್ಟಿ-ಸಿ-ದು-ದಾ-ಯಿತು
ಮುಟ್ಟಿ-ಸು-ತ್ತದೆ
ಮುಟ್ಟು
ಮುಟ್ಟು-ತ್ತಲೆ
ಮುಟ್ಟು-ತ್ತಿದೆ
ಮುಟ್ಟುವ
ಮುಟ್ಟು-ವಂ-ತಹ
ಮುಟ್ಟು-ವಷ್ಟು
ಮುಡಿ
ಮುಡಿದ
ಮುಡಿ-ದಿ-ದ್ದಾಳೆ
ಮುಡಿದು
ಮುಡಿ-ಪಾ-ಗಲಿ
ಮುಡಿ-ಯನ್ನು
ಮುಡಿ-ಯ-ಲ್ಲಿದ್ದ
ಮುಡಿ-ಯುತ್ತ
ಮುಡಿಸಿ
ಮುಡಿ-ಸಿ-ರ-ಬೇಕು
ಮುಡಿ-ಸುತ್ತ
ಮುಡುವಿ
ಮುತ್ತ
ಮುತ್ತಿ-ಕೊಂ-ಡರು
ಮುತ್ತಿ-ಕೊಂ-ಡಿ-ರುವ
ಮುತ್ತಿ-ಕೊಂಡು
ಮುತ್ತಿ-ಕ್ಕಿ-ದಳು
ಮುತ್ತಿಗೆ
ಮುತ್ತಿ-ಗೆ-ಹಾ-ಕಿ-ದನು
ಮುತ್ತಿ-ಟ್ಟು-ಕೊ-ಳ್ಳು-ವಳು
ಮುತ್ತಿ-ಡು-ವ-ವ-ನಂತೆ
ಮುತ್ತಿತು
ಮುತ್ತಿ-ತ್ತನು
ಮುತ್ತಿ-ದನು
ಮುತ್ತಿ-ದರು
ಮುತ್ತಿ-ದರೆ
ಮುತ್ತಿನ
ಮುತ್ತೈದೆ
ಮುತ್ತೈ-ದೆ-ಯ-ರನ್ನು
ಮುತ್ತೈ-ದೆ-ಯ-ರ-ನ್ನೆಲ್ಲ
ಮುತ್ತೈ-ದೆ-ಯ-ರಿಗೆ
ಮುತ್ತೈ-ದೆ-ಯ-ರಿ-ಗೆಲ್ಲ
ಮುತ್ತೈ-ದೆ-ಯರು
ಮುದಿ
ಮುದಿ-ಗಂ-ಡನ
ಮುದಿ-ದಾವು
ಮುದುಕ
ಮುದು-ಕನ
ಮುದು-ಕ-ನನ್ನು
ಮುದು-ಕ-ನಾಗಿ
ಮುದು-ಕ-ನಾದ
ಮುದು-ಕ-ನಾ-ದಾಗ
ಮುದು-ಕನೆ
ಮುದು-ಕ-ನೆಂದು
ಮುದು-ಕ-ರನ್ನೂ
ಮುದು-ಕ-ರಾದ
ಮುದು-ಕಿಯ
ಮುದ್ದಾ
ಮುದ್ದಾ-ಡಿ-ದನು
ಮುದ್ದಾ-ಡಿ-ದರು
ಮುದ್ದಾ-ಡು-ತ್ತಿದ್ದ
ಮುದ್ದಾ-ಡು-ತ್ತಿ-ದ್ದರು
ಮುದ್ದಾ-ಡು-ವನು
ಮುದ್ದಾ-ಡು-ವುದು
ಮುದ್ದಾದ
ಮುದ್ದಿ-ಕ್ಕು-ವರು
ಮುದ್ದಿನ
ಮುದ್ದಿ-ನಿಂದ
ಮುದ್ದು
ಮುದ್ದು-ಕು-ಮಾರ
ಮುದ್ದು-ಕೃ-ಷ್ಣ-ನನ್ನು
ಮುದ್ದು-ಮಗ
ಮುದ್ದು-ಮ-ಗ-ನಿಗೆ
ಮುದ್ದು-ಮ-ಗಳು
ಮುದ್ದು-ಮ-ಗ-ಳೆ-ನಿ-ಸಿ-ಕೊಂ-ಡಿದ್ದ
ಮುದ್ದು-ಮರಿ
ಮುದ್ದು-ಮ-ರಿಯ
ಮುದ್ದು-ಮಾ-ತು-ಗಳನ್ನು
ಮುದ್ದು-ಮು-ಖ-ವನ್ನು
ಮುದ್ದು-ಮೊ-ಗ-ಗಳ
ಮುದ್ದೆ-ಗ-ಳಂ-ತಿದ್ದ
ಮುದ್ದೆ-ಯಂತೆ
ಮುದ್ದೆ-ಯಾದ
ಮುದ್ದೆಯೋ
ಮುದ್ರ-ಣ-ಗಳನ್ನು
ಮುಧು-ರಾ-ಪು-ರಿಯ
ಮುನಿ
ಮುನಿ-ಗಳನ್ನು
ಮುನಿ-ಗಳಿಂದ
ಮುನಿ-ಗ-ಳಿಗೆ
ಮುನಿ-ಗಳೂ
ಮುನಿ-ಗಳೆ
ಮುನಿಗೆ
ಮುನಿ-ಜ-ನರ
ಮುನಿಯ
ಮುನಿ-ಯಿಂದ
ಮುನಿಯು
ಮುನಿ-ವೃಂ-ದ-ಹೃ-ದಿ-ಸ್ಥ-ಪದಂ
ಮುನೀಂದ್ರ
ಮುನೀ-ಶ್ವರ
ಮುನೀ-ಶ್ವರಾ
ಮುನ್ನ
ಮುನ್ನ-ಡೆದು
ಮುನ್ನವೆ
ಮುನ್ನವೇ
ಮುನ್ನಾ-ದಿನ
ಮುನ್ನುಗ್ಗಿ
ಮುನ್ನುಡಿ
ಮುನ್ನು-ಡಿಯ
ಮುನ್ನು-ಡಿ-ಯನ್ನು
ಮುನ್ನೂರ
ಮುನ್ನೂ-ರ-ರ-ವತ್ತು
ಮುನ್ನೂರು
ಮುಪ್ಪನ್ನು
ಮುಪ್ಪಿನ
ಮುಪ್ಪಿ-ನ-ಕಾ-ಲ-ದಲ್ಲಿ
ಮುಪ್ಪಿ-ನ-ಮು-ದು-ಕ-ನಾ-ಗಿ-ರಲಿ
ಮುಪ್ಪಿ-ನಲ್ಲಿ
ಮುಪ್ಪಿ-ನಿಂದ
ಮುಪ್ಪಿಲ್ಲ
ಮುಪ್ಪು
ಮುಮು-ಕ್ಷ-ಗಳ
ಮುಮು-ಕ್ಷು-ಗ-ಳಿಗೆ
ಮುಮು-ಕ್ಷುವೇ
ಮುರ-ನೆಂಬ
ಮುರ-ಪಾಶ
ಮುರ-ಳೀ-ಗಾ-ನ-ಲೋಲ
ಮುರ-ಳೀ-ಧರ
ಮುರಾ-ಸು-ರನು
ಮುರಾ-ಸು-ರ-ನೆಂಬ
ಮುರಿದ
ಮುರಿ-ದ-ನೆಂ-ಬು-ದನ್ನು
ಮುರಿದು
ಮುರಿ-ದು-ಬಿತ್ತು
ಮುರಿ-ದು-ಬಿ-ದ್ದವು
ಮುರಿ-ದು-ಬಿ-ದ್ದೊ-ಡ-ನೆಯೇ
ಮುರಿ-ದೇ-ಬಿ-ಟ್ಟನು
ಮುರಿ-ಯ-ಬೇ-ಕಾ-ದೀತು
ಮುರಿ-ಯ-ಬೇಕು
ಮುರಿ-ಯಲು
ಮುರಿ-ಯಿತು
ಮುರಿ-ಯು-ತ್ತಾನೆ
ಮುರಿ-ಯು-ವಂತೆ
ಮುರಿ-ಯು-ವು-ದ-ಕ್ಕಾ-ಗಿಯೆ
ಮುರಿ-ಯು-ವು-ದ-ಕ್ಕಾಯೆ
ಮುರಿ-ಯು-ವುದೋ
ಮುರುಂಡ
ಮುಳು-ಗ-ದಂತೆ
ಮುಳುಗಿ
ಮುಳು-ಗಿತ್ತು
ಮುಳು-ಗಿದ
ಮುಳು-ಗಿ-ದರೂ
ಮುಳು-ಗಿದ್ದ
ಮುಳು-ಗಿ-ದ್ದರು
ಮುಳು-ಗಿ-ದ್ದ-ವನು
ಮುಳು-ಗಿ-ರ-ಬ-ಲ್ಲ-ವ-ನಾ-ದರೂ
ಮುಳು-ಗಿ-ರು-ತ್ತಾರೆ
ಮುಳು-ಗಿ-ರುವ
ಮುಳು-ಗಿ-ರು-ವ-ವ-ರನ್ನು
ಮುಳು-ಗಿ-ರು-ವುದನ್ನು
ಮುಳು-ಗಿಸಿ
ಮುಳು-ಗಿ-ಹೋ-ಗಿದೆ
ಮುಳು-ಗಿ-ಹೋ-ಗಿ-ದ್ದಾಗ
ಮುಳು-ಗಿ-ಹೋಗು
ಮುಳು-ಗಿ-ಹೋ-ಗು-ತ್ತಲೆ
ಮುಳು-ಗಿ-ಹೋ-ಗು-ತ್ತವೆ
ಮುಳು-ಗಿ-ಹೋ-ಗು-ತ್ತಿ-ದ್ದಾಗ
ಮುಳು-ಗಿ-ಹೋ-ಗು-ವ-ವ-ರನ್ನು
ಮುಳು-ಗಿ-ಹೋ-ದನು
ಮುಳು-ಗಿ-ಹೋ-ದುದೇ
ಮುಳು-ಗಿ-ಹೋ-ಯಿತು
ಮುಳುಗು
ಮುಳು-ಗು-ತ್ತಿದ್ದ
ಮುಳು-ಗು-ತ್ತಿ-ದ್ದಾನೆ
ಮುಳು-ಗುವ
ಮುಳು-ಗು-ವ-ವ-ನಿಗೆ
ಮುಳ್ಳನ್ನು
ಮುಳ್ಳಿ-ನಿಂದ
ಮುಳ್ಳಿ-ನಿಂ-ದಲೇ
ಮುಳ್ಳು
ಮುಳ್ಳು-ಕ-ಳ್ಳೆ-ಗ-ಳ-ನ್ನಾ-ಗಲಿ
ಮುಷ್ಕಾ-ರ-ವೆಂಬ
ಮುಷ್ಟಿ
ಮುಷ್ಟಿಕ
ಮುಷ್ಠಿ-ಕನ
ಮುಷ್ಠಿ-ಕ-ನೊ-ಡನೆ
ಮುಷ್ಠಿ-ಕರೂ
ಮುಷ್ಠಿ-ಯಿಂದ
ಮುಸ-ಲಾ-ಯು-ಧದ
ಮುಸ-ಲಾ-ಯು-ಧ-ದೊ-ಡನೆ
ಮುಸು-ಕ-ದಂತೆ
ಮುಸು-ಕ-ಲಾ-ರದು
ಮುಸುಕಿ
ಮುಸು-ಕಿ-ಕೊ-ಳ್ಳು-ವಂತೆ
ಮುಸು-ಕಿ-ಕೊ-ಳ್ಳು-ವು-ದಿಲ್ಲ
ಮುಸು-ಕಿತು
ಮುಸು-ಕಿತ್ತು
ಮುಸು-ಕಿದ
ಮುಸು-ಕಿ-ದಂ-ತಾ-ಗಿತ್ತು
ಮುಸು-ಕಿ-ನಲ್ಲಿ
ಮುಸು-ಕುತ್ತಾ
ಮುಸು-ಗಿ-ರುವ
ಮುಹುಃ
ಮುಹೂರ್ತ
ಮುಹೂ-ರ್ತ-ಕಾಲ
ಮುಹೂ-ರ್ತ-ದಲ್ಲಿ
ಮುಹೂ-ರ್ತ-ವನ್ನು
ಮುಹೂ-ರ್ತವು
ಮೂಕ-ನಂತೆ
ಮೂಕ-ರಂತೆ
ಮೂಗ-ನಂತೆ
ಮೂಗಿಗೆ
ಮೂಗಿನ
ಮೂಗಿ-ನಿಂದ
ಮೂಗು
ಮೂಗು-ಇ-ತ್ಯಾದಿ
ಮೂಗು-ಗಳನ್ನು
ಮೂಗು-ದಾರ
ಮೂಗು-ದಾ-ರ-ವನ್ನು
ಮೂಗು-ದಾ-ರ-ಹಾ-ಕಿದ
ಮೂಗೊ
ಮೂಡಣ
ಮೂಡ-ದಂತೆ
ಮೂಡಲ
ಮೂಡಿ
ಮೂಡಿತು
ಮೂಡಿ-ತೆಂ-ದರೆ
ಮೂಡಿದ
ಮೂಡಿ-ದಂತಾ
ಮೂಡಿ-ದಂತೆ
ಮೂಡಿ-ದವು
ಮೂಡಿದ್ದ
ಮೂಡಿ-ದ್ದುವು
ಮೂಡಿ-ಬಂತು
ಮೂಡಿ-ಬಂದ
ಮೂಡಿ-ಬಂ-ದರು
ಮೂಡಿ-ಬ-ರುವ
ಮೂಡಿ-ರು-ವಂತೆ
ಮೂಡಿ-ಸಿದ
ಮೂಡು-ತ್ತದೆ
ಮೂಡು-ತ್ತಲೆ
ಮೂಡು-ತ್ತವೆ
ಮೂಡುತ್ತಾ
ಮೂಡುವ
ಮೂಡು-ವು-ದಕ್ಕೂ
ಮೂಡು-ವುದು
ಮೂಢ
ಮೂಢ-ತನ
ಮೂಢ-ತ-ನ-ಕ್ಕಾಗಿ
ಮೂಢನ
ಮೂಢ-ನಂತೆ
ಮೂಢ-ನಂ-ಬಿಕೆ
ಮೂಢ-ನಾದ
ಮೂಢ-ನಿ-ಗೇನು
ಮೂಢನೂ
ಮೂಢ-ಭಕ್ತ
ಮೂಢ-ರಾದ
ಮೂಢ-ರಿಗೂ
ಮೂಢ-ಳಾದ
ಮೂತಿ-ಗಳನ್ನು
ಮೂತಿ-ಯ-ಮೇಲೆ
ಮೂದ-ಲಿಸಿ
ಮೂದ-ಲಿ-ಸಿದ
ಮೂದಲೆ
ಮೂನ್ನೂರು
ಮೂರ
ಮೂರಡಿ
ಮೂರ-ಡಿ-ಗಳಿಂದ
ಮೂರ-ನೆಯ
ಮೂರ-ನೆ-ಯ-ದಾದ
ಮೂರ-ರಲ್ಲಿ
ಮೂರ-ರ-ಲ್ಲಿಯೂ
ಮೂರಾಗಿ
ಮೂರು
ಮೂರು-ಕೋಟಿ
ಮೂರು-ತಿಗೆ
ಮೂರು-ದಿ-ನ-ಗಳು
ಮೂರು-ಲಕ್ಷ
ಮೂರು-ಲೋಕ
ಮೂರು-ಲೋ-ಕ-ಗಳ
ಮೂರು-ಲೋ-ಕ-ಗಳನ್ನೂ
ಮೂರು-ಲೋ-ಕದ
ಮೂರು-ವಿಧ
ಮೂರು-ಸಲ
ಮೂರು-ಹೊತ್ತು
ಮೂರೂ
ಮೂರೇ
ಮೂರ್ಖ
ಮೂರ್ಖ-ನತ್ತ
ಮೂರ್ಛಿ-ತ-ನಾ-ದನು
ಮೂರ್ಛಿ-ತ-ಳಾ-ದಳು
ಮೂರ್ಛೆ
ಮೂರ್ಛೆ-ಗೊಂಡು
ಮೂರ್ಛೆ-ಬಿದ್ದ
ಮೂರ್ಛೆ-ಯಿಂದ
ಮೂರ್ಛೆ-ಹೋ-ಗು-ತ್ತಲೆ
ಮೂರ್ಛೆ-ಹೋ-ದರು
ಮೂರ್ತಿ
ಮೂರ್ತಿ-ಗ-ಳಿಗೆ
ಮೂರ್ತಿಗೆ
ಮೂರ್ತಿ-ದೇ-ವಿ-ಯಲ್ಲಿ
ಮೂರ್ತಿಯ
ಮೂರ್ತಿ-ಯನ್ನು
ಮೂರ್ತಿ-ಯನ್ನೊ
ಮೂರ್ತಿ-ಯಾ-ಗಿತ್ತು
ಮೂರ್ತಿ-ಯಾದ
ಮೂರ್ತಿಯು
ಮೂರ್ತಿಯೂ
ಮೂರ್ತಿ-ಯೊಂದು
ಮೂರ್ತಿ-ಯೊಬ್ಬ
ಮೂರ್ತಿ-ವೆ-ತ್ತಂತೆ
ಮೂಲ
ಮೂಲಕ
ಮೂಲ-ಕ-ವಾದ
ಮೂಲ-ಕವೆ
ಮೂಲ-ಕವೇ
ಮೂಲ-ಕಾ-ರಣ
ಮೂಲ-ಕಾರಣ-ನಾಗಿ
ಮೂಲ-ಕಾರಣ-ನಾದ
ಮೂಲ-ಕಾರಣನು
ಮೂಲ-ಕಾರಣ-ರಾ-ದರು
ಮೂಲ-ಕಾರಣ-ವೇನು
ಮೂಲಕ್ಕೆ
ಮೂಲ-ಗಳು
ಮೂಲ-ತ-ತ್ವದ
ಮೂಲದ
ಮೂಲ-ದಲ್ಲಿ
ಮೂಲ-ಪು-ರು-ಷ-ನಾ-ಗು-ವಂತೆ
ಮೂಲ-ಪು-ರು-ಷ-ನಾ-ಗು-ವನು
ಮೂಲ-ಪು-ರು-ಷ-ನಾದ
ಮೂಲ-ಭೂ-ತ-ವಾದ
ಮೂಲ-ವನ್ನು
ಮೂಲ-ವ-ಲ್ಲವೆ
ಮೂಲ-ವಸ್ತು
ಮೂಲ-ವಾ-ದುದು
ಮೂಲಾ-ಧಾ-ರ-ದ-ಲ್ಲಿ-ರುವ
ಮೂಲಿ-ಕೆ-ಗಳನ್ನೂ
ಮೂಲೆ
ಮೂಲೆ-ಗಳನ್ನೂ
ಮೂಲೆ-ಗ-ಳ-ಲ್ಲಿಯೂ
ಮೂಲೆ-ಯನ್ನೂ
ಮೂಲೆ-ಯ-ಲ್ಲಿದ್ದ
ಮೂಳೆ
ಮೂಳೆ-ಗಳಿಂದ
ಮೂವ-ತ್ತಾರು
ಮೂವತ್ತು
ಮೂವ-ತ್ತೆ-ರಡು
ಮೂವರ
ಮೂವ-ರಂತೆ
ಮೂವ-ರಲ್ಲಿ
ಮೂವ-ರಿಗೂ
ಮೂವರು
ಮೂವರೂ
ಮೂಸಿ
ಮೂಸಿ-ನೋ-ಡಿದ
ಮೂಸು-ತ್ತ-ದೆಯೆ
ಮೃಕಂಡು
ಮೃಗ
ಮೃಗ-ಗಳ
ಮೃಗ-ಗಳನ್ನು
ಮೃಗ-ಗಳಲ್ಲಿ
ಮೃಗ-ಗಳು
ಮೃಗ-ಗ-ಳೆಲ್ಲ
ಮೃಗ-ಚ-ರ್ಮ-ಗಳು
ಮೃಗದ
ಮೃಗ-ದಂ-ತಿ-ರುವ
ಮೃಗ-ದಿಂದ
ಮೃಗ-ಪ-ಕ್ಷಿ-ಗಳನ್ನೆಲ್ಲ
ಮೃಗ-ಯು-ರಿವ
ಮೃಗ-ವನ್ನು
ಮೃಜ್ಜಾ-ತಿ-ಸ್ತಸ್ಮೈ
ಮೃಣ್ಮ-ಯೇ-ಷ್ವಿವ
ಮೃತ-ವೆಂ-ದು-ಕೊಂಡು
ಮೃತ-ಸಂ-ಜೀ-ವಿ-ಕ-ಯಾ-ನಯಾ
ಮೃತಿ
ಮೃತಿ-ಜ-ನ್ಮ-ಜ-ರಾ-ಶ-ಮನಂ
ಮೃತ್ತಿ-ಕೆ-ಯಂತೆ
ಮೃತ್ಯು
ಮೃತ್ಯು-ಭ-ಯ-ವನ್ನು
ಮೃತ್ಯು-ಭ-ಯ-ವಿ-ರ-ಕೂ-ಡದು
ಮೃತ್ಯು-ಭ-ಯ-ವಿಲ್ಲ
ಮೃತ್ಯು-ವನ್ನು
ಮೃತ್ಯು-ವಾ-ಗ-ಬೇಕೆ
ಮೃತ್ಯು-ವಾಗಿ
ಮೃತ್ಯು-ವಾ-ಗು-ತ್ತಾ-ನೆಯೊ
ಮೃತ್ಯು-ವಾದ
ಮೃತ್ಯು-ವಾ-ಯಿತು
ಮೃತ್ಯು-ವಿಗೂ
ಮೃತ್ಯು-ವಿಗೆ
ಮೃತ್ಯು-ವಿನ
ಮೃತ್ಯು-ವಿ-ನಂ-ತಿದ್ದ
ಮೃತ್ಯು-ವಿ-ನಂತೆ
ಮೃತ್ಯು-ವಿ-ನೊ-ಡನೆ
ಮೃತ್ಯುವು
ಮೃತ್ಯುವೂ
ಮೃತ್ಯು-ವೆಂದು
ಮೃತ್ಯು-ವೇನೂ
ಮೃತ್ಯು-ಸ್ವ-ರೂ-ಪ-ರಾ-ದ-ವರು
ಮೃತ್ಯು-ಸ್ವ-ರೂ-ಪರು
ಮೃತ್ಯೋಶ್ಚ
ಮೃದಂ-ಗ-ಗಳ
ಮೃದು
ಮೃದು-ದೇ-ಹಕ್ಕೆ
ಮೃದು-ನುಡಿ
ಮೃದು-ನು-ಡಿ-ಗಳಿಂದ
ಮೃದು-ನು-ಡಿ-ಗ-ಳಿಂ-ದಲೂ
ಮೃದು-ಪಾ-ದ-ಗಳ
ಮೃದು-ಮ-ಧುರ
ಮೃದು-ಮ-ಧು-ರ-ವಾಗಿ
ಮೃದು-ಮ-ಧು-ರ-ವಾದ
ಮೃದು-ವಾದ
ಮೃದು-ಶ-ರೀರ
ಮೃಷ್ಟಾ
ಮೃಷ್ಟಾನ್ನ
ಮೃಷ್ಟಾ-ನ್ನ-ವನ್ನೂ
ಮೃಷ್ಟಾ-ನ್ನವೋ
ಮೆಗಾ-ಸ್ತ-ನೀ-ಸನು
ಮೆಚ್ಚದೆ
ಮೆಚ್ಚ-ಲಿಲ್ಲ
ಮೆಚ್ಚಿ
ಮೆಚ್ಚಿ-ಕೊಂಡು
ಮೆಚ್ಚಿಗೆ
ಮೆಚ್ಚಿ-ಗೆ-ಯಾ-ಗದ
ಮೆಚ್ಚಿ-ಗೆ-ಯಾ-ದರೂ
ಮೆಚ್ಚಿದ
ಮೆಚ್ಚಿದೆ
ಮೆಚ್ಚಿ-ದ್ದೇನೆ
ಮೆಚ್ಚಿ-ಸ-ಬೇಕು
ಮೆಚ್ಚಿಸಿ
ಮೆಚ್ಚಿ-ಸಿ-ದರೆ
ಮೆಚ್ಚು-ಗೆ-ಯಾ-ಗುವ
ಮೆಚ್ಚು-ವಂತೆ
ಮೆಚ್ಚು-ವು-ದಿಲ್ಲ
ಮೆಟ್ಟ-ಲಾಗಿ
ಮೆಟ್ಟಲು
ಮೆಟ್ಟಿ
ಮೆಟ್ಟಿಂ-ಗಾ-ಲಿ-ನಿಂದ
ಮೆಟ್ಟಿ-ಕೊಂಡು
ಮೆಟ್ಟಿದ
ಮೆಟ್ಟಿ-ದವ
ಮೆಟ್ಟಿ-ಬಿದ್ದು
ಮೆಟ್ಟಿ-ಬೀ-ಳು-ವನು
ಮೆತ್ತ-ನೆಯ
ಮೆತ್ತಿ-ಕೊ-ಳ್ಳು-ತ್ತವೆ
ಮೆತ್ತು-ತ್ತಿ-ದ್ದಾರೆ
ಮೆದ್ದು-ಅ-ವನ್ನು
ಮೆರ-ಗನ್ನು
ಮೆರ-ವ-ಣಿಗೆ
ಮೆರ-ವ-ಣಿ-ಗೆ-ಯಲ್ಲಿ
ಮೆರೆ
ಮೆರೆ-ದಿ-ದ್ದಾನೆ
ಮೆರೆ-ಯುತ್ತಾ
ಮೆರೆ-ಯು-ತ್ತಿತ್ತು
ಮೆರೆ-ಯು-ತ್ತಿ-ರು-ವೆ-ಯೆಂದು
ಮೆರೆ-ಯು-ವಂತೆ
ಮೆರೆ-ಯು-ವ-ವ-ರನ್ನು
ಮೆಲುಕು
ಮೆಲು-ಕು-ಹಾ-ಕುತ್ತಾ
ಮೆಲೆ
ಮೆಲ್ಲ
ಮೆಲ್ಲಗೆ
ಮೆಲ್ಲ-ಗೆ-ಹೋಗಿ
ಮೆಲ್ಲ-ಡಿ-ಗೆಲ್ಲಿ
ಮೆಲ್ಲನೆ
ಮೆಲ್ಲ-ಮೆ-ಲ್ಲನೆ
ಮೆಲ್ಲುತ್ತಾ
ಮೆಳೆ
ಮೆಳೆ-ಯನ್ನು
ಮೆಳೆ-ಯ-ನ್ನೆಲ್ಲ
ಮೇ
ಮೇಂಗ
ಮೇಕೆ
ಮೇಕೆ-ಗಳು
ಮೇಕೆ-ಗ-ಳೊ-ಡನೆ
ಮೇಕೆಯ
ಮೇಕೆ-ಯೊ-ಡನೆ
ಮೇಘ
ಮೇಘ-ಗಳ
ಮೇಘ-ಗಳನ್ನು
ಮೇಘ-ಗಳಿಂದ
ಮೇಘ-ಗಳೆ
ಮೇಘ-ಗಳೇ
ಮೇಘದ
ಮೇಘ-ದಂತೆ
ಮೇಘ-ರಾಜ
ಮೇಘ-ವನ್ನು
ಮೇಘ-ಶ್ಯಾ-ಮ-ವಾದ
ಮೇಧಾ-ತಿ-ಥಿ-ಯೆಂಬ
ಮೇನ-ಕೆ-ಯಲ್ಲಿ
ಮೇನೆ-ಯಲ್ಲಿ
ಮೇಯಿ
ಮೇಯಿಸ
ಮೇಯಿ-ಸಲು
ಮೇಯಿಸಿ
ಮೇಯಿ-ಸುತ್ತಾ
ಮೇಯಿ-ಸುವ
ಮೇಯಿ-ಸು-ವ-ವನೂ
ಮೇಯಿ-ಸು-ವು-ದ-ಕ್ಕಾಗಿ
ಮೇಯು
ಮೇಯು-ತ್ತಿ-ದ್ದರೆ
ಮೇಯು-ವ-ವನೂ
ಮೇಯು-ವು-ದಕ್ಕೆ
ಮೇರು
ಮೇರು-ಗಿ-ರಿ-ಯನ್ನು
ಮೇರು-ದೇ-ವಿಯ
ಮೇರು-ದೇ-ವಿ-ಯೊ-ಡನೆ
ಮೇರು-ಪ-ರ್ವತ
ಮೇರು-ಪ-ರ್ವ-ತ-ವನ್ನು
ಮೇರು-ಪ-ರ್ವ-ತವು
ಮೇರು-ವಿನ
ಮೇರೆ
ಮೇರೆ-ದಪ್ಪಿ
ಮೇಲ
ಮೇಲಕ್ಕೆ
ಮೇಲ-ಕ್ಕೆ-ತ್ತದೆ
ಮೇಲ-ಕ್ಕೆತ್ತಿ
ಮೇಲ-ಕ್ಕೆ-ತ್ತಿ-ಕೊಂಡು
ಮೇಲ-ಕ್ಕೆ-ತ್ತಿದ
ಮೇಲ-ಕ್ಕೆ-ತ್ತಿ-ದನು
ಮೇಲ-ಕ್ಕೆ-ತ್ತಿ-ದರು
ಮೇಲ-ಕ್ಕೆ-ತ್ತಿ-ರುವೆ
ಮೇಲ-ಕ್ಕೆತ್ತು
ಮೇಲ-ಕ್ಕೆ-ದ್ದನು
ಮೇಲ-ಕ್ಕೆ-ದ್ದರು
ಮೇಲ-ಕ್ಕೆ-ದ್ದಳು
ಮೇಲ-ಕ್ಕೆ-ದ್ದ-ವನೆ
ಮೇಲ-ಕ್ಕೆದ್ದು
ಮೇಲ-ಕ್ಕೆದ್ದೆ
ಮೇಲ-ಕ್ಕೆ-ಬ್ಬಿಸಿ
ಮೇಲ-ಕ್ಕೆ-ಬ್ಬಿ-ಸಿ-ದನು
ಮೇಲ-ಕ್ಕೆ-ಳೆ-ದರು
ಮೇಲ-ಕ್ಕೆ-ಳೆ-ದು-ಕೊಂ-ಡನು
ಮೇಲ-ಕ್ಕೆ-ಸೆ-ಯು-ತ್ತಿತ್ತು
ಮೇಲ-ಕ್ಕೇ-ಳ-ದಿದ್ದು
ಮೇಲ-ಕ್ಕೇ-ಳ-ದಿ-ರಲು
ಮೇಲ-ಕ್ಕೇ-ಳ-ಬೇಕಾ
ಮೇಲ-ಕ್ಕೇ-ಳ-ಲಿಲ್ಲ
ಮೇಲ-ಕ್ಕೇಳು
ಮೇಲ-ಕ್ಕೇ-ಳು-ವಂತೆ
ಮೇಲ-ಕ್ಕೇ-ಳು-ವ-ವ-ರೆಗೆ
ಮೇಲ-ಕ್ಕೇ-ಳು-ವಷ್ಟು
ಮೇಲಾ-ಗಲಿ
ಮೇಲಾ-ಗಿ-ರು-ವುದನ್ನು
ಮೇಲಾ-ದ-ವರು
ಮೇಲಿ
ಮೇಲಿಂದ
ಮೇಲಿಂ-ದ-ಮೇಲೆ
ಮೇಲಿ-ಟ್ಟರೆ
ಮೇಲಿ-ಟ್ಟಿದ್ದ
ಮೇಲಿಟ್ಟು
ಮೇಲಿ-ಟ್ಟು-ಕೊಂಡ
ಮೇಲಿ-ಟ್ಟು-ಕೊಂ-ಡಳು
ಮೇಲಿ-ಟ್ಟು-ಕೊಂಡು
ಮೇಲಿ-ಟ್ಟು-ಕೊ-ಳ್ಳು-ವಳು
ಮೇಲಿದ್ದ
ಮೇಲಿ-ದ್ದರೆ
ಮೇಲಿನ
ಮೇಲಿ-ನದೇ
ಮೇಲಿ-ನಿಂದ
ಮೇಲಿ-ರ-ದಂತೆ
ಮೇಲಿ-ರಲು
ಮೇಲಿ-ರುವ
ಮೇಲು
ಮೇಲುಗೈ
ಮೇಲು-ಸಿರು
ಮೇಲು-ಹೊ-ದಿಕೆ
ಮೇಲೂ
ಮೇಲೆ
ಮೇಲೆಂದು
ಮೇಲೆ-ಕ್ಕೆದ್ದು
ಮೇಲೆ-ತ್ತ-ಬೇ-ಕಾ-ಯಿತು
ಮೇಲೆ-ದ್ದನು
ಮೇಲೆ-ದ್ದರು
ಮೇಲೆ-ದ್ದಳು
ಮೇಲೆ-ದ್ದಿ-ರುವ
ಮೇಲೆದ್ದು
ಮೇಲೆಯೂ
ಮೇಲೆಯೆ
ಮೇಲೆಯೇ
ಮೇಲೆ-ರಗಿ
ಮೇಲೆಲ್ಲ
ಮೇಲೇ
ಮೇಲೇ-ನಾ-ದರೂ
ಮೇಲೇ-ರಿ-ದ-ವರು
ಮೇಲೇ-ರಿ-ಬ-ರು-ತ್ತಿದ್ದ
ಮೇಲೇ-ರಿಸಿ
ಮೇಲೇ-ಳು-ವುದು
ಮೇಲೊಂದು
ಮೇಲ್ಕ-ಟ್ಟು-ಗಳು
ಮೇಲ್ಕ-ಟ್ಟೆಲ್ಲ
ಮೇಲ್ಗ-ಡೆಯ
ಮೇಲ್ಗೈ-ಯಾ-ದರೆ
ಮೇಲ್ಪಂ-ಕ್ತಿ-ಯಾ-ಗು-ವಂತೆ
ಮೇಲ್ಪಂ-ಕ್ತಿ-ಯಾ-ದ-ವೆಂದು
ಮೇಲ್ಭಾ-ಗ-ದಲ್ಲಿ
ಮೇಲ್ಭಾ-ಗ-ದ-ಲ್ಲಿ-ರುವ
ಮೇಲ್ಮು-ಸು-ಕನ್ನು
ಮೇಲ್ವಿ-ಚಾ-ರ-ಣೆ-ಗೊ-ಳ-ಪಟ್ಟ
ಮೇವನ್ನು
ಮೇಶ್ವರ
ಮೇಷ್ಟಿ-ಯನ್ನು
ಮೇಽನ-ನ್ಯ-ವಿ-ಷಯಾ
ಮೈ
ಮೈಕ-ಟ್ಟನ್ನು
ಮೈಕೂ-ದಲು
ಮೈಗೂ-ಡಿ-ಸಿ-ಕೊಂ-ಡ-ವನು
ಮೈಗೂ-ಡಿ-ಸಿ-ಕೊಂಡು
ಮೈಗೂ-ದಲು
ಮೈಗೆ
ಮೈಗೆಲ್ಲ
ಮೈಗೆಲ್ಲಾ
ಮೈತಿ-ಳಿದು
ಮೈತ್ರಿ
ಮೈತ್ರೇಯ
ಮೈತ್ರೇ-ಯ-ನನ್ನು
ಮೈತ್ರೇ-ಯ-ನಿಗೆ
ಮೈತ್ರೇ-ಯನು
ಮೈತ್ರೇ-ಯರ
ಮೈತ್ರೇ-ಯ-ರದು
ಮೈತ್ರೇ-ಯ-ರನ್ನು
ಮೈತ್ರೇ-ಯರು
ಮೈತ್ರೇ-ಯರೂ
ಮೈದಾನ
ಮೈದಾ-ನಕ್ಕೆ
ಮೈದುಂಬಿ
ಮೈದುಂ-ಬಿದ
ಮೈದು-ನ-ರನ್ನು
ಮೈದೋ-ರಿ-ದಾಗ
ಮೈಬಣ್ಣ
ಮೈಬ-ಣ್ಣ-ದಿಂದ
ಮೈಮ-ರೆತ
ಮೈಮ-ರೆ-ತರು
ಮೈಮ-ರೆ-ತಿದ್ದ
ಮೈಮ-ರೆ-ತಿ-ದ್ದರು
ಮೈಮ-ರೆ-ತಿ-ದ್ದಾಳೆ
ಮೈಮ-ರೆ-ತಿ-ರುವ
ಮೈಮ-ರೆ-ತಿ-ರು-ವ-ಷ್ಟ-ರಲ್ಲಿ
ಮೈಮ-ರೆ-ತಿ-ರು-ವಾಗ
ಮೈಮ-ರೆತು
ಮೈಮ-ರೆಯ
ಮೈಮ-ರೆ-ಯಿತು
ಮೈಮ-ರೆ-ವಿ-ನಿಂದ
ಮೈಮು-ಚ್ಚಿ-ಕೊಂಡು
ಮೈಮೇಲೆ
ಮೈಮೇ-ಲೆಲ್ಲ
ಮೈಯನ್ನು
ಮೈಯಲ್ಲ
ಮೈಯಲ್ಲಿ
ಮೈಯಲ್ಲೆ
ಮೈಯಿಂದ
ಮೈಯು-ಜ್ಜಿ-ಕೊಂಡು
ಮೈಯೆಲ್ಲ
ಮೈಯ್ಯ-ಲ್ಲಿ-ರುವ
ಮೈರೇ-ಯ-ವೆಂಬ
ಮೈಲಿ
ಮೈಸೂ-ರಿ-ನಿಂದ
ಮೈಸೂರು
ಮೊಗದ
ಮೊಗ್ಗು-ಗ-ಳಂ-ತಿ-ರುವ
ಮೊಟ್ಟ
ಮೊಟ್ಟ-ಮೊ-ದಲ
ಮೊಟ್ಟ-ಮೊ-ದ-ಲ-ನೆ-ಯ-ದಾದ
ಮೊಟ್ಟ-ಮೊ-ದ-ಲ-ಬಾ-ರಿಗೆ
ಮೊಟ್ಟ-ಮೊ-ದ-ಲ-ಲ್ಲಿಯೇ
ಮೊಟ್ಟ-ಮೊ-ದಲು
ಮೊಟ್ಟೆಯೋ
ಮೊಣ-ಕಾಲು
ಮೊದ
ಮೊದ-ಮೊ-ದಲು
ಮೊದಲ
ಮೊದ-ಲನೆ
ಮೊದ-ಲ-ನೆಯ
ಮೊದ-ಲ-ನೆ-ಯ-ದಾದ
ಮೊದ-ಲ-ನೆ-ಯದೇ
ಮೊದ-ಲ-ಬಾರಿ
ಮೊದ-ಲರ್ಧ
ಮೊದ-ಲಾದ
ಮೊದ-ಲಾ-ದ-ವರ
ಮೊದ-ಲಾ-ದ-ವ-ರನ್ನು
ಮೊದ-ಲಾ-ದ-ವರು
ಮೊದ-ಲಾ-ದ-ವರೆಲ್ಲ
ಮೊದ-ಲಾ-ದ-ವರೆ-ಲ್ಲರೂ
ಮೊದ-ಲಾ-ದವು
ಮೊದ-ಲಾ-ದ-ವು-ಗಳ
ಮೊದ-ಲಾ-ದ-ವು-ಗಳಿಂದ
ಮೊದ-ಲಾ-ದು-ವನ್ನು
ಮೊದ-ಲಾ-ದು-ವು-ಗ-ಳಿಗೂ
ಮೊದ-ಲಾ-ದು-ವೆಲ್ಲ
ಮೊದ-ಲಾ-ಯಿತು
ಮೊದಲಿ
ಮೊದ-ಲಿಂದ
ಮೊದ-ಲಿಂ-ದಲೂ
ಮೊದ-ಲಿ-ಗಿಂತ
ಮೊದ-ಲಿ-ಗಿಂ-ತಲೂ
ಮೊದ-ಲಿ-ದ್ದಂತೆ
ಮೊದ-ಲಿನ
ಮೊದ-ಲಿ-ನಂ-ತಾ-ಗ-ಬೇಕು
ಮೊದ-ಲಿ-ನಂತೆ
ಮೊದ-ಲಿ-ನಿಂದ
ಮೊದ-ಲಿ-ನಿಂ-ದಲೂ
ಮೊದ-ಲಿ-ಲ್ಲದ
ಮೊದ-ಲಿ-ಲ್ಲ-ವಾ-ಯಿತು
ಮೊದಲು
ಮೊದ-ಲು-ಮಾ-ಡಿತು
ಮೊದ-ಲು-ಮಾ-ಡಿ-ದಳು
ಮೊದಲೆ
ಮೊದ-ಲೆಂದು
ಮೊದ-ಲೆ-ರ-ಡ-ಕ್ಷ-ರ-ಗಳನ್ನು
ಮೊದಲೇ
ಮೊನ-ಚನ್ನು
ಮೊನ-ಚಾದ
ಮೊನೆ
ಮೊನೆಗೆ
ಮೊನೆ-ಯಾಗಿ
ಮೊನೆ-ಯುಳ್ಳ
ಮೊನ್ನೆ-ತಾನೆ
ಮೊಮ್ಮ-ಕ್ಕ-ಳಿಂದ
ಮೊಮ್ಮ-ಕ್ಕಳು
ಮೊಮ್ಮಗ
ಮೊಮ್ಮ-ಗನ
ಮೊಮ್ಮ-ಗ-ನಾದ
ಮೊಮ್ಮ-ಗ-ನಿಗೆ
ಮೊಮ್ಮ-ಗನು
ಮೊಮ್ಮ-ಗ-ಳಾದ
ಮೊರೆ
ಮೊರೆದು
ಮೊರೆ-ಯನ್ನು
ಮೊರೆ-ಯಿ-ಟ್ಟನು
ಮೊರೆ-ಯು-ತ್ತಿದೆ
ಮೊರೆ-ಹೊ-ಕ್ಕನು
ಮೊರೆ-ಹೊ-ಕ್ಕರು
ಮೊರೆ-ಹೊ-ಕ್ಕ-ವನ
ಮೊರೆ-ಹೊ-ಕ್ಕ-ವ-ರ-ನ್ನೆಲ್ಲ
ಮೊರೆ-ಹೊ-ಕ್ಕಿ-ದ್ದಾನೆ
ಮೊಲ
ಮೊಲ-ವನ್ನು
ಮೊಲೆ
ಮೊಲೆ-ಕು-ಡಿ-ಯು-ವು-ದನ್ನೂ
ಮೊಲೆ-ಗಳಲ್ಲಿ
ಮೊಲೆ-ಗಳು
ಮೊಲೆ-ಗು-ಡಿದು
ಮೊಲೆ-ಗು-ಡಿಯ
ಮೊಲೆ-ಗು-ಡಿ-ಯು-ವುದು
ಮೊಲೆ-ಯನ್ನು
ಮೊಲೆ-ಯ-ಲ್ಲಿದ್ದ
ಮೊಲೆ-ಯು-ಣಿ-ಸಿ-ದಳು
ಮೊಲೆ-ಯು-ಣಿ-ಸು-ತ್ತಿ-ದ್ದಳು
ಮೊಲೆ-ಯು-ಣಿ-ಸು-ವರು
ಮೊಲೆ-ಹಾ-ಲನ್ನು
ಮೊಲೆ-ಹಾಲು
ಮೊಲ್ಲೆ
ಮೊಳ-ಕಾಲು
ಮೊಳಗಿ
ಮೊಳ-ಗಿ-ದವು
ಮೊಳ-ಗಿ-ಸ-ಲ್ಪ-ಡುವ
ಮೊಳ-ಗು-ವಂತೆ
ಮೊಳೆ
ಮೊಳೆತು
ಮೊಳೆ-ಯ-ಲಾರ
ಮೊಳೆ-ಯು-ತ್ತಲೆ
ಮೊಳೆ-ಯು-ವು-ದಕ್ಕೆ
ಮೊಸ-ರ-ನ್ನದ
ಮೊಸ-ರನ್ನು
ಮೊಸ-ರಿನ
ಮೊಸರು
ಮೊಸ-ರು-ಬೆ-ಣ್ಣೆ-ಗಳನ್ನು
ಮೊಸಳೆ
ಮೊಸ-ಳೆಯ
ಮೊಸ-ಳೆ-ಯನ್ನು
ಮೊಸ-ಳೆ-ಯಾ-ಗಿದ್ದ
ಮೊಸ-ಳೆ-ಯಿಂದ
ಮೊಸ-ಳೆ-ಯೊಂದು
ಮೊಸ-ಳೆ-ಯೊ-ಡನೆ
ಮೋಂ
ಮೋಕ್ಷ
ಮೋಕ್ಷ-ಎ-ಲ್ಲವೂ
ಮೋಕ್ಷ-ಕ್ಕಾಗಿ
ಮೋಕ್ಷ-ಕ್ಕಿಂ-ತಲೂ
ಮೋಕ್ಷಕ್ಕೂ
ಮೋಕ್ಷಕ್ಕೆ
ಮೋಕ್ಷ-ಗಳನ್ನು
ಮೋಕ್ಷ-ಗಳೂ
ಮೋಕ್ಷ-ಗ-ಳೆ-ರ-ಡಕ್ಕೂ
ಮೋಕ್ಷದ
ಮೋಕ್ಷ-ದತ್ತ
ಮೋಕ್ಷ-ದಲ್ಲಿ
ಮೋಕ್ಷ-ಧರ್ಮ
ಮೋಕ್ಷ-ಪ್ರ-ದ-ವಾದ
ಮೋಕ್ಷ-ಪ್ರಾ-ಪ್ತಿಯ
ಮೋಕ್ಷ-ಮಾ-ರ್ಗ-ವನ್ನು
ಮೋಕ್ಷ-ವನ್ನು
ಮೋಕ್ಷ-ವಾ-ಗಲಿ
ಮೋಕ್ಷವು
ಮೋಕ್ಷವೂ
ಮೋಕ್ಷವೇ
ಮೋಕ್ಷ-ವೊಂ-ದನ್ನು
ಮೋಕ್ಷ-ಶಾ-ಸ್ತ್ರ-ವೆಂದೂ
ಮೋಕ್ಷ-ಸಂ-ಪಾ-ದನೆ
ಮೋಕ್ಷ-ಸಾ-ಧನ
ಮೋಕ್ಷ-ಸಾ-ಧ-ನ-ಗಳು
ಮೋಕ್ಷ-ಸಾ-ಧನೆ
ಮೋಕ್ಷ-ಸಾ-ಧ-ನೆ-ಗಾ-ಗಿಯೇ
ಮೋಕ್ಷ-ಸಾ-ಮ್ರಾ-ಜ್ಯದ
ಮೋಕ್ಷ-ಹೇ-ತು-ವಾ-ಗ-ಬೇಕು
ಮೋಕ್ಷ-ಹೊಂ-ದಿದ
ಮೋಕ್ಷಾ-ಪೇಕ್ಷಿ
ಮೋಕ್ಷಾ-ಪೇ-ಕ್ಷಿಯ
ಮೋಕ್ಷಾ-ಪೇ-ಕ್ಷಿ-ಯಾದ
ಮೋಟು
ಮೋಟು-ಮ-ರ-ದಂತೆ
ಮೋಡ
ಮೋಡ-ಗಳಿಂದ
ಮೋಡ-ಗ-ಳಿ-ದ್ದಂತೆ
ಮೋಡ-ಗಳು
ಮೋಡ-ಗ-ಳೆದ್ದು
ಮೋಡ-ಗಳೇ
ಮೋಡ-ದಂತೆ
ಮೋಡ-ದೊ-ಡನೆ
ಮೋಡ-ವನ್ನು
ಮೋಡವು
ಮೋಡ-ವೊಂದು
ಮೋಡಿಗೆ
ಮೋಡಿಯೊ
ಮೋರೆ-ಯಿಂದ
ಮೋಸ
ಮೋಸಕ್ಕೆ
ಮೋಸ-ಗಾರ
ಮೋಸ-ಗಾ-ರ-ನಾದ
ಮೋಸದ
ಮೋಸ-ದಿಂದ
ಮೋಸ-ಮಾ-ಡ-ಲೆಂದು
ಮೋಸ-ಮಾ-ಡಿ-ದೆ-ಯಲ್ಲಾ
ಮೋಸ-ಮಾ-ಡಿ-ದ್ದಾರೆ
ಮೋಸ-ವನ್ನು
ಮೋಸ-ವಾ-ಗ-ದಿ-ರ-ಬೇ-ಕಾ-ದರೆ
ಮೋಸ-ವಿದ್ಯೆ
ಮೋಸ-ವಿ-ದ್ಯೆ-ಯಲ್ಲಿ
ಮೋಸ-ವೆಂ-ಬುದು
ಮೋಸ-ಹೋ-ಗ-ದಿ-ರು-ತ್ತಾರೆ
ಮೋಸ-ಹೋ-ಗ-ಬೇಡ
ಮೋಸ-ಹೋ-ಗು-ತ್ತಿ-ರುವೆ
ಮೋಸ-ಹೋ-ಗು-ತ್ತೇವೆ
ಮೋಸ-ಹೋ-ಗು-ವು-ದಿ-ಲ್ಲಮ್ಮ
ಮೋಸ-ಹೋದ
ಮೋಸ-ಹೋ-ದೆ-ನಲ್ಲಾ
ಮೋಹ
ಮೋಹಕ
ಮೋಹ-ಕ-ನಾ-ಗಿ-ದ್ದಾನೆ
ಮೋಹ-ಕ-ಮೂರ್ತಿ
ಮೋಹ-ಕ-ರೂ-ಪ-ವನ್ನು
ಮೋಹ-ಕ-ವಾಗಿ
ಮೋಹ-ಕ-ವಾದ
ಮೋಹ-ಕ್ಕಾ-ಗಲಿ
ಮೋಹಕ್ಕೆ
ಮೋಹ-ಗಳನ್ನು
ಮೋಹ-ಗೊಂ-ಡಳು
ಮೋಹ-ಗೊಂ-ಡಿದ್ದ
ಮೋಹ-ಗೊಂಡು
ಮೋಹ-ಗೊ-ಳಿ-ಸಿ-ದು-ದೊ-ಒಂದೆ
ಮೋಹ-ಗೊ-ಳಿ-ಸುವ
ಮೋಹ-ಗೊ-ಳ್ಳು-ತ್ತಿ-ರು-ವಾಗ
ಮೋಹ-ಗೊ-ಳ್ಳು-ವು-ದಿಲ್ಲ
ಮೋಹದ
ಮೋಹ-ದಿಂದ
ಮೋಹ-ದೂ-ರ-ನೆಂ-ಬು-ದನ್ನು
ಮೋಹನ
ಮೋಹ-ನ-ಷ್ಟ-ವಾ-ಯಿತು
ಮೋಹ-ನಾಂಗ
ಮೋಹ-ನಾಂ-ಗ-ನಿಗೆ
ಮೋಹ-ನಾಂಗಿ
ಮೋಹ-ನಾಂ-ಗಿ-ಯನ್ನು
ಮೋಹ-ನಾ-ಕಾರ
ಮೋಹ-ನಾ-ಕಾ-ರ-ದಿಂದ
ಮೋಹ-ನಾ-ಕಾ-ರ-ವನ್ನು
ಮೋಹ-ಪ-ರ-ವ-ಶ-ನ-ನ್ನಾಗಿ
ಮೋಹ-ಪ-ರ-ವ-ಶ-ನಾಗಿ
ಮೋಹ-ಪ-ರ-ವ-ಶ-ರಾ-ದರು
ಮೋಹ-ಪಾಶ
ಮೋಹ-ಪಾ-ಶಕ್ಕೆ
ಮೋಹ-ಪಾ-ಶ-ವನ್ನು
ಮೋಹ-ಬಂ-ಧನ
ಮೋಹ-ಬಂ-ಧ-ನ-ವನ್ನು
ಮೋಹ-ಮ-ಗ್ನ-ಳಾ-ಗಿದ್ದ
ಮೋಹ-ವನ್ನು
ಮೋಹ-ವನ್ನೂ
ಮೋಹ-ವಿತ್ತು
ಮೋಹ-ವಿಲ್ಲ
ಮೋಹವೂ
ಮೋಹ-ವೆ-ಲ್ಲಿ-ಯದು
ಮೋಹವೊ
ಮೋಹಾತಿ
ಮೋಹಿ-ತ-ರಾಗಿ
ಮೋಹಿನಿ
ಮೋಹಿ-ನಿಗೆ
ಮೋಹಿ-ನಿಯ
ಮೋಹಿ-ನಿ-ಯಂತೆ
ಮೋಹಿ-ನಿ-ಯಾಗಿ
ಮೋಹಿ-ನಿಯು
ಮೋಹಿ-ನಿಯೂ
ಮೋಹಿ-ನೀಂ
ಮೋಹಿ-ನೀ-ರೂ-ಪ-ವನ್ನು
ಮೋಹಿ-ಸ-ದಿ-ರು-ವುದು
ಮೋಹಿಸಿ
ಮೋಹಿ-ಸಿದ
ಮೋಹಿ-ಸಿ-ದ-ನಂತೆ
ಮೋಹಿ-ಸಿ-ದರೆ
ಮೋಹಿ-ಸಿ-ದಳು
ಮೋಹಿ-ಸಿ-ದ್ದನು
ಮೋಹಿ-ಸಿ-ರು-ವ-ಳೆಂಬ
ಮೋಹಿ-ಸು-ವಂ-ತಹ
ಮೌಂಜಿ
ಮೌನಕ್ಕೆ
ಮೌನ-ದಿಂದ
ಮೌನ-ದಿಂ-ದಿ-ರಲು
ಮೌನ-ಧಾ-ರಿ-ಯಾಗಿ
ಮೌನ-ಧಾ-ರಿ-ಯಾ-ಗಿ-ದ್ದಳು
ಮೌನ-ವನ್ನು
ಮೌನ-ವಾಗಿ
ಮೌನ-ವಾ-ಗಿದ್ದು
ಮೌನ-ವಾ-ಗಿ-ರಲು
ಮೌನಿ-ಯಾ-ದನು
ಮೌರ್ಯ-ವಂ-ಶದ
ಮೌಲ್ಯ-ದಲ್ಲಿ
ಮ್ನನು
ಮ್ಯಾ
ಮ್ಯಾಕ್ಡೊ-ನಲ್
ಮ್ಲೇಚ್ಛ-ರಾ-ಗು-ವಂತೆ
ಯ
ಯಂ
ಯಂತಹ
ಯಂತಾ-ಗಿತ್ತು
ಯಂತಾ-ಯಿತು
ಯಂತಿ
ಯಂತಿ-ದ್ದವು
ಯಂತಿ-ರುವ
ಯಂತೆ
ಯಂತ್ರ
ಯಂತ್ರದ
ಯಂತ್ರ-ದಂತೆ
ಯಂತ್ರ-ವನ್ನು
ಯಕ್ಷ
ಯಕ್ಷ-ನೊ-ಬ್ಬನ
ಯಕ್ಷ-ನೊ-ಬ್ಬನು
ಯಕ್ಷರ
ಯಕ್ಷ-ರಾ-ಕ್ಷ-ಸರ
ಯಕ್ಷ-ರಾ-ಜ-ನಾದ
ಯಕ್ಷ-ರಾ-ದರು
ಯಕ್ಷರು
ಯಕ್ಷ-ವಂ-ಶ-ವನ್ನೆ
ಯಕ್ಷ್ಮಣಾ
ಯಜ-ಮಾನ
ಯಜ-ಮಾ-ನನ
ಯಜ-ಮಾ-ನ-ನನ್ನು
ಯಜ-ಮಾ-ನ-ನಾದ
ಯಜ-ಮಾ-ನಿ-ಯಾದ
ಯಜೇ-ತಿಹೇ
ಯಜ್ಞ
ಯಜ್ಞ-ಕ-ರ್ತೃ-ವಿನ
ಯಜ್ಞ-ಕ-ರ್ಮವು
ಯಜ್ಞ-ಕಲ್ಪ
ಯಜ್ಞ-ಕಾರ್ಯ
ಯಜ್ಞ-ಕಾ-ಲ-ದಲ್ಲಿ
ಯಜ್ಞ-ಕುಂ-ಡ-ದಿಂದ
ಯಜ್ಞಕ್ಕೆ
ಯಜ್ಞ-ಕ್ರ-ತವೇ
ಯಜ್ಞ-ಕ್ರ-ತು-ಗಳ
ಯಜ್ಞ-ಕ್ರ-ತು-ಗಳೇ
ಯಜ್ಞ-ಗಳನ್ನು
ಯಜ್ಞ-ಗಳಲ್ಲಿ
ಯಜ್ಞ-ಗ-ಳಿಗೆ
ಯಜ್ಞದ
ಯಜ್ಞ-ದಲ್ಲಿ
ಯಜ್ಞ-ದಿಂದ
ಯಜ್ಞ-ದೀ-ಕ್ಷಿ-ತ-ನಾದ
ಯಜ್ಞ-ದೃ-ಷ್ಟಿ-ಯಿಂದ
ಯಜ್ಞ-ನಾ-ಯ-ಕನೂ
ಯಜ್ಞ-ನಿ-ಯಾ-ಮಕ
ಯಜ್ಞನು
ಯಜ್ಞ-ನೆಂಬ
ಯಜ್ಞ-ಪತಿ
ಯಜ್ಞ-ಪ-ಶು-ವನ್ನು
ಯಜ್ಞ-ಪು-ರುಷ
ಯಜ್ಞ-ಪು-ರು-ಷ-ನಾದ
ಯಜ್ಞ-ಪು-ರು-ಷ-ನೆಂ-ಬು-ವ-ನನ್ನು
ಯಜ್ಞ-ಬಾ-ಹು-ಅ-ಧಿ-ಪ-ತಿ-ಯಾಗಿ
ಯಜ್ಞ-ಮಂ-ಟ-ಪ-ದಲ್ಲಿ
ಯಜ್ಞ-ಮ-ಯ-ವ-ನ್ನಾಗಿ
ಯಜ್ಞ-ಮಾಡಿ
ಯಜ್ಞ-ಮಾ-ಡಿ-ದ-ವನೇ
ಯಜ್ಞ-ಮಾ-ಡು-ವಂತೆ
ಯಜ್ಞ-ಮೂ-ರ್ತಿಯು
ಯಜ್ಞ-ಯಾ-ಗಾದಿ
ಯಜ್ಞ-ಯಾ-ಗಾ-ದಿ-ಗಳನ್ನು
ಯಜ್ಞ-ಯಾ-ಗಾ-ದಿ-ಗಳಿಂದ
ಯಜ್ಞ-ರೂ-ಪದ
ಯಜ್ಞ-ವನ್ನು
ಯಜ್ಞವು
ಯಜ್ಞ-ಶಾ-ಲೆಗೆ
ಯಜ್ಞಶ್ಚ
ಯಜ್ಞ-ಸಂ-ಭಾ-ರ-ಗಳನ್ನು
ಯಜ್ಞ-ಸ್ವ-ರೂಪ
ಯಜ್ಞಾದಿ
ಯಜ್ಞಾ-ನು-ಷ್ಠಾ-ತೃ-ವಾದ
ಯಜ್ಞೇನ
ಯಜ್ಞೇ-ಶ್ವರ
ಯಜ್ಞೇ-ಶ್ವ-ರ-ನಂತೆ
ಯಜ್ಞೋ
ಯಜ್ಞೋ-ಪ-ವೀತ
ಯಣ
ಯಣ-ನಂತೆ
ಯತ-ಶ್ಚೇದಂ
ಯತಿಯು
ಯತೀ-ಶ್ವ-ರಾ-ನಂ-ದರ
ಯತೀ-ಶ್ವ-ರಾ-ನಂ-ದರು
ಯತ್ತ-ತ್ಕರ್ಮ
ಯತ್ನಿ-ಸ-ಬೇಡಿ
ಯತ್ನಿ-ಸಿತು
ಯತ್ನಿ-ಸಿ-ದನು
ಯತ್ನಿ-ಸಿ-ದರೂ
ಯತ್ನಿ-ಸಿ-ದುದು
ಯತ್ನಿ-ಸು-ತ್ತಿ-ರು-ವನು
ಯತ್ನಿ-ಸು-ವರು
ಯತ್ರ
ಯತ್ಸ-ತ್ಯಸ್ಯ
ಯಥಾ
ಯಥಾ-ವ-ತ್ತಾಗಿ
ಯಥೇ-ಚ್ಛ-ವಾಗಿ
ಯಥೇಚ್ಛೆ
ಯಥೇ-ಷ್ಟ-ವಾಗಿ
ಯಥೈ-ಕಾ-ತ್ಮ್ಯಾನು
ಯದಂ-ವಿ-ದ್ಧಾಃ
ಯದ-ಕರ್ಮ
ಯದ-ನು-ಚ-ರಿ-ತ-ಲೀ-ಲಾ-ಕ-ರ್ಣ-ಪೀ-ಯೂ-ಷ-ವಿ-ಪ್ರುಟ್
ಯದು
ಯದು-ಕು-ಮಾ-ರರ
ಯದು-ದೇವ
ಯದು-ಪ-ತೇಃ
ಯದು-ಮ-ಹಾ-ರಾಜ
ಯದು-ಮ-ಹಾ-ರಾ-ಜನು
ಯದು-ರಾ-ಜನು
ಯದು-ವಂ-ಶದ
ಯದು-ವಂ-ಶ-ದಲ್ಲಿ
ಯದು-ವಂ-ಶ-ದ್ರೋ-ಹಿ-ಯಾದ
ಯದು-ವಿನ
ಯದೂ-ನಾಂ
ಯದ್ವ-ದ್ವಯಂ
ಯನ್ನಲ್ಲ
ಯನ್ನ-ಸ್ಪೃ-ಶಂತಿ
ಯನ್ನಾಗಿ
ಯನ್ನು
ಯನ್ನೂ
ಯನ್ನೆ
ಯನ್ನೆಲ್ಲ
ಯನ್ನೇ
ಯನ್ನೋ
ಯಮ
ಯಮ-ದೂ-ತ-ರಂತೆ
ಯಮ-ದೂ-ತ-ರಿಗೂ
ಯಮ-ದೂ-ತರು
ಯಮ-ದೂ-ತರೆ
ಯಮ-ಧ-ರ್ಮ-ಎಂಬ
ಯಮ-ಧ-ರ್ಮ-ನಿಗೂ
ಯಮ-ಧ-ರ್ಮ-ನಿಗೆ
ಯಮ-ಧ-ರ್ಮನೂ
ಯಮ-ಧ-ರ್ಮ-ರಾಯ
ಯಮ-ಧ-ರ್ಮ-ರಾ-ಯನ
ಯಮ-ಧ-ರ್ಮ-ರಾ-ಯ-ನನ್ನೆ
ಯಮ-ಧ-ರ್ಮ-ರಾ-ಯನು
ಯಮನ
ಯಮ-ನಂತೆ
ಯಮ-ನನ್ನು
ಯಮ-ನಿಗೂ
ಯಮನು
ಯಮ-ನೆಂಬ
ಯಮನೇ
ಯಮ-ರಾ-ಜ-ನ-ಲ್ಲಿಗೆ
ಯಮ-ಲೋ-ಕಕ್ಕೆ
ಯಮ-ಲೋ-ಕ-ದಲ್ಲಿ
ಯಮ-ಲೋ-ಕ-ದಿಂದ
ಯಮ-ಳ-ರಾಗಿ
ಯಮ-ಶಿಕ್ಷೆ
ಯಮುನಾ
ಯಮು-ನಾ-ನದಿ
ಯಮು-ನಾ-ನ-ದಿಯ
ಯಮು-ನಾ-ನ-ದಿ-ಯಲ್ಲಿ
ಯಮು-ನಾ-ನದೀ
ಯಮುನೆ
ಯಮು-ನೆ-ಯಲ್ಲಿ
ಯಯಾತಿ
ಯಯಾ-ತಿಗೆ
ಯಯಾ-ತಿಯ
ಯಯಾ-ತಿ-ಯನ್ನು
ಯಯಾ-ತಿಯು
ಯರ
ಯರನ್ನು
ಯರು
ಯರೂ
ಯರೆ
ಯರೊ-ಡನೆ
ಯಲ್ಲ
ಯಲ್ಲವೆ
ಯಲ್ಲಾ
ಯಲ್ಲಿ
ಯಲ್ಲಿದ್ದ
ಯಲ್ಲಿಯೂ
ಯಲ್ಲಿಯೆ
ಯಲ್ಲಿಯೇ
ಯಲ್ಲೊ
ಯವನ
ಯವ-ನ-ನೊ-ಡನೆ
ಯವ-ನ-ರಾ-ಜನು
ಯವನೇ
ಯವ್ವ-ನದ
ಯವ್ವ-ನ-ವನ್ನು
ಯವ್ವ-ನವೇ
ಯಶ-ಸಾ-ವ-ದಾತಃ
ಯಶ-ಸ್ವಿ-ಯಾಗಿ
ಯಶ-ಸ್ಸು-ಇ-ವು-ಗಳ
ಯಶೋ-ದಾ-ದೇ-ವಿಯೆ
ಯಶೋದೆ
ಯಶೋ-ದೆಗೂ
ಯಶೋ-ದೆಯ
ಯಶೋ-ದೆ-ಯದು
ಯಶೋ-ದೆ-ಯನ್ನು
ಯಶೋ-ದೆ-ಯನ್ನೂ
ಯಶೋ-ದೆ-ಯರ
ಯಶೋ-ದೆ-ಯ-ರಿಗೆ
ಯಶೋ-ದೆ-ಯ-ರಿ-ಬ್ಬರೂ
ಯಶೋ-ದೆ-ಯರು
ಯಶೋ-ದೆ-ಯರೂ
ಯಶೋ-ದೆಯು
ಯಶೋ-ದೆ-ಯೊ-ಡನೆ
ಯಶೋ-ಧೆಗೆ
ಯಶ್ಚಕ್ರೇ
ಯಸ್ಮಿ-ನ್ನಿದಂ
ಯಸ್ಯ
ಯಾ
ಯಾಂತು
ಯಾಂತ್ರಿ-ಕ-ವಾಗಿ
ಯಾಗ
ಯಾಗ-ಕಾ-ರ್ಯ-ದಲ್ಲಿ
ಯಾಗ-ಕಾ-ರ್ಯ-ವನ್ನು
ಯಾಗ-ಕಾ-ಲ-ದಲ್ಲಿ
ಯಾಗ-ಕ್ಕಾಗಿ
ಯಾಗ-ಕ್ಕಿಂ-ತಲೂ
ಯಾಗಕ್ಕೆ
ಯಾಗ-ಗಳನ್ನು
ಯಾಗ-ಗಳಲ್ಲಿ
ಯಾಗ-ಗ-ಳ-ಲ್ಲೆಲ್ಲ
ಯಾಗ-ಗ-ಳಿಂ-ದಲೆ
ಯಾಗ-ಗ-ಳಿಗೆ
ಯಾಗ-ಗಳು
ಯಾಗದ
ಯಾಗ-ದಲ್ಲಿ
ಯಾಗ-ದಿಂದ
ಯಾಗ-ದೀ-ಕ್ಷೆ-ಯನ್ನು
ಯಾಗ-ಪ-ಶು-ವಾಗಿ
ಯಾಗ-ಬೇಕು
ಯಾಗ-ಭೂಮಿ
ಯಾಗ-ಭೂ-ಮಿಯು
ಯಾಗ-ಮಂ-ಟ-ಪಕ್ಕಿ-ಳಿದು
ಯಾಗ-ಮಂ-ಟ-ಪಕ್ಕೆ
ಯಾಗ-ಮಂ-ಟ-ಪ-ವನ್ನು
ಯಾಗ-ಮಾ-ಡು-ವರೋ
ಯಾಗ-ಲಿಲ್ಲ
ಯಾಗ-ವನ್ನು
ಯಾಗ-ವನ್ನೂ
ಯಾಗ-ವಲ್ಲ
ಯಾಗವು
ಯಾಗ-ವೇನೋ
ಯಾಗ-ವೈ-ಭವ
ಯಾಗ-ವೈ-ಭ-ವ-ಗಳನ್ನು
ಯಾಗ-ಶಾ-ಲೆಗೆ
ಯಾಗ-ಶಾ-ಲೆ-ಯನ್ನು
ಯಾಗ-ಶಾ-ಲೆ-ಯಲ್ಲಿ
ಯಾಗ-ಶಾ-ಲೆ-ಯಿಂದ
ಯಾಗಾ-ದಿ-ಗಳನ್ನು
ಯಾಗಿ
ಯಾಗಿತ್ತು
ಯಾಗಿದೆ
ಯಾಗಿದ್ದ
ಯಾಗಿ-ದ್ದ-ನಂತೆ
ಯಾಗಿ-ದ್ದಾಗ
ಯಾಗಿಯೆ
ಯಾಗಿಯೇ
ಯಾಗಿ-ರುವ
ಯಾಗಿ-ರು-ವಂತೆ
ಯಾಗು-ತ್ತಲೆ
ಯಾಗು-ವಂತೆ
ಯಾಗು-ವು-ದಿ-ಲ್ಲವೆ
ಯಾಚನೆ
ಯಾಚ-ನೆ-ಗಾಗಿ
ಯಾಚಿ-ಸಿ-ದನು
ಯಾಚಿ-ಸಿ-ದಳು
ಯಾಚಿ-ಸು-ವು-ದ-ಕ್ಕಾಗಿ
ಯಾಜ್ಞ-ವ-ಲ್ಕ್ಯರೇ
ಯಾಜ್ಞ-ಸೇನಿ
ಯಾಟದ
ಯಾತ-ನೆ-ಗಳನ್ನು
ಯಾತು-ಧಾ-ನ-ಪ್ರ-ಮ-ಥ-ಪ್ರೇ-ತ-ಮಾತೃ
ಯಾತ್ಮ
ಯಾತ್ರೆ
ಯಾತ್ರೆ-ಗಾಗಿ
ಯಾತ್ರೆ-ಗೆಂದು
ಯಾತ್ರೆ-ಯನ್ನು
ಯಾದ
ಯಾದನು
ಯಾದರೂ
ಯಾದರೆ
ಯಾದಳು
ಯಾದವ
ಯಾದ-ವ-ಕುಲ
ಯಾದ-ವ-ಕು-ಲದ
ಯಾದ-ವ-ಕು-ಲ-ವೊಂ-ದನ್ನು
ಯಾದ-ವ-ಗಿರಿ
ಯಾದ-ವನ
ಯಾದ-ವ-ನನ್ನು
ಯಾದ-ವನು
ಯಾದ-ವ-ಭ-ಟರು
ಯಾದ-ವರ
ಯಾದ-ವ-ರನ್ನು
ಯಾದ-ವ-ರನ್ನೂ
ಯಾದ-ವ-ರ-ನ್ನೆಲ್ಲ
ಯಾದ-ವ-ರ-ನ್ನೆಲ್ಲಾ
ಯಾದ-ವ-ರಲ್ಲಿ
ಯಾದ-ವ-ರಾ-ಜ್ಯದ
ಯಾದ-ವ-ರಾದ
ಯಾದ-ವ-ರಿಗೆ
ಯಾದ-ವರು
ಯಾದ-ವ-ರು-ಎ-ಲ್ಲರೂ
ಯಾದ-ವರೂ
ಯಾದ-ವರೆಲ್ಲ
ಯಾದ-ವರೆ-ಲ್ಲ-ರ-ನ್ನೂ-ಕೊ-ನೆಗೆ
ಯಾದ-ವರೆ-ಲ್ಲರೂ
ಯಾದ-ವರೆಲ್ಲಾ
ಯಾದ-ವ-ರೊ-ಡನೆ
ಯಾದ-ವ-ವಂಶ
ಯಾದ-ವ-ವಂ-ಶ-ದಲ್ಲಿ
ಯಾದ-ವ-ವೀ-ರರ
ಯಾದ-ವ-ವೀ-ರ-ರಲ್ಲಿ
ಯಾದ-ವ-ವೀ-ರರು
ಯಾದ-ವ-ಸೇನೆ
ಯಾದ-ವೇಂ-ದ್ರಸ್ಯ
ಯಾದು-ದ-ರಿಂದ
ಯಾದೆ
ಯಾದೋ
ಯಾನಿ
ಯಾನಿಯ
ಯಾಮಿ
ಯಾಯಾತಿ
ಯಾಯಿತು
ಯಾರ
ಯಾರನ್ನು
ಯಾರನ್ನೂ
ಯಾರಾ-ದರೂ
ಯಾರಿ
ಯಾರಿಂದ
ಯಾರಿಂ-ದಲೂ
ಯಾರಿ-ಗಾಗಿ
ಯಾರಿಗೂ
ಯಾರಿಗೆ
ಯಾರೀತ
ಯಾರು
ಯಾರು-ಯಾ-ರಿಗೆ
ಯಾರೂ
ಯಾರೆ
ಯಾರೆಂದು
ಯಾರೆಂ-ಬು-ದನ್ನು
ಯಾರೇನು
ಯಾರೊ
ಯಾರೊ-ಡ-ನೆಯೂ
ಯಾರೊ-ಬ್ಬ-ರಿಗೂ
ಯಾರೊ-ಬ್ಬ-ರೊ-ಡನೆ
ಯಾರೋ
ಯಾವ
ಯಾವನ
ಯಾವ-ನನ್ನು
ಯಾವನು
ಯಾವನೆ
ಯಾವನೊ
ಯಾವನೋ
ಯಾವ-ಯಾವ
ಯಾವಳು
ಯಾವಾ
ಯಾವಾಗ
ಯಾವಾ-ಗ-ಲಾ-ದರೂ
ಯಾವಾ-ಗಲೂ
ಯಾವು-ದಕ್ಕೂ
ಯಾವು-ದನ್ನೂ
ಯಾವು-ದರ
ಯಾವುದಾ
ಯಾವು-ದಾದ
ಯಾವು-ದಾ-ದರೂ
ಯಾವು-ದಿದೆ
ಯಾವು-ದಿ-ದ್ದರೂ
ಯಾವುದು
ಯಾವುದೂ
ಯಾವು-ದೆಂದು
ಯಾವುದೊ
ಯಾವುದೋ
ಯಾಽವ್ಯ-ಕ್ತಾಯ
ಯಿಂದ
ಯಿಂದಲೂ
ಯಿಂದಲೇ
ಯಿಂದಿರು
ಯಿಟ್ಟನು
ಯಿಟ್ಟು
ಯಿಡು-ತ್ತಿ-ರು-ವುದನ್ನು
ಯಿಡು-ವುದು
ಯಿತು
ಯಿತ್ತನು
ಯಿತ್ತರು
ಯಿಲ್ಲ
ಯಿಲ್ಲದ
ಯಿಲ್ಲ-ದಂತೆ
ಯಿಲ್ಲದೆ
ಯಿಸು-ವದೋ
ಯುಕ್ತ
ಯುಕ್ತ-ನೆ-ನ್ನ-ಬೇ-ಕಾ-ಯಿತು
ಯುಕ್ತಾ-ಯುಕ್ತ
ಯುಕ್ತಿ-ಯಿಂದ
ಯುಕ್ತಿಯೂ
ಯುಗ
ಯುಗ-ಗ-ಳಾದ
ಯುಗ-ಗ-ಳಿ-ಗಿಂ-ತಲೂ
ಯುಗ-ತ್ರ-ಯ-ಸ್ವ-ರೂ-ಪ-ನಾದ
ಯುಗ-ದಂತೆ
ಯುಗ-ದಲ್ಲಿ
ಯುಗ-ಧ-ರ್ಮಕ್ಕೆ
ಯುಗ-ವಾಗಿ
ಯುಗಾಂ-ತಾ-ನಲ
ಯುತ್ತ
ಯುತ್ತದೆ
ಯುತ್ತಾ
ಯುತ್ತಿದ್ದ
ಯುತ್ತಿ-ದ್ದೇವೆ
ಯುತ್ತಿ-ರುವ
ಯುತ್ತೇನೆ
ಯುದ್ದಕ್ಕೆ
ಯುದ್ಧ
ಯುದ್ಧ-ಕ-ವ-ಚ-ವನ್ನು
ಯುದ್ಧಕ್ಕೂ
ಯುದ್ಧಕ್ಕೆ
ಯುದ್ಧ-ದಲ್ಲಿ
ಯುದ್ಧ-ದ-ಲ್ಲಿಯೂ
ಯುದ್ಧ-ಭಿ-ಕ್ಷೆಗೆ
ಯುದ್ಧ-ಭಿ-ಕ್ಷೆ-ಯನ್ನು
ಯುದ್ಧ-ಭೂಮಿ
ಯುದ್ಧ-ಭೂ-ಮಿ-ಯಿಂದ
ಯುದ್ಧ-ಮಾ-ಡ-ಬ-ಲ್ಲ-ವನು
ಯುದ್ಧ-ಮಾ-ಡ-ಬೇಕು
ಯುದ್ಧ-ಮಾ-ಡ-ಲಾ-ರದೆ
ಯುದ್ಧ-ಮಾ-ಡಲು
ಯುದ್ಧ-ಮಾಡಿ
ಯುದ್ಧ-ಮಾಡು
ಯುದ್ಧ-ಮಾ-ಡು-ವುದು
ಯುದ್ಧ-ರಂ-ಗಕ್ಕೆ
ಯುದ್ಧ-ರಂ-ಗ-ದಲ್ಲಿ
ಯುದ್ಧ-ರಂ-ಗ-ದಿಂದ
ಯುದ್ಧ-ವ-ನ್ನಾ-ದರೂ
ಯುದ್ಧ-ವನ್ನು
ಯುದ್ಧ-ವಾಗಿ
ಯುದ್ಧ-ವೆಂ-ಬುದು
ಯುದ್ಧ-ವೆಲ್ಲ
ಯುದ್ಧೋ-ತ್ಸಾ-ಹ-ದಿಂದ
ಯುಧಿ-ಷ್ಠಿ-ರ-ನೆಂಬ
ಯುಮು-ನಾ-ತೀ-ರಕ್ಕೆ
ಯುಮು-ನಾ-ದಿ-ಯ-ಲ್ಲಿದ್ದ
ಯುಳ್ಳ-ವನು
ಯುಳ್ಳುದು
ಯುವ
ಯುವಕ
ಯುವ-ಕ-ನನ್ನು
ಯುವ-ಕ-ನಾಗಿ
ಯುವ-ಕ-ನಾದ
ಯುವ-ಕ-ನಾ-ದನು
ಯುವ-ಕ-ನೊ-ಬ್ಬನು
ಯುವ-ಕ-ರಲ್ಲಿ
ಯುವ-ಕ-ರಾಗಿ
ಯುವ-ಕ-ರಿಗೆ
ಯುವ-ನಾ-ಶ್ವ-ನಿಗೆ
ಯುವ-ನಾ-ಶ್ವನು
ಯುವ-ನಾ-ಶ್ವ-ನೆಂಬ
ಯುವು-ದಕ್ಕೆ
ಯುವು-ದೇನು
ಯೆಂದು
ಯೆಂಬ
ಯೆಂಬು-ದನ್ನು
ಯೆಂಬು-ದ-ರಲ್ಲಿ
ಯೆಂಬುದೇ
ಯೆತ್ತಿ
ಯೆತ್ತಿ-ದನು
ಯೆತ್ತು-ವಂತೆ
ಯೆರೆ-ದು-ಕೊಟ್ಟು
ಯೆಲ್ಲ
ಯೆಲ್ಲಿತ್ತು
ಯೇ
ಯೇನ
ಯೇನಾ-ಗಿದೆ
ಯೊಂದಿ-ದ್ದರೆ
ಯೊಂದು
ಯೊಡನೆ
ಯೊಬ್ಬ
ಯೊಬ್ಬನ
ಯೊಳಗೆ
ಯೋ
ಯೋಗ
ಯೋಗ-ಕ್ಕಿಂ-ತಲೂ
ಯೋಗ-ಕ್ಷೆ-ಮ-ವ-ನ್ನೆಲ್ಲ
ಯೋಗ-ಕ್ಷೇ-ಮ-ವನ್ನು
ಯೋಗ-ಕ್ಷೇ-ಮ-ವನ್ನೂ
ಯೋಗ-ಗಳ
ಯೋಗ-ದಿಂದ
ಯೋಗ-ದೃ-ಷ್ಟಿ-ಯಿಂದ
ಯೋಗ-ಧಾ-ರ-ಣೆ-ಯಿಂದ
ಯೋಗ-ನಾಥಃ
ಯೋಗ-ನಿ-ದ್ರೆ-ಯಲ್ಲಿ
ಯೋಗ-ನಿ-ದ್ರೆ-ಯಿಂದ
ಯೋಗ-ನಿ-ರ-ತ-ನಾದ
ಯೋಗ-ಪ್ರ-ಭಾ-ವ-ದಿಂದ
ಯೋಗ-ಪ್ರ-ಸ್ಥಾ-ನದ
ಯೋಗ-ಮ-ಹಿ-ಮೆ-ಯಿಂದ
ಯೋಗ-ಮಾ-ಯೆ-ಯನ್ನು
ಯೋಗ-ಮಾ-ಯೆ-ಯಿಂದ
ಯೋಗ-ಮಾ-ರ್ಗ-ದಿಂದ
ಯೋಗ-ಮಾ-ರ್ಗ-ವನ್ನು
ಯೋಗ-ಮು-ದ್ರೆ-ಯಲ್ಲಿ
ಯೋಗ-ವನ್ನು
ಯೋಗ-ವಿ-ದ್ಯೆಯ
ಯೋಗವೂ
ಯೋಗ-ಶಕ್ತಿ
ಯೋಗ-ಶ-ಕ್ತಿಯ
ಯೋಗ-ಶ-ಕ್ತಿ-ಯಿಂದ
ಯೋಗ-ಸಿದ್ಧಿ
ಯೋಗ-ಸಿ-ದ್ಧಿ-ಯನ್ನು
ಯೋಗಾ-ಗ್ನಿ-ಯಿಂದ
ಯೋಗಾ-ದೇ-ಶ-ವೆಂದು
ಯೋಗಾ-ಭ್ಯಾಸ
ಯೋಗಾ-ಭ್ಯಾ-ಸ-ದಿಂದ
ಯೋಗಾ-ಭ್ಯಾ-ಸ-ನಿ-ರ-ತ-ನಾ-ದ-ವನು
ಯೋಗಾ-ರೂ-ಢ-ನಾಗಿ
ಯೋಗಾ-ಸ-ನ-ದಲ್ಲಿ
ಯೋಗಿ
ಯೋಗಿ-ಗಳ
ಯೋಗಿ-ಗಳಲ್ಲಿ
ಯೋಗಿ-ಗ-ಳಿಗೂ
ಯೋಗಿ-ಗ-ಳಿಗೆ
ಯೋಗಿ-ಗಳು
ಯೋಗಿ-ಗ-ಳೆ-ನಿ-ಸಿ-ಕೊಂಡ
ಯೋಗಿಗೆ
ಯೋಗಿ-ಯಾದ
ಯೋಗಿ-ಯಾ-ದ-ವ-ನನ್ನು
ಯೋಗಿ-ಯಾ-ದ-ವನು
ಯೋಗಿಯು
ಯೋಗೀ-ಶ್ವರ
ಯೋಗೇ
ಯೋಗೇ-ಶ್ವ-ರ-ನಾದ
ಯೋಗ್ಯ
ಯೋಗ್ಯತೆ
ಯೋಗ್ಯ-ತೆಗೆ
ಯೋಗ್ಯ-ತೆ-ಯಲ್ಲಿ
ಯೋಗ್ಯ-ತೆ-ಯಿಲ್ಲ
ಯೋಗ್ಯ-ತೆ-ಯಿ-ಲ್ಲದೆ
ಯೋಗ್ಯ-ನಲ್ಲ
ಯೋಗ್ಯ-ನ-ಲ್ಲ-ಎಂದು
ಯೋಗ್ಯ-ನಾ-ದರೆ
ಯೋಗ್ಯ-ಮಾ-ರ್ಗ-ದಲ್ಲಿ
ಯೋಗ್ಯ-ರಲ್ಲ
ಯೋಗ್ಯ-ರೀತಿ
ಯೋಗ್ಯರು
ಯೋಗ್ಯ-ವಾದ
ಯೋಗ್ಯ-ವಾ-ದುದೆ
ಯೋಗ್ಯವೂ
ಯೋಚ-ನಾ-ಮಗ್ನ
ಯೋಚನೆ
ಯೋಚ-ನೆಗೂ
ಯೋಚಿ-ಸ-ಬೇಡ
ಯೋಚಿಸಿ
ಯೋಚಿ-ಸಿ-ದನು
ಯೋಚಿ-ಸಿ-ದರೂ
ಯೋಚಿ-ಸಿ-ದ್ದಾನೆ
ಯೋಚಿಸು
ಯೋಚಿ-ಸುತ್ತಾ
ಯೋಚಿ-ಸು-ತ್ತಾನೆ
ಯೋಚಿ-ಸು-ತ್ತಿ-ದ್ದಂ-ತೆಯೆ
ಯೋಚಿ-ಸು-ತ್ತಿ-ರಲು
ಯೋಚಿ-ಸು-ತ್ತಿ-ರು-ವಂ-ತಿದೆ
ಯೋಚಿ-ಸು-ತ್ತಿ-ರು-ವಾಗ
ಯೋಚಿ-ಸು-ವ-ವ-ನಂತೆ
ಯೋಚಿ-ಸು-ವು-ದು-ಹೀ-ಗಾಗಿ
ಯೋಜನ
ಯೋಜ-ನ-ಗಳ
ಯೋಜ-ನ-ದಲ್ಲಿ
ಯೋಜ-ನ-ದಷ್ಟು
ಯೋಜ-ನ-ದಾಚೆ
ಯೋಜ-ನೆ-ಗ-ಳಷ್ಟು
ಯೋಜ-ನೆ-ಯೊಂ-ದನ್ನು
ಯೋಧರು
ಯೋಭಿ
ಯೋಽನಿ-ಶ-ಮೇವ
ಯೌಜಸೇ
ಯೌವನ
ಯೌವ-ನ-ಎಂದು
ಯೌವ-ನ-ಗಳನ್ನು
ಯೌವ-ನ-ದಿಂದ
ಯೌವ-ನಾ-ವ-ಸ್ಥೆಯ
ಯೌವ-ನೆ-ಯ-ರಾದ
ಯ್ದನು
ಯ್ಯಾಲೆ-ಯಂ-ತಾ-ಗು-ತಿತ್ತು
ರಂಗ
ರಂಗ-ವಲ್ಲಿ
ರಂಗ-ವ-ಲ್ಲಿ-ಗಳ
ರಂಗ-ವ-ಲ್ಲಿ-ಗಳಿಂದ
ರಂಗ-ಸ-ಜ್ಜಿಕೆ
ರಂಗು-ರಂ-ಗಾಗಿ
ರಂತಿ
ರಂತಿ-ದೇವ
ರಂತಿ-ದೇ-ವನ
ರಂತಿ-ದೇ-ವನು
ರಂತೆ
ರಂಧಯ
ರಂಧ್ರಕ್ಕೆ
ರಂಧ್ರ-ದಲ್ಲಿ
ರಂಧ್ರ-ದೊ-ಳ-ಗಿ-ನಿಂದ
ರಂಧ್ರ-ವನ್ನು
ರಕ್ಕಸ
ರಕ್ಕ-ಸ-ದೇ-ಹದ
ರಕ್ಕ-ಸನ
ರಕ್ಕ-ಸ-ನನ್ನು
ರಕ್ಕ-ಸ-ನನ್ನೂ
ರಕ್ಕ-ಸ-ನಿಗೆ
ರಕ್ಕ-ಸ-ನಿದ್ದ
ರಕ್ಕ-ಸನು
ರಕ್ಕ-ಸನೇ
ರಕ್ಕ-ಸ-ನೊ-ಬ್ಬನು
ರಕ್ಕ-ಸರ
ರಕ್ಕ-ಸ-ರನ್ನು
ರಕ್ಕ-ಸ-ರ-ನ್ನೆಲ್ಲ
ರಕ್ಕ-ಸ-ರಿಗೆ
ರಕ್ಕ-ಸರು
ರಕ್ಕ-ಸರೂ
ರಕ್ಕಸಿ
ರಕ್ಕ-ಸಿಯ
ರಕ್ತ
ರಕ್ತ-ಗ-ತ-ವಾ-ಗಿವೆ
ರಕ್ತ-ದಿಂದ
ರಕ್ತ-ಮ-ಯ-ವಾ-ದವು
ರಕ್ಷಕ
ರಕ್ಷ-ಕ-ನಾ-ಗಿದ್ದ
ರಕ್ಷ-ಕ-ನಾ-ಗಿದ್ದು
ರಕ್ಷ-ಕ-ನಾ-ಗಿ-ರಲು
ರಕ್ಷ-ಕ-ನಾದ
ರಕ್ಷ-ಕನು
ರಕ್ಷ-ಕ-ರಾ-ಗಿ-ದ್ದ-ವರೇ
ರಕ್ಷ-ಕ-ರಾದ
ರಕ್ಷಣ
ರಕ್ಷ-ಣ-ಕಾ-ರ್ಯ-ದಲ್ಲಿ
ರಕ್ಷಣೆ
ರಕ್ಷ-ಣೆ-ಇವು
ರಕ್ಷ-ಣೆ-ಗಾಗಿ
ರಕ್ಷ-ಣೆಗೆ
ರಕ್ಷ-ಣೆಯ
ರಕ್ಷ-ಣೆ-ಯನ್ನು
ರಕ್ಷ-ಣೆ-ಯಲ್ಲಿ
ರಕ್ಷ-ಣೆ-ಯ-ಲ್ಲಿ-ರುವ
ರಕ್ಷ-ಣೆ-ಯ-ಲ್ಲಿ-ರು-ವು-ದ-ರಿಂ-ದಲೆ
ರಕ್ಷ-ಣೆ-ಯಾ-ಗ-ದಿ-ದ್ದರೆ
ರಕ್ಷತ
ರಕ್ಷತು
ರಕ್ಷ-ತ್ವ-ಶೇ-ಷ-ಕೃಚ್ಛೆ
ರಕ್ಷ-ತ್ವಸೌ
ರಕ್ಷಾ-ಮಣಿ
ರಕ್ಷಿ
ರಕ್ಷಿ-ಸ-ಬೇ-ಕ-ಲ್ಲವೆ
ರಕ್ಷಿ-ಸ-ಬೇಕೇ
ರಕ್ಷಿ-ಸ-ಬೇಡಿ
ರಕ್ಷಿ-ಸಲಿ
ರಕ್ಷಿಸಿ
ರಕ್ಷಿ-ಸಿದ
ರಕ್ಷಿ-ಸಿ-ದನು
ರಕ್ಷಿ-ಸಿ-ದ್ದರು
ರಕ್ಷಿ-ಸಿ-ರುವೆ
ರಕ್ಷಿಸು
ರಕ್ಷಿ-ಸು-ತ್ತಿ-ರಲಿ
ರಕ್ಷಿ-ಸು-ತ್ತಿರು
ರಕ್ಷಿ-ಸು-ತ್ತೀಯೆ
ರಕ್ಷಿ-ಸು-ತ್ತೇನೆ
ರಕ್ಷಿ-ಸುವ
ರಕ್ಷಿ-ಸು-ವಂತೆ
ರಕ್ಷಿ-ಸು-ವ-ವನು
ರಕ್ಷಿ-ಸು-ವ-ವ-ರಾರು
ರಕ್ಷಿ-ಸು-ವ-ವ-ರಿ-ನ್ನಾರು
ರಕ್ಷಿ-ಸುವು
ರಕ್ಷಿ-ಸು-ವು-ದ-ಕ್ಕಾಗಿ
ರಕ್ಷಿ-ಸು-ವು-ದ-ಕ್ಕಾ-ಗಿಯೇ
ರಕ್ಷಿ-ಸು-ವುದು
ರಕ್ಷಿ-ಸುವೆ
ರಕ್ಷೆ
ರಕ್ಷೆಗೆ
ರಕ್ಷೆ-ಯನ್ನು
ರಘು
ರಘು-ವಿನ
ರಚ-ನಾ-ಕಾ-ಲ-ವನ್ನು
ರಚನೆ
ರಚ-ನೆ-ಯಾ-ಗಿ-ರ-ಬೇಕು
ರಚಿ-ತ-ವಾ-ಗಿ-ರ-ಬೇ-ಕೆಂದೂ
ರಚಿಸಿ
ರಚಿ-ಸಿದ
ರಚಿ-ಸಿ-ದನು
ರಜ
ರಜಸ್
ರಜ-ಸ್ತ-ಮೋ-ಗು-ಣ-ವು-ಳ್ಳ-ವರು
ರಜೋ-ಗುಣ
ರಜೋ-ಗು-ಣಕ್ಕೆ
ರಜೋ-ಗು-ಣ-ಗ-ಳಿಗೆ
ರಜೋ-ಗು-ಣ-ಪ್ರ-ಧಾ-ನ-ರಾಗಿ
ರಜೋ-ಗು-ಣ-ವನ್ನು
ರಜೋ-ಗು-ಣವೆ
ರಣ-ಭೂಮಿ
ರಣ-ಭೂ-ಮಿ-ಯಲ್ಲಿ
ರಣ-ರಂ-ಗಕ್ಕೆ
ರಣ-ರಂ-ಗ-ದಲ್ಲಿ
ರತಿ
ರತಿ-ಕ್ರೀ-ಡೆ-ಗೆಂದು
ರತಿ-ಕ್ರೀ-ಡೆಯ
ರತಿ-ಮು-ದ್ವ-ಹ-ತಾ-ದದ್ಧಾ
ರತಿ-ಯನ್ನು
ರತಿ-ಯ-ರಷ್ಟು
ರತಿ-ಸು-ಖಕ್ಕೆ
ರತಿ-ಸು-ಖ-ವನ್ನು
ರತ್ತ
ರತ್ನ
ರತ್ನ-ಕಿ-ರೀಟ
ರತ್ನ-ಕುಂ-ಡ-ಲ-ಗಳನ್ನು
ರತ್ನ-ಕ್ಕಾಗಿ
ರತ್ನ-ಕ್ಕಾ-ಗಿಯೂ
ರತ್ನಕ್ಕೆ
ರತ್ನ-ಖ-ಚಿ-ತ-ಗ-ಳಾದ
ರತ್ನ-ಖ-ಚಿ-ತ-ವಾದ
ರತ್ನ-ಗಳ
ರತ್ನ-ಗಳನ್ನು
ರತ್ನ-ಗಳಿಂದ
ರತ್ನ-ಗ-ಳು-ಅವೇ
ರತ್ನ-ಗ-ಳೊ-ಡನೆ
ರತ್ನದ
ರತ್ನ-ದವು
ರತ್ನ-ದಿಂದ
ರತ್ನ-ರಾ-ಶಿ-ಗಳನ್ನೂ
ರತ್ನ-ವನ್ನು
ರತ್ನ-ವನ್ನೂ
ರತ್ನ-ವಲ್ಲ
ರತ್ನ-ವಿ-ಲ್ಲ-ದು-ದನ್ನು
ರತ್ನ-ವೆಂದರೆ
ರತ್ನ-ವೇನೂ
ರತ್ನ-ಸಿಂ-ಹಾ-ಸ-ನ-ದಲ್ಲಿ
ರತ್ನಾ-ಭ-ರಣ
ರತ್ನಾ-ಭ-ರ-ಣ-ಗಳಿಂದ
ರಥ
ರಥಕ್ಕೆ
ರಥ-ಗಳನ್ನು
ರಥ-ಗಳು
ರಥ-ಗಳೂ
ರಥ-ಗ-ಳೆ-ರಡೂ
ರಥದ
ರಥ-ದಂತೆ
ರಥ-ದಲ್ಲಿ
ರಥ-ದ-ಲ್ಲಿ-ಡ-ಬೇಕು
ರಥ-ದ-ಲ್ಲಿಯೆ
ರಥ-ದಿಂದ
ರಥ-ದಿಂ-ದಿ-ಳಿದು
ರಥ-ದೊ-ಳಕ್ಕೆ
ರಥ-ವ-ನ್ನಿ-ಳಿದು
ರಥ-ವನ್ನು
ರಥ-ವನ್ನೂ
ರಥ-ವ-ನ್ನೇರಿ
ರಥ-ವ-ನ್ನೇ-ರಿ-ದನು
ರಥ-ವ-ನ್ನೇರು
ರಥ-ವ-ನ್ನೇ-ರು-ತ್ತಿ-ದ್ದಂ-ತೆಯೆ
ರಥವು
ರಥ-ವೆಂಬ
ರಥ-ವೇರಿ
ರಥ-ವೇ-ರಿ-ದರು
ರಥ-ವೊಂದು
ರಥಿ-ಕ-ನಾಗಿ
ರನು-ನಯ
ರನ್ನದ
ರನ್ನು
ರನ್ನೂ
ರನ್ನೆಲ್ಲ
ರಭ-ಸಕ್ಕೆ
ರಭ-ಸ-ದಿಂದ
ರಭ-ಸ-ವನ್ನು
ರಮಣ
ರಮ-ಣಕ
ರಮ-ಣ-ರೊ-ಡನೆ
ರಮ-ಣಾ-ಭಿ-ಧಾನ
ರಮಣಿ
ರಮ-ಣೀಯ
ರಮ-ಣೀ-ಯ-ವಾ-ಗಿತ್ತು
ರಮಾ
ರಮಾ-ದೇವಿ
ರಮಾ-ದೇ-ವಿ-ಯಂತೆ
ರಮಾ-ದೇ-ವಿ-ಯನ್ನೂ
ರಮಾ-ಧಿ-ಪ-ತಿಂ
ರಮಾ-ನಂದ
ರಮಾ-ಪ-ತ್ಯ-ಷ್ಟ-ಕಮ್
ರಮಿ-ಸ-ಕೂ-ಡದು
ರಮಿ-ಸ-ತ-ಕ್ಕ-ವನು
ರಮಿಸಿ
ರಮಿ-ಸಿ-ದಳು
ರಮಿ-ಸಿ-ದೊ-ಡನೆ
ರಮಿ-ಸುತ್ತಾ
ರಮಿ-ಸು-ತ್ತಿ-ದ್ದಳು
ರಮಿ-ಸು-ತ್ತಿ-ರು-ವ-ಳೆಂ-ಬು-ದನ್ನು
ರಮಿ-ಸು-ವಂತೆ
ರಮ್ಯ-ಕ-ವರ್ಷ
ರಮ್ಯ-ಕ-ವ-ರ್ಷ-ದಲ್ಲಿ
ರಮ್ಯ-ವಾ-ಗಿತ್ತು
ರಮ್ಯ-ವಾದ
ರಲು
ರಲ್ಲ
ರಲ್ಲ-ವೆಂ-ದು-ಕೊಂ-ಡರು
ರಲ್ಲಾ
ರಲ್ಲಿ
ರಲ್ಲಿ-ಮಾ-ವಿನ
ರಲ್ಲಿಯೂ
ರವಿ-ಭಾನು
ರಷ್ಟು
ರಸ
ರಸ-ಕ-ವ-ಳ-ವ-ನ್ನಿತ್ತು
ರಸ-ಚಿ-ತ್ರ-ಗಳು
ರಸ-ಜೀ-ವ-ನ-ವನ್ನು
ರಸ-ದಂತೆ
ರಸ-ನಿ-ಮಿ-ಷ-ಗಳು
ರಸ-ಭ-ರಿ-ತ-ವಾದ
ರಸ-ಭಾ-ಸ-ಮಾ-ಡು-ತ್ತಾರೊ
ರಸ-ಮಯ
ರಸ-ರೂ-ಪ-ವಾಗಿ
ರಸ-ವ-ತ್ತಾ-ಗು-ವು-ದೆಂದು
ರಸ-ವಾ-ಗು-ತ್ತಾನೆ
ರಸಾ-ತ-ಲ-ವನ್ನು
ರಸಾ-ತಳ
ರಸಾ-ಭಾ-ಸ-ವನ್ನು
ರಸಾಯಾ
ರಸಾ-ಸ್ವಾ-ದದ
ರಸಿಕ
ರಸಿ-ಕ-ರ-ಸ-ಪು-ಷಿ-ಯಾದ
ರಸಿ-ಕರು
ರಸ್ತೆ-ಗಳು
ರಸ್ಸ-ನೊ-ಡನೆ
ರಹಸ್ಯ
ರಹ-ಸ್ಯ-ವಾಗಿ
ರಹ-ಸ್ಯ-ವಾ-ಗಿ-ಟ್ಟುಕೊ
ರಹ-ಸ್ಯ-ವಾದ
ರಹಿ-ತನೂ
ರಹಿ-ತ-ವಾದ
ರಹೂ-ಗ-ಣನ
ರಹೂ-ಗ-ಣನು
ರಾ
ರಾಂ
ರಾಕ್ಷಸ
ರಾಕ್ಷ-ಸನ
ರಾಕ್ಷ-ಸ-ನಂತೆ
ರಾಕ್ಷ-ಸ-ನದು
ರಾಕ್ಷ-ಸ-ನನ್ನು
ರಾಕ್ಷ-ಸ-ನಲ್ಲ
ರಾಕ್ಷ-ಸ-ನಾಗಿ
ರಾಕ್ಷ-ಸ-ನಾ-ಗಿದ್ದ
ರಾಕ್ಷ-ಸ-ನಾ-ಗಿ-ರುವೆ
ರಾಕ್ಷ-ಸ-ನಾ-ಗೆಂದು
ರಾಕ್ಷ-ಸ-ನಾದ
ರಾಕ್ಷ-ಸ-ನಿಗೆ
ರಾಕ್ಷ-ಸನು
ರಾಕ್ಷ-ಸ-ನೊ-ಬ್ಬನು
ರಾಕ್ಷ-ಸ-ಬಲ
ರಾಕ್ಷ-ಸ-ಮಾಯೆ
ರಾಕ್ಷ-ಸ-ಮಾ-ಯೆ-ಯನ್ನು
ರಾಕ್ಷ-ಸರ
ರಾಕ್ಷ-ಸ-ರಂತೂ
ರಾಕ್ಷ-ಸ-ರನ್ನು
ರಾಕ್ಷ-ಸ-ರಾಗಿ
ರಾಕ್ಷ-ಸ-ರಾಜ
ರಾಕ್ಷ-ಸ-ರಾ-ಜನ
ರಾಕ್ಷ-ಸ-ರಾ-ಜ-ನಾದ
ರಾಕ್ಷ-ಸ-ರಿಗೂ
ರಾಕ್ಷ-ಸ-ರಿಗೆ
ರಾಕ್ಷ-ಸರು
ರಾಕ್ಷ-ಸರೂ
ರಾಕ್ಷ-ಸ-ರೂ-ಪ-ದಿಂದ
ರಾಕ್ಷ-ಸ-ರೊ-ಬ್ಬರೂ
ರಾಕ್ಷ-ಸ-ವಂ-ಶಕ್ಕೆ
ರಾಕ್ಷ-ಸ-ವೀ-ರ-ರನ್ನೂ
ರಾಕ್ಷ-ಸ-ವೀ-ರ-ರೆಲ್ಲ
ರಾಕ್ಷ-ಸ-ಸೈ-ನ್ಯ-ದೊ-ಡನೆ
ರಾಕ್ಷ-ಸಿಯ
ರಾಗ
ರಾಗ-ದ್ವೇ-ಷ-ಗಳನ್ನು
ರಾಗ-ದ್ವೇ-ಷ-ಗ-ಳಿ-ಲ್ಲದ
ರಾಗ-ದ್ವೇ-ಷಾದಿ
ರಾಗ-ದ್ವೇ-ಷಾ-ದಿ-ಗಳನ್ನು
ರಾಗ-ಬೇಕು
ರಾಗ-ಭೋ-ಗ-ಗಳಲ್ಲಿ
ರಾಗ-ರ-ಹಿ-ತನೂ
ರಾಗ-ರಾ-ಗ-ವಾಗಿ
ರಾಗ-ವ-ನ್ನಿ-ಡ-ಬೇಕು
ರಾಗಿ
ರಾಗಿ-ದ್ದರು
ರಾಗಿ-ದ್ದ-ವರು
ರಾಗಿ-ದ್ದಾರೆ
ರಾಗಿದ್ದು
ರಾಗಿ-ಯಾ-ದರೂ
ರಾಗಿ-ರು-ವರು
ರಾಗು-ತ್ತಾರೆ
ರಾಜ
ರಾಜ-ಆಳು
ರಾಜ-ಕ-ನ್ಯೆ-ಯರು
ರಾಜ-ಕ-ನ್ಯೆ-ಯಾ-ದು-ದ-ರಿಂದ
ರಾಜ-ಕಾರಣ-ಪ-ಟು-ತ್ವ-ಆ-ತನ
ರಾಜ-ಕಾ-ರಣಿ
ರಾಜ-ಕಾ-ರ್ಯ-ಗಳಲ್ಲಿ
ರಾಜ-ಕೀ-ಯಕ್ಕೆ
ರಾಜ-ಕು-ಮಾರ
ರಾಜ-ಕು-ಮಾ-ರ-ನನ್ನು
ರಾಜ-ಕು-ಮಾ-ರ-ನಾದ
ರಾಜ-ಕು-ಮಾ-ರ-ನಿಗೆ
ರಾಜ-ಕು-ಮಾ-ರನು
ರಾಜ-ಕು-ಮಾ-ರ-ರಿಗೂ
ರಾಜ-ಕು-ಮಾ-ರರು
ರಾಜ-ಕು-ಮಾರಿ
ರಾಜ-ಕು-ಮಾ-ರಿ-ಯರ
ರಾಜ-ಕು-ಮಾ-ರಿ-ಯ-ರಿದ್ದ
ರಾಜ-ಕು-ಮಾ-ರಿ-ಯರು
ರಾಜ-ಕು-ಮಾ-ರಿ-ಯಾದ
ರಾಜ-ಕು-ಮಾ-ರಿಯೇ
ರಾಜ-ತ್ವ-ವನ್ನು
ರಾಜ-ದೂ-ತ-ನನ್ನು
ರಾಜ-ಧರ್ಮ
ರಾಜ-ಧ-ರ್ಮ-ವನ್ನು
ರಾಜ-ಧ-ರ್ಮ-ವನ್ನೂ
ರಾಜ-ಧಾನಿ
ರಾಜ-ಧಾ-ನಿ-ಗ-ಳಿವೆ
ರಾಜ-ಧಾ-ನಿಗೆ
ರಾಜ-ಧಾ-ನಿಯ
ರಾಜ-ಧಾ-ನಿ-ಯನ್ನು
ರಾಜ-ಧಾ-ನಿ-ಯ-ಲ್ಲಿದ್ದ
ರಾಜ-ಧಾ-ನಿ-ಯ-ಲ್ಲಿ-ರು-ವ-ವರು
ರಾಜ-ಧಾ-ನಿ-ಯಾಗಿ
ರಾಜ-ಧಾ-ನಿ-ಯಾದ
ರಾಜ-ಧಾ-ನಿ-ಯಾ-ಯಿತು
ರಾಜ-ಧಾ-ನಿ-ಯೆ-ಲ್ಲವೂ
ರಾಜ-ಧಾ-ನಿ-ಯೊಂ-ದನ್ನು
ರಾಜನ
ರಾಜ-ನಂತೂ
ರಾಜ-ನತ್ತ
ರಾಜ-ನ-ದಲ್ಲ
ರಾಜ-ನ-ನ್ನಾಗಿ
ರಾಜ-ನನ್ನು
ರಾಜ-ನಾ-ಗ-ಬೇ-ಕೆಂಬ
ರಾಜ-ನಾ-ಗ-ಲೆಂದು
ರಾಜ-ನಾಗಿ
ರಾಜ-ನಾ-ಗಿದ್ದ
ರಾಜ-ನಾ-ಗಿದ್ದೆ
ರಾಜ-ನಾ-ಗಿ-ರು-ವಾಗ
ರಾಜ-ನಾಗು
ರಾಜ-ನಾ-ಗು-ವನು
ರಾಜ-ನಾದ
ರಾಜ-ನಾ-ದರೂ
ರಾಜ-ನಾ-ದ-ವನು
ರಾಜ-ನಾರು
ರಾಜ-ನಿಂದ
ರಾಜ-ನಿಗೂ
ರಾಜ-ನಿಗೆ
ರಾಜ-ನಿತ್ತ
ರಾಜ-ನಿದ್ದ
ರಾಜ-ನಿ-ಲ್ಲದೆ
ರಾಜ-ನೀತಿ
ರಾಜನು
ರಾಜನೂ
ರಾಜನೆ
ರಾಜ-ನೆಂ-ದರೆ
ರಾಜ-ನೆಂದು
ರಾಜ-ನೆಂಬ
ರಾಜನೇ
ರಾಜ-ನೊ-ಡನೆ
ರಾಜ-ನೊ-ಬ್ಬ-ನಿ-ದ್ದ-ನಂತೆ
ರಾಜ-ಪ-ದ-ವಿ-ಯನ್ನು
ರಾಜ-ಪ-ದ-ವಿ-ಯಲ್ಲಿ
ರಾಜ-ಪ-ದ-ವಿ-ಯಿಂದ
ರಾಜ-ಪು-ತ್ರನು
ರಾಜ-ಪುತ್ರಿ
ರಾಜ-ಪು-ತ್ರಿ-ಯ-ರನ್ನು
ರಾಜ-ಪು-ತ್ರಿ-ಯರೂ
ರಾಜ-ಪು-ತ್ರಿ-ಯಾದ
ರಾಜ-ಪುಷಿ
ರಾಜ-ಪು-ಷಿಗೆ
ರಾಜ-ಪು-ಷಿ-ಯೊ-ಬ್ಬನು
ರಾಜ-ಬೀ-ದಿ-ಗಳ
ರಾಜ-ಬೀ-ದಿ-ಯಲ್ಲಿ
ರಾಜ-ಭ-ಟರ
ರಾಜ-ಭ-ಟರು
ರಾಜ-ಭೋಗ
ರಾಜ-ಭೋ-ಗ-ಗಳನ್ನು
ರಾಜ-ಮ-ಹಾ-ರಾ-ಜರು
ರಾಜ-ಮಾ-ರ್ಗ-ದಲ್ಲಿ
ರಾಜ-ಯೋ-ಗ್ಯ-ವಾದ
ರಾಜರ
ರಾಜ-ರನ್ನು
ರಾಜ-ರ-ನ್ನೆಲ್ಲ
ರಾಜ-ರ-ಲ್ಲೆಲ್ಲ
ರಾಜ-ರಾ-ಗಲಿ
ರಾಜ-ರಾ-ಗು-ವಿರಿ
ರಾಜ-ರಾ-ಣಿ-ಯರು
ರಾಜ-ರಾ-ದರು
ರಾಜ-ರಾ-ದ-ವರೇ
ರಾಜ-ರಿಗೆ
ರಾಜ-ರಿ-ಗೆಲ್ಲ
ರಾಜರು
ರಾಜ-ರು-ಗಳನ್ನೆಲ್ಲ
ರಾಜ-ರು-ಗಳನ್ನೆಲ್ಲಾ
ರಾಜ-ರು-ಗ-ಳೆಲ್ಲ
ರಾಜರೂ
ರಾಜರೆ
ರಾಜ-ರೆಲ್ಲ
ರಾಜ-ರೆ-ಲ್ಲರೂ
ರಾಜ-ರ್ಷಿ-ಯಾದ
ರಾಜ-ವಂ-ಶ-ಗಳ
ರಾಜ-ವಂ-ಶ-ಗಳು
ರಾಜ-ವೇ-ಷ-ವನ್ನು
ರಾಜ-ವೈ-ಭವ
ರಾಜಸ
ರಾಜ-ಸ-ಗು-ಣ-ಪ್ರ-ಧಾ-ನ-ವಾಗಿ
ರಾಜ-ಸ-ಭಕ್ತಿ
ರಾಜ-ಸಭೆ
ರಾಜ-ಸ-ಭೆಗೆ
ರಾಜ-ಸ-ಭೆ-ಯಲ್ಲಿ
ರಾಜ-ಸ-ಭೆ-ಯ-ಲ್ಲಿದ್ದ
ರಾಜ-ಸಾ-ಹಂ-ಕಾರ
ರಾಜ-ಸಾ-ಹಂ-ಕಾ-ರ-ದಿಂದ
ರಾಜ-ಸಿಂ-ಹನ
ರಾಜ-ಸೂಯ
ರಾಜ-ಸೂ-ಯ-ಯಾ-ಗ-ಕ್ಕಿದ್ದ
ರಾಜ-ಸೂ-ಯ-ಯಾ-ಗಕ್ಕೆ
ರಾಜ-ಸೂ-ಯ-ಯಾ-ಗ-ವನ್ನು
ರಾಜಾ
ರಾಜಾ-ಧಿ-ದೇವಿ
ರಾಜಾ-ಧಿ-ರಾ-ಜ-ನಾಗಿ
ರಾಜಾ-ಸ್ಥಾ-ನಕ್ಕೆ
ರಾಜೇಂದ್ರ
ರಾಜ್ಯ
ರಾಜ್ಯ-ಕೋ-ಶ-ಗಳನ್ನೆಲ್ಲ
ರಾಜ್ಯ-ಕೋ-ಶ-ಗಳು
ರಾಜ್ಯಕ್ಕೆ
ರಾಜ್ಯ-ಗಳ
ರಾಜ್ಯ-ಗಳನ್ನು
ರಾಜ್ಯ-ಗ-ಳಿಗೆ
ರಾಜ್ಯದ
ರಾಜ್ಯ-ದಲ್ಲಿ
ರಾಜ್ಯ-ದಲ್ಲೆಲ್ಲ
ರಾಜ್ಯ-ದಿಂದ
ರಾಜ್ಯ-ಭಾರ
ರಾಜ್ಯ-ಭಾ-ರಕ್ಕೆ
ರಾಜ್ಯ-ಭಾ-ರದ
ರಾಜ್ಯ-ಭಾ-ರ-ಮಾ-ಡಿದ
ರಾಜ್ಯ-ಭಾ-ರ-ಮಾಡು
ರಾಜ್ಯ-ಭಾ-ರ-ಮಾ-ಡುತ್ತಾ
ರಾಜ್ಯ-ಭಾ-ರ-ಮಾ-ಡು-ತ್ತಿದ್ದ
ರಾಜ್ಯ-ಭಾ-ರ-ಮಾ-ಡು-ತ್ತಿ-ದ್ದನು
ರಾಜ್ಯ-ಭಾ-ರ-ಮಾ-ಡು-ತ್ತಿ-ದ್ದಾಗ
ರಾಜ್ಯ-ಭಾ-ರ-ಮಾ-ಡು-ತ್ತಿದ್ದು
ರಾಜ್ಯ-ಭಾ-ರ-ಮಾ-ಡು-ವನು
ರಾಜ್ಯ-ಭಾ-ರ-ವನ್ನು
ರಾಜ್ಯ-ಭೋ-ಗ-ಗಳನ್ನು
ರಾಜ್ಯ-ಭೋ-ಗ-ಗಳು
ರಾಜ್ಯ-ಭೋ-ಗ-ಗ-ಳೊಂದೂ
ರಾಜ್ಯ-ಭ್ರ-ಷ್ಟ-ನಾದ
ರಾಜ್ಯ-ವನ್ನು
ರಾಜ್ಯ-ವನ್ನೂ
ರಾಜ್ಯ-ವ-ನ್ನೆಲ್ಲ
ರಾಜ್ಯ-ವಾಳಿ
ರಾಜ್ಯ-ವಾ-ಳಿದ
ರಾಜ್ಯ-ವಾ-ಳು-ತ್ತಿ-ದ್ದರು
ರಾಜ್ಯ-ವಾ-ಳು-ವ-ವನು
ರಾಜ್ಯವು
ರಾಜ್ಯ-ವೆಲ್ಲ
ರಾಜ್ಯಾ-ಧಿ-ಕಾ-ರ-ವನ್ನು
ರಾಡು-ತ್ತಿ-ರು-ವಾಗ
ರಾಣಿ
ರಾಣಿಯ
ರಾಣಿ-ಯನ್ನು
ರಾಣಿ-ಯ-ರಂತೆ
ರಾಣಿ-ಯ-ರಿಗೂ
ರಾಣಿ-ಯ-ರಿಗೆ
ರಾಣಿ-ಯ-ರಿ-ಗೆಲ್ಲ
ರಾಣಿ-ಯರು
ರಾಣಿ-ಯರೂ
ರಾಣಿ-ಯ-ರೆ-ನಿ-ಸಿ-ಕೊ-ಳ್ಳು-ತ್ತಾರೆ
ರಾಣಿ-ಯ-ರೆಲ್ಲ
ರಾಣಿ-ಯರೇ
ರಾಣಿ-ಯ-ರೊ-ಡನೆ
ರಾಣಿ-ಯಲ್ಲಿ
ರಾಣಿ-ಯಾದ
ರಾಣಿ-ಯಿಂದ
ರಾಣಿ-ವಾ-ಸ-ದ-ವ-ರನ್ನು
ರಾಣಿ-ವಾ-ಸ-ದ-ವರು
ರಾಣಿ-ವಾ-ಸ-ದ-ವರೂ
ರಾಣಿ-ವಾ-ಸ-ವಂತೂ
ರಾಣೀ-ವಾ-ಸದ
ರಾತನೂ
ರಾತ್ರಿ
ರಾತ್ರಿ-ಗಳನ್ನು
ರಾತ್ರಿ-ಗ-ಳ-ಲ್ಲೆಲ್ಲ
ರಾತ್ರಿ-ಗ-ಳಾ-ಗು-ತ್ತವೆ
ರಾತ್ರಿ-ಗಳೇ
ರಾತ್ರಿಯ
ರಾತ್ರಿ-ಯನ್ನು
ರಾತ್ರಿ-ಯಲ್ಲಿ
ರಾತ್ರಿ-ಯಾ-ಗಲಿ
ರಾತ್ರಿ-ಯಾ-ದಾಗ
ರಾತ್ರಿ-ಯಾ-ಯಿತು
ರಾತ್ರಿ-ಯೆಲ್ಲ
ರಾತ್ರ್ಯಾ-ಮೀ-ಶ್ವರೋ
ರಾದ
ರಾದರು
ರಾದರೂ
ರಾದು-ದನ್ನೂ
ರಾದುದು
ರಾಧ
ರಾಧ-ನೆ-ಯಿಂದ
ರಾಧಾ-ಕೃ-ಷ್ಣ-ಲೀಲೆ
ರಾಮ
ರಾಮ-ಕೃಷ್ಣ
ರಾಮ-ಕೃ-ಷ್ಣ-ರನ್ನು
ರಾಮ-ಕೃ-ಷ್ಣರು
ರಾಮನ
ರಾಮ-ನಂತೆ
ರಾಮ-ನದು
ರಾಮ-ನನ್ನು
ರಾಮನು
ರಾಮನೂ
ರಾಮ-ನೆಂದು
ರಾಮ-ರಾ-ಜ್ಯವು
ರಾಮ-ಲ-ಕ್ಷ್ಮ-ಣರು
ರಾಮ-ಸೇ-ತು-ವೆಗೆ
ರಾಮಾ
ರಾಮಾ-ದಿ-ಗ-ಳಿಂ-ದಲೂ
ರಾಮಾ-ನು-ಜರು
ರಾಮಾ-ನು-ಜಾ-ಚಾ-ರ್ಯರು
ರಾಮಾ-ಯಣ
ರಾಮಾ-ವ-ತಾ-ರ್ಅ
ರಾಮೋ-ಽದ್ರಿ-ಕೂ-ಟೇ-ಷ್ವಥ
ರಾಯನ
ರಾಯನು
ರಾಯ-ಭಾರಿ
ರಾರಿಂ-ದಲೂ
ರಾರು
ರಾರೂ
ರಾವಣ
ರಾವ-ಣನ
ರಾವ-ಣ-ನಿಗೆ
ರಾವ-ಣನು
ರಾವ-ಣನೆ
ರಾವ-ಣ-ನೊ-ಡನೆ
ರಾವ-ಣಾ-ಸು-ರನ
ರಾಶಿ
ರಾಶಿ-ಗಳಲ್ಲಿ
ರಾಶಿಯ
ರಾಶಿ-ಯಲ್ಲಿ
ರಾಶಿ-ಹಾಕಿ
ರಾಷ್ಟ್ರ-ಪಾ-ಲಿ-ಕೆ-ವೃಷ
ರಾಸ-ಕ್ರೀಡೆ
ರಾಸ-ಕ್ರೀ-ಡೆಯ
ರಾಸ-ಕ್ರೀ-ಡೆ-ಯಲ್ಲಿ
ರಾಸ-ಕ್ರೀ-ಡೆ-ಯಿಂದ
ರಾಸ-ಲೀಲೆ
ರಾಸ-ಲೀ-ಲೆಯ
ರಾಹು
ರಾಹು-ಗ್ರ-ಹ-ವಿದೆ
ರಿಂದ
ರಿಂದಲೆ
ರಿಂದಲೇ
ರಿಗೂ
ರಿಗೆ
ರಿಬ್ಬರ
ರಿರಲಿ
ರಿಲ್ಲದ
ರಿಲ್ಲೊ
ರಿಸ-ಬೇಕು
ರಿಸಿ
ರಿಸಿ-ಹಾ-ಕಿರೊ
ರೀತ
ರೀತಿ
ರೀತಿ-ಯನ್ನು
ರೀತಿ-ಯಲ್ಲಿ
ರೀತಿ-ಯಾ-ಗಿವೆ
ರೀತಿ-ಯಿಂದ
ರುಂಡ
ರುಂಡ-ಗಳಿಂದ
ರುಂಡ-ಗಳು
ರುಕ್ಮ-ಣಿಯೂ
ರುಕ್ಮ-ವ-ತಿ-ಯನ್ನು
ರುಕ್ಮ-ವ-ತಿಯು
ರುಕ್ಮಿ
ರುಕ್ಮಿಗೆ
ರುಕ್ಮಿಣಿ
ರುಕ್ಮಿ-ಣಿಗೆ
ರುಕ್ಮಿ-ಣಿ-ಪ್ರ-ದ್ಯುಮ್ನ
ರುಕ್ಮಿ-ಣಿಯ
ರುಕ್ಮಿ-ಣಿ-ಯನ್ನು
ರುಕ್ಮಿ-ಣಿ-ಯ-ರನ್ನು
ರುಕ್ಮಿ-ಣಿಯು
ರುಕ್ಮಿ-ಣಿಯೇ
ರುಕ್ಮಿ-ಣಿ-ಯೇನೊ
ರುಕ್ಮಿ-ಣಿ-ಯೊ-ಡನೆ
ರುಕ್ಮಿಣೀ
ರುಕ್ಮಿ-ಣೀ-ದೇವಿ
ರುಕ್ಮಿ-ಣೀ-ದೇ-ವಿಯ
ರುಕ್ಮಿ-ಣೀ-ದೇ-ವಿಯು
ರುಕ್ಮಿಯ
ರುಕ್ಮಿ-ಯನ್ನು
ರುಕ್ಮಿ-ಯನ್ನೂ
ರುಕ್ಮಿಯು
ರುಕ್ಮಿಯೆ
ರುಗಿ
ರುಚಿ
ರುಚಿ-ಆ-ಕೂ-ತಿ-ಗಳ
ರುಚಿ-ಕರ
ರುಚಿ-ಕ-ರ-ವಾದ
ರುಚಿ-ಯನ್ನು
ರುಚಿ-ಯಾ-ಗಿ-ರುವ
ರುಚಿ-ಯೆಂಬ
ರುಚಿಯೇ
ರುಚಿ-ರ-ಹಾ-ಸ-ಭ್ರೂ-ವಿ-ಜೃಂ-ಭಸ್ಯ
ರುಚಿ-ಸ-ಲಿಲ್ಲ
ರುಚಿ-ಸಿತು
ರುದ್ಧ
ರುದ್ರ
ರುದ್ರ-ಎ-ಲ್ಲರೂ
ರುದ್ರ-ಗ-ಣ-ಗಳನ್ನು
ರುದ್ರ-ಗೀತೆ
ರುದ್ರ-ಗೀ-ತೆಯು
ರುದ್ರ-ದೇವ
ರುದ್ರ-ದೇ-ವನ
ರುದ್ರ-ದೇ-ವನು
ರುದ್ರ-ದೇ-ವನೇ
ರುದ್ರ-ದೇ-ವರು
ರುದ್ರನ
ರುದ್ರ-ನಂತೆ
ರುದ್ರ-ನನ್ನು
ರುದ್ರ-ನಾಗಿ
ರುದ್ರ-ನಿಗೆ
ರುದ್ರನು
ರುದ್ರನೂ
ರುದ್ರ-ನೆಂದು
ರುದ್ರ-ನೆಂ-ಬು-ವನು
ರುದ್ರನೇ
ರುದ್ರ-ಭ-ಯಾ-ನ-ಕ-ರಾ-ಗಿ-ದ್ದಾರೆ
ರುದ್ರ-ಭೂ-ಮಿ-ಯಾ-ಯಿತು
ರುದ್ರ-ರನ್ನು
ರುದ್ರರು
ರುದ್ರ-ರು-ನ-ನಗೆ
ರುದ್ರ-ರೆಂಬ
ರುದ್ರರೇ
ರುಮಾ-ಲು-ಗಳನ್ನು
ರುಳಿ-ದವು
ರುಳು
ರುವ
ರುವುದು
ರುವೆ
ರೂಪ
ರೂಪಂ
ರೂಪಕ್ಕೆ
ರೂಪ-ಗಳ
ರೂಪ-ಗಳನ್ನು
ರೂಪ-ಗಳನ್ನೂ
ರೂಪ-ಗಳಿಂದ
ರೂಪ-ಗ-ಳಿಗೆ
ರೂಪ-ಗುಣ
ರೂಪದ
ರೂಪ-ದಲ್ಲಿ
ರೂಪ-ದ-ಲ್ಲಿಯೇ
ರೂಪ-ದ-ಲ್ಲಿ-ರುವ
ರೂಪ-ದಿಂದ
ರೂಪ-ದಿಂ-ದಲೆ
ರೂಪ-ನಾ-ಗಲು
ರೂಪ-ನಾಗಿ
ರೂಪ-ನಾ-ಗಿಯೂ
ರೂಪ-ನಾ-ಗು-ವು-ದಕ್ಕೆ
ರೂಪ-ನಾದ
ರೂಪನ್ನು
ರೂಪ-ಭೇ-ದ-ಗ-ಳಾ-ಗಲಿ
ರೂಪ-ಮ-ದ-ದಿಂದ
ರೂಪ-ಯಾ-ನಾ-ಯು-ಧಾನಿ
ರೂಪ-ವ-ತಿ-ಯ-ರಿ-ಲ್ಲ-ವೆಂದೂ
ರೂಪ-ವನ್ನು
ರೂಪ-ವನ್ನೂ
ರೂಪ-ವಾದ
ರೂಪ-ವೈ-ಭವ
ರೂಪ-ಸಂ-ಪತ್ತು
ರೂಪ-ಸಿ-ಯೊ-ಬ್ಬಳು
ರೂಪ-ಸೌಂ-ದ-ರ್ಯ-ವನ್ನು
ರೂಪಾಂ-ತರ
ರೂಪಿ
ರೂಪಿಗೆ
ರೂಪಿನ
ರೂಪಿ-ನಿಂದ
ರೂಪಿ-ನಿಂ-ದಲೇ
ರೂಪು
ರೂಪು-ವೆ-ತ್ತಂತೆ
ರೆಂದಾ-ಯಿತು
ರೆಂದು
ರೆಂಬು-ದನ್ನು
ರೆಂಬೆ
ರೆಂಬೆ-ಗಳನ್ನು
ರೆಂಬೆ-ಯನ್ನು
ರೆಕ್ಕೆ
ರೆಕ್ಕೆ-ಗಳ
ರೆಕ್ಕೆಯ
ರೆಕ್ಕೆ-ಯಿಂದ
ರೆಪ್ಪೆ-ಗಳನ್ನು
ರೆಪ್ಪೆ-ಗಳು
ರೆಪ್ಪೆ-ಯಲ್ಲಿ
ರೆಲ್ಲ
ರೆಲ್ಲರ
ರೆಲ್ಲ-ರನ್ನೂ
ರೆಲ್ಲರೂ
ರೆಲ್ಲಿ
ರೇಕು-ಗ-ಳಂತೆ
ರೇಖೆ-ಗಳು
ರೇಗಿ
ರೇಗಿತು
ರೇಗಿ-ದನು
ರೇಗಿ-ದರು
ರೇಗಿ-ದ-ರೇನು
ರೇಗಿ-ದಳು
ರೇಗಿಸಿ
ರೇಗಿ-ಹೋ-ಯಿತು
ರೇಣುಕೆ
ರೇಣು-ಕೆಯ
ರೇನು
ರೇವ-ತಿ-ಯೆಂಬ
ರೇವಾ-ಖಂ-ಡ-ದಲ್ಲಿ
ರೇವಾ-ನ-ದಿಯ
ರೇಷ್ಮೆ
ರೇಷ್ಮೆಯ
ರೈಲು-ಗಾ-ಡಿ-ಗಳು
ರೈವ-ತ-ಎಂಬ
ರೈವ-ತ-ಗಿ-ರಿ-ಯ-ಮೇಲೆ
ರೈವ-ತನು
ರೈವ-ತ-ಮ-ನ್ವಂ-ತ-ರ-ದಲ್ಲಿ
ರೊಂದು
ರೊಟ್ಟಿ
ರೊಡನೆ
ರೊಡ-ನೆಯೂ
ರೊಯ್ಯೆಂದು
ರೋಗ
ರೋಗಕ್ಕೆ
ರೋಗ-ಗಳು
ರೋಗದ
ರೋಗ-ದಿಂದ
ರೋಗ-ರು-ಜಿ-ನ-ಗಳ
ರೋಗ-ವಿದ್ದ
ರೋಗ-ವಿಲ್ಲ
ರೋಗವೂ
ರೋಗಿ
ರೋಗಿ-ಯಾಗಿ
ರೋಗಿ-ಯಾ-ದ-ವನು
ರೋಚ-ನಿ-ಅ-ನಿ-ರು-ದ್ಧರ
ರೋಚ-ನೆ-ಹಸ್ತ
ರೋಚಿ-ನಿ-ಯನ್ನು
ರೋದನ
ರೋದ-ನದ
ರೋದ-ನ-ವನ್ನು
ರೋದ-ನವೆ
ರೋದಿ-ಸಿದು
ರೋದಿ-ಸು-ತ್ತಿ-ರಲು
ರೋಧಿ-ಸು-ತ್ತದೆ
ರೋಮ-ಗಳು
ರೋಮಾಂಚ
ರೋಮಾಂ-ಚ-ಗೊಂಡು
ರೋಮಾಂ-ಚ-ಗೊ-ಳ್ಳು-ತ್ತಿತ್ತು
ರೋಮಾಂ-ಚನ
ರೋಮಾಂ-ಚ-ನ-ಗೊಂ-ಡರೂ
ರೋಮಾಂ-ಚ-ನ-ಗೊಂ-ಡಿತು
ರೋಮಾಂ-ಚ-ನ-ಗೊಂಡು
ರೋಮಾಂ-ಚ-ನ-ಗೊ-ಳ್ಳದ
ರೋರ-ವೀಷಿ
ರೋಷ
ರೋಷಕ್ಕೆ
ರೋಷ-ಗಳು
ರೋಷ-ಗೊಂಡ
ರೋಷ-ದಿಂದ
ರೋಷ-ಭೀ-ಷ-ಣ-ನಾಗಿ
ರೋಷ-ಭೀ-ಷ-ಣ-ನಾದ
ರೋಷ-ಭೀ-ಷ-ಣ-ರಾ-ದರು
ರೋಷ-ವು-ಕ್ಕಿತು
ರೋಷಾ-ವೇ-ಶ-ಗೊಂ-ಡಿ-ದ್ದಾನೆ
ರೋಷಾ-ವೇ-ಶ-ದಿಂದ
ರೋಸಿ
ರೋಹಿಣಿ
ರೋಹಿ-ಣಿ-ಬ-ಲ-ರಾಮ
ರೋಹಿ-ಣಿಯ
ರೋಹಿ-ಣಿ-ಯರು
ರೋಹಿ-ಣಿ-ಯ-ಲ್ಲಿಯೇ
ರೋಹಿ-ಣಿ-ಯಿಂದ
ರೋಹಿ-ಣಿಯು
ರೋಹಿಣೀ
ರೋಹಿತ
ರೋಹಿ-ತ-ನಿಗೆ
ರೋಹಿ-ತನು
ರೋಹಿ-ತಾ-ಶ್ವ-ನಿಗೆ
ಲಂಕಾರ
ಲಂಕಾ-ರ-ಗಳಿಂದ
ಲಂಕಾ-ರ-ಭೂ-ಷಿ-ತೆ-ಯಾಗಿ
ಲಂಕೆಯ
ಲಂಕೆ-ಯನ್ನು
ಲಂತೂ
ಲಂಪ-ಟ-ನಾದ
ಲಕ್ಷ
ಲಕ್ಷಣ
ಲಕ್ಷ-ಣ-ಗಳ
ಲಕ್ಷ-ಣ-ಗ-ಳಿಂ-ದಲೂ
ಲಕ್ಷ-ಣ-ಗಳು
ಲಕ್ಷ-ಣ-ಗ-ಳುಳ್ಳ
ಲಕ್ಷ-ಣ-ಗ-ಳೆಲ್ಲ
ಲಕ್ಷ-ಣ-ವನ್ನು
ಲಕ್ಷಣೆ
ಲಕ್ಷ-ಣೆಗೆ
ಲಕ್ಷ-ಣೆ-ಪ್ರ-ಘೋಷ
ಲಕ್ಷ-ಣೆ-ಯನ್ನು
ಲಕ್ಷ-ಯೋ-ಜನ
ಲಕ್ಷ-ಯೋ-ಜ-ನ-ದಲ್ಲಿ
ಲಕ್ಷ-ವನ್ನು
ಲಕ್ಷ-ವಾಗಿ
ಲಕ್ಷಿಸ
ಲಕ್ಷಿ-ಸದ
ಲಕ್ಷಿ-ಸದೆ
ಲಕ್ಷೋಪ
ಲಕ್ಷ್ಮಣ
ಲಕ್ಷ್ಮ-ಣ-ನಿಗೆ
ಲಕ್ಷ್ಮ-ಣ-ನೊ-ಡ-ಗೂ-ಡಿದ
ಲಕ್ಷ್ಮ-ಣ-ನೊ-ಡನೆ
ಲಕ್ಷ್ಮಣಾ
ಲಕ್ಷ್ಮಿ
ಲಕ್ಷ್ಮಿಗೆ
ಲಕ್ಷ್ಮಿಯ
ಲಕ್ಷ್ಮಿ-ಯರ
ಲಕ್ಷ್ಮಿ-ಯ-ಹಾಗೆ
ಲಕ್ಷ್ಮಿ-ಯಾದ
ಲಕ್ಷ್ಮಿಯು
ಲಕ್ಷ್ಮಿ-ಯೆಂದು
ಲಕ್ಷ್ಮಿಯೇ
ಲಕ್ಷ್ಮಿಯೊ
ಲಕ್ಷ್ಮೀ
ಲಕ್ಷ್ಮೀ-ದೇ-ವಿ-ಯನ್ನು
ಲಕ್ಷ್ಮೀ-ದೇ-ವಿಯು
ಲಕ್ಷ್ಮೀ-ದೇ-ವಿಯೇ
ಲಕ್ಷ್ಮೀ-ಪ-ತಿ-ಯಾದ
ಲಕ್ಷ್ಮೀ-ಪ-ತಿಯೆ
ಲಕ್ಷ್ಯವೂ
ಲಗಾಮು
ಲಗ್ಗೆ
ಲಗ್ನದ
ಲಜ್ಜಾ-ರ-ಹಿ-ತ-ನಾಗಿ
ಲಜ್ಜೆ
ಲಜ್ಜೆ-ಯನ್ನು
ಲಜ್ಜೆ-ಯೊ-ಡನೆ
ಲನ್ನು
ಲನ್ನೂ
ಲಪ-ಟಾ-ಯಿ-ಸುವ
ಲಭತೇ
ಲಭಿ-ಸ-ಬೇಕು
ಲಭಿ-ಸ-ಲೆಂದು
ಲಭಿ-ಸಿ-ದೊ-ಡ-ನೆಯೆ
ಲಭಿ-ಸಿವೆ
ಲಭಿಸು
ಲಭಿ-ಸು-ತ್ತದೆ
ಲಭಿ-ಸು-ತ್ತವೆ
ಲಭಿ-ಸು-ತ್ತಿತ್ತು
ಲಭಿ-ಸು-ವು-ದೆಂದೂ
ಲಭ್ಯ-ವಾ-ಯಿತು
ಲಯ
ಲಯ-ಗಳನ್ನು
ಲಯ-ಗ-ಳಿಗೆ
ಲಯ-ಗಳು
ಲಯ-ಗ-ಳೆಂಬ
ಲಯ-ಗೊ-ಳಿಸಿ
ಲಯ-ಗೊ-ಳಿ-ಸುತ್ತಿ
ಲಲನಾ
ಲಲಿ-ತಮ್
ಲಲ್ಲೆ-ವಾ-ತು-ಗಳಿಂದ
ಲಲ್ಲೆ-ವಾ-ತು-ಗ-ಳಿಗೆ
ಲವ
ಲವ-ಲೇ-ಶವೂ
ಲಹ-ರಿ-ಯೊಂದು
ಲಾಕ್ಷ-ಣಿ-ಕರು
ಲಾಗು-ವು-ದಿಲ್ಲ
ಲಾಘ-ವ-ದಿಂದ
ಲಾಡಿಸು
ಲಾಡಿ-ಸು-ತ್ತೇನೆ
ಲಾಡಿ-ಸುವ
ಲಾದ
ಲಾಭ-ವನ್ನು
ಲಾಯ-ಗಳು
ಲಾರದ
ಲಾರ-ದಷ್ಟು
ಲಾರದೆ
ಲಾರರು
ಲಾರೆ
ಲಾಲನೆ
ಲಾಲಿಸಿ
ಲಾಲಿಸು
ಲಾವ-ಣ್ಯಕ್ಕೆ
ಲಾವ-ಣ್ಯ-ವನ್ನು
ಲಾವಾ-ರ-ಸ-ವನ್ನು
ಲಿಂ-ಲಿಂಗ
ಲಿಂಗ
ಲಿಂಗಂ
ಲಿಂಗ-ಪು-ರಾಣ
ಲಿಂಗ-ಶ-ರೀರ
ಲಿಂಗ-ಶ-ರೀ-ರಕ್ಕೆ
ಲಿಂಗ-ಶ-ರೀ-ರ-ವೆಂಬ
ಲಿಂಗಾ-ತೀತ
ಲಿಡು-ವಂ-ತಾ-ದರು
ಲಿಬಿ-ಲಿಬಿ
ಲಿಲ್ಲ
ಲಿಲ್ಲ-ವೆಂಬ
ಲೀನ-ವಾಗಿ
ಲೀನ-ವಾ-ಗಿತ್ತು
ಲೀನ-ವಾ-ಗು-ತ್ತದೆ
ಲೀನ-ವಾ-ಗು-ತ್ತವೆ
ಲೀಲಾ-ಕ-ಮ-ಲ-ವನ್ನು
ಲೀಲಾ-ಜಾ-ಲ-ವಾಗಿ
ಲೀಲಾ-ಮಾ-ನು-ಷ-ವಿ-ಗ್ರ-ಹ-ವಾದ
ಲೀಲಾ-ಮೂ-ರ್ತಿ-ಯಾ-ಗಿ-ರುವೆ
ಲೀಲಾ-ವ-ರ್ಣ-ನೆಯ
ಲೀಲಾ-ವಿ-ಭೂ-ತಿ-ಗ-ಳಂ-ತಿ-ರುವ
ಲೀಲಾ-ವಿ-ಲಾ-ಸ-ಗ-ಳೆಂಬ
ಲೀಲೆ
ಲೀಲೆ-ಗಳನ್ನು
ಲೀಲೆ-ಗಳನ್ನೆಲ್ಲಾ
ಲೀಲೆ-ಗ-ಳ-ಲ್ಲಿಯೇ
ಲೀಲೆ-ಗ-ಳಿಗೆ
ಲೀಲೆ-ಗಳು
ಲೀಲೆ-ಗ-ಳೆಲ್ಲ
ಲೀಲೆ-ಗಾಗಿ
ಲೀಲೆ-ಯಂತೂ
ಲೀಲೆ-ಯನ್ನು
ಲೀಲೆ-ಯಲ್ಲಿ
ಲೀಲೆ-ಯಿಂದ
ಲೀಲೆಯೆ
ಲೀಲೆಯೇ
ಲೀಲೇ-ಕ್ಷಿ-ತೇನ
ಲುಬ್ಧ-ಧರ್ಮಾ
ಲೆಂದು
ಲೆಂದೇ
ಲೆಕ್ಕ
ಲೆಕ್ಕ-ವಲ್ಲ
ಲೆಕ್ಕ-ವಿ-ಲ್ಲ-ದಷ್ಟು
ಲೆಕ್ಕವೆ
ಲೆಕ್ಕಿ-ಸದೆ
ಲೇಖ-ಕರು
ಲೇಸೆ-ನಿ-ಸಿತು
ಲೊಳಗೆ
ಲೋಕ
ಲೋಕ-ಕಂ-ಟ-ಕ-ನಾದ
ಲೋಕ-ಕಂ-ಟ-ಕ-ನೆಂದೂ
ಲೋಕ-ಕಂ-ಟ-ಕ-ರಾ-ಗು-ತ್ತಾರೆ
ಲೋಕ-ಕರ್ತ
ಲೋಕ-ಕ-ಲ್ಯಾ-ಣ-ಕ್ಕಾಗಿ
ಲೋಕ-ಕ-ಲ್ಯಾ-ಣ-ವನ್ನು
ಲೋಕಕ್ಕೂ
ಲೋಕಕ್ಕೆ
ಲೋಕ-ಕ್ಕೆಲ್ಲ
ಲೋಕಕ್ಕೇ
ಲೋಕ-ಕ್ಷೇಮ
ಲೋಕ-ಕ್ಷೇ-ಮ-ಕ್ಕಾಗಿ
ಲೋಕ-ಗಳ
ಲೋಕ-ಗಳನ್ನು
ಲೋಕ-ಗಳನ್ನೂ
ಲೋಕ-ಗ-ಳನ್ನೆ
ಲೋಕ-ಗಳನ್ನೆಲ್ಲ
ಲೋಕ-ಗಳಲ್ಲಿ
ಲೋಕ-ಗ-ಳ-ಲ್ಲಿಯೂ
ಲೋಕ-ಗ-ಳಿಗೂ
ಲೋಕ-ಗ-ಳಿಗೆ
ಲೋಕ-ಗಳು
ಲೋಕ-ಗಳೂ
ಲೋಕ-ಗ-ಳೆಲ್ಲ
ಲೋಕ-ಗ-ಳೆ-ಲ್ಲವೂ
ಲೋಕ-ಗಳೇ
ಲೋಕ-ಗುರು
ಲೋಕ-ಗು-ರು-ವಂತೆ
ಲೋಕ-ಗು-ರು-ವನ್ನು
ಲೋಕ-ಗು-ರು-ವಾದ
ಲೋಕದ
ಲೋಕ-ದಲ್ಲಿ
ಲೋಕ-ದ-ಲ್ಲಿಯೂ
ಲೋಕ-ದ-ಲ್ಲಿ-ರುವ
ಲೋಕ-ದಲ್ಲೆಲ್ಲ
ಲೋಕ-ದ-ವರೆಲ್ಲ
ಲೋಕ-ನಾಥ
ಲೋಕ-ಪತಿ
ಲೋಕ-ಪಾ-ಲ-ಕ-ನಂತೆ
ಲೋಕ-ಪಾ-ಲ-ಕ-ರಿಗೆ
ಲೋಕ-ಪಾ-ಲ-ಕರು
ಲೋಕ-ಪಾ-ಲ-ಕರೂ
ಲೋಕ-ಪಾ-ಲ-ಕ-ರೆ-ಲ್ಲರೂ
ಲೋಕ-ಪಾ-ಲ-ನೆಂಬ
ಲೋಕ-ಪಾ-ಲ-ರನ್ನೂ
ಲೋಕ-ಪೂಜ್ಯ
ಲೋಕ-ಪ್ರ-ಸಿ-ದ್ಧ-ರಾ-ದರು
ಲೋಕ-ಭಯಂ
ಲೋಕ-ಮಾ-ತೆ-ಯಾದ
ಲೋಕ-ಮೋ-ಹಕ
ಲೋಕ-ಮೋ-ಹ-ಕ-ಳಾದ
ಲೋಕ-ಮೋ-ಹ-ಕ-ವಾದ
ಲೋಕ-ರಂ-ಜ-ಕ-ನಾದ
ಲೋಕ-ರ-ಕ್ಷಕ
ಲೋಕ-ರ-ಕ್ಷಣೆ
ಲೋಕ-ರ-ಕ್ಷ-ಣೆ-ಗಾಗಿ
ಲೋಕ-ರ-ಕ್ಷ-ಣೆ-ಗಾ-ದರೆ
ಲೋಕ-ಲೋ-ಕ-ದ-ವ-ರಿ-ಗೆಲ್ಲ
ಲೋಕ-ವನ್ನು
ಲೋಕ-ವನ್ನೂ
ಲೋಕ-ವನ್ನೆ
ಲೋಕ-ವ-ನ್ನೆಲ್ಲ
ಲೋಕ-ವಾ-ದರೆ
ಲೋಕ-ವಾ-ಸಿ-ಗ-ಳಿಗೆ
ಲೋಕ-ವುಂಟೊ
ಲೋಕವೂ
ಲೋಕ-ವೆಲ್ಲ
ಲೋಕ-ವೆ-ಲ್ಲ-ವನ್ನೂ
ಲೋಕ-ವೆ-ಲ್ಲವೂ
ಲೋಕ-ಸಂ-ಚಾರ
ಲೋಕ-ಸಾ-ಮಾ-ನ್ಯ-ವಾದ
ಲೋಕ-ಸುಂ-ದರ
ಲೋಕ-ಸೃ-ಷ್ಟಿ-ಯಾದ
ಲೋಕ-ಹಿ-ತಕ್ಕೆ
ಲೋಕಾಂ-ತ-ರ-ಗಳನ್ನು
ಲೋಕಾಂ-ತ-ರ-ಗ-ಳ-ಲ್ಲೆಲ್ಲಾ
ಲೋಕಾ-ದ-ವ-ತಾ-ಜ್ಜ-ನಾಂ-ತಾತ್
ಲೋಕಾ-ಧಾರ
ಲೋಕಾ-ಧ್ಯ-ಕ್ಷನೂ
ಲೋಕಾ-ನು-ಗ್ರ-ಹವೇ
ಲೋಕಾ-ಭಿ-ರಾ-ಮ-ವಾಗಿ
ಲೋಕಾಯ
ಲೋಕಾ-ಲೋಕ
ಲೋಕಾ-ಲೋ-ಕ-ವೆಂಬ
ಲೋಕಾ-ಶ್ರ-ಯ-ನಾ-ಗಿಯೂ
ಲೋಕಾ-ಸ್ಸ-ಮ-ಸ್ತಾಃ
ಲೋಕೇ-ಶ್ವರ
ಲೋಕೇ-ಶ್ವ-ರ-ನಂತೆ
ಲೋಕೇ-ಶ್ವ-ರ-ನಾದ
ಲೋಕೇ-ಶ್ವ-ರ-ನೆಂಬ
ಲೋಕೇ-ಶ್ವರಾ
ಲೋಕೋ-ತ್ತ-ರ-ಳಾದ
ಲೋಕೋ-ತ್ತ-ರ-ವಾದ
ಲೋಕೋ-ದ್ಧಾ-ರ-ಕ್ಕಾಗಿ
ಲೋಪ
ಲೋಪ-ದೋ-ಷ-ಗ-ಳಿಗೂ
ಲೋಪ-ವಾಗ
ಲೋಪ-ವಾ-ಗ-ದಂತೆ
ಲೋಪ-ವಾ-ದರೆ
ಲೋಪ-ವಾ-ಯಿತೆ
ಲೋಪವೂ
ಲೋಪ-ವೊ-ದ-ಗಿತು
ಲೋಭ-ರನ್ನು
ಲೋಲ-ನಾ-ಗಿಯೇ
ಲೋಹ-ಗ-ಳಿಗೂ
ಲೋಹ-ಗಳು
ಲೋಹ-ದಂ-ಡ-ಗ-ಳಾ-ದವು
ಲೋಹ-ದಿಂ-ದಲೆ
ಲೌಕಿಕ
ಲೌಕಿ-ಕ-ವಾದ
ಲ್ಲಂತೂ
ಲ್ಲರೂ
ಲ್ಲವೆ
ಲ್ಲಿಟ್ಟಳು
ಲ್ಲಿಟ್ಟಿದ್ದ
ಲ್ಲಿಟ್ಟು
ಲ್ಲಿದ್ದ
ಲ್ಲಿಯೂ
ಲ್ಲಿಯೆ
ಲ್ಲಿಯೇ
ಲ್ಲಿರ-ಬೇ-ಕಾದ
ಲ್ಲಿರ-ಬೇಕು
ಲ್ಲಿರುವ
ಲ್ಲಿರು-ವಂತೆ
ಲ್ಲೆಲ್ಲೊ
ಳಲ್ಲಿ
ಳಾಗಿ-ದ್ದಳು
ಳಾಗು-ತ್ತೇನೆ
ಳಾದ
ಳಿಗೆ
ಳೆಂದು
ಳೆರ-ಡನ್ನೂ
ಳೊಂದಿಗೆ
ಳೊಡನೆ
ಳ್ನಗುತ್ತಾ
ಳ್ನಗೆ-ಯನ್ನು
ಳ್ನಗೆ-ಯಿಂದ
ವ
ವಂ
ವಂಚ-ಕ-ನಾದ
ವಂಚ-ನೆಯೆ
ವಂಚಿ-ತ-ರಾಗಿ
ವಂಚಿ-ತ-ರಾಗು
ವಂಚಿ-ಸ-ಲೆಂದೊ
ವಂಚಿಸಿ
ವಂಚಿ-ಸುವು
ವಂಚಿ-ಸು-ವು-ದಕ್ಕೆ
ವಂತನ
ವಂತ-ನಿಗೆ
ವಂತನು
ವಂತನೆ
ವಂತಹ
ವಂತಾ-ಯಿತು
ವಂತಿ-ರುವ
ವಂತೆ
ವಂತೆಯೆ
ವಂತೆಯೇ
ವಂದನ
ವಂದನೆ
ವಂದಿ
ವಂದಿಸಿ
ವಂದಿ-ಸುತ್ತ
ವಂದಿ-ಸು-ತ್ತಿ-ದ್ದರು
ವಂದಿ-ಸು-ತ್ತೇವೆ
ವಂದಿ-ಸು-ವುದು
ವಂಶ
ವಂಶಕ್ಕೆ
ವಂಶ-ಗಳನ್ನು
ವಂಶ-ಗ-ಳಿಗೂ
ವಂಶದ
ವಂಶ-ದಲ್ಲಿ
ವಂಶ-ದ-ಲ್ಲಿಯೇ
ವಂಶ-ದ-ವ-ನಾದ
ವಂಶ-ದ-ವರ
ವಂಶ-ದ-ವ-ರನ್ನು
ವಂಶ-ದ-ವರೂ
ವಂಶ-ದ-ವರೆಲ್ಲ
ವಂಶ-ಪ-ರಂ-ಪರೆ
ವಂಶ-ಪ-ರಂ-ಪ-ರೆ-ಯಲ್ಲಿ
ವಂಶ-ಪಾ-ರಂ-ಪರ್ಯ
ವಂಶ-ವನ್ನು
ವಂಶವು
ವಂಶ-ವೃ-ಕ್ಷ-ದಲ್ಲಿ
ವಂಶವೆ
ವಂಶವೇ
ವಂಶ-ವೇನು
ವಂಶಾ-ನು-ಚ-ರಿತ
ವಂಶಾ-ನು-ಚ-ರಿ-ತ-ಎಂಬ
ವಂಶಾ-ನು-ಚ-ರಿ-ತ-ವನ್ನು
ವಂಶಾ-ಭಿ-ವೃ-ದ್ಧಿ-ಗಾಗಿ
ವಂಶಾ-ವ-ಳಿ-ಯನ್ನೂ
ವಂಶೋ
ವಂಶೋ-ದ್ಧಾ-ರ-ಕ-ನ-ನ್ನು-ಬ-ಲಿ-ಕೊ-ಡುವ
ವಕ್ತ್ರನು
ವಕ್ತ್ರೇಭ್ಯೋ
ವಕ್ರ-ಗ-ತಿ-ಯಿ-ಲ್ಲ-ದಾಗ
ವಚ-ನ-ಭಾ-ಗ-ವತ
ವಚ-ನ-ಭಾ-ಗ-ವ-ತ-ವನ್ನು
ವಚ-ನ-ಭಾ-ಗ-ವ-ತವು
ವಚ-ನ-ಭಾ-ಗ-ವ-ತ-ವೆಂಬ
ವಚ-ನ-ವೇದ
ವಚ-ನ-ಶೂ-ರ-ನೇನೂ
ವಚ-ಸ್ಯು-ಪ-ರತೇ
ವಜ್ರ
ವಜ್ರ-ಕಾಯ
ವಜ್ರದ
ವಜ್ರ-ದಂತೆ
ವಜ್ರ-ದಂಷ್ಟ್ರ
ವಜ್ರ-ದಿಂದ
ವಜ್ರ-ನಖ
ವಜ್ರ-ನಿಗೂ
ವಜ್ರ-ನಿಗೆ
ವಜ್ರ-ಮು-ಷ್ಟಿ-ಯಿಂದ
ವಜ್ರ-ಮು-ಷ್ಠಿ-ಯಿಂದ
ವಜ್ರ-ವೆಂದರೆ
ವಜ್ರ-ವೈ-ಢೂ-ರ್ಯ-ಗಳಿಂದ
ವಜ್ರ-ಸ-ಮಾ-ನ-ವಾದ
ವಜ್ರಾ-ಯುಧ
ವಜ್ರಾ-ಯು-ಧ-ಕ್ಕಾ-ಗಲಿ
ವಜ್ರಾ-ಯು-ಧ-ದಿಂದ
ವಜ್ರಾ-ಯು-ಧ-ವನ್ನು
ವಟು
ವಟು-ವಾದ
ವಟು-ವಿಗೆ
ವಣಿ-ಯು-ವಂ-ತಿ-ರು-ವುದು
ವಣೆ
ವತವು
ವತ-ವೆಂ-ದೊ-ಡ-ನೆಯೆ
ವತಿಗೆ
ವತಿ-ಯಿಂದ
ವತ್ತಾಗಿ
ವತ್ಸ
ವತ್ಸಕ
ವತ್ಸ-ಕ-ನೆಂಬ
ವತ್ಸರ
ವತ್ಸಾ
ವತ್ಸಾ-ಸು-ರ-ನೆಂದು
ವದ
ವಧಿ-ಸು-ವೆನು
ವಧೂ
ವಧೂ-ವ-ರರ
ವಧೂ-ವ-ರ-ರನ್ನು
ವಧೆ
ವಧೆಗೆ
ವಧೆ-ಯಿಂದ
ವಧೆಯೆ
ವಧ್ವೋ
ವನ
ವನ-ಕಾ-ನ-ನ-ಗಳ
ವನಕ್ಕೆ
ವನ-ಗ-ಳ-ಲ್ಲೆಲ್ಲ
ವನ-ಗ-ಳೆಲ್ಲ
ವನದ
ವನ-ದಲ್ಲಿ
ವನ-ಭೋ-ಜನ
ವನ-ಲ-ಕ್ಷ್ಮಿಯ
ವನಲ್ಲಾ
ವನ-ವನ್ನು
ವನ-ವ-ನ್ನೆಲ್ಲ
ವನ-ವಾಸ
ವನ-ವಾ-ಸಕ್ಕೆ
ವನ-ವಿ-ಹಾರ
ವನ-ವಿ-ಹಾ-ರಕ್ಕೆ
ವನ-ವೆ-ಲ್ಲವೂ
ವನು
ವನೆ
ವನೊ
ವನೋ
ವನ್ನಾಗಿ
ವನ್ನಾ-ಡುವೆ
ವನ್ನು
ವನ್ನೂ
ವನ್ನೆ
ವನ್ನೆಲ್ಲ
ವನ್ನೆಲ್ಲಾ
ವನ್ನೇನೊ
ವಪ್ಪ
ವಯಂ
ವಯಃ-ಏನು
ವಯ-ಮಿವ
ವಯ-ಮಿ-ವಾ-ಚ್ಯು-ತ-ಪಾ-ದ-ಜು-ಷ್ಟಾಂ
ವಯ-ಮೃ-ತ-ಮಿವ
ವಯ-ಸ್ಕ-ನಾದ
ವಯ-ಸ್ಸಾ-ಗಿ-ರ-ಬೇ-ಕಾ-ಗಿತ್ತು
ವಯ-ಸ್ಸಾ-ಗಿರು
ವಯ-ಸ್ಸಾದ
ವಯ-ಸ್ಸಾ-ದ-ವರು
ವಯ-ಸ್ಸಾ-ದ-ವರೆ-ಲ್ಲರೂ
ವಯ-ಸ್ಸಾ-ಯಿತು
ವಯಸ್ಸಿ
ವಯ-ಸ್ಸಿಗೂ
ವಯ-ಸ್ಸಿಗೆ
ವಯ-ಸ್ಸಿನ
ವಯ-ಸ್ಸಿ-ನಲ್ಲಿ
ವಯ-ಸ್ಸಿ-ನ-ಲ್ಲಿಯೂ
ವಯ-ಸ್ಸಿ-ನ-ಲ್ಲಿಯೇ
ವಯ-ಸ್ಸಿ-ನ-ವ-ನಾದ
ವಯ-ಸ್ಸಿ-ನ-ವ-ನಾ-ದರೂ
ವಯ-ಸ್ಸಿ-ನ-ವ-ರೆಗೆ
ವಯ-ಸ್ಸಿ-ನೊ-ಡನೆ
ವಯಸ್ಸು
ವಯ್ಯಾ-ರ-ದಿಂದ
ವರ
ವರ-ಕೊ-ಡ-ಲೆಂದು
ವರ-ಕೊಡು
ವರ-ಗಳ
ವರ-ಗಳನ್ನು
ವರ-ಗಳಲ್ಲಿ
ವರದ
ವರ-ದಂತೆ
ವರದಿ
ವರ-ದಿಂದ
ವರ-ದಿ-ಮಾ-ಡಿ-ದರು
ವರ-ನಾ-ಗ-ಲಿ-ರುವ
ವರ-ನಿ-ಗಾಗಿ
ವರನು
ವರ-ಪ್ರ-ಧಾ-ನ-ಗಳು
ವರ-ಪ್ರ-ಸಾ-ದ-ದಿಂದ
ವರಯ
ವರ-ರಿಗೆ
ವರ-ವನ್ನು
ವರ-ವನ್ನೂ
ವರ-ವಾಗಿ
ವರ-ವಿತ್ತ
ವರ-ವಿ-ತ್ತರು
ವರ-ವಿ-ತ್ತಿ-ದ್ದೇನೆ
ವರ-ವಿ-ತ್ತುದು
ವರ-ವಿದೆ
ವರ-ವೊಂ-ದಿದೆ
ವರ-ಹಾ-ವ-ತಾ-ರದ
ವರಾದ್ದ
ವರಾ-ರಿ-ಸೈನ್ಯ
ವರಾರು
ವರಾಹ
ವರಾಹಃ
ವರಾ-ಹ-ನನ್ನು
ವರಾ-ಹ-ಪು-ರಾಣ
ವರಾ-ಹ-ಮೂರ್ತಿ
ವರಾ-ಹ-ಮೂ-ರ್ತಿಯ
ವರಾ-ಹ-ಮೂ-ರ್ತಿ-ಯನ್ನು
ವರಾ-ಹ-ಮೂ-ರ್ತಿಯು
ವರಾ-ಹ-ರೂ-ಪ-ದಿಂದ
ವರಾ-ಹ-ರೂ-ಪಿ-ನಿಂದ
ವರಾ-ಹ-ರೂ-ಪಿ-ನಿಂ-ದಲೂ
ವರಾ-ಹ-ರೂ-ಪಿ-ಯಾದ
ವರಾ-ಹ-ವನ್ನು
ವರಾ-ಹವು
ವರಾ-ಹಾವ
ವರಾ-ಹಾ-ವ-ತಾರ
ವರಿ-ಯುವ
ವರಿ-ಸ-ಬೇ-ಕೆಂದು
ವರಿ-ಸಲು
ವರಿ-ಸಲೆ
ವರಿ-ಸ-ಹೊ-ರ-ಡಲು
ವರಿಸಿ
ವರಿ-ಸಿತು
ವರಿ-ಸಿ-ದಳು
ವರಿ-ಸಿ-ದವು
ವರಿ-ಸಿ-ದುದೇ
ವರಿ-ಸಿದೆ
ವರಿ-ಸಿ-ದೆ-ಯಲ್ಲ
ವರಿ-ಸಿ-ದ್ದರೆ
ವರಿ-ಸಿ-ದ್ದಳು
ವರಿ-ಸಿ-ದ್ದಾಳೆ
ವರಿ-ಸಿದ್ದೆ
ವರಿ-ಸಿ-ದ್ದೇನೆ
ವರಿ-ಸಿ-ಯಾರು
ವರಿ-ಸಿರಿ
ವರಿ-ಸು-ತ್ತಾಳೊ
ವರಿ-ಸು-ತ್ತೇನೆ
ವರಿ-ಸುವ
ವರಿ-ಸು-ವಂ-ತಿಲ್ಲ
ವರಿ-ಸು-ವ-ವ-ಳನ್ನು
ವರಿ-ಸು-ವೆ-ನೆಂ-ದಾಗ
ವರು
ವರುಣ
ವರು-ಣ-ದೇ-ವನ
ವರು-ಣ-ದೇ-ವ-ನನ್ನು
ವರು-ಣ-ದೇ-ವ-ನಿಗೆ
ವರು-ಣ-ದೇ-ವನು
ವರು-ಣನ
ವರು-ಣ-ನನ್ನು
ವರು-ಣ-ನಿಗೆ
ವರು-ಣನು
ವರು-ಣ-ಪಾ-ಶ-ದಿಂದ
ವರು-ಣ-ಲೋ-ಕಕ್ಕೆ
ವರು-ಣ-ಲೋ-ಕ-ದಿಂದ
ವರು-ಣಸ್ಯ
ವರು-ಣೋ-ದ್ಯಾ-ನದ
ವರುಷ
ವರು-ಷಕ್ಕೆ
ವರೂ
ವರೆಂಬ
ವರೆಗೂ
ವರೆಗೆ
ವರೊ
ವರೋ
ವರ್ಗ
ವರ್ಣ
ವರ್ಣಕ್ಕೆ
ವರ್ಣ-ಗಳ
ವರ್ಣ-ಗ-ಳ-ಲ್ಲೆಲ್ಲ
ವರ್ಣ-ಗ-ಳಿಗೆ
ವರ್ಣ-ಗ-ಳಿವೆ
ವರ್ಣ-ಗಳು
ವರ್ಣ-ಗ-ಳೆಲ್ಲ
ವರ್ಣದ
ವರ್ಣ-ದ-ವರೂ
ವರ್ಣ-ನಾ-ತೀತ
ವರ್ಣನೆ
ವರ್ಣ-ನೆ-ಗಳೂ
ವರ್ಣ-ನೆಗೆ
ವರ್ಣ-ನೆಯೇ
ವರ್ಣ-ವಿ-ಭಾ-ಗವೂ
ವರ್ಣಾ
ವರ್ಣಾ-ಶ್ರಮ
ವರ್ಣಾ-ಶ್ರ-ಮದ
ವರ್ಣಾ-ಶ್ರ-ಮ-ಧರ್ಮ
ವರ್ಣಾ-ಶ್ರ-ಮ-ಧ-ರ್ಮ-ಗಳನ್ನು
ವರ್ಣಾ-ಶ್ರ-ಮ-ಧ-ರ್ಮ-ಗಳನ್ನೆಲ್ಲಾ
ವರ್ಣಾ-ಶ್ರ-ಮ-ಧ-ರ್ಮ-ಗ-ಳಿ-ಗಾ-ಗಿಯೂ
ವರ್ಣಾ-ಶ್ರ-ಮ-ಧ-ರ್ಮ-ಗ-ಳುಂಟೆ
ವರ್ಣಿ-ಸಲು
ವರ್ಣಿಸಿ
ವರ್ಣಿ-ಸಿದ
ವರ್ಣಿ-ಸಿ-ದಳು
ವರ್ಣಿ-ಸು-ತ್ತಿ-ದ್ದರು
ವರ್ಣಿ-ಸುವ
ವರ್ಣಿ-ಸು-ವವ
ವರ್ಣಿ-ಸು-ವು-ದಕ್ಕೆ
ವರ್ತ
ವರ್ತ-ಮಾ-ನ-ಕಾ-ಲ-ಗಳಲ್ಲಿ
ವರ್ತಿಯು
ವರ್ಷ
ವರ್ಷ-ಕಾಲ
ವರ್ಷಕ್ಕೆ
ವರ್ಷ-ಕ್ಕೊಂ-ದ-ರಂತೆ
ವರ್ಷ-ಗಳ
ವರ್ಷ-ಗಳನ್ನೂ
ವರ್ಷ-ಗ-ಳ-ಲ್ಲಿಯೂ
ವರ್ಷ-ಗ-ಳ-ವ-ರೆಗೆ
ವರ್ಷ-ಗ-ಳಾಗಿ
ವರ್ಷ-ಗ-ಳಾ-ಗಿತ್ತು
ವರ್ಷ-ಗ-ಳಾ-ಗಿ-ದ್ದುವು
ವರ್ಷ-ಗ-ಳಾ-ಗು-ತ್ತಲೆ
ವರ್ಷ-ಗ-ಳಾದ
ವರ್ಷ-ಗ-ಳಿವೆ
ವರ್ಷ-ಗಳು
ವರ್ಷ-ಗ-ಳೆಲ್ಲ
ವರ್ಷದ
ವರ್ಷ-ದ-ಲ್ಲಿಯೂ
ವರ್ಷ-ದ-ವ-ರೆಗೂ
ವರ್ಷ-ದಷ್ಟು
ವರ್ಷ-ವಾದ
ವರ್ಷ-ವಾ-ದ-ಮೇಲೆ
ವರ್ಷವೂ
ವರ್ಷ್ಮಣೇ
ವಲ್ಕಲ
ವಲ್ಲ
ವಲ್ಲ-ಎಂದು
ವಲ್ಲ-ತ-ಕಂಠ
ವಲ್ಲದೆ
ವಲ್ಲ-ಭಾ-ಚಾ-ರ್ಯರು
ವಲ್ಲಿ
ವಲ್ಲು-ಗಳ
ವಳಾಗಿ
ವಳಿ-ಕೆ-ಗಳನ್ನೂ
ವಳು
ವಳೆ
ವವನ
ವವ-ನಂತೆ
ವವ-ನಲ್ಲ
ವವ-ನಾ-ಗಿದ್ದ
ವವ-ನಾ-ಗಿಯೂ
ವವ-ನಾ-ದ್ದ-ರಿಂದ
ವವ-ನಿಗೆ
ವವನು
ವವ-ನು-ಎಂಬ
ವವನೂ
ವವನೇ
ವವರ
ವವ-ರನ್ನು
ವವ-ರಿಗೆ
ವವರು
ವವರೂ
ವವರೆ
ವಶಕ್ಕೆ
ವಶ-ಕ್ಕೊ-ಪ್ಪಿ-ಸಿ-ದನು
ವಶ-ನಾಗಿ
ವಶ-ನಾ-ಗಿ-ಹೋ-ಗು-ತ್ತಾನೆ
ವಶ-ನಾಗು
ವಶ-ಪ-ಡಿ-ಸಿ-ಕೊಂಡ
ವಶ-ಪ-ಡಿ-ಸಿ-ಕೊಂ-ಡನು
ವಶ-ಪ-ಡಿ-ಸಿ-ಕೊಂ-ಡಿ-ದ್ದಾರೆ
ವಶ-ಮಾ-ಡಿ-ಕೊಂ-ಡರೆ
ವಶ-ವರ್ತಿ
ವಶ-ವ-ರ್ತಿ-ಯಾಗಿ
ವಶ-ವ-ರ್ತಿ-ಯಾ-ಗಿವೆ
ವಶ-ವಾ-ಗ-ದಂತೆ
ವಶ-ವಾಗಿ
ವಶ-ವಾ-ಗು-ವುದು
ವಶ-ವಾ-ದುವು
ವಷಟ್
ವಷ್ಟ-ರಲ್ಲಿ
ವಷ್ಟು
ವಷ್ಟೆ
ವಸಂತ
ವಸಂ-ತ-ಋತು
ವಸಂ-ತ-ಋ-ತು-ವಿನ
ವಸಂ-ತ-ಕಾ-ಲದ
ವಸಂ-ತ-ದಂ-ತೆಯೇ
ವಸ-ತಿಗೂ
ವಸ-ತಿ-ಯನ್ನೂ
ವಸಿಷ್ಠ
ವಸಿ-ಷ್ಠನು
ವಸಿ-ಷ್ಠನೂ
ವಸಿ-ಷ್ಠರ
ವಸಿ-ಷ್ಠ-ರನ್ನು
ವಸಿ-ಷ್ಠ-ರಲ್ಲಿ
ವಸಿ-ಷ್ಠ-ರಿಗೆ
ವಸಿ-ಷ್ಠರು
ವಸು
ವಸುಂ-ಧರ
ವಸು-ದೇವ
ವಸು-ದೇ-ವ-ಇ-ವರೂ
ವಸು-ದೇ-ವನ
ವಸು-ದೇ-ವ-ನಂತೆ
ವಸು-ದೇ-ವ-ನಂ-ದ-ನಾಂ-ಘ್ರಿಂ
ವಸು-ದೇ-ವ-ನನ್ನು
ವಸು-ದೇ-ವ-ನಲ್ಲಿ
ವಸು-ದೇ-ವ-ನಿಂದ
ವಸು-ದೇ-ವ-ನಿಗೂ
ವಸು-ದೇ-ವ-ನಿಗೆ
ವಸು-ದೇ-ವನು
ವಸು-ದೇ-ವನೂ
ವಸು-ದೇ-ವನೆ
ವಸು-ದೇ-ವರ
ವಸು-ದೇ-ವ-ರನ್ನು
ವಸು-ದೇ-ವರು
ವಸು-ವಾ-ಗಿ-ದ್ದಾಗ
ವಸ್ತು
ವಸ್ತು-ಅ-ವೆ-ರಡೂ
ವಸ್ತು-ಗಳನ್ನು
ವಸ್ತು-ಗಳನ್ನೂ
ವಸ್ತು-ಗಳನ್ನೆಲ್ಲ
ವಸ್ತು-ಗಳಲ್ಲಿ
ವಸ್ತು-ಗ-ಳ-ಲ್ಲಿಯೂ
ವಸ್ತು-ಗ-ಳಾ-ಗಿ-ದ್ದವು
ವಸ್ತು-ಗಳಿ
ವಸ್ತು-ಗ-ಳಿಂ-ದಾ-ಗಲಿ
ವಸ್ತು-ಗ-ಳಿ-ಗಾಗಿ
ವಸ್ತು-ಗಳು
ವಸ್ತು-ಗ-ಳೆಲ್ಲ
ವಸ್ತು-ಗ-ಳೆ-ಲ್ಲವೂ
ವಸ್ತು-ಜ್ಞಾನ
ವಸ್ತುತಃ
ವಸ್ತು-ವನ್ನು
ವಸ್ತು-ವನ್ನೆ
ವಸ್ತು-ವಾ-ಗ-ಬೇ-ಕಾದ
ವಸ್ತು-ವಾ-ಗಿತ್ತು
ವಸ್ತು-ವಾ-ಗಿದೆ
ವಸ್ತು-ವಾ-ಗಿ-ದ್ದರೆ
ವಸ್ತು-ವಾ-ವು-ದನ್ನೂ
ವಸ್ತು-ವಿನ
ವಸ್ತುವೂ
ವಸ್ತು-ವೆಂದು
ವಸ್ತುವೇ
ವಸ್ತು-ಸ್ಥಿತಿ
ವಸ್ತ್ರ
ವಸ್ತ್ರ-ಗಳನ್ನು
ವಸ್ತ್ರ-ಗಳು
ವಸ್ತ್ರ-ಧಾ-ರ-ಣೆಯೆ
ವಸ್ತ್ರ-ಭೂ-ಷ-ಣ-ಗ-ಳ-ನ್ನಿತ್ತು
ವಸ್ತ್ರ-ಭೂ-ಷ-ಣ-ಗಳಿಂದ
ವಸ್ತ್ರ-ವನ್ನು
ವಸ್ತ್ರ-ವನ್ನೂ
ವಸ್ತ್ರ-ವೆಂ-ದಾಗ
ವಸ್ತ್ರಾ
ವಸ್ತ್ರಾ-ಪ-ಹ-ರಣ
ವಸ್ತ್ರಾ-ಭ-ರ-ಣ-ಗಳ
ವಸ್ತ್ರಾ-ಭ-ರ-ಣ-ಗಳನ್ನು
ವಸ್ತ್ರಾ-ಭ-ರ-ಣ-ಗಳನ್ನೂ
ವಸ್ತ್ರಾ-ಭ-ರ-ಣ-ಗಳಿಂದ
ವಸ್ತ್ರಾ-ಭ-ರ-ಣ-ಗಳೂ
ವಸ್ತ್ರಾ-ಲಂ-ಕಾ-ರ-ಭೂ-ಷಿತ
ವಹತು
ವಹಿಸಿ
ವಹಿ-ಸಿ-ಕೊಂಡ
ವಹಿ-ಸಿ-ಕೊಂಡು
ವಹಿ-ಸಿ-ಕೊಟ್ಟು
ವಹಿ-ಸಿ-ದನು
ವಹಿ-ಸಿ-ದರು
ವಹಿ-ಸಿ-ದ-ವ-ರ-ನ್ನೆಲ್ಲ
ವಹಿ-ಸಿ-ದ್ದಾನೆ
ವಹಿಸು
ವಹ್ನಿ
ವಾ
ವಾಂ
ವಾಕಿಂಗ್
ವಾಕ್
ವಾಕ್ಕು
ವಾಕ್ಯ-ಗಳನ್ನು
ವಾಗ
ವಾಗ-ದಂತೆ
ವಾಗ-ದೆಂಬ
ವಾಗ-ಬ-ಲ್ಲದು
ವಾಗ-ಬೇಕು
ವಾಗಲಿ
ವಾಗ-ಲೆಲ್ಲ
ವಾಗಿ
ವಾಗಿತ್ತು
ವಾಗಿದೆ
ವಾಗಿ-ದೆಯೋ
ವಾಗಿದ್ದ
ವಾಗಿದ್ದು
ವಾಗಿಯೂ
ವಾಗಿಯೆ
ವಾಗಿರ
ವಾಗಿ-ರ-ಬೇಕು
ವಾಗಿ-ರಲಿ
ವಾಗಿ-ರು-ತ್ತದೆ
ವಾಗು-ತ್ತದೆ
ವಾಗು-ತ್ತ-ದೆಯೋ
ವಾಗು-ತ್ತಲೆ
ವಾಗು-ತ್ತವೆ
ವಾಗುತ್ತಾ
ವಾಗು-ವಂತೆ
ವಾಗು-ವು-ದಕ್ಕೆ
ವಾಗು-ವು-ದಿಲ್ಲ
ವಾಗು-ವುದು
ವಾಙ್ಮ-ಯ-ದಲ್ಲಿ
ವಾಚಾ
ವಾಜ-ಪೇ-ಯ-ವೆಂಬ
ವಾಡಿ-ಕೆಗೆ
ವಾಡು-ತ್ತಿದ್ದು
ವಾಣಿ
ವಾತಾ-ವ-ರಣ
ವಾತಾ-ವ-ರ-ಣ-ದಲ್ಲಿ
ವಾತಾ-ವ-ರ-ಣ-ವೆಲ್ಲ
ವಾತ್ಸಲ್ಯ
ವಾತ್ಸ-ಲ್ಯ-ದಿಂದ
ವಾತ್ಸ-ಲ್ಯ-ಭ-ಕ್ತಿ-ಯನ್ನು
ವಾದ
ವಾದಂದು
ವಾದನು
ವಾದರೂ
ವಾದವು
ವಾದಷ್ಟು
ವಾದಾಗ
ವಾದಿ-ಸಿ-ದರೆ
ವಾದೀತು
ವಾದು-ದನ್ನು
ವಾದು-ದ-ರಿಂದ
ವಾದು-ದಲ್ಲ
ವಾದುದು
ವಾದು-ದೆಂದೂ
ವಾದುವು
ವಾದ್ದ-ರಿಂದ
ವಾದ್ಯ-ಗಳನ್ನು
ವಾದ್ಯ-ಗಳು
ವಾದ್ಯ-ಗ-ಳೊ-ಡನೆ
ವಾನ-ಪ್ರ-ಸ್ಥ-ವೆಂಬ
ವಾನರ
ವಾನ-ರ-ನನ್ನು
ವಾನಾ-ಹಾ-ಟಿನ
ವಾಮ
ವಾಮ-ದೇವ
ವಾಮನ
ವಾಮ-ನನ
ವಾಮ-ನ-ನಿಗೆ
ವಾಮ-ನನು
ವಾಮ-ನ-ನೆಂದು
ವಾಮ-ನ-ನೆಂಬ
ವಾಮ-ನ-ಮೂ-ರ್ತಿ-ಯನ್ನು
ವಾಮ-ನ-ಮೂ-ರ್ತಿಯು
ವಾಮ-ನಾ-ವ-ತಾ-ರಿ-ಯಾದ
ವಾಮ-ನಾ-ವ-ತಾ-ರ್-ಅ-ದಲ್ಲಿ
ವಾಯಿತು
ವಾಯಿ-ತೆಂ-ದರೆ
ವಾಯಿ-ತೆಂದು
ವಾಯು
ವಾಯು-ದೇ-ವನು
ವಾಯು-ದೇ-ವರು
ವಾಯು-ಪು-ರಾ-ಣ-ದ
ವಾಯು-ಪು-ರಾ-ಣದ
ವಾಯು-ಮಂ-ಡಲ
ವಾಯು-ರೂಪೀ
ವಾಯು-ವಿನ
ವಾಯು-ವಿ-ನಿಂದ
ವಾಯುವು
ವಾಯುವೂ
ವಾಯು-ಸಖೋ
ವಾಯ್ತೆಂದು
ವಾರು
ವಾರು-ಣಾಸ್ತ್ರ
ವಾರುಣಿ
ವಾರು-ಣಿ-ಯನ್ನು
ವಾರ್ಕ್ಷಿ
ವಾರ್ಕ್ಷಿ-ಎಂದು
ವಾರ್ಕ್ಷಿ-ಯನ್ನು
ವಾರ್ಕ್ಷಿ-ಯಲ್ಲಿ
ವಾಲಿ-ಯನ್ನು
ವಾಲ್ಮೀಕಿ
ವಾಲ್ಮೀ-ಕಿಗೆ
ವಾಳ
ವಾಸ
ವಾಸಕ್ಕೆ
ವಾಸ-ದ-ವ-ರೊ-ಡನೆ
ವಾಸ-ನಾ-ನು-ಸಾ-ರ-ವಾಗಿ
ವಾಸ-ನಾ-ಸ-ಮುದ್ರ
ವಾಸನೆ
ವಾಸ-ನೆ-ಗ-ಳೆಲ್ಲ
ವಾಸ-ನೆಯ
ವಾಸ-ನೆ-ಯನ್ನು
ವಾಸ-ನೆ-ಯಾ-ದರೂ
ವಾಸ-ನೆಯೂ
ವಾಸ-ಮಾ-ಡಲು
ವಾಸ-ಮಾ-ಡು-ತ್ತಿ-ದ್ದರು
ವಾಸ-ವಾ-ಗಿದ್ದ
ವಾಸವು
ವಾಸ-ಸ್ಥ-ಳ-ವಾದ
ವಾಸ-ಸ್ಥಾನ
ವಾಸ-ಸ್ಥಾ-ನ-ವನ್ನು
ವಾಸಿ
ವಾಸಿ-ಯಾ-ಗು-ವು-ದೇನೂ
ವಾಸಿ-ಸುತ್ತಾ
ವಾಸಿ-ಸು-ತ್ತಿ-ದ್ದವು
ವಾಸಿ-ಸು-ತ್ತಿ-ರುವ
ವಾಸಿ-ಸು-ತ್ತಿ-ರುವೆ
ವಾಸು
ವಾಸುಕಿ
ವಾಸು-ಕಿಯ
ವಾಸು-ಕಿಯೇ
ವಾಸು-ದೇವ
ವಾಸು-ದೇ-ವ
ವಾಸು-ದೇ-ವ-ಸಂ-ಕ-ರ್ಷಣ
ವಾಸು-ದೇ-ವನ
ವಾಸು-ದೇ-ವ-ನ-ದಲ್ಲ
ವಾಸು-ದೇ-ವ-ನದು
ವಾಸು-ದೇ-ವ-ನನ್ನು
ವಾಸು-ದೇ-ವ-ನಾದ
ವಾಸು-ದೇ-ವ-ನಿಗೆ
ವಾಸು-ದೇ-ವನು
ವಾಸು-ದೇ-ವಾಯ
ವಾಸು-ದೇ-ವೋ-ಪಾ-ಸನೆ
ವಾಸ್ರಜಂ
ವಾಹನ
ವಾಹ-ನ-ಗಳನ್ನು
ವಾಹ-ನ-ಗಳು
ವಾಹ-ನ-ವಾದ
ವಾಹ-ನ-ವಾ-ದು-ದ-ರಿಂದ
ವಿಂ
ವಿಂಗ-ಡಿಸ
ವಿಂಗ-ಡಿ-ಸಿ-ದನು
ವಿಂದ
ವಿಂದ-ಗಳಲ್ಲಿ
ವಿಂದ-ಯು-ಗಳ
ವಿಂದಾ-ನು-ವಿಂ-ದರು
ವಿಂಧ್ಯ-ಪ-ರ್ವತ್ಕೆ
ವಿಂಧ್ಯಾ-ವ-ಳಿಯು
ವಿಂಧ್ಯಾ-ವ-ಳಿಯೂ
ವಿಕ
ವಿಕ-ಟ-ವೇ-ಷದ
ವಿಕ-ಟಾ-ಕಾ-ರದ
ವಿಕಲಂ
ವಿಕ-ಲ್ಪ-ಗಳನ್ನು
ವಿಕ-ಲ್ಪ-ರ-ಹಿ-ತ-ಸ್ಸ್ವಯಂ
ವಿಕಾರ
ವಿಕಾ-ರಕ್ಕೂ
ವಿಕಾ-ರ-ಗ-ಳಿಗೆ
ವಿಕಾ-ರ-ಗಳು
ವಿಕಾ-ರ-ಗಳೇ
ವಿಕಾ-ರ-ಗ-ಳೇ-ನಿ-ದ್ದರೂ
ವಿಕಾ-ರ-ಗ-ಳೊಂ-ದಕ್ಕೂ
ವಿಕಾ-ರ-ಧ್ವ-ನಿ-ಯಿಂದ
ವಿಕಾ-ರ-ರೂ-ಪಿನ
ವಿಕಾ-ರ-ವನ್ನು
ವಿಕಾ-ರ-ವಾ-ವುದೂ
ವಿಕಾ-ಸದ
ವಿಕುಕ್ಷಿ
ವಿಕು-ಕ್ಷಿಯ
ವಿಕ್ರ-ಯ-ಗಳಲ್ಲಿ
ವಿಖ್ಯಾ-ತ-ನಾ-ಗಿ-ದ್ದನು
ವಿಗದಂ
ವಿಗು-ಣಮ್
ವಿಗೆ
ವಿಗ್ರಹ
ವಿಗ್ರ-ಹಕ್ಕೆ
ವಿಗ್ರ-ಹ-ಗಳು
ವಿಗ್ರ-ಹದ
ವಿಗ್ರ-ಹ-ದಲ್ಲಿ
ವಿಗ್ರ-ಹ-ದೆ-ದು-ರಿಗೆ
ವಿಗ್ರ-ಹ-ವನ್ನು
ವಿಗ್ರ-ಹವೇ
ವಿಘ್ನ
ವಿಘ್ನ-ವಾ-ಗ-ದಂ-ತೆಯೂ
ವಿಘ್ನ-ವಾ-ದು-ದನ್ನು
ವಿಚಾರ
ವಿಚಾ-ರ-ಗಳನ್ನು
ವಿಚಾ-ರ-ಗಳಲ್ಲಿ
ವಿಚಾ-ರ-ಗ-ಳ-ಲ್ಲಿಯೂ
ವಿಚಾ-ರ-ಗಳೂ
ವಿಚಾ-ರ-ದಲ್ಲಿ
ವಿಚಾ-ರ-ದ-ಲ್ಲಿಯೇ
ವಿಚಾ-ರ-ದಲ್ಲೂ
ವಿಚಾ-ರ-ಪ-ರ-ನಾದ
ವಿಚಾ-ರ-ಪ-ರ-ರಾದ
ವಿಚಾ-ರ-ಮಾಡಿ
ವಿಚಾ-ರ-ವನ್ನು
ವಿಚಾ-ರ-ವಾಗಿ
ವಿಚಾ-ರ-ವಾ-ಗಿಯೆ
ವಿಚಾರಿ
ವಿಚಾ-ರಿ-ಸ-ಬೇ-ಕಾದ
ವಿಚಾ-ರಿ-ಸ-ಬೇ-ಡವೆ
ವಿಚಾ-ರಿ-ಸಲು
ವಿಚಾ-ರಿಸಿ
ವಿಚಾ-ರಿ-ಸಿದ
ವಿಚಾ-ರಿ-ಸಿ-ದನು
ವಿಚಾ-ರಿ-ಸಿ-ದರು
ವಿಚಾ-ರಿ-ಸುತ್ತಾ
ವಿಚಾ-ರಿ-ಸು-ವಂತೆ
ವಿಚಾ-ರಿ-ಸು-ವು-ದ-ಕ್ಕಾಗಿ
ವಿಚಾ-ರಿ-ಸೋಣ
ವಿಚಾರು
ವಿಚಿತ್ರ
ವಿಚಿತ್ರಃ
ವಿಚಿ-ತ್ರ-ವಾದ
ವಿಚಿ-ತ್ರ-ವಾದು
ವಿಚಿ-ತ್ರ-ವಾ-ದುದು
ವಿಚಿ-ತ್ರ-ವೀ-ರ್ಯನ
ವಿಜಯ
ವಿಜ-ಯ-ದಿಂದ
ವಿಜ-ಯ-ನೆಂ-ಬು-ವನ
ವಿಜ-ಯ-ಯಾತ್ರೆ
ವಿಜ-ಯ-ರಿಗೆ
ವಿಜ-ಯರು
ವಿಜ-ಯ-ರೆಂಬ
ವಿಜ-ಯ-ಸ-ಖೀ-ನಾಂ
ವಿಜಯಿ
ವಿಜ-ಯಿ-ಯಾದ
ವಿಜ-ಯೀ-ಭವ
ವಿಜಿ-ತಾ-ಶ್ವನ
ವಿಜಿ-ತಾ-ಶ್ವ-ನನ್ನು
ವಿಜಿ-ತಾ-ಶ್ವನು
ವಿಜಿ-ತಾ-ಶ್ವ-ನೆಂದು
ವಿಜೃಂ-ಭ-ಣೆ-ಯಿಂದ
ವಿಜೃಂ-ಭಿ-ಸು-ತ್ತದೆ
ವಿಜೃಂ-ಭಿ-ಸು-ತ್ತಿ-ದ್ದರು
ವಿಜ್ಞಾತಂ
ವಿಜ್ಞಾತೇ
ವಿಜ್ಞಾ-ನ-ಮಾ-ತ್ರಾಯ
ವಿಜ್ಞಾ-ನ-ವಾಗಿ
ವಿಟ
ವಿಟ-ನನ್ನು
ವಿಟ-ಪು-ರು-ಷ-ನಿಗೆ
ವಿಟ-ಪು-ರು-ಷರು
ವಿಟರು
ವಿಟ-ರೆಂದು
ವಿಟ್ಟು
ವಿಡಂಬಂ
ವಿಡೂ-ರ-ಥನು
ವಿತತಂ
ವಿತ-ನುಮ್
ವಿತಳ
ವಿತ್ತ-ನಲ್ಲ
ವಿತ್ತು
ವಿದರ್ಭ
ವಿದ-ರ್ಭ-ದೇ-ಶದ
ವಿದ-ರ್ಭ-ರಾ-ಜ-ನಾದ
ವಿದ-ರ್ಭ-ರಾ-ಜ್ಯದ
ವಿದ-ಸ್ಥೂಲ
ವಿದಾಮ
ವಿದಿಕ್ಷು
ವಿದುರ
ವಿದು-ರನ
ವಿದು-ರ-ನನ್ನು
ವಿದು-ರ-ನಿಗೆ
ವಿದು-ರನು
ವಿದು-ರನೂ
ವಿದು-ರರು
ವಿದು-ರ-ರೊ-ಡನೆ
ವಿದು-ರ್ಮನೋ
ವಿದು-ಷ-ಸ್ತೇ-ಭ್ಯೇತ್ಯ
ವಿದುಷಾ
ವಿದೆ
ವಿದೇಹ
ವಿದೇ-ಹ-ನೆಂದೂ
ವಿದ್ದಾ-ಗಲೆ
ವಿದ್ಯಾ
ವಿದ್ಯಾ-ತೇ-ಸ್ತ-ಜ-ಪೋ-ಮೂ-ರ್ತಿ-ಮಿಮಂ
ವಿದ್ಯಾ-ಧರ
ವಿದ್ಯಾ-ಧ-ರರ
ವಿದ್ಯಾ-ಧ-ರ-ರನ್ನೂ
ವಿದ್ಯಾ-ಧ-ರರು
ವಿದ್ಯಾ-ಧ-ರರೂ
ವಿದ್ಯಾ-ಧ-ರಾದಿ
ವಿದ್ಯಾ-ಧ-ರಾ-ದಿ-ಗಳು
ವಿದ್ಯಾ-ಪಾ-ರಂ-ಗ-ತ-ನಾಗಿ
ವಿದ್ಯಾ-ಭ್ಯಾಸ
ವಿದ್ಯಾ-ಭ್ಯಾ-ಸ-ಕ್ಕಾಗಿ
ವಿದ್ಯಾ-ಭ್ಯಾ-ಸ-ಮಾಡಿ
ವಿದ್ಯಾ-ಭ್ಯಾ-ಸ-ವನ್ನು
ವಿದ್ಯಾ-ರ್ಜನೆ
ವಿದ್ಯಾರ್ಥಿ
ವಿದ್ಯಾ-ವಂತ
ವಿದ್ಯಾ-ವಂ-ತರು
ವಿದ್ಯೆ
ವಿದ್ಯೆ-ಗಳನ್ನು
ವಿದ್ಯೆ-ಗಳನ್ನೂ
ವಿದ್ಯೆ-ಗ-ಳಲ್ಲೂ
ವಿದ್ಯೆ-ಗ-ಳಿಗೂ
ವಿದ್ಯೆ-ಗಳು
ವಿದ್ಯೆಯ
ವಿದ್ಯೆ-ಯನ್ನು
ವಿದ್ಯೆ-ಯಲ್ಲಿ
ವಿದ್ಯೆ-ಯಿಂದ
ವಿದ್ರಾ-ವಯ
ವಿದ್ವಾಂಸ
ವಿದ್ವಾಂ-ಸರು
ವಿಧ
ವಿಧ-ದ-ಲ್ಲಿಯೂ
ವಿಧ-ದಿಂದ
ವಿಧ-ದಿಂ-ದಲೂ
ವಿಧ-ವಾದ
ವಿಧ-ವೆ-ಯ-ರಾಗಿ
ವಿಧಾ-ನ-ಗ-ಳಿ-ವೆ-ದೇ-ಹ-ಮ-ನ-ಸ್ಸು-ಗ-ಳೆ-ರಡೂ
ವಿಧಾ-ನ-ಗಳೂ
ವಿಧಾ-ನ-ವನ್ನು
ವಿಧಾ-ನ-ವನ್ನೂ
ವಿಧಿ
ವಿಧಿ-ಪೂ-ರ್ವಕ
ವಿಧಿ-ಪೂ-ರ್ವ-ಕ-ವಾಗಿ
ವಿಧಿ-ವಿ-ಧಾ-ನ-ಗಳನ್ನು
ವಿಧಿಸಿ
ವಿಧಿ-ಸಿ-ದ್ದರು
ವಿಧಿ-ಸುವ
ವಿಧಿ-ಸು-ವೆನು
ವಿಧಿ-ಸು-ವೆಯೋ
ವಿಧೇ-ಯ-ರಾಗಿ
ವಿಧೇ-ಯರು
ವಿನ
ವಿನಃ
ವಿನ-ಮ್ರ-ನಾ-ಗು-ತ್ತಿ-ದ್ದನು
ವಿನಯ
ವಿನ-ಯ-ದಿಂದ
ವಿನ-ಯ-ಶೀ-ಲರು
ವಿನ-ಷ್ಟಾಃ
ವಿನಾ-ಶ-ಕಾ-ಲಕ್ಕೆ
ವಿನಾ-ಶ-ಹೊಂ-ದು-ತ್ತದೆ
ವಿನಿಃ-ಸೃ-ತಾಃ
ವಿನಿ-ಯೋ-ಗಿ-ಸಿ-ದರೂ
ವಿನೇ-ದು-ರ್ನ್ಯ-ಪ-ತಂಶ್ಚ
ವಿನೋದ
ವಿನೋ-ದ-ಕೇ-ಳಿ-ಗೆಂದು
ವಿನೋ-ದಕ್ಕೂ
ವಿನೋ-ದದ
ವಿನೋ-ದ-ದ-ಲ್ಲಿ-ದ್ದಾನೆ
ವಿನೋ-ದ-ದಿಂದ
ವಿನೋ-ದ-ವನ್ನು
ವಿನೋ-ದ-ವಾಗಿ
ವಿನೋ-ದ-ವಾ-ಗಿ-ದ್ದುದು
ವಿನೋ-ದ-ವೆಂದು
ವಿಪ
ವಿಪ-ತ್ತನ್ನು
ವಿಪತ್ತು
ವಿಪ-ತ್ತು-ಗಳಿಂದ
ವಿಪ-ರೀತ
ವಿಪ-ರೀ-ತ-ಜ್ಞಾನ
ವಿಪುಲ
ವಿಪ್ರ-ಚಿತ್ತ
ವಿಪ್ರರ
ವಿಪ್ರ-ವಾಸೇ
ವಿಫಲ
ವಿಫ-ಲ-ರಾ-ದರು
ವಿಫ-ಲ-ವಾ-ಗಲು
ವಿಫ-ಲ-ವಾ-ದವು
ವಿಬುಧ
ವಿಭ-ಜನ್
ವಿಭಾಗ
ವಿಭಾ-ಗ-ಕಾರ್ಯ
ವಿಭಾ-ಗ-ವಾ-ಗುವ
ವಿಭಾ-ಗ-ವಾ-ದರು
ವಿಭಾ-ಗಶಃ
ವಿಭಾಗಿ
ವಿಭಾ-ಗಿ-ಸಿ-ಕೊ-ಡುವ
ವಿಭಾ-ಗಿ-ಸಿ-ದನು
ವಿಭಾ-ಗಿ-ಸು-ತ್ತಾರೆ
ವಿಭೀ-ಷ-ಣ-ನನ್ನು
ವಿಭೂ-ತ-ದ್ವಂ-ದ್ವ-ಧರ್ಮಾ
ವಿಭೂ-ತಿ-ಗಳೇ
ವಿಭೂ-ತಿ-ಯನ್ನು
ವಿಭೂ-ತಿ-ಯಿಂದ
ವಿಭ್ರಮ
ವಿಮದಂ
ವಿಮ-ರ್ಶ-ಕರ
ವಿಮ-ರ್ಶಿ-ಸುವ
ವಿಮ-ರ್ಶೆ-ಯಿಂದ
ವಿಮ-ಲ-ಎಂಬ
ವಿಮ-ಲಮ್
ವಿಮಾನ
ವಿಮಾ-ನಕ್ಕೆ
ವಿಮಾ-ನ-ಗ-ಳಂತೆ
ವಿಮಾ-ನ-ಗಳಲ್ಲಿ
ವಿಮಾ-ನದ
ವಿಮಾ-ನ-ದತ್ತ
ವಿಮಾ-ನ-ದಿಂದ
ವಿಮಾ-ನ-ವನ್ನು
ವಿಮಾ-ನ-ವ-ನ್ನೇರಿ
ವಿಮಾ-ನ-ವ-ನ್ನೇ-ರಿ-ದರು
ವಿಮಾನವು
ವಿಮಾ-ನ-ವೇರಿ
ವಿಮಾ-ನ-ವೊಂ-ದನ್ನು
ವಿಮಾ-ನ-ವೊಂದು
ವಿಮುಂ-ಚತೋ
ವಿಮು-ಕ್ತ-ರಾಗಿ
ವಿಮು-ಖ-ತೆ-ಯೇನೂ
ವಿಮು-ಖ-ನಾದ
ವಿಮು-ಖ-ರಾಗಿ
ವಿಮು-ಖ-ಳ-ನ್ನಾಗಿ
ವಿಮು-ಖ-ವಾ-ದರೆ
ವಿಮೋ-ಚ-ನೆ-ಗಾಗಿ
ವಿಮೋ-ಚ-ನೆ-ಯಾ-ಗ-ಲೆಂದು
ವಿಯಂತೆ
ವಿರ-ಕ್ತ-ಚಿ-ತ್ತ-ನಾಗಿ
ವಿರ-ಕ್ತ-ನಾಗಿ
ವಿರ-ಕ್ತ-ನಾ-ದನು
ವಿರ-ಕ್ತ-ರನ್ನು
ವಿರ-ಕ್ತ-ರನ್ನೇ
ವಿರ-ಕ್ತ-ರಾಗಿ
ವಿರ-ಕ್ತರು
ವಿರ-ಕ್ತರೇ
ವಿರಕ್ತಿ
ವಿರ-ಕ್ತಿ-ಇದೇ
ವಿರ-ಕ್ತಿ-ಗಳೂ
ವಿರತಂ
ವಿರಳ
ವಿರ-ಹ-ತಾಪ
ವಿರ-ಹ-ತಾ-ಪದ
ವಿರ-ಹ-ತಾ-ಪ-ವನ್ನು
ವಿರ-ಹ-ವೇ-ದನೆ
ವಿರ-ಹ-ವೇ-ದ-ನೆ-ಯನ್ನು
ವಿರ-ಹ-ವೇ-ದ-ನೆ-ಯಿಂದ
ವಿರ-ಹಾ-ಗ್ನಿ-ತ-ಪ್ತ-ಹೃ-ದ-ಯ-ಯೇ-ಷ್ವಲಂ
ವಿರ-ಹಾ-ಗ್ನಿ-ಯಿಂದ
ವಿರ-ಹಾ-ಗ್ನಿಯು
ವಿರಾ-ಜ-ಮಾ-ನ-ವಾ-ಗಿತ್ತು
ವಿರಾಟ್
ವಿರಾ-ಟ್ಪು-ರು-ಷ-ನಿಂದ
ವಿರಾ-ಟ್ಪು-ರು-ಷನು
ವಿರಾ-ಟ್ಪು-ರು-ಷನೇ
ವಿರಾ-ಟ್ರೂ-ಪ-ನಾದ
ವಿರಾ-ಟ್ರೂ-ಪ-ವೆ-ನ್ನು-ತ್ತಾರೆ
ವಿರಾ-ಡ್ರೂ-ಪದ
ವಿರಾ-ಡ್ರೂ-ಪ-ದಲ್ಲಿ
ವಿರಾ-ಡ್ರೂ-ಪ-ನಿಂದ
ವಿರೂ-ಪ-ಗೊ-ಳಿ-ಸಿದ
ವಿರೋ-ಧ-ವ-ಲ್ಲ-ವಾ-ದರೆ
ವಿರೋ-ಧ-ವಾ-ಗಿಲ್ಲ
ವಿರೋ-ಧ-ವಾದ
ವಿರೋ-ಧ-ವಾ-ದುದು
ವಿರೋ-ಧಿ-ಗ-ಳ-ಲ್ಲಿಯೂ
ವಿರ್ಯಾಸ್ಥ
ವಿಲ-ಕ್ಷಣ
ವಿಲ-ಕ್ಷ-ಣ-ನಾಗಿ
ವಿಲ-ಕ್ಷ-ಣ-ವಾದ
ವಿಲ-ಕ್ಷ-ಣ-ವಾ-ದುದು
ವಿಲ-ಪಸಿ
ವಿಲಾಸ
ವಿಲಾ-ಸಕ್ಕೂ
ವಿಲಾ-ಸ-ವತಿ
ವಿಲೀ-ನ-ಗೊ-ಳಿ-ಸಿ-ದನು
ವಿಲ್ಲ
ವಿಲ್ಲ-ದಂ-ತಹ
ವಿಲ್ಲ-ದಂತೆ
ವಿವರ
ವಿವ-ರ-ಗಳನ್ನು
ವಿವ-ರಣೆ
ವಿವ-ರ-ವನ್ನು
ವಿವ-ರ-ವಾಗಿ
ವಿವ-ರ-ವಿ-ವ-ರ-ವಾಗಿ
ವಿವ-ರಿಸಿ
ವಿವ-ರಿ-ಸಿ-ದರು
ವಿವ-ರಿ-ಸಿ-ದ್ದೇನೆ
ವಿವ-ರಿ-ಸಿ-ರು-ವನು
ವಿವ-ರಿ-ಸುತ್ತಾ
ವಿವ-ರಿ-ಸು-ತ್ತೇನೆ
ವಿವ-ರಿ-ಸು-ವಾಗ
ವಿವ-ರಿ-ಸು-ವುದು
ವಿವ-ಸ್ವ-ತ-ನೆಂಬ
ವಿವಾದ
ವಿವಾ-ದ-ವನ್ನು
ವಿವಾಹ
ವಿವಾ-ಹ-ಕಾಲ
ವಿವಾ-ಹಕ್ಕೆ
ವಿವಾ-ಹದ
ವಿವಾ-ಹ-ಮಾಡಿ
ವಿವಾ-ಹ-ಮಾ-ಡಿ-ಕೊ-ಟ್ಟ-ನಾ-ದರೂ
ವಿವಾ-ಹ-ಮಾ-ಡಿ-ಕೊ-ಡ-ಬೇ-ಕೆಂದು
ವಿವಾ-ಹ-ಮಾ-ಡಿ-ದಳು
ವಿವಾ-ಹ-ವನ್ನು
ವಿವಾ-ಹ-ವನ್ನೂ
ವಿವಾ-ಹ-ವಾ-ಗ-ಬೇ-ಕಾ-ದರೆ
ವಿವಾ-ಹ-ವಾ-ಗ-ಬೇ-ಕೆಂಬ
ವಿವಾ-ಹ-ವಾ-ಗಲು
ವಿವಾ-ಹ-ವಾ-ಗಿತ್ತು
ವಿವಾ-ಹ-ವಾ-ಗಿ-ದ್ದರು
ವಿವಾ-ಹ-ವಾಗು
ವಿವಾ-ಹ-ವಾ-ಗು-ವ-ವನೇ
ವಿವಾ-ಹ-ವಾ-ದನು
ವಿವಾ-ಹ-ವಾ-ಯಿತು
ವಿವಾ-ಹವು
ವಿವಿ-ಕ್ತ-ವಾದ
ವಿವಿಧ
ವಿವಿ-ಧಾಂ-ಶ-ಗ-ಳಿಗೆ
ವಿವೇಕ
ವಿವೇ-ಕ-ಜ್ಞಾನ
ವಿವೇ-ಕ-ಬೋಧೆ
ವಿವೇ-ಕ-ವನ್ನು
ವಿವೇ-ಕ-ವನ್ನೂ
ವಿವೇಕಾನಂದರು
ವಿವೇ-ಕಿ-ಗ-ಳಾದ
ವಿವೇ-ಕಿ-ಗ-ಳಾ-ದ-ವರು
ವಿವೇ-ಕಿ-ಯಾ-ಗಿ-ದ್ದರೆ
ವಿವೇ-ಕಿ-ಯಾ-ಗಿ-ರು-ವ-ದಂತೂ
ವಿವೇ-ಕಿ-ಯಾ-ದಾನು
ವಿವೇ-ಕಿ-ಯೆಂ-ದಾ-ಯಿತು
ವಿವೇ-ಚಿ-ಸೋಣ
ವಿವ್ಯಧೇ
ವಿಶಂತಿ
ವಿಶ-ದ-ವಾಗಿ
ವಿಶಾ-ಲ-ವಾದ
ವಿಶಿ-ಷ್ಟ-ವಾ-ದುದು
ವಿಶು-ದ್ಧ-ಎಂಬ
ವಿಶೇಷ
ವಿಶೇ-ಷ-ಣಾಂ
ವಿಶೇ-ಷ-ಣಾ-ಽನು-ಪ-ಲ-ಕ್ಷಿ-ತ-ಸ್ಥಾ-ನಾಯ
ವಿಶೇ-ಷ-ವಾಗಿ
ವಿಶೇ-ಷೈ-ರ್ವಿ-ಲ-ಕ್ಷಿ-ತಾ-ತ್ಮನೇ
ವಿಶೋ-ಧ-ನಾಯ
ವಿಶ್ರ-ಕೇ-ಶಿ-ವೃಕ
ವಿಶ್ರಮಿ
ವಿಶ್ರಾಂತಿ
ವಿಶ್ರಾಂ-ತಿ-ಗ-ಳಾದ
ವಿಶ್ರಾಂ-ತಿ-ಗಾಗಿ
ವಿಶ್ರಾಂ-ತಿ-ಗೃ-ಹ-ಹೆ-ಚ್ಚೇನು
ವಿಶ್ರಾಂ-ತಿ-ಯನ್ನು
ವಿಶ್ವ
ವಿಶ್ವ-ಕರ್ಮ
ವಿಶ್ವ-ಕ-ರ್ಮನ
ವಿಶ್ವ-ಕ-ರ್ಮನು
ವಿಶ್ವ-ಕ-ರ್ಮ-ನೆಂಬ
ವಿಶ್ವ-ಕು-ಟುಂಬಿ
ವಿಶ್ವ-ಜಿ-ತ್ಯಾ-ಗ-ವನ್ನು
ವಿಶ್ವ-ರೂಪ
ವಿಶ್ವ-ರೂಪಃ
ವಿಶ್ವ-ರೂ-ಪ-ದ-ರ್ಶ-ನದ
ವಿಶ್ವ-ರೂ-ಪನ
ವಿಶ್ವ-ರೂ-ಪ-ನಂತೂ
ವಿಶ್ವ-ರೂ-ಪ-ನಿಗೆ
ವಿಶ್ವ-ರೂ-ಪನು
ವಿಶ್ವ-ರೂ-ಪ-ವನ್ನು
ವಿಶ್ವ-ರೂ-ಪ-ವನ್ನೆ
ವಿಶ್ವ-ವೆ-ಲ್ಲವೂ
ವಿಶ್ವ-ವ್ಯಾ-ಪಿ-ಯಾ-ದರೂ
ವಿಶ್ವ-ಸ್ಫೂ-ರ್ಜಿ-ಯೆಂ-ಬು-ವನು
ವಿಶ್ವಾ-ಮಿತ್ರ
ವಿಶ್ವಾ-ಮಿ-ತ್ರನ
ವಿಶ್ವಾ-ಮಿ-ತ್ರ-ಮ-ಹ-ರ್ಷಿ-ಯಿಂದ
ವಿಶ್ವಾ-ಮಿ-ತ್ರರ
ವಿಶ್ವಾ-ಮಿ-ತ್ರ-ರನ್ನು
ವಿಶ್ವಾ-ಮಿ-ತ್ರ-ರಿ-ಬ್ಬರೂ
ವಿಶ್ವಾ-ವ-ಸು-ವೆಂಬ
ವಿಶ್ವಾ-ವ-ಸುವೇ
ವಿಶ್ವಾಸ
ವಿಶ್ವಾ-ಸ-ಗಳಿಂದ
ವಿಶ್ವೇ-ಶ್ವ-ರನು
ವಿಶ್ವೇ-ಶ್ವರೋ
ವಿಶ್ವೋ-ತ್ಪತ್ತಿ
ವಿಶ್ವೋ-ತ್ಪ-ತ್ತಿಗೆ
ವಿಷ
ವಿಷ-ಜಂ-ತು-ಗ-ಳಾ-ದವು
ವಿಷದ
ವಿಷ-ದಿಂದ
ವಿಷ-ದಿಂ-ದಲೊ
ವಿಷ-ಬಾ-ಧೆ-ಯನ್ನು
ವಿಷ-ಮ-ಸ್ಥಿ-ತಿ-ಯುಂ-ಟಾಗಿ
ವಿಷಯ
ವಿಷ-ಯ-ಕ್ಕಾಗಿ
ವಿಷ-ಯ-ಗಳ
ವಿಷ-ಯ-ಗಳನ್ನು
ವಿಷ-ಯ-ಗಳನ್ನೂ
ವಿಷ-ಯ-ಗಳಲ್ಲಿ
ವಿಷ-ಯ-ಗ-ಳ-ಲ್ಲೆಲ್ಲಾ
ವಿಷ-ಯ-ಗಳು
ವಿಷ-ಯ-ಗಳೂ
ವಿಷ-ಯ-ಗಳೇ
ವಿಷ-ಯ-ಗು-ಣ-ಗಳಲ್ಲಿ
ವಿಷ-ಯ-ದಲ್ಲಿ
ವಿಷ-ಯ-ವನ್ನು
ವಿಷ-ಯ-ವ-ಸ್ತು-ಗಳ
ವಿಷ-ಯ-ವಾಗಿ
ವಿಷ-ಯವೂ
ವಿಷ-ಯ-ವೇ-ನಿದೆ
ವಿಷ-ಯ-ವೇನೂ
ವಿಷ-ಯ-ಸುಖ
ವಿಷ-ಯ-ಸು-ಖಕ್ಕೆ
ವಿಷ-ಯ-ಸು-ಖ-ಗಳ
ವಿಷ-ಯ-ಸು-ಖ-ಗಳನ್ನು
ವಿಷ-ಯ-ಸು-ಖ-ಗ-ಳಿಗೆ
ವಿಷ-ಯ-ಸು-ಖದ
ವಿಷಯಾ
ವಿಷ-ಯಾ-ಭಿ-ಲಾ-ಷೆ-ಯನ್ನು
ವಿಷ-ಯಾ-ಸಕ್ತಿ
ವಿಷ-ಯಾ-ಸ-ಕ್ತಿಯೇ
ವಿಷ-ವನ್ನು
ವಿಷ-ವಿ-ಟ್ಟರು
ವಿಷ-ವಿ-ಟ್ಟ-ವರೂ
ವಿಷ-ವು-ಕ್ಕು-ವಂತೆ
ವಿಷವೂ
ವಿಷ-ವೈ-ದ್ಯನು
ವಿಷ್ಣು
ವಿಷ್ಣು-ಇ-ವರು
ವಿಷ್ಣು-ಚಕ್ರ
ವಿಷ್ಣು-ಚ-ಕ್ರವು
ವಿಷ್ಣು-ದೂ-ತರ
ವಿಷ್ಣು-ದೂ-ತ-ರಿಗೂ
ವಿಷ್ಣು-ದೂ-ತ-ರಿಗೆ
ವಿಷ್ಣು-ದೂ-ತರು
ವಿಷ್ಣು-ದ್ವೇ-ಷ-ದಲ್ಲಿ
ವಿಷ್ಣು-ಪದಂ
ವಿಷ್ಣು-ಪ-ದ-ವನ್ನು
ವಿಷ್ಣು-ಪಾ-ರ-ಮ್ಯ-ವನ್ನು
ವಿಷ್ಣು-ಪು-ರಾಣ
ವಿಷ್ಣು-ಪು-ರಾ-ಣ-ದಲ್ಲಿ
ವಿಷ್ಣು-ಮಯಂ
ವಿಷ್ಣು-ಮ-ಯ-ವಾ-ಯಿತು
ವಿಷ್ಣು-ಮಾ-ಯೆಯ
ವಿಷ್ಣು-ಮಾ-ಯೆ-ಯಂತೆ
ವಿಷ್ಣು-ಮಾ-ಯೆ-ಯನ್ನು
ವಿಷ್ಣು-ಮಾ-ಯೆ-ಯೆಂದು
ವಿಷ್ಣು-ಯ-ಶ-ನೆಂಬ
ವಿಷ್ಣು-ರ-ರೀಂ-ದ್ರ-ಪಾ-ಣಿಃ
ವಿಷ್ಣು-ರೂ-ಪ-ವನ್ನು
ವಿಷ್ಣು-ರೂ-ಪ-ವಾ-ಗಿಯೋ
ವಿಷ್ಣು-ಲೋ-ಕ-ವನ್ನು
ವಿಷ್ಣು-ವನ್ನು
ವಿಷ್ಣು-ವ-ಲ್ಲದ
ವಿಷ್ಣು-ವಿಗೆ
ವಿಷ್ಣು-ವಿನ
ವಿಷ್ಣು-ವಿ-ನಂತೆ
ವಿಷ್ಣು-ವಿ-ನಿಂದ
ವಿಷ್ಣುವು
ವಿಷ್ಣುವೇ
ವಿಷ್ಣು-ಸ-ನ್ನಿ-ಧಿ-ಯನ್ನು
ವಿಷ್ಣು-ಸಾ-ಯು-ಜ್ಯ-ವನ್ನು
ವಿಷ್ಣೋ-ರೂಪಂ
ವಿಷ್ವ-ಕ್ಸೇನ
ವಿಷ್ವ-ಕ್ಸೇನಃ
ವಿಷ್ವ-ಕ್ಸೇ-ನನು
ವಿಸರ್ಗ
ವಿಸ-ರ್ಜ-ನೆ-ಯಾಗಿ
ವಿಸೃಜ
ವಿಸೃ-ಜಸಿ
ವಿಸೃ-ಷ್ಟಾ-ಪತ್ಯ
ವಿಸ್ತ-ರಿಸಿ
ವಿಸ್ತ-ರಿ-ಸಿ-ದನು
ವಿಸ್ತ-ರಿ-ಸು-ವನು
ವಿಸ್ತಾರ
ವಿಸ್ತಾ-ರ-ವಾಗಿ
ವಿಸ್ತಾ-ರ-ವಾ-ಗಿದೆ
ವಿಸ್ತಾ-ರ-ವಾ-ಗಿಯೂ
ವಿಸ್ತಾ-ರ-ವಾ-ಗಿವೆ
ವಿಸ್ತಾ-ರ-ವಾದ
ವಿಸ್ತಾ-ರ-ವಾ-ಯಿತು
ವಿಸ್ತೀ-ರ್ಣ-ವಿ-ರುವ
ವಿಸ್ತೀ-ರ್ಣ-ವುಳ್ಳ
ವಿಸ್ಮೃತ್ಯ
ವಿಹಂಗಾ
ವಿಹ-ರಿಸ
ವಿಹ-ರಿ-ಸ-ಬೇ-ಕೆಂದು
ವಿಹ-ರಿಸಿ
ವಿಹ-ರಿ-ಸಿ-ದನು
ವಿಹ-ರಿ-ಸಿ-ದರು
ವಿಹ-ರಿಸು
ವಿಹ-ರಿ-ಸು-ತ್ತಲೆ
ವಿಹ-ರಿ-ಸುತ್ತಾ
ವಿಹ-ರಿ-ಸು-ತ್ತಿದ್ದ
ವಿಹ-ರಿ-ಸು-ತ್ತಿ-ದ್ದನು
ವಿಹ-ರಿ-ಸು-ತ್ತಿ-ದ್ದಾನೆ
ವಿಹ-ರಿ-ಸು-ತ್ತಿದ್ದು
ವಿಹ-ರಿ-ಸು-ತ್ತಿ-ರಲು
ವಿಹ-ರಿ-ಸು-ತ್ತಿ-ರು-ವಾಗ
ವಿಹ-ರಿ-ಸು-ತ್ತಿ-ರು-ವುದು
ವಿಹ-ರಿ-ಸು-ವನು
ವಿಹ-ರಿ-ಸು-ವಾಗ
ವಿಹ-ರಿ-ಸು-ವು-ದಕ್ಕೆ
ವಿಹ-ರಿ-ಸು-ವುದು
ವಿಹಾರ
ವಿಹಾ-ರಕ್ಕೆ
ವಿಹಾ-ರ-ಕ್ಕೆಂದು
ವಿಹಾ-ರ-ದಲ್ಲಿ
ವಿಹಾ-ರ-ಸ್ಥಾನ
ವಿಹಾ-ಹ-ವಾ-ದನು
ವಿಹಿ-ತ-ವಾ-ಗಿ-ರುವ
ವೀಕ್ಷ-ಣ-ವಿ-ಲ್ಲದೆ
ವೀಣೆ-ಯಲ್ಲಿ
ವೀಣೆ-ಯಿಂದ
ವೀತ-ನಿ-ದ್ರಾ-ನ-ಶೇಷೇ
ವೀರ
ವೀರನೆ
ವೀರ-ಭದ್ರ
ವೀರ-ಭ-ದ್ರನು
ವೀರ-ರ-ಸ-ವನ್ನು
ವೀರರು
ವೀರರೆ
ವೀರರೇ
ವೀರ-ವರ
ವೀರ-ಸ್ವರ್ಗ
ವೀರ-ಸ್ವ-ರ್ಗಕ್ಕೆ
ವೀರ್ಯ
ವೀರ್ಯ-ದಲ್ಲಿ
ವುದ-ಕ್ಕಾಗಿ
ವುದ-ಕ್ಕಾ-ಗು-ತ್ತ-ದೆಯೆ
ವುದಕ್ಕೆ
ವುದನ್ನು
ವುದ-ನ್ನೆಲ್ಲ
ವುದರ
ವುದ-ರಿಂದ
ವುದಿಲ್ಲ
ವುದು
ವುದೂ
ವುದೆಂ-ದರೆ
ವುದೆಂ-ಬು-ದನ್ನೂ
ವುದೆಲ್ಲ
ವುದೇ
ವುದೊಂ-ದು-ಮೂರು
ವುದೋ
ವುಳ್ಳ
ವೃಕ
ವೃಕ-ನಿಗೆ
ವೃಕಾ-ಸುರ
ವೃಕಾ-ಸು-ರ-ನನ್ನೂ
ವೃಕಾ-ಸು-ರ-ನಿಗೆ
ವೃಕಾ-ಸು-ರನು
ವೃಕ್ಷಕ್ಕೆ
ವೃಕ್ಷ-ಗಳ
ವೃಕ್ಷ-ಗ-ಳ-ಲ್ಲಿಟ್ಟು
ವೃಕ್ಷ-ಗಳು
ವೃಕ್ಷ-ಗ-ಳೆಲ್ಲ
ವೃಕ್ಷ-ಗಳೇ
ವೃಕ್ಷ-ದಲ್ಲಿ
ವೃತ್ತಾಂತ
ವೃತ್ತಾಂ-ತ-ವನ್ನು
ವೃತ್ತಾಂ-ತ-ವನ್ನೂ
ವೃತ್ತಾಂ-ತ-ವ-ನ್ನೆಲ್ಲ
ವೃತ್ತಿ
ವೃತ್ತಿ-ಯನ್ನು
ವೃತ್ತಿ-ಯಾಗಿ
ವೃತ್ತಿ-ಯಿಂದ
ವೃತ್ರ
ವೃತ್ರನ
ವೃತ್ರ-ನಿಂದ
ವೃತ್ರನು
ವೃತ್ರನೇ
ವೃತ್ರಾ
ವೃತ್ರಾ-ಸುರ
ವೃತ್ರಾ-ಸು-ರನ
ವೃತ್ರಾ-ಸು-ರ-ನನ್ನು
ವೃತ್ರಾ-ಸು-ರ-ನಲ್ಲಿ
ವೃತ್ರಾ-ಸು-ರ-ನಾಗಿ
ವೃತ್ರಾ-ಸು-ರನು
ವೃದ್ಧ-ಕ-ರ್ಮ-ದಂ-ತ-ವಕ್ತ್ರ
ವೃದ್ಧ-ಪಿ-ತಾ-ಮ-ಹ-ನಿಗೆ
ವೃದ್ಧ-ರ-ನ್ನೆಲ್ಲ
ವೃದ್ಧಿ-ಕ್ಷ-ಯ-ಗಳು
ವೃದ್ಧಿ-ಯಾ-ಗು-ತ್ತದೆ
ವೃಷ
ವೃಷ-ದ್ಭಾನು
ವೃಷ-ಪ-ರ್ವ-ನಿಗೆ
ವೃಷ-ಪ-ರ್ವನು
ವೃಷ-ಪ-ರ್ವ-ನೆಂಬ
ವೃಷಭ
ವೃಷ-ಭ-ರೂ-ಪಿನ
ವೃಷ-ಭಾ-ರೂ-ಢ-ನಾಗಿ
ವೆಂದರೆ
ವೆಂದ-ರೇನು
ವೆಂದು
ವೆಂಬ
ವೆಂಬಷ್ಟು
ವೆಂಬು-ದನ್ನೂ
ವೆಚ್ಚ
ವೆಚ್ಚ-ಮಾ-ಡಿ-ದ್ದನು
ವೆನಿ-ಸಿ-ಕೊಂಡ
ವೆನು
ವೆನೆಂದು
ವೆನ್ನು-ವಳೆ
ವೆಯಲ್ಲ
ವೆಯಾ
ವೆಲ್ಲವೂ
ವೇಂ
ವೇಕಿ-ಗಳು
ವೇಗಕ್ಕೆ
ವೇಗ-ದಿಂದ
ವೇಗ-ವಂತ
ವೇಗ-ವನ್ನು
ವೇಗ-ವಾಗಿ
ವೇಣು-ಗಾ-ನ-ದಿಂದ
ವೇಣು-ಗಾ-ನ-ವನ್ನು
ವೇಣು-ಧರ
ವೇದ
ವೇದ-ಕ-ರ್ಮ-ಗಳಲ್ಲಿ
ವೇದ-ಕ್ಕಿಂತ
ವೇದಕ್ಕೆ
ವೇದ-ಗಳ
ವೇದ-ಗ-ಳಂತೆ
ವೇದ-ಗ-ಳ-ಗಿಂತ
ವೇದ-ಗಳನ್ನು
ವೇದ-ಗ-ಳಿಗೂ
ವೇದ-ಗ-ಳಿಗೆ
ವೇದ-ಗಳು
ವೇದ-ಗಳೂ
ವೇದ-ಗ-ಳೆ-ಲ್ಲವೂ
ವೇದದ
ವೇದ-ದಲ್ಲಿ
ವೇದ-ದಿಂ-ದಲೇ
ವೇದ-ಧ-ರ್ಮ-ವನ್ನು
ವೇದ-ನಿ-ಷ್ಠ-ರಾದ
ವೇದ-ನೆ-ಯಿಂದ
ವೇದ-ಪಾ-ಠ-ಕರು
ವೇದ-ಪು-ರು-ಷನ
ವೇದ-ಪು-ರು-ಷ-ನಾದ
ವೇದ-ಪ್ರ-ತಿ-ಪಾ-ದ್ಯನೂ
ವೇದ-ಮಂ-ತ್ರ-ಗಳನ್ನು
ವೇದ-ಮ-ಯರು
ವೇದ-ಮ-ಯ-ವಾದ
ವೇದ-ಮಾತೆ
ವೇದ-ವನ್ನು
ವೇದ-ವಾಕ್ಯ
ವೇದ-ವಿ-ದ-ರಾದ
ವೇದ-ವಿ-ಹಿ-ತ-ವಾ-ಗಿ-ದ್ದರೂ
ವೇದ-ವಿ-ಹಿ-ತ-ವಾದ
ವೇದ-ವಿ-ಹಿ-ತ-ವಾ-ದುದು
ವೇದ-ವೆಂದು
ವೇದ-ವೆಂ-ದು-ಮ-ಹಾ-ಭಾ-ರ-ತ-ದಂತೆ
ವೇದ-ವೆಂಬ
ವೇದವೇ
ವೇದ-ವೇ-ದಾಂ-ತ-ಗಳ
ವೇದ-ವೇ-ದಾಂ-ತ-ಗಳನ್ನು
ವೇದ-ವೇ-ದಾಂ-ತ-ಗಳಲ್ಲಿ
ವೇದ-ವೇ-ದಾಂ-ತ-ಗ-ಳೊ-ಡನೆ
ವೇದ-ವ್ಯಾ-ಸರ
ವೇದ-ಶಾ-ಸ್ತ್ರ-ಗಳ
ವೇದ-ಸ-ಮಾ-ನ-ವಾದ
ವೇದ-ಸಾ-ರ-ವನ್ನು
ವೇದ-ಸ್ಮೃತಿ
ವೇದ-ಸ್ವ-ರೂಪ
ವೇದ-ಸ್ವ-ರೂ-ಪ-ನಾದ
ವೇದ-ಸ್ವ-ರೂ-ಪ-ವೆಂದು
ವೇದ-ಸ್ವ-ರೂ-ಪಿ-ಯಾದ
ವೇದಾ
ವೇದಾಂತ
ವೇದಾಂ-ತ-ಗಳ
ವೇದಾಂ-ತದ
ವೇದಾಂ-ತ-ರ್ಗ-ತ-ವಾದ
ವೇದಾಂ-ತ-ಸಾ-ರ-ವೆಲ್ಲ
ವೇದಾ-ಧ್ಯ-ಯನ
ವೇದಾ-ಧ್ಯ-ಯ-ನ-ಸಂ-ಪ-ನ್ನ-ನಾಗ
ವೇದಾ-ರ್ಥ-ವನ್ನು
ವೇದೋಕ್ತ
ವೇದೋ-ಕ್ತ-ವಾದ
ವೇದೋ-ಕ್ತ-ವಿ-ಧಿಯ
ವೇದೋಸಿ
ವೇದ್ಮ್ಯಹಂ
ವೇದ್ಯ
ವೇದ್ಯ-ವಾ-ಗು-ತ್ತದೆ
ವೇನ
ವೇನನ
ವೇನ-ನಿಗೆ
ವೇನನು
ವೇನಲ್ಲ
ವೇನು
ವೇನೂ
ವೇನೆಂದು
ವೇನೋ
ವೇಬರ್ನ
ವೇರದು
ವೇರಿ-ಸಿ-ದಳು
ವೇಳೆ
ವೇಳೆ-ಗಾ-ಗಲೆ
ವೇಳೆಗೆ
ವೇಳೆ-ಯಲ್ಲಿ
ವೇಶ-ವನ್ನು
ವೇಶಾಂ-ತ-ರ-ದಿಂದ
ವೇಶ್ಯೆ
ವೇಶ್ಯೆ-ಯರ
ವೇಶ್ಯೆ-ಯರು
ವೇಶ್ಯೆ-ಯೊ-ಬ್ಬ-ಳಿ-ದ್ದಳು
ವೇಷ
ವೇಷ-ಗಳು
ವೇಷ-ದ-ಲ್ಲಿದ್ದ
ವೇಷ-ದಿಂದ
ವೇಷ-ಧಾರಿ
ವೇಷ-ಮಾ-ತ್ರ-ದಿಂ-ದಲೇ
ವೇಷ-ವನ್ನು
ವೇಷ-ವನ್ನೂ
ವೇಷ-ವ-ನ್ನೆಲ್ಲ
ವೇಷ-ಹಾಕಿ
ವೇಷ-ಹಾ-ಕಿ-ಕೊಂಡು
ವೈ
ವೈಕುಂಠ
ವೈಕುಂ-ಠಕ್ಕೆ
ವೈಕುಂ-ಠ-ಚ್ಯು-ತ-ರಾ-ಗುವ
ವೈಕುಂ-ಠ-ದ-ರ್ಶನ
ವೈಕುಂ-ಠ-ದಲ್ಲಿ
ವೈಕುಂ-ಠ-ವನ್ನು
ವೈಕೃ-ತ-ಸೃ-ಷ್ಟಿ-ಯಲ್ಲಿ
ವೈಕೃ-ತ-ಸೃ-ಷ್ಟಿ-ಯ-ಲ್ಲಿಯೇ
ವೈಖರಿ
ವೈಜ-ಯಂ-ತಿ-ಮಾ-ಲೆ-ಯನ್ನೂ
ವೈಜ್ಞಾ-ನಿಕ
ವೈಡೂರ
ವೈಡೂರ್ಯ
ವೈದಿಕ
ವೈದ್ಯ
ವೈನಾ-ಯಕ
ವೈನಾ-ಯ-ಕ-ಯ-ಕ್ಷ-ರಕ್ಷೋ
ವೈಭವ
ವೈಭ-ವ-ಗಳನ್ನು
ವೈಭ-ವ-ಗ-ಳಿಗೆ
ವೈಭ-ವ-ಗಳು
ವೈಭ-ವ-ದಿಂದ
ವೈಭ-ವ-ದಿಂ-ದಿ-ರು-ತ್ತಿ-ದ್ದರು
ವೈಭ-ವ-ವನ್ನು
ವೈಭ-ವವೆ
ವೈಭ-ವವೇ
ವೈಭ-ವ-ಶಿ-ಖ-ರ-ದ-ಮೇಲೆ
ವೈಯ್ಯಾ-ರ-ದಿಂದ
ವೈರ
ವೈರ-ವೆಂಬ
ವೈರ-ವೆಂ-ಬುದೆ
ವೈರಾಗ್ಯ
ವೈರಾ-ಗ್ಯಕ್ಕೆ
ವೈರಾ-ಗ್ಯ-ಗಳ
ವೈರಾ-ಗ್ಯ-ಗಳನ್ನು
ವೈರಾ-ಗ್ಯ-ಗಳು
ವೈರಾ-ಗ್ಯ-ಗಳೂ
ವೈರಾ-ಗ್ಯ-ದತ್ತ
ವೈರಾ-ಗ್ಯ-ದಿಂದ
ವೈರಾ-ಗ್ಯ-ಪರ
ವೈರಾ-ಗ್ಯ-ಪ-ರ-ನಾಗಿ
ವೈರಾ-ಗ್ಯ-ಪ-ರ-ನಾ-ಗಿದ್ದ
ವೈರಾ-ಗ್ಯ-ಪ-ರ-ನಾ-ದನು
ವೈರಾ-ಗ್ಯ-ಪ-ರ-ರಾ-ಗಿ-ದ್ದ-ರಷ್ಟೆ
ವೈರಾ-ಗ್ಯ-ಮೂ-ರ್ತಿ-ಯಾದ
ವೈರಾ-ಗ್ಯ-ವನ್ನು
ವೈರಾ-ಗ್ಯ-ವಾ-ಗಲಿ
ವೈರಾ-ಗ್ಯ-ವೆಂಬ
ವೈರಾ-ಜನ
ವೈರಿ
ವೈರಿ-ಗಳ
ವೈರಿ-ಗ-ಳಂತೆ
ವೈರಿ-ಯಾದ
ವೈವ-ಸ್ವತ
ವೈವ-ಸ್ವ-ತನ
ವೈವ-ಸ್ವ-ತ-ನಿಗೆ
ವೈವ-ಸ್ವ-ತ-ಮನು
ವೈವ-ಸ್ವ-ತ-ಮ-ನು-ವಿನ
ವೈಶ್ಯ
ವೈಶ್ಯನೂ
ವೈಶ್ಯರು
ವೈಶ್ಯರೇ
ವೈಷ್ಣವ
ವೈಷ್ಣವಿ
ವೈಹಾ-ಯ-ಸ-ವೆಂಬ
ವೊಂದನ್ನು
ವೊಂದು
ವೋಢುಂ
ವೌಷಟ್
ವ್ಯಕ್ತ
ವ್ಯಕ್ತ-ಪ-ಡಿ-ಸಿ-ದನೋ
ವ್ಯಕ್ತ-ವಾಗಿ
ವ್ಯಕ್ತ-ವಾ-ಗಿವೆ
ವ್ಯಕ್ತ-ವಾ-ಗು-ತ್ತದೆ
ವ್ಯಕ್ತ-ವಾ-ದಾಗ
ವ್ಯಕ್ತಿ
ವ್ಯಕ್ತಿ-ಗಳ
ವ್ಯಕ್ತಿಗೆ
ವ್ಯಕ್ತಿತ್ವ
ವ್ಯಕ್ತಿ-ತ್ವ-ದಲ್ಲಿ
ವ್ಯಕ್ತಿ-ತ್ವ-ವನ್ನು
ವ್ಯಕ್ತಿ-ತ್ವ-ವಿ-ಲ್ಲ-ದ-ವ-ನಂತೆ
ವ್ಯಕ್ತಿ-ತ್ವವು
ವ್ಯಕ್ತಿಯ
ವ್ಯಕ್ತಿ-ಯನ್ನೆ
ವ್ಯಕ್ತಿ-ಯಾ-ಗಿ-ರ-ಬೇಕು
ವ್ಯಕ್ತಿ-ಯಿಂ-ದಲೋ
ವ್ಯಕ್ತಿಯೇ
ವ್ಯಕ್ತಿ-ಯೊ-ಬ್ಬನು
ವ್ಯತ್ಯಾಸ
ವ್ಯತ್ಯಾ-ಸಕ್ಕೆ
ವ್ಯತ್ಯಾ-ಸ-ದಿಂದ
ವ್ಯತ್ಯಾ-ಸ-ದಿಂ-ದಲೂ
ವ್ಯತ್ಯಾ-ಸ-ವನ್ನು
ವ್ಯತ್ಯಾ-ಸ-ವಿಲ್ಲ
ವ್ಯತ್ಯಾ-ಸ-ವಿ-ಲ್ಲ-ದಂ-ತಾ-ಗು-ತ್ತದೆ
ವ್ಯತ್ಯಾ-ಸ-ವಿ-ಷ್ಟೆ-ಪ್ರತಿ
ವ್ಯತ್ಯಾ-ಸವೂ
ವ್ಯಥೆ-ಗೊ-ಳ-ಗಾ-ದ-ವನೆ
ವ್ಯಥೆ-ಪ-ಡ-ಲಿಲ್ಲ
ವ್ಯಥೆ-ಪ-ಡು-ವು-ದಿಲ್ಲ
ವ್ಯಥೆ-ಯನ್ನು
ವ್ಯಥೆ-ಯಾ-ಯಿ-ತಾ-ದರೂ
ವ್ಯಥೆ-ಯಾ-ಯಿತು
ವ್ಯರ್ಥ
ವ್ಯರ್ಥ-ಗೊ-ಳಿ-ಸಿತು
ವ್ಯರ್ಥ-ವಾ-ಗ-ದಂತೆ
ವ್ಯರ್ಥ-ವಾ-ಗ-ಲಿಲ್ಲ
ವ್ಯರ್ಥ-ವಾಗಿ
ವ್ಯರ್ಥ-ವಾ-ದವು
ವ್ಯರ್ಥ-ವಾದು
ವ್ಯರ್ಥ-ವಾ-ಯಿತು
ವ್ಯರ್ಥ-ವೆಂದು
ವ್ಯರ್ಥ-ವೆಂ-ಬುದೂ
ವ್ಯರ್ಥ-ವೆ-ನಿಸು
ವ್ಯವ-ಹ-ರಿ-ಸ-ಬೇ-ಕೆಂಬ
ವ್ಯವ-ಹ-ರಿ-ಸುತ್ತಾ
ವ್ಯವ-ಹ-ರಿ-ಸು-ತ್ತಾನೆ
ವ್ಯವ-ಹ-ರಿ-ಸು-ತ್ತಿ-ದ್ದನು
ವ್ಯವ-ಹ-ರಿ-ಸು-ತ್ತಿ-ರು-ವ-ನಲ್ಲಾ
ವ್ಯವ-ಹ-ರಿ-ಸು-ತ್ತಿ-ರು-ವು-ದ-ರಿಂದ
ವ್ಯವ-ಹ-ರಿ-ಸುವ
ವ್ಯವ-ಹ-ರಿ-ಸು-ವುದು
ವ್ಯವ-ಹಾ-ರ-ವನ್ನೂ
ವ್ಯಸೃ-ಜ-ದ-ಕೃತ
ವ್ಯಹ-ರಿ-ಸ-ಬೇ-ಕಷ್ಟೆ
ವ್ಯಾಕ-ರ್ತ-ವ್ಯ-ಗಳನ್ನು
ವ್ಯಾಖ್ಯಾ-ನ-ವನ್ನು
ವ್ಯಾಜ್ಯಾಂ-ತ-ರ-ವಾಗಿ
ವ್ಯಾಪಾರ
ವ್ಯಾಪಾ-ರಕ್ಕೆ
ವ್ಯಾಪಾ-ರ-ಗಳೆ
ವ್ಯಾಪಾ-ರದ
ವ್ಯಾಪಾ-ರ-ದಲ್ಲಿ
ವ್ಯಾಪಾ-ರ-ವನ್ನು
ವ್ಯಾಪಾ-ರ-ವಲ್ಲ
ವ್ಯಾಪಾ-ರ-ವಿಲ್ಲ
ವ್ಯಾಪಾ-ರ-ವೆಲ್ಲ
ವ್ಯಾಪಾ-ರ-ವೆ-ಲ್ಲವೂ
ವ್ಯಾಪಿಸಿ
ವ್ಯಾಪಿ-ಸಿತು
ವ್ಯಾಪಿ-ಸಿತ್ತು
ವ್ಯಾಪಿ-ಸಿದೆ
ವ್ಯಾಪಿ-ಸಿ-ದ್ದರೂ
ವ್ಯಾಪಿ-ಸಿ-ದ್ದಾನೆ
ವ್ಯಾಪಿ-ಸಿ-ದ್ದೇನೆ
ವ್ಯಾಪಿ-ಸಿ-ರುವ
ವ್ಯಾಪಿ-ಸಿ-ರು-ವನು
ವ್ಯಾಪಿ-ಸು-ತ್ತ-ದೆ-ಯ-ಲ್ಲವೆ
ವ್ಯಾಪಿ-ಸುತ್ತಾ
ವ್ಯಾಪಿ-ಸುವ
ವ್ಯಾಪಿ-ಸು-ವಂ-ತಿ-ದ್ದರೆ
ವ್ಯಾಪಿ-ಸು-ವು-ದಿಲ್ಲ
ವ್ಯಾಸ
ವ್ಯಾಸ-ಕೃತ
ವ್ಯಾಸ-ಕೃ-ತ-ವ-ಲ್ಲ-ವೆಂದು
ವ್ಯಾಸ-ಕೃ-ತ-ವೆಂದು
ವ್ಯಾಸನು
ವ್ಯಾಸ-ಪುತ್ರ
ವ್ಯಾಸ-ಮ-ಹರ್ಷಿ
ವ್ಯಾಸ-ಮ-ಹ-ರ್ಷಿಯು
ವ್ಯಾಸ-ಮುನಿ
ವ್ಯಾಸ-ಮು-ನಿ-ಯನ್ನು
ವ್ಯಾಸ-ಮು-ನಿಯು
ವ್ಯಾಸರ
ವ್ಯಾಸ-ರನ್ನು
ವ್ಯಾಸ-ರಲ್ಲ
ವ್ಯಾಸ-ರಿಂದ
ವ್ಯಾಸ-ರಿಗೆ
ವ್ಯಾಸರು
ವ್ಯಾಸ-ರೂ-ಪ-ದಿಂದ
ವ್ಯಾಸ-ರೇನು
ವ್ಯಾಸ್-ಅ-ನೆಂಬ
ವ್ಯೂಹ-ಗಳಿಂದ
ವ್ಯೋಮ-ನೆಂಬ
ವ್ಯೋಮ-ವ-ತ್ತ-ನ್ನ-ತೋ-ಽಸ್ಮ್ಯ-ಹಮ್
ವ್ರತ
ವ್ರತದ
ವ್ರತ-ದಲ್ಲಿ
ವ್ರತ-ವನ್ನು
ವ್ರತ-ವಾಗಿ
ವ್ರತ್ಯದ
ಶಂಕರ
ಶಂಕ-ರನ
ಶಂಕ-ರ-ನನ್ನು
ಶಂಕ-ರನು
ಶಂಕ-ರನೂ
ಶಂಕಿ-ಸ-ಬೇಡ
ಶಂಕಿ-ಸಿ-ದಿ-ರ-ಲ್ಲವೆ
ಶಂಕಿ-ಸುತ್ತಾ
ಶಂಕು
ಶಂಕೆ
ಶಂಕೆ-ಯಿಂದ
ಶಂಖ
ಶಂಖ-ಚ-ಕ್ರ-ಧಾ-ರಿ-ಯಾ-ಗಿ-ರುವ
ಶಂಖ-ಚ-ಕ್ರಾದಿ
ಶಂಖ-ಚೂಡ
ಶಂಖ-ಚೂ-ಡನು
ಶಂಖ-ಚೂ-ಡ-ನೆಂಬ
ಶಂಖದ
ಶಂಖ-ಮಾತ್ರ
ಶಂಖ-ವನ್ನು
ಶಂಖವೇ
ಶಂಖು
ಶಂಬರ
ಶಂಬ-ರ-ನನ್ನು
ಶಂಬ-ರನು
ಶಂಬ-ರ-ನೆಂಬ
ಶಂಬ-ರಾ-ಸು-ರ-ನನ್ನು
ಶಂಬ-ಲ-ವೆಂಬ
ಶಂಬಾ-ಸು-ರ-ನಿಗೆ
ಶಕ
ಶಕಟ
ಶಕ-ಟಾ-ಸುರ
ಶಕುಂ-ತಲೆ
ಶಕುಂ-ತ-ಲೆಯ
ಶಕುಂ-ತ-ಳೆಯೂ
ಶಕುನಿ
ಶಕು-ನಿಯ
ಶಕ್ತ
ಶಕ್ತ-ನಾದ
ಶಕ್ತಯೇ
ಶಕ್ತ-ವಾ-ಗ-ಕೂ-ಡದು
ಶಕ್ತ-ವಾ-ಗ-ಲಿಲ್ಲ
ಶಕ್ತ-ವಾ-ಗಿಲ್ಲ
ಶಕ್ತ-ವಾ-ಗುವು
ಶಕ್ತಿ
ಶಕ್ತಿ-ಗಳನ್ನು
ಶಕ್ತಿ-ಗಳನ್ನೂ
ಶಕ್ತಿ-ಗಳಿಂದ
ಶಕ್ತಿ-ಗ-ಳುಳ್ಳ
ಶಕ್ತಿಗೂ
ಶಕ್ತಿಗೆ
ಶಕ್ತಿ-ಪರೀಕ್ಷೆ--ಯನ್ನು
ಶಕ್ತಿಯ
ಶಕ್ತಿ-ಯನ್ನು
ಶಕ್ತಿ-ಯನ್ನೂ
ಶಕ್ತಿ-ಯನ್ನೆ
ಶಕ್ತಿ-ಯ-ನ್ನೆಲ್ಲ
ಶಕ್ತಿ-ಯನ್ನೇ
ಶಕ್ತಿ-ಯಲ್ಲಿ
ಶಕ್ತಿ-ಯಿಂದ
ಶಕ್ತಿ-ಯಿತ್ತು
ಶಕ್ತಿ-ಯಿ-ದ್ದರೂ
ಶಕ್ತಿ-ಯಿ-ದ್ದರೆ
ಶಕ್ತಿ-ಯಿ-ದ್ದಲ್ಲಿ
ಶಕ್ತಿ-ಯಿ-ರು-ತ್ತದೆ
ಶಕ್ತಿ-ಯಿ-ಲ್ಲ-ದಿ-ದ್ದರೆ
ಶಕ್ತಿ-ಯುಳ್ಳ
ಶಕ್ತಿ-ಯು-ಳ್ಳ-ವರು
ಶಕ್ತಿಯೆ
ಶಕ್ತಿ-ಯೆ-ಲ್ಲವೂ
ಶಕ್ತಿ-ಯೆ-ಲ್ಲಿ-ಯದು
ಶಕ್ತಿಯೇ
ಶಕ್ತಿ-ಯೊ-ಡ-ಗೂಡಿ
ಶಕ್ತಿ-ರೂಪ
ಶಕ್ತಿ-ವಂ-ತ-ರಿ-ಲ್ಲ-ವೆಂದು
ಶಕ್ತೀ
ಶಕ್ತ್ಯಾ-ಯು-ಧ-ವನ್ನು
ಶಚಿ-ಯನ್ನು
ಶಚಿ-ಯೊ-ಡನೆ
ಶಚೀ-ದೇ-ವಿಯ
ಶತ-ಕೋಟಿ
ಶತ-ಚಂದ್ರ
ಶತ-ಚಿತ್ತು
ಶತ-ಧನ್ವ
ಶತ-ಧ-ನ್ವನ
ಶತ-ಧ-ನ್ವ-ನನ್ನು
ಶತ-ಧ-ನ್ವ-ನಿಗೆ
ಶತ-ಧ-ನ್ವನು
ಶತ-ಧ್ರು-ತಿ-ಯನ್ನು
ಶತ-ಧ್ರು-ತಿ-ಯಲ್ಲಿ
ಶತ-ಬಿಂ-ದು-ವಿನ
ಶತ-ಮಾ-ನ-ಕ್ಕಿಂತ
ಶತ-ಮಾ-ನಕ್ಕೂ
ಶತ-ಮಾ-ನಕ್ಕೆ
ಶತ-ಮಾ-ನ-ಗಳ
ಶತ-ಮಾ-ನದ
ಶತ-ಮಾ-ನ-ದಲ್ಲಿ
ಶತ-ರೂಪೆ
ಶತ-ರೂ-ಪೆ-ಯಲ್ಲಿ
ಶತ-ರೂ-ಪೆ-ಯೆಂಬ
ಶತ-ರೂ-ಪೆ-ಯೊ-ಡನೆ
ಶತ-ಸ್ಸಿದ್ಧ
ಶತ್ರು
ಶತ್ರು-ಗಳ
ಶತ್ರು-ಗಳನ್ನು
ಶತ್ರು-ಗಳನ್ನೆಲ್ಲ
ಶತ್ರು-ಗಳನ್ನೆಲ್ಲಾ
ಶತ್ರು-ಗಳಲ್ಲಿ
ಶತ್ರು-ಗ-ಳಾ-ದರೂ
ಶತ್ರು-ಗ-ಳಿಗೆ
ಶತ್ರು-ಗಳು
ಶತ್ರು-ಘ್ನ-ನಿಗೆ
ಶತ್ರು-ಘ್ನ-ರೆಂಬ
ಶತ್ರು-ಭ-ಯ-ವಿಲ್ಲ
ಶತ್ರು-ವನ್ನು
ಶತ್ರು-ವಲ್ಲ
ಶತ್ರು-ವಾದ
ಶತ್ರು-ವಿಗೆ
ಶತ್ರು-ವಿನ
ಶತ್ರು-ವಿ-ನೊ-ಡನೆ
ಶತ್ರು-ವೆಂದು
ಶತ್ರು-ವೆಂಬ
ಶತ್ರು-ಸೇನೆ
ಶತ್ರು-ಸೇ-ನೆಯ
ಶತ್ರು-ಸೇ-ನೆ-ಯನ್ನು
ಶನಿ-ಗ್ರಹ
ಶಪಿ-ಸ-ದಿ-ದ್ದರೆ
ಶಪಿಸಿ
ಶಪಿ-ಸಿದ
ಶಪಿ-ಸಿ-ದ-ನಲ್ಲಾ
ಶಪಿ-ಸಿ-ದನು
ಶಪಿ-ಸಿ-ದ-ನು-ಇದು
ಶಪಿ-ಸಿ-ದರು
ಶಪಿ-ಸಿ-ದಳು
ಶಪಿ-ಸಿ-ದು-ದ-ಕ್ಕಾಗಿ
ಶಬ್ದ
ಶಬ್ದ-ಬ್ರ-ಹ್ಮದ
ಶಬ್ದ-ಮಾಡಿ
ಶಬ್ದ-ಮಾ-ಡು-ತ್ತಿ-ದ್ದವು
ಶಬ್ದ-ವನ್ನು
ಶಬ್ದ-ವಾ-ಗು-ತ್ತಿ-ರಲು
ಶಬ್ದ-ವಾ-ಯಿತು
ಶಬ್ದವು
ಶಬ್ದ-ವೆಂಬ
ಶಬ್ದವೇ
ಶಬ್ದಾದಿ
ಶಬ್ದಾ-ದಿ-ತ-ನ್ಮಾ-ತ್ರ-ಗ-ಳಿಗೂ
ಶಮ
ಶಮಂ-ತ-ಪಂ-ಚ-ಕ-ವೆಂಬ
ಶಮ-ದ-ಮಾ-ದಿ-ಗಳಿಂದ
ಶಮೀಕ
ಶಮೀ-ಕ-ಪು-ಷಿಗೆ
ಶಯ-ದಿಂದ
ಶರಣಂ
ಶರ-ಣಾ-ಗತ
ಶರ-ಣಾ-ಗ-ತ-ನಾಗ
ಶರ-ಣಾ-ಗ-ತ-ನಾಗಿ
ಶರ-ಣಾ-ಗ-ತ-ನಾ-ಗಿ-ದ್ದೇನೆ
ಶರ-ಣಾ-ಗ-ತ-ನಾದ
ಶರ-ಣಾ-ಗ-ತ-ನಾ-ದನು
ಶರ-ಣಾ-ಗ-ತ-ಭೀ-ತ-ಹರಂ
ಶರ-ಣಾ-ಗ-ತ-ರ-ಕ್ಷ-ಕ-ರಾದ
ಶರ-ಣಾ-ಗ-ತ-ರಾ-ಗ-ಬೇಕು
ಶರ-ಣಾ-ಗ-ತ-ರಾಗಿ
ಶರ-ಣಾ-ಗ-ತ-ರಾ-ಗಿ-ರುವ
ಶರ-ಣಾ-ಗ-ತ-ರಾದ
ಶರ-ಣಾ-ಗ-ತ-ಳಾ-ಗಿ-ರುವ
ಶರ-ಣಾ-ಗಲು
ಶರ-ಣಾಗಿ
ಶರ-ಣಾ-ಗಿ-ದ್ದೇನೆ
ಶರ-ಣಾ-ಗಿ-ದ್ದೇ-ನೆ-ಎಂಬ
ಶರ-ಣಾ-ಗು-ತ್ತೇನೆ
ಶರ-ಣಾ-ದನು
ಶರ-ಣಾ-ದ-ವ-ರನ್ನು
ಶರ-ಣಾ-ದ-ವ-ರಿಗೆ
ಶರ-ಣಾ-ದ-ವ-ಳಂತೆ
ಶರ-ಣಾ-ಯಿತು
ಶರಣು
ಶರ-ಣು-ಹೊಂದಿ
ಶರ-ಣು-ಹೊ-ಕ್ಕಿ-ದ್ದೇನೆ
ಶರ-ಣು-ಹೋಗಿ
ಶರ-ಣು-ಹೋಗು
ಶರ-ತ್ಕಾ-ಲದ
ಶರ-ದಂ-ಬ-ರ-ತು-ಲ್ಯ-ತ-ನುಂ
ಶರ-ದೃ-ತು-ವಿನ
ಶರ-ಪಂ-ಜ-ರ-ವನ್ನು
ಶರ-ಶ-ಯ್ಯೆ-ಯಲ್ಲಿ
ಶರೀರ
ಶರೀ-ರಕ್ಕೆ
ಶರೀ-ರ-ಗಳ
ಶರೀ-ರ-ಗಳನ್ನು
ಶರೀ-ರ-ಗಳಲ್ಲಿ
ಶರೀ-ರ-ಗಳು
ಶರೀ-ರದ
ಶರೀ-ರ-ದಲ್ಲಿ
ಶರೀ-ರ-ದಿಂದ
ಶರೀ-ರ-ದೊ-ಡನೆ
ಶರೀ-ರ-ದೊ-ಡ-ನೆಯೆ
ಶರೀ-ರ-ನಾ-ಗಿಯೂ
ಶರೀ-ರ-ವ-ನ್ನಾಗಿ
ಶರೀ-ರ-ವನ್ನು
ಶರೀ-ರ-ವನ್ನೆ
ಶರೀ-ರವು
ಶರೀ-ರ-ವುಳ್ಳ
ಶರೀ-ರ-ವೆಂದೂ
ಶರೀ-ರವೇ
ಶರೀ-ರ-ಸಂ-ಬಂಧ
ಶರೀ-ರ-ಸಂ-ಬಂ-ಧ-ವನ್ನು
ಶರ್ಕರ
ಶರ್ಮಿಷ್ಠೆ
ಶರ್ಮಿ-ಷ್ಠೆಗೂ
ಶರ್ಮಿ-ಷ್ಠೆಗೆ
ಶರ್ಮಿ-ಷ್ಠೆಯ
ಶರ್ಮಿ-ಷ್ಠೆ-ಯನ್ನೂ
ಶರ್ಮಿ-ಷ್ಠೆಯೂ
ಶರ್ಯಾತಿ
ಶರ್ಯಾ-ತಿಗೆ
ಶರ್ಯಾ-ತಿಯು
ಶರ್ಯಾ-ತಿ-ರಾ-ಜನ
ಶಲ
ಶಲ್ಯ
ಶಶ
ಶಶಾದ
ಶಸ್ತ್ರ-ಗಳ
ಶಸ್ತ್ರ-ಗ-ಳಿಂ-ದಲೂ
ಶಸ್ತ್ರಾ-ಸ್ತ್ರ-ಗಳ
ಶಾಂತ
ಶಾಂತಂ
ಶಾಂತ-ಗೊಂ-ಡಿತು
ಶಾಂತ-ಗೊ-ಳಿ-ಸಿ-ದರೆ
ಶಾಂತ-ಚಿತ್ತ
ಶಾಂತ-ಚಿ-ತ್ತ-ನಾ-ಗಿ-ರು-ತ್ತಿ-ದ್ದನು
ಶಾಂತ-ಚಿ-ತ್ತ-ನಾಗು
ಶಾಂತ-ಚಿ-ತ್ತ-ನಾ-ದನು
ಶಾಂತನಾ
ಶಾಂತ-ನಾಗಿ
ಶಾಂತ-ನಾ-ಗಿರು
ಶಾಂತ-ನಾಗು
ಶಾಂತ-ನಾದ
ಶಾಂತ-ನಾ-ದು-ದಷ್ಟೇ
ಶಾಂತನೂ
ಶಾಂತ-ಮ-ನ-ಸ್ಸಿ-ನಿಂದ
ಶಾಂತ-ಮೂರ್ತಿ
ಶಾಂತ-ಮೂ-ರ್ತಿ-ಯಾದ
ಶಾಂತ-ವಾಗ
ಶಾಂತ-ವಾ-ಗ-ಲಿಲ್ಲ
ಶಾಂತ-ವಾಗಿ
ಶಾಂತ-ವಾ-ಯಿತು
ಶಾಂತಾಯ
ಶಾಂತಿ
ಶಾಂತಿ-ಗಳು
ಶಾಂತಿಗೆ
ಶಾಂತಿ-ಗೊಂ-ಡಿಲ್ಲ
ಶಾಂತಿ-ದೇ-ವಾ-ಪ್ರ-ಶ್ರಮ
ಶಾಂತಿ-ದೇ-ವಿ-ಯನ್ನೂ
ಶಾಂತಿ-ಪ-ರ್ವ-ದಲ್ಲಿ
ಶಾಂತಿಯ
ಶಾಂತಿ-ಯನ್ನು
ಶಾಂತಿ-ಯಿಂದ
ಶಾಂತಿ-ಯಿಲ್ಲ
ಶಾಂತಿ-ಯಿ-ಲ್ಲ-ದಂ-ತಾ-ಯಿತು
ಶಾಂತಿ-ಯುಂ-ಟಾ-ಗ-ಲಿಲ್ಲ
ಶಾಂತಿಯೆ
ಶಾಂತಿ-ಯೆಂ-ದ-ರೇನು
ಶಾಂತಿ-ರ-ಸ-ವನ್ನು
ಶಾಂತಿ-ಸು-ಖ-ವನ್ನು
ಶಾಕ-ದ್ವೀ-ಪದ
ಶಾಕ-ದ್ವೀ-ಪ-ವಿದೆ
ಶಾಕ-ವೆಂಬ
ಶಾಪ
ಶಾಪ-ಕೊ-ಟ್ಟಿ-ದ್ದನು
ಶಾಪಕ್ಕಲ್ಲ
ಶಾಪಕ್ಕೆ
ಶಾಪದ
ಶಾಪ-ದಂತೆ
ಶಾಪ-ದಿಂದ
ಶಾಪ-ಭಯ
ಶಾಪ-ವ-ನ್ನಿತ್ತ
ಶಾಪ-ವನ್ನು
ಶಾಪ-ವನ್ನೂ
ಶಾಪ-ವಾ-ದ-ರೇನು
ಶಾಪ-ವಿ-ತ್ತಈ
ಶಾಪ-ವಿ-ತ್ತನು
ಶಾಪ-ವಿತ್ತು
ಶಾಪವು
ಶಾಪವೂ
ಶಾಪ-ವೊಂದು
ಶಾಮ-ರಾಯ
ಶಾಮ-ರಾ-ಯರು
ಶಾಮ-ಲ-ವರ್ಣ
ಶಾರದೆ
ಶಾರ್ಙ್ಗ-ದಿಂದ
ಶಾರ್ಙ್ಗ-ಧನು
ಶಾರ್ಙ್ಗ-ವೆಂದು
ಶಾರ್ಙ್ಗ-ವೆಂಬ
ಶಾಲಿ
ಶಾಲಿ-ಗ-ಳಾ-ಗು-ವಿರಿ
ಶಾಲ್ಮಲಿ
ಶಾಲ್ಮ-ಲಿ-ದ್ವೀ-ಪದ
ಶಾಶ್ವತ
ಶಾಶ್ವ-ತ-ಪ-ದವಿ
ಶಾಶ್ವ-ತ-ವ-ಲ್ಲ-ವೆಂ-ಬು-ದನ್ನು
ಶಾಶ್ವ-ತ-ವಾ-ಗಲಿ
ಶಾಶ್ವ-ತ-ವಾ-ಗು-ತ್ತದೆ
ಶಾಶ್ವ-ತ-ವಾದ
ಶಾಶ್ವ-ತ-ಸುಖ
ಶಾಶ್ವ-ತ-ಸು-ಖ-ದಿಂದ
ಶಾಸ-ನ-ದಲ್ಲಿ
ಶಾಸ್ತ್ರ
ಶಾಸ್ತ್ರ-ಉ-ಪ-ದೇಶ
ಶಾಸ್ತ್ರ-ಇ-ವೆ-ಲ್ಲವೂ
ಶಾಸ್ತ್ರ-ಗಳನ್ನು
ಶಾಸ್ತ್ರ-ಗಳನ್ನೂ
ಶಾಸ್ತ್ರ-ಗಳಲ್ಲಿ
ಶಾಸ್ತ್ರ-ಗಳು
ಶಾಸ್ತ್ರ-ಜ್ಞ-ರನ್ನು
ಶಾಸ್ತ್ರ-ಜ್ಞಾನ
ಶಾಸ್ತ್ರ-ಜ್ಞಾ-ನ-ದಿಂದ
ಶಾಸ್ತ್ರದ
ಶಾಸ್ತ್ರ-ರೀ-ತಿ-ಯಾಗಿ
ಶಾಸ್ತ್ರ-ವನ್ನು
ಶಾಸ್ತ್ರ-ಸ-ಮ್ಮ-ತ-ವಾ-ಗಿಯೇ
ಶಾಸ್ತ್ರಾ-ಣಾಂ
ಶಾಸ್ತ್ರಾ-ನಂ-ದರು
ಶಾಸ್ತ್ರೋ-ಕ್ತ-ವಾಗಿ
ಶಿಂಶು-ಮಾರ
ಶಿಕ್ಷಕ
ಶಿಕ್ಷ-ಣ-ವನ್ನು
ಶಿಕ್ಷಿ-ಸ-ಬ-ಹುದು
ಶಿಕ್ಷಿ-ಸ-ಬೇ-ಕೆಂದು
ಶಿಕ್ಷಿ-ಸ-ಲಾ-ರದೆ
ಶಿಕ್ಷಿಸಿ
ಶಿಕ್ಷಿ-ಸಿದ
ಶಿಕ್ಷಿ-ಸಿ-ದಂ-ತಾ-ಯಿತು
ಶಿಕ್ಷಿ-ಸು-ವು-ದು-ಇ-ವೆಲ್ಲ
ಶಿಕ್ಷಿ-ಸುವೆ
ಶಿಕ್ಷಿ-ಸು-ವೆ-ನೆಂದು
ಶಿಕ್ಷೆ
ಶಿಕ್ಷೆ-ಗಳನ್ನೆಲ್ಲ
ಶಿಕ್ಷೆ-ಗಾಗಿ
ಶಿಕ್ಷೆಗೆ
ಶಿಕ್ಷೆ-ಯನ್ನು
ಶಿಕ್ಷೆ-ಯಾ-ಯಿತು
ಶಿಕ್ಷೆ-ಯೇ-ನೆಂ-ಬು-ದನ್ನು
ಶಿಖಂ-ಡಿನಿ
ಶಿಖರ
ಶಿಖ-ರಕ್ಕೆ
ಶಿಖ-ರ-ಗಳ
ಶಿಖ-ರ-ಗಳಿಂದ
ಶಿಖ-ರ-ಗ-ಳಿ-ದ್ದರೂ
ಶಿಖ-ರ-ಗಳು
ಶಿಖ-ರ-ದಂತೆ
ಶಿಖ-ರ-ಪ್ರಾ-ಯ-ವಾ-ದು-ದಾ-ಗಿದೆ
ಶಿಖ-ರ-ವನ್ನು
ಶಿಖಾಯೈ
ಶಿಪ್ಪ-ದಿ-ಕಾ-ರಂ-ನಲ್ಲಿ
ಶಿಬಿ
ಶಿಬಿ-ರ-ದಿಂದ
ಶಿರಸಿ
ಶಿರಸೇ
ಶಿರ-ಸೇ-ನಮಃ
ಶಿರ-ಸ್ಸಿ-ನಲ್ಲಿ
ಶಿರಸ್ಸು
ಶಿರೋ-ಮ-ಣಿ-ಯಂ-ತಿ-ರುವ
ಶಿರೋ-ಮ-ಣಿ-ಯಾದ
ಶಿಲ-ಪ್ಪ-ದಿ-ಕಾ-ರಂ-ನಲ್ಲಿ
ಶಿಲೆ
ಶಿಲ್ಪ-ಕ-ಲೆಗೆ
ಶಿಲ್ಪ-ಚಾ-ತು-ರ್ಯ-ವನ್ನು
ಶಿಲ್ಪಿ
ಶಿಲ್ಪಿಗೆ
ಶಿಲ್ಪಿ-ಯಾದ
ಶಿವ
ಶಿವ-ಪಾ-ರ್ವ-ತಿ-ಯರ
ಶಿವ-ಶಿ-ವೆ-ಯರು
ಶಿವ-ದ್ವೇಷಿ
ಶಿವ-ದ್ವೇ-ಷಿ-ಯಾದ
ಶಿವ-ಧ-ನು-ಸ್ಸನ್ನು
ಶಿವನ
ಶಿವ-ನನ್ನು
ಶಿವ-ನಿಗೆ
ಶಿವನು
ಶಿವ-ನೆಂಬ
ಶಿವ-ನೊ-ಡನೆ
ಶಿವ-ಪು-ರಾಣ
ಶಿವ-ಭ-ಕ್ತ-ನಾದ
ಶಿವ-ರೂ-ಪ-ವಾ-ಗಿಯೋ
ಶಿವೆ-ಯೊ-ಡನೆ
ಶಿಶು
ಶಿಶು-ಗ-ಳಾಗಿ
ಶಿಶು-ಗಳೂ
ಶಿಶು-ನಾ-ಗನೂ
ಶಿಶು-ನಾ-ಗ-ನೆಂ-ಬು-ವನು
ಶಿಶು-ಪಾಲ
ಶಿಶು-ಪಾ-ಲನ
ಶಿಶು-ಪಾ-ಲ-ನನ್ನು
ಶಿಶು-ಪಾ-ಲ-ನಿಗೆ
ಶಿಶು-ಪಾ-ಲ-ನಿಗೇ
ಶಿಶು-ಪಾ-ಲನು
ಶಿಶು-ಪಾ-ಲನೇ
ಶಿಶು-ಪಾ-ಲ-ನೊ-ಡನೆ
ಶಿಶು-ಪಾ-ಲ-ವ-ಧೆ-ಯನ್ನು
ಶಿಶು-ಪಾ-ಲಾ-ದಿ-ಗಳು
ಶಿಶು-ಮಾರ
ಶಿಶು-ಮಾ-ರ-ನೆಂಬ
ಶಿಶು-ರೂ-ಪ-ದ-ಲ್ಲಿ-ರುವ
ಶಿಶು-ರೂ-ಪ-ದಿಂದ
ಶಿಶು-ರೂ-ಪಿ-ನಿಂದ
ಶಿಶು-ರೂ-ಪಿ-ಯನ್ನು
ಶಿಶು-ರೂ-ಪಿ-ಯಾದ
ಶಿಶು-ಲೀ-ಲೆ-ಯನ್ನು
ಶಿಶು-ವನ್ನು
ಶಿಶು-ವಾ-ಗಿ-ದ್ದಾ-ಗಲೇ
ಶಿಶು-ವಿನ
ಶಿಶು-ವಿ-ನಂತೆ
ಶಿಶುವೂ
ಶಿಶು-ಹ-ತ್ಯೆ-ಯನ್ನು
ಶಿಷ್ಟ-ಪ-ರಿ-ಪಾ-ಲ-ನೆ-ಗಾಗಿ
ಶಿಷ್ಟ-ರ-ಕ್ಷ-ಣೆ-ಗಾಗಿ
ಶಿಷ್ಟ-ರನ್ನು
ಶಿಷ್ಯ
ಶಿಷ್ಯನ
ಶಿಷ್ಯ-ನಾಗಿ
ಶಿಷ್ಯ-ನಾದ
ಶಿಷ್ಯ-ನಿಗೆ
ಶಿಷ್ಯನು
ಶಿಷ್ಯ-ನೊ-ಬ್ಬನು
ಶಿಷ್ಯ-ರಂ-ತಿ-ರುವ
ಶಿಷ್ಯ-ರನ್ನು
ಶಿಷ್ಯ-ರೊ-ಡ-ಗೂ-ಡಿದ
ಶಿಷ್ಯ-ರೊ-ಡನೆ
ಶಿಸಿ
ಶಿಸ್ತನ್ನು
ಶಿಸ್ತು
ಶಿಸ್ತೂ
ಶೀಘ್ರ
ಶೀಘ್ರ-ವಾಗಿ
ಶೀಘ್ರ-ವಾ-ಗಿಯೇ
ಶೀತ
ಶೀತೋಷ್ಣ
ಶೀಲ
ಶೀಲ-ಗಳನ್ನೂ
ಶೀಲ-ಗಳಲ್ಲಿ
ಶೀಲ-ಗ-ಳಿಗೆ
ಶೀಲ-ದಲ್ಲಿ
ಶುಂಗ
ಶುಂಗ-ನನ್ನು
ಶುಕ
ಶುಕ-ನನ್ನು
ಶುಕ-ನಿಗೆ
ಶುಕನು
ಶುಕ-ಮ-ಹರ್ಷಿ
ಶುಕ-ಮ-ಹ-ರ್ಷಿ-ಗಳನ್ನು
ಶುಕ-ಮ-ಹ-ರ್ಷಿ-ಗಳಲ್ಲಿ
ಶುಕ-ಮ-ಹ-ರ್ಷಿಗೆ
ಶುಕ-ಮ-ಹ-ರ್ಷಿಯು
ಶುಕ-ಮು-ಖ-ದಿಂದ
ಶುಕ-ಮುನಿ
ಶುಕ-ಮು-ನಿ-ಕೃತ
ಶುಕ-ಮು-ನಿ-ಗಳನ್ನೂ
ಶುಕ-ಮು-ನಿಗೆ
ಶುಕ-ಮು-ನಿಯ
ಶುಕ-ಮು-ನಿ-ಯನ್ನು
ಶುಕ-ಮು-ನಿಯು
ಶುಕ-ರಾಜ
ಶುಕ್ರ-ಗ್ರ-ಹ-ವಿದೆ
ಶುಕ್ರ-ಚಾ-ರ್ಯ-ನಿಗೂ
ಶುಕ್ರನ
ಶುಕ್ರ-ನಂ-ತಹ
ಶುಕ್ರನು
ಶುಕ್ರ-ಮು-ನಿಗೆ
ಶುಕ್ರಾ-ಚಾರ್ಯ
ಶುಕ್ರಾ-ಚಾ-ರ್ಯನ
ಶುಕ್ರಾ-ಚಾ-ರ್ಯ-ನಿಗೆ
ಶುಕ್ರಾ-ಚಾ-ರ್ಯ-ನಿ-ಲ್ಲದೆ
ಶುಕ್ರಾ-ಚಾ-ರ್ಯನು
ಶುಕ್ರಾ-ಚಾ-ರ್ಯರ
ಶುಕ್ರಾ-ಚಾ-ರ್ಯ-ರಲ್ಲಿ
ಶುಕ್ರಾ-ಚಾ-ರ್ಯ-ರಿಗೆ
ಶುಕ್ರಾ-ಚಾ-ರ್ಯರು
ಶುಕ್ಲ
ಶುಕ್ಲ-ಪಕ್ಷ
ಶುಕ್ಲ-ಪ-ಕ್ಷದ
ಶುಕ್ಲ-ಪ-ಕ್ಷ-ದಲ್ಲಿ
ಶುಚಿ
ಶುಚಿ-ಯಾ-ಗಿ-ರ-ಬೇಕು
ಶುಚಿ-ರ್ಭೂ-ತ-ನಾಗಿ
ಶುದ್ಧ
ಶುದ್ಧ-ಜ್ಞಾನ
ಶುದ್ಧ-ಭ-ಕ್ತಿಯೂ
ಶುದ್ಧ-ಜೀ-ವ-ನಿಗೆ
ಶುದ್ಧ-ತೆ-ಯನ್ನು
ಶುದ್ಧ-ನಾದ
ಶುದ್ಧ-ಬ್ರಾ-ಹ್ಮ-ಣ-ರಾ-ದರು
ಶುದ್ಧ-ಮ-ನ-ಸ್ಸಿ-ನಲ್ಲಿ
ಶುದ್ಧ-ಮ-ನ-ಸ್ಸಿ-ನ-ವರು
ಶುದ್ಧ-ಮ-ನ-ಸ್ಸಿ-ನಿಂದ
ಶುದ್ಧ-ಮಾಡಿ
ಶುದ್ಧ-ವಾ-ಗಿ-ರು-ವುದು
ಶುದ್ಧ-ವಾ-ಗು-ತ್ತದೆ
ಶುದ್ಧ-ವಾದ
ಶುದ್ಧ-ವಾ-ಯಿತು
ಶುದ್ಧಾ-ನ್ನ-ವನ್ನು
ಶುದ್ಧಿ-ಗೊ-ಳಿ-ಸ-ಬೇಕು
ಶುದ್ಧಿ-ಮಾ-ಡ-ಬೇ-ಕೆಂದು
ಶುದ್ಧಿ-ಮಾ-ಡಿ-ದರು
ಶುನ-ಕ-ನೆಂಬ
ಶುನ-ಶ್ಯೇ-ಫನು
ಶುನ-ಶ್ಶೇ-ಫ-ನನ್ನು
ಶುನ-ಶ್ಶೇ-ಫ-ನೆಂಬ
ಶುಭ
ಶುಭ-ಚ-ರಿ-ತ್ರೆ-ಯನ್ನು
ಶುಭ-ದಾ-ಯಕ
ಶುಭ-ದಿನ
ಶುಭ-ದಿ-ನ-ವಾ-ಗಿತ್ತು
ಶುಭ-ಮು-ಹೂರ್ತ
ಶುಭ-ಮು-ಹೂ-ರ್ತ-ದಲ್ಲಿ
ಶುಭ-ಲ-ಗ್ನ-ದಲ್ಲಿ
ಶುಭ-ಸ್ಥಾನ
ಶುಭ್ರ
ಶುಭ್ರ-ಧ-ವ-ಳ-ಕಾಂ-ತಿ-ಯಿಂದ
ಶುಭ್ರ-ಧ-ವ-ಳ-ವಾದ
ಶುಭ್ರ-ವಾಗಿ
ಶುಭ್ರ-ವಾದ
ಶುಷ್ಯ-ದ್ಧ್ರ-ದಾಃ
ಶೂದ್ರ
ಶೂದ್ರ-ನಾಗಿ
ಶೂದ್ರನೂ
ಶೂದ್ರ-ನೊ-ಬ್ಬನು
ಶೂದ್ರ-ರಾ-ಣಿ-ಯಲ್ಲಿ
ಶೂದ್ರರು
ಶೂದ್ರರೂ
ಶೂದ್ರ-ರೆಂಬ
ಶೂದ್ರರೇ
ಶೂನ್ಯ-ಹೃ-ದ-ಯೆ-ಯ-ರಾಗಿ
ಶೂರ
ಶೂರನ
ಶೂರ-ನಾದ
ಶೂರ-ನಿಗೆ
ಶೂರ-ನಿ-ರ-ಲಿಲ್ಲ
ಶೂರನೂ
ಶೂರ-ನೆಂದು
ಶೂರ-ನೆಂ-ಬು-ವನು
ಶೂರನೋ
ಶೂರ-ರನ್ನು
ಶೂರ-ರಲ್ಲಿ
ಶೂರ-ರಾ-ಗು-ವು-ದಿಲ್ಲ
ಶೂರ-ರಾದ
ಶೂರ-ರಾ-ದ-ರೇನು
ಶೂರರು
ಶೂರ-ರೆ-ಲ್ಲರೂ
ಶೂರರೇ
ಶೂರ-ಸೇನ
ಶೂರ-ಸೇ-ನನ
ಶೂರ-ಸೇ-ನ-ನೆಂ-ಬು-ವನು
ಶೂರ್ಪ
ಶೂರ್ಪ-ಣಖಿ
ಶೂಲ
ಶೂಲ-ಗಳಿಂದ
ಶೂಲ-ದಿಂದ
ಶೂಲ-ದಿಂ-ದಲೊ
ಶೂಲ-ವನ್ನು
ಶೂಲ-ವನ್ನೂ
ಶೃಂಗಾರ
ಶೃಂಗಾ-ರ-ಜೀ-ವ-ನದ
ಶೃಂಗಾ-ರ-ರ-ಸ-ವನ್ನು
ಶೃಂಗಿಯು
ಶೃತ
ಶೇಖ-ರಿ-ಸುವ
ಶೇಷಕ್ಕೆ
ಶೇಷನು
ಶೇಷ-ಶಾ-ಯಿ-ಯಾಗಿ
ಶೇಷ-ಶಾ-ಯಿ-ಯಾದ
ಶೈತ್ಯಾ-ಧಿ-ಕ್ಯ-ದಿಂದ
ಶೈಲ-ದಂತೆ
ಶೈಲಿ-ಯಿಂದ
ಶೈವ-ರನ್ನು
ಶೈಶವ
ಶೈಶ-ವ-ದಲ್ಲೆಲ್ಲ
ಶೈಶ-ವ-ಬಾ-ಲ್ಯ-ಗಳನ್ನು
ಶೋಣಿ-ತ-ಪು-ರಕ್ಕೆ
ಶೋಣಿ-ತ-ಪು-ರದ
ಶೋಣಿ-ತ-ಪು-ರ-ದಲ್ಲಿ
ಶೋಧ-ಕನೂ
ಶೋಧಿಸಿ
ಶೋಧಿ-ಸಿ-ದರೂ
ಶೌಚ
ಶೌನಕ
ಶೌನ-ಕನೇ
ಶೌನ-ಕ-ಪು-ಷಿಯು
ಶೌನ-ಕಾದಿ
ಶೌನ-ಕಾ-ದಿ-ಗಳನ್ನು
ಶೌರ-ಸೇ-ನಿ-ಯರು
ಶೌರೇಃ
ಶೌರ್ಯ
ಶ್ಚಂದ್ರ
ಶ್ಚಚಾರ
ಶ್ಮಶಾ-ನ-ವಾ-ಗಿದೆ
ಶ್ಮಶಾ-ನ-ವಾ-ಸಿ-ಯೆಂ-ಬಂತೆ
ಶ್ಯಾಮ
ಶ್ಯಾಮಕ
ಶ್ಯಾಮಲ
ಶ್ಯಾಮ-ಲ-ವ-ರ್ಣದ
ಶ್ಯಾಮ-ಲ-ವಾದ
ಶ್ರತ-ಶ್ರವಾ
ಶ್ರದ್ದ-ಧಾ-ನಾಃ
ಶ್ರದ್ಧಾ-ದೇ-ವಿ-ಯನ್ನೂ
ಶ್ರದ್ಧೆ-ಗಳಿಂದ
ಶ್ರದ್ಧೆಗೆ
ಶ್ರದ್ಧೆಯ
ಶ್ರದ್ಧೆ-ಯನ್ನು
ಶ್ರದ್ಧೆ-ಯಿಂದ
ಶ್ರಮ
ಶ್ರಮ-ಧ-ರ್ಮ-ಗಳನ್ನು
ಶ್ರಮ-ವೆಲ್ಲ
ಶ್ರವಣ
ಶ್ರವ-ಣ-ಗಳ
ಶ್ರವ-ಣ-ದಿಂದ
ಶ್ರವ-ಣ-ಮಂ-ಗಳಂ
ಶ್ರವಸ್ಸು
ಶ್ರಾದ್ಧ-ಕಾಲ
ಶ್ರಾದ್ಧ-ಕ್ಕಾಗಿ
ಶ್ರಾದ್ಧಕ್ಕೆ
ಶ್ರಾದ್ಧಾ-ದಿ-ಗಳಿಂದ
ಶ್ರಿಯಃ-ಪತಿ
ಶ್ರಿಯ-ಮೃತೇ
ಶ್ರೀ
ಶ್ರೀಕೃಷ್ಣ
ಶ್ರೀಕೃ-ಷ್ಣ-ಇವೂ
ಶ್ರೀಕೃ-ಷ್ಣ-ಕ-ಥಾ-ಮೃ-ತ-ದಿಂದ
ಶ್ರೀಕೃ-ಷ್ಣ-ದ-ರ್ಶ-ನ-ಕ್ಕಾಗಿ
ಶ್ರೀಕೃ-ಷ್ಣನ
ಶ್ರೀಕೃ-ಷ್ಣ-ನಂತೆ
ಶ್ರೀಕೃ-ಷ್ಣ-ನಂ-ತೆಯೆ
ಶ್ರೀಕೃ-ಷ್ಣ-ನಂ-ತೆಯೇ
ಶ್ರೀಕೃ-ಷ್ಣ-ನ-ದಿ-ರ-ಬೇ-ಕೆ-ನಿ-ಸಿತು
ಶ್ರೀಕೃ-ಷ್ಣ-ನದೇ
ಶ್ರೀಕೃ-ಷ್ಣ-ನನ್ನು
ಶ್ರೀಕೃ-ಷ್ಣ-ನನ್ನೂ
ಶ್ರೀಕೃ-ಷ್ಣ-ನನ್ನೆ
ಶ್ರೀಕೃ-ಷ್ಣ-ನನ್ನೇ
ಶ್ರೀಕೃ-ಷ್ಣ-ನಲ್ಲಿ
ಶ್ರೀಕೃ-ಷ್ಣ-ನ-ಲ್ಲಿ-ರುವ
ಶ್ರೀಕೃ-ಷ್ಣ-ನಾಗಿ
ಶ್ರೀಕೃ-ಷ್ಣ-ನಾ-ಗಿಯೂ
ಶ್ರೀಕೃ-ಷ್ಣ-ನಾ-ದರೂ
ಶ್ರೀಕೃ-ಷ್ಣನಿ
ಶ್ರೀಕೃ-ಷ್ಣ-ನಿಂದ
ಶ್ರೀಕೃ-ಷ್ಣ-ನಿ-ಗಿಂತ
ಶ್ರೀಕೃ-ಷ್ಣ-ನಿಗೂ
ಶ್ರೀಕೃ-ಷ್ಣ-ನಿಗೆ
ಶ್ರೀಕೃ-ಷ್ಣ-ನಿ-ದ್ದನೊ
ಶ್ರೀಕೃ-ಷ್ಣ-ನಿ-ಲ್ಲದೆ
ಶ್ರೀಕೃ-ಷ್ಣನು
ಶ್ರೀಕೃ-ಷ್ಣನೂ
ಶ್ರೀಕೃ-ಷ್ಣನೆ
ಶ್ರೀಕೃ-ಷ್ಣ-ನೆಂ-ದರೆ
ಶ್ರೀಕೃ-ಷ್ಣ-ನೆಂಬ
ಶ್ರೀಕೃ-ಷ್ಣನೇ
ಶ್ರೀಕೃ-ಷ್ಣ-ನೇನು
ಶ್ರೀಕೃ-ಷ್ಣ-ನೇ-ನೆ-ನ್ನು-ವನೊ
ಶ್ರೀಕೃ-ಷ್ಣ-ನೊ-ಡನೆ
ಶ್ರೀಕೃ-ಷ್ಣ-ನೊ-ಬ್ಬನೆ
ಶ್ರೀಕೃ-ಷ್ಣ-ಪ-ರ-ಮಾತ್ಮ
ಶ್ರೀಕೃ-ಷ್ಣ-ಪ-ರ-ಮಾ-ತ್ಮನ
ಶ್ರೀಕೃ-ಷ್ಣ-ಪ-ರ-ಮಾ-ತ್ಮ-ನನ್ನು
ಶ್ರೀಕೃ-ಷ್ಣ-ಪ-ರ-ಮಾ-ತ್ಮನು
ಶ್ರೀಕೃ-ಷ್ಣ-ಪ-ರ-ಮಾ-ತ್ಮನೆ
ಶ್ರೀಕೃ-ಷ್ಣ-ಬ-ಲ-ರಾ-ಮರ
ಶ್ರೀಕೃ-ಷ್ಣ-ಬ-ಲ-ರಾ-ಮರು
ಶ್ರೀಕೃ-ಷ್ಣ-ಬ-ಲ-ರಾ-ಮ-ರೊ-ಡನೆ
ಶ್ರೀಕೃ-ಷ್ಣ-ಭ-ಕ್ತ-ನಾದ
ಶ್ರೀಕೃ-ಷ್ಣ-ಭ-ಗ-ವಾ-ನನು
ಶ್ರೀಕೃ-ಷ್ಣ-ಮ-ಯ-ವಾ-ಗಿತ್ತು
ಶ್ರೀಕೃ-ಷ್ಣ-ಮ-ಹಿ-ಮೆ-ಯನ್ನು
ಶ್ರೀಕೃ-ಷ್ಣ-ಮಾಯೆ
ಶ್ರೀಕೃ-ಷ್ಣ-ಮಾ-ಯೆಯ
ಶ್ರೀಕೃ-ಷ್ಣ-ಮೂ-ರ್ತಿ-ಯಿಂದ
ಶ್ರೀಕೃ-ಷ್ಣರ
ಶ್ರೀಕೃ-ಷ್ಣ-ರಾ-ದರೂ
ಶ್ರೀಕೃ-ಷ್ಣ-ರಿ-ದ್ದರೊ
ಶ್ರೀಕೃ-ಷ್ಣರು
ಶ್ರೀಕೃ-ಷ್ಣ-ರು-ಕ್ಮಿ-ಣಿ-ಯರು
ಶ್ರೀಕೃ-ಷ್ಣ-ರೂ-ಪ-ದಿಂದ
ಶ್ರೀಕೃ-ಷ್ಣ-ವ-ತಾ-ರವು
ಶ್ರೀಕೃ-ಷ್ಣಾ-ಕೃ-ತಿ-ಗಳನ್ನು
ಶ್ರೀಕೃ-ಷ್ಣಾ-ಭಿ-ಮು-ಖ-ರಾ-ಗು-ತ್ತಾರೆ
ಶ್ರೀಕೃ-ಷ್ಣಾ-ರ್ಜು-ನರು
ಶ್ರೀಕೃ-ಷ್ಮನು
ಶ್ರೀಗಂಧ
ಶ್ರೀಗಂ-ಧ-ಗಳ
ಶ್ರೀಗಂ-ಧದ
ಶ್ರೀದೇ-ವಾ-ವ-ಸು-ಹಂಸ
ಶ್ರೀನ-ರ-ಸಿಂ-ಹ-ಮೂ-ರ್ತಿಗೆ
ಶ್ರೀಭಾನು
ಶ್ರೀಮ
ಶ್ರೀಮಂತ
ಶ್ರೀಮಂ-ತ-ರಾ-ದರು
ಶ್ರೀಮಂ-ತ-ರಿಗೆ
ಶ್ರೀಮಂ-ತರು
ಶ್ರೀಮಂ-ತ-ವಾ-ಯಿತು
ಶ್ರೀಮಂ-ಸ್ತ್ವ-ಮ-ತಿ-ದ-ಯಿತೋ
ಶ್ರೀಮ-ದಾ-ತತಂ
ಶ್ರೀಮ-ದ್ಬಾ-ಗ-ವ-ತದ
ಶ್ರೀಮ-ದ್ಭಾ-ಗ-ವತ
ಶ್ರೀಮ-ದ್ಭಾ-ಗ-ವ-ತದ
ಶ್ರೀಮ-ದ್ಭಾ-ಗ-ವ-ತ-ದಲ್ಲಿ
ಶ್ರೀಮ-ದ್ಭಾ-ಗ-ವ-ತ-ವನ್ನು
ಶ್ರೀಮ-ದ್ಭಾ-ಗ-ವ-ತವು
ಶ್ರೀಮ-ದ್ಭಾ-ಗ-ವ-ತೋ-ಧ್ಯಾ-ಯ-ಸಾರೋ
ಶ್ರೀಮ-ನ್ನಾ-ರಾ-ಯ-ಣ-ನನ್ನು
ಶ್ರೀಮ-ನ್ನಾ-ರಾ-ಯ-ಣನು
ಶ್ರೀರಂಗ
ಶ್ರೀರಾಮ
ಶ್ರೀರಾ-ಮ-ಕೃಷ್ಣ
ಶ್ರೀರಾ-ಮ-ಕೃ-ಷ್ಣರ
ಶ್ರೀರಾ-ಮ-ಕೃ-ಷ್ಣ-ರಿಗೆ
ಶ್ರೀರಾ-ಮ-ಕೃ-ಷ್ಣರು
ಶ್ರೀರಾ-ಮ-ಕೃ-ಷ್ಣಾ-ಶ್ರ-ಮದ
ಶ್ರೀರಾ-ಮ-ಕೃ-ಷ್ಣಾ-ಶ್ರ-ಮ-ದಲ್ಲಿ
ಶ್ರೀರಾ-ಮ-ಚಂದ್ರ
ಶ್ರೀರಾ-ಮ-ಚಂ-ದ್ರನು
ಶ್ರೀರಾ-ಮನ
ಶ್ರೀರಾ-ಮ-ನಂತೆ
ಶ್ರೀರಾ-ಮ-ನನ್ನು
ಶ್ರೀರಾ-ಮ-ನಿಗೆ
ಶ್ರೀರಾ-ಮನು
ಶ್ರೀರಾ-ಮನೆ
ಶ್ರೀರಾ-ಮ-ನೆಂಬ
ಶ್ರೀರಾ-ಮ-ನೊ-ಡನೆ
ಶ್ರೀರಾ-ಮ-ಮೂ-ರ್ತಿ-ಯನ್ನು
ಶ್ರೀವತ್ಸ
ಶ್ರೀವ-ತ್ಸ-ವೆಂಬ
ಶ್ರೀವ-ತ್ಸಾಂಕಂ
ಶ್ರೀವ-ಧೂ-ಸ್ಸಾ-ಕ-ಮಾಸ್ತೇ
ಶ್ರೀವಿಷ್ಣು
ಶ್ರೀಶೈ-ಲಕ್ಕೆ
ಶ್ರೀಸಾ-ಮಾ-ನ್ಯನ
ಶ್ರೀಹರಿ
ಶ್ರೀಹ-ರಿಗೆ
ಶ್ರೀಹ-ರಿಯ
ಶ್ರೀಹ-ರಿ-ಯನ್ನು
ಶ್ರೀಹ-ರಿ-ಯಲ್ಲಿ
ಶ್ರೀಹ-ರಿ-ಯ-ಲ್ಲಿಗೆ
ಶ್ರೀಹ-ರಿ-ಯಿಂದ
ಶ್ರೀಹ-ರಿಯು
ಶ್ರೀಹ-ರಿಯೂ
ಶ್ರೀಹ-ರಿಯೆ
ಶ್ರೀಹ-ರಿ-ಯೆಂದೆ
ಶ್ರೀಹ-ರಿ-ಯೆಂ-ಬು-ವನು
ಶ್ರೀಹ-ರಿಯೇ
ಶ್ರುತ-ಕೀರ್ತಿ
ಶ್ರುತ-ಕೀ-ರ್ತಿ-ಯೆಂಬ
ಶ್ರುತ-ದೇ-ವನು
ಶ್ರುತ-ದೇ-ವ-ನೆಂಬ
ಶ್ರುತ-ದೇ-ವರೂ
ಶ್ರುತ-ದೇವಾ
ಶ್ರುತ-ಧರ
ಶ್ರುತಮ್
ಶ್ರುತ-ಸೇ-ನ-ರೆಂಬ
ಶ್ರುತಿ
ಶ್ರುತಿ-ಯನ್ನು
ಶ್ರುತಿ-ಯಲ್ಲಿ
ಶ್ರುತಿ-ಯೊ-ಡನೆ
ಶ್ರುಮ
ಶ್ರೇಯ
ಶ್ರೇಯಃ-ಪ್ರ-ತೀ-ಪ-ಕಾಃ
ಶ್ರೇಯ-ಸ್ಕರ
ಶ್ರೇಯ-ಸ್ಕ-ರ-ವಾ-ಗಿಯೂ
ಶ್ರೇಯ-ಸ್ಸನ್ನು
ಶ್ರೇಯ-ಸ್ಸಾಗು
ಶ್ರೇಯ-ಸ್ಸಿ-ಗಾಗಿ
ಶ್ರೇಯ-ಸ್ಸಿ-ಗಾ-ಗಿಯೆ
ಶ್ರೇಯ-ಸ್ಸಿನ
ಶ್ರೇಯಸ್ಸು
ಶ್ರೇಯ-ಸ್ಸುಂ-ಟಾ-ಗು-ತ್ತದೆ
ಶ್ರೇಯ-ಸ್ಸೇ-ನಿದೆ
ಶ್ರೇಯಸ್ಸೋ
ಶ್ರೇಷ್ಠ
ಶ್ರೇಷ್ಠ-ನಾ-ಗ-ಬೇ-ಕೆಂದು
ಶ್ರೇಷ್ಠ-ನಾ-ಗಿ-ರು-ವ-ವನು
ಶ್ರೇಷ್ಠನೂ
ಶ್ರೇಷ್ಠರು
ಶ್ರೇಷ್ಠ-ವಾದ
ಶ್ರೇಷ್ಠ-ವಾದು
ಶ್ರೇಷ್ಠ-ವಾ-ದುದು
ಶ್ರೇಷ್ಠ-ವಾ-ದು-ದೆಂದು
ಶ್ರೋತ್ರಿಯ
ಶ್ಲೇಷ್ಮ
ಶ್ಲೋಕ-ಎಂದು
ಶ್ಲೋಕ-ಗಳು
ಶ್ಲೋಕ-ದೊ-ಡನೆ
ಶ್ಲೋಕ-ವಂತೆ
ಶ್ಲೋಕವು
ಶ್ವಪ-ಚ-ನಿಗೆ
ಶ್ವಫಲ್ಕ
ಶ್ವರ
ಶ್ವರ-ನಲ್ಲಿ
ಶ್ವರನು
ಶ್ವರು
ಶ್ವಾಸ-ಧಾ-ರಣೆ
ಷಂ
ಷಂಡ-ತ-ನಕ್ಕೆ
ಷಡಂಘ್ರೇ
ಷಡ-ಕ್ಷರೀ
ಷಡ್ಗು-ಣೈ-ಶ್ವರ್ಯ
ಷಡ್ವ-ರ್ಗ-ಗಳಿಂದ
ಷಣ-ನನ್ನು
ಷಣ್ಮು-ಖನು
ಷೇಕ
ಷೋಡ-ಶ-ಕ-ಲಾಯ
ಷೋಡ-ಶೋ-ಪ-ಚಾ-ರ-ಗ-ಳಿಗೆ
ಸ
ಸಂಕಟ
ಸಂಕ-ಟ-ಕ-ರ-ವಾ-ಗಿ-ದ್ದರೂ
ಸಂಕ-ಟ-ಗಳನ್ನು
ಸಂಕ-ಟ-ಗ-ಳಿಂ-ದಲೂ
ಸಂಕ-ಟ-ಗ-ಳಿಗೆ
ಸಂಕ-ಟ-ಗಳು
ಸಂಕ-ಟ-ದಲ್ಲಿ
ಸಂಕ-ಟ-ದ-ಲ್ಲಿ-ರು-ವಾ-ಗಲೆ
ಸಂಕ-ಟ-ದಿಂದ
ಸಂಕ-ಟ-ಪ-ಟು-ತ್ತಿ-ರುವ
ಸಂಕ-ಟ-ಪ-ಟ್ಟರು
ಸಂಕ-ಟ-ಪ-ಟ್ಟೆವು
ಸಂಕ-ಟ-ಪ-ಡಿ-ಸುವಿ
ಸಂಕ-ಟ-ಪ-ಡುತ್ತಾ
ಸಂಕ-ಟ-ಪ-ಡು-ತ್ತಾನೆ
ಸಂಕ-ಟ-ಪ-ಡು-ತ್ತಾರೆ
ಸಂಕ-ಟ-ಪ-ಡು-ತ್ತಿದ್ದ
ಸಂಕ-ಟ-ಪ-ಡು-ತ್ತಿ-ದ್ದರು
ಸಂಕ-ಟ-ಪ-ಡು-ತ್ತಿ-ದ್ದ-ವ-ರು-ತಮ್ಮ
ಸಂಕ-ಟ-ಪ-ಡು-ತ್ತಿ-ರು-ವರೊ
ಸಂಕ-ಟ-ಪ-ಡು-ತ್ತಿ-ರು-ವಿ-ರಲ್ಲಾ
ಸಂಕ-ಟ-ಪ-ಡು-ತ್ತಿ-ರು-ವೆಯಾ
ಸಂಕ-ಟ-ಪ-ಡು-ವುದು
ಸಂಕ-ಟ-ವನ್ನು
ಸಂಕ-ಟ-ವ-ನ್ನುಂಟು
ಸಂಕ-ಟ-ವಾ-ಗು-ತ್ತದೆ
ಸಂಕ-ಟ-ವಾ-ಯಿತು
ಸಂಕ-ಟ-ವೆಂದರೆ
ಸಂಕ-ಟ-ವೊ-ದ-ಗಿ-ದಾಗ
ಸಂಕರ್
ಸಂಕ-ರ್ಷಣ
ಸಂಕ-ರ್ಷ-ಣ
ಸಂಕ-ರ್ಷ-ಣ-ವಾ-ಸು-ದೇವ
ಸಂಕ-ರ್ಷ-ಣ-ಎಂಬ
ಸಂಕ-ರ್ಷ-ಣದ
ಸಂಕ-ರ್ಷ-ಣ-ನನ್ನು
ಸಂಕ-ರ್ಷ-ಣನು
ಸಂಕ-ರ್ಷ-ಣ-ನೆಂದೂ
ಸಂಕ-ರ್ಷ-ಣ-ಮೂ-ರ್ತಿಯ
ಸಂಕ-ರ್ಷ-ಣ-ರೆಂಬ
ಸಂಕ-ರ್ಷ-ಣ-ಸ್ವಾ-ಮಿಯ
ಸಂಕ-ಲೆ-ಗಳನ್ನು
ಸಂಕ-ಲೆ-ಯನ್ನು
ಸಂಕಲ್ಪ
ಸಂಕ-ಲ್ಪಕ್ಕೆ
ಸಂಕ-ಲ್ಪ-ದಿಂದ
ಸಂಕ-ಲ್ಪ-ದಿಂ-ದಲೇ
ಸಂಕ-ಲ್ಪ-ಮಾತ್ರ
ಸಂಕ-ಲ್ಪ-ರೂ-ಪ-ವಾದ
ಸಂಕ-ಲ್ಪ-ವನ್ನು
ಸಂಕ-ಲ್ಪವೆ
ಸಂಕ-ಲ್ಪಿ-ಸಿದ
ಸಂಕೀ-ರ್ತನ
ಸಂಕೀ-ರ್ತ-ನ-ದಿಂದ
ಸಂಕೀ-ರ್ತನೆ
ಸಂಕೀ-ರ್ತ-ನೆ-ಯನ್ನು
ಸಂಕೀ-ರ್ತ-ನೆ-ಯೊಂ-ದ-ರಿಂ-ದಲೆ
ಸಂಕೃ-ತಿ-ಯೆಂ-ಬು-ವ-ನಿಗೆ
ಸಂಕೋಚ
ಸಂಕೋಲೆ
ಸಂಕೋ-ಲೆ-ಗ-ಳೊ-ಡನೆ
ಸಂಕ್ಷಯಂ
ಸಂಖ್ಯಾ-ನಾ-ಯ-ಽನಂತಾ
ಸಂಖ್ಯೆ
ಸಂಖ್ಯೆಯ
ಸಂಖ್ಯೆ-ಯಲ್ಲಿ
ಸಂಗ
ಸಂಗಡಿ
ಸಂಗ-ಡಿ-ಗ-ರ-ನ್ನೆಲ್ಲ
ಸಂಗ-ಡಿ-ಗ-ರಿಗೂ
ಸಂಗ-ಡಿ-ಗ-ರೊ-ಡನೆ
ಸಂಗತಿ
ಸಂಗ-ತಿ-ಗಳನ್ನು
ಸಂಗ-ತಿ-ಗಳು
ಸಂಗ-ತಿ-ಯನ್ನು
ಸಂಗ-ತಿ-ಯ-ನ್ನೆಲ್ಲ
ಸಂಗ-ತಿ-ಯೆಲ್ಲ
ಸಂಗ-ದಲ್ಲಿ
ಸಂಗ-ದಿಂದ
ಸಂಗಮ
ಸಂಗ-ಮ-ದಲ್ಲಿ
ಸಂಗ-ಮ-ಸ್ಥ-ಳ-ದಲ್ಲಿ
ಸಂಗ-ಮ-ಸ್ಥಾ-ನ-ದಲ್ಲಿ
ಸಂಗ-ವನ್ನೂ
ಸಂಗವೇ
ಸಂಗೀತ
ಸಂಗೀ-ತಕ್ಕೆ
ಸಂಗೀ-ತ-ಗಾತಿ
ಸಂಗೀ-ತದ
ಸಂಗೀ-ತ-ವನ್ನು
ಸಂಗ್ರ-ಹ-ವಾಗಿ
ಸಂಗ್ರಹಿ
ಸಂಗ್ರ-ಹಿಸಿ
ಸಂಗ್ರ-ಹಿ-ಸಿ-ಕೊಂಡು
ಸಂಗ್ರ-ಹಿ-ಸಿ-ಟ್ಟು-ಕೊಂ-ಡಿದ್ದ
ಸಂಗ್ರ-ಹಿ-ಸಿ-ಡುವ
ಸಂಗ್ರ-ಹಿ-ಸಿದ
ಸಂಗ್ರ-ಹಿ-ಸಿ-ದಾಗ
ಸಂಗ್ರ-ಹಿ-ಸಿ-ದ್ದೇನೆ
ಸಂಗ್ರ-ಹಿ-ಸು-ವಂತೆ
ಸಂಗ್ರ-ಹಿ-ಸು-ವುದು
ಸಂಗ್ರ-ಹಿ-ಸು-ವು-ದೇಕೆ
ಸಂಚ-ಕಾರ
ಸಂಚನ್ನು
ಸಂಚರಿ
ಸಂಚ-ರಿ-ಸ-ಬಲ್ಲ
ಸಂಚ-ರಿ-ಸ-ಬ-ಹುದು
ಸಂಚ-ರಿ-ಸ-ಲಾ-ರ-ದಂತೆ
ಸಂಚ-ರಿಸಿ
ಸಂಚ-ರಿ-ಸಿತು
ಸಂಚ-ರಿಸು
ಸಂಚ-ರಿ-ಸುತ್ತ
ಸಂಚ-ರಿ-ಸುತ್ತಾ
ಸಂಚ-ರಿ-ಸು-ತ್ತಿದ್ದ
ಸಂಚ-ರಿ-ಸು-ತ್ತಿ-ದ್ದನು
ಸಂಚ-ರಿ-ಸು-ತ್ತಿ-ದ್ದಾನೆ
ಸಂಚ-ರಿ-ಸು-ತ್ತಿದ್ದು
ಸಂಚ-ರಿ-ಸು-ತ್ತಿ-ದ್ದೇನೆ
ಸಂಚ-ರಿ-ಸು-ತ್ತಿ-ರ-ಬ-ಹುದೆ
ಸಂಚ-ರಿ-ಸು-ತ್ತಿ-ರು-ವನು
ಸಂಚ-ರಿ-ಸು-ತ್ತಿ-ರು-ವಾಗ
ಸಂಚ-ರಿ-ಸು-ತ್ತಿ-ರುವೆ
ಸಂಚ-ರಿ-ಸು-ತ್ತಿ-ರು-ವೆನು
ಸಂಚ-ರಿ-ಸುವ
ಸಂಚ-ರಿ-ಸು-ವಂ-ತಾ-ಗ-ಬೇಕು
ಸಂಚ-ರಿ-ಸು-ವಂ-ತಿ-ರ-ಲಿಲ್ಲ
ಸಂಚ-ರಿ-ಸು-ವಾಗ
ಸಂಚಾರ
ಸಂಚಾ-ರಕ್ಕೆ
ಸಂಚಾ-ರ-ಮಾ-ಡುತ್ತಾ
ಸಂಚಾ-ರ-ವಿ-ಲ್ಲ-ದು-ದ-ರಿಂದ
ಸಂಚಿನ
ಸಂಚು
ಸಂಜೀ-ವಿನೀ
ಸಂಜೆ
ಸಂಜೆಯ
ಸಂಜೆ-ಯ-ವ-ರೆಗೆ
ಸಂಜೆ-ಯ-ವೇ-ಳೆಗೆ
ಸಂಜೆ-ಯಾ-ಗಿತ್ತು
ಸಂಜೆ-ಯಾ-ಗು-ತ್ತಲೆ
ಸಂಜೆ-ಯಾ-ಗು-ತ್ತಿ-ದ್ದಂ-ತೆಯೇ
ಸಂತ
ಸಂತ-ತಿಯ
ಸಂತ-ತಿ-ಯನ್ನು
ಸಂತ-ತಿಯೆ
ಸಂತ-ತಿಯೇ
ಸಂತ-ದ-ರ್-ಅನ
ಸಂತ-ಯಿ-ಸು-ತ್ತಿ-ರಲು
ಸಂತರ
ಸಂತ-ಸ-ಗೊಂಡ
ಸಂತ-ಸ-ಗೊಂ-ಡಿತು
ಸಂತ-ಸ-ಗೊ-ಳಿ-ಸಿದೆ
ಸಂತ-ಸ-ಪ-ಡು-ತ್ತವೆ
ಸಂತಾನ
ಸಂತಾ-ನ-ಕ್ಕಾಗಿ
ಸಂತಾ-ನ-ಪ-ರಂ-ಪರೆ
ಸಂತಾ-ನ-ವನ್ನು
ಸಂತಾ-ನ-ವೃ-ದ್ಧಿ-ಯಾದು
ಸಂತಾ-ನ-ಸೌ-ಭಾ-ಗ್ಯ-ದಿಂದ
ಸಂತಾ-ನಾ-ಪೇ-ಕ್ಷೆ-ಯಿಂದ
ಸಂತುಷ್ಟ
ಸಂತು-ಷ್ಟ-ಗೊಂಡ
ಸಂತು-ಷ್ಟ-ನಾಗಿ
ಸಂತು-ಷ್ಟ-ನಾ-ಗಿ-ದ್ದೇನೆ
ಸಂತು-ಷ್ಟ-ನಾದ
ಸಂತು-ಷ್ಟ-ರಾ-ಗಿ-ರು-ವ-ರೆಂ-ಬು-ದನ್ನು
ಸಂತು-ಷ್ಟ-ರಾದ
ಸಂತು-ಷ್ಟಿ-ಗೊಂಡ
ಸಂತು-ಷ್ಟಿ-ಗೊಂ-ಡವು
ಸಂತೆ
ಸಂತೈ-ಸಿ-ದನು
ಸಂತೋಷ
ಸಂತೋ-ಷ-ಕ್ಕಾಗಿ
ಸಂತೋ-ಷ-ಕ್ಕಿಂ-ತಲೂ
ಸಂತೋ-ಷಕ್ಕೆ
ಸಂತೋ-ಷ-ಗಳನ್ನು
ಸಂತೋ-ಷ-ಗಳಿಂದ
ಸಂತೋ-ಷ-ಗೊಂಡ
ಸಂತೋ-ಷ-ಗೊಂ-ಡರು
ಸಂತೋ-ಷ-ಗೊಂ-ಡರೆ
ಸಂತೋ-ಷ-ಗೊಂ-ಡವು
ಸಂತೋ-ಷ-ಗೊಂ-ಡಿ-ರು-ವಾಗ
ಸಂತೋ-ಷ-ಗೊಂಡು
ಸಂತೋ-ಷ-ಗೊ-ಳಿ-ಸಿ-ದನು
ಸಂತೋ-ಷ-ಗೊ-ಳಿಸು
ಸಂತೋ-ಷ-ಗೊ-ಳ್ಳು-ತ್ತಿದ್ದ
ಸಂತೋ-ಷ-ಚಿ-ತ್ತ-ದ-ಲ್ಲಿದ್ದ
ಸಂತೋ-ಷದ
ಸಂತೋ-ಷ-ದಿಂದ
ಸಂತೋ-ಷ-ಪ-ಟ್ಟನು
ಸಂತೋ-ಷ-ಪ-ಟ್ಟರು
ಸಂತೋ-ಷ-ಪಟ್ಟು
ಸಂತೋ-ಷ-ಪ-ಡ-ಬಾ-ರ-ದೆ-ಎಂಬ
ಸಂತೋ-ಷ-ಪ-ಡಲಿ
ಸಂತೋ-ಷ-ಪ-ಡಿ-ಸಲಿ
ಸಂತೋ-ಷ-ಪ-ಡಿ-ಸಲು
ಸಂತೋ-ಷ-ಪ-ಡಿ-ಸಿ-ದನು
ಸಂತೋ-ಷ-ಪ-ಡಿ-ಸಿ-ದರು
ಸಂತೋ-ಷ-ಪ-ಡಿ-ಸುತ್ತಾ
ಸಂತೋ-ಷ-ಪ-ಡಿ-ಸು-ತ್ತಾನೆ
ಸಂತೋ-ಷ-ಪ-ಡಿ-ಸು-ತ್ತಿ-ದ್ದರು
ಸಂತೋ-ಷ-ಪ-ಡಿ-ಸು-ವನು
ಸಂತೋ-ಷ-ಪ-ಡಿ-ಸು-ವ-ವ-ನಂತೆ
ಸಂತೋ-ಷ-ಪ-ಡಿ-ಸು-ವ-ವನು
ಸಂತೋ-ಷ-ಪ-ಡಿ-ಸು-ವು-ದ-ಕ್ಕಾಗಿ
ಸಂತೋ-ಷ-ಪ-ಡು-ತ್ತಾರೆ
ಸಂತೋ-ಷ-ಪ-ಡು-ತ್ತಿದೆ
ಸಂತೋ-ಷ-ಪ-ಡು-ತ್ತಿ-ದ್ದರೆ
ಸಂತೋ-ಷ-ಪ-ಡು-ವು-ದಕ್ಕೆ
ಸಂತೋ-ಷ-ಭಾ-ವ-ದ-ಲ್ಲಿ-ರು-ವಾಗ
ಸಂತೋ-ಷ-ವ-ನ್ನುಂ-ಟು-ಮಾ-ಡಿತು
ಸಂತೋ-ಷ-ವ-ನ್ನುಂ-ಟು-ಮಾ-ಡಿದ
ಸಂತೋ-ಷ-ವಾಗಿ
ಸಂತೋ-ಷ-ವಾ-ಗಿತ್ತು
ಸಂತೋ-ಷ-ವಾ-ಗಿದೆ
ಸಂತೋ-ಷ-ವಾ-ಗು-ತ್ತದೆ
ಸಂತೋ-ಷ-ವಾ-ಗು-ವಂ-ತಹ
ಸಂತೋ-ಷ-ವಾ-ಗು-ವಂತೆ
ಸಂತೋ-ಷ-ವಾ-ಗು-ವಷ್ಟೇ
ಸಂತೋ-ಷ-ವಾ-ಯಿ-ತಾ-ದರೂ
ಸಂತೋ-ಷ-ವಾ-ಯಿತು
ಸಂತೋ-ಷ-ವುಂ-ಟು-ಮಾ-ಡ-ಬೇ-ಕೆಂದು
ಸಂತೋ-ಷಾ-ತಿ-ಶ-ಯ-ದಿಂದ
ಸಂತೋ-ಷಿ-ತ-ರಾ-ದರು
ಸಂತೋ-ಷಿ-ಸಿದ
ಸಂತೋ-ಷಿ-ಸು-ತ್ತಿ-ರಲು
ಸಂದರ್ಭ
ಸಂದ-ರ್ಭ-ಕೂಡ
ಸಂದ-ರ್ಭ-ಕ್ಕಾಗಿ
ಸಂದ-ರ್ಭಕ್ಕೆ
ಸಂದ-ರ್ಭ-ವನ್ನು
ಸಂದ-ರ್ಶನ
ಸಂದ-ರ್ಶಿಸಿ
ಸಂದ-ರ್ಶಿ-ಸೋಣ
ಸಂದವು
ಸಂದೇ-ಶ-ಗಳು
ಸಂದೇ-ಶ-ವನ್ನು
ಸಂದೇಹ
ಸಂದೇ-ಹಕ್ಕೆ
ಸಂದೇ-ಹ-ಗಳನ್ನು
ಸಂದೇ-ಹ-ಗಳನ್ನೂ
ಸಂದೇ-ಹ-ನಿ-ವಾ-ರ-ಣೆ-ಗಾಗಿ
ಸಂದೇ-ಹ-ಪ-ಡ-ಬೇಡ
ಸಂದೇ-ಹ-ಪ-ಡು-ತ್ತಿ-ದ್ದಾನೆ
ಸಂದೇ-ಹ-ವನ್ನು
ಸಂದೇ-ಹ-ವಿ-ದ್ದರೆ
ಸಂದೇ-ಹ-ವಿ-ಲ್ಲ-ವಾ-ಯಿತು
ಸಂದೇ-ಹವೂ
ಸಂದೇ-ಹ-ವೇಕೆ
ಸಂಧ
ಸಂಧಾದಿ
ಸಂಧಾ-ನ-ದಿಂದ
ಸಂಧಿ
ಸಂಧಿ-ಕಾಲ
ಸಂಧಿ-ಗಾಗಿ
ಸಂಧಿಯ
ಸಂಧಿ-ಯನ್ನು
ಸಂಧಿಯೆ
ಸಂಧಿಸಿ
ಸಂಧಿ-ಸಿತು
ಸಂಧಿ-ಸಿ-ದರು
ಸಂಧೇ-ಯ-ಮ-ಸ್ಮಿನ್
ಸಂಧ್ಯಾ-ಕ-ರ್ಮ-ದಲ್ಲಿ
ಸಂಧ್ಯಾ-ಕಾ-ಲ-ಗ-ಳ-ಲ್ಲಿಯೂ
ಸಂಧ್ಯಾ-ಕಾ-ಲ-ದಲ್ಲಿ
ಸಂಧ್ಯಾ-ಕಾ-ಲವು
ಸಂಧ್ಯಾ-ದಿ-ಗಳು
ಸಂಧ್ಯಾ-ದೇ-ವಿಯು
ಸಂಧ್ಯಾ-ವಂ-ದನೆ
ಸಂಧ್ಯಾ-ವಂ-ದ-ನೆಗೆ
ಸಂಧ್ಯಾ-ಸೃ-ಷ್ಟಿಗೆ
ಸಂಧ್ಯೆ
ಸಂಧ್ಯೆ-ಗಳನ್ನು
ಸಂಧ್ಯೆ-ಗ-ಳಾ-ಗಲಿ
ಸಂಧ್ಯೆ-ಗಳೇ
ಸಂಧ್ಯೆಯ
ಸಂಧ್ಯೆ-ಯನ್ನು
ಸಂಧ್ಯೆ-ಯಾಗಿ
ಸಂನ್ಯಾಸ
ಸಂನ್ಯಾ-ಸ-ವನ್ನು
ಸಂನ್ಯಾ-ಸ-ವೆಂದು
ಸಂನ್ಯಾಸಿ
ಸಂನ್ಯಾ-ಸಿ-ಗ-ಳಾ-ದರು
ಸಂನ್ಯಾ-ಸಿಯ
ಸಂನ್ಯಾ-ಸಿ-ಯನ್ನು
ಸಂನ್ಯಾ-ಸಿ-ಯಾ-ಗ-ತ-ಕ್ಕ-ವನು
ಸಂನ್ಯಾ-ಸಿ-ಯಾಗಿ
ಸಂನ್ಯಾ-ಸಿ-ಯೆಂದೆ
ಸಂಪ-ಗೆ-ಯಂತೆ
ಸಂಪ-ತ್ತನ್ನು
ಸಂಪ-ತ್ತ-ನ್ನೆಲ್ಲ
ಸಂಪ-ತ್ತನ್ನೊ
ಸಂಪ-ತ್ತಿ-ನಿಂದ
ಸಂಪತ್ತು
ಸಂಪ-ತ್ತು-ಗಳನ್ನು
ಸಂಪ-ತ್ಪ್ರ-ದ-ವಾ-ಗಿದೆ
ಸಂಪ-ನ್ನ-ನಾಗಿ
ಸಂಪ-ನ್ನ-ನಾದ
ಸಂಪ-ರ್ಕ-ವಾ-ದ-ಮೇಲೆ
ಸಂಪ-ರ್ಕ-ವಿಲ್ಲ
ಸಂಪ-ರ್ಕವೂ
ಸಂಪಾ
ಸಂಪಾ-ದ-ಕರು
ಸಂಪಾ-ದಿಸಿ
ಸಂಪಾ-ದಿ-ಸಿ-ಕೊಂ-ಡಿರು
ಸಂಪಾ-ದಿ-ಸಿ-ಕೊಂಡು
ಸಂಪಾ-ದಿ-ಸಿ-ಕೊ-ಟ್ಟನು
ಸಂಪಾ-ದಿ-ಸಿ-ಕೊ-ಳ್ಳ-ದಂ-ತೆಯೂ
ಸಂಪಾ-ದಿ-ಸಿದ
ಸಂಪಾ-ದಿ-ಸಿ-ದ-ನೆಂದು
ಸಂಪಾ-ದಿ-ಸಿ-ದು-ದ-ಲ್ಲದೆ
ಸಂಪಾ-ದಿ-ಸಿದೆ
ಸಂಪಾ-ದಿ-ಸಿ-ರುವ
ಸಂಪಾ-ದಿ-ಸು-ತ್ತಾನೆ
ಸಂಪಾ-ದಿ-ಸು-ವುದೇ
ಸಂಪಿಗೆ
ಸಂಪೂರ್ಣ
ಸಂಪೂ-ರ್ಣ-ವಾಗಿ
ಸಂಪ್ರ-ತ್ಯ-ಪಾ-ಸ್ತ-ಕ-ಮ-ಲ-ಶ್ರಿಯ
ಸಂಪ್ರ-ದಾ-ಯ-ಗಳೂ
ಸಂಪ್ರ-ದಾ-ಯ-ವಾ-ದಿ-ಗಳ
ಸಂಪ್ರ-ಮೋ-ಹಾತ್
ಸಂಬಂಧ
ಸಂಬಂ-ಧ-ದಿಂದ
ಸಂಬಂ-ಧ-ದೊ-ಡನೆ
ಸಂಬಂ-ಧ-ಪ-ಟ್ಟುವೆ
ಸಂಬಂ-ಧ-ಪ-ಟ್ಟುವೇ
ಸಂಬಂ-ಧ-ವನ್ನು
ಸಂಬಂ-ಧ-ವಾಗಿ
ಸಂಬಂ-ಧ-ವಾ-ದೊ-ಡ-ನೆಯೆ
ಸಂಬಂ-ಧ-ವಿ-ರು-ವ-ವ-ರೆಗೂ
ಸಂಬಂ-ಧ-ವಿಲ್ಲ
ಸಂಬಂ-ಧ-ವಿ-ಲ್ಲದ
ಸಂಬಂ-ಧ-ವಿ-ಲ್ಲ-ದಾಗ
ಸಂಬಂ-ಧವೂ
ಸಂಬಂ-ಧ-ವೇನೂ
ಸಂಬಂಧಿ
ಸಂಬಂ-ಧಿ-ಗಳು
ಸಂಬಂ-ಧಿಸಿ
ಸಂಬಂ-ಧಿ-ಸಿದ
ಸಂಬಂ-ಧಿ-ಸಿವೆ
ಸಂಭ-ವ-ವಿದೆ
ಸಂಭವಿ
ಸಂಭ-ವಿ-ಸುವ
ಸಂಭಾ-ಷಣೆ
ಸಂಭಾ-ಷ-ಣೆ-ಯನ್ನು
ಸಂಭಾ-ಷಿ-ಸು-ತ್ತಿದ್ದ
ಸಂಭೂ-ತಿ-ಯಲ್ಲಿ
ಸಂಭ್ರಮ
ಸಂಭ್ರ-ಮ-ಗಳನ್ನು
ಸಂಭ್ರ-ಮ-ಗಳಿಂದ
ಸಂಭ್ರ-ಮ-ಗ-ಳಿಗೆ
ಸಂಭ್ರ-ಮ-ದಿಂದ
ಸಂಭ್ರ-ಮ-ವನ್ನು
ಸಂಯ-ಮ-ನಿಗೆ
ಸಂರ-ಕ್ಷ-ಕನು
ಸಂರ-ಕ್ಷ-ಕನೂ
ಸಂರ-ಕ್ಷಿ-ಸಲಿ
ಸಂರ-ಕ್ಷಿ-ಸ-ಲಿ-ಎಂ-ದನು
ಸಂರ-ಕ್ಷಿಸು
ಸಂರ-ಕ್ಷಿ-ಸು-ವಂತೆ
ಸಂರ-ಕ್ಷಿ-ಸು-ವು-ದ-ಕ್ಕಾಗಿ
ಸಂವ-ತ್ಸ-ರ-ನೆಂ-ಬು-ವನು
ಸಂವ-ರ್ತ-ಕ-ವೆಂಬ
ಸಂವಾ-ದ-ಆದಿ
ಸಂವಾ-ದ-ಕ-ರ್ದಮ
ಸಂಶಯ
ಸಂಶ-ಯ-ದಿಂದ
ಸಂಶ-ಯ-ವಿ-ಲ್ಲ-ಎಂಬ
ಸಂಶ-ಯವೂ
ಸಂಶ-ಯ-ವೆಲ್ಲ
ಸಂಸಾರ
ಸಂಸಾ-ರಕ್ಕೂ
ಸಂಸಾ-ರಕ್ಕೆ
ಸಂಸಾ-ರ-ಜೀ-ವನ
ಸಂಸಾ-ರ-ಜೀ-ವ-ನ-ದಲ್ಲಿ
ಸಂಸಾ-ರ-ಜೀ-ವ-ನ-ವನ್ನು
ಸಂಸಾ-ರ-ತಾ-ಪ-ಗಳೂ
ಸಂಸಾ-ರ-ತಾ-ಪ-ತ್ರ-ಯದ
ಸಂಸಾ-ರದ
ಸಂಸಾ-ರ-ದಲ್ಲಿ
ಸಂಸಾ-ರ-ದ-ಲ್ಲಿದ್ದು
ಸಂಸಾ-ರ-ದ-ಲ್ಲಿ-ದ್ದು-ಕೊಂಡು
ಸಂಸಾ-ರ-ದ-ವ-ರ-ನ್ನೆಲ್ಲ
ಸಂಸಾ-ರ-ದಿಂದ
ಸಂಸಾ-ರ-ದೊ-ಡನೆ
ಸಂಸಾ-ರ-ಬಂ-ಧನ
ಸಂಸಾ-ರ-ಬಂ-ಧ-ನಕ್ಕೆ
ಸಂಸಾ-ರ-ಬಂ-ಧ-ನ-ದ-ಲ್ಲಿಯೇ
ಸಂಸಾ-ರ-ಬಂ-ಧ-ನ-ದಿಂದ
ಸಂಸಾ-ರ-ಬಂ-ಧ-ನ-ದಿಂ-ದಲೂ
ಸಂಸಾ-ರ-ಭಯ
ಸಂಸಾ-ರ-ಭ-ಯ-ವಿಲ್ಲ
ಸಂಸಾ-ರ-ಮೋ-ಹ-ವನ್ನು
ಸಂಸಾ-ರ-ಯಾ-ತ್ರೆ-ಯನ್ನು
ಸಂಸಾ-ರ-ರೂ-ಪದ
ಸಂಸಾ-ರ-ವನ್ನು
ಸಂಸಾ-ರವು
ಸಂಸಾ-ರವೂ
ಸಂಸಾ-ರ-ವೆಂಬ
ಸಂಸಾ-ರ-ವೆಲ್ಲ
ಸಂಸಾ-ರವೇ
ಸಂಸಾ-ರ-ಸಾ-ಗ-ರ-ದಲ್ಲಿ
ಸಂಸಾ-ರ-ಸಾ-ಗ-ರ-ವನ್ನು
ಸಂಸಾ-ರ-ಸಾರ
ಸಂಸಾ-ರ-ಸು-ಖಕ್ಕೆ
ಸಂಸಾ-ರ-ಸು-ಖದ
ಸಂಸಾರಿ
ಸಂಸಾ-ರಿ-ಗಳ
ಸಂಸಾ-ರಿ-ಗಳನ್ನು
ಸಂಸಾ-ರಿ-ಗಳಲ್ಲಿ
ಸಂಸಾ-ರಿ-ಗ-ಳಾಗಿ
ಸಂಸಾ-ರಿ-ಗ-ಳಾ-ಗಿದ್ದೂ
ಸಂಸಾ-ರಿ-ಗ-ಳಾ-ದ-ವರು
ಸಂಸಾ-ರಿ-ಗ-ಳಿಗೆ
ಸಂಸಾ-ರಿ-ಗ-ಳೆಂ-ಬು-ವರು
ಸಂಸಾ-ರಿ-ಯಂತೆ
ಸಂಸಾ-ರಿ-ಯಾಗ
ಸಂಸಾ-ರಿ-ಯಾಗಿ
ಸಂಸಾ-ರಿ-ಯಾ-ಗಿದ್ದ
ಸಂಸಾ-ರಿ-ಯಾ-ಗು-ತ್ತಾನೆ
ಸಂಸಾ-ರಿ-ಯಾ-ದ-ವನು
ಸಂಸಾ-ರೋ-ಯ-ಮ-ತೀವ
ಸಂಸ್ಕಾ-ರ-ಗ-ಳೆಲ್ಲ
ಸಂಸ್ಕಾ-ರ-ವನ್ನು
ಸಂಸ್ಕೃ-ತದ
ಸಂಸ್ಕೃ-ತಿ-ಯನ್ನು
ಸಂಸ್ಥಾ-ಪ-ನೆ-ಯನ್ನು
ಸಂಸ್ಥೆ
ಸಂಹ-ರಿ-ಸ-ಬಾ-ರದೆ
ಸಂಹ-ರಿಸಿ
ಸಂಹ-ರಿ-ಸಿದ
ಸಂಹ-ರಿ-ಸಿ-ದನು
ಸಂಹ-ರಿ-ಸು-ತ್ತಿ-ರಲು
ಸಂಹ-ರಿ-ಸು-ತ್ತೇನೆ
ಸಂಹ-ರಿ-ಸು-ವ-ನಾ-ದರೂ
ಸಂಹಾರ
ಸಂಹಾ-ರ-ಇದು
ಸಂಹಾ-ರಕ್ಕೆ
ಸಂಹಾ-ರ-ವಾ-ಗು-ತ್ತಿದೆ
ಸಂಹಾ-ರ-ವಾ-ದ-ಮೇಲೆ
ಸಂಹ್ಲಾದ
ಸಕಲ
ಸಕ-ಲ-ಗುಣ
ಸಕ-ಲ-ಧ-ರ್ಮ-ಗಳನ್ನೂ
ಸಕ-ಲ-ಪ್ರಾ-ಣಿ-ವ-ರ್ಗಕ್ಕೂ
ಸಕ-ಲ-ರನ್ನೂ
ಸಕ-ಲವೂ
ಸಕ-ಲ-ಶಾ-ಸ್ತ್ರ-ಗಳ
ಸಕ-ಲಾ-ಗ-ಮ-ಗೀ-ತ-ಗುಣಂ
ಸಕ-ಲಾ-ಲಂ-ಕಾ-ರ-ಭೂ-ಷಿ-ತ-ವಾಗಿ
ಸಕ-ಲಾ-ವ-ನಿ-ಬಿಂ-ಬ-ಧರಂ
ಸಕ-ಲೇಂ
ಸಕ-ಲೋ-ಪ-ಚಾರ
ಸಕಾಮ
ಸಕಾರಣವೇ
ಸಕಾ-ಲಕ್ಕೆ
ಸಕೃ-ದ-ದನ
ಸಕೃ-ದ-ಧ-ರ-ಸು-ಧಾಂ
ಸಕೃ-ದೇ-ತ-ತ್ತ-ನ್ನ-ಖ-ಸ್ಪ-ರ್ಶ-ತೀವ್ರ
ಸಕ್ಕರೆ
ಸಕ್ಕ-ರೆ-ಯಂ-ತಹ
ಸಕ್ಕ-ರೆ-ಯಂತೆ
ಸಕ್ತ-ವಾ-ದರೆ
ಸಖಿ
ಸಖಿ-ಯ-ರಲ್ಲಿ
ಸಖಿ-ಯರು
ಸಖಿ-ಯರೆ
ಸಖಿ-ಯ-ರೊ-ಡನೆ
ಸಖೀ-ಜ-ನ-ರೊ-ಡನೆ
ಸಖೀ-ಭಾ-ವ-ದಾ-ಸಿ-ಭಾವ
ಸಗರ
ಸಗ-ರ-ಚ-ಕ್ರ-ವರ್ತಿ
ಸಗ-ರನ
ಸಗ-ರ-ನಿಗೆ
ಸಗ-ರನು
ಸಗ-ರ-ಪು-ತ್ರರ
ಸಗುಣ
ಸಗು-ಣ-ಬ್ರ-ಹ್ಮ-ನಲ್ಲಿ
ಸಚ್ಚಿ-ದಾ-ನಂದ
ಸಚ್ಚಿ-ದಾ-ನಂ-ದನ
ಸಚ್ಚಿ-ದಾ-ನಂ-ದ-ನಲ್ಲಿ
ಸಚ್ಚಿ-ದಾ-ನಂ-ದ-ಮೂ-ರ್ತಿ-ಯನ್ನು
ಸಚ್ಚಿ-ದಾ-ನಂ-ದ-ಶಕ್ತಿ
ಸಚ್ಚಿ-ದಾ-ನಂ-ದ-ಸ್ವ-ರೂ-ಪ-ನೆನ್ನು
ಸಜ್ಜನ
ಸಜ್ಜ-ನ-ಇ-ವ-ರನ್ನು
ಸಜ್ಜ-ನ-ನಾಗಿ
ಸಜ್ಜ-ನ-ನಾದ
ಸಜ್ಜ-ನರ
ಸಜ್ಜ-ನ-ರನ್ನು
ಸಜ್ಜ-ನ-ರಾ-ಗು-ತ್ತಾರೆ
ಸಜ್ಜ-ನ-ರಾದ
ಸಜ್ಜ-ನ-ರಾ-ದ-ವರು
ಸಜ್ಜ-ನ-ರಿಗೂ
ಸಜ್ಜ-ನ-ರಿಗೆ
ಸಜ್ಜ-ನರು
ಸಜ್ಜಾಗಿ
ಸಜ್ಜಾ-ಗಿತ್ತು
ಸಜ್ಜಾ-ಗಿದ್ದ
ಸಜ್ಞನ
ಸಟೆ-ಯಾ-ಗು-ವುದು
ಸಡ-ಗರ
ಸಡ-ಗ-ರ-ಕ್ಕಂತೂ
ಸಡ-ಗ-ರ-ಗಳಿಂದ
ಸಡ-ಗ-ರ-ದ-ಲ್ಲಿದ್ದ
ಸಡ-ಗ-ರ-ದಿಂದ
ಸಡ-ಗ-ರವೋ
ಸಡ-ಗ-ರಿ-ಸು-ವ-ಷ್ಟ-ರಲ್ಲಿ
ಸಡ-ಲಿತು
ಸಡ-ಲಿ-ದಂತಾ
ಸಣ್ಣ
ಸಣ್ಣ-ಕೀ-ಟ-ದ-ವ-ರೆಗೆ
ಸಣ್ಣ-ಗಾ-ಗಿ-ಹೋದ
ಸಣ್ಣ-ತ-ಪ್ಪಿಗೆ
ಸಣ್ಣ-ದೊಂದು
ಸತ-ತ-ಮು-ರಸಿ
ಸತಾಂ-ಪತಿ
ಸತಿ
ಸತಿಗೆ
ಸತಿ-ಪ-ತಿ-ಗ-ಳಾ-ದರು
ಸತಿಯ
ಸತೀ
ಸತೀ-ದೇವಿ
ಸತೀ-ದೇ-ವಿಗೆ
ಸತೀ-ದೇ-ವಿಯು
ಸತೀ-ಮಣಿ
ಸತೀ-ಮ-ಣಿ-ಯಾದ
ಸತ್
ಸತ್ಕ-ಥಾಂ
ಸತ್ಕ-ರಿಸಿ
ಸತ್ಕ-ರಿ-ಸಿದ
ಸತ್ಕ-ರಿ-ಸಿ-ದನು
ಸತ್ಕ-ರಿ-ಸಿ-ದರು
ಸತ್ಕ-ರಿ-ಸು-ತ್ತಿ-ದ್ದರು
ಸತ್ಕ-ರಿ-ಸು-ತ್ತಿ-ರಲು
ಸತ್ಕ-ರಿ-ಸುವ
ಸತ್ಕ-ರ್ಮಕ್ಕೆ
ಸತ್ಕ-ರ್ಮ-ಗಳಿಂದ
ಸತ್ಕ-ರ್ಮ-ದ-ಲ್ಲಿಯೂ
ಸತ್ಕ-ರ್ಮ-ದಿಂದ
ಸತ್ಕ-ರ್ಮ-ವನ್ನು
ಸತ್ಕಾ-ರ್ಯ-ದಲ್ಲಿ
ಸತ್ಕೃ-ತ-ನಾಗಿ
ಸತ್ತ
ಸತ್ತಂ-ತಾ-ಯಿತು
ಸತ್ತಂ-ತೆಯೆ
ಸತ್ತ-ನೆಂಬ
ಸತ್ತ-ಮೇಲೆ
ಸತ್ತರೆ
ಸತ್ತ-ರೆಂಬ
ಸತ್ತ-ರೇನು
ಸತ್ತಳು
ಸತ್ತ-ವನ
ಸತ್ತ-ವ-ನಂ-ತಾ-ಗಿದ್ದ
ಸತ್ತ-ವ-ನಂ-ತೆಯೆ
ಸತ್ತ-ವ-ನ-ಲ್ಲಿದ್ದ
ಸತ್ತ-ವನು
ಸತ್ತ-ವರ
ಸತ್ತ-ವ-ರನ್ನು
ಸತ್ತ-ವ-ರಿ-ಗೆಲ್ಲ
ಸತ್ತ-ವರೆಲ್ಲ
ಸತ್ತ-ವ-ಳಂ-ತಾಗಿ
ಸತ್ತ-ವ-ಳಂತೆ
ಸತ್ತಿತು
ಸತ್ತಿ-ತೆಂದು
ಸತ್ತು
ಸತ್ತು-ದನ್ನು
ಸತ್ತು-ದ-ರಿಂದ
ಸತ್ತು-ದು-ದನ್ನು
ಸತ್ತು-ಬಿತ್ತು
ಸತ್ತು-ಬಿದ್ದ
ಸತ್ತು-ಬಿ-ದ್ದರು
ಸತ್ತು-ಬಿ-ದ್ದವು
ಸತ್ತು-ಬಿ-ದ್ದಿದ್ದ
ಸತ್ತು-ಬಿ-ದ್ದಿ-ರುವ
ಸತ್ತು-ಬಿ-ದ್ದಿ-ರು-ವುದು
ಸತ್ತು-ಹು-ಟ್ಟಿ-ದಂತಾ
ಸತ್ತು-ಹೋ-ಗ-ಬೇಕು
ಸತ್ತು-ಹೋ-ಗಲು
ಸತ್ತು-ಹೋಗಿ
ಸತ್ತು-ಹೋ-ಗಿದೆ
ಸತ್ತು-ಹೋ-ಗು-ತ್ತೇವೆ
ಸತ್ತು-ಹೋ-ಗು-ವನು
ಸತ್ತು-ಹೋ-ಗು-ವರು
ಸತ್ತು-ಹೋ-ಗು-ವಳು
ಸತ್ತು-ಹೋದ
ಸತ್ತು-ಹೋ-ದ-ನಂತೆ
ಸತ್ತು-ಹೋ-ದನು
ಸತ್ತು-ಹೋ-ದ-ಮೇಲೆ
ಸತ್ತು-ಹೋ-ದರು
ಸತ್ತು-ಹೋ-ದಳು
ಸತ್ತು-ಹೋ-ದವು
ಸತ್ತು-ಹೋದೆ
ಸತ್ತು-ಹೋ-ಯಿತು
ಸತ್ತೇ
ಸತ್ತೇ-ಹೋ-ಗಿ-ರ-ಬ-ಹು-ದೆಂಬ
ಸತ್ತ್ವಾಯ
ಸತ್ಪು-ತ್ರನ
ಸತ್ಪು-ತ್ರ-ನನ್ನು
ಸತ್ಪು-ತ್ರ-ರಾ-ಗು-ವಂತೆ
ಸತ್ಪು-ರು-ಷರ
ಸತ್ಯ
ಸತ್ಯಕ
ಸತ್ಯ-ಕ್ಕಾಗಿ
ಸತ್ಯಕ್ಕೂ
ಸತ್ಯ-ಜಿತ್
ಸತ್ಯದ
ಸತ್ಯ-ದೆ-ಡೆಗೆ
ಸತ್ಯ-ನಾದ
ಸತ್ಯ-ಭಾ-ಮ-ಸ-ಹಿ-ತ-ನಾದ
ಸತ್ಯ-ಭಾಮೆ
ಸತ್ಯ-ಭಾ-ಮೆಗೆ
ಸತ್ಯ-ಭಾ-ಮೆ-ಭಾನು
ಸತ್ಯ-ಭಾ-ಮೆಯ
ಸತ್ಯ-ಭಾ-ಮೆ-ಯನ್ನು
ಸತ್ಯ-ಭಾ-ಮೆ-ಯನ್ನೂ
ಸತ್ಯ-ಭಾ-ಮೆ-ಯ-ರನ್ನು
ಸತ್ಯ-ಭಾ-ಮೆಯೆ
ಸತ್ಯ-ಭಾ-ಮೆ-ಯೊ-ಡನೆ
ಸತ್ಯ-ಮಾ-ನೇನ
ಸತ್ಯ-ಲೋಕ
ಸತ್ಯ-ಲೋ-ಕಕ್ಕೆ
ಸತ್ಯ-ಲೋ-ಕದ
ಸತ್ಯ-ಲೋ-ಕವೇ
ಸತ್ಯ-ವಂತ
ಸತ್ಯ-ವಂ-ತ-ನಾ-ಗಿ-ರು-ವುದು
ಸತ್ಯ-ವತಿ
ಸತ್ಯ-ವ-ತಿ-ಗಾಗಿ
ಸತ್ಯ-ವ-ತಿಯ
ಸತ್ಯ-ವ-ತಿ-ಯರ
ಸತ್ಯ-ವ-ತಿ-ಯೆಂಬ
ಸತ್ಯ-ವನ್ನು
ಸತ್ಯ-ವಾ-ಗಿ-ದ್ದಲ್ಲಿ
ಸತ್ಯ-ವಾ-ಗಿ-ರು-ವಷ್ಟೆ
ಸತ್ಯ-ವಾದಿ
ಸತ್ಯ-ವೆಂದೇ
ಸತ್ಯ-ವೆ-ನಿಸಿ
ಸತ್ಯವೇ
ಸತ್ಯ-ವ್ರತ
ಸತ್ಯ-ವ್ರ-ತನ
ಸತ್ಯ-ವ್ರ-ತ-ನನ್ನು
ಸತ್ಯ-ವ್ರ-ತನು
ಸತ್ಯ-ವ್ರ-ತ-ನೆಂಬ
ಸತ್ಯ-ಸಂಧ
ಸತ್ಯ-ಸಂ-ಧ-ನಾದ
ಸತ್ಯ-ಸಂ-ಧ-ರಿಗೆ
ಸತ್ಯ-ಸ್ವ-ರೂ-ಪ-ನಾದ
ಸತ್ಯ-ಸ್ವ-ರೂ-ಪವೂ
ಸತ್ಯಾಂ-ಗ-ಎಂಬ
ಸತ್ಯಾ-ನು-ಸಂ-ಧಾ-ನದ
ಸತ್ಯೆ
ಸತ್ಯೆಯು
ಸತ್ಯೆ-ಯೊ-ಡನೆ
ಸತ್ಯೇ-ನಾ-ಽನೇನ
ಸತ್ರ-ಗಳನ್ನು
ಸತ್ರ-ಗಳು
ಸತ್ರ-ಯಾಗ
ಸತ್ರ-ಯಾ-ಗದ
ಸತ್ರ-ಯಾ-ಗ-ವನ್ನು
ಸತ್ರಾ
ಸತ್ರಾ-ಜಿತ
ಸತ್ರಾ-ಜಿ-ತನ
ಸತ್ರಾ-ಜಿ-ತ-ನನ್ನು
ಸತ್ರಾ-ಜಿ-ತ-ನಿಗೆ
ಸತ್ರಾ-ಜಿ-ತ-ನಿಗೇ
ಸತ್ರಾ-ಜಿ-ತನು
ಸತ್ರಾ-ಜಿ-ತನೂ
ಸತ್ವ
ಸತ್ವ-ಗುಣ
ಸತ್ವ-ಗು-ಣ-ಪ್ರ-ಧಾನ
ಸತ್ವ-ಗು-ಣವು
ಸತ್ವ-ಗು-ಣಾ-ಶ್ರ-ಯನೂ
ಸತ್ವ-ಪ-ರೀಕ್ಷೆ
ಸತ್ವ-ಪರೀಕ್ಷೆ--ಗಾಗಿ
ಸತ್ವ-ಪರೀಕ್ಷೆ-ಯ
ಸತ್ವ-ರ-ಜ-ಸ್ತ-ಮೋ-ಗು-ಣ-ಗ-ಳಿಗೆ
ಸತ್ವ-ರೂ-ಪಿ-ಯಾಗಿ
ಸತ್ವ-ವನ್ನು
ಸತ್ವ-ವನ್ನೂ
ಸತ್ವ-ಸ್ವ-ರೂಪಿ
ಸತ್ವಾದಿ
ಸತ್ಸಂಗ
ಸದ-ಸ-ಚ್ಚ-ಯತ್
ಸದಸಿ
ಸದಾ
ಸದಾ-ಚಾರ
ಸದಾ-ಚಾ-ರ-ಗಳನ್ನು
ಸದಾ-ಚಾ-ರಿ-ಗಳು
ಸದಾ-ಶಿ-ವನ
ಸದು-ತ್ತ-ರ-ವಿ-ತ್ತರು
ಸದೃ-ಶ-ರಾ-ದ-ವರು
ಸದೃ-ಶ-ವಾದ
ಸದೆ-ಬ-ಡಿ-ದರು
ಸದೆ-ಬ-ಡಿದು
ಸದೆ-ಬ-ಡೆ-ಯಿರಿ
ಸದ್ಗತಿ
ಸದ್ಗ-ತಿಗೆ
ಸದ್ಗ-ತಿ-ಯನ್ನು
ಸದ್ಗ-ತಿ-ಯನ್ನೂ
ಸದ್ಗ-ತಿ-ಯಿಲ್ಲ
ಸದ್ಗ-ತಿಯೂ
ಸದ್ಗು-ಣ-ಗಳನ್ನು
ಸದ್ಗು-ಣ-ಗಳು
ಸದ್ದನ್ನು
ಸದ್ಬುದ್ಧಿ
ಸದ್ಭ-ಕ್ತ-ರಾ-ದ-ವರ
ಸದ್ಯಕ್ಕೆ
ಸದ್ಯ-ಸ್ತ-ತ್ಯ-ಜೇ-ಽಸ್ಮಾನ್
ಸದ್ಯೋ
ಸದ್ಯೋ-ಜಾ-ತಾದಿ
ಸದ್ವ-ಸ್ತು-ಲಕ್ಷ್ಯ
ಸಧ್ಯಕ್ಕೆ
ಸಧ್ಯದ
ಸನಂ
ಸನಂ-ದನ
ಸನಂ-ದಾದಿ
ಸನಂ-ದಾ-ದಿ-ಗಳು
ಸನಕ
ಸನ-ಕನೇ
ಸನ-ಕ-ಸ-ನಂ-ದಾದಿ
ಸನ-ಕಾದಿ
ಸನ-ಕಾ-ದಿ-ಗಳನ್ನು
ಸನ-ಕಾ-ದಿ-ಗಳು
ಸನ-ತ್ಕು-ಮಾರ
ಸನ-ತ್ಕು-ಮಾ-ರನ
ಸನ-ತ್ಕು-ಮಾ-ರನು
ಸನ-ತ್ಕು-ಮಾ-ರೋ-ಽವತು
ಸನ-ತ್ಸು-ಜಾತ
ಸನ-ದಲ್ಲಿ
ಸನಾ-ತನ
ಸನಾ-ಥ-ವಾ-ಯಿತು
ಸನ್ನಿ-ಧಿ-ಯನ್ನು
ಸನ್ನಿ-ವೇಶ
ಸನ್ನಿ-ವೇ-ಶ-ಗ-ಳಿಗೆ
ಸನ್ನಿ-ವೇ-ಶ-ಗಳು
ಸನ್ನಿ-ವೇ-ಶ-ದಲ್ಲಿ
ಸನ್ನಿ-ವೇ-ಶ-ವನ್ನು
ಸನ್ನಿ-ವೇ-ಶವೂ
ಸನ್ನಿ-ಹಿ-ತ-ನಾ-ಗಿ-ರು-ವ-ನೆಂದೇ
ಸನ್ನಿ-ಹಿ-ತ-ವಾ-ಗಿ-ರು-ವು-ದಾಗಿ
ಸನ್ನಿ-ಹಿ-ತ-ವಾ-ಗು-ತ್ತಲೇ
ಸನ್ನಿ-ಹಿ-ತ-ವಾ-ದಂತೆ
ಸನ್ನಿ-ಹಿ-ತ-ವಾ-ಯಿತು
ಸನ್ಮಂ-ಗ-ಳ-ಗಳೂ
ಸನ್ಮಾನ
ಸನ್ಮಾ-ನ-ಗ-ಳೊ-ಡನೆ
ಸನ್ಮಾ-ನ-ವನ್ನು
ಸನ್ಮಾ-ನಿ-ತ-ನಾ-ದನು
ಸನ್ಮಾನ್ಯ
ಸನ್ಮಾ-ರ್ಗ-ದಲ್ಲಿ
ಸನ್ಯಾ-ಸಿ-ಯಾಗಿ
ಸನ್ಯಾ-ಸಿ-ವೇ-ಷ-ದಿಂದ
ಸಪ-ತ್ನ್ಯಾಃ
ಸಪದಿ
ಸಪಿ-ತೃ-ಗೇ-ಹಾನ್
ಸಪ್ತ
ಸಪ್ತ-ದ್ವೀ-ಪ-ಗಳನ್ನೂ
ಸಪ್ತ-ದ್ವೀ-ಪ-ಗಳಿಂದ
ಸಪ್ತ-ದ್ವೀ-ಪ-ಗ-ಳಿಗೆ
ಸಪ್ತ-ಧಾ-ತು-ಗಳು
ಸಪ್ತ-ಪುಷಿ
ಸಪ್ತ-ಪು-ಷಿ-ಗಳ
ಸಪ್ತ-ಪು-ಷಿ-ಗ-ಳೊ-ಡನೆ
ಸಪ್ತ-ಪು-ಷಿ-ಮಂ-ಡ-ಲಕ್ಕೂ
ಸಪ್ತ-ಮಾ-ತೃ-ಕೆ-ಯ-ರು-ಮೊ-ದ-ಲಾದ
ಸಪ್ತ-ರ್ಷಿ-ಗ-ಳೊ-ಡನೆ
ಸಪ್ತ-ವಾ-ರಿ-ಧೀನ್
ಸಪ್ತ-ಸ-ಮು-ದ್ರ-ಗಳನ್ನು
ಸಪ್ತ-ಸ-ಮು-ದ್ರ-ಗಳನ್ನೂ
ಸಪ್ತ-ಸ-ಮು-ದ್ರದ
ಸಪ್ಪು-ಳಿ-ಲ್ಲದೆ
ಸಪ್ರ-ಮಾಣ
ಸಫ-ಲ-ಗೊ-ಳಿ-ಸ-ಲೆಂದು
ಸಫ-ಲ-ಗೊ-ಳಿ-ಸಿ-ದಳು
ಸಫ-ಲ-ವಾ-ಗು-ವಂತೆ
ಸಫ-ಲ-ವಾ-ದಂ-ತಾ-ಯಿತು
ಸಫ-ಲ-ವಾ-ದುದೇ
ಸಫ-ಲ-ವಾ-ದುವು
ಸಫ-ಲ-ವಾ-ಯಿ-ತೆಂದು
ಸಫ-ಲ-ವಾ-ಯಿತೋ
ಸಭಾ
ಸಭಾ-ಧ್ಯ-ಕ್ಷ-ರನ್ನು
ಸಭಾ-ಭ-ವ-ನಕ್ಕೆ
ಸಭಾ-ಭ-ವ-ನದ
ಸಭಾ-ಭ-ವ-ನ-ದಲ್ಲಿ
ಸಭಾ-ಭ-ವ-ನ-ವನ್ನು
ಸಭಾ-ಮಂ-ಟಪ
ಸಭಾ-ಮಂ-ಟ-ಪಕ್ಕೆ
ಸಭಾ-ಮಂ-ಟ-ಪ-ದಲ್ಲಿ
ಸಭಾ-ಸ್ಥಾ-ನ-ವನ್ನೂ
ಸಭಿ-ಕರೆ
ಸಭೆ
ಸಭೆ-ಇ-ವು-ಗಳನ್ನು
ಸಭೆಗೆ
ಸಭೆಯ
ಸಭೆ-ಯನ್ನು
ಸಭೆ-ಯಲ್ಲಿ
ಸಭೆ-ಯ-ಲ್ಲಿ-ದ-ವರೆಲ್ಲ
ಸಭೆ-ಯ-ಲ್ಲಿದ್ದ
ಸಭೆ-ಯ-ಲ್ಲಿ-ದ್ದ-ವರೂ
ಸಭೆ-ಯಿಂದ
ಸಮ
ಸಮಂ-ಜ-ಸ-ವಾ-ಗಿ-ರು-ವು-ದೆಂದು
ಸಮ-ಕಾ-ಲ-ದ-ವರೆಲ್ಲ
ಸಮ-ಗ್ರ-ವಾಗಿ
ಸಮ-ಚಿ-ತ್ತ-ನಾಗಿ
ಸಮ-ಚಿ-ತ್ತ-ನೆಂದು
ಸಮತೆ
ಸಮ-ದನ
ಸಮ-ದರ್ಶಿ
ಸಮ-ದೃ-ಷ್ಟಿಗೆ
ಸಮ-ದೃ-ಷ್ಟಿ-ಯಿಂದ
ಸಮ-ನಾಗಿ
ಸಮ-ನಾ-ಗಿದೆ
ಸಮ-ನಾದ
ಸಮ-ನಿ-ಸ-ಕೂ-ಡದು
ಸಮನೆ
ಸಮ-ಪಾಸ್ತ
ಸಮ-ಬ-ಲರು
ಸಮ-ಬುದ್ಧಿ
ಸಮ-ಭಾ-ವ-ದಿಂದ
ಸಮಯ
ಸಮ-ಯ-ಕ್ಕಾ-ಗಿಯೆ
ಸಮ-ಯಕ್ಕೆ
ಸಮ-ಯ-ಗಳಲ್ಲಿ
ಸಮ-ಯ-ದಲ್ಲಿ
ಸಮ-ಯ-ದ-ಲ್ಲಿಯೆ
ಸಮ-ಯ-ವನ್ನು
ಸಮ-ಯ-ವನ್ನೆ
ಸಮ-ಯ-ವನ್ನೇ
ಸಮ-ಯ-ವೆಂದು
ಸಮ-ರ-ಸ-ವಾಗಿ
ಸಮರ್ಥ
ಸಮ-ರ್ಥ-ನಾ-ದ-ಮೇಲೆ
ಸಮ-ರ್ಥ-ರಾದ
ಸಮ-ರ್ಪ-ಕ-ವಾಗ
ಸಮ-ರ್ಪ-ಕ-ವಾಗಿ
ಸಮ-ರ್ಪಿಸಿ
ಸಮ-ರ್ಪಿ-ಸಿದ
ಸಮ-ರ್ಪಿ-ಸಿ-ದಳು
ಸಮ-ವ-ಯ-ಸ್ಕ-ನಾದ
ಸಮ-ವಾ-ಗು-ವು-ದಿಲ್ಲ
ಸಮ-ಸ-ಮ-ವಾಗಿ
ಸಮ-ಸ-ಮ-ವಾ-ಗಿಯೇ
ಸಮಸ್ತ
ಸಮ-ಸ್ತ-ತೇ-ಜಾಃ
ಸಮ-ಸ್ತ-ರಿಂ-ದಲೂ
ಸಮ-ಸ್ತರೂ
ಸಮ-ಸ್ತ-ವನ್ನೂ
ಸಮ-ಸ್ತವೂ
ಸಮ-ಸ್ಥಿತಿ
ಸಮ-ಸ್ಥಿ-ತಿ-ಯ-ಲ್ಲಿದ್ದ
ಸಮ-ಸ್ಥಿ-ತಿ-ಯ-ಲ್ಲಿ-ರು-ವಷ್ಟು
ಸಮಸ್ಯೆ
ಸಮ-ಸ್ಯೆ-ಯಾ-ಗಿತ್ತು
ಸಮಾ
ಸಮಾ-ಗಮ
ಸಮಾ-ಗ-ಮ-ವನ್ನು
ಸಮಾ-ಚಾರ
ಸಮಾ-ಚಾ-ರ-ಗ-ಳೊಂದೂ
ಸಮಾ-ಚಾ-ರ-ವನ್ನು
ಸಮಾ-ಚಾ-ರ-ವ-ನ್ನೆಲ್ಲ
ಸಮಾ-ಚಾ-ರ-ವ-ನ್ನೆಲ್ಲಾ
ಸಮಾ-ಚಾ-ರ-ವೇನು
ಸಮಾಜ
ಸಮಾ-ಜದ
ಸಮಾ-ಜ-ದಲ್ಲಿ
ಸಮಾ-ಜ-ವನ್ನೆ
ಸಮಾ-ಧಾನ
ಸಮಾ-ಧಾ-ನ-ಗೊಂಡ
ಸಮಾ-ಧಾ-ನ-ಗೊ-ಳಿ-ಸಿ-ದ-ಮೇಲೆ
ಸಮಾ-ಧಾ-ನ-ಚಿತ್ತ
ಸಮಾ-ಧಾ-ನದ
ಸಮಾ-ಧಾ-ನ-ಪಡಿ
ಸಮಾ-ಧಾ-ನ-ಪ-ಡಿಸಿ
ಸಮಾ-ಧಾ-ನ-ಪ-ಡಿ-ಸಿ-ದನು
ಸಮಾ-ಧಾ-ನ-ಮಾ-ಡ-ಲೆಂದು
ಸಮಾ-ಧಾ-ನ-ಮಾಡಿ
ಸಮಾ-ಧಾ-ನ-ಮಾ-ಡಿದ
ಸಮಾ-ಧಾ-ನ-ಮಾ-ಡಿ-ದನು
ಸಮಾ-ಧಾ-ನ-ವಾ-ಗ-ಲೆಂದು
ಸಮಾ-ಧಾ-ನ-ವಾ-ದಂ-ತಾ-ಯಿತು
ಸಮಾ-ಧಾ-ನ-ವಿಲ್ಲ
ಸಮಾಧಿ
ಸಮಾ-ಧಿ-ಇ-ವು-ಗಳಿಂದ
ಸಮಾ-ಧಿ-ಗಳಿಂದ
ಸಮಾ-ಧಿ-ಗೇ-ರಿ-ದನು
ಸಮಾ-ಧಿ-ಯ-ಲ್ಲಿ-ದ್ದನು
ಸಮಾ-ಧಿ-ಯಿಂದ
ಸಮಾ-ಧಿ-ಯಿಂ-ದೆದ್ದು
ಸಮಾ-ಧಿ-ಯೋ-ಗ-ದಿಂದ
ಸಮಾ-ಧಿ-ಸ್ಥ-ನಾ-ದನು
ಸಮಾ-ಧಿ-ಸ್ಥಿ-ತಿ-ಯಲ್ಲಿ
ಸಮಾನ
ಸಮಾ-ನತೆ
ಸಮಾ-ನ-ನ-ಲ್ಲವೆ
ಸಮಾ-ನ-ನಾದ
ಸಮಾ-ನ-ನಾ-ದ-ವನು
ಸಮಾ-ನ-ಪ್ರೀ-ತಿ-ಯಿಂದ
ಸಮಾ-ನ-ಬ-ಲ-ಶಾ-ಲಿ-ಗ-ಳಂತೆ
ಸಮಾ-ನ-ರಾದ
ಸಮಾ-ನ-ರಿಲ್ಲ
ಸಮಾ-ನ-ರಿ-ಲ್ಲ-ವೆಂದು
ಸಮಾ-ನ-ವಾದ
ಸಮಾ-ನ-ವಾ-ದುದು
ಸಮಾ-ನ-ವೆಂದು
ಸಮಾ-ಲೋ-ಚಿ-ಸ-ಬ-ಹುದು
ಸಮಿ-ತ್ತು-ಗಳನ್ನೂ
ಸಮೀಪ
ಸಮೀ-ಪಕ್ಕೆ
ಸಮೀ-ಪದ
ಸಮೀ-ಪ-ದಲ್ಲಿ
ಸಮೀ-ಪ-ದ-ಲ್ಲಿದ್ದ
ಸಮೀ-ಪ-ದ-ಲ್ಲಿ-ದ್ದಾರೆ
ಸಮೀ-ಪ-ದ-ಲ್ಲಿಯೆ
ಸಮೀ-ಪ-ದ-ಲ್ಲಿಯೇ
ಸಮೀ-ಪ-ದಲ್ಲೆ
ಸಮೀ-ಪಿ-ಸಿ-ರು-ವು-ದೆಂದು
ಸಮೀ-ಪಿ-ಸು-ತ್ತಲೆ
ಸಮೀ-ಪಿ-ಸು-ತ್ತಿ-ದೆ-ಯೆಂದು
ಸಮುದ್ರ
ಸಮು-ದ್ರ-ಕ್ಕಿಂತ
ಸಮು-ದ್ರಕ್ಕೆ
ಸಮು-ದ್ರ-ಗಳ
ಸಮು-ದ್ರ-ಗಳು
ಸಮು-ದ್ರ-ಗ-ಳೆಲ್ಲ
ಸಮು-ದ್ರ-ತೀ-ರಕ್ಕೆ
ಸಮು-ದ್ರ-ತೀ-ರ-ದಲ್ಲಿ
ಸಮು-ದ್ರದ
ಸಮು-ದ್ರ-ದಂ-ತಾ-ಗ-ಬೇಕು
ಸಮು-ದ್ರ-ದಂತೆ
ಸಮು-ದ್ರ-ದತ್ತ
ಸಮು-ದ್ರ-ದಲ್ಲಿ
ಸಮು-ದ್ರ-ದ-ಲ್ಲಿನ
ಸಮು-ದ್ರ-ದಿಂದ
ಸಮು-ದ್ರ-ಮ-ಥನ
ಸಮು-ದ್ರ-ಮ-ಥ-ನಕ್ಕೆ
ಸಮು-ದ್ರ-ಮ-ಧ್ಯ-ದಲ್ಲಿ
ಸಮು-ದ್ರ-ಮ-ಧ್ಯ-ದಿಂದ
ಸಮು-ದ್ರ-ರಾಜ
ಸಮು-ದ್ರ-ರಾ-ಜನ
ಸಮು-ದ್ರ-ರಾ-ಜ-ನನ್ನು
ಸಮು-ದ್ರ-ರಾ-ಜ-ನಾದ
ಸಮು-ದ್ರ-ರಾ-ಜನು
ಸಮು-ದ್ರ-ರಾ-ಜರು
ಸಮು-ದ್ರ-ವನ್ನು
ಸಮು-ದ್ರ-ವನ್ನೇ
ಸಮು-ದ್ರ-ವಾಗಿ
ಸಮು-ದ್ರ-ವಿದೆ
ಸಮು-ದ್ರವು
ಸಮು-ದ್ರವೂ
ಸಮು-ದ್ರ-ವೆಲ್ಲ
ಸಮು-ದ್ರ-ಸ್ನಾ-ನಕ್ಕೆ
ಸಮೂ-ಲ-ವಾಗಿ
ಸಮೂ-ಹಕ್ಕೆ
ಸಮೂ-ಹದ
ಸಮೂ-ಹ-ದೊ-ಡನೆ
ಸಮೃ-ದ್ಧ-ವಾ-ಗಿ-ದ್ದವು
ಸಮೃ-ದ್ಧ-ವಾಗು
ಸಮೆ-ದು-ಹೋ-ಗು-ವಂತೆ
ಸಮೇ-ತ-ನಾದ
ಸಮ್ಮತ
ಸಮ್ಮ-ತ-ವಾದ
ಸಮ್ಮ-ತವೆ
ಸಮ್ಮ-ತ-ವೆಂದೆ
ಸಮ್ಮ-ತಿ-ಯ-ನ್ನೀ-ಯುತ್ತಾ
ಸಮ್ಮ-ತಿಸಿ
ಸಮ್ಮ-ತಿ-ಸಿ-ದನು
ಸಮ್ಮ-ತಿ-ಸಿ-ದರು
ಸಮ್ಮಿ-ಳ-ನ-ವನ್ನು
ಸರ
ಸರ-ಣಿ-ಯನ್ನು
ಸರ-ಪ-ಣಿಯ
ಸರ-ಪ-ಣಿ-ಯಲ್ಲಿ
ಸರ-ಪ-ಳಿ-ಗಳಿಂದ
ಸರಯೂ
ಸರ-ಯೂ-ನ-ದಿ-ಯಲ್ಲಿ
ಸರಳ
ಸರ-ಳ-ನಾದ
ಸರ-ಳ-ವಾದ
ಸರ-ವಾ-ಯಿತು
ಸರಸ
ಸರಸಂ
ಸರ-ಸ-ವ-ನ್ನಾ-ಡುತ್ತಾ
ಸರ-ಸ-ವಾ-ಡು-ತ್ತಿ-ದ್ದನು
ಸರ-ಸ-ವಾ-ಡು-ತ್ತಿ-ರು-ವೆ-ಯಲ್ಲ
ಸರ-ಸ-ವಾ-ಡು-ತ್ತಿ-ರು-ವೆ-ವೆಂದು
ಸರ-ಸ-ವಾದ
ಸರ-ಸ-ಸ-ಲ್ಲಾಪ
ಸರ-ಸ-ಸ-ಲ್ಲಾ-ಪ-ಇವು
ಸರ-ಸ-ಸ-ಲ್ಲಾ-ಪಕ್ಕೆ
ಸರ-ಸ-ಸ-ಲ್ಲಾ-ಪ-ವಾ-ಡುತ್ತಾ
ಸರ-ಸಿ-ಯಾ-ಗಿತ್ತು
ಸರ-ಸ್ವತಿ
ಸರ-ಸ್ವ-ತಿಯು
ಸರ-ಸ್ವತೀ
ಸರ-ಸ್ವ-ತೀ-ನ-ದಿಯ
ಸರ-ಸ್ವ-ತೀ-ನ-ದಿಯು
ಸರಾ-ಗ-ವಾಗಿ
ಸರಾ-ಗ-ವಾ-ಗು-ವಂತೆ
ಸರಿ
ಸರಿ
ಸರಿ-ಅ-ಧೋ-ಗತಿ
ಸರಿ-ಎಂದು
ಸರಿ-ದಾ-ಡು-ವು-ದಕ್ಕೂ
ಸರಿ-ದೂ-ಗ-ಲಾ-ರದು
ಸರಿ-ದ್ಗಿ-ರಿ-ವ-ನಾ-ದಿ-ಭಿಃ
ಸರಿ-ಪ-ಡಿ-ಸಿ-ಕೊ-ಳ್ಳುತ್ತಾ
ಸರಿ-ಬಾ-ರ-ದೆಂದು
ಸರಿ-ಬೀ-ಳ-ಲಿಲ್ಲ
ಸರಿ-ಯಪ್ಪ
ಸರಿ-ಯಲ್ಲ
ಸರಿ-ಯಷ್ಟೆ
ಸರಿ-ಯಾಗಿ
ಸರಿ-ಯಾದ
ಸರಿ-ಯಾ-ದ-ವರೇ
ಸರಿ-ಯು-ತ್ತಿ-ದ್ದವು
ಸರಿಯೆ
ಸರಿ-ಯೆಂದು
ಸರಿ-ಯೆಂದೇ
ಸರಿ-ಸ-ಬೇ-ಕಾ-ಗು-ತ್ತದೆ
ಸರಿ-ಸಿ-ಕೊಂಡು
ಸರಿ-ಹೋ-ಗು-ತ್ತದೆ
ಸರೀ-ಸೃ-ಪೇಭ್ಯೋ
ಸರೋ
ಸರೋ-ವರ
ಸರೋ-ವ-ರಕ್ಕೆ
ಸರೋ-ವ-ರ-ಗಳ
ಸರೋ-ವ-ರ-ಗಳು
ಸರೋ-ವ-ರ-ಗಳೂ
ಸರೋ-ವ-ರದ
ಸರೋ-ವ-ರ-ದಲ್ಲಿ
ಸರೋ-ವ-ರ-ವ-ನ್ನಾಗಿ
ಸರೋ-ವ-ರ-ವನ್ನು
ಸರೋ-ವ-ರ-ವನ್ನೇ
ಸರೋ-ವ-ರ-ವಾ-ಗ-ಬೇಕು
ಸರೋ-ವ-ರವು
ಸರೋ-ವ-ರ-ವೊಂ-ದರ
ಸರ್ಗ
ಸರ್ಪ
ಸರ್ಪ-ಗಳ
ಸರ್ಪ-ಗ-ಳಂ-ತಿ-ರುವ
ಸರ್ಪ-ಗಳನ್ನು
ಸರ್ಪ-ಗಳಿಂದ
ಸರ್ಪ-ಜ-ನ್ಮ-ವನ್ನು
ಸರ್ಪ-ದ-ಷ್ಟ-ನಾಗಿ
ಸರ್ಪ-ದಷ್ಟೆ
ಸರ್ಪ-ಯಾ-ಗ-ವನ್ನು
ಸರ್ಪ-ರಾ-ಜ-ಇ-ವು-ಗಳ
ಸರ್ಪ-ರಾ-ಜ-ನಾದ
ಸರ್ಪ-ರಾ-ಜನು
ಸರ್ಪ-ವನ್ನು
ಸರ್ಪ-ವೊಂದು
ಸರ್ಪಾ-ಧ-ಮ-ನಾದ
ಸರ್ಪಿ
ಸರ್ವ
ಸರ್ವಂ
ಸರ್ವಕ್ಕೂ
ಸರ್ವಗಃ
ಸರ್ವ-ಗುಣ
ಸರ್ವ-ಗ್ರಾ-ಹ್ಯ-ವೆಂ-ಬು-ದ-ರಲ್ಲಿ
ಸರ್ವ-ಜೀ-ವಿ-ಗಳ
ಸರ್ವಜ್ಞ
ಸರ್ವ-ಜ್ಞ-ತ್ವಾದಿ
ಸರ್ವ-ಜ್ಞ-ನಾಗಿ
ಸರ್ವ-ಜ್ಞ-ನಾದ
ಸರ್ವ-ಜ್ಞ-ನಾ-ದರೂ
ಸರ್ವ-ಜ್ಞನೂ
ಸರ್ವ-ಜ್ಞ-ರೆ-ನಿ-ಸಿ-ಕೊಂಡ
ಸರ್ವಜ್ಞೋ
ಸರ್ವ-ತಂತ್ರ
ಸರ್ವ-ತಂ-ತ್ರ-ಸ್ವ-ತಂತ್ರ
ಸರ್ವ-ತಂ-ತ್ರ-ಸ್ವ-ತಂ-ತ್ರನು
ಸರ್ವತ್ರ
ಸರ್ವ-ನಾ-ಶಕ್ಕೆ
ಸರ್ವ-ಪ್ರಾ-ಣಿ-ಗಳ
ಸರ್ವ-ಪ್ರಾ-ಣಿ-ಗಳೂ
ಸರ್ವ-ಭೂ-ತ-ಗಳ
ಸರ್ವ-ಭೂ-ತ-ಸಮ
ಸರ್ವ-ಭೂ-ತಾ-ನಾ-ಮಂ-ತ-ರ್ಬ-ಹಿ-ರ-ವ-ಸ್ಥಿ-ತಮ್
ಸರ್ವ-ಮಂ-ಗಳೆ
ಸರ್ವ-ಮಯ
ಸರ್ವ-ರ-ಕ್ಷಾಂ
ಸರ್ವ-ವಿ-ಧ-ದ-ಲ್ಲಿಯೂ
ಸರ್ವ-ವ್ಯಾ-ಪಕ
ಸರ್ವ-ವ್ಯಾ-ಪ-ಕನೂ
ಸರ್ವ-ವ್ಯಾ-ಪಿ-ಯಾಗಿ
ಸರ್ವ-ವ್ಯಾ-ಪಿ-ಯಾದ
ಸರ್ವ-ಶಕ್ತ
ಸರ್ವ-ಶ-ಕ್ತ-ನಾ-ಗಿ-ರುವ
ಸರ್ವ-ಶ-ಕ್ತ-ನಾದ
ಸರ್ವ-ಶ-ಕ್ತರು
ಸರ್ವ-ಶ್ರೇಷ್ಠ
ಸರ್ವ-ಶ್ರೇ-ಷ್ಠ-ವಾದ
ಸರ್ವ-ಸಂಗ
ಸರ್ವ-ಸಂ-ಗ-ಪರಿ
ಸರ್ವ-ಸಂ-ಗ-ಪ-ರಿ-ತ್ಯಾಗ
ಸರ್ವ-ಸಂ-ಗ-ವನ್ನು
ಸರ್ವ-ಸ-ತ್ವ-ಗುಣ
ಸರ್ವ-ಸ-ಮಾ-ನ-ಭಾ-ವ-ದಿಂ-ದಿ-ರು-ವನು
ಸರ್ವ-ಸಾಕ್ಷಿ
ಸರ್ವ-ಸು-ಖಕ್ಕೂ
ಸರ್ವಸ್ವ
ಸರ್ವ-ಸ್ವನ್ನು
ಸರ್ವ-ಸ್ವ-ರೂಪ
ಸರ್ವ-ಸ್ವ-ವನ್ನೂ
ಸರ್ವ-ಸ್ವವೂ
ಸರ್ವ-ಸ್ವ-ವೂ-ಎಂದು
ಸರ್ವ-ಸ್ವಾ-ಮ್ಯ-ವನ್ನು
ಸರ್ವಾ
ಸರ್ವಾಂಗ
ಸರ್ವಾಂ-ತರ್
ಸರ್ವಾಂ-ತ-ರ್ಯಾಮಿ
ಸರ್ವಾಂ-ತ-ರ್ಯಾ-ಮಿ-ಯಾಗಿಯೂ
ಸರ್ವಾಂ-ತ-ರ್ಯಾ-ಮಿ-ಯಾದ
ಸರ್ವಾಂ-ತ-ರ್ಯಾ-ಯಾಮಿ
ಸರ್ವಾ-ಣ್ಯೇ-ತಾನಿ
ಸರ್ವಾ-ತ್ಮಕ
ಸರ್ವಾ-ತ್ಮ-ಕ-ನಾ-ಗಿ-ರುವ
ಸರ್ವಾ-ತ್ಮ-ಕ-ನಾದ
ಸರ್ವಾ-ತ್ಮ-ಕ-ವಾ-ದು-ದ-ರಿಂ-ದಲೇ
ಸರ್ವಾ-ತ್ಮ-ನಾ-ಗಿ-ದ್ದರೂ
ಸರ್ವಾ-ತ್ಮ-ನಾದ
ಸರ್ವಾ-ಧಾರ
ಸರ್ವಾ-ಧಿ-ಕಾರ
ಸರ್ವಾ-ಧಿ-ಕಾ-ರ-ಗಳನ್ನು
ಸರ್ವಾ-ಧಿ-ಕಾ-ರಿ-ಯಾ-ದನು
ಸರ್ವಾ-ನು-ಕೂ-ಲೆ-ಯಾ-ಗಿ-ರು-ವಳು
ಸರ್ವಾ-ಪದ್ಭ್ಯೋ
ಸರ್ವಾ-ಲಂ-ಕಾರ
ಸರ್ವಾ-ಲಂ-ಕಾ-ರ-ಭೂ-ಷಿತ
ಸರ್ವಾ-ಲಂ-ಕಾ-ರ-ಭೂ-ಷಿ-ತೆ-ಯಾಗಿ
ಸರ್ವಾ-ಶ್ರ-ಯನೂ
ಸರ್ವಾ-ಸಾಂ
ಸರ್ವೇ
ಸರ್ವೇ-ಶ್ವರ
ಸರ್ವೇ-ಶ್ವ-ರತ್ವ
ಸರ್ವೇ-ಶ್ವ-ರನ
ಸರ್ವೇ-ಶ್ವ-ರ-ನಯ್ಯ
ಸರ್ವೇ-ಶ್ವ-ರ-ನಾದ
ಸರ್ವೇ-ಶ್ವ-ರನೂ
ಸರ್ವೇ-ಶ್ವ-ರ-ನೆಂದು
ಸರ್ವೇ-ಶ್ವ-ರ-ನೆ-ನಿ-ಸಿ-ಕೊಂಡ
ಸರ್ವೈ-ಸ್ಸ್ವ-ರೂ-ಪೈರ್ನಃ
ಸರ್ವೋ-ತ್ತಮ
ಸಲ
ಸಲ-ಕ್ಷ್ಮ-ಣೋ-ಽವ್ಯಾ-ದ್ಭ-ರ-ತಾ-ಗ್ರಜೋ
ಸಲ-ಗ-ಗ-ಳಂತೆ
ಸಲ-ಗ-ನಂತೆ
ಸಲವೂ
ಸಲ-ಸ-ಲವೂ
ಸಲ-ಹಲಿ
ಸಲ-ಹಿ-ದಿರಿ
ಸಲ-ಹು-ತ್ತಿ-ರ-ಬೇಕು
ಸಲ-ಹು-ತ್ತಿ-ರು-ವನೋ
ಸಲಹೆ
ಸಲ-ಹೆ-ಗಳನ್ನು
ಸಲ-ಹೆ-ಯಂತೆ
ಸಲ-ಹೆ-ಯಿತ್ತ
ಸಲ-ಹೆ-ಯೇನು
ಸಲಿ
ಸಲಿ-ಗೆ-ಯ-ನ್ನಿ-ತ್ತುದೆ
ಸಲಿ-ಗೆ-ಯನ್ನು
ಸಲಿ-ಗೆ-ಯಿಂದ
ಸಲೀ-ಸಾಗಿ
ಸಲು
ಸಲ್ಲ
ಸಲ್ಲ-ತ-ಕ್ಕು-ದೆಂದು
ಸಲ್ಲದು
ಸಲ್ಲ-ಬೇ-ಕಾದ
ಸಲ್ಲ-ಬೇ-ಕೆಂದು
ಸಲ್ಲ-ಲಿ-ಲ್ಲ-ವೆಂದು
ಸಲ್ಲಾಪ
ಸಲ್ಲಾ-ಪ-ದಲ್ಲಿ
ಸಲ್ಲಾ-ಪ-ಮಾ-ಡುತ್ತಾ
ಸಲ್ಲಿ-ಸ-ಬ-ಹುದು
ಸಲ್ಲಿ-ಸ-ಬೇ-ಕಾದ
ಸಲ್ಲಿ-ಸ-ಬೇ-ಕೆಂದು
ಸಲ್ಲಿ-ಸಲು
ಸಲ್ಲಿಸಿ
ಸಲ್ಲಿ-ಸಿತು
ಸಲ್ಲಿ-ಸಿದ
ಸಲ್ಲಿ-ಸಿ-ದನು
ಸಲ್ಲಿ-ಸಿ-ದರು
ಸಲ್ಲಿಸು
ಸಲ್ಲಿ-ಸು-ತ್ತೇನೆ
ಸಲ್ಲಿ-ಸು-ವು-ದ-ಕ್ಕಾಗಿ
ಸಲ್ಲಿ-ಸು-ವು-ದಾಗಿ
ಸಲ್ಲಿ-ಸು-ವೆನು
ಸಲ್ಲು
ಸಲ್ಲು-ತ್ತದೆ
ಸಲ್ಲು-ತ್ತಿತ್ತು
ಸವ-ತಿಯ
ಸವ-ತಿ-ಯರ
ಸವ-ತಿ-ಯರು
ಸವ-ನ-ಎಂಬ
ಸವ-ರಿದ
ಸವ-ರಿ-ದನು
ಸವ-ರಿ-ಹಾ-ಕಿದ
ಸವ-ರುತ್ತಾ
ಸವ-ರು-ತ್ತಿತ್ತು
ಸವಾರಿ
ಸವಾ-ಲೆಂ-ಬು-ವಂತೆ
ಸವಾ-ಸೇ-ರಾಗಿ
ಸವಿ
ಸವಿ-ತು-ರ್ಜಾ-ತ-ವೇದೋ
ಸವಿದು
ಸವಿ-ದು-ದ-ರಿಂದ
ಸವಿ-ನುಡಿ
ಸವಿ-ನು-ಡಿ-ಗಳಿಂದ
ಸವಿ-ನು-ಡಿಗೆ
ಸವಿ-ನೆ-ನ-ಪಾಗಿ
ಸವಿ-ಯನ್ನು
ಸವಿ-ಯ-ಬೇ-ಕೆಂ
ಸವಿ-ಯಾದ
ಸವಿ-ಯುತ್ತ
ಸವಿ-ಯು-ತ್ತಲೆ
ಸವಿ-ಯುತ್ತಾ
ಸವಿ-ಯುವ
ಸವಿ-ಸ್ತಾ-ರ-ವಾಗಿ
ಸವೆದು
ಸವೆ-ದು-ಹೋ-ಯಿತು
ಸವೆ-ಯದು
ಸವೆ-ಯು-ತ್ತದೆ
ಸವೆ-ಸಿ-ದನು
ಸವೆ-ಸು-ತ್ತಾನೆ
ಸವೆ-ಸು-ವಂತೆ
ಸವೆ-ಸು-ವ-ವರು
ಸಶ-ರೀ-ರ-ನಾ-ಗಿಯೆ
ಸಸ್ಯ
ಸಸ್ಯ-ಗಳ
ಸಸ್ಯ-ಗಳನ್ನು
ಸಸ್ಯ-ಗಳನ್ನೂ
ಸಸ್ಯ-ಗ-ಳಿಗೆ
ಸಸ್ಯ-ಗ-ಳೆಲ್ಲ
ಸಸ್ಯ-ವ-ರ್ಗಕ್ಕೆ
ಸಹ
ಸಹ-ಗ-ಮನ
ಸಹ-ಗ-ಮ-ನಕ್ಕೆ
ಸಹ-ಗ-ಮ-ನ-ಮಾ-ಡದೆ
ಸಹಜ
ಸಹ-ಜ-ದ್ವೇಷ
ಸಹ-ಜ-ವಾಗಿ
ಸಹ-ಜ-ವಾ-ಗಿಯೇ
ಸಹ-ಜವೇ
ಸಹ-ಜ-ಸ್ಥಿ-ತಿ-ಯಲ್ಲಿ
ಸಹ-ದೇವ
ಸಹ-ದೇ-ವನ
ಸಹ-ದೇ-ವನು
ಸಹ-ದೇ-ವ-ನೆಂ-ಬು-ವನು
ಸಹ-ದೇ-ವ-ರಿಂದ
ಸಹ-ದೇ-ವರೂ
ಸಹ-ದೇ-ವಾ-ಪ್ರ-ರೂಢ
ಸಹನೆ
ಸಹ-ನೆ-ಯನ್ನು
ಸಹ-ಪಾ-ಠಿ-ಯಾ-ಗಿದ್ದ
ಸಹ-ಪಾ-ಠಿ-ಯಾ-ಗುವ
ಸಹ-ವಾಸ
ಸಹ-ವಾ-ಸ-ದಲ್ಲಿ
ಸಹ-ವಾ-ಸ-ದ-ಲ್ಲಿ-ದ್ದು-ಕೊಂಡು
ಸಹ-ವಾ-ಸ-ದ-ಲ್ಲಿ-ರುವ
ಸಹ-ವಾ-ಸ-ದಿಂದ
ಸಹ-ವಾ-ಸ-ದೋಷ
ಸಹ-ವಾ-ಸ-ವನ್ನಾ
ಸಹ-ವಾ-ಸ-ವನ್ನು
ಸಹ-ವಾ-ಸ-ವನ್ನೂ
ಸಹಸೇ
ಸಹಸ್ರ
ಸಹ-ಸ್ರ-ಜನ
ಸಹ-ಸ್ರ-ಜಿತ್ತು
ಸಹ-ಸ್ರದ
ಸಹ-ಸ್ರ-ದ-ಳ-ಗಳಿಂದ
ಸಹ-ಸ್ರ-ನೇತ್ರ
ಸಹ-ಸ್ರ-ಪಾದ
ಸಹ-ಸ್ರ-ಬಾಹು
ಸಹ-ಸ್ರ-ವರ್ಷ
ಸಹ-ಸ್ರ-ವ-ರ್ಷ-ಗ-ಳ-ವ-ರೆಗೂ
ಸಹ-ಸ್ರ-ವ-ರ್ಷ-ಗ-ಳ-ವ-ರೆಗೆ
ಸಹ-ಸ್ರಾರು
ಸಹಾ-ನು-ಭೂ-ತಿಯ
ಸಹಾಯ
ಸಹಾ-ಯಕ
ಸಹಾ-ಯ-ಕ-ನಾಗಿ
ಸಹಾ-ಯ-ಕ-ನಾ-ಗಿ-ರು-ವು-ದ-ರಿಂದ
ಸಹಾ-ಯ-ಕ-ರಾ-ಗಿ-ದ್ದಾರೆ
ಸಹಾ-ಯ-ಕ-ವಾಗಿ
ಸಹಾ-ಯ-ಕ-ವಾ-ದುದು
ಸಹಾ-ಯ-ಕ-ವಾ-ಯಿತು
ಸಹಾ-ಯ-ಕ್ಕಾಗಿ
ಸಹಾ-ಯಕ್ಕೆ
ಸಹಾ-ಯ-ಕ್ಕೆಂದು
ಸಹಾ-ಯ-ದಿಂದ
ಸಹಾ-ಯ-ದಿಂ-ದಲೆ
ಸಹಾ-ಯ-ವನ್ನು
ಸಹಿ-ತ-ನಾಗಿ
ಸಹಿ-ತ-ವಾಗಿ
ಸಹಿಸ
ಸಹಿ-ಸದೆ
ಸಹಿ-ಸ-ಲಾರ
ಸಹಿ-ಸ-ಲಾ-ರದ
ಸಹಿ-ಸ-ಲಾ-ರದೆ
ಸಹಿ-ಸ-ಲಾರೆ
ಸಹಿ-ಸಲಿ
ಸಹಿ-ಸ-ಲಿಲ್ಲ
ಸಹಿ-ಸಿ-ಕೊಂಡು
ಸಹಿ-ಸಿ-ಕೊ-ಳ್ಳ-ಲೇ-ಬೇಕು
ಸಹೇಂ-ದ್ರಾಯ
ಸಹೋ-ದ-ರರೂ
ಸಾಂಕ್ರಾ-ಮಿ-ಕ-ದಂತೆ
ಸಾಂಖ್ಯಾ-ಶಾ-ಸ್ತ್ರ-ವನ್ನು
ಸಾಂಗ-ವಾಗಿ
ಸಾಂದೀ-ಪನ
ಸಾಂದೀ-ಪ-ನ-ರಿಂದ
ಸಾಂದೀ-ಪ-ನರು
ಸಾಂದೀ-ಪ-ನ-ರೆಂಬ
ಸಾಂದೀ-ಪರ
ಸಾಂದೀ-ಪಿ-ನಿ-ಯೆಂಬ
ಸಾಂಬ
ಸಾಂಬನ
ಸಾಂಬ-ನನ್ನು
ಸಾಂಬ-ನನ್ನೂ
ಸಾಂಬ-ನಿಗೆ
ಸಾಂಬನು
ಸಾಂಬ-ನೇನು
ಸಾಕಲ್ಲಾ
ಸಾಕಷ್ಟಾ
ಸಾಕಷ್ಟು
ಸಾಕಾಗಿ
ಸಾಕಾ-ಗಿದೆ
ಸಾಕಾದ
ಸಾಕಾ-ಯಿತು
ಸಾಕಿ
ಸಾಕು
ಸಾಕು-ತಂ-ದೆ-ತಾ-ಯಿ-ಯ-ರನ್ನು
ಸಾಕು-ತಂ-ದೆ-ಯನ್ನು
ಸಾಕು-ಮ-ಗ-ಳು-ಪಾಂ-ಡು-ಕರ್ಣ
ಸಾಕೆ-ನಿ-ಸಿತು
ಸಾಕ್ಷತ್
ಸಾಕ್ಷಾ
ಸಾಕ್ಷಾತ್
ಸಾಕ್ಷಾ-ತ್ಕಾರ
ಸಾಕ್ಷಾ-ತ್ಕಾ-ರಕ್ಕೆ
ಸಾಕ್ಷಾ-ತ್ಕಾ-ರದ
ಸಾಕ್ಷಾ-ತ್ಕಾ-ರ-ಮಾ-ಡಿ-ಕೊಂ-ಡ-ವ-ನಾ-ದರೂ
ಸಾಕ್ಷಾ-ತ್ಕಾ-ರ-ವನ್ನು
ಸಾಕ್ಷಾ-ತ್ಕಾ-ರ-ವಾ-ಗು-ತ್ತದೆ
ಸಾಕ್ಷಾ-ತ್ಕಾ-ರ-ವಾ-ದರೆ
ಸಾಕ್ಷಾ-ತ್ತಾಗಿ
ಸಾಕ್ಷಾ-ತ್ಪ-ರ-ಮೇ-ಶ್ವ-ರನೇ
ಸಾಕ್ಷಿ
ಸಾಕ್ಷಿ-ಗ-ಳಾಗಿ
ಸಾಕ್ಷಿ-ಭೂ-ತ-ನಾ-ಗಿ-ರು-ವೆಯೊ
ಸಾಕ್ಷಿ-ಯಾ-ಗಿ-ರು-ವ-ವನು
ಸಾಕ್ಷಿ-ಯೆಂಬ
ಸಾಗ-ಕ-ಳು-ಹಿಸಿ
ಸಾಗರ
ಸಾಗ-ರ-ಕಾಂ-ತೆ-ಯರೆ
ಸಾಗ-ರದ
ಸಾಗ-ರ-ದಲ್ಲಿ
ಸಾಗ-ರ-ದಿಂದ
ಸಾಗ-ರ-ವಾಗಿ
ಸಾಗ-ರವೆ
ಸಾಗ-ಲಿಲ್ಲ
ಸಾಗಿ
ಸಾಗಿತ್ತು
ಸಾಗಿದ
ಸಾಗಿ-ದವು
ಸಾಗಿ-ಸ-ಬೇಕು
ಸಾಗಿ-ಸ-ಹೊ-ರ-ಟರು
ಸಾಗಿಸಿ
ಸಾಗಿ-ಸಿದ
ಸಾಗಿ-ಸು-ತ್ತಿದ್ದ
ಸಾಗಿ-ಸು-ತ್ತಿ-ದ್ದನು
ಸಾಗಿ-ಸು-ವನು
ಸಾಗು-ತ್ತಿತ್ತು
ಸಾಗು-ತ್ತಿ-ದೆಯೊ
ಸಾಗು-ತ್ತಿ-ದ್ದಳು
ಸಾಗು-ತ್ತಿ-ರುವ
ಸಾಗು-ತ್ತಿ-ರು-ವಾಗ
ಸಾಗು-ವ-ಷ್ಟ-ರಲ್ಲಿ
ಸಾಟಿ
ಸಾಟಿಯೆ
ಸಾಣೆ-ಯಿ-ಕ್ಕದ
ಸಾತ್ತ್ವತ
ಸಾತ್ಯಕಿ
ಸಾತ್ಯ-ಕಿ-ಯನ್ನು
ಸಾತ್ವ-ತಾಂ-ಪತಿ
ಸಾತ್ವಿಕ
ಸಾತ್ವಿ-ಕ-ಗುಣ
ಸಾತ್ವಿ-ಕ-ನಾದ
ಸಾತ್ವಿ-ಕ-ಭಕ್ತಿ
ಸಾತ್ವಿ-ಕರು
ಸಾತ್ವಿ-ಕ-ವಾ-ಯಿತು
ಸಾತ್ವಿ-ಕಾ-ಹಂ-ಕಾರ
ಸಾತ್ವಿ-ಕಾ-ಹಂ-ಕಾ-ರ-ದಿಂದ
ಸಾತ್ವಿ-ಕಾ-ಹಾ-ರ-ವನ್ನು
ಸಾದಃ
ಸಾದ-ರ-ದಿಂದ
ಸಾಧ-ಕ-ನಿಗೆ
ಸಾಧ-ಕ-ವಾದ
ಸಾಧನ
ಸಾಧ-ನ-ಗಳೂ
ಸಾಧ-ನ-ಗ-ಳೆಂದು
ಸಾಧ-ನ-ವಿದೆ
ಸಾಧ-ನ-ವಿ-ಧಾ-ನ-ವನ್ನು
ಸಾಧ-ನ-ವೆ-ನಿ-ಸಿದ
ಸಾಧನಾ
ಸಾಧನೆ
ಸಾಧ-ನೆ-ಗಾಗಿ
ಸಾಧ-ನೆಗೆ
ಸಾಧ-ನೆಯ
ಸಾಧ-ನೆ-ಯಾಗಿ
ಸಾಧಾ-ರ-ಣ-ನಲ್ಲ
ಸಾಧಿಸ
ಸಾಧಿ-ಸ-ಬ-ಹುದು
ಸಾಧಿ-ಸ-ಬೇ-ಕೆಂ
ಸಾಧಿ-ಸ-ಬೇ-ಕೆಂಬ
ಸಾಧಿ-ಸಿ-ಕೊ-ಳ್ಳು-ವುದು
ಸಾಧಿ-ಸಿದ
ಸಾಧಿ-ಸಿ-ದನು
ಸಾಧಿ-ಸಿ-ದರು
ಸಾಧಿ-ಸಿ-ದ-ವ-ನಾಗಿ
ಸಾಧಿ-ಸುತ್ತಾ
ಸಾಧಿ-ಸುವ
ಸಾಧಿ-ಸು-ವು-ದ-ಕ್ಕಾಗಿ
ಸಾಧು
ಸಾಧು-ಗಳ
ಸಾಧು-ಗ-ಳಾ-ಗಿ-ರು-ವ-ವರ
ಸಾಧು-ಗ-ಳಾದ
ಸಾಧು-ಗಳು
ಸಾಧು-ಗಳೇ
ಸಾಧು-ಜ-ನರ
ಸಾಧು-ತ್ವಕ್ಕೆ
ಸಾಧು-ಪು-ರು-ಷ-ನಿಗೆ
ಸಾಧು-ರ-ಕ್ಷ-ಕ-ನಾದ
ಸಾಧು-ವಾ-ಗಿ-ರುವ
ಸಾಧು-ವಾದ
ಸಾಧು-ವಿ-ನಂತೆ
ಸಾಧು-ವೆಂದು
ಸಾಧು-ಸಂ-ತರ
ಸಾಧು-ಸಂ-ತ-ರನ್ನು
ಸಾಧು-ಸಂ-ತರು
ಸಾಧು-ಸ-ಜ್ಜ-ನರ
ಸಾಧು-ಸ-ಜ್ಜ-ನ-ರನ್ನು
ಸಾಧು-ಸ-ಜ್ಜ-ನ-ರಾಗಿ
ಸಾಧು-ಸ-ಜ್ಜ-ನ-ರಿಗೆ
ಸಾಧು-ಸ-ತ್ಪು-ರು-ಷರ
ಸಾಧ್ಯ
ಸಾಧ್ಯ-ರನ್ನೂ
ಸಾಧ್ಯ-ವಾ-ಗದೆ
ಸಾಧ್ಯ-ವಾ-ಗ-ಲಿಲ್ಲ
ಸಾಧ್ಯ-ವಾ-ಗಿ-ರ-ಬೇ-ಕಾ-ದರೆ
ಸಾಧ್ಯ-ವಾ-ಗಿಲ್ಲ
ಸಾಧ್ಯ-ವಾ-ಗು-ತ್ತದೆ
ಸಾಧ್ಯ-ವಾ-ಗು-ತ್ತ-ದೆಯೆ
ಸಾಧ್ಯ-ವಾ-ಗುತ್ತಾ
ಸಾಧ್ಯ-ವಾ-ಗು-ವಂ-ತಹ
ಸಾಧ್ಯ-ವಾ-ಗು-ವಂತೆ
ಸಾಧ್ಯ-ವಾದ
ಸಾಧ್ಯ-ವಾ-ದರೆ
ಸಾಧ್ಯ-ವಾ-ಯಿತು
ಸಾಧ್ಯ-ವಿ-ರ-ಲಿಲ್ಲ
ಸಾಧ್ಯ-ವಿಲ್ಲ
ಸಾಧ್ಯ-ವಿ-ಲ್ಲ-ದು-ದ-ಕ್ಕಾಗಿ
ಸಾಧ್ಯ-ವಿ-ಲ್ಲದೆ
ಸಾಧ್ಯ-ವಿ-ಲ್ಲ-ವಾ-ದರೂ
ಸಾಧ್ಯ-ವಿ-ಲ್ಲ-ವಾ-ದುದ
ಸಾಧ್ಯ-ವಿ-ಲ್ಲ-ವೆ-ನ್ನು-ವಂ-ತಹ
ಸಾಧ್ಯ-ವಿ-ಲ್ಲವೋ
ಸಾಧ್ಯವೂ
ಸಾಧ್ಯವೆ
ಸಾಧ್ಯವೇ
ಸಾಧ್ಯ-ವೇ-ನಪ್ಪ
ಸಾನ-ವನ್ನು
ಸಾನ್ನಿಧ್ಯ
ಸಾನ್ನಿ-ಧ್ಯ-ದಲ್ಲಿ
ಸಾನ್ನಿ-ಧ್ಯ-ದಿಂದ
ಸಾನ್ನಿ-ಧ್ಯ-ವನ್ನು
ಸಾಮ
ಸಾಮ-ಗಾ-ನ-ದಿಂದ
ಸಾಮ-ಗ್ರಿ-ಗಳನ್ನು
ಸಾಮ-ಗ್ರಿ-ಗಳನ್ನೆಲ್ಲಾ
ಸಾಮ-ಗ್ರಿ-ಗಳು
ಸಾಮ-ಗ್ರಿ-ಗ-ಳೆ-ಲ್ಲವೂ
ಸಾಮ-ಗ್ರಿ-ಗಳೇ
ಸಾಮ-ಗ್ರಿ-ಗ-ಳೊ-ಡನೆ
ಸಾಮ-ರಸ್ಯ
ಸಾಮ-ವೇ-ದ-ಮಂ-ತ್ರ-ಗಳಿಂದ
ಸಾಮಾ-ನನ್ನು
ಸಾಮಾ-ನು-ಗಳನ್ನು
ಸಾಮಾ-ನು-ಗಳನ್ನೆಲ್ಲ
ಸಾಮಾ-ನು-ಗ-ಳೊ-ಡನೆ
ಸಾಮಾನ್ಯ
ಸಾಮಾ-ನ್ಯ-ಜ-ನರೂ
ಸಾಮಾ-ನ್ಯ-ನಲ್ಲ
ಸಾಮಾ-ನ್ಯ-ನಾದ
ಸಾಮಾ-ನ್ಯನು
ಸಾಮಾ-ನ್ಯನೆ
ಸಾಮಾ-ನ್ಯ-ನೆಂದು
ಸಾಮಾ-ನ್ಯನೇ
ಸಾಮಾ-ನ್ಯರ
ಸಾಮಾ-ನ್ಯ-ರಲ್ಲ
ಸಾಮಾ-ನ್ಯ-ರಾದ
ಸಾಮಾ-ನ್ಯರೂ
ಸಾಮಾ-ನ್ಯರೇ
ಸಾಮಾ-ನ್ಯ-ಳಲ್ಲ
ಸಾಮಾ-ನ್ಯ-ವಲ್ಲ
ಸಾಮಾ-ನ್ಯ-ವಾಗಿ
ಸಾಮಾ-ನ್ಯ-ವಾದ
ಸಾಮಾ-ನ್ಯ-ವಾ-ದುದೆ
ಸಾಮಾ-ನ್ಯವೆ
ಸಾಮಾ-ನ್ಯವೇ
ಸಾಮೀಪ್ಯ
ಸಾಮೀ-ಪ್ಯ-ದ-ಲ್ಲಿ-ದ್ದರೂ
ಸಾಮ್ರಾ-ಜ್ಯದ
ಸಾಮ್ರಾ-ಜ್ಯ-ವನ್ನು
ಸಾಮ್ರಾ-ಜ್ಯ-ಸು-ಖ-ದಿಂದ
ಸಾಯಂ
ಸಾಯ-ಕೂ-ಡದು
ಸಾಯದೆ
ಸಾಯ-ಬೇ-ಕಾ-ಗು-ತ್ತದೆ
ಸಾಯ-ಬೇ-ಕೆಂದೇ
ಸಾಯಲಿ
ಸಾಯ-ಲಿಲ್ಲ
ಸಾಯಲೂ
ಸಾಯ-ಲೆಂದೊ
ಸಾಯಲೇ
ಸಾಯ-ಲೇ-ಬೇಕು
ಸಾಯಾಂ
ಸಾಯಿ-ಸಿದ
ಸಾಯುಜ್ಯ
ಸಾಯು-ಜ್ಯ-ಪ-ದ-ವಿ-ಯನ್ನು
ಸಾಯುತ್ತ
ಸಾಯು-ತ್ತದೆ
ಸಾಯು-ತ್ತಲೆ
ಸಾಯುತ್ತಾ
ಸಾಯು-ತ್ತಾರೆ
ಸಾಯು-ತ್ತಿ-ದ್ದೇವೆ
ಸಾಯು-ತ್ತಿ-ರಲು
ಸಾಯು-ತ್ತಿರು
ಸಾಯು-ತ್ತಿ-ರುವ
ಸಾಯುವ
ಸಾಯು-ವಾಗ
ಸಾಯು-ವು-ದಕ್ಕೆ
ಸಾಯು-ವು-ದ-ರ-ಲ್ಲಿಯೂ
ಸಾಯು-ವು-ದಲ್ಲಾ
ಸಾಯು-ವು-ದಿಲ್ಲ
ಸಾಯು-ವುದು
ಸಾಯು-ವು-ದು-ಇವು
ಸಾಯು-ವುದೂ
ಸಾಯೋಣ
ಸಾರ
ಸಾರಣೆ
ಸಾರಥಿ
ಸಾರ-ಥಿ-ಗ-ಳೊ-ಡನೆ
ಸಾರ-ಥಿ-ಯನ್ನು
ಸಾರ-ಥಿ-ಯಾಗಿ
ಸಾರ-ಥಿ-ಯಾದ
ಸಾರ-ಥಿ-ಯಾ-ದನು
ಸಾರ-ವ-ತ್ತಾಗಿ
ಸಾರ-ವನ್ನು
ಸಾರ-ವನ್ನೂ
ಸಾರ-ವಾಗಿ
ಸಾರ-ವಾದ
ಸಾರ-ವೆಲ್ಲ
ಸಾರವೇ
ಸಾರಸ
ಸಾರ-ಸ-ರ್ವ-ಸ್ವ-ವೆಂದು
ಸಾರಿ
ಸಾರಿದೆ
ಸಾರಿ-ಯಾ-ದರೂ
ಸಾರಿ-ಸ-ಬೇ-ಕೆಂಬ
ಸಾರಿ-ಹೇ-ಳು-ತ್ತಿವೆ
ಸಾರು-ತ್ತಿತ್ತು
ಸಾರು-ತ್ತಿ-ದ್ದವು
ಸಾರುವ
ಸಾರ್ಥಕ
ಸಾರ್ಥ-ಕ-ಪ-ಡಿ-ಸಿ-ಕೊಂಡ
ಸಾರ್ಥ-ಕ-ವಾ-ದವು
ಸಾರ್ಥ-ಕ-ವಾ-ದುವು
ಸಾರ್ಥ-ಕ-ವಾ-ಯಿತು
ಸಾರ್ಥ-ಗೊ-ಳಿ-ಸಿ-ಕೊಂ-ಡಿದೆ
ಸಾಲದು
ಸಾಲದೆ
ಸಾಲ-ದೆಂದು
ಸಾಲ-ದೆಂ-ಬಂತೆ
ಸಾಲವೆ
ಸಾಲಾ-ಗಿ-ರುವ
ಸಾಲಿಗೆ
ಸಾಲಿ-ಗ್ರಾ-ಮ-ಗಳು
ಸಾಲಿ-ನಲ್ಲಿ
ಸಾಲಿ-ನ-ಲ್ಲಿಯೇ
ಸಾಲು
ಸಾಲು-ಆಗಿ
ಸಾಲು-ಗಳು
ಸಾಲೋಕ್ಯ
ಸಾಲೋ-ಕ್ಯ-ಪ-ದ-ವಿ-ಯನ್ನು
ಸಾಲ್ವ
ಸಾಲ್ವನ
ಸಾಲ್ವ-ನನ್ನು
ಸಾಲ್ವ-ನಿ-ಗಾದ
ಸಾಲ್ವ-ನಿಗೆ
ಸಾಲ್ವನು
ಸಾಲ್ವನೂ
ಸಾವ-ಧಾ-ನ-ವಾಗಿ
ಸಾವನ್ನು
ಸಾವನ್ನೇ
ಸಾವರ್ಣಿ
ಸಾವಾ-ಗಲಿ
ಸಾವಿ
ಸಾವಿಗೆ
ಸಾವಿತ್ರಿ
ಸಾವಿ-ನಿಂದ
ಸಾವಿರ
ಸಾವಿ-ರಕ್ಕೆ
ಸಾವಿ-ರ-ಜ-ನದ
ಸಾವಿ-ರದ
ಸಾವಿ-ರ-ದೆಂ-ಟು-ನೂರು
ಸಾವಿ-ರ-ವ-ರ್ಷ-ಗಳ
ಸಾವಿ-ರಾರು
ಸಾವಿಲ್ಲ
ಸಾವು
ಸಾವು-ಗಳ
ಸಾವು-ಗಳು
ಸಾವು-ಗ-ಳೆಂಬ
ಸಾವೆ-ಲ್ಲಿ-ಯದು
ಸಾವೇ
ಸಾಷ್ಟಾಂಗ
ಸಾಹಸ
ಸಾಹ-ಸ-ಕಾ-ರ್ಯಕ್ಕೆ
ಸಾಹ-ಸ-ಕೃ-ತ್ಯ-ಗಳನ್ನು
ಸಾಹ-ಸಕ್ಕೆ
ಸಾಹ-ಸ-ಗಳ
ಸಾಹ-ಸ-ಗಳನ್ನೂ
ಸಾಹ-ಸ-ವನ್ನು
ಸಾಹ-ಸ-ವ-ನ್ನೆಲ್ಲ
ಸಾಹ-ಸವು
ಸಾಹ-ಸವೇ
ಸಾಹಿ-ತಿ-ಗ-ಳಾದ
ಸಾಹಿತ್ಯ
ಸಾಹಿ-ತ್ಯ-ಗಳ
ಸಾಹಿ-ತ್ಯ-ದಲ್ಲಿ
ಸಾಹಿ-ತ್ಯ-ದೃಷ್ಟಿ
ಸಾಹಿ-ತ್ಯ-ದೃ-ಷ್ಟಿ-ಯಿಂದ
ಸಾಹಿ-ತ್ಯೇ-ತಿ-ಹಾ-ಸ-ಜ್ಞರು
ಸಿಂಗ-ರದ
ಸಿಂಗ-ರಿ-ಸ-ಬೇ-ಕೆಂದು
ಸಿಂಗ-ರಿಸಿ
ಸಿಂಗ-ರಿ-ಸಿ-ದರು
ಸಿಂಗಾರ
ಸಿಂಗಾ-ರ-ಗಳಿಂದ
ಸಿಂಗಾ-ರ-ವನ್ನು
ಸಿಂಗಾ-ರ-ವಾ-ಗಿದ್ದ
ಸಿಂಚ
ಸಿಂಧು
ಸಿಂಧೂ
ಸಿಂಬ
ಸಿಂಹ
ಸಿಂಹಕ್ಕೆ
ಸಿಂಹ-ಗಳನ್ನು
ಸಿಂಹ-ಗಳು
ಸಿಂಹ-ಗಳೂ
ಸಿಂಹದ
ಸಿಂಹ-ದಂ-ತಿದ್ದ
ಸಿಂಹ-ದಂತೆ
ಸಿಂಹ-ದ-ಮೇಲೆ
ಸಿಂಹ-ದೊ-ಡನೆ
ಸಿಂಹ-ನಾದ
ಸಿಂಹ-ರಾ-ಜನು
ಸಿಂಹವು
ಸಿಂಹಾ-ಸನ
ಸಿಂಹಾ-ಸ-ನದ
ಸಿಂಹಾ-ಸ-ನ-ದಲ್ಲಿ
ಸಿಂಹಾ-ಸ-ನ-ದಿಂದ
ಸಿಂಹಾ-ಸ-ನ-ವ-ನ್ನಲ್ಲ
ಸಿಂಹಾ-ಸ-ನ-ವನ್ನು
ಸಿಂಹಾ-ಸ-ನ-ವ-ನ್ನೇರಿ
ಸಿಂಹಾ-ಸ-ನ-ವ-ನ್ನೇ-ರಿ-ದಾಗ
ಸಿಂಹಾ-ಸ-ನ-ವೇರಿ
ಸಿಂಹಿ-ಣಿ-ಯಾ-ದಳು
ಸಿಕ್ಕ
ಸಿಕ್ಕಂ-ತಾ-ಯಿತು
ಸಿಕ್ಕದ
ಸಿಕ್ಕ-ದಂತೆ
ಸಿಕ್ಕ-ದಿ-ದ್ದರೆ
ಸಿಕ್ಕದೆ
ಸಿಕ್ಕ-ನೆಂಬ
ಸಿಕ್ಕರೂ
ಸಿಕ್ಕರೆ
ಸಿಕ್ಕ-ಲಾ-ರದ
ಸಿಕ್ಕ-ಲಿಲ್ಲ
ಸಿಕ್ಕ-ಲಿ-ಲ್ಲ-ವೆಂದರೆ
ಸಿಕ್ಕ-ಲೆಂದು
ಸಿಕ್ಕವು
ಸಿಕ್ಕ-ಷ್ಟ-ರಿಂ-ದಲೆ
ಸಿಕ್ಕಷ್ಟೂ
ಸಿಕ್ಕಾಗ
ಸಿಕ್ಕಾನು
ಸಿಕ್ಕಿ
ಸಿಕ್ಕಿ-ಕೊಂ-ಡವು
ಸಿಕ್ಕಿ-ಕೊಂ-ಡಿತು
ಸಿಕ್ಕಿ-ಕೊಂಡು
ಸಿಕ್ಕಿತು
ಸಿಕ್ಕಿ-ತೆಂ-ದು-ಕೊಂಡು
ಸಿಕ್ಕಿದ
ಸಿಕ್ಕಿ-ದಂ-ತ-ಹು-ದನ್ನು
ಸಿಕ್ಕಿ-ದಂ-ತಾ-ಯಿತು
ಸಿಕ್ಕಿ-ದರು
ಸಿಕ್ಕಿ-ದ-ವ-ರನ್ನು
ಸಿಕ್ಕಿ-ದ-ಷ್ಟನ್ನು
ಸಿಕ್ಕಿ-ದಾಗ
ಸಿಕ್ಕಿ-ದು-ದನ್ನು
ಸಿಕ್ಕಿ-ದು-ದ-ರಿಂದ
ಸಿಕ್ಕಿ-ಬಿತ್ತು
ಸಿಕ್ಕಿ-ಬಿದ್ದ
ಸಿಕ್ಕಿ-ಬಿ-ದ್ದಾ-ಗಲೂ
ಸಿಕ್ಕಿ-ಬಿ-ದ್ದಿದ್ದ
ಸಿಕ್ಕಿ-ಬಿ-ದ್ದಿ-ರುವ
ಸಿಕ್ಕಿ-ಬಿದ್ದು
ಸಿಕ್ಕಿ-ರು-ವಾಗ
ಸಿಕ್ಕಿ-ರು-ವು-ದ-ರಿಂದ
ಸಿಕ್ಕಿ-ಸಿ-ಕೊಂ-ಡರೂ
ಸಿಕ್ಕಿ-ಸಿ-ಕೊಂ-ಡಿದ್ದ
ಸಿಕ್ಕು-ತ್ತ-ದೆಯೆ
ಸಿಕ್ಕುತ್ತಿ
ಸಿಕ್ಕು-ದನ್ನು
ಸಿಕ್ಕುದು
ಸಿಕ್ಕುವ
ಸಿಕ್ಕು-ವು-ದಿಲ್ಲ
ಸಿಕ್ಕು-ವುದು
ಸಿಗು-ವುದು
ಸಿಗು-ವುದೇ
ಸಿಟ್ಟು
ಸಿಡಿ
ಸಿಡಿ-ದೆ-ದ್ದ-ವನೆ
ಸಿಡಿ-ಲ-ದ-ನಿ-ಯನ್ನೂ
ಸಿಡಿ-ಲ-ನಂ-ತಹ
ಸಿಡಿಲಿ
ಸಿಡಿ-ಲಿನ
ಸಿಡಿ-ಲಿ-ನಂ-ತಹ
ಸಿಡಿ-ಲಿ-ನಂತೆ
ಸಿಡಿಲು
ಸಿಡಿ-ಲು-ಗ-ಳೊ-ಡನೆ
ಸಿಡಿ-ಲು-ಬ-ಡಿ-ದಂ-ತಾ-ಯಿತು
ಸಿತು
ಸಿದ
ಸಿದ-ನಂತೆ
ಸಿದನು
ಸಿದ-ಮೇಲೆ
ಸಿದರು
ಸಿದಾಗ
ಸಿದೆ
ಸಿದ್ದ
ಸಿದ್ಧ
ಸಿದ್ಧ-ಗ-ಣ-ಗಳಿಂದ
ಸಿದ್ಧ-ಗೊ-ಳಿಸು
ಸಿದ್ಧ-ಚಾ-ರ-ಣರ
ಸಿದ್ಧ-ತೆ-ಗಳನ್ನೂ
ಸಿದ್ಧ-ತೆ-ಗಳೂ
ಸಿದ್ಧ-ನಾಗಿ
ಸಿದ್ಧ-ನಾ-ಗಿದ್ದ
ಸಿದ್ಧ-ನಾ-ಗಿ-ದ್ದೇನೆ
ಸಿದ್ಧ-ನಾ-ಗಿಯೇ
ಸಿದ್ಧ-ನಾಗು
ಸಿದ್ಧ-ನಾ-ಗು-ತ್ತಿ-ರು-ವಾಗ
ಸಿದ್ಧ-ನಾದ
ಸಿದ್ಧ-ನಾ-ದನು
ಸಿದ್ಧ-ಪ-ಡಿ-ಸಿ-ದರು
ಸಿದ್ಧ-ಪ-ಡಿ-ಸಿರಿ
ಸಿದ್ಧ-ಪ-ಡಿ-ಸು-ವಂತೆ
ಸಿದ್ಧ-ಮಾ-ಡಿ-ಕೊಂಡು
ಸಿದ್ಧ-ರನ್ನೂ
ಸಿದ್ಧ-ರ-ಸದ
ಸಿದ್ಧ-ರಾಗಿ
ಸಿದ್ಧ-ರಾ-ಗಿದ್ದ
ಸಿದ್ಧ-ರಾಗು
ಸಿದ್ಧ-ರಾ-ಗು-ವಂತೆ
ಸಿದ್ಧ-ರಾ-ದರು
ಸಿದ್ಧರು
ಸಿದ್ಧ-ಳಾ-ಗಿ-ದ್ದಳು
ಸಿದ್ಧ-ಳಾ-ದಳು
ಸಿದ್ಧ-ವಾಗಿ
ಸಿದ್ಧ-ವಾ-ಗಿತ್ತು
ಸಿದ್ಧ-ವಾ-ಗಿ-ದ್ದವು
ಸಿದ್ಧ-ವಾ-ಗಿ-ದ್ದಾಳೆ
ಸಿದ್ಧ-ವಾ-ಗಿ-ರುವ
ಸಿದ್ಧ-ವಾ-ಗಿ-ರು-ವಂತೆ
ಸಿದ್ಧ-ವಾ-ಗು-ತ್ತಿತ್ತು
ಸಿದ್ಧ-ವಾ-ದಳು
ಸಿದ್ಧ-ವಾ-ದವು
ಸಿದ್ಧ-ವಾ-ದು-ದನ್ನು
ಸಿದ್ಧ-ವಾ-ಯಿತು
ಸಿದ್ಧ-ವಿ-ದ್ಯಾ-ಧ-ರರು
ಸಿದ್ಧಾಂ-ತಕ್ಕೆ
ಸಿದ್ಧಾ-ಶ್ರ-ಮ-ವೆಂಬ
ಸಿದ್ಧಿ
ಸಿದ್ಧಿ-ಗಳನ್ನು
ಸಿದ್ಧಿ-ಗಳು
ಸಿದ್ಧಿಗೆ
ಸಿದ್ಧಿ-ಪ್ರ-ದ-ನ-ಲ್ಲವೆ
ಸಿದ್ಧಿ-ಯಂತೆ
ಸಿದ್ಧಿ-ಯನ್ನು
ಸಿದ್ಧಿ-ಯಾ-ಗು-ತ್ತದೆ
ಸಿದ್ಧಿಯೇ
ಸಿದ್ಧಿ-ಸಿತು
ಸಿದ್ಧೌ-ಷ-ಧ-ದಂ-ತಿ-ರುವ
ಸಿರಲು
ಸಿರಿ-ಯನ್ನೂ
ಸಿರಿ-ಸಂ-ಪ-ತ್ತಿಗೂ
ಸಿರುವ
ಸಿರು-ವ-ನೆಂಬ
ಸಿಲು-ಕ-ದಂತೆ
ಸಿಲುಕಿ
ಸಿಹಿ-ತಿಂ-ಡಿ-ಯಂತೆ
ಸಿಹಿ-ನೀ-ರಿನ
ಸೀಕ-ನ್ನೊ-ಕೊ-ನೆಗೆ
ಸೀತಾ-ದೇ-ವಿಗೆ
ಸೀತಾ-ದೇ-ವಿ-ಯೊ-ಡನೆ
ಸೀತಾ-ರಾಮ
ಸೀತಾ-ಲ-ಕ್ಷ್ಮ-ಣ-ರೊ-ಡನೆ
ಸೀತೆ
ಸೀತೆಯ
ಸೀತೆ-ಯನ್ನು
ಸೀತೆ-ಯಿ-ರುವ
ಸೀತೇ
ಸೀದು
ಸೀನ
ಸೀನಲು
ಸೀನೀರು
ಸೀಮಾ
ಸೀಮೆ-ಗಳು
ಸೀರೆ
ಸೀರೆ-ಗಳನ್ನು
ಸೀರೆ-ಗಳನ್ನೆಲ್ಲ
ಸೀರೆ-ಗಳು
ಸೀರೆ-ಗ-ಳೆಲ್ಲ
ಸೀರೆ-ಗಳೇ
ಸೀರೆಯ
ಸೀರೆ-ಯನ್ನು
ಸೀರೆ-ಯು-ಟ್ಟ-ವಳು
ಸೀರೆ-ಯುಟ್ಟು
ಸೀಳಿ
ಸೀಳಿ-ಕೊಂಡು
ಸೀಳಿ-ದನು
ಸೀಳಿ-ಬಿ-ಡು-ತ್ತೇನೆ
ಸೀಳಿ-ಹಾ-ಕಿತು
ಸೀಳಿ-ಹಾ-ಕಿ-ದನು
ಸೀಳಿ-ಹಾ-ಕು-ವಂತೆ
ಸೀಳಿ-ಹಾ-ಕು-ವು-ದೇನೂ
ಸೀಳಿ-ಹೋ-ಗಿ-ದ್ದರೂ
ಸೀಳು-ವಂತೆ
ಸೀಸ-ವನ್ನು
ಸು
ಸುಂ
ಸುಂಕ
ಸುಂಕ-ವಿ-ಲ್ಲ-ವೆಂ-ದು-ಕೊಂಡು
ಸುಂಟ-ರ-ಗಾಳಿ
ಸುಂಟ-ರ-ಗಾ-ಳಿ-ಯೊಂದು
ಸುಂದರ
ಸುಂದ-ರ-ದೃಶ್ಯ
ಸುಂದ-ರ-ನಾದ
ಸುಂದ-ರ-ನಾ-ದರೆ
ಸುಂದ-ರ-ಪು-ರು-ಷರು
ಸುಂದ-ರ-ಮೂರ್ತಿ
ಸುಂದ-ರ-ಮೂ-ರ್ತಿ-ಯನ್ನು
ಸುಂದ-ರ-ರೂ-ಪ-ವನ್ನು
ಸುಂದ-ರ-ರೂ-ಪವು
ಸುಂದ-ರ-ಳಾದ
ಸುಂದ-ರಳು
ಸುಂದ-ರ-ವ-ನ-ದಲ್ಲಿ
ಸುಂದ-ರ-ವ-ನವು
ಸುಂದ-ರ-ವಾಗಿ
ಸುಂದ-ರ-ವಾ-ಗಿತ್ತು
ಸುಂದ-ರ-ವಾ-ಗಿದೆ
ಸುಂದ-ರ-ವಾ-ಗಿದ್ದ
ಸುಂದ-ರ-ವಾ-ಗಿ-ರು-ವಂತೆ
ಸುಂದ-ರ-ವಾದ
ಸುಂದ-ರಾಂಗ
ಸುಂದ-ರಾಂ-ಗ-ನಾರು
ಸುಂದ-ರಾಂ-ಗನೂ
ಸುಂದ-ರಾಂ-ಗ-ರಾದ
ಸುಂದ-ರಾಂಗಿ
ಸುಂದರಿ
ಸುಂದ-ರಿಯ
ಸುಂದ-ರಿ-ಯನ್ನು
ಸುಂದ-ರಿ-ಯ-ರಲ್ಲಿ
ಸುಂದ-ರಿ-ಯ-ರಾದ
ಸುಂದ-ರಿ-ಯರು
ಸುಂದ-ರಿ-ಯರೆ
ಸುಂದ-ರಿ-ಯ-ರೇ-ನು-ಹಿ-ರ-ಣ್ಯ-ಕ-ಶಿ-ಪು-ವಿನ
ಸುಂದ-ರಿ-ಯಾಗಿ
ಸುಂದ-ರಿ-ಯಾದ
ಸುಂದ-ರಿ-ಯಾ-ದರೂ
ಸುಂದ-ರಿ-ಯೆಂದು
ಸುಂದ-ರಿ-ಯೊಂ-ದಿಗೆ
ಸುಕನ್ಯೆ
ಸುಕ-ನ್ಯೆಗೆ
ಸುಕ-ನ್ಯೆ-ಯರು
ಸುಕ-ನ್ಯೆ-ಯೊ-ಡನೆ
ಸುಕರಂ
ಸುಕು-ಮಾ-ರ-ನನ್ನು
ಸುಕು-ಮಾ-ರಿ-ಯನ್ನು
ಸುಕು-ಮಾ-ರಿ-ಯಾಗಿ
ಸುಕ್ಕು-ಗ-ಟ್ಟು-ತ್ತದೆ
ಸುಕ್ಕು-ಬಿದ್ದ
ಸುಕ್ಷೇ-ಮ-ನಾ-ಗಿದ್ದ
ಸುಖ
ಸುಖಂ
ಸುಖ-ಕರ
ಸುಖ-ಕ-ರ-ಗ-ಳೆಂದು
ಸುಖ-ಕ-ರ-ವಾ-ಗಿತ್ತೆ
ಸುಖ-ಕ-ರ-ವಾ-ಗು-ವಂ-ತಹ
ಸುಖ-ಕ್ಕಾಗಿ
ಸುಖಕ್ಕೂ
ಸುಖಕ್ಕೆ
ಸುಖ-ಕ್ಕೇಕೆ
ಸುಖ-ಗಳ
ಸುಖ-ಗಳನ್ನು
ಸುಖ-ಗ-ಳಿ-ಗೆಲ್ಲ
ಸುಖ-ಗಳು
ಸುಖದ
ಸುಖ-ದಂತೆ
ಸುಖ-ದಿಂದ
ಸುಖ-ದಿಂ-ದಿ-ದ್ದರು
ಸುಖ-ದಿಂ-ದಿ-ದ್ದುವು
ಸುಖ-ದಿಂ-ದಿ-ರು-ವ-ರ-ಲ್ಲವೆ
ಸುಖ-ದಿಂ-ದಿ-ರು-ವರೇ
ಸುಖ-ದುಃಖ
ಸುಖ-ದುಃ-ಖ-ಗಳ
ಸುಖ-ದುಃ-ಖ-ಗಳನ್ನು
ಸುಖ-ದುಃ-ಖ-ಗಳನ್ನೆಲ್ಲ
ಸುಖ-ದುಃ-ಖ-ಗ-ಳಿಗೆ
ಸುಖ-ದುಃ-ಖ-ಗ-ಳಿ-ಗೆಲ್ಲಾ
ಸುಖ-ದುಃ-ಖ-ಗ-ಳಿ-ಲ್ಲ-ವೆಂ-ದಾ-ದರೂ
ಸುಖ-ದುಃ-ಖ-ಗಳು
ಸುಖ-ದುಃ-ಖ-ಗ-ಳೆಂಬ
ಸುಖ-ದುಃ-ಖಾದಿ
ಸುಖ-ಪ-ಡ-ಬೇ-ಕೆಂಬ
ಸುಖ-ಪು-ರು-ಷನೇ
ಸುಖ-ಪ್ರಾಪ್ತಿ
ಸುಖ-ಭೋಗ
ಸುಖ-ಭೋ-ಗ-ಗಳ
ಸುಖ-ಭೋ-ಗ-ಗಳನ್ನು
ಸುಖ-ಭೋ-ಗ-ಗಳಲ್ಲಿ
ಸುಖ-ಭೋ-ಜನ
ಸುಖ-ಭ್ರಾಂತಿ
ಸುಖ-ರಾ-ಶಿ-ಕರಂ
ಸುಖ-ವನ್ನು
ಸುಖ-ವನ್ನೊ
ಸುಖ-ವ-ಸ-ತಿ-ಗಳನ್ನು
ಸುಖ-ವಾ-ಗಲಿ
ಸುಖ-ವಾಗಿ
ಸುಖ-ವಾ-ಗಿ-ದ್ದನು
ಸುಖ-ವಾ-ಗಿ-ದ್ದರು
ಸುಖ-ವಾ-ಗಿ-ದ್ದಾನೆ
ಸುಖ-ವಾ-ಗಿ-ದ್ದಾ-ನೆಯೆ
ಸುಖ-ವಾ-ಗಿ-ದ್ದಾ-ನೆಯೊ
ಸುಖ-ವಾ-ಗಿ-ದ್ದಾ-ರೆಯೊ
ಸುಖ-ವಾ-ಗಿಯೇ
ಸುಖ-ವಾ-ಗಿ-ರ-ಬಾ-ರದೆ
ಸುಖ-ವಾ-ಗಿ-ರ-ಬೇ-ಕಾ-ದರೆ
ಸುಖ-ವಾ-ಗಿ-ರ-ಬೇಕು
ಸುಖ-ವಾ-ಗಿ-ರಿಸು
ಸುಖ-ವಾ-ಗಿ-ರು-ವರೆ
ಸುಖ-ವಾ-ದ-ರೇನು
ಸುಖ-ವಿಲ್ಲ
ಸುಖವೂ
ಸುಖ-ವೆಂದು
ಸುಖ-ವೆಲ್ಲಿ
ಸುಖ-ಶಾಂ-ತಿ-ಗಳನ್ನು
ಸುಖ-ಶಾಂ-ತಿ-ಗಳಿಂದ
ಸುಖ-ಶಾಂ-ತಿ-ಗಳು
ಸುಖ-ಸಂ-ತೋ-ಷ-ಗಳಿಂದ
ಸುಖ-ಸಂ-ತೋ-ಷ-ಗಳು
ಸುಖ-ಸಾ-ಧ-ನೆಗೂ
ಸುಖ-ಸೌ-ಭಾ-ಗ್ಯ-ಗಳು
ಸುಖಾ-ಗ-ಮ-ನವೆ
ಸುಖಾ-ನು-ಭ-ವ-ವ-ಲ್ಲ-ವೆಂದು
ಸುಖಾ-ಭಿ-ಲಾ-ಷಿ-ಯಾಗಿ
ಸುಖಾ-ಭಿ-ಲಾಷೆ
ಸುಖಾ-ಸ-ನ-ಗಳು
ಸುಖಾ-ಸ-ನ-ದಲ್ಲಿ
ಸುಖಾ-ಸೀ-ನ-ನಾ-ಗಿ-ದ್ದನು
ಸುಖಿ-ಗ-ಳಾ-ಗಿ-ರು-ವರು
ಸುಖಿ-ಗ-ಳಾದ
ಸುಖಿ-ಗಳು
ಸುಖಿನೋ
ಸುಖಿ-ಯಾ-ಗ-ಬ-ಯ-ಸು-ವ-ವ-ನಿಗೆ
ಸುಖಿ-ಯಾ-ಗಿದ್ದು
ಸುಗಂಧ
ಸುಗಂ-ಧ-ದಿಂದ
ಸುಗಂ-ಧ-ಯು-ಕ್ತ-ವಾ-ಗಿ-ರು-ವುದು
ಸುಗಣಂ
ಸುಗ-ಮ-ವಾಗಿ
ಸುಗುಣ
ಸುಗು-ಣ-ವನ್ನು
ಸುಗು-ಮ-ವಾಗಿ
ಸುಗ್ರೀ-ವನ
ಸುಗ್ರೀ-ವ-ನನ್ನು
ಸುಗ್ರೀ-ವ-ನಿಗೆ
ಸುಗ್ರೀ-ವ-ನೆಂಬ
ಸುಚಾರು
ಸುಜ-ನೈ-ರ-ಲಭಂ
ಸುಜ್ಞಾ-ನ-ನಿ-ಧಿ-ಯಾದ
ಸುಟ್ಟ
ಸುಟ್ಟರು
ಸುಟ್ಟರೆ
ಸುಟ್ಟು
ಸುಟ್ಟು-ಬಿ-ಡ-ಬೇ-ಕೆನ್ನು
ಸುಟ್ಟು-ಬಿ-ಡು-ವಂತೆ
ಸುಟ್ಟು-ಹಾ-ಕಿತು
ಸುಟ್ಟು-ಹಾ-ಕಿದ
ಸುಟ್ಟು-ಹಾ-ಕು-ತ್ತಿ-ದ್ದನು
ಸುಟ್ಟು-ಹೋ-ಗಿ-ರ-ಬೇ-ಕೆಂ-ದು-ಕೊಂಡು
ಸುಡ
ಸುಡ-ಲಾ-ಗು-ವು-ದಿಲ್ಲ
ಸುಡಲು
ಸುಡಿ-ಸು-ವು-ದಕ್ಕೆ
ಸುಡು
ಸುಡು-ತ್ತದೆ
ಸುಡುತ್ತಾ
ಸುಡು-ವಂತೆ
ಸುಡು-ವ-ವ-ನಂತೆ
ಸುಣ್ಣ
ಸುತ-ಪ-ಪೃಶ್ನಿ
ಸುತಲ
ಸುತ-ಲ-ಲೋ-ಕಕ್ಕೆ
ಸುತಳ
ಸುತ್ತ
ಸುತ್ತದೆ
ಸುತ್ತ-ಮುತ್ತ
ಸುತ್ತ-ಮು-ತ್ತ-ಲಿದ್ದ
ಸುತ್ತ-ಮು-ತ್ತ-ಲಿನ
ಸುತ್ತ-ಮು-ತ್ತಿನ
ಸುತ್ತ-ಲಿಂ-ದಲೂ
ಸುತ್ತ-ಲಿದ್ದ
ಸುತ್ತ-ಲಿ-ದ್ದ-ವ-ರನ್ನು
ಸುತ್ತ-ಲಿನ
ಸುತ್ತಲೂ
ಸುತ್ತ-ಹೊ-ರ-ಟನು
ಸುತ್ತಾ
ಸುತ್ತಾಡಿ
ಸುತ್ತಾಡು
ಸುತ್ತಾ-ಡು-ತ್ತಿ-ದ್ದನು
ಸುತ್ತಾ-ಡು-ತ್ತಿರು
ಸುತ್ತಾ-ಡು-ತ್ತಿ-ರುವ
ಸುತ್ತಾನೆ
ಸುತ್ತಿ
ಸುತ್ತಿ-ಕೊಂ-ಡಂ-ತಾ-ಯಿತು
ಸುತ್ತಿ-ಕೊಂ-ಡ-ರಂತೆ
ಸುತ್ತಿ-ಕೊಂ-ಡರು
ಸುತ್ತಿ-ಕೊಂ-ಡಳು
ಸುತ್ತಿ-ಕೊಂ-ಡಿತು
ಸುತ್ತಿ-ಕೊಂಡು
ಸುತ್ತಿ-ಕೊಳ್ಳ
ಸುತ್ತಿ-ದನು
ಸುತ್ತಿ-ದಳು
ಸುತ್ತಿ-ದಾಗ
ಸುತ್ತಿ-ದುದೂ
ಸುತ್ತಿದ್ದ
ಸುತ್ತಿ-ದ್ದ-ವಳು
ಸುತ್ತಿ-ದ್ದಾನೆ
ಸುತ್ತಿ-ರ-ಬೇಕು
ಸುತ್ತಿ-ರು-ವಾಗ
ಸುತ್ತಿ-ರುವೆ
ಸುತ್ತಿ-ಸು-ತ್ತಿ-ರು-ತ್ತೇನೆ
ಸುತ್ತು-ಗ-ಟ್ಟಿ-ದರು
ಸುತ್ತುತ್ತ
ಸುತ್ತು-ತ್ತವೆ
ಸುತ್ತು-ತ್ತಿ-ದ್ದನು
ಸುತ್ತು-ತ್ತಿರು
ಸುತ್ತು-ತ್ತಿ-ರುವ
ಸುತ್ತು-ಮು-ತ್ತಿನ
ಸುತ್ತು-ವ-ರಿದ
ಸುತ್ತು-ವಾಗ
ಸುತ್ತೇನೆ
ಸುದ-ಕ್ಷಿಣ
ಸುದ-ಕ್ಷಿ-ಣ-ನನ್ನೂ
ಸುದ-ಕ್ಷಿ-ಣನು
ಸುದ-ಕ್ಷಿ-ಣ-ನೆಂ-ಬು-ವನು
ಸುದಯಂ
ಸುದರಂ
ಸುದ-ರ್ಶನ
ಸುದ-ರ್ಶ-ನದ
ಸುದಾ-ಮ-ನೆಂಬ
ಸುದೀ-ರ್ಘ-ವಾಗಿ
ಸುದೃ-ಶಮ್
ಸುದ್ದಿ
ಸುದ್ದಿ-ಯನ್ನು
ಸುದ್ದಿ-ಯನ್ನೇ
ಸುದ್ದಿಯು
ಸುದ್ದಿಯೂ
ಸುದ್ದಿ-ಯೆಲ್ಲ
ಸುದ್ಯುಮ್ನ
ಸುದ್ಯು-ಮ್ನನ
ಸುದ್ಯು-ಮ್ನ-ನಿಂದ
ಸುದ್ಯು-ಮ್ನನು
ಸುಧನ್ವ
ಸುಧಯಾ
ಸುಧ-ರ್ಮೆ-ಯೆಂಬ
ಸುಧಾ
ಸುಧಾಮ
ಸುಧಾ-ಮನ
ಸುಧಾ-ಮನು
ಸುಧಾ-ಮ-ಸು-ಮಿತ್ರ
ಸುಧಾ-ರ-ಸವು
ಸುಧಾ-ರಿ-ಸಿಕೊ
ಸುಧೀ
ಸುಧೇಷ್ಮ
ಸುನಂದ
ಸುನಂ-ದಾದಿ
ಸುನೀತಿ
ಸುನೀಥೆ
ಸುನೀ-ಥೆಗೆ
ಸುನೀ-ಲ-ವಾದ
ಸುಪದಂ
ಸುಪ್ತ-ಸ್ಥಿ-ತಿ-ಯ-ಲ್ಲಿ-ದ್ದರೂ
ಸುಪ್ಪ
ಸುಪ್ಪ-ತ್ತಿ-ಗೆ-ಗಳಿಂದ
ಸುಪ್ಪ-ತ್ತಿ-ಗೆ-ಗಳು
ಸುಪ್ರೀತ
ಸುಪ್ರೀ-ತ-ನಾ-ಗು-ವನು
ಸುಪ್ರೀ-ತ-ನಾದ
ಸುಪ್ರೀ-ತ-ರಾ-ಗು-ತ್ತಾರೆ
ಸುಪ್ರೀ-ತ-ರಾದ
ಸುಬಲಂ
ಸುಬಾಹು
ಸುಭದ್ರ
ಸುಭದ್ರೆ
ಸುಭ-ದ್ರೆಗೂ
ಸುಭ-ದ್ರೆಯ
ಸುಭ-ದ್ರೆ-ಯನ್ನು
ಸುಭ-ದ್ರೆ-ಯನ್ನೆ
ಸುಭ-ದ್ರೆಯೂ
ಸುಭು-ಜಮ್
ಸುಮತಿ
ಸುಮ-ತಿಯ
ಸುಮ-ನೋ-ಹರ
ಸುಮ-ಸನ
ಸುಮಾರು
ಸುಮಾ-ಲಿ-ಗಳೂ
ಸುಮಿತ್ರ
ಸುಮ್ಮ-ನಾ-ದನು
ಸುಮ್ಮ-ನಾ-ದರು
ಸುಮ್ಮನಿ
ಸುಮ್ಮ-ನಿದ್ದ
ಸುಮ್ಮ-ನಿ-ದ್ದಾನೆ
ಸುಮ್ಮ-ನಿ-ರ-ಲಿಲ್ಲ
ಸುಮ್ಮ-ನಿ-ರಲು
ಸುಮ್ಮ-ನಿರು
ಸುಮ್ಮ-ನಿ-ರು-ವ-ರಲ್ಲಾ
ಸುಮ್ಮ-ನಿ-ರುವೆ
ಸುಮ್ಮನೆ
ಸುಯ-ಜ್ಞನ
ಸುಯ-ಜ್ಞ-ನೆಂಬ
ಸುಯೋ-ಗ-ವಿ-ದೆ-ಯಂತೆ
ಸುರ
ಸುರ-ಕ್ಷಿ-ತ-ವಾ-ಗಿ-ಟ್ಟಿ-ದ್ದನು
ಸುರ-ಕ್ಷಿ-ತ-ವಾದ
ಸುರ-ನಾ-ರಿ-ಯರು
ಸುರ-ನಿಗೆ
ಸುರನು
ಸುರ-ಭೂ-ಮಿ-ಹ-ರಿ-ಕೇಶ
ಸುರ-ಸುಂ-ದ-ರಿ-ಯನ್ನು
ಸುರ-ಸುಂ-ದ-ರಿ-ಯಾದ
ಸುರಾ-ದೇವಿ
ಸುರಿ
ಸುರಿ-ದಂ-ತಾ-ಯಿತು
ಸುರಿ-ದವು
ಸುರಿ-ದಷ್ಟೂ
ಸುರಿದು
ಸುರಿ-ದು-ಕೊಂ-ಡನು
ಸುರಿ-ದು-ಕೊಂಡು
ಸುರಿ-ಯಲು
ಸುರಿ-ಯಿತು
ಸುರಿ-ಯು-ತ್ತದೆ
ಸುರಿ-ಯು-ತ್ತಿತ್ತು
ಸುರಿ-ಯು-ತ್ತಿದೆ
ಸುರಿ-ಯು-ತ್ತಿ-ರುವ
ಸುರಿ-ಯುವ
ಸುರಿ-ಯುವು
ಸುರಿ-ಯು-ವು-ದಕ್ಕೆ
ಸುರಿ-ಸದ
ಸುರಿ-ಸಲು
ಸುರಿ-ಸ-ಹ-ತ್ತಿ-ದನು
ಸುರಿಸಿ
ಸುರಿ-ಸಿ-ಕೊಂಡು
ಸುರಿ-ಸಿತು
ಸುರಿ-ಸಿ-ದ-ನಂತೆ
ಸುರಿ-ಸಿ-ದನು
ಸುರಿ-ಸಿ-ದರು
ಸುರಿ-ಸಿ-ದಳು
ಸುರಿ-ಸಿ-ದವು
ಸುರಿ-ಸಿ-ದುವು
ಸುರಿ-ಸುತ್ತ
ಸುರಿ-ಸುತ್ತಾ
ಸುರಿ-ಸು-ತ್ತಾರೆ
ಸುರಿ-ಸು-ತ್ತಿತ್ತು
ಸುರಿ-ಸು-ತ್ತಿ-ರುವ
ಸುರಿ-ಸು-ವನು
ಸುರಿ-ಸುವೆ
ಸುರುಚಿ
ಸುರು-ಚಿಯ
ಸುರು-ಚಿ-ಯಾ-ಡಿದ
ಸುರು-ಚಿಯು
ಸುರು-ಳಿ-ಸು-ರು-ಳಿ-ಯಾದ
ಸುರೆಯ
ಸುರೆ-ಯನ್ನು
ಸುರೇತ
ಸುರ್ಭಾನು
ಸುಲಭ
ಸುಲಭಂ
ಸುಲ-ಭ-ಭ-ಗ-ವಂ-ತನ
ಸುಲ-ಭ-ಮಾರ್ಗ
ಸುಲ-ಭ-ವಲ್ಲ
ಸುಲ-ಭ-ವ-ಲ್ಲ-ದಿ-ದ್ದರೂ
ಸುಲ-ಭ-ವಾಗಿ
ಸುಲ-ಭ-ವಾದ
ಸುಲ-ಭ-ವಾ-ದುದೂ
ಸುಲ-ಭ-ವೇನೂ
ಸುಲಿ-ಗೆ-ಗಳು
ಸುಳಿ
ಸುಳಿ-ಗಳು
ಸುಳಿ-ಗು-ರು-ಳಿನ
ಸುಳಿಗೆ
ಸುಳಿ-ದಾ-ಡುತ್ತಾ
ಸುಳಿ-ದಾ-ಡು-ತ್ತಿ-ದ್ದನು
ಸುಳಿ-ದಾ-ಡು-ತ್ತಿ-ರುವೆ
ಸುಳಿದು
ಸುಳಿ-ಯದು
ಸುಳಿ-ಯಾದ
ಸುಳಿ-ವನ್ನು
ಸುಳಿ-ವಿ-ನಿಂದ
ಸುಳಿ-ವಿಲ್ಲ
ಸುಳಿವೇ
ಸುಳ್ಳ-ನೆಂದು
ಸುಳ್ಳನ್ನು
ಸುಳ್ಳಾ-ಗಿ-ಹೋ-ದರೆ
ಸುಳ್ಳಾ-ಡ-ಬೇಡ
ಸುಳ್ಳಾಡು
ಸುಳ್ಳಿ-ಗಿಂತ
ಸುಳ್ಳಿಗೆ
ಸುಳ್ಳು
ಸುಳ್ಳೇ
ಸುಳ್ಳೇನೂ
ಸುವ
ಸುವ-ರ್ಣ-ವನ್ನು
ಸುವಾ-ಸನ
ಸುವಾ-ಸ-ನಾ-ಯು-ಕ್ತ-ವಾ-ಗಿದೆ
ಸುವಾ-ಸನೆ
ಸುವಾ-ಸ-ನೆಗೆ
ಸುವಾ-ಸ-ನೆ-ಯಂತೆ
ಸುವಾ-ಸ-ನೆ-ಯನ್ನು
ಸುವಾ-ಸ-ನೆ-ಯಿಂದ
ಸುವಾ-ಸ-ನೆಯೆ
ಸುವಿ-ಸ್ತಾ-ರ-ವಾಗಿ
ಸುವು-ದ-ಕ್ಕಾಗಿ
ಸುವು-ದಿ-ಲ್ಲ-ವೆಂ-ಬುದು
ಸುವುದು
ಸುಶರ್ಮ
ಸುಷು-ಪ್ತಿ-ಯಲ್ಲಿ
ಸುಷು-ಪ್ತಿ-ಯೆಂಬ
ಸುಷೇಣ
ಸುಸ-ಮ-ಯ-ವನ್ನೇ
ಸುಸ್ಥಿ-ರ-ನಾ-ದ-ವನು
ಸುಸ್ಪಷ್ಟ
ಸುಸ್ಪ-ಷ್ಟ-ವಾಗು
ಸುಹೃ-ದಮ್
ಸೂಕ್ತ
ಸೂಕ್ತಿ
ಸೂಕ್ತಿ-ಯೊಂದು
ಸೂಕ್ಷ್ಮ
ಸೂಕ್ಷ್ಮ-ಗ-ಳೆ-ರ-ಡಕ್ಕೂ
ಸೂಕ್ಷ್ಮ-ದೇ-ಹ-ಗಳ
ಸೂಕ್ಷ್ಮ-ದೇ-ಹ-ದಿಂದ
ಸೂಕ್ಷ್ಮ-ದೇ-ಹ-ದಿಂ-ದಲೆ
ಸೂಕ್ಷ್ಮ-ದೇ-ಹವು
ಸೂಕ್ಷ್ಮ-ದೇ-ಹ-ವೊಂದು
ಸೂಕ್ಷ್ಮ-ಪ್ರ-ಕೃ-ತಿ-ಯ-ಲ್ಲಿ-ರು-ವುದು
ಸೂಕ್ಷ್ಮ-ವಾಗಿ
ಸೂಕ್ಷ್ಮ-ವಾ-ಗಿ-ರು-ತ್ತವೆ
ಸೂಕ್ಷ್ಮ-ವಾದ
ಸೂಕ್ಷ್ಮ-ಶ-ರೀ-ರ-ವನ್ನು
ಸೂಕ್ಷ್ಮಾ-ತಿ-ಸೂಕ್ಷ್ಮ
ಸೂಚನೆ
ಸೂಚ-ನೆ-ಯಂತೆ
ಸೂಚ-ನೆ-ಯಿಂದ
ಸೂಚಿ-ತ-ವಾದ
ಸೂಚಿ-ಸಿ-ದನು
ಸೂಚಿ-ಸಿ-ದ್ದಂತೆ
ಸೂಚಿಸು
ಸೂಚಿ-ಸು-ತ್ತದೆ
ಸೂಚಿ-ಸು-ತ್ತಿ-ದ್ದರು
ಸೂಚಿ-ಸು-ತ್ತಿ-ರುವ
ಸೂಚಿ-ಸುವ
ಸೂಚಿ-ಸು-ವಂತೆ
ಸೂಜಿ-ಗ-ಲ್ಲಿ-ನಂತೆ
ಸೂಜಿ-ಗಲ್ಲು
ಸೂತ
ಸೂತನ
ಸೂತ-ಪು-ರಾ-ಣಿಕ
ಸೂತ-ಪು-ರಾ-ಣಿ-ಕನ
ಸೂತ-ಪು-ರಾ-ಣಿ-ಕನು
ಸೂತ-ಪು-ರಾ-ಣಿ-ಕರು
ಸೂತ-ಮ-ಹರ್ಷಿ
ಸೂತ-ಮ-ಹ-ರ್ಷಿಯು
ಸೂತ-ಮುನಿ
ಸೂತ-ಮು-ನೀ-ಶ್ವ-ರ-ನಿಗೆ
ಸೂತ-ರನ್ನು
ಸೂತ-ರೆಂದು
ಸೂತಿ-ಕಾ-ಗೃ-ಹ-ದಲ್ಲಿ
ಸೂತ್ರ
ಸೂತ್ರ-ಧಾ-ರಿಯ
ಸೂತ್ರ-ಧಾ-ರಿ-ಯಾದ
ಸೂತ್ರ-ಧಾ-ರಿಯು
ಸೂತ್ರ-ವನ್ನು
ಸೂಯ-ಯಾ-ಗ-ದಲ್ಲಿ
ಸೂಯ-ಯಾ-ಗ-ವನ್ನು
ಸೂರೆ
ಸೂರೆ-ಗ-ಳ್ಳುತ್ತಾ
ಸೂರೆ-ಗೊಂಡ
ಸೂರೆ-ಗೊಂ-ಡಿತ್ತಾ
ಸೂರೆ-ಗೊಂ-ಡಿವೆ
ಸೂರೆ-ಗೊ-ಳ್ಳ-ಬಾ-ರದೆ
ಸೂರೆ-ಗೊ-ಳ್ಳು-ತ್ತವೆ
ಸೂರೆ-ಗೊ-ಳ್ಳುತ್ತಾ
ಸೂರೆ-ಗೊ-ಳ್ಳು-ತ್ತಿ-ದ್ದರು
ಸೂರೆ-ಮಾ-ಡು-ತ್ತಿ-ದ್ದವು
ಸೂರೆ-ಮಾ-ಡು-ವನು
ಸೂರೆ-ಯಾ-ಗಿ-ಹೋ-ಗಿದೆ
ಸೂರೆ-ಹೋ-ಯಿತು
ಸೂರ್ಯ
ಸೂರ್ಯ-ಗ್ರ-ಹಣ
ಸೂರ್ಯ-ಗ್ರ-ಹ-ಣ-ವೆಂದು
ಸೂರ್ಯ-ಚಂ-ದ್ರಾ-ದಿ-ಗಳು
ಸೂರ್ಯನ
ಸೂರ್ಯ-ನಂ-ತಿ-ದ್ದವು
ಸೂರ್ಯ-ನಂ-ತಿ-ರುವ
ಸೂರ್ಯ-ನಂತೆ
ಸೂರ್ಯ-ನನ್ನು
ಸೂರ್ಯ-ನಲ್ಲ
ಸೂರ್ಯ-ನಿಂದ
ಸೂರ್ಯ-ನಿಗೂ
ಸೂರ್ಯ-ನಿಗೆ
ಸೂರ್ಯನು
ಸೂರ್ಯನೇ
ಸೂರ್ಯ-ಭ-ಗ-ವಂತ
ಸೂರ್ಯ-ಭ-ಗ-ವಾ-ನನೆ
ಸೂರ್ಯ-ಮಂ-ಡ-ಲಕ್ಕೆ
ಸೂರ್ಯ-ಮಂ-ಡ-ಲದ
ಸೂರ್ಯ-ಮಾ-ತ್ಮಾ-ನ-ಮೀ-ಮ-ಹೀ-ತಿಹೇ
ಸೂರ್ಯರ
ಸೂರ್ಯರು
ಸೂರ್ಯಾ-ಸ್ತ-ದೊ-ಳ-ಗಾಗಿ
ಸೂರ್ಯೋ-ದ-ಯಕ್ಕೆ
ಸೂಳೆಯೂ
ಸೂಸುವ
ಸೃಂಜಯ
ಸೃಷ್ಟಿ
ಸೃಷ್ಟಿ-ಕರ್ತ
ಸೃಷ್ಟಿ-ಕ-ರ್ತ-ನನ್ನು
ಸೃಷ್ಟಿ-ಕ-ರ್ತ-ನಾಗಿ
ಸೃಷ್ಟಿ-ಕ-ರ್ತ-ನಾದ
ಸೃಷ್ಟಿ-ಕಾರ್ಯ
ಸೃಷ್ಟಿ-ಕಾ-ರ್ಯ-ವನ್ನು
ಸೃಷ್ಟಿ-ಕಾ-ರ್ಯವು
ಸೃಷ್ಟಿ-ಕ್ರ-ಮ-ವನ್ನು
ಸೃಷ್ಟಿ-ಗಳ
ಸೃಷ್ಟಿ-ಗಳಲ್ಲಿ
ಸೃಷ್ಟಿಗೂ
ಸೃಷ್ಟಿಗೆ
ಸೃಷ್ಟಿ-ಗೆಂದು
ಸೃಷ್ಟಿ-ಗೆಲ್ಲ
ಸೃಷ್ಟಿ-ಮಾ-ಡ-ಬೇ-ಕೆಂದು
ಸೃಷ್ಟಿಯ
ಸೃಷ್ಟಿ-ಯನ್ನು
ಸೃಷ್ಟಿ-ಯಲ್ಲಿ
ಸೃಷ್ಟಿ-ಯ-ಲ್ಲಿಯೇ
ಸೃಷ್ಟಿ-ಯಾ-ಗಿದೆ
ಸೃಷ್ಟಿ-ಯಾದ
ಸೃಷ್ಟಿ-ಯಾ-ದವು
ಸೃಷ್ಟಿ-ಯಾ-ದುದು
ಸೃಷ್ಟಿ-ಯಾ-ದು-ವು-ಗ-ಳೆಂದು
ಸೃಷ್ಟಿ-ಯಾ-ಯಿತು
ಸೃಷ್ಟಿಯು
ಸೃಷ್ಟಿ-ಯೆಂ-ಬುದು
ಸೃಷ್ಟಿ-ಯೆ-ಲ್ಲವೂ
ಸೃಷ್ಟಿಯೇ
ಸೃಷ್ಟಿ-ಯೊ-ಳ-ಗಿ-ರುವ
ಸೃಷ್ಟಿ-ರ-ಹ-ಸ್ಯ-ವನ್ನು
ಸೃಷ್ಟಿ-ಶ-ಕ್ತಿ-ಯನ್ನು
ಸೃಷ್ಟಿ-ಸ-ಬೇ-ಕೆಂ-ದಿ-ರುವ
ಸೃಷ್ಟಿ-ಸ-ಬೇಡ
ಸೃಷ್ಟಿ-ಸ-ಹೊ-ರಟು
ಸೃಷ್ಟಿಸಿ
ಸೃಷ್ಟಿ-ಸಿದ
ಸೃಷ್ಟಿ-ಸಿ-ದನು
ಸೃಷ್ಟಿ-ಸಿ-ದ-ವನೂ
ಸೃಷ್ಟಿ-ಸಿ-ದುದು
ಸೃಷ್ಟಿ-ಸಿ-ದ್ದಾನೆ
ಸೃಷ್ಟಿ-ಸಿ-ದ್ದೇನೆ
ಸೃಷ್ಟಿ-ಸಿ-ರುವ
ಸೃಷ್ಟಿಸು
ಸೃಷ್ಟಿ-ಸು-ತ್ತಿ-ದ್ದನು
ಸೃಷ್ಟಿ-ಸು-ತ್ತೀಯೆ
ಸೃಷ್ಟಿ-ಸು-ತ್ತೇನೆ
ಸೃಷ್ಟಿ-ಸುವ
ಸೃಷ್ಟಿ-ಸು-ವವ
ಸೃಷ್ಟಿ-ಸು-ವಾಗ
ಸೃಷ್ಟಿ-ಸು-ವುದು
ಸೆಡ-ಕು-ಗಾತಿ
ಸೆಣ-ಸಲು
ಸೆರ-ಗನ್ನು
ಸೆರ-ಗನ್ನೂ
ಸೆರ-ಗಿನ
ಸೆರ-ಗಿ-ನಲ್ಲಿ
ಸೆರಗು
ಸೆರ-ಮ-ನೆಗೆ
ಸೆರೆ
ಸೆರೆ-ಮ-ನೆಗೆ
ಸೆರೆ-ಮ-ನೆ-ಯನ್ನು
ಸೆರೆ-ಮ-ನೆ-ಯಿಂದ
ಸೆರೆ-ಮ-ನೆ-ಯೆಂ-ದ-ಮೇಲೆ
ಸೆರೆಯ
ಸೆರೆ-ಯಲ್ಲಿ
ಸೆರೆ-ಯ-ಲ್ಲಿ-ಟ್ಟನು
ಸೆರೆ-ಯ-ಲ್ಲಿ-ಟ್ಟಿದ್ದ
ಸೆರೆ-ಯ-ಲ್ಲಿಟ್ಟು
ಸೆರೆ-ಯ-ಲ್ಲಿ-ಡಿ-ಸಿರು
ಸೆರೆ-ಯ-ಲ್ಲಿದ್ದ
ಸೆರೆ-ಯ-ಲ್ಲಿದ್ದು
ಸೆರೆ-ಯ-ಲ್ಲಿ-ರು-ವು-ದನ್ನೂ
ಸೆರೆ-ಯಿಂದ
ಸೆರೆ-ಯಿ-ಲ್ಲಿ-ಡಿರಿ
ಸೆರೆ-ಹಿ-ಡಿ-ದನು
ಸೆರೆ-ಹಿ-ಡಿದು
ಸೆಳೆ-ತಕ್ಕೆ
ಸೆಳೆದ
ಸೆಳೆದು
ಸೆಳೆ-ದು-ಕೊಂ-ಡನು
ಸೆಳೆ-ದು-ಕೊಂಡು
ಸೆಳೆ-ದು-ಕೊ-ಳ್ಳು-ತ್ತಲೆ
ಸೆಳೆ-ಯಿತು
ಸೆಳೆ-ಯು-ತ್ತದೆ
ಸೆಳೆ-ಯುತ್ತಾ
ಸೆಳೆ-ಸೀ-ರೆಗೆ
ಸೇಡನ್ನು
ಸೇಡು
ಸೇತುವೆ
ಸೇತು-ವೆ-ಯಂ-ತಿದೆ
ಸೇತು-ವೆ-ಯಂ-ತಿ-ರುವ
ಸೇತು-ವೆ-ಯನ್ನು
ಸೇತು-ವೆ-ಯನ್ನೆ
ಸೇನನ
ಸೇನ-ನಿಗೆ
ಸೇನನು
ಸೇನಾ-ನಾ-ಯ-ಕರು
ಸೇನಾ-ನಿ-ಗಳು
ಸೇನಾ-ಪತಿ
ಸೇನಾ-ಪ-ತಿ-ಗಳು
ಸೇನಾ-ಪ-ತಿ-ಯ-ನ್ನಾಗಿ
ಸೇನಾ-ಪ-ತಿ-ಯಾದ
ಸೇನಾ-ಪ-ತಿ-ಯೊ-ಡನೆ
ಸೇನೆ
ಸೇನೆ-ಗಳ
ಸೇನೆ-ಗಳನ್ನೂ
ಸೇನೆ-ಗಳಲ್ಲಿ
ಸೇನೆ-ಗ-ಳೊ-ಡನೆ
ಸೇನೆಗೂ
ಸೇನೆಗೆ
ಸೇನೆಯ
ಸೇನೆ-ಯನ್ನು
ಸೇನೆ-ಯನ್ನೂ
ಸೇನೆ-ಯ-ನ್ನೆಲ್ಲ
ಸೇನೆ-ಯಿತ್ತು
ಸೇನೆಯು
ಸೇನೆಯೂ
ಸೇನೆ-ಯೊ-ಡನೆ
ಸೇನೆ-ಯೊ-ಡ-ನೆಯೂ
ಸೇರ-ದಂತೆ
ಸೇರ-ಬ-ಹುದು
ಸೇರ-ಬೇಕು
ಸೇರ-ಬೇ-ಕು-ಎಂದು
ಸೇರಿ
ಸೇರಿ-ಕೊಂಡ
ಸೇರಿ-ಕೊಂ-ಡನು
ಸೇರಿ-ಕೊಂ-ಡರು
ಸೇರಿ-ಕೊಂ-ಡಿತ್ತು
ಸೇರಿ-ಕೊಂ-ಡಿದ್ದ
ಸೇರಿ-ಕೊಂ-ಡಿ-ದ್ದಾನೆ
ಸೇರಿ-ಕೊಂ-ಡಿದ್ದು
ಸೇರಿ-ಕೊಂಡು
ಸೇರಿ-ಕೊ-ಳ್ಳ-ಬ-ಹುದು
ಸೇರಿ-ಕೊ-ಳ್ಳು-ತ್ತಾರೆ
ಸೇರಿಗೆ
ಸೇರಿತು
ಸೇರಿದ
ಸೇರಿ-ದನು
ಸೇರಿ-ದರು
ಸೇರಿ-ದರೂ
ಸೇರಿ-ದರೆ
ಸೇರಿ-ದವ
ಸೇರಿ-ದ-ವ-ನಲ್ಲ
ಸೇರಿ-ದ-ವನು
ಸೇರಿ-ದ-ವನೆ
ಸೇರಿ-ದ-ವ-ರಾ-ದರೂ
ಸೇರಿ-ದ-ವರು
ಸೇರಿ-ದ-ವ-ರೆಂದು
ಸೇರಿ-ದ-ವಳು
ಸೇರಿ-ದವು
ಸೇರಿ-ದ-ವು-ಗಳು
ಸೇರಿ-ದಾಗ
ಸೇರಿದು
ಸೇರಿ-ದುದು
ಸೇರಿ-ದುವು
ಸೇರಿದೆ
ಸೇರಿ-ದ್ದರು
ಸೇರಿ-ದ್ದಾನೆ
ಸೇರಿದ್ದು
ಸೇರಿ-ದ್ದು-ದನ್ನು
ಸೇರಿ-ದ್ದು-ದ-ರಿಂದ
ಸೇರಿ-ದ್ದೇವೆ
ಸೇರಿ-ಯಾರು
ಸೇರಿಯೂ
ಸೇರಿ-ರುವೆ
ಸೇರಿವೆ
ಸೇರಿ-ಸದೆ
ಸೇರಿ-ಸ-ಬೇಕು
ಸೇರಿಸಿ
ಸೇರಿ-ಸಿ-ಕೊಂಡ
ಸೇರಿ-ಸಿ-ಕೊ-ಳ್ಳು-ವು-ದಕ್ಕೆ
ಸೇರಿ-ಸುತ್ತಾ
ಸೇರಿ-ಹೋ-ಗಿ-ದೆ-ಆ-ತನ
ಸೇರಿ-ಹೋ-ಗು-ತ್ತಾನೆ
ಸೇರಿ-ಹೋ-ಯಿತು
ಸೇರು-ತ್ತದೆ
ಸೇರು-ತ್ತಲೆ
ಸೇರು-ತ್ತವೆ
ಸೇರು-ತ್ತಾನೆ
ಸೇರು-ತ್ತಾರೆ
ಸೇರುವ
ಸೇರು-ವಂ-ತಹ
ಸೇರು-ವಂತೆ
ಸೇರು-ವನು
ಸೇರು-ವ-ಷ್ಟ-ರಲ್ಲಿ
ಸೇರು-ವಿರಿ
ಸೇರು-ವುದೂ
ಸೇರು-ವುದೇ
ಸೇರುವೆ
ಸೇವಕ
ಸೇವ-ಕ-ನಂತೆ
ಸೇವ-ಕ-ನಾ-ಗಿದ್ದ
ಸೇವ-ಕ-ನಾ-ಗಿ-ರು-ವಾಗ
ಸೇವ-ಕ-ನಾದ
ಸೇವ-ಕ-ರಾ-ಗಿ-ದ್ದರು
ಸೇವ-ಕ-ರಾದ
ಸೇವ-ಕ-ರಾ-ದರು
ಸೇವ-ಕರು
ಸೇವ-ಕಿ-ಯರು
ಸೇವ-ಕಿ-ಯರೂ
ಸೇವಾ-ರೂ-ಪ-ವಾದ
ಸೇವಿ-ತ-ವಾದ
ಸೇವಿ-ತ-ವಾ-ಯಿತು
ಸೇವಿಸ
ಸೇವಿ-ಸ-ಬೇಕು
ಸೇವಿಸಿ
ಸೇವಿಸು
ಸೇವಿ-ಸು-ತ್ತಲೆ
ಸೇವಿ-ಸುತ್ತಾ
ಸೇವಿ-ಸು-ತ್ತಾರೆ
ಸೇವಿ-ಸು-ತ್ತಾಳೆ
ಸೇವಿ-ಸು-ತ್ತಿ-ದ್ದರು
ಸೇವಿ-ಸು-ತ್ತಿ-ದ್ದಳು
ಸೇವಿ-ಸು-ತ್ತಿವೆ
ಸೇವಿ-ಸುವ
ಸೇವೆ
ಸೇವೆ-ಗಾಗಿ
ಸೇವೆ-ಯನ್ನು
ಸೇವೆ-ಯಲ್ಲಿ
ಸೇವೆ-ಯ-ಲ್ಲಿಯೇ
ಸೇವೆ-ಯಿಂದ
ಸೇವೆ-ಯಿಂ-ದಲೇ
ಸೇವೈ-ಕ-ನಿಷ್ಠಾ
ಸೈ
ಸೈನಿ-ಕರೂ
ಸೈನಿ-ಕ-ರೆಲ್ಲ
ಸೈನ್ಯ
ಸೈನ್ಯಕ್ಕೂ
ಸೈನ್ಯಕ್ಕೆ
ಸೈನ್ಯ-ಗಳು
ಸೈನ್ಯ-ಗ-ಳೊ-ಡನೆ
ಸೈನ್ಯದ
ಸೈನ್ಯ-ದೊ-ಡನೆ
ಸೈನ್ಯ-ವನ್ನು
ಸೈನ್ಯ-ವ-ನ್ನೆಲ್ಲ
ಸೈನ್ಯ-ವೆಲ್ಲ
ಸೈನ್ಯ-ವೆ-ಲ್ಲ-ವನ್ನೂ
ಸೈನ್ಯ-ವೆ-ಲ್ಲವೂ
ಸೈರಣೆ
ಸೊಂಕು
ಸೊಂಟಕ್ಕೆ
ಸೊಂಟದ
ಸೊಂಟ-ದಲ್ಲಿ
ಸೊಂಡ-ಲಿ-ನಂತೆ
ಸೊಂಡ-ಲಿ-ನಿಂದ
ಸೊಂಡಿ-ಲನ್ನು
ಸೊಂಡಿ-ಲಿ-ನಿಂದ
ಸೊಂಪನ್ನೂ
ಸೊಂಪಾ
ಸೊಂಪಾದ
ಸೊಕ್ಕನ್ನು
ಸೊಕ್ಕಿ
ಸೊಕ್ಕು
ಸೊಗ-ಸಾಗಿ
ಸೊಗ-ಸಾದ
ಸೊಪ್ಪಾಗಿ
ಸೊಪ್ಪಾ-ಯಿತು
ಸೊಬ-ಗನ್ನು
ಸೊಬಗು
ಸೊರಗಿ
ಸೊಳ್ಳೆ
ಸೊಳ್ಳೆ-ಗ-ಳಂ-ತಿ-ರುವ
ಸೊಳ್ಳೆಯ
ಸೊಳ್ಳೆ-ಯಂತೆ
ಸೊಸೆ-ಇ-ವ-ರನ್ನು
ಸೊಸೆ-ಯ-ರೊ-ಡನೆ
ಸೊಸೆ-ಯಾದ
ಸೋಂಕು
ಸೋಂಕುವ
ಸೋಕಿ
ಸೋಕಿಸಿ
ಸೋಕು-ತ್ತಲೆ
ಸೋಕು-ತ್ತಿ-ದ್ದಂ-ತೆಯೇ
ಸೋಕು-ತ್ತಿ-ರು-ವು-ದ-ರಿಂದ
ಸೋಕುವ
ಸೋಜಿಗ
ಸೋಜಿ-ಗದ
ಸೋತ
ಸೋತಂ-ತೆಲ್ಲ
ಸೋತರೂ
ಸೋತ-ರೇನು
ಸೋತ-ವ-ನಂತೆ
ಸೋತ-ವರು
ಸೋತಾಗ
ಸೋತಾ-ಗಲೂ
ಸೋತಿವೆ
ಸೋತು
ಸೋತು-ದೆಲ್ಲ
ಸೋತೆ
ಸೋತೆ-ವೆಂದು
ಸೋತ್ರ-ಮಾಡಿ
ಸೋದರ
ಸೋದ-ರ-ತ್ತೆ-ಯನ್ನೂ
ಸೋದ-ರ-ತ್ತೆ-ಯಾದ
ಸೋದ-ರ-ಮಾ-ವನ
ಸೋದ-ರ-ಮಾ-ವ-ನನ್ನೆ
ಸೋದ-ರ-ಮಾ-ವ-ನಾದ
ಸೋದ-ರ-ಮಾ-ವ-ನೆಂದು
ಸೋದ-ರ-ರಾದ
ಸೋದ-ರ-ರಿಗೂ
ಸೋದ-ರರು
ಸೋದ-ರರೂ
ಸೋದ-ರರೇ
ಸೋದ-ರ-ರೊ-ಡ-ನೆಯೂ
ಸೋದ-ರ-ಳಿ-ಯ-ನನ್ನು
ಸೋದ-ರ-ಸೊಸೆ
ಸೋದ-ರಿ-ಯರು
ಸೋಮ
ಸೋಮಕ
ಸೋಮ-ನಾ-ಥಾ-ನಂದ
ಸೋಮ-ಪಾ-ನದ
ಸೋಮ-ಭಾ-ಗದ
ಸೋಮ-ಭಾ-ಗ-ವನ್ನು
ಸೋಮ-ರಸ
ಸೋಮ-ರ-ಸ-ವೆಂಬ
ಸೋಮ-ರೆಂಬ
ಸೋಮ-ವನ್ನು
ಸೋರಿ
ಸೋರಿ-ಹೋ-ಯಿತು
ಸೋರು-ತ್ತಿ-ರುವ
ಸೋರ್ಮು-ಡಿ-ಯಾಗಿ
ಸೋಲ-ದಂ-ತಹ
ಸೋಲದೆ
ಸೋಲನ್ನೆ
ಸೋಲಾದ
ಸೋಲಾ-ಯಿತು
ಸೋಲಿಸ
ಸೋಲಿ-ಸಲು
ಸೋಲಿ-ಸ-ಲೆಂದೊ
ಸೋಲಿಸಿ
ಸೋಲಿ-ಸಿದ
ಸೋಲು
ಸೋಲುತ್ತಾ
ಸೋಲು-ತ್ತಿ-ದ್ದರೂ
ಸೋಲುವ
ಸೋಲು-ವು-ದಿಲ್ಲ
ಸೌಂದರ್ಯ
ಸೌಂದ-ರ್ಯಕ್ಕೆ
ಸೌಂದ-ರ್ಯ-ಗಳನ್ನು
ಸೌಂದ-ರ್ಯ-ರ-ತ್ನ-ವನ್ನು
ಸೌಂದ-ರ್ಯ-ವನ್ನು
ಸೌಂದ-ರ್ಯ-ವನ್ನೂ
ಸೌಂದ-ರ್ಯ-ಸಾ-ರ-ದಿಂದ
ಸೌಖ್ಯ-ವನ್ನು
ಸೌಗಂ-ಧಿ-ಕಾ-ವ-ನವೂ
ಸೌಭರಿ
ಸೌಭ-ರಿ-ಪು-ಷಿಯು
ಸೌಭ-ರಿ-ಮ-ಹರ್ಷಿ
ಸೌಭ-ರಿಯ
ಸೌಭ-ರಿ-ಯಂ-ತಹ
ಸೌಭ-ರಿಯು
ಸೌಭ-ರಿ-ಯೆಂಬ
ಸೌಭ-ವೆಂಬ
ಸೌಭಾಗ್ಯ
ಸೌಭಾ-ಗ್ಯ-ವನ್ನು
ಸೌಮ್ಯ
ಸೌಮ್ಯ-ರೂ-ಪವು
ಸೌರಾಷ್ಟ್ರ
ಸೌವೀರ
ಸ್ಕ-ಸ್ಕಾಂದ
ಸ್ಕಂಧ
ಸ್ಕರ-ವಾ-ದುವು
ಸ್ಕರಿಸಿ
ಸ್ಕಾಂದ
ಸ್ಕಾಂದ-ಪು-ರಾಣ
ಸ್ಕಾರ-ಎಂದು
ಸ್ಕಾರ-ಮಾಡಿ
ಸ್ತನ
ಸ್ತನ-ಗಳ
ಸ್ತನಾ-ಗ್ರ-ಗಳಲ್ಲಿ
ಸ್ತನೈ-ರ್ವಿ-ಧರ್ತು
ಸ್ತವ
ಸ್ತಸ್ಯ
ಸ್ತುತಿ-ಯನ್ನು
ಸ್ತುತಿ-ಯಿಂದ
ಸ್ತುತಿ-ಸ-ಬೇಕೆ
ಸ್ತುತಿ-ಸ-ಹೊ-ರ-ಟ-ರುಹೇ
ಸ್ತುತಿಸಿ
ಸ್ತುತಿ-ಸಿತು
ಸ್ತುತಿ-ಸಿದ
ಸ್ತುತಿ-ಸಿ-ದನು
ಸ್ತುತಿ-ಸಿ-ದರು
ಸ್ತುತಿ-ಸಿ-ದಳು
ಸ್ತುತಿ-ಸಿ-ರುವ
ಸ್ತುತಿಸು
ಸ್ತುತಿ-ಸುತ್ತಾ
ಸ್ತುತಿ-ಸು-ತ್ತಾರೆ
ಸ್ತುತಿ-ಸು-ತ್ತಿ-ರುವ
ಸ್ತುತಿ-ಸು-ತ್ತಿ-ರು-ವಂ-ತೆಯೇ
ಸ್ತುತಿ-ಸು-ತ್ತಿ-ರು-ವನು
ಸ್ತುತಿ-ಸು-ವನು
ಸ್ತುತಿ-ಸು-ವು-ದ-ರ-ಲ್ಲಿಯೆ
ಸ್ತುತ್ಯ-ನಾಗಿ
ಸ್ತೇಜಸ್ತೇ
ಸ್ತೋತ್ರ
ಸ್ತೋತ್ರ-ದಿಂದ
ಸ್ತೋತ್ರ-ಪ್ರ-ವಾಹ
ಸ್ತೋತ್ರ-ಮಾ-ಡಿತು
ಸ್ತೋತ್ರ-ಮಾ-ಡಿದ
ಸ್ತೋತ್ರ-ಮಾ-ಡಿ-ದನು
ಸ್ತೋತ್ರ-ಮಾ-ಡಿ-ದ-ನು-ಸ-ಹಸ್ರ
ಸ್ತೋತ್ರ-ಮಾ-ಡಿ-ದರು
ಸ್ತೋತ್ರ-ಮಾ-ಡುತ್ತಾ
ಸ್ತೋತ್ರ-ಮಾ-ಡು-ತ್ತಾನೆ
ಸ್ತೋತ್ರ-ಮಾ-ಡು-ತ್ತಿ-ದ್ದನು
ಸ್ತೋತ್ರ-ಮಾ-ಡು-ತ್ತಿ-ರು-ವನು
ಸ್ತೋತ್ರ-ಮಾ-ಡು-ವಳು
ಸ್ತೋತ್ರ-ಮಾ-ಡು-ವ-ವ-ರನ್ನು
ಸ್ತೋತ್ರ-ರೂ-ಪ-ವಾದ
ಸ್ತೋತ್ರ-ವನ್ನು
ಸ್ತೋತ್ರವೇ
ಸ್ತೋತ್ರ-ಸ್ತೋ-ಭ-ಶ್ಛಂ-ದೋ-ಮಯಃ
ಸ್ತೋತ್ರಾರ್ಹ
ಸ್ತ್ರಿಯ-ಮ-ಕೃ-ತ-ವಿ-ರೂ-ಪಾಂ
ಸ್ತ್ರೀ
ಸ್ತ್ರೀಜಿತಃ
ಸ್ತ್ರೀಪ-ರಿ-ವಾ-ರ-ದೊ-ಡನೆ
ಸ್ತ್ರೀಪು-ರುಷ
ಸ್ತ್ರೀಪು-ರು-ಷರು
ಸ್ತ್ರೀಪು-ರು-ಷ-ರೂ-ಪದ
ಸ್ತ್ರೀಯ
ಸ್ತ್ರೀಯ-ರನ್ನೂ
ಸ್ತ್ರೀಯ-ರ-ನ್ನೆಲ್ಲ
ಸ್ತ್ರೀಯರು
ಸ್ತ್ರೀಯರೇ
ಸ್ತ್ರೀರೂಪು
ಸ್ತ್ರೀಲಂ-ಪ-ಟ-ನಾದ
ಸ್ತ್ರೀಲೋ-ಲ-ನಾದ
ಸ್ತ್ರೀಸಂ-ಗ-ವನ್ನು
ಸ್ತ್ರೋತ್ರ
ಸ್ತ್ರೋತ್ರ-ಮಾ-ಡುವ
ಸ್ಥಗಿತ
ಸ್ಥಲಿಗೆ
ಸ್ಥಲೇಷು
ಸ್ಥಳ
ಸ್ಥಳಕ್ಕೆ
ಸ್ಥಳ-ಗಳನ್ನೂ
ಸ್ಥಳ-ಗಳಲ್ಲಿ
ಸ್ಥಳ-ಗ-ಳಿಗೆ
ಸ್ಥಳ-ದಲ್ಲಿ
ಸ್ಥಳ-ದಿಂದ
ಸ್ಥಳ-ವನ್ನು
ಸ್ಥಳ-ವಿಲ್ಲ
ಸ್ಥಾಣು-ವಿ-ನಂತೆ
ಸ್ಥಾನ
ಸ್ಥಾನಕ್ಕೆ
ಸ್ಥಾನ-ಗಳಲ್ಲಿ
ಸ್ಥಾನ-ಗ-ಳ-ಲ್ಲಿಯೇ
ಸ್ಥಾನ-ಗಳು
ಸ್ಥಾನ-ಮಾನ
ಸ್ಥಾನ-ಮಾ-ನ-ಗಳನ್ನು
ಸ್ಥಾನ-ವನ್ನು
ಸ್ಥಾನ-ವಾ-ಯಿತು
ಸ್ಥಾನ-ವಿಲ್ಲ
ಸ್ಥಾನವು
ಸ್ಥಾನವೂ
ಸ್ಥಾನೇಷು
ಸ್ಥಾಪಿಸಿ
ಸ್ಥಾಪಿ-ಸಿ-ದನು
ಸ್ಥಾಪಿ-ಸಿ-ದು-ದಾ-ದ-ಮೇಲೆ
ಸ್ಥಾಪಿ-ಸುತ್ತ
ಸ್ಥಾಪಿ-ಸು-ವು-ದ-ಕ್ಕಾ-ಗಿಯೆ
ಸ್ಥಾಯಿ-ಯಾ-ಗದೆ
ಸ್ಥಾವರ
ಸ್ಥಾವ-ರ-ಗಳು
ಸ್ಥಿತಿ
ಸ್ಥಿತಿ-ಗ-ತಿ-ಗಳನ್ನು
ಸ್ಥಿತಿ-ಗ-ಳ-ಲ್ಲಿಯೂ
ಸ್ಥಿತಿ-ಯನ್ನು
ಸ್ಥಿತಿ-ಯಲ್ಲಿ
ಸ್ಥಿತಿ-ಯ-ಲ್ಲಿ-ದ್ದರು
ಸ್ಥಿತಿ-ಯ-ಲ್ಲಿ-ದ್ದೆವೆ
ಸ್ಥಿತಿ-ಯ-ಲ್ಲಿಯೇ
ಸ್ಥಿತಿ-ಯೆ-ಲ್ಲವೂ
ಸ್ಥಿರ
ಸ್ಥಿರ-ವಲ್ಲ
ಸ್ಥಿರ-ವಾಗಿ
ಸ್ಥಿರ-ವಾ-ಗಿ-ರು-ತ್ತದೆ
ಸ್ಥಿರ-ವಾ-ಗಿ-ರು-ತ್ತವೆ
ಸ್ಥಿರ-ವಾ-ಗಿ-ರು-ವಂತೆ
ಸ್ಥಿರ-ವಾ-ಗು-ವುದು
ಸ್ಥಿರ-ವಾದ
ಸ್ಥಿರ-ವಾ-ದು-ದಕ್ಕೆ
ಸ್ಥಿರ-ವಾ-ದುದು
ಸ್ಥಿರ-ವಾ-ಯಿತು
ಸ್ಥಿರ-ವೆಂದು
ಸ್ಥಿರವೇ
ಸ್ಥೂಲ
ಸ್ಥೂಲ-ದೆ-ಶೆ-ಯನ್ನು
ಸ್ಥೂಲ-ದೇಹ
ಸ್ಥೂಲ-ದೇ-ಹದ
ಸ್ಥೂಲ-ದೇ-ಹ-ದಿಂದ
ಸ್ಥೂಲ-ದೇ-ಹವು
ಸ್ಥೂಲ-ಪ-ರಿ-ಚ-ಯ-ವನ್ನು
ಸ್ಥೂಲ-ರೂಪ
ಸ್ಥೂಲ-ರೂ-ಪ-ದ-ಲ್ಲಿ-ರುವ
ಸ್ಥೂಲ-ರೂ-ಪ-ವನ್ನು
ಸ್ಥೂಲ-ರೂ-ಪ-ವಾಗಿ
ಸ್ಥೂಲ-ವಾಗಿ
ಸ್ಥೂಲ-ಶ-ರೀ-ರ-ವನ್ನು
ಸ್ಥೂಲ-ಸೂಕ್ಷ್ಮ
ಸ್ನಾನ
ಸ್ನಾನ-ಕ್ಕಿ-ಳಿ-ದನು
ಸ್ನಾನ-ಕ್ಕಿ-ಳಿ-ದರು
ಸ್ನಾನ-ಕ್ಕೆಂದು
ಸ್ನಾನ-ಪಾ-ನ-ಗಳನ್ನು
ಸ್ನಾನ-ಮಾಡಿ
ಸ್ನಾನ-ಮಾ-ಡಿ-ದನು
ಸ್ನಾನ-ಮಾ-ಡಿ-ದರೆ
ಸ್ನಾನ-ಮಾ-ಡಿಸಿ
ಸ್ನಾನ-ಮಾ-ಡಿ-ಸಿತು
ಸ್ನಾನ-ಮಾಡು
ಸ್ನಾನ-ಮಾ-ಡುತ್ತಾ
ಸ್ನಾನ-ಮಾ-ಡು-ತ್ತಿದ್ದ
ಸ್ನಾನ-ಮಾ-ಡು-ತ್ತೇನೆ
ಸ್ನಾನ-ವನ್ನು
ಸ್ನಾನ-ಸಂ-ಧ್ಯೆ-ಗಳನ್ನು
ಸ್ನಾನಾದಿ
ಸ್ನಾನಾ-ನಂ-ತರ
ಸ್ನಿಗ್ಧ-ವಾಗಿ
ಸ್ನೇಹ
ಸ್ನೇಹಕ್ಕೆ
ಸ್ನೇಹ-ದಿಂ-ದಿತ್ತು
ಸ್ನೇಹ-ಬಂ-ಧನ
ಸ್ನೇಹ-ಮ-ಯ-ವಾ-ಗಿರ
ಸ್ನೇಹ-ವನ್ನು
ಸ್ನೇಹ-ವಾ-ಯಿತು
ಸ್ನೇಹ-ವೆಲ್ಲಿ
ಸ್ನೇಹಿತ
ಸ್ನೇಹಿ-ತ-ನಂ-ತಿ-ರು-ವುದು
ಸ್ಪಂದನ
ಸ್ಪಂದಿ-ಸು-ತ್ತಿ-ರ-ಬೇಕು
ಸ್ಪತಿಯು
ಸ್ಪರ್ಧಿ-ಯನ್ನು
ಸ್ಪರ್ಧೆ-ಯಿಂದ
ಸ್ಪರ್ಶ
ಸ್ಪರ್ಶ-ಗುಣ
ಸ್ಪರ್ಶನ
ಸ್ಪರ್ಶ-ನ-ವಿ-ಷ್ಫು-ಲಿಂಗೇ
ಸ್ಪರ್ಶಾ-ದಿ-ಗಳನ್ನು
ಸ್ಪರ್ಷ-ದಿಂದ
ಸ್ಪಷ್ಟ
ಸ್ಪಷ್ಟ-ವಾಗಿ
ಸ್ಪಷ್ಟ-ವಾದ
ಸ್ಪಷ್ಟ-ವೇ-ಅ-ವ-ರಿ-ಗೆಲ್ಲ
ಸ್ಪೃಶ-ತಾ-ಮಾ-ತ್ಮನಾ
ಸ್ಪೃಶಾಂ-ಘ್ರಿಂ
ಸ್ಪೃಹ-ಯಸೇ
ಸ್ಫಟಿಕ
ಸ್ಫಟಿ-ಕದ
ಸ್ಫಟಿ-ಕ-ದಂತೆ
ಸ್ಫುಟಂ-ಯಾರು
ಸ್ಫುರಿ-ಸು-ತ್ತಿ-ರು-ತ್ತದೆ
ಸ್ಫೂರ್ತಿ
ಸ್ಮರಣ
ಸ್ಮರ-ಣ-ಮಾ-ತ್ರ-ದಿಂದ
ಸ್ಮರಣೆ
ಸ್ಮರ-ಣೆ-ಯಿತ್ತು
ಸ್ಮರ-ಣೆ-ಯಿದೆ
ಸ್ಮರ-ಣೆ-ಯಿದ್ದ
ಸ್ಮರ-ಣೆ-ಯಿಲ್ಲ
ಸ್ಮರ-ಣೆ-ಯುಂ-ಟಾ-ಗುವ
ಸ್ಮರ-ಣೆ-ಯುಂ-ಟಾ-ಯಿತು
ಸ್ಮರ-ಣೆಯೂ
ಸ್ಮರತಿ
ಸ್ಮರ-ರುಜ
ಸ್ಮರಿಸ
ಸ್ಮರಿ-ಸ-ಬೇ-ಕಾದ
ಸ್ಮರಿ-ಸಿ-ಕೊಂಡು
ಸ್ಮರಿ-ಸಿ-ಕೊ-ಳ್ಳು-ತ್ತಾ-ನೆಯೊ
ಸ್ಮರಿ-ಸು-ವ-ವರು
ಸ್ಮಶಾನ
ಸ್ಮಾರಕ
ಸ್ಮಾರ್ತ-ರಾ-ದರೂ
ಸ್ಮೃತಂ
ಸ್ಮೃತಃ
ಸ್ಮೃತಿ
ಸ್ಮೃತಿ-ಗಳ
ಸ್ಮೃತಿ-ಗಳಲ್ಲಿ
ಸ್ಮೃತಿ-ಗಳು
ಸ್ಮೃತಿ-ಲ-ಬ್ಧ-ವಾ-ಯಿತು
ಸ್ಮೃತ್ವಾ
ಸ್ಯಮಂತ
ಸ್ಯಮಂ-ತಕ
ಸ್ಯಮಂ-ತ-ಕ-ಮಣಿ
ಸ್ಯಮಂ-ತ-ಕ-ಮ-ಣಿ-ಗಾಗಿ
ಸ್ಯಮಂ-ತ-ಕ-ಮ-ಣಿ-ಯನ್ನು
ಸ್ಯಮಂ-ತ-ಕ-ಮ-ಣಿ-ಯೊ-ಡನೆ
ಸ್ಯಮಂ-ತ-ಕ-ರತ್ನ
ಸ್ಯಮಂ-ತ-ಕ-ವೆಂಬ
ಸ್ಯಮಂ-ತ-ಪಂ-ಚ-ಕ-ದ-ಲ್ಲಿಯೆ
ಸ್ಯಮಂ-ತ-ಪಂ-ಚ-ಕ-ವೆಂಬ
ಸ್ಯುಃ
ಸ್ವ
ಸ್ವಂತ
ಸ್ವಕೃತ
ಸ್ವಗೋ-ಭಿಃ
ಸ್ವಚ್ಛ-ವಾ-ಗಿ-ರ-ಬೇಕು
ಸ್ವಚ್ಛ-ವಾದ
ಸ್ವಜನ
ಸ್ವಜ-ನಾ-ಸ್ತ-ಸ-ಮ-ಸ್ತ-ಮಲಂ
ಸ್ವತಂತ್ರ
ಸ್ವತಂ-ತ್ರ-ನಾಗಿ
ಸ್ವತಂ-ತ್ರ-ನಾದ
ಸ್ವತಂ-ತ್ರ-ನಾ-ದ-ವನು
ಸ್ವತಂ-ತ್ರ-ರೆಂದು
ಸ್ವತಂ-ತ್ರ-ವಾ-ದುದು
ಸ್ವತಃ
ಸ್ವತೇ-ಜ-ಸಾ-ಗ್ರಸ್ತ
ಸ್ವತ್ತನ್ನು
ಸ್ವತ್ತಾ-ಗ-ಬೇ-ಕೆಂ-ಬುದು
ಸ್ವತ್ತಾ-ಯಿತು
ಸ್ವತ್ತೆಂ-ಬು-ದನ್ನೆ
ಸ್ವದಂ-ಷ್ಟ್ರ-ಯೋ-ನ್ನಿ-ತ-ಧರೋ
ಸ್ವಧರಂ
ಸ್ವಧ-ರ್ಮ-ದಲ್ಲಿ
ಸ್ವಧ-ರ್ಮ-ವನ್ನು
ಸ್ವಧ-ರ್ಮ-ವೆಂದರೆ
ಸ್ವಧಾ
ಸ್ವನಾ-ಮ-ಭಿಃ
ಸ್ವನೇನ
ಸ್ವಪಿತಿ
ಸ್ವಪ್ನ
ಸ್ವಪ್ನ-ಜ-ಗ-ತ್ತನ್ನು
ಸ್ವಪ್ನದ
ಸ್ವಪ್ನ-ದಲ್ಲಿ
ಸ್ವಪ್ನ-ವೊಂದು
ಸ್ವಪ್ನ-ಸುಂ-ದ-ರಿ-ಯಂ-ತಿದ್ದ
ಸ್ವಪ್ರ-ಕಾ-ಶ-ನಾದ
ಸ್ವಪ್ರ-ಕಾಶಿ
ಸ್ವಪ್ರ-ಯೋ-ಜ-ನಕ್ಕೆ
ಸ್ವಬು-ದ್ಧಿ-ಯಿಂ-ದಲೊ
ಸ್ವಬು-ದ್ಧಿ-ಯಿ-ಲ್ಲದ
ಸ್ವಭಾವ
ಸ್ವಭಾ-ವಕ್ಕೆ
ಸ್ವಭಾ-ವ-ಗಳನ್ನು
ಸ್ವಭಾ-ವ-ಗಳಲ್ಲಿ
ಸ್ವಭಾ-ವ-ಗ-ಳಾ-ದವು
ಸ್ವಭಾ-ವ-ಗಳು
ಸ್ವಭಾ-ವ-ಚ-ಪ-ಲೆ-ಯಾದ
ಸ್ವಭಾ-ವತಃ
ಸ್ವಭಾ-ವ-ದಲ್ಲಿ
ಸ್ವಭಾ-ವ-ದ-ವ-ರಾ-ಗಿದ್ದ
ಸ್ವಭಾ-ವ-ದ-ವ-ರಿಗೆ
ಸ್ವಭಾ-ವ-ದ-ವರು
ಸ್ವಭಾ-ವ-ವ-ನ್ನ-ರಿತ
ಸ್ವಭಾ-ವ-ವನ್ನು
ಸ್ವಭಾ-ವ-ವಾಗಿ
ಸ್ವಭಾ-ವವು
ಸ್ವಭಾ-ವ-ವು-ಳ್ಳು-ದಾ-ದು-ದ-ರಿಂದ
ಸ್ವಭಾ-ವವೇ
ಸ್ವಮಾ-ಯಯಾ
ಸ್ವಯಂ
ಸ್ವಯಂ-ಪ್ರ-ಕಾಶ
ಸ್ವಯಂ-ಭು-ವಾದ
ಸ್ವಯಂ-ವರ
ಸ್ವಯಂ-ವ-ರ-ಕಾ-ಲ-ದಲ್ಲಿ
ಸ್ವಯಂ-ವ-ರಕ್ಕೆ
ಸ್ವಯಂ-ವ-ರ-ದಲ್ಲಿ
ಸ್ವಯಂ-ವ-ರ-ವನ್ನು
ಸ್ವಯೂ-ಥ-ಕ-ಲ-ಹ-ದಲ್ಲಿ
ಸ್ವರ-ಗಳಲ್ಲಿ
ಸ್ವರ-ದಿಂದ
ಸ್ವರೂಪ
ಸ್ವರೂ-ಪ-ಜ್ಞಾ-ನ-ದಿಂದ
ಸ್ವರೂ-ಪ-ನಾ-ಗಿಯೂ
ಸ್ವರೂ-ಪ-ನಾದ
ಸ್ವರೂ-ಪನು
ಸ್ವರೂ-ಪನೂ
ಸ್ವರೂ-ಪ-ನೆಂದೂ
ಸ್ವರೂ-ಪನೇ
ಸ್ವರೂ-ಪ-ವನ್ನು
ಸ್ವರೂ-ಪ-ವನ್ನೂ
ಸ್ವರೂ-ಪ-ವಾಗಿ
ಸ್ವರೂ-ಪ-ವಾ-ಗಿತ್ತು
ಸ್ವರೂ-ಪವು
ಸ್ವರೂ-ಪ-ವೆಂತ
ಸ್ವರೂ-ಪ-ವೆಂದು
ಸ್ವರೂ-ಪ-ವೆ-ನಿಸಿ
ಸ್ವರೂ-ಪ-ವೆ-ನಿ-ಸಿದ
ಸ್ವರೂ-ಪವೇ
ಸ್ವರೂ-ಪ-ವೇನು
ಸ್ವರೂಪಿ
ಸ್ವರೂ-ಪಿ-ಯಾದ
ಸ್ವರೂ-ಪಿ-ಯೆಂಬ
ಸ್ವರ್ಗ
ಸ್ವರ್ಗ-ನ-ರಕ
ಸ್ವರ್ಗ-ಕ್ಕಿಂ-ತಲೂ
ಸ್ವರ್ಗಕ್ಕೆ
ಸ್ವರ್ಗ-ಗಳು
ಸ್ವರ್ಗ-ಗ-ಳೆ-ನ್ನು-ತ್ತಾರೆ
ಸ್ವರ್ಗದ
ಸ್ವರ್ಗ-ದಂ-ತಿತ್ತು
ಸ್ವರ್ಗ-ದ-ಮೇಲೆ
ಸ್ವರ್ಗ-ದಲ್ಲಿ
ಸ್ವರ್ಗ-ದ-ಲ್ಲಿದ್ದ
ಸ್ವರ್ಗ-ದ-ಲ್ಲಿ-ರುವ
ಸ್ವರ್ಗ-ದಿಂದ
ಸ್ವರ್ಗ-ಪ-ದ-ವಿ-ಯನ್ನು
ಸ್ವರ್ಗ-ಲೋ-ಕಕ್ಕೆ
ಸ್ವರ್ಗ-ಲೋ-ಕದ
ಸ್ವರ್ಗ-ಲೋ-ಕವು
ಸ್ವರ್ಗ-ವನ್ನು
ಸ್ವರ್ಗ-ವನ್ನೊ
ಸ್ವರ್ಗ-ವಾ-ದ-ರೇನು
ಸ್ವರ್ಗ-ವೆಂಬ
ಸ್ವರ್ಗವೇ
ಸ್ವರ್ಗ-ವೇ-ರಿ-ದ-ವರು
ಸ್ವರ್ಗ-ಸು-ಖ-ಕ್ಕಾಗಿ
ಸ್ವರ್ಗ-ಸು-ಖ-ವನ್ನು
ಸ್ವರ್ಗ-ಸು-ಖವೂ
ಸ್ವರ್ಗಾದಿ
ಸ್ವರ್ಗಾ-ದಿ-ಗ-ಳಿಗೂ
ಸ್ವರ್ಣ-ಸ್ವಪ್ನ
ಸ್ವರ್ಣ-ಸ್ವ-ಪ್ನ-ದಂ-ತಿದ್ದ
ಸ್ವರ್ಣಾ-ಭ-ರ-ಣ-ಗಳಿಂದ
ಸ್ವರ್ಭಾನು
ಸ್ವರ್ಭಾ-ನು-ವಿನ
ಸ್ವಲ್ಪ
ಸ್ವಲ್ಪ-ಕಾಲ
ಸ್ವಲ್ಪ-ಕಾ-ಲಕ್ಕೆ
ಸ್ವಲ್ಪ-ಕಾ-ಲ-ವಾದ
ಸ್ವಲ್ಪ-ವನ್ನು
ಸ್ವಲ್ಪ-ವಾ-ದರೂ
ಸ್ವಲ್ಪವೂ
ಸ್ವಲ್ಪವೆ
ಸ್ವಲ್ಪ-ಸ್ವ-ಲ್ಪ-ವನ್ನು
ಸ್ವಷ್ಟ-ವಾಗಿ
ಸ್ವಸ್ತಿ-ವಾ-ಚ-ನ-ವನ್ನು
ಸ್ವಸ್ತ್ಯಾಸ್ತ
ಸ್ವಸ್ಥ
ಸ್ವಸ್ಥ-ಚಿ-ತ್ತ-ದಿಂದ
ಸ್ವಸ್ಥಳ
ಸ್ವಸ್ಥಾ-ನಕ್ಕೆ
ಸ್ವಸ್ಥಾ-ನ-ಗ-ಳಿಗೆ
ಸ್ವಸ್ಥಾ-ನ-ವನ್ನು
ಸ್ವಸ್ಥಾ-ನ-ವಾದ
ಸ್ವಸ್ವ-ರೂಪ
ಸ್ವಸ್ವ-ರೂ-ಪ-ಜ್ಞಾ-ನ-ವುಂ-ಟಾ-ಗು-ತ್ತದೆ
ಸ್ವಸ್ವ-ರೂ-ಪ-ದಿಂದ
ಸ್ವಸ್ವ-ರೂ-ಪ-ವನ್ನು
ಸ್ವಾಂ
ಸ್ವಾಗ-ತ-ಮಾ-ಸ್ಯ-ತಾಂ
ಸ್ವಾಧೀ-ನಕ್ಕೆ
ಸ್ವಾಧೀ-ನ-ಮಾ-ಡಿ-ಕೊಂಡು
ಸ್ವಾಧ್ಯಾಯ
ಸ್ವಾನಂ-ದ-ದಲ್ಲಿ
ಸ್ವಾಭಾ
ಸ್ವಾಭಾ-ವಿ-ಕ-ವಾಗಿ
ಸ್ವಾಭಾ-ವಿ-ಕ-ವಾ-ಗಿಯೇ
ಸ್ವಾಮಿ
ಸ್ವಾಮಿ-ಸೇ-ವಕ
ಸ್ವಾಮಿಗೆ
ಸ್ವಾಮಿ-ದ್ರೋ-ಹಿ-ಯಾದ
ಸ್ವಾಮಿನ್
ಸ್ವಾಮಿ-ಯನ್ನು
ಸ್ವಾಮಿ-ಯಲ್ಲಿ
ಸ್ವಾಮಿ-ಯಾಗಿ
ಸ್ವಾಮಿ-ಯಾ-ಗಿದ್ದ
ಸ್ವಾಮಿ-ಯಾದ
ಸ್ವಾಮಿ-ಯಾ-ದನು
ಸ್ವಾಮಿ-ಯಾ-ದರೂ
ಸ್ವಾಮಿ-ಯಾ-ದುದು
ಸ್ವಾಮಿಯು
ಸ್ವಾಮಿಯೂ
ಸ್ವಾಯಂಭು
ಸ್ವಾಯಂ-ಭು-ಮ-ನು-ವಿನ
ಸ್ವಾಯಂ-ಭುವ
ಸ್ವಾಯಂ-ಭು-ವನ
ಸ್ವಾಯಂ-ಭು-ವನು
ಸ್ವಾಯಂ-ಭು-ವ-ಮನು
ಸ್ವಾಯಂ-ಭು-ವ-ಮ-ನು-ವನ್ನು
ಸ್ವಾಯಂ-ಭು-ವ-ಮ-ನು-ವಿನ
ಸ್ವಾಯಂ-ಭು-ವ-ಮ-ನುವು
ಸ್ವಾಯಂ-ಭು-ವಿಗೆ
ಸ್ವಾಯಂ-ಭು-ವಿನ
ಸ್ವಾಯುಂ-ಭು-ವ-ಮ-ನು-ವಿನ
ಸ್ವಾರ-ಸ್ಯ-ವಾ-ಗಿದೆ
ಸ್ವಾರ್ಥ
ಸ್ವಾರ್ಥ-ದಲ್ಲಿ
ಸ್ವಾರ್ಥ-ಪ-ರ-ರಾದ
ಸ್ವಾರ್ಥ-ವಿಲ್ಲ
ಸ್ವಾಹಾ
ಸ್ವಾಹಾ-ಎಂದು
ಸ್ವಾಹಾ-ದೇವಿ
ಸ್ವೀಕರಿ
ಸ್ವೀಕ-ರಿ-ಸ-ದಿ-ರು-ವುದು
ಸ್ವೀಕ-ರಿ-ಸ-ಬ-ಹು-ದಲ್ಲ
ಸ್ವೀಕ-ರಿ-ಸ-ಬ-ಹುದು
ಸ್ವೀಕ-ರಿ-ಸ-ಬೇಕು
ಸ್ವೀಕ-ರಿ-ಸ-ಲಿಲ್ಲ
ಸ್ವೀಕ-ರಿ-ಸಲು
ಸ್ವೀಕ-ರಿಸಿ
ಸ್ವೀಕ-ರಿ-ಸಿದ
ಸ್ವೀಕ-ರಿ-ಸಿ-ದನು
ಸ್ವೀಕ-ರಿ-ಸಿ-ದರು
ಸ್ವೀಕ-ರಿ-ಸಿ-ದ್ದರು
ಸ್ವೀಕ-ರಿ-ಸಿದ್ದು
ಸ್ವೀಕ-ರಿ-ಸಿ-ರುವ
ಸ್ವೀಕ-ರಿಸು
ಸ್ವೀಕ-ರಿ-ಸುತ್ತ
ಸ್ವೀಕ-ರಿ-ಸುತ್ತಾ
ಸ್ವೀಕ-ರಿ-ಸು-ತ್ತಾನೆ
ಸ್ವೀಕ-ರಿ-ಸು-ತ್ತಿರು
ಸ್ವೀಕ-ರಿ-ಸು-ತ್ತೇನೆ
ಸ್ವೀಕ-ರಿ-ಸುವ
ಸ್ವೀಕ-ರಿ-ಸು-ವಂತೆ
ಸ್ವೀಕ-ರಿ-ಸು-ವ-ನೆಂದು
ಸ್ವೀಕ-ರಿ-ಸು-ವು-ದಾಗಿ
ಸ್ವೀಕ-ರಿ-ಸು-ವು-ದಿಲ್ಲ
ಸ್ವೀಕಾರ
ಸ್ವೀಕಾ-ರ-ಮಾ-ಡಿಯೆ
ಸ್ಸನ್ನು
ಸ್ಸೇಮ
ಹಂಗಿಸಿ
ಹಂಗಿ-ಸು-ತ್ತಿದ್ದಿ
ಹಂಗಿ-ಸು-ತ್ತಿದ್ದೆ
ಹಂಗು-ತೊ-ರೆದು
ಹಂಗು-ದೊ-ರೆದು
ಹಂಚ
ಹಂಚಲು
ಹಂಚಿ
ಹಂಚಿ-ಕೊಂ-ಡರು
ಹಂಚಿ-ಕೊಂಡು
ಹಂಚಿ-ಕೊ-ಟ್ಟನು
ಹಂಚಿ-ಕೊ-ಟ್ಟಿ-ದ್ದಾನೆ
ಹಂಚಿ-ಕೊಟ್ಟು
ಹಂಚಿ-ದಳು
ಹಂಚಿ-ದು-ದಾ-ಯಿತು
ಹಂಚಿ-ದ್ದಾನೆ
ಹಂಚಿಯ
ಹಂಚು-ತ್ತೇನೆ
ಹಂಡೆಯ
ಹಂಡೆ-ಯಲ್ಲಿ
ಹಂದಿ
ಹಂದಿಯ
ಹಂದಿ-ಯಾಗಿ
ಹಂದಿಯು
ಹಂಬಲಿ
ಹಂಬ-ಲಿಸ
ಹಂಬ-ಲಿಸಿ
ಹಂಬ-ಲಿ-ಸು-ತ್ತಿ-ದ್ದನು
ಹಂಬ-ಲಿ-ಸು-ತ್ತಿ-ರುವ
ಹಂಬಲು
ಹಂಸ
ಹಂಸಂ
ಹಂಸಕ್ಕೆ
ಹಂಸ-ಗ-ಳಾಗಿ
ಹಂಸ-ಗು-ಹ್ಯ-ವೆಂಬ
ಹಂಸ-ತೂ-ಲಿ-ಕಾ-ತ-ಲ್ಪ-ಗಳು
ಹಂಸ-ದಂತೆ
ಹಂಸನೆ
ಹಂಸ-ರಾದ
ಹಂಸರು
ಹಂಸ-ವ-ನ್ನೇರಿ
ಹಂಸ-ವೆಂದರೆ
ಹಕ್ಕಿ
ಹಕ್ಕಿ-ಗಳ
ಹಕ್ಕಿ-ಗ-ಳಂತೆ
ಹಕ್ಕಿ-ಗ-ಳಾಗಿ
ಹಕ್ಕಿ-ಗಳು
ಹಕ್ಕಿ-ಗ-ಳೆ-ರಡೂ
ಹಕ್ಕಿ-ಗಳೇ
ಹಕ್ಕಿಗೆ
ಹಕ್ಕಿ-ಯಂತೂ
ಹಕ್ಕಿ-ಯನ್ನು
ಹಕ್ಕಿಯೂ
ಹಕ್ಕು-ಗಳೂ
ಹಗ-ರಣ
ಹಗ-ಲನ್ನು
ಹಗ-ಲಾ-ಗಲಿ
ಹಗ-ಲಿಗೆ
ಹಗ-ಲಿನ
ಹಗ-ಲಿ-ರ-ಳು-ಗಳು
ಹಗ-ಲಿ-ರು-ಳಿನ
ಹಗ-ಲಿ-ರುಳು
ಹಗ-ಲಿ-ರುಳೂ
ಹಗ-ಲಿ-ರು-ಳೂ-ಕೂ-ತಾಗ
ಹಗ-ಲಿ-ರು-ಳೂ-ನಿಂ-ತಿ-ರು-ವಾಗ
ಹಗ-ಲಿ-ರು-ಳೆಂಬ
ಹಗ-ಲಿ-ರು-ಳೆ-ನ್ನದೆ
ಹಗಲು
ಹಗ-ಲು-ರಾತ್ರಿ
ಹಗ-ಲು-ರಾ-ತ್ರಿ-ಗ-ಳಿಲ್ಲ
ಹಗಲೂ
ಹಗ-ಲೆಲ್ಲ
ಹಗುರ
ಹಗೆ
ಹಗೆ-ಗಳ
ಹಗೆ-ಗಳನ್ನು
ಹಗೆ-ಗಳನ್ನೆಲ್ಲ
ಹಗೆ-ಗ-ಳಿಗೆ
ಹಗೆ-ಗ-ಳೊ-ಡನೆ
ಹಗ್ಗ
ಹಗ್ಗ-ಗಳನ್ನು
ಹಗ್ಗ-ದಂತೆ
ಹಗ್ಗ-ದಿಂದ
ಹಗ್ಗ-ಬೇ-ಕಲ್ಲ
ಹಗ್ಗ-ವನ್ನು
ಹಗ್ಗ-ವಾ-ಗಲು
ಹಗ್ಗ-ವಾ-ಗು-ವಂತೆ
ಹಗ್ಗ-ವಾ-ಗು-ವು-ದಕ್ಕೆ
ಹಗ್ಗ-ವೆಂದು
ಹಗ್ಗ-ವೆಂ-ದುದು
ಹಚ್ಚ
ಹಚ್ಚ-ಹ-ಸು-ರಾಗಿ
ಹಚ್ಚ-ಹ-ಸು-ರಾದ
ಹಚ್ಚಿ
ಹಚ್ಚಿ-ಕೊಂ-ಡರು
ಹಚ್ಚಿ-ಕೊ-ಳ್ಳದೆ
ಹಚ್ಚಿ-ಕೊ-ಳ್ಳ-ಬೇಕು
ಹಚ್ಚಿ-ಕೊ-ಳ್ಳ-ಬೇಡ
ಹಚ್ಚಿ-ಕೊ-ಳ್ಳು-ವುದೆ
ಹಚ್ಚಿದ
ಹಚ್ಚಿ-ದ-ವಳು
ಹಟ
ಹಟ-ಮಾ-ರಿ-ಯಾಗಿ
ಹಟ-ವನ್ನು
ಹಟ-ಹಿಡಿ
ಹಡ-ಗನ್ನು
ಹಡ-ಗ-ನ್ನೇ-ರಿ-ದನು
ಹಡಗಿ
ಹಡ-ಗಿಗೆ
ಹಡ-ಗಿ-ನಂತೆ
ಹಡಗು
ಹಡ-ಗು-ಗಳನ್ನು
ಹಣ
ಹಣ-ಕಾ-ಸಿನ
ಹಣ-ಕಾಸು
ಹಣಕ್ಕೆ
ಹಣ-ದೊ-ಡನೆ
ಹಣ-ವನ್ನು
ಹಣ-ವನ್ನೆ
ಹಣ-ವೆಲ್ಲ
ಹಣ-ವೊಂದೆ
ಹಣೆಗೆ
ಹಣೆಯ
ಹಣೆ-ಯ-ಬ-ರಹ
ಹಣೆ-ಯ-ಮೇ-ಲಿನ
ಹಣೆ-ಯಲ್ಲಿ
ಹಣೆ-ಹಣೆ
ಹಣ್ಣನ್ನು
ಹಣ್ಣಿನ
ಹಣ್ಣಿ-ನಂತೆ
ಹಣ್ಣಿ-ನಲ್ಲಿ
ಹಣ್ಣು
ಹಣ್ಣು-ಕಾ-ಯಿ-ಗಳನ್ನು
ಹಣ್ಣು-ಗಳ
ಹಣ್ಣು-ಗಳನ್ನು
ಹಣ್ಣು-ಗಳನ್ನೂ
ಹಣ್ಣು-ಗಳಿಂದ
ಹಣ್ಣು-ಗಳು
ಹಣ್ಣು-ಗ-ಳು-ದು-ರಿ-ದವು
ಹಣ್ಣು-ಗ-ಳೆಂ-ದರೆ
ಹಣ್ಣು-ಗ-ಳೆಂಬ
ಹತ
ಹತ-ನಾಗಿ
ಹತ-ನಾ-ಗಿದ್ದ
ಹತ-ನಾದು
ಹತ-ಭಾ-ಗ್ಯ-ರಾದ
ಹತ-ರಾ-ಗಿ-ಹೋ-ದ-ಮೇಲೆ
ಹತ-ರಾ-ದರು
ಹತ-ರಾ-ದು-ದನ್ನು
ಹತೋ-ಟಿಗೆ
ಹತೋ-ಟಿಯ
ಹತೋ-ಟಿ-ಯ-ಲ್ಲಿ-ಟ್ಟು-ಕೊ-ಳ್ಳು-ವುದು
ಹತ್ತ-ನೆಯ
ಹತ್ತ-ನೆ-ಯದು
ಹತ್ತ-ಬೇ-ಕಾ-ದರೆ
ಹತ್ತ-ಲಿಲ್ಲ
ಹತ್ತಿ
ಹತ್ತಿ-ಕೊಂ-ಡರು
ಹತ್ತಿ-ಕೊಂ-ಡಿದ್ದ
ಹತ್ತಿ-ಕೊಂಡು
ಹತ್ತಿತು
ಹತ್ತಿ-ದರು
ಹತ್ತಿ-ದು-ದನ್ನು
ಹತ್ತಿರ
ಹತ್ತಿ-ರಕ್ಕೆ
ಹತ್ತಿ-ರಕ್ಕೇ
ಹತ್ತಿ-ರದ
ಹತ್ತಿ-ರ-ದಲ್ಲಿ
ಹತ್ತಿ-ರ-ದ-ಲ್ಲಿದ್ದ
ಹತ್ತಿ-ರ-ದ-ಲ್ಲಿಯೆ
ಹತ್ತಿ-ರ-ದ-ಲ್ಲಿಯೇ
ಹತ್ತಿ-ರ-ದಲ್ಲೆ
ಹತ್ತಿ-ರ-ವಾಗಿ
ಹತ್ತಿ-ರ-ವಾ-ಗಿ-ದೆ-ಯಾ-ದರೂ
ಹತ್ತಿ-ರ-ವಾ-ಗುತ್ತಾ
ಹತ್ತಿ-ರ-ವಾ-ಗು-ತ್ತಿತ್ತು
ಹತ್ತಿ-ರ-ವಾ-ದಾಗ
ಹತ್ತಿ-ರ-ವಾ-ದು-ದ-ರಿಂದ
ಹತ್ತಿ-ರವೆ
ಹತ್ತಿ-ರವೇ
ಹತ್ತಿ-ರ-ಹೋಗಿ
ಹತ್ತಿ-ಹೋ-ಗಿದೆ
ಹತ್ತು
ಹತ್ತು-ಜನ
ಹತ್ತು-ವು-ದಿಲ್ಲ
ಹತ್ತು-ಸ-ಹಸ್ರ
ಹತ್ತು-ಸಾ-ವಿರ
ಹತ್ತು-ಸಾ-ವಿ-ರ-ಜನ
ಹತ್ತೊಂ-ಬ-ತ್ತ-ರಲ್ಲಿ
ಹತ್ತೊತ್ತಿ
ಹತ್ಯಾ-ದೋಷ
ಹದ-ಗೊ-ಳಿ-ಸಿ-ದಾಗ
ಹದವಾ
ಹದಿ
ಹದಿ-ನಾ-ರ-ನೆಯ
ಹದಿ-ನಾರು
ಹದಿ-ನಾ-ರು-ಸಾ-ವಿರ
ಹದಿ-ನಾ-ಲ್ಕ-ನೆಯ
ಹದಿ-ನಾ-ಲ್ಕ-ನೆ-ಯದು
ಹದಿ-ನಾಲ್ಕು
ಹದಿ-ನೆಂ-ಟ-ನೆಯ
ಹದಿ-ನೆಂ-ಟ-ನೆ-ಯದು
ಹದಿ-ನೆಂಟು
ಹದಿ-ನೇಳು
ಹದಿ-ನೈ-ದನೆ
ಹದಿ-ನೈ-ದ-ನೆಯ
ಹದಿ-ನೈದು
ಹದಿ-ಮೂ-ರ-ನೆಯ
ಹದಿ-ಮೂ-ರ-ನೆ-ಯ-ಳಾದ
ಹದಿ-ಮೂರು
ಹದ್ದಿಗೆ
ಹದ್ದು
ಹದ್ದು-ಗ-ಳಿಗೆ
ಹನಿ
ಹನಿ-ಗ-ಳಿ-ವೆಯೋ
ಹನಿ-ಗೂ-ಡಿತು
ಹನಿ-ಯನ್ನು
ಹನ್ನೆ-ರ-ಡ-ನೆಯ
ಹನ್ನೆ-ರಡು
ಹನ್ನೆ-ರ-ಡು-ವರ್ಷ
ಹನ್ನೊಂ-ದ-ನೆಯ
ಹನ್ನೊಂದು
ಹಬ್ಬ
ಹಬ್ಬ-ವನ್ನು
ಹಬ್ಬ-ವ-ನ್ನುಂ-ಟು-ಮಾ-ಡುತ್ತಾ
ಹಬ್ಬ-ವಾ-ಗು-ವಂತೆ
ಹಬ್ಬಿ
ಹಬ್ಬಿತು
ಹಬ್ಬಿದೆ
ಹಬ್ಬಿದ್ದ
ಹಬ್ಬಿ-ರುವ
ಹಬ್ಬು-ವು-ದಲ್ಲಾ
ಹಮ್ಮೀ-ರರು
ಹಮ್ಮು
ಹಯ-ಗ್ರೀವ
ಹಯ-ಗ್ರೀ-ವನು
ಹಯ-ಗ್ರೀ-ವ-ನೆಂಬ
ಹಯ-ಗ್ರೀ-ವ-ಮೂ-ರ್ತಿ-ಯನ್ನು
ಹಯಾ-ನನೋ
ಹರ
ಹರ-ಕ-ರು-ಣ-ದಿಂದ
ಹರಕು
ಹರ-ಕು-ಬ-ಟ್ಟೆ-ಯನ್ನು
ಹರ-ಕೃಪೆ
ಹರಕೆ
ಹರ-ಕೆಗೆ
ಹರ-ಕೆ-ಯನ್ನು
ಹರ-ಟುವ
ಹರಡಿ
ಹರ-ಡಿ-ಕೊಂ-ಡಿತ್ತು
ಹರ-ಡಿತು
ಹರ-ಡಿದೆ
ಹರ-ಡು-ತ್ತಿ-ದ್ದು-ದ-ರಿಂದ
ಹರ-ಡು-ವಂತೆ
ಹರ-ಡು-ವು-ದಕ್ಕೆ
ಹರ-ಡು-ವುದು
ಹರ-ಣ-ಕ್ಕಾಗಿ
ಹರ-ಳು-ಗಳು
ಹರಸಿ
ಹರ-ಸಿ-ಕೊಂಡು
ಹರ-ಸಿ-ದನು
ಹರ-ಸಿ-ದರು
ಹರ-ಸಿ-ದಳು
ಹರಸು
ಹರ-ಸುತ್ತಾ
ಹರ-ಸು-ವು-ದಕ್ಕೆ
ಹರಿ
ಹರಿಃ
ಹರಿ-ಕ-ಥೆ-ಯಲ್ಲಿ
ಹರಿ-ಕೀ-ರ್ತ-ನೆ-ಯಲ್ಲಿ
ಹರಿ-ಕೃ-ಪೆ-ಯಿಂದ
ಹರಿ-ಕ್ಷೇ-ತ್ರ-ವೆಂಬ
ಹರಿ-ಚ-ರ-ಣದ
ಹರಿ-ಣೀ-ದೇ-ವಿಗೆ
ಹರಿಣ್ಯಃ
ಹರಿ-ತ-ವಾದ
ಹರಿ-ದರೆ
ಹರಿ-ದಳು
ಹರಿದು
ಹರಿ-ದು-ಕೊಂಡು
ಹರಿ-ದು-ಬಂತು
ಹರಿ-ದು-ಬಂದ
ಹರಿ-ದು-ಬಂ-ದಿತು
ಹರಿ-ದು-ಬಂ-ದಿ-ರುವ
ಹರಿ-ದು-ಬಂದು
ಹರಿ-ದು-ಹಾ-ಕಿ-ದನು
ಹರಿ-ದು-ಹೋ-ಗು-ತ್ತಿತ್ತು
ಹರಿ-ದು-ಹೋ-ದಳು
ಹರಿ-ಧ್ಯಾ-ನ-ತ-ತ್ಪರ
ಹರಿ-ಧ್ಯಾ-ನ-ದಲ್ಲಿ
ಹರಿ-ನೈ-ವೇ-ದ್ಯದ
ಹರಿ-ಭ-ಕ್ತ-ರಾ-ದರು
ಹರಿ-ಭಕ್ತಿ
ಹರಿ-ಭ-ಜ-ನೆಗೆ
ಹರಿ-ಮೇ-ಧನ
ಹರಿಯ
ಹರಿ-ಯ-ನ್ನಲ್ಲ
ಹರಿ-ಯನ್ನು
ಹರಿ-ಯಲಿ
ಹರಿ-ಯಲು
ಹರಿ-ಯಿತು
ಹರಿ-ಯು-ತ್ತದೆ
ಹರಿ-ಯು-ತ್ತವೆ
ಹರಿ-ಯು-ತ್ತಾಳೆ
ಹರಿ-ಯು-ತ್ತಿತ್ತು
ಹರಿ-ಯು-ತ್ತಿದೆ
ಹರಿ-ಯು-ತ್ತಿದ್ದ
ಹರಿ-ಯು-ತ್ತಿ-ದ್ದಂತೆ
ಹರಿ-ಯು-ತ್ತಿ-ದ್ದವು
ಹರಿ-ಯು-ತ್ತಿ-ರ-ಲಿಲ್ಲ
ಹರಿ-ಯು-ತ್ತಿ-ರುವ
ಹರಿ-ಯು-ತ್ತಿ-ರು-ವಂ-ತೆಯೆ
ಹರಿ-ಯುವ
ಹರಿ-ಯೆಂ-ಬು-ವನು
ಹರಿ-ರ್ಮಾಂ
ಹರಿ-ರ್ವಿ-ದ-ಧ್ಯಾ-ನ್ಮಮ
ಹರಿ-ಲೀ-ಲೆಯ
ಹರಿ-ಲೀ-ಲೆ-ಯನ್ನು
ಹರಿವ
ಹರಿ-ವಂಶ
ಹರಿ-ವರ್ಷ
ಹರಿ-ಶ್ಚಂದ್ರ
ಹರಿ-ಶ್ಚಂ-ದ್ರನ
ಹರಿ-ಶ್ಚಂ-ದ್ರ-ನಿಗೆ
ಹರಿ-ಶ್ಚಂ-ದ್ರನು
ಹರಿ-ಸಂ-ಕೀ-ರ್ತ-ನೆಯ
ಹರಿ-ಸಿತು
ಹರಿ-ಸಿದ
ಹರಿ-ಸಿ-ದನು
ಹರಿ-ಸುತ್ತಾ
ಹರಿ-ಸು-ತ್ತಿದ್ದ
ಹರಿ-ಸು-ತ್ತಿ-ರು-ವಂತೆ
ಹರೇ-ರ್ನಾಮ
ಹರ್ಯ
ಹರ್ಯ-ಕ್ಷ-ನಿಗೆ
ಹರ್ಯ-ಶ್ವ-ರೆಂಬ
ಹರ್ಷ
ಹಲ
ಹಲ-ವರು
ಹಲ-ವಾರು
ಹಲವು
ಹಲ-ವು-ಕಾಲ
ಹಲ-ಸಿ-ನ-ಮ-ರವೆ
ಹಲಸು
ಹಲುಬಿ
ಹಲು-ಬು-ತ್ತಿ-ರುವ
ಹಲ್ಲು
ಹಲ್ಲು-ಕ-ಡಿ-ಯು-ತ್ತಿ-ರಲು
ಹಲ್ಲು-ಗಳನ್ನು
ಹಲ್ಲು-ಗಳನ್ನೆಲ್ಲ
ಹಲ್ಲು-ಗಳು
ಹಲ್ಲು-ಬಿದ್ದ
ಹಲ್ಲೂ
ಹಳದಿ
ಹಳಿ
ಹಳುವಿ
ಹಳೆಯ
ಹಳೆ-ಯ-ದಾದ
ಹಳ್ಳ-ಕೊ-ಳ್ಳ-ಗಳು
ಹಳ್ಳಕ್ಕೆ
ಹಳ್ಳ-ಗಳೇ
ಹಳ್ಳಿ
ಹಳ್ಳಿ-ಗ-ಳಲ್ಲೊ
ಹಳ್ಳಿಯ
ಹವ-ಣಿ-ಸಿ-ದ್ದೇನೆ
ಹವ-ಣಿ-ಸು-ತ್ತಿ-ದ್ದನು
ಹವ-ಣಿ-ಸು-ತ್ತಿ-ರು-ವ-ಷ್ಟ-ರಲ್ಲಿ
ಹವಳ
ಹವ-ಳದ
ಹವ-ಳ-ದಂ-ತಿ-ರುವ
ಹವಿರ್
ಹವಿ-ರ್ಧಾ-ನ-ನೆಂಬ
ಹವಿ-ರ್ಭೂ-ದೇವಿ
ಹವಿ-ಸ್ಸನ್ನು
ಹವಿ-ಸ್ಸ-ನ್ನೆಲ್ಲ
ಹವಿ-ಸ್ಸಿನ
ಹವಿ-ಸ್ಸಿ-ನಲ್ಲಿ
ಹವಿಸ್ಸು
ಹವಿಸ್ಸೆ
ಹವ್ಯ-ಕವ್ಯ
ಹವ್ಯ-ವಾಟ್
ಹಸನು
ಹಸರು
ಹಸಾದ
ಹಸಿದ
ಹಸಿ-ದರೆ
ಹಸಿದು
ಹಸಿಯ
ಹಸಿ-ಯಾ-ಗಿದೆ
ಹಸಿ-ರನ್ನು
ಹಸಿರು
ಹಸಿ-ವನ್ನು
ಹಸಿ-ವಾ-ಗಿದೆ
ಹಸಿ-ವಾ-ದಾಗ
ಹಸಿ-ವಾ-ಯಿತು
ಹಸಿವಿ
ಹಸಿ-ವಿ-ನಿಂದ
ಹಸಿವು
ಹಸಿ-ವೆ-ಯಿಂದ
ಹಸಿವೋ
ಹಸು
ಹಸು-ಗಳ
ಹಸು-ಗಳನ್ನು
ಹಸು-ರಾ-ಗಿದ್ದ
ಹಸು-ರಾ-ಗಿಯೇ
ಹಸು-ರಾದ
ಹಸುಳೆ
ಹಸು-ಳೆ-ಯಾದ
ಹಸು-ವನ್ನು
ಹಸು-ವಿನ
ಹಸೆಯ
ಹಸ್ತ
ಹಸ್ತಿನಾ
ಹಸ್ತಿ-ನಾ-ವತಿ
ಹಸ್ತಿ-ನಾ-ವ-ತಿಗೆ
ಹಸ್ತಿ-ನಾ-ವ-ತಿ-ಗೆ-ಹೋಗಿ
ಹಸ್ತಿ-ನಾ-ವ-ತಿಯ
ಹಸ್ತಿ-ನಾ-ವ-ತಿ-ಯತ್ತ
ಹಸ್ತಿ-ನಾ-ವ-ತಿ-ಯಲ್ಲಿ
ಹಸ್ತಿ-ನಾ-ವ-ತಿ-ಯಾ-ಗಲಿ
ಹಸ್ತಿ-ನಾ-ವ-ತಿ-ಯಿಂದ
ಹಸ್ತಿ-ನಾ-ವ-ತೀ-ಪ-ಟ್ಟ-ಣ-ವನ್ನೆ
ಹಾ
ಹಾಕ-ಬ-ಹುದು
ಹಾಕ-ಬೇ-ಕಂತೋ
ಹಾಕ-ಬೇ-ಕಾ-ಗಿತ್ತು
ಹಾಕ-ಬೇಕು
ಹಾಕ-ಬೇ-ಕೆಂದು
ಹಾಕಲಿ
ಹಾಕ-ಲೇ-ಬೇ-ಕೆಂದು
ಹಾಕಿ
ಹಾಕಿ-ಕೊಂಡು
ಹಾಕಿ-ಕೊ-ಳ್ಳದೆ
ಹಾಕಿ-ಕೊ-ಳ್ಳ-ಬೇ-ಕೆ-ನ್ನು-ವ-ಷ್ಟ-ರಲ್ಲಿ
ಹಾಕಿ-ಕೊ-ಳ್ಳ-ಲಿಲ್ಲ
ಹಾಕಿ-ಕೊ-ಳ್ಳಲೇ
ಹಾಕಿ-ಕೊ-ಳ್ಳು-ತ್ತಾನೆ
ಹಾಕಿ-ಕೊ-ಳ್ಳುವ
ಹಾಕಿತು
ಹಾಕಿದ
ಹಾಕಿ-ದಂತೆ
ಹಾಕಿ-ದ-ಈ-ಗಲೆ
ಹಾಕಿ-ದ-ನಂತೆ
ಹಾಕಿ-ದ-ನಷ್ಟೆ
ಹಾಕಿ-ದನು
ಹಾಕಿ-ದರು
ಹಾಕಿ-ದರೆ
ಹಾಕಿ-ದಳು
ಹಾಕಿ-ದ-ವನೇ
ಹಾಕಿ-ದು-ದ-ನ್ನೆಲ್ಲ
ಹಾಕಿ-ದುದು
ಹಾಕಿ-ದುದೊ
ಹಾಕಿದ್ದ
ಹಾಕಿ-ದ್ದನು
ಹಾಕಿ-ದ್ದರು
ಹಾಕಿ-ದ್ದಾನೆ
ಹಾಕಿ-ದ್ದೇನೆ
ಹಾಕಿ-ಬಿ-ಡ-ಬ-ಹುದು
ಹಾಕಿರಿ
ಹಾಕಿ-ರುವ
ಹಾಕಿ-ರು-ವಿರಿ
ಹಾಕಿಲ್ಲ
ಹಾಕಿವೆ
ಹಾಕಿ-ಸಿ-ದ-ನಲ್ಲಾ
ಹಾಕಿ-ಸಿ-ದನು
ಹಾಕು
ಹಾಕು-ತ್ತದೆ
ಹಾಕು-ತ್ತಲೆ
ಹಾಕುತ್ತಾ
ಹಾಕು-ತ್ತಾನೆ
ಹಾಕು-ತ್ತಿತ್ತು
ಹಾಕು-ತ್ತಿದೆ
ಹಾಕು-ತ್ತಿ-ದ್ದರು
ಹಾಕು-ತ್ತಿ-ದ್ದಾಳೆ
ಹಾಕು-ತ್ತೇನೆ
ಹಾಕು-ತ್ತೇವೆ
ಹಾಕುವ
ಹಾಕು-ವಂತೆ
ಹಾಕು-ವರು
ಹಾಕು-ವ-ಳಲ್ಲಾ
ಹಾಕು-ವುದು
ಹಾಗಲ್ಲ
ಹಾಗಾ
ಹಾಗಾಗ
ಹಾಗಾ-ಗ-ಲಿ-ಲ್ಲ-ವಾ-ದರೂ
ಹಾಗಾ-ಗಿ-ದೆಯೆ
ಹಾಗಾ-ಗಿ-ರು-ವುದು
ಹಾಗಾ-ಗು-ವ-ವ-ರೆಗೆ
ಹಾಗಾ-ದರೆ
ಹಾಗಾ-ಯಿತು
ಹಾಗಿಂದ
ಹಾಗಿ-ದ್ದರೆ
ಹಾಗಿ-ರಲಿ
ಹಾಗಿ-ರು-ವು-ದ-ರಿಂದ
ಹಾಗೂ
ಹಾಗೆ
ಹಾಗೆಂ-ದ-ರೇನು
ಹಾಗೆಂದು
ಹಾಗೆಯೆ
ಹಾಗೆಯೇ
ಹಾಗೇಕೆ
ಹಾಗೇ-ನಾ-ದರೂ
ಹಾಡ-ತೊ-ಡ-ಗಿ-ದರು
ಹಾಡನ್ನು
ಹಾಡ-ಬೇಕು
ಹಾಡಿ
ಹಾಡಿ-ಕೊಂಡು
ಹಾಡಿ-ಕೊಂ-ಡುದೆ
ಹಾಡಿ-ಕೊ-ಳ್ಳುತ್ತಾ
ಹಾಡಿ-ದರು
ಹಾಡಿ-ದ-ರು-ಶ್ರೀ-ಕೃ-ಷ್ಣ-ನಲ್ಲಿ
ಹಾಡಿ-ದ-ವರ
ಹಾಡಿ-ಹಾ-ಡಿ-ಕೊಂಡು
ಹಾಡು
ಹಾಡುತ್ತ
ಹಾಡುತ್ತಾ
ಹಾಡು-ತ್ತಾನೆ
ಹಾಡು-ತ್ತಾರೆ
ಹಾಡುತ್ತಿ
ಹಾಡು-ತ್ತಿ-ದ್ದರು
ಹಾಡು-ತ್ತಿ-ದ್ದವು
ಹಾಡು-ತ್ತಿ-ದ್ದಾರೆ
ಹಾಡು-ತ್ತಿ-ರ-ಬೇಕು
ಹಾಡು-ತ್ತಿ-ರುವ
ಹಾಡು-ತ್ತಿವೆ
ಹಾಡುವ
ಹಾಡು-ವರು
ಹಾಡು-ವಳು
ಹಾಡು-ವ-ವ-ರನ್ನು
ಹಾಡು-ವು-ದಕ್ಕೆ
ಹಾದಿ
ಹಾದಿಗೆ
ಹಾದಿಯ
ಹಾದಿ-ಯನ್ನು
ಹಾದಿ-ಯನ್ನೆ
ಹಾದಿ-ಯಲ್ಲಿ
ಹಾದಿ-ಯಿಂ-ದಲೆ
ಹಾದಿ-ಯಿದೆ
ಹಾದಿ-ಹೋ-ಕ-ನೊ-ಬ್ಬ-ನನ್ನು
ಹಾದಿ-ಹೋ-ಕರು
ಹಾದು
ಹಾದು-ಬಂ-ದಿತ್ತು
ಹಾದು-ಹೋ-ಗು-ವಾಗ
ಹಾನಿ-ಕರ
ಹಾಯಾಗಿ
ಹಾಯಾ-ಗಿ-ರ-ಬೇ-ಕೆ-ನಿ-ಸಿತು
ಹಾಯಾ-ಗಿ-ರು-ತ್ತದೆ
ಹಾರ
ಹಾರ-ಗಳು
ಹಾರ-ವನ್ನು
ಹಾರ-ಹೊ-ಡೆ-ಯು-ವಂತೆ
ಹಾರಾ-ಡಿತು
ಹಾರಾಡು
ಹಾರಾ-ಡು-ತ್ತಿದ್ದ
ಹಾರಾ-ಡು-ತ್ತಿ-ರುವ
ಹಾರಾ-ಡು-ವಾಗ
ಹಾರಿ
ಹಾರಿ-ಕೊಂ-ಡರು
ಹಾರಿತು
ಹಾರಿ-ತೆಂ-ದರೆ
ಹಾರಿದ
ಹಾರಿ-ದರು
ಹಾರಿ-ದವು
ಹಾರಿ-ಬಂ-ದಿತು
ಹಾರಿ-ಬಂದು
ಹಾರಿ-ಬ-ರು-ವಾಗ
ಹಾರಿಸಿ
ಹಾರಿ-ಸಿ-ಕೊಂಡು
ಹಾರಿ-ಸಿ-ಕೊಂ-ಡು-ಹೋ-ದಂತೆ
ಹಾರಿ-ಸಿತು
ಹಾರಿ-ಸಿದ
ಹಾರಿ-ಸಿ-ದ-ಳೆಂ-ದರೆ
ಹಾರಿ-ಸುತ್ತಾ
ಹಾರಿ-ಸು-ವಳು
ಹಾರಿ-ಹೋಗಿ
ಹಾರಿ-ಹೋ-ಗಿತ್ತು
ಹಾರಿ-ಹೋಗು
ಹಾರಿ-ಹೋ-ಗು-ತ್ತದೆ
ಹಾರಿ-ಹೋ-ಗು-ತ್ತಲೆ
ಹಾರಿ-ಹೋ-ಗು-ತ್ತಿತ್ತು
ಹಾರಿ-ಹೋ-ಗುವ
ಹಾರಿ-ಹೋ-ಗು-ವಂ-ತಿ-ರ-ಲಿಲ್ಲ
ಹಾರಿ-ಹೋ-ಗು-ವಂತೆ
ಹಾರಿ-ಹೋ-ಗು-ವುವು
ಹಾರಿ-ಹೋ-ದಂ-ತಾ-ಯಿತು
ಹಾರಿ-ಹೋ-ದರು
ಹಾರಿ-ಹೋ-ಯಿತು
ಹಾರಿ-ಹೋ-ಯಿತೊ
ಹಾರು-ತ್ತಾನೆ
ಹಾರು-ತ್ತಿ-ರುವ
ಹಾರುವ
ಹಾರು-ವಂತೆ
ಹಾರೈ-ಸಿ-ದಳು
ಹಾರೈ-ಸು-ತ್ತಿ-ದ್ದರು
ಹಾರೈ-ಸು-ತ್ತೇನೆ
ಹಾರೈ-ಸು-ತ್ತೇವೆ
ಹಾಲನ್ನು
ಹಾಲ-ನ್ನೆಲ್ಲ
ಹಾಲಾ-ಹಲ
ಹಾಲಿ
ಹಾಲಿನ
ಹಾಲಿ-ನಿಂದ
ಹಾಲು
ಹಾಲು-ಅ-ನ್ನ-ವ-ನುಂಡು
ಹಾಲು-ಕೊ-ಡದೆ
ಹಾಲೆ
ಹಾಲೆ-ರೆ-ದಂ-ತಾ-ಯಿತು
ಹಾಲ್ಗ-ಡಲ
ಹಾಲ್ಗ-ಡ-ಲಿಗೆ
ಹಾಲ್ಗ-ಡಲು
ಹಾಳಾಗ
ಹಾಳಾಗಿ
ಹಾಳಾ-ಗಿ-ಹೋ-ಗಲಿ
ಹಾಳಾ-ಗು-ತ್ತ-ದೆ-ಯಷ್ಟೆ
ಹಾಳಾ-ಗು-ತ್ತವೆ
ಹಾಳಾ-ದ-ವರು
ಹಾಳಾ-ದವು
ಹಾಳಾದೆ
ಹಾಳು
ಹಾಳು-ಗ-ವಿ-ಗಳು
ಹಾಳು-ಬಾವಿ
ಹಾಳು-ಬಾ-ವಿ-ಯಂತೆ
ಹಾಳು-ಬಿದ್ದು
ಹಾಳು-ಮಾ-ಡ-ಬೇ-ಕೆಂದು
ಹಾಳು-ಮಾ-ಡಿ-ಕೊ-ಳ್ಳ-ಬಾ-ರದು
ಹಾಳು-ಮಾ-ಡಿದೆ
ಹಾಳು-ಮಾ-ಡುವ
ಹಾವ
ಹಾವ-ನ್ನಾ-ಡಿ-ಸುವ
ಹಾವನ್ನು
ಹಾವನ್ನೂ
ಹಾವ-ಭಾವ
ಹಾವಲ್ಲ
ಹಾವಳಿ
ಹಾವ-ಳಿ-ಯಿಂದ
ಹಾವಾ-ಗಿತ್ತು
ಹಾವಾ-ಗೆಂದು
ಹಾವಾ-ಡಿ-ಗನ
ಹಾವಾದೆ
ಹಾವಾ-ಯಿ-ತಲ್ಲ
ಹಾವಾ-ಯಿ-ತಲ್ಲೆ
ಹಾವಾ-ಯಿತು
ಹಾವಾ-ಯಿ-ತೆಂದು
ಹಾವಿಗೆ
ಹಾವಿನ
ಹಾವಿ-ನಂ-ತಾ-ಯಿತು
ಹಾವಿ-ನಂತೆ
ಹಾವಿ-ನ-ಮ-ರಿ-ಗ-ಳಂತೆ
ಹಾವು
ಹಾವು-ಗಳ
ಹಾವು-ಗಳಿಂದ
ಹಾವು-ಗ-ಳಿಗೆ
ಹಾವು-ಗಳು
ಹಾವು-ಗ-ಳೊ-ಡನೆ
ಹಾಸ
ಹಾಸಕ್ಕೆ
ಹಾಸ-ವಾದ
ಹಾಸಾತ್
ಹಾಸಿ
ಹಾಸಿ-ಕೊಂಡು
ಹಾಸಿಗೆ
ಹಾಸಿ-ಗೆ-ಗ-ಳೇನು
ಹಾಸಿ-ಗೆಯ
ಹಾಸಿ-ಗೆ-ಯನ್ನು
ಹಾಸಿ-ಗೆ-ಯ-ಮೇಲೆ
ಹಾಸಿ-ಗೆ-ಯಲ್ಲಿ
ಹಾಸಿ-ಗೆ-ಯಿಂ-ದೆದ್ದು
ಹಾಸು-ಹೊ-ಕ್ಕು-ಗ-ಳಂತೆ
ಹಾಸೋ-ದಾರ
ಹಾಸ್ಯ
ಹಾಸ್ಯಕ್ಕೂ
ಹಾಸ್ಯಕ್ಕೆ
ಹಾಸ್ಯ-ಗಳಲ್ಲಿ
ಹಾಸ್ಯದ
ಹಾಸ್ಯ-ದಿಂದ
ಹಾಸ್ಯ-ಪ್ರಿ-ಯ-ರಾದ
ಹಾಸ್ಯ-ಭ-ರಿತ
ಹಾಸ್ಯ-ಮಾಡಿ
ಹಾಸ್ಯ-ಮಾ-ಡು-ವನು
ಹಾಹಾ-ಕಾರ
ಹಾಹಾ-ಕಾ-ರ-ದಿಂದ
ಹಿ
ಹಿಂಗಾ-ಲು-ಗಳನ್ನು
ಹಿಂಗಾ-ಲು-ಗ-ಳ-ನ್ನೆತ್ತಿ
ಹಿಂಗಾ-ಲು-ಗಳಿಂದ
ಹಿಂಗಾ-ಲು-ಗ-ಳೆ-ರ-ಡನ್ನೂ
ಹಿಂಗಿ-ಸ-ಬೇ-ಕಾ-ದುದು
ಹಿಂಗಿ-ಸು-ತ್ತದೆ
ಹಿಂಗು-ವಷ್ಟು
ಹಿಂಜ-ರಿ-ದನು
ಹಿಂಜ-ರಿ-ದರೆ
ಹಿಂಜ-ರಿ-ಯದ
ಹಿಂಡಾಗಿ
ಹಿಂಡಿ-ತೌ-ಡು-ಗ-ಳನ್ನೊ
ಹಿಂಡಿ-ದರೆ
ಹಿಂಡು
ಹಿಂಡು-ತ್ತಾನೆ
ಹಿಂಡು-ವಂತೆ
ಹಿಂಡು-ಹಿಂ-ಡಾಗಿ
ಹಿಂತಿ-ರುಗಿ
ಹಿಂತಿ-ರು-ಗಿ-ದರು
ಹಿಂತಿ-ರು-ಗು-ತ್ತಲೆ
ಹಿಂತಿ-ರು-ಗು-ವವೋ
ಹಿಂದ-ಕ್ಕಿತ್ತು
ಹಿಂದಕ್ಕೆ
ಹಿಂದ-ಕ್ಕೆ-ಳೆದು
ಹಿಂದಟ್ಟಿ
ಹಿಂದಣ
ಹಿಂದಿ
ಹಿಂದಿನ
ಹಿಂದಿ-ನಂ-ತೆಯೆ
ಹಿಂದಿ-ನಂ-ತೆಯೇ
ಹಿಂದಿ-ನ-ದನ್ನು
ಹಿಂದಿ-ನ-ದೆಂದು
ಹಿಂದಿರು
ಹಿಂದಿ-ರುಗ
ಹಿಂದಿ-ರು-ಗ-ದಿ-ರಲು
ಹಿಂದಿ-ರು-ಗ-ದು-ದನ್ನು
ಹಿಂದಿ-ರು-ಗ-ಬ-ಹುದು
ಹಿಂದಿ-ರು-ಗ-ಬೇ-ಕಾ-ಯಿತು
ಹಿಂದಿ-ರು-ಗ-ಬೇಕು
ಹಿಂದಿ-ರು-ಗ-ಬೇ-ಕೆಂದು
ಹಿಂದಿ-ರು-ಗ-ಬೇ-ಕೆಂಬ
ಹಿಂದಿ-ರು-ಗ-ಲಿಲ್ಲ
ಹಿಂದಿ-ರು-ಗಲು
ಹಿಂದಿ-ರುಗಿ
ಹಿಂದಿ-ರು-ಗಿತು
ಹಿಂದಿ-ರು-ಗಿದ
ಹಿಂದಿ-ರು-ಗಿ-ದನು
ಹಿಂದಿ-ರು-ಗಿ-ದ-ಮೇಲೆ
ಹಿಂದಿ-ರು-ಗಿ-ದರು
ಹಿಂದಿ-ರು-ಗಿ-ದರೆ
ಹಿಂದಿ-ರು-ಗಿ-ದಳು
ಹಿಂದಿ-ರು-ಗಿ-ದವು
ಹಿಂದಿ-ರು-ಗಿ-ದಾಗ
ಹಿಂದಿ-ರು-ಗಿದೆ
ಹಿಂದಿ-ರು-ಗಿ-ದೊ-ಡನೆ
ಹಿಂದಿ-ರು-ಗಿರಿ
ಹಿಂದಿ-ರು-ಗಿ-ರೆಂ-ದರೆ
ಹಿಂದಿ-ರು-ಗಿ-ಸ-ಬೇ-ಕೆಂದು
ಹಿಂದಿ-ರುಗು
ಹಿಂದಿ-ರು-ಗು-ತ್ತಲೆ
ಹಿಂದಿ-ರು-ಗುತ್ತಾ
ಹಿಂದಿ-ರು-ಗು-ತ್ತಾನೆ
ಹಿಂದಿ-ರು-ಗು-ತ್ತಿ-ದ್ದರು
ಹಿಂದಿ-ರು-ಗುವ
ಹಿಂದಿ-ರು-ಗು-ವಂತೆ
ಹಿಂದಿ-ರು-ಗು-ವ-ರೆಂದು
ಹಿಂದಿ-ರು-ಗು-ವ-ವ-ರೆಗೆ
ಹಿಂದಿ-ರು-ಗು-ವ-ಷ್ಟ-ರಲ್ಲಿ
ಹಿಂದಿ-ರು-ಗು-ವಾಗ
ಹಿಂದಿ-ರು-ಗು-ವು-ದಕ್ಕೆ
ಹಿಂದಿ-ರು-ಗು-ವು-ದನ್ನೆ
ಹಿಂದಿ-ರು-ಗು-ವೆ-ನೆಂದು
ಹಿಂದು
ಹಿಂದು-ರು-ಗಿದ
ಹಿಂದೂ
ಹಿಂದೂ-ಧ-ರ್ಮದ
ಹಿಂದೆ
ಹಿಂದೆಂದೂ
ಹಿಂದೆಯೆ
ಹಿಂದೆಯೇ
ಹಿಂದೆಲ್ಲ
ಹಿಂದೊಮ್ಮೆ
ಹಿಂಬಾ-ಲ-ಕ-ರೊ-ಡನೆ
ಹಿಂಬಾ-ಲಿ-ಸ-ಬೇ-ಕಂತೆ
ಹಿಂಬಾ-ಲಿಸಿ
ಹಿಂಬಾ-ಲಿ-ಸಿತು
ಹಿಂಬಾ-ಲಿ-ಸಿ-ದನು
ಹಿಂಬಾ-ಲಿ-ಸಿ-ದ-ರಂತೆ
ಹಿಂಬಾ-ಲಿ-ಸಿ-ದರು
ಹಿಂಬಾ-ಲಿ-ಸಿ-ದಳು
ಹಿಂಬಾ-ಲಿ-ಸು-ವಂತೆ
ಹಿಂಬಾ-ಲಿ-ಸು-ವರು
ಹಿಂಸಾ-ಕಾ-ರ್ಯ-ವನ್ನು
ಹಿಂಸಿ-ಸ-ತೊ-ಡ-ಗಿ-ದನು
ಹಿಂಸಿ-ಸ-ದಿ-ರು-ವುದು
ಹಿಂಸಿ-ಸಲು
ಹಿಂಸಿ-ಸ-ಹೊ-ರ-ಡು-ತ್ತಾನೆ
ಹಿಂಸಿಸಿ
ಹಿಂಸಿಸು
ಹಿಂಸಿ-ಸು-ತ್ತಾನೆ
ಹಿಂಸಿ-ಸು-ತ್ತಿದ್ದ
ಹಿಂಸಿ-ಸು-ತ್ತಿ-ದ್ದನು
ಹಿಂಸಿ-ಸು-ತ್ತಿ-ರುವೆ
ಹಿಂಸಿ-ಸು-ವನು
ಹಿಂಸೆ
ಹಿಂಸೆಈ
ಹಿಂಸೆ-ಪ-ಡಿ-ಸು-ವ-ನೆಂದು
ಹಿಂಸೆ-ಪ-ಡು-ತ್ತಾನೆ
ಹಿಂಸೆ-ಮಾಡಿ
ಹಿಂಸೆ-ಯನ್ನು
ಹಿಂಸೆ-ಯಾ-ಗ-ದಂತೆ
ಹಿಂಸೆ-ಯಾ-ಗು-ವುದೋ
ಹಿಗ್ಗದೆ
ಹಿಗ್ಗ-ಬಾ-ರದು
ಹಿಗ್ಗ-ಲಿಸಿ
ಹಿಗ್ಗಿ
ಹಿಗ್ಗಿತು
ಹಿಗ್ಗಿದ
ಹಿಗ್ಗಿ-ದರು
ಹಿಗ್ಗಿ-ನಿಂದ
ಹಿಗ್ಗಿ-ಹೋದ
ಹಿಗ್ಗಿ-ಹೋ-ದರು
ಹಿಗ್ಗು
ಹಿಗ್ಗುತ್ತಾ
ಹಿಗ್ಗು-ವನು
ಹಿಗ್ಗೋ
ಹಿಚಿಕಿ
ಹಿಚು-ಕಿ-ಹಾ-ಕಲು
ಹಿಟ್ಟಿಗೆ
ಹಿಟ್ಟಿ-ಲ್ಲದೆ
ಹಿಟ್ಟು
ಹಿಡಿ
ಹಿಡಿತ
ಹಿಡಿ-ತ-ದಿಂದ
ಹಿಡಿ-ತ-ರಿಸಿ
ಹಿಡಿ-ತ-ರಿ-ಸು-ತ್ತೇನೆ
ಹಿಡಿ-ತ-ರಿ-ಸು-ವುದು
ಹಿಡಿ-ತ-ರು-ವು-ದ-ಕ್ಕಾಗಿ
ಹಿಡಿದ
ಹಿಡಿ-ದಂ-ತಾ-ಯಿತು
ಹಿಡಿ-ದ-ನಲ್ಲಾ
ಹಿಡಿ-ದನು
ಹಿಡಿ-ದರು
ಹಿಡಿ-ದರೂ
ಹಿಡಿ-ದಳು
ಹಿಡಿ-ದ-ವನೆ
ಹಿಡಿ-ದಿದ್ದ
ಹಿಡಿ-ದಿ-ದ್ದಾರೆ
ಹಿಡಿ-ದಿ-ರುವ
ಹಿಡಿ-ದಿ-ರು-ವಿ-ರಂತೆ
ಹಿಡಿದು
ಹಿಡಿ-ದು-ಕೊಂ-ಡನು
ಹಿಡಿ-ದು-ಕೊಂ-ಡರು
ಹಿಡಿ-ದು-ಕೊಂ-ಡಳು
ಹಿಡಿ-ದು-ಕೊಂಡು
ಹಿಡಿ-ದು-ಕೊಂ-ಡು-ಬಿ-ಟ್ಟಿದೆ
ಹಿಡಿ-ದು-ಕೊಂಡೆ
ಹಿಡಿ-ದು-ಕೊ-ಳ್ಳಲು
ಹಿಡಿ-ದು-ಕೊ-ಳ್ಳುವ
ಹಿಡಿ-ದು-ಕೊ-ಳ್ಳು-ವುದು
ಹಿಡಿ-ದುದು
ಹಿಡಿ-ದೆ-ತ್ತಿ-ದನು
ಹಿಡಿ-ದೆ-ಳೆದು
ಹಿಡಿದೇ
ಹಿಡಿ-ದೇ-ಬಿ-ಟ್ಟಿತು
ಹಿಡಿ-ಯನ್ನು
ಹಿಡಿ-ಯ-ಬೇ-ಕಾ-ಯಿತು
ಹಿಡಿ-ಯ-ಬೇಕು
ಹಿಡಿ-ಯ-ಬೇ-ಕೆಂಬ
ಹಿಡಿ-ಯ-ಲಾ-ರ-ದಷ್ಟು
ಹಿಡಿ-ಯ-ವ-ಲ-ಕ್ಕಿಗೆ
ಹಿಡಿ-ಯ-ವ-ಲ-ಕ್ಕಿ-ಯಿಂದ
ಹಿಡಿ-ಯ-ಷ್ಟನ್ನು
ಹಿಡಿ-ಯ-ಹೋದ
ಹಿಡಿ-ಯಿತು
ಹಿಡಿ-ಯಿರಿ
ಹಿಡಿ-ಯು-ತ್ತಲೆ
ಹಿಡಿ-ಯು-ತ್ತಾರೆ
ಹಿಡಿ-ಯು-ತ್ತಿತ್ತೊ
ಹಿಡಿ-ಯುಳ್ಳ
ಹಿಡಿ-ಯುವ
ಹಿಡಿ-ಯು-ವಂತೆ
ಹಿಡಿ-ಯು-ವುದು
ಹಿಡಿ-ಯು-ವು-ದೇನು
ಹಿಡಿವ
ಹಿಡಿಸ
ಹಿಡಿ-ಸ-ಲಾ-ರ-ದಷ್ಟು
ಹಿಡಿ-ಸ-ಲಿಲ್ಲ
ಹಿಡಿ-ಸಿತು
ಹಿಡಿ-ಸುವ
ಹಿಡಿ-ಸು-ವಂ-ತಿಲ್ಲ
ಹಿಡಿ-ಸು-ವು-ದಿಲ್ಲ
ಹಿಡಿ-ಹಿ-ಡಿ-ಯಾಗಿ
ಹಿತ
ಹಿತ-ನಾ-ಗ-ಬೇ-ಕೆಂದು
ಹಿತ-ಮಿ-ತ-ವಾದ
ಹಿತ-ವ-ಚನ
ಹಿತ-ವನ್ನು
ಹಿತ-ವಾಗು
ಹಿತೈಷಿ
ಹಿತ್ತಲ
ಹಿತ್ತಾ-ಳೆಯ
ಹಿತ್ತಿ-ಲನ್ನು
ಹಿತ್ತಿ-ಲ-ಲ್ಲಿಯೆ
ಹಿನ್ನೆಲೆ
ಹಿಮ-ವಂ-ತನ
ಹಿಮ-ವ-ತ್ಪ-ರ್ವ-ತ-ದಂತೆ
ಹಿಮಾ-ಲಯ
ಹಿಮಾ-ಲ-ಯದ
ಹಿಮಾ-ಲ-ಯ-ಪ-ರ್ವ-ತದ
ಹಿಮ್ಮುಖ
ಹಿರ-ಣ್ಮಯ
ಹಿರ-ಣ್ಮ-ಯ-ದಲ್ಲಿ
ಹಿರಣ್ಯ
ಹಿರ-ಣ್ಯ-ಕ-ನೆಂ-ಬು-ವನು
ಹಿರ-ಣ್ಯ-ಕ-ಶಿಪು
ಹಿರ-ಣ್ಯ-ಕ-ಶಿ-ಪು-ವನ್ನು
ಹಿರ-ಣ್ಯ-ಕ-ಶಿ-ಪು-ವಿಗೆ
ಹಿರ-ಣ್ಯ-ಕ-ಶಿ-ಪು-ವಿನ
ಹಿರ-ಣ್ಯ-ಕ-ಶಿ-ಪು-ವಿ-ನಿಂದ
ಹಿರ-ಣ್ಯ-ಕ-ಶಿ-ಪುವು
ಹಿರ-ಣ್ಯ-ಕ-ಶಿ-ಪು-ವೆಂದು
ಹಿರ-ಣ್ಯಾಕ್ಷ
ಹಿರ-ಣ್ಯಾ-ಕ್ಷನ
ಹಿರ-ಣ್ಯಾ-ಕ್ಷ-ನತ್ತ
ಹಿರ-ಣ್ಯಾ-ಕ್ಷ-ನನ್ನು
ಹಿರ-ಣ್ಯಾ-ಕ್ಷನು
ಹಿರ-ಣ್ಯಾ-ಕ್ಷ-ರಿಗೆ
ಹಿರ-ಣ್ಯಾ-ಕ್ಷರು
ಹಿರಿದ
ಹಿರಿದು
ಹಿರಿಮೆ
ಹಿರಿಯ
ಹಿರಿ-ಯ-ಇ-ವರ
ಹಿರಿ-ಯ-ನ-ನ್ನಾಗಿ
ಹಿರಿ-ಯ-ನಾಗಿ
ಹಿರಿ-ಯ-ನಾದ
ಹಿರಿ-ಯ-ನಾ-ದ-ವನು
ಹಿರಿ-ಯ-ನೆ-ನಿ-ಸಿದ್ದ
ಹಿರಿ-ಯ-ನೆ-ನಿ-ಸಿ-ದ್ದಾನೆ
ಹಿರಿ-ಯ-ಮ-ಗ-ನಾದ
ಹಿರಿ-ಯ-ಮ-ಗ-ಳಾದ
ಹಿರಿ-ಯರ
ಹಿರಿ-ಯ-ರಂತೆ
ಹಿರಿ-ಯ-ರ-ನ್ನಾಗಿ
ಹಿರಿ-ಯ-ರನ್ನೂ
ಹಿರಿ-ಯ-ರಾದ
ಹಿರಿ-ಯ-ರಾ-ದುದ
ಹಿರಿ-ಯ-ರಾ-ರಿಗೂ
ಹಿರಿ-ಯ-ರಿಗೂ
ಹಿರಿ-ಯರು
ಹಿರಿ-ಯರೂ
ಹಿರಿ-ಯ-ರೊ-ಡ-ನೆಯೂ
ಹಿರಿ-ಯ-ಳಾದ
ಹಿರಿ-ಯ-ಳಾ-ದ-ವಳು
ಹಿರಿ-ಯ-ಳಿ-ಗೊ-ಪ್ಪಿಸಿ
ಹಿರಿ-ಯ-ವ-ನಾದ
ಹಿರಿ-ಹಿರಿ
ಹಿಸಿ-ಕಿ-ಹಾ-ಕಿದ
ಹಿಸು
ಹಿಸುಕಿ
ಹೀಗಾ-ಗ-ಬೇ-ಕಾ-ಗಿ-ದ್ದರೆ
ಹೀಗಾ-ಗ-ಬೇ-ಕಾ-ದುದು
ಹೀಗಾಗಿ
ಹೀಗಾ-ಗಿ-ದೆಯೇ
ಹೀಗಾ-ದರೆ
ಹೀಗಿದೆ
ಹೀಗಿ-ರಲು
ಹೀಗಿರು
ಹೀಗಿ-ರು-ತ್ತದೆ
ಹೀಗಿ-ರು-ತ್ತಿ-ರಲು
ಹೀಗಿ-ರು-ವಾಗ
ಹೀಗಿ-ರು-ವು-ದ-ರಿಂದ
ಹೀಗಿ-ರು-ವುದು
ಹೀಗೂ
ಹೀಗೆ
ಹೀಗೆಂದು
ಹೀಗೆಂ-ದು-ಕೊಂಡ
ಹೀಗೆಂ-ದು-ಕೊಂಡು
ಹೀಗೆಯೆ
ಹೀಗೆಯೇ
ಹೀಗೆಲ್ಲ
ಹೀಗೇಕೆ
ಹೀನ-ಕೆ-ಲಸ
ಹೀಯಾಳಿ
ಹೀಯಾ-ಳಿಸಿ
ಹೀಯಾ-ಳಿ-ಸಿ-ದನು
ಹೀಯಾ-ಳಿ-ಸುವ
ಹೀರಿ
ಹೀರಿತ್ತು
ಹೀರಿ-ದ-ಮೇಲೆ
ಹೀರಿ-ದರು
ಹೀರು
ಹೀರುತ್ತಾ
ಹೀರು-ವು-ದ-ಕ್ಕಾಗಿ
ಹೀರು-ವು-ದಕ್ಕೆ
ಹುಂಕಾ-ರ-ದಿಂದ
ಹುಚ್ಚ
ಹುಚ್ಚನ
ಹುಚ್ಚ-ನಂ-ತಿ-ದ್ದೇನೆ
ಹುಚ್ಚ-ನಂತೆ
ಹುಚ್ಚ-ನೆಂದು
ಹುಚ್ಚಪ್ಪ
ಹುಚ್ಚ-ರಂತೆ
ಹುಚ್ಚ-ರಾ-ಗಿ-ಹೋ-ದರು
ಹುಚ್ಚ-ರಾ-ದರು
ಹುಚ್ಚಾ
ಹುಚ್ಚಿ
ಹುಚ್ಚಿ-ಯಂತೆ
ಹುಚ್ಚು
ಹುಚ್ಚು-ಹು-ಚ್ಚಾಗಿ
ಹುಟ್ಟ
ಹುಟ್ಟಡ
ಹುಟ್ಟ-ಡ-ಗಿ-ಸಿ-ಬಿ-ಡು-ತ್ತೇವೆ
ಹುಟ್ಟ-ತಕ್ಕ
ಹುಟ್ಟ-ದಂ-ತಹ
ಹುಟ್ಟ-ದಿ-ರಲು
ಹುಟ್ಟದು
ಹುಟ್ಟದೆ
ಹುಟ್ಟನ್ನು
ಹುಟ್ಟ-ಬ-ಯ-ಸು-ತ್ತಾರೆ
ಹುಟ್ಟ-ಬೇ-ಕಾ-ಗು-ತ್ತದೆ
ಹುಟ್ಟ-ಬೇ-ಕಾ-ಯಿತು
ಹುಟ್ಟ-ಬೇಕು
ಹುಟ್ಟ-ಬೇಕೆ
ಹುಟ್ಟ-ಬೇ-ಕೆಂದು
ಹುಟ್ಟ-ಲಾ-ರದು
ಹುಟ್ಟಲಿ
ಹುಟ್ಟ-ಲಿಲ್ಲ
ಹುಟ್ಟಲು
ಹುಟ್ಟ-ಲೇ-ಬೇಕು
ಹುಟ್ಟಾ
ಹುಟ್ಟಿ
ಹುಟ್ಟಿಗೆ
ಹುಟ್ಟಿತು
ಹುಟ್ಟಿ-ತೆಂಬ
ಹುಟ್ಟಿತ್ತು
ಹುಟ್ಟಿದ
ಹುಟ್ಟಿ-ದಂತೆ
ಹುಟ್ಟಿ-ದಂದೆ
ಹುಟ್ಟಿ-ದನು
ಹುಟ್ಟಿ-ದ-ನೆಂದು
ಹುಟ್ಟಿ-ದ-ಮೇಲೆ
ಹುಟ್ಟಿ-ದರು
ಹುಟ್ಟಿ-ದರೂ
ಹುಟ್ಟಿ-ದರೆ
ಹುಟ್ಟಿ-ದಳು
ಹುಟ್ಟಿ-ದವ
ಹುಟ್ಟಿ-ದ-ವ-ನಾ-ದು-ದ-ರಿಂದ
ಹುಟ್ಟಿ-ದ-ವನು
ಹುಟ್ಟಿ-ದ-ವನೇ
ಹುಟ್ಟಿ-ದ-ವರು
ಹುಟ್ಟಿ-ದ-ವಳು
ಹುಟ್ಟಿ-ದವು
ಹುಟ್ಟಿ-ದ-ವು-ಗ-ಳಾ-ದ್ದ-ರಿಂದ
ಹುಟ್ಟಿ-ದಾಗ
ಹುಟ್ಟಿ-ದಾ-ಗಲೂ
ಹುಟ್ಟಿ-ದಾ-ಗಲೇ
ಹುಟ್ಟಿ-ದು-ದರ
ಹುಟ್ಟಿ-ದು-ದ-ರಿಂದ
ಹುಟ್ಟಿ-ದುದು
ಹುಟ್ಟಿ-ದುದೂ
ಹುಟ್ಟಿ-ದುದೆ
ಹುಟ್ಟಿದೆ
ಹುಟ್ಟಿ-ದೆಯೊ
ಹುಟ್ಟಿ-ದೊ-ಡ-ನೆಯೆ
ಹುಟ್ಟಿದ್ದ
ಹುಟ್ಟಿ-ದ್ದಾನೆ
ಹುಟ್ಟಿ-ದ್ದಾಳೆ
ಹುಟ್ಟಿದ್ದಿ
ಹುಟ್ಟಿ-ನಿಂದ
ಹುಟ್ಟಿ-ಬಂತು
ಹುಟ್ಟಿ-ಬಂದ
ಹುಟ್ಟಿ-ಬಂ-ದ-ನಂತೆ
ಹುಟ್ಟಿ-ಬಂ-ದ-ವರು
ಹುಟ್ಟಿ-ಬಂ-ದಷ್ಟು
ಹುಟ್ಟಿ-ಬಂ-ದಿತು
ಹುಟ್ಟಿ-ಬಂ-ದಿ-ದ್ದಾನೆ
ಹುಟ್ಟಿ-ಬಂ-ದಿ-ರುವೆ
ಹುಟ್ಟಿ-ಬಂದು
ಹುಟ್ಟಿ-ಬಂ-ದೆ-ಯ-ಲ್ಲವೆ
ಹುಟ್ಟಿ-ರ-ಬೇಕು
ಹುಟ್ಟಿ-ರುವ
ಹುಟ್ಟಿ-ರು-ವ-ರ-ಲ್ಲಾ-ಎಂದು
ಹುಟ್ಟಿ-ರು-ವ-ವನು
ಹುಟ್ಟಿ-ರು-ವಾಗ
ಹುಟ್ಟಿ-ರು-ವಾ-ಗಲೂ
ಹುಟ್ಟಿ-ರು-ವುದು
ಹುಟ್ಟಿ-ರು-ವುದೇ
ಹುಟ್ಟಿ-ರುವೆ
ಹುಟ್ಟಿಸಿ
ಹುಟ್ಟಿ-ಸಿದ
ಹುಟ್ಟಿ-ಸು-ತ್ತವೆ
ಹುಟ್ಟಿ-ಸು-ತ್ತಾನೆ
ಹುಟ್ಟಿ-ಸುವ
ಹುಟ್ಟಿ-ಸು-ವಂ-ತಹ
ಹುಟ್ಟಿ-ಸು-ವಂ-ತ-ಹುದೇ
ಹುಟ್ಟಿ-ಸು-ವಂ-ತಿ-ದ್ದರು
ಹುಟ್ಟಿ-ಸು-ವು-ದ-ಕ್ಕಾ-ಗಿಯೆ
ಹುಟ್ಟಿ-ಸುವೆ
ಹುಟ್ಟು
ಹುಟ್ಟು-ಗುಣ
ಹುಟ್ಟು-ತ್ತದೆ
ಹುಟ್ಟು-ತ್ತ-ದೆಯೇ
ಹುಟ್ಟು-ತ್ತಲೆ
ಹುಟ್ಟು-ತ್ತಲೇ
ಹುಟ್ಟು-ತ್ತವೆ
ಹುಟ್ಟುತ್ತಾ
ಹುಟ್ಟು-ತ್ತಾನೆ
ಹುಟ್ಟು-ತ್ತಾರೆ
ಹುಟ್ಟು-ತ್ತಾಳೆ
ಹುಟ್ಟು-ತ್ತಿ-ರುವ
ಹುಟ್ಟು-ತ್ತೇನೆ
ಹುಟ್ಟುವ
ಹುಟ್ಟು-ವಂತೆ
ಹುಟ್ಟು-ವನು
ಹುಟ್ಟು-ವರು
ಹುಟ್ಟು-ವ-ವನು
ಹುಟ್ಟು-ವ-ವ-ರೆಗೆ
ಹುಟ್ಟು-ವಾ-ಗಲೆ
ಹುಟ್ಟು-ವು-ದ-ಕ್ಕಾಗಿ
ಹುಟ್ಟು-ವು-ದ-ಕ್ಕಿಂತ
ಹುಟ್ಟು-ವು-ದಕ್ಕೂ
ಹುಟ್ಟು-ವು-ದಕ್ಕೆ
ಹುಟ್ಟು-ವು-ದಾಗಿ
ಹುಟ್ಟು-ವು-ದಾ-ದರೆ
ಹುಟ್ಟು-ವು-ದಿಲ್ಲ
ಹುಟ್ಟು-ವುದು
ಹುಟ್ಟು-ವು-ದೆಂಬ
ಹುಟ್ಟು-ಸಾವು
ಹುಟ್ಟು-ಸಾ-ವು-ಗಳ
ಹುಡಿ-ಗು-ಟ್ಟಿ-ದನು
ಹುಡು
ಹುಡುಕ
ಹುಡು-ಕಲಿ
ಹುಡು-ಕ-ಹೊ-ರ-ಟರು
ಹುಡುಕಿ
ಹುಡು-ಕಿ-ಕೊಂಡು
ಹುಡು-ಕಿ-ತ-ರು-ವಂತೆ
ಹುಡು-ಕಿದ
ಹುಡು-ಕಿ-ದರೂ
ಹುಡು-ಕಿ-ಸಿದ
ಹುಡು-ಕುತ್ತಾ
ಹುಡು-ಕು-ತ್ತಿದ್ದ
ಹುಡು-ಕು-ತ್ತಿ-ದ್ದಾಳೆ
ಹುಡು-ಕು-ತ್ತಿ-ರುವ
ಹುಡು-ಕು-ವು-ದೆಂ-ದರೆ
ಹುಡು-ಕು-ವು-ದೇಕೆ
ಹುಡುಗ
ಹುಡು-ಗನ
ಹುಡು-ಗ-ನಂತೆ
ಹುಡು-ಗ-ನನ್ನು
ಹುಡು-ಗ-ನಾ-ದರೂ
ಹುಡು-ಗ-ನೊಬ್ಬ
ಹುಡು-ಗ-ಬು-ದ್ಧಿಗೆ
ಹುಡು-ಗರ
ಹುಡು-ಗ-ರಂತೆ
ಹುಡು-ಗ-ರನ್ನೂ
ಹುಡು-ಗ-ರ-ನ್ನೆಲ್ಲ
ಹುಡು-ಗ-ರ-ಮೇ-ಲಾ-ದರೆ
ಹುಡು-ಗರಾ
ಹುಡು-ಗ-ರಿಗೆ
ಹುಡು-ಗ-ರಿ-ಬ್ಬ-ರನ್ನೂ
ಹುಡು-ಗರು
ಹುಡು-ಗರೆ
ಹುಡು-ಗ-ರೆಲ್ಲ
ಹುಡು-ಗ-ರೊ-ಡನೆ
ಹುಡು-ಗಾಟ
ಹುಡು-ಗಾ-ಟ-ಕ್ಕಾ-ಗಿಯೋ
ಹುಡು-ಗಾ-ಟ-ವ-ನ್ನೆಲ್ಲ
ಹುಡು-ಗಾ-ಟ-ವೆಂದು
ಹುಡು-ಗಾ-ಟ-ವೇನೂ
ಹುಡುಗಿ
ಹುಡು-ಗಿಗೆ
ಹುಡು-ಗಿಯ
ಹುಡು-ಗಿ-ಯಂತೂ
ಹುಡು-ಗಿ-ಯನ್ನು
ಹುಡು-ಗಿ-ಯ-ರಿಗೆ
ಹುಡು-ಗಿ-ಯ-ರಿ-ಗೆಲ್ಲ
ಹುಡು-ಗಿ-ಯರು
ಹುಡು-ಗಿ-ಯರೆ
ಹುಡು-ಗಿ-ಯ-ರೊ-ಬ್ಬ-ರಿಗೂ
ಹುಡು-ಗಿ-ಯಾ-ದರೂ
ಹುಡು-ಗಿಯು
ಹುಡು-ಗಿ-ಯೊ-ಬ್ಬಳು
ಹುಡು-ಗು-ತ-ನದ
ಹುಣ್ಣಾದ
ಹುಣ್ಣಿ-ಮೆ-ಗಳಲ್ಲಿ
ಹುಣ್ಣು
ಹುತಾಶಃ
ಹುತ್ತ
ಹುತ್ತದ
ಹುತ್ತ-ದಲ್ಲಿ
ಹುತ್ತ-ದಿಂದ
ಹುತ್ತ-ವನ್ನು
ಹುದು
ಹುದು-ಗಿ-ಟ್ಟು-ಕೊಂ-ಡಿರ
ಹುದು-ಗಿ-ಸಿ-ಕೊಂಡು
ಹುಬ್ಬಿನ
ಹುಬ್ಬಿ-ನಿಂದ
ಹುಬ್ಬು
ಹುಬ್ಬು-ಗಳ
ಹುಬ್ಬು-ಗಳನ್ನು
ಹುಬ್ಬು-ಗಳಿಂದ
ಹುಬ್ಬು-ಗಳು
ಹುಯ-ಲನ್ನು
ಹುರಿ-ಗೊಂ-ಡಿತು
ಹುರಿದ
ಹುರಿ-ದಿದೆ
ಹುರಿದು
ಹುಲಿ
ಹುಲಿಗೆ
ಹುಲಿಯ
ಹುಲಿಯೇ
ಹುಲಿ-ಹಿ-ಡಿದ
ಹುಲು-ಸಾದ
ಹುಲ್ಲನ್ನು
ಹುಲ್ಲನ್ನೂ
ಹುಲ್ಲಿಗೆ
ಹುಲ್ಲಿನ
ಹುಲ್ಲು
ಹುಲ್ಲು-ಕಡ್ಡಿ
ಹುಲ್ಲು-ಕ-ಡ್ಡಿ-ಗಿಂ-ತಲೂ
ಹುಲ್ಲು-ಕ-ಡ್ಡಿ-ಯಂತೆ
ಹುಲ್ಲು-ಕ-ಡ್ಡಿ-ಯನ್ನು
ಹುಲ್ಲು-ಕ-ಡ್ಡಿ-ಯ-ವ-ರೆಗೆ
ಹುಲ್ಲು-ತಿಂ-ದರೂ
ಹುಲ್ಲೆ
ಹುಲ್ಲೆ-ಗ-ಳಂತೆ
ಹುಲ್ಲೆಯ
ಹುಲ್ಲೆ-ಯಂ-ತಾ-ದಳು
ಹುಲ್ಲೆ-ಯಂತೆ
ಹುಲ್ಲೆ-ಯನ್ನು
ಹುಳ
ಹುಳ-ಗ-ಳಂತೆ
ಹುಳ-ವನ್ನು
ಹುಳ-ವನ್ನೆ
ಹುಳವೆ
ಹುಳಿ
ಹುಳು
ಹುಳು-ಗ-ಳಂತೆ
ಹುಳುವಿ
ಹುಳು-ವಿ-ನಂತೆ
ಹೂ
ಹೂಂಕಾರ
ಹೂಗಳ
ಹೂಗಳನ್ನು
ಹೂಗಳಲ್ಲಿ
ಹೂಗಳಿಂದ
ಹೂಗಳು
ಹೂಗಿಡ
ಹೂಡಿ
ಹೂಡಿತ್ತು
ಹೂಡಿದ
ಹೂಡಿ-ದನು
ಹೂಡಿ-ದರು
ಹೂಡಿ-ದ-ವನೆ
ಹೂಡಿ-ದ್ದುದು
ಹೂಡು
ಹೂಡು-ಗಿ-ಯ-ರೆಲ್ಲ
ಹೂಣ
ಹೂತಿಯ
ಹೂದೋ-ಟ-ವೊಂ-ದರ
ಹೂಬಾ-ಣ-ವಿ-ಟ್ಟು-ಕೊಂಡು
ಹೂಬಿ-ಸಿ-ಲಲ್ಲಿ
ಹೂಮಳೆ
ಹೂಮ-ಳೆ-ಗ-ರೆ-ದರು
ಹೂಮ-ಳೆ-ಗ-ರೆ-ಯಿತು
ಹೂಮ-ಳೆ-ಯನ್ನು
ಹೂಮಾಲೆ
ಹೂಮಾ-ಲೆ-ಗಳ
ಹೂಮಾ-ಲೆ-ಗ-ಳಿಂ-ದಲೂ
ಹೂಮಾ-ಲೆ-ಗಳು
ಹೂಮಾ-ಲೆ-ಯನ್ನು
ಹೂಮಾ-ಲೆ-ಯಿಂದ
ಹೂಳಿ-ಸಿ-ದನು
ಹೂವನ್ನು
ಹೂವನ್ನೊ
ಹೂವಾ-ಡಿಗ
ಹೂವಿ
ಹೂವಿಂದ
ಹೂವಿಗೆ
ಹೂವಿನ
ಹೂವಿ-ನಂತೆ
ಹೂವಿ-ನಷ್ಟು
ಹೂವಿ-ನಿಂದ
ಹೂವು
ಹೂವು-ಗಳ
ಹೂವೆ
ಹೂಹಣ್ಣು
ಹೂಹಾ-ರ-ಗಳು
ಹೃತ-ಚೇತಾ
ಹೃತ-ದಾ-ನ-ವ-ದೃ-ಪ್ತ-ಬಲಂ
ಹೃತ-ವಾ-ಸ-ವ-ಮು-ಖ್ಯ-ಮದಂ
ಹೃದಯ
ಹೃದ-ಯಂ-ಗ-ಮ-ವಾಗಿ
ಹೃದ-ಯಕ್ಕೆ
ಹೃದ-ಯ-ಗಳಲ್ಲಿ
ಹೃದ-ಯ-ಗಳು
ಹೃದ-ಯ-ಗೋ-ಚ-ರ-ನಾದ
ಹೃದ-ಯ-ಗ್ರಂಥಿ
ಹೃದ-ಯದ
ಹೃದ-ಯ-ದಲ್ಲಿ
ಹೃದ-ಯ-ದ-ಲ್ಲಿಯೆ
ಹೃದ-ಯ-ದ-ಲ್ಲಿ-ರುವ
ಹೃದ-ಯ-ದಿಂದ
ಹೃದ-ಯ-ನಾದ
ಹೃದ-ಯ-ಪೀ-ಠ-ದಲ್ಲಿ
ಹೃದ-ಯ-ವನ್ನು
ಹೃದ-ಯವು
ಹೃದ-ಯ-ವೆ-ಲ್ಲವೂ
ಹೃದ-ಯಾ-ಕಾ-ಶ-ದ-ಲ್ಲಿ-ರುವ
ಹೃದ-ಯಾದಿ
ಹೃದ-ಯಾಯ
ಹೃದ-ಯಾ-ಯ-ನಮಃ
ಹೃದ-ಯೇ-ಶ್ವರಿ
ಹೃದಿ-ಯೋ-ಗಿ-ಜ-ನೈಃ
ಹೃದೀ-ರ-ಯಸಿ
ಹೃಷೀ-ಕೇಶ
ಹೃಷೀ-ಕೇ-ಶನು
ಹೃಷೀ-ಕೇ-ಶಾಯ
ಹೆ
ಹೆಂಗ-ರುಳು
ಹೆಂಗಸ
ಹೆಂಗ-ಸರ
ಹೆಂಗ-ಸ-ರನ್ನು
ಹೆಂಗ-ಸ-ರನ್ನೂ
ಹೆಂಗ-ಸರು
ಹೆಂಗ-ಸರೂ
ಹೆಂಗ-ಸ-ರೆಲ್ಲ
ಹೆಂಗ-ಸರೇ
ಹೆಂಗ-ಸಿಗೆ
ಹೆಂಗ-ಸಿನ
ಹೆಂಗ-ಸಿ-ನಂತೆ
ಹೆಂಗ-ಸಿ-ನೊ-ಡನೆ
ಹೆಂಗಸು
ಹೆಂಗಸೆ
ಹೆಂಗ-ಸೊ-ಬ್ಬಳು
ಹೆಂಗೊ-ಲೆಗೆ
ಹೆಂಗೊ-ಲೆಯ
ಹೆಂಡ
ಹೆಂಡತಿ
ಹೆಂಡ-ತಿಗೂ
ಹೆಂಡ-ತಿಗೆ
ಹೆಂಡ-ತಿಯ
ಹೆಂಡ-ತಿ-ಯನ್ನು
ಹೆಂಡ-ತಿ-ಯನ್ನೂ
ಹೆಂಡ-ತಿ-ಯರು
ಹೆಂಡ-ತಿ-ಯಲ್ಲಿ
ಹೆಂಡ-ತಿ-ಯಾದ
ಹೆಂಡ-ತಿ-ಯಿ-ದ್ದಳು
ಹೆಂಡ-ತಿಯೂ
ಹೆಂಡ-ತಿ-ಯೊ-ಡನೆ
ಹೆಂಡ-ತಿ-ರೆಲ್ಲ
ಹೆಂಡದ
ಹೆಂಡಿತಿ
ಹೆಂಡಿರ
ಹೆಂಡಿ-ರನ್ನು
ಹೆಂಡಿ-ರನ್ನೂ
ಹೆಂಡಿ-ರಲ್ಲಿ
ಹೆಂಡಿ-ರಾಗಿ
ಹೆಂಡಿ-ರಿ-ದ್ದರು
ಹೆಂಡಿರು
ಹೆಂಡಿ-ರೊ-ಡನೆ
ಹೆಗಲ
ಹೆಗ-ಲಲ್ಲಿ
ಹೆಗ-ಲಿನ
ಹೆಗಲು
ಹೆಗ್ಗುರಿ
ಹೆಚ್ಚಾ
ಹೆಚ್ಚಾಗಿ
ಹೆಚ್ಚಾ-ಗಿ-ರು-ವುದು
ಹೆಚ್ಚಾ-ಯಿತು
ಹೆಚ್ಚಾ-ಯಿ-ತೆಂ-ದಾ-ಗಲಿ
ಹೆಚ್ಚಿ-ಕೊ-ಳ್ಳು-ವು-ದಕ್ಕೆ
ಹೆಚ್ಚಿತು
ಹೆಚ್ಚಿ-ದಂ-ತೆಲ್ಲ
ಹೆಚ್ಚಿ-ದಂ-ತೆಲ್ಲಾ
ಹೆಚ್ಚಿ-ದಾಗ
ಹೆಚ್ಚಿನ
ಹೆಚ್ಚಿಸಿ
ಹೆಚ್ಚಿ-ಸು-ತ್ತಾನೆ
ಹೆಚ್ಚು
ಹೆಚ್ಚು-ಕ-ಡ-ಮೆ-ಯಾ-ಗು-ವಂ-ತ-ಹು-ದಲ್ಲ
ಹೆಚ್ಚು-ಕಾಲ
ಹೆಚ್ಚು-ತ್ತದೆ
ಹೆಚ್ಚುತ್ತಾ
ಹೆಚ್ಚು-ತ್ತಿತ್ತು
ಹೆಚ್ಚೇನು
ಹೆಜ್ಜೆ
ಹೆಜ್ಜೆ-ಗ-ಳ-ನ್ನಿ-ಡುತ್ತಾ
ಹೆಜ್ಜೆ-ಗ-ಳನ್ನೆ
ಹೆಜ್ಜೆ-ಗ-ಳಾ-ಗು-ವಷ್ಟು
ಹೆಜ್ಜೆ-ಗಳು
ಹೆಜ್ಜೆಗೂ
ಹೆಜ್ಜೆಗೆ
ಹೆಜ್ಜೆಯ
ಹೆಜ್ಜೆ-ಯ-ನ್ನಿ-ಡಲು
ಹೆಜ್ಜೆ-ಯ-ನ್ನಿಡು
ಹೆಜ್ಜೆ-ಯ-ನ್ನಿ-ಡುತ್ತಾ
ಹೆಜ್ಜೆ-ಯಲ್ಲಿ
ಹೆಜ್ಜೆ-ಯಿ-ಡು-ತ್ತಲೆ
ಹೆಜ್ಜೆ-ಯಿ-ಡು-ವುದು
ಹೆಜ್ಜೆಯೂ
ಹೆಜ್ಜೆ-ಹೆ-ಜ್ಜೆಗೂ
ಹೆಡ-ತ-ಲೆಯ
ಹೆಡೆ
ಹೆಡೆ-ಗಳ
ಹೆಡೆ-ಗಳನ್ನು
ಹೆಡೆ-ಗಳನ್ನೂ
ಹೆಡೆ-ಗ-ಳ-ಲ್ಲಿದ್ದ
ಹೆಡೆ-ಗಳಿಂದ
ಹೆಡೆ-ಗಳು
ಹೆಡೆ-ಗ-ಳೆಲ್ಲ
ಹೆಡೆ-ಗ-ಳೊ-ಡನೆ
ಹೆಡೆ-ತ-ಲೆಯ
ಹೆಡೆ-ತು-ಳಿದ
ಹೆಡೆ-ಮೆ-ಟ್ಟಿದ
ಹೆಡೆಯ
ಹೆಡೆ-ಯನ್ನು
ಹೆಣ
ಹೆಣ-ಗಳ
ಹೆಣ-ಗ-ಳಂತೆ
ಹೆಣದ
ಹೆಣ-ದಂತೆ
ಹೆಣ-ದ-ಮೇಲೆ
ಹೆಣ-ವನ್ನು
ಹೆಣೆ-ದಿದೆ
ಹೆಣೆದು
ಹೆಣ್ಣನ್ನು
ಹೆಣ್ಣ-ನ್ನು-ಅ-ದ-ರ-ಲ್ಲಿಯೂ
ಹೆಣ್ಣನ್ನೆ
ಹೆಣ್ಣನ್ನೇ
ಹೆಣ್ಣಾ-ಗಲಿ
ಹೆಣ್ಣಾಗಿ
ಹೆಣ್ಣಾದ
ಹೆಣ್ಣಾ-ನೆ-ಗಳ
ಹೆಣ್ಣಾ-ನೆ-ಗಳನ್ನೂ
ಹೆಣ್ಣಾ-ನೆ-ಗ-ಳೊ-ಡನೆ
ಹೆಣ್ಣಾ-ನೆಗೆ
ಹೆಣ್ಣಾ-ನೆ-ಯನ್ನು
ಹೆಣ್ಣಿ-ಗ-ನಲ್ಲ
ಹೆಣ್ಣಿ-ಗಿಂತ
ಹೆಣ್ಣಿಗೂ
ಹೆಣ್ಣಿಗೆ
ಹೆಣ್ಣಿನ
ಹೆಣ್ಣಿ-ನಂತೆ
ಹೆಣ್ಣಿ-ನತ್ತ
ಹೆಣ್ಣಿ-ನಲ್ಲಿ
ಹೆಣ್ಣಿ-ನಿಂ-ದ-ಮ-ರ-ದಿಂದ
ಹೆಣ್ಣಿ-ನೊ-ಡನೆ
ಹೆಣ್ಣು
ಹೆಣ್ಣು-ಗಂ-ಡು-ಗಳು
ಹೆಣ್ಣು-ಗಂ-ಡೆಂಬ
ಹೆಣ್ಣು-ಗಳ
ಹೆಣ್ಣು-ಗಳನ್ನು
ಹೆಣ್ಣು-ಗಳನ್ನೂ
ಹೆಣ್ಣು-ಗಳನ್ನೆಲ್ಲ
ಹೆಣ್ಣು-ಗಳಲ್ಲಿ
ಹೆಣ್ಣು-ಗ-ಳ-ಲ್ಲೊ-ಬ್ಬಳು
ಹೆಣ್ಣು-ಗ-ಳಿಗೂ
ಹೆಣ್ಣು-ಗ-ಳಿಗೆ
ಹೆಣ್ಣು-ಗ-ಳಿ-ಗೆಲ್ಲ
ಹೆಣ್ಣು-ಗಳು
ಹೆಣ್ಣು-ಗಳೂ
ಹೆಣ್ಣು-ಗಳೆ
ಹೆಣ್ಣು-ಗ-ಳೆಲ್ಲ
ಹೆಣ್ಣು-ಗ-ಳೆ-ಲ್ಲರೂ
ಹೆಣ್ಣು-ಜನ್ಮ
ಹೆಣ್ಣು-ಮ-ಕ್ಕಳ
ಹೆಣ್ಣು-ಮ-ಕ್ಕ-ಳಂತೂ
ಹೆಣ್ಣು-ಮ-ಕ್ಕ-ಳನ್ನು
ಹೆಣ್ಣು-ಮ-ಕ್ಕ-ಳನ್ನೂ
ಹೆಣ್ಣು-ಮ-ಕ್ಕ-ಳಿಗೆ
ಹೆಣ್ಣು-ಮ-ಕ್ಕ-ಳಿದ್ದಾ
ಹೆಣ್ಣು-ಮ-ಕ್ಕ-ಳಿ-ಬ್ಬರೂ
ಹೆಣ್ಣು-ಮ-ಕ್ಕಳು
ಹೆಣ್ಣು-ಮ-ಕ್ಕಳೂ
ಹೆಣ್ಣು-ಮ-ಕ್ಕ-ಳೆಲ್ಲ
ಹೆಣ್ಣು-ಮ-ಕ್ಕ-ಳೊ-ಡ-ನೆಯೂ
ಹೆಣ್ಣು-ಮ-ಗ-ಳಾಗ
ಹೆಣ್ಣು-ಮ-ಗಳು
ಹೆಣ್ಣು-ಮ-ಗಳೂ
ಹೆಣ್ಣು-ಮಗು
ಹೆಣ್ಣು-ವೇ-ಷ-ವನ್ನು
ಹೆಣ್ಣೂ
ಹೆಣ್ಣೆ
ಹೆಣ್ಣೆಂ-ದರೆ
ಹೆಣ್ಣೆಂದು
ಹೆಣ್ಣೇ
ಹೆಣ್ಣೇಕೆ
ಹೆಣ್ಣೊ
ಹೆಣ್ಣೊ-ಬ್ಬಳ
ಹೆಣ್ಣೊ-ಬ್ಬಳು
ಹೆತ್ತ
ಹೆತ್ತ-ಮೇಲೆ
ಹೆತ್ತಳು
ಹೆತ್ತ-ವ-ರಿಂದ
ಹೆತ್ತ-ವ-ರಿಗೆ
ಹೆತ್ತ-ವ-ಳಂತೆ
ಹೆತ್ತಿ-ರುವ
ಹೆತ್ತು
ಹೆತ್ತೊಡ
ಹೆತ್ತೊ-ಡ-ನೆಯೇ
ಹೆದ-ಯೇ-ರಿ-ಸುವ
ಹೆದ-ರದೆ
ಹೆದ-ರ-ಬೇ-ಕಾ-ಗಿಲ್ಲ
ಹೆದ-ರ-ಬೇ-ಕಾ-ದು-ದಿಲ್ಲ
ಹೆದ-ರ-ಬೇಡ
ಹೆದ-ರ-ಬೇಡಿ
ಹೆದರಿ
ಹೆದ-ರಿ-ಕೆ-ಯಿಂದ
ಹೆದ-ರಿದ
ಹೆದ-ರಿ-ದ-ವ-ನಂತೆ
ಹೆದ-ರಿ-ದ-ವ-ರಂತೆ
ಹೆದ-ರಿ-ಯಲ್ಲ
ಹೆದ-ರಿಸಿ
ಹೆದ-ರಿ-ಸಿ-ದರು
ಹೆದ-ರಿ-ಸು-ತ್ತಿ-ರು-ವೆ-ಯ-ಲ್ಲವೆ
ಹೆದ-ರಿ-ಸು-ತ್ತಿ-ರು-ವೆ-ಯಲ್ಲಾ
ಹೆದ-ರಿ-ಸು-ತ್ತೀಯಾ
ಹೆದ-ರು-ತ್ತಾರೆ
ಹೆದ-ರು-ತ್ತೇನೆ
ಹೆದ-ರು-ತ್ತೇ-ನೆಯೆ
ಹೆದ-ರು-ವನೇ
ಹೆದ-ರು-ವ-ವ-ನಲ್ಲ
ಹೆದ-ರು-ವ-ವರ
ಹೆದ-ರು-ವು-ದಾ-ದರೆ
ಹೆದ-ರು-ವು-ದಿಲ್ಲ
ಹೆದ-ರು-ವುದೂ
ಹೆದೆ-ಯೇ-ರಿ-ಸಿದ
ಹೆದೆ-ಯೇ-ರಿ-ಸುವ
ಹೆದ್ದಾರಿ
ಹೆದ್ದಾ-ರಿ-ಯಂ-ತಿ-ರುವ
ಹೆಬ್ಬಂ-ಡೆ-ಗಳು
ಹೆಬ್ಬಾ-ಗಿ-ಲನ್ನು
ಹೆಬ್ಬಾ-ಗಿ-ಲಲ್ಲಿ
ಹೆಬ್ಬಾ-ಗಿಲು
ಹೆಬ್ಬಾ-ಗಿ-ಲು-ಗ-ಳುಳ್ಳ
ಹೆಬ್ಬಾ-ವನ್ನು
ಹೆಬ್ಬಾ-ವಲ್ಲ
ಹೆಬ್ಬಾ-ವಾದ
ಹೆಬ್ಬಾ-ವಿ-ನಂತೆ
ಹೆಬ್ಬಾ-ವಿ-ನಿಂದ
ಹೆಬ್ಬಾವು
ಹೆಬ್ಬಾ-ವು-ಗಳು
ಹೆಬ್ಬಾ-ವೆಂದರೆ
ಹೆಬ್ಬಾ-ವೊಂದು
ಹೆಬ್ಬುಲಿ
ಹೆಬ್ಬೆ-ಟ್ಟಿನ
ಹೆಬ್ಬೆ-ರಳ
ಹೆಬ್ಬೆ-ರ-ಳನ್ನು
ಹೆಮ್ಮ-ರ-ದಂತೆ
ಹೆಮ್ಮ-ರ-ವಾಗಿ
ಹೆಮ್ಮೆ
ಹೆಮ್ಮೆ-ಪಟ್ಟು
ಹೆಮ್ಮೆ-ಯಿಂದ
ಹೆಮ್ಮೆ-ಯೆಲ್ಲ
ಹೆರಿಗೆ
ಹೆರಿ-ಗೆಯ
ಹೆರು-ತ್ತಾಳೆ
ಹೆರು-ವು-ದಕ್ಕೆ
ಹೆಸ-ರ-ನ್ನಿ-ಟ್ಟು-ಕೊಂಡು
ಹೆಸ-ರನ್ನು
ಹೆಸ-ರನ್ನೆ
ಹೆಸರಾ
ಹೆಸ-ರಾ-ದರು
ಹೆಸ-ರಾ-ದ-ವರು
ಹೆಸ-ರಾ-ದುದು
ಹೆಸ-ರಾ-ಯಿತು
ಹೆಸರಿ
ಹೆಸ-ರಿಗೆ
ಹೆಸ-ರಿನ
ಹೆಸ-ರಿ-ನಲ್ಲಿ
ಹೆಸ-ರಿ-ನ-ಲ್ಲಿಯೇ
ಹೆಸ-ರಿ-ನಿಂದ
ಹೆಸ-ರಿ-ರಲಿ
ಹೆಸ-ರಿ-ಲ್ಲ-ದಂತೆ
ಹೆಸ-ರಿ-ಸುವ
ಹೆಸರು
ಹೆಸ-ರು-ಎ-ಲ್ಲದ
ಹೆಸ-ರು-ಗಳನ್ನು
ಹೆಸ-ರು-ಗ-ಳಾ-ದವು
ಹೆಸ-ರು-ಗಳಿಂದ
ಹೆಸ-ರು-ಗಳು
ಹೆಸ-ರುಳ್ಳ
ಹೆಸರೂ
ಹೆಸರೇ
ಹೆಸ-ರೇನು
ಹೆಸ-ರೊಂ-ದನ್ನು
ಹೆಸ-ರೊಂದು
ಹೇ
ಹೇಗಾ
ಹೇಗಾ-ಗಿ-ಬಿಟ್ಟೆ
ಹೇಗಾ-ದರೂ
ಹೇಗಾ-ದೀಯೆ
ಹೇಗಿ-ದೆಯೊ
ಹೇಗಿ-ದ್ದರೂ
ಹೇಗಿ-ದ್ದ-ವನು
ಹೇಗಿ-ದ್ದ-ವರು
ಹೇಗಿ-ರ-ಬ-ಹುದು
ಹೇಗಿ-ರ-ಬೇಕು
ಹೇಗಿ-ರ-ಬೇ-ಕೆಂ-ಬ-ದನ್ನು
ಹೇಗಿ-ರು-ವುದೊ
ಹೇಗೂ
ಹೇಗೆ
ಹೇಗೆಂದು
ಹೇಗೆಂ-ಬು-ದನ್ನು
ಹೇಗೆಂ-ಬುದು
ಹೇಗೊ
ಹೇಗೋ
ಹೇಚಿ-ಕೆ-ಯೆಂ-ಬುದು
ಹೇಡಿ
ಹೇಡಿ-ಗ-ಳೆಲ್ಲ
ಹೇಡಿ-ತನ
ಹೇಡಿ-ಯಾದ
ಹೇಡಿ-ಯಾ-ದರೂ
ಹೇಡಿ-ಯೊ-ಡನೆ
ಹೇತು
ಹೇತುಂ
ಹೇಮಾಂ-ಗದ
ಹೇಮಾ-ದ್ರಿ-ತು-ಷ್ಟಯೇ
ಹೇಮಾ-ದ್ರಿ-ಯೆಂಬ
ಹೇರಿ-ದನು
ಹೇಳ
ಹೇಳ-ತೊ-ಡ-ಗಿದ
ಹೇಳದೆ
ಹೇಳ-ಬ-ಲ್ಲನು
ಹೇಳ-ಬ-ಹು-ದಾ-ದರೂ
ಹೇಳ-ಬ-ಹುದು
ಹೇಳ-ಬ-ಹುದೆ
ಹೇಳ-ಬೇ-ಕಾದು
ಹೇಳ-ಬೇ-ಕಾ-ದುದೇ
ಹೇಳ-ಬೇಕು
ಹೇಳ-ಬೇಕೆ
ಹೇಳ-ಬೇ-ಕೆಂದು
ಹೇಳ-ಬೇ-ಕೆ-ನ್ನಿ-ಸು-ತ್ತದೆ
ಹೇಳ-ಬೇಕೊ
ಹೇಳಯ್ಯ
ಹೇಳ-ಲಾಗು
ಹೇಳ-ಲಾ-ರದೆ
ಹೇಳ-ಲಾರೆ
ಹೇಳಲಿ
ಹೇಳ-ಲಿಲ್ಲ
ಹೇಳಲು
ಹೇಳಲೆ
ಹೇಳಿ
ಹೇಳಿ-ಕ-ಳು-ಹಿ-ಸಿದ
ಹೇಳಿ-ಕ-ಳು-ಹಿ-ಸಿ-ದನು
ಹೇಳಿ-ಕ-ಳು-ಹಿ-ಸಿ-ದಳು
ಹೇಳಿ-ಕ-ಳು-ಹಿ-ಸಿ-ದೊ-ಡ-ನೆಯೇ
ಹೇಳಿ-ಕ-ಳು-ಹಿ-ಸಿ-ದ್ದಾನೆ
ಹೇಳಿ-ಕ-ಳು-ಹಿ-ಸಿ-ರು-ವಾಗ
ಹೇಳಿಕೆ
ಹೇಳಿ-ಕೆ-ಯಂತೆ
ಹೇಳಿ-ಕೆ-ಯಿದೆ
ಹೇಳಿ-ಕೊಂಡ
ಹೇಳಿ-ಕೊಂ-ಡನು
ಹೇಳಿ-ಕೊಂ-ಡರು
ಹೇಳಿ-ಕೊಂ-ಡಳು
ಹೇಳಿ-ಕೊಂ-ಡಿ-ದ್ದಳು
ಹೇಳಿ-ಕೊಂಡು
ಹೇಳಿ-ಕೊ-ಟ್ಟರು
ಹೇಳಿ-ಕೊ-ಟ್ಟರೆ
ಹೇಳಿ-ಕೊ-ಟ್ಟ-ವರು
ಹೇಳಿ-ಕೊ-ಟ್ಟು-ದ-ನ್ನೆಲ್ಲ
ಹೇಳಿ-ಕೊ-ಡ-ಬೇಕೆ
ಹೇಳಿ-ಕೊಡು
ಹೇಳಿ-ಕೊ-ಡು-ತ್ತಾನೆ
ಹೇಳಿ-ಕೊ-ಡು-ತ್ತಿ-ರುವ
ಹೇಳಿ-ಕೊ-ಡು-ತ್ತೇನೆ
ಹೇಳಿ-ಕೊ-ಳ್ಳದೆ
ಹೇಳಿ-ಕೊ-ಳ್ಳ-ಲಾ-ರದ
ಹೇಳಿ-ಕೊ-ಳ್ಳಲು
ಹೇಳಿ-ಕೊ-ಳ್ಳುತ್ತಾ
ಹೇಳಿ-ಕೊ-ಳ್ಳು-ವಂತೆ
ಹೇಳಿ-ಕೊ-ಳ್ಳು-ವ-ವರು
ಹೇಳಿ-ಕೊ-ಳ್ಳು-ವು-ದಕ್ಕೆ
ಹೇಳಿ-ಕೊ-ಳ್ಳು-ವುದು
ಹೇಳಿ-ಕೊ-ಳ್ಳು-ವು-ದೇನು
ಹೇಳಿ-ತು-ಈತ
ಹೇಳಿದ
ಹೇಳಿ-ದಂ-ತಹ
ಹೇಳಿ-ದಂ-ತಾ-ಗಿತ್ತು
ಹೇಳಿ-ದಂತೆ
ಹೇಳಿ-ದಂ-ತೆಯೆ
ಹೇಳಿ-ದ-ಗೆ-ಳೆ-ಯರೆ
ಹೇಳಿ-ದ-ತಂ-ದೆಯೆ
ಹೇಳಿ-ದ-ನಂತೆ
ಹೇಳಿ-ದ-ನಾ-ದರೂ
ಹೇಳಿ-ದನು
ಹೇಳಿ-ದ-ನು-ಹ-ದಿ-ನೆಂಟು
ಹೇಳಿ-ದ-ಮಗು
ಹೇಳಿ-ದರು
ಹೇಳಿ-ದ-ರು-ಪು-ಷಿಯ
ಹೇಳಿ-ದ-ರು-ಮ-ಹಾ-ರಾಜ
ಹೇಳಿ-ದ-ರು-ಸ್ವಾಮಿ
ಹೇಳಿ-ದರೂ
ಹೇಳಿ-ದರೆ
ಹೇಳಿ-ದಳು
ಹೇಳಿ-ದ-ಳು-ಅಮ್ಮ
ಹೇಳಿ-ದ-ಳು-ಅಯ್ಯಾ
ಹೇಳಿ-ದ-ವನು
ಹೇಳಿ-ದ-ವರ
ಹೇಳಿ-ದ-ವ-ರಾ-ರೆಂ-ಬುದೇ
ಹೇಳಿ-ದ-ವರು
ಹೇಳಿ-ದಾ-ಗಲೇ
ಹೇಳಿ-ದಿರಿ
ಹೇಳಿ-ದು-ದನ್ನು
ಹೇಳಿ-ದು-ದ-ರಿಂದ
ಹೇಳಿ-ದು-ದಾದ
ಹೇಳಿ-ದುದು
ಹೇಳಿ-ದು-ದೆಲ್ಲ
ಹೇಳಿ-ದು-ದೆ-ಲ್ಲವೂ
ಹೇಳಿದೆ
ಹೇಳಿ-ದೆ-ನಷ್ಟೆ
ಹೇಳಿ-ದೆ-ಯಷ್ಟೆ
ಹೇಳಿ-ದೆ-ಯಾ-ದರೂ
ಹೇಳಿ-ದ್ದಂತೆ
ಹೇಳಿ-ದ್ದರು
ಹೇಳಿ-ದ್ದರೂ
ಹೇಳಿ-ದ್ದಾನೆ
ಹೇಳಿ-ದ್ದಾಳೆ
ಹೇಳಿದ್ದೆ
ಹೇಳಿ-ರುವ
ಹೇಳಿ-ರು-ವುದು
ಹೇಳಿ-ರು-ವು-ದು-ಹೇ-ಳು-ವ-ವರು
ಹೇಳಿ-ರು-ವೆನು
ಹೇಳಿರೆ
ಹೇಳಿಲ್ಲ
ಹೇಳಿಸಿ
ಹೇಳಿ-ಸಿ-ದರೂ
ಹೇಳಿ-ಸು-ವುದು
ಹೇಳಿ-ಹೋ-ಯಿತು
ಹೇಳು
ಹೇಳು-ಎಂ-ದನು
ಹೇಳು-ತ್ತದೆ
ಹೇಳು-ತ್ತಲೆ
ಹೇಳು-ತ್ತವೆ
ಹೇಳುತ್ತಾ
ಹೇಳು-ತ್ತಾನೆ
ಹೇಳು-ತ್ತಾ-ನೆ-ಒಂದು
ಹೇಳು-ತ್ತಾ-ನೆ-ಯೋ-ಗಿ-ಗಳು
ಹೇಳು-ತ್ತಾರೆ
ಹೇಳು-ತ್ತಾಳೆ
ಹೇಳುತ್ತಿ
ಹೇಳು-ತ್ತಿದ್ದ
ಹೇಳು-ತ್ತಿ-ದ್ದರು
ಹೇಳು-ತ್ತಿ-ದ್ದ-ರು-ಇದು
ಹೇಳು-ತ್ತಿ-ದ್ದರೆ
ಹೇಳು-ತ್ತಿ-ದ್ದಾಗ
ಹೇಳು-ತ್ತಿ-ದ್ದಾನೆ
ಹೇಳು-ತ್ತಿ-ದ್ದುದು
ಹೇಳು-ತ್ತಿ-ದ್ದೇನೆ
ಹೇಳು-ತ್ತಿರು
ಹೇಳು-ತ್ತಿ-ರು-ವ-ನೆಂ-ಬು-ದನ್ನೂ
ಹೇಳು-ತ್ತಿ-ರು-ವ-ವನು
ಹೇಳು-ತ್ತಿ-ರು-ವಾ-ಗಲೆ
ಹೇಳು-ತ್ತಿ-ರು-ವುದು
ಹೇಳು-ತ್ತಿ-ರು-ವೆ-ಯಲ್ಲ
ಹೇಳು-ತ್ತಿ-ರು-ವೆ-ಯಲ್ಲಾ
ಹೇಳು-ತ್ತಿ-ರು-ವೆಯಾ
ಹೇಳು-ತ್ತೇನೆ
ಹೇಳು-ತ್ತೇವೆ
ಹೇಳುವ
ಹೇಳು-ವಂ-ತಿಲ್ಲ
ಹೇಳು-ವಂತೆ
ಹೇಳು-ವನು
ಹೇಳು-ವರು
ಹೇಳು-ವ-ವನ
ಹೇಳು-ವ-ವ-ನಂತೆ
ಹೇಳು-ವ-ವನೂ
ಹೇಳು-ವ-ವ-ರನ್ನೂ
ಹೇಳು-ವ-ವರು
ಹೇಳು-ವ-ವರೂ
ಹೇಳು-ವ-ಷ್ಟ-ರಲ್ಲಿ
ಹೇಳು-ವಷ್ಟು
ಹೇಳು-ವಾಗ
ಹೇಳುವು
ಹೇಳು-ವು-ದಂತೂ
ಹೇಳು-ವು-ದಕ್ಕೆ
ಹೇಳು-ವುದನ್ನು
ಹೇಳು-ವು-ದರ
ಹೇಳು-ವು-ದ-ರಲ್ಲಿ
ಹೇಳು-ವು-ದ-ರಿಂದ
ಹೇಳು-ವು-ದಾ-ದರೆ
ಹೇಳು-ವುದು
ಹೇಳು-ವುದೂ
ಹೇಳು-ವು-ದೆಂದು
ಹೇಳು-ವು-ದೇನು
ಹೇಳು-ವೆ-ಯಲ್ಲಾ
ಹೇಳೋಣ
ಹೇಸದೆ
ಹೇಸು-ವು-ದಿಲ್ಲ
ಹೈಹಯ
ಹೊಂಗ-ನ-ಸೆ-ಲ್ಲವೂ
ಹೊಂಚು
ಹೊಂಚು-ಕಾ-ಯು-ತ್ತಿದ್ದ
ಹೊಂಚು-ಹಾ-ಕಿ-ಕೊಂ-ಡಿದೆ
ಹೊಂಚು-ಹಾ-ಕು-ತ್ತಿತ್ತು
ಹೊಂಚು-ಹಾ-ಕು-ತ್ತಿದ್ದು
ಹೊಂಚು-ಹಾ-ಕು-ತ್ತಿವೆ
ಹೊಂದ-ಲಾ-ರದೆ
ಹೊಂದಿ
ಹೊಂದಿ-ಕೆ-ಯಾದ
ಹೊಂದಿ-ಕೊಂಡು
ಹೊಂದಿ-ಕೊ-ಳ್ಳ-ಬೇ-ಕಾ-ಯಿತು
ಹೊಂದಿದ
ಹೊಂದಿ-ದರು
ಹೊಂದಿ-ದ-ವನು
ಹೊಂದಿ-ದ-ವರ
ಹೊಂದಿದ್ದ
ಹೊಂದಿ-ದ್ದುದು
ಹೊಂದಿ-ರು-ವುದು
ಹೊಂದು
ಹೊಂದು-ತ್ತಲೆ
ಹೊಂದುತ್ತಾ
ಹೊಂದು-ವನು
ಹೊಂಬ-ಣ್ಣದ
ಹೊಂಬಾ-ಳೆ-ಗ-ಳಿಂ-ದಲೂ
ಹೊಕ್ಕ
ಹೊಕ್ಕನು
ಹೊಕ್ಕರು
ಹೊಕ್ಕರೆ
ಹೊಕ್ಕ-ಳಿ-ನಿಂದ
ಹೊಕ್ಕಳು
ಹೊಕ್ಕ-ವರು
ಹೊಕ್ಕಿತು
ಹೊಕ್ಕಿ-ದ್ದೇನೆ
ಹೊಕ್ಕಿ-ರ-ವನೋ
ಹೊಕ್ಕಿ-ರು-ವ-ನೆಂ-ಬುದು
ಹೊಕ್ಕಿ-ರು-ವೆಯಾ
ಹೊಕ್ಕು
ಹೊಕ್ಕು-ಳಿ-ನಿಂದ
ಹೊಗ-ಳ-ದಿ-ರಲು
ಹೊಗ-ಳ-ಬಾ-ರದು
ಹೊಗ-ಳ-ಬೇ-ಕೆ-ನಿಸಿ
ಹೊಗಳಿ
ಹೊಗ-ಳಿ-ಕೆಗೆ
ಹೊಗ-ಳಿ-ಕೊಂಡು
ಹೊಗ-ಳಿ-ದನು
ಹೊಗ-ಳಿ-ದ-ರ-ಲ್ಲವೆ
ಹೊಗ-ಳಿ-ದರು
ಹೊಗ-ಳಿ-ಸಿ-ಕೊ-ಳ್ಳುತ್ತಾ
ಹೊಗಳು
ಹೊಗ-ಳುತ್ತ
ಹೊಗ-ಳುತ್ತಾ
ಹೊಗ-ಳು-ತ್ತಿದೆ
ಹೊಗ-ಳು-ತ್ತಿ-ದ್ದಾನೆ
ಹೊಗ-ಳು-ತ್ತಿ-ರಲು
ಹೊಗ-ಳುವ
ಹೊಗ-ಳು-ವನು
ಹೊಗ-ಳು-ವ-ವರು
ಹೊಗ-ಳು-ವುದು
ಹೊಗಿಸಿ
ಹೊಗು-ತ್ತಿ-ರುವ
ಹೊಗು-ಳು-ವನು
ಹೊಗುವ
ಹೊಗೆ
ಹೊಗೆ-ಯಿಂದ
ಹೊಗೆ-ಯಿ-ಲ್ಲದ
ಹೊಗೆ-ಯೆಲ್ಲಿ
ಹೊಟ್ಟೆ
ಹೊಟ್ಟೆ-ಕಿ-ಚ್ಚಿ-ನಿಂದ
ಹೊಟ್ಟೆ-ಕಿಚ್ಚು
ಹೊಟ್ಟೆ-ಕಿ-ಚ್ಚು-ಪಟ್ಟು
ಹೊಟ್ಟೆ-ಗಾಗಿ
ಹೊಟ್ಟೆ-ಗಿ-ಲ್ಲ-ವಲ್ಲ
ಹೊಟ್ಟೆಗೆ
ಹೊಟ್ಟೆ-ತುಂಬ
ಹೊಟ್ಟೆ-ನೋ-ವಾ-ದಾಗ
ಹೊಟ್ಟೆ-ನೋ-ವೆಂದು
ಹೊಟ್ಟೆಯ
ಹೊಟ್ಟೆ-ಯನ್ನು
ಹೊಟ್ಟೆ-ಯಲ್ಲಿ
ಹೊಟ್ಟೆ-ಯ-ಲ್ಲಿ-ಟ್ಟು-ಕೊಂಡಿ
ಹೊಟ್ಟೆ-ಯ-ಲ್ಲಿದ್ದ
ಹೊಟ್ಟೆ-ಯ-ಲ್ಲಿಯೇ
ಹೊಟ್ಟೆ-ಯ-ಲ್ಲಿ-ರುವ
ಹೊಟ್ಟೆ-ಯಿಂದ
ಹೊಟ್ಟೆಯೆ
ಹೊಟ್ಟೆ-ಹಿ-ಚಿ-ಕಿ-ಕೊ-ಳ್ಳು-ವಷ್ಟು
ಹೊಟ್ಟೆ-ಹೊ-ರೆ-ಯುತ್ತ
ಹೊಟ್ಟೆ-ಹೊ-ರೆ-ಯುವ
ಹೊಡಿ
ಹೊಡೆತ
ಹೊಡೆ-ತಕ್ಕೆ
ಹೊಡೆ-ತ-ಗಳಿಂದ
ಹೊಡೆ-ತ-ದಿಂದ
ಹೊಡೆ-ತ-ವನ್ನು
ಹೊಡೆದ
ಹೊಡೆ-ದಂ-ತಾ-ಯಿತು
ಹೊಡೆ-ದ-ಟ್ಟಿರಿ
ಹೊಡೆ-ದನು
ಹೊಡೆ-ದರು
ಹೊಡೆ-ದಾ-ಟಕ್ಕೆ
ಹೊಡೆ-ದಾಡಿ
ಹೊಡೆ-ದಾ-ಡಿಯೆ
ಹೊಡೆದು
ಹೊಡೆ-ದು-ಕೊಂಡು
ಹೊಡೆ-ದು-ಹಾಕ
ಹೊಡೆ-ದು-ಹಾ-ಕ-ಬೇ-ಕು-ಎಂ-ಬುದೇ
ಹೊಡೆ-ದು-ಹಾ-ಕಿ-ದನು
ಹೊಡೆ-ದು-ಹಾ-ಕಿ-ದರು
ಹೊಡೆ-ದೋ-ಡಿ-ಸಿದೆ
ಹೊಡೆ-ಯದೆ
ಹೊಡೆ-ಯ-ದೆಯೆ
ಹೊಡೆ-ಯ-ಬೇ-ಡಮ್ಮ
ಹೊಡೆ-ಯಲು
ಹೊಡೆ-ಯ-ಲೆಂದು
ಹೊಡೆ-ಯ-ಹೊ-ರ-ಟರು
ಹೊಡೆ-ಯ-ಹೋ-ದನು
ಹೊಡೆ-ಯು-ತ್ತಲೆ
ಹೊಡೆ-ಸ-ಬೇಕು
ಹೊಡೆ-ಸಿ-ದನು
ಹೊಣೆ
ಹೊಣೆ-ಯ-ಲ್ಲವೆ
ಹೊತ್ತ
ಹೊತ್ತರು
ಹೊತ್ತ-ವನೂ
ಹೊತ್ತ-ವನೆ
ಹೊತ್ತ-ವರು
ಹೊತ್ತಾ-ಗಿ-ದ್ದು-ದ-ರಿಂದ
ಹೊತ್ತಾ-ದರೂ
ಹೊತ್ತಾ-ಯಿತು
ಹೊತ್ತಿ
ಹೊತ್ತಿ-ಕೊಂ-ಡವು
ಹೊತ್ತಿ-ಕೊಂಡಿ
ಹೊತ್ತಿಗೆ
ಹೊತ್ತಿತು
ಹೊತ್ತಿದ್ದ
ಹೊತ್ತಿನ
ಹೊತ್ತಿ-ನಲ್ಲೆ
ಹೊತ್ತಿ-ನ-ವ-ರೆಗೆ
ಹೊತ್ತಿ-ನೊಳ
ಹೊತ್ತಿ-ನೊ-ಳ-ಗಾಗಿ
ಹೊತ್ತಿ-ರುವ
ಹೊತ್ತಿ-ರು-ವುದನ್ನು
ಹೊತ್ತಿ-ಸಿ-ಟ್ಟಂತೆ
ಹೊತ್ತಿ-ಸಿ-ದರು
ಹೊತ್ತೀಯೆ
ಹೊತ್ತು
ಹೊತ್ತು-ಕೊಂಡು
ಹೊತ್ತು-ಕೊಂ-ಡು-ಹೋಗಿ
ಹೊತ್ತು-ಕೊಂ-ಡು-ಹೋ-ಗಿ-ದ್ದಾನೆ
ಹೊತ್ತು-ಕೊಂಡೇ
ಹೊತ್ತೂ
ಹೊತ್ತೇ
ಹೊದಿಕೆ
ಹೊದಿ-ಕೆಯ
ಹೊದಿಸಿ
ಹೊದಿ-ಸಿದ
ಹೊದೆ-ದಿದ್ದ
ಹೊದ್ದ
ಹೊದ್ದಿಕೆ
ಹೊದ್ದಿದ್ದ
ಹೊದ್ದಿ-ದ್ದಾರೆ
ಹೊದ್ದು
ಹೊಯಿ-ಸಿ-ಕೊ-ಳ್ಳುವ
ಹೊಯ್ದಂ-ತಾ-ಯಿತು
ಹೊಯ್ದು-ಕೊಂಡು
ಹೊರ
ಹೊರಕ್ಕೆ
ಹೊರ-ಕ್ಕೆದ್ದು
ಹೊರ-ಕ್ಕೆ-ಳೆದ
ಹೊರ-ಕ್ಕೊಯ್ದು
ಹೊರ-ಕ್ಕೋಡಿ
ಹೊರ-ಕ್ಕೋ-ಡಿ-ದನು
ಹೊರ-ಗಣ್ಣು
ಹೊರ-ಗಾ-ಗಲಿ
ಹೊರ-ಗಿದ್ದ
ಹೊರ-ಗಿನ
ಹೊರ-ಗಿ-ನ-ವ-ರೆಗೆ
ಹೊರ-ಗಿ-ನಿಂದ
ಹೊರ-ಗಿ-ರುವ
ಹೊರಗೂ
ಹೊರಗೆ
ಹೊರ-ಗೆಲ್ಲ
ಹೊರ-ಗೆಲ್ಲೊ
ಹೊರ-ಗೆ-ಳೆ-ದನು
ಹೊರ-ಗೆ-ಳೆ-ಯು-ತ್ತಿ-ರಲು
ಹೊರ-ಗೋ-ಡಿ-ಬಂ-ದರು
ಹೊರ-ಚಾ-ಚು-ವುದು
ಹೊರ-ಚೆ-ಲ್ಲುತ್ತಾ
ಹೊರಟ
ಹೊರ-ಟಂತೆ
ಹೊರ-ಟನು
ಹೊರ-ಟ-ನೆಂ-ದರೆ
ಹೊರ-ಟ-ಮೇಲೆ
ಹೊರ-ಟರು
ಹೊರ-ಟರೆ
ಹೊರ-ಟಳು
ಹೊರ-ಟ-ವನು
ಹೊರ-ಟ-ವ-ರಲ್ಲಿ
ಹೊರ-ಟ-ವರೆಲ್ಲ
ಹೊರ-ಟವು
ಹೊರ-ಟಾಗ
ಹೊರಟಿ
ಹೊರ-ಟಿತು
ಹೊರ-ಟಿದ್ದ
ಹೊರ-ಟಿ-ದ್ದಾನೆ
ಹೊರ-ಟಿ-ದ್ದಾರೆ
ಹೊರ-ಟಿ-ದ್ದುದೇ
ಹೊರ-ಟಿ-ರುವ
ಹೊರ-ಟಿ-ರು-ವಂತೆ
ಹೊರಟು
ಹೊರ-ಟು-ದಕ್ಕೆ
ಹೊರ-ಟು-ದನ್ನು
ಹೊರ-ಟುದು
ಹೊರ-ಟು-ನಿಂ-ತವು
ಹೊರ-ಟು-ಬಂದ
ಹೊರ-ಟು-ಬಂ-ದರು
ಹೊರ-ಟು-ಬಂ-ದಳು
ಹೊರ-ಟು-ಬಂ-ದಿತು
ಹೊರ-ಟು-ಬ-ರು-ತ್ತೇನೆ
ಹೊರ-ಟು-ಬಿ-ಟ್ಟೆ-ಯಲ್ಲಾ
ಹೊರ-ಟು-ಹೊ-ದನು
ಹೊರ-ಟು-ಹೋ-ಗಿ-ದ್ದನು
ಹೊರ-ಟು-ಹೋಗು
ಹೊರ-ಟು-ಹೋ-ಗು-ತ್ತದೆ
ಹೊರ-ಟು-ಹೋ-ಗು-ತ್ತಲೆ
ಹೊರ-ಟು-ಹೋ-ಗು-ತ್ತಿ-ದ್ದಂ-ತೆಯೆ
ಹೊರ-ಟು-ಹೋ-ಗು-ವಂತೆ
ಹೊರ-ಟು-ಹೋ-ಗು-ವುದು
ಹೊರ-ಟು-ಹೋದ
ಹೊರ-ಟು-ಹೋ-ದಂತೆ
ಹೊರ-ಟು-ಹೋ-ದನು
ಹೊರ-ಟು-ಹೋ-ದ-ಮೇಲೆ
ಹೊರ-ಟು-ಹೋ-ದ-ರಾ-ದರೂ
ಹೊರ-ಟು-ಹೋ-ದರು
ಹೊರ-ಟು-ಹೋ-ದಳು
ಹೊರ-ಟು-ಹೋ-ದವು
ಹೊರ-ಟೇ-ಹೋದ
ಹೊರಡ
ಹೊರ-ಡ-ದಂ-ತಾ-ಯಿತು
ಹೊರ-ಡ-ಬ-ಹುದು
ಹೊರ-ಡ-ಬೇ-ಕೆ-ನ್ನು-ವ-ಷ್ಟ-ರ-ಲ್ಲಿಯೇ
ಹೊರ-ಡಲು
ಹೊರ-ಡಿರಿ
ಹೊರಡು
ಹೊರ-ಡು-ತ್ತಲೆ
ಹೊರ-ಡು-ತ್ತಾನೆ
ಹೊರ-ಡು-ತ್ತಾರೆ
ಹೊರ-ಡು-ತ್ತಾ-ರೆಯೆ
ಹೊರ-ಡು-ತ್ತೇನೆ
ಹೊರ-ಡುವ
ಹೊರ-ಡು-ವಂತೆ
ಹೊರ-ಡು-ವನು
ಹೊರ-ಡು-ವು-ದಕ್ಕೆ
ಹೊರ-ಡು-ವುದು
ಹೊರ-ಡು-ವು-ದೆಂ-ದರೆ
ಹೊರ-ಡುವೆ
ಹೊರ-ತ-ರುವ
ಹೊರತು
ಹೊರ-ತೆಗೆ
ಹೊರ-ತೆ-ಗೆ-ದನು
ಹೊರ-ತೆ-ಗೆ-ಯಲು
ಹೊರ-ದೋ-ರದೆ
ಹೊರ-ದೋ-ರಿ-ದರು
ಹೊರ-ನೋ-ಟಕ್ಕೆ
ಹೊರ-ಪ-ಡಿ-ಸ-ಲಾ-ರವು
ಹೊರ-ಬಂದ
ಹೊರ-ಬಂ-ದಳು
ಹೊರ-ಬಂ-ದವು
ಹೊರ-ಬಂ-ದಿತು
ಹೊರ-ಬಂದು
ಹೊರ-ಬ-ರಲು
ಹೊರ-ಬ-ರು-ತ್ತದೆ
ಹೊರ-ಬ-ರು-ತ್ತವೆ
ಹೊರ-ಬ-ಲ್ಲದು
ಹೊರ-ಬೀ-ಳು-ವ-ಷ್ಟ-ರಲ್ಲಿ
ಹೊರ-ಭಾ-ಗ-ದ-ಲ್ಲಿದ್ದ
ಹೊರ-ಮುಖ
ಹೊರ-ಲಾ-ರದೆ
ಹೊರಲು
ಹೊರ-ಳಾ-ಡಿತು
ಹೊರ-ಳಾ-ಡುತ್ತಾ
ಹೊರ-ಳಾ-ಡು-ತ್ತಿ-ರು-ವಾಗ
ಹೊರಳಿ
ಹೊರ-ಳಿದ
ಹೊರ-ಸೂ-ಸಿ-ದವು
ಹೊರ-ಹಾಕು
ಹೊರ-ಹೊಮ್ಮಿ
ಹೊರ-ಹೊ-ಮ್ಮಿತು
ಹೊರ-ಹೊ-ಮ್ಮು-ತ್ತಿದ್ದ
ಹೊರ-ಹೊ-ರ-ಟಿತು
ಹೊರ-ಹೊ-ರಟು
ಹೊರಿ-ಸ-ಬೇಡ
ಹೊರು-ವ-ಷ್ಟನ್ನು
ಹೊರು-ವು-ದ-ಕ್ಕೆಲ್ಲಿ
ಹೊರು-ವುದು
ಹೊರೆ-ಗಳು
ಹೊರೆ-ಯಿಂದ
ಹೊರೆ-ಯುವ
ಹೊಲ
ಹೊಲಕ್ಕೆ
ಹೊಲೆ-ಗೇರಿ
ಹೊಲೆಯ
ಹೊಲೆ-ಯನ್ನು
ಹೊಳೆ
ಹೊಳೆ-ಕೊ-ಳ-ಗಳು
ಹೊಳೆ-ಯನ್ನು
ಹೊಳೆ-ಯನ್ನೆ
ಹೊಳೆ-ಯಲ್ಲಿ
ಹೊಳೆ-ಯಾಗಿ
ಹೊಳೆ-ಯಾ-ಯಿತು
ಹೊಳೆ-ಯಿತು
ಹೊಳೆಯು
ಹೊಳೆ-ಯು-ತ್ತಿತ್ತು
ಹೊಳೆ-ಯು-ತ್ತಿದ್ದ
ಹೊಳೆ-ಯು-ತ್ತಿ-ದ್ದವು
ಹೊಳೆ-ಯು-ತ್ತಿ-ರುವ
ಹೊಳೆ-ಯುವ
ಹೊಳೆವ
ಹೊಳ್ಳೆ-ಗಳು
ಹೊಳ್ಳೆ-ಯಿಂದ
ಹೊಸ
ಹೊಸಕಿ
ಹೊಸ-ಕಿ-ಹಾ-ಕು-ವು-ದಕ್ಕೆ
ಹೊಸ-ದಲ್ಲ
ಹೊಸ-ದಾಗಿ
ಹೊಸದು
ಹೊಸ-ದೊಂದು
ಹೊಸಬ
ಹೊಸ-ಬಟ್ಟೆ
ಹೊಸ-ಬ-ಟ್ಟೆ-ಗಳನ್ನು
ಹೊಸ-ಬ-ಟ್ಟೆ-ಗ-ಳ-ನ್ನುಟ್ಟು
ಹೊಸ-ಬ-ಟ್ಟೆ-ಯನ್ನು
ಹೊಸ-ಬ-ನನ್ನು
ಹೊಸ್ತಿಲ
ಹೋ
ಹೋಗ
ಹೋಗದು
ಹೋಗದೆ
ಹೋಗ-ಬ-ಹು-ದಾದ
ಹೋಗ-ಬಾ-ರದೆ
ಹೋಗ-ಬಾ-ರ-ದೆಂದು
ಹೋಗ-ಬೇಕಾ
ಹೋಗ-ಬೇ-ಕಾ-ಗಿದೆ
ಹೋಗ-ಬೇ-ಕಾ-ದರೆ
ಹೋಗ-ಬೇ-ಕಾ-ದುದು
ಹೋಗ-ಬೇ-ಕಾ-ಯಿತು
ಹೋಗ-ಬೇಕು
ಹೋಗ-ಬೇ-ಕೆಂದು
ಹೋಗ-ಬೇ-ಕೆ-ನ್ನಿ-ಸಿತು
ಹೋಗ-ಬೇಕೇ
ಹೋಗ-ಬೇಡ
ಹೋಗ-ಬೇ-ಡ-ವೆಂದು
ಹೋಗ-ಲಾ-ಡಿಸ
ಹೋಗ-ಲಾ-ಡಿ-ಸ-ಬ-ಲ್ಲುದು
ಹೋಗ-ಲಾ-ಡಿ-ಸ-ಬೇಕು
ಹೋಗ-ಲಾ-ಡಿಸಿ
ಹೋಗ-ಲಾ-ಡಿ-ಸಿ-ಕೊ-ಳ್ಳು-ವು-ದ-ಕ್ಕಾಗಿ
ಹೋಗ-ಲಾ-ಡಿ-ಸಿದ
ಹೋಗ-ಲಾ-ಡಿ-ಸಿ-ದನು
ಹೋಗ-ಲಾ-ಡಿಸು
ಹೋಗ-ಲಾ-ಡಿ-ಸು-ತ್ತಾನೆ
ಹೋಗ-ಲಾ-ಡಿ-ಸು-ತ್ತೇನೆ
ಹೋಗ-ಲಾ-ಡಿ-ಸುವ
ಹೋಗ-ಲಾ-ಡಿ-ಸು-ವು-ದ-ಕ್ಕಾಗಿ
ಹೋಗ-ಲಾ-ರದೆ
ಹೋಗಲಿ
ಹೋಗ-ಲಿಲ್ಲ
ಹೋಗಲು
ಹೋಗಿ
ಹೋಗಿ-ತ್ತಾ-ದರೂ
ಹೋಗಿತ್ತು
ಹೋಗಿದೆ
ಹೋಗಿದ್ದ
ಹೋಗಿ-ದ್ದನು
ಹೋಗಿ-ದ್ದ-ನೆಂದು
ಹೋಗಿ-ದ್ದರು
ಹೋಗಿ-ದ್ದ-ರೆಂದು
ಹೋಗಿ-ದ್ದಳು
ಹೋಗಿ-ದ್ದ-ವನು
ಹೋಗಿ-ದ್ದವು
ಹೋಗಿ-ದ್ದಾಗ
ಹೋಗಿ-ದ್ದಾನೆ
ಹೋಗಿ-ದ್ದಾರೆ
ಹೋಗಿ-ಬ-ರ-ಬೇ-ಕಾ-ಗಿದೆ
ಹೋಗಿ-ಬ-ರು-ವು-ದಾಗಿ
ಹೋಗಿ-ಬ-ರು-ವೆನು
ಹೋಗಿ-ರ-ಬ-ಹು-ದೆಂದು
ಹೋಗಿ-ರುವ
ಹೋಗಿ-ರು-ವ-ನೆಂಬ
ಹೋಗಿ-ರು-ವಾಗ
ಹೋಗಿ-ರು-ವು-ದ-ರಿಂದ
ಹೋಗಿಲ್ಲ
ಹೋಗು
ಹೋಗುತ್ತ
ಹೋಗು-ತ್ತದೆ
ಹೋಗು-ತ್ತಲೆ
ಹೋಗು-ತ್ತವೆ
ಹೋಗು-ತ್ತ-ವೆಂ-ಬುದು
ಹೋಗುತ್ತಾ
ಹೋಗು-ತ್ತಾ-ನಂತೆ
ಹೋಗು-ತ್ತಾನೆ
ಹೋಗು-ತ್ತಾರೆ
ಹೋಗು-ತ್ತಾ-ರೆಯೆ
ಹೋಗುತ್ತಿ
ಹೋಗು-ತ್ತಿತ್ತು
ಹೋಗು-ತ್ತಿದೆ
ಹೋಗು-ತ್ತಿದ್ದ
ಹೋಗು-ತ್ತಿ-ದ್ದಂ-ತೆಯೆ
ಹೋಗು-ತ್ತಿ-ದ್ದರು
ಹೋಗು-ತ್ತಿ-ದ್ದವು
ಹೋಗು-ತ್ತಿ-ದ್ದಾಗ
ಹೋಗು-ತ್ತಿ-ದ್ದಿ-ರ-ಬ-ಹುದು
ಹೋಗು-ತ್ತಿ-ದ್ದುದು
ಹೋಗು-ತ್ತಿ-ದ್ದುದೂ
ಹೋಗು-ತ್ತಿ-ರಲು
ಹೋಗು-ತ್ತಿ-ರುವ
ಹೋಗು-ತ್ತಿ-ರು-ವಂ-ತಯೆ
ಹೋಗು-ತ್ತಿ-ರು-ವಾಗ
ಹೋಗು-ತ್ತಿವೆ
ಹೋಗುತ್ತೀ
ಹೋಗು-ತ್ತೇ-ನಮ್ಮಾ
ಹೋಗು-ತ್ತೇನೆ
ಹೋಗು-ತ್ತೇ-ನೆಂದು
ಹೋಗು-ತ್ತೇವೆ
ಹೋಗುವ
ಹೋಗು-ವಂ-ತಹ
ಹೋಗು-ವಂ-ತಾ-ಗಲು
ಹೋಗು-ವಂ-ತಿಲ್ಲ
ಹೋಗು-ವಂತೆ
ಹೋಗು-ವನು
ಹೋಗು-ವರು
ಹೋಗು-ವ-ವನು
ಹೋಗು-ವ-ಷ್ಟ-ರಲ್ಲಿ
ಹೋಗು-ವಾಗ
ಹೋಗುವು
ಹೋಗು-ವು-ದ-ಕ್ಕಾಗಿ
ಹೋಗು-ವು-ದಕ್ಕೆ
ಹೋಗು-ವುದನ್ನು
ಹೋಗು-ವು-ದ-ರಿಂದ
ಹೋಗು-ವು-ದ-ರೊ-ಳ-ಗಾಗಿ
ಹೋಗು-ವು-ದ-ಲ್ಲವೆ
ಹೋಗು-ವು-ದಿಲ್ಲ
ಹೋಗು-ವುದು
ಹೋಗು-ವುದೇ
ಹೋಗು-ವು-ದೇಕೆ
ಹೋಗು-ವುದೋ
ಹೋಗುವೆ
ಹೋಗು-ವೆ-ನೆಂ-ದರೆ
ಹೋಗೋಣ
ಹೋಗೋ-ಣವೇ
ಹೋಡೆ-ದೋ-ಡಿ-ಸ-ಬೇ-ಕೆಂದು
ಹೋತ್ರ
ಹೋತ್ರದ
ಹೋದ
ಹೋದಂ-ತಾ-ಯಿತು
ಹೋದಂತೆ
ಹೋದ-ಕೊ-ಡಲೆ
ಹೋದತ್ತ
ಹೋದ-ನಂತೆ
ಹೋದನು
ಹೋದನೆ
ಹೋದ-ನೆಂಬ
ಹೋದನೊ
ಹೋದ-ಮೇಲೂ
ಹೋದ-ಮೇಲೆ
ಹೋದರು
ಹೋದರೂ
ಹೋದರೆ
ಹೋದ-ರೆಂದು
ಹೋದ-ರೆಂ-ಬುದು
ಹೋದ-ರೆಂ-ಬುದೇ
ಹೋದ-ರೇನು
ಹೋದರೊ
ಹೋದ-ಳಂತೆ
ಹೋದಳು
ಹೋದ-ವನು
ಹೋದ-ವನೆ
ಹೋದ-ವಳೆ
ಹೋದವು
ಹೋದ-ವೆಂ-ದಾ-ಗಲಿ
ಹೋದ-ಹಾ-ದಿ-ಯನ್ನು
ಹೋದಾಗ
ಹೋದಾರು
ಹೋದು
ಹೋದು-ದ-ಕ್ಕಾಗಿ
ಹೋದು-ದನ್ನು
ಹೋದು-ದನ್ನೂ
ಹೋದು-ದ-ರಿಂದ
ಹೋದುದು
ಹೋದು-ದೇಕೆ
ಹೋದೆ
ಹೋದೆಯಾ
ಹೋದೆವು
ಹೋಮ
ಹೋಮ-ಕಾ-ರ್ಯ-ವನ್ನು
ಹೋಮ-ಕುಂ-ಡದ
ಹೋಮ-ಕುಂ-ಡ-ದಿಂದ
ಹೋಮದ
ಹೋಮ-ಭಾ-ಗ-ವಿಲ್ಲ
ಹೋಮ-ಮಾಡ
ಹೋಮ-ಮಾಡಿ
ಹೋಮ-ಮಾ-ಡಿದ
ಹೋಮ-ಮಾ-ಡಿ-ದಂ-ತಾ-ಯಿತು
ಹೋಮ-ಮಾ-ಡುತ್ತಾ
ಹೋಮ-ವನ್ನು
ಹೋಮಾ-ಗ್ನಿ-ಯಲ್ಲಿ
ಹೋಮಾದಿ
ಹೋಯಿ-ತಲ್ಲ
ಹೋಯಿ-ತಾ-ದರೂ
ಹೋಯಿತು
ಹೋಯಿ-ತೆಂದು
ಹೋಯಿತೊ
ಹೋಯೆಂದು
ಹೋರಾ
ಹೋರಾಟ
ಹೋರಾ-ಟಕ್ಕೆ
ಹೋರಾ-ಟ-ದಲ್ಲಿ
ಹೋರಾ-ಟ-ವನ್ನು
ಹೋರಾಡಿ
ಹೋರಾ-ಡಿ-ದರು
ಹೋರಾ-ಡಿ-ದವು
ಹೋರಾ-ಡುತ್ತಾ
ಹೋರಾ-ಡು-ತ್ತಿದ್ದ
ಹೋರಾ-ಡು-ತ್ತಿ-ದ್ದನು
ಹೋರಾ-ಡು-ತ್ತಿ-ದ್ದರು
ಹೋರಾ-ಡು-ತ್ತಿ-ದ್ದರೆ
ಹೋರಾ-ಡು-ವುದು
ಹೋರಾ-ಡು-ವು-ದೆಂ-ದರೆ
ಹೋಲಿಕೆ
ಹೋಲಿಸಿ
ಹೋಲು-ತ್ತಿದ್ದ
ಹೋಲು-ತ್ತಿ-ದ್ದಾನೆ
ಹೋಳಾಗಿ
ಹೋಳಾ-ಯಿತು
ಹೌದೆಂದು
ಹೌಹಾ-ರಿತು
ಹ್ಯತ್ರ
ಹ್ಯಪಿ
ಹ್ಯುತ್ತ-ಮ-ಶ್ಲೋ-ಕ-ಜ-ಲ್ಪೈಃ
ಹ್ಯುತ್ತ-ಮ-ಶ್ಲೋ-ಕ-ಶಬ್ದಃ
ಹ್ರಾಂ
ಹ್ರೀಂ
ಹ್ಲಾದ
}
