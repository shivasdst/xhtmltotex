
\chapter{ಶ್ರೀಶಿವಸಹಸ್ರನಾಮಸ್ತೋತ್ರಮ್​}

\begin{verse}
ಸ್ಥಿರಃ ಸ್ಥಾಣುಃ ಪ್ರಭುರ್ಭೀಮಃ ಪ್ರವರೋ ವರದೋ ವರಃ ।\\ಸರ್ವಾತ್ಮಾ ಸರ್ವವಿಖ್ಯಾತಃ ಸರ್ವಃ ಸರ್ವಕರೋ ಭವಃ \num{॥ ೧ ॥}
\end{verse}

\begin{verse}
ಜಟೀ ಚರ್ಮೀ ಶಿಖಂಡೀ ಚ ಸರ್ವಾಂಗಃ ಸರ್ವಭಾವನಃ ।\\ಹರಶ್ಚ ಹರಿಣಾಕ್ಷಶ್ಚ ಸರ್ವಭೂತಹರಃ ಪ್ರಭುಃ \num{॥ ೨ ॥}
\end{verse}

\begin{verse}
ಪ್ರವೃತ್ತಿಶ್ಚ ನಿವೃತ್ತಿಶ್ಚ ನಿಯತಃ ಶಾಶ್ವತೋ ಧ್ರುವಃ ।\\ಶ್ಮಶಾನವಾಸೀ ಭಗವನ್ ಖಚರೋ ಗೋಚರೋಽರ್ದನಃ \num{॥ ೩ ॥}
\end{verse}

\begin{verse}
ಅಭಿವಾದ್ಯೋ ಮಹಾಕರ್ಮಾ ತಪಸ್ವೀ ಭೂತಭಾವನಃ ।\\ಉನ್ಮತ್ತವೇಷಪ್ರಚ್ಛನ್ನಃ ಸರ್ವಲೋಕಪ್ರಜಾಪತಿಃ \num{॥ ೪ ॥}
\end{verse}

\begin{verse}
ಮಹಾರೂಪೋ ಮಹಾಕಾಯೋ ವೃಷರೂಪೋ ಮಹಾಯಶಾಃ ।\\ಮಹಾತ್ಮಾ ಸರ್ವಭೂತಾತ್ಮಾ ವಿಶ್ವರೂಪೋ ಮಹಾಹನುಃ \num{॥ ೫ ॥}
\end{verse}

\begin{verse}
ಲೋಕಪಾಲೋಽಂತರ್ಹಿತಾತ್ಮಾ ಪ್ರಸಾದೋ ಹಯಗರ್ದಭಿಃ ।\\ಪವಿತ್ರಂ ಚ ಮಹಾಂಶ್ಚೈವ ನಿಯಮೋ ನಿಯಮಾಶ್ರಿತಃ \num{॥ ೬ ॥}
\end{verse}

\begin{verse}
ಸರ್ವಕರ್ಮಾ ಸ್ವಯಂಭೂತ ಆದಿರಾದಿಕರೋ ನಿಧಿಃ ।\\ಸಹಸ್ರಾಕ್ಷೋ ವಿಶಾಲಾಕ್ಷಃ ಸೋಮೋ ನಕ್ಷತ್ರಸಾಧಕಃ \num{॥ ೭ ॥}
\end{verse}

\begin{verse}
ಚಂದ್ರಃ ಸೂರ್ಯಃ ಶನಿಃ ಕೇತುರ್ಗ್ರಹೋ ಗ್ರಹಪತಿರ್ವರಃ ।\\ಅತ್ರಿರತ್ರ್ಯಾ ನಮಸ್ಕರ್ತಾ ಮೃಗಬಾಣಾರ್ಪಣೋಽನಘಃ \num{॥ ೮ ॥}
\end{verse}

\begin{verse}
ಮಹಾತಪಾ ಘೋರತಪಾ ಅದೀನೋ ದೀನಸಾಧಕಃ ।\\ಸಂವತ್ಸರಕರೋ ಮಂತ್ರಃ ಪ್ರಮಾಣಂ ಪರಮಂ ತಪಃ \num{॥ ೯ ॥}
\end{verse}

\begin{verse}
ಯೋಗೀ ಯೋಜ್ಯೋ ಮಹಾಬೀಜೋ ಮಹಾರೇತಾ ಮಹಾಬಲಃ ।\\ಸುವರ್ಣರೇತಾಃ ಸರ್ವಜ್ಞಃ ಸುಬೀಜೋ ಬೀಜವಾಹನಃ \num{॥ ೧೦ ॥}
\end{verse}

\begin{verse}
ದಶಬಾಹುಸ್ತ್ವನಿಮಿಷೋ ನೀಲಕಂಠ ಉಮಾಪತಿಃ ।\\ವಿಶ್ವರೂಪಃ ಸ್ವಯಂ ಶ್ರೇಷ್ಠೋ ಬಲವೀರೋಽಬಲೋ ಗಣಃ \num{॥ ೧೧ ॥}
\end{verse}

\begin{verse}
ಗಣಕರ್ತಾ ಗಣಪತಿರ್ದಿಗ್ವಾಸಾಃ ಕಾಮ ಏವ ಚ ।\\ಮಂತ್ರವಿತ್ಪರಮೋ ಮಂತ್ರಃ ಸರ್ವಭಾವಕರೋ ಹರಃ \num{॥ ೧೨ ॥}
\end{verse}

\begin{verse}
ಕಮಂಡಲುಧರೋ ಧನ್ವೀ ಬಾಣಹಸ್ತಃ ಕಪಾಲವಾನ್ ।\\ಅಶನೀ ಶತಘ್ನೀ ಖಡ್ಗೀ ಪಟ್ಟಿಶೀ ಚಾಯುಧೀ ಮಹಾನ್ \num{॥ ೧೩ ॥}
\end{verse}

\begin{verse}
ಸ್ರುವಹಸ್ತಃ ಸುರೂಪಶ್ಚ ತೇಜಸ್ತೇಜಸ್ಕರೋ ನಿಧಿಃ ।\\ಉಷ್ಣೀಷೀ ಚ ಸುವಕ್ತ್ರಶ್ಚ ಉದಗ್ರೋ ವಿನತಸ್ತಥಾ \num{॥ ೧೪ ॥}
\end{verse}

\begin{verse}
ದೀರ್ಘಶ್ಚ ಹರಿಕೇಶಶ್ಚ ಸುತೀರ್ಥಃ ಕೃಷ್ಣ ಏವ ಚ ।\\ಶೃಗಾಲರೂಪಃ ಸಿದ್ಧಾರ್ಥೋ ಮುಂಡಃ ಸರ್ವಶುಭಂಕರಃ \num{॥ ೧೫ ॥}
\end{verse}

\begin{verse}
ಅಜಶ್ಚ ಬಹುರೂಪಶ್ಚ ಗಂಧಧಾರೀ ಕಪರ್ದ್ಯಪಿ ।\\ಊರ್ಧ್ವರೇತಾ ಊರ್ಧ್ವಲಿಂಗ ಊರ್ಧ್ವಶಾಯೀ ನಭಃಸ್ಥಲಃ \num{॥ ೧೬ ॥}
\end{verse}

\begin{verse}
ತ್ರಿಜಟೀ ಚೀರವಾಸಾಶ್ಚ ರುದ್ರಃ ಸೇನಾಪತಿರ್ವಿಭುಃ ।\\ಅಹಶ್ಚರೋ ನಕ್ತಂಚರಸ್ತಿಗ್ಮಮನ್ಯುಃ ಸುವರ್ಚಸಃ \num{॥ ೧೭ ॥}
\end{verse}

\begin{verse}
ಗಜಹಾ ದೈತ್ಯಹಾ ಕಾಲೋ ಲೋಕಧಾತಾ ಗುಣಾಕರಃ ।\\ಸಿಂಹಶಾರ್ದೂಲರೂಪಶ್ಚ ಆರ್ದ್ರಚರ್ಮಾಂಬರಾವೃತಃ \num{॥ ೧೮ ॥}
\end{verse}

\begin{verse}
ಕಾಲಯೋಗೀ ಮಹಾನಾದಃ ಸರ್ವಕಾಮಾಶ್ಚತುಷ್ಪದಃ ।\\ನಿಶಾಚರಃ ಪ್ರೇತಚಾರೀ ಭೂತಚಾರೀ ಮಹೇಶ್ವರಃ \num{॥ ೧೯ ॥}
\end{verse}

\begin{verse}
ಬಹುಭೂತೋ ಬಹುಧರಃ ಸ್ವರ್ಭಾನುರಮಿತೋ ಗತಿಃ ।\\ನೃತ್ಯಪ್ರಿಯೋ ನಿತ್ಯನರ್ತೋ ನರ್ತಕಃ ಸರ್ವಲಾಲಸಃ \num{॥ ೨೦ ॥}
\end{verse}

\begin{verse}
ಘೋರೋ ಮಹಾತಪಾಃ ಪಾಶೋ ನಿತ್ಯೋ ಗಿರಿರುಹೋ ನಭಃ ।\\ಸಹಸ್ರಹಸ್ತೋ ವಿಜಯೋ ವ್ಯವಸಾಯೋ ಹ್ಯತಂದ್ರಿತಃ \num{॥ ೨೧ ॥}
\end{verse}

\begin{verse}
ಅಧರ್ಷಣೋ ಧರ್ಷಣಾತ್ಮಾ ಯಜ್ಞಹಾ ಕಾಮನಾಶಕಃ ।\\ದಕ್ಷಯಾಗಾಪಹಾರೀ ಚ ಸುಸಹೋ ಮಧ್ಯಮಸ್ತಥಾ \num{॥ ೨೨ ॥}
\end{verse}

\begin{verse}
ತೇಜೋಽಪಹಾರೀ ಬಲಹಾ ಮುದಿತೋಽರ್ಧೋಽಜಿತೋಽವರಃ ।\\ಗಂಭೀರಘೋಷೋ ಗಂಭೀರೋ ಗಂಭೀರಬಲವಾಹನಃ \num{॥ ೨೩ ॥}
\end{verse}

\begin{verse}
ನ್ಯಗ್ರೋಧರೂಪೋ ನ್ಯಗ್ರೋಧೋ ವೃಕ್ಷಕರ್ಣಸ್ಥಿತಿರ್ವಿಭುಃ ।\\ಸುತೀಕ್ಷ್ಣದಶನಶ್ಚೈವ ಮಹಾಕಾಯೋ ಮಹಾಽನನಃ \num{॥ ೨೪ ॥}
\end{verse}

\begin{verse}
ವಿಷ್ವಕ್ಸೇನೋ ಹರಿರ್ಯಜ್ಞಃ ಸಂಯುಗಾಪೀಡವಾಹನಃ ।\\ತೀಕ್ಷ್ಣತಾಪಶ್ಚ ಹರ್ಯಶ್ವಃ ಸಹಾಯಃ ಕರ್ಮಕಾಲವಿತ್ \num{॥ ೨೫ ॥}
\end{verse}

\begin{verse}
ವಿಷ್ಣುಪ್ರಸಾದಿತೋ ಯಜ್ಞಃ ಸಮುದ್ರೋ ವಡವಾಮುಖಃ ।\\ಹುತಾಶನಸಹಾಯಶ್ಚ ಪ್ರಶಾಂತಾತ್ಮಾ ಹುತಾಶನಃ \num{॥ ೨೬ ॥}
\end{verse}

\begin{verse}
ಉಗ್ರತೇಜಾ ಮಹಾತೇಜಾ ಜನ್ಯೋ ವಿಜಯಕಾಲವಿತ್ ।\\ಜ್ಯೋತಿಷಾಮಯನಂ ಸಿದ್ಧಿಃ ಸರ್ವವಿಗ್ರಹ ಏವ ಚ \num{॥ ೨೭ ॥}
\end{verse}

\begin{verse}
ಶಿಖೀ ಮುಂಡೀ ಜಟೀ ಜ್ವಾಲೀ ಮೂರ್ತಿಜೋ ಮೂರ್ಧಗೋ ಬಲೀ ।\\ವೇಣವೀ ಪಣವೀ ತಾಲೀ ಖಲೀ ಕಾಲಕಟಂಕಟಃ \num{॥ ೨೮ ॥}
\end{verse}

\begin{verse}
ನಕ್ಷತ್ರ ವಿಗ್ರಹಮತಿರ್ಗುಣಬುದ್ಧಿರ್ಲಯೋಽಗಮಃ ।\\ಪ್ರಜಾಪತಿರ್ವಿಶ್ವಬಾಹುರ್ವಿಭಾಗಃ ಸರ್ವಗೋಽಮುಖಃ \num{॥ ೨೯ ॥}
\end{verse}

\begin{verse}
ವಿಮೋಚನಃ ಸುಸರಣೋ ಹಿರಣ್ಯಕವಚೋದ್ಭವಃ ।\\ಮೇಢ್ರಜೋ ಬಲಚಾರೀ ಚ ಮಹೀಚಾರೀ ಸ್ರುತಸ್ತಥಾ \num{॥ ೩೦ ॥}
\end{verse}

\begin{verse}
ಸರ್ವತೂರ್ಯನಿನಾದೀ ಚ ಸರ್ವಾತೋದ್ಯಪರಿಗ್ರಹಃ ।\\ವ್ಯಾಲರೂಪೋ ಗುಹಾವಾಸೀ ಗುಹೋ ಮಾಲೀ ತರಂಗವಿತ್ \num{॥ ೩೧ ॥}
\end{verse}

\begin{verse}
ತ್ರಿದಶಸ್ತ್ರಿಕಾಲಧೃತ್ ಕರ್ಮಸರ್ವಬಂಧವಿಮೋಚನಃ ।\\ಬಂಧನಸ್ತ್ವಸುರೇಂದ್ರಾಣಾಂ ಯುಧಿ ಶತ್ರುವಿನಾಶನಃ \num{॥ ೩೨ ॥}
\end{verse}

\begin{verse}
ಸಾಂಖ್ಯಪ್ರಸಾದೋ ದುರ್ವಾಸಾಃ ಸರ್ವಸಾಧುನಿಷೇವಿತಃ ।\\ಪ್ರಸ್ಕಂದನೋ ವಿಭಾಗಜ್ಞೋಽತುಲ್ಯೋ ಯಜ್ಞವಿಭಾಗವಿತ್ \num{॥ ೩೩ ॥}
\end{verse}

\begin{verse}
ಸರ್ವವಾಸಃ ಸರ್ವಚಾರೀ ದುರ್ವಾಸಾ ವಾಸವೋಽಮರಃ ।\\ಹೈಮೋ ಹೇಮಕರೋಽಯಜ್ಞಃ ಸರ್ವಧಾರೀ ಧರೋತ್ತಮಃ \num{॥ ೩೪ ॥}
\end{verse}

\begin{verse}
ಲೋಹಿತಾಕ್ಷೋ ಮಹಾಕ್ಷಶ್ಚ ವಿಜಯಾಕ್ಷೋ ವಿಶಾರದಃ ।\\ಸಂಗ್ರಹೋ ನಿಗ್ರಹಃ ಕರ್ತಾ ಸರ್ಪಚೀರನಿವಾಸನಃ \num{॥ ೩೫ ॥}
\end{verse}

\begin{verse}
ಮುಖ್ಯೋಽಮುಖ್ಯಶ್ಚ ದೇಹಶ್ಚ ಕಾಹಲಿಃ ಸರ್ವಕಾಮದಃ ।\\ಸರ್ವಕಾಲಪ್ರಸಾದಶ್ಚ ಸುಬಲೋ ಬಲರೂಪಧೃತ್ \num{॥ ೩೬ ॥}
\end{verse}

\begin{verse}
ಸರ್ವಕಾಮವರಶ್ಚೈವ ಸರ್ವದಃ ಸರ್ವತೋಮುಖಃ ।\\ಆಕಾಶನಿರ್ವಿರೂಪಶ್ಚ ನಿಪಾತೀ ಹ್ಯವಶಃ ಖಗಃ \num{॥ ೩೭ ॥}
\end{verse}

\begin{verse}
ರೌದ್ರರೂಪೋಽಂಶುರಾದಿತ್ಯೋ ಬಹುರಶ್ಮಿಃ ಸುವರ್ಚಸೀ ।\\ವಸುವೇಗೋ ಮಹಾವೇಗೋ ಮನೋವೇಗೋ ನಿಶಾಚರಃ \num{॥ ೩೮ ॥}
\end{verse}

\begin{verse}
ಸರ್ವವಾಸೀ ಶ್ರಿಯಾವಾಸೀ ಉಪದೇಶಕರೋಽಕರಃ ।\\ಮುನಿರಾತ್ಮನಿರಾಲೋಕಃ ಸಂಭಗ್ನಶ್ಚ ಸಹಸ್ರದಃ \num{॥ ೩೯ ॥}
\end{verse}

\begin{verse}
ಪಕ್ಷೀ ಚ ಪಕ್ಷರೂಪಶ್ಚ ಅತಿದೀಪ್ತೋ ವಿಶಾಂಪತಿಃ ।\\ಉನ್ಮಾದೋ ಮದನಃ ಕಾಮೋ ಹ್ಯಶ್ವತ್ಥೋಽರ್ಥಕರೋ ಯಶಃ \num{॥ ೪೦ ॥}
\end{verse}

\begin{verse}
ವಾಮದೇವಶ್ಚ ವಾಮಶ್ಚ ಪ್ರಾಗ್ದಕ್ಷಿಣಶ್ಚ ವಾಮನಃ ।\\ಸಿದ್ಧಯೋಗೀ ಮಹರ್ಷಿಶ್ಚ ಸಿದ್ಧಾರ್ಥಃ ಸಿದ್ಧಸಾಧಕಃ \num{॥ ೪೧ ॥}
\end{verse}

\begin{verse}
ಭಿಕ್ಷುಶ್ಚ ಭಿಕ್ಷುರೂಪಶ್ಚ ವಿಪಣೋ ಮೃದುರವ್ಯಯಃ ।\\ಮಹಾಸೇನೋ ವಿಶಾಖಶ್ಚ ಷಷ್ಟಿಭಾಗೋ ಗವಾಂಪತಿಃ \num{॥ ೪೨ ॥}
\end{verse}

\begin{verse}
ವಜ್ರಹಸ್ತಶ್ಚ ವಿಷ್ಕಂಭೀ ಚಮೂಸ್ತಂಭನ ಏವ ಚ ।\\ವೃತ್ತಾವೃತ್ತಕರಸ್ತಾಲೋ ಮಧುರ್ಮಧುಕಲೋಚನಃ \num{॥ ೪೩ ॥}
\end{verse}

\begin{verse}
ವಾಚಸ್ಪತ್ಯೋ ವಾಜಸನೋ ನಿತ್ಯಮಾಶ್ರಮಪೂಜಿತಃ ।\\ಬ್ರಹ್ಮಚಾರೀ ಲೋಕಚಾರೀ ಸರ್ವಚಾರೀ ವಿಚಾರವಿತ್ \num{॥ ೪೪ ॥}
\end{verse}

\begin{verse}
ಈಶಾನ ಈಶ್ವರಃ ಕಾಲೋ ನಿಶಾಚಾರೀ ಪಿನಾಕವಾನ್ ।\\ನಿಮಿತ್ತಸ್ಥೋ ನಿಮಿತ್ತಂ ಚ ನಂದಿರ್ನಂದಕರೋ ಹರಿಃ \num{॥ ೪೫ ॥}
\end{verse}

\begin{verse}
ನಂದೀಶ್ವರಶ್ಚ ನಂದೀ ಚ ನಂದನೋ ನಂದಿವರ್ಧನಃ ।\\ಭಗಹಾರೀ ನಿಹಂತಾ ಚ ಕಾಲೋ ಬ್ರಹ್ಮಾ ಪಿತಾಮಹಃ \num{॥ ೪೬ ॥}
\end{verse}

\begin{verse}
ಚತುರ್ಮುಖೋ ಮಹಾಲಿಂಗಶ್ಚಾರುಲಿಂಗಸ್ತಥೈವ ಚ ।\\ಲಿಂಗಾಧ್ಯಕ್ಷಃ ಸುರಾಧ್ಯಕ್ಷೋ ಯೋಗಾಧ್ಯಕ್ಷೋ ಯುಗಾವಹಃ \num{॥ ೪೭ ॥}
\end{verse}

\begin{verse}
ಬೀಜಾಧ್ಯಕ್ಷೋ ಬೀಜಕರ್ತಾ ಅಧ್ಯಾತ್ಮಾನುಗತೋ ಬಲಃ ।\\ಇತಿಹಾಸಃ ಸಕಲ್ಪಶ್ಚ ಗೌತುಮೋಽಥ ನಿಶಾಕರಃ \num{॥ ೪೮ ॥}
\end{verse}

\begin{verse}
ದಂಭೋ ಹ್ಯದಂಭೋ ವೈದಂಭೋ ವಶ್ಯೋ ವಶಕರಃ ಕಲಿಃ ।\\ಲೋಕಕರ್ತಾ ಪಶುಪತಿರ್ಮಹಾಕರ್ತಾ ಹ್ಯನೌಷಧಃ \num{॥ ೪೯ ॥}
\end{verse}

\begin{verse}
ಅಕ್ಷರಂ ಪರಮಂ ಬ್ರಹ್ಮ ಬಲವಚ್ಛಕ್ರ ಏವ ಚ ।\\ನೀತಿರ್ಹ್ಯನೀತಿಃ ಶುದ್ಧಾತ್ಮಾ ಶುದ್ಧೋ ಮಾನ್ಯೋ ಗತಾಗತಃ \num{॥ ೫೦ ॥}
\end{verse}

\begin{verse}
ಬಹುಪ್ರಸಾದಃ ಸುಸ್ವಪ್ನೋ ದರ್ಪಣೋಽಥ ತ್ವಮಿತ್ರಜಿತ್ ।\\ವೇದಕಾರೋ ಮಂತ್ರಕಾರೋ ವಿದ್ವಾನ್ ಸಮರಮರ್ದನಃ \num{॥ ೫೧ ॥}
\end{verse}

\begin{verse}
ಮಹಾಮೇಘನಿವಾಸೀ ಚ ಮಹಾಘೋರೋ ವಶೀ ಕರಃ ।\\ಅಗ್ನಿಜ್ವಾಲೋ ಮಹಾಜ್ವಾಲೋ ಅತಿಧೂಮ್ರೋ ಹುತೋ ಹವಿಃ \num{॥ ೫೨ ॥}
\end{verse}

\begin{verse}
ವೃಷಣಃ ಶಂಕರೋ ನಿತ್ಯಂ ವರ್ಚಸ್ವೀ ಧೂಮಕೇತನಃ ।\\ನೀಲಸ್ತಥಾಂಗಲುಬ್ಧಶ್ಚ ಶೋಭನೋ ನಿರವಗ್ರಹಃ \num{॥ ೫೩ ॥}
\end{verse}

\begin{verse}
ಸ್ವಸ್ತಿದಃ ಸ್ವಸ್ತಿಭಾವಶ್ಚ ಭಾಗೀ ಭಾಗಕರೋ ಲಘುಃ ।\\ಉತ್ಸಂಗಶ್ಚ ಮಹಾಂಗಶ್ಚ ಮಹಾಗರ್ಭಪರಾಯಣಃ \num{॥ ೫೪ ॥}
\end{verse}

\begin{verse}
ಕೃಷ್ಣವರ್ಣಃ ಸುವರ್ಣಶ್ಚ ಇಂದ್ರಿಯಂ ಸರ್ವದೇಹಿನಾಮ್ ।\\ಮಹಾಪಾದೋ ಮಹಾಹಸ್ತೋ ಮಹಾಕಾಯೋ ಮಹಾಯಶಾಃ \num{॥ ೫೫ ॥}
\end{verse}

\begin{verse}
ಮಹಾಮೂರ್ಧಾ ಮಹಾಮಾತ್ರೋ ಮಹಾನೇತ್ರೋ ನಿಶಾಲಯಃ ।\\ಮಹಾಂತಕೋ ಮಹಾಕರ್ಣೋ ಮಹೋಷ್ಠಶ್ಚ ಮಹಾಹನುಃ \num{॥ ೫೬ ॥}
\end{verse}

\begin{verse}
ಮಹಾನಾಸೋ ಮಹಾಕಂಬುರ್ಮಹಾಗ್ರೀವಃ ಶ್ಮಶಾನಭಾಕ್ ।\\ಮಹಾವಕ್ಷಾ ಮಹೋರಸ್ಕೋ ಹ್ಯಂತರಾತ್ಮಾ ಮೃಗಾಲಯಃ \num{॥ ೫೭ ॥}
\end{verse}

\begin{verse}
ಲಂಬನೋ ಲಂಬಿತೋಷ್ಠಶ್ಚ ಮಹಾಮಾಯಃ ಪಯೋನಿಧಿಃ ।\\ಮಹಾದಂತೋ ಮಹಾದಂಷ್ಟ್ರೋ ಮಹಾಜಿಹ್ವೋ ಮಹಾಮುಖಃ \num{॥ ೫೮ ॥}
\end{verse}

\begin{verse}
ಮಹಾನಖೋ ಮಹಾರೋಮಾ ಮಹಾಕೇಶೋ ಮಹಾಜಟಃ ।\\ಪ್ರಸನ್ನಶ್ಚ ಪ್ರಸಾದಶ್ಚ ಪ್ರತ್ಯಯೋ ಗಿರಿಸಾಧನಃ \num{॥ ೫೯ ॥}
\end{verse}

\begin{verse}
ಸ್ನೇಹನೋಽಸ್ನೇಹನಶ್ಚೈವ ಅಜಿತಶ್ಚ ಮಹಾಮುನಿಃ ।\\ವೃಕ್ಷಾಕಾರೋ ವೃಕ್ಷಕೇತುರನಲೋ ವಾಯುವಾಹನಃ \num{॥ ೬೦ ॥}
\end{verse}

\begin{verse}
ಗಂಡಲೀ ಮೇರುಧಾಮಾ ಚ ದೇವಾಧಿಪತಿರೇವ ಚ ।\\ಅಥರ್ವಶೀರ್ಷಃ ಸಾಮಾಸ್ಯ ಪುಕ್ಸಹಸ್ರಾಮಿತೇಕ್ಷಣಃ \num{॥ ೬೧ ॥}
\end{verse}

\begin{verse}
ಯಜುಃಪಾದಭುಜೋ ಗುಹ್ಯಃ ಪ್ರಕಾಶೋ ಜಂಗಮಸ್ತಥಾ ।\\ಅಮೋಘಾರ್ಥಃ ಪ್ರಸಾದಶ್ಚ ಅಭಿಗಮ್ಯಃ ಸುದರ್ಶನಃ \num{॥ ೬೨ ॥}
\end{verse}

\begin{verse}
ಉಪಕಾರಃ ಪ್ರಿಯಃ ಸರ್ವಃ ಕನಕಃ ಕಾಂಚನಚ್ಛವಿಃ ।\\ನಾಭಿರ್ನಂದಿಕರೋ ಭಾವಃ ಪುಷ್ಕರಸ್ಥಪತಿಃ ಸ್ಥಿರಃ \num{॥ ೬೩ ॥}
\end{verse}

\begin{verse}
ದ್ವಾದಶಸ್ತ್ರಾಸನಶ್ಚಾದ್ಯೋ ಯಜ್ಞೋ ಯಜ್ಞಸಮಾಹಿತಃ ।\\ನಕ್ತಂ ಕಲಿಶ್ಚ ಕಾಲಶ್ಚ ಮಕರಃ ಕಾಲಪೂಜಿತಃ \num{॥ ೬೪ ॥}
\end{verse}

\begin{verse}
ಸಗಣೋ ಗಣಕಾರಶ್ಚ ಭೂತವಾಹನಸಾರಥಿಃ ।\\ಭಸ್ಮಶಯೋ ಭಸ್ಮಗೋಪ್ತಾ ಭಸ್ಮಭೂತಸ್ತರುರ್ಗಣಃ \num{॥ ೬೫ ॥}
\end{verse}

\begin{verse}
ಲೋಕಪಾಲಸ್ತಥಾಲೋಕೋ ಮಹಾತ್ಮಾ ಸರ್ವಪೂಜಿತಃ ।\\ಶುಕ್ಲಸ್ತ್ರಿಶುಕ್ಲಃ ಸಂಪನ್ನಃ ಶುಚಿರ್ಭೂತನಿಷೇವಿತಃ \num{॥ ೬೬ ॥}
\end{verse}

\begin{verse}
ಆಶ್ರಮಸ್ಥಃ ಕ್ರಿಯಾವಸ್ಥೋ ವಿಶ್ವಕರ್ಮಮತಿರ್ವರಃ ।\\ವಿಶಾಲಶಾಖಸ್ತಾಮ್ರೋಷ್ಠೋ ಹ್ಯಂಬುಜಾಲಃ ಸುನಿಶ್ಚಲಃ \num{॥ ೬೭ ॥}
\end{verse}

\begin{verse}
ಕಪಿಲಃ ಕಪಿಶಃ ಶುಕ್ಲ ಆಯುಶ್ಚೈವ ಪರೋಽಪರಃ ।\\ಗಂಧರ್ವೋ ಹ್ಯದಿತಿಸ್ತಾರ್ಕ್ಷ್ಯಃ ಸುವಿಜ್ಞೇಯಃ ಸುಶಾರದಃ \num{॥ ೬೮ ॥}
\end{verse}

\begin{verse}
ಪರಶ್ವಧಾಯುಧೋ ದೇವೋ ಅನುಕಾರೀ ಸುಬಾಂಧವಃ ।\\ತುಂಬವೀಣೋ ಮಹಾಕ್ರೋಧ ಊರ್ಧ್ವರೇತಾ ಜಲೇಶಯಃ \num{॥ ೬೯ ॥}
\end{verse}

\begin{verse}
ಉಗ್ರೋ ವಂಶಕರೋ ವಂಶೋ ವಂಶನಾದೋ ಹ್ಯನಿಂದಿತಃ ।\\ಸರ್ವಾಂಗರೂಪೋ ಮಾಯಾವೀ ಸುಹೃದೋ ಹ್ಯನಿಲೋಽನಲಃ \num{॥ ೭೦ ॥}
\end{verse}

\begin{verse}
ಬಂಧನೋ ಬಂಧಕರ್ತಾ ಚ ಸುಬಂಧನವಿಮೋಚನಃ ।\\ಸಯಜ್ಞಾರಿಃ ಸಕಾಮಾರಿರ್ಮಹಾದಂಷ್ಟ್ರೋ ಮಹಾಯುಧಃ \num{॥ ೭೧ ॥}
\end{verse}

\begin{verse}
ಬಹುಧಾ ನಿಂದಿತಃ ಶರ್ವಃ ಶಂಕರಃ ಶಂಕರೋಽಧನಃ ।\\ಅಮರೇಶೋ ಮಹಾದೇವೋ ವಿಶ್ವದೇವಃ ಸುರಾರಿಹಾ \num{॥ ೭೨ ॥}
\end{verse}

\begin{verse}
ಅಹಿರ್ಬುಧ್ನ್ಯೋಽನಿಲಾಭಶ್ಚ ಚೇಕಿತಾನೋ ಹವಿಸ್ತಥಾ ।\\ಅಜೈಕಪಾಚ್ಚ ಕಾಪಾಲೀ ತ್ರಿಶಂಕುರಜಿತಃ ಶಿವಃ \num{॥ ೭೩ ॥}
\end{verse}

\begin{verse}
ಧನ್ವಂತರಿರ್ಧೂಮಕೇತುಃ ಸ್ಕಂದೋ ವೈಶ್ರವಣಸ್ತಥಾ ।\\ಧಾತಾ ಶಕ್ರಶ್ಚ ವಿಷ್ಣುಶ್ಚ ಮಿತ್ರಸ್ತ್ವಷ್ಟಾಧ್ರುವೋ ಧರಃ \num{॥ ೭೪ ॥}
\end{verse}

\begin{verse}
ಪ್ರಭಾವಃ ಸರ್ವಗೋ ವಾಯುರರ್ಯಮಾ ಸವಿತಾ ರವಿಃ ।\\ಉಷಂಗುಶ್ಚ ವಿಧಾತಾ ಚ ಮಾಂಧಾತಾ ಭೂತಭಾವನಃ \num{॥ ೭೫ ॥}
\end{verse}

\begin{verse}
ವಿಭುರ್ವರ್ಣವಿಭಾವೀ ಚ ಸರ್ವಕಾಮಗುಣಾವಹಃ ।\\ಪದ್ಮನಾಭೋ ಮಹಾಗರ್ಭಶ್ಚಂದ್ರವಕ್ತ್ರೋಽನಿಲೋಽನಲಃ \num{॥ ೭೬ ॥}
\end{verse}

\begin{verse}
ಬಲವಾಂಶ್ಚೋಪಶಾಂತಶ್ಚ ಪುರಾಣಃ ಪುಣ್ಯಚುಂಚುರೀ ।\\ಕುರುಕರ್ತಾ ಕುರುವಾಸೀ ಕುರುಭೂತೋ ಗುಣೌಷಧಃ \num{॥ ೭೭ ॥}
\end{verse}

\begin{verse}
ಸರ್ವಾಶಯೋ ದರ್ಭಚಾರೀ ಸರ್ವೇಷಾಂ ಪ್ರಾಣಿನಾಂ ಪತಿಃ ।\\ದೇವದೇವಃ ಸುಖಾಸಕ್ತಃ ಸದಸತ್ಸರ್ವರತ್ನವಿತ್ \num{॥ ೭೮ ॥}
\end{verse}

\begin{verse}
ಕೈಲಾಸಗಿರಿವಾಸೀ ಚ ಹಿಮವದ್ಗಿರಿಸಂಶ್ರಯಃ ।\\ಕೂಲಹಾರೀ ಕೂಲಕರ್ತಾ ಬಹುವಿದ್ಯೋ ಬಹುಪ್ರದಃ \num{॥೭೯ ॥}
\end{verse}

\begin{verse}
ವಣಿಜೋ ವರ್ಧಕೀ ವೃಕ್ಷೋ ಬಕುಲಶ್ಚಂದನಶ್ಛದಃ ।\\ಸಾರಗ್ರೀವೋ ಮಹಾಜತ್ರುರಲೋಲಶ್ಚ ಮಹೌಷಧಃ \num{॥ ೮೦ ॥}
\end{verse}

\begin{verse}
ಸಿದ್ಧಾರ್ಥಕಾರೀ ಸಿದ್ಧಾರ್ಥಶ್ಛಂದೋವ್ಯಾಕರಣೋತ್ತರಃ ।\\ಸಿಂಹನಾದಃ ಸಿಂಹದಂಷ್ಟ್ರಃ ಸಿಂಹಗಃ ಸಿಂಹವಾಹನಃ \num{॥ ೮೧ ॥}
\end{verse}

\begin{verse}
ಪ್ರಭಾವಾತ್ಮಾ ಜಗತ್ಕಾಲಸ್ಥಾಲೋ ಲೋಕಹಿತಸ್ತರುಃ ।\\ಸಾರಂಗೋ ನವಚಕ್ರಾಂಗಃ ಕೇತುಮಾಲೀ ಸಭಾವನಃ \num{॥ ೮೨ ॥}
\end{verse}

\begin{verse}
ಭೂತಾಲಯೋ ಭೂತಪತಿರಹೋರಾತ್ರಮನಿಂದಿತಃ \num{॥ ೮೩ ॥}
\end{verse}

\begin{verse}
ವಾಹಿತಾ ಸರ್ವಭೂತಾನಾಂ ನಿಲಯಶ್ಚ ವಿಭುರ್ಭವಃ ।\\ಅಮೋಘಃ ಸಂಯತೋ ಹ್ಯಶ್ವೋ ಭೋಜನಃ ಪ್ರಾಣಧಾರಣಃ \num{॥ ೮೪ ॥}
\end{verse}

\begin{verse}
ಧೃತಿಮಾನ್ ಮತಿಮಾನ್ ದಕ್ಷಃ ಸತ್ಕೃತಶ್ಚ ಯುಗಾಧಿಪಃ ।\\ಗೋಪಾಲಿರ್ಗೋಪತಿರ್ಗ್ರಾಮೋ ಗೋಚರ್ಮವಸನೋ ಹರಿಃ \num{॥ ೮೫ ॥}
\end{verse}

\begin{verse}
ಹಿರಣ್ಯಬಾಹುಶ್ಚ ತಥಾ ಗುಹಾಪಾಲಃ ಪ್ರವೇಶಿನಾಮ್ ।\\ಪ್ರಕೃಷ್ಟಾರಿರ್ಮಹಾಹರ್ಷೋ ಜಿತಕಾಮೋ ಜಿತೇಂದ್ರಿಯಃ \num{॥ ೮೬ ॥}
\end{verse}

\begin{verse}
ಗಾಂಧಾರಶ್ಚ ಸುವಾಸಶ್ಚ ತಪಸ್ಸಕ್ತೋ ರತಿರ್ನರಃ ।\\ಮಹಾಗೀತೋ ಮಹಾನೃತ್ಯೋ ಹ್ಯಪ್ಸರೋಗಣಸೇವಿತಃ \num{॥ ೮೭ ॥}
\end{verse}

\begin{verse}
ಮಹಾಕೇತುರ್ಮಹಾಧಾತುರ್ನೈಕಸಾನುಚರಶ್ಚಲಃ ।\\ಆವೇದನೀಯ ಆದೇಶಃ ಸರ್ವಗಂಧಸುಖಾವಹಃ \num{॥ ೮೮ ॥}
\end{verse}

\begin{verse}
ತೋರಣಸ್ತಾರಣೋ ವಾತಃ ಪರಿಧೀ ಪತಿಖೇಚರಃ ।\\ಸಂಯೋಗೋ ವರ್ಧನೋ ವೃದ್ಧೋ ಅತಿವೃದ್ಧೋ ಗುಣಾಧಿಕಃ \num{॥ ೮೯ ॥}
\end{verse}

\begin{verse}
ನಿತ್ಯ ಆತ್ಮ ಸಹಾಯಶ್ಚ ದೇವಾಸುರಪತಿಃ ಪತಿಃ ।\\ಯುಕ್ತಶ್ಚ ಯುಕ್ತಬಾಹುಶ್ಚ ದೇವೋ ದಿವಿಸುಪರ್ವಣಃ \num{॥ ೯೦ ॥}
\end{verse}

\begin{verse}
ಆಷಾಢಶ್ಚ ಸುಷಾಢಶ್ಚ ಧ್ರುವೋಽಥ ಹರಿಣೋ ಹರಃ ।\\ವಪುರಾವರ್ತಮಾನೇಭ್ಯೋ ವಸುಶ್ರೇಷ್ಠೋ ಮಹಾಪಥಃ \num{॥ ೯೧ ॥}
\end{verse}

\begin{verse}
ಶಿರೋಹಾರೀ ವಿಮರ್ಶಶ್ಚ ಸರ್ವಲಕ್ಷಣಲಕ್ಷಿತಃ ।\\ಅಕ್ಷಶ್ಚ ರಥಯೋಗೀ ಚ ಸರ್ವಯೋಗೀ ಮಹಾಬಲಃ \num{॥ ೯೨ ॥}
\end{verse}

\begin{verse}
ಸಮಾಮ್ನಾ ಯೋಽಸಮಾಮ್ನಾಯಸ್ತೀರ್ಥದೇವೋ ಮಹಾರಥಃ ।\\ನಿರ್ಜೀವೋ ಜೀವನೋ ಮಂತ್ರಃ ಶುಭಾಕ್ಷೋ ಬಹುಕರ್ಕಶಃ \num{॥ ೯೩ ॥}
\end{verse}

\begin{verse}
ರತ್ನಪ್ರಭೂತೋ ರತ್ನಾಂಗೋ ಮಹಾರ್ಣವನಿಪಾನವಿತ್ ।\\ಮೂಲಂ ವಿಶಾಲೋ ಹ್ಯಮೃತೋ ವ್ಯಕ್ತಾವ್ಯಕ್ತಸ್ತಪೋನಿಧಿಃ \num{॥ ೯೪ ॥}
\end{verse}

\begin{verse}
ಆರೋಹಣೋಽಧಿರೋಹಶ್ಚ ಶೀಲಧಾರೀ ಮಹಾಯಶಾಃ ।\\ಸೇನಾಕಲ್ಪೋ ಮಹಾಕಲ್ಪೋ ಯೋಗೋ ಯುಗಕರೋ ಹರಿಃ \num{॥ ೯೫ ॥}
\end{verse}

\begin{verse}
ಯುಗರೂಪೋ ಮಹಾರೂಪೋ ಮಹಾನಾಗಹನೋ ವಧಃ ।\\ನ್ಯಾಯನಿರ್ವಪಣಃ ಪಾದಃ ಪಂಡಿತೋ ಹ್ಯಚಲೋಪಮಃ \num{॥ ೯೬ ॥}
\end{verse}

\begin{verse}
ಬಹುಮಾಲೋ ಮಹಾಮಾಲಃ ಶಶೀ ಹರಸುಲೋಚನಃ ।\\ವಿಸ್ತಾರೋ ಲವಣಃ ಕೂಪಸ್ತ್ರಿಯುಗಃ ಸಫಲೋದಯಃ \num{॥ ೯೭ ॥}
\end{verse}

\begin{verse}
ತ್ರಿಲೋಚನೋ ವಿಷಣ್ಣಾಂಗೋ ಮಣಿವಿದ್ಧೋ ಜಟಾಧರಃ ।\\ಬಿಂದುರ್ವಿಸರ್ಗಃ ಸುಮುಖಃ ಶರಃ ಸರ್ವಾಯುಧಃ ಸಹಃ \num{॥ ೯೮ ॥}
\end{verse}

\begin{verse}
ನಿವೇದನಃ ಸುಖಾಜಾತಃ ಸುಗಂಧಾರೋ ಮಹಾಧನುಃ ।\\ಗಂಧಪಾಲೀ ಚ ಭಗವಾನುತ್ಥಾನಃ ಸರ್ವಕರ್ಮಣಾಮ್ \num{॥ ೯೯ ॥}
\end{verse}

\begin{verse}
ಮಂಥಾನೋ ಬಹುಲೋ ವಾಯುಃ ಸಕಲಃ ಸರ್ವಲೋಚನಃ ।\\ತಲಸ್ತಾಲಃ ಕರಸ್ಥಾಲೀ ಊರ್ಧ್ವಸಂಹನನೋ ಮಹಾನ್ \num{॥ ೧೦೦ ॥}
\end{verse}

\begin{verse}
ಛತ್ರಂ ಸುಚ್ಛತ್ರೋ ವಿಖ್ಯಾತೋ ಲೋಕಃ ಸರ್ವಾಶ್ರಯಃ ಕ್ರಮಃ ।\\ಮುಂಡೋ ವಿರೂಪೋ ವಿಕೃತೋ ದಂಡೀ ಕುಂಡೀ ವಿಕುರ್ವಣಃ \num{॥ ೧೦೧ ॥}
\end{verse}

\begin{verse}
ಹರ್ಯಕ್ಷಃ ಕಕುಭೋ ವಜ್ರೀ ಶತಜಿಹ್ವಃ ಸಹಸ್ರಪಾತ್ ।\\ಸಹಸ್ರಮೂರ್ಧಾ ದೇವೇಂದ್ರಃ ಸರ್ವದೇವಮಯೋ ಗುರುಃ \num{॥ ೧೦೨ ॥}
\end{verse}

\begin{verse}
ಸಹಸ್ರಬಾಹುಃ ಸರ್ವಾಂಗಃ ಶರಣ್ಯಃ ಸರ್ವಲೋಕಕೃತ್ ।\\ಪವಿತ್ರಂ ತ್ರಿಕಕುನ್ಮಂತ್ರಃ ಕನಿಷ್ಠಃ ಕೃಷ್ಣಪಿಂಗಲಃ \num{॥ ೧೦೩ ॥}
\end{verse}

\begin{verse}
ಬ್ರಹ್ಮದಂಡವಿನಿರ್ಮಾತಾ ಶತಘ್ನೀಪಾಶಶಕ್ತಿಮಾನ್ ।\\ಪದ್ಮಗರ್ಭೋ ಮಹಾಗರ್ಭೋ ಬ್ರಹ್ಮಗರ್ಭೋ ಜಲೋದ್ಭವಃ \num{॥ ೧೦೪ ॥}
\end{verse}

\begin{verse}
ಗಭಸ್ತಿರ್ಬ್ರಹ್ಮಕೃದ್ಬ್ರಹ್ಮೀ ಬ್ರಹ್ಮವಿದ್ಬ್ರಾಹ್ಮಣೋ ಗತಿಃ ।\\ಅನಂತರೂಪೋ ನೈಕಾತ್ಮಾ ತಿಗ್ಮತೇಜಾಃ ಸ್ವಯಂಭುವಃ \num{॥ ೧೦೫ ॥}
\end{verse}

\begin{verse}
ಊರ್ಧ್ವಗಾತ್ಮಾ ಪಶುಪತಿರ್ವಾತರಂಹಾ ಮನೋಜವಃ ।\\ಚಂದನೀ ಪದ್ಮನಾಲಾಗ್ರಃ ಸುರಭ್ಯುತ್ತರಣೋ ನರಃ \num{॥ ೧೦೬ ॥}
\end{verse}

\begin{verse}
ಕರ್ಣಿಕಾರಮಹಾಸ್ರಗ್ವೀ ನೀಲಮೌಲಿಃ ಪಿನಾಕಧೃತ್ ।\\ಉಮಾಪತಿರುಮಾಕಾಂತೋ ಜಾಹ್ನವೀಧೃದುಮಾಧವಃ \num{॥ ೧೦೭ ॥}
\end{verse}

\begin{verse}
ವರೋ ವರಾಹೋ ವರದೋ ವರೇಣ್ಯಃ ಸುಮಹಾಸ್ವನಃ ।\\ಮಹಾಪ್ರಸಾದೋ ದಮನಃ ಶತ್ರುಹಾ ಶ್ವೇತಪಿಂಗಲಃ \num{॥ ೧೦೮ ॥}
\end{verse}

\begin{verse}
ಪೀತಾತ್ಮಾ ಪರಮಾತ್ಮಾ ಚ ಪ್ರಯತಾತ್ಮಾಪ್ರಧಾನಧೃತ್ ।\\ಸರ್ವಪಾರ್ಶ್ವಮುಖಸ್ತ್ರ್ಯಕ್ಷೋ ಧರ್ಮಸಾಧಾರಣೋ ವರಃ \num{॥ ೧೦೯ ॥}
\end{verse}

\begin{verse}
ಚರಾಚರಾತ್ಮಾ ಸೂಕ್ಷ್ಮಾತ್ಮಾ ಅಮೃತೋ ಗೋವೃಷೇಶ್ವರಃ ।\\ಸಾಧ್ಯರ್ಷಿರ್ವಸುರಾದಿತ್ಯೋ ವಿವಸ್ವಾನ್ ಸವಿತಾಽಮೃತಃ \num{॥ ೧೧೦ ॥}
\end{verse}

\begin{verse}
ವ್ಯಾಸಃ ಸರ್ಗಃ ಸುಸಂಕ್ಷೇಪೋ ವಿಸ್ತರಃ ಪರ್ಯಯೋ ನರಃ ।\\ಪುತುಃ ಸಂವತ್ಸರೋ ಮಾಸಃ ಪಕ್ಷಃ ಸಂಖ್ಯಾಸಮಾಪನಃ \num{॥ ೧೧೧ ॥}
\end{verse}

\begin{verse}
ಕಲಾ ಕಾಷ್ಠಾ ಲವಾ ಮಾತ್ರ ಮುಹೂರ್ತಾಹಃಕ್ಷಪಾಃ ಕ್ಷಣಾಃ ।\\ವಿಶ್ವಕ್ಷೇತ್ರಂ ಪ್ರಜಾಬೀಜಂ ಲಿಂಗಮಾದ್ಯಸ್ತು ನಿರ್ಗಮಃ \num{॥ ೧೧೨ ॥}
\end{verse}

\begin{verse}
ಸದಸದ್ವ್ಯಕ್ತಮವ್ಯಕ್ತಂ ಪಿತಾ ಮಾತಾ ಪಿತಾಮಹಃ ।\\ಸ್ವರ್ಗದ್ವಾರಂ ಪ್ರಜಾದ್ವಾರಂ ಮೋಕ್ಷದ್ವಾರಂ ತ್ರಿವಿಷ್ಟಪಮ್ \num{॥ ೧೧೩ ॥}
\end{verse}

\begin{verse}
ನಿರ್ವಾಣಂ ಹ್ಲಾದನಶ್ಚೈವ ಬ್ರಹ್ಮಲೋಕಃ ಪರಾ ಗತಿಃ ।\\ದೇವಾಸುರವಿನಿರ್ಮಾತಾ ದೇವಾಸುರಪರಾಯಣಃ \num{॥ ೧೧೪ ॥}
\end{verse}

\begin{verse}
ದೇವಾಸುರಗುರುರ್ದೇವೋ ದೇವಾಸುರನಮಸ್ಕೃತಃ ।\\ದೇವಾಸುರಮಹಾಮಾತ್ರೋ ದೇವಾಸುರಗಣಾಶ್ರಯಃ \num{॥ ೧೧೫ ॥}
\end{verse}

\begin{verse}
ದೇವಾಸುರಗಣಾಧ್ಯಕ್ಷೋ ದೇವಾಸುರಗಣಾಗ್ರಣೀಃ ।\\ದೇವಾತಿದೇವೋ ದೇವರ್ಷಿರ್ದೇವಾಸುರವರಪ್ರದಃ \num{॥ ೧೧೬ ॥}
\end{verse}

\begin{verse}
ದೇವಾಸುರೇಶ್ವರೋ ವಿಶ್ವೋ ದೇವಾಸುರಮಹೇಶ್ವರಃ ।\\ಸರ್ವದೇವಮಯೋಽಚಿಂತ್ಯೋ ದೇವತಾತ್ಮಾಽತ್ಮಸಂಭವಃ \num{॥ ೧೧೭ ॥}
\end{verse}

\begin{verse}
ಉದ್ಭಿತ್ ತ್ರಿವಿಕ್ರಮೋ ವೈದ್ಯೋ ವಿರಜೋ ನೀರಜೋಽಮರಃ ।\\ಈಡ್ಯೋ ಹಸ್ತೀಶ್ವರೋ ವ್ಯಾಘ್ರೋ ದೇವಸಿಂಹೋ ನರರ್ಷಭಃ \num{॥ ೧೧೮ ॥}
\end{verse}

\begin{verse}
ವಿಬುಧೋಽಗ್ರವರಃ ಸೂಕ್ಷ್ಮಃ ಸರ್ವದೇವಸ್ತಪೋಮಯಃ ।\\ಸುಯುಕ್ತಃ ಶೋಭನೋ ವಜ್ರೀ ಪ್ರಾಸಾನಾಂ ಪ್ರಭವೋಽವ್ಯಯಃ \num{॥ ೧೧೯ ॥}
\end{verse}

\begin{verse}
ಗುಹಃ ಕಾಂತೋ ನಿಜಃ ಸರ್ಗಃ ಪವಿತ್ರಂ ಸರ್ವಪಾವನಃ ।\\ಶೃಂಗೀ ಶೃಂಗಪ್ರಿಯೋ ಬಭ್ರೂ ರಾಜರಾಜೋ ನಿರಾಮಯಃ \num{॥ ೧೨೦ ॥}
\end{verse}

\begin{verse}
ಅಭಿರಾಮಃ ಸುರಗಣೋ ವಿರಾಮಃ ಸರ್ವಸಾಧನಃ ।\\ಲಲಾಟಾಕ್ಷೋ ವಿಶ್ವದೇವೋ ಹರಿಣೋ ಬ್ರಹ್ಮವರ್ಚಸಃ \num{॥ ೧೨೧ ॥}
\end{verse}

\begin{verse}
ಸ್ಥಾವರಾಣಾಂ ಪತಿಶ್ಚೈವ ನಿಯಮೇಂದ್ರಿಯವರ್ಧನಃ ।\\ಸಿದ್ಧಾರ್ಥಃ ಸಿದ್ಧಭೂತಾರ್ಥೋಽಚಿಂತ್ಯಃ ಸತ್ಯವ್ರತಃ ಶುಚಿಃ \num{॥ ೧೨೨ ॥}
\end{verse}

\begin{verse}
ವ್ರತಾಧಿಪಃ ಪರಂ ಬ್ರಹ್ಮ ಭಕ್ತಾನಾಂ ಪರಮಾ ಗತಿಃ ।\\ವಿಮುಕ್ತೋ ಮುಕ್ತತೇಜಾಶ್ಚ ಶ್ರೀಮಾನ್​ಶ್ರೀವರ್ಧನೋ ಜಗತ್ \num{॥ ೧೨೩ ॥}
\end{verse}

