
\chapter{ಶ್ರೀಶಿವಾಷ್ಟೋತ್ತರಶತನಾಮಾವಲಿಃ}

\begin{center}
ಅರ್ಚನ ಪ್ರಕಾರವು ಸಹಸ್ರನಾಮಾವಲಿಯ ಪ್ರಾರಂಭದಲ್ಲಿ ಕೊಟ್ಟಿರುವಂತೆಯೇ.
\end{center}

ಶಿವಾಯ ನಮಃ

ಮಹೇಶ್ವರಾಯ ನಮಃ

ಶಂಭವೇ ನಮಃ

ಪಿನಾಕಿನೇ ನಮಃ

ಶಶಿಶೇಖರಾಯ ನಮಃ

ವಾಮದೇವಾಯ ನಮಃ

ವಿರೂಪಾಕ್ಷಾಯ ನಮಃ

ಕಪರ್ದಿನೇ ನಮಃ

ನೀಲಲೋಹಿತಾಯ ನಮಃ

ಶಂಕರಾಯ ನಮಃ \num{೧೦}

ಶೂಲಪಾಣಯೇ ನಮಃ

ಖಟ್ವಾಂಗಿನೇ ನಮಃ

ವಿಷ್ಣುವಲ್ಲಭಾಯ ನಮಃ

ಶಿಪಿವಿಷ್ಟಾಯ ನಮಃ

ಅಂಬಿಕಾನಾಥಾಯ ನಮಃ

ಶ್ರೀಕಂಠಾಯ ನಮಃ

ಭಕ್ತವತ್ಸಲಾಯ ನಮಃ

ಭವಾಯ ನಮಃ

ಶರ್ವಾಯ ನಮಃ

ತ್ರಿಲೋಕೇಶಾಯ ನಮಃ \num{೨೦}

ಶಿತಿಕಂಠಾಯ ನಮಃ

ಶಿವಾಪ್ರಿಯಾಯ ನಮಃ

ಉಗ್ರಾಯ ನಮಃ

ಕಪಾಲಿನೇ ನಮಃ

ಕಾಮಾರಯೇ ನಮಃ

ಅಂಧಕಾಸುರಸೂದನಾಯ ನಮಃ

ಗಂಗಾಧರಾಯ ನಮಃ

ಲಲಾಟಾಕ್ಷಾಯ ನಮಃ

ಕಾಲಕಾಲಾಯ ನಮಃ

ಕೃಪಾನಿಧಯೇ ನಮಃ \num{೩೦}

ಭೀಮಾಯ ನಮಃ

ಪರಶುಹಸ್ತಾಯ ನಮಃ

ಮೃಗಪಾಣಯೇ ನಮಃ

ಜಟಾಧರಾಯ ನಮಃ

ಕೈಲಾಸವಾಸಿನೇ ನಮಃ

ಕವಚಿನೇ ನಮಃ

ಕಠೋರಾಯ ನಮಃ

ತ್ರಿಪುರಾಂತಕಾಯ ನಮಃ

ವೃಷಾಂಕಾಯ ನಮಃ

ವೃಷಭಾರೂಢಾಯ ನಮಃ \num{೪೦}

ಭಸ್ಮೋದ್ಧೂಲಿತವಿಗ್ರಹಾಯ ನಮಃ

ಸಾಮಪ್ರಿಯಾಯ ನಮಃ

ಸ್ವರಮಯಾಯ ನಮಃ

ತ್ರಯೀಮೂರ್ತಯೇ ನಮಃ

ಅನೀಶ್ವರಾಯ ನಮಃ

ಸರ್ವಜ್ಞಾಯ ನಮಃ

ಪರಮಾತ್ಮನೇ ನಮಃ

ಸೋಮಸೂರ್ಯಾಗ್ನಿಲೋಚನಾಯ ನಮಃ

ಹವಿಷೇ ನಮಃ

ಯಜ್ಞಮಯಾಯ ನಮಃ \num{೫೦}

ಸೋಮಾಯ ನಮಃ

ಪಂಚವಕ್ತ್ರಾಯ ನಮಃ

ಸದಾಶಿವಾಯ ನಮಃ

ವಿಶ್ವೇಶ್ವರಾಯ ನಮಃ

ವೀರಭದ್ರಾಯ ನಮಃ

ಗಣನಾಥಾಯ ನಮಃ

ಪ್ರಜಾಪತಯೇ ನಮಃ

ಹಿರಣ್ಯರೇತಸೇ ನಮಃ

ದುರ್ಧರ್ಷಾಯ ನಮಃ

ಗಿರೀಶಾಯ ನಮಃ \num{೬೦}

ಗಿರಿಶಾಯ ನಮಃ

ಅನಘಾಯ ನಮಃ

ಭುಜಂಗಭೂಷಣಾಯ ನಮಃ

ಭರ್ಗಾಯ ನಮಃ

ಗಿರಿಧನ್ವನೇ ನಮಃ

ಗಿರಿಪ್ರಿಯಾಯ ನಮಃ

ಕೃತ್ತಿವಾಸಸೇ ನಮಃ

ಪುರಾರಾತಯೇ ನಮಃ

ಭಗವತೇ ನಮಃ

ಪ್ರಮಥಾಧಿಪಾಯ ನಮಃ \num{೭೦}

ಮೃತ್ಯುಂಜಯಾಯ ನಮಃ

ಸೂಕ್ಷ್ಮತನವೇ ನಮಃ

ಜಗದ್ವ್ಯಾಪಿನೇ ನಮಃ

ಜಗದ್ಗುರವೇ ನಮಃ

ವ್ಯೋಮಕೇಶಾಯ ನಮಃ

ಮಹಾಸೇನಜನಕಾಯ ನಮಃ

ಚಾರುವಿಕ್ರಮಾಯ ನಮಃ

ರುದ್ರಾಯ ನಮಃ

ಭೂತಪತಯೇ ನಮಃ

ಸ್ಥಾಣವೇ ನಮಃ \num{೮೦}

ಅಹಿರ್ಬುಧ್ನ್ಯಾಯ ನಮಃ

ದಿಗಂಬರಾಯ ನಮಃ

ಅಷ್ಟಮೂರ್ತಯೇ ನಮಃ

ಅನೇಕಾತ್ಮನೇ ನಮಃ

ಸಾತ್ತ್ವಿಕಾಯ ನಮಃ

ಶುದ್ಧವಿಗ್ರಹಾಯ ನಮಃ

ಶಾಶ್ವತಾಯ ನಮಃ

ಖಂಡಪರಶವೇ ನಮಃ

ಅಜಾಯ ನಮಃ

ಪಾಶವಿಮೋಚನಾಯ ನಮಃ \num{೯೦}

ಮೃಡಾಯ ನಮಃ

ಪಶುಪತಯೇ ನಮಃ

ದೇವಾಯ ನಮಃ

ಮಹಾದೇವಾಯ ನಮಃ

ಅವ್ಯಯಾಯ ನಮಃ

ಹರಯೇ ನಮಃ

ಪೂಷದಂತಭಿದೇ ನಮಃ

ಅವ್ಯಗ್ರಾಯ ನಮಃ

ದಕ್ಷಾಧ್ವರಹರಾಯ ನಮಃ

ಹರಾಯ ನಮಃ \num{೧೦೦}

ಭಗನೇತ್ರಭಿದೇ ನಮಃ

ಅವ್ಯಕ್ತಾಯ ನಮಃ

ಸಹಸ್ರಾಕ್ಷಾಯ ನಮಃ

ಸಹಸ್ರಪದೇ ನಮಃ

ಅಪವರ್ಗಪ್ರದಾಯ ನಮಃ

ಅನಂತಾಯ ನಮಃ

ತಾರಕಾಯ ನಮಃ

ಪರಮೇಶ್ವರಾಯ ನಮಃ \num{೧೦೮}

\begin{center}
ಶ್ರೀಶಿವಾಷ್ಟೋತ್ತರಶತನಾಮಾವಲಿಃ ಸಮಾಪ್ತಾಃ
\end{center}

