
\chapter{ಫಲಶ್ರುತಿಃ}

\begin{verse}
ಯಥಾಪ್ರಧಾನಂ ಭಗವಾನಿತಿ ಭಕ್ತ್ಯಾ ಸ್ತುತೋ ಮಯಾ ।\\ಯನ್ನ ಬ್ರಹ್ಮಾದಯೋ ದೇವಾ ವಿದುಸ್ತತ್ವೇನ ನರ್ಷಯಃ \num{॥ ೧ ॥}
\end{verse}

\begin{verse}
ಸ್ತೋತವ್ಯಮರ್ಚ್ಯಂ ವಂದ್ಯಂ ಚ ಕಃ ಸ್ತೋಷ್ಯತಿ ಜಗತ್ಪತಿಮ್ ।\\ಭಕ್ತ್ಯಾತ್ವೇವಂ ಪುರಸ್ಕೃತ್ಯ ಮಯಾ ಯಜ್ಞಪತಿರ್ವಿಭುಃ ।\\ತತೋಽಭ್ಯನುಜ್ಞಾಂ ಸಂಪ್ರಾಪ್ಯ ಸ್ತುತೋ ಮತಿಮತಾಂ ವರಃ \num{॥ ೨ ॥}
\end{verse}

\begin{verse}
ಶಿವಮೇಭಿಃ ಸ್ತುವನ್ ದೇವಂ ನಾಮಭಿಃ ಪುಷ್ಟಿವರ್ಧನೈಃ ।\\ನಿತ್ಯಯುಕ್ತಃ ಶುಚಿರ್ಭಕ್ತಃ ಪ್ರಾಪ್ನೋತ್ಯಾತ್ಮಾನಮಾತ್ಮನಾ \num{॥ ೩ ॥}
\end{verse}

\begin{verse}
ಏತದ್ಧಿ ಪರಮಂ ಬ್ರಹ್ಮ ಪರಂ ಬ್ರಹ್ಮಾಧಿಗಚ್ಛತಿ ।\\ಪುಷಯಶ್ಚೈವ ದೇವಾಶ್ಚ ಸ್ತುವಂತ್ಯೇತೇನ ತತ್ಪರಮ್ \num{॥ ೪ ॥}
\end{verse}

\begin{verse}
ಸ್ತೂಯಮಾನೋ ಮಹಾದೇವಸ್ತುಷ್ಯತೇ ನಿಯತಾತ್ಮಭಿಃ ।\\ಭಕ್ತಾನುಕಂಪೀ ಭಗವಾನಾತ್ಮಸಂಸ್ಥಾಕರೋ ವಿಭುಃ \num{॥ ೫ ॥}
\end{verse}

\begin{verse}
ತಥೈವ ಚ ಮನುಷ್ಯೇಷು ಯೇ ಮನುಷ್ಯಾಃ ಪ್ರಧಾನತಃ ।\\ಆಸ್ತಿಕಾಃ ಶ್ರದ್ದಧಾನಾಶ್ಚ ಬಹುಭಿರ್ಜನ್ಮಭಿಃ ಸ್ತವೈಃ \num{॥ ೬ ॥}
\end{verse}

\begin{verse}
ಭಕ್ತ್ಯಾ ಹ್ಯನನ್ಯಮೀಶಾನಂ ಪರಂ ದೇವಂ ಸನಾತನಮ್ ।\\ಕರ್ಮಣಾ ಮನಸಾ ವಾಚಾ ಭಾವೇನಾಮಿತತೇಜಸಃ \num{॥ ೭ ॥}
\end{verse}

\begin{verse}
ಶಯಾನಾ ಜಾಗ್ರಮಾಣಾಶ್ಚ ವ್ರಜನ್ನುಪವಿಶಂಸ್ತಥಾ ।\\ಉನ್ಮಿಷನ್ನಿಮಿಷಂಶ್ಚೈವ ಚಿಂತಯಂತಃ ಪುನಃ ಪುನಃ \num{॥ ೮ ॥}
\end{verse}

\begin{verse}
ಶೃಣ್ವಂತಃ ಶ್ರಾವಯಂತಶ್ಚ ಕಥಯಂತಶ್ಚ ತೇ ಭವಮ್ ।\\ಸ್ತುವಂತಃ ಸ್ತೂಯಮಾನಾಶ್ಚ ತುಷ್ಯಂತಿ ಚ ರಮಂತಿ ಚ \num{॥ ೯ ॥}
\end{verse}

\begin{verse}
ಜನ್ಮಕೋಟಿಸಹಸ್ರೇಷು ನಾನಾಸಂಸಾರಯೋನಿಷು ।\\ಜಂತೋರ್ವಿಗತಪಾಪಸ್ಯ ಭವೇ ಭಕ್ತಿಃ ಪ್ರಜಾಯತೇ \num{॥ ೧೦ ॥}
\end{verse}

\begin{verse}
ಉತ್ಪನ್ನಾ ಚ ಭವೇ ಭಕ್ತಿರನನ್ಯಾ ಸರ್ವಭಾವತಃ ।\\ಭಾವಿನಃ ಕಾರಣೇ ಚಾಸ್ಯ ಸರ್ವಯುಕ್ತಸ್ಯ ಸರ್ವಥಾ \num{॥ ೧೧ ॥}
\end{verse}

\begin{verse}
ಏತದ್ದೇವೇಷು ದುಷ್ಪ್ರಾಪಂ ಮನುಷ್ಯೇಷು ನ ಲಭ್ಯತೇ ।\\ನಿರ್ವಿಘ್ನಾ ನಿಶ್ಚಲಾ ರುದ್ರೇ ಭಕ್ತಿರವ್ಯಭಿಚಾರಿಣೀ \num{॥ ೧೨ ॥}
\end{verse}

\begin{verse}
ತಸ್ಯೈವ ಚ ಪ್ರಸಾದೇನ ಭಕ್ತಿರುತ್ಪದ್ಯತೇ ನೃಣಾಮ್ ।\\ಯೇನ ಯಾಂತಿ ಪರಾಂ ಸಿದ್ಧಿಂ ತದ್ಭಾವಗತತೇಜಸಃ \num{॥ ೧೩ ॥}
\end{verse}

\begin{verse}
ಯೇ ಸರ್ವಭಾವಾನುಗತಾಃ ಪ್ರಪದ್ಯಂತೇ ಮಹೇಶ್ವರಮ್ ।\\ಪ್ರಪನ್ನವತ್ಸಲೋ ದೇವಃ ಸಂಸಾರಾತ್ತಾನ್ ಸಮುದ್ಧರೇತ್ \num{॥ ೧೪ ॥}
\end{verse}

\begin{verse}
ಏವಮನ್ಯೇ ವಿಕುರ್ವಂತಿ ದೇವಾಃ ಸಂಸಾರಮೋಚನಮ್ ।\\ಮನುಷ್ಯಾಣಾಮೃತೇ ದೇವಂ ನಾನ್ಯಾ ಶಕ್ತಿಸ್ತಪೋಬಲಮ್ \num{॥ ೧೫ ॥}
\end{verse}

\begin{verse}
ಇತಿ ತೇನೇಂದ್ರಕಲ್ಪೇನ ಭಗವಾನ್ ಸದಸತ್ಪತಿಃ ।\\ಕೃತ್ತಿವಾಸಾಃ ಸ್ತುತಃ ಕೃಷ್ಣ ತಂಡಿನಾ ಶುಭಬುದ್ಧಿನಾ \num{॥ ೧೬ ॥}
\end{verse}

\begin{verse}
ಸ್ತವಮೇತಂ ಭಗವತೋ ಬ್ರಹ್ಮಾ ಸ್ವಯಮಧಾರಯತ್ ।\\ಗೀಯತೇ ಚ ಸ ಬುದ್ಧ್ಯೇತ ಬ್ರಹ್ಮಾ ಶಂಕರಸಂನಿಧೌ \num{॥ ೧೭ ॥}
\end{verse}

\begin{verse}
ಇದಂ ಪುಣ್ಯಂ ಪವಿತ್ರಂ ಚ ಸರ್ವದಾ ಪಾಪನಾಶನಮ್ ।\\ಯೋಗದಂ ಮೋಕ್ಷದಂ ಚೈವ ಸ್ವರ್ಗದಂ ತೋಷದಂ ತಥಾ \num{॥ ೧೮ ॥}
\end{verse}

\begin{verse}
ಏವಮೇತತ್ಪಠಂತೇ ಯ ಏಕಭಕ್ತ್ಯಾ ತು ಶಂಕರಮ್ ।\\ಯಾ ಗತಿಃ ಸಾಂಖ್ಯಯೋಗಾನಾಂ ವ್ರಜಂತ್ಯೇತಾಂ ಗತಿಂ ತದಾ \num{॥ ೧೯ ॥}
\end{verse}

\begin{verse}
ಸ್ತವಮೇತಂ ಪ್ರಯತ್ನೇನ ಸದಾ ರುದ್ರಸ್ಯ ಸಂನಿಧೌ ।\\ಅಬ್ದಮೇಕಂ ಚರೇದ್ಭಕ್ತಃ ಪ್ರಾಪ್ನುಯಾದೀಪ್ಸಿತಂ ಫಲಮ್ \num{॥ ೨೦ ॥}
\end{verse}

\begin{verse}
ಏತದ್ರಹಸ್ಯಂ ಪರಮಂ ಬ್ರಹ್ಮಣೋ ಹೃದಿ ಸಂಸ್ಥಿತಮ್ ।\\ಬ್ರಹ್ಮಾ ಪ್ರೋವಾಚ ಶಕ್ರಾಯ ಶಕ್ರಃ ಪ್ರೋವಾಚ ಮೃತ್ಯವೇ \num{॥ ೨೧ ॥}
\end{verse}

\begin{verse}
ಮೃತ್ಯುಃ ಪ್ರೋವಾಚ ರುದ್ರೇಭ್ಯೋ ರುದ್ರೇಭ್ಯಸ್ತಂಡಿಮಾಗಮತ್ ।\\ಮಹತಾ ತಪಸಾ ಪ್ರಾಪ್ತಸ್ತಂಡಿನಾ ಬ್ರಹ್ಮಸದ್ಮನಿ \num{॥ ೨೨ ॥}
\end{verse}

\begin{verse}
ತಂಡಿಃ ಪ್ರೋವಾಚ ಶುಕ್ರಾಯ ಗೌತಮಾಯ ಚ ಭಾರ್ಗವಃ ।\\ವೈವಸ್ವತಾಯ ಮನವೇ ಗೌತಮಃ ಪ್ರಾಹ ಮಾಧವ \num{॥ ೨೩ ॥}
\end{verse}

\begin{verse}
ನಾರಾಯಣಾಯ ಸಾಧ್ಯಾಯ ಸಮಾಧಿಷ್ಠಾಯ ಧೀಮತೇ ।\\ಯಮಾಯ ಪ್ರಾಹ ಭಗವಾನ್ ಸಾಧ್ಯೋ ನಾರಾಯಣೋಽಚ್ಯುತಃ \num{॥ ೨೪ ॥}
\end{verse}

\begin{verse}
ನಾಚಿಕೇತಾಯ ಭಗವಾನಾಹ ವೈವಸ್ವತೋ ಯಮಃ ।\\ಮಾರ್ಕಂಡೇಯಾಯ ವಾರ್ಷ್ಣೇಯ ನಾಚಿಕೇತೋಽಭ್ಯಭಾಷತ \num{॥ ೨೫ ॥}
\end{verse}

\begin{verse}
ಮಾರ್ಕಂಡೇಯಾನ್ಮಯಾ ಪ್ರಾಪ್ತೋ ನಿಯಮೇನ ಜನಾರ್ದನ ।\\ತವಾಪ್ಯಹಮಮಿತ್ರಘ್ನಸ್ತವಂ ದದ್ಯಾಂ ಹ್ಯವಿಶ್ರುತಮ್ \num{॥ ೨೬ ॥}
\end{verse}

\begin{verse}
ಸ್ವರ್ಗ್ಯಮಾರೋಗ್ಯಮಾಯುಷ್ಯಂ ಧನ್ಯಂ ವೇದೇನ ಸಂಮಿತಮ್ ।\\ನಾಸ್ಯ ವಿಘ್ನಂ ವಿಕುರ್ವಂತಿ ದಾನವಾ ಯಕ್ಷರಾಕ್ಷಸಾಃ ।\\ಪಿಶಾಚಾ ಯಾತುಧಾನಾ ವಾ ಗುಹ್ಯಕಾ ಭುಜಗಾ ಅಪಿ \num{॥ ೨೭ ॥}
\end{verse}

\begin{verse}
ಯಃ ಪಠೇತ ಶುಚಿಃ ಪಾರ್ಥ ಬ್ರಹ್ಮಚಾರೀ ಜಿತೇಂದ್ರಿಯಃ ।\\ಅಭಗ್ನಯೋಗೋ ವರ್ಷಂ ತು ಸೋಽಶ್ವಮೇಧಫಲಂ ಲಭೇತ್ \num{॥ ೨೮ ॥}
\end{verse}

