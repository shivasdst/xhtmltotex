\sethyphenation{kannada}{
ಅಂಗ-ಲು-ಬ್ಧಾಯ
ಅಂತ-ಕಾ-ಲೇಽಪಿ
ಅಂತ-ರಾ-ತ್ಮನೇ
ಅಂತ-ರ್ಹಿ-ತಾ-ತ್ಮನೇ
ಅಂಧ-ಕಾ-ಸು-ರ-ಸೂ-ದ-ನಾಯ
ಅಂಬಿ-ಕಾ-ನಾ-ಥಾಯ
ಅಂಬು-ಜಾ-ಲಾಯ
ಅಂಶವೇ
ಅಕ-ರಾಯ
ಅಕ್ಷರಂ
ಅಕ್ಷ-ರ-ಥ-ಯೋ-ಗಿನೇ
ಅಕ್ಷ-ರಾಯ
ಅಕ್ಷಶ್ಚ
ಅಗ-ಮಾಯ
ಅಗ್ನಿ-ಜ್ವಾ-ಲಾಯ
ಅಗ್ನಿ-ಜ್ವಾಲೋ
ಅಗ್ರ-ವ-ರಾಯ
ಅಚ-ಲೋ-ಪ-ಮಾಯ
ಅಚಿಂ-ತ್ಯಾಯ
ಅಜಶ್ಚ
ಅಜಾಯ
ಅಜಿ-ತಶ್ಚ
ಅಜಿ-ತಾಯ
ಅಜೈ-ಕ-ಪಾಚ್ಚ
ಅಜೈ-ಕ-ಪಾದೇ
ಅತಂ-ದ್ರಿ-ತಾಯ
ಅತಿ-ದೀ-ಪ್ತಾಯ
ಅತಿ-ದೀಪ್ತೋ
ಅತಿ-ಧೂ-ಮ್ರಾಯ
ಅತಿ-ಧೂಮ್ರೋ
ಅತಿ-ವೃ-ದ್ಧಾಯ
ಅತಿ-ವೃದ್ಧೋ
ಅತು-ಲ್ಯಾಯ
ಅತ್ರಯೇ
ಅತ್ರಿ-ರತ್ರ್ಯಾ
ಅತ್ರ್ಯಾ-ನ-ಮ-ಸ್ಕರ್ತ್ರೇ
ಅಥ-ರ್ವ-ಶೀರ್ಷಃ
ಅಥ-ರ್ವ-ಶೀ-ರ್ಷಾಯ
ಅಥವಾ
ಅದಂ-ಭಾಯ
ಅದನ್ನು
ಅದಿ-ತಯೇ
ಅದೀ-ನಾಯ
ಅದೀನೋ
ಅಧ-ನಾಯ
ಅಧ-ರ್ಷ-ಣಾಯ
ಅಧ-ರ್ಷಣೋ
ಅಧಿ-ರೋ-ಹಾಯ
ಅಧ್ಯ-ಕ್ಷರು
ಅಧ್ಯಾ-ತ್ಮಾ-ನು-ಗ-ತಾಯ
ಅಧ್ಯಾ-ತ್ಮಾ-ನು-ಗತೋ
ಅನಂ-ತ-ರೂ-ಪಾಯ
ಅನಂ-ತ-ರೂಪೋ
ಅನಂ-ತಾಯ
ಅನ-ಘಾಯ
ಅನ-ಲಾಯ
ಅನಾ-ದಿ-ನಿ-ಧ-ನ-ಸ್ಯಾಹಂ
ಅನಿಂ-ದಿ-ತಾಯ
ಅನಿ-ಮಿ-ಷಾಯ
ಅನಿ-ಲಾ-ಭಾಯ
ಅನಿ-ಲಾಯ
ಅನೀ-ತಯೇ
ಅನೀ-ಶ್ವ-ರಾಯ
ಅನು-ಕಾ-ರಿಣೇ
ಅನು-ಕಾರೀ
ಅನು-ಷ್ಟುಪ್
ಅನೇ-ಕಾ-ತ್ಮನೇ
ಅನೌ-ಷ-ಧಾಯ
ಅಪ-ರಾಯ
ಅಪ-ವ-ರ್ಗ-ಪ್ರ-ದಾಯ
ಅಪಿ
ಅಪ್ರಾಪ್ಯ
ಅಪ್ಸ-ರೋ-ಗಣ
ಅಬ-ಲಾಯ
ಅಬ್ದ-ಮೇಕಂ
ಅಭ-ಗ್ನ-ಯೋಗೋ
ಅಭಿ-ಗಮ್ಯಃ
ಅಭಿ-ಗ-ಮ್ಯಾಯ
ಅಭಿ-ರಾಮಃ
ಅಭಿ-ರಾ-ಮಾಯ
ಅಭಿ-ವಾ-ದ್ಯಾಯ
ಅಭಿ-ವಾದ್ಯೋ
ಅಮ-ರಾಯ
ಅಮ-ರೇ-ಶಾಯ
ಅಮ-ರೇಶೋ
ಅಮಿ-ತಾಯ
ಅಮಿ-ತ್ರ-ಜಿತೇ
ಅಮು-ಖಾಯ
ಅಮು-ಖ್ಯಾಯ
ಅಮೃ-ತ-ಗೋ-ವೃ-ಷೇ-ಶ್ವ-ರಾಯ
ಅಮೃ-ತಾಯ
ಅಮೃತೋ
ಅಮೋಘಃ
ಅಮೋ-ಘಾಯ
ಅಮೋ-ಘಾರ್ಥಃ
ಅಮೋ-ಘಾ-ರ್ಥಾಯ
ಅಯ-ಜ್ಞಾಯ
ಅರ್ಚನ
ಅರ್ಥ-ಕ-ರಾಯ
ಅರ್ಥಾಯ
ಅರ್ದ-ನಾಯ
ಅರ್ಯಮ್ಣೇ
ಅಲೋ-ಕಾಯ
ಅವ-ರಾಯ
ಅವ-ಶಾಯ
ಅವ್ಯ-ಕ್ತಾಯ
ಅವ್ಯ-ಗ್ರಾಯ
ಅವ್ಯ-ಯಾಯ
ಅಶ-ನಿನೇ
ಅಶನೀ
ಅಶ್ವ-ತ್ಥಾಯ
ಅಶ್ವಾಯ
ಅಷ್ಟ-ಮೂ-ರ್ತಯೇ
ಅಷ್ಟೋ-ತ್ತ-ರ-ಸ-ಹಸ್ರಂ
ಅಸತೇ
ಅಸ-ಮಾ-ಮ್ನಾ-ಯಾಯ
ಅಸು-ರೇಂ-ದ್ರಾ-ಣಾಂ
ಅಸ್ನೇ-ಹ-ನಾಯ
ಅಸ್ಯ
ಅಹ-ಶ್ಚ-ರಾಯ
ಅಹ-ಶ್ಚರೋ
ಅಹಿ-ರ್ಬು-ಧ್ನ್ಯಾಯ
ಅಹಿ-ರ್ಬು-ಧ್ನ್ಯೋ-ಽನಿ-ಲಾ-ಭಶ್ಚ
ಅಹೋ-ರಾ-ತ್ರಾಯ
ಆಕಾ-ಶ-ನಿ-ರ್ವಿ-ರೂ-ಪಶ್ಚ
ಆಕಾ-ಶ-ನಿ-ರ್ವಿ-ರೂ-ಪಾಯ
ಆಚ-ಮ-ನ-ಪ್ರಾ-ಣಾ-ಯಾ-ಮ-ಸಂ-ಕ-ಲ್ಪ-ಗಳನ್ನು
ಆತ್ಮ
ಆತ್ಮ-ನಿ-ರಾ-ಲೋ-ಕಾಯ
ಆತ್ಮ-ಸಂ-ಭ-ವಾಯ
ಆದಯೇ
ಆದಿ-ಕ-ರಾಯ
ಆದಿ-ತ್ಯ-ವ-ಸವೇ
ಆದಿ-ತ್ಯಾಯ
ಆದಿ-ರಾ-ದಿ-ಕರೋ
ಆದೇಶಃ
ಆದೇ-ಶಾಯ
ಆದ್ಯ-ನಿ-ರ್ಗ-ಮಾಯ
ಆದ್ಯಾಯ
ಆಯು-ಧಿನೇ
ಆಯು-ಶ್ಚೈವ
ಆಯುಷೇ
ಆರೋ-ಹ-ಣಾಯ
ಆರೋ-ಹ-ಣೋ-ಽಧಿ-ರೋ-ಹಶ್ಚ
ಆರ್ದ್ರ-ಚ-ರ್ಮಾಂ-ಬ-ರಾ-ವೃತಃ
ಆರ್ದ್ರ-ಚ-ರ್ಮಾಂ-ಬ-ರಾ-ವೃ-ತಾಯ
ಆಲೋ-ಲಾಯ
ಆವ-ರಣ
ಆವ-ರ್ತ-ಮಾ-ನ-ವ-ಪುಷೇ
ಆವೇ-ದ-ನೀಯ
ಆವೇ-ದ-ನೀ-ಯಾಯ
ಆಶ್ರಮ
ಆಶ್ರ-ಮಸ್ಥಃ
ಆಶ್ರ-ಮ-ಸ್ಥಾಯ
ಆಷಾ-ಢಶ್ಚ
ಆಷಾ-ಢಾಯ
ಆಸ್ತಿ-ಕಾಃ
ಇಂದ್ರಿಯಂ
ಇತಿ
ಇತಿ-ಹಾಸಃ
ಇತಿ-ಹಾ-ಸಾಯ
ಇದಂ
ಈ
ಈಡ್ಯಾಯ
ಈಡ್ಯೋ
ಈಶಾನ
ಈಶಾ-ನಾಯ
ಈಶ್ವರಃ
ಈಶ್ವ-ರಾಯ
ಉಗ್ರ-ತೇ-ಜಸೇ
ಉಗ್ರ-ತೇಜಾ
ಉಗ್ರಾಯ
ಉಗ್ರೋ
ಉತ್ಪನ್ನಾ
ಉತ್ಸಂ-ಗಶ್ಚ
ಉತ್ಸಂ-ಗಾಯ
ಉದ-ಗ್ರಾಯ
ಉದಗ್ರೋ
ಉದ್ಭಿತ್
ಉದ್ಭಿದೇ
ಉನ್ಮ-ತ್ತ-ವೇ-ಷ-ಪ್ರ-ಚ್ಛನ್ನಃ
ಉನ್ಮ-ತ್ತ-ವೇ-ಷ-ಪ್ರ-ಚ್ಛ-ನ್ನಾಯ
ಉನ್ಮಾ-ದಾಯ
ಉನ್ಮಾದೋ
ಉನ್ಮಿ-ಷ-ನ್ನಿ-ಮಿ-ಷಂ-ಶ್ಚೈವ
ಉಪ-ಕಾರಃ
ಉಪ-ಕಾ-ರಾಯ
ಉಪ-ಚಾ-ರ-ಗಳನ್ನು
ಉಪ-ದೇ-ಶ-ಕ-ರಾಯ
ಉಪ-ದೇ-ಶ-ಕ-ರೋ-ಽಕರಃ
ಉಪ-ಮ-ನ್ಯು-ರು-ವಾಚ
ಉಪ-ಯೋ-ಗಿ-ಸ-ಬೇಕು
ಉಪ-ಶಾಂ-ತಾಯ
ಉಮಾ-ಕಾಂ-ತಾಯ
ಉಮಾ-ಧ-ವಾಯ
ಉಮಾ-ಪ-ತಯೇ
ಉಮಾ-ಪ-ತಿಃ
ಉಮಾ-ಪ-ತಿ-ರು-ಮಾ-ಕಾಂತೋ
ಉವಾಚ
ಉಷಂ-ಗವೇ
ಉಷಂ-ಗುಶ್ಚ
ಉಷ್ಣೀ-ಷಿಣೇ
ಉಷ್ಣೀಷೀ
ಊರ್ಧ್ವ-ಗಾ-ತ್ಮನೇ
ಊರ್ಧ್ವ-ಗಾತ್ಮಾ
ಊರ್ಧ್ವ-ರೇ-ತಸೇ
ಊರ್ಧ್ವ-ರೇತಾ
ಊರ್ಧ್ವ-ಲಿಂಗ
ಊರ್ಧ್ವ-ಲಿಂ-ಗಾಯ
ಊರ್ಧ್ವ-ಶಾ-ಯಿನೇ
ಊರ್ಧ್ವ-ಶಾಯೀ
ಊರ್ಧ್ವ-ಸಂ-ಹ-ನ-ನಾಯ
ಊರ್ಧ್ವ-ಸಂ-ಹ-ನನೋ
ಋಷ್ಯಾ-ದಿ-ನ್ಯಾಸಃ
ಎಲೆ-ಕ್ಟ್ರಿಕ್
ಏಕ-ಭಕ್ತ್ಯಾ
ಏತ-ದ್ದೇ-ವೇಷು
ಏತದ್ಧಿ
ಏತ-ದ್ರ-ಹಸ್ಯಂ
ಏವ
ಏವ-ಮನ್ಯೇ
ಏವ-ಮೇ-ತ-ತ್ಪ-ಠಂತೇ
ಓಂ
ಕಂಚಿತ್
ಕಃ
ಕಕು-ಭಾಯ
ಕಕುಭೋ
ಕಠೋ-ರಾಯ
ಕಥ-ಯಂ-ತಶ್ಚ
ಕನಕಃ
ಕನ-ಕಾಯ
ಕನಿಷ್ಠಃ
ಕನಿ-ಷ್ಠಾಯ
ಕಪ-ರ್ದಿನೇ
ಕಪ-ರ್ದ್ಯಪಿ
ಕಪಾ-ಲ-ವತೇ
ಕಪಾ-ಲ-ವಾನ್
ಕಪಾ-ಲಿನೇ
ಕಪಿಲಃ
ಕಪಿ-ಲಾಯ
ಕಪಿಶಃ
ಕಪಿ-ಶಾಯ
ಕಮಂ-ಡ-ಲು-ಧ-ರಾಯ
ಕಮಂ-ಡ-ಲು-ಧರೋ
ಕರಃ
ಕರ-ಸ್ಥಾ-ಲಿನೇ
ಕರ-ಸ್ಥಾಲೀ
ಕರಾಯ
ಕರ್ಣಿ-ಕಾ-ರ-ಮ-ಹಾ-ಸ್ರ-ಗ್ವಿಣೇ
ಕರ್ಣಿ-ಕಾ-ರ-ಮ-ಹಾ-ಸ್ರಗ್ವೀ
ಕರ್ತಾ
ಕರ್ತ್ರೇ
ಕರ್ಮ-ಕಾ-ಲ-ವಿತ್
ಕರ್ಮ-ಕಾ-ಲ-ವಿದೇ
ಕರ್ಮಣಾ
ಕರ್ಮ-ಸ-ರ್ವ-ಬಂ-ಧ-ವಿ-ಮೋ-ಚನಃ
ಕರ್ಮ-ಸ-ರ್ವ-ಬಂ-ಧ-ವಿ-ಮೋ-ಚ-ನಾ-ಯ-ನಮಃ
ಕಲಯೇ
ಕಲಾ
ಕಲಾಭ್ಯೋ
ಕಲಿಃ
ಕಲಿ-ಯ-ಬೇಕು
ಕಲಿಶ್ಚ
ಕಲ್ಯಾ-ಣ-ಮಿ-ದ-ಮು-ತ್ತ-ಮಮ್
ಕವ-ಚಿನೇ
ಕಸ್ತಸ್ಯ
ಕಾಂಚ-ನ-ಚ್ಛ-ವಯೇ
ಕಾಂಚ-ನ-ಚ್ಛ-ವಿಃ
ಕಾಂತಾಯ
ಕಾಂತೋ
ಕಾಪಾ-ಲಿನೇ
ಕಾಪಾಲೀ
ಕಾಮ
ಕಾಮ-ನಾ-ಶಕಃ
ಕಾಮ-ನಾ-ಶ-ಕಾಯ
ಕಾಮಾಯ
ಕಾಮಾ-ರಯೇ
ಕಾಮೋ
ಕಾರ-ಣಮ್
ಕಾರ-ಣಾ-ತ್ಮಾ-ನ-ಮೀ-ಶ್ವ-ರಮ್
ಕಾರ-ಣಾ-ನಾಂ
ಕಾರಣೇ
ಕಾರ್ತೆ-್ಸ್ನ್ಯನ
ಕಾಲ-ಕ-ಟಂ-ಕಟಃ
ಕಾಲ-ಕ-ಟಂ-ಕ-ಟಾಯ
ಕಾಲ-ಕಾ-ಲಾಯ
ಕಾಲ-ಪೂ-ಜಿತಃ
ಕಾಲ-ಪೂ-ಜಿ-ತಾಯ
ಕಾಲ-ಯೋ-ಗಿನೇ
ಕಾಲ-ಯೋಗೀ
ಕಾಲಶ್ಚ
ಕಾಲಾಯ
ಕಾಲೋ
ಕಾಷ್ಠಾ
ಕಾಷ್ಠಾಭ್ಯೋ
ಕಾಹ-ಲಯೇ
ಕಾಹ-ಲಿಃ
ಕಿಂ
ಕುಂಡಿನೇ
ಕುಂಡೀ
ಕುರು-ಕರ್ತಾ
ಕುರು-ಕರ್ತ್ರೇ
ಕುರು-ಭೂತೋ
ಕುರು-ವಾ-ಸಿನೇ
ಕುರು-ವಾಸೀ
ಕೂಪ-ಸ್ತ್ರಿ-ಯುಗಃ
ಕೂಲ-ಕರ್ತಾ
ಕೂಲ-ಕರ್ತ್ರೇ
ಕೂಲ-ಹಾ-ರಿಣೇ
ಕೂಲ-ಹಾರೀ
ಕೃತೈ-ರ್ವೇ-ದ-ಕೃ-ತಾ-ತ್ಮನಾ
ಕೃತ್ತಿ-ವಾ-ಸಸೇ
ಕೃತ್ತಿ-ವಾ-ಸಾಃ
ಕೃತ್ವಾ
ಕೃತ್ಸ್ನಂ
ಕೃಪಾ-ನಿ-ಧಯೇ
ಕೃಷ್ಣ
ಕೃಷ್ಣ-ಪಿಂ-ಗಲಃ
ಕೃಷ್ಣ-ಪಿಂ-ಗ-ಲಾಯ
ಕೃಷ್ಣ-ವರ್ಣಃ
ಕೃಷ್ಣ-ವ-ರ್ಣಾಯ
ಕೃಷ್ಣಾಯ
ಕೇತವೇ
ಕೇತು-ಮಾ-ಲಿನೇ
ಕೇತು-ಮಾಲೀ
ಕೇತು-ರ್ಗ್ರಹೋ
ಕೇನ-ಚಿತ್
ಕೈಲಾ-ಸ-ಗಿ-ರಿ-ವಾ-ಸಿನೇ
ಕೈಲಾ-ಸ-ಗಿ-ರಿ-ವಾಸೀ
ಕೈಲಾ-ಸ-ವಾ-ಸಿನೇ
ಕೊಟ್ಟಿ-ರು-ವಂ-ತೆಯೇ
ಕ್ರಮಃ
ಕ್ರಿಯಾ-ವ-ಸ್ಥಾಯ
ಕ್ರಿಯಾ-ವಸ್ಥೋ
ಕ್ಷಣಾಃ
ಕ್ಷಣೇಭ್ಯೋ
ಖಂಡ-ಪ-ರ-ಶವೇ
ಖಗಃ
ಖಗಾಯ
ಖಚರೋ
ಖಟ್ವಾಂ-ಗಿನೇ
ಖಡ್ಗಿನೇ
ಖಡ್ಗೀ
ಖಲಿನೇ
ಖಲೀ
ಖೇಚ-ರಾಯ
ಖ್ಯಾತೈ-ರ್ಮು-ನಿ-ಭಿ-ಸ್ತ-ತ್ತ್ವ-ದ-ರ್ಶಿ-ಭಿಃ
ಖ್ಯಾತೋ
ಗಂಗಾ-ಧ-ರಾಯ
ಗಂಡ-ಲಿನೇ
ಗಂಡಲೀ
ಗಂಧ-ಧಾ-ರಿಣೇ
ಗಂಧ-ಧಾರೀ
ಗಂಧ-ಪಾ-ಲಿ-ಭ-ಗ-ವತೇ
ಗಂಧ-ಪಾಲೀ
ಗಂಧ-ರ್ವಾಯ
ಗಂಧರ್ವೋ
ಗಂಭೀ-ರ-ಘೋ-ಷಾಯ
ಗಂಭೀ-ರ-ಘೋಷೋ
ಗಂಭೀ-ರ-ಬ-ಲ-ವಾ-ಹನಃ
ಗಂಭೀ-ರ-ಬ-ಲ-ವಾ-ಹ-ನಾಯ
ಗಂಭೀ-ರಾಯ
ಗಂಭೀರೋ
ಗಚ್ಛೇತ
ಗಜಘ್ನೇ
ಗಜಹಾ
ಗಣಃ
ಗಣ-ಕರ್ತಾ
ಗಣ-ಕರ್ತ್ರೇ
ಗಣ-ಕಾ-ರಶ್ಚ
ಗಣ-ಕಾ-ರಾಯ
ಗಣ-ನಾ-ಥಾಯ
ಗಣ-ಪ-ತಯೇ
ಗಣ-ಪ-ತಿ-ರ್ದಿ-ಗ್ವಾ-ಸಾಃ
ಗಣಾ-ಧ್ಯ-ಕ್ಷಾಯ
ಗಣಾಯ
ಗತಯೇ
ಗತಾ-ಗತಃ
ಗತಾ-ಗ-ತಾಯ
ಗತಿಂ
ಗತಿಃ
ಗತಿಮ್
ಗಭ-ಸ್ತಯೇ
ಗಭ-ಸ್ತಿ-ರ್ಬ್ರ-ಹ್ಮ-ಕೃ-ದ್ಬ್ರಹ್ಮೀ
ಗಮ್ಯತೇ
ಗವಾಂ-ಪ-ತಯೇ
ಗವಾಂ-ಪ-ತಿಃ
ಗಾಂಧಾ-ರಶ್ಚ
ಗಾಂಧಾ-ರಾಯ
ಗಿರಿ-ಧ-ನ್ವನೇ
ಗಿರಿ-ಪ್ರಿ-ಯಾಯ
ಗಿರಿ-ರು-ಹಾಯ
ಗಿರಿ-ರುಹೋ
ಗಿರಿ-ಶಾಯ
ಗಿರಿ-ಸಾ-ಧನಃ
ಗಿರಿ-ಸಾ-ಧ-ನಾಯ
ಗಿರೀ-ಶಾಯ
ಗಿರೇಃ
ಗೀಯತೇ
ಗುಣ-ಬು-ದ್ಧಯೇ
ಗುಣಾ-ಕರಃ
ಗುಣಾ-ಕ-ರಾಯ
ಗುಣಾ-ಧಿಕಃ
ಗುಣಾ-ಧಿ-ಕಾಯ
ಗುಣಾನ್
ಗುಣೌ-ಷಧಃ
ಗುಣೌ-ಷ-ಧಾಯ
ಗುರವೇ
ಗುರುಃ
ಗುಹಃ
ಗುಹಾ-ಪಾಲಃ
ಗುಹಾಯ
ಗುಹಾ-ವಾ-ಸಿನೇ
ಗುಹಾ-ವಾಸೀ
ಗುಹೋ
ಗುಹ್ಯಃ
ಗುಹ್ಯಕಾ
ಗುಹ್ಯಾಯ
ಗೋಚ-ರಾಯ
ಗೋಚ-ರೋ-ಽರ್ದನಃ
ಗೋಚ-ರ್ಮ-ವ-ಸ-ನಾಯ
ಗೋಚ-ರ್ಮ-ವ-ಸನೋ
ಗೋಪ-ತಯೇ
ಗೋಪಾ-ಲಯೇ
ಗೋಪಾ-ಲಿ-ರ್ಗೋ-ಪ-ತಿ-ರ್ಗ್ರಾಮೋ
ಗೋವೃ-ಷೇ-ಶ್ವರಃ
ಗೌತಮಃ
ಗೌತ-ಮಾಯ
ಗೌತು-ಮೋಽಥ
ಗೌರೀ
ಗ್ರಂಥ-ಗ-ಳಿಂ-ದಾ-ಗಲೀ
ಗ್ರಹ-ಪ-ತಯೇ
ಗ್ರಹ-ಪ-ತಿ-ರ್ವರಃ
ಗ್ರಹಾಯ
ಗ್ರಾಮಾಯ
ಘೃತ-ಮಿ-ವೋ-ದ್ಧೃ-ತಮ್
ಘೃತಾ-ತ್ಸಾರಂ
ಘೋರ-ತ-ಪಸೇ
ಘೋರ-ತಪಾ
ಘೋರಾಯ
ಘೋರೋ
ಚ
ಚಂದ-ನಾಯ
ಚಂದ-ನಿನೇ
ಚಂದನೀ
ಚಂದ್ರಃ
ಚಂದ್ರ-ವ-ಕ್ತ್ರಾಯ
ಚಂದ್ರಾಯ
ಚತು-ರ್ಮು-ಖಾಯ
ಚತು-ರ್ಮುಖೋ
ಚತು-ರ್ವೇ-ದ-ಸ-ಮ-ನ್ವಿ-ತಮ್
ಚತು-ಷ್ಪ-ಥಾಯ
ಚಮೂ-ಸ್ತಂ-ಭನ
ಚಮೂ-ಸ್ತಂ-ಭ-ನಾಯ
ಚಯಂ
ಚರಾ-ಚ-ರಾ-ತ್ಮನೇ
ಚರಾ-ಚ-ರಾತ್ಮಾ
ಚರೇ-ದ್ಭಕ್ತಃ
ಚರ್ಮಿಣೇ
ಚರ್ಮೀ
ಚಲಾಯ
ಚಾಯುಧೀ
ಚಾರು-ಚಂ-ದ್ರಾ-ವ-ತಂಸಂ
ಚಾರು-ಲಿಂ-ಗಾಯ
ಚಾರು-ವಿ-ಕ್ರ-ಮಾಯ
ಚಾಸ್ಯ
ಚಿಂತ-ಯಂತಃ
ಚೀರ-ವಾ-ಸಸೇ
ಚೀರ-ವಾ-ಸಾಶ್ಚ
ಚೇಕಿ-ತಾ-ನಾಯ
ಚೇಕಿ-ತಾನೋ
ಚೈವ
ಚೈವಾ-ಯ-ಮೀ-ಶ್ವ-ರಸ್ಯ
ಛಂದಃ
ಛಂದೋ-ವ್ಯಾ-ಕ-ರ-ಣೋ-ತ್ತರ
ಛತ್ರಂ
ಛತ್ರಾಯ
ಛದಾಯ
ಜಂಗ-ಮ-ಸ್ತಥಾ
ಜಂಗ-ಮಾಯ
ಜಂತೋ-ರ್ವಿ-ಗ-ತ-ಪಾ-ಪಸ್ಯ
ಜಗತಿ
ಜಗತೇ
ಜಗತ್
ಜಗ-ತ್ಕಾ-ಲ-ಸ್ಥಾ-ಲಾಯ
ಜಗ-ತ್ಕಾ-ಲ-ಸ್ಥಾಲೋ
ಜಗ-ತ್ಪ-ತಿಮ್
ಜಗ-ತ್ಯ-ಮ-ರ-ಪೂ-ಜಿತಃ
ಜಗ-ದ್ಗು-ರವೇ
ಜಗ-ದ್ಯೋ-ನೇ-ರ್ಮ-ಹಾ-ತ್ಮನಃ
ಜಗ-ದ್ವ್ಯಾ-ಪಿನೇ
ಜಟಾ-ಧರಃ
ಜಟಾ-ಧ-ರಾಯ
ಜಟಿನೇ
ಜಟೀ
ಜನಾ-ರ್ದನ
ಜನ್ಮ-ಕೋ-ಟಿ-ಸ-ಹ-ಸ್ರೇಷು
ಜನ್ಯಾಯ
ಜನ್ಯೋ
ಜಪೇ
ಜಪ್ಯ-ಮಿದಂ
ಜಯ-ನ-ಗರ
ಜಲ-ಜೋ-ದ್ಭ-ವಾಯ
ಜಲೇ-ಶಯಃ
ಜಲೇ-ಶ-ಯಾಯ
ಜಲೋ-ದ್ಭವಃ
ಜಾಗ್ರ-ಮಾ-ಣಾಶ್ಚ
ಜಾಹ್ನ-ವೀ-ಧೃತೇ
ಜಾಹ್ನ-ವೀ-ಧೃ-ದು-ಮಾ-ಧವಃ
ಜಿತ-ಕಾ-ಮಾಯ
ಜಿತ-ಕಾಮೋ
ಜಿತೇಂ-ದ್ರಿಯಃ
ಜಿತೇಂ-ದ್ರಿ-ಯಾಯ
ಜೀವ-ನಾಯ
ಜೀವನೋ
ಜ್ಞಾತ್ವಾ
ಜ್ಞಾನಂ
ಜ್ಯೋತಿ-ಷಾ-ಮ-ಯನಂ
ಜ್ಯೋತಿ-ಷಾ-ಮ-ಯ-ನಾಯ
ಜ್ವಾಲಿನೇ
ಜ್ವಾಲೀ
ತಂಡಿಃ
ತಂಡಿ-ಕೃ-ತೋ-ಽಭ-ವತ್
ತಂಡಿನಾ
ತತಃ
ತತೋ-ಽನು-ಜ್ಞಾಂ
ತತೋ-ಽಭ್ಯ-ನು-ಜ್ಞಾಂ
ತತ್ಪ-ರಮ್
ತಥಾ
ತಥೈವ
ತದಾ
ತದಾ-ಪ್ರ-ಭೃತಿ
ತದ್ಭಾ-ವ-ಗ-ತ-ತೇ-ಜಸಃ
ತಪಃ
ತಪಸಾ
ತಪ-ಸ್ವಿನೇ
ತಪಸ್ವೀ
ತಪ-ಸ್ಸಕ್ತೋ
ತಪೋ-ನಿ-ಧಯೇ
ತಪೋ-ಮ-ಯಾಯ
ತರಂ-ಗ-ವಿತ್
ತರಂ-ಗ-ವಿದೇ
ತರವೇ
ತಲ-ಸ್ತಾಲಃ
ತಲ-ಸ್ತಾ-ಲಾಯ
ತವಾ-ಪ್ಯ-ಹ-ಮ-ಮಿ-ತ್ರ-ಘ್ನ-ಸ್ತವಂ
ತಸ್ಯೈವ
ತಾತ
ತಾನಿ
ತಾಮ್ರೋ-ಷ್ಠಾಯ
ತಾರ-ಕಾಯ
ತಾರ-ಣಾಯ
ತಾರ್ಕ್ಷ-್ಯಾಯ
ತಾಲಾಯ
ತಾಲಿನೇ
ತಾಲೀ
ತಿಗ್ಮ-ತೇ-ಜಾಃ
ತಿಗ್ಮ-ಮ-ನ್ಯವೇ
ತೀಕ್ಷ
ತೀಕ್ಷ್ಣ-ತಾ-ಪಶ್ಚ
ತೀರ್ಥ-ದೇ-ವಾಯ
ತು
ತುಂಬ-ವೀ-ಣಾಯ
ತುಂಬ-ವೀಣೋ
ತುಷ್ಯಂತಿ
ತೇ
ತೇಜ-ಸಾ-ಮಪಿ
ತೇಜಸೇ
ತೇಜ-ಸ್ಕ-ರ-ನಿ-ಧಯೇ
ತೇಜ-ಸ್ತೇ-ಜ-ಸ್ಕರೋ
ತೇಜೋ-ಪ-ಹಾ-ರಿಣೇ
ತೇಜೋ-ಽಪ-ಹಾರೀ
ತೇನ
ತೇನಾ-ಭ್ಯ-ನು-ಜ್ಞಾತಃ
ತೇನೇಂ-ದ್ರ-ಕ-ಲ್ಪೇನ
ತೋರ-ಣ-ಸ್ತಾ-ರಣೋ
ತೋರ-ಣಾಯ
ತೋಷದಂ
ತ್ರಯೀ-ಮೂ-ರ್ತಯೇ
ತ್ರಾಸ-ನಾಯ
ತ್ರಿಕ-ಕು-ನ್ಮಂತ್ರಃ
ತ್ರಿಕ-ಕು-ನ್ಮಂ-ತ್ರಾಯ
ತ್ರಿಕಾ-ಲ-ಧೃತೇ
ತ್ರಿಜ-ಟಿನೇ
ತ್ರಿಜಟೀ
ತ್ರಿದ-ಶ-ಸ್ತ್ರಿ-ಕಾ-ಲ-ಧೃತ್
ತ್ರಿದ-ಶಾಯ
ತ್ರಿನೇ-ತ್ರಮ್
ತ್ರಿನೇ-ತ್ರಾಯ
ತ್ರಿಪು-ರಾಂ-ತ-ಕಾಯ
ತ್ರಿಯು-ಗಾಯ
ತ್ರಿಲೋ-ಕೇ-ಶಾಯ
ತ್ರಿಲೋ-ಚನೋ
ತ್ರಿವಿ-ಕ್ರ-ಮಾಯ
ತ್ರಿವಿ-ಕ್ರಮೋ
ತ್ರಿವಿ-ಷ್ಟ-ಪಮ್
ತ್ರಿವಿ-ಷ್ಟ-ಪಾಯ
ತ್ರಿಶಂ-ಕವೇ
ತ್ರಿಶಂ-ಕು-ರ-ಜಿತಃ
ತ್ರಿಶು-ಕ್ಲಾಯ
ತ್ರ್ಯಕ್ಷಾಯ
ತ್ವಮಿ-ತ್ರ-ಜಿತ್
ತ್ವಷ್ಟ್ರೇ
ದಂಡಿನೇ
ದಂಡೀ
ದಂಭಾಯ
ದಂಭೋ
ದಕ್ಷಃ
ದಕ್ಷ-ಯಾ-ಗಾ-ಪ-ಹಾ-ರಿಣೇ
ದಕ್ಷ-ಯಾ-ಗಾ-ಪ-ಹಾರೀ
ದಕ್ಷಾ-ಧ್ವ-ರ-ಹ-ರಾಯ
ದಕ್ಷಾಯ
ದಕ್ಷಿ-ಣಾಯ
ದದ್ಯಾಂ
ದಧ್ನೋ
ದಮನಃ
ದಮ-ನಾಯ
ದರ್ಪ-ಣಾಯ
ದರ್ಪ-ಣೋಽಥ
ದರ್ಭ-ಚಾ-ರಿಣೇ
ದರ್ಭ-ಚಾರೀ
ದಶ
ದಶ-ಬಾ-ಹವೇ
ದಶ-ಬಾ-ಹು-ಸ್ತ್ವ-ನಿ-ಮಿಷೋ
ದಾಂತಾ-ನಾ-ಮಪಿ
ದಾಂತೋ
ದಾತವ್ಯಂ
ದಾನವಾ
ದಿಗಂ-ಬ-ರಾಯ
ದಿಗ್ವಾ-ಸಸೇ
ದಿವಿ-ಸು-ಪ-ರ್ವಣಃ
ದಿವಿ-ಸು-ಪ-ರ್ವ-ಣ-ದೇ-ವಾಯ
ದಿವ್ಯಾ-ನಾಂ
ದೀನ-ಸಾ-ಧಕಃ
ದೀನ-ಸಾ-ಧ-ಕಾಯ
ದೀರ್ಘಶ್ಚ
ದೀರ್ಘಾಯ
ದುರ್ಧ-ರ್ಷಾಯ
ದುರ್ವಾ-ಸಸೇ
ದುರ್ವಾಸಾ
ದುರ್ವಾ-ಸಾಃ
ದುಷ್ಪ್ರಾಪಂ
ದೇವ
ದೇವಂ
ದೇವಃ
ದೇವತಾ
ದೇವ-ತಾ-ತ್ಮನೇ
ದೇವ-ತಾ-ತ್ಮಾ-ಽತ್ಮ-ಸಂ-ಭವಃ
ದೇವ-ದೇವಃ
ದೇವ-ದೇ-ವಾಯ
ದೇವ-ರ್ಷಯೇ
ದೇವ-ರ್ಷಿ-ರ್ದೇ-ವಾ-ಸು-ರ-ವ-ರ-ಪ್ರದಃ
ದೇವ-ಸಿಂ-ಹಾಯ
ದೇವ-ಸಿಂಹೋ
ದೇವಸ್ಯ
ದೇವಾ
ದೇವಾಃ
ದೇವಾ-ತಿ-ದೇ-ವಾಯ
ದೇವಾ-ತಿ-ದೇವೋ
ದೇವಾ-ಧಿ-ಪ-ತಯೇ
ದೇವಾ-ಧಿ-ಪ-ತಿ-ರೇವ
ದೇವಾ-ನಾ-ಮಪಿ
ದೇವಾಯ
ದೇವಾಶ್ಚ
ದೇವಾ-ಸುರ
ದೇವಾ-ಸು-ರ-ಗ-ಣಾ-ಗ್ರ-ಣೀಃ
ದೇವಾ-ಸು-ರ-ಗ-ಣಾ-ಗ್ರಣ್ಯೇ
ದೇವಾ-ಸು-ರ-ಗ-ಣಾ-ಧ್ಯಕ್ಷೋ
ದೇವಾ-ಸು-ರ-ಗ-ಣಾ-ಶ್ರಯಃ
ದೇವಾ-ಸು-ರ-ಗ-ಣಾ-ಶ್ರ-ಯಾಯ
ದೇವಾ-ಸು-ರ-ಗು-ರು-ರ್ದೇವೋ
ದೇವಾ-ಸು-ರ-ನ-ಮ-ಸ್ಕೃತಃ
ದೇವಾ-ಸು-ರ-ಪ-ತಯೇ
ದೇವಾ-ಸು-ರ-ಪ-ತಿಃ
ದೇವಾ-ಸು-ರ-ಪ-ರಾ-ಯಣಃ
ದೇವಾ-ಸು-ರ-ಮ-ಹಾ-ಮಾತ್ರೋ
ದೇವಾ-ಸು-ರ-ಮ-ಹೇ-ಶ್ವರಃ
ದೇವಾ-ಸು-ರ-ವಿ-ನಿ-ರ್ಮಾತಾ
ದೇವಾ-ಸು-ರೇ-ಶ್ವ-ರಾಯ
ದೇವಾ-ಸು-ರೇ-ಶ್ವರೋ
ದೇವೇಂದ್ರಃ
ದೇವೇಂ-ದ್ರಾಯ
ದೇವೋ
ದೇಹಶ್ಚ
ದೇಹಾಯ
ದೈತ್ಯಘ್ನೇ
ದೈತ್ಯಹಾ
ದ್ಯುತಿಃ
ದ್ವಾದ-ಶ-ಸ್ತ್ರಾ-ಸ-ನ-ಶ್ಚಾದ್ಯೋ
ದ್ವಾದ-ಶಾಯ
ಧನ್ಯಂ
ಧನ್ವಂ-ತ-ರಯೇ
ಧನ್ವಂ-ತ-ರಿ-ರ್ಧೂ-ಮ-ಕೇ-ತುಃ
ಧನ್ವಿನೇ
ಧನ್ವೀ
ಧರಃ
ಧರಾಯ
ಧರೋ-ತ್ತಮಃ
ಧರೋ-ತ್ತ-ಮಾಯ
ಧರ್ಮ-ಸಾ-ಧಾ-ರಣೋ
ಧರ್ಷ-ಣಾ-ತ್ಮನೇ
ಧರ್ಷ-ಣಾತ್ಮಾ
ಧಾತಾ
ಧಾತ್ರೇ
ಧಾರ್ಯಂ
ಧೀಃ
ಧೀಮತಃ
ಧೀಮ-ತಾ-ಮಪಿ
ಧೀಮತೇ
ಧೂಪಾದಿ
ಧೂಮ-ಕೇ-ತನಃ
ಧೂಮ-ಕೇ-ತ-ನಾಯ
ಧೂಮ-ಕೇ-ತವೇ
ಧೃತಿ-ಮತೇ
ಧೃತಿ-ಮಾನ್
ಧ್ಯಾನ-ಮಿದಂ
ಧ್ಯಾನಮ್
ಧ್ಯಾಯೇ-ನ್ನಿತ್ಯಂ
ಧ್ಯೇಯ-ಮ-ನು-ತ್ತ-ಮಮ್
ಧ್ರುವಃ
ಧ್ರುವಾಯ
ಧ್ರುವೋಽಥ
ನ
ನಂದ-ನಾಯ
ನಂದನೋ
ನಂದಯೇ
ನಂದಿ-ಕ-ರಾಯ
ನಂದಿನೇ
ನಂದಿ-ರ್ನಂ-ದ-ಕರೋ
ನಂದಿ-ವ-ರ್ಧನಃ
ನಂದಿ-ವ-ರ್ಧ-ನಾಯ
ನಂದೀ
ನಂದೀ-ಶ್ವ-ರಶ್ಚ
ನಂದೀ-ಶ್ವ-ರಾಯ
ನಕ್ತಂ
ನಕ್ತಂ-ಚ-ರ-ಸ್ತಿ-ಗ್ಮ-ಮ-ನ್ಯುಃ
ನಕ್ತಂ-ಚ-ರಾಯ
ನಕ್ತಾಯ
ನಕ್ಷತ್ರ
ನಕ್ಷ-ತ್ರ-ಸಾ-ಧಕಃ
ನಕ್ಷ-ತ್ರ-ಸಾ-ಧ-ಕಾಯ
ನಭಃ
ನಭಃ-ಸ್ಥಲಃ
ನಭಸೇ
ನಭ-ಸ್ಥ-ಲಾಯ
ನಮಃ
ನಮ-ಸ್ಕರ್ತಾ
ನಮ-ಸ್ಕೃ-ತಾಯ
ನರಃ
ನರಕಂ
ನರ-ರ್ಷಭಃ
ನರ-ರ್ಷ-ಭಾಯ
ನರಾಯ
ನರ್ತಕಃ
ನರ್ತ-ಕಾಯ
ನರ್ಷಯಃ
ನವ-ಚ-ಕ್ರಾಂಗಃ
ನವ-ಚ-ಕ್ರಾಂ-ಗಾಯ
ನಾಚಿ-ಕೇ-ತಾಯ
ನಾಚಿ-ಕೇ-ತೋ-ಽಭ್ಯ-ಭಾ-ಷತ
ನಾನಾ-ಸಂ-ಸಾ-ರ-ಯೋ-ನಿಷು
ನಾನ್ಯಾ
ನಾಭಯೇ
ನಾಭಿ-ರ್ನಂ-ದಿ-ಕರೋ
ನಾಮ-ಕ್ಕೊಂ-ದ-ರಂತೆ
ನಾಮ-ಭಿಃ
ನಾಮ-ಸ-ಹ-ಸ್ರಾಣಿ
ನಾಮ್ನಾಂ
ನಾರಾ-ಯಣ
ನಾರಾ-ಯ-ಣಾಯ
ನಾರಾ-ಯ-ಣೋ-ಽಚ್ಯುತಃ
ನಾಶ್ರ-ದ್ದ-ಧಾ-ನ-ರೂ-ಪಾಯ
ನಾಸ್ತಿ-ಕಾ-ಯಾ-ಜಿ-ತಾ-ತ್ಮನೇ
ನಾಸ್ಯ
ನಿಂದಿತಃ
ನಿಖಿ-ಲ-ಭ-ಯ-ಹರಂ
ನಿಗ-ದಿಷ್ಯೇ
ನಿಗ್ರಹಃ
ನಿಗ್ರ-ಹಾಯ
ನಿಜಃ
ನಿಜ-ಸ-ರ್ಗಾಯ
ನಿತ್ಯ
ನಿತ್ಯಂ
ನಿತ್ಯ-ನ-ರ್ತಾಯ
ನಿತ್ಯ-ನರ್ತೋ
ನಿತ್ಯ-ಮಾ-ಶ್ರ-ಮ-ಪೂ-ಜಿತಃ
ನಿತ್ಯ-ಮಾ-ಶ್ರ-ಮ-ಪೂ-ಜಿ-ತಾಯ
ನಿತ್ಯ-ಯುಕ್ತಃ
ನಿತ್ಯಾ-ತ್ಮ-ಸ-ಹಾ-ಯಾಯ
ನಿತ್ಯಾಯ
ನಿತ್ಯೋ
ನಿಧಯೇ
ನಿಧಿಃ
ನಿಪಾ-ತಿನೇ
ನಿಪಾತೀ
ನಿಮಿತ್ತಂ
ನಿಮಿ-ತ್ತಸ್ಥೋ
ನಿಮಿ-ತ್ತ-ಸ್ಮಾಯ
ನಿಮಿ-ತ್ತಾಯ
ನಿಯತಃ
ನಿಯ-ತಾ-ತ್ಮ-ಭಿಃ
ನಿಯ-ತಾಯ
ನಿಯ-ಮಾಯ
ನಿಯ-ಮಾ-ಶ್ರಿತಃ
ನಿಯ-ಮಾ-ಶ್ರಿ-ತಾಯ
ನಿಯ-ಮೇಂ-ದ್ರಿ-ಯ-ವ-ರ್ಧನಃ
ನಿಯ-ಮೇಂ-ದ್ರಿ-ಯ-ವ-ರ್ಧ-ನಾಯ
ನಿಯ-ಮೇನ
ನಿಯಮೋ
ನಿರ-ವ-ಗ್ರಹಃ
ನಿರ-ವ-ಗ್ರ-ಹಾಯ
ನಿರಾ-ಮಯಃ
ನಿರಾ-ಮ-ಯಾಯ
ನಿರ್ಗಮಃ
ನಿರ್ಜೀ-ವಾಯ
ನಿರ್ಜೀವೋ
ನಿರ್ಮಥ್ಯ
ನಿರ್ವಾಣಂ
ನಿರ್ವಾ-ಣಾಯ
ನಿರ್ವಿಘ್ನಾ
ನಿಲ-ಯಶ್ಚ
ನಿವೃ-ತ್ತಯೇ
ನಿವೃ-ತ್ತಿಶ್ಚ
ನಿವೇ-ದನಃ
ನಿವೇ-ದ-ನಾಯ
ನಿಶಾ-ಕರಃ
ನಿಶಾ-ಕ-ರಾಯ
ನಿಶಾ-ಚರಃ
ನಿಶಾ-ಚ-ರಾಯ
ನಿಶಾ-ಚಾ-ರಿಣೇ
ನಿಶಾ-ಚಾರೀ
ನಿಶಾ-ಲಯಃ
ನಿಶಾ-ಲ-ಯಾಯ
ನಿಶ್ಚಲಾ
ನಿಹಂತಾ
ನಿಹಂತ್ರೇ
ನೀತಯೇ
ನೀತಿ-ರ್ಹ್ಯ-ನೀ-ತಿಃ
ನೀರ-ಜಾಯ
ನೀರ-ಜೋ-ಽಮರಃ
ನೀಲ-ಕಂಠ
ನೀಲ-ಕಂ-ಠಾಯ
ನೀಲ-ಮೌ-ಲಯೇ
ನೀಲ-ಮೌ-ಲಿಃ
ನೀಲ-ಲೋ-ಹಿ-ತಾಯ
ನೀಲ-ಸ್ತ-ಥಾಂ-ಗ-ಲು-ಬ್ಧಶ್ಚ
ನೀಲಾಯ
ನೃಣಾಮ್
ನೃತ್ಯ-ಪ್ರಿ-ಯಾಯ
ನೃತ್ಯ-ಪ್ರಿಯೋ
ನೆರ-ವೇ-ರಿಸಿ
ನೈಕ-ಸಾ-ನು-ಚ-ರಾಯ
ನೈಕಾ-ತ್ಮನೇ
ನೈಕಾತ್ಮಾ
ನ್ಯಗ್ರೋ-ಧ-ರೂ-ಪಾಯ
ನ್ಯಗ್ರೋ-ಧ-ರೂಪೋ
ನ್ಯಗ್ರೋ-ಧಾಯ
ನ್ಯಗ್ರೋಧೋ
ನ್ಯಾಯ-ನಿ-ರ್ವ-ಪಣಃ
ನ್ಯಾಯ-ನಿ-ರ್ವ-ಪ-ಣಾಯ
ಪಂಚ-ವಕ್ತ್ರಂ
ಪಂಚ-ವ-ಕ್ತ್ರಾಯ
ಪಂಡಿ-ತಾಯ
ಪಂಡಿತೋ
ಪಕ್ಷಃ
ಪಕ್ಷ-ರೂ-ಪಶ್ಚ
ಪಕ್ಷ-ರೂ-ಪಾಯ
ಪಕ್ಷಾಯ
ಪಕ್ಷಿಣೇ
ಪಕ್ಷೀ
ಪಟ್ಟಿ-ಶಿನೇ
ಪಟ್ಟಿಶೀ
ಪಠೇತ
ಪಣ-ವಿನೇ
ಪಣವೀ
ಪತಿಃ
ಪತಿ-ಖೇ-ಚರಃ
ಪತಿ-ಖೇ-ಚ-ರಾಯ
ಪತಿ-ಶ್ಚೈವ
ಪತ್ಯೇ
ಪದ್ಮ-ಗ-ರ್ಭಾಯ
ಪದ್ಮ-ಗರ್ಭೋ
ಪದ್ಮ-ನಾ-ಭಾಯ
ಪದ್ಮ-ನಾಭೋ
ಪದ್ಮ-ನಾ-ಲಾಗ್ರಃ
ಪದ್ಮ-ನಾ-ಲಾ-ಗ್ರಾಯ
ಪದ್ಮ-ಯೋ-ನಿನಾ
ಪದ್ಮಾ-ಸೀನಂ
ಪಯೋ-ನಿ-ಧಯೇ
ಪಯೋ-ನಿ-ಧಿಃ
ಪರಂ
ಪರ-ಬ್ರ-ಹ್ಮಣೇ
ಪರಮಂ
ಪರ-ಮ-ತ-ಪಸೇ
ಪರಮಾ
ಪರ-ಮಾಂ
ಪರ-ಮಾ-ಗ-ತಯೇ
ಪರ-ಮಾ-ತ್ಮನೇ
ಪರ-ಮಾತ್ಮಾ
ಪರ-ಮಾಯ
ಪರ-ಮೇ-ಶ್ವ-ರಮ್
ಪರ-ಮೇ-ಶ್ವ-ರಾಯ
ಪರ-ಶು-ಮೃ-ಗ-ವ-ರಾ-ಭೀ-ತಿ-ಹಸ್ತಂ
ಪರ-ಶು-ಹ-ಸ್ತಾಯ
ಪರ-ಶ್ವ-ಧಾ-ಯು-ಧಾಯ
ಪರ-ಶ್ವ-ಧಾ-ಯುಧೋ
ಪರಸ್ಮೈ
ಪರಾ
ಪರಾಂ
ಪರಾ-ಗ-ತಯೇ
ಪರಾ-ಣಾ-ಮಪಿ
ಪರಾ-ಯ-ಣಾಯ
ಪರಿ-ಧಿನೇ
ಪರಿಧೀ
ಪರೋ-ಽಪರಃ
ಪರ್ಯ-ಯ-ನ-ರಾಯ
ಪರ್ಯಯೋ
ಪವಿತ್ರಂ
ಪವಿ-ತ್ರಾಯ
ಪಶು-ಪ-ತಯೇ
ಪಶು-ಪ-ತಿ-ರ್ಮ-ಹಾ-ಕರ್ತಾ
ಪಶು-ಪ-ತಿ-ರ್ವಾ-ತ-ರಂಹಾ
ಪಾದಃ
ಪಾದಾಯ
ಪಾಪ-ನಾ-ಶ-ನಮ್
ಪಾರ್ಥ
ಪಾವನಂ
ಪಾಶ-ವಿ-ಮೋ-ಚ-ನಾಯ
ಪಾಶಾಯ
ಪಾಶೋ
ಪಿತಾ
ಪಿತಾ-ಮಹಃ
ಪಿತಾ-ಮ-ಹಾಯ
ಪಿತ್ರೇ
ಪಿನಾ-ಕ-ಧೃತೇ
ಪಿನಾ-ಕ-ಧೃತ್
ಪಿನಾ-ಕ-ವತೇ
ಪಿನಾ-ಕ-ವಾನ್
ಪಿನಾ-ಕಿನೇ
ಪಿಶಾಚಾ
ಪೀತಾ-ತ್ಮನೇ
ಪೀತಾತ್ಮಾ
ಪುಕ್ಸ-ಹ-ಸ್ರಾ-ಮಿ-ತೇ-ಕ್ಷಣಃ
ಪುಕ್ಸ-ಹ-ಸ್ರಾ-ಮಿ-ತೇ-ಕ್ಷ-ಣಾಯ
ಪುಣ್ಯಂ
ಪುಣ್ಯ-ಚುಂ-ಚವೇ
ಪುಣ್ಯ-ಚುಂ-ಚುರೀ
ಪುತವೇ
ಪುತುಃ
ಪುನಃ
ಪುರ-ಸ್ಕೃತ್ಯ
ಪುರಾ
ಪುರಾಣಃ
ಪುರಾ-ಣಾಯ
ಪುರಾ-ರಾ-ತಯೇ
ಪುರು-ಭೂ-ತಾಯ
ಪುಷ-ಯ-ಶ್ಚೈವ
ಪುಷಿಃ
ಪುಷಿಣಾ
ಪುಷೀ-ಣಾ-ಮಪಿ
ಪುಷ್ಕ-ರ-ಸ್ಥ-ಪ-ತಯೇ
ಪುಷ್ಕ-ರ-ಸ್ಥ-ಪ-ತಿಃ
ಪುಷ್ಟಿ-ವ-ರ್ಧ-ನೈಃ
ಪುಷ್ಪ-ಸಾರಂ
ಪೂಜಾ-ವಿ-ಧಿ-ಗಳನ್ನು
ಪೂಜೆ-ಯಾದ
ಪೂರ್ವ-ಪೀ-ಠಿಕಾ
ಪೂಷ-ದಂ-ತ-ಭಿದೇ
ಪೌಷ್ಟಿಕಂ
ಪ್ರಕಾ-ರವು
ಪ್ರಕಾ-ಶ-ಕರು
ಪ್ರಕಾ-ಶಾಯ
ಪ್ರಕಾಶೋ
ಪ್ರಕೃ-ಷ್ಟಾ-ರಯೇ
ಪ್ರಕೃ-ಷ್ಟಾ-ರಿ-ರ್ಮ-ಹಾ-ಹರ್ಷೋ
ಪ್ರಚಾ-ಪ-ತಯೇ
ಪ್ರಜಾ-ದ್ವಾರಂ
ಪ್ರಜಾ-ದ್ವಾ-ರಾಯ
ಪ್ರಜಾ-ಪ-ತಯೇ
ಪ್ರಜಾ-ಪ-ತಿ-ರ್ವಿ-ಶ್ವ-ಬಾ-ಹು-ರ್ವಿ-ಭಾಗಃ
ಪ್ರಜಾ-ಬೀಜಂ
ಪ್ರಜಾ-ಬೀ-ಜಾಯ
ಪ್ರಜಾ-ಯತೇ
ಪ್ರತಿ-ಗಳು
ಪ್ರತ್ಯ-ಯಾಯ
ಪ್ರತ್ಯಯೋ
ಪ್ರಥಮಂ
ಪ್ರಧಾ-ನತಃ
ಪ್ರಧಾ-ನ-ಧೃತೇ
ಪ್ರಪ-ದ್ಯಂತೇ
ಪ್ರಪ-ನ್ನ-ವ-ತ್ಸಲೋ
ಪ್ರಪಿ-ತಾ-ಮಹಃ
ಪ್ರಭ-ವ-ತಾ-ಮಪಿ
ಪ್ರಭ-ವಾಯ
ಪ್ರಭವೇ
ಪ್ರಭ-ವೋ-ಽವ್ಯಯಃ
ಪ್ರಭಾ
ಪ್ರಭಾವಃ
ಪ್ರಭಾ-ವಾ-ತ್ಮನೇ
ಪ್ರಭಾ-ವಾತ್ಮಾ
ಪ್ರಭಾ-ವಾಯ
ಪ್ರಭುಃ
ಪ್ರಭು-ರ್ಭೀಮಃ
ಪ್ರಮ-ಥಾ-ಧಿ-ಪಾಯ
ಪ್ರಮಾಣಂ
ಪ್ರಮಾ-ಣಾಯ
ಪ್ರಯ-ತಾ-ತ್ಮನಾ
ಪ್ರಯ-ತಾ-ತ್ಮನೇ
ಪ್ರಯ-ತಾ-ತ್ಮಾ-ಪ್ರ-ಧಾ-ನ-ಧೃತ್
ಪ್ರಯತೋ
ಪ್ರಯ-ತ್ನೇನ
ಪ್ರವರಂ
ಪ್ರವ-ರಾಯ
ಪ್ರವರೋ
ಪ್ರವೃ-ತ್ತಯೇ
ಪ್ರವೃ-ತ್ತಿಶ್ಚ
ಪ್ರವೇ-ಶಿ-ನಾಂ-ಗು-ಹಾ-ಪಾ-ಲಾಯ
ಪ್ರವೇ-ಶಿ-ನಾಮ್
ಪ್ರಶಾಂ-ತಾ-ತ್ಮನೇ
ಪ್ರಶಾಂ-ತಾತ್ಮಾ
ಪ್ರಸ-ನ್ನಮ್
ಪ್ರಸ-ನ್ನಶ್ಚ
ಪ್ರಸ-ನ್ನಾಯ
ಪ್ರಸಾ-ದಶ್ಚ
ಪ್ರಸಾ-ದಾ-ತ್ತಸ್ಯ
ಪ್ರಸಾ-ದಾಯ
ಪ್ರಸಾ-ದೇನ
ಪ್ರಸಾದೋ
ಪ್ರಸ್ಕಂ-ದ-ನಾಯ
ಪ್ರಸ್ಕಂ-ದನೋ
ಪ್ರಾಂಜ-ಲಿಃ
ಪ್ರಾಗ್ದ-ಕ್ಷಿ-ಣಶ್ಚ
ಪ್ರಾಚೇ
ಪ್ರಾಣ-ಧಾ-ರಣಃ
ಪ್ರಾಣ-ಧಾ-ರ-ಣಾಯ
ಪ್ರಾಣಿ-ನಾಂ
ಪ್ರಾಪ
ಪ್ರಾಪ್ತ-ಸ್ತಂ-ಡಿನಾ
ಪ್ರಾಪ್ತೋ
ಪ್ರಾಪ್ನು-ಯಾ-ದೀ-ಪ್ಸಿತಂ
ಪ್ರಾಪ್ನೋ-ತ್ಯಾ-ತ್ಮಾ-ನ-ಮಾ-ತ್ಮನಾ
ಪ್ರಾರಂಭಃ
ಪ್ರಾರಂ-ಭ-ದಲ್ಲಿ
ಪ್ರಾಸಾ-ನಾಂ
ಪ್ರಾಹ
ಪ್ರಿಯಃ
ಪ್ರಿಯಾಯ
ಪ್ರೆಯ-ತ್ನೇ-ನಾ-ಧಿ-ಗಂ-ತವ್ಯಂ
ಪ್ರೆಸ್
ಪ್ರೇತ-ಚಾ-ರಿಣೇ
ಪ್ರೇತ-ಚಾರೀ
ಪ್ರೋವಾಚ
ಫಲಮ್
ಫಲ-ಶ್ರು-ತಿಃ
ಬಂಧ-ಕರ್ತಾ
ಬಂಧ-ಕರ್ತ್ರೇ
ಬಂಧ-ನ-ಸ್ತ್ವ-ಸು-ರೇಂ-ದ್ರಾ-ಣಾಂ
ಬಂಧ-ನಾಯ
ಬಂಧ-ನಾ-ಯ-ನಮಃ
ಬಂಧನೋ
ಬಕು-ಲ-ಶ್ಚಂ-ದ-ನ-ಶ್ಛದಃ
ಬಕು-ಲಾಯ
ಬಭ್ರವೇ
ಬಭ್ರೂ
ಬಲಃ
ಬಲಘ್ನೇ
ಬಲ-ಚಾ-ರಿಣೇ
ಬಲ-ಚಾರೀ
ಬಲ-ರೂ-ಪ-ಧೃತೇ
ಬಲ-ರೂ-ಪ-ಧೃತ್
ಬಲ-ವ-ಚ್ಛಕ್ರ
ಬಲ-ವತೇ
ಬಲ-ವಾಂ-ಶ್ಚೋ-ಪ-ಶಾಂ-ತಶ್ಚ
ಬಲ-ವೀ-ರಾಯ
ಬಲ-ವೀ-ರೋ-ಽಬಲೋ
ಬಲಹಾ
ಬಲಾಯ
ಬಲಿನೇ
ಬಲೀ
ಬಲ್ಲ-ವ-ರಿಂ-ದಾ-ಗಲೀ
ಬಹು-ಕ-ರ್ಕಶಃ
ಬಹು-ಕ-ರ್ಕ-ಶಾಯ
ಬಹು-ಧರಃ
ಬಹು-ಧ-ರಾಯ
ಬಹುಧಾ
ಬಹು-ಧಾ-ನಿಂ-ದಿ-ತಾಯ
ಬಹು-ಪ್ರದಃ
ಬಹು-ಪ್ರ-ದಾಯ
ಬಹು-ಪ್ರ-ಸಾದಃ
ಬಹು-ಪ್ರ-ಸಾ-ದಾಯ
ಬಹು-ಭಿ-ರ್ಜ-ನ್ಮ-ಭಿಃ
ಬಹು-ಭೂ-ತಾಯ
ಬಹು-ಭೂತೋ
ಬಹು-ಮಾ-ಲಾಯ
ಬಹು-ಮಾಲೋ
ಬಹು-ರ-ಶ್ಮಯೇ
ಬಹು-ರ-ಶ್ಮಿಃ
ಬಹು-ರೂ-ಪಶ್ಚ
ಬಹು-ರೂ-ಪಾಯ
ಬಹುಲೋ
ಬಹು-ವಿ-ದ್ಯಾಯ
ಬಹು-ವಿದ್ಯೋ
ಬಾಣ-ಹಸ್ತಃ
ಬಾಣ-ಹ-ಸ್ತಾಯ
ಬಿಂದವೇ
ಬಿಂದು-ರ್ವಿ-ಸರ್ಗಃ
ಬಿಲ್ವಾ-ರ್ಚನೆ
ಬಿಲ್ವಾ-ರ್ಚ-ನೆ-ಗಾಗಿ
ಬೀಜ-ಕರ್ತಾ
ಬೀಜ-ಕರ್ತ್ರೇ
ಬೀಜಮ್
ಬೀಜ-ವಾ-ಹನಃ
ಬೀಜ-ವಾ-ಹ-ನಾಯ
ಬೀಜಾ-ಧ್ಯ-ಕ್ಷಾಯ
ಬೀಜಾ-ಧ್ಯಕ್ಷೋ
ಬುದ್ಧ್ಯೇತ
ಬ್ರಹ್ಮ
ಬ್ರಹ್ಮ-ಕೃತೇ
ಬ್ರಹ್ಮ-ಗ-ರ್ಭಾಯ
ಬ್ರಹ್ಮ-ಗರ್ಭೋ
ಬ್ರಹ್ಮ-ಚಾ-ರಿಣೇ
ಬ್ರಹ್ಮ-ಚಾರೀ
ಬ್ರಹ್ಮ-ಣಾ-ಮಪಿ
ಬ್ರಹ್ಮಣೇ
ಬ್ರಹ್ಮಣೋ
ಬ್ರಹ್ಮ-ದಂ-ಡ-ವಿ-ನಿ-ರ್ಮಾತಾ
ಬ್ರಹ್ಮ-ದಂ-ಡ-ವಿ-ನಿ-ರ್ಮಾತ್ರೇ
ಬ್ರಹ್ಮ-ಪ್ರೋಕ್ತಂ
ಬ್ರಹ್ಮ-ಪ್ರೋ-ಕ್ತೈ-ರ್ಪು-ಷಿ-ಪ್ರೋ-ಕ್ತೈ-ರ್ವೇ-ದ-ವೇ-ದಾಂ-ಗ-ಸಂ-ಭ-ವೈಃ
ಬ್ರಹ್ಮ-ಲೋಕಃ
ಬ್ರಹ್ಮ-ಲೋ-ಕಾ-ದಯಂ
ಬ್ರಹ್ಮ-ಲೋ-ಕಾಯ
ಬ್ರಹ್ಮ-ಲೋ-ಕಾ-ವ-ತಾ-ರಿ-ತೈಃ
ಬ್ರಹ್ಮ-ವ-ರ್ಚಸಃ
ಬ್ರಹ್ಮ-ವ-ರ್ಚ-ಸಾಯ
ಬ್ರಹ್ಮ-ವಿದೇ
ಬ್ರಹ್ಮ-ವಿ-ದ್ಬ್ರಾ-ಹ್ಮಣೋ
ಬ್ರಹ್ಮ-ಸ-ದ್ಮನಿ
ಬ್ರಹ್ಮಾ
ಬ್ರಹ್ಮಾ-ದಯೋ
ಬ್ರಹ್ಮಾ-ಧಿ-ಗ-ಚ್ಛತಿ
ಬ್ರಹ್ಮಿಣೇ
ಬ್ರಾಹ್ಮ-ಣಾಯ
ಭಕ್ತ-ವ-ತ್ಸ-ಲಾಯ
ಭಕ್ತಸ್ತ್ವಂ
ಭಕ್ತಾ-ನಾಂ
ಭಕ್ತಾ-ನು-ಕಂಪೀ
ಭಕ್ತಾಯ
ಭಕ್ತಿಃ
ಭಕ್ತಿ-ರ-ನನ್ಯಾ
ಭಕ್ತಿ-ರ-ವ್ಯ-ಭಿ-ಚಾ-ರಿಣೀ
ಭಕ್ತಿ-ರು-ತ್ಪ-ದ್ಯತೇ
ಭಕ್ತ್ಯಾ
ಭಕ್ತ್ಯಾ-ತ್ವೇವಂ
ಭಗ-ನೇ-ತ್ರ-ಭಿದೇ
ಭಗ-ವತೇ
ಭಗ-ವತೋ
ಭಗ-ವನ್
ಭಗ-ವಾ-ನಾ-ತ್ಮ-ಸಂ-ಸ್ಥಾ-ಕರೋ
ಭಗ-ವಾ-ನಾಹ
ಭಗ-ವಾ-ನಿತಿ
ಭಗ-ವಾ-ನು-ತ್ಥಾನಃ
ಭಗ-ವಾನ್
ಭಗ-ಹಾ-ರಿಣೇ
ಭಗ-ಹಾರೀ
ಭರ್ಗಾಯ
ಭವಂ
ಭವಂತಿ
ಭವಃ
ಭವಮ್
ಭವಾಯ
ಭವೇ
ಭಸ್ಮ-ಗೋಪ್ತಾ
ಭಸ್ಮ-ಗೋಪ್ತ್ರೇ
ಭಸ್ಮ-ಭೂ-ತ-ಸ್ತ-ರು-ರ್ಗಣಃ
ಭಸ್ಮ-ಭೂ-ತಾಯ
ಭಸ್ಮ-ಶ-ಯಾಯ
ಭಸ್ಮ-ಶಯೋ
ಭಸ್ಮೋ-ದ್ಧೂ-ಲಿ-ತ-ವಿ-ಗ್ರ-ಹಾಯ
ಭಾಗ-ಕ-ರಾಯ
ಭಾಗ-ಕರೋ
ಭಾಗಿನೇ
ಭಾಗೀ
ಭಾರ್ಗವಃ
ಭಾವಃ
ಭಾವಾಯ
ಭಾವಿನಃ
ಭಾವೇ-ನಾ-ಮಿ-ತ-ತೇ-ಜಸಃ
ಭಿಕ್ಷವೇ
ಭಿಕ್ಷು-ರೂ-ಪಶ್ಚ
ಭಿಕ್ಷು-ರೂ-ಪಾಯ
ಭಿಕ್ಷುಶ್ಚ
ಭೀಮಾಯ
ಭುಜಂ-ಗ-ಭೂ-ಷ-ಣಾಯ
ಭುಜಗಾ
ಭೂತ-ಚಾ-ರಿಣೇ
ಭೂತ-ಚಾರೀ
ಭೂತ-ನಿ-ಷೇ-ವಿ-ತಾಯ
ಭೂತ-ಪ-ತಯೇ
ಭೂತ-ಪ-ತಿ-ರ-ಹೋ-ರಾ-ತ್ರ-ಮ-ನಿಂ-ದಿತಃ
ಭೂತ-ಭಾ-ವನಃ
ಭೂತ-ಭಾ-ವ-ನಾಯ
ಭೂತ-ವಾ-ಹನ
ಭೂತ-ವಾ-ಹ-ನ-ಸಾ-ರ-ಥಿಃ
ಭೂತಾ-ಲ-ಯಾಯ
ಭೂತಾ-ಲಯೋ
ಭೂತ್ವಾ
ಭೂರ್ಲೋಕಂ
ಭೋಜನಃ
ಭೋಜ-ನಾಯ
ಮಂಗಲಂ
ಮಂಡ-ಸ್ತ-ಥೈ-ತ-ತ್ಸಾ-ರ-ಮು-ದ್ಧೃ-ತಮ್
ಮಂತ್ರಃ
ಮಂತ್ರ-ಕಾ-ರಾಯ
ಮಂತ್ರ-ಕಾರೋ
ಮಂತ್ರ-ವಿ-ತ್ಪ-ರಮೋ
ಮಂತ್ರ-ವಿದೇ
ಮಂತ್ರಾಯ
ಮಂಥಾ-ನ-ಬ-ಹು-ಲ-ವಾ-ಯವೇ
ಮಂಥಾನೋ
ಮಕರಃ
ಮಕ-ರಾಯ
ಮಣಿ-ವಿ-ದ್ಧಾಯ
ಮಣಿ-ವಿದ್ಧೋ
ಮತಿ-ಮ-ತಾಂ
ಮತಿ-ಮತೇ
ಮತಿ-ಮಾನ್
ಮದನಃ
ಮದ-ನಾಯ
ಮಧವೇ
ಮಧು
ಮಧು-ಕ-ಲೋ-ಚ-ನಾಯ
ಮಧು-ರ್ಮ-ಧು-ಕ-ಲೋ-ಚನಃ
ಮಧ್ಯ-ಮ-ಸ್ತಥಾ
ಮಧ್ಯ-ಮಾಯ
ಮನವೇ
ಮನಸಾ
ಮನು-ಜ-ವ್ಯಾಘ್ರ
ಮನು-ಷ್ಯಾಃ
ಮನು-ಷ್ಯಾ-ಣಾ-ಮೃತೇ
ಮನು-ಷ್ಯೇಷು
ಮನೋ-ಜವಃ
ಮನೋ-ಜ-ವಾಯ
ಮನೋ-ವೇ-ಗಾಯ
ಮನೋ-ವೇಗೋ
ಮಮ
ಮಯಾ
ಮಹತಃ
ಮಹತಾ
ಮಹತೇ
ಮಹತ್
ಮಹ-ದ್ಭಿ-ರ್ವಿ-ಹಿ-ತೈಃ
ಮಹ-ರ್ಷಯೇ
ಮಹ-ರ್ಷಿಶ್ಚ
ಮಹಾಂ-ಗಶ್ಚ
ಮಹಾಂ-ಗಾಯ
ಮಹಾಂ-ತ-ಕಾಯ
ಮಹಾಂ-ತಕೋ
ಮಹಾಂ-ಶ್ಚೈವ
ಮಹಾ-ಕಂ-ಬವೇ
ಮಹಾ-ಕಂ-ಬು-ರ್ಮ-ಹಾ-ಗ್ರೀವಃ
ಮಹಾ-ಕ-ರ್ಣಾಯ
ಮಹಾ-ಕರ್ಣೋ
ಮಹಾ-ಕರ್ತ್ರೇ
ಮಹಾ-ಕ-ರ್ಮಣೇ
ಮಹಾ-ಕರ್ಮಾ
ಮಹಾ-ಕ-ಲ್ಪಾಯ
ಮಹಾ-ಕಲ್ಪೋ
ಮಹಾ-ಕಾ-ಯಾಯ
ಮಹಾ-ಕಾಯೋ
ಮಹಾ-ಕೇ-ತವೇ
ಮಹಾ-ಕೇ-ತು-ರ್ಮ-ಹಾ-ಧಾ-ತು-ರ್ನೈ-ಕ-ಸಾ-ನು-ಚ-ರ-ಶ್ಚಲಃ
ಮಹಾ-ಕೇ-ಶಾಯ
ಮಹಾ-ಕೇಶೋ
ಮಹಾ-ಕ್ರೋಧ
ಮಹಾ-ಕ್ರೋ-ಧಾಯ
ಮಹಾ-ಕ್ಷಶ್ಚ
ಮಹಾ-ಕ್ಷಾಯ
ಮಹಾ-ಗ-ರ್ಭ-ಪ-ರಾ-ಯಣಃ
ಮಹಾ-ಗ-ರ್ಭ-ಪ-ರಾ-ಯ-ಣಾಯ
ಮಹಾ-ಗ-ರ್ಭ-ಶ್ಚಂ-ದ್ರ-ವ-ಕ್ತ್ರೋ-ಽನಿ-ಲೋ-ಽನಲಃ
ಮಹಾ-ಗ-ರ್ಭಾಯ
ಮಹಾ-ಗರ್ಭೋ
ಮಹಾ-ಗ-ಹ-ನಾಯ
ಮಹಾ-ಗೀ-ತಾಯ
ಮಹಾ-ಗೀತೋ
ಮಹಾ-ಗ್ರೀ-ವಾಯ
ಮಹಾ-ಘೋ-ರಾಯ
ಮಹಾ-ಘೋರೋ
ಮಹಾ-ಜಟಃ
ಮಹಾ-ಜ-ಟಾಯ
ಮಹಾ-ಜ-ತ್ರವೇ
ಮಹಾ-ಜ-ತ್ರು-ರ-ಲೋ-ಲಶ್ಚ
ಮಹಾ-ಜಿ-ಹ್ವಾಯ
ಮಹಾ-ಜಿಹ್ವೋ
ಮಹಾ-ಜ್ವಾ-ಲಾಯ
ಮಹಾ-ಜ್ವಾಲೋ
ಮಹಾ-ತ-ಪಸೇ
ಮಹಾ-ತಪಾ
ಮಹಾ-ತ-ಪಾಃ
ಮಹಾ-ತೇ-ಜಸೇ
ಮಹಾ-ತೇಜಾ
ಮಹಾ-ತ್ಮನಃ
ಮಹಾ-ತ್ಮನೇ
ಮಹಾತ್ಮಾ
ಮಹಾ-ದಂ-ತಾಯ
ಮಹಾ-ದಂತೋ
ಮಹಾ-ದಂ-ಷ್ಟ್ರಾಯ
ಮಹಾ-ದಂಷ್ಟ್ರೋ
ಮಹಾ-ದೇ-ವ-ಸ್ತು-ಷ್ಯತೇ
ಮಹಾ-ದೇ-ವಾಯ
ಮಹಾ-ದೇವೋ
ಮಹಾ-ಧ-ನುಃ
ಮಹಾ-ಧ-ನುಷೇ
ಮಹಾ-ಧಾ-ತವೇ
ಮಹಾ-ನ-ಖಾಯ
ಮಹಾ-ನಖೋ
ಮಹಾ-ನ-ನಾಯ
ಮಹಾ-ನಾ-ಗ-ಹನೋ
ಮಹಾ-ನಾದಃ
ಮಹಾ-ನಾ-ದಾಯ
ಮಹಾ-ನಾ-ಸಾಯ
ಮಹಾ-ನಾಸೋ
ಮಹಾ-ನೃ-ತ್ಯಾಯ
ಮಹಾ-ನೃತ್ಯೋ
ಮಹಾ-ನೇ-ತ್ರಾಯ
ಮಹಾ-ನೇತ್ರೋ
ಮಹಾನ್
ಮಹಾ-ಪಥಃ
ಮಹಾ-ಪ-ಥಾಯ
ಮಹಾ-ಪಾ-ದಾಯ
ಮಹಾ-ಪಾದೋ
ಮಹಾ-ಪ್ರ-ಸಾ-ದಾಯ
ಮಹಾ-ಪ್ರ-ಸಾದೋ
ಮಹಾ-ಬಲಃ
ಮಹಾ-ಬ-ಲಾಯ
ಮಹಾ-ಬಾಹೋ
ಮಹಾ-ಬೀ-ಜಾಯ
ಮಹಾ-ಬೀಜೋ
ಮಹಾ-ಮಾ-ತ್ರಾಯ
ಮಹಾ-ಮಾತ್ರೋ
ಮಹಾ-ಮಾಯಃ
ಮಹಾ-ಮಾ-ಯಾಯ
ಮಹಾ-ಮಾಲಃ
ಮಹಾ-ಮಾ-ಲಾಯ
ಮಹಾ-ಮುಖಃ
ಮಹಾ-ಮು-ಖಾಯ
ಮಹಾ-ಮು-ನಯೇ
ಮಹಾ-ಮು-ನಿಃ
ಮಹಾ-ಮೂರ್ಧಾ
ಮಹಾ-ಮೂರ್ಧ್ನೇ
ಮಹಾ-ಮೇ-ಘ-ನಿ-ವಾ-ಸಿನೇ
ಮಹಾ-ಮೇ-ಘ-ನಿ-ವಾಸೀ
ಮಹಾ-ಯ-ಶಸೇ
ಮಹಾ-ಯ-ಶಾಃ
ಮಹಾ-ಯುಧಃ
ಮಹಾ-ಯು-ಧಾಯ
ಮಹಾ-ರಥಃ
ಮಹಾ-ರ-ಥಾಯ
ಮಹಾ-ರೂ-ಪಾಯ
ಮಹಾ-ರೂಪೋ
ಮಹಾ-ರೇ-ತಸೇ
ಮಹಾ-ರೇತಾ
ಮಹಾ-ರೋಮಾ
ಮಹಾ-ರೋಮ್ಣೇ
ಮಹಾ-ರ್ಣ-ವ-ನಿ-ಪಾ-ನ-ವಿತ್
ಮಹಾ-ರ್ಣ-ವ-ನಿ-ಪಾ-ನ-ವಿದೇ
ಮಹಾ-ಲಿಂ-ಗ-ಶ್ಚಾ-ರು-ಲಿಂ-ಗ-ಸ್ತ-ಥೈವ
ಮಹಾ-ಲಿಂ-ಗಾಯ
ಮಹಾ-ವ-ಕ್ಷಸೇ
ಮಹಾ-ವಕ್ಷಾ
ಮಹಾ-ವೇ-ಗಾಯ
ಮಹಾ-ವೇಗೋ
ಮಹಾ-ಸೇ-ನ-ಜ-ನ-ಕಾಯ
ಮಹಾ-ಸೇ-ನಾಯ
ಮಹಾ-ಸೇನೋ
ಮಹಾ-ಹ-ನವೇ
ಮಹಾ-ಹ-ನುಃ
ಮಹಾ-ಹ-ರ್ಷಾಯ
ಮಹಾ-ಹ-ಸ್ತಾಯ
ಮಹಾ-ಹಸ್ತೋ
ಮಹಾ-ಽನನಃ
ಮಹಿ-ಚಾ-ರಿಣೇ
ಮಹೀ-ಚಾರೀ
ಮಹೇಶಂ
ಮಹೇ-ಶ್ವರಃ
ಮಹೇ-ಶ್ವ-ರಮ್
ಮಹೇ-ಶ್ವ-ರಾಯ
ಮಹೋ-ರ-ಸ್ಕಾಯ
ಮಹೋ-ರಸ್ಕೋ
ಮಹೋ-ಷ್ಠಶ್ಚ
ಮಹೋ-ಷ್ಠಾಯ
ಮಹೌ-ಷಧಃ
ಮಹೌ-ಷ-ಧಾಯ
ಮಾಂಗಲ್ಯಂ
ಮಾಂಧಾತಾ
ಮಾಂಧಾತ್ರೇ
ಮಾಡ-ಬೇಕು
ಮಾತಾ
ಮಾತ್ರ
ಮಾತ್ರಾಭ್ಯೋ
ಮಾತ್ರೇ
ಮಾಧವ
ಮಾನ್ಯಾಯ
ಮಾನ್ಯೋ
ಮಾಯಾ-ವಿನೇ
ಮಾಯಾವೀ
ಮಾರ್ಕಂ-ಡೇ-ಯಾ-ನ್ಮಯಾ
ಮಾರ್ಕಂ-ಡೇ-ಯಾಯ
ಮಾಲಿನೇ
ಮಾಲೀ
ಮಾಸಃ
ಮಾಸಾಯ
ಮಿತ್ರ-ಸ್ತ್ವ-ಷ್ಟಾ-ಧ್ರುವೋ
ಮಿತ್ರಾಯ
ಮುಂಡಃ
ಮುಂಡಾಯ
ಮುಂಡಿನೇ
ಮುಂಡೀ
ಮುಂಡೋ
ಮುಕ್ತ-ತೇ-ಜಸೇ
ಮುಕ್ತ-ತೇ-ಜಾಶ್ಚ
ಮುಖ್ಯಾಯ
ಮುಖ್ಯೋ-ಽಮು-ಖ್ಯಶ್ಚ
ಮುದಿ-ತಾಯ
ಮುದಿ-ತೋ-ಽರ್ಧೋ-ಽಜಿ-ತೋ-ಽವರಃ
ಮುದ್ರ-ಕರು
ಮುದ್ರಣ
ಮುನಯೇ
ಮುನಿ-ರಾ-ತ್ಮ-ನಿ-ರಾ-ಲೋಕಃ
ಮುಹೂ-ರ್ತಾ-ಹಃ-ಕ್ಷ-ಪಾಃ
ಮುಹೂ-ರ್ತಾ-ಹ-ಕ್ಷ-ಪಾಭ್ಯೋ
ಮೂರ್ತಿ-ಜಾಯ
ಮೂರ್ತಿಜೋ
ಮೂರ್ಧ-ಗಾಯ
ಮೂರ್ಧಗೋ
ಮೂಲಂ
ಮೂಲಾಯ
ಮೃಗ-ಪಾ-ಣಯೇ
ಮೃಗ-ಬಾ-ಣಾ-ರ್ಪ-ಣಾಯ
ಮೃಗ-ಬಾ-ಣಾ-ರ್ಪ-ಣೋ-ಽನಘಃ
ಮೃಗಾ-ಲಯಃ
ಮೃಗಾ-ಲ-ಯಾಯ
ಮೃಡಾಯ
ಮೃತ್ಯವೇ
ಮೃತ್ಯುಂ-ಜ-ಯಾಯ
ಮೃತ್ಯುಃ
ಮೃದವೇ
ಮೃದು-ರ-ವ್ಯಯಃ
ಮೇ
ಮೇಢ್ರ-ಜಾಯ
ಮೇಢ್ರಜೋ
ಮೇಧ್ಯಂ
ಮೇರು-ಧಾಮಾ
ಮೇರು-ಧಾಮ್ನೇ
ಮೇಲೆ
ಮೈಸೂರು
ಮೊದ-ಲ-ನೆಯ
ಮೊದಲು
ಮೋಕ್ಷದಂ
ಮೋಕ್ಷ-ದ್ವಾರಂ
ಮೋಕ್ಷ-ದ್ವಾ-ರಾಯ
ಯ
ಯಂ
ಯಃ
ಯಕ್ಷ-ರಾ-ಕ್ಷ-ಸಾಃ
ಯಚ್ಛ್ರುತ್ವಾ
ಯಜುಃ-ಪಾ-ದ-ಭು-ಜಾಯ
ಯಜುಃ-ಪಾ-ದ-ಭುಜೋ
ಯಜ್ಞಃ
ಯಜ್ಞಘ್ನೇ
ಯಜ್ಞ-ಪ-ತಿ-ರ್ವಿ-ಭುಃ
ಯಜ್ಞ-ಭಾ-ಗ-ವಿದೇ
ಯಜ್ಞ-ಮ-ಯಾಯ
ಯಜ್ಞ-ವಿ-ಭಾ-ಗ-ವಿತ್
ಯಜ್ಞ-ಸ-ಮಾ-ಹಿತಃ
ಯಜ್ಞ-ಸ-ಮಾ-ಹಿ-ತಾಯ
ಯಜ್ಞಹಾ
ಯಜ್ಞಾ-ನಾ-ಮಪಿ
ಯಜ್ಞಾಯ
ಯಜ್ಞೋ
ಯತಃ
ಯತ-ಸ್ತಂ-ಡಿಃ
ಯತೋ
ಯತ್ತ-ದ್ಬ್ರಹ್ಮ
ಯತ್ತಪಃ
ಯತ್ತೇ-ಜ-ಸ್ತ-ಪ-ಸಾ-ಮಪಿ
ಯತ್ಪ-ರಮ್
ಯಥಾ
ಯಥಾ-ಪ್ರ-ಧಾನಂ
ಯಥೋ-ಕ್ತೈಃ
ಯದಾ
ಯದು-ಕು-ಲ-ಶ್ರೇಷ್ಠ
ಯದುಕ್ತಂ
ಯದ್ಬ್ರಹ್ಮ
ಯಧಿ-ಷ್ಠಿರ
ಯನ್ನ
ಯಮಃ
ಯಮಾಯ
ಯಶಃ
ಯಶಸೇ
ಯಶ್ಚಾ-ಭ್ಯ-ಸೂ-ಯತೇ
ಯಸ್ತ್ವಿ
ಯಸ್ಯಾ-ದಿ-ರ್ಮ-ಧ್ಯ-ಮಂತಂ
ಯಾ
ಯಾಂತಿ
ಯಾತಿ
ಯಾತು-ಧಾನಾ
ಯಾದ-ವ-ಗಿರಿ
ಯಾನ್ಯಾಹ
ಯುಕ್ತ-ಬಾ-ಹವೇ
ಯುಕ್ತ-ಬಾ-ಹುಶ್ಚ
ಯುಕ್ತಶ್ಚ
ಯುಕ್ತಾಯ
ಯುಕ್ತೇ-ನಾಪಿ
ಯುಗ-ಕ-ರಾಯ
ಯುಗ-ಕರೋ
ಯುಗ-ರೂ-ಪಾಯ
ಯುಗ-ರೂಪೋ
ಯುಗಾ-ಧಿಪಃ
ಯುಗಾ-ಧಿ-ಪಾಯ
ಯುಗಾ-ವಹಃ
ಯುಗಾ-ವ-ಹಾಯ
ಯುಧಿ
ಯುಧಿ-ಶ-ತ್ರು-ವಿ-ನಾ-ಶ-ನಾಯ
ಯೇ
ಯೇನ
ಯೈ
ಯೋ
ಯೋಗದಂ
ಯೋಗ-ಮಿದಂ
ಯೋಗಾ-ಧ್ಯ-ಕ್ಷಾಯ
ಯೋಗಾ-ಧ್ಯಕ್ಷೋ
ಯೋಗಾಯ
ಯೋಗಿ-ನಾ-ಮಪಿ
ಯೋಗಿನೇ
ಯೋಗೀ
ಯೋಗೋ
ಯೋಜ್ಯಾಯ
ಯೋಜ್ಯೋ
ಯೋಽಸ-ಮಾ-ಮ್ನಾ-ಯ-ಸ್ತೀ-ರ್ಥ-ದೇವೋ
ರಕ್ಷೋಘ್ನಂ
ರಜ-ತ-ಗಿ-ರಿ-ನಿಭಂ
ರತಯೇ
ರತಿ-ರ್ನರಃ
ರತ್ನ-ಪ್ರ-ಭೂ-ತಾಯ
ರತ್ನ-ಪ್ರ-ಭೂತೋ
ರತ್ನಾಂ-ಗಾಯ
ರತ್ನಾಂಗೋ
ರತ್ನಾ-ಕ-ಲ್ಪೋ-ಜ್ಜ್ವ-ಲಾಂಗಂ
ರಥ-ಯೋಗೀ
ರಮಂತಿ
ರವಯೇ
ರವಿಃ
ರಹ-ಸ್ಯ-ಮಿ-ದ-ಮು-ತ್ತ-ಮಮ್
ರಾಜತ್ವೇ
ರಾಜ-ರಾ-ಜಾಯ
ರಾಜ-ರಾಜೋ
ರುದ್ರಃ
ರುದ್ರಸ್ಯ
ರುದ್ರಾ-ಣಾ-ಮಪಿ
ರುದ್ರಾಯ
ರುದ್ರೇ
ರುದ್ರೇ-ಭ್ಯ-ಸ್ತಂ-ಡಿ-ಮಾ-ಗ-ಮತ್
ರುದ್ರೇಭ್ಯೋ
ರೌದ್ರ-ರೂ-ಪಾಯ
ರೌದ್ರ-ರೂ-ಪೋ-ಽಂಶು-ರಾ-ದಿತ್ಯೋ
ಲಂಬ-ನಾಯ
ಲಂಬನೋ
ಲಂಬಿ-ತೋ-ಷ್ಠಶ್ಚ
ಲಂಬಿ-ತೋ-ಷ್ಠಾಯ
ಲಘವೇ
ಲಘುಃ
ಲಭೇತ್
ಲಭ್ಯತೇ
ಲಯಾಯ
ಲಲಾ-ಟಾ-ಕ್ಷಾಯ
ಲಲಾ-ಟಾಕ್ಷೋ
ಲವಣಃ
ಲವಾ
ಲವೇಭ್ಯೋ
ಲಿಂಗ-ಮಾ-ದ್ಯಸ್ತು
ಲಿಂಗಾ-ಧ್ಯಕ್ಷಃ
ಲಿಂಗಾ-ಧ್ಯ-ಕ್ಷಾಯ
ಲಿಂಗಾಯ
ಲೋಕಃ
ಲೋಕ-ಕರ್ತಾ
ಲೋಕ-ಕರ್ತ್ರೇ
ಲೋಕ-ಚಾ-ರಿಣೇ
ಲೋಕ-ಚಾರೀ
ಲೋಕ-ಧಾತಾ
ಲೋಕ-ಧಾತ್ರೇ
ಲೋಕ-ಪಾ-ಲ-ಸ್ತ-ಥಾ-ಲೋಕೋ
ಲೋಕ-ಪಾ-ಲಾಯ
ಲೋಕ-ಪಾ-ಲೋ-ಽಂತ-ರ್ಹಿ-ತಾತ್ಮಾ
ಲೋಕ-ಹಿ-ತ-ಸ್ತ-ರುಃ
ಲೋಕ-ಹಿ-ತಾಯ
ಲೋಕಾಃ
ಲೋಹಿ-ತಾ-ಕ್ಷಾಯ
ಲೋಹಿ-ತಾಕ್ಷೋ
ವಂದ್ಯಂ
ವಂಶ-ಕ-ರಾಯ
ವಂಶ-ಕರೋ
ವಂಶ-ನಾ-ದಾಯ
ವಂಶ-ನಾದೋ
ವಂಶಾಯ
ವಂಶೋ
ವಕ್ತುಂ
ವಕ್ಷ್ಯಾ-ಮ್ಯ-ವ್ಯ-ಕ್ತ-ಯೋ-ನಿನಃ
ವಕ್ಷ್ಯೇ
ವಜ್ರ-ಹ-ಸ್ತಶ್ಚ
ವಜ್ರ-ಹ-ಸ್ತಾಯ
ವಜ್ರಿಣೇ
ವಜ್ರೀ
ವಡ-ವಾ-ಮುಖಃ
ವಡ-ವಾ-ಮು-ಖಾಯ
ವಣಿ-ಜಾಯ
ವಣಿಜೋ
ವಧಃ
ವಧಾಯ
ವಪು-ರಾ-ವ-ರ್ತ-ಮಾ-ನೇಭ್ಯೋ
ವರಃ
ವರ-ದಸ್ಯ
ವರ-ದಾಯ
ವರದೋ
ವರ-ಪ್ರ-ದಾಯ
ವರ-ಯೈನಂ
ವರ-ವ-ರಾ-ಹಾಯ
ವರಾಯ
ವರಾಹೋ
ವರೇಣ್ಯಃ
ವರೇ-ಣ್ಯಸ್ಯ
ವರೇ-ಣ್ಯಾಯ
ವರೋ
ವರ್ಚ-ಸ್ವಿನೇ
ವರ್ಚಸ್ವೀ
ವರ್ಣ-ವಿ-ಭಾ-ವಿನೇ
ವರ್ಧ-ಕಿನೇ
ವರ್ಧಕೀ
ವರ್ಧನೋ
ವರ್ಷಂ
ವರ್ಷ-ಶ-ತೈ-ರಪಿ
ವಶ-ಕರಃ
ವಶ-ಕ-ರಾಯ
ವಶಿನೇ
ವಶೀ
ವಶ್ಯಾಯ
ವಶ್ಯೋ
ವಸಾನಂ
ವಸು-ವೇ-ಗಾಯ
ವಸು-ವೇಗೋ
ವಸು-ಶ್ರೇ-ಷ್ಠಾಯ
ವಸು-ಶ್ರೇಷ್ಠೋ
ವಾ
ವಾಚ-ಸ್ಪ-ತ್ಯಾಯ
ವಾಚ-ಸ್ಪತ್ಯೋ
ವಾಚಾ
ವಾಜ-ಸ-ನಾಯ
ವಾಜ-ಸನೋ
ವಾತಃ
ವಾತ-ರಂ-ಹಸೇ
ವಾತಾಯ
ವಾಮ-ದೇ-ವಶ್ಚ
ವಾಮ-ದೇ-ವಾಯ
ವಾಮನಃ
ವಾಮ-ನಾಯ
ವಾಮಶ್ಚ
ವಾಮಾಯ
ವಾಯುಃ
ವಾಯು-ರ-ರ್ಯಮಾ
ವಾಯು-ವಾ-ಹನಃ
ವಾಯು-ವಾ-ಹ-ನಾಯ
ವಾರ್ಷ್ಣೇಯ
ವಾಸ-ವಾಯ
ವಾಸ-ವೋ-ಽಮರಃ
ವಾಸು-ದೇವ
ವಾಹಿತಾ
ವಿಕು-ರ್ವಂತಿ
ವಿಕು-ರ್ವಣಃ
ವಿಕು-ರ್ವಾ-ಣಾಯ
ವಿಕೃ-ತಾಯ
ವಿಕೃತೋ
ವಿಖ್ಯಾತಂ
ವಿಖ್ಯಾ-ತ-ಲೋ-ಕಾಯ
ವಿಖ್ಯಾತೋ
ವಿಗ್ರ-ಹ-ಮ-ತಯೇ
ವಿಗ್ರ-ಹ-ಮ-ತಿ-ರ್ಗು-ಣ-ಬು-ದ್ಧಿ-ರ್ಲ-ಯೋ-ಽಗಮಃ
ವಿಘ್ನಂ
ವಿಚಾ-ರ-ವಿತ್
ವಿಚಾ-ರ-ವಿದೇ
ವಿಜ-ಯ-ಕಾ-ಲ-ವಿತ್
ವಿಜ-ಯ-ಕಾ-ಲ-ವಿದೇ
ವಿಜ-ಯಾ-ಕ್ಷಾಯ
ವಿಜ-ಯಾಕ್ಷೋ
ವಿಜ-ಯಾಯ
ವಿಜಯೋ
ವಿದುಷೇ
ವಿದು-ಸ್ತ-ತ್ವೇನ
ವಿದ್ವಾನ್
ವಿಧಾತಾ
ವಿಧಾತ್ರೇ
ವಿನ-ತ-ಸ್ತಥಾ
ವಿನ-ತಾಯ
ವಿನಿ-ಯೋಗಃ
ವಿನಿ-ರ್ಮಾತ್ರೇ
ವಿಪ-ಣಾಯ
ವಿಪಣೋ
ವಿಪ್ರ-ರ್ಷಿ-ರ್ನಾ-ಮ-ಸಂ-ಗ್ರ-ಹ-ಮಾ-ದಿತಃ
ವಿಬು-ಧಾಯ
ವಿಬು-ಧೋ-ಽಗ್ರ-ವರಃ
ವಿಭವೇ
ವಿಭಾ-ಗ-ಜ್ಞಾಯ
ವಿಭಾ-ಗ-ಜ್ಞೋ-ಽತುಲ್ಯೋ
ವಿಭಾ-ಗಾಯ
ವಿಭುಃ
ವಿಭು-ರ್ಭವಃ
ವಿಭು-ರ್ವ-ರ್ಣ-ವಿ-ಭಾವೀ
ವಿಭೂ-ತೀ-ನಾ-ಮಪಿ
ವಿಮ-ರ್ಶ-ಶಿ-ರೋ-ಹಾ-ರಿಣೇ
ವಿಮ-ರ್ಶಶ್ಚ
ವಿಮು-ಕ್ತಾಯ
ವಿಮುಕ್ತೋ
ವಿಮೋ-ಚನಃ
ವಿಮೋ-ಚ-ನಾಯ
ವಿರ-ಜಾಯ
ವಿರಜೋ
ವಿರಾಮಃ
ವಿರಾ-ಮಾಯ
ವಿರೂ-ಪಾ-ಕ್ಷಾಯ
ವಿರೂ-ಪಾಯ
ವಿರೂಪೋ
ವಿವ-ರ-ಗಳನ್ನು
ವಿವ-ರಿ-ಸುವ
ವಿವ-ಸ್ವ-ತ್ಸ-ವಿ-ತ್ರ-ಮೃ-ತಾಯ
ವಿವ-ಸ್ವಾನ್
ವಿಶಾಂ-ಪ-ತಯೇ
ವಿಶಾಂ-ಪ-ತಿಃ
ವಿಶಾ-ಖಶ್ಚ
ವಿಶಾ-ಖಾಯ
ವಿಶಾ-ರದಃ
ವಿಶಾ-ರ-ದಾಯ
ವಿಶಾ-ಲ-ಶಾ-ಖ-ಸ್ತಾ-ಮ್ರೋಷ್ಠೋ
ವಿಶಾ-ಲ-ಶಾ-ಖಾಯ
ವಿಶಾ-ಲಾಕ್ಷಃ
ವಿಶಾ-ಲಾ-ಕ್ಷಾಯ
ವಿಶಾ-ಲಾಯ
ವಿಶಾಲೋ
ವಿಶ್ವ-ಕ-ರ್ಮ-ಮ-ತಯೇ
ವಿಶ್ವ-ಕ-ರ್ಮ-ಮ-ತಿ-ರ್ವರಃ
ವಿಶ್ವ-ಕ್ಷೇತ್ರಂ
ವಿಶ್ವ-ಕ್ಷೇ-ತ್ರಾಯ
ವಿಶ್ವ-ದೇವಃ
ವಿಶ್ವ-ದೇ-ವಾಯ
ವಿಶ್ವ-ದೇವೋ
ವಿಶ್ವ-ಬಾ-ಹವೇ
ವಿಶ್ವ-ರೂಪಃ
ವಿಶ್ವ-ರೂ-ಪಸ್ಯ
ವಿಶ್ವ-ರೂ-ಪಾಯ
ವಿಶ್ವ-ರೂಪೋ
ವಿಶ್ವ-ವಂದ್ಯಂ
ವಿಶ್ವಾದ್ಯಂ
ವಿಶ್ವಾಯ
ವಿಶ್ವೇ-ಶ್ವ-ರಾಯ
ವಿಶ್ವೋ
ವಿಷ-ಣ್ಣಾಂ-ಗಾಯ
ವಿಷ-ಣ್ಣಾಂಗೋ
ವಿಷ್ಕಂ-ಭಿನೇ
ವಿಷ್ಕಂಭೀ
ವಿಷ್ಣವೇ
ವಿಷ್ಣು-ಪ್ರ-ಸಾ-ದಿ-ತಾಯ
ವಿಷ್ಣು-ಪ್ರ-ಸಾ-ದಿತೋ
ವಿಷ್ಣು-ವ-ಲ್ಲ-ಭಾಯ
ವಿಷ್ಣುಶ್ಚ
ವಿಷ್ವ-ಕ್ಸೇ-ನಾಯ
ವಿಷ್ವ-ಕ್ಸೇನೋ
ವಿಸ-ರ್ಗಾಯ
ವಿಸ್ತರಃ
ವಿಸ್ತ-ರಾತ್
ವಿಸ್ತಾ-ರ-ಲ-ವ-ಣ-ಕೂ-ಪಾಯ
ವಿಸ್ತಾರೋ
ವೀರ-ಭ-ದ್ರಾಯ
ವೃಕ್ಷ-ಕ-ರ್ಣ-ಸ್ಥಿ-ತಯೇ
ವೃಕ್ಷ-ಕ-ರ್ಣ-ಸ್ಥಿ-ತಿ-ರ್ವಿ-ಭುಃ
ವೃಕ್ಷ-ಕೇ-ತವೇ
ವೃಕ್ಷ-ಕೇ-ತು-ರ-ನಲೋ
ವೃಕ್ಷಾ-ಕಾ-ರಾಯ
ವೃಕ್ಷಾ-ಕಾರೋ
ವೃಕ್ಷಾಯ
ವೃಕ್ಷೋ
ವೃತ್ತಾ-ವೃ-ತ್ತ-ಕ-ರ-ಸ್ತಾಲೋ
ವೃತ್ತಾ-ವೃ-ತ್ತ-ಕ-ರಾಯ
ವೃದ್ಧಾಯ
ವೃದ್ಧೋ
ವೃಷಣಃ
ವೃಷ-ಣಾಯ
ವೃಷ-ಭಾ-ರೂ-ಢಾಯ
ವೃಷ-ರೂ-ಪಾಯ
ವೃಷ-ರೂಪೋ
ವೃಷಾಂ-ಕಾಯ
ವೇಣ-ವಿನೇ
ವೇಣವೀ
ವೇದ-ಕಾ-ರಾಯ
ವೇದ-ಕಾರೋ
ವೇದೇನ
ವೈ
ವೈದಂ-ಭಾಯ
ವೈದಂಭೋ
ವೈದ್ಯಾಯ
ವೈದ್ಯೋ
ವೈವ-ಸ್ವ-ತಾಯ
ವೈವ-ಸ್ವತೋ
ವೈಶ್ರ-ವ-ಣ-ಸ್ತಥಾ
ವೈಶ್ರ-ವ-ಣಾಯ
ವ್ಯಕ್ತಾಯ
ವ್ಯಕ್ತಾ-ವ್ಯ-ಕ್ತ-ಸ್ತ-ಪೋ-ನಿ-ಧಿಃ
ವ್ಯಕ್ತಾ-ವ್ಯ-ಕ್ತಾಯ
ವ್ಯವ-ಸಾ-ಯಾಯ
ವ್ಯವ-ಸಾಯೋ
ವ್ಯಾಘ್ರಾಯ
ವ್ಯಾಘ್ರೋ
ವ್ಯಾಲ-ರೂ-ಪಾಯ
ವ್ಯಾಲ-ರೂಪೋ
ವ್ಯಾಸಃ
ವ್ಯಾಸಾಯ
ವ್ಯೋಮ-ಕೇ-ಶಾಯ
ವ್ರಜಂ-ತ್ಯೇ-ತಾಂ
ವ್ರಜ-ನ್ನು-ಪ-ವಿ-ಶಂ-ಸ್ತಥಾ
ವ್ರತಾ-ಧಿಪಃ
ವ್ರತಾ-ಧಿ-ಪಾಯ
ಶಂಕರಃ
ಶಂಕ-ರಮ್
ಶಂಕ-ರ-ಸಂ-ನಿಧೌ
ಶಂಕ-ರಾಯ
ಶಂಕರೋ
ಶಂಕ-ರೋ-ಽಧನಃ
ಶಂಭವೇ
ಶಕ್ತಿ
ಶಕ್ತಿಃ
ಶಕ್ತಿ-ತ-ಶ್ಚ-ರಿತಂ
ಶಕ್ತಿ-ಸ್ತ-ಪೋ-ಬ-ಲಮ್
ಶಕ್ನು-ಯಾ-ದ್ವ-ಕ್ತುಂ
ಶಕ್ಯಂ
ಶಕ್ಯಃ
ಶಕ್ರಃ
ಶಕ್ರಶ್ಚ
ಶಕ್ರಾಯ
ಶತ-ಘ್ನಿನೇ
ಶತಘ್ನೀ
ಶತ-ಘ್ನೀ-ಪಾ-ಶ-ಶ-ಕ್ತಿ-ಮತೇ
ಶತ-ಘ್ನೀ-ಪಾ-ಶ-ಶ-ಕ್ತಿ-ಮಾನ್
ಶತ-ಜಿಹ್ವಃ
ಶತ-ಜಿ-ಹ್ವಾಯ
ಶತ್ರುಘ್ನೇ
ಶತ್ರು-ವಿ-ನಾ-ಶನಃ
ಶತ್ರುಹಾ
ಶನಯೇ
ಶನಿಃ
ಶಯಾನಾ
ಶರಃ
ಶರಣ್ಯಃ
ಶರ-ಣ್ಯಾಯ
ಶರಾಯ
ಶರ್ವಃ
ಶರ್ವಸ್ಯ
ಶರ್ವಾಯ
ಶಶಿ-ಶೇ-ಖ-ರಾಯ
ಶಶಿ-ಹರ
ಶಶೀ
ಶಾಂತಿ-ರ್ದ್ಯು-ತೀ-ನಾ-ಮಪಿ
ಶಾಂತೀ-ನಾ-ಮಪಿ
ಶಾಶ್ವ-ತಾಯ
ಶಾಶ್ವತೋ
ಶಿಖಂ-ಡಿನೇ
ಶಿಖಂಡೀ
ಶಿಖಿನೇ
ಶಿಖೀ
ಶಿತಿ-ಕಂ-ಠಾಯ
ಶಿಪಿ-ವಿ-ಷ್ಟಾಯ
ಶಿರೋ-ಹಾರೀ
ಶಿವಃ
ಶಿವ-ಪೂ-ಜಾದಿ
ಶಿವ-ಪೂ-ಜೆ-ಯಲ್ಲಿ
ಶಿವ-ಮೇ-ಭಿಃ
ಶಿವ-ಸ-ಹ-ಸ್ರ-ನಾಮ
ಶಿವ-ಸ-ಹ-ಸ್ರ-ನಾ-ಮ-ಸ್ತೋ-ತ್ರಮ್
ಶಿವಾ-ನಾ-ಮಪಿ
ಶಿವಾ-ಪ್ರಿ-ಯಾಯ
ಶಿವಾಯ
ಶೀಲ-ಧಾ-ರಿಣೇ
ಶೀಲ-ಧಾರೀ
ಶುಕ್ರಾಯ
ಶುಕ್ಲ
ಶುಕ್ಲ-ಸ್ತ್ರಿ-ಶುಕ್ಲಃ
ಶುಕ್ಲಾಯ
ಶುಚಯೇ
ಶುಚಿಃ
ಶುಚಿ-ರ್ಭಕ್ತಃ
ಶುಚಿ-ರ್ಭೂ-ತ-ನಿ-ಷೇ-ವಿತಃ
ಶುದ್ಧ-ವಿ-ಗ್ರ-ಹಾಯ
ಶುದ್ಧಾ-ತ್ಮನೇ
ಶುದ್ಧಾತ್ಮಾ
ಶುದ್ಧಾಯ
ಶುದ್ಧೋ
ಶುಭ-ಬು-ದ್ಧಿನಾ
ಶುಭಮ್
ಶುಭಾ-ಕ್ಷಾಯ
ಶುಭಾಕ್ಷೋ
ಶೂಲ-ಪಾ-ಣಯೇ
ಶೃಂಗ-ಪ್ರಿ-ಯಾಯ
ಶೃಂಗ-ಪ್ರಿಯೋ
ಶೃಂಗಿಣೇ
ಶೃಂಗೀ
ಶೃಗಾ-ಲ-ರೂಪಃ
ಶೃಗಾ-ಲ-ರೂ-ಪಾಯ
ಶೃಣು
ಶೃಣು-ಷ್ವಾ-ವ-ಹಿತೋ
ಶೃಣ್ವಂತಃ
ಶೋಭ-ನಾಯ
ಶೋಭನೋ
ಶ್ಮಶಾ-ನ-ಭಾಕ್
ಶ್ಮಶಾ-ನ-ಭಾಜೇ
ಶ್ಮಶಾ-ನ-ವಾ-ಸಿನೇ
ಶ್ಮಶಾ-ನ-ವಾಸೀ
ಶ್ರದ್ದ-ಧಾ-ನಾಶ್ಚ
ಶ್ರದ್ಧ-ಧಾ-ನಾ-ಸ್ತಿ-ಕಾಯ
ಶ್ರಾವ-ಯಂ-ತಶ್ಚ
ಶ್ರಾವ-ಯಿ-ಷ್ಯಾಮಿ
ಶ್ರಿಯಾ-ವಾ-ಸಿನೇ
ಶ್ರಿಯಾ-ವಾಸೀ
ಶ್ರೀ
ಶ್ರೀಕಂ-ಠಾಯ
ಶ್ರೀಮತೇ
ಶ್ರೀಮಾ-ನ್ಶ್ರೀ-ವ-ರ್ಧನೋ
ಶ್ರೀರಾ-ಮ-ಕೃಷ್ಣ
ಶ್ರೀವ-ರ್ಧ-ನಾಯ
ಶ್ರೀಶಿ-ವ-ಸ-ಹ-ಸ್ರ-ನಾ-ಮ-ಸ್ತೋ-ತ್ರಮ್
ಶ್ರೀಶಿ-ವ-ಸ-ಹ-ಸ್ರ-ನಾ-ಮಾ-ವ-ಲಿಃ
ಶ್ರೀಶಿ-ವಾ-ಷ್ಟೋ-ತ್ತ-ರ-ಶ-ತ-ನಾ-ಮಾ-ವ-ಲಿಃ
ಶ್ರೀಸ-ದಾ-ಶಿ-ವ-ಸ-ಹ-ಸ್ರ-ನಾ-ಮ-ಸ್ತೋ-ತ್ರ-ಮಂ-ತ್ರಸ್ಯ
ಶ್ರುತೈಃ
ಶ್ರೇಷ್ಠೋ
ಶ್ವೇತ-ಪಿಂ-ಗಲಃ
ಶ್ವೇತ-ಪಿಂ-ಗ-ಲಾಯ
ಷಷ್ಟಿ-ಭಾ-ಗಾಯ
ಷಷ್ಟಿ-ಭಾಗೋ
ಸ
ಸಂಕ್ಷಿ-ಪ್ತಾ-ರ್ಥ-ಪ-ದಾ-ಕ್ಷ-ರಮ್
ಸಂಖ್ಯಾ-ಸ-ಮಾ-ಪನಃ
ಸಂಖ್ಯಾ-ಸ-ಮಾ-ಪ-ನಾಯ
ಸಂಗ್ರ-ಹಾಯ
ಸಂಗ್ರಹೋ
ಸಂನಿಧೌ
ಸಂಪನ್ನಃ
ಸಂಪ-ನ್ನಾಯ
ಸಂಪ್ರಾಪ್ಯ
ಸಂಭ-ಗ್ನಶ್ಚ
ಸಂಭ-ಗ್ನಾಯ
ಸಂಭ-ವಂತಿ
ಸಂಮಿ-ತಮ್
ಸಂಯ-ತಾಯ
ಸಂಯತೋ
ಸಂಯು-ಗಾ-ಪೀ-ಡ-ವಾ-ಹನಃ
ಸಂಯು-ಗಾ-ಪೀ-ಡ-ವಾ-ಹ-ನಾಯ
ಸಂಯೋ-ಗ-ವ-ರ್ಧ-ನಾಯ
ಸಂಯೋಗೋ
ಸಂವ-ತ್ಸ-ರ-ಕ-ರಾಯ
ಸಂವ-ತ್ಸ-ರ-ಕರೋ
ಸಂವ-ತ್ಸ-ರಾಯ
ಸಂವ-ತ್ಸರೋ
ಸಂಸಾ-ರ-ಮೋ-ಚ-ನಮ್
ಸಂಸಾ-ರಾ-ತ್ತಾನ್
ಸಂಸ್ಥಿ-ತಮ್
ಸಕಲಃ
ಸಕ-ಲಾಯ
ಸಕ-ಲ್ಪಶ್ಚ
ಸಕ-ಲ್ಪಾಯ
ಸಕಾ-ಮಾ-ರಯೇ
ಸಕಾ-ಮಾ-ರಿ-ರ್ಮ-ಹಾ-ದಂಷ್ಟ್ರೋ
ಸಕ್ತಾಯ
ಸಗ-ಣಾಯ
ಸಗಣೋ
ಸತೇ
ಸತ್ಕೃ-ತಶ್ಚ
ಸತ್ಕೃ-ತಾಯ
ಸತ್ಯ-ವ್ರತಃ
ಸತ್ಯ-ವ್ರ-ತಾಯ
ಸತ್ಯೈಃ
ಸತ್ಯೈ-ಸ್ತತ್
ಸದ-ಸತೇ
ಸದ-ಸ-ತ್ಪ-ತಿಃ
ಸದ-ಸ-ತ್ಸ-ರ್ವ-ರ-ತ್ನ-ವಿತ್
ಸದ-ಸ-ದ್ವ್ಯ-ಕ್ತ-ಮ-ವ್ಯಕ್ತಂ
ಸದಾ
ಸದಾ-ಶಿ-ವ-ಪ್ರೀ-ತ್ಯರ್ಥಂ
ಸದಾ-ಶಿ-ವಾಯ
ಸದಾ-ಶಿವೋ
ಸನಾ-ತ-ನಮ್
ಸಫ-ಲೋ-ದಯಃ
ಸಫ-ಲೋ-ದ-ಯಾಯ
ಸಭಾ-ವನಃ
ಸಭಾ-ವ-ನಾಯ
ಸಮಂ-ತಾತ್
ಸಮ-ಕ-ಲ್ಪ-ಯತ್
ಸಮ-ರ-ಮ-ರ್ದನಃ
ಸಮ-ರ-ಮ-ರ್ದ-ನಾಯ
ಸಮ-ರ್ಪಿ-ಸು-ವು-ದಕ್ಕೆ
ಸಮಾ-ಧಿ-ಷ್ಠಾಯ
ಸಮಾ-ಪ್ತಮ್
ಸಮಾ-ಪ್ತಾಃ
ಸಮಾಮ್ನಾ
ಸಮಾ-ಮ್ನಾ-ಯಾಯ
ಸಮು-ದ್ದೇಶಂ
ಸಮು-ದ್ಧ-ರೇತ್
ಸಮು-ದ್ರಾಯ
ಸಮುದ್ರೋ
ಸಯ-ಜ್ಞಾ-ರಯೇ
ಸಯ-ಜ್ಞಾ-ರಿಃ
ಸರ್ಗಃ
ಸರ್ಗ-ಸು-ಸಂ-ಕ್ಷೇ-ಪ-ವಿ-ಸ್ತ-ರಾಯ
ಸರ್ಪ-ಚೀ-ರ-ನಿ-ವಾ-ಸನಃ
ಸರ್ಪ-ಚೀ-ರ-ನಿ-ವಾ-ಸ-ನಾಯ
ಸರ್ವಃ
ಸರ್ವ-ಕ-ರಾಯ
ಸರ್ವ-ಕರೋ
ಸರ್ವ-ಕ-ರ್ಮ-ಣಾಮ್
ಸರ್ವ-ಕ-ರ್ಮಣೇ
ಸರ್ವ-ಕರ್ಮಾ
ಸರ್ವ-ಕ-ರ್ಮೋ-ತ್ಥಾ-ನಾಯ
ಸರ್ವ-ಕಾ-ಮ-ಗು-ಣಾ-ವಹಃ
ಸರ್ವ-ಕಾ-ಮ-ಗು-ಣಾ-ವ-ಹಾಯ
ಸರ್ವ-ಕಾ-ಮದಃ
ಸರ್ವ-ಕಾ-ಮ-ದಾಯ
ಸರ್ವ-ಕಾ-ಮ-ವ-ರ-ಶ್ಚೈವ
ಸರ್ವ-ಕಾ-ಮ-ವ-ರಾಯ
ಸರ್ವ-ಕಾ-ಮಾಯ
ಸರ್ವ-ಕಾ-ಮಾ-ಶ್ಚ-ತು-ಷ್ಪದಃ
ಸರ್ವ-ಕಾ-ಲ-ಪ್ರ-ಸಾ-ದಶ್ಚ
ಸರ್ವ-ಕಾ-ಲ-ಪ್ರ-ಸಾ-ದಾಯ
ಸರ್ವ-ಗಂ-ಧ-ಸು-ಖಾ-ವಹಃ
ಸರ್ವ-ಗಂ-ಧ-ಸು-ಖಾ-ವ-ಹಾಯ
ಸರ್ವ-ಗ-ವಾ-ಯವೇ
ಸರ್ವ-ಗಾಯ
ಸರ್ವಗೋ
ಸರ್ವ-ಗೋ-ಽಮುಖಃ
ಸರ್ವ-ಚಾ-ರಿಣೇ
ಸರ್ವ-ಚಾರೀ
ಸರ್ವಜ್ಞಃ
ಸರ್ವ-ಜ್ಞಾಯ
ಸರ್ವ-ತೂ-ರ್ಯ-ನಿ-ನಾ-ದಿನೇ
ಸರ್ವ-ತೂ-ರ್ಯ-ನಿ-ನಾದೀ
ಸರ್ವ-ತೋ-ಮುಖಃ
ಸರ್ವ-ತೋ-ಮು-ಖಾಯ
ಸರ್ವತ್ರ
ಸರ್ವಥಾ
ಸರ್ವದಃ
ಸರ್ವದಾ
ಸರ್ವ-ದಾಯ
ಸರ್ವ-ದೇ-ವ-ಮ-ಯಾಯ
ಸರ್ವ-ದೇ-ವ-ಮಯೋ
ಸರ್ವ-ದೇ-ವ-ಮ-ಯೋ-ಽಚಿಂತ್ಯೋ
ಸರ್ವ-ದೇ-ವ-ಸ್ತ-ಪೋ-ಮಯಃ
ಸರ್ವ-ದೇ-ವಾಯ
ಸರ್ವ-ದೇ-ಹಿ-ನಾ-ಮಿಂ-ದ್ರಿ-ಯಾಯ
ಸರ್ವ-ದೇ-ಹಿ-ನಾಮ್
ಸರ್ವ-ಧಾ-ರಿಣೇ
ಸರ್ವ-ಧಾರೀ
ಸರ್ವ-ಪಾ-ಪ-ಪ್ರ-ಣಾ-ಶ-ನಮ್
ಸರ್ವ-ಪಾ-ಪಾ-ಪ-ಹ-ಮಿದಂ
ಸರ್ವ-ಪಾ-ರ್ಶ್ವ-ಮು-ಖ-ಸ್ತ್ರ-್ಯಕ್ಷೋ
ಸರ್ವ-ಪಾ-ರ್ಶ್ವ-ಮು-ಖಾಯ
ಸರ್ವ-ಪಾ-ವನಃ
ಸರ್ವ-ಪಾ-ವ-ನಾಯ
ಸರ್ವ-ಪೂ-ಜಿತಃ
ಸರ್ವ-ಪೂ-ಜಿ-ತಾಯ
ಸರ್ವ-ಪ್ರಾ-ಣಿ-ಪ-ತಯೇ
ಸರ್ವ-ಭಾ-ವ-ಕ-ರಾಯ
ಸರ್ವ-ಭಾ-ವ-ಕರೋ
ಸರ್ವ-ಭಾ-ವತಃ
ಸರ್ವ-ಭಾ-ವನಃ
ಸರ್ವ-ಭಾ-ವ-ನಾಯ
ಸರ್ವ-ಭಾ-ವಾ-ನು-ಗ-ತಾಃ
ಸರ್ವ-ಭೂ-ತ-ನಿ-ಲ-ಯಾಯ
ಸರ್ವ-ಭೂ-ತ-ವಾ-ಹಿತ್ರೇ
ಸರ್ವ-ಭೂ-ತ-ಹರಃ
ಸರ್ವ-ಭೂ-ತ-ಹ-ರಾಯ
ಸರ್ವ-ಭೂ-ತ-ಹಿತಂ
ಸರ್ವ-ಭೂ-ತಾ-ತ್ಮನೇ
ಸರ್ವ-ಭೂ-ತಾ-ತ್ಮ-ಭೂ-ತಸ್ಯ
ಸರ್ವ-ಭೂ-ತಾತ್ಮಾ
ಸರ್ವ-ಭೂ-ತಾ-ನಾಂ
ಸರ್ವ-ಮಂ-ಗ-ಲ-ಮಾಂ-ಗಲ್ಯಂ
ಸರ್ವ-ಯು-ಕ್ತಸ್ಯ
ಸರ್ವ-ಯೋ-ಗಿನೇ
ಸರ್ವ-ಯೋಗೀ
ಸರ್ವ-ರ-ತ್ನ-ವಿದೇ
ಸರ್ವ-ಲ-ಕ್ಷ-ಣ-ಲ-ಕ್ಷಿತಃ
ಸರ್ವ-ಲ-ಕ್ಷ-ಣ-ಲ-ಕ್ಷಿ-ತಾಯ
ಸರ್ವ-ಲಾ-ಲಸಃ
ಸರ್ವ-ಲಾ-ಲ-ಸಾಯ
ಸರ್ವ-ಲೋ-ಕ-ಕೃತೇ
ಸರ್ವ-ಲೋ-ಕ-ಕೃತ್
ಸರ್ವ-ಲೋ-ಕ-ಪಿ-ತಾ-ಮಹಃ
ಸರ್ವ-ಲೋ-ಕ-ಪ್ರ-ಜಾ-ಪ-ತಯೇ
ಸರ್ವ-ಲೋ-ಕ-ಪ್ರ-ಜಾ-ಪ-ತಿಃ
ಸರ್ವ-ಲೋ-ಕೇಷು
ಸರ್ವ-ಲೋ-ಚನಃ
ಸರ್ವ-ಲೋ-ಚ-ನಾಯ
ಸರ್ವ-ವಾಸಃ
ಸರ್ವ-ವಾ-ಸಾಯ
ಸರ್ವ-ವಾ-ಸಿನೇ
ಸರ್ವ-ವಾಸೀ
ಸರ್ವ-ವಿ-ಖ್ಯಾತಃ
ಸರ್ವ-ವಿ-ಖ್ಯಾ-ತಾಯ
ಸರ್ವ-ವಿ-ಗ್ರಹ
ಸರ್ವ-ವಿ-ಗ್ರ-ಹಾಯ
ಸರ್ವ-ಶು-ಭಂ-ಕರಃ
ಸರ್ವ-ಶು-ಭ-ಕ-ರಾಯ
ಸರ್ವ-ಸಾ-ಧನಃ
ಸರ್ವ-ಸಾ-ಧ-ನಾಯ
ಸರ್ವ-ಸಾ-ಧಾ-ರ-ಣ-ವ-ರಾಯ
ಸರ್ವ-ಸಾ-ಧು-ನಿ-ಷೇ-ವಿತಃ
ಸರ್ವ-ಸಾ-ಧು-ನಿ-ಷೇ-ವಿ-ತಾಯ
ಸರ್ವ-ಸ್ತ-ವಾ-ನಾಂ
ಸರ್ವಸ್ಮೈ
ಸರ್ವಸ್ಯ
ಸರ್ವಾಂಗಃ
ಸರ್ವಾಂ-ಗ-ರೂ-ಪಾಯ
ಸರ್ವಾಂ-ಗ-ರೂಪೋ
ಸರ್ವಾಂ-ಗಾಯ
ಸರ್ವಾ-ತೋ-ದ್ಯ-ಪ-ರಿ-ಗ್ರಹಃ
ಸರ್ವಾ-ತೋ-ದ್ಯ-ಪ-ರಿ-ಗ್ರ-ಹಾಯ
ಸರ್ವಾ-ತ್ಮನೇ
ಸರ್ವಾತ್ಮಾ
ಸರ್ವಾ-ನ್ಕಾ-ಮಾ-ನ-ವಾ-ಪ್ಸ್ಯಸಿ
ಸರ್ವಾ-ಯುಧಃ
ಸರ್ವಾ-ಯು-ಧಾಯ
ಸರ್ವಾ-ರ್ಥ-ಸಾ-ಧ-ಕೈಃ
ಸರ್ವಾ-ಶ-ಯಾಯ
ಸರ್ವಾ-ಶಯೋ
ಸರ್ವಾ-ಶ್ರಯಃ
ಸರ್ವಾ-ಶ್ರ-ಯ-ಕ್ರ-ಮಾಯ
ಸರ್ವೇ-ಷಾಂ
ಸವಿತಾ
ಸವಿ-ತಾ-ಽಮೃತಃ
ಸವಿತ್ರೇ
ಸಹಃ
ಸಹ-ಪೂ-ರ್ವೈಃ
ಸಹ-ಸ್ರದಃ
ಸಹ-ಸ್ರ-ದಾಯ
ಸಹ-ಸ್ರ-ನಾ-ಮ-ವನ್ನು
ಸಹ-ಸ್ರ-ನಾ-ಮಾ-ವ-ಲಿಯ
ಸಹ-ಸ್ರ-ಪದೇ
ಸಹ-ಸ್ರ-ಪಾತ್
ಸಹ-ಸ್ರ-ಬಾ-ಹವೇ
ಸಹ-ಸ್ರ-ಬಾ-ಹುಃ
ಸಹ-ಸ್ರ-ಮೂರ್ಧಾ
ಸಹ-ಸ್ರ-ಮೂರ್ಧ್ನೇ
ಸಹ-ಸ್ರ-ಹ-ಸ್ತಾಯ
ಸಹ-ಸ್ರ-ಹಸ್ತೋ
ಸಹ-ಸ್ರಾ-ಕ್ಷಾಯ
ಸಹ-ಸ್ರಾಕ್ಷೋ
ಸಹಾ-ತ್ಮ-ಜೈಃ
ಸಹಾಯ
ಸಹಾಯಃ
ಸಹಾ-ಯಶ್ಚ
ಸಹಾ-ಯಾಯ
ಸಾಂಖ್ಯ-ಪ್ರ-ಸಾ-ದಾಯ
ಸಾಂಖ್ಯ-ಪ್ರ-ಸಾದೋ
ಸಾಂಖ್ಯ-ಯೋ-ಗಾ-ನಾಂ
ಸಾತ್ತ್ವಿ-ಕಾಯ
ಸಾಧು-ಭಿಃ
ಸಾಧ್ಯ-ರ್ಷಯೇ
ಸಾಧ್ಯ-ರ್ಷಿ-ರ್ವ-ಸು-ರಾ-ದಿತ್ಯೋ
ಸಾಧ್ಯಾಯ
ಸಾಧ್ಯೋ
ಸಾಮ-ಪ್ರಿ-ಯಾಯ
ಸಾಮಾಸ್ಯ
ಸಾಮಾ-ಸ್ಯಾಯ
ಸಾರಂ
ಸಾರಂ-ಗಾಯ
ಸಾರಂಗೋ
ಸಾರ-ಗ್ರೀ-ವಾಯ
ಸಾರ-ಗ್ರೀವೋ
ಸಾರ-ಥಯೇ
ಸಿಂಹಗಃ
ಸಿಂಹ-ಗಾಯ
ಸಿಂಹ-ದಂಷ್ಟ್ರಃ
ಸಿಂಹ-ದಂ-ಷ್ಟ್ರಾಯ
ಸಿಂಹ-ನಾದಃ
ಸಿಂಹ-ನಾ-ದಾಯ
ಸಿಂಹ-ವಾ-ಹನಃ
ಸಿಂಹ-ವಾ-ಹ-ನಾಯ
ಸಿಂಹ-ಶಾ-ರ್ದೂ-ಲ-ರೂ-ಪಶ್ಚ
ಸಿಂಹ-ಶಾ-ರ್ದೂ-ಲ-ರೂ-ಪಾಯ
ಸಿದ್ಧ-ಭೂ-ತಾ-ರ್ಥಾಯ
ಸಿದ್ಧ-ಭೂ-ತಾ-ರ್ಥೋ-ಽಚಿಂತ್ಯಃ
ಸಿದ್ಧಯೇ
ಸಿದ್ಧ-ಯೋ-ಗಿನೇ
ಸಿದ್ಧ-ಯೋಗೀ
ಸಿದ್ಧ-ಸಾ-ಧಕಃ
ಸಿದ್ಧ-ಸಾ-ಧ-ಕಾಯ
ಸಿದ್ಧಾರ್ಥಃ
ಸಿದ್ಧಾ-ರ್ಥ-ಕಾ-ರಿಣೇ
ಸಿದ್ಧಾ-ರ್ಥ-ಕಾರೀ
ಸಿದ್ಧಾ-ರ್ಥ-ಶ್ಛಂ-ದೋ-ವ್ಯಾ-ಕ-ರ-ಣೋ-ತ್ತರಃ
ಸಿದ್ಧಾ-ರ್ಥಾಯ
ಸಿದ್ಧಾರ್ಥೋ
ಸಿದ್ಧಿಂ
ಸಿದ್ಧಿಃ
ಸಿದ್ಧೈಃ
ಸುಖಾ-ಜಾತಃ
ಸುಖಾ-ಜಾ-ತಾಯ
ಸುಖಾ-ಸಕ್ತಃ
ಸುಖಾ-ಸ-ಕ್ತಾಯ
ಸುಗಂ-ಧಾ-ರಾ-ಯ-ನಮಃ
ಸುಗಂ-ಧಾರೋ
ಸುಚ್ಛ-ತ್ರಾಯ
ಸುಚ್ಛತ್ರೋ
ಸುತೀ-ಕ್ಷ್ಣ-ದ-ಶ-ನ-ಶ್ಚೈವ
ಸುತೀ-ಕ್ಷ್ಣ-ದ-ಶ-ನಾಯ
ಸುತೀರ್ಥಃ
ಸುತೀ-ರ್ಥಾಯ
ಸುದ-ರ್ಶನಃ
ಸುದ-ರ್ಶ-ನಾಯ
ಸುನಿ-ಶ್ಚಲಃ
ಸುನಿ-ಶ್ಚ-ಲಾಯ
ಸುಬಂ-ಧ-ನ-ವಿ-ಮೋ-ಚನಃ
ಸುಬಂ-ಧ-ನ-ವಿ-ಮೋ-ಚ-ನಾಯ
ಸುಬ-ಲಾಯ
ಸುಬಲೋ
ಸುಬಾಂ-ಧವಃ
ಸುಬಾಂ-ಧ-ವಾಯ
ಸುಬೀ-ಜಾಯ
ಸುಬೀಜೋ
ಸುಮ-ಹಾ-ಸ್ವನಃ
ಸುಮ-ಹಾ-ಸ್ವ-ನಾಯ
ಸುಮುಖಃ
ಸುಮು-ಖಾಯ
ಸುಯುಕ್ತಃ
ಸುಯು-ಕ್ತಾಯ
ಸುರ-ಗ-ಣಾಯ
ಸುರ-ಗಣೋ
ಸುರ-ಭ್ಯು-ತ್ತ-ರಣೋ
ಸುರ-ಭ್ಯು-ತ್ತಾ-ರ-ಣಾಯ
ಸುರಾ-ಧ್ಯ-ಕ್ಷಾಯ
ಸುರಾ-ಧ್ಯಕ್ಷೋ
ಸುರಾ-ರಿಘ್ನೇ
ಸುರಾ-ರಿಹಾ
ಸುರೂ-ಪಶ್ಚ
ಸುರೂ-ಪಾಯ
ಸುರೈ-ರಪಿ
ಸುಲೋ-ಚ-ನಾಯ
ಸುವ-ಕ್ತ್ರಶ್ಚ
ಸುವ-ಕ್ತ್ರಾಯ
ಸುವ-ರ್ಚಸಃ
ಸುವ-ರ್ಚ-ಸಾಯ
ಸುವ-ರ್ಚ-ಸಿನೇ
ಸುವ-ರ್ಚಸೀ
ಸುವ-ರ್ಣ-ರೇ-ತಸೇ
ಸುವ-ರ್ಣ-ರೇ-ತಾಃ
ಸುವ-ರ್ಣಶ್ಚ
ಸುವ-ರ್ಣಾಯ
ಸುವಾ-ಸಶ್ಚ
ಸುವಾ-ಸಾಯ
ಸುವಿ-ಜ್ಞೇಯಃ
ಸುವಿ-ಜ್ಞೇ-ಯಾಯ
ಸುಶಾ-ರದಃ
ಸುಶಾ-ರ-ದಾಯ
ಸುಷಾ-ಢಶ್ಚ
ಸುಷಾ-ಢಾಯ
ಸುಸಂ-ಕ್ಷೇಪೋ
ಸುಸ-ರ-ಣಾಯ
ಸುಸ-ರಣೋ
ಸುಸ-ಹಾಯ
ಸುಸಹೋ
ಸುಸ್ವ-ಪ್ನಾಯ
ಸುಸ್ವಪ್ನೋ
ಸುಹೃ-ದಾಯ
ಸುಹೃದೋ
ಸೂಕ್ಷ್ಮಃ
ಸೂಕ್ಷ್ಮ-ತ-ನವೇ
ಸೂಕ್ಷ್ಮಾ-ತ್ಮನೇ
ಸೂಕ್ಷ್ಮಾತ್ಮಾ
ಸೂಕ್ಷ್ಮಾಯ
ಸೂರ್ಯಃ
ಸೂರ್ಯಾಯ
ಸೇನಾ-ಕ-ಲ್ಪಾಯ
ಸೇನಾ-ಕಲ್ಪೋ
ಸೇನಾ-ಪ-ತಯೇ
ಸೇನಾ-ಪ-ತಿ-ರ್ವಿ-ಭುಃ
ಸೇವಿ-ತಾಯ
ಸೋಮ-ಸೂ-ರ್ಯಾ-ಗ್ನಿ-ಲೋ-ಚ-ನಾಯ
ಸೋಮಾಯ
ಸೋಮೋ
ಸೋಽಶ್ವ-ಮೇ-ಧ-ಫಲಂ
ಸ್ಕಂದಾಯ
ಸ್ಕಂದೋ
ಸ್ತವ-ಮೇತಂ
ಸ್ತವಮ್
ಸ್ತವ-ರಾಜ
ಸ್ತವ-ರಾ-ಜೋ-ಽವ-ತಾ-ರಿತಃ
ಸ್ತವಾ-ನಾ-ಮು-ತ್ತಮಂ
ಸ್ತವೈಃ
ಸ್ತುತಃ
ಸ್ತುತ-ಮ-ಮ-ರ-ಗ-ಣೈ-ರ್ವ್ಯಾ-ಘ್ರ-ಕೃ-ತ್ತಿಂ
ಸ್ತುತೋ
ಸ್ತುತ್ಯಂ
ಸ್ತುವಂತಃ
ಸ್ತುವಂ-ತ್ಯೇ-ತೇನ
ಸ್ತುವನ್
ಸ್ತೂಯ-ಮಾ-ನಾಶ್ಚ
ಸ್ತೂಯ-ಮಾನೋ
ಸ್ತೋತ-ವ್ಯ-ಮರ್ಚ್ಯಂ
ಸ್ತೋತು-ಮೀ-ಶ್ವರಃ
ಸ್ತೋತ್ರ
ಸ್ತೋಷ್ಯತಿ
ಸ್ತೋಷ್ಯಾಮಿ
ಸ್ಥಾಣವೇ
ಸ್ಥಾಣುಃ
ಸ್ಥಾವ-ರ-ಪ-ತಯೇ
ಸ್ಥಾವ-ರಾ-ಣಾಂ
ಸ್ಥಿರಃ
ಸ್ಥಿರಾಯ
ಸ್ನೇಹ-ನಾಯ
ಸ್ನೇಹ-ನೋ-ಽಸ್ನೇ-ಹ-ನ-ಶ್ಚೈವ
ಸ್ರುತ-ಸ್ತಥಾ
ಸ್ರುತಾಯ
ಸ್ರುವ-ಹಸ್ತಃ
ಸ್ರುವ-ಹ-ಸ್ತಾಯ
ಸ್ವಯಂ
ಸ್ವಯಂ-ಭುವಃ
ಸ್ವಯಂ-ಭು-ವ-ತಿ-ಗ್ಮ-ತೇ-ಜಸೇ
ಸ್ವಯಂ-ಭೂತ
ಸ್ವಯಂ-ಭೂ-ತಾಯ
ಸ್ವಯಂ-ಶ್ರೇ-ಷ್ಠಾಯ
ಸ್ವಯ-ಮ-ಧಾ-ರ-ಯತ್
ಸ್ವರ-ಮ-ಯಾಯ
ಸ್ವರ್ಗದಂ
ಸ್ವರ್ಗ-ದ್ವಾರಂ
ಸ್ವರ್ಗ-ದ್ವಾ-ರಾಯ
ಸ್ವರ್ಗಾ-ಚ್ಚೈ-ವಾತ್ರ
ಸ್ವರ್ಗೇ
ಸ್ವರ್ಗ್ಯಂ
ಸ್ವರ್ಗ್ಯ-ಮಾ-ರೋ-ಗ್ಯ-ಮಾ-ಯುಷ್ಯಂ
ಸ್ವರ್ಭಾ-ನವೇ
ಸ್ವರ್ಭಾ-ನು-ರ-ಮಿತೋ
ಸ್ವಸ್ತಿದಃ
ಸ್ವಸ್ತಿ-ದಾಯ
ಸ್ವಸ್ತಿ-ಭಾ-ವಶ್ಚ
ಸ್ವಸ್ತಿ-ಭಾ-ವಾಯ
ಹಯ-ಗ-ರ್ದ-ಭಯೇ
ಹಯ-ಗ-ರ್ದ-ಭಿಃ
ಹರಃ
ಹರಯೇ
ಹರಶ್ಚ
ಹರ-ಸು-ಲೋ-ಚನಃ
ಹರ-ಸ್ಯಾ-ಮಿ-ತ-ತೇ-ಜಸಃ
ಹರಾಯ
ಹರಿಃ
ಹರಿ-ಕೇ-ಶಶ್ಚ
ಹರಿ-ಕೇ-ಶಾಯ
ಹರಿ-ಣಾ-ಕ್ಷಶ್ಚ
ಹರಿ-ಣಾ-ಕ್ಷಾಯ
ಹರಿ-ಣಾಯ
ಹರಿಣೋ
ಹರಿ-ರ್ಯಜ್ಞಃ
ಹರ್ಯಕ್ಷಃ
ಹರ್ಯ-ಕ್ಷಾಯ
ಹರ್ಯಶ್ವಃ
ಹರ್ಯ-ಶ್ವಾಯ
ಹವಿಃ
ಹವಿಷೇ
ಹವಿ-ಸ್ತಥಾ
ಹಸ್ತೀ-ಶ್ವ-ರಾಯ
ಹಸ್ತೀ-ಶ್ವರೋ
ಹಿಮ-ವ-ದ್ಗಿ-ರಿ-ಸಂ-ಶ್ರಯಃ
ಹಿಮ-ವ-ದ್ಗಿ-ರಿ-ಸಂ-ಶ್ರ-ಯಾಯ
ಹಿರ-ಣ್ಯ-ಕ-ವ-ಚೋ-ದ್ಭವಃ
ಹಿರ-ಣ್ಯ-ಕ-ವ-ಚೋ-ದ್ಭ-ವಾಯ
ಹಿರ-ಣ್ಯ-ಬಾ-ಹವೇ
ಹಿರ-ಣ್ಯ-ಬಾ-ಹುಶ್ಚ
ಹಿರ-ಣ್ಯ-ರೇ-ತಸೇ
ಹುತಾಯ
ಹುತಾ-ಶನ
ಹುತಾ-ಶನಃ
ಹುತಾ-ಶ-ನ-ಸ-ಹಾ-ಯಶ್ಚ
ಹುತಾ-ಶ-ನಾಯ
ಹುತೋ
ಹೃದಿ
ಹೇಮ
ಹೇಮ-ಕ-ರಾಯ
ಹೇಮ-ಕ-ರೋ-ಽಯಜ್ಞಃ
ಹೈಮಾಯ
ಹೈಮೋ
ಹ್ಯಂತ-ರಾತ್ಮಾ
ಹ್ಯಂಬು-ಜಾಲಃ
ಹ್ಯಚ-ಲೋ-ಪಮಃ
ಹ್ಯತಂ-ದ್ರಿತಃ
ಹ್ಯದಂಭೋ
ಹ್ಯದಿ-ತಿ-ಸ್ತಾ-ರ್ಕ್ಷ್ಯಃ
ಹ್ಯನ-ನ್ಯ-ಮೀ-ಶಾನಂ
ಹ್ಯನಿಂ-ದಿತಃ
ಹ್ಯನಿ-ಲೋ-ಽನಲಃ
ಹ್ಯನೌ-ಷಧಃ
ಹ್ಯಪ್ಸ-ರೋ-ಗ-ಣ-ಸೇ-ವಿತಃ
ಹ್ಯಮೃತೋ
ಹ್ಯವ-ತಾ-ರಿತಃ
ಹ್ಯವಶಃ
ಹ್ಯವಿ-ಶ್ರು-ತಮ್
ಹ್ಯಶ್ವ-ತ್ಥೋ-ಽರ್ಥ-ಕರೋ
ಹ್ಯಶ್ವೋ
ಹ್ಲಾದ-ನ-ಶ್ಚೈವ
ಹ್ಲಾದ-ನಾಯ
ೃಷಿಃ
್ಣತಾ-ಪಾಯ
}
