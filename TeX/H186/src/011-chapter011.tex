
\chapter{अग्निमयी उक्तियाँ }

\toendnotes{अग्निमयी उक्तियाँ}

\addtoendnotes{\protect\begin{multicols}{3}}

हमें ‘साहसी’ शब्द और उससे भी अधिक ‘साहसी’ कर्मों की आवश्यकता है। उठो! उठो! संसार दुःख से जल रहा है। क्या तुम सो सकते हो? हम बार-बार पुकारें, जब तक सोते हुए देवता न जाग उठें, जब तक अन्तर्यामी देव उस पुकार का उत्तर न दें। जीवन में और क्या है? इससे महान् कर्म क्या है?\endnote{ ४/४०८;} 

मेरे बच्चो, याद रखना कि कायर तथा दुर्बल लोग ही पापाचरण करते हैं और झूठ बोलते हैं। साहसी तथा शक्तिशाली लोग सदा ही नीतिपरायण होते हैं। नीतिपरायण, साहसी तथा सहानुभूति-सम्पन्न बनने का प्रयास करो।\endnote{ १/१५१;}

आओ, मनुष्य बनो! अपनी कूप-मण्डूकता छोड़ो और बाहर दृष्टि डालो। देखो, दुनिया के दूसरे देश किस प्रकार आगे बढ़ रहे हैं! क्या तुम्हें मनुष्य से प्रेम है? क्या तुम्हें अपने देश से प्रेम है? यदि -‘हाँ’ - तो आओ, हम लोग उच्चतर और बेहतर चीजों के लिए संघर्ष करें। पीछे मुड़कर मत देखो। तुम्हारे परम निकट और प्रियजन रोते हों, तो रोने दो। पीछे देखो ही मत; बस, केवल आगे बढ़ते जाओ! 

भारतमाता कम-से-कम एक हजार युवकों की बलि चाहती है, ध्यान रहे - मनुष्यों की, पशुओं की नहीं।\endnote{ १/३९९;} 

धीरता और दृढ़ता के साथ चुपचाप काम करना होगा। समाचार-पत्रों के द्वारा हल्ला मचाने से काम न होगा। सर्वदा याद रखो - नाम कमाना हमारा उद्देश्य नहीं है।\endnote{ १/३९९;} 

क्या समता, स्वतंत्रता, कर्मठता और पौरुष में तुम पाश्चात्यों के भी गुरु बन सकते हो? क्या तुम इसके साथ ही अपने धर्म-विश्वास तथा साधनाओं में एक कट्टर हिन्दू हो सकते हो? यह हमें करना है और हम इसे अवश्य करेंगे। तुम सबने इसी कार्य के लिए जन्म लिया है। स्वयं पर विश्वास रखो। दृढ़ विश्वास महान् कार्यों का मूल है। निरन्तर बढ़े चलो। मरते दम तक गरीबों और पददलितों के लिए सहानुभूति - यही हमारा मूलमंत्र है। वीर युवको! बढ़े चलो!\endnote{ २/३२१-२२;} 

प्रत्येक मनुष्य एवं प्रत्येक राष्ट्र को महान् बनाने के लिए तीन चीजें आवश्यक हैं -

\begin{enumerate}
\item सदाचार की शक्ति में विश्वास 

 \item ईर्ष्या और सन्देह का परित्याग 

 \item जो भला बनने या भलाई के कार्यों में प्रयत्नशील हों, उनकी सहायता करना।\endnote{ २/३२४;} 

\end{enumerate}

किसी भी व्यक्ति या किसी वस्तु की आशा मत करो। तुम स्वयं जितना कुछ कर सकते हो, करो। किसी के ऊपर अपनी आशा का महल खड़ा मत करो।\endnote{ २/३५५;} 

\newpage

महान् कर्म, केवल महान् बलिदानों द्वारा ही सम्भव हैं। स्वार्थपरता की आवश्यकता नहीं, नाम की भी नहीं, यश की भी नहीं - तुम्हारी नहीं, मेरी नहीं, यहाँ तक कि मेरे गुरुदेव की भी नहीं। मेरे बच्चो, मेरे वीर-भले-महान् आत्माओ, कर्म करो, भावों - योजनाओं को कार्य रूप में परिणत करो, अपने कन्धों को पहिये से लगा दो। नाम, यश या किसी अन्य तुच्छ वस्तु के लिए मुड़कर देखने के लिए मत ठहरो। स्वार्थ को उखाड़ फेंको और काम करो।\endnote{ २/३५६;} 

वेद कहते हैं - \textbf{उत्तिष्ठत जाग्रत प्राप्य वरान् निबोधत } - उठो! जागो! लक्ष्य पर पहुँच जाने के पहले रुको मत। जागो! जागो! लम्बी रात बीत रही है, सूर्योदय का प्रकाश दिखायी दे रहा है। महातरंग उठ चुकी है, कोई भी इसका प्रतिरोध नहीं कर सकेगा।... उत्साह, मेरे बच्चो, उत्साह! प्रेम, मेरे बच्चो, प्रेम! श्रद्धा और विश्वास। डरो नहीं। भय ही सबसे बड़ा पाप है।\endnote{ २/३५६;} 

कार्य की सामान्य शुरुआत को देखकर डरो मत। बड़ी चीजें बाद में आएँगी। साहस रखो। अपने भाइयों का नेता बनने की कोशिश मत करो, बल्कि उनकी सेवा करते रहो। नेता बनने की इस पाशविक प्रवृत्ति ने जीवन-रूपी समुद्र में अनेक बड़े-बड़े जहाजों को डुबा दिया है। इस विषय में सावधान रहना, अर्थात् मृत्यु तक को तुच्छ समझकर निःस्वार्थ हो जाओ और काम करते रहो।\endnote{ २/३५७-५८;} 

खूब परिश्रम करते रहो। पवित्र और शुद्ध बनो - उत्साहाग्नि स्वयं ही प्रज्वलित हो उठेगी।\endnote{ २/३८४;} 

मेरे वीर हृदय युवको, विश्वास रखो कि तुम सबका जन्म अनेक महान् कार्य करने के लिए हुआ है। कुत्तों के भौंकने से मत डरो - नहीं, स्वर्ग के वज्र से भी मत डरो। उठकर खड़े हो जाओ और कार्य में लगे रहो।\endnote{ २/३९६;२०५} 

लोग जो चाहें, कहने दो; अपनी निष्ठाओं में दृढ़ रहो - फिर कोई चिन्ता नहीं, दुनिया तुम्हारे पैरों तले आ जाएगी। लोग कहते हैं - “इस पर विश्वास करो, उस पर विश्वास करो”, मैं कहता हूँ - “पहले स्वयं पर विश्वास करो।” यही सही रास्ता है।\endnote{ ३/३११;} 

कहो - “मैं सब कुछ कर सकता हूँ।” “नहीं, नहीं” - कहने से तो साँप का विष भी निष्प्रभ हो जाता है।\endnote{ ३/३११;} 

अदम्य ऊर्जा के साथ कार्य आरम्भ कर दो। भय की क्या बात है? किसमें शक्ति है, जो बाधा डाले?... भय! किसका भय? कोई शंका मत करो।\endnote{ ३/३१२;} 

त्याग-त्याग - इसी का अच्छी तरह प्रचार करना होगा। त्यागी हुए बिना तेजस्विता नहीं आती।\endnote{ ३/३१३;} 

\textbf{मा भैः मा भैः } - डरो मत! धीरे-धीरे सब कुछ हो जाएगा। मैं तुम लोगों से यही आशा करता हूँ कि आत्म-प्रदर्शन, दलबन्दी तथा ईर्ष्या को सदा के लिए त्याग दो। पृथ्वी की तरह सब कुछ सहन करने की शक्ति अर्जित करो। यदि यह कर सको, तो दुनिया अपने आप तुम्हारे चरणों में आ गिरेगी।\endnote{ ३/३१९-२०;} 

लोग चाहे जो भी क्यों न सोचें; तुम पवित्रता, नैतिकता तथा भगवत्-प्रेम के अपने आदर्श को कभी नीचे मत उतारना।... ईश्वर से प्रेम करनेवाले को किसी भी इन्द्रजाल से नहीं डरना चाहिए। स्वर्ग तथा मर्त्यलोक में - सर्वत्र केवल पवित्रता ही सर्वोच्च तथा दिव्यतम शक्ति है।\endnote{ ५/३७१;} 

मैं उस ईश्वर या धर्म में विश्वास नहीं करता, जो न विधवाओं के आँसू पोंछ सकता है और न अनाथों के मुँह में एक टुकड़ा रोटी ही पहुँचा सकता है। किसी धर्म के सिद्धान्त चाहे जितने भी उदात्त क्यों न हो, उसका दर्शन चाहे जितना भी सुव्यवस्थित क्यों न हो; जब तक वह ग्रन्थों तथा मतों तक ही सीमित है, मैं उसे ‘धर्म’ नहीं मानता। हमारी आँखें पीछे नहीं, सामने हैं। सामने बढ़ते रहो और जिसे तुम अपना धर्म कहकर गौरवबोध करते हो, उसे कार्यरूप में परिणत करो। ईश्वर तुम्हारा कल्याण करें!\endnote{ ३/३२२-२३;} 

वत्स, कोई मनुष्य, कोई जाति, दूसरों से घृणा करते हुए जी नहीं सकती। भारत के भाग्य का निपटारा उसी दिन हो चुका, जब उसने इस म्लेच्छ शब्द का अविष्कार किया और दूसरों से अपना नाता तोड़ लिया। खबरदार, जो तुमने इस विचार की पुष्टि की! वेदान्त की बातें बघारना तो खूब सरल है, पर इसके छोटे-से-छोटे सिद्धान्त को काम में लाना कितना कठिन है!\endnote{ ३/३२४;} 

जिस संन्यासी में मनुष्यों के कल्याण करने की कोई इच्छा नहीं है; वह संन्यासी नहीं, वह तो पशु है!\endnote{ ३/३२६;} 

जिनके ऐशो-आराम में लालन-पालन तथा शिक्षा - लाखों पददलित परिश्रमी गरीबों के हृदय के रक्त से हो रही है और इसके बावजूद जो उनकी ओर ध्यान नहीं देता, उन्हें मैं विश्वासघाती कहता हूँ। इतिहास में कहाँ और किस काल में आपके धनवानों, कुलीनों, पुरोहितों और राजाओं ने गरीबों की ओर ध्यान दिया था - उन गरीबों की ओर, जिन्हें कोल्हू के बैलों की भाँति पेरने से ही उनकी शक्ति संचित हुई थी।\endnote{ ३/३२९;} 

करोड़ों पिछड़े असहाय लोगों को ऊपर उठाने की किसे चिन्ता है? विश्वविद्यालय के उपाधिधारी कुछ हजार व्यक्तियों से राष्ट्र का निर्माण नहीं हो सकता, कुछ धनवानों से राष्ट्र नहीं बनता। यह सच है कि हमारे पास सुअवसर कम हैं; तथापि तीस करोड़ व्यक्तियों को खिलाने और पहनाने के लिए, उन्हें आराम से रखने के लिए, बल्कि उन्हें विलासितापूर्वक रखने के लिए हमारे पास पर्याप्त संसाधन~हैं।\endnote{ ३/३३०;} 

प्रेम और सहानुभूति ही एकमात्र मार्ग है। प्रेम ही एकमात्र उपासना है।\endnote{ ३/३३१;} 

मुझे पूर्ण विश्वास है कि कोई भी मनुष्य या राष्ट्र अपने को दूसरों से अलग रखकर जी नहीं सकता; और जब कभी गौरव, नीति या पवित्रता की भ्रान्त धारणा के चलते ऐसा प्रयत्न किया गया है, तो इसका परिणाम उस पृथक् होनेवाले के लिए हमेशा घातक सिद्ध हुआ है।\endnote{ ३/३३१;} 

लेन-देन ही संसार का नियम है; और यदि भारत फिर से उठना चाहे, तो यह परम आवश्यक है कि वह अपने रत्नों को बाहर लाकर पृथ्वी की जातियों में बिखेर दे; और इसके बदले में वे जो कुछ दे सकें, उसे सहर्ष ग्रहण करे। विस्तार ही जीवन है और संकुचन मृत्यु; प्रेम ही जीवन है और द्वेष ही मृत्यु।\endnote{ ३/३३२;} 

जो लोग दूसरों को स्वाधीनता देने के लिए तैयार नहीं, क्या वे स्वयं स्वाधीनता पाने योग्य हैं? व्यर्थ का असन्तोष जताते हुए शक्तिक्षय करने के बदले हम चुपचाप वीरता के साथ काम करते रहें। मेरा तो पूर्ण विश्वास है कि संसार की कोई भी शक्ति किसी से वह वस्तु अलग नहीं रख सकती, जिसके लिए वह सचमुच ही योग्य हो। अतीत तो हमारा गौरवमय था ही, पर मेरा आन्तरिक विश्वास है कि भविष्य और भी गौरवमय होगा।\endnote{ ३/३३२;} 

\vskip 2.6pt

परोपकार ही जीवन है, परोपकार न करना ही मृत्यु है। जितने नरपशु तुम देखते हो, उनमें नब्बे प्रतिशत मृत हैं, प्रेत है, क्योंकि जिसमें प्रेम नहीं, वह जी भी नहीं सकता। मेरे बच्चो, सबके लिए तुम्हारे दिल में दर्द हो - गरीब, मूर्ख तथा पददलित लोगों के दुःख को तुम महसूस करो; और तब तक महसूस करो, जब तक कि तुम्हारे हृदय की धड़कन न रुक जाए, मस्तिष्क न चकराने लगे और तुम्हें ऐसा न प्रतीत होने लगे कि तुम पागल हो जाओगे।\endnote{ ३/३३३;} 

\vskip 2.6pt

डरो मत मेरे बच्चो। अनन्त नक्षत्रखचित आकाश की ओर भयभीत दृष्टि से ऐसे मत ताको, मानो वह तुम्हें कुचल ही डालेगा। धैर्य रखो। देखोगे, कुछ ही घण्टों में वह सब-का-सब तुम्हारे पैरों तले आ गया है। धैर्य रखो। न धन से काम होता है, न नाम से; न यश काम आता है, न विद्या; प्रेम से ही सब कुछ होता है। चरित्र ही कठिनाइयों की संगीन दीवारें भेदकर अपना रास्ता बना सकता है।\endnote{ ३/३३३;} 

\vskip 2.6pt

रोटी! रोटी! मुझे इस बात का विश्वास नहीं है कि वह ईश्वर, जो यहाँ पर मुझे रोटी नहीं दे सकता, वही स्वर्ग में मुझे अनन्त सुख देगा! राम कहो! भारत को उठाना होगा, गरीबों को भोजन देना होगा, शिक्षा का विस्तार करना होगा और पुरोहित-प्रपंच की बुराइयों को मिटाना होगा। पुरोहित-प्रपंच और सामाजिक अत्याचारों का कहीं नामोनिशान न रहे! सबके लिए अधिक अन्न और सबको अधिकाधिक अवसर मिलते रहें।\endnote{ ३/३३४;} 

\vskip 2.6pt

उत्साह से हृदय को भर लो और सर्वत्र फैल जाओ। काम करो, काम करो। नेतृत्व करते समय सबके दास हो जाओ; निःस्वार्थ होओ और कभी एक मित्र को दूसरे के पीठ पीछे उसकी निन्दा करते मत सुनो। अनन्त धैर्य रखो। तभी सफलता तुम्हारे हाथ आएगी।\endnote{ ३/३३५;} 

\vskip 2.6pt

मेरे साहसी बच्चो, आगे बढ़ो - चाहे धन आए, या न आए, आदमी मिलें या न मिलें। क्या तुम्हारे पास प्रेम है? क्या तुम्हें ईश्वर पर भरोसा है? बस, आगे बढ़ो, तुम्हें कोई नहीं रोक सकेगा।\endnote{ ३/३३५-३६;} 

\vskip 2.6pt

जल्दबाजी से कोई काम नहीं होगा। पवित्रता, धैर्य और अध्यवसाय - इन्हीं तीन गुणों से सफलता मिलती है; और सर्वोपरि है प्रेम। तुम्हारे सामने अनन्त समय है, अतः हड़बड़ी करने की कोई आवश्यकता नहीं। यदि तुम पवित्र और निष्कपट हो, तो सभी काम ठीक हो जाएँगे। हमें तुम्हारे जैसे हजारों की जरूरत है, जो समाज पर टूट पड़ें और जहाँ कहीं भी जाएँ, वही नये जीवन और नयी शक्ति का संचार कर दें।\endnote{ ३/३३८;} 

समाचार-पत्रों की निन्दा और व्यर्थ बातों पर ध्यान मत दो। निष्कपट रहो और अपने कर्तव्य का पालन करो, बाकी सब ठीक हो जाएगा। सत्य की विजय अवश्यम्भावी है।\endnote{ ३/३४२-४३;} 

अपने तन, मन एवं वाणी को ‘जगद्धिताय’ अर्पित करो। तुमने पढ़ा है - \textbf{मातृदेवो भव, पितृदेवो भव } - अपनी माता को ईश्वर समझो, अपने पिता को ईश्वर समझो, परन्तु मैं कहता हूँ - \textbf{दरिद्रदेवो भव मूर्खदेवो भव } - गरीब, निरक्षर, मूर्ख और दुःखी - इन्ही को अपना ईश्वर मानो। इनकी सेवा करना ही परम धर्म समझो।\endnote{ ३/३५७;} 

शासक बनने की कोशिश मत करो - सबसे अच्छा शासक वह है, जो सबकी सेवा कर सकता है। मृत्युपर्यन्त सत्य-पथ से विचलित मत होना। हम काम चाहते हैं - धन, नाम और यश की हमें चाह नहीं है।\endnote{ ३/३४८;} 

यदि तुम सचमुच मेरी सन्तान हो, तो तुम किसी चीज से नहीं डरोगे, किसी बात पर नहीं रुकोगे। तुम सिंहतुल्य होगे। हमें भारत और पूरे संसार को जगाना है। कायरता को पास न फटकने देना। मैं ‘नहीं’ नहीं सुनूँगा, समझे? मृत्युपर्यन्त सत्य-पथ पर अटल रहना होगा।... इसका रहस्य है गुरुभक्ति, मृत्युपर्यन्त गुरु में विश्वास।\endnote{ ३/३४९;} 

यदि तुम शासन करना चाहते हो, तो दास बनो। यही सच्चा रहस्य है।\endnote{ २/३६०;} 

याद रखना - पुरुष तथा नारी - दोनों की ही आवश्यकता है। आत्मा में नारीपुरुष का कोई भेद नहीं है।... हजारों की संख्या में पुरुष तथा नारियाँ चाहिए, जो अग्नि की भाँति हिमालय से कन्याकुमारी तथा उत्तरी ध्रुव से दक्षिणी ध्रुव तक पूरी दुनिया में फैल जाएँगे। वह बच्चों का खेल नहीं है और न उसके लिए समय ही है। जो बच्चों का खेल खेलना चाहते हैं, उन्हें अभी अलग हो जाना चाहिए, नहीं तो आगे उनके लिए बड़ी विपत्ति खड़ी हो जाएगी।\endnote{ ३/३०१;} 

हममें से प्रत्येक दिन-रात भारत के उन करोड़ों पददलितों के लिए प्रार्थना करे; जो दारिद्र्य, पुरोहित-प्रपंच तथा बलवानों के अत्याचार से पीड़ित हैं। उनके लिए दिनरात प्रार्थना करो। मैं धनवान और उच्च श्रेणी की अपेक्षा इन पीड़ितों को ही धर्म का उपदेश देना अधिक पसन्द करता हूँ। मैं न कोई तत्त्व-जिज्ञासु हूँ, न दार्शनिक और न सिद्ध पुरुष; मैं तो निर्धन हूँ और निर्धनों से प्रेम करता~हूँ।\endnote{ ३/३४५;} 

जब तक करोड़ों भूखे और अशिक्षित रहेंगे, तब तक मैं प्रत्येक उस व्यक्ति को विश्वासघाती समझूँगा, जो उनके खर्च पर शिक्षित हुआ है, परन्तु जो उन पर तनिक भी ध्यान नहीं देता! वे लोग जिन्होंने गरीबों को कुचलकर धन पैदा किया है और अब ठाट-बाट से अकड़कर चलते हैं; यदि वे उन बीस करोड़ देशवासियों के लिए कुछ नहीं करते, जो इस समय भूखे और असभ्य बने हुए हैं, तो वे घृणा के पात्र हैं।\endnote{ ३/३४५-४६;} 

प्रत्येक राष्ट्र के जीवन में एक मुख्य प्रवाह रहता है। भारत में वह धर्म है। उसी को प्रबल बनाओ - बस, दोनों ओर के अन्य प्रवाह उसी के साथ चलेंगे।\endnote{ ३/३६५;} 

सदा याद रखो कि प्रत्येक राष्ट्र को अपनी रक्षा स्वयं करनी होगी। इसी तरह प्रत्येक व्यक्ति को भी अपनी रक्षा स्वयं करनी होगी। दूसरों से सहायता की आशा न रखो।\endnote{ ३/३७०;} 

मेरी सतत प्रार्थना है कि मेरे भीतर जो आग जल रही है, वही तुम्हारे भीतर जल उठे; तुम अत्यन्त निष्कपट बनो और संसार के रणक्षेत्र में तुम्हें वीरगति प्राप्त हो।\endnote{ ३/३७१;} 

मैं ऐसा कोई मार्ग नहीं निकाल पाया, जिससे सबको प्रसन्न रख सकूँ। अतः मैं वही रहूँगा, जो स्वाभाविक रूप से हूँ - अपनी अन्तरात्मा के प्रति पूर्ण रूप से ईमानदार। सौन्दर्य और यौवन का नाश हो जाता है; जीवन और धन का नाश हो जाता है; नाम और यश का भी नाश हो जाता है; पर्वत भी चूरचूर होकर मिट्टी हो जाते हैं; मित्रता और प्रेम भी नश्वर है - एकमात्र सत्य ही चिरस्थायी~है।\endnote{ ३/३७९;} 

शुरू में ही बड़ी-बड़ी योजनाएँ मत बनाओ, धीरे-धीरे कार्य आरम्भ करो - जिस जमीन पर तुम खड़े हो, उसे अच्छी तरह पकड़कर क्रमशः ऊँचाइयों को पाने की चेष्टा करो।\endnote{ ३/३८६;} 

व्यक्ति को अपनी अन्तःप्रेरणा से कार्य करना चाहिए और यदि वह कार्य शुभ तथा हितकर है, तो समाज की भावना में परिवर्तन अवश्य होगा, ऐसा चाहे उसकी मृत्यु के शताब्दियों बाद ही क्यों न हो! हमें तन-मन से कार्य में लग जाना चाहिए। जब तक हम एक - केवल एक ही आदर्श के लिए अपना सर्वस्व न्यौछावर करने को तैयार नहीं होंगे, तब तक हम कदापि आलोक नहीं देख सकेंगे।\endnote{ ३/३८७;} 

संसार के इतिहास में क्या कभी धनवानों ने कुछ काम किया है? काम हमेशा हृदय और बुद्धि से होता है, धन से नहीं।\endnote{ ३/३८७;} 

जब-जब तुम्हें दुर्बलता का बोध हो; तब-तब यह समझो कि तुम न केवल स्वयं को, बल्कि अपने उद्देश्य को भी हानि पहुँचा रहे हो। अनन्त श्रद्धा तथा शक्ति ही सफलता का मूल है।\endnote{ ३/३९०;} 

चाहिए - पूर्ण सरलता, पवित्रता, विशाल बुद्धि और सर्वजयी इच्छाशक्ति। इन गुणों से सम्पन्न मुट्ठी भर लोग काम करें, तो सारी दुनिया उलट-पलट हो जाएगी।\endnote{ ४/२७६;} 

\textbf{सत्यमेव जयते नानृतम् } - सत्य की ही विजय होती है, असत्य की नहीं। जो लोग सोचते हैं कि मिथ्या का कुछ पुट रहे बिना सत्य का प्रचार सहज ही सम्भव नहीं होगा, वे भ्रान्त हैं। समय आने पर वे देखेंगे कि विष की एक बूँद भी सारे भोजन को विषाक्त कर देती है।... जो पवित्र तथा साहसी है, वही जगत् में सब कुछ कर सकता है।\endnote{ ४/२७६;} 

धर्म ही है भारत की वह जीवनी-शक्ति और जब तक हिन्दू लोग अपने पूर्वजों से प्राप्त उत्तराधिकार को नहीं भूलेंगे, तब तक संसार की कोई भी शक्ति उनका ध्वंस नहीं कर सकती।\endnote{ ९/३५३;} 

आज्ञा-पालन के गुण को अपनाओ, लेकिन अपने धर्मविश्वास को मत खोओ। बड़ों के अधीन हुए बिना शक्ति कभी-भी केन्द्रीभूत नहीं हो सकती और बिखरी हुई शक्तियों को केन्द्रीभूत किए बिना कोई महान् कार्य नहीं हो सकता।... ईर्ष्या तथा अहंभाव को दूर करो। संगठित होकर दूसरों के लिए कार्य करना सीखो। हमारे देश में इसकी बहुत आवश्यकता है।\endnote{ ४/२८०;} 

अनन्त धैर्य, अनन्त पवित्रता तथा अनन्त अध्यवसाय - ये ही सत्कार्य में सफलता के रहस्य हैं।\endnote{ ४/२८६;} 

‘अनेक जोगी मठ उजाड़’। भारत में जिस एक चीज का हममें अभाव है, वह है मेल तथा संगठन-शक्ति; और उसे प्राप्त करने का प्रधान उपाय है - आज्ञा-पालन।\endnote{ ४/३९५;} 

वीरता से आगे बढ़ो। एक दिन या एक वर्ष में सफलता की आशा न रखो। उच्चतम आदर्श को पकड़े रहो। दृढ़ रहो। स्वार्थपरता और ईर्ष्या से बचो। आज्ञा-पालन करो। सत्य, मानवता और अपने देश के कार्य में सदा के लिए अटल रहो; और तुम संसार को हिला दोगे। याद रखो - व्यक्ति और उसका जीवन ही शक्ति का उद्गम है, अन्य कुछ भी नहीं।\endnote{ ४/३९५;} 

चारों ओर से अन्धकार जितना ही बढ़ता है, उद्देश्य उतना ही निकट आता है; मनुष्य उतना ही अधिक जीवन का सही अर्थ समझने लगता है। तब लगता है कि यह जीवन एक स्वप्न है और हम समझने लगते हैं कि क्यों हर व्यक्ति इसका अनुभव करने में असफल रहता है - इसका कारण यह है कि वह एक अर्थहीन वस्तु का अर्थ निकालने का प्रयास करता है।... सन्त यह जानकर कि ‘प्रत्येक वस्तु क्षणिक है’, ‘प्रत्येक वस्तु परिवर्तनशील है’ - सुख तथा दुःख, दोनों का परित्याग कर देता है और किसी भी वस्तु के प्रति आसक्ति न रखकर इस विश्व-परिदृश्य का द्रष्टा मात्र बन जाता है।\endnote{ ४/२९७;} 

न संख्याबल, न धन, न पाण्डित्य, न वाक्चातुरी - कुछ भी नहीं; बल्कि पवित्रता, आदर्श जीवन, एक शब्द में - अनुभूति को ही सफलता मिलेगी! हर देश में दस-बारह सिंह जैसी शक्तिमान आत्माएँ होने दो, जिन्होंने अपने बन्धन तोड़ डाले हैं, जिन्होंने ‘अनन्त’ का स्पर्श कर लिया है, जिनका चित्त ब्रह्मानुसन्धान में लीन है और जो न धन की चिन्ता करते हैं, न बल की, न नाम की - ये लोग ही संसार को हिला डालने के लिए पर्याप्त होंगे।\endnote{ ४/३३६;} 

आओ, हम नाम-यश तथा दूसरों पर शासन करने की इ्च्छा को छोड़कर काम करें। काम, क्रोध तथा लोभ - हम इन तीनों बन्धनों से मुक्त हो जाएँ। तब सत्य हमारे साथ होगा।\endnote{ ४/३३८;} 

बिना विघ्न-बाधाओं के क्या कभी कोई महान् कार्य हो सकता है? समय, धैर्य तथा अदम्य इच्छा-शक्ति से ही कार्य हुआ करता है। मैं तुम लोगों को ऐसी बहुत-सी बातें बताता, जिससे तुम्हारा हृदय उछल पड़ता, परन्तु मैं ऐसा नहीं करूँगा। मैं तो लोहे के समान दृढ़ इच्छा-शक्ति से सम्पन्न हृदय चाहता हूँ, जो कभी कम्पित न हो। दृढ़ता के साथ लगे रहो।\endnote{ ४/३३९-४०;} 

मैं कायरता से घृणा करता हूँ। कायरों तथा राजनीतिक मूर्खताओं के साथ मैं कोई सम्बन्ध नहीं रखता। मुझे किसी भी प्रकार की राजनीति में विश्वास नहीं है। ईश्वर तथा सत्य ही जगत् में एकमात्र राजनीति है, बाकी सब कूड़ा-करकट है।\endnote{ ४/३४६;} 

पवित्रता, दृढ़ता तथा उद्यम - ये तीनों ही गुण मैं एक साथ चाहता हूँ।\endnote{ ४/३४७;} 

पवित्रता, धैर्य तथा प्रयत्न के द्वारा सारी बाधाएँ दूर हो जाती हैं। निःसन्देह सभी महान् कार्य धीरे-धीरे ही होते हैं।\endnote{ ४/३५१;} 

साहसी बनो और कार्य करो। धैर्य और दृढ़तापूर्वक कार्य - यही एकमात्र मार्ग है।\endnote{ ४/३५६;} 

जो आज्ञा-पालन करना जानते हैं, वे ही आज्ञा देना भी जानते हैं। पहले आदेशपालन करना सीखो। इन पाश्चात्य राष्ट्रों में जैसे स्वाधीनता का भाव प्रबल है, वैसे ही आदेश-पालन करने का भाव भी प्रबल है। हम सभी स्वयं को बड़ा समझते हैं, इसी से कोई काम नहीं बनता। महान् उद्यम, महान् साहस, महान् वीरता और सबसे पहले आज्ञापालन - ये ही गुण व्यक्तिगत अथवा राष्ट्रीय उन्नति के लिए एकमात्र उपाय हैं।\endnote{ ४/३६०;} 

अपने देश में क्या मनुष्य भी हैं? यह तो श्मशान-सदृश है। यदि निम्न श्रेणी के लोगों को शिक्षा दे सको, तो कार्य हो सकता है। ज्ञान से बढ़कर दूसरा कौन-सा बल है? क्या तुम उन्हें शिक्षा दे सकते हो? बड़े लोगों ने किस देश में कभी किसी का उपकार किया है? सभी देशों में मध्यवर्गीय लोगों ने ही महान् कार्य किए हैं। रुपये मिलने में क्या देरी लगती है? मनुष्य कहाँ हैं? क्या हमारे देश में एक भी मनुष्य है?\endnote{ ४/३१५;} 

किसी को उसकी योजना में हतोत्साहित मत करो। निन्दा की प्रवृत्ति पूरी तौर से त्याग दो। जब तक देखो कि लोग सही मार्ग पर जा रहे हैं, तब तक उनके कार्य में सहायता करो; और जब कभी तुम्हें लगे कि वे गलत कर रहे हैं, तो नम्रतापूर्वक उन्हें उनकी गलतियाँ दिखा दो। एक-दूसरे की निन्दा ही सब दोषों की जड़ है। किसी भी संगठन के विनाश में इसका बहुत बड़ा हाथ है।\endnote{ ४/३१५;} 

चालाकी से कोई महान् कार्य नहीं होता। प्रेम, सत्यानुराग और प्रचण्ड ऊर्जा के द्वारा ही सभी कार्य सम्पन्न होते हैं। \textbf{तत्कुरु पौरुषम् } - इसलिए पुरुषार्थ को प्रकट करो।\endnote{ ४/३१७;} 

स्वयं कुछ न करना और यदि दूसरा कोई कुछ करना चाहे, तो उसकी हँसी उड़ाना - यही हमारी जाति का एक महान् दोष है और इसी से हम लोगों का सर्वनाश हुआ है। हृदयहीनता और उद्यम का अभाव ही सब दुःखों का मूल है, अतः इन दोनों को त्याग दो।\endnote{ ४/३८०;} 

पीछे मुड़कर देखने की आवश्यकता नहीं है। आगे बढ़ो! हमें अनन्त शक्ति, अनन्त उत्साह, अनन्त साहस तथा अनन्त धैर्य चाहिए। केवल तभी महान् कार्य सम्पन्न होंगे।\endnote{ ५/४०३;} 

जो सबका दास है, वही उनका सच्चा स्वामी है। जिसके प्रेम में ऊँच-नीच का विचार है, वह कभी नेता नहीं बन सकता। जिसके प्रेम में कोई सीमा नहीं, जो ऊँचनीच सोचने के लिए कभी रुकता नहीं, उसके चरणों में सारा संसार लोट जाता है।\endnote{ ४/४०२-०३;} 

संसार के धर्म प्राणहीन उपहास की वस्तु हो गए हैं। जगत् को जिस चीज की जरूरत है, वह है चरित्र। संसार को ऐसे लोग चाहिए, जिनका जीवन स्वार्थहीन ज्वलन्त प्रेम का उदाहरण हो। वह प्रेम हर शब्द को वज्र के समान प्रभावी बना देगा।\endnote{ ४/४०८;} 

तुम पवित्र और सर्वोपरि निष्ठावान बनो। एक मुहूर्त के लिए भी भगवान के प्रति अपनी आस्था न खोओ, इसी से तुम्हें प्रकाश दिखायी देगा। जो कुछ सत्य है, वही चिरस्थायी रहेगा; जो सत्य नहीं है, उसकी कोई भी रक्षा नहीं कर सकता।\endnote{ ५/३७१;} 

मैं जो चाहता हूँ, वह है लोहे की नसें और फौलाद के स्नायु, जिनके भीतर ऐसा मन वास करता हो, जो वज्र के समान पदार्थ का बना हो। बल, मनुष्यता, क्षात्रवीर्य और ब्रह्मतेज।\endnote{ ५/३९८;} 

आनन्दपूर्वक रहो। अपने आदर्श में स्थिर रहो।... कभी दूसरों को मार्ग दिखाने या उन पर हुक्म चलाने का प्रयत्न मत करना।... शासन मत करो। सबके दास बने रहो।\endnote{ ३/३९०;} 

पीठ पीछे किसी की निन्दा करना पाप है। इससे पूरी तरह बचकर रहना चाहिए। मन में कई बातें आती हैं, परन्तु उन्हें प्रगट करने से तिल का ताड़ बन जाता है। यदि क्षमा कर दो और भूल जाओ, तो उन बातों का अन्त हो जाता है।\endnote{ ३/३९१;} 

एक और सत्य जिसका मैंने अनुभव किया है, वह यह है कि निःस्वार्थ सेवा ही धर्म है और बाह्य विधि-अनुष्ठान आदि केवल पागलपन हैं। यहाँ तक कि अपनी मुक्ति की अभिलाषा करना भी अनुचित है। मुक्ति केवल उसके लिए है, जो दूसरों के लिए सर्वस्व त्याग देता है; परन्तु जो लोग दिन-रात ‘मेरी मुक्ति, मेरी मुक्ति’ की रट लगाये रहते हैं, वे अपने इहलोक तथा परलोक के सच्चे कल्याण को नष्ट करके इधर-उधर भटकते रह जाते हैं।\endnote{ ६/३२७;} 

परोपकार धर्म है; परपीड़न पाप। शक्ति तथा पौरुष पुण्य है; दुर्बलता तथा कायरता पाप। स्वतंत्रता पुण्य है, पराधीनता पाप। दूसरों से प्रेम करना पुण्य है, दूसरों से घृणा करना पाप। परमात्मा में और स्वयं में विश्वास करना पुण्य है, सन्देह करना पाप। एकता का ज्ञान पुण्य है, अनेकता देखना पाप।\endnote{ १०/२२२;} 

डटे रहो, मेरे बहादुर बच्चो। हमने अभी शुरुआत भर की है। निराश मत होना! कभी मत कहना - बहुत हुआ।\endnote{ ५/४००;} 

मैं जितना उन्नत बन सकता था, मैं चाहता हूँ कि मेरे सब बच्चे उससे सौ गुना उन्नत बनें। तुम लोगों में से प्रत्येक को खूब शक्तिशाली बनना होगा। मैं कहता हूँ - अवश्य बनना होगा। आज्ञा-पालन, आलस्यहीनता और ध्येय के प्रति अनुराग - ये तीनों रहें, तो कोई भी तुम्हें अपने मार्ग से डिगा नहीं सकता।\endnote{ ६/३५२;} 

रुपये आदि सब कुछ अपने आप आते रहेंगे। रुपये नहीं, मनुष्य चाहिए। मनुष्य सब कुछ कर सकता है। रुपयें क्या कर सकते हैं? मनुष्य चाहिए - जितने मिलें, उतना ही अच्छा।\endnote{ ६/३६४;} 

जगत् की सारी धनराशि की अपेक्षा ‘मनुष्य’ कहीं अधिक मूल्यवान है।\endnote{ - ४/२८६;} 

वत्स, प्रेम कभी निष्फल नहीं होता। आज या कल या फिर युगों के बाद - परन्तु सत्य की विजय अवश्य होगी। प्रेम की विजय अवश्य होगी। क्या तुम अपने भाई, मनुष्यजाति से प्रेम करते हो? तो फिर ईश्वर को ढूँढ़ने कहाँ चले - क्या ये गरीब, दुखी, दुर्बल लोग ईश्वर नहीं हैं? पहले इन्हीं की पूजा क्यों नहीं करते? गंगा-तट पर कुँआ खोदने क्यों जाते हो? प्रेम की सर्व-शक्तिमत्ता में विश्वास करो। इस झूठे चकाचौंध वाले नाम-यश की परवाह कौन करता है? समाचार-पत्रों में क्या छपता है, इसकी मैं कभी खबर नहीं लेता। क्या तुम्हारे पास प्रेम है? तो फिर तुम सर्व-शक्तिमान हो। क्या तुम पूर्णतः निःस्वार्थ हो? यदि हो, तो फिर तुम्हें कौन रोक सकता है? चरित्र की ही सर्वत्र विजय होती है। प्रभु ही समुद्र के तल में भी अपनी सन्तानों की रक्षा करते हैं। तुम्हारे देश के लिए वीरों की आवश्यकता है। वीर बनो।\endnote{ ३/३२३;} 

मृत्युपर्यन्त काम करो - मैं तुम्हारे साथ हूँ और जब मैं नहीं रहूँगा, तब मेरी आत्मा तुम्हारे साथ काम करेगी।\endnote{ ५/३७१} से अपनी वर्तमान तथा भविष्य में भी आनेवाली समस्याओं का हल निकालना होगा। 

\delimiter

\addtoendnotes{\protect\end{multicols}}

