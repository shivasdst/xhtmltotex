
\chapter{स्वामी विवेकानन्द के जीवन की स्मरणीय घटनाएँ }

नरेन्द्रनाथ की माँ भुवनेश्वरी देवी एक महीयसी नारी थीं। अपने जीवन के परवर्ती काल में स्वामी विवेकानन्द ने अनेकों बार कहा था - “अपने ज्ञान के विकास के लिए मैं अपनी माँ का ऋणी हूँ।” उनकी शिक्षा देने की पद्धति विलक्षण थी। एक घटना के द्वारा इसे समझा जा सकता है - 

\vskip 1.5pt

स्कूल में एक दिन नरेन्द्र को बिना किसी दोष के ही सजा भुगतनी पड़ी। भूगोल की कक्षा में नरेन्द्र ने सही उत्तर दिया था, परन्तु मास्टरजी को लगा कि उत्तर गलत है और इस कारण उन्होंने नरेन्द्र को दण्डित किया। नरेन्द्र ने बारम्बार विरोध जताते हुए कहा - “मुझसे गलती नहीं हुई है, मैं ठीक ही बोल रहा हूँ।” परन्तु इससे मास्टरजी और भी अधिक नाराज हो गए और निर्दयतापूर्वक उन्हें बेंत से पीटने लगे। 

\vskip 1.5pt

नरेन्द्रनाथ घर लौटे और उन्होंने रोते हुए सारी बातें माँ को बतायीं। परन्तु माँ ने उन्हें सांत्वना देते हुए कहा - “बेटा, यदि तुमसे भूल नहीं हुई, तो फिर तुम इस विषय में चिन्ता ही क्यों करते हो? फल चाहे जो भी हो, सभी अवस्थाओं में सर्वदा सत्य को पकड़े रहना।” 

\vskip 1.5pt

नरेन्द्रनाथ को अपने गुरुदेव में माँ के इस उपदेश की जीवन्त प्रतिमूर्ति देखने को मिली। श्रीरामकृष्ण कहते - “हर अवस्था में सत्य का पालन करना चाहिए। इस कलियुग में यदि कोई सत्य को पकड़े रहता है, तो वह निश्चय ही ईश्वर का दर्शन पा लेगा।” श्रीरामकृष्ण जो कुछ कहते, वही उनके आचरण में भी दीख पड़ता था। 

\vskip 1.5pt

स्वामीजी ने अपनी माँ तथा बाद में अपने गुरुदेव में सत्य के प्रति जो अटल निष्ठा देखी, उसे उन्होंने अपने जीवन की प्रत्येक क्रिया में प्रकट किया। परवर्ती काल में जगत् ने उनकी यह स्वाभाविक घोषणा सुनी - “सत्य के लिए सब कुछ का त्याग किया जा सकता है, परन्तु किसी भी चीज के लिए सत्य का त्याग नहीं किया जा सकता।” 

\delimiter

एक बार उनकी माँ ने सलाह दी - “आजीवन पवित्र रहो। अपनी मर्यादा की रक्षा करो और दूसरों को सम्मान प्रदान करो। खूब शान्त रहना, परन्तु आवश्यक होने पर दृढ़ता दिखाने में भी संकोच मत करना।” 

\vskip 1.5pt

इस सदुपदेश ने नरेन्द्र के चरित्र-गठन में बड़ी सहायता की। उन्हें अपने बचपन से ही ज्ञात हो गया कि अपने स्वाभिमान की कैसे रक्षा की जाए। वे दूसरों के प्रति सम्मान दिखाने में कभी आगापीछा नहीं करते; परन्तु साथ ही वे किसी के द्वारा किए गए निरर्थक अपमान को भी सहन नहीं कर पाते थे। 

एक बार नरेन्द्र के पिता के एक मित्र ने अकारण ही उनसे अपमानपूर्ण आचरण किया। इस पर उन्हें बड़ा आश्चर्य हुआ, क्योंकि ऐसा आचरण उन्हें पहले कभी देखने को नहीं मिला था - उनके माता-पिता कभी उन्हें छोटा मानकर उनकी अवज्ञा नहीं करते थे। उन्होंने सोचा - “यह कितने आश्चर्य की बात है! यहाँ तक कि मेरे पिता भी मुझे इतना तुच्छ नहीं मानते।” अतः वे एक घायल नाग की भाँति तनकर खड़े हो गए और दृढ़ स्वर में बोले - “देखिये महाशय, आपके समान अनेक लोग हैं, जो सोचते हैं कि आयु कम होने से बुद्धि भी कम होती है, परन्तु वास्तविकता ऐसी नहीं है।” उन सज्जन ने तत्काल अपनी भूल स्वीकार कर ली। 

कठोपनिषद् में हमें एक अन्य बालक का चरित्र देखने को मिलता है, जो निर्भयता तथा\break स्वाभिमान से परिपूर्ण है। स्वामी विवेकानन्द सर्वदा नचिकेता की प्रशंसा किया करते थे, जिसने कहा था - “बहुत-से लोगों के बीच मैं प्रथम या मध्यम हो सकता हूँ, परन्तु अधम कदापि नहीं हो सकता।” 

\delimiter

नरेन्द्र के पिता खुले-हाथों दान किया करते थे। उनके मँझले पुत्र महेन्द्रनाथ ने लिखा है - “दीन-दुखियों की सहायता करना उनके लिए एक बीमारी के समान हो गया था।” मुहल्ले के लोग उन्हें - दाता विश्वनाथ के नाम से जानते थे। किसी की भी आर्थिक कठिनाई को देखकर वे स्वयं को सँभाल नहीं पाते थे। उनके दूर के सम्बन्धी छात्र उनके घर में रहकर उन्हीं के खर्च पर पढ़ाई-लिखाई करते थे। पड़ोसियों में से किसी के अभाव की बात ज्ञात होते ही वे उसकी सहायता में तत्पर हो जाते थे। इस विषय में वे पूर्णतः बेहिसाबी थे। यहाँ तक कि वे कुछ नशेबाज लोगों की भी आर्थिक सहायता किया करते थे। 

इस ओर नरेन्द्रनाथ का ध्यान जाने पर, उन्होंने धन के इस दुरुपयोग की ओर जब अपने पिता की दृष्टि आकर्षित किया, तो इस पर विश्वनाथ बोले - “बेटा, जीवन कष्टों से परिपूर्ण है। बड़े होने पर तुम स्वयं ही इन बातों का अनुभव करोगे और उन लोगों के प्रति भी करुणा दिखाना सीखोगे, जो जीवन के अनन्त कष्टों से सामयिक रूप से छुटकारा पाने के लिए नशे या किसी अन्य साधन का भी सहारा लेते हैं।” 

नरेन्द्रनाथ को इस सहानुभूति की पूर्ण अभिव्यक्ति श्रीरामकृष्ण के जीवन तथा उपदेशों में देखने को मिली। इसके फलस्वरूप उनकी ‘दया की दृष्टि’ प्रत्येक व्यक्ति तथा प्रत्येक जीव के प्रति ‘प्रेम तथा श्रद्धा’ की दृष्टि में रूपान्तरित हो गयी थी। श्रीरामकृष्ण से उन्होंने सीखा कि केवल दया ही यथेष्ट नहीं है। मनुष्य जीवन्त ईश्वर का एक प्रतिरूप है। क्या हम कभी ईश्वर पर दया दिखाने की बात सोच सकते हैं? नहीं, बल्कि उल्टे हम उनकी सेवा तथा पूजा करके स्वयं को धन्य मानते हैं। अतः इस सन्दर्भ में ‘दया’ शब्द उपयुक्त नहीं है। ‘शिव’ बोध के साथ ‘जीव’ की सेवा करना - ब्रह्म की अभिव्यक्ति के रूप में मानवता की सेवा करना - यही सही दृष्टिकोण है। एक ही ‘नारायण’ चोर-उचक्कों के रूप में भी विद्यमान हैं और भले तथा सुसभ्य लोगों के रूप में भी। 

इस प्रकार स्वामी विवेकानन्द ने अपने गुरुदेव से यह सीखा कि हर व्यक्ति को, यहाँ तक कि पापी-पतितों को भी सम्मान की दृष्टि से देखना चाहिए। वे कहा करते थे - “दुष्टरूपी ईश्वर, पापीरूपी ईश्वर!” वे प्रायः कहते - “जीवन भर दुःख-भोग करने के बाद मैंने यही समझा है कि इस संसार-रूपी नरक-कुण्ड में यदि एक दिन के लिए भी एक व्यक्ति के मन में भी थोड़ा-सा सुख तथा शान्ति उत्पन्न किया जा सके, तो केवल उतना ही सत्य है। 

\delimiter

नरेन्द्र कहानी सुनाने में भी उस्ताद थे। उनके शब्द भी उतने ही आकर्षक थे, जितना कि उनका व्यक्तित्व। उनके व्यक्तित्व तथा बातों में ऐसा आकर्षण था कि एक बार जब वे कोई कथा सुनाना आरम्भ कर देते, तो लोग अपना काम-काज तक भूलकर उनकी बातें सुनने में मशगूल जाते। स्कूल में एक दिन नरेन्द्र अवकाश के समय अपने मित्रों को बड़े सजीव रूप से कुछ बातें सुना रहे थे। इसी बीच शिक्षक ने कक्षा में प्रवेश किया और अपना विषय पढ़ाना आरम्भ कर दिया। परन्तु छात्रगण नरेन्द्र की कथा में इतने डूबे हुए थे कि उन्हें पाठ सुनाई ही नहीं दे रहा था। थोड़ी देर बाद शिक्षक ने खुसफुसाहट की आवाज सुनी और सारी बात समझ ली। उन्होंने नाराज होकर, उस दिन जो कुछ पढ़ाया जा रहा था, उसके विषय में प्रत्येक छात्र से पूछना आरम्भ किया। कोई भी ठीक उत्तर नहीं दे सका। परन्तु नरेन्द्र की प्रतिभा अद्भुत थी, उनका मस्तिष्क एक साथ ही दो स्तरों पर कार्य कर सकता था। अपने मन के एक भाग से बातचीत करते हुए भी, उनके मन का दूसरा भाग पाठ सुनने में लगा हुआ था। अतः जब शिक्षक ने उनसे भी वही प्रश्न किया, तो उन्होंने सही उत्तर बता दिया। 

शिक्षक बड़ी कठिनाई में पड़े। उन्होंने छात्रों से पूछा कि अब तक कौन बातें कर रहा था? सबने नरेन्द्र की ओर इशारा किया, परन्तु शिक्षक ने उन लोगों की बात पर विश्वास करने से इनकार कर दिया। इसके बाद उन्होंने नरेन्द्र को छोड़ बाकी सभी छात्रों को बेंच पर खड़े हो जाने का आदेश दिया। नरेन्द्र भी अपने मित्रों के साथ बेंच पर खड़े हो गए। शिक्षक ने उनसे बैठ जाने को कहा। परन्तु नरेन्द्र ने उत्तर दिया - “नहीं महाशय, मुझे भी खड़े होना पड़ेगा, क्योंकि मैं ही तो उन लोगों के साथ बातें कर रहा था।” 

\delimiter

बी. ए. की परीक्षा के लिए फीस जमा करने का समय आ पहुँचा था। हरिदास चट्टोपाध्याय को छोड़ नरेन्द्र के अन्य सभी सहपाठी फीस जमा कर चुके थे। हरिदास बड़ा गरीब था। उसने किसी प्रकार परीक्षा का शुल्क एकत्र कर लिया था, परन्तु जब तक वह कॉलेज की भी पूरी फीस न जमा कर दे, तब तक अधिकारी उसकी परीक्षा की फीस स्वीकार करने को राजी न थे। इस प्रकार हरिदास की परीक्षा में बैठने की आशा धूल में मिल चुकी थी। 

हरिदास के इस संकट की बात सुनकर नरेन्द्रनाथ को बड़ा सदमा लगा। सहसा उन्हें याद आया कि विशेष परिस्थितियों में कॉलेज-फीस माफ करने की व्यवस्था भी है और कॉलेज के वरिष्ठ क्लर्क राजकुमार बाबू हरिदास का संकट मिटा सकते हैं। इसमें उन्हें आशा की एक किरण दिखाई दी। अतः नरेन्द्र ने अपने मित्र को आश्वस्त किया कि वे यथासम्भव उनकी सहायता करेंगे। 

कुछ दिनों बाद जब छात्रगण अपना बाकी शुल्क तथा परीक्षा-शुल्क जमा कर रहे थे, तभी नरेन्द्र ने राजकुमार बाबू के पास जाकर अनुरोधपूर्वक कहा - “महाशय, हरिदास अपने कॉलेज की फीस देने में असमर्थ है। क्या आप कृपापूर्वक उसकी फीस माफ कर देंगे? यदि आप उसे परीक्षा में सम्मिलित होने दें, तो वह अच्छे अंकों के साथ पास हो जाएगा, नहीं तो उसका जीवन बरबाद हो जाएगा।” 

राजकुमार बाबू क्रोधी स्वभाव के व्यक्ति थे। उन्होंने मुँह बिचकाकर नरेन्द्र से कहा - “तुम्हारी धृष्टतापूर्ण सिफारिश की कोई जरूरत नहीं। अच्छा तो यह होगा कि तुम अपने काम से काम रखो। फीस जमा किए बिना मैं उसे कदापि परीक्षा में नहीं बैठने दूँगा।” इस प्रकार दुतकार दिये जाने पर नरेन्द्र अपने मित्र के विषय में और भी अधिक चिन्तित हो उठे। 

उन्हें परिस्थिति निराशाजनक प्रतीत हुई, क्योंकि वे जानते थे कि वे स्वयं उतनी राशि कदापि जुटा नहीं सकते और उसका तत्काल भुगतान आवश्यक था। ऐसी स्थिति में उन्हें कुछ समझ में नहीं रहा था कि क्या किया जाए! तभी उन्हें एक तरकीब सूझी। 

वे जानते थे कि राजकुमार बाबू अफीम खाने के आदी हैं और प्रतिदिन शाम को नशे की गोली लेने के लिए हेदुआ के एक खास चण्डूखाने में जाया करते हैं। नरेन्द्र उस दिन कॉलेज की छुट्टी होने के बाद घर नहीं लौटे और व्यग्रतापूर्वक संध्या होने की प्रतीक्षा करते रहे। धीरे-धीरे नगर में अन्धकार फैलने लगा। नरेन्द्रनाथ धीरे-धीरे अड्डे के निकट जा पहुँचे और उधर जानेवालों की निगरानी करने लगे। थोड़ी देर बाद नरेन्द्रनाथ ने देखा कि वृद्ध राजकुमार बाबू चोरी-छिपे धीरे-धीरे अफीम के अड्डे की ओर जा रहे हैं। मछली को जाल में आ पहुँची देख, नरेन्द्रनाथ तेजी से राजकुमार बाबू के सामने जा पहुँचे। उन्हें देखते ही बाबू को तो मानो साँप सूंघ गया। नरेन्द्रनाथ ने अवसर देखकर एक बार फिर हरिदास का प्रसंग उठाया और चेतावनी भी दे दी कि यदि वह काम नहीं हुआ, तो उनके नशे की बात पूरे कॉलेज में फैल जाएगी। 

राजकुमार बाबू ने खतरा भाँप लिया और धीमे स्वर में नरेन्द्र से बोले - “अरे भाई, नाराज क्यों होते हो? तुम जो चाहते हो, वह हो जाएगा। क्या मैं कभी तुम्हारे अनुरोध को टाल सकता हूँ?” नरेन्द्रनाथ समझ गए कि उनका काम हो गया है और वे अपने उद्देश्य में सफल रहे हैं। अतः वे अपने घर लौट आए और वे वृद्ध सज्जन उस चण्डूखाने में प्रविष्ट हो गए। 

अगले दिन सुबह सबेरा होने के पूर्व ही नरेन्द्रनाथ हरिदास के घर जा पहुँचे और उससे बोले - “चलो, आनन्द मनाओ। हमारी योजना सफल हुई। अब तुम्हें कॉलेज की अपनी बकाया फीस नहीं भरनी होगी।” इसके बाद नरेन्द्र ने पिछली शाम की घटना का बड़ा सजीव वर्णन किया और दोनों हँसते-हँसते लोटपोट हो गए। 

\vskip -8pt

\delimiter

अगली घटना तब हुई, जब स्वामीजी परिव्राजक के रूप में भ्रमण के दौरान वाराणसी में ठहरे हुए थे। एक दिन दुर्गा-मन्दिर से लौटते समय बन्दरों का एक झुण्ड उनके पीछे पड़ गया। स्वामीजी पहले तो उनसे अपनी जान बचाने के लिए दौड़ने लगे, परन्तु बन्दर उनसे भी अधिक तेज दौड़े और क्रमशः अधिकाधिक आक्रामक होते गए। तभी इस सारी घटना को देख रहे एक वृद्ध संन्यासी ने चिल्लाकर उनसे कहा - “ठहर जाओ। बन्दरों का सामना करो।” स्वामीजी ने उसकी आवाज सुनी और मुड़कर बन्दरों का सामना किया। बन्दरों ने तत्काल उनका पीछा छोड़ दिया। 

इस साधारण-सी घटना से स्वामीजी ने एक महान् शिक्षा ग्रहण की। वे समझ गए कि कोई कठिनाई या संकट आने पर व्यक्ति को भागना नहीं, अपितु उसका साहसपूर्वक सामना करना चाहिए। परवर्ती काल में, न्यूयार्क में एक व्याख्यान देते समय स्वामीजी ने इस घटना का उल्लेख करते हुए कहा था - “समस्त जीवन में हमें यही शिक्षा लेनी होगी - जो कुछ भी भयानक है, उसका सामना करना पड़ेगा, साहसपूर्वक उसके सामने खड़ा होना पड़ेगा। जैसे बन्दरों के सामने से न भागकर, उनका सामना करने पर वे भाग गए थे, उसी प्रकार हमारे जीवन में जो कुछ कष्टप्रद बातें हैं, उनका सामना करने ही पर वे भाग जाती हैं। यदि हमें स्वाधीनता अर्जित करनी हो, तो हम प्रकृति को जीतने पर ही उसे पाएँगे, उससे भागकर नहीं। कापुरुष कभी विजय नहीं पा सकता। हमको भय, कष्ट और अज्ञान के साथ संग्राम करना होगा, तभी वे हमारे सामने से भाग जाएँगे।” 

\vskip -8pt

\delimiter

एक बार किसी ने स्वामीजी से कहा कि संन्यासी को अपने देश के प्रति विशेष लगाव नहीं रखना चाहिए; बल्कि उसे तो प्रत्येक राष्ट्र को अपना ही मानना चाहिए। इस पर स्वामीजी ने उत्तर दिया - “जो व्यक्ति अपनी ही माँ को प्रेम तथा सेवा नहीं दे सकता, वह भला दूसरे की माँ को सहानुभूति कैसे दे सकेगा?” 

स्वामीजी का तात्पर्य यह था कि संन्यासी को भी अपनी मातृभूमि से प्रेम करना चाहिए। जो व्यक्ति अपने ही देश से प्रेम नहीं कर सकता, वह पूरी पृथ्वी को भला कैसे अपना बना सकता है? पहले देशभक्ति; और उसके बाद विश्वप्रेम! 

\vskip -8pt

\delimiter

एक बार स्वामीजी हिमालय की एक लम्बी यात्रा कर रहे थे। उन्होंने देखा कि एक वृद्ध यात्री खूब थका हुआ एक चढ़ाई के नीचे असहाय-सा खड़ा है। वृद्ध ने निराश होकर स्वामीजी से कहा, “महाराज, इतना लम्बा रास्ता कैसे पार कर सकूँगा? मैं तो अब और चल भी नहीं सकता - मेरी तो छाती ही फट जाएगी।” 

स्वामीजी ने धैर्यपूर्वक उस वृद्ध की बातें सुनीं और उसके बाद कहा, “जरा अपने पाँवों की ओर देखिए! पीछे आपके पाँवों के नीचे जो रास्ता है, उसे आप पार कर आए हैं; और उसी रास्ते को आप सामने भी देख रहे हैं। वह भी शीघ्र ही आपके पाँवों तले होगा।” स्वामीजी की यह बात सुनकर वृद्ध की निराशा दूर हो गयी और वे फिर से अपने पथ पर अग्रसर हुए। 

\delimiter

१८८८ ई. के अगस्त में स्वामीजी आगरा से पैदल ही वृन्दावन की ओर चले जा रहे थे। नगर के बाहर निकलते ही उन्होंने एक व्यक्ति को सड़क के किनारे बैठकर चिलम पीते देखा। काफी दूर तक पैदल चलने के कारण स्वामीजी थक गए थे। उनके मन में थोड़ा-सा धूम्रपान तथा विश्राम करने की इच्छा हुई। अतः वे उस व्यक्ति के पास गए और पूछा कि क्या वे भी उसकी चिलम से धूम्रपान कर सकते हैं? यह प्रस्ताव सुनते ही वह व्यक्ति काँप उठा। उसने स्वामीजी से कहा - “महाराज, मुझे खेद है कि मैं अपना जूठा चिलम आपको पीने के लिए नहीं दे सकता, क्योंकि आप संन्यासी हैं और मैं एक अछूत भंगी हूँ!” 

स्वामीजी ने बिना कुछ कहे अपनी यात्रा जारी रखी। परन्तु थोड़ी दूर जाने के बाद ही उनके मन में विद्युत् के समान एक विचार कौंधा - “यह क्या! एक संन्यासी के रूप में क्या मैंने जातिबोध, कुल-अभिमान आदि सहित सब कुछ का त्याग नहीं कर दिया है? यह मेरे लिए कितनी लज्जा की बात है कि मैंने केवल इसलिए उसकी चिलम नहीं पी, क्योंकि वह जाति का भंगी है! इस विचार ने उन्हें इतना बेचैन कर दिया कि वे तत्काल लौटकर उस व्यक्ति के पास जा पहुँचे और उसकी अनिच्छा के बावजूद उससे एक चिलम बनवायी। इसके बाद स्वामीजी उसके पास बैठकर आनन्दपूर्वक धूम्रपान करने लगे। 

\vskip -8pt

\delimiter

स्वामीजी तब वाराणसी में ठहरे हुए थे। १३ अप्रैल, १८९० को उन्हें अपने गुरुभाई बलराम बोस के निधन की दुःखद सूचना मिली। इससे उन्हें इतना सदमा पहुँचा कि वे अपने आँसुओं को रोक नहीं सके। वाराणसी के सुप्रसिद्ध विद्वान् प्रमदादास मित्र का ध्यान इस ओर आकृष्ट हुआ और वे बोले - “स्वामीजी, आप एक संन्यासी हैं। आपका इस प्रकार शोक करना शोभा नहीं देता।” 

प्रमदाबाबू की टिप्पणी सुनकर स्वामीजी के मन को चोट लगी और वे बोल उठे - “आप कहते क्या हैं, प्रमदाबाबू! यह सच है कि मैं संन्यासी हूँ। परन्तु क्या एक संन्यासी के लिए हृदयहीन होना आवश्यक है?” उन्होंने आगे कहा - “देखिए, एक सच्चा संन्यासी सामान्य लोगों से कहीं अधिक कोमल हृदयवाला होता है। जो भी हो, हम मनुष्य तो हैं न! इसके अतिरिक्त बलराम बाबू मेरे गुरुभाई भी थे। मैं किसी ऐसे संन्यास में विश्वास नहीं करता, जो व्यक्ति को संवेदनहीन तथा निर्दय बना देता~है!” 

स्वामीजी प्रायः ही अपने शिष्यों से कहा करते थे कि जो कोई दूसरों का भला करने की चेष्टा नहीं करता, वह संन्यासी कहलाने के योग्य नहीं है। वे बारम्बार कहते कि दूसरों के हितार्थ प्राण देने के लिए, जीवों के गगनभेदी रुदन को मिटाने के लिए, विधवाओं के आँसू पोंछने के लिए, पुत्र-वियोग से पीड़ित माता को सांत्वना प्रदान करने के लिए, अज्ञ जनता को जीवन-संग्राम के उपयुक्त बनाने के लिए, सबका लौकिक तथा पारलौकिक कल्याण करने के लिए और सबको ज्ञान का आलोक देकर उनके भीतर सोये हुए ब्रह्म-भाव को जगाने के लिए ही जगत् में संन्यासी का जन्म हुआ है। 

\vskip -8pt

\delimiter

स्वामीजी की पवहारी बाबा के प्रति बड़ी श्रद्धा थी। उन्होंने बाबाजी का दर्शन भी किया था। गाजीपुर में निवास करते समय स्वामीजी ने सुना कि बाबाजी के पास जो थोड़ी-बहुत चीजें थीं, उन्हें चुराने के लिए एक बार उनकी कुटिया में एक चोर आया था। चोर जब चुराई हुई चीजों के साथ चलने ही वाला था कि बाबाजी की नींद खुल गयी। इससे चोर डर गया और सारी वस्तुएँ वहीं छोड़कर भागने लगा। पवहारी बाबा ने तत्काल वे चीजें उठायीं और चोर के पीछे दौड़ पड़े। काफी दूर तक पीछा करने के बाद आखिरकार उन्होंने चोर को पकड़ ही लिया और उससे उन वस्तुओं को स्वीकार कर लेने का अनुरोध करने लगे। बाबाजी ने चोर से कहा - “नारायण, ये सारी चीजें आपकी ही हैं।” परन्तु चोर अविश्वासपूर्वक उनकी ओर देखता ही रह गया। 

अनेक वर्षों बाद हिमालय में भ्रमण करते समय एक बार स्वामीजी की एक तेजस्वी साधु से भेंट हुई। उनके साथ थोड़ी देर बातचीत के बाद स्वामीजी को विश्वास हो गया कि ये साधु बड़े उच्च कोटि के हैं। परन्तु बाद में यह सुनकर वे विस्मित रह गए, “स्वामीजी, मैं वही चोर हूँ, जिसने बाबाजी की कुटिया में चोरी करने की चेष्टा की थी!” 

साधु ने आगे कहा - “जब पवहारी बाबा ने मुस्कुराते हुए मुझे अपनी सारी चीजें सौंप दी और मुझे ‘नारायण’ कहकर सम्बोधित किया, तो मुझे बोध हुआ कि मैंने कितना जघन्य कार्य किया है और मैं कितना नीच हूँ। उसी क्षण से मैंने सांसारिक सम्पदा के पीछे दौड़ना छोड़ दिया और परमार्थिक धन की खोज में लग गया।” उनकी कथा सुनकर स्वामीजी बड़े प्रभावित हुए और बाद में वे कहा करते थे - “प्रत्येक पापी के भीतर साधुत्व का बीज विद्यमान है।” 

\vskip -8pt

\delimiter

स्वामी विवेकानन्द के परिव्राजक जीवन में, और भी एक रोचक घटना हुई थी। सम्भवतः नवम्बर १८९० ई. के दूसरे सप्ताह में स्वामीजी मेरठ आए और वहीं संयोगवश उनकी भेंट अपने कुछ गुरुभाइयों से हुई, जो अलग से तीर्थयात्रा हेतु निकले हुए थे। जैसा कि स्वाभाविक था, इतने दिनों बाद वे लोग एक-दूसरे से मिलकर बड़े प्रसन्न थे। वे लोग वहाँ एक साथ रहने लगे और वह स्थान मानो एक द्वितीय वराहनगर मठ में रूपान्तरित हो गया। 

जैसा कि सर्वविदित है - स्वामीजी पुस्तकों के बड़े प्रेमी थे। मेरठ में उन्होंने अपना बहुत-सा समय ग्रन्थ पढ़ने में ही बिताया। उनके कहने पर स्वामी अखण्डानन्द प्रतिदिन एक स्थानीय पुस्तकालय में जाकर सर जॉन लबक की ग्रन्थावली का एक बड़ा-सा खण्ड ले आते और अगले दिन उसे लौटाकर उसके बाद का खण्ड ले आते। 

ग्रन्थपाल ने सोचा कि स्वामीजी पुस्तकें पढ़ नहीं रहे हैं, बल्कि केवल दूसरों पर अपना रुआब जमाने का प्रयास कर रहे हैं। एक दिन स्वामी अखण्डानन्द जब लबक का अगला खण्ड लेने गए, तो ग्रन्थपाल ने उनके समक्ष अपनी शंका व्यक्त की। 

अखण्डानन्दजी ने लौटकर यह बात स्वामीजी को बतायी। इसे सुनकर स्वामीजी एक दिन स्वयं ही पुस्तकालय में उपस्थित हुए और विनम्रता के साथ ग्रन्थपाल से बोले - “महाशय, मैंने बड़े ध्यानपूर्वक इन ग्रन्थों को पढ़ा है। यदि आपको इस विषय में कोई शंका हो, तो आप अपनी इच्छानुसार इन पुस्तकों में से कहीं से कोई भी प्रश्न पूछ सकते हैं।” 

इस पर ग्रन्थपाल ने स्वामीजी से अनेक प्रश्न पूछे और स्वामीजी ने उन सभी प्रश्नों के सटीक उत्तर दिये। ग्रन्थपाल आश्चर्यचकित हो उठा। उसे अपने जीवन में कभी स्वामीजी जैसा व्यक्ति देखने को नहीं मिला था। 

खेतड़ी के राजा भी स्वामीजी के पढ़ने की पद्धति देखकर बड़े विस्मित हुए थे। स्वामीजी कोई भी पुस्तक हाथ में लेकर शीघ्रतापूर्वक उसके शुरू से अन्त तक के पृष्ठ पलटते जाते; बस, इतने से ही उनकी पढ़ाई पूरी हो जाती। राजा ने जब पूछा कि यह भला कैसे सम्भव है! तो स्वामीजी ने बताया कि जैसे एक शिशु पढ़ना सीखता है, तो सर्वप्रथम वह किसी शब्द के किसी विशेष अक्षर पर अपना ध्यान एकाग्र करता है। उसका दो-तीन बार उच्चारण करता है, फिर अगले अक्षर के साथ भी ऐसा ही करने के बाद वह पूरे शब्द का उच्चारण करता है। जब वह पढ़ने की कला में और भी आगे बढ़ता है, तो वह प्रत्येक शब्द पर ध्यान देता है। फिर काफी अभ्यास के बाद वह दृष्टि मात्र डालकर ही पूरा वाक्य पड़ लेता है। इसी प्रकार यदि कोई अपनी एकाग्रता की शक्ति बढ़ाता जाए, तो वह पलक झपकते ही एक पूरा पृष्ठ पढ़ सकता है। स्वामीजी ने बताया कि वे ऐसा ही करते हैं और साथ ही यह भी बताया कि इसके लिए ब्रह्मचर्य, अभ्यास तथा एकाग्रता की आवश्यकता है। इन तीन चीजों का आश्रय लेकर कोई भी व्यक्ति यह क्षमता अर्जित कर सकता है। 

बेलगाम के वन-अधिकारी हरिपद मित्र भी इसी प्रकार आश्चर्यचकित हुए थे। एक बार उनके साथ बातचीत के दौरान स्वामीजी ने अपनी याददाश्त से चार्ल्स डिकेंस द्वारा रचित ‘पिकविक पेपर्स’ ग्रन्थ का एक बड़ा अंश उद्धृत किया। इसके बाद, यह सुनकर तो मित्र महाशय के विस्मय की सीमा ही न रही कि स्वामीजी ने वह ग्रन्थ केवल दो बार ही पढ़ा है। स्वामीजी ने उन्हें बताया कि एकाग्रता तथा ब्रह्मचर्य के पालन से बुद्धि तीक्ष्ण हो जाती है। 

एक बार स्वामीजी अस्वस्थ थे और बेलूड़ मठ में निवास कर रहे थे। एक दिन उनके शिष्य शरत् चन्द्र चक्रवर्ती उनसे मिलने उनके कमरे में आए और देखा कि वहाँ ब्रिटानिका विश्वकोष का एक नया सेट रखा हुआ है। शरत् चन्द्र ने उन विशाल चमचमाते ग्रन्थों को देखकर कहा, “इतनी सारी पुस्तकें तो एक जीवन में पढ़ पाना कठिन है।” परन्तु स्वामीजी इस बात से सहमत नहीं हुए। वे बोले, “क्या कहता है? इन दस खण्डों को मैं पढ़ चुका हूँ; इनमें से तू जहाँ से चाहे, पूछ ले।” 

शिष्य उन खण्डों में से चुन-चुनकर कठिन प्रश्न करने लगा और स्वामीजी के मुख से उन सबके उपयुक्त उत्तर सुनकर उसके विस्मय का ठिकाना न रहा। इतना ही नहीं अनेक स्थानों पर स्वामीजी ने उन खण्डों से लम्बे उद्धरण भी दिए। अन्त में शिष्य में आत्मविश्वास जाग्रत करने के उद्देश्य से स्वामीजी बोले, “देख, ब्रह्मचर्य का ही ठीकठीक पालन करने से सारी इच्छित विद्याएँ प्राप्त हो जाती हैं और व्यक्ति में आसानी से ऐसी स्मृति आ जाती है।” उनकी विलक्षण स्मरण-शक्ति पर जर्मन दार्शनिक पॉल डॉयसन के भी विस्मित हो जाने पर स्वामीजी ने उन्हें भी यही बताया था। 

\delimiter

१८९१ ई. में स्वामीजी माउंट आबू में एक मुसलमान वकील के अतिथि के रूप में निवास कर रहे थे। खेतड़ी-राजा के नीजी सचिव मुंशी जगमोहन लाल एक दिन वकील साहब के बँगले पर आए और स्वामीजी को उनके यहाँ ठहरे देखकर बड़े चकित हुए। जगमोहन लाल अपने विस्मय को छिपाये रखने में असमर्थ होकर बोल उठे - “स्वामीजी, आप एक हिन्दू संन्यासी हैं। आप एक मुसलमान के यहाँ कैसे निवास कर रहे हैं?” 

स्वामीजी के लिए किसी भी प्रकार का जाति-पाति या धर्म-विषयक भेदभाव सह पाना असम्भव था; अतः उन्होंने कठोर स्वर में उत्तर दिया - “महाशय, आप कहना क्या चाहते हैं? मैं एक संन्यासी हूँ। मैं आपकी समस्त सामाजिक मान्यताओं के परे हूँ। यहाँ तक कि मैं भंगी - तथाकथित अछूत से साथ भी भोजन कर सकता हूँ। मैं ईश्वर से नहीं डरता, क्योंकि वे इसकी अनुमति देते हैं; मैं शास्त्रों से भी नहीं डरता, क्योंकि वे भी इसके लिए अनुमति देते हैं; परन्तु मैं आप लोगों और आपके समाज से डरता हूँ। आप लोग ईश्वर तथा शास्त्रों के विषय में कुछ भी नहीं जानते। मैं तो सर्वत्र - यहाँ तक कि सबसे छुद्र कीट तक में ब्रह्म की ही अभिव्यक्ति देखता हूँ। मेरी दृष्टि में कुछ भी ऊँचा या नीचा नहीं है। शिव! शिव!!” स्वामीजी के प्रत्येक शब्द से आग बरस रही थी और मुंशी जगमोहन लाल सम्मोहित होकर खड़े उनके अद्भुत व्यक्तित्व को देखते रह गए। 

\delimiter

उस काल की परम्परा के अनुसार एक दिन खेतड़ी के राजा ने एक नर्तकी के गायन का आयोजन किया। उन दिनों स्वामीजी भी राजा अजीतसिंह के मेहमान के रूप में खेतड़ी के राजमहल में ठहरे हुए थे। राजाजी ने स्वामीजी को भी गायन सुनने के लिए निमंत्रण भेजा, परन्तु स्वामीजी ने आने से मना करते हुए यह सन्देश भेजा कि संन्यासी के लिए ऐसे आयोजन में भाग लेना उसके आदर्शों के प्रतिकूल है। 

गायिका को यह सुनकर बड़ा दुःख हुआ। उसने मानो इसके उत्तर में ही सुप्रसिद्ध वैष्णव सन्त सूरदास का एक भजन गाना आरम्भ किया। शाम की मन्द वायु के साथ नर्तकी के वेदनापूर्ण भजन की मधुर स्वर-लहरी स्वामीजी के संवेदनशील कानों तक जा पहुँची। उस भावपूर्ण भजन के शब्द इस प्रकार थे -

\begin{verse}
 प्रभु मोरे अवगुन चित न धरो,\\
 समदरशी है नाम तिहारो चाहे तो पार करो।\\
 इक लोहा पूजा में राखत, इक घर बधिक परो,\\
 पारस गुन अवगुन नहीं चितवे, कंचन करत खरो।\\
 इक नदिया इक नार कहावत, मैलो नीर भरो,\\
 जब दोनों मिल एक बरन भये, सुरसरि नाम परो।\\
 स्वामी विवेकानन्द के जीवन की स्मरणीय घटनाएँ\\
 इक माया इक ब्रह्म कहावत सूर-श्याम झगरो,\\
 अज्ञान से भेद होवे, ज्ञानी काहे भेद करो।
\end{verse}

पूरे भजन के द्वारा यही भाव व्यक्त हुआ कि ईश्वर प्रत्येक व्यक्ति तथा वस्तु में ओतप्रोत हैं। इसे सुनकर स्वामीजी भावविभोर हो उठे। वे स्वयं को धिक्कारते हुए कह उठे - “यह मैं कैसा संन्यासी हूँ, जो एक नारी तथा स्वयं के बीच ऐसा भेदभाव रखता हूँ?” इस भजन से उन्हें एक महान् शिक्षा मिली। स्वामीजी को बोध हुआ कि सभी के भीतर एक ही आत्मा निवास करती है, अतः वे किसी का तिरस्कार नहीं कर सकते। वे तत्काल दौड़ते हुए वहाँ जा पहुँचे, जहाँ भजन चल रहा था। कहते हैं कि स्वामीजी ने उस नर्तकी को ‘माँ’ के रूप में सम्बोधित किया और अपने अनुचित व्यवहार के लिए उससे क्षमा-याचना की। 

\delimiter

एक संन्यासी सामाजिक मान्यताओं के परे एक चिरमुक्त आत्मा होता है। वह एक नदी के समान निरन्तर चलता रहता है। कभी वह अपनी रात श्मशान में बिताता है, तो कभी एक महाराजा के महल में सोता है और कभी वह रेलवे स्टेशन में रात गुजारता है; परन्तु वह हर हालत में प्रसन्न रहता है। स्वामीजी एक ऐसे ही संन्यासी थे, जो इस समय राजस्थान के एक रेलवे स्टेशन पर ठहरे हुए थे। पूरे दिन भर उनके पास आनेवाले लोगों का ताँता लगा रहता था। वे लोग उनसे धर्म-विषयक तरह-तरह के प्रश्न पूछते और स्वामीजी अथक भाव से उनके उत्तर देते रहते। इसी प्रकार तीन दिन तथा तीन रात बीत गए। स्वामीजी आध्यात्मिक चर्चा में ऐसे डूबे हुए थे कि कुछ खाने के लिए भी उन्होंने अपनी वाणी को विराम नहीं दिया। जो लोग उनके उपदेश सुनने आए थे, उन लोगों के मन में भी यह विचार नहीं आया कि स्वामीजी से पूछें कि उनका भोजन हुआ है या नहीं। 

उनके वहाँ निवास की तीसरी रात, जब सभी आगन्तुक लौट चुके थे, तभी एक निर्धन व्यक्ति सामने आया और उनसे बड़े प्रेम के साथ बोला - “स्वामीजी, मैंने देखा कि पिछले तीन दिनों से आप बातें ही किए जा रहे हैं। इस दौरान आपने जल की एक बूँद तक ग्रहण नहीं किया! इससे मेरे मन को बड़ी पीड़ा हुई है।” 

स्वामीजी को ऐसा महसूस हुआ मानो ईश्वर स्वयं ही उनके समक्ष उस निर्धन व्यक्ति के रूप में अवतीर्ण हुए हों। उन्होंने उसकी ओर देखा और बोले - “क्या आप मुझे कुछ खाने को देंगे?” वह व्यक्ति जाति का मोची था, इसलिए कुछ हिचकिचाहट के साथ बोला - “स्वामीजी, मेरा हृदय आपको कुछ रोटियाँ खिलाने को आतुर है, परन्तु मैं उन्हें कैसे दे सकता हूँ? वे मेरे द्वारा स्पर्श की हुई हैं! यदि आप अनुमति दें, तो मैं आपके लिए थोड़ा-सा कच्चा आटा तथा दाल ले आऊँ, आप स्वयं अपनी इच्छानुसार उसे पका लें।” 

स्वामीजी ने कहा - “नहीं, मेरे भाई, तुमने जो रोटियाँ बनायी हैं, वे ही मुझे दे दो। मैं उन्हें बड़े आनन्द से खाऊँगा।” वह गरीब आदमी यह सुनकर डर गया। उसे भय हुआ कि यदि राजा को पता चला कि इस अस्पृश्य व्यक्ति ने एक संन्यासी को अपना खाना दिया है, तो वे उसके लिए कठोर दण्ड की व्यवस्था कर सकते हैं। परन्तु एक संन्यासी की सेवा करने की आकांक्षा ने उसके भय को परे धकेल दिया। वह शीघ्रतापूर्वक घर गया और स्वामीजी के लिए ताजी बनी हुई रोटियों के साथ वापस लौटा। इस निर्धन व्यक्ति की दयालुता तथा निःस्वार्थ प्रेम को देखकर स्वामीजी की आँखों में आँसू आ गए। उन्होंने सोचा कि ऐसे कितने ही उदार लोग अज्ञात रूप से हमारे देश की झोपड़ियों में निवास करते हैं। ये लोग आर्थिक दृष्टि से निर्धन और तथाकथित निम्न जातियों में जन्मे हैं, तथापि ये कितने महान् तथा उदार हृदयवाले हैं! 

इसी बीच कुछ सज्जनों ने देखा कि स्वामीजी चर्मकार का दिया हुआ भोजन खा रहे हैं, तो वे इस पर खूब चिढ़ गए। वे लोग स्वामीजी के पास आकर बोले कि उनके लिए निम्न जाति के व्यक्ति द्वारा प्रदत्त भोजन ग्रहण करना अनुचित है। स्वामीजी ने धैर्यपूर्वक उनकी बातें सुनी और उसके बाद कहने लगे - “आप लोगों ने मुझे पिछले तीन दिनों से निरन्तर अविराम बोलने को मजबूर किया, परन्तु आपने मुझसे इतना तक नहीं पूछा कि मेरा भोजन तथा विश्राम हुआ है या नहीं। इसके बावजूद आप लोगों का दावा है कि आप लोग ऊँची जाति के और सज्जन व्यक्ति हैं। इससे भी बढ़कर शर्म की बात तो यह है कि आप इसे अछूत कहकर इसका अपमान कर रहे हैं। क्या आप लोग इसके भीतर से प्रकट हुई उदारता की अनदेखी करते हुए निर्लज्ज भाव से इसका तिरस्कार कर सकते हैं?” 

\delimiter

उच्च कोटि के सन्तों तथा महापुरुषों के मन में भी कभी-कभी आध्यात्मिक हताशा का भाव दीख पड़ता है। स्वामीजी भी इसके अपवाद न थे। जब वे परिव्राजक का जीवन बिताते हुए भ्रमण कर रहे थे, तभी एक दिन यह सोचकर उनके मन में घोर निराशा उत्पन्न हुई कि जीवन के जिस लक्ष्य की प्राप्ति के लिए मैंने संसार का त्याग किया था, अहा, वह ईश्वर-दर्शन मुझे अब भी प्राप्त नहीं हो सका! उन्होंने मन-ही-मन सोचा - “अब इस जीवन की मेरे लिए कोई उपयोगिता नहीं;” और निश्चय किया कि - अब मैं निराहार रहकर ध्यान करते हुए इस शरीर को सूखी पत्ती के समान त्याग दूँगा। 

इसी उद्देश्य को ध्यान में रखकर वे एक घने जंगल में प्रविष्ट हुए और बिना कुछ खाये-पीये दिन भर चलते रहे। संध्या के बाद वे थकान के कारण चलने में असमर्थ हो गए और एक वृक्ष के नीचे लेट गए। थोड़ी देर बाद उन्होंने देखा कि एक बाघ छिपते हुए उन्हीं की ओर अग्रसर हो रहा है। वे उसी प्रकार चुपचाप लेटे हुए सोचने लगे - “चलो, एक भूखे प्राणी का शिकार बनकर कम-से-कम उसकी सेवा का मौका तो मिला, क्योंकि ऐसा तो नहीं लगता कि मैं कभी जगत् का कोई अन्य उपकार कर सकूँगा।” इसके बाद वे शान्तिपूर्वक प्रतीक्षा करने लगे कि बाघ आकर उन पर झपट्टा मारे। बाघ स्वामीजी के काफी पास तक आया, परन्तु किसी अज्ञात कारणवश वह पीछे मुड़ा और जंगल के अँधियारे में गायब हो गया। स्वामीजी को अपनी आँखों पर विश्वास नहीं हुआ! वे सोचते रहे कि बाघ फिर लौटेगा। उसी की प्रतीक्षा में उन्होंने सारी रात उसी वृक्ष के नीचे बितायी, परन्तु वह फिर नहीं आया। 

\delimiter

उन दिनों परिव्राजक के रूप में भ्रमण कर रहे स्वामीजी के तेजोमय व्यक्तित्व की ओर आकृष्ट होनेवाले विभिन्न रियासतों के राजाओं की ही भाँति, मैसूर के महाराजा चामराजेन्द्र वाडियार भी उनके सम्पर्क में आए और अल्प काल के भीतर ही दोनों के बीच बड़ी घनिष्ठता हो गयी। एक दिन महाराजा ने कहा, “स्वामीजी, मेरे दरबारियों के विषय में आपकी क्या राय है?” दरबारी लोग भी वहीं उपस्थित थे, परन्तु स्वामीजी सर्वदा सत्य बोलने के अभ्यस्त थे। अतः वे बेहिचक बोले, “दरबारी जैसे सर्वत्र होते हैं, वैसे ही यहाँ भी हैं।” 

महाराजा ने स्वामीजी का तात्पर्य समझ लिया। उन्होंने मन-ही-मन यह मान लिया कि स्वामीजी का कहना सत्य है। तो भी उन्होंने बाहर से स्वामीजी के साथ पूर्ण सहमति नहीं जतायी और धीमे स्वर में अपने दरबारियों का पक्ष लेते हुए बोले, “नहीं, नहीं, स्वामीजी, कम-से-कम मेरे दीवान तो वैसे नहीं हैं। ये तो बड़े ही बुद्धिमान और विश्वस्त हैं।” स्वामीजी ने उत्तर दिया - “परन्तु महाराज, दीवान ही वह व्यक्ति है, जो राजा को लूटकर अंग्रेज सरकार के राज-प्रतिनिधि की जेब भरता है।” 

महाराजा ने देखा कि स्वामीजी कुछ ज्यादा ही कह जा रहे हैं; अतः उन्होंने बातचीत का विषय बदल दिया और थोड़ी देर बाद स्वामीजी को अपने भीतरी कक्ष में ले जाकर बोले - “स्वामीजी महाराज, इतनी स्पष्टवादिता सर्वदा सुरक्षित नहीं होती। मेरे दरबारियों के समक्ष आप जैसे बोल रहे थे, यदि वैसे ही बोलते रहे, तो मुझे आशंका है कि कोई आपको जहर भी दे सकता है।” स्वामीजी दहाड़ उठे - “क्या? तो आप ऐसा सोचते हैं कि एक सच्चा संन्यासी अपने प्राणों के भय से सत्य नहीं बोलेगा? महाराज, मान लीजिए कि कल ही आपका पुत्र मुझसे पूछे, ‘स्वामीजी, मेरे पिता के विषय में आपके क्या विचार हैं?’ तो क्या मैं आपके उन सारे गुणों का बखान करूँगा, जो मैं निश्चित रूप से जानता हूँ कि आपमें हैं ही नहीं? क्या मैं इस विषय में झूठ बोलूँगा? ऐसा कदापि नहीं हो सकता!” 

अपनी स्पष्टवादिता के बावजूद स्वामीजी कभी किसी के पीठ-पीछे उसकी निन्दा नहीं करते थे। यदि वे किसी का कोई दोष आवश्यक भी समझते, तो वे स्वयं उसी व्यक्ति को बताते। उसकी अनुपस्थिति में, वे उसके चरित्र के नकारात्मक पहलुओं को भुलाकर, केवल उसके गुणों की प्रशंसा ही किया करते थे। 

\vskip -8pt

\delimiter

स्वामी तुरीयानन्द (हरि महाराज) ने एक घटना का वर्णन किया है। उस समय स्वामीजी अमेरिका-यात्रा के लिए मुम्बई जा रहे थे और तुरीयानन्दजी स्वामी ब्रह्मानन्द के साथ उनसे मिलने आबूरोड स्टेशन पर आए हुए थे। उस समय स्वामीजी ने कहा था, “हरिभाई, मैं अब भी तुम्हारे तथाकथित धर्म को समझने में असमर्थ हूँ।” इतना कहने के बाद वे भावुकता के आवेग में काँप उठे और उनके चेहरे पर गहन वेदना का भाव झलक उठा। लेकिन उन्होंने शीघ्र ही स्वयं को सँभाल लिया और अपने हाथ को सीने पर रखते हुए बोले, “परन्तु मेरा हृदय काफी विशाल हो गया है और मैं दूसरों के कष्टों को अनुभव करना सीख गया हूँ! विश्वास करो, जब मैं किसी को कष्ट पाते देखता हूँ, तो सच कहता हूँ मेरा हृदय पीड़ा से अभिभूत हो जाता है!” 

स्वामीजी पुनः भावविभोर हो उठे और आँसुओं से उनके गाल भीग गए। स्वामी तुरीयानन्द यह देखकर हक्के-बक्के रह गए। वे सोचने लगे, “क्या बुद्ध ने भी ऐसा ही अनुभव नहीं किया था! क्या उनके मुख से भी प्रेम के ये ही शब्द नहीं निकले थे!” उन्होंने स्पष्ट रूप से देखा कि करोड़ों लोगों की असीम पीड़ा स्वामीजी के हृदय में तूफान मचा रही है। उन्हें ऐसा लगा कि स्वामीजी का हृदय मानो एक ऐसा पात्र है, जिसमें मानव-जाति के सारे दुःखों को शान्त करने के लिए एक मलहम तैयार हो रहा है।” 

\delimiter

स्वामीजी में स्वाभिमान की गहन भावना थी और वे चाहते थे कि इसी प्रकार हर भारतवासी अपने स्वाभिमान के प्रति सचेत हो। आगे वर्णित होनेवाली घटना, जो उनके अमेरिका जाने के पूर्व घटी थी, इसी बात को भलीभाँति प्रस्तुत करती है। स्वामीजी तथा मुंशी जगमोहन लाल आबूरोड स्टेशन पर रेलगाड़ी के डिब्बे में बैठे ट्रेन के मुम्बई रवाना होने का इन्तजार कर रहे थे। स्वामीजी को विदा करने आए हुए उनके एक बंगाली मित्र भी उसी डिब्बे में बैठे हुए थे। तभी एक अंग्रेज टिकट-निरीक्षक आया और कटु शब्दों में बोला कि वे सज्जन ट्रेन से नीचे उतर जाएँ। वे सज्जन भी रेल-विभाग के ही कर्मचारी थे। उन्होंने टिकट-निरीक्षक को समझाने का प्रयास किया कि उन्होंने कुछ भी नियम-विरुद्ध नहीं किया है। परन्तु टिकट-निरीक्षक अपनी बात पर अड़ा रहा। इसके फलस्वरूप दोनों के बीच गर्मागर्म बहस छिड़ गयी, जिसमें बाद में स्वामीजी को भी हस्तक्षेप करना पड़ा। परन्तु उस अंग्रेज ने उन्हें कोई साधारण साधु समझा और अपमानसूचक शब्दों में बोला, “तुम काँहें बात करते~हो?” 

स्वामीजी यह ‘तुम’ का सम्बोधन सुनकर भड़क गए और अंग्रेजी में बोले, “यह ‘तुम’ किसे कहते हो? क्या तुम नहीं जानते कि उच्च श्रेणी के यात्रियों के साथ कैसे बात करनी चाहिए? ‘आप’ क्यों नहीं कहा?” अपनी गलती को समझकर टिकट-निरीक्षक बोला, “मुझे खेद है कि मैं यह भाषा ठीक से नहीं जानता। मैं तो केवल चाहता था कि यह आदमी...~।” स्वामीजी ने उसकी बात बीच में ही काटते हुए कहा, “अभी-अभी तुमने कहा कि तुम हिन्दी भाषा नहीं जानते; अब देखता हूँ कि तुम अपनी मातृभाषा भी नहीं जानते। जिन्हें तुम ‘आदमी’ कह रहे हो, वे एक ‘सज्जन’ है। मैं इस अनादरपूर्ण आचरण की शिकायत तुम्हारे ऊपरवाले अधिकारी से करूँगा।” इस पर वह टिकटनिरीक्षक डर गया और शीघ्रतापूर्वक उस डिब्बे से चलता बना। 

टिकट-निरीक्षक के जाते ही स्वामीजी जगमोहन लाल की ओर उन्मुख हुए और बोले, “इन यूरोपियनों के साथ व्यवहार करते समय हमें अपने स्वाभिमान का ध्यान रखना चाहिए। हमें भी दूसरों के समान ही अपनी प्रतिष्ठा तथा स्तर के विषय में सचेत रहना चाहिए। दुर्भाग्यवश हम ऐसा नहीं करते और इसी के फलस्वरूप इन लोगों को हमें नीचा दिखाने का मौका मिल जाता है। हमें हर कीमत पर अपने स्वाभिमान की रक्षा करनी होगी, अन्यथा हमें पग-पग पर अपमान सहना पड़ेगा। ध्यान रहे, कायरता से ही सभी प्रकार के दुराचरण तथा बुराइयों को बढ़ावा मिलता है। सभ्यता में हिन्दू लोग दुनिया की अन्य किसी भी जाति से कम नहीं हैं, परन्तु वे अपने को अत्यन्त हीन मानते हैं। यही कारण है कि कोई भी ऐरो-गैरो हमें अपमानित करने का साहस करता है और हम लोग उस अपमान का घूंट चुपचाप पी जाते हैं।” 

इस प्रकार स्वामीजी ने भारतवासियों को अपने देश के प्रति प्रेम तथा सम्मान का भाव रखना सिखाया। उन्हें पूर्ण विश्वास था कि भारत को पाश्चात्य भौतिकवाद सीखने की जितनी आवश्यकता है, उससे कहीं अधिक दुनिया को भारत के आध्यात्मिक खजाने की जरूरत है। इस विश्वास के कारण ही वे कभी पाश्चात्य सभ्यता की चमक-दमक भरी समृद्धि से भ्रमित नहीं हुए और न ही वे कभी क्षण भर के लिए भी हीन भावना से ग्रस्त हुए। इसी अटल निष्ठा के बल पर उन्होंने भारतीय सभ्यता तथा संस्कृति की महिमाघोषणा की। इसमें कोई सन्देह नहीं कि उनके बल, साहस तथा आत्मविश्वास ने पश्चिमी जगत् के हजारों लोगों को भारत तथा इसकी संस्कृति से प्रेम करने की प्रेरणा दी। स्वामीजी ने उस ऐतिहासिक धर्म-महासभा को किस भाव से देखा था, इस विषय में डॉ. एनी बेसेंट के शब्दों में एक नयी अन्तर्दृष्टि प्राप्त होती है। 

डॉ. बेसेंट ने लिखा था - “शिकागो के धूम्रमलिन क्षितिज पर भारतीय सूर्य के समान दीप्तिमान, सिंह के समान उन्नत सिर, अन्तर्भेदी दृष्टि, चंचल ओष्ठद्वय, मनोहर तथा द्रुत चाल, गैरिक वस्त्रों में विभूषित एक महिमामय मूर्ति - ऐसी हुई स्वामी विवेकानन्द के बारे में मेरी धारणा, जब मैं महासभा के प्रतिनिधियों के लिए निर्धारित कमरे में पहली बार उनसे मिली। वे एक संन्यासी के रूप में परिचित थे, जो कि उचित ही था, परन्तु वे थे एक योद्धा संन्यासी; और पहली भेंट के समय वे मुझे संन्यासी की अपेक्षा योद्धा ही अधिक प्रतीत हुए थे; क्योंकि जब वे प्राचीनतम जीवित धर्म के प्रतिनिधि के रूप में मंच पर से उतरकर आते,... तब उनके अंग-अंग से देश तथा जाति का गर्व फूट पड़ता-सा प्रतीत होता था। चंचल, तेज और उद्धत पश्चिम में अपने इस सन्देशवाहक सन्तान को भेजकर भारत को लज्जित होने का कोई कारण नहीं। वे भारतवर्ष का सन्देश लेकर आए थे और भारत के नाम पर ही उन्होंने उसका प्रचार किया। जिस महिमा-मण्डित देश से प्रतिनिधि के रूप में वे आए थे, उसकी मर्यादा का उन्हें सदैव भान रहता था। वे उद्यमशील, शक्तिमान और अपने उद्देश्य में अटल थे; वे मनुष्यों के बीच एक मनुष्य के रूप में अपना सिर ऊँचा करके खड़े होते थे और अपने मतों का समर्थन करने की क्षमता उनमें सर्वदा विद्यमान थी।” 

\delimiter

धर्म-महासभा में स्वामीजी के प्रथम व्याख्यान ने ही उन्हें इतना प्रसिद्ध बना दिया कि शिकागो नगर के सर्वाधिक प्रतिष्ठित लोगों ने उन्हें अपने घर पर आमंत्रित करना आरम्भ कर दिया। प्रत्येक व्यक्ति उन्हें अपना अतिथि बनाना चाहता था। 

महासभा के पहले दिन का अधिवेशन पूरा हो जाने के बाद स्वामीजी एक करोड़पति के भवन में ले जाए गए और वहाँ उनके सम्मान में राजोचित सत्कार का आयोजन किया गया। मेजबान ने स्वामीजी की सुविधा के लिए सारे उपाय किए, परन्तु स्वामीजी को न तो नाम-यश की आकांक्षा थी और न ही वे शारीरिक सुख-सुविधा के इच्छुक थे। अतः इस भव्यता, तड़क-भड़क तथा अमेरिकी लोगों के स्वाभाविक प्रशंसा भाव के बीच स्वामीजी को बेचैनी महसूस होने लगी। वे यह बात नहीं भूल पा रहे थे कि उनके देशवासी कितने कष्टों के बीच जीवन-यापन कर रहे हैं। उनका हृदय भारत के लिए क्रन्दन करता रहा और उन्हें विलासितापूर्ण बिस्तर पर नींद नहीं आयी। बिस्तर से उतरकर वे फर्श पर लेट गए और सारी रात एक शिशु के समान रोते रहे। उन्होंने प्रार्थना की - “हे माँ, मैं इस नाम-यश को लेकर क्या करूँ, जबकि मेरे देशवासी घोर निर्धनता में डूबे हुए हैं! हम गरीब भारतवासी ऐसी बुरी हालत तक पहुँच गए हैं कि मुट्ठी भर अन्न के अभाव में लाखों लोग प्राण त्याग देते हैं और यहाँ लोग अपने व्यक्तिगत सुख-सुविधा के लिए लाखों रुपये खर्च करते हैं! भारत की जनता को कौन उठाएगा? कौन उनके मुख में अन्न देगा? मैं उनकी किस प्रकार सेवा कर सकता हूँ?” स्वामीजी का भारत के प्रति ऐसा ही ज्वलन्त प्रेम था। 

स्वामी विवेकानन्द के पाश्चात्य देशों से लौटने के बाद बेलूड़ मठ में भी एक ऐसी ही घटना हुई थी। स्वामीजी के एक गुरुभाई स्वामी विज्ञानानन्द उन दिनों मठ में ही ठहरे हुए थे। स्वामीजी का उनके प्रति बड़ा लगाव था और वे उन्हें स्नेहपूर्वक ‘पेशन’ कहकर पुकारा करते थे, क्योंकि मठ में संन्यास लेने के पूर्व उनका नाम हरिप्रसन्न था। विज्ञानानन्दजी स्वामीजी के पास वाले कमरे में निवास करते थे। एक रात सुबकने की आवाज सुनकर उनकी नींद खुल गयी और वे कारण जानने के लिए स्वामीजी के कमरे की ओर दौड़े। स्वामीजी के कमरे में पहुँचकर उन्होंने देखा कि वे फूट-फूटकर रो रहे हैं। स्वामीजी को पता ही नहीं चला कि उनके गुरुभाई कमरे में आ पहुँचे हैं। 

विज्ञानानन्दजी ने पूछा - “स्वामीजी, क्या आपकी तबीयत खराब है?” स्वामीजी हड़बड़ा गए और बोले - “ओह, पेशन, मुझे लगा कि तुम सो रहे हो। नहीं भाई, मैं बीमार नहीं हूँ। परन्तु जब तक मेरा देश पीड़ित है, तब तक मैं सो नहीं सकता। मैं रोरोकर श्रीरामकृष्ण से प्रार्थना कर रहा था कि वे शीघ्रातिशीघ्र देश की हालत सुधार दें।” 

स्वामीजी भारत तथा भारतवासियों के प्रति प्रेम की प्रतिमूर्ति थे। जो कोई भी उनके सम्पर्क में आता, उसे वे भारत से प्रेम करने की शिक्षा देते। भगिनी क्रिष्टिन लिखती हैं - “मुझे लगता है कि जब हमने पहली बार उनकी उस अद्भुत आवाज में ‘भारतवर्ष’ शब्द सुना, तभी से हमारे हृदय में भारत के प्रति प्रेम प्रकट हुआ। यह बड़ा ही विस्मयकर लगता है कि कैसे पाँच अक्षरों के इस छोटे-से शब्द में इतना कुछ समा सकता है। उसमें प्रेम था, आवेग था, गर्व था, आकुलता थी, श्रद्धा थी, विषाद था, वीरता थी, गृहविरह था तथा सर्वोपरि - और भी अधिक प्रेम था। कितने भी ग्रन्थ दूसरों में ऐसे भाव उत्पन्न नहीं कर पाते! स्वामीजी में श्रोताओं के मन में प्रेम संचरित करने की जादुई शक्ति थी। इसके बाद भारत सदा के लिए हमारा मनोवांछित देश बन गया। उसके विषय में हर चीज - उसके लोग, उसका इतिहास, उसकी शिल्प-कला, उसकी प्रथाएँ, उसकी नदियाँ, पर्वत तथा मैदानी अंचल, उसकी संस्कृति, उसकी महान् आध्यात्मिक धारणाएँ, उसके शास्त्र - सब कुछ हमारे लिए रुचिकर और जीवन्त हो उठा।” 

\vskip -8pt

\delimiter

लन्दन में स्वामीजी के व्याख्यान बड़े ही लोकप्रिय हुए। एक शाम वे ‘राजयोग’ विषय पर बोल रहे थे और सभी उपस्थित लोग पूर्ण तन्मयता के साथ सुन रहे थे। परन्तु अच्छी चीजें भी हमेशा सबको पसन्द नहीं आतीं। एक एंग्लो-इंडियन श्रोता ने व्याख्यान के दौरान बीच-बीच में मूर्खतापूर्ण टिप्पणियाँ करनी आरम्भ कर दीं। स्वामीजी ने प्रारम्भ में तो उसकी ओर ध्यान ही नहीं दिया और देवप्रेरित के समान अपने विषय पर बोलते रहे। श्रोतागण भी शुरू में बड़े क्षुब्ध हुए, परन्तु स्वामीजी को अविचलित देखकर वे लोग भी चुप रहे। 

वह एंग्लो-इंडियन व्यक्ति क्रमशः अधिकाधिक शोरगुल मचाने लगा और मर्यादा की सभी सीमाओं का उल्लंघन करने लगा। जब स्वामीजी बुद्धदेव के प्रति अपनी श्रद्धा व्यक्त कर रहे थे, तो उसने बुद्धदेव की निन्दा की; जब स्वामीजी संन्यासियों की प्रशंसा कर रहे थे, तो वह उन्हें चोर तथा मिथ्याचारी कहने लगा; और जब उसे पता चला कि स्वामीजी बंगाली हैं, तो वह बंगाल के लोगों की निन्दा तथा अंग्रेजों की प्रशंसा करने लगा। 

व्याख्यान के दौरान इस प्रकार बारम्बार बाधा दिये जाने पर, आखिरकार स्वामीजी उसकी ओर मुड़े और इतिहास के अनेक पृष्ठों से अंग्रेजों के आपराधिक आचरण के अनेक उदाहरण प्रस्तुत करने लगे। ब्रिटेन के हृदय प्रदेश (लंदन) में एक भारतीय संन्यासी के मुख से इस प्रकार यह सब सुनकर उस एंग्लो-इंडियन की आँखों से आँसू बहने लगे। स्वामीजी सहज भाव से अपने विषय पर वापस लौट आए और इस प्रकार अपना व्याख्यान पूरा किया, मानो कुछ हुआ ही न हो। 

\delimiter

पश्चिमी अमेरिका के एक नगर में स्वामीजी ने अपने एक व्याख्यान में कहा कि जिस व्यक्ति को पूर्ण सत्य या ज्ञान की उपलब्धि हो जाती है, वह हर परिस्थिति में सम भाव बनाए रखता है; वह बाह्य घटनाओं से अविचलित रहकर सर्वदा शान्त बना रहता है। कुछ अक्खड़ चरवाहे बालकों ने स्वामीजी का यह व्याख्यान सुना और उनकी परीक्षा लेने की ठानी। जब स्वामीजी एक व्याख्यान देने उनके कस्बे में गए, तो उन लोगों ने उनसे एक ड्रम को उलटकर उसी पर खड़े होकर बोलने का अनुरोध किया। स्वामीजी ने उनका अनुरोध स्वीकार किया और अपने विषय की प्रस्तुति में तल्लीन हो गए। इस बीच उन युवकों ने निकट से ही बन्दूक चलाना शुरू किया, जिसकी गोलियाँ स्वामीजी के कान के पास से होकर निकल रही थी। परन्तु स्वामीजी इससे जरा भी विचलित नहीं हुए। उन्होंने उसी प्रकार शान्तिपूर्वक अपना भाषण जारी रखा, जैसा कि आरम्भ किया था। उनका वक्तव्य समाप्त होने पर चरवाहे युवकों ने उन्हें चारों ओर से घेर लिया और प्रशंसा के भाव से उनके साथ हाथ मिलाते हुए घोषित किया - “हाँ, स्वामीजी, आप पूरी तौर से ईमानदार हैं। आप अपने उपदेशों की जीवन्त प्रतिमूर्ति हैं।” 

\delimiter

स्वामीजी अमेरिका में सुपरिचित हो गए। एक बार एक स्टेशन पर ट्रेन से उतरने पर उनका बड़ा भव्य स्वागत किया गया। यह देखकर एक निग्रो कुली ने आगे बढ़कर उनसे हाथ मिलाया और बोला - “बधाई हो! मुझे बड़ी खुशी है कि मेरी जाति का एक व्यक्ति इतना सम्मानित किया गया है। इस देश की सम्पूर्ण निग्रो जाति को आपके ऊपर गर्व है।” स्वामीजी ने भी भी बड़े उत्साहपूर्वक हाथ मिलाते हुए गर्मजोशी के साथ कहा - “धन्यवाद! धन्यवाद, बन्धु!” उन्होंने उसे आभास तक नहीं होने दिया कि वे निग्रो नहीं हैं। 

निग्रो होने के सन्देह में स्वामीजी को दक्षिणी अमेरिका के कई होटलों में प्रवेश नहीं करने दिया गया। परन्तु उन्होंने कभी इस बात का खण्डन नहीं किया और न बताया ही कि वे एक भारतीय हैं। एक बार एक पाश्चात्य शिष्य ने उनसे पूछा कि ऐसी परिस्थितियों में उन्होंने क्यों नहीं बता दिया कि वे निग्रो नहीं, बल्कि एक भारतवासी हैं। स्वामीजी ने उत्तर दिया - “क्या! किसी दूसरे की कीमत पर बड़ा बनना! ऐसा करने के लिए मेरा पृथ्वी पर आगमन नहीं हुआ है।” 

\vskip -8pt

\delimiter

मादाम एम्मा काल्वे फ्रांस की एक सुप्रसिद्ध ओपेरा गायिका थीं। अमेरिका में भी वे अत्यन्त लोकप्रिय थीं। परन्तु व्यावसायिक सफलता के बावजूद उनका व्यक्तिगत जीवन बड़ा ही दुःखमय था - वे बड़ी जिद्दी तथा उग्र स्वभाव की थीं और उनमें बड़ा कठोर राग-द्वेष का भाव था। जैसा कि स्वाभाविक है, उनके मन में जरा भी शान्ति न थी। इसके अतिरिक्त १८९४ ई. के मार्च में हुई एक दुर्घटना में उनकी इकलौती पुत्री की भी मृत्यु हो गयी थी। इस दुःखद घटना के बाद वे अपना मानसिक सन्तुलन प्रायः खो बैठी थीं। उन्हीं दिनों उनकी एक मित्र उन्हें स्वामीजी के पास ले जाकर उनसे मिलाना चाहती थी, परन्तु काल्वे ने मना कर दिया, क्योंकि उन्होंने सोच लिया था कि स्थायी शान्ति प्राप्त करने के लिए उनके पास आत्महत्या के अतिरिक्त अन्य कोई उपाय नहीं बचा है। उन्होंने इसके लिए चार दफा प्रयास भी किया, परन्तु असफल रहीं। उन्होंने जब देखा कि कैसे स्वामीजी ने उनके कई मित्रों की सहायता की है, तो आखिरकार उन्होंने भी उनसे जाकर मिलने का निर्णय किया। 

वे स्वामीजी के आवास पर जा पहुँची और उनके अध्ययन-कक्ष में बैठकर प्रतीक्षा करने लगीं। जब उन्हें अन्दर बुलाया गया, तो उन्होंने देखा कि स्वामीजी ध्यान की सौम्य मुद्रा में बैठे हुए हैं और उनका गैरिक वस्त्र फर्श तक फैला हुआ है, पगड़ी से युक्त उनका सिर सामने की ओर झुका है और आँखें भूमि की ओर निबद्ध हैं। 

स्वामीजी ने ऊपर की ओर नहीं देखा, परन्तु एक मृदु स्नेहपूर्ण स्वर में बोले - “पुत्री, तुमने अपने चारों ओर कैसा अशान्त परिवेश बना रखा है! शान्त हो जाओ! यह बड़ा आवश्यक है!” मादाम काल्वे अपनी स्मृतिकथा में लिखती हैं - “इसके बाद मेरे नाम तक से अपरिचित उन व्यक्ति ने अनुद्विग्न तथा उदासीन भाव से शान्त स्वर में मेरे जीवन की गोपनीय समस्याओं तथा चिन्ताओं के बारे में अनेक बातें कहीं। उन्होंने ऐसी बातें भी बताईं, जो शायद मेरे परम अन्तरंग मित्रों को भी नहीं मालूम थीं।” 

जब उनके विदा लेने का समय आया, तो स्वामीजी ने उन पर आशीर्वादों की वर्षा करते हुए कहा - “बीती बातों को भूल जाओ। पुनः आनन्द और उत्साह के साथ जीवन-यापन करो। अपने स्वास्थ्य को सुधारो। एकान्त में बैठकर दुःखद प्रसंगों का चिन्तन मत करो।” काल्वे को क्षण भर में ही बोध होने लगा कि उनके दुःख तथा चिन्ताएँ दूर हो चुकी हैं। वे लिखती हैं - “ऐसा लगता था मानो उन्होंने मेरे व्याधिग्रस्त मन की सारी कुण्ठाओं को दूर करके उसे पवित्र एवं शान्तिमय भावों से परिपूर्ण कर दिया हो।” 

\vskip -8pt

\delimiter

अमेरिका के सुप्रसिद्ध धनकुबेर जॉन डी. रॉकफेलर के साथ स्वामीजी की जो भेंट हुई थी, उसका विवरण मादाम काल्वे ने अपनी मित्र मादाम पॉल वर्डियर को इस प्रकार दिया था। 

मादाम वर्डियर के लिखा है कि श्री रॉकफेलर ने अपने मित्रों से स्वामीजी के बारे में सुन रखा था। मित्रगण चाहते थे कि वे इन असाधारण भारतीय संन्यासी से मिलें, परन्तु वे हर बार किसी-न-किसी कारणवश टाल गए। वे एक सुदृढ़ इच्छाशक्ति के व्यक्ति थे और किसी के लिए भी उनके निर्णय में परिवर्तन करा लेना बड़ा कठिन कार्य था। परन्तु एक दिन रॉकफेलर एक विशिष्ट मनोवेग से प्रेरित होकर सीधे शिकागो के अपने उन मित्र के घर पर जा पहुँचे, जहाँ स्वामीजी ठहरे हुए थे। रसोइये द्वारा दरवाजा खोलने पर उन्होंने उसे किनारे ढकेलते हुए हिन्दू संन्यासी के रहने का कमरा पूछा। 

रसोइया उन्हें स्वामीजी के कमरे में ले गया। रॉकफेलर बिना सूचना दिये या इन्तजार किए स्वामीजी के अध्ययन-कक्ष में प्रविष्ट हो गए। यह देखकर उनके विस्मय की सीमा न रही कि लिखने की मेज पर बैठे स्वामीजी ने उनकी ओर नजर तक उठाकर नहीं देखा कि कौन आया है। क्षण भर की खामोशी के पश्चात् स्वामीजी ने काल्वे के समान ही, रॉकफेलर को भी उनके विगत जीवन से सम्बन्धित ऐसी अनेक बातें बताई, जो उनके खुद के अतिरिक्त दूसरा कोई भी नहीं जानता था। साथ ही उन्होंने रॉकफेलर को यह भी समझा दिया कि जो धन उन्होंने एकत्र किया है, वह उनका अपना नहीं है, वरन् वे उसके संरक्षक (ट्रस्टी) मात्र हैं; और ईश्वर ने उन्हें यह अतुल सम्पदा इसलिए प्रदान की है, ताकि वे उसके द्वारा लोगों की सहायता तथा उपकार करके धन्य हो सकें। भला रॉकफेलर को कोई यह बताने का साहस करे कि उन्हें क्या करना चाहिए और क्या नहीं! वे झल्लाकर, विदा माँगे बिना ही कमरे से बाहर निकल गए। 

लगभग एक सप्ताह बाद वे पुनः बिना किसी पूर्व सूचना के आए और स्वामीजी के अध्ययन-कक्ष में प्रवेश किया। उन्होंने स्वामीजी को उसी मुद्रा में बैठे देख उनकी मेज पर एक कागज फेंक दिया, जिसमें एक सार्वजनिक संस्था को एक बड़ी धनराशि दान करने की योजना थी। रॉकफेलर ने कहा, “महाशय! अब तो आप सन्तुष्ट होंगे और मुझे इसके लिए धन्यवाद देंगे।” स्वामीजी ने, न तो आँखें उठाईं और न ही अपनी जगह से हिले। उस कागज को हाथ में उठाते हुए वे बोले, “धन्यवाद तो बल्कि आपको मुझे देना चाहिए।” 

जनहित के कार्यों हेतु रॉकफेलर का यह प्रथम बड़ा दान था। परवर्ती काल में उन्होंने जनहित के कार्यों में काफी धन व्यय किया और खूब प्रसिद्धि हासिल की। 

\vskip -6pt

\delimiter

सुप्रसिद्ध अमेरिकी वक्ता तथा अज्ञेयवादी रॉबर्ट इंगरसोल से भी स्वामीजी की शिकागो में ही भेंट हुई थी। दोनों की विचारधारा में जमीन-आसमान का भेद था। एक ओर - स्वामीजी अतीन्द्रिय सत्य में अटल विश्वास रखनेवाले और आध्यात्मिकता की एक जीवन्त प्रतिमूर्ति थे। यदि उनका किसी से विरोध था, तो धार्मिक कट्टरता और ढोंग से ही था। दूसरी ओर - इंगरसोल हर तरह की धार्मिक मान्यताओं के विरोधी थे और अतीन्द्रिय सत्य में उनकी जरा भी श्रद्धा न थी। इसके बावजूद दोनों की कई बार भेंट हुई और आपस में धार्मिक तथा दार्शनिक विषयों पर चर्चा भी हुई। 

एक बार इंगरसोल ने कहा - “स्वामीजी, मेरा विचार है कि इस संसार से जितना भी सम्भव हो लाभ उठाने का प्रयास करना चाहिए। सन्तरे को निचोड़कर जितना अधिक हो सके रस निकाल लेना होगा, क्योंकि इस जगत् के अलावा अन्य किसी लोक के अस्तित्व के बारे में कोई प्रमाण नहीं~है।” 

स्वामीजी ने उत्तर दिया, “इस जगत् रूपी सन्तरे को निचोड़ने की मैं आप से अच्छी पद्धति जानता हूँ और उससे मुझे रस भी ज्यादा मिलता है। मैं जानता हूँ कि मैं मर नहीं सकता, अतः मुझे रस निकालने की हड़बड़ी नहीं है। मैं जानता हूँ कि मेरे लिए भय का कोई कारण नहीं, अतः आराम से निचोड़ता हूँ। मेरा कोई कर्तव्य नहीं, स्त्री-पुत्रादि तथा विषय-सम्पत्ति का कोई बन्धन नहीं है; अतः मैं समस्त नर-नारियों से प्रेम कर सकता हूँ, सभी मेरे लिए भगवत्-स्वरूप हैं। मनुष्य को भगवान समझकर उससे प्रेम करने में कितना आनन्द है! अपने सन्तरे को इस प्रकार निचोड़कर देखिए, इससे आपको दस हजार गुना अधिक रस मिलेगा। रस का एक बूँद भी व्यर्थ नहीं जाएगी।” 

\vskip -6pt

\delimiter

लन्दन से विदा लेने के पूर्व स्वामीजी के एक अंग्रेज मित्र ने उनसे पूछा, “स्वामीजी, चार वर्षों तक विलासिता, चकाचौंध तथा शक्ति से परिपूर्ण इस पश्चिमी जगत् का अनुभव लेने के बाद अब अपनी मातृभूमि आपको कैसी लगेगी?” स्वामीजी ने उत्तर दिया, “इधर आने के पूर्व मैं भारत से प्रेम करता था; परन्तु अब तो भारत की धूलिकण तक मेरे लिए पवित्र हो गयी है; अब मेरे लिए वह एक पुण्यभूमि है - एक तीर्थस्थान है!” 

\vskip -6pt

\delimiter

कोलकाता में एक दिन स्वामीजी ने अपने शिष्य प्रियनाथ सिन्हा से कहा कि जब कोई व्यक्ति अपने धर्म से प्रेम करता है, तो वह बड़ा हिम्मती तथा साहसी बन जाता है; और ऐसा अटल प्रेम ही भारतवासियों के बीच एकता ला सकता है, जिसका उनमें नितान्त अभाव है। इस सन्दर्भ में स्वामीजी ने एक घटना का वर्णन किया, जो उस जलयान पर घटी थी, जिस पर सवार होकर वे भारत लौट रहे~थे। 

उस यात्रा के दौरान दो ईसाई मिशनरी स्वामीजी के पास आए और उनके साथ हिन्दू तथा ईसाई धर्मों के गुण-दोषों पर चर्चा करने के लिए उनसे अनुरोध करने लगे। चर्चा के दौरान जब मिशनरी उनके साथ तर्क करने में असमर्थ होने लगे, तो उनकी आक्रामकता उत्तरोत्तर बढ़ने लगी और वे हिन्दुओं तथा उनके धर्म के विषय में ऊलजलूल बकने लगे। स्वामीजी से जितना भी सम्भव था, उनकी बातें सहते रहे। इसके बाद वे उन दोनों के निकट गए और एक मिशनरी का कॉलर पकड़कर व्यंग्यपूर्वक, किन्तु साथ ही दृढ़ता के साथ बोले, “यदि तुमने मेरे धर्म के बारे में एक भी अपशब्द कहा, तो मैं तुम्हें समुद्र में फेंक दूँगा।” मिशनरी सिर से पाँव तक काँप उठा और गिड़गिड़ाते हुए बोला, “महाशय, मुझे जाने दीजिए। मैं दुबारा कभी ऐसा नहीं करूँगा।” स्वामीजी ने प्रियनाथ सिन्हा को बताया कि इसके बाद वह मिशनरी जब कभी स्वामीजी को देखता, तो उनके प्रति बड़ा सम्मान प्रदर्शित करता। 

इस घटना का वर्णन करने के बाद स्वामीजी अपने शिष्य की ओर मुड़े और उससे पूछा कि यदि कोई उसकी माँ का अपमान करे, तो वह क्या करेगा? प्रियनाथ सिन्हा ने कहा, “महाराज, मैं तो उस पर चढ़ बैठूँगा और उसकी अच्छी ठुकाई करूँगा।” उनके इस उत्तर पर प्रसन्न होकर स्वामीजी बोले, “अच्छा सिंहा, यदि तुम्हारे मन में अपने धर्म के विषय में भी ऐसी ही अटल भक्ति रहती, तो तुम कभी अपने हिन्दू भाइयों को ईसाई होते देखकर उसे सहन नहीं पाते। तुम प्रतिदिन ऐसा ही होते हुए देखते हो, परन्तु उधर बिल्कुल भी ध्यान नहीं देते। कहाँ है तुम्हारी श्रद्धा? और कहाँ है तुम्हारा देशप्रेम? प्रतिदिन ईसाई मिशनरी तुम्हारे मुँह पर हिन्दू धर्म को गालियाँ देते रहते हैं; परन्तु इसे सुनकर तुममें से कितने लोगों का खून खौलता है और कितने लोग अपने धर्म की रक्षा के लिए उठ खड़े होते हैं?” 

\vskip -6pt

\delimiter

पूर्वोक्त समुद्र-यात्रा के समय स्वामीजी के कई यूरोपीय शिष्य भी उनके साथ यात्रा कर रहे थे। जब जहाज अदन पहुँचकर वहाँ रुका, तो स्वामीजी बन्दरगाह से तीन मील दूर कोई स्थान देखने के लिए नीचे उतरे। शिष्यगण भी उनके पीछे-पीछे चल रहे थे। 

जब यह टोली इस प्रकार घूम रही थी, तभी स्वामीजी ने वहाँ एक भारतीय पानविक्रेता को देखा। वे अपने संगियों को पीछे छोड़ते हुए तेजी से उसके पास जा पहुँचे। इतने दिनों बाद एक भारतवासी से मिलकर उससे बातें करना उनके लिए असीम आनन्द की बात थी। 

इस दौरान उनके शिष्यगण उन्हें इधर-उधर ढूँढ़ रहे थे। उन्हें पता ही न था कि स्वामीजी सहसा किधर चले गए। थोड़ी देर बाद उन लोगों ने देखा कि वे एक पानविक्रेता के बगल में आराम से बैठे हुए हैं। वह एक अद्भुत नजारा था! स्वामीजी उस सीधे-सादे पानवाले से कह रहे थे, “भाई, जरा अपनी चिलम तो देना!” यह सुनकर वे लोग मुस्कुराये बिना नहीं रह सके। पानवाले ने उन्हें चिलम दी और वे उसे लेकर वे बड़े आनन्दपूर्वक धूम्रपान करने लगे। 

स्वामीजी की बालसुलभ सरलता, स्वदेशवासियों के प्रति प्रेम और छोटी-मोटी चीजों में भी आनन्द लेने की क्षमता ने कैप्टेन सेवियर तथा अन्य लोगों को बड़ा प्रभावित किया। 

\delimiter

पश्चिमी जगत् से लौटने के बाद स्वामीजी का स्वास्थ्य ठीक नहीं चल रहा था। दिन-पर-दिन उनका स्वास्थ्य बिगड़ता जा रहा था। अतः अपने गुरुभाइयों तथा शुभचिन्तकों की सलाह पर वे जलवायु-परिवर्तन हेतु दार्जिलिंग गए। 

वहाँ एक दिन सुबह जब वे टहलने के लिए निकले, तो यह देखकर उन्हें बड़ी पीड़ा हुई कि एक भूटिया महिला अपनी पीठ पर भारी बोझ उठाये सड़क पर चली जा रही है। स्वामीजी के साथ चलनेवालों का ध्यान, स्वामीजी की उसके प्रति सहानुभूति पर गया, जो मानो उसकी सम्पूर्ण पीड़ा का स्वयं ही अनुभव कर रहे थे! सहसा वह लड़खड़ायी और अपने पूरे बोझ के साथ नीचे गिर पड़ी। उसके अस्थि-पंजरों में काफी चोट आयी। परन्तु इसके साथ-ही-साथ स्वामीजी ने भी अपने पंजरों में तीव्र पीड़ा का अनुभव किया। वे थोड़ी देर तक स्थिर खड़े रहे और उसके बाद बोले - “मैं हिल नहीं सकता। मुझे बड़ी तीव्र पीड़ा का बोध हो रहा है।” साथ के लोगों ने पूछा - “स्वामीजी, आपको कहाँ पीड़ा हो रही है?” अपने पंजरों की ओर संकेत करते हुए स्वामीजी बोले - “ठीक यहाँ पर! क्या तुमने नहीं देखा कि कुछ ही क्षणों पूर्व उस महिला को यहाँ पर कितने जोर की चोट लगी?” 

यद्यपि ऐसी घटनाएँ दुर्लभ हैं, परन्तु ऐसे लोगों के जीवन में घटती हैं, जो बड़े संवेदनशील हैं तथा दूसरों के प्रति सहानुभूति-सम्पन्न हैं। ऐसी घटनाएँ हमें श्रीरामकृष्ण के जीवन में भी देखने को मिलती हैं। एक बार जब गंगा पर एक नौका में एक मल्लाह ने दूसरे मल्लाह पर प्रहार किया, तो श्रीरामकृष्ण पीड़ा से चिल्ला उठे थे। 

\delimiter

जब स्वामीजी प्रसिद्ध हो गए तथा देश-विदेश में हजारों भक्तों द्वारा पूजित होने लगे, तो भी उनके स्वभाव तथा मित्रों के प्रति उनके व्यवहार में कोई अन्तर नहीं आया। वे उन लोगों के लिए सदैव वही - ‘नरेन’ ही बने रहे। 

स्वामीजी जब लाहौर गए, तो वहाँ संयोगवश उनकी भेंट अपने बचपन के मित्र मोतीलाल बोस से हो गयी। मोतीलाल ग्रेट इंडियन सर्कस के मालिक थे और वहाँ पर अपनी टोली के साथ खेल दिखाने के लिए आए हुए थे। काफी काल बाद भेंट होने के कारण स्वामीजी अपने मित्र के साथ बातें करने में मशगूल हो गए। परन्तु मोतीलाल को बेचैनी महसूस होने लगी। वे बोले - “भाई, मैं तुम्हें कैसे सम्बोधित करूँ - ‘नरेन’ कहकर या ‘स्वामीजी’ कहकर?” 

स्वामीजी खिलखिलाकर हँसने लगे और बोले - “मोती, क्या तू पागल हो गया है? कुछ भी बदला नहीं है। तू मेरे लिए वही ‘मोती’ है और मैं वही ‘नरेन’ हूँ।” स्वामीजी ने ये शब्द इतने स्नेह तथा भावुकता के साथ कहे कि मोतीलाल अत्यन्त अभिभूत हो उठे और अपना सारा संकोच भूल गए। 

\delimiter

स्वामीजी के अमेरिका से लौटने के बाद राजस्थान के अलवर नगर में एक और भी हृदयस्पर्शी घटना घटी थी। वे वहाँ अपने अन्तरंग मित्रों तथा शिष्यों से मिलने गए थे। उनके अनेक मित्र तथा गण्यमान्य लोग उनका स्वागत करने के लिए अलवर के रेलवे स्टेशन पर उपस्थित थे। परन्तु वहाँ स्वामीजी का ध्यान एक व्यक्ति की ओर आकृष्ट हुआ, जो कि ऊपर से बड़ा ही साधारण-सा दीख पड़ता था और विनयपूर्वक भीड़ में एक किनारे खड़ा था। उस व्यक्ति ने फटे-पुराने वस्त्र धारण कर रखे थे, परन्तु इतने दिनों बाद स्वामीजी को देखकर उसका चेहरा खुशी से दमक रहा था। वह भी स्वामीजी का सान्निध्य चाहता था, परन्तु उसमें इतना साहस न था कि भीड़ को ठेलकर उनके पास पहुँच सके। स्वामीजी ने उसे दूर से देखा और पुकार उठे - “रामसनेही! रामसनेही!” भीड़ ने उस साधारण-से दिख रहे व्यक्ति के लिए रास्ता दे दिया और वह स्वामीजी का अभिवादन करने को आगे बढ़ा। स्वामीजी ने उसका स्वागत किया और स्नेहपूर्वक उसके साथ बातें कीं। 

\delimiter

स्वामीजी के अल्पकालिक अलवर-प्रवास के दौरान उनके प्रति श्रद्धा-भक्ति रखनेवाले अनेक धनाढ्य लोगों ने उन्हें निशाभोज के लिए आमंत्रित किया; परन्तु स्वामीजी ने सर्वप्रथम एक निर्धन बुढ़िया का निमंत्रण स्वीकार किया। इस निर्धन वृद्धा ने उन्हें उस समय खिलाया था, जब वे परिव्राजक संन्यासी के रूप में वहाँ आए थे और उनके पास खाने को कुछ भी न था। स्वामीजी उसकी हृदयवत्ता को नहीं भूले थे। अलवर पहुँचते ही स्वामीजी ने उसे सूचना भेजी थी कि उनके लिए उन्हीं मोटे टिक्कड़ों की व्यवस्था की जाए, जिन्हें कई साल पहले वे आनन्दपूर्वक खाया करते थे। 

वृद्धा खुशी से उन्मत्त हो उठी। उसने बड़े यत्नपूर्वक टिक्कड़ बनाए और बड़ी बेसब्री के साथ स्वामीजी तथा उनके शिष्यों की प्रतीक्षा करने लगी। उन लोगों के आने पर उसने बड़े प्रेम के साथ अपना साधारण-सा भोजन परोसा। स्वामीजी ने भोजन का रसास्वादन किया और अपने शिष्यों से बोले - “देखो, यह बूढ़ी माँ कितनी वात्सल्यमयी है! और यह भोजन कितना पवित्र तथा अनाडम्बर है!” वहाँ से विदा लेने के पूर्व स्वामीजी ने उस घर के स्वामी के हाथ में सौ रुपयों का एक नोट पकड़ा दिया और बता दिया कि इसे बाद में बूढ़ी-माँ को दे देना। 

\delimiter

१८९० ई. में स्वामी विवेकानन्द अपने गुरुभाई अखण्डानन्द जी के साथ हिमालय में भ्रमण कर रहे थे। एक दिन जब वे अल्मोड़ा से मात्र दो मील दूर रह गए थे, तभी स्वामीजी भूख तथा थकान के कारण अचेत और मरणासन्न हो गए। निकट ही एक मुस्लिम फकीर की कुटिया थी। एक अज्ञात संन्सासी को ऐसी हालत में पड़ा देखकर वह जल्दी से जाकर एक ककड़ी ले आया। परन्तु स्वामीजी का शरीर इतना दुर्बल हो चुका था कि वे उसे उठाकर मुँह से भी नहीं लगा सकते थे, अतः फकीर ने अपने हाथ से उन्हें वह ककड़ी खिलायी। इसके बाद स्वामीजी थोड़ा स्वस्थ महसूस करने लगे। 

इसके सात वर्ष बाद जब स्वामीजी पुनः अल्मोड़ा आए, तो स्थानीय जनता ने उनका स्वागत करने के लिए एक आमसभा का आयोजन किया था। क्रार्यक्रम के बीच में ही स्वामीजी का ध्यान सहसा भीड़ में खड़े एक व्यक्ति की ओर आकृष्ट हुआ। वह व्यक्ति वही फकीर था, जिसने वर्षों पूर्व स्वामीजी की जान बचायी थी। स्वामीजी ने उसे पहचान लिया और सबके सामने लाकर घोषित किया कि इस फकीर ने ही कभी उनकी जान बचायी थी। बाद में उन्होंने उसे कुछ धन भी दिया। परन्तु फकीर स्वामीजी को भूल चुका था। 

\delimiter

एक बार जाड़े के मौसम में स्वामीजी स्वामी निरंजनानन्द को साथ लेकर देवघर-वैद्यनाथ गए और वहाँ प्रियनाथ मुखर्जी के अतिथि हुए। एक दिन अपने गुरुभाई के साथ टहलते समय स्वामीजी ने देखा कि सड़क के किनारे एक पीड़ा से तड़पता हुआ व्यक्ति असहाय पड़ा है। स्वामीजी उस व्यक्ति के पास गए और देखा कि वह भयंकर पेचिस रोग से ग्रस्त है। उन्हें लगा कि उसे तत्काल चिकित्सा की जरूरत है, परन्तु सबसे पहले तो उसे सड़क के किनारे से हटाना होगा। परन्तु उसे ले कहाँ जाया जाए? उन्होंने सोचा कि प्रियनाथ के घर ले चलें। परन्तु स्वामीजी स्वयं ही तो वहाँ मेहमान थे, अतः वे एक अपरिचित व्यक्ति को वहाँ कैसे ले जा सकते थे? प्रियनाथ बाबू कहीं बुरा न मान जाएँ! इस प्रकार वे क्षण भर के लिए हिचके और उसके बाद किसी भी कीमत पर इस असहाय व्यक्ति की सेवा करने का निश्चय कर लिया। अतः वे स्वामी निरंजनानन्द की सहायता से पीड़ित व्यक्ति को प्रियनाथ बाबू के घर ले आए। वहाँ उन्होंने उसे एक बिस्तर पर लिटा दिया, उसकी भलीभाँति सफाई की, कपड़े बदले और गरम सेंक देने लगे। इसके फलस्वरूप व्यक्ति शीघ्र ही चंगा हो उठा। यह सब देखकर प्रियनाथ बाबू नाराज होना तो दूर, स्वामीजी के प्रेम की इस अद्भुत अभिव्यक्ति पर मुग्ध हो उठे। 

\delimiter

किसी गोरक्षण समिति के एक प्रचारक आर्थिक सहायता के लिए स्वामीजी से मिलने आए। उन्होंने स्वामीजी को बताया कि उनकी समिति गो-माताओं की कसाइयों के हाथ से बचाने हेतु देश में जगह-जगह पिंजरापोलों की स्थापना करती है। स्वामीजी ने बड़े ध्यानपूर्वक उनकी बातें सुनीं और उसके बाद पूछा, “मध्य भारत में इस समय भयानक अकाल पड़ा हुआ है। भारत सरकार ने अनाहार के फलस्वरूप मृत्यु को प्राप्त हुए नौ लाख लोगों की तालिका दी है। क्या आपकी समिति ने उनकी सहायता के लिए कुछ किया है?” प्रचारक ने इन्कार करते हुए बताया कि उनकी समिति केवल गो-माताओं की रक्षा के लिए ही स्थापित हुई है; और वे अकाल-पीड़ितों की सहायता करने का कोई कारण नहीं देखते, क्योंकि लोग अपने कर्मों का फल ही तो भोग रहे हैं! 

यह असंवेदनापूर्ण उत्तर सुनकर स्वामीजी नाराज हो उठे, परन्तु उन्होंने स्वयं को सँभालते हुए कहा, “मुझे किसी भी ऐसी सभा-समिति के प्रति जरा भी सहानुभूति नहीं है; जो मनुष्य के प्रति संवेदना का भाव नहीं रखती, जो अपने ही भाइयों को भूखे मरते देखकर भी उनकी प्राणरक्षा के लिए एक मुट्ठी अन्न न देकर, पशु-पक्षियों के रक्षणार्थ ढेर-के-ढेर अन्न वितरित करती है। मुझे लगता कि ऐसी समितियों के द्वारा समाज का कुछ विशेष उपकार नहीं होगा। यदि व्यक्ति हर बात में कर्मफल के ही सिद्धान्त के अनुसार चले, तो फिर पृथ्वी पर होनेवाले सारे उद्यम और यहाँ तक कि आपके गोरक्षा का प्रयास भी, पूरी तौर से निरर्थक सिद्ध होंगे। तब तो यह भी कहा जा सकता है कि गो-माताएँ भी अपने कर्मफल से ही कसाइयों के हाथ में पहुँचती हैं और हमें उनके लिए कुछ भी करने की आवश्यकता नहीं है।” 

प्रचारक को कोई उत्तर नहीं सूझा, तथापि वे अपने तर्क को आगे बढ़ाते हुए बोले, “हाँ स्वामीजी, आपने जो कहा, वह सत्य है, परन्तु शास्त्र कहते हैं कि गो हमारी माता है।” इस पर स्वामीजी ने व्यंग करते हुए कहा, “आप बिलकुल सही कहते हैं, अन्यथा ऐसी सुयोग्य सन्तान और कौन प्रसव कर सकती है?” 

प्रचारक सम्भवतः स्वामीजी के तात्पर्य को समझ नहीं सके और एक बार फिर अपनी समिति के लिए दान की याचना करने लगे। उनका वक्तव्य समाप्त हो जाने पर स्वामीजी बोले, “देखो, मैं तो ठहरा एक फकीर संन्यासी! मेरे पास धन कहाँ है, जो मैं आपकी आर्थिक सहायता कर सकूँ? परन्तु यदि कभी मेरे हाथ में धन आया भी, तो सर्वप्रथम मैं उसे मानव-सेवा में व्यय करूँगा। पहले मनुष्य की सेवा तथा रक्षा आवश्यक है - उसे अन्न, विद्या तथा धर्म का दान करना होगा। इन मानवीय आवश्यकताओं की पूर्ति के बाद यदि कुछ बचा, तो शायद आपकी समिति को भी कुछ भेज दिया जाएगा।” 

\vskip -7pt

\delimiter

‘हितवादी’ पत्रिका के सुप्रसिद्ध सम्पादक पण्डित सखाराम गणेश देउस्कर अपने दो मित्रों के साथ स्वामीजी से मिलने आए। मित्रों में से एक पंजाबी था - यह जानकर स्वामीजी पंजाब में अकाल अर्थात् खाद्यान्नों के अभाव की गम्भीर समस्या पर बातें करने लगे। उन दिनों भारत में फैले अकाल ने स्वामीजी के मन को ऐसा आच्छन्न कर लिया था कि उन्होंने देउस्कर तथा उनके मित्रों के साथ धर्म या आध्यात्मिक विषयों पर चर्चा नहीं की। विदा लेते समय पंजाबी सज्जन ने स्वामीजी से कहा - “महाराज, आज हम आपसे कुछ आध्यात्मिक बातें सुनने की आशा लेकर आए थे, परन्तु दुर्भाग्यवश हमारी चर्चा सांसारिक विषयों की ओर मुड़ गयी। मुझे लगता है कि यह समय की बरबादी मात्र हुई।” 

इस बात को सुनकर स्वामीजी बड़े गम्भीर हो गए और बोले - “महाशय, जब तक मेरे देश का एक कुत्ता भी भूखा है, तब तक उसे खिलाना तथा उसके देखभाल करना ही मेरा धर्म होगा। बाकी सब कुछ - या तो अधर्म है या फिर मिथ्या-धर्म।” स्वामीजी की अग्निमयी वाणी सुनकर तीनों आगन्तुक गूंगे-से रह गए। स्वामीजी के देहावसान के वर्षों बाद पण्डित देउस्कर ने इस घटना का वर्णन करते हुए कहा था कि तभी स्वामीजी के ये शब्द सदा के लिए उनके मन में अंकित हो गए और जीवन में पहली बार उनकी समझ में आया कि सच्ची देशभक्ति किसे कहते हैं! 

\delimiter

वेदान्त पर स्वामीजी से तर्क-वितर्क करके उन्हें नीचा दिखाने के उद्देश्य से एक उत्तर-भारतीय पण्डित उनसे मिलने आए। परन्तु स्वामीजी उस समय वेदान्त पर चर्चा करने की मनःस्थिति में नहीं थे। वे तो निरन्तर देश भर में फैले अकाल के बोझ से कराह रही जनता के विषय में चिन्ता कर रहे थे। वे बोले - “पण्डितजी, सर्वत्र भयंकर अकाल फैला हुआ है। सबसे पहले तो आप, मुट्ठी भर अन्न के लिए हृदय-विदारक क्रन्दन कर रहे, अपने भूखे देशवासियों की समस्या को दूर करने की चेष्टा कीजिए; उसके बाद आप मेरे पास वेदान्त पर शास्त्रार्थ करने के लिए आइए। अनाहार तथा भूख से मर रहे हजारों लोगों को बचाने के लिए अपने जीवन तथा प्राणों को दाँव पर लगा देना - यही वेदान्त-धर्म का सार-सर्वस्व है।” 

\delimiter

स्वामीजी काशीपुर (कोलकाता) के गोपाल लाल सील के उद्यान-भवन में ठहरे हुए थे। एक दिन एक युवक उनके पास आया और बोला - “स्वामीजी, मैंने अनेक स्थानों का भ्रमण किया है और मेरा अनेक धार्मिक सम्प्रदायों के साथ घनिष्ठ सम्पर्क रहा है; तो भी मैं अब तक यह समझने में असमर्थ रहा हूँ कि सत्य क्या है। मैं प्रतिदिन अपने कमरे का दरवाजा बन्द करके ध्यान में बैठता हूँ, परन्तु शान्ति मुझसे कोसों दूर रहती हैं। स्वामीजी, बताइए ऐसा क्यों है?” 

स्वामीजी ने ध्यानपूर्वक उसकी बातें सुनीं और उसके बाद बोले - “वत्स, यदि तुम मन की शान्ति चाहते हो, तो अब तक तुम जो कुछ करते रहे हो, उसके ठीक उलटा करना होगा। तुम्हें अपने द्वार खुले रखने होंगे और चारों ओर दृष्टि घुमाकर देखना होगा। ऐसा करने पर तुम्हें यह देखकर बड़ा विस्मय होगा कि बहुत-से लोग उद्विग्नतापूर्वक तुमसे सहायता की अपेक्षा कर रहे हैं। उनकी सहायता करो, उन्हें खिलाओ, उन्हें पीने को पानी दो - यथासम्भव उनकी सेवा करो। मैं विश्वास दिलाता हूँ कि तुम्हें शान्ति अवश्य मिलेगी।” 

\delimiter

देवघर से लौटने के बाद वाले दिन स्वामीजी ने मठ के अपने गुरुभाई संन्यासियों को अपनी इस इच्छा से अवगत कराया कि वे लोग बुद्धदेव के शिष्यों के समान ही दुनिया के ओर-छोर तक जाकर श्रीरामकृष्ण के सन्देश का प्रचार करने के लिए तैयार हो जाएँ। प्रारम्भ में स्वामीजी ने अपने दो शिष्यों - स्वामी विरजानन्द तथा स्वामी प्रकाशानन्द को इस कार्य के लिए चुना और उनसे धर्म-प्रचार हेतु पूर्वी बंगाल (अब बांग्लादेश) जाने को कहा। 

विरजानन्दजी ने कहा - “स्वामीजी, मैं तो कुछ जानता ही नहीं। मैं भला क्या प्रचार करूँगा?” स्वामीजी ने उत्तर दिया - “तो फिर तुम जाकर इसी बात का प्रचार करो। यह भी एक बड़ा अद्भुत सन्देश है!” 

शिष्य को तब भी विश्वास नहीं हो रहा था। उसने स्वामीजी से प्रार्थना की कि इस कार्य में लगने के पूर्व उसे तीव्र साधना में डूबकर आत्मा की उपलब्धि करने की अनुमति प्रदान की जाए। स्वामीजी को यह विचार पसन्द नहीं आया और उन्होंने अपने युवा शिष्य को डाँटते हुए कहा - “यदि तुम अपने लिए मुक्ति की चेष्टा करोगे, तो नरक में जाओगे! पहले इसी कामना का नाश करो। यही सबसे कठिन साधना है। यदि तुम आध्यात्मिक उपलब्धि की चरम अवस्था प्राप्त करना चाहते हो, तो दूसरों की मुक्ति के लिए कठोर परिश्रम करो।” 

इसके बाद उन्होंने अत्यन्त मृदु-मधुर आवाज में कहा - “कर्म करो, मेरे बच्चो, फल की परवाह किए बिना जी-जान से कर्म करो। दूसरों का भला करने के प्रयास में यदि नरक भी जाना पड़े, तो क्या? मैं कहता हूँ कि यह केवल अपने लिए स्वर्ग अर्जित करने की अपेक्षा अनन्त गुना श्रेष्ठ है।” 

\delimiter

एक दिन बलराम बोस के मकान में स्वामीजी अपने शिष्य शरत् चन्द्र चक्रवर्ती के साथ बड़े उत्साहपूर्वक वेदों के विषय में चर्चा कर रहे थे। श्रीरामकृष्ण के एक गृही शिष्य गिरीश चन्द्र घोष भी वहाँ उपस्थित थे। दोनों बड़े गौर के साथ इस विषय पर स्वामीजी की मनमोहक व्याख्या सुन रहे थे। स्वामीजी द्वारा सविस्तार बोलने के बाद गिरीश बोले - “नरेन, मैं तुमसे एक बात पूछना चाहता हूँ। मैं जानता हूँ कि तुमने वेद-वेदान्त के दर्शन का बड़ी गहराई से अध्ययन किया है, परन्तु क्या उनमें निर्धनता, भूखमरी और देश को निष्प्राण कर रही अन्य भयंकर समस्याओं का भी कोई समाधान लिखा है?” श्रीमती अमुक, जो प्रतिदिन पचास लोगों को भोजन कराया करती थीं, वे स्वयं तीन दिनों से अनाहार का कष्ट भोग रही हैं; बदमाशों ने एक गृहिणी को मार डाला है; और मेरी परिचित एक विधवा अपने सम्बन्धियों द्वारा अपनी सम्पत्ति से बेदखल कर दी गयी है। क्या तुम्हारे वेद उसे न्याय दिला सकते हैं? क्या तुम्हारे शास्त्र उसके कष्ट दूर करके हमें आश्वस्त कर सकते हैं कि ऐसी घटनाएँ दुबारा नहीं होंगी?” 

गिरीश बाबू और भी उदाहरण देते हुए समाज की भयंकर दुर्दशा का चित्रण करने लगे। स्वामीजी निस्तब्ध बैठे रहे। उनकी आँखों से आँसू बहने लगे और वे शीघ्रतापूर्वक उठकर कमरे से बाहर चले गए। तब गिरीश बाबू ने शरत् चन्द्र की ओर उन्मुख होकर कहा, “देखा तुमने! तुम्हारे गुरु का हृदय कितना विशाल है! स्वामीजी की वेद-वेदान्त में विद्वत्ता के कारण नहीं, बल्कि उनके अनन्त प्रेम तथा दूसरों के दुःख दूर करने के लिए उनकी आकुलता के लिए ही, मैं उनके प्रति आदर का भाव रखता हूँ। तुमने अपनी आँखों से ही देख लिया कि अपने देशवासियों के कष्ट की बात सुनकर वे किस प्रकार फूट-फूटकर रोने लगे! वेद-वेदान्त नहीं, अपितु मानवता के प्रति प्रेम तथा करुणा ही स्वामीजी के हृदय को अभिभूत कर देते हैं।” 

\vskip -6pt

\delimiter

पश्चिमी दुनिया से लौटने के बाद स्वामीजी का स्वास्थ्य बड़ी तेजी से बिगड़ रहा था। उनके चिकित्सकों ने उन्हें पूर्ण विश्राम की सलाह दी, अतः वे दार्जिलिंग गए। परन्तु कुछ दिनों बाद ही उन्हें सूचना मिली कि कोलकाता में प्लेग की महामारी फैल गयी है। उस समय स्वामीजी के हृदय में जो भावनाएँ उठी थीं, उन्हें उन्होंने जोसेफीन मैक्लाउड के नाम २९ अप्रैल, १८९८ को लिखे अपने पत्र में व्यक्त किया है। उसमें लिखा था - मैंने निश्चय किया है कि जिस नगर में मैंने जन्म लिया है, वहाँ के प्लेग-पीड़ित लोगों की सेवा में मैं अपना जीवन बलिदान कर दूँगा और यही निर्वाण-प्राप्ति का सर्वश्रेष्ठ उपाय होगा। स्वामीजी तत्काल कोलकाता आ पहुँचे और वहाँ उन्होंने देखा कि कोलकाता के लोगों को रोग से भी अधिक रोग का भय आतंकित किए हुए है। आतंक से घबराये हुए लोग घर छोड़-छोड़कर भाग रहे थे। स्वामीजी ने परिस्थिति की गम्भीरता को समझ लिया और लोगों के भय को दूर करने के लिए एक प्लेग-विषयक घोषणा-पत्र प्रकाशित कराया। लोगों के बीच वितरित किए जानेवाले इस घोषणा-पत्र में बताया गया था कि रामकृष्ण मिशन उनकी हर प्रकार से यथासम्भव सहायता देने को तैयार है। उन्होंने नगर के विभिन्न अंचलों में सेवा-केन्द्र खोलने का भी निर्णय लिया। 

परन्तु इस राहत-कार्य के लिए काफी धन की आवश्यकता थी। जब एक गुरुभाई ने स्वामीजी से पूछा कि इसके लिए धन कहाँ से आएगा, तो उन्होंने बिना किसी हिचक के उत्तर दिया - “हम लोग संन्यासी हैं। हम तो भिक्षा का अन्न खाकर पेड़ों के नीचे भी सो लेंगे। यदि मठ बेचकर करोड़ों लोगों का जीवन बचाया जा सके, तो यह भी मुझे स्वीकार है।” 

सौभाग्यवश इस चरम उपाय का आश्रय नहीं लेना पड़ा, क्योंकि धन दूसरे स्रोतों से आ गया। परन्तु स्वामीजी की घोषणा में उनकी हृदयवत्ता और मानव-मात्र के प्रति उनका असीम प्रेम तथा करुणा प्रकट होती है। उन्होंने अपनी योजनाओं को साकार रूप देने हेतु - मठ-निर्माण करने के लिए जमीन खरीदने को खून-पसीना एक करके धन जुटाया था। इसके बावजूद यदि मठ को बेचकर लोगों की सहायता हो पाती, तो वे उसे भी बेचने को तैयार थे। 

\vskip -6pt

\delimiter

मानव-जाति के प्रति स्वामीजी का प्रेम इतना गहन था कि वे प्रायः लोगों की दुर्दशा की बात सोचकर एकान्त में आँसू बहाया करते थे। निम्नलिखित घटना उनके प्रथम अमेरिका-यात्रा के बाद हुई थी। उन दिनों वे बलराम बोस के मकान में निवास कर रहे थे। एक दिन स्वामी तुरीयानन्द उनसे मिलने आए। उन्होंने देखा कि स्वामीजी अकेले ही मकान के बरामदे में टहल रहे हैं। वे अपने विचारों में इतने खोये हुए थे कि इस ओर उनका ध्यान ही नहीं गया कि उनके गुरुभाई उनसे मिलने आए हुए हैं। थोड़ी देर बाद वे मीराबाई के एक सुपरिचित भजन की पंक्तियाँ गुनगुनाने लगे और उनकी आँखों से आँसू निकलकर गालों को भिगोने लगे। इसके बाद उन्होंने अपने दोनों हाथों से अपने चेहरे को ढँक लिया और छज्जे के सहारे झुककर गाते रहे - “मेरो दरद न जाने कोय! घायल की गति घायल जाने और न जाने कोय!” बाद में इस घटना का वर्णन करते हुए स्वामी तुरीयानन्द ने कहा था - “उनकी वाणी मेरे हृदय को तीर की भाँति बेध गयी और मेरी भी आँखों में आँसू आ गए। स्वामीजी की पीड़ा का कारण न समझ पाने के कारण मुझे बड़ी बेचैनी हुई। परन्तु शीघ्र ही मेरे ध्यान में आया कि सारे विश्व के दीन-दुखियों तथा पीड़ितों के प्रति प्रबल सहानुभूति ही उनकी उस मनःस्थिति का कारण थी।” 

\delimiter

स्वामीजी जब दूसरी बार अमेरिका गए, तो वहाँ एक बड़ी रोचक घटना हुई थी। एक दिन जब वे एक नदी के तट पर टहल रहे थे, तो उन्होंने देखा कि कुछ युवक पानी के ऊपर तैर रहे अण्डे की खोलों पर निशानेबाजी का अभ्यास कर रहे हैं। बारीबारी से प्रयास करके भी कोई युवक लक्ष्य को भेदने में सफल नहीं हो पा रहा था। वह खेल देखकर स्वामीजी को बड़ा मजा आया। वे अपने चेहरे पर आयी मुस्कान को दबा न सके और एक युवक ने इसे देख भी लिया। उसने चुनौती-भरे स्वर में कहा - “महाशय, यह काम उतना आसान नहीं है, जितना कि दूर से दीख रहा है। आइये, जरा देखें कि आप क्या कर पाते हैं!” 

स्वामीजी ने चुपचाप जाकर लड़के के हाथ से बन्दूक ले लिया और एक-एक करके लगातार बारह खोलों को उड़ा दिया। युवकगण आश्चर्यचकित रह गए। उन लोगों ने सोचा कि स्वामीजी जरूर कोई कुशल निशानेबाज होंगे। स्वामीजी ने उनके चेहरे के भाव पढ़ लिए और उन लोगों को बताया कि उन्होंने अपने जीवन में इसके पहले कभी एक भी गोली नहीं चलायी है और उनकी इस सफलता का एकमात्र रहस्य है - मन की एकाग्रता। 

\delimiter

भारत लौटते समय स्वामीजी काहिरा में ठहरे हुए थे। कुछ पाश्चात्य शिष्य तथा मित्र भी उनके साथ थे। वहाँ एक दिन टहलते समय स्वामीजी की टोली रास्ता भूल गयी और सहसा उन लोगों ने पाया कि वे वेश्याओं के मुहल्ले में पहुँच गए हैं। स्वामीजी तथा उनके मित्र पल भर में ही समझ गए कि वे लोग गलत जगह पर आ गए हैं। उन लोगों ने स्वामीजी को उस गन्दे तथा दुर्गन्धपूर्ण मार्ग से दूर ले जाने का प्रयास किया। परन्तु स्वामीजी टोली से अलग हो गए और सड़क के किनारे एक बेंच पर बैठीं अर्धनग्न महिलाओं के पास पहुँच गए। उन्होंने उनकी ओर करुणा-भरी दृष्टि से देखा और बुदबुदाने लगे - “अहा बच्चियो! अहा अभागिनो! इन्होंने अपने सौन्दर्य के लिए अपना देवत्व बलिदान कर दिया है। अब इनकी हालत तो देखो!” यह कहकर स्वामीजी रोने लगे और वे महिलाएँ, जो क्षण भर पूर्व भद्दे हावभाव दिखा रही थीं, लज्जित होकर संकोच से गड़ गयीं। उनमें से एक ने स्वामीजी के वस्त्र के छोर को चूमा और बोली - “यह देखो, ईश्वरद्रष्टा पुरुष!” उसने स्वामीजी की ओर उन्मुख होकर कहा - “आप ईश्वर के दूत हैं!” एक अन्य महिला ने पश्चात्ताप में अपने चेहरे को ढँक लिया। 

\vskip -6pt

\delimiter

एक बार स्वामीजी एक ट्रेन से यात्रा कर रहे थे। एक निर्धन मुसलमान फेरीवाला उबले हुए चने बेचता हुआ उनके डिब्बे में सवार हुआ। स्वामीजी ने उसे देखते ही अपने साथ के ब्रह्मचारी के सामने चने का गुणगान शुरू कर दिया। वे बोले - “देखो, चना आदमी को बलवान बनाता है।” इसके बाद वे फेरीवाले की ओर इंगित करते हुए ब्रह्मचारी से बोले - “अगर उससे थोड़ा-सा लिया जाए, तो कैसा रहेगा?” ब्रह्मचारी स्वामीजी के स्वभाव से भलीभाँति परिचित थे। वे तत्काल समझ गए कि स्वामीजी चने खाना नहीं, बल्कि उस गरीब आदमी की सहायता करना चाहते हैं। अतः ब्रह्मचारी ने उससे एक पैसे के चने लेकर उसे चार आने दे दिए। 

स्वामीजी की निगाहें बड़ी तेज थीं। उन्होंने ब्रह्मचारी से पूछा कि उसने कितने पैसे दिये। उसके - “चार आने” बताने पर स्वामीजी ने उसे स्नेहपूर्वक कहा - “मेरे बच्चे, यह पर्याप्त नहीं है। उसके घर में पत्नी और बच्चे हैं। उसे एक रुपया दे दे।” ब्रह्मचारी ने स्वामीजी के निर्देश का पालन किया। परन्तु स्वामीजी ने वे चने खाये नहीं। 

\vskip -6pt

\delimiter

एक बार स्वामी विज्ञानानन्द ने एक घटना बतायी थी, जिससे स्वामीजी की अलौकिक सहानुभूति प्रकट होती है। बेलूड़ मठ में वे स्वामीजी के कमरे के बगल वाले कक्ष में निवास किया करते थे। एक रात लगभग दो बजे वे अपने कमरे से बाहर आए और यह देखकर विस्मित रह गए कि स्वामीजी बेचैन होकर बरामदे में टहल रहे हैं। विज्ञानानन्द ने उनसे पूछा - “स्वामीजी, आप सोये नहीं? क्या आपको नींद नहीं आ रही है?” 

स्वामीजी बोले - “मैं तो भलीभाँति सो रहा था, परन्तु अचानक ही मुझे एक झटका-सा लगा और मैं उठ बैठा। मुझे लगता है कि दुनिया में कहीं कोई दुर्घटना हुई है और उसमें बहुत-से लोगों के प्राण गए हैं।” 

स्वामी विज्ञानानन्द ने पहले तो स्वामीजी की बात को गम्भीरता से नहीं लिया। यह उन्हें बड़ा अविश्वसनीय लगा कि अपने बिस्तर में पड़े हुए भी स्वामीजी को सुदूर हुई घटना का बोध हो सकता है; परन्तु अगली सुबह उन्हें समाचार-पत्रों में यह पढ़कर बड़ा ही आश्चर्य हुआ कि फीजी द्वीप के निकट हुए एक ज्वालामुखी-विस्फोट में अनेक लोगों की मृत्यु हो गयी है। विस्फोट का समय ठीक वही था, जिस क्षण स्वामीजी के हृदय को सदमा पहुँचा था। 

\vskip -5pt

\delimiter

स्वामीजी अपने शिष्यों के समक्ष भारत के प्राचीन गौरव का बखान किया करते थे, परन्तु साथ ही वे यह भी कहते कि भविष्य का भारत उससे भी कहीं अधिक महान् होगा। एक दिन बेलूड़ मठ में उन्होंने कहा - “मुझ पर विश्वास करो। मुझे एक दिव्य दर्शन हुआ है, जिसमें मैंने स्पष्ट रूप से देखा कि आगामी चार-पाँच शताब्दियों के दौरान भारत में क्या घटित होने वाला है।” 

एक अन्य समय भी उन्होंने कई उल्लेखनीय भविष्य-वाणियाँ की थीं। वे बोले - “अगले पचास वर्षों में भारत स्वाधीन हो जाएगा; और यह स्वाधीनता एक अप्रत्याशित ढंग से आएगी। अगले बीस वर्षों के भीतर ही एक महायुद्ध भड़क उठेगा और यदि पाश्चात्य देश अपने विशुद्ध भौतिकवाद का त्याग नहीं करते, तो एक और युद्ध अवश्यम्भावी है।” 

एक अन्य भविष्य-वाणी करते हुए स्वामीजी ने कहा -“स्वाधीन होने के बाद भारत पाश्चात्य भौतिकवाद को स्वीकार करेगा और इस हद तक भौतिक समृद्धि प्राप्त करेगा कि इस क्षेत्र में वह अपनी प्राचीन गरिमा को भी पीछे छोड़ जाएगा।” उन्होंने यह भी बताया कि अमेरिका जैसे देश क्रमशः अधिकाधिक आध्यात्मिक होते जाएँगे, क्योंकि भौतिक समृद्धि की ऊँचाइयों से वे लोग इस सरल सत्य को समझ लेंगे कि स्थूल भौतिकवाद से चिर-शान्ति की उपलब्धि नहीं हो सकती। 

एक अन्य सन्दर्भ में स्वामीजी ने कहा कि जब अंग्रेज लोग भारत छोड़कर चले जाएँगे, उसके बाद चीन द्वारा भारत पर आक्रमण किए जाने की आशंका है। 

\vskip -5pt

\delimiter

स्वामीजी की महान् प्रशंसिका तथा उन्हें अपना मित्र माननेवाली अमेरिकी महिला - जोसेफीन मैक्लाउड ने एक बार उनसे पूछा था - “मैं आपकी सर्वाधिक सहायता कैसे कर सकती हूँ?” स्वामीजी ने उत्तर दिया - “भारत से प्रेम करो।” 

\vskip -5pt

\delimiter

स्वयं के विषय में बोलते हुए एक बार स्वामीजी ने कहा था कि वे ‘घनीभूत भारत’ हैं। वस्तुतः उनका भारत-प्रेम इतना गहन था कि आखिरकार वे भारत की साकार प्रतिमूर्ति ही बन गए थे। विवेकानन्द तथा भारत मिलकर एकाकार हो गए थे। भगिनी निवेदिता के निम्नलिखित शब्दों में इसी विश्वास की प्रतिध्वनि है - “भारत ही स्वामीजी का महानतम भाव था।... भारत ही उनके हृदय में धड़कता था, भारत ही उनकी धमनियों में प्रवाहित होता था, भारत ही उनका दिवा-स्वप्न था और भारत ही उनकी सनक थी। इतना ही नहीं, वे स्वयं ही भारत बन गए थे। वे भारत की सजीव प्रतिमूर्ति थे। वे स्वयं ही - साक्षात् भारत, उसकी आध्यात्मिकता, उसकी पवित्रता, उसकी मेधा, उसकी शक्ति, उसकी अन्तर्दृष्टि तथा उसकी नियति के प्रतीक बन गए थे।” 

स्वामीजी की जीवनी का अध्ययन करने पर हमारे मन में ऐसी दृढ़ धारणा बन जाती है कि वे सभी दृष्टियों से अतुल्य थे। ऐसा कोई भी न था, जो भारत के प्रति उनसे अधिक लगाव रखता हो, जो भारत के प्रति उनसे अधिक गर्व का अनुभव करता रहा हो और जिसने उनसे अधिक उत्साहपूर्वक इसके हित के लिए कार्य किया हो। तथापि, यह भी सत्य है कि ऐसा कोई भी न था, जिसने भारतवासियों की दुर्बलता, कायरता तथा अयोग्यता पर उनसे अधिक निर्ममता तथा कठोरतापूर्वक प्रहार किया हो। उन्होंने दोनों ही छोरों का स्पर्श किया, क्योंकि वे भारतवासियों को अन्तरंग रूप से जानते थे। जैसे एक माता अपनी सन्तान के मन को तथा उसकी आवश्यकताओं को, स्वयं उससे भी अधिक अच्छी तरह समझ सकती है, उसी प्रकार स्वामीजी भी भारत की हर जरूरत को भलीभाँति समझ सकते थे। स्वामीजी के विचारों में हमें भारत और उसके भूतकाल, वर्तमान तथा भविष्य का एक यथार्थ चित्र प्राप्त होता है। इसीलिए रवीन्द्रनाथ ठाकुर ने रोमाँ रोलाँ से कहा था - “यदि आप भारत को समझना चाहते हैं, तो विवेकानन्द को पढ़िये।” 

\section*{सन्दर्भ ग्रन्थ }

\centerline{\textbf{हिन्दी: }}

\begin{enumerate}
\item युगनायक विवेकानन्द, स्वामी गम्भीरानन्द, तीन खण्डों में, द्वितीय संस्करण। 

 \item शिवानन्द स्मृति-संग्रह, भाग १, प्रथम संस्करण। 

\end{enumerate}

\centerline{\textbf{बंगला: }}

\begin{enumerate}
\item विश्वविवेक, असित कुमार वन्द्योपाध्याय, शंकरी प्रसाद बसु तथा शंकर, द्वितीय संस्करण। 

 \item प्रत्यक्षदर्शीर स्मृतिपटे स्वामी विज्ञानानन्द, ज्योतिर्मय वसुराय तथा सुरेश चन्द्र दास, प्रथम संस्करण। 

 \item स्वामीजीर स्मृति-संचयन, स्वामी निर्लेपानन्द, प्रथम संस्करण। 

 \item उद्बोधन, मासिक, वर्ष ६२, अंक १०। 

\end{enumerate}

\bgroup\renewcommand\labelenumi{\theenumi.}

\centerline{\textbf{अंग्रेजी:}}

\begin{enumerate}
\item \enginline{The Life of Swami Vivekananda, Eastern and Western Disciples, Vols. I and II, 5th Edition;}

 \item \enginline{Reminiscences of Swami Vivekananda – Eastern and Western Admirers, Ist Edition;}

 \item \enginline{Swami Vivekananda in America: New Discoveries, Marie Louise Burke, Vol. I, 3rd Edition}

\end{enumerate}

\egroup

\delimiter

