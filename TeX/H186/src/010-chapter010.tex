
\chapter{विश्व के पथ-प्रदर्शक }

\indentsecionsintoc

\addtoendnotes{\protect\newpage}

\toendnotes{विश्व के पथ-प्रदर्शक}

\addtoendnotes{\protect\begin{multicols}{3}}

\section*{सत्य के दो आदर्श - श्रुति और स्मृति}

\addsectiontoTOC{सत्य के दो आदर्श - श्रुति और स्मृति}

हमारे शास्त्रों में सत्य के दो आदर्श हैं - पहले को हम सनातन सत्य कहते हैं; और दूसरे प्रकार का सत्य पहले वाले की तरह प्रामाणिक न होने पर भी, विशेष-विशेष देश, काल और पात्र पर लागू होता है। श्रुति अर्थात् वेदों में जीवात्मा तथा परमात्मा के स्वरूप का पारस्परिक सम्बन्ध वर्णित है। मनु आदि की स्मृतियों, याज्ञवल्क्य आदि की संहिताओं और पुराणों तथा तंत्रों में दूसरे प्रकार का सत्य है। ये द्वितीय श्रेणी के ग्रन्थ तथा उपदेश श्रुति या वेदों के अधीन हैं, क्योंकि यदि स्मृति और श्रुति में विरोध हो, तो श्रुति को ही प्रमाण रूप में ग्रहण करना होगा। यही शास्त्र का विधान है। अभिप्राय यह कि श्रुति में जीवात्मा की नियति तथा उसके चरम लक्ष्य-विषयक मुख्य सिद्धान्तों का वर्णन है; और इनकी व्याख्या तथा विस्तार का काम स्मृतियों तथा पुराणों पर छोड़ दिया गया है - ये प्रथमोक्त सत्य के ही सविस्तार वर्णन हैं। साधारणतया पथ-प्रदर्शन के लिए श्रुति ही पर्याप्त है। धार्मिक जीवन बिताने के लिए, सार-तत्त्व के विषय में, श्रुति के कहे उपदेशों से अधिक, न कुछ कहा जा सकता है और न कुछ जानने की आवश्यकता ही है। इस विषय में जो भी आवश्यक है, वह श्रुति में है; जीवात्मा की सिद्धि-प्राप्ति के लिए जो-जो उपदेश चाहिए, श्रुति में उनका सम्पूर्ण वर्णन है। केवल विशेष अवस्थाओं के विधान श्रुति में नहीं है। समय-समय पर स्मृतियों में इनकी व्यवस्था दी गयी है। 

श्रुति की एक अन्य विशेषता यह है कि अनेक महर्षियों ने श्रुति में विभिन्न सत्य संकलित किए हैं, इनमें पुरुष अधिक हैं, परन्तु कुछ महिलाएँ भी हैं। इनके व्यक्तिगत जीवन या जन्मकाल आदि के विषय में हमें बहुत कम जानकारी है, परन्तु उनके सर्वोत्कृष्ट विचार, जिन्हें श्रेष्ठ आविष्कार कहना ही उपयुक्त होगा, हमारे देश के धर्म-साहित्य - वेदों में लिपिबद्ध तथा संरक्षित हैं। परन्तु स्मृतियों में ऋषियों की जीवनी और प्रायः उनके कार्यकलाप विशेष रूप से देखने को मिलते हैं, स्मृतियों में ही हम सर्वप्रथम अद्भुत, महा-शक्तिशाली, प्रभावोत्पादक और संसार को संचालित करनेवाले व्यक्तियों का परिचय प्राप्त करते हैं। कभी-कभी उनके समुन्नत और उज्ज्वल चरित्र उनके उपदेशों से भी अधिक उत्कृष्ट जान पड़ते हैं।... 

परन्तु श्रुति अथवा वेद ही हमारे धर्म के मूल स्रोत हैं, जो पूर्णतः निराकारवादी हैं। बड़े-बड़े आचार्यों, बड़े-बड़े अवतारों और महर्षियों का उल्लेख स्मृतियों और पुराणों में है।\endnote{ ५/१४३-४४;}


\section*{ऋषि कौन थे?}

\addsectiontoTOC{ऋषि कौन थे?}

वेदान्त नामक ज्ञानराशि ऋषि नामधारी व्यक्तियों के द्वारा आविष्कृत हुई। ऋषि शब्द का अर्थ है - मंत्रद्रष्टा। उन्होंने पहले ही से विद्यमान ज्ञान को प्रत्यक्ष किया था; वह ज्ञान तथा भाव उनके अपने विचार का फल नहीं था। जब कभी तुम सुनो कि वेदों के अमुक अंश के ऋषि अमुक हैं, तब यह मत सोचो कि उन्होंने उसे लिखा या अपनी बुद्धि द्वारा रचा है, बल्कि वे पहले से ही विद्यमान भावराशि के वे द्रष्टा मात्र हैं - वे भाव अनादि काल से ही इस संसार में विद्यमान थे, ऋषि ने उनका आविष्कार मात्र किया। ऋषिगण आध्यात्मिक आविष्कारक थे।\endnote{ ५/१९-२०;} 

ऋषि कौन हैं? उपनिषद् कहते हैं कि ‘ऋषि कोई साधारण व्यक्ति नहीं, वे मंत्रद्रष्टा हैं।’ ऋषि वे हैं, जिन्होंने धर्म को प्रत्यक्ष किया है, जिनके लिए धर्म - केवल पुस्तकों का अध्ययन, या युक्तजाल, या व्यावसायिक ज्ञान या वाग्वितण्डा मात्र नहीं, अपितु प्रत्यक्ष अनुभव है, अतीन्द्रिय सत्य का साक्षात् बोध है। यही ऋषित्व है।\endnote{ ५/७१;} 

चेतना पंचेन्द्रियों द्वारा सीमाबद्ध है। आध्यात्मिक जगत् के सत्य को प्राप्त करने के लिए मनुष्यों को चेतना की अतीत भूमि में, इन्द्रियों के परे जाना होगा। अब भी ऐसे व्यक्ति हैं, जो पंचेन्द्रियों की सीमा के परे जा सकते हैं। ये ऋषि कहलाते हैं, क्योंकि उन्होंने आध्यात्मिक सत्यों का साक्षात्कार किया है। 

जिस प्रकार हम अपने सामने के इस मेज को प्रत्यक्ष प्रमाण से जानते हैं, उसी प्रकार वेदोक्त सत्यों का प्रमाण भी प्रत्यक्ष अनुभव है। यह हम इन्द्रियों से देख रहे हैं और आध्यात्मिक सत्यों का भी हम जीवात्मा की ज्ञानतीत अवस्था में साक्षात् करते हैं। ऐसा ऋषित्व प्राप्त करना देश, काल, लिंग या जाति-विशेष के ऊपर निर्भर नहीं करता। वात्स्यायन निर्भयतापूर्वक घोषणा करते हैं कि यह ऋषित्व - ऋषियों की सन्तानों, आर्यों, अनार्यों; और यहाँ तक कि म्लेच्छों की भी संयुक्त सम्पत्ति है। 

यही वेदों का ऋषित्व है। हमको भारतीय धर्म के इस आदर्श को सर्वदा स्मरण रखना होगा; और मेरी तो इच्छा है कि संसार की अन्य जातियाँ भी इस आदर्श को समझकर याद रखें, क्योंकि इससे धार्मिक लड़ाई-झगड़े कम हो जाएँगे। शास्त्र-ग्रन्थों में धर्म नहीं होता; या फिर सिद्धान्तों, मतवादों, चर्चाओं या तार्किक उक्तियों में भी धर्म की प्राप्ति नहीं होती। धर्म तो स्वयं होने तथा बनने की चीज है। मेरे मित्रो, जब तक तुम ऋषि नहीं बनोगे और जब तक तुम्हारा आध्यात्मिक सत्यों के साथ साक्षात्कार नहीं होगा, तब तक तुम्हारा धार्मिक जीवन आरम्भ ही नहीं हुआ है।\endnote{ ५/१४८;} 

हमारे समाज के नेता कभी सेनानायक या राजा नहीं, अपितु ऋषि थे।... यह ऋषित्व किसी आयु या समय या किसी सम्प्रदाय या जाति की अपेक्षा नहीं रखता। वात्स्यायन कहते हैं - “सत्य का साक्षात्कार करना होगा और स्मरण रखना होगा कि हममें से प्रत्येक को ऋषि होना है।” साथ ही हमें अगाध आत्मविश्वास से भी सम्पन्न होना चाहिए। हम लोग सारे संसार में शक्ति-संचार करेंगे; क्योंकि सारी शक्ति हममें ही विद्यमान है। हमें धर्म का प्रत्यक्ष साक्षात्कार करना होगा, उसकी उपलब्धि करनी होगी, तभी हम ऋषित्व की उज्ज्वल ज्योति से पूर्ण होकर महापुरुष-पद प्राप्त कर सकेंगे, तभी हमारे मुख से जो वाणी निकलेगी, वह सुरक्षा की असीम स्वीकृति से पूर्ण होगी।\endnote{ ५/७१-७२;}


\section*{श्रुति और धर्माचार्य}

\addsectiontoTOC{श्रुति और धर्माचार्य}

केवल हमारे धर्म को छोड़कर संसार में प्रत्येक अन्य धर्म किसी धर्म-प्रवर्तक अथवा धर्म-प्रवर्तकों के जीवन से ही अविच्छिन्न रूप से सम्बद्ध है। ईसाई धर्म ईसा के, इस्लाम मुहम्मद के, बौद्ध धर्म बुद्ध के, जैन धर्म जिनों के और अन्य धर्म अन्य व्यक्तियों के जीवन के ऊपर प्रतिष्ठित हैं। इसीलिए इन महापुरुषों के जीवन के ऐतिहासिक प्रमाणों को लेकर उन धर्मों में जो काफी वाद-विवाद होता है, वह स्वाभाविक है। यदि कभी इन प्राचीन महापुरुषों के अस्तित्व-विषयक ऐतिहासिक प्रमाण दुर्बल होते हैं, तो उनकी धर्मरूपी अट्टालिका गिरकर चकनाचूर हो जाती है। हमारा धर्म व्यक्ति-विशेष पर नहीं, अपितु सनातन सिद्धान्तों पर प्रतिष्ठित है, अतः हम उस संकट से मुक्त हैं। ऐसा नहीं है कि किसी महापुरुष, यहाँ तक कि किसी अवतार के कथन को ही तुम अपना धर्म मानते हो। कृष्ण के वाणी से वेदों की प्रामाणिकता सिद्ध नहीं होती; बल्कि वे ही वेदों के अनुगामी हैं, इसी कारण कृष्ण के वाक्य प्रमाण माने जाते हैं। कृष्ण वेदों के प्रमाण नहीं हैं, अपितु वेद ही कृष्ण के प्रमाण हैं। कृष्ण की महानता इस बात में है कि वेदों के जितने प्रचारक हुए हैं, वे ही उनमें सर्वश्रेष्ठ हैं। अन्य अवतारों तथा सभी ऋषियों के सन्दर्भ में भी ऐसा ही समझो।\endnote{ ५/१४४;} 

हमारा प्रथम सिद्धान्त यह है कि मनुष्य की पूर्णता-प्राप्ति या उसकी मुक्ति के लिए, जो कुछ आवश्यक है, उसका वर्णन वेदों में है। कोई और नया आविष्कार नहीं हो सकता। समस्त ज्ञान के चरम लक्ष्य-स्वरूप पूर्ण एकत्व के आगे तुम कभी बढ़ नहीं सकते। इस पूर्ण एकत्व का आविष्कार बहुत पहले ही वेदों ने किया है, इससे अधिक अग्रसर होना असम्भव है। \textbf{तत्त्वमसि } - (वह ब्रह्म तुम्हीं हो) का आविष्कार होते ही आध्यात्मिक ज्ञान सम्पूर्ण हो गया। यह \textbf{‘तत्त्वमसि’ } वेदों में ही है। विभिन्न स्थान, काल, परिस्थिति तथा परिवेश के अनुसार समय-समय पर लोगों को केवल इसकी शिक्षा देना ही बाकी रह गया था। इस प्राचीन सनातन मार्ग पर व्यक्तियों का चलना ही शेष रह गया था। इसीलिए बीच-बीच में विभिन्न आचार्यों तथा महापुरुषों का आविर्भाव होता रहता है। उस तत्त्व का वर्णन गीता में श्रीकृष्ण की इस प्रसिद्ध वाणी के अतिरिक्त ऐसे सुन्दर और स्पष्ट रूप से अन्यत्र कहीं नहीं हुआ है -

\begin{verse}
यदा यदा हि धर्मस्य ग्लानिर्भवति भारत।\\अभ्युत्थानम् अधर्मस्य तदात्मानं सृजाम्यहम्॥ ४. ७ 
\end{verse}

\newpage

- “हे अर्जुन, जब-जब धर्म की हानि और अधर्म की वृद्धि होती है, तब-तब मैं धर्म की रक्षा तथा अधर्म के नाश हेतु समय-समय पर अवतार लेता रहता हूँ।” यही भारतीय धारणा है।\endnote{ ५/१४४-४५;} 

जिस प्रकार हमारे ईश्वर सगुण और निर्गुण दोनों हैं, ठीक उसी प्रकार हमारा धर्म भी पूर्णतः निर्गुण है - अर्थात् हमारा धर्म किसी व्यक्ति-विशेष के ऊपर निर्भर नहीं करता; तथापि इसमें असंख्य अवतार तथा महापुरुष स्थान पा सकते हैं। हमारे धर्म में जितने अवतार, महापुरुष और ऋषि हैं, उतने अन्य किस धर्म में हैं? इतना ही नहीं, हमारा धर्म यहाँ तक कहता है कि वर्तमान तथा भविष्य में और भी अनेक महापुरुष तथा अवतार आदि आविर्भूत होंगे। श्रीमद्-भागवत में कहा है - \textbf{अवताराः ह्यसंख्येयाः। } अतः हमारे धर्म में नये-नये धर्म-प्रवर्तकों के आने के मार्ग में कोई रुकावट नहीं है।\endnote{ ५/८०;} 

हमारे ऋषियों ने अत्यन्त प्राचीन काल में ही समझ लिया था कि मानव-जाति के अधिकांश लोगों को एक व्यक्तित्व की जरूरत है। उनको किसी-न-किसी प्रकार का साकार ईश्वर चाहिए। जिन बुद्धदेव ने साकार ईश्वर के विरुद्ध प्रचार किया था, उनके देहत्याग के बाद पचास वर्षों के भीतर ही उनके शिष्यों ने उनको ईश्वर मान लिया। साकार ईश्वर की भी आवश्यकता है और हम जानते हैं कि साकार ईश्वर की वृथा कल्पनाओं से भी बढ़कर, सजीव ईश्वर इस लोक में बीच-बीच में उत्पन्न होकर हम लोगों के बीच रहते भी हैं।... किसी भी प्रकार काल्पनिक ईश्वर की अपेक्षा, हमारी\break किसी भी काल्पनिक रचना की अपेक्षा, अर्थात् हम ईश्वर-विषयक जो भी धारणा बना सकते हैं, उसकी अपेक्षा वे पूजा के अधिक योग्य हैं। ईश्वर के सम्बन्ध में हम लोग जो भी धारणा रख\break सकते हैं, उसकी अपेक्षा श्रीकृष्ण बहुत बड़े हैं। हम अपने मन में जितने उच्च आदर्श का विचार\break कर सकते हैं, उसकी अपेक्षा श्रीकृष्ण बहुत बड़े हैं। हम अपने मन में जितने उच्च आदर्श का\break विचार कर सकते हैं, उसकी अपेक्षा बुद्धदेव अधिक उच्च आदर्श हैं, अधिक जीवन्त आदर्श हैं। इसीलिए सब प्रकार के काल्पनिक देवताओं को पदच्युत करके वे चिर काल से मनुष्यों द्वारा पूजे जा रहे हैं। 

हमारे ऋषि यह बात जानते थे, इसीलिए उन्होंने सारे भारतवासियों के लिए इन महापुरुषों की, इन अवतारों की पूजा का मार्ग खोल दिया है। इतना ही नहीं, जो हमारे सर्वश्रेष्ठ अवतार हैं, उन्होंने और भी आगे बढ़कर कहा है -

\begin{verse}
यद्यत विभूतिमत् सत्त्वं श्रीमदूर्जितमेव वा।\\तत्तदेवावगच्छ त्वं मम तेजोंऽशसम्भवम्॥ \vauthor{- गीता, १०. ४१ } 
\end{verse}

- “मनुष्यों में जहाँ भी अद्भुत आध्यात्मिक शक्ति का प्रकाश होता है, समझो वहाँ मैं वर्तमान हूँ, मुझसे ही इस आध्यात्मिक शक्ति का प्रकाश होता है।” यह हिन्दुओं के लिए सभी देशों के समस्त अवतारों की उपासना करने का द्वार खोल देता है। हिन्दू किसी भी देश के किसी भी साधु-महात्मा की पूजा कर सकते हैं।\endnote{ ५/१४५-४६;} 

तुम चाहे जिस अवतार या आचार्य को अपने जीवन का आदर्श बनाकर विशेष रूप से उपासना करना चाहो, कर सकते हो। यहाँ तक कि तुमको यह सोचने की भी स्वाधीनता है कि जिसको तुमने स्वीकार किया है, वे सब पैगम्बरों में महान् है और सब अवतारों में श्रेष्ठ है; इसमें कोई आपत्ति नहीं है, परन्तु तुम्हारे धर्म-साधन की नींव सनातन तत्त्व-समूह पर ही होनी चाहिए। यहाँ अद्भुत तथ्य यह है कि हमारे अवतार वहीं तक मान्य हैं, जहाँ तक वे वैदिक सनातन सत्य सिद्धान्तों के ज्वलन्त उदाहरण हैं।\endnote{ ५/८०-८१;}


\section*{महान् धर्माचार्यगण}

\vskip -4pt\addsectiontoTOC{महान् धर्माचार्यगण}

प्रत्येक देश के आध्यात्मिक जीवन में पतन और उत्थान के युग होते हैं। जब देश की अवनति होती है, तो लगता है कि उसकी जीवनी-शक्ति नष्ट हो गयी है - छिन्नभिन्न हो गयी है। परन्तु वह पुनः बल संग्रह करती है - उन्नति करने लगती है - जागृति की एक विशाल लहर उठती है और हर बार यही देखने में आता है कि इस विशालकाय तरंग के उच्चतम शिखर पर कोई दिव्य महापुरुष विराजमान हैं। एक ओर जहाँ वे उस तरंग - राष्ट्र के अभ्युत्थान के शक्तिदाता होते हैं, वहीं दूसरी ओर वे स्वयं उस महती शक्ति के फलस्वरूप होते हैं, जो उस अभ्युदय - उस तरंग का मूल है। इस प्रकार वे एक दूसरे पर क्रिया-प्रतिक्रिया करते रहते हैं - परस्पर के स्रष्टा तथा सृष्ट हैं - जनक तथा जन्य हैं। वे एक ओर समाज को अपनी महान् शक्ति से प्रभावित तथा अभिभूत करते हैं, वहीं दूसरी ओर समाज ही उनकी इस प्रचण्ड शक्ति के आविर्भाव का कारण होता है। ये ही संसार के महान् विचारक तथा मनीषी होते हैं, ये ही दुनिया के पैगम्बर, जीवन-दर्शन के सन्देश-वाहक ऋषि और ईश्वर के अवतार कहलाते हैं।\endnote{ ७/१७७;}


\section*{श्रीराम और सीता}

\vskip -4pt\addsectiontoTOC{श्रीराम और सीता}

प्राचीन वीर युगों के आदर्श स्वरूप, सत्य-परायणता तथा समग्र नैतिकता के साकार मूर्ति स्वरूप, आदर्श तनय, आदर्श पति, आदर्श पिता और सर्वोपरि आदर्श राजा - श्रीराम का चरित्र हमारे सम्मुख महान ऋषि वाल्मीकि के द्वारा प्रस्तुत किया गया है।... और सीता के विषय में क्या कहा जाए। तुम संसार के समस्त प्राचीन साहित्य को छान डालो, और मैं तुमसे निःसंकोच कहता हूँ कि तुम संसार के भावी साहित्य का भी मन्थन कर सकते हो, परन्तु उसमें से तुम सीता के समान दूसरा चरित्र नहीं निकाल सकोगे। सीता-चरित्र अद्वितीय है। यह चरित्र सदा के लिए एक ही बार चित्रित हुआ है। राम तो कदाचित् अनेक हो गए है, किन्तु सीता दूसरी नहीं हुईं। भारतीय स्त्रियों को जैसा होना चाहिए, सीता उनके लिए आदर्श हैं। स्त्री-चरित्र के जितने भारतीय आदर्श हैं, वे सब सीता के ही चरित्र से उत्पन्न हुए हैं।\endnote{ ५/१४९-५०;} 

राम, वृद्ध महाराजा (दशरथ) की आत्मा थे, लेकिन वे राजा थे और अपने वचन से पीछे नहीं हट सकते थे।\endnote{ ७/१४७;} 

सीता मूर्तिमान सतीत्व थीं। उन्होंने अपने पति के अतिरिक्त किसी अन्य पुरुष के शरीर का स्पर्श तक नहीं किया। 

\vskip 2pt

“पवित्र? वह तो सतीत्व-स्वरूपिणी है” - राम ने कहा।... 

\vskip 2pt

राम ने अपने शरीर का उत्सर्ग कर परलोक में सीता को प्राप्त किया। 

\vskip 2pt

सीता - पवित्र, निर्मल, समस्त दुःख झेलनेवाली! 

\vskip 2pt

भारत में उस प्रत्येक वस्तु को सीता नाम दिया जाता हैं, जो शुभ, निर्मल और पवित्र होती है; नारी में नारीत्व का जो गुण है, वह सीता है। सीता - धैर्यवान, सब दुःखों को झेलनेवाली, पतिव्रता, नित्य साध्वी पत्नी! उन्होंने तमाम कष्ट झेले, पर राम के विरुद्ध एक भी कठोर शब्द नहीं कहा। सीता में प्रतिहिंसा कभी नहीं थी। ‘सीता बनो!’\endnote{ ७/१४७;} 

\vskip 2pt


\section*{भगवान श्रीकृष्ण}

\addsectiontoTOC{भगवान श्रीकृष्ण}

वे एक ही स्वरूप में अपूर्व संन्यासी और अद्भुत गृहस्थ थे; उनमें अत्यन्त अद्भुत रजोगुण तथा शक्ति का विकास था और साथ ही वे अत्यन्त अद्भुत त्याग का जीवन बिताते थे। बिना गीता का अध्ययन किए कृष्ण-चरित्र कभी समझ नहीं आ सकता, क्योंकि वे अपने उपेदशों के जीवन्त स्वरूप थे। प्रत्येक अवतार, जिस सन्देश का प्रचार करने आए थे, वे उसके जीवित उदाहरण के रूप में अवतरित हुए। गीता के प्रचारक श्रीकृष्ण सदा भगवद्-गीता के उपदेशों की साकार मूर्ति थे, वे अनासक्ति के उज्ज्वल उदाहरण थे। उन्होंने अपना सिंहासन त्याग दिया और कभी उसकी चिन्ता नहीं की। जिनके कहने से ही राजा अपने सिंहासनों को छोड़ देते थे, ऐसे समग्र भारत के नेता ने स्वयं कभी राजा नहीं होना चाहा। उन्होंने बाल्यकाल में जिस सरल भाव से गोपियों के साथ क्रीड़ा की, जीवन की अन्य अवस्थाओं में भी उनका वह सरल स्वभाव नहीं छूटा। उनके जीवन की वह चिर-स्मरणीय घटना याद आती है, जिसे समझना अत्यन्त कठिन है। जब तक कोई पूर्ण ब्रह्मचारी और पवित्र स्वभाव का नहीं बनता, तब तक उसे इसके समझने की चेष्टा करना उचित नहीं। उस प्रेम के अत्यन्त अद्भुत विकास को, जो उस वृन्दावन की मधुर लीला में रूपक भाव से वर्णित हुआ है, प्रेमरूपी मदिरा के पान से जो उन्मत्त हुआ हो, उसको छोड़कर और कोई नहीं समझ सकता। कौन उन गोपियों के प्रेम से उत्पन्न विरह-यंत्रणा के भाव को समझ सकता है; जो प्रेम आदर्श स्वरूप है, जो प्रेम प्रेम के अतिरिक्त और कुछ नहीं चाहता, जो प्रेम स्वर्ग की भी आकांक्षा नहीं करता, जो प्रेम इहलोक और परलोक की किसी भी वस्तु की कामना नहीं करता?... वे श्रीकृष्ण के प्रति उपयोग किए जानेवाले किसी भी विशेषण को घृणा करती है, वे यह जानने की चिन्ता नहीं करतीं कि कृष्ण सृष्टिकर्ता हैं, वे यह जानने की चिन्ता नहीं करती कि वे सर्वशक्तिमान हैं, वे यह जानने की भी चिन्ता नहीं करती कि वे सर्व-सामर्थ्यवान हैं। वे केवल यही समझती हैं कि कृष्ण प्रेममय हैं, यही उनके लिए पर्याप्त है। गोपियाँ श्रीकृष्ण को केवल वृन्दावन का कृष्ण ही समझती हैं। अनेक सेनाओं के नेता, राजाधिराज श्रीकृष्ण उनके निकट सदा एक गोप ही रहे।

\begin{verse}
न धनं न जनं सुन्दरीं कवितां वा जगदीश कामये।\\मम जन्मनि जन्मनीश्वरे भवताद् भक्तिरहैतुकी त्वयि॥ 
\end{verse}

- ‘हे जगदीश! मैं धन, जन, कविता अथवा सुन्दरी - कुछ भी नहीं चाहता। हे ईश्वर, आपके प्रति जन्म-जन्मान्तरों में मेरी अहैतुकी भक्ति हो।’ यह अहैतुकी भक्ति, यह निष्काम कर्म, यह निरपेक्ष कर्तव्य-निष्ठा का आदर्श धर्म के इतिहास में एक नया अध्याय है। मानव-इतिहास में प्रथम बार भारतभूमि पर सर्वश्रेष्ठ अवतार श्रीकृष्ण के मुँह से पहली बार यह तत्त्व निकला था। भय और प्रलोभनों के धर्म सदा के लिए विदा हो गए और मनुष्य-हृदय में नरक-भय और स्वर्ग-सुख-भोग के प्रलोभन होते हुए भी, इस सर्वोत्तम आदर्श का उदय हुआ - प्रेम के निमित्त -प्रेम, कर्तव्य के निमित्त कर्तव्य, कर्म के निमित्त कर्म। 

और यह प्रेम कैसा है? मैंने तुम लोगों से कहा है कि गोपी-प्रेम को समझना बड़ा कठिन है।... पहले कांचन, नाम तथा यश और क्षुद्र मिथ्या संसार के प्रति आसक्ति को छोड़ो। तभी, केवल तभी तुम गोपी-प्रेम को समझोगे। यह इतना विशुद्ध है कि बिना सब कुछ छोड़े इसको समझने की चेष्टा करना ही अनुचित है। जब तक अन्तःकरण पूर्ण रूप से पवित्र नहीं होता, तब तक इसको समझने की चेष्टा करना वृथा है। हर समय जिनके हृदय में काम, धन, यशोलिप्सा के बुलबुले उठते हैं, ऐसे लोग गोपी-प्रेम की चर्चा करने तथा समझने का साहस करते हैं! 

कृष्ण-अवतार का मुख्य उद्देश्य इसी गोपी-प्रेम की शिक्षा है, यहाँ तक कि गीता का महान् दर्शन भी उस प्रेमोन्मत्तता की बराबरी नहीं कर सकता। क्योंकि गीता में साधक को धीरे-धीरे उसी चरम लक्ष्य - मुक्ति के साधन का उपदेश दिया गया है, परन्तु इसमें रसास्वाद की उन्मत्तता, प्रेम की मदोन्मत्तता विद्यमान है, यहाँ गुरु तथा शिष्य, शास्त्र तथा उपदेश, ईश्वर तथा स्वर्ग - सब एकाकार हैं, भय के भाव का चिह्न तक नहीं है। सब बह गया है - शेष रह गयी है केवल प्रेमोन्मत्तता। उस समय संसार का कुछ भी स्मरण नहीं रहता, भक्त उस समय संसार में उसी कृष्ण, एकमात्र उसी कृष्ण के अतिरिक्त और कुछ नहीं देखता, उस समय वह समस्त प्राणियों में कृष्ण के ही दर्शन करता है, उसका मुँह भी उस समय कृष्ण के ही समान दीखता है, उसकी आत्मा उस समय कृष्णमय हो जाती है। यह है कृष्ण की महिमा!... 

कृष्ण के जीवन-चरित्र में बहुत से ऐतिहासिक अन्तर्विरोध मिल सकते हैं, कृष्ण के चरित्र में बहुत से प्रक्षेप हो सकते हैं। ये सभी सत्य हो सकते हैं, तो भी उस समय समाज में जो एक अपूर्व नये भाव का उदय हुआ था, उसका कुछ आधार अवश्य था। अन्य किसी भी महापुरुष या पैगम्बर के जीवन पर विचार करने से यह जान पड़ता है कि वह पैगम्बर अपने पूर्ववर्ती कितने ही भावों का विकास मात्र है। हम देखते हैं कि उसने अपने देश में, यहाँ तक कि उस समय जैसी शिक्षा प्रचलित थी, केवल उसी का प्रचार किया है, यहाँ तक कि उस महापुरुष के अस्तित्व पर भी सन्देह हो सकता है, परन्तु मैं चुनौती देता हूँ कि कोई यह साबित कर दे कि कृष्ण के निष्काम कर्म, निरपेक्ष कर्तव्य-निष्ठा और निष्काम प्रेम-तत्त्व के ये उपदेश संसार में मौलिक आविष्कार नहीं है। यदि ऐसा नहीं कर सकते, तो यह अवश्य स्वीकार करना पड़ेगा कि किसी एक व्यक्ति ने निश्चय ही इन तत्त्वों को प्रस्तुत किया है। यह स्वीकार नहीं किया जा सकता कि ये तत्त्व किसी दूसरे मनुष्य से लिए गए हैं। कारण यह कि कृष्ण के उत्पन्न होने के समय सर्व-साधारण में इन तत्त्वों का प्रचार नहीं था। भगवान श्रीकृष्ण ही इनके प्रथम प्रचारक हैं। उनके शिष्य व्यासदेव ने पूर्वोक्त तत्त्वों का आम लोगों में प्रचार किया।... 

हम उस आदर्श-प्रेमी श्रीकृष्ण का वर्णन छोड़कर और भी नीचे की तह में प्रवेश करके गीता-प्रचारक श्रीकृष्ण की विवेचना करेंगे। यहाँ भी हम देखते हैं कि गीता के समान वेदों का भाष्य कभी नहीं बना है और न बनेगा। श्रुति या उपनिषदों का तात्पर्य समझना बड़ा कठिन है, क्योंकि विभिन्न भाष्यकारों ने अपने-अपने मतानुसार उनकी व्याख्या करने की चेष्टा की है। अन्त में, जो स्वयं श्रुति के प्रेरक हैं, उन्हीं भगवान ने आविर्भूत होकर गीता के प्रचारक रूप से श्रुति का जैसा अर्थ समझाया, आज भारत को उसी व्याख्या-प्रणाली की आवश्यकता है, सारे संसार को उसी की आवश्यकता है।... एक अद्वैतवादी भाष्यकार ने किसी उपनिषद् की व्याख्या की, जिसमें बहुत से द्वैतभाव के वाक्य हैं। उसने उनको तोड़-मरोड़कर कुछ अर्थ निकाला और उन सबका अपनी व्याख्या के अनुरूप मनमाना अर्थ लगा लिया। फिर द्वैतवादी भाष्यकार ने भी व्याख्या करनी चाही, उनमें अनेक अद्वैतमूलक अंश हैं, जिनसे द्वैतमूलक अर्थ ग्रहण करने के लिए उसने भी खींचतान की। परन्तु गीता में इस तरह के किसी भाव को बिगाड़ने की चेष्टा आपको नहीं मिलेगी।\endnote{ ५/१५१-५५;} 

गीता के मूल नायक हैं श्रीकृष्ण।... भारत में अन्य अवतारों की अपेक्षा श्रीकृष्ण के उपासक संख्या में अधिक हैं। उनके उपासकों का विश्वास है कि श्रीकृष्ण पूर्णावतार हैं और शंका करने पर वे कहते हैं - बुद्ध तथा अन्य अवतारों की ओर देखो। वे केवल संन्यासी थे, गृहस्थों के प्रति उनके हृदय में कोई सहानुभूति नहीं थी, और होती भी कैसे? पर श्रीकृष्ण के जीवन को देखो, पुत्र, पिता, राजा - सभी दृष्टियों से वे महान् हैं और वे आजीवन इस महान् शिक्षा को आचरण में लाते रहे - “जो मनुष्य प्रबल कर्मशीलता के बीच रहता हुआ भी निष्कर्म भाव की मधुर शान्ति का उपभोग करता है और महा-निस्तब्धता में भी जो अत्यन्त कर्मशील रह सकता है, उसी ने जीवन के रहस्य को ठीक-ठीक जाना है।” श्रीकृष्ण ने इस स्थिति को प्राप्त करने का जो मार्ग बताया है - वह है अनासक्ति योग। सारे कर्म करो, परन्तु उसमें आसक्त मत होओ। तुम सर्वदा निर्विकार, शुद्ध-बुद्ध और मुक्त आत्मा हो - निर्लिप्त और साक्षी हो। हमारे दुःखों का मूल कारण कर्म नहीं, आसक्ति है। उदाहरणार्थ, धन की ही बात लो! सम्पत्तिशाली होना बड़ी अच्छी बात है। कृष्ण कहेंगे - धनोपार्जन करो, उसके लिए जीतोड़ परिश्रम करो, पर उसमें आसक्ति मत रखो। सन्तान, पत्नी, पति, कुटुम्बी, यश आदि के विषय में भी यही भाव रखो। उनका त्याग करने की कोई आवश्यकता नहीं है, केवल उनमें आसक्त मत बनो। आसक्ति के भाजन तो केवल प्रभु ही हो सकते हैं - और कुछ नहीं। सबके लिए परिश्रम करो, उन्हें प्यार करो, उनका हित सम्पन्न करो, अवसर आने पर उनके लिए अपने जीवन का बलिदान भी कर दो, परन्तु उनमें आसक्त मत होओ। श्रीकृष्ण का अपना स्वयं का जीवन उनके इस उपदेश का एक प्रज्वलन्त उदाहरण है।\endnote{ ७/१६३-६४;}


\section*{गौतम बुद्ध}

\addsectiontoTOC{गौतम बुद्ध}

इसके बाद ही भारतीय इतिहास का एक शोकजनक अध्याय शुरू होता है। हम गीता में भी भिन्न-भिन्न सम्प्रदायों के विरोध के कोलाहल की दूर से आती हुई आवाज सुन पाते हैं और देखते हैं कि समन्वय के वे अद्भुत प्रचारक भगवान श्रीकृष्ण बीच में पड़कर विरोध को हटा रहे हैं। वे कहते हैं - सारा जगत् मुझमें उसी प्रकार गुँथा हुआ है, जिस तरह धागे में मणियाँ गुँथी रहती है -

\begin{verse}
मत्तः परतरं नान्यत्किंचिदस्ति धनंजय।\\मयि सर्वमिदं प्रोतं सूत्रे मणिगणा इव॥ ७. ७ 
\end{verse}

साम्प्रदायिक झगड़ों की दूर से सुनायी देनेवाली धीमी आवाज हम तभी से सुन रहे हैं। सम्भव है कि श्रीकृष्ण के उपदेश से ये झगड़े कुछ देर के लिए रुक गए हों और समन्वय तथा शान्ति का संचार हुआ हो, परन्तु यह विरोध फिर उत्पन्न हुआ। केवल धर्ममत के आधार पर ही नहीं, सम्भवतः वर्ण के आधार पर भी यह विवाद चलता रहा - हमारे समाज के दो प्रबल अंग ब्राह्मणों तथा क्षत्रियों, राजाओं तथा पुरोहितों के बीच विवाद आरम्भ हुआ था। और एक हजार वर्ष तक जिस विशाल तरंग ने समग्र भारत को सराबोर कर दिया था, उसके सर्वोच्च शिखर पर हम एक अन्य महामहिम मूर्ति को देखते हैं और वे हमारे शाक्यमुनि गौतम हैं। उनके उपदेशों और प्रचार-कार्य से तुम सभी अवगत हो। हम उनको ईश्वरावतार समझकर उनकी पूजा करते हैं, नैतिकता का इतना बड़ा निर्भीक प्रचारक संसार में दूसरा कोई पैदा नहीं हुआ, कर्मयोगियों में सर्वश्रेष्ठ स्वयं श्रीकृष्ण ही मानो शिष्य रूप से अपने उपदेशों को कार्यरूप में परिणत करने के लिए पैदा हुए।\endnote{ ५/१५६-५७;} 

जिस समय बुद्ध ने जन्म लिया, भारत को एक महान् धर्माचार्य - एक पैगम्बर की आवश्यकता थी। पुरोहितों का एक प्रबल संगठन यहाँ पहले से ही विद्यमान था।... पुरोहित एक ईश्वर में विश्वास करते हैं, परन्तु उनका कहना है कि इस ईश्वर के निकट पहुँच पाना और उसे जान पाना - केवल उन्हीं के माध्यम से हो सकता है।\endnote{ ७/२०१-०६;} 

परन्तु जहाँ पुरोहित फल-फूल रहे थे, वहीं संन्यासी कहलाने वाले कवि-मनीषियों का भी अस्तित्व था।... इस प्रकार प्राचीन भारत के इन कवि-मनीषियों ने पुरोहितों के मार्ग को नकार कर शुद्ध सत्य की घोषणा की। उन्होंने पुरोहितों की शक्ति को ध्वस्त करने का प्रयास किया और थोड़े सफल भी हुए। लेकिन दो ही पीढ़ियों में उनके शिष्य पुनः अन्धविश्वासों - पुरोहितों के पेचीले रास्तों - में वापस लौट गए और स्वयं भी पुरोहित बन गए - “सत्य को तुम हमारे द्वारा ही पा सकते हो।” सत्य फिर जम गया और पपड़ियों को तोड़ने तथा सत्य को मुक्त करने के लिए पैगम्बर फिर आए और यह क्रम इसी प्रकार चलता रहा। हाँ, सदा उस मानव का, उस पैगम्बर का आविर्भाव होते रहना अनिवार्य है, अन्यथा मानवता मर जाएगी। 

पुरोहितों का कहना है कि केवल वे ही सत्य के योग्य लोग हैं! जनता उसके योग्य नहीं! सत्य को पतला करना जरूरी है! उसमें थोड़ा पानी मिला लो! 

गीता और ‘पर्वत पर उपदेश’ (\enginline{Sermon on the Mount }) को लो, वे मानो साक्षात् सरलता हैं। राह चलनेवाला भी उन्हें समझ सकता है। कितने महान्! उनमें सत्य को स्पष्टता तथा सरलता से प्रकट किया गया है। लेकिन नहीं, पुरोहित यह मान ही नहीं सकते कि सत्य को इतने सीधे ढंग से प्रकट किया जा सकता है। वे दो हजार स्वर्गों और दो हजार नरकों की बात करते हैं। यदि लोग उनके नुस्खों का सेवन करेंगे, तो वे स्वर्ग में जाएँगे। यदि वे नियमों का पालन नहीं करते, तो नरक में जाएँगे! 

लेकिन लोग सत्य से अवगत हो ही जाएँगे। कुछ लोग डरते हैं कि यदि सम्पूर्ण सत्य सबको दे दिया जाएगा, तो उससे उन्हें हानि पहुँचेगी। वे कहते हैं कि सबको विशुद्ध सत्य नहीं दिया जाना चाहिए। लेकिन सत्य के साथ समझौता करते रहने से जगत् को कोई विशेष लाभ नहीं हुआ। वह जैसा है, अब वह उससे भी बुरा और क्या होगा? सत्य को बाहर लाओ! यदि वह वास्तविक है, तो कल्याण ही करेगा। जब लोग इसका विरोध करते है और अन्य तरीके प्रस्तावित करते हैं, तो वे केवल जादू-टोनों (अन्धविश्वासों) का ही मण्डन करते हैं। 

बुद्ध के समय में भारत इनसे परिपूर्ण था। जन-समुदाय थे, परन्तु उन्हें समस्त ज्ञान से वंचित कर दिया गया था। यदि वेदों का एक शब्द भी किसी मनुष्य के कान में पड़ जाता, तो उसे भीषण दण्ड दिया जाता था। जिन वेदों में प्राचीन हिन्दुओं द्वारा खोजे आध्यात्मिक सत्य संचित हैं, पुरोहितों ने उन वेदों को रहस्य बना रखा था। 

अन्ततः एक व्यक्ति इसे और अधिक सहन नहीं कर सका। उसके पास मस्तिष्क, शक्ति और हृदय - विस्तीर्ण आकाश जैसा असीम हृदय था। उसने देखा कि पुरोहित लोग किस प्रकार जनता का नेतृत्व कर रहे हैं और वे किस प्रकार से अपनी शक्ति में गौरव का अनुभव कर रहे हैं। उसने इस सम्बन्ध में कुछ करना चाहा। वह किसी पर अपना शक्तिपूर्ण अधिकार नहीं चाहता था। वह मनुष्यों के मानसिक और आध्यात्मिक बन्धनों को तोड़ डालना चाहता था। उसका हृदय विशाल था। वैसा हृदय हमारे आसपास के अनेक लोगों में हो सकता है और हम भी दूसरों की सहायता करना चाहते हैं। परन्तु हमारे पास वह मस्तिष्क नहीं है; हम वे साधन तथा उपाय नहीं जानते, जिनके द्वारा सहायता दी जा सकती है। परन्तु इस व्यक्ति के पास आत्माओं के बन्धनों को तोड़ फेंकने के उपायों को खोज निकालने वाला मस्तिष्क था। उसने जान लिया कि मनुष्य दुःख से पीड़ित क्यों होता है और उसने दुःख से निवृत्त होने का मार्ग ढूँढ़ निकाला। वह अत्यन्त कुशल व्यक्ति था। उसने सब बातों का समाधान कर लिया। उसने बिना किसी भेद-भाव के सभी लोगों को उपदेश दिया और उन्हें सम्बोधि की शान्ति प्राप्त करने के लिए प्रेरित किया। वह व्यक्ति थे बुद्ध!... 

उन्होंने हर किसी को, बिना कोई भेद-भाव किए, वेदों के दर्शन के सारांश की ही शिक्षा दी। उन्होंने ये उपदेश सारे विश्व को दिए, क्योंकि मानव की समता उनके महान् सन्देशों में से एक है। सब मनुष्य बराबर हैं। वहाँ किसी के साथ कोई रियायत नहीं! बुद्ध समता के महान उपदेशक थे। उनकी शिक्षा थी कि आध्यात्मिकता प्राप्त करने में हर स्त्री-पुरुष को समान अधिकार है। उन्होंने पुरोहितों तथा अन्य जातियों के बीच का अन्तर मिटा दिया। उन्होंने निम्नतम लोगों को भी उच्चतम उपलब्धियों का अधिकारी बताया। उन्होंने हर किसी के लिए निर्वाण के द्वार खोल दिए।... 

उनका धर्म-सिद्धान्त यह था - हमारे जीवन में दुःख क्यों है? - इसलिए कि हम स्वार्थी हैं। हम अपने लिए वस्तुओं की कामना करते हैं - इसी कारण दुःख का अस्तित्व है। इससे छुटकारा पाने का मार्ग क्या हैं? - आत्मा का परित्याग करना। आत्मा की सत्ता नहीं है; यह प्रपंचात्मक जगत् - जिसे हम प्रत्यक्ष देखते हैं, इसी की सत्ता है। जन्म-मरण-चक्र के पीछे विद्यमान आत्मा नामक कोई वस्तु नहीं है। है - विचार-प्रवाह - एक विचार उत्तरोत्तर दूसरे विचार के पीछे चलता रहता है, प्रत्येक विचार एक ही क्षण में अस्तित्व को प्राप्त करता है और अस्तित्वहीन हो जाता है, बस केवल इतना; विचार का कोई ज्ञाता नहीं है, आत्मा नहीं है। शरीर प्रति क्षण परिवर्तित होता रहता है, वैसे ही मन तथा चेतना भी। अतः आत्मा एक भ्रम है। सारी स्वार्थपरता इस आत्मा को, इस भ्रान्तिजन्य आत्मा को पकड़े रहने से उत्पन्न होती है। यदि हम इस सत्य को जान लें कि आत्मा नहीं है, तो हम स्वयं सुखी होंगे और दूसरों को भी सुखी बनाएँगे। 

यही था वह तत्त्व, जिसका बुद्ध ने उपदेश दिया। उन्होंने केवल बातें नहीं की - वे संसार के लिए स्वयं अपना जीवन तक देने को प्रस्तुत थे।\endnotemark[\theendnote] 

दुनिया में वे ही एकमात्र ऐसे थे, जो यज्ञों में पशुबलि रोगने हेतु, किसी प्राणी के जीवन की रक्षा के लिए अपना जीवन भी न्यौछावर करने को तत्पर रहते थे। एक बार उन्होंने एक राजा से कहा - “यदि किसी निरीह पशु की बलि देने से तुम्हें स्वर्ग-प्राप्ति हो सकती है, तो मनुष्य की बलि देने से और भी उच्च फल की प्राप्ति होगी। राजन्, उस पशु के पाश काटकर मेरी आहुति दे दो - इससे शायद तुम्हारा अधिक कल्याण हो।” राजा स्तब्ध रह गया।\endnote{ ७/१९८;} 

उन्होंने कहा - “यह पशुबलि एक दूसरा अन्धविश्वास है। ईश्वर और आत्मा - दो बड़े अन्धविश्वास हैं। ईश्वर पुरोहितों द्वारा आविष्कृत एक अन्धविश्वास मात्र है। जैसा कि ब्राह्मण शिक्षा देते हैं - यदि ईश्वर है, तो इस जगत् में इतना दुःख क्यों हैं? वह भी मेरे ही समान कारण के नियम का दास है। यदि वह कारण के नियम से आबद्ध नहीं है, तो सृष्टि क्यों रचता है? ऐसा ईश्वर जरा भी सन्तोषजनक नहीं है। जगत् का शासक स्वर्ग में है, जो विश्व पर अपनी मनमौजी इच्छा से शासन करता है और हमें यहाँ दुःख में मरने के लिए छोड़ देता है - हम पर एक क्षण को दृष्टि डाल लेने तक की उसमें भलमनसाहत नहीं है। हमारा सारा जीवन अविच्छिन्न दुःख है। लेकिन इतना ही दण्ड काफी नहीं है - मृत्यु के बाद हमें ऐसे स्थानों को जाना है, जहाँ हमें दूसरे दण्ड प्राप्त होंगे। तो भी हम जगत् के इस स्रष्टा को प्रसन्न करने के लिए निरन्तर तरहतरह के विधि-विधान और अनुष्ठान किया करते हैं!” 

बुद्ध ने कहा - “ये सारे अनुष्ठान गलत हैं। जगत् में केवल एक ही आदर्श है। सारे मोह को ध्वस्त कर दो, जो सत्य है, वही बचा रहेगा। बादल जैसे ही हटेंगे, सूर्य चमक उठेगा।” इस आत्मा को कैसे मारा जाए? पूर्णरूपेण निःस्वार्थ हो जाओ, एक चींटी तक के लिए अपना जीवन देने को तैयार रहो। किसी अन्धविश्वास के लिए कर्म न करो, न किसी ईश्वर को प्रसन्न करने के लिए, न कोई पुरस्कार पाने के लिए; वरन् इसलिए करो कि तुम अपनी आत्मा को मारकर अपनी मुक्ति पाना चाहते हो। उपासना, प्रार्थना और उसी तरह का बाकी सब निरर्थक है। तुम सभी कहते हो, ‘मैं ईश्वर को धन्यवाद देता हूँ - लेकिन वह रहता कहाँ है? तुम नहीं जानते, परन्तु इसके बावजूद तुम सभी उस ईश्वर के पीछे दीवाने हो रहे हो। 

हिन्दू लोग अपने ईश्वर के सिवा बाकी सब कुछ छोड़ सकते हैं। ईश्वर को अस्वीकार करने का अर्थ है - भक्ति के पैरों-तले से धरती ही खींच लेना। भक्ति और ईश्वर से हिन्दुओं को चिपके ही रहना होगा।... 

तो भी बुद्ध का धर्म तेजी से फैला। ऐसा उस अद्भुत प्रेम के कारण हुआ, जो मानवता के इतिहास में पहली बार एक विशाल हृदय से प्रवाहित हुआ और जिसने अपने को केवल मानव-मात्र की ही नहीं, प्राणि-मात्र की सेवा में अर्पित कर दिया था - ऐसा प्रेम, जिसे जीव-मात्र के लिए मुक्ति का एक मार्ग खोज निकालने के अतिरिक्त अन्य किसी बात की चिन्ता नहीं थी। 

मानव ईश्वर से तो प्रेम करता था, लेकिन अपने मनुष्य भाई के बारे में सब कुछ भुला बैठा था। जो मनुष्य ईश्वर के नाम पर अपने प्राण दे सकता है, वह पलटकर ईश्वर के ही नाम पर अपने बन्धु-मानव की हत्या भी कर सकता है। संसार की यही दशा थी। लोग ईश्वर की महिमा के लिए पुत्र की बलि दे देते थे, ईश्वर की महिमा के लिए राष्ट्रों को लूट लेते थे, ईश्वर की महिमा के लिए हजारों प्राणियों का वध कर डालते थे और ईश्वर की महिमा के लिए धरती को रक्तरंजित कर डालते थे। यह पहला अवसर था, जब उन्होंने एक अन्य ईश्वर - मानव - की ओर देखा। मनुष्य से प्रेम किया जाना चाहिए। मानव-मात्र के प्रति उत्कट प्रेम की यह पहली लहर थी - मिलवटों से रहित, ज्ञान की प्रथम लहर - जिसने भारत से निकलकर, दक्षिण, उत्तर, पूर्व, पश्चिम एक-एक करके बहुत-से देशों को आप्लावित कर डाला।... 

मैं आजीवन बुद्ध का परम अनुरागी रहा हूँ।... अन्य किसी की अपेक्षा मैं उस चरित्र के प्रति सबसे अधिक श्रद्धा रखता हूँ - वह साहस, वह निर्भीकता, वह विराट् प्रेम! उनका जन्म मनुष्य के कल्याण के लिए हुआ था। लोग अपने लिए ईश्वर की, सत्य की खोज कर सकते हैं; परन्तु उन्होंने अपने निमित्त सत्य का ज्ञान प्राप्त करने की चिन्ता भी नहीं की। सत्य की खोज उन्होंने इसलिए की कि लोग दुःख से पीड़ित थे। उनकी सहायता कैसे की जाए - यही उनकी एकमात्र चिन्ता थी। अपने सारे जीवन उन्होंने स्वयं के लिए एक विचार तक नहीं किया।... 

यदि वे जीवन में महान् थे, तो मृत्यु में भी महान् थे। उन्होंने तुम्हारे अमेरीकी आदिवासियों से मिलती-जुलती जाति के एक सदस्य द्वारा भिक्षा में दिये गए खाद्य पदार्थ को खा लिया। इस जाति के लोग बिना भक्ष्य-अभक्ष्य का कोई विचार किए सब कुछ खा लेते हैं, इसलिए हिन्दू लोग इनका स्पर्श तक नहीं करते। उन्होंने अपने शिष्यों से कहा, “तुम लोग इस चीज को मत खाना, परन्तु मैं इसे अस्वीकार नहीं कर सकता। उस आदमी के पास जाओ और कहो कि उसने मेरी - मेरे जीवन की एक बहुत बड़ी सेवा की है - उसने मुझे मेरे शरीर से मुक्त कर दिया है।” एक बूढ़ा आदमी आया और उनके निकट बैठ गया - वह सैकड़ों मील पैदल चलकर शास्ता के दर्शन करने आया था। बुद्ध ने उसे उपदेश दिया। जब उन्होंने एक शिष्य को रोते देखा, तो यह कहते हुए उसकी भर्त्सना की, “यह क्या? क्या यही मेरी सारी शिक्षाओं का फल है? किसी मिथ्या बन्धन को न रहने दो - न मुझ पर निर्भरता को, और न इस जाते हुए व्यक्तित्व के मिथ्या महिमा-मण्डन को। बुद्ध व्यक्ति नहीं है, वह एक अनुभूति है। अपनी मुक्ति स्वयं ही प्राप्त करो।” 

मृत्यु के समय भी उन्होंने अपने लिए किसी विशिष्टता का दावा नहीं किया। इसीलिए मैं उन्हें श्रद्धा करता हूँ।\endnote{ ७/२०६-१२;}


\section*{श्री शंकराचार्य}

\addsectiontoTOC{श्री शंकराचार्य}

बुद्ध के उपदेश में खतरे का एक तत्त्व था - वह एक सुधारक धर्म था। लोगों में विराट् आध्यात्मिक परिवर्तन लाने के लिए उन्हें अनेक नकारात्मक शिक्षाएँ देनी पड़ी। परन्तु यदि कोई धर्म नकारात्मक पक्ष को अत्यधिक महत्त्व देता है, तो अन्ततः उसके नष्ट हो जाने का खतरा भी रहता है। कोई भी सुधारक-सम्प्रदाय केवल सुधारक होकर जीवित नहीं रह सकता। केवल निर्माणात्मक तत्त्व - सच्ची प्रेरणा अर्थात् सिद्धान्त - ही जीवित रहते हैं। सुधार सम्पन्न हो जाने के उपरान्त सकारात्मक पक्ष को महत्त्व दिया जाना चाहिए। भवन का निर्माण हो चुकने के बाद मचान को हटा लेना आवश्यक है। 

भारत में कुछ ऐसा हुआ कि समय बीतने के साथ-साथ बुद्ध के अनुयायी उनकी शिक्षाओं के नकारात्मक पक्ष पर अत्यधिक बल देने लगे और यही अन्ततः उनके धर्म के अधःपतन का कारण सिद्ध हुआ।\endnote{ ७/२१०;} 

बौद्ध-धर्म की अवनति से जिन घृणित आचारों का आविर्भाव हुआ, उनका वर्णन करने के लिए न मेरे पास समय है और न इच्छा ही। अति कुत्सित अनुष्ठान-पद्धतियाँ, बड़े भयानक तथा अश्लील ग्रन्थ - जो मनुष्यों द्वारा न तो कभी लिखे गए थे और न मनुष्य ने जिनकी कभी कल्पना तक की थी - अत्यन्त भीषण पाशविक अनुष्ठान-पद्धतियाँ, जो कभी धर्म के नाम पर प्रचलित नहीं हुई थीं - ये सभी पतनशील बौद्ध धर्म की सृष्टि थीं।\endnote{ ५/१५९;} 

परन्तु भारत को जीवित रहना था, इसीलिए पुनः भगवान का आविर्भाव हुआ। जिन्होंने कहा था, ‘जब-जब धर्म की हानि होती है, तब-तब मैं आता हूँ’ - वे फिर से आये। इस बार भगवान का दक्षिणी प्रदेश में आविर्भाव हुआ। उस ब्राह्मण युवक का, उस अद्भुत प्रतिभाशाली शंकराचार्य का अभ्युदय हुआ, जिसके बारे में कहते हैं कि उसने सोलह वर्ष की उम्र में ही अपनी सारी ग्रन्थ-रचना समाप्त कर ली थी। इस सोलह वर्ष के अद्भुत बालक और उसकी रचनाओं पर आधुनिक सभ्य संसार विस्मित हो रहा है। उसने समग्र भारत को उसके प्राचीन विशुद्ध मार्ग पर ले आने का संकल्प किया था। परन्तु यह कार्य कितना कठिन और विशाल था, इसका जरा विचार करो। (भारत की तत्कालीन परिस्थितियों के विषय में कुछ बातें मैं पहले ही बता चुका हूँ।)... तातार, बलूची आदि दुनिया की सभी भयानक जातियों के लोग भारत में आकर बौद्ध बने और हमारे साथ मिल गए। वे लोग अपने राष्ट्रीय आचारों को भी साथ लाए थे। इस प्रकार हमारा राष्ट्रीय जीवन अत्यन्त भयानक पाशविक आचारों से परिपूर्ण हो गया। उक्त ब्राह्मण युवक को बौद्धों से विरासत में यही मिला था और तभी से अब तक भारत में इसी अधःपतित बौद्ध धर्म पर वेदान्त की पुनर्विजय का कार्य सम्पन्न हो रहा है। अब भी यह कार्य जारी है, अब भी यह पूरा नहीं हुआ है। महान् दार्शनिक शंकर ने आकर दिखलाया कि बौद्ध धर्म और वेदान्त के सारांश में विशेष अन्तर नहीं है। परन्तु तथागत के शिष्यों ने उनके उपदेशों का मर्म न समझकर, स्वयं को हीन बना लिया और आत्मा तथा ईश्वर का अस्तित्व अस्वीकार करके नास्तिक हो गये। आचार्य शंकर ने यही दिखलाया और तब बौद्ध लोगों ने अपने प्राचीन धर्म में लौटना आरम्भ किया।... 

शंकर की प्रतिभा प्रखर थी, परन्तु उनका हृदय... उतना उदार नहीं था।... मेरी समझ में नहीं आता कि लोग शंकर को अनुदार मत का पोषक क्यों कहते हैं। उनके लिखे ग्रन्थों में ऐसा कुछ भी नहीं मिलता, जो उनकी संकीर्णता का परिचय दे। जिस प्रकार भगवान बुद्धदेव के उपदेश उनके शिष्यों के हाथों में बिगड़ गए, उसी प्रकार शंकराचार्य के उपदेशों पर संकीर्णता का जो दोष लगाया जाता है, सभ्भवतः वह उनकी शिक्षा के कारण नहीं, अपितु उनके शिष्यों की अयोग्यता के कारण~है।\endnote{ ५/१५९-६०;}


\section*{आचार्य रामानुज}

\addsectiontoTOC{आचार्य रामानुज}

तब मेधावी रामानुज का अभ्युदय हुआ।... रामानुज का हृदय शंकर की अपेक्षा अधिक विशाल था। उन्होंने पददलितों की पीड़ा का अनुभव किया और उनके प्रति सहानुभूति दिखायी। उन्होंने उन दिनों प्रचलित अनुष्ठान-पद्धतियों में यथाशक्ति सुधार किया और उन लोगों के लिए नयी अनुष्ठान-पद्धतियों, नयी उपासना-प्रणालियों की सृष्टि की, जिनके लिए ये अति आवश्यक थीं। इसी के साथ-साथ उन्होंने ब्राह्मण से लेकर चाण्डाल तक - सबके लिए सर्वोच्च आध्यात्मिक उपासना का द्वार खोल दिया। यह था रामानुज का कार्य!... 

रामानुज के समय से धर्म-प्रचार की एक विशेषता लक्षणीय है - तब से सर्वसाधारण के लिए धर्म का द्वार खुल गया।\endnote{ ५/१६०;}


\section*{श्री चैतन्य महाप्रभु}

\addsectiontoTOC{श्री चैतन्य महाप्रभु}

उत्तर भारत के महान् सन्त - चैतन्यदेव गोपियों के प्रेमोन्मत्त भाव के प्रतिनिधि थे। चैतन्यदेव स्वयं एक ब्राह्मण थे, उस समय के एक प्रसिद्ध नैयायिक वंश में उनका जन्म हुआ था। वे न्याय के अध्यापक थे, तर्क द्वारा सबको परास्त करते थे - यही उन्होंने बचपन से जीवन का उच्चतम आदर्श समझ रखा था। किसी महापुरुष की कृपा से इनका सम्पूर्ण जीवन बदल गया। तब उन्होंने वाद-विवाद, तर्क, न्याय का अध्यापन - सब कुछ छोड़ दिया। संसार में भक्ति के जितने बड़े-बड़े आचार्य हुए हैं, प्रेमोन्मत्त चैतन्य उनमें से एक श्रेष्ठ आचार्य हैं। उनकी भक्ति-तरंग सारे बंगाल में फैल गयी, जिससे सबके हृदय को शान्ति मिली। उनके प्रेम की सीमा न थी। साधु, असाधु, हिन्दू, मुसलमान, पवित्र, अपवित्र, वेश्या, पतित - सभी उनके प्रेम के भागी थे, वे सब पर दया रखते थे। यद्यपि काल के प्रभाव से सभी अवनति को प्राप्त होते हैं और उनका चलाया हुआ सम्प्रदाय घोर अवनति की दशा को पहुँच गया है। तो भी आज तक वह ऐसे लोगों का आश्रय-स्थल है, जो निर्धन, दुर्बल, जातिच्युत, पतित हैं और जिनका किसी भी समाज में स्थान नहीं है।\endnote{ ५/१६०-६१;} 

चैतन्य महाप्रभु महान् त्यागी थे। उनका कामिनी और इन्द्रिय-भोग से कोई नाता नहीं था। परन्तु बाद में, उनके अनुयाइयों ने स्त्रियों को भी अपने सम्प्रदाय में सम्मिलित कर लिया, चैतन्यदेव के नाम पर अन्धाधुन्ध उनके सम्पर्क में रहे और इस प्रकार उनके महान् आदर्शों को मिट्टी में मिला दिया। प्रेम का जो आदर्श चैतन्य महाप्रभु ने अपने जीवन में प्रकट किया था, उसमें तो अहंभाव और वासना का लेश तक नहीं था; वह काम-विहीन प्रेम सर्वसाधारण के लिए सुलभ नहीं हो सकता था। परन्तु उनके परवर्ती वैष्णव गुरुओं ने, चैतन्य महाप्रभु के जीवन के त्याग तथा निःस्पृहता के आदर्शों पर बल न देकर, सर्व-साधारण के बीच उनके प्रेम के आदर्श का ही प्रचार किया। परिणाम यह हुआ कि सर्व-साधारण उस स्वर्गीय प्रेम के तत्त्व को नहीं समझ सका और उस प्रेम को निकृष्टतम प्रकार के स्त्री-पुरुष-प्रेम का रूप प्राप्त हुआ।... जब तक हृदय में वासना के लिए तिल मात्र भी स्थान है, तब वह प्रेम सम्भव नहीं। केवल महान् त्यागी, निःस्पृह तथा संन्यस्त व्यक्ति, जो मानवों में अतिमानव हैं - केवल उन्हें ही उस स्वर्गीय प्रेम का अधिकार है। यदि वह प्रेम का उच्चतम आदर्श सर्व-साधारण में प्रचलित कर दिया जाए, तो वह प्रकारान्तर से मनुष्य के हृदय में प्रबल सांसारिक प्रेम को ही उद्दीप्त करेगा - क्योंकि ईश्वर का प्रिया-भाव से ध्यान करते-करते साधारण मनुष्य अधिकांश समय अपनी ही प्रिया के ध्यान में खोया रहेगा।\endnote{ ८/२७९-८०;}


\section*{श्रीरामकृष्ण}

\addsectiontoTOC{श्रीरामकृष्ण}

एक (शंकराचार्य) का था अद्भुत मस्तिष्क और दूसरे (चैतन्य) का था विशाल हृदय। अब एक ऐसे अद्भुत पुरुष के जन्म लेने का समय आ गया था, जिसमें ऐसा ही हृदय और मस्तिष्क - दोनों एक साथ विराजमान हों, जो एक साथ ही शंकर के प्रतिभा-सम्पन्न मस्तिष्क एवं चैतन्य के अद्भुत, विशाल, अनन्त हृदय का अधिकारी हो, जो देखे कि सब सम्प्रदाय एक ही आत्मा, एक ही ईश्वर की शक्ति से परिचालित हो रहे हैं और प्रत्येक प्राणी में वही ईश्वर विद्यमान है, जिसका हृदय भारत के तथा भारत के बाहर के दीन, दुर्बल, पतित सबके लिए द्रवित हो; लेकिन साथ ही जिसकी विशाल बुद्धि ऐसे महान् तत्त्वों की परिकल्पना करे, जिनसे भारत में तथा भारत के बाहर सब विरोधी सम्प्रदायों में समन्वय साधित हो और इस अद्भुत समन्वय द्वारा वह एक हृदय और मस्तिष्क के सार्वभौम धर्म को प्रकट करे। 

एक ऐसे ही पुरुष ने जन्म ग्रहण किया और मैंने वर्षों तक उनके चरणों तले बैठकर शिक्षा-लाभ का सौभाग्य प्राप्त किया। समय हो चुका था। एक व्यक्ति के आविर्भाव की आवश्यकता थी और वह उत्पन्न हुआ। सबसे अधिक आश्चर्य की बात यह थी कि उसका समग्र जीवन एक ऐसे नगर (कोलकाता) के पास व्यतीत हुआ, जो पाश्चात्य भावों से उन्मत्त हो रहा था, जो भारत के सब शहरों की अपेक्षा विदेशी भावों से अधिक भरा हुआ था। वह वहीं निवास करता था। परन्तु यह महा-प्रतिभाशाली व्यक्ति हर प्रकार के किताबी ज्ञान से रहित था और अपना नाम तक लिखना नहीं जानता था।\footnote{ पाश्चात्य दृष्टिकोण से श्रीरामकृष्ण प्रायः निरक्षर थे, परन्तु बाद में शोध से पता चला कि वे थोड़ा-बहुत लिखना-पढ़ना भी जानते~थे।} परन्तु हमारे विश्वविद्यालय के बड़े-बड़े विद्वान् स्नातकों ने उनको एक महान् बौद्धिक प्रतिभा के रूप में स्वीकार किया। वे अद्भुत महापुरुष थे - श्रीरामकृष्ण परमहंस।\endnote{ ५/१६१;} 

मेरे गुरुदेव कोलकाता नगर के समीप रहने आए। यह नगर भारत की राजधानी\footnote{ उन दिनों कोलकाता ही भारत का राजधानी थी। बाद में अंग्रेजों ने ही राजधानी को दिल्ली स्थानान्तरित कर दिया।} तथा हमारे देश का सबसे महत्त्वपूर्ण विश्वविद्यालय-नगर है, जहाँ से प्रतिवर्ष सैकड़ों नास्तिक तथा भौतिकवादी बाहर निकलते हैं - तो भी विश्वविद्यालय के इन्हीं सन्देहवादी तथा अज्ञेयवादी व्यक्तियों में से अनेक लोग उनके पास आते और उनकी बातें सुनते थे। मैंने भी उस व्यक्ति के बारे में सुना और उनके उपदेश सुनने उनके पास गया। मेरे गुरुदेव एक अत्यन्त साधारण मनुष्य जैसे ही प्रतीत होते थे। उनमें कोई विशेषता नहीं दिखती थी। वे बहुत साधारण भाषा का प्रयोग करते थे। उस समय मेरे मन में यह प्रश्न उठा - “क्या यह व्यक्ति वास्तव में महान् ज्ञानी है?’ मैं धीरे से खिसककर उनके पास गया और उनके सामने वही प्रश्न रख दिया, जो मैं अन्य सभी से पूछा करता था - “महाराज, क्या आप ईश्वर में विश्वास करते हैं?” उन्होंने उत्तर दिया - “हाँ।” मैंने कहा - “क्या आप उन्हें सिद्ध करके दिखा सकते हैं?” उन्होंने उत्तर दिया - ‘हाँ।” मैंने कहा - “कैसे?” उन्होंने उत्तर दिया - “यहाँ जैसे मैं तुम्हें देख रहा हूँ, वैसे ही मैं ईश्वर को देखता हूँ - बल्कि उससे भी अधिक स्पष्ट रूप से।” इस उत्तर का मेरे मन पर तत्काल असर हुआ, क्योंकि जीवन में मुझे पहली बार ऐसा व्यक्ति मिला था, जिसने तत्काल कह दिया कि उसने ईश्वर को देखा है, जिसने यह भी बताया कि धर्म एक वास्तविक सत्य है और जिस प्रकार हम अपनी इन्द्रियों के द्वारा विश्व का अनुभव करते हैं, उससे कहीं अधिक स्पष्ट रूप से उसका अनुभव किया जा सकता है। मैं दिन-पर-दिन उनके पास जाने लगा और मैंने यह प्रत्यक्ष अनुभव किया कि धर्म भी दूसरे को दिया जा सकता है, केवल एक ही स्पर्श तथा एक ही दृष्टि में सारा जीवन बदला जा सकता है।... इन महापुरुष का प्रत्यक्ष दर्शन करने के बाद मेरी सारी नास्तिकता दूर हो गयी।\endnote{ ७/२५९-६०;} 

अपने गुरुदेव के सान्निध्य में रहकर मैंने जान लिया कि इस जीवन में ही मनुष्य पूर्णावस्था को पहुँच सकता है। उनके मुख से कभी किसी के लिए अभिशाप या निन्दा के वचन नहीं निकले। उनकी आँखें कोई बुरी चीज देख ही नहीं सकती थीं और न उनके मन में कभी बुरे विचार ही प्रवेश कर सकते थे। उन्हें जो कुछ दिखा, वह अच्छा ही दिखा। यह महान् पवित्रता तथा महान् त्याग ही आध्यात्मिक जीवन का रहस्य हैं।\endnote{ ७/२६४;} 

त्याग के बिना महान् आध्यात्मिकता कैसे प्राप्त हो सकती है? त्याग ही सभी प्रकार के धर्मभावों की पृष्ठभूमि है और तुम यह सदैव देखोगे कि जैसे-जैसे त्याग का भाव क्षीण होता जाता है, वैसे-वैसे धर्म के क्षेत्र में इन्द्रियों का प्रभाव बढ़ता जाता है और उसी परिमाण में आध्यात्मिकता का ह्रास होता जाता है। 

मेरे गुरुदेव त्याग की साकार मूर्ति थे। हमारे देश में संन्यासी होनेवाले व्यक्ति के लिए यह आवश्यक होता है कि वह सारी सांसारिक सम्पत्ति तथा सामाजिक स्थिति का परित्याग कर दे और मेरे गुरुदेव ने इस सिद्धान्त का अक्षरशः पालन किया। ऐसे बहुतसे लोग थे, जिनसे मेरे गुरुदेव यदि कोई भेंट ग्रहण कर लेते, तो वे अपने को धन्य मानते; और यदि वे स्वीकार करते, तो वे लोग उन्हें हजारों रुपये देने को प्रस्तुत थे, परन्तु मेरे गुरुदेव ऐसे प्रलोभनों दूर भागते थे। काम-कांचन पर पूर्ण विजय के वे जीवन्त तथा ज्वलन्त उदाहरण थे। इन दोनों भावनाओं का उनमें पूर्ण अभाव था और इस शताब्दी के लिए ऐसे ही व्यक्तियों की नितान्त आवश्यकता है।\endnote{ ७/२६४;} 

पश्चिम में हम प्रायः नारीपूजा की बात सुनते हैं, परन्तु यह पूजा प्रायः उसके तारुण्य तथा लावण्य के कारण होती है। परन्तु मेरे गुरुदेव के स्त्री-पूजन का भाव यह था कि प्रत्येक स्त्री का मुखारविन्द उन आनन्दमयी जगदम्बा का ही मुखारविन्द है, उसके अतिरिक्त अन्य कुछ नहीं। मैंने प्रत्यक्ष देखा है कि मेरे गुरुदेव उन स्त्रियों के चरणों में गिर पड़ते, जिनको समाज स्पर्श तक नहीं करता था और उन स्त्रियों से भी वे रोते हुए यही कहते, “हे जगदम्बे, एक रूप में तुम सड़कों पर घूमती हो और दूसरे रूप में तुम विश्वव्यापिनी हो। तुम्हें प्रणाम करता हूँ, माँ, मैं तुम्हें प्रणाम करता हूँ।” सोचकर देखो, उनका जीवन कैसा धन्य है, जिनका काम-भाव पूर्णतः नष्ट हो गया है, जो प्रत्येक नारी का भक्ति-भाव से दर्शन कर रहे हैं और जिनके लिए प्रत्येक नारी के मुख ने एक ऐसा रूप धारण कर लिया है, जिसमें साक्षात् आनन्दमयी भगवती जगद्धात्री का मुख ही प्रतिबिम्बित हो रहा है!... यदि आध्यात्मिकता अर्जित करनी हो, तो ऐसी पवित्रता अति आवश्यक है।\endnote{ ७/२५६-५७;} 

मेरे गुरुदेव के जीवन का दूसरा तत्त्व दूसरों के प्रति अगाध प्रेम था। उनके जीवन का प्रारम्भिक भाग धर्म के उपार्जन में लगा रहा तथा अन्तिम भाग उसके वितरण में।... झुण्ड-के-झुण्ड लोग मेरे गुरुदेव के श्रीवचन सुनने आते और वे चौबीस घण्टों में से बीस घण्टे तक उनके साथ बातें करते रहते; और वह भी कोई एक दिन नहीं, बल्कि महीनों तक यही क्रम जारी रहा। इसका फल यह हुआ कि अत्यन्त परिश्रम के कारण अन्त में उनका शरीर टूट गया। उनका मानव-जाति के प्रति इतना अगाध प्रेम था कि उनके पास कृपा पाने आनेवाले हजारों में से अति साधारण व्यक्ति भी उस कृपा-लाभ से वंचित नहीं रहता था। इसके फलस्वरूप धीरे-धीरे उनके गले में एक बड़ा भयंकर रोग हो गया। परन्तु बहुत आग्रह करने के बावजूद वे इस परिश्रम से विरत नहीं होते थे। जैसे ही वे सुनते कि उनसे मिलने के इच्छुक लोग आए हुए हैं, तो वे उन्हें अन्दर बुलाये बिना नहीं मानते और उनके सब प्रश्नों का उत्तर देते। किसी के समझाने का प्रयास करने पर उन्होंने कहा था - “मैं परवाह नहीं करता। यदि एक मनुष्य की भी सहायता हो सके, तो मैं ऐसे हजारों शरीर छोड़ने को तैयार हूँ। एक व्यक्ति की सहायता कर पाना भी गौरव की बात है।\endnote{ ७/२६५;} 

यदि मैंने जीवन भर में एक भी सत्य वाक्य कहा है, तो वह उन्हीं का - केवल उन्हीं का वाक्य है; परन्तु यदि मैंने ऐसे वाक्य कहे हों, जो असत्य, भ्रामक या मानवजाति के लिए हितकर न हों, तो वे सब मेरे ही वाक्य हैं और उनका पूरा उत्तरदायी मेरे ऊपर है।\endnote{ ५/१६२;} 

किसी राष्ट्र की उन्नति के लिए उसके पास एक आदर्श होना आवश्यक है। वस्तुतः वह आदर्श है - निर्गुण ब्रह्म। लेकिन चूंकि तुम लोग किसी निराकार आदर्श से प्रेरणा नहीं प्राप्त कर सकते, इसलिए तुम्हें साकार आदर्श चाहिए। वह तुम्हें श्रीरामकृष्ण के व्यक्तित्व के रूप में मिला है। अन्य व्यक्ति अब हमारे आदर्श क्यों नहीं बन सकते, इसका कारण यह है कि उनके दिन बीत चुके हैं और वेदान्त को सबके लिए सुलभ बनाने के लिए निश्चय ही ऐसा व्यक्ति चाहिए, जिसकी सहानुभूति वर्तमान पीढ़ी से हो। श्रीरामकृष्ण से इसकी परिपूर्ति होती है। अतः अब तुम्हें चाहिए कि उनको सबके समक्ष रखो। चाहे कोई उन्हें साधु माने या अवतार, इससे कोई फरक नहीं पड़ता।\endnote{ ७/२७१;} 

ईश्वर यद्यपि सर्वत्र है, तो भी हम उसे केवल मनुष्य चरित्र के माध्यम से ही जान सकते है। श्रीरामकृष्ण के जैसा पूर्ण-चरित्र पहले कभी कोई नहीं हुआ, इसलिए हमें उन्हीं को केन्द्र बनाकर संगठित होना पड़ेगा। हाँ, हर व्यक्ति उन्हें अपनी-अपनी भावना के अनुसार स्वीकार करे - जो जैसा चाहे, उन्हें माने - ईश्वर, परित्राता, आचार्य, आदर्श मनुष्य अथवा एक महापुरुष।\endnote{ २/३२६;} 

संकीर्ण समाज में आध्यात्मिकता प्रबल और गम्भीर होती है - जैसे सँकरी नदी में प्रवाह अधिक होता है। उदार समाज में एक ओर दृष्टिकोण का विस्तार होने के बावजूद उसी अनुपात में गम्भीरता तथा तीव्रता की कमी दिखाई देती है। परन्तु श्रीरामकृष्ण का जीवन इतिहास के इस नियम का अपवाद है। यह एक अभूतपूर्व घटना है कि श्रीरामकृष्ण का जीवन एक ही आधार में समुद्र से भी गम्भीर और आकाश से भी विस्तृत भावों का सम्मिलन है। 

हमें श्रीरामकृष्ण की अनुभूतियों के आलोक में वेदों की व्याख्या करनी होगी।... प्राचीन काल में गीता के प्रवक्ता भगवान श्रीकृष्ण ने (वेदों के) इन आपात् विरोधी उक्तियों के बीच आंशिक रूप से सामंजस्य स्थापित किया था। काल के अन्तराल में काफी विराट् रूप धारण कर चुके, उसी विवाद को पूरी तौर से हल करने के निमित्त वे स्वयं श्रीरामकृष्ण के रूप में आए हैं। उन्होंने पहले अपने जीवन में रूपायित करके और तदुपरान्त अपने उपदेशों द्वारा यह बताया कि आपात् दृष्टि से परस्पर-विरोधी प्रतीत होनेवाली शास्त्रों की ये उक्तियाँ विभिन्न स्तरों के साधकों के लिए उपयोगी हैं और उनके विकास का क्रम प्रदर्शित करती हैं। अतः श्रीरामकृष्ण की उक्तियों के आलोक में ही वेद-वेदान्त के यथार्थ तात्पर्य को समझा जा सकता है।... 

श्रीरामकृष्ण ने हमें काम-कांचन-त्याग के समान सावधानी के साथ कोई चीज छोड़ने को कहा है, तो वह है - ईश्वर के असीम भावों को अपने दायरे में सीमाबद्ध करना।... 

ऐसा अभूतपूर्व व्यक्तित्व, ज्ञान-योग-भक्ति तथा कर्म का ऐसा अद्भुत सामंजस्य - मानव-जाति ने इसके पहले कभी नहीं देखा। श्रीरामकृष्ण का जीवन यह सिद्ध करता है कि महानतम विस्तार, सर्वोच्च उदारता तथा परम तीव्रता - एक ही व्यक्ति में एक साथ निवास कर सकती हैं और इसी प्रकार के समाज का भी गठन हो सकता है, क्योंकि समाज भी तो व्यक्तियों का ही एक समूह मात्र है। 

वही व्यक्ति श्रीरामकृष्ण का यथार्थ शिष्य तथा अनुयायी है, जिसका चरित्र उन्हीं के समान आदर्श तथा सर्वांग-सम्पूर्ण है। ऐसे पूर्ण चरित्र का निर्माण ही इस युग का आदर्श है और प्रत्येक व्यक्ति को केवल इसी के लिए प्रयास करना चाहिए।\endnote{ ण. एं. ७/४११-१२;} 

इसलिए मैं इस निष्कर्ष पर पहुँचा हूँ कि श्रीरामकृष्ण की बराबरी का दूसरा कोई नहीं। वैसी अपूर्व सिद्धि, सबके लिए वैसी अहैतुकी दया, जन्म-मरण से जकड़े हुए जीव के लिए वैसी प्रगाढ़ सहानुभूति - इस संसार में अन्यत्र कहीं भी नहीं है।\endnote{ १/३६५;} 

पहले रामकृष्ण परमहंस का अध्ययन किए बिना वेद, वेदान्त, भागवत तथा अन्य पुराणों का तात्पर्य समझना असम्भव है। उनका जीवन, भारत की सम्पूर्ण धार्मिक विचारराशि को आलोकित करनेवाला एक अनन्त शक्तिशाली सर्चलाइट है। वे वेदों तथा उनके लक्ष्य के जीवित भाष्य थे। अपने एक जीवन काल के दौरान वे भारत के राष्ट्रीय धार्मिक अस्तित्व का एक पूरा कल्प ही बिता गये।\endnote{ २/३६०;} 

रामकृष्ण परमहंस सबसे आधुनिक है और सबसे पूर्ण हैं - ज्ञान, प्रेम, त्याग, उदारता और लोकहित-कामना के वे मूर्तिमान स्वरूप हैं। अतः कौन है, जिसके साथ उनकी तुलना की जा सके? जो उन्हें समझ नहीं सकता, उसका जीवन व्यर्थ है। मेरा परम सौभाग्य है कि मैं उनका जन्म-जन्मान्तर का दास हूँ। उनकी एक उक्ति भी मेरे लिए वेद-वेदान्त से अधिक मूल्यवान है। \textbf{तस्य दास-दास-दासोऽहम् - } मैं उनके दासों-केदासों का दास हूँ! परन्तु एकांगी कट्टरता से उनकी भावधारा में अवरोध आता है, उसी से मैं चिढ़ता हूँ। भले ही उनका नाम डूब जाए, परन्तु उनकी शिक्षा फलप्रसू होनी चाहिए। क्या वे नाम के दास थे?\endnote{ २/३६०;} 

जिस दिन श्रीरामकृष्ण देव ने जन्म लिया है, उसी दिन से आधुनिक भारत तथा सत्युग का आविर्भाव हुआ है।\endnote{ ४/३०९;} 

श्रीरामकृष्ण-अवतार में ज्ञान, भक्ति तथा प्रेम - तीनों ही विद्यमान हैं। उनमें अनन्त ज्ञान, अनन्त प्रेम, अनन्त कर्म तथा प्राणियों के लिए अनन्त दया है। अभी तक तुम्हें इसका अनुभव नहीं हुआ है।... युग-युग से समग्र हिन्दू जाति के लिए जो चिन्तन का विषय रहा, उसकी उन्होंने अपने एक ही जीवन में उपलब्धि की। उनका जीवन सभी देशों के धर्म-शास्त्रों का सजीव भाष्य-स्वरूप है।\endnote{ ४/३१०;} 

भारत चिर काल से दुःख भोगता आया है; सनातन धर्म दीर्घ काल से अत्याचार सहता रहा है। परन्तु ईश्वर दयामय हैं। अपनी सन्तानों के परित्राण हेतु वे एक बार फिर आये हैं; पतित भारत को एक बार फिर उठने का सुयोग मिला है। श्रीरामकृष्ण के चरणों में बैठने से ही भारत का उत्थान हो सकता है। उनकी जीवनी तथा शिक्षाओं को चारों ओर फैलाना होगा, उन्हें हिन्दू समाज के रोम-रोम में भरना होगा। यह कौन करेगा? श्रीरामकृष्ण की पताका हाथ में लेकर संसार के उद्धार हेतु कौन अभियान करेगा? कौन है, जो नाम-यश, भोग-ऐश्वर्य और यहाँ तक कि इहलोक तथा परलोक की सारी आशाओं का बलिदान करके अवनति की इस बाढ़ को रोकने हेतु अग्रसर होगा?... वह धन्य है, जिसे प्रभु ने चुन लिया है।\endnote{ ३/३३७-३८;} 

जगत् के कल्याण हेतु ही परमहंस श्रीरामकृष्ण का आविर्भाव हुआ था। तुम अपनी-अपनी भावना के अनुसार उनको मनुष्य, ईश्वर, अवतार - जो कुछ भी कहना चाहो - कह सकते हो। जो कोई उनको प्रणाम करेगा, वह तत्काल ही स्वर्ण बन जाएगा। वत्स, इस सन्देश को लेकर तुम घर-घर जाओ - देखोगे कि तुम्हारी सारी अशान्ति दूर हो गयी है।\endnote{ ३/३०१;} 

हमें समाज में - संसार में, चेतना का संचार करना होगा। बैठे-बैठे गप्पें लड़ाने और घण्टा हिलाने से काम नहीं चलेगा।... आध्यात्मिकता की बड़ी भारी तरंग आ रही है - साधारण व्यक्ति महान् बन जाएँगे, अनपढ़ व्यक्ति उनकी कृपा से बड़े-बड़े पण्डितों के आचार्य बन जाएँगे - \textbf{उत्तिष्ठत जाग्रत प्राप्य वरान् निबोधत } - ‘उठो, जागो और जब तक लक्ष्य तक न पहुँच जाओ, रुको नहीं।’ निरन्तर विस्तार करना ही जीवन है और संकुचन मृत्यु। जो व्यक्ति अपना ही स्वार्थ देखता है, आराम-तलब और आलसी है, उसके लिए नरक में भी जगह नहीं है। जिसमें जीवों के लिए इतनी करुणा है कि वह खुद उनके लिए नरक में भी जाने को तैयार रहता है, उनके लिए कुछ भी उठा नहीं रखता - वही श्रीरामकृष्ण का पुत्र है, \textbf{इतरे कृपणाः } - दूसरे तो हीन बुद्धि वाले हैं। इस आध्यात्मिक जागृति के सन्धि-काल में जो कोई भी कमर कसकर खड़ा हो जाएगा; गाँव-गाँव, तथा घर-घर में उनका संवाद देता फिरेगा, वही मेरा भाई है - वही ‘उनका’ पुत्र है। यही कसौटी है - जो रामकृष्ण के पुत्र हैं, वे अपना भला नहीं चाहते, वे प्राण चले जाने पर भी दूसरों का भला चाहते हैं - \textbf{प्राणात्ययेऽपि परकल्याण-चिकीर्षवः। }\endnote{ ३/३५५-५६;} 

मैं जो कुछ हुआ हूँ, सारी दुनिया भविष्य में जो कुछ बनेगी, उन सबके मूल में हैं मेरे गुरुदेव श्रीरामकृष्ण।\endnote{ १०/२१८} 

\delimiter

\addtoendnotes{\protect\end{multicols}}

\addtocontents{toc}{\protect\par\egroup}

