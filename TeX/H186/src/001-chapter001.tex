
\chapter{स्वामी विवेकानन्द का सन्देश }

\begin{flushright}
\textit{\textbf{- स्वामी लोकेश्वरानन्द}}
\end{flushright}

स्वामी विवेकानन्द के एक पाश्चात्य अनुरागी ने उनका वर्णन करते हुए कहा था कि वे ‘आयु में कम, परन्तु ज्ञान में असीम थे।’ यदि आप इस भावुकतापूर्ण उक्ति को शब्दशः स्वीकार करते हैं, तो यह स्वामी विवेकानन्द के दर्शन की आज के युग में प्रासंगिकता को प्रस्तुत करती है। उनकी आयु अल्प थी, ४० वर्षों से भी कम; पिछली शताब्दी के प्रारम्भ में ही उनका देहान्त हो गया। तब से दो विश्वयुद्ध हो चुके हैं और विज्ञान तथा प्रौद्योगिकी के क्षेत्र में विस्मयजनक विकास हो चुका है। इसके फलस्वरूप पूरी दुनिया बदल चुकी है और साथ ही मनुष्य के दृष्टिकोण तथा जीवन-शैली में भी परिवर्तन आ चुका है। भारत, जो (भगिनी निवेदिता के शब्दों में) ‘उनकी उपासना की महारानी थीं’, अब एक स्वाधीन राष्ट्र हो चुका है (जैसी कि उन्होंने भविष्यवाणी की थी - अगले ५० वर्षों के भीतर यह देश अत्यन्त अकल्पनीय परिस्थितियों में स्वाधीन हो जाएगा) और सारे विश्व के समान ही भारत भी ऐसी समस्याओं का सामना कर रहा है, जो उनके समय में अज्ञात थीं। तो क्या इसके बावजूद यह कहा जा सकता है कि स्वामी विवेकानन्द आज भी प्रासंगिक हैं?

\section*{मनुष्य निर्माण - उनका जीवनोद्देश्य}

स्वामी विवेकानन्द की प्रासंगिकता इस बात पर निर्भर नहीं करती कि हमारे सम्मुख आज कौन-सी समस्याएँ खड़ी हैं, बल्कि इस बात पर निर्भर करती है कि हमें किस भाव के साथ उनका सामना करना है। उनका जोर स्वयं मनुष्य पर ही था, क्योंकि सही प्रकार के मनुष्य उपलब्ध हों, तो कोई भी समस्या असाध्य नहीं है। वे कहा करते, “मनुष्य निर्माण ही मेरा जीवनोद्देश्य है।” सचमुच ही किसी भी राष्ट्र का भविष्य इस बात पर निर्भर करता है कि उसकी जनता कितनी अच्छी, बुद्धिमान तथा सुयोग्य है। कोई भी राष्ट्र एक या दो महान् व्यक्तियों को पैदा कर सकता है, परन्तु यह इस बात की गारंटी नहीं देता कि वह देश महान् होगा। वह उस देश की क्षमता को सिद्ध कर सकता है, परन्तु जब तक किसी देश के औसत नर-नारी का जीवन-स्तर उच्च नहीं हो जाता, तब तक उस देश को महान् नहीं कहा जा सकता। स्वामीजी कहा करते थे कि एक बुद्ध या एक ईसा ने किसी भी देश के भाग्य का निर्धारण नहीं किया, बल्कि आम जनता ने यह निर्धारित किया कि उस देश का भविष्य क्या होगा। उनके मतानुसार राष्ट्र की सच्ची शक्ति उसकी आम जनता में निहित है। उन्होंने जनता की उपेक्षा को राष्ट्र का एक महान् पाप बताया। उन्होंने कहा कि यही उसकी अधिकांश बुराइयों का कारण है। आम जनता (जिसे उन्होंने ‘निद्रामग्न जलदैत्य’ कहा था) में अनन्त शक्ति निहित है, परन्तु राष्ट्रीय समस्याओं के समाधान में उन्हें कभी अपनी भूमिका निभाने का अवसर नहीं दिया गया। 

विश्वविद्यालय की उपाधियों के आधार पर विशेषाधिकार का दावा करनेवाले, आम लोगों की समस्याओं से पूर्णतः अनभिज्ञ, कुछ मुट्ठी भर तथाकथित बुद्धिजीवी ही राष्ट्र के भाग्य का फैसला करते रहे, जबकि जनता मूक द्रष्टा बनी रही! वस्तुतः वे राष्ट्र की सत्ता में अपनी हिस्सेदारी के विषय में कहीं अधिक चिन्तित थे और उनकी शिकायतों का कारण केवल यह बोध था कि उन्हें अपनी योग्यता के अनुसार अधिकारों की प्राप्ति नहीं हो रही है। आम जनता की क्या हालत है, कैसे वे ऊँची जातियों द्वारा सताये जा रहे हैं और उनकी निरक्षरता तथा निर्धनता आदि बातों के विषय में उन्हें जरा भी चिन्ता नहीं थी। इसके बावजूद वे सम्पूर्ण राष्ट्र के हित में बोलने का दावा करते थे। वे ऐसे रीढ़विहीन लोग थे, जिनकी राष्ट्र या जनता के प्रति कोई निष्ठा न थी, जिनमें इतना भी साहस नहीं था कि स्वयं का या राष्ट्र का अपमान करनेवालों का सामना कर पाते। स्वामी विवेकानन्द ऐसे लोगों पर चिढ़कर उन्हें ऐसे ‘चलते-फिरते शव’ कहा करते, जो मरने के बाद भी अपने विगत गौरव के चिह्न के रूप में लोगों को प्रभावित करने की चेष्टा में लगे रहते हैं।


\section*{आम जनता की उन्नति}

भारत की स्वाधीनता के ६० वर्षों बाद, क्या अब भी इस परिदृश्य में कोई बदलाव आया है? अब भी नगरीय मध्य-वर्ग का ही राष्ट्र पर दबदबा है। देश में जो कुछ हो रहा है, उसमें आम जनता को शायद ही कुछ कहने का अधिकार है। यद्यपि राष्ट्र-निर्माण की प्रक्रिया में उनके सम्मिलित होने की आवश्यकता को स्वीकार किया जाता है, तथापि न तो कोई योजना बनाने में और न ही उनके क्रियान्वन में उन्हें कोई भूमिका होती है। अतः राष्ट्र की प्रगति में जो विलम्ब हो रहा है, इसमें आश्चर्य की कोई बात नहीं। अपनी नवीन भारत की परिकल्पना में स्वामी विवेकानन्द ने शुद्रों (अर्थात् श्रमजीवियों) को सर्वोपरि महत्त्व दिया जाए। उन्होंने कहा कि वे एक समाजवादी हैं, ‘इसलिए नहीं कि समाजवाद एक आदर्श व्यवस्था है, बल्कि इसलिए कि खाली पेट से थोड़ा कुछ भी बेहतर है।’ इतिहास की अपनी समझ से उन्हें विश्वास हो गया था कि देर-सबेर क्रान्ति का आना अवश्यम्भावी है। उन्होंने दो देशों का नाम लिया, जहाँ यह सबसे पहले आनेवाली थी - रूस और चीन। यह घटना १८९० वाले दशक में हुई थी। कोई नहीं जानता कि उन्हें कैसे इस बात का अनुमान हो गया था। यह भी कोई नहीं जानता कि उक्त दो देशों में जैसी क्रान्ति हुई, क्या उन्होंने उसका अनुमोदन किया होता! 

{जहाँ तक भारत का सवाल है, यह स्पष्ट है कि उन्होंने इसके लिए क्रान्ति के स्थान पर विकास को बेहतर माना। क्या उन्हें मालूम था कि क्रान्ति के लिए देश को क्या कीमत चुकानी पड़ती है? यह स्पष्ट है कि वे भारत में हिंसक परिवर्तन नहीं चाहते थे, क्योंकि उन्होंने कहा वे ऊपरवालों को गिराना नहीं, अपितु नीचेवालों को उठाना चाहते हैं। अर्थात् वे चाहते थे कि श्रमजीवियों को इतने\parfillskip=0pt}\newpage\noindent अधिक अवसर प्रदान किए जाएँ कि वे भी बुद्धिजीवियों के स्तर तक पहुँच जाएँ। श्रमजीवियों को ऊपर उठाने के लिए बुद्धिजीवियों को नीचे उतारने की आवश्यकता नहीं है।


\section*{शिक्षा द्वारा चरित्र-निर्माण}

श्रमजीवियों की काफी काल से उपेक्षा हुई और उन्हें शिक्षित होने का अवसर नहीं मिला। अब उन पर विशेष ध्यान दिया जाना चाहिए, ताकि वे यथाशीघ्र अपनी प्रारम्भिक कठिनाइयों पर विजय पा सकें। वे लोग शिक्षा पाने के लिए आएँ, इसकी जगह स्वामीजी चाहते थे कि शिक्षा को ही उनके पास पहुँचाया जाए। सही आकलन करते हुए उन्होंने कहा था कि शिक्षा को निःशुल्क बनाने मात्र से काम न होगा। इसके अतिरिक्त भी प्रोत्साहन देना होगा - शिक्षा को उनके द्वार तक पहुँचाना होगा। और यह शिक्षा न केवल मुफ्त होगी, बल्कि एक प्रबुद्ध कुल के बच्चे के लिए एक शिक्षक की तुलना में एक श्रमजीवी के बच्चे के लिए पाँच शिक्षकों की व्यवस्था करनी होगी। जब मैसूर के महाराजा अपने राज्य में मुफ्त शिक्षा की योजना बना रहे थे, तब स्वामीजी ने उन्हें यही सलाह दी थी। स्वामीजी ने शिक्षा पर काफी बल दिया। वे भारत की सारी बीमारियों के लिए इसी को सर्वरोगहर औषधि मानते थे। 

परन्तु शिक्षा किस प्रकार की होगी? निश्चित रूप से यह केवल पुस्तकें पढ़ना, परीक्षाएँ पास करना तथा उपाधियाँ प्राप्त करने तक ही सीमित नहीं होगा। उनकी दृष्टि में शिक्षा का अर्थ केवल जानकारियाँ देना ही नहीं, बल्कि कुछ और भी सार्थक वस्तु था। शिक्षा का अर्थ है विचारों को आत्मसात् करना; इसे मनुष्य का निर्माण करनेवाली, जीवन प्रदान करनेवाली तथा चरित्र का निर्माण करनेवाली होनी चाहिए। इसमें कार्यकुशलता भी शामिल होनी चाहिए, ताकि यह उत्पादनशील भी हो। उन्हें इस बात का खेद था कि शिक्षा की वर्तमान व्यवस्था व्यक्ति को अपने पाँवों पर खड़ा होने के योग्य नहीं बना पाती और न ही उसमें स्वाभिमान या आत्मविश्वास जगा पाती है। वे भारत के लिए ऐसी शिक्षा चाहते थे, जिसमें उसके अपने आदर्शवाद के साथ पाश्चात्य कुशलता का सामंजस्य हो। भारत ने अनेक उच्च विचारों को जन्म दिया है, पर उन्हें शायद ही कभी कार्य रूप में परिणत किया गया है। वे इसी कमी को भारत की सारी सामाजिक बुराइयों का कारण मानते थे। इसके उदाहरण के रूप में उन्होंने जातिप्रथा की ओर इंगित किया। श्रम-विभाजन के सिद्धान्त पर आधारित यह एक आदर्श संस्था थी। प्रत्येक व्यक्ति को उसकी अधिकतम क्षमता के अनुसार विकास का अवसर देना ही इसका उद्देश्य था, परन्तु जब इसमें ऊँच-नीच का भाव आ गया और यह जन्म पर आधारित हो गया, तब यह पूरी तौर से निरर्थक तथा हानिकारक हो गया। स्वामी विवेकानन्द ने जाति-व्यवस्था की जितनी कठोर भर्त्सना की है, उतनी शायद ही किसी ने की होगी। परन्तु इसका समाधान क्या है? स्वामीजी पुनः कहते हैं - शिक्षा। अच्छी शिक्षा मिलने पर अभी जो लोग पिछड़े हुए हैं, वे स्वयं ही उन्नत हो जाएँगे।


\section*{आध्यात्मिक शक्ति के द्वारा उद्धार}

स्वामीजी को आशा थी कि पिछड़े लोगों की उन्नति के लिए समाज जो भी कदम उठाएगा, उच्च वर्गों के लोग उसका स्वागत करेंगे। वे यह भी उम्मीद करते थे कि वे लोग स्वयं ही इस दिशा में आगे बढ़ेंगे और इस विकास को गति देने के लिए आवश्यक बलिदान भी करेंगे। यह पूरी तौर से ‘त्याग और सेवा’ रूपी भारत के राष्ट्रीय आदर्शों के अनुरूप होगा। यदि वे ऐसा करने में असफल रहे, तो यह स्वयं उनके लिए तथा राष्ट्र के लिए हानिकर सिद्ध होगा। इतिहास सिद्ध करता है कि यह चेतावनी आवश्यक थी। जब निहित-स्वार्थवाले लोगों द्वारा सामाजिक परिवर्तनों का प्रतिरोध होता है, तो हिंसा भड़क उठती है। इसीलिए स्वामीजी चाहते थे कि भारत में, जहाँ यह काफी काल से स्थगित है, यह परिवर्तन सहज तथा शान्तिपूर्ण हो। भारत के सुदीर्घ इतिहास को देखें, तो यह देश के भीतरी तथा बाहरी शक्तियों द्वारा बारम्बार आक्रान्त होता रहा है, परन्तु अपने लचीलेपन के कारण हर बार उस संकट पर विजय पा लेता रहा है। उसने अपने मूलभूत आध्यात्मिक स्वरूप को त्यागे बिना ही उन परिवर्तनों को अंगीकार कर लिया। इसी प्रकार वह शताब्दियों तक लगभग वही राष्ट्र बना रहा। नवीन विचारों को आत्मसात् करने की, नये दबावों के साथ समायोजन करने की, नवीन परिस्थितियों के अनुसार स्वयं को ढालने की, इस क्षमता के कारण ही यह अपने पूर्ण विनाश को टालने में समर्थ हुआ है। जब-जब उसे संकट की लहरों ने डुबा दिया है, तब-तब वह और भी सबल होकर उभरा है, सम्भवतः एक नये रूप में, परन्तु अपने मूलभूत भावों को सुरक्षित रहते हुए। 

{स्वामीजी को जैसे श्रमिक-वर्ग के उत्थान का पूर्वानुमान हुआ, वैसे ही उन्हें आशंका थी कि इसके साथ-साथ सांस्कृतिक स्तर में गिरावट आएगी। बाद में जिन देशों में क्रान्तियाँ हुईं, उनके प्रमाण यह सिद्ध करते हैं कि स्वामीजी ने ठीक ही कहा था। भारत में भी कहीं ऐसा न हो, इसीलिए स्वामीजी चाहते थे कि जिस समय श्रमिक-वर्ग के उत्थान के लिए जमीन तैयार हो रही है, उसी समय आम जनता को भारत की आध्यात्मिक परम्परा के अनुसार शिक्षित करने पर समुचित ध्यान दिया जाना चाहिए। उन्होंने आवाहन किया - “देश में आध्यात्मिक भावों की बाढ़ ला दो।” खास कर कौन-से भावों की बात उनके मन में थी? स्पष्टतः - सत्य, न्याय, प्रेम, शान्ति, समन्वय आदि के वे ही आदर्श, जिन्होंने भारतीय संस्कृति को सर्वाधिक प्रभावित किया है। इन आदर्शों ने भारतीय मानस को हिंसा के प्रति एक सहज विरोध-भाव से परिपूर्ण कर दिया है। स्वामी विवेकानन्द इस बात को लेकर बड़े चिन्तित थे कि उनके द्वारा पूर्वदृष्ट सामाजिक परिवर्तन भारतीय परम्पराओं को हानि पहुँचाने के स्थान पर, उनका संरक्षण करे। इन परम्पराओं का संरक्षण उनकी पहली प्राथमिकता थी। उन्हीं में भारत की शक्ति निहित है। जब तक भारत ने इन परम्पराओं में निष्ठा रखी, तब तक वह सुरक्षित रहा। भारत को न केवल अपने, बल्कि दुनिया की विरासत के रूप में भी इन परम्पराओं को\parfillskip=0pt}\newpage\noindent सुरक्षित रखना होगा। भारत ने हजारों वर्षों से जिन परम्पराओं का विकास तथा संरक्षण किया है, वैसी परम्पराएँ दुनिया के किसी भी देश में नहीं हैं।


\section*{विज्ञान तथा प्रौद्योगिकी के द्वारा विकास}

स्वामी विवेकानन्द का चित्त भारत की निर्धनता से व्यथित था। भारत के ग्रामीण अंचलों का भ्रमण करते समय उन्होंने लोगों के कष्ट तथा पीड़ाएँ देखीं। परन्तु साथ ही वे लोगों के सद्गुणों को देखकर अभिभूत भी हुए थे। वे अंग्रेज लोगों से केवल इस कारण घृणा करते थे कि उन लोगों ने योजनाबद्ध रूप से राष्ट्र के धन का शोषण कर लिया था। वे तथाकथित कुलीन भारतवासियों की स्वार्थपरता तथा अपने देशवासियों की हालत के प्रति उदासीनता के कारण, उनसे भी कम घृणा नहीं करते थे। अपने जीवन के अन्तिम वर्षों में वे यथासम्भव उन लोगों के सान्निध्य से बचने का प्रयास करते थे। 

परन्तु निर्धनता की समस्या को कैसे हल किया जाए? उन्हें लगा कि विज्ञान तथा प्रौद्योगिकी में ही इस समस्या का समाधान निहित है। भारत को अपने देश में एक औद्योगिक क्रान्ति लाने के लिए पाश्चात्य विज्ञान तथा तकनीकी का व्यापक रूप से प्रयोग करना होगा। उन्होंने देखा था कि किस प्रकार पाश्चात्य देशों ने विज्ञान तथा तकनीकी की सहायता से निर्धनता पर विजय प्राप्त कर ली है। एशिया में जापान ने भी वैसा ही कर दिखाया था। वे चाहते थे कि निर्धनता के खिलाफ संघर्ष में भारत पश्चिम के पदचिह्नों का अनुसरण करे, परन्तु अन्य किसी भी क्षेत्र में वह पश्चिम की नकल न करे।


\section*{भारतीय जीवन-पद्धति}

स्वामीजी पश्चिमी देशों की भौतिक समृद्धि से प्रभावित थे, परन्तु उन्होंने यह भी देख लिया था कि इसने किस प्रकार उनके नैतिक दृष्टिकोण को भोथरा कर दिया था। वे चाहते थे कि भारत भी वैसी ही भौतिक समृद्धि हासिल करे, परन्तु साथ ही नैतिक आदर्शों के प्रति अपने लगाव को बनाए रखे। पश्चिम के लोगों में इन्द्रिय-सुखों के प्रति एक तरह का पागलपन था, जो स्वामीजी को पसन्द नहीं आया। वे भौतिक समृद्धि तथा गहन नैतिक संवेदनशीलता का समन्वय देखना चाहते थे। भारत के सम्पूर्ण इतिहास में भारत का यही मार्ग रहा है। 

स्वामी विवेकानन्द का कोई भी उपदेश पुराना नहीं पड़ा है। उन्होंने आधुनिक विचारों का प्रचार किया, परन्तु वे एक दूरदृष्टि-सम्पन्न व्यक्ति थे; उनके लिए न तो केवल तात्कालिक विश्व महत्त्वपूर्ण था और न ही आनेवाला सुदूर भविष्य। उनकी मनुष्य से सम्बन्धित हर विषय में - न केवल धर्म, बल्कि विज्ञान, कला, साहित्य, इतिहास, राजनीति में भी रुचि थी। इन कुछ विषयों पर भी उनके विचार युगान्तरकारी थे। उदाहरणार्थ वे चाहते थे कि भारत में एक जाति-वर्ग से रहित समाज हो। उनका कहना था कि यह समाज इस्लामी शरीर तथा वेदान्ती बुद्धि वाला हो। इस्लाम में जाति-वर्ग का भेद स्वीकार नहीं किया जाता और इस मामले में वे इस्लाम के प्रशंसक थे। परन्तु उनके मतानुसार समाज को आदर्श होने के लिए उसे केवल जाति-वर्गहीन होना ही काफी नहीं था, बल्कि उसे वेदान्त में प्रतिपादित आध्यात्मिक विकास की ओर भी दृष्टि रखनी होगी। इसीलिए वे एक ऐसे समाज के हामी थे, जो इस्लामी लक्षणवाला होगा, परन्तु साथ ही उसमें अधिकतम आध्यात्मिक उन्नति पर बल दिया जाएगा। 

वर्तमान में भारत के सामने कई समस्याएँ हैं - निर्धनता, निरक्षरता, जातिवाद तथा एकता का अभाव। परन्तु ये कोई नई समस्याएँ नहीं हैं। इनका स्वामीजी के समय में भी अस्तित्व था। उन्होंने इन पर काफी विचार किया और उनके समाधान के विषय में अपने विचार व्यक्त किए, जो सदा के लिए उपयोगी हैं। वे किसी के द्वारा पूर्व निर्दिष्ट समाधान के नहीं, बल्कि ऐसे समाधानों के पक्षधर थे, जो इतिहास तथा परिस्थितियों को ध्यान में रखकर निकाले गए थे। उन्हें दूसरे देशों का अनुकरण भी पसन्द नहीं था। प्रत्येक देश को अपने ढंग से, अपनी प्रतिभा की सहायता से अपनी समस्याओं से निपटना होगा। उन्होंने लोगों की इच्छा-शक्ति, सही दृष्टिकोण तथा चरित्र पर बल दिया। वर्तमान समस्याएँ हल हो सकती हैं, परन्तु शीघ्र ही अन्य समस्याएँ उठ खड़ी होंगी। बिना समस्याओं के जीवन ही अकल्पनीय है। परन्तु ये गुण आ जाएँ, तो फिर कोई भी समस्या कठिन नहीं रह जाएगी।


\section*{स्वामी विवेकानन्द की कार्ययोजना}

स्वामी विवेकानन्द ने तीन भविष्यवाणियाँ की थीं, जिनमें से दो सत्य सिद्ध हो चुकी हैं। इनमें से पहली तथा सर्वाधिक महत्त्वपूर्ण थी भारत की स्वाधीनता के विषय में। १८९० के दशक में ही उन्होंने कहा था, “भारत अकल्पनीय परिस्थितियों के बीच अगले पचास वर्षों में ही स्वाधीन हो जाएगा।” ठीक ऐसा ही हुआ। जब उन्होंने यह बात कही, तब कुछ ही लोगों ने इस पर ध्यान दिया और शायद ही किसी ने इस उक्ति को महत्त्व दिया हो। उस समय ऐसा होने की कोई सम्भावना नहीं दीख रही थी। अधिकांश लोग अंग्रेजों द्वारा शासित होकर सन्तुष्ट थे। उनके लिए अंग्रेजी राज्य का अर्थ था कानून का शासन, समानता तथा न्याय का शासन। उन दिनों लोगों में शायद की कहीं कोई राजनीतिक चेतना दिखायी पड़ती थी। उन्हें राजनीतिक स्वाधीनता की कोई धारणा न थी। यहाँ तक कि बुद्धिजीवी वर्ग भी यह निश्चित रूप से नहीं कह सकता था कि वह क्या चाहता है। उनमें से कुछ मुट्ठी भर लोग अपने लिए बेहतर कार्य और यदि सम्भव हुआ तो देश के प्रशासन में एक कनिष्ठ हिस्सेदार की भूमिका चाहते थे। उनके मन में स्वाधीन भारत का विचार कभी आया ही नहीं था। उनमें से कुछ लोग तो यहाँ तक सोचते थे कि अंग्रेजों का शासन एक वरदान है। वे चाहते थे कि देश की शान्ति तथा प्रगति के लिए इसे जारी रहना चाहिए। ऐसी पृष्ठभूमि में स्वामी विवेकानन्द ने भविष्यवाणी की कि देश अगले पचास वर्षों के भीतर ही स्वाधीन हो जाएगा। 

\newpage

उनकी दूसरी महत्त्वपूर्ण भविष्यवाणी थी कि रूस में पहली बार श्रमिक-क्रान्ति होगी, जिसके होने या हो सकने के बारे में किसी को कल्पना तक न थी। श्रमिक-क्रान्ति के मुख्य प्रवक्ता मार्क्स का कहना था कि यह वहीं होगा, जहाँ ट्रेड-यूनियन आन्दोलन काफी मजबूत होगा। इसी आधार पर उन्होंने कहा कि यह जर्मनी में होगा। तथापि मार्क्स के कथन के विपरीत, पहली श्रमिक-क्रान्ति एक ऐसे देश रूस में हुई, जो मूलतः कृषिप्रधान था और जहाँ किसी संगठित श्रमिक आन्दोलन का अभाव था। तो फिर यह कैसे घटित हुआ? स्वामी विवेकानन्द ने किस आधार पर ये सही भविष्यवाणियाँ कीं? यह बता पाना कठिन है। यह सर्वविदित है कि वे इतिहास के एक अच्छे अध्येता थे और ऐतिहासिक शक्तियों के विषय में उनकी अन्तर्दृष्टि गहन तथा ठोस थी। 

विवेकानन्द ने एक भविष्यवाणी और की थी, जिसका सत्य सिद्ध होना अभी बाकी है। उन्होंने कहा था कि भारत एक बार फिर समृद्धि तथा शक्ति की महान् ऊँचाइयों तक उठेगा और अपने समस्त प्राचीन गौरव को पीछे छोड़ जाएगा। भारत की वर्तमान अवस्था को देखने से ऐसी किसी भी सम्भावना का संकेत नहीं मिलता। उलटे यह ऐसी समस्याओं से घिरा हुआ है, जो कुछ लोगों के मतानुसार उसके विनाश का कारण बन सकती हैं। यह सच है कि इनमें से कुछ समस्याएँ - निर्धनता, अशिक्षा, जातिवाद आदि तो बड़ी पुरानी हैं, दीर्घ काल से - स्वाधीनता के पहले से ही चली आ रही हैं और अब भी वैसी ही सबल तथा भयंकर रूप धारण किए हुए हैं। परन्तु भारत ने किसी भी प्रकार उनके साथ रहना सीख लिया है और वे कोई विशेष चिन्ता का कारण नहीं हैं। नयी समस्याएँ, जो सचमुच ही अत्यन्त अशुभ हैं, वे हैं विभिन्न आंचलिक समुदायों का पृथकतावाद और साम्प्रदायिक कट्टरता। देश पहले ही सम्प्रदाय के आधार पर एक बड़े विभाजन का शिकार हो चुका है और एक प्रश्न हर व्यक्ति के मन को चिन्तित किए हुए है कि यदि और भी विभाजन हुए, तो भारत का अस्तित्व ही खतरे में पड़ जाएगा।


\section*{जियो और जीने दो}

स्वामीजी ने अपनी अन्तर्दृष्टि से यह स्पष्ट जान लिया था कि यदि भारत ने राजनीति में पाश्चात्य तरीकों को अपनाया, तो ऐसी ही परिस्थिति का निर्माण होगा और उन्होंने भारत को ऐसे अनुकरण के विषय में सावधान कर दिया था। पाश्चात्य राष्ट्र स्वभाव से ही एकांगी और इस कारण असहिष्णु हैं। दूसरी ओर भारत सदा से ही नये-नये प्रजातीय समूहों और उनके साथ ही नये विचारों तथा जीवन-शैलियों का स्वागत करता हुआ, उन्हें आत्मसात् करके अपनाता रहा है। यहाँ आए यूनानियों, सीथियनों, मंगोलों तथा हूणों का अब अलग अस्तित्व कहाँ रह गया है? वे सभी भारतीयता की पहचान में समाहित हो गए हैं। आज भारत अनेक प्रजातियों, संस्कृतियों तथा परम्पराओं की एक सुन्दर कलाकृति बन गया है, जिसका प्रत्येक तत्त्व आपसी प्रेम तथा सद्भाव के धागे से जुड़ा हुआ है। भारतवर्ष सदा से ही एक बहु-प्रजातीय तथा बहु-धर्मीय राष्ट्र रहा है। 

अत्यन्त उन्नत लोग ऐसे लोगों के बीच में रहते आए हैं, जो उनकी तुलना में पिछड़े हुए हैं। तथापि वे सदा से आपसी शान्ति तथा सद्भाव के साथ रहते आए हैं। उनके बीच शायद ही कभी आपसी सम्बन्धों को कटु बनानेवाले संघर्ष होते हैं। उनमें से अधिकांश लोग पूरी तौर से आत्मसात् हो गए हैं, परन्तु यदि कुछ ने अपना अलग अस्तित्व बनाए रखना चाहा है, तो वे वैसे ही रहे हैं। ‘जियो और जीने दो’ - यही भारत की नीति रही है। भारत सदा से ही ‘बहुत्व में एकत्व’ के दर्शन में विश्वास करता आया है। ईश्वर एक है और मनुष्य भी एक है। भारत ने कभी किसी भी समूह की स्वाधीनता में हस्तक्षेप करने का प्रयास नहीं किया। इसीलिए प्रत्येक धार्मिक समुदाय के भीतर अनेक सम्प्रदायों का अस्तित्व विद्यमान है। स्वामीजी ने सम्प्रदायों के बहुत्व का स्वागत किया। वे जानते थे कि जहाँ स्वाधीनता है, वहीं उन्नति भी सम्भव है। उन्नति में स्वाधीनता होने पर विविधता अवश्यम्भावी है। विविधता स्वाभाविक है और एकरूपता कृत्रिम है। प्रत्येक को अपने ही तरीके विकास करना होगा, अन्यथा वह विकसित ही नहीं हो सकता। भारत सदा से ही इस सिद्धान्त में विश्वास करता रहा है और इसी के अनुसार जीता आया है। तो फिर एकता कहाँ रही? एकता एक-दूसरे के सिद्धान्तों का सम्मान करने में और आदर्शों की समानता में है। स्वामीजी के मतानुसार भारत के राष्ट्रीय आदर्श हैं - त्याग और सेवा। प्रत्येक समूह दूसरों के लिए त्याग करता है और इस प्रकार राष्ट्र की सेवा करता है। इसी पद्धति से विविधता ने भारतीय एकता को सुदृढ़ करने में मदद की है। इसी पद्धति से किसी भी बहु-प्रजातीय समाज में एकता की स्थापना सम्भव है। आशा है कि भारत अपने इन राष्ट्रीय आदर्शों में निष्ठा बनाए रखेगा और सत्ता के उस लोभ से सम्मोहित नहीं होगा, जो कि पाश्चात्य राजनीति का प्रमुख वैशिष्ट्य है। भारत में राजनीति का उद्देश्य शासन नहीं, बल्कि सेवा होना चाहिए।


\section*{श्रमिक का बुद्धिजीवी वर्गों में समता}

स्वामीजी जानते थे कि वह दिन दूर नहीं, जब एक नये भारत का उदय होगा, जिसमें शक्ति बुद्धिजीवी-वर्ग (ब्राह्मणों) के हाथ में नहीं, बल्कि श्रमिक-वर्ग (शूद्रों) के हाथ में होगी। यह जितना शीघ्र हो, देश के लिए उतना ही अच्छा होगा। उन्हें यह भी आशा थी कि बुद्धिजीवी वर्ग, अपने तथा राष्ट्र के हित में भी, इस परिवर्तन का स्वागत करेगा। अन्य स्थानों में यह परिवर्तन हिंसा के साथ आया है। परन्तु भारत में किसी हिंसा की आवश्यकता नहीं है, बशर्ते कि बुद्धिजीवी-वर्ग अपने राष्ट्रीय आदर्शों के अनुसार चलें। भारतीय समाज सदा से ही उल्लेखनीय मात्रा में लचीलापन दिखाता आया है और समय की आवश्यकताओं के अनुसार स्वयं को समायोजित करता रहा है। 

पूरी तौर से नहीं, तो आंशिक रूप से ही सही, स्वामीजी की भविष्यवाणी सत्य सिद्ध हुई है। आज का श्रमिक-वर्ग पिछले किसी भी काल की अपेक्षा अधिक सशक्त हो चुका है। वे लोग अब बेहतर स्थिति में हैं और क्रमशः बुद्धिजीवी वर्ग की बराबरी पर पहुँच रहे हैं। स्वामीजी नहीं चाहते थे कि इस तरह की सामाजिक परिवर्तन की प्रक्रिया में देश के बौद्धिक तथा नैतिक स्तर में ह्रास हो। इसीलिए वे चाहते थे कि श्रमिक-वर्ग के बच्चों को बुद्धिजीवी-वर्ग के बच्चों की तुलना में शिक्षा के अधिक अवसर प्रदान किए जाएँ। सबसे अधिक तो वे चाहते थे कि सांस्कृतिक तथा नैतिक दृष्टि से भारत पहले के समान ही सबल बना रहे। वे ‘ऊपर उठाने’ में विश्वास करते थे, ‘नीचे उतारने’ में नहीं। आज के स्वाधीन भारत में ठीक यही हो रहा है और यह एक शुभ लक्षण है।


\section*{इस्लामी शरीर और वेदान्ती बुद्धि}

स्वामीजी की परिकल्पना थी कि श्रमिक-वर्ग सत्ता में आएगा, उनका सांस्कृतिक तथा नैतिक स्तर उन्नत होगा और भारतीय समाज जाति-वर्ग-विहीन हो जाएगा। इस समाज के विषय में उनकी प्रिय उक्ति थी - ‘इस्लामी शरीर में वेदान्ती बुद्धि’। यह आदर्श समाज ऐसा होना चाहिए, जो निरन्तर नैतिक पूर्णता (जिसे उन्होंने ‘वेदान्ती’ कहा) के उच्चतर स्तर की उपलब्धि के लिए प्रयास करता रहेगा। 

अपने देश के लिए उन्होंने जो कार्ययोजना बनायी, उसमें पहली प्राथमिकता के रूप में, उन्होंने जिस समस्या को रेखांकित किया, वह था निर्धनता का उन्मूलन। वे जानते थे कि भारत को एक औद्योगिक क्रान्ति की आवश्यकता है। इसके लिए आवश्यक था कि सम्पूर्ण देश में विज्ञान तथा प्रौद्योगिकी की शिक्षा का विस्तार किया जाए। किसी काल में भारत विज्ञान तथा प्रौद्योगिकी के क्षेत्र में अन्य देशों की तुलना में काफी आगे रहा है, परन्तु पिछली कुछ शताब्दियों से यह जड़ीभूत हो गया है और अन्य देशों की तुलना में हर दृष्टि से पिछड़ गया है। जनता की निर्धनता उन्हें भौतिक तथा आध्यात्मिक दृष्टि से निर्जीव बनाती जा रही थी। सर्वोपरि, जनता में एक तरह की जड़ता छा गयी है, जिसके फलस्वरूप लोग परिस्थिति को अपरिहार्य मानकर उसे स्वीकार कर लेते हैं। इसी बात से स्वामीजी सर्वाधिक व्यथित होते थे। अंग्रेज लोगों ने इस स्थिति का गलत फायदा उठाया और निर्लज्ज क्रूरता के साथ जनता का शोषण किया। स्वामीजी ने जमशेदजी टाटा को एक ऐसा संस्थान स्थापित करने की सलाह दी, जिसमें उन्नत कोटि का शोध-कार्य किया जाए। दोनों एक ही जलयान पर जापान से अमेरिका की यात्रा कर रहे थे। टाटा जापान के औद्योगिक उत्पादों का आयात करने की चेष्टा कर रहे थे। परन्तु स्वामीजी को यह बात पसन्द नहीं आयी। वे चाहते थे कि टाटा न केवल भारत के औद्योगीकरण में एक अग्रदूत का कार्य करें, अपितु एक शोध-केन्द्र भी आरम्भ करें, ताकि नवीनतम तकनीकी का सतत प्रवाह औद्योगिक विस्तार को गति प्रदान करता रहे। टाटा ने स्वामीजी की सलाह मानी। उन्होंने एक शोध-केन्द्र की स्थापना की, जो कि सम्पूर्ण विश्व में विख्यात है। मजे की बात तो यह है कि उन्होंने स्वामीजी को इसका प्रथम निदेशक बनने को आमंत्रित किया था। स्वामीजी ने उसका क्या उत्तर दिया, यह ज्ञात नहीं हो सका है।


\section*{विज्ञान तथा धर्म का गठजोड़}

स्वामीजी जानते थे कि किसी राष्ट्र की शक्ति उसके आकार या धन पर नहीं, अपितु उसके चरित्र पर निर्भर करती है। उन्हें आशा थी कि भारत इस बात का एक उदाहरण प्रस्तुत करेगा कि किस प्रकार विज्ञान तथा धर्म का सम्मिलन हो सकता है - जिसमें विज्ञान मनुष्य की भौतिक आवश्यकताओं की पूर्ति करेगा और धर्म उसकी नैतिक तथा आध्यात्मिक जरूरतों की। यह आवश्यक नहीं कि एक धनाढ्य व्यक्ति नैतिक भी हो। स्वामीजी ने कहा - “मनुष्य-निर्माण ही मेरा जीवनोद्देश्य है।” उन्हें बोध हुआ कि किसी भी समाज में यदि उसके मानवीय घटक भले तथा सबल नहीं हैं, तो वह टिक नहीं सकता। वे पश्चिमी दुनिया की कर्मठता, संगठन-क्षमता और विज्ञान तथा प्रौद्योगिकी के क्षेत्र में उनकी उपलब्धियों से प्रभावित थे। परन्तु उन्होंने उन लोगों की इन्द्रिय-सुखों के प्रति घोर आसक्ति तथा नैतिक विकास के विषय में उदासीनता भी देखी। वे जानते थे कि किसी समाज में जब तक भौतिक समृद्धि तथा नैतिक विकास के बीच सही सन्तुलन नहीं होता, तब तक व्यक्ति को शान्ति नहीं मिल सकती और उस समाज की सर्वांगीण उन्नति भी नहीं हो सकती। 

साल-दर-साल भारत निर्धनता के विरुद्ध अपनी लड़ाई में विजय की ओर बढ़ता जा रहा है। खाद्यान्न के क्षेत्र में उसे उल्लेखनीय उपलब्धि हासिल हुई है। स्वाधीनता के तत्काल बाद उसे अपनी जनता को भरपेट भोजन कराने के लिए भी मुख्यतः आयात पर निर्भर रहना पड़ता था। आज वह खाद्य-उत्पादन के क्षेत्र में आत्मनिर्भरता की स्थिति में पहुँच गया है। भारत एक हरित क्रान्ति से होकर गुजर चुका है। इसे प्रायः बाढ़ों तथा अकालों का सामना करना पड़ता है। तथापि वह बिना किसी बाह्य सहायता के अपनी जनता का पेट भरने में सक्षम हो चुका है। विज्ञान तथा प्रौद्योगिकी के क्षेत्र में भी उसने उल्लेखनीय प्रगति की है। आज वह विश्व के सर्वाधिक उन्नत प्रौद्योगिकी वाले देशों के बीच स्थान बना चुका है। इंजीनियरी की तकनीकी में वह आत्मनिर्भर है और अनेक विकासशील देशों की सहायता करने में भी सक्षम है। 

परन्तु खेद की बात यह है कि अब भी देश के कुछ अंचलों में निर्धनता फैली हुई है। सामाजिक अन्याय भी अभी पूरी तौर से दूर नहीं हो सका है और अशिक्षा भी एक समस्या बनी हुई है। जब तक ये समस्याएँ सुलझ नहीं जातीं, तब तक स्वामी विवेकानन्द के भारत-विषयक विचार कैसे रूपायित हो सकते हैं! वे एक अधीर व्यक्ति थे और जानते थे कि भारत को साहस तथा आत्मविश्वास की जरूरत है। समस्याएँ भारत की हैं और स्वयं भारत को ही उनका समाधान ढूँढ़ निकालना होगा। उसे न तो दूसरों का अनुकरण करना है, और न ही दूसरों पर निर्भर रहना है। उसे अपने स्वयं के प्रयासों से अपनी वर्तमान तथा भविष्य में भी आनेवाली समस्याओं का हल निकालना होगा। 

\delimiter

