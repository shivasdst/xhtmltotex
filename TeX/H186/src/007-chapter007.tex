
\chapter{सच्चे धर्म का स्वरूप }

\indentsecionsintoc

\toendnotes{सच्चे धर्म का स्वरूप}

\addtoendnotes{\protect\begin{multicols}{3}}

\section*{धर्म की परिभाषा और सच्चा धर्म}

\addsectiontoTOC{धर्म की परिभाषा और सच्चा धर्म}

धर्म वह वस्तु है, जिससे पशु मनुष्य तक और मनुष्य परमात्मा तक उठ सकता है।\endnote{ १०/२१३;} 

हमारे मन का एक भाव कहता है - करो; और उसके पीछे एक दूसरा स्वर उठता है, जो कहता है - मत करो। हमारे मन में धारणाओं का एक समूह है, जो सर्वदा इन्द्रियों के द्वारा बाहर जाने की चेष्टा करता रहता है; और उनके पीछे चाहे कितना ही क्षीण क्यों न हो, एक स्वर कहता रहता है - बाहर मत जाना। इन दो वृत्तियों के संस्कृत में बड़ा सुन्दर नाम हैं - प्रवृत्ति और निवृत्ति।... इस ‘मत करना’ से ही धर्म और आध्यात्मिकता आरम्भ होती है। जहाँ यह ‘मत करना’ (निवृत्ति) नहीं है, वहाँ जानना कि धर्म आरम्भ ही नहीं हुआ।\endnote{ २/६३;} 

प्रत्येक आत्मा अव्यक्त ईश्वर है। बाह्य तथा आन्तरिक प्रकृतियों को वशीभूत करके, आत्मा के इस ईश्वरत्व को व्यक्त करना ही जीवन का चरम लक्ष्य है। कर्म, उपासना, मनःसंयम अथवा ज्ञान - इनमें से एक, कुछ या सभी उपायों का सहारा लेकर अपने ब्रह्मभाव को व्यक्त करो और मुक्त हो जाओ। बस, यही धर्म का सार-सर्वस्व है। मतवाद, पूजा-पद्धति, शास्त्र, मन्दिर अथवा अन्य बाह्य क्रिया-कलाप तो उसके गौण विवरण मात्र हैं।\endnote{ १/१७३;} 

धर्म का अर्थ है, उस ब्रह्मत्व की अभिव्यक्ति, जो सब मनुष्यों में पहले ही से विद्यमान है।\endnote{ २/३२८;} शास्त्र-ग्रन्थों में धर्म नहीं होता, अथवा सिद्धान्तों, मतवादों, चर्चाओं तथा तार्किक उक्तियों में भी धर्म-प्राप्ति नहीं होती, धर्म तो साक्षात्कार करने की वस्तु है। तुम्हें ऋषि होना होगा। ऐ मेरे मित्रो, जब तक तुम ऋषि नहीं बनोगे, जब तक तुम्हारा आध्यात्मिक सत्य के साथ साक्षात् नहीं होगा, निश्चय ही तब तक तुम्हारा धार्मिक जीवन आरम्भ नहीं हुआ। जब तक तुम्हारी यह अतिचेतन (समाधि) अवस्था शुरू नहीं होती, तब तक धर्म केवल कहने भर की बात है, तब तक यह केवल धर्म-प्राप्ति के लिए तैयारी मात्र है। तुम केवल दूसरों से सुनी-सुनायी बातों को दुहराते-तिहराते भर हो।\endnote{ ५/१४८;} 

सबसे बढ़कर एक अन्य बात भी हमें याद रखनी चाहिए; खेद है कि इसे हम प्रायः भूल जाते हैं। वह बात यह है कि भारत में धर्म का तात्पर्य ‘प्रत्यक्षानुभूति’ है, इससे जरा भी कम नहीं। कोई भी हमें ऐसा धर्म नहीं सिखा सकता कि ‘यदि तुम इस मत को मान लो, तो तुम्हारा उद्धार हो जाएगा’; क्योंकि ऐसी बात पर हम विश्वास करते ही नहीं। तुम अपने को जैसा बनाओगे, अपने को जैसे साँचे में ढालोगे, वैसे ही बनोगे। तुम जो भी हो, जैसे हो, वह ईश्वर की कृपा और अपनी चेष्टा से बने हो। किसी मत में विश्वास मात्र से तुम्हारा कोई विशेष उपकार नहीं होगा। ‘अनुभूति’ की यह महती शक्तिमयी वाणी भारत के ही आध्यात्मिक गगन-मण्डल से आविर्भूत हुई है और केवल हमारे ही शास्त्रों ने बारम्बार कहा है कि ‘ईश्वर के दर्शन’ करने होंगे। निःसन्देह यह बात बड़े साहस की है, पर इसका लेशमात्र भी मिथ्या नहीं है, यह अक्षरशः सत्य है। धर्म की प्रत्यक्ष अनुभूति करनी होगी, केवल सुनने से काम नहीं चलेगा; तोते की तरह कुछ शब्द और धर्म-विषयक बातें रट लेने से काम नहीं चलेगा; केवल बुद्धि द्वारा स्वीकार कर लेने से भी काम न चलेगा - जरूरत है अपने भीतर धर्म को प्रवेश कराने की। अतः ईश्वर के अस्तित्व पर विश्वास रखने का सबसे बड़ा प्रमाण यह नहीं है कि यह तर्क से सिद्ध है; वरन् ईश्वर के अस्तित्व का सर्वोच्च प्रमाण तो यह है कि हमारे यहाँ के प्राचीन तथा अर्वाचीन सभी सिद्ध लोगों ने ईश्वर का साक्षात्कार किया है।\endnote{ ५/२६८-६९;} 

पवित्र और निःस्वार्थी बनने की कोशिश करो - सारा धर्म इसी में है।\endnote{ १/३७९, १४४;} धर्म का रहस्य आचरण से जाना जा सकता है, व्यर्थ के मतवादों से नहीं। भले बनना तथा भलाई करना - इसी में सारा धर्म निहित है।\endnote{ १/३८०;} 

निःस्वार्थता ही ईश्वर है। कोई मनुष्य भले ही रत्नखचित सिंहासन पर आसीन हो, सोने के महल में रहता हो, परन्तु यदि वह पूर्ण रूप से निःस्वार्थ है, तो वह ब्रह्म में ही स्थित है। परन्तु दूसरा व्यक्ति, चाहे झोपड़ी में ही क्यों न रहता हो, चिथड़े ही क्यों न पहनता हो, सर्वथा दीन-हीन भाव से ही क्यों न रहता हो, परन्तु यदि वह स्वार्थी है, तो वह संसार में पूरी तौर से डूबा हुआ है।\endnote{ ३/६१;} 

स्वार्थपरता अर्थात् स्वयं के सम्बन्ध में पहले सोचना ही सबसे बड़ा पाप है। जो मनुष्य यह सोचता रहता है कि मैं ही पहले खा लूँ, मुझे ही सबसे अधिक धन मिल जाए, मैं ही सर्वस्व का अधिकारी बन जाऊँ, मेरी ही सबसे पहले मुक्ति हो और मैं ही दूसरों से पहले सीधा स्वर्ग चला जाऊँ; वही व्यक्ति स्वार्थी है। निःस्वार्थी व्यक्ति तो कहता है, ‘मुझे अपनी चिन्ता नहीं है, मुझे स्वर्ग जाने की भी कोई आकांक्षा नहीं है, मेरे नरक में जाने से भी यदि किसी को लाभ हो सकता है, तो मैं उसके लिए भी तैयार हूँ।’ यह स्वार्थपरता ही धर्म की कसौटी है। जिसमें जितनी अधिक निःस्वार्थता है, वह उतना ही आध्यात्मिक है और उतना ही शिव के समीप है।\endnote{ ५/४०;} 

पुण्य वह है, जो हमारी उन्नति में सहायता करता है; और पाप वह है जो हमें पतन की ओर ले जाता है। मनुष्य तीन प्रकार के गुणों से निर्मित है - पाशविक, मानवीय और दैवी। जो तुममें दैवी गुण बढ़ाता है, वह पुण्य है; और जो तुममें पशुता बढ़ाता है, वह पाप है। तुम्हें पाशविक वृत्ति को मारकर मनुष्य अर्थात् प्रेममय तथा उदार बनना होगा। तुमको और भी ऊपर उठना होगा; और शुद्ध आनन्द, सच्चिदानन्द, अदाहक अग्नि के समान; अद्भुत प्रेममय, किन्तु मानवीय प्रेम की दुर्बलता से रहित, दुःख की भावना से रहित बनना होगा।\endnote{ १/२९४-९५;} 

मनुष्य जहाँ भी और जिस स्थिति में भी है, यदि धर्म वहीं उसकी सहायता नहीं कर सकता, तो उसकी उपयोगिता अधिक नहीं है - तब वह केवल कुछ विशिष्ट व्यक्तियों के लिए कोरा सिद्धान्त मात्र होकर रह जाएगा। धर्म यदि मानवता का हित करना चाहता है, तो उसके लिए यह आवश्यक है कि वह मनुष्य की प्रत्येक दशा में उसकी सहायता कर सकने में तत्पर और सक्षम हो - चाहे गुलामी हो या आजादी, चाहे घोर पतन हो या अतीव पवित्रता, उसे सर्वत्र मानव की सहायता कर पाने में समर्थ होना चाहिए।\endnote{ ८/१२;} 

जिस किसी वस्तु से आध्यात्मिक, मानसिक या शारीरिक दुर्बलता उत्पन्न हो, उसे पैर की उंगलियों से भी मत छुओ। मनुष्य में जो स्वाभाविक बल है, उसकी अभिव्यक्ति धर्म है। असीम शक्ति का स्प्रिंग इस छोटी-सी काया में कुण्डली मारे विद्यमान है और वह स्प्रिंग स्वयं फैल रहा है। ज्यों-ज्यों यह फैलता है, त्यों-त्यों एक-एक शरीर अपर्याप्त होता जाता है और वह उनका परित्याग कर उच्चतर शरीर धारण करता है। यही है मनुष्य का धर्म, सभ्यता या प्रगति का इतिहास।\endnote{ ९/१५५-५६;}


\section*{ईश्वर प्राप्ति के मार्ग}

\addsectiontoTOC{ईश्वर प्राप्ति के मार्ग}

दुनिया में साधारणतया चार प्रकार के लोग होते हैं - बुद्धिवादी, भावुक, रहस्यवादी, कर्मठ। हमें इनमें से प्रत्येक के लिए उचित प्रकार की पूजा-विधि देनी होगी। बुद्धिवादी कहता है - “मुझे इस तरह का पूजा-विधि पसन्द नहीं। मुझे दार्शनिक, विवेक-सिद्ध सामग्री दो - मैं वही चाहता हूँ।” अतः बुद्धिवादी मनुष्य के लिए बुद्धिसम्मत दार्शनिक पूजा है। फिर आता है कर्मठ। वह कहता है - “दार्शनिक की पूजा मेरे किसी काम की नहीं। मुझे अपने मानव-बन्धुओं की सेवा करने दो।” उनके लिए सेवा ही सबसे बड़ी पूजा है। रहस्यवादी और भावुक के लिए भी उनके योग्य पूजा-विधियाँ हैं। धर्म में इन सब लोगों के लिए विश्वास के तत्त्व हैं।\endnote{ ९/२२१;} 

सारे मानव जाति का, सभी धर्मों का एक ही चरम लक्ष्य है और वह है - भगवान से पुनर्मिलन या दूसरे शब्दों में उस ईश्वरीय स्वरूप की प्राप्ति और वही प्रत्येक व्यक्ति का सच्चा स्वभाव है। परन्तु यद्यपि लक्ष्य एक ही है, तो भी लोगों के विभिन्न स्वभावों के अनुसार उसकी प्राप्ति के साधनों में भिन्नता हो सकती है। 

लक्ष्य और उसकी प्राप्ति के साधन - इन दोनों को मिलाकर ‘योग’ कहा जाता है। ‘योग’ शब्द संस्कृत के उसी धातु से उत्पन्न हुआ है, जिससे अंग्रेजी शब्द ‘योक’ निकला है, जिसका अर्थ है ‘जोड़ना’, अर्थात् अपने को उस परमात्मा से जोड़ना, जो कि हमारा सच्चा स्वरूप है। इस प्रकार के योग अथवा मिलन के विभिन्न साधन हैं, पर उनमें मुख्य हैं - कर्मयोग, भक्तियोग, राजयोग और ज्ञानयोग। 

हर मनुष्य का विकास उसके अपने स्वभाव के अनुसार ही होना चाहिए। जैसे हर विज्ञान के अपने अलग-अलग तरीके होते हैं, वैसे ही प्रत्येक धर्म के भी हैं। धर्म के चरम लक्ष्य की प्राप्ति के तरीकों या साधनों को हम ‘योग’ कहते हैं। विभिन्न प्रकृतियों और स्वभावों के अनुसार योग के भी विभिन्न प्रकार हैं। उनके निम्नलिखित चार विभाग हैं -

\begin{enumerate}
\item कर्मयोग - इसके अनुसार मनुष्य कर्म और कर्तव्य के द्वारा अपने ईश्वरीय स्वरूप की अनुभूति करता है। 

 \item भक्तियोग - इसमें सगुण ईश्वर के प्रति भक्ति और प्रेम के द्वारा अपने ईश्वरीय स्वरूप की अनुभूति होती है। 

 \item राजयोग - इसके अनुसार मनःसंयम के द्वारा मनुष्य अपने ईश्वरीय स्वरूप की अनुभूति करता है। 

 \item ज्ञानयोग - इसके अनुसार ज्ञान के द्वारा व्यक्ति को अपने ईश्वरीय स्वरूप की अनुभूति होती है। 

\end{enumerate}

ये सब एक ही केन्द्र - परमात्मा की ओर ले जानेवाले भिन्न-भिन्न मार्ग हैं।\endnote{ ३/१६९-७०;} 

ज्ञान, भक्ति, योग और कर्म - ये चार मार्ग मुक्ति की ओर ले जाने वाले हैं। हर व्यक्ति को उस मार्ग का अनुसरण करना चाहिए, जिसके लिए वह योग्य है, लेकिन इस युग में कर्मयोग पर विशेष बल देना होगा।\endnote{ १०/२१८;}


\section*{ज्ञानयोग - विचार का मार्ग}

\addsectiontoTOC{ज्ञानयोग - विचार का मार्ग}

तुम्हें सदा स्मरण रखना होगा कि वेदान्त का मूल सिद्धान्त एकत्व या अखण्डता भाव है। द्वैत कहीं भी नहीं है; दो जगतों के लिए दो भिन्न प्रकार के जीवन नहीं हैं।... एक ही जीवन है, एक ही जगत् है और एक ही सत्ता है। सब कुछ वही एक सत्ता मात्र है; भेद केवल मात्रा का है, प्रकार का नहीं।... अमीबा और मैं - एक ही हूँ, अन्तर केवल परिमाण का है; और सर्वोच्च जीवन की दृष्टि से देखने पर सारे भेद मिट जाते हैं।\endnote{ ८/८-९;} 

कोई भी पूर्णता हासिल करने की जरूरत नहीं है। तुम पहले से ही मुक्त और पूर्ण हो। धर्म, ईश्वर या परलोक-विषयक ये सब धारणाएँ कहाँ से आयीं? मनुष्य क्यों ‘ईश्वरईश्वर’ करता घूमता फिरता है? सभी देशों में, सभी समाजों में मनुष्य क्यों पूर्ण आदर्श का अन्वेषण करता फिरता है - चाहे वह आदर्श मनुष्य में हो, अथवा ईश्वर या किसी अन्य वस्तु के रूप में? इसलिए कि वह भाव तुम्हारे ही भीतर वर्तमान है। वह थी तुम्हारे हृदय की धड़कन और तुम उसे नहीं जानते थे; तुम सोचते थे कि बाहर की कोई वस्तु यह ध्वनि कर रही है। तुम्हारी आत्मा में विराजमान ईश्वर ही तुम्हें अपनी खोज करने को - अपनी उपलब्धि करने को प्रेरित कर रहा है। यहाँ, वहाँ, मन्दिर में, गिरजाघर में, स्वर्ग में, मर्त्य में, विभिन्न स्थानों में, अनेक उपायों से खोज करने के बाद, अन्त में हम एक चक्कर पूरा करके; हमने जहाँ से आरम्भ किया था, वहीं अर्थात् अपनी आत्मा में ही वापस आ जाते हैं और देखते हैं कि जिसको हम सारे जगत् में खोजते फिर रहे थे, जिसके लिए हमने मन्दिरों और गिरजों में जा-जाकर कातर होकर प्रार्थनाएँ कीं, आँसू बहाए, जिसको हम सुदूर आकाश में मेघराशि के पीछे छिपा हुआ अव्यक्त और रहस्यमय समझते रहे, वह हमारे निकट से भी निकट है, प्राणों का भी प्राण है, हमारा शरीर है, हमारी आत्मा है - तुम्ही ‘मैं’ हो, मैं ही ‘तुम’ हूँ। यही तुम्हारा स्वरूप है - इसी को अभिव्यक्त करो। तुम्हें पवित्र होना नहीं पड़ेगा - तुम स्वयं पवित्र ही तो हो। तुम्हें पूर्ण होना नहीं पड़ेगा - तुम पूर्ण ही तो हो। सारी प्रकृति देश-कालातीत सत्य को परदे के समान ढँके हुए है। तुम जो कुछ भी अच्छा विचार या अच्छा कार्य करते हो, उससे मानो वह आवरण धीरे-धीरे छिन्न होता रहता है और वह देश-कालातीत शुद्ध स्वरूप, अनन्त ईश्वर स्वयं अभिव्यक्त होता रहता है।\endnote{ २/१४;} 

वेदान्त का समाधान यह है कि हम बद्ध नहीं, वरन् नित्य-मुक्त हैं। यही नहीं, बल्कि अपने को बद्ध सोचना भी अनिष्टकार है, वह भ्रम है - आत्म-सम्मोहन है। ज्योंही तुमने कहा कि मैं बद्ध हूँ, दुर्बल हूँ, असहाय हूँ, त्योंही तुम्हारा दुर्भाग्य आरम्भ हो गया, तुमने अपने पैरों में एक बेड़ी और डाल ली। अतः ऐसी बात न कभी कहना और न कभी सोचना ही।\endnote{ १०/३६७;} 

वेदान्त कहता है कि दुर्बलता ही दुनिया के सारे दुःखों का कारण है, इसी से सारे दुःख-कष्ट पैदा होते हैं। हम दुर्बल हैं, इसीलिए इतना दुःख भोगते हैं। हम दुर्बलता के कारण ही चोरी-डकैती, झूठ-ठगी तथा इसी प्रकार के अनेकानेक दुष्कर्म करते हैं। दुर्बल होने के कारण ही हम मृत्यु के मुख में गिरते हैं। जहाँ हमें दुर्बल बनाने वाला कोई नहीं है, वहाँ न मृत्यु है, न दुःख। हम लोग केवल भ्रान्तिवश ही दुःख भोगते हैं। इस भ्रान्ति को दूर कर दो और तत्काल सारे दुःख चले जाएँगे।\endnote{ २/१८६;} 

मेरा उद्देश्य यह दिखलाना है कि नैतिकता और निःस्वार्थता के सर्वोच्च आदर्श, उच्चतम दार्शनिक धारणा के साथ असंगत नहीं हैं; नैतिकता और नीतिशास्त्र की उपलब्धि के लिए तुमको अपनी दार्शनिक धारणा को नीचा नहीं करना पड़ता, वरन् नैतिकता और नीतिशास्त्र को ठोस आधार देने के लिए तुमको उच्चतम दार्शनिक और वैज्ञानिक धारणाएँ स्वीकार करनी होंगी। मनुष्य का ज्ञान मानवीय हित का विरोधी नहीं है, बल्कि जीवन के प्रत्येक विभाग में ज्ञान हमारी रक्षा करता है। ज्ञान ही उपासना है। हम जितना जान सकें, उसी में हमारा मंगल है। वेदान्ती कहते हैं, इन समस्त प्रतीयमान बुराइयों का कारण है - असीम का सीमाबद्ध हो जाना। जो प्रेम सीमाबद्ध होकर क्षुद्र-भावापन्न हो जाता है तथा बुरा प्रतीत होता है, वही अपनी चरमावस्था में स्वयं को ईश्वर के रूप में व्यक्त करता है। वेदान्त यह भी कहता है कि इस प्रतीयमान सम्पूर्ण बुराई का कारण हमारे भीतर ही है। किसी अलौकिक विधाता को दोष न दो, न निराश या दुखी होओ, न यह सोचो कि तुम गड्ढे में पड़े हो और जब तक कोई दूसरा आकर तुम्हारी सहायता नहीं करता, तब तक तुम इससे निकल नहीं सकते। वेदान्त कहता है कि ऐसा नहीं हो सकता।... बाहर से कोई सहायता नहीं मिलती, सहायता मिलती है भीतर से। तुम दुनिया के सारे देवताओं के समक्ष रो सकते हो, मैं भी अनेक वर्ष ऐसे ही रोता रहा, अन्त में देखा कि मुझे सहायता मिल रही है, किन्तु यह सहायता भीतर से मिली। भ्रान्तिवश इतने दिनों तक मैं जो अनेक प्रकार के काम करता रहा, वह भ्रान्ति मुझे दूर करनी पड़ी। यही एकमात्र उपाय है। मैंने स्वयं को जिस जाल में फँसा रखा है, वह मुझे ही काटना पड़ेगा और उसे काटने की शक्ति भी मुझमें ही है। इस विषय में मैं निश्चयपूर्वक कह सकता हूँ कि मेरे जीवन की भली या बुरी कोई भी प्रवृति व्यर्थ नहीं गयी - मैं उसी अतीत के भले-बुरे - दोनों प्रकार के कर्मों का समष्टि-स्वरूप हूँ। मैंने जीवन में बहुतसी भूलें की हैं, किन्तु इनको किए बिना आज जो मैं हूँ, वह कभी न होता। मैं अब अपने जीवन से परम सन्तुष्ट हूँ। पर मेरे कहने का अर्थ यह नहीं कि तुम घर जाकर यथेच्छा अन्याय करते रहो। मेरी बात का गलत अर्थ न समझ लेना। मेरे कहने का अभिप्राय यही है कि कुछ भूल-चूक हो गयी है, इसलिए एकदम हाथ-पर-हाथ रखकर मत बैठे रहो, अपितु यह समझ लो कि अन्त में फल सबका अच्छा ही होता है। इसके विपरीत और कुछ कभी नहीं हो सकता, क्योंकि शिवत्व और विशुद्धत्व हमारा स्वाभाविक धर्म है। उसका किसी भी प्रकार नाश नहीं हो सकता। हम लोगों का यथार्थ स्वरूप सदा ही एकरूप रहता है।\endnote{ ८/६०-६१;} 

‘पाप’ की ही बात लो। मैं अभी वेदान्त के अनुसार ‘पाप की धारणा’ और यह धारणा कि ‘मनुष्य पापी है’ पर चर्चा कर रहा था। वस्तुतः दोनों एक ही हैं - एक सकारात्मक है और दूसरी नकारात्मक। पहली, मनुष्य को उसकी दुर्बलता दिखा देती है और दूसरी, उसकी शक्ति। वेदान्त कहता है कि यदि दुर्बलता है, तो चिन्ता की कोई बात नहीं, हमें विकास करना है। जब मनुष्य ने पहले-पहल जन्म लिया, तभी जान लिया गया कि उसका रोग क्या है। सभी अपना-अपना रोग जानते हैं - किसी दूसरे को बतलाने की जरूरत नहीं पड़ती। ‘हम रोगी हैं’ - केवल यह सोचते रहने से ही हम स्वस्थ नहीं हो सकते। इसके लिए दवा आवश्यक है। बाहर की सारी चीजें हम भूला सकते हैं, बाह्य जगत् के प्रति हम कपटाचारी हो सकते हैं, परन्तु हम सभी अपने मन के अन्तराल में अपनी दुर्बलताओं को जानते हैं। वेदान्त कहता है कि मनुष्य को सदैव उसकी दुर्बलता की याद कराते रहना अधिक सहायता नहीं करता। उसको बल प्रदान करो और सदैव निर्बलता का चिन्तन करते रहने से बल नहीं प्राप्त होता। \textbf{दुर्बलता का उपचार सदैव उसका चिन्तन करते रहना नहीं है, वरन् बल का चिन्तन करना है। } मनुष्य में जो शक्ति पहले से ही विद्यमान है, उसे उसकी याद दिला दो। मनुष्य को पापी न बताकर वेदान्त ठीक उसका विपरीत मार्ग ग्रहण करता है और कहता है ‘तुम पूर्ण और शुद्धस्वरूप हो और जिसे तुम पाप कहते हो, वह तुममें है ही नहीं’। जिसे तुम ‘पाप’ कहते हो, वह तुम्हारी आत्म-अभिव्यक्ति का निम्नतम रूप है, अपनी आत्मा को उच्चतर भाव में व्यक्त करो। यह एक बात हम सबको सदैव याद रखनी चाहिए; और इसे हम सभी कर सकते हैं। कभी ‘नहीं’ मत कहना, यह न कहना कि ‘मैं नहीं कर सकता’, क्योंकि तुम अनन्त-स्वरूप हो। तुम्हारे स्वरूप की तुलना में देश-काल भी कुछ नहीं हैं। तुम सब कुछ कर सकते हो, तुम सर्वशक्तिमान हो।\endnote{ ८/११;} 

दो शक्तियाँ सदा समानान्तर रेखाओं में एक-दूसरे के साथ कार्य कर रही हैं। एक कहती है ‘मैं’ और दूसरी कहती है ‘मैं नहीं’। उनकी अभिव्यक्ति केवल मनुष्य में ही नहीं, अपितु पशुओं में और क्षुद्रतम कीटों में भी दीख पड़ती है। नर-रक्त-पिपासु लपलपाती जीभवाली बाघिन भी अपने बच्चे की रक्षा हेतु जान देने को तैयार रहती है। सबसे बुरा आदमी भी, जो सहज ही अपने भाई का गला काट सकता है - भूख से मरती हुई अपनी स्त्री तथा बच्चों के लिए निःसंकोच अपने प्राण दे देता है। सृष्टि के भीतर ये दोनों शक्तियाँ एक साथ काम कर रही हैं - जहाँ एक शक्ति देखोगे, वहीं दूसरी भी दीख पड़ेगी। एक स्वार्थपरता है और दूसरी निःस्वार्थता। एक है ग्रहण, दूसरी त्याग। एक लेती है, दूसरी देती है। क्षुद्रतम प्राणी से लेकर उच्चतम प्राणी तक सम्पूर्ण ब्रह्माण्ड इन्हीं दो शक्तियाँ का लीलाक्षेत्र है। इसके लिए किसी प्रमाण की जरूरत नहीं - यह स्वतःसिद्ध है। 

समाज के एक अंश के लोगों को जगत् के सारे कार्यों तथा विकास को - प्रतियोगिता और संघर्ष - इन दो में से केवल एक घटक पर आधारित बताने का क्या अधिकार है? विश्व के सारे कार्यों को द्वेष, युद्ध, प्रतियोगिता और संघर्ष पर अधिष्ठित मानने का उन्हें क्या अधिकार है? हम इनके अस्तित्व को अस्वीकार नहीं करते; परन्तु दूसरी शक्ति की क्रिया को बिलकुल न मानने का उन्हें क्या अधिकार है? क्या कोई व्यक्ति इस बात से इन्कार कर सकता है कि प्रेम, निरहंकारिता या त्याग ही जगत् की एकमात्र सकारात्मक शक्ति है? वह दूसरी शक्ति इस प्रेम-शक्ति का ही अनुचित प्रयोग है; प्रेम से ही प्रतिद्वन्द्विता की उत्पत्ति होती है, प्रेम ही प्रतियोगिता का मूल है। निःस्वार्थता ही बुराई की माता है। अच्छाई ही बुराई का जनक है; और बुराई का परिणाम भी अच्छाई के सिवा और कुछ नहीं है। जो व्यक्ति दूसरे की हत्या करता है, वह भी प्रायः अपने पुत्रादि के प्रति स्नेह की प्रेरणा से तथा उनके लालन-पालन के लिए ही करता है। उसका प्रेम संसार के अन्य लाखों लोगों से हटकर केवल अपने बच्चे में सीमित हो जाता है, पर ससीम हो या असीम, वह मूलतः प्रेम ही है। 

अतः सारे जगत् की परिचालक, जगत् में एकमात्र सच्ची तथा जीवन्त शक्ति वही एक अदभुत वस्तु है - वह किसी भी आकार में व्यक्त क्यों न हो; और वह है प्रेम, निःस्वार्थता तथा त्याग। इसीलिए वेदान्त अद्वैत पर जोर देता है।\endnote{ ८/५९-६०;} 

केवल अद्वैतवाद से ही नैतिकता की व्याख्या हो सकती है। हर धर्म यही प्रचार कर रहा है कि समस्त नैतिक तत्त्वों का सार दूसरों का भला करना ही है। - हम दूसरों का हित क्यों करें? हमें निःस्वार्थ होना चाहिए। - क्यों निःस्वार्थ होना चाहिए? कोई देवता ऐसा कह गए हैं? - मैं उन्हें नहीं मानता। शास्त्रों ने ऐसा कहा है। - लेकिन हम उन्हें क्यों मानें? शास्त्र यदि ऐसा कहते हैं तो मेरे लिए उनका क्या महत्त्व है? संसार के अधिकांश लोगों की यही नीति है कि हर कोई बाकी लोगों की चिन्ता छोड़कर केवल अपना ही भला देखता है। मैं क्यों नैतिक बनूँ? गीता में वर्णित इस सत्य को जाने बिना तुम उत्तर नहीं पा सकते - “जो व्यक्ति अपनी आत्मा को सब जीवों में स्थित देखता है और आत्मा में सब जीवों को देखता है, वह इस तरह ईश्वर को सर्वत्र समभाव से विद्यमान देखता हुआ आत्मा द्वारा आत्मा की हिंसा नहीं करता।” अद्वैतवाद की शिक्षा से तुम्हें यह ज्ञान होता है कि दूसरों की हिंसा करके तुम अपनी ही हिंसा करते हो, क्योंकि सब तुम्हारे ही स्वरूप हैं। तुम्हें मालूम हो या न हो, सब हाथों से तुम्हीं कार्य कर रहे हो, सब पैरों से तुम्हीं चल रहे हो, राजा के रूप में तुम्हीं महल में सुख-भोग कर रहे हो, फिर तुम्हीं रास्ते के भिखारी के रूप में दुःखमय जीवन बिता रहे हो। अज्ञ में भी तुम हो, विद्वान् में भी तुम हो, दुर्बल में भी तुम हो, सबल में भी तुम हो। इस तत्त्व का ज्ञान प्राप्त करके तुम्हें सबके प्रति सहानुभूति रखनी चाहिए। चूँकि दूसरे को कष्ट देना स्वयं को ही कष्ट पहुँचाना है, अतः हमें कदापि दूसरों को कष्ट नहीं देना चाहिए। यदि मैं बिना भोजन के मर भी जाऊँ, तो भी मुझे इसकी चिन्ता नहीं, क्योंकि जब मैं भूखा मर रहा हूँ, उसी समय मैं लाखों मुखों से भोजन भी कर रहा हूँ। अतः हमें ‘मैं’‘मेरा’ पर ध्यान ही नहीं देना चाहिए, यह सम्पूर्ण संसार ‘मेरा’ ही है, ‘मैं’ ही एक दूसरी रीति से संसार के सम्पूर्ण आनन्द का भोग कर रहा हूँ। मेरा या इस संसार का विनाश भी भला कौन कर सकता है? इस प्रकार अद्वैतवाद ही नैतिक तत्त्वों की एकमात्र व्याख्या है। अन्य मतवाद तुम्हें नैतिकता की शिक्षा दे सकते हैं, पर हम क्यों नीतिपरायण हों, इसका हेतु नहीं बता सकते।\endnote{ ५/३१५-१६;} 

व्यक्ति तथा खण्डित जीवों के रूप में हम अपना स्वरूप भूल जाते हैं। अतः अद्वैतवाद हमें भेद को त्यागने की शिक्षा नहीं देता, वरन् उसके रूप को समझ लेने को कहता है। हम वस्तुतः वही अनन्त पुरुष हैं; हमारे व्यक्तित्व जल की उन धाराओं के सदृश हैं, जिनके द्वारा वह अनन्त सत्ता अपने को अभिव्यक्त कर रही है; और यह विराट् परिवर्तनसमष्टि, जिसे हम ‘विकासवाद’ कहते हैं, अपनी अनन्त शक्ति को व्यक्त करने में चेष्टमान आत्मा के द्वारा सम्पादित होती है। परन्तु हम अनन्त के इस पार कहीं रुक नहीं सकते। हमारे आनन्द-ज्ञान तथा शक्ति को अनन्त होना ही है। अनन्त सत्ता, अनन्त शक्ति, अनन्त आनन्द हमारे हैं। हम लोगों को उन्हें उपार्जित नहीं करना है, वे सब हममें हैं, हमें तो उन्हें केवल प्रकट मात्र करना है। अद्वैतवाद का यही केन्द्रीय भाव है।\endnote{ ८/४६-४७;} 

जगत् को जानने का साहस करनेवाला कोई भी व्यक्ति इस बात से इनकार नहीं कर सकता कि जीवन और जगत् दोषमय हैं। परन्तु संसार के सभी धर्म इसका क्या प्रतिकार बताते हैं? वे कहते हैं कि यह संसार कुछ नहीं है, वास्तविक सत्य इस संसार के बाहर है। यहीं से कठिनाई शुरू होती है। यह उपाय तो मानो हमें अपना सब कुछ नष्ट कर देने का उपदेश देता है।... तो क्या कोई उपाय नहीं है? वेदान्त कहता है - विभिन्न धर्म जो कुछ कहते हैं, सब सत्य है, पर इसका ठीक-ठीक अर्थ समझ लेना होगा।... वेदान्त वस्तुतः जगत् की नकारता नहीं। भले ही वेदान्त में, जैसे परम वैराग्य के उपदेश हैं, वैसे अन्यत्र कहीं भी नहीं हैं, परन्तु इस वैराग्य का अर्थ शुष्क आत्महत्या नहीं है। वेदान्त में वैराग्य का अर्थ है - जगत् को ब्रह्म-रूप देखना - जगत् को हम जिस भाव से देखते हैं, उसे हम जैसा जानते हैं, वह हमें जैसा प्रतीत होता है, उसका त्याग करना और उसके सच्चे स्वरूप को पहचानना। उसे ब्रह्म-स्वरूप देखो - वास्तव में वह ब्रह्म के अतिरिक्त और कुछ भी नहीं है। इसी कारण सबसे प्राचीन उपनिषद् में हम पाते हैं - \textbf{ईशावास्यम् इदं सर्वं यत्किंच जगत्यां जगत् }- जगत् में जो कुछ है, वह सब ईश्वर से आच्छन्न है। सारे जगत् को ईश्वर से ढँक लेना होगा। किसी मिथ्या आशावादिता से नहीं, जगत् के दोषों तथा दुःख-कष्ट के प्रति आँखें मूँदकर नहीं, वरन् वास्तविक रूप से हर वस्तु के भीतर ईश्वर का दर्शन करना होगा। इसी प्रकार हमें संसार का त्याग करना होगा; और जब संसार का त्याग हो गया, तो शेष क्या रहा? - ईश्वर। इसका क्या तात्पर्य है? यही कि तुम्हारी स्त्री रहे, उससे कोई हानि नहीं, उसको छोड़ना नहीं होगा, वरन् स्त्री में तुम्हें ईश्वर-दर्शन करना होगा। सन्तान का त्याग करो, इसका अर्थ क्या यह है कि बाल-बच्चों को रास्ते में फेंक देना होगा, जैसा कि हर देश में कुछ नरपशु करते हैं? कदापि नहीं! वह धर्म नहीं - वह तो पैशाचिक कार्य है। तो फिर क्या करें? उनमें और इसी प्रकार सभी वस्तुओं में ईश्वर का दर्शन करो। जीवन में, मरण में, सुख में, दुःख में - सभी अवस्थाओं में ईश्वर समान रूप से विद्यमान है। केवल आँखें खोलकर उसके दर्शन करो।... आँखें खोलकर देखो कि अब तक तुम जगत् को जैसा देख रहे थे, वैसा वह कभी नहीं था - वह स्वप्न था, माया थी। थे तो एकमात्र प्रभु। वे ही सन्तान में, वे ही स्त्री में, वे ही पति में, वे ही अच्छे में, वे ही बुरे में, वे ही पाप में, वे ही पापी में, वे ही जीवन में और वे ही मरण में विद्यमान हैं।\endnote{ २/१४८;} 

\vskip 3pt


\section*{राजयोग - मनःसंयम का मार्ग}

\addsectiontoTOC{राजयोग - मनःसंयम का मार्ग}

हमारा उद्देश्य है - प्रकृति के पंजे से छुटकारा पाना। यही सारे धर्मों का एकमात्र लक्ष्य है।... योगी मनःसंयम के द्वारा इस परम लक्ष्य तक पहुँचने की चेष्टा करते हैं। जब तक हम प्रकृति के हाथ से अपना उद्धार नहीं कर लेते, तब तक हम गुलाम है; वह जैसा कहती है, हम वैसे ही चलने को लाचार हैं। योगी दावा करते हैं कि मन को वशीभूत कर लेनेवाला पदार्थों को भी वशीभूत कर लेता है। आन्तरिक प्रकृति बाह्य प्रकृति की अपेक्षा कहीं उच्चतर है और उस पर अधिकार प्राप्त करना बहुत कठिन है। इसीलिए जो लोग आन्तरिक प्रकृति को वशीभूत कर सकते हैं, सारा जगत् उनके वशीभूत हो जाता है - उनका दास हो जाता है। प्रकृति को इस प्रकार वश में लाने का उपाय ही राजयोग है।\endnote{ १/१७३;} 

\vskip 4pt

राजयोग-विद्या पहले मनुष्य को उसकी अपनी आन्तरिक अवस्थाओं के निरीक्षण का उपाय दे देती है। मन ही उस निरीक्षण का यंत्र है।... यहाँ विषय अन्दर की वस्तु है - मन ही विषय है; मन का अध्ययन करना ही यहाँ उद्देश्य है और मन ही, मन का अध्येता है। हम मन की एक ऐसी शक्ति के बारे में जानते हैं, जिसे मनन कहते हैं।... तुम एक साथ ही कर्म और चिन्तन दोनों कर रहे हो, परन्तु तुम्हारे मन का एक अंश मानो बाहर खड़े होकर, तुम जो चिन्तन कर रहे हो, उसे देख रहा है। मन की सारी शक्तियों को एकत्र करके मन पर ही उनका प्रयोग करना होगा। जैसे सूर्य की तीक्ष्ण किरणों के समक्ष घने अन्धकारमय स्थान भी अपने गुप्त तथ्य खोल देते हैं, वैसे ही यह एकाग्र मन अपने सब आन्तरिक रहस्य प्रकट कर देगा।... राजयोग हमें यही शिक्षा देता है। इसमें जितने उपदेश हैं, उन सबका उद्देश्य प्रथमतः मन की एकाग्रता का साधन है, इसके बाद - उसके गहनतम प्रदेश में कितने प्रकार के भिन्न-भिन्न कार्य हो रहे हैं, उनका ज्ञान प्राप्त करना और तत्पश्चात् उनसे सामान्य सत्यों को निकालकर उनसे अपने एक सिद्धान्त पर पहुँचना है।\endnote{ १/३९-४१;} 

मनुष्य यदि स्वयं को कामुक बोध करता है, तो उसका सारा धर्मभाव चला जाता है; चरित्र-बल और मानसिक तेज नष्ट हो जाता है। इसीलिए संसार में जिन सम्प्रदायों में बड़े-बड़े धर्मवीर पैदा हुए हैं, उन सभी ने ब्रह्मचर्य पर विशेष जोर दिया है। इसीलिए विवाह-त्यागी संन्यासी-दल की उत्पत्ति हुई। इस ब्रह्मचर्य का पूर्ण रूप से - तन-मन-वचन से - पालन नितान्त आवश्यक है। ब्रह्मचर्य के बिना राजयोग की साधना बड़े खतरे की है, क्योंकि उससे विषम मानसिक विकार पैदा हो सकता है। यदि कोई राजयोग का अभ्यास करे और साथ ही अपवित्र जीवन बिताये, तो वह योगी होने की आशा भला कैसे कर सकता है?\endnote{ १/८२;}


\section*{भक्तियोग - प्रेम का मार्ग}

\addsectiontoTOC{भक्तियोग - प्रेम का मार्ग}

भक्ति सभी धर्मों में है - कहीं वह ईश्वर के प्रति भक्ति है, तो कहीं महापुरुषों के प्रति। सर्वत्र इस भक्ति-रूप उपासना का सर्वोपरि प्रभाव दीख पड़ता है। ज्ञान की अपेक्षा भक्ति-लाभ सहज है। ज्ञान-लाभ के लिए कठिन अभ्यास और अनुकूल परिस्थितियों की जरूरत होती है। शरीर सर्वथा स्वस्थ तथा नीरोग न हो और मन सर्वथा विषयों से अनासक्त न हो, तो योग का अभ्यास नहीं किया जा सकता, परन्तु भक्ति की साधना लोग सभी अवस्थाओं के बड़ी सरलता के साथ कर सकते हैं।... 

पुत्र-प्राप्ति, धनी होने या स्वर्ग-लाभ के लिए की जानेवाली ईश्वर की उपासना को हम भक्ति नहीं कह सकते, यहाँ तक कि नरक की पीड़ा से छूटने के लिए भी की गयी उपासना को भी हम भक्ति नहीं कह सकते। भय या लोभ से कभी भक्ति की उत्पत्ति नहीं हो सकती। वे ही सच्चे भक्त हैं, जो कह सकते हैं - “हे जगदीश्वर! मैं धन, जन, परम सुन्दरी स्त्री या विद्वत्ता - कुछ भी नहीं चाहता। हे प्रभो! मैं हर जन्म में केवल आपकी अहेतुकी भक्ति चाहता हूँ।”\endnote{ ५/२४८, २५४;} 

भारतीय भक्ति पाश्चात्य देशों की भक्ति के समान नहीं है। भक्ति के बारे में हमारी मुख्य धारणा यह है कि उसमें भय का नामो-निशान तक नहीं रहता - रहता है केवल भगवान के प्रति प्रेम।... भक्ति की बातें हमारी प्राचीनतम उपनिषदों तक में विद्यमान हैं, जो ईसाई बाइबिल से भी काफी अधिक प्राचीन हैं। वैदिक संहिताओं में भी भक्ति के बीज देखने में आते हैं। फिर ‘भक्ति’ शब्द भी कोई पाश्चात्य शब्द नहीं है। वेद-मन्त्र में ‘श्रद्धा’ शब्द का जो उल्लेख है, उसी से क्रमशः भक्तिवाद का उद्भव हुआ था।\endnote{ १०/३८५;} 

निम्नतम रूप में प्रेम की अभिव्यक्ति को ‘शान्त’ कहते हैं। जब ईश्वर की उपासना करते समय व्यक्ति के हृदय में प्रेमाग्नि प्रज्वलित नहीं रहती, जब वह प्रेम से उन्मत्त होकर अपनी सुध-बुध नहीं खो बैठता, जब उसका प्रेम बाह्य क्रिया-कलापों और अनुष्ठानों से थोड़ा-ही उन्नत एक साधारण-सा प्रेम रहता है, जब उसकी उपासना में प्रबल प्रेम का उन्माद नहीं रहता, तब वह उपासना ‘शान्त-भक्ति’ कहलाती है। संसार में कुछ ऐसे लोग होते हैं, जो साधन-पथ पर धीरे-धीरे अग्रसर होना पसन्द करते हैं और कुछ आँधी के समान जोर से चलना। शान्त भक्त धीर, प्रशान्त तथा नम्र होता है। 

इससे कुछ ऊँची अवस्था है - ‘दास्य’। इस अवस्था में व्यक्ति अपने को ईश्वर का दास समझता है। विश्वासी सेवक की अपने स्वामी के प्रति अनन्य भक्ति ही उसका आदर्श है। 

इसके बाद है - ‘सख्य’ प्रेम। सख्य-प्रेम का साधक भगवान से कहता है, ‘तुम मेरे प्रिय सखा हो।’ जैसे एक व्यक्ति अपने मित्र के सम्मुख अपना हृदय खोल देता है और यह जानता है कि उसका मित्र उसके अवगुणों पर कभी ध्यान न देगा, वरन् उसकी सदा सहायता ही करेगा - उन दोनों में जिस प्रकार समानता का एक भाव रहता है, वैसे ही ‘सख्य-प्रेम’ के साधक और उसके सखा भगवान के बीच भी मानो एक प्रकार की समानता का भाव रहता है। इस प्रकार भगवान हमारा अन्तरंग मित्र हो जाता है, जिसको हम अपने जीवन की सारी बातें दिल खोलकर बता सकते हैं।... 

इसके बाद है ‘वात्सल्य-प्रेम’। इसमें भगवान से अपने पिता के रूप में नहीं, बल्कि अपनी सन्तान के रूप में प्रेम करना पड़ता है। सम्भव है, यह कुछ अजीब-सा लगे, पर उसका उद्देश्य है - अपनी ईश्वर-विषयक धारणा से ऐश्वर्य के सारे भावों को दूर कर देना। ऐश्वर्य की भावना के साथ ही भय आता है। प्रेम में भय के लिए कोई स्थान नहीं है।... प्रेमी कहता है कि भगवान को महामहिम, ऐश्वर्यशाली, जगन्नाथ या देवाधिदेव के रूप में सोचने की मेरी इच्छा ही नहीं होती। भगवान के सम्बन्धित यह जो भयोत्पादक ऐश्वर्य की भावना है, उसी को दूर करने हेतु भगवान को वह अपनी सन्तान के रूप में प्यार करता है। माता-पिता अपने बच्चे से भयभीत नहीं होते, उसके प्रति उनकी श्रद्धा नहीं होती। वे उस बच्चे से कुछ याचना नहीं करते। बच्चा तो सदा पानेवाला ही होता है और उसके लिए वे लोग सौ बार भी मरने को तैयार रहते हैं। अपने एक बच्चे के लिए लोग हजार जीवन भी न्यौछावर करने को प्रस्तुत रहते हैं। बस, इसी प्रकार भगवान के साथ वात्सल्य-भाव से प्रेम किया जाता है।... 

प्रेम का यह दिव्य रूप एक अन्य मानवीय भाव में व्यक्त होता है। उसे ‘मधुर’ कहते हैं और वही सब प्रकार के प्रेमों में श्रेष्ठ है। संसार में प्रेम की जो सर्वोच्च अभिव्यक्ति है, वही इसकी नींव है और मानवीय प्रेमों में यही सबसे प्रबल है। पुरुष और स्त्री के बीच जो प्रेम रहता है, उसके समान और कौन-सा प्रेम है, जो मनुष्य की सारी प्रकृति को बिल्कुल उलट-पलट दे, जो उसके हर परमाणु में संचरित होकर उसको पागल बना दे, उसके अपने स्वभाव को ही भुला दे; और उसे देवता या दैत्य बना दे? दैवी प्रेम के इस मधुर-भाव में भगवान का चिन्तन पति के रूप में किया जाता है - ऐसा विचार कि हम सभी स्त्रियाँ हैं, इस संसार में और कोई पुरुष नहीं, एक ही पुरुष है - हमारा प्रेमास्पद भगवान। पुरुष स्त्री के प्रति और स्त्री पुरुष के प्रति जो प्रेम प्रदर्शित करती है, वही प्रेम भगवान को देना होगा।... 

परन्तु प्रेमी-भक्त यहाँ भी नहीं रुकता, उसके लिए पति और पत्नी की प्रेमोन्मत्तता भी पर्याप्त नहीं। ऐसे भक्त अवैध-प्रेम का भाव ग्रहण करते हैं, क्योंकि उसमें बड़ी प्रबलता होती है। परन्तु देखो, उसकी अवैधता उनका लक्ष्य नहीं है। इस प्रेम का स्वभाव ही ऐसा है कि उसे जितनी बाधा मिलती है, वह उतना ही उग्र रूप धारण करता है। पति-पत्नी का प्रेम अबाध रहता है - उसमें किसी प्रकार की विघ्न-बाधा नहीं आती; इसीलिए भक्त कल्पना करता है, मानो कोई स्त्री पर-पुरुष में आसक्त है और माता, पिता या पति उसके इस प्रेम का विरोध कर रहे हैं। इस प्रेम के मार्ग में जितनी ही बाधाएँ आती हैं, वह उतना ही प्रबल रूप धारण करता जाता है। श्रीकृष्ण वृन्दावन के कुंजों में किस प्रकार लीला करते थे, किस प्रकार सब लोग उन्मत्त होकर उनसे प्रेम करते थे, किस प्रकार उनकी बाँसुरी की मधुर तान सुनते ही चिरधन्य गोपियाँ सब कुछ भूलकर, संसार तथा इसके सारे बन्धनों को भूलकर, दुनिया के सारे कर्तव्य तथा सुख-दुःख को बिसराकर, उन्मत्त-सी उनसे मिलने के लिए दौड़ पड़ती थीं - यह सब मानवी भाषा द्वारा व्यक्त नहीं किया जा सकता। हे मानव, तुम दैवी प्रेम की बातें तो करते हो, पर साथ ही इस संसार की असार चीजों में भी मन लगाये रहते हो - क्या तुम सच्चे हो? ‘जहाँ राम हैं, वहाँ काम नहीं, और जहाँ काम है वहाँ राम नहीं।’ ये दोनों कभी एक साथ नहीं रह सकते - प्रकाश और अन्धकार क्या कभी एक साथ रहे हैं?\endnote{ ४/६८-७३;} 

सांसारिक प्रेमी जिस प्रकार अपने प्रियतम से प्रेम करते हैं, उसी प्रकार हमें भी ईश्वर से प्रेम करना होगा। कृष्ण स्वयं ईश्वर थे, राधा उनके प्रेम में उन्मत्त थीं।... परन्तु इस अपूर्व प्रेम के तत्त्व को कितने लोग समझते हैं? बहुत-से ऐसे लोग हैं, जिनका हृदय पाप से परिपूर्ण है, वे नहीं जानते कि पवित्रता या नैतिकता किसे कहते हैं। वे क्या इन तत्त्वों को समझ सकते हैं? वे कैसे भी इन तत्त्वों को समझ ही नहीं सकते। जब मन से सारे सांसारिक वासनापूर्ण विचार दूर हो जाते हैं और जब मन निर्मल नैतिक तथा आध्यात्मिक भाव-जगत् में स्थित हो जाता है, उस समय व्यक्ति अशिक्षित होने पर भी शास्त्र की अति जटिल समस्याओं के रहस्य को समझने में समर्थ होता है। परन्तु इस प्रकार के मनुष्य संसार में कितने हैं या हो सकते है?\endnote{ ५/२५५;} 

भक्ति तो तुम्हारे भीतर ही है - केवल उसके ऊपर काम-कांचन का एक आवरणसा पड़ा हुआ है। उसको हटाते ही भीतर की वह भक्ति स्वयमेव प्रकट हो जाएगी।\endnote{ १०/३७१;} 

प्रेम बड़ी ही उच्च वस्तु है, परन्तु उसके निरर्थक भावुकता में परिणत होकर नष्ट हो जाने का भय रहता है।\endnote{ २/३२५;} 

भक्तियोग का एक बड़ा लाभ यह है कि वह हमारे महान् दिव्य लक्ष्य की प्राप्ति का सबसे सरल और स्वाभाविक मार्ग है। परन्तु साथ ही उसमें एक विशेष आशंका यह है कि अपनी निम्न अवस्था में वह मनुष्य को बहुधा भयानक मतान्ध और कट्टर बना देता है।\endnote{ ४/५;} 

\newpage

चैतन्य महाप्रभु महात्यागी थे, उनका कामिनी और इन्द्रिय -भोग से कोई नाता न था। पर बाद में, उनके अनुयाइयों ने स्त्रियों को भी अपने सम्प्रदाय में सम्मिलित कर लिया, उनके नाम पर अन्धाधुन्ध उनसे मेलजोल रखा और इस प्रकार उनके महान् आदर्शों को मिट्टी में मिला दिया। प्रेम का जो आदर्श चैतन्यदेव ने अपने जीवन में प्रकट किया था, उसमें तो अहं-भाव तथा वासना का लेश तक नहीं था; वह काम-विहीन प्रेम सर्वसाधारण के लिए भला कैसे सुलभ हो सकता था? परन्तु उनके परवर्ती वैष्णव गुरुओं ने, चैतन्य महाप्रभु के जीवन के त्याग तथा निष्कामता के आदर्शों पर ध्यान न देकर, जन-साधारण में उनके प्रेम के आदर्श का ही प्रचार किया। परिणाम यह हुआ कि जनता उस स्वर्गीय प्रेम के तत्त्व को समझ नहीं सकी और वह प्रेम स्त्री-पुरुष के निकृष्टतम प्रकार के प्रेम में परिणत हो गया।... 

जब तक हृदय में जरा भी वासना है, तब तक यह प्रेम सम्भव नहीं। केवल महात्यागी, निःस्पृह और संन्यस्त व्यक्ति, जो मानवों में अतिमानव हैं, केवल उन्हें ही इस स्वर्गीय प्रेम में अधिकार है। यदि वह प्रेम का उच्चतम आदर्श सर्वसाधारण में प्रचलित कर दिया जाए, तो वह अज्ञात भाव से मानव के हृदय में प्रबल सांसारिक प्रेम को ही उद्दीप्त करेगा - क्योंकि साधारण व्यक्ति ईश्वर का प्रिया-भाव से ध्यान करते-करते अधिकांश समय अपनी प्रिया के ध्यान में ही खोया रहेगा और इसका परिणाम जो होगा - वह स्पष्ट है।... 

मधुर भाव को छोड़ क्या अन्य कोई भाव या सम्बन्ध नहीं है, जिनके द्वारा उपासना की जा सके? अन्य चारों मार्गों का अनुसरण करके पूरे हृदय से ईश्वर का नाम-स्मरण करो। पहले हृदय के द्वार तो खुलने दो, बाकी सब अपने आप ही आ जाएगा। पर यह बात अच्छी तरह से समझ लो कि जब तक काम-वासना है, तब तक उस प्रेम का आविर्भाव नहीं होगा। क्यों न पहले अपनी भोगों की लालसा का ही त्याग कर डालो!... 

यह मत समझना कि कीर्तन का अर्थ केवल नाचना ही है। कीर्तन का अर्थ है - चाहे जैसे भी हो सके, ईश्वर का गुणगान करना। वैष्णवों का भावावेश में आकर नाचना और मस्त हो जाना - निस्सन्देह बड़ा मनोहारी है, पर उसमें एक खतरा भी है, जिससे स्वयं को बचाना होगा। वह खतरा है - उसकी प्रतिक्रिया में। एक ओर तो भावनाएँ सर्वोच्च शिखर तक पहुँच जाती हैं, आँखों से अश्रुप्रवाह होने लगता है और फिर शरीर मस्ती में झूमने लगता है, परन्तु दूसरी ओर संकीर्तन ज्योंही समाप्त होता है, त्योंही भावनाओं की इन प्रबल लहरों का उतनी ही शीघ्रता से पतन भी होता है। समुद्र में लहरें जितनी ऊपर उठतीं है, उतनी ही नीचे गिरती हैं। उस अवस्था में प्रतिक्रिया का आघात सह पाना आसान नहीं है। जब तक सद्-असद्-विवेक-बुद्धि का विकास नहीं हो जाता, व्यक्ति के ऐसी अवस्था में वासना आदि दुर्बलताओं का शिकार बन जाने की सम्भावना रहती है।... 

\newpage

ईश्वर की ज्ञान-मिश्रित भक्ति से आराधना करो। भक्ति के साथ विवेक का लोप न हो। इसके अतिरिक्त महाप्रभु से उनकी सहृदयता, समस्त प्राणियों के लिए उनकी प्रेममय दया, ईश्वर के लिए उनका उत्कट प्रेम सीखो; और उनकी निस्पृहता को अपने जीवन का लक्ष्य बनाओ।\endnote{ ८/२७९-८१;}


\section*{कर्मयोग - अनासक्ति का मार्ग}

\addsectiontoTOC{कर्मयोग - अनासक्ति का मार्ग}

हमारा एकमात्र सच्चा कर्तव्य है - अनासक्त होकर एक स्वाधीन व्यक्ति की तरह कार्य करना तथा अपने समस्त कर्म प्रभु परमेश्वर को समर्पित कर देना। हमारे समस्त कर्तव्य उन्हीं के तो हैं।\endnote{ ३/७६;} 

कोई कार्य, कोई विचार, जो फल उत्पन्न करता है; वह ‘कर्म’ कहलाता है।\endnote{ ३/६८;} 

अतः ठीक ढंग से कर्म करने के लिए आवश्यक है कि सर्वप्रथम हम आसक्ति का भाव त्याग दें। दूसरी बात यह है कि हमें स्वयं झंझट में उलझ नहीं जाना चाहिए, बल्कि अपने को एक साक्षी के समान रखना और अपना काम करते रहना चाहिए। मेरे गुरुदेव कहा करते थे - “अपने बच्चों के प्रति वही भावना रखो, जो एक दासी की होती है।” वह तुम्हारे बच्चे को गोद में लेती है, उसे खिलाती है और उसे इस प्रकार प्यार करती है, मानो वह उसी का बच्चा हो। परन्तु तुम ज्योंही उसे काम से अलग कर देते हो, त्योंही वह अपना बोरिया-बिस्तर समेटकर घर छोड़ने को तैयार हो जाती है। बच्चों के प्रति उसका जो इतना प्रेम था, उसे वह बिल्कुल भूल जाती है। एक साधारण दासी को तुम्हारे बच्चों को छोड़कर दूसरे के बच्चों को लेने में जरा भी दुःख नहीं होगा। तुम भी अपने बच्चों के प्रति यही भाव धारण करो। तुम उनकी दासी हो - और यदि तुम्हारा ईश्वर में विश्वास है, तो विश्वास करो कि ये सब चीजें, जिन्हें तुम अपनी समझते हो, वास्तव में ईश्वर की हैं।\endnote{ ३/६३;} 

हमें भलाई क्यों करना चाहिए? इसलिए कि भलाई करना ही अच्छा है। कर्मयोगी का कहना है कि जो स्वर्ग प्राप्त करने की इच्छा से भी सत्कर्म करता है, वह अपने को बन्धन में डाल लेता है। किसी कार्य में यदि थोड़ी भी स्वार्थपरता रहे, तो वह हमें मुक्त करने की जगह, हमारे पैरों में एक बेड़ी और डाल देता है। 

अतः एकमात्र उपाय है - समस्त कर्मफलों का त्याग कर देना, अनासक्त हो जाना।... बिना किसी स्वार्थ के किया हुआ प्रत्येक सत्कार्य हमारे पैरों में एक और बेड़ी डालने के बदले पहले की ही एक बेड़ी को तोड़ देता है। बिना किसी बदले की आशा से संसार में भेजा गया प्रत्येक शुभ विचार संचित होता जाएगा - वह हमारे पैरों की बेड़ियों में से एक को काट देगा और हमें अधिकाधिक पवित्र बनाता जाएगा; और अन्ततः हम पवित्रतम बन जाएँगे।\endnote{ ३/८८-८९;} 

जिस कर्म के द्वारा हमारी आध्यात्मिकता का विकास होता है, केवल वही कर्म है; और जो कर्म हममें भौतिकता को बढ़ावा देता है, वही अकर्म है।\endnote{ ४/३१७;} 

इस मानव जीवन में व्यक्ति को सर्वदा किसी-न-किसी प्रकार का कर्म करते रहना पड़ता है। जब मनुष्य कर्म करने को बाध्य है, तो कर्मयोग हमें इस प्रकार कर्म करने की सलाह देता है, जिससे आत्मा की अनुभूति के माध्यम से हमें मुक्ति प्राप्त हो सके।\endnote{ ६/१५३;} 

कर्म के द्वारा मुक्ति-लाभ करना हो, तो अपने को कर्म में लगाओ, परन्तु बिना किसी कामना के - बिना किसी फल की आकांक्षा के कर्म करो। इस प्रकार के कर्मों के द्वारा ज्ञान-लाभ होता है और इस ज्ञान के द्वारा मुक्ति होती है। ज्ञान-प्राप्ति के पूर्व कर्मत्याग करने से दुःख ही होता है। आत्मा के लिए कर्म करने पर किसी प्रकार का कर्मजनित बन्धन नहीं होता। \textbf{कर्म से सुख की आकांक्षा भी मत करो और ऐसा भय भी मत रखो कि कर्म करने से कष्ट होगा। }... सारे कर्म भगवान को अर्पित कर दो। संसार में रहो, परन्तु संसारी मत बनो - जैसे कमल की जड़ कीचड़ में रहती है, परन्तु वह स्वयं सर्वदा शुद्ध रहती है। लोग तुम्हारे प्रति चाहे जैसा व्यवहार करें, पर तुम सभी से प्रेम करते रहो।\endnote{ ७/७४;} 

यदि तुम्हारी दृष्टि कर्म के फलों की ओर न रहे और यदि तुममें सभी तरह की कामनाओं तथा वासनाओं के परे जाने के लिए प्रबल आग्रह हो, तो ये सारे सत्कर्म तुम्हारे कर्म-बन्धन को काट डालने में सहायता करेंगे। यह सोचना मूर्खता है कि ऐसे कर्म से बन्धन आएगा। दूसरों के लिए किए हुए ऐसे कर्म ही कर्म-बन्धनों की जड़ को काटने के एकमात्र उपाय हैं। \textbf{नान्यः पन्था विद्यतेऽयनाय } - इसके अतिरिक्त कोई दूसरा मार्ग नहीं है।\endnote{ ६/१२२;} 

श्रीकृष्ण का चित्रण वैसा ही होना चाहिए, जैसे वे थे - गीता के मूर्तस्वरूप और उनके पूरे व्यक्तित्व से गीता का केन्द्रीय भाव अभिव्यक्त होना चाहिए।... उनके शरीर का प्रत्येक अंग सक्रिय है और इसके बावजूद मुख पर नील गगन की गम्भीर शान्ति एवं प्रसन्नता व्याप्त है। यही तो गीता (४. १८) का मूल तत्त्व है - शरीर, मन तथा आत्मा को ईश्वर के चरणों में लगाये रखकर, सभी परिस्थितियों में शान्त और स्थिर बने रहना -

\begin{verse}
कर्मणि-अकर्म यः पश्येद्-अकर्मणि च कर्म यः।\\स बुद्धिमान्-मनुष्येषु स युक्तः कृत्स्न-कर्मकृत्॥ 
\end{verse}

कर्म करते हुए भी जिसका मन शान्त है; और जब कोई बाह्य चेष्टा नहीं हो रही है, तब भी जिसमें सतत ब्रह्म-चिन्तन रूपी महान् कर्म की धारा बह रही है - वही मनुष्यों में बुद्धिमान् है, वही योगी है और वही कर्म-कुशल भी है।... 

फल की चिन्ता त्यागकर, मन तथा आत्मा को प्रभु के चरण-कमलों में लगाकर कर्म करना - अनन्त कर्म करना - गीता के निष्काम कर्मयोग का यह सन्देश प्रत्येक व्यक्ति तक पहुँचना चाहिए। 

तांत्रिक साधना में फिसलने का बहुत ज्यादा डर है।... अब इसे त्यागकर आगे बढ़ना चाहिए। वेदों का अध्ययन होना चाहिए। चारों योगों के एक मधुर समन्वय का अभ्यास होना चाहिए और पूर्ण ब्रह्मचर्य का व्रत ग्रहण करना चाहिए। 

सत् और असत् का विवेक, वैराग्य तथा भक्ति, कर्म तथा ध्यानयोग का अभ्यास - और इसके साथ-साथ स्त्रियों के प्रति आदर का भाव होना चाहिए। 

स्त्रियाँ आदिशक्ति जगदम्बा की प्रतीक हैं।... जिस दिन से हम जगदम्बा की सच्ची पूजा करने लगेंगे और हर व्यक्ति माँ की वेदी पर अपना बलिदान दे देगा, उसी दिन से भारत का सच्ची भलाई होने लगेगी और वह समृद्धि के मार्ग पर अग्रसर होने लगेगा।\endnote{ ८/२३८-२४२;}


\section*{धर्म-समन्वय}

\addsectiontoTOC{धर्म-समन्वय}

संसार के लिए यह बड़े ही दुर्भाग्य की बात होगी, यदि सभी मनुष्य एक ही धर्म, उपासना की एक ही सार्वभौमिक पद्धति और नैतिकता के एक ही आदर्श को स्वीकार कर लें। इससे सभी धार्मिक और आध्यात्मिक उन्नति को मृत्यु-जैसा आघात पहुँचेगा। अतः भले या बुरे उपायों द्वारा दूसरों को अपने धर्म और सत्य के उच्चतम आदर्श पर लाने की चेष्टा करने की जगह, हमें चाहिए कि हम उनकी वे सब बाधाएँ हटा देने का प्रयत्न करें, जो उनके अपने धर्म के उच्चतम आदर्श के अनुसार विकास में बाधा डालती हैं, और इस तरह उन लोगों की चेष्टाएँ विफल कर दें, जो एक सार्वजनीन धर्म की स्थापना के प्रयास में लगे हुए हैं।\endnote{ ३/१६९;} 

हम लोग सब धर्मों के प्रति, न केवल सहिष्णुता में विश्वास करते हैं, अपितु सभी धर्मों को सच्चा मानकर स्वीकार करते हैं। मुझे एक ऐसे देश का व्यक्ति होने का अभिमान है, जिसने पृथ्वी के सभी धर्मों और सभी देशों के उत्पीड़ितों और शरणार्थियों को आश्रय दिया है।\endnote{ १/३;} 

दूसरे धर्म या मत के लिए हमें केवल सहनशीलता नहीं दिखानी है, बल्कि प्रत्यक्ष रूप से उन्हें स्वीकार करना होगा; और यही सत्य सब धर्मों की नींव है।\endnote{ ३/३३;} 

किसी की निन्दा मत करो। किसी की सहायता कर सकते हो, तो करो; नहीं कर सकते, तो हाथ-पर-हाथ रखकर चुपचाप बैठे रहो; उन्हें आशीर्वाद दो, अपने रास्ते जाने दो। गाली देने अथवा निन्दा करने से कोई उन्नति नहीं होती। इस प्रकार से कभी कोई कार्य नहीं होता। हम अपनी शक्ति दूसरे की निन्दा करने में लगाते हैं। आलोचना और निन्दा अपनी शक्ति खर्च करने के निरर्थक उपाय हैं, क्योंकि अन्त में हम देखते हैं कि सभी लोग एक ही वस्तु देख रहे हैं, या फिर कमो-बेश उसी आदर्श की ओर पहुँच रहे हैं; और हम लोगों में जो भेद दीख पड़ते हैं, वे केवल अभिव्यक्ति के भेद हैं।\endnote{ ८/११;} 

वस्तुतः धर्ममतों की अनेकता लाभदायक है, क्योंकि वे सभी मनुष्यों को धार्मिक जीवन व्यतीत करने की प्रेरणा ही देते हैं; और इस कारण सभी अच्छे हैं। जितने ही अधिक धर्म-सम्प्रदाय होंगे, मनुष्य की भगवद्-भावना को सफलतापूर्वक जाग्रत करने के उतने ही अधिक सुयोग मिलेंगे।\endnote{ ३/१७०;} 

मेरे गुरुदेव का कहना था - धर्म एक ही है; सभी पैगम्बरों की शिक्षा एक ही होती है; पर उस तत्त्व को प्रकट करने हेतु सबको उसे कोई-न-कोई आकार देना पड़ा। इसलिए उन्होंने उसके पुराने आकार को त्यागकर उसे नये आकार में हमारे सामने रखा है।\endnote{ ७/२५;} 

\newpage

सभी धर्मों में समाधि या तुरीयावस्था एक है। देहज्ञान के पार जाने पर हिन्दू, मुसलमान, ईसाई, बौद्ध, यहाँ तक कि जो लोग किसी प्रकार का धर्ममत स्वीकार नहीं करते, सभी को ठीक एक ही तरह की अनुभूति होती है।\endnote{ ७/५३;} 

\vskip 2pt

ईसाई को हिन्दू या बौद्ध नहीं हो जाना चाहिए; और न हिन्दू अथवा बौद्ध को ईसाई ही। पर हाँ, प्रत्येक को चाहिए कि वह दूसरों के सार-भाग को आत्मसात् करके पुष्टिलाभ करे और अपने वैशिष्ट्य की रक्षा करते हुये अपनी निजी वृद्धि के नियम के अनुसार विकसित हो।... 

\vskip 2pt

साधुता, पवित्रता और सेवा किसी सम्प्रदाय-विशेष की सम्पत्ति नहीं है; और प्रत्येक धर्म ने श्रेष्ठ तथा अतिशय उन्नत चरित्र के नर-नारियों को जन्म दिया है। इन प्रत्यक्ष प्रमाणों के बावजूद भी यदि कोई ऐसा स्वप्न देखे कि अन्य सारे धर्म नष्ट हो जाएँगे और केवल उसी का धर्म जीवित रहेगा, तो मैं अपने हृदय के अन्तस्तल से उस पर दया करता हूँ और उसे स्पष्ट बता देना चाहता हूँ कि सारे प्रतिरोधों के बावजूद, शीघ्र ही प्रत्येक धर्म की पताका पर यह लिखा होगा - ‘संघर्ष नहीं, सहयोग’; ‘पर-भाव-विनाश नहीं, पर-भाव-ग्रहण’; ‘मतभेद और कलह नहीं, समन्वय और शान्ति’!\endnote{ १/२७;} 

\vskip 2pt

यदि कभी कोई सार्वभौमिक धर्म होना है, तो वह किसी देश या काल से सीमाबद्ध नहीं होगा। वह उस असीम ईश्वर के समान असीम होगा, जिसका कि वह उपदेश देता है; जिसका सूर्य कृष्ण और ईसा के अनुयायियों पर, सन्तों और पापियों को समान रूप से आलोक-वर्षा करेगा; जो न तो हिन्दू होगा, न बौद्ध, न ईसाई और न इस्लाम, बल्कि इन सबकी समष्टि होगा; और इसके बावजूद जिसमें विकास के लिए अनन्त अवकाश होगा; जो इतना उदार होगा कि पशुओं के स्तर से किंचित् उन्नत निम्नतम घृणित जंगली मनुष्य से लेकर अपने हृदय और मस्तिष्क के गुणों के कारण मानवता से इतना ऊपर उठ गए उच्चतम मनुष्य तक को, जिसके प्रति सारा समाज श्रद्धानत हो जाता है और लोग जिसके मनुष्य होने में सन्देह करते हैं, उन सबको अपनी अनन्त भुजाओं में समेट सकेगा और सबको स्थान दे सकेगा। वह एक ऐसा धर्म होगा, जिसकी नीति में उत्पीड़न या असहिष्णुता के लिए कोई स्थान न होगा, जो हर स्त्री-पुरुष में दिव्यता को स्वीकार करेगा और जिसका सारी शक्ति तथा सामर्थ्य मानवता को उसकी सच्ची दिव्य प्रकृति का साक्षात्कार करने में सहायता देने में ही केन्द्रित होगा। 

\vskip 2pt

आप ऐसा ही धर्म सामने रखिए, और सारे राष्ट्र आपके अनुयायी बन जाएँगे।\endnote{ १/२०-२१;} 

\vskip 2pt


\section*{धार्मिक व्यक्ति के लक्षण}

\addsectiontoTOC{धार्मिक व्यक्ति के लक्षण}

धर्म-पथ में अपनी उन्नति का पहला लक्षण यह देखोगे कि तुम दिन-पर-दिन बड़े प्रफुल्ल होते जा रहे हो। यदि कोई व्यक्ति विषादयुक्त दिखे, तो वह अपच के कारण भले ही हो, पर यह धर्म का लक्षण नहीं हो सकता।... पाप ही दुःखों का कारण है, उसका अन्य कोई कारण नहीं।... उतरा हुआ चेहरा लेकर बाहर मत जाना! किसी दिन ऐसा होने पर दरवाजा बन्द करके समय बिता देना। संसार में इस बीमारी को संक्रमित करने का तुम्हें क्या अधिकार है?\endnote{ १/१८०;} 

सच्चे आध्यात्मिक व्यक्ति सर्वत्र उदार होते हैं। ‘उसका’ प्रेम उन्हें विवश कर देता है। परन्तु धर्म ही जिनका व्यापार है, वे लोग संसार की स्पर्धा, उसकी लड़ाकू तथा स्वार्थी चाल को धर्म के क्षेत्र में ले आते हैं और इस कारण संकीर्ण तथा धूर्त होने पर विवश हो जाते हैं।\endnote{ ३/३४७;} 

हम संसार में पाप तथा दुर्बलता क्यों देखते हैं?... हम स्वयं जैसे हैं, वैसा ही जगत् को भी देखते हैं। मान लो, कमरे में मेज पर सोने की एक थैली रखी है और एक छोटा बच्चा वहाँ खेल रहा है। इतने में एक चोर वहाँ आता है और उस थैली को चुरा लेता है। तो क्या बच्चा यह समझेगा कि चोरी हो गयी? हमारे भीतर जो है, वही हम बाहर भी देखते हैं। बच्चे के मन में चोर नहीं है, अतः वह बाहर भी चोर नहीं देखता। सब प्रकार के ज्ञान के विषय में ऐसा ही है। संसार के पाप तथा दुर्बलताओं की बात मन में न लाओ, बल्कि रोओ कि तुम्हें जगत् में अब भी पाप दिखता है; रोओ कि तुम्हें अब भी सर्वत्र अत्याचार दिखता है; और यदि तुम जगत् का उपकार करना चाहते हो, तो इस पर दोषारोपण करना छोड़ दो। उसे और भी दुर्बल मत करो। आखिर ये सब पाप, दुःख आदि हैं क्या? ये सब दुर्बलता के ही तो फल हैं। लोग बचपन से ही शिक्षा पाते हैं कि वे दुर्बल हैं, पापी हैं। इस प्रकार की शिक्षा से संसार दिन-पर-दिन और भी दुर्बल होता जा रहा है। उनको सिखाओ कि वे सब उसी अमृत की सन्तान हैं - यहाँ तक कि जिनके भीतर आत्मा का प्रकाश अत्यन्त क्षीण है, उन्हें भी यही शिक्षा दो। बचपन से ही उनके मस्तिष्क में इस प्रकार के विचार प्रविष्ट हो जाए, जिनसे उनकी यथार्थ सहायता हो सके, जो उनको सबल बना दें, जिनसे उनका कुछ यथार्थ हित हो। दुर्बलता और विषादकारक विचार उनके मस्तिष्क में प्रवेश ही न करें। सत्-चिन्तन के स्रोत में स्वयं को बहा दो; मन-ही-मन सर्वदा कहते रहो, ‘\textbf{सोऽहं सोऽहम् }- मैं ही वह हूँ, मैं ही वह हूँ।’ तुम्हारे भी मन में रात-दिन यही स्वर संगीत की भाँति झंकृत होता रहे और मृत्यु के समय भी तुम्हारे अधरों पर - \textbf{सोऽहं सोऽहम् } खेलता रहे। यही सत्य है - जगत् की अनन्त शक्ति तुम्हारे भीतर है।... आओ साहसी बनें; सत्य को जानें और उसे जीवन में परिणत करें। लक्ष्य भले ही बहुत दूर हो; पर उठो, जागो और लक्ष्य को प्राप्त किए बिना रुको मत।\endnote{ २/१९-२०;} 

क्या तुम्हें विश्व-इतिहास में पैगम्बरों की शक्ति के स्रोत का पता चला? बुद्धि में? क्या उनमें से कोई दर्शन-विषयक सुन्दर ग्रन्थ लिखकर छोड़ गया है; या न्याय के कूट तर्कों पर कोई पुस्तक लिख गया है? किसी ने ऐसा नहीं किया। वे केवल थोड़ी-सी बातें कह गए हैं। ईसा की भाँति भावना करो, तुम भी ईसा हो जाओगे; बुद्ध के समान भावना करो, तुम भी बुद्ध बन जाओगे। भावना ही जीवन है, भावना ही बल है, भावना ही तेज है - भावना के बिना, कितनी ही बुद्धि क्यों न लगाओ, ईश्वर-प्राप्ति नहीं होगी।\endnote{ ८/१८;} 

साहस दो प्रकार का होता है। एक प्रकार का साहस है - तोप के मुँह में दौड़ जाना और दूसरे प्रकार का साहस है - आध्यात्मिक विश्वास। एक बार एक दिग्विजयी सम्राट् भारतवर्ष में आया। उसके गुरु ने उसे भारतीय साधुओं से मिलने का आदेश दिया था। बहुत खोज करने के बाद उसने देखा कि एक वृद्ध साधु एक पत्थर पर बैठे हैं। सम्राट् ने उनसे कुछ देर बातचीत की उनके ज्ञान से बड़ा प्रभावित हुआ। उसने साधु को अपने साथ देश ले जाने की इच्छा व्यक्त की। साधु ने इसे अस्वीकार करते हुए कहा, “मैं इस वन में बड़े आनन्द से हूँ।” सम्राट् बोला, “मैं सारी पृथ्वी का सम्राट् हूँ; आपको असीम ऐश्वर्य और उच्च पद-मर्यादा दूँगा।” साधु बोले, “ऐश्वर्य, पद-मर्यादा आदि किसी चीज की मुझे इच्छा नहीं है।” तब सम्राट् ने कहा, “यदि आप मेरे साथ नहीं चलेंगे, तो मैं आपको मार डालूँगा।” इस पर साधु शान्तिपूर्वक हँसे और बोले, “राजन्, आज तुमने अपने जीवन में सबसे मूर्खतापूर्ण बात कही है। तुम्हारी क्या हस्ती, जो मुझे मारो? सूर्य मुझे सुखा नहीं सकता, अग्नि मुझे जला नहीं सकती, तलवार मुझे काट नहीं सकती, क्योंकि मैं तो जन्मरहित, अविनाशी, नित्य-विद्यमान, सर्वव्यापी, सर्व-शक्तिमान आत्मा हूँ।” यह आध्यात्मिक साहस है और दूसरा है सिंह या बाघ का साहस। १८५७ ई. के गदर के समय एक मुसलमान सिपाही ने एक संन्यासी महात्मा को बुरी तरह घायल कर दिया था। हिन्दू विद्रोहियों ने उस मुसलमान को पकड़ लिया और महात्मा के पास लाकर कहा, “आप कहें, तो हम इसका वध कर दें।” महात्मा ने शान्तिपूर्वक उसकी ओर देखा और कहा, “भाई, तुम वही हो, तुम वही हो - \textbf{तत्त्वमसि। }” और यह कहते-कहते उन्होंने शरीर छोड़ दिया। यही साहस का दूसरा उदाहरण हैं।\endnote{ २/१७;}


\section*{जीवन्त ईश्वर की पूजा}

\addsectiontoTOC{जीवन्त ईश्वर की पूजा}

धर्म का रहस्य मतवादों से नहीं, बल्कि आचरण से जाना जा सकता है। भले बनना तथा भलाई करना - इसी में सारा धर्म निहित है। “जो केवल ‘प्रभु प्रभु’ की रट लगाता है, वह नहीं, अपितु जो उस परम पिता के इच्छानुसार कार्य करता है” - वही धार्मिक है।\endnote{ १/३८०;} 

नीतिपरायण और साहसी बनो, अन्तःकरण पूर्णतः शुद्ध होना चाहिए। पूर्ण नीतिपरायण तथा साहसी बनो - प्राणों के लिए भी कभी न डरो। धार्मिक मत-मतान्तरों को लेकर बेकार की माथापच्ची मत करो। कायर लोग ही पापाचरण करते हैं, वीर कभी पाप नहीं करते - यहाँ तक कि वे कभी मन में भी पाप का विचार नहीं लाते।\endnote{ १/३५०-५१;} 

यदि तुम सचमुच पवित्र हो, तो तुम्हें अपवित्रता कैसे दिखायी दे सकती है? क्योंकि जो भीतर है, वही बाहर दीख पड़ता है। हमारे अन्दर यदि अपवित्रता न होती, तो हम उसे बाहर कभी देख ही न पाते। वेदान्त की यह भी एक साधना है। आशा है, हम सभी लोग जीवन में इसको व्यवहार में लाने की चेष्टा करेंगे।\endnote{ ८/३५;} 

\newpage

परोपकार ही धर्म है; परपीड़न ही पाप। शक्ति और पौरुष पुण्य है, दुर्बलता और कायरता पाप। स्वाधीनता पुण्य है; पराधीनता पाप। दूसरों से प्रेम करना पुण्य है, घृणा करना पाप। परमात्मा में और स्वयं में विश्वास पुण्य है, सन्देह करना पाप। एकता का बोध पुण्य है, अनेकता देखना पाप।\endnote{ १०/२२२;} 

सब प्रकार की नीति, शुभ तथा मंगल का मूलमंत्र ‘मैं’ नहीं, ‘तुम’ है। स्वर्ग और नरक हैं या नहीं, आत्मा है या नहीं, कोई अनश्वर सत्ता है या नहीं - इसकी कौन परवाह करता है? हमारे सामने यह संसार है और वह दुःखों से पूर्ण है। बुद्ध के समान इस संसार सागर में गोता लगाकर या तो इस संसार के दुःखों को दूर करो या इस प्रयत्न में प्राण त्याग दो। अपने को भूल जाओ, आस्तिक हो या नास्तिक, अज्ञेयवादी ही हो या वेदान्ती, ईसाई हो या मुसलमान - प्रत्येक के लिए यही प्रथम पाठ है। और जो पाठ सबको स्पष्ट है, वह है तुच्छ ‘अहं’ का उन्मूलन और वास्तविक आत्मा का विकास।\endnote{ ८/५९;} 

मैंने इतनी तपस्या करके यही सार समझा है कि हर जीव में वे ही विराजित हैं; इसके सिवा ईश्वर और कुछ भी नहीं। जीवों पर दया करनेवाला ही ईश्वर की सेवा कर रहा है।”\endnote{ ६/२१६;}

\begin{verse}
बहु रूपों में खड़े तुम्हारे आगे, और कहाँ हैं ईश?\\व्यर्थ खोज यह, जीव-प्रेम की ही सेवा पाते जगदीश।\endnote{ ९/३२५;} 
\end{verse}

तुमने पढ़ा है - \textbf{मातृदेवो भव, पितृदेवो भव } - ‘अपनी माता को ईश्वर समझो, अपने पिता को ईश्वर समझो’ - परन्तु मैं कहता हूँ - \textbf{दरिद्रदेवो भव, मूर्खदेवो भव } - गरीब निरक्षर, मूर्ख और दु;खी, इन्हें अपना ईश्वर मानो। इनकी सेवा करना ही परम धर्म समझो।\endnote{ ३/३५७;} 

हम जीवित ईश्वर की पूजा करना चाहते हैं। मैंने सम्पूर्ण जीवन ईश्वर के सिवा और कुछ नहीं देखा। तुमने भी नहीं देखा। इस कुर्सी को देखने से पहले तुम्हें ईश्वर को देखना पड़ता है, उसके बाद उसी में और उसके माध्यम से कुर्सी को देखना पड़ता है। वह दिन-रात जगत् में रहकर प्रतिक्षण - ‘मैं हूँ’ ‘मैं हूँ’ कह रहा है। जिस क्षण तुम बोलते हो - ‘मैं हूँ’, उसी क्षण तुम उस सत्ता को जान रहे हो। तुम ईश्वर को कहाँ ढूँढ़ने जाओगे, यदि तुम उसे अपने हृदय में हर प्राणी में नहीं देख पाते?\endnote{ ८/२८;} 

इस संसाररूपी नरक-कुण्ड में यदि एक दिन के लिए भी किसी व्यक्ति के चित्त में थोड़ा-सा आनन्द तथा शान्ति दिया जा सके, तो उतना ही सत्य है, आजन्म कष्ट भोगते हुए मैंने यही तो देखा है - बाकी सब कुछ व्यर्थ है।\endnote{ ८/३९१;} 

यदि भलाई चाहते हो, तो घण्टा आदि को गंगाजी में सौंपकर साक्षात् ईश्वर की - विराट् और स्वराट् की - मानव-देहधारी प्रत्येक मनुष्य की पूजा में लग जाओ। यह जगत् ईश्वर का विराट् रूप है और उसकी पूजा का अर्थ है, उसकी सेवा - वास्तव में कर्म इसी का नाम है, निरर्थक अनुष्ठानों का नहीं।... लाखों रुपये खर्च कर काशी तथा वृन्दावन के मन्दिरों के कपाट खुलते और बन्द होते हैं। कहीं ठाकुरजी वस्त्र बदल रहे हैं, तो कहीं भोजन या और कुछ कर रहे हैं, जिसका ठीक-ठीक तात्पर्य हम नहीं समझ पाते,... जबकि दूसरी ओर जीवित ठाकुर भोजन तथा विद्या के बिना मरे जा रहे है! मुम्बई के बनिया लोग खटमलों के लिए अस्पताल बनवा रहे हैं, पर मनुष्यों की ओर उनका जरा भी ध्यान नहीं है - चाहे वे मर ही क्यों न जाएँ। तुम लोगों में इस सहज बात को समझने तक की बुद्धि नहीं है, यह हमारे देश के लिए प्लेग के समान है और पूरा देश पागलों का अड्डा है।\endnote{ ३/२९९;} 

हर स्त्री, हर पुरुष और सभी को ईश्वर के ही समान देखो। तुम किसी की सहायता नहीं कर सकते, तुम्हें केवल सेवा करने का ही अधिकार है - प्रभु की सन्तानों की, यदि भाग्यवान हो तो, स्वयं प्रभु की ही सेवा करो। ईश्वर के अनुग्रह से यदि तुम उसकी किसी सन्तान की सेवा कर सके, तो धन्य हो जाओगे। अपने को बहुत बड़ा मत समझो। तुम धन्य हो, क्योंकि सेवा करने का अधिकार तुमको मिला, दूसरो को नहीं। केवल पूजा के भाव से सेवा करो। हमें निर्धनों में ईश्वर को देखना चाहिए, अपनी ही मुक्ति के लिए उनके निकट जाकर हमें उनकी पूजा करनी चाहिए। निर्धन तथा दुखी लोग हमारी मुक्ति के साधन हैं, ताकि हम रोगी, पागल, कोढ़ी, पापी आदि स्वरूपों में विचरते हुये प्रभु की सेवा करके अपना उद्धार कर सकें।\endnote{ ५/१४१;} 

सबसे पहले उस विराट् की पूजा करो, जिसे तुम अपने चारों ओर देख रहे हो - ‘उसकी’ पूजा करो। ये मनुष्य और पशु - जिन्हें हम अपने आसपास और आगेपीछे देख रहे हैं - ये ही हमारे ईश्वर हैं। इनमें सबसे पहले पूज्य हैं - हमारे अपने देशवासी। परस्पर ईर्ष्या-द्वेष करने और झगड़ने की जगह हमें उनकी पूजा करनी होगी।\endnote{ ५/१९४;} 

समस्त उपासनाओं का यही सार है कि व्यक्ति पवित्र रहे और सदैव दूसरों का भला करे। वह मनुष्य जो शिव को निर्धन, दुर्बल तथा रुग्ण व्यक्ति में भी देखता है, वही सचमुच शिव की उपासना करता है, पर यदि वह उन्हें केवल मूर्ति में ही देखता है, तो उसकी उपासना अभी नितान्त प्रारम्भिक ही है। यदि किसी मनुष्य ने किसी एक निर्धन मनुष्य की सेवा-शुश्रूषा बिना जाति-पाँति अथवा ऊँच-नीच के भेद-भाव के, यह विचार कर की है कि उसमें साक्षात् शिव विराजमान हैं, तो शिव उस मनुष्य से उस दूसरे मनुष्य की अपेक्षा अधिक प्रसन्न होंगे, जो कि उन्हें केवल मन्दिर में ही देखता है। 

एक धनी व्यक्ति का एक बगीचा था जिसमें दो माली काम करते थे। एक माली बड़ा सुस्त तथा कमजोर था, परन्तु जब कभी वह अपने मालिक को आते देखता, तो झट से उठकर खड़ा हो जाता और हाथ जोड़कर कहता, “मेरे स्वामी का मुख कैसा सुन्दर है!” और उनके सम्मुख नाचने लगता। दूसरा माली ज्यादा बातचीत नहीं करता था, उसे तो बस अपने काम से काम था। वह बड़ी मेहनत से बगीचे में तरह-तरह के फल-सब्जियाँ पैदा करके, उन्हें स्वयं अपने सिर पर रखकर मालिक के घर पहुँचाता था, यद्यपि मालिक का घर बहुत दूर था। अब इन दो मालियों में से मालिक किसको अधिक चाहेगा? बस, ठीक इसी प्रकार यह संसार एक बगीचा है, जिसके मालिक शिव हैं। यहाँ भी दो प्रकार के माली हैं - एक तो वह जो सुस्त, अकर्मण्य तथा ढोंगी है और कभी-कभी शिव के सुन्दर नेत्र, नासिका तथा अन्य अंगों की प्रशंसा करता रहता है; और दूसरा ऐसा है जो शिव की सन्तान की, समस्त दीन-दुखी प्राणियों की और उनकी सारे सृष्टि की चिन्ता रखता है। इन दो प्रकार के लोगों में से कौन शिव को अधिक प्यारा होगा? निश्चय ही, वही जो उनकी सन्तान की सेवा करता है। जो व्यक्ति अपने पिता की सेवा करना चाहता है, उसे सबसे पहले अपने भाइयों की सेवा करनी चाहिए, इसी प्रकार जो शिव की सेवा करना चाहता है, उसे पहले उनकी सन्तान की, विश्व के प्राणी-मात्र की सेवा करनी चाहिए। शास्त्रों में कहा भी गया है कि जो भगवान के दासों की सेवा करता है, वही उनका सर्वश्रेष्ठ दास है। यह बात सर्वदा ध्यान में रखनी चाहिए। 

मैं फिर कहता हूँ कि तुम्हें स्वयं शुद्ध रहना चाहिए तथा यदि कोई तुम्हारे पास सहायतार्थ आए, तो जितना तुमसे बन सके, उतनी उसकी सेवा अवश्य करनी चाहिए। यही सर्वश्रेष्ठ धर्म कहलाता है। इसी श्रेष्ठ कर्म की शक्ति से तुम्हारा चित्त शुद्ध हो जाएगा और फिर शिव, जो प्रत्येक हृदय में निवास करते हैं, प्रकट हो जाएँगे।... परन्तु इसके विपरीत यदि कोई मनुष्य स्वार्थी है, तो चाहे उसने संसार के सब मन्दिरों के ही दर्शन क्यों न किए हो, चाहे वह सारे तीर्थों में क्यों न गया हो और चाहे रंग-भभूत रमाकर अपनी शक्ल चीते-जैसी क्यों न बना ली हो, वह शिव से दूर है - बहुत दूर है।\endnote{ ५/३८-३९;} 

तुम्हीं ईश्वर के सर्वश्रेष्ठ मन्दिर हो। मैं किसी मन्दिर, किसी प्रतिमा या किसी बाइबिल की उपासना न करके तुम्हारी ही उपासना करूँगा। कुछ लोग इतना परस्पर-विरोधी विचार क्यों रखते हैं?... लोग कहते हैं, हम ठेठ प्रत्यक्षवादी है; ठीक है, परन्तु तुम्हारी उपासना करने की अपेक्षा और अधिक प्रत्यक्ष क्या हो सकता है? मैं तुम्हें देख रहा हूँ, तुम्हारा अनुभव कर रहा हूँ और जानता हूँ कि तुम ईश्वर हो।... मनुष्य-देह में स्थित मानव-आत्मा ही एकमात्र उपास्य ईश्वर है। वैसे पशु भी भगवान के मन्दिर है, पर मनुष्य ही सर्वश्रेष्ठ मन्दिर है - ताजमहल जैसा। यदि मैं उसकी उपासना नहीं कर सका, तो अन्य किसी मन्दिर से कुछ भी उपकार नहीं होगा। जिस क्षण मैं प्रत्येक मनुष्य-देह रूपी मन्दिर में विराजित ईश्वर की उपलब्धि कर सकूँगा, जिस क्षण मैं प्रत्येक मनुष्य के सम्मुख भक्तिभाव से खड़ा हो सकूँगा और वास्तव में उसमें ईश्वर को देख सकूँगा, जिस क्षण मेरे अन्दर यह भाव आ जाएगा, उसी क्षण मैं सम्पूर्ण बन्धनों से मुक्त हो जाऊँगा।\endnote{ ८/२९-३०;} 

{\stretchpara दूसरा प्रश्न यह है कि आत्मा या ईश्वर की प्राप्ति के बाद क्या होता है?... धर्म की इस प्रत्यक्षानुभूति से जगत् का पूरा उपकार होता है। लोगों को भय होता है कि जब वे यह अवस्था प्राप्त कर लेंगे, जब उन्हें ज्ञान हो जाएगा कि सभी एक है, तब उनके प्रेम का स्रोत सूख जाएगा, जीवन में जो कुछ मूल्यवान है, वह सब चला जाएगा, इस जीवन में और पर-जीवन में जो कुछ उन्हें प्रिय था, उसमें से कुछ भी न बचा रहेगा। पर लोग एक बार भी यह सोचकर नहीं देखते कि जो लोग अपने\par}\newpage\noindent सुख की चिन्ता की ओर से उदासीन हो गए हैं, वे ही जगत् में सर्वश्रेष्ठ कर्मी हुए है। मनुष्य तभी वास्तव में प्रेम करता है, जब वह देखता है कि उसके प्रेम का पात्र कोई क्षुद्र मर्त्य जीव नहीं है। मनुष्य तभी वास्तविक प्रेम कर सकता है, जब वह देखता है कि उसके प्रेम का पात्र एक मिट्टी का ढेला नहीं, अपितु स्वयं साक्षात् भगवान हैं। स्त्री यदि यह समझे कि उसके पति साक्षात् ब्रह्म-स्वरूप हैं, तो उससे और भी अधिक प्रेम करेगी। पति भी यदि यह जाने कि स्त्री स्वयं ब्रह्म-स्वरूप है, तो वह भी स्त्री से अधिक प्रेम करेगा। सन्तान को ब्रह्म-स्वरूप देखनेवाली माताएँ अपनी सन्तान से अधिक स्नेह कर सकेंगी। जो लोग यह जानेंगे कि उनके शत्रु साक्षात् ब्रह्म-स्वरूप हैं, वे लोग अपने महान् शत्रुओं के प्रति भी प्रेमभाव रख सकेंगे। जो लोग यह समझेंगे कि साधु व्यक्ति साक्षात् ब्रह्म-स्वरूप हैं, वे ही लोग पवित्र लोगों से प्रेम करेंगे। जो लोग यह जान लेंगे कि इन महादुष्टों के भी पीछे वे प्रभु ही विराजमान हैं, वे अत्यन्त अपवित्र व्यक्तियों से भी प्रेम करेंगे। जिनका क्षुद्र अहं पूर्णतः मर चुका है और उसके स्थान पर ईश्वर ने अधिकार जमा लिया है, वे लोग ही जगत् के प्रेरक हो सकते हैं। उनके लिए समग्र विश्व दिव्य भाव से रूपान्तरित हो जाता है। जो कुछ भी दुःखकर या क्लेशकर है, वह सब उनकी दृष्टि से लुप्त हो जाता है, सभी प्रकार के द्वन्द्व और संघर्ष समाप्त हो जाते हैं। तब यह जगत्, जहाँ हम प्रतिदिन रोटी के एक टुकड़े के लिए झगड़ा और मारपीट करते हैं, उनके लिए कारागार होने के बदले एक क्रीड़ाक्षेत्र बन जाता है। तब जगत् बड़ा सुन्दर रूप धारण कर लेता है। ऐसे ही व्यक्ति को यह कहने का अधिकार है कि ‘यह जगत् कितना सुन्दर है!’ उन्हीं को यह कहने का अधिकार है कि सब मंगल-स्वरूप है। 

धर्म की इस प्रत्यक्ष उपलब्धि से जगत् का यह महान् हित होगा कि ये अविराम विवाद, द्वन्द्व आदि सब दूर हो जाएँगे और जगत् में शान्ति का राज्य हो जाएगा। यदि जगत् के सभी लोग आज इस महान् सत्य के एक बिन्दु की भी उपलब्धि कर सके, तो उनके लिए यह सारा जगत् एक दूसरा ही रूप धारण कर लेगा और यह सब झगड़ा समाप्त होकर शान्ति का राज्य आ जाएगा। यह घिनौना उतावलापन, यह स्पर्धा - जो हमें अन्य सबों को ठेलकर आगे बढ़ निकलने के लिए बाध्य करती है, इस संसार से उठ जाएगी। इसके साथ-साथ सब प्रकार की अशान्ति, घृणा, ईर्ष्या तथा सभी प्रकार का बुराइयाँ सदा के लिए चली जाएँगी। तब इस जगत् में देवता लोग निवास करेंगे। तब यही जगत् स्वर्ग हो जाएगा। जब देवता देवता से खेलेगा, देवता देवता से मिलकर कार्य करेगा, देवता देवता से प्रेम करेगा, तब क्या इसमें बुराई ठहर सकती है? ईश्वर की प्रत्यक्ष उपलब्धि की यही एक बड़ी उपयोगिता है। समाज में तुम जो कुछ भी देख रहे हो, वह सब उस समय परिवर्तित होकर एक दिव्य रूप धारण कर लेगा। तब तुम किसी मनुष्य को बुरा नहीं समझोगे। यही प्रथम महालाभ है। उस समय तुम लोग किसी अन्याय करनेवाले बेचारे नर-नारी की ओर घृणापूर्ण दृष्टि से नहीं देखोगे।... तब तुममें ईर्ष्या अथवा दूसरों पर शासन करने का भाव नहीं रहेगा; वह सब चला जाएगा। तब प्रेम इतना प्रबल हो जाएगा कि फिर मानव-जाति को सत्पथ पर चलने के लिए चाबुक की आवश्यकता नहीं रह जाएगी। 

यदि संसार के नर-नारियों का लाखवाँ भाग भी बिल्कुल चुप रहकर एक क्षण के लिए कहे, “तुम सभी ईश्वर हो, हे मानवो, हे पशुओ, हे सब प्रकार के जीवित प्राणियो! तुम सभी एक जीवन्त ईश्वर के प्रकाश हो!” तो आधे घण्टे के अन्दर ही सारे जगत् का परिवर्तन हो जाए। तब सभी देशों के लोग चारों ओर घृणा के बीज न बोकर, ईर्ष्या और असत् चिन्ता का प्रवाह न फैलाकर, सोचेंगे कि सभी ‘वह’ है। जो कुछ तुम देख रहे हो या अनुभव कर रहे हो, वह सब ‘वही’ है। तुम्हारे भीतर बुराई न रहने पर, तुम बुराई कैसे देखोगे? तुम्हारे भीतर यदि चोर न हो, तो तुम किस प्रकार चोर देखोगे? तुम स्वयं यदि खूनी नहीं हो, तो किस प्रकार खूनी देखोगे? साधु हो जाओ, तो तुम्हारे भीतर से असाधु-भाव एकदम चला जाएगा। इस प्रकार सारे जगत् का परिवर्तन हो जाएगा। यही समाज का सबसे बड़ा लाभ है। मनुष्य के लिए यही महान् लाभ है।\endnote{ २/३९-४१;} 

संसार में ज्ञान के प्रकाश का विस्तार करो। प्रकाश! सिर्फ प्रकाश लाओ! प्रत्येक व्यक्ति ज्ञान के प्रकाश को प्राप्त करे। जब तक सब लोग भगवान के निकट न पहुँच जाएँ, तब तक तुम्हारा कार्य पूरा नहीं हुआ है। गरीबों में ज्ञान का विस्तार करो, धनिकों पर और भी अधिक प्रकाश डालो; क्योंकि निर्धनों की अपेक्षा धनिकों को अधिक प्रकाश की आवश्यकता है। अपढ़ लोगों को भी प्रकाश दिखाओ। शिक्षित मनुष्यों के लिए और अधिक प्रकाश चाहिए, क्योंकि आजकल शिक्षा का मिथ्याभिमान खूब प्रबल हो रहा है। इसी तरह सबके निकट प्रकाश का विस्तार करो। और शेष सब भगवान पर छोड़ दो, क्योंकि स्वयं भगवान के शब्दों में -

\begin{verse}
कर्मण्येवाधिकारस्ते मा फलेषु कदाचन।\\मा कर्मफलहेतुर्भू मा ते संगोऽस्त्वकर्मणि॥ \vauthor{- गीता, २. ४७ } 
\end{verse}

- कर्म में ही तुम्हारा अधिकार है, फल में नहीं; तुम इस भाव से कर्म मत करो जिससे तुम्हें फल-भोग करना पड़े। तुम्हारी प्रवृत्ति कर्म त्याग करने की ओर न हो।\endnote{ ५/१४२} 

\delimiter

\addtoendnotes{\protect\end{multicols}}

\addtocontents{toc}{\protect\par\egroup}

