
\chapter{शिक्षा: एकमात्र समाधान }

\indentsecionsintoc

\toendnotes{शिक्षा : एकमात्र समाधान}

\addtoendnotes{\protect\begin{multicols}{3}}

\section*{ज्ञान मनुष्य के भीतर है}

\vskip -5pt\addsectiontoTOC{ज्ञान मनुष्य के भीतर है}

हम जो कहते हैं कि मनुष्य ‘जानता’ है, उसे ठीक मनोवैज्ञानिक भाषा में व्यक्त करने के लिए हमें कहना चाहिए कि वह ‘आविष्कार’ करता है। मनुष्य जो कुछ ‘सीखता’ है, वह वास्तव में ‘आविष्कार’ करना ही है। आविष्कार का अर्थ है - मनुष्य का अपनी अनन्त ज्ञान-स्वरूप आत्मा के ऊपर से आवरण को हटा लेना। 

हम कहते हैं कि न्यूटन ने गुरुत्वाकर्षण का आविष्कार किया। तो क्या वह आविष्कार किसी कोने में पड़ा हुआ न्यूटन की प्रतीक्षा कर रहा था? वह उसके मन में ही था। समय आया और उसने उसे ढूँढ़ निकाला। संसार ने जितना भी ज्ञान प्राप्त किया है, वह मन से ही निकला है। विश्व का असीम पुस्तकालय तुम्हारे मन में ही विद्यमान है। बाह्य जगत् तो तुम्हें अपने मन के अध्ययन में लगाने के लिए उद्दीपक तथा अवसर मात्र है; पर सर्वदा तुम्हारे अध्ययन का विषय तुम्हारा मन ही रहता है। सेव के गिरने ने न्यूटन को उद्दीपना प्रदान की और उसने अपने मन का अध्ययन किया। उसने अपने मन में पूर्व से स्थित विचार-शृंखला की कड़ियों को एक बार फिर से सँजोया और उनमें एक नयी कड़ी का आविष्कार किया। उसी को हम गुरुत्वाकर्षण का नियम कहते हैं। यह न सेव में था और न पृथ्वी के केन्द्र में स्थित किसी अन्य वस्तु में। 

अतः सारा ज्ञान - चाहे वह व्यावहारिक हो या पारमार्थिक - मनुष्य के मन में ही निहित है। बहुधा यह प्रकट न होकर ढँका रहता है और जब आवरण धीरे-धीरे हटता जाता है, तो हम कहते हैं कि ‘हमें ज्ञान हो रहा है’ और ज्यों-ज्यों इस आविष्करण की क्रिया बढ़ती जाती है, त्यों-त्यों हमारे ज्ञान की वृद्धि हो जाती है। जिस मनुष्य पर से यह आवरण उठता जा रहा है, वह अन्य व्यक्तियों की अपेक्षा अधिक ज्ञानी है और जिस मनुष्य पर यह आवरण तह-दर-तह पड़ा है, वह अज्ञानी है। जिस मनुष्य पर से यह आवरण बिल्कुल चला जाता है, वह सर्वज्ञ पुरुष कहलाता है। अतीत में कितने ही सर्वज्ञ व्यक्ति हो चुके हैं और मेरा विश्वास है कि अब भी बहुत से होंगे तथा आगामी युगों में भी ऐसे असंख्य पुरुष जन्म लेंगे।\endnote{ ३/३;}


\section*{शिक्षा का तात्पर्य}

\vskip -5pt\addsectiontoTOC{शिक्षा का तात्पर्य}

शिक्षा का अर्थ है, उस पूर्णता की अभिव्यक्ति, जो सब मनुष्यों में पहले ही से विद्यमान है। अतः... शिक्षक का कार्य केवल रास्ते की सभी रुकावटें हटा देना मात्र है।\endnote{ २/३२८;} 

सच्ची शिक्षा की तो अभी हम लोगों में कल्पना भी नहीं की गयी है।... यह शब्दों का रटना मात्र नहीं है, तो भी इसे मानसिक शक्तियों का विकास अथवा व्यक्तियों को ठीक तथा कुशल ढंग से इच्छा करने का प्रशिक्षण कहा जा सकता है।\endnote{ ४/२६८;} 

यंत्रचालित अति विशाल जहाज और महा-बलवान रेल का इंजन जड़ हैं, वे हिलते हैं और चलते हैं, परन्तु वे जड़ हैं। और वह जो दूर से नन्हा-सा कीड़ा अपने जीवन की रक्षा के लिए रेल की पटरी से हट गया, वह चैतन्य क्यों है? यंत्र में इच्छा-शक्ति का कोई विकास नहीं है। यंत्र कभी नियम का उल्लंघन करने की इच्छा नहीं रखता। कीड़ा नियम का विरोध करना चाहता है और नियम के विरुद्ध जाता है, चाहे उस प्रयत्न में वह सफल हो, या असफल, इसलिए वह चेतन है। जिस अंश में इच्छा-शक्ति के प्रकट होने में सफलता होती है, उसी अंश में सुख अधिक होता है और जीव उतना ही ऊँचा होता है। परमेश्वर की इच्छा-शक्ति पूर्ण रूप से सफल होती है, इसलिए वह उच्चतम है। 

शिक्षा किसे कहते हैं? क्या वह पठन मात्र है? - नहीं। क्या वह नाना प्रकार का ज्ञानार्जन है? - नहीं, वह भी नहीं। \textbf{जिस संयम के द्वारा इच्छाशक्ति का प्रवाह तथा विकास वश में लाया जाता है और फलदायी होता है, उसे शिक्षा कहते हैं। } अब सोचो कि क्या वह शिक्षा है, जिसने निरन्तर इच्छाशक्ति को बलपूर्वक पीढ़ी-दर-पीढ़ी रोककर उसे प्रायः नष्ट कर दिया है, जिसके प्रभाव से नये विचारों की तो बात ही जाने दो, पुराने विचार भी एक-एक कर लुप्त होते चले जा रहे हैं; क्या वह शिक्षा है, जो मनुष्य को धीरे-धीरे यंत्र बना रही है? जो स्वचालित यंत्र के समान सुकर्म करता है, उसकी अपेक्षा अपनी स्वतंत्र इच्छाशक्ति और बुद्धि के बल से अनुचित कर्म करनेवाला मेरे विचार से श्रेयस्कर है।\endnote{ ७/३५८;} 

जानकारियों से मन को भर देना ही शिक्षा नहीं है। (शिक्षा का आदर्श है) मनरूपी यंत्र को योग्य बनाना और उस पर पूर्ण अधिकार प्राप्त करना।\endnote{ ४/१५७;} 

मेरे विचार से \textbf{शिक्षा का सार - तथ्यों का संकलन नहीं, बल्कि मन की एकाग्रता प्राप्त करना है। } यदि मुझे फिर से अपनी शिक्षा आरम्भ करनी हो और इस पर मेरा वश चले, तो मैं तथ्यों का अध्ययन कदापि न करूँ। मैं अपने मन की एकाग्रता और अनासक्ति की क्षमता बढ़ाऊँगा और उपकरण के पूर्णतया तैयार हो जाने पर उससे यथेच्छा तथ्यों का संकलन करूँगा। बच्चे में मन की एकाग्रता और अनासक्ति की सामर्थ्य एक साथ विकसित होनी चाहिए।\endnote{ ४/१०९;} 

शिक्षा का मतलब यह नहीं है कि तुम्हारे दिमाग में ऐसी बहुत-सी बातें इस प्रकार ठूँस दी जाएँ कि उनमें अन्तर्द्वन्द्व होने लगे और तुम्हारा दिमाग उन्हें जीवन भर पचा ही न सके। जिस शिक्षा से हम अपना जीवन निर्माण कर सकें, मनुष्य बन सकें, चरित्रगठन कर सकें और विचारों का सामंजस्य कर सकें, वही वास्तव में शिक्षा कहलाने योग्य है। यदि तुम पाँच ही भावों को पचाकर उसके अनुसार अपना जीवन और चरित्र गठित कर सके हो, तो तुम्हारी शिक्षा उस आदमी की अपेक्षा बहुत अधिक है, जिसने एक पूरे पुस्तकालय को कण्ठस्थ कर लिया है।... यदि अनेक प्रकार की जानकारियों का संचय करना ही शिक्षा हो, तब तो ये पुस्तकालय संसार में सर्वश्रेष्ठ मुनि और विश्वकोष ही ऋषि हैं।\endnote{ ५/१९५;} 

हमें ऐसी शिक्षा की आवश्यकता है, जिससे चरित्र-निर्माण हो, मानसिक शक्ति बढ़े, बुद्धि विकसित हो और देश के युवक अपने पैरों पर खड़े होना सीखें।\endnote{ ८/२७७;} 

जो शिक्षा सामान्य व्यक्ति को जीवन संग्राम में समर्थ नहीं बना सकती, जो मनुष्य में चरित्र-बल, परोपकार की भावना और सिंह के समान साहस नहीं ला सकती, वह भी क्या कोई शिक्षा है? शिक्षा वही है, जिसके द्वारा जीवन में अपने पैरों पर खड़ा हुआ जाता है।\endnote{ ६/१०६;}


\section*{शिक्षक का कर्तव्य}

\addsectiontoTOC{शिक्षक का कर्तव्य}

एक अन्य चीज, जिसकी हमें जरूरत है, वह है उस शिक्षा-पद्धति का निर्मूलन, जो मार-मार कर गधों को घोड़ा बनाना चाहती है।... कोई भी किसी को कुछ नहीं सिखा सकता। जो शिक्षक यह समझता है कि वह कुछ सिखा रहा है, वह सब गुड़गोबर कर देता है। वेदान्त का सिद्धान्त है कि मनुष्य के अन्तर में ज्ञान का समस्त भण्डार निहित है - एक अबोध शिशु में भी - केवल उसको जाग्रत कर देने की आवश्यकता है और यही आचार्य का काम है। हमें बच्चों के लिए बस इतना ही करना है कि वे अपने हाथ-पैर तथा आँख-कान का समुचित उपयोग करना भर सीख लें और फिर सब आसान है।\endnote{ ८/२२९;} 

हस्तक्षेप मत करो! अपनी सीमा के भीतर रहो और सब ठीक हो जाएगा।\endnote{ २/३२८;} 

\textbf{किसी व्यक्ति की श्रद्धा नष्ट करने का प्रयत्न मत करो। } यदि हो सके तो उसे जो कुछ अधिक अच्छा हो, दे दो, यदि हो सके तो जिस स्तर पर वह खड़ा हो, उसे सहायता देकर उससे ऊपर उठा दो - परन्तु जिस स्थान पर वह था, उस जगह से उसे नीचे मत गिराओ। सच्चा शिक्षक वही है, जो क्षण भर में ही अपने को हजारों व्यक्तियों में परिणत कर सके। सच्चा शिक्षक वही है, जो छात्र को सिखाने के लिए तत्काल छात्र की ही मनोभूमि में उतर आये तथा अपनी आत्मा अपने छात्र की आत्मा में एकरूप कर सके और जो छात्र की ही दृष्टि से देख सके, उसी के कानों से सुन सके तथा उसी के मस्तिष्क से समझ सके। ऐसा ही आचार्य शिक्षा दे सकता है - दूसरा नहीं। बाकी निषेधकारी, निरुत्साहक तथा संहारक आचार्य कभी भलाई नहीं कर सकते।\endnote{ ७/२६३;}


\section*{शिक्षक और छात्र का सम्बन्ध}

\addsectiontoTOC{शिक्षक और छात्र का सम्बन्ध}

मैं शिक्षा को गुरु के साथ व्यक्तिगत सम्पर्क - गुरुगृह-वास समझता हूँ। गुरु के व्यक्तिगत जीवन के अभाव में शिक्षा नहीं हो सकती। अपने विश्वविद्यालयों को ही लीजिए। अपने पचास वर्ष के अस्तित्व में उन्होंने क्या किया है? उन्होंने एक भी मौलिक व्यक्ति पैदा नहीं किया। वे केवल परीक्षा लेने की संस्थाएँ हैं। सबके कल्याण के लिए बलिदान की भावना का अभी हमारे देश में विकास नहीं हुआ है।\endnote{ ४/ २६२;} 

\newpage

बाल्यावस्था से ही जाज्वल्यमान, उज्ज्वल चरित्रयुक्त किसी तपस्वी महापुरुष के सान्निध्य में रहना चाहिए, ताकि उच्चतम ज्ञान का जीवन्त आदर्श आँखों के सामने रहे। केवल यह पढ़ लेने भर से कि ‘झूठ बोलना पाप है’ - कोई लाभ नहीं। हर एक को पूर्ण ब्रह्मचर्य पालन करने का व्रत लेना चाहिए, तभी हृदय में श्रद्धा और भक्ति का उदय होगा। नहीं तो, जिसमें श्रद्धा और भक्ति नहीं, वह झूठ क्यों नहीं बोलेगा? हमारे देश में अध्यापन का महान् कार्य सदैव निःस्पृह और त्यागी पुरुषों ने ही किया है।... जब तक अध्यापन का कार्य त्यागी लोगों ने किया, तब तक भारत समृद्ध बना रहा।\endnote{ ८/२३१;}


\section*{शिक्षा में धर्म का समावेश}

\addsectiontoTOC{शिक्षा में धर्म का समावेश}

मैं धर्म को शिक्षा का अन्तरतम अंग समझता हूँ।\endnote{ ४/२६८;} 

सबका मूल धर्म है, वही मुख्य है। (शिक्षा में) धर्म ही भात के समान है, शेष सब वस्तुएँ सब्जी-चटनी जैसी हैं। केवल सब्जी और चटनी खाने से अपच हो जाता है; और केवल भात खाने से भी वैसा ही होता है।\endnote{ ८/२२९;} 

धर्म-प्रचार के बाद उसके साथ-साथ लौकिक विद्या तथा अन्य आवश्यक विद्याएँ आप ही आ जाएँगी; पर यदि तुम बिना धर्म के लौकिक विद्या ग्रहण करना चाहो, तो मैं तुम्हें स्पष्ट बता देता हूँ कि भारत में तुम्हारा यह प्रयास व्यर्थ सिद्ध होगा, वह लोगों के हृदय में स्थान नहीं बना सकेगा।\endnote{ ५/११८;}


\section*{धर्म-केन्द्रित शिक्षा की योजना}

\addsectiontoTOC{धर्म-केन्द्रित शिक्षा की योजना}

सबसे पहले हमें एक मन्दिर की जरूरत है, क्योंकि हिन्दू लोग सब कार्यों में धर्म को ही प्रथम स्थान देते हैं। तुम कहोगे कि ऐसा होने पर हिन्दुओं के विभिन्न मतावलम्बियों में परस्पर झगड़े होने लगेंगे। पर मैं तुमको किसी विशेष मत के अनुसार मन्दिर बनाने को नहीं कहता। वह इन साम्प्रदायिक भेद-भावों के परे होगा। उसका एकमात्र प्रतीक होगा - ॐ, जो हमारे किसी भी सम्प्रदाय के लिए सर्वोच्च प्रतीक है। यदि हिन्दुओं में कोई ऐसा सम्प्रदाय हो, जो इस ओंकार को न माने, तो जान लो कि वह हिन्दू कहलाने के योग्य नहीं है। वहाँ सब लोग अपने-अपने सम्प्रदाय के अनुसार ही हिन्दुत्व की व्याख्या कर सकेंगे, पर मन्दिर हम सबके लिए एक ही होना चाहिए। जो अपने सम्प्रदाय के अनुसार देवी-देवताओं की प्रतिमा-पूजा करना चाहें, अन्यत्र जाकर करें, पर इस मन्दिर में वे औरों से झगड़ा न करें। इस मन्दिर में वे ही धार्मिक तत्त्व समझाए जाएँगे, जो सब सम्प्रदायों में समान हैं। साथ ही यहाँ हर सम्प्रदायवाले को अपने मत की शिक्षा देने का अधिकार होगा, पर एक प्रतिबन्ध रहेगा कि वे अन्य सम्प्रदायों से झगड़ा नहीं कर सकेंगे। बोलो, तुम क्या कहते हो? संसार तुम्हारी राय जानना चाहता है, उसे यह सुनने का समय नहीं है कि तुम औरों के विषय में क्या विचार प्रकट कर रहे हो। औरों की बात छोड़कर तुम अपनी ओर ध्यान दो। 

दूसरी बात यह है कि इस मन्दिर के साथ ही एक और संस्था हो, जिसमें धार्मिक शिक्षक और प्रचारक तैयार किए जाएँ और वे सभी घूम-फिरकर धर्म-प्रचार करने भेजे जाएँ। पर ये केवल धर्म का प्रचार न करें, वरन् उसके साथ-साथ लौकिक शिक्षा का भी प्रचार करें। जैसे हम द्वार-द्वार जाकर धर्म का प्रचार करते हैं, वैसे ही हमें लौकिक शिक्षा का भी प्रचार करना पड़ेगा। यह काम सहज ही हो सकता है। शिक्षकों तथा धर्म-प्रचारकों द्वारा हमारे कार्य का विस्तार होता जाएगा और क्रमशः अन्य स्थानों में ऐसे ही मन्दिर स्थापित होंगे और इस प्रकार यह कार्य पूरे भारत में फैल जाएगा। यही मेरी योजना है।\endnote{ ५/११८;}


\section*{नकारात्मक नहीं, सकारात्मक शिक्षा}

\addsectiontoTOC{नकारात्मक नहीं, सकारात्मक शिक्षा}

जन-साधारण के सामने सकारात्मक आदर्श रखना होगा। ‘नहीं-नहीं’ की भावना मनुष्य को दुर्बल बना डालती है। देखते नहीं, जो माता-पिता दिन-रात बच्चों के लिखनेपढ़ने पर जोर देते रहते हैं, कहते हैं, ‘इसका कुछ सुधार नहीं होगा, यह मूर्ख है, गधा है’, आदि-आदि - उनके बच्चे अधिकांशतः वैसे ही बन जाते हैं। बच्चों को अच्छा कहने से और प्रोत्साहन देने से - समय आने पर वे स्वयं ही अच्छे बन जाते हैं।... यदि लोगों को रचनात्मक भाव दिये जा सकें, तो साधारण व्यक्ति भी मनुष्य बन जाएगा और अपने पैरों पर खड़ा होना सीख सकेगा। मनुष्य भाषा, साहित्य, दर्शन, कविता, शिल्प आदि विविध क्षेत्रों में जो प्रयत्न कर रहा है, उसमें वह अनेक गलतियाँ करता है। उचित यह होगा कि हम उसे उन गलतियों को न बताकर, प्रगति के मार्ग पर धीरे-धीरे अग्रसर होने में सहायता दें। गलतियाँ दिखाने से लोगों की भावना को ठेस पहुँचती है और वे हतोत्साहित हो जाते हैं। श्रीरामकृष्ण को हमने देखा है - जिन्हें हम त्याज्य मानते थे, उन्हें भी वे प्रोत्साहित करके उनके जीवन की गति मोड़ देते थे। शिक्षा देने का उनका ढंग बड़ा अद्भुत था।\endnote{ ६/११२;} 

न्यूयार्क में मैं आइरिश नवागन्तुकों को देखा करता था - पददलित, कान्तिहीन, निःसम्बल, अति निर्धन और अशिक्षित - साथ में एक लाठी और उसके सिर पर लटकती हुई फटे कपड़ों की एक छोटी-सी पोटली। उसकी चाल में भय और आँखों में आशंका होती थी। छह महीने में ही यह दृश्य बिल्कुल बदल जाता था। अब वह तनकर चलता था, उसका वेश बदल गया था, उसकी चाल और चितवन में पहले का भय नहीं दिखायी पड़ता। ऐसा क्यों हुआ? हमारा वेदान्त कहता है कि वह आइरिश अपने देश में चारों तरफ घृणा से घिरा हुआ रहता था - सारी प्रकृति एक स्वर में उससे कह कही थी - “बच्चू, तेरे लिए और कोई आशा नहीं है; तू गुलाम ही पैदा हुआ और सदा गुलाम ही बना रहेगा।” आजन्म सुनते-सुनते बच्चू को वैसा ही विश्वास हो गया। बच्चू ने अपने को सम्मोहित कर डाला कि वह अति नीच है। इससे उसका ब्रह्मभाव संकुचित हो गया। पर जब उसने अमेरिका में पैर रखा, तो चारों ओर से ध्वनि उठी - “बच्चू, तू भी वही आदमी है, जो हम लोग हैं। मनुष्य ने ही सब काम किए हैं; तेरे और मेरे जैसे मनुष्य ही सब कुछ कर सकते हैं। धीरज धर।” बच्चू ने सिर उठाया और देखा कि बात तो ठीक ही है - बस, उसके अन्दर सोया हुआ ब्रह्म जाग उठा; मानो प्रकृति ने स्वयं ही कहा हो - “उठो, जागो और जब तक मंजिल तक न पहुँच जाओ, रुको मत।” 

हमारे लड़के जो शिक्षा पा रहे हैं, वह बड़ी निषेधात्मक है। स्कूल के लड़के कुछ भी नहीं सीखते, बल्कि जो कुछ अपना है, उसका भी नाश हो जाता है और इसका फल होता है - श्रद्धा का अभाव। जो श्रद्धा वेद-वेदान्त का मूलमंत्र है, जिस श्रद्धा के बल से यह संसार चल रहा है - उसी श्रद्धा का लोप! गीता में कहा है -\textbf{अज्ञश्च अश्रद्दधानश्च संशयात्मा विनश्यति } - अज्ञ, श्रद्धाहीन तथा संशययुक्त व्यक्ति का नाश हो जाता है, इसीलिए हम मृत्यु के इतने पास हैं। अब उपाय है - शिक्षा का प्रसार। पहले आत्मज्ञान। इससे मेरा तात्पर्य जटाजूट, दण्ड, कमण्डलु और पहाड़ों की कन्दराओं से नहीं, जो इस शब्द के उच्चारण करते ही याद आते हैं। तो मेरा मतलब क्या है?... द्वैत, विशिष्टाद्वैत, अद्वैत, शैव-सिद्धान्त, वैष्णव, शाक्त, यहाँ तक कि बौद्ध और जैन आदि भारत में जितने सम्प्रदाय स्थापित हुए हैं, सभी इस विषय पर सहमत है कि इस जीवात्मा में अव्यक्त भाव से अनन्त शक्ति निहित है; चींटी से लेकर सर्वोच्च सिद्ध पुरुष तक - सभी में वह आत्मा विराजमान है, भेद केवल उसकी अभिव्यक्ति की मात्रा में है।... जैसे किसान खेतों की मेड़ तोड़ देता है और एक खेत का पानी दूसरे खेत में चला जाता हैं, वैसे ही आत्मा भी आवरण टूटते ही प्रकट हो जाती है। उपयुक्त अवसर और उपयुक्त्त देशकाल मिलते ही उस शक्ति का विकास हो जाता है; परन्तु चाहे विकास हो, या न हो, वह शक्ति प्रत्येक जीव - ब्रह्मा से लेकर तिनके तक में - विद्यमान है। सर्वत्र जा-जाकर इस शक्ति को जगाना होगा। 

यह हुई पहली बात। दूसरी बात यह है कि इसके साथ-साथ शिक्षा भी देनी होगी।\endnote{ ६/३११-१२;} 

पहले हमें गुरुगृह-वास और वैसी ही अन्य शिक्षा-प्रणालियों को पुनर्जीवित करना होगा। आज हमें आवश्यकता है - वेदान्तयुक्त पाश्चात्य विज्ञान की, ब्रह्मचर्य के आदर्श और श्रद्धा तथा आत्मविश्वास की।\endnote{ ८/२२९;} 

आज हमें जरूरत है - विदेशी नियंत्रण हटाकर, हमारे विविध विद्याओं का अध्ययन हो; और साथ-ही-साथ अंग्रेजी भाषा तथा पाश्चात्य विज्ञान भी सिखाया जाए। हमें उद्योग-धन्धों की उन्नति के लिए तकनीकी शिक्षा भी प्राप्त करनी होगी, ताकि देश के युवक नौकरी ढूँढ़ने के बजाय अपनी जीविका के लिए समुचित धनोपार्जन कर सकें और दुर्दिन के लिए कुछ बचाकर रख भी सकें।\endnote{ ८/२३१;}


\section*{दुर्बल बनानेवाली शिक्षा नहीं चाहिए}

\addsectiontoTOC{दुर्बल बनानेवाली शिक्षा नहीं चाहिए}

द्वैतवाद के कई प्रकारों के विषय में मुझे कोई आपत्ति नहीं है, किन्तु जो कोई उपदेश दुर्बलता की शिक्षा देता है, उस पर मुझे विशेष आपत्ति है। स्त्री-पुरुष बालकबालिका, जिस समय दैहिक, मानसिक अथवा आध्यात्मिक शिक्षा पाते हैं, उस समय मैं उनसे एकमात्र यही प्रश्न करता हूँ - “क्या तुम्हें इससे बल प्राप्त होता है?” क्योंकि जानता हूँ, एकमात्र सत्य ही बल प्रदान करता है। मैं जानता हूँ कि एकमात्र सत्य ही प्राणप्रद है। सत्य की ओर गए बिना हम अन्य किसी भी उपाय से वीर नहीं हो सकते; और वीर हुए बिना हम सत्य के समीप नहीं पहुँच सकते। इसीलिए जो मत या जो शिक्षा-प्रणाली मन तथा मस्तिष्क को दुर्बल कर दे और मनुष्य को अन्धविश्वासों से भर दे, जिससे वह अन्धकार में टटोलता रहे, ख्याली पुलाव पकाता रहे और सब प्रकार की अजीबो-गरीब और अन्धविश्वासपूर्ण बातों की तह छानता रहे, उस मत या प्रणाली को मैं पसन्द नहीं करता, क्योंकि मनुष्य पर उसका परिणाम बड़ा भयानक होता है। ऐसी प्रणालियों से कभी कोई उपकार नहीं होता। बल्कि उल्टे वे मन में रोगात्मकता ला देती हैं, उसे दुर्बल बना देती हैं - इतना दुर्बल कि कालान्तर में मन सत्य को ग्रहण करने और उसके अनुसार जीवन-गठन करने में सर्वथा असमर्थ हो जाता है।\endnote{ २/१८८;}


\section*{हमारा लक्ष्य स्वावलम्बन हो}

\addsectiontoTOC{हमारा लक्ष्य स्वावलम्बन हो}

लोगों को यदि आत्मनिर्भर बनने की शिक्षा न दी जाए, तो सारे संसार की दौलत से भी भारत के एक छोटे से गाँव की सहायता नहीं की जा सकती है। नैतिक तथा बौद्धिक - दोनों ही प्रकार की शिक्षा प्रदान करना ही हमारा पहला कार्य होना चाहिए,... ताकि इस शिक्षा के फलस्वरूप लोग आत्मनिर्भर तथा मितव्ययी बन सकें; विवाह की ओर उनका अस्वाभाविक रुझान दूर हो और इस प्रकार वे भविष्य में अकाल के मुख में जाने से अपने को बचा सकें।\endnote{ ५/८;}


\section*{वर्तमान शिक्षा-प्रणाली के दोष}

\addsectiontoTOC{वर्तमान शिक्षा-प्रणाली के दोष}

आज की विश्वविद्यालय की शिक्षा में दोष-ही-दोष भरे हैं। यह क्लर्क पैदा करने की मशीन के सिवाय कुछ नहीं है। यदि इतना ही होता, तो भी ठीक था, पर नहीं - इस शिक्षा से लोग किस प्रकार श्रद्धा-विश्वास छोड़ते जा रहे हैं। वे कहते हैं कि गीता तो एक प्रक्षिप्त ग्रन्थ है और वेद ग्राम्य-गीत मात्र हैं। वे भारत के बाहर के देशों तथा विषयों के बारे में तो हर बात जानना चाहते हैं, पर यदि उनसे कोई उनके पूर्वजों के नाम पूछे, तो चौदह पीढ़ी तो दूर रही, सात पीढ़ी तक भी नहीं बता सकते।... जिस राष्ट्र का कोई अपना इतिहास नहीं, उसके पास इस संसार में कुछ भी नहीं। जिस व्यक्ति को सदैव इस बात का विश्वास तथा अभिमान है कि वह उच्च कुल में पैदा हुआ है, क्या वह कभी दुश्चरित्र हो सकेगा? ऐसा क्यों है? उसमें निहित आत्मविश्वास तथा स्वाभिमान का भाव सदैव उसके विचार और कार्य को इतना नियंत्रित रखता है कि ऐसा व्यक्ति सन्मार्ग से च्युत होने की अपेक्षा हँसते-हँसते मृत्यु का आलिंगन कर लेगा। इसी तरह राष्ट्र का गौरवमय अतीत राष्ट्र को नियंत्रण में रखता है और उसका अधःपतन नहीं होने देता।... यदि एक दृष्टि से देखा जाए, तो तुम्हें वायसराय\footnote{ भारत के तत्कालीन वायसराय लार्ड कर्जन ने विश्वविद्यालयी शिक्षा के स्तर को ऊँचा करने के उद्देश्य से उसे इतना महँगा बना देना चाहा था कि वह मध्य-वर्ग के छात्रों की पहुँच के बाहर हो जाती।} का कृतज्ञ होना चाहिए कि उन्होंने विश्वविद्यालयी शिक्षा में सुधार का प्रस्ताव रखा है। इससे उच्च शिक्षा करीबकरीब बन्द ही हो जाएगी और देश को कम-से-कम कुछ साँस लेने और विचार करने का समय तो मिलेगा। वाह! ग्रेजुएट बनने के लिए क्या दौड़धूप लगी है और कुछ दिन बाद फिर ठण्डी पड़ जाती है। और आखिर में वे सीखते क्या हैं - बस यही कि हमारा धर्म, आचार-विचार और रीति-रिवाज सब खराब हैं, पाश्चात्यों की सब बातें अच्छी हैं! इस प्रकार हम महा विनाश को आमंत्रित कर रहे हैं। आखिर इस उच्च शिक्षा के रहने या न रहने से क्या बनता या बिगड़ता है? इससे कहीं ज्यादा अच्छा तो यह होगा कि उच्च शिक्षा पाकर नौकरी के लिए दफ्तरों की खाक छानने के बजाय लोग थोड़ी-सी यांत्रिक शिक्षा प्राप्त करें, ताकि किसी काम-धन्धे से लगकर अपना पेट तो पाल सकेंगे।\endnote{ ५/२०८;} 

\vskip 3pt

जो शिक्षा तुम अभी पा रहे हो, उसमें कुछ अच्छा अंश भी है, परन्तु दोष बहुत हैं। इसलिए ये बुराइयाँ उसके भले अंश को दबा देती हैं। पहली बात तो यह कि यह शिक्षा मनुष्य बनानेवाली नहीं कही जा सकती। यह शिक्षा केवल तथा पूर्णतः निषेधात्मक है। निषेधात्मक शिक्षा या निषेध की बुनियाद पर आधारित शिक्षा मृत्यु से भी भयानक है। कोमल-मति बालक पाठशाला में भर्ती होता है और उसे सबसे पहली बात यह सिखायी जाती है कि तुम्हारा बाप मूर्ख है। दूसरी बात वह सीखता है कि तुम्हारा दादा पागल है। तीसरी बात यह कि तुम्हारे सारे शिक्षक और आचार्य पाखण्डी हैं। और चौथी बात कि तुम्हारे सारे पवित्र धर्म-ग्रन्थों में झूठी और कपोल-कल्पित बातें भरी हुई हैं! ऐसी निषेधात्मक बातें सीखते हुए बालक जब सोलह वर्ष की आयु का होता है, तब तक वह निषेधों की ढेर बन चुका होता है - उसमें न जान रहती है और न रीढ़। इसका परिणाम जो होना था, वही हुआ। पिछले पचास वर्षों से दी जानेवाली इस शिक्षा ने तीनों प्रान्तों में एक भी स्वतंत्र विचारों का मनुष्य पैदा नहीं किया और जो मौलिक विचार के लोग हैं, उन्होंने यहाँ नहीं, विदेशों में शिक्षा पायी है या फिर अपने भ्रमपूर्ण अन्धविश्वासों को दूर करने हेतु पुनः अपने पुराने शिक्षालयों में जाकर अध्ययन किया है।\endnote{ ५/६-७;} 

\vskip 2pt


\section*{शिक्षा जनता की दुर्दशा को दूर कर सकती है}

\addsectiontoTOC{शिक्षा जनता की दुर्दशा को दूर कर सकती है}

शिक्षा! शिक्षा! केवल शिक्षा! यूरोप के अनेक नगरों में घूमकर और वहाँ के गरीबों के भी अमन-चैन और शिक्षा को देखकर अपने गरीब देशवासियों की याद आती थी और मैं आँसू बहाता था। यह अन्तर क्यों हुआ? उत्तर मिला - शिक्षा से। शिक्षा से आत्मविश्वास आया और आत्मविश्वास से उनका अन्तर्निहित ब्रह्मभाव जाग उठा; जबकि हमारा ब्रह्मभाव क्रमशः निद्रित - संकुचित होता जा रहा है।\endnote{ ५/५३-५४;} 

\vskip 3pt

शिक्षा का विस्तार तथा ज्ञान का उन्मेष हुए बिना देश की उन्नति कैसे होगी?... परन्तु स्मरण रहे कि आम जनता और महिलाओं में शिक्षा का प्रसार हुए बिना उन्नति का अन्य कोई उपाय नहीं।... पुराण, इतिहास, गृहकार्य, शिल्प-कला, गृहस्थी के नियम आदि आधुनिक विज्ञान की सहायता से सिखाने होंगे और आदर्श चरित्र गठन करने के लिए उपयुक्त आचरण की भी शिक्षा देनी होगी।... जिनकी माताएँ शिक्षित और नीतिपरायण हैं, उन्हीं घरों में बड़े लोग जन्म लेते हैं।\endnote{ ५/८४;}


\section*{जन-साधारण की शिक्षा}

\addsectiontoTOC{जन-साधारण की शिक्षा}

जनता को शिक्षित और उन्नत कीजिए। इसी तरह एक राष्ट्र का निर्माण होता है।... सारा दोष यहाँ है: यथार्थ राष्ट्र जो कि झोपड़ियों में बसता है, अपना मनुष्यत्व भूल चुका है, अपना व्यक्तित्व खो चुका है। हिन्दू, मुसलमान, ईसाई - हर किसी के पैरोंतले कुचले जाकर ये लोग यह समझने लगे हैं कि जिसके पास भी पर्याप्त धन है, उसी के पैरों-तले कुचले जाने के लिए उनका जन्म हुआ है। उन्हें उनका खोया हुआ व्यक्तित्व लौटाना होगा। उन्हें शिक्षित करना होगा।... हमारा कर्तव्य केवल रासायनिक पदार्थों को एकत्र भर कर देना है, इसके बाद ईश्वरीय विधान से वे स्वयं ही रवे में परिणत हो जाएँगे। हमें केवल उनके मस्तिष्क को विचारों से भर देना है, बाकी सब कुछ वे स्वयं ही कर लेंगे। इसका अर्थ हुआ - आम जनता को शिक्षित करना। इसी में कठिनाइयाँ हैं। एक टुटपुजिया सरकार कभी कुछ नहीं कर सकती है, न करेगी; अतः उस दिशा से किसी सहायता की आशा नहीं है। 

अब मान लो कि हम प्रत्येक गाँव में निःशुल्क पाठशाला खोलने में समर्थ हैं, तो भी गरीब लड़के स्कूलों में आने की अपेक्षा अपने जीविकोपार्जन हेतु हल चलाने जाएँगे। न तो हमारे पास (शिक्षा देने हेतु) धन है और न हम उनको शिक्षा के लिए बुला ही सकते हैं। समस्या निराशाजनक प्रतीत होती है। मैंने एक रास्ता ढूँढ़ निकाला है। वह यह है कि यदि पहाड़ मुहम्मद के पास नहीं आता, तो मुहम्मद को ही पहाड़ के पास जाना होगा। यदि गरीब शिक्षा लेने नहीं आ सकते, तो शिक्षा को ही उनके पास - खेत में, कारखाने में और हर जगह पहुँचना होगा। यह कैसे? आपने मेरे गुरुभाइयों को देखा है। मुझे सारे भारत से ऐसे निःस्वार्थ, अच्छे एवं शिक्षित सैकड़ों नवयुवक मिल सकते हैं। ये लोग गाँव-गाँव जाकर हर द्वार पर केवल धर्म ही नहीं, शिक्षा भी पहुँचाएँगे। इसलिए मैं भारत की नारियों के लिए शिक्षिकाओं के रूप में विधवाओं की भी एक छोटीसी टोली संगठित कर रहा हूँ। 

अब मान लो कि ग्रामीण लोग अपना दिन भर का काम समाप्त करके अपने गाँव में लौट आए हैं और किसी पेड़ के नीचे या कहीं और बैठकर, हुक्का पीते गप लड़ाते हुए समय बिता रहे हैं। मान लो, दो शिक्षित संन्यासी वहाँ पहुँच जाएँ और प्रोजेक्टर की सहायता से उन्हें ग्रह-नक्षत्रों या भिन्न-भिन्न देशों के या ऐतिहासिक दृश्यों के चित्र दिखाने लगें। इस प्रकार ग्लोब, नक्शे आदि के द्वारा जबानी ही कितना काम हो सकता है?... केवल आँख ही ज्ञान का एकमात्र द्वार नहीं है, कान से भी यह काम हो सकता है। इस प्रकार नये-नये विचारों तथा नीतियों से उनका परिचय होगा और वे एक बेहतर जीवन की आशा करने लगेंगे। हमारा काम यहीं समाप्त हो जाता है। बाकी सब उन्हीं पर छोड़ देना होगा।\endnote{ ५/१६६;} 

\newpage

हमारे देश में हजारों निष्ठावान और त्यागी साधु हैं, जो गाँव-गाँव धर्म की शिक्षा देते फिरते हैं। यदि उनमें से कुछ को सांसारिक विषयों के शिक्षकों के रूप में भी संगठित किया जा सके; तो गाँव-गाँव और द्वार-द्वार पर जाकर वे केवल धर्मशिक्षा ही नहीं, बल्कि भौतिक शिक्षा भी देंगे। मान लो, इनमें से दो लोग शाम को अपने साथ एक प्रोजेक्टर, एक ग्लोब और कुछ नक्शे आदि लेकर किसी गाँव में जाते हैं। वहाँ वे अपढ़ लोगों को गणित, ज्योतिष और भूगोल की बहुत कुछ शिक्षा दे सकते हैं। वे गरीब पुस्तकों से जीवन भर में जितनी जानकारी न पा सकेंगे, उससे सौगुनी अधिक जानकारी वे उन्हें बातचीत के जरिये - विभिन्न देशों के बारे में कहानियाँ सुनाकर दे सकते हैं।\endnote{ ४/३३५;} 

यदि स्वभाव में समता न भी हो, तो भी सबको समान सुविधा मिलनी चाहिए। फिर भी यदि किसी को अधिक तथा किसी को कम सुविधा देनी हो, तो बलवान की अपेक्षा दुर्बल को अधिक सुविधा देनी होगी। अर्थात् शूद्र को शिक्षा की जितनी जरूरत है, उतनी ब्राह्मण को नहीं। यदि किसी ब्राह्मण के पुत्र के लिए एक शिक्षक जरूरत हो, तो शूद्र के लड़के के लिए दस शिक्षक चाहिए। कारण यह है कि जिसकी बुद्धि को प्रकृति के द्वारा स्वाभाविक प्रखरता नहीं प्राप्त हुई है, उसकी बाहर से अधिक सहायता करनी होगी।\endnote{ ५/४४;}


\section*{नारियों की शिक्षा}

\addsectiontoTOC{नारियों की शिक्षा}

धर्म, शिल्प, विज्ञान, गृहकार्य, रसोई, सिलाई, स्वास्थ्य आदि सब विषयों की मोटी-मोटी बातें सिखलाना उचित है। नाटक और उपन्यास तो उनके पास पहुँचने ही नहीं चाहिए।... परन्तु केवल पूजा-पाठ सिखाने से ही काम नहीं बनेगा। सब विषयों में उनकी आँखें खोल देनी होगी। छात्राओं के सामने सर्वदा आदर्श नारी-चरित्र रखकर त्यागरूप व्रत में उनका अनुराग जगाना होगा। कुमारियों को सीता, सावित्री, दमयन्ती, लीलावती, खना, मीराबाई, आदि के जीवनियाँ बताकर उनको वैसा ही जीवन बनाने का उपदेश देना होगा।\endnote{ ६/४०;} 

असल बात है कि शिक्षा हो या दीक्षा, धर्महीन होने पर उसमें त्रुटि रह ही जाती है। धर्म को केन्द्र बनाकर स्त्री-शिक्षा का प्रचार करना होगा। धर्म के अतिरिक्त दूसरी शिक्षाएँ गौण होंगी। धर्म-शिक्षा, चरित्र-गठन तथा ब्रह्यचर्य-पालन - इन्हीं के लिए तो शिक्षा की आवश्यकता है। वर्तमान काल में आज तक भारत में स्त्री-शिक्षा का जो प्रचार हुआ है, उसमें धर्म को ही गौण बनाकर रखा गया है।\endnote{ ६/१८६;}


\section*{बोलचाल की भाषा में शिक्षा}

\addsectiontoTOC{बोलचाल की भाषा में शिक्षा}

हमारे देश में प्राचीन काल से सभी विद्याएँ संस्कृत में ही विद्यमान रहने के कारण, विद्वानों तथा आम जनता के बीच एक अगाध समुद्र-सी दूरी बना रही। बुद्धकाल से लेकर चैतन्य तथा श्रीरामकृष्ण तक जो भी महापुरुष लोक-कल्याण हेतु अवतीर्ण हुए, उन सबने आम जनता की भाषा में जनता को उपदेश दिया है। पाण्डित्य उत्तम है, मगर पाण्डित्य का प्रदर्शन क्या जटिल, अप्राकृतिक तथा कल्पित भाषा को छोड़ अन्य किसी भाषा में नहीं हो सकता? कलात्मक निपुणता क्या बोलचाल की भाषा में नहीं प्रदर्शित की जा सकती? स्वाभाविक भाषा को छोड़ एक अस्वाभाविक भाषा को तैयार करने से क्या लाभ? घर में हम जिस भाषा में बातचीत करते हैं, उसी में हम सारे पाण्डित्यपूर्ण विषयों पर विचार भी करते हैं; तो फिर लिखते समय हम क्यों जटिल भाषा का प्रयोग करने लगते हैं? जिस भाषा में तुम अपने मन में दर्शन या विज्ञान के बारे में सोचते हो, आपस में बातें करते हो, उसी भाषा में क्या दर्शन या विज्ञान नहीं लिखा जा सकता? यदि कहो - नहीं, तो फिर उस भाषा में तुम अपने मन में अथवा कुछ व्यक्तियों के साथ उन सब तत्त्वों पर विचार-विमर्ष कैसे करते हो? जिस भाषा में हम स्वाभाविक रूप से अपने मन के विचार प्रकट करते हैं, जिस भाषा में हम अपना क्रोध, दुःख तथा प्रेम आदि व्यक्त करते हैं, उससे अधिक उपयुक्त भाषा और कौन-सी हो सकती है! अतः हमें उसी भाव, उसी शैली को बनाए रखना होगा। उस भाषा में जितनी शक्ति है, उसमें जैसे थोड़े-से शब्दों में अनेक विचार व्यक्त हो सकते हैं तथा उसे जैसे चाहो, घुमा-फिरा सकते हो, वैसे गुण किसी कृत्रिम भाषा में कदापि नहीं आ सकते। भाषा को ऐसा बनाना होगा - जैसे शुद्ध इस्पात, उसे जैसा चाहो मोड़ लो, पर फिर वैसे का वैसा; कहो तो एक चोट में ही पत्थर को काट दे, पर धार न टूटें। हमारी भाषा, संस्कृत के समान बड़े-बड़े निरर्थक शब्दों का प्रयोग करके और उसके आडम्बर की - केवल उसके इसी एक पहलू की - नकल करते-करते अस्वाभाविक होती जा रही है। भाषा ही तो राष्ट्र की उन्नति का प्रधान लक्षण एवं उपाय है।... \textbf{भाषा विचारों की वाहक है। भाव ही प्रधान है, भाषा गौण है। }... जो भाषा, शिल्प तथा संगीत भावहीन तथा निष्प्राण है; वह किसी भी काम का नहीं। अब लोग समझेंगे कि राष्ट्रीय जीवन में ज्यों-ज्यों स्फूर्ति आती जाएगी; त्यों-त्यों भाषा, शिल्प, संगीत आदि अपने आप भावमय तथा जीवन्त होते जाएँगे; दो प्रचलित शब्दों से जो तात्पर्य प्रकट होंगे, वे दो हजार छँटे हुए विशेषणों से भी नहीं होंगे।\endnote{ १०/१६७-६८;} 

शिक्षक की महानता उसकी सरल भाषा में निहित है।\endnote{ ४/३९१;} 

भाषा सम्बन्धी मेरा आदर्श मेरे गुरुदेव श्रीरामकृष्णदेव की भाषा है, जो थी तो निन्तात बोल-चाल की भाषा, परन्तु साथ ही भावों को परम अभिव्यक्ति प्रदान करनेवाली भी थी।\endnote{ १०/४२} 

\delimiter

\addtoendnotes{\protect\end{multicols}}

\addtocontents{toc}{\protect\par\egroup}

