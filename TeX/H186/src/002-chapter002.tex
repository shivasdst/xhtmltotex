
\chapter{मेरा भारत, अमर भारत }

\indentsecionsintoc

\toendnotes{मेरा भारत, अमर भारत}

\addtoendnotes{\protect\begin{multicols}{3}}

\section*{भारत का इतिहास, संस्कृति तथा आदर्श}

\addsectiontoTOC{भारत का इतिहास, संस्कृति तथा आदर्श}

यदि पृथ्वी पर ऐसा कोई देश है, जिसे हम धन्य पुण्य-भूमि कह सकते हैं,... यदि ऐसा कोई देश है, जहाँ मानव जाति की क्षमा, धैर्य, दया, शुद्धता आदि सद्वृत्तियों का सर्वाधिक विकास हुआ है और यदि ऐसा कोई देश है जहाँ आध्यात्मिकता तथा सर्वाधिक आत्मान्वेषण का विकास हुआ है, तो वह भूमि भारत ही है।\endnote{ ५/५} 

राजनीतिक महानता या सामरिक शक्ति की प्राप्ति करना - न कभी भारत का जीवनोद्देश्य रहा है, न अब है; और याद रखो ऐसा भविष्य में भी कभी नहीं होगा।\endnote{ ५/९;} 

हम जानते हैं कि हिन्दू जाति ने कभी धन को महत्त्व नहीं दिया। धन उसे खूब मिला - दूसरे राष्ट्रों से कहीं अधिक धन मिला, पर हिन्दू जाति ने धन को कभी जीवन का उद्देश्य नहीं माना। भारत युगों तक शक्तिशाली बना रहा, तो भी शक्ति उसका उद्देश्य नहीं बना। उसने अपनी शक्ति का उपयोग कभी अपने देश के बाहर किसी पर विजय प्राप्त करने में नहीं किया। वह अपनी सीमाओं से सन्तुष्ट रहा, इसलिए कभी-भी किसी से युद्ध करने नहीं गया, उसने कभी भी साम्राज्यवादी गौरव को महत्त्व नहीं दिया। धन और शक्ति कभी भी इस देश के आदर्श नहीं बने सके।\endnote{ १०/४;} 

हाँ, मेरे बन्धुओ, यही हमारे देश का गौरवमय भाग्य है कि सुदूर अतीत में, उपनिषदों के काल में ही हमने संसार के समक्ष एक चुनौती रखी थी - \textbf{न प्रजया धनेन त्यागेनैके अमृतत्वमानशुः } - “न तो सन्तानों द्वारा और न ही सम्पत्ति के द्वारा, बल्कि केवल त्याग द्वारा ही अमृतत्व की प्राप्ति होती है।” एक-एक कर कई राष्ट्रों ने इस चुनौती को स्वीकार किया और अपनी शक्ति भर संसार की इस पहेली को कामनाओं के स्तर पर सुलझाने का प्रयत्न किया। वे सभी अतीत काल में असफल रहे - पुराने राष्ट्र शक्ति तथा धन की लोलुपता से उत्पन्न होनेवाले पापाचार और दुःखों के बोझ से दबकर मिट गए, और नये राष्ट्र डगमगाते कदमों से पतन की ओर बढ़ रहे हैं। इस प्रश्न का हल होना अभी भी बाकी है कि शान्ति की जय होगी या युद्ध की, सहिष्णुता की विजय होगी या असहिष्णुता की, भलाई की विजय होगी या बुराई की, शरीर की विजय होगी या बुद्धि की, सांसारिकता की विजय होगी या आध्यात्मिकता की। हमने तो युगों पहले ही इस प्रश्न का हल ढूँढ़ लिया था और सौभाग्य या दुर्भाग्य के बीच हम अपने उसी समाधान पर दृढ़तापूर्वक डटे हैं और चिर काल तक उसी पर अटल रहने को कृतसंकल्प हैं। हमारा समाधान है - असांसारिकता - त्याग। 

मानव जाति का आध्यात्मीकरण - यही भारतीय जीवन-रचना का प्रतिपाद्य विषय है, यही उसके अनन्त संगीत का मूल सुर है, यही उसके अस्तित्व का मेरुदण्ड है, यही उसके जीवन की आधारशिला और उसके अस्तित्व का एकमात्र हेतु है। चाहे तातारों का शासन रहा हो या तुर्कों का, चाहे मुगलों ने राज्य किया हो या अंग्रेजों ने, परन्तु अपने इस सुदीर्घ जीवन-प्रवाह में भारत कभी भी अपने इस मार्ग से विचलित नहीं हुआ है।\endnote{ ९/२९९-३००;} 

पूर्व की नारियों का पश्चिमी मानदण्ड से मूल्यांकन करना उचित नहीं। पश्चिम में नारी पत्नी है, पूर्व में वह माँ है। हिन्दू मातृ-भाव को श्रद्धा की दृष्टि से देखते हैं, यहाँ तक कि संन्यासियों को भी अपनी माँ के सामने अपना मस्तक भूमि पर टेकना पड़ता है। पातिव्रत्य का यहाँ बहुत सम्मान है।\endnote{ १०/२६३;} 

भारत में माँ ही परिवार का केन्द्र और हमारा सर्वोच्च आदर्श है। वह हमारे लिए ईश्वर की प्रतिनिधि है, क्योंकि ईश्वर ब्रह्माण्ड की माँ हैं। एक नारी ऋषि ने ही सबसे पहले ईश्वर की एकता की अनुभूति की और यह सिद्धान्त वेदों की प्रारम्भिक ऋचाओं में व्यक्त किया।... जो प्रार्थना के द्वारा जन्म पाता है, वही आर्य है; और जिसका जन्म कामुकता से होता है, वह अनार्य है।... भारत में यह बात इतनी गम्भीरतापूर्वक मान्य हो गई है कि वहाँ यदि विवाह की परिणति प्रार्थना में न हो, तो हम विवाह में भी व्यभिचार की बात कहते हैं।... यही - सतीत्व ही हमारी जाति का रहस्य है।\endnote{ १०/३०२-०३;} 

संस्कृत भाषा में दो महाकाव्य अत्यन्त प्राचीन हैं।... इन दोनों में प्राचीन आर्यावर्त की सभ्यता और संस्कृति, तत्कालीन आचार-विचार एवं सामाजिक अवस्था लिपिबद्ध है। इन महाकाव्यों में प्राचीनतर ‘रामायण’ है, जिसमें राम के जीवन की कथा निरूपित हुई है।... राम और सीता भारतीय राष्ट्र के आदर्श हैं। सभी बालक-बालिकाएँ, विशेषतः कुमारियाँ सीता की पूजा करती हैं। भारतीय नारी की उच्चतम महत्त्वाकांक्षा यही होती है कि वह सीता के समान शुद्ध, पतिपरायणा और सर्वसहा बने।\endnote{ ७/१३२,१४४;} 

महाभारत शब्द का अर्थ है - महान् भारत देश, अथवा भरत के महान् वंशजों का आख्यान।... यह महाकाव्य भारत में सर्वाधिक लोकप्रिय है, और इसका भारतीय जीवन पर उतना ही प्रभाव पड़ा है, जितना की यूनान देश पर होमर-प्रणीत काव्य का।... धर्मभीरु किन्तु वृद्ध, अन्ध और निर्बल धृतराष्ट्र के हृदय में चलनेवाला पुत्र-प्रेम और कर्तव्य का द्वन्द्व; पितामह भीष्म का उदात्त और उन्नत चरित्र, महाराज युधिष्ठिर का उदार तथा धार्मिक स्वभाव; और उनके चारों बन्धुओं का उन्नत चरित्र, स्वामी-निष्ठा और अप्रतिम वीरता; मानवीय ज्ञान की चरम सीमा प्राप्त श्रीकृष्ण का अद्वितीय व्यक्तित्व; और महासती तपस्विनी रानी गान्धारी, पुत्र-वत्सला कुन्ती, पतिपरायणा और सर्वसहिष्णु द्रौपदी आदि नारियों के चरित्र - जो पुरुषों से किसी भाँति कम नहीं हैं - तथा इस महाग्रन्थ और रामायण के अन्य असंख्य चरित्र-नायक विगत हजारों वर्षों से समस्त हिन्दू जाति की यत्न-संचित राष्ट्रीय सम्पत्ति रहे हैं और उसके विचारों एवं कर्तव्य-अकर्तव्य तथा नीतिसम्बन्धी सिद्धान्तों की आधारशिला हैं। वस्तुतः रामायण और महाभारत प्राचीन आर्यजीवन तथा ज्ञान के दो ऐसे विश्वकोष हैं, जिनमें एक ऐसी उन्नत ‘सभ्यता’ का चित्रण किया गया है, जो मानव जाति को अब भी प्राप्त करनी है।\endnote{ ७/१४८,१६८;} 

इन चरित्रों का अध्ययन करने पर तुमको सहज ही बोध होने लगता है कि भारतीय और पाश्चात्य आदर्शों में कितना महान् अन्तर है।... पश्चिम कहता है - “कर्म करो - कर्म द्वारा अपनी शक्ति दिखाओ।” भारत कहता है - “सहनशीलता द्वारा अपनी शक्ति दिखाओ।” पश्चिम ने इस समस्या का समाधान किया है कि मनुष्य कितनी अधिक वस्तुओं का स्वामी बन सकता है; परन्तु इस प्रश्न का उत्तर भारत ने दिया है कि मनुष्य कितने अल्प में जीवनयापन कर सकता है।\endnote{ ७/१४४;}


\section*{विश्व सभ्यता को भारत का अवदान}

\addsectiontoTOC{विश्व सभ्यता को भारत का अवदान}

संसार हमारे देश का अत्यन्त ऋणी है।\endnote{ ५/५;} 

जब मैं अपने देश के प्राचीन इतिहास का सिंहावलोकन करता हूँ, तो सम्पूर्ण विश्व में मुझे ऐसा कोई भी देश नहीं दिखता, जिसने मानवीय हृदय को उन्नत और सुसंस्कृत बनाने में भारत के समान चेष्टा की हो। इसलिए, न तो मैं अपनी हिन्दू जाति को दोषी ठहराता हूँ और न इसकी निन्दा करता हूँ। मैं तो कहता हूँ - तुमने जो कुछ किया है, अच्छा किया है; पर इससे भी अच्छा करने की चेष्टा करो।’\endnote{ ५/९१-९२;} 

हम हिन्दू तुम्हारे (ईसाई) धर्म की प्राचीनता स्वीकार करने को तैयार हैं; वैसे हमारा धर्म उस समय से करीब तीन सौ वर्ष पूर्व अस्तित्व में आ चुका था, जबकि तुम्हारे धर्म की कल्पना भी नहीं हुई थी। यही बात विज्ञानों के विषय में भी सत्य है। प्राचीन काल में भारत ने ही सर्वप्रथम चिकित्सा-वैज्ञानिक उत्पन्न किए थे और सर विलियम हंटर के मतानुसार इसने विभिन्न रसायनों का पता लगाकर और तुम्हें विरूप कानों और नाकों को सुडौल बनाने की विधि (प्लास्टिक सर्जरी) सिखाकर आधुनिक चिकित्सा विज्ञान में भी योग दिया है। गणित में तो उसने और भी अधिक योगदान किया है; क्योंकि बीजगणित, ज्यामिति, ज्योतिष तथा आधुनिक विज्ञान का गौरव - मिश्र गणित - इन सबका आविष्कार भारत में हुआ; यहाँ तक कि पूरी वर्तमान सभ्यता की मूल आधारशिला-स्वरूप वे दस अंक भी भारत में ही आविष्कृत हुए हैं और उन्हें सूचित करनेवाले शब्द भी वस्तुतः संस्कृत के हैं। 

दर्शन-शास्त्र के क्षेत्र में तो, जैसा कि महान् जर्मन दार्शनिक शापेनहॉवर ने स्वीकार किया है, हम अब भी दूसरे राष्ट्रों से बहुत ऊँचे हैं। संगीत में, भारत ने संसार को सात प्रधान स्वरों और उनके मापन-क्रम सहित अपनी वह अंकन-पद्धति प्रदान की है, जिनका आनन्द हम ईसा से लगभग तीन सौ पचास वर्ष पूर्व से ले रहे थे, जबकि यूरोप में वह ग्यारहवीं सदी में ही पहुँच सकी। भाषा-विज्ञान में, अब हमारी संस्कृत भाषा सभी विद्वानों द्वारा सारी यूरोपीय भाषाओं की आधार के रूप में स्वीकार की जाती है, जो वस्तुतः संस्कृत के अपभ्रंशों के सिवा और कुछ नहीं हैं। 

साहित्य में हमारे महाकाव्य, काव्य तथा नाटक किसी भी भाषा की ऐसी सर्वोच्च रचनाओं के समकक्ष हैं। जर्मनी के महानतम कवि (गेटे) ने शकुन्तला के सार का उल्लेख करते हुए कहा है कि यह ‘स्वर्ग और धरा का सम्मिलन है।’ भारत ने संसार को ईसप की कथाएँ दी हैं। ईसप ने इन्हें एक पुरानी संस्कृत पुस्तक से लिया था। भारत ने ‘सहस्त्र-रजनी-चरित’ (\enginline{Arabian Nights }) दिया है और हाँ, सिन्ड्रैला और बीन स्टाक्स की कहानियाँ भी वहीं (भारत) से आयी हैं। वस्तुओं के उत्पादन में, भारत ने ही सर्वप्रथम रूई तथा बैगनी रंग बनाया। वह रत्नों से सम्बन्धित सभी कलाओं में कुशल था और ‘शुगर’ शब्द तथा उसके द्वारा निर्दिष्ट वस्तु भी भारतीय उत्पादन हैं। अन्त में, उसने शतरंज, ताश और चौपड़ के खेलों का आविष्कार भी किया है। वस्तुतः सभी बातों में भारत की उच्चता इतनी अधिक थी कि यूरोप की भुक्खड़ टोलियाँ उसकी ओर आकृष्ट हुईं, जिसके फलस्वरूप परोक्ष रूप से अमेरिका का भी आविष्कार हुआ।\endnote{ १०/२८४-८५;} 

मैं चुनौती देता हूँ कि कोई भी व्यक्ति भारत के राष्ट्रीय जीवन का कोई भी ऐसा काल मुझे दिखा दे, जिसमें यहाँ सम्पूर्ण विश्व को हिला देने की क्षमता रखनेवाले आध्यात्मिक महापुरुषों का अभाव रहा हो। पर भारत का कार्य आध्यात्मिक है। और यह कार्य रण-भेरी के निनाद से या सैन्यदलों के अभियानों से तो पूरा नहीं किया जा सकता। धरती पर भारत का प्रभाव सर्वदा मृदुल ओस-कणों की भाँति नीरव तथा अव्यक्त रूप से बरसा है, तथापि इस प्रकार वह सर्वदा धरती के सुन्दरतम पुष्पों को विकसित करता रहा है।\endnote{ ९/३००;}


\section*{भारतीय जीवन में धर्म का स्थान}

\addsectiontoTOC{भारतीय जीवन में धर्म का स्थान}

हमारी इस पवित्र मातृभूमि का मेरुदण्ड, आधार-भित्ति या जीवन-केन्द्र एकमात्र धर्म ही है। दूसरे लोग भले ही राजनीति को, व्यापार के बल पर अगाध धनराशि अर्जित करने के गौरव को, वाणिज्य-नीति की शक्ति तथा उसके प्रचार को, अथवा बाह्य स्वाधीनता प्राप्ति के अपूर्व सुख को महत्त्व दें, परन्तु हिन्दू अपने मन में, न तो इनके महत्त्व को मानते हैं और न मानना चाहते हैं। हिन्दुओं के साथ धर्म, ईश्वर, आत्मा, अनन्त और मुक्ति के सम्बन्ध में बातें कीजिए; मैं आपको विश्वास दिलाता हूँ, यहाँ का एक साधारण कृषक भी इन विषयों में अन्य देशों के दार्शनिक कहे जानेवाले व्यक्तियों की अपेक्षा अधिक जानकारी रखता है।... संसार को सिखाने के लिए अब भी हमारे पास कुछ है। इसीलिए सैकड़ों वर्षों के अत्याचार और करीब हजार वर्षों के विदेशी शासन तथा शोषण के बावजूद यह देश जीवित है। इस देश के अब भी जीवित रहने का मुख्य प्रयोजन यह है कि इसने अब भी ईश्वर और धर्म तथा अध्यात्म रूप रत्नकोश का परित्याग नहीं किया है।\endnote{ ५/४४-४५;} 

प्राच्य और पाश्चात्य राष्ट्रों में घूमकर मुझे दुनिया का कुछ अनुभव प्राप्त हुआ है और मैंने सर्वत्र सब देशों का कोई-न-कोई ऐसा आदर्श देखा है, जिसे उस देश का मेरुदण्ड कह सकते हैं। कहीं राजनीति, कहीं समाज-संस्कृति, कहीं मानसिक उन्नति; और इसी प्रकार कुछ-न-कुछ प्रत्येक के मेरुदण्ड का काम करता है। परन्तु हमारी मातृभूमि भारतवर्ष का मेरुदण्ड धर्म - केवल धर्म ही है। धर्म ही के आधार पर, उसी की नींव पर, हमारे राष्ट्रीय जीवन का प्रासाद खड़ा है।\endnote{ ५/७४;} 

रोम की ओर देखो। रोम का ध्येय था साम्राज्य-लिप्सा - शक्ति-विस्तार। ज्योंही उस पर आघात हुआ, त्योंही रोम छिन्न-भिन्न हो गया, विनष्ट हो गया। यूनान की प्रेरणा थी - बुद्धि। ज्योंही उस पर आघात हुआ, त्योंही यूनान समाप्त हो गया। वर्तमान युग में स्पेन आदि वर्तमान देशों का भी यही हाल हुआ है। प्रत्येक राष्ट्र का विश्व के लिए एक ध्येय होता है; और जब तक वह ध्येय आक्रान्त नहीं होता, तब तक असंख्य संकटों के बीच भी वह राष्ट्र जीवित रहता है, पर ज्योंही वह ध्येय नष्ट हुआ, त्योंही वह राष्ट्र ढह जाता है। 

भारत की वह प्राण-शक्ति अब भी आक्रान्त नहीं हुई है। भारतवासियों ने उसका त्याग नहीं किया है और अन्धविश्वासों के बावजूद वह आज भी सबल है। यहाँ भयानक अन्धविश्वास हैं और उनमें से कुछ तो अत्यन्त जघन्य तथा घृणास्पद हैं, परन्तु उनकी चिन्ता मत करो; क्योंकि हमारी राष्ट्रीय जीवन-धारा - हमारा राष्ट्रीय ध्येय अभी भी जीवित है।... 

भारत, मृत्यु की भाँति दृढ़तापूर्वक ईश्वर, केवल ईश्वर से चिपका हुआ है, इसीलिए उसके लिए अभी भी आशा है।\endnote{ १०/४-५;} 

हिन्दू का खाना धार्मिक, पीना धार्मिक, सोना धार्मिक, उसकी चाल-ढाल धार्मिक, विवाह आदि धार्मिक और यहाँ तक कि उसकी चोरी करने की प्रेरणा भी धार्मिक होती है।... इसका एक ही कारण है और वह यह कि इस देश की प्राणशक्ति - इसका ध्येय धर्म है; और चूँकि धर्म पर आघात नहीं हुआ, इसीलिए यह देश अभी तक जीवित है।\endnote{ १०/४;} 

अन्य सभी विषयों को अपने जीवन के इस मूल उद्देश्य के अधीन करना होगा। संगीत में भी सुर-सामंजस्य का यही नियम है। मूल सुर के अनुगत होने से संगीत में ठीक लय आती है। यहाँ भी वही करना होगा। ऐसा भी कोई राष्ट्र हो सकता है, जिसका मूलमंत्र राजनीतिक प्रबलता हो, निश्चय ही धर्म और अन्य सभी विषय उसके जीवन के प्रमुख मूलमंत्र के नीचे दब जाएँगे, पर यहाँ एक ऐसा राष्ट्र है, जिसका प्रधान जीवनोद्देश्य धर्म और वैराग्य है। हिन्दुओं का एकमात्र मूलमंत्र है - यह जगत् क्षणभंगुर और दो दिनों की भ्रान्ति मात्र है।\endnote{ ५/४८-४९;}


\section*{चिरन्तन भारत}

\addsectiontoTOC{चिरन्तन भारत}

पश्चिम में आने से पहले मैं भारत से केवल प्रेम ही करता था, परन्तु अब (विदेश से लौटते समय) मुझे ऐसा प्रतीत होता है कि भारत की धूलि तक मेरे लिए पवित्र है, भारत की हवा तक मेरे लिए पावन है, भारत अब मेरे लिए पुण्यभूमि है, तीर्थस्थान है।\endnote{ ५/२०३;} 

हम सभी भारत के पतन के बारे में काफी कुछ सुनते हैं। कभी मैं भी इस पर विश्वास करता था। मगर आज अनुभव की दृढ़ भूमि पर खड़े होकर, दृष्टि को पूर्वाग्रहों से मुक्त करके और सर्वोपरि अन्य देशों के अतिरंजित चित्रों को प्रत्यक्ष रूप से उचित प्रकाश तथा छायाओं में देखकर, मैं बड़ी विनम्रता के साथ स्वीकार करता हूँ कि मैं गलत था। हे आर्यों के पावन देश! तू कभी पतित नहीं हुआ। राजदण्ड टूटते रहे और फेंके जाते रहे, शक्ति की गेंद एक हाथ से दूसरे में उछलती रही, पर भारत में दरबारों तथा राजाओं का प्रभाव सदा अल्प लोगों को ही छू सका - उच्चतम से निम्नतम तक जनता की विशाल राशि अपनी अनिवार्य जीवन-धारा का अनुसरण करने के लिए मुक्त रही है, और राष्ट्रीय जीवन-धारा कभी मन्द तथा अर्धचेतन गति से और कभी प्रबल तथा प्रबुद्ध गति से प्रवाहित होती रही है। मैं उन बीसों ज्योतिर्मय सदियों की अटूट शृंखला के सम्मुख विस्मयाकुल खड़ा हूँ, जिनके बीच यहाँ-वहाँ एकाध धूमिल कड़ी है, जो अगली कड़ी को और भी अधिक ज्योतिर्मय बना देती है और इनके बीच मेरी यह जन्मभूमि - पशु-मानव को देव-मानव में रूपान्तरित करने के अपने यशोपूरित लक्ष्य को पाने हेतु - जिसे धरती या आकाश की कोई शक्ति रोक नहीं सकती - अपने सहज महिमामय पदक्षेप के साथ अग्रसर हो रही है।\endnote{ ९/२९९;} 

हे मेरे देशवासियो, मेरे मित्रो, मेरे बच्चो, राष्ट्रीय जीवनरूपी यह जहाज लाखों लोगों को जीवनरूपी समुद्र के पार करता रहा है। कई शताब्दियों से इसका यह कार्य चल रहा है और इसकी सहायता से लाखों आत्माएँ, इस सागर के उस पार अमृतधाम में पहुँची हैं। पर आज शायद तुम्हारे ही दोष से इस पोत में कुछ खराबी आ गई है, इसमें एक-दो छेद हो गए हैं, तो क्या तुम इसे कोसोगे? संसार में जिसने तुम्हारा सर्वाधिक उपकार किया है, उसके विरुद्ध खड़े होकर उस पर गाली बरसाना क्या तुम्हारे लिए उचित है? यदि हमारे इस समाज में, इस राष्ट्रीय जीवनरूपी जहाज में छेद है, तो हम तो उसकी सन्तान हैं। आओ चलो, उन छेदों को बन्द कर दें - उसके लिए हँसतेहँसते अपने हृदय का रक्त बहा दें। और यदि हम ऐसा न कर सकें, तो हमें मर जाना ही उचित है। हम अपना भेजा निकालकर उसकी डाट बनाएँगे और जहाज के उन छेदों में भर देंगे। पर मुख से कभी उसकी भर्त्सना न करो? इस समाज के विरुद्ध एक कड़ा शब्द तक न निकालो। इसके अतीत की गौरव-गरिमा के लिए मेरा उस पर प्रेम है। मैं तुम सबको प्यार करता हूँ, क्योंकि तुम देवताओं की सन्तान हो, महिमाशाली पूर्वजों के वंशज हो। तब भला मैं तुम्हें कैसे कोस सकता हूँ? यह असम्भव है। तुम्हारा सब प्रकार से मंगल हो। ऐ मेरे बच्चो, मैं अपनी सारी योजनाएँ तुम्हारे सामने रखने के लिए तुम्हारे पास आया हूँ। यदि तुम उन्हें सुनो, तो मैं तुम्हारे साथ काम करने को तैयार हूँ। पर यदि तुम उनको न सुनो, और मुझे ठुकराकर अपने देश के बाहर भी निकाल दो, तो भी मैं तुम्हारे पास वापस आकर यही कहूँगा, “भाई, हम सब डूब रहे हैं। मैं आज तुम्हारे बीच बैठने आया हूँ। और यदि हमें डूबना है, तो आओ, हम सब साथ ही डूबें, पर एक भी कटु शब्द हमारे ओठों पर न आने पाए।”\endnote{ ५/१२२-३;}


\section*{हे भारत, मत भूलना!}

\addsectiontoTOC{हे भारत, मत भूलना!}

ऐ भारत! तुम मत भूलना कि तुम्हारी स्त्रियों का आदर्श सीता, सावित्री, दमयन्ती हैं; मत भूलना कि तुम्हारे उपास्य सर्वत्यागी उमानाथ शंकर हैं; मत भूलना कि तुम्हारा विवाह, तुम्हारा धन और तुम्हारा जीवन इन्द्रिय-सुख के लिए - अपने व्यक्तिगत सुख के लिए नहीं है; मत भूलना कि तुम जन्म से ही ‘माता’ के लिए बलि-स्वरूप रखे गए हो; मत भूलना कि तुम्हारा समाज उस विराट् महामाया की छाया मात्र है; मत भूलना कि नीच, अज्ञानी, दरिद्र, चमार और मेहतर तुम्हारा रक्त और तुम्हारे भाई हैं। ऐ वीर! साहस का आश्रय लो। गर्व से कहो कि मैं भारतवासी हूँ और प्रत्येक भारतवासी मेरा भाई है, बोलो कि अज्ञानी भारतवासी, निर्धन भारतवासी, ब्राह्मण भारतवासी, चाण्डाल भारतवासी - सब मेरे भाई हैं; तुम भी कटिमात्र में वस्त्र लपेटकर गर्व से पुकारकर कहो कि भारतवासी मेरा भाई है, भारतवासी मेरे प्राण हैं, भारत के देवी-देवता मेरे ईश्वर हैं, भारत का समाज मेरा पालना, मेरे यौवन का उपवन और मेरे वार्धक्य की वाराणसी है। भाई, बोलो कि भारत की मिट्टी मेरा स्वर्ग है, भारत के कल्याण में मेरा कल्याण है, और रात-दिन कहते रहो कि ‘हे गौरीनाथ! हे जगदम्बे! मुझे मनुष्यत्व दो; माँ, मेरी दुर्बलता और कापुरुषता दूर कर दो, मुझे मनुष्य बनाओ।’\endnote{ ९/२२८;} 

\delimiter

\addtoendnotes{\protect\end{multicols}}

\addtocontents{toc}{\protect\par\egroup}

