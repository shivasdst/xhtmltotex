
\chapter{हमारे राष्ट्रीय पतन के कारण }

\indentsecionsintoc

\toendnotes{हमारे राष्ट्रीय पतन के कारण}

\addtoendnotes{\protect\begin{multicols}{3}}

\section*{आम जनता की उपेक्षा - महान् राष्ट्रीय पाप है}

\addsectiontoTOC{आम जनता की उपेक्षा - महान् राष्ट्रीय पाप है}

मैं समझता हूँ कि हमारा सबसे बड़ा राष्ट्रीय पाप जनसमुदाय की उपेक्षा है, और वह भी हमारे पतन का एक कारण है।\endnote{ ४/२६०;} 

वे लोग जो किसान हैं; वे मछुवारे, जुलाहे, जो भारत के नगण्य मनुष्य हैं, विजातिविजित स्वजाति-निन्दित छोटी-छोटी जातियाँ हैं, वे ही लगातार चुपचाप काम किए जा रही हैं, अपने परिश्रम का फल भी नहीं पा रही हैं।\endnote{ ८/१८८-८९;} 

भारत के दरिद्रों, पतितों और पापियों का कोई साथी नहीं, कोई सहायता देनेवाला नहीं - वे चाहे जितनी भी कोशिश क्यों न करें, उनकी उन्नति का कोई उपाय नहीं। वे दिन-पर-दिन डूबते जा रहे हैं। क्रूर समाज उन पर जो लगातार चोटें कर रहा है, उसका अनुभव तो वे खूब कर रहे हैं, पर वे जानते नहीं कि ये चोटें कहाँ से आ रही हैं।\endnote{ १/४०२;} 

भारत की सारी बुराइयों की जड़ है - गरीबों की दुर्दशा।... पुरोहिती शक्ति और विदेशी विजेतागण सदियों से उन्हें कुचलते रहे हैं, जिसके फलस्वरूप भारत के गरीब बेचारे भूल गए हैं कि वे भी मनुष्य हैं।\endnote{ २/३६९;}


\section*{जनता पर अत्याचार - इतिहास के साक्ष्य}

\addsectiontoTOC{जनता पर अत्याचार - इतिहास के साक्ष्य}

प्राचीन भारत अनेक शताब्दियों तक ब्राह्मण और क्षत्रिय - अपनी इन दो प्रधान जातियों की महत्वाकांक्षा की पूर्ति के लिए एक युद्धक्षेत्र बना रहा। 

एक ओर पुरोहितगण आम जनता पर होनेवाले क्षत्रियों के उन अन्यायपूर्ण सामाजिक अत्याचार के विरुद्ध थे - जो प्रजा को अपने धर्मसंगत खाद्य के रूप में देखा करते थे - और दूसरी ओर भारत की एकमात्र शक्ति-सम्पन्न क्षत्रिय जाति ने जनता को पुरोहितों के धार्मिक अत्याचार से बचाने तथा उनके निरन्तर बढ़ते हुए कर्मकाण्डों से छुड़ाने के लिए कमर कसी थी। इसमें क्षत्रियों को कुछ हद तक सफलता भी मिली थी। 

यह संघर्ष हमारे इतिहास के एकदम प्रारम्भिक काल से ही शुरू हुआ था। सारे वेदों से यह स्पष्ट रूप से प्रकट होता है। जब क्षत्रियों तथा ज्ञानकाण्ड के नेता श्रीकृष्ण ने समन्वय का मार्ग दिखलाया, तो कुछ समय के लिए यह विरोध कम हो गया। इसका परिणाम है गीता की शिक्षा - जो दर्शन, उदारता तथा धर्म का सार-स्वरूप है। मगर संघर्ष का कारण तब भी विद्यमान था, अतः उसका परिणाम अनिवार्य था। 

\newpage

निर्धन तथा अशिक्षित जनता पर प्रभुत्व स्थापित करने की महत्त्वाकांक्षा इन दोनों जातियों में विद्यमान थी, अतः संघर्ष फिर भयानक हो उठा। उस काल का थोड़ा-बहुत साहित्य जो उपलब्ध है, वह प्राचीन काल के उसी प्रबल संघर्ष की क्षीण प्रतिध्वनि मात्र है। पर अन्त में क्षत्रियों की विजय हुई, ज्ञान की जीत हुई, स्वाधीनता की जीत हुई; कर्मकाण्डों को नीचा देखना पड़ा और उसका अधिकांश सदा के लिए विदा हो गया। यह वही क्रान्ति थी, जिसे हम बौद्ध सुधारवाद के नाम से जानते हैं। धर्म की दृष्टि से यह कर्मकाण्ड के हाथों से मुक्ति का सूचक है और राजनीति के दृष्टिकोण से यह क्षत्रियों के हाथों पुरोहितों का पराभव सूचित करता है। 

यह विशेष रूप से ध्यान देने योग्य बात है कि प्राचीन भारत ने जिन दो सर्वश्रेष्ठ व्यक्तियों को जन्म दिया था - कृष्ण और बुद्ध - दोनों ही क्षत्रिय हैं। और यह उससे भी अधिक ध्यान देने योग्य बात है कि इन दोनों ही देव-मानवों ने लिंग तथा जाति पर आधारित भेद को न मानकर सबके लिए ज्ञान का द्वार उन्मुक्त कर दिया था। 

बौद्ध धर्म में अद्भुत नैतिक बल विद्यमान होने पर भी, वह अतीव ध्वंसात्मक था। उसकी\break अधिकांश शक्ति नकारात्मक प्रयासों में ही व्यय हो जाने के कारण, उसे अपनी जन्मभूमि में ही अपना विनाश देखना पड़ा; और उसका जो कुछ बाकी रहा, वह जिन अन्धविश्वासों तथा\break कर्मकाण्डों के निवारण हेतु नियोजित किया गया था, उनसे भी सैकड़ों-गुना अधिक बीभत्स था। 

इस प्रकार मनुष्य-देह धारण करनेवाले में सर्वश्रेष्ठ व्यक्ति - स्वयं भगवान बुद्ध द्वारा परिचालित संजीवनी-शक्ति-प्रवाह भी एक दुर्गन्धमय रोग-कीटाणु-पूर्ण क्षुद्र गन्दे जलाशय में बदल गया। तब भारत को भी अनेक शताब्दियों तक प्रतीक्षा करनी पड़ी, जब तक कि भगवान शंकर और उनके कुछ ही समय बाद रामानुज तथा मध्वाचार्य आविर्भूत नहीं हुए।... 

इसके फलस्वरूप भारत में वेदों का पुनरभ्युदय हुआ - वेदान्त का ऐसा प्रबल पुनरुत्थान हुआ, जैसा भारत ने इसके पूर्व कभी नहीं देखा था; यहाँ तक कि गृहस्थाश्रमी भी आरण्यकों के अध्ययन में संलग्न हो गए। 

बौद्ध आन्दोलन में वस्तुतः क्षत्रियगण ही नेता थे और उन्होंने बड़ी संख्या में बौद्ध धर्म स्वीकार किया था। सुधार तथा धर्म-परिवर्तन के उत्साह में संस्कृत भाषा उपेक्षित हो गयी और केवल लोक-भाषाओं का ही विकास होने लगा। अधिकांश क्षत्रिय वैदिक साहित्य तथा संस्कृत शिक्षा के क्षेत्र से अलग हो गये। अतः दक्षिणात्यों से यह जो सुधारतरंग उत्थित हुई, उससे कुछ सीमा तक केवल पुरोहितों का ही उपकार हुआ, पर भारत की शेष कोटि-कोटि जनता के पैरों में उसने पहले से भी अधिक शृंखलाएँ डाल दीं। 

क्षत्रियगण सदा से ही भारत का मेरुदण्ड रहे हैं, अतः वे ही विज्ञान और स्वतंत्रता के चिर रक्षक हैं। देश से अन्धविश्वासों को हटा देने के लिए चिरकाल से ही उनकी वाणी प्रतिध्वनित हुई है और भारत के इतिहास के आदि से अन्त तक पुरोहितों के अत्याचार से साधारण जनता की रक्षा करने के लिए वे स्वयं एक अभेद्य दीवार की भाँति खड़े रहे हैं। 

जब उनमें से अधिकांश घोर अज्ञानता में निमग्न हो गए और बचे-खुचों ने मध्य एशिया की जंगली जातियों के साथ रोटी-बेटी का सम्बन्ध स्थापित करके भारत में पुरोहितों की शक्ति को सुदृढ़ करने के लिए तलवार उठा ली, तब भारत के पाप का प्याला लबालब भर गया और भारत-भूमि एकदम नीचे डूब गयी। इससे भारत का उद्धार तब तक नहीं होगा, जब तक कि क्षत्रियगण स्वयं न जागेंगे तथा स्वयं को मुक्त करके बाकी लोगों के पैरों से जंजीरों को न खोल देंगे।\endnote{ ९/३५३-५६;} 

राष्ट्र डूब रहा है। करोड़ों प्राणियों का शाप हमारे सिर पर है,... जिन असंख्य करोड़ों लोगों को हमने अद्वैतवाद का तत्त्व सुनाया और जिनसे तीव्र घृणा की, जिनके विरोध में हमने लोकाचार का आविष्कार किया, जिन्हें हमने मुख से तो कहा कि सब बराबर हैं, सब एक ब्रह्म ही हैं, परन्तु इस उक्ति को काम में लाने का तिल मात्र भी प्रयत्न नहीं किया।\endnote{ ५/३२१;}


\section*{धर्म के नाम पर जन-शोषण}

\addsectiontoTOC{धर्म के नाम पर जन-शोषण}

पृथ्वी पर ऐसा कोई धर्म नहीं, जो हिन्दू धर्म जैसा इतने उच्च स्वर में मानवता के गौरव का उपदेश देता हो और पृथ्वी पर ऐसा कोई धर्म नहीं, जो हिन्दू धर्म के समान गरीबों और निम्न जातिवालों का गला ऐसी क्रूरता से घोटता हो।\endnote{ १/४०३;} 

यह रोना-धोना मचा है कि हम बड़े गरीब हैं, पर गरीबों की सहायता के लिए कितनी दानशील संस्थाएँ हैं? भारत के करोड़ों अनाथों के लिए कितने लोग रोते हैं? हे भगवान! हम लोग भी क्या मनुष्य हैं? तुम लोगों के घरों के चारों ओर जो पशुवत् भंगी-डोम हैं! उनकी उन्नति के लिए क्या कर रहे हो? उनके मुख में एक ग्रास अन्न देने के लिए क्या करते हो? बताओ न। तुम उन्हें छूते भी नहीं और उन्हें ‘दुर-दुर’ कहकर भगा देते हो! क्या हम मनुष्य हैं? वे हजारों साधु-ब्राह्मण भारत की नीच-दलित-दरिद्र जनता के लिए क्या कर रहे हैं? बस ‘मत छू’, ‘मत छू’ की रट ही तो लगाते हैं! ऐसे सनानत धर्म कैसा सन्यानाश कर डाला है! अब धर्म कहाँ है? केवल छुआछूत में - मुझे छुओ मत, छुओ मत!\endnote{ २/३१६;} 

पुरोहिती प्रपंच ही भारत के पतन का मूल कारण है। मनुष्य अपने भाई को पतित बनाकर क्या स्वयं पतित होने से बच सकता है?... स्मरण रखो, तुम्हारे पूर्वजों द्वारा आविष्कृत सत्यों में सर्वश्रेष्ठ सत्य है - इस ब्रह्माण्ड का एकत्व। क्या कोई व्यक्ति स्वयं का किसी प्रकार अनिष्ट किए बिना दूसरों को हानि पहुँचा सकता है? ब्राह्मण तथा क्षत्रियों के ये ही अत्याचार चक्रवृद्धि ब्याज के साथ अब स्वयं उन्हीं के सिर पर आ पड़े हैं और यह हजारों वर्ष की पराधीनता तथा पतन निश्चय ही उन्हीं के कर्मों के अनिवार्य फल है।\endnote{ ९/३५६;}


\section*{एक अन्य कारण - शिक्षा पर एकाधिकार}

\addsectiontoTOC{एक अन्य कारण - शिक्षा पर एकाधिकार}

भारत के सत्यानाश का मुख्य कारण यही है कि देश की सम्पूर्ण विद्या-बुद्धि; राज-शासन और दम्भ के बल से मुट्ठी भर लोगों के एकाधिकार में रखी गयी।\endnote{ ६/३१०-११;} 

पूर्व तथा पश्चिम में सारा अन्तर यह है कि उनमें राष्ट्रीयता की भावना है, हम लोगों में नहीं है, अर्थात् वहाँ सभ्यता तथा शिक्षा का प्रसार व्यापक है, आम जनता में व्याप्त है। भारत और अमेरिका में उच्च वर्ग के लोग समान हैं, परन्तु दोनों देशों के निम्न वर्गों में जमीन-आसमान का भेद है। अंग्रेजों के लिए भारत को जीतना इतना आसान क्यों सिद्ध हुआ? इसलिए कि वे एक राष्ट्र हैं, हम नहीं। जब हमारा कोई महापुरुष गुजर जाता है, तो अगले महापुरुष के लिए हमें सैकड़ों वर्ष बैठे रहना पड़ता है; और ये अमेरिकी लोग उनका सर्जन उसी अनुपात में कर सकते हैं, जिस अनुपात में उनकी मृत्यु होती है।... यहाँ महापुरुषों का अभाव है। ऐसा क्यों? क्योंकि महापुरुषों के चुनाव के लिए उनके पास बहुत बड़ा क्षेत्र है, जबकि हमारे पास बहुत ही छोटा। तीन, चार या छह करोड़ लोगों के राष्ट्रों की अपेक्षा तीस करोड़ लोगों के राष्ट्र के पास अपने महापुरुषों के चुनाव के लिए क्षेत्र सबसे छोटा है। क्योंकि उन राष्ट्रों में शिक्षित नर-नारियों की संख्या बहुत अधिक है।... यही हमारे देश का बहुत बड़ा दोष है और हमें इसे दूर करना ही होगा।\endnote{ २/३६५;}


\section*{लोगों को कहा गया कि वे कुछ भी नहीं है}

\addsectiontoTOC{लोगों को कहा गया कि वे कुछ भी नहीं है}

सैकड़ों वर्षों से लोगों को उनकी हीन अवस्था का ही ज्ञान कराया गया है। उनसे कहा गया है कि वे कुछ नहीं है। सारे जगत् में सर्वत्र जन-साधारण से कहा गया है कि तुम लोग मनुष्य ही नहीं हो। शताब्दियों से इस प्रकार डराये जाने के कारण वे बेचारे सचमुच ही करीब-करीब पशुत्व को प्राप्त हो गए हैं। उन्हें कभी आत्मतत्त्व के विषय में सुनने का मौका नहीं दिया गया।\endnote{ ५/११९;}


\section*{आलस्य और नीचता}

\addsectiontoTOC{आलस्य और नीचता}

स्वयं कुछ करना नहीं और यदि दूसरा कोई कुछ करना चाहे, तो उसकी हँसी उड़ाना भारतवासियों का एक महान् दोष है और इसी से भारतवर्ष का सर्वनाश हुआ है। हृदयहीनता तथा उद्यम का अभाव सब दुःखों का मूल है। अतः इन दोनों को त्याग दो। किसके अन्दर क्या है, प्रभु के बिना कौन जान सकता है? सबको मौका मिलना चाहिए। आगे प्रभु की इच्छा। सब पर समान स्नेह रखना बड़ा कठिन है; किन्तु उसके बिना मुक्ति नहीं मिल सकती।\endnote{ ४/३८०;}


\section*{एक और कारण - संकीर्णता}

\addsectiontoTOC{एक और कारण - संकीर्णता}

भारत के दुःख-दारिद्र्य तथा पतन का प्रधान कारण यह है कि उसने घोंघे की तरह अपने अंगों को समेटकर अपने कार्य-क्षेत्र को संकुचित कर लिया और अन्य देशों के सत्य-पिपासुओं के लिए अपने जीवनदायी रत्नों का भण्डार नहीं खोला। हमारे पतन का एक अन्य प्रधान कारण यह है कि हमने बाहर जाकर दूसरे राष्ट्रों से अपनी तुलना नहीं की।\endnote{ ५/२१०;} 

कोई भी मनुष्य, कोई भी राष्ट्र, दूसरों से घृणा करते हुए जी नहीं सकता। भारत के भाग्य का निपटारा उसी दिन हो चुका था, जिस दिन उसने ‘म्लेच्छ’ शब्द का आविष्कार किया और दूसरे राष्ट्रों से अपना नाता तोड़ लिया।\endnote{ ३/३४;} 

\vskip 3pt

प्राचीन या नवीन तर्कजाल इसे चाहे जैसे भी ढँकने की चेष्टा करे, पर उस सामान्य नैतिक नियम के अनुसार कि कोई भी बिना अपने को अधःपतित किए दूसरों से घृणा नहीं कर सकता - इसका अनिवार्य फल यह हुआ कि जो आर्य जाति सभी प्राचीन जातियों में सर्वश्रेष्ठ थी, उसका नाम पृथ्वी की जातियों में एक सामान्य घृणासूचक शब्द-सा हो गया है।\endnote{ ३/३३१;} 

\vskip 2pt


\section*{संगठन का अभाव}

\addsectiontoTOC{संगठन का अभाव}

तुम लोगों में संगठन की शक्ति का एकदम अभाव है। वही अभाव सब अनर्थों का मूल है। मिल-जुलकर कार्य करने के लिए कोई भी तैयार नहीं। संगठन के लिए सर्वप्रथम आज्ञा-पालन की आवश्यकता है।\endnote{ ४/३११;} 

\vskip 3pt

हमारे उपनिषद् कितने ही महत्त्वपूर्ण क्यों न हों, अन्य देशों के साथ तुलना में हम अपने पूर्वज ऋषियों पर कितना ही गर्व क्यों न करें, परन्तु मैं तुम लोगों से स्पष्ट शब्दों में कहता हूँ कि हम दुर्बल हैं, अत्यन्त दुर्बल हैं। प्रथम तो है हमारी शारीरिक दुर्बलता। यह शारीरिक दुर्बलता कम-से-कम हमारे एक तिहाई दुःखों का कारण है। हम आलसी हैं, हम कार्य नहीं कर सकते; हम पारस्परिक एकता स्थापित नहीं कर सकते, हम एक-दूसरे से प्रेम नहीं करते, हम बड़े स्वार्थी हैं, हम तीन मनुष्य एकत्र होते ही आपस में घृणा करते हैं, ईर्ष्या करते हैं।... हम बातें बहुत कहते हैं, परन्तु उनके अनुसार कभी कार्य नहीं करते। इस प्रकार तोते के समान बातें करना हमारा अभ्यास हो गया है - आचरण में हम बहुत पिछड़े हुए हैं। इसका कारण क्या है? शारीरिक दुर्बलता। दुर्बल मस्तिष्क कुछ नहीं कर सकता, हमें अपने मस्तिष्क को सबल बनाना होगा।\endnote{ ५/१३६-३७} 

\vskip 2pt


\section*{एक अन्य पाप - नारी की अवहेलना}

\addsectiontoTOC{एक अन्य पाप - नारी की अवहेलना}

भारत में दो बड़ी बुरी बातें हैं। स्त्रियों का तिरस्कार और गरीबों को जाति-भेद के द्वारा पीसना।\endnote{ ४/३२३;} 

\vskip 3pt

क्या तुम ‘शाक्त’ शब्द का अर्थ जानते हो?... जो ईश्वर को समग्र जगत् में महाशक्ति के रूप में जानता है और स्त्रियों में इस शक्ति का प्रकाश मानता है, वही शाक्त है।... महर्षि मनु ने भी कहा है कि जिन परिवारों में स्त्रियों से अच्छा बर्ताव किया जाता है और वे सुखी हैं, उन पर देवताओं की कृपा रहती है। यहाँ (पश्चिम) के पुरुष ऐसा ही करते हैं और इसी कारण सुखी, विद्वान्, स्वतंत्र और उद्योगी हैं। दूसरी ओर हम लोग स्त्रियों को नीच, अधम तथा अतिहेय कहते हैं, इसीलिए हम लोग पशुवत्, दास, उद्यमहीन तथा दरिद्र हो गए।\endnote{ २/३१५;} 

क्या कारण है कि संसार के सब देशों में हमारा देश ही सबसे अधम है, शक्तिहीन है और पिछड़ा हुआ है? इसका कारण यही है कि हमारे यहाँ शक्ति का अनादर होता है।\endnote{ २/३६१} 

\delimiter

\addtoendnotes{\protect\end{multicols}}

\addtocontents{toc}{\protect\par\egroup}

