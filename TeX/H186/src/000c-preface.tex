
\newpage

\begin{center}
\dedication{\textbf{स्वामी वीरेश्वरानन्दजी की\general{\\[5pt]} स्मृति में समर्पित }}
\end{center}

\newpage

\chapter*{प्रकाशकीय }

‘मेरा भारत, अमर भारत’ यह पुस्तक पाठकों के सम्मुख रखते हमें अत्यन्त हर्ष हो रहा है। 

कोलकाता के रामकृष्ण मिशन इंस्टीट्यूट ऑफ कल्चर द्वारा भारत-विषयक स्वामीजी के सन्देश पर आधारित ‘आमार भारत अमर भारत’ नामक बंगला पुस्तक सर्वप्रथम १९८६ ई. में प्रकाशित हुआ। उसके बाद इसके अनेक पुनर्मुद्रण हुए। अनेक लोगों की माँग पर १९९३ ई. में इसका \enginline{My India, The India Eternal } शीर्षक के साथ अंग्रेजी संस्करण भी प्रकाशित किया गया। इन दोनों भाषाओं में इस ग्रन्थ की अब तक तीन लाख से भी अधिक प्रतियाँ मुद्रित हो चुकी हैं। 

पुस्तक की लोकप्रियता का कारण सहज ही समझ में आ जाता है - स्वामी विवेकानन्द के जिन उद्धरणों का इसमें संकलन किया गया है, वे पाठकों को साहस, आशा तथा बल की नई ऊँचाइयों तक उन्नत करती हैं। रवीन्द्रनाथ ठाकुर ने कहा है, “उनमें (विवेकानन्द) सब कुछ सकारात्मक है, नकारात्मक कुछ भी नहीं।’ यह पुस्तक इसी उक्ति का एक प्रमाण तथा निदर्शन प्रस्तुत करता है। 

रामकृष्ण मठ तथा रामकृष्ण मिशन के दशम अध्यक्ष स्वामी वीरेश्वरानन्दजी ने इस ग्रन्थ के लिए प्रेरणा प्रदान की थी। यह ग्रन्थ सविनय उन्हीं को समर्पित है। 

इसका हिन्दी अनुवाद रामकृष्ण मिशन विवेकानन्द आश्रम, रायपुर से प्रकाशित होनेवाली ‘विवेक-ज्योति’ मासिक पत्रिका के फरवरी २००८ से धारावाहिक रूप से मुद्रित हो रहा है~। इसकी लोकप्रियता को देखते हुए वहीं से इसका हिन्दी रूपान्तरण ग्रन्थाकार प्रकाशित किया जा रहा है। आशा है यह भी अपने पूर्ववर्तियों की भाँति ही लोकप्रिय होगा। ग्रन्थ के हिन्दी अनुवाद के सम्पादन तथा उत्तरार्ध के तीन अंशों - ‘स्वामी विवेकानन्द का सन्देश’, ‘स्वामी विवेकानन्द के जीवन की स्मरणीय घटनाएँ’ तथा ‘मनीषियों की दृष्टि में स्वामी विवेकानन्द’ के अनुवाद के लिए हम उक्त पत्रिका के सम्पादक स्वामी विदेहात्मानन्द के प्रति विशेष आभारी हैं। इसके प्रूफ-संशोधन आदि में उसी आश्रम के स्वामी प्रपत्त्यानन्द से हमें विशेष सहायता मिली है। 

\bigskip

\noindent नागपुर\hfill{\large\textbf{- प्रकाशक}}

\noindent दि. २० जुलाई २००९\\स्वामी रामकृष्णानन्द जयन्ती 

