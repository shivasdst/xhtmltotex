
\chapter{नारी-शक्ति और उसका जागरण }

\indentsecionsintoc

\toendnotes{नारी-शक्ति और उसका जागरण}

\addtoendnotes{\protect\begin{multicols}{3}}

\section*{भारतीय नारी का आदर्श}

\addsectiontoTOC{भारतीय नारी का आदर्श}

हे भारत! तुम मत भूलना कि तुम्हारी स्त्रियों का आदर्श सीता, सावित्री, दमयन्ती हैं।\endnote{ ९/२२८;} 

भारतवर्ष में स्त्रीत्व मातृत्व का ही बोधक है, मातृत्व में महानता, निःस्वार्थता, कष्ट-सहिष्णुता और क्षमाशीलता का भाव निहित है।\endnote{ १/३११;} 

पूर्व की स्त्रियों को पश्चिमी मानदण्ड से जाँचना उचित नहीं है। पश्चिम में स्त्री पत्नी है, पूर्व में वह माँ है।\endnote{ १०/३६३;} 

भारत में नारी ईश्वर की साक्षात् अभिव्यक्ति है तथा उसका सारा जीवन इस विचार से ओतप्रोत है कि वह माँ है और पूर्ण माँ बनने के लिए उसे पतिव्रता रहना आवश्यक है।\endnote{ १०/३६३;} हमारी स्त्रियाँ इतनी विदुषी नहीं, किन्तु वे अधिक पवित्र हैं। प्रत्येक स्त्री के लिए अपने पति को छोड़ अन्य कोई भी पुरुष पुत्र जैसा होना चाहिए।\endnote{ १०/२१६;} 

प्रत्येक पुरुष के लिए अपनी पत्नी को छोड़कर अन्य सब स्त्रियाँ माता के समान होनी चाहिए। जब मैं अपने आसपास देखता हूँ और (पश्चिमी देशों में) नारी-भक्ति के नाम पर जो कुछ चलता है, वह देखता हूँ, तो मेरी आत्मा ग्लानि से भर उठती है। जब तक तुम्हारी स्त्रियाँ यौन-सम्बन्धी प्रश्न की उपेक्षा करके सामान्य मानवता के स्तर पर नहीं मिलती, उनका सच्चा विकास नहीं होगा। तब तक वे केवल खिलौना बनी रहेंगी, और कुछ नहीं। यही सब तलाक का कारण है। तुम्हारे पुरुष नीचे झुकते हैं और कुर्सी देते हैं, फिर अगले ही क्षण वे प्रशंसा में कहना शुरू करते हैं - “देवीजी, तुम्हारी आँखें कितनी सुन्दर हैं!” उन्हें यह करने का क्या अधिकार है? एक पुरुष इतना साहस क्यों कर पाता है और तुम स्त्रियाँ इसकी अनुमति कैसे दे सकती हो? ऐसी चीजों से मानवता के अधमतर पक्ष का विकास होता है। ये हमें उच्चतर आदर्शों की ओर नहीं ले जातीं। 

हम स्त्री है या पुरुष हैं - हमें यह न सोचकर यह सोचना चाहिए कि हम मानव हैं; जो एक-दूसरे की सहायता करने और एक-दूसरे के काम आने के लिए जन्मे हैं।\endnotemark[\theendnote]


\section*{नारियों की उपेक्षा ही भारत के पतन का कारण}

\addsectiontoTOC{नारियों की उपेक्षा ही भारत के पतन का कारण}

भारत में दो बड़ी बुरी बातें हैं। स्त्रियों का तिरस्कार और गरीबों को जाति-भेद के द्वारा पीसना।\endnote{ ४/३२३;} 

यहाँ (अमेरिका) के पुरुष अपनी स्त्रियों के साथ अच्छा व्यवहार करते हैं; इसीलिए ये सुखी, विद्वान्, स्वतंत्र और उद्योगी हैं। दूसरी ओर हम भारत के लोग स्त्री-जाति को नीच, अधम, परम हेय तथा अपवित्र कहते हैं। अतः हम लोग पशु, दास, उद्यमहीन और दरिद्र हो गए।\endnote{ २/३१५;} 

\newpage

शक्ति के बिना जगत् का उद्धार नहीं हो सकता। क्या कारण है कि संसार के सब देशों में हमारा देश ही सबसे अधम है, शक्तिहीन है, पिछड़ा हुआ है? इसका कारण यही है कि वहाँ शक्ति का अनादर होता है।\endnote{ २/३६१;}


\section*{स्त्री-पुरुष के बीच भेदभाव अनुचित}

\addsectiontoTOC{स्त्री-पुरुष के बीच भेदभाव अनुचित}

यह समझना कठिन है कि इस देश में पुरुष तथा स्त्रियों के बीच इतना भेद क्यों किया जाता है! वेदान्त शास्त्र में तो कहा है कि एक ही चैतन्य सत्ता सर्वभूतों में विद्यमान है। तुम लोग स्त्रियों की निन्दा ही करते हो। उनकी उन्नति के लिए तुमने क्या किया है, बोलो तो?\endnote{ ६/१८१;} 

आत्मा में भी क्या कहीं लिंग-भेद है? स्त्री और पुरुष का भाव दूर करो, सब आत्मा है।\endnote{ ३/३०९;}


\section*{नारियों का सम्मान हो}

\addsectiontoTOC{नारियों का सम्मान हो}

क्या तुम अपने देश की महिलाओं की अवस्था सुधार सकते हो? तभी तुम्हारे कल्याण की आशा की जा सकती है, नहीं तो तुम ऐसे ही पिछड़े पड़े रहोगे।\endnote{ २/३१६;} 

स्त्री में जो दिव्यता निहित है, उसे हम कभी ठग नहीं सकते। वह न कभी ठगी गयी है, न ठगी जाएगी। यह सदैव अपना प्रभाव जमा लेती है तथा सदैव ही अचूक रूप से बेईमानी तथा ढोंग को पहचान लेती है; और सत्य के तेज, आध्यात्मिकता के आलोक तथा पवित्रता की शक्ति का उसे निश्चित रूप से पता चल जाता है। यदि हम वास्तविक धर्मलाभ करना चाहते हैं, तो ऐसी पवित्रता अनिवार्य है।\endnote{ ७/२५७;} 

स्त्रियों की दशा सुधारे बिना जगत् के कल्याण की कोई सम्भावना नहीं है। पक्षी के लिए एक पंख से उड़ना सम्भव नहीं है। इसीलिए रामकृष्ण-अवतार में ‘स्त्री-गुरु’ को ग्रहण किया गया है, इसीलिए उन्होंने स्त्री-वेश तथा स्त्री-भाव\footnote{ श्रीरामकृष्ण ने अपने मन से पुरुष-नारी का भेद दूर करने हेतु, कुछ काल के लिए नारियों का वेश धारण करके, स्वयं को एक नारी मानते हुए साधना की थी।} में साधना की और इसी कारण उन्होंने नारियों के मातृभाव में जगदम्बा के रूप का दर्शन करने का उपदेश दिया। 

अतः मेरा पहला प्रयत्न स्त्रियों के लिए मठ स्थापित करने का है। इस मठ से गार्गी और मैत्रेयी और उनसे भी अधिक योग्यता रखनेवाली स्त्रियों की उत्पत्ति होगी।\endnote{ ४/३१७;} 

श्रीरामकृष्ण को देखा है - वे सभी स्त्रियों के प्रति मातृभाव रखते थे - चाहे वह किसी भी जाति की या कैसी भी स्त्री क्यों न हो। मैंने देखा है न, इसीलिए इसे समझकर तुम लोगों को वैसा ही बनने को कहता हूँ और बालिकाओं के लिए गाँव-गाँव में पाठशालाएँ खोलकर उन्हें शिक्षित करने के लिए कहता हूँ। स्त्रियाँ जब शिक्षित होंगी, तभी तो उनकी सन्तानों द्वारा देश का मुख उज्ज्वल होगा और देश में विद्या, ज्ञान, शक्ति, भक्ति जाग उठेगी।\endnote{ ६/१८५;} 

किस शास्त्र में ऐसी बात है कि स्त्रियाँ ज्ञान-भक्ति की अधिकारिणी नहीं होंगी? भारत का अधःपतन तभी से शुरू हुआ, जब पुरोहितों ने ब्राह्मणेतर जातियों को वेदपाठ का अनधिकारी घोषित किया और साथ ही स्त्रियों के सारे अधिकार छीन लिए। नहीं तो, वेदों-उपनिषदों के युग में मैत्रेयी, गार्गी आदि प्रातःस्मरणीय स्त्रियाँ ब्रह्म-विचार में ऋषितुल्य हो गयी हैं। हजार वेदज्ञ ब्राह्मणों की सभा में गार्गी ने गर्व के साथ याज्ञवल्क्य को ब्रह्मज्ञान में शास्त्रार्थ के लिए चुनौती दी थी। इन आदर्श विदुषी स्त्रियों को जब उन दिनों अध्यात्मज्ञान का अधिकार था, तो फिर आज की स्त्रियों को वह अधिकार क्यों न रहेगा? एक बार जो हुआ है, वह फिर अवश्य हो सकता है। इतिहास की पुनरावृत्ति हुआ करती है। स्त्रियों की पूजा करके ही सभी राष्ट्र बड़े बने हैं। जिस देश में, जिस राष्ट्र में, स्त्रियों की पूजा नहीं होती; वह देश, वह राष्ट्र, न कभी बड़ा बन सका और न कभी बन सकेगा। तुम्हारे देश का जो इतना अधःपतन हुआ, उसका प्रधान कारण है - इन शक्ति-मूर्तियों का अपमान। मनु ने कहा है - \textbf{यत्र नार्यस्तु पूज्यन्ते रमन्ते तत्र देवताः। यत्रैतास्तु न पूज्यन्ते सर्वास्तत्राफलाः क्रियाः॥ } - “जहाँ स्त्रियों का आदर होता है, वहाँ देवता प्रसन्न होते है और जहाँ उनका सम्मान नहीं होता है, वहाँ सारे कार्य तथा प्रयत्न असफल हो जाते हैं।” जहाँ स्त्रियों का सम्मान नहीं होता, वे दुखी रहती हैं; उस परिवार की, उस देश की उन्नति की आशा नहीं की जा सकती। अतः पहले उन्हें उठाना होगा।\endnote{ ६/१८१-८२;}


\section*{शिक्षा द्वारा नारियों की समस्याओं का समाधान}

\addsectiontoTOC{शिक्षा द्वारा नारियों की समस्याओं का समाधान}

निश्चय ही... उनकी समस्याएँ बहुत-सी और गम्भीर हैं, पर उनमें एक भी ऐसी नहीं है, जो जादू भरे ‘शिक्षा’ शब्द से हल न की जा सकती हो।\endnote{ ४/२६८;} पहले अपने स्त्रियों को शिक्षा दो और उन्हें उनकी स्थिति पर छोड़ दो, तब वे तुम्हें स्वयं बताएँगी कि उनके लिए कौन-से सुधार आवश्यक हैं।\endnote{ १/२९७;} 

इस प्रकार की शिक्षा प्राप्त होने पर स्त्रियाँ अपनी समस्याएँ स्वयं ही हल कर लेंगी। अब तक तो उन्होंने केवल असहाय अवस्था में दूसरों पर आश्रित होकर जीवन बिताना और जरा भी अनिष्ट या संकट की आशंका होने पर आँसू बहाना ही सीखा है। परन्तु अब दूसरी बातों के साथ-साथ उन्हें बहादुर भी बनना होगा। आज के जमाने में उनके लिए आत्मरक्षा सीखना भी बहुत जरूरी हो गया है। देखो, झाँसी की रानी कैसी महान् थीं।\endnote{ ८/२७७;} 

हर नारी ऐसा कुछ सीखे, जिसके द्वारा वह जरूरत पड़ने पर अपना जीविकोपार्जन कर सके।\endnote{ श. ए. २६३;} 

स्त्री-शिक्षा कैसी हो; प्रथम तो - हिन्दू स्त्री के लिए सतीत्व का अर्थ समझना सरल ही है; क्योंकि यह उसकी विरासत है, परम्परागत सम्पत्ति है। अतः सबसे पहले, भारतीय नारी के हृदय में यह ज्वलन्त आदर्श सर्वोपरि रहे, ताकि वे इतनी दृढ़चरित्र बन जाएँ कि चाहे विवाहित हों या कुमारी, जीवन की हर अवस्था में, अपने सतीत्व से तिल भर भी डिगने की अपेक्षा, निडर भाव से जीवन की आहुति दे दें। अपने आदर्श की रक्षा के लिए अपने जीवन की भी बलि दे देना - यह क्या कम वीरता है?... साथ ही महिलाओं को विज्ञान तथा अन्य विषय भी सिखाए जाएँ, जिनसे न केवल उनका, अपितु अन्य लोगों का भी हित हो। यह जानकर कि परोपकार के लिए यह करना है, भारतीय नारी प्रसन्नता से और सरलतापूर्वक कोई भी विषय सीख लेगी।\endnote{ ८/२७७-८;} 

“मैं बहुत चाहता हूँ कि हमारी स्त्रियों में अमेरिकी नारियों जितनी बौद्धिकता होती, परन्तु मैं उसे चारित्रिक पवित्रता के मूल्य पर कदापि नहीं चाहूँगा।\endnote{ १०/२१६;}


\section*{सीता: भारतीय नारी का आदर्श}

\addsectiontoTOC{सीता : भारतीय नारी का आदर्श}

सीता भारत की आदर्श हैं, भारतीय भावों की प्रतिनिधि हैं, मूर्तिमती भारतमाता हैं। नहीं पता कि सीता सचमुच जन्मी थीं या नहीं, रामायण की कथा किसी ऐतिहासिक तथ्य पर आधारित है या कपोल-कल्पित। परन्तु यह सत्य है कि हजारों वर्षों से सीता का चरित्र ही भारतीय राष्ट्र का आदर्श रहा है। ऐसी अन्य कोई पौराणिक कथा नहीं है, जिसने सीता के चरित्र की भाँति पूरे भारतीय राष्ट्र को आच्छादित और प्रभावित किया हो, उसके जीवन में इतनी गहराई तक प्रवेश किया हो, जो देश की नस-नस में, उसके रक्त की एक-एक बूँद में इतनी प्रवाहित हुई हो। भारत में जो कुछ पवित्र है, विशुद्ध है, जो कुछ पावन है, उन सबका ‘सीता’ शब्द से बोध हो जाता है। नारी में नारीजनोचित जो भी गुण माने गए हैं, ‘सीता’ शब्द उन सबका परिचायक है। इसीलिए ब्राह्मण जब किसी कुल-वधू को आशीर्वाद देते हैं, तो कहते हैं - ‘सीता बनो’; जब किसी बालिका को आशीर्वाद देते हैं, तो कहते हैं ‘सीता बनो’। वे सब (हिन्दू) सीता की सन्तान हैं - जीवन में उनका एकमेव यही प्रयत्न होता है कि वे सीता बनें - सीता-सी शुद्ध, धीर, सर्वसहा, पति-परायणा और पतिव्रता बनें।... सीता इस भारतीय आदर्श की सच्ची प्रतिनिधि हैं। उनके हृदय में अत्याचारों के प्रतिशोध का विचार तक नहीं आया।\endnote{ ७/१४४-४५;} 

सीता के विषय में क्या कहा जाए! संसार के सम्पूर्ण प्राचीन साहित्य को छान डालो और मैं तुम्हें विश्वास दिलाता हूँ कि तुम संसार के भावी साहित्य का भी मन्थन कर सकते हो, परन्तु उसमें भी तुम सीता के समान दूसरा चरित्र नहीं पा सकोगे। सीताचरित्र अद्वितीय है। यह चरित्र सदा के लिए एक ही बार चित्रित हुआ है। राम तो कदाचित् कई हो गए हैं, पर सीता दूसरी नहीं हुईं। भारतीय स्त्रियों को जैसा होना चाहिए, सीता उनके लिए आदर्श हैं। स्त्री-चरित्र के जितने भारतीय आदर्श हैं, वे सब सीता के ही चरित्र से उद्भूत हैं और सारे भारतवर्ष में हजारों वर्षों से वे स्त्री-पुरुष-बालकों की पूजा पा रही हैं। महा-महिमामयी सीता, स्वयं पवित्रता से भी अधिक पवित्र, धैर्य एवं सहनशीलता का सर्वोच्च आदर्श - सीता, वे सदा इसी भाव से पूजी जाएँगी। जिन्होंने अविचलित भाव से ऐसे महादुःख का जीवन व्यतीत किया, वे ही नित्य साध्वी, सदा शुद्ध-स्वभाव सीता, आदर्श-पत्नी सीता, मनुष्य-लोक की आदर्श, देवलोक की भी आदर्श नारी, पुण्यचरित्र सीता सदा हमारी राष्ट्रीय देवी बनी रहेंगी। हम सभी उनके चरित्र को भलीभाँति जानते हैं, इसलिए उनका विशेष वर्णन करने की जरूरत नहीं। चाहे हमारे सारे पुराण नष्ट हो जाएँ, यहाँ तक की हमारे वेद भी लुप्त हो जाएँ, चाहे हमारी संस्कृत भाषा सदा के लिए कालस्रोत में विलुप्त हो जाए, किन्तु मेरी बात ध्यानपूर्वक सुनो, जब तक भारत में अति ग्राम्य भाषा बोलनेवाले पाँच भी हिन्दू रहेंगे, तब तक सीता की कथा विद्यमान रहेगी। सीता का प्रवेश हमारी जाति की अस्थि-मज्जा तक में हो चुका है; हर हिन्दू नरनारी के रक्त में सीता विराजमान हैं; हम सभी सीता की सन्तान हैं। हमारी नारियों को आधुनिक भावों में रँगने की जो चेष्टाएँ हो रही हैं, उन सब प्रयत्नों में यदि उनको सीताचरित्र के आदर्श से भ्रष्ट करने की चेष्टा होगी, तो वे सब असफल होंगी, जैसा कि हम प्रतिदिन देखते हैं। भारतीय नारियों से सीता के चरण-चिह्नों का अनुसरण कराकर हमें अपनी उन्नति की चेष्टा करनी होगी, यही एकमात्र पथ है।\endnote{ ५/१५०;} 

मैं जानता हूँ कि जिस जाति ने सीता को उत्पन्न किया है - चाहे उसने उसकी कल्पना ही की हो - नारी के प्रति उसका आदर पृथ्वी पर अद्वितीय है।\endnote{ ४/२६८;}


\section*{आधुनिक नारी का आदर्श}

\addsectiontoTOC{आधुनिक नारी का आदर्श}

हमारी माँ (सारदा देवी) यद्यपि बाहर से समुद्र के समान प्रशान्त हैं, पर वे आध्यात्मिक शक्ति की एक विशाल आधार हैं। उनके आविर्भाव से भारतीय इतिहास का एक नवयुग शुरू हुआ है। जिन आदर्शों को उन्होंने अपने जीवन में प्रकट किया है और दूसरों को अपनाने के लिए प्रेरित किया है, वे आदर्श भारतीय नारी की बन्धन-मुक्ति के हर प्रयास को संजीवित करने के साथ-ही सम्पूर्ण पृथ्वी की नारी-जाति को प्रभावित करते हुए उनके मन-प्राण में प्रविष्ट हो जाएँगे।\endnote{ झ. /. ५०६;} 

माताजी को केन्द्र बनाकर गंगा के पूर्व तट पर स्त्रियों के मठ की स्थापना करनी होगी। जैसे इस मठ में ब्रह्मचारी-साधु तैयार होंगे, वैसे ही उस पार के स्त्री-मठ में भी ब्रह्मचारिणी और साध्वी स्त्रियाँ तैयार होंगी।\endnote{ ६/१८१;} 

माँ का स्वरूप वस्तुतः क्या है, तुम लोग अभी नहीं समझ सके हो - तुममें से एक भी नहीं। धीरे-धीरे जानोगे। भाई, शक्ति के बिना जगत् का उद्धार नहीं हो सकता। क्या कारण है कि संसार के सब देशों में हमारा देश ही सबसे अधम, शक्तिहीन और पिछड़ा हुआ है? इसका कारण यही है कि यहाँ शक्ति का अनादर होता है। उस महाशक्ति को भारत में पुनः जगाने के लिए ही माँ का आविर्भाव हुआ है और उन्हें केन्द्र बनाकर जगत् में फिर से गार्गी और मैत्रेयी जैसी नारियों का जन्म होगा। भाई, अभी नहीं, धीरे-धीरे सब समझोगे। इसीलिए उनके मठ का होना पहले आवश्यक है।... शक्ति की कृपा के बिना कुछ भी नहीं हो सकता। अमेरिका और यूरोप में क्या देख रहा हूँ? - शक्ति की उपासना। परन्तु वे उसकी उपासना अज्ञानवश करते हैं, इन्द्रिय-भोग द्वारा करते हैं। तो फिर जो लोग पवित्रतापूर्वक सात्त्विक भाव द्वारा उसे पूजेंगे, उनका कितना कल्याण होगा! दिन-पर-दिन सब समझता जा रहा हूँ। मेरी आँखें खुलती जा रही हैं। अतः पहले माँ का मठ बनाना होगा। पहले माँ और उनकी पुत्रियाँ, फिर पिता और उनके पुत्र - समझे?... भाई, नाराज न होना, तुममें से कोई भी अब तक माँ को समझ नहीं सका है। मुझ पर माँ की कृपा, बाप की कृपा से लाख-गुनी है। माँ की कृपा, माँ का आशीष मेरे लिए सर्वोपरि है।... मुझे क्षमा करो, माँ के विषय में मैं थोड़ा कट्टर हूँ। माँ की आज्ञा होने पर उनके वीरभद्र भूत-प्रेत कुछ भी कर सकते हैं।... अमेरिका के लिए प्रस्थान करने से पहले मैंने माँ को लिखा था कि वे मुझे आशीर्वाद दें। उनका आशीर्वाद आया और मैं एक ही छलाँग में समुद्र पार हो गया। इसी से समझ लो! इस विकट जाड़े में मैं जगह-जगह भाषण दे रहा हूँ और विषम बाधाओं से लड़ रहा हूँ, ताकि माँ के मठ हेतु कुछ धन एकत्र हो सके। 

बाबूराम की माँ का बुढ़ापे में बुद्धिलोप हुआ है, जो जीती-जागती दुर्गा को छोड़कर मिट्टी की दुर्गा पूजने चली हैं। भाई, विश्वास अमोल धन है। जीती-जागती दुर्गा की पूजा न दिखाया, तो मेरा नाम नहीं! जब जमीन खरीद कर जीती-जागती दुर्गा - हमारी माँ को वहाँ ले जाकर बिठा दोगे, तभी मैं चैन की साँस लूँगा। उसके पहले मैं देश नहीं लौट रहा हूँ। जितनी जल्दी हो सके, यह करो। यदि रुपये भेज सकूँ, तो दम लेकर सुस्ताऊँ। तुम लोग साजो-सामान जुटाकर मेरा यह दुर्गोत्सव सम्पन्न कर दो, तो जानूँ। गिरीश घोष माँ की खूब पूजा कर रहा है, वह धन्य है, उसका कुल धन्य है! भाई, माँ का स्मरण होने पर बीच-बीच में कहता हूँ, ‘को रामः?’ भाई, यही जो मैं कहता हूँ, इसी में मेरी कट्टरता है।\endnote{ २/३६१-६२;}


\section*{नारी मठ विषयक स्वामीजी के विचार}

\addsectiontoTOC{नारी मठ विषयक स्वामीजी के विचार}

जहाँ पर स्त्रियों का सम्मान नहीं होता, वे दुःखी रहती हैं; उस परिवार की, उस देश की उन्नति की आशा नहीं की जा सकती। इसलिए पहले इन्हें ही उठाना होगा। इनके लिए आदर्श मठ की स्थापना करनी होगी।... जिस महामाया का रूप-रसात्मक बाह्य विकास मनुष्य को पागल बनाए रखता है, जिस माया का ज्ञान-भक्ति-विवेक-वैराग्यात्मक अन्तर्विकास मनुष्य को सर्वज्ञ, सिद्ध-संकल्प, ब्रह्मज्ञ बना देता है - उन प्रत्यक्ष मातृरूपा स्त्रियों की पूजा करने से मैंने कभी मना नहीं किया। \textbf{सैषा प्रसन्ना वरदा नृणां भवति मुक्तये } - प्रसन्न होने पर वह वरदायिनी तथा मनुष्यों की मुक्ति का कारण होती है। इस महामाया को पूजा-प्रणाम द्वारा प्रसन्न किए बिना - क्या मजाल कि ब्रह्मा, विष्णु तक उनके पंजे से छूटकर मुक्त हो जाएँ? गृह-लक्ष्मियों की पूजा के उद्देश्य से, उनमें ब्रह्मविद्या के विकास के निमित्त मैं उनके लिए मठ स्थापित करूँगा।... 

अब भी श्रीरामकृष्ण की अनेक भक्तिमती शिष्याएँ हैं। उन्हें लेकर स्त्री-मठ का प्रारम्भ करूँगा। माताजी उनका केन्द्र बनेंगी। पहले उसमें श्रीरामकृष्ण के भक्तों की स्त्री-कन्याएँ निवास करेंगी, क्योंकि वे उस स्त्री-मठ की उपयोगिता आसानी से समझ सकेंगी। उसके बाद उन्हें देखकर अन्य गृहस्थ लोग भी इस महान् कार्य में सहायक बनेंगे।... 

जगत् का कोई भी महान् कार्य त्याग के बिना नहीं हुआ है। वटवृक्ष का अंकुर देखकर कौन कह सकता है कि समय आने पर वह एक विराट् वृक्ष बनेगा? अभी तो इसी रूप में मठ की स्थापना करूँगा। फिर एकाध पीढ़ी के बाद देशवासी इस मठ की कद्र करने लगेंगे। जो विदेशी स्त्रियाँ मेरी शिष्याएँ बनी हैं, ये ही इस कार्य में जीवन उत्सर्ग करेंगी। तुम लोग भय तथा कापुरुषता छोड़कर इस कार्य में लग जाओ और इस उच्च आदर्श को सभी के सामने रख दो। देखना, समय आने पर इसकी प्रभा से देश आलोकित हो उठेगा।... 

गंगाजी के उस पार एक विशाल भूखण्ड लिया जाएगा। उसमें अविवाहित कुमारियाँ तथा विधवा ब्रह्मचारिणियाँ भी रहेंगी। बीच-बीच में गृहस्थ-घरों की भक्तिमती स्त्रियाँ भी वहाँ आकर ठहर सकेंगी। इस मठ से पुरुषों का कोई ताल्लुक नहीं रहेगा। पुरुष-मठ के वृद्ध संन्यासी दूर से स्त्री-मठ का काम चलाएँगे। स्त्री-मठ में बालिकाओं का एक स्कूल रहेगा। उसमें धर्मशास्त्र, साहित्य, संस्कृत व्याकरण और साथ-ही थोड़ी-बहुत अंग्रेजी भी सिखायी जाएगी। सिलाई का काम, रसोई बनाना, घर-गृहस्थी के सारे नियम तथा शिशुपालन आदि मोटे-मोटे विषयों की शिक्षा भी दी जाएगी। साथ ही जप, ध्यान, पूजा - ये सब तो शिक्षा के अंग रहेंगे ही। जो स्त्रियाँ घर छोड़कर हमेशा के लिए यहीं रह सकेंगी, उनके भोजन-वस्त्र का प्रबन्ध मठ की ओर से किया जाएगा। जो ऐसा नहीं कर सकेंगी, वे इस मठ में दैनिक छात्राओं के रूप में आकर अध्ययन कर सकेंगी। यदि सम्भव होगा तो मठ के अध्यक्ष की अनुमति से वे यहाँ पर रहेंगी और जितने दिन रहेंगी, भोजन भी पा सकेंगी। बालिकाओं से ब्रह्मचर्य-पालन कराने हेतु वृद्धा ब्रह्मचारिणियाँ छात्राओं की शिक्षा का भार लेंगी। ५-७ वर्ष तक इस मठ में शिक्षा पाने के बाद बालिकाओं के अभिभावक उनका विवाह कर सकेंगे। यदि कोई अधिकारिणी समझी जाएगी, तो अपने अभिभावकों की सम्मति लेकर चिर कौमार्य-व्रत का पालन करती हुई वहाँ ठहर सकेगी। जो स्त्रियाँ चिर कौमार्य व्रत का अवलम्बन करेंगी, वे ही समय पर मठ की शिक्षिकाएँ तथा प्रचारिकाएँ बन जाएँगी और गाँव-गाँव, नगर-नगर में शिक्षा-केन्द्र खोलकर स्त्रियों की शिक्षा के विस्तार की चेष्टा करेंगी। चरित्रवान तथा धर्म-भावापन्न प्रचारिकाओं द्वारा देश में यथार्थ स्त्री-शिक्षा का प्रसार होगा। वे स्त्री-मठ के सम्पर्क में जितने दिन रहेंगी, उतने दिन तक ब्रह्मचर्य की रक्षा करना इस मठ का अनिवार्य नियम होगा। धर्म-परायणता, त्याग और संयम यहाँ की छात्राओं के अलंकार होंगे और सेवा-धर्म उनके जीवन का व्रत होगा। इस प्रकार के आदर्श जीवन को देखकर कौन उनका सम्मान न करेगा? कौन उन पर अविश्वास करेगा? देश की स्त्रियों का जीवन इस प्रकार गठित हो जाने पर ही तो तुम्हारे देश में सीता, सावित्री, गार्गी का फिर से आविर्भाव हो सकेगा? देशाचार के घोर बन्धन से प्राणहीन, स्पन्दनहीन बनकर तुम्हारी बालिकाएँ कितनी दयनीय बन गयी हैं, यह तुम एक बार पाश्चात्य देशों की यात्रा करने पर ही समझ सकोगे। स्त्रियों की इस दुर्दशा के लिए तुम्हीं लोग जिम्मेदार हो। देश की स्त्रियों को पुनः जाग्रत करने का भार तुम्हीं पर है। इसीलिए तो मैं कह रहा हूँ कि काम में लग जाओ।... 

\newpage

उन्हें शिक्षा देकर छोड़ देना होगा। इसके बाद वे स्वयं ही सोच-समझकर, जो उचित होगा, करेंगी। विवाह करके गृहस्थी में लग जाने पर भी वैसी लड़कियाँ अपने पतियों को उच्च भाव की प्रेरणा देंगी और वीर पुत्रों की जननी बनेंगी। परन्तु यह नियम रखना होगा कि स्त्री-मठ की छात्राओं के अभिभावक १५ वर्ष की आयु के पूर्व उनके विवाह का नाम नहीं लेंगे।... इन विदुषी तथा कर्मठ बालिकाओं के लिए वरों की कमी न होगी। \textbf{दशमे कन्यका-प्राप्तिः } - इन बाल-विवाह मूलक वचनों पर आज समाज चल नहीं रहा है, चलेगा भी नहीं।\endnote{ ६/१८२-८३;}


\section*{पुरुष हस्तक्षेप न करें}

\addsectiontoTOC{पुरुष हस्तक्षेप न करें}

पहले अपनी स्त्रियों को शिक्षा दो और बाकी उन्हीं पर छोड़ दो, वे ही तुम्हें बताएँगी कि उनके लिए क्या सुधार जरूरी हैं।\endnote{ १/२९७;} हमारा अधिकार केवल शिक्षा के प्रचार तक ही सीमित होगा। उन्हें ऐसी स्थिति में पहुँचा देना होगा, जहाँ वे अपनी समस्या को स्वयं अपने ढंग से सुलझा सकें। उनके लिए यह काम न कोई कर सकता है और न किसी को करना ही चाहिए। हमारी भारतीय नारियाँ संसार की अन्य किन्हीं भी नारियों की भाँति इसे करने की क्षमता रखती हैं।\endnote{ ४/२६७;} 

उन्नति के लिए सबसे पहले स्वाधीनता की जरूरत है। यदि तुममें से कोई यह कहने का साहस करे कि मैं अमुक स्त्री या अमुक बालक की मुक्ति के लिए कार्य करूँगा, तो यह गलत होगा, हजार बार गलत होगा। मुझसे बार-बार पूछा जाता है कि विधवाओं की समस्या के बारे में और स्त्रियों के प्रश्न पर आप क्या सोचते हैं? मैं इस प्रश्न का अन्तिम उत्तर यही देता हूँ - क्या मैं विधवा हूँ, जो तुम मुझसे ऐसा निरर्थक प्रश्न पूछते हो? क्या मैं स्त्री हूँ, जो तुम बार-बार मुझसे यही प्रश्न करते हो? स्त्री-जाति के प्रश्न को हल करने के लिए अग्रसर होनेवाले तुम हो कौन? क्या तुम हर विधवा और हर स्त्री के भाग्य-विधाता - भगवान हो? दूर रहो! अपनी समस्याओं का समाधान वे स्वयं कर लेंगी।\endnote{ ५/१४१;}


\section*{भारतीय नारियों के प्रति}

\addsectiontoTOC{भारतीय नारियों के प्रति}

इस देश की नारियों से भी मैं वही बात कहूँगा, जो पुरुषों से कहता हूँ। भारत में विश्वास करो और अपने भारतीय धर्म में विश्वास करो। शक्तिमान बनो, आशावान बनो, संकोच छोड़ो और याद रखो कि यदि हम विदेश से कोई वस्तु लेते हैं; तो उसके बदले में देने को संसार के किसी भी अन्य देश... की तुलना में हिन्दू के पास उससे अनन्त गुना अधिक है।\endnote{ ४/२६९;} 

गौरी-माँ कहाँ हैं? हजारों गौरी-माताओं की आवश्यकता है, जिनमें उन्हीं के समान महान् एवं तेजोमय भाव हो।\endnote{ ३/३६१;} 

भारत के लिए, विशेषकर नारी-समाज के लिए पुरुषों की अपेक्षा नारियों में एक सच्ची सिंहिनी की आवश्यकता है।\endnote{ ६/३४९;} 

पुरुष तथा नारी, दोनों ही आवश्यक हैं।... हजारों पुरुष तथा नारी चाहिए, जो अग्नि की भाँति हिमालय से कन्याकुमारी तथा उत्तरी से दक्षिणी ध्रुव तक पूरी दुनिया में फैल जाएँ। वह बच्चों का खेल नहीं है और न उसके लिए समय ही है। जो बच्चों का खेल खेलना चाहते हैं, उन्हें अभी अलग हो जाना चाहिए, नहीं तो आगे उनके लिए बड़ी विपत्ति खड़ी हो जाएगी। हमें संगठन चाहिए, आलस्य को दूर कर दो; फैलो! फैलो! अग्नि की तरह चारों ओर फैल जाओ।\endnote{ ३/३०१-०२;} 

जगत् को प्रकाश कौन देगा? बलिदान ही भूतकाल से नियम रहा है और हाय! युगों तक इसी को रहना है। संसार के वीरों और सर्वश्रेष्ठों को ‘बहुजन हिताय, बहुजन सुखाय’ अपना बलिदान करना होगा। असीम दया और प्रेम से परिपूर्ण सैकड़ों बुद्धों की आवश्यकता है। 

विश्व के धर्म प्राणहीन हास्यास्पद वस्तु हो गए हैं। जगत् को जिस चीज की जरूरत है, वह है चरित्र। संसार को ऐसे लोग चाहिए, जिनका जीवन निःस्वार्थ ज्वलन्त प्रेम का उदाहरण हो। वह प्रेम एक-एक शब्द को वज्र के समान प्रभावी बना देगा।... हमें ‘साहसी’ शब्द और उससे भी अधिक ‘साहसी’ कर्मों की आवश्यकता है। उठो! उठो! संसार दुःख से जल रहा है। क्या तुम सो सकते हो?\endnote{ ४/४०७-०८;} 

पाँच सौ पुरुषों के द्वारा भारत को जय करने में पचास वर्ष लग सकते हैं, परन्तु पाँच सौ नारियों के द्वारा यह मात्र कुछ सप्ताहों में ही सम्पन्न हो सकता है।\endnote{ \engfoot{M. S.} २६०} 

\delimiter

\addtoendnotes{\protect\end{multicols}}

\addtocontents{toc}{\protect\par\egroup}

