
\chapter{परिशिष्ट - २ }

\section*{सन्दर्भ सूची}

इस संकलन में मुख्यतः अद्वैत आश्रम (मायावती) द्वारा प्रकाशित ‘विवेकानन्द-साहित्य’ (१० खण्ड) (प्रथम संस्करण) का उपयोग किया गया है। जहाँ अन्य भाषाओं के अन्य ग्रन्थों के उद्धरण आए हैं, उनका यथास्थान उल्लेख किया गया है। 

\tabcolsep=2pt

\begin{tabular}{@{}lc>{\raggedright}p{9cm}@{}}
\general{\enginline{\textbf{L. S. V.}}} & – & \general{\enginline{Letters of Swami Vivekananda}} \tabularnewline
\general{\enginline{\textbf{S. V. P.}}} & – & \general{\enginline{Swami Vivekananda : Patriot-Prophet \\\general{\bgroup\footnotesize} – Bhupendranath Datta, Nababharat Publishers,\\ Culcutta, 1954\general{\egroup}}} \tabularnewline
\general{\enginline{\textbf{C. W.}}} & – & \general{\enginline{The complete Works of Swami Vivekananda}} \tabularnewline
\general{\enginline{\textbf{M. S.}}} & – & \general{\enginline{The Master as I Saw Him \\\general{\bgroup\footnotesize} – Sister Nivedita, Udbodhan Office, Calcutta, 1977\general{\egroup}}} \tabularnewline
\general{\enginline{\textbf{P. B.}}} & – & \general{\enginline{Prabuddha Bharat}} \tabularnewline
\end{tabular}

\delimiter

