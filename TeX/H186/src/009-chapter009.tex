
\chapter{राष्ट्रीय एकता }

\indentsecionsintoc

\toendnotes{राष्ट्रीय एकता}

\addtoendnotes{\protect\begin{multicols}{3}}

\section*{एकता की आवश्यकता}

\addsectiontoTOC{एकता की आवश्यकता}

इच्छाशक्ति ही जगत् में अमोघ शक्ति है।... वह कौन-सी वस्तु है, जिसके द्वारा कुल चार करोड़ अंग्रेज पूरे तीस करोड़ भारतवासियों पर शासन करते हैं? इस प्रश्न का मनोवैज्ञानिक समाधान क्या है? यही, कि वे चार करोड़ लोग अपनी-अपनी इच्छाशक्ति को एकत्र कर देते हैं अर्थात् शक्ति का अनन्त भण्डार बना लेते हैं और तुम तीस करोड़ लोग अपनी-अपनी इच्छाओं को एक दूसरे से पृथक् किए रहते हो। बस, यही इसका रहस्य है कि वे संख्या में कम होकर भी तुम्हारे ऊपर शासन करते हैं। अतः संगठन ही शक्ति का मूल है। यदि भारत को महान् बनाना है, उसका भविष्य उज्ज्वल बनाना है, तो इसके लिए आवश्यकता है - संगठन की, शक्ति-संग्रह की और बिखरी हुई इच्छाशक्ति को एकत्र करके उसमें समन्वय लाने की। 

अथर्ववेद संहिता का एक विलक्षण मंत्र याद आ रहा है, जिसमें कहा गया है - तुम सब लोग एकमन हो जाओ, सब लोग एक ही विचार के बन जाओ, क्योंकि प्राचीन काल में एकमन होने के कारण ही देवताओं ने बलि पायी थी। देवता मनुष्य द्वारा इसीलिए पूजे गए कि वे एकचित्त थे, एकमन हो जाना ही समाज गठन का रहस्य है। और यदि तुम ‘आर्य’ और ‘द्रविड़’, ‘ब्राह्मण’ और ‘अब्राह्मण’ जैसे तुच्छ विषयों को लेकर ‘तूतू’ ‘मैं-मैं’ करोगे - झगड़े और पारस्परिक विरोध भाव को बढ़ाओगे - तो समझ लो कि तुम उस शक्ति-संग्रह से दूर हटते जाओगे, जिसके द्वारा भारत का भविष्य बनने जा रहा है। इस बात को याद रखो कि भारत का भविष्य पूर्णतः इसी पर निर्भर करता है। बस, इच्छाशक्ति का संचय और उनका समन्वय करके उन्हें एकमुखी करना ही सारा रहस्य है। प्रत्येक चीनी अपनी शक्तियों को भिन्न-भिन्न मार्गों से परिचालित करता है, परन्तु मुट्ठी भर जापानी अपनी इच्छाशक्ति एक ही मार्ग से परिचालित करते हैं; और इसका जो फल हुआ, वह तुम लोगों से छिपा नहीं है। इसी तरह की बात सारे संसार में देखने में आती है।... ये सब मतभेद और झगड़े एकदम बन्द हो जाने चाहिए।\endnote{ ५/१९३-९४;}


\section*{पंजाब के सिखों तथा हिन्दुओं को सलाह}

\addsectiontoTOC{पंजाब के सिखों तथा हिन्दुओं को सलाह}

यह वही वीरभूमि है, जिसे भारत पर चढ़ाई करनेवाले शत्रुओं के सभी आक्रमणों तथा अतिक्रमणों का आघात सबसे पहले सहना पड़ा था। इसी वीरभूमि को अपनी छाती खोलकर, आर्यावर्त में घुसनेवाली बाहरी बर्बर जातियों के प्रत्येक हमले का सामना करना पड़ा था। यह वही भूमि है, जिसने इतनी विपत्तियाँ झेलने के बाद भी, अब तक अपने गौरव तथा शक्ति को पूरी तौर से नहीं खोया है। यही वह भूमि है, जहाँ बाद में दयालु नानक ने अद्भुत विश्वप्रेम का उपदेश दिया, जहाँ उन्होंने अपना विशाल हृदय खोलकर सारे संसार को - केवल हिन्दुओं को नहीं, वरन् मुसलमानों को भी - गले लगाने के लिए अपने हाथ फैलाये। यहीं पर हमारी जाति के सबसे बाद के तथा महान तेजस्वी वीरों में से एक - गुरु गोविन्द सिंह ने धर्म की रक्षा के लिए अपना तथा अपने प्राणप्रिय कुटुम्बियों का रक्त बहा दिया। जिन लोगों के लिए यह खून की नदी बहायी गयी, उन्होंने भी जब उनका साथ छोड़ दिया, तब वे मर्माहत सिंह की भाँति चुपचाप दक्षिण भारत में निर्जन-वास के लिए चले गए और अपने देशभाइयों के प्रति अधरों पर एक भी कटु वचन लाए बिना, तनिक भी असन्तोष प्रकट किए बिना, शान्त भाव से इस लोक से प्रयाण करके चले गए। 

हे पंचनदवासी भाइयो! यहाँ अपनी इस प्राचीन पवित्र भूमि में, मैं तुम लोगों के सामने आचार्य के रूप में नहीं खड़ा हूँ, क्योंकि तुम्हें शिक्षा देने योग्य ज्ञान मेरे पास बहुत ही कम है। मैं तो पूर्वी प्रान्त से अपने पश्चिमी प्रान्त के भाइयों के पास इसीलिए आया हूँ कि उनके साथ हृदय खोलकर वार्तालाप करूँ, उन्हें अपने अनुभव बताऊँ और उनके अनुभव से स्वयं लाभ उठाऊँ। मैं यहाँ यह देखने नहीं आया कि हमारे बीच क्याक्या मतभेद है, वरन् मैं तो यह खोजने आया हूँ कि हम लोगों की मिलन-भूमि कौनसी है। यहाँ मैं यह जानने का प्रयत्न कर रहा हूँ कि वह कौन-सा आधार है, जिस पर हम लोग आपस में सदा भाई बने रह सकते है; किस नींव पर प्रतिष्ठित होने से, अनन्त काल से सुनायी देनेवाली वह वाणी, उत्तरोत्तर अधिक प्रबल होती रहेगी। मैं यहाँ तुम्हारे सामने कुछ रचनात्मक कार्यक्रम रखने आया हूँ, ध्वंसात्मक नहीं। क्योंकि आलोचना के दिन जा चुके हैं और अब हम रचनात्मक कार्य करने के लिए उत्सुक हैं।... 

सज्जनो! इसी उद्देश्य से प्रेरित होकर मैं आपके सामने आया हूँ और आरम्भ में ही यह बता देना चाहता हूँ कि मैं किसी दल या विशिष्ट सम्प्रदाय का नहीं हूँ। सभी दल और सभी सम्प्रदाय मेरे लिए महान् और गौरवशाली हैं। मैं उन सबसे प्रेम करता हूँ और अपने जीवन भर मैं यही ढूँढ़ने का प्रयत्न करता रहा हूँ कि उनमें कौन-कौनसी बातें अच्छी और सच्ची हैं।... मैं वे बातें प्रस्तुत करूँगा, जिनमें हम एकमत हैं और यदि ईश्वर के अनुग्रह से यह सम्भव हो, तो आओ, हम उसे ग्रहण करें और कार्यरूप में परिणत करें।\endnote{ ५/२५७, २६२-३, २७०;} 

इस देश में अनेक पन्थ या सम्प्रदाय हुए हैं। आज भी ये काफी संख्या में हैं और भविष्य में भी बड़ी संख्या में होंगे।... सम्प्रदाय अवश्य रहें, पर साम्प्रदायिकता दूर हो जाए। साम्प्रदायिकता से संसार की कोई उन्नति नहीं होगी, परन्तु सम्प्रदायों के बिना संसार का काम नहीं चल सकता। एक ही साम्प्रदायिक विचार के लोग सारे काम नहीं कर सकते। संसार की यह अनन्त शक्ति कुछ थोड़े-से लोगों के हाथों परिचालित नहीं हो सकती। यह बात समझ लेने पर यह भी हमारी समझ में आ जाएगा कि क्यों यह सम्प्रदायभेदरूपी श्रमविभाग अनिवार्य रूप से हमारे भीतर आ गया है। भिन्न-भिन्न आध्यात्मिक शक्ति-समूहों का परिचालन करने के लिए सम्प्रदाय कायम रहें। परन्तु जब हम देखते हैं कि हमारे प्राचीन शास्त्र इस बात की घोषणा कर रहे हैं कि यह सब भेद-भाव केवल ऊपर का है, सतही मात्र है और इन सारी विभिन्नताओं के बावजूद इनको एक साथ बाँधे रहनेवाला परम मनोहर स्वर्णसूत्र इनके भीतर पिरोया हुआ है, तब इसके लिए हमें एक-दूसरे के साथ लड़ने-झगड़ने की कोई आवश्यकता दिखाई नहीं देती।... 

यदि कोई तुमसे साम्प्रदायिक झगड़ा करने को तैयार हो, तो उससे पूछो, ‘क्या तुमने ईश्वर के दर्शन किए हैं? क्या तुम्हें कभी आत्म-दर्शन प्राप्त हुआ है? यदि नहीं, तो तुम्हें ईश्वर के नाम का प्रचार करने का क्या अधिकार है?... सबको अपनी-अपनी राह से चलने दो - ‘प्रत्यक्ष अनुभूति’ की ओर अग्रसर होने दो। सभी अपने-अपने हृदय में उस सत्यस्वरूप आत्मा का दर्शन पाने का प्रयत्न करें। और जब वे उस विराट् अनावृत सत्य के दर्शन कर लेंगे, तभी उससे प्राप्त होनेवाले अपूर्व आनन्द का अनुभव कर सकेंगे। आत्मोपलब्धि से प्रसूत होनेवाला यह अपूर्व आनन्द कपोल-कल्पित नहीं हैं, वरन् भारत के प्रत्येक ऋषि ने, प्रत्येक सत्यद्रष्टा व्यक्ति ने इसका प्रत्यक्ष अनुभव किया है। तब उस आत्मदर्शी हृदय से आप-ही-आप प्रेम की वाणी फूट निकलेगी, क्योंकि उसे ऐसे परम पुरुष का स्पर्श प्राप्त हुआ है, जो स्वयं प्रेमस्वरूप है। बस, तभी हमारे सारे साम्प्रदायिक लड़ाई-झगड़े दूर होंगे।\endnotemark[\theendnote]


\section*{हिन्दू-मुस्लिम-सम्बन्ध}

\addsectiontoTOC{हिन्दू-मुस्लिम-सम्बन्ध}

वेदान्त मत की आध्यात्मिक उदारता ने इस्लाम धर्म पर अपना विशेष प्रभाव डाला था। भारत का इस्लाम धर्म संसार के अन्यान्य देशों के इस्लाम धर्म की अपेक्षा पूर्ण रूप से भिन्न है। जब दूसरे देशों के मुसलमान यहाँ आकर भारतीय मुसलमानों को फुसलाते हैं कि तुम विधर्मियों के साथ मिल-जुलकर कैसे रहते हो, तभी अशिक्षित कट्टर मुसलमान उत्तेजित होकर दंगा-फंसाद मचाते हैं।\endnote{ १०/३७७;} 

चाहे हम उसे वेदान्त कहें या किसी अन्य नाम से पुकारें, परन्तु सत्य तो यह है कि धर्म तथा विचार में अद्वैत ही अन्तिम शब्द है और केवल उसी के दृष्टिकोण से सब धर्मों तथा सम्प्रदायों को प्रेम से देखा जा सकता है। हमें विश्वास है कि भविष्य के प्रबुद्ध मानवी समाज का यही धर्म है। अन्य जातियों की अपेक्षा हिन्दुओं को यह श्रेय प्राप्त होगा कि उन्होंने इसकी सर्वप्रथम खोज की। इसका कारण यह है कि वे अरबी और हिब्रू - दोनों जातियों से अधिक प्राचीन हैं। परन्तु साथ ही व्यावहारिक अद्वैतवाद का - जो समस्त मनुष्य-जाति को अपनी ही आत्मा का स्वरूप समझता है, तथा उसी के अनुकूल आचरण करता है - विकास हिन्दुओं में सार्वभौमिक भाव से होना अभी भी बाकी है। 

दूसरी ओर, हमारा अनुभव है कि यदि किसी धर्म के अनुयायी इस समानता को व्यावहारिक जगत् के दैनिक कार्यों के क्षेत्र में, उल्लेखनीय स्तर तक अपना सके हैं, तो वे इस्लाम और केवल इस्लाम के अनुयायी ही हैं।... 

इसलिए हमें दृढ़ विश्वास है कि वेदान्त के सिद्धान्त चाहे जितने भी उदार और विलक्षण क्यों न हो, परन्तु व्यावहारिक इस्लाम की सहायता के बिना, वे मनुष्य-जाति रूपी विशाल जन-समुदाय के लिए निरर्थक हैं। हम मनुष्य-जाति को उस स्थान पर पहुँचाना चाहते हैं; जहाँ न वेद है, न बाइबिल और न कुरान; परन्तु वेद, बाइबिल और कुरान के समन्वय से ही ऐसा हो सकता है। मनुष्य जाति को यह शिक्षा देनी होगी कि सभी धर्म एक उसी धर्म के - एकत्व धर्म के ही भिन्न-भिन्न रूप हैं, ताकि प्रत्येक व्यक्ति इनमें से अपना मनोनुकूल मार्ग चुन सके। 

हमारी मातृभूमि के लिए इन दोनों विशाल मतों का सामंजस्य - हिन्दुत्व और इस्लाम - वेदान्ती बुद्धि और इस्लामी शरीर - यही एक आशा है। 

इस विप्लव और संघर्ष के बीच मैं अपने मनश्चक्षुओं से भारत की उस पूर्णावस्था को देखता हूँ, जिसका भविष्य में तेजस्वी और अजेय रूप में वेदान्ती बुद्धि और इस्लामी शरीर के साथ उत्थान होगा।\endnote{ ६/४०५-६;}


\section*{बहुरंगी राष्ट्रीय पट: बहुत्व में एकत्व}

\addsectiontoTOC{बहुरंगी राष्ट्रीय पट : बहुत्व में एकत्व}

सचमुच एक विविध जातियों का अजायबघर! हाल ही में प्राप्त हुए सुमात्रा-शृंखला के अर्द्ध-वानर (\enginline{half-ape }) का कंकाल खोजने पर यहाँ भी कहीं मिल जाएगा। डोलमेनों की कमी नहीं है। पत्थर के औजार कहीं से भी खोदकर निकाले जा सकते हैं। किसी समय यहाँ झीलवासी - कम-से-कम सरितावासी लोगों की बहुतायत रही होगी। गुहावासी और पत्तियाँ पहनेवाले अब भी मिलते हैं। देश के विभिन्न भागों में आज भी जंगलों में रहनेवाले आदिम शिकारी दिखायी देते है। फिर और अधिक ऐतिहासिक विभेद है - निग्रीटो-कोलारी, द्रविड़ और आर्य। 

फिर बीच-बीच में इनमें समाते गए हैं लगभग सभी ज्ञात और अब तक अज्ञात अनेक जातियों के अंश - विविध मंगोल वर्ग, मंगोल, तातार और भाषाविज्ञानियों की तथाकथित आर्यजाति। फिर यहाँ हैं - फारसी, यूनानी, यूँची, हूण, चीन, सीथियन और अन्य अनेक; जो घुल-मिलकर एक हो गए हैं - यहूदी, पारसी, अरब, मंगोल आदि से लेकर समुद्री डाकुओं तथा जर्मन जंगली सरदारों के वंशज तक, जो अभी तक आत्मसात् नहीं किए जा सके है - इन जातीय-तरंगों से निर्मित मानवता का महासागर - जो (तरंगें) उद्वेलित, उत्तेजित, उबलती, संघर्षरत, निरन्तर परिवर्तनशील रूप धारण करती हुई; धरातल तक उठती, फैलती और छोटी लहरों को उदरस्थ करती हुई, फिर शान्त हो जाती हैं - यह है भारत का इतिहास!\endnote{ ९/२८१;} 

जाति धर्म, भाषा तथा शासन-प्रणाली - ये सब एक साथ मिलकर एक राष्ट्र की सृष्टि करते हैं। यदि एक-एक जाति को लेकर हमारे राष्ट्र से तुलना की जाए, तो हम देखेंगे कि जिन उपादानों से संसार के दूसरे राष्ट्र गठित हुए हैं, वे संख्या में यहाँ के उपादनों से कम हैं। यहाँ आर्य हैं, द्रविड़ हैं, तातार हैं, तुर्क हैं, मुगल हैं, यूरोपीय हैं, मानो - संसार की सभी जातियाँ इस भूमि में अपना-अपना खून मिला रही हैं। भाषा का यहाँ एक विचित्र-सा जमावड़ा है, आचार-व्यवहारों के सम्बन्ध में दो भारतीय जातियों में जितना अन्तर है, उतना प्राच्य और यूरोपीय जातियों में भी नहीं है। 

हमारे पास एकमात्र मिलन-भूमि है - हमारी पवित्र परम्परा, हमारा धर्म। एकमात्र सामान्य आधार वही हैं और उसी पर हमें संगठित होना पड़ेगा। यूरोप में राजनीतिक विचार ही राष्ट्रीय एकता का कारण है; परन्तु एशिया में राष्ट्रीय एकता का आधार धर्म है, अतः भारत के भावी संगठन की पहली शर्त के तौर पर इस धार्मिक एकता की ही आवश्यकता है।\endnote{ ५/१८०;} 

हम देखते हैं कि एशिया में, और विशेषतः भारत में जाति, भाषा, समाज सम्बन्धी सारी बाधाएँ धर्म की इस एकीकरण की शक्ति के सामने उड़ जाती हैं। हम जानते हैं कि भारतीय मन के लिए धार्मिक आदर्श से बड़ा और कुछ भी नहीं है। धर्म ही भारतीय जीवन का मूलमंत्र है और हम केवल सबसे कम बाधावाले मार्ग का अनुसरण करके ही कार्य में अग्रसर हो सकते हैं। धार्मिक आदर्श यहाँ सबसे बड़ा आदर्श है, यह बात न केवल सत्य है, अपितु यही भारत के लिए कार्य करवाने का एकमात्र सम्भाव्य उपाय है। पहले उस पथ को सुदृढ़ किए बिना, दूसरे मार्ग से कार्य करने पर उसका फल घातक होगा। इसीलिए भविष्य के भारत-निर्माण का पहला कार्य, वह पहला सोपान, जिसे युगों के उस चट्टान पर खोदकर बनाना होगा, भारत की यह धार्मिक एकता ही है।\endnote{ ५/१८१;} 

अपने राष्ट्रीय हित के लिए, जैसा कि अतीत काल में किया गया था और चिर काल तक किया जाएगा, हमें सबसे पहले अपनी राष्ट्र की समस्त आध्यात्मिक शक्तियों को ढूँढ़ निकालना होगा। अपनी बिखरी हुई आध्यात्मिक शक्तियों को एकत्र करना ही भारत में राष्ट्रीय एकता स्थापित करने का एकमात्र उपाय है। जिनकी हृत्तंत्री एक ही आध्यात्मिक स्वर में बँधी है, उन सबके सम्मिलन से ही भारत में राष्ट्र का गठन होगा।\endnote{ ५/२६२;} 

ईसाई को हिन्दू या बौद्ध नहीं हो जाना चाहिए; और न हिन्दू या बौद्ध को ईसाई बनना होगा। पर हाँ, प्रत्येक को चाहिए कि वह दूसरों के सार-भाग को आत्मसात् करके पुष्टिलाभ करे और अपने वैशिष्ट्य की रक्षा करते हुए अपने स्वयं के विकास के नियम के अनुसार विकास करे।\endnote{ १/२६-२७;} 

सम्प्रदाय अवश्य रहें, पर साम्प्रदायिकता दूर हो जाए।\endnote{ ५/२६३;}


\section*{एकता की कुंजी}

\addsectiontoTOC{एकता की कुंजी}

हम जानते हैं कि हमारे विभिन्न सम्प्रदायों के सिद्धान्त तथा दावे चाहे कितने ही भिन्न क्यों न हों, हमारे धर्म में कुछ सिद्धान्त ऐसे हैं, जो सभी सम्प्रदायों में मान्य हैं। अतः हमारे सम्प्रदायों के ऐसे कुछ सामान्य आधार अवश्य हैं, उनको स्वीकार करने पर हमारे धर्म में अद्भुत विविधता के लिए गुंजाइश हो जाती है और साथ ही विचार तथा अपनी रुचि के अनुसार जीवन-निर्वाह के लिए हमें पूर्ण स्वाधीनता भी प्राप्त हो जाती है। कम-से-कम हम लोग, जिन्होंने इस पर विचार किया है, यह बात जानते हैं। हमारे लिए आवश्यक है कि अपने धर्म के ये जीवनप्रद सामान्य तत्त्व हम सबके सामने लाएँ और देश के सभी स्त्री-पुरुष, बाल-वृद्ध, उन्हें जानें-समझें तथा जीवन में उतारें। सर्वप्रथम यही हमारा कार्य है।\endnote{ ५/१८०-८१;} 

हमें उन बातों का प्रचार करना होगा, जिन पर हम सभी सहमत हैं और तब आपसी मतभेद अपने आप ही दूर हो जाएँगे। मैं भारतवासियों से बारम्बार कहता रहा हूँ कि कमरे में यदि सैकड़ों वर्षों से अँधेरा फैला हुआ है, तो क्या - “घोर अन्धकार! भयंकर अन्धकार!!” - कहकर चिल्लाने मात्र से अँधेरा दूर हो जाएगा? नहीं। रोशनी जला दो - फिर देखोगे कि अँधेरा तत्काल दूर हो जाएगा।\endnote{ ५/२७५} 

\delimiter

\addtoendnotes{\protect\end{multicols}}

\addtocontents{toc}{\protect\par\egroup}

