
\chapter{परिशिष्ट – १}

\section*{लियो टाल्स्टाय (१८२८-१९१०) }

रूस के विख्यात उपन्यासकार, जिन्होंने १८५२ से १८५४ तक रूसी सेना को सेवा दी। ‘युद्ध और शान्ति’ (१८६५-६९) तथा ‘अन्ना केरेनिना’ (१८७५-७७) का लेखन। १८७६ के बाद उन्होंने ईसाई अनार्किज्म का एक सिद्धान्त विकसित किया और अपना जीवन समाज-सुधार के कार्य में लगा दिया। 

\section*{रवीन्द्रनाथ ठाकुर (१८६१-१९४३) }

सुप्रसिद्ध साहित्यकार, दार्शनिक तथा शिक्षाशास्त्री। बंगला काव्य-संग्रह ‘गीतांजलि’ के अंग्रेजी अनुवाद पर १९१३ ई. में साहित्य का नोबल पुरस्कार पानेवाले प्रथम एशियाई। १९१९ ई. में\break जलियाँवाला बाग में हुए अंग्रेजी सेना के अत्याचार के विरोध में उन्होंने नाइट (सर) की उपाधि को लौटा दिया। 

\section*{श्री अरविन्द (१८७२-१९५०) }

अरविन्द घोष ने आइ. सी. एस (इंडियन सिविल सर्विस) की प्रतियोगिता में भाग लेकर सफलता प्राप्त की, परन्तु स्वाधीनता आन्दोलन (१९०२-१९१०) में भाग लेने के उद्देश्य से विदेशी सरकार की प्रशासनिक सेवा में योगदान नहीं किया। वे राजनीति से संन्यास लेकर पाण्डीचेरी चले गए और अपना बाकी जीवन वहीं बिताया। उनकी रचनाओं में ‘दिव्य-जीवन’ तथा ‘योग-समन्वय’ विशेष उल्लेखनीय हैं। 

\section*{ब्राह्मबान्धव उपाध्याय (१८६१-१९०७) }

पूर्वनाम भवानी चरण बन्द्योपाध्याय। ये जनरल असेम्बलीज इंस्टीट्यूशन में नरेन्द्रनाथ दत्त (स्वामी विवेकानन्द) के सहपाठी थे। परवर्ती काल में वे श्रीरामकृष्ण तथा केशवचन्द्र सेन के सम्पर्क में आए। हिन्दू धर्म छोड़कर पहले वे ईसाई धर्म के प्रोटेस्टेंट सम्प्रदाय में सम्मिलित हुए और फिर १८९४ ई. में कैथॅलिक होकर ‘सोफिया’ नामक पत्रिका आरम्भ की। बाद में उन्होंने ‘स्वराज’ तथा ‘संध्या’ पत्रिकाएँ भी शुरू कीं और ‘मेरा भारत-उद्धार’, ‘समाज-तत्त्व’, ‘विलायत-यात्री संन्यासी के पत्र’ आदि ग्रन्थ लिखे। 

\section*{बाल गंगाधर तिलक (१८५६-१९२०) }

महाराष्ट्र के रत्नागिरि में जन्मे लोकमान्य तिलक, भारतीय राष्ट्रीय आन्दोलन के एक प्रमुख नेता थे। सर्वमान्य रूप से उन्हें ‘फादर ऑफ इंडियन अनरेस्ट’ (भारतीय असन्तोष का जनक) माना जाता था। स्वामी विवेकानन्द तथा स्वामी दयानन्द से प्रभावित होकर उन्होंने वैदिक दर्शन का गहन अध्ययन किया। वे संस्कृत तथा गणित के महान् विद्वान् थे। उनके ग्रन्थों में प्रमुख हैं - ‘गीता-रहस्य’, ‘द आर्कटिक होम इन द वेदाज’ (वेदों में आर्यों के ध्रुवीय मूल के संकेत)। 

\section*{बिपिन चन्द्र पाल (१८५८-१९३२) }

भारतीय स्वाधीनता आन्दोलन के एक सुप्रसिद्ध नेता। राजनीति में उन्होंने तिलक, लाजपत राय तथा अरविन्द के साथ मिलकर कार्य किया। १९०६ ई. में उन्होंने ‘वन्देमातरम्’ नामक एक दैनिक तथा १९१३ में ‘हिन्दू रिविउ’ नाम से एक मासिक पत्रिका निकाला। अपनी युवावस्था में वे ब्रह्मसमाजी हो गए थे, परन्तु परवर्ती जीवन में वे शांकर तथा वैष्णव दर्शन से काफी प्रभावित रहे। वे एक महान् प्रचारक, एक कुशल लेखक तथा विचारक भी थे। अरविन्द के अनुसार वे राष्ट्रीयता के सबलतम मार्ग-दर्शकों में से एक थे। 

\section*{मोहनदास करमचन्द गाँधी (१८६९-१९४८) }

‘बापूजी’ तथा ‘राष्ट्रपिता’ के रूप में विख्यात भारतीय राष्ट्रवादी। कानून का अध्ययन करने लन्दन गए (१८८८-९१), भारत में वकालत (१८९३), दक्षिणी अफ्रिका में निग्रो लोगों के पक्ष में आन्दोलन (१८९३)। भारतीय राष्ट्रीय कांग्रेस के अध्यक्ष (१९२५३४)। ‘हिन्द स्वराज’ (१९०९), ‘सत्य के मेरे प्रयोग’ आदि ग्रन्थों के लेखक। 

\section*{जवाहर लाल नेहरू (१८८९-१९६४) }

सुप्रसिद्ध भारतीय राजनीतिज्ञ तथा स्वाधीन भारत के प्रथम प्रधान-मंत्री। वे गांधीजी के एक निष्ठावान अनुयायी तथा भारत की विदेश नीति के स्रष्टा थे। वे एक कुशल लेखक भी थे। उनके प्रमुख ग्रन्थ हैं - ‘मेरी आत्मकथा’, ‘भारत की खोज’, ‘विश्व-इतिहास की झलकियाँ’, ‘पिता के पत्र - पुत्री के नाम’, आदि। १९५५ ई. में उन्हें ‘भारतरत्न’ की उपाधि दी गयी। 

\section*{सुभाष चन्द्र बोस (१८९७-१९४५) }

भारतीय राजनीतिज्ञ। गाँधी का समर्थन किया और स्वराज पार्टी में सम्मिलित हुए (१९२३);\break कोलकाता के मुख्य प्रशासनिक अधिकारी (१९२४); बंगाल प्रदेश कांग्रेस के अध्यक्ष (१९२७); राष्ट्रीय कांग्रेस के बंगाल डेलीगेशन का नेतृत्व किया (१९२८); भारत के लिए पूर्ण स्वाधीनता का समर्थन किया; अनेकों बार कारावास; ‘भारतीय संग्राम’; ‘तरुणाई के सपने’ आदि ग्रन्थों का लेखन; भारतीय राष्ट्रीय कांग्रेस के अध्यक्ष (१९३८)। 

\section*{विनोबा भावे (१८९५-१९८२) }

एक राष्ट्रीय नेता तथा समाज सुधारक। राष्ट्रीय आन्दोलन के समय महात्मा गाँधी के प्रमुख सहायक। उन्होंने एक तपस्वी का जीवन बिताया। भारत की स्वाधीनता के बाद (१९५१ में) विनोबा ने भूदान-आन्दोलन आरम्भ किया। इसके लिए सम्पूर्ण भारत में हजारों मील की पदयात्रा की और जमींदारों तथा धनिक किसानों से आग्रह किया कि वे स्वेच्छापूर्वक निर्धन भूमिहीन किसानों को भूमि का दान करें। इसके अतिरिक्त उन्होंने गोवध के विरुद्ध अखिल भारतीय आन्दोलन का भी नेतृत्व किया। 

\section*{रोमाँ रोलाँ (१८६६-१९४४) }

फ्रांसीसी साहित्यकार। १९१५ ई. में उनके ‘जीन क्रिस्टोफे’ (१९०४-१२) नामक उपन्यास-माला तथा एंडेसुस लेमली में संग्रहित पेसिफिस्ट मेनीफेस्टो (१९१५) पर नोबल पुरस्कार मिला। उनकी अन्य रचनाओं में हैं - ‘लक्मे एन्चैन्टे’ उपन्यास-माला (१९२२३३); ‘ले थ्येटर डे ला रिवाल्यूशन’ तथा ‘लेस ट्रेजेडीज डे ला फोइ’ (१९१३) नामक ग्रन्थों में संग्रहित ऐतिहासिक तथा दार्शनिक नाटक; बिथोवेन (१९०३), माइकेल-एंजेलो (१९०५), टाल्सटाय (१९११), महात्मा गाँधी (१९२८), श्रीरामकृष्ण और स्वामी विवेकानन्द की जीवनियाँ; उनका आइंस्टीन, स्वाइटजर, बर्ट्रेंड रसेल तथा रवीन्द्रनाथ ठाकुर के साथ पत्र-व्यवहार होता था; वे वीरोचित आदर्शवाद के अग्रदूत माने जाते थे; ‘अन्तर्राष्ट्रीय रिविउ यूरोप’ की स्थापना में सहायता की (१९२३)। 

\section*{विल डुरांट (१८८५-१९८१) }

‘विल’ के नाम से सुपरिचित विलियम जेम्स डुरांट एक अमेरिकी इतिहासकार थे, न्यूयार्क हैबर टेम्पल स्कूल में प्राध्यापक थे (१९१४-२७)। ‘स्टोरी ऑफ फिलॉसाफी’ (दर्शन की कहानी) (१९२६) ग्रन्थ की सफलता के बाद, उन्होंने अपनी पत्नी एरियल डुरांट के साथ मिलकर ११ खण्डों में ‘स्टोरी ऑफ सिविलाइजेशन’ (सभ्यता की कहानी) नामक ग्रन्थमाला लिखी। इसमें निम्नलिखित खण्ड थे - ‘\enginline{Our Oriental Heritage }’ (हमारी प्राच्य विरासत) (१९३५), ‘\enginline{The Life of Greece }’ (यूनान का जीवन) (१९३९), ‘\enginline{Caesar and Christ }’ (सीजर और ईसा) (१९४४), \enginline{‘Age of Faith }’ (विश्वास का काल) (१९५०), ‘\enginline{Renaissance }’ (पुनर्जागरण का काल) (१९५३), ‘\enginline{Reformation }’ (सुधार का काल) (१९५७), ‘\enginline{Age of Reason Begins }’ (बुद्धियुग का प्रारम्भ) (१९६१), \enginline{‘Age of Louis XIV }’ (चौदहवें लूई का काल) (१९६३), ‘\enginline{Age of Voltaire }’ (वोल्टायर का काल) (१९६५), ‘\enginline{Rousseau and Revolution }’ (रूसो और क्रान्ति) (१९६७, \enginline{Pulitzer prize }), \enginline{‘Age of Napoleon }’ (नेपोलियन का काल) (१९७५); \enginline{also wrote ‘Dual Biography }’ (दो जीवनियाँ) (१९७७)।

\section*{चक्रवर्ती राजगोपालाचारी (१८७९-१९७२) }

भारतीय राष्ट्रीय नेता। १९१८ ई. से गाँधीजी के साथ जुड़े; भारतीय राष्ट्रीय कांग्रेस की कार्यकारिणी समिति के सदस्य (१९२२-४२); मद्रास के मुख्य-मंत्री (१९३७-३९, १९५२-५४); भारत के गवर्नर-जनरल (१९४८-५०); मध्यमार्गी ‘स्वतंत्र पार्टी’ के संस्थापक (१९५९)। 

\section*{सर्वपल्ली राधाकृष्णन (१८८८-१९७५) }

वे एक दार्शनिक, मानवतावादी, शिक्षाशास्त्री तथा प्राच्यविद् थे। १९५२ में भारत के उप-राष्ट्रपति चुने गए; बाद में निर्विरोध भारतीय संघ के राष्ट्रपति चुने गए (१९६२६७)। उनके द्वारा लिखित अनेक ग्रन्थों में प्रमुख हैं - ‘भारतीय दर्शन’ (१९२३-२७), ‘उपनिषदों का दर्शन’ (१९२४), ‘प्राच्य धर्म तथा पाश्चात्य विचार’ (१९३९) तथा ‘प्राच्य और पाश्चात्य’ (१९५५)। 

\section*{रमेश चन्द्र मजुमदार (१८८८-१९८०) }

एक प्रमुख इतिहासकार तथा प्राध्यापक। ये ‘मानव-जाति का इतिहास: सांस्कृतिक तथा\break वैज्ञानिक विकास’ ग्रन्थ के प्रकाशन हेतु अन्तर्राष्ट्रीय समिति के उपाध्यक्ष थे। उन्हें कोलकाता विश्व\-विद्यालय, रवीन्द्र भारती विश्वविद्यालय तथा जादवपुर विश्वविद्यालयों से मानद डी. लिट् की उपाधियाँ मिली। उनके ग्रन्थों में प्रमुख हैं - ‘भारतवासियों का इतिहास और संस्कृति’ (११ खण्ड), ‘भारत में स्वाधीनता संग्राम का इतिहास’ (३ खण्ड), ‘भारत का इतिहास’ (४ खण्ड), ‘सुदूर पूर्व में प्राचीन भारतीय उपनिवेश’, ‘स्वामी विवेकानन्द’ आदि। 

\section*{सुनीति कुमार चैटर्जी (१८९०-१९७७) }

अन्तर्राष्ट्रीय ख्यातिप्राप्त शिक्षाशास्त्री, भाषाविद् तथा शिक्षा-जगत् के एक असाधारण विभूति। उनके अंग्रेजी ग्रन्थों में प्रमुख हैं - ‘\enginline{Bengali Self-taught }’ (बंगला स्वशिक्षा), ‘\enginline{A Bengali Phonetic Reader }’ (बंगला उच्चारण पाठिका), ‘\enginline{Indo-Aryan and Hindi }’ (इंडो-आर्यन तथा हिन्दी), ‘\enginline{Languages and Literatures of Modern India }’ (आधुनिक भारत की भाषाएँ तथा उनका साहित्य), \enginline{‘Africanism }’ (अफ्रीकनवाद), ‘\enginline{Balts and Aryans in their Indo-European Background }’ (अपनी इंडो-यूरोपियन पृष्ठभूमि में बाल्टिक तथा आर्य), और ‘\enginline{India and Ethiopia from the Seventh Century B.C. }’ (ईसापूर्व ७वीं शताब्दी से भारत और इथोपिया)। इसके अतिरिक्त उन्होंने अनेक लेख तथा प्रबन्ध भी लिखे। 

\section*{ए. एल. बासम (१९१४-८६) }

एक प्रसिद्ध भारतविद्। ब्रिटेन, अमेरिका, भारत, पाकिस्तान तथा श्रीलंका के विश्वविद्यालयों द्वारा अतिथि-प्राध्यापक के रूप में आमंत्रित किए गए। १९८५ में विश्वभारती विश्वविद्यालय द्वारा ‘देशि\-कोत्तम’ की उपाधि दी गयी। कोलकाता एशियाटिक सोसायटी के विवेकानन्द-प्राध्यापक और रामकृष्ण-भावधारा के अध्यक्ष रहे। ‘\enginline{The Wonder that was India }’ (अद्भुत भारत) उनका सर्वाधिक प्रसिद्ध ग्रन्थ है। 

\section*{ई. पी. चेलीसेव (१९२१) }

प्रोफेसर चेलीसेव सोवियत रूस के एक प्रमुख भारतविद् और आधुनिक भारतीय साहित्य -\break विशेषकर हिन्दी साहित्य के एक सुप्रसिद्ध विद्वान् हैं। उन्हें ‘जवाहरलाल नेहरू शान्ति पुरस्कार’ प्रदान किया गया। पिछले तीस वर्षों से विवेकानन्द की संस्कृति तथा शोध के कार्य से जुड़े रहे। रामकृष्ण-विवेकानन्द भावधारा के सर्वांगीण अध्ययन के लिए गठित कमेटी के उपाध्यक्षों में एक~थे। 

\section*{हुआंग जिन चुआन }

चीन के बेजिंग विश्वविद्यालय में इतिहास के प्राध्यापक और वहीं के सामाजिक विज्ञान अकादमी में एशियाई अध्ययन संस्थान के उप-निदेशक। उन्होंने चीनी भाषा में स्वामी विवेकानन्द पर एक ग्रन्थ लिखा है, जिसका नाम है - ‘आधुनिक भारतीय दार्शनिक विवेकानन्द: एक अध्ययन’। ये भी रामकृष्ण-विवेकानन्द भावधारा के सर्वांगीण अध्ययन के लिए गठित कमेटी के उपाध्यक्षों में एक~थे। 

\delimiter

