
\chapter{भारत के पुनरुत्थान के उपाय }

\indentsecionsintoc

\toendnotes{भारत के पुनरुत्थान के उपाय}

\addtoendnotes{\protect\begin{multicols}{3}}

\section*{कुछ भी नष्ट मत करो}

\addsectiontoTOC{कुछ भी नष्ट मत करो}

‘कुछ भी नष्ट मत करो’ - सर्वप्रथम मैं मनुष्य जाति से यह उक्ति स्वीकार कर लेने का अनुरोध करता हूँ। मूर्तिभंजक सुधारक लोग संसार का कोई उपकार नहीं कर सकते। किसी वस्तु को तोड़कर धूल में मत मिलाओ, वरन् उसका गठन करो। यदि हो सके तो सहायता करो, नहीं तो चुपचाप हाथ उठाकर खड़े हो जाओ और देखो कि मामला कहाँ तक जाता है। यदि सहायता नहीं कर सकते, तो हानि मत पहुँचाओ।... जो जहाँ पर है, उसे वहीं से उठाने की चेष्टा करो।... क्या तुम सोचते हो कि तुम एक शिशु को भी कुछ सिखा सकते हो? नहीं, तुम नहीं सिखा सकते। शिशु स्वयं ही शिक्षा प्राप्त करता है - बाधाएँ हटा देना मात्र ही तुम्हारा कर्तव्य है।\endnote{ ३/१४८;}


\section*{आम जनता - शक्ति की स्रोत}

\addsectiontoTOC{आम जनता - शक्ति की स्रोत}

समाज का नेतृत्व चाहे विद्या-बल से प्राप्त हुआ हो, या बाहु-बल से अथवा धनबल से; परन्तु उस शक्ति का आधार प्रजा ही है। इस शक्ति के आधार - प्रजा से शासक-वर्ग जितना ही अलग रहेगा, वह उतना ही दुर्बल होगा।\endnote{ ९/२२१;} 

स्वार्थपरता ही निःस्वार्थता का पहला शिक्षक है। व्यष्टि के स्वार्थों की रक्षा हेतु लोगों का ध्यान समष्टि के हित की ओर जाता है। स्वदेश के स्वार्थ में अपना स्वार्थ और स्वदेश के हित में अपना हित है। अधिकांश कार्य दूसरों के सहयोग के बिना नहीं चल सकते, आत्मरक्षा तक नहीं हो सकती।\endnote{ ९/२२२;} 

समष्टि (समाज) के जीवन में व्यष्टि (व्यक्ति) का जीवन है; समष्टि के सुख में व्यष्टि का सुख है; समष्टि के बिना व्यष्टि का अस्तित्व ही असम्भव है - यही अनन्त सत्य जगत् का मूल आधार है। अनन्त समष्टि के साथ सहानुभूति रखते हुए उसके सुख में सुख और उसके दुःख में दुःख मानकर धीरे-धीरे आगे बढ़ना ही व्यष्टि का एकमात्र कर्तव्य है।\endnote{ ४/४६३;}


\section*{हे भारत के श्रमजीवियो}

\addsectiontoTOC{हे भारत के श्रमजीवियो}

हे भारत के श्रमजीवियो, तुम्हारे नीरव, सदा ही निन्दित हुए परिश्रम के फलस्वरूप बाबिल, ईरान, सिकन्दरिया, यूनान, रोम, वेनिस, जिनेवा, बगदाद, समरकन्द, पुर्तगाल, स्पेन, फ्रांसीसी, दिनेमार, डच, और अंग्रेजों का क्रमशः आधिपत्य हुआ और उनको ऐश्वर्य मिला। और तुम? कौन सोचता है इस बात को! तुम्हारे पुरखे दो दर्शन लिख गए हैं, दस काव्य तैयार कर गए हैं, दस मन्दिर खड़े कर गए हैं और तुम्हारी बुलन्द आवाज से आकाश फट रहा है; और जिनके रुधिर-स्राव से मनुष्य जाति की यह सारी उन्नति हुई है, उनके गुणों का बखान भला कौन करता है? लोकजयी, धर्मवीर, रणवीर, काव्यवीर सबकी आँखों पर चढ़ते हैं, सबके पूज्य हैं, परन्तु जहाँ कोई नहीं देखता, जहाँ कोई एक वाहवाही भी नहीं करता, जहाँ सभी लोग घृणा करते हैं; वहाँ निवास करती है अपार सहनशीलता, अनन्य प्रीति और निर्भीक कार्यक्षमता; हमारे गरीब, घर में तथा बाहर दिन-रात मुँह बन्द किए कर्म किए जा रहे हैं, उसमें क्या वीरता नहीं है? बड़ा काम आने पर अधिकांश लोग वीर हो जाते हैं; दस हजार लोगों की वाहवाही के सामने कापुरुष भी सहज ही प्राण दे देता है; घोर स्वार्थी भी निष्काम हो जाता है; परन्तु अत्यन्त छोटे से कार्य में भी, सबके अज्ञात भाव से जो वैसी ही निःस्वार्थता एवं कर्तव्यपराणता दिखाते हैं, वे ही धन्य हैं - वे तुम लोग हो - भारत के हमेशा के पैरों तले कुचले श्रमजीवियो! मैं तुम्हें प्रणाम करता हूँ।\endnote{ ८/१९०;}


\section*{आम जनता का उत्थान}

\addsectiontoTOC{आम जनता का उत्थान}

‘धर्म को हानि पहुँचाए बिना जनता की उन्नति’ - इसी को अपना आदर्श-वाक्य बना लो। 

याद रखो कि राष्ट्र झोपड़ी में बसा हुआ है;... राष्ट्र की भावी उन्नति उसकी विधवाओं को मिले पति की संख्या पर नहीं, अपितु ‘आम जनता’ की अवस्था पर निर्भर है। क्या तुम जनता की उन्नति कर सकते हो? क्या उनकी स्वाभाविक आध्यात्मिक वृत्ति को नष्ट किए बिना, उन्हें उनका खोया हुआ व्यक्तित्व वापस दिला सकते हो?\endnote{ २/३२१;} 

मेरा मानना है कि हमारा सबसे बड़ा राष्ट्रीय पाप आम जनता की उपेक्षा है और वह भी हमारे पतन का एक कारण है। हम चाहे जितनी राजनीति करें, उससे तब तक कोई लाभ नहीं होगा, जब तक कि भारत की जनता एक बार फिर सुशिक्षित, सुपोषित तथा सुपालित नहीं होती। वे हमारी शिक्षा के लिए (राजकर के द्वारा) धन देते हैं, (शारीरिक श्रम के द्वारा) हमारे मन्दिर बनाते हैं और बदले में ठोकर पाते हैं। वे व्यवहारतः हमारे दास हैं। यदि हम भारत को पुनर्जीवित करना चाहते हैं, तो हमें उनके लिए काम करना होगा।\endnote{ ४/२६०-६१;} 

हमारा संघ दीन-हीन निर्धन, निरक्षर किसानों तथा श्रमिकों के लिए है और उनके लिए सब कुछ करने के बाद यदि समय बचेगा, तभी कुलीनों की बारी आएगी। प्रेम द्वारा तुम उन किसानों तथा श्रमिकों को जीत सकोगे।... \textbf{उद्धरेत् आत्मना आत्मानम् - } अपने प्रयास से अपना उद्धार करो। यह हर परिस्थिति पर लागू होता है।... जिस क्षण उन्हें अपनी दशा का बोध हो जाएगा और वे सहायता तथा उन्नति की जरूरत समझेंगे, तब जानना कि तुम्हारा प्रभाव पड़ रहा है और तुम ठीक रास्ते पर हो। धनवान लोग दयापूर्वक गरीबों के लिए जो थोड़ी-सी भलाई करते हैं, वह स्थायी नहीं होती और अन्त में दोनों पक्षों को हानि पहुँचती है। किसान और श्रमिक समाज मरणासन्न अवस्था में हैं; और उन्हें जिस चीज की जरूरत है, वह यह है कि धनवान उन्हें अपनी शक्ति को पुनः प्राप्त करने में सहायता दें और कुछ नहीं। उसके बाद किसानों और मजदूरों को स्वयं ही अपनी समस्याओं का सामना और समाधान करने दो।\endnote{ ७/४१२-१३;}


\section*{दोष धर्म का नहीं}

\addsectiontoTOC{दोष धर्म का नहीं}

भारतवर्ष में हम लोग गरीबों और पतितों के बारे में जो धारणा रखते हैं, उसे सोचकर मेरे प्राण बेचैन हो गए। उनके लिए न कोई अवसर है, न बचने की कोई राह और न उन्नति के लिए कोई मार्ग ही है। भारत के निर्धनों, पतितों तथा पापियों का कोई साथी नहीं, कोई सहायक नहीं।... वे दिन-पर-दिन डूबते जा रहे हैं।... पिछले कुछ वर्षों से विचारशील लोग समाज की यह दुर्दशा समझ रहे हैं, परन्तु दुर्भाग्यवश इसका दोष वे हिन्दू धर्म के मत्थे मढ़ रहे हैं। वे सोचते हैं कि जगत् के इस सर्वश्रेष्ठ धर्म का नाश ही समाज की उन्नति का एकमात्र उपाय है। सुनो मित्र, प्रभु की कृपा से मुझे इसका रहस्य ज्ञात हो गया है। इसमें दोष धर्म का नहीं, बल्कि उल्टे - तुम्हारा धर्म ही तो तुम्हें सिखाता है कि संसार के सारे प्राणी तुम्हारी आत्मा के विविध रूप हैं। इस तत्त्व को व्यावहारिक आचरण में न लाना - सहानभूति का अभाव - हृदय का अभाव - यही समाज की वर्तमान दुरवस्था का कारण है।... समाज की यह दशा दूर करनी होगी - परन्तु धर्म का नाश करके नहीं वरन् हिन्दू धर्म के महान् उपदेशों का अनुसरण करके और उसके साथ हिन्दू धर्म के स्वाभाविक विकास-रूप बौद्ध धर्म की अपूर्व सहृदयता को जोड़कर।\endnote{ १/४०२-०३;} 

विकास के लिए पहले स्वाधीनता चाहिए। तुम्हारे पूर्वजों ने आत्मा को स्वाधीनता दी थी, इसीलिए धर्म में उत्तरोत्तर विकास हुआ; परन्तु शरीर को उन्होंने सैकड़ों बन्धनों में डाल दिया, इसी से समाज का विकास रुक गया। पाश्चात्य देशों का हाल ठीक इसके उल्टा है। उनके समाज में स्वाधीनता है, पर धर्म में बिल्कुल नहीं। इसके फलस्वरूप वहाँ धर्म बड़ा अधूरा रह गया, पर समाज की भारी उन्नति हुई है। अब प्राच्य-समाज के पैरों से जंजीरें धीरे-धीरे खुल रही हैं और पाश्चात्य-धर्म के लिए भी वैसा ही हो रहा है।... पश्चिमी देश आध्यात्मिकता को सामाजिक उन्नति के माध्यम से ही प्राप्त करना चाहते हैं, पर प्राच्य देश थोड़ी भी सामाजिक शक्ति को धर्म के द्वारा ही पाना चाहते हैं। इसीलिए आधुनिक सुधारकों को पहले भारत के धर्म का नाश किए बिना सुधार का और कोई दूसरा उपाय ही नहीं सूझता। उन्होंने इस दिशा में चेष्टा भी की है, पर असफल रहे। इसका क्या कारण है? यह कि उनमें से किसी ने अपने धर्म का अच्छी तरह अध्ययनमनन नहीं किया, उनमें से किसी ने भी वह प्रशिक्षण नहीं लिया, जो इस सब धर्मों की माता को समझने के लिए जरूरी है! मेरा दावा है कि हिन्दू समाज की उन्नति के लिए हिन्दू धर्म के विनाश की जरूरत नहीं है; और ऐसी बात नहीं कि समाज की वर्तमान दशा धर्म के कारण हुई हो, वरन् इसलिए हुई कि धर्म का समाज में उपयोग नहीं किया गया। मैं इस कथन का हर शब्द अपने प्राचीन शास्त्रों से प्रमाणित करने को तैयार हूँ। मैं यही शिक्षा दे रहा हूँ और हमें इसी को कार्यरूप में परिणत करने के लिए जीवन भर चेष्टा करनी होगी।\endnote{ ३/३१७-१८;}


\section*{अपने इतिहास को जानो}

\vskip -3pt\addsectiontoTOC{अपने इतिहास को जानो}

अतीत से ही भविष्य बनता है। अतः यथासम्भव अतीत की ओर देखो, पीछे जो चिरन्तन निर्झर बह रहा है, भरपेट उसका जल पिओ और उसके बाद सामने देखो और भारत को उज्ज्वलतर, महत्तर और पहले से अधिक ऊँचा उठाओ। हमारे पूर्वज महान् थे। पहले हमें यह याद रखकर समझना होगा कि हम किन उपादानों से बने हैं, कौनसा खून हमारी नसों से बह रहा है।... इस विश्वास और अतीत गौरव के ज्ञान से हम निश्चय ही पहले से भी श्रेष्ठ भारत बनाएँगे। 

जो लोग सदा अपने अतीत की ओर दृष्टि लगाये रखते हैं, आजकल सभी लोग उनकी निन्दा करते हैं। वे कहते हैं कि इस प्रकार सदा अतीत की ओर देखते रहने के कारण ही हिन्दू जाति को नाना प्रकार के दुःख तथा संकट भोगने पड़े हैं। पर मेरी धारणा है कि इसका विपरीत ही सत्य है। जब तक हिन्दू जाति अपने अतीत को भूली हुई थी, तब तक वह अचेत अवस्था में पड़ी रही और अतीत की ओर दृष्टि जाते ही चहुँ ओर पुनर्जीवन के लक्षण दीख रहे हैं। अतीत के साँचे में भविष्य को ढालना होगा, अतीत ही भविष्य होगा।\endnote{ ५/१७९-८०;} 

अतः हिन्दू लोग अतीत का जितना ही अध्ययन करेंगे, उनका भविष्य उतना ही उज्ज्वल होगा और जो भी हर व्यक्ति को इस अतीत के बारे में शिक्षित करने की चेष्टा कर रहा है, वह स्वदेश का परम हितकारी है। भारत की अवनति इसलिए नहीं हुई कि हमारे पूर्वजों के नियम तथा आचार-व्यवहार बुरे थे, वरन् उनकी अवनति का कारण यह था कि उन नियमों और आचार-व्यवहारों को उनकी न्यायसंगत परिणति तक नहीं ले जाने दिया गया।\endnote{ ९/३५३;} 

सर्वप्रथम, हमारे उपनिषदों, पुराणों और अन्य सब शास्त्रों में जो अपूर्व सत्य छिपे हुए हैं, उन्हें इन ग्रन्थों के पन्नों से बाहर निकालकर, मठों की चहारदीवारियाँ भेदकर, वनों की निर्जनता से निकालकर, कुछ विशेष सम्प्रदायों के हाथ से छीनकर देश में सर्वत्र बिखेर देना होगा, ताकि ये सत्य दावानल के समान पूरे देश को चारों ओर से लपेट लें - उत्तर से दक्षिण और पूर्व से पश्चिम तक सर्वत्र फैल जाएँ - हिमालय से कन्याकुमारी और सिन्धु से ब्रह्मपुत्र तक सर्वत्र धधक उठें।\endnote{ ५/११६;}


\section*{राष्ट्रीय महापुरुषों के प्रति श्रद्धा}

\vskip -3pt\addsectiontoTOC{राष्ट्रीय महापुरुषों के प्रति श्रद्धा}

सर्वप्रथम महापुरुषों की पूजा चलानी होगी। जो लोग उन सब सनातन तत्त्वों की अनुभूति कर गए हैं, जैसे भारत में श्रीराम, श्रीकृष्ण, महावीर हनुमान तथा श्रीरामकृष्ण - इन्हें लोगों के समक्ष आदर्श या इष्ट के रूप में प्रस्तुत करना होगा। देश में भी रामचन्द्र और महावीर की पूजा चला दो तो जानूँ?... गीता का सिंहनाद करनेवाले श्रीकृष्ण की पूजा चला दो - शक्ति की पूजा चला दो!... इस समय आवश्यकता है महान् त्याग, महान् निष्ठा तथा महान् धैर्य की। स्वार्थगन्ध-रहित शुद्ध बुद्धि की सहायता से महान् उद्यम के साथ कमर कसकर सब कुछ ठीक-ठीक जानने के प्रयास में लग जाओ।\endnote{ ६/१३८;}


\section*{धर्म पर आघात मत करो}

\addsectiontoTOC{धर्म पर आघात मत करो}

धर्म को क्षति पहुँचाये बिना ही जनता की उन्नति - इसी को अपना आदर्श बना लो।... क्या समता, स्वतंत्रता, कार्य-कौशल तथा पुरुषार्थ में तुम पाश्चात्यों के भी गुरु बन सकते हो? क्या तुम उसी के साथ-ही-साथ स्वाभाविक आध्यात्मिक अन्तःप्रेरणा तथा साधनाओं में एक कट्टर सनातनी हिन्दू हो सकते हो? यही करना है और हम इसे अवश्य करेंगे~।\endnote{ २/३२१;} 

मेरा यही दावा है कि हिन्दू समाज की उन्नति के लिए हिन्दू धर्म का विनाश करने की कोई जरूरत नहीं है। ऐसी बात नहीं कि समाज की वर्तमान दुर्दशा हिन्दू धर्म की प्राचीन रीति-नीतियों और आचार-अनुष्ठानों के समर्थन के कारण हुई हो, वरन् ऐसा इसलिए हुआ कि धार्मिक तत्त्वों का सभी सामाजिक विषयों में भलीभाँति उपयोग नहीं हुआ।\endnote{ ३/३१७;} 

धर्म ही भारत की जीवनी-शक्ति है; और जब तक हिन्दू जाति अपने पूर्वजों से प्राप्त उत्तराधिकार को नहीं भूलेगी, तब तक संसार की कोई भी शक्ति उसका नाश नहीं कर सकती।\endnote{ ९/३५३;} 

यदि जीवन का रक्त सशक्त तथा शुद्ध है, तो शरीर में रोग के जीवाणु नहीं रह सकते। हमारी आध्यात्मिकता ही हमारा जीवन-रक्त है। यदि यह साफ बहता रहे, यदि यह शुद्ध तथा सशक्त बना रहे, तो सब कुछ ठीक है। राजनीतिक, सामाजिक, या चाहे जैसी भी जागतिक त्रुटियाँ हों, चाहे देश की निर्धनता ही क्यों न हो; यदि खून शुद्ध है, तो सब कुछ ठीक हो जाएगा।\endnote{ ५/१८१-८२} 

यदि तुम धर्म को फेंककर राजनीति, समाज-नीति अथवा अन्य किसी दूसरी नीति को अपनी जीवन-शक्ति का केन्द्र बनाने में सफल हो जाओ, तो उसका फल यह होगा कि तुम्हारा अस्तित्व तक न रह जाएगा। यदि तुम इससे बचना चाहो, तो तुम्हें अपने सारे कार्य अपनी जीवन-शक्ति-रूपी धर्म के भीतर से ही करने होंगे।... इस संसार में जैसे हर व्यक्ति को अपना-अपना मार्ग चुन लेना पड़ता है, वैसे ही हर राष्ट्र को भी चुन लेना पड़ता है। हमने युगों पूर्व अपना पथ निर्धारित कर लिया था और अब हमें उसी को पकड़े रहना चाहिए - उसी के अनुसार चलना चाहिए। फिर, हमारा यह चयन भी तो उतना कोई बुरा नहीं है। जड़ के बदले चैतन्य का, मनुष्य के बदले ईश्वर का चिन्तन करना, क्या संसार में इतनी बुरी चीज है? परलोक में दृढ़ आस्था, इस लोक के प्रति विरक्ति, प्रबल त्याग-शक्ति और ईश्वर तथा अविनाशी आत्मा में दृढ़ विश्वास - तुम लोगों में सतत विद्यमान है। क्या तुम इसे छोड़ सकते हो? नहीं, तुम इसे कभी नहीं छोड़ सकते। तुम कुछ दिन भौतिकवादी होकर और भौतिकवाद की चर्चा करके भले ही मुझे भ्रमित करने की चेष्टा करो, पर मैं जानता हूँ कि तुम क्या हो! तुम्हें बस, धर्म को थोड़ा ठीक से समझा देने भर की देर है, बस, तुम परम आस्तिक हो जाओगे। सोचो, अपना स्वभाव भला तुम कैसे बदल सकते हो? 

अतः भारत में किसी प्रकार का सुधार या उन्नति की चेष्टा करने के पहले धर्म-प्रचार आवश्यक है। भारत को समाजवादी या राजनीतिक विचारों से प्लावित करने के पहले जरूरी है कि उसमें आध्यात्मिक विचारों की बाढ़ ला दी जाए।\endnote{ ५/११५-१६;} 

हिन्दुओं को अपना धर्म छोड़ने की जरूरत नहीं। परन्तु उन्हें चाहिए कि धर्म को एक उचित मर्यादा के भीतर सीमित रखें और समाज को उन्नत करने के लिए स्वाधीनता प्रदान करें। भारत के सभी समाज-सुधारकों ने पुरोहितों के अत्याचारों और अवनति का उत्तरदायित्व धर्म के मत्थे मढ़ने की एक भयंकर भूल की और उसके अभेद्य गढ़ को ढहाने का प्रयत्न किया। नतीजा क्या हुआ? असफलता! बुद्धदेव से लेकर राजा राममोहन राय तक - सबने जाति-भेद को धर्म का एक अंग माना और जाति-भेद के साथ-हीसाथ धर्म पर भी पूरा आघात किया और वे सभी असफल रहे।\endnote{ २/३११;} 

भला हो या बुरा, भारत में हजारों वर्षों से धार्मिक आदर्श की धारा प्रवाहित हो रही है। भला हो या बुरा, भारत का वायु-मण्डल इसी धार्मिक आदर्श से अगणित शताब्दियों तक पूर्ण रहकर जगमगाता रहा है। भला हो या बुरा, हम इसी धार्मिक आदर्श के भीतर पैदा हुए और पले हैं - यहाँ तक कि अब धर्मभाव हमारे जन्म से ही रक्त में मिल गया है; हमारे रोम-रोम में वही धार्मिक आदर्श रम रहा है, वह हमारे शरीर का अंश और हमारी जीवनी-शक्ति बन गया है। इस वेगवती नदी ने हजारों वर्ष में अपने लिए जो पाट बनाया है, उसे भरे बिना; इस शक्ति की प्रतिक्रिया जगाये बिना, क्या तुम धर्म का परित्याग कर सकते हो? क्या तुम चाहते हो कि गंगा की धारा फिर बर्फ से ढँके हुए हिमालय को लौट जाए और फिर वहाँ से नवीन धारा बन कर प्रवाहित हो? यदि ऐसा होना सम्भव भी हो, तो भी यह कदापि सम्भव नहीं हो सकता कि यह देश अपने धर्ममय जीवन के विशिष्ट मार्ग को छोड़ सके और अपने लिए राजनीति या किसी अन्य नये मार्ग पर चलना प्रारम्भ कर दे। तुम उसी रास्ते से काम कर सकते हो, जिस पर बाधाएँ कम हों; और भारत के लिए धर्म का मार्ग ही अल्पतम बाधावाला मार्ग है। धर्म के पथ का अनुसरण करना हमारे जीवन का मार्ग है, हमारी उन्नति का मार्ग है और यही हमारे कल्याण का भी मार्ग है।\endnote{ ५/७५-७६;} 

मेरे कहने का यह अर्थ नहीं कि हमें दूसरी चीजों की जरूरत ही नहीं है, यह अर्थ नहीं कि राजनीतिक या सामाजिक उन्नति अनावश्यक है।... मैं तुम्हें सदा याद दिलाना चाहता हूँ कि यहाँ ये सारे विषय गौण हैं, मुख्य विषय धर्म ही है। भारतीय मन पहले धार्मिक है, उसके बाद ही कुछ और है।\endnote{ ५/१८३;} 

भारत में सामाजिक सुधार का प्रचार तभी हो सकता है, जब यह दिखा दिया जाए कि उस नयी प्रथा से आध्यात्मिक जीवन की उन्नति में किस प्रकार विशेष सहायता मिलेगी। राजनीति का प्रचार करने के लिए हमें दिखाना होगा कि उसके द्वारा हमारे राष्ट्रीय जीवन की आकांक्षा - आध्यात्मिक उन्नति - की इतनी अधिक पूर्ति हो सकेगी।\endnote{ ५/११६;}


\section*{भूखे भजन न होत गोपाला}

\addsectiontoTOC{भूखे भजन न होत गोपाला}

पहले रोटी और तब धर्म।\endnote{ ५/३२२;} वर्तमान हिन्दू समाज केवल उन्नत आध्यात्मिक लोगों के लिए ही गठित है, बाकी सबको वह निर्दयता से पीस डालता है। ऐसा क्यों? जो लोग थोड़ा-बहुत तुच्छ सांसारिक चीजों का भोग करना चाहते हैं, उनका क्या होगा? हमारा धर्म जैसे उत्तम, मध्यम और अधम - सभी तरह के अधिकारियों को अपने भीतर स्वीकार कर लेता है, वैसे ही हमारे समाज को भी उच्च-नीच सभी भाववाले लोगों को ले लेना होगा। इसके उपाय के रूप में, पहले हमें अपने धर्म का यथार्थ तत्त्व समझना और तब उसे सामाजिक विषयों में लगाना होगा। यह बहुत ही धीरे-धीरे होनेवाला, पर ठोस काम है। इसे करते रहना होगा।\endnote{ ३/३१८;} 

हमारे जो भाई अभी उच्चतम सत्य के योग्य नहीं हुए हैं, उनके लिए हल्का-सा भौतिकवाद शायद हितकर हो; पर उसे अपनी जरूरत के अनुरूप ढालकर लेना होगा। सभी देशों तथा समाजों में एक भ्रम फैला हुआ है; और विशेष दुःख की बात तो यह है कि भारत में भी थोड़े दिन हुए पहली बार इस भ्रान्ति ने प्रवेश किया है कि अधिकारी का विचार किए बिना सबके लिए समान व्यवस्था देना। सच तो यह है कि सबके लिए एक मार्ग हो ही नहीं सकता। यह आवश्यक नहीं है कि मेरी पद्धति ही आपकी भी हो।\endnote{ ५/४६;} 

भौतिक सभ्यता, यहाँ तक कि विलासिता की भी जरूरत होती है, क्योंकि उससे गरीबों को काम मिलता है। रोटी! रोटी! मुझे इस बात का विश्वास नहीं है कि जो भगवान मुझे यहाँ पर रोटी नहीं दे सकता, वह स्वर्ग में मुझे अनन्त सुख देगा! राम कहो! भारत को उठाना होगा, गरीबों को भोजन देना होगा, शिक्षा का प्रसार करना होगा और पुरोहितप्रपंच की बुराइयों को दूर करना होगा। पुरोहित-प्रपंच और सामाजिक अत्याचारों का कहीं नामो-निशान तक न रहे! सबके लिए अधिक अन्न और सबको अधिकाधिक सुविधाएँ मिलती रहें।... 

अपने धर्म पर अधिक बल और समाज को स्वाधीनता देते हुए हमें धीरे-धीरे यह अवस्था लानी होगी। प्राचीन धर्म से पुरोहित-प्रपंच की बुराइयों को उखाड़ दो, तो तुम्हें संसार का सबसे अच्छा धर्म प्राप्त हो जाएगा। मेरी बात समझते हो न? क्या तुम भारत का धर्म लेकर एक यूरोपीय समाज का निर्माण कर सकते हो? मुझे विश्वास है कि यह सम्भव है और एक दिन ऐसा अवश्य होगा।\endnote{ ३/३३४;} 

सभी लक्षणों से यही प्रकट हो रहा है कि समाजवाद या जनता द्वारा शासन का कोई भी रूप - उसे आप चाहे जिस भी नाम से पुकारें - आ रहा है। लोग निश्चय ही चाहेंगे कि उनकी सांसारिक जरूरतों की पूर्ति हो, उन्हें काम कम करना पड़े, उनका शोषण न हो, युद्ध न हो, खाने को अधिक मिले। इस देश की या कोई अन्य सभ्यता, यदि धर्म तथा मनुष्य की अच्छाई पर आधारित न हो, तो उसके टिक पाने की क्या गारंटी है?\endnote{ ४/३४३;}


\section*{नारी जागरण}

\addsectiontoTOC{नारी जागरण}

सर्वप्रथम स्त्रियों का वर्तमान दशा से उद्धार करना होगा। आम जनता को जगाना होगा। तभी तो भारतवर्ष का कल्याण होगा।\endnote{ ६/३७;} 

स्त्रियों की पूजा करके ही सभी राष्ट्र बड़े बने हैं। जिस देश में, जिस राष्ट्र में स्त्रियों की पूजा नहीं होती, वह देश या राष्ट्र, न कभी बड़ा बन सका है और न भविष्य में कभी बन सकेगा।\endnote{ ६/१८२;}


\section*{शिक्षा - मूलभूत आवश्यकता}

\addsectiontoTOC{शिक्षा - मूलभूत आवश्यकता}

शिक्षा! शिक्षा! केवल शिक्षा! यूरोप के अनेक नगरों में भ्रमण करते हुए और वहाँ के गरीबों के भी सुख-सुविधाओं तथा शिक्षा को देखकर मुझे अपने गरीब देशवासियों की याद आती थी और मैं आँसू बहाता था। यह अन्तर क्यों हुआ? उत्तर मिला - शिक्षा से। शिक्षा और आत्मविश्वास से उनके भीतर स्थित ब्रह्मभाव जाग गया है, जबकि हमारा क्रमशः संकुचित हो रहा है।\endnote{ ६/३११;} 

अपने निम्न वर्ग के लोगों के प्रति हमारा एकमात्र कर्त्तव्य है - उन्हें शिक्षा देना और उनके खोये हुए व्यक्तित्व का पुनः विकास करना।... उन्हें उत्तम विचार देने होंगे। उनके चारों ओर की दुनिया में जो कुछ हो रहा है, उसके बारे में उनकी आँखें खोल देनी होंगी; इसके बाद अपना उद्धार वे स्वयं कर लेंगे। हर राष्ट्र, हर पुरुष और हर स्त्री को अपना उद्धार स्वयं करना होगा। उन्हें विचार दे दो - बस, एक इसी सहायता की उन्हें जरूरत है, इसके फलस्वरूप बाकी सब कुछ स्वयं ही हो जाएगा। हमें केवल रासायनिक पदार्थों को एकत्र कर देना है, रवा बँधना तो प्राकृतिक नियमों से ही सम्पन्न होगा। उनके दिमागों में विचार भर देना ही हमारा कर्तव्य है, बाकी सब वे स्वयं कर लेंगे। भारत में बस यही करना है।\endnote{ २/३६९-७०;} 

अपनी समस्याओं को हल कर लेनेवाला एक कल्याणकारी और प्रबल जनमत तैयार करने में समय लगता है - काफी लम्बा समय लगता है और इस दौरान हमें प्रतीक्षा करनी होगी। अतः सामाजिक सुधार की पूरी समस्या का रूप यह है - कहाँ हैं वे लोग, जो सुधार चाहते हैं? पहले उन्हें तैयार करो। सुधार चाहनेवाले लोग हैं कहाँ? थोड़ेसे लोग किसी बात को उचित समझते हैं और उसे अन्य सब पर जबरन लादना चाहते हैं।... राष्ट्र को पहले शिक्षित करो।... अतः समाज-सुधार के लिए भी पहला कर्तव्य है - लोगों को शिक्षित करना। जब तक यह कार्य सम्पन्न नहीं होता, तब तक प्रतीक्षा करनी होगी।\endnote{ ५/११०-११;} 

जिस राष्ट्र की जनता में विद्या-बुद्धि का जितना ही अधिक प्रचार है, वह राष्ट्र उतना ही उन्नत है। भारत के सर्वनाश का मुख्य कारण यही है कि देश की सारी विद्या-बुद्धि, राज-शासन और दम्भ के बल पर मुट्ठी भर लोगों के एकाधिकार में रखी गयी। यदि हमें फिर से उन्नति करनी है, तो हमको उसी मार्ग पर चलना होगा, अर्थात् जनता में विद्या का प्रसार करना होगा।\endnote{ ६/३१०-११;}


\section*{आत्मनिर्भरता के लिए शिक्षा}

\addsectiontoTOC{आत्मनिर्भरता के लिए शिक्षा}

लोगों को यदि आत्मनिर्भर बनने की शिक्षा न दी जाए, तो सारे संसार की दौलत से भारत के एक छोटे-से गाँव की भी सहायता नहीं की जा सकती है। नैतिक तथा बौद्धिक - दोनों ही प्रकार की शिक्षा प्रदान करना हमारा पहला कार्य होना चाहिए।\endnote{ ६/३५०;}


\section*{विदेशों के साथ आदान-प्रदान}

\addsectiontoTOC{विदेशों के साथ आदान-प्रदान}

लेन-देन ही संसार का नियम है और यदि भारत फिर से उठना चाहे, तो यह परम आवश्यक है कि वह अपने रत्नों को बाहर लाकर पृथ्वी के राष्ट्रों में बिखेर दे और इसके बदले में वे जो कुछ भी दे सकें, उसे सहर्ष ग्रहण करे। \textbf{विस्तार ही जीवन है और संकोच मृत्यु; प्रेम ही जीवन है और द्वेष ही मृत्यु। } हमने उसी दिन से मरना शुरू कर दिया, जब से हम अन्य जातियों से घृणा करने लगे, और यह मृत्यु बिना इसके किसी दूसरे उपाय से रुक नहीं सकती कि हम फिर से विस्तार को अपनाएँ, जो कि जीवन का चिह्न है। 

अतः हमें पृथ्वी के सभी राष्ट्रों से मिलना-जुलना पड़ेगा। प्रत्येक हिन्दू जो विदेश भ्रमण करने जाता है, उन सैकड़ों लोगों की अपेक्षा अपने देश को अधिक लाभ पहुँचाता है, जो केवल अन्धविश्वासों तथा स्वार्थपरता की पोटली मात्र है।\endnote{ ३/३३२;} 

भारत के भाग्य का निपटारा उसी दिन हो चुका, जब उसने इस ‘म्लेच्छ’ शब्द का आविष्कार किया और दूसरों से अपना नाता तोड़ लिया।\endnote{ ३/३२४;} 

भारत के पतन और दुःख-दारिद्र्य का प्रधान कारण यह है कि उसने घोंघे की तरह अपना सर्वांग समेटकर अपना कार्यक्षेत्र संकुचित कर लिया था तथा आर्येतर दूसरे राष्ट्रों के लिए, जिन्हें सत्य की तृष्णा थी, अपने जीवनप्रद सत्य-रत्नों का भण्डार नहीं खोला था। हमारे पतन का एक और प्रधान कारण यह भी है कि हम लोगों ने बाहर जाकर दूसरे राष्ट्रों से अपनी तुलना नहीं की।... अतः तुम्हें विदेश जाना होगा, \textbf{आदान-प्रदान ही अभ्युदय का रहस्य है। } क्या हम सर्वदा दूसरों से लेते ही रहेंगे? क्या हम लोग सदा ही पश्चिमवासियों के चरणों में बैठकर ही सब बातें, यहाँ तक कि धर्म भी सीखेंगे? हाँ, हम उन लोगों से कल-कारखाने के काम सीख सकते हैं, और भी दूसरी अनेक बातें उनसे सीख सकते हैं, परन्तु हमें भी उन्हें कुछ सिखाना होगा; और वह है - हमारा धर्म, हमारी आध्यात्मिकता। संसार एक सर्वांगीण सभ्यता की अपेक्षा कर रहा है। शताब्दियों की अवनति, दुःखों तथा दुर्भाग्य के चक्र में पड़कर भी हिन्दू जाति उत्तराधिकार में प्राप्त धर्मरूपी जिन अमूल्य रत्नों को यत्नपूर्वक अपने हृदय से लगाए हुए है, उन्हीं रत्नों की आशा में संसार उसकी ओर आग्रहभरी दृष्टि से निहार रहा है। तुम्हारे पूर्वजों के उन्हीं अपूर्व रत्नों के लिए भारत से बाहर के लोग कैसे लालायित हो रहे हैं, यह मैं तुम्हें कैसे समझाऊँ।... इसलिए हमें भारत के बाहर जाना ही होगा। हमारी आध्यात्मिकता के बदले में वे लोग जो भी कुछ दें, हमें लेना होगा। चेतना-राज्य के अपूर्व तत्त्वों के बदले हम जड़-राज्य के अद्भुत तत्त्वों को प्राप्त करेंगे। चिर काल तक शिष्य रहने से हमारा काम नहीं चलेगा, हमें आचार्य भी बनना होगा। समभाव के न रहने पर मित्रता सम्भव नहीं। जब तक एक पक्ष सदा ही आचार्य का आसन पाता रहता है और दूसरा पक्ष सदा ही उसके चरणों में बैठकर शिक्षा ग्रहण किया करता है, तब तक दोनों में कभी समभाव की स्थापना नहीं हो सकती। यदि तुम्हारी अँग्रेज और अमरीकी जाति से समभाव रखने की इच्छा हो, तो जैसे तुम्हें उनसे शिक्षा प्राप्त करनी है, वैसे ही उन्हें शिक्षा देनी भी होगी और अब भी तुम्हारे पास अनेक शताब्दियों तक संसार को शिक्षा देने योग्य पर्याप्त सामग्री है।\endnote{ ५/२१०;} 

आज हमें जरूरत है - वेदान्तयुक्त पाश्चात्य विज्ञान की, ब्रह्मचर्य के आदर्श, और श्रद्धा तथा आत्मविश्वास की।... आज जरूरत है - विदेशी नियंत्रण को हटाकर हमारे विविध शास्त्रों, विद्याओं का अध्ययन हो और साथ-ही-साथ अंग्रेजी भाषा और पाश्चात्य विज्ञान भी सीखा जाए। हमें उद्योग-धन्धों की उन्नति के लिए यांत्रिक-शिक्षा भी प्राप्त करनी होगी, जिससे देश के युवक नौकरी ढूँढ़ने के बजाय अपनी आजीविका के लिए समुचित धनोपार्जन भी कर सकें।\endnote{ ८/२२९-३१;} 

क्या समता, स्वतंत्रता, कार्य-कुशलता तथा पुरुषार्थ में तुम पाश्चात्यों के भी गुरु बन सकते हो? क्या तुम उसी के साथ-साथ स्वाभाविक आध्यात्मिक अन्तःप्रेरणा तथा अध्यात्म-साधनाओं में एक कट्टर सनातनी हिन्दू हो सकते हो।\endnote{ २/२३१;} 

जैसे संघ स्थापना की पश्चिमी कार्यप्रणाली और बाह्य सभ्यता के भाव हमारे देश की नस-नस में समा रहे हैं, चाहे हम उनका ग्रहण करें या न करें, वैसे ही भारत की आध्यात्मिकता और दर्शन पाश्चात्य देशों को प्लावित कर रहे हैं। इस गति को कोई रोक नहीं सकता और हम भी पश्चिम की किसी-न-किसी प्रकार की भौतिकवादी सभ्यता का पूर्णतः प्रतिरोध नहीं कर सकते। इसका कुछ अंश सम्भव है, हमारे लिए अच्छा हो और आध्यात्मिकता का कुछ अंश पश्चिम के लिए हितकर हो। इसी तरह सामंजस्य की रक्षा हो सकेगी। यह बात नहीं कि हर चीज हमें पश्चिमवालों से सीखनी चाहिए, या पश्चिमवालों को जो कुछ सीखना है, वह सब हमसे ही सीखें।\endnote{ ५/६८;} 

गरीब लोगों के जीवन को इतने कड़े धार्मिक तथा नैतिक बन्धनों में जकड़ दिया गया है, जिससे उनका कोई लाभ नहीं है। उनके कामों में हस्तक्षेप मत करो। उन्हें भी संसार का थोड़ा आनन्द लेने दो। तुम देखोगे कि वे क्रमशः उन्नत होते जाते हैं और बिना किसी विशेष चेष्टा के उनके हृदय में अपने आप ही त्याग का उद्रेक होगा। पाश्चात्य जातियों से इस दिशा में (भोग के विषय में) हम थोड़ा-बहुत सीख सकते हैं, किन्तु यह शिक्षा ग्रहण करते समय हमें खूब सावधान रहना होगा। मुझे बड़े दुःख से कहना पड़ता है कि आजकल हम पाश्चात्य भावनाओं से अनुप्राणित जितने लोगों के उदाहरण पाते हैं, वे अधिकतर असफलता के हैं, इस समय भारत में हमारे मार्ग में दो बड़ी रुकावटें हैं - एक ओर हमारा प्राचीन हिन्दू समाज और दूसरी ओर अर्वाचीन यूरोपीन सभ्यता। इन दोनों में यदि कोई मुझसे एक को पसन्द करने के लिए कहे, तो मैं प्राचीन हिन्दू समाज को ही पसन्द करूँगा।\endnote{ ५/४७;} 

भय इस बात का है कि इस पाश्चात्य भाव-तरंग में कहीं हमारे चिर काल से अर्जित अमूल्य रत्न बह तो नहीं जाएँगे! उस प्रबल भँवर में पड़कर कहीं भारतभूमि भी ऐहिक सुख प्राप्त करने की रणभूमि में तो नहीं बदल जाएगी! असाध्य, असम्भव तथा जड़ से उखाड़ देनेवाले विदेशी ढंग का अनुकरण करने से हमारी ‘न घर के रहे न घाट के’ जैसी दशा तो नहीं हो जाएगी - और हम ‘इतो नष्टः ततो भ्रष्टः’ के उदाहरण तो नहीं बन जाएँगे! 

इसलिए हमें अपने घर की सम्पदा सर्वदा सम्मुख रखनी होगी और ऐसी चेष्टा करनी होगी, जिससे आम जनता तक अपने पैतृक धन को सर्वदा देख और जान सकें और इसी के साथ-साथ बाहर से प्रकाश प्राप्त करने के लिए हमको निर्भीक होकर अपने घर के सारे द्वार खोल देने होंगे। संसार के चारों ओर से प्रकाश की किरणें आएँ, पश्चिम का तीव्र प्रकाश भी आए! जो दुर्बल, दोषयुक्त है, उसका नाश होगा ही, उसे रखकर क्या लाभ? जो सबल है, शक्तिदायी है, वह अविनाशी है; उसका नाश कौन कर सकता है?\endnote{ १०/१३७;} 

अब प्रश्न यह है कि हमें भी संसार से कुछ सीखना है या नहीं? शायद दूसरी जातियों से हमें भौतिक विज्ञान सीखना पड़े। किस तरह संगठन और उसका परिचालन हो, विभिन्न शक्तियों को नियमानुसार काम में लगाकर कैसे थोड़े प्रयास से अधिक लाभ हो, आदि बातें हमें अवश्य दूसरों से सीखनी होंगी। पाश्चात्य लोगों से हमें शायद ये सब बातें कुछ-कुछ सीखनी ही होंगी। पर स्मरण रखना होगा कि हमारा उद्देश्य त्याग ही है। यदि कोई भोग और ऐहिक सुख को ही परम पुरुषार्थ मानकर भारत में उनका प्रचार करना चाहे, यदि कोई जड़-जगत् को ही भारतवासियों का ईश्वर कहने की धृष्टता करे, तो वह मिथ्यावादी है। इस पवित्र भारतभूमि में उसके लिए कोई स्थान नहीं है, भारतवासी उसकी बात नहीं सुनेंगे।\endnote{ ५/४५-४६;} 

जानते हो, मेरा मत क्या है? हम वेदान्त-धर्म के गूढ़ रहस्यों का पाश्चात्य जगत् में प्रचार करके उन महा शक्तिशाली राष्ट्रों की श्रद्धा तथा सहानुभूति प्राप्त करेंगे और आध्यात्मिक विषयों में सदा उनके गुरु-स्थानीय बने रहेंगे। दूसरी ओर अन्य ऐहिक विषयों में वे हमारे गुरु बने रहेंगे। जिस दिन भारतवासी धर्म-शिक्षा के लिए पाश्चात्यों के पीछे चलेंगे, उसी दिन इस अधःपतित राष्ट्र का अस्तित्व सदा के लिए नष्ट हो जाएगा। “हमें यह दे दो, वह दे दो” - ऐसे आन्दोलन से सफलता नहीं मिलेगी। वरन् उपर्युक्त आदान-प्रदान के फलस्वरूप जब दोनों पक्षों में पारस्परिक श्रद्धा और सहानुभूति का आकर्षण पैदा होगा, तो अधिक चिल्लाने की जरूरत ही नहीं रहेगी। वे स्वयं हमारे लिए सब कुछ कर देंगे। मेरा विश्वास है कि वेदान्त की चर्चा और वेदान्त का सर्वत्र प्रचार होने से हमारा तथा उनका - दोनों का ही विशेष लाभ होगा। इसके सामने राजनीतिक चर्चा मेरी समझ में निम्न स्तर का उपाय है। अपने इस विश्वास को कार्य में परिणत करने के लिए मैं अपने प्राण तक दे दूँगा। यदि तुम समझते हो कि किसी दूसरे उपाय से भारत का कल्याण होगा, तो उसी उपाय को अपना कर आगे बढ़ते जाओ।\endnote{ ६/९-१०;} 

भारत को यूरोप से बाह्य प्रकृति पर विजय प्राप्त करना सीखना है और यूरोप को भारत से अन्तःप्रकृति की विजय सीखनी होगी। तब न हिन्दू होंगे, न यूरोपियन - होगी आदर्श मानव-जाति, जिसने बाह्य और अन्तः - दोनों प्रकृतियों को जीत लिया होगा। हमने मानव-जाति के एक पहलू का विकास किया है, तो उन्होंने दूसरे का। चाहिए यह कि दोनों का मेल हो।\endnote{ ४/२५५;}


\section*{भारतीय परम्पराओं में ही नये भारत का गठन}

\addsectiontoTOC{भारतीय परम्पराओं में ही नये भारत का गठन}

सुधारकों से मैं कहूँगा कि मैं उनसे कहीं बढ़कर सुधारक हूँ। वे लोग केवल इधर-उधर थोड़ा-सा सुधार चाहते हैं और मैं आमूल सुधार चाहता हूँ। हम लोगों का मतभेद केवल सुधार की प्रणाली में है। उनकी प्रणाली विनाशात्मक है और मेरी रचनात्मक। मैं सुधार में नहीं, स्वाभाविक उन्नति में विश्वास करता हूँ। मैं अपने को ईश्वर के स्थान पर प्रतिष्ठित कर अपने समाज के लोगों के सिर पर यह उपदेश मढ़ने का साहस नहीं कर सकता कि ‘तुम्हें इसी भाँति चलना होगा, दूसरी तरह नहीं।’ मैं तो सिर्फ उस गिलहरी की तरह होना चाहता हूँ, जो राम के सेतु बाँधने के समय अपने योगदान-स्वरूप थोड़ासा बालू लाकर ही सन्तुष्ट हो गयी थी।... यह अद्भुत राष्ट्र-जीवन-रूपी यंत्र युग-युग से कार्य करता आया है, राष्ट्रीय जीवन का यह अद्भुत प्रवाह हम लोगों के सम्मुख बह रहा है। कौन जानता है, कौन साहसपूर्वक कह सकता है कि यह अच्छा है या बुरा, और यह किस प्रकार चलेगा?... राष्ट्रीय जीवन को जिस ईंधन की जरूरत है, देते जाओ, बस, वह अपने ढंग से उन्नति करता जाएगा; कोई उसकी उन्नति का मार्ग निर्दिष्ट नहीं कर सकता। हमारे समाज में अनेक बुराइयाँ हैं, पर ऐसी बुराइयाँ तो दूसरे समाजों में भी हैं।... दोषारोपण या निन्दा करने की क्या जरूरत?... बुराई तो हर कोई दिखा सकता है। मानव का सच्चा हितैषी तो वह है, जो इन कठिनाइयों से बाहर निकलने का उपाय बताये। 

क्या भारत में कभी सुधारकों का अभाव था? क्या तुमने भारत का इतिहास पढ़ा है? रामानुज, शंकर, नानक, चैतन्य, कबीर और दादू कौन थे? ये सब बड़े-बड़े धर्माचार्य कौन थे, जो भारत-गगन में अति उज्ज्वल नक्षत्रों की तरह एक-एक कर उदित हुए थे?... इन सबने प्रयत्न किया और उनका काम आज भी जारी है। भेद केवल इतना है कि वे आज के सुधारकों की तरह दम्भी नहीं थे; वे इनके जैसे अपने मुँह से कभी अभिशाप नहीं उगलते थे। उनके मुँह से मात्र आशीर्वाद ही निकलता था। उन्होंने कभी भर्त्सना नहीं की।... उन्होंने यह नहीं कहा, “पहले तुम दुष्ट थे, और अब तुम्हें अच्छा होना होगा।” उन्होंने यही कहा, “पहले तुम अच्छे थे, अब और भी अच्छे बनो।” इससे जमीन-आसमान का फर्क पैदा हो जाता है। हमें अपने स्वभाव के अनुसार उन्नति करनी होगी। विदेशी संस्थाओं ने बलपूर्वक जिस कृत्रिम प्रणाली को हममें प्रचलित करने की चेष्टा की है, उसके अनुसार काम करना वृथा है। वह असम्भव है। जय हो प्रभु! हम लोगों को तोड़-मरोड़कर नये सिरे से दूसरे राष्ट्रों के ढाँचे में गढ़ना असम्भव है। मैं दूसरी कौमों की सामाजिक प्रथाओं की निन्दा नहीं करता। वे उनके लिए अच्छी है, पर हमारे लिए नहीं। उनके लिए जो कुछ अमृत है, वही हमारे लिए विष हो सकता है। पहले यही बात सीखनी है। अन्य प्रकार के विज्ञान, अन्य प्रकार के परम्परागत संस्कार और अन्य प्रकार के आचारों से उनकी वर्तमान सामाजिक प्रथा गठित हुई है। पर हमारे पीछे हमारे अपने परम्परागत संस्कार और हजारों वर्षों के कर्म हैं। अतः हमें स्वभावतः अपने संस्कारों के अनुसार ही चलना होगा और हमें यह करना ही होगा।\endnote{ ५/१०९-१४;}


\section*{चाहिए सच्चे देशभक्तों की एक टोली}

\addsectiontoTOC{चाहिए सच्चे देशभक्तों की एक टोली}

मैंने जापान में सुना कि वहाँ की बालिकाओं का विश्वास है कि यदि उनकी गुड़ियों को हृदय से प्यार किया जाए, तो वे जीवित हो उठेंगी। जापानी बालिका अपनी गुड़िया को कभी नहीं तोड़ती।... मेरा भी विश्वास है कि यदि हतश्री, अभागे, निर्बुद्धि, पददलित, चिर-बुभुक्षित, झगड़ालू और ईर्ष्यालु भारतवासियों को भी कोई हृदय से प्यार करने लगे, तो भारत पुनः जाग्रत हो जाएगा। भारत तभी जागेगा, जब विशाल हृदयवाले सैकड़ों स्त्री-पुरुष भोग-विलास और सुख की सभी इच्छाओं को विसर्जित कर मन, वचन और शरीर से उन करोड़ों भारतीयों के हित के लिए सचेष्ट होंगे जो दरिद्रता तथा मूर्खता के अगाध सागर में निरन्तर डूबते जा रहे हैं।\endnote{ ६/३०७;} 

देशभक्त बनो, जिस राष्ट्र ने अतीत में हमारे लिए इतने बड़े-बड़े काम किए हैं, उसे प्राणों से भी प्यारा समझो।\endnote{ ५/९५;} 

ऐ मेरे भावी सुधारको, मेरे भावी देशभक्तो, तुम अनुभव करो, हृदय से अनुभव करो। क्या तुम हृदय से अनुभव करते हो कि देवों तथा ऋषियों की करोड़ों सन्तानें आज पशुतुल्य हो गयी हैं? क्या तुम हृदय से अनुभव करते हो कि लाखों लोग आज भूखों मर रहे हैं और लाखों लोग शताब्दियों से इसी भाँति भूखों मरते आए हैं? क्या तुम अनुभव करते हो कि अज्ञान के काले बादल ने सारे भारत को ढँक लिया है? क्या तुम यह सब सोचकर बेचैन हो जाते हो? क्या इस भावना ने तुम्हारी निद्रा छीन ली है? क्या यह भावना तुम्हारे रक्त के साथ मिलकर तुम्हारी धमनियों में बहती हैं? क्या वह तुम्हारे हृदय के स्पन्दन में मिल गयी है? क्या उसने तुम्हें पागल-सा बना दिया है?... यदि - ‘हाँ’, तो जानो कि तुमने देशभक्त होने की पहली सीढ़ी पर पैर रखा है।... माना कि तुम अनुभव करते हो; पर पूछता हूँ, क्या तुमने केवल व्यर्थ की बातों में शक्ति क्षय न करके, इस दुर्दशा के निवारण हेतु कोई यथार्थ कर्तव्य-पथ निश्चित किया है? क्या लोगों की भर्त्सना न करके, उनकी सहायता का कोई उपाय सोचा है? क्या स्वदेशवासियों को उनकी इस जीवन्मृत दशा से बाहर निकालने के लिए कोई मार्ग ठीक किया है? क्या उनके दुःखों को कम करने के लिए दो सांत्वनादायक शब्दों को खोजा है? यही दूसरी बात है।... पर इतने ही से पूरा न होगा। क्या तुम पर्वताकार विघ्नबाधाओं को लाँघकर कार्य करने के लिए तैयार हो? यदि सारी दुनिया हाथ में नंगी तलवार लेकर तुम्हारे विरोध में खड़ी हो जाए, तो भी क्या तुम जिसे सत्य समझते हो, उसे पूरा करने का साहस करोगे? यदि तुम्हारे पत्नी-पुत्र तुम्हारे प्रतिकूल हो जाएँ, भाग्यलक्ष्मी तुमसे रूठकर चली जाए, कीर्ति भी तुम्हारा साथ छोड़ दे, तो भी क्या तुम उस सत्य में लगे रहोगे? फिर भी क्या तुम उसके पीछे लगे रहकर अपने लक्ष्य की ओर बढ़ते रहोगे।... क्या तुममें ऐसी दृढ़ता है? बस यही तीसरी बात है। यदि तुममें ये तीन बातें हैं, तो तुममें से हर एक अद्भुत कार्य कर सकता है।\endnote{ ५/१२०-२१;} 

\vskip 2pt


\section*{राष्ट्र-निर्माण के अग्रदूतों से}

\addsectiontoTOC{राष्ट्र-निर्माण के अग्रदूतों से}

तेजस्वी युवकों का एक दल गठित करो और उसे अपनी उत्साह की अग्नि से प्रज्वलित कर दो। क्रमशः इसकी परिधि का विस्तार करते हुए इस संघ को बढ़ाते रहो।\endnote{ २/३५७;} 

\vskip 2pt

मेरा विश्वास युवा पीढ़ी - नयी पीढ़ी में है; मेरे कार्यकर्ता उन्हीं में से आएँगे और वे सिंहों की भाँति सभी समस्याओं के हल निकालेंगे।\endnote{ ४/२६१;} 

\vskip 2pt

मेरे वीरहृदय युवको!... अन्य किसी बात की जरूरत नहीं, जरूरत है तो केवल प्रेम, निश्छलता और धैर्य की। जीवन का अर्थ ही वृद्धि - विस्तार यानी प्रेम है। अतः प्रेम ही जीवन है, यही जीवन का एकमात्र नियम है; और स्वार्थपरता ही मृत्यु है। इस लोक तथा परलोक में भी यही बात सत्य है। परोपकार ही जीवन है, परोपकार न करना ही मृत्यु। तुम जितने नरपशु देखते हो, उनमें से नब्बे प्रतिशत मृत हैं, प्रेत हैं; क्योंकि मेरे बच्चो, जिसमें प्रेम नहीं है, वह जी भी नहीं सकता। सबके लिए तुम्हारे दिल में दर्द हो - गरीब, अपढ़ तथा पददलितों के दुःख को महसूस करो; तब तक महसूस करो, जब तक कि तुम्हारे हृदय की धड़कन न रुक जाए, मस्तिष्क न चकराने लगे और तुम्हें ऐसा प्रतीत होने लगे कि तुम पागल हो जाओगे - इसके बाद अपना दिल खोलकर ईश्वर के चरणों में रख दो; और तब तुम्हें शक्ति, सहायता और अदम्य उत्साह की प्राप्ति होगी। पिछले दस वर्षों से मैं अपना मूलमंत्र घोषित करता आया हूँ - संघर्ष करते रहो; और अब भी कहता हूँ - सतत संघर्ष करते चलो। जब चारों ओर अँधेराही-अँधेरा दिखता था, तब भी मैं कहता था - संघर्ष करते रहो; और अब जब थोड़ाथोड़ा उजाला दिखायी दे रहा है, तब भी कहता हूँ - संघर्ष करते चलो। डरो मत, मेरे बच्चो! अनन्त नक्षत्रखचित आकाश की ओर भयभीत दृष्टि से ऐसे मत ताको, मानो वह हमें कुचल ही डालेगा। धीरज रखो। देखोगे कि कुछ ही घण्टों में वह पूरा-का-पूरा तुम्हारे पैरों तले आ गया है। धैर्य रखो। न धन से काम होता है, न नाम से; न यश काम आता है, न विद्या; प्रेम से ही सब कुछ होता है। चरित्र ही कठिनाइयों की संगीन दीवारें तोड़कर अपना रास्ता बना सकता है।\endnote{ ३/३३२-३३;} 

\vskip 2pt

तथाकथित धनिकों पर भरोसा न करो, वे जीवित की अपेक्षा मृत ही अधिक हैं। आशा तुम लोगों से है - जो विनीत, निरभिमानी और विश्वास-परायण हैं। ईश्वर के प्रति आस्था रखो। किसी चालबाजी की जरूरत नहीं; उससे कुछ नहीं होता। दुखियों का दर्द समझो और ईश्वर से सहायता की प्रार्थना करो - वह अवश्य मिलेगी। 

मैं बारह वर्ष तक हृदय में यह बोझ लादे और सिर में यह विचार लिए बहुतसे तथाकथित धनिकों और अमीरों के द्वार-द्वार पर घूमा। हृदय का रक्त बहाते हुए मैं आधी पृथ्वी का चक्कर लगाकर इस अजनबी देश (अमेरिका) में सहायता माँगने आया। परन्तु ईश्वर अनन्त शक्तिमान हैं - मैं जानता हूँ, वे मेरी सहायता करेंगे। मैं इस देश में भूख या जाड़े से भले ही मर जाऊँ, पर युवको! मैं गरीबों, अशिक्षितों और उत्पीड़ितों के लिए इस सहानुभूति तथा प्राणपण चेष्टा को, तुम्हें थाती के तौर पर सौंपता हूँ। जाओ - इसी क्षण उन पार्थसारथी (श्रीकृष्ण) के मन्दिर में जाओ, जो गोकुल के दीन-हीन ग्वालों के सखा थे, जो गुहक चाण्डाल को भी गले लगाने में नहीं हिचके, जिन्होंने अपने बुद्धावतार-काल में अमीरों का निमंत्रण अस्वीकार करके एक गणिका के भोजन का निमंत्रण स्वीकार किया और उसे उबारा; जाओ उनके पास, जाकर साष्टांग प्रणाम करो और उनके सम्मुख एक महाबलि दो, अपने समस्त जीवन की बलि दो - उन दीन-हीनों और उत्पीड़ितों के लिए, जिनके लिए प्रभु युग-युग में अवतार लिया करते हैं; और जिन्हें वे सबसे अधिक प्यार करते हैं। और तब प्रतिज्ञा करो कि अपना सारा जीवन इन तीस करोड़ लोगों के उद्धार-कार्य में लगा दोगे, जो दिनो-दिन अवनति के गर्त में गिरते जा रहे हैं।... 

तथाकथित धनिकों या बड़े लोगों का रुख मत देखो - हृदयहीन, कोरे बुद्धिवादी लेखकों और समाचार-पत्रों में प्रकाशित उनके निस्तेज लेखों की भी परवाह न करो। विश्वास, सहानुभूति - दृढ़ विश्वास और ज्वलन्त सहानुभूति चाहिए! जीवन तुच्छ है, मरण भी तुच्छ है; भूख तुच्छ है, जाड़ा भी तुच्छ है। जय हो प्रभु की! आगे कूच करो - प्रभु ही हमारे सेनानायक हैं। पीछे मत देखो। कौन गिरा, पीछे मत देखो - आगे बढ़ो, बढ़ते चलो! भाइयो, इसी तरह हम आगे बढ़ते जाएँगे - एक गिरेगा, तो दूसरा वहाँ डट जाएगा।\endnote{ १/४०४-०६;}


\section*{हमारा मूलमंत्र त्याग और सेवा}

\addsectiontoTOC{हमारा मूलमंत्र त्याग और सेवा}

हमारी कार्यविधि बड़ी सरलता के साथ समझायी जा सकती है। वह है - बस, राष्ट्रीय जीवन को पुनः स्थापित करना। बुद्ध ने ‘त्याग’ का प्रचार किया था, भारत ने सुना और इसके बावजूद छह शताब्दियों में ही वह अपनी समृद्धि के सर्वोच्च शिखर पर पहुँच गया। यही रहस्य है। भारत के राष्ट्रीय आदर्श हैं - त्याग और सेवा। आप उसकी इन धाराओं में तीव्रता लाइए और बाकी सब अपने आप ठीक हो जाएगा।\endnote{ ४/२६५;}


\section*{आत्म-विश्वास और आत्मश्रद्धा}

\addsectiontoTOC{आत्म-विश्वास और आत्मश्रद्धा}

सबसे बड़ी बात यह है कि ईश्वर में विश्वास होने से भी पहले, स्वयं में विश्वास रहे, परन्तु कठिनाई की बात यह है कि दिन-प्रतिदिन हमारा स्वयं पर से विश्वास घटता जा रहा है। सुधारकों के विरुद्ध मेरी यही आपत्ति है।\endnote{ ४/२६१;} 

हमें जिस चीज की जरूरत है, वह है श्रद्धा। दुर्भाग्यवश भारत से इसका प्रायः लोप हो गया है और हमारी वर्तमान दुर्दशा का कारण भी यही है। एकमात्र इस श्रद्धा के भेद से ही मनुष्य-मनुष्य में अन्तर पाया जाता है? इसका और दूसरा कारण नहीं। यह श्रद्धा ही है, जो एक मनुष्य को बड़ा और दूसरे को दुर्बल और छोटा बनाती है। मेरे गुरुदेव कहा करते थे - “जो अपने को दुर्बल सोचता है, वह दुर्बल ही हो जाता है” - और यह बिल्कुल सही है। इस श्रद्धा को तुम्हें पाना ही होगा। पश्चिमी जातियों द्वारा अर्जित की हुई, जो भौतिक शक्ति तुम देख रहे हो, वह इस श्रद्धा का ही फल है। वे अपने दैहिक बल में विश्वासी हैं और यदि तुम अपनी आत्मा पर विश्वास करो, तो वह कितना अधिक कारगर होगा?\endnote{ ५/२१३;}


\section*{साथ ही तीव्र कर्मठता भी चाहिए}

\addsectiontoTOC{साथ ही तीव्र कर्मठता भी चाहिए}

हम वही उद्यम, वही स्वाधीनता का प्रेम, वही आत्मनिर्भरता, वही अटल धैर्य, वही कार्यदक्षता, वही एकता और वही उन्नति-तृष्णा चाहते हैं। हम बीती बातों की उधेड़बुन छोड़कर अनन्त तक विस्तारित अग्रसर दृष्टि चाहते हैं। और आ-पाद-मस्तक नसनस में बहनेवाला रजोगुण चाहते हैं।\endnote{ १०/१३५;} 

प्रत्येक मनुष्य और प्रत्येक राष्ट्र को महान् बनाने के लिए तीन चीजें आवश्यक हैं -

\begin{enumerate}
\item सदाचार की शक्ति में विश्वास। 

 \item ईर्ष्या और सन्देह का परित्याग। 

 \item जो सत् बनने या सत् कर्म करने के लिए यत्नवान हो, उनकी सहायता करना।\endnote{ २/३२४;} 

\end{enumerate}

संगठन की शक्ति का हमारे स्वभाव में पूर्ण अभाव है, परन्तु हमें उसका विकास करना होगा। इसका सबसे बड़ा रहस्य है - ईर्ष्या का अभाव होना। अपने भाइयों के विचारों को मान लेने के लिए सदैव प्रस्तुत रहो और उनसे हमेशा मेल बनाए रखने की चेष्टा करो। यही सम्पूर्ण रहस्य है।\endnote{ २/३८२;} 

तुम्हारे लिए यह परम आवश्यक है कि अपनी शक्ति को व्यर्थ नष्ट करना और प्रायः निरर्थक बातें बनाना छोड़कर, अंग्रेजों से नेताओं की आज्ञा का तुरन्त पालन, ईर्ष्याहीनता, अथक लगन और अटूट आत्मविश्वास की शिक्षा प्राप्त करना। अंग्रेज जब किसी काम के लिए एक नेता चुन लेते हैं, तो हार-जीत में सदा उसका साथ देते हैं और उसकी आज्ञा का पालन करते हैं। यहाँ भारत में हर व्यक्ति नेता बनना चाहता है, आज्ञा-पालन करनेवाला कोई भी नहीं है। आज्ञा देने की क्षमता प्राप्त करने से पहले प्रत्येक व्यक्ति को आज्ञा-पालन करना सीखना होगा। हमारी ईर्ष्याओं का कहीं अन्त नहीं है; और जो हिन्दू जितना अधिक महत्त्वपूर्ण है, वह उतना ही अधिक ईर्ष्यालु है। जब तक हिन्दू ईर्ष्या से बचना और नेताओं की आज्ञा का पालन करना नहीं सीखते, उनमें संगठन की क्षमता नहीं आएगी।\endnote{ ४/२५५;} 

\vskip 2pt

मुझे आशा है, हममें से प्रत्येक में पर्याप्त उदारता होगी और साथ ही खूब दृढ़ निष्ठा भी होगी...~। मैं ‘कट्टरता’ वाली निष्ठा भी चाहता हूँ और भौतिकवादियों का उदार भाव भी चाहता हूँ। हमें ऐसे ही हृदय की जरूरत है, जो समुद्र-सा गम्भीर और आकाशसा उदार हो। हमें संसार की किसी भी उन्नत जाति की तरह उन्नतिशील होना है और साथ ही अपनी परम्पराओं के प्रति वही श्रद्धा तथा निष्ठा रखनी है।\endnote{ ५/७१;}


\section*{श्रमिकों का जागरण होगा}

\addsectiontoTOC{श्रमिकों का जागरण होगा}

पुरोहित, सैनिक, व्यापारी और मजदूर - चारों वर्ण बारी-बारी से मानवी समाज पर शासन करते हैं। हर शासन का अपना गौरव और अपना दोष होता है। ब्राह्मण के राज्य में वंश के आधार पर भयंकर पृथकता रहती है - पुरोहित तथा उनके वंशज सब प्रकार के अधिकारों से सुरक्षित रहते हैं। उनके सिवा किसी को कोई ज्ञान नहीं होता और उनके सिवा किसी को शिक्षा देने का अधिकार नहीं है। इस विशिष्ट युग में सब विद्याओं की नींव पड़ती है, यह इसका गौरव है। ब्राह्मण मन को उन्नत करते हैं, क्योंकि मन द्वारा ही वे राज्य करते हैं। 

\vskip 2pt

क्षत्रिय शासन क्रूर और अन्यायी होता है, परन्तु उनमें पृथकता नहीं रहती और उनके युग में कला और सामाजिक संस्कृति उन्नति के शिखर तक पहुँच जाती है। 

\vskip 2pt

तब वैश्य शासन आता है। इसमें कुचलने और खून चूसने की मौन शक्ति बड़ी भीषण होती है। इसका लाभ यह है कि व्यापारी सर्वत्र जाता है, अतः वह पहले दोनों युगों में एकत्र हुए विचारों को फैलाने में सफल होता है। उनमें क्षत्रियों से भी कम पृथकता होती है, पर सभ्यता की अवनति शुरू हो जाती है। 

\vskip 2pt

अन्त में आएगा मजदूरों का शासन। उसका लाभ होगा भौतिक सुखों का समान वितरण - और उससे हानि होगी, शायद संस्कृति का निम्न स्तर पर गिर जाना। साधारण शिक्षा का बहुत प्रचार होगा, परन्तु समाज में असामान्य प्रतिभाशाली लोग कम होते जाएँगे। परन्तु पहले तीनों का राज्य हो चुका है। अब शूद्र शासन का युग आ गया है - वे अवश्य राज्य करेंगे, और उन्हें कोई रोक नहीं सकता।\endnote{ ५/३८६-८७;} 

\vskip 2pt

एक ऐसा समय आएगा, जब शूद्रत्व सहित शूद्रों का प्राबल्य होगा; आजकल जैसे शूद्र जाति वैश्यत्व या क्षत्रियत्व लाभ कर अपना बल दिखा रही है, वैसे नहीं; वरन् शूद्र जाति सब देशों में अपने शूद्रोचित धर्म-कर्म सहित समाज में आधिपत्य प्राप्त करेगी। यूरोप में अभी से आकाश में इसकी लालिमा दिखने लगी है और इसका फलाफल सोचकर सभी लोग घबराये हुए हैं। सोशलिज्म, अनार्किज्म, निहिलिज्म आदि सम्प्रदाय इस विप्लव के आगे चलनेवाली घ्वजाएँ हैं।\endnote{ ९/२१९-२०;} 

तुम नहीं देखते, परन्तु मैं आवरण के पीछे भविष्य की दुनिया में होनेवाली घटनाओं की छाया को देख रहा हूँ। ईश्वर की कृपा से वर्षों के सूक्ष्म निरीक्षण से मुझे यह अन्तर्दृष्टि प्राप्त हुई है। अध्ययन और यात्रा करो, यही साधना है। जैसे ज्योतिर्विज्ञानी दूरवीक्षण यंत्र के द्वारा नक्षत्रों की गति का अध्ययन करते हैं, वैसे ही इस संसार की भावी घटनावली मेरे दृष्टिक्षेत्र के सम्मुख प्रकट हो जाती है। मेरी बात पर विश्वास करो, शूद्रों का यह उत्थान पहले रूस में होगा और तब चीन में। उसके बाद भारत का उदय होगा और वह भावी जगत् के रूपायन में एक महत्त्वपूर्ण भूमिका अदा करेगा।\endnote{ ए. न्न. झ. ३३५;} 

यदि ऐसा राज्य स्थापित करना सम्भव हो, जिसमें ब्राह्मण युग का ज्ञान, क्षत्रिय युग की सभ्यता, वैश्य युग का प्रचार-भाव और शूद्र युग की समानता लायी जा सके - उनके दोषों को त्यागकर - तो वह एक आदर्श राज्य होगा। परन्तु क्या यह सम्भव है? 

अन्य सभी मतवाद अपनाये गए हैं और दोषयुक्त सिद्ध हुए हैं। इसकी भी अब परीक्षा होने दो - यदि और किसी कारण से नहीं, तो उसकी नवीनता के लिए ही सही। सर्वदा एक ही वर्ग के लोगों को सुख और दुःख मिलने की अपेक्षा, सुख और दुःख का बँटवारा बेहतर है। संसार में भलाई और बुराई का योग समान ही रहता है। नये विधान में, वह भार बस, एक कन्धे से दूसरा कन्धा बदल लेगा।\endnote{ ५/३८७;}


\section*{नया भारत}

\addsectiontoTOC{नया भारत}

एक नवीन भारत निकल पड़े - हल पकड़कर, किसानों की कुटी भेदकर, मछुए, माली, मोची, मेहतरों की कुटीरों से। निकल पड़े बनियों की दुकानों से, भुजवा के भाड़ के पास से, कारखाने से, हाट से, बाजार से। निकल पड़े झाड़ियों, जंगलों, पहाड़ों, पर्वतों से। इन लोगों ने हजारों वर्षों तक नीरव अत्याचार सहन किया है - उससे पायी है अपूर्व सहनशीलता। सनातन दुःख उठाया, जिससे पायी है अटल जीवनी शक्ति। ये लोग मुट्ठी भर सत्तू खाकर दुनिया को उलट सकेंगे। आधी रोटी मिली, तो तीनों लोक में इनका तेज न अटेगा? ये रक्तबीज के प्राणों से युक्त हैं। और पाया है सदाचार-बल, जो तीनों लोकों में नहीं है। 

इतनी शान्ति, इतनी प्रीति, मौन रहकर दिन-रात इतना खटना और काम के वक्त सिंह-विक्रम!! अतीत के कंकालो! यही है तुम्हारे सामने तुम्हारा उत्तराधिकारी भावी भारत।\endnote{ ८/१६७-६८;}


\section*{इस युग का केन्द्र होगा - भारत}

\addsectiontoTOC{इस युग का केन्द्र होगा - भारत}

क्या भारत मर जाएगा? तब तो संसार से सारी आध्यात्मिकता का समूल नाश हो जाएगा, सारे सदाचारपूर्ण आदर्श जीवन का विनाश हो जाएगा, धर्मों के प्रति सारी मधुर सहानुभूति नष्ट हो जाएगी, सारी भावुकता का भी लोप हो जाएगा और उसके स्थान पर कामरूपी देव और विलासिता-रूपी देवी राज्य करेंगे। धन उनका पुरोहित होगा। छल, पाशविक बल और स्पर्धा - ये ही उनकी पूजा-पद्धति होंगी और मानवात्मा उनकी बलिसामग्री हो जाएगी। ऐसा कभी नहीं हो सकता। क्रियाशक्ति की अपेक्षा सहनशक्ति कई गुना प्रबल होती है। घृणा के बल से प्रेम का बल अनन्त गुना सबल है।\endnote{ ९/३७७;} 

सुदीर्घ रजनी अब समाप्त होती हुई जान पड़ती है। महादुःख का प्रायः अन्त ही प्रतीत होता है। महानिद्रा में निमग्न शव मानो जाग्रत हो रहा है। इतिहास की बात तो दूर रही, सुदूर अतीत के जिस घोर अँधेरे को भेदने में जनश्रुतियाँ भी असमर्थ हैं, वहीं से एक आवाज हमारे पास आ रही है। ज्ञान, भक्ति और कर्म के अनन्त हिमालय स्वरूप हमारी मातृभूमि भारत की हर चोटी पर प्रतिध्वनित होकर यह आवाज मृदु, दृढ़, परन्तु अभ्रान्त स्वर में हमारे पास तक आ रही है। जितना समय बीतता है, उतनी ही वह और भी स्पष्ट तथा गम्भीर होती जाती है - और देखो, वह निद्रित भारत अब जागने लगा है। मानो हिमालय के प्राणप्रद वायु-स्पर्श से मृतदेह के शिथिलप्राय अस्थि-मांस तक में प्राण-संचार हो रहा है, जड़ता धीरे-धीरे दूर हो रही है। जो अन्धे हैं, वे ही देख नहीं सकते; जो विकृत-बुद्धि हैं, वे ही समझ नहीं सकते कि हमारी मातृभूमि अपनी गम्भीर निद्रा से अब जाग रही है। अब कोई उसे रोक नहीं सकता। अब यह फिर सो भी नहीं सकती। कोई बाह्य शक्ति अब इसे दबा नहीं सकती।\endnote{ ५/४२;} 

प्राचीन काल में बहुत-सी चीजें अच्छी थीं और अनेक बुरी भी। उत्तम वस्तुओं की रक्षा करनी होगी, परन्तु प्राचीन भारत से भविष्य का भारत कहीं अधिक महान् होगा।\endnote{ ४/३०९;} 

अतीत तो हमारा गौरवमय था ही, परन्तु मेरा दृढ़ विश्वास है कि हमारा भविष्य और भी अधिक गौरवमय होगा।\endnote{ ३/३३२;} 

भारत का पुनरुत्थान होगा, पर जड़ की शक्ति से नहीं, वरन् आत्मा की शक्ति से। यह उत्थान विनाश से नहीं, वरन् शान्ति और प्रेम की ध्वजा लेकर, संन्यासियों के वेश से होगा - धन की शक्ति से नहीं, बल्कि भिक्षापात्र की शक्ति से सम्पन्न होगा।... अपने समक्ष मैं एक स्पष्ट दृश्य देख रहा हूँ कि हमारी यह प्राचीन माता एक बार पुनः जाग्रत होकर नव-यौवनपूर्ण और पूर्व से कहीं अधिक महिमान्वित होकर अपने सिंहासन पर विराजित है। शान्ति और आशीर्वाद की वाणी के साथ सारे संसार में उसके नाम की घोषणा कर दो।\endnote{ ९/३८०-८१;} 

जगत् में बड़ी-बड़ी विजयी जातियाँ हो चुकी हैं, हम भी महान् विजेता रह चुके हैं। हमारी विजय की गाथा को भारत के महान् सम्राट् अशोक ने धर्म तथा आध्यात्मिकता की ही विजय बताया है। एक बार फिर भारत को विश्वविजय करना होगा। यही मेरे जीवन का स्वप्न है।... यही हमारे सामने वह महान् आदर्श है और प्रत्येक को इसके लिए तैयार रहना होगा - वह आदर्श है भारत की विश्वविजय - इससे छोटा कोई आदर्श न चलेगा और हम सभी को इसके लिए तैयार रहना होगा। इसके लिए भरसक कोशिश करनी होगी।... उठो भारत, तुम अपनी आध्यात्मिकता के द्वारा विश्व पर विजय प्राप्त करो! जैसा कि इसी देश में सर्वप्रथम प्रचार किया गया है, प्रेम ही घृणा पर विजय प्राप्त करेगा, घृणा घृणा को नहीं जीत सकती, हमें भी वैसा ही करना पड़ेगा। भौतिकवाद और उससे उत्पन्न क्लेश भौतिकवाद से कभी दूर नहीं हो सकते। जब एक सेना दूसरी सेना पर विजय प्राप्त करने की चेष्टा करती है, तो वह मानव-जाति को पशु बना देती है और इस प्रकार वह पशुओं की संख्या बढ़ा देती है। आध्यात्मिकता पाश्चात्य देशों पर अवश्य विजय प्राप्त करेगी। पाश्चात्य लोग क्रमशः अनुभव कर रहे हैं कि उन्हें राष्ट्र के रूप में बने रहने के लिए आध्यात्मिकता की जरूरत है। वे इसकी प्रतीक्षा कर रहे हैं; चाव से इसकी बाट जोह रहे हैं। उसकी पूर्ति कहाँ से होगी? वे आदमी कहाँ हैं, जो भारतीय ऋषियों के उपदेश जगत् के सब देशों में पहुँचाने के लिए तैयार हों? कहाँ हैं वे लोग, जो सब कुछ इसलिए छोड़ने को तैयार हों कि ये हितकर उपदेश संसार के कोने-कोने तक फैल जाएँ? सत्य के प्रचार के लिए ऐसे ही वीरहृदय लोगों की जरूरत है। वेदान्त के महासत्यों को फैलाने के लिए ऐसे ही वीर कर्मियों को बाहर जाना होगा। जगत् इसे चाहता है, जगत् इसके बिना नष्ट हो जाएगा। सारा पाश्चात्य जगत् मानो एक ज्वालामुखी के मुख पर बैठा है, जो कल ही फूटकर उसे चूर-चूर कर सकता है। उन्होंने सारा संसार छान डाला, पर उन्हें तनिक भी शान्ति नहीं मिली। उन्होंने इन्द्रिय-सुख का प्याला पीकर खाली कर डाला, पर फिर भी उससे उन्हें तृप्ति नहीं मिली। भारत के धार्मिक विचारों को पाश्चात्य देशों की नस-नस में भर देने का यही समय है। इसलिए,... हमें बाहर जाना ही पड़ेगा, अपनी आध्यात्मिकता तथा दार्शनिकता से हमें जगत् को जीतना होगा। दूसरा कोई उपाय ही नहीं है, हमें निश्चित रूप से इसी को लेकर करना या मरना होगा।\endnote{ ५/१७०;} 

अति प्राचीन काल में एक बार भारतीय अध्यात्म-विद्या यूनानी उत्साह के साथ मिलकर, रोमन, ईरानी आदि शक्तिशाली राष्ट्रों के अभ्युदय में सहायक हुई। सिकन्दर शाह के दिग्विजय के बाद इन दोनों महा जलप्रपातों के संघर्ष के फलस्वरूप ईसा आदि नाम से प्रसिद्ध आध्यात्मिक तरंग ने प्रायः आधे संसार को प्लावित कर दिया। पुनः इस प्रकार के मिश्रण से अरब का अभ्युदय हुआ, जिससे आधुनिक यूरोपीय सभ्यता की नींव पड़ी और ऐसा लगता है कि इस युग में भी पुनः इन दोनों महाशक्तियों का सम्मिलनकाल आ पहुँचा है।\endnote{ १०/१३४} 

\delimiter

\addtoendnotes{\protect\end{multicols}}

\addtocontents{toc}{\protect\par\egroup}

