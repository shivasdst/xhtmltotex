
\chapter{स्वयं पर विश्वास }

\toendnotes{स्वयं पर विश्वास}

\addtoendnotes{\protect\begin{multicols}{3}}

विश्वास - यही महानता का एकमात्र रहस्य है। यदि पुराणों में कहे गए तैतीस करोड़ देवताओं के ऊपर; और विदेशियों ने बीच-बीच में जिन देवताओं को तुम्हारे बीच घुसा दिया है, उन सब पर भी, यदि तुम्हारी श्रद्धा हो और अपने आप पर श्रद्धा न हो, तो तुम कदापि मोक्ष के अधिकारी नहीं हो सकते।\endnote{ ५/८६;} 

\vskip 2pt

संसार का इतिहास उन थोड़े से व्यक्तियों का इतिहास है, जिनमें आत्मविश्वास था। यह विश्वास अन्तःस्थित देवत्व को ललकार कर प्रकट कर देता है। तब व्यक्ति कुछ भी कर सकता है, सर्वसमर्थ हो जाता है। असफलता तभी होती है, जब तुम अन्तःस्थ अमोघ शक्ति को अभिव्यक्त करने का यथेष्ट प्रयत्न नहीं करते। जिस क्षण व्यक्ति या राष्ट्र आत्मविश्वास खो देता है, उसी क्षण उसकी मृत्यु आ जाती है।\endnote{ ९/१९५;} 

\vskip 2pt

जिसमें आत्मविश्वास नहीं है, वही नास्तिक है। प्राचीन धर्मों के अनुसार जो ईश्वर में विश्वास नहीं करता, वह नास्तिक है। नूतन धर्म कहता है,जो आत्मविश्वास नहीं रखता, वही नास्तिक है।\endnote{ ८/१२;} 

\vskip 2pt

यह कभी न सोचना कि आत्मा के लिए कुछ भी असम्भव है। ऐसा सोचना ही भयानक\break नास्तिकता है। यदि पाप नामक कोई वस्तु है, तो यह कहना कि मैं दुर्बल हूँ या अन्य कोई\break दुर्बल है - यही पाप है।\endnote{ ८/१८;} 

\vskip 2pt

तुम जो कुछ भी सोचोगे, वही हो जाओगे। यदि तुम अपने को दुर्बल समझोगे, तो दुर्बल हो जाओगे; बलवान सोचोगे, तो बलवान हो जाओगे।\endnote{ ५/२९;} 

\vskip 2pt

इन असफलताओं, इन त्रुटियों के रहने में हर्ज भी क्या है? मैंने गाय को कभी झूठ बोलते नहीं सुना, पर वह सदा गाय ही रहती है; मनुष्य कभी नहीं हो जाती। अतः यदि बार-बार असफल हो जाओ, तो भी क्या? कोई हानि नहीं। हजार बार इस आदर्श को हृदय में धारण करो; और यदि हजार बार भी असफल हो जाओ, तो एक बार फिर प्रयत्न करो।\endnote{ २/१५६;} 

\vskip 2pt

दुर्बलता का उपचार सदैव उसका चिन्तन करते रहना नहीं है, वरन् बल का चिन्तन करना है।\endnote{ ८/११;} 

\vskip 2pt

हे मेरे युवक बन्धुओ! बलवान बनो - यही तुम्हारे लिए मेरा उपदेश है। गीतापाठ करने की अपेक्षा फुटबाल खेलने से तुम्हें ईश्वर कहीं अधिक सुलभ होगा। मैंने अत्यन्त साहसपूर्वक ये बातें कही हैं और इन्हें कहना अति आवश्यक है, क्योंकि मैं तुम लोगों को प्यार करता हूँ। मैं जानता हूँ कि कंकड़ कहाँ चुभता है। मैंने कुछ अनुभव प्राप्त किया है। बलवान शरीर या मजबूत पुट्ठों से तुम गीता को और अच्छी तरह समझोगे। शरीर में ताजा रक्त होने से तुम कृष्ण की महती प्रतिभा तथा महान् तेजस्विता को अच्छी तरह समझ सकोगे। जब तुम्हारा शरीर तुम्हारे पैरों के बल दृढ़ भाव से खड़ा होगा, जब तुम अपने को मनुष्य समझोगे, तब तुम उपनिषद् और आत्मा की महिमा भलीभाँति समझोगे।\endnote{ ५/३७;} 

\vskip 1.7pt

यह एक बड़ा सत्य है कि बल ही जीवन है और दुर्बलता ही मृत्यु है। बल ही अनन्त सुख है और अमर तथा शाश्वत जीवन है और दुर्बलता ही मृत्यु है।\endnote{ ९/१७७;} 

\vskip 1.7pt

सफल होने के लिए प्रबल अध्यवसाय चाहिए, मन का अमित बल चाहिए। अध्यवसायी साधक कहता है, “मैं चुल्लू से समुद्र पी जाऊँगा। मेरी इच्छा मात्र से पर्वत चूर-चूर हो जाएँगे।” इस प्रकार का तेज, इस प्रकार का दृढ़ संकल्प लेकर कठोर साधना करो और तुम ध्येय को अवश्य प्राप्त करोगे।\endnote{ १/९०;} 

\vskip 1.7pt

निराश मत होओ, मार्ग बड़ा कठिन है - छुरे की धार पर चलने के समान दुर्गम है, फिर भी निराश मत होओ - उठो, जागो और अपने चरम लक्ष्य को प्राप्त करो।\endnote{ २/७९;} 

\vskip 1.7pt

कोई कुछ भी कहे, अपने विश्वास में दृढ़ रहो - दुनिया तुम्हारे पैरों तले आ जाएगी, चिन्ता मत करो। लोग कहते हैं - “इस पर विश्वास करो, उस पर विश्वास करो,” पर मैं कहता हूँ - “पहले अपने आप पर विश्वास करो।” यही रास्ता है। \enginline{Have faith in yourself – all power is in you – be conscious and bring it out. } (अपने आप पर विश्वास करो - सब शक्ति तुम में है - इसे जान लो और प्रकट करो।)\endnote{ ३/३११-१२;} 

\vskip 1.7pt

प्रसन्न होओ और इस बात का विश्वास रखो कि प्रभु ने बड़े-बड़े कार्य करने के लिए हम लोगों को चुना है और हम उन्हें करके ही रहेंगे।\endnote{ २/३१२;} 

\vskip 1.7pt

जो हमारे पास नहीं है, जो शायद हमारे पूर्वजों के पास भी नहीं था - जो यवनों के पास था और जिसका स्पन्दन यूरोपीय डायनेमो से उस महाशक्ति को बड़े वेग से उत्पन्न कर रहा है, जिसका संचार समस्त भूमण्डल में हो रहा है - हम उसी को चाहते हैं। हम वही उद्यम, स्वाधीनता के लिए वही प्रेम, वही आत्मनिर्भरता, वही अटल धैर्य, वही कार्यकुशलता, वही एकता और वही उन्नति-पिपासा चाहते है। हम बीती हुई बातों की उधेड़-बुन छोड़कर अनन्त तक विस्तारित दूरदृष्टि चाहते हैं और आपाद-मस्तक नसनस में बहनेवाला रजोगुण चाहते हैं।\endnote{ १०/१३५;} 

\vskip 1.7pt

पूर्ण नीतिपरायण तथा साहसी बनो - प्राणों के लिए भी कभी न डरो। धार्मिक मत-मतान्तरों को लेकर व्यर्थ में माथापच्ची मत करो। कायर लोग ही पापाचरण करते हैं, वीर कभी भी पाप नहीं करते - यहाँ तक कि वे मन में भी कभी पाप के विचार नहीं लाते।\endnote{ १/३५१;} 

\vskip 1.7pt

सबसे पास जा-जाकर कहो, “उठो, जागो और सोओ मत। सारे अभाव और दुःख नष्ट करने की शक्ति तुम्हीं में है - इस बात पर विश्वास करने से ही वह शक्ति जाग उठेगी।”\endnote{ ६/१४;} 

\vskip 1.7pt

सिंह-गर्जन के साथ आत्मा की महिमा घोषित करो। जीव को अभय देकर कहो \textbf{- उत्तिष्ठत जाग्रत प्राप्य वरान् निबोधत। }\endnote{ ६/९८;} 

कहो कि जिन कष्टों को हम अभी झेल रहे हैं, वे हमारे ही किए हुए कर्मों के फल हैं। यदि यह मान लिया जाए, तो यह भी प्रमाणित हो जाता है कि वे फिर हमारे द्वारा नष्ट भी किए जा सकते हैं। जो कुछ हमने सृजन किया है, उसका हम ध्वंस भी कर सकते हैं और जो कुछ दूसरों ने किया हैं, उसका नाश हमसे कभी नहीं हो सकता। अतः उठो, साहसी बनो, वीर बनो। सारा उत्तरदायित्व अपने कन्धों पर लो और याद रखो कि तुम स्वयं अपने भाग्य के निर्माता हो। तुम जो कुछ बल या सहायता चाहो, सब तुम्हारे भीतर ही विद्यमान है।\endnote{ २/१२०;} 

आत्मविश्वास का आदर्श ही हमारी सबसे अधिक सहायता कर सकता है। यदि इस आत्म\-विश्वास का और भी व्यापक रूप से प्रचार होता और यह कार्यरूप में परिणत हो जाता, तो मेरा दृढ़ विश्वास है कि जगत् में जितने दुःख तथा बुराइयाँ हैं, उनका अधिकांश लुप्त हो जाता। मानव-जाति के समग्र इतिहास में सभी महान् स्त्री-पुरुषों में यदि कोई प्रेरणा सर्वाधिक सशक्त रही है, तो वह आत्मविश्वास ही है।\endnote{ ८/१२;} 

विश्वास, सहानुभूति - दृढ़ विश्वास और ज्वलन्त सहानुभूति चाहिए! जीवन तुच्छ है, मरण भी तुच्छ है; भूख तुच्छ है और जाड़ा भी तुच्छ है। प्रभु की जय हो! आगे कूच करो - प्रभु ही हमारे सेनानायक हैं। पीछे मत देखो। कौन गिरा, पीछे मत देखो - आगे बढ़ो, बढ़ते चलो! भाइयो, इसी तरह हम आगे बढ़ते जाएँगे, - एक गिरेगा, तो दूसरा वहाँ डट जाएगा।\endnote{ १/४०५-०६;} 

अनुकरण, कायर की तरह अनुकरण करके कोई उन्नति के पथ पर आगे नहीं बढ़ सकता। वह तो मनुष्य के अधःपतन का लक्षण है। जब मनुष्य स्वयं से घृणा करने लग जाता है, तब समझना चाहिए कि उस पर अन्तिम चोट बैठ चुकी है।... अतः भाइयो, आत्मविश्वासी बनो। अपने पूर्वजों के नाम से स्वयं को लज्जित नहीं, गौरवान्वित समझो। याद रहे, किसी का अनुकरण कदापि न करो। कदापि नहीं। जब तुम औरों के विचारों का अनुकरण करते हो, तो अपनी स्वाधीनता गँवा बैठते हो। यहाँ तक कि आध्यात्मिक विषय में भी यदि दूसरों के आज्ञाधीन होकर कार्य करोगे, तो अपनी सारी शक्ति, यहाँ तक कि विचार की शक्ति भी खो बैठोगे। अपने स्वयं के प्रयत्नों के द्वारा अपने अन्दर की शक्तियों का विकास करो। पर देखो, दूसरे का अनुकरण न करो। हाँ, दूसरों के पास जो कुछ अच्छाई हो, उसे अवश्य ग्रहण करो। हमें दूसरों से अवश्य सीखना होगा।... औरों से उत्तम बातें सीखकर उन्नत बनो। जो सीखना नहीं चाहता, वह तो पहले ही मर चुका है।\endnote{ ५/२७२-७३;} 

परिश्रम करो, अटल रहो और भगवान पर श्रद्धा रखो। काम शुरू कर दो।... ‘धर्म को बिना हानि पहुँचाये जनता की उन्नति’ - इसी को अपना आदर्श वाक्य बना लो। 

याद रखो कि राष्ट्र झोपड़ी में बसा हुआ है, परन्तु हाय! उन लोगों के लिए कभी किसी ने कुछ किया नहीं।... बिना उनकी स्वाभाविक आध्यात्मिक वृत्ति को नष्ट किए, क्या तुम उनका खोया हुआ व्यक्तित्व उन्हें वापस दिला सकते हो? क्या समता, स्वतंत्रता, कार्य-कुशलता तथा पुरुषार्थ में तुम पाश्चात्यों के भी गुरु बन सकते हो? क्या तुम उसी के साथ-साथ स्वाभाविक आध्यात्मिक अन्तःप्रेरणा तथा अध्यात्म-साधनाओं में एक निष्ठावान सनातनी हिन्दू हो सकते हो। यह हमें करना है और हम इसे अवश्य करेंगे। तुम सबने इसी के लिए जन्म लिया है। अपने आप पर विश्वास रखो। दृढ़ संकल्प महान् कार्यों की जननी हैं।\endnote{ २/३२१-२२;} 

पूर्ण निष्कपटता, पवित्रता, विशाल बुद्धि और सर्वविजयी इच्छाशक्ति। इन गुणों से सम्पन्न मुट्ठी भर आदमियों को यह काम करने दो और सारे संसार में क्रान्तिकारी परिवर्तन आ जाएगा।\endnote{ ४/२७५;} 

अपने आप पर श्रद्धा करना सीखो! इसी आत्मश्रद्धा के बल पर अपने पैरों पर खड़े हो जाओ और शक्तिशाली बनो। इस समय हमें इसी की आवश्यकता है।\endnote{ ५/८६;} 

मेरा दृढ़ विश्वास है और मैं तुम लोगों से भी यह बात अच्छी तरह समझ लेने को कहता हूँ कि जो व्यक्ति दिन-रात अपने को दीन-हीन या अयोग्य समझते हुए बैठा रहेगा, उसके द्वारा कुछ भी नहीं हो सकता। वास्तव में यदि वह दिन-रात स्वयं को दीनहीन तथा ‘कुछ नहीं’ समझता है, तो वह ‘कुछ नहीं’ ही बन जाता है। यदि तुम कहो - ‘मेरे अन्दर शक्ति है’, तो तुममें शक्ति जाग उठेगी। और यदि तुम सोचो कि ‘मैं कुछ नहीं हूँ’ - दिन-रात यही सोचा करो, तो तुम सचमुच ही ‘कुछ नहीं’ हो जाओगे। तुम्हें यह महान् तत्त्व सदा स्मरण रखना होगा। हम तो उसी सर्व-शक्तिमान परम पिता की सन्तान हैं, उसी अनन्त ब्रह्माग्नि की चिनगारियाँ हैं - भला हम ‘कुछ नहीं’ कैसे हो सकते हैं? हम सब कुछ हैं, हम सब कुछ कर सकते हैं और मनुष्य को सब कुछ करना ही होगा।\endnote{ ५/२६७;} 

यदि भौतिक दृष्टि से बड़े होना चाहो, तो विश्वास करो - तुम वैसे ही हो जाओगे। मैं एक छोटा-सा बुलबुला हो सकता हूँ, तुम पर्वताकार ऊँची तरंग हो सकते हो, परन्तु यह जान रखो कि हम दोनों के लिए पृष्ठभूमि अनन्त समुद्र ही है। अनन्त ब्रह्म हमारी समग्र शक्तियों तथा बल का भण्डार है और हम दोनों ही उससे अपनी इच्छा भर शक्तिसंग्रह कर सकते हैं। इसलिए अपने आप पर विश्वास करो।\endnote{ ५/३१६-१७;} 

\delimiter

\addtoendnotes{\protect\end{multicols}}

