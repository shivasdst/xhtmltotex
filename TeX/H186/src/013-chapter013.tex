
\chapter{कुछ मनीषियों की दृष्टि में स्वामी विवेकानन्द }

\indentsecionsintoc

\toendnotes{कुछ मनीषियों की दृष्टि में स्वामी विवेकानन्द}

\section*{लियो टाल्स्टाय}

\addsectiontoTOC{लियो टाल्स्टाय}

अलेक्जेंडर शिफमैन लिखते हैं - “मध्यकालीन भारतीय दार्शनिकों में शंकराचार्य का और आधुनिक दार्शनिकों में रामकृष्ण परमहंस तथा उनके शिष्य स्वामी विवेकानन्द का उन्होंने काफी गहराई से अध्ययन किया।... 

“अपने जीवन के अन्तिम वर्षों में टाल्स्टाय ने श्रीरामकृष्ण पर ज्यादा ध्यान नहीं दिया। प्राचीन उपदेशों के अपने संकलन के नये संस्करण के लिए उन्होंने उनके (श्रीरामकृष्ण) ग्रन्थों से केवल कुछ उक्तियों को चुनने का कार्य किया। इन दिनों वे विवेकानन्द के उपदेशों में काफी अधिक रुचि ले रहे थे।... 

“विवेकानन्द के दर्शन के साथ टाल्स्टाय का प्रथम परिचय १८९६ ई. के सितम्बर में हुआ था। तब उन्होंने पहली बार अपनी डायरी में लिखा कि उन्होंने अपने पास भेजा गया ‘भारतीय विद्या का एक मनमोहक ग्रन्थ’ पढ़ा है। (टाल्स्टाय ग्रन्थावली, खण्ड ५३, पृ. १०६)। यह १८९५-९६ के सर्दियों में न्यूयार्क में विवेकानन्द द्वारा भारतीय दर्शन पर प्रदत्त व्याख्यान-माला थी। श्री अनेन्द्र कुमार दत्त नामक एक भारतीय विद्वान् ने यह ग्रन्थ टाल्स्टाय को भेजते हुए लिखा था - 

“आपको यह जानकर प्रसन्नता होगी कि हमें प्राप्त होनेवाली प्राचीनतम भारतीय दर्शन की चरम उपलब्धियों के साथ आपके सिद्धान्तों का पूर्ण सामंजस्य है।” (यह पत्र प्रकाशित नहीं हुआ। यह टाल्स्टाय के अभिलेखागार में रखा हुआ है। उस ग्रन्थ का नाम था - राजयोग। १९११ में इस ग्रन्थ का रूसी भाषा में अनुवाद हुआ।) 

इस पत्र के उत्तर में टाल्स्टाय ने लिखा कि उन्होंने ग्रन्थ को पसन्द किया है और उसमें निरूपित आत्मा के वास्तविक अस्तित्व के पक्ष में दी हुई युक्तियों का समर्थन किया। (टाल्स्टाय ग्रन्थावली, खण्ड ६९, पृ. १४६)। 

“पूँजीवादी भौतिक सभ्यता के खिलाफ विवेकानन्द के आवेगपूर्ण निन्दावाद में और मनुष्य के ‘भौतिक आवरण’ की तुलना में उसके आध्यात्मिक सार-भाग को दी गयी प्राथमिकता के समर्थन में टाल्स्टाय को प्राचीन भारतवासियों की प्रारम्भिक उपदेशों की प्रतिध्वनि सुनाई दी, विशेषकर वेदों के ऐसे अनेक भाव दृष्टिगोचर हुए, जो उन्हें स्वीकार्य थे। 

“टाल्स्टाय ने स्वामीजी की जो दूसरी पुस्तक पढ़ी, वह उनके व्याख्यानों तथा लेखों का (अंग्रेजी भाषा में) एक संकलन था, जो १९०७ ई. में उनके परिचित आइ. एफ. नाझीविन ने भेजा था। नाझीविन ने उनसे पूछा था कि क्या वे उस पुस्तक को रखना चाहेंगे, जिसका उत्तर देते हुए टाल्स्टाय ने ७ जुलाई १९०७ को लिखा - “उस ब्राह्मण (हिन्दू) की लिखी पुस्तक मुझे भेजें। ऐसी पुस्तकें पढ़ने में परम आनन्द का बोध होता है, अन्तरात्मा का विस्तार होता है।” (टाल्स्टाय ग्रन्थावली, खण्ड ७७, पृ. १५१)। 

“१९०८ ई. में नाझीविन ने \enginline{Voices of the People }ए (जन-वाणी) नामक ग्रन्थ प्रकाशित किया, जिसमें स्वामीजी का ‘लोक-स्तुति’ तथा ‘ईश्वर और मानव’ लेख भी संकलित हुआ था। परवर्ती लेख ने टाल्स्टाय को काफी प्रभावित किया। इसे पढ़ने के बाद उन्होंने नाझीविन को लिखा, “यह असाधारण रूप से अच्छा है।’ (टाल्स्टाय ग्रन्थावली, खण्ड ७८, पृ. ८४)। 

“एक बार टाल्स्टाय ने शॉपेनहावर के सन्दर्भ में ईश्वर-विषयक विवेकानन्द के उत्कृष्ट तर्कों के लिए उनकी प्रशंसा की और इस भारतीय दार्शनिक की अंग्रेजी के विषय में कहा, “विवेकानन्द की अंग्रेजी कितनी अद्भुत है! उन्होंने इसकी सारी बारीकियाँ सीख ली हैं।” (डी. पी. मैकोवित्सकी, यस्नया पोल्यना नोट्स, ३ जुलाई १९०८)। 

“१९०९ ई. के मार्च में टाल्स्टाय जनता के लिए नवीन लोकप्रिय पुस्तकों की एक तालिका बना रहे थे। इसके प्रकाशन में उन्होंने श्रीरामकृष्ण तथा विवेकानन्द की उक्तियों को भी शामिल करने के योजना बनायी (टाल्स्टाय ग्रन्थावली, खण्ड ५७, पृ. ४०); और उसी वर्ष के अप्रैल में उन्होंने प्राच्यविद् एल. ओ. आइनहार्न को सूचित किया, ‘हम लोग विवेकानन्द के चुनिंदा विचारों को प्रकाशन हेतु प्रस्तुत कर रहे हैं; मैं उनका बड़ा प्रशंसक हूँ।’ (टाल्स्टाय ग्रन्थावली, खण्ड ७९, पृ. १४२)। परन्तु यह ग्रन्थ प्रकाशित नहीं हो सका था।\endnote{ \engfoot{\textit{Tolstoy and India}, Sahitya Akademi, New Delhi, 1969, pp.\ 25--39}}


\section*{रवीन्द्रनाथ ठाकुर}

\addsectiontoTOC{रवीन्द्रनाथ ठाकुर}

यदि आप भारत को समझना चाहते हैं, तो विवेकानन्द का अध्ययन कीजिए, उनमें सब कुछ विधेयात्मक है, निषेधात्मक कुछ भी नहीं।\endnote{ रवीन्द्रनाथ ने रोमाँ रोलाँ से कहा था। रोमाँ रोलाँ ने अपने एक पत्र के द्वारा स्वामी अशोकानन्द को बताया। हमें स्वामी अशोकानन्द से प्राप्त हुआ।} 

कुछ काल पूर्व विवेकानन्द ने कहा था कि प्रत्येक मनुष्य में ब्रह्म की शक्ति विद्यमान है। कहा था कि निर्धन के माध्यम से नारायण हमारी सेवा पाना चाहते हैं। इसी को मैं सच्चा सन्देश कहता हूँ। इस सन्देश ने व्यक्ति को उसकी स्वार्थपरता की सीमा से बाहर निकालकर, आत्मबोध को असीम मुक्ति का मार्ग दिखाया। यह किसी विशेष आचार-व्यवहार का उपदेश नहीं था; और न व्यक्ति के बाह्य जीवन पर थोपा गया कोई संकीर्ण अनुशासन ही था। छूतमार्ग का विरोध इसमें स्वतः ही आ गया है। इस कारण नहीं कि इसके द्वारा राष्ट्रीय स्वाधीनता मिलने में सहायता हो सकती है, बल्कि इस कारण कि इसके द्वारा मनुष्य का अपमान दूर किया जा सकेगा। उस अपमान में हममें से प्रत्येक की आत्मा का अपमान निहित है। 

विवेकानन्द का यह सन्देश सम्पूर्ण मानव-जाति को एक आह्वान है, इसीलिए यह कर्म तथा त्याग के द्वारा हमारे युवकों को मुक्ति के विभिन्न मार्गों पर चलने को प्रेरित कर रहा है।\endnote{ विश्वभारती के सौजन्य से} 

आधुनिक भारत में एकमात्र विवेकानन्द ने ही एक ऐसे महान् सन्देश का प्रचार किया जो कि किन्हीं विधि-निषेधों में ही आबद्ध नहीं हैं। सम्पूर्ण राष्ट्र का आह्वान करते हुए उन्होंने कहा था, ‘तुम सबके भीतर ब्रह्म की शक्ति विद्यमान है - निर्धनों में विराजमान ईश्वर तुम्हारी सेवा माँगते हैं। इस सन्देश ने नवयुवकों के चित्त को पूर्ण रूप से जाग्रत कर दिया है। इसीलिए यह सन्देश देशभक्ति के विभिन्न रूपों में तथा विविध प्रकार के त्यागों में फलप्रसू हुआ। उनके सन्देश ने मानव को श्रद्धा एवं सम्मान के साथ ही शक्ति एवं उत्साह भी प्रदान किया है।... इस सन्देश ने मानव-मात्र को जो सन्देश दिया है, वह किसी एक बिन्दु में आबद्ध नहीं है; और न ही यह किसी शारीरिक गतियों की पुनरावृत्ति तक सीमित है। इसने सचमुच ही उनके जीवन को विविध क्षेत्रों में एक अद्भुत कर्मठता से समृद्ध किया है। आज के भारतीय युवकों के साहसपूर्ण गतिविधियों के प्रेरणा-स्रोत के रूप में स्थित है विवेकानन्द का सन्देश - जो मनुष्य की उंगलियों का नहीं, अपितु उसकी अन्तरात्मा का आह्वान करता है।\endnote{ प्रबासी, ज्येष्ठ १३३५ (बंगाब्द), पृ. २८५-८६}


\section*{श्री अरविन्द}

\addsectiontoTOC{श्री अरविन्द}

भारत की अन्तरात्मा सर्वप्रथम धर्म के क्षेत्र में जागी और विजय प्राप्त किया। इसके पहले से ही संकेत मिल रहे थे, पहले से ही कई महापुरुष आ चुके थे; परन्तु जब कोलकाता के श्रेष्ठ शिक्षित युवक उन अशिक्षित हिन्दू तपस्वी, आत्मदर्शी समाधिवान महात्मा (श्रीरामकृष्ण) के चरणों में जाकर अवनत हुए, जिनमें विदेशी भावों या शिक्षा का लेश या कण मात्र भी न था, तभी युद्ध जीता जा चुका था। गुरु द्वारा निर्दिष्ट होकर, एक वीर पुरुष की भाँति, पूरे जगत् को अपने दोनों हाथों में लेकर, बदल डालने के सुनिश्चित उद्देश्य के साथ विवेकानन्द का बाहर जाना, विश्व के समक्ष पहला स्पष्ट संकेत था कि भारत न केवल आत्मरक्षा के लिए, अपितु विजय करने को जागा है।... राष्ट्र की अन्तरात्मा एक बार जब धर्म के क्षेत्र में जाग्रत हो गयी, तब तो केवल समय तथा अवसर की ही अपेक्षा थी, जब वह राष्ट्रीय अस्तित्व के सभी तरह ही आध्यात्मिक तथा बौद्धिक क्रियाशीलता में प्रविष्ट हो जाती और उन्हें अपने अधिकार में कर लेती।\endnote{ \engfoot{\textit{Sri Aurobindo}, Vol. 2, 1972, p. 37}} 

यदि दुनिया में कोई सामर्थ्यवान पुरुष हुआ है, तो वे थे स्वामी विवेकानन्द। मनुष्यों के बीच वे एक साक्षात् सिंह थे। उनकी सर्जनात्मक प्रतिभा और क्षमता के बारे में हमारे मन में जो धारणा है, उसकी तुलना में, अपने पीछे वे जो एक सुनिश्चित कार्य छोड़ गए हैं, वह बिल्कुल अपर्याप्त है। उनके प्रभाव को हम आज भी व्यापक स्तर पर क्रियाशील अनुभव करते हैं। हमें ठीक मालूम नहीं कि किस प्रकार और कहाँ, किसी ऐसी चीज में जिसे अभी तक पूरा आकार नहीं मिला है, कुछ सिंह-सदृश, महान् अन्तःप्रेरक उन्नायक वस्तु भारत की अन्तरात्मा में प्रविष्ट हो गयी हैं। और हम कहते हैं - वह देखो! विवेकानन्द अब भी अपनी माता और उनकी सन्तानों की आत्मा में जीवित हैं।\endnote{ \engfoot{\textit{Ibid.,} Vol. 17, 1971, p. 332}} 

\newpage

स्वामी विवेकानन्द का अमेरिका जाना और बाद में उनका अनुसरण करनेवालों के द्वारा भारत के हित में जो कार्य हुआ, वह सौ लन्दन-कांग्रेसों द्वारा हो सकनेवाले कार्य से भी कहीं अधिक था। सहानुभूति जगाने का यही सच्चा उपाय था - अपने को एक ऐसे राष्ट्र के रूप में प्रस्तुत करना, जिसके पास उसका अपना एक महान् इतिहास तथा प्राचीन सभ्यता है, जिनमें अब भी उनके पूर्वजों की प्रतिभा तथा गुण शेष हैं, और जिनके पास अब भी विश्व को देने के लिए कुछ है, इसलिए वे स्वाधीनता के हकदार हैं - इस प्रकार अपना पौरुष तथा योग्यता दिखाकर, न कि भिक्षावृत्ति के द्वारा।\endnote{ \engfoot{\textit{Ibid.,} Vol. 2, p. 171}}


\section*{ब्राह्मबान्धव उपाध्याय}

\addsectiontoTOC{ब्राह्मबान्धव उपाध्याय}

कुछ दिनों के लिए भ्रमण हेतु मैं बोलपुर आश्रम गया था। लौटते समय मैं ज्योंही हावड़ा स्टेशन पर पहुँचा, त्योंही किसी ने कहा - कल विवेकानन्द ने अपनी नर-लीला का संवरण कर लिया। सच कहता हूँ, सुनते ही मेरे सीने में छुरी भिंद गयी। वेदना का ज्वार थोड़ा कम होने पर मैंने सोचा - विवेकानन्द का काम अब कैसे चलेगा? फिर मन में आया - क्यों? उनके अनेक सुयोग्य गुरुभाई हैं, वे लोग चलाएँगे। तो भी प्रेरणा हुई - मुझमें जो थोड़ी-बहुत शक्ति है, उसे काम में लगाऊँ, विवेकानन्द के विश्व-विजय का कार्य आगे बढ़ाने का प्रयास करूँ। उसी क्षण मैंने यूरोप जाने का निश्चय कर लिया। मैंने सपने में भी नहीं सोचा था कि कभी यूरोप देखना होगा। पर उस हावड़ा स्टेशन पर ही मैंने संकल्प कर लिया कि मैं यूरोप जाकर वहाँ वेदान्त का प्रचार करूँगा। तब मेरी समझ में आया कि विवेकानन्द कौन हैं! जिनकी प्रेरणा-शक्ति मुझ जैसे क्षुद्र व्यक्ति को समुद्र-पार ले जा सकती है, वे कोई साधारण मनुष्य नहीं हो सकते। इसके कुछ ही दिनों बाद मैं केवल सत्ताइस रुपये लेकर कोलकाता नगरी से विदा हुआ और यूरोप की यात्रा पर चल पड़ा। अन्ततः यूरोप पहुँचकर मैंने ऑक्सफोर्ड और कैम्ब्रिज के विश्वविद्यालयों में वेदान्त पर व्याख्यान दिए। वहाँ बड़े-बड़े (अंग्रेज) प्राध्यापकों ने मेरे भाषण सुने और मुझे वचन दिया कि वे अपने यहाँ हिन्दू प्राध्यापकों को नियुक्त कर वेदान्त-शास्त्र के शिक्षण की व्यवस्था करेंगे। उन प्राध्यापकों ने मुझे जो पत्र लिखे, उन्हें मैंने प्रकाशित नहीं कराया है। उनके प्रकाशित होने पर पता चलेगा कि वहाँ वेदान्त का प्रचार कितना प्रभावी प्रचार हुआ था। मैं एक साधारण-सा आदमी हूँ। मेरे द्वारा जो यह इतना बड़ा कार्य हो गया, वह मुझे एक स्वप्न जैसा प्रतीत होता है। यह सब विवेकानन्द की प्रेरणा-शक्ति से ही सका है, मानो असम्भव सम्भव हो गया। इसीलिए मैं कभी-कभी सोचता हूँ विवेकानन्द कौन हैं? विवेकानन्द जिस महान् कार्य का विस्तार कर गए हैं, उस पर विचार करने से उनके महत्त्व का ओर-छोर नहीं मिलता। 

एक बार कोलकाता के हेदुआ तालाब के किनारे मेरी विवेकानन्द से भेंट हुई थी। मैंने कहा, “भाई! चुपचाप क्यों बैठे हो? चलो, एक बार कोलकाता शहर में वेदान्त का शोर मचाएँ। मैं सब व्यवस्था कर दूँगा, तुम बस एक बार मंच पर आ जाना।” विवेकानन्द ने कातर स्वर में कहा, “भवानी भाई! अब मैं बचूँगा नहीं (यह उनके देहावसान के छह महीने पूर्व की बात है)। मैं इस समय अपने मठ के निर्माण में लगा हूँ और चाहता हूँ कि इसकी सुव्यवस्था करके जाऊँ। मैं इसी कार्य में व्यस्त हूँ, मेरे पास समय नहीं है।” उसी दिन उनका करुणा-विगलित चित्त देखकर मैं समझ गया कि इस व्यक्ति का हृदय वेदनामय-व्यथा से परिपूर्ण है। पर वेदना किसके लिए, व्यथा किसके लिए? - देश के लिए। आर्य ज्ञान, आर्य सभ्यता - सब नष्ट-भ्रष्ट तथा विध्वंश होता जा रहा है; अन्य अनार्य तत्त्व उस सूक्ष्म, उदार, आर्य तत्त्व को पराभूत किए जा रहे हैं, और तुम हो कि जागते ही नहीं, व्यथा भी नहीं होती तुम्हें। विवेकानन्द के हृदय में इसकी वेदनामय प्रतिक्रिया हुई थी। वह व्यथा, वह प्रतिक्रिया इतनी गम्भीर थी कि उसने अमेरिका और यूरोप में चेतना जगा दी। उसी व्यथा की बात सोचता हूँ, उसी वेदना पर चिन्तन करता हूँ और स्वयं से पूछता हूँ विवेकानन्द कौन थे? देश के लिए उनकी व्यथा क्या कभी मूर्तिमान हो सकती है? यदि हो सकती हो, तभी विवेकानन्द को समझा जा सकता है।\endnote{ स्वराज, २२ बैशाख १३१४ (बंगाब्द), पृ. ९९} 

स्वामीजी! मैं तुम्हारे यौवन का मित्र हूँ। तुम्हारे साथ मैंने कितना आमोद-प्रमोद किया है, वनभोज किया है, बातचीत की है। तब मैं भला कहाँ जानता था, कि तुम्हारे हृदय में भारत के लिए पीड़ा का ज्वालामुखी सुलग रहा है! आज मैं भी अपनी क्षुद्र शक्ति के साथ तुम्हारे ही व्रत को पूरा करने में लगा हूँ।... इस घोर संग्राम में जब मैं क्षत-विक्षत होकर गिर पड़ता हूँ, अवसाद आकर हृदय को आच्छन्न कर लेता है, तब मैं तुम्हारे द्वारा प्रदर्शित आदर्श की ओर देखता हूँ, तुम्हारे सिंह-विक्रम की बात सोचता हूँ और तुम्हारी गहन वेदना का स्मरण करता हूँ। इसके साथ ही मेरा सारा विषाद चला जाता है और न जाने कहाँ से एक दिव्य आलोक और एक दिव्य-शक्ति आकर मेरे मन-प्राण को परिपूर्ण कर देती है।\endnote{ विवेकानन्द ओ समकालीन भारतवर्ष (बंगला), खण्ड १, मण्डल बुक हाउस, कोलकाता, १९८२, पृ. ३५१}


\section*{लोकमान्य तिलक}

\addsectiontoTOC{लोकमान्य तिलक}

शायद ही ऐसा कोई हिन्दू होगा, जो विवेकानन्द के नाम से परिचित न हो। उन्नीसवीं शताब्दी में भौतिक-विज्ञान ने असाधारण प्रगति करके (विश्व में) अपने लिए उच्च स्थान बना लिया था। ऐसी शताब्दी के उत्तरार्ध में, हिन्दुस्तान में सहस्त्रों वर्ष पूर्व से प्रचलित अध्यात्म-शास्त्र को पाश्चात्य विद्वानों के समक्ष समझाकर प्रस्तुत करना तथा उनसे उसकी अपूर्वता की बात मनवाना; और जिस राष्ट्र में उन शास्त्रों का सृजन हुआ, उसके लोगों के बारे में सहानुभूति उत्पन्न कराना - यह कोई मामूली काम नहीं है। अंग्रेजी शिक्षा के साथ ही साथ पश्चिम की भौतिकता का प्रवाह भी भारत में इतनी तेजी के साथ बहा चला आ रहा था कि उसे वापस लौटाने के लिए एक असाधारण धैर्यशाली और बुद्धिमान पुरूष के आविर्भाव की आवश्यकता थी। स्वामी विवेकानन्द के पहले यह कार्य थियोसॉफिकल सोसायटी ने आरम्भ किया था, परन्तु इसमें कोई दो राय नहीं है कि उस दिशा में सच्चे हिन्दुत्व का आनयन सर्वप्रथम स्वामीजी ने ही किया।... विश्व के समस्त देशों में अद्वैतवाद की पताका फहराने का और सम्पूर्ण जग में हिन्दू धर्म अर्थात् हिन्दुओं का गौरव बढ़ाते हुए धर्म की स्थापना का कार्य स्वामी विवेकानन्द ने अपने हाथ में ले लिया था। उन्होंने अपनी विद्वत्ता, वाग्मिता, उत्साह तथा आत्मविश्वास के द्वारा दृढ़ भूमि पर इस कार्य की आधारशिला रखी। हिन्दू धर्म का उज्ज्वल स्वरूप कौन-सा है? हमारे देश में विकसित हुआ हमारा धर्म ही, हमारा अमोल धन और बल है; जग भर में इसका प्रचार करना ही हमारा सच्चा कर्तव्य है - ऐसा केवल मुख से न कहकर, सारी दुनिया के समक्ष सिद्ध कर दिखाने वाले सत्पुरुष, एक तो हजार-बारह सौ वर्ष पूर्व शंकराचार्य हुए थे और दूसरे उन्नीसवीं शताब्दी के अन्त में स्वामी विवेकानन्द हुए।\endnote{ केसरी, ८ जुलाई १९०२: (मराठी से अनूदित)}


\section*{बिपिन चन्द्र पाल}

\addsectiontoTOC{बिपिन चन्द्र पाल}

\centerline{\textbf{एक}}

स्वामी विवेकानन्द अकेले नहीं हैं। अपने गुरुदेव परमहंस श्रीरामकृष्ण के साथ वे अविच्छेद्य रूप से जुड़े हुए हैं। केवल भारतवर्ष के ही नहीं, बल्कि बृहत्तर विश्व के भी आधुनिक मानव के लिए, ये दोनों मानो संयुक्त रूप धारण किए हुए हैं। आधुनिक मनुष्य केवल विवेकानन्द के माध्यम से ही परमहंसदेव को समझ सकता है; इसी प्रकार विवेकानन्द को भी उनके गुरुदेव के जीवनालोक में ही समझा जा सकता है। श्रीरामकृष्ण एक महान् आध्यात्मिक शक्ति थे। इसी कारण वे उस पीढ़ी के लिए एक सुनिश्चित रहस्य बने रहे, जो ‘बुद्धिवाद’ के नारों को ही समझती थी और उसी को लेकर अभिभूत थी। बुद्धिवाद का अर्थ है - भावुकता का अभाव, और यह भावुकता ही आध्यात्मिक जीवन का प्राण है। भावुकता कल्पना मात्र नहीं है। वस्तुतः यह इन्द्रियों तथा बुद्धि के स्तर के भी परे स्थित तत्त्व की कल्पना नहीं, अपितु अनुभव करने की शक्ति है। श्रीरामकृष्ण जिस पीढ़ी में आए, उसमें इस भाव-प्रवणता का अभाव था। इस कारण वे उन लोगों के लिए एक रहस्य बने रहे। अतः श्रीरामकृष्ण परमहंस की अन्तरात्मा तथा जीवन-सन्देश को इस पीढ़ी के लिए बोधगम्य भाषा में व्याख्या तथा प्रस्तुत करने का कार्य विवेकानन्द के ऊपर ही न्यस्त था। 

रामकृष्ण परमहंस किसी मतवाद या सम्प्रदाय की सीमा में नहीं आते थे; या फिर इसे दूसरे शब्दों में कहें, तो वे सभी भारतीय तथा अभारतीय धर्म-सम्प्रदायों तथा मतवादों का प्रतिनिधित्व करते थे। वे एक सच्चे सार्वभौमिक व्यक्ति थे, परन्तु उनकी सार्वभौमिकता कल्पनामूलक नहीं थी। उन्होंने सार्वभौमिक धर्म की स्थापना के लिए विभिन्न धर्मों की विशिष्टताओं का परित्याग नहीं किया। उनकी दृष्टि में तो सार्वभौमिकता तथा विशिष्टता - सूर्य तथा उसकी छाया के समान सर्वदा आपस में अभिन्न रूप से जुड़े हुए थे। इस कारण उन्होंने जीवन तथा चिन्तन के असंख्य रूपों के भीतर सार्वभौमिकता के सत्य का अनुभव किया। विवेकानन्द ने अपने गुरुदेव की इस अनुभूति को आधुनिक मानवतावाद की भाषा में प्रस्तुत किया। 

\newpage

रामकृष्ण परमहंस के ईश्वर - तर्कशास्त्र या दर्शन के ईश्वर न थे, अपितु वे उनके प्रत्यक्ष, व्यक्तिगत तथा आन्तरिक अनुभूति के विषय थे। रामकृष्ण ने किन्हीं प्राचीन शास्त्रों या परम्पराओं, अथवा किसी गुरु के प्रमाण के आधार पर नहीं, अपितु अपने स्वयं के प्रत्यक्ष, व्यक्तिगत अनुभूतियों के आधार पर ईश्वर के अस्तित्व को स्वीकार किया। वे एक वेदान्ती थे, क्योंकि प्रारम्भ में उनकी निष्ठा तथा साधनाएँ शक्ति-सम्प्रदाय से जुड़ी थीं। बंगाल का शक्ति-सम्प्रदाय वेदान्त के आधार पर ही खड़ा है। परन्तु रामकृष्ण परमहंस के वेदान्त को शांकर-वेदान्त नहीं कहा जा सकता, और न ही इसे वैष्णव-वेदान्त के विभिन्न सम्प्रदायों के अन्तर्गत रखा जा सकता है। ये नामकरण उन लोगों के लिए हैं, जिन्होंने महान् चिन्तकों के विचारों के आधार पर अपने धर्ममत का निर्माण किया है। परन्तु रामकृष्ण परमहंस इस श्रेणी के नहीं थे। वे न तो दार्शनिक थे, न कोई महान् विद्वान् थे, न वे प्राचीन या नव्य न्यायशास्र के पण्डित थे; बल्कि वे तो सहज भाव से युक्त ऋषि थे। उन्होंने जो कुछ देखा, उसी में विश्वास किया। 

ऋषि सर्वदा ही अनुभूति-सम्पन्न होता है - ईसा भी वैसे ही थे और मानव-जाति के सभी महान् धर्मगुरु वैसे ही थे। लोगों की भीड़ उन्हें नहीं समझ सकती; और उनके युग के विद्वान् तथा दार्शनिक तो उन्हें बिल्कुल भी नहीं समझ सकते। तथापि वे उस तत्त्व को अभिव्यक्त करते हैं, जिसे सारे दर्शन-शास्त्र टटोलकर जानने की चेष्टा कर रहे हैं। ईसा के समान ही रामकृष्ण परमहंस को भी एक ऐसे प्रचारक की जरूरत थी, जो उनके जीवन की व्याख्या करता तथा उनका सन्देश उनके युग को समझा देता। ईसा को सेंट पॉल के रूप में और रामकृष्ण को विवेकानन्द के रूप में ऐसे व्याख्याकार प्राप्त हुए। इसीलिए विवेकानन्द को रामकृष्ण परमहंस की अनुभूतियों के आलोक में ही समझना होगा। 

\centerline{\textbf{दो }}

विवेकानन्द के रूपान्तरण की कथा अब तक बतायी नहीं गयी है। शायद ही कोई जानता है कि यह चमत्कार कैसे घटित हुआ। विवेकानन्द एक बुद्धिवादी थे, यद्यपि वे सोचते थे कि वे आस्तिक हैं। प्रारम्भ में उनके धार्मिक विचार ब्रह्मसमाज से प्रभावित थे, परन्तु ये विचार सन्त-महात्माओं के प्रति श्रद्धा के विकास के लिए विशेष सहायक नहीं थे। तथापि ब्रह्मसमाज के अनेक सदस्य रामकृष्ण परमहंस की महान् आध्यात्मिक शक्तियों, और विशेषकर उनके उत्कट ईश्वर-प्रेम को देखकर उनकी ओर आकृष्ट हुए। परन्तु वे लोग परमहंसदेव के जीवन तथा अनुभूतियों के रहस्यमय प्रवाहों को खोल नहीं सके। वस्तुतः परमहंसदेव ने उनमें से अधिकांश लोगों के समक्ष अपने धर्मभाव के गूढ़ रहस्यों को प्रकट नहीं किया। इस मामले में विवेकानन्द उनके कृपापात्र हुए। 

परमहंस रामकृष्ण ने विवेकानन्द के भाव तथा स्वभाव की आन्तरिक संरचना को देख लिया और उन्हें अपने जीवन-सन्देश के प्रचार हेतु एक उपयुक्त यंत्र के रूप में पहचान लिया। यही स्वामी विवेकानन्द के रूपान्तरण की सच्ची कहानी है।... स्वामी विवेकानन्द ने अपने गुरुदेव के प्रति एक अज्ञात आकर्षण का अनुभव किया। यह अन्तरात्मा की शक्ति का कार्य था। आध्यात्मिकता की अतल गहराइयों में, जब एक आत्मा दूसरी आत्मा का स्पर्श करती है, तो दोनों सदा के लिए अटूट आध्यात्मिक बन्धनों से जुड़ जाती हैं। व्यावहारिक रूप से दोनों मिलकर एक हो जाते हैं; तब गुरु ही शिष्य के माध्यम से अपना कार्य करता है और शिष्य को इसका भान तक नहीं होता कि वह अपने गुरुदेव के द्वारा ही परिचालित हो रहा है। लोग इसे प्रेरणा कहते हैं। स्वामी विवेकानन्द ने अपने रूपान्तरण के बाद, अपने गुरुदेव की प्रेरणा से कार्य सम्पन्न किया। 

\centerline{\textbf{तीन}}

स्वामी विवेकानन्द का सन्देश, यद्यपि सर्वविदित वेदान्त की शब्दावली में दिया गया था, तथापि वह वस्तुतः आधुनिक मानवता को उनके गुरुदेव का सन्देश था। स्वामीजी का सन्देश वस्तुतः आधुनिक मानवता का सन्देश था। उन्होंने अपने देशवासियों का आह्वान करते हुए कहा - “मनुष्य बनो।” भारत में धार्मिक व्यक्ति एक मध्ययुगीन व्यक्ति हुआ करता था। उसका धर्म सामान्यतः परलोक से सम्बन्धित होता था। यह एक ऐसा धर्म था, जो जगत् का और समस्त जागतिक तथा सामाजिक उत्तरदायित्वों के त्याग का आदेश देता था। परन्तु श्रीरामकृष्ण परमहंस का सन्देश ऐसा नहीं था। वे जितने वेदान्ती थे, उसने ही वैष्णव भी थे। हिन्दू धर्म के दो परस्पर-विरोधी मतवादों के बीच समन्वय ही उनके लिए धार्मिकता का तात्पर्य था। उनका मातृ-सम्प्रदाय वस्तुतः ‘भक्ति’ का ही सम्प्रदाय था, जिसमें भगवत्-प्रेम को मानवीय मातृत्व के रूप में अनुभव किया जाता था। बंगाल के वैष्णवों के समान ही परमहंसदेव के लिए भी परम तत्त्व जगत् से भिन्न नहीं था। यह भौतिक नहीं था, परन्तु इस कारण वह निराकार नहीं था। परम तत्त्व का सच्चा रूप मानवीय रूप है - परन्तु वह हमारी आँखों से दिखनेवाला इन्द्रिय-गोचर रूप नहीं है, बल्कि उसके पीछे स्थित रहनेवाला आध्यात्मिक रूप है, जो मानवीय आँखों से अतीत होता है। मनुष्य तथा ईश्वर मूलतः एक हैं। 

सभी धार्मिक साधनाओं का उद्देश्य है - मनुष्य को उसकी मूलभूत दिव्यता की अनुभूति करने में सहायता करना। जब स्वामीजी ने अपने देशवासियों से मनुष्य बनने का आह्वान किया, तो वस्तुतः उनका यही तात्पर्य था। ब्राह्मण के संध्या-वन्दन में एक पंक्ति आती है, जिसका अर्थ है - “मैं ब्रह्म हूँ। मैं अन्य कुछ भी नहीं हूँ। दुःख तथा शोक मेरा स्पर्श नहीं कर सकते। मैं सत्य-स्वरूप, आत्म-चैतन्य तथा चिरंजीवी हूँ। मैं स्वरूपतः चिर-मुक्त हूँ।” स्वामी विवेकानन्द द्वारा उद्घोषित आधुनिक जगत् के प्रति उनके गुरुदेव का सन्देश वस्तुतः यही था। 

यह स्वाधीनता का सन्देश नकारात्मक नहीं है, अपितु सकारात्मक तथा परम सर्वांगीण तात्पर्यों से युक्त है। स्वाधीनता का अर्थ है - सारे बाह्य बन्धनों का निवारण। परन्तु हमारा ऐसा निर्माण हुआ है कि हम अपने बाह्य सम्बन्धों से - प्राकृतिक या सामाजिक परिवेश से स्वयं को अलग नहीं कर सकते। ऐसा अलगाव भौतिक तथा आध्यात्मिक, दोनों ही दृष्टियों से घातक होता है। अतः जीवन का नियम अलगाव नहीं, बल्कि मेलजोल है; असहयोग नहीं, बल्कि सहयोग है। सच्ची स्वाधीनता युद्ध के द्वारा नहीं, बल्कि शान्ति के द्वारा ही पायी जा सकती है। इसमें कोई सन्देह नहीं कि जीवन-योजना में युद्ध, त्याग तथा अलगाव का भी स्थान है, परन्तु केवल अस्थायी रूप से ही है और परम लक्ष्य की प्राप्ति के लिए एक साधन के रूप में है, जिसमें अस्तित्व के लिए अपरिहार्य संघर्ष चिर काल तक नहीं चलता, बल्कि एक अन्तरंग तथा स्थायी सम्मिलन में इन सारे संघर्षों का समाधान तथा निराकरण हो जाता है। स्वाधीनता एक है। अपनी कामनाओं तथा वासनाओं के चंगुल से मुक्ति पाना ही लक्ष्य की प्राप्ति की ओर पहला कदम है। अगला कदम है - अपने मानव-भाइयों के भय से मुक्ति। इसके बाद आता है किसी बाह्य सत्ता के अधिकार से मुक्ति। इस प्रकार मनुष्य को अपनी व्यक्तिगत स्वाधीनता से आरम्भ करके, सामाजिक तथा राजनीतिक स्वाधीनताओं से होकर गुजरते हुए, अपनी वास्तविक स्वाधीनता प्राप्त कर लेनी चाहिए। जब उसे इसकी उपलब्धि हो जाती है, तब अन्ततः उसे बोध होता है कि वह तथा ईश्वर अभिन्न हैं। स्वामीजी द्वारा प्रतिपादित वेदान्त का यही सन्देश है। यही वस्तुतः आधुनिक जगत् को उनके गुरुदेव का सन्देश है।\endnote{ प्रबुद्ध भारत (अंग्रेजी मासिक), जुलाई १९३२, पृ. ३२३-२५} 

भारत के कुछ लोग सोचते हैं कि इंग्लैंड में स्वामी विवेकानन्द द्वारा प्रदत्त व्याख्यानों का कोई विशेष फल नहीं हुआ और उनके मित्रों तथा अनुयाइयों ने उनके कार्य को बढ़ाचढ़ाकर प्रस्तुत किया है। परन्तु यहाँ आकर मैंने देखा कि उन्होंने यहाँ पर सर्वत्र एक उल्लेखनीय प्रभाव डाला है। इंग्लैंड के अनेक अंचलों में मेरी ऐसे बहुत-से लोगों से मुलाकात हुई है, जो स्वामी विवेकानन्द के प्रति अत्यन्त श्रद्धा तथा सम्मान का भाव रखते हैं। यद्यपि मैं उनके सम्प्रदाय का नहीं हूँ और यह सत्य है कि मेरा उनसे काफी मतभेद है, तथापि मैं यह कहने को बाध्य हूँ कि विवेकानन्द ने यहाँ के अनेक लोगों की आँखें खोल दी हैं तथा उनके हृदय को उदार बना दिया है। उनकी शिक्षाओं के फलस्वरूप अब यहाँ के अधिकांश लोग दृढ़तापूर्वक विश्वास करने लगे हैं कि प्राचीन हिन्दू शास्त्रों में अद्भुत आध्यात्मिक सत्य छिपे पड़े हैं। उन्होंने केवल इसी भाव को फैलाया हो, ऐसी बात नहीं; वे इंग्लैंड तथा भारत के बीच एक स्वर्णिम सम्बन्ध स्थापित करने में भी सफल हुए हैं। श्री हैविस द्वारा लिखित ‘द डेड पुलपिट’ (ईसाई धर्म का अवसान) नामक ग्रन्थ से ‘विवेकानन्दवाद’ पर मैंने जो उद्धरण दिया है, उससे आप स्पष्ट रूप से समझ सकते हैं कि विवेकानन्द के विचारों के प्रसार के फलस्वरूप सैकड़ों लोगों ने ईसाई धर्म से नाता तोड़ लिया है। और निम्नलिखित घटना से यह बोध हो जाता है कि इस देश में उनका कार्य कितना प्रभावी तथा विस्तृत रूप से हुआ है। 

कल शाम को मैं दक्षिणी लन्दन के निवासी अपने एक मित्र से मिलने जा रहा था। रास्ता भूलकर मैं एक मोड़ के किनारे खड़ा होकर सोच रहा था कि किस ओर जाऊँ! उसी समय एक महिला एक बालक के साथ मेरी ओर आयीं।... मुझे लगा कि सम्भवतः वे मुझे रास्ता दिखा देना चाहती हैं। उन्होंने मुझसे कहा, “महाशय, शायद आप अपना रास्ता ढूँढ़ रहे हैं, क्या मैं आपकी सहायता कर सकती हूँ?”... मुझे रास्ता बताने के बाद वे बोलीं, “किन्हीं समाचार-पत्रों से मुझे पता चला है कि आप लन्दन आ रहे हैं। आपके ऊपर दृष्टि पड़ते ही मैंने अपने पुत्र से कहा - वह देखो स्वामी विवेकानन्द खड़े हैं।” चूँकि मुझे ट्रेन पकड़ने की जल्दी थी, अतः मेरे पास ठहरकर यह बताने का समय नहीं था कि मैं विवेकानन्द नहीं हूँ और मैं शीघ्रतापूर्वक वहाँ से चला गया। तथापि विवेकानन्द को देखे बिना ही उनके प्रति उक्त महिला की महान् श्रद्धा को देखकर मैं सचमुच ही आश्चर्यचकित रह गया। इस सुखद घटना पर मैंने अतीव आनन्द का अनुभव किया और अपनी गेरुआ पगड़ी को धन्यवाद दिया, जिसने मुझे इतना सम्मान दिलाया था। इस घटना के अतिरिक्त भी मैंने यहाँ ऐसे अनेक शिक्षित अंग्रेज सज्जनों को देखा है, जो भारत के प्रति श्रद्धा रखते हैं और भारत के धार्मिक तथा आध्यात्मिक तत्त्वों को अतीव आग्रह के साथ सुनते हैं।\endnote{ इंडियन मिरर (अंग्रेजी दैनिक), १५ फरवरी, १८९८}


\section*{महात्मा गांधी}

\addsectiontoTOC{महात्मा गांधी}

मैं आज (६ फरवरी १९२१ को) यहाँ (बेलुड़ मठ में) स्वामी विवेकानन्द के जन्म-दिवस पर उनकी पुण्य स्मृति में श्रद्धांजलि अर्पित करने आया हूँ। मैंने स्वामीजी के ग्रन्थ बड़े ही मनोयोग के साथ पढ़े हैं और इसके फलस्वरूप देश के प्रति मेरा प्रेम हजारों-गुना बढ़ गया है। युवकों से मेरा अनुरोध है कि जिस स्थान पर स्वामी विवेकानन्द ने निवास और देहत्याग किया, वहाँ से कुछ प्रेरणा लिए बिना, खाली हाथ न लौटें।\endnote{ प्रबुद्ध भारत, मई १९६३, पृ. १७०}


\section*{पं. जवाहर लाल नेहरू}

\addsectiontoTOC{पं. जवाहर लाल नेहरू}

प्राचीन भारत में पगे और भारतीय परम्परा के गर्व से परिपूर्ण, विवेकानन्द का जीवन की समस्याओं के प्रति दृष्टिकोण अत्याधुनिक था। प्राचीन और वर्तमान भारत के बीच वे एक सेतु के समान थे।... उनका व्यक्तित्व प्रभावशाली था, गम्भीरता और आत्मसम्मान उनमें भरा हुआ था, अपने तथा अपने कार्य के प्रति उनमें श्रद्धा थी और साथ ही वे क्रियाशीलता और अदम्य शक्ति से ओतप्रोत थे। भारत को आगे बढ़ाने की उनमें तीव्र उत्कण्ठा थी। हताश और निरुत्साहित हिन्दू मानस के लिए वे एक संजीवनी औषधि के रूप में आए। उन्होंने हमें आत्मविश्वास प्रदान किया और प्राचीन धारा से जुड़ने के कुछ सूत्र दिये।\endnote{ \engfoot{\textit{The Discovery of India,} Meridian Books Limited, London, 1960, p. 338}} 

मुझे पता नहीं कि आज की युवा पीढ़ी में से कितने लोग स्वामी विवेकानन्द के लेख तथा व्याख्यान पढ़ते हैं, पर मैं अपने काल की बात कह सकता हूँ। मेरी पीढ़ी के अनेक युवक उनसे अत्यधिक प्रभावित हुए थे और मेरा विचार है कि वर्तमान पीढ़ी भी यदि स्वामीजी के व्याख्यान और उनकी रचनाओं का अनुशीलन करे, तो उसका बड़ा उपकार होगा, वे काफी कुछ सीख सकेंगे। स्वामीजी की वाणी थोथे शब्द मात्र न थे, बल्कि उनमें उनके मन-प्राण में प्रज्वलित होने वाली उस अग्नि की झलक मिल जाती है, उनकी वाग्मिता और जीवनदायी भाषा के माध्यम से अभिव्यक्त होनेवाली उस ज्वाला के दर्शन होते हैं, जिसने अत्यल्प आयु में ही उन्हें स्वाहा कर दिया था। अपने मुख से उच्चरित होनेवाले शब्दों में उन्होंने अपनी अन्तरात्मा उड़ेल दी और इसी कारण वे एक महान् वक्ता हो गए थे। उन्होंने वाक्चातुर्य की चमक-दमक तथा अलंकरण का नहीं, बल्कि गहन श्रद्धा और आत्मा में आस्था का आश्रय लिया था, इसीलिए भारत में बहुसंख्य लोगों पर उन्होंने काफी गहरा प्रभाव डाला और इसमें सन्देह नहीं कि युवक और युवतियों की दो-तीन पीढ़ियाँ उनसे प्रभावित होती रही हैं।... 

जिन लोगों ने नये भारत के प्रारम्भिक और कठिनाई के दिनों में आकर पहले से ही इसे तैयार किया और आकार दिया, तब से ऐसा काफी कुछ घटा है जिसके कारण सम्भवतः कुछ लोग उन्हें भूल जाते हैं। यदि आप स्वामी विवेकानन्द की रचनाएँ और व्याख्यान पढ़ें, तो उनमें आप एक अद्भुत बात देखेंगे कि वे कभी पुरानी नहीं प्रतीत होतीं। ये बातें ५६ वर्ष पूर्व\footnote{ पं. नेहरू का यह व्याख्यान १९४९ ई. में हुआ था।} कही गयी थीं, पर आज भी वैसी ही तरोताजा हैं, क्योंकि उन्होंने जो कुछ भी लिखा या कहा, वहाँ हमारी अथवा विश्व समस्याओं के मूलभूत तत्त्वों तथा पहलुओं से सम्बन्धित था। यही कारण है कि वे पुराने नहीं लगते। आज भी यदि आप उन्हें पढ़ें, तो वे नये ही प्रतीत होंगे। 

उन्होंने ऐसा कुछ दिया, जो हमें अपनी विरासत के प्रति एक तरह का गर्व प्रदान करता है। पर उन्होंने हमें बख्शा नहीं। उन्होंने हमारे दोषों और असफलताओं के बारे में भी बहुत-कुछ कहा। उन्हें कुछ भी छिपा रखने की इच्छा न थीं और वस्तुतः उनके लिए ऐसा करना उचित भी नहीं था। चूँकि हमें उन दोषों को दूर करने की जरूरत है, इसलिए उन्होंने उन दोषों की ओर इंगित भी किया। कभी-कभी वे हमारे ऊपर कठोर आघात करते हैं, पर कभी-कभी वे भारत के उन महान् तत्त्वों का भी बोध कराते हैं, जो भारत का वैशिष्ट्य रहे हैं और जिन्होंने हमारे दुर्दिनों में भी हमारी महानता को कुछ हद तक बरकरार रखा है। 

इसलिए, स्वामीजी ने जो कुछ भी लिखा या कहा है, वह हमारे हित में है और होना भी चाहिए तथा वह आने वाले लम्बे अरसे तक हमें प्रभावित करता रहेगा। वे साधारण अर्थ में कोई राजनीतिज्ञ नहीं थे, फिर भी, मेरी राय में, वे भारत के राष्ट्रीय आन्दोलन के महान् संस्थापकों में से एक थे; और आगे चलकर जिन लोगों ने उस आन्दोलन में थोड़ा या अधिक सक्रिय भाग लिया, उनमें से अनेकों ने स्वामी विवेकानन्द से ही प्रेरणा ग्रहण की थी। 

प्रत्यक्ष या अप्रत्यक्ष रूप से उन्होंने वर्तमान भारत को सशक्त रूप से प्रभावित किया था और मेरा विश्वास है कि हमारी युवा पीढ़ी स्वामी विवेकानन्द के अन्तर से प्रवहमान ज्ञान, प्रेरणा और उत्साह के स्रोत से लाभ उठाएगी।... 

श्रीरामकृष्ण परमहंस जैसे लोग, स्वामी विवेकानन्द जैसे लोग और महात्मा गांधी जैसे लोग - विश्व की महान् संयोजक शक्तियाँ हैं, विश्व की महान् रचनात्मक प्रतिभाएँ हैं; और केवल उनके द्वारा प्रदत्त कुछ विशिष्ट विचारों के सन्दर्भ में ही नहीं, अपितु विश्व के प्रति उनके दृष्टिकोण तथा इस पर जाने या अनजाने पड़नेवाला उनका प्रभाव - हमारे लिए सर्वाधिक महत्त्व की चीज है।\endnote{ \engfoot{\textit{Sri Ramakrishna and Vivekanunda,} Advaita Ashrama, Calcutta, 1960, pp. 4-13}}


\section*{सुभाषचन्द्र बोस}

\addsectiontoTOC{सुभाषचन्द्र बोस}

पिछली शताब्दी के ८० वाले दशक में भारत में दो धार्मिक महापुरुषों का उदय हुआ, जिनका देश के नव-जागरण की धारा पर विशेष प्रभाव पड़ा। वे थे श्रीरामकृष्ण परमहंस और उनके शिष्य स्वामी विवेकानन्द।... रामकृष्ण परमहंस ने सभी धर्मों की मूलभूत एकता का और धर्मों-कर्मों के आपसी विद्वेष की समाप्ति का उपदेश किया।... अपने देहत्याग के पूर्व उन्होंने अपने शिष्य को अपने धार्मिक उपदेशों का भारत और विश्व भर में प्रचार करने का गुरुभार सौंप दिया था। तदनुसार स्वामी विवेकानन्द ने रामकृष्ण मिशन की स्थापना की, जो एक प्रकार का संघ था। इसके साधु पूर्णतः हिन्दू जीवन जीकर देश और विदेश में, खासकर अमेरिका में विशुद्ध हिन्दू धर्म का प्रचार करते हैं। स्वामीजी राष्ट्र में हर प्रकार के स्वस्थ क्रिया-कलापों के प्रेरणा-स्रोत रहे; उनके लिए धर्म राष्ट्रवाद का प्रेरक था। उन्होंने भारत की नयी पीढ़ी में अपने अतीत के प्रति गर्व, भविष्य के प्रति विश्वास और स्वयं में आत्मविश्वास तथा आत्मसम्मान की भावना जगाने का प्रयास किया। यद्यपि स्वामी विवेकानन्द ने कोई राजनीतिक विचार या सन्देश नहीं दिया, तथापि हर व्यक्ति, जो उनके सम्पर्क में आया या जिसने भी उनके लेखों को पढ़ा, वह देशभक्ति की भावना से ओतप्रोत हो गया और स्वतः ही उसमें राजनीतिक चेतना पैदा हो गयी। कम-से-कम जहाँ तक बंगाल का प्रश्न है, स्वामी विवेकानन्द को वहाँ के आधुनिक राष्ट्रवादी आन्दोलन का जनक माना जा सकता है। यद्यपि उनका देहावसान बहुत जल्दी १९०२ ई. में ही हो गया, लेकिन उनका प्रभाव उनकी मृत्यु के बाद और अधिक बढ़ गया।”\endnote{ \engfoot{\textit{The Indian Struggle,} Asia Publishing House, Bombay etc., 1964, p. 21}} 

स्वामीजी के बारे में कुछ लिखते समय मैं आनन्द-विभोर हुए बिना नहीं रह पाता। जिन लोगों को उनके साथ घनिष्ठतापूर्वक मिलने-जुलने का सौभाग्य हुआ था, उनमें भी शायद ही कोई-कोई उन्हें यथार्थ रूप से समझ सके और उनकी गम्भीरता की थाह पा सके। उनका व्यक्तित्व समृद्ध, गहन तथा बहुमुखी था, जो उनके उपदेशों तथा लेखों से भी विलक्षण था; और उनके इसी व्यक्तित्व ने उनके स्वदेशवासियों पर अद्भुत प्रभाव विस्तार किया था।... वे अपने त्याग में असंयमित, कर्म में अथक, प्रेम में असीम, विद्या में बहुमुखी, बुद्धि में गहन, भावुकता में अतिरेकी और संघर्ष में निर्मम थे, पर इसके साथ ही वे एक शिशु के समान सरल थे। हमारे इस जगत् में उनके समान व्यक्तित्व दुर्लभ हैं।... 

\newpage

स्वामीजी पूर्ण विकसित पौरुष से सम्पन्न थे, उनके रग-रग में योद्धापन भरा था। इसीलिए वे शक्ति के उपासक थे और इसी कारण उन्होंने अपने देशवासियों का उत्थान करने हेतु वेदान्त की एक नवीन व्यावहारिक व्याख्या दी।... वे इतने महान्, गहन तथा बहुमुखी थे कि मैं घण्टों लिखकर भी इन महापुरुष की महिमा के प्रति तनिक भी न्याय नहीं कर सकूँगा। वे विश्व के प्रथम ऐसे सर्वोच्च कोटि के योगी थे, जिन्होंने ब्रह्म का साक्षात्कार करने के बाद भी स्वदेश तथा मानवता के नैतिक एवं आध्यात्मिक उत्थान के लिए अपना सम्पूर्ण जीवन अर्पित कर दिया था। आज यदि वे जीवित होते, तो मैं उनके चरणों में होता। यदि मैं भूल नहीं करता, तो आधुनिक भारत उन्हीं की सृष्टि है।\endnote{ प्रबुद्ध भारत, जुलाई १९३२, पृ. ३५२} 

श्रीरामकृष्ण और स्वामी विवेकानन्द के प्रति मैं कितना ऋणी हूँ - इसे मैं शब्दों के माध्यम से व्यक्त नहीं कर सकता। उन्हीं के पुण्य प्रभाव से मेरे जीवन में चेतना का प्रथम प्रदुर्भाव हुआ था। निवेदिता के समान ही मेरा भी विश्वास है कि रामकृष्ण और विवेकानन्द एक ही अखण्ड व्यक्तित्व के दो पहलू हैं। आज यदि स्वामीजी जीवित होते, तो निश्चय ही वे मेरे गुरु होते अर्थात् मैंने अवश्य ही उनका गुरु-रूप में वरण कर लिया होता। अस्तु, कहना न होगा कि जब तक मैं जीवित रहूँगा, ‘रामकृष्ण-विवेकानन्द’ का अनन्य अनुगत तथा अनुरागी बना रहूँगा।\endnote{ उद्बोधन (बंगला मासिक), आश्विन १३५४ (बंगाब्द), पृ. ४५९} 

स्वामी विवेकानन्द के बहुमुखी प्रतिभा की व्याख्या करना बड़ा कठिन है। मेरे समय का छात्र-समुदाय स्वामीजी की रचनाओं और व्याख्यानों से जैसा प्रभावित होता था, वैसा और किसी से भी नहीं होता था। वे मानो उन (छात्रों) की आशाओं-आकांक्षाओं को पूर्णरूप से अभिव्यक्त करते थे। (परन्तु) स्वामीजी का यथार्थ मूल्यांकन करने के लिए उन्हें परमहंसदेव के साथ मिलाकर देखना होगा। वर्तमान स्वाधीनता आन्दोलन की नींव स्वामीजी की वाणी पर ही आश्रित है। भारतवर्ष को यदि स्वाधीन होना है, तो उसमें हिन्दुत्व या इस्लाम का प्रभुत्व होने से काम न होगा - उसे राष्ट्रीयता के आदर्श से अनुप्राणित कर विभिन्न सम्प्रदायों का सम्मिलित निवास-स्थान बनाना होगा। रामकृष्ण-विवेकानन्द के ‘धर्मसमन्वय’ का सन्देश भारतवासियों को सम्पूर्ण हृदय के साथ अपनाना होगा।... 

स्वामीजी ने प्राच्य तथा पाश्चात्य, धर्म तथा विज्ञान और अतीत एवं वर्तमान के बीच समन्वय साधित किया; वे इसी कारण महान् हैं। उनकी शिक्षाओं में हमारे देशवासी अभूतपूर्व आत्म-सम्मान, आत्म-विश्वास और आत्म-प्रतिष्ठा का बोध कर रहे हैं।”\endnote{ वही, फाल्गुन १३३७ (बंगाब्द)} 

रामकृष्ण परमहंस ने अपनी साधना के द्वारा जो सर्व-धर्म-समन्वय कर दिखाया था, वही स्वामीजी के जीवन का मूलमंत्र बना और वही भावी भारत के मूलमंत्र का भी आधार हुआ। इस सर्व-धर्म-समन्वय और सभी मतों में सहिष्णुता के बिना हमारे इस विविधतापूर्ण देश में राष्ट्रीयता का बोध स्थापित नहीं हो सकता।... 

राममोहन राय के युग से विभिन्न आन्दोलनों के जरिये भारत की मुक्ति-कामना धीरे-धीरे प्रकट हुई। यह आकांक्षा उन्नीसवीं शताब्दी के चिन्तन तथा सामाजिक सुधारों में दीख पड़ी थी, परन्तु यह राजनैतिक क्षेत्र में कभी प्रकट नहीं हुई। इसका कारण यह था कि तब भी भारतवासी पराधीनता की मोहनिद्रा में डूबे हुए थे और सोचते थे कि अंग्रेजों का भारत-विजय एक दैवी वरदान है। उन्नीसवीं शताब्दी के अन्त में और बीसवीं शताब्दी के प्रारम्भ में स्वाधीनता के अखण्ड रूप का आभास रामकृष्ण-विवेकानन्द के मध्य से झलकता है। ‘फ्रीडम, फ्रीडम इज दी सांग आफ दी सोल - स्वाधीनता हमारी आत्मा का संगीत है।’ यह सन्देश जब स्वामीजी के हृदय से निकला, तब उन्होंने समग्र देशवासियों को मुग्ध और उन्मत्त कर दिया। उनकी साधना के द्वारा, आचरण के द्वारा, वचनों और भाषणों के द्वारा यह सत्य प्रकट हुआ। 

\vskip 2pt

स्वामी विवेकानन्द ने मनुष्य को तरह-तरह के बन्धनों से मुक्त होकर सही मनुष्य बनने को कहा और दूसरी तरफ सर्व-धर्म-समन्वय के प्रचार के जरिये भारतीय राष्ट्रीयता की आधार-शिला स्थापित की।\endnote{ नूतनेर सन्धान (बंगला), कोलकाता, पृ. २४-२६}


\section*{विनोबा भावे}

\addsectiontoTOC{विनोबा भावे}

विवेकानन्द ने हमें, न केवल अपनी शक्ति के विषय में सजग बनाया, बल्कि उन्होंने हमारे दोषों तथा दुर्बलताओं के प्रति भी हमारा ध्यान आकृष्ट किया।... भारतवर्ष उन दिनों तमोगुण (अज्ञान तथा अविवेक) में डूबा हुआ था और अपनी दुर्बलता को ही अनासक्ति तथा शान्ति समझने की भूल कर बैठा था। इसीलिए स्वामीजी ने तो यहाँ तक कहा कि आलस्य तथा प्रमाद की तुलना में आपराधिक सक्रियता भी वांछनीय है। उन्होंने लोगों को इस विषय में सचेत किया कि वे जिस तामसिक अवस्था में स्थित हैं, इस बात को तथा उसे तोड़कर बाहर निकल आने की आवश्यकता को समझें, ताकि वे सीधे खड़े होकर अपने जीवन में वेदान्त की शक्ति की अनुभूति कर सकें। 

\vskip 2pt

जो लोग निर्जन में जाकर समाज से कटकर दर्शन-शास्त्र तथा धर्मग्रन्थों के अध्ययन का आनन्द ले रहे थे, उनके विषय में उन्होंने कहा कि इस प्रकार के जीवन की अपेक्षा फुटबाल खेलना कहीं अधिक अच्छा है। विभिन्न प्रसंगों में अपनी उक्तियों के द्वारा उन्होंने उस तमोगुण की ओर संकेत किया, जो भारत की आत्मिक शक्ति को आच्छन्न किए हुए थी और उसके गौरव को पुनः स्थापित किया। उनकी शिक्षा थी - “हर प्राणी के भीतर एक ही अन्तरात्मा विराजती है; और यदि तुम्हें इस बात पर विश्वास है, तो तुम्हारा यह कर्तव्य हो जाता है कि सबको अपना भाई मानो और सम्पूर्ण मानव-जाति की सेवा करो।” लोगों की ऐसी धारणा थी कि यद्यपि सभी लोगों का तत्त्व-ज्ञान में समान अधिकार है, तथापि दिन-प्रतिदिन के व्यवहार तथा सम्बन्धों में ऊँच-नीच के भेद को बनाए रखना चाहिए। स्वामीजी ने हमें इस सत्य का बोध कराया कि जिस तत्त्व-ज्ञान का अपने मानव-भाइयों के साथ हमारे प्रतिदिन के व्यवहार तथा क्रिया-कलापों में कोई स्थान नहीं है, वह (तत्त्व-ज्ञान) व्यर्थ तथा निरर्थक है। इसीलिए उन्होंने हमें सलाह दी कि हमें दरिद्रनारायण की उन्नति के लिए अपना जीवन उनकी सेवा में अर्पित कर देना चाहिए। ‘दरिद्रनारायण’ शब्द स्वामीजी ने बनाया और उसे लोकप्रिय बनाने का कार्य गाँधीजी ने किया।\endnote{ प्रबुद्ध भारत, मई १९६३, पृ. १७२-७३}


\section*{रोमाँ रोलाँ}

\addsectiontoTOC{रोमाँ रोलाँ}

वे (विवेकानन्द) शक्ति की सजीव प्रतिमा थे और कर्म ही मानवता के प्रति उनका सन्देश था। बीथोवन के समान ही उनके लिए भी यही सभी सद्गुणों का मूल था।... 

उनका सम्राट् जैसा भाव ही उनका वैशिष्ट्य था। वे मानो एक जन्मजात महाराजा थे। भारत अथवा अमेरिका में जो कोई भी उनके सम्पर्क में आया, उनकी तेजस्विता के सम्मुख मस्तक झुकाने को बाध्य हुआ। 

१८९३ ई. के सितम्बर में, शिकागो में आयोजित सर्वधर्म सम्मेलन के कार्डिनल गिबन्स द्वारा उद्घाटन के अवसर पर, जब तीस वर्ष का यह अज्ञात युवक प्रकट हुआ, तो उसकी भव्य उपस्थिति में अन्य सभी प्रतिनिधि विस्मृत हो गए। एंग्लो-सैक्सन जाति की विपुल श्रोतृ-मण्डली पहले तो उनके रंग के कारण उनके प्रति पूर्वाग्रह से युक्त थी, पर उनका बल और सौंदर्य, उनका ओज और शालीनता उनके नेत्रों की गहरी चमक तथा प्रभावशाली मुद्रा देखकर; और जब उन्होंने बोलना आरम्भ किया, तो उनकी गुरुगम्भीर वाणी के भव्य संगीत को सुनकर, ठगी सी रह गयी। इन योद्धा उपदेशक ने संयुक्त राज्य अमेरिका पर गहरी छाप छोड़ी। 

उन्हें कहीं भी दूसरे स्थान पर सोच पाना असम्भव था। वे जहाँ कहीं भी गए, प्रथम रहे।... प्रत्येक व्यक्ति ने प्रथम दृष्टि में ही उन्हें पहचान लिया कि वे एक नायक है, ईश्वर-प्रेरित व्यक्ति हैं; एक ऐसे आधिकारिक व्यक्ति, जिनमें लोगों पर आदेश चलाने की क्षमता है। हिमालय में यात्रा के दौरान उनके सामने से आता हुआ एक यात्री - उन्हें पहचाने बिना ही, उन्हें देखकर स्तम्भित होकर खड़ा हो गया और चिल्ला उठा - “शिव!” 

मानो उनके प्रिय देवता ने स्वयं ही उनके मस्तक पर अपना नाम अंकित कर दिया था।... 

उन्होंने अपनी आयु के अभी अपनी चालीस वर्ष भी पूरे नहीं किए थे कि उनका बलिष्ठ शरीर चिता पर सुला दिया गया।... 

परन्तु उस चिता की अग्नि मानो अब भी प्रज्वलित है। उसी चिता के भस्म से भारत की\break अन्तश्चेतना प्राचीन आख्यान के अलौकिक पक्षी ‘फोनिक्स’ की भाँति नवजीवन पाकर जाग उठी है; जाग उठी है अपनी एकता और उस महान् सन्देश के प्रति श्रद्धा के रूप में, जिसका उसकी प्राचीन जाति की स्वप्नदर्शी आत्मा वैदिक काल से ही चिन्तनमनन करती आयी है और जो सन्देश उसे बाकी मानवता को थाती रूप में सौंपना है। 

\delimiter

कोलम्बो में उनके व्याख्यान और रामेश्वरम् में उनके उपदेश हृदयग्राही थे, परन्तु अपनी\break महानतम शक्तियों को उन्होंने मद्रास के लिए रख छोड़ा था। मद्रास एक तरह की उद्दाम विह्वलता के साथ कई सप्ताह से उनकी बाट जोह रहा था।... 

लोगों की उद्दाम आकांक्षा को उन्होंने भारत के प्रति अपना सन्देश देकर सन्तुष्ट किया। वह सन्देश मानो राम, कृष्ण और शिव की भूमि के जागरण का शंखनाद था, अमर आत्मा के वीरत्व के प्रति युद्ध में मार्च करने का आह्वान था। वे, एक सेनानायक, अपने लोगों को अपनी समर-नीति सुझाते हुए एक साथ उठ खड़े होने को पुकार रहे थे। 

“मेरे भारत, जागो!... 

“आगामी पचास वर्षो तक... हमारे मन से अन्य सभी व्यर्थ के देवी-देवता लुप्त हो जाएँ। हमारा राष्ट्र - यही हमारा एकमात्र ईश्वर है, सर्वत्र उसके कान हैं, वह सब कुछ आवृत्त किए हुए हैं और बाकी सभी देवता सो रहे हैं। जब उस देवता को, जिसे हम अपने चारों ओर देख पाते हैं, उस विराट् की पूजा नहीं कर पाते, तो फिर हम अन्य व्यर्थ के देवताओं के पीछे क्यों पड़े?... हमारे चारों तरफ जो लोग हैं सर्वप्रथम उन विराट् की पूजा करो। मानव और पशु - सभी हमारे देवता हैं, और सबसे पहले हमें जिन देवताओं की पूजा करनी होगी, वे हैं हमारे अपने देशवासी...~।” 

इन तड़ित् के समान शब्दों की कैसी प्रतिध्वनि हुई होगी, इसकी आप स्वयं कल्पना करें! 

झंझावात आया और चला गया, पर अपने पीछे वह सर्वत्र आत्मशक्ति, मानव में प्रच्छन्न ब्रह्म और उसकी अनन्त सम्भावनाओं में अदम्य निष्ठा का अग्नि-प्रपात फैलाता गया। यहाँ मेरे मनश्चक्षुओं के समक्ष रेम्ब्राँ द्वारा उत्कीर्ण वह मूर्तिशिल्प भास रहा है, जिसमें लाजरस की कब्र पर ईसा मसीह के समान प्रज्ञापुरुष तनकर खड़े हैं, उनके हाथ उठे हुए हैं और मृतक को जगाने तथा जीवनदान करने के लिए उनकी प्रभावान मुखमुद्रा से शक्ति का प्रवाह निःश्रित हो रहा हैं।... 

क्या मृतक में प्राण-संचार हुआ? क्या भारत ने अपने सन्देशवाहक के शब्दों पर रोमांचित होकर, उनकी आशा को भी पूर्ण किया? क्या उनका मुखर उत्साह कार्य में रूपायित हुआ? लगता है कि उस समय के लिए यह सम्पूर्ण अग्नि धूम्र रूप में परिणत हो गयी थी और विवेकानन्द ने दो वर्ष बाद खेदपूर्वक कहा कि मुझे अपनी सेना के लिए अपेक्षित भारतीय युवकों की फसल प्राप्त नहीं हुई। स्वप्नों में आच्छन्न, पूर्वग्रह से ग्रस्त और अल्प उद्यम के भार को भी सह पाने में असमर्थ एक राष्ट्र की आदतों को क्षण भर में बदल पाना असम्भव था, परन्तु आचार्यदेव के कठोर आघात ने अपनी निद्रा में उसे पहली बार करवट बदलने को विवश किया। स्वप्न में डूबे भारत को पहली बार अपने ब्रह्मबोध के साथ आगे कूच करने का वीरतापूर्ण शंखनाद सुनाई पड़ा। वह कभी इसे विस्मृत नहीं कर सका। उसी दिन से तन्द्रालु महामूर्ति का जागरण आरम्भ हुआ। विवेकानन्द के निधन के तीन वर्ष बाद ही, आगामी पीढ़ी ने, तिलक और गाँधी के महान् आन्दोलन के पूर्वाभास रूप में बंगाल का जो विद्रोह देखा; और आज यदि भारत ने संगठित और सामूहिक जन-आन्दोलन में निश्चित रूप से भाग लिया है, तो यह उनके मद्रास के सन्देश में निहित ‘लाजरस, उठो! जागो।’ के उस प्रारम्भिक आघात के कारण ही था। 

शक्ति के इस सन्देश का दोहरा तात्पर्प था, जिनमें एक था राष्ट्रीय और दूसरा सार्वभौमिक। यद्यपि अद्वैतवादी महान् संन्यासी के लिए तो इसका सार्वभौमिक अर्थ ही प्रधान था। परन्तु इसके दूसरे अर्थ ने भारतीय बल को पुनर्जीवन प्रदान किया। 

\delimiter

उनके शब्द महान् संगीत हैं, बीथोवन शैली के टुकड़े हैं, हैंडेल के समवेत गान के छन्द-प्रवाह की भाँति उद्दीपक लय हैं। शरीर में विद्युत्स्पर्श के-से आघात की सिहरन का अनुभव किए बिना, मैं उनकी उस वाणी का स्पर्श नहीं कर सकता, जो तीस वर्ष की दूरी पर पुस्तकों के पृष्ठों में बिखरे पड़े हैं। और जब वे ज्वलन्त शब्दों के रूप में नायक के मुख से निकले थे, तब तो न जाने कैसे-कैसे आघात और आवेग पैदा हुए होंगे। 

\delimiter

अनेक शताब्दियों से भारतवर्ष जिस ऊपर विचारधारा की चलमान बालुका-राशि में फँसा हुआ था, उससे उद्धार उसी के एक संन्यासी के हाथों हुआ; और इसके फलस्वरूप उसके नीचे छिपी हुई, आध्यात्मिकता की जलराशि बड़ी-बड़ी लहरों के रूप में प्रवाहित होने लगी। इस प्रकार जो प्रचण्ड शक्तियाँ उन्मुक्त हुई थीं, उनसे पश्चिम को अवगत कराना आवश्यक है। 

आज विश्व जागरणशील भारत के सम्मुखीन हुआ है। इस विराट् प्रायद्वीप के सुविस्तृत क्षेत्र में फैलकर लेटी हुई इसकी महान् काया अब अपने हाथ-पाँव फैला रही है; अपनी बिखरी शक्तियों को एकत्र कर रहीं है। विगत शताब्दियों में तीन पीढ़ियों के तूर्यवादकों ने इस पुनर्जागरण में चाहे जो भी भूमिका क्यों न निभायी हो, (जिनमें सर्वोच्च और सच्चे अग्रदूत राममोहन राय हैं, जो हमारे प्रणम्य हैं), परन्तु चरम तूर्यनाद तो (स्वामीजी के) कोलम्बो तथा मद्रास के व्याख्यानों के माध्यम से ही हुआ था। 

और वह जादुई मूलमंत्र था - एकता - प्रत्येक भारतीय नर-नारी के बीच एकता, आत्मा की समस्त शक्तियों - कल्पना और क्रिया के बीच एकता, मुक्ति, प्रेम एवं कर्म के बीच एकता, अपनी सैकड़ों भिन्न-भिन्न भाषाओं के बीच एकता, वर्तमान एवं भावी पुनर्निर्माण के मूल एक ही धार्मिक केन्द्र से उत्थित होने वाले लाखों देवी-देवताओं के बीच एकता और प्राच्य तथा पाश्चात्य के, भूत तथा वर्तमान के समस्त धार्मिक विचारों के विशाल सागर के बीच एकता। आज भारत पश्चिम की उद्धत सभ्यता के प्रति निष्ठा दिखाने से इन्कार करता है, अपने निजी विचारों की रक्षा के लिए संघर्ष करता हैं, अपने युगों पूर्व की विरासत को उसके किसी भी अंश को परित्याग न करने के संकल्प के साथ उसने अपना लिया हैं, बाकी जगत् को इससे लाभ उठाने की अनुमति प्रदान की है और उसके बदले में पाश्चात्य बौद्धिकता की उपलब्धियों को स्वीकार किया है। इसी में रामकृष्ण-विवेकानन्द और राममोहन-ब्रह्मसमाज के जागरण के बीच का भेद निहित है। एक अपूर्ण और भेदभावपूर्ण सभ्यता के प्रभुत्व के दिन अब बीत चुके हैं। दो दीर्घकाय - एशिया और यूरोप - पहली बार समानता के आधार पर आज आमने-सामने खड़े हैं। यदि वे विवेकवान हुए, तो मिलकर कार्य करेंगे और उनके परिश्रम का फल सबको प्राप्त होगा। 

यह ‘बृहत्तर भारत’, यह ‘अभिवन भारत’, जिसके अभ्युदय की बात राजनीतिज्ञ और विद्वज्जन हमसे शुतुरमुर्ग की भाँति छिपाते आए हैं और जिसका विलक्षण प्रभाव अब हमारे सामने स्पष्ट हो उठा है, रामकृष्ण की आत्मा से ओतप्रोत है। परमहंस और उनके विचारों को रूपायित करनेवाले वीर (विवेकानन्द) ये युग्म-नक्षत्र, वर्तमान भारत को प्रभावित और प्रेरित कर रहे हैं। उनकी उष्ण कान्ति खमीर के समान क्रियाशील होकर भारतभूमि को उर्वर बना रही है। भारत के महान् नेतागण - मनीषा-शिरोमणि अरविन्द घोष, कवीन्द्र रवीन्द्र और महात्मा गाँधी - ये सभी राजहंस और गरुड़ के युग्म-नक्षत्र-आलोक में ही पल्लवित, पुष्पित तथा फलित हुए हैं। अरविन्द और गाँधी ने यह बात सार्वजनिक तौर पर स्वीकार भी की है।... 

जहाँ तक गेटे के समान प्रतिभाशाली टैगोर का सवाल है, वे भारत की सभी नदियों के संगम पर स्थित थे; शायद यह मान लेना उचित होगा कि उनके भीतर - अपने महर्षि पिता से प्राप्त ब्रह्मसमाज और रामकृष्ण-विवेकानन्द का नव-वेदान्त - दो धाराएँ संयुक्त तथा समायोजित हुई थीं। दोनों से समृद्ध, दोनों में सहज गति प्राप्त, उन्होंने शान्तिपूर्वक अपने भाव में पश्चिम तथा पूरब को जोड़ा। यदि मैं भूल नहीं करता हूँ, तो सामाजिक तथा राजनीतिक दृष्टिकोण से उन्होंने केवल एक ही बार - स्वामीजी के देहान्त के चार वर्ष बाद लगभग १९०६ ई. में स्वदेशी आन्दोलन के आरम्भ में अपने विचारों की सार्वजनिक रूप से घोषणा की थी। इसमें कोई सन्देह नहीं कि ऐसे पूर्ववर्ती की साँस ने उनके विकास में कुछ भूमिका अवश्य निभाई होगी। 

\delimiter

गाँधीजी का स्वभाव रामकृष्ण या विवेकानन्द से बिलकुल विपरीत है। परन्तु हाल ही में मुझे गाँधी का वह वक्तव्य सुनकर बड़ी खुशी हुई, जिसमें उन्होंने अपने अन्तर्राष्ट्रीय संघ के मित्रों का आह्वान किया है कि वे सभी सम्प्रदायों की धार्मिक ‘स्वीकृति’ के महान् सार्वभौमिक सिद्धान्त को अंगीकार करें, जिसका कि विवेकानन्द ने प्रचार किया था।... 

मानवीय प्रगति की वर्तमान अवस्था में, जबकि अन्धी तथा सचेतन - दोनों ही प्रकार की शक्तियाँ प्रत्येक स्वभाव के लोगों को एक साथ एकत्र करके ‘सहयोग, या फिर मृत्यु’ की ओर संकेत कर रही हैं, यह परम आवश्यक है कि मानवीय चेतना को इस भाव से सराबोर कर दिया जाए, ताकि यह अपरिहार्य सिद्धान्त एक स्वीकृत सत्य बन जाए कि प्रत्येक मतवाद को जीवित रहने का समान अधिकार है और साथ ही प्रत्येक व्यक्ति का यह भी कर्तव्य है कि वह अपने पड़ोसी के धर्मविश्वास को भी श्रद्धा की दृष्टि से देखे। मेरे मतानुसार गाँधीजी ने जब इस बात की स्पष्ट रूप से घोषणा की, तब वे स्वयं को रामकृष्ण के उत्तराधिकारी के रूप में प्रस्तुत कर रहे थे। 

हममें से कोई एक भी ऐसा नहीं है, जो इस शिक्षा को हृदयंगम कर सके। इन पंक्तियों का लेखक, जिसने इस उदार आदर्श को रूपायित करने की अपने पूरे जीवन भर अस्पष्ट चेष्टा की है, अब बड़ी गम्भीरता के साथ यह महसूस करता है कि उसकी इस आकांक्षा के बावजूद उसके प्रयास में बहुत-सी त्रुटियाँ रह गयी हैं; और वह अपने लक्ष्य की ओर अग्रसर होने में सहायता पाने के लिए गाँधीजी की उसी महान् शिक्षा के प्रति कृतज्ञ है, जो विवेकानन्द - और उनसे भी अधिक रामकृष्ण द्वारा प्रचारित हुआ था।\endnote{ \engfoot{\textit{The Life of Vivekananda and the Universal Gospel,} Advaita Ashrama, Calcutta, 1970, pp. 4-7, 106-14, 146, 286-89, 307-70}}


\section*{विल डुराण्ट}

\addsectiontoTOC{विल डुराण्ट}

उन्होंने (स्वामी विवेकानन्द) अपने देशवासियों को जैसा शक्तिदायी मतवाद प्रदान किया, वैसा वैदिक काल से अब तक किसी भी हिन्दू धर्म-प्रचारक ने नहीं दिया था - 

हम मनुष्य बनानेवाला धर्म चाहते हैं।... उन दुर्बल बनानेवाली रहस्यमय मतवादों को छोड़ो और बलवान बनो।... अगले पचास वर्षों के लिए अपने मन से अन्य व्यर्थ के देवताओं को लुप्त हो जाने दो। एकमात्र यही देवता जाग्रत है - हमारा अपना राष्ट्र, सर्वत्र उसके हाथ हैं, सर्वत्र उसके पाँव हैं, सर्वत्र उसके नेत्र हैं; वह सब कुछ को आवृत्त किए हुए है।... हमें सर्वप्रथम उन लोगों की पूजा करनी होगी, जो हमारे चारों ओर विद्यमान हैं।... ये हमारे ईश्वर हैं - मानव तथा जीव-जन्तु; और पूजा के योग्य प्रथम देवता हैं हमारे अपने देशवासी। 

इससे एक कदम आगे बढ़ते ही गाँधी का आविर्भाव हुआ।\endnote{ \engfoot{\textit{The Story of Civilization: Our Oriental Heritage,} Vol. I, Simon \& Schuster, New York, 1954, p. 618}}


\section*{चक्रवर्ती राजगोपालचारी}

\addsectiontoTOC{चक्रवर्ती राजगोपालचारी}

स्वामी विवेकानन्द ने हिन्दू धर्म को बचाया और इस प्रकार भारत की रक्षा की। वे न होते तो हम अपना धर्म गँवा बैठते और स्वाधीन भी नहीं हो पाते। अतः सभी बातों के लिए हम स्वामी विवेकानन्द के ऋणी हैं। मेरी कामना है कि उनका विश्वास, उनका साहस और उनका विवेक हमें सदा-सर्वदा प्रेरित करता रहे, ताकि उनसे मिली सम्पदा को हम सुरक्षित रख सकें।\endnote{ \engfoot{\textit{Swami Vivekananda Centenary Memorial Volume,} Calcutta, 1963, p. xiii}}


\section*{डॉ. सर्वपल्ली राधाकृष्णन्}

\addsectiontoTOC{डॉ. सर्वपल्ली राधाकृष्णन्}

आज हम केवल अपने देश के नहीं, अपितु सारे विश्व के इतिहास में एक संकट के दौर से गुजर रहे हैं। बहुतों का मत है कि इस समय हम एक अथाह गर्त की कगार पर खड़े हैं। आदर्शों में विकृति आ गयी है, नैतिकता में गिरावट आ गयी है, पलायनवाद का विस्तार हो रहा है और जनमानस मानो एक उन्माद से ग्रस्त हो रहा है। जनमानस इस पर विचार करता हुआ निराशा, पराजय तथा व्यर्थता की अनुभूति से अवसन्न होता जा रहा है। ये ही वे चीजें हैं, जो आज हमें दृष्टिगोचर हो रही हैं। मानव-आत्मा की शक्ति पर इस तरह का अविश्वास मनुष्य के आत्मसम्मान के प्रति विश्वासघात है, यह मानव-प्रकृति का अपमान है। जगत् में जितने भी महान् परिवर्तन हुए हैं, वे मानव-प्रकृति के द्वारा हुए हैं। यदि स्वामी विवेकानन्द ने हमें कोई सन्देश दिया है, तो वह यही है कि अपनी निज की आध्यात्मिक शक्तियों पर विश्वास रखो।... मनुष्य के पास आध्यात्मिक शक्तियों का अक्षय भण्डार है। उसकी आत्मा सर्वोपरि है। मानव अद्वितीय है। जगत् में कुछ भी अपरिहार्य नहीं है और हम अपने सम्मुख आनेवाली भीषणतम विपत्तियों तथा बाधाओं पर विजय पा सकते हैं। बस, हमें आशा का परित्याग नहीं करना चाहिए। उन्होंने हमें संकट में सहनशीलता की शिक्षा दी, दुःख में धैर्य और निराशा में उत्साह दिया। उन्होंने कहा - “बाह्य रूपों से भ्रमित न होओ। सबके अन्तराल में दैवी इच्छा छिपी हुई है, इस विश्व में एक उद्देश्य निहित है। इस उद्देश्य के साथ सहयोग करो, उसे प्रयत्नपूर्वक अर्जित करो।”\endnote{ \engfoot{\textit{Ibid.,} pp. x-xi}}


\section*{रमेश चन्द्र मजुमदार}

\addsectiontoTOC{रमेश चन्द्र मजुमदार}

कोलम्बस द्वारा अमेरिका की खोज की ४०० वीं वर्षगाँठ के अवसर पर १८९३ ई. में शिकागो में आयोजित धर्म-महासभा में स्वामी विवेकानन्द ने हिन्दू धर्म के पक्ष में अपने विचार रखे। वहाँ विश्व के प्रायः सभी देशों से आए सभी धर्मों के प्रतिनिधियों की उपस्थिति में भारत के इन युवा संन्यासी ने वेदान्त के सिद्धान्तों तथा हिन्दू धर्म की महिमा का इतनी सुदृढ़ वक्तृत्व-शक्ति द्वारा निरूपण किया कि पहले ही दिन से उन्होंने विशाल श्रोतृ-मण्डली के हृदय को सम्मोहित कर लिया। यह कहना अतिशयोक्ति न होगा कि स्वामी विवेकानन्द ने वर्तमान विश्व के सांस्कृतिक मानचित्र पर हिन्दू धर्म के लिए एक स्थान निर्धारित कर दिया। पश्चिम के सभ्य देश अब तक हिन्दू धर्म को बड़ी हेय दृष्टि से देखा करते थे। वे सोचते कि हिन्दू धर्म अन्धविश्वासों, दुराचारों तथा अनैतिक परम्पराओं का एक पुलिन्दा मात्र है और आज के प्रगतिशील विश्व में किसी भी दृष्टि से गम्भीरतापूर्वक लेने के योग्य नहीं है। परन्तु अब, इतिहास में पहली बार, उन लोगों ने स्वामी विवेकानन्द द्वारा प्रतिपादित हिन्दू धर्म के उदात्त सिद्धान्तों को, न केवल हार्दिक अभिनन्दन के साथ स्वागत किया, अपितु विश्व की सभ्यताओं तथा संस्कृतियों के बीच एक अत्यन्त उच्च स्थान भी प्रदान किया। विराट् हिन्दू समाज पर इस घटना की प्रतिक्रिया का सहज ही अनुमान लगाया जा सकता है। 

हिन्दू समाज तथा धर्म के अनेक दोषों तथा कमियों के विषय में, शिक्षित हिन्दू बुद्धिजीवी सदा से ही पाश्चात्य लोगों, विशेषकर मिशनरियों की आलोचना के प्रति काफी संवेदनशील थे; और अपने युक्तिवादी दृष्टिकोण के कारण इस आलोचना का अधिकांश भाग स्वीकार करने को बाध्य थे। जब कभी हिन्दू तथा पाश्चात्य संस्कृतियों का तुलनात्मक मूल्यांकन होता, तो वे लोग सदा ही आत्मरक्षा का रुख दिखाते और उनका दृष्टिकोण प्रायः खेदसूचक ही होता। पाश्चात्य विद्वान् बड़े विश्वासपूर्वक अपनी संस्कृति की तुलना में हिन्दू संस्कृति को हीन सिद्ध किया करते थे और इन लोगों (भारतीय बुद्धिजीवियों) ने इसे प्रायः निश्चयपूर्वक स्वीकार कर लिया था। परन्तु अब, सहसा पासा पलट गया था और पाश्चात्य प्रतिनिधिगण भी एक स्वर में हिन्दू धर्म में निहित सत्यों की उच्च प्रशंसा करने लगे। यह एक ऐसी घटना थी, जिसकी हिन्दुओं के शत्रुओं या मित्रों - किसी ने भी कल्पना तक नहीं की थी। इसने न केवल हिन्दुओं में उनके अतीत सभ्यता तथा संस्कृति में आत्मविश्वास को लौटा दिया, बल्कि उनके राष्ट्रीय गौरव तथा देशभक्ति की भावना को तत्काल जगा दिया। यह भावना स्वामी विवेकानन्द के भारत लौटने पर सारे भारत के - कन्याकुमारी से हिमालय तक के हिन्दुओं द्वारा उन्हें अर्पित किए गए असंख्य अभिनन्दन-पत्रों में ध्वनित तथा प्रतिध्वनित हुई है। विकासमान हिन्दू राष्ट्रवाद के लिए यह एक महान् अवदान था। 

भारत लौटने पर स्वामी विवेकानन्द ने प्रचारित किया कि आध्यात्मिकता ही हिन्दू सभ्यता का आधार है। अपने लेखों तथा व्याख्यानों के द्वारा उन्होंने दिखाया कि मानवजाति के कल्याण में, हमारी आँखों में चकाचौंध उत्पन्न करनेवाली पाश्चात्य भौतिकता की अपेक्षा भारतीय\break आध्यात्मिकता का मूल्य या महत्त्व कम नहीं है। वे भारतवासियों से यह कहते कभी नहीं थके कि वे पश्चिम की चमक-दमक से चौंधियाई आँखों को अपने स्वयं के आदर्शों तथा संस्थाओं की ओर मोड़ें। पाश्चात्य आदर्शों तथा संस्थाओं के साथ तुलना करते हुए उन्होंने भारतीय आदर्शों तथा संस्थाओं को उत्कृष्टतर माना और अपने देशवासियों से कहा कि वे सोने को छोड़कर उसकी जगह भड़कीली घटिया वस्तु को न अपनाएँ।... 

परन्तु विवेकानन्द, न तो पश्चिम के प्रति किसी पूर्वाग्रह से ग्रस्त थे और न ही उसकी उपलब्धियों के महत्त्व के प्रति असंवेदनशील थे। उन्होंने स्पष्ट रूप से स्वीकार किया कि भारतीय संस्कृति न तो निर्दोष है और न पूर्ण। उसे पश्चिम से बहुत कुछ सीखना होगा, परन्तु अपनी मूलभूत विशेषताओं को छोड़कर नहीं। 

स्वामी विवेकानन्द के चरित्र में हमें एक महान् संन्यासी तथा एक निष्ठावान देशभक्त का समन्वय दीख पड़ता है। उन्होंने भारतीय राष्ट्रवाद को उसकी प्राचीन महिमा की उच्च वेदी पर प्रतिष्ठित किया और इस राष्ट्रवाद ने भारत के करोड़ों छोटे-बड़े, धनी-गरीब लोगों को अपने आप में समाहित कर लिया। उन्होंने भारत की राष्ट्रीय चेतना को जगाने में अपना जीवन उत्सर्ग कर दिया। उनकी अनेक मर्मस्पर्शी उक्तियाँ आज भी भारत की राष्ट्रीय भावनाओं को उसकी गहराई तक आन्दोलित करने में समर्थ हैं।... 

\newpage

एक तपस्वी होने के बावजूद स्वामी विवेकानन्द देशभक्तों के भी शिरोमणि थे। उनके मन में सदा-सर्वदा यही विचार घूमता रहता था कि किस प्रकार भारतवासियों की मृतप्राय, परन्तु सोयी हुई आध्यात्मिक ऊर्जा को जगाकर, भारत की प्राचीन महिमा को पुनः स्थापित किया जाए।... 

इन महान् संन्यासी ने अपने गुरुदेव श्रीरामकृष्ण के आह्वान पर अपने घर-द्वार का त्याग किया, आध्यात्मिक अनुभूतियों में गहराई तक गोता लगाया और इस बात को प्रचारित करते हुए कभी नहीं थके कि आज भारत को धर्म तथा दर्शन की आवश्यकता नहीं, वह तो उसके पास यथेष्ट मात्रा में विद्यमान है; आज भारत को आवश्यकता है, उसके करोड़ों भूखों के लिए भोजन की, निम्न वर्गों के लिए सामाजिक न्याय की, उसके निर्बल लोगों के लिए शक्ति तथा ऊर्जा की और विश्व के एक महान् राष्ट्र के रूप में गर्व तथा सम्मान के भाव की। उन्होंने समस्त भारतवासियों का आह्वान करते हुए कहा कि वे हर प्रकार के भय को त्याग दें और वेदान्त द्वारा घोषित सनातन सत्य के आधार पर याद दिलाया कि वे सभी ईश्वर के अंश हैं तथा इस भाव के द्वारा शक्तिमान होकर मनुष्यों के रूप में खड़े हो जाएँ। इन महान् संन्यासी के उपदेशों तथा जीवनादर्श ने राष्ट्रीय जीवन के प्रवाह को नया वेग प्रदान किया, उसमें नवीन आशा तथा प्रेरणा का संचार किया, और देश की सेवा को आध्यात्मिक स्तर तक पहुँचा दिया।... 

इस प्रकार स्वामी विवेकानन्द ने भारतीय राष्ट्रवाद को एक आध्यात्मिक आधार प्रदान किया। वेदान्त तथा गीता के उपदेश अनेक राष्ट्रवादियों के जीवन तथा कार्यों में ओतप्रोत रहे। अनेक शहीदों ने उनके उपदेशों से प्रेरित होकर हँसते हुए चरम कष्ट तथा बलिदान सहे, निर्भयतापूर्वक मृत्यु को गले लगाया और कभी-कभी अपने ऊपर होनेवाली मृत्यु से भी बढ़कर अमानुषिक यातनाओं को शान्तिपूर्वक सहन किया।\endnote{ \engfoot{\textit{History of the Freedom Movement in India,} Vol. I, Firma K. L. Mukhopadhyay, Calcutta, 1962, pp. 358-63}}


\section*{सुनीति कुमार चट्टोपाध्याय}

\addsectiontoTOC{सुनीति कुमार चट्टोपाध्याय}

स्वामी विवेकानन्द मेरे सम्मुख एक ऐसे मनुष्य के रूप में प्रगट हुए, जो मानवजाति और विशेषकर भारतवासियों की पीड़ाओं से काफी अभिभूत हो गए थे। इस सन्दर्भ में मध्यम तथा उच्च वर्ग के विरुद्ध उनकी कुछ तिरस्कार-सूचक उक्तियाँ हमें अपनी अन्तरात्मा की गहराई तक हिलाकर रख देती हैं। उन्होंने हमारे लिए मनुष्य की, विशेषकर उन निम्नतर श्रेणियों के लोगों की महिमा का आविष्कार किया, जिन्हें भारतीय समाज में उपेक्षा का शिकार होकर रहना पड़ा था। इसके साथ ही उन्होंने हमें वेदान्त द्वारा विशुद्ध तथा सर्वोत्कृष्ट रूप में निरूपित भारतीय चिन्तन के महत्त्व के प्रति सचेत किया। वे हम लोगों को यह विश्वास दिलाने में सफल हुए कि वेदान्त-दर्शन के रूप में हमारे पूर्वजों द्वारा छोड़ी गयी विरासत, न केवल हम भारतवासियों के लिए, अपितु बाकी मानवजाति के लिए भी चिरकालिक महत्त्व की है। इसने हमारे हृदय को उदार बनाया और हमें एक ऐसे देश के वासी होने के गर्व का बोध कराया, जिसने सदा से ही मानवजाति के सेवा को अपना उद्देश्य तथा पुनीत कर्तव्य बनाए रखा। हिन्दू जाति अपना साहस खोती जा रही थी और स्वामी विवेकानन्द ने ही हमें पुनः उस साहस की प्राप्ति में सहायता की। विशेषकर पुराने ढर्रे के ईसाई मिशनरियों द्वारा, हमारी परम्पराओं तथा जीवन-शैली की अविचारपूर्वक तथा सहानुभूतिहीन निन्दा होती रहती थी; और स्वामीजी ने इसे ध्वस्त कर दिया। अपने इन्हीं कृतित्वों के कारण वे हमारे अत्यन्त प्रिय हो गए और हम उन्हें एक ऐसे नये प्रकार के महान् आचार्य या अवतार के रूप में मानने लगे, जो हमें एक बेहतर तथा सबल व्यक्ति का जीवन बिताने का मार्ग दिखाने के लिए इस धरातल पर अवतीर्ण हुए थे। 

स्वामीजी ने हिन्दुओं के सामाजिक जीवन से अनेक अन्धविश्वासों को दूर कर दिया; और सामाजिक लोकाचारों तथा उसी तरह के अन्य सामयिक विषयों के स्थान पर, हमारा ध्यान शाश्वत विषयों की ओर आकृष्ट किया। जातिवाद के वे घोर शत्रु थे। एक संन्यासी और एक सामान्य हिन्दू के रूप में भी छूआछूत से तो उन्हें विशेष घृणा थी। उन्होंने एक नया शब्द बनाया, जो भारतीय अंग्रेजी में प्रायः ही उपयोग में आता है - \enginline{‘don’t }\enginline{}\enginline{touchism’ } (मुझे-मत-छूओ-वाद)। उनका हृदय आम जनता के लिए प्रेम तथा सहानुभूति से लबालब भरा हुआ था और वे सभी जीवों में ईश्वर को देखनेवाले एक वेदान्ती के रूप में उन सभी की परम उत्साह के साथ सेवा करना चाहते थे। उन्होंने हमारे भारतीय भाषाओं के लिए एक नया शब्द बनाया - ‘दरिद्र-नारायण’ अर्थात् निर्धन तथा पीड़ित रूपी ईश्वर। यह शब्द अब सम्पूर्ण भारत में स्वीकृत हो चुका है और यह एक औसत आदमी के लिए एक सम्मान-सूचक भाव प्रकट करता है। समाज के निर्धन तथा पिछड़े, पीड़ित तथा हताश लोगों को ईश्वर के अंश या अवतार के रूप में देखना होगा; और उनकी सेवा ईश्वर की ही सेवा होगी। गुजरात के पुराने वैष्णव कवियों द्वारा उपयोग में लाए गए ‘हरिजन’ शब्द में बड़े अच्छे भाव थे और उसे महात्मा गाँधी ने पुनः लोकप्रिय बनाया, परन्तु ‘दरिद्र-नारायण’ शब्द में व्यक्ति के लिए एक कर्तव्य की भावना भी निहित है और वह यह है कि यदि वह ईश्वर की सेवा करना चाहता है, तो निर्धनों-पीड़ितों की सेवा करे। 

स्वामी विवेकानन्द को एक महान् धर्माचार्य के रूप में देखा जाता है। सचमुच ही एक ओर तो उन्होंने हिन्दू धर्म तथा दर्शन के अध्ययन के क्षेत्र में और दूसरी ओर इसके ज्ञान को प्रशंसित तथा प्रसारित करने में भी सुनिश्चित योगदान किया। सैद्धान्तिक तथा व्यावहारिक वेदान्त के विभिन्न पहलुओं पर रचित उनकी महान् कृतियाँ अब भी विश्व के हजारों जिज्ञासुओं को प्रेरणा प्रदान करती हैं। परन्तु साथ ही यह कहा जाता है कि वे एक धर्म-प्रचारक से भी अधिक एक ऐसे परोपकारी व्यक्ति थे, जिन्होंने अपना सम्पूर्ण जीवन मानव की सेवा में अर्पित कर दिया था। परन्तु स्वामीजी के व्यक्तित्व का इस प्रकार विश्लेषण करने की आवश्यकता नहीं है। उचित तो यह होगा कि मानव-सेवा को भी ईश्वर-सेवा का ही एक रूप मान लिया जाए, क्योंकि व्यावहारिक धर्म की दृष्टि से देखें, तो ईश्वर तथा मनुष्य एक ही सिक्के के दो पहलू हैं। दो दृष्टियों से स्वामीजी को नये विचारों का अग्रदूत कहा जा सकता है। एक तो, जैसा कि उनकी शिष्या भगिनी निवेदिता ने कहा है - वे ही ऐसे पहले व्यक्ति थे, जिन्होंने हिन्दू धर्म के मूलभूत लक्षणों को आधुनिक काल की दृष्टि से एक विचार-प्रणाली तथा एक जीवन-शैली के रूप में प्रस्तुत किया। यही वह पहली चीज है, जो हम भारतवासियों को स्वामीजी में उल्लेखनीय प्रतीत होती है। द्वितीयतः, स्वामीजी ने पाश्चात्य जगत् के समक्ष धार्मिक चिन्तन के क्षेत्र में - साम्प्रदायिक समस्याओं के सन्दर्भ में एक नया दृष्टिकोण प्रस्तुत किया, जिसकी उन्हें परम आवश्यकता थी। इसके साथ ही यह भी जोड़ा जा सकता है कि स्वामीजी ने वर्तमान भारत के एक प्रमुख विचारक के रूप में आधुनिक भारतीय संस्कृति को एक नयी आभा प्रदान की। उन्होंने समस्त मानवीय धर्मों तथा संस्कृतियों के बीच एकत्व स्थापित करने की परिकल्पना की थी, जो एक अखण्ड सत्ता के रूप में सबकी श्रद्धा का भाजन होगा। 

बहुत-से लोग मेरे इस विचार से सहमत होंगे कि १८९३ ई. में शिकागो में आयोजित अन्तर्राष्ट्रीय धर्म-महासभा में स्वामी विवेकानन्द का भाग लेना तथा अधिकारपूर्वक मधुर तथा युक्तिसंगत वक्तव्य देना, आधुनिक मानव के बौद्धिक इतिहास की एक अत्यन्त महत्त्वपूर्ण घटना है। वहाँ उन्होंने पहली बार एक नवीन तथा युक्तिसंगत धार्मिक समझ तथा सहिष्णुता की आवश्यकता की घोषणा की; और यह उस अमेरिका के लिए विशेष रूप से आवश्यक था, जो बड़ी तेजी से विज्ञान तथा प्रौद्योगिकी में, समृद्धि तथा शक्ति में आगे बढ़ रहा था; और जो उसकी सर्वहित की आकांक्षाओं तथा उपलब्धियों के विरोधी नहीं थे। परन्तु संयुक्त राज्य अमेरिका के न्यू-इंग्लैंड अंचल के कुछ अति महत्त्वपूर्ण व्यक्तियों को छोड़ दें, तो आम तौर पर वहाँ की धार्मिक पृष्टभूमि अनगढ़ तथा अविकसित थी। उन लोगों ने बाइबिल की शाब्दिक व्याख्या को पकड़ रखा था और समस्त मतवादों में इतनी दृढ़ता के साथ विश्वासी थे, जो अपनी सहजता तथा कट्टरता, और अन्धविश्वास तथा अनगढ़ता के सम्मिश्रण से बड़ी कारुणिक अवस्था को प्राप्त हो चुका था। यह अत्यन्त आदिम प्रकार का धर्म, उन लोगों को सन्तोषजनक नहीं लगा, जो उच्चतर तथा अधिक सभ्य स्तर पर जिज्ञासा की वृत्ति से प्रेरित थे; और उनके लिए विवेकानन्द का सन्देश मानो सूखी धरती पर वर्षा के समान था।... 

इस कारण हम कह सकते हैं कि अमेरिका से लेकर, सम्पूर्ण पश्चिमी जगत् के बहुत-से बुद्धिजीवी नर-नारियों के मन में एक बिल्कुल नये प्रकार का आध्यात्मिक रूपान्तरण आया है; और यहाँ हम देखते हैं कि वेदान्त का खमीर स्वामीजी के माध्यम से काम कर रहा है। डी. एच. लॉरेंस ने मैक्सिको के जीवन पर झ्त्ल्स् एीजहू (पंखोंवाला सर्प) नामक एक उपन्यास लिखा है, जिसमें मैक्सिको के कुछ राजनेताओं में, उन लोगों के कैथॅलिक ईसाई होने के पूर्व के वहाँ के मूलभूत अज्टेक नामक प्राचीन धर्म के पुनर्जागरण का चित्रण किया गया है। वहाँ इस आन्दोलन के कुछ नेताओं में व्यक्त होनेवाली मानसिकता अद्भुत रूप से आधुनिक है। इस उपन्यास का मुख्य पात्र रैमोन वहाँ के रोमन कैथॅलिक बिशप से जो बातें कह रहा है, वह पूरा-का-पूरा ही विवेकानन्द-साहित्य से लिया हुआ प्रतीत होता है। इस प्रकार यद्यपि सामान्य जनों के अज्ञात रूप से ही, स्वामीजी द्वारा १८९३ ई. में शिकागो में अमेरिका तथा पाश्चात्य जगत् को, और बाद में इंग्लैंड तथा भारत के लोगों को जो सन्देश दिया गया था, वह मानव-जाति के धार्मिक दृष्टिकोण को मुक्त करने में एक प्रभावी तत्त्व सिद्ध हुआ है। 

\vskip 2pt

स्वामीजी के विषय में मैंने जिस प्रथम बिन्दु का उल्लेख किया, वह था - विश्व के समक्ष हिन्दू धर्म को सार-रूप में परिभाषित करना। यह सेवा केवल भारत के लिए ही नहीं थी, अपितु एक अन्य रूप से पूरी मानव-जाति के लिए थी।... 

\vskip 2pt

समाज में जो लोग अन्याय का शिकार हुए थे, स्वामीजी उन लोगों के प्रति विशेष प्रेम था। उन लोगों में मानवीय स्वाभिमान का भाव प्रकट करने की उन्होंने जी-जान से चेष्टा की।... भारत की राजनीतिक पराधीनता तथा आध्यात्मिक शून्यता के उस काल में, जब सब कुछ निराशाजनक प्रतीत हो रहा था और आम जनता पूरी तौर से आत्मविश्वास खो चुकी थी, उस समय ‘कार्यक्षेत्र में उतर पड़ने का आह्वान’ लेकर ‘विवेकानन्द’ नामक एक शक्ति का आविर्भाव कैसे सम्भव हो सका था, यह निश्चय ही एक उल्लेखनीय घटना है। जब हम पूरी तौर से हताश हो चुके थे, हमारी सारी आशाएँ धूमिल हो चुकी थीं, ऐसे समय में ऐसे एक व्यक्ति का आगमन इस बात का द्योतक है कि कृपालु ईश्वर कभी अपने जनों को त्यागते नहीं और यह एक प्रकार से गीता के प्रायः ही उद्धृत होनेवाले श्लोक के महान् भाव का प्रतिफलन है - जब-जब धर्म का क्षय तथा अधर्म का प्राबल्य होता है, तब-तब ईश्वर स्वयं को एक महान् अवतार के रूप में प्रकट करके लोगों को नेतृत्व प्रदान करते हुए, मुक्ति के सच्चे मार्ग पर ले जाते हैं। इस अर्थ में विवेकानन्द एक अवतार थे; और न केवल भारत के, अपितु वर्तमान युग की सम्पूर्ण मानवता के लिए, एक दैवी प्रेरणा-सम्पन्न तथा ईश्वर-निर्दिष्ट पथ-प्रदर्शक थे।\endnote{ \engfoot{\textit{Swami Vivekananda Centenary Memorial Volume,} pp. 228-33}} 

\vskip 2pt


\section*{डॉ. ए. एल. बासम}

\addsectiontoTOC{डॉ. ए. एल. बासम}

परवर्ती काल में स्वामी विवेकानन्द के रूप में परिचित होनेवाले नरेन्द्रनाथ दत्त के जन्म के सौ साल बाद - आज भी विश्व-इतिहास की तुला पर उनके महत्त्व का आकलन कर पाना बड़ा कठिन है। पाश्चात्य तथा अधिकांश भारतीय इतिहासकारों के लिए यह कार्य, उनके देहान्त के समय जितना कठिन था, अब निश्चित रूप से यह और भी कठिन हो गया है; क्योंकि तब से अब तक के अन्तराल के वर्षों में होनेवाली अनेक विस्मयकारी एवं अप्रत्याशित घटनाओं से ऐसा संकेत मिलता है कि आनेवाली शताब्दियों में, विशेषकर जहाँ तक एशिया का सवाल है, वे आधुनिक विश्व को गढ़नेवाले लोगों में एक के रूप में याद किए जाएँगे; और भारतीय धर्म के सम्पूर्ण इतिहास में वे सर्वाधिक महत्त्वपूर्ण व्यक्तियों में से एक के रूप में गिने जाएँगे, जिनके महत्त्व की तुलना शंकर तथा रामानुज जैसे महान् आचार्यों से ही हो सकती है; परन्तु कबीर, चैतन्य तथा दक्षिण भारत के आलवार एवं नायनार आदि स्थानीय तथा आंचलिक सन्तों की तुलना में तो वे निःसन्देह अधिक महत्त्वपूर्ण~थे। 

\delimiter

मेरा यह भी विश्वास है कि विवेकानन्द विश्व-इतिहास में, अपने उस कार्य को प्रारम्भ करने के लिए भी सदा स्मरण किए जाएँगे, जिसे डॉ. सी. ई. एम. जोड ने ‘पूरब का प्रति-आक्रमण’ कहा था। लगभग हजार वर्ष पहले दक्षिण-पूर्व एशिया तथा चीन में बौद्ध तथा हिन्दू धर्मों का प्रचार करते हुए भ्रमण करनेवाले भारतीय धर्माचार्यों के बाद, वे ही एक ऐसे प्रथम धर्म-प्रचारक हुए, जिन्होंने भारत के बाहर अपने प्रभाव का विस्तार किया।\endnote{ \engfoot{\textit{Swami Vivekananda in East and West,} Ramakrishna Vedanta Centre, London, pp. 210-14}}


\section*{ई. पी. चेलीशेव}

\addsectiontoTOC{ई. पी. चेलीशेव}

विवेकानन्द के ग्रन्थों का बारम्बार पठन करने पर हर बार मुझे उसमें कुछ ऐसी नयी बात मिल जाती है, जिससे मुझे प्राचीन और अर्वाचीन भारत, उसके दर्शन, उसकी जीवन-पद्धति और वहाँ के लोगों के रीति-रिवाज एवं भावी आकांक्षाओं को और भी गहराई से समझने में सहायता मिलती है। विवेकानन्द के सन्देश में भारतीय संस्कृति के सर्वश्रेष्ठ तत्त्वों के साथ समन्वित होकर मानवतावाद के उदात्त आदर्शों का विकास हुआ है और मुझे लगता है कि यही उनकी महानतम देन हैं। 

समकालीन भारतीय साहित्य का अध्ययन करते समय कई बार मैंने पाया है कि विवेकानन्द के मानवतावादी आदर्शों ने अनेक लेखकों की कृतियों पर अत्यधिक प्रभाव विस्तार किया है। मेरी राय में, मनुष्य को पौरुषहीन मानकर ईश्वर से कृपा-याचना करने वाले ईसाई आदर्श के साथ विवेकानन्द के मानवतावाद का बिल्कुल भी साम्य नहीं है। धार्मिक आदर्शों को उन्होंने राष्ट्रीय हितों की रक्षा और अपने गुलाम देशवासियों की मुक्ति के काम में लगाने का प्रयास किया था। विवेकानन्द ने लिखा है कि साम्राज्यवादी लोग भारत में एक-पर-एक चर्च बनवाते जा रहे हैं, जबकि प्राच्य देशों में धर्म की नहीं, रोटी की जरूरत है; मैं सभी लोगों को अन्धविश्वासी मूर्ख की अपेक्षा सच्चा नास्तिक देखना कहीं अधिक पसन्द करूँगा। मनुष्य को ऊपर उठाने के लिए विवेकानन्द उन्हें ईश्वर से एकीभूत कर देते हैं।... 

मनुष्य के सार-तत्त्व की ऐसी मानवतावादी व्याख्या, विवेकानन्द के वैश्विक दृष्टिकोण के\break लोकतांत्रिक स्वरूप को सुनिश्चित करती है।... 

यद्यपि हम विवेकानन्द द्वारा प्रतिपादित मानवतावाद के आदर्शवादी आधार से सहमत नहीं है, तथापि हम स्वीकार करते हैं कि इसमें सक्रिय मानवतावाद के अनेक तत्त्व विद्यमान हैं, जो सर्वोपरि - मनुष्य को ऊपर उठाने की उनकी अदभ्य आकांक्षा, उसमें आत्मगौरव का भाव लाना, अपने तथा सबके भाग्य के बारे में उत्तरदायित्व महसूस कराना, अपने भले सच्चे तथा न्यायपूर्ण आदर्शों के लिए प्रयासी बनाना, किसी भी प्रकार के कष्ट के प्रति घृणा का भाव जगाना आदि रूपों अभिव्यक्त हुए हैं।... 

अनेक वर्ष बीत जाएँगे, अनेक पीढ़ियाँ आएँगी और चली जाएँगी, विवेकानन्द और उनका काल सुदूर अतीत की वस्तु हो जाएँगे, परन्तु उनकी याद कभी धुँधली नहीं होगी, जो अपने जीवन भर अपने देशवासियों के लिए बेहतर भविष्य का स्वप्न देखते रहे, जिन्होंने भारतवासियों को जगाने एवं भारत की प्रगति का पथ प्रशस्त करने के लिए इतना कुछ किया और अपने अति पीड़ित लोगों को अन्याय एवं अत्याचार से बचाने के लिए संघर्ष करते रहे। जिस प्रकार एक खड़ी चट्टान एक सागर-तटीय घाटी को तूफानी हवा के तेज झोकों, ऊँची लहरों और बुरे मौसम से रक्षा करती है; उसी प्रकार विवेकानन्द भी, साहस तथा निःस्वार्थ भाव के साथ अपनी मातृभूमि के शत्रुओं के साथ जूझते रहे। 

भारतीय जनता के समान ही, सोवियत रूस की जनता भी - जो रूस में प्रकाशित स्वामीजी के कुछ ग्रन्थों के माध्यम से उन्हें जानती है - इन महान् भारतीय देशभक्त, मानवतावादी, जनतांत्रिक और अपने देशवासियों तथा सम्पूर्ण मानवता के लिए संघर्ष करनेवाले योद्धा की पुण्य स्मृति को अत्यन्त श्रद्धा की दृष्टि से देखती है।\endnote{ \engfoot{\textit{Swami Vivekananda Centenary Memorial Volume,} pp. 506-18}}


\section*{हुआंग जिन चुआन}

\addsectiontoTOC{हुआंग जिन चुआन}

आज के चीन में विवेकानन्द ही भारत के सबसे प्रसिद्ध दार्शनिक तथा समाजसेवी माने जाते हैं। उनकी दार्शनिक तथा सामाजिक चिन्तनराशि और उद्दाम देशप्रेम ने, न केवल स्वदेश में राष्ट्रवादी आन्दोलन के विकास की प्रेरणा दी, अपितु विदेशों में भी महान् प्रभाव का विस्तार किया। १८९३ ई. में स्वामीजी ने कैंटन और उसके आसपास के क्षेत्रों का परिदर्शन किया था। उन्होंने मद्रासवासियों के नाम अपने एक पत्र में इस भ्रमण का विवरण दिया है। उन्हें चीन में इतिहास के बारे में थोड़ी-बहुत जानकारी और समझ भी थी। अपने लेखों एवं व्याख्यानों में उन्होंने बहुधा चीन के बारे में उच्च भाव व्यक्त किए हैं। उन्होंने एक भविष्यवाणी भी की थी कि चीनी संस्कृति एक दिन निश्चित रूप से ‘फोनिक्स पक्षी’ की भाँति पुनर्जीवित होगी और प्राच्य एवं पाश्चात्य संस्कृतियों के समन्वय का महान् उत्तरदायित्व ग्रहण करेगी। उनके जीवनीकार रोमाँ रोलाँ ने इस विषय में उनके विचारों के विकास का विवरण भी दिया है। विवेकानन्द जब पहली बार अमेरिका गए तो उन्हें आशा थी कि वह राष्ट्र मेरे लक्ष्य की प्राप्ति में सहायक होगा। परन्तु अपनी द्वितीय पाश्चात्य यात्रा के दौरान उन्हें बोध हुआ कि वे डालर-केन्द्रित साम्राज्यवाद के भुलावे में आ गए थे। अतः वे इस निष्कर्ष पर पहुँचे कि अमेरिका नहीं, बल्कि चीन ही मेरे इस कार्य के सम्पादन में यंत्र स्वरूप हो सकता है। 

सामंतवाद एवं साम्राज्यवाद की त्रासदी के बीच जी रहे चीनी लोगों के बारे में विवेकानन्द के मन में असीम सहानुभूति का भाव था। अपनी चीन-यात्रा के पश्चात् उन्होंने एक बड़ी रोचक टिप्पणी की। उन्होंने लिखा, “चीनी बच्चों को पूरा दार्शनिक ही समझो। जिस उम्र में भारतीय बच्चे घुटनों के बल भी नहीं चल पाते, उस उम्र में वह स्थिर भाव से चुपचाप काम पर जाता है। आवश्यकता का दर्शन उसने अच्छी तरह सीख और समझ लिया है।” इस उक्ति से यह स्पष्ट हो जाता है कि उस काल के चीनी शिशुओं के कष्ट के प्रति विवेकानन्द के हृदय में कितनी सहानुभूति का भाव था। 

अपनी कल्पना के समाजवाद की व्याख्या करते हुए विवेकानन्द ने एक उपयोगी सिद्धान्त प्रस्तुत किया है। उन्होंने कहा कि भावी समाज श्रमिक वर्ग द्वारा शासित होगा और यह घटना पहले चीन में होगी। अपनी ‘वर्तमान भारत’ पुस्तक में वे कहते हैं - “परन्तु फिर भी आशा है। काल के प्रभाव से ब्राह्मण आदि वर्ण भी शूद्रों का नीच स्थान प्राप्त कर रहे हैं।... और शूद्र जाति ऊँचा स्थान पा रही है।... महा-बलवान चीन हम लोगों के सामने ही बड़ी शीघ्रता से शूद्रत्व प्राप्त कर रहा है।... तो भी एक ऐसा समय आएगा, जब शूद्रत्वसहित शूद्रों का प्राधान्य होगा,... एक समय आएगा, जब प्रत्येक देश के शूद्र अपने-अपने समाज का पूर्ण आधिपत्य करेंगे।... समाजवाद, अराजकतावाद, नाशवाद आदि सम्प्रदाय आनेवाले सामाजिक विप्लव की ध्वजाएँ हैं।”... 

उनके जीवन तथा साहित्य से पूर्वोद्धृत सामग्री के आधार पर हम इतना तो देख ही सकते हैं कि विवेकानन्द ने साम्राज्यवाद तथा जागीरदारी व्यवस्था के तले दबी हुई चीनी जनता के प्रति बड़ी हमदर्दी तथा सहानुभूति दिखायी और उनके प्रति काफी आशा व्यक्त की। परन्तु हम भूपेन्द्र नाथ दत्त की इस बात से सहमत नहीं हैं कि चीनी तथा रूसी क्रान्तियों की सफलता के सुदृढ़ ऐतिहासिक क्षणों के रूपायन का श्रेय विवेकानन्द के ‘सन्देश’ को जाता है। यह तो उन्हें एक दिव्य अनुभूतिवादी व्यक्तित्व बना देगा। हमने देखा है कि सामाजिक विकास के नियमों से सम्बन्धित विवेकानन्द का दृष्टिकोण अवैज्ञानिक था। तथापि किसी भी उन्नत विचारक के लिए इतिहास की अग्रगति तथा उसके चरणों और घटनाओं के विषय में सही भविष्य-वाणी कर पाना सम्भव नहीं है। अतः विवेकानन्द का महत्त्व इस बात में है कि वे तथ्यों में से सत्य को निकाल पाने में सक्षम थे। 

निष्कर्ष के रूप में हम कह सकते हैं कि विवेकानन्द भारत के लोकतंत्रवादी देशभक्तों में सर्वप्रमुख थे। उन्होंने हमारी प्राचीन गरिमामय संस्कृति के प्रति अतीव सम्मान का भाव व्यक्त किया और चीन के पीड़ाग्रस्त लोगों से उन्हें प्रेम था। 

हम उन्हें श्रद्धांजलि अर्पित करते हैं।\endnote{ हुआंग जिन चुआन ने ४ जनवरी १९८० को कोलकाता के एशियाटिक सोसायटी में ‘विवेकानन्द और चीन’ विषय पर एक व्याख्यान दिया था। यह सामग्री उनके व्याख्यान के साइक्लोस्टाइल्ड सारांश से प्रस्तुत की गयी है, जो उस अवसर पर श्रोताओं के बीच वितरित की गयी थी। प्राध्यापक चुआन द्वारा इस सारांश की एक स्व-हस्ताक्षरित प्रति ७ जनवरी १९८० को रामकृष्ण मिशन इंस्टीट्यूट ऑफ कल्चर के सचिव स्वामी लोकेश्वरानन्द को भी भेंट-स्वरूप प्रदान की गयी थी।} 

\delimiter

\addtocontents{toc}{\protect\par\egroup}

