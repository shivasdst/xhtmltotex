
\chapter{ತಂದೆಯ ಮಾರ್ಗದರ್ಶನದಲ್ಲಿ}

\noindent

ನರೇಂದ್ರನಲ್ಲಿ ಧೈರ್ಯ, ಧೈರ್ಯದೊಂದಿಗೆ ಔದಾರ್ಯ, ಔದಾರ್ಯದೊಂದಿಗೆ ಕಾರುಣ್ಯ. ಇವುಗಳ ಜೊತೆಗೆ ಸಮಯಸ್ಫೂರ್ತಿ, ದಣಿವರಿಯದ ಶಕ್ತಿ, ಹಾಸ್ಯಪ್ರವೃತ್ತಿ, ಸಾಹಸಪ್ರವೃತ್ತಿ– ಇವೆಲ್ಲ ಅವನ ಬಾಲ್ಯ-ಕೌಮಾರ್ಯಗಳಿಂದಲೇ ಸ್ಪಷ್ಟವಾಗಿ ವ್ಯಕ್ತವಾಗುತ್ತಿದ್ದವು. ಇವುಗಳಿಂದಾ ಗಿಯೇ ಆತ ಯಾವಾಗಲೂ ತನ್ನ ಸಹಪಾಠಿಗಳ ಸಂಗಾತಿಗಳ ಮುಂದಾಳಾಗಿಯೇ ಇರುತ್ತಿದ್ದ.

ಆದರೆ ಬೆಳೆದು ದೊಡ್ಡವನಾದಂತೆಲ್ಲ ಅವನ ಸ್ವಭಾವದಲ್ಲಿ ಒಂದು ನಿಶ್ಚಿತವಾದ ಬದಲಾವಣೆ ಯಾಗುತ್ತಿರುವುದನ್ನು ಕಾಣಬಹುದಾಗಿತ್ತು. ಅವನಿಗೀಗ ಸುಮಾರು ಹದಿಮೂರು-ಹದಿನಾಲ್ಕು ವರ್ಷ. ಆಗಲೇ ಅವನಿಗೆ ಬೌದ್ಧಿಕ ವಿಷಯಗಳಲ್ಲಿ, ವಿಚಾರಪರ ವಿಷಯಗಳಲ್ಲಿ ವಿಶೇಷ ಒಲವು ಮೂಡಿಬರತೊಡಗಿತ್ತು. ಈಗ ಅವನಲ್ಲಿ ಒಳ್ಳೆಯ ಪುಸ್ತಕಗಳನ್ನು, ವೃತ್ತಪತ್ರಿಕೆಗಳನ್ನು ಓದುವ ಮತ್ತು ಸಾರ್ವಜನಿಕ ಉಪನ್ಯಾಸಗಳನ್ನು ಕೇಳುವ ಇಚ್ಛೆ ಉಂಟಾಗಿತ್ತು. ಅಲ್ಲದೆ ಆ ಪುಸ್ತಕಗಳ ಮತ್ತು ಉಪನ್ಯಾಸಗಳ ವಿಷಯಗಳನ್ನೆಲ್ಲ ನೆನಪಿನಲ್ಲಿಟ್ಟುಕೊಂಡು, ತನ್ನ ಸ್ನೇಹಿತರಿಗೆ ಅವುಗಳ ಸಾರಾಂಶವನ್ನು ವಿವರಿಸುತ್ತಿದ್ದ. ಜೊತೆಗೆ, ಅವನು ತನ್ನ ವಿಚಾರಶಕ್ತಿಯಿಂದ ಆ ವಿಷಯಗಳನ್ನೆಲ್ಲ ತುಲನೆ ಮಾಡಿ, ತನ್ನ ಸ್ವಂತ ವಿಮರ್ಶೆಯನ್ನು ಹೇಳುವುದನ್ನು ಕಂಡು ಅವನ ಸ್ನೇಹಿತರು ವಿಸ್ಮಯ ಭರಿತರಾಗುತ್ತಿದ್ದರು. ಈ ವೇಳೆಗೆ ಅವನಲ್ಲಿ ಚೆನ್ನಾಗಿ ವಾದ ಮಾಡುವ ಸಾಮರ್ಥ್ಯ ಕೂಡ ಬೆಳೆದುಬಂದಿತ್ತು. ಅವನಿಗೆ ಇದಿರಾಗಿ ವಾದಕ್ಕೆ ನಿಂತು ಗೆಲ್ಲಲು ಯಾರಿಂದಲೂ ಸಾಧ್ಯವಿರಲಿಲ್ಲ.

ಒಂದು ದಿನ, ಅವನ ಸ್ನೇಹಿತನೊಬ್ಬ ಒಳ್ಳೇ ಗವಾಯಿಗಳ ತರಹ ಹಾಡುತ್ತಿದ್ದ. ಅವನ ಹಾಡುಗಾರಿಕೆಯನ್ನು ಕೇಳಿದ ನರೇಂದ್ರ ಅದರ ಕುರಿತಾಗಿ, ತುಂಬ ಸ್ವಾರಸ್ಯಕರವಾಗಿ ತನ್ನ ವಿಮರ್ಶೆಯನ್ನು ಹೇಳುತ್ತಾನೆ: “ಕೇವಲ ಲಯಬದ್ಧವಾಗಿ, ಶ್ರುತಿಬದ್ಧವಾಗಿ ಹಾಡಿದ ಮಾತ್ರಕ್ಕೆ ಅದು ಸಂಗೀತವೆನಿಸಲಾರದು. ಸಂಗೀತವೆನ್ನುವುದು ಒಂದು ನಿರ್ದಿಷ್ಟವಾದ ಭಾವವನ್ನು ಹೊಮ್ಮಿ ಸುವಂತಿರಬೇಕು. ಹಾಡನ್ನು ಸುಮ್ಮನೆ ಎಳೆದೆಳೆದು ಹಾಡುತ್ತಿದ್ದರೆ, ಅದನ್ನು ಯಾರಾದರೂ ಮೆಚ್ಚಿಕೊಂಡಾರೇನು? ಹಾಡಿನ ಅರ್ಥ-ಭಾವಗಳು ಸ್ವಯಂ ಆ ಹಾಡುಗಾರನ ಭಾವವನ್ನು ಪ್ರಚೋದಿಸಿರಬೇಕು. ಹಾಡುಗಾರ ಪ್ರತಿಯೊಂದು ಶಬ್ದವನ್ನೂ ಸ್ಪಷ್ಟವಾಗಿ ಉಚ್ಚರಿಸಬೇಕು. ಮತ್ತು ಇವುಗಳ ಜೊತೆಗೆ, ತಾಳ ಹಾಗೂ ಶ್ರುತಿಯ ಕಡೆಗೂ ಗಮನಕೊಟ್ಟಿರಬೇಕು. ಹಾಡಿನಲ್ಲಿ ಶ್ರುತಿ-ಲಯ ಇವೆರಡೂ ಸರಿಯಾಗಿದ್ದುಬಿಟ್ಟ ಮಾತ್ರಕ್ಕೆ ಅದು ಸಂಗೀತವೆನಿಸಲಾರದು. ಗಾಯಕ ನಲ್ಲೇ ಹಾಡಿನ ಭಾವ ಸ್ಫುರಣೆಯಾಗದಿದ್ದಲ್ಲಿ ಅದು ಸಂಗೀತವೇ ಅಲ್ಲ.” ನರೇಂದ್ರ ಇನ್ನೂ ಹುಡುಗನಾದರೂ ಅವನ ವಿಮಾರ್ಶಾತ್ಮಕ ಬುದ್ಧಿ ಅದೆಷ್ಟು ತೀಕ್ಷ್ಣ!

ಹಿಂದೆಯೇ ನೋಡಿದಂತೆ, ವಿಶ್ವನಾಥದತ್ತ ತನ್ನ ವಕೀಲಿ ವೃತ್ತಿಯ ಸಂಬಂಧವಾಗಿ ಮಧ್ಯಭಾರತ ಹಾಗೂ ಉತ್ತರಭಾರತದ ಬೇರೆಬೇರೆ ಸ್ಥಳಗಳಿಗೆ ಹೋಗಬೇಕಾಗುತ್ತಿತ್ತು. ೧೮೭೭ ರಲ್ಲಿ, ಎಂದರೆ ನರೇಂದ್ರನ ಹದಿನಾಲ್ಕನೇ ವಯಸ್ಸಿನಲ್ಲಿ, ಅವನು ಮಧ್ಯ ಪ್ರದೇಶದ ರಾಯಪುರಕ್ಕೆ ಹೋಗಬೇಕಾಯಿತು. ಆದರೆ ಅಲ್ಲೇ ಬಹಳ ಕಾಲ ಇರಬೇಕಾಗಿತ್ತಾದ್ದರಿಂದ ತಾನು ಹೋದ ಮೇಲೆ ಅವನು ತನ್ನ ಹೆಂಡತಿ-ಮಕ್ಕಳನ್ನೂ ಅಲ್ಲಿಗೇ ಕರೆಯಿಸಿಕೊಳ್ಳಲು ವ್ಯವಸ್ಥೆಮಾಡಿದ. ಕಲ್ಕತ್ತದಿಂದ ರಾಯಪುರ ಬಹಳ ದೂರ. ಆಗಿನ್ನೂ ಅಲ್ಲಿಯವರೆಗೆ ರೈಲುದಾರಿ ಇರಲಿಲ್ಲ. ಆದ್ದರಿಂದ ಎತ್ತಿನಗಾಡಿಯಲ್ಲೇ ಪ್ರಯಾಣ ಸಾಗಬೇಕಾಗಿತ್ತು–ಹದಿನೈದು ದಿನಗಳ ಸುದೀರ್ಘ ಪ್ರಯಾಣ! ರಸ್ತೆಯಾದರೂ ಸುಗಮವಾದದ್ದೇನಲ್ಲ; ಕ್ರೂರ ಪ್ರಾಣಿಗಳಿಂದ ಕೂಡಿದ ಕಾಡುದಾರಿ. ಭುವನೇಶ್ವರೀ ದೇವಿ ಮಕ್ಕಳನ್ನೆಲ್ಲ ಕರೆದುಕೊಂಡು ರಾಯಪುರಕ್ಕೆ ಹೊರಟಳು. ನರೇಂದ್ರನೇ ಹಿರಿಯ ಗಂಡಾಳು, ಆದ್ದರಿಂದ ಅವನೇ ಅವರೆಲ್ಲರ ಮುಂದಾಳು. ಪ್ರಯಾಣದ ಜವಾ ಬ್ದಾರಿಯೆಲ್ಲ ಅವನದೇ. ದಾರಿಯಲ್ಲಿ ಅವನು ಬಗೆಬಗೆಯ ಕಷ್ಟಗಳಿಗೆ ಈಡಾದರೂ, ದಾರಿ ಯುದ್ದಕ್ಕೂ ಕಾಣಸಿಗುವ ನಯನಮನೋಹರ ವನರಾಜಿಗಳ ಸೌಂದರ್ಯವನ್ನು ನೋಡುತ್ತ ತನ್ನ ಕಷ್ಟಗಳನ್ನು ಮರೆಯುತ್ತಿದ್ದ. ಭಗವಂತ ತನ್ನ ಅಪಾರ ಶಕ್ತಿಯಿಂದ, ಅನುಪಮ ಪ್ರೇಮದಿಂದ, ಹೇಗೆ ಈ ಭೂದೇವಿಯನ್ನು ಅತ್ಯಪೂರ್ವ ವಸ್ತ್ರಾಭರಣಗಳಿಂದ ಅಲಂಕರಿಸಿ ಶೋಭೆಗೊಳಿಸಿದ್ದಾನೆ ಎನ್ನುವುದನ್ನು ಭಾವಿಸುತ್ತ ಅವನ ಹೃದಯ ಭಾವಪರವಶವಾಗಿಬಿಡುತ್ತಿತ್ತು. ಈ ಸಂದರ್ಭದಲ್ಲಿನ ಒಂದು ದಿವ್ಯ ಮಧುರ ಅನುಭವವನ್ನು ಮುಂದೆ ಅವನೇ ಬಣ್ಣಿಸುತ್ತಾನೆ:

“ನಾನು ಅಂದು ಆ ಕಾನನದ ದಾರಿಯಲ್ಲಿ ಹೋಗುತ್ತಿದ್ದಾಗ ಏನೇನು ಕಂಡೆನೋ, ಏನೇನು ಅನುಭವಿಸಿದೆನೋ ಅವೆಲ್ಲವೂ ನನ್ನ ಮನಸ್ಸಿನಲ್ಲಿ ಚಿರಮುದ್ರಿತವಾಗಿ ಉಳಿದುಕೊಂಡಿವೆ. ಒಂದು ದಿನ ನಾವು ವಿಂಧ್ಯಪರ್ವತದ ಕಣಿವೆಯೊಂದರ ಮೂಲಕ ಪ್ರಯಾಣ ಮಾಡುತ್ತಿದ್ದೆವು. ದಾರಿಯ ಅಕ್ಕಪಕ್ಕದಲ್ಲಿ ಉನ್ನತ ಪರ್ವತಶಿಖರಗಳು ಆಗಸದೆತ್ತರಕ್ಕೆ ನಿಮಿರಿ ನಿಂತಿದ್ದುವು. ಬಗೆಬಗೆಯ ಮರಗಿಡಬಳ್ಳಿಗಳು ಫಲಪುಷ್ಪಗಳ ಭಾರದಿಂದ ಬಾಗಿ ನಿಂತು ಪರ್ವತ ಶ್ರೇಣಿಯನ್ನು ಸೊಬಗು ಗೊಳಿಸಿದ್ದುವು. ಚಿತ್ರ ವಿಚಿತ್ರ ವರ್ಣದ ಪಕ್ಷಿಗಳು ಮರದಿಂದ ಮರಕ್ಕೆ ಹಾರಾಡುತ್ತ ತಮ್ಮ ಮಧುರ ಇಂಚರಗಳಿಂದ ತುಂಬಿಬಿಟ್ಟಿದ್ದುವು. ಇವುಗಳನ್ನೆಲ್ಲ ನೋಡುತ್ತ ನೋಡುತ್ತ ನನ್ನ ಮನಸ್ಸು ಒಂದು ಅಪೂರ್ವ ಶಾಂತಿಯನ್ನು ಅನುಭವಿಸಿತು. ಮೆಲ್ಲನೆ ಸಾಗುತ್ತಿದ್ದ ನಮ್ಮ ಎತ್ತಿನಗಾಡಿ ಈಗ, ರಸ್ತೆಯ ಮೇಲ್ಗಡೆ ಬಹುಎತ್ತರದಲ್ಲಿ ಎರಡು ಪರ್ವತ ಶಿಖರಗಳು ಕೂಡಿ ಕೊಳ್ಳುವಂತಿದ್ದ ಸ್ಥಳಕ್ಕೆ ಬಂದು ತಲುಪಿತು. ಶಿಖರಗಳು ಪರಸ್ಪರ ಪ್ರೀತಿಯಿಂದಲೋ ಎಂಬಂತೆ ಒಂದು ಇನ್ನೊಂದರತ್ತ ವಾಲಿಕೊಂಡಿರುವಂತೆ ಕಾಣುತ್ತಿತ್ತು. ಆ ಶೃಂಗಾಗ್ರಗಳ ಕೆಳಭಾಗವನ್ನು ಗಮನಿಸಿದೆ–ಒಂದು ಶಿಖರದಲ್ಲಿ ಮೇಲಿನಿಂದ ಕೆಳಗಿನವರೆಗೂ ಒಂದು ದೊಡ್ಡ ಬಿರುಕು; ಆ ಬಿರುಕನ್ನೇ ಮುಚ್ಚುವಂತಹ ಒಂದು ಬೃಹದಾಕಾರದ ಜೇನುಗೂಡು ಅಲ್ಲಿ ತೂಗಾಡುತ್ತಿದೆ. ಅಸಂಖ್ಯಾತ ಜೇನ್ನೊಣಗಳ ಅದೆಷ್ಟು ವರ್ಷಗಳ ಪರಿಶ್ರಮದ ಫಲವೋ ಆ ಜೇನುಗೂಡು! ಜೇನ್ನೊಣಗಳ ರಾಜ್ಯದ ಆದಿ-ಅಂತ್ಯಗಳ ಕುರಿತಾಗಿ ಯೋಚಿಸುತ್ತಿದ್ದಂತೆ ನನ್ನ ಮನಸ್ಸು ತ್ರಿಭುವನ ಗಳ ಒಡೆಯನಾದ ಭಗವಂತನ ಅದ್ಭುತ ಸೃಷ್ಟಿರಚನಾ ಕೌಶಲವನ್ನು ಭಾವಿಸುತ್ತ ಪರವಶ ವಾಯಿತು; ನಾನು ಬಾಹ್ಯಪ್ರಜ್ಞೆ ತಪ್ಪಿ ಹಾಗೆಯೇ ಕುಳಿತುಬಿಟ್ಟೆ. ಆ ಸ್ಥಿತಿಯಲ್ಲಿ ನಾನು ಎಷ್ಟು ಹೊತ್ತು ಇದ್ದೆನೋ ನನಗೆ ತಿಳಿಯದು. ನಾನು ಮೈತಿಳಿದೆದ್ದಾಗ ನಾವು ಆ ಜಾಗದಿಂದ ಎಷ್ಟೋ ದೂರ ಮುಂದಕ್ಕೆ ಹೊರಟುಹೋಗಿದ್ದೆವು.” ನರೇಂದ್ರ ಇಷ್ಟು ಗಾಢವಾಗಿ ಭಾವದಲ್ಲಿ ಮುಳುಗಿ ಮೈಮರೆತದ್ದು ಬಹುಶಃ ಇದೇ ಮೊದಲ ಸಲ.

ಆ ದಿನಗಳಲ್ಲಿ ರಾಯಪುರದಲ್ಲಿ ಶಾಲೆ ಇರಲಿಲ್ಲ; ಆದ್ದರಿಂದ ನರೇಂದ್ರನಿಗೆ ಇಲ್ಲಿ ಶಾಲಾವಿದ್ಯಾಭ್ಯಾಸಕ್ಕೆ ಅವಕಾಶವಾಗಲಿಲ್ಲ. ಆದರೆ ಅವನಿಗೆ ಬೌದ್ಧಿಕ ಸ್ತರದಲ್ಲಿ ತನ್ನ ತಂದೆ ಯೊಂದಿಗೆ ನಿಕಟ ಸಂಪರ್ಕವನ್ನು ಬೆಳೆಸಿಕೊಳ್ಳುವ ಸದವಕಾಶ ಲಭ್ಯವಾಯಿತು. ಕಲ್ಕತ್ತದಲ್ಲಿದ್ದಾಗ ಅವನು ತನ್ನ ಹೆಚ್ಚಿನ ಸಮಯವನ್ನೆಲ್ಲ ಶಾಲೆ ಅಧ್ಯಯನ ಸ್ನೇಹಿತರು ವ್ಯಾಯಾಮ ವಿಹಾರ ಇವುಗಳಲ್ಲೇ ಕಳೆಯುತ್ತಿದ್ದ. ಆದರೆ ಈಗ ರಾಯಪುರದಲ್ಲಿ ತಂದೆಗೂ ಸಾಕಷ್ಟು ಬಿಡುವಿದೆ; ಮಗನಿಗಂತೂ ಶಾಲೆಯೂ ಇಲ್ಲ, ಸ್ನೇಹಿತರೂ ಇಲ್ಲ. ಆದ್ದರಿಂದ ತಂದೆಯ ಆತ್ಮೀಯ ಸಂಪರ್ಕಕ್ಕೆ ಬರುವುದಕ್ಕೊಂದು ಸದವಕಾಶವಾಯಿತು. ಅಲ್ಲದೆ, ವಿಶ್ವನಾಥನೇನೂ ಸಾಮಾನ್ಯನಲ್ಲ. ಆತ ಉದಾತ್ತ ಚರಿತ; ಮತ್ತು ಹಲವಾರು ವಿಷಯಗಳಲ್ಲಿ ಅವನಿಗೆ ವಿಸ್ತಾರವಾದ ಪರಿಜ್ಞಾನ ವಿತ್ತು. ಈಗ ಅವನು ತಾರುಣ್ಯಕ್ಕೆ ಕಾಲಿಡುತ್ತಿದ್ದ ಮಗನ ಬುದ್ಧಿಯನ್ನು ಪ್ರಚೋದಿಸುವ, ವಿಚಾರ ಶಕ್ತಿಯನ್ನು ಸ್ಫುರಿಸುವ ಕಾರ್ಯದಲ್ಲಿ ತೊಡಗಿದ. ಮಗನೊಂದಿಗೆ ಅವನು ಹಲವಾರು ಉನ್ನತ ವಿಚಾರಗಳ ಕುರಿತಾಗಿ ದೀರ್ಘ ಸಂಭಾಷಣೆಗಳನ್ನು ನಡೆಸುತ್ತಿದ್ದ; ವಾದವಿವಾದಗಳನ್ನು ಮಾಡು ತ್ತಿದ್ದ. ಅವನು ಮಗನಿಗೆ ಸಂಪೂರ್ಣ ವಿಚಾರಸ್ವಾತಂತ್ರ್ಯವನ್ನಿತ್ತಿದ್ದ. ತನ್ನ ಅನಿಸಿಕೆ ಭಾವನೆ ಗಳನ್ನೆಲ್ಲಾ ನಿರಾತಂಕವಾಗಿ ಮಂಡಿಸುವ ಸ್ವಾತಂತ್ರ್ಯ ಕೊಟ್ಟಿದ್ದ. ಏಕೆಂದರೆ, ವಿದ್ಯಾಭ್ಯಾಸ ಕ್ರಮದಲ್ಲಿ ಸ್ವತಂತ್ರ ಆಲೋಚನೆಗೆ, ವಿಚಾರ ಸ್ಫುರಣೆಗೆ ಅವಕಾಶ ಇರಬೇಕು, ಒಬ್ಬರ ಆಲೋಚನೆ ಯನ್ನು ಇನ್ನೊಬ್ಬರ ಮೇಲೆ ಹೇರುವುದು ತರವಲ್ಲ ಎನ್ನುವುದು ಆತನ ನಿಶ್ಚಿತಾಭಿಪ್ರಾಯ. ತನ್ನ ತಂದೆಯಿಂದ ನರೇಂದ್ರ ಅತ್ಯಮೂಲ್ಯವಾದ ಹಲವಾರು ವಿಚಾರಗಳನ್ನು ಕಲಿತುಕೊಂಡ. ಉದಾ ಹರಣೆಗೆ, ಯಾವುದೇ ವಿಷಯದಲ್ಲಾದರೂ ಅದರ ಪ್ರಮುಖವಾದ ಅಂಶಗಳನ್ನು ಗ್ರಹಿಸುವ ಸಾಮರ್ಥ್ಯವನ್ನೂ, ಸತ್ಯಸಂಗತಿಯನ್ನು ವಿಶಾಲದೃಷ್ಟಿಯಿಂದ ಹಾಗೂ ಸಮಗ್ರದೃಷ್ಟಿಯಿಂದ ಕಾಣುವ ಶಕ್ತಿಯನ್ನೂ, ಯಾವುದಾದರೊಂದು ವಿಷಯದ ಕುರಿತಾಗಿ ಚರ್ಚೆ ನಡೆಸುವಾಗ ಅದರ ಮುಖ್ಯಾಂಶವನ್ನು ಬಿಟ್ಟು ಕದಲದಿರುವಂತಹ ದೃಢಚಿತ್ತತೆಯನ್ನೂ ನರೇಂದ್ರ ತನ್ನ ತಂದೆಯಿಂದ ಪಡೆದುಕೊಂಡ.

ರಾಯಪುರದಲ್ಲಿ ವಿಶ್ವನಾಥನ ಮನೆಗೆ ಎಷ್ಟೋ ಜನ ವಿದ್ವಾಂಸರು ಬರುತ್ತಿದ್ದರು. ಆಗ ಹಲವಾರು ವಿಷಯಗಳ ಮೇಲೆ ಚರ್ಚೆ ನಡೆಯುತ್ತಿತ್ತು. ನರೇಂದ್ರ ಬದಿಯಲ್ಲೇ ಕುಳಿತು ಆ ಚರ್ಚೆಗಳನ್ನು ಗಮನವಿಟ್ಟು ಕೇಳುತ್ತಿದ್ದ; ಕೆಲವೊಮ್ಮೆ ಅವನು ತನ್ನ ಸ್ವಂತ ಅಭಿಪ್ರಾಯವನ್ನೂ ವ್ಯಕ್ತಗೊಳಿಸುತ್ತಿದ್ದ. ಆಗ ಅವನ ಬುದ್ಧಿಶಕ್ತಿಯನ್ನು ಕಂಡ ಹಿರಿಯರು ಬಹಳ ಆಶ್ಚರ್ಯ ಪಡುತ್ತಿದ್ದರು. ಅಷ್ಟೇ ಅಲ್ಲ, ಬೌದ್ಧಿಕವಾಗಿ ಅವನನ್ನು ತಮ್ಮ ಸಮಾನಸ್ಕಂಧನೆಂಬಂತೆ ನಡೆಸಿ ಕೊಳ್ಳುತ್ತಿದ್ದರು. ಅದನ್ನು ಕಂಡಾಗ ವಿಶ್ವನಾಥನ ಹೃದಯ ಒಳಗೊಳಗೇ ಹಿಗ್ಗುತ್ತಿತ್ತು. ಒಮ್ಮೆ ನರೇಂದ್ರ, ಬಂಗಾಳೀ ಸಾಹಿತ್ಯದ ದೊಡ್ಡ ವಿದ್ವಾಂಸರೊಬ್ಬರೊಂದಿಗೆ ಮಾತನಾಡುತ್ತಿದ್ದ. ಆಗ ಅವನು ಸಾಂದರ್ಭಿಕವಾಗಿ ಅನೇಕ ಶ್ರೇಷ್ಠ ಬಂಗಾಳೀ ಗ್ರಂಥಗಳ ವಾಕ್ಯವಾಕ್ಯಗಳನ್ನೇ ಪುಟ ಪುಟಗಳನ್ನೇ ಉದ್ಧರಿಸಿ ಹೇಳಿಬಿಟ್ಟ! ಅದನ್ನು ಕಂಡು ಆ ಹಿರಿಯರು ಬೆಕ್ಕಸಬೆರಗಾಗಿ ಉದ್ಗರಿ ಸಿದರು, “ಮಗು, ಒಂದಲ್ಲ ಒಂದು ದಿನ ನಿನ್ನ ಕೀರ್ತಿಯನ್ನು ನಾವು ಕೇಳಲಿಕ್ಕಿದೆ.” ಆ ಭವಿಷ್ಯವಾಣಿ ಮುಂದೆ ನಿಜವಾಗುವುದನ್ನು ನಾವು ನೋಡಲಿದ್ದೇವೆ.*

ನರೇಂದ್ರನಿಗೆ ತನ್ನ ಬುದ್ಧಿಶಕ್ತಿಯ ಮೇಲೆ ವಿಶೇಷವಾದ ವಿಶ್ವಾಸ. ಇತರರೂ ತನ್ನ ಬುದ್ಧಿ ಮತ್ತೆಯನ್ನು ಗುರುತಿಸಿ, ತನ್ನನ್ನು ಅದಕ್ಕೆ ತಕ್ಕಂತೆ ನಡೆಸಿಕೊಳ್ಳಬೇಕೆಂಬುದು ಅವನ ನಿರೀಕ್ಷೆ. ಈ ವಿಷಯದಲ್ಲಿ ಅವನ ಆಕಾಂಕ್ಷೆ ನಿಜಕ್ಕೂ ಬಹಳ ದೊಡ್ಡದು. ತನ್ನ ಬುದ್ಧಿಶಕ್ತಿಗೆ ಸರಿಯಾದ ಮನ್ನಣೆ ಕೊಡದಿದ್ದರೆ ಆತ ಯಾರನ್ನೂ ಬಿಡುವವನಲ್ಲ. ಒಮ್ಮೆ ಅವನು ತನ್ನ ಬೌದ್ಧಿಕ ಪ್ರತಿಭೆಗೆ ತಕ್ಕ ಮನ್ನಣೆ ಕೊಡಲಿಲ್ಲವೆಂದು ತನ್ನ ತಂದೆಯ ಒಬ್ಬ ಸ್ನೇಹಿತರನ್ನೇ ತರಾಟೆಗೆ ತೆಗೆದು ಕೊಂಡುಬಿಟ್ಟಿದ್ದ! ಕೊನೆಗೂ ಅವರು ತಮ್ಮ ತಪ್ಪನ್ನು ಒಪ್ಪಿಕೊಳ್ಳುವವರೆಗೆ ಬಿಡಲೇ ಇಲ್ಲ. ಆದರೆ ಅವನು ಹೀಗೆ ಹಿರಿಯರ ಮೇಲೆಲ್ಲ ರೇಗಾಡುವುದನ್ನು ವಿಶ್ವನಾಥನಂತಹ ತಂದೆ ಸಹಿಸಿಯಾನೆ? ನರೇಂದ್ರ ಹಾಗೆ ನಡೆದುಕೊಂಡಾಗಲೆಲ್ಲ ಅವನು ಕಟುವಾಗಿ ಛೀಮಾರಿ ಹಾಕುತ್ತಿದ್ದ. ಆದರೂ ತನ್ನ ಮಗನ ಸ್ವಾಭಿಮಾನದ ಕಾವನ್ನು ಕಂಡು ಮನದೊಳಗೇ ಹೆಮ್ಮೆಪಟ್ಟು ಕೊಳ್ಳುತ್ತಿದ್ದ.

ನಿಜಕ್ಕೂ ರಾಯಪುರದಲ್ಲಿದ್ದ ಎರಡು ವರ್ಷಗಳ ಅವಧಿಯಲ್ಲಿ ನರೇಂದ್ರನ ಸ್ವಾಭಿಮಾನಪ್ರಜ್ಞೆ ವಿಶೇಷವಾಗಿ ಜಾಗೃತವಾಗಿಬಿಟ್ಟಿತ್ತು. ಕಲ್ಕತ್ತಕ್ಕೆ ಹಿಂದಿರುಗುವುದರೊಳಗೆ ಅವನು ಬೇರೆಯೇ ವ್ಯಕ್ತಿಯಾಗಿಬಿಟ್ಟಿದ್ದ. ಅವನ ಶರೀರರಚನೆಯೂ ಈಗ ಧೀರಗಂಭೀರವಾಗಿ ಕಾಣುತ್ತಿತ್ತು. ಮೊದಲಿ ನಿಂದಲೂ ಅವನ ಶರೀರ ಸರ್ವಾಂಗ ಸುಂದರವಾಗಿಯೇ ಇತ್ತು; ಆದರೆ ಈಗ ಅವನಿಗೆ ರಾಜಕುಮಾರನ ಕಳೆ ಬಂದುಬಿಟ್ಟಿದೆ. ಎಲ್ಲರ ಗಮನವನ್ನೂ ಸೆಳೆಯುವಂತಿತ್ತು ಅವನ ನಡಿಗೆಯ ಠೀವಿ. ಅಲ್ಲದೆ ಈಗ ಅವನು ತನ್ನ ಬುದ್ಧಿಮತ್ತೆಗೆ ಸರಿದೂಗುವಂಥವರನ್ನು ಮಾತ್ರ ಸ್ನೇಹಿತರನ್ನಾಗಿ ಸ್ವೀಕರಿಸಲಾರಂಭಿಸಿದ. ಆದರೆ ಅವನು ಅಹಂಕಾರಿಯಾಗಿಬಿಟ್ಟಿದ್ದನೆಂದು ಅರ್ಥವಲ್ಲ; ಅವನಲ್ಲಿ ರಕ್ತಗತವಾಗಿ ಬಂದಿದ್ದ ಔದಾರ್ಯ-ಸೌಜನ್ಯಗಳು ಹಾಗೆಯೇ ಇದ್ದುವು.

ರಾಯಪುರದಲ್ಲಿ ನರೇಂದ್ರ, ಪುರಾತನ ಭಾರತದ ಪ್ರಸಿದ್ಧ ಆಟವಾದ ಚದುರಂಗವನ್ನು ಕರಗತ ಮಾಡಿಕೊಂಡ. ಎಂತಹ ಕಷ್ಟದ ಪಂದ್ಯಗಳಲ್ಲೂ ಅವನು ಜಯಶಾಲಿಯಾಗುತ್ತಿದ್ದ. ಈ ದಿನ ಗಳಲ್ಲೇ ಅವನು ತನ್ನ ತಂದೆಯಿಂದ ಪಾಕಶಾಸ್ತ್ರದ ರೀತಿ-ರಹಸ್ಯಗಳನ್ನೂ ಕಲಿತು ಸ್ವಾಧೀನ ಪಡಿಸಿಕೊಂಡ. ಮುಂದೆ ಅವನು ಸ್ವಾಮಿ ವಿವೇಕಾನಂದರಾದ ಮೇಲೆ ಕೆಲವು ಪಾಶ್ಚಾತ್ಯ ಶಿಷ್ಯರಿಗೂ ಆ ಅಡಿಗೆಯ ರುಚಿಯನ್ನು ತೋರಿಸಿದ್ದುಂಟು–ಆದರೆ ಆಗಲೂ ಸ್ವಲ್ಪ ಖಾರ ಮುಂದು. ವಿವೇಕಾನಂದರು ಯಾವಾಗಲೂ ಸ್ವಲ್ಪ ಖಾರವೇ!

ಹಿಂದೆಯೇ ನೋಡಿದಂತೆ, ವಿಶ್ವನಾಥ ದತ್ತನಿಗೆ ಸಂಗೀತದಲ್ಲಿ ವಿಶೇಷ ಅಭಿರುಚಿ; ಸ್ವತಃ ಅವನೂ ಚೆನ್ನಾಗಿ ಹಾಡಬಲ್ಲವನಾಗಿದ್ದ. ಸಹಜವಾಗಿಯೇ ಅವನ ಮನೆಯಲ್ಲಿ ಸಂಗೀತದ ಒಂದು ವಿಶೇಷ ವಾತಾವರಣ ಏರ್ಪಟ್ಟಿತ್ತು. ನರೇಂದ್ರನ ಸಂಗೀತಪ್ರೇಮವನ್ನೂ ಗಾಯನ ಪ್ರತಿಭೆ ಗಳನ್ನೂ ಅವನ ಬಾಲ್ಯದಲ್ಲೇ ವಿಶ್ವನಾಥ ಗುರುತಿಸಿದ್ದ. ನುರಿತ ಸಂಗೀತಗಾರರಿಂದ ಶಾಸ್ತ್ರೀಯ ವಾಗಿ ಶಿಕ್ಷಣ ಪಡೆದ ಹೊರತು ಯಾರೂ ಆ ಕಲೆಯಲ್ಲಿ ಪ್ರವೀಣರಾಗಲು ಸಾಧ್ಯವಿಲ್ಲ ಎಂದೂ ಅವನು ಅರಿತಿದ್ದ. ಅವನೇ ಹಿಂದೆ ನರೇಂದ್ರನಿಗೆ ಸಂಗೀತದ ಪ್ರಾಥಮಿಕ ಪಾಠಗಳನ್ನು ಕಲಿಸಿದ್ದ. ಅವನು ಮಗನಿಗೆ ನಾನಾ ಬಗೆಯ ಹಾಡುಗಳನ್ನು ಹೇಳಿಕೊಟ್ಟ. ಮುಂದೆ ಕಲ್ಕತ್ತಕ್ಕೆ ಹಿಂದಿರುಗಿದ ಮೇಲೆ, ಆಗಿನ ಪ್ರಸಿದ್ಧ ಸಂಗೀತಗಾರರಾದ ವೇಣೀ ವಸ್ತಾದ, ಬಳಿಕ ಈತನ ಗುರು ಅಹಮದ್ ಖಾನ್​–ಇವರಿಂದಲೂ ಶಾಸ್ತ್ರೀಯ ಸಂಗೀತವನ್ನು ಕಲಿಸಿದ. ಅಹಮದ್ ಖಾನನಿಂದ ನರೇಂದ್ರ ಹಲವಾರು ಹಿಂದಿ, ಉರ್ದು ಹಾಗೂ ಪಾರಸೀ ಹಾಡುಗಳನ್ನು ಕಲಿತುಕೊಂಡ. ಇವುಗಳಲ್ಲಿ ಹೆಚ್ಚಿನವು ಭಕ್ತಿಗೀತೆಗಳು. ಅಲ್ಲದೆ, ಉಜಿರ್ ಖಾನ್, ದುನ್ನಿಖಾನ್, ಕನ್ನೈಲಾಲ್ ಧೇಂಡಿ, ಜಗನ್ನಾಥ ಮಿಶ್ರ ಎಂಬ ವಿದ್ವಾಂಸರಿಂದಲೂ ಅವನಿಗೆ ಬೇರೆಬೇರೆ ಸಮಯಗಳಲ್ಲಿ ಸಂಗೀತ ಪಾಠವಾಗಿತ್ತು. ಹಾಡುಗಾರಿಕೆಯಲ್ಲಲ್ಲದೆ ಅವನು ವಾದ್ಯ ಸಂಗೀತದಲ್ಲೂ ಪ್ರವೀಣನಾಗಿದ್ದ. ಪಖವಾಜ್​-ತಬಲಾಗಳನ್ನೂ, ಸಿತಾರ್ ಮತ್ತು ಎಸ್ರಾಜ್ ಎನ್ನುವ ತಂತಿವಾದ್ಯಗಳನ್ನೂ ನುಡಿಸ ಬಲ್ಲವನಾಗಿದ್ದ. ಆದರೆ ಅವನ ವಿಶೇಷ ಒಲವೂ ಬಲವೂ ಹಾಡುಗಾರಿಕೆಯಲ್ಲೇ. ಅದರಲ್ಲಿ ಅವನು ತನ್ನ ಗುರುಗಳನ್ನೇ ಮೀರಿಸಿಬಿಟ್ಟ. ದೈವದತ್ತ ಕಂಠಮಾಧುರ್ಯವಿದ್ದ ಅವನನ್ನು ಆ ಗುರುಗಳು ಶ್ರೇಷ್ಠ ಮಟ್ಟದ ಗಾಯಕನನ್ನಾಗಿ ರೂಪಿಸಿದರು. ಅವನು ಅಷ್ಟೇ ಶ್ರದ್ಧೆಯಿಂದ ಸಂಗೀತವನ್ನು ಅಭ್ಯಾಸ ಮಾಡಿದ್ದ. ಗಂಟೆಗಟ್ಟಲೆ ಕಾಲ ಮೈಮರೆತು ಅಭ್ಯಾಸ ಮಾಡುತ್ತಿದ್ದ. ಅದನ್ನು ಕೇಳುವುದಕ್ಕೆಂದೇ ಅವನ ಸ್ನೇಹಿತರು ಅಲ್ಲಿ ಬಂದು ಕುಳಿತಿರುತ್ತಿದ್ದರು. ಇತರರ ವಿಷಯ ಹಾಗಿರಲಿ, ಅವನ ಮನೆಮಂದಿಗೂ ಆ ಕಂಠಮಾಧುರ್ಯವನ್ನು ಸವಿಯಲು ಅದೆಷ್ಟು ಆಸೆ! ಕೆಲವೊಮ್ಮೆ ಅವನು ತನ್ನ ತಂದೆಯ ಸಂತೋಷಕ್ಕಾಗಿ ಹಾಡುತ್ತಿದ್ದ–ಒಮ್ಮೊಮ್ಮೆ ಗಂಭೀರಭಾವ ದಿಂದ; ಮತ್ತೆ ಕೆಲವೊಮ್ಮೆ ಉಲ್ಲಾಸಭಾವದಿಂದ. ಬಾಲಕ ನರೇಂದ್ರ ಹಲವಾರು ವಿಷಯಗಳಲ್ಲಿ ಹಲವಾರು ವಿದ್ಯೆಗಳಲ್ಲಿ ಪ್ರವೀಣನಾಗಿದ್ದರೂ ಅವುಗಳಲ್ಲೆಲ್ಲ ಹಾಡುಗಾರಿಕೆಯೇ ಅತ್ಯಂತ ಪ್ರಮುಖವಾದದ್ದು ಎನ್ನಬಹುದು. ಏಕೆಂದರೆ ಅವನ ಗಾಯನಪ್ರತಿಭೆಯೆನ್ನುವುದು ಇತರೆಲ್ಲ ಪ್ರತಿಭೆಗಳನ್ನೂ ಮೀರಿ ಎದ್ದುಕಾಣುತ್ತಿತ್ತು.

ನಿಜಕ್ಕೂ ಆತನ ಕಂಠ ಎಷ್ಟು ಮಧುರವಾಗಿತ್ತು, ಎಷ್ಟು ಚೇತೋಹಾರಿಯಾಗಿತ್ತು ಎಂದರೆ, ಅವನು ಹಾಡೊಂದನ್ನು ಹಾಡತೊಡಗಿದನೆಂದರೆ ಆ ರಾಗದ ಅಧಿದೇವತೆಯೇ ಅಲ್ಲಿ ಮೈದಳೆದು ನಿಂತಂತೆ ಭಾಸವಾಗುತ್ತಿತ್ತು. ಅವನಿಗೆ ಭಗವಂತನನ್ನು ಆರಾಧಿಸಿ ಒಲಿಸಿಕೊಳ್ಳಲು ಸಂಗೀತವೇ ಒಂದು ಪ್ರಧಾನ ಸಾಧನವಾಯಿತು ಎನ್ನಬಹುದು. ಮುಂದೆ ದಿವ್ಯಗುರು ಶ್ರೀರಾಮಕೃಷ್ಣರೊಂದಿಗೆ ಆತನ ಪ್ರಥಮ ಸಮಾಗಮವಾದದ್ದು ಸಂಗೀತದ ಮೂಲಕವೇ.

ಸಂಗೀತದ ಜೊತೆಗೆ ನರೇಂದ್ರ ನೃತ್ಯಕಲೆಯನ್ನೂ ಅಭ್ಯಸಿಸಿದ. ಭಾರತೀಯರಾದ ನಮಗೆ ಈ ನರ್ತನಕಲೆ ಪ್ರಾಪ್ತವಾದದ್ದು ಬ್ರಹ್ಮಾದಿ ದೇವತೆಗಳಿಂದಲೇ. ಶ್ರೀಕೃಷ್ಣನೂ ನಾಟ್ಯನಿಪುಣ. ಶಿವನ ಹೆಸರೇ ನಟರಾಜ–ನಾಟ್ಯವಾಡುವವರೆಲ್ಲರ ರಾಜ! ಇಂತಹ ಈ ದೈವಿಕ ನೃತ್ಯವನ್ನು ನರೇಂದ್ರ ಕಲಿತುಕೊಂಡ. ಅಲ್ಲದೆ, ನರ್ತಿಸಿ ತೋರಿಸುತ್ತಲೂ ಇದ್ದ. ಅದನ್ನು ಕಂಡವರು ವಿಶೇಷ ಆನಂದವನ್ನನುಭವಿಸುತ್ತಿದ್ದರು. ತನ್ನ ದೃಢವಾದ ಸುಪುಷ್ಟವಾದ ಸುಂದರ ಶರೀರ, ಮತ್ತು ಅದಕ್ಕಿಂತ ಹೆಚ್ಚಾಗಿ ತನ್ನಲ್ಲಿ ಆಳವಾಗಿ ಬೆಳೆದು ಬಂದ ಕಲಾತ್ಮಕ ಮನೋಭಾವ–ಇವುಗಳಿಂದ ಕೂಡಿ ಅವನು ನರ್ತನ ಮಾಡುತ್ತಿದ್ದರೆ ಪ್ರೇಕ್ಷಕರು ತಲ್ಲೀನರಾಗಿ ನೋಡುತ್ತಿದ್ದರು. ಅವನು ತನ್ನ ಜೇನುಕಂಠದಿಂದ ಹಾಡುತ್ತ ಆ ಹಾಡಿನ ಭಾವಕ್ಕೆ ತಕ್ಕಂತೆ ತನ್ನ ದುಂಡುಗಿನ ಬಾಹುಗಳನ್ನು, ಶರೀರದ ಅಂಗಾಂಗಗಳನ್ನು ಚಲಿಸುತ್ತ ನೃತ್ಯ ಮಾಡುತ್ತಿದ್ದರೆ ಒಂದು ದಿವ್ಯ ಸೌಂದರ್ಯ ಅಲ್ಲಿ ಹೊರಸೂಸುತ್ತಿತ್ತು. ಅದನ್ನು ನೋಡುತ್ತಿದ್ದ ಪ್ರೇಕ್ಷಕರ ಕಣ್ಣುಗಳಿಗೆ, ಅವನ ಗಾನವನ್ನು ಕೇಳು ತ್ತಿದ್ದ ಕಿವಿಗಳಿಗೆ ದಿವ್ಯಾನುಭವದ ರಸದೌತಣವಾಗುತ್ತಿತ್ತು.

ಮಗ ನರೇಂದ್ರನನ್ನು ವಿಶ್ವನಾಥ ಬಗೆಬಗೆಯ ವಿದ್ಯೆಗಳಲ್ಲಿ ಪರಿಣತನನ್ನಾಗಿ ಮಾಡಿದ್ದಲ್ಲದೆ ಧೀರತೆಯ ಪಾಠವನ್ನೂ ಹೇಳಿಕೊಟ್ಟ.

ತಂದೆಯ ಅತಿಧಾರಾಳತನವನ್ನು ನೋಡುತ್ತಿದ್ದ ನರೇಂದ್ರ ಒಂದು ದಿನ ನೇರವಾಗಿ ತಂದೆ ಯನ್ನು ಕೇಳಿದ, “ಅಪ್ಪಾ, ನನಗಾಗಿ ಏನು ಮಾಡಿದ್ದೀ ನೀನು?” ಮರುಕ್ಷಣದಲ್ಲೇ ಚಿಮ್ಮಿತು ಉತ್ತರ–“ಹೋಗು, ಆ ಕನ್ನಡಿಯ ಮುಂದೆ ನಿಂತು ನೋಡಿಕೊ.”

ಏನರ್ಥ ಮಗನ ಆ ಪ್ರಶ್ನೆಗೆ!....

ಏನರ್ಥ ತಂದೆಯ ಈ ಉತ್ತರಕ್ಕೆ!....

ನಾವು ನೋಡಿದಂತೆ ವಿಶ್ವನಾಥ ತನ್ನ ಸಾಮರ್ಥ್ಯದಿಂದ ಬೇಕೆಂಬಷ್ಟು ಹಣವನ್ನು ಗಳಿಸುತ್ತಿದ್ದ. ಆದರೆ ಅವನು ಅತಿ ಧಾರಾಳಿ; ತನ್ನ ಸೋಮಾರಿ ಬಂಧುಗಳಿಗೆ ಮದ್ಯಪಾನ ಮಾಡಲು ಹಣ ಕೊಡುವಷ್ಟು ಉದಾರಿ! ಅವನ ಇಂತಹ ಅತಿ ಔದಾರ್ಯವನ್ನು ಕಂಡೇ ನರೇಂದ್ರ ಈಗ ‘ಅಪ್ಪಾ ನನಗಾಗಿ ಏನು ಮಾಡಿದ್ದೀ ನೀನು?’ ಎಂದು ಕೇಳುತ್ತಿದ್ದಾನೆ. ಆದರೆ ವಿಶ್ವನಾಥನಿಗೆ ತನ್ನ ಮಕ್ಕಳಿ ಗಾಗಿ ಹಣ ಕೂಡಿಡುವುದರಲ್ಲಿ ನಂಬಿಕೆಯಿಲ್ಲ. ಅವನ ಪ್ರಕಾರ, ತಂದೆಯಾದವನು ಮಕ್ಕಳನ್ನು ಗುಣವಂತರನ್ನಾಗಿ ಮಾಡಬೇಕೇ ಹೊರತು ಧನವಂತರನ್ನಾಗಲ್ಲ. ಆದ್ದರಿಂದಲೇ ಮಗ ಆ ಪ್ರಶ್ನೆ ಯನ್ನು ಕೇಳಿದಾಗ ತಕ್ಷಣ ಹೇಳುತ್ತಿದ್ದಾನೆ: ‘ಹೋಗು, ಆ ಕನ್ನಡಿಯ ಮುಂದೆ ನಿಂತು ನೋಡಿಕೊ.’ ಕನ್ನಡಿಯ ಮುಂದೆ ನಿಂತು ನೋಡಿಕೊಂಡರೆ ಏನು ಕಾಣುತ್ತದೆ–ಶರೀರ, ಅಷ್ಟೆ! ಆದರೆ ಅಷ್ಟಲ್ಲದೆ ಮತ್ತಿನ್ನೇನು ಬೇಕು? ನರೇಂದ್ರ ಕನ್ನಡಿಯ ಮುಂದೆ ನಿಂತು ನೋಡಿಕೊಂಡರೆ ಏನು ಕಾಣುತ್ತದೆ?–ತೇಜಃಪುಂಜವಾದ ಕಣ್ಣುಗಳು, ಓಜೋಮಯವಾದ ಶರೀರ! ಅವನ ಕಣ್ಣುಗಳ ಮೂಲಕ ಹೊರಸೂಸುತ್ತಿದೆ!–ಅವನು ಕಲಿತ ವಿದ್ಯೆ, ಸೌಜನ್ಯಾದಿ ಸದ್ಗುಣಗಳು. ಅವನ ಶರೀರದಿಂದ ಚಿಮ್ಮುತ್ತಿದೆ–ಓಜಸ್ಸು ತೇಜಸ್ಸು ಶಕ್ತಿ ಶೌರ್ಯ! ಇಷ್ಟಿದ್ದರೆ ಸಾಲದೆ? ಇಷ್ಟಿದ್ದು ಬಿಟ್ಟರೆ ಇನ್ನೇನು ಬೇಕು? ಇನ್ನೇನೇನು ಬೇಕೋ ಅವನ್ನೆಲ್ಲ ಸ್ವತಃ ಸಂಪಾದಿಸಿಕೊಳ್ಳಲು ಇಷ್ಟಿದ್ದರೆ ಸಾಲದೆ? ನರೇಂದ್ರನ ಸೂಕ್ಷಬುದ್ಧಿಗೆ ತಂದೆಯ ಇಂಗಿತ ಅರ್ಥವಾಯಿತು; ಸುಮ್ಮನಾದ.

ಇನ್ನೊಂದು ದಿನ ಅವನು ತನ್ನ ತಂದೆಯನ್ನು ಕೇಳುತ್ತಾನೆ, “ಅಪ್ಪ ಉತ್ತಮ ನಡವಳಿಕೆಯ ಲಕ್ಷಣವೇನು?” ಇದಕ್ಕೆ ತಂದೆ ವಿಶ್ವನಾಥ ಉತ್ತರಿಸುತ್ತಾನೆ, “ಯಾವುದಕ್ಕೂ ಆಶ್ಚರ್ಯವನ್ನು ತೋರಿಸಬೇಡ.” ನರೇಂದ್ರ ಕೇಳಿದ ಪ್ರಶ್ನೆಯೂ ಅಪೂರ್ವವಾದದ್ದು. ವಿಶ್ವನಾಥ ನೀಡಿದ ಉತ್ತರವೂ ಅಪೂರ್ವವಾದದ್ದು. ಈ ಪುಟ್ಟ ಹಿತ ನುಡಿಯ ಕುರಿತಾಗಿ ಆಲೋಚನೆ ಮಾಡಿದಷ್ಟೂ ಅದರ ಮಹತ್ವ ಚೆನ್ನಾಗಿ ಮನವರಿಕೆಯಾಗುತ್ತ ಬರುತ್ತದೆ. ಈ ಮಾತನ್ನು ನರೇಂದ್ರ ಆಳವಾಗಿ ಅರ್ಥಮಾಡಿಕೊಂಡು ಅರಗಿಸಿಕೊಂಡದ್ದರಿಂದಲೇ ಅವನು ಮುಂದೆ ಸ್ವಾಮಿ ವಿವೇಕಾನಂದರಾಗಿ ಜಗತ್ತಿನಾದ್ಯಂತ ಸಂಚರಿಸುವಾಗ ಹಲವಾರು ರಾಜಮಹಾರಾಜರುಗಳನ್ನೂ ದೇಶವಿದೇಶಗಳ ಸುಪ್ರಸಿದ್ಧ ವ್ಯಕ್ತಿಗಳನ್ನೂ ಕೋಟ್ಯಧೀಶರನ್ನೂ ಸಂಧಿಸಿದಾಗಲೂ ಬೆರಗಾಗದೆ, ಬೆಚ್ಚಿ ಬೀಳದೆ ಅವರೊಂದಿಗೆಲ್ಲ ಸಮಾನ ಗಾಂಭೀರ್ಯದಿಂದ ನಡೆದುಕೊಳ್ಳಲು ಸಾಧ್ಯವಾಯಿತು ಎನ್ನಬಹುದು. ಅಂತೂ ನರೇಂದ್ರ ತನ್ನ ತಂದೆಯಿಂದ ಕಲಿತ ಪಾಠಗಳೆಲ್ಲವೂ ರತ್ನದಂಥವು.

ವಿಶ್ವನಾಥ ರಾಯಪುರದಲ್ಲಿ ಎರಡು ವರ್ಷಗಳ ಕಾಲ ಇದ್ದು ೧೮೭೯ ರಲ್ಲಿ ಸಂಸಾರ ಸಮೇತವಾಗಿ ಕಲ್ಕತ್ತಕ್ಕೆ ಮರಳಿದ. ಈಗ ನರೇಂದ್ರನನ್ನು ಪುನಃ ಶಾಲೆಗೆ ಸೇರಿಸುವುದು ಸ್ವಲ್ಪ ಕಷ್ಟವಾಯಿತು. ಏಕೆಂದರೆ ಅಷ್ಟು ದೀರ್ಘಕಾಲ ಆತ ಗೈರುಹಾಜರಾಗಿದ್ದಾನೆ. ಆದರೆ ಆ ಶಾಲೆಯ ಅಧ್ಯಾಪಕರೆಲ್ಲ ಅವನ ಬುದ್ಧಿಶಕ್ತಿ-ಸಾಮರ್ಥ್ಯಗಳನ್ನು ಚೆನ್ನಾಗಿ ಕಂಡುಕೊಂಡಿದ್ದರು. ಅಲ್ಲದೆ ಅವನು ಅವರ ವಿಶೇಷ ಪ್ರೀತಿವಿಶ್ವಾಸಗಳಿಗೆ ಪಾತ್ರನಾಗಿದ್ದ. ಆದ್ದರಿಂದ ಅವನ ವಿಷಯದಲ್ಲಿ ವಿಶೇಷ ರಿಯಾಯಿತಿ ತೋರಿಸಿ ಮತ್ತೆ ಶಾಲೆಗೆ ಸೇರಿಸಿಕೊಂಡರು. ಅವನ ಹಿಂದಿನ ಸಹಪಾಠಿಗಳೆಲ್ಲ ಈಗ ಎಂಟ್ರೆನ್ಸ್ ತರಗತಿಯಲ್ಲಿದ್ದರು. ಅವನೂ ಈ ವರ್ಷ ಎಂಟ್ರೆನ್ಸ್ ಪರೀಕ್ಷೆಗೆ ಕುಳಿತುಕೊಳ್ಳ ಬೇಕಾದರೆ ಮೂರು ವರ್ಷಗಳ ಪಾಠಗಳನ್ನು ಒಟ್ಟಿಗೆ ಅಭ್ಯಾಸ ಮಾಡಬೇಕಾಗಿತ್ತು. ಅವನೇನೋ ಇದಕ್ಕೆ ಸಿದ್ಧನಾಗಿಯೇ ಇದ್ದ. ಅದನ್ನು ತಾನು ಸಾಧಿಸಬಲ್ಲೆನೆಂಬ ವಿಶ್ವಾಸ ಅವನಿಗಿತ್ತು.

ನರೇಂದ್ರ ಈಗ ಮತ್ತೊಮ್ಮೆ ಮೆಟ್ರೊಪಾಲಿಟನ್ ಶಾಲೆಯ ವಿದ್ಯಾರ್ಥಿಯಾಗಿದ್ದಾನೆ. ಆ ವರ್ಷ ಶಾಲೆಯ ಹಿರಿಯ ಉಪಾಧ್ಯಾಯರೊಬ್ಬರು ನಿವೃತ್ತಿ ಹೊಂದಲಿದ್ದರು. ಅವರನ್ನು ಸೂಕ್ತ ರೀತಿ ಯಲ್ಲಿ ಗೌರವಿಸಿ ಬೀಳ್ಗೊಡಲು ಶಾಲೆಯಲ್ಲಿ ಸಮಾರಂಭವೊಂದನ್ನು ಏರ್ಪಡಿಸಲಾಯಿತು. ಆ ಕಾಲದ ಅಗ್ರಗಣ್ಯ ರಾಷ್ಟ್ರೀಯ ಧುರೀಣರೂ ಶ್ರೇಷ್ಠ ವಾಗ್ಮಿಗಳೂ ಆದ ಸುರೇಂದ್ರನಾಥ ಬ್ಯಾನರ್ಜಿಯವರನ್ನು ಕಾರ್ಯಕ್ರಮದ ಅಧ್ಯಕ್ಷರನ್ನಾಗಿ ಆಹ್ವಾನಿಸಲಾಯಿತು. ಅಂತಹ ವಿಖ್ಯಾತ ರೊಬ್ಬರು ಬರಲು ಒಪ್ಪಿದ್ದು ದೊಡ್ಡ ವಿಷಯವೇ ಸರಿ. ಆದರೆ ಕಾರ್ಯಕ್ರಮದಲ್ಲಿ ನಿವೃತ್ತ ಅಧ್ಯಾಪಕರ ಗುಣಗಾನ ರೂಪವಾದ ಭಾಷಣವನ್ನು ಹುಡುಗರಲ್ಲೇ ಯಾರಾದರೂ ಮಾಡಬೇಕಲ್ಲ! ಅಂಥ ಸುಪ್ರಸಿದ್ಧ ವಾಗ್ಮಿಯ ಮುಂದೆ ನಿಂತು ಮಾತನಾಡಬೇಕು ಎನ್ನುವುದನ್ನು ನೆನೆಸಿಕೊಂಡು ಹುಡುಗರೆಲ್ಲ ದಿಗಿಲುಬಿದ್ದರು. ಒಬ್ಬನೂ ಮಾತನಾಡಲು ಮುಂದಾಗಲಿಲ್ಲ. ಆಗ ಹುಡುಗರೆಲ್ಲ ನರೇಂದ್ರನ ಬಳಿಗೆ ಬಂದು, “ನರೇನ್, ಈ ಕೆಲಸಕ್ಕೆ ನೀನೇ ಸರಿ ಕಣೊ” ಎಂದರು. ನರೇಂದ್ರ ಒಪ್ಪಿಕೊಂಡ. ಕಾರ್ಯಕ್ರಮದ ದಿನ ಬಂದಿತು; ಸಭೆ ಸೇರಿದೆ; ಬ್ಯಾನರ್ಜಿಯವರು ಅಧ್ಯಕ್ಷ ಸ್ಥಾನದಲ್ಲಿ ಕುಳಿತಿದ್ದಾರೆ. ತನ್ನ ಸರದಿ ಬಂದಾಗ ನರೇಂದ್ರ ಸಭೆಯ ಮಧ್ಯದಿಂದ ಎದ್ದು ಬಂದು ಭಾಷಣವನ್ನು ಪ್ರಾರಂಭಿಸಿದ–ಈಗ ನಿವೃತ್ತಿ ಹೊಂದುತ್ತಿರುವ ತಮ್ಮ ಅಧ್ಯಾಪಕರ ಮೇಲೆ ವಿದ್ಯಾರ್ಥಿಗಳೆಲ್ಲ ಎಷ್ಟೊಂದು ಅಭಿಮಾನವಿಟ್ಟಿದ್ದಾರೆ, ಈಗ ಅವರನ್ನು ಶಾಲೆಯಿಂದ ಬೀಳ್ಗೊಡ ಬೇಕಾಗಿರುವುದರಿಂದ ಹುಡುಗರೆಲ್ಲರಿಗೂ ಅದೆಷ್ಟು ದುಃಖವಾಗಿದೆ ಎಂಬಿತ್ಯಾದಿಯಾಗಿ ಸುಮಾರು ಅರ್ಧ ಗಂಟೆಯ ಕಾಲ ನಿರರ್ಗಳವಾಗಿ ಭಾಷಣ ಮಾಡಿದ. ತರುವಾಯ, ಅಧ್ಯಕ್ಷಮಹೋದಯ ರಾದ ಸುರೇಂದ್ರನಾಥ ಬ್ಯಾನರ್ಜಿಯವರ ಭಾಷಣ. ಅದರಲ್ಲಿ ಅವರು ಎಳೆಯ ಭಾಷಣಕಾರನಾದ ನರೇಂದ್ರನ ಬಗ್ಗೆ ಪ್ರಸ್ತಾಪಿಸಿ, ಅವನನ್ನು ಬಹಳ ಹೊಗಳಿದರು. ಹುಡುಗರೆಲ್ಲ ತಮಗೇ ಸನ್ಮಾನ ವಾದಷ್ಟು ಸಂತೋಷಪಟ್ಟರು.

ನರೇಂದ್ರ ಈ ವರ್ಷ ಎಂಟ್ರೆನ್ಸ್ ಪರೀಕ್ಷೆಗೆ ಕುಳಿತಿದ್ದಾನೆ. ಅದಕ್ಕಾಗಿ ಇಡೀ ಮೂರು ವರ್ಷಗಳ ಪಾಠವನ್ನು ಓದಿ ಮನನ ಮಾಡಬೇಕಾಗಿದೆ. ಆದರೆ ಅವನ ಗಮನವೆಲ್ಲ ಪಠ್ಯೇತರ ಪುಸ್ತಕಗಳ ಕಡೆಗೇ! ಈ ಅವಧಿಯಲ್ಲಿ ಅವನು ಇಂಗ್ಲಿಷ್ ಹಾಗೂ ಬಂಗಾಳೀ ಭಾಷೆಗಳ ಹಲವಾರು ಉತ್ತಮ ಗ್ರಂಥಗಳನ್ನು ಓದಿ ಹೃದ್ಗತ ಮಾಡಿಕೊಂಡ. ಸಂಸ್ಕೃತ ಸಾಹಿತ್ಯದಲ್ಲಿ ತನ್ನ ವಯಸ್ಸಿಗೆ ಮೀರಿದ ಅಧ್ಯಯನ ನಡೆಸಿ ವಿಶೇಷ ಮಾಹಿತಿ ಪಡೆದುಕೊಂಡ. ಭಾರತದ ಇತಿಹಾಸವನ್ನು ಆಳವಾಗಿ ಅಧ್ಯಯನ ಮಾಡಿದ. ರಾಷ್ಟ್ರಪ್ರೇಮಿಯಾಗಬೇಕಾದವನು ತನ್ನ ರಾಷ್ಟ್ರದ ಇತಿಹಾಸದ ಬಗ್ಗೆ, ಸಂಸ್ಕೃತಿ ಪರಂಪರೆಯ ಬಗ್ಗೆ ನಿಖರವಾದ ತಿಳಿವಳಿಕೆ ಹೊಂದಿರಬೇಕು. ಮುಂದೆ ಪಾಶ್ಚಾತ್ಯ ರಾಷ್ಟ್ರಗಳಲ್ಲಿ ತನ್ನ ಮಾತೃಭೂಮಿಯ ಭವ್ಯತೆಯನ್ನು ಘೋಷಿಸುವುದಕ್ಕಾಗಿಯೇ ಎಂಬಂತಿತ್ತು– ನರೇಂದ್ರನ ಭಾರತೀಯ ಇತಿಹಾಸದ ಅಧ್ಯಯನ. ಹೀಗೆ, ಇಂತಹ ಸಾಹಿತ್ಯಕೃತಿಗಳ ಅಧ್ಯಯನ ದಲ್ಲಿ ಕಳೆದವು ದಿನಗಳು.

ಆದರೆ ಈಗ ಎಂಟ್ರೆನ್ಸ್ ಪರೀಕ್ಷೆ ಸಮೀಪಿಸುತ್ತಿದೆ. ನರೇಂದ್ರ ಪರೀಕ್ಷೆಗಾಗಿ ಅಧ್ಯಯನ ಪ್ರಾರಂಭಿಸಿದ. ಎಷ್ಟೇ ಬುದ್ಧಿಶಕ್ತಿಸಂಪನ್ನನಾದರೂ ಓದಬೇಕಾದ ವಿಷಯಗಳೆಲ್ಲ ಪೇರಿಕೊಂಡು ಬಿಟ್ಟಿದ್ದರೆ ಏನು ಮಾಡಲಾದೀತು? ಆದರೆ ಅವನು ಹಟತೊಟ್ಟು, ಕಷ್ಟಪಟ್ಟು ಓದಿದ. ಉದಾ ಹರಣೆಗೆ, ರೇಖಾಗಣಿತವನ್ನು ತಾನು ಯಾವ ರೀತಿಯಲ್ಲಿ ಅಧ್ಯಯನ ಮಾಡಿದ ಎಂಬುದನ್ನು ಅವನೇ ವಿವರಿಸುತ್ತಾನೆ: “ಪರೀಕ್ಷೆಗೆ ಇನ್ನು ಕೇವಲ ಎರಡು-ಮೂರು ದಿನಗಳಿವೆ ಎನ್ನುವಾಗ ನೋಡುತ್ತೇನೆ, ನಾನು ರೇಖಾಗಣಿತದ ವಿಷಯದಲ್ಲಿ ಏನೇನೂ ಅಭ್ಯಾಸ ಮಾಡಿಲ್ಲ! ಆಗ ನಾನು ಪಟ್ಟಾಗಿ ರಾತ್ರಿಯೆಲ್ಲ ಕುಳಿತು ಅದನ್ನು ಅಭ್ಯಾಸ ಮಾಡಲಾರಂಭಿಸಿದೆ, ಮತ್ತು ಇಪ್ಪತ್ತನಾಲ್ಕು ಗಂಟೆಗಳ ಕಾಲದಲ್ಲಿ ರೇಖಾಗಣಿತದ ನಾಲ್ಕು ಪುಸ್ತಕಗಳನ್ನು ಅಧ್ಯಯನ ಮಾಡಿ ಮುಗಿಸಿದೆ.” ಹೀಗೆ ಅವನು ಕಷ್ಟಪಟ್ಟು ಓದಿದ್ದೇ ಅಲ್ಲದೆ, ಪರೀಕ್ಷೆಯಲ್ಲಿ ಶಾಲೆಗೇ ಮೊದಲಿಗನಾಗಿ ಉನ್ನತ ಶ್ರೇಣಿಯಲ್ಲಿ ಉತ್ತೀರ್ಣನಾದ. ಆ ವರ್ಷ ಅವನ ಶಾಲೆಯಲ್ಲಿ ಅಷ್ಟು ಉನ್ನತ ಶ್ರೇಣಿಯಲ್ಲಿ ತೇರ್ಗಡೆಯಾದವನು ಅವನೊಬ್ಬನೇ! ವಿಶ್ವನಾಥ ಮಗನಿಗೆ ಒಂದು ಒಳ್ಳೆಯ ಕೈಗಡಿಯಾರವನ್ನು ಬಹುಮಾನವಾಗಿ ಕೊಟ್ಟ.

ನರೇಂದ್ರ ಈ ವೇಳೆಗೆ ಪಠ್ಯೇತರ ಪುಸ್ತಕಗಳನ್ನು ಅದೆಷ್ಟು ಓದಿಬಿಟ್ಟಿದ್ದನೆಂದರೆ ಅವನ ಲ್ಲೊಂದು ಆಶ್ಚರ್ಯಕರ ಸಾಮರ್ಥ್ಯ ಬೆಳೆದುಬಂದಿತ್ತು. ಆ ಕುರಿತು ಅವನೇ ಹೇಳುತ್ತಾನೆ: “ನಾನು ಲೇಖಕನ ಅಭಿಪ್ರಾಯಗಳನ್ನು ತಿಳಿಯಲು ಅವನ ಪುಸ್ತಕವನ್ನು ಆಮೂಲಾಗ್ರವಾಗಿ ಓದಲೇಬೇಕಾಗಿರಲಿಲ್ಲ. ಒಂದು ಪ್ಯಾರಾದ ಮೊದಲನೆಯ ಹಾಗೂ ಕೊನೆಯ ವಾಕ್ಯಗಳನ್ನು ಓದಿದರೆ ಸಾಕಾಗಿತ್ತು–ಇಡೀ ಪ್ಯಾರಾದ ಸಾರಾಂಶ ಅರ್ಥವಾಗುತ್ತಿತ್ತು. ಕ್ರಮೇಣ ನನ್ನ ಈ ಶಕ್ತಿ ಇನ್ನೂ ಬೆಳೆಯಿತು; ಆಗ ಪುಟದ ಮೊದಲ ಹಾಗೂ ಕೊನೆಯ ವಾಕ್ಯಗಳನ್ನು ಓದಿಬಿಟ್ಟರೆ ಸಾಕು, ಇಡೀ ಪುಟದ ವಿಷಯವೆಲ್ಲ ನನಗೆ ತಿಳಿದುಬಿಡುತ್ತಿತ್ತು. ಬರಬರುತ್ತ ಈ ಶಕ್ತಿ ಇನ್ನೂ ವೃದ್ಧಿ ಗೊಂಡಿತು, ಎಷ್ಟರಮಟ್ಟಿಗೆಂದರೆ ಲೇಖಕನು ಒಂದು ವಿಷಯವನ್ನು ಅರ್ಥಪಡಿಸಲು ನಾಲ್ಕೈದು ಪುಟಕ್ಕಿಂತ ಹೆಚ್ಚಾಗಿ ಬರೆಯಬೇಕಾಗಿ ಬಂದಿದ್ದರೆ, ನನಗೆ ಆ ಇಡೀ ವಿಷಯವನ್ನು ಗ್ರಹಿಸಲು ಅದರ ಮೊದಲಿನ ಕೆಲವು ವಾಕ್ಯಗಳನ್ನು ಓದಿಬಿಟ್ಟರೆ ಸಾಕಾಗಿತ್ತು.”

