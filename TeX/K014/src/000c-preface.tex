
\chapter*{ಮುನ್ನುಡಿ}

\noindent

ಜೀವನವೊಂದು ಹೋರಾಟವೆಂದು ಸ್ವಾಮಿ ವಿವೇಕಾನಂದರು ಹೇಳುತ್ತಿದ್ದರು. ನಾವು ಅನೇಕ ಶಕ್ತಿಗಳ ವಿರುದ್ಧ ಹೋರಾಡಬೇಕಾಗುತ್ತದೆ. ಹೀಗೆ ಹೋರಾಡುತ್ತಲೇ ಜೀವನವು ವಿಕಾಸಹೊಂದು ತ್ತದೆ. ಪ್ರಾಕೃತಿಕ ಶಕ್ತಿಗಳ ವಿರುದ್ಧ ನಾವು ಸಮರ ನಡೆಸಬೇಕಾಗುತ್ತದೆ; ಸಾಮಾಜಿಕ ಪರಿಸ್ಥಿತಿಗಳ ವಿರುದ್ಧ ನಾವು ಸೆಣಸಬೇಕಾಗುತ್ತದೆ. ಇವೆಲ್ಲಕ್ಕಿಂತ ಮಿಗಿಲಾಗಿ ನಮ್ಮೊಳಗೇ ಇರುವ ಶತ್ರುಗಳ ವಿರುದ್ಧ ನಾವು ನಿರಂತರ ಯುದ್ಧ ನಡೆಸಬೇಕಾಗುತ್ತದೆ. ನಮ್ಮ ಮನಶ್ಚಾಂಚಲ್ಯದ ವಿರುದ್ಧ, ನಮ್ಮ ನಿಷೇಧಾತ್ಮಕ ಭಾವೋದ್ವೇಗಗಳ ವಿರುದ್ಧ, ನಮ್ಮ ಅಹಂಕಾರದ ವಿರುದ್ಧ, ವಿವಿಧ ಪ್ರಲೋಭನೆಗಳ ವಿರುದ್ಧ, ನಾವೇ ಸೃಷ್ಟಿಸಿಕೊಂಡ ಸಂಸ್ಕಾರಗಳ ವಿರುದ್ಧ–ಹೀಗೆ ಅನೇಕ ಪ್ರತಿಕೂಲ ಶಕ್ತಿಗಳ ವಿರುದ್ಧ ನಾವು ಹೋರಾಡಬೇಕಾಗುತ್ತದೆ. ಬಾಹ್ಯ ರಣರಂಗದಲ್ಲಿಯೂ ನಾವು ಹೋರಾಡಬೇಕು ಹಾಗೂ ಅಂತರಂಗದ ರಣರಂಗದಲ್ಲಿಯೂ ನಾವು ಹೋರಾಡಬೇಕು. ಈ ಹೋರಾಟ ನಡೆಸಲು ನಾವು ಸೋಲರಿಯದ ವೀರ ಯೋಧರಾಗಬೇಕು. ಈ ಹೋರಾಟಕ್ಕೆ ಬೇಕಾಗಿರುವ ಗುಣಗಳು ಸಾಹಸಪ್ರವೃತ್ತಿ, ಧೈರ್ಯ, ದೈಹಿಕ ಮತ್ತು ಮಾನಸಿಕ ಬಲ, ಪ್ರಚಂಡ ಇಚ್ಛಾಶಕ್ತಿ, ಅಪಾರ ಆತ್ಮವಿಶ್ವಾಸ ಮತ್ತು ಸ್ವಾತಂತ್ರ್ಯಪ್ರೇಮ. ಸ್ವಾಮಿ ವಿವೇಕಾನಂದರಲ್ಲಿ ಎದ್ದು ಕಾಣುವ ಗುಣಗಳೇ ಇವು. ಆದ್ದರಿಂದಲೇ ಅವರನ್ನು ವೀರ ಸಂನ್ಯಾಸಿ ಎಂದೂ ಯೋಧ ಸಂನ್ಯಾಸಿ ಎಂದೂ ಕರೆಯಲಾಗಿದೆ. ಅವರು ಸಾಮಾನ್ಯ ಯೋಧರಂತೆ ಕತ್ತಿ ಗುರಾಣಿ ಹಿಡಿದೊ ಬಂದೂಕು ಹಿಡಿದೊ ಯುದ್ಧ ಮಾಡಿದವರಲ್ಲ. ವಿವೇಕ ವೈರಾಗ್ಯಾದಿ ಆಧ್ಯಾತ್ಮಿಕ ಗುಣಗಳೇ ಅವರ ಆಯುಧಗಳಾಗಿದ್ದವು. ಈ ಮಾಯಾ ಪ್ರಪಂಚದ ಸಮರಾಂಗಣದಲ್ಲಿ ಮಹಾ ಪರಾಕ್ರಮದಿಂದ ಹೋರಾಡಿ ಆಧ್ಯಾತ್ಮಿಕ ಸಾಮ್ರಾಜ್ಯವನ್ನು ಗೆದ್ದ ಮಹಾ ಸಾಹಸಿ ಸ್ವಾಮಿ ವಿವೇಕಾನಂದರು.

ಅವರಲ್ಲಿ ಈ ಸಾಹಸ ಗುಣಗಳು ಬಾಲ್ಯಕಾಲದಿಂದಲೂ ಇದ್ದುದನ್ನು ನೋಡುತ್ತೇವೆ. ಬಾಲ್ಯ ಕಾಲದ ಅವರ ಹೆಸರು ನರೇಂದ್ರ ಎಂದು. ಆದ್ದರಿಂದ ‘ವೀರ ನರೇಂದ್ರ’ ಎಂದು ಅವರನ್ನು ಕರೆಯುವುದು ಔಚಿತ್ಯಪೂರ್ಣವಾಗಿದೆ. ನರೇಂದ್ರನ ಬಾಲ್ಯ ಕಾಲದ ಜೀವನವನ್ನೂ ವೀರೋಚಿತ ರೋಮಾಂಚಕ ಘಟನೆಗಳನ್ನೂ ಬಾಲಕರೆಲ್ಲರೂ ಅನುಸರಿಸಬೇಕಾದ ವಿಶಿಷ್ಟ ಗುಣಗಳನ್ನೂ ಕೂಡಿರುವ ‘ವೀರ ನರೇಂದ್ರ’ ಎಂಬ ಈ ಪುಸ್ತಕವು ಸ್ವಾಮಿ ಪುರುಷೋತ್ತಮಾನಂದರ ‘ವೀರ ಸಂನ್ಯಾಸಿ ವಿವೇಕಾನಂದ’ ಎಂಬ ಗ್ರಂಥದ ಮೊದಲ ಆರು ಅಧ್ಯಾಯಗಳನ್ನು ಹೊಂದಿದೆ. ಈ ಗ್ರಂಥವು ಸ್ವಾಮಿ ವಿವೇಕಾನಂದರ ಸವಿಸ್ತಾರ ಜೀವನದ ಮೊದಲ ಸಂಪುಟವಾಗಿದೆ. ಉಳಿದೆರಡು ಸಂಪುಟಗಳ ಹೆಸರು ‘ವಿಶ್ವವಿಜೇತ ವಿವೇಕಾನಂದ’ ಮತ್ತು ‘ವಿಶ್ವಮಾನವ ವಿವೇಕಾನಂದ’.

\begin{flushright}
ಪ್ರಕಾಶಕರು
\end{flushright}

