
\chapter{ಸದ್ಗುಣ ಸಾಗರ}

\noindent

ಎಂಟ್ರೆನ್ಸ್ ಪರೀಕ್ಷೆ ಮುಗಿಯುವ ಹೊತ್ತಿಗೆ ನರೇಂದ್ರನ ಬಾಲ್ಯದ ಆಟ-ಓಡಾಟಗಳು, ತಂಟೆ- ತಮಾಷೆಗಳು ಎಲ್ಲ ಮುಗಿದುಹೋಗಿದ್ದುವು. ಈಗ ಅವನು ಯೌವನದ ಮೆಟ್ಟಿಲೇರಿದ್ದಾನೆ. ಅವನ ವಿಚಾರಧಾರೆ ಹೆಚ್ಚು ಗಂಭೀರವಾಗುತ್ತ ಬರುತ್ತಿದೆ. ಈಗ ತನ್ನ ಹದೀನೇಳನೆಯ ವಯಸ್ಸಿನಲ್ಲಿ ಅವನು ಕಲ್ಕತ್ತದ ಪ್ರೆಸಿಡೆನ್ಸಿ ಕಾಲೇಜಿಗೆ ಸೇರಿದ. ಇನ್ನು ಮುಂದೆ ನಾವು ಅವನನ್ನು ತೀವ್ರ ಚಿಂತನಶೀಲನಾದ ವಿದ್ಯಾರ್ಥಿಯಾಗಿ ಕಾಣಲಿದ್ದೇವೆ. ಅವನೀಗ ಬಾಲಜಗತ್ತನ್ನು ದಾಟಿ ಯುವ ಪ್ರಪಂಚದಲ್ಲಿ ವಿಹರಿಸುತ್ತಿದ್ದಾನೆ. ಈಗ ಅವನ ಕಲ್ಪನೆಗಳ ವಿನ್ಯಾಸವೇ ಬೇರೆ, ಅನಿಸಿಕೆಗಳೇ ಬೇರೆ, ಅನುಭವಗಳೇ ಬೇರೆ. ಅವನಿನ್ನೂ ಹದಿನೇಳರ ತರುಣನಾದರೂ ಅವನದು ಕಟ್ಟುಮಸ್ತಾದ ಮಾಂಸಲವಾದ ಮೈಕಟ್ಟು; ನೋಡಲು ಸ್ವಲ್ಪ ಸ್ಥೂಲವಾಗಿ ಕಂಡರೂ ತುಂಬ ಚಟುವಟಿಕೆ ಯಿಂದಿರುವ ಶರೀರ. ಅವನು ಸೇರಿದ್ದ ಪ್ರೆಸಿಡೆನ್ಸಿ ಕಾಲೇಜಿನ ವಿದ್ಯಾರ್ಥಿಗಳು ಒಂದೋ ಐರೋಪ್ಯ ಉಡಿಗೆಯಲ್ಲಿ, ಇಲ್ಲವೆ ಭಾರತೀಯ ಉಡಿಗೆಯಾದ ಪೈಜಾಮ-ಜುಬ್ಬಗಳಲ್ಲಿ ಬರಬೇಕಾಗಿತ್ತು. ಜೊತೆಗೆ, ಎಲ್ಲರೂ ಕೈಗಡಿಯಾರವನ್ನು ಕಟ್ಟಿಕೊಂಡಿರಬೇಕಾಗಿತ್ತು. ನರೇಂದ್ರ ಭಾರತೀಯ ಸಂಪ್ರದಾಯಪ್ರಿಯ; ಆದ್ದರಿಂದ ಅವನು ಭಾರತೀಯ ಉಡಿಗೆಯಾದ ಪೈಜಾಮ-ಜುಬ್ಬವನ್ನೇ ಆರಿಸಿಕೊಂಡ.

ನರೇಂದ್ರನಿಗೆ ಮೊದಲನೆಯ ವರ್ಷವೇನೋ ಕಾಲೇಜಿಗೆ ಪ್ರತಿದಿನ ತಪ್ಪದೆ ಹೋಗಲು ಸಾಧ್ಯವಾಯಿತು. ಆದರೆ ಎರಡನೆಯ ವರ್ಷ ಮಲೇರಿಯಾ ಜ್ವರ ಹಿಡಿದು, ಸರಿಯಾಗಿ ಹೋಗಲು ಸಾಧ್ಯವಾಗದೆ, ಹಾಜರಿ ಕಡಿಮೆಯಾಯಿತು. ಆದ್ದರಿಂದ ಪರೀಕ್ಷೆಗೆ ಕುಳಿತುಕೊಳ್ಳಲು ಪ್ರೆಸಿಡೆನ್ಸಿ ಕಾಲೇಜಿನಲ್ಲಿ ನರೇಂದ್ರನಿಗೆ ಅನುಮತಿ ಸಿಗಲಿಲ್ಲ. ಆದರೆ ಅವನ ಅದೃಷ್ಟಕ್ಕೆ. ‘ಸ್ಕಾಟಿಷ್ ಜನರಲ್ ಮಿಷನರಿ ಬೋರ್ಡ್​’ ಎಂಬ ಮತ್ತೊಂದು ವಿದ್ಯಾಸಂಸ್ಥೆ ನಡೆಸುತ್ತಿದ್ದ ‘ಸ್ಕಾಟಿಷ್ ಚರ್ಚ್ ಕಾಲೇಜು’ ಅವನಿಗೆ ಪರೀಕ್ಷೆಗೆ ಕುಳಿತುಕೊಳ್ಳಲು ಅವಕಾಶ ನೀಡಿತು. ನರೇಂದ್ರ ಎಫ್.ಎ. ಪರೀಕ್ಷೆ ಯಲ್ಲಿ ಉತ್ತೀರ್ಣನಾದ; ಆದರೆ ಅನಾರೋಗ್ಯ-ನಿಶ್ಶಕ್ತಿಗಳ ದೆಸೆಯಿಂದ ದ್ವಿತೀಯ ದರ್ಜೆಗಳಿಸಿದ. ಮುಂದೆ ಅವನು ಇದೇ ಕಾಲೇಜಿನಲ್ಲಿ ವಿದ್ಯಾಭ್ಯಾಸ ಮುಂದುವರಿಸಿ, ಬಿ.ಎ.ವರೆಗೆ ಓದಿ ಉತ್ತೀರ್ಣನಾದ. ಈ ಸಂದರ್ಭದಲ್ಲಿ, ಇದೇ ಕಾಲೇಜಿಗೆ ಸಂಬಂಧಪಟ್ಟಂತೆ ಇನ್ನೊಂದು ವಿಷಯ ವನ್ನು ಸ್ಮರಿಸಬಹುದಾಗಿದೆ. ನರೇಂದ್ರ ಮುಂದೆ ಲೋಕವಿಖ್ಯಾತ ಸ್ವಾಮಿ ವಿವೇಕಾನಂದರಾಗಿ ಪಾಶ್ಚಾತ್ಯ ಜಗತ್ತಿನಲ್ಲಿ ಜಯಭೇರಿ ಮೊಳಗಿಸಿ ಕಲ್ಕತ್ತಕ್ಕೆ ಹಿಂದಿರುಗಿದಾಗ ಸಾರೋಟಿನಲ್ಲಿ ಅವರನ್ನು ಮೆರವಣಿಗೆಯಲ್ಲಿ ಕರೆತರಲಾಯಿತು. ಆಗ ಆ ಸಾರೋಟಿನ ಕುದುರೆಗಳನ್ನು ಬಿಚ್ಚಿ, ಇದೇ ‘ಸ್ಕಾಟಿಷ್ ಚರ್ಚ್ ಕಾಲೇಜಿ’ನ ಯುವಕರು ಸ್ವತಃ ತಮ್ಮ ಕೈಯಿಂದಲೇ ಎಳೆದು, ತಮ್ಮ ಗೌರವಾದರ ಅಭಿನಂದನೆಗಳನ್ನು ಅರ್ಪಿಸಿದರು.

ನರೇಂದ್ರನ ಕಾಲೇಜು ವಿದ್ಯಾಭ್ಯಾಸ ಈಗ ಭರದಿಂದ ಸಾಗತೊಡಗಿತು. ಈ ಅವಧಿಯಲ್ಲೇ ಅವನ ಜ್ಞಾನದಾಹವೂ ಆಧ್ಯಾತ್ಮಿಕ ಪಿಪಾಸೆಯೂ ಅತಿಶಯವಾಗಿ ಜಾಗೃತವಾಗಿಬಿಟ್ಟುವು. ಇನ್ನು ಮುಂದಿನ ವರ್ಷಗಳಲ್ಲಿ ಆ ಬಗ್ಗೆ ಇನ್ನೂ ವಿವರವಾಗಿ ನೋಡಲಿದ್ದೇವೆ. ಎಫ್.ಎ. ತರಗತಿಯಲ್ಲಿ ಅವನು ಆಯ್ದುಕೊಂಡಿದ್ದ ಮುಖ್ಯ ವಿಷಯಗಳೆಂದರೆ ಇಂಗ್ಲಿಷ್, ಬಂಗಾಳಿ, ಇತಿಹಾಸ, ಗಣಿತ, ತರ್ಕಶಾಸ್ತ್ರ ಹಾಗೂ ಮನಶ್ಶಾಸ್ತ್ರ. ಇವು ಅವನು ಎಂಟ್ರೆನ್ಸ್ ಪರೀಕ್ಷೆಗಾಗಿ ಆರಿಸಿಕೊಂಡಿದ್ದ ವಿಷಯಗಳ ಮುಂದುವರಿಕೆಯೇ ಎನ್ನಬಹುದು. ಮುಂದೆ ಅವನು ಬಿ.ಎ. ತರಗತಿಯಲ್ಲಿ ವಿಶೇಷ ಅಧ್ಯಯನಕ್ಕಾಗಿ ಆಯ್ದುಕೊಂಡ ವಿಷಯವೆಂದರೆ, ತರ್ಕಶಾಸ್ತ್ರ ಮನಶ್ಶಾಸ್ತ್ರಗಳಿಗೆ ಬದಲಾಗಿ ತತ್ತ್ವಶಾಸ್ತ್ರ; ಉಳಿದಂತೆ ಯಾವುದೇ ಬದಲಾವಣೆ ಇರಲಿಲ್ಲ. ಅವನು ಈ ವಿಷಯಗಳನ್ನು ಆರಿಸಿಕೊಂಡದ್ದು ತಂದೆ ವಿಶ್ವನಾಥನ ನಿರ್ದೇಶನದ ಮೇರೆಗೆ. ತರ್ಕಶಾಸ್ತ್ರ ಮನಶ್ಶಾಸ್ತ್ರ ತತ್ತ್ವ ಶಾಸ್ತ್ರಗಳು ಬುದ್ಧಿಯ ಸಂಸ್ಕರಣಕ್ಕೆ ಹಾಗೂ ವಿವೇಕಪ್ರಜ್ಞೆಯ ವಿಕಸನಕ್ಕೆ ಬಹಳ ಸಹಾಯಕಾರಿ ಎಂದು ನಂಬಿದ್ದರಿಂದಲೇ ಅವನು ಅವುಗಳನ್ನು ವಿಶೇಷವಾಗಿ ಆರಿಸಿಕೊಂಡ. ಜೊತೆಗೆ ಸಾಹಿತ್ಯ ದಲ್ಲಿ ಹೆಚ್ಚಾಗಿ ಅಧ್ಯಯನ ಮಾಡಿದ. ಅಲ್ಲದೆ ಲೇಖನ-ಭಾಷಣಗಳ ಕಲೆಯಲ್ಲೂ ಪ್ರವೀಣನಾದ. ಇಂಗ್ಲಿಷ್ ಭಾಷೆಯನ್ನು, ಅದರಲ್ಲೂ ವಿಶೇಷವಾಗಿ ಸಂಭಾಷಣೆಯ ಹಾಗೂ ಚರ್ಚೆಯ ಕಲೆ ಯನ್ನು ಸ್ವಾಧೀನಪಡಿಸಿಕೊಳ್ಳುವುದಕ್ಕಾಗಿ ತೀವ್ರ ಅಧ್ಯಯನ ಪ್ರಯತ್ನ ಮಾಡಿದ, ಮತ್ತು ಅದರಲ್ಲಿ ಯಶಸ್ವಿಯೂ ಆಗಿ ಆ ವಿಷಯದಲ್ಲಿ ಕಾಲೇಜಿಗೇ ಮೊದಲಿಗನಾದ. ಅವನು ತನ್ನ ಅಧ್ಯಯನವನ್ನು ಪಾಠಪಟ್ಟಿಗಷ್ಟೇ ಸೀಮಿತಗೊಳಿಸಿದವನಲ್ಲ. ಕಾಲೇಜಿನ ಮೊದಲ ಎರಡು ವರ್ಷಗಳಲ್ಲಿ ಅವನು ಪಾಶ್ಚಾತ್ಯ ತರ್ಕಶಾಸ್ತ್ರಜ್ಞರ ಕೃತಿಗಳನ್ನು ಓದಿ ಅರಗಿಸಿಕೊಂಡ. ಮೂರನೇ ಹಾಗೂ ನಾಲ್ಕನೇ ವರ್ಷಗಳಲ್ಲಿ ಪಾಶ್ಚಾತ್ಯ ತತ್ತ್ವಶಾಸ್ತ್ರವನ್ನೂ ಪ್ರಾಚೀನ-ಅರ್ವಾಚೀನ ಐರೋಪ್ಯ ಇತಿಹಾಸವನ್ನೂ ವಿಶದವಾಗಿ ಅಧ್ಯಯನ ಮಾಡಿ ಕರತಲಾಮಲಕವಾಗಿಸಿಕೊಂಡ.

ನರೇಂದ್ರನ ಎಫ್.ಎ. ತರಗತಿಯ ಎರಡನೇ ವರ್ಷವನ್ನು ಬಹಳ ಮಹತ್ವಪೂರ್ಣ ಎನ್ನ ಬಹುದು. ಏಕೆಂದರೆ ಅವನ ಜೀವನವನ್ನೇ ಪರಿವರ್ತಿಸಿ ಹೊಸ ರೂಪ ಕೊಟ್ಟು ಅವನನ್ನು ವೀರಸಂನ್ಯಾಸಿ ವಿವೇಕಾನಂದರನ್ನಾಗಿ ಮಾಡಿ ಜಗತ್ತಿನ ಜ್ಞಾನಪೀಠದ ಮೇಲೆ ನೆಲೆಗೊಳಿಸಿದ ಶ್ರೀರಾಮಕೃಷ್ಣರ ವಿಷಯವಾಗಿ ಅವನು ಮೊತ್ತಮೊದಲನೆಯದಾಗಿ ಕೇಳಿದ್ದು ಈ ತರಗತಿ ಯಲ್ಲಿದ್ದಾಗಲೇ. ಪ್ರತಿಯೊಂದು ಮಹಾಕಾರ್ಯಕ್ಕೂ ಒಂದು ಸ್ವಾರಸ್ಯಕರ ಪ್ರಾರಂಭವಿರುತ್ತದೆ. ಯಾವುದೋ ಒಂದು ಪುಟ್ಟ ಮಾತೋ ಅಥವಾ ಘಟನೆಯೋ ಮುಂದಿನ ಯಾವುದೋ ಒಂದು ಮಹಾಕಾರ್ಯಕ್ಕೆ ನಾಂದಿಯಾಗಿಬಿಟ್ಟಿರುತ್ತದೆ. ನರೇಂದ್ರ ಓದುತ್ತಿದ್ದ ಸ್ಕಾಟಿಷ್ ಚರ್ಚ್ ಕಾಲೇಜಿನ ಪ್ರಿನ್ಸಿಪಾಲರ ಹೆಸರು ಪ್ರೊಫೆಸರ್ ವಿಲಿಯಂ ಹೇಸ್ಟೀ ಎಂದು. ಒಂದು ದಿನ ನರೇಂದ್ರನ ತರಗತಿಯನ್ನು ತೆಗೆದುಕೊಳ್ಳಬೇಕಾಗಿದ್ದ ಪ್ರಾಧ್ಯಾಪಕರು ಕಾಲೇಜಿಗೆ ಬಂದಿರಲಿಲ್ಲ. ಆದ್ದರಿಂದ ಅವರ ಬದಲಿಗೆ ಸ್ವತಃ ಪ್ರಿನ್ಸಿಪಾಲರೇ ಅಂದು ಇಂಗ್ಲಿಷ್ ಸಾಹಿತ್ಯದ ವಿಷಯವಾಗಿ ಪಾಠ ಮಾಡತೊಡಗಿದರು. ಅವರು ವರ್ಡ್ಸ್​ವರ್ತ್ ಕವಿಯ \eng{‘Excursion’} ಎನ್ನುವ ಕಾವ್ಯವನ್ನು ವಿವರಿಸುವಾಗ ಅದರಲ್ಲಿ ಭಾವಸಮಾಧಿಯ ಅಥವಾ ಆನಂದಸಮಾಧಿಯ ಕುರಿತಾಗಿ ಒಂದು ಅಂಶ ಬಂದಿತು; ಅಲ್ಲಿ, ಪ್ರಕೃತಿ ಸೌಂದರ್ಯವನ್ನು ವರ್ಣಿಸುತ್ತ, ಸ್ವತಃ ವರ್ಡ್ಸ್​ವರ್ತ್ ಕವಿಯೇ ಭಾವಪರವಶನಾದಂತೆ ಕಂಡುಬರುತ್ತದೆ. ಈ ಭಾವಸಮಾಧಿಯ ಕುರಿತಂತೆ ವಿಲಿಯಂ ಹೇಸ್ಟೀ ವಿವರಿಸಿದ್ದು ಹುಡುಗರಿಗೆ ಸರಿಯಾಗಿ ಅರ್ಥವಾಗಲಿಲ್ಲ. ಅಲ್ಲದೆ ಭಾವಸಮಾಧಿಯಂತಹ ವಿಷಯ ವನ್ನು ಮಾತಿನಿಂದ ವಿವರಿಸಿ ಅರ್ಥಪಡಿಸಲು ಸಾಧ್ಯವೂ ಇಲ್ಲ. ಅದನ್ನು ಮನಗಂಡ ಹೇಸ್ಟೀ ಹೇಳುತ್ತಾರೆ: “ಇದೊಂದು ವಿಶೇಷವಾದ ಅನುಭವ. ಶುದ್ಧವಾದ ಹಾಗೂ ಏಕಾಗ್ರಗೊಂಡ ಮನಸ್ಸಿಗೆ ಮಾತ್ರ ಈ ಬಗೆಯ ಅನುಭವ ಸಾಧ್ಯ. ಈಗಿನ ಕಾಲದಲ್ಲಿ ಇದು ಬಹಳ ಅಪರೂಪ. ಆದರೆ ಒಬ್ಬ ವ್ಯಕ್ತಿಗೆ ಮಾತ್ರ ಈ ಬಗೆಯ ಅನುಭವವಾಗುವುದನ್ನು ನಾನೇ ಕಂಡಿದ್ದೇನೆ–ಅವರೇ ದಕ್ಷಿಣೇಶ್ವರದ ರಾಮಕೃಷ್ಣ ಪರಮಹಂಸರು. ಅಲ್ಲಿಗೆ ಹೋದರೆ ನೀವೇ ಅದನ್ನು ಕಂಡು ಅರ್ಥ ಮಾಡಿಕೊಳ್ಳಬಹುದು.” ನರೇಂದ್ರನಿಗೆ ತನ್ನ ಗುರುವಿನ ಮೊದಲ ಪರಿಚಯವಾದ ಬಗೆ ಹೀಗೆ.

ಈ ದಿನಗಳಲ್ಲಿ ತನ್ನ ಸ್ನೇಹಿತರ ಹಾಗೂ ಇತರ ಪರಿಚಯಸ್ಥರ ದೃಷ್ಟಿಗೆ ನರೇಂದ್ರ ಒಬ್ಬ ರಾಜಕುಮಾರನಂತೆ ಕಾಣುತ್ತಿದ್ದ–ಅವನ ವ್ಯಕ್ತಿತ್ವದಲ್ಲಿ ಅಂಥಾ ಘನತೆ, ಗಾಂಭೀರ್ಯ, ತೇಜಸ್ಸು. ಅವನ ಕಾಲೇಜಿನ ಭಾರತೀಯ ಹಾಗೂ ಆಂಗ್ಲ–ಈ ಎರಡೂ ವರ್ಗದ ಪ್ರಾಧ್ಯಾಪಕರು ಅವನಿಂದ ವಿಶೇಷವಾಗಿ ಆಕರ್ಷಿತರಾಗಿದ್ದರು. ಅವನ ಅಪಾರ ಬುದ್ಧಿಸಾಮರ್ಥ್ಯವನ್ನೂ ಅವನ ಅದ್ಭುತ ಅಂತಶ್ಶಕ್ತಿಯನ್ನೂ ಅವರು ಸ್ಪಷ್ಟವಾಗಿ ಕಂಡುಕೊಂಡಿದ್ದರು. ನರೇಂದ್ರನ ವಿಷಯವಾಗಿ ಪ್ರಿನ್ಸಿಪಾಲ್ ವಿಲಿಯಂ ಹೇಸ್ಟೀ ಅವರಂತೂ, “ನರೇಂದ್ರನಾಥ ನಿಜಕ್ಕೂ ಒಬ್ಬ ಅಸಾಧಾರಣ ಪ್ರತಿಭಾವಂತ. ನಾನು ವ್ಯಾಪಕವಾಗಿ ಸಂಚರಿಸಿ ನೋಡಿದ್ದೇನೆ–ಆದರೆ ಅವನಂತಹ ಮೇಧಾವಿ ಹಾಗೂ ಸಾಮರ್ಥ್ಯಶಾಲೀ ಹುಡುಗನನ್ನು ನಾನೆಲ್ಲೂ ಕಂಡಿಲ್ಲ. ಅವನು ಈ ಜಗತ್ತಿನಲ್ಲಿ ತನ್ನ ಪ್ರಭಾವ ಬೀರುವುದು ಖಂಡಿತ!” ಎಂದು ಕೊಂಡಾಡುತ್ತಾರೆ.

ಎಫ್.ಎ. ಅಧ್ಯಯನ ಮುಗಿಯುವುದರೊಳಗೆ ನರೇಂದ್ರನ ಆಲೋಚನಾ ವಿಧಾನದಲ್ಲಿ ಒಂದು ಅದ್ಭುತವಾದ ಸ್ವಂತಿಕೆ ಬೆಳೆದುಬಂದಿತ್ತು. ಯಾವ ಹೊಸ ವಿಚಾರವನ್ನು ಕಾಣಲಿ, ಕೇಳಲಿ, ಅದನ್ನು ತನ್ನ ವಿಚಾರದ ಒರೆಗಲ್ಲಿಗೆ ಹಚ್ಚಿನೋಡದೆ ಬಿಡುತ್ತಿರಲಿಲ್ಲ. ಸ್ನೇಹಿತರೆಲ್ಲ ಸೇರಿ ಅಧ್ಯಯನ ಮಾಡುತ್ತಿರುವಾಗ ಹುಟ್ಟಿಕೊಂಡ ಯಾವುದಾದರೊಂದು ವಿಷಯದ ಮೇಲಿನ ಚರ್ಚೆ ಇತ್ಯರ್ಥ ವಾಗದಿದ್ದರೆ, ನರೇಂದ್ರ ವಿರಾಮದ ವೇಳೆಯಲ್ಲೂ ಆ ಚರ್ಚೆಯನ್ನು ಮುಂದುವರಿಸುತ್ತಿದ್ದ. ಅವನ ಅಭಿಪ್ರಾಯವನ್ನು ವಿರೋಧಿಸುವ ಕೆಲಸಕ್ಕೆ ಕೈಹಾಕಿದವರು ಸಿಂಹದ ಬಾಯಿಗೆ ಕೈ ಹಾಕಿದಂತೆಯೇ! ತನ್ನ ಪ್ರಖರ ಪ್ರತಿವಾದದಿಂದ ಅವನು ಎದುರಾಳಿಯ ಬಾಯಿ ಮುಚ್ಚಿಸಿ ಬಿಡುತ್ತಿದ್ದ. ತನ್ನ ಸಹಪಾಠಿಗಳ ನಡುವೆ ಅವನೊಂದು ಸಿಂಹದಂತೆ. ಸಾಂಪ್ರದಾಯಿಕ ಆಲೋಚನೆ-ನಂಬಿಕೆಗಳಿಗೆ ಬದ್ಧ ವಿರೋಧಿ ಅವನು; ಯಾವುದನ್ನೂ ಕಣ್ಮುಚ್ಚಿಕೊಂಡು ಒಪ್ಪಿ ಕೊಳ್ಳುವವನಲ್ಲ. ಗೆಳೆಯರೊಂದಿಗಿನ ಅವನ ಚರ್ಚೆಯ ವಿಷಯಗಳಂತೂ ಅಸಂಖ್ಯಾತ.

ನಿಜಕ್ಕೂ ಈ ದಿನಗಳಲ್ಲಿ ನರೇಂದ್ರನಲ್ಲೊಂದು ಅದ್ಭುತ ಮಾನಸಿಕ ಪರಿವರ್ತನೆಯುಂಟಾ ಯಿತು. ಅವನಲ್ಲಿ ಕಟುವಾದ ವಿಮರ್ಶಾತ್ಮಕ ಮನೋಭಾವ ಬೆಳೆದು ಬಿಟ್ಟಿತು. ಸಾಮಾನ್ಯವಾಗಿ ಯುವಜನರು ಭಾವುಕ ಸ್ವಭಾವದವರಾಗಿರುತ್ತಾರೆ. ಆದರೆ ನರೇಂದ್ರನಲ್ಲಿ ವಿಚಾರಶಕ್ತಿಯೆನ್ನು ವುದು ಭಾವುಕತೆಯನ್ನು ಮೆಟ್ಟಿನಿಂತಿತು. ಜೀವನ ರಂಗಕ್ಕೆ ಸಂಬಂಧಿಸಿದ ಪ್ರತಿಯೊಂದು ವಿಷಯವನ್ನೂ ವಿಮರ್ಶೆಮಾಡಿ ನೋಡಿ, ಅದರ ಒಳಹೊರಗನ್ನು ಜಾಲಾಡಿ, ಅದರಲ್ಲಿ ತಥ್ಯಾಂಶ ವಿದ್ದರೆ ಮಾತ್ರ ಸ್ವೀಕರಿಸುವುದನ್ನು ಅವನು ರೂಢಿಸಿಕೊಂಡ. ತನ್ನ ಧರ್ಮವಾದ ಹಿಂದೂ ಧರ್ಮದ ಒಳಹೊರಗುಗಳನ್ನೂ ಅವನು ವಿಶ್ಲೇಷಿಸಿ ನೋಡಲಾರಂಭಿಸಿದ. ಧರ್ಮ ಎನ್ನುವು ದೊಂದು ಬೃಹತ್ತಾದ ವಿಷಯ. ಈ ಧರ್ಮದ ಮರ್ಮ ಏನು? ಇದರ ಸಾಧಕ-ಬಾಧಕಗಳೇನು? ತನ್ನ ಜೀವನದಲ್ಲಿ ಈ ಧರ್ಮದ ಪಾತ್ರವೇನು? ಇದರಲ್ಲೇನಾದರೂ ಸತ್ತ್ವವಿದೆಯೆ? ಈ ವಿಷಯಗಳನ್ನು ಕಟುತರ ವಿಮರ್ಶೆಗೆ ಗುರಿಪಡಿಸಿ ನೋಡಿದ. ಹಾಗೆಯೇ ತನ್ನ ಸಮಕಾಲೀನ ಭಾರತದ ಸಾಮಾಜಿಕ ರಾಜಕೀಯ ಸ್ಥಿತಿಗತಿಗಳ ಬಗ್ಗೆ ಗಾಢಚಿಂತನೆ ನಡೆಸಿದ. ಅವನು ಯಾವಾ ಗಲೂ ತನ್ನದೇ ಆದ ನಿರ್ಧಾರಕ್ಕೆ ಬರುವ ಸ್ವಾತಂತ್ರ್ಯವನ್ನು ಬಯಸುವವನು. ಅವನ ಆ ನಿರ್ಧಾರ ಯಾವ ಬಗೆಯದೇ ಆಗಲಿ, ಅದನ್ನು ಯಾರೂ ಸುಲಭದಲ್ಲಿ ತಳ್ಳಿಹಾಕುವಂತಿರಲಿಲ್ಲ. ಆದರೆ ಅವನ ಈ ನಂಬಿಕೆ ಎನ್ನುವುದು ಓದರಿಯದವರ ಮೊಂಡುವಾದವಲ್ಲ, ಕುರುಡು ನಂಬಿಕೆಯಲ್ಲ. ಅವನು ಅಸಂಖ್ಯಾತ ಗ್ರಂಥಗಳನ್ನು ಅಧ್ಯಯಿಸಿ ಅಮೂಲ್ಯ ವಿಷಯಸಂಗ್ರಹಣೆ ಮಾಡಿಕೊಂಡಿದ್ದ ಯುವಕ! ಅಲ್ಲದೆ ಅವನೊಬ್ಬ ಆದರ್ಶವಾದಿ, ಸತ್ಯಶೋಧಕ!–ಕೇವಲ ಲೌಕಿಕ ವಿಲಾಸಗಳಿಂದ ತೃಪ್ತಿಪಟ್ಟುಕೊಳ್ಳುವ ಸಾಮಾನ್ಯ ಯುವಕರಂತಲ್ಲ. ಪ್ರಕೃತಿಯ ದಟ್ಟವಾದ ತೆರೆಯನ್ನು ಹರಿದು, ಒಳಗೇನಿದೆಯೆಂದು ಕಾಣಬಯಸುವವನು. ಅವನ ಅಂತರಾಳದಲ್ಲಿ ಸತ್ಯ ಸಾಕ್ಷಾತ್ಕಾರದ ಹಂಬಲ ಪ್ರವಾಹದಂತೆ ಭೋರ್ಗರೆಯುತ್ತಿತ್ತು. ಈ ಹಂಬಲದ ಮಿಡಿತವೆನ್ನುವುದು ಮೊದಲಿನಿಂಲೂ ಬಡಿಬಡಿದು ಅವನಿಗೊಂದು ಅಂಶವನ್ನು ಸೂಚಿಸುತ್ತಿತ್ತು– ‘ತಾನು ಲೋಕದ ಎಲ್ಲ ವ್ಯಕ್ತಿಗಳಂತಲ್ಲ, ತನ್ನ ಜೀವನ ಇತರರ ಜೀವನದಂತಲ್ಲ’ ಎಂದು.

ಆದರೆ ನರೇಂದ್ರ ಹೀಗೆ ಗಂಭೀರ ವಿಚಾರಧಾರೆಯಲ್ಲಿ ಮುಳುಗಿದ್ದರೂ ಅವನ ಹಾಸ್ಯಪ್ರವೃತ್ತಿಯೇನೂ ಬತ್ತಿಹೋಗಿರಲಿಲ್ಲ; ಅವನ ಸಾಹಸಪ್ರವೃತ್ತಿ ಕಡಿಮೆಯಾಗಿರಲಿಲ್ಲ. ಸ್ನೇಹಿತರ ಗುಂಪಿನಲ್ಲಿ ಅತ್ಯಂತ ಹಾಸ್ಯಪ್ರಿಯನೆಂದರೆ ಅವನೇ. ಯಾವುದೇ ಸನ್ನಿವೇಶದಲ್ಲಿ ಅಡಕವಾಗಿರಬಹುದಾದ ತಮಾಷೆಯನ್ನುಗುರುತಿಸುವಲ್ಲಿ ಅವನೇ ಮೊದಲಿಗ. ಅವನು ಹಾಸ್ಯದ ಚಟಾಕಿಗಳನ್ನು ಹಾರಿಸಿ ಸ್ನೇಹಿತರನ್ನೆಲ್ಲ ನಗೆಗಡಲಿನಲ್ಲಿ ಮುಳುಗಿಸಿಬಿಡುತ್ತಿದ್ದ. ಕಾಲೇಜು ಹುಡಗುರೆಲ್ಲ ಸೇರಿ ಪ್ರವಾಸ ಹೊರಟರೆ ಅವರಲ್ಲಿ ಇವನೇ ಎಲ್ಲರಿಗಿಂತ ಉಲ್ಲಾಸಿ. ಕೆಲವೊಮ್ಮೆ ಈ ಹುಡುಗರೆಲ್ಲ ಸಾರೋಟುಗಳಿಲ್ಲಿ ತುಂಬಿಕೊಂಡು ರಸ್ತೆಯುದ್ದಕ್ಕೂ ಹಾಡಿಕೊಂಡು ಹೊರಟುಬಿಡುತ್ತಿದ್ದರು. ಹಬ್ಬ ಹರಿ ದಿನಗಳಲ್ಲಿ ಸ್ನೇಹಿತರೆಲ್ಲ ಸೇರಿ, ಗಂಗಾನದಿಗೆ ಹೋಗಿ ಮನದಣಿಯೆ ಈಜಾಡಿಕೊಂಡು ಬರುತ್ತಿದ್ದರು – ಇದರಿಂದ ಪವಿತ್ರ ಗಂಗಾಸ್ನಾನವೂ ಆದಂತಾಯಿತು, ಮನಸ್ಸಿಗೆ ಉಲ್ಲಾಸವೂ ದೊರೆತಂತಾಯಿತು. ಇನ್ನು ದುರ್ಗಾಪೂಜೆ ದೀಪಾವಳಿಗಳಂತಹ ಹಬ್ಬದ ದಿನಗಳಲ್ಲಿ ಬೇರೆ ಬೇರೆ ಬೀದಿಗಳಲ್ಲಿ ಏರ್ಪಡಿಸಿದ ಉತ್ಸವಗಳನ್ನು ವೀಕ್ಷಿಸಿ ಆನಂದಿಸಲು ಹೋಗುತ್ತಿದ್ದರು. ಈ ಎಲ್ಲ ಸಂದರ್ಭಗಳಲ್ಲೂ ನರೇಂದ್ರನೇ ಅವರೆಲ್ಲರ ನಾಯಕ. ಈ ಎಲ್ಲ ಹಬ್ಬ ಹರಿದಿನ ಉತ್ಸವಗಳಲ್ಲಿ ಎಲ್ಲರೂ ಅತಿ ಹೆಚ್ಚಿನ ಸಂತೋಷವನ್ನು ಅನುಭವಿಸುವಂತೆ ಅವನು ನೋಡಿಕೊಳ್ಳುತ್ತಿದ್ದ. ಆದರೆ ಇಲ್ಲಿನ ಒಂದು ಬಹುಮುಖ್ಯವಾದ ಅಂಶವೇನೆಂದರೆ ಈ ತಮಾಷೆಗಳಲ್ಲಿ ಚೆಲ್ಲುತನವಿರಲಿಲ್ಲ; ಗಾಂಭೀರ್ಯವಿತ್ತು. ನರೇಂದ್ರ ಈ ರೀತಿ ತನ್ನ ಸಹಪಾಠಿಗಳೊಂದಿಗೆ ಸರಸವಾಗಿ, ಆತ್ಮೀಯವಾಗಿ ಒಡನಾಡುತ್ತ ತನಗರಿವಿಲ್ಲದಂತೆಯೇ ಅನೇಕ ವಿಶ್ವಾಸಪಾತ್ರ ಸ್ನೇಹಿತರನ್ನು ಗಳಿಸಿದ. ಇವರು ಕೊನೆಯವರೆಗೂ ಇವನ ಆಪ್ತರಾಗಿಯೇ ಉಳಿದುಕೊಂಡರು.

ನರೇಂದ್ರನಲ್ಲಿ ಬೆಳೆದ ಇನ್ನೊಂದು ಮನೋವೃತ್ತಿಯೆಂದರೆ ಶೋಕಿಯ ಕುರಿತಾದ ತಿರಸ್ಕಾರ. ಶೋಕಿ ಎಂದರೆ ಚಿತ್ರವಿಚಿತ್ರವಾಗಿ ಬಟ್ಟೆಗಳನ್ನು ಧರಿಸುವುದು. ಚಿತ್ರವಿಚಿತ್ರವಾಗಿ ತಲೆಕೂದಲನ್ನು ಚಾಚಿಕೊಳ್ಳುವುದು ಇತ್ಯಾದಿ. ನರೇಂದ್ರನಿಗೆ ಈ ಬಗೆಯ ಶೋಕಿಲಾಲರನ್ನು ಕಂಡರೆ ಆಗುತ್ತಿರಲಿಲ್ಲ. ಅಂಥವರಿಗೆ ಎದುರಿಗೇ ಹೇಳಿಬಿಡುತ್ತಿದ್ದ, ಸಭ್ಯರಂತೆ ಬಟ್ಟೆ ಧರಿಸಬೇಕಾದ್ದು ಅವರ ಕರ್ತವ್ಯ ಎಂದು. ಹುಡುಗರು ವರ್ತನೆಯಲ್ಲಾಗಲಿ ವೇಷಭೂಷಣದಲ್ಲಾಗಲಿ ಸ್ತ್ರೀಯರ ಅನುಕರಣೆ ಮಾಡುವುದನ್ನು ಅವನು ತಿರಸ್ಕರಿಸುತ್ದಿದ್ದ ಹಡುಗರು ಗಂಡುಗಲಿಗಳಂತಿರಬೇಕೆಂಬುದು ಅವನ ಅಭಿಮತ.

ಆಪತ್ತಿಗೆ ಗುರಯಾದವರ ಸಹಾಯಕ್ಕೆ ಮುನ್ನುಗ್ಗುವುದು ನರೇಂದ್ರನ ಇನ್ನೊಂದು ವಿಶೇಷ ಗುಣ. ಅದಕ್ಕೆ ದೃಷ್ಟಾಂತವಾಗಿ ಒಂದು ರಂಜನೀಯ ಘಟನೆಯನ್ನು ನೋಡಬಹುದು. ಅವನು ಓದುತ್ತಿದ್ದ ಕಾಲೇಜಿನಲ್ಲಿ ಶುಲ್ಕ ಕೊಗಡಲು ಸಾಧ್ಯವಾಗದ ಬಡ ವಿದ್ಯಾರ್ಥಿಗಳಿಗೆ ಅದರಿಂದ ವಿನಾಯಿತಿ ದೊರೆಯುತ್ತಿತ್ತು. ಕೆಲವು ವಿಶೇಷ ಸಂದರ್ಭಗಳಲ್ಲಿ ಧನಸಹಾಯದ ಸವಲತ್ತೂ ಇತ್ತು. ಆದರೆ ಶುಲ್ಕ ವಿನಾಯಿತಿಯನ್ನೇ ಆಗಲಿ ಧನಸಹಾಯವನ್ನೇ ಆಗಲಿ ಬಯಸುವ ವಿದ್ಯಾರ್ಥಿಗಳು ತಮ್ಮ ಅರ್ಜಿಯನ್ನು ಸಕಾಲದಲ್ಲಿ ಅಧಿಕಾರಿಗಳ ಕೈಗೆ ತಲುಪಿಸಬೇಕಾಗಿತ್ತು. ರಾಜ್​ಕುಮಾರ್ ಎಂಬವನು ಆ ಕಾಲೇಜಿನ ಹಿರಿಯ ಗುಮಾಸ್ತೆ. ಬಡ ವಿದ್ಯಾರ್ಥಿಗಳ ಅರ್ಜಿಗಳನ್ನು ನೋಡಿ ತದನುಸಾರವಾಗಿ ಶುಲ್ಕ ವಿನಾಯಿತಿಯನ್ನೋ ಧನಸಹಾಯವನ್ನೋ ಮಂಜೂರು ಮಾಡುವ ವಿಷಯದಲ್ಲಿ ನಿರ್ಧಾರ ತೆಗೆದುಕೊಳ್ಳುವ ಅಧಿಕಾರ ಈತನದಾಗಿತ್ತು. ಒಮ್ಮೆ, ನರೇಂದ್ರನ ಸಹಪಾಠಿಯಾದ ಹರಿದಾಸ ಚಟರ್ಜಿ ಎಂಬ ಒಬ್ಬ ಬಡ ವಿದ್ಯಾರ್ಥಿ ವರ್ಷವಿಡೀ ಫೀಸು ಕಟ್ಟಿಯೇ ಇರಲಿಲ್ಲ. ಈಗ ವರ್ಷಾಂತ್ಯದಲ್ಲಿ ಅವನು ನರೇಂದ್ರನ ಬಳಿ ತನ್ನ ಕಷ್ಟವನ್ನು ತೋಡಿಕೊಂಡ. ವಾರ್ಷಿಕ ಪರೀಕ್ಷೆ ಸಮೀಪಿಸುತ್ತಿದೆ ದ್ ಆದರೆ ಪರೀಕ್ಷಾ ಶುಲ್ಕ ಕಟ್ಟಲೂ ಅವನಿಗೆ ಶಕ್ತಿ ಇರಲಿಲ್ಲ. ಆಗ ನರೇಂದ್ರ, ಶ್ ತಾಳು, ಈ ವಿಷಯದಲ್ಲಿ ಏನಾದರೂ ಮಾಡಲು ಸಾಧ್ಯಾವಾಗುತ್ತದೆಯೋ ನೋಡೋಣ ಷ್ ಎಂದು ಧೈರ್ಯ ತುಂಬಿದ.

ಸರಿ, ಒಂದೆರಡು ದಿನಗಳಾದುವು; ವಿದ್ಯಾರ್ಥಿಗಳೆಲ್ಲ ಫೀಸು ಕಟ್ಟಲು ಮುಖ್ಯ ಗುಮಾಸ್ತೆಯ ಆಫೀಸಿನ ಮುಂದೆ ಸಾಲಾಗಿ ನಿಂತಿದ್ದರು. ನಡುವೆ ಹರಿದಾಸ ಮತ್ತು ನರೇಂದ್ರರೂ ಇದ್ದಾರೆ. ತನ್ನ ಸರದಿ ಬಂದಾಗ ನರೇಂದ್ರ ಮುಖ್ಯ ಗುಮಾಸ್ತೆಯ ಹತ್ತಿರ, “ಸರ್, ನಮ್ಮ ಹರಿದಾಸ್ ತುಂಬ ಬಡವ; ಫೀಸ್ ಕಟ್ಟುವಷ್ಟು ಅನುಕೂಲವಿಲ್ಲ. ದಯವಿಟ್ಟು ಅವನಿಗೆ ವಿನಾಯತಿ ಕೊಡಿಸಬೇಕು. ನೀವು ವಿನಾಯಿತಿ ಕೊಡಿಸಿ ಪರೀಕ್ಷೆಗೆ ಕುಳಿತುಕೊಳ್ಳಲು ಅನುಕೂಲ ಮಾಡಿಕೊಟ್ಟರೆ ಅವನು ಖಂಡಿತವಾಗಿಯೂ ಉನ್ನತ ಶ್ರೇಣಿಯಲ್ಲಿ ಪಾಸಾಗುತ್ತಾನೆ. ಇಲ್ಲದೆ ಹೋದರೆ ಅವನ ಕಥೆ ಮುಗಿದಹಾಗೆಯೇ... ” ಎಂದು ವಿನಯದಿಂದ ಕೇಳಿಕೊಂಡ. ಆದರೆ ಆ ಗುಮಾಸ್ತೆ ಗಂಟುಮೋರೆ ಹಾಕಿಕೊಂಡು, “ಸರಿಸರಿ! ನಿನ್ನ ವಕಾಲತ್ತೇನೂ ಬೇಕಾಗಿಲ್ಲ. ನಿನ್ನ ಕೆಲಸ ನೀನು ನೋಡಿಕೋ ಹೋಗು. ಹಣ ಕಟ್ಟದಿದ್ದರೆ ಅವನನ್ನು ಪರೀಕ್ಷೆಗೆ ಕೂಡಿಸಲು ಸಾಧ್ಯವೇ ಇಲ್ಲ”ಎಂದುಬಿಟ್ಟ. ಇದನ್ನು ಕೇಳಿ ನರೇಂದ್ರನಿಗೆ ಸಿಟ್ಟು ಉಕ್ಕೇರಿದರೂ, ಅಷ್ಟು ಜನ ವಿದ್ಯಾರ್ಥಿಗಳ ಮುಂದೆ ಪ್ರತಿಮಾತಾಡದೆ ಬಂದುಬಿಟ್ಟ. ಹರಿದಾಸನಿಗೆ ತುಂಬ ನಿರಾಶೆಯಾಯಿತು. ಆದರೆ ನರೇಂದ್ರನೇನೂ ಹತಾಶನಾಗಿರಲಿಲ್ಲ. ಅವನು ಗೆಳೆಯನನ್ನು ಸಾಮಾಧಾನಪಡಿಸುತ್ತ ಹೇಳಿದ, “ಅಷ್ಟೇಕೆ ನಿರಾಶನಾಗುತ್ತೀಯೋ? ಹಾಗೆ ಮಾತನಾಡುವುದೇ ಆ ಮುದುಕನ ಸ್ವಭಾವ. ನಾನು ನಿನಗೆ ಏನಾದರೊಂದು ವ್ಯವಸ್ಥೆ ಮಾಡಿಯೇ ತೀರುತ್ತೇನೆ ತಾಳು. ನೀನೇನು ಚಿಂತಿಸಬೇಡ.” ನರೇಂದ್ರ ಒಂದು ಕೆಲಸವನ್ನು ಕೈಗೆ ತೆಗೆದುಕೊಂಡನೆಂದರೆ ಅದನ್ನು ಸಾಧಿಸಿಯೇ ಸಿದ್ಧ. ಅವನಿನ್ನೂ ಹುಡುಗನಾಗಿದ್ದಾಗಲೇ ಸ್ನೇಹಿತರೊಂದಿಗೆ ಯುದ್ಧನೌಕೆಯನ್ನು ನೋಡಲು ಆಸೆಪಟ್ಟು ಹೋದಾಗ ಆ ದರ್ವಾನ‘ಹುಡುಗರಿಗೆಲ್ಲ ಪ್ರವೇಶವಿಲ್ಲ’ ಎಂದು ಗದರಿದರೂ ಉಪಾಯವಾಗಿ ‘ಬಡಾಸಾಹೇಬರ ’ ಬಳಿಗೇ ಹೋಗಿ, ಅನುಮತಿ ಪತ್ರ ತಂದ ಧೀರನಲ್ಲವೆ ಅವನು? ಮತ್ತು ಆ ಅಂಬಿಗರು ಗದರಿಸಿದಾಗ, ಸಿಪಾಯಿಗಳಿಂದ ಅವರನ್ನೇ ಹೆದರಿಸಿ ತನ್ನ ಕಾರ್ಯ ಸಾಧಿಸಿಕೊಂಡ ಪರಾಕ್ರಮಿಯಲ್ಲಿವೆ ಅವನು? ಈಗ ಈ ಗುಮಾಸ್ತೆಯನ್ನು ಹೇಗೆ ಬಗ್ಗಿಸಬೇಕು ಎನ್ನುವುದು ತಿಳಿಯದೆ ಅವನಿಗೆ! ಕಾರ್ಯಸಾಧನೆಗೆ ಧೈರ್ಯ ಬೇಕು. ಜೊತೆಗೆ ಯುಕ್ತಿ ಬೇಕು. ಈ ಎರಡೂ ತುಂಬಿಕೊಂಡಿವೆ ಅವನಲ್ಲಿ. ಹೀಗಿದ್ದ ಮೇಲೆ ಅವನು ತನ್ನ ಕಾರ್ಯವನ್ನು ಸಾಧಿಸದೆ ಬಿಡುತ್ತಾನೆಯೆ?

ಸಂಜೆಯ ವೇಳೆಗೆ ಕಾಲೇಜು ಮುಗಿಯಿತು; ಆದರೆ ನರೇಂದ್ರ ಎಂದಿನಂತೆ ಮನೆಯ ಕಡೆಗೆ ಹೆಜ್ಜೆ ಹಾಕಲಿಲ್ಲ. ಬದಲಾಗಿ, ಬೇರೆ ಯಾವುದೋ ಒಂದು ಗಲ್ಲಿಯ ಕಡೆಗೆ ನಡೆದ. ಅಲ್ಲೇನಪ್ಪ ಕೆಲಸ ಇವನಿಗೆ....? ಅಲ್ಲಿ ಒಂದು ಅಫೀಮಿನ ಅಂಗಡಿ ಇದೆ! ಮಾತ್ರವಲ್ಲ, ಅದು ಮುಖ್ಯಗುಮಾಸ್ತೆ ರಾಜ್​ಕುಮಾರ ದಿನಾಲೂ ಸಾಯಂಕಾಲ ಭೇಟಿ ನೀಡುವಂಥ ಅಂಗಡಿ! ಇದನ್ನು ನರೇಂದ್ರ ಹೇಗೋ ಪತ್ತೆ ಹಚ್ಚಿಬಿಟ್ಟಿದ್ದಾನೆ. ರಾಜ್​ಕುಮಾರನ ಬರವನ್ನೇ ಎದುರುನೋಡುತ್ತ ಅವನು ಮರೆಯಲ್ಲಿ ನಿಂತ. ಸೂರ್ಯ ಮುಳುಗಿ ನಸುಗತ್ತಲಾಗುತ್ತಲೇ ಅಲ್ಲಿ ನುಸುಳಿದ ರಾಜ್​ಕುಮಾರನಿಗೆ ಗಾಬರಿಯಾಯಿತು. ಆದರೂ ಸಾಧ್ಯವಾದಷ್ಟು ಸಾವರಿಸಿಕೊಂಡು, “ಏ... ಏನು ದತ್ತ! ಇಲ್ಲಿದ್ದೀಯ!! ಇಲ್ಲಿಗೇಕೆ ಬಂದುಬಿಟ್ಟೆ?” ಎಂದು ತೊದಲಿದ. ಆಗ ನರೇಂದ್ರ ಶಾಂತವಾಗಿ ನುಡಿದ: “ಸರ್, ದಯವಿಟ್ಟು ನೀವು ಹರಿದಾಸನಿಗೆ ಶುಲ್ಕವಿನಾಯಿತಿ ಕೊಡಿಸಿ ಪರೀಕ್ಷೆಗೆ ಕುಳಿತುಕೊಳ್ಳಲು ಅವಕಾಶ ಕೊಡಬೇಕು. ಒಂದು ವೇಳೆ ನೀವು ಇದಕ್ಕೆ ಒಪ್ಪಿಕೊಳ್ಳದೆ ಹೋದರೆ ನಿಮ್ಮ ಗುಟ್ಟು ಬಯಲಾಗುತ್ತದೆ!” ಗುಮಾಸ್ತೆ ಬೆವತು ಹೋದ. ಅವನು ಅಫೀಮಿನ ಚಟ ಅಂಟಿಸಿಕೊಂಡಿದ್ದರೂ ಕಾಲೇಜಿನಲ್ಲಿ, ಸಮಾಜದಲ್ಲಿ ಸಭ್ಯಸ್ಥನ ಮುಖವಾಡ ಹಾಕಿಕೊಂಡಿದ್ದ. ಈಗ ನರೇಂದ್ರ ಹೀಗೆಂದಾಗ ಅವನು ಪೂರ್ತಿ ಇಳಿದುಹೋಗಿ ನಯವಾದ ದನಿಯಲ್ಲಿ ನುಡಿದ: “ಏನಪ್ಪಾ ನರೇಂದ್ರ! ಅದಕ್ಕೆಲ್ಲ ಕೋಪಿಸಿಕೊಂಡುಬಿಟ್ಟೆಯಾ! ನಿನ್ನಿಷ್ಟದಂತೆಯೇ ಮಾಡೋಣ. ನೀನು ಕೇಳಿದ ಮೇಲೆ‘ಇಲ್ಲ’ಎನ್ನುವುದಕ್ಕಾಗುತ್ತದೆಯೆ...ಆದರೆ ನೋಡಪ್ಪ, ಅವನು ಕೊಡಬೇಕಾದ ಕಾಲೇಜು ಫೀಸನ್ನೇನೋ ಮಾಫಿ ಮಾಡಬಹುದು; ಪರೀಕ್ಷಾ ಶುಲ್ಕವನ್ನು ಮಾತ್ರ ಕಟ್ಟಬೇಕಾಗುತ್ತದೆ.” ನರೇಂದ್ರ ಇದಕ್ಕೊಪ್ಪಿ ಅವನ ಪಾಡಿಗೆ ಅವನನ್ನು ಬಿಟ್ಟ ಹೊರಟುಹೋದ.

ಮರುದಿನ ಮುಂಜಾವಿನಲ್ಲೇ ಅವನು ಹರಿದಾಸನ ಮನೆಗೆ ಹೋಗಿ ಉಲ್ಲಾಸದಿಂದ ಒಂದು ಹಾಡು ಹೇಳಾರಂಭಿಸಿದ. ಅದರ ಅರ್ಥ ಹೀಗಿದೆ:

‘ಓ ಗೆಳೆಯ, ಈ ಮಧುರ ಮುಂಜಾನೆಯಲ್ಲಿ ಎದ್ದು ಪರಮಾತ್ಮನ ಅಪಾರ ವೈಭವದ ಮೇಲೆ ಧ್ಯಾನಮಾಡು. ಅದೋ ನೋಡು, ಆ ದೂರದ ಪರ್ವತಶಿಖರಗಳ ಮಧ್ಯದಲ್ಲಿ ಅರುಣಕಿರಣಧಾರಿ ಯಾದ ಸುಂದರ ಸೂರ್ಯ ಮೆಲ್ಲನೆ ಮೇಲೇರಿ ಬರುತ್ತಿದ್ದಾನೆ! ಇಂದಿನ ಈ ಮಂಗಲಕರ ದಿನದಂದು ಎಂಥ ಮಂಜುಳ ಮಂದ ಮಾರುತ ಬೀಸುತ್ತಿದೆ ನೋಡು! ಈ ಮಂದಮಾರುತವೂ ಕೂಡ ಭಗವಂತನ ವೈಭವವನ್ನು ಮಧುರವಾಗಿ ಹಾಡುತ್ತ, ಮಧುವನ್ನೇ ಸೂಸುತ್ತಿದೆ ನೋಡು! ನಾವೆಲ್ಲ ನಮ್ಮ ಹೃದಯದಲ್ಲಿ ಪ್ರೇಮಭಕ್ತಿಯ ಕಾಣಿಕೆಯನ್ನು ತುಂಬಿಕೊಂಡು, ಆ ಭಗವಂತನ ಧಾಮಕ್ಕೆ ಹೋಗೋಣ!”

ಬಳಿಕ, ನರೇಂದ್ರ ತನ್ನ ಗೆಳೆಯನಿಗೆ ಸಂತೋಷದ ಸುದ್ದಿಯನ್ನು ತಿಳಿಸಿ, ಹಿಂದಿನ ಸಂಜೆ ನಡೆದ ಘಟನೆಯನ್ನು ರಸವತ್ತಾಗಿ, ನಾಟಕೀಯವಾಗಿ ಬಣ್ಣಿಸಿದ. ಅದನ್ನು ಕೇಳಿದ ಹರಿದಾಸ ಹೊಟ್ಟೆ ಹುಣ್ಣಾಗುವಷ್ಟು ನಕ್ಕ. ನರೇಂದ್ರನೂ ಆ ನಗುವಿನಲ್ಲಿ ಸೇರಿಕೊಂಡ. 

ಯುವಕ ನರೇಂದ್ರನ ಸ್ವಭಾವವೆಂಥದು, ಹೃದಯದ ಭಾವಸೂಕ್ಷ್ಮ ಎಂಥದು, ಇಡೀ ವ್ಯಕ್ತಿತ್ವ ವೆಂಥದು ಎನ್ನುವುದು ಆತನ ಒಡನಾಡಿಗಳು ಅವನ ಮೇಲಿಟ್ಟಿದ್ದ ಭಾವನೆಗಳ ಮೂಲಕ ತಿಳಿದುಬರುತ್ತದೆ. ಮುಂದೆ ಅವನ ಒಬ್ಬ ಸ್ನೇಹಿತ ಹೇಳುತ್ತಾನೆ:

“ನರೇಂದ್ರನ ಮಾತು ಕೇಳುವುದಕ್ಕೆ ಅದೆಷ್ಟು ಸೊಗಸು! ಅವನ ಕಂಠಸ್ವರವೇ ಕಿವಿಗಳಿಗೆ ಸಂಗೀತ. ಸುಮ್ಮನೆ ಅವನ ಮಾತುಗಳನ್ನು ಕೇಳಿ ಆನಂದಿಸುವುದಕ್ಕಾಗಿಯೇ ನಾವು ಏನಾದ ರೊಂದು ಚರ್ಚೆಯ ವಿಷಯವನ್ನು ಎತ್ತುತ್ತಿದ್ದೆವು. ಅವನ ವಿಚಾರಧಾರೆ ತುಂಬ ಕುತೂಹಲಕಾರಿ ಯಾಗಿರುತ್ತಿತ್ತು. ಅಲ್ಲದೆ, ಸಂಪೂರ್ಣ ಸ್ವಂತಿಕೆಯಿಂದ ಕೂಡಿದ್ದಾಗಿರುತ್ತಿತ್ತು. ಅವನಿಂದ ಕೇಳಿ ಕೇಳಿಯೇ ನಾವು ಎಷ್ಟೋ ಕಲಿತುಕೊಳ್ಳುತ್ತಿದ್ದೆವು. ಆದರೆ ತನ್ನ ಮಾತನ್ನು ಯಾರಾದರೂ ಪ್ರತಿಭಟಿಸಿದರೋ, ತಕ್ಷಣ ತನ್ನ ಪ್ರಖರ ವಿಚಾರಧಾರೆಯಿಂದ ಮತ್ತು ಶಕ್ತಿಯುತ ಭಾಷಾಶೈಲಿ ಯಿಂದ ಪ್ರತಿವಾದಿಯ ಮೇಲೆರಗಿ ನಿಮಿಷಾರ್ಧದಲ್ಲಿ ಅವನ ಬಾಯಿ ಮುಚ್ಚಿಸಿಬಿಡುತ್ತಿದ್ದ! ದುರ್ಬಲತೆಯನ್ನು ಕಂಡರೇ ಆಗುತ್ತಿರಲಿಲ್ಲ ಅವನಿಗೆ. ನೆಪೋಲಿಯನ್ನನ ಮೇಲೆ ಅವನಿಗೆ ತುಂಬ ಮೆಚ್ಚುಗೆ. ಅವನು ಯಾವಾಗಲೂ ಒಂದು ವಿಷಯವನ್ನು ಒತ್ತಿ ಹೇಳುತ್ತಿದ್ದ, ಏನೆಂದರೆ –ಯಾವುದೇ ಒಂದು ಮಹತ್ಕಾರ್ಯ ಸಾಧ್ಯವಾಗಬೇಕಾದರೆ ಕಾರ್ಯಕರ್ತರೆಲ್ಲರೂ ತಮ್ಮ ನಾಯಕನಲ್ಲಿ ಮರುಮಾತಿಲ್ಲದೆ ವಿಧೇಯತೆಯಿಂದಿರಬೇಕಾಗುತ್ತದೆ, ಅಂತ. ನಮ್ಮ ಪಾಲಿಗೆ ನರೇಂದ್ರ ಕರುಣೆಯ ಮೂರ್ತಿಯೇ ಆಗಿದ್ದ. ಒಮ್ಮೆ ಅವನು ಜ್ವರದಿಂದ ಹಾಸಿಗೆ ಹಿಡಿದಿದ್ದ. ನಾನು ಅವನನ್ನು ನೋಡಿಕೊಂಡುಬರಲು ಹೋದೆ. ಅವನು ಸಂಪೂರ್ಣ ನಿತ್ರಾಣನಾಗಿ ಮಲಗಿದ್ದ. ಆದರೆ ನನ್ನನ್ನು ನೋಡಿದ ಕೂಡಲೇ ಎದ್ದುನಿಂತು ನನ್ನನ್ನು ಸ್ವಾಗತಿಸಿ ಉಪಚರಿಸಲು ಮುಂದಾದ! ನಾನು ಅವನನ್ನು ತಡೆಯಲು ನೋಡಿದೆ, ಆದರೆ ಆತ ಕೇಳಲೇ ಇಲ್ಲ. ನಾನು ಅವನ ಅತಿಥಿಯಂತೆ, ಆದ್ದರಿಂದ ಅವನು ನನ್ನನ್ನು ಸತ್ಕರಿಸಲೇಬೇಕಂತೆ! ತನ್ನ ಕೈಯಿಂದಲೇ ನನಗಾಗಿ ಹುಕ್ಕಾ ತಯಾರಿಸಲು ತೊಡಗಿದ. ಮನೆಯ ಆಳು ಬಂದು ಈ ದೃಶ್ಯವನ್ನು ನೋಡಿ ದಿಗ್ಭ್ರಾಂತನಾಗಿ ನಿಂತ. ಆತನಿಗೆ ನರೇಂದ್ರನನ್ನು ಕಂಡರೆ ಭಕ್ತನಿಗೆ ಸಾಧುವೊಬ್ಬನ ಮೇಲಿರುವಂತಹ ಭಕ್ತಿ-ಪ್ರೇಮ. ನರೇಂದ್ರನೇ ಅವನಿಗೆ ಸರ್ವಸ್ವ. ಆದ್ದರಿಂದ ಆ ಸೇವಕ ನರೇಂದ್ರನನ್ನು ಬೈದು, ಒತ್ತಾಯಪಡಿಸಿ, ಸುಮ್ಮನಿರುವಂತೆ ಕೇಳಿಕೊಂಡ... ”

ಸುಪ್ರಸಿದ್ಧ ಶಿಕ್ಷಣತಜ್ಞರೂ ಪಂಡಿತರೂ ಆದ ಡಾ. ವ್ರಜೇಂದ್ರನಾಥ ಸೀಲರು ತರುಣ ನರೇಂದ್ರನನ್ನು ನಿಕಟವಾಗಿ ಬಲ್ಲವರು; ಕಾಲೇಜಿನಲ್ಲಿ ಅವನಿಗಿಂತ ಒಂದು ವರ್ಷದ ಹಿರಿಯರು. ನರೇಂದ್ರನ ಆಗಿನ ವ್ಯಕ್ತಿತ್ವದ ವೈಭವವನ್ನು ಬಣ್ಣಿಸುತ್ತ ಮುಂದೆ ಇವರು ಹೀಗೆನ್ನುತ್ತಾರೆ:

“ನಿಸ್ಸಂದೇಹವಾಗಿ ಅವನೊಬ್ಬ ಪ್ರತಿಭಾ ಸಂಪನ್ನನಾದ ತರುಣ; ಜನರೊಂದಿಗೆ ಸಹಜ ಸುಲಭವಾಗಿ ಬೆರೆಯಬಲ್ಲ ಸರಳ; ಶ್ರೇಷ್ಠ ದರ್ಜೆಯ ಹಾಡುಗಾರ; ಗೆಳೆಯರ ಗೋಷ್ಠಿಗಳಿಗೆ ಅವನಿದ್ದರೇ ಜೀವಕಳೆ! ಸಂಭಾಷಣ ಕಲೆಯಲ್ಲಿ ಮಹಾಪ್ರಚಂಡ–ಆದರೆ ಮಾತು ಸ್ವಲ್ಪ ಖಾರ; ವ್ಯಂಗ್ಯದ ಮೊನೆಯುಳ್ಳ ತನ್ನ ಚಾಟೂಕ್ತಿಗಳಿಂದ ಸಾಮಾಜಿಕರ ಬಾಹ್ಯಾಡಂಬರದ, ತೋರಾ ಣಿಕೆಯ ಮುಸುಕನ್ನು ಇರಿದು ಈಡಾಡುತ್ತಿದ್ದ. ಮೇಲ್ನೋಟಕ್ಕೆ ವಕ್ರೋಕ್ತಿಯ ನಿಂದಕ; ಆದರೆ ಆ ಸರ್ವತಿರಸ್ಕಾರದ ಮುಸುಕಿನಲ್ಲಿ ಅಡಗಿದ್ದುದು ಮೃದುಲತರ ಹೃದಯ. ನೋಡುವವರ ಕಣ್ಣಿಗೆ ಲಂಗುಲಗಾಮಿಲ್ಲದ ಸ್ವೇಚ್ಛಾಚಾರಿ; ಆದರೆ ಅವನೊಳಗಿತ್ತು–ಉಕ್ಕಿನಂಥ ಇಚ್ಛಾಶಕ್ತಿ. ಮಹಾ ಖಂಡಿತವಾದಿ; ಕಡ್ಡಿ ಮುರಿದಂತೆ ಮಾತು; ಅದರಲ್ಲಿ ಅಧಿಕಾರವಾಣಿಯ ಕಿಮ್ಮತ್ತು ತಾನೇ ತಾನಾಗಿತ್ತು. ಅದರೊಂದಿಗೆ, ಮಾತನಾಡುವಾಗ ಅವನ ಕಣ್ಣಿಂದ ಒಂದು ವಿಚಿತ್ರ ಶಕ್ತಿ ಹೊಮ್ಮುತ್ತಿತ್ತು; ಕೇಳುಗರನ್ನದು ಸರ್ಪಾಸ್ತ್ರದಂತೆ ಕಟ್ಟಿಹಾಕಿಬಿಡುತ್ತಿತ್ತು!... ಆತನಲ್ಲಿ ಪ್ರಖರ ಬುದ್ಧಿ ಸೂಕ್ಷ್ಮತೆಯಿಂದ ಬೆಳಗುತ್ತಿರುವ ಉದಾತ್ತ-ಪರಿಶುದ್ಧ ಚಾರಿತ್ರ್ಯವನ್ನು ನಾನು ಕಂಡು ಕೊಂಡೆ. ಆದರೆ, ಅವನು ಮೊಂಡುಬುದ್ಧಿಯ ಸೋಗಿನ ಮಡಿವಂತನೆಂದು ಇದರ ಅರ್ಥವಲ್ಲ... ಸಂಪ್ರದಾಯಶರಣರನ್ನು, ದೊಡ್ಡಸ್ತಿಕೆಯವರನ್ನು ಅವರ ಸೌಧಗಳಲ್ಲೇ ಛೇಡಿಸಿ ತಬ್ಬಿಬ್ಬು ಮಾಡುವುದೆಂದರೆ ಅವನಿಗೆ ಅದೇನು ಖುಷಿಯೊ! ಅವನು ತನ್ನ ಈ ವರ್ತನೆಯಿಂದ ನಿಕಟ ಗೆಳೆಯರನ್ನು ಬಿಟ್ಟು ಉಳಿದವರ ಪಾಲಿಗೆ ಬಿಡಿಸಲಾರದ ಒಗಟಾಗಿ ತೋರುತ್ತಿದ್ದ.”

ಇದು ತರುಣ ನರೇಂದ್ರನ ವ್ಯಕ್ತಿಚಿತ್ರಣದ ಒಂದು ಭಾಗ ಮಾತ್ರ. ನಿಜಕ್ಕೂ ಅವನ ವ್ಯಕ್ತಿತ್ವ ಅದೆಷ್ಟು ಗಹನವಾದದ್ದು, ಬಹುಮುಖವಾದದ್ದು ಎಂದರೆ ಯಾರಿಗೂ ಅದನ್ನು ಅಳೆಯಲು ಸಾಧ್ಯವಿರಲಿಲ್ಲ.

ಈಗ ನರೇಂದ್ರನ ವಾರ್ಷಿಕ ಪರೀಕ್ಷೆಗಳು ಸಮೀಪಿಸುತ್ತಿವೆ. ಪರೀಕ್ಷೆಯೆಂದರೆ ಎಲ್ಲ ಕಾಲ ದಲ್ಲೂ ವಿದ್ಯಾರ್ಥಿಗಳಿಗೆ ಅಗ್ನಿಪರೀಕ್ಷೆಯೇ ಸರಿ. ಏಕೆಂದರೆ ಅವರ ಭವಿಷ್ಯ ಈ ಪರೀಕ್ಷೆಯನ್ನೇ ಅವಲಂಬಿಸಿಕೊಂಡಿರುತ್ತದೆ. ಹೆಚ್ಚು ಹಣ ತರುವ ಉದ್ಯೋಗ ಸಿಗಬೇಕಾದರೂ ಸನ್ಮಾನ ತರುವ ಪದವಿ-ಪ್ರಶಸ್ತಿ ಪಡೆಯಬೇಕಾದರೂ ಪರೀಕ್ಷೆಯಲ್ಲಿ ಚೆನ್ನಾಗಿ ಉತ್ತೀರ್ಣರಾಗಲೇಬೇಕು. ಸಹಜ ವಾಗಿಯೇ ಅವರು ಇದಕ್ಕಾಗಿ ಬಹಳಷ್ಟು ಶಕ್ತಿಯನ್ನು ವಿನಿಯೋಗಿಸಬೇಕಾಗುತ್ತದೆ. ಭಗೀರಥ ಪ್ರಯತ್ನವನ್ನೇ ಮಾಡಬೇಕಾಗುತ್ತದೆ. ನರೇಂದ್ರ ಮಾತ್ರ ಈ ವಿಷಯದಲ್ಲಿ ಇತರರಂತಲ್ಲ. ಉಳಿದ ವಿದ್ಯಾರ್ಥಿಗಳಂತೆ ಅವನು ವರ್ಷವಿಡೀ ಕಷ್ಟಪಟ್ಟು ಓದುತ್ತಿರಲಿಲ್ಲ. ಕಾಲೇಜು ಅಧ್ಯಯನಕ್ಕಾಗಿ ಅವನು ಕನಿಷ್ಠ ಶ್ರಮವನ್ನಷ್ಟೇ ವಿನಿಯೋಗಿಸುತ್ತಿದ್ದ. ಉಳಿದ ಸಮಯ ಹಾಗೂ ಶಕ್ತಿಯನ್ನೆಲ್ಲ ಪಠ್ಯೇತರ ವಿಷಯಗಳ ಅಧ್ಯಯನ, ಧ್ಯಾನ, ಸಂಗೀತ, ವ್ಯಾಯಾಮ, ಚರ್ಚೆ–ಇವುಗಳಿಗೆ ಉಪಯೋಗಿಸುತ್ತಿದ್ದ. ಇವೆಲ್ಲಕ್ಕಿಂತ ಮುಖ್ಯವಾಗಿ, ಅವನ ಆಧ್ಯಾತ್ಮಿಕ ಪಿಪಾಸೆ ಅವನನ್ನು ಲೌಕಿಕ ವಿದ್ಯೆಯ ಕಡೆಗೆ ಹೆಚ್ಚಾಗಿ ಗಮನ ಕೊಡಲು ಬಿಡುತ್ತಿರಲಿಲ್ಲ. ವಿಶ್ವವಿದ್ಯಾಲಯಗಳು ನೀಡುವ ಪದವಿಯಂತಹ ಲೌಕಿಕ ಸಂಪತ್ತಿಗಿಂತ ಆಧ್ಯಾತ್ಮಿಕ ಸಂಪತ್ತು ಎನ್ನುವುದು ಎಷ್ಟೋ ಪಾಲು ಘನತರವಾದದ್ದು ಎಂಬ ನಂಬಿಕೆ ಅವನಲ್ಲುಂಟಾಗಿತ್ತು. ಆದ್ದರಿಂದ, ಇಂತಹ ಲೌಕಿಕ ವಿದ್ಯೆ ಯಲ್ಲಿ ತಾನು ಅತ್ಯುತ್ತಮ ಶ್ರೇಣಿಯಲ್ಲಿ ಉತ್ತೀರ್ಣನಾಗಬೇಕು. ಸ್ವರ್ಣಪದಕ ಗಳಿಸಬೇಕು ಎಂಬ ಉತ್ಸಾಹವೇನೂ ಅವನಿಗಿರಲಿಲ್ಲ. ಪರೀಕ್ಷೆ ತೀರ ಹತ್ತಿರ ಬಂದಾಗ, ತೇರ್ಗಡೆಯಾಗಲು ಎಷ್ಟು ಬೇಕೋ ಅಷ್ಟು ಮಾತ್ರ ಓದುತ್ತಿದ್ದ, ಅಷ್ಟೆ. ನಿಜಕ್ಕೂ ಅವನು ತಯಾರಿಯನ್ನು ಪ್ರಾರಂಭಿಸು ತ್ತಿದ್ದುದು ಪರೀಕ್ಷೆಗೆ ಕೆಲವೇ ದಿನಗಳ ಮೊದಲು. ಆದರೆ ಅವನ ಗ್ರಹಣಸಾಮರ್ಥ್ಯ, ನೆನಪಿನ ಶಕ್ತಿ ಎರಡೂ ಅದ್ಭುತವಾದದ್ದು. ಆದ್ದರಿಂದ ಅವನು ಹಾಗೆ ಮಾಡಿದರೂ ನಡೆದುಹೋಗುತ್ತಿತ್ತು.

ಈಗ ನರೇಂದ್ರನ ಬಿ. ಎ. ಪರೀಕ್ಷೆಗೆ ಒಂದೇ ತಿಂಗಳು ಉಳಿದಿದೆ. ಆದರೆ ಪರೀಕ್ಷೆಗೆ ಸೂಚಿತವಾಗಿದ್ದ ಒಂದು ಇತಿಹಾಸಗ್ರಂಥವನ್ನು ಅವನು ಏನೇನೂ ಓದಿರಲಿಲ್ಲ. ಅಲ್ಲದೆ ಅವನ ಹತ್ತಿರ ಆ ಪುಸ್ತಕ ಕೂಡ ಇರಲಿಲ್ಲ. ಆಗ ಹೋಗಿ ಎಲ್ಲಿಂದಲೋ ಒಂದು ಪ್ರತಿಯನ್ನು ತಂದ. ತಂದವನೇ, ಅದನ್ನು ಅಧ್ಯಯನ ಮಾಡಿ ವಿಷಯವನ್ನು ಸ್ವಾಧೀನಪಡಿಸಿಕೊಳ್ಳುವವರೆಗೆ ಹೊರಗೆ ಹೋಗುವುದೇ ಇಲ್ಲ ಎಂದು ವ್ರತಧಾರಿಯಾಗಿ ಕುಳಿತುಬಿಟ್ಟ. ಮತ್ತು ಮೂರೇ ದಿನಗಳಲ್ಲಿ ಆ ಪುಸ್ತಕವನ್ನು ಆಮೂಲಾಗ್ರವಾಗಿ ಅಧ್ಯಯನ ಮಾಡಿ ಮುಗಿಸಿಯೂಬಿಟ್ಟ! ಆದರೆ ಅವನ ಮನೆ ಯಲ್ಲಿ ಜನ ಹೆಚ್ಚು, ಗಲಾಟೆಯೂ ಹೆಚ್ಚು. ಆದ್ದರಿಂದ ಅವನ ಉಳಿದ ಪುಸ್ತಕಗಳ ಅಧ್ಯಯನವನ್ನು ಮುಗಿಸಲು, ಸಮೀಪದಲ್ಲೇ ಇದ್ದ ತನ್ನ ಅಜ್ಜಿಯ ಮನೆಗೆ ಹೋದ. ಅಲ್ಲಿ ಒಂದೆಡೆ ಸ್ಥಿರವಾಗಿ ಕುಳಿತು ಹಗಲೂರಾತ್ರಿ ಅಧ್ಯಯನ ಆರಂಭಿಸಿದ. ಆಯಾ ದಿನ ಎಷ್ಟೆಷ್ಟು ಓದಿ ಮುಗಿಸಬೇಕು ಎಂದು ಗುರುತು ಹಾಕಿಕೊಂಡು, ಅಷ್ಟನ್ನೂ ಓದಿ ಮುಗಿಸದೆ ಅಲ್ಲಿಂದ ಕದಲುವುದಿಲ್ಲ ಎಂದು ತನಗೆ ತಾನೇ ಕಟ್ಟುಪಾಡು ಹಾಕಿಕೊಂಡು, ಕುಳಿತುಬಿಡುತ್ತಿದ್ದ. ಆಗ ತಾವು ಪರೀಕ್ಷೆಗಾಗಿ ಓದಿದ ಕ್ರಮವನ್ನು ಮುಂದೆ ಸ್ವಾಮಿ ವಿವೇಕಾನಂದರು ತಿಳಿಸುತ್ತಾರೆ: “ನಾನು ಪುಸ್ತಕ ಹಿಡಿದುಕೊಂಡು ಕೋಣೆಯಲ್ಲಿ ಕುಳಿತುಬಿಡುತ್ತಿದ್ದೆ. ಓದಿಓದಿ ಮೆದುಳಿಗೆ ದಣಿವಾದಾಗ ಅದನ್ನು ಸ್ವಲ್ಪ ತಣಿಸಲು ಹತ್ತಿರದಲ್ಲೇ ಒಳ್ಳೇ ಸ್ಟ್ರಾಂಗ್ ಚಹಾ ಅಥವಾ ಕಾಫಿಯನ್ನು ಇಟ್ಟುಕೊಂಡಿರುತ್ತಿದ್ದೆ. ರಾತ್ರಿ ಕಣ್ಣಿಗೆ ಮಂಪರು ಕವಿಯುವಂತಾದಾಗ ಕಾಲಿಗೆ ಒಂದು ಹಗ್ಗ ಕಟ್ಟಿಕೊಳ್ಳುತ್ತಿದ್ದೆ. ಓದುತ್ತಿರುವಂತೆಯೇ ನಿದ್ರೆ ಬಂದು ಹಿಂದಕ್ಕೆ ಒರಗಿಕೊಂಡರೆ ಆ ಹಗ್ಗ ತಕ್ಷಣ ನನ್ನನ್ನು ಜಗ್ಗಿ ಎಳೆದಂತಾಗಿ ಎಚ್ಚರಗೊಳ್ಳುತ್ತಿದ್ದೆ, ಮತ್ತೆ ಓದಲು ಆರಂಭಿಸುತ್ತಿದ್ದೆ.”

ಅಜ್ಜಿಯ ಮನೆಯ ತನ್ನ ಕೋಣೆಯಲ್ಲಿ ಓದಿಕೊಳ್ಳುತ್ತಿದ್ದಾಗ ನರೇಂದ್ರ ಮಧ್ಯೆ ಮಧ್ಯೆ ಬಿಡುವಿನ ವೇಳೆಯಲ್ಲಿ ತನ್ನಷ್ಟಕ್ಕೆ ತಾನೇ ಹಾಡಿಕೊಳ್ಳುತ್ತಿದ್ದ. ಆ ಮನೆಯ ಎದುರುಸಾಲಿನಲ್ಲೇ ಇನ್ನೊಂದು ಮನೆಯಲ್ಲಿ ಒಬ್ಬಳು ಬಾಲವಿಧವೆಯಿದ್ದಳು; ಇನ್ನೂ ಚಿಕ್ಕ ವಯಸ್ಸು. ಈಕೆ ತನ್ನ ಕೋಣೆಯ ಕಿಟಕಿಯ ಬಳಿ ನಿಂತು ನರೇಂದ್ರನ ಮಧುರಗಾಯನವನ್ನು ಆಲಿಸುತ್ತಿದ್ದಳು. ಎಷ್ಟೋ ದಿನಗಳಿಂದ ಹೀಗೆಯೇ ನಡೆಯುತ್ತಿತ್ತು. ಅವಳು ನರೇಂದ್ರನನ್ನು ಅವನಿಗೆ ತಿಳಿಯದಂತೆಯೇ ಹಲವಾರು ಸಲ ನೋಡಿದ್ದಳು. ಅವನ ರೂಪ ಕಣ್ಣಿಗೆ ಒಪ್ಪಿತು; ಅವನ ವ್ಯಕ್ತಿತ್ವ ಮನಸ್ಸಿಗೆ ಬಂದಿತ್ತು. ಒಂದು ದಿನ ಮಾತ್ರ ಅವಳು ತನ್ನ ಮನೆಯಿಂದ ಹೊರಟುಬಂದು ನರೇಂದ್ರನ ಕೋಣೆಯ ಬಾಗಿಲಲ್ಲೇ ನಿಂತುಬಿಟ್ಟಿದ್ದಾಳೆ! ಸಂಜೆಯಾದ್ದರಿಂದ ಅರೆಗತ್ತಲು. ನರೇಂದ್ರ ನೋಡುತ್ತಾನೆ–ಅಪರಿಚಿತ ಯುವತಿ ನಿಂತಿದ್ದಾಳೆ; ಅವಳ ಮುಖದಲ್ಲಿ ಶೃಂಗಾರಭಾವ ಮೂಡಿ ಬರುತ್ತಿದೆ... ನರೇಂದ್ರನಿಗೆ ತಬ್ಬಿಬ್ಬಾಯಿತು. ಅವನು ಈ ಮೊದಲು ಅವಳನ್ನೆಂದೂ ಕಂಡವನಲ್ಲ. ಕೂಡಲೇ ಅವನು “ತಾಯೀ, ತಾಯೀ!” ಎಂದು ಉದ್ಗರಿಸುತ್ತ ಅವಳ ಪಾದಕ್ಕೆ ಬಿದ್ದುಬಿಟ್ಟ! ಮತ್ತೆ ಗಟ್ಟಿಯಾಗಿ ನುಡಿದ, “ತಾಯಿ! ಯಾಕಿಲ್ಲಿಗೆ ಬಂದಿರುವೆ? ನೀನು ನನ್ನ ಹೆತ್ತತಾಯಿಗೆ ಸಮ!” ಹುಡುಗಿಗೆ ಅರ್ಥವಾಯಿತು. ಮರುಕ್ಷಣವೇ ಅವಳು ಅಲ್ಲಿಂದ ಕಣ್ಮರೆ ಯಾದಳು. ನರೇಂದ್ರ ತನ್ನ ಜೀವನದ ಒಂದು ಅಗ್ನಿಪರೀಕ್ಷೆಯಲ್ಲಿ ಉತ್ತೀರ್ಣನಾಗಿ ಅಪರಂಜಿ ಯಂತೆ ಹೊರಬಂದಿದ್ದ. ಮರುದಿನವೇ ಅವನು ಆ ಮನೆಯನ್ನು ಬಿಟ್ಟುಬಿಟ್ಟ. ಮತ್ತೆ ಆ ಕೋಣೆಗೆ ಅವನು ಕಾಲುಹಾಕಲೇ ಇಲ್ಲ.

ಬಿ. ಎ. ಪರೀಕ್ಷೆಗೆ ನರೇಂದ್ರನ ತಯಾರಿಯೆಲ್ಲ ಮುಗಿಯಿತು. ಆದರೆ ಪರೀಕ್ಷೆಯ ಹಿಂದಿನ ದಿನ ಬೆಳಗ್ಗೆ ಅವನೊಂದು ವಿಚಿತ್ರ ಮನಸ್ಥಿತಿಯಲ್ಲಿದ್ದ. ‘ಈ ಪುಸ್ತಕಗಳನ್ನೆಲ್ಲ ಎಸೆದುಬಿಡೋಣ; ಈ ವಿದ್ಯೆಯಿಂದ ಏನು ಪ್ರಯೋಜನ?’ ಎನ್ನುವ ಭಾವನೆ ಅವನಲ್ಲುಂಟಾಗಿಬಿಟ್ಟಿತ್ತು. ಇದು ಕೆಲವು ವಿದ್ಯಾರ್ಥಿಗಳಿಗೆ ಪರೀಕ್ಷೆಯ ಸಮಯದಲ್ಲಿ ಉದ್ಭವಿಸುವ ವೈರಾಗ್ಯದಂತಲ್ಲ. ಅವನಲ್ಲಿ ಅಧ್ಯಾತ್ಮಭಾವ ಜಾಗೃತವಾಗಿಬಿಟ್ಟಿದೆ. ಅವನ ಮುಖ ತೇಜಃಪುಂಜವಾಗಿ ಬೆಳಗುತ್ತಿದೆ. ತನ್ನ ಸಹಪಾಠಿಗಳ ಮುಂದೆ ಆತ ಭಾವಭರಿತನಾಗಿ, ಆನಂದಪರವಶನಾಗಿ ಹಾಡುತ್ತಾನೆ:

\begin{verse}
ಬರಿಯ ಮಕ್ಕಳು ನಾವು, ತಿಳಿವು ಸಾಲದು ನಮಗೆ\\ಜ್ಞಾನ ಸಾಗರನಿಹನು ಶರಣಾಗು ಅವಗೆ ॥
\end{verse}

\noindent

ಮತ್ತೆ ಭಗವಂತನನ್ನು ಗುಣಗಾನ ಮಾಡುವ ಹಾಡೊಂದನ್ನು ಹಾಡುತ್ತಾನೆ:

\begin{verse}
ಓ ಪರ್ವತಶಿಖರಗಳೇ ನೀವು ಹಾಡಿರಿ\\ಓಲಾಡುವ ಮೋಡಗಳೇ ನೀವು ಹಾಡಿರಿ\\ಬೀಸುವ ಮರುತ್ತುಗಳೇ ನೀವು ಹಾಡಿರಿ\\ಹಾಡಿ ಹಾಡಿ ದಣಿಯಿರೈ ಅವನ ಮಹಿಮೆಯ ॥\\ಓ ಸೂರ್ಯಚಂದ್ರರೇ, ಹೊಳೆವ ತಾರೆಗಳೇ, \\ಹಾಡಿ ಹಾಡಿ ಕುಣಿಯಿರೈ ಅವನ ಮಹಿಮೆಯ ॥
\end{verse}

\noindent

ಹೀಗೆ ಆ ದಿನ ಬೆಳಗ್ಗೆ ಸುಮಾರು ಒಂಬತ್ತು ಗಂಟೆಯವರೆಗೂ ಆನಂದದಿಂದ ಹಾಡುತ್ತಲೇ ಇದ್ದ; ಸಹಪಾಠಿಗಳ ಜೊತೆಗೆ ಮಾತನಾಡುತ್ತಲೇ ಇದ್ದ. ಆದರೆ ಇತರರು ಅಂತಹ ಸ್ಥಿತಿಗೇರಲು ಸಾಧ್ಯವೆ? ಅದನ್ನು ಅರ್ಥಮಾಡಿಕೊಳ್ಳಲಾದರೂ ಸಾಧ್ಯವೆ? ಕೊನೆಗೆ ಒಬ್ಬ ಸ್ನೇಹಿತ ಮಧ್ಯೆ ಪ್ರವೇಶಿಸಿ, “ನಾಳೆ ಪರೀಕ್ಷೆ ಇದೆ, ನರೇನ್!” ಎಂದು ಹೇಳಿಯೇಬಿಟ್ಟ. ಆದರೆ ನರೇಂದ್ರ ಅದರ ಕಡೆ ಗಮನ ಕೊಡಬೇಕಲ್ಲ! ಮರುದಿನದ ಪರೀಕ್ಷೆಯಲ್ಲಿ ಅವನು ಬರೆದದ್ದೂ ಆಯಿತು, ಉತ್ತೀರ್ಣನೂ ಆದ. ಅವನಂತಹ ಅಸಾಮಾನ್ಯ ಬುದ್ಧಿಶಾಲಿಗೆ ಇಂತಹ ಪರೀಕ್ಷೆಗಳೆಲ್ಲ ಯಾವ ಲೆಕ್ಕ.

