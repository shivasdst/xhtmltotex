\sethyphenation{kannada}{
ಅಂಗಡಿ
ಅಂಗ-ಡಿಗೆ
ಅಂಗಳ
ಅಂಗ-ಸಾ-ಧನೆ
ಅಂಗ-ಸಾ-ಧ-ನೆ-ಯನ್ನು
ಅಂಗಾಂ-ಗ-ಗಳನ್ನು
ಅಂಗೀ-ಕ-ರಿಸಿ
ಅಂಗೈ
ಅಂಚಿಗೆ
ಅಂಚಿ-ನಲ್ಲಿ
ಅಂಟಿ-ಕೊಂ-ಡಿತು
ಅಂಟಿ-ಸಿ-ಕೊಂ-ಡಿ-ದ್ದರೂ
ಅಂತ
ಅಂತ-ರಂಗ
ಅಂತ-ರಂ-ಗದ
ಅಂತ-ರಂ-ಗ-ದಿಂದ
ಅಂತ-ರಾ-ಳ-ದಲ್ಲಿ
ಅಂತಲೂ
ಅಂತಲೇ
ಅಂತ-ಶ್ಶಕ್ತಿ-ಯನ್ನೂ
ಅಂತಹ
ಅಂತೂ
ಅಂತೆಯೇ
ಅಂಥ
ಅಂಥ-ದೇ-ನಾ-ದರೂ
ಅಂಥವ-ನಿಗೆ
ಅಂಥವ-ರಿಗೆ
ಅಂಥವು-ಗಳನ್ನೆಲ್ಲ
ಅಂಥಾ
ಅಂದರೂ
ಅಂದರೆ
ಅಂದಿನ
ಅಂದಿ-ನಿಂದ
ಅಂದು
ಅಂದೂ
ಅಂಬಿ-ಗ-ರನ್ನೇ
ಅಂಬಿ-ಗ-ರಿಂದ
ಅಂಬಿ-ಗ-ರಿಂ-ದಲೂ
ಅಂಬಿ-ಗ-ರಿಗೆ
ಅಂಬಿ-ಗ-ರಿಗೋ
ಅಂಬಿ-ಗರು
ಅಂಬು
ಅಂಶ
ಅಂಶ-ಗಳನ್ನು
ಅಂಶ-ವನ್ನು
ಅಂಶ-ವೇನೆಂದರೆ
ಅಕ-ಸ್ಮಾ-ತ್ತಾಗಿ
ಅಕ್ಕ
ಅಕ್ಕಂ-ದಿ-ರನ್ನು
ಅಕ್ಕಂ-ದಿ-ರಿಗೆ
ಅಕ್ಕಂ-ದಿರು
ಅಕ್ಕ-ಪಕ್ಕ-ದಲ್ಲಿ
ಅಕ್ಕಿ
ಅಕ್ಬರ್
ಅಕ್ಷ-ರ-ಗಳನ್ನು
ಅಕ್ಷ-ರಶಃ
ಅಕ್ಷ-ರಾ-ಭ್ಯಾಸ
ಅಗತ್ಯ-ವಿ-ದ್ದಷ್ಟು
ಅಗತ್ಯವೂ
ಅಗಳಿ
ಅಗ-ಳಿ-ಹಾ-ಕಿತ್ತು
ಅಗಾ-ಧ-ವಾ-ದುದು
ಅಗೋ-ಚರ
ಅಗ್ನಿ-ಜ್ವಾಲೆ
ಅಗ್ನಿ-ಪರೀಕ್ಷೆ--ಯಲ್ಲಿ
ಅಗ್ನಿ-ಪರೀಕ್ಷೆ-ಯೇ
ಅಗ್ನಿ-ಮಾಂ-ದ್ಯ-ದಿಂ-ದಾ-ಗಿ-ಎಂ-ದರೆ
ಅಗ್ರ-ಗಣ್ಯ
ಅಚ್ಚ-ಳಿ-ಯದ
ಅಚ್ಚು-ಮೆ-ಚ್ಚಿನ
ಅಚ್ಚು-ಮೆಚ್ಚು
ಅಜೀರ್ಣ
ಅಜೀ-ರ್ಣ-ದಿಂ-ದಾಗಿ
ಅಜ್ಜಿ
ಅಜ್ಜಿಯ
ಅಜ್ಜಿಯೂ
ಅಟ್ಟಿ-ಬಿಟ್ಟ
ಅಟ್ಟಿ-ಸಿ-ಕೊಂಡು
ಅಡ-ಕ-ವಾ-ಗಿ-ರ-ಬ-ಹು-ದಾದ
ಅಡಗಿ-ಕೊಂ-ಡಿತ್ತು
ಅಡಗಿ-ದ್ದುದು
ಅಡಿಗೆ
ಅಡಿ-ಗೆಗೆ
ಅಡಿ-ಗೆ-ಮ-ನೆ-ಯೊ-ಳಗೆ
ಅಡಿ-ಗೆಯ
ಅಡ್ಡ
ಅಡ್ಡ-ಗಟ್ಟಿ
ಅಡ್ಡ-ದಾರಿ-ಗೆ-ಳೆ-ಯು-ವಂ-ತಹ
ಅಡ್ಡಾ-ಡು-ತ್ತಿ-ರು-ವುದು
ಅಣ-ಕಿ-ಸುತ್ತ
ಅಣು-ಕು-ಚೇ-ಷ್ಟೆ-ಗಳನ್ನು
ಅಣ್ಣನ
ಅತಿ
ಅತಿ-ಥಿ-ಗ-ಳಾಗಿ
ಅತಿ-ಥಿ-ಗ-ಳಿಗೆ
ಅತಿ-ಥಿ-ಯಂತೆ
ಅತಿ-ಧಾ-ರಾ-ಳ-ತ-ನ-ವನ್ನು
ಅತಿ-ಯಾ-ಗಿ-ರ-ಬೇಕು
ಅತಿ-ರೇ-ಕ-ವನ್ನು
ಅತಿ-ಶ-ಯ-ವಾಗಿ
ಅತಿ-ಶ-ಯೋ-ಕ್ತಿ-ಯಲ್ಲ
ಅತೀಂ-ದ್ರಿಯ
ಅತ್ತ
ಅತ್ತಿತ್ತ
ಅತ್ಯಂತ
ಅತ್ಯ-ಪೂರ್ವ
ಅತ್ಯ-ಮೂ-ಲ್ಯ-ವಾದ
ಅತ್ಯಾ-ಸ-ಕ್ತಿ-ಯಿಂದ
ಅತ್ಯು-ತ್ತಮ
ಅತ್ಯು-ತ್ಸಾ-ಹ-ದಿಂದ
ಅಥವಾ
ಅದ
ಅದ-ಕ್ಕ-ನು-ಗು-ಣ-ವಾ-ಗಿಯೇ
ಅದ-ಕ್ಕಾಗಿ
ಅದ-ಕ್ಕಿಂತ
ಅದಕ್ಕೂ
ಅದಕ್ಕೆ
ಅದ-ಕ್ಕೆಲ್ಲ
ಅದ-ಕ್ಕೊ-ಪ್ಪ-ಲಿಲ್ಲ
ಅದ-ಕ್ಕೊಪ್ಪಿ
ಅದನ್ನು
ಅದನ್ನೇ
ಅದಮ್ಯ
ಅದರ
ಅದ-ರಂ-ತೆಯೇ
ಅದ-ರಲ್ಲಿ
ಅದ-ರ-ಲ್ಲಿದ್ದ
ಅದ-ರಲ್ಲೂ
ಅದ-ರಲ್ಲೇ
ಅದ-ರ-ಲ್ಲೊಂದು
ಅದ-ರಿಂದ
ಅದ-ರೊಂ-ದಿಗೆ
ಅದು
ಅದೂ
ಅದೃಷ್ಟ
ಅದೃ-ಷ್ಟಕ್ಕೆ
ಅದೃ-ಷ್ಟ-ವ-ಶಾತ್
ಅದೆಷ್ಟು
ಅದೆಷ್ಟೋ
ಅದೇ
ಅದೇಕೆ
ಅದೇನು
ಅದೇನೋ
ಅದೊಂದು
ಅದೊಂದೂ
ಅದೋ
ಅದ್ಭುತ
ಅದ್ಭು-ತ-ವನ್ನು
ಅದ್ಭು-ತ-ವಾದ
ಅದ್ಭು-ತ-ವಾ-ದದ್ದು
ಅದ್ಭು-ತ-ವಾ-ದದ್ದೇ
ಅಧಿ-ಕಾರ
ಅಧಿ-ಕಾ-ರ-ವನ್ನು
ಅಧಿ-ಕಾ-ರ-ವಾ-ಣಿಯ
ಅಧಿ-ಕಾ-ರ-ವಾ-ಣಿ-ಯಿಂದ
ಅಧಿ-ಕಾ-ರಿ-ಗಳ
ಅಧಿ-ದೇ-ವ-ತೆಯೇ
ಅಧ್ಯಕ್ಷ
ಅಧ್ಯ-ಕ್ಷ-ಮ-ಹೋ-ದಯ
ಅಧ್ಯ-ಕ್ಷ-ರ-ನ್ನಾಗಿ
ಅಧ್ಯ-ಕ್ಷರು
ಅಧ್ಯ-ಯನ
ಅಧ್ಯ-ಯ-ನ-ಕ್ಕಾಗಿ
ಅಧ್ಯ-ಯ-ನ-ವನ್ನು
ಅಧ್ಯ-ಯಿಸಿ
ಅಧ್ಯಾ-ತ್ಮಕ್ಕೆ
ಅಧ್ಯಾ-ತ್ಮ-ಭಾವ
ಅಧ್ಯಾ-ಪ-ಕರ
ಅಧ್ಯಾ-ಪ-ಕ-ರಿಗೆ
ಅಧ್ಯಾ-ಪ-ಕರು
ಅಧ್ಯಾ-ಪ-ಕ-ರೆಲ್ಲ
ಅಧ್ಯಾ-ಯ-ಗಳನ್ನು
ಅನಂತ
ಅನಂ-ತರ
ಅನ-ಕ್ಷ-ರ-ತೆಯ
ಅನಾ-ದಿ-ಯಿಂದ
ಅನಾ-ರೋ-ಗ್ಯ-ನಿ-ಶ್ಶಕ್ತಿ-ಗಳ
ಅನಾ-ಹು-ತದ
ಅನಿ-ಲದ
ಅನಿಷ್ಟ
ಅನಿ-ಸಿಕೆ
ಅನಿ-ಸಿ-ಕೆ-ಗಳೇ
ಅನು-ಕ-ರಣೆ
ಅನು-ಕೂಲ
ಅನು-ಕೂ-ಲ-ಕ-ರ-ವಾ-ಗಿತ್ತು
ಅನು-ಕೂ-ಲ-ವಿಲ್ಲ
ಅನು-ಪಮ
ಅನು-ಭವ
ಅನು-ಭ-ವ-ಗಳೇ
ಅನು-ಭ-ವ-ವನ್ನು
ಅನು-ಭ-ವ-ವಾ-ಗು-ತ್ತಿತ್ತು
ಅನು-ಭ-ವ-ವಾ-ಗು-ವುದನ್ನು
ಅನು-ಭ-ವ-ವೇನೂ
ಅನು-ಭ-ವಿ-ಸಿತು
ಅನು-ಭ-ವಿ-ಸಿ-ದೆನೋ
ಅನು-ಭ-ವಿ-ಸು-ತ್ತಿ-ದ್ದುದು
ಅನು-ಭ-ವಿ-ಸುವ
ಅನು-ಭ-ವಿ-ಸು-ವಂ-ತಾ-ಗು-ತ್ತದೆ
ಅನು-ಭ-ವಿ-ಸು-ವಂತೆ
ಅನು-ಮತಿ
ಅನು-ಮ-ತಿ-ಪತ್ರ
ಅನು-ಮ-ತಿ-ಪ-ತ್ರ-ಗಳು
ಅನು-ಮ-ತಿ-ಯನ್ನು
ಅನು-ಸ-ರಿ-ಸ-ಬೇ-ಕಾದ
ಅನೇಕ
ಅನೇ-ಕ-ರಿ-ದ್ದರು
ಅನೇ-ಕಾ-ನೇಕ
ಅನೈ-ತಿಕ
ಅನ್ನ
ಅನ್ನಿ-ಸ-ತೊ-ಡ-ಗಿ-ದೆ-ಭ-ಗ-ವಂ-ತ-ನನ್ನು
ಅನ್ನುತ್ತಿ
ಅನ್ಯಾ-ಯದ
ಅನ್ಯಾ-ಯ-ವಾ-ಗ-ಬ-ಹುದು
ಅಪ-ಖ್ಯಾತಿ
ಅಪ-ಘಾತ
ಅಪ-ಘಾ-ತ-ದಿಂ-ದಾಗಿ
ಅಪ-ಮಾನ
ಅಪ-ರಂಜಿ
ಅಪ-ರಾಧ
ಅಪ-ರಾ-ಧ-ಗಳನ್ನು
ಅಪ-ರಿ-ಮಿತ
ಅಪ-ರೂಪ
ಅಪ-ರೂ-ಪದ್ದು
ಅಪ-ವಾ-ದ-ವಾ-ಗಿ-ರ-ಲಿಲ್ಲ
ಅಪಾ-ಯಕ್ಕೆ
ಅಪಾ-ಯದ
ಅಪಾ-ಯ-ದಿಂದ
ಅಪಾರ
ಅಪಾ-ರ-ವಾದ
ಅಪಾ-ರ್ಥ-ಮಾಡಿ
ಅಪೂರ್ವ
ಅಪೂ-ರ್ವ-ವಾ-ದದ್ದು
ಅಪ್ಪ
ಅಪ್ಪಾ
ಅಪ್ರ-ತಿಮ
ಅಪ್ರ-ತಿ-ಹ-ತ-ವಾದ
ಅಪ್ರಿಯ
ಅಫ-ಘಾ-ನಿ-ಸ್ತಾ-ನ-ದ-ವ-ರೆಗೆ
ಅಫೀ-ಮಿನ
ಅಭಾ-ವ-ವೆ-ನ್ನು-ವುದು
ಅಭಿ-ನಂ-ದ-ನೆ-ಗಳನ್ನು
ಅಭಿ-ನಂ-ದಿ-ಸಿ-ದರು
ಅಭಿ-ಪ್ರಾಯ
ಅಭಿ-ಪ್ರಾ-ಯ-ಗಳನ್ನು
ಅಭಿ-ಪ್ರಾ-ಯ-ಭೇ-ದ-ವೇ-ರ್ಪಟ್ಟು
ಅಭಿ-ಪ್ರಾ-ಯ-ಮೂ-ಡಿತು
ಅಭಿ-ಪ್ರಾ-ಯ-ವನ್ನು
ಅಭಿ-ಪ್ರಾ-ಯ-ವನ್ನೂ
ಅಭಿ-ಪ್ರಾ-ಯ-ವೇ-ನೆಂ-ಬುದು
ಅಭಿ-ಮತ
ಅಭಿ-ಮ-ತ-ವಾ-ಗಿತ್ತು
ಅಭಿ-ಮಾನ
ಅಭಿ-ಮಾ-ನಕ್ಕೂ
ಅಭಿ-ಮಾ-ನಕ್ಕೆ
ಅಭಿ-ಮಾ-ನ-ವಿ-ಟ್ಟಿ-ದ್ದಾರೆ
ಅಭಿ-ರುಚಿ
ಅಭಿ-ಲಾಷೆ
ಅಭಿ-ವೃ-ದ್ಧಿ-ಗಾಗಿ
ಅಭ್ಯ-ಸಿ-ಸಿದ
ಅಭ್ಯಾ-ಗ-ತರ
ಅಭ್ಯಾಸ
ಅಭ್ಯಾ-ಸ-ದಲ್ಲಿ
ಅಭ್ಯಾ-ಸ-ಮಾಡು
ಅಮಿತ
ಅಮೀನ
ಅಮೂಲ್ಯ
ಅಮ್ಮ
ಅಮ್ಮಂ-ದಿರೂ
ಅಮ್ಮಾ
ಅಯ-ಸ್ಕಾಂ-ತೀಯ
ಅಯ್ಯಯ್ಯೋ
ಅಯ್ಯೋ
ಅರ-ಗಿ-ಸಿ-ಕೊಂಡ
ಅರ-ಗಿ-ಸಿ-ಕೊಂ-ಡ-ದ್ದ-ರಿಂ-ದಲೇ
ಅರ-ಬರು
ಅರ-ಮನೆ
ಅರ-ಳಲು
ಅರಳಿ
ಅರ-ಳಿ-ಸಿ-ಕೊಂಡು
ಅರ-ಳಿ-ಸಿದ
ಅರ-ಳು-ತ್ತಿ-ರುವ
ಅರ-ಸಿ-ನ-ಕುಂ-ಕುಮ
ಅರ-ಸುತ್ತ
ಅರಿ-ತಿದ್ದ
ಅರಿತು
ಅರಿ-ತು-ಕೊಂಡ
ಅರಿ-ತು-ಕೊ-ಳ್ಳ-ಲಿಲ್ಲ
ಅರಿ-ವಾ-ಯಿತು
ಅರಿವು
ಅರಿವೂ
ಅರು-ಣ-ಕಿ-ರ-ಣ-ಧಾರಿ
ಅರೆ
ಅರೆ-ಗಣ್ಣು
ಅರೆ-ಗ-ತ್ತಲು
ಅರೆ-ಮ-ನ-ಸ್ಸಿನ
ಅರೇ-ಬಿಕ್
ಅರ್ಜಿ-ಗಳನ್ನು
ಅರ್ಜಿ-ಯನ್ನು
ಅರ್ಥ
ಅರ್ಥ-ಭಾ-ವ-ಗಳು
ಅರ್ಥ-ಗ-ರ್ಭಿತ
ಅರ್ಥ-ಪ-ಡಿ-ಸಲು
ಅರ್ಥ-ಪೂರ್ಣ
ಅರ್ಥ-ಮಾ-ಡಿ-ಕೊಂಡು
ಅರ್ಥ-ಮಾ-ಡಿ-ಕೊ-ಳ್ಳ-ಬೇ-ಕೆಂದೇ
ಅರ್ಥ-ಮಾ-ಡಿ-ಕೊ-ಳ್ಳ-ಲಾ-ದರೂ
ಅರ್ಥ-ಮಾ-ಡಿ-ಕೊ-ಳ್ಳಲು
ಅರ್ಥ-ವಲ್ಲ
ಅರ್ಥ-ವಾ-ಗ-ಲಿಲ್ಲ
ಅರ್ಥ-ವಾ-ಗಲೇ
ಅರ್ಥ-ವಾ-ಗು-ತ್ತದೆ
ಅರ್ಥ-ವಾ-ಗು-ತ್ತಿತ್ತು
ಅರ್ಥ-ವಾ-ಗು-ತ್ತಿ-ರ-ಲಿಲ್ಲ
ಅರ್ಥ-ವಾ-ದಾಗ
ಅರ್ಥ-ವಾ-ಯಿತು
ಅರ್ಥ-ವಾ-ಯಿತೋ
ಅರ್ಧ
ಅರ್ಪಿ-ಸಿ-ದರು
ಅಲಂ-ಕ-ರಿಸಿ
ಅಲು-ಗಾ-ಡಿ-ಸಿ-ದಳು
ಅಲು-ಗಾ-ಡಿ-ಸಿ-ಬಿ-ಟ್ಟರು
ಅಲೆ-ಗಳು
ಅಲೆ-ಯ-ಬೇಡ
ಅಲೌ-ಕಿ-ಕ-ತೆ-ಯಿ-ರು-ತ್ತಿತ್ತು
ಅಲ್ಲ
ಅಲ್ಲದೆ
ಅಲ್ಲಲ್ಲೇ
ಅಲ್ಲವೆ
ಅಲ್ಲವೋ
ಅಲ್ಲಾ
ಅಲ್ಲಾ-ಡಿಸಿ
ಅಲ್ಲಿ
ಅಲ್ಲಿ-ಇಲ್ಲಿ
ಅಲ್ಲಿಂದ
ಅಲ್ಲಿಂ-ದಲೇ
ಅಲ್ಲಿಗೆ
ಅಲ್ಲಿ-ಗೆಲ್ಲ
ಅಲ್ಲಿಗೇ
ಅಲ್ಲಿದ್ದ
ಅಲ್ಲಿ-ದ್ದ-ವ-ರ-ಲ್ಲೆಲ್ಲ
ಅಲ್ಲಿ-ದ್ದ-ವರೆಲ್ಲ
ಅಲ್ಲಿನ
ಅಲ್ಲಿ-ಯ-ವ-ರೆಗೆ
ಅಲ್ಲಿಯೇ
ಅಲ್ಲೀಗ
ಅಲ್ಲೇ
ಅಲ್ಲೇ-ನಪ್ಪ
ಅಲ್ಲೊಂದು
ಅಳ-ತೊ-ಡ-ಗಿದ
ಅಳ-ದಿ-ರು-ವ-ವನು
ಅಳ-ಲಾ-ರಂ-ಭಿ-ಸಿದ
ಅಳಿ-ಸಿಯೇ
ಅಳಿ-ಸಿ-ಹೋ-ಗಿ-ರ-ಲಿಲ್ಲ
ಅಳುಕಿ
ಅಳುತ್ತ
ಅಳು-ತ್ತ-ಳು-ತ್ತಲೇ
ಅಳುತ್ತಾ
ಅಳು-ತ್ತಿದ್ದೀ
ಅಳೆ-ಯಲು
ಅವ-ಕಾಶ
ಅವ-ಕಾ-ಶ-ವಾ-ಗ-ಲಿಲ್ಲ
ಅವಗೆ
ಅವ-ತ-ರಿ-ಸಿದ
ಅವ-ತಾ-ರ-ತ-ತ್ತ್ವ-ವನ್ನೂ
ಅವ-ಧಿ-ಯಲ್ಲಿ
ಅವ-ಧಿ-ಯಲ್ಲೇ
ಅವನ
ಅವ-ನಂ-ತಹ
ಅವ-ನಂತೂ
ಅವ-ನಂ-ತೆಯೇ
ಅವ-ನ-ತಿಗೆ
ಅವ-ನ-ದನ್ನು
ಅವ-ನ-ದಲ್ಲ
ಅವ-ನ-ದಾ-ಯಿತು
ಅವ-ನದು
ಅವ-ನದೇ
ಅವ-ನನ್ನು
ಅವ-ನನ್ನೇ
ಅವ-ನಲ್ಲಿ
ಅವ-ನ-ಲ್ಲಿಗೆ
ಅವ-ನ-ಲ್ಲುಂ-ಟಾ-ಗಿತ್ತು
ಅವ-ನ-ಲ್ಲುಂ-ಟಾ-ಗಿ-ಬಿ-ಟ್ಟಿತ್ತು
ಅವ-ನ-ಲ್ಲು-ದಿ-ಸಿತು
ಅವ-ನ-ಲ್ಲೊಂದು
ಅವ-ನ-ಲ್ಲೊಬ್ಬ
ಅವ-ನಾ-ಡಿದ
ಅವನಿ
ಅವ-ನಿಂದ
ಅವ-ನಿ-ಗ-ನ್ನಿ-ಸಿತು
ಅವ-ನಿ-ಗಾಗ
ಅವ-ನಿ-ಗಿಂತ
ಅವ-ನಿ-ಗಿತ್ತು
ಅವ-ನಿ-ಗಿ-ರ-ಲಿಲ್ಲ
ಅವ-ನಿ-ಗೀಗ
ಅವ-ನಿಗೂ
ಅವ-ನಿಗೆ
ಅವ-ನಿ-ಗೆಂದೂ
ಅವ-ನಿ-ಗೇನು
ಅವ-ನಿ-ಗೇನೋ
ಅವ-ನಿ-ಗೊಂದು
ಅವ-ನಿಚ್ಛೆ
ಅವ-ನಿ-ದ್ದರೇ
ಅವ-ನಿನ್ನೂ
ಅವ-ನೀಗ
ಅವನು
ಅವನೂ
ಅವ-ನೆಂದ
ಅವ-ನೆಂ-ದರೆ
ಅವ-ನೆಂದೂ
ಅವ-ನೆ-ಡೆಗೆ
ಅವ-ನೆ-ದೆ-ಯೊ-ಳ-ಗಿಂದ
ಅವನೇ
ಅವ-ನೇನು
ಅವ-ನೇನೋ
ಅವ-ನೊಂದು
ಅವ-ನೊಬ್ಬ
ಅವ-ನೊ-ಬ್ಬನೇ
ಅವ-ನೊ-ಳ-ಗಿ-ತ್ತು-ಉ-ಕ್ಕಿ-ನಂಥ
ಅವ-ನೊ-ಳಗೆ
ಅವನ್ನೂ
ಅವ-ನ್ನೆಲ್ಲ
ಅವ-ಮಾನ
ಅವರ
ಅವ-ರಂತೂ
ಅವ-ರನ್ನು
ಅವ-ರನ್ನೂ
ಅವ-ರ-ನ್ನೆಲ್ಲ
ಅವ-ರನ್ನೇ
ಅವ-ರಲ್ಲಿ
ಅವ-ರಲ್ಲೇ
ಅವ-ರ-ಲ್ಲೊಬ್ಬ
ಅವ-ರ-ಲ್ಲೊ-ಬ್ಬನ
ಅವ-ರಿ-ಗಾಗಿ
ಅವ-ರಿ-ಗಿನ್ನೂ
ಅವ-ರಿಗೆ
ಅವ-ರಿ-ಗೆಲ್ಲ
ಅವ-ರಿ-ಗೇನು
ಅವ-ರಿನ್ನೂ
ಅವ-ರಿ-ಬ್ಬರ
ಅವರು
ಅವರೂ
ಅವ-ರೆಂದೂ
ಅವರೆ-ದುರು
ಅವರೆಲ್ಲ
ಅವರೆ-ಲ್ಲರ
ಅವರೆ-ಲ್ಲ-ರಿ-ಗಿಂ-ತಲೂ
ಅವರೇ
ಅವ-ರೊಂ-ದಿ-ಗೆಲ್ಲ
ಅವರೋ
ಅವ-ಲಂ-ಬಿ-ಸಿ-ಕೊಂ-ಡಿ-ರು-ತ್ತದೆ
ಅವಳ
ಅವ-ಳದು
ಅವ-ಳನ್ನು
ಅವ-ಳ-ನ್ನೆಂದೂ
ಅವ-ಳ-ಲ್ಲಿತ್ತು
ಅವ-ಳಿ-ಗಾ-ಗಲೇ
ಅವ-ಳಿ-ಗಿನ್ನೂ
ಅವ-ಳಿಗೂ
ಅವ-ಳಿಗೆ
ಅವ-ಳಿ-ಗೇನೂ
ಅವಳು
ಅವ-ಳೆದೆ
ಅವ-ಳೊಂದು
ಅವಸ್ಥೆ
ಅವ-ಸ್ಥೆ-ಯನ್ನು
ಅವಿ-ಧೇ-ಯ-ತೆ-ಯನ್ನು
ಅವಿ-ಧೇ-ಯ-ರಾಗಿ
ಅವಿ-ಭಕ್ತ
ಅವಿ-ವಾ-ಹಿ-ತ-ನಾದ
ಅವು
ಅವು-ಗಳ
ಅವು-ಗಳನ್ನು
ಅವು-ಗಳನ್ನೆಲ್ಲ
ಅವು-ಗ-ಳ-ಲ್ಲೆಲ್ಲ
ಅವು-ಗಳಿಂದ
ಅವೆ-ಲ್ಲ-ವನ್ನೂ
ಅವೆ-ಲ್ಲವೂ
ಅಶ್ಲೀಲ
ಅಷ್ಟಕ್ಕೆ
ಅಷ್ಟಕ್ಕೇ
ಅಷ್ಟನ್ನೂ
ಅಷ್ಟ-ರಲ್ಲಿ
ಅಷ್ಟ-ರೊ-ಳಗೇ
ಅಷ್ಟ-ಲ್ಲದೆ
ಅಷ್ಟಾಗಿ
ಅಷ್ಟಿ-ಷ್ಟಲ್ಲ
ಅಷ್ಟು
ಅಷ್ಟು-ದ್ದದ
ಅಷ್ಟು-ರ-ಸ-ವ-ತ್ತಾಗಿ
ಅಷ್ಟೆ
ಅಷ್ಟೆ-ತ್ತ-ರದ
ಅಷ್ಟೇ
ಅಷ್ಟೇಕೆ
ಅಷ್ಟೇನೂ
ಅಷ್ಟೊಂದು
ಅಷ್ಟೊ-ತ್ತಿಗೆ
ಅಸಂ-ಖ್ಯಾತ
ಅಸಂ-ಬ-ದ್ಧದ
ಅಸ-ಹ-ನೀಯ
ಅಸ-ಹಾ-ಯಕ
ಅಸ-ಹಾ-ಯ-ಕ-ನಾದ
ಅಸಾ-ಧಾ-ರಣ
ಅಸಾ-ಮಾನ್ಯ
ಅಸ್ತಿ-ತ್ವ-ವನ್ನೂ
ಅಸ್ತಿ-ತ್ವ-ವನ್ನೇ
ಅಹಂ-ಕಾ-ರದ
ಅಹಂ-ಕಾ-ರಿ-ಯಾ-ಗಿ-ಬಿ-ಟ್ಟಿ-ದ್ದ-ನೆಂದು
ಅಹ-ಮದ್
ಆ
ಆಂಗ್ಲ
ಆಂಗ್ಲಈ
ಆಂಗ್ಲರೂ
ಆಂದೋ-ಲನ
ಆಂದೋ-ಲ-ನ-ಗಳು
ಆಕ-ರ್ಷ-ಕ-ವಾಗಿ
ಆಕ-ರ್ಷಣೆ
ಆಕ-ರ್ಷ-ಣೆ-ಯಿತ್ತು
ಆಕ-ರ್ಷಿ-ತ-ರಾಗಿ
ಆಕ-ರ್ಷಿ-ತ-ರಾ-ಗಿ-ದ್ದರು
ಆಕ-ರ್ಷಿ-ತ-ರಾ-ಗಿ-ಬಿ-ಟ್ಟಿ-ದ್ದರು
ಆಕ-ರ್ಷಿ-ತ-ರಾ-ದರು
ಆಕ-ರ್ಷಿಸಿ
ಆಕ-ರ್ಷಿ-ಸಿದ
ಆಕ-ರ್ಷಿ-ಸಿ-ಬಿ-ಟ್ಟಿದ್ದ
ಆಕ-ಸ್ಮಿಕ
ಆಕಾಂಕ್ಷೆ
ಆಕಾ-ರ-ರ-ಹಿತ
ಆಕಾ-ರವೇ
ಆಕಾಶ
ಆಕಾ-ಶ-ದಂತೆ
ಆಕೃತಿ
ಆಕೃ-ತಿ-ಗಳನ್ನು
ಆಕೃ-ತಿ-ಗಳು
ಆಕೆ
ಆಕೆಯ
ಆಕ್ರ-ಮ-ಣ-ಶೀಲ
ಆಕ್ರ-ಮಿ-ಸಿ-ಬಿ-ಟ್ಟಿದೆ
ಆಖಾ-ಡ-ವನ್ನು
ಆಗ
ಆಗ-ಬೇಕು
ಆಗ-ಮನ
ಆಗ-ಮ-ನ-ವನ್ನೇ
ಆಗಲಿ
ಆಗಲೂ
ಆಗಲೇ
ಆಗ-ಸ-ದೆ-ತ್ತ-ರಕ್ಕೆ
ಆಗಾಗ
ಆಗಿ
ಆಗಿತ್ತು
ಆಗಿದ್ದ
ಆಗಿನ
ಆಗಿ-ನಿಂ-ದಲೂ
ಆಗಿನ್ನೂ
ಆಗಿ-ಬಿ-ಡು-ತ್ತಿದ್ದ
ಆಗಿ-ರ-ಬ-ಹುದು
ಆಗಿ-ರ-ಲಿಲ್ಲ
ಆಗು-ತ್ತಿ-ದ್ದು-ದ-ರಿಂದ
ಆಗು-ತ್ತಿ-ರ-ಲಿಲ್ಲ
ಆಗುವ
ಆಗು-ವು-ದೇನು
ಆಗೆಲ್ಲ
ಆಗ್ತೀ-ನಪ್ಪಾ
ಆಚ-ರಿ-ಸು-ತ್ತಿ-ದ್ದಳು
ಆಚ-ರಿ-ಸು-ವು-ದು-ಇದು
ಆಚಾ-ರ್ಯರು
ಆಚೆಗೆ
ಆಜ್ಞಾ-ಪನೆ
ಆಜ್ಞಾ-ಪಿ-ಸಿ-ದರು
ಆಜ್ಞೆ
ಆಟ
ಆಟ-ಓ-ಡಾ-ಟ-ಗಳು
ಆಟ-ಕ್ಕಿ-ಳಿ-ದರೆ
ಆಟ-ಗ-ಳ-ನ್ನಾ-ಡ-ಬೇ-ಕೆಂಬ
ಆಟ-ಗಳಲ್ಲಿ
ಆಟದ
ಆಟ-ದಲ್ಲಿ
ಆಟ-ದಲ್ಲೂ
ಆಟ-ವನ್ನು
ಆಟ-ವನ್ನೂ
ಆಟ-ವ-ಲ್ಲವೆ
ಆಟ-ವಾ-ಡಿ-ಕೊಂ-ಡಿರು
ಆಟ-ವಾ-ಡು-ತ್ತಿದ್ದ
ಆಟ-ವಾ-ಡು-ತ್ತಿ-ದ್ದ-ವನು
ಆಟ-ವಾ-ಡು-ತ್ತಿ-ದ್ದಾ-ನೆಯೋ
ಆಟ-ವಾದ
ಆಡ-ಬ-ಹುದು
ಆಡಲು
ಆಡಿದ
ಆಡಿ-ರ-ಬ-ಹುದು
ಆಡು
ಆಡು-ತ್ತಿದ್ದ
ಆಡು-ತ್ತಿ-ದ್ದರು
ಆಡು-ತ್ತಿ-ದ್ದಾಗ
ಆಡು-ತ್ತೇನೆ
ಆಡು-ವಾ-ಗಲೂ
ಆಡು-ವುದನ್ನು
ಆಡು-ವು-ದೆಂ-ದರೆ
ಆತ
ಆತಂಕ
ಆತನ
ಆತ-ನನ್ನು
ಆತ-ನಲ್ಲಿ
ಆತ-ನಿಗೂ
ಆತ-ನಿಗೆ
ಆತ್ಮ-ಗೌ-ರ-ವ-ವನ್ನು
ಆತ್ಮ-ವಿ-ಶ್ವಾಸ
ಆತ್ಮ-ವಿ-ಶ್ವಾ-ಸ-ವಿತ್ತು
ಆತ್ಮೀಯ
ಆತ್ಮೀ-ಯ-ವಾಗಿ
ಆದ
ಆದಂ-ತಾ-ಯಿತು
ಆದರ
ಆದ-ರ-ದಿಂದ
ಆದ-ರವೂ
ಆದ-ರೀಗ
ಆದರೂ
ಆದರೆ
ಆದ-ರೇನು
ಆದರ್ಶ
ಆದ-ರ್ಶ-ಗಳನ್ನು
ಆದ-ರ್ಶ-ಗಳು
ಆದ-ರ್ಶದ
ಆದ-ರ್ಶ-ಪ-ರಿ-ಪಾ-ಲ-ನೆ-ಗಾಗಿ
ಆದ-ರ್ಶ-ವನ್ನು
ಆದ-ರ್ಶ-ವಾದಿ
ಆದಿ-ಅಂ-ತ್ಯ-ಗಳ
ಆದೆಲ್ಲ
ಆದೇ-ಶ-ಗಳನ್ನು
ಆದ್ದ-ರಿಂದ
ಆದ್ದ-ರಿಂ-ದಲೇ
ಆಧ್ಯಾ-ತ್ಮಿಕ
ಆಧ್ಯಾ-ತ್ಮಿ-ಕ-ತೆಯ
ಆನಂದ
ಆನಂ-ದಕ್ಕೆ
ಆನಂ-ದ-ದಿಂದ
ಆನಂ-ದ-ಪ-ರ-ವ-ಶ-ನಾಗಿ
ಆನಂ-ದ-ಮೋ-ಹನ
ಆನಂ-ದ-ವ-ನ್ನ-ನು-ಭ-ವಿ-ಸು-ತ್ತಿ-ದ್ದರು
ಆನಂ-ದ-ವನ್ನು
ಆನಂ-ದ-ಸ-ಮಾ-ಧಿಯ
ಆನಂ-ದಾಶ್ರು
ಆನಂ-ದಿ-ಸ-ಬೇ-ಕಾ-ದರೆ
ಆನಂ-ದಿ-ಸಲು
ಆನಂ-ದಿ-ಸು-ವು-ದ-ಕ್ಕಾ-ಗಿಯೇ
ಆಪ-ತ್ತಿಗೆ
ಆಪ್ತ
ಆಪ್ತ-ರಾ-ಗಿಯೇ
ಆಪ್ತರೇ
ಆಫೀ-ಸಿಗೆ
ಆಫೀ-ಸಿನ
ಆಫೀ-ಸಿ-ನಲ್ಲಿ
ಆಫೀಸು
ಆಮಂ-ತ್ರ-ಣ-ವಿ-ತ್ತಿ-ದ್ದರು
ಆಮೂ-ಲಾ-ಗ್ರ-ವಾಗಿ
ಆಮೇಲೆ
ಆಮೋ-ದ-ಪ್ರ-ಮೋ-ದ-ಗಳಲ್ಲಿ
ಆಯಾ
ಆಯಾಸ
ಆಯಿತು
ಆಯು-ಧ-ಗ-ಳಾ-ಗಿ-ದ್ದವು
ಆಯ್ದು-ಕೊಂಡ
ಆಯ್ದು-ಕೊಂ-ಡಿದ್ದ
ಆರಂ-ಭ-ವಾ-ಯಿತು
ಆರಂ-ಭಿ-ಸಿದ
ಆರಂ-ಭಿ-ಸಿ-ದ್ದನ್ನು
ಆರಂ-ಭಿ-ಸು-ತ್ತಿದ್ದೆ
ಆರ-ನೆಯ
ಆರಾ-ಧ-ಕರು
ಆರಾ-ಧ-ನೆ-ಯನ್ನೂ
ಆರಾ-ಧಿ-ಸಲಿ
ಆರಾ-ಧಿಸಿ
ಆರಾಧ್ಯ
ಆರಾ-ಧ್ಯ-ದೇ-ವ-ತೆಯೇ
ಆರಿ-ಸಿ-ಕೊಂಡ
ಆರಿ-ಸಿ-ಕೊಂ-ಡದ್ದು
ಆರಿ-ಸಿ-ಕೊಂ-ಡರೂ
ಆರಿ-ಸಿ-ಕೊಂ-ಡಿದ್ದ
ಆರು
ಆರೋ-ಗ್ಯ-ಶಾ-ಲಿ-ಯಾದ
ಆರೋ-ಪಿ-ಸು-ತ್ತಾರೆ
ಆಲದ
ಆಲಿ-ಸಿದ
ಆಲಿ-ಸು-ತ್ತಿದ್ದ
ಆಲಿ-ಸು-ತ್ತಿ-ದ್ದಳು
ಆಲೋ-ಚನಾ
ಆಲೋ-ಚನೆ
ಆಲೋ-ಚ-ನೆ-ನಂ-ಬಿ-ಕೆ-ಗ-ಳಿಗೆ
ಆಲೋ-ಚ-ನೆ-ಗಳನ್ನೆಲ್ಲ
ಆಲೋ-ಚ-ನೆಗೆ
ಆಲೋ-ಚ-ನೆ-ಯನ್ನು
ಆಲೋ-ಚಿ-ಸ-ತೊ-ಡ-ಗಿದ
ಆಲೋ-ಚಿಸಿ
ಆಲೋ-ಚಿ-ಸಿ-ವಿ-ಮ-ರ್ಶಿಸಿ
ಆಲೋ-ಚಿ-ಸಿದ
ಆಲೋ-ಚಿ-ಸಿ-ದರು
ಆಳ-ಕ್ಕಿ-ಳಿದು
ಆಳ-ವಾಗಿ
ಆಳು
ಆಳು-ಕಾಳು-ಗ-ಳಿ-ದ್ದರು
ಆಳು-ಕಾಳು-ಗಳು
ಆಳ್ವಿ-ಕೆಯ
ಆವ-ರಿ-ಸಿ-ಬಿ-ಡುತ್ತಿ
ಆವ-ಶ್ಯ-ಕತೆ
ಆವ-ಶ್ಯ-ಕ-ತೆಗೆ
ಆವಿ-ಷ್ಕ-ರಿ-ಸಿ-ದರು
ಆವೇ-ಶ-ಭ-ರಿ-ತ-ರಾಗಿ
ಆಶೀ-ರ್ವ-ದಿ-ಸಿ-ದಳು
ಆಶೀ-ರ್ವ-ದಿ-ಸುತ್ತ
ಆಶೀ-ರ್ವಾದ
ಆಶ್ಚರ್ಯ
ಆಶ್ಚ-ರ್ಯ-ಕರ
ಆಶ್ಚ-ರ್ಯ-ಕು-ತೂ-ಹಲ
ಆಶ್ಚ-ರ್ಯದ
ಆಶ್ಚ-ರ್ಯ-ದಿಂದ
ಆಶ್ಚ-ರ್ಯ-ವನ್ನು
ಆಶ್ಚ-ರ್ಯ-ವಾ-ಗ-ದಿ-ರ-ಲಿಲ್ಲ
ಆಶ್ಚ-ರ್ಯ-ವಾ-ಗಿ-ರ-ಬೇಕು
ಆಶ್ಚ-ರ್ಯ-ವಿಲ್ಲ
ಆಶ್ಚ-ರ್ಯವೂ
ಆಶ್ಚ-ರ್ಯ-ವೇ-ನಿದೆ
ಆಶ್ಚ-ರ್ಯ-ವೇನೂ
ಆಶ್ರಮ
ಆಶ್ರಯ
ಆಶ್ರ-ಯಕ್ಕೆ
ಆಶ್ರ-ಯ-ದಲ್ಲಿ
ಆಸಕ್ತಿ
ಆಸ-ಕ್ತಿ-ಯಿಂದ
ಆಸ-ನ-ಗಳನ್ನು
ಆಸೆ
ಆಸೆ-ಪಟ್ಟು
ಆಸೆ-ಯಾಗಿ
ಆಸೆ-ಯಾಗು
ಆಸ್ತಿ
ಆಸ್ತಿ-ಪಾ-ಸ್ತಿಯ
ಆಸ್ತಿ-ಯೆಲ್ಲ
ಆಸ್ಥಾ-ನಿ-ಕ-ರನ್ನು
ಆಹಾ
ಆಹಾರ
ಆಹಾ-ರದ
ಆಹ್ವಾ-ನಿ-ತ-ರಿ-ಗೆಲ್ಲ
ಆಹ್ವಾ-ನಿ-ಸ-ಲಾ-ಯಿತು
ಇಂಗಿತ
ಇಂಗ್ಲಿ-ಷನ್ನು
ಇಂಗ್ಲಿ-ಷಿ-ನಲ್ಲೇ
ಇಂಗ್ಲಿಷ್
ಇಂಗ್ಲೆಂ-ಡಿನ
ಇಂಚ-ರ-ಗಳಿಂದ
ಇಂತಹ
ಇಂತಿಂಥ
ಇಂಥ
ಇಂಥಾ
ಇಂದಿನ
ಇಂದು
ಇಂದೇ
ಇಂದೇಕೆ
ಇಂದ್ರಿ-ಯ-ಜೀ-ವ-ನದ
ಇಕ್ಕೆ-ಲ-ಗ-ಳ-ಲ್ಲಿನ
ಇಚ್ಛಾ-ಶಕ್ತಿ
ಇಚ್ಛೆ
ಇಚ್ಛೆಗೆ
ಇಟ್ಟಿದ್ದ
ಇಟ್ಟಿ-ರು-ವುದು
ಇಟ್ಟು-ಕೊಂ-ಡಿ-ರು-ತ್ತಿದ್ದೆ
ಇಟ್ಟು-ಕೊಂಡು
ಇಟ್ಟು-ಕೊ-ಳ್ಳು-ತ್ತಿದ್ದ
ಇಡಿಯ
ಇಡೀ
ಇಡು-ತ್ತಿದ್ದ
ಇಡುವು
ಇತರ
ಇತ-ರರ
ಇತ-ರ-ರಂ-ತಲ್ಲ
ಇತ-ರ-ರನ್ನು
ಇತ-ರ-ರನ್ನೂ
ಇತ-ರ-ರಿಂದ
ಇತ-ರ-ರಿಗೆ
ಇತ-ರರು
ಇತ-ರರೂ
ಇತ-ರ-ರೆಲ್ಲ
ಇತ-ರೆಲ್ಲ
ಇತಿ-ಹಾಸ
ಇತಿ-ಹಾ-ಸ-ಗ್ರಂ-ಥ-ವನ್ನು
ಇತಿ-ಹಾ-ಸದ
ಇತಿ-ಹಾ-ಸ-ದಲ್ಲಿ
ಇತಿ-ಹಾ-ಸ-ದ-ಲ್ಲೊಂದು
ಇತಿ-ಹಾ-ಸ-ವನ್ನು
ಇತಿ-ಹಾ-ಸ-ವನ್ನೂ
ಇತ್ತ
ಇತ್ತ-ಕಡೆ
ಇತ್ತಾ-ದರೂ
ಇತ್ತು
ಇತ್ಯರ್ಥ
ಇತ್ಯಾದಿ
ಇದ
ಇದ-ಕ್ಕಾಗಿ
ಇದಕ್ಕೂ
ಇದಕ್ಕೆ
ಇದ-ಕ್ಕೊಂದು
ಇದ-ಕ್ಕೊಪ್ಪಿ
ಇದನ್ನು
ಇದ-ನ್ನೆಲ್ಲ
ಇದನ್ನೇ
ಇದರ
ಇದ-ರಲ್ಲಿ
ಇದ-ರ-ಲ್ಲೇ-ನಾ-ದರೂ
ಇದ-ರಿಂದ
ಇದ-ರಿಂ-ದಾಗಿ
ಇದ-ರೊಂ-ದಿಗೆ
ಇದ-ಲ್ಲದೆ
ಇದಾದ
ಇದಾ-ವು-ದಕ್ಕೂ
ಇದಿ-ರಾಗಿ
ಇದು
ಇದೂ
ಇದೆ
ಇದೆಂಥ
ಇದೆ-ಯ-ಲ್ಲವೆ
ಇದೆ-ಯೆಂದು
ಇದೆಯೋ
ಇದೆಲ್ಲ
ಇದೆ-ಲ್ಲ-ಕ್ಕಿಂತ
ಇದೇ
ಇದೇ-ನಪ್ಪ
ಇದೇ-ನಿದು
ಇದೊಂದು
ಇದ್ದ
ಇದ್ದಂ-ತಿ-ರ-ಲಿಲ್ಲ
ಇದ್ದಂ-ತಿಲ್ಲ
ಇದ್ದ-ಕ್ಕಿ-ದಂತೆ
ಇದ್ದ-ಕ್ಕಿ-ದ್ದಂತೆ
ಇದ್ದರೂ
ಇದ್ದರೆ
ಇದ್ದ-ವ-ರಿಗೆ
ಇದ್ದಾನೆ
ಇದ್ದಾರೆ
ಇದ್ದಿ-ದ್ದರೆ
ಇದ್ದೀಯ
ಇದ್ದು
ಇದ್ದು-ದನ್ನು
ಇದ್ದು-ದೆಂ-ದರೆ
ಇದ್ದು-ಬಿಟ್ಟ
ಇದ್ದುವು
ಇದ್ದೆನೋ
ಇದ್ದೇ
ಇದ್ದೇನೆ
ಇನ್ನಷ್ಟು
ಇನ್ನಾ-ರದೂ
ಇನ್ನಾರೂ
ಇನ್ನಾ-ವುದೇ
ಇನ್ನು
ಇನ್ನೂ
ಇನ್ನೂ-ಬ್ಬಳು
ಇನ್ನೆ-ಷ್ಟೆಷ್ಟೋ
ಇನ್ನೆಷ್ಟೋ
ಇನ್ನೇ-ನಾ-ದರೂ
ಇನ್ನೇನು
ಇನ್ನೇನೂ
ಇನ್ನೇ-ನೇನು
ಇನ್ನೇ-ನ್ಮಾ-ಡತ್ತೆ
ಇನ್ನೈದು
ಇನ್ನೊಂ-ದ-ರತ್ತ
ಇನ್ನೊಂದು
ಇನ್ನೊಂ-ದು-ಕ-ರುಣೆ
ಇನ್ನೊಂ-ದು-ಧ್ಯಾ-ನಸ್ಥ
ಇನ್ನೊಬ್ಬ
ಇನ್ನೊ-ಬ್ಬನು
ಇನ್ನೊ-ಬ್ಬರ
ಇನ್ನೊಮ್ಮೆ
ಇಪ್ಪ-ತ್ತ-ನಾಲ್ಕು
ಇಪ್ಪತ್ತು
ಇಪ್ಪ-ತ್ತೈ-ದ-ನೆಯ
ಇಬ್ಬರ
ಇಬ್ಬ-ರನ್ನೂ
ಇಬ್ಬರು
ಇಬ್ಬರೂ
ಇಬ್ಭಾಗ
ಇರ-ಬೇ-ಕಾ-ಗಿ-ತ್ತಾ-ದ್ದ-ರಿಂದ
ಇರ-ಬೇಕು
ಇರಲಿ
ಇರ-ಲಿಲ್ಲ
ಇರಿದು
ಇರಿ-ಸ-ಲಾ-ಗಿತ್ತು
ಇರಿ-ಸಿ-ಕೊಂಡು
ಇರು
ಇರು-ತ್ತದೆ
ಇರು-ತ್ತಿತ್ತು
ಇರು-ತ್ತಿದ್ದ
ಇರು-ತ್ತಿ-ದ್ದ-ವನು
ಇರು-ತ್ತಿ-ದ್ದುದು
ಇರು-ತ್ತಿದ್ದೆ
ಇರು-ತ್ತಿ-ರ-ಲಿಲ್ಲ
ಇರುವ
ಇರು-ವವ
ಇರು-ವು-ದಾ-ದರೆ
ಇರು-ವುದು
ಇರು-ವುದೇ
ಇಲಿ-ಗಳ
ಇಲಿ-ಗ-ಳು-ಇ-ವೆಲ್ಲ
ಇಲ್ಲ
ಇಲ್ಲ-ಎಂ-ದು-ಬಿಟ್ಟ
ಇಲ್ಲದ
ಇಲ್ಲ-ದಿ-ದ್ದ-ರಾ-ಯಿತು
ಇಲ್ಲ-ದಿ-ದ್ದರೆ
ಇಲ್ಲದೆ
ಇಲ್ಲ-ವಲ್ಲ
ಇಲ್ಲ-ವಾ-ಯಿತು
ಇಲ್ಲವೆ
ಇಲ್ಲ-ವೆಂ-ಬಂತೆ
ಇಲ್ಲವೋ
ಇಲ್ಲಿ
ಇಲ್ಲಿಂದ
ಇಲ್ಲಿಗೆ
ಇಲ್ಲಿ-ಗೇಕೆ
ಇಲ್ಲಿ-ದ್ದೀಯ
ಇಲ್ಲಿನ
ಇಲ್ಲಿಯೂ
ಇಲ್ಲಿ-ರು-ವುದು
ಇಲ್ಲೆಲ್ಲ
ಇಲ್ಲೇ
ಇಳಿ-ದಾಗ
ಇಳಿ-ದು-ಹೋಗಿ
ಇಳಿ-ಯಲು
ಇಳಿ-ಯಿರಿ
ಇಳಿ-ಯು-ತ್ತವೆ
ಇಳಿ-ಯು-ತ್ತಿದೆ
ಇವತ್ತು
ಇವನ
ಇವ-ನನ್ನು
ಇವ-ನಿಗೆ
ಇವ-ನೀಗ
ಇವನು
ಇವನೂ
ಇವನೇ
ಇವ-ನ್ನೆಲ್ಲ
ಇವ-ರಿ-ಗಾಗಿ
ಇವ-ರಿ-ಬ್ಬರ
ಇವರು
ಇವ-ರು-ಗಳನ್ನೂ
ಇವರೂ
ಇವರೆಲ್ಲ
ಇವರೆ-ಲ್ಲರ
ಇವರೇ
ಇವಳ
ಇವು
ಇವು-ಗಳ
ಇವು-ಗಳನ್ನು
ಇವು-ಗಳನ್ನೆಲ್ಲ
ಇವು-ಗಳಲ್ಲಿ
ಇವು-ಗ-ಳಲ್ಲೇ
ಇವು-ಗಳಿಂದ
ಇವು-ಗಳಿಂದಾ
ಇವು-ಗ-ಳಿಂ-ದಾಗಿ
ಇವೆ-ರಡೂ
ಇವೆಲ್ಲ
ಇವೆ-ಲ್ಲ-ಕ್ಕಿಂತ
ಇಷ್ಟ
ಇಷ್ಟ-ದೇ-ವ-ತೆ-ಯಾದ
ಇಷ್ಟ-ಪ-ಡ-ಲಿಲ್ಲ
ಇಷ್ಟ-ಪ-ಡು-ತ್ತಿದ್ದ
ಇಷ್ಟ-ಲ್ಲದೆ
ಇಷ್ಟ-ವಾ-ಯಿತು
ಇಷ್ಟ-ವಿಲ್ಲ
ಇಷ್ಟವೇ
ಇಷ್ಟ-ಸ-ರಿ-ಯಾ-ಗಿಯೇ
ಇಷ್ಟಾ-ದರೂ
ಇಷ್ಟಿ-ದ್ದರೆ
ಇಷ್ಟಿದ್ದು
ಇಷ್ಟು
ಇಷ್ಟೆಲ್ಲ
ಇಷ್ಟೇ
ಇಷ್ಟೊಂದು
ಇಸವಿ
ಈ
ಈಕೆ
ಈಗ
ಈಗಂತೂ
ಈಗಲೂ
ಈಗಲೇ
ಈಗಾ-ಗಲೇ
ಈಗಿಂ-ದಲೇ
ಈಗಿನ
ಈಗೀಗ
ಈಗೊಂದು
ಈಚೀ-ಚೆ-ಗಂತೂ
ಈಚೆಗೆ
ಈಜಾ-ಡಿ-ಕೊಂಡು
ಈಜು
ಈಜು-ವು-ದ-ರಲ್ಲಿ
ಈಡಾ-ಡು-ತ್ತಿದ್ದ
ಈಡಾ-ದರೂ
ಈಡೇ-ರ-ಬ-ಹು-ದೆಂಬ
ಈಡೇ-ರಿ-ಸಿ-ಕೊ-ಡ-ಬಲ್ಲ
ಈತ
ಈತನ
ಈತ-ನ-ದಾ-ಗಿತ್ತು
ಈಶ್ವ-ರ-ಚಂದ್ರ
ಉಂಟಾ-ಗಿತ್ತು
ಉಂಟಾ-ಗುವ
ಉಂಟಾದ
ಉಂಟು
ಉಕ್ಕಿ
ಉಕ್ಕಿ-ಬಂತು
ಉಕ್ಕು-ತ್ತಿ-ರು-ವುದನ್ನು
ಉಕ್ಕೇ-ರಿ-ದರೂ
ಉಗು-ಳುತ್ತ
ಉಚ್ಚ-ಕಂ-ಠ-ದಿಂದ
ಉಚ್ಚ-ರಿ-ಸ-ಬಾ-ರ-ದಂಥ
ಉಚ್ಚ-ರಿ-ಸ-ಬೇಕು
ಉಚ್ಚ-ರಿ-ಸುತ್ತ
ಉಚ್ಚ-ರಿ-ಸು-ವಾಗ
ಉಜಿರ್
ಉಜ್ವಲ
ಉಜ್ವ-ಲ-ತೆ-ಯನ್ನು
ಉಜ್ವ-ಲ-ವಾ-ಗಿದೆ
ಉಡಿ-ಗೆ-ಯಲ್ಲಿ
ಉಡಿ-ಗೆ-ಯಾದ
ಉಡಿ-ಸಿದ್ದ
ಉಡಿ-ಸಿ-ದ್ದರು
ಉಡಿ-ಸು-ತ್ತಾರೆ
ಉತ್ಕಟ
ಉತ್ತಮ
ಉತ್ತರ
ಉತ್ತ-ರಕ್ಕೆ
ಉತ್ತ-ರ-ಭಾ-ರ-ತದ
ಉತ್ತ-ರ-ವನ್ನು
ಉತ್ತ-ರ-ವನ್ನೇ
ಉತ್ತ-ರ-ವಿಲ್ಲ
ಉತ್ತ-ರವೂ
ಉತ್ತ-ರವೇ
ಉತ್ತ-ರ-ವೇನು
ಉತ್ತ-ರ-ಹೋಗು
ಉತ್ತ-ರಿ-ಸಲು
ಉತ್ತ-ರಿ-ಸಿದ
ಉತ್ತ-ರಿ-ಸಿ-ದಾಗ
ಉತ್ತ-ರಿ-ಸಿಯಾ
ಉತ್ತ-ರಿ-ಸು-ತ್ತಾನೆ
ಉತ್ತೀ-ರ್ಣ-ನಾ-ಗ-ಬೇಕು
ಉತ್ತೀ-ರ್ಣ-ನಾಗಿ
ಉತ್ತೀ-ರ್ಣ-ನಾದ
ಉತ್ತೀ-ರ್ಣನೂ
ಉತ್ತೀ-ರ್ಣ-ರಾ-ಗ-ಲೇ-ಬೇಕು
ಉತ್ಪ-ನ್ನ-ವಾ-ದಾಗ
ಉತ್ಪಾ-ದ-ನೆ-ಗಾಗಿ
ಉತ್ಪ್ರೇ-ಕ್ಷೆ-ಯ-ಲ್ಲ-ವೆಂಬ
ಉತ್ಸ-ವ-ಗಳನ್ನು
ಉತ್ಸ-ವ-ಗಳಲ್ಲಿ
ಉತ್ಸಾಹ
ಉತ್ಸಾ-ಹ-ದಿಂದ
ಉತ್ಸಾ-ಹ-ಪೂ-ಣನ್
ಉತ್ಸಾ-ಹ-ವೇನೂ
ಉದಾ
ಉದಾತ್ತ
ಉದಾ-ತ್ತ-ಗಂ-ಭೀರ
ಉದಾ-ತ್ತ-ಪ-ರಿ-ಶುದ್ಧ
ಉದಾರ
ಉದಾ-ರ-ದೃಷ್ಟಿ
ಉದಾರಿ
ಉದಾ-ರಿ-ಗಳೇ
ಉದಾ-ಸೀನ
ಉದಾ-ಹ-ರ-ಣೆಗೆ
ಉದಾ-ಹ-ರಿ-ಸು-ವು-ದಿತ್ತು
ಉದು-ರಿ-ಸಿ-ಬಿ-ಡು-ತ್ತಾರೆ
ಉದ್ಗರಿ
ಉದ್ಗ-ರಿಸಿ
ಉದ್ಗ-ರಿ-ಸುತ್ತ
ಉದ್ದ
ಉದ್ದಕ್ಕೂ
ಉದ್ದವೋ
ಉದ್ದು-ದ್ದದ
ಉದ್ದೇಶ
ಉದ್ಧ-ರಿಸಿ
ಉದ್ಭ-ವಿ-ಸುವ
ಉದ್ಯಾನ
ಉದ್ಯಾ-ನ-ಗ-ಳಿಗೋ
ಉದ್ಯೋಗ
ಉದ್ಯೋ-ಗ-ಸ್ಥ-ನಾಗಿ
ಉದ್ರೇ-ಕ-ಕಾರೀ
ಉನ್ನತ
ಉನ್ನ-ತಾ-ಧಿ-ಕಾರಿ
ಉಪ-ಕ-ರ-ಣ-ಗಳನ್ನೆಲ್ಲ
ಉಪ-ಚ-ರಿ-ಸಲು
ಉಪ-ಚ-ರಿ-ಸಿದ
ಉಪ-ದೇ-ಶಿ-ಸಿದ
ಉಪ-ನಿ-ಷ-ತ್ತು-ಪು-ರಾ-ಣ-ಗ-ಳೇನೋ
ಉಪ-ನ್ಯಾ-ಸ-ಗಳ
ಉಪ-ನ್ಯಾ-ಸ-ಗಳನ್ನು
ಉಪ-ಯೋ-ಗ-ವಾ-ಗ-ಲಿಲ್ಲ
ಉಪ-ಯೋ-ಗಿ-ಸ-ಕೊ-ಳ್ಳ-ಬೇ-ಕಾ-ಗಿತ್ತು
ಉಪ-ಯೋ-ಗಿ-ಸು-ತ್ತಿದ್ದ
ಉಪ-ವಾಸ
ಉಪಾ-ಧ್ಯಾ-ಯರ
ಉಪಾ-ಧ್ಯಾ-ಯ-ರನ್ನು
ಉಪಾ-ಧ್ಯಾ-ಯ-ರ-ನ್ನೆಲ್ಲ
ಉಪಾ-ಧ್ಯಾ-ಯ-ರಿಂದ
ಉಪಾ-ಧ್ಯಾ-ಯರು
ಉಪಾ-ಧ್ಯಾ-ಯ-ರೊ-ಬ್ಬರು
ಉಪಾಯ
ಉಪಾ-ಯ-ಗಾ-ಣದೆ
ಉಪಾ-ಯ-ವಾಗಿ
ಉಪಾ-ಸ-ಕರು
ಉಪಾ-ಸನೆ
ಉಪ್ಪು
ಉಭ-ಯ-ಸಂ-ಕ-ಟಕ್ಕೆ
ಉಭ-ಯ-ಸಂ-ಕ-ಟ-ದಿಂದ
ಉಯ್ಯಾ-ಲೆಯ
ಉರಿ-ಯಿ-ಟ್ಟಂ-ತಾ-ಯಿತು
ಉರಿಸಿ
ಉರಿ-ಸು-ತ್ತಿ-ದ್ದರು
ಉರ್ದು
ಉಲ್ಲಾ-ಸ-ದಿಂದ
ಉಲ್ಲಾ-ಸ-ಭಾ-ವ-ದಿಂದ
ಉಲ್ಲಾ-ಸ-ಮಯ
ಉಲ್ಲಾ-ಸವೂ
ಉಲ್ಲಾಸಿ
ಉಲ್ಲೇ-ಖಿ-ಸ-ಬ-ಹುದು
ಉಳಿದ
ಉಳಿ-ದಂತೆ
ಉಳಿ-ದ-ವರ
ಉಳಿ-ದ-ವರೆಲ್ಲ
ಉಳಿ-ದಿದೆ
ಉಳಿ-ದು-ಕೊಂ-ಡರು
ಉಳಿ-ದು-ಕೊಂ-ಡಿತ್ತು
ಉಳಿ-ದು-ಕೊಂ-ಡಿ-ದ್ದ-ವರು
ಉಳಿ-ದು-ಕೊಂ-ಡಿವೆ
ಉಳಿ-ದು-ಕೊಂ-ಡು-ಬಿ-ಟ್ಟಿತು
ಉಳಿ-ದು-ದೆಲ್ಲ
ಉಳಿ-ದೆ-ರಡು
ಉಳಿ-ಯ-ಲಿಲ್ಲ
ಉಳಿ-ಯು-ವು-ದಿಲ್ಲ
ಉಳಿ-ಸಿ-ಕೊಂ-ಡರು
ಉಳಿ-ಸಿ-ಕೊಂಡು
ಉಳಿ-ಸಿದ
ಉಸಿರು
ಊಟ
ಊಟ-ವಾದ
ಊರಿಂ-ದೂ-ರಿಗೆ
ಊರು-ಗ-ಳ-ಲ್ಲಿ-ರು-ವ-ವರು
ಊಹಿಸಿ
ಊಹಿ-ಸಿ-ದರು
ಊಹಿ-ಸು-ವು-ದಕ್ಕೂ
ಊಹೆಗೆ
ಎ
ಎಂಟ್ರೆ-ನ್ಸ್
ಎಂತಹ
ಎಂಥ
ಎಂಥದು
ಎಂಥ-ವ-ರನ್ನೇ
ಎಂಥಾ
ಎಂದ
ಎಂದನೋ
ಎಂದ-ರಾ-ಗದು
ಎಂದರು
ಎಂದರೂ
ಎಂದರೆ
ಎಂದಳು
ಎಂದಾ-ಯಿತು
ಎಂದಿಗೆ
ಎಂದಿದ್ದ
ಎಂದಿನ
ಎಂದಿ-ನಂ-ತಾದ
ಎಂದಿ-ನಂ-ತಾ-ದಾಗ
ಎಂದಿ-ನಂತೆ
ಎಂದಿ-ನಂ-ತೆಯೇ
ಎಂದು
ಎಂದು-ಕೊ-ಳ್ಳುತ್ತ
ಎಂದು-ದ್ಗ-ರಿ-ಸುತ್ತ
ಎಂದು-ಬಿಟ್ಟ
ಎಂದು-ಬಿ-ಟ್ಟರೆ
ಎಂದೂ
ಎಂದೇ
ಎಂಬ
ಎಂಬಂ-ತಿತ್ತು
ಎಂಬಂತೆ
ಎಂಬ-ವನ
ಎಂಬ-ವ-ನಿಗೆ
ಎಂಬ-ವನು
ಎಂಬಷ್ಟೇ
ಎಂಬಾ-ತನ
ಎಂಬಿ-ತ್ಯಾ-ದಿ-ಯಾಗಿ
ಎಂಬಿ-ಬ್ಬರು
ಎಂಬು-ದನ್ನು
ಎಂಬು-ದರ
ಎಂಬುದು
ಎಂಬು-ದೊಂದು
ಎಚ್ಚ-ತ್ತಿ-ದ್ದರೂ
ಎಚ್ಚ-ತ್ತು-ಕೊಂಡ
ಎಚ್ಚ-ರ-ಗೊಂಡ
ಎಚ್ಚ-ರ-ಗೊ-ಳ್ಳು-ತ್ತಿದ್ದೆ
ಎಚ್ಚ-ರ-ದಿಂದ
ಎಚ್ಚ-ರ-ವಾ-ಗಿಯೇ
ಎಟು-ಕ-ದಂತೆ
ಎಡದ
ಎಡ್ವ-ರ್ಡ್
ಎಣ್ಣೆ
ಎತ್ತ-ರಕ್ಕೆ
ಎತ್ತ-ರದ
ಎತ್ತ-ರ-ದಲ್ಲಿ
ಎತ್ತಲು
ಎತ್ತಿ
ಎತ್ತಿ-ನ-ಗಾಡಿ
ಎತ್ತಿ-ನ-ಗಾ-ಡಿ-ಯಲ್ಲೇ
ಎತ್ತು-ತ್ತಿ-ದ್ದರು
ಎತ್ತು-ತ್ತಿ-ದ್ದೆವು
ಎತ್ತುವ
ಎತ್ತು-ವಾಗ
ಎತ್ತೆತ್ತಿ
ಎದುರಾ
ಎದು-ರಾ-ಗು-ತ್ತವೆ
ಎದು-ರಾ-ಗುವ
ಎದು-ರಾ-ಳಿಯ
ಎದು-ರಿಗೇ
ಎದು-ರಿ-ನಲ್ಲಿ
ಎದು-ರಿ-ನಿಂದ
ಎದು-ರಿ-ಸ-ಬೇ-ಕಾ-ಗು-ತ್ತದೆ
ಎದು-ರಿ-ಸ-ಬೇಕು
ಎದು-ರಿ-ಸಿಯೇ
ಎದು-ರು-ನೋ-ಡುತ್ತ
ಎದು-ರು-ಸಾ-ಲಿ-ನಲ್ಲೇ
ಎದೆ
ಎದೆ-ಗಾ-ರಿಕೆ
ಎದೆ-ಗಾ-ರಿ-ಕೆ-ಯನ್ನು
ಎದೆ-ಗಾ-ರಿ-ಕೆ-ಯಾ-ವುದೂ
ಎದೆ-ಗುಂ-ದ-ಲಿಲ್ಲ
ಎದೆ-ಮುಟ್ಟಿ
ಎದ್ದರೆ
ಎದ್ದಿತು
ಎದ್ದು
ಎದ್ದು-ಕಾ-ಣು-ತ್ತಿತ್ತು
ಎದ್ದು-ನಿಂತು
ಎದ್ದು-ಬಂದ
ಎದ್ದು-ಬಂದು
ಎದ್ದೇ-ಳು-ತ್ತಿತ್ತು
ಎನ್ನ
ಎನ್ನ-ತೊ-ಡ-ಗಿ-ದಾಗ
ಎನ್ನದೆ
ಎನ್ನ-ಬ-ಹುದು
ಎನ್ನ-ಬೇಕು
ಎನ್ನುತ್ತ
ಎನ್ನು-ತ್ತಾ-ನ-ವನು
ಎನ್ನು-ತ್ತಿದ್ದ
ಎನ್ನು-ತ್ತಿ-ದ್ದರು
ಎನ್ನುವ
ಎನ್ನು-ವಂತೆ
ಎನ್ನು-ವ-ವನ
ಎನ್ನು-ವಾಗ
ಎನ್ನುವು
ಎನ್ನು-ವುದನ್ನು
ಎನ್ನು-ವು-ದ-ನ್ನೆಲ್ಲ
ಎನ್ನು-ವುದು
ಎನ್ನು-ವುದೇ
ಎನ್ನು-ವು-ದೇನೋ
ಎನ್ನು-ವು-ದೊಂದು
ಎಫ್ಎ
ಎಬ್ಬಿ-ಸ-ಬೇ-ಕಾ-ಗು-ತ್ತಿತ್ತು
ಎಬ್ಬಿ-ಸಲು
ಎರ-ಡ-ನೆಯ
ಎರ-ಡ-ನೆ-ಯ-ದಾಗಿ
ಎರ-ಡನೇ
ಎರ-ಡ-ರಲ್ಲಿ
ಎರ-ಡ-ರಷ್ಟು
ಎರ-ಡ-ರ್ಥ-ಒಂ-ದ-ನೆ-ಯ-ದಾಗಿ
ಎರಡು
ಎರ-ಡು-ಮೂರು
ಎರಡೂ
ಎರ-ಡೆ-ರಡು
ಎಲಾ
ಎಲೆಗೆ
ಎಲ್ಲ
ಎಲ್ಲಕ್ಕೂ
ಎಲ್ಲರ
ಎಲ್ಲ-ರಂತೆ
ಎಲ್ಲ-ರಿ-ಗಿಂತ
ಎಲ್ಲ-ರಿಗೂ
ಎಲ್ಲರೂ
ಎಲ್ಲ-ವನ್ನೂ
ಎಲ್ಲ-ವು-ಗಳ
ಎಲ್ಲವೂ
ಎಲ್ಲಿ
ಎಲ್ಲಿಂ-ದಲೋ
ಎಲ್ಲಿಗೋ
ಎಲ್ಲೆ
ಎಲ್ಲೋ
ಎಳೆ-ದಂ-ತಾಗಿ
ಎಳೆದು
ಎಳೆ-ದು-ಕೊಂ-ಡು-ಬಿಟ್ಟ
ಎಳೆ-ದೆ-ಳೆದು
ಎಳೆಯ
ಎಳೆ-ಯುತ್ತ
ಎಳೆ-ಯು-ತ್ತಿದ್ದ
ಎಷ್ಟ-ರ-ಮ-ಟ್ಟಿ-ಗಿ-ತ್ತೆಂ-ದರೆ
ಎಷ್ಟ-ರ-ಮ-ಟ್ಟಿ-ಗೆಂ-ದರೆ
ಎಷ್ಟಾ-ದರೂ
ಎಷ್ಟು
ಎಷ್ಟೆಷ್ಟು
ಎಷ್ಟೇ
ಎಷ್ಟೊಂದು
ಎಷ್ಟೋ
ಎಸೆದು
ಎಸೆ-ದು-ಬಿ-ಡೋಣ
ಎಸೆ-ಯು-ವಲ್ಲಿ
ಎಸ್ರಾಜ್
ಏ
ಏಕ-ಕಾ-ಲ-ದಲ್ಲಿ
ಏಕ-ದೇ-ವತಾ
ಏಕಾಂ-ತ-ದಲ್ಲಿ
ಏಕಾ-ಗ್ರ-ಗೊಂಡ
ಏಕಾ-ಗ್ರ-ಗೊ-ಳಿಸಿ
ಏಕಾ-ಗ್ರ-ತೆ-ಯಿಂದ
ಏಕೆ
ಏಕೆಂ-ದರೆ
ಏತ-ಕ್ಕೋ-ಸ್ಕರ
ಏನ-ಡಿಗೆ
ಏನನ್ನಾ
ಏನ-ನ್ನಾ-ದರೂ
ಏನನ್ನು
ಏನನ್ನೂ
ಏನಪ್ಪ
ಏನಪ್ಪಾ
ಏನರ್ಥ
ಏನಾಗ
ಏನಾ-ಗಿತ್ತು
ಏನಾ-ಗಿ-ರ-ಬ-ಹು-ದೆಂ-ಬು-ದನ್ನು
ಏನಾ-ಗು-ತ್ತಿತ್ತೋ
ಏನಾ-ಗುತ್ತೆ
ಏನಾದ
ಏನಾ-ದರೂ
ಏನಾ-ದ-ರೊಂದು
ಏನಾ-ಯಿ-ತಪ್ಪ
ಏನಾ-ಯಿತು
ಏನಿದು
ಏನಿಲ್ಲ
ಏನು
ಏನೂ
ಏನೆಂ-ದರೆ
ಏನೆಂದು
ಏನೇ
ಏನೇನು
ಏನೇನೂ
ಏನೇನೋ
ಏನೋ
ಏನ್
ಏಯ್
ಏರಿ-ಕೊಂಡು
ಏರಿ-ಸುವ
ಏರ್ಪ-ಟ್ಟಿತ್ತು
ಏರ್ಪ-ಡಿ-ಸ-ಲಾ-ಯಿತು
ಏರ್ಪ-ಡಿಸಿ
ಏರ್ಪ-ಡಿ-ಸಿದ
ಏರ್ಪ-ಡಿ-ಸಿ-ದ್ದರು
ಏಳನೇ
ಏಳು-ತ್ತಲೇ
ಏಳುವ
ಐತಿ-ಹಾ-ಸಿಕ
ಐದೇ
ಐನೂರು
ಐರೋಪ್ಯ
ಐಶ್ವರ್ಯ
ಐಶ್ವ-ರ್ಯವೇ
ಒಂಟೆ-ಗಳ
ಒಂದ-ಕ್ಕೊಂದು
ಒಂದಲ್ಲ
ಒಂದಷ್ಟು
ಒಂದಾ-ದ-ಮೇ-ಲೊಂದು
ಒಂದಿಂಚು
ಒಂದಿ-ಬ್ಬರು
ಒಂದಿಷ್ಟು
ಒಂದು
ಒಂದು-ತುಂ-ಟ-ತನ
ಒಂದು-ತುಂ-ಡನ್ನು
ಒಂದು-ಧ್ಯಾ-ನ-ಸ್ಥ-ನಾದ
ಒಂದೆಡೆ
ಒಂದೆ-ರಡು
ಒಂದೇ
ಒಂದೊಂ-ದಾಗಿ
ಒಂದೊಂದು
ಒಂದೊಂದೇ
ಒಂದೋ
ಒಂಬತ್ತು
ಒಗ-ಟಾಗಿ
ಒಗ್ಗು-ವು-ದಿ-ಲ್ಲವೋ
ಒಟ್ಟಾಗಿ
ಒಟ್ಟಿಗೆ
ಒಟ್ಟಿ-ನಲ್ಲಿ
ಒಟ್ಟು
ಒಡ-ಕು-ಗ-ಳುಂ-ಟಾ-ಗು-ತ್ತಿ-ದ್ದುವು
ಒಡ-ನಾ-ಡಿ-ಗಳು
ಒಡ-ನಾ-ಡುತ್ತ
ಒಡ-ನೆಯೇ
ಒಡೆದು
ಒಡೆ-ದು-ಕೊಂ-ಡ-ರೇ-ನಪ್ಪ
ಒಡೆ-ಯ-ನಾದ
ಒಣ-ಹುಲ್ಲು
ಒತ್ತಡ
ಒತ್ತಾ-ಯ-ಪ-ಡಿಸಿ
ಒತ್ತಾ-ಯ-ಪ-ಡಿ-ಸಿ-ದರು
ಒತ್ತಿ
ಒತ್ತಿ-ದಷ್ಟೂ
ಒದ-ಗಿ-ಬಂತು
ಒದ-ಗಿ-ಬಂದ
ಒದ-ಗಿ-ಸಿ-ಕೊ-ಡು-ತ್ತಿದ್ದ
ಒದ-ಗಿ-ಸಿ-ಕೊ-ಡು-ವಲ್ಲಿ
ಒದ್ದಾಗ
ಒದ್ದಾ-ಡ-ಬೇ-ಕಾ-ಗು-ತ್ತದೆ
ಒದ್ದಾ-ಡಿತು
ಒಪ್ಪಲೇ
ಒಪ್ಪಿ
ಒಪ್ಪಿ-ಕೊಂಡ
ಒಪ್ಪಿ-ಕೊಂ-ಡರು
ಒಪ್ಪಿ-ಕೊಂಡು
ಒಪ್ಪಿ-ಕೊ-ಳ್ಳದೆ
ಒಪ್ಪಿ-ಕೊ-ಳ್ಳ-ಬ-ಹು-ದಾ-ಗಿತ್ತು
ಒಪ್ಪಿ-ಕೊ-ಳ್ಳು-ವ-ವ-ರೆಗೆ
ಒಪ್ಪಿ-ಗೆ-ಯಾ-ಗ-ಲಿಲ್ಲ
ಒಪ್ಪಿ-ಗೆ-ಯಾ-ದವು
ಒಪ್ಪಿ-ಗೆ-ಯಾ-ಯಿತು
ಒಪ್ಪಿತು
ಒಪ್ಪಿ-ದರು
ಒಪ್ಪಿದ್ದು
ಒಪ್ಪಿ-ಸಿ-ಬಿಟ್ಟ
ಒಪ್ಪು-ತ್ತಿ-ರ-ಲಿಲ್ಲ
ಒಬ್ಬ
ಒಬ್ಬನ
ಒಬ್ಬ-ನಿ-ಗಾಗಿ
ಒಬ್ಬನೂ
ಒಬ್ಬನೇ
ಒಬ್ಬರ
ಒಬ್ಬ-ರಲ್ಲ
ಒಬ್ಬ-ಳಿ-ದ್ದಳು
ಒಬ್ಬಳು
ಒಬ್ಬಳೇ
ಒಬ್ಬೊ-ಬ್ಬ-ರಾಗಿ
ಒಮ್ಮೆ
ಒರ-ಗಿ-ಕೊಂ-ಡರೆ
ಒರ-ಗಿ-ಕೊಂ-ಡಿ-ರು-ತ್ತಿದ್ದ
ಒರ-ಗಿ-ಕೊಂಡು
ಒರ-ಟೊ-ರ-ಟಾಗಿ
ಒರೆ-ಗ-ಲ್ಲಿಗೆ
ಒರೆ-ಸಿ-ಕೊ-ಳ್ಳುತ್ತ
ಒಲವು
ಒಲವೂ
ಒಲಿದ
ಒಲಿ-ಸಿ-ಕೊ-ಳ್ಳಲು
ಒಲೆ
ಒಲೆ-ಗುಂ-ಡು-ಕಲ್ಲು
ಒಳಕ್ಕೆ
ಒಳ-ಗಾ-ಗು-ತ್ತಾರೆ
ಒಳ-ಗಾ-ದರು
ಒಳ-ಗಿಂದ
ಒಳಗೆ
ಒಳಗೇ
ಒಳ-ಗೇ-ನಿ-ದೆ-ಯೆಂದು
ಒಳ-ಗೊ-ಳಗೇ
ಒಳ-ಶುಂಠಿ
ಒಳ-ಹೊ-ರ-ಗನ್ನು
ಒಳ-ಹೊ-ರ-ಗು-ಗಳನ್ನೂ
ಒಳ್ಳೆ
ಒಳ್ಳೆಯ
ಒಳ್ಳೆ-ಯ-ವ-ನಾ-ದರೆ
ಒಳ್ಳೆ-ಯ-ವನು
ಒಳ್ಳೇ
ಒಳ್ಳೊ-ಳ್ಳೆಯ
ಓ
ಓಜೋ-ಮ-ಯ-ವಾದ
ಓಟ
ಓಡಾ
ಓಡಾ-ಡಿ-ಕೊಂ-ಡಿ-ದ್ದರೆ
ಓಡಾ-ಡು-ತ್ತಿದ್ದ
ಓಡಿದ
ಓಡಿ-ಬಂದು
ಓಡಿ-ಸಲು
ಓಡಿಸಿ
ಓಡಿ-ಸೋ-ಣ-ವೆಂದರೆ
ಓಡಿ-ಹೋ-ಗ-ದಿ-ರ-ಲೆಂದು
ಓಡಿ-ಹೋಗಿ
ಓಡು-ತ್ತಿದ್ದ
ಓಡು-ತ್ತಿ-ದ್ದುದು
ಓಡು-ತ್ತಿ-ರುವ
ಓಡುವ
ಓಡೋಡಿ
ಓದ-ಬೇ-ಕಾದ
ಓದ-ರಿ-ಯ-ದ-ವರ
ಓದಲು
ಓದಲೂ
ಓದ-ಲೇ-ಬೇ-ಕಾ-ಗಿ-ರ-ಲಿಲ್ಲ
ಓದಿ
ಓದಿ-ಓದಿ
ಓದಿ-ಕೊಂಡು
ಓದಿ-ಕೊ-ಳ್ಳು-ತ್ತಿ-ದ್ದಾಗ
ಓದಿದ
ಓದಿ-ದರೆ
ಓದಿದ್ದೇ
ಓದಿ-ಬ-ರೆ-ಯುವ
ಓದಿ-ಬಿ-ಟ್ಟರು
ಓದಿ-ಬಿ-ಟ್ಟರೆ
ಓದಿ-ಬಿ-ಟ್ಟಿ-ದ್ದ-ನೆಂ-ದರೆ
ಓದಿ-ರ-ಲಿಲ್ಲ
ಓದಿಸಿ
ಓದು-ಗೀ-ದಿನ
ಓದುತ್ತ
ಓದು-ತ್ತಿದ್ದ
ಓದು-ತ್ತಿ-ದ್ದರೆ
ಓದು-ತ್ತಿ-ರ-ಲಿಲ್ಲ
ಓದು-ತ್ತಿ-ರು-ವಂ-ತೆಯೇ
ಓದು-ತ್ತೇ-ನೆ-ನ-ರೇಂ-ದ್ರ-ನೆಂದ
ಓದುವ
ಓರ-ಗೆ-ಯ-ವ-ರಿ-ಗೆಲ್ಲ
ಓಲಾ-ಡುವ
ಓಳ್ಳೇ
ಓಹೋ
ಔಚಿ-ತ್ಯ-ಪೂ-ರ್ಣ-ವಾ-ಗಿದೆ
ಔತಣ
ಔದಾರ್ಯ
ಔದಾ-ರ್ಯ-ಸೌ-ಜ-ನ್ಯ-ಗಳು
ಔದಾ-ರ್ಯ-ದೊಂ-ದಿಗೆ
ಔದಾ-ರ್ಯ-ವನ್ನು
ಕ
ಕಂಗಳು
ಕಂಗಾ-ಲಾ-ಗದೆ
ಕಂಗಾ-ಲಾದ
ಕಂಠ
ಕಂಠ-ದಂತೆ
ಕಂಠ-ಮಾ-ಧು-ರ್ಯ-ವನ್ನು
ಕಂಠ-ಮಾ-ಧು-ರ್ಯ-ವಿದ್ದ
ಕಂಠವೂ
ಕಂಠಶ್ರೀ
ಕಂಠಸ್ಥ
ಕಂಠ-ಸ್ವರ
ಕಂಠ-ಸ್ವ-ರವೇ
ಕಂಡ
ಕಂಡ-ದ್ದನ್ನೇ
ಕಂಡ-ರಾ-ಗದು
ಕಂಡರೂ
ಕಂಡರೆ
ಕಂಡರೇ
ಕಂಡ-ವ-ನಲ್ಲ
ಕಂಡ-ವನೇ
ಕಂಡ-ವ-ರಾರು
ಕಂಡ-ವ-ರಿ-ರ-ಬ-ಹು-ದ-ಲ್ಲವೆ
ಕಂಡ-ವರು
ಕಂಡಾಗ
ಕಂಡಿ-ತು-ಅದೂ
ಕಂಡಿದ್ದ
ಕಂಡಿ-ದ್ದೀರಾ
ಕಂಡಿ-ದ್ದೇ-ನೆ-ಅ-ವರೇ
ಕಂಡಿ-ರ-ಬೇಕು
ಕಂಡಿ-ರ-ಲಿ-ಕ್ಕಿಲ್ಲ
ಕಂಡಿ-ರ-ಲಿಲ್ಲ
ಕಂಡಿಲ್ಲ
ಕಂಡು
ಕಂಡು-ಕೊಂಡ
ಕಂಡು-ಕೊಂ-ಡಳು
ಕಂಡು-ಕೊಂ-ಡಿದ್ದ
ಕಂಡು-ಕೊಂ-ಡಿ-ದ್ದರು
ಕಂಡು-ಕೊಂಡು
ಕಂಡು-ಕೊಳ್ಳಿ
ಕಂಡು-ಬಂ-ದರೂ
ಕಂಡು-ಬಂ-ದಿತು
ಕಂಡು-ಬ-ರು-ತ್ತದೆ
ಕಂಡು-ಬ-ರು-ತ್ತಿತ್ತು
ಕಂಡು-ಬ-ರು-ತ್ತಿದೆ
ಕಂಡು-ಬ-ರು-ತ್ತಿದ್ದ
ಕಂಡು-ಬ-ರು-ತ್ತಿ-ರು-ವಾಗ
ಕಂಡು-ಬ-ರುವ
ಕಂಡೆನೋ
ಕಂಡೇ
ಕಂತೆ-ಯನ್ನು
ಕಂದಾ-ಚಾ-ರ-ಗಳನ್ನೂ
ಕಂದಾ-ಚಾ-ರ-ಗಳಿಂದ
ಕಂದಾ-ಚಾ-ರ-ಗಳು
ಕಂಬ-ನಿ-ದುಂ-ಬಿತು
ಕಕ್ಷಿ-ದಾರ
ಕಕ್ಷಿ-ದಾ-ರ-ನಿಗೆ
ಕಕ್ಷಿ-ದಾ-ರ-ರನ್ನು
ಕಕ್ಷಿ-ದಾ-ರ-ರಿ-ಗೆಲ್ಲ
ಕಕ್ಷಿ-ದಾ-ರರು
ಕಕ್ಷಿ-ದಾ-ರ-ರೆಲ್ಲ
ಕಛೇ-ರಿ-ಗಳನ್ನು
ಕಛೇ-ರಿ-ಯಲ್ಲಿ
ಕಟು-ತರ
ಕಟು-ವಾಗಿ
ಕಟು-ವಾದ
ಕಟು-ಶ-ಬ್ದ-ಗಳನ್ನು
ಕಟ್ಟ-ಡಕ್ಕೆ
ಕಟ್ಟ-ಡ-ವ-ನ್ನೊಮ್ಮೆ
ಕಟ್ಟ-ದಿ-ದ್ದರೆ
ಕಟ್ಟ-ಬೇ-ಕಾ-ಗು-ತ್ತದೆ
ಕಟ್ಟ-ಲಾ-ಗಿತ್ತು
ಕಟ್ಟಲು
ಕಟ್ಟಲೂ
ಕಟ್ಟಿ-ಕೊಂ-ಡಿ-ರ-ಬೇ-ಕಾ-ಗಿತ್ತು
ಕಟ್ಟಿ-ಕೊಂಡು
ಕಟ್ಟಿ-ಕೊ-ಳ್ಳು-ತ್ತಿದ್ದೆ
ಕಟ್ಟಿದ
ಕಟ್ಟಿ-ದರು
ಕಟ್ಟಿದ್ದ
ಕಟ್ಟಿಯೇ
ಕಟ್ಟಿ-ಹಾ-ಕಿ-ಬಿ-ಡು-ತ್ತಿತ್ತು
ಕಟ್ಟು-ಪಾಡು
ಕಟ್ಟು-ಮ-ಸ್ತಾ-ಗಿ-ರು-ವುದನ್ನು
ಕಟ್ಟು-ಮ-ಸ್ತಾದ
ಕಟ್ಟು-ವಷ್ಟು
ಕಠೋರ
ಕಡಿ-ಮೆ-ಯ-ದೇ-ನಲ್ಲ
ಕಡಿ-ಮೆ-ಯಾ-ಗಿ-ರ-ಲಿಲ್ಲ
ಕಡಿ-ಮೆ-ಯಾ-ಯಿತು
ಕಡಿ-ಮೆ-ಯೇ-ನಿ-ರ-ಲಿಲ್ಲ
ಕಡೆ
ಕಡೆ-ಗ-ಣಿ-ಸಿ-ಬಿ-ಟ್ಟಿ-ದ್ದ-ರುಈ
ಕಡೆಗೂ
ಕಡೆಗೆ
ಕಡೆಗೇ
ಕಡೆ-ಯ-ವ-ರಿಂ-ದಲೂ
ಕಡೆ-ಯಿಂದ
ಕಡ್ಡಿ
ಕಡ್ಡಿ-ಯನ್ನು
ಕಡ್ಡಿ-ಯ-ನ್ನು-ಅ-ದರ
ಕಣಪ್ಪಾ
ಕಣಮ್ಮಾ
ಕಣಿ-ವೆ-ಯೊಂ-ದರ
ಕಣೊ
ಕಣ್ಕಣ್ಣು
ಕಣ್ಣ-ಗ-ಲಿ-ಸಿ-ಕೊಂಡು
ಕಣ್ಣನ್ನೇ
ಕಣ್ಣ-ರ-ಳಿ-ಸಿ-ಕೊಂಡು
ಕಣ್ಣಲ್ಲಿ
ಕಣ್ಣಾ-ಮು-ಚ್ಚಾಲೆ
ಕಣ್ಣಾರೆ
ಕಣ್ಣಿಂದ
ಕಣ್ಣಿಗೂ
ಕಣ್ಣಿಗೆ
ಕಣ್ಣಿ-ಟ್ಟಿ-ರ-ಬೇ-ಕಾ-ಯಿತು
ಕಣ್ಣಿಟ್ಟು
ಕಣ್ಣೀ-ರಿ-ಳಿ-ಸುತ್ತ
ಕಣ್ಣೀರು
ಕಣ್ಣು
ಕಣ್ಣು-ಮೂ-ಗು-ಗಳನ್ನು
ಕಣ್ಣು-ಕೆಂ-ಪಗೆ
ಕಣ್ಣು-ಗಳ
ಕಣ್ಣು-ಗಳನ್ನು
ಕಣ್ಣು-ಗಳಲ್ಲಿ
ಕಣ್ಣು-ಗ-ಳಿಗೆ
ಕಣ್ಣು-ಗ-ಳಿವೆ
ಕಣ್ಣು-ಗಳು
ಕಣ್ಣು-ಗಳೇ
ಕಣ್ಣು-ಜ್ಜಿ-ಕೊಂಡು
ಕಣ್ಣು-ಮು-ಚ್ಚಿ-ಕೊ-ಳ್ಳುವ
ಕಣ್ಣೆ-ವೆ-ಯಿಕ್ಕು
ಕಣ್ಣೊ-ರೆ-ಸುವ
ಕಣ್ಣೊ-ಳಗೆ
ಕಣ್ತೆ-ರೆದು
ಕಣ್ಮರೆ
ಕಣ್ಮ-ರೆ-ಯಾ-ಗಿ-ಬಿಟ್ಟ
ಕಣ್ಮುಚ್ಚಿ
ಕಣ್ಮು-ಚ್ಚಿ-ಕೊಂಡು
ಕಣ್ಮು-ಚ್ಚಿ-ದ-ನೆಂ-ದರೆ
ಕಣ್ಮು-ಮು-ಚ್ಚಿ-ಕೊಂ-ಡಿ-ದ್ದಾನೆ
ಕಣ್ರೊ
ಕತೆ-ಗಳನ್ನು
ಕತೆ-ಗಳೂ
ಕತೆ-ಯನ್ನು
ಕತ್ತನ್ನ
ಕತ್ತನ್ನು
ಕತ್ತ-ರಿಸಿ
ಕತ್ತಿ
ಕತ್ತಿ-ಯ-ಲ-ಗಿನ
ಕತ್ತು
ಕಥ-ನ-ಕಾ-ರರ
ಕಥ-ನ-ಕಾ-ರರು
ಕಥ-ನ-ಕಾ-ರ-ರೊ-ಬ್ಬರು
ಕಥಾ-ಕಾ-ಲ-ಕ್ಷೇಪ
ಕಥಾ-ಪ್ರ-ಸಂ-ಗ-ಗಳನ್ನು
ಕಥೆ
ಕಥೆ-ಗಳ
ಕಥೆ-ಗಳನ್ನು
ಕಥೆ-ಗಳನ್ನೆಲ್ಲ
ಕಥೆ-ಗ-ಳಿ-ದ್ದುವು
ಕಥೆ-ಗಳು
ಕಥೆ-ಗಳೂ
ಕಥೆ-ಯನ್ನು
ಕದ-ಲ-ದಿ-ರು-ವಂ-ತಹ
ಕದ-ಲು-ವು-ದಿಲ್ಲ
ಕದ್ದು
ಕನ-ಸಿನ
ಕನಿ-ಕ-ರ-ದಿಂದ
ಕನಿಷ್ಠ
ಕನ್ನ-ಡಿಯ
ಕನ್ನೈ-ಲಾಲ್
ಕಪಿ
ಕರ-ಗತ
ಕರ-ಗಿ-ಸಿ-ಬಿ-ಟ್ಟಿ-ದ್ದರು
ಕರ-ಗಿ-ಹೋ-ಗು-ತ್ತಿದ್ದ
ಕರ-ಗುತ್ತ
ಕರ-ತ-ಲಾ-ಮ-ಲ-ಕ-ವಾ-ಗಿ-ಸಿ-ಕೊಂಡ
ಕರು-ಣೆಯ
ಕರು-ಣೆ-ಯಿಂ-ದಿ-ರು-ತ್ತಿದ್ದ
ಕರು-ಳಿಗೆ
ಕರೆಗೆ
ಕರೆ-ತಂದ
ಕರೆ-ತಂದು
ಕರೆ-ತ-ರ-ಲಾ-ಯಿತು
ಕರೆ-ದರು
ಕರೆ-ದರೆ
ಕರೆ-ದಳು
ಕರೆದು
ಕರೆ-ದು-ಕೊಂಡು
ಕರೆ-ದೊ-ಯ್ದಿತು
ಕರೆ-ದೊಯ್ಯು
ಕರೆ-ದೊ-ಯ್ಯು-ತ್ತಿದ್ದ
ಕರೆ-ಯ-ಲಾಗಿದೆ
ಕರೆ-ಯಿ-ಸಿ-ಕೊಂಡು
ಕರೆ-ಯಿ-ಸಿ-ಕೊ-ಳ್ಳಲು
ಕರೆ-ಯು-ತ್ತಿದ್ದ
ಕರೆ-ಯು-ವುದು
ಕರೆ-ಸಿ-ಕೊಂ-ಡಿದ್ದ
ಕರ್ತವ್ಯ
ಕಲಾ-ತ್ಮಕ
ಕಲಾ-ವಿದ
ಕಲಾ-ವಿ-ದರು
ಕಲಿತ
ಕಲಿ-ತ-ಕೆ-ಲವು
ಕಲಿ-ತ-ದ್ದ-ಲ್ಲದೆ
ಕಲಿ-ತದ್ದು
ಕಲಿ-ತರು
ಕಲಿ-ತಾ-ಯಿತು
ಕಲಿ-ತಿದ್ದ
ಕಲಿ-ತಿ-ದ್ದ-ನೆಂ-ದರೆ
ಕಲಿತು
ಕಲಿ-ತು-ಕೊಂಡ
ಕಲಿ-ತು-ಕೊಂ-ಡಿದ್ದ
ಕಲಿ-ತು-ಕೊಂಡು
ಕಲಿ-ತು-ಕೊ-ಳ್ಳು-ತ್ತಿ-ದ್ದೆವು
ಕಲಿ-ಯ-ತೊ-ಡ-ಗಿದ
ಕಲಿ-ಯ-ದಿ-ದ್ದರೆ
ಕಲಿ-ಯ-ಬೇ-ಕಾ-ಗು-ತ್ತದೆ
ಕಲಿ-ಯ-ಬೇಕು
ಕಲಿ-ಯ-ಬೇ-ಡವೆ
ಕಲಿ-ಯ-ಲಾ-ರಂ-ಭಿ-ಸಿದ
ಕಲಿ-ಯ-ಲಾರೆ
ಕಲಿ-ಯಲು
ಕಲಿ-ಯ-ಲೇ-ಬೇ-ಕಾ-ಗಿತ್ತು
ಕಲಿ-ಯು-ವಂ-ಥದು
ಕಲಿ-ಯು-ವು-ದೆಂ-ದರೆ
ಕಲಿ-ಸ-ಲೇ-ಬೇಕು
ಕಲಿ-ಸಿ-ಕೊ-ಡ-ಬೇಕು
ಕಲಿ-ಸಿದ
ಕಲಿ-ಸಿದ್ದ
ಕಲಿ-ಸುವ
ಕಲೆ
ಕಲೆ-ಯಲ್ಲಿ
ಕಲೆ-ಯಲ್ಲೂ
ಕಲ್ಕ-ತ್ತಕ್ಕೆ
ಕಲ್ಕ-ತ್ತದ
ಕಲ್ಕ-ತ್ತ-ದಲ್ಲಿ
ಕಲ್ಕ-ತ್ತ-ದ-ಲ್ಲಿ-ದ್ದಾಗ
ಕಲ್ಕ-ತ್ತ-ದಿಂದ
ಕಲ್ಕತ್ತಾ
ಕಲ್ಪ-ನೆ-ಗಳ
ಕಲ್ಪ-ನೆ-ಗಳು
ಕಲ್ಪ-ನೆಯ
ಕಲ್ಪ-ನೆ-ಯೊಂದು
ಕಲ್ಯಾಣ
ಕಲ್ಲಿನ
ಕಲ್ಲು-ಚ-ಪ್ಪಡಿ
ಕಲ್ಲೆ-ದೆ-ಯ-ವ-ನಾಗಿ
ಕಳ-ಚಿತು
ಕಳಿ-ಸಿ-ಕೊಟ್ಟ
ಕಳಿ-ಸು-ವುದು
ಕಳೆ
ಕಳೆ-ದವು
ಕಳೆದು
ಕಳೆ-ದು-ಕೊಂ-ಡ-ವ-ನಿ-ಗಿಂತ
ಕಳೆ-ದು-ಕೊ-ಳ್ಳಲು
ಕಳೆ-ದು-ಕೊ-ಳ್ಳು-ತ್ತಿ-ರ-ಲಿಲ್ಲ
ಕಳೆ-ಯ-ಬೇಕು
ಕಳೆ-ಯಲು
ಕಳೆ-ಯಿತು
ಕಳೆ-ಯುತ್ತ
ಕಳೆ-ಯು-ತ್ತಿದ್ದ
ಕಳೆ-ಯುವ
ಕವಿಯ
ಕವಿ-ಯು-ತ್ತಿದ್ದ
ಕವಿ-ಯು-ವಂ-ತಾ-ದಾಗ
ಕವಿಯೇ
ಕಷ್ಟ
ಕಷ್ಟ-ಕಾ-ರ್ಪ-ಣ್ಯ-ಗಳನ್ನು
ಕಷ್ಟಕ್ಕೆ
ಕಷ್ಟ-ಗಳನ್ನು
ಕಷ್ಟ-ಗಳಿಂದ
ಕಷ್ಟ-ಗ-ಳಿಗೆ
ಕಷ್ಟದ
ಕಷ್ಟ-ದ-ಲ್ಲಿ-ರು-ವುದನ್ನು
ಕಷ್ಟ-ದಲ್ಲೇ
ಕಷ್ಟ-ಪಟ್ಟು
ಕಷ್ಟ-ವನ್ನು
ಕಷ್ಟ-ವಾ-ಗಿ-ರ-ಬೇಕು
ಕಷ್ಟ-ವಾ-ಯಿತು
ಕಸ-ಕ-ಡ್ಡಿ-ಗಳನ್ನು
ಕಸ-ರತ್ತು
ಕಸೂತಿ
ಕಹಿ-ಯನ್ನು
ಕಾಂಪೌ-ಡಿ-ನಲ್ಲಿ
ಕಾಗು-ಣಿ-ತ-ದೊಂ-ದಿಗೇ
ಕಾಡು-ತ್ತಿದ್ದ
ಕಾಡು-ತ್ತಿ-ದ್ದರು
ಕಾಡು-ದಾರಿ
ಕಾಣ
ಕಾಣ-ತೊ-ಡ-ಗಿತು
ಕಾಣ-ದಂತೆ
ಕಾಣ-ದಿ-ದ್ದು-ದ-ರಿಂದ
ಕಾಣ-ದಿ-ರಲು
ಕಾಣ-ಬ-ಯ-ಸು-ವ-ವನು
ಕಾಣ-ಬ-ಹು-ದಾ-ಗಿತ್ತು
ಕಾಣ-ಬ-ಹುದು
ಕಾಣ-ಬೇ-ಕೆಂ-ಬ-ವನು
ಕಾಣ-ಲಾ-ರಂ-ಭಿ-ಸಿತು
ಕಾಣಲಿ
ಕಾಣ-ಲಿ-ದ್ದೇವೆ
ಕಾಣ-ಲೇ-ಬೇಕು
ಕಾಣ-ಲೇ-ಬೇ-ಕೆಂಬ
ಕಾಣ-ಸಿ-ಗುವ
ಕಾಣಿ-ಕೆ-ಯನ್ನು
ಕಾಣಿ-ಸಲು
ಕಾಣಿ-ಸಿ-ಕೊಂಡ
ಕಾಣಿ-ಸಿ-ಕೊ-ಡ-ಬ-ಲ್ಲ-ವ-ರಾರು
ಕಾಣಿ-ಸು-ತ್ತದೆ
ಕಾಣಿ-ಸು-ತ್ತ-ದೆ-ಯಲ್ಲಾ
ಕಾಣಿ-ಸು-ತ್ತ-ದೆಯೆ
ಕಾಣು
ಕಾಣು-ತ್ತದೆ
ಕಾಣು-ತ್ತ-ದೆ-ಶ-ರೀರ
ಕಾಣು-ತ್ತಲ್ಲ
ಕಾಣು-ತ್ತಿತ್ತು
ಕಾಣು-ತ್ತಿ-ದ್ದ-ಅ-ವನ
ಕಾಣು-ತ್ತಿ-ದ್ದರು
ಕಾಣು-ತ್ತಿ-ದ್ದುದು
ಕಾಣು-ತ್ತಿ-ರ-ಲಿಲ್ಲ
ಕಾಣು-ತ್ತಿ-ರುವ
ಕಾಣುವ
ಕಾಣು-ವಂ-ತಹ
ಕಾಣು-ವಂ-ತಿಲ್ಲ
ಕಾಣು-ವುದು
ಕಾತ-ರತೆ
ಕಾತ-ರ-ದಲ್ಲಿ
ಕಾದು-ಕಾದು
ಕಾನ-ನದ
ಕಾನೂ-ನಿನ
ಕಾನೂನು
ಕಾಪಾ-ಡಿಕೊ
ಕಾಫಿ-ಯನ್ನು
ಕಾಯಿ-ಲೆ-ಯಾ-ಗಿ-ಬಿ-ಡು-ತ್ತಿತ್ತು
ಕಾಯು-ತ್ತಿ-ದ್ದ-ವ-ನಂತೆ
ಕಾಯು-ತ್ತಿರು
ಕಾರಣ
ಕಾರ-ಣ-ಕ್ಕಾಗಿ
ಕಾರ-ಣ-ಗಳಿಂದ
ಕಾರ-ಣ-ವಾ-ಗಿದ್ದ
ಕಾರ-ಣ-ವಾ-ದರೆ
ಕಾರ-ಣ-ವಾ-ಯಿತು
ಕಾರ-ಣ-ವಿತ್ತು
ಕಾರು
ಕಾರುಣ್ಯ
ಕಾರು-ಬಾ-ರನ್ನು
ಕಾರ್ತಿಕ
ಕಾರ್ಯ
ಕಾರ್ಯ-ಕ-ರ್ತ-ರೆ-ಲ್ಲರೂ
ಕಾರ್ಯ-ಕ-ಲಾ-ಪ-ಗಳನ್ನು
ಕಾರ್ಯ-ಕ-ಲಾ-ಪ-ಗಳಲ್ಲಿ
ಕಾರ್ಯ-ಕ್ರ-ಮ-ಗಳು
ಕಾರ್ಯ-ಕ್ರ-ಮದ
ಕಾರ್ಯ-ಕ್ರ-ಮ-ದಲ್ಲಿ
ಕಾರ್ಯ-ಗ-ತ-ಗೊ-ಳಿ-ಸಲು
ಕಾರ್ಯ-ಗಳನ್ನು
ಕಾರ್ಯ-ಗಳನ್ನೂ
ಕಾರ್ಯ-ಗಳಲ್ಲಿ
ಕಾರ್ಯ-ದಲ್ಲಿ
ಕಾರ್ಯ-ವನ್ನು
ಕಾರ್ಯ-ವಿ-ಧಾ-ನ-ಗಳನ್ನು
ಕಾರ್ಯ-ಶೀ-ಲತೆ
ಕಾರ್ಯ-ಸಾ-ಧ-ನೆಗೆ
ಕಾಲ
ಕಾಲಕ್ಕೆ
ಕಾಲ-ಕ್ರ-ಮ-ದಿಂದ
ಕಾಲದ
ಕಾಲ-ದಲ್ಲಿ
ಕಾಲ-ವಿ-ಳಂಬ
ಕಾಲವೂ
ಕಾಲ-ಸ್ಥಿ-ತಿಗೆ
ಕಾಲಾ-ನಂ-ತರ
ಕಾಲಿಗೆ
ಕಾಲಿ-ಟ್ಟಾಗ
ಕಾಲಿ-ಡು-ತ್ತಿದ್ದ
ಕಾಲು
ಕಾಲು-ಹಾ-ಕಲೇ
ಕಾಲೇ-ಜಿಗೆ
ಕಾಲೇ-ಜಿಗೇ
ಕಾಲೇ-ಜಿನ
ಕಾಲೇ-ಜಿ-ನಲ್ಲಿ
ಕಾಲೇಜು
ಕಾಳೀ-ದೇ-ವಾ-ಲ-ಯ-ದಲ್ಲಿ
ಕಾಳೀ-ಪ್ರ-ಸಾದ
ಕಾಳೀ-ಪ್ರ-ಸಾ-ದನ
ಕಾಳೀ-ಪ್ರ-ಸಾ-ದನೂ
ಕಾಳೀ-ಪ್ರ-ಸಾ-ದನೇ
ಕಾವನ್ನು
ಕಾವಿ-ಬಟ್ಟೆ
ಕಾವ್ಯ-ಮಯ
ಕಾವ್ಯ-ವನ್ನು
ಕಾವ್ಯ-ವೆಲ್ಲ
ಕಾಶಿಗೆ
ಕಾಶಿಯ
ಕಾಶಿ-ಯಲ್ಲಿ
ಕಾಶೀ
ಕಾಶೀ-ಯಾತ್ರೆ
ಕಿಂಚಿತ್ತೂ
ಕಿಟ-ಕಿಯ
ಕಿಟ-ಕಿ-ಯಿಂದ
ಕಿಡಿ
ಕಿತ್ತ
ಕಿತ್ತು-ಹೋಗಿ
ಕಿತ್ತೆ-ಸೆ-ದು-ಬಿಟ್ಟ
ಕಿಮ್ಮತ್ತು
ಕಿರಿ-ದಾದ
ಕಿರಿದು
ಕಿರಿ-ಯುತ್ತ
ಕಿರು-ಚಿ-ಕೊಂಡ
ಕಿವಿ
ಕಿವಿ-ಗಳನ್ನು
ಕಿವಿ-ಗ-ಳಿಗೆ
ಕಿವಿಗೆ
ಕಿವಿ-ತೆ-ರೆ-ದು-ಕೊಂಡು
ಕಿವಿಯ
ಕಿವಿ-ಯನ್ನು
ಕಿವಿ-ಯಲ್ಲಿ
ಕಿವಿ-ಯಿಂದ
ಕೀಟಲೆ
ಕೀರಲು
ಕೀರ್ತ-ನ-ಕಾ-ರರು
ಕೀರ್ತಿ-ಯನ್ನು
ಕುಂಚ-ಗಳಿಂದ
ಕುಂಭಾ-ಭಿ-ಷೇಕ
ಕುಕ್ಕು-ತ್ತಿತ್ತು
ಕುಟುಂಬ
ಕುಟುಂ-ಬ-ದ-ಲ್ಲಾ-ದ್ದ-ರಿಂದ
ಕುಡಿಕೆ
ಕುಡಿ-ತಕ್ಕೂ
ಕುಣಿ-ಯಿರೈ
ಕುತೂ-ಹಲ
ಕುತೂ-ಹ-ಲ-ಕಾರಿ
ಕುತ್ತಿ-ಗೆಗೆ
ಕುದುರೆ
ಕುದು-ರೆ-ಗಳನ್ನು
ಕುದು-ರೆ-ಯೊಂ-ದನ್ನು
ಕುದು-ರೆ-ಸ-ವಾ-ರಿ-ಯಲ್ಲೂ
ಕುರಿ-ತಂತೆ
ಕುರಿ-ತಾಗಿ
ಕುರಿ-ತಾದ
ಕುರಿತು
ಕುರುಡು
ಕುಲ-ಕು-ಕಿ-ದರೂ
ಕುಲಕ್ಕೂ
ಕುಲದ
ಕುಲ-ದೇ-ವ-ತೆಯ
ಕುಲ-ಪು-ರೋ-ಹಿ-ತರು
ಕುಲುಕಿ
ಕುಳಿತ
ಕುಳಿ-ತರೆ
ಕುಳಿ-ತ-ವ-ನಲ್ಲ
ಕುಳಿ-ತಾಗ
ಕುಳಿ-ತಿದ್ದ
ಕುಳಿ-ತಿ-ದ್ದರು
ಕುಳಿ-ತಿ-ದ್ದರೆ
ಕುಳಿ-ತಿ-ದ್ದ-ವರೆಲ್ಲ
ಕುಳಿ-ತಿ-ದ್ದಾನೆ
ಕುಳಿ-ತಿ-ದ್ದಾರೆ
ಕುಳಿ-ತಿದ್ದೆ
ಕುಳಿ-ತಿ-ರಲು
ಕುಳಿ-ತಿ-ರು-ತ್ತಿದ್ದ
ಕುಳಿ-ತಿ-ರು-ತ್ತಿ-ದ್ದರು
ಕುಳಿ-ತಿ-ರು-ವುದನ್ನು
ಕುಳಿ-ತಿ-ರು-ವುದು
ಕುಳಿತು
ಕುಳಿ-ತು-ಕೊಂ-ಡ-ವನೇ
ಕುಳಿ-ತು-ಕೊಂಡು
ಕುಳಿ-ತು-ಕೊಳ್ಳ
ಕುಳಿ-ತು-ಕೊ-ಳ್ಳಲು
ಕುಳಿ-ತು-ಕೊ-ಳ್ಳು-ತ್ತಿದ್ದ
ಕುಳಿ-ತು-ಕೊ-ಳ್ಳುವ
ಕುಳಿ-ತು-ಕೊ-ಳ್ಳು-ವು-ದೇಕೆ
ಕುಳಿ-ತು-ಬಿಟ್ಟ
ಕುಳಿ-ತು-ಬಿ-ಟ್ಟದ್ದ
ಕುಳಿ-ತು-ಬಿ-ಟ್ಟರು
ಕುಳಿ-ತು-ಬಿಟ್ಟೆ
ಕುಳಿ-ತು-ಬಿ-ಡು-ತ್ತಿದ್ದ
ಕುಳಿ-ತು-ಬಿ-ಡು-ತ್ತಿದ್ದೆ
ಕುಳಿತೇ
ಕುಳ್ಳಿ-ರಿ-ಸಿ-ಕೊಂಡು
ಕುಶ-ಲಿ-ಗ-ಳಾದ
ಕುಸಿದು
ಕುಸ್ತಿ
ಕುಸ್ತಿ-ಯಾ-ಡು-ತ್ತಿ-ದ್ದರೆ
ಕೂಗದೇ
ಕೂಗಾ-ಡು-ತ್ತಿ-ರ-ಲಿಲ್ಲ
ಕೂಗಿ
ಕೂಗಿ-ಕೊಂ-ಡರು
ಕೂಗಿದ
ಕೂಗಿ-ದರೂ
ಕೂಗಿ-ದಾಗ
ಕೂಡ
ಕೂಡಲೇ
ಕೂಡಿ
ಕೂಡಿಟ್ಟ
ಕೂಡಿ-ಟ್ಟಿದ್ದ
ಕೂಡಿಟ್ಟು
ಕೂಡಿ-ಡು-ವು-ದ-ರಲ್ಲಿ
ಕೂಡಿದ
ಕೂಡಿ-ದ್ದಾ-ಗಿ-ರು-ತ್ತಿತ್ತು
ಕೂಡಿ-ರುವ
ಕೂಡಿ-ಸಲು
ಕೂಡಿಸಿ
ಕೂಡಿ-ಸಿ-ಕೊಂಡು
ಕೂಡಿ-ಸಿ-ಬಿ-ಟ್ಟರೆ
ಕೂಡಿ-ಹಾ-ಕಿ-ದಳು
ಕೂತುಂಡೇ
ಕೂತುಂ-ಬಗೆ
ಕೂದಲು
ಕೂರಿ-ಸಿ-ಕೊಂಡು
ಕೃತ-ಜ್ಞ-ತೆ-ಯ-ನ್ನ-ರ್ಪಿಸಿ
ಕೃತಿ-ಗಳನ್ನು
ಕೃಪೆ-ಯಿಂದ
ಕೃಷಿ-ಕನ
ಕೆಂಗಣ್ಣು
ಕೆಚ್ಚೆ-ದೆಯ
ಕೆಟ್ಟ
ಕೆಟ್ಟಂ-ತಾ-ಗಿ-ಬಿ-ಡು-ತ್ತಿತ್ತು
ಕೆರ-ಳಿತು
ಕೆರ-ವಾನ್
ಕೆಲ-ಕಾಲ
ಕೆಲ-ಕಾ-ಲದ
ಕೆಲ-ತಿಂ-ಗಳ
ಕೆಲ-ದಿ-ನ-ಗ-ಳಲ್ಲೇ
ಕೆಲ-ಮೊಮ್ಮೆ
ಕೆಲ-ವ-ರನ್ನು
ಕೆಲ-ವ-ರಿಗೆ
ಕೆಲ-ವರು
ಕೆಲವು
ಕೆಲವೇ
ಕೆಲ-ವೊಮ್ಮೆ
ಕೆಲಸ
ಕೆಲ-ಸಆ
ಕೆಲ-ಸ-ಕಾರ್ಯ
ಕೆಲ-ಸ-ಕಾ-ರ್ಯ-ಗಳನ್ನು
ಕೆಲ-ಸಕ್ಕೆ
ಕೆಲ-ಸ-ವನ್ನು
ಕೆಲ-ಸವೂ
ಕೆಲ-ಸ-ವೆಂದರೆ
ಕೆಲ-ಸ-ವೇನೂ
ಕೆಳ-ಕ್ಕೆ-ಸೆ-ದು-ಬಿಟ್ಟ
ಕೆಳ-ಗಾಗಿ
ಕೆಳ-ಗಿನ
ಕೆಳ-ಗಿ-ನ-ವ-ರೆಗೂ
ಕೆಳಗೆ
ಕೆಳಗೋ
ಕೆಳ-ಭಾ-ಗ-ವನ್ನು
ಕೇಳ-ಬೇಕು
ಕೇಳಲಿ
ಕೇಳ-ಲಿ-ಕ್ಕಿದೆ
ಕೇಳಲು
ಕೇಳಲೇ
ಕೇಳಿ
ಕೇಳಿ-ಕೊಂಡ
ಕೇಳಿ-ಕೊಂಡು
ಕೇಳಿ-ಕೊ-ಳ್ಳಲಿ
ಕೇಳಿದ
ಕೇಳಿ-ದರು
ಕೇಳಿ-ದರೂ
ಕೇಳಿ-ದರೆ
ಕೇಳಿ-ದಳು
ಕೇಳಿ-ದ-ವ-ರಿ-ಗೆಲ್ಲ
ಕೇಳಿ-ದಾಗ
ಕೇಳಿದ್ದ
ಕೇಳಿದ್ದು
ಕೇಳಿ-ನೋ-ಡ-ಬಾ-ರದು
ಕೇಳಿಯೇ
ಕೇಳಿ-ಸಿ-ಕೊಂಡ
ಕೇಳಿ-ಸಿತು
ಕೇಳು
ಕೇಳು-ಗ-ರ-ನ್ನದು
ಕೇಳುತ್ತ
ಕೇಳು-ತ್ತಾನೆ
ಕೇಳು-ತ್ತಿದ್ದ
ಕೇಳು-ತ್ತಿ-ದ್ದಂತೆ
ಕೇಳು-ತ್ತಿ-ದ್ದರು
ಕೇಳು-ತ್ತಿ-ದ್ದಾನೆ
ಕೇಳುವ
ಕೇಳು-ವಂ-ತೆಯೂ
ಕೇಳು-ವು-ದಕ್ಕೆ
ಕೇಳು-ವು-ದ-ಕ್ಕೆಂದೇ
ಕೇಳು-ವುದನ್ನು
ಕೇಳ್ತೀಯೋ
ಕೇವಲ
ಕೇಶ-ವ-ಚಂದ್ರ
ಕೇಶ-ವ-ಸೇ-ನ-ನಿಂದ
ಕೇಶಾ-ಕೇ-ಶಿ-ಗ-ಳಿಗೆ
ಕೈ
ಕೈಗ-ಡಿ-ಯಾ-ರ-ವನ್ನು
ಕೈಗಳಿಂದ
ಕೈಗೆ
ಕೈಗೆ-ತ್ತಿ-ಕೊಂಡ
ಕೈಗೊಂಡ
ಕೈಗೊಂಡು
ಕೈಗೊಂ-ಡೇನೋ
ಕೈಚ-ಳಕ
ಕೈತುಂಬ
ಕೈನೋಡಿ
ಕೈಬೀ-ಸಿ-ಕೊಂಡು
ಕೈಮೂಳೆ
ಕೈಯನ್ನು
ಕೈಯನ್ನೇ
ಕೈಯಲ್ಲಿ
ಕೈಯ-ಲ್ಲಿ-ರುವ
ಕೈಯಲ್ಲೂ
ಕೈಯಿಂದ
ಕೈಯಿಂ-ದಲೇ
ಕೈಲಾ-ಸಕ್ಕೆ
ಕೈಸು-ಟ್ಟು-ಕೊಂ-ಡಿರ
ಕೈಹ-ಚ್ಚಿ-ದರು
ಕೈಹ-ತ್ತ-ಲಿಲ್ಲ
ಕೈಹಾ-ಕಿದ
ಕೈಹಾ-ಕಿ-ದ-ವರು
ಕೈಹಿ-ಡಿದು
ಕೊಂಡ-ಕೊಂಡ
ಕೊಂಡರು
ಕೊಂಡಳು
ಕೊಂಡಾ-ಡು-ತ್ತಾರೆ
ಕೊಂಡಿತು
ಕೊಂಡಿ-ದ್ದರೆ
ಕೊಂಡು
ಕೊಂಡು-ಬಿ-ಟ್ಟಿದ್ದ
ಕೊಂಡೆ
ಕೊಂಡೇ-ಬಿಟ್ಟ
ಕೊಂಬೆಗೆ
ಕೊಂಬೆ-ಯಲ್ಲಿ
ಕೊಂಬೆ-ಯಿಂದ
ಕೊಗ-ಡಲು
ಕೊಟ್ಟ
ಕೊಟ್ಟದ್ದೂ
ಕೊಟ್ಟನೆ
ಕೊಟ್ಟಿತ್ತು
ಕೊಟ್ಟಿದ್ದ
ಕೊಟ್ಟು
ಕೊಟ್ಟು-ಬಿಟ್ಟ
ಕೊಟ್ಟು-ಬಿಟ್ಟೆ
ಕೊಟ್ಟು-ಬಿ-ಡು-ತ್ತಿದ್ದ
ಕೊಠ-ಡಿ-ಯಲ್ಲಿ
ಕೊಡ
ಕೊಡ-ದಿ-ದ್ದರೆ
ಕೊಡದೇ
ಕೊಡ-ಬೇ-ಕಲ್ಲ
ಕೊಡ-ಬೇ-ಕಾದ
ಕೊಡ-ಬೇಕು
ಕೊಡ-ಬೇಕೇ
ಕೊಡ-ಲಿ-ಲ್ಲ-ವೆಂದು
ಕೊಡಲು
ಕೊಡ-ಲೇ-ಬೇಕು
ಕೊಡಿ-ಸ-ಬ-ಹು-ದಾ-ಗಿತ್ತು
ಕೊಡಿ-ಸ-ಬೇಕು
ಕೊಡಿಸಿ
ಕೊಡಿ-ಸೋಣ
ಕೊಡು
ಕೊಡು-ಗೆಯೂ
ಕೊಡು-ತ್ತಿದ್ದ
ಕೊಡು-ತ್ತಿ-ದ್ದಳು
ಕೊಡು-ತ್ತಿಯೋ
ಕೊಡು-ತ್ತೇವೆ
ಕೊಡುವ
ಕೊಡು-ವಂತೆ
ಕೊಡು-ವಷ್ಟು
ಕೊಡು-ವಾಗ
ಕೊಡು-ವು-ದಕ್ಕೆ
ಕೊಡು-ವು-ದ-ಕ್ಕೋ-ಸ್ಕರ
ಕೊಡು-ವುದು
ಕೊಡ್ತಾರೆ
ಕೊನೆ-ಗಾಲ
ಕೊನೆಗೂ
ಕೊನೆಗೆ
ಕೊನೆಯ
ಕೊನೆ-ಯಲ್ಲಿ
ಕೊನೆ-ಯ-ವ-ರೆಗೂ
ಕೊರೆ-ಯು-ತ್ತಿ-ತ್ತು-ಒಂ-ದಾ-ದರೂ
ಕೊಲೆ-ಗ-ಡು-ಕ-ನನ್ನು
ಕೊಳವೆ
ಕೊಳ-ವೆ-ಗಳಿಂದ
ಕೊಳ್ಳಲು
ಕೊಳ್ಳು-ತ್ತಿದ್ದ
ಕೊಳ್ಳು-ತ್ತಿ-ದ್ದರು
ಕೊಳ್ಳು-ತ್ತಿ-ದ್ದಳು
ಕೊಳ್ಳು-ವಂ-ತಿದ್ದ
ಕೊಳ್ಳು-ವ-ವ-ನಲ್ಲ
ಕೋಂಡವು
ಕೋಟಿ-ಗ-ಟ್ಟಲೆ
ಕೋಟೆ-ಯನ್ನು
ಕೋಟ್ಯ-ಧೀ-ಶ-ರನ್ನೂ
ಕೋಣೆಗೆ
ಕೋಣೆಯ
ಕೋಣೆ-ಯನ್ನು
ಕೋಣೆ-ಯಲ್ಲಿ
ಕೋಣೆ-ಯೊ-ಳಗೆ
ಕೋತಿ
ಕೋನ-ಗಳಿಂದ
ಕೋಪ
ಕೋಪ-ಗೊಂಡು
ಕೋಪ-ದಿಂದ
ಕೋಪ-ವನ್ನು
ಕೋಪಿ-ಷ್ಟರು
ಕೋಪಿ-ಸಿ-ಕೊಂಡು
ಕೋಪಿ-ಸಿ-ಕೊಂ-ಡು-ಬಿ-ಟ್ಟೆಯಾ
ಕೋರಿದ
ಕೌಟುಂ-ಬಿಕ
ಕೌಮಾರ್ಯ
ಕೌಮಾ-ರ್ಯಕ್ಕೆ
ಕೌಶ-ಲ-ದಿಂ-ದಲೇ
ಕೌಶ-ಲ-ವನ್ನು
ಕ್ಕಾಗಿಯೇ
ಕ್ಕೇನೂ
ಕ್ಕೊಳ-ಗಾ-ದ-ವನೇ
ಕ್ರಮ-ದಲ್ಲಿ
ಕ್ರಮ-ಪ್ರ-ಕಾರ
ಕ್ರಮ-ಪ್ರ-ಕಾ-ರ-ವಾ-ಗಿಯೇ
ಕ್ರಮ-ವನ್ನು
ಕ್ರಮ-ವೆಲ್ಲ
ಕ್ರಮವೇ
ಕ್ರಮೇಣ
ಕ್ರಾಂತಿ-ಕಾರೀ
ಕ್ರಿಕೆಟ್
ಕ್ರಿಕೆ-ಟ್ನಲ್ಲಿ
ಕ್ರಿಯಾ-ತ್ಮಕ
ಕ್ರಿಯಾ-ಶೀ-ಲ-ವಾಗಿ
ಕ್ರೀಡೆ-ಗಳಲ್ಲಿ
ಕ್ರೂರ
ಕ್ರೂರ-ತನ
ಕ್ರೈಸ್ತ
ಕ್ಷಣ
ಕ್ಷಣ-ಕಾಲ
ಕ್ಷಣ-ಕಾ-ಲವೂ
ಕ್ಷಣವೇ
ಕ್ಷತ್ರಿಯ
ಕ್ಷೆಮೆ
ಕ್ಷೇತ್ರ
ಖ
ಖಂಡಿತ
ಖಂಡಿ-ತ-ವಾಗಿ
ಖಂಡಿ-ತ-ವಾ-ಗಿಯೂ
ಖಂಡಿ-ತ-ವಾದಿ
ಖಂಡಿ-ಸಿ-ಬಿ-ಡು-ತ್ತಿದ್ದ
ಖಡ-ಕ್ಕಾಗಿ
ಖರ್ಚಿ
ಖರ್ಚಿಗೆ
ಖರ್ಚು
ಖರ್ಚೂ
ಖಾನ-ನಿಂದ
ಖಾನ್
ಖಾನ್-ಇ-ವ-ರಿಂ-ದಲೂ
ಖಾರ
ಖಾರ-ಮುಂದು
ಖಾರವೇ
ಖಾರವೋ
ಖಿಚ-ಡಿ-ಒಂದು
ಖಿಚ-ಡಿಗೆ
ಖುಷಿ
ಖುಷಿ-ಪ-ಡಿ-ಸಲು
ಖುಷಿ-ಯಾ-ಗಿ-ಡು-ತ್ತಿದ್ದ
ಖುಷಿ-ಯಿಂ-ದಿ-ರ-ಬೇಕು
ಖುಷಿಯೊ
ಗಂಗಾ-ನ-ದಿಗೆ
ಗಂಗಾ-ನ-ದಿಯ
ಗಂಗಾ-ನ-ದಿ-ಯಲ್ಲಿ
ಗಂಗಾ-ಸ್ನಾ-ನವೂ
ಗಂಗೆಯ
ಗಂಟಿ-ಕ್ಕು-ವು-ದರ
ಗಂಟು-ಮೋರೆ
ಗಂಟೆ
ಗಂಟೆ-ಗ-ಟ್ಟಲೆ
ಗಂಟೆ-ಗಳ
ಗಂಟೆ-ಗಳನ್ನು
ಗಂಟೆಯ
ಗಂಟೆ-ಯ-ವ-ರೆಗೂ
ಗಂಡಲ್ಲ
ಗಂಡಾಂ-ತರ
ಗಂಡಾಳು
ಗಂಡು
ಗಂಡು-ಗ-ಲಿ-ಗ-ಳಂ-ತಿ-ರ-ಬೇ-ಕೆಂ-ಬುದು
ಗಂಡು-ಗ-ಲಿ-ಯಾ-ಗ-ಲಿ-ರುವ
ಗಂಡು-ಮ-ಗುವೂ
ಗಂಡು-ಸಂ-ತಾ-ನ-ವನ್ನು
ಗಂಭೀರ
ಗಂಭೀ-ರ-ಭಾವ
ಗಂಭೀ-ರ-ವಾಗಿ
ಗಂಭೀ-ರ-ವಾ-ಗುತ್ತ
ಗಂಭೀ-ರ-ವಾ-ಹಿ-ನಿ-ಯಾದ
ಗಟ್ಟಿ
ಗಟ್ಟಿ-ಯಾಗಿ
ಗಟ್ಟಿ-ಯಾಗೇ
ಗಡ-ಸು-ದ-ನಿ-ಯಲ್ಲಿ
ಗಣಿತ
ಗಣಿ-ತ-ದಲ್ಲಿ
ಗಣೇಶ
ಗಣ್ಯ
ಗತಿ
ಗತ್ತು
ಗದ-ರಿ-ದರೂ
ಗದ-ರಿಸಿ
ಗದ-ರಿ-ಸಿ-ದಾಗ
ಗದ-ರಿ-ಸುತ್ತ
ಗದ್ಯ-ವಾಗಿ
ಗಮನ
ಗಮ-ನ-ಕೊ-ಟ್ಟಿ-ರ-ಬೇಕು
ಗಮ-ನ-ಕೊಟ್ಟು
ಗಮ-ನ-ಕೊ-ಡದೆ
ಗಮ-ನಕ್ಕೆ
ಗಮ-ನ-ವನ್ನೂ
ಗಮ-ನ-ವಿಟ್ಟು
ಗಮ-ನ-ವಿತ್ತು
ಗಮ-ನ-ವೆಲ್ಲ
ಗಮ-ನಾರ್ಹ
ಗಮ-ನಿಸ
ಗಮ-ನಿಸಿ
ಗಮ-ನಿ-ಸಿದ
ಗಮ-ನಿ-ಸಿ-ದೆ-ಒಂದು
ಗಮ-ನಿ-ಸು-ತ್ತಿದ್ದೆ
ಗರ-ಡಿ-ಮ-ನೆಗೆ
ಗರಿ-ಮುರಿ
ಗರುಡ
ಗಲ-ಭೆ-ಯೆ-ಬ್ಬಿ-ಸಿ-ದರು
ಗಲಾಟೆ
ಗಲಾ-ಟೆ-ಗಿಟ್ಟು
ಗಲಾ-ಟೆ-ಗೆಲ್ಲ
ಗಲಾ-ಟೆ-ಯಾ-ದರೂ
ಗಲಾ-ಟೆಯೂ
ಗಲಾ-ಟೆ-ಯೆಲ್ಲ
ಗಲೂ
ಗಲೆಲ್ಲ
ಗಲ್ಲಿಯ
ಗಳ
ಗಳನ್ನು
ಗಳನ್ನೂ
ಗಳ-ನ್ನೆಲ್ಲ
ಗಳ-ನ್ನೆಲ್ಲಾ
ಗಳಲ್ಲಿ
ಗಳಲ್ಲೇ
ಗಳಿಂದ
ಗಳಿಂ-ದಲೂ
ಗಳಿ-ಸ-ಬೇಕು
ಗಳಿ-ಸಲು
ಗಳಿ-ಸಿದ
ಗಳಿ-ಸಿದ್ದ
ಗಳಿ-ಸು-ತ್ತಿದ್ದ
ಗಳಿ-ಸು-ವುದೇ
ಗಳು
ಗಳುಈ
ಗಳೆಲ್ಲ
ಗವಾ-ಯಿ-ಗಳ
ಗಹ-ನ-ವಾ-ದದ್ದು
ಗಾಂಭೀರ್ಯ
ಗಾಂಭೀ-ರ್ಯ-ದಿಂದ
ಗಾಂಭೀ-ರ್ಯ-ವಿತ್ತು
ಗಾಗಿ
ಗಾಟಿ-ಕೆ-ತುಂ-ಟ-ತ-ನ-ಗಳ
ಗಾಡು-ತ್ತಾನೆ
ಗಾಢ
ಗಾಢ-ಚಿಂ-ತನೆ
ಗಾಢ-ವಾಗಿ
ಗಾಢ-ವಾ-ಗಿರು
ಗಾನ-ವನ್ನು
ಗಾಬರಿ
ಗಾಬ-ರಿ-ಯಾಗಿ
ಗಾಬ-ರಿ-ಯಾ-ಯಿತು
ಗಾಯ
ಗಾಯಕ
ಗಾಯ-ಕ-ನ-ನ್ನಾಗಿ
ಗಾಯಕ್ಕೆ
ಗಾಯದ
ಗಾಯನ
ಗಾಯ-ನ-ಪ್ರ-ತಿ-ಭೆ-ಯೆ-ನ್ನು-ವುದು
ಗಾಯ-ವಾಗಿ
ಗಾಯವೇ
ಗಾಳಿ
ಗಿಯೇ
ಗಿಳಿ-ಗಳು
ಗೀತೆ-ಗಳು
ಗುಂಡು-ಗುಂ-ಡಾದ
ಗುಂಪಿ-ನಲ್ಲಿ
ಗುಟ್ಟು
ಗುಡು-ಗ-ಲಿ-ರುವ
ಗುಡು-ಗಿ-ದರು
ಗುಣ
ಗುಣ-ಗಳನ್ನೂ
ಗುಣ-ಗಳಿಂದ
ಗುಣ-ಗ-ಳಿವೆ
ಗುಣ-ಗಳು
ಗುಣ-ಗಳೇ
ಗುಣ-ಗಾನ
ಗುಣ-ನ-ಡ-ತೆಈ
ಗುಣ-ಮಾ-ಧುರ್ಯ
ಗುಣ-ವಂ-ತ-ರ-ನ್ನಾಗಿ
ಗುಣ-ವಿ-ಶೇ-ಷ-ಗಳನ್ನು
ಗುಣ-ವಿ-ಶೇ-ಷ-ವೆಂದರೆ
ಗುದ್ದನ್ನೂ
ಗುದ್ದು
ಗುಮಾಸ್ತೆ
ಗುಮಾ-ಸ್ತೆಯ
ಗುಮಾ-ಸ್ತೆ-ಯನ್ನು
ಗುರ-ಯಾ-ದ-ವರ
ಗುರಾಣಿ
ಗುರಿ-ಪ-ಡಿಸಿ
ಗುರಿ-ಯಾಗು
ಗುರಿ-ಯಾ-ದಾಗ
ಗುರು
ಗುರು-ಕು-ಲ-ದಂ-ತಹ
ಗುರು-ಗ-ಳನ್ನೇ
ಗುರು-ಗಳು
ಗುರು-ತರ
ಗುರು-ತಿಸಿ
ಗುರು-ತಿ-ಸಿ-ದ-ವ-ರುಂಟು
ಗುರು-ತಿ-ಸಿದ್ದ
ಗುರುತು
ಗುರು-ವಿನ
ಗುಳು-ಗು-ಳು-ಗುಳು
ಗುಳು-ಗು-ಳು-ಗು-ಳು-ಗುಳು
ಗುಹ
ಗೃಹ-ಸ್ಥ-ನಂತೆ
ಗೆದ್ದ
ಗೆದ್ದು
ಗೆಲು
ಗೆಲ್ಲ
ಗೆಲ್ಲಲು
ಗೆಲ್ಲು-ವು-ದಂತೂ
ಗೆಳೆಯ
ಗೆಳೆ-ಯ-ನನ್ನು
ಗೆಳೆ-ಯ-ನಿಗೆ
ಗೆಳೆ-ಯರ
ಗೆಳೆ-ಯ-ರನ್ನು
ಗೆಳೆ-ಯ-ರೊಂ-ದಿ-ಗಿನ
ಗೆಳೆ-ಯ-ರೊಂ-ದಿಗೆ
ಗೇನೂ
ಗೇಲಿ
ಗೈರು-ಹಾ-ಜ-ರಾ-ಗಿ-ದ್ದಾನೆ
ಗೊಂಡರು
ಗೊಂಡಿತು
ಗೊಂಬೆ-ಗಳನ್ನು
ಗೊತ್ತಾ
ಗೊತ್ತಾ-ಗ-ಲಿಲ್ಲ
ಗೊತ್ತಾಗಿ
ಗೊತ್ತಾ-ಯಿತು
ಗೊತ್ತಿತ್ತು
ಗೊತ್ತಿಲ್ಲ
ಗೊತ್ತು
ಗೊಳಿ-ಸಿ-ದ್ದುವು
ಗೋಚ-ರ-ವಾ-ಯಿತು
ಗೋಚ-ರಿ-ಸು-ವು-ದುಂಟು
ಗೋಜಿಗೆ
ಗೋಡೆಯ
ಗೋಪು-ರ-ಗ-ಳೆಲ್ಲ
ಗೋಲಿ-ಯಾ-ಟ-ಎ-ಲ್ಲವೂ
ಗೋಳು
ಗೋಳು-ಹೊ-ಯ್ದು-ಕೊ-ಳ್ಳು-ವು-ದೆಂ-ದರೆ
ಗೋಷ್ಠಿ-ಗ-ಳಿಗೆ
ಗೋಸ್ಕ-ರವೇ
ಗೋಸ್ವಾಮಿ
ಗೌರವ
ಗೌರ-ವಕ್ಕೂ
ಗೌರ-ವ-ಭಾ-ವ-ವಿತ್ತು
ಗೌರವಾ
ಗೌರ-ವಾ-ದರ
ಗೌರ-ವಾ-ದ-ರ-ಗಳಿಂದ
ಗೌರ-ವಿಸಿ
ಗೌರ-ವಿ-ಸು-ವು-ದಿ-ಲ್ಲವೋ
ಗ್ರಂಥ-ಗಳ
ಗ್ರಂಥ-ಗಳನ್ನು
ಗ್ರಂಥದ
ಗ್ರಂಥ-ವನ್ನೇ
ಗ್ರಂಥವು
ಗ್ರಹ-ಣ-ಸಾ-ಮರ್ಥ್ಯ
ಗ್ರಹ-ಬಲ
ಗ್ರಹಿ-ಸಲು
ಗ್ರಹಿಸಿ
ಗ್ರಹಿ-ಸಿ-ದ್ದಳು
ಗ್ರಹಿ-ಸುವ
ಘಟನೆ
ಘಟ-ನೆ-ಗಳನ್ನೂ
ಘಟ-ನೆ-ಗಳು
ಘಟ-ನೆಯ
ಘಟ-ನೆ-ಯನ್ನು
ಘಟ-ನೆ-ಯನ್ನೂ
ಘಟ-ನೆ-ಯಾ-ಯಿತು
ಘಟ-ನೆ-ಯಿಂ-ದಾಗಿ
ಘಟ-ನೆ-ಯಿದೆ
ಘಟ-ನೆಯೋ
ಘನ
ಘನ-ತ-ರ-ವಾ-ದದ್ದು
ಘನತೆ
ಘನ-ತೆ-ವೆ-ತ್ತ-ವರು
ಘೋಷ-ಣೆ-ಯಾ-ಗಿತ್ತು
ಘೋಷಿ-ಸಿ-ಕೊಂ-ಡಿತ್ತು
ಘೋಷಿಸು
ಘೋಷಿ-ಸು-ವು-ದ-ಕ್ಕಾ-ಗಿಯೇ
ಚಂದ
ಚಂದವೋ
ಚಂಪ-ಕ-ಪುಷ್ಪ
ಚಕ್ರ-ವರ್ತಿ
ಚಕ್ರ-ವ-ರ್ತಿಗೆ
ಚಕ್ರ-ವ-ರ್ತಿಯ
ಚಕ್ರ-ವ-ರ್ತಿ-ಯಾಗಿ
ಚಕ್ರ-ವ-ರ್ತಿ-ಯಾದ
ಚಟ
ಚಟರ್ಜಿ
ಚಟಾ-ಕಿ-ಗಳನ್ನು
ಚಟು-ವ-ಟಿಕೆ
ಚಟು-ವ-ಟಿ-ಕೆ-ಯಿಂದ
ಚಟು-ವ-ಟಿ-ಕೆ-ಯಿಂ-ದಿ-ರು-ತ್ತಿ-ದ್ದು-ದೇನೋ
ಚಡ-ಪ-ಡಿ-ಕೆ-ಯ-ನ್ನುಂ-ಟು-ಮಾ-ಡು-ತ್ತಿತ್ತು
ಚಡ-ಪ-ಡಿ-ಸಿದ
ಚಡ-ಪ-ಡಿ-ಸು-ತ್ತಿದ್ದೆ
ಚಡ್ಡಿ
ಚತುರ
ಚದು-ರಂ-ಗ-ವನ್ನು
ಚಪ್ಪಾಳೆ
ಚರಿತ
ಚರಿ-ತ್ರೆ-ಯಲ್ಲಿ
ಚರ್ಚೆ
ಚರ್ಚೆ-ಇ-ವು-ಗ-ಳಿಗೆ
ಚರ್ಚೆ-ಗಳನ್ನು
ಚರ್ಚೆಯ
ಚರ್ಚೆ-ಯನ್ನು
ಚರ್ಚೆಯೇ
ಚರ್ಚ್
ಚಲಾ-ಯಿ-ಸುತ್ತ
ಚಲಿ-ಸುತ್ತ
ಚಳ-ವ-ಳಿ-ಯನ್ನು
ಚಳ-ವ-ಳಿ-ಯಾ-ಗಲಿ
ಚಹರೆ
ಚಹಾ
ಚಾಕ-ಚ-ಕ್ಯ-ತೆ-ಯಿಂದ
ಚಾಚಿ-ಕೊ-ಳ್ಳು-ವುದು
ಚಾಟೂ-ಕ್ತಿ-ಗಳಿಂದ
ಚಾತು-ರ್ಯ-ದಿಂದ
ಚಾಪೆ
ಚಾಪೆ-ಯನ್ನು
ಚಾರಿತ್ರ್ಯ
ಚಾರಿ-ತ್ರ್ಯ-ವನ್ನು
ಚಾವಟಿ
ಚಿಂತ-ನ-ಶೀ-ಲ-ನಾದ
ಚಿಂತಿ-ಸ-ಬೇಡ
ಚಿಂತಿ-ಸುತ್ತ
ಚಿಂತೆ
ಚಿಕ್ಕ
ಚಿಕ್ಕ-ಚಿಕ್ಕ
ಚಿಕ್ಕಪ್ಪ
ಚಿಕ್ಕ-ಪ್ಪ-ನನ್ನು
ಚಿಕ್ಕ-ಪ್ಪ-ನಾದ
ಚಿಕ್ಕಮ್ಮ
ಚಿಕ್ಕ-ಮ್ಮನ
ಚಿಕ್ಕ-ಮ್ಮ-ನಿ-ಗೊಂದು
ಚಿಕ್ಕ-ವನು
ಚಿಟ್ಟೆ
ಚಿತ್ರ
ಚಿತ್ರ-ಗಳನ್ನು
ಚಿತ್ರಣ
ಚಿತ್ರ-ವಿ-ಚಿ-ತ್ರ-ವಾಗಿ
ಚಿತ್ರವೇ
ಚಿತ್ರಿ-ಸಿ-ದರೆ
ಚಿಮ್ಮಿತು
ಚಿಮ್ಮು-ತ್ತಿ-ದೆ-ಓ-ಜಸ್ಸು
ಚಿರ-ಮು-ದ್ರಿ-ತ-ವಾಗಿ
ಚೀಲ-ದಲ್ಲಿ
ಚುರು-ಕಾಗಿ
ಚೂರು-ಚೂ-ರಾ-ದುವು
ಚೆನ್ನಾಗಿ
ಚೆನ್ನಾ-ಗಿದೆ
ಚೆನ್ನಾ-ಗಿಯೇ
ಚೆಲ್ಲು-ತ-ನ-ವಿ-ರ-ಲಿಲ್ಲ
ಚೇತ-ರಿ-ಸಿ-ಕೊಂ-ಡ-ವ-ನಂತೆ
ಚೇತ-ರಿ-ಸಿ-ಕೊಂಡು
ಚೇತೋ-ಹಾ-ರಿ-ಯಾ-ಗಿತ್ತು
ಚೇಷ್ಟೆ
ಚೇಷ್ಟೆ-ಗಳನ್ನೆಲ್ಲ
ಚೇಷ್ಟೆ-ಯನ್ನು
ಚೋಟುದ್ದ
ಚೌಕ-ಟ್ಟನ್ನು
ಛೀಛೀ-ಛೀಛೀ
ಛೀಮಾರಿ
ಛೆ
ಛೇಡಿಸಿ
ಜಂಬ-ದಿಂದ
ಜಗ-ತ್ತನ್ನೇ
ಜಗ-ತ್ತಿನ
ಜಗ-ತ್ತಿ-ನಲ್ಲಿ
ಜಗ-ತ್ತಿ-ನಾ-ದ್ಯಂತ
ಜಗ-ನ್ನಾಥ
ಜಗ-ಲಿ-ಯಿಂದ
ಜಗಳ
ಜಗ್ಗಾ-ಟ-ವಿ-ರ-ಲಿಲ್ಲ
ಜಗ್ಗಿ
ಜಗ್ಗು-ವ-ವ-ನಲ್ಲ
ಜಜ್ಜಿ
ಜಟ-ಕಾ-ದಲ್ಲಿ
ಜಟೆ-ಗ-ಟ್ಟ-ಬೇ-ಕಾ-ದರೆ
ಜಟೆ-ಗ-ಟ್ಟಲೇ
ಜಟೆ-ಗಳು
ಜಟೆ-ಗ-ಳೆಲ್ಲ
ಜಟೆ-ಯಾ-ಗಲೂ
ಜಟೆ-ಯಾ-ಗಿ-ದೆಯೇ
ಜಡ
ಜನ
ಜನಕ
ಜನ-ಜೀ-ವ-ನದ
ಜನ-ಜೀ-ವ-ನ-ವನ್ನು
ಜನ-ತೆ-ಯನ್ನು
ಜನ-ಪದ
ಜನ-ಪ್ರಿ-ಯತೆ
ಜನ-ಪ್ರಿ-ಯ-ವಾ-ಯಿತು
ಜನರ
ಜನ-ರಲ್
ಜನ-ರಲ್ಲಿ
ಜನ-ರಿ-ಗೆಲ್ಲ
ಜನರು
ಜನರೂ
ಜನ-ರೆಲ್ಲ
ಜನ-ರೊಂ-ದಿಗೆ
ಜನ-ವರಿ
ಜನ-ವ-ರ್ಗ-ಗಳು
ಜನ-ಸಂ-ದ-ಣಿ-ಯ-ಲ್ಲೆಲ್ಲೋ
ಜನ-ಸಾ-ಮಾ-ನ್ಯರು
ಜನಾಂಗ
ಜನಿ-ಸು-ತ್ತೇನೆ
ಜನ್ಮ
ಜನ್ಮ-ಗಳ
ಜಪ
ಜಪ-ಧ್ಯಾ-ನ-ಗಳನ್ನು
ಜಯ-ಭೇರಿ
ಜಯ-ವಾ-ಗಲಿ
ಜಯ-ಶಾಲಿ
ಜಯ-ಶಾ-ಲಿ-ಯಾ-ಗು-ತ್ತಿದ್ದ
ಜಯ-ಶಾ-ಲಿ-ಯಾ-ಯಿತು
ಜರು-ಗಿ-ದಾಗ
ಜಲ-ಮಾ-ರ್ಗ-ವಾ-ಗಿಯೇ
ಜವಾ
ಜವಾ-ನರು
ಜವಾ-ಬ್ದಾ-ರಿ-ಯನ್ನೂ
ಜಾಗ
ಜಾಗಕ್ಕೆ
ಜಾಗ-ದಲ್ಲಿ
ಜಾಗ-ದಿಂದ
ಜಾಗೃ-ತ-ವಾ-ಗಿ-ತ್ತೆ-ನ್ನು-ವುದು
ಜಾಗೃ-ತ-ವಾ-ಗಿ-ಬಿ-ಟ್ಟಿತ್ತು
ಜಾಗೃ-ತ-ವಾ-ಗಿ-ಬಿ-ಟ್ಟಿದೆ
ಜಾಗೃ-ತ-ವಾ-ಗಿ-ಬಿ-ಟ್ಟುವು
ಜಾಗೃ-ತ-ವಾ-ಗು-ತ್ತಿತ್ತು
ಜಾಣ-ತನ
ಜಾತಕ
ಜಾತ-ಕ-ವನ್ನು
ಜಾತಿ-ಕು-ಲ-ಗೋತ್ರ
ಜಾತಿ-ನಿ-ಯ-ಮ-ವನ್ನು
ಜಾತಿ-ಪ-ದ್ಧತಿ
ಜಾತಿ-ಭೇ-ದ-ದಿಂದ
ಜಾತಿ-ಭೇ-ದ-ವನ್ನು
ಜಾತಿ-ಯ-ವರ
ಜಾತಿ-ಯ-ವ-ರಾ-ದರೂ
ಜಾತಿ-ಯ-ವ-ರಿಗೆ
ಜಾತಿ-ಯ-ವರು
ಜಾತಿ-ಯ-ವ-ರೆಂದು
ಜಾತಿ-ಯೆ-ನ್ನು-ವು-ದೊಂದು
ಜಾತ್ರೆಗೆ
ಜಾತ್ರೆ-ಯಲ್ಲಿ
ಜಾದು-ಗಾರ
ಜಾರಿ
ಜಾಲಾಡಿ
ಜಿಗಿದ
ಜಿಗಿ-ಯ-ಬ-ಹುದು
ಜಿಗಿ-ಯು-ತ್ತಾನೆ
ಜಿನು-ಗುವ
ಜೀವ-ಕಳೆ
ಜೀವದ
ಜೀವನ
ಜೀವ-ನ-ಕ್ಕಿ-ಳಿದು
ಜೀವ-ನ-ಕ್ರ-ಮದ
ಜೀವ-ನ-ಕ್ರ-ಮ-ವನ್ನು
ಜೀವ-ನ-ತ-ತ್ತ್ವಕ್ಕೆ
ಜೀವ-ನದ
ಜೀವ-ನ-ದಂ-ತಲ್ಲ
ಜೀವ-ನ-ದಲ್ಲಿ
ಜೀವ-ನ-ದು-ದ್ದಕ್ಕೂ
ಜೀವ-ನ-ರಂ-ಗಕ್ಕೆ
ಜೀವ-ನ-ವನ್ನು
ಜೀವ-ನ-ವನ್ನೂ
ಜೀವ-ನ-ವನ್ನೇ
ಜೀವ-ನವು
ಜೀವ-ನವೇ
ಜೀವ-ನ-ವೊಂದು
ಜೀವಿ-ಸುತ್ತ
ಜೀವಿ-ಸುವ
ಜುಗುಪ್ಸೆ
ಜೇಡಿ-ಮ-ಣ್ಣಿ-ನಿಂದ
ಜೇನು-ಕಂ-ಠ-ದಿಂದ
ಜೇನು-ಗೂಡು
ಜೇನ್ನೊ-ಣ-ಗಳ
ಜೊತೆ
ಜೊತೆಗೆ
ಜೊತೆ-ಯಲ್ಲಿ
ಜೊತೆ-ಯ-ಲ್ಲಿ-ರು-ವಾಗ
ಜೊತೆ-ಯಲ್ಲೂ
ಜೊತೆ-ಯಲ್ಲೇ
ಜೋತಾ-ಡ-ಬ-ಹುದು
ಜೋತಾ-ಡಿ-ಸಿ-ಕೊಂಡು
ಜೋತಾಡು
ಜೋತಾ-ಡು-ತ್ತಿದ್ದ
ಜೋತಾ-ಡು-ತ್ತಿ-ದ್ದಾನೆ
ಜೋರಾಗಿ
ಜೋಲು
ಜ್ಞಾನ
ಜ್ಞಾನ-ದಾ-ಹವೂ
ಜ್ಞಾನ-ಪೀ-ಠದ
ಜ್ಯೋತಿ
ಜ್ಯೋತಿ-ದ-ರ್ಶನ
ಜ್ಯೋತಿಯ
ಜ್ಯೋತಿ-ರ್ಮಯ
ಜ್ಯೋತಿಷ
ಜ್ಯೋತಿ-ಷ-ವನ್ನೂ
ಜ್ವರ
ಜ್ವರ-ದಿಂದ
ಜ್ವಲಿ-ಸ-ಬೇ-ಕಾದ
ಟಾಗೋ-ರರು
ಟಾಗೋರ್
ಟೀಕಿ-ಸಿ-ದ್ದುಂಟು
ಟೀಕಿ-ಸುತ್ತ
ಟೀಕೆ
ಠಣಲ್
ಠೀವಿ
ಠೀವಿ-ಯಿಂದ
ಡಾ
ಡ್ತಿರುತ್ತೆ
ಣಿಕೆಯ
ತಂಟೆ
ತಂಟೆ
ತಂಟೆ-ಮಾರಿ
ತಂಟೆಯ
ತಂಡ
ತಂತಮ್ಮ
ತಂತಿ-ವಾ-ದ್ಯ-ಗಳನ್ನೂ
ತಂದ
ತಂದದ್ದೂ
ತಂದರು
ತಂದ-ವನೇ
ತಂದಿತು
ತಂದಿದ್ದ
ತಂದು
ತಂದು-ಕೊಂ-ಡಳು
ತಂದು-ಕೊಟ್ಟ
ತಂದು-ಬಿ-ಟ್ಟಿದ್ದ
ತಂದೆ
ತಂದೆ-ತಾಯಿ
ತಂದೆಗೂ
ತಂದೆ-ತಾ-ಯಂ-ದಿರು
ತಂದೆ-ತಾ-ಯಿ-ಗಳು
ತಂದೆಯ
ತಂದೆ-ಯಂತೆ
ತಂದೆ-ಯನ್ನು
ತಂದೆ-ಯಾ-ದ-ವನು
ತಂದೆ-ಯಿಂದ
ತಕ್ಕ
ತಕ್ಕಂತೆ
ತಕ್ಷಣ
ತಕ್ಷ-ಣವೇ
ತಗು-ಲಿತು
ತಗು-ಲಿ-ಸಿ-ಕೊಂಡು
ತಟ್ಟಿ
ತಟ್ಟಿ-ದರು
ತಟ್ಟಿ-ದಾ-ಗಲೇ
ತಟ್ಟುತ್ತ
ತಡ-ಮಾ-ಡದೆ
ತಡೆ-ದು-ಕೊಂಡು
ತಡೆ-ದು-ಕೊ-ಳ್ಳ-ಲಾ-ರದೆ
ತಡೆ-ಯಲು
ತಡೆ-ಹಿ-ಡಿ-ಯು-ತ್ತಿತ್ತು
ತಣಿ-ಸಲು
ತಣ್ಣ-ಗಾ-ಗಿ-ಬಿ-ಡು-ತ್ತಿದ್ದ
ತಣ್ಣೀ-ರನ್ನು
ತಣ್ಣೀರು
ತತ್ತ್ವ
ತತ್ತ್ವ-ಸಂ-ದೇ-ಶ-ಗಳನ್ನು
ತತ್ತ್ವ-ಗಳನ್ನು
ತತ್ತ್ವ-ವನ್ನು
ತತ್ತ್ವಶಃ
ತತ್ತ್ವ-ಶಾಸ್ತ್ರ
ತತ್ತ್ವ-ಶಾ-ಸ್ತ್ರ-ಜ್ಞರ
ತತ್ತ್ವ-ಶಾ-ಸ್ತ್ರ-ವನ್ನೂ
ತಥ್ಯಾಂಶ
ತದ-ನು-ಸಾ-ರ-ವಾಗಿ
ತನಕ
ತನ-ಗ-ರಿ-ವಿ-ಲ್ಲ-ದಂ-ತೆಯೇ
ತನಗೂ
ತನಗೆ
ತನಗೇ
ತನ-ಗೇನು
ತನ-ಗೇನೂ
ತನ್ನ
ತನ್ನಂ-ತಹ
ತನ್ನದೇ
ತನ್ನನ್ನು
ತನ್ನನ್ನೇ
ತನ್ನಲ್ಲಿ
ತನ್ನವ
ತನ್ನ-ವ-ರ-ನ್ನಾ-ಗಿ-ಸಿ-ಕೊ-ಳ್ಳುವ
ತನ್ನ-ಷ್ಟಕ್ಕೆ
ತನ್ನಿಂದ
ತನ್ನೆ-ಡೆಗೆ
ತನ್ನೊಂ-ದಿಗೆ
ತನ್ನೊ-ಳಗೆ
ತನ್ಮ-ಯ-ನಾ-ಗಿ-ದ್ದು-ಕೊಂಡು
ತಪಸ್ಸು
ತಪ್ಪದೆ
ತಪ್ಪನ್ನು
ತಪ್ಪ-ಲಲ್ಲೋ
ತಪ್ಪಿ
ತಪ್ಪಿ-ಸಿ-ಕೊಂಡು
ತಪ್ಪಿ-ಸಿ-ಕೊ-ಳ್ಳಲೇ
ತಪ್ಪಿ-ಸುವ
ತಪ್ಪಿ-ಹೋ-ಯಿತು
ತಪ್ಪು
ತಪ್ಪು-ಗಳನ್ನು
ತಪ್ಪು-ದಾರಿ-ಗೆ-ಳೆ-ಯುವ
ತಬ್ಬಲಿ
ತಬ್ಬ-ಲಿ-ಯಾಗಿ
ತಬ್ಬ-ಲಿ-ಯಾದ
ತಬ್ಬ-ಲಿ-ಯಾ-ದರೂ
ತಬ್ಬಿ-ಕೊಂಡು
ತಬ್ಬಿ-ಬ್ಬಾ-ದರು
ತಬ್ಬಿ-ಬ್ಬಾ-ಯಿತು
ತಬ್ಬಿಬ್ಬು
ತಮ-ಗ-ರಿ-ವಿ-ಲ್ಲ-ದೆಯೇ
ತಮಗೇ
ತಮ-ತ-ಮಗೆ
ತಮಾಷೆ
ತಮಾ-ಷೆ-ಗಳಲ್ಲಿ
ತಮಾ-ಷೆ-ಗಳಿಂದ
ತಮಾ-ಷೆ-ಗಳು
ತಮಾ-ಷೆ-ಗ-ಳೆಲ್ಲ
ತಮಾ-ಷೆಯ
ತಮಾ-ಷೆ-ಯ-ನ್ನು-ಗು-ರು-ತಿ-ಸು-ವಲ್ಲಿ
ತಮಾ-ಷೆಯೂ
ತಮ್ಮ
ತಮ್ಮಂ-ದಿರು
ತಮ್ಮ-ಊ-ಹಾ-ಶ-ಕ್ತಿ-ಯನ್ನು
ತಮ್ಮ-ತಮ್ಮ
ತಮ್ಮದೇ
ತಮ್ಮ-ನಿಗೆ
ತಮ್ಮನ್ನು
ತಮ್ಮ-ನ್ನೆಂ-ದಿಗೂ
ತಮ್ಮ-ವನು
ತಮ್ಮಿಂದ
ತಮ್ಮೊಬ್ಬ
ತಯಾ-ರಾ-ಯಿತು
ತಯಾ-ರಿ-ಯನ್ನು
ತಯಾ-ರಿ-ಯೆಲ್ಲ
ತಯಾ-ರಿ-ಸಲು
ತಯಾ-ರಿಸಿ
ತಯಾ-ರಿ-ಸಿದ
ತಯಾ-ರಿ-ಸು-ವ-ವನು
ತರ-ಕಾರಿ
ತರ-ಕಾ-ರಿ-ಮೀನು
ತರ-ಕಾ-ರಿ-ಯನ್ನು
ತರ-ಕಾ-ರಿ-ಯೊಂ-ದಿಗೆ
ತರ-ಗತಿ
ತರ-ಗ-ತಿಗೇ
ತರ-ಗ-ತಿಯ
ತರ-ಗ-ತಿ-ಯನ್ನು
ತರ-ಗ-ತಿ-ಯಲ್ಲಿ
ತರ-ಗ-ತಿ-ಯ-ಲ್ಲಿ-ದ್ದರು
ತರ-ತ-ರದ
ತರ-ದಂತೆ
ತರ-ಬಲ್ಲ
ತರ-ಬೇ-ಕಾ-ಗಿತ್ತು
ತರ-ಬೇ-ಕಾ-ಗಿದೆ
ತರ-ಬೇತಿ
ತರಲೆ
ತರ-ಳರು
ತರ-ವಲ್ಲ
ತರಹ
ತರ-ಹದ
ತರಾ-ಟೆಗೆ
ತರುಣ
ತರು-ಣ-ನಾ-ದರೂ
ತರು-ಣರ
ತರು-ಣ-ರಿ-ಗೆಲ್ಲ
ತರು-ತ್ತಿದ್ದ
ತರುವ
ತರು-ವಾಯ
ತರು-ವು-ದ-ಕ್ಕಾಗಿ
ತರ್ಕ-ಶಾಸ್ತ್ರ
ತರ್ಕ-ಶಾ-ಸ್ತ್ರ-ಜ್ಞರ
ತಲು-ಪಿತು
ತಲು-ಪಿ-ದಂತೆ
ತಲು-ಪಿ-ದರು
ತಲು-ಪಿ-ಸ-ಬೇ-ಕಾ-ಗಿತ್ತು
ತಲೆ
ತಲೆ-ಕೂ-ದ-ಲನ್ನು
ತಲೆ-ಕೆ-ಳ-ಗಾಗಿ
ತಲೆ-ಗಿಲೆ
ತಲೆಗೆ
ತಲೆ-ತ-ಲಾಂ-ತ-ರ-ದಿಂ-ದಲೂ
ತಲೆ-ಮುಟ್ಟಿ
ತಲೆ-ಮೇಲೆ
ತಲೆಯ
ತಲೆ-ಯೆತ್ತಿ
ತಲೆ-ಯೊ-ಳ-ಗೊಂದು
ತಲ್ಲ-ಣಿಸು
ತಲ್ಲ-ಣಿ-ಸು-ತ್ತಿ-ದ್ದಾರೆ
ತಲ್ಲೀ-ನ-ರಾಗಿ
ತಲ್ಲೀ-ನ-ಳಾ-ದಳು
ತಳ-ವಾರ
ತಳೆದ
ತಳ್ಳಿ-ಹಾ-ಕು-ವಂ-ತಿ-ರ-ಲಿಲ್ಲ
ತವಕ
ತಾತ
ತಾತನ
ತಾತ-ನಂತೆ
ತಾತ-ನಾದ
ತಾತ-ನೇ-ನಾ-ದರೂ
ತಾತಯ್ಯ
ತಾತ-ಯ್ಯನ
ತಾತ-ಯ್ಯ-ನಿಗೆ
ತಾತಾ
ತಾತಾಯ್ಯ
ತಾತ್ಕಾ-ಲಿ-ಕ-ವಾ-ಗಿ-ಯಾ-ದರೂ
ತಾನಾಗಿ
ತಾನಾ-ಗಿತ್ತು
ತಾನಾ-ಗಿಯೇ
ತಾನು
ತಾನೂ
ತಾನೆ
ತಾನೇ
ತಾನೇ-ತಾ-ನಾಗಿ
ತಾನೇನೋ
ತಾನೊಬ್ಬ
ತಾಪ-ತ್ರಯ
ತಾಪ-ತ್ರ-ಯವೂ
ತಾಪಸಿ
ತಾಯಂ-ದಿರ
ತಾಯಿ
ತಾಯಿ
ತಾಯಿ-ಮ-ಕ್ಕಳು
ತಾಯಿಗೆ
ತಾಯಿ-ತಂದೆ
ತಾಯಿ-ತಂ-ದೆ-ಯರ
ತಾಯಿ-ಮ-ಕ್ಕಳು
ತಾಯಿಯ
ತಾಯಿ-ಯನ್ನು
ತಾಯಿ-ಯಾದ
ತಾಯಿ-ಯಿಂದ
ತಾಯೀ
ತಾಯ್ತಂ-ದೆ-ಯ-ರಿಗೆ
ತಾಯ್ತಂ-ದೆ-ಯರು
ತಾರ-ಸಿಯ
ತಾರಾ-ಬಲ
ತಾರು-ಣ್ಯಕ್ಕೆ
ತಾರೆ-ಗಳೇ
ತಾಳ
ತಾಳಿ
ತಾಳು
ತಾಳ್ಮೆ-ಯಿರು
ತಾವು
ತಾವು-ತಾವೇ
ತಾವೂ
ತಾವೇ
ತಾವೇನು
ತಿಂಗಳ
ತಿಂಗ-ಳ-ವ-ರೆಗೂ
ತಿಂಗಳು
ತಿಂಗ-ಳು-ಗಳು
ತಿಂಡಿ
ತಿಂಡಿ-ಗಳನ್ನು
ತಿಂದ-ದ್ದಕ್ಕೆ
ತಿಂದರೆ
ತಿಂದೇ
ತಿಂದೇ-ಬಿಟ್ಟ
ತಿದ್ದಿ-ಕೊಂಡು
ತಿದ್ದಿ-ಸಿ-ದರು
ತಿದ್ದಿ-ಸು-ವಾಗ
ತಿದ್ದು-ವು-ದ-ರ-ಲ್ಲೊಂದು
ತಿನ್ನ-ಬೇ-ಕೆಂದು
ತಿನ್ನಿ-ಸ-ಬೇಕು
ತಿನ್ನು-ವು-ದಿಲ್ಲ
ತಿರ-ಸ್ಕ-ರಿ-ಸು-ತ್ತಿತ್ತು
ತಿರ-ಸ್ಕ-ರಿ-ಸು-ತ್ದಿದ್ದ
ತಿರ-ಸ್ಕಾರ
ತಿರ-ಸ್ಕಾ-ರ-ವೇ-ನಿಲ್ಲ
ತಿರಿ-ಚಿ-ಮು-ರುಚಿ
ತಿರು-ಗಾ-ಟಕ್ಕೆ
ತಿರುಗಿ
ತಿರು-ಗಿ-ಕೊಂ-ಡಿತು
ತಿರು-ಗಿ-ಕೊಳ್ಳು
ತಿರು-ಗಿ-ದುವು
ತಿರು-ಗಿ-ಬಿ-ಟ್ಟಿತು
ತಿರು-ವುತ್ತ
ತಿಳಿ-ದಾಗ
ತಿಳಿದು
ತಿಳಿ-ದು-ಕೊಂಡ
ತಿಳಿ-ದು-ಕೊಂ-ಡರು
ತಿಳಿ-ದು-ಕೊಂ-ಡ-ವರು
ತಿಳಿ-ದು-ಕೊಂಡು
ತಿಳಿ-ದು-ಕೊ-ಳ್ಳಲು
ತಿಳಿ-ದು-ಬ-ರು-ತ್ತದೆ
ತಿಳಿ-ದು-ಬಿ-ಡು-ತ್ತಿತ್ತು
ತಿಳಿ-ಯ-ದಂ-ತೆಯೇ
ತಿಳಿ-ಯದು
ತಿಳಿ-ಯದೆ
ತಿಳಿ-ಯ-ಬೇಕು
ತಿಳಿ-ಯ-ಲಿಲ್ಲ
ತಿಳಿ-ಯಲು
ತಿಳಿ-ಯಿತು
ತಿಳಿ-ಯು-ವಂತೆ
ತಿಳಿ-ವ-ಳಿಕೆ
ತಿಳಿವು
ತಿಳಿಸಿ
ತಿಳಿ-ಸಿ-ಕೊಟ್ಟ
ತಿಳಿ-ಸಿ-ಕೊ-ಟ್ಟು-ಬಿ-ಟ್ಟಿ-ದ್ದಾ-ನಲ್ಲ
ತಿಳಿ-ಸಿ-ಕೊ-ಡು-ತ್ತಿ-ದ್ದುವು
ತಿಳಿ-ಸಿದ
ತಿಳಿ-ಸಿ-ದರು
ತಿಳಿ-ಸಿ-ಬಿ-ಟ್ಟನೋ
ತಿಳಿ-ಸು-ತ್ತಾರೆ
ತೀಕ್ಷ್ಣ
ತೀರ
ತೀರಾ
ತೀರಿ
ತೀರಿ-ಕೊಂ-ಡಳು
ತೀರಿ-ಹೋ-ಗಿ-ದ್ದಳು
ತೀರು-ತ್ತೇನೆ
ತೀರ್ಮಾ-ನಕ್ಕೆ
ತೀರ್ಮಾ-ನಿ-ಸಿ-ದರು
ತೀರ್ಮಾ-ನಿ-ಸಿದ್ದ
ತೀರ್ಮಾ-ನಿ-ಸಿ-ದ್ದರೋ
ತೀರ್ಮಾ-ನಿ-ಸಿ-ಬಿ-ಟ್ಟಿ-ದ್ದರು
ತೀವ್ರ
ತೀವ್ರ-ಗಾ-ಮಿ-ಯಾ-ಗ-ಬೇಕು
ತೀವ್ರ-ತೆಯೂ
ತೀವ್ರ-ವಾಗಿ
ತೀವ್ರ-ವಾ-ಗಿ-ಬಿ-ಟ್ಟಿತು
ತೀವ್ರ-ವಾ-ಯಿತು
ತುಂಟ
ತುಂಟ-ತನ
ತುಂಟ-ತ-ನ-ಗ-ಳಿಗೆ
ತುಂಟ-ತ-ನದ
ತುಂಟ-ನಗೆ
ತುಂಟ-ನಾ-ದರೂ
ತುಂಟಾಟ
ತುಂಟಾ-ಟ-ಚ-ಟು-ವ-ಟಿ-ಕೆ-ಗಳು
ತುಂಡಾಗಿ
ತುಂಡು
ತುಂಬ
ತುಂಬಿ
ತುಂಬಿ-ಕೊಂ-ಡಿತ್ತು
ತುಂಬಿ-ಕೊಂ-ಡಿದೆ
ತುಂಬಿ-ಕೊಂ-ಡಿವೆ
ತುಂಬಿ-ಕೊಂಡು
ತುಂಬಿದ
ತುಂಬಿ-ಬಿ-ಟ್ಟಿ-ದ್ದಾನೆ
ತುಂಬಿ-ಬಿ-ಟ್ಟಿ-ದ್ದುವು
ತುಂಬಿ-ರು-ವುದು
ತುಂಬು-ಹೃ-ದ-ಯದ
ತುಂಬೆಲ್ಲ
ತುಟಿ-ಪಿ-ಟ-ಕ್ಕೆ-ನ್ನದೆ
ತುಲನೆ
ತೂಗಾ-ಡು-ತ್ತಿದೆ
ತೃಪ್ತಿ
ತೃಪ್ತಿ-ಪ-ಟ್ಟು-ಕೊ-ಳ್ಳುವ
ತೃಪ್ತಿ-ಯಿಂ-ದಿ-ರ-ಬೇಕು
ತೃಷೆಗೂ
ತೆಗೆ-ದರೂ
ತೆಗೆದು
ತೆಗೆ-ದು-ಕೊಂ-ಡ-ನೆಂ-ದರೆ
ತೆಗೆ-ದು-ಕೊಂಡು
ತೆಗೆ-ದು-ಕೊ-ಳ್ಳದೆ
ತೆಗೆ-ದು-ಕೊ-ಳ್ಳ-ಬೇ-ಕಾ-ಗಿದ್ದ
ತೆಗೆ-ದು-ಕೊ-ಳ್ಳುವ
ತೆಗೆ-ದು-ಬಿ-ಟ್ಟರೆ
ತೆಗೆ-ಯುವ
ತೆಗೆ-ಸಿ-ಕೊ-ಟ್ಟಿದ್ದ
ತೆರೆ-ದಿ-ಟ್ಟು-ಕೊಂಡ
ತೆರೆದು
ತೆರೆ-ದು-ಕೊಂಡ
ತೆರೆ-ದು-ಕೊಂ-ಡಿತು
ತೆರೆ-ದು-ಕೊಂಡು
ತೆರೆದೇ
ತೆರೆ-ಯನ್ನು
ತೆರೆ-ಯಲು
ತೆರೆ-ಯುವ
ತೆರೆ-ಯು-ವ-ಷ್ಟ-ರಲ್ಲಿ
ತೇಜಃ-ಪುಂ-ಜ-ವಾಗಿ
ತೇಜಃ-ಪುಂ-ಜ-ವಾದ
ತೇಜಸ್ವಿ
ತೇಜ-ಸ್ವಿ-ಯಾಗಿ
ತೇಜ-ಸ್ವಿ-ಯಾದ
ತೇಜಸ್ಸು
ತೇರ್ಗ-ಡೆ-ಯಾ-ಗಲು
ತೇರ್ಗ-ಡೆ-ಯಾ-ದ-ವನು
ತೇಲುತ್ತ
ತೇವ-ವಾ-ಗು-ತ್ತಿ-ದ್ದುವು
ತೈನಾ-ತಿ-ಗಳು
ತೊಂದರೆ
ತೊಂದ-ರೆ-ಯಾ-ಗಿ-ರ-ಬೇಕು
ತೊಟ್ಟು-ಕೊಂಡು
ತೊಡ-ಗ-ದಂತೆ
ತೊಡಗಿ
ತೊಡ-ಗಿದ
ತೊಡ-ಗಿ-ದರು
ತೊಡ-ಗಿ-ರು-ವಾಗ
ತೊಡೆ-ದು-ಹಾ-ಕುವ
ತೊದ-ಲಿದ
ತೊರೆ-ದ-ವರು
ತೊರೆದು
ತೊಲೆ
ತೊಲೆ-ಯನ್ನು
ತೊಲೆ-ಯೊಂ-ದನ್ನು
ತೊಳ-ಲು-ತ್ತಿ-ರು-ವ-ವ-ನೊಬ್ಬ
ತೊಳ-ಸಿ-ಬಂದು
ತೊವ್ವೆ-ಅನ್ನ
ತೋಚದೆ
ತೋಚ-ಲಿಲ್ಲ
ತೋಟ-ದಲ್ಲಿ
ತೋಟ-ವಿತ್ತು
ತೋಡಿ-ಕೊಂಡ
ತೋರ-ಗೊ-ಡದೆ
ತೋರ-ಲಾ-ರಂ-ಭಿ-ಸಿತು
ತೋರಾ
ತೋರಿತು
ತೋರಿದ
ತೋರಿ-ಸ-ಬೇಡ
ತೋರಿಸಿ
ತೋರಿ-ಸಿ-ಕೊ-ಟ್ಟ-ನಂತೆ
ತೋರಿ-ಸಿ-ಕೊ-ಡ-ಬ-ಲ್ಲ-ವರು
ತೋರಿ-ಸಿ-ಕೊ-ಡ-ಬ-ಲ್ಲ-ವ-ರೊ-ಬ್ಬರು
ತೋರಿ-ಸಿ-ಕೊ-ಡ-ಬ-ಹು-ದ-ಲ್ಲವೆ
ತೋರಿ-ಸಿ-ಕೊ-ಡು-ವ-ವ-ರಿ-ದ್ದಿ-ದ್ದರೆ
ತೋರಿ-ಸಿದ
ತೋರಿ-ಸಿ-ದ್ದುಂ-ಟು-ಆ-ದರೆ
ತೋರಿ-ಸುತ್ತ
ತೋರಿ-ಸು-ತ್ತಲೂ
ತೋರಿ-ಸು-ತ್ತಿದ್ದ
ತೋರಿ-ಸು-ತ್ತಿ-ರು-ವಾಗ
ತೋರಿ-ಸು-ವಂ-ತಹ
ತೋರಿ-ಸು-ವಂತೆ
ತೋರು-ತ್ತಿದ್ದ
ತೋಳ-ಲ್ಲೆ-ತ್ತಿ-ಕೊಂ-ಡಾಗ
ತ್ಕಾರದ
ತ್ತದೆ
ತ್ತಾನೆ
ತ್ತಿತ್ತು
ತ್ತಿತ್ತು-ಎಲ್ಲ
ತ್ತಿತ್ತೆಂ-ದರೆ
ತ್ತಿದ್ದ
ತ್ತಿದ್ದರು
ತ್ತಿದ್ದಾಗ
ತ್ತಿದ್ದಾನೆ
ತ್ತಿದ್ದು-ದ-ರಿಂದ
ತ್ತಿದ್ದುದು
ತ್ತೀಯೋ
ತ್ಯಜಿ-ಸಿ-ಬಿ-ಡುವ
ತ್ಯಾಗಕ್ಕೂ
ತ್ಯಾಗಕ್ಕೆ
ತ್ಯಾಗ-ಜೀ-ವ-ನದ
ತ್ಯಾಗ-ಜೀ-ವ-ನ-ವನ್ನೇ
ತ್ಯಾಗದ
ತ್ಯಾಗ-ಬುದ್ಧಿ
ತ್ಯಾಗ-ಬು-ದ್ಧಿ-ಯನ್ನು
ತ್ರಿಭು-ವನ
ತ್ವರಿ-ತದ
ಥಳಿ-ಸ-ತೊ-ಡ-ಗಿ-ದರು
ದಂಗು
ದಂತೆಲ್ಲ
ದಕ್ಕೇ
ದಕ್ಷ-ತೆ-ಯಿಂದ
ದಕ್ಷಿ-ಣೇ-ಶ್ವ-ರಕ್ಕೆ
ದಕ್ಷಿ-ಣೇ-ಶ್ವ-ರ-ಕ್ಕೊಮ್ಮೆ
ದಕ್ಷಿ-ಣೇ-ಶ್ವ-ರದ
ದಟ್ಟ-ವಾದ
ದಡ
ದಡಕ್ಕೆ
ದಡದ
ದಡ-ದಿಂದ
ದಡ-ಬಡ
ದಡ-ಸೇ-ರಿತು
ದಣಿ-ಯಿರೈ
ದಣಿ-ವ-ರಿ-ಯದ
ದಣಿ-ವಾ-ದಾಗ
ದಣಿ-ವೆ-ನ್ನದೆ
ದತ್ತ
ದತ್ತನ
ದತ್ತ-ನಿಗೆ
ದತ್ತನೂ
ದನಿ-ಯಲ್ಲಿ
ದನ್ನು
ದಬ್ಬಾ-ಳಿಕೆ
ದಯ-ಪಾ-ಲಿಸು
ದಯ-ವಿಟ್ಟು
ದರ-ಗ-ಳ-ನ್ನಿ-ಟ್ಟಿದ್ದ
ದರು
ದರೂ
ದರೆ
ದರ್ಜೆ-ಗ-ಳಿ-ಸಿದ
ದರ್ಜೆಯ
ದರ್ಬಾ-ರಿ-ನಲ್ಲಿ
ದರ್ಬಾರು
ದರ್ವಾ-ನ-ಹು-ಡು-ಗ-ರಿ-ಗೆಲ್ಲ
ದರ್ಶನ
ದರ್ಶ-ನ-ಲಾ-ಭ-ವಾ-ಗ-ದಿದ್ದ
ದರ್ಶ-ನ-ವಾ-ಗಿತ್ತು
ದರ್ಶ-ನಾ-ನು-ಭ-ವ-ವಾ-ಗು-ತ್ತಿತ್ತು
ದಲ್ಲಿ
ದಲ್ಲಿ-ರುವ
ದಲ್ಲೂ
ದಲ್ಲೇ
ದಷ್ಟ-ಪು-ಷ್ಟ-ವಾ-ಗಿತ್ತು
ದಸ್ತ-ಗಿರಿ
ದಾಗ
ದಾಗಿತ್ತು
ದಾಟಿ
ದಾಟಿದೆ
ದಾತಾ
ದಾದಿ-ಯ-ರನ್ನು
ದಾನ-ಧ-ರ್ಮಾದಿ
ದಾರದ
ದಾರಿ
ದಾರಿ-ಗಾ-ಣದೆ
ದಾರಿಯ
ದಾರಿ-ಯನ್ನು
ದಾರಿ-ಯಲ್ಲಿ
ದಾಸ-ಯ್ಯ-ಗಳನ್ನು
ದಿಂದ
ದಿಂದಲೂ
ದಿಂದಲೇ
ದಿಂದಿದ್ದೆ
ದಿಂದಿ-ರು-ತ್ತಿದ್ದ
ದಿಂಬು-ಗಳನ್ನು
ದಿಗಿ-ಲು-ಬಿ-ದ್ದರು
ದಿಗ್ಭ್ರಾಂ-ತ-ನಾಗಿ
ದಿಟ್ಟ-ತ-ನ-ವನ್ನು
ದಿಟ್ಟಿ-ಸಿದ
ದಿಟ್ಟಿ-ಸುತ್ತ
ದಿನ
ದಿನಂ-ಪ್ರ-ತಿಯ
ದಿನ-ಗಳ
ದಿನ-ಗಳನ್ನು
ದಿನ-ಗಳಲ್ಲಿ
ದಿನ-ಗ-ಳ-ವ-ರೆಗೆ
ದಿನ-ಗ-ಳಾ-ಗ-ಬೇ-ಕಿತ್ತು
ದಿನ-ಗ-ಳಾ-ದುವು
ದಿನ-ಗಳಿಂದ
ದಿನ-ಗ-ಳಿವೆ
ದಿನ-ಗಳು
ದಿನ-ಗಳೇ
ದಿನದ
ದಿನ-ದಂದು
ದಿನ-ದೊ-ಳ-ಗೆಲ್ಲ
ದಿನ-ವಾ-ದರೂ
ದಿನ-ವಿಡೀ
ದಿನಾಲೂ
ದಿನೇ-ದಿನೇ
ದಿವಾ-ನ್-ಇ-ಹ-ಫೀಜ್
ದಿವ್ಯ
ದಿವ್ಯ-ಗುರು
ದಿವ್ಯ-ಜ್ಯೋ-ತಿ-ಯನ್ನು
ದಿವ್ಯ-ಸಂ-ದೇ-ಶ-ವನ್ನು
ದಿವ್ಯಾ-ನು-ಭ-ವ-ಗ-ಳಾ-ದಾವು
ದಿವ್ಯಾ-ನು-ಭ-ವದ
ದಿವ್ಯಾ-ನು-ಭ-ವ-ದಲ್ಲಿ
ದಿಸೆ-ಯಲ್ಲಿ
ದೀನ-ದ-ಲಿ-ತರ
ದೀನ-ಬಂಧು
ದೀನ-ರೊಂ-ದಿಗೆ
ದೀಪ
ದೀಪ-ಗಳನ್ನು
ದೀಪದ
ದೀಪಾ-ವ-ಳಿ-ಗ-ಳಂ-ತಹ
ದೀರ್ಘ
ದೀರ್ಘ-ಕಾಲ
ದುಂಡು-ಗಿನ
ದುಃಖ-ದಿಂದ
ದುಃಖ-ವಾ-ಗಿದೆ
ದುಃಖ-ವಾ-ಯಿತು
ದುಃಖ-ವಾ-ಯಿ-ತೆಂ-ಬು-ದನ್ನು
ದುಡಿದು
ದುಡಿ-ಯ-ಬಲ್ಲ
ದುನ್ನಿ-ಖಾನ್
ದುರ-ದೃ-ಷ್ಟಕ್ಕೆ
ದುರು-ಗುಟ್ಟಿ
ದುರು-ಗು-ಟ್ಟಿ-ಕೊಂಡು
ದುರ್ಗಾ
ದುರ್ಗಾ-ಪೂಜೆ
ದುರ್ಗಾ-ಪ್ರ-ಸಾದ
ದುರ್ಗಾ-ಪ್ರ-ಸಾ-ದನ
ದುರ್ಗಾ-ಪ್ರ-ಸಾ-ದ-ನನ್ನೇ
ದುರ್ಗುಣ
ದುರ್ಗೆ
ದುರ್ಘ-ಟ-ನೆ-ಯನ್ನು
ದುರ್ಬಲ
ದುರ್ಬ-ಲ-ತೆ-ಯನ್ನು
ದುರ್ಬಾ-ರನ್ನು
ದುಷ್ಟ-ಸಂ-ಪ್ರ-ದಾಯ
ದುಷ್ಪ-ರಿ-ಣಾ-ಮ-ಗಳನ್ನೂ
ದೂಗುವ
ದೂರ
ದೂರಕ್ಕೆ
ದೂರದ
ದೂರ-ದಲ್ಲೇ
ದೂರ-ದಿಂ-ದಲೇ
ದೂರ-ದೂ-ರದ
ದೂರು-ಗಳು
ದೃಢ-ಚಿ-ತ್ತ-ತೆ-ಯನ್ನೂ
ದೃಢ-ನಂ-ಬಿ-ಕೆ-ಯಿದೆ
ದೃಢ-ನಿ-ರ್ಧಾ-ರ-ವನ್ನು
ದೃಢ-ವಾದ
ದೃಶ್ಯ
ದೃಶ್ಯ-ಗಳ
ದೃಶ್ಯ-ಗಳು
ದೃಶ್ಯ-ದಲ್ಲಿ
ದೃಶ್ಯ-ವನ್ನು
ದೃಷ್ಟಾಂ-ತ-ವಾಗಿ
ದೃಷ್ಟಿ
ದೃಷ್ಟಿ-ಕೋನ
ದೃಷ್ಟಿಗೆ
ದೆಂಬ
ದೆವ್ವದ
ದೆಸೆ-ಯಿಂದ
ದೆಹಲಿ
ದೇವ
ದೇವ-ದೇ-ವ-ತೆ-ಗಳ
ದೇವ-ತೆ-ಗ-ಳಿಂ-ದಲೇ
ದೇವ-ತೆ-ಗ-ಳೆಂದು
ದೇವ-ತೆಯ
ದೇವ-ದೇ-ವಿ-ಯರ
ದೇವರ
ದೇವ-ರ-ನಾ-ಮ-ಗಳ
ದೇವ-ರ-ನಾ-ಮ-ಗಳನ್ನು
ದೇವ-ರನ್ನು
ದೇವ-ರ-ಪೂಜೆ
ದೇವ-ರಾದ
ದೇವ-ರಿ-ದ್ದಾನೆ
ದೇವರು
ದೇವ-ರೆಂ-ದರೆ
ದೇವರೇ
ದೇವ-ಲೋ-ಕ-ದಲ್ಲಿ
ದೇವ-ಸ್ಥಾ-ನಕ್ಕೆ
ದೇವ-ಸ್ಥಾ-ನದ
ದೇವಾ-ಲ-ಯ-ಗ-ಳಿಗೂ
ದೇವಾ-ಲ-ಯದ
ದೇವಿ
ದೇವಿಯ
ದೇವಿಯೂ
ದೇವೇಂ-ದ್ರ-ನಾಥ
ದೇಶ-ವಿ-ದೇ-ಶ-ಗಳ
ದೇಹಾ-ರೋಗ್ಯ
ದೈನ್ಯದ
ದೈವ-ದತ್ತ
ದೈವ-ಭಕ್ತೆ
ದೈವಿಕ
ದೈಹಿಕ
ದೊಂದು
ದೊಡ್ಡ
ದೊಡ್ಡದು
ದೊಡ್ಡ-ವನಾ
ದೊಡ್ಡ-ವ-ನಾಗಿ
ದೊಡ್ಡ-ವ-ನಾದ
ದೊಡ್ಡ-ವ-ನಾ-ದಂ-ತೆಲ್ಲ
ದೊಡ್ಡ-ವ-ರು-ಚಿ-ಕ್ಕ-ವರು
ದೊಡ್ಡ-ವ-ಳಾ-ದಂತೆ
ದೊಡ್ಡ-ಸ್ತಿ-ಕೆ-ಯ-ವ-ರನ್ನು
ದೊಪ್ಪನೆ
ದೊರ-ಕಿತ್ತು
ದೊರ-ಕು-ವಂ-ತಾ-ಗ-ಬೇಕು
ದೊರೆ-ತಂ-ತಾ-ಯಿತು
ದೊರೆ-ಯ-ಲಿಲ್ಲ
ದೊರೆ-ಯು-ತ್ತಿತ್ತು
ದೋಣಿ
ದೋಣಿ-ಮ-ನೆಯ
ದೋಣಿಯ
ದೋಣಿ-ಯನ್ನು
ದೋಣಿ-ಯಲ್ಲಿ
ದೋಣಿ-ಯಲ್ಲೇ
ದೋಣಿ-ಯ-ವ-ರಿಗೆ
ದೋಣಿ-ಯ-ವರು
ದೋಣಿ-ಯಿಂದ
ದೋಷ-ವೆಂದರೆ
ದ್
ದ್ದರು
ದ್ದಲ್ಲದೆ
ದ್ದಾನಲ್ಲ
ದ್ದಾನೆ
ದ್ದುವು
ದ್ದೇವೆ
ದ್ವಾರ-ಪಾ-ಲಕ
ದ್ವಾರ-ಪಾ-ಲ-ಕನ
ದ್ವಾರ-ಪಾ-ಲ-ಕ-ನೊಬ್ಬ
ದ್ವಿತೀಯ
ಧಕ್ಕೆ
ಧಕ್ಕೆ-ಯಾ-ಯಿತು
ಧನ-ವಂ-ತ-ರ-ನ್ನಾ-ಗಲ್ಲ
ಧನ-ಸ-ಹಾ-ಯದ
ಧನ-ಸ-ಹಾ-ಯ-ವನ್ನೇ
ಧನ-ಸ-ಹಾ-ಯ-ವನ್ನೋ
ಧನ್ಯ-ತೆಯ
ಧನ್ಯ-ವಾ-ದ-ಗ-ಳ-ನ್ನ-ರ್ಪಿಸಿ
ಧರಿ-ಸ-ಬೇ-ಕಾದ್ದು
ಧರಿ-ಸಿದ್ದ
ಧರಿ-ಸು-ವುದು
ಧರೆ-ಗಿ-ಳಿದ
ಧರೆಗೆ
ಧರ್ಮ
ಧರ್ಮ-ಗ-ಳಿಂ-ದಲೂ
ಧರ್ಮ-ಗು-ರು-ಗಳ
ಧರ್ಮ-ಗು-ರು-ಗ-ಳಂತೆ
ಧರ್ಮ-ಗು-ರು-ವಿನ
ಧರ್ಮದ
ಧರ್ಮ-ರ-ಕ್ಷ-ಣ-ಕೋ-ವಿದ
ಧರ್ಮ-ವಾದ
ಧರ್ಮವೂ
ಧಾಂಡಿಗ
ಧಾಂಡಿ-ಗನೇ
ಧಾಮಕ್ಕೆ
ಧಾರ-ಳ-ವಾ-ಗಿಯೇ
ಧಾರಾ-ಳದ
ಧಾರಾ-ಳ-ವಾಗಿ
ಧಾರಾಳಿ
ಧಾರ್ಮಿಕ
ಧಾವಿ-ಸಿ-ಬಂ-ದರು
ಧೀರ
ಧೀರ-ಗಂ-ಭೀ-ರ-ವಾಗಿ
ಧೀರ-ತೆಯ
ಧೀರ-ನ-ಲ್ಲವೆ
ಧೀರ-ನಾ-ಗಿರು
ಧೀರ-ನಾ-ಗು-ವುದು
ಧೀರ-ಪ್ರ-ಯತ್ನ
ಧುಮು-ಕಿ-ದ್ದ-ರಿಂದ
ಧುರೀ-ಣರು
ಧುರೀ-ಣರೂ
ಧೇಂಡಿ
ಧೈರ್ಯ
ಧೈರ್ಯ-ಔ-ದಾ-ರ್ಯ-ಗಳು
ಧೈರ್ಯ-ದೊಂ-ದಿಗೆ
ಧೈರ್ಯ-ಮಾಡಿ
ಧೈರ್ಯ-ಶಾ-ಲಿನಿ
ಧೋತಿ
ಧೋತಿ-ಯ-ನ್ನಲ್ಲ
ಧೋತಿ-ಯನ್ನು
ಧೋತಿ-ಯನ್ನೇ
ಧೋರ-ಣೆ-ಪ್ರ-ಯ-ತ್ನ-ಗಳು
ಧೋರ-ಣೆ-ಗಳಿಂದ
ಧ್ಯಾನ
ಧ್ಯಾನಕ್ಕೆ
ಧ್ಯಾನದ
ಧ್ಯಾನ-ದಲ್ಲಿ
ಧ್ಯಾನ-ದಲ್ಲೇ
ಧ್ಯಾನ-ದಿಂದ
ಧ್ಯಾನ-ದಿಂ-ದೆದ್ದು
ಧ್ಯಾನ-ನಿ-ರ-ತ-ರಾ-ಗಿ-ದ್ದರು
ಧ್ಯಾನ-ಮಗ್ನ
ಧ್ಯಾನ-ಮ-ಗ್ನ-ನಾದ
ಧ್ಯಾನ-ಮಾಡು
ಧ್ಯಾನ-ಲೀ-ನ-ನಾಗಿ
ಧ್ಯಾನ-ಲೀ-ನ-ವಾ-ಗಿ-ಬಿ-ಟ್ಟಿತು
ಧ್ಯಾನ-ವನ್ನು
ಧ್ಯಾನಾ-ನಂ-ದದ
ಧ್ಯಾನಿ-ಸಲಿ
ಧ್ಯಾನಿ-ಸಿದ
ಧ್ಯೇಯ
ಧ್ವನಿ
ಧ್ವನಿ-ಗಳು
ಧ್ವನಿ-ಯಲ್ಲಿ
ಧ್ವನಿ-ಸ-ಬೇ-ಕಾ-ಗಿ-ರು-ವುದು
ನಂಟ-ನಾದ
ನಂದ-ಭ-ರಿ-ತ-ರಾ-ದರು
ನಂದ-ರಾಗಿ
ನಂಬ-ಬೇಡಿ
ನಂಬಿಕೆ
ನಂಬಿ-ಕೆ-ಗಳು
ನಂಬಿ-ಕೆ-ಯಲ್ಲ
ನಂಬಿ-ಕೆ-ಯಾ-ಯಿತು
ನಂಬಿ-ಕೆ-ಯಿಲ್ಲ
ನಂಬಿ-ಕೆ-ಯಿ-ಲ್ಲ-ವಲ್ಲ
ನಂಬಿ-ದ್ದ-ರಿಂ-ದಲೇ
ನಂಬಿ-ಬಿ-ಡೋ-ಣವೇ
ನಂಬು-ತ್ತಾರೆ
ನಕ್ಕ
ನಕ್ಕ-ದ್ದಲ್ಲ
ನಕ್ಕದ್ದು
ನಕ್ಕು
ನಕ್ಕು-ಬಿಟ್ಟ
ನಗರ
ನಗ-ರ-ಗಳು
ನಗ-ರದ
ನಗಿಸಿ
ನಗಿ-ಸು-ತ್ತಾನೆ
ನಗಿ-ಸು-ತ್ತಿದ್ದ
ನಗಿ-ಸು-ತ್ತಿ-ರುವ
ನಗು
ನಗುತ್ತ
ನಗು-ತ್ತಾನೆ
ನಗು-ನ-ಗುತ್ತ
ನಗು-ವಿ-ನಲ್ಲಿ
ನಗು-ವುದನ್ನು
ನಗು-ವು-ದಿಲ್ಲ
ನಗುವೋ
ನಗೆ-ಗ-ಡ-ಲಿ-ನಲ್ಲಿ
ನಗೆ-ಯನ್ನು
ನಟ-ರಾ-ಜ-ನಾ-ಟ್ಯ-ವಾ-ಡು-ವ-ವರೆ-ಲ್ಲರ
ನಡ-ವ-ಳಿಕೆ
ನಡ-ವ-ಳಿ-ಕೆಯ
ನಡ-ಸಿ-ಕೊಂಡು
ನಡಿ
ನಡಿಗೆ
ನಡಿ-ಗೆಯ
ನಡು-ಗಿ-ಸುವ
ನಡು-ಗುವ
ನಡು-ರ-ಸ್ತೆ-ಯಲ್ಲಿ
ನಡುವೆ
ನಡೆ
ನಡೆದ
ನಡೆ-ದರು
ನಡೆ-ದು-ಕೊಂ-ಡರೆ
ನಡೆ-ದು-ಕೊಂ-ಡಾ-ಗ-ಲೆಲ್ಲ
ನಡೆ-ದು-ಕೊಂಡು
ನಡೆ-ದು-ಕೊಳ್ಳ
ನಡೆ-ದು-ಕೊ-ಳ್ಳಲು
ನಡೆ-ದು-ದ-ನ್ನೆಲ್ಲ
ನಡೆ-ದುವು
ನಡೆ-ದು-ಹೋ-ಗು-ತ್ತಿತ್ತು
ನಡೆ-ದೇ-ಬಿ-ಟ್ಟರು
ನಡೆ-ಯ-ಬಾ-ರದು
ನಡೆ-ಯಲೇ
ನಡೆ-ಯಿತು
ನಡೆ-ಯಿರಿ
ನಡೆ-ಯು-ತ್ತ-ದೆಯೆ
ನಡೆ-ಯು-ತ್ತಲೇ
ನಡೆ-ಯು-ತ್ತಿತ್ತು
ನಡೆ-ಯು-ತ್ತಿದೆ
ನಡೆ-ಯು-ತ್ತಿ-ರು-ವಾಗ
ನಡೆ-ಸ-ಬೇ-ಕಾ-ಗು-ತ್ತದೆ
ನಡೆ-ಸಲು
ನಡೆಸಿ
ನಡೆ-ಸಿ-ಕೊಂಡು
ನಡೆ-ಸಿ-ಕೊ-ಳ್ಳ-ಬೇ-ಕೆಂ-ಬುದು
ನಡೆ-ಸಿದ
ನಡೆ-ಸು-ತ್ತಿತ್ತು
ನಡೆ-ಸು-ತ್ತಿದ್ದ
ನಡೆ-ಸುವ
ನಡೆ-ಸು-ವಂ-ತಾ-ದರು
ನಡೆ-ಸು-ವಾಗ
ನಡೆ-ಸು-ವು-ದ-ರಲ್ಲಿ
ನದಿಯ
ನದಿ-ಯೊ-ಳಗೆ
ನನ-ಗ-ನಿ-ಸು-ತ್ತಿದೆ
ನನ-ಗಾಗಿ
ನನ-ಗೀಗ
ನನಗೆ
ನನ-ಗೊಂದು
ನನ್ನ
ನನ್ನನ್ನು
ನನ್ನನ್ನೂ
ನನ್ನ-ನ್ನೇ-ನಾ-ದರೂ
ನನ್ನಲ್ಲಿ
ನನ್ನ-ಷ್ಟಕ್ಕೆ
ನನ್ನಿಂದ
ನನ್ನು
ನನ್ನೊ-ಳಗೆ
ನಮ-ಗಾ-ದೀತು
ನಮಗೆ
ನಮ-ಸ್ಕ-ರಿ-ಸಿ-ದಳು
ನಮ-ಸ್ಕ-ರಿ-ಸು-ತ್ತಿದ್ದ
ನಮ-ಸ್ಕಾರ
ನಮೂ-ನೆಯ
ನಮ್ಮ
ನಮ್ಮೊ-ಳಗೇ
ನಯ-ನ-ಗಳು
ನಯ-ನ-ಮ-ನೋ-ಹರ
ನಯ-ವಾದ
ನರೇಂದ್ರ
ನರೇಂ-ದ್ರ-ಚ-ಕ್ರ-ವರ್ತಿ
ನರೇಂ-ದ್ರನ
ನರೇಂ-ದ್ರ-ನಗೆ
ನರೇಂ-ದ್ರ-ನದೇ
ನರೇಂ-ದ್ರ-ನನ್ನು
ನರೇಂ-ದ್ರ-ನಲ್ಲಿ
ನರೇಂ-ದ್ರ-ನ-ಲ್ಲೊಂದು
ನರೇಂ-ದ್ರ-ನಾಥ
ನರೇಂ-ದ್ರ-ನಿಂದ
ನರೇಂ-ದ್ರ-ನಿ-ಗಾಗಿ
ನರೇಂ-ದ್ರ-ನಿ-ಗಾದ
ನರೇಂ-ದ್ರ-ನಿಗೂ
ನರೇಂ-ದ್ರ-ನಿಗೆ
ನರೇಂ-ದ್ರ-ನಿ-ಗೆ-ಆ-ಗ-ಲಪ್ಪ
ನರೇಂ-ದ್ರ-ನಿ-ಗೇನೋ
ನರೇಂ-ದ್ರ-ನಿಗೋ
ನರೇಂ-ದ್ರ-ನೀಗ
ನರೇಂ-ದ್ರನೂ
ನರೇಂ-ದ್ರ-ನೆಂಬ
ನರೇಂ-ದ್ರನೇ
ನರೇಂ-ದ್ರ-ನೇನೂ
ನರೇಂ-ದ್ರರೂ
ನರೇನ್
ನರ್ತನ
ನರ್ತ-ನ-ಕಲೆ
ನರ್ತಿಸಿ
ನಲ್ಲೇ
ನವ-ಗೋ-ಪಾಲ
ನವ-ಚೇ-ತ-ನ-ವನ್ನು
ನವ-ಜಾತ
ನವಾ-ಬ-ನಂತೆ
ನವಿಲು
ನವೆಂ-ಬ-ರಿ-ನಲ್ಲಿ
ನಸು-ಗ-ತ್ತ-ಲಾ-ಗು-ತ್ತಲೇ
ನಸು-ಗೆಂ-ಪಿನ
ನಸು-ನಕ್ಕ
ನಾಂದಿ-ಯಾ-ಗಿ-ಬಿ-ಟ್ಟಿ-ರು-ತ್ತದೆ
ನಾಗಿ-ಬಿ-ಟ್ಟಿ-ದ್ದಾನೆ
ನಾಗಿ-ಬಿ-ಟ್ಟಿ-ದ್ದಾ-ನೆಯೋ
ನಾಗಿ-ಬಿ-ಡ-ಲಿಲ್ಲ
ನಾಚಿಕೆ
ನಾಟಕ
ನಾಟ-ಕಕ್ಕೆ
ನಾಟ-ಕ-ಗಳನ್ನು
ನಾಟ-ಕದ
ನಾಟ-ಕಾ-ಭ್ಯಾಸ
ನಾಟ-ಕೀ-ಯ-ವಾಗಿ
ನಾಟ್ಯ-ನಿ-ಪುಣ
ನಾಡಿದ್ದು
ನಾದ
ನಾದರೂ
ನಾದ್ದ-ರಿಂದ
ನಾನ-ದನ್ನು
ನಾನಾ
ನಾನಿನ್ನು
ನಾನಿಲ್ಲಿ
ನಾನು
ನಾನೆಲ್ಲೂ
ನಾನೇ
ನಾನೊಬ್ಬ
ನಾಮ-ಕ-ರ-ಣದ
ನಾಮ-ಕ-ರ-ಣ-ವಾ-ಯಿತು
ನಾಮ-ದಿಂದ
ನಾಯಕ
ನಾಯ-ಕತ್ವ
ನಾಯ-ಕ-ತ್ವವೇ
ನಾಯ-ಕ-ನ-ನ್ನಾಗಿ
ನಾಯ-ಕ-ನಲ್ಲಿ
ನಾಯ-ಕ-ನಾ-ಗಿದ್ದ
ನಾಯ-ಕ-ನಾ-ಗು-ವು-ದೆಂ-ದರೆ
ನಾರೀ-ಕುಲ
ನಾಲ್ಕನೇ
ನಾಲ್ಕು
ನಾಲ್ಕೈದು
ನಾಳೆ
ನಾಳೆಯ
ನಾವಾ-ಗಲೇ
ನಾವಿಕ
ನಾವಿ-ಕನ
ನಾವಿ-ಕ-ನನ್ನು
ನಾವಿ-ಕ-ನಿಗೆ
ನಾವಿ-ಕನೂ
ನಾವು
ನಾವೆಲ್ಲ
ನಾವೇ
ನಾವೇಕೆ
ನಿಂತ
ನಿಂತಂತೆ
ನಿಂತರು
ನಿಂತಾ-ರೆಯೆ
ನಿಂತಿತು
ನಿಂತಿತ್ತು
ನಿಂತಿದೆ
ನಿಂತಿದ್ದ
ನಿಂತಿ-ದ್ದರು
ನಿಂತಿ-ದ್ದಾನೆ
ನಿಂತಿ-ದ್ದಾಳೆ
ನಿಂತಿ-ದ್ದುವು
ನಿಂತಿ-ರ-ಲಿಲ್ಲ
ನಿಂತಿ-ರು-ತ್ತಿದ್ದ
ನಿಂತು
ನಿಂತು-ಕೊ-ಳ್ಳು-ವಂತೆ
ನಿಂತು-ಬಿಟ್ಟ
ನಿಂತು-ಬಿ-ಟ್ಟಿ-ದ್ದಾಳೆ
ನಿಂತು-ಬಿ-ಡು-ತ್ತಿದ್ದ
ನಿಂದಕ
ನಿಂದಲೂ
ನಿಂದಿಸಿ
ನಿಕಟ
ನಿಕ-ಟ-ವಾಗಿ
ನಿಖ-ರ-ವಾಗಿ
ನಿಖ-ರ-ವಾದ
ನಿಗೆ
ನಿಜ
ನಿಜಕ್ಕೂ
ನಿಜ-ತ್ವ-ವನ್ನು
ನಿಜ-ವಾ-ಗಿ-ದ್ದರೆ
ನಿಜ-ವಾ-ಗಿಯೂ
ನಿಜ-ವಾ-ಗು-ವುದನ್ನು
ನಿಜ-ವಾದ
ನಿಜ-ವಾ-ದಲ್ಲಿ
ನಿಜವೇ
ನಿಜ-ಸ್ವ-ರೂ-ಪ-ವನ್ನು
ನಿಟ್ಟಿ-ಸಿ-ದಳು
ನಿಟ್ಟಿ-ಸುತ್ತ
ನಿಟ್ಟು-ಸಿ-ರೊಂದು
ನಿತ್ಯದ
ನಿತ್ಯಾ-ನಂ-ದ-ಮೂರ್ತಿ
ನಿತ್ರಾ-ಣ-ನಾಗಿ
ನಿದ-ರ್ಶನ
ನಿದ್ರೆ
ನಿದ್ರೆಯ
ನಿದ್ರೆ-ಯಲ್ಲಿ
ನಿದ್ರೆ-ಹೋ-ಗು-ತ್ತಿದ್ದ
ನಿಧಾ-ನ-ವಾಗಿ
ನಿಧಾ-ನ-ವಾ-ಗುತ್ತ
ನಿನಗೆ
ನಿನಗೇ
ನಿನ-ಗ್ಯಾರೋ
ನಿನ್ನ
ನಿನ್ನನ್ನು
ನಿನ್ನಲ್ಲಿ
ನಿನ್ನಿಷ್ಟ
ನಿನ್ನಿ-ಷ್ಟ-ದಂ-ತೆಯೇ
ನಿಭಾ-ಯಿ-ಸ-ಬಲ್ಲ
ನಿಭಾ-ಯಿ-ಸು-ವ-ವ-ರಿಗೆ
ನಿಮಗೆ
ನಿಮ-ಗೆಲ್ಲ
ನಿಮಿರಿ
ನಿಮಿ-ಷ-ಗಳಲ್ಲಿ
ನಿಮಿ-ಷ-ದಲ್ಲಿ
ನಿಮಿ-ಷಾ-ರ್ಧ-ದಲ್ಲಿ
ನಿಮ್ಮ
ನಿಮ್ಮನ್ನು
ನಿಯ-ಮ-ಗ-ಳ-ಲ್ಲೊಂದು
ನಿಯ-ಮದ
ನಿರಂ-ತರ
ನಿರ-ತ-ನಾದ
ನಿರ-ತ-ರಾ-ಗಿ-ಬಿ-ಟ್ಟರು
ನಿರ-ತ-ರಾ-ಗಿ-ರು-ತ್ತಿ-ದ್ದರು
ನಿರ-ತ-ಳಾ-ಗು-ತ್ತಿ-ದ್ದಳು
ನಿರ-ರ್ಗ-ಳ-ವಾಗಿ
ನಿರಾ-ಕ-ರಿ-ಸಿ-ತ್ತೆಂ-ದಲ್ಲ
ನಿರಾ-ಕಾರ
ನಿರಾ-ಕಾ-ರ-ಇದು
ನಿರಾ-ತಂ-ಕ-ವಾಗಿ
ನಿರಾ-ಶ-ನಾ-ಗು-ತ್ತೀಯೋ
ನಿರಾ-ಶ-ರಾಗಿ
ನಿರಾ-ಶೆಯ
ನಿರಾ-ಶೆ-ಯಾ-ಯಿತು
ನಿರಾ-ಸ-ಕ್ತ-ಸಪ್ಪೆ
ನಿರೀ-ಕ್ಷಿ-ಸಿ-ಕೊಂಡು
ನಿರೀ-ಕ್ಷಿ-ಸಿದ
ನಿರೀ-ಕ್ಷಿ-ಸುತ್ತಾ
ನಿರೀಕ್ಷೆ
ನಿರ್ಗು-ಣ-ನಿ-ರಾ-ಕಾ-ರ-ನೆಂದು
ನಿರ್ದಿಷ್ಟ
ನಿರ್ದಿ-ಷ್ಟ-ವಾದ
ನಿರ್ದೇ-ಶ-ನದ
ನಿರ್ಧ-ರಿ-ಸಿ-ಬಿಟ್ಚ
ನಿರ್ಧಾರ
ನಿರ್ಧಾ-ರಕ್ಕೆ
ನಿರ್ಧಾ-ರ-ವನ್ನು
ನಿರ್ನಾಮ
ನಿರ್ಮಾಣ
ನಿರ್ಮಾ-ಣ-ಗೊ-ಳ್ಳು-ವುದು
ನಿರ್ಮಾ-ಣ-ದಲ್ಲಿ
ನಿರ್ಮಿ-ಸಿದ್ದ
ನಿರ್ಮಿ-ಸಿ-ದ್ದುವು
ನಿರ್ಮೂಲ
ನಿರ್ಮೂ-ಲ-ಗೊ-ಳಿ-ಸಲು
ನಿರ್ಮೂ-ಲನ
ನಿರ್ಮೂ-ಲ-ನ-ಕ್ಕಾಗಿ
ನಿರ್ಲ-ಕ್ಷಿ-ಸಿ-ಬಿಟ್ಟ
ನಿರ್ವ-ಹಿ-ಸಿ-ಕೊಂಡು
ನಿರ್ವ-ಹಿ-ಸಿ-ಕೊ-ಳ್ಳು-ತ್ತಾರೆ
ನಿರ್ವ-ಹಿಸು
ನಿಲ್ಲಿ-ಸ-ಲಾ-ಗು-ತ್ತಿತ್ತು
ನಿಲ್ಲಿ-ಸು-ತ್ತಿದ್ದ
ನಿಲ್ಲು-ತ್ತವೆ
ನಿಲ್ಲು-ತ್ತಿತ್ತು
ನಿಲ್ಲು-ತ್ತಿ-ದ್ದರು
ನಿಲ್ಲುವ
ನಿವೃತ್ತ
ನಿವೃತ್ತಿ
ನಿವೃ-ತ್ತಿ-ಪ-ರ-ವಾದ
ನಿಶ್ಚ-ಯಕ್ಕೆ
ನಿಶ್ಚ-ಯ-ವಾಗಿ
ನಿಶ್ಚಿ-ತ-ವಾದ
ನಿಶ್ಚಿ-ತಾ-ಭಿ-ಪ್ರಾಯ
ನಿಷೇ-ಧಾ-ತ್ಮಕ
ನಿಷ್ಠೆ-ಯನ್ನು
ನಿಸ್ಸಂ-ದೇ-ಹ-ವಾಗಿ
ನಿಸ್ಸೀಮ
ನಿಸ್ಸೀ-ಮ-ನಾದ
ನೀಡಿತು
ನೀಡಿದ
ನೀಡಿ-ದಳು
ನೀಡಿ-ದ್ದಳು
ನೀಡು-ತ್ತಿತ್ತು
ನೀಡುವ
ನೀಡು-ವಂತೆ
ನೀಡು-ವಂಥ
ನೀತಿ-ಕ-ತೆ-ಗಳನ್ನು
ನೀತಿ-ಪರ
ನೀತಿಯ
ನೀನು
ನೀನೂ
ನೀನೇ
ನೀನೇಕೆ
ನೀನೇನು
ನೀನೊಬ್ಬ
ನೀರಸ
ನೀರಿಗೆ
ನೀರು
ನೀರೊ-ಳಗೆ
ನೀವು
ನೀವೆಲ್ಲ
ನೀವೇ
ನುಂಗಿ-ಕೊಂಡು
ನುಗ್ಗಿ
ನುಗ್ಗಿ-ದ-ನ-ಲ್ಲದೆ
ನುಗ್ಗಿ-ಬಿಟ್ಟ
ನುಗ್ಗು-ತ್ತಾನೆ
ನುಗ್ಗು-ತ್ತಿದ್ದ
ನುಚ್ಚು-ನೂರು
ನುಡಿ
ನುಡಿದ
ನುಡಿಯ
ನುಡಿಸ
ನುರಿತ
ನುರಿ-ತ-ವನು
ನುಸು-ಳಿದ
ನೂತನ
ನೂರಾರು
ನೃತ್ಯ
ನೃತ್ಯ-ಕ-ಲೆ-ಯನ್ನೂ
ನೃತ್ಯ-ವನ್ನು
ನೃಸಿಂ-ಹ-ದ-ತ್ತ-ನಿಗೆ
ನೆಗೆ-ಯು-ವುದು
ನೆಟ್ಟ-ನೋ-ಟ-ದಿಂದ
ನೆನಪಾ
ನೆನ-ಪಾ-ಯಿ-ತಂತೆ
ನೆನ-ಪಾ-ಯಿತು
ನೆನ-ಪಿದೆ
ನೆನ-ಪಿನ
ನೆನ-ಪಿ-ನ-ಲ್ಲಿ-ಟ್ಟು-ಕೊಂಡು
ನೆನ-ಪಿ-ಸಿ-ಕ-ಳ್ಳ-ಬೇಕು
ನೆನ-ಪಿ-ಸಿ-ಕೊಂಡು
ನೆನಪು
ನೆನೆ-ಸಿ-ಕೊಂಡು
ನೆಪೋ-ಲಿ-ಯ-ನ್ನನ
ನೆರ-ಳಾಟ
ನೆರ-ಳಿ-ರು-ವು-ದ-ರಿಂದ
ನೆರ-ವಾ-ಗ-ಬೇಕು
ನೆರ-ವಾ-ಗು-ವಂ-ತಹ
ನೆರ-ವಿಗೆ
ನೆರ-ವೇ-ರಿ-ಸು-ವಲ್ಲಿ
ನೆರೆ-ಕೆ-ರೆ-ಯಲ್ಲಿ
ನೆರೆ-ದಿದ್ದ
ನೆರೆ-ಹೊ-ರೆಯ
ನೆರೆ-ಹೊ-ರೆ-ಯ-ವರೂ
ನೆಲದ
ನೆಲ-ದೊ-ಳಗೆ
ನೆಲೆ-ಗೊ-ಳಿ-ಸಿದ
ನೇ
ನೇಮಕ
ನೇರ
ನೇರ-ವಾಗಿ
ನೈತಿಕ
ನೈತಿ-ಕ-ಆ-ಧ್ಯಾ-ತ್ಮಿಕ
ನೈತಿ-ಕ-ತೆಯ
ನೊಂದಿಗೆ
ನೊಂದಿ-ದ್ದಾರೆ
ನೊಬ್ಬನೇ
ನೋಡ
ನೋಡದೆ
ನೋಡ-ನೋ-ಡು-ತ್ತಿ-ದ್ದಂತೆ
ನೋಡಪ್ಪ
ನೋಡ-ಬ-ಹುದು
ನೋಡ-ಬ-ಹುದೆ
ನೋಡ-ಬೇಕು
ನೋಡ-ಲಾ-ರಂ-ಭಿ-ಸಿದ
ನೋಡ-ಲಾ-ರಂ-ಭಿ-ಸಿ-ದರು
ನೋಡ-ಲಿ-ದ್ದೇವೆ
ನೋಡಲು
ನೋಡಿ
ನೋಡಿಕೊ
ನೋಡಿ-ಕೊಂಡ
ನೋಡಿ-ಕೊಂ-ಡರೆ
ನೋಡಿ-ಕೊಂಡು
ನೋಡಿ-ಕೊಂ-ಡು-ಬ-ರಲು
ನೋಡಿ-ಕೊ-ಳ್ಳ-ತೊ-ಡ-ಗಿ-ದರು
ನೋಡಿ-ಕೊ-ಳ್ಳ-ಬೇ-ಕಾ-ದ-ವನು
ನೋಡಿ-ಕೊ-ಳ್ಳಲು
ನೋಡಿ-ಕೊಳ್ಳಿ
ನೋಡಿ-ಕೊ-ಳ್ಳು-ತ್ತಿದ್ದ
ನೋಡಿಕೋ
ನೋಡಿತು
ನೋಡಿದ
ನೋಡಿ-ದಂತೆ
ನೋಡಿ-ದ-ಅ-ತ್ಯಂತ
ನೋಡಿ-ದರೂ
ನೋಡಿ-ದರೆ
ನೋಡಿ-ದ-ರೇಽಽ
ನೋಡಿ-ದಳು
ನೋಡಿ-ದ-ವನೇ
ನೋಡಿ-ದಾಗ
ನೋಡಿದೆ
ನೋಡಿ-ದೆವು
ನೋಡಿ-ದ್ದಳು
ನೋಡಿ-ದ್ದೇ-ನೆ-ಆ-ದರೆ
ನೋಡಿ-ದ್ದೇವೆ
ನೋಡಿ-ಬಿಟ್ಟ
ನೋಡು
ನೋಡುತ್ತ
ನೋಡು-ತ್ತಾನೆ
ನೋಡು-ತ್ತಾ-ನೆ-ಅ-ಪ-ರಿ-ಚಿತ
ನೋಡು-ತ್ತಾರೆ
ನೋಡು-ತ್ತಾ-ರೆ-ಎದೆ
ನೋಡು-ತ್ತಾ-ಳೆಆ
ನೋಡು-ತ್ತಾ-ಳೆ-ಎ-ದು-ರಿಗೆ
ನೋಡು-ತ್ತಾ-ಳೆ-ಮ-ಗನ
ನೋಡು-ತ್ತಿತ್ತು
ನೋಡು-ತ್ತಿದೆ
ನೋಡು-ತ್ತಿದ್ದ
ನೋಡು-ತ್ತಿ-ದ್ದರು
ನೋಡು-ತ್ತಿ-ದ್ದರೆ
ನೋಡು-ತ್ತಿ-ದ್ದಾರೆ
ನೋಡು-ತ್ತಿ-ದ್ದೇನೆ
ನೋಡು-ತ್ತಿ-ರ-ಲಿಲ್ಲ
ನೋಡು-ತ್ತೇನೆ
ನೋಡು-ತ್ತೇವೆ
ನೋಡುವ
ನೋಡು-ವ-ವರ
ನೋಡು-ವುದನ್ನು
ನೋಡು-ವು-ದೆಂ-ದರೇ
ನೋಡೇ-ಬಿ-ಡು-ವುದು
ನೋಡೋಣ
ನೋಡ್ರಪ್ಪ
ನೋವನ್ನು
ನೋವಾ-ಗು-ವಂತೆ
ನೋವಾ-ಯಿತು
ನ್ನೆಲ್ಲ
ನ್ಯಾಯ
ನ್ಯಾಯ-ವಿ-ತ-ರಣೆ
ನ್ಯಾಯಾ-ಲ-ಯದ
ಪಂಗ-ಡದ
ಪಂಜಾ-ಬಿ-ನಿಂದ
ಪಂಡಿತ
ಪಂಡಿ-ತರೂ
ಪಂಥ
ಪಂಥ-ವನ್ನು
ಪಂಥವು
ಪಂಥ-ವೇ-ನಾ-ಗಿ-ರ-ಲಿಲ್ಲ
ಪಂದ್ಯ-ಗ-ಳಲ್ಲೂ
ಪಂದ್ಯ-ಗಳು
ಪಂದ್ಯ-ಗ-ಳೇನೋ
ಪಂದ್ಯಾ-ವ-ಳಿ-ಯೊಂ-ದ-ರಲ್ಲಿ
ಪಕ್ಕಕ್ಕೆ
ಪಕ್ಕದ
ಪಕ್ಕ-ದ-ಲ್ಲಿ-ರುವ
ಪಕ್ಷಿ-ಗಳು
ಪಕ್ಷಿ-ಗಳೂ
ಪಖ-ವಾ-ಜ್-ತ-ಬ-ಲಾ-ಗಳನ್ನೂ
ಪಟ್ಟಾಗಿ
ಪಟ್ಟಿ
ಪಠ್ಯೇ-ತರ
ಪಡಿ-ಸಿ-ಕೊಂಡ
ಪಡಿ-ಸು-ತ್ತಿದ್ದ
ಪಡಿ-ಸುವ
ಪಡು-ತ್ತಿ-ದ್ದರು
ಪಡು-ವಂ-ತಿತ್ತು
ಪಡೆದ
ಪಡೆ-ದರು
ಪಡೆದು
ಪಡೆ-ದು-ಕೊಂಡ
ಪಡೆ-ಯ-ಬಾ-ರದು
ಪಡೆ-ಯ-ಬೇ-ಕಾ-ಗಿತ್ತು
ಪಡೆ-ಯ-ಬೇ-ಕಾ-ದರೂ
ಪಡೆ-ಯಲು
ಪಡೆ-ಯು-ತ್ತಿ-ದ್ದರು
ಪತಿ
ಪತಿ-ಗೃಹ
ಪತ್ತೆ
ಪತ್ನಿ
ಪತ್ನಿಯ
ಪತ್ರ
ಪತ್ರ-ವನ್ನು
ಪದ-ವಿ-ಪ್ರ-ಶಸ್ತಿ
ಪದ-ವಿ-ಯಂ-ತಹ
ಪದ-ವೀ-ಧರ
ಪದ್ಧತಿ
ಪದ್ಧ-ತಿ-ಗಳನ್ನು
ಪದ್ಧ-ತಿಯ
ಪದ್ಯ-ಗಳನ್ನು
ಪರಂ-ಪ-ರೆಯ
ಪರ-ಕೀ-ಯರ
ಪರದೆ
ಪರ-ದೆ-ಗಿ-ರ-ದೆ-ಗಳನ್ನೆಲ್ಲ
ಪರ-ಮ-ತ-ಗಳ
ಪರ-ಮ-ಪು-ಷಿ-ಯೊ-ಬ್ಬನು
ಪರ-ಮ-ಹಂ-ಸರು
ಪರಮಾ
ಪರ-ಮಾತ್ಮ
ಪರ-ಮಾ-ತ್ಮನ
ಪರ-ಮೋ-ದ್ದೇಶ
ಪರ-ವಶ
ಪರ-ವಾಗಿ
ಪರ-ವಾ-ದರು
ಪರ-ಸ್ಪರ
ಪರ-ಹಿತ
ಪರ-ಹಿ-ತ-ದೃ-ಷ್ಟಿ-ಯನ್ನು
ಪರಾ-ಕ್ರಮ
ಪರಾ-ಕ್ರ-ಮ-ದಿಂದ
ಪರಾ-ಕ್ರ-ಮಿ-ಯ-ಲ್ಲಿವೆ
ಪರಾರಿ
ಪರಾ-ರಿ-ಯಾ-ದರು
ಪರಿ
ಪರಿ-ಗ-ಣಿ-ಸಿ-ದ-ವ-ನೇ-ನಲ್ಲ
ಪರಿ-ಚಯ
ಪರಿ-ಚ-ಯ-ವಾದ
ಪರಿ-ಚ-ಯ-ಸ್ಥರ
ಪರಿ-ಚಿ-ತರು
ಪರಿ-ಚಿ-ತ-ರು-ಅ-ಪ-ರಿ-ಚಿ-ತರು
ಪರಿ-ಜ್ಞಾನ
ಪರಿ-ಣ-ತ-ನ-ನ್ನಾಗಿ
ಪರಿ-ಣತಿ
ಪರಿ-ಣ-ಮಿಸಿ
ಪರಿ-ಣಾ-ಮ-ದಿಂ-ದಾಗಿ
ಪರಿ-ಣಾ-ಮ-ವಾಗಿ
ಪರಿ-ತ್ಯಾಗ
ಪರಿ-ತ್ಯಾಗಿ
ಪರಿ-ಪಾಠ
ಪರಿ-ಪಾ-ಠ-ವಿದೆ
ಪರಿ-ಪೂ-ರ್ಣ-ವಾಗಿ
ಪರಿ-ಭಾ-ವಿ-ಸಿ-ಸ-ದ್ದ-ರಲ್ಲೂ
ಪರಿ-ಮಾ-ಣ-ದ-ಲ್ಲಿ-ದ್ದರೆ
ಪರಿ-ವ-ರ್ತ-ನೆ-ಯುಂಟಾ
ಪರಿ-ವ-ರ್ತಿತ
ಪರಿ-ವ-ರ್ತಿಸಿ
ಪರಿವೆ
ಪರಿ-ವ್ರಾ-ಜಕ
ಪರಿ-ಶು-ದ್ಧ-ನಾ-ಗಿ-ರ-ಬೇಕು
ಪರಿ-ಶ್ರ-ಮದ
ಪರಿ-ಸ್ಥಿತಿ
ಪರಿ-ಸ್ಥಿ-ತಿ-ಗಳ
ಪರಿ-ಸ್ಥಿ-ತಿ-ಗಳನ್ನು
ಪರಿ-ಸ್ಥಿ-ತಿ-ಗಳು
ಪರಿ-ಸ್ಥಿ-ತಿ-ಗಾಗಿ
ಪರಿ-ಸ್ಥಿ-ತಿ-ಯನ್ನು
ಪರಿ-ಸ್ಥಿ-ತಿ-ಯಲ್ಲೇ
ಪರಿ-ಸ್ಥಿ-ತಿ-ಯಿಂದ
ಪರಿ-ಹಾ-ರ-ವನ್ನು
ಪರಿ-ಹಾ-ರ-ವಾ-ಗು-ವ-ವ-ರೆಗೂ
ಪರೀಕ್ಷಾ
ಪರೀಕ್ಷೆ
ಪರೀ-ಕ್ಷೆ-ಗಳು
ಪರೀ-ಕ್ಷೆ-ಗ-ಳೆಲ್ಲ
ಪರೀ-ಕ್ಷೆ-ಗಾಗಿ
ಪರೀ-ಕ್ಷೆಗೆ
ಪರೀ-ಕ್ಷೆಯ
ಪರೀ-ಕ್ಷೆ-ಯನ್ನೇ
ಪರೀ-ಕ್ಷೆ-ಯಲ್ಲಿ
ಪರೀ-ಕ್ಷೆ-ಯೆಂ-ದರೆ
ಪರ್ವ-ಕಾಲ
ಪರ್ವತ
ಪರ್ವ-ತ-ಶಿ-ಖ-ರ-ಗಳ
ಪರ್ವ-ತ-ಶಿ-ಖ-ರ-ಗಳು
ಪರ್ವ-ತ-ಶಿ-ಖ-ರ-ಗಳೇ
ಪರ್ವ-ದಿನ
ಪರ್ಷಿ-ಯನ್
ಪಲಾ-ವಿನ
ಪಲ್ಯ
ಪಲ್ಯ-ಗಳನ್ನು
ಪಲ್ಯ-ಗ-ಳಿಗೆ
ಪಲ್ಯ-ದ-ಲ್ಲಿ-ರುವ
ಪಲ್ಯ-ದಿಂದ
ಪಳ-ಗಿ-ದ-ವನೇ
ಪಳ-ಗಿ-ಸುವ
ಪವಿತ್ರ
ಪವಿ-ತ್ರ-ತೆ-ಪ-ರಿ-ಶು-ದ್ಧತೆ
ಪವಿ-ತ್ರ-ತೆಯ
ಪವಿ-ತ್ರ-ತೆ-ಯನ್ನು
ಪವಿ-ತ್ರ-ನಾ-ಗಿ-ರ-ಬೇಕು
ಪಶ್ಚಾ-ತ್ತಾ-ಪ-ಗೊಂಡು
ಪಾಂಡ-ವರ
ಪಾಕ
ಪಾಕ-ಶಾ-ಸ್ತ್ರದ
ಪಾಕ-ಶಾ-ಸ್ತ್ರ-ನಿ-ಪು-ಣ-ರೆಲ್ಲ
ಪಾಠ
ಪಾಠ-ಗಳನ್ನು
ಪಾಠ-ಗಳನ್ನೆಲ್ಲ
ಪಾಠ-ಗ-ಳೆ-ಲ್ಲವೂ
ಪಾಠದ
ಪಾಠ-ದಲ್ಲೂ
ಪಾಠ-ಪ-ಟ್ಟಿ-ಗಷ್ಟೇ
ಪಾಠ-ಪ-ಟ್ಟಿ-ಯ-ಲ್ಲಿ-ರುವ
ಪಾಠ-ವನ್ನು
ಪಾಠ-ವನ್ನೂ
ಪಾಠ-ವಾ-ಗಿತ್ತು
ಪಾಠ-ಶಾ-ಲೆಗೆ
ಪಾಠ-ಶಾ-ಲೆ-ಯಲ್ಲಿ
ಪಾಡನ್ನು
ಪಾಡಿಗೆ
ಪಾತ್ರ-ಧಾ-ರಿಗೆ
ಪಾತ್ರ-ಧಾ-ರಿ-ಯನ್ನು
ಪಾತ್ರ-ನಾ-ಗಿದ್ದ
ಪಾತ್ರ-ರಾ-ಗಿ-ದ್ದ-ವರು
ಪಾತ್ರ-ವೇನು
ಪಾದಕ್ಕೆ
ಪಾದ-ವನ್ನು
ಪಾಪ
ಪಾರವೇ
ಪಾರಸೀ
ಪಾರ-ಸೀ-ಸಂ-ಸ್ಕೃತ
ಪಾರಾ-ಗಲೇ
ಪಾರಿ-ವಾ-ಳ-ಗಳು
ಪಾರು-ಪ-ತ್ಯ-ವನ್ನು
ಪಾರು-ಮಾ-ಡಿದ
ಪಾರು-ಮಾ-ಡಿ-ಬಿ-ಟ್ಟ-ನಲ್ಲ
ಪಾಲಿ-ಗಿ-ದ್ದುವು
ಪಾಲಿಗೆ
ಪಾಲಿಗೇ
ಪಾಲಿ-ಟನ್
ಪಾಲಿಸಿ
ಪಾಲಿ-ಸಿದ
ಪಾಲು
ಪಾಶ್ಚಾತ್ಯ
ಪಾಶ್ಚಾ-ತ್ಯರ
ಪಾಸಾ-ಗು-ತ್ತಾನೆ
ಪಿತ್ರಾ-ರ್ಜಿತ
ಪಿಪಾಸೆ
ಪಿಪಾ-ಸೆಯೂ
ಪೀಠದ
ಪೀಡಿ-ಸಿ-ದಾ-ಗ-ಲೆಲ್ಲ
ಪೀಡಿ-ಸು-ವು-ದಿತ್ತು
ಪೀಳಿ-ಗೆ-ಯ-ವರೆಲ್ಲ
ಪುಕ್ಕ-ಲ್ರಯ್ಯ
ಪುಟ
ಪುಟ-ಕ್ಕಿಂತ
ಪುಟ-ಗಳ
ಪುಟ-ಗಳನ್ನು
ಪುಟ-ಗ-ಳನ್ನೇ
ಪುಟದ
ಪುಟಿ-ಯ-ಬಲ್ಲ
ಪುಟಿ-ಯು-ತ್ತಿತ್ತು
ಪುಟ್ಟ
ಪುಟ್ಟ-ಪುಟ್ಟ
ಪುಡಿ-ಪು-ಡಿ-ಯಾ-ಯಿತು
ಪುಣಿ
ಪುತ್ರ-ಪ್ರ-ಸ-ವಕ್ಕೆ
ಪುತ್ರ-ಸಂ-ತಾನ
ಪುನಃ
ಪುನ-ರ್ವಿ-ವಾ-ಹ-ವನ್ನು
ಪುರಾ-ತನ
ಪುರು-ಷೋ-ತ್ತ-ಮಾ-ನಂದ
ಪುರು-ಷೋ-ತ್ತ-ಮಾ-ನಂ-ದರ
ಪುರೋ-ಗಾಮಿ
ಪುರೋ-ಹಿ-ತರು
ಪುಳ-ಕಿ-ತ-ಗೊಂಡ
ಪುಷಿ-ಮು-ನಿ-ಗಳ
ಪುಷ್ಪಾ-ರ್ಚನೆ
ಪುಸ್ತಕ
ಪುಸ್ತ-ಕ-ಗಳ
ಪುಸ್ತ-ಕ-ಗಳನ್ನು
ಪುಸ್ತ-ಕ-ಗಳನ್ನೆಲ್ಲ
ಪುಸ್ತ-ಕದ
ಪುಸ್ತ-ಕ-ದಲ್ಲಿ
ಪುಸ್ತ-ಕ-ವನ್ನು
ಪುಸ್ತ-ಕವು
ಪೂಜಿಸಿ
ಪೂಜೆ
ಪೂಜೆ-ಪ್ರಾ-ರ್ಥ-ನೆ-ಗಳ
ಪೂಜೆ-ಪ್ರಾ-ರ್ಥ-ನೆ-ಗಳನ್ನು
ಪೂಜೆಗೆ
ಪೂಜೆಯ
ಪೂಜೆ-ಯಲ್ಲಿ
ಪೂರ್ಣ
ಪೂರ್ಣ-ವಾಗಿ
ಪೂರ್ಣ-ಶ್ರ-ದ್ಧೆ-ಯಿಂದ
ಪೂರ್ತಿ
ಪೂರ್ವ-ಜರು
ಪೂರ್ವ-ಭಾಗ
ಪೂರ್ವ-ಸೂ-ಚಿ-ಗಳು
ಪೇಚಿಗೆ
ಪೇಟ
ಪೇರಿ-ಕೊಂಡು
ಪೇರಿ-ಸಿ-ಕೊಂಡು
ಪೈಜಾ-ಮ-ಜು-ಬ್ಬ-ಗಳಲ್ಲಿ
ಪೈಜಾ-ಮ-ಜು-ಬ್ಬ-ವನ್ನೇ
ಪೈಲ್ವಾ-ನ-ರಿಂದ
ಪೋತ್ಸಾ-ಹಿ-ಸು-ತ್ತಿದ್ದ
ಪೋಷಿಸಿ
ಪೌಷ್ಟಿ-ಕಾ-ಹಾ-ರ-ವನ್ನು
ಪ್ಯಾರಾದ
ಪ್ರಕ-ಟ-ಪ-ಡಿ-ಸಿ-ದ್ದಾನೆ
ಪ್ರಕ-ಟಿ-ಸುವು
ಪ್ರಕಾರ
ಪ್ರಕಾ-ರವೇ
ಪ್ರಕಾ-ಶ-ಕ-ರಿಗೆ
ಪ್ರಕಾ-ಶ-ಕರು
ಪ್ರಕಾ-ಶ-ದಿಂದ
ಪ್ರಕೃತಿ
ಪ್ರಕೃ-ತಿಯ
ಪ್ರಖರ
ಪ್ರಖ-ರ-ವಾ-ಗಿತ್ತು
ಪ್ರಖ್ಯಾತ
ಪ್ರಗ-ತಿ-ಪರ
ಪ್ರಚಂಡ
ಪ್ರಚ-ಲಿ-ತ-ವಾ-ಗ-ತೊ-ಡ-ಗಿತು
ಪ್ರಚ-ಲಿ-ತ-ವಿತ್ತು
ಪ್ರಚ-ಲಿ-ತ-ವಿದ್ದ
ಪ್ರಚಾರ
ಪ್ರಚೋ-ದಿ-ಸಿ-ರ-ಬೇಕು
ಪ್ರಚೋ-ದಿ-ಸುವ
ಪ್ರಜೆ-ಗಳ
ಪ್ರಜೆ-ಗಳು
ಪ್ರಜ್ಞೆ
ಪ್ರಜ್ಞೆಯೇ
ಪ್ರಣಾಮ
ಪ್ರಣಾ-ಳಿ-ಗಳನ್ನು
ಪ್ರಣಾ-ಳಿ-ಯ-ನ್ನೇನೋ
ಪ್ರತಿ
ಪ್ರತಿ-ಕೂಲ
ಪ್ರತಿ-ದಿನ
ಪ್ರತಿ-ದಿ-ನವೂ
ಪ್ರತಿ-ಪಾ-ದಿ-ಸು-ತ್ತಿತ್ತು
ಪ್ರತಿ-ಭ-ಟಿಸಿ
ಪ್ರತಿ-ಭ-ಟಿ-ಸಿ-ದರೋ
ಪ್ರತಿಭಾ
ಪ್ರತಿ-ಭಾ-ವಂತ
ಪ್ರತಿಭೆ
ಪ್ರತಿ-ಭೆ-ಗಳನ್ನೂ
ಪ್ರತಿ-ಭೆಗೆ
ಪ್ರತಿ-ಭೆ-ಯನ್ನು
ಪ್ರತಿ-ಮಾ-ತಾ-ಡದೆ
ಪ್ರತಿ-ಯನ್ನು
ಪ್ರತಿ-ಯಾಗಿ
ಪ್ರತಿ-ಯೊಂದು
ಪ್ರತಿ-ಯೊಬ್ಬ
ಪ್ರತಿ-ಯೊ-ಬ್ಬ-ರಿಗೂ
ಪ್ರತಿ-ರಾ-ತ್ರಿಯೂ
ಪ್ರತಿ-ವಾ-ದ-ದಿಂದ
ಪ್ರತಿ-ವಾ-ದಿಯ
ಪ್ರತಿ-ಷ್ಠಾ-ಪಿಸಿ
ಪ್ರತಿ-ಷ್ಠಾ-ಪಿ-ಸಿ-ಕೊಂಡು
ಪ್ರತಿ-ಷ್ಠಾ-ಪಿ-ಸಿದ
ಪ್ರತಿ-ಷ್ಠಾ-ಪಿ-ಸಿದ್ದ
ಪ್ರತಿ-ಷ್ಠಿ-ತ-ನಾದ
ಪ್ರತ್ಯಕ್ಷ
ಪ್ರತ್ಯು-ತ್ತರ
ಪ್ರತ್ಯು-ತ್ಪ-ನ್ನ-ಮತಿ
ಪ್ರತ್ಯೇಕ
ಪ್ರಥಮ
ಪ್ರದ-ರ್ಶನ
ಪ್ರದ-ರ್ಶ-ನ-ಗಳನ್ನು
ಪ್ರದ-ರ್ಶ-ನಾ-ಲ-ಗ-ಳಿಗೋ
ಪ್ರದೇ-ಶ-ಗಳಲ್ಲಿ
ಪ್ರದೇ-ಶದ
ಪ್ರಧಾನ
ಪ್ರಪಂ-ಚದ
ಪ್ರಪಂ-ಚ-ದಲ್ಲಿ
ಪ್ರಬಲ
ಪ್ರಬ-ಲ-ವಾದ
ಪ್ರಭಾವ
ಪ್ರಭಾ-ವಕ್ಕೆ
ಪ್ರಭಾ-ವ-ಗ-ಳಿಗೆ
ಪ್ರಭಾ-ವಿತ
ಪ್ರಭಾ-ವಿ-ತ-ನಾಗಿ
ಪ್ರಮುಖ
ಪ್ರಮು-ಖ-ನಾ-ದ-ವನು
ಪ್ರಮು-ಖ-ವಾದ
ಪ್ರಮು-ಖ-ವಾ-ದದ್ದು
ಪ್ರಯತ್ನ
ಪ್ರಯ-ತ್ನ-ದಲ್ಲಿ
ಪ್ರಯ-ತ್ನ-ವನ್ನು
ಪ್ರಯ-ತ್ನ-ವನ್ನೇ
ಪ್ರಯ-ತ್ನಿ-ಸು-ತ್ತಿತ್ತು
ಪ್ರಯಾಣ
ಪ್ರಯಾ-ಣದ
ಪ್ರಯಾ-ಣಿ-ಕ-ರಿಗೆ
ಪ್ರಯೋ-ಗಿ-ಸಿ-ಬಿಟ್ಟ
ಪ್ರಯೋ-ಗಿ-ಸಿ-ಯೂ-ಬಿಟ್ಟ
ಪ್ರಯೋ-ಜನ
ಪ್ರಯೋ-ಜ-ನ-ವಾ-ಗ-ಲಿಲ್ಲ
ಪ್ರಯೋ-ಜ-ನ-ವಾ-ದರೂ
ಪ್ರಲೋ-ಭ-ನೆ-ಗಳ
ಪ್ರಲೋ-ಭ-ನೆ-ಗಳು
ಪ್ರವ-ಚ-ನ-ಗಳು
ಪ್ರವಾಸ
ಪ್ರವಾ-ಸದ
ಪ್ರವಾ-ಹ-ದಂತೆ
ಪ್ರವೀಣ
ಪ್ರವೀ-ಣ-ನಾ-ಗಿದ್ದ
ಪ್ರವೀ-ಣ-ನಾ-ಗಿ-ದ್ದರೂ
ಪ್ರವೀ-ಣ-ನಾದ
ಪ್ರವೀ-ಣ-ರಾ-ಗಲು
ಪ್ರವೃತ್ತಿ
ಪ್ರವೃ-ತ್ತಿ-ಯನ್ನು
ಪ್ರವೇಶ
ಪ್ರವೇ-ಶ-ಮಾ-ಡಿತು
ಪ್ರವೇ-ಶ-ವಿಲ್ಲ
ಪ್ರವೇ-ಶಿಸಿ
ಪ್ರಶಂ-ಸಿ-ಸಲು
ಪ್ರಶ್ನೆ
ಪ್ರಶ್ನೆ-ಗಳನ್ನು
ಪ್ರಶ್ನೆಗೆ
ಪ್ರಶ್ನೆ-ಯನ್ನು
ಪ್ರಶ್ನೆಯೂ
ಪ್ರಶ್ನೆ-ಯೇನು
ಪ್ರಸ-ನ್ನ-ಗಂ-ಭೀರ
ಪ್ರಸಾ-ದನ
ಪ್ರಸಿದ್ಧ
ಪ್ರಸಿ-ದ್ಧ-ನಾ-ಗಿದ್ದ
ಪ್ರಸಿ-ದ್ಧಿ-ಗೊಂಡು
ಪ್ರಸೂತ
ಪ್ರಸ್ತಾ-ಪಿಸಿ
ಪ್ರಾಕೃ-ತಿಕ
ಪ್ರಾಚೀ-ನ-ಅ-ರ್ವಾ-ಚೀನ
ಪ್ರಾಣ
ಪ್ರಾಣ-ಗಳನ್ನು
ಪ್ರಾಣದ
ಪ್ರಾಣಿ
ಪ್ರಾಣಿ-ಗಳಿಂದ
ಪ್ರಾಣಿ-ವ-ನ-ವೊಂ-ದಕ್ಕೆ
ಪ್ರಾಥ-ಮಿಕ
ಪ್ರಾಧಾನ
ಪ್ರಾಧ್ಯಾ-ಪ-ಕರು
ಪ್ರಾಪಂ-ಚಿಕ
ಪ್ರಾಪಂ-ಚಿ-ಕ-ತೆ-ಯನ್ನು
ಪ್ರಾಪ್ತ
ಪ್ರಾಪ್ತ-ವಾ-ದದ್ದು
ಪ್ರಾಮಾ-ಣಿಕ
ಪ್ರಾಮಾ-ಣಿ-ಕತೆ
ಪ್ರಾಮು-ಖ್ಯ-ವಾದ
ಪ್ರಾಯ-ಪ್ರ-ಬು-ದ್ಧ-ನಾ-ಗು-ವ-ವರೆ-ಗಾ-ದರೂ
ಪ್ರಾರಂಭ
ಪ್ರಾರಂ-ಭ-ವಾಗಿ
ಪ್ರಾರಂ-ಭ-ವಾ-ಗಿದ್ದ
ಪ್ರಾರಂ-ಭ-ವಾ-ದಾಗ
ಪ್ರಾರಂ-ಭ-ವಿ-ರು-ತ್ತದೆ
ಪ್ರಾರಂ-ಭಿ-ಸಿದ
ಪ್ರಾರಂ-ಭಿ-ಸಿ-ದ-ಈಗ
ಪ್ರಾರಂ-ಭಿಸು
ಪ್ರಾರಂ-ಭಿ-ಸು-ತ್ತಿದ್ದ
ಪ್ರಾರ್ಥನಾ
ಪ್ರಾರ್ಥ-ನಾ-ದಿ-ಗಳ
ಪ್ರಾರ್ಥ-ನಾ-ಮಂ-ದಿ-ರ-ದಲ್ಲಿ
ಪ್ರಾರ್ಥ-ನಾ-ಶ್ಲೋ-ಕ-ಗಳನ್ನು
ಪ್ರಾರ್ಥನೆ
ಪ್ರಾರ್ಥ-ನೆ-ಭ-ಜ-ನೆಯ
ಪ್ರಾರ್ಥ-ನೆ-ಗಳ
ಪ್ರಾರ್ಥ-ನೆಗೆ
ಪ್ರಾರ್ಥಿಸಿ
ಪ್ರಾರ್ಥಿ-ಸಿ-ಕೊ-ಳ್ಳ-ಲಾ-ರಂ-ಭಿ-ಸಿ-ದಳು
ಪ್ರಾರ್ಥಿ-ಸಿ-ಕೊ-ಳ್ಳು-ವುದು
ಪ್ರಿನ್ಸಿ-ಪಾ-ಲರ
ಪ್ರಿನ್ಸಿ-ಪಾ-ಲರೇ
ಪ್ರಿನ್ಸಿ-ಪಾಲ್
ಪ್ರಿಯ
ಪ್ರಿಯ-ನಾ-ಗಿದ್ದ
ಪ್ರಿಯ-ವಲ್ಲ
ಪ್ರಿಯ-ವ-ಲ್ಲವೆ
ಪ್ರಿಯ-ವಾದ
ಪ್ರೀತಿ
ಪ್ರೀತಿ-ಯಿಂದ
ಪ್ರೀತಿ-ಯಿಂ-ದಲೋ
ಪ್ರೀತಿ-ವಿ-ಶ್ವಾ-ಸ-ಗ-ಳಿಗೆ
ಪ್ರೀತಿ-ಸು-ತ್ತಿ-ದ್ದು-ದ-ರಲ್ಲಿ
ಪ್ರೆಸಿ-ಡೆನ್ಸಿ
ಪ್ರೇಕ್ಷಕ
ಪ್ರೇಕ್ಷ-ಕರ
ಪ್ರೇಕ್ಷ-ಕ-ರಲ್ಲಿ
ಪ್ರೇಕ್ಷ-ಕ-ರ-ಲ್ಲೊಬ್ಬ
ಪ್ರೇಕ್ಷ-ಕರು
ಪ್ರೇಕ್ಷ-ಕ-ರೆಲ್ಲ
ಪ್ರೇತ
ಪ್ರೇತ-ವಾ-ದರೂ
ಪ್ರೇತ-ವಾ-ದ್ದ-ರಿಂದ
ಪ್ರೇಮ
ಪ್ರೇಮದ
ಪ್ರೇಮ-ದಿಂದ
ಪ್ರೇಮ-ಭ-ಕ್ತಿಯ
ಪ್ರೇಮ-ಸಂ-ಬಂ-ಧ-ವನ್ನು
ಪ್ರೊಜೆ-ಕ್ಟರ್
ಪ್ರೊಫೆ-ಸರ್
ಪ್ರೋತ್ಸಾ-ಹಿ-ಸಿದ
ಪ್ರೋತ್ಸಾ-ಹಿ-ಸಿ-ದರು
ಪ್ರೋತ್ಸಾ-ಹಿ-ಸು-ತ್ತಿ-ದ್ದುದು
ಫಕೀ-ರರೂ
ಫಲ-ಪು-ಷ್ಪ-ಗಳ
ಫಲ-ವಾಗಿ
ಫಲವೋ
ಫಲಿ-ಸಿತು
ಫಲಿ-ಸಿ-ದ್ದನ್ನು
ಫೀಸ-ನ್ನೇನೋ
ಫೀಸು
ಫೀಸ್
ಬಂಗಾಳ
ಬಂಗಾ-ಳದ
ಬಂಗಾ-ಳ-ದಲ್ಲಿ
ಬಂಗಾಳಿ
ಬಂಗಾಳೀ
ಬಂಟ
ಬಂತು
ಬಂದ
ಬಂದ-ದ್ದ-ರಿಂದ
ಬಂದದ್ದೂ
ಬಂದ-ಬರಿ
ಬಂದ-ಮೇಲೆ
ಬಂದ-ರಿಗೆ
ಬಂದ-ರಿ-ನಲ್ಲಿ
ಬಂದರು
ಬಂದ-ರು-ನ-ರೇಂ-ದ್ರ-ನನ್ನು
ಬಂದರೆ
ಬಂದಳು
ಬಂದ-ವರು
ಬಂದಾಗ
ಬಂದಾ-ಗ-ಲೆಲ್ಲ
ಬಂದಿ-ತಲ್ಲ
ಬಂದಿತು
ಬಂದಿ-ತೆಂ-ದರೆ
ಬಂದಿತ್ತು
ಬಂದಿತ್ತೆ
ಬಂದಿದ್ದ
ಬಂದಿ-ದ್ದರೆ
ಬಂದಿ-ರ-ಬ-ಹುದು
ಬಂದಿ-ರ-ಬ-ಹುದೇ
ಬಂದಿ-ರ-ಲಿಲ್ಲ
ಬಂದಿ-ರುವೆ
ಬಂದು
ಬಂದು-ಬಿಟ್ಟ
ಬಂದು-ಬಿಟ್ಟಿ
ಬಂದು-ಬಿ-ಟ್ಟಿತ್ತು
ಬಂದು-ಬಿ-ಟ್ಟಿದೆ
ಬಂದು-ಬಿಟ್ಟೆ
ಬಂದು-ಹೋಗು
ಬಂದೂಕು
ಬಂದೆ
ಬಂದೇ
ಬಂದ್
ಬಂಧನ
ಬಂಧ-ನ-ದಿಂದ
ಬಂಧು-ಗ-ಳಿಗೆ
ಬಂಧು-ಗಳು
ಬಂಧು-ಗ-ಳೊಂ-ದಿಗೆ
ಬಗ-ಲಲ್ಲಿ
ಬಗ-ಲ-ಲ್ಲಿ-ಟ್ಟಿ-ದ್ದರು
ಬಗ-ಲಿ-ನಲ್ಲಿ
ಬಗೆ
ಬಗೆ-ಗ-ಣ್ಣಿನ
ಬಗೆ-ಬ-ಗೆಯ
ಬಗೆ-ಬ-ಗೆ-ಯಾಗಿ
ಬಗೆಯ
ಬಗೆ-ಯದು
ಬಗೆ-ಯದೇ
ಬಗೆ-ಯಾಗಿ
ಬಗ್ಗಿ
ಬಗ್ಗಿ-ಸ-ಬೇಕು
ಬಗ್ಗೆ
ಬಟ್ಟೆ
ಬಟ್ಟೆ-ಗಳನ್ನು
ಬಟ್ವಿ
ಬಡ
ಬಡ-ಬ-ಗ್ಗರ
ಬಡವ
ಬಡ-ವರ
ಬಡ-ವ-ರಾ-ಗಿ-ರ-ಬ-ಹುದು
ಬಡ-ವ-ರಾ-ದರು
ಬಡ-ವಾ-ಗು-ತ್ತವೆ
ಬಡಾ
ಬಡಾ-ಸಾ-ಹೇ-ಬರ
ಬಡಾ-ಸಾ-ಹೇ-ಬ-ರನ್ನು
ಬಡಿ-ದ-ದ್ದ-ಲ್ಲದೆ
ಬಡಿ-ದುವು
ಬಡಿ-ಬ-ಡಿದು
ಬಡಿ-ಸಿ-ಕೊಂ-ಡರು
ಬಡಿ-ಸಿ-ದಳು
ಬಡಿ-ಸು-ತ್ತಿದ್ದ
ಬಡಿ-ಸು-ವಂಥ
ಬಣ್ಣದ
ಬಣ್ಣ-ನೆ-ಯೆಲ್ಲ
ಬಣ್ಣಿ-ಸಿದ
ಬಣ್ಣಿ-ಸುತ್ತ
ಬಣ್ಣಿ-ಸು-ತ್ತದೆ
ಬಣ್ಣಿ-ಸು-ತ್ತಾನೆ
ಬಣ್ಣಿ-ಸು-ತ್ತಿ-ದ್ದಳು
ಬಣ್ಣಿ-ಸು-ವ-ವರು
ಬತ್ತಿ-ಹೋ-ಗಿ-ರ-ಲಿಲ್ಲ
ಬದ-ಲಾಗಿ
ಬದ-ಲಾ-ಗು-ತ್ತಿತ್ತು
ಬದ-ಲಾ-ಯಿ-ಸುವ
ಬದ-ಲಾ-ವಣೆ
ಬದ-ಲಿಗೆ
ಬದಿ-ಗೊ-ತ್ತಿ-ದಳು
ಬದಿ-ಯಲ್ಲೇ
ಬದು-ಕಿ-ದರು
ಬದ್ಧ
ಬನ್ನಿ
ಬಯ-ಕೆ-ಗಳನ್ನು
ಬಯ-ಲಾ-ಗು-ತ್ತದೆ
ಬಯ-ಸುವ
ಬಯ-ಸು-ವ-ವನು
ಬರದೆ
ಬರದೇ
ಬರ-ಬ-ರುತ್ತ
ಬರ-ಬೇ-ಕಾ-ಗಿತ್ತು
ಬರ-ಬೇ-ಕಾ-ಗು-ತ್ತದೆ
ಬರ-ಬೇ-ಕಾ-ದರೆ
ಬರ-ಬೇ-ಕೆಂದು
ಬರ-ಮಾ-ಡಿ-ಕೊಂ-ಡಿದ್ದ
ಬರ-ಮಾ-ಡಿ-ಕೊಂಡು
ಬರ-ಲಾ-ರಂ-ಭ-ವಾ-ಯಿತು
ಬರ-ಲಿಲ್ಲ
ಬರ-ಲಿ-ಲ್ಲ-ವೆಂ-ದೇ-ನಲ್ಲ
ಬರಲು
ಬರಲೇ
ಬರ-ವನ್ನೇ
ಬರಿ-ಗೈ-ಯಲ್ಲಿ
ಬರಿ-ದಾ-ಯಿತು
ಬರಿಯ
ಬರಿ-ಸು-ತ್ತಿದ್ದ
ಬರು
ಬರು-ತ್ತದೆ
ಬರು-ತ್ತಾನೆ
ಬರು-ತ್ತಾ-ರ-ಲ್ಲವೆ
ಬರು-ತ್ತಿತ್ತು
ಬರು-ತ್ತಿದೆ
ಬರು-ತ್ತಿ-ದೆ-ಯೇನೋ
ಬರು-ತ್ತಿದ್ದ
ಬರು-ತ್ತಿ-ದ್ದರು
ಬರು-ತ್ತಿ-ದ್ದವು
ಬರು-ತ್ತಿ-ದ್ದಾನೆ
ಬರು-ತ್ತಿ-ರ-ಲಿಲ್ಲ
ಬರು-ತ್ತಿ-ರ-ವುದನ್ನು
ಬರು-ತ್ತಿ-ರುವ
ಬರು-ತ್ತಿ-ರು-ವಾಗ
ಬರು-ತ್ತೀರೋ
ಬರುವ
ಬರು-ವ-ವಳ
ಬರು-ವಾಗ
ಬರು-ವು-ದ-ಕ್ಕೊಂದು
ಬರು-ವು-ದಿಲ್ಲ
ಬರು-ವುದೂ
ಬರೆದ
ಬರೆ-ದದ್ದೂ
ಬರೆ-ದಳು
ಬರೆ-ದಿದೆ
ಬರೆ-ದಿ-ದ್ದನ್ನು
ಬರೆ-ದಿಲ್ಲ
ಬರೆದು
ಬರೆ-ದು-ಬಿಟ್ಟ
ಬರೆ-ಯ-ಬ-ಲ್ಲ-ವ-ನಾ-ಗಿದ್ದ
ಬರೆ-ಯ-ಬೇ-ಕಾಗಿ
ಬರೆ-ಯಲೂ
ಬಲ
ಬಲ-ಗೈ-ಯಲ್ಲಿ
ಬಲ-ಗೊಂಡು
ಬಲ-ವಾಗಿ
ಬಲ-ವಾದ
ಬಲವೂ
ಬಲ-ಶಾ-ಲಿ-ಯಾ-ಗಿ-ರ-ಬೇಕು
ಬಲ-ಹು-ಬ್ಬಿನ
ಬಲಿ-ಯು-ತ್ತಿತ್ತು
ಬಲು
ಬಲ್ಲ
ಬಲ್ಲ-ನೆಂದೂ
ಬಲ್ಲರು
ಬಲ್ಲ-ವ-ನಾ-ಗಿದ್ದ
ಬಲ್ಲ-ವರು
ಬಲ್ಲೆ
ಬಳ-ಪದ
ಬಳ-ಲಿದ
ಬಳಿ
ಬಳಿಕ
ಬಳಿಗೆ
ಬಳಿಗೇ
ಬಳಿ-ಗೋಡಿ
ಬಳಿ-ಗೋ-ಡಿದ
ಬಳಿ-ಗೋ-ಡು-ತ್ತಿದ್ದ
ಬಳಿ-ಯಲ್ಲಿ
ಬಳಿ-ಯಿ-ರು-ವ-ವರೆ-ಲ್ಲರೂ
ಬವ-ಣೆ-ಗಳನ್ನು
ಬಸು
ಬಹಳ
ಬಹ-ಳ-ವಾಗಿ
ಬಹ-ಳಷ್ಟು
ಬಹು-ಎ-ತ್ತ-ರ-ದಲ್ಲಿ
ಬಹು-ಕಾಲ
ಬಹು-ಜನ
ಬಹು-ತೇಕ
ಬಹು-ದಾ-ಗಿತ್ತು
ಬಹು-ದಿ-ನ-ಗ-ಳ-ವ-ರೆಗೆ
ಬಹುದು
ಬಹು-ದೂರ
ಬಹು-ದೇ-ವತಾ
ಬಹು-ಭಾ-ಗ-ವನ್ನು
ಬಹು-ಮ-ಟ್ಟಿಗೆ
ಬಹು-ಮಾ-ನ-ವಾಗಿ
ಬಹು-ಮು-ಖ-ವಾ-ದದ್ದು
ಬಹು-ಮು-ಖ್ಯ-ವಾದ
ಬಹು-ವಾಗಿ
ಬಹುಶಃ
ಬಾ
ಬಾಗಿ
ಬಾಗಿಲ
ಬಾಗಿ-ಲನ್ನು
ಬಾಗಿ-ಲಲ್ಲಿ
ಬಾಗಿ-ಲಲ್ಲೇ
ಬಾಗಿ-ಲಿಗೆ
ಬಾಗಿಲು
ಬಾಗಿ-ಲು-ಗಳನ್ನು
ಬಾಡಿಗೆ
ಬಾಣ-ಸಿಗ
ಬಾಯ-ಲ್ಲಿ-ರು-ತ್ತಿತ್ತು
ಬಾಯಿ
ಬಾಯಿಂದ
ಬಾಯಿ-ಗಿ-ಟ್ಟು-ಕೊಂ-ಡರೆ
ಬಾಯಿಗೆ
ಬಾಯ್ತುಂಬ
ಬಾರ-ದಂತೆ
ಬಾರದೆ
ಬಾರಯ್ಯ
ಬಾರಿ
ಬಾರಿ-ಸ-ತೊ-ಡ-ಗಿ-ದರು
ಬಾರಿ-ಸು-ತ್ತಲೇ
ಬಾರು
ಬಾಲ
ಬಾಲಕ
ಬಾಲ-ಕ-ರೆ-ಲ್ಲರೂ
ಬಾಲ-ಕ-ಲ್ಪನೆ
ಬಾಲ-ಜ-ಗ-ತ್ತನ್ನು
ಬಾಲ-ನಾ-ಗಿ-ರು-ವಾ-ಗಲೇ
ಬಾಲ-ಭಾ-ಷೆ-ಯಲ್ಲಿ
ಬಾಲ-ವಿ-ಧವೆ
ಬಾಲ-ವಿ-ಧ-ವೆ-ಯಾ-ಗುವ
ಬಾಲ-ವಿ-ಧ-ವೆ-ಯಿ-ದ್ದಳು
ಬಾಲ-ಹೃ-ದ-ಯಕ್ಕೆ
ಬಾಲ್ಯ
ಬಾಲ್ಯ-ಕೌ-ಮಾ-ರ್ಯ-ಗ-ಳಿಂ-ದಲೇ
ಬಾಲ್ಯ-ಕಾ-ಲ-ದಿಂ-ದಲೂ
ಬಾಲ್ಯದ
ಬಾಲ್ಯ-ದಲ್ಲಿ
ಬಾಲ್ಯ-ದಲ್ಲೇ
ಬಾಲ್ಯ-ದಿಂ-ದಲೂ
ಬಾಲ್ಯ-ದಿಂ-ದಲೇ
ಬಾಲ್ಯ-ವನ್ನು
ಬಾಲ್ಯ-ವಿ-ವಾಹ
ಬಾಲ್ಯ-ವಿ-ವಾ-ಹದ
ಬಾಲ್ಯ-ವಿ-ವಾ-ಹವೇ
ಬಾಲ್ಯ-ವೆಂಬ
ಬಾಲ್ಯವೇ
ಬಾಳೆಯ
ಬಾಳ್ವೆ
ಬಾವಲಿ
ಬಾವುಲ್
ಬಾವು-ಲ್ಗಳು
ಬಾವು-ಲ್ಗ-ಳೆ-ನ್ನು-ತ್ತಾ-ರೆಈ
ಬಾಹು-ಗಳನ್ನು
ಬಾಹ್ಯ
ಬಾಹ್ಯ-ಪ್ರಜ್ಞೆ
ಬಾಹ್ಯ-ಪ್ರ-ಜ್ಞೆಯೇ
ಬಾಹ್ಯಾ-ಡಂ-ಬ-ರದ
ಬಿ
ಬಿಎ
ಬಿಎ-ವ-ರೆಗೆ
ಬಿಕ್ಕಿ-ಬಿಕ್ಕಿ
ಬಿಗಿದ
ಬಿಗಿ-ಹಿ-ಡಿ-ದು-ಕೊಂಡು
ಬಿಚ್ಚಿ
ಬಿಚ್ಚಿ-ಕೊಂಡು
ಬಿಟ್ಟ
ಬಿಟ್ಟ-ನಾ-ದರೂ
ಬಿಟ್ಟರೆ
ಬಿಟ್ಟಳು
ಬಿಟ್ಟಿತು
ಬಿಟ್ಟಿತ್ತು
ಬಿಟ್ಟಿದೆ
ಬಿಟ್ಟಿದ್ದ
ಬಿಟ್ಟಿ-ದ್ದರೆ
ಬಿಟ್ಟು
ಬಿಟ್ಟು-ಬಿಟ್ಟ
ಬಿಟ್ಟು-ಬಿ-ಟ್ಟರೆ
ಬಿಟ್ಟು-ಬಿಟ್ಟೆ
ಬಿಟ್ಟು-ಹೋ-ಗಲು
ಬಿಡದೆ
ಬಿಡ-ಬಲ್ಲ
ಬಿಡ-ಬೇ-ಕಾ-ಯಿತು
ಬಿಡ-ಲಿಲ್ಲ
ಬಿಡಲೇ
ಬಿಡಿ
ಬಿಡಿ-ಸ-ಲಾ-ರದ
ಬಿಡಿ-ಸಲು
ಬಿಡಿಸಿ
ಬಿಡಿ-ಸಿ-ಕೊಂಡು
ಬಿಡಿ-ಸಿ-ದರು
ಬಿಡಿ-ಸು-ತ್ತಿದ್ದ
ಬಿಡುತ್ತ
ಬಿಡು-ತ್ತಾ-ನೆಯೆ
ಬಿಡು-ತ್ತಿದ್ದ
ಬಿಡು-ತ್ತಿ-ದ್ದರು
ಬಿಡು-ತ್ತಿ-ರ-ಲಿಲ್ಲ
ಬಿಡು-ತ್ತೇನೆ
ಬಿಡು-ವ-ವ-ನಲ್ಲ
ಬಿಡು-ವಿದೆ
ಬಿಡು-ವಿನ
ಬಿಡು-ವು-ದಕ್ಕೂ
ಬಿತ್ತಿ-ದಂ-ತಾ-ಯಿತು
ಬಿತ್ತು
ಬಿದ್ದ
ಬಿದ್ದಾಗ
ಬಿದ್ದು
ಬಿದ್ದು-ಬಿಟ್ಟ
ಬಿದ್ದು-ಬಿ-ಟ್ಟಳು
ಬಿದ್ದು-ಬಿ-ಡು-ತ್ತಿದ್ದ
ಬಿರು-ಕನ್ನೇ
ಬಿರುಕು
ಬಿರು-ಸಾದ
ಬಿರು-ಸಿ-ನಿಂದ
ಬಿರುಸೇ
ಬಿಲಾ-ಸ-ಪುರ
ಬಿಲೇ
ಬಿಲ್ಲಂ-ಬೆ-ರ-ಗಾ-ದರು
ಬಿಳಿ
ಬಿಳಿ-ಯ-ರನ್ನು
ಬಿಳೀ
ಬಿಸಿ
ಬಿಸಿ-ಯೇ-ರಿತು
ಬಿಸಿ-ರ-ಕ್ತದ
ಬಿಸಿ-ಲಿ-ನಲ್ಲೂ
ಬೀಗು-ತ್ತಿತ್ತು
ಬೀಜ
ಬೀದಿ-ಗಳಲ್ಲಿ
ಬೀಭತ್ಸ
ಬೀರ-ಲಾ-ರಂ-ಭಿ-ಸಿತ್ತು
ಬೀರಿತು
ಬೀರಿ-ದುವು
ಬೀರಿ-ಬಿ-ಟ್ಟಿತ್ತು
ಬೀರುತ್ತ
ಬೀರು-ತ್ತಿತ್ತು
ಬೀರು-ವಷ್ಟು
ಬೀರು-ವುದು
ಬೀಳ-ದಂತೆ
ಬೀಳದೆ
ಬೀಳ-ಬ-ಹು-ದೆಂ-ಬುದು
ಬೀಳ-ಲಿಲ್ಲ
ಬೀಳು-ತ್ತ-ದೆಯೇ
ಬೀಳು-ತ್ತಿದ್ದ
ಬೀಳುವ
ಬೀಳು-ವಂತೆ
ಬೀಳ್ಗೊಂಡ
ಬೀಳ್ಗೊಡ
ಬೀಳ್ಗೊ-ಡಲು
ಬೀಳ್ಗೊ-ಳ್ಳಲು
ಬೀಸ-ಬಲ್ಲೆ
ಬೀಸಿದ
ಬೀಸಿ-ದ-ನೆಂ-ದರೆ
ಬೀಸುತ್ತ
ಬೀಸು-ತ್ತಿದೆ
ಬೀಸು-ತ್ತಿ-ದ್ದಾನೆ
ಬೀಸುವ
ಬೀಸು-ವು-ದಕ್ಕೆ
ಬೀಸು-ವು-ದ-ರಲ್ಲಿ
ಬುಡ-ದಲ್ಲಿ
ಬುದ್ಧಿ
ಬುದ್ಧಿ-ಮ-ತ್ತೆಗೆ
ಬುದ್ಧಿಯ
ಬುದ್ಧಿ-ಯನ್ನು
ಬುದ್ಧಿ-ಯಲ್ಲಿ
ಬುದ್ಧಿ-ವಂ-ತ-ನಾದ
ಬುದ್ಧಿ-ಶಕ್ತಿ
ಬುದ್ಧಿ-ಶ-ಕ್ತಿ-ಸಾ-ಮ-ರ್ಥ್ಯ-ಗಳನ್ನು
ಬುದ್ಧಿ-ಶ-ಕ್ತಿಗೆ
ಬುದ್ಧಿ-ಶ-ಕ್ತಿಯ
ಬುದ್ಧಿ-ಶ-ಕ್ತಿ-ಯನ್ನು
ಬುದ್ಧಿ-ಶ-ಕ್ತಿ-ಯಲ್ಲಿ
ಬುದ್ಧಿ-ಶ-ಕ್ತಿ-ಸಂ-ಪ-ನ್ನ-ನಾ-ದರೂ
ಬುದ್ಧಿ-ಶಾ-ಲಿಗೆ
ಬುದ್ಧಿ-ಸಾ-ಮ-ರ್ಥ್ಯದ
ಬುದ್ಧಿ-ಸಾ-ಮ-ರ್ಥ್ಯ-ವನ್ನು
ಬುದ್ಧಿ-ಸಾ-ಮ-ರ್ಥ್ಯ-ವನ್ನೂ
ಬೃಹ-ತ್ಕಾ-ರ್ಯ-ಗಳನ್ನು
ಬೃಹ-ತ್ತಾದ
ಬೃಹ-ದಾ-ಕಾ-ರದ
ಬೆಂಚಿನ
ಬೆಂಚಿ-ನಲ್ಲೇ
ಬೆಂಚಿ-ನಿಂದ
ಬೆಂಬ-ಲಿ-ಸು-ತ್ತಿದ್ದ
ಬೆಕ್ಕಸ
ಬೆಕ್ಕ-ಸ-ಬೆ-ರ-ಗಾಗಿ
ಬೆಚ್ಚಿ
ಬೆಚ್ಚಿತು
ಬೆಚ್ಚಿ-ಬಿದ್ದ
ಬೆಚ್ಚಿ-ಬಿದ್ದು
ಬೆಟ್ಟದ
ಬೆತ್ತ
ಬೆತ್ತ-ದಿಂದ
ಬೆತ್ತ-ದೇಟು
ಬೆನ್ನ-ಮೇಲೆ
ಬೆನ್ನು
ಬೆಪ್ಪಾಗಿ
ಬೆಪ್ಪು-ಗಟ್ಟಿ
ಬೆರ-ಗಾ-ಗದೆ
ಬೆರ-ಗಾಗಿ
ಬೆರ-ಗಾ-ಗು-ತ್ತಿ-ದ್ದರು
ಬೆರ-ಗಾ-ದರು
ಬೆರಳ
ಬೆರ-ಳು-ಗಳನ್ನು
ಬೆರೆ-ಯ-ಬಲ್ಲ
ಬೆರೆ-ಯು-ತ್ತಿದ್ದ
ಬೆರೆಸಿ
ಬೆಲೆ-ಬಾ-ಳುವ
ಬೆಳ-ಕನ್ನು
ಬೆಳಕು
ಬೆಳ-ಗ-ಲಿ-ರುವ
ಬೆಳಗಿ
ಬೆಳ-ಗು-ತ್ತಿದೆ
ಬೆಳ-ಗು-ತ್ತಿದ್ದ
ಬೆಳ-ಗು-ತ್ತಿ-ರುವ
ಬೆಳಗ್ಗೆ
ಬೆಳೆದ
ಬೆಳೆ-ದರೂ
ಬೆಳೆದು
ಬೆಳೆ-ದು-ಕೊಂಡು
ಬೆಳೆ-ದು-ನಿಂತ
ಬೆಳೆ-ದು-ಬಂ-ದದ್ದು
ಬೆಳೆ-ದು-ಬಂ-ದಿತ್ತು
ಬೆಳೆ-ದು-ಬಂ-ದಿದ್ದ
ಬೆಳೆ-ದು-ಬ-ರು-ತ್ತಿತ್ತು
ಬೆಳೆಯ
ಬೆಳೆ-ಯಲೂ
ಬೆಳೆ-ಯಿತು
ಬೆಳೆ-ಯುತ್ತ
ಬೆಳೆ-ಯು-ತ್ತವೆ
ಬೆಳೆ-ಯು-ತ್ತಿತ್ತು
ಬೆಳೆ-ಯುವ
ಬೆಳೆ-ಯು-ವು-ದಿ-ಲ್ಲಪ್ಪ
ಬೆಳೆ-ಸ-ಬೇ-ಕಾ-ಯಿತು
ಬೆಳೆ-ಸ-ಬೇಕು
ಬೆಳೆ-ಸಿ-ಕೊ-ಳ್ಳುವ
ಬೆಳೆ-ಸಿದ್ದ
ಬೆಳೆ-ಸು-ವುದು
ಬೆಳ್ಳಿಯ
ಬೆವತು
ಬೇಕ-ಲ್ಲವೆ
ಬೇಕಾ-ಗಿದೆ
ಬೇಕಾ-ಗಿ-ದ್ದಾರೆ
ಬೇಕಾ-ಗಿ-ರುವ
ಬೇಕಾ-ಗಿ-ರು-ವು-ದ-ರಿಂದ
ಬೇಕಾ-ಗಿಲ್ಲ
ಬೇಕಾ-ಗು-ತ್ತದೆ
ಬೇಕಾ-ಗು-ತ್ತವೆ
ಬೇಕಾ-ಗು-ತ್ತಿತ್ತು
ಬೇಕಾ-ಗು-ವು-ದಿಲ್ಲ
ಬೇಕಾದ
ಬೇಕಾ-ದರೂ
ಬೇಕಾ-ದರೆ
ಬೇಕಾ-ದುವು
ಬೇಕಿ-ದ್ದರೆ
ಬೇಕು
ಬೇಕೆಂ-ಬಷ್ಟು
ಬೇಕೆಂ-ಬು-ದನ್ನು
ಬೇಕೆ-ನ್ನುವ
ಬೇಕೋ
ಬೇಡ
ಬೇಡವೆ
ಬೇಡಿ
ಬೇಡಿ-ದರು
ಬೇಡಿ-ದರೆ
ಬೇರಾವ
ಬೇರಿ-ನಂತೆ
ಬೇರೆ
ಬೇರೆ-ಬೇರೆ
ಬೇರೆ-ಯಾಗಿ
ಬೇರೆಯೇ
ಬೇಳೆ-ಯಿಂದ
ಬೇಸ-ರ-ವಾ-ದಾ-ಗ-ಲೆಲ್ಲ
ಬೈದರು
ಬೈದಿ-ದ್ದಾನೆ
ಬೈದು
ಬೈಬ-ಲಿನ
ಬೈಬಲ್
ಬೈಯ-ಲಿಲ್ಲ
ಬೈರಾಗಿ
ಬೈರಾ-ಗಿಗೆ
ಬೊಂಬೆ-ಗಳನ್ನು
ಬೊಚ್ಚು-ಬಾಯಿ
ಬೋಧನೆ
ಬೋಧ-ನೆ-ಗಳನ್ನು
ಬೋಧ-ನೆ-ಯನ್ನು
ಬೋರ-ಲಾಗಿ
ಬೋರ್ಡ್
ಬೌದ್ಧಿಕ
ಬೌದ್ಧಿ-ಕ-ವಾಗಿ
ಬ್ದಾರಿ-ಯೆಲ್ಲ
ಬ್ಯಾನರ್ಜಿ
ಬ್ಯಾನ-ರ್ಜಿ-ಯ-ವರ
ಬ್ಯಾನ-ರ್ಜಿ-ಯ-ವ-ರನ್ನು
ಬ್ಯಾನ-ರ್ಜಿ-ಯ-ವರು
ಬ್ರಹ್ಮ
ಬ್ರಹ್ಮ-ದೈತ್ಯ
ಬ್ರಹ್ಮ-ರಾ-ಕ್ಷಸ
ಬ್ರಹ್ಮ-ವನ್ನು
ಬ್ರಹ್ಮ-ವನ್ನೇ
ಬ್ರಹ್ಮಾದಿ
ಬ್ರಾಹ್ಮ
ಬ್ರಾಹ್ಮಣ
ಬ್ರಾಹ್ಮ-ಧು-ರೀ-ಣ-ನಾದ
ಬ್ರಾಹ್ಮ-ಸ-ಮಾಜ
ಬ್ರಾಹ್ಮ-ಸ-ಮಾ-ಜಕ್ಕೆ
ಬ್ರಾಹ್ಮ-ಸ-ಮಾ-ಜದ
ಬ್ರಾಹ್ಮ-ಸ-ಮಾ-ಜ-ದಲ್ಲಿ
ಬ್ರಾಹ್ಮ-ಸ-ಮಾ-ಜ-ದ-ವರು
ಬ್ರಾಹ್ಮ-ಸ-ಮಾ-ಜ-ದಿಂದ
ಬ್ರಾಹ್ಮ-ಸ-ಮಾ-ಜ-ವನ್ನು
ಬ್ರಾಹ್ಮ-ಸ-ಮಾ-ಜವು
ಬ್ರಾಹ್ಮ-ಸ-ಮಾ-ಜೀ-ಯರು
ಬ್ರಿಟಿ-ಷರ
ಬ್ರಿಟಿಷ್
ಭಂಗ
ಭಂಗ-ವಾಗಿ
ಭಕ್ತ-ನಾದ
ಭಕ್ತ-ನಾ-ದ-ವನು
ಭಕ್ತ-ನಿಗೆ
ಭಕ್ತ-ವರ
ಭಕ್ತಿ
ಭಕ್ತಿ-ಪ್ರೇಮ
ಭಕ್ತಿ-ಗೀ-ತೆ-ಗಳು
ಭಕ್ತಿ-ಪರ
ಭಕ್ತಿ-ಮತಿ
ಭಕ್ತಿ-ಯಾ-ಗಲಿ
ಭಕ್ತಿ-ಯಿಂದ
ಭಕ್ತಿ-ಯುತ
ಭಕ್ತಿ-ಸಾ-ಧ-ನೆಯ
ಭಕ್ಷ್ಯ-ಗಳನ್ನು
ಭಗ
ಭಗ-ವಂತ
ಭಗ-ವಂ-ತನ
ಭಗ-ವಂ-ತ-ನನ್ನು
ಭಗ-ವಂ-ತ-ನಲ್ಲಿ
ಭಗ-ವಂ-ತ-ನಿಗೆ
ಭಗ-ವಂ-ತನು
ಭಗ-ವ-ತ್ಸಾಕ್ಷಾ
ಭಗ-ವ-ತ್ಸಾ-ಕ್ಷಾ-ತ್ಕಾ-ರದ
ಭಗ-ವ-ದಿ-ಚ್ಛೆಗೆ
ಭಗ-ವ-ದ್ದ-ರ್ಶನ
ಭಗ-ವ-ದ್ಭಾವ
ಭಗ-ವಾನ್
ಭಗೀ-ರಥ
ಭಜನ
ಭಜ-ನೆ-ಪ್ರಾ-ರ್ಥ-ನೆ-ಧ್ಯಾನ
ಭಜ-ನೆ-ಪ್ರಾ-ರ್ಥ-ನೆ-ಧ್ಯಾ-ನ-ಗಳನ್ನು
ಭಜ-ನೆ-ಪ್ರಾ-ರ್ಥ-ನೆ-ಧ್ಯಾ-ನಾ-ದಿ-ಗಳನ್ನು
ಭಜ-ನೆ-ಗ-ಳ-ಷ್ಟ-ರಿಂ-ದಲೇ
ಭಟರು
ಭಯ
ಭಯಂ-ಕರ
ಭಯಂ-ಕ-ರ-ವಾ-ಗಿತ್ತು
ಭಯಂ-ಕ-ರ-ವಾದ್ದು
ಭಯ-ಗೊಂ-ಡಿ-ದ್ದರು
ಭಯ-ಗೌ-ರ-ವ-ಗಳಿಂದ
ಭಯ-ದಿಂದ
ಭಯ-ವಾ-ಗ-ದಿ-ರು-ತ್ತ-ದೆಯೆ
ಭಯ-ವಾಗಿ
ಭರ-ದಿಂದ
ಭರ-ವಸೆ
ಭರಾ-ಟೆ-ಯನ್ನು
ಭರಿ-ತ-ರಾ-ಗು-ತ್ತಿ-ದ್ದರು
ಭರ್ಜರಿ
ಭವಿಷ್ಯ
ಭವಿ-ಷ್ಯ-ಜೀ-ವ-ನದ
ಭವಿ-ಷ್ಯದ
ಭವಿ-ಷ್ಯ-ವಾಣಿ
ಭವಿ-ಷ್ಯ-ವಿದೆ
ಭವಿ-ಷ್ಯ-ವೆಲ್ಲ
ಭವ್ಯ
ಭವ್ಯ-ತೆ-ಯನ್ನು
ಭಾಗ
ಭಾಗ-ವ-ತದ
ಭಾಗ-ವ-ಹಿ-ಸ-ಲಾ-ರಂ-ಭಿ-ಸಿದ
ಭಾಗ-ವ-ಹಿ-ಸು-ತ್ತಿದ್ದ
ಭಾಗಿ-ಯಾ-ಗು-ತ್ತಿತ್ತು
ಭಾಗ್ಯವೂ
ಭಾರ-ತಕ್ಕೆ
ಭಾರ-ತದ
ಭಾರ-ತ-ದಲ್ಲಿ
ಭಾರ-ತ-ದಾ-ದ್ಯಂತ
ಭಾರ-ತೀಯ
ಭಾರ-ತೀ-ಯ-ರಾದ
ಭಾರ-ದಿಂದ
ಭಾರ-ವ-ನ್ನೆಲ್ಲ
ಭಾರ-ವಾದ
ಭಾರೀ
ಭಾವ
ಭಾವಕ್ಕೆ
ಭಾವ-ದಂ-ತ-ರಂ-ಗ-ವನ್ನು
ಭಾವ-ದಲ್ಲಿ
ಭಾವನೆ
ಭಾವ-ನೆ-ಗಳ
ಭಾವ-ನೆ-ಗ-ಳ-ಲ್ಲೆಲ್ಲ
ಭಾವ-ನೆಯೇ
ಭಾವ-ಪ-ರ-ವ-ಶ-ನಾ-ದಂತೆ
ಭಾವ-ಪ-ರ-ವ-ಶ-ವಾ-ಗಿ-ಬಿ-ಡು-ತ್ತಿತ್ತು
ಭಾವ-ಭ-ರಿತ
ಭಾವ-ಭ-ರಿ-ತ-ನಾಗಿ
ಭಾವ-ವನ್ನು
ಭಾವ-ಸ-ಮಾ-ಧಿಯ
ಭಾವ-ಸ-ಮಾ-ಧಿ-ಯಂ-ತಹ
ಭಾವ-ಸೂಕ್ಷ್ಮ
ಭಾವಿಸಿ
ಭಾವಿ-ಸಿ-ಕೊ-ಳ್ಳು-ತ್ತಿದ್ದ
ಭಾವಿ-ಸಿ-ದರೆ
ಭಾವಿ-ಸಿದ್ದ
ಭಾವಿ-ಸಿಯೇ
ಭಾವಿ-ಸಿ-ರ-ಬೇಕು
ಭಾವಿ-ಸುತ್ತ
ಭಾವೀ
ಭಾವುಕ
ಭಾವು-ಕ-ತೆ-ಯನ್ನು
ಭಾವೈ-ಕ್ಯಕ್ಕೆ
ಭಾವೋ-ದ್ವೇಗ
ಭಾವೋ-ದ್ವೇ-ಗ-ಗಳ
ಭಾಷಣ
ಭಾಷ-ಣ-ಕಾ-ರ-ನಾದ
ಭಾಷ-ಣ-ವನ್ನು
ಭಾಷ-ಣ-ವನ್ನೇ
ಭಾಷಾ-ಶೈಲಿ
ಭಾಷೆ
ಭಾಷೆ-ಗಳ
ಭಾಷೆ-ಗಳನ್ನು
ಭಾಷೆ-ಗಳಲ್ಲಿ
ಭಾಷೆ-ಗ-ಳು-ಇ-ವೆ-ಲ್ಲ-ದರ
ಭಾಷೆ-ಯನ್ನು
ಭಾಷೆ-ಯಲ್ಲಿ
ಭಾಷೆ-ಯಲ್ಲೇ
ಭಾಷೆ-ಯಿಂದ
ಭಾಸ-ವಾ-ಗು-ತ್ತಿತ್ತು
ಭಿಕ್ಷು-ಕ-ರನ್ನು
ಭಿಕ್ಷು-ಕ-ರು-ಇ-ವ-ರನ್ನು
ಭಿಕ್ಷೆ
ಭಿಕ್ಷೆ-ಗಾಗಿ
ಭಿಕ್ಷೆ-ಯಿಂದ
ಭುವ-ನ-ವನ್ನೇ
ಭುವನೇ
ಭುವ-ನೇ-ಶ್ವರಿ
ಭುವ-ನೇ-ಶ್ವ-ರಿಗೆ
ಭುವ-ನೇ-ಶ್ವ-ರಿಯ
ಭುವ-ನೇ-ಶ್ವ-ರಿಯೂ
ಭುವ-ನೇ-ಶ್ವರೀ
ಭುವಿ-ಯನ್ನು
ಭೂಗೋ-ಳದ
ಭೂಗೋ-ಳ-ದಲ್ಲಿ
ಭೂತ
ಭೂತ-ವ-ರ್ತ-ಮಾ-ನ-ಭ-ವಿ-ಷ್ಯ-ತ್ಕಾ-ಲ-ಗಳನ್ನೆಲ್ಲ
ಭೂದೇ-ವಿ-ಯನ್ನು
ಭೂಮಿ
ಭೇಟಿ
ಭೇದ-ವಿ-ಲ್ಲದೆ
ಭೇದವೇ
ಭೇಷ್
ಭೋಗ
ಭೋರ್ಗ-ರೆ-ಯು-ತ್ತಿತ್ತು
ಭ್ರೂಮಧ್ಯ
ಮಂಗ-ಮಾಯ
ಮಂಗ-ಲ-ಕರ
ಮಂಜುಳ
ಮಂಜೂರು
ಮಂಡಿ-ಸುವ
ಮಂತ್ರ-ದಂತೆ
ಮಂತ್ರಿ-ಗಳು
ಮಂತ್ರಿ-ಯಂತೆ
ಮಂದ
ಮಂದ-ಮಾ-ರು-ತವೂ
ಮಂದಿ
ಮಂದಿ-ಯೆಲ್ಲ
ಮಂದಿ-ರದ
ಮಂದಿ-ರ-ದಲ್ಲಿ
ಮಂಪರು
ಮಕ-ರ-ಸಂ-ಕ್ರಾಂ-ತಿಯ
ಮಕ್ಕಳ
ಮಕ್ಕ-ಳನ್ನು
ಮಕ್ಕ-ಳ-ನ್ನೆಲ್ಲ
ಮಕ್ಕ-ಳಾ-ಗಿ-ದ್ದುವು
ಮಕ್ಕಳಿ
ಮಕ್ಕ-ಳಿಂದ
ಮಕ್ಕ-ಳಿ-ಗಾಗಿ
ಮಕ್ಕ-ಳಿಗೆ
ಮಕ್ಕಳು
ಮಕ್ಕ-ಳು-ದು-ರ್ಗಾ-ಪ್ರ-ಸಾದ
ಮಕ್ಕಳೂ
ಮಕ್ಕ-ಳೆಲ್ಲ
ಮಕ್ಕ-ಳೆಷ್ಟೋ
ಮಗ
ಮಗನ
ಮಗ-ನನ್ನು
ಮಗ-ನಾಗಿ
ಮಗನಿ
ಮಗ-ನಿ-ಗಂತೂ
ಮಗ-ನಿಗೆ
ಮಗ-ನೊಂ-ದಿಗೆ
ಮಗಳು
ಮಗು
ಮಗು-ವನ್ನು
ಮಗು-ವ-ನ್ನೇನೋ
ಮಗು-ವಲ್ಲ
ಮಗು-ವ-ಲ್ಲವೇ
ಮಗುವಿ
ಮಗು-ವಿಗೆ
ಮಗು-ವಿನ
ಮಗು-ವಿ-ಲ್ಲ-ವೆಂಬ
ಮಗೂ
ಮಗ್ನ
ಮಗ್ನ-ವಾ-ಗಿ-ದ್ದರೆ
ಮಗ್ನ-ವಾ-ಗು-ತ್ತಿತ್ತು
ಮಟ-ಗು-ಟ್ಟುತ್ತ
ಮಟ್ಟದ
ಮಡದಿ
ಮಡಿ
ಮಡಿ-ಲಲ್ಲಿ
ಮಡಿ-ಲಿಗೆ
ಮಡಿ-ವಂ-ತ-ನೆಂದು
ಮಡಿ-ವಂ-ತಿಕೆ
ಮಡಿ-ಸಿ-ಬಿ-ಡಿಸಿ
ಮಣಿದು
ಮತ
ಮತ-ಗಳ
ಮತೀ-ಯರ
ಮತ್ತಷ್ಟು
ಮತ್ತಿ-ನಲ್ಲಿ
ಮತ್ತಿ-ನ್ನೇನು
ಮತ್ತು
ಮತ್ತೂ
ಮತ್ತೆ
ಮತ್ತೆಂದೂ
ಮತ್ತೆ-ಯನ್ನು
ಮತ್ತೊಂದು
ಮತ್ತೊಮ್ಮೆ
ಮದುವೆ
ಮದು-ವೆ-ಮಾ-ಡಿ-ಕೊಂಡು
ಮದು-ವೆಯ
ಮದು-ವೆ-ಯನ್ನು
ಮದು-ವೆ-ಯಾಗಿ
ಮದು-ವೆ-ಯಾ-ಗು-ವು-ದ-ರಿಂದ
ಮದು-ವೆ-ಯಾ-ಗು-ವು-ದಿಲ್ಲ
ಮದು-ವೆ-ಯಾ-ಗು-ವು-ದಿ-ಲ್ಲ-ವೆಂಬ
ಮದು-ವೆ-ಯಾ-ದ-ವ-ರ-ಲ್ಲವೆ
ಮದ್ಯ-ಪಾನ
ಮದ್ಯ-ಪಾ-ನದ
ಮಧುರ
ಮಧು-ರ-ಗಾ-ಯ-ನ-ವನ್ನು
ಮಧು-ರ-ವಾಗಿ
ಮಧು-ರ-ವಾ-ಗಿತ್ತು
ಮಧು-ರ-ವಾದ
ಮಧು-ವನ್ನೇ
ಮಧ್ಯ
ಮಧ್ಯ-ದಲ್ಲಿ
ಮಧ್ಯ-ದಿಂದ
ಮಧ್ಯ-ಭಾ-ರತ
ಮಧ್ಯಾಹ್ನ
ಮಧ್ಯಾ-ಹ್ನದ
ಮಧ್ಯೆ
ಮಧ್ಯೆ-ಮಧ್ಯೆ
ಮನಃ-ಪ-ಟ-ಲದ
ಮನಃ-ಸ್ಥಿತಿ
ಮನ-ಗಂಡ
ಮನ-ಗಂ-ಡಿದ್ದ
ಮನ-ದಟ್ಟು
ಮನ-ದ-ಣಿಯೆ
ಮನ-ದಲ್ಲಿ
ಮನ-ದೊ-ಳಗೇ
ಮನನ
ಮನ-ರಂ-ಜನೆ
ಮನ-ವ-ರಿ-ಕೆ-ಯಾ-ಗುತ್ತ
ಮನ-ವೊ-ಪ್ಪುವ
ಮನ-ವೊ-ಲಿ-ಸಿ-ಕೊ-ಳ್ಳುವ
ಮನ-ಶ್ಚಾಂ-ಚ-ಲ್ಯದ
ಮನ-ಶ್ಶಾಸ್ತ್ರ
ಮನ-ಶ್ಶಾ-ಸ್ತ್ರ-ಗ-ಳಿಗೆ
ಮನ-ಸಾರೆ
ಮನ-ಸೂ-ರೆ-ಗೊಂ-ಡಿ-ದ್ದುವು
ಮನ-ಸ್ಥಿ-ತಿ-ಯ-ಲ್ಲಿದ್ದ
ಮನ-ಸ್ಸನ್ನು
ಮನಸ್ಸಿ
ಮನ-ಸ್ಸಿಗೂ
ಮನ-ಸ್ಸಿಗೆ
ಮನ-ಸ್ಸಿನ
ಮನ-ಸ್ಸಿ-ನಲ್ಲಿ
ಮನ-ಸ್ಸಿ-ನ-ಲ್ಲಿದ್ದ
ಮನ-ಸ್ಸಿ-ನಲ್ಲೂ
ಮನ-ಸ್ಸಿ-ನಿಂದ
ಮನಸ್ಸು
ಮನ-ಸ್ಸೆಲ್ಲ
ಮನಸ್ಸೇ
ಮನುಷ್ಯ
ಮನೆ
ಮನೆ-ಕಡೆ
ಮನೆ-ಗಳಿಂದ
ಮನೆಗೆ
ಮನೆ-ಗೆ-ಲ-ಸಕ್ಕೆ
ಮನೆಗೇ
ಮನೆ-ತನ
ಮನೆ-ತ-ನದ
ಮನೆ-ತ-ನ-ದ-ವರು
ಮನೆ-ತ-ನ-ದ-ವರೆಲ್ಲ
ಮನೆ-ಬಾ-ಗಿ-ಲಿಗೆ
ಮನೆ-ಮಂದಿ
ಮನೆ-ಮಂ-ದಿಗೂ
ಮನೆ-ಮಂ-ದಿ-ಯ-ನ್ನೆಲ್ಲ
ಮನೆ-ಮಂ-ದಿಯೂ
ಮನೆ-ಮಂ-ದಿ-ಯೆಲ್ಲ
ಮನೆ-ಮಂ-ದಿ-ಯೊ-ಳಗೆ
ಮನೆಯ
ಮನೆ-ಯನ್ನು
ಮನೆ-ಯಲ್ಲಿ
ಮನೆ-ಯಲ್ಲೇ
ಮನೆ-ಯ-ಲ್ಲೊಂದು
ಮನೆ-ಯ-ಲ್ಲೊಬ್ಬ
ಮನೆ-ಯ-ವ-ರ-ನ್ನೆಲ್ಲ
ಮನೆ-ಯ-ವ-ರಾ-ರಿಗೂ
ಮನೆ-ಯ-ವ-ರಿಗೆ
ಮನೆ-ಯ-ವರು
ಮನೆ-ಯ-ವರೂ
ಮನೆ-ಯ-ವರೆ-ಲ್ಲರೂ
ಮನೆ-ಯ-ವ-ರೊಂ-ದಿಗೆ
ಮನೆ-ಯಿಂದ
ಮನೆ-ವಾ-ರ್ತೆಯ
ಮನೊ-ಭಾ-ವವೇ
ಮನೋ
ಮನೋ-ಧರ್ಮ
ಮನೋ-ಭಾವ
ಮನೋ-ಭಾ-ವ-ಇ-ವು-ಗಳಿಂದ
ಮನೋ-ಭಾ-ವದ
ಮನೋ-ಭಾ-ವ-ದವ
ಮನೋ-ಭಾ-ವ-ದ-ವರು
ಮನೋ-ರಥ
ಮನೋ-ವೃ-ತ್ತಿ-ಯೆಂ-ದರೆ
ಮನೋ-ಸಾ-ಮ್ರಾ-ಜ್ಯ-ದಲ್ಲಿ
ಮನ್ನಣೆ
ಮನ್ನ-ಣೆಗೆ
ಮನ್ನ-ಣೆ-ಯಿ-ರ-ಲಿಲ್ಲ
ಮನ್ನಿ-ಸು-ತ್ತಿ-ರ-ಲಿಲ್ಲ
ಮಮತೆ
ಮಯಿ
ಮರ
ಮರ-ಕೋ-ತಿ-ಯಾಟ
ಮರಕ್ಕೆ
ಮರ-ಗಿ-ಡ-ಬ-ಳ್ಳಿ-ಗಳು
ಮರ-ಣ-ಕಾ-ಲ-ದಲ್ಲಿ
ಮರ-ಣ-ಶ-ಯ್ಯೆ-ಯಲ್ಲಿ
ಮರದ
ಮರ-ದಿಂದ
ಮರ-ಳಿದ
ಮರ-ವನ್ನು
ಮರ-ವಿತ್ತು
ಮರ-ವೆಂದರೆ
ಮರಿ-ತು-ಟಿ-ಗಿ-ಟ್ಟು-ಕೊಂಡು
ಮರಿ-ಸಿಂಹ
ಮರು-ಕ್ಷಣ
ಮರು-ಕ್ಷ-ಣ-ದಲ್ಲೇ
ಮರು-ಕ್ಷ-ಣವೇ
ಮರು-ಗುತ್ತೀ
ಮರು-ತ್ತು-ಗಳೇ
ಮರು-ದಿನ
ಮರು-ದಿ-ನದ
ಮರು-ದಿ-ನ-ದಿಂ-ದಲೇ
ಮರು-ದಿ-ನವೇ
ಮರು-ಮಾ-ತಿ-ಲ್ಲದೆ
ಮರೆ-ತಾ-ಗಿತ್ತು
ಮರೆತು
ಮರೆ-ಯಲ್ಲಿ
ಮರೆ-ಯು-ತ್ತಿದ್ದ
ಮರೆ-ಯುವ
ಮರ್ಮ
ಮರ್ಯಾ-ದೆ-ಯನ್ನೇ
ಮಲಗಿ
ಮಲ-ಗಿ-ಕೊಂಡ
ಮಲ-ಗಿ-ಕೊಂ-ಡಾಗ
ಮಲ-ಗಿ-ಕೊ-ಳ್ಳಲು
ಮಲ-ಗಿ-ಕೊ-ಳ್ಳು-ತ್ತಿದ್ದ
ಮಲ-ಗಿ-ಕೊ-ಳ್ಳು-ವಾಗ
ಮಲ-ಗಿದ್ದ
ಮಲ-ಗಿ-ಸಿದ
ಮಲೇ-ರಿಯಾ
ಮಲ್ಲ-ಯು-ದ್ಧ-ಗಳಲ್ಲಿ
ಮಸ್ತ-ಕ-ಕ್ಕಿ-ಳಿ-ಸಿ-ಬಿ-ಡು-ತ್ತಿದ್ದ
ಮಹ-ಡಿಯ
ಮಹ-ಡಿ-ಯ-ನ್ನೇರಿ
ಮಹ-ತ್ಕಾರ್ಯ
ಮಹ-ತ್ತ-ರ-ವಾದ
ಮಹತ್ವ
ಮಹ-ತ್ವ-ಪೂರ್ಣ
ಮಹ-ದ-ಭಿ-ಲಾಷೆ
ಮಹರ್ಷಿ
ಮಹ-ರ್ಷಿ-ಗ-ಳಿಗೆ
ಮಹ-ರ್ಷಿ-ಗಳು
ಮಹಾ
ಮಹಾ-ಕಾ-ರ್ಯಕ್ಕೂ
ಮಹಾ-ಕಾ-ರ್ಯಕ್ಕೆ
ಮಹಾ-ಕಾ-ರ್ಯ-ವನ್ನು
ಮಹಾ-ಕಾ-ವ್ಯ-ಗಳನ್ನು
ಮಹಾ-ಜ್ಞಾನಿ
ಮಹಾ-ತ್ಮ-ನನ್ನೇ
ಮಹಾ-ಪು-ರು-ಷನ
ಮಹಾ-ಪು-ರು-ಷ-ರಲ್ಲಿ
ಮಹಾ-ಪ್ರ-ಚಂ-ಡ-ಆ-ದರೆ
ಮಹಾ-ಭಾ-ರ-ತ-ವನ್ನು
ಮಹಾ-ಮಾ-ಲೆಯ
ಮಹಾ-ಮೇಳ
ಮಹಾ-ರಾ-ಜನೇ
ಮಹಾ-ರಾಯ
ಮಹಾ-ಶ-ಯರೆ
ಮಹಾ-ಸಿಂಹ
ಮಹಾ-ಸಿಂ-ಹ-ದೆ-ಡೆಗೆ
ಮಹಿ-ಮೆ-ಗಳನ್ನು
ಮಹಿ-ಮೆಯ
ಮಹಿ-ಮೆ-ಯನ್ನು
ಮಹಿಳೆ
ಮಹಿ-ಳೆ-ಯರು
ಮಹಿ-ಳೆ-ಯ-ರೆಲ್ಲ
ಮಾಂಸ
ಮಾಂಸ-ಲ-ವಾದ
ಮಾಂಸಾ-ಹಾರ
ಮಾಂಸಾ-ಹಾ-ರ-ವನ್ನು
ಮಾಡ
ಮಾಡ-ತೊ-ಡಗಿ
ಮಾಡ-ತೊ-ಡ-ಗಿದ
ಮಾಡ-ತೊ-ಡ-ಗಿ-ದರು
ಮಾಡತ್ತೆ
ಮಾಡದೆ
ಮಾಡ-ಬಲ್ಲ
ಮಾಡ-ಬ-ಹುದು
ಮಾಡ-ಬೇ-ಕಲ್ಲ
ಮಾಡ-ಬೇ-ಕಾಗಿ
ಮಾಡ-ಬೇ-ಕಾ-ಗಿತ್ತು
ಮಾಡ-ಬೇ-ಕಾ-ಗಿದೆ
ಮಾಡ-ಬೇ-ಕಾ-ಗು-ತ್ತದೆ
ಮಾಡ-ಬೇ-ಕಾ-ದರೆ
ಮಾಡ-ಬೇ-ಕಾ-ಯಿತು
ಮಾಡ-ಬೇಕು
ಮಾಡ-ಬೇಕೇ
ಮಾಡ-ಬೇಡ
ಮಾಡ-ಲಾಗಿದೆ
ಮಾಡ-ಲಾ-ದೀತು
ಮಾಡ-ಲಾ-ರಂ-ಭಿ-ಸಿದ
ಮಾಡ-ಲಾ-ರಂ-ಭಿ-ಸಿದೆ
ಮಾಡಲಿ
ಮಾಡ-ಲಿಲ್ಲ
ಮಾಡಲು
ಮಾಡಲೂ
ಮಾಡಿ
ಮಾಡಿ-ಕೊಂಡ
ಮಾಡಿ-ಕೊಂ-ಡಿದ್ದ
ಮಾಡಿ-ಕೊಂಡು
ಮಾಡಿ-ಕೊಂ-ಡು-ಬಿಟ್ಟ
ಮಾಡಿ-ಕೊ-ಟ್ಟರೆ
ಮಾಡಿ-ಕೊ-ಳ್ಳ-ಬ-ಹುದು
ಮಾಡಿ-ಕೊ-ಳ್ಳ-ಬೇ-ಕಾ-ಯಿ-ತಂತೆ
ಮಾಡಿ-ಕೊ-ಳ್ಳ-ಲೇ-ಬೇಕು
ಮಾಡಿ-ಕೊ-ಳ್ಳು-ತ್ತಿದ್ದ
ಮಾಡಿ-ಕೊ-ಳ್ಳುವ
ಮಾಡಿ-ಕೊ-ಳ್ಳು-ವಂ-ತಾಗ
ಮಾಡಿ-ಕೊ-ಳ್ಳು-ವಂತೆ
ಮಾಡಿ-ಕೊ-ಳ್ಳು-ವುದು
ಮಾಡಿಟ್ಟು
ಮಾಡಿ-ಟ್ಟು-ಹೋಗು
ಮಾಡಿದ
ಮಾಡಿ-ದರು
ಮಾಡಿ-ದರೂ
ಮಾಡಿ-ದರೆ
ಮಾಡಿ-ದ-ವ-ರಲ್ಲ
ಮಾಡಿ-ದಷ್ಟೂ
ಮಾಡಿ-ದಾಗ
ಮಾಡಿ-ದಾ-ಗ-ಲೆಲ್ಲ
ಮಾಡಿ-ದಾ-ಗಲೇ
ಮಾಡಿದೆ
ಮಾಡಿದ್ದ
ಮಾಡಿ-ದ್ದರೂ
ಮಾಡಿ-ದ್ದ-ಲ್ಲದೆ
ಮಾಡಿ-ದ್ದಾ-ರಲ್ಲ
ಮಾಡಿದ್ದೀ
ಮಾಡಿದ್ದು
ಮಾಡಿ-ದ್ದುವು
ಮಾಡಿ-ಬಿ-ಟ್ಟಿ-ದ್ದಳು
ಮಾಡಿ-ಬಿ-ಡ-ಬೇಕು
ಮಾಡಿ-ಬಿ-ಡಲು
ಮಾಡಿ-ಬಿಡಿ
ಮಾಡಿ-ಬಿ-ಡು-ತ್ತಿದ್ದ
ಮಾಡಿ-ಬಿ-ಡು-ತ್ತೇವೆ
ಮಾಡಿಯೇ
ಮಾಡಿ-ರ-ಬೇಕು
ಮಾಡಿಲ್ಲ
ಮಾಡಿ-ಲ್ಲವೋ
ಮಾಡಿ-ಸಲು
ಮಾಡಿಸಿ
ಮಾಡಿ-ಸಿ-ಕೊ-ಡ-ಬ-ಲ್ಲ-ವ-ರನ್ನು
ಮಾಡಿ-ಸಿ-ಕೊಳ್ಳಿ
ಮಾಡಿ-ಸಿ-ದರು
ಮಾಡಿ-ಸಿ-ದಳು
ಮಾಡಿ-ಸಿ-ದ್ದಳು
ಮಾಡಿ-ಸು-ತ್ತಿದ್ದ
ಮಾಡು
ಮಾಡುತ್ತ
ಮಾಡು-ತ್ತಲೇ
ಮಾಡು-ತ್ತಾ-ನೆ-ಎಲ್ಲ
ಮಾಡು-ತ್ತಿತ್ತು
ಮಾಡು-ತ್ತಿದ್ದ
ಮಾಡು-ತ್ತಿ-ದ್ದರೂ
ಮಾಡು-ತ್ತಿ-ದ್ದರೆ
ಮಾಡು-ತ್ತಿ-ದ್ದಳು
ಮಾಡು-ತ್ತಿ-ದ್ದಾ-ನಲ್ಲ
ಮಾಡು-ತ್ತಿ-ದ್ದಾರೆ
ಮಾಡು-ತ್ತಿ-ದ್ದೀಯಾ
ಮಾಡು-ತ್ತಿ-ದ್ದು-ದ-ಕ್ಕಾಗಿ
ಮಾಡು-ತ್ತಿದ್ದೆ
ಮಾಡು-ತ್ತಿ-ದ್ದೆವು
ಮಾಡು-ತ್ತಿರು
ಮಾಡು-ತ್ತಿ-ರುವ
ಮಾಡು-ತ್ತಿ-ರು-ವಂತೆ
ಮಾಡು-ತ್ತಿ-ರು-ವಾಗ
ಮಾಡು-ತ್ತಿ-ರು-ವಾ-ಗಲೂ
ಮಾಡು-ತ್ತಿ-ರು-ವಾ-ಗಲೇ
ಮಾಡು-ತ್ತಿ-ರು-ವುದನ್ನು
ಮಾಡು-ತ್ತಿ-ರು-ವುದು
ಮಾಡು-ತ್ತೀಯ
ಮಾಡು-ತ್ತೇನೆ
ಮಾಡುವ
ಮಾಡು-ವಂತೆ
ಮಾಡು-ವಾಗ
ಮಾಡು-ವು-ದ-ಕ್ಕಾಗಿ
ಮಾಡು-ವು-ದಕ್ಕೂ
ಮಾಡು-ವು-ದಕ್ಕೆ
ಮಾಡು-ವುದನ್ನು
ಮಾಡು-ವು-ದೀಗ
ಮಾಡು-ವುದು
ಮಾಡು-ವು-ದೆಂ-ದರೆ
ಮಾಡು-ವು-ದೆಂದು
ಮಾಡೋಣ
ಮಾತ-ನಾ-ಡ-ಬಲ್ಲ
ಮಾತ-ನಾ-ಡ-ಬಲ್ಲೆ
ಮಾತ-ನಾ-ಡ-ಬೇಕು
ಮಾತ-ನಾ-ಡಲು
ಮಾತ-ನಾ-ಡಿ-ಕೊ-ಳ್ಳು-ವುದೂ
ಮಾತ-ನಾ-ಡಿ-ಯಾನು
ಮಾತ-ನಾ-ಡಿ-ಸಲೇ
ಮಾತ-ನಾ-ಡಿ-ಸುತ್ತ
ಮಾತ-ನಾ-ಡಿ-ಸು-ತ್ತಿದ್ದ
ಮಾತ-ನಾ-ಡುತ್ತ
ಮಾತ-ನಾ-ಡು-ತ್ತಲೇ
ಮಾತ-ನಾ-ಡು-ತ್ತಿದ್ದ
ಮಾತ-ನಾ-ಡು-ತ್ತಿ-ರ-ಲಿಲ್ಲ
ಮಾತ-ನಾ-ಡು-ವಾಗ
ಮಾತ-ನಾ-ಡು-ವು-ದಕ್ಕೆ
ಮಾತ-ನಾ-ಡು-ವುದೇ
ಮಾತನ್ನು
ಮಾತಾಡಿ
ಮಾತಾ-ಡಿ-ಕೊಂ-ಡರು
ಮಾತಿಗೂ
ಮಾತಿದೆ
ಮಾತಿನ
ಮಾತಿ-ನಲ್ಲಿ
ಮಾತಿ-ನ-ವ-ಳಲ್ಲ
ಮಾತಿ-ನಿಂದ
ಮಾತು
ಮಾತು-ಗ-ಳ-ನ್ನಾ-ಡಿ-ದರೆ
ಮಾತು-ಗಳನ್ನು
ಮಾತು-ಗ-ಳಿಗೂ
ಮಾತೂ
ಮಾತೃ
ಮಾತೃ-ಬಾಷೆ
ಮಾತೃ-ಭೂ-ಮಿಯ
ಮಾತೆಯ
ಮಾತೆ-ಯರ
ಮಾತೋ
ಮಾತ್ರ
ಮಾತ್ರಕ್ಕೆ
ಮಾತ್ರ-ವಲ್ಲ
ಮಾತ್ರ-ವ-ಲ್ಲದೆ
ಮಾಧುರ್ಯ
ಮಾನ-ವನ
ಮಾನ-ಸಿಕ
ಮಾಫಿ
ಮಾಮ
ಮಾಮನ
ಮಾಮಾ
ಮಾಯ-ವಾ-ಗಿ-ಬಿ-ಟ್ಟಿ-ರು-ತ್ತಿ-ದ್ದುವು
ಮಾಯಾ
ಮಾಯಾ-ಲಾಂ-ದ್ರ-ದ-ಸ್ಲೈಡ್
ಮಾಯೆ
ಮಾರುತ
ಮಾರು-ಹೋ-ಗಿತ್ತು
ಮಾರ್ಗ-ಗಳು
ಮಾರ್ಗ-ದ-ರ್ಶ-ನ-ದಲ್ಲಿ
ಮಾರ್ಗ-ದಲ್ಲಿ
ಮಾರ್ಗ-ವಾಗಿ
ಮಾಲಿ-ಕ-ನಾದ
ಮಾಸ-ಲಾ-ಯಿತು
ಮಾಸ್ತ-ರರ
ಮಾಸ್ತ-ರಿಗೆ
ಮಾಸ್ತರು
ಮಾಹಿತಿ
ಮಿಂಚಿನ
ಮಿಂಚಿ-ನಂತೆ
ಮಿಗಿ-ಲಾಗಿ
ಮಿಗು-ತ್ತಿತ್ತು
ಮಿಠಾಯಿ
ಮಿಡಿ-ತ-ವೆ-ನ್ನು-ವುದು
ಮಿತಿ-ಮೀ-ರು-ತ್ತಿತ್ತು
ಮಿತ್ರ
ಮಿತ್ರನ
ಮಿತ್ರ-ನಿಗೂ
ಮಿಶ್ರ
ಮಿಶ್ರ-ಣ-ದಿಂ-ದಾಗಿ
ಮಿಷ-ನರಿ
ಮಿಸು-ಕಲೇ
ಮೀನಿನ
ಮೀನು
ಮೀನು-ಮಾಂಸ
ಮೀರ-ದಿ-ರು-ವುದನ್ನು
ಮೀರಿ
ಮೀರಿತು
ಮೀರಿದ
ಮೀರಿ-ಸಿ-ಬಿಟ್ಟ
ಮೀರಿ-ಸುವ
ಮೀರಿ-ಸು-ವ-ವರು
ಮೀಸ-ಲಾದ
ಮುಂಜಾ-ನೆ-ಯಲ್ಲಿ
ಮುಂಜಾ-ವಿ-ನಲ್ಲೇ
ಮುಂದಕ್ಕೆ
ಮುಂದಾ-ಗ-ದಿ-ದ್ದು-ದನ್ನು
ಮುಂದಾ-ಗ-ಲಿಲ್ಲ
ಮುಂದಾ-ಗು-ತ್ತಿದ್ದ
ಮುಂದಾದ
ಮುಂದಾ-ದು-ದ-ರಿಂ-ದ-ಲಾ-ದರೂ
ಮುಂದಾ-ಳಾ-ಗಿಯೇ
ಮುಂದಾಳು
ಮುಂದಿಟ್ಟ
ಮುಂದಿಟ್ಟು
ಮುಂದಿನ
ಮುಂದಿ-ನಿಂದ
ಮುಂದು
ಮುಂದು-ಮುಂ-ದಕ್ಕೆ
ಮುಂದು-ವ-ರಿ-ಕೆಯೇ
ಮುಂದು-ವ-ರಿ-ದಳು
ಮುಂದು-ವ-ರಿ-ಯ-ಬೇ-ಕಾ-ಗು-ತ್ತದೆ
ಮುಂದು-ವ-ರಿ-ಯಿತು
ಮುಂದು-ವ-ರಿ-ಯು-ವಂತೆ
ಮುಂದು-ವ-ರಿಸಿ
ಮುಂದು-ವ-ರಿ-ಸಿ-ಕೊಂಡು
ಮುಂದು-ವ-ರಿ-ಸು-ತ್ತಿದ್ದ
ಮುಂದೆ
ಮುಂದೊಮ್ಮೆ
ಮುಂಬೆ-ಳ-ಕನ್ನು
ಮುಕ್ತ-ಕಂ-ಠ-ದಿಂದ
ಮುಕ್ತ-ನಾ-ಗಿ-ರ-ಬೇ-ಕೆಂಬ
ಮುಖ
ಮುಖಂಡ
ಮುಖಂ-ಡ-ನಂತೆ
ಮುಖಂ-ಡ-ನಾ-ಗಿದ್ದ
ಮುಖಂ-ಡರ
ಮುಖಂ-ಡರು
ಮುಖದ
ಮುಖ-ದಲ್ಲಿ
ಮುಖ-ಭಾವ
ಮುಖ-ವನ್ನೇ
ಮುಖ-ವಾಡ
ಮುಖ್ಯ
ಮುಖ್ಯ-ಗು-ಮಾಸ್ತೆ
ಮುಖ್ಯ-ವಾಗಿ
ಮುಖ್ಯ-ವಾದ
ಮುಖ್ಯಾಂ-ಶ-ವನ್ನು
ಮುಗಿದ
ಮುಗಿ-ದ-ಹಾ-ಗೆಯೇ
ಮುಗಿ-ದು-ಹೋ-ಗಿ-ದ್ದುವು
ಮುಗಿ-ಯಿತು
ಮುಗಿ-ಯುವ
ಮುಗಿ-ಯು-ವು-ದ-ರೊ-ಳಗೆ
ಮುಗಿ-ಸದೆ
ಮುಗಿ-ಸ-ಬೇಕು
ಮುಗಿ-ಸಲು
ಮುಗಿಸಿ
ಮುಗಿ-ಸಿ-ಕೊಂಡು
ಮುಗಿ-ಸಿದ
ಮುಗಿ-ಸಿದೆ
ಮುಗಿ-ಸಿ-ಯೂ-ಬಿಟ್ಟ
ಮುಗು-ಳ್ನ-ಗುತ್ತ
ಮುಗ್ಧ
ಮುಗ್ಧ-ಭಾ-ವ-ದಿಂದ
ಮುಚ್ಚಿ
ಮುಚ್ಚಿ-ಕೊಂಡು
ಮುಚ್ಚಿಸಿ
ಮುಚ್ಚಿ-ಸಿ-ಬಿ-ಡು-ತ್ತಿದ್ದ
ಮುಚ್ಚು-ವಂ-ತಹ
ಮುಟ್ಟದ
ಮುಟ್ಟಿ
ಮುಡಿ-ಪಾ-ಗಿ-ಟ್ಟಿ-ರು-ವು-ದ-ರಿಂದ
ಮುತ್ತ-ಜ್ಜಿಯೂ
ಮುತ್ತಾ-ತ-ರಾ-ದಿ-ಯಾಗಿ
ಮುದಿ
ಮುದಿ-ಕಣ್ಣು
ಮುದುಕ
ಮುದು-ಕನ
ಮುದು-ಕಪ್ಪ
ಮುದು-ಡಿ-ಹೋ-ಗು-ತ್ತಿತ್ತು
ಮುದ್ದಾದ
ಮುದ್ದಿಗೆ
ಮುದ್ದಿನ
ಮುದ್ದು
ಮುದ್ರೆ-ಇವು
ಮುದ್ರೆ-ಯ-ನ್ನೊ-ತ್ತಿ-ದುವು
ಮುನ್ನ-ಡೆದ
ಮುನ್ನ-ಡೆದು
ಮುನ್ನ-ಡೆ-ಸುವ
ಮುನ್ನು-ಗ್ಗಲು
ಮುನ್ನು-ಗ್ಗಿದ
ಮುನ್ನು-ಗ್ಗು-ವುದು
ಮುನ್ನುಡಿ
ಮುರಿದ
ಮುರಿ-ದಂತೆ
ಮುರಿ-ದರೆ
ಮುರಿದು
ಮುರಿ-ದು-ಕೊಂಡ
ಮುರಿ-ದು-ಕೊಂಡು
ಮುರಿ-ದು-ಬಿ-ದ್ದರೂ
ಮುರು-ಕಲು
ಮುಳುಗಿ
ಮುಳು-ಗಿ-ದಳು
ಮುಳು-ಗಿ-ದ್ದರೂ
ಮುಳು-ಗಿ-ಸಿ-ಬಿ-ಡು-ತ್ತಿದ್ದ
ಮುಷ್ಟಾ-ಮುಷ್ಟಿ
ಮುಷ್ಟಿ-ಯು-ದ್ಧ-ದಲ್ಲಿ
ಮುಸ-ಲ್ಮಾನ
ಮುಸ-ಲ್ಮಾ-ನ-ನನ್ನು
ಮುಸ-ಲ್ಮಾ-ನನೂ
ಮುಸ-ಲ್ಮಾ-ನರ
ಮುಸ-ಲ್ಮಾ-ನರೂ
ಮುಸು-ಕನ್ನು
ಮುಸು-ಕಿ-ನಲ್ಲಿ
ಮುಸ್ಲಿ-ಮರ
ಮುಹೂರ್ತ
ಮೂಡಿ
ಮೂಡಿತು
ಮೂಡಿ-ಬ-ರ-ತೊ-ಡ-ಗಿತ್ತು
ಮೂಡಿ-ಬ-ರ-ಬೇಕೇ
ಮೂಡಿ-ಬರು
ಮೂಡಿ-ಬ-ರು-ತ್ತಿ-ರುವ
ಮೂಢ
ಮೂಢ-ನಂ-ಬಿ-ಕೆ-ಗಳನ್ನೂ
ಮೂರ-ನೆ-ಯ-ವಳು
ಮೂರನೇ
ಮೂರು
ಮೂರೂ
ಮೂರೇ
ಮೂರ್ತಿ
ಮೂರ್ತಿ-ಪೂ-ಜೆ-ಯನ್ನು
ಮೂರ್ತಿ-ಪೂ-ಜೆ-ಯಲ್ಲಿ
ಮೂರ್ತಿಯೇ
ಮೂರ್ನಾಲ್ಕು
ಮೂಲ
ಮೂಲಕ
ಮೂಲ-ಕವೇ
ಮೂಲತಃ
ಮೂಲೆ-ಗಳೇ
ಮೂಲೆ-ಮೂ-ಲೆ-ಯ-ನ್ನೆಲ್ಲ
ಮೂಲೋ-ದ್ದೇಶ
ಮೃತ-ಪ್ರಾ-ಯ-ವಾ-ಗಿ-ದ್ದರೆ
ಮೃತ್ಯು-ಮು-ಖ-ದಿಂದ
ಮೃತ್ಯು-ಸ್ವ-ರೂ-ಪ-ವಾದ
ಮೃದು-ಲ-ತರ
ಮೆಚ್ಚಿ
ಮೆಚ್ಚಿ-ಕೊಂಡ
ಮೆಚ್ಚಿ-ಕೊಂ-ಡಾ-ರೇನು
ಮೆಚ್ಚಿದ್ದ
ಮೆಚ್ಚುಗೆ
ಮೆಚ್ಚು-ಗೆ-ಯಾ-ದುವು
ಮೆಟ್ಟಿ
ಮೆಟ್ಟಿ-ನಿಂ-ತಿತು
ಮೆಟ್ಟಿ-ನಿಂತು
ಮೆಟ್ಟಿಲ
ಮೆಟ್ಟಿಲು
ಮೆಟ್ಟಿ-ಲು-ಗಳ
ಮೆಟ್ಟಿ-ಲು-ಗಳನ್ನು
ಮೆಟ್ಟಿ-ಲೇರಿ
ಮೆಟ್ಟಿ-ಲೇ-ರಿ-ದ್ದಾನೆ
ಮೆಟ್ರೊ
ಮೆಟ್ರೊ-ಪಾ-ಲಿ-ಟನ್
ಮೆಟ್ರೋ
ಮೆಟ್ರೋ-ಪಾ-ಲಿ-ಟನ್
ಮೆಣ-ಸಿ-ನ-ಕಾ-ಯಿ-ಯನ್ನು
ಮೆದು-ಳಿಗೆ
ಮೆರ-ವ-ಣಿ-ಗೆ-ಯಲ್ಲಿ
ಮೆರೆ-ದಾ-ಡುತ್ತ
ಮೆರೆ-ಸಿ-ಯಾರು
ಮೆಲು-ದ-ನಿ-ಯಲ್ಲಿ
ಮೆಲ್ಲಗೆ
ಮೆಲ್ಲನೆ
ಮೇಧಾವಿ
ಮೇರೆಗೆ
ಮೇಲಕ್ಕೆ
ಮೇಲ-ಕ್ಕೆತ್ತಿ
ಮೇಲ-ಕ್ಕೆ-ತ್ತಿ-ಬಿ-ಟ್ಟರು
ಮೇಲಾ-ಗಿದ್ದ
ಮೇಲಿ-ಟ್ಟಿದ್ದ
ಮೇಲಿನ
ಮೇಲಿ-ನಿಂದ
ಮೇಲಿ-ರು-ವಂ-ತಹ
ಮೇಲೂ
ಮೇಲೆ
ಮೇಲೆ-ಇ-ಲ್ಲ-ಎ-ನ್ನು-ವು-ದ-ಕ್ಕಾ-ಗು-ತ್ತ-ದೆಯೆ
ಮೇಲೆಯೇ
ಮೇಲೆ-ರಗಿ
ಮೇಲೆಲ್ಲ
ಮೇಲೇ
ಮೇಲೇರಿ
ಮೇಲೇ-ರಿದ
ಮೇಲೊಂದು
ಮೇಲ್ಗಡೆ
ಮೇಲ್ನೋ-ಟಕ್ಕೆ
ಮೇಲ್ಭಾ-ಗ-ದಲ್ಲಿ
ಮೈ
ಮೈಕ-ಟ್ಟನ್ನು
ಮೈಕಟ್ಟು
ಮೈಕಾಂತಿ
ಮೈಗೂ
ಮೈಗೂ-ಡಿ-ಸಿ-ಕೊ-ಳ್ಳ-ಬೇಕು
ಮೈಗೂ-ಡಿ-ಸಿ-ಕೊ-ಳ್ಳು-ವಂ-ತಾ-ಗಲಿ
ಮೈಗೆ
ಮೈತಿ-ಳಿದು
ಮೈತಿ-ಳಿ-ದೆ-ದ್ದಾಗ
ಮೈತುಂಬ
ಮೈದ-ಳೆದು
ಮೈದಾನ
ಮೈದೋ-ರಿತ್ತು
ಮೈಮ-ರೆ-ತದ್ದು
ಮೈಮ-ರೆತು
ಮೈಮು-ಟ್ಟಿ-ದಿರೋ
ಮೈಮೇಲೆ
ಮೈಮೇ-ಲೇನೂ
ಮೈಯನ್ನು
ಮೈಲಿ
ಮೈಲಿ-ಗೆಯ
ಮೈಸೂರು
ಮೊಂಡಾಟ
ಮೊಂಡು-ಬು-ದ್ಧಿಯ
ಮೊಂಡು-ವಾ-ದ-ವಲ್ಲ
ಮೊಗ್ಗೇ
ಮೊತ್ತ-ಮೊ-ದ-ಲ-ನೆ-ಯ-ದಾಗಿ
ಮೊದ-ಮೊ-ದಲು
ಮೊದಲ
ಮೊದ-ಲ-ನೆಯ
ಮೊದ-ಲ-ನೆ-ಯ-ದಾಗಿ
ಮೊದ-ಲಾದ
ಮೊದಲಿ
ಮೊದ-ಲಿಗ
ಮೊದ-ಲಿ-ಗ-ನಾಗಿ
ಮೊದ-ಲಿ-ಗ-ನಾದ
ಮೊದ-ಲಿನ
ಮೊದ-ಲಿ-ನಿಂಲೂ
ಮೊದಲು
ಮೊದಲೇ
ಮೊನೆ-ಯುಳ್ಳ
ಮೊಳಕೆ
ಮೊಳ-ಕೆ-ಯಲ್ಲಿ
ಮೊಳ-ಗ-ಬೇ-ಕಾ-ಗಿ-ದೆ-ಯ-ಲ್ಲವೆ
ಮೊಳ-ಗಿತು
ಮೊಳ-ಗಿ-ದ್ದಾನೆ
ಮೊಳ-ಗಿಸಿ
ಮೋಜೂ
ಮೋಡ-ಗಳೇ
ಮೋರಿ-ಯೊ-ಳಗೆ
ಮೋರೆ
ಮೋಸ
ಮೌಢ್ಯ-ವೆಂದು
ಮೌನ-ದಲ್ಲಿ
ಯಂತಿದ್ದ
ಯಂತೆ
ಯಂತ್ರೋ-ಪ-ಕ-ರ-ಣ-ಗಳನ್ನೆಲ್ಲ
ಯಜ-ಮಾನ
ಯಜ-ಮಾ-ನ-ನಾದ
ಯಣದ
ಯಥಾ
ಯನ್ನು
ಯಲ್ಲಿ
ಯಲ್ಲಿ-ದ್ದಾ-ಗಲೇ
ಯಲ್ಲೇ
ಯವ-ನಾ-ದರೂ
ಯಶ-ಸ್ವಿ-ಯಾಗಿ
ಯಶ-ಸ್ವಿಯೂ
ಯಶಸ್ಸು
ಯಶ್ವಸೀ
ಯಾ
ಯಾಕಾ-ಗ-ಬಾ-ರದು
ಯಾಕಿ-ಲ್ಲಿಗೆ
ಯಾಕೆ
ಯಾಗ-ಲಾರ
ಯಾಗಿ-ದ್ದರೆ
ಯಾಗಿ-ರು-ತ್ತಿತ್ತು
ಯಾಗು-ತ್ತಿ-ರು-ವುದನ್ನು
ಯಾತ್ರಿಕ
ಯಾದ
ಯಾದಳು
ಯಾದ-ವ-ಗಿರಿ
ಯಾಯಿತು
ಯಾರ
ಯಾರ-ಕೈ-ಲಾ-ದರೂ
ಯಾರ-ದ್ದಪ್ಪಾ
ಯಾರ-ನ್ನಾ-ದರೂ
ಯಾರನ್ನೂ
ಯಾರಲ್ಲಿ
ಯಾರಾ-ದರೂ
ಯಾರಾ-ದ-ರೊಬ್ಬ
ಯಾರಾ-ದ-ರೊ-ಬ್ಬ-ರಿಂದ
ಯಾರಾ-ದ-ರೊ-ಬ್ಬರು
ಯಾರಿಂ-ದಲೂ
ಯಾರಿ-ಗಿ-ರು-ತ್ತದೆ
ಯಾರಿಗೂ
ಯಾರಿಗೆ
ಯಾರಿ-ದ್ದಾರೆ
ಯಾರು
ಯಾರು-ತಾನೆ
ಯಾರೂ
ಯಾರೇ
ಯಾರೊ-ಬ್ಬರೂ
ಯಾರೋ
ಯಾವ
ಯಾವ-ಗಲೂ
ಯಾವನು
ಯಾವನೋ
ಯಾವ-ಯಾವ
ಯಾವಾ
ಯಾವಾಗ
ಯಾವಾ-ಗ-ಲಾ-ದ-ರೊಮ್ಮೆ
ಯಾವಾ-ಗಲೂ
ಯಾವು-ದಕ್ಕೂ
ಯಾವು-ದ-ನ್ನಾ-ದರೂ
ಯಾವು-ದನ್ನೂ
ಯಾವು-ದಾ-ದರೂ
ಯಾವು-ದಾ-ದ-ರೊಂದು
ಯಾವುದು
ಯಾವು-ದೆಂ-ದರೆ
ಯಾವುದೇ
ಯಾವುದೋ
ಯಿಂದ
ಯಿಂದಿ-ರುವ
ಯಿತು
ಯಿತೋ
ಯುಂಟಾ-ಗಿದೆ
ಯುಕ್ತಿ
ಯುದ್ದಕ್ಕೂ
ಯುದ್ಧ
ಯುದ್ಧದ
ಯುದ್ಧ-ನೌ-ಕೆ-ಯನ್ನು
ಯುವ
ಯುವಕ
ಯುವ-ಕ-ಯು-ವ-ತಿ-ಯರು
ಯುವ-ಕನೂ
ಯುವ-ಕ-ರಂ-ತಲ್ಲ
ಯುವ-ಕ-ರಲ್ಲಿ
ಯುವ-ಕರು
ಯುವ-ಜ-ನರ
ಯುವ-ಜ-ನರು
ಯುವತಿ
ಯೊಂದಿಗೆ
ಯೋಗ
ಯೋಗ-ಕ್ಷೇಮ
ಯೋಗಿಯ
ಯೋಗ್ಯ
ಯೋಗ್ಯವೂ
ಯೋಚ-ನೆಯೇ
ಯೋಚಿ-ಸಿದ
ಯೋಚಿ-ಸಿದ್ದ
ಯೋಚಿ-ಸು-ತ್ತಿದ್ದ
ಯೋಚಿ-ಸು-ತ್ತಿ-ದ್ದಂತೆ
ಯೋಧ
ಯೋಧ-ರಂತೆ
ಯೋಧ-ರಾ-ಗ-ಬೇಕು
ಯೌವನ
ಯೌವ-ನದ
ಯೌವ-ನ-ದಲ್ಲಿ
ರ
ರಂಗಕ್ಕೆ
ರಂಗ-ಮಂ-ಟ-ಪ-ದ-ಲ್ಲಿ-ರು-ವುದನ್ನು
ರಂಗ-ಮಂ-ಟ-ಪ-ವಾ-ಯಿತು
ರಂಗು-ರಂ-ಗಾಗಿ
ರಂಜ-ನೀಯ
ರಂಜಿ-ಸು-ತ್ತಿದ್ದ
ರಂಪ
ರಂಪಾಟ
ರಕ್ತ
ರಕ್ತ-ಗ-ತ-ವಾಗಿ
ರಕ್ತ-ವನ್ನು
ರಕ್ಷಿಸಿ
ರಕ್ಷಿ-ಸಿ-ಕೊ-ಳ್ಳ-ಬೇಕು
ರಚಿಸು
ರಚಿ-ಸು-ತ್ತಿ-ದ್ದಂ-ತಿ-ತ್ತು-ಒಬ್ಬ
ರಜ-ಪು-ತಾನ
ರಣ-ರಂ-ಗ-ದ-ಲ್ಲಿಯೂ
ರತ್ನ-ದಂ-ಥವು
ರತ್ನ-ವಾದ
ರಪ-ರ-ಪನೆ
ರಭ-ಸಕ್ಕೆ
ರಭ-ಸ-ದಲ್ಲಿ
ರಮಿ-ಸು-ತ್ತಿದ್ದ
ರಲ್ಲದೆ
ರಲ್ಲಿ
ರಲ್ಲೊಂದು
ರಸ
ರಸಜ್ಞ
ರಸ-ದೌ-ತ-ಣ-ವಾ-ಗು-ತ್ತಿತ್ತು
ರಸ-ಭ-ರಿ-ತ-ವಾದ
ರಸ-ವ-ತ್ತಾಗಿ
ರಸ್ತೆ
ರಸ್ತೆ-ಗಳಲ್ಲಿ
ರಸ್ತೆಯ
ರಸ್ತೆ-ಯ-ವ-ರೆಗೆ
ರಸ್ತೆ-ಯಾ-ದರೂ
ರಸ್ತೆ-ಯು-ದ್ದಕ್ಕೂ
ರಹ-ಸ್ಯ-ವಾಗಿ
ರಾಕ್ಷ-ಸ-ವಿ-ದೆ-ಯಪ್ಪ
ರಾಖಾ-ಲನೂ
ರಾಗದ
ರಾಗಿ
ರಾಗಿ-ದ್ದರು
ರಾಜ
ರಾಜ-ಕೀಯ
ರಾಜ-ಕು-ಮಾ-ರನ
ರಾಜ-ಕು-ಮಾ-ರ-ನಂತೆ
ರಾಜ-ಧಾನಿ
ರಾಜನ
ರಾಜ-ನಾಗಿ
ರಾಜ-ನಾ-ಗು-ತ್ತೇನೆ
ರಾಜ-ಮ-ಹಾ-ರಾ-ಜ-ರು-ಗಳನ್ನೂ
ರಾಜ-ಯ-ತಿ-ರಾಜ
ರಾಜರು
ರಾಜಾಧಿ
ರಾಜಾ-ಧಿ-ರಾ-ಜ-ನಿಗೆ
ರಾಜ್ಕು-ಮಾರ
ರಾಜ್ಕು-ಮಾ-ರನ
ರಾಜ್ಕು-ಮಾ-ರ-ನಿಗೆ
ರಾಜ್ಕು-ಮಾರ್
ರಾಜ್ಯದ
ರಾತ್ರಿ
ರಾತ್ರಿ-ಯನ್ನು
ರಾತ್ರಿ-ಯಾ-ಯಿತು
ರಾತ್ರಿ-ಯಿಡೀ
ರಾತ್ರಿಯೂ
ರಾತ್ರಿ-ಯೆಲ್ಲ
ರಾದ
ರಾಮ
ರಾಮ-ಸೀ-ತೆ-ಯರ
ರಾಮ-ಕೃಷ್ಣ
ರಾಮ-ಕೃ-ಷ್ಣರ
ರಾಮ-ಕೃ-ಷ್ಣರು
ರಾಮ-ಚಂದ್ರ
ರಾಮ-ಚಂ-ದ್ರ-ನಿಗೆ
ರಾಮನ
ರಾಮ-ನನ್ನು
ರಾಮ-ಮೋ-ಹನ
ರಾಮಾ
ರಾಮಾ-ಯಣ
ರಾಮಾ-ಯ-ಣ-ಮ-ಹಾ-ಭಾ-ರ-ತ-ಗಳ
ರಾಮಾ-ಯ-ಣದ
ರಾಮಾ-ಯ-ಣ-ವನ್ನು
ರಾಮ್ಖಾ-ಡಿ-ನ-ರೇಂ-ದ್ರನ
ರಾಯ-ಪುರ
ರಾಯ-ಪು-ರಕ್ಕೆ
ರಾಯ-ಪು-ರ-ಗ-ಳಿ-ಗೆಲ್ಲ
ರಾಯ-ಪು-ರ-ದಲ್ಲಿ
ರಾಯ-ಪು-ರ-ದ-ಲ್ಲಿದ್ದ
ರಾಯ್
ರಾಷ್ಟ್ರಕ್ಕೇ
ರಾಷ್ಟ್ರ-ಗಳಲ್ಲಿ
ರಾಷ್ಟ್ರದ
ರಾಷ್ಟ್ರ-ನಾ-ಡಿ-ಯಲ್ಲಿ
ರಾಷ್ಟ್ರ-ಪ್ರಜ್ಞೆ
ರಾಷ್ಟ್ರ-ಪ್ರೇ-ಮದ
ರಾಷ್ಟ್ರ-ಪ್ರೇ-ಮಿ-ಯಾ-ಗ-ಬೇ-ಕಾ-ದ-ವನು
ರಾಷ್ಟ್ರ-ಭ-ಕ್ತಿಯ
ರಾಷ್ಟ್ರ-ಹಿತ
ರಾಷ್ಟ್ರೀಯ
ರಿಂದ
ರಿಗೂ
ರಿಗೆ
ರಿಯಾ-ಯಿತಿ
ರೀತಿ
ರೀತಿ-ರ-ಹ-ಸ್ಯ-ಗಳನ್ನೂ
ರೀತಿಯ
ರೀತಿ-ಯಲ್ಲಿ
ರೀತಿಯೇ
ರುಚಿ-ಕ-ರ-ವಾದ
ರುಚಿ-ಯನ್ನು
ರುಚಿ-ಯಾ-ಗಿಯೇ
ರುಜು
ರುವು-ದಕ್ಕೂ
ರುವುದನ್ನು
ರೂಢಿಗೆ
ರೂಢಿ-ಗೊ-ಳಿ-ಸು-ವು-ದ-ಕ್ಕಾಗಿ
ರೂಢಿ-ಯಾ-ಗಿತ್ತು
ರೂಢಿ-ಸಿ-ಕೊಂಡ
ರೂಪ
ರೂಪಕ್ಕೆ
ರೂಪ-ಗಳನ್ನು
ರೂಪ-ಲಾ-ವ-ಣ್ಯ-ವನ್ನು
ರೂಪ-ವಾದ
ರೂಪಿ-ಸಲು
ರೂಪಿ-ಸಿ-ಕೊಂ-ಡಿ-ದ್ದರು
ರೂಪಿ-ಸಿ-ದರು
ರೂಪಿ-ಸಿ-ದಳು
ರೆನ್ನುವ
ರೆಲ್ಲ
ರೇಖಾ-ಗ-ಣಿ-ತದ
ರೇಖಾ-ಗ-ಣಿ-ತ-ವನ್ನು
ರೇಖೆ
ರೇಗಾ-ಡು-ವುದನ್ನು
ರೇಗಿ
ರೇಗಿತು
ರೇಗಿ-ಸು-ತ್ತಿ-ದ್ದುದೂ
ರೇನು
ರೈಲು-ದಾರಿ
ರೈಲು-ಬಂಡಿ
ರೊಂದು
ರೊಬ್ಬರು
ರೊಯ್ಯನೆ
ರೊಳಕ್ಕೆ
ರೋಗ-ದಿಂ-ದಾಗಿ
ರೋಮಾಂ-ಚಕ
ರೋಮಾಂ-ಚ-ಕಾರಿ
ರೋಮಾಂ-ಚ-ಕಾ-ರಿ-ಯಾ-ಯಿತು
ರೋಮಾಂ-ಚ-ಕಾರೀ
ಲಂಗು-ಲ-ಗಾ-ಮಿ-ಲ್ಲದ
ಲಕ್ನೋ
ಲಕ್ಷ-ಣ-ಗ-ಳಿವೆ
ಲಕ್ಷ-ಣ-ವೇನು
ಲಕ್ಷ್ಮಿ-ಸ-ರ-ಸ್ವತಿ
ಲಭಿ-ಸ-ಲಿಲ್ಲ
ಲಭ್ಯ-ವಾ-ಯಿತು
ಲಭ್ಯ-ವಿ-ದ್ದು-ವು-ಯಾ-ರೆಂ-ದ-ರ-ವ-ರಿ-ಗಲ್ಲ
ಲಯ-ಬ-ದ್ಧ-ವಾಗಿ
ಲಾಠಿ
ಲಾಠಿ-ಗಳೂ
ಲಾಠಿ-ಯನ್ನು
ಲಾರಂ-ಭಿ-ಸಿ-ದರು
ಲಾಹೋರ್
ಲೆಕ್ಕ
ಲೆಕ್ಕ-ವೆಂ-ದ-ರಾ-ಗದು
ಲೇಖ-ಕನ
ಲೇಖ-ಕನು
ಲೇಖ-ನ-ಭಾ-ಷ-ಣ-ಗಳ
ಲೇರಿದ
ಲೋಕಕ್ಕೆ
ಲೋಕದ
ಲೋಕ-ದಿಂದ
ಲೋಕ-ವಿ-ಖ್ಯಾತ
ಲೋಕ-ಹಿತ
ಲೋಕಾ-ನು-ಭ-ವಿ-ಯಾದ
ಲೋಪ-ದೋಷ
ಲೋಪ-ದೋ-ಷ-ಗಳನ್ನೂ
ಲೋಪ-ದೋ-ಷ-ಗ-ಳೇನೇ
ಲೌಕಿಕ
ಲ್ಲೆಲ್ಲ
ಲ್ಲೊಂದು
ವಂಚಿ-ತ-ರಾ-ಗ-ಲಿಲ್ಲ
ವಂತನ
ವಂತಹ
ವಂತಿಗೆ
ವಂಶದ
ವಂಶ-ಸ್ಥರು
ವಕಾ-ಲತ್ತು
ವಕಾ-ಲ-ತ್ತೇನೂ
ವಕೀ-ಲ-ನಾಗಿ
ವಕೀ-ಲ-ನಾದ
ವಕೀ-ಲ-ನಾ-ದ್ದ-ರಿಂದ
ವಕೀ-ಲ-ರಿಂದ
ವಕೀಲಿ
ವಕ್ರೋ-ಕ್ತಿಯ
ವಜ್ರ-ಸ-ಮ-ವಾಗಿ
ವತ್ತಾಗಿ
ವನ-ಭೋ-ಜ-ನಕ್ಕೆ
ವನ-ರಾ-ಜಿ-ಗಳ
ವನ್ನು
ವಪ್ಪ
ವಯ-ಸ್ಸಿಗೆ
ವಯ-ಸ್ಸಿನ
ವಯ-ಸ್ಸಿ-ನಲ್ಲಿ
ವಯ-ಸ್ಸಿ-ನ-ವ-ರೆಗೆ
ವಯಸ್ಸು
ವಯೋ-ಗು-ಣಕ್ಕೆ
ವಯೋ-ಮಿ-ತಿ-ಯನ್ನು
ವರ-ಪ್ರ-ಸಾ-ದ-ವ-ಲ್ಲ-ವೇನು
ವರಿ-ಗೆಲ್ಲ
ವರ್ಗದ
ವರ್ಗ-ದಲ್ಲಿ
ವರ್ಗ-ದ-ವರೆಲ್ಲ
ವರ್ಜಿ-ಸ-ಬೇ-ಕಾ-ಯಿತು
ವರ್ಜಿ-ಸ-ಬೇ-ಕೆಂ-ಬುದು
ವರ್ಡ್ಸ-್ವ-ರ್ತ್
ವರ್ಣ
ವರ್ಣದ
ವರ್ಣನೆ
ವರ್ಣ-ನೆ-ಯನ್ನು
ವರ್ಣ-ಮಾ-ಲೆ-ಯನ್ನು
ವರ್ಣಿ-ಸುತ್ತ
ವರ್ಣಿ-ಸುವ
ವರ್ತನೆ
ವರ್ತ-ನೆ-ಯ-ಲ್ಲಾ-ಗಲಿ
ವರ್ತ-ನೆ-ಯಿಂದ
ವರ್ತಿ-ಸು-ತ್ತಿದ್ದ
ವರ್ಷ
ವರ್ಷ-ಕಾಲ
ವರ್ಷ-ಗಳ
ವರ್ಷ-ಗಳಲ್ಲಿ
ವರ್ಷ-ಗ-ಳ-ವ-ರೆಗೂ
ವರ್ಷ-ಗಳು
ವರ್ಷ-ಗಳೇ
ವರ್ಷದ
ವರ್ಷ-ದ-ವ-ನಿ-ರು-ವಾಗ
ವರ್ಷ-ವನ್ನು
ವರ್ಷ-ವಷ್ಟೆ
ವರ್ಷ-ವಿಡೀ
ವರ್ಷ-ವೇನೋ
ವರ್ಷಾಂ-ತ್ಯ-ದಲ್ಲಿ
ವಲ್ಲವೆ
ವಷ್ಟ-ರಲ್ಲಿ
ವಸೂಲು
ವಸ್ತಾದ
ವಸ್ತು-ವನ್ನು
ವಸ್ತು-ವಾ-ಗಿದ್ದ
ವಸ್ತ್ರಾ-ಭ-ರ-ಣ-ಗಳಿಂದ
ವಹಿ-ಸಿ-ಕೊ-ಳ್ಳು-ತ್ತೇನೆ
ವಾಂತಿ
ವಾಂತಿ-ಯನ್ನು
ವಾಕ್ಪ-ಟು-ತ್ವಕ್ಕೆ
ವಾಕ್ಯ-ಗಳನ್ನು
ವಾಕ್ಯ-ವಾ-ಕ್ಯ-ಗ-ಳನ್ನೇ
ವಾಗ-ದಿ-ದ್ದರೆ
ವಾಗಿ
ವಾಗಿತ್ತು
ವಾಗಿಯೇ
ವಾಗಿ-ರು-ತ್ತಿತ್ತು
ವಾಗಿ-ರು-ತ್ತಿದ್ದ
ವಾಗ್ಮಿ-ಗಳೂ
ವಾಗ್ಮಿಯ
ವಾಗ್ಮಿ-ವೀರ
ವಾಗ್ವಾದ
ವಾಣಿ-ಯನ್ನು
ವಾತಾ-ವ-ರಣ
ವಾತಾ-ವ-ರ-ಣ-ದಲ್ಲಿ
ವಾತಾ-ವ-ರ-ಣ-ದಿಂದ
ವಾತಾ-ವ-ರ-ಣ-ವನ್ನು
ವಾತ್ಸ-ಲ್ಯ-ದಿಂದ
ವಾದ
ವಾದಕ್ಕೆ
ವಾದ-ವನ್ನು
ವಾದ-ವನ್ನೂ
ವಾದ-ವಿ-ವಾ-ದ-ಗಳನ್ನು
ವಾದಷ್ಟು
ವಾದಿ-ಸು-ತ್ತಿ-ದ್ದರು
ವಾದು-ದ-ರಿಂದ
ವಾದುವು
ವಾದ್ಯ
ವಾಯಿತು
ವಾರಂಟು
ವಾರದ
ವಾರಾ-ಣ-ಸಿಗೆ
ವಾರಾ-ಣ-ಸಿ-ಯನ್ನು
ವಾರಾ-ಣ-ಸಿ-ಯಲ್ಲಿ
ವಾರ್ಷಿಕ
ವಾಲನ
ವಾಲಿ-ಕೊಂ-ಡಿತ್ತು
ವಾಲಿ-ಕೊಂ-ಡಿ-ರು-ವಂತೆ
ವಾಸ-ವಾ-ಗಿ-ದ್ದಾರೆ
ವಾಸ-ವಾ-ಗಿ-ದ್ದು-ಕೊಂಡು
ವಾಸ-ವಾ-ಗಿ-ರು-ತ್ತಾನೆ
ವಾಸಿ-ಯಾ-ಗಲು
ವಾಸಿ-ಯಾ-ದ-ವರು
ವಿಂಧ್ಯ-ಪ-ರ್ವ-ತದ
ವಿಕ-ಸ-ನಕ್ಕೆ
ವಿಕಾ-ಸ-ಹೊಂದು
ವಿಖ್ಯಾತ
ವಿಗ್ರಹ
ವಿಗ್ರ-ಹ-ಗ-ಳ-ನ್ನಿ-ಟ್ಟಿದ್ದ
ವಿಗ್ರ-ಹ-ಗಳನ್ನು
ವಿಗ್ರ-ಹ-ಗ-ಳ-ಲ್ಲವೆ
ವಿಗ್ರ-ಹ-ಗಳು
ವಿಗ್ರ-ಹ-ಗ-ಳೆ-ದು-ರಿ-ನಲ್ಲಿ
ವಿಗ್ರ-ಹದ
ವಿಗ್ರ-ಹ-ವನ್ನು
ವಿಗ್ರ-ಹ-ವಾಗಿ
ವಿಚ-ಲಿ-ತ-ನಾ-ಗು-ತ್ತಿ-ರ-ಲಿಲ್ಲ
ವಿಚ-ಲಿ-ತ-ರಾ-ದರು
ವಿಚಾರ
ವಿಚಾ-ರ-ಗಳ
ವಿಚಾ-ರ-ಗಳನ್ನು
ವಿಚಾ-ರ-ಗಳನ್ನೆಲ್ಲ
ವಿಚಾ-ರ-ಗಳಿಂದ
ವಿಚಾ-ರಣೆ
ವಿಚಾ-ರದ
ವಿಚಾ-ರ-ದಲ್ಲಿ
ವಿಚಾ-ರ-ಧಾರೆ
ವಿಚಾ-ರ-ಧಾ-ರೆ-ಇ-ವು-ಗಳನ್ನು
ವಿಚಾ-ರ-ಧಾ-ರೆ-ಯಲ್ಲಿ
ವಿಚಾ-ರ-ಧಾ-ರೆ-ಯಿಂದ
ವಿಚಾ-ರ-ಪರ
ವಿಚಾ-ರ-ವಂ-ತರ
ವಿಚಾ-ರ-ವಂ-ತಿ-ಕೆಗೆ
ವಿಚಾ-ರ-ವನ್ನು
ವಿಚಾ-ರ-ಶ-ಕ್ತಿ-ಯಿಂದ
ವಿಚಾ-ರ-ಶ-ಕ್ತಿ-ಯೆನ್ನು
ವಿಚಾ-ರ-ಸ್ವಾ-ತಂ-ತ್ರ್ಯ-ವ-ನ್ನಿ-ತ್ತಿದ್ದ
ವಿಚಾ-ರಿ-ಸಿ-ಕೊ-ಳ್ಳು-ತಿದ್ದ
ವಿಚಿತ್ರ
ವಿಚಿ-ತ್ರ-ವಪ್ಪಾ
ವಿಜ-ಯ-ಕೃಷ್ಣ
ವಿಜ-ಯೋ-ತ್ಸಾ-ಹ-ದಿಂದ
ವಿತ್ತು
ವಿದ್ದರೆ
ವಿದ್ದು-ದ-ರಿಂದ
ವಿದ್ಯಾ-ಧಿ-ದೇ-ವ-ತೆ-ಯಾದ
ವಿದ್ಯಾ-ಬು-ದ್ಧಿ-ಯನ್ನೂ
ವಿದ್ಯಾ-ಭ್ಯಾಸ
ವಿದ್ಯಾ-ಭ್ಯಾ-ಸದ
ವಿದ್ಯಾ-ಭ್ಯಾ-ಸ-ದಲ್ಲಿ
ವಿದ್ಯಾ-ಭ್ಯಾ-ಸ-ವನ್ನು
ವಿದ್ಯಾರ್ಥಿ
ವಿದ್ಯಾ-ರ್ಥಿ-ಗಳ
ವಿದ್ಯಾ-ರ್ಥಿ-ಗ-ಳಂತೆ
ವಿದ್ಯಾ-ರ್ಥಿ-ಗ-ಳಿಗೆ
ವಿದ್ಯಾ-ರ್ಥಿ-ಗಳು
ವಿದ್ಯಾ-ರ್ಥಿ-ಗ-ಳೆಲ್ಲ
ವಿದ್ಯಾ-ರ್ಥಿ-ಯಾಗಿ
ವಿದ್ಯಾ-ರ್ಥಿ-ಯಾಗಿದ್ದಾನೆ
ವಿದ್ಯಾ-ವಂತ
ವಿದ್ಯಾ-ವಂ-ತರು
ವಿದ್ಯಾ-ಸಂಸ್ಥೆ
ವಿದ್ಯಾ-ಸಾ-ಗ-ರ-ರಿಂದ
ವಿದ್ಯಾ-ಸಾ-ಗ-ರರು
ವಿದ್ಯೆ
ವಿದ್ಯೆ-ಗಳನ್ನು
ವಿದ್ಯೆ-ಗಳಲ್ಲಿ
ವಿದ್ಯೆಯ
ವಿದ್ಯೆ-ಯಿಂದ
ವಿದ್ವಾಂ-ಸ-ರಿಂ-ದಲೂ
ವಿದ್ವಾಂ-ಸರು
ವಿದ್ವಾಂ-ಸ-ರೊ-ಬ್ಬ-ರೊಂ-ದಿಗೆ
ವಿಧ-ದಲ್ಲಿ
ವಿಧವಾ
ವಿಧ-ವೆ-ಯರ
ವಿಧಾ-ನ-ದಲ್ಲಿ
ವಿಧಾ-ನ-ವನ್ನು
ವಿಧಿ-ಸುವ
ವಿಧೇ-ಯ-ತೆ-ಯಿಂ-ದಿ-ರ-ಬೇ-ಕಾ-ಗು-ತ್ತದೆ
ವಿಧ್ಯಾ-ಭ್ಯಾಸ
ವಿನ
ವಿನ-ತೆ-ಯನ್ನು
ವಿನ-ಯ-ದಿಂದ
ವಿನಾ-ಯತಿ
ವಿನಾ-ಯಿತಿ
ವಿನಾ-ಯಿ-ತಿ-ಯನ್ನೇ
ವಿನಾ-ಯಿ-ತಿ-ಯನ್ನೋ
ವಿನಿ-ಯೋ-ಗಿ-ಸ-ಬೇ-ಕಾ-ಗು-ತ್ತದೆ
ವಿನಿ-ಯೋ-ಗಿ-ಸು-ತ್ತಿದ್ದ
ವಿನ್ಯಾ-ಸವೇ
ವಿಪ-ರೀತ
ವಿಭಿನ್ನ
ವಿಭಿ-ನ್ನ-ವಾದ
ವಿಮ-ರ್ಶಾ-ತ್ಮಕ
ವಿಮ-ರ್ಶೆಗೆ
ವಿಮ-ರ್ಶೆ-ಮಾಡಿ
ವಿಮ-ರ್ಶೆ-ಯನ್ನು
ವಿಮಾ-ರ್ಶಾ-ತ್ಮಕ
ವಿರಸ
ವಿರಾ-ಗಿಯ
ವಿರಾ-ಮದ
ವಿರುದ್ಧ
ವಿರು-ದ್ಧ-ವಾ-ಗ-ಬ-ಹುದು
ವಿರು-ದ್ಧ-ಹೀಗೆ
ವಿರೋಧಿ
ವಿರೋ-ಧಿ-ಸು-ತ್ತಿತ್ತು
ವಿರೋ-ಧಿ-ಸು-ತ್ತಿದ್ದ
ವಿರೋ-ಧಿ-ಸುವ
ವಿಲಾ-ಸ-ಗಳ
ವಿಲಾ-ಸ-ಗಳಿಂದ
ವಿಲಾ-ಸ-ಪೂ-ರ್ಣ-ವಾದ
ವಿಲಿಯಂ
ವಿವ-ರ-ವಾಗಿ
ವಿವ-ರಿಸಿ
ವಿವ-ರಿ-ಸಿದ್ದು
ವಿವ-ರಿ-ಸು-ತ್ತಾನೆ
ವಿವ-ರಿ-ಸು-ತ್ತಿದ್ದ
ವಿವ-ರಿ-ಸು-ವಾಗ
ವಿವಾಹ
ವಿವಾ-ಹ-ಗಳು
ವಿವಾ-ಹದ
ವಿವಿಧ
ವಿವೇಕ
ವಿವೇ-ಕ-ಪ್ರ-ಜ್ಞೆಯ
ವಿವೇ-ಕ-ಬು-ದ್ಧಿ-ಯನ್ನು
ವಿವೇಕಾ
ವಿವೇಕಾನಂದ
ವಿವೇಕಾನಂದರ
ವಿವೇಕಾನಂದ-ರ-ನ್ನಾಗಿ
ವಿವೇಕಾನಂದ-ರಲ್ಲಿ
ವಿವೇಕಾನಂದ-ರಾಗಿ
ವಿವೇಕಾನಂದ-ರಾದ
ವಿವೇಕಾನಂದರು
ವಿವೇಕಾನಂದರೇ
ವಿಶ-ದ-ವಾಗಿ
ವಿಶಾಲ
ವಿಶಾ-ಲ-ದೃ-ಷ್ಟಿ-ಯಿಂದ
ವಿಶಾ-ಲ-ವಾಗಿ
ವಿಶಾ-ಲ-ವಾದ
ವಿಶಿಷ್ಟ
ವಿಶೇಷ
ವಿಶೇ-ಷ-ತೆ-ಯಿತ್ತು
ವಿಶೇ-ಷ-ವಾಗಿ
ವಿಶೇ-ಷ-ವಾದ
ವಿಶ್ಲೇ-ಷಿಸಿ
ವಿಶ್ವದ
ವಿಶ್ವ-ನಾಥ
ವಿಶ್ವ-ನಾ-ಥ-ದತ್ತ
ವಿಶ್ವ-ನಾ-ಥನ
ವಿಶ್ವ-ನಾ-ಥ-ನಂ-ತಹ
ವಿಶ್ವ-ನಾ-ಥ-ನಂ-ತೆಯೇ
ವಿಶ್ವ-ನಾ-ಥ-ನನ್ನು
ವಿಶ್ವ-ನಾ-ಥ-ನಲ್ಲಿ
ವಿಶ್ವ-ನಾ-ಥ-ನಿ-ಗಿನ್ನೂ
ವಿಶ್ವ-ನಾ-ಥ-ನಿಗೆ
ವಿಶ್ವ-ನಾ-ಥನು
ವಿಶ್ವ-ನಾ-ಥನೂ
ವಿಶ್ವ-ನಾ-ಥ-ನೇನೂ
ವಿಶ್ವ-ಮಾನವ
ವಿಶ್ವ-ವಿ-ಜೇತ
ವಿಶ್ವ-ವಿ-ದ್ಯಾ-ಲ-ಯ-ಗಳು
ವಿಶ್ವ-ವ್ಯಾ-ಪ-ಕ-ವಾಗಿ
ವಿಶ್ವಾ-ನಾಥ
ವಿಶ್ವಾಸ
ವಿಶ್ವಾ-ಸ-ದಿಂದ
ವಿಶ್ವಾ-ಸ-ಪಾತ್ರ
ವಿಶ್ವಾ-ಸ-ವಿ-ಟ್ಟಿ-ದ್ದರು
ವಿಷಮ
ವಿಷಯ
ವಿಷ-ಯ-ಕ್ಕಾಗಿ
ವಿಷ-ಯಕ್ಕೆ
ವಿಷ-ಯ-ಗಳ
ವಿಷ-ಯ-ಗ-ಳಂತೂ
ವಿಷ-ಯ-ಗಳನ್ನು
ವಿಷ-ಯ-ಗಳನ್ನೆಲ್ಲ
ವಿಷ-ಯ-ಗಳಲ್ಲಿ
ವಿಷ-ಯ-ಗ-ಳೆಂ-ದರೆ
ವಿಷ-ಯ-ಗ-ಳೆಲ್ಲ
ವಿಷ-ಯದ
ವಿಷ-ಯ-ದ-ಲ್ಲಾ-ದರೂ
ವಿಷ-ಯ-ದಲ್ಲಿ
ವಿಷ-ಯ-ವನ್ನು
ವಿಷ-ಯ-ವನ್ನೂ
ವಿಷ-ಯ-ವ-ನ್ನೆಲ್ಲ
ವಿಷ-ಯ-ವಾಗಿ
ವಿಷ-ಯವೂ
ವಿಷ-ಯ-ವೆಂದರೆ
ವಿಷ-ಯ-ವೆಲ್ಲ
ವಿಷ-ಯವೇ
ವಿಷ-ಯ-ವೊಂ-ದನ್ನು
ವಿಷ-ಯ-ಸಂ-ಗ್ರ-ಹಣೆ
ವಿಸ್ತಾ-ರ-ವಾದ
ವಿಸ್ಮಯ
ವಿಸ್ಮ-ಯ-ಮೂ-ಕ-ರಾಗಿ
ವಿಸ್ಮಯಾ
ವಿಹ-ರಿ-ಸು-ತ್ತಿ-ದ್ದಾನೆ
ವಿಹಾರ
ವಿಹಾ-ರಕ್ಕೆ
ವೀಕ್ಷಿಸಿ
ವೀರ
ವೀರ-ಸಂನ್ಯಾಸಿ
ವೀರಾ-ವೇ-ಶ-ದಿಂದ
ವೀರೇ-ಶ್ವರ
ವೀರೋ-ಚಿತ
ವುದ-ರ-ಲ್ಲಿ-ದ್ದರು
ವುದಲ್ಲ
ವುದಿಲ್ಲ
ವುದು
ವೃತ್ತ-ಪ-ತ್ರಿ-ಕೆ-ಗಳನ್ನು
ವೃತ್ತಿ
ವೃತ್ತಿಯ
ವೃತ್ತಿ-ಯಲ್ಲಿ
ವೃದ್ಧ
ವೃದ್ಧಿ
ವೃದ್ಧಿ-ಯಾ-ಗು-ವು-ದಿಲ್ಲ
ವೆಂಥದು
ವೆನ್ನು-ವುದು
ವೇಗ-ದಲ್ಲಿ
ವೇಗ-ವಾಗಿ
ವೇಣೀ
ವೇದಿ-ಕೆ-ಗಳ
ವೇಳ
ವೇಳೆ
ವೇಳೆ-ಗಾ-ಗಲೇ
ವೇಳೆಗೆ
ವೇಳೆ-ಯಲ್ಲಿ
ವೇಳೆ-ಯಲ್ಲೂ
ವೇಳೆ-ಯ-ಲ್ಲೆಲ್ಲ
ವೇಷ-ಭೂ-ಷ-ಣ-ದ-ಲ್ಲಾ-ಗಲಿ
ವೈದ್ಯ-ರಿಗೂ
ವೈಭ-ವದ
ವೈಭ-ವ-ಪೂರ್ಣ
ವೈಭ-ವ-ವನ್ನು
ವೈಭೋ-ಗದ
ವೈಯ-ಕ್ತಿಕ
ವೈರಾಗ್ಯ
ವೈರಾ-ಗ್ಯ-ದಂ-ತಲ್ಲ
ವೈರಾ-ಗ್ಯಾದಿ
ವೈವಾ-ಹಿಕ
ವೈವಿ-ಧ್ಯ-ಪೂ-ರ್ಣ-ವಾದ
ವ್ಯಂಗ್ಯದ
ವ್ಯಕ್ತ-ಗೊ-ಳಿ-ಸು-ತ್ತಿದ್ದ
ವ್ಯಕ್ತ-ಪ-ಡಿ-ಸು-ವಾಗ
ವ್ಯಕ್ತ-ವಾ-ಗ-ತೊ-ಡ-ಗಿತು
ವ್ಯಕ್ತ-ವಾ-ಗಲು
ವ್ಯಕ್ತ-ವಾ-ಗು-ತ್ತಿ-ದ್ದವು
ವ್ಯಕ್ತಿ
ವ್ಯಕ್ತಿ-ಗಳ
ವ್ಯಕ್ತಿ-ಗ-ಳಂ-ತಲ್ಲ
ವ್ಯಕ್ತಿ-ಗ-ಳಂತೆ
ವ್ಯಕ್ತಿ-ಗಳನ್ನೂ
ವ್ಯಕ್ತಿ-ಗ-ಳಿಗೆ
ವ್ಯಕ್ತಿ-ಗ-ಳೆಂ-ದರೆ
ವ್ಯಕ್ತಿಗೆ
ವ್ಯಕ್ತಿ-ಚಿ-ತ್ರ-ಣದ
ವ್ಯಕ್ತಿತ್ವ
ವ್ಯಕ್ತಿ-ತ್ವದ
ವ್ಯಕ್ತಿ-ತ್ವ-ದಲ್ಲಿ
ವ್ಯಕ್ತಿ-ತ್ವ-ದಿಂದ
ವ್ಯಕ್ತಿ-ತ್ವ-ವನ್ನು
ವ್ಯಕ್ತಿಯ
ವ್ಯಕ್ತಿ-ಯಾ-ಗಿ-ಬಿ-ಟ್ಟಿದ್ದ
ವ್ಯಕ್ತಿಯು
ವ್ಯತಿ-ರಿ-ಕ್ತ-ವಾದ
ವ್ಯತ್ಯಾ-ಸ-ವಿತ್ತು
ವ್ಯರ್ಥ-ಇದು
ವ್ಯವ-ಧಾ-ನವೇ
ವ್ಯವಸ್ಥೆ
ವ್ಯವ-ಸ್ಥೆ-ಮಾ-ಡಿದ
ವ್ಯವ-ಹಾ-ರ-ದಲ್ಲಿ
ವ್ಯವ-ಹಾ-ರ-ವ-ನ್ನೆಲ್ಲ
ವ್ಯಾಕು-ಲತೆ
ವ್ಯಾಕು-ಲ-ತೆ-ಯುಂ-ಟಾ-ಗಿದೆ
ವ್ಯಾಪ-ಕ-ವಾಗಿ
ವ್ಯಾಪಾ-ರೋ-ದ್ಯಮ
ವ್ಯಾಪಿ-ಸಿ-ಕೊಂ-ಡಿತು
ವ್ಯಾಯ-ಮ-ಶಾ-ಲೆಯ
ವ್ಯಾಯಾಮ
ವ್ಯಾಯಾ-ಮ-ಮ-ಲ್ಲ-ಯು-ದ್ಧಾ-ದಿ-ಗಳ
ವ್ಯಾಯಾ-ಮ-ಗಳ
ವ್ಯಾಯಾ-ಮ-ಶಾಲೆ
ವ್ಯಾಯಾ-ಮ-ಶಾ-ಲೆಗೆ
ವ್ಯಾಯಾ-ಮ-ಶಾ-ಲೆಯ
ವ್ರಜೇಂ-ದ್ರ-ನಾಥ
ವ್ರತ-ಕ್ಕೀಗ
ವ್ರತ-ಗಳನ್ನು
ವ್ರತ-ಧಾ-ರಿ-ಯಾಗಿ
ವ್ರತ-ನಿ-ಯ-ಮ-ಗ-ಳ-ನ್ನಾ-ಚ-ರಿಸಿ
ಶಕ್ತಿ
ಶಕ್ತಿ-ಕೇಂ-ದ್ರವೇ
ಶಕ್ತಿ-ಗಳ
ಶಕ್ತಿ-ಯನ್ನು
ಶಕ್ತಿ-ಯನ್ನೂ
ಶಕ್ತಿ-ಯ-ನ್ನೆಲ್ಲ
ಶಕ್ತಿ-ಯಿಂದ
ಶಕ್ತಿ-ಯಿ-ರು-ವ-ವ-ರನ್ನು
ಶಕ್ತಿ-ಯುತ
ಶಕ್ತಿ-ಯು-ತ-ವಾಗಿ
ಶಕ್ತಿ-ಶಾ-ಲಿ-ಯಾದ
ಶತ-ಮಾ-ನ-ಗಳ
ಶತ-ಮಾ-ನದ
ಶತ್ರು-ಗಳ
ಶಬ್ದ
ಶಬ್ದ-ಕೋ-ಶ-ವೆಲ್ಲ
ಶಬ್ದಕ್ಕೆ
ಶಬ್ದ-ಗಳನ್ನು
ಶಬ್ದ-ಗಳಿಂದ
ಶಬ್ದ-ಗಳು
ಶಬ್ದ-ವನ್ನೂ
ಶರ-ಣಾ-ಗಿ-ದ್ದು-ಕೊಂಡು
ಶರ-ಣಾಗು
ಶರೀರ
ಶರೀ-ರದ
ಶರೀ-ರ-ದಲ್ಲೂ
ಶರೀ-ರ-ದಿಂದ
ಶರೀ-ರ-ರ-ಚ-ನೆಯೂ
ಶರೀ-ರ-ವನ್ನು
ಶರೀ-ರ-ವ-ನ್ನೆಲ್ಲ
ಶರೀ-ರ-ಸಂ-ಪ-ತ್ತನ್ನು
ಶರೀ-ರ-ಸಂ-ಪ-ತ್ತನ್ನೂ
ಶಾಂತ-ಗಂ-ಭೀರ
ಶಾಂತ-ನಾ-ಗು-ವುದು
ಶಾಂತ-ನಾದ
ಶಾಂತ-ವಾಗಿ
ಶಾಂತಿ-ಯನ್ನು
ಶಾಕ್ತ
ಶಾರೀ-ರಿಕ
ಶಾಲಾ
ಶಾಲಾ-ವಿ-ದ್ಯಾ-ಭ್ಯಾ-ಸಕ್ಕೆ
ಶಾಲೆ
ಶಾಲೆಗೆ
ಶಾಲೆಗೇ
ಶಾಲೆಯ
ಶಾಲೆ-ಯನ್ನು
ಶಾಲೆ-ಯಲ್ಲಿ
ಶಾಲೆ-ಯಲ್ಲೇ
ಶಾಲೆ-ಯಿಂದ
ಶಾಲೆಯೂ
ಶಾಸ್ತ್ರ-ಗಳು
ಶಾಸ್ತ್ರ-ದಲ್ಲಿ
ಶಾಸ್ತ್ರಿ
ಶಾಸ್ತ್ರೀಯ
ಶಾಸ್ತ್ರೀ-ಯಂ-ತಹ
ಶಿಕ್ಷಣ
ಶಿಕ್ಷ-ಣ-ತ-ಜ್ಞರೂ
ಶಿಕ್ಷೆ-ಗಳನ್ನು
ಶಿಖ-ರ-ಗಳು
ಶಿಖ-ರ-ದಲ್ಲಿ
ಶಿರ-ಚ್ಛೇ-ದನ
ಶಿವ
ಶಿವ-ಕೃಪೆ
ಶಿವ-ಧಾ-ಮ-ವನ್ನು
ಶಿವನ
ಶಿವ-ನನ್ನು
ಶಿವ-ನ-ಲ್ಲವೇ
ಶಿವ-ನಾಥ
ಶಿವ-ನಿಗೆ
ಶಿವನೇ
ಶಿವ-ಪೂಜೆ
ಶಿವ-ಪೂ-ಜೆ-ಶಿ-ವ-ಧ್ಯಾ-ನ-ಗ-ಳಲ್ಲೇ
ಶಿವ-ಪೂ-ಜೆ-ಯಲ್ಲಿ
ಶಿವ-ಮ-ಯವೇ
ಶಿವ-ಸ್ಮ-ರಣೆ
ಶಿಶು-ವನ್ನು
ಶಿಶು-ವಿನ
ಶಿಷ್ಟಾ-ಚಾರ
ಶಿಷ್ಯ
ಶಿಷ್ಯನ
ಶಿಷ್ಯ-ನಾ-ಗು-ತ್ತಾನೆ
ಶಿಷ್ಯ-ರಿಗೂ
ಶೀಲ-ವೆಂಬ
ಶುಚಿ
ಶುಚಿ-ಮಾಡಿ
ಶುದ್ಧ-ವಾದ
ಶುಭ
ಶುಭ-ದಿನ
ಶುಭ-ದಿ-ನ-ದಲ್ಲಿ
ಶುಭ-ರಾತ್ರಿ
ಶುಭ್ರ
ಶುರು-ಮಾ-ಡಿ-ದನೋ
ಶುರು-ಮಾ-ಡಿ-ದರು
ಶುಲ್ಕ
ಶುಲ್ಕ-ವನ್ನು
ಶುಲ್ಕ-ವಿ-ನಾ-ಯಿತಿ
ಶೂಟ್
ಶೃಂಗಾ-ಗ್ರ-ಗಳ
ಶೃಂಗಾ-ರ-ಭಾವ
ಶೋಕಿ
ಶೋಕಿಯ
ಶೋಕಿ-ಲಾ-ಲ-ರನ್ನು
ಶೋಭೆ-ಗೊ-ಳಿ-ಸಿ-ದ್ದಾನೆ
ಶೌರ್ಯ
ಶ್
ಶ್ಚರ್ಯದ
ಶ್ಯಾಮ-ಸುಂ-ದರಿ
ಶ್ರದ್ಧೆ
ಶ್ರದ್ಧೆ-ಗಳನ್ನು
ಶ್ರದ್ಧೆ-ಗಳಿಂದ
ಶ್ರದ್ಧೆ-ಯಾ-ಗಲಿ
ಶ್ರದ್ಧೆ-ಯಿಂದ
ಶ್ರಮ-ವ-ನ್ನಷ್ಟೇ
ಶ್ರಮಿ-ಸ-ಬೇ-ಕೆ-ನ್ನುವ
ಶ್ರೀ
ಶ್ರೀಕೃ-ಷ್ಣನ
ಶ್ರೀಕೃ-ಷ್ಣನೂ
ಶ್ರೀಮಂ-ತ-ನಾ-ಗಿದ್ದ
ಶ್ರೀಮಂ-ತ-ರಾ-ಗಿ-ರ-ಬ-ಹುದು
ಶ್ರೀಮಂ-ತರು
ಶ್ರೀಮ-ದ್ಭಾ-ಗ-ವ-ತ-ವನ್ನು
ಶ್ರೀರಾ-ಮ-ಕೃಷ್ಣ
ಶ್ರೀರಾ-ಮ-ಕೃ-ಷ್ಣರ
ಶ್ರೀರಾ-ಮ-ಕೃ-ಷ್ಣ-ರನ್ನು
ಶ್ರೀರಾ-ಮ-ಕೃ-ಷ್ಣರು
ಶ್ರೀರಾ-ಮ-ಕೃ-ಷ್ಣ-ರೆಂಬ
ಶ್ರೀರಾ-ಮ-ಕೃ-ಷ್ಣ-ರೊಂ-ದಿಗೆ
ಶ್ರೀರಾ-ಮನ
ಶ್ರೀರಾ-ಮನು
ಶ್ರುತಿ-ಲಯ
ಶ್ರುತಿ-ಬ-ದ್ಧ-ವಾಗಿ
ಶ್ರುತಿಯ
ಶ್ರೇಣಿ-ಯನ್ನು
ಶ್ರೇಣಿ-ಯಲ್ಲಿ
ಶ್ರೇಯ-ಸ್ಸಿ-ಗಾಗಿ
ಶ್ರೇಷ್ಠ
ಶ್ರೇಷ್ಠ-ನಾದ
ಶ್ರೇಷ್ಠ-ವಾ-ದ-ವಿ-ಚಾ-ರ-ಸ-ಮ್ಮ-ತ-ವಾದ
ಶ್ಲೋಕ-ಗಳನ್ನು
ಶ್ವರಿಯ
ಶ್ವರ್ಯ-ವನ್ನೂ
ಶ್ವಾಸೋ-ಚ್ಛ್ವಾಸ
ಷರ-ತ್ತಿಗೆ
ಷ್
ಸಂಕಟ
ಸಂಕ-ಟ-ಗಳನ್ನು
ಸಂಕ-ಲ್ಪ-ಗಳನ್ನು
ಸಂಕು-ಚಿತ
ಸಂಕು-ಚಿ-ತ-ವಾ-ದ-ವು-ಗಳು
ಸಂಕೇ-ತವೇ
ಸಂಗ-ಡಿಗ
ಸಂಗ-ಡಿ-ಗ-ನೊ-ಬ್ಬ-ನನ್ನು
ಸಂಗ-ಡಿ-ಗ-ರ-ನ್ನೆಲ್ಲ
ಸಂಗ-ಡಿ-ಗ-ರೆಲ್ಲ
ಸಂಗ-ತಿ-ಯೆಂ-ದರೆ
ಸಂಗಾ-ತಿ-ಗಳ
ಸಂಗಾ-ತಿ-ಗಳು
ಸಂಗಾ-ತಿ-ಗ-ಳೆ-ಲ್ಲರ
ಸಂಗೀತ
ಸಂಗೀ-ತಕ್ಕೆ
ಸಂಗೀ-ತ-ಗಾ-ರ-ರಾದ
ಸಂಗೀ-ತ-ಗಾ-ರ-ರಿಂದ
ಸಂಗೀ-ತದ
ಸಂಗೀ-ತ-ದಲ್ಲಿ
ಸಂಗೀ-ತ-ದಲ್ಲೂ
ಸಂಗೀ-ತ-ಪ್ರೇ-ಮ-ವನ್ನೂ
ಸಂಗೀ-ತ-ವನ್ನು
ಸಂಗೀ-ತ-ವನ್ನೂ
ಸಂಗೀ-ತ-ವೆ-ನಿ-ಸ-ಲಾ-ರದು
ಸಂಗೀ-ತ-ವೆ-ನ್ನು-ವುದು
ಸಂಗೀ-ತವೇ
ಸಂಗೀ-ತಾ-ಭ್ಯಾಸ
ಸಂಗ್ರ-ಹ-ವಾ-ಯಿತು
ಸಂಘ-ಟಿ-ಸಿತು
ಸಂಘರ್ಷ
ಸಂಘ-ಸಂ-ಸ್ಥೆ-ಗಳನ್ನು
ಸಂಚ-ರಿಸಿ
ಸಂಚ-ರಿ-ಸುತ್ತ
ಸಂಚ-ರಿ-ಸು-ತ್ತಿ-ದ್ದಾಗ
ಸಂಚ-ರಿ-ಸು-ವಾಗ
ಸಂಚ-ಲ-ನಕ್ಕೂ
ಸಂಚಾ-ರಕ್ಕೆ
ಸಂಚಾ-ರ-ವಾ-ಯಿತು
ಸಂಜೆ
ಸಂಜೆ-ಗ-ತ್ತ-ಲಾ-ಗು-ತ್ತಿತ್ತು
ಸಂಜೆ-ಗ-ತ್ತಲು
ಸಂಜೆಯ
ಸಂಜೆ-ಯಾ-ದ್ದ-ರಿಂದ
ಸಂತ-ಸ-ಪಟ್ಟ
ಸಂತುಷ್ಟ
ಸಂತೋಷ
ಸಂತೋ-ಷ-ಕ್ಕಾಗಿ
ಸಂತೋ-ಷ-ಗಳಿಂದ
ಸಂತೋ-ಷದ
ಸಂತೋ-ಷ-ದಿಂದ
ಸಂತೋ-ಷ-ಪ-ಟ್ಟರು
ಸಂತೋ-ಷ-ವನ್ನು
ಸಂತೋ-ಷವೂ
ಸಂದರ್ಭ
ಸಂದ-ರ್ಭ-ಗಳಲ್ಲಿ
ಸಂದ-ರ್ಭ-ಗ-ಳಲ್ಲೂ
ಸಂದ-ರ್ಭ-ಗ-ಳ-ಲ್ಲೆಲ್ಲ
ಸಂದ-ರ್ಭ-ಗ-ಳಲ್ಲೇ
ಸಂದ-ರ್ಭ-ದಲ್ಲಿ
ಸಂದ-ರ್ಭ-ದ-ಲ್ಲಿನ
ಸಂಧಿ-ಸಿ-ದಾ-ಗಲೂ
ಸಂನ್ಯಾಸ
ಸಂನ್ಯಾ-ಸ-ಜೀ-ವ-ನ-ವನ್ನು
ಸಂನ್ಯಾ-ಸದ
ಸಂನ್ಯಾ-ಸ-ರೇಖೆ
ಸಂನ್ಯಾ-ಸ-ರೇ-ಖೆ-ಯಂತೆ
ಸಂನ್ಯಾಸಿ
ಸಂನ್ಯಾ-ಸಿ-ಗಳ
ಸಂನ್ಯಾ-ಸಿ-ಗ-ಳಾ-ಗ-ಬೇ-ಕೆಂದು
ಸಂನ್ಯಾ-ಸಿ-ಗ-ಳಾಗಿ
ಸಂನ್ಯಾ-ಸಿ-ಗಳೇ
ಸಂನ್ಯಾ-ಸಿಯ
ಸಂನ್ಯಾ-ಸಿ-ಯಾ-ಗ-ಬೇ-ಕೆಂಬ
ಸಂನ್ಯಾ-ಸಿ-ಯಾಗಿ
ಸಂನ್ಯಾ-ಸಿ-ಯಾ-ಗು-ತ್ತಾ-ನಂ-ತಲ್ಲ
ಸಂನ್ಯಾ-ಸಿ-ಯಾ-ಗು-ವು-ದ-ರಲ್ಲಿ
ಸಂನ್ಯಾ-ಸಿ-ಯಾದ
ಸಂನ್ಯಾ-ಸಿ-ಯಾ-ದರೂ
ಸಂಪ-ತ್ತನ್ನು
ಸಂಪ-ತ್ತಿ-ಗಿಂತ
ಸಂಪತ್ತು
ಸಂಪ-ನ್ನ-ನಾದ
ಸಂಪ-ನ್ಮೂ-ಲ-ಗ-ಳಾ-ಗಲಿ
ಸಂಪ-ರ್ಕಕ್ಕೆ
ಸಂಪ-ರ್ಕ-ದಿಂ-ದಾಗಿ
ಸಂಪ-ರ್ಕ-ವನ್ನು
ಸಂಪ-ರ್ಕ-ವಿ-ಟ್ಟ-ಕೊಂ-ಡಿದ್ದ
ಸಂಪಾ-ದನೆ
ಸಂಪಾ-ದಿಸಿ
ಸಂಪಾ-ದಿ-ಸಿ-ಕೊ-ಳ್ಳಲು
ಸಂಪಾ-ದಿ-ಸಿ-ದ್ದನ್ನು
ಸಂಪಿಗೆ
ಸಂಪು-ಟ-ಗಳ
ಸಂಪು-ಟ-ವಾ-ಗಿದೆ
ಸಂಪೂರ್ಣ
ಸಂಪೂ-ರ್ಣ-ವಾಗಿ
ಸಂಪ್ರ-ದಾಯ
ಸಂಪ್ರ-ದಾ-ಯ-ಗಳನ್ನು
ಸಂಪ್ರ-ದಾ-ಯ-ಗಳನ್ನೆಲ್ಲ
ಸಂಪ್ರ-ದಾ-ಯ-ಗಳಲ್ಲಿ
ಸಂಪ್ರ-ದಾ-ಯದ
ಸಂಪ್ರ-ದಾ-ಯ-ದ-ವರು
ಸಂಪ್ರ-ದಾ-ಯ-ಪ್ರಿಯ
ಸಂಪ್ರ-ದಾ-ಯ-ಶ-ರ-ಣ-ರನ್ನು
ಸಂಬಂ-ಧ-ಪಟ್ಟ
ಸಂಬಂ-ಧ-ಪ-ಟ್ಟಂತೆ
ಸಂಬಂ-ಧ-ವಾಗಿ
ಸಂಬಂ-ಧ-ವಾದ
ಸಂಬಂ-ಧವೇ
ಸಂಬಂಧಿ
ಸಂಬಂ-ಧಿ-ಕರ
ಸಂಬಂ-ಧಿ-ಕ-ರಿ-ರಲಿ
ಸಂಬಂ-ಧಿ-ಗ-ಳಿಗೆ
ಸಂಬಂ-ಧಿ-ಯಾದ
ಸಂಬಂ-ಧಿ-ಸಿದ
ಸಂಬಂ-ಧಿ-ಸಿದ್ದೇ
ಸಂಭವ
ಸಂಭ-ವ-ವಿತ್ತು
ಸಂಭ-ವಿಸಿ
ಸಂಭ-ವಿ-ಸಿ-ದ್ದ-ರಿಂದ
ಸಂಭಾ-ಷಣ
ಸಂಭಾ-ಷ-ಣೆ-ಗಳನ್ನು
ಸಂಭಾ-ಷ-ಣೆಯ
ಸಂಮಿ-ಶ್ರಣ
ಸಂಶ-ಯವೇ
ಸಂಶೋ-ಧನಾ
ಸಂಸಾರ
ಸಂಸಾ-ರ-ಜೀ-ವನ
ಸಂಸಾ-ರದ
ಸಂಸಾ-ರ-ವ-ನ್ನಂತೂ
ಸಂಸ್ಕ-ರ-ಣಕ್ಕೆ
ಸಂಸ್ಕಾ-ರ-ಗಳ
ಸಂಸ್ಕಾ-ರವೇ
ಸಂಸ್ಕೃತ
ಸಂಸ್ಕೃ-ತ-ದಲ್ಲಿ
ಸಂಸ್ಕೃತಿ
ಸಂಸ್ಕೃ-ತಿ-ನಾ-ಗ-ರಿ-ಕ-ತೆ-ಗಳು
ಸಂಸ್ಕೃ-ತಿ-ಗಳ
ಸಂಸ್ಕೃ-ತಿಯ
ಸಂಸ್ಕೃ-ತಿ-ಯನ್ನು
ಸಂಸ್ಕೃ-ತಿಯೂ
ಸಂಸ್ಕೃ-ತಿ-ಯೊಂದು
ಸಂಸ್ಥೆಯ
ಸಂಸ್ಥೆ-ಯನ್ನು
ಸಂಸ್ಥೆ-ಯಾ-ಗಲಿ
ಸಂಸ್ಥೆ-ಯಿದು
ಸಕಲೈ
ಸಕಾ-ಲ-ದಲ್ಲಿ
ಸಗುಣ
ಸಗು-ಣ-ನಿ-ರಾ-ಕಾ-ರ-ಬ್ರ-ಹ್ಮದ
ಸಗು-ಣ-ಸಾ-ಕಾ-ರ-ನೆಂದು
ಸಜ್ಜನ
ಸಜ್ಜ-ನ-ಸು-ಚ-ರಿ-ತ್ರ-ನೆ-ನಿ-ಸಿ-ಕೊ-ಳ್ಳ-ಬೇಕಾ
ಸಣ್ಣ
ಸಣ್ಣ-ಗಾಗಿ
ಸತ್ಕ-ರಿ-ಸ-ಲೇ-ಬೇ-ಕಂತೆ
ಸತ್ತೇ-ಹೋ-ದ-ನೆಂದು
ಸತ್ತ್ವ-ವಿ-ದೆಯೆ
ಸತ್ಪ-ರಿ-ಣಾ-ಮ-ವ-ನ್ನುಂ-ಟು-ಮಾ-ಡಿ-ದವು
ಸತ್ಯ
ಸತ್ಯ-ದ-ರ್ಶನ
ಸತ್ಯ-ವನ್ನು
ಸತ್ಯ-ಶೋ-ಧಕ
ಸತ್ಯ-ಸಂ-ಗ-ತಿ-ಯನ್ನು
ಸತ್ಯ-ಸಾ-ಕ್ಷಾ-ತ್ಕಾ-ರ-ವೆಂದರೆ
ಸತ್ಯಾಂ-ಶ-ಗಳು
ಸತ್ವ-ರ-ಹಿತ
ಸದ-ವ-ಕಾಶ
ಸದ-ವ-ಕಾ-ಶ-ವಾ-ಯಿತು
ಸದ-ಸ್ಯ-ರನ್ನು
ಸದ-ಸ್ಯ-ರಾ-ಗಿ-ದ್ದರು
ಸದ-ಸ್ಯ-ರಿಗೂ
ಸದ-ಸ್ಯರು
ಸದಾ
ಸದ್ಗುಣ
ಸದ್ಗು-ಣ-ಗಳು
ಸನಾ-ತನ
ಸನ್ನಾ-ಹ-ದ-ಲ್ಲಿತ್ತು
ಸನ್ನಿ-ಧಿ-ಯಲ್ಲಿ
ಸನ್ನಿ-ವೇ-ಶಕ್ಕೆ
ಸನ್ನಿ-ವೇ-ಶ-ಗಳಿಂದ
ಸನ್ನಿ-ವೇ-ಶ-ದಲ್ಲಿ
ಸನ್ನಿ-ವೇ-ಶ-ವನ್ನು
ಸನ್ಮಾನ
ಸಪ್ಪೆ-ಮೋರೆ
ಸಭೆ
ಸಭೆ-ಗಳಲ್ಲಿ
ಸಭೆಗೆ
ಸಭೆಯ
ಸಭೆ-ಯಲ್ಲಿ
ಸಭ್ಯ-ರಂತೆ
ಸಭ್ಯ-ಸ್ಥನ
ಸಮ
ಸಮ-ಕಾ-ಲೀನ
ಸಮ-ಗ್ರ-ದೃ-ಷ್ಟಿ-ಯಿಂದ
ಸಮನೆ
ಸಮಯ
ಸಮ-ಯಕ್ಕೆ
ಸಮ-ಯ-ಗಳಲ್ಲಿ
ಸಮ-ಯ-ದಲ್ಲಿ
ಸಮ-ಯ-ವನ್ನು
ಸಮ-ಯ-ವ-ನ್ನೆಲ್ಲ
ಸಮ-ಯ-ಸ್ಫೂರ್ತಿ
ಸಮರ
ಸಮ-ರಾಂ-ಗ-ಣ-ದಲ್ಲಿ
ಸಮ-ರ್ಥಿ-ಸಿ-ದರು
ಸಮ-ವಸ್ತ್ರ
ಸಮಸ್ತ
ಸಮಸ್ಯೆ
ಸಮ-ಸ್ಯೆ-ಗಳನ್ನು
ಸಮಾ-ಗ-ಮ-ವಾ-ದದ್ದು
ಸಮಾ-ಚಾರ
ಸಮಾಜ
ಸಮಾ-ಜದ
ಸಮಾ-ಜ-ದಲ್ಲಿ
ಸಮಾ-ಜ-ದ-ವರು
ಸಮಾ-ಜ-ವನ್ನು
ಸಮಾ-ಜವು
ಸಮಾ-ಜ-ಹಿತ
ಸಮಾ-ಧಾನ
ಸಮಾ-ಧಾ-ನ-ಕರ
ಸಮಾ-ಧಾ-ನ-ದಿಂ-ದಲೇ
ಸಮಾ-ಧಾ-ನ-ದಿಂ-ದಿ-ರು-ವುದನ್ನು
ಸಮಾ-ಧಾ-ನ-ಪ-ಡಿ-ಸಿ-ದರು
ಸಮಾ-ಧಾ-ನ-ಪ-ಡಿ-ಸುತ್ತ
ಸಮಾ-ಧಾ-ನ-ವಾ-ಗು-ತ್ತಿಲ್ಲ
ಸಮಾ-ಧಾ-ನ-ವಿಲ್ಲ
ಸಮಾನ
ಸಮಾ-ನ-ಸ್ಕಂ-ಧ-ನೆಂ-ಬಂತೆ
ಸಮಾ-ರಂ-ಭ-ವೊಂ-ದನ್ನು
ಸಮೀ-ಪ-ದಲ್ಲೇ
ಸಮೀ-ಪಿ-ಸಿದೆ
ಸಮೀ-ಪಿ-ಸು-ತ್ತಿದೆ
ಸಮೀ-ಪಿ-ಸು-ತ್ತಿವೆ
ಸಮೃ-ದ್ಧಿಯ
ಸಮೇ-ತ-ವಾಗಿ
ಸಮ್ಮನೆ
ಸಮ್ಮಾ-ನಿತ
ಸರಕು
ಸರದಿ
ಸರಳ
ಸರ-ಳ-ಶುದ್ಧ
ಸರ-ಸ-ವಾಗಿ
ಸರ-ಸ್ವ-ತಿಗೆ
ಸರಿ
ಸರಿ-ದೂ-ಗು-ವಂ-ಥ-ವ-ರನ್ನು
ಸರಿ-ಯಲ್ಲ
ಸರಿ-ಯಾಗಿ
ಸರಿ-ಯಾ-ಗಿ-ದ್ದು-ಬಿಟ್ಟ
ಸರಿ-ಯಾ-ಗಿಯೇ
ಸರಿ-ಯಾ-ಗಿ-ರು-ವಾಗ
ಸರಿ-ಯಾದ
ಸರಿ-ಯಾ-ದ-ದ್ದನ್ನೇ
ಸರಿ-ಯು-ತ್ತಾನೆ
ಸರಿ-ಯು-ತ್ತಿದೆ
ಸರಿಯೆ
ಸರಿ-ವಿ-ರು-ದ್ಧ-ವಾಗಿ
ಸರಿ-ಸ-ಮ-ರಿಲ್ಲ
ಸರಿ-ಸರಿ
ಸರಿ-ಸಾ-ಟಿ-ಯಿ-ಲ್ಲದ
ಸರಿಸಿ
ಸರಿ-ಹೋ-ದಾನು
ಸರಿ-ಹೋ-ಯಿತು
ಸರ್
ಸರ್ಕಸ್
ಸರ್ಪ
ಸರ್ಪವೋ
ಸರ್ಪಾ-ಸ್ತ್ರ-ದಂತೆ
ಸರ್ರನೆ
ಸರ್ವ
ಸರ್ವಜ್ಞ
ಸರ್ವ-ಜ್ಞ-ನಾದ
ಸರ್ವ-ತಿ-ರ-ಸ್ಕಾ-ರದ
ಸರ್ವತೋ
ಸರ್ವ-ಸಂಗ
ಸರ್ವ-ಸ-ಮ-ರ್ಪಣೆ
ಸರ್ವಸ್ವ
ಸರ್ವಾಂಗ
ಸರ್ವಾಂ-ಗ-ಸುಂ-ದರ
ಸರ್ವಾಂ-ಗ-ಸುಂ-ದ-ರ-ವಾಗಿ
ಸರ್ವೇ
ಸರ್ವೇ-ಸಾ-ಮಾನ್ಯ
ಸಲ
ಸಲ-ವಂತೂ
ಸಲ-ವೆಂತೂ
ಸಲಹೆ
ಸಲಿ-ಗೆ-ಯಿಂದ
ಸಲಿ-ಲದ
ಸಲೀ-ಸಾಗಿ
ಸಲ್ಲಾ-ಪ-ದಲ್ಲಿ
ಸಲ್ಲಿ-ಸ-ಬೇಕು
ಸಲ್ಲಿಸಿ
ಸಲ್ಲಿ-ಸಿ-ದರೆ
ಸಲ್ಲಿ-ಸಿ-ದಳು
ಸಲ್ಲಿ-ಸಿ-ದಾಗ
ಸಲ್ಲಿ-ಸು-ತ್ತಿ-ದ್ದರು
ಸಲ್ಲಿ-ಸು-ತ್ತಿ-ರು-ವಾಗ
ಸಲ್ಲಿ-ಸು-ತ್ತೇನೆ
ಸವರಿ
ಸವ-ಲತ್ತೂ
ಸವಾರ
ಸವಾರಿ
ಸವಿ-ಯಲು
ಸವಿ-ಸ್ತಾರ
ಸಸ್ಯಾ-ಹಾ-ರವೇ
ಸಸ್ಯಾ-ಹಾ-ರಿ-ಯಾ-ಗಿ-ಬಿ-ಟ್ಟಿ-ದ್ದಾನೆ
ಸಸ್ಯಾ-ಹಾ-ರಿಯೇ
ಸಹ-ಕಾ-ರ-ವಾ-ಗಲಿ
ಸಹಜ
ಸಹ-ಜ-ವಾಗಿ
ಸಹ-ಜ-ವಾ-ಗಿಯೇ
ಸಹ-ಜ-ವಾದ
ಸಹ-ಪಾ-ಠಿ-ಗಳ
ಸಹ-ಪಾ-ಠಿ-ಗ-ಳೆಲ್ಲ
ಸಹ-ಪಾ-ಠಿ-ಗ-ಳೊಂ-ದಿಗೆ
ಸಹ-ಪಾ-ಠಿಯ
ಸಹ-ಪಾ-ಠಿ-ಯಾದ
ಸಹ-ವಿ-ದ್ಯಾರ್ಥಿ
ಸಹ-ಶಿ-ಷ್ಯ-ನಾದ
ಸಹಾ-ನು-ಭೂತಿ
ಸಹಾ-ನು-ಭೂ-ತಿ-ಯಲ್ಲ
ಸಹಾಯ
ಸಹಾ-ಯ-ಸ-ಹ-ಕಾ-ರ-ಗ-ಳಿ-ಗಾಗಿ
ಸಹಾ-ಯ-ಕಾರಿ
ಸಹಾ-ಯಕ್ಕೆ
ಸಹಾ-ಯ-ದಿಂದ
ಸಹಾ-ಯ-ವನ್ನು
ಸಹಾ-ಯ-ವ-ನ್ನೇಕೆ
ಸಹಾ-ಯ-ಹ-ಸ್ತ-ದಿಂದ
ಸಹಿ-ಸಿ-ಕೊಂಡು
ಸಹಿ-ಸಿ-ಯಾನೆ
ಸಾಂಗ-ವಾಗಿ
ಸಾಂದ-ರ್ಭಿಕ
ಸಾಂದ-ರ್ಭಿ-ಕ-ವಾಗಿ
ಸಾಂಪ್ರ-ದಾ-ಯಿಕ
ಸಾಕಪ್ಪ
ಸಾಕ-ವ-ನಿಗೆ
ಸಾಕಷ್ಟು
ಸಾಕಾ-ಗಿತ್ತು
ಸಾಕಾ-ಗಿ-ತ್ತು-ಇಡೀ
ಸಾಕಾ-ಗು-ತ್ತಿತ್ತು
ಸಾಕಿ-ಸ-ಲಹಿ
ಸಾಕು
ಸಾಕು-ತ್ತಿ-ದ್ದೀಯ
ಸಾಕ್ಷಾ-ತ್ಕ-ರಿ-ಸಿ-ಕೊ-ಳ್ಳ-ಲೂ-ಬ-ಹುದು
ಸಾಕ್ಷಾ-ತ್ಕಾರ
ಸಾಕ್ಷಾ-ತ್ಕಾ-ರ-ಕ್ಕಾಗಿ
ಸಾಕ್ಷಾ-ತ್ಕಾ-ರಕ್ಕೆ
ಸಾಕ್ಷಾ-ತ್ಕಾ-ರದ
ಸಾಕ್ಷಾ-ತ್ಕಾ-ರ-ವನ್ನು
ಸಾಕ್ಷಾ-ತ್ಕಾ-ರವೇ
ಸಾಗ-ತೊ-ಡ-ಗಿತು
ಸಾಗ-ಬೇ-ಕಾ-ಗಿ-ತ್ತು-ಹ-ದಿ-ನೈದು
ಸಾಗರ
ಸಾಗ-ರ-ನಿ-ಹನು
ಸಾಗಿದೆ
ಸಾಗು-ತ್ತಿದೆ
ಸಾಗು-ತ್ತಿದ್ದ
ಸಾದ್ಯವೆ
ಸಾಧ-ಕ-ಬಾ-ಧ-ಕ-ಗ-ಳೇನು
ಸಾಧ-ನ-ವಾ-ಯಿತು
ಸಾಧನೆ
ಸಾಧ-ನೆ-ಗಳ
ಸಾಧ-ನೆಯ
ಸಾಧ-ನೆ-ಯನ್ನೇ
ಸಾಧಾ-ರಣ
ಸಾಧಿ-ಸದೆ
ಸಾಧಿ-ಸ-ಬ-ಲ್ಲೆ-ನೆಂಬ
ಸಾಧಿ-ಸ-ಲಾ-ಗು-ವು-ದಿಲ್ಲ
ಸಾಧಿ-ಸಲು
ಸಾಧಿ-ಸಿ-ಕೊಂಡ
ಸಾಧಿ-ಸಿ-ದ-ವನ
ಸಾಧಿ-ಸಿಯೇ
ಸಾಧು
ಸಾಧು-ಗಳನ್ನು
ಸಾಧು-ಗಳು
ಸಾಧು-ಗ-ಳು-ಭಿ-ಕ್ಷು-ಕರು
ಸಾಧು-ಗಳೂ
ಸಾಧು-ವಾ-ಗಿದ್ದೆ
ಸಾಧು-ವೊ-ಬ್ಬನ
ಸಾಧು-ಸಂ-ನ್ಯಾ-ಸಿ-ಗಳನ್ನು
ಸಾಧ್ಯ
ಸಾಧ್ಯ-ವಾ-ಗದ
ಸಾಧ್ಯ-ವಾ-ಗದೆ
ಸಾಧ್ಯ-ವಾ-ಗ-ಬೇ-ಕಾ-ದರೆ
ಸಾಧ್ಯ-ವಾ-ಗ-ಲಿಲ್ಲ
ಸಾಧ್ಯ-ವಾ-ಗು-ತ್ತಿ-ರ-ಲಿಲ್ಲ
ಸಾಧ್ಯ-ವಾ-ದಷ್ಟು
ಸಾಧ್ಯ-ವಾ-ಯಿತು
ಸಾಧ್ಯ-ವಾ-ಯಿತೇ
ಸಾಧ್ಯ-ವಾ-ಯಿತೋ
ಸಾಧ್ಯ-ವಿ-ರ-ಲಿಲ್ಲ
ಸಾಧ್ಯ-ವಿಲ್ಲ
ಸಾಧ್ಯವೂ
ಸಾಧ್ಯವೆ
ಸಾಧ್ಯವೇ
ಸಾಧ್ಯಾ-ವಾ-ಗು-ತ್ತ-ದೆಯೋ
ಸಾಬೀ-ತು-ಪ-ಡಿ-ಸು-ತ್ತಿ-ದ್ದರು
ಸಾಮಂತ
ಸಾಮರ್ಥ್ಯ
ಸಾಮ-ರ್ಥ್ಯದ
ಸಾಮ-ರ್ಥ್ಯ-ದಿಂದ
ಸಾಮ-ರ್ಥ್ಯ-ವನ್ನು
ಸಾಮ-ರ್ಥ್ಯ-ವನ್ನೂ
ಸಾಮ-ರ್ಥ್ಯ-ವೆಂ-ಥದು
ಸಾಮ-ರ್ಥ್ಯ-ಶಾಲೀ
ಸಾಮಾ-ಜಿಕ
ಸಾಮಾ-ಜಿ-ಕರ
ಸಾಮಾ-ಧಾ-ನ-ಪ-ಡಿ-ಸುತ್ತ
ಸಾಮಾ-ನು-ಗಳನ್ನು
ಸಾಮಾ-ನು-ಗಳೂ
ಸಾಮಾನ್ಯ
ಸಾಮಾ-ನ್ಯ-ನಲ್ಲ
ಸಾಮಾ-ನ್ಯ-ರಿಗೆ
ಸಾಮಾ-ನ್ಯ-ವಾಗಿ
ಸಾಮ್ರಾ-ಜ್ಯ-ವನ್ನು
ಸಾಯಂ-ಕಾಲ
ಸಾಯ-ಲಿಲ್ಲ
ಸಾಯು-ತ್ತಾ-ರೆಯೇ
ಸಾರ-ವನ್ನು
ಸಾರಾಂಶ
ಸಾರಾಂ-ಶ-ವನ್ನು
ಸಾರೋ-ಟಿಗೆ
ಸಾರೋ-ಟಿನ
ಸಾರೋ-ಟಿ-ನಲ್ಲಿ
ಸಾರೋಟು
ಸಾರೋ-ಟು-ಗ-ಳಿಲ್ಲಿ
ಸಾರೋ-ಟು-ವಾಲ
ಸಾರೋ-ಟು-ವಾ-ಲನ
ಸಾರೋ-ಟು-ವಾ-ಲನೇ
ಸಾರೋ-ಟು-ವಾಲಾ
ಸಾರೋ-ಟು-ಸ-ವಾರ
ಸಾರ್
ಸಾರ್ವ-ಜ-ನಿಕ
ಸಾರ್ವ-ಭೌಮ
ಸಾಲದು
ಸಾಲದೆ
ಸಾಲ-ದ್ದಕ್ಕೆ
ಸಾಲಾಗಿ
ಸಾಲಿ-ನಲ್ಲಿ
ಸಾಲಿ-ರು-ವುದು
ಸಾವ-ರಿ-ಸಿ-ಕೊಂಡು
ಸಾವಿನ
ಸಾಸಿವೆ
ಸಾಹಸ
ಸಾಹ-ಸ-ಕೃ-ತ್ಯ-ವನ್ನು
ಸಾಹ-ಸ-ಗಳಲ್ಲಿ
ಸಾಹ-ಸ-ಗಳು
ಸಾಹ-ಸ-ಪ್ರ-ವೃತ್ತಿ
ಸಾಹ-ಸ-ಪ್ರ-ವೃ-ತ್ತಿ-ಯೆ-ನ್ನು-ವುದು
ಸಾಹ-ಸ-ವನ್ನು
ಸಾಹ-ಸ-ವಾಗಿ
ಸಾಹಸಿ
ಸಾಹಿತ್ಯ
ಸಾಹಿ-ತ್ಯ-ಕೃ-ತಿ-ಗಳ
ಸಾಹಿ-ತ್ಯದ
ಸಾಹಿ-ತ್ಯ-ದಲ್ಲಿ
ಸಾಹೇ-ಬರ
ಸಾಹೇ-ಬರು
ಸಿಂಡ-ರಿ-ಸಿ-ಕೊಂಡು
ಸಿಂಪ-ಡಿಸಿ
ಸಿಂಹದ
ಸಿಂಹ-ದಂತೆ
ಸಿಕ್ಕ
ಸಿಕ್ಕಾ
ಸಿಕ್ಕಿ
ಸಿಕ್ಕಿ-ಕೊಂ-ಡ-ವ-ರಂತೆ
ಸಿಕ್ಕಿತು
ಸಿಕ್ಕಿತೋ
ಸಿಕ್ಕಿದ
ಸಿಕ್ಕಿ-ದ್ದನ್ನು
ಸಿಕ್ಕಿ-ದ್ದ-ನ್ನೆಲ್ಲ
ಸಿಕ್ಕಿಲ್ಲ
ಸಿಕ್ಕಿ-ಹಾ-ಕಿ-ಕೊಂ-ಡರೆ
ಸಿಗ-ಬೇ-ಕಲ್ಲ
ಸಿಗ-ಬೇ-ಕಾ-ದರೂ
ಸಿಗ-ಲಿಲ್ಲ
ಸಿಗು-ತ್ತಿತ್ತು
ಸಿಗು-ವನೋ
ಸಿಟ್ಟು
ಸಿಡಿಲ
ಸಿತಾರ್
ಸಿದರು
ಸಿದ್ಧ
ಸಿದ್ಧ-ನಾ-ಗಿಯೇ
ಸಿದ್ಧ-ನಾ-ಗಿರ
ಸಿದ್ಧ-ನಾ-ಗಿರು
ಸಿದ್ಧ-ಳಾ-ದಳು
ಸಿದ್ಧಾಂತ
ಸಿದ್ಧಾಂ-ತ-ಗಳನ್ನು
ಸಿದ್ಧಾಂ-ತ-ಗಳು
ಸಿದ್ಧಾಂ-ತ-ವನ್ನು
ಸಿದ್ಧಿಸಿ
ಸಿದ್ಧಿ-ಸಿ-ಕೊ-ಳ್ಳ-ಬಲ್ಲೆ
ಸಿಪಾ-ಯಿ-ಗಳ
ಸಿಪಾ-ಯಿ-ಗಳಿಂದ
ಸಿಪಾ-ಯಿ-ಗಳು
ಸಿಮು-ಲಿಯಾ
ಸಿಮ್ಲಾ
ಸಿಯೇ
ಸಿರಾ-ಪಿಸ್
ಸಿಲು-ಕಿದ
ಸೀತಾ-ರಾ-ಮರ
ಸೀತಾ-ರಾ-ಮ-ರನ್ನು
ಸೀತಾ-ರಾ-ಮರು
ಸೀತಾ-ರಾ-ಮರ
ಸೀತೆ
ಸೀದಾ
ಸೀಮಿ-ತ-ಗೊ-ಳಿ-ಸಿ-ದ-ವ-ನಲ್ಲ
ಸೀರೆ
ಸೀರೆ-ಯೊಂದು
ಸೀಲರು
ಸುಂದರ
ಸುಂದ-ರ-ವಾ-ಗಿಯೇ
ಸುಂದ-ರ-ವಾದ
ಸುಖ
ಸುಖದ
ಸುಖ-ವಾ-ಗಿ-ರು-ವುದೇ
ಸುಗಂ-ಧ-ವನ್ನು
ಸುಗ-ಮ-ವಾ-ದ-ದ್ದೇ-ನಲ್ಲ
ಸುತ್ತ
ಸುತ್ತ-ಮು-ತ್ತಲ
ಸುತ್ತ-ಲಿನ
ಸುತ್ತಿ
ಸುತ್ತಿ-ಕೊಂ-ಡಿದ್ದ
ಸುತ್ತಿ-ಕೊಂಡು
ಸುತ್ತಿದ
ಸುತ್ತು-ಗಟ್ಟಿ
ಸುತ್ತು-ಹಾಕಿ
ಸುದೀರ್ಘ
ಸುದ್ದಿ
ಸುದ್ದಿ-ಯನ್ನು
ಸುದ್ದಿಯೇ
ಸುಧಾ-ರಣೆ
ಸುಧಾ-ರ-ಣೆ-ಗ-ಳೆಲ್ಲ
ಸುಧಾ-ರ-ಣೆಯ
ಸುಧಾ-ರಿ-ಸಿ-ಕೊಂಡು
ಸುಧಾ-ರಿ-ಸಿ-ಕೊಂ-ಡೆದ್ದು
ಸುಪು-ಷ್ಟ-ವಾದ
ಸುಪ್ಪ-ತ್ತಿಗೆ
ಸುಪ್ರ-ಸಿದ್ಧ
ಸುಮ-ಧು-ರ-ವಾಗಿ
ಸುಮಾರು
ಸುಮ್ಮ
ಸುಮ್ಮ-ನಾ-ಗಿ-ಬಿಟ್ಟ
ಸುಮ್ಮ-ನಾದ
ಸುಮ್ಮ-ನಿದ್ದ
ಸುಮ್ಮ-ನಿ-ದ್ದು-ಬಿಟ್ಟ
ಸುಮ್ಮ-ನಿ-ರಲು
ಸುಮ್ಮ-ನಿ-ರು-ತ್ತಿ-ದ್ದರು
ಸುಮ್ಮ-ನಿ-ರು-ವಂತೆ
ಸುಮ್ಮನೆ
ಸುರ-ಕ್ಷಿ-ತ-ವಾಗಿ
ಸುರಿ-ಯ-ತೊ-ಡ-ಗಿತು
ಸುರಿ-ಯು-ತ್ತಿ-ದ್ದಳು
ಸುರಿ-ಸು-ತ್ತಿ-ದ್ದಾನೆ
ಸುರುಳಿ
ಸುರೇಂದ್ರ
ಸುರೇಂ-ದ್ರ-ನಾಥ
ಸುಲ-ಭ-ದಲ್ಲಿ
ಸುಲ-ಭ-ವಾಗಿ
ಸುಲ-ಭ-ವಾ-ಗಿ-ರ-ಲಿಲ್ಲ
ಸುಲ-ಭ-ವಾ-ಯಿತು
ಸುಲ-ಭವೆ
ಸುಳಿ-ಯಿಂದ
ಸುಳ್ಳಿನ
ಸುವಂ-ತಿ-ರ-ಬೇಕು
ಸುವಾ-ಸನೆ
ಸುಸ್ತಾಗಿ
ಸೂಕ್ತ
ಸೂಕ್ತ-ಮಾ-ರ್ಗ-ದ-ರ್ಶನ
ಸೂಕ್ತ-ವಾಗಿ
ಸೂಕ್ತ-ವಾ-ದು-ದನ್ನು
ಸೂಕ್ಷ-ಬು-ದ್ಧಿಗೆ
ಸೂಕ್ಷ್ಮ-ತೆ-ಯಿಂದ
ಸೂಚಿ-ತ-ವಾ-ಗಿದ್ದ
ಸೂಚಿ-ಸು-ತ್ತಿತ್ತು
ಸೂಚಿ-ಸುವ
ಸೂರ್ಯ
ಸೂರ್ಯ-ಚಂ-ದ್ರರೇ
ಸೂರ್ಯ-ವಂಶ
ಸೂರ್ಯೋ-ದ-ಯಕ್ಕೆ
ಸೂಸು-ತ್ತಿದೆ
ಸೃಷ್ಟಿ-ರ-ಚನಾ
ಸೃಷ್ಟಿ-ಸಿ-ಕೊಂಡ
ಸೆಣ-ಸ-ಬೇ-ಕಾ-ಗು-ತ್ತದೆ
ಸೆಳೆ-ಯು-ತ್ತಿ-ದ್ದುವು
ಸೆಳೆ-ಯು-ವಂ-ತಿತ್ತು
ಸೇದಿ
ಸೇದಿ-ದರು
ಸೇದಿ-ದರೆ
ಸೇದಿ-ದ-ವರು
ಸೇದಿ-ನೋ-ಡಿದ
ಸೇನ
ಸೇನ-ಇ-ವರು
ಸೇನ-ನಂ-ಥ-ವನ
ಸೇನಾ-ಧಿ-ಪ-ತಿ-ಯಂತೆ
ಸೇರಿ
ಸೇರಿ-ಕೊಂಡ
ಸೇರಿ-ಕೊಂ-ಡರೆ
ಸೇರಿ-ಕೊಂ-ಡಿದೆ
ಸೇರಿ-ಕೊಂ-ಡಿ-ರು-ವು-ದಿ-ಲ್ಲವೆ
ಸೇರಿ-ಕೊಂಡು
ಸೇರಿತು
ಸೇರಿತ್ತು
ಸೇರಿದ
ಸೇರಿ-ದರು
ಸೇರಿ-ದ-ವ-ನಾದ್ದ
ಸೇರಿದೆ
ಸೇರಿದ್ದ
ಸೇರಿ-ದ್ದರು
ಸೇರಿ-ಬಿ-ಟ್ಟರು
ಸೇರಿ-ಬಿ-ಡು-ತ್ತಾನೋ
ಸೇರಿವೆ
ಸೇರಿ-ಸ-ಲಾ-ಯಿತು
ಸೇರಿಸಿ
ಸೇರಿ-ಸಿ-ಕೊಂ-ಡರು
ಸೇರಿ-ಸಿ-ಕೊಂ-ಡಿದ್ದ
ಸೇರಿ-ಸಿ-ಕೊಂಡು
ಸೇರಿ-ಸು-ತ್ತಾ-ನೇ-ನಮ್ಮಾ
ಸೇರಿ-ಸುವ
ಸೇರಿ-ಸು-ವು-ದಿಲ್ಲ
ಸೇರಿ-ಸು-ವುದು
ಸೇರುವ
ಸೇವಕ
ಸೇವಾ-ಪ-ರತೆ
ಸೇವೆ-ಶು-ಶ್ರೂ-ಷೆ-ಗಳನ್ನು
ಸೈನಿ-ಕರ
ಸೈನಿ-ಕ-ರನ್ನು
ಸೈನಿ-ಕ-ರನ್ನೇ
ಸೈನಿ-ಕ-ರಿಂ-ದಲೂ
ಸೈನಿ-ಕ-ರಿಗೆ
ಸೈನಿ-ಕರು
ಸೈನಿ-ಕ-ಸ-ಹ-ಜ-ವಾದ
ಸೊಂಟ
ಸೊಂಟಕ್ಕೆ
ಸೊಂಟ-ಕ್ಕೊಂದು
ಸೊಗಸು
ಸೊತ್ತೂ
ಸೊಬಗು
ಸೊರಗಿ
ಸೊರ-ಗಿ-ದರೂ
ಸೋಂಕಿತು
ಸೋಂಬೇ-ರಿ-ಗಳನ್ನೆಲ್ಲ
ಸೋಗಿನ
ಸೋತ
ಸೋದ-ರ-ಸೋ-ದ-ರಿ-ಯ-ರಲ್ಲಿ
ಸೋದ-ರ-ನಾದ
ಸೋದ-ರ-ಮಾವ
ಸೋಮ-ವಾರ
ಸೋಮ-ವಾ-ರ-ಗ-ಳಂದು
ಸೋಮ-ವಾ-ರ-ದಂದು
ಸೋಮಾರಿ
ಸೋಮಾ-ರಿ-ಗ-ಳಾದ
ಸೋರಿತು
ಸೋಲ-ರಿ-ಯದ
ಸೋಲು-ಗಳನ್ನು
ಸೌಂದರ್ಯ
ಸೌಂದ-ರ್ಯ-ವನ್ನು
ಸೌಜ-ನ್ಯಾದಿ
ಸೌಧ-ಗ-ಳಲ್ಲೇ
ಸೌಭಾ-ಗ್ಯ-ಗಳೂ
ಸ್ಕಾಟಿಷ್
ಸ್ಟೇಜಿನ
ಸ್ಟ್ರಾಂಗ್
ಸ್ತಬ್ಧ-ವಾಗಿ
ಸ್ತರ-ದಲ್ಲಿ
ಸ್ತ್ರೀಯರ
ಸ್ತ್ರೀವಿ-ದ್ಯಾ-ಭ್ಯಾ-ಸ-ವನ್ನು
ಸ್ಥಳ
ಸ್ಥಳಕ್ಕೆ
ಸ್ಥಳ-ಗ-ಳಾದ
ಸ್ಥಳ-ಗ-ಳಿಗೆ
ಸ್ಥಾನ-ದಲ್ಲಿ
ಸ್ಥಾನ-ವನ್ನು
ಸ್ಥಾಪ-ಕ-ರಾದ
ಸ್ಥಾಪಿ-ತ-ವಾದ
ಸ್ಥಾಪಿ-ಸಿದ
ಸ್ಥಾಪಿ-ಸಿ-ದ್ದ-ವನು
ಸ್ಥಾವ-ರ-ಗ-ಳಿ-ದ್ದವು
ಸ್ಥಾವ-ರ-ವೊಂ-ದನ್ನು
ಸ್ಥಿತಿ
ಸ್ಥಿತಿ-ಗ-ತಿ-ಗಳ
ಸ್ಥಿತಿಗೆ
ಸ್ಥಿತಿ-ಗೇ-ರಲು
ಸ್ಥಿತಿ-ಯನ್ನು
ಸ್ಥಿತಿ-ಯಲ್ಲಿ
ಸ್ಥಿತಿ-ಯ-ಲ್ಲಿ-ದ್ದಾಗ
ಸ್ಥಿತಿ-ಯ-ಲ್ಲಿ-ದ್ದೆನೋ
ಸ್ಥಿತಿ-ಯ-ಲ್ಲಿ-ರ-ಲಿಲ್ಲ
ಸ್ಥಿತಿ-ಯಿಂದ
ಸ್ಥಿರತೆ
ಸ್ಥಿರ-ವಾಗಿ
ಸ್ಥೂಲ-ವಾಗಿ
ಸ್ನಾನ
ಸ್ನೇಹ
ಸ್ನೇಹಿತ
ಸ್ನೇಹಿ-ತನ
ಸ್ನೇಹಿ-ತ-ನೊಂ-ದಿಗೆ
ಸ್ನೇಹಿ-ತ-ನೊಬ್ಬ
ಸ್ನೇಹಿ-ತರ
ಸ್ನೇಹಿ-ತ-ರ-ನ್ನಾಗಿ
ಸ್ನೇಹಿ-ತ-ರನ್ನು
ಸ್ನೇಹಿ-ತ-ರ-ನ್ನೆಲ್ಲ
ಸ್ನೇಹಿ-ತ-ರನ್ನೇ
ಸ್ನೇಹಿ-ತ-ರಲ್ಲಿ
ಸ್ನೇಹಿ-ತ-ರಲ್ಲೇ
ಸ್ನೇಹಿ-ತ-ರಿಂ-ದಲೂ
ಸ್ನೇಹಿ-ತ-ರಿಗೆ
ಸ್ನೇಹಿ-ತರು
ಸ್ನೇಹಿ-ತರೂ
ಸ್ನೇಹಿ-ತ-ರೆಲ್ಲ
ಸ್ನೇಹಿ-ತ-ರೊಂ-ದಿಗೆ
ಸ್ನೇಹಿ-ತ-ರೊ-ಡನೆ
ಸ್ಪಂದಿ-ಸುತ್ತ
ಸ್ಪಷ್ಟ-ಕ-ಲ್ಪನೆ
ಸ್ಪಷ್ಟ-ವಾಗಿ
ಸ್ಫುರಣೆ
ಸ್ಫುರ-ಣೆಗೆ
ಸ್ಫುರ-ಣೆ-ಯಾ-ಗ-ದಿ-ದ್ದಲ್ಲಿ
ಸ್ಫುರಿ-ಸುವ
ಸ್ಫೋಟ-ಗೊ-ಳ್ಳು-ತ್ತಿತ್ತು
ಸ್ಮರಿ-ಸ-ಬ-ಹು-ದಾ-ಗಿದೆ
ಸ್ಲೇಟನ್ನು
ಸ್ವಂತ
ಸ್ವಂತದ
ಸ್ವಂತದ್ದೇ
ಸ್ವಂತಿಕೆ
ಸ್ವಂತಿ-ಕೆ-ಯಿಂದ
ಸ್ವತಂತ್ರ
ಸ್ವತಂ-ತ್ರ-ನಾಗಿ
ಸ್ವತಂ-ತ್ರ-ನಾದ
ಸ್ವತಃ
ಸ್ವಪ್ನ-ದ-ರ್ಶ-ನದ
ಸ್ವಪ್ರ-ಯ-ತ್ನ-ದಿಂದ
ಸ್ವಭಾವ
ಸ್ವಭಾ-ವಕ್ಕೆ
ಸ್ವಭಾ-ವ-ಗಳು
ಸ್ವಭಾ-ವತಃ
ಸ್ವಭಾ-ವದ
ಸ್ವಭಾ-ವ-ದಲ್ಲಿ
ಸ್ವಭಾ-ವ-ದ-ವ-ರಾ-ಗಿ-ರು-ತ್ತಾರೆ
ಸ್ವಭಾ-ವ-ವೆಂ-ಥದು
ಸ್ವಭಾ-ವ-ವೆ-ನ್ನು-ವುದು
ಸ್ವಯಂ
ಸ್ವರ್ಣ
ಸ್ವರ್ಣ-ಪ-ದಕ
ಸ್ವರ್ಣ-ಮುಖಿ
ಸ್ವಲ್ಪ
ಸ್ವಲ್ಪ-ಮಾತ್ರ
ಸ್ವಲ್ಪವೂ
ಸ್ವಲ್ಪ-ಸ್ವಲ್ಪ
ಸ್ವಲ್ಪ-ಹೊ-ತ್ತಿನ
ಸ್ವಷ್ಟ
ಸ್ವಷ್ಟ-ವಾಗಿ
ಸ್ವಷ್ಟ-ವಾ-ಗು-ತ್ತದೆ
ಸ್ವಸ್ಥ
ಸ್ವಾಗ-ತಿಸಿ
ಸ್ವಾಗ-ತಿ-ಸಿ-ದ-ರೆ-ನ್ನ-ಬ-ಹುದು
ಸ್ವಾಗ-ತಿ-ಸು-ತ್ತಿ-ದ್ದಳು
ಸ್ವಾತಂತ್ರ್ಯ
ಸ್ವಾತಂ-ತ್ರ್ಯಕ್ಕೆ
ಸ್ವಾತಂ-ತ್ರ್ಯ-ಪ್ರಿ-ಯರು
ಸ್ವಾತಂ-ತ್ರ್ಯ-ಪ್ರೇಮ
ಸ್ವಾತಂ-ತ್ರ್ಯ-ವನ್ನು
ಸ್ವಾಧೀನ
ಸ್ವಾಧೀ-ನ-ಪ-ಡಿ-ಸಿ-ಕೊಂ-ಡಿದ್ದ
ಸ್ವಾಧೀ-ನ-ಪ-ಡಿ-ಸಿ-ಕೊ-ಳ್ಳು-ವ-ವ-ರೆಗೆ
ಸ್ವಾಧೀ-ನ-ಪ-ಡಿ-ಸಿ-ಕೊ-ಳ್ಳು-ವು-ದ-ಕ್ಕಾಗಿ
ಸ್ವಾಭಿ-ಮಾ-ನದ
ಸ್ವಾಭಿ-ಮಾ-ನ-ಪ್ರಜ್ಞೆ
ಸ್ವಾಮಿ
ಸ್ವಾಮೀ
ಸ್ವಾರ-ಸ್ಯ-ಕರ
ಸ್ವಾರ-ಸ್ಯ-ಕ-ರ-ವಾಗಿ
ಸ್ವಾರ-ಸ್ಯ-ವನ್ನು
ಸ್ವಾರ-ಸ್ಯ-ವಾಗಿ
ಸ್ವೀಕ-ರಿ-ಸ-ಲಾ-ರಂ-ಭಿ-ಸಿದ
ಸ್ವೀಕ-ರಿ-ಸಿದ
ಸ್ವೀಕ-ರಿ-ಸಿ-ಯಾ-ಗಿ-ಬಿ-ಟ್ಟಿದೆ
ಸ್ವೀಕ-ರಿ-ಸು-ವುದನ್ನು
ಸ್ವೇಚ್ಛಾ-ಚಾರಿ
ಹಂಗನ್ನೂ
ಹಂಗನ್ನೇ
ಹಂಬಲ
ಹಂಬ-ಲದ
ಹಂಬ-ಲ-ವನ್ನು
ಹಂಬ-ಲ-ವ-ಲ್ಲವೆ
ಹಕ್ಕು-ಗ-ಳಿ-ಗಾಗಿ
ಹಕ್ಕು-ಗಳೂ
ಹಗ-ಲಿ-ನಲ್ಲಿ
ಹಗಲು
ಹಗ-ಲೂ-ರಾತ್ರಿ
ಹಗ-ಲೆಲ್ಲ
ಹಗ್ಗ
ಹಗ್ಗ-ರಾ-ಟೆ-ಗಳ
ಹಚ್ಚಿ
ಹಚ್ಚಿ-ಕೊಂಡು
ಹಚ್ಚಿ-ನೋ-ಡದೆ
ಹಚ್ಚಿ-ಬಿ-ಟ್ಟಿ-ದ್ದಾನೆ
ಹಜಾ-ರ-ದಲ್ಲಿ
ಹಟ
ಹಟ-ತೊಟ್ಟು
ಹಠ
ಹಡು-ಗರು
ಹಡುಗು
ಹಣ
ಹಣ-ವನ್ನು
ಹಣ-ವನ್ನೇ
ಹಣವೂ
ಹಣೆ
ಹಣೆಗೆ
ಹಣೆ-ಯಲ್ಲೂ
ಹತಾ-ಶ-ನಾ-ಗಿ-ರ-ಲಿಲ್ಲ
ಹತೋ-ಟಿಗೆ
ಹತ್ತ-ಬಾ-ರದು
ಹತ್ತ-ಬೇ-ಡವೋ
ಹತ್ತ-ಬೇ-ಡ್ರಪ್ಪ
ಹತ್ತಿ
ಹತ್ತಿ-ಕೊಂ-ಡರೆ
ಹತ್ತಿದ
ಹತ್ತಿ-ದ-ವರ
ಹತ್ತಿರ
ಹತ್ತಿ-ರಕ್ಕೆ
ಹತ್ತಿ-ರದ
ಹತ್ತಿ-ರ-ದಲ್ಲೇ
ಹತ್ತಿ-ರವೂ
ಹತ್ತಿ-ರವೇ
ಹತ್ತು
ಹತ್ತು-ಹ-ನ್ನೊಂದು
ಹತ್ತೂ
ಹತ್ತೊಂ-ಬ-ತ್ತನೇ
ಹದಿ-ನಾ-ರರ
ಹದಿ-ನಾ-ಲ್ಕನೇ
ಹದಿ-ನಾಲ್ಕು
ಹದಿ-ನೆಂ-ಟು-ಹ-ತ್ತೊಂ-ಬ-ತ್ತ-ನೆಯ
ಹದಿ-ನೇ-ಳರ
ಹದಿ-ಮೂ-ರು-ಹ-ದಿ-ನಾಲ್ಕು
ಹದೀ-ನೇ-ಳ-ನೆಯ
ಹನು-ಮಂತ
ಹನು-ಮಂ-ತನ
ಹನು-ಮಂ-ತ-ನನ್ನು
ಹನು-ಮಂ-ತನು
ಹನು-ಮಂ-ತ-ನೇಕೋ
ಹನ್ನೆ-ರ-ಡು-ಹ-ದಿ-ನಾಲ್ಕು
ಹಬ್ಬ
ಹಬ್ಬದ
ಹರಕೆ
ಹರ-ಡಿದ
ಹರ-ಣೆಗೆ
ಹರ-ಮೋ-ಹಿನಿ
ಹರ-ಸಿದ
ಹರ-ಸಿ-ದರು
ಹರ-ಸು-ತ್ತಿ-ದ್ದರು
ಹರಿ
ಹರಿತ
ಹರಿ-ದಾ-ಡಿ-ಕೊಂಡು
ಹರಿ-ದಾಸ
ಹರಿ-ದಾ-ಸನ
ಹರಿ-ದಾ-ಸ-ನಿಗೆ
ಹರಿ-ದಾಸ್
ಹರಿ-ದಿನ
ಹರಿ-ದಿ-ರು-ತ್ತಿತ್ತು
ಹರಿದು
ಹರಿ-ಯನ್ನು
ಹರಿ-ಯು-ತ್ತಿದ್ದ
ಹರಿ-ಯು-ವಂ-ತಿ-ರು-ತ್ತಿತ್ತು
ಹರಿ-ಯು-ವು-ದ-ರ-ಲ್ಲಿದೆ
ಹರಿ-ಯೇನೋ
ಹರಿ-ಸ-ಬಲ್ಲ
ಹರಿ-ಸುವ
ಹಲ-ವಾರು
ಹಲವು
ಹಲ್ಲು
ಹಳ-ಬ-ರು-ಹೊ-ಸ-ಬರು
ಹಳಿ-ದನೋ
ಹವ್ಯಾ-ಸ-ವನ್ನು
ಹಸಿರು
ಹಸಿ-ವಿ-ನಿಂದ
ಹಸಿವು
ಹಸಿ-ವೆ-ಯಿಂದ
ಹಸು-ವಿತ್ತು
ಹಸ್ತ-ಸಾ-ಮು-ದ್ರಿಕ
ಹಾಕ-ಬ-ಲ್ಲ-ವ-ರನ್ನು
ಹಾಕ-ಬೇ-ಕಾ-ಗುತ್ತೆ
ಹಾಕ-ಲಿಲ್ಲ
ಹಾಕಿ
ಹಾಕಿ-ಕೊಂಡ
ಹಾಕಿ-ಕೊಂ-ಡಿತ್ತು
ಹಾಕಿ-ಕೊಂ-ಡಿದ್ದ
ಹಾಕಿ-ಕೊಂ-ಡಿ-ರು-ತ್ತಿ-ರ-ಲಿಲ್ಲ
ಹಾಕಿ-ಕೊಂಡು
ಹಾಕಿ-ಕೊ-ಟ್ಟರು
ಹಾಕಿ-ಕೊ-ಳ್ಳು-ತ್ತಿದ್ದ
ಹಾಕಿ-ದಂ-ತೆಯೇ
ಹಾಕಿದ್ದ
ಹಾಕಿ-ದ್ದರು
ಹಾಕಿ-ಬಿಟ್ಟ
ಹಾಕಿ-ಬಿ-ಟ್ಟಿದ್ದ
ಹಾಕು-ತ್ತಿದ್ದ
ಹಾಕುವು
ಹಾಕು-ವುದು
ಹಾಗಾ-ದರೆ
ಹಾಗಿ-ದ್ದರೆ
ಹಾಗಿ-ರಲಿ
ಹಾಗೂ
ಹಾಗೆ
ಹಾಗೆಂದು
ಹಾಗೆಯೇ
ಹಾಗೆಲ್ಲ
ಹಾಗೆ-ಹಾ-ಗೆಯೇ
ಹಾಗೇ
ಹಾಜರಿ
ಹಾಡ-ತೊ-ಡ-ಗಿ-ದ-ನೆಂ-ದರೆ
ಹಾಡನ್ನು
ಹಾಡ-ಬ-ಲ್ಲರು
ಹಾಡ-ಬ-ಲ್ಲ-ವ-ನಾ-ಗಿದ್ದ
ಹಾಡಲು
ಹಾಡಿ
ಹಾಡಿ-ಕೊಂಡು
ಹಾಡಿ-ಕೊ-ಳ್ಳು-ತ್ತಿದ್ದ
ಹಾಡಿದ
ಹಾಡಿನ
ಹಾಡಿ-ನಲ್ಲಿ
ಹಾಡಿರಿ
ಹಾಡು
ಹಾಡು-ಗಳನ್ನು
ಹಾಡು-ಗಾರ
ಹಾಡು-ಗಾ-ರನ
ಹಾಡು-ಗಾ-ರ-ನಾಗಿ
ಹಾಡು-ಗಾ-ರರು
ಹಾಡು-ಗಾ-ರಿ-ಕೆ-ಯನ್ನು
ಹಾಡು-ಗಾ-ರಿ-ಕೆ-ಯ-ಲ್ಲ-ಲ್ಲದೆ
ಹಾಡು-ಗಾ-ರಿ-ಕೆ-ಯಲ್ಲೇ
ಹಾಡು-ಗಾ-ರಿ-ಕೆಯೇ
ಹಾಡುತ್ತ
ಹಾಡು-ತ್ತಲೇ
ಹಾಡು-ತ್ತಾನೆ
ಹಾಡು-ತ್ತಿದ್ದ
ಹಾಡು-ತ್ತಿ-ದ್ದ-ಒ-ಮ್ಮೊಮ್ಮೆ
ಹಾಡು-ತ್ತಿ-ದ್ದರೆ
ಹಾಡೊಂ-ದನ್ನು
ಹಾನಿ
ಹಾರ
ಹಾರಾ-ಡುತ್ತ
ಹಾರಿ
ಹಾರಿದ
ಹಾರಿಯೇ
ಹಾರಿಸಿ
ಹಾರಿ-ಹೋಗಿ
ಹಾರು-ತ್ತಿ-ರುವ
ಹಾರು-ವುದು
ಹಾರೈಕೆ
ಹಾಳಾ-ಗಿ-ಹೋ-ಗು-ತ್ತಿತ್ತು
ಹಾಳಾ-ಗುವ
ಹಾವು
ಹಾಸಿಗೆ
ಹಾಸಿದ
ಹಾಸ್ಯ-ಕ್ಕೊಂದು
ಹಾಸ್ಯದ
ಹಾಸ್ಯ-ಪ್ರ-ವೃತ್ತಿ
ಹಾಸ್ಯ-ಪ್ರ-ವೃ-ತ್ತಿ-ಯೇನೂ
ಹಾಸ್ಯ-ಪ್ರಿ-ಯ-ನೆಂ-ದರೆ
ಹಾಸ್ಯ-ಮಯ
ಹಿಂಡು-ವುದು
ಹಿಂದಕ್ಕೆ
ಹಿಂದಿ
ಹಿಂದಿಈ
ಹಿಂದಿನ
ಹಿಂದಿ-ರುಗ
ಹಿಂದಿ-ರು-ಗ-ಲಿಲ್ಲ
ಹಿಂದಿ-ರುಗಿ
ಹಿಂದಿ-ರು-ಗಿದ
ಹಿಂದಿ-ರು-ಗಿ-ದರು
ಹಿಂದಿ-ರು-ಗಿ-ದಾಗ
ಹಿಂದಿ-ರು-ಗಿ-ಸು-ವು-ದಿಲ್ಲ
ಹಿಂದಿ-ರು-ಗು-ವಾಗ
ಹಿಂದಿ-ರು-ಗು-ವು-ದ-ರೊ-ಳಗೆ
ಹಿಂದೀ
ಹಿಂದು-ಗಳ
ಹಿಂದು-ಮುಂದು
ಹಿಂದು-ಳಿದ
ಹಿಂದೂ
ಹಿಂದೂ-ಇ-ಸ್ಲಾಂ
ಹಿಂದೂ-ಗಳ
ಹಿಂದೂ-ಧರ್ಮ
ಹಿಂದೂ-ಧ-ರ್ಮಕ್ಕೆ
ಹಿಂದೂ-ಧ-ರ್ಮದ
ಹಿಂದೂ-ಧ-ರ್ಮ-ದಲ್ಲಿ
ಹಿಂದೂ-ಧ-ರ್ಮ-ದಿಂದ
ಹಿಂದೂ-ರಾ-ಷ್ಟ್ರದ
ಹಿಂದೂ-ಸಂ-ಪ್ರ-ದಾ-ಯ-ಗಳಲ್ಲಿ
ಹಿಂದೂ-ಸ-ಮಾ-ಜವು
ಹಿಂದೂ-ಸ್ಥಾ-ನದ
ಹಿಂದೂ-ಸ್ಥಾ-ನ-ದಲ್ಲೆಲ್ಲ
ಹಿಂದೆ
ಹಿಂದೆ-ಮುಂದೆ
ಹಿಂದೆ-ಯಷ್ಟೆ
ಹಿಂದೆಯೇ
ಹಿಂದೊಂದು
ಹಿಂದೊಮ್ಮೆ
ಹಿಂಭಾ-ಗ-ದಲ್ಲಿ
ಹಿಗ್ಗು-ತ್ತಿತ್ತು
ಹಿಡ-ಕೊಂಡು
ಹಿಡಿ-ತ-ದಿಂದ
ಹಿಡಿದ
ಹಿಡಿ-ದಿದ್ದ
ಹಿಡಿದು
ಹಿಡಿ-ದು-ಕೊಂ-ಡಳು
ಹಿಡಿ-ದು-ಕೊಂ-ಡಿ-ದ್ದ-ಳೆಂ-ದರೆ
ಹಿಡಿ-ದು-ಕೊಂಡು
ಹಿಡಿ-ದು-ಕೊ-ಳ್ಳುತ್ತೆ
ಹಿಡಿದೊ
ಹಿಡಿ-ಯಲು
ಹಿಡಿಸಿ
ಹಿಡೀರಿ
ಹಿಡು-ದು-ಕೊಂಡು
ಹಿತ
ಹಿತ-ವಾದ
ಹಿತ-ಸಾ-ಧನೆ
ಹಿತೈ-ಷಿ-ಗಳು
ಹಿನ್ನಲೆ
ಹಿರಿಯ
ಹಿರಿ-ಯ-ನಾದ
ಹಿರಿ-ಯರ
ಹಿರಿ-ಯರು
ಹಿಸು-ಕಿ-ಹಾ-ಕಿ-ಬಿ-ಡು-ತ್ತದೆ
ಹಿಸು-ಕಿ-ಹಾ-ಕಿ-ಬಿ-ಡುತ್ತೆ
ಹೀಗ-ಳೆಯು
ಹೀಗಿದೆ
ಹೀಗಿದ್ದ
ಹೀಗಿ-ದ್ದರೂ
ಹೀಗಿ-ರು-ವಾಗ
ಹೀಗೆ
ಹೀಗೆಂ-ದಾಗ
ಹೀಗೆ-ನ್ನು-ತ್ತಾರೆ
ಹೀಗೆ-ನ್ನು-ತ್ತಿದ್ದೀ
ಹೀಗೆಯೇ
ಹೀಗೆಲ್ಲ
ಹೀಗೇ
ಹೀಗೋ
ಹೀಗ್
ಹೀನ
ಹುಕ್ಕದ
ಹುಕ್ಕಾ
ಹುಕ್ಕಾ-ಕೊ-ಳ-ವೆ-ಗಳನ್ನೂ
ಹುಕ್ಕಾ-ವನ್ನು
ಹುಚ್ಚುಚ್ಚಾ
ಹುಚ್ಚೆ-ದ್ದು-ಬಿ-ಟ್ಟಿ-ದ್ದಾರೆ
ಹುಟ್ಟಿ-ಕೊಂಡ
ಹುಟ್ಟಿ-ಕೊಂ-ಡಿತು
ಹುಟ್ಟಿ-ಕೊಂ-ಡಿತ್ತು
ಹುಟ್ಟಿ-ಕೊ-ಳ್ಳು-ತ್ತಿತ್ತು
ಹುಟ್ಟಿ-ನಿಂದ
ಹುಟ್ಟಿ-ಬಂದ
ಹುಟ್ಟಿ-ಬಿ-ಟ್ಟಿತು
ಹುಟ್ಟಿ-ಬಿ-ಟ್ಟಿತ್ತು
ಹುಟ್ಟು-ಗು-ಣವೇ
ಹುಡ-ಗು-ರೆಲ್ಲ
ಹುಡು
ಹುಡು-ಕ-ತೊ-ಡ-ಗಿ-ದರು
ಹುಡು-ಕಾಡಿ
ಹುಡು-ಕಿ-ಕೊಂಡು
ಹುಡು-ಕಿ-ನೋ-ಡಿ-ದರೂ
ಹುಡು-ಕು-ತ್ತಿದ್ದ
ಹುಡುಗ
ಹುಡು-ಗನ
ಹುಡು-ಗ-ನನ್ನು
ಹುಡು-ಗ-ನಾ-ಗಿ-ದ್ದಾ-ಗಲೇ
ಹುಡು-ಗ-ನಾ-ಗಿ-ರ-ಬ-ಹುದು
ಹುಡು-ಗ-ನಾ-ದರೂ
ಹುಡು-ಗ-ನಿಗೆ
ಹುಡು-ಗ-ನಿದ್ದ
ಹುಡು-ಗನೋ
ಹುಡು-ಗರ
ಹುಡು-ಗ-ರಂತೆ
ಹುಡು-ಗ-ರನ್ನು
ಹುಡು-ಗ-ರನ್ನೂ
ಹುಡು-ಗ-ರ-ಲ್ಲವೆ
ಹುಡು-ಗ-ರಲ್ಲಿ
ಹುಡು-ಗ-ರಲ್ಲೇ
ಹುಡು-ಗ-ರಷ್ಟೆ
ಹುಡು-ಗರಿ
ಹುಡು-ಗ-ರಿಂದ
ಹುಡು-ಗ-ರಿಗೂ
ಹುಡು-ಗ-ರಿಗೆ
ಹುಡು-ಗ-ರಿ-ಬ್ಬರೂ
ಹುಡು-ಗ-ರಿಲ್ಲ
ಹುಡು-ಗರು
ಹುಡು-ಗ-ರು-ನಾ-ವೆಲ್ಲ
ಹುಡು-ಗರೂ
ಹುಡು-ಗ-ರೆಲ್ಲ
ಹುಡು-ಗ-ರೆ-ಲ್ಲ-ರಿಗೂ
ಹುಡು-ಗ-ರೊಂ-ದಿಗೆ
ಹುಡುಗಿ
ಹುಡು-ಗಿಗೆ
ಹುಡು-ಗು-ತ-ನ-ವಿಲ್ಲ
ಹುಡುಗ್ರಾ
ಹುಣ್ಣಾ-ಗು-ವಷ್ಟು
ಹುಬ್ಬು-ಗಳ
ಹುಬ್ಬು-ಗಳನ್ನು
ಹುರಿ-ದುಂ-ಬಿ-ಸಿ-ದರು
ಹುರು-ಪಿ-ನಿಂದ
ಹೂಡ-ಬೇಕು
ಹೂಡಿಯೂ
ಹೂವಿನ
ಹೂವು-ಗಳನ್ನೆಲ್ಲ
ಹೃತ್ಪೂ-ರ್ವಕ
ಹೃತ್ಪೂ-ರ್ವ-ಕ-ವಾಗಿ
ಹೃದಯ
ಹೃದ-ಯದ
ಹೃದ-ಯ-ದ-ಲ್ಲಾ-ಗಲೇ
ಹೃದ-ಯ-ದಲ್ಲಿ
ಹೃದ-ಯ-ದಿಂದ
ಹೃದ-ಯ-ವನ್ನು
ಹೃದ್ಗತ
ಹೆಂಗಸು
ಹೆಂಡ-ತಿ-ಮ-ಕ್ಕ-ಳನ್ನೂ
ಹೆಂಡ-ತಿ-ಯ-ಲ್ಲವೆ
ಹೆಂಡ-ತಿ-ಯಾಗಿ
ಹೆಗ್ಗು-ರುತು
ಹೆಚ್ಚಾಗಿ
ಹೆಚ್ಚಾ-ಗಿಯೇ
ಹೆಚ್ಚಿತು
ಹೆಚ್ಚಿನ
ಹೆಚ್ಚಿ-ನವು
ಹೆಚ್ಚು
ಹೆಚ್ಚು-ಕ-ಡಿ-ಮೆ-ಯಾ-ಗಿ-ಬಿ-ಟ್ಟಿ-ದ್ದರೂ
ಹೆಚ್ಚೇ
ಹೆಜ್ಜೆ
ಹೆಡೆ
ಹೆಡೆ-ಬಿ-ಚ್ಚಿ-ಕೊಂಡು
ಹೆಣ್ಣು-ಒ-ಬ್ಬಳು
ಹೆಣ್ಣು-ಮ-ಕ್ಕ-ಳೆಂದು
ಹೆತ್ತ
ಹೆತ್ತ-ತಾ-ಯಿಗೆ
ಹೆತ್ತ-ವ-ರಿಗೆ
ಹೆತ್ತ-ವ-ಳಾದ
ಹೆದರಿ
ಹೆದ-ರಿ-ಕೆ-ಯಾ-ಯಿತು
ಹೆದ-ರಿ-ಕೊ-ಳ್ಳ-ಬೇಡ
ಹೆದ-ರಿ-ಕೊ-ಳ್ಳು-ವುದೂ
ಹೆದ-ರಿ-ಬಿ-ಟ್ಟಿ-ದ್ದರು
ಹೆದ-ರಿ-ಸಲು
ಹೆದ-ರಿಸಿ
ಹೆದ-ರಿ-ಸಿ-ದರೂ
ಹೆದ-ರಿ-ಸಿಯೂ
ಹೆದ-ರಿ-ಸು-ತ್ತಿ-ದ್ದಳು
ಹೆಮ್ಮೆ
ಹೆಮ್ಮೆ-ಪಟ್ಟು
ಹೆಮ್ಮೆಯ
ಹೆಮ್ಮೆ-ಯಿಂದ
ಹೆಸ-ರನ್ನೇ
ಹೆಸ-ರಿಗೆ
ಹೆಸ-ರಿ-ನಲ್ಲಿ
ಹೆಸರು
ಹೆಸ-ರು-ಕೀ-ರ್ತಿ-ಗಳನ್ನು
ಹೆಸ-ರು-ಗಳು
ಹೆಸರೇ
ಹೇ
ಹೇಗಾ-ದರೂ
ಹೇಗಿತ್ತೋ
ಹೇಗಿ-ದ್ದರೆ
ಹೇಗಿ-ರು-ತ್ತ-ದೆಂದು
ಹೇಗೆ
ಹೇಗೋ
ಹೇಯ್
ಹೇರಳ
ಹೇರ-ಳ-ವಾಗಿ
ಹೇರಿ-ಕೊಂಡು
ಹೇರು-ವು-ದರ
ಹೇರು-ವುದು
ಹೇಳ-ಬ-ರು-ವಂ-ತಿಲ್ಲ
ಹೇಳ-ಲಾಗು
ಹೇಳ-ಲಾ-ರಂ-ಭಿ-ಸಿ-ದ-ರು-ಸ-ಕಾ-ಲ-ದಲ್ಲಿ
ಹೇಳ-ಲಾ-ರದ
ಹೇಳ-ಲಿಲ್ಲ
ಹೇಳ-ಲೇ-ಬೇ-ಕಾ-ಗಿಲ್ಲ
ಹೇಳ-ಹೋ-ಗಿ-ರ-ಲಿಲ್ಲ
ಹೇಳಾ-ರಂ-ಭಿ-ಸಿದ
ಹೇಳಿ
ಹೇಳಿ-ಕ-ಳಿ-ಸಿದ
ಹೇಳಿ-ಕೊಟ್ಟ
ಹೇಳಿ-ಕೊ-ಟ್ಟರು
ಹೇಳಿ-ಕೊ-ಳ್ಳ-ಬಹು
ಹೇಳಿದ
ಹೇಳಿ-ದರು
ಹೇಳಿ-ದಳು
ಹೇಳಿ-ದ-ವನೇ
ಹೇಳಿದ್ದ
ಹೇಳಿ-ದ್ದನ್ನು
ಹೇಳಿ-ದ್ದರು
ಹೇಳಿ-ದ್ದರೋ
ಹೇಳಿದ್ದು
ಹೇಳಿ-ಬಿಟ್ಟ
ಹೇಳಿ-ಬಿ-ಟ್ಟ-ತಾನು
ಹೇಳಿ-ಬಿ-ಟ್ಟರು
ಹೇಳಿ-ಬಿ-ಟ್ಟಾಗ
ಹೇಳಿ-ಬಿ-ಡು-ತ್ತಿದ್ದ
ಹೇಳಿ-ಯೇ-ಬಿಟ್ಟ
ಹೇಳಿ-ರ-ಬ-ಹುದು
ಹೇಳಿ-ಸಿ-ಕೊ-ಳ್ಳುವ
ಹೇಳಿ-ಸುವ
ಹೇಳು
ಹೇಳುತ್ತ
ಹೇಳು-ತ್ತಾನೆ
ಹೇಳು-ತ್ತಾರೆ
ಹೇಳು-ತ್ತಿದೆ
ಹೇಳು-ತ್ತಿದ್ದ
ಹೇಳು-ತ್ತಿ-ದ್ದಂತೆ
ಹೇಳು-ತ್ತಿ-ದ್ದರು
ಹೇಳು-ತ್ತಿ-ದ್ದಳು
ಹೇಳು-ತ್ತಿ-ದ್ದಾನೆ
ಹೇಳುವ
ಹೇಳು-ವಂತೆ
ಹೇಳು-ವ-ವ-ರಿ-ಲ್ಲ-ವಲ್ಲ
ಹೇಳು-ವ-ಷ್ಟ-ರಲ್ಲಿ
ಹೇಳು-ವುದನ್ನು
ಹೇಳು-ವು-ದಲ್ಲೂ
ಹೇಳು-ವು-ದಾ-ದರೆ
ಹೇಳು-ವುದು
ಹೇಸ್ಟೀ
ಹೈಕೋ-ರ್ಟಿನ
ಹೊಂಗ-ನ-ಸಿಗೆ
ಹೊಂಗ-ನಸು
ಹೊಂದದ
ಹೊಂದ-ಲಿ-ದ್ದರು
ಹೊಂದಿದೆ
ಹೊಂದಿದ್ದು
ಹೊಂದಿ-ರ-ಬೇಕು
ಹೊಂದು-ತ್ತಿ-ರುವ
ಹೊಂದುವ
ಹೊಕ್ಕ
ಹೊಗ-ಳಿ-ದರು
ಹೊಗೆ
ಹೊಚ್ಚ
ಹೊಟ್ಟೆ
ಹೊಟ್ಟೆ-ತುಂಬ
ಹೊಡೆ-ದರು
ಹೊಡೆ-ಯಲು
ಹೊಡೆ-ಯು-ತ್ತಿ-ದ್ದರೆ
ಹೊತ್ತಾ-ಗಿ-ಹೋ-ಯಿತು
ಹೊತ್ತಾ-ದರೂ
ಹೊತ್ತಾ-ಯಿ-ತೆನ್ನಿ
ಹೊತ್ತಿ-ಗಾ-ಗಲೇ
ಹೊತ್ತಿಗೆ
ಹೊತ್ತಿ-ನಿಂದ
ಹೊತ್ತು
ಹೊತ್ತು-ಕೊಂಡು
ಹೊಮ್ಮಿ
ಹೊಮ್ಮು-ತ್ತಿತ್ತು
ಹೊರಕ್ಕೆ
ಹೊರ-ಗ-ಡೆಯೇ
ಹೊರ-ಗಿ-ನಿಂದ
ಹೊರಗೆ
ಹೊರ-ಗೆ-ಸೆದು
ಹೊರಟ
ಹೊರ-ಟರು
ಹೊರ-ಟರೆ
ಹೊರ-ಟಳು
ಹೊರ-ಟ-ವರು
ಹೊರ-ಟಿತು
ಹೊರ-ಟಿ-ದ್ದಾಗ
ಹೊರ-ಟಿದ್ದೀ
ಹೊರಟು
ಹೊರ-ಟು-ಬಂದ
ಹೊರ-ಟು-ಬಂದು
ಹೊರ-ಟು-ಬಿಟ್ಟ
ಹೊರ-ಟು-ಬಿ-ಟ್ಟಿದೆ
ಹೊರ-ಟು-ಬಿ-ಟ್ಟಿ-ದ್ದಾನೆ
ಹೊರ-ಟು-ಬಿ-ಡು-ತ್ತಾನೆ
ಹೊರ-ಟು-ಬಿ-ಡು-ತ್ತಿದ್ದ
ಹೊರ-ಟು-ಬಿ-ಡು-ತ್ತಿ-ದ್ದರು
ಹೊರ-ಟು-ಬಿ-ಡು-ತ್ತಿದ್ದೆ
ಹೊರ-ಟು-ಹೋಗಿ
ಹೊರ-ಟು-ಹೋ-ಗಿ-ದ್ದೆವು
ಹೊರ-ಟು-ಹೋದ
ಹೊರ-ಟೇ-ಬಿಟ್ಟ
ಹೊರ-ಡುವ
ಹೊರತು
ಹೊರ-ತೆ-ಗೆದು
ಹೊರ-ದೂ-ಡು-ವು-ದಾ-ದರೂ
ಹೊರ-ಪ್ರ-ಪಂ-ಚದ
ಹೊರ-ಬಂ-ದಿದ್ದ
ಹೊರ-ಸೂ-ಸ-ಲಾ-ರಂ-ಭಿ-ಸಿ-ದಾಗ
ಹೊರ-ಸೂ-ಸು-ತ್ತಿತ್ತು
ಹೊರ-ಸೂ-ಸು-ತ್ತಿದೆ
ಹೊರ-ಹೊ-ಮ್ಮಿತು
ಹೊಲಿ-ಗೆಯ
ಹೊಳೆ-ಯಿತು
ಹೊಳೆ-ಯಿ-ತು-ಮ-ಕ್ಕ-ಳನ್ನು
ಹೊಳೆವ
ಹೊಳೆ-ಹೊ-ಳೆ-ಯುವ
ಹೊಸ
ಹೊಸ-ದಾ-ಗಿತ್ತು
ಹೊಸ-ದೊಂದು
ಹೊಸ-ಹೊಸ
ಹೊಸೆ-ಯುವ
ಹೋ
ಹೋಗ-ಬ-ಹು-ದೆಂದು
ಹೋಗ-ಬೇ-ಕಾ-ಗು-ತ್ತಿತ್ತು
ಹೋಗ-ಬೇ-ಕಾ-ದರೆ
ಹೋಗ-ಬೇ-ಕಾ-ಯಿತು
ಹೋಗ-ಬೇಕು
ಹೋಗ-ಲಾ-ರಂ-ಭಿ-ಸಿದ
ಹೋಗ-ಲಾ-ರಂ-ಭಿ-ಸಿ-ದ್ದರು
ಹೋಗ-ಲಾ-ರರು
ಹೋಗಲಿ
ಹೋಗ-ಲಿಲ್ಲ
ಹೋಗಲು
ಹೋಗಲೇ
ಹೋಗ-ಲೇ-ಬೇಡ
ಹೋಗಾ-ಕಡೆ
ಹೋಗಿ
ಹೋಗಿದ್ದ
ಹೋಗಿ-ದ್ದಂತೆ
ಹೋಗಿ-ದ್ದರೆ
ಹೋಗಿ-ರ-ಬ-ಹುದು
ಹೋಗಿ-ರು-ತ್ತಿದ್ದ
ಹೋಗು
ಹೋಗುತ್ತ
ಹೋಗು-ತ್ತ-ದೆಯೇ
ಹೋಗು-ತ್ತಾನೋ
ಹೋಗು-ತ್ತಿದ್ದ
ಹೋಗು-ತ್ತಿ-ದ್ದರು
ಹೋಗು-ತ್ತಿ-ದ್ದ-ರೆ-ಹು-ಡುಗ
ಹೋಗು-ತ್ತಿ-ದ್ದಾಗ
ಹೋಗು-ತ್ತಿ-ದ್ದುದು
ಹೋಗು-ತ್ತೇನೆ
ಹೋಗುವ
ಹೋಗು-ವು-ದ-ಕ್ಕಾಗಿ
ಹೋಗು-ವು-ದಿತ್ತು
ಹೋಗು-ವುದು
ಹೋಗು-ವು-ದೆಂ-ದರೆ
ಹೋಗು-ವುದೇ
ಹೋಗೋಣ
ಹೋದ
ಹೋದಂ-ತಾ-ಯಿತು
ಹೋದಂತೆ
ಹೋದನೋ
ಹೋದರು
ಹೋದರೆ
ಹೋದಳು
ಹೋದ-ವ-ರಿ-ದ್ದಾ-ರೆಯೇ
ಹೋದಾಗ
ಹೋದಾ-ಗಿ-ನಿಂದ
ಹೋದೆ
ಹೋಯಿತು
ಹೋಯ್
ಹೋರಾಟ
ಹೋರಾ-ಟಕ್ಕೆ
ಹೋರಾ-ಟ-ವೆಂದು
ಹೋರಾ-ಡ-ಬೇ-ಕಾ-ಗು-ತ್ತದೆ
ಹೋರಾ-ಡ-ಬೇಕು
ಹೋರಾಡಿ
ಹೋರಾ-ಡು-ತ್ತಲೇ
ಹೋರಾ-ಡು-ತ್ತಿತ್ತು
ಹೋರಾ-ಡು-ತ್ತಿ-ದ್ದಾ-ನೆ-ಇ-ದ್ದ-ಕ್ಕಿ-ದ್ದಂತೆ
ಹೋರಾ-ಡು-ವುದನ್ನು
ಹೋಲುತ್ತಿ
ಹೋಳು-ಗಳನ್ನು
ಹೌದಾ
ಹೌದು
ಹೌಹಾ-ರಿ-ಬಿ-ಟ್ಟ-ರಲ್ಲ
ಹ್ಞೂ
}
