%importing packages
\usepackage[utf8]{inputenc}
\usepackage{polyglossia}
\usepackage{fontspec}
\usepackage{graphicx}
\usepackage{array}
\usepackage{fancyhdr}
\usepackage{setspace}
\usepackage{rotating}

%%page settings for the book
\usepackage[papersize={140mm,215mm},textwidth=100mm,
textheight=170mm,headheight=6mm,headsep=4mm,topmargin=17.5mm,botmargin=1.15cm,
leftmargin=20mm,rightmargin=20mm,footskip=0.6cm]{zwpagelayout}
\usepackage[justification=centering]{caption}

\defaultfontfeatures{Ligatures=TeX}

\setdefaultlanguage{kannada} % (english)
\setotherlanguages{english}

\setmainfont[
	Script=Kannada,
	BoldFont=SHREE-KAN-OTF-0850-Bold,
	ItalicFont=SHREE-KAN-OTF-0850-Italic,
	BoldItalicFont=SHREE-KAN-OTF-0850-Bold-Italic,
	HyphenChar="200C
]{SHREE-KAN-OTF-0850}


\newfontfamily\englishfont[
	Script=Latin,
	BoldFont=GentiumBasic-Bold,
	ItalicFont=GentiumBasic-Italic,
	BoldItalicFont=GentiumBasic-BoldItalic
]{GentiumBasic}

\newfontfamily\kannadafont[
	Script=Kannada,
	BoldFont=SHREE-KAN-OTF-0850-Bold,
	ItalicFont=SHREE-KAN-OTF-0850-Italic,
	BoldItalicFont=SHREE-KAN-OTF-0850-Bold-Italic,
	HyphenChar="200C
]{SHREE-KAN-OTF-0850}

\newfontfamily\gentiumplus[
	Script=Latin
]{GentiumPlus}

\def\eng#1{{\englishfont\textenglish{#1}}}
\def\gplus#1{{\gentiumplus #1}}
\def\kan#1{{\kannadafont\textkannada{#1}}}


\def\enginline#1{{\fontsize{10}{12}\selectfont\eng{#1}}}

\long\def\bookTitle#1{\vfill\centerline{{\fontsize{30}{32}\selectfont\textbf{#1}}}\vfill}
\def\titleauthor#1{\centerline{{\LARGE\textbf{#1}}}\vfill}
\def\delimiter{\bigskip\centerline{*\quad*\quad*}\bigskip}
\def\general#1{#1}
\long\def\supskpt#1{$^{#1}$}
\def\namesinorder#1{\\ {\rm\sl\small #1}}

%%fancy header settings
\fancypagestyle{plain}{%
\chead[]{}
\lhead[]{}
\rhead[]{}
\cfoot[{\thepage}]{{\thepage}}
}
\renewcommand{\headrulewidth}{0pt}
\pagestyle{fancy}

%%user defined commands

\def\sizenine{\fontsize{9}{11}\selectfont}

%%general settings
\pretolerance=9000
\parindent=0pt
\setlength{\parskip}{5pt}


%%redefining macros
\makeatletter

\renewcommand\chaptermark[1]{\markboth{#1}{}}
\renewcommand\section{\@startsection {section}{1}{\z@}%
                                   {-2.2ex \@plus -1.3ex \@minus -.3ex}%
                                   {1ex \@plus.2ex}%
                                   {\iflanguage{sanskrit}{\devanagarifont\Large\bfseries}{\englishfont\Large\bfseries}}}

\renewcommand\subsection{\@startsection{subsection}{2}{\z@}%
                                     {-2.25ex\@plus -1ex \@minus -.2ex}%
                                     {1.5ex \@plus .2ex}%
                                     {\iflanguage{sanskrit}{\devanagarifont\large\bfseries}{\englishfont\large\bfseries}}}

\renewcommand\subsubsection{\@startsection{subsubsection}{3}{\z@}%
                                     {-2.25ex\@plus -1ex \@minus -.2ex}%
                                     {1.5ex \@plus .2ex}%
                                     {\iflanguage{sanskrit}{\devanagarifont\bfseries}{\englishfont\bfseries}}}

\renewcommand\paragraph{\@startsection{paragraph}{4}{\z@}%
                                    {-3.25ex \@plus -1ex \@minus -.2ex}%
                                    {1em}%
                                    {\iflanguage{sanskrit}{\devanagarifont\bfseries}{\englishfont\bfseries}}}
                                     

\renewcommand\thesection{\@arabic\c@section}
\renewcommand\thesubsubsection{\@arabic\c@subsubsection}
\renewcommand\theparagraph{\thesubsubsection.\@alph\c@paragraph}

\makeatother


\def\fsize#1#2{\fontsize{#1}{#2}\selectfont }

\newcommand{\sethyphenation}[3][]{%
  \sbox0{\begin{otherlanguage}[#1]{#2}
    \hyphenation{#3}\end{otherlanguage}}}
    
\input dictionary    
