
\chapter{\enginline{5.} ಸಮಾರೋಪ}

'ಯೋ ಗ' ಭಾರತೀಯ ಪ್ರಾಚೀನ ಸಂಸ್ಕೃತಿಯ ಒಂದು ಪ್ರಮುಖ ಅಂಗವಾಗಿದೆ. ಆದರೆ ಹಿಂದಿನ ಕಾಲದಲ್ಲಿ ಹಿಮಾಲಯದಂತಹ ನಿರ್ಜನಾರಣ್ಯಗಳಲ್ಲಿ ವಾಸ ವಾಗಿರುವ ಕೆಲವು ಸಾಧುಸಂತರಿಗೆ ಮಾತ್ರ ಇದರ ಪರಿಜ್ಞಾನವಿತ್ತು. ಆದರೆ ಕಳೆದ ಶತಮಾನದ ಅಂತ್ಯಭಾಗದಿಂದೀಚೆ ಕೆಲವೆಲ್ಲ ವಿದ್ವಾಂಸರಿಗೆ ಈ ವಿಚಾರದಲ್ಲಿ ಆಸಕ್ತಿ ಮೂಡಿತು. ಆದರ ಪರಿಣಾಮವಾಗಿ ಅವರು ಆಳವಾದ ಅಭ್ಯಾಸ ನಡೆಸ ತೊಡಗಿ ದರು. ಜನಸಾಮಾನ್ಯರಿಗೂ ಅದರ ಪ್ರಯೋಜನ ದೊರಕುವಂತಾಗಲು ತಕ್ಕ ಕಾರ್ಯಕ್ರಮ ಕೈಗೊಂಡು ಅದನ್ನು ಪ್ರಚಾರಗೊಳಿಸಿದರು. ಅಂತಹವರಲ್ಲಿ ಸ್ವಾಮಿ ವಿವೇಕಾನಂದರ ಹಾಗೂ ಮಹರ್ಷಿ ಅರವಿಂದರ ಕೊಡುಗೆ ಗಣನೀಯವಾದುದು. ಇವರಲ್ಲದೆ ಹಲವಾರು ಉದ್ದಾಮ ಯೋಗಿಗಳು ಈ ಉದಾತ್ತ ತತ್ವವನ್ನು ಭಾರತ ದಲ್ಲಿ ಮಾತ್ರವೇ ಅಲ್ಲ; ಜಗತ್ತಿನಾದ್ಯಂತ ಪ್ರಸಾರಗೊಳಿಸುವ ಮಹತ್ಕಾರ್ಯ ಸಹ ನಡೆಸಿಕೊಂಡು ಬಂದರು.

ಆದರೆ ವೈದ್ಯವಿಜ್ಞಾನಿಗಳು ತಮ್ಮ ದೃಷ್ಟಿಯನ್ನು ಈ ವಿಚಾರದತ್ತ ಬಹಳವಾಗಿ ಬೀರಿರಲಿಲ್ಲ. ಅವರ ದೃಷ್ಟಿ ಪ್ರಪಂಚದಾದ್ಯಂತ ಸಾಮೂಹಿಕವಾಗಿ ಜನರನ್ನು ಬಲಿ ಗೊಳ್ಳುತ್ತಿದ್ದ ಭಯಂಕರ ಸಾಂಕ್ರಾಮಿಕ ರೋಗಗಳತ್ತ ಹೆಚ್ಚಿನಂಶ ಕೇಂದ್ರೀಕೃತ ವಾಗಿತ್ತು. ಅದರೊಂದಿಗೆ ಮಿತಿಮೀರಿದ ಕೈಗಾರಿಕೆಗಳ ವಿಸ್ತರಣೆಯಿಂದಾಗಿ ಜನ ನಿಬಿಡ ನಗರೀಕರಣ, ಅನಾರೋಗ್ಯಕರ ವಾತಾವರಣ ಮುಂತಾದವುಗಳಿಂದಾಗಿ ಜನ ಜೀವನವೇ ಕುಸಿದುಹೋಗುವ ಸಂದರ್ಭ ಬಂದಿತ್ತು. ಇಂತಹ ಸಂದಿಗ್ಧ ಪರಿಸ್ಥಿತಿ ಯಲ್ಲಿ ಜನಸಾಮಾನ್ಯರನ್ನು ಹೇಗೆ ಕಾಪಾಡುವುದು ಎಂದು ವೈದ್ಯ ವಿಜ್ಞಾನಿಗಳು ತಮ್ಮ ಆಸಕ್ತಿಯನ್ನು ತೋರಿ ಅಪಾರ ಸಂಶೋಧನಾ ಕಾರ್ಯ ನಡೆಸಿದರು.

ಅದರ ಪರಿಣಾಮವಾಗಿ ಅಂತಹ ಭಯಂಕರ ಸಾಂಕ್ರಾಮಿಕ ರೋಗಗಳ ಪ್ರತಿಬಂಧಕೋಪಾಯ ರೋಗನಿರೋಧಕ ಔಷಧಿ ಮುಂತಾದುವುಗಳ ಆವಿಷ್ಕಾರ ನಡೆದು, ಇಂದು—ಸಿಡುಬು, ಪ್ಲೇಗು, ಕಾಲರ, ಮಲೇರಿಯಾ, ಟೈಫೈಡ್​— ಮುಂತಾದ ಸಾಮೂಹಿಕ ಮಾರಕರೋಗಗಳೆಲ್ಲ ನಿರ್ನಾಮವಾದಂತಾದುದು ನಮ್ಮ ಭಾಗ್ಯವೆನ್ನಬೇಕು. ಮುಂದುವರಿದ ದೇಶಗಳಲ್ಲಂತು ಇವೆಲ್ಲ ಹೇಳಹೆಸರಿಲ್ಲದಂತಾ ದುದು ನಿಜವಾಗಿಯೂ ಬಹು ದೊಡ್ಡ ಸಾಧನೆ ಎನ್ನಬೇಕು. ಆದರೆ ಈ ಸಾಂಕ್ರಾಮಿಕ ರೋಗಗಳು ಮಾಯವಾಗಿ ಹೋಗುವುದರೊಂದಿಗೆಯೇ ಇತ್ತೀಚೆಗೆ ಹೊಸ ಹೊಸ ಮಾನಸಿಕ ಒತ್ತಡದ ರೋಗಗಳು ತಲೆಯೆತ್ತಿ ಬಂದಿರುವುದು ಒಂದು ವಿಪರ್ಯಾಸ ವೆಂದೇ ಹೇಳಬೇಕು. ಅದರಿಂದಾಗಿ ಮಿತಿಮೀರಿದ ರಕ್ತದ ಒತ್ತಡ, ಮಧುಮೇಹ, ಉಬ್ಬಸ, ಹೃದಯಾಘಾತ, ಹೊಟ್ಟೆಹುಣ್ಣು, ಕ್ಯಾನ್ಸರ್, ಅರೆ ತಲೆನೋವು— ಮೊದಲಾದ ಕಾಯಿಲೆಗಳು ಜನರನ್ನು ಶಕ್ತಿಹೀನರನ್ನಾಗಿಸಿ, ಮಾರಕ ರೋಗಗಳಾಗಿ ಪರಿಣಮಿಸಿರುವುದನ್ನು ನಾವೀಗ ಕಾಣುತ್ತಿದ್ದೇವೆ. ಅಭಿವೃದ್ಧಿ ಹೊಂದಿದ ರಾಷ್ಟ್ರ ಗಳಲ್ಲಂತೂ, ಇವು ಮಿತಿಮೀರಿ ಹೋಗುತ್ತಿವೆ.

ಒತ್ತಡದ ಕಾಯಿಲೆಗಳು ಮೂಲತಃ ಮಾನಸಿಕ ಗೊಂದಲಗಳಿಂದಾಗಿ ಬೆಳೆದು ಬರುವವುಗಳಾದ್ದರಿಂದ, ಬರಿಯ ಔಷಧಿಗಳಿಂದಾಗಲೀ, ಶಸ್ತ್ರಚಿಕಿತ್ಸೆಯಿಂದಾಗಲಿ ಗುಣಪಡಿಸಲಾದವುಗಳಾಗಿವೆ. ಇಂತಹ ಸಂದರ್ಭದಲ್ಲಿ ವಿವಿಧ ರೀತಿಯ ಯೋಗಾ ಭ್ಯಾಸಗಳು ಈ ಕಾಯಿಲೆ ಬಾರದಂತೆಯೂ, ಬಂದರೂ ಉಪಶಮನಗೊಳ್ಳು ವಂತೆಯೂ, ಸಹಾಯಕವಾಗಬಲ್ಲವೆಂಬುದನ್ನು ಇತ್ತೀಚೆಗೆ ಜಗತ್ತಿನಾದ್ಯಾಂತ ಕಂಡು ಕೊಂಡಿದ್ದಾರೆ. ಈ ರೋಗದ ಅನಾಹುತಗಳು ಜನರನ್ನು ಮಿತಿ ಮೀರಿ ಕಾಡ ತೊಡಗಿದ್ದರಿಂದ ವೈದ್ಯವಿಜ್ಞಾನಿಗಳು ಯೋಗ ವಿಚಾರದಲ್ಲಿ ಆಳವಾದ ಅಧ್ಯಯನ ನಡೆಸಲಾರಂಭಿಸಿದ್ದಾರೆ. ಅದರ ಅವಶ್ಯಕತೆ ತೋರಿಬಂದ ಕಾರಣ ನಾವು ಸಹ ನಮ್ಮ ದೊಡ್ಡ ಆಸ್ಪತ್ರೆ ಮತ್ತು ವಿಸ್ತಾರವಾದ ಪ್ರಯೋಗಶಾಲೆಗಳಲ್ಲಿ ಬೇರೆಬೇರೆ ವಿಧದ ಸಂಶೋಧನಾ ಕಾರ್ಯಗಳನ್ನು ನಡೆಸಿದವು. ಅದರ ಫಲಿತಾಂಶವನ್ನೇ ಈ ಕಿರು ಹೊತ್ತಗೆಯಲ್ಲಿ ಪ್ರಕಟಿಸಲಾಗಿದೆ.

ಸಾಮಾನ್ಯವಾಗಿ 'ಯೋಗ' ಪದವನ್ನು ಬೇರೆಬೇರೆ ಅರ್ಥಗಳಲ್ಲಿ ಜನ ಅರಿ ತಿರುತ್ತಾರೆ. ಕೆಲವರು 'ಹಠಯೋಗ'ವೆಂದು ಹೇಳಲ್ಪಡುವ ದೈಹಿಕ ವ್ಯಾಯಾಮ ಕ್ರಿಯೆಯನ್ನು 'ಯೋಗ'ವೆಂದು ತಿಳಿದಿರುತ್ತಾರೆ. ಇನ್ನು ಕೆಲವರು ಧ್ಯಾನವನ್ನು 'ಯೋಗ'ವೆಂದು ಹೇಳುತ್ತಾರೆ. ಆದರೆ ನಾವು ನಡೆಸಿದ ಆಳವಾದ ಅಧ್ಯಯನದ ಪ್ರಕಾರ ಪತಂಜಲಿ ಮಹರ್ಷಿ ಹೇಳಿದ 'ಅಷ್ಟಾಂಗಯೋಗ'ವೇ ಸರಿಯಾದ ಸಮಗ್ರ ಯೋಗವೆಂದು ಸ್ಪಷ್ಟಪಡಿಸಬಹುದಾಗಿದೆ. ಅದರೊಂದಿಗೆ, ಭಗವದ್ಗೀತೆಯಲ್ಲಿ ಹೇಳಲಾದ ಕರ್ಮಯೋಗ, ಜ್ಞಾನಯೋಗ, ಭಕ್ತಿಯೋಗಗಳ ತತ್ವವನ್ನು ಅರಿತು ಕೊಂಡು ಅವನ್ನು ಜೀವನದಲ್ಲಿ ಅಳವಡಿಸಿಕೊಂಡು ಬರಬೇಕು. ಇವು ಎಲ್ಲ ಸಮಸ್ಯೆ ಗಳ ಪರಿಹಾರಸೂಚಕವಾಗಿಯೂ ಜೀವನ ಸುಧಾರಣೆಯಲ್ಲಿ ಸಹಾಯಕವಾಗಿಯೂ ಒದಗಿಬರುವುದರಲ್ಲಿ ಸಂಶಯವಿಲ್ಲ.

ವಿಜ್ಞಾನದ ಪ್ರಭಾವ ಜೀವನದ ಎಲ್ಲ ರಂಗಗಳಲ್ಲಿಯೂ ಹರಡಿರುವುದರಿಂದ ವೈಜ್ಞಾನಿಕವಾಗಿ ಸಿದ್ಧವಾಗದಿದ್ದರೆ ಇಂದು ಯಾವ ವಿಚಾರವನ್ನೂ ಜನ ಒಪ್ಪಿ ಕೊಳ್ಳಲು ತಯಾರಿಲ್ಲ. ಅದಕ್ಕೆ 'ಯೋಗ'ವೂ ಒಂದು ಅಪವಾದವಲ್ಲ. ಆದಕಾರಣ ಯೋಗಶಕ್ತಿಯನ್ನು ವೈಜ್ಞಾನಿಕ ಅಧ್ಯಯನ ಮಾಡುವುದು ಅತ್ಯವಶ್ಯವೆಂದು ಬೇರೆ ಹೇಳಬೇಕಾಗಿಲ್ಲ. ಆ ದಿಸೆಯಲ್ಲಿ ನಡೆಸಲಾದ ನಮ್ಮ ಪ್ರಯತ್ನ ಇನ್ನೂ ಹೆಚ್ಚಿನ ಅಧ್ಯಯನ ಕಾರ್ಯಗಳಿಗೆ ಬುನಾದಿಯಂತಾಗಲೂಬಹುದು.

ಅಂತೂ ನಮ್ಮೆಲ್ಲ ಅಧ್ಯಯನಗಳ ಸಾರಾಂಶ—ಮಾನವ ತನ್ನ ಸುಖಮಯ, ಸಮರ್ಥ, ಸುದೀರ್ಘ, ಜೀವನಕ್ಕೆ—ಆಸನ, ಪ್ರಾಣಾಯಾಮ ಮತ್ತು ಧ್ಯಾನಗಳನ್ನು ಸರಿಯಾದ ರೀತಿಯಲ್ಲಿ ನಡೆಸಿಕೊಂಡು ಬರಬೇಕು. ಅಂತಹ ಪ್ರಯತ್ನ ಉತ್ತಮ ಮಟ್ಟದ್ದಾಗಿರಬೇಕು. ಮಾಡುವ ಕೆಲಸ ಉದಾತ್ತಮಟ್ಟದಾದರೆ ಅದೇ 'ಯೋಗ'. 'ಯೋಗಃ ಕರ್ಮಸು ಕೌಶಲಂ' ಎಂಬ ಗೀತಾ ವಾಕ್ಯವೂ ಅದನ್ನೆ ಹೇಳುತ್ತದೆ. ಆ ಮೂಲಕ ವಿಶ್ವದ ಮಾನವರಿಗೆಲ್ಲ ಸುಖ ಶಾಂತಿ ಒದಗಿಬರಲೆಂದು ಹಾರೈಸುತ್ತೇವೆ.

\begin{center}
\textbf{ಅನುಬಂಧ}
\end{center}

\textbf{ಯೋಗದ ಮೂಲಕ \general{\enginline{110}} ವರ್ಷಗಳ ಯೌವನ\supskpt{\footnote{

* ಭಾರತ ಸರ್ಕಾರದ ಉಚ್ಚಾಧಿಕಾರಿಯಾಗಿದ್ದು, \enginline{110} ವರ್ಷಕ್ಕೂ ಮಿಕ್ಕಿ ಬದುಕಿದ್ದ ದಿ~। ಚಾರ್ಲ್ಸ್ ಹೆನ್ರಿ ಆರ್ನಾಲ್ಡ್ ಅವರು ಇಂಗ್ಲೇಂಡಿನಿಂದ ವಾರಾಣಸಿಯ ಸ್ನೇಹಿತರಿಗೆ ಬರೆದಿದ್ದ ಪತ್ರದ ಸಾರಾಂಶ.}} }

'ಯೋ ಗವೆಂದರೆ ಅತೀಂದ್ರಿಯ ಧ್ಯಾನ. ಅದನ್ನು ಏಕಾಗ್ರತೆಯ ಮೂಲಕ ಸಾಧಿಸ ಬಹುದು. ಅಂತಹ ಧ್ಯಾನಯೋಗ, ಮಾನವನ ಅಪಾರಶಕ್ತಿಗೆ ಮೂಲಾಧಾರ ವಾಗುತ್ತದೆ. ಬೆಳಗ್ಗೆ ಎದ್ದೊಡನೆ ಹಾಗೂ ಮಲಗುವ ಮೊದಲಿನ ಪ್ರಶಸ್ತಕಾಲದಲ್ಲಿ ಹದಿನೈದು ನಿಮಿಷ ಏಕಾಗ್ರತಾಪೂರ್ವಕವಾದ ಧ್ಯಾನ ನಡೆಸಬೇಕು. ಅದನ್ನು ಪ್ರತಿದಿನ ಶಿಸ್ತುಬದ್ಧವಾಗಿ ಮಾಡುತ್ತಿರಬೇಕು. ಮನಸ್ಸಿಗೆ ತೋರಿದಾಗ ಸ್ವಲ್ಪ ಹೊತ್ತು ಧ್ಯಾನಮಾಡುವುದು, ಮತ್ತೆ ಕೆಲದಿನ ಇಷ್ಟಬಂದಂತೆ ಇದ್ದುಬಿಡುವುದು. ಆಮೇಲೊಂದು ದಿನ ಹಳೆಯ ಬಾಕಿಯನ್ನೆಲ್ಲ ತೀರಿಸುವುದಕ್ಕಾಗಿ ಘಂಟೆಗಟ್ಲೆ ಧ್ಯಾನ ಗೈವುದು—ಇಂತಹ ಧ್ಯಾನಯೋಗವನ್ನು ದಯವಿಟ್ಟು ಮಾಡಬೇಡಿರಿ. ಅದರಿಂದೇನೂ ಪ್ರಯೋಜನವಾಗದು.

ಧ್ಯಾನಕಾರ್ಯವನ್ನು ನಿಶ್ಚಿತ ಸಮಯದಲ್ಲಿ, ನಿರ್ದಿಷ್ಟ ರೀತಿಯಲ್ಲಿ ನಡೆಸಿ ಕೊಂಡು ಬರಬೇಕು. ಆಗ ಅದರ ಉತ್ತಮ ಪ್ರಯೋಜನ ಬೇಗನೆ ತೋರಿಬರು ವುದು. ಒಮ್ಮೆ ಅದರ ಸಲ್ಲಾಭದ ರುಚಿ ಗೋಚರವಾದ ಮೇಲೆ ಮತ್ತೆ ನೀವಾಗಿಯೇ ಮುಂದುವರಿಸಿಕೊಂಡು ಅವಶ್ಯ ಹೋಗುವಿರಿ. ಮತ್ತೂ ಅಭ್ಯಾಸಗೈದು ಧ್ಯಾನ ಯೋಗದ ಪರಿಪೂರ್ಣತೆಯನ್ನು ಸಾಧಿಸಿಕೊಂಡ ನಂತರ ಅದರ ಅಪಾರ ಶಕ್ತಿಯ ಅರಿವೂ ಉಂಟಾಗುವುದು. ಆಗ ಜೀವನದ ಕನಸುಗಳನ್ನೆಲ್ಲ ನನಸಾಗಿಸುವುದರಲ್ಲಿ ನೀವು ಅವಶ್ಯ ಜಯಶೀಲರಾಗುವಿರಿ.

ನನ್ನೀಪತ್ರವನ್ನೊದಿ ನನಗೀಗ \enginline{110} ವರ್ಷ ವಯಸ್ಸಾಗಿದೆ ಎಂದು ತಿಳಿದಾಗ ಅನೇಕರು—ನಾನು ಕೈಗಾಡಿಯಲ್ಲಿ ದೂಡಿಸಿಕೊಂಡು ಹೋಗುವ ಅಶಕ್ತ, ಅಥವಾ ಹಾಸಿಗೆ ಹಿಡಿದಿದ್ದು ಹೇಗೋ ಪ್ರಾಣಧಾರಣೆ ಮಾಡಿಕೊಂಡು ಇರುವ ಅಸಹಾಯಕ ಎಂದು ಮೊದಲಾಗಿ ಭಾವಿಸಿಕೊಂಡಿರಬಹುದು.

ಅದಕ್ಕಾಗಿ ಮೊತ್ತಮೊದಲು, ನನ್ನ ಈಗಿನ ಸಾಮರ್ಥ್ಯ ವಿಚಾರವನ್ನು ಸ್ಪಷ್ಟ ಪಡಿಸಬಯಸುತ್ತೇನೆ. ನಾನೆಷ್ಟಕ್ಕೂ ಅಂಗವಿಕಲನಾಗಿರದೆ, ಅಸಹಾಯಕನಾಗಿರದೆ, ಅಂತಹ ಪರಿಸ್ಥಿತಿಯಿಂದ ಇನ್ನೂ ಬಹುದೂರ ಇದ್ದೇನೆಂದು ಹೆಮ್ಮೆಯಿಂದ ಹೇಳಿಕೊಳ್ಳುವ ಸುಸ್ಥಿತಿಯಲ್ಲಿದ್ದೇನೆ. ಈಗಲೂ ಚುರುಕಾಗಿದ್ದು ವಿವಿಧ ಚಟುವಟಿಕೆ ಗಳಲ್ಲಿ ಮಗ್ನನಾಗಿದ್ದೇನೆ. ನನ್ನ ಬುದ್ಧಿಶಕ್ತಿ ಮೊದಲಿನಂತೆಯೇಕೆಲಸ ಮಾಡುತ್ತಿದೆ. ನನ್ನ ಆರೋಗ್ಯ ಚೆನ್ನಾಗಿದೆ. ನನ್ನ ದೃಷ್ಟಿ ಮಾತ್ರ ಸ್ವಲ್ಪ ಮಸುಕಾದಂತೆ ತೋರಿದರೂ ಇತ್ತೀಚೆಗೆ ಕನ್ನಡಕ ಧರಿಸತೊಡಗಿದಮೇಲೆ ನನ್ನ ಓದು, ಬರವಣಿಗೆ ಕೆಲಸಗಳನ್ನು ಇನ್ನೊಬ್ಬರ ಸಹಾಯವಿಲ್ಲದೆ ನಾನೇ ಮಾಡಿಕೊಳ್ಳುತ್ತಿದ್ದೇನೆ. ಬಾಯಿರುಚಿ ಚೆನ್ನಾಗಿದ್ದು ಸರಿಯಾಗಿ ಉಣ್ಣುತ್ತೇನೆ. ಮನಸ್ಸಾದರೆ ಸ್ವಲ್ಪ ಕುಡಿಯುವುದೂ ಉಂಟು.

ನನ್ನ ಜ್ಞಾಪಕಶಕ್ತಿ—ನಾನು ಐದು ವರ್ಷ ವಯಸ್ಸಿನವನಾಗಿದ್ದಲ್ಲಿನವರೆಗೆ ನನ್ನನ್ನೊಯ್ಯತ್ತದೆ. ಒಂದು ಶತಮಾನಕ್ಕೂ ಮೀರಿದ ನನ್ನ ಸುದೀರ್ಘ ಜೀವನದಲ್ಲಿ ಪ್ರಪಂಚದಾದ್ಯಂತ ಎಷ್ಟೆಷ್ಟೋ ಬದಲಾವಣೆಗಳಾದುದನ್ನು ಕಂಡಿದ್ದೇನೆ; ತಿಳಿದಿ ದ್ದೇನೆ. ನಾನು ಪತ್ರಿಕಾಪ್ರಪಂಚದಲ್ಲಿ ಕೆಲಸಮಾಡುತ್ತಿದ್ದಾಗ ಮಹಾಮಹಾ ಪುರುಷ ರನ್ನು ಕಂಡು ಮಾತನಾಡಿಸಿದ್ದೇನೆ. ದೊಡ್ಡ ದೊಡ್ಡ ಘಟನೆಗಳನ್ನು ಕಂಡಿದ್ದೇನೆ. ಲಂಡನ್ನಿನಲ್ಲಿ ಪಾರ್ಲಿಮೆಂಟ್ ಹೌಸ್ ಸುಟ್ಟುಹೋದ ದುರ್ಘಟನೆ, ರಾಣಿ ವಿಕ್ಟೋರಿಯಾಳ ಸಿಂಹಾಸನಾರೋಹಣದ ಸಂಭ್ರಮ—ಮುಂತಾದ ದೃಶ್ಯಗಳು ನೆನಪಿನ ಕಣ್ಣಿಗೆ ಈಗಲೂ ಸ್ಪಷ್ಟವಾಗಿ ತೋರಿಬರುತ್ತಿವೆ. ಅಂದಿನ ಮಹಾಮೇಧಾವಿ ಗಳಾದ ಚಾರ್ಲ್ಸ್ ಡಿಕ್ಕನ್ಸ್, ಡಿಸರೇಲಿ—ಅವರೊಡನೆ ನಾನು ನಡೆಸಿದ್ದ ಮಾತು ಕತೆಯ ನೆನಪು ಈಗಲೂ ಅಚ್ಚಳಿಯದೆ ಇದೆ. ಅಂದಿನ ಯುವರಾಜ ಪ್ರಿನ್ಸ್ ಆಫ್ ವೇಲ್ಸ್​ಗೆ (ನಂತರ ಏಳನೆಯ ಎಡ್ವರ್ಡ್ ರಾಜ) ಬಂದಿದ್ದ ಭಯಂಕರ ಕಾಯಿಲೆ, ನಂತರ ಆತ ಗುಣ ಹೊಂದಿದಾಗ ಲಂಡನಿನ್ನ ಜನರಿಗಾದ ಅಪಾರ ಆನಂದ— ಇವೆಲ್ಲದರ ನೆನಪು ಹಸಿರಾಗಿಯೇ ಉಳಿದಿದೆ.

ಜನಸಾಮಾನ್ಯರ ದೃಷ್ಟಿಯಲ್ಲಿ ಅಸಾಧಾರಣವೆಂದು ತೋರಬಹುದಾದ ನೂರಾರು ವಿಸ್ಮಯಕರ ಘಟನೆಗಳು ನನ್ನ ಜೀವನದಲ್ಲಿ ಬಂದು ಹೋಗಿವೆ. ಅವಕ್ಕೆಲ್ಲ ನಾನು ಅಷ್ಟೊಂದು ಮಹತ್ವ ಕೊಡುವುದಿಲ್ಲ. ಆದರೆ, ನನಗೆ 'ಯೋಗ' ಮಾರ್ಗಕ್ಕೆ ಪರಿಚಯ ಮಾಡಿಕೊಟ್ಟಿದ್ದ ಒಂದು ಅತ್ಯಂತ ಅದ್ಭುತ ಸನ್ನಿವೇಶವನ್ನು ಮಾತ್ರ ನಾನೆಂದಿಗೂ ಮರೆಯಲು ಸಾಧ್ಯವೇ ಇಲ್ಲ. ನನ್ನೀ ಸುದೀರ್ಘ ಸುಖಜೀವನಕ್ಕೆ ಕಾರಣಪುರುಷನಾದ ಆ ಪರಮಪೂಜ್ಯ ಯೋಗ ಗುರುವಿನ ಆಕಸ್ಮಿಕ ಪರಿಚಯ ಒದಗಿದ ಘಟನೆಯನ್ನು ಸಂಗ್ರಹಿಸಿ ಹೇಳುವುದು ಸಂದರ್ಭೋಚಿತವಲ್ಲವೇ?

\enginline{'1825} ರಲ್ಲಿ ಇಂಡಿಯಾ ಸರ್ಕಾರದ ಉಚ್ಚ ಅಧಿಕಾರಿಯಾಗಿದ್ದ ನಾನು ವಾರಾಣಸಿಗೆ ಒಮ್ಮೆ ಭೇಟಿ ಕೊಟ್ಟಿದ್ದೆ. ಆಗ ಬೇರೆಬೇರೆ ಊರುಗಳಿಂದ ಸಹಸ್ರಾರು ಜನರು ಅಲ್ಲಿ ನಡೆಯುತ್ತಿದ್ದ ಬಹು ದೊಡ್ಡ ಜಾತ್ರೆಗಾಗಿ ಬಂದು ಸೇರುತ್ತಿದ್ದರು. ಬೆಳಗ್ಗೆದ್ದು ನಾನು ವಾಯುವಿಹಾರಕ್ಕಾಗಿ ಹೊರಹೋಗಿದ್ದಾಗ ದೊಡ್ಡ ಗುಂಪು ಗಳೆರಡು ಪರಸ್ಪರ ಹೊಡೆದಾಟದಲ್ಲಿ ತೊಡಗಿರುವುದನ್ನು ನೋಡಿದೆ. ಆ ಗುಂಪಿನಲ್ಲಿ ಒಬ್ಬ ಸನ್ಯಾಸಿಯೂ ಸಿಕ್ಕಿಹೋಗಿದ್ದ. ನಾನು ಒಡನೆಯೇ ಮುನ್ನುಗಿದಾಗ, ನನ್ನನ್ನು ಕಂಡು, ಇಂಗ್ಲಿಷ್ ಅಧಿಕಾರಿಯಾದ ಕಾರಣ ಜನರೆಲ್ಲ ಹೆದರಿ ಓಡಿಹೋದರು. ಗಾಯಗೊಂಡಿದ್ದ ಸಂನ್ಯಾಸಿ ಮಾತ್ರ ಅಲ್ಲೇ ನಿಂತಿದ್ದ. ನಾನವನನ್ನು ಮನೆಗೆ ಕರೆದೊಯ್ದು ಗಾಯಗಳನ್ನು ತೊಳೆದು ತಕ್ಕ ಔಷಧೋಪಚಾರ ನಡೆಸಿದೆ. ವಿಚಾರ ಕೇಳಿದಾಗ ಆತ—ಮಹಾಯೋಗಗುರು ಸ್ವಾಮಿ ಚಕ್ರಾನಂದ ತಿಳಿದುಬಂತು.

ನಾನು ತೋರಿದ ಆದರ ಮತ್ತು ಮಾಡಿದ ಸಹಾಯವನ್ನು ಮೆಚ್ಚಿದ ಸ್ವಾಮೀಜಿ, ಮಾನವನಿಗೆ ಅತಿ ದೊಡ್ಡ ವರಪ್ರದಾನವೆಂದು ಹೇಳಲ್ಪಡಬೇಕಾದ ಯೋಗಾಭ್ಯಾಸ ತತ್ವವನ್ನು ನನಗುಪದೇಶಗೈದರು. ಅವರೇ ನನ್ನ ಯೋಗ ಗುರು. ಅವರಿಗಾಗ ಒಂದುನೂರು ನಲ್ವತ್ತೇಳು ವರ್ಷ ವಯಸ್ಸಾಗಿತ್ತು. ಯೋಚಿಸಿ ನೋಡಿದರೆ, ಆ ಗುರುಮಹಾಶಯರ ಪರಿಚಯವಾದ ದಿನದಿಂದಲೇ ನನ್ನ ಹೊಸ ಜೀವನ ಅಥವಾ ಪುನರ್ಜನ್ಮ ಆರಂಭಗೊಂಡಿದೆ ಎಂದು ನಾನು ಅವರನ್ನು ಪ್ರತಿದಿನ ಸ್ಮರಿಸುತ್ತಿರುತ್ತೇನೆ.'

\delimiter

