
\chapter*{ನಿವೇದನ}

\begin{center}
ಮೊದಲನೆಯ ಮುದ್ರಣ
\end{center}

ಇಂದು ನಮ್ಮ ಸುತ್ತಮುತ್ತ ನಾನಾ ತರದ ಗೊಂದಲ, ಸಂಘರ್ಷಗಳನ್ನು ಕಾಣುತ್ತಿ ದ್ದೇವೆ. ಕೊಲೆ ದರೋಡೆಗಳ ವಾರ್ತೆಗಳನ್ನು ಪತ್ರಿಕೆಗಳಲ್ಲಿ ನೋಡುತ್ತೇವೆ. ಉದ್ವೇಗ ಆವೇಶಮಯ ಹೊಡೆದಾಟಗಳು ದೇಶದಲ್ಲಿ ಸರ್ವಸಾಮಾನ್ಯವಾಗುತ್ತಿವೆ. ಅಂತಾ ರಾಷ್ಟ್ರೀಯ ಮಟ್ಟದಲ್ಲೂ ಬಗೆಹರಿಸಲಾಗದ ಯುದ್ಧಸನ್ನಿವೇಶಗಳು ದಿನ ದಿನ ಬೆಳೆದುಬರುತ್ತಿವೆ. ಸಾವಿರಾರು ಜನರನ್ನು ಬಲಿಗೊಳ್ಳುತ್ತಲೂ ಇವೆ. ಇವು ಇನ್ನಷ್ಟು ಬೆಳೆದುಬಂದರೆ ಜಾಗತಿಕ ಯುದ್ಧದಲ್ಲೇ ಪರ್ಯವಸಾನವಾಗುವ ಭಯಂಕರ ಪರಿಸ್ಥಿತಿ ಸನ್ನಿಹಿತವಾಗುವಂತೆಯೂ ಇದೆ. ಹಾಗಾದಾಗ, ಯುಗಯುಗಾಂತಗಳಿಂದ ವಿಕಾಸಗೊಂಡು ಬಂದ ನಮ್ಮ ಸಾಧನೆಗಳೆಲ್ಲ ಸರ್ವನಾಶದಲ್ಲಿ ಮುಕ್ತಾಯವಾಗುವು ದರಲ್ಲಿ ಸಂಶಯವಿಲ್ಲ.

ಇಂತಹ ವಾತಾವರಣದ ಸೃಷ್ಟಿಗೆ ಕಾರಣವೇನು? ಅದಕ್ಕಿರುವ ಪರಿಹಾರೋ ಪಾಯವೇನು? ಇದೀಗ ಎಲ್ಲರನ್ನೂ ಕಾಡುತ್ತಿರುವ ಸಮಸ್ಯೆಯಾಗಿದೆ.

\enginline{'War starts in the heart of the man'.} ಯುದ್ಧ ಆರಂಭಗೊಳ್ಳುವುದು ಮಾನವನ ಹೃದಯಲ್ಲಿ—ಎಂದು ಬಲ್ಲವರು ಹೇಳಿದ್ದಾರೆ.

ಮಾನವನ ಹೃದಯದಲ್ಲಿ ದ್ವೇಷ, ಅಸೂಯೆಗಳು ಬೆಳೆದುಬಂದರೆ ಅದು ಇತರರೊಡನೆ ಜಗಳಕ್ಕೆ ಕಾರಣವಾಗುತ್ತದೆ. ಜನಸಮುದಾಯದಲ್ಲಿ ಮಿತಿ ಮೀರಿ ಕಾಮಕ್ರೋಧಗಳು ಬೆಳೆದುಬಂದರೆ ಅದು ಇನ್ನೊಂದು ಸಮಾಜ ಅಥವಾ ರಾಷ್ಟ್ರ ದೊಡನೆ ಯುದ್ಧಕ್ಕೆ ನಾಂದಿಯಾಗುತ್ತದೆ. ಹಾಗಾಗಿ, ಮೂಲತಃ ಮಾನವ ರಾಗದ್ವೇಷ ಗಳನ್ನು ಸಾಧ್ಯವಿದ್ದಮಟ್ಟಿಗೆ ನಿಗ್ರಹಿಸಲು ಕಲಿಯಬೇಕು. ಆಗ ದ್ವೇಷದ ಬದಲಾಗಿ ಸ್ನೇಹ, ಮಾತ್ಸರ್ಯದ ಬದಲಾಗಿ ಸಹನೆ ಮೂಡಿಬರುತ್ತವೆ. ಹೀಗೆ ಜನಸಾಮಾನ್ಯರೂ ಶಾಂತಿಪ್ರಿಯರಾಗತೊಡಗಿದಾಗ ಸಮಾಜ, ದೇಶ, ರಾಷ್ಟ್ರ—ಎಲ್ಲವೂ ಜಗಳ ಹೊಡೆದಾಟಗಳಿಂದ ವಿಮುಕ್ತವಾಗುತ್ತವೆ. ವಿಶ್ವಶಾಂತಿ ತಾನಾಗಿಯೇ ಬರುತ್ತದೆ.

ಅಂತಹ ವೈಯಕ್ತಿಕ ಶಾಂತಿ, ಸಹನೆ ಬೆಳೆಸುವ ವಿಚಾರದಲ್ಲಿ 'ಯೋಗಾಭ್ಯಾಸ' ತುಂಬ ಸಹಾಯ ಮಾಡುತ್ತದೆಂಬುದು ಹಿರಿಯ ಯೋಗಿಗಳ ಸಿದ್ಧಾಂತ. ವಿಜ್ಞಾನಿ ಗಳು ಸಹ ಒಪ್ಪಿಕೊಂಡಿರುವ ವಿಚಾರ. ಇದು ಅನುಭವದಿಂದ ಕಂಡುಕೊಂಡ ಸತ್ಯಾಂಶವೂ ಹೌದು.

ಆಧುನಿಕ ವಿಜ್ಞಾನದ ಪ್ರಭಾವದಿಂದಾಗಿ ನಾವು ಪ್ರಕೃತಿಯನ್ನು ನಮಗಾಗುವಂತೆ ವಶಪಡಿಸಿಕೊಂಡಿದ್ದೇವೆ. ಮಹಾ ಮಹಾ ರೋಗಗಳಿಗೆ ಔಷಧಗಳನ್ನು ಕಂಡುಹಿಡಿ ದಿದ್ದೇವೆ. ಸುಖಮಯ ಜನಜೀವನಕ್ಕೆ ಬೇಕಾದ್ದನ್ನೆಲ್ಲ ಸಾಧಿಸಿದ್ದೇವೆ; ನಿಜ. ಆದರೆ, ಬಾಹ್ಯಪ್ರಪಂಚವನ್ನೆಲ್ಲ ಗೆದ್ದುಕೊಂಡರೂ ನಮನ್ನೇ ನಮಗೆ ಗೆಲ್ಲಲಾಗಲಿಲ್ಲ. ಮನೋಬುದ್ಧಿಗಳು ಗೊಂದಲ ಘರ್ಷಣೆಗಳಿಗೊಳಗಾಗಿ ಹೊಸ ಸಮಸ್ಯೆಗಳನ್ನು ತಂದೊಡ್ಡಿವೆ. ವೈದ್ಯ ವಿಜ್ಞಾನವು ಸಾಮೂಹಿಕ ಮಾರಕ ರೋಗಗಳಾದ ಪ್ಲೇಗ್, ಕಾಲರಾ, ಸಿಡುಬು ಮುಂತಾದ ಭಯಂಕರ ರೋಗಗಳನ್ನು ಹಿಡಿತಕ್ಕೆ ತಂದುಕೊಟ್ಟಿದೆ; ನಿಜ. ಆದರೂ ಮಿತಿಮೀರಿದ ರಕ್ತದ ಒತ್ತಡ, ಹೃದಯಾಘಾತ, ಮನೋವೈಕಲ್ಯ, ಉಬ್ಬಸ ಮುಂತಾದ ಹೊಸ ಹೊಸ ಮನೋದೈಹಿಕ ರೋಗಗಳು (Psycho Somatic Diseases) ಇಂದಿನ ಮಾನವನನ್ನು ಬಹಳವಾಗಿ ಕಾಡುತ್ತಿವೆ. ಇದು ಬಹು ದೊಡ್ಡ ಸಮಸ್ಯೆಯಾಗಿಯೂ ಒದಗಿಬಂದಿದೆ.

ಈ ಮನೋದೈಹಿಕ ಒತ್ತಡದ \enginline{(Stress and Strain)} ಕಾಯಿಲೆಗಳಿಗೆ ಔಷಧಿಯಾಗಲಿ, ಶಸ್ತ್ರಚಿಕಿತ್ಸೆಯಾಗಲಿ ತಾಗುವುದಿಲ್ಲ. ದೈಹಿಕ ಮಾನಸಿಕ ವ್ಯಾಯಾಮ ರೂಪವಾದ ಯೋಗಾಸನ, ಪ್ರಾಣಾಯಾಮ, ಧ್ಯಾನ ಮುಂತಾದವುಗಳೇ ಈ ರೋಗ ನಿವಾರಣೆಗಿರುವ, ವೆಚ್ಚವಿಲ್ಲದ ಉತ್ತಮ ಸಾಧನ. ಇದನ್ನು ನಮ್ಮ ಯೋಗೀಶ್ವ ರರೂ, ವೈದ್ಯ ವಿಜ್ಞಾನಿಗಳೂ ಕಂಡುಕೊಂಡಿದ್ದಾರೆ.

ನಮ್ಮ ದೇಹ, ಮನಸ್ಸು, ಬುದ್ಧಿಗಳ ಶಕ್ತಿಯ ಬಹುಭಾಗವನ್ನು ಇಂದು ನಾವು ಬಳಸುತ್ತಿಲ್ಲ. ಯಾವುದೇ ಮತ್ತು ಅಥವಾ ಶಕ್ತಿಯನ್ನು ಬಳಸದೇ ಇದ್ದರೆ ಅದು ತುಕ್ಕು ಹಿಡಿದಂತಾಗಿ ತನ್ನ ಸಾಮರ್ಥ್ಯವನ್ನು ಕಳೆದುಕೊಳ್ಳುತ್ತದೆ. ಇದು ಪ್ರಕೃತಿ ನಿಯಮ.

ಇದಕ್ಕಾಗಿ, ಮುಂದೆ ವಿವರಿಸಲಾಗುವ ಯೋಗಾಭ್ಯಾಸ ವಿಧಾನಗಳಿಂದ ದೇಹ, ಮನಸ್ಸು ಬುದ್ಧಿಗಳನ್ನು ಸರಿಯಾಗಿ ಬಳಸುತ್ತ ಚುರುಕಾಗಿಟ್ಟುಕೊಳ್ಳಲು ಬರುತ್ತದೆ. ಆಗ ಶಕ್ತಿ ಕುಗ್ಗುವುದಿಲ್ಲ ಮಾತ್ರವೇ ಅಲ್ಲ; ಅದು ಹಿಗ್ಗುತ್ತ ಹೋಗುತ್ತದೆ. ನಮ್ಮ ಅಗಾಧ ನಿಗೂಢಶಕ್ತಿಯ ಬಳಕೆ ಸರಿಯಾದ ರೀತಿಯಲ್ಲಿ ನಡೆಸಲ್ಪಡುವಂತಾದರೆ, ಸ್ವಾಭಾವಿಕವಾಗಿಯೇ ವ್ಯಕ್ತಿ, ಜೀವನದಲ್ಲಿ ಪ್ರಗತಿ ಸಾಧಿಸುತ್ತಾನೆ. ಆಗ ಸಮಾಜವೂ ಮುಂದೆ ಬರುತ್ತದೆ.

ಅಂತಹ ನಿಗೂಢಶಕ್ತಿಯ ವಿಕಸನಕ್ಕಿರುವ ಸಾಧನ—ಪತಂಜಲಿ ಮಹರ್ಷಿ ಹೇಳಿದ 'ಅಷ್ಟಾಂಗಯೋಗ' ವಿಧಾನ. ಇದರ ವಿವರಣೆಯನ್ನು ನಾವು ಮುಂದೆ ಕೊಡಲಿದ್ದೇವೆ.

ಯೋಗಿ ಮಹರ್ಷಿಗಳು ಹೇಳಿದ್ದಾರೆಂಬ ಕಾರಣದಿಂದ ಮಾತ್ರ ಈ 'ಯೋಗ' ವಿಚಾರವನ್ನು ಒಪ್ಪಿಕೊಳ್ಳಬೇಕಾಗಿಲ್ಲ. \enginline{Science and Philosophy of Yoga} ಎಂಬ ಮೂಲಗ್ರಂಥದ ಲೇಖಕ ಡಾ~॥ ಕೆ.ಎನ್. ಉಡುಪ ಅವರು ಸ್ವಾನುಭವದಿಂದ ಇದನ್ನು ಕಂಡುಕೊಂಡಿದ್ದಾರೆ. ವೈಜ್ಞಾನಿಕ ಪರೀಕ್ಷೆಗಳಿಂದ ದೃಢಪಡಿಸಿಕೊಂಡೂ ಇದ್ದಾರೆ. ಡಾ~॥ ಉಡುಪರು, ವಾರಾಣಸಿ ಹಿಂದೂ ವಿಶ್ವವಿದ್ಯಾಲಯದ 'ಇನ್ಸ್ಟಿ ಟ್ಯೂಟ್ ಆಫ್ ಮೆಡಿಕಲ್ ಸೈನ್ಸಸ್​' ಸಂಸ್ಥೆಯ ನಿರ್ದೇಶಕರಾಗಿ, ಆ ಮಹಾಸಂಸ್ಥೆಯನ್ನು ಕಟ್ಟಿ ಬೆಳೆಸಿದವರು. ಎರಡು ದಶಕಕ್ಕೂ ಮೀರಿದ ತನ್ನ ಅಧಿಕಾರಾ ವಧಿಯಲ್ಲಿ ಅಸಾಧಾರಣ ಕಾರ್ಯ ಕಲಾಪ ನಡೆಸಿದವರು. \enginline{W.H.O.} ಮುಂತಾದ ಪ್ರಪಂಚದಾದ್ಯಂತದ ದೊಡ್ಡ ದೊಡ್ಡ ಸಂಸ್ಥೆಗಳ ನಿಕಟ ಸಂಪರ್ಕವನ್ನಿಟ್ಟುಕೊಂಡ, ಅಂತಾರಾಷ್ಟ್ರೀಯ ಖ್ಯಾತಿಯ ಡಾ~॥ ಉಡುಪರಿಗೆ \enginline{1972} ರಲ್ಲಿ ಭಾರತ ಸರ್ಕಾರ ಪದ್ಮಶ‍್ರೀ ಬಿರುದನ್ನಿತ್ತು ಗೌರವಿಸಿದೆ. ಅವರ ಅದ್ಭುತ ಕಾರ್ಯದಕ್ಷತೆ, ಅವಿರತ ಪರಿಶ್ರಮ, ನಾನಾ ಮುಖವಾದ ಕಾರ್ಯಪ್ರವೃತ್ತಿ—ಇವುಗಳಿಗೆಲ್ಲ ಬೇಕಾದ ದೈಹಿಕ, ಮಾನಸಿಕ, ಬೌದ್ಧಿಕ ಚೈತನ್ಯ—ತನಗೆ ಯೋಗಾಭ್ಯಾಸದಿಂದ ಬಂದುದಾಗಿದೆ ಎಂಬು ದನ್ನು ಸ್ವಾನುಭವದಿಂದ ಅವರು ಹೇಳಿಕೊಂಡಿದ್ದಾರೆ. ಅವರೊಬ್ಬ ಹಿರಿಯ ವೈದ್ಯ ವಿಜ್ಞಾನಿಯಾಗಿದ್ದ ಕಾರಣ, ಆ ಅನುಭವವನ್ನು ವೈಜ್ಞಾನಿಕ ಪರಿಶೀಲನೆ, ಅಧ್ಯಯನ ಗಳಿಗೆ ಗುರಿಪಡಿಸಿ ಅದರ ಸಂತ್ಯಾಂಶವನ್ನು ಕಂಡುಕೊಂಡಿದ್ದಾರೆ. ತಮ್ಮ ಆಸ್ಪತ್ರೆಯ ರೋಗಿಗಳ ಮೇಲೆ, ಕಾಲೇಜಿನ ನಿರೋಗಿ ಯುವಕರ ಮೇಲೆ, ಯೋಗಾಭ್ಯಾಸದ ಪ್ರಯೋಗ ನಡೆಸಿದ್ದಾರೆ. ಅದರ ಪರಿಣಾಮವನ್ನು ವೈಜ್ಞಾನಿಕವಾಗಿ ಅಧ್ಯಯನಗೈದು ಅಂತಾರಾಷ್ಟ್ರೀಯ ಮಟ್ಟದ ವೈದ್ಯವಿಜ್ಞಾನ ಪ್ರತಿಕೆಗಳಲ್ಲಿ ಪ್ರಕಟಿಸಿದ್ದಾರೆ. ಅದರಿಂದಾಗಿ ಪ್ರಮುಖ ಪಾಶ್ಚಾತ್ಯ ಹಾಗೂ ಪೌರ್ವಾತ್ಯ ರಾಷ್ಟ್ರಗಳ ವಿಜ್ಞಾನಿಗಳು ಡಾ~॥ ಉಡುಪರನ್ನು ತಮ್ಮ ವಿಜ್ಞಾನಸಂಸ್ಥೆಗಳಿಗೆ ಆಮಂತ್ರಿಸಿ, ಅಥವಾ ಅವರೇ ಇಲ್ಲಿ ಬಂದು, ವಿಚಾರವಿನಿಮಯಗೈದು ಮತ್ತೂ ಹೆಚ್ಚಿನ ಸಂಶೋಧನೆಗಳಲ್ಲಿ ತೊಡಗಿದ್ದಾರೆ.

ಯೋಗವಿಚಾರದಲ್ಲಿ, ಹಲವಾರು ಗ್ರಂಥಗಳು ಪ್ರಕಟಗೊಂಡಿವೆಯಾದರೂ ವೈಜ್ಞಾನಿಕ ಅಧ್ಯಯನ ನಡೆಸಿ ಪ್ರಕಟಗೊಂಡ ಗ್ರಂಥಗಳು ವಿರಳವಾಗಿವೆ. ಆ ಕೆಲಸವನ್ನು ಡಾ~॥ ಉಡುಪರು ಸಮರ್ಥ ರೀತಿಯಲ್ಲಿ ಮಾಡಿದ್ದಾರೆ.

ಅಂತಹ ಯೋಗದ ವೈಜ್ಞಾನಿಕ ಪರಿಶೀಲನೆಯ ಫಲಿತಾಂಶವನ್ನು ಕನ್ನಡದ ಜನತೆಗೂ ಮುಟ್ಟಿಸುವ ಉದ್ದೇಶದಿಂದ ಮೂಲ ಇಂಗ್ಲಿಷ್ ಪುಸ್ತಕದ ಕನ್ನಡ ಸಂಗ್ರಹಾನುವಾದವನ್ನು ನಾನು ಮಾಡಿದ್ದೇನೆ.

ಈ ಸಂದರ್ಭದಲ್ಲಿ ಕೆಲಮಟ್ಟಿನ ಸ್ವಾನುಭವ ನಿರೂಪಣೆಯೂ ಅಪ್ರಸ್ತುತವಾಗ ಲಾರದು.

ನನ್ನ \enginline{45} ನೇ ವಯಸ್ಸಿನಿಂದಾರಂಭಿಸಿ \enginline{20} ವರ್ಷಗಳ ಕಾಲ ದೊಡ್ಡ ಕಾಲೇಜು ಗಳೆರಡರ ಪ್ರಿನ್ಸಿಪಾಲನಾಗಿ ಸಾರ್ಥಕ ಸೇವೆ ಸಲ್ಲಿಸುವ ಅವಕಾಶ ನನಗೆ ದೊರಕಿತ್ತು. ಮೊದಲ \enginline{12} ವರ್ಷ ಕೋಲಾರದ ಸರ್ಕಾರೀ ಕಾಲೇಜು ನಂತರ \enginline{8} ವರ್ಷ ಕೊಡಗಿನ ಕಾವೇರಿ ಕಾಲೇಜು; ಹೀಗೆ ಎರಡು ಕಾಲೇಜುಗಳನ್ನು, ಕಟ್ಟಿ ಬೆಳೆಸಿ, ಅಸಾಧಾರಣ ಮಟ್ಟದ ಉತ್ತಮ ಕಾಲೇಜುಗಳೆಂದು ಸಾಧಿಸಿ ಜಯಶೀಲನಾಗಿದ್ದೇನೆ. ಸಾಮಾನ್ಯ \enginline{2000} ತರುಣ ವಿದ್ಯಾರ್ಥಿ ವೃಂದ, \enginline{150} ಕ್ಕೂ ಮಿಕ್ಕಿದ ವಿವಿಧ ಮಟ್ಟದ ಸಿಬ್ಬಂದಿ ವರ್ಗ—ಇವರನ್ನೆಲ್ಲ ನಿಭಾಯಿಸಿಕೊಂಡು ರಾಜ್ಯ ಹಾಗೂ ಕೇಂದ್ರ ಸರ್ಕಾರಗಳಿಂದ, ಜನಸಾಮಾನ್ಯರಿಂದ ಮತ್ತು ನೆಚ್ಚಿನ ವಿದ್ಯಾರ್ಥಿಗಳಿಂದ 'ಸೈ' ಎನಿಸಿಕೊಂಡಿದ್ದೇನೆ. ಈ \enginline{20} ವರ್ಷಗಳ ಅವಧಿಯಲ್ಲಿ ಅನಾರೋಗ್ಯ ನಿಮಿತ್ತದ ರಜೆಯನ್ನಾಗಲಿ, 'ಅರ್ಜೆಂಟ್​' ಕೆಲಸದ ರಜೆಯನ್ನಾಗಲಿ ನಾನು ಪಡೆದುದಿಲ್ಲ. ವರ್ಷದ \enginline{365} ದಿನ, ದಿನಕ್ಕೆ \enginline{12} ಘಂಟೆಗೆ ಕಡಿಮೆಯಿಲ್ಲದ ದುಡಿಮೆಯನ್ನು ಕಾಲೇಜಿನಲ್ಲಿ ನಡೆಸುತ್ತಿ ದ್ದರೂ ದೈಹಿಕ ಮಾನಸಿಕ, ಬೌದ್ಧಿಕ ಆಯಾಸಗಳು ನನ್ನನ್ನು ಕಾಡಿದ್ದಿಲ್ಲ. ರಕ್ತದ ಒತ್ತಡ, ತಲೆನೋವು ಸೋಂಕಿದ್ದಿಲ್ಲ. ಇವಕ್ಕೆಲ್ಲ ಕಾರಣ, ನಾನು ಶಿಸ್ತುಬದ್ಧನಾಗಿ ಮಾಡುತ್ತಿದ್ದ—ಸಮಗ್ರ ಯೋಗಾಭ್ಯಾಸವೆಂದು ಧಾರಾಳ ಹೇಳಬಹುದು.

ಇತ್ತೀಚೆ ನಿವೃತ್ತನಾದಮೇಲೆ ಸಹ ದಿನಕ್ಕೆ ಸಾಮಾನ್ಯ \enginline{12} ಘಂಟೆಗಳ ಕಾಲ, ತೋಟಗಾರಿಕೆ, ವಾಚನ, ಲೇಖನ ಮುಂತಾದ ದೈಹಿಕ ಬೌದ್ಧಿಕ ಕಾರ್ಯಗಳನ್ನು ನಡೆಸುತ್ತಿದ್ದೇನೆ. ದ್ವಿಚಕ್ರ ಬೈಕಿನಲ್ಲಿ ವಿಭಿನ್ನ ಕೆಲಸಗಳಿಗಾಗಿ ಸಂಚಾರಗೈಯುತ್ತಿ ದ್ದೇನೆ. ಎರಡು ದಶಕಗಳ ಹಿಂದೆ ಇದ್ದ ರೀತಿಯಲ್ಲೇ ನನ್ನ ಚಟುವಟಿಕೆಗಳು ಇಂದು ನಡೆಯುತ್ತಿವೆ.

ಇವಕ್ಕೆಲ್ಲ 'ಯೋಗಾಭ್ಯಾಸ' ಕಾರಣವೆಂದು ತಿಳಿದ ಮೇಲೆ ಆ ವಿಜಯದ ಗುಟ್ಟನ್ನು \enginline{(Secret of Success)} ಇತರರೊಡನೆ ಹಂಚಿಕೊಳ್ಳುವುದು ಸಮಯೋಚಿತ ವಲ್ಲವೇ? ಇಂದು ಬಹು ಮಂದಿ ಅಧಿಕಾರಸ್ಥ ಮಹಾಶಯರು ನಡುವಯಸ್ಸಿನಲ್ಲೇ ಸೋತು ಸುಣ್ಣವಾಗಿ ಸಮಾಜಕ್ಕೆ ಭಾರವಾಗಿರುವವರಿದ್ದಾರೆ. ಅಂಥವರು ಚೇತರಿಸಿ ಕೊಳ್ಳುವುದಕ್ಕೆ ಈ ಪುಸ್ತಕ ಅವಶ್ಯ ಸಹಾಯಕವಾಗಬಹುದೆಂದು ಭಾವಿಸುವೆ.

ಈ ಕನ್ನಡ 'ಯೋಗವಿಜ್ಞಾನ', ಜನಸಾಮಾನ್ಯರಿಗೂ ಅರ್ಥವಾಗುವ ಸರಳ ಭಾಷೆಯಲ್ಲಿ ಬರೆಯಲು ಪ್ರಯತ್ನಿಸಿದ್ದೇನೆ. ವೈಜ್ಞಾನಿಕ ವಿವರಣೆಯಲ್ಲಿ ಮತ್ತು ತಾತ್ವಿಕ ವಿಶ್ಲೇಷಣೆಯಲ್ಲಿ ಕೆಲವೆಡೆ ಸಂಗ್ರಹಕಾರ್ಯವನ್ನೂ ಮಾಡಿದ್ದೇನೆ.

ಈ ಕನ್ನಡ ಆವೃತ್ತಿಗೆ ಮೊತ್ತಮೊದಲು ತಕ್ಕ ಮಾರ್ಗದರ್ಶನಗೈದವನು ಇಂಗ್ಲಿಷ್ ಮೂಲ ಲೇಖಕ, ನನ್ನ ಸಹೋದರ ಡಾ~॥ ಕೆ. ಎನ್. ಉಡುಪ; ಇದರ ಪ್ರಕಟನಕಾರ್ಯವನ್ನು ಸಂತೋಷದಿಂದ ನಡೆಸಿ ಕೊಟ್ಟವರು ಕರ್ನಾಟಕದ ಸುಪ್ರಸಿದ್ಧ ಪ್ರಕಾಶಕ, ಶ‍್ರೀಯುತ ಡಿ. ವಿ. ಕೆ. ಮೂರ್ತಿ ಅವರು; ಇದನ್ನು ಅಂದವಾಗಿ ಸಕಾಲದಲ್ಲಿ ಮುದ್ರಿಸಿ ಕೊಟ್ಟವರು; ಮೈಸೂರು ಪ್ರಿಂಟಿಂಗ್ ಪ್ರೆಸ್​ನ ಶ‍್ರೀ ಜಿ. ಎಚ್. ಕೃಷ್ಣಮೂರ್ತಿ ಅವರು; ಇವರೆಲ್ಲರಿಗೂ ನನ್ನ ಹೃತ್ಪೂರ್ವಕ ಕೃತಜ್ಞತೆಗಳನ್ನುರ್ಪಿಸುತ್ತೇನೆ.

\begin{flushright}
\textbf{ಪ್ರೊ~॥ ಕೆ. ರಾಮಕೃಷ್ಣ ಉಡುಪ}\\ಅನುವಾದ
\end{flushright}

ಎಂ.ಐ.ಜಿ \enginline{—5, '} ಶ‍್ರೀಕರ'\\ಉದಯರವಿ, ಕುವೆಂಪುನಗರ\\ಮೈಸೂರು \enginline{570 023}

