
\chapter{ಗುಣತ್ರಯವಿಭಾಗಯೋಗ}

ಶ‍್ರೀಕೃಷ್ಣ ಅರ್ಜುನನಿಗೆ ಹೇಳುತ್ತಾನೆ:

\begin{verse}
ಪರಂ ಭೂಯಃ ಪ್ರವಕ್ಷ್ಯಾಮಿ ಜ್ಞಾನಾನಾಂ ಜ್ಞಾನಮುತ್ತಮಮ್~।\\ಯಜ್ಜ್ಞಾತ್ವಾ ಮುನಯಃ ಸರ್ವೇ ಪರಾಂ ಸಿದ್ಧಿಮಿತೋ ಗತಾಃ \versenum{॥ ೧~॥}
\end{verse}

{\small ಯಾವುದನ್ನು ತಿಳಿದುಕೊಂಡು ಮುನಿಗಳು ಇಲ್ಲಿಂದ ಪರಮ ಸಿದ್ಧಿಯನ್ನು ಹೊಂದಿದರೋ, ಅಂತಹ ಜ್ಞಾನಗಳಲ್ಲಿ ಉತ್ತಮವೂ ಶ್ರೇಷ್ಠವೂ ಆದ ಜ್ಞಾನವನ್ನು ಪುನಃ ಹೇಳುತ್ತೇನೆ.}

ಈ ಅಧ್ಯಾಯವನ್ನು ಗುಣತ್ರಯವಿಭಾಗಯೋಗ ಎಂದು ಕರೆಯುತ್ತಾರೆ. ಗೀತೆಯ ಪ್ರಕಾರ ಈ ಪ್ರಪಂಚವು ಆಗಿರುವುದು ಸತ್ತ್ವ, ರಜಸ್ಸು ಮತ್ತು ತಮೋಗುಣಗಳ ಸಂಮಿಶ್ರಣದಿಂದ. ಇಲ್ಲಿ ಯಾವ ಗುಣವೂ ಮತ್ತೊಂದನ್ನು ಬಿಟ್ಟು ತಾನು ಬೇರೆ ಇಲ್ಲ. ಯಾವಾಗಲೂ ಮೂರೂ ಒಟ್ಟಿಗೆ ಇರುವುವು. ಆದರೆ ಸಮಪ್ರಮಾಣದಲ್ಲಿ ಇರುವುದಿಲ್ಲ. ಅದರಲ್ಲೂ ಯಾವುದಾದರೂ ಒಂದು ಉಳಿದ ಎರಡು ಗುಣಗಳಿಗಿಂತ ಹೆಚ್ಚಾಗಿದೆ. ಅದಕ್ಕಾಗಿಯೇ ಆ ಹೆಸರು ಅದಕ್ಕೆ ಬರುವುದು. ಮನುಷ್ಯರೆಲ್ಲ ಈ ಗುಣಗಳಿಗೆ ಸೇರಿದ್ದಾರೆ. ಇವುಗಳಲ್ಲಿ ತಮೋಗುಣ ಪ್ರಧಾನವಾದವರೇ ಅತ್ಯಂತ ಕೆಳಗಿನ ಮಟ್ಟ ದವರು. ಇವರಲ್ಲಿ ಅಜ್ಞಾನ ಆಲಸ್ಯ ಜಡತ್ವ ಮುಂತಾದುವುಗಳು ಹೆಚ್ಚಾಗಿವೆ. ಅದಕ್ಕಿಂತ ಉತ್ತಮ ರಾದವರೆ ರಜೋಗುಣಿಗಳು. ಅವರಲ್ಲಿ ಚಟುವಟಿಕೆಗಳನ್ನು ನೋಡುತ್ತೇವೆ. ಆಸೆ ಆಕಾಂಕ್ಷೆಗಳಿವೆ; ಫಲಾಪೇಕ್ಷೆ ಇದೆ; ಕೀರ್ತಿ, ಲಾಭ, ಯಶಸ್ಸು, ಅಧಿಕಾರ ಮುಂತಾದುವುಗಳನ್ನು ಪಡೆಯುವುದಕ್ಕೆ ಹಾತೊರೆಯುತ್ತಿರುವರು ಯಾವಾಗಲೂ. ಸುಮ್ಮನೆ ಇರುವುದಿಲ್ಲ. ಕೆಲವು ವೇಳೆ ಚೆನ್ನಾಗಿ ಪೆಟ್ಟು ಬಿದ್ದಾಗ ಇನ್ನು ಮೇಲೆ ಯಾವ ಕೆಲಸಕ್ಕೂ ಕೈಹಾಕುವುದಿಲ್ಲವೆಂದು ಮನಸ್ಸು ಮಾಡುತ್ತಾರೆ. ಆದರೆ ಬಹಳ ಬೇಗ ಇದನ್ನು ಮರೆಯುವರು; ಎಂದಿನಂತೆ ಮತ್ತಾವುದೋ ಚಟುವಟಿಕೆಯ ಬಿರುಗಾಳಿ ಯಲ್ಲಿ ಸಿಕ್ಕಿಕೊಳ್ಳುವರು. ಅಂತೂ ಒಂದು ಬಿಟ್ಟರೆ ಮತ್ತೊಂದನ್ನು ಬೇಗ ಹಿಡಿಯುತ್ತಾರೆ. ಮೂರನೇ ಮನುಷೃನೇ ಸತ್ತ್ವಗುಣ ಪ್ರಧಾನವಾಗಿರುವವನು. ಇಲ್ಲಿ ಶಾಂತಿ ಜ್ಞಾನ ಮುಂತಾದ ಒಳ್ಳೆಯ ಗುಣಗಳೆಲ್ಲ ಇವೆ. ಆದರೆ ಇವನೂ ಕೂಡ ಬದ್ಧ. ಈ ಗುಣಗಳಿಗೆ ಅಂಟಿಕೊಂಡಿರುವನು. ಇದನ್ನು ಬಿಟ್ಟು ಹೋಗಲಾರ. ಉಳಿದ ಎರಡು ಗುಣಗಳೊಂದಿಗೆ ಹೋಲಿಸಿ ನೋಡಿದರೆ ಇದು ಶ್ರೇಷ್ಠವಾಗಿ ಕಾಣುವುದು. ಆದರೆ ನಾವು ಸೇರುವ ಗುರಿಯ ದೃಷ್ಟಿಯಿಂದ ನೋಡಿದರೆ ಇದು ಕೂಡಾ ನಮ್ಮನ್ನು ಪ್ರಪಂಚಕ್ಕೆ ಬಂಧಿಸಿರುವುದು; ನಾವು ಇದರಿಂದಲೂ ಕಿತ್ತುಕೊಂಡು ಹೋಗಬೇಕಾಗಿದೆ.

ಮಾನವನ ವಿಕಾಸವೆಲ್ಲ ತಮೋಗುಣದಿಂದ ರಜೋಗುಣಕ್ಕೆ ಬರುವುದು. ಅಲ್ಲಿಂದ ಸತ್ತ ್ವಗುಣಕ್ಕೆ, ಅನಂತರ ಗುಣಗಳಿಗೆ ಅತೀತವಾಗಿ ಹೋಗುವುದು ಆಗಿದೆ. ಒಂದು ದೃಷ್ಟಿಯಿಂದ ನೋಡಿದರೆ ಇರುವುದು ಒಂದೇ ಗುಣ, ಅದು ಮೂರು ಸ್ಥಿತಿಯಲ್ಲಿದೆ. ಒಂದೊಂದು ಸ್ಥಿತಿಯಲ್ಲಿಯೂ ಅದು ಬೇರೆಯೋ ಎಂಬಂತೆ ನಮಗೆ ಕಾಣುತ್ತಿರುವುದು. ನಾವು ನೀರಿನ ಮೂರು ಸ್ಥಿತಿಗೆ ಇದನ್ನು ಹೋಲಿಸಬಹುದು. ನೀರು ಹಿಮಾಲಯದಲ್ಲಿ ಘನೀಭೂತವಾಗಿ ನೀರ್ಗಲ್ಲಿನಂತೆ ಇದೆ. ಆಗ ಅದು ಗಟ್ಟಿಯಾಗಿ ಇರುವುದು, ಜಡವಾಗಿರುವುದು. ಬಿದ್ದ ಕಡೆ ಬಿದ್ದಿರುವುದು. ಅದೇ ನೀರ್ಗಲ್ಲು ಕರಗಿದರೆ ನೀರಿನಂತೆ ಚಲಿಸುವುದು, ಚಲಿಸುವಾಗ ಅದೊಂದು ಶಬ್ದ ಮಾಡಿಕೊಂಡು ಹೋಗುವುದು, ಮೇಲಿ ನಿಂದ ಉರುಳಿಕೊಂಡು ಹೋಗುವುದು. ಯಾವ ಆತಂಕ ಅದರ ದಾರಿಯಲ್ಲಿ ಬಂದರೂ, ಅದನ್ನು ಸಾಧ್ಯವಾದರೆ ಭೇದಿಸಿಕೊಂಡು ಹೋಗುವುದು, ಇಲ್ಲದೆ ಇದ್ದರೆ ಅದನ್ನು ಬಳಿಸಿಕೊಂಡು ಹೋಗು ವುದು. ಅದೇ ನೀರು ಆವಿಯಾದರೆ ಕೆಳಗಿನಿಂದ ಮೇಲೆಕ್ಕೆ ಹೋಗುವುದು. ಹೋಗುವಾಗ ಹಗುರವಾಗು ವುದು. ತನ್ನಲ್ಲಿ ಇರುವ ಕೊಳೆಯನ್ನು ಇಲ್ಲೇ ಬಿಟ್ಚುಪರಿಶುದ್ಧವಾಗಿ ಏಳುವುದು. ಮೂರು ಗುಣಗಳ ಸ್ಥಿತಿ ಹೀಗೆಯೆ. ನೀರ್ಗಲ್ಲಿನಂತೆ ಇರುವುದು ತಮೋಗುಣ. ನೀರಿನಂತೆ ಹರಿಯುತ್ತಿರುವುದು ರಜೋ ಗುಣ, ಆವಿಯಂತೆ ಮೇಲೇಳುವುದು ಸತ್ತ್ವಗುಣ. ಈ ಮೂರು ಗುಣಗಳ ವಿಷಯವನ್ನು ಶ‍್ರೀಕೃಷ್ಣ ಈ ಅಧ್ಯಾಯದಲ್ಲಿ ವಿವರಿಸುತ್ತಾನೆ.

ಯಾವುದನ್ನು ತಿಳಿದುಕೊಂಡು ಮುನಿಗಳು ಇಲ್ಲಿಂದ ಪರಮ ಸಿದ್ಧಿಯನ್ನು ಹೊಂದಿ ಹೋದರೊ ಎನ್ನುವನು. ಆ ಜ್ಞಾನವನ್ನು ತಿಳಿದುಕೊಂಡರೆ ಶ್ರೇಷ್ಠವಾದ ಸಿದ್ಧಿಯನ್ನು ಪಡೆಯುವರು. ಪರಮ ಸಿದ್ಧಿ ಎಂದರೆ ಪರಮಾತ್ಮನನ್ನು ಮುಟ್ಟುವುದು. ಒಮ್ಮೆ ನಾವು ಅವನನ್ನು ಮುಟ್ಟಿತು ಎಂದರೆ ಈ ಸಂಸಾರದ ಆಟದಿಂದ ಪಾರಾದಂತೆ. ಪಗಡೆಕಾಯಿ ಆಟ ಆಡುತ್ತಿರುವಾಗ, ಕಾಯಿ ಹಣ್ಣಾದಂತೆ ಅದು. ಆ ಕಾಯಿ ಇನ್ನು ತಾನು ಬಂದ ಸ್ಥಳಕ್ಕೆ ಹಿಂತಿರುಗಲಾರದು. ಯಾರೂ ಅದನ್ನು ಇನ್ನು ಕೊಲ್ಲಲಾರರು. ಅದರಂತೆಯೇ ಜೀವಿ ಪರಮ ಸಿದ್ಧಿಯನ್ನು ಪಡೆದ ಮೇಲೆ ಸಂಸಾರ ಚಕ್ರದಿಂದ ಪಾರಾಗಿ ಹೋಗುತ್ತಾನೆ. ಜನನ ಮರಣಗಳ ಬಲೆಯಿಂದ ತಪ್ಪಿಸಿಕೊಂಡು ಹೋಗುತ್ತಾನೆ.

ಈ ಸಂಸಾರದಿಂದ ತಪ್ಪಿಸಿಕೊಂಡು ಹೋಗುವುದಕ್ಕೆ ಹಲವು ಮಾರ್ಗಗಳಿವೆ. ಜ್ಞಾನ ಇದೆ, ಕರ್ಮ ಇದೆ, ಭಕ್ತಿಮಾರ್ಗವಿದೆ. ಈ ಮಾರ್ಗಗಳಲ್ಲೆಲ್ಲ ಯಾವುದು ಉತ್ತಮವೊ ಶ್ರೇಷ್ಠವೊ ಅದನ್ನು ಹೇಳುತ್ತೇನೆ ಎನ್ನುತ್ತಾನೆ. ಎಲ್ಲಾ ಮಾರ್ಗಗಳೂ ಗುರಿ ಕಡೆಗೆ ನಮ್ಮನ್ನು ಕರೆದೊಯ್ಯಬಹುದು, ಆದರೆ ಒಬ್ಬನ ಸ್ವಭಾವಕ್ಕೆ ಯಾವುದು ಹಿಡಿಸುವುದೊ ಆ ದೃಷ್ಟಿಯಿಂದ ನೋಡಬೇಕಾಗಿದೆ. ಈ ಮಾರ್ಗದಲ್ಲಿ ಒಬ್ಬೊಬ್ಬರಿಗೆ ಒಂದೊಂದು ಹಿಡಿಸುವುದು. ಪ್ರತಿಯೊಬ್ಬನೂ ತನ್ನ ಮನೋರುಚಿಯ ಮೇಲೆ ಮತ್ತು ತನ್ನ ಸಾಮರ್ಥ್ಯದ ಮೇಲೆ ಇದನ್ನು ನಿರ್ಧರಿಸಬೇಕಾಗಿದೆ. ಇಲ್ಲಿ ಅರ್ಜುನನಂತಹವನಿಗೆ ಯಾವುದು ಶ್ರೇಷ್ಠವಾದ ಹಾದಿ ಎಂಬುದನ್ನು ಶ‍್ರೀಕೃಷ್ಣ ಹೇಳುತ್ತಾನೆ. ಶ‍್ರೀಕೃಷ್ಣ ಏತಕ್ಕೆ ಹೇಳಬೇಕು? ಯಾವುದು ಯಾರಿಗೆ ಸರಿ ಎಂಬುದು ಅವರಿಗೆ ಗೊತ್ತಿಲ್ಲವೆ? ಅವರೇ ನಿರ್ಧರಿಸಲಿ ಎಂದು ಹೇಳಬಹುದು. ಅಂಗಡಿಯವನು ತನ್ನ ಸರಕನ್ನೆಲ್ಲಾ ಗಿರಾಕಿ ಎದುರಿಗೆ ಪ್ರದರ್ಶಿಸುತ್ತಾನೆ. ಗಿರಾಕಿ ತನಗೆ ಯಾವುದು ಹಿಡಿಸುವುದು ಮತ್ತು ತನ್ನಲ್ಲಿ ಎಷ್ಟು ದುಡ್ಡು ಇದೆಯೋ ಅದಕ್ಕೆ ಅನುಗುಣವಾಗಿ ಕೊಂಡುಕೊಳ್ಳಲಿ ಎಂದು. ಆದರೆ ಯಾವುದು ಚೆನ್ನ ನಮಗೆ ಎನ್ನುವುದು ಗೊತ್ತಿಲ್ಲ. ಎಲ್ಲಾ ಚೆನ್ನಾಗಿ ಕಾಣುವುವು. ಕೆಲವು ವೇಳೆ ಅಂಗಡಿವನಿಗೇ, ನಿಮಗೆ ಯಾವುದು ಚೆನ್ನಾಗಿ ಕಾಣುತ್ತದೆ, ಅದನ್ನು ನಿಶ್ಚಯಿಸಿ ಹೇಳಿ ಎನ್ನುವೆವು. ಅದರಂತೆಯೇ ಗುರು, ಶಿಷ್ಯನಿಗೆ ಯಾವುದೊ ಒಂದು ಮಾರ್ಗದಲ್ಲಿ ಹೋಗುವಂತೆ ಆಸೆ ಇದೆ. ಆದರೆ ಅದಕ್ಕೆ ಇರುವ ಯೋಗ್ಯತೆಯನ್ನು ಅವನು ತಿಳಿದುಕೊಂಡಿಲ್ಲ. ನನ್ನ ಯೋಗ್ಯತೆಗೆ ತಕ್ಕ ಮಾರ್ಗವನ್ನು ನಾನು ಹಿಡಿಯಬೇಕಾಗಿದೆ. ಗುರುವಿಗೆ ಇದು ಚೆನ್ನಾಗಿ ವೇದ್ಯವಾಗುವುದು. ಅರ್ಜುನ ಕುರುಕ್ಷೇತ್ರ ಯುದ್ಧರಂಗದಲ್ಲಿ ಮೊದಲ ಬಾರಿ ನಿಂತಾಗ ಅವನ ಮನಸ್ಸಿನಲ್ಲಿ ಎದ್ದ ಪ್ರತಿಕ್ರಿಯೆ, ಯಾವ ಯುದ್ಧವನ್ನೂ ಮಾಡದೆ ಹಿಂತಿರುಗಿ ಹೋಗಬೇಕು ಎನ್ನುವುದಾಗಿತ್ತು. ಆದರೆ ಶ‍್ರೀಕೃಷ್ಣ ಅದಕ್ಕೆ ಒಪ್ಪಿಕೊಂಡನೇ? ಇಲ್ಲ. ಬಲತ್ಕಾರವಾಗಿ ಅರ್ಜುನನ್ನು ಯುದ್ಧ ಮಾಡಲು ಪ್ರೇರೇಪಿಸುವನು. ಏಕೆಂದರೆ ಈಗ ಯುದ್ಧ ಬೇಡವೆಂದು ಹಿಂತಿರುಗಿದರೂ ನಾಳೆ ಜನ ಅರ್ಜುನನ್ನು ಹೇಡಿ ದುರ್ಬಲ ಎಂದು ಆಡಿಕೊಳ್ಳುವಾಗ ಇದನ್ನು ಉದಾಸೀನನಾಗಿ ನೋಡಬಲ್ಲ ಸಾತ್ತ್ವಿಕ ಶಕ್ತಿ ಕ್ಷತ್ರಿಯನಾದ ಅರ್ಜುನನಲ್ಲಿ ಇಲ್ಲ. ಆಗ ಅವನು ಅವಮಾನಿತನಾಗಿ, ಈಗ ಏನನ್ನು ಮಾಡುವುದಿಲ್ಲ ಎಂದು ಹೇಳುತ್ತಾನೆಯೊ ಅದನ್ನು ಮಾಡಿ ಹಾಕುತ್ತಾನೆ ಎಂಬುದನ್ನು ಶ‍್ರೀಕೃಷ್ಣ ಚೆನ್ನಾಗಿ ಬಲ್ಲ. ಅರ್ಜುನನಿಗೆ ಇದು ಗೊತ್ತಿಲ್ಲ. ಆದರೆ ಶ‍್ರೀಕೃಷ್ಣನಿಗೆ ಇದು ಗೊತ್ತು. ಅದಕ್ಕೇ ನಿಜವಾದ ಗುರು ಶಿಷ್ಯನ ಜವಾಬ್ದಾರಿಯನ್ನು ತೆಗೆದುಕೊಂಡಿರುವನು. ಅವನಿಗೆ ಯಾವುದು ಮೇಲೋ ಅದನ್ನು ಹೇಳುತ್ತಾನೆ.

ಆ ಜ್ಞಾನವನ್ನು ಪುನಃ ಹೇಳುತ್ತೇನೆ ಎನ್ನುವನು. ಇದನ್ನು ಇದುವರೆಗೆ ಎಷ್ಟೋ ಸಲ ಒಂದಲ್ಲ ಮತ್ತೊಂದು ರೀತಿಯಲ್ಲಿ ಹೇಳಿರುವನು. ಏನೊ ಹೊಸದಾಗಿರುವ ವಿಷಯವನ್ನು ಹೇಳುತ್ತೇನೆ ಎಂದು ಹೇಳುವುದಿಲ್ಲ. ಒಂದೇ ಸತ್ಯವನ್ನು ಹಲವು ದೃಷ್ಟಿಕೋನಗಳಿಂದ ನೋಡುವನು. ಒಂದೊಂದು ಸಲ ಹೇಳಿದಾಗಲೂ ಶಿಷ್ಯನ ಮನಸ್ಸಿನ ಆಳ ಆಳಕ್ಕೆ ಅದು ಹೋಗುವುದು. ಇಲ್ಲಿ ಪುನರುಕ್ತಿ ಒಂದು ದೋಷವಲ್ಲ. ರೋಗ ಗುಣವಾಗುವವರೆಗೆ ಔಷಧವನ್ನು ತೆಗೆದುಕೊಳ್ಳುತ್ತ ಇರಬೇಕಾಗುವುದು.

\begin{verse}
ಇದಂ ಜ್ಞಾನಮುಪಾಶ್ರಿತ್ಯ ಮಮ ಸಾಧರ್ಮ್ಯಮಾಗತಾಃ~।\\ಸರ್ಗೇಽಪಿ ನೋಪಜಾಯಂತೇ ಪ್ರಲಯೇ ನ ವ್ಯಥಂತಿ ಚ \versenum{॥ ೨~॥}
\end{verse}

{\small ಈ ಜ್ಞಾನವನ್ನು ಆಶ್ರಯಿಸಿಕೊಂಡು ನನ್ನ ಸಾಧರ್ಮ್ಯವನ್ನು ಹೊಂದಿದವರು ಸೃಷ್ಟಿಕಾಲದಲ್ಲಿ ಹುಟ್ಟುವುದಿಲ್ಲ, ಪ್ರಳಯಕಾಲದಲ್ಲಿ ವ್ಯಥೆಪಡುವುದಿಲ್ಲ.}

ಯಾರು ಈ ಅಧ್ಯಾಯದಲ್ಲಿ ಬರುವ ಜ್ಞಾನವನ್ನು ತಿಳಿದುಕೊಳ್ಳುತ್ತಾರೆಯೊ ಅವರು ಭಗವಂತ ನಂತೆ ಆಗುತ್ತಾರೆ. ಗುಣಗಳು ಹೇಗೆ ನಮ್ಮನ್ನು ಬಂಧನಕ್ಕೆ ಒಳಗುಮಾಡುತ್ತವೆ ಮತ್ತು ಅವುಗಳಿಂದ ನಾವು ಪಾರಾಗುವುದು ಹೇಗೆ ಎಂಬುದನ್ನು ತಿಳಿದುಕೊಂಡು ಪಾರಾಗುವರು. ಇಲ್ಲಿ ಸಾಧರ್ಮ್ಯವನ್ನು ಪಡೆದುಕೊಳ್ಳುವುದು ಎಂದರೆ ಅವನಂತೆ ಆಗುವುದು. ಇವನು ಹೇಗೆ ನಿತ್ಯವೊ, ಬಂಧನಕ್ಕೆ ಸಿಕ್ಕಿಲ್ಲವೊ ಹಾಗೆ ಆಗುವುದು. ಇದೇ ಸಾಧರ್ಮ್ಯ. ಯಾರು ಇಂತಹ ಸ್ಥಿತಿಯನ್ನು ಪಡೆದುಕೊಂಡಿರುವರೊ ಅವರು ಸೃಷ್ಟಿಕಾಲದಲ್ಲಿ ಹುಟ್ಟುವುದಿಲ್ಲ. ಈ ಪ್ರಪಂಚಕ್ಕೆ ಪುನಃ ಬದ್ಧಜೀವಿಗಳಂತೆ ಬರುವುದಿಲ್ಲ. ಯಾರು ಇನ್ನೂ ಜ್ಞಾನವನ್ನು ಪಡೆದುಕೊಳ್ಳಬೇಕಾಗಿದೆಯೋ ಅವನು ಬರುತ್ತಾನೆ. ಯಾರಲ್ಲಿ ವಾಸನೆಗಳು ಇನ್ನೂ ಇವೆಯೋ ಅದನ್ನು ತೃಪ್ತಿಪಡಿಸಿಕೊಳ್ಳುವುದಕ್ಕೆ ಬರಬೇಕಾಗಿದೆ. ಆದರೆ ಯಾರು ತಮ್ಮ ಹೃದಯದ ವಾಸನೆಗಳನ್ನೆಲ್ಲ ಪರಮಾತ್ಮನ ಜ್ಞಾನದಿಂದ ದಹಿಸಿಕೊಂಡಿರುವರೊ ಅವರು ಇನ್ನು ಮೇಲೆ ಪ್ರಪಂಚಕ್ಕೆ ಬರಬೇಕಾಗಿಲ್ಲ. ಅದರಂತೆಯೇ ಪ್ರಳಯದ ಸಮಯದಲ್ಲಿ ಅವರು ವ್ಯಥೆಯನ್ನೂ ಪಡೆಯುವುದಿಲ್ಲ. ಯಾರಿಗೆ ಇನ್ನೂ ಆಸಕ್ತಿ ಇದೆಯೋ ಅವರು ಇದನ್ನು ಬಿಟ್ಟುಹೋಗಬೇಕಾಗಿದೆ ಯಲ್ಲ; ಅಥವಾ ಅದು ನನ್ನನ್ನು ಬಿಟ್ಟುಹೋಗುವುದಲ್ಲ ಎಂದು ವ್ಯಥೆಪಡುವರು. ಜ್ಞಾನಿಯಾದರೊ ಹೊರಗಿನ ಪ್ರಕೃತಿ ಬಂದು ಅವನಿಂದ ಕಿತ್ತುಕೊಳ್ಳುವುದಕ್ಕಿಂತ ಮುಂಚೆಯೇ ಅವುಗಳನ್ನು ಆಚೆಗೆ ಎಸೆದಿರುವನು. ಅವನು ಯಾವುದಕ್ಕೂ ಬದ್ಧನಲ್ಲ, ಆಸಕ್ತನಲ್ಲ. ಅವನು ತನ್ನ ಕೈಯನ್ನು ಎಲ್ಲದ ರಿಂದಲೂ ತೊಳೆದುಕೊಂಡಿರುವನು.

\begin{verse}
ಮಮ ಯೋನಿರ್ಮಹದ್ಬ್ರಹ್ಮ ತಸ್ಮಿನ್ ಗರ್ಭಂ ದದಾಮ್ಯಹಮ್~।\\ಸಂಭವಃ ಸರ್ವಭೂತಾನಾಂ ತತೋ ಭವತಿ ಭಾರತ \versenum{॥ ೩~॥}
\end{verse}

{\small ಅರ್ಜುನ, ಮಹತ್ತಾದ ಪ್ರಕೃತಿಯೇ ನನ್ನ ಯೋನಿ; ಅದರಲ್ಲಿ ನಾನು ಗರ್ಭವನ್ನು ಇಡುತ್ತೇನೆ. ಇದರಿಂದ ಸರ್ವಪ್ರಾಣಿಗಳ ಉತ್ಪತ್ತಿಯೂ ಆಗುವುದು.}

ತ್ರಿಗುಣಗಳಿಂದ ಕೂಡಿದ ಪ್ರಕೃತಿಯೇ ಕ್ಷೇತ್ರ. ಇಲ್ಲಿ ಪಂಚಭೂತಗಳು ಜ್ಞಾನೇಂದ್ರಿಯ, ಕರ್ಮೇಂ ದ್ರಿಯ, ಮನಸ್ಸು ಬುದ್ಧಿ ಚಿತ್ತ ಅಹಂಕಾರಗಳೆಲ್ಲ ಇರುವುವು. ಇಲ್ಲಿಯೇ ಪರಮಾತ್ಮ ಜೀವಿಗಳೆಂಬ ಬೀಜವನ್ನು ಇಡುವುದು. ಇದು ಅನಂತರ ಈ ದೇಹಗಳನ್ನು ಧರಿಸಿ ಹಲವು ನಾಮರೂಪದಿಂದ ಈ ಪ್ರಪಂಚದಲ್ಲಿ ವಿಕಾಸವಾಗುವುದು. ನಿಜವಾಗಿ ಎಲ್ಲರಿಗೂ ತಂದೆ ದೇವರೊಬ್ಬನೇ. ನಮ್ಮ ದೇಹದ ತಾಯಿತಂದೆಗಳು ಕೇವಲ ನಿಮಿತ್ತಮಾತ್ರ. ಅವರು ನಮ್ಮ ಒಂದು ದೇಹಕ್ಕೆ ತಂದೆ ತಾಯಿಗಳು. ಭಗವಂತನಾದರೊ ಎಲ್ಲರಿಗೂ ತಂದೆ. ಯಾವಾಗ ಅವನು ನಮ್ಮೆಲ್ಲರಿಗೂ ತಂದೆಯಾಗುತ್ತಾನೆಯೋ ಆಗ ಈ ಪ್ರಪಂಚದಲ್ಲಿ ಎಲ್ಲರೂ ನಮ್ಮ ಬಂಧುಬಾಂಧವರಾಗುತ್ತಾರೆ. ಏಕೆಂದರೆ ನಾವೆಲ್ಲ ಒಂದು ವಿಶ್ವ ಸಂಸಾರಕ್ಕೆ ಸೇರಿದವರು.

\begin{verse}
ಸರ್ವಯೋನಿಷು ಕೌಂತೇಯ ಮೂರ್ತಯಃ ಸಂಭವಂತಿ ಯಾಃ~।\\ತಾಸಾಂ ಬ್ರಹ್ಮ ಮಹದ್ಯೋನಿರಹಂ ಬೀಜಪ್ರದಃ ಪಿತಾ \versenum{॥ ೪~॥}
\end{verse}

{\small ಸರ್ವಯೋನಿಗಳಲ್ಲೂ ಯಾವ ಶರೀರಗಳು ಹುಟ್ಟುತ್ತಿರುವುವೋ ಅವುಗಳಿಗೆ ಮಹತ್ ಪ್ರಕೃತಿಯೇ ಯೋನಿ ಬೀಜಪ್ರದನಾದ ತಂದೆಯೇ ನಾನು.}

ಈ ಪ್ರಪಂಚದಲ್ಲಿ ಎಲ್ಲಾ ಬಗೆಯ ಜೀವರಾಶಿಗಳಿಗೂ ತಾಯಿಯೇ ಪ್ರಕೃತಿ. ಅವಳ ಮೂಲ ದಿಂದಲೇ ಇವುಗಳೆಲ್ಲ ಬರುವುವು. ಬಗೆಬಗೆಯ ಜೀವರಾಶಿಗಳೆಲ್ಲ ಅಲ್ಲಿಂದ ಬರುತ್ತಿವೆ. ಆದರೆ ಪ್ರಕೃತಿ ಇವುಗಳನ್ನು ಸೃಷ್ಟಿಸುವುದಿಲ್ಲ. ದೇವರು ಅಲ್ಲಿ ಬೀಜವನ್ನು ಇಡುವನು. ಹೊಲ ಆ ಬೀಜವನ್ನು ತೆಗೆದುಕೊಂಡು ಅದರಿಂದ ಒಂದು ಸಸಿಯನ್ನು ಮಾಡುತ್ತದೆ. ಪ್ರಕೃತಿಯೇ ಇವುಗಳನ್ನು ತಯಾರು ಮಾಡಲಾರದು. ಭಗವಂತ ಅಲ್ಲಿ ಬೀಜವನ್ನು ಇಟ್ಟರೆ, ಆ ಬೀಜಕ್ಕೆ ಬೇಕಾದುದೆಲ್ಲ ಅಲ್ಲಿದೆ. ಚೇತನ ಭಗವಂತನಿಂದ ಬರುವುದು. ಪ್ರಕೃತಿ ಆ ಚೇತನಕ್ಕೆ ಒಂದು ದೇಹವನ್ನು ಒದಗಿಸುವುದು.

\begin{verse}
ಸತ್ತ್ವಂ ರಜಸ್ತಮ ಇತಿ ಗುಣಾಃ ಪ್ರಕೃತಿಸಂಭವಾಃ~।\\ನಿಬಧ್ನಂತಿ ಮಹಾಬಾಹೋ ದೇಹೇ ದೇಹಿನಮವ್ಯಯಮ್ \versenum{॥ ೫~॥}
\end{verse}

{\small ಅರ್ಜುನ, ಸತ್ತ್ವ ರಜಸ್ಸು, ತಮಸ್ಸು ಎಂಬ ಪ್ರಕೃತಿಯಿಂದ ಹುಟ್ಟಿದ ಗುಣಗಳು ಅವ್ಯಯನಾದ ಆತ್ಮನನ್ನು ದೇಹದಲ್ಲಿ ಬಂಧಿಸುತ್ತವೆ.}

ಯಾವಾಗ ಜೀವಿ ಪ್ರಕೃತಿಯ ಮಾಲಕ ಬರುತ್ತಾನೆಯೋ ಆಗ ಪ್ರಕೃತಿಯ ಗುಣಗಳು ಇವನಿಗೆ ಅಂಟಿಕೊಳ್ಳುತ್ತವೆ. ಅವು ಇವನನ್ನು ದೇಹದಲ್ಲಿ ಬಂಧಿಸುತ್ತವೆ. ಆತ್ಮವಾದರೊ ಅವ್ಯಯ, ಯಾವ ಬದಲಾವಣೆಗೂ ಸಿಕ್ಕಿಲ್ಲ. ಅವನು ಎಲ್ಲಾ ಗುಣಗಳಿಗೆ ಅತೀತನಾದವನು. ಆದರೆ ಈಗ ಈ ಗುಣಗಳ ಬಲೆಗೆ ಸಿಕ್ಕಿದಂತಾಗಿರುವನು. ತನ್ನ ನೈಜ ಗುಣಗಳನ್ನೆಲ್ಲಾ ಕಳೆದುಕೊಂಡಂತಾಗಿರುವನು. ಪ್ರಕೃತಿಯ ಧರ್ಮವನ್ನು ಆರೋಪಣಮಾಡಿಕೊಂಡು ಅದು ತಾನು ಎಂದು ಭ್ರಮಿಸುತ್ತಿರುವನು.

ತಾನೊಂದು ದೇಹವೆಂದು ಭ್ರಮಿಸುವನು. ನೀರು ಒಂದು ಗ್ಲಾಸಿನ ಬುಡ್ಡಿಯೊಳಗೆ ಹೋದಾಗ ಅದಕ್ಕೆ ತಾತ್ಕಾಲಿಕವಾಗಿ ಬುಡ್ಡಿಯ ಬಣ್ಣ ಮತ್ತು ಆಕಾರ ಬರುವುದು. ಆ ನೀರು ತನಗೆ ಯಾವ ಬಣ್ಣವೂ ಇಲ್ಲ, ಆಕಾರವೂ ಇಲ್ಲ ಎಂಬುದನ್ನು ಮರೆತು ಉಪಾಧಿಗಳೆಲ್ಲ ನನ್ನವೇ ಎಂದು ಭ್ರಮಿಸಿದಂತೆ ಜೀವ ಆಗುವುದು. ದೇಹ ತನ್ನದು ಎಂದು ಭ್ರಮಿಸಿದೊಡನೆಯೆ ಅದರ ಮೇಲೆ ಆಸಕ್ತಿ ಹೆಚ್ಚುವುದು. ಅದಅನ್ನು ಚೆನ್ನಾಗಿ ಇಡಲು ಯತ್ನಿಸುವನು. ಅದರಲ್ಲಿ ಇರುವ ಇಂದ್ರಿಯ ಕೇಳಿದ್ದನ್ನು ವಿಷಯ ಪ್ರಪಂಚದಿಂದ ಹುಡುಕಿ ತಂದು ಕೊಡುವನು. ಇವನ ವ್ಯವಹಾರವೆಲ್ಲ ಬಾಹ್ಯದಲ್ಲಿ ಆಗುತ್ತಿರುವುದು. ಮನಸ್ಸು ಮತ್ತು ಬುದ್ಧಿಗಳನ್ನು ತನ್ನ ಸುಖಸಾಧನಗಳಿಗಾಗಿ ಉಪಯೋಗಿಸುವನು. ಆತ್ಮ ಸಂಪೂರ್ಣವಾಗಿ ತನ್ನದಲ್ಲದ ಆದರೆ ಅಕಸ್ಮಾತ್ ಈಗ ಇರುವ ದೇಹದ ಪಂಜರದಲ್ಲಿ ಸಿಕ್ಕಿಬೀಳುವನು. ನೊಣ ಒಂದು ಹಲಸಿನ ಹಣ್ಣನ್ನು ತಿನ್ನಲು ಬಂತು. ತಿಂದಾದ ಮೇಲೆ ಹೋಗಲು ಯತ್ನಿಸಿತು. ಆದರೆ ಕಾಲಿಗೆ ರೆಕ್ಕೆಗೆ ಅಂಟು ಮೆತ್ತಿಕೊಂಡಿತು. ಎಷ್ಟು ಪ್ರಯತ್ನಪಟ್ಟರೂ ಹಾರಿ ಹೋಗಲು ಸಾಧ್ಯವಾಗಲಿಲ್ಲ. ಅಲ್ಲಿಯೇ ಸಾಯಬೇಕಾಯಿತು. ಅದರಂತೆಯೇ ನಾವು ಇಲಿಯನ್ನು ಹಿಡಿಯುತ್ತೇವೆ. ಪಂಜರದಲ್ಲಿ ಯಾವುದಾದರೂ ರುಚಿಕರವಾದ ತಿಂಡಿಯನ್ನು ಸಿಕ್ಕಿಸಿ ಬಾಗಿಲನ್ನು ತೆಗೆದು ಇಡುತ್ತೇವೆ. ಆದರೆ ಇಲಿ ಒಳಗೆ ಹೋದ ತತ್​ಕ್ಷಣವೇ ತಿಂಡಿಗೆ ಬಾಯಿಹಾಕಿದೊಡನೆಯೇ ಹಿಂದಿನಿಂದ ಬಾಗಿಲು ಮುಚ್ಚಿಕೊಳ್ಳುವುದು. ಆ ಚೂರನ್ನು ತಿಂದದ್ದಕ್ಕೆ ತನ್ನ ದೇಹವನ್ನೇ ಬಲಿ ಕೊಡಬೇಕು ಅದು. ಅದರಂತೆಯೇ ನಾವು ಈಗ ಆಗಿದ್ದೇವೆ. ದೇಹದಲ್ಲಿರುವ ಪಂಚೇಂದ್ರಿಯಗಳ ಮೂಲಕ ಅನುಭವಿಸಲು ಬಂದೆವು. ಈಗ ನಾವು ಅದರಿಂದ ಪಾರಾಗಲು ಸಾಧ್ಯವೇ ಇಲ್ಲ. ಅಷ್ಟು ಬಲವಾಗಿ ನಮ್ಮನ್ನು ಅವು ಹಿಡಿದಿವೆ. ಆಧ್ಯಾತ್ಮಿಕ ಜೀವನದ ಸಾಧನೆಯೆಲ್ಲ ಇವುಗಳಿಂದ ಕಳಚಿ ಕೊಂಡು ಹೋಗುವುದಾಗಿದೆ. ಇದಕ್ಕೆ ಎಷ್ಟೊಂದು ಸಾಹಸ ಬೇಕು, ಬುದ್ಧಿವಂತಿಕೆ ಬೇಕು. ಜನ್ಮ ಜನ್ಮಗಳಿಂದಲೂ ನಾವು ಇದಕ್ಕಾಗಿ ಹೋರಾಡುತ್ತಿದ್ದೇವೆ. ಈ ತ್ರಿಗುಣಗಳಲ್ಲಿ ನೆಯ್ದ ಮಾಯೆ ಅಷ್ಟು ಮೋಹಕವಾಗಿದೆ. ದೊಡ್ಡ ದೊಡ್ಡ ದೇವರುಗಳೇ ಅದಕ್ಕೆ ಬಂದುಬಿದ್ದರೆ, ಅದರಿಂದ ಪಾರಾಗಲು ಕಷ್ಟಪಡಬೇಕಾಗುವುದು. ಇನ್ನು ನಮ್ಮ ಪಾಡೇನು? ಇದನ್ನು ವಿವರಿಸುವುದಕ್ಕೆ ಶ‍್ರೀರಾಮಕೃಷ್ಣರು ಒಂದು ಕಥೆಯನ್ನು ಹೇಳುವರು: ಅಸುರನನ್ನು ಕೊಲ್ಲಲು ವಿಷ್ಣು ವರಾಹ ರೂಪವನ್ನು ಧಾರಣೆ ಮಾಡುತ್ತಾನೆ. ಅಸುರನನ್ನು ಕೊಂದಾದ ಮೇಲೆ ತನ್ನ ವೈಕುಂಠಕ್ಕೆ ಹೋಗದೆ ಇಲ್ಲಿ ಮತ್ತೊಂದು ಹಂದಿಯೊಂದಿಗೆ ಸಂಸಾರ ಮಾಡಿಕೊಂಡು ಹಲವು ಮರಿಗಳನ್ನು ಹೊತ್ತುಕೊಂಡು ಗೊಬ್ಬರದ ಗುಂಡಿಯಲ್ಲಿ ವಿಹರಿಸುತ್ತಿದ್ದ. ದೇವತೆಗಳೆಲ್ಲ ಬಂದು ಹೇಳಿದರು. ತನ್ನ ನಿಜಸ್ವರೂಪವನ್ನು ಪಡೆದು ವೈಕುಂಠಕ್ಕೆ ಬಾ ಎಂದು. ಆದರೆ ಹಂದಿಗೆ ಇದೇ ವೈಕುಂಠಕ್ಕಿಂತ ಮಿಗಿಲಾಗಿತ್ತು. ಕೊನೆಗೆ ಇಂದ್ರ ಬಂದು ಅದರ ಕತ್ತನ್ನು ತನ್ನ ಕತ್ತಿಯಿಂದ ಕಡಿಯಬೇಕಾಯಿತು. ಅನಂತರವೇ ಒಳಗಿನಿಂದ ನಗುತ್ತಾ ವಿಷ್ಣು ಎದ್ದು ಬಂದ. ತನ್ನ ಸ್ವಧಾಮಕ್ಕೆ ಹೋದ. ಎಂದರೆ ದೊಡ್ಡ ದೊಡ್ಡ ವ್ಯಕ್ತಿಗಳ ಪಾಡೇ ಹೀಗಾದರೆ ನಮ್ಮಂತಹ ನರಮನುಷ್ಯರ ಪಾಡೇನಾಗಬೇಕು? ಒಬ್ಬ ಇಂದ್ರ ಬಂದು ನಮ್ಮನ್ನು ಕಡಿದಲ್ಲದೆ ಈ ಬಂಧನದಿಂದ ಪಾರಾಗುವುದಿಲ್ಲ. ಇಂದ್ರನಂತೆ ಬರುವವನೆ ಗುರು. ಹಂದಿಯ ರೂಪವನ್ನು ಧಾರಣೆ ಮಾಡಿ ಇಂದ್ರಿಯ ಕೆಸರಿನಲ್ಲಿ ವಿಹರಿಸುತ್ತಿರವವರನ್ನು ಗುರು ಮಾಯೆಯಿಂದ ಬಿಡಿಸಿ ಆತ್ಮ ಸ್ವರಾಜ್ಯದ ಅವರ ತೌರೂರಿಗೆ ಕರೆದುಕೊಂಡು ಹೋಗುವನು.

\begin{verse}
ತತ್ರ ಸತ್ತ್ವಂ ನಿರ್ಮಲತ್ವಾತ್ ಪ್ರಕಾಶಕಮನಾಮಯಮ್~।\\ಸುಖಸಂಗೇನ ಬಧ್ನಾತಿ ಜ್ಞಾನಸಂಗೇನ ಚಾನಘ \versenum{॥ ೬~॥}
\end{verse}

{\small ಅರ್ಜುನ ಅವುಗಳಲ್ಲಿ ಸತ್ತ್ವಗುಣ ನಿರ್ಮಲವಾದುದರಿಂದ ಪ್ರಕಾಶಮಾನವಾಗಿದೆ, ಪಾಪದ ಲೇಪದಿಂದ ಪಾರಾಗಿದೆ. ಅದು ಆತ್ಮನನ್ನು ಸುಖಸಂಗದಿಂದಲೂ ಮತ್ತು ಜ್ಞಾನಸಂಗದಿಂದಲೂ ಬಂಧಿಸುವುದು.}

ಗುಣಗಳಲ್ಲಿ ಸತ್ತ್ವಗುಣ ಶ್ರೇಷ್ಠವಾದುದು. ತಮೋಗುಣ ಮತ್ತು ರಜೋಗುಣದಲ್ಲಿರುವವರಿಗೆ ಸತ್ತ್ವಗುಣವೇ ಒಂದು ಆದರ್ಶವಾಗಿ ತೋರುವುದು. ಆದರೆ ಸತ್ತ್ವಗುಣಕ್ಕೆ ಬಂದ ಮೇಲೆಯೆ ಅದರ ಮಿತಿ ಗೊತ್ತಾಗಬೇಕಾದರೆ. ಪ್ರೈಮರಿ ಶಾಲೆಯಲ್ಲಿ ಓದುತ್ತಿರುವ ಹುಡುಗನಿಗೆ ತನಗೆ ಪಾಠ ಹೇಳುವ ಉಪಾಧ್ಯಾಯ ದೊಡ್ಡ ವಿದ್ಯಾವಂತ. ಆದರೆ ಆ ಮಗು ಕಾಲೇಜಿಗೆ ಏನಾದರೂ ಹೋದರೆ ಆಗ ತನ್ನ ಪ್ರೈಮರಿ ಶಾಲೆಯ ಉಪಾಧ್ಯಾಯರಿಗೆ ತನಗೆ ಪಾಠ ಹೇಳಲು ಆಗುವುದಿಲ್ಲ ಎಂದು ಗೊತ್ತಾಗುವುದು.

ಆ ಸತ್ತ್ವಗುಣದ ವಿವರಗಳನ್ನು ಕೊಡುತ್ತಾನೆ: ಅಲ್ಲಿ ಯಾವ ಕಶ್ಮಲವೂ ಇಲ್ಲ. ಒಂದು ದೀಪ ಬೆಳಗುತ್ತಿದ್ದರೆ, ಅದರ ಸುತ್ತಲೂ ಇರುವ ಶುದ್ಧವಾದ ಚಿಮಿಣಿಯಂತೆ. ಅದರ ಮೂಲಕ ದೀಪದ ಕುಡಿಯು ಹೇಗಿದೆಯೊ ಹಾಗೆ ಕಾಣುವುದು. ಇಂತಹ ಮನುಷ್ಯನಲ್ಲಿರುವ ಗುಣಗಳು ಎಲ್ಲರಿಗೂ ಕಾಣುತ್ತಿರುವುವು. ಅವನಲ್ಲಿ ಯಾವ ತೋರಿಕೆಯೂ ಇಲ್ಲ, ಅಟ್ಟಹಾಸವೂ ಇಲ್ಲ. ಶಿಶು ಸಹಜ ಸರಳತೆಯನ್ನು ನೋಡುವೆವು.

ಅದು ಪ್ರಕಾಶವಾಗಿದೆ. ದೀವಿಗೆಯಂತೆ ಪ್ರಕಾಶಿಸುತ್ತಿದೆ. ಅದರ ಬೆಳಕಿನಲ್ಲಿ ಸುತ್ತಲೂ ಇರುವ ವಸ್ತುವಿನ ನೈಜ ಸ್ಥಿತಿಯನ್ನು ಅರಿಯಬಹುದು. ಅವನಲ್ಲಿ ಜ್ಞಾನ ಉರಿಯುತ್ತಿದೆ. ಅದು ವ್ಯಕ್ತವಾಗು ವುದಕ್ಕೆ ಇರುವ ಆತಂಕಗಳನ್ನೆಲ್ಲ ನಿವಾರಿಸಿಕೊಂಡಿರುವನು. ದೀಪಕ್ಕೆ ಹಾಕುವ ಚಿಮಣಿಯನ್ನು ಚೆನ್ನಾಗಿ ಶುದ್ಧ ಮಾಡಿರುವನು.

ಅನಾಮಯವಾಗಿದೆ. ಅಲ್ಲಿ ಯಾವ ಒಂದು ಲೇಪವೂ ಇಲ್ಲ. ಅಜ್ಞಾನದ ಕೊಳೆಯನ್ನೆಲ್ಲ ಅವನು ಚೆನ್ನಾಗಿ ತೊಳೆದಿರುವನು. ಅವನು ಹೊರಗೆ ವಸ್ತು ಹೇಗಿದೆಯೋ ಹಾಗೆ ನೋಡಬಲ್ಲ. ಅವನು ಕಣ್ಣಿಗೆ ಹಾಕಿಕೊಂಡಿರುವ ಬಣ್ಣದ ಕನ್ನಡಿಯನ್ನು ತೆಗೆದು ಹಾಕಿರುವನು. ಬಣ್ಣದ ಕನ್ನಡಕ ತನ್ನ ಬಣ್ಣವನ್ನು ನೋಡುವ ವಸ್ತುವಿನ ಮೇಲೆಲ್ಲ ಬಳಿಯುವುದು. ಸತ್ತ್ವಗುಣಿ ಇದನ್ನು ತೆಗೆದುಹಾಕಿರುವನು. ಇದರಲ್ಲಿಯೂ ಕೆಲವು ಮಿತಿಗಳಿವೆ. ಇದು ಒಬ್ಬ ಮನುಷ್ಯನನ್ನು ಒಂದು ರೀತಿ ಬಂಧಿಸುವುದು. ಅದೇ ಸುಖದ ಸಂಗ ಮತ್ತು ಜ್ಞಾನದ ಸಂಗ. ಇಲ್ಲಿ ಸುಖ ಎಂದರೆ ವಿಷಯ ವಸ್ತುಗಳ ಮೂಲಕ ಬರುವ ಇಂದ್ರಿಯ ಸುಖವಲ್ಲ. ಸತ್ತ್ವಗುಣಿ ಇದನ್ನು ಮೀರಿ ಹೋಗಿರುವನು. ಇದು ಕಟ್ಟಿ ಹಾಕುವುದು ಒಳ್ಳೆಯ ಸುಖಕ್ಕೆ. ಇದನ್ನೇ ಯೋಗಜೀವನದಲ್ಲಿ ರಸಾಸ್ವಾದನೆ ಎನ್ನುವುದು. ಆನಂದವನ್ನು ಅನುಭವಿಸುತ್ತೇವೆ. ಇದು ಇಂದ್ರಿಯ ಸುಖವಲ್ಲ. ಇಂದ್ರಿಯಾತೀತ ಅನುಭವ. ಆದರೆ ಇದು ಕೂಡ ಒಂದು ಬಂಧನವೇ. ಇದನ್ನು ಕಿತ್ತುಕೊಂಡು ಹೋಗುವುದಕ್ಕೆ ಆಗುವುದಿಲ್ಲ. ಒಂದು ಗೂಟಕ್ಕೆ ಕಟ್ಟಿ ಹಾಕಿದ ದನ ತನ್ನ ದಾರ ಎಷ್ಟು ಉದ್ದವಾಗಿದೆಯೊ ಅಷ್ಟು ದೂರ ಮಾತ್ರ ನಡೆದು ಗೂಟದ ಸುತ್ತಲೂ ಅದು ಮೇಯಬಹುದು. ಅದನ್ನು ಮೀರಿ ಹೋಗುವುದಕ್ಕೆ ಆಗುವುದಿಲ್ಲ. ತಮೋಗುಣಿ, ರಜೋಗುಣಿ ಗಳು ಹಾಗೆಯೇ ಯಾವುದಾದರೂ ಒಂದೊಂದು ಗೂಟಕ್ಕೆ ಬಂಧಿಯಾಗಿರುವರು. ಇವುಗಳಿಗಿಂತ ಮೇಲು ಸತ್ತ್ವಗುಣದ ಗೂಟ. ಇದರ ಸುತ್ತಲೂ ಮೇಯುವುದಕ್ಕೆ ಚೆನ್ನಾದ ಹುಲ್ಲಿದೆ ದನಕ್ಕೆ. ಆದರೆ ಅದನ್ನು ಮೀರಿ ಹೋಗಲಾರದು. ಶ‍್ರೀರಾಮಕೃಷ್ಣರಿಗೆ ಅದ್ವೈತವನ್ನು ಬೋಧನೆ ಮಾಡಲು ತೋತಾ ಪುರಿ ಎಂಬ ದೊಡ್ಡ ಜ್ಞಾನಿ ಬಂದ. ಅವನು ಕೆಲವು ವೇದಾಂತದ ಬೋಧನೆಗಳನ್ನೆಲ್ಲ ಮಾಡಿಯಾದ ಮೇಲೆ ಶ‍್ರೀರಾಮಕೃಷ್ಣರಿಗೆ ಈಗ ಪರಬ್ರಹ್ಮನನ್ನು ಕುರಿತು ಧ್ಯಾನಿಸು ಎಂದ. ಅದು ಎಲ್ಲಾ ನಾಮರೂಪಗಳಾಚೆ, ಎಲ್ಲಾ ಗುಣಗಳಾಚೆ ಇದೆ. ಶ‍್ರೀರಾಮಕೃಷ್ಣರು ಯಾವಾಗ ಧ್ಯಾನಿಸುವುದಕ್ಕೆ ಪ್ರಯತ್ನಿಸಿದರೋ ಇಲ್ಲವೋ ಆಗ ಅವರ ಇಷ್ಟದೇವರಾದ ಕಾಳಿಕಾಮಾತೆ ಅವರ ಮುಂದೆ ಬಂದು ನಿಲ್ಲುತ್ತಿದ್ದಳು. ಎಷ್ಟು ಪ್ರಯತ್ನಮಾಡಿದರೂ ಆ ಗೂಟದಿಂದ ಕಿತ್ತುಕೊಂಡು ಹೋಗುವುದಕ್ಕೆ ಆಗುವುದಿಲ್ಲ. ಅಯ್ಯೊ ನನ್ನ ಕೈಯಲ್ಲಿ ಇದು ಸಾಧ್ಯವಿಲ್ಲ ಎಂದು ಹೇಳಿದಾಗ, ತೋತಾಪುರಿ ಅದು ಸಾಧ್ಯವಾಗಲೇ ಬೇಕು; ನಿನ್ನ ಹುಬ್ಬಿನ ನಡುವೆ ಅದನ್ನು ಧ್ಯಾನಿಸು. ಇನ್ನೊಮ್ಮೆ ಸಗುಣ ದೇವತೆ ನಿನ್ನ ಕಣ್ಣು ಮುಂದೆ ಬಂದು ನಿಂತರೆ, ಜ್ಞಾನಖಡ್ಗದಿಂದ ಅದನ್ನು ಛೇದಿಸು ಎಂದನು. ಹಾಗೆಯೇ ಶ‍್ರೀರಾಮಕೃಷ್ಣರು ಜ್ಞಾನಖಡ್ಗದಿಂದ ಛೇದಿಸಿದಾಗಲೇ ಪರಬ್ರಹ್ಮನ ಉಪಾಸನೆಗೆ ಯೋಗ್ಯರಾದರು. ಸಾಧಾರಣ ಮನುಷ್ಯನಿಗಾದರೋ ದೇವಿಯನ್ನು ನೋಡುವುದೇ ಒಂದು ಗುರಿ. ಆದರೆ ಯಾರು ದೇವಿಯನ್ನು ನೋಡಿರುವರೋ ಅವರು ಅದನ್ನು ಮೀರಿದ ಸತ್ಯವನ್ನು ತಿಳಿದುಕೊಳ್ಳಬೇಕಾಗಿದೆ. ಆದರೆ ಇದುವರೆಗೆ ನಾವು ಅನುಭವಿಸಿದ ಸತ್ಯ ನಮ್ಮನ್ನು ಬಿಟ್ಟುಕೊಡುವುದಿಲ್ಲ. ಅದರಿಂದಲೂ ನಾವು ಕಿತ್ತುಕೊಂಡು ಹೋಗಬೇಕು. ಪರಮ ಸತ್ಯ ನನಗೆ ಬೇಕಾದರೆ ಎಲ್ಲವನ್ನೂ ಬಲಿಕೊಡಲು ಸಿದ್ಧನಾಗಿರ ಬೇಕು. ತನ್ನ ನಂಟರಿಷ್ಟರು ಕೀರ್ತಿ ಗೌರವ ಧನ ಮಾನ ಇವುಗಳನ್ನು ತ್ಯಜಿಸಬೇಕು. ಯಾವುದಕ್ಕೆ ಇದುವರೆಗೂ ಅಂಟಿಕೊಂಡಿರವನೊ ಅದನ್ನು ತ್ಯಜಿಸಿದಲ್ಲದೆ ಪರಮ ಸತ್ಯ ನನಗೆ ಲಭಿಸದು. ಕೆಳಗಿನದನ್ನು ಬಿಡದೆ ಮೇಲಿನದು ನಮಗೆ ದೊರಕುವುದಿಲ್ಲ. ಇದೇ ಸುಖಸಂಗ.

ಅನಂತರ ಜ್ಞಾನವೂ ಕೂಡಾ ನಮ್ಮನ್ನು ಕಟ್ಟಿಹಾಕುವುದು. ಸತ್ತ್ವಗುಣದಿಂದ ಜ್ಞಾನ ಬರುವುದು. ಜ್ಞಾನ ನಮ್ಮನ್ನು ಅಜ್ಞಾನದಿಂದ ಪಾರುಮಾಡುವುದು. ಅನಂತರ ತನಗೇ ನಮ್ಮನ್ನು ದಾಸನನ್ನಾಗಿ ಮಾಡಿಕೊಳ್ಳುವುದು. ಅಜ್ಞಾನಕ್ಕಿಂತ ಜ್ಞಾನ ಮೇಲು. ಅಜ್ಞಾನದ ಮುಳ್ಳು ನಮಗೆ ಮೆಟ್ಟಿರುವಾಗ ಜ್ಞಾನದ ಸೂಜಿಯನ್ನು ತೆಗೆದುಕೊಂಡು ಮುಳ್ಳನ್ನು ಕಿತ್ತುಹಾಕಬೇಕು. ಅನಂತರ ಜ್ಞಾನದ ಸೂಜಿಯನ್ನೂ ಬಿಸುಡಬೇಕಾಗಿದೆ. ಜ್ಞಾನವೇ ಗುರಿಯಲ್ಲ. ಅದೊಂದು ದಾರಿಯಲ್ಲಿ ಸಿಕ್ಕುವ ಸ್ಟೇಷನ್, ಅದೇ ನಾವು ಇಳಿಯಬೇಕಾದ ಸ್ಥಳವಲ್ಲ. ಎಲ್ಲಾ ರೈಲ್ವೆ ಸ್ಟೇಷನ್​ಗಳಿಗಿಂತ ಈ ಸ್ಟೇಷನ್ ನಾವು ಸೇರುವ ಊರಿಗೆ ಸಮೀಪದಲ್ಲಿದೆ. ಆದರೆ ಇಲ್ಲಿಯೇ ಇಳಿದರೆ ನಮ್ಮ ಮನೆ ಸಿಕ್ಕುವುದಿಲ್ಲ. ನಾವು ಇನ್ನೂ ನಡೆಯ ಬೇಕಾಗಿದೆ. ಜ್ಞಾನಕ್ಕೆ ಒಂದು ಮಿತಿ ಇದೆ. ಅದು ಎಲ್ಲವನ್ನೂ ವಿವರಿಸಲಾರದು. ಜ್ಞಾನ ಕೆಲಸ ಮಾಡುವುದೇ ದೇಶ ಕಾಲ ನಿಮಿತ್ತದ ಪ್ರಪಂಚದಲ್ಲಿ. ಆದರೆ ಕಾರ್ಯಕಾರಣ ಸಂಬಂಧ ಎಂಬುದು ಎಲ್ಲವನ್ನೂ ವಿವರಿಸಲಾರದು. ಈ ವಿಶ್ವದಲ್ಲಿ ಯಾವುದೋ ಸಣ್ಣ ಭಾಗ ಮಾತ್ರ ದೇಶ ಕಾಲ ನಿಮಿತ್ತದ ಬೇಲಿಯಲ್ಲಿರುವುದು. ಬಹುಭಾಗ ಅದನ್ನು ಮೀರಿದೆ. ಈ ಭೂಮಿಯನ್ನೆಲ್ಲ ಆವರಿಸಿರುವ ನೀರಿನಲ್ಲಿ ಯಾವುದೋ ಸಣ್ಣ ಭಾಗ ಹಿಮದಂತೆ ಘಟ್ಟಿಯಾಗಿದೆ. ಬಹುಭಾಗ ನೀರಿನಂತೆ ದ್ರವರೂಪದಲ್ಲಿದೆ. ಅದೂ ಅಲ್ಲದೆ ಜ್ಞಾನ ಎಂಬುದು ಸಾಪೇಕ್ಷವಸ್ತು. ಜ್ಞಾನವಿರಬೇಕಾದರೆ ಅಜ್ಞಾನವೂ ಇರಬೇಕು. ಒಂದೇ ಇರಲಾರದು. ಶ‍್ರೀರಾಮಕೃಷ್ಣರು ಈ ದೃಷ್ಟಾಂತವನ್ನು ಕೊಡುತ್ತಾರೆ: ಒಮ್ಮೆ ವಸಿಷ್ಠ ಪುಷಿಗಳು ತಮ್ಮ ಮಕ್ಕಳನ್ನು ಕಳೆದುಕೊಂಡು ವ್ಯಥೆ ಪಟ್ಟರು. ಆಗ ಲಕ್ಷ್ಮಣ ಶ‍್ರೀರಾಮನನ್ನು “ಇಂತಹ ದೊಡ್ಡ ಜ್ಞಾನಿಗಳು ಕೂಡ ಅಳುತ್ತಾರಲ್ಲ?”ಎಂದು ಕೇಳಿದ. ಆಗ ಶ‍್ರೀರಾಮನು ಹೇಳುತ್ತಾನೆ: “ತಮ್ಮ, ಎಲ್ಲಿ ಜ್ಞಾನವಿದೆಯೋ ಅಲ್ಲಿ ಅಜ್ಞಾನವೂ ಇರಬೇಕಾಗಿದೆ” ಎಂದು. ನಮ್ಮ ಗುರಿ ಜ್ಞಾನ ಅಜ್ಞಾನಗಳನ್ನು ಮೀರಿ ಹೋಗುವುದಾಗಿದೆ. ಮುಂಚೆ ಜ್ಞಾನದ ಸಹಾಯವನ್ನು ಪಡೆದುಕೊಂಡು ಅಜ್ಞಾನದಿಂದ ಪಾರಾಗಬೇಕು. ಅನಂತರ ಜ್ಞಾನಕ್ಕೂ ಅತೀತರಾಗಿ ಹೋಗಬೇಕು. ರೋಗವಿರುವಾಗ ಔಷಧಿ ತೆಗೆದು ಕೊಂಡು ಗುಣಮಾಡಿಕೊಳ್ಳುತ್ತೇವೆ. ಅನಂತರ ಔಷಧಿ ತೆಗೆದುಕೊಳ್ಳವುದನ್ನು ನಿಲ್ಲಿಸುತ್ತೇವೆ. ಹಾಗೆಯೇ ಜ್ಞಾನವೆಂಬ ಔಷಧಿ ಕೂಡಾ. ಅದನ್ನು ತೆಗೆದುಕೊಳ್ಳುತ್ತಿರುವುದೇ ನಮ್ಮ ಗುರಿಯಲ್ಲ. ಅದರಿಂದ ನಾವು ಪಾರಾಗಿ ಹೋಗಬೇಕು. ಗುರಿ ಜ್ಞಾನಾರ್ಜನೆಗಳಿಗೆ ಅತೀತವಾದುದು.

\begin{verse}
ರಜೋ ರಾಗಾತ್ಮಕಂ ವಿದ್ಧಿ ತೃಷ್ಣಾಸಂಗಸಮುದ್ಭವಮ್~।\\ತನ್ನಿಬಧ್ನಾತಿ ಕೌಂತೇಯ ಕರ್ಮಸಂಗೇನ ದೇಹಿನಮ್ \versenum{॥ ೭~॥}
\end{verse}

{\small ರಜೊಗುಣ ರಾಗದಿಂದ ಆಗಿರುವುದು. ಇದು ತೃಷ್ಣೆ ಮತ್ತು ಸಂಗ ಇವುಗಳಿಗೆ ಕಾರಣ. ಇದು ಕರ್ಮಸಂಗದಿಂದ ಆತ್ಮನನ್ನು ಬಂಧಿಸುವುದು.}

ಎರಡನೆಯದೇ ಮನುಷ್ಯನನ್ನು ಮುತ್ತಿರುವ ರಜೋಗುಣ. ಇದು ರಾಗದಿಂದ ಆಗಿದೆ, ಆಸೆಯಿಂದ ಆಗಿದೆ. ಅದಕ್ಕೆ ಹೊರಗಿನ ವಸ್ತುಗಳ ಮೇಲೆ ಆಸೆ. ತೆಪ್ಪನೆ ಇರಲಾರದು. ಹೊಸ ಬಯಕೆಗಳು ಮನಸ್ಸಿಗೆ ಬರುತ್ತಲೇ ಇರುವುವು. ಬಂದದ್ದನ್ನೆಲ್ಲಾ ಅನುಭವಿಸಲೇ ಬೇಕು. ಸುಮ್ಮನೆ ಇರುವುದಕ್ಕೆ ಆಗುವು ದಿಲ್ಲ.

ಅದು ತೃಷ್ಣೆಯಿಂದ ಕೂಡಿದೆ, ಬರೀ ಆಸೆಯಲ್ಲ. ಇದು ಆಸೆಯ ಬಾಯಾರಿಕೆ. ಅದನ್ನು ಕೊಟ್ಟೇ ಶಮನ ಮಾಡಬೇಕು. ಏನಾದರೂ ಕೊಟ್ಟರಾದರೂ ಶಮನವಾದೀತೇ? ತೃಷ್ಣೆಯನ್ನು ಯಾರೂ ತೃಪ್ತಿಪಡಿಸಿ ಪಾರಾಗಿಲ್ಲ. ಅದನ್ನು ಒಂದು ಪ್ರಮಾಣದಲ್ಲಿ ತೃಪ್ತಿಪಡಿಸಿದರೆ ಮತ್ತೊಂದು ಪ್ರಮಾಣ ದಲ್ಲಿ ಅದು ಏಳುವುದು. ಸಾಗರಕ್ಕೆ ಹಗಲು ರಾತ್ರಿ ನದಿಗಳು ಬಂದು ಬೀಳುತ್ತಿರುವುವು. ಹಾಗಾದರೂ ಸಾಗರ ಉಕ್ಕಿ ಹರಿದಿದೆಯೇ? ಎಂದಿಗೂ ಇಲ್ಲ. ಹಾಗೆಯೇ ತೃಷ್ಣೆ ಎಂಬುದು. ಏನೋ ಕೇಳುತ್ತದಲ್ಲ ಮನಸ್ಸು ಎಂದು ಕೊಡುತ್ತೇವೆ. ಒಂದು ಸಲ ಕೊಟ್ಟಾದರೂ ತೆಪ್ಪಗಿರಿಸುವ ಮನಸ್ಸಿನ ಚಪಲವನ್ನು ಎಂದು ನೋಡುತ್ತೇವೆ. ಆದರೆ ಮನಸ್ಸಿನ ಚಪಲ ತೀರೀತೆ? ಅದು ಪುನಃ ಪುನಃ ಕೇಳುವುದು, ಯಾವಾಗಲೂ ಕೇಳುತ್ತಲೇ ಇರುವುದು.

ಅದು ಕೇಳುವುದನ್ನು ಕೊಡುವುದಕ್ಕೆ ಮೀಸಲಾಗಿಡಬೇಕು. ಇಷ್ಟಾದರೂ ತೃಪ್ತಿ ಏನಾದರೂ ಬರುವುದೇ? ತೃಪ್ತಿ ಎಂದಿಗೂ ಇಲ್ಲ. ಅದರ ಬದಲು ಅತೃಪ್ತಿ ಜಾಸ್ತಿಯಾಗುವುದು. ಹಣ ಎಷ್ಟಿದ್ದರೂ ಸಾಲದು, ಇನ್ನೂ ಬೇಕು. ಸಾವಿರ ಇರುವವನಿಗೆ ಲಕ್ಷದ ಮೇಲೆ ಆಸೆ, ಲಕ್ಷ ಇರುವವನಿಗೆ ಕೋಟಿಯ ಮೇಲೆ ಆಸೆ. ಕೋಟಿ ಇರುವವನಿಗೆ ಹತ್ತು ಕೋಟಿಯ ಮೇಲೆ ಆಸೆ. ಎಲ್ಲಿ ನೋಡಿದರೂ ತನ್ನಿಂದ ಮೇಲೆ ಇರುವವನಂತೆ ತಾನು ಯಾವಾಗ ಆಗುತ್ತೇನೆಯೋ ಎಂದು ಚಿಂತಿಸುತ್ತಿರುವನೆ ಹೊರತು, ಇರುವ ಸ್ಥಿತಿಯಲ್ಲಿ ತೃಪ್ತಿಯಿಲ್ಲ. ಹಾಗೆಯೇ ತೃಪ್ತಿಯ ಒಲೆಗೆ ಯಾವ ಸೌದೆಯನ್ನು ಹಾಕಿದರೂ ಬಹಳ ಬೇಗ ಉರಿದು ಹೋಗುವುದು. ಅದರೊಳಗೆ ಇರುವ ಧಗಧಗಿಸುವ ಕೆಂಡ ಬೇಗ ಸೌದೆಯನ್ನು ತೆಗೆದುಕೊಂಡು ಬಾ ಎಂದು ಕಾಡುವುದು. ಆ ಸೌದೆಯನ್ನು ನ್ಯಾಯವಾಗಿಯೋ ಅನ್ಯಾಯವಾಗಿಯೋ ಯಾರಿಂದಲೋ ತಂದು ಅಂತೂ ಯಾವಾಗಲೂ ಒಲೆಗೆ ಒಟ್ಟುತ್ತಿರುವುದೇ ನಮ್ಮ ಬಾಳಿನ ಗುರಿಯಾಗುವುದು.

ಈ ತೃಷ್ಣೆಯಿಂದ ಸಂಗ ಉಂಟಾಗುವುದು, ಆ ವಸ್ತುವಿಗೆ ಕಟ್ಟಿಹಾಕಿಕೊಳ್ಳುತ್ತೇವೆ, ಅದರ ಮೇಲೆ ಸ್ವಾಮಿತ್ವವನ್ನು ಸ್ಥಾಪಿಸುತ್ತೇವೆ. ಅದು ನನಗೆ ಸೇರಿದ್ದು ಎಂದು ಭಾವಿಸುತ್ತೇವೆ. ಅದರ ಸುತ್ತಲೂ ಬೇಲಿ ಕಟ್ಟುತ್ತೇವೆ. ಅದನ್ನು ನಾನು ಮಾತ್ರ ಅನುಭವಿಸಬೇಕು, ಇತರರು ಯಾರೂ ಅದನ್ನು ಮುಟ್ಟಕೂಡದು, ಅನುಭವಿಸಕೂಡದು. ಹಾಗೇನಾದರೂ ಅವರು ಮಾಡಿದರೆ ನಮಗೆ ಅವರ ಮೇಲೆ ಕೋಪ. ನಾವು ಅದನ್ನು ಉಪಯೋಗಿಸಲಿ ಬಿಡಲಿ, ಅದು ಯಾವಾಗಲೂ ನನ್ನ ಹತ್ತಿರವೇ ಇರಬೇಕು. ಕ್ಷಣ ಕಾಲವೂ ಅದನ್ನು ಮರೆಯುವುದಕ್ಕೆ ಆಗುವುದಿಲ್ಲ, ಅದರಿಂದ ದೂರವಿರುವುದಕ್ಕೆ ಆಗುವು ದಿಲ್ಲ. ಒಂದು ಕಾಣದ ಹಗ್ಗದಿಂದ ನಾವು ಅದಕ್ಕೆ ಕಟ್ಟಿ ಹಾಕಿಕೊಳ್ಳುತ್ತೇವೆ.

ಇದು ಕರ್ಮಸಂಗದಿಂದ ಆತ್ಮನನ್ನು ಬಂಧಿಸುತ್ತದೆ. ಜೀವಿಗೆ ಯಾವಾಗ ತೃಷ್ಣೆ ಇದೆಯೋ, ಆಗ ಅವನು ಸುಮ್ಮನೆ ಇರಲಾರ. ಅದನ್ನು ತೃಪ್ತಿಪಡಿಸಿಕೊಳ್ಳುವುದಕ್ಕೆ ಏನಾದರೂ ಕರ್ಮವನ್ನು ಅವನು ಮಾಡಲೇ ಬೇಕಾಗಿದೆ. ತೃಷ್ಣೆ ಎಂಬುದು ಅಂಕುಶದಂತೆ ಯಾವಾಗಲೂ ನಮ್ಮನ್ನು ತಿವಿಯುತ್ತಿರು ವುದು. ನಾವು ಸುಮ್ಮನಿರುವುದಕ್ಕೆ ಆಗುವುದಿಲ್ಲ. ಕೆಲಸ ಮಾಡಲೇ ಬೇಕಾಗುವುದು. ತೃಪ್ತಿಪಡಿಸಿ ಕೊಳ್ಳುವುದಕ್ಕೆ ಕೆಲಸ ಮಾಡುವಾಗ ಯಾವುದಾದರೂ ಪೆಟ್ಟು ಬೀಳಬಹುದು. ಆಗ ಸ್ವಲ್ಪ ದಿಗ್ಭ್ರಾಂತ ನಾಗಿ ಸ್ವಲ್ಪ ಕಾಲ ನಿಲ್ಲಿಸಿರುತ್ತಾನೆ. ಆ ಮೇಲೆ ಪುನಃ ಶುರುಮಾಡುವನು. ಹೇಗೆ ಗಾಣಕ್ಕೆ ಕಟ್ಟಿದ ಎತ್ತು ಯಾವಾಗಲೂ ಗಾಣದ ಸುತ್ತಲೂ ಸುತ್ತುತ್ತಿರುಬೇಕೋ ಹಾಗೆ ತೃಷ್ಣೆಯ ಗಾಣಕ್ಕೆ ಕಟ್ಟಿದ ಬಡ ಎತ್ತುಗಳಂತೆ ನಾವು ಹಗಲು ರಾತ್ರಿ ದುಡಿಯುತ್ತಿರುವೆವು. ಅದಕ್ಕೆ ಕೆಲವು ವೇಳೆ ದೊಡ್ಡ ಹೆಸರುಗಳನ್ನು ಕೊಡುತ್ತೇವೆ. ಇದು ನಮ್ಮ ಕರ್ತವ್ಯ, ಧರ್ಮ, ನಾವು ಇದನ್ನು ಮಾಡದೆ ಇದ್ದರೆ ಇನ್ನಾರು ಮಾಡುವರು ಇತ್ಯಾದಿಯಾಗೆಲ್ಲ ಹೇಳಿ ಅಂತೂ ತೃಷ್ಣೆಯ ನೊಗಕ್ಕೆ ನಮ್ಮ ಕತ್ತನ್ನು ಕೊಟ್ಟು ಈ ಕರ್ಮದ ಗಾಡಿಯನ್ನು ಎಳೆಯುತ್ತಿರುವೆವು. ಇದು ನಮ್ಮ ಹಣೆಯ ಬರಹ, ಇದರಿಂದ ನಾವು ತಪ್ಪಿಸಿಕೊಂಡು ಬರಲು ಸಾಧ್ಯವಿಲ್ಲ, ಅದಕ್ಕಾಗಿ ಇದನ್ನು ಮಾಡುತ್ತೇವೆ ಎಂದು ಅರಿಯುವವರು ಬಹಳ ಅಲ್ಪ ಮಂದಿ.

\begin{verse}
ತಮಸ್ತ್ವಜ್ಞಾನಜಂ ವಿದ್ಧಿ ಮೋಹನಂ ಸರ್ವದೇಹಿನಾಮ್~।\\ಪ್ರಮಾದಾಲಸ್ಯನಿದ್ರಾಭಿಸ್ತನ್ನಿಬಧ್ನಾತಿ ಭಾರತ \versenum{॥ ೮~॥}
\end{verse}

{\small ಅರ್ಜುನ, ತಮಸ್ಸು ಅಜ್ಞಾನದಿಂದ ಹುಟ್ಟಿರುವುದು. ಇದು ಸಮಸ್ತ ಪ್ರಾಣಿಗಳನ್ನು ಮೋಹಪಡಿಸುವುದು ಎಂದು ತಿಳಿ. ಪ್ರಮಾದ, ಆಲಸ್ಯ, ನಿದ್ರೆ ಇವುಗಳಿಂದ ಬಂಧಿಸುವುದು.}

ಮನುಷ್ಯನನ್ನು ಆವರಿಸಿರುವ ಮತ್ತೊಂದು ಗುಣವೇ ತಮಸ್ಸು. ಇದು ಗುಣಗಳಲ್ಲೆಲ್ಲ ಬಹಳ ಕೆಳಗಿನ ಮಟ್ಟದ್ದು. ಈ ಗುಣಕ್ಕೆ ಕಾರಣ ಅಜ್ಞಾನ. ಬುದ್ಧಿಯ ಕಾಂತಿಯನ್ನು ಬರದಂತೆ ತಡೆಗಟ್ಟು ವುದು. ಒಂದು ವಿದ್ಯುದ್ದೀಪ ಬೆಳಗುತ್ತಿದ್ದರೆ, ಅದಕ್ಕೆ ದಪ್ಪವಾದ ಒಂದು ಬಟ್ಟೆಯನ್ನು ಪದರ ಪದರಗಳಾಗಿ ಕಟ್ಟಿದಂತಿದೆ. ಆತ್ಮಪ್ರಭೆಯನ್ನು ಹೊರಗೆ ಬರದಂತೆ ಮಾಡುವುದು. ಇದು ವಸ್ತುಗಳ ಮೇಲೆ ಬಿದ್ದರೆ ಮಾತ್ರ ಅದರ ಯಥಾರ್ಥ ಸ್ಥಿತಿಯನ್ನು ಅರಿಯಬಹುದು. ಆದರೆ ಈ ಆವರಣ ವಾದರೊ ಕಾಂತಿ ಹೊರಗೆ ಬರುವುದಕ್ಕೆ ಆಸ್ಪದ ಕೊಡುವುದಿಲ್ಲ.

ತಮೋಗುಣ ಎಲ್ಲಾ ಪ್ರಾಣಿಗಳನ್ನೂ ಮೋಹಪಡಿಸುವುದು. ವಸ್ತುವಿನ ನೈಜ ಸ್ಥಿತಿಯನ್ನು ಮರೆಸುವುದು. ಮುಂದೇನಾಗುವುದು ಒಂದು ವಸ್ತು ಅನುಭವಿಸಿದರೆ ಎಂದು ಪರ್ಯಾಲೋಚನೆಯನ್ನೇ ಮಾಡುವುದಿಲ್ಲ. ಒಂದು ಇದ್ದರೆ ಮತ್ತೊಂದು ರೀತಿ ಅದನ್ನು ಅನುಭವಿಸುವುದು. ಎಲ್ಲಿ ಬಂಧನ ವಿದೆಯೋ ಅಲ್ಲೇ ಹೋಗುವುದು ಅದರ ಸ್ವಭಾವ. ಯಾರು ಏನು ಹೇಳಿದರೂ ಅದಕ್ಕೆ ಕಿವಿಗೊಡುವು ದಿಲ್ಲ. ಇಲಿ ಬೋನಿನೊಳಗೆ ಧಾವಿಸುವುದಕ್ಕೆ ಪ್ರಯತ್ನಿಸುವುದೇ ಹೊರತು, ಹೋದರೆ ಏನು ಗತಿ ಎಂದು ಆಲೋಚಿಸುವುದಿಲ್ಲ.

ಇದು ಮನುಷ್ಯನನ್ನು ಪ್ರಮಾದ ಆಲಸ್ಯ ನಿದ್ರೆ ಇವುಗಳಿಂದ ಬಂಧಿಸುವುದು. ಪ್ರಮಾದ ಎಂದರೆ ತಪ್ಪು ಮಾಡುವುದು. ಯಾವ ಕಾಲದಲ್ಲಿ ಏನು ಮಾಡಬೇಕೋ ಅದನ್ನು ಮರೆಯುವುದು. ಅಷ್ಟೇ ಅಲ್ಲ ಏನು ಮಾಡಬಾರದೋ ಅದನ್ನು ಮಾಡುವುದು. ಎರಡರಿಂದಲೂ ಕಷ್ಟ ಕಾದಿದೆ. ಎರಡನೆಯದೆ ಆ ಮನುಷ್ಯ ಆಲಸಿ. ಅವನ ಬುದ್ಧಿ ಯೋಚಿಸುವುದಿಲ್ಲ. ಯಾವಾಗಲೂ ಯೋಚನೆ ಮಾಡುವುದು ಮನಸ್ಸಿಗೆ ಕಷ್ಟ. ಅದರ ನೆಮ್ಮದಿಗೆ ಭಂಗ ಬರುವುದು. ಅದರಂತೆಯೇ ಮೈಬಗ್ಗಿ ಕೆಲಸ ಮಾಡುವು ದಿಲ್ಲ. ಏನಾದರೂ ಮಾಡಬೇಕಾಗಿ ಬಂದರೆ, ನಾಳೆ ಮಾಡಿದರಾಯಿತು, ಎಂದು ಮುಂದೆ ಹಾಕುತ್ತಿರು ವನು. ಅವನೇನು ನಿಜವಾಗಿ ನಾಳೆ ಮಾಡಬೇಕೆಂದು ಅಲ್ಲ ಇರುವುದು. ನಾಳೆ ತನಕವಾದರೂ ಸುಮ್ಮನಿರುವ ಅಥವಾ ನಾಳೆ ಹೊತ್ತಿಗೆ ಆ ಕೆಲಸವನ್ನು ಮಾಡುವ ಪ್ರಮೇಯವೇ ತಪ್ಪಬಹುದು ಎಂದು ಆಲೋಚಿಸುವನು. ಮೂರನೆಯದೇ ನಿದ್ರೆ. ತಮಸ್ಸಿನ ಒಂದು ಪ್ರಧಾನ ಲಕ್ಷಣ ಇದು. ಅವನು ಎಲ್ಲರಿಗಿಂತ ಹೆಚ್ಚಾಗಿ ನಿದ್ರೆ ಮಾಡುವನು. ಯಾವಾಗ ಒಬ್ಬ ಅತಿ ನಿದ್ರಿಸುವನೋ ಅವನ ಬುದ್ಧಿ ಜಡವಾಗುವುದು. ಯಾವುದನ್ನೂ ಸರಿಯಾಗಿ ವಿಚಾರಿಸಲಾರ. ನಿದ್ರೆಯೇ ಇರಬಾರದು ಎಂದಲ್ಲ. ಶ‍್ರೀಕೃಷ್ಣನೇ ಇದಕ್ಕೆ ವಿರೋಧ. ಅತಿ ನಿದ್ರೆ ಎಷ್ಟು ಅಪಾಯಾಕಾರಿಯೋ ಹಾಗೆಯೇ ನಿದ್ರೆ ಇಲ್ಲದೆ ಇರುವುದು ಕೂಡಾ ಅಪಾಯಾಕಾರಿ. ಅದರಲ್ಲಿ ಒಂದು ಸಮತ್ವ ಇರಬೇಕು. ಈ ತಮೋಗುಣಿ ಯಲ್ಲಿ ಅದನ್ನು ಕಾಣುವುದಿಲ್ಲ.

\begin{verse}
ಸತ್ತ್ವಂ ಸುಖೇ ಸಂಜಯತಿ ರಜಃ ಕರ್ಮಣಿ ಭಾರತ~।\\ಜ್ಞಾನಮಾವೃತ್ಯ ತು ತಮಃ ಪ್ರಮಾದೇ ಸಂಜಯತ್ಯುತ \versenum{॥ ೯~॥}
\end{verse}

{\small ಅರ್ಜುನ, ಸತ್ತ್ವಗುಣ ಸುಖದಲ್ಲಿ ಸಂಗವನ್ನು ಉಂಟುಮಾಡುವುದು. ರಜೋಗುಣ ಕರ್ಮದಲ್ಲಿ ಸಂಗವನ್ನು ಉಂಟುಮಾಡುವುದು. ಆದರೆ ತಮೋಗುಣ ಜ್ಞಾನವನ್ನು ಆವರಿಸಿಕೊಂಡು ಪ್ರಮಾದದಲ್ಲಿ ಸಂಗವನ್ನು ಉಂಟು ಮಾಡುವುದು.}

ಈ ಮೂರು ಗುಣಗಳೂ ಕೂಡ ಜೀವಿಯನ್ನು ಕಟ್ಟಿಹಾಕುವುವು. ಸತ್ತ್ವಗುಣ ಮೇಲೆ ಕಟ್ಟಿಹಾಕು ವುದು. ರಜೋಗುಣ ಮಧ್ಯದಲ್ಲಿ, ತಮೋಗುಣ ಕೆಳಗಡೆ ನಮ್ಮನ್ನು ಕಟ್ಟಿಹಾಕುವುದು. ತಮೋಗುಣ ಕೀಳು, ರಜೋಗುಣ ಮೇಲು, ರಜೋಗುಣಕ್ಕಿಂತ ಸತ್ತ್ವಗುಣ ಮೇಲೆ. ಆದರೆ ಸತ್ತ್ವಕ್ಕೂ ಅತೀತನಾಗಿ ಹೋಗಬೇಕು. ಗುಣಾತೀತನಾಗುವುದೇ ನಮ್ಮ ಗುರಿ.

ಸತ್ತ್ವಗುಣ ಕೆಳಮಟ್ಟದಿಂದ ಮನಸ್ಸನ್ನು ಮೇಲಕ್ಕೆ ತರುವುದು. ಆದರೆ ಯಾವಾಗ ಮನಸ್ಸು ಅದನ್ನು ಅತಿಕ್ರಮಿಸಲು ಯತ್ನಿಸುವುದೋ, ಆಗ ಅದನ್ನು ತಡೆಯುವುದು. ಸತ್ತ್ವಗುಣ ರಸಾಸ್ವಾದನೆ ಯಲ್ಲಿ ನಮ್ಮನ್ನು ಕಟ್ಟಿ ಹಾಕುವುದು. ಶಾಂತಿ, ಆನಂದ, ಜ್ಞಾನ ಇವುಗಳಿಂದ ನಮ್ಮನ್ನು ಬಂಧಿಸು ವುದು. ಈ ಗುಣಗಳೇನೊ ಒಳ್ಳೆಯವು. ಆದರೆ ಬಂಧನದ ದೃಷ್ಟಿಯಿಂದ ನೋಡಿದರೆ, ಒಂದು ಕೈಗೆ ಹಾಕಿರುವ ಕಬ್ಬಿಣದ ಬೇಡಿ, ಮತ್ತೊಂದು ಕೈಗೆ ಹಾಕಿಕೊಂಡಿರುವ ಚಿನ್ನದ ಬೇಡಿ. ಎರಡೂ ಬೇಡಿಯೇ.

ರಜೋಗುಣದ ಸ್ವಭಾವವೇ ಯಾವಾಗಲೂ ಪಾದರಸದಂತೆ ಚಲಿಸುವುದು. ಯಾವಾಗಲೂ ಏನನ್ನಾದರೂ ಮಾಡುತ್ತಿರುವುದು. ಸುಮ್ಮನೆ ಎಂದಿಗೂ ಕುಳಿತುಕೊಳ್ಳುವುದಿಲ್ಲ. ಮನಸ್ಸಿನಲ್ಲಿ ಹಲವಾರು ಆಸೆ ಆಕಾಂಕ್ಷೆಗಳಿವೆ. ಅವು ಯಾವಾಗಲೂ ನಮ್ಮನ್ನು ತಿವಿಯುತ್ತಿವೆ, ಅದು ಮಾಡು ಇದು ಮಾಡು ಎಂದು. ನಾವು ಸುಮ್ಮನೆ ಕುಳಿತುಕೊಳ್ಳುವುದಕ್ಕೆ ಆಗುವುದಿಲ್ಲ. ಉರಿಯುತ್ತಿರುವ ಒಲೆಯ ಮೇಲೆ ಪಾತ್ರೆಯನ್ನು ಇಟ್ಟರೆ ಅದರಲ್ಲಿರುವ ನೀರು ಕುದಿಯುವುದಕ್ಕೆ ಮೊದಲಾಗುವುದು. ಅದು ಕುದಿಯದೆ ಬಹಳ ಕಾಲ ಸುಮ್ಮನೆ ಇರುವುದಕ್ಕೆ ಆಗುವುದಿಲ್ಲ. ಆಸೆಯ ಕೊಳ್ಳಿಗಳು ಬಲವಾಗಿ ಉರಿಯುತ್ತಿವೆ ಕೆಳಗೆ. ಮೇಲಿರುವ ವಸ್ತುವನ್ನು ಸುಮ್ಮನೆ ಇರಗೊಡಿಸುವುದಿಲ್ಲ.

ತಮೋಗುಣವಾದರೊ ಬುದ್ಧಿಯನ್ನು ಜಡಮಾಡುವುದು. ಕೆಲಸಮಾಡುವುದಕ್ಕೆ ಪ್ರಯತ್ನಪಟ್ಟರೆ ಬುದ್ಧಿ ಕರಗುವುದು. ಅದು ಯಾವುದು ಸರಿ, ಯಾವುದು ತಪ್ಪು ಎಂಬುದನ್ನು ನಿಷ್ಕರ್ಷಿಸುವ ಸ್ಥಿತಿಯಲ್ಲಿ ಇರುವುದಿಲ್ಲ. ಸುಮ್ಮನೆ ಕುಳಿತಿರಲಾರದೆ ಏನೇನೋ ಮಾಡುವನು. ಅದು ಯಾವಾಗಲೂ ತಪ್ಪೇ ಆಗಿರುವುದು. ನೋಡಿದವರು ಅವನು ಆ ಕೆಲಸವನ್ನು ಮಾಡದೆ ಇದ್ದರೆ ಮೇಲು ಎಂದು ಭಾವಿಸುವರು. ಆದರೆ ಇವನಂತು ಸುಮ್ಮನೆ ಇರುವುದಿಲ್ಲ. ತಪ್ಪುತಪ್ಪಾಗಿ ಏನನ್ನೋ ಮಾಡಿಹಾಕು ವನು.

\begin{verse}
ರಜಸ್ತಮಶ್ಚಾಭಿಭೂಯ ಸತ್ತ್ವಂ ಭವತಿ ಭಾರತ~।\\ರಜಃ ಸತ್ತ್ವಂ ತಮಶ್ಚೈವ ತಮಃ ಸತ್ತ್ವಂ ರಜಸ್ತಥಾ \versenum{॥ ೧ಂ~॥}
\end{verse}

{\small ಅರ್ಜುನ, ರಜಸ್ಸು ತಮಸ್ಸನ್ನು ಅಡಗಿಸಿ ಸತ್ತ್ವಗುಣ ಮೇಲೇಳುವುದು. ಸತ್ತ್ವಗುಣ ತಮೋಗುಣಗಳನ್ನು ಅಡಗಿಸಿ ರಜೋಗುಣ ಮೇಲೇಳುವುದು. ಸತ್ತ್ವ ಮತ್ತು ರಜೋಗುಣಗಳನ್ನು ಅಡಗಿಸಿ ತಮಸ್ಸು ಮೇಲೆ ಏಳುವುದು.}

ಮೂರು ಗುಣಗಳೂ ಎಲ್ಲರಲ್ಲಿಯೂ ಇರುತ್ತವೆ. ಆದರೆ ಒಬ್ಬೊಬ್ಬರಲ್ಲಿ ಒಂದೊಂದು ಮೇಲೆದ್ದು ಕಾಣುತ್ತದೆ. ಅದನ್ನೇ ಆ ಹೆಸರಿನಿಂದ ಕರೆಯುತ್ತಾರೆ. ಒಂದಿರುವಾಗ ಮತ್ತೆರಡು ಅದರ ಹಿಂಬದಿಯಲ್ಲಿಇರುತ್ತವೆ. ಯಾವಾಗ ಸತ್ತ್ವಗುಣ ಅಧಿಕವಾಗಿರುವುದೋ ಆಗ ರಜೋಗುಣ ಮತ್ತು ತಮೋಗುಣಗಳು ಹಿಂಬದಿಯಲ್ಲಿ ಇರುತ್ತವೆ. ತಮೋಗುಣ ಅಧಿಕವಾಗಿರುವಾಗ ಸತ್ತ್ವ ಮತ್ತು ರಜೋಗುಣಗಳು ಹಿಂಬದಿಯಲ್ಲಿರುತ್ತವೆ. ಎಂತಹ ದೊಡ್ಡ ಸಾತ್ವಿಕ ಜೀವಿಯಾದರೂ, ಅವನು ಏನಾದರೂ ಕೆಲಸಮಾಡುತ್ತಾನೆ, ಅದು ರಾಜಸ; ಅವನು ನಿದ್ರಿಸುತ್ತಾನೆ, ಅದು ತಾಮಸಿಕ. ಆದರೆ ಇವೆರಡೂ ಸತ್ತ್ವಗುಣಕ್ಕೆ ಅಧೀನವಾಗಿವೆ. ಎಂತಹ ಪ್ರಚಂಡ ರಜೋಗುಣಿಯಾದರೂ ಕೆಲವು ವೇಳೆ ಶಾಂತಿ ಜ್ಞಾನ ಆನಂದಗಳನ್ನು ಅನುಭವಿಸುತ್ತಾನೆ. ಇದು ಸಾತ್ತ್ವಿಕ. ಮತ್ತು ಇಂತಹ ವ್ಯಕ್ತಿಯೂ ನಿದ್ರಿಸುತ್ತಾನೆ. ನಿದ್ರೆ, ಆಲಸ್ಯ, ಮರೆವು ಮುಂತಾದುವು ಅವನನ್ನೂ ಕಾಡುತ್ತವೆ. ಆದರೆ ತಾಮಸಿ ಯಷ್ಟು ಹೆಚ್ಚಲ್ಲ. ಇದೇ ರಾಜಸಿಕ ಗುಣ. ಹಾಗೆಯೇ ಬರೀ ತಮೋಗುಣಿಯೂ ಕೂಡ. ಅವನಲ್ಲಿ ಸತ್ತ್ವ ಮತ್ತು ರಜೋಗುಣ ಇಲ್ಲದೆ ಇಲ್ಲ. ಅದರ ಅಂಶ ಕಡಿಮೆ ಅಷ್ಟೆ.

\begin{verse}
ಸರ್ವದ್ವಾರೇಷು ದೇಹೇಽಸ್ಮಿನ್ ಪ್ರಕಾಶ ಉಪಜಾಯತೇ~।\\ಜ್ಞಾನಂ ಯದಾ ತದಾ ವಿದ್ಯಾದ್ವಿವೃದ್ಧಂ ಸತ್ತ್ವಮಿತ್ಯುತ \versenum{॥ ೧೧~॥}
\end{verse}

{\small ಯಾವಾಗ ಈ ದೇಹದ ಸಕಲ ಇಂದ್ರಿಯಗಳ ಮೂಲಕವೂ ಜ್ಞಾನವೆಂಬ ಪ್ರಕಾಶ ಬೆಳಗುವುದೋ ಆಗ ಸತ್ತ್ವಗುಣ ವೃದ್ಧಿಯಾಗಿದೆ ಎಂದು ತಿಳಿಯಬೇಕು.}

ಇನ್ನು ಮೇಲೆ ಯಾವಾಗ ಯಾವ ಗುಣ ವ್ಯಕ್ತವಾಗುತ್ತಿದೆ ಎಂಬುದನ್ನು ವಿವರಿಸುತ್ತಾನೆ. ಮನುಷೃನಿಗೆ ಇಂದ್ರಿಯಗಳೆಲ್ಲ ಕಿಟಕಿಗಳಂತೆ. ಅದರ ಮೂಲಕ ಬರುವ ಕಾಂತಿಯಿಂದ ಒಳಗೆ ಏನಿದೆ ಎಂಬುದನ್ನು ಹೇಳಬಹುದು. ಸತ್ತ್ವಗುಣಿ ಯಾವುದನ್ನೂ ಬಚ್ಚಿಡುವುದಿಲ್ಲ. ಅವನಲ್ಲಿ ಒಳಗೆ ಒಂದು ಹೊರಗೆ ಒಂದು ಇಲ್ಲ. ಒಳಗಿರುವುದೆಲ್ಲ ಕಾಣುತ್ತಿದೆ. ಗ್ಲಾಸಿನ ಆಲ್ಮೈರಾ ಹಿಂದುಗಡೆ ಏನಿದೆಯೋ ಅದು ಯಾವಾಗಲೂ ಕಾಣುತ್ತಿರುವಂತೆ ಅವನಲ್ಲಿರುವುದು ಹೊರಗೆ ವ್ಯಕ್ತವಾಗುತ್ತಿರುವುದು.

ಕರ್ಮೇಂದ್ರಿಯ ಜ್ಞಾನೇಂದ್ರಿಯದ ಮೂಲಕ ಒಳಗಿರುವ ಜ್ಞಾನ ಪ್ರಕಾಶವಾಗುತ್ತಿರುವುದು. ಇದೇ ಅವನು ಸತ್ತ್ವಗುಣಿ ಎಂಬುದನ್ನು ತೋರುವುದು. ಅವನು ತಾನು ಜ್ಞಾನಿ ಎಂದು ಹೇಳಿಕೊಳ್ಳುವುದಿಲ್ಲ. ಆದರೆ ಅವನ ನಡೆನುಡಿ ನೋಟ ಅಭಿರುಚಿ ಇವುಗಳ ಮೂಲಕವಾಗಿಯೇ ಅವನಲ್ಲಿರುವ ಜ್ಞಾನ ವ್ಯಕ್ತವಾಗುವುದು. ನಾವು ತಿಂದದ್ದು ನಮ್ಮ ತೇಗಿನಲ್ಲಿ ಗೊತ್ತಾಗುವಂತೆ, ಸತ್ತ್ವಗುಣವನ್ನು ಅವನ ಕ್ರಿಯೆಗಳ ಮೂಲಕ ಕಂಡುಹಿಡಿಯಬಹುದು. ಅವನಾವಾಗಲೂ ಒಳ್ಳೆಯ ವಿಷಯಗಳನ್ನು ಮಾತ ನಾಡುತ್ತಾನೆ. ಆ ಮಾತು ಸುಮ್ಮನೆ ಆಡಿದ್ದಲ್ಲ. ಅವನು ಚೆನ್ನಾಗಿ ವಿಚಾರಮಾಡಿ ಸತ್ಯವನ್ನು ಕಂಡುಹಿಡಿದಿರುವನು. ಅವನಾಡುವ ಮಾತುಗಳು ತೂಕದ ಮಾತುಗಳು. ಅವು ಬರೀ ಉದ್ವಿಗ್ನತೆ ಯಿಂದ ಬಂದವಲ್ಲ. ಅವನು ಯಾರನ್ನೂ ನೋಯಿಸುವುದಿಲ್ಲ. ತಪ್ಪನ್ನು ಕಂಡುಹಿಡಿಯುವುದಿಲ್ಲ. ಇನ್ನೊಬ್ಬನಲ್ಲಿರುವ ತಪ್ಪನ್ನು ನಿಕೃಷ್ಚವಾಗಿ ಕಾಣುತ್ತಾನೆ, ಅವನು ಒಳ್ಳೆಯದನ್ನು ಸ್ಪಷ್ಟವಾಗಿ ನೋಡುತ್ತಾನೆ. ಅವನ ಅಂತರಾಳದಲ್ಲಿ ಶಾಂತಿ ಇದೆ. ವಸ್ತುವಿನ ಯಥಾರ್ಥಸ್ಥಿತಿಯನ್ನು ತಿಳಿದು ಕೊಂಡಿರುವನು ಅವನು. ಹಾಗೆಯೇ ಅವನ ನೋಟ. ಅದರಲ್ಲಿ ಚಾಂಚಲ್ಯವಿಲ್ಲ. ಅಶುಚಿಯ ಕಡೆ ಕಣ್ಣನ್ನು ಹೊರಳಿಸುವುದಿಲ್ಲ. ಪವಿತ್ರವಾದುದನ್ನು ನೋಡುತ್ತಾನೆ. ಅಶ್ಲೀಲವನ್ನು ನೋಡಬೇಕಾಗಿ ಬಂದರೂ ಅದರ ಹಿಂದೆ ಇರುವ ಪವಿತ್ರತೆಯನ್ನು ಮರೆಯುವುದಿಲ್ಲ. ಅವನ ಕೆಲಸವೂ ಹಾಗೆಯೇ ಏನುಮಾಡಿದರೂ ತನ್ನ ಲಾಭಕ್ಕೆ ಕೀರ್ತಿಗೆ ಸ್ವಾರ್ಥಕ್ಕೆ ಮಾಡುವುದಿಲ್ಲ. ಎಲ್ಲವನ್ನೂ ಭಗವದರ್ಪಿತ ವಾಗಲಿ ಎಂಬ ದೃಷ್ಟಿಯಿಂದ ಮಾಡುವನು. ಅವನು ಮಾಡುವ ಕೆಲಸದ ಹಿಂದೆ ಗಲಾಟೆಯಿಲ್ಲ, ಉದ್ವೇಗವಿಲ್ಲ. ಇನ್ನೊಬ್ಬನಿಗೆ ತೋರಿಸಿಕೊಳ್ಳುವುದಕ್ಕಾಗಿ ಅವನು ಏನನ್ನೂ ಮಾಡುವುದಿಲ್ಲ. ಸತ್ತ್ವ ಗುಣಿಯ ಅಭಿರುಚಿ ಕೂಡ ಜೀವನದಲ್ಲಿ ಶ್ರೇಷ್ಠವಾದುದು. ಅಲ್ಲಿ ಸಣ್ಣತನಕ್ಕೆ ಆಸ್ಪದವೇ ಇಲ್ಲ. ಸಣ್ಣತನ ಇವನ ಹತ್ತಿರ ಸುಳಿಯಲೇ ಆರದು. ಅವನಲ್ಲಿ ಸಣ್ಣ ಮಾತಿಲ್ಲ, ಸಣ್ಣ ಕೆಲಸವಿಲ್ಲ, ಸಣ್ಣ ರುಚಿ ಇಲ್ಲ. ಎಲ್ಲಾ ಭವ್ಯವಾದುದು. ಅವನ ಸಾನ್ನಿಧ್ಯವೇ ಒಳ್ಳೆಯದನ್ನು ಸ್ಪಂದಿಸುತ್ತಿರುವುದು, ಗುಲಾಬಿಯ ಹೂವಿನ ಹತ್ತಿರ ನಿಂತರೆ ಅದರ ಪರಿಮಳ ನಮ್ಮ ಮೂಗಿಗೆ ಹೇಗೆ ಬರುವುದೋ ಹಾಗೆ ಇಂತಹ ಸತ್ತ್ವಗುಣಿಯ ಸಾನ್ನಿಧ್ಯವೇ ಪವಿತ್ರ ವಾತಾವರಣದಿಂದ ಆಚ್ಛಾದಿತವಾಗಿರುವುದು. ಎಲ್ಲರ ಮೇಲೆಯೂ ಇದು ತನ್ನ ಪ್ರಭಾವವನ್ನು ಬೀರುವುದು.

\begin{verse}
ಲೋಭಃ ಪ್ರವೃತ್ತಿರಾರಂಭಃ ಕರ್ಮಣಾಮಶಮಃ ಸ್ಪೃಹಾ~।\\ರಜಸ್ಯೇತಾನಿ ಜಾಯಂತೇ ವಿವೃದ್ಧೇ ಭರತರ್ಷಭ \versenum{॥ ೧೨~॥}
\end{verse}

{\small ಅರ್ಜುನ, ರಜೋಗುಣ ವೃದ್ಧಿಯಾಗಿರುವಾಗ ಲೋಭ, ಪ್ರವೃತ್ತಿ ಕರ್ಮಗಳ ಆರಂಭ, ಅಶಮ, ಸ್ಪೃಹೆ ಇವು ಉಂಟಾಗುವುವು.}

ಯಾವಾಗ ರಜೋಗುಣ ಮೇಲಾಗಿರುವುದೊ ಆಗ ಈ ಪ್ರವೃತ್ತಿಗಳನ್ನು ಅವನಲ್ಲಿ ಕಾಣುತ್ತೇವೆ. ಅವನು ಲೋಭಿ. ತನ್ನಲ್ಲಿರುವುದನ್ನು ಇನ್ನೊಬ್ಬನಿಗೆ ಕೊಡ. ಅದನ್ನು ಯಾವಾಗಲೂ ವೃದ್ಧಿಮಾಡ ಬೇಕೆಂಬುದೇ ಅವನ ಗುರಿ. ಅವನು ಯಾವುದನ್ನೂ ಕಳೆದುಕೊಳ್ಳುವುದಕ್ಕೆ ಇಚ್ಛೆಪಡುವುದಿಲ್ಲ. ಅವನಿಗೆ ಗೊತ್ತಿರುವುದೊಂದೆ.... ಕೂಡುವುದು. ಕಳೆಯುವುದು ಅವನಿಗೆ ಗೊತ್ತಿಲ್ಲ. ಪ್ರಕೃತಿ ಇವನಿಂದ ಬಿಡದೆ ವಸೂಲಿಮಾಡಬೇಕೇ ಹೊರತು ಅವನಾಗಿ ಕೊಡುವವನಲ್ಲ. ಅವನು ತನ್ನಲ್ಲಿ ಇರುವುದನ್ನು ಕೊಡುವುದಿಲ್ಲ. ಆದರೆ ಇತರರಲ್ಲಿರುವುದು ಬೇಕು. ಚೆನ್ನಾಗಿ ಇರುವುದು ಹೊರಗಡೆ ಯಾರ ಯಾರ ಹತ್ತಿರ ಇರುವುದೋ ಅದರ ಮೇಲೆಲ್ಲ ಇವನಿಗೆ ಕಣ್ಣಿದೆ. ಅವಕಾಶವಾದರೆ ಅದನ್ನು ತನ್ನ ಹತ್ತಿರ ಸೇರಿಸಿಕೊಳ್ಳಲು ಸ್ವಲ್ಪವೂ ಅನುಮಾನಿಸುವುದಿಲ್ಲ.

ಅವನು ಪ್ರವೃತ್ತಿ, ಎಂದರೆ ಕರ್ಮಗಳಲ್ಲಿ ನಿರತನಾಗಿರುವನು. ಇನ್ನೂ ಹೆಚ್ಚು ಗೋಜಿನಲ್ಲಿ ನಮ್ಮನ್ನು ಸಿಕ್ಕಿಸುವುದು ಆ ಕರ್ಮಗಳು. ಕರ್ಮಮಾಡಿ ಬಂಧನದಿಂದ ಪಾರಾಗುವವನಲ್ಲ ಇವನು. ಅದರಿಂದ ಮತ್ತಷ್ಟು ಬಂಧನವನ್ನು ಮಾಡಿಕೊಳ್ಳುವನು. ಯಾವಾಗಲೂ ಲಾಭಕ್ಕೆ, ಕೀರ್ತಿಗೆ ಅಧಿಕಾರಕ್ಕೆ ಇವನು ಹಾತೊರೆಯುತ್ತಾ ಕರ್ಮಮಾಡುವನು. ಇವನು ಸುಮ್ಮನೆ ಕೆಲಸಮಾಡುವವ ನಲ್ಲ. ಇವನಿಗೆ ತತ್​ಕ್ಷಣ ಪ್ರತಿಫಲ ಬೇಕು. ಹಾಗೆ ಕರ್ಮ ಮಾಡುತ್ತಿರುವಾಗ ಕೆಲವು ವೇಳೆ ಕೈ ಸುಟ್ಟುಕೊಂಡರೆ ಅವನು ಆ ಕೆಲಸವನ್ನು ಬಿಡುತ್ತಾನೆ, ಮತ್ತೊಂದಕ್ಕೆ ಕೈಹಾಕುತ್ತಾನೆ. ಅಂತೂ ಯಾವಾಗಲೂ ಏನನ್ನಾದರೂ ಮಾಡುವುದೇ ಅವನ ಸ್ವಭಾವ. ಶಾಂತಿಯಿಂದ ಕೆಲಸಮಾಡುವುದಿಲ್ಲ. ಗಲಾಟೆ ಜಾಸ್ತಿ. ಪುರೋಹಿತರು ಮಂತ್ರ ಹೇಳುವುದು ಕಡಿಮೆಯಾದರೂ ಉಗುಳು ಜಾಸ್ತಿ ಎಂಬ ಗಾದೆಯಂತೆ ಈ ಮನುಷ್ಯ. ಗ್ರೀಸಿಲ್ಲದೆ ಎಣ್ಣೆಯಿಲ್ಲದೆ ಓಡುವ ಹಳೆಯ ಕಾರಿನಂತೆ ಇದು. ಅದ್ಭುತವಾಗಿ ಶಬ್ದಮಾಡುವುದು. ಆದರೆ ಅದರ ವೇಗವೋ ಹೇಳತೀರದು.

ಇಷ್ಟೊಂದು ಕೆಲಸಮಾಡುತ್ತಿರುವನು. ಆದರೂ ಅವನ ಮನಸ್ಸಿನಲ್ಲಿ ಏನಾದರೂ ಶಾಂತಿ ಸಮಾಧಾನ ಇದೆಯೆ?--ಎಂದು ನೋಡಿದರೆ ಅದು ಮಾತ್ರ ಇಲ್ಲ. ಯಾವಾಗಲೂ ಅಶಾಂತಿ. ಇರುವುದರ ಮೇಲೆ ತೃಪ್ತಿ ಇಲ್ಲ. ಬರುವುದರ ಮೇಲೆ ಕಣ್ಣು. ಎಷ್ಟು ಬಂದರೂ ಮತ್ತೂ ಬೇಕೆಂಬ ಆಸೆ ಇವನನ್ನು ಯಾವಾಗಲೂ ಕಾಡುತ್ತಲೇ ಇರುವುದು. ಯಾವಾಗ ಫಲಾಪೇಕ್ಷೆಯಿಂದ ಕೆಲಸಮಾಡು ವನೊ, ಆಗ ಅದು ಬರದೇ ಇದ್ದರೆ, ನಿರೀಕ್ಷಿಸಿದುದಕ್ಕಿಂತ ಕಡಿಮೆ ಬಂದರೆ, ಅಥವಾ ಫಲವನ್ನು ಇನ್ನೊಬ್ಬನೊಂದಿಗೆ ಹಂಚಿಕೊಳ್ಳಬೇಕಾಗಿ ಬಂದರೆ ವ್ಯಥೆಪಡುವನು. ಅಂತೂ ಯಾವಾಗಲೂ ಯಾವುದರಲ್ಲಿ ಆದರೂ ತಪ್ಪನ್ನು ಕಂಡುಹಿಡಿಯುತ್ತಲೇ ಇರುವನು. ಇದಕ್ಕೆಲ್ಲ ಕಾರಣ ತನ್ನಲ್ಲಿದೆ ಎಂದು ತಿಳಿಯುವುದೇ ಇಲ್ಲ. ಇಪ್ಪತ್ತೆಂಟು ಸುದ್ಧಿ ಸಮಾಚಾರಗಳು ಹೊರಗಿನಿಂದ ಇವನ ಚಿತ್ತ ಸರೋವರದ ಮೇಲೆ ಬಿದ್ದು ಅಲ್ಲಿ ಅಶಾಂತಿಯ ಅಲೆಯನ್ನು ಎಬ್ಬಿಸುತ್ತಿರುವುದು. ಕರ್ಮಯೋಗದ ಆದರ್ಶದಂತೆ ಕೆಲಸ ಮಾಡಿದಷ್ಟೂ ಚಿತ್ತಶುದ್ಧಿಯಾಗಿರಬೇಕು, ಮನುಷ್ಯ ಪ್ರಶಾಂತಚಿತ್ತದವನಾಗ ಬೇಕು. ಆದರೆ ಈತ ಯೋಗವನ್ನು ಮರೆತಿರುವನು. ಕರ್ಮಒಂದೇ ಗೊತ್ತಿರುವುದು. ಯಾವಾಗ ಯೋಗವನ್ನು ಮರೆತು ಕರ್ಮವನ್ನು ಮಾಡುತ್ತ ಹೋಗುವನೊ ಆಗ ಅದು ಬಂಧನಗಳನ್ನು ಹೆಚ್ಚಿಸುತ್ತ ಬರುವುದು.

ಅವನಲ್ಲಿ ಸ್ಪೃಹೆ ಇದೆ. ಆಸೆ ಬಿಟ್ಟಿಲ್ಲ. ಇನ್ನೂ ಹಲವಾರು ವಸ್ತುಗಳನ್ನು ಅವನು ಅನುಭವಿಸ ಬೇಕಾಗಿ ಬರುವುದು. ಎಷ್ಟೋ ಕಷ್ಟನಷ್ಟಗಳನ್ನು ಪಟ್ಟಿರುವನು. ಆದರೆ ಅದರಿಂದ ಬುದ್ಧಿ ಕಲಿತಿಲ್ಲ. ಏನೊ ಮುಂದೆ ಬರುವುದು ಚೆನ್ನಾಗಿದ್ದೀತೇನೊ ಎಂದು ಯಾವಾಗಲೂ ಬಯಕೆಗಳನ್ನು ತೀರಿಸಿಕೊಳ್ಳು ವುದರಲ್ಲೇ ನಿರತನಾಗಿರುವನು. ಈ ಸ್ಪೃಹೆ ಎಂಬ ಬೆಂಕಿಗೆ ಎಷ್ಟು ಕೊಟ್ಟರೂ ಸಾಕು ಎನ್ನುವುದಿಲ್ಲ. ಹೊಸ ಹೊಸ ಸೌದೆಯನ್ನು ಕೇಳುತ್ತಿರುವುದು. ಅದನ್ನು ಒದಗಿಸುವುದಕ್ಕಾಗಿ ಹಗಲಿರುಳೂ ನಿರತನಾಗಿ ರುವನು.

\begin{verse}
ಅಪ್ರಕಾಶೋಽಪ್ರವೃತ್ತಿಶ್ಚ ಪ್ರಮಾದೋ ಮೋಹ ಏವ ಚ~।\\ತಮಸ್ಯೇತಾನಿ ಜಾಯಂತೇ ವಿವೃದ್ಧೇ ಕುರುನಂದನ \versenum{॥ ೧೩~॥}
\end{verse}

{\small ಅರ್ಜುನ, ತಮೋಗುಣ ವೃದ್ಧಿಯಾದಾಗ ಅಪ್ರಕಾಶ, ಅಪ್ರವೃತ್ತಿ, ಪ್ರಮಾದ, ಮೋಹ ಇವು ಉಂಟಾಗುವುವು.}

ಮೂರು ಗುಣಗಳಲ್ಲಿ ತಮಸ್ಸು ತುಂಬಾ ಕೆಳಮಟ್ಟದ ಗುಣ. ಎಲ್ಲಿ ತಮಸ್ಸು ಇದೆಯೊ ಅಲ್ಲಿ ಜ್ಞಾನವಿನ್ನೂ ವಿಕಾಸವಾಗಿಲ್ಲ. ಜ್ಞಾನದ ಬೆಳಕಿದ್ದರೆ ಯುಕ್ತಾಯುಕ್ತ ಪರಿಜ್ಞಾನವಿರುವುದು. ಯಾವಾಗ ಜ್ಞಾನದ ಜ್ಯೋತಿ ಇನ್ನೂ ವಿಕಾಸವಾಗಿಲ್ಲವೋ ಅವನಲ್ಲಿ ವಿಚಾರವನ್ನು ತಿಳಿಯುವ ಶಕ್ತಿ ಇಲ್ಲ.

ಕೆಲಸವನ್ನು ಮಾಡುವ ಪ್ರವೃತ್ತಿ ಇಲ್ಲ ಅಂದರೆ ಅವನಲ್ಲಿ ನಿವೃತ್ತಿ ಏನೂ ಇಲ್ಲ. ಕಷ್ಟಪಟ್ಟು ಕೆಲಸ ಮಾಡುವುದಕ್ಕೆ ಮನಸ್ಸಿಲ್ಲ. ಮತ್ತೆ ಯಾರಾದರೂ ತೊಂದರೆ ತೆಗೆದುಕೊಂಡು ಆ ಕೆಲಸವನ್ನು ಮಾಡಿ ಬಂದ ಫಲವನ್ನು ಕೊಟ್ಟರೆ ಇವನು ಅನುಭವಿಸುವುದಕ್ಕೆ ಸಿದ್ಧನಾಗಿರುವನು. ಆದರೆ ಇವನೇ ಮೈಬಗ್ಗಿ ಕೆಲಸ ಮಾಡುವುದಿಲ್ಲ. ಮಾಡಬೇಕಾದ ಕೆಲಸವನ್ನು ಮುಂದೂಡುವನು. ಮರೆವು, ಅಸಡ್ಡೆ ಯಾವಾಗಲೂ ಇವನ ಬೆನ್ನು ಬಿಡುವುದಿಲ್ಲ. ಇವನೇನಾದರೂ ಕೆಲಸ ಮಾಡಿದರೆ ಹಾಳು ಮಾಡುವನು ಅಥವಾ ಮತ್ತೇನನ್ನೊ ಮಾಡುವನು. ಮಾಡಬೇಕಾದುದನ್ನು ಬಿಡುವನು. ಮಾಡಬಾರದುದನ್ನು ಮಾಡುವನು. ಇದೇ ಪ್ರಮಾದ.

ಯಾವಾಗಲೂ ಒಂದು ವಸ್ತುವಿನ ಮೇಲೆ ಮೋಹ ಇರುವುದು. ಅದನ್ನು ಬಿಟ್ಟು ಇರಲಾರ. ಅದನ್ನು ಕಟ್ಟಿಕೊಂಡು ಎಷ್ಟು ಅನುಭವಿಸಿದರೂ ಅದನ್ನು ತಳ್ಳಿ ಸುಮ್ಮನಿರಲಾರ. ಯಾವಾಗಲೂ ಅದನ್ನು ಕಟ್ಟಿಕೊಂಡು ಒದ್ದಾಡುವನು. ಅವನು ಹಿಂದಿನ ಅನುಭವದಿಂದ ಬುದ್ಧಿಯನ್ನು ಕಲಿತು ಕೊಳ್ಳುವುದಿಲ್ಲ.

\begin{verse}
ಯದಾ ಸತ್ತ್ವೇ ಪ್ರವೃದ್ಧೇ ತು ಪ್ರಲಯಂ ಯಾತಿ ದೇಹಭೃತ್~।\\ತದೋತ್ತಮವಿದಾಂ ಲೋಕಾನಮಲಾನ್ ಪ್ರತಿಪದ್ಯತೇ \versenum{॥ ೧೪~॥}
\end{verse}

{\small ಸತ್ತ್ವಗುಣ ತುಂಬಿರುವಾಗ ಜೀವಿಯು ಕಾಲವಾದರೆ ಅವನು ಉತ್ತಮ ಜ್ಞಾನಿಗಳ ಪವಿತ್ರವಾದ ಲೋಕವನ್ನು ಪಡೆಯುತ್ತಾನೆ.}

ಸತ್ತ್ವಗುಣ ಮೇಲಿರುವಾಗ ಸತ್ತರೆ ಉತ್ತಮ ಸ್ಥಿತಿ ಅವನಿಗೆ ಪ್ರಾಪ್ತವಾಗುವುದು. ಸಾಯುವ ಸಮಯದಲ್ಲಿ ಸತ್ತ್ವಗುಣ ಪ್ರಧಾನವಾಗಿರಬೇಕಾದರೆ ಬದುಕಿರುವಾಗ ಅವನು ಆ ಗುಣವನ್ನು ವೃದ್ಧಿ ಮಾಡಿಕೊಂಡಿರಬೇಕು. ಉತ್ತಮವಾದ ಜ್ಞಾನಿಗಳ ಪವಿತ್ರವಾದ ಲೋಕವೆಂದರೆ, ಈ ಪ್ರಪಂಚವನ್ನು ಬಿಟ್ಟು ಬೇರೆ ಯಾವುದೋ ಲೋಕದಲ್ಲಿ ಹುಟ್ಟುತ್ತಾನೆ ಎಂದು ನಾವು ತಿಳಿಯಬೇಕಾಗಿಲ್ಲ. ಇಲ್ಲಿಯೇ ಅನೇಕ ಮನೆಗಳಿವೆ. ಯೋಗ್ಯ ಸಾತ್ತ್ವಿಕ ಪುರುಷರು ಇರುವರು. ಅಂತಹ ಒಂದು ವಾತಾವರಣದಲ್ಲಿ ಹುಟ್ಟಿ ತನ್ನ ಜೀವನದ ಬೆಳವಣಿಗೆಯನ್ನು ಮುಂದುವರಿಸುವನು. ಪೂರ್ಣತೆಯನ್ನು ಪಡೆಯಬೇಕಾರೆ ಅವನಲ್ಲಿರುವ ಆಸೆಯ ಸ್ವಲ್ಪ ನೀರು ಇನ್ನೂ ಇಂಗಬೇಕಾಗಿದೆ. ಅಲ್ಲಿಯವರೆಗೆ ಇವನು ಯೋಗ್ಯರ ಸಹವಾಸದಲ್ಲಿರುವನು. ಗಿಟಕಿನಲ್ಲಿರುವ ನೀರು ಸಂಪೂರ್ಣವಾಗಿ ಇಂಗುವವರೆಗೆ ಅದು ಕರಟದಿಂದ ಬೇರೆ ಆಗಲಾರದು. ಹಾಗೆ ಆಗುವುದೇ ಮುಕ್ತನ ಸ್ಥಿತಿ. ದೇಹವೆಂಬ ಕರಟದಿಂದ ಜೀವ ಬದುಕಿರು ವಾಗಲೇ ಬೇರೆಯಾಗಿರುವುದು. ಇಂತಹ ಸ್ಥಿತಿಯನ್ನು ಪಡೆಯಬೇಕಾದರೆ ಇನ್ನೂ ಸ್ಪಲ್ಪ ಕಾಲ ಬೇಕಾಗಿದೆ ಸಾತ್ತ್ವಿಕ ಜೀವಿಗೆ.

\begin{verse}
ರಜಸಿ ಪ್ರಲಯಂ ಗತ್ವಾ ಕರ್ಮಸಂಗಿಷು ಜಾಯತೇ~।\\ತಥಾ ಪ್ರಲೀನಸ್ತಮಸಿ ಮೂಢಯೋನಿಷು ಜಾಯತೇ \versenum{॥ ೧೫~॥}
\end{verse}

{\small ರಜೋಗುಣ ಮೇಲಿರುವಾಗ ಜೀವಿ ಕಾಲವಾದರೆ ಕರ್ಮಸಂಗಿಗಳಲ್ಲಿ ಹುಟ್ಟುವನು. ತಮೋಗುಣ ಮೇಲಿರುವಾಗ ಕಾಲವಾದರೆ ಮೂಢ ಮನುಷ್ಯರ ಮಧ್ಯೆ ಜನಿಸುವನು.}

ರಜೋಗುಣಿ ಕಾಲವಾದರೆ, ಮುಂದೆ ಆ ರಜೋಗುಣದಲ್ಲಿ ಮುಂದುವರಿಯಬೇಕಾಗಿರುವುದ ರಿಂದ, ಅಂತಹ ವಾತಾವರಣದಲ್ಲಿಯೇ ಹುಟ್ಟುವನು. ಪ್ರತಿಯೊಂದು ಜೀವಿಯೂ ತನ್ನ ಸ್ವಭಾವಕ್ಕೆ ತಕ್ಕ ವಾತಾವರಣವನ್ನು ಹುಡುಕಿಕೊಂಡು ಹೋಗುವುದು. ಅವನಿಗೆ ಕರ್ಮಸಂಗಿಗಳ ಸಹವಾಸ ಬೇಕು. ಆಗಲೇ ಅವನು ಚೆನ್ನಾಗಿ ಬೆಳೆಯುವನು. ಅದರಂತೆಯೇ ತಮೋಗುಣಿ ಕೂಡಾ. ಅವನು ಕೂಡಾ ಮೂಢರ ಮಧ್ಯೆ ಹುಟ್ಟುವನು. ದೇವರು ಕೊಟ್ಟ ವಿಚಾರವನ್ನು ಅವರು ಉಪಯೋಗಿಸುವುದಿಲ್ಲ. ನಾವು ಕಾಲವಾದ ಮೇಲೆ ಯಾರೋ ನಮ್ಮನ್ನು ಬೇರೆ ಬೇರೆ ವಾತಾವರಣದಲ್ಲಿ ಬಿಡುವುದಿಲ್ಲ. ನಾವೇ ಸೂಕ್ಷ್ಮಾವಸ್ಥೆ ಯಲ್ಲಿ ಅಂತಹ ವಾತಾವರಣವನ್ನು ಹುಡುಕಿಕೊಂಡು ಹೋಗುತ್ತೇವೆ. ಬದುಕಿರುವಾಗ ನಾವು ಹೇಗೆ ನಮ್ಮಂತಿರುವ ಜನರ ಸ್ನೇಹವನ್ನು ಮಾಡಿಕೊಳ್ಳುವೆವೋ ಹಾಗೆಯೇ ಕಾಲವಾದ ಮೇಲೆಯೂ ನಮ್ಮ ಸ್ವಭಾವಕ್ಕೆ ತಕ್ಕ ವಾತಾವರಣವನ್ನು ಹುಡುಕಿಕೊಂಡು ಹೋಗಿ ಅಲ್ಲಿ ಜನಿಸುವೆವು.

\begin{verse}
ಕರ್ಮಣಃ ಸುಕೃತಸ್ಯಾಹುಃ ಸಾತ್ತ್ವಿಕಂ ನಿರ್ಮಲಂ ಫಲಮ್~।\\ರಜಸಸ್ತು ಫಲಂ ದುಃಖಮಜ್ಞಾನಂ ತಮಸಃ ಫಲಮ್ \versenum{॥ ೧೬~॥}
\end{verse}

{\small ಕರ್ಮಕ್ಕೆ ನಿರ್ಮಲವಾದ ಸಾತ್ತ್ವಿಕ ಫಲ, ರಜಸ್ಸಿಗೆ ದುಃಖದ ಫಲ; ತಮಸ್ಸಿಗೆ ಅಜ್ಞಾನದ ಫಲ ಎಂದು ಹೇಳುತ್ತಾರೆ.}

ಈ ಗುಣಗಳಲ್ಲೆಲ್ಲ ಕಾರ್ಯಕಾರಣ ಸಂಬಂಧ ಇದ್ದೇ ಇದೆ. ಯಾವಾಗ ಒಬ್ಬ ಒಳ್ಳೆಯ ಕೆಲಸವನ್ನು ಫಲಾಪೇಕ್ಷೆ ಇಲ್ಲದೆ ಮಾಡುತ್ತಾನೆಯೋ ಅದು ಅವನ ಚಿತ್ತವನ್ನು ಶುದ್ಧಿ ಮಾಡುವುದು. ಇದರಿಂದ ಅವನ ಮನಸ್ಸು ತಿಳಿಯಾಗುತ್ತ ಬರುವುದು. ವಸ್ತುವಿನ ನೈಜ ಸ್ವಭಾವವನ್ನು ಅರಿತುಕೊಳ್ಳುವ ಸ್ಥಿತಿಗೆ ಬರುವನು. ಯುಕ್ತಾಯುಕ್ತ ಪರಿಜ್ಞಾನ ಉಂಟಾಗುವುದು. ಕೆಟ್ಟದ್ದನ್ನು ಬಿಡುವನು. ಒಳ್ಳೆಯದನ್ನು ಮಾಡುವನು. ಶಾಂತಿ ಆನಂದ ಇವನೆಡೆಗೆ ಬರುವುವು. ಇದನ್ನು ಹುಡುಕಿಕೊಂಡು ಹೋಗಬೇಕಾಗಿಲ್ಲ. ತಗ್ಗಿನ ಪ್ರದೇಶಕ್ಕೆ ಹೇಗೆ ನೀರು ಎಲ್ಲಾ ಕಡೆಯಿಂದಲೂ ಹರಿದುಕೊಂಡು ಬರುವುದೊ ಹಾಗೆ.

ರಜಸ್ಸಿಗೆ ದುಃಖದ ಫಲ ಕಟ್ಟಿ ಇಟ್ಟಿದ್ದು. ಯಾವಾಗ ಫಲಾಪೇಕ್ಷೆಯಿಂದ ಕೆಲಸ ಮಾಡುವನೋ ಇವನಿಗೆ ಬೇಕಾದುದು ಸಿಕ್ಕುವುದು. ಆದರೆ ಅದರೊಡನೆ ಬೇಡದ್ದು ನೆರಳಿನಂತೆ ಬರುವುದು. ಐಶ್ವರ್ಯ ಬರುವುದು. ಆದರೆ ಅದರ ಹಿಂದೆ, ಅದನ್ನು ವೃದ್ಧಿ ಮಾಡಬೇಕು ಎನ್ನುವ ಆಸೆಯೂ ಬರುವುದು. ಯಾವ ಸಮಯದಲ್ಲಿ ಐಶ್ವರ್ಯ ಕಡಿಮೆಯಾಗಿ ಬಿಡುವುದೊ ಎಂಬ ಚಿಂತನೆ ಯಾವಾಗಲೂ ಬಾಧಿಸುತ್ತಿರುವುದು. ಹಣವನ್ನು ಹಲವು ಕಡೆ ಬಿತ್ತಿರುವನು. ಆದರೆ ಎಲ್ಲಾ ಕಡೆಯೂ ಅದು ಚೆನ್ನಾಗಿ ಬೆಳೆಯುತ್ತಿಲ್ಲ, ಒಂದೊಂದು ಕಡೆ ಅದು ಒಂದು ರೀತಿ ಇರುವುದು. ಕಾರ್ಮಿಕ ಸ್ಟ್ರೈಕುಗಳು, ಸೇಲ್ಸ್ ಟ್ಯಾಕ್ಸ್​ಗಳು ಅತಿವೃಷ್ಟಿ ಅನಾವೃಷ್ಟಿಗಳು ಅವನನ್ನು ಯಾವಾಗಲೂ ಬಾಧಿಸುತ್ತಿರುವುವು. ಹಣವಿಲ್ಲದವನಿಗೆ ಒಂದೇ ಚಿಂತೆ--ಅದು ಇಲ್ಲ ಎಂದು. ಇದ್ದವನಿಗೆ ನಾನಾ ಚಿಂತೆಗಳು. ಈ ಚಿಂತೆಗಳು ಅವನನ್ನು ಜೀವಸಹಿತ ಹೀರುವುವು. ಇವುಗಳೆಲ್ಲ ಅವನ ದೇಹದ ಮೇಲೆ ತಮ್ಮ ಪ್ರಭಾವವನ್ನು ಬೀರಿ ಹಲವಾರು ರೋಗ ರುಜಿನಗಳಿಗೆ ತುತ್ತಾಗುತ್ತಿರುವನು. ಹೃದಯದ ಖಾಯಿಲೆಗಳು, ಮಧುಮೇಹಗಳು, ವ್ರಣ\enginline{(cancer)} ಇವುಗಳೆಲ್ಲ ಬಹುಪಾಲು ಮೀಸಲು ದುಡ್ಡಿದ್ದವನಿಗೆ. ದೇಹ ಹೇಗಾದರೂ ಇರಲಿ. ಮನಸ್ಸಾದರೂ ನೆಮ್ಮದಿಯಿಂದ ಇದಯೆ? ಅದು ಮಾತ್ರ ಇಲ್ಲ. ಇದರಂತೆಯೇ ಅಧಿಕಾರ. ಹಣದ ಬಲದಿಂದ ಓಟಿಗೆ ನಿಂತು ಗೆಲ್ಲುವನು. ಅಧಿಕಾರ ಸಂಪಾದಿಸುವನು. ಒಮ್ಮೆ ಅಧಿಕಾರದ ರುಚಿ ನೋಡಿದರೆ ಅವನಿನ್ನು ಅದಿಲ್ಲದೆ ಸುಮ್ಮನೆ ಇರಲಾರ. ಅಧಿಕಾರವಿಲ್ಲದವನ ಜೀವನ ಸಪ್ಪೆಯಾಗುವುದು. ಅದೆಲ್ಲಿ ಬಿಟ್ಟು ಹೋಗುವುದೋ ಎಂಬ ಭಯ ಒಂದು ಕಡೆ. ಇತರರು ಎಲ್ಲಿ ಅದನ್ನು ಅಪಹರಿಸುತ್ತಾರೊ ಎಂಬ ಕಳವಳ ಒಂದು ಕಡೆ. ಅಧಿಕಾರ ಸಂಪಾದಿಸಿದ. ಅದರಿಂದ ಶಾಂತಿ ಬಂತೆ, ನೆಮ್ಮದಿ ಬಂತೆ? ಅದು ಮಾತ್ರ ಇಲ್ಲ. ಹಾಗೆಯೇ ಕೀರ್ತಿ. ಕೀರ್ತಿಯನ್ನು ಉರುಳಿಸುದಕ್ಕೆ ಅಪಕೀರ್ತಿ ಯಾವಾಗಲೂ ಸನ್ನಾಹ ಮಾಡುತ್ತಿರುವುದು. ರಜೋಗುಣಿಯ ಸುಖದ ಆಧಾರ ಈ ಪ್ರಪಂಚ. ಇಲ್ಲಿ ಒಂದು ಸರಿಯಾಗಿದ್ದರೆ ಇನ್ನೊಂದು ದುಃಖ ಕಾಡುತ್ತಿರುವುದು. ಇದಿಲ್ಲದವನೇ ಇಲ್ಲ.

ತಮಸ್ಸಿನ ಫಲ ಅಜ್ಞಾನ. ಯಾವಾಗ ಕೈಯಲ್ಲಿ ಬೆಳಕಿಲ್ಲವೊ, ಗಾಂಢಾಂಧಕಾರದಲ್ಲಿ ನಡೆದು ಕೊಂಡು ಹೋಗುತ್ತಿರುವೆವೊ ಆಗ ಎಡಹುವೆವು, ಮುಗ್ಗರಿಸುವೆವು. ಕಲ್ಲು ಮುಳ್ಳು ಕೊರಕಲುಗಳು ಬೇರೆ ಇವೆ ದಾರಿಯಲ್ಲಿ. ಈ ಪ್ರಪಂಚದ ಘಟನಾವಳಿಗಳ ಕೈಯಲ್ಲಿ ನೂಕಿಸಿಕೊಂಡು ಅವನು ಮುಂದೆ ಹೋಗುತ್ತಿರುವನು. ಇದೇನು, ಇದೇಕೆ ಆಗುತ್ತಿದೆ ಎಂದು ವಿಚಾರಮಾಡುವ ಗೋಜಿಗೆ ಹೋಗುವು ದಿಲ್ಲ. ಅವನು ತಳ್ಳಿದತ್ತ ಹೋಗುವನು. ಬೇಕಾದಷ್ಟು ದುಃಖಪಟ್ಟಿರುವನು. ಆದರೆ ಇದರಿಂದ ಅವನಿಗೆ ಬುದ್ಧಿ ಬರುವುದಿಲ್ಲ. ವಿಚಾರ ಜಾಗ್ರತವಾಗಿದ್ದರೆ ಇವುಗಳಿಂದ ಬುದ್ಧಿ ಕಲಿಯುತ್ತಾನೆ. ಆದರೆ ತಮೋಗುಣಿಯಲ್ಲಿ ವಿಚಾರದ ಹಣತೆ ತುಂಬಾ ಮಂಕಾಗಿ ಉರಿಯುತ್ತಿದೆ.

\begin{verse}
ಸತ್ತ್ವಾತ್ ಸಂಜಾಯತೇ ಜ್ಞಾನಂ ರಜಸೋ ಲೋಭ ಏವ ಚ~।\\ಪ್ರಮಾದಮೋಹೌ ತಮಸೋ ಭವತೋಽಜ್ಞಾನಮೇವ ಚ \versenum{॥ ೧೭~॥}
\end{verse}

{\small ಸತ್ತ್ವಗುಣದಿಂದ ಜ್ಞಾನ, ರಜೋಗುಣದಿಂದ ಲೋಭ, ತಮೋಗುಣದಿಂದ ಪ್ರಮಾದ ಮೋಹ ಅಜ್ಞಾನ ಇವು ಉಂಟಾಗುತ್ತವೆ.}

ಸತ್ತ್ವಗುಣಿಯ ಚಿತ್ತ ನಿರ್ಮಲವಾಗಿದೆ. ಅವನ ಬುದ್ಧಿ ಜಾಗ್ರತವಾಗಿದೆ. ಅವನಿಗಾದ ಅನುಭವ ಗಳಿಂದ ಅವನು ಜ್ಞಾನವನ್ನು ಪಡೆಯುತ್ತಾನೆ. ಅನುಭವ ಎಲ್ಲರಿಗೂ ಆಗುವುದು. ತಮೋಗುಣಿಗೂ ಆಗುವುದು. ಆದರೆ ಅವನು ಬುದ್ಧಿ ಕಲಿಯುವುದಿಲ್ಲ. ಸತ್ತ್ವಗುಣಿಯಾದರೊ ಆದ ಅನುಭವದಿಂದ ಬುದ್ಧಿ ಕಲಿಯುತ್ತಾನೆ. ಅದನ್ನು ತನ್ನ ಜೀವನದಲ್ಲಿ ಉಪಯೋಗಿಸಿಕೊಳ್ಳುತ್ತಾನೆ. ಈ ಸಂಸಾರದಿಂದ ಹೇಗೆ ತಪ್ಪಿಸಿಕೊಂಡು ಹೋಗಬಹುದು, ಅದಕ್ಕೆ ಏನಾದರೂ ಉಪಾಯಗಳಿವೆಯೆ ಎಂಬುದನ್ನು ಕುರಿತು ಯೋಚಿಸುವನು. ಅವನಿಗೆ ಆದ ಯಾವ ಅನುಭವವೂ ವ್ಯರ್ಥವಾಗಿರುವುದಿಲ್ಲ. ಅವನೇ ಪ್ರತ್ಯಕ್ಷ ಅನುಭವಿಸಿದ್ದು, ಜೊತೆಗೆ ತಾನು ನೋಡಿದ್ದು, ಕೇಳಿದ್ದು, ಓದಿದ್ದು ಎಲ್ಲದರಿಂದಲೂ ಅವನು ಬುದ್ಧಿಯನ್ನು ಕಲಿಯುವನು. ಈ ಪ್ರಪಂಚದಲ್ಲಿ ಪ್ರತಿಯೊಂದು ಜೀವಿಗೂ ಅವಕಾಶಗಳು ದೊರೆಯು ತ್ತವೆ ಜ್ಞಾನಾರ್ಜನೆಗೆ. ಆದರೆ ಅದನ್ನು ಉಪಯೋಗಿಸಿಕೊಳ್ಳುವವರು ಅತಿ ವಿರಳ. ಸತ್ತ್ವಗುಣಿ ಅಂತಹ ವಿರಳರ ಗುಂಪಿಗೆ ಸೇರಿದವನು.

ರಜೋಗುಣದಿಂದ ಲೋಭ ಬರುವುದು. ನಾವು ಸಂಪಾದಿಸಿದ್ದನ್ನು ಬಿಡಲು ಇಚ್ಛೆಯಿಲ್ಲ. ಅದಕ್ಕೆ ಅಂಟಿಕೊಂಡಿರುವೆವು. ಅದು ಹಣವಾಗಬಹುದು, ಅಧಿಕಾರವಾಗಬಹುದು. ಯಾವುದಾದರೂ ಒಮ್ಮೆ ಅವನ ಹತ್ತಿರಕ್ಕೆ ಬಂದರೆ ಸಾಕು ಕಪಿಮುಷ್ಟಿಯಿಂದ ಹಿಡಿದುಕೊಳ್ಳುವನು. ಅದನ್ನು ಮತ್ತೊಬ್ಬನಿಗೆ ಕೊಡುವುದಿಲ್ಲ ಮತ್ತು ಈಗ ಇರುವುದನ್ನು ವೃದ್ಧಿ ಮಾಡಿಕೊಳ್ಳಲು ಇಚ್ಛಿಸುವನು. ಆದರೆ ಇತರರೂ ಹೀಗೆಯೇ ಇರುವರು. ಕೊನೆಗೆ ಅವರವರಿಗೇ ಕಿತ್ತಾಟವಾಗುವುದು. ಇವರಿಗೆ ಬೇರೆ ಶತ್ರುಗಳಾರೂ ಇಲ್ಲ, ಇವನಂತೆ ಇರುವ ಮತ್ತೊಬ್ಬನೇ ಶತ್ರು. ಇಂತಹವರಿಗೆ ಬರಗಾಲವಿಲ್ಲ ಪ್ರಪಂಚದಲ್ಲಿ. ಲೋಭಿಗೆ ತೃಪ್ತಿ ಹೇಗೆ ಬರುವುದು? ತೃಪ್ತಿ ಇಲ್ಲದೆ ಇದ್ದರೆ ಇನ್ನು ಶಾಂತಿ ಬರುವುದು ಹೇಗೆ? ಯಾವಾಗ ಶಾಂತಿ ಇಲ್ಲವೋ, ಇನ್ನು ಏನು ಇದ್ದರೆ ತಾನೆ ಏನು? ಯಾವುದರಿಂದಲೂ ಸುಖಿ ಆಗಲಾರ. ಹಲವು ವಸ್ತುಗಳಿಗೆ ಒಡೆಯ ಎಂಬ ಹೆಸರನ್ನು ಪಡೆಯಬಹುದು. ಯಾರು ತನ್ನ ಇಂದ್ರಿಯಕ್ಕೆ, ಒಡೆಯನಲ್ಲವೊ ಅವನು ಹೊರಗಡೆ ಎಷ್ಟು ವಸ್ತುಗಳಿಗೆ ಒಡೆಯನಾದರೆ ಏನು ಬಂತು?

ತಮೋಗುಣಿ ತಪ್ಪುಮಾಡುವನು. ಮೋಹ ಅವನನ್ನು ಆವರಿಸಿಕೊಂಡಿದೆ. ವಸ್ತುವಿನ ನೈಜಸ್ಥಿತಿ ಯನ್ನು ಅವನು ಅರಿಯಲಾರ. ಅಜ್ಞಾನದ ಇರುಳಲ್ಲಿ ಅವನು ಸಂಚರಿಸುತ್ತಿರುವನು. ಕೈಯಲ್ಲಿ ವಿಚಾರದ ದೀಪವಿಲ್ಲ. ಶ್ರದ್ಧೆ ಅಥವಾ ನಂಬಿಕೆಯ ಊರುಗೋಲಿಲ್ಲ. ಮೃಗಸದೃಶ ಬಾಳುವೆ ಅವನದು. ಬುದ್ಧಿ ಇಲ್ಲದೆ ನರಳುತ್ತಿವೆ ಪಶುಗಳು. ತಮೋಗುಣಿ ತನ್ನಲ್ಲಿ ಬುದ್ಧಿಯ ಹಣತೆ ಇದ್ದರೂ ಅದನ್ನು ಹತ್ತಿಸದೆ ಇರುವನು. ಹತ್ತಿಸುವ ತೊಂದರೆಯನ್ನು ತೆಗೆದುಕೊಳ್ಳುವುದಿಲ್ಲ.

\begin{verse}
ಊರ್ಧ್ವಂ ಗಚ್ಛಂತಿ ಸತ್ತ್ವಸ್ಥಾ ಮಧ್ಯೇ ತಿಷ್ಠಂತಿ ರಾಜಸಾಃ~।\\ಜಘನ್ಯಗುಣವೃತ್ತಿಸ್ಥಾ ಅಧೋ ಗಚ್ಛಂತಿ ತಾಮಸಾಃ \versenum{॥ ೧೮~॥}
\end{verse}

{\small ಸತ್ತ್ವಗುಣದಲ್ಲಿರುವವರು ಮೇಲಕ್ಕೆ ಹೋಗುತ್ತಾರೆ. ರಜೋಗುಣಿಗಳು ಮಧ್ಯದಲ್ಲಿ ನಿಲ್ಲುವರು. ಕೊನೆಯ ಗುಣದ ತಾಮಸಿಕರು ಕೆಳಗೆ ಹೋಗುತ್ತಾರೆ.}

ಸತ್ತ್ವಗುಣ ಪ್ರಧಾನವಾಗಿರುವಾಗ ಸತ್ತರೆ ಉತ್ತಮ ಲೋಕಗಳಿಗೆ ಹೋಗುವರು. ಇದೇ ಲೋಕಕ್ಕೆ ಬಂದರೆ ಶುದ್ಧ ಸಾತ್ವಿಕ ಸ್ವಭಾವದ ಉತ್ತಮ ಕುಲದಲ್ಲಿ ಅವರು ಹುಟ್ಟುವರು. ರಜೋಗುಣಿಗಳಲ್ಲಿ ಸ್ವಲ್ಪ ಒಳ್ಳೆಯದು ಕೆಟ್ಟದ್ದು ಎರಡೂ ಬೆರತಿರುವುದು. ಆದಕಾರಣ ಅವರು ಕಾಲವಾದರೆ ಅಂತಹ ಒಂದು ಮಿಶ್ರ ವಾತಾವಾರಣದಲ್ಲೆ ಹುಟ್ಟುವರು. ತಮೋಗುಣಿಗಳು ಕಾಲವಾದರೆ, ತುಂಬ ಕೆಳ ಮಟ್ಟದ ಮಾನವರಾಗಿ ಹುಟ್ಟುವರು. ಯಾವಾಗ ನಾವು ವಿಚಾರವನ್ನು ಉಪಯೋಗಿಸುವುದಿಲ್ಲವೊ, ಅದು ಇದ್ದರೂ ಒಂದೇ ಇಲ್ಲದಿದ್ದರೂ ಒಂದೇ. ಇದ್ದು ಉಪಯೋಗಿಸದೆ ಇರುವವನು ಮನುಷ್ಯ. ಇಲ್ಲದಿದ್ದುದರಿಂದ ಉಪಯೋಗಿಸದೆ ಇರುವುವು ಪ್ರಾಣಿಗಳು.

\begin{verse}
ನಾನ್ಯಂ ಗುಣೇಭ್ಯಃ ಕರ್ತಾರಂ ಯದಾ ದ್ರಷ್ಟಾನುಪಶ್ಯತಿ~।\\ಗುಣೇಭ್ಯಶ್ಚ ಪರಂ ವೇತ್ತಿ ಮದ್ಭಾವಂ ಸೋಽಧಿಗಚ್ಛತಿ \versenum{॥ ೧೯~॥}
\end{verse}

{\small ನೋಡುವವನು ಯಾವಾಗ ಗುಣಗಳಿಗಿಂತ ಬೇರೊಬ್ಬ ಕರ್ತೃವನ್ನು ನೋಡುವುದಿಲ್ಲವೋ ಮತ್ತು ಗುಣಗಳಿಗಿಂತ ಬೇರೆಯಾದ ಏನನ್ನು ತಿಳಿಯುವನೋ, ಆಗ ಅವನು ನನ್ನ ಭಾವವನ್ನು ಹೊಂದುವನು.}

ನಿಜವಾಗಿ ನೋಡುವವನು ತೋರಿಕೆಯನ್ನು ಭೇದಿಸಿ ನೋಡುತ್ತಾನೆ. ಈ ನಾಮರೂಪದ ತೆರೆ ಸತ್ಯವನ್ನು ಅವನ ಮುಂದೆ ಬಚ್ಚಿಡಲಾರದು. ಹೇಗೆ ಎಕ್ಸ್​ರೇ ದೇಹವನ್ನು ತೂರಿ ಹಿಂದೆ ಇರುವ ಎಲುಬಿನ ಗೂಡನ್ನು ತೋರುವುದೋ ಹಾಗೆ ನಿಜವಾಗಿ ನೋಡುವವನು ನಾಮರೂಪದ ಹಿಂದೆ ಇರುವುದನ್ನು ನೋಡುತ್ತಾನೆ. ಅವನಿಗೆ ಗುಣಗಳೇ ಎಲ್ಲಾ ಕೆಲಸಗಳನ್ನು ಮಾಡುತ್ತಿರುವ ಹಾಗೆ ಕಾಣುವುದು. ಮೂರು ಗುಣ ಪ್ರಕೃತಿಯಲ್ಲಿದೆ. ನಮ್ಮ ದೇಹ ಮನಸ್ಸು ಬುದ್ಧಿ ಇಂದ್ರಿಯದಲ್ಲಿ ಮನೆ ಮಾಡಿಕೊಂಡಿರುವುದೂ ಅದೆ, ಹೊರಗೆ ವಿರಾಟ್ ಪ್ರಪಂಚದಂತೆ ಕಾಣುವುದೂ ಅದೇ. ಇಲ್ಲಿ ಇಂದ್ರಿಯಗಳಂತೆ ಇರುವುದು. ಹೊರಗೆ ವಿಷಯ ವಸ್ತುಗಳಂತೆ ಇರುವುದು. ಇಂದ್ರಿಯ ವಿಷಯ ವಸ್ತುವಿನೊಂದಿಗೆ ಸೇರಿ ಒಂದು ಕೆಲಸವನ್ನು ಮಾಡುವುದು. ಅಂದರೆ ಕೆಲಸವೆಲ್ಲ ಯಾರೊ ಹೊರಗಡೆಯವರಿಂದ ಆದ ಹಾಗಾಯಿತು. ಇವನಾದರೊ ಸಾಕ್ಷಿಯಾಗಿ ನಿಂತುಕೊಂಡು ಇರುವನು. ಯಾವಾಗ ಕೇವಲ ಸಾಕ್ಷೀಭಾವವನ್ನು ನೋಡುವನೊ, ರೂಢಿಸುವನೊ, ಅವನ ಮೂಲಕ ಆಗತಕ್ಕ ಕರ್ಮಗಳು, ಕೇವಲ ದೇಹ ವ್ಯಾಪಾರಕ್ಕೆ ಮತ್ತು ಉಳಿದ ಉತ್ತಮ ಸಂಸ್ಕಾರಗಳು ಕ್ಷಯವಾಗುವುದಕ್ಕೆ. ಅಂತಹ ಮನುಷ್ಯ ಎಂದಿಗೂ ಕೆಟ್ಟದ್ದನ್ನು ಮಾಡಲಾರನು.

ಅವನು ಗುಣಗಳಿಗಿಂತ ಬೇರೆ ಆದವನನ್ನು ತಿಳಿಯುತ್ತಾನೆ. ಹೀಗೆ ಪಂಚಭೂತಗಳು ಒಳಗೆ ದೇಹ ಇಂದ್ರಿಯ ಮನಸ್ಸು, ಬುದ್ಧಿ, ಅಹಂಕಾರ ಇವುಗಳನ್ನೆಲ್ಲ ಅತಿಕ್ರಮಿಸಿ ಹೋದ ಮೇಲೆ ಒಂದೇ ಪರಮಾತ್ಮ ಎಲ್ಲಾ ಕಡೆಯೂ ವ್ಯಾಪಿಸಿಕೊಂಡಿರುವುದು ಕಾಣುವುದು. ಅವನು ನಾನು ಎಂಬ ಬಿಲದಿಂದ ಹೊರಬರುವನು. ಆಗ ಎಲ್ಲಾ ಕಡೆಯೂ ಪರಮಾತ್ಮನೆಂಬ ಬಯಲೇ ಕಾಣುವುದು. ಒಂದು ಮಡಕೆ ತನ್ನ ಆಕಾರ ಮತ್ತು ಬಣ್ಣದಿಂದ ಒಳಗಿರುವ ಆಕಾಶವನ್ನು ಹೊರಗಿರುವ ಆಕಾಶಕ್ಕಿಂತ ಬೇರೆ ಎಂಬ ಕಲ್ಪನೆ ಬರುವಂತೆ ಮಾಡಿತ್ತು. ಯಾವಾಗ ಮಡಕೆಯನ್ನು ಒಡೆಯುತ್ತೇವೆಯೊ ಆಗ, ಮಡಕೆಯಲ್ಲಿರುವ ಆಕಾಶವೇ ಅನಂತ ಆಕಾಶವಾಗಿ ಎಲ್ಲ ಕಡೆಯೂ ವ್ಯಾಪಿಸಿಕೊಂಡಿದೆ ಎಂದು ಅರಿಯುವನು. ನಾವು ತಗುಲಿಹಾಕಿಕೊಂಡಿರುವ ಉಪಾಧಿಗಳಿಂದ ನಮ್ಮ ಹಿಂದೆ ಇರುವ ಪರಮಾತ್ಮ ನನ್ನು ಮರೆಯುತ್ತೇವೆ. ನನ್ನ ಉಪಾಧಿಗಳನ್ನೆಲ್ಲ ಕಳಚಿಬಿಟ್ಟರೆ ನನ್ನ ಹಿಂದೆ ಮಾತ್ರವಲ್ಲ, ಎಲ್ಲರ ಹಿಂದೆಯೂ ಇರುವ ಅವನು ಗೋಚರಿಸುವನು. ನನ್ನ ಉಪಾಧಿಯನ್ನು ಕಳಚಿಟ್ಟರೆ ಎಲ್ಲರ ಉಪಾಧಿಗಳನ್ನು ಕಳಚಿಟ್ಟಂತೆಯೆ. ನನ್ನ ಉಪಾಧಿಯ ಹಳ್ಳದಿಂದ ಮೇಲೆದ್ದರೆ, ಎಲ್ಲರ ಉಪಾಧಿಯ ಹಳ್ಳದಿಂದ ಮೇಲಕ್ಕೆ ಎದ್ದಂತೆಯೇ. ಆಗ ವಿಶ್ವವ್ಯಾಪಿಯಾದ ಆಕಾಶ ಎಲ್ಲೆಲ್ಲಿಯೂ ವ್ಯಾಪಿಸಿ ಕೊಂಡಿರುವುದು ಕಾಣುವುದು.

ಗುಣ ಎಂಬುದು ನಮ್ಮನ್ನು ಬೇರ್ಪಡಿಸುವ ಮಡಕೆ. ಇದು ಹೊರಗಿನದು ಬೇರೆ, ಒಳಗಿನದು ಬೇರೆ ಎಂಬ ಭ್ರಾಂತಿಯನ್ನು ಉಂಟುಮಾಡಿದೆ. ಯಾವಾಗ ಅದನ್ನು ಬಿಟ್ಟು ಮೇಲೆ ಬರುತ್ತಾನೋ, ಆಗ ಅವನು ಸರ್ವವ್ಯಾಪಿಯಾದ ಆಕಾಶದಲ್ಲಿ ಒಂದಾಗುವನು. ಇದೇ ಅವನ ಭಾವವನ್ನು ಹೊಂದು ವುದು ಎಂದು ಅರ್ಥ. ನದಿ ತನ್ನ ನಾಮರೂಪಗಳನ್ನೆಲ್ಲ ಕಳೆದುಕೊಂಡು ಸಾಗರಕ್ಕೆ ಸೇರುವುದು. ತಕ್ಷಣವೇ ಅದು ಸಾಗರವೇ ಆಗುವುದು. ಭಗವಂತನನ್ನು ಅರಿತವನು ಅವನಂತೆಯೇ ಆಗುತ್ತಾನೆ. ನಾವು ಇನ್ನು ಮೇಲೆ ಹಳೆಯ ಮನುಷೃರಂತೆ ಇರುವುದಕ್ಕೆ ಆಗುವುದಿಲ್ಲ. ನಾವು ದೇವರನ್ನು ತಿಳಿದುಕೊಂಡಿರುವುದೂ ಅದೇ ದೇಹೇಂದ್ರಿಯದ ಬಿಲದಲ್ಲಿ ಹಿಂದಿನಂತೆಯೇ ವಾಸಮಾಡಿಕೊಂಡಿರು ವುದೂ ಒಟ್ಟಿಗೆ ಹೋಗುವುದಕ್ಕೆ ಆಗುವುದಿಲ್ಲ. ಅವನು ಇದನ್ನು ಉಪಯೋಗಿಸಬಹುದು. ಹಾಗೆ ಉಪಯೋಗಿಸುವಾಗ ನಿರ್ಲಿಪ್ತನಾಗಿ ಉಪಯೋಗಿಸುವನು. ಅವನಲ್ಲಿ ಯಾವ ಹೀನ ಸಂಸ್ಕಾರಗಳೂ ಇರುವುದಿಲ್ಲ. ಆದಕಾರಣ ಅವನು ಯಾವ ಹೀನ ಕೆಲಸಗಳನ್ನೂ ಮಾಡಲಾರ. ಆಕಾಶ ಮಡಕೆಯ ಒಳಗೆ ಇದ್ದಂತೆ ಕಂಡರೂ ಹೊರಗಿರುವ ಮಹಾಕಾಶದೊಂದಿಗೆ ಸಂಬಂಧವನ್ನು ಕಲ್ಪಿಸಿಕೊಂಡು, ಹೊರಗಿನದೆ ಎಲ್ಲಾ ಮಡಕೆ ಕುಡಿಕೆಗಳಲ್ಲಿ ಸಣ್ಣದರಲ್ಲಿ ದೊಡ್ಡದರಲ್ಲಿ ಇರುವುದು ಎಂಬುದನ್ನು ಸದಾಕಾಲದಲ್ಲಿಯೂ ಅವನು ಅರಿಯುತ್ತಿರುವನು.

\begin{verse}
ಗುಣಾನೇತಾನತೀತ್ಯ ತ್ರೀನ್ ದೇಹೀ ದೇಹಸಮುದ್ಭವಾನ್~।\\ಜನ್ಮಮೃತ್ಯುಜರಾದುಃಖೈರ್ವಿಮುಕ್ತೋಽಮೃತಮಶ್ನುತೇ \versenum{॥ ೨ಂ~॥}
\end{verse}

{\small ಜೀವಿಯು ಈ ದೇಹದ ಉತ್ಪತ್ತಿ ಕಾರಣವಾಗಿರುವ ಮೂರು ಗುಣಗಳನ್ನು ದಾಟಿ, ಜನ್ಮ, ಮೃತ್ಯು ಜರಾ ಇವುಗಳಿಂದ ಪಾರಾಗಿ ಅಮೃತತ್ವವನ್ನು ಪಡೆಯುತ್ತಾನೆ.}

ಈ ದೇಹದ ಉತ್ಪತ್ತಿಗೆ ಮೂರು ಗುಣಗಳೇ ಕಾರಣ. ಈ ಗುಣಗಳೆಲ್ಲ ಜೀವಿಯನ್ನು ಒಂದು ದೇಹಕ್ಕೆ ಕಟ್ಟುತ್ತವೆ. ಅವನು ಅದರಲ್ಲಿ ಇರುವ ಇಂದ್ರಿಯದ ಮೂಲಕ ಹೊರಗಿನಿಂದ ಬರುವ ವಿಷಯಗಳನ್ನು ಅನುಭವಿಸುತ್ತಿರುವನು. ನಾವು ಮೃಗಾಲಯದಲ್ಲಿರುವ ಪ್ರಾಣಿಗಳಂತೆ ಪಂಜರದಲ್ಲಿ ಸಿಕ್ಕಿ ನರಳುವೆವು. ಈ ಪಂಜರವನ್ನು ನಾವೇ ಪ್ರವೇಶ ಮಾಡಿದ್ದು, ಬಾಗಿಲನ್ನು ನಾವೇ ಹಾಕಿಕೊಂಡಿರು ವುದು. ಅದನ್ನು ಬಿಟ್ಟು ಬರಲು ನಮಗೆ ಸಾಧ್ಯ. ಆದರೂ ಮನುಷ್ಯ ಇಂದ್ರಿಯದ ಪಂಜರ ಬಿಟ್ಟು ಹೋದರೆ ತನ್ನ ಗತಿ ಏನು ಎಂದು ಅಂಜಿ ಅಲ್ಲಿಯೇ ವಾಸವಾಗಿರುವನು. ಜನ್ಮ ಜನ್ಮಗಳಿಂದಲೂ ಯಾವಾಗ ಜ್ಞಾನದಿಂದ ಅವನಿಗೆ ತನ್ನಲ್ಲಿರುವ ವಸ್ತು ಯಾವುದು ಎಂದು ಗೊತ್ತಾಗುವುದೊ, ಅದನ್ನು ಬಿಡುವುದರಲ್ಲಿ ಇವನು ಜಯಶೀಲನಾದರೆ, ಮುಕ್ತನಾದಂತೆಯೇ. ನಮ್ಮ ವ್ಯಕ್ತಿತ್ವ ಎಂಬುದು ಮೂರು ಗುಣಗಳಿಂದ ಹಾಕಿಕೊಂಡ ಬೇಲಿ. ಇನ್ನೊಬ್ಬರು ಇಲ್ಲಿಗೆ ಬರದಿರಲಿ ಎಂದು ಹಾಕಿಕೊಂಡೆವು. ಈಗ ನಾವೇ ಗೂಟಕ್ಕೆ ಕಟ್ಟಿಹಾಕಿಕೊಂಡು ಇದರಿಂದ ಪಾರಾಗದಂತೆ ಮಾಡಿಕೊಂಡೆವು. ರೇಷ್ಮೆಯ ಹುಳು ತನ್ನ ದೇಹದಿಂದಲೇ ಉತ್ಪತ್ತಿಯಾದ ತಂತುವನ್ನು ತನ್ನ ಸುತ್ತಲೂ ನೆಯ್ದುಕೊಂಡು ಬಂಧನವನ್ನು ಸೃಷ್ಟಿಸಿಕೊಳ್ಳುವುದು. ಅದನ್ನು ಹೊರಗಡೆಯವರು ಯಾರೂ ಕೂಡಿ ಹಾಕಲಿಲ್ಲ. ಅದು ತಾನೇ ಎಲ್ಲವನ್ನೂ ಮಾಡಿಕೊಂಡಿತು. ಕೊನೆಗೆ ಅದು ಬಿಡುಗಡೆಯಾಗಬೇಕಾದರೂ ಯಾರೂ ಹೊರಗಡೆಯವರು ಅದನ್ನು ಬಿಡಿಸುವುದಿಲ್ಲ. ತಾನೇ ಗೂಡನ್ನು ಸೀಳಿ ಬರಬೇಕಾಗಿದೆ.

ಅದರಂತೆಯೇ ಜೀವಿ ಗುಣದ ಗೂಡಿನಿಂದ ಹೊರಗಡೆ ಬಂದರೆ ಮುಕ್ತ. ಈ ಗೂಡೋ ಜನನ, ಮರಣ, ಮುಪ್ಪು ಇವುಗಳಿಂದ ಕೂಡಿದೆ. ಇದು ಇದ್ದಂತೆ ಇರುವುದಿಲ್ಲ. ಯಾವಾಗಲೂ ಬದಲಾಯಿ ಸುತ್ತಿರುವುದು. ಮಗುವಿನಿಂದ ಮುದುಕನವರೆಗೆ ದಿನಕಳೆದಂತೆ ಸಾಗುತ್ತಿರುವುದು. ಈ ದೇಹ, ಇಂದ್ರಿಯ, ಮನಸ್ಸು, ಬುದ್ಧಿಗಳ ಮೂಲಕ ನಾವು ಹಲವು ಬಗೆಯ ಸುಖ ಅನುಭವಿಸಿದೆವು. ಅದಕ್ಕಿಂತ ಹೆಚ್ಚಾಗಿ ದುಃಖವನ್ನು ಅನುಭವಿಸಿದೆವು. ಕೊನೆಗೆ ಮೃತ್ಯು ಸಮನ್ ಜಾರಿ ಮಾಡುವುದು. ಈ ದೇಹ ಬಿಟ್ಟು ಹೊರಡಬೇಕು. ಒಂದೇ ಸಲ ಹೋಗುವಷ್ಟು ಪುಣ್ಯ ಸಂಪಾದಿಸಿಲ್ಲ. ನಾವು ಸಾಲಗಾರರಾಗಿ ಹೋಗುತ್ತೇವೆ. ನಮ್ಮ ಕರ್ಮ ತೀರಿಲ್ಲ. ಅದನ್ನು ತೀರಿಸುವುದಕ್ಕೆ ಪುನಃ ಪುನಃ ಬರುತ್ತಿರುವೆವು. ತೀರಿಸುವುದು ಇರಲಿ. ಹೊಸ ಹೊಸ ಸಾಲವನ್ನು ಮಾಡಿಕೊಳ್ಳುತ್ತಿರುವೆವು. ಯಾವಾಗ ಮೂರು ಗುಣಗಳು ಎಂಬ ಆಕರ್ಷಣೆಯಿಂದ ಪಾರಾಗುವನೊ, ಆಗ ಸಾಲದಿಂದ ಮುಕ್ತನಾಗುತ್ತಾನೆ. ಅವನು ಇನ್ನು ಮೇಲೆ ಪ್ರಪಂಚದಲ್ಲಿರುವಾಗಲೇ ಮುಕ್ತ. ಯಾವುದಕ್ಕೂ ಅಂಟಿಕೊಂಡಿಲ್ಲ, ಎಲ್ಲದಕ್ಕೂ ಸಾಕ್ಷಿ. ಇನ್ನು ಮೇಲೆ ಅವನು ಹುರಿಯುತ್ತಿರುವ ಬಾಂಡಲೆಯಿಂದ ಸಿಡಿದು ಬಿದ್ದ ಬೀಜದಂತೆ ಆಗುತ್ತಾನೆ. ಪ್ರಕೃತಿ ಮತ್ತೊಮ್ಮೆ ಅವನನ್ನು ಬಾಂಡಲೆಯ ಮೇಲೆ ತರುವುದಿಲ್ಲ.

ಅರ್ಜುನ ಶ‍್ರೀಕೃಷ್ಣನನ್ನು ಕೇಳುತ್ತಾನೆ:

\begin{verse}
ಕೈರ್ಲಿಂಗೈಸ್ತ್ರೀನ್ ಗುಣಾನೇತಾನತೀತೋ ಭವತಿ ಪ್ರಭೋ~।\\ಕಿಮಾಚಾರಃ ಕಥಂ ಚೈತಾಂಸ್ತ್ರೀನ್ ಗುಣಾನತಿವರ್ತತೇ \versenum{॥ ೨೧~॥}
\end{verse}

{\small ಪ್ರಭು, ಮೂರು ಗುಣಗಳನ್ನು ದಾಟಿದವನು ಯಾವ ಚಿಹ್ನೆಗಳಿಂದ ಕೂಡಿರುವನು. ಅವನ ನಡವಳಿಕೆ ಹೇಗೆ? ಅವನು ಈ ಮೂರು ಗುಣಗಳನ್ನು ಹೇಗೆ ಅತಿಕ್ರಮಿಸಿರುತ್ತಾನೆ?}

ಅರ್ಜುನ ಇಲ್ಲಿ ಗುಣಾತೀತನ ಅವಸ್ಥೆ ಎಂತಹುದು ಎಂಬುದನ್ನು ಕೇಳುತ್ತಾನೆ. ಈ ಪ್ರಪಂಚದಲ್ಲಿ ಬದ್ಧರು, ಮುಕ್ತರೂ ಇಬ್ಬರೂ ಇರುತ್ತಾರೆ. ಇಬ್ಬರೂ ಹೊರಗಡೆಯಿಂದ ನೋಡಿದರೆ ಒಂದೇ ಸಮನಾಗಿ ಕಾಣುತ್ತಾರೆ. ಅವರ ವ್ಯತ್ಯಾಸವೆಲ್ಲ ಅವರು ಈ ಪ್ರಪಂಚವನ್ನು ನೋಡುವ ದೃಷ್ಟಿಯಲ್ಲಿದೆ. ಇದು ಆಂತರಿಕ ವಿಷಯ. ಇದು ಎಲ್ಲರಿಗೂ ಗೊತ್ತಾಗುವುದಿಲ್ಲ. ಹೇಗೆ ವೈದ್ಯನಿಗೆ ಮಾತ್ರ ನಾಡೀಶಾಸ್ತ್ರ ಗೊತ್ತೊ ಹಾಗೆಯೆ ಆಧ್ಯಾತ್ಮಿಕ ಜೀವಿಯನ್ನು ತಿಳಿದುಕೊಳ್ಳಬೇಕಾದರೆ ಒಬ್ಬ ಆಧ್ಯಾತ್ಮಿಕ ಜೀವಿಗೆ ಮಾತ್ರ ಸಾಧ್ಯ. ವಜ್ರದ ವ್ಯಾಪಾರಿ ಮಾತ್ರ ವಜ್ರ ಯಾವುದು, ಗ್ಲಾಸಿನ ಚೂರು ಯಾವುದು ಎಂಬುದನ್ನು ಕಂಡುಹಿಡಿಯಬಲ್ಲ. ಆದಕಾರಣವೆ ಗುಣಾತೀತನನ್ನು ಕಂಡು ಹಿಡಿಯುವುದು ಹೇಗೆ ಎಂದು ಕೇಳುತ್ತಾನೆ.

ಅವನ ಆಚಾರ ಎಂತಹುದು ಎಂದರೆ ಅವನು ಹೊರಗಿನಿಂದ ಬರುವ ವೇದನೆಗಳಿಗೆ ಹೇಗೆ ಪ್ರತಿಕ್ರಿಯೆಯನ್ನು ತೋರುತ್ತಾನೆ ಎಂಬುದು. ಗುಣಾತೀತನೂ ಈ ಪ್ರಪಂಚದಲ್ಲಿಯೇ ಇರುವನು. ಇವನಿಗೆ ಈ ಪ್ರಪಂಚದಲ್ಲಿ ಬೇರೆ ಒಂದು ಸ್ಥಳವನ್ನು ಏರ್ಪಾಡು ಮಾಡಿಲ್ಲ. ರೂಪ ರಸ ಶಬ್ದ ಗಂಧಗಳೆಂಬುದು ಇವನ ಇಂದ್ರಿಯದ ಬಾಗಿಲಿನ ಮೂಲಕವಾಗಿಯೂ ಪ್ರವೇಶಿಸುವುವು. ಒಬ್ಬೊ ಬ್ಬರು ಒಂದೊಂದು ರೀತಿಯಲ್ಲಿಇವನನ್ನು ನೋಡುತ್ತಾರೆ, ಒಂದೊಂದು ರೀತಿ ಇವನ ವಿಷಯವಾಗಿ ಆಡಿಕೊಳ್ಳುತ್ತಾರೆ. ಆಗ ಇವನು ಅದನ್ನು ಹೇಗೆ ನೋಡುವನು? ಇವನ ಮಾತುಕತೆ ಹೇಗಿರುವುದು ಈ ಪ್ರಪಂಚದಲ್ಲಿ. ಇವುಗಳನ್ನೆಲ್ಲ ತಿಳಿಯಬಯಸುವನು ಅರ್ಜುನ.

ಅವನು ಈ ಗುಣಗಳನ್ನು ಹೇಗೆ ಅತಿಕ್ರಮಿಸುವನು? ಇದು ಅತ್ಯಂತ ಮುಖ್ಯವಾಗಿರುವುದು. ನಮ್ಮನ್ನು ಈ ಪ್ರಪಂಚಕ್ಕೆ ಕಟ್ಟಿಹಾಕಿರುವುದೇ ಈ ಗುಣಗಳು. ನಾವು ಬೇಕಾದರೆ ಈ ಗುಣಗಳ ಸುತ್ತಲೂ ಸುತ್ತುತ್ತಿರಬಹುದೇ ಹೊರತು, ಇದನ್ನು ತಪ್ಪಿಸಿಕೊಂಡು ಹೋಗುವುದಕ್ಕೆ ಗುಣಗಳು ಬಿಡುವುದಿಲ್ಲ. ಗೂಟಕ್ಕೆ ಕಟ್ಟಿದ ದನದ ಹಾಗೆ ಇರುವುದು. ಹಗ್ಗವನ್ನು ಜಗ್ಗಿಸಿ ಎಳೆದಲ್ಲದೆ ಅದು ಕಿತ್ತುಹೋಗುವುದಿಲ್ಲ. ಜಗ್ಗಿಸಿ ಎಳೆಯುವಾಗಲೂ ಅದರಲ್ಲಿ ಅಷ್ಟೊಂದು ಶಕ್ತಿ ಇರಬೇಕು, ಆಗಲೇ ಅದು ಕಿತ್ತುಹೋಗಬೇಕಾದರೆ, ಇಲ್ಲದಿದ್ದರೆ ಸಾಧ್ಯವಿಲ್ಲ. ನಾವೊಂದು ವಸ್ತುವನ್ನು, ಮೇಲಕ್ಕೆ ಎಸೆದರೆ ತಕ್ಷಣ ಕೆಳಕ್ಕೆ ಬರುವುದು. ಆದರೆ ಭೂಮಿ ಯಾವ ವೇಗದಲ್ಲಿ ಸೆಳೆಯುತ್ತಿದೆಯೊ ಅದಕ್ಕಿಂತ ಹೆಚ್ಚು ವೇಗದಲ್ಲಿ ಎಸೆದರೆ, ಅಂದರೆ ಗಂಟೆಗೆ ಸಮಾರು ೧೮ಂಂಂ ಮೈಲಿಗಳ ವೇಗದಲ್ಲಿ ಎಸೆದರೆ ಅದು ಭೂಮಿಯ ಆಕರ್ಷಣವನ್ನು ಕಿತ್ತುಕೊಂಡು ಹೋಗಿ ಬಿಡುವುದು. ಅದು ಪುನಃ ಇಲ್ಲಿಗೆ ಬರುವುದಿಲ್ಲ. ಅದೇ ಒಂದು ಸ್ವತಂತ್ರ ಗ್ರಹವಾಗಿಬಿಡುವುದು. ಇದು ಚೇಳು ಕೊಳವಿಯನ್ನು ದೀಪಾವಳಿ ಸಮಯದಲ್ಲಿ ಹಾರಿಸಿದಂತೆ. ಕೊಳವಿಯಲ್ಲಿ ಮದ್ದಿರುವುದರಿಂದ ಅದನ್ನು ಹಚ್ಚಿದೊಡನೆ ಆಕಾಶಕ್ಕೆ ಏರುವುದು. ಆದರೆ ಪುನಃ ಕೆಳಗೆ ಬರುವುದು. ಪುನಃ ಕೆಳಗೆ ಬರದಂತೆ ಮೇಲಕ್ಕೆ ಕಳುಹಿಸಬೇಕಾದರೆ ಎಂತಹ ಮದ್ದಿನಿಂದ ತುಂಬಬೇಕು ಎಂದು ಶ‍್ರೀಕೃಷ್ಣನನ್ನು ಕೇಳುತ್ತಾನೆ. ಶ‍್ರೀಕೃಷ್ಣ ಅರ್ಜುನನ ಪ್ರಶ್ನೆಗೆ ಉತ್ತರ ಕೊಡುತ್ತಾನೆ.

\begin{verse}
ಪ್ರಕಾಶಂ ಚ ಪ್ರವೃತ್ತಿಂ ಚ ಮೋಹಮೇವ ಚ ಪಾಂಡವ~।\\ನ ದ್ವೇಷ್ಟಿ ಸಂಪ್ರವೃತ್ತಾನಿ ನ ನಿವೃತ್ತಾನಿ ಕಾಂಕ್ಷತಿ \versenum{॥ ೨೨~॥}
\end{verse}

{\small ಅರ್ಜುನ, ಯಾವನು ಪ್ರಕಾಶ, ಪ್ರವೃತ್ತಿ, ಮೋಹ ಇವುಗಳು ಬಂದರೆ ದ್ವೇಷಿಸುವುದಿಲ್ಲವೊ ಹೋದರೆ ಬಯಸುವುದಿಲ್ಲವೊ--}

ಸತ್ತ್ವ, ರಜಸ್ಸು, ತಮಸ್ಸು ಮೋಡದಂತೆ ಬಂದು ಹೋಗುತ್ತಿರುವುವು. ಸೂರ್ಯ ಹೇಗೆ ಮೋಡ ಗಳಿಂದ ಆವೃತನಾಗಿರುವಂತೆ ಕಾಣುವನೋ ಹಾಗೆ. ಆದರೆ ಸೂರ್ಯ ಅವುಗಳ ಮೇಲೆ ಇರುವನು. ಅವು ಬಂದರೆ ವ್ಯಥೆ ಪಡುವುದೂ ಇಲ್ಲ, ಹೋದರೆ ದುಃಖ ಪಡುವುದೂ ಇಲ್ಲ. ಗುಣಾತೀತ ಉದಾಸೀನನಾಗಿರುತ್ತಾನೆ. ಈ ಮೂರು ಗುಣಗಳಿಗೆ ಸಂಬಂಧಪಟ್ಟ ಕ್ರಿಯೆಗಳನ್ನು ಮಾಡುತ್ತ ಇರುತ್ತಾನೆ. ಆದರೆ ಅವನು ಯಾವುದಕ್ಕೂ ಅಂಟಿಕೊಂಡಿರುವುದಿಲ್ಲ. ಅವನಿಗೆ ಸತ್ತ್ವಗುಣದ ಜ್ಞಾನ, ಶಾಂತಿ, ಆನಂದ ಮುಂತಾದುವು ಬರುವುವು. ಅವನು ಮೋಡಕ್ಕೆ ಹೇಳಿಕಳಿಸಲಿಲ್ಲ. ಅವು ಹೋಗು ವುವು. ಅವನು ಎರಡಕ್ಕೂ ಉದಾಸೀನ. ಅದರಂತೆಯೇ ಗುಣಾತೀತ ಈ ಪ್ರಪಂಚದಲ್ಲಿರುವಾಗ ಕೆಲಸ ಮಾಡದೆ ಸುಮ್ಮನೆ ತುಕ್ಕು ಹಿಡಿದುಹೋಗುವುದಿಲ್ಲ. ಅವನೇನಾದರೂ ಕೆಲಸವನ್ನು ಮಾಡುತ್ತಲೇ ಇರುವನು. ಅನೇಕ ವೇಳೆ ಇವು ಲೋಕ ಸಂಗ್ರಹ ಕೆಲಸ. ಇದನ್ನು ಮಾಡುತ್ತಿರುವಾಗಲೂ ಅವನು ಇದಕ್ಕೆ ಅಂಟಿಕೊಂಡು ಇರುವುದಿಲ್ಲ. ಯಾವ ಸಮಯದಲ್ಲಿ ಬೇಕಾದರೂ ಅವನು ಇವುಗಳನ್ನು ಬಿಡಲು ಸಿದ್ಧನಾಗಿರುವನು. ಅವನಿಗೆ ಕೆಲಸದ ಮೇಲೆ ಆಸಕ್ತಿಯಿಲ್ಲ. ಫಲದ ಮೇಲೆ ಆಕಾಂಕ್ಷೆ ಇಲ್ಲ. ಹಾಗೆಯೇ ಅವನು ನಿದ್ರಿಸುವನು. ಆದರೆ ನಿದ್ರೆಯ ಮೇಲೂ ಅವನಿಗೆ ಆಸಕ್ತಿಯಿಲ್ಲ. ಇವೆಲ್ಲ ದೇಹ ಧರ್ಮಗಳು. ಅವನು ಯಾವಾಗಲೂ ಇವುಗಳಿಗೆ ಸಾಕ್ಷಿಯೇ ಹೊರತು ಇವುಗಳಲ್ಲಿ ಬೆರೆತುಕೊಂಡಿಲ್ಲ.

ಯಾರು ಇನ್ನೂ ಗುಣಕ್ಕೆ ಬಂಧಿಗಳೊ, ಅವರು ಪ್ರಿಯವಾಗಿರುವುದು ಬಂದರೆ ಕುಣಿದಾಡುವರು, ಅಪ್ರಿಯವಾಗಿರುವುದು ಬಂದರೆ ಗೊಣಗಾಡುವರು. ಗುಣಾತೀತನಲ್ಲಿ ಔದಾಸೀನ್ಯಭಾವವನ್ನು ನೋಡುವೆವು. ಯಾವ ಮೋಡ ಬಂದು ಹೋಗುವುದೋ ಅದನ್ನು ಲೆಕ್ಕಿಸುವುದೇ ಇಲ್ಲ. ಮೋಡದ ಮೇಲೆ ಇರುವವರು ಮೋಡದಿಂದ ಬಾಧಿತರಾಗದವರು. ಮೋಡದ ಕೆಳಗೆ ಇರುವವರು, ಬಾಧಿತ ರಾಗುವರು. ಗುಣಾತೀತ ಮೇಲೆ ಇರುವವನು. ಮೋಡದ ಕೆಳಗೆ ಇರುವ ನಮಗೆ ಸೂರ್ಯ ಮೋಡದಿಂದ ಮುಚ್ಚಿಕೊಂಡು ತನ್ನ ಕಾಂತಿಯನ್ನು ಕಳೆದುಕೊಂಡಂತೆ ಇರುವನು. ಆದರೆ ಸೂರ್ಯನಿಗೆ ಹಾಗೆ ಅನ್ನಿಸುವುದಿಲ್ಲ. ಅವನು ಯಾವಾಗಲೂ ಎಂದಿನಂತೆಯೇ ಪ್ರಕಾಶಿಸುತ್ತಿರುವನು.

\begin{verse}
ಉದಾಸೀನವದಾಸೀನೋ ಗುಣೈರ್ಯೋ ನ ವಿಚಾಲ್ಯತೇ~।\\ಗುಣಾ ವರ್ತಂತ ಇತ್ಯೇವ ಯೋಽವತಿಷ್ಠತಿ ನೇಂಗತೇ \versenum{॥ ೨೬~॥}
\end{verse}

{\small ಉದಾಸೀನನಾಗಿ ಗುಣಗಳಿಂದ ಬಾಧಿತನಾಗದೆ ಗುಣಗಳು ವರ್ತಿಸುತ್ತವೆ ಎಂದು ತಿಳಿದು ಚಲಿಸದೆ ಇರುವನು.}

ಗುಣಾತೀತ ಉದಾಸೀನನಂತೆ ಇರುತ್ತಾನೆ. ಅವನು ಯಾವ ಗುಣದ ಪಕ್ಷವನ್ನೂ ವಹಿಸುವುದಿಲ್ಲ. ಅವನಿಗೆ ಯಾವ ಗುಣದ ಮೇಲೆಯೂ ಆಸಕ್ತಿ ಇಲ್ಲ. ಆಸಕ್ತಿ ಇದ್ದರೆ ತಾನೆ ಅದು ಬಂದಾಗ ಸಂತೋಷಪಡುವುದು, ಹೋದಾಗ ದುಃಖ ಪಡುವುದು? ನಾವು ಎಲ್ಲಿಯವರೆಗೆ ಈ ಪ್ರಪಂಚದಲ್ಲಿರು ವೆವೊ ಅಲ್ಲಿಯವರೆಗೆ ಈ ಗುಣಗಳೆಂಬ ಮೋಡಗಳು ಒಂದಾದ ಮೇಲೊಂದು ಬಂದುಹೋಗುತ್ತ ಇರುತ್ತವೆ. ಆದರೆ ಗುಣಾತೀತ ಇವುಗಳಾವುದಕ್ಕೂ ಅಂಟಿಕೊಂಡಿಲ್ಲ.

ಅವನು ಗುಣಗಳಿಂದ ವಿಚಲಿತನಾಗುವುದಿಲ್ಲ. ಮನಸ್ಸಿನಲ್ಲಿ ದೌರ್ಬಲ್ಯವಿದ್ದರೆ ವಿಚಲಿತನಾಗು ವನು. ಬೆಂಕಿ ಕಂಡರೆ ಬೆಣ್ಣೆ ಕರಗುವುದು. ಹಾಗೆಯೆ ಒಳಗೆ ವಾಸನೆ ಇದ್ದರೆ ಹೊರಗೆ ವಿಷಯ ವಸ್ತುಗಳು ಬಂದಾಗ ಮನಸ್ಸು ವಿಕಾರಕ್ಕೆ ಒಳಪಡುವುದು. ಆದರೆ ಆಸೆಯೆ ಇಲ್ಲದೇ ಇದ್ದರೆ ಯಾವುದು ಬಂದರೂ, ಎಷ್ಟು ಬಂದರೂ, ಏನೂ ಆಗುವುದಿಲ್ಲ. ಕನ್ನಡಿ ತನ್ನ ಮುಂದಿರುವುದನ್ನೆಲ್ಲ ಪ್ರತಿಬಿಂಬಿಸುವುದು. ಆದರೆ ಅದರಿಂದ ಇದು ಬಾಧಿತವಾಗಿ ಹೋಗುವುದಿಲ್ಲ. ಅದರ ಮೇಲೆ ಯಾವುದೂ ಅಂಟಿಕೊಳ್ಳುವುದಿಲ್ಲ. ಸಿನಿಮಾ ಪರದೆಯ ಮೇಲೆ ಏನೇನೊ ಗೊಂಬೆಗಳು ಕಾಣಿಸಿ ಕೊಳ್ಳುವುವು. ಆದರೆ ಆ ಪರದೆ ಆದರೊ ಹಿಂದಿನಂತೆಯೇ ಇರುವುದು. ಹಾಗೆಯೇ ಗುಣಾತೀತನ ಮನಸ್ಸು ಯಾವುದಕ್ಕೂ ಅಂಟಿಕೊಂಡಿರುವುದಿಲ್ಲ.

ಈ ಘಟನಾವಳಿಗಳನ್ನೆಲ್ಲಾ ಗುಣಗಳ ಲೀಲೆ ಎಂದು ಭಾವಿಸುವನು. ಸತ್ತ್ವ ರಜಸ್ಸು ತಮೋಗುಣ ಗಳು ಬಂಧನಕ್ಕೆ ಒಳಗಾದ ವ್ಯಕ್ತಿಯ ಮೂಲಕ ಹಲವಾರು ಕ್ರಿಯೆಗಳನ್ನು ಮಾಡಿಸುವುವು. ಒಳ್ಳೆಯ ದನ್ನು ಮಾಡಿಸುವುದು ಸತ್ತ್ವಗುಣ. ಒಳ್ಳೆಯದು ಮತ್ತು ಕೆಟ್ಟದ್ದನ್ನು ಮಾಡಿಸುವುದು ರಜೋಗುಣ. ಅಜ್ಞಾನ ಮರೆವುಗಳಿಂದ ಕೂಡಿದ ಕ್ರಿಯೆಯನ್ನು ಮಾಡಿಸುವುದು ತಮೋಗುಣ. ಯಾವ ಕ್ರಿಯೆಯನ್ನು ತೆಗೆದುಕೊಂಡು ಅದರ ಮೂಲಕ್ಕೆ ಹೋದರೂ, ಯಾವುದಾದರೊಂದು ಗುಣ ಅದರ ಹಿಂದಿದೆ. ಆ ಗುಣವೇ ಇಂದ್ರಿಯವನ್ನು ಪ್ರಚೋದಿಸಿ ಕೆಲಸವನ್ನು ಮಾಡಿಸುವುದು. ಎಲ್ಲಿಯವರೆಗೆ ಆ ಗುಣ ನಮ್ಮಲ್ಲಿರುವುದೊ ಅಲ್ಲಿಯವರೆಗೆ ನಾವು ಸುಮ್ಮನೆ ಕೈಕಟ್ಟಿಕೊಂಡು ಕುಳಿತಿರುವುದಕ್ಕೆ ಆಗುವುದಿಲ್ಲ. ಅವು ಅಂಕುಶದಂತೆ ತಿವಿದು ಆಯಾ ಕೆಲಸಗಳನ್ನು ನಮ್ಮ ಕೈಯಿಂದ ಮಾಡಿಸಿಯೇ ಮಾಡಿಸುವುವು. ಗುಣಾತೀತ ಇವುಗಳಿಗೆ ಸಾಕ್ಷಿಯಾಗಿರುವನು. ಇವನ ಮೂಲಕವಾಗಿಯೂ ತ್ರಿಗುಣಗಳಿಗೆ ಸಂಬಂಧ ಪಟ್ಟ ಕ್ರಿಯೆಗಳಾಗುತ್ತಿರುವುವು. ಆದರೆ ಅವನು ಮನಸ್ಸಿಟ್ಟು ಮಾಡುವುದಿಲ್ಲ. ಹಿಂದಿನ ವೇಗದಿಂದ ಅವು ಚಲಿಸುತ್ತ ಇರುತ್ತವೆ. ಹೊಸದಾಗಿ ಅವನ್ನು ನೂಕಿದ್ದಲ್ಲ. ಗಡಿಯಾರಕ್ಕೆ ಆಗಲೇ ಕೀಲು ಕೊಟ್ಟಿದೆ. ಅದು ಖರ್ಚಾಗುವವರೆಗೆ ಅದು ನಡೆದುಕೊಂಡು ಹೋಗುವುದು. ಅನಂತರ ನಿಲ್ಲುವುದು. ಅವನು ಪುನಃ ಗಡಿಯಾರಕ್ಕೆ ಕೀಲುಕೊಡಲು ಹೋಗುವುದಿಲ್ಲ.

ಗುಣಾತೀತ ಈ ಚಲನವಲನಗಳನ್ನು ನೋಡಿದಾಗ ವಿಚಲಿತನಾಗುವುದಿಲ್ಲ. ತನ್ನ ಸ್ವಾಸ್ಥ್ಯವನ್ನು ಕಳೆದುಕೊಳ್ಳುವುದಿಲ್ಲ. ಅಕ್ಕಸಾಲಿಗನ ಅಡಿಗಲ್ಲಿನಂತೆ ಅವನು. ಅದಕ್ಕೆ ಬೆಳಗಿನಿಂದ ಸಾಯಂಕಾಲದ ವರೆಗೆ ಬೇಕಾದಷ್ಟು ಸುತ್ತಿಗೆಯ ಪೆಟ್ಟು ಬೀಳುವುದು. ಅದು ಮಾತ್ರ ತಾನು ಎಂದಿನಂತೆಯೇ ಇರುವುದು. ಹಾವು ಪೊರೆಯನ್ನು ಬಿಟ್ಟಂತೆ ಗುಣಗಳ ಪೊರೆಯಿಂದ ಗುಣಾತೀತ ಕಳಚಿಕೊಂಡಿರು ವನು. ಅವನು ಪುನಃ ಅವಕ್ಕೆ ಬದ್ಧನಾಗುವುದಿಲ್ಲ.

\begin{verse}
ಸಮದುಃಖಸುಖಃ ಸ್ವಸ್ಥಃ ಸಮಲೋಷ್ವಾಶ್ಮ ಕಾಂಚನಃ~।\\ತುಲ್ಯಪ್ರಿಯಾಪ್ರಿಯೋ ಧೀರಸ್ತುಲ್ಯನಿಂದಾತ್ಮಸಂಸ್ತುತಿಃ \versenum{॥ ೨೪~॥}
\end{verse}

{\small ಸುಖದುಃಖಗಳಲ್ಲಿ ಸಮನಾಗಿರುವನು, ಸ್ವಸ್ಥನು, ಮಣ್ಣು ಕಲ್ಲು ಚಿನ್ನ ಇವುಗಳನ್ನು ಸಮನಾಗಿ ನೋಡುವನು, ಪ್ರಿಯ ಅಪ್ರಿಯಗಳಲ್ಲಿ ಸಮನಾಗಿರುವನು, ಧೀರನು ಸ್ತುತಿ ನಿಂದೆಗಳಲ್ಲಿ ಸಮನಾಗಿರುವನು.}

ಗುಣಾತೀತನಲ್ಲಿ ಸಮತ್ವಗುಣ ಚಿರಮುದ್ರಿತವಾಗಿರುವುದನ್ನು ನೋಡುತ್ತೇವೆ. ಜೀವನದಲ್ಲಿ ಯಾವಾಗಲೂ ಜೀವಿಗೆ ಸುಖ ಅಥವಾ ದುಃಖ ಒಂದಾದ ಮೇಲೆ ಒಂದು ಬರುತ್ತಿರುವುದು. ಸುಖ ಬಂದಾಗ ಹಿಗ್ಗುವುದೂ ಇಲ್ಲ. ದುಃಖ ಬಂದಾಗ ಕುಗ್ಗುವುದೂ ಇಲ್ಲ. ಏಕೆಂದರೆ ಇವೆರಡೂ ಬಾಳಿನ ಹಗಲು ರಾತ್ರಿಗಳಂತೆ. ಹಗಲು ಒಂದೇ ಇರುವುದಕ್ಕೆ ಆಗುವುದಿಲ್ಲ. ರಾತ್ರಿ ಒಂದೇ ಇರುವುದಕ್ಕೆ ಆಗುವುದಿಲ್ಲ. ಉತ್ತರ ಅಥವಾ ದಕ್ಷಿಣ ಧ್ರುವದ ಸಮೀಪದಲ್ಲಿ ಇಲ್ಲಿ ನಮಗೆ ಆಗುವಂತೆ ಅರ್ಧ ಹಗಲು ರಾತ್ರಿ ಆಗುವುದಿಲ್ಲ. ಆರು ತಿಂಗಳು ರಾತ್ರಿ, ಆರು ತಿಂಗಳು ಹಗಲು ಆಗುವುದು. ಅಂತೂ ಎಲ್ಲಿ ಆದರೂ ಇರಲಿ, ಎಲ್ಲರಿಗೂ ಈ ದ್ವಂದ್ವ ಅನುಭವ ಆಗಲೇ ಬೇಕು. ಗುಣಾತೀತನಿಗೂ ಇದು ಆಗುತ್ತದೆ. ಆದರೆ ಅವನು ಇದನ್ನು ಮನಸ್ಸಿಗೆ ತರುವುದಿಲ್ಲ. ಪ್ರಪಂಚದ ಸ್ವಭಾವವೇ ಇದು ಎಂದರಿತು ಉದಾಸೀನನಾಗಿರುತ್ತಾನೆ.

ಅವನು ಸ್ವಸ್ಥ, ತನ್ನ ನೈಜವಾದ ಸ್ಥಿತಿಯ ಮೇಲೆ ನಿಂತವನು. ಬೇಕಾದಷ್ಟು ನಾಮ ರೂಪುಗಳ ಸಂತೆಯೇ ಅವನ ಮುಂದೆ ಬರಬಹುದು. ಆದರೆ ಅವನು ಯಾವುದರ ಬಲೆಗೂ ಬೀಳುವುದಿಲ್ಲ. ಸಚ್ಚಿದಾನಂದ ಸ್ವರೂಪನಾದ ಪರಮಾತ್ಮನ ಮೂಲದಿಂದ ಬಂದಿರುವುದೇ ನಮ್ಮ ನೈಜಸ್ಥಿತಿ. ಅದು ನಮ್ಮಲ್ಲಿ ಸದಾಕಾಲದಲ್ಲಿಯೂ ಇರುವುದು. ಅಜ್ಞಾನಿ ಅದನ್ನು ಮರೆತಿದ್ದಾನೆ. ಸಾಧಕ ಅದನ್ನು ಪಡೆಯಲು ಯತ್ನಿಸುತ್ತಿರುವನು. ಗುಣಾತೀತನೇ ಸಿದ್ಧ, ಅದನ್ನು ಪಡೆದವನು. ಇನ್ನುಮೇಲೆ ಅವನು ಭ್ರಾಂತಿಯ ಕನಸಿಗೆ ಒಳಗಾಗುವುದಿಲ್ಲ.

ಅವನು ಮಣ್ಣು ಕಲ್ಲು ಚಿನ್ನ ಇವನ್ನು ಒಂದೇ ಸಮನಾಗಿ ಕಾಣುತ್ತಾನೆ. ಹಾಗಾದರೆ ಅವನಿಗೆ ಇವು ಬೇರೆಬೇರೆ ಎಂಬುದು ಮರೆತು ಹೋಗಿದೆಯೆ? ಮರೆತು ಹೋಗಿಲ್ಲ. ಪ್ರಪಂಚದ ಸಂತೆಯಲ್ಲಿ ಅದಕ್ಕೆ ಬೆಲೆ ಇದೆ. ಚಿನ್ನಕ್ಕೆ ಕೊಡುವ ಕಡೆ ಕಲ್ಲು ಕೊಟ್ಟರೆ ಬೆಲೆ ಇಲ್ಲ. ಎಲ್ಲಾ ಮಣ್ಣಿನ ಸಮ. ಇವುಗಳಾವುದರಿಂದಲೂ ಪರಮಾತ್ಮನ ಸಾಕ್ಷಾತ್ಕಾರವಾಗುವುದಿಲ್ಲ. ಶ‍್ರೀರಾಮಕೃಷ್ಣರು ಒಂದು ನಿದರ್ಶನವನ್ನು ಮನೋಜ್ಞವಾಗಿ ಬಣ್ಣಿಸುವರು. ಒಮ್ಮೆ ಗಂಡ ಹೆಂಡತಿ ಇಬ್ಬರು ದಾರಿಯಲ್ಲಿ ಹೋಗುತ್ತಿದ್ದರು. ಅವರಿಬ್ಬರೂ ಜ್ಞಾನಿಗಳು. ದಾರಿಯಲ್ಲಿ ಚಿನ್ನದ ನಾಣ್ಯವೊಂದು ಬಿದ್ದಿತ್ತು. ಗಂಡ ಭಾವಿಸಿದ, ಹೆಂಡತಿ ನೋಡಿದರೆ, ಅದನ್ನು ತೆಗೆದುಕೊಳ್ಳಬಹುದೇನೋ ಎಂದು. ಆಗ ವ್ರತ ಭಂಗವಾಗುವುದು. ನಿಂತಕಡೆ ಕಾಲಿನಿಂದ ಮಣ್ಣನ್ನು ಕೆರೆದು ಆ ಹೊನ್ನಿನ ಮೇಲೆ ತಳ್ಳಿ ಮುಚ್ಚುತ್ತಿದ್ದ. ಹೆಂಡತಿ ಗಂಡ ಏನೋ ನೆಲ ಕೆರೆಯುವುದನ್ನು ನೋಡಿದಳು. ನೀವೇನು ಮಾಡುತ್ತಿರುವುದು ಅಲ್ಲಿ ಎಂದು ಕೇಳಿದಳು. ಆತ ತಾನು ಏನು ಮಾಡುತ್ತಿರುವನೋ ಅದನ್ನು ಹೇಳಿದ. ಆಗ ಹೆಂಡತಿ, ನಿಮಗೆ ಇನ್ನೂ ಅದು ಹೊನ್ನು,ಇದು ಮಣ್ಣು ಎಂಬ ಭೇದಭಾವ ಬಿಟ್ಟು ಹೋಗಿಲ್ಲವಲ್ಲ ಎಂದು ಹೇಳಿದಳು. ಗುಣಾತೀತ ಈ ಗುಂಪಿಗೆ ಸೇರಿದವನು. ಅಷ್ಟೈಶ್ವರ್ಯಗಳೇ ಅವನ ಸುತ್ತಲೂ ಬಿದ್ದಿರಬಹುದು. ಅವನು ಅದರ ಕಡೆ ಕಣ್ಣನ್ನೂ ಹೊರಳಿಸುವುದಿಲ್ಲ.

ಅವನು ಪ್ರಿಯ ಅಪ್ರಿಯಗಳಲ್ಲಿ ಸಮನಾಗಿರುವನು. ಈ ಪ್ರಪಂಚದಲ್ಲಿ ಸುಖ ದುಃಖಗಳು ಹೇಗೆ ಪ್ರತಿಯೊಬ್ಬರಿಗೂ ಬರುವುವೋ ಹಾಗೆಯೇ ಇವನಿಗೂ ಬರುವುವು. ಆದರೆ ಇವುಗಳು ಯಾವುವೂ ಶಾಶ್ವತವಲ್ಲ. ಒಂದನ್ನು ಬಿಟ್ಟು ಮತ್ತೊಂದಿಲ್ಲ. ಒಂದೇ ನಾಣ್ಯದ ಎರಡು ಭಾಗಗಳಂತೆ ಇವು. ಪ್ರಿಯವಾಗಿರುವುದು ಬರುತ್ತದೆ ಎಂಬುದನ್ನು ಕುರಿತು ಆಲೋಚಿಸುವುದರಿಂದಲೇ ನಮ್ಮ ಮನಸ್ಸಿಗೆ ಒಂದು ಆನಂದವಾಗುವುದು. ಅದು ಬಂತು ಆದರೆ ಯಾವಾಗಲೂ ನಮ್ಮ ಹತ್ತಿರ ಇರಲಾರದು. ಕೆಲವು ಕಾಲವಾದ ಮೇಲೆ ಹೊರಟು ಹೋಗುವುದು. ಅಯ್ಯೊ ಹೊರಟು ಹೋಗುವುದಲ್ಲ ಎಂದು ವ್ಯಥೆ ಪಡುವೆವು. ಅಪ್ರಿಯ ಬರದಿರುವುದಕ್ಕೆ ಏನೇನೋ ಮಾಡುವೆವು. ಆದರೆ ಅದೋ ನಮಗಿಂತ ಬುದ್ಧಿವಂತ. ನಾವು ಚಾಪೆಯ ಕೆಳಗೆ ನುಸುಳಿದರೆ ಅದು ರಂಗೋಲಿಯ ಕೆಳಗೆ ನುಸುಳಿ ಬರುವುದು. ಅದು ಬಂದು ನಮ್ಮನ್ನು ಕಾಡುವುದು. ಆದರೆ ಅದು ಎಂದೆಂದಿಗೂ ಕಾಡುತ್ತಿರುವುದಕ್ಕೆ ಆಗುವುದಿಲ್ಲ. ಕೆಲವು ಕಾಲವಾದಮೇಲೆ ಹೊರಟು ಹೋಗುವುದು. ಒಂದು ವೇಳೆ ದೀರ್ಘಕಾಲ ಅಪ್ರಿಯ ನಮ್ಮಲ್ಲೇ ಮನೆಮಾಡಿಕೊಂಡಿತು ಎಂದರೆ, ನಾವು ಅದಕ್ಕೆ ಒಗ್ಗಿಹೋಗುವೆವು. ಅದು ಅಪ್ರಿಯ ಎಂದೇ ಕಾಣುವುದಿಲ್ಲ. ಆದಕಾರಣವೇ ಅವೆರಡೂ ಸರದಿಯಮೇಲೆ ಒಂದಾದ ಮೇಲೊಂದು ಬರುವುವು. ನಮ್ಮ ಜೀವನವನ್ನು ರೂಪಿಸುವುದಕ್ಕೆ ಇವೆರಡಕ್ಕೂ ಸಮಾನವಾದ ಪಾತ್ರವಿದೆ. ನಮಗೆ ಟೀಕೆಯೂ ಬೇಕು, ಪ್ರೋತ್ಸಾಹವೂ ಬೇಕು. ಹಾಗೆಂದು ಅರಿತು ಗುಣಾತೀತ ಸುಮ್ಮನಿರುವನು.

ಅವನು ಧೀರ, ಯಾವುದರ ಆಕರ್ಷಣೆಗೂ ಬೀಳದವನು. ಈ ಪ್ರಪಂಚದಲ್ಲಿ ಹಲವು ಬಗೆಯ ಧೀರರಿದ್ದಾರೆ. ಗೌರೀಶಂಕರ ಶಿಖರಕ್ಕೆ ಬೇಕಾದರೆ ಪ್ರಾಣದ ಹಂಗನ್ನು ತೊರೆದು ಹತ್ತುವವರಿದ್ದಾರೆ. ಗುಂಡಿನೇಟಿಗೆ ಧೈರ್ಯದಿಂದ ಎದೆಯೊಡ್ಡುವವರು ಇದ್ದಾರೆ. ಸ್ಪುಟ್ನಿಕ್​ನಲ್ಲಿ ಕುಳಿತು ಚಂದ್ರಲೋಕಕ್ಕೆ ಹೋಗಲು ಸಿದ್ಧರಾಗಿರುವವರು ಕೆಲವರು. ಇವೆಲ್ಲ ಮಕ್ಕಳಾಟ, ಗುಣಾತೀತನ ಧೈರ್ಯದೊಂದಿಗೆ ಹೋಲಿಸಿದರೆ. ಇವನು ಅರಿಷಡ್ವರ್ಗಗಳ ಸೆಳೆತವನ್ನು ಕಿತ್ತುಕೊಂಡು ಹೋಗಲು ಸಿದ್ಧನಾಗಿರುವನು. ಭೂಮಿಯ ಆಕರ್ಷಣವನ್ನು ಕಿತ್ತುಕೊಂಡು ಹೋಗುವುದಕ್ಕಿಂತ ಕಷ್ಟ ಇದು. ಈ ಪ್ರಪಂಚದಲ್ಲಿ ಯಾವುದೂ ಇನ್ನು ಮುಂದೆ ಅವನನ್ನು ಇಂದ್ರಿಯದ ಗುಹೆಯಲ್ಲಿ ಸೆರೆ ಹಿಡಿಯಲಾರದು. ಅವನು ಎಲ್ಲಾ ನಾಮರೂಪಗಳನ್ನು ಭೇದಿಸಿಕೊಂಡು ಹೋಗಿ ಹಿಂದಿರುವ ಪರಮಾತ್ಮ ವಸ್ತುವನ್ನು ಕಾಣು ತ್ತಾನೆ.

ಸ್ತುತಿ ನಿಂದೆಗಳಲ್ಲಿ ಸಮ ಅವನು. ಈ ಪ್ರಪಂಚದಲ್ಲಿ ಅಂತಹ ಗುಣಾತೀತನನ್ನು ಕೆಲವರು ಹೊಗಳುವರು. ಏಕೆಂದರೆ ಅವನ ಯೋಗ್ಯತೆ ಎಲ್ಲರಿಗೂ ಗೊತ್ತಾಗುವುದಿಲ್ಲ. ಹಾಗೆಯೇ ಕೆಲವರು ಅವನನ್ನು ತಿಳಿದುಕೊಳ್ಳುವುದಕ್ಕೆ ಆಗದವರು, ಇಂತಹ ಗುಣಾತೀತರಿಂದ ತಮ್ಮ ಸ್ವಾರ್ಥಕ್ಕೆ ಅಡ್ಡಿ ಬಂದುದರಿಂದ ಅವನನ್ನು ಜರೆಯುವರು. ಈ ಜೀವನದಲ್ಲಿ ನಾವು ಯಾವಾಗ ಒಂದಕ್ಕೆ ಕಿವಿ ಕೊಡುತ್ತೇವೆಯೊ ಆಗ ಮತ್ತೊಂದನ್ನು ಗಮನಿಸಬೇಕಾಗುವುದು. ತಿರಸ್ಕಾರ ಪುರಸ್ಕಾರಗಳೆರಡೂ ಅವನಿಗೆ ಸಮ. ಜನರು ಪುರಸ್ಕರಿಸಿದರೆ ಇವನಿಗೇನೂ ಹೊಸದಾಗಿ ಬರುವುದಿಲ್ಲ. ಆತ್ಮ ಸಾಕ್ಷಾತ್ಕಾರ ದೃಷ್ಟಿಯಿಂದ ಇವನಾಗಲೇ ಪೂರ್ಣ. ಜನರ ಹೊಗಳಿಕೆ ಇದನ್ನು ಹೆಚ್ಚು ಮಾಡಲಾರದು. ಜನರ ತೆಗಳಿಕೆ ಅವನಲ್ಲಿರುವುದನ್ನು ತೆಗೆದುಕೊಳ್ಳಲಾರದು. ಟೀಕಿಸುವವರಿಗೆ ಏನು ಗೊತ್ತು ಗುಣಾತೀತನ ಮಹಿಮೆ? ಆನೆಯೊಂದು ದಾರಿಯಲ್ಲಿ ಹೋಗುತ್ತಿದ್ದರೆ ಮನೆಮನೆಯಲ್ಲಿರುವ ನಾಯಿಗಳು ಹೊರಗೆ ಬಂದು ಬೊಗಳುತ್ತಿರುತ್ತವೆ. ಆನೆ ಇದನ್ನು ಗಮನಿಸುವುದೇ ಇಲ್ಲ. ಸುಮ್ಮನೆ ಬೊಗಳಿ ಬೊಗಳಿ ನಾಯಿಗೆ ಗಂಟಲು ನೋವು.

\begin{verse}
ಮಾನಾಪಮಾನಯೋಸ್ತುಲ್ಯಸ್ತುಲ್ಯೋ ಮಿತ್ರಾರಿಪಕ್ಷಯೋಃ~।\\ಸರ್ವಾರಂಭಪರಿತ್ಯಾಗೀ ಗುಣಾತೀತಃ ಸ ಉಚ್ಯತೇ \versenum{॥ ೨೫~॥}
\end{verse}

{\small ಯಾರು ಮಾನಾಪಮಾನಗಳಲ್ಲಿ ಸಮಬುದ್ಧಿಯುಳ್ಳವನೊ, ಶತ್ರು ಮಿತ್ರರನ್ನು ಒಂದೇ ಸಮನಾಗಿ ನೋಡುವನೊ, ಸಮಸ್ತ ಕರ್ಮಗಳನ್ನು ಬಿಟ್ಟಿರುವನೊ, ಅವನನ್ನು ಗುಣಾತೀತ ಎಂದು ಕರೆಯುತ್ತಾರೆ.}

ಗುಣಾತೀತ ಈ ಪ್ರಪಂಚದಲ್ಲಿದ್ದರೂ ಇದಕ್ಕೆ ಸೇರದವನು. ಇವನೊಬ್ಬ ಅಪರಿಚಿತನಂತೆ ಇರುವನು. ಕೆಲವರು ಇವನನ್ನು ತುಂಬಾ ಗೌರವದಿಂದ ನೋಡುವರು. ಮತ್ತೆ ಕೆಲವರು, ಅಗೌರವ ದಿಂದ ಕಾಣುವರು. ವ್ಯಾಸರ ಮಗನಾದ ಶುಕನಂತೆ ಇವನ ವ್ಯವಹಾರ. ವ್ಯಾಸರು ತಮ್ಮ ಮಗನಾದ ಶುಕನನ್ನು ಜನಕರಾಜನ ಹತ್ತಿರ ಬ್ರಹ್ಮಜ್ಞಾನವನ್ನು ಪಡೆದುಕೊಂಡು ಬಾ ಎಂದು ಕಳಿಸಿದರು. ಶುಕ ಜನಕರಾಜನ ರಾಜಧಾನಿಗೆ ಹೋದ. ಅರಮನೆಯ ಹೊರಗೆ ನಿಂತ. ಆಳು ಜನಕನಿಗೆ ಹೋಗಿ ಯಾರು ಏತಕ್ಕೆ ಬಂದಿದ್ದಾರೆ ಎಂದು ತಿಳಿಸಿದನು. ಜನಕ ಹುಡುಗನನ್ನು ಪರೀಕ್ಷಿಸುವುದಕ್ಕೆ ಕೆಲವು ದಿನ ಅನಾದರವನ್ನು ತೋರಿದ. ವ್ಯಾಸರಂತಹ ಮಹಾಜ್ಞಾನಿಗಳ ಮಗ. ಆ ಮಗನಾದರೋ ಹುಟ್ಟುವಾಗಲೇ ಮಹಾ ಜ್ಞಾನಿಯಾದವನು. ಮನೆ ಬಾಗಿಲಿನಲ್ಲೆ ಇದ್ದರೂ ಅವನನ್ನು ಸ್ವಲ್ಪವೂ ಗಮನಿಸಲಿಲ್ಲ. ಒಂದೆರಡು ದಿನಗಳಾದಮೇಲೆ ಸಾಕ್ಷಾತ್ ಜನಕನೇ ತಾಳ ಮೇಳಗಳೊಡನೆ ಶುಕನನ್ನು ರಾಜಾಸ್ಥಾನಕ್ಕೆ ಬರಮಾಡಿಕೊಂಡು ಎಲ್ಲಾ ವಿಧವಾದ ಗೌರವವನ್ನು ತೋರಿದನು. ಶುಕ ಒಂದೇ ಸಮನಾಗಿ ಇದ್ದ. ಅನಾದರ ತೋರಿಸಿದಾಗ ಕೋಪಗೊಳ್ಳಲಿಲ್ಲ. ಬೇಕಾದಷ್ಟು ಗೌರವ ತೋರಿದಾಗ ಅದಕ್ಕೆ ತಲೆ ತಿರುಗಲೂ ಇಲ್ಲ. ಇದನ್ನು ನೋಡಿ ಜನಕ ಶುಕನಿಗೆ ನಾನು ನಿನಗೆ ಏನನ್ನೂ ಹೇಳಿಕೊಡಬೇಕಾಗಿಲ್ಲ. ನೀನು ಆಗಲೇ ಬ್ರಹ್ಮಜ್ಞಾನದಲ್ಲಿ ಸ್ಥಿರನಾಗಿದ್ದೀಯೆ ಎಂದ.

ಗುಣಾತೀತ ಯಾರಿಗೂ ತೊಂದರೆ ಕೊಡುವವನಲ್ಲ. ಯಾರ ಆಸ್ತಿಯನ್ನೂ ಅಪಹರಿಸುವವನಲ್ಲ. ಅಂತಹವನಿಗೆ ಯಾರು ಶತ್ರುಗಳು ಇದ್ದಾರು ಎಂದು ನಾವು ಆಶ್ಚರ್ಯಪಡಬಹುದು. ಇವರಾಗಿ ಯಾರ ಹತ್ತಿರವೂ ಶತ್ರುತ್ವವನ್ನು ಕಟ್ಟಿಕೊಂಡಿಲ್ಲ. ಆದರೂ ಕೆಲವರು ಇವರನ್ನು ತಮ್ಮ ಶತ್ರುಗಳು ಎಂದು ಭಾವಿಸುತ್ತಾರೆ. ಶ‍್ರೀರಾಮಕೃಷ್ಣರ ಜೀವನದಲ್ಲಿ ಇಂತಹ ಒಂದು ಘಟನೆಯನ್ನು ನೋಡು ತ್ತೇವೆ. ಶ‍್ರೀರಾಮಕೃಷ್ಣರ ಸಂಪರ್ಕ ಆಗುವುದಕ್ಕೆ ಮುಂಚೆ ಮಥುರನಾಥ ಎಂಬ ಶ‍್ರೀಮಂತ ಅವನ ಕುಲ ಪುರೋಹಿತರಿಗೆ ಗೌರವ ಕೊಡುತ್ತಿದ್ದ. ಮಥುರನಾಥನಿಗೆ ಶ‍್ರೀರಾಮಕೃಷ್ಣರ ಪರಿಚಯವಾದ ಮೇಲೆ ಕುಲಪುರೋಹಿತರಿಗೆ ಕೊಡುತ್ತಿದ್ದ ಯಾವ ದಾನಧರ್ಮಗಳನ್ನೂ ನಿಲ್ಲಿಸದೇ ಇದ್ದರೂ, ಬ್ರಾಹ್ಮಣನ ಮೇಲೆ ಗೌರವ ಕಡಿಮೆಯಾಯಿತು. ಅವನು ಪ್ರತ್ಯಕ್ಷವಾಗಿ ಒಬ್ಬ ಮಹಾತ್ಮನನ್ನು ಶ‍್ರೀರಾಮಕೃಷ್ಣರಲ್ಲಿ ಕಾಣುತ್ತಿದ್ದ. ಅವನ ಗೌರವವೆಲ್ಲ ಶ‍್ರೀರಾಮಕೃಷ್ಣರ ಕಡೆಗೆ ಹೋಯಿತು. ಇದನ್ನು ಕುಲಪುರೋಹಿತನಿಗೆ ಸಹಿಸಲಾಗಲಿಲ್ಲ. ಒಂದು ದಿನ ಶ‍್ರೀರಾಮಕೃಷ್ಣರು ಒಬ್ಬರೆ ಇದ್ದಾಗ ಅವರ ಕೋಣೆಗೆ ಹೋಗಿ “ನೀನು ಯಾವ ಮಂತ್ರ ಮಾಡಿ ಮಥುರನಾಥನ ಮನಸ್ಸನ್ನು ನಿನ್ನ ಕಡೆಗೆ ಸೆಳೆದುಕೊಂಡಿರುವೆ ಹೇಳು” ಎಂದು ಕೇಳಿದ. ಶ‍್ರೀರಾಮಕೃಷ್ಣರು ತಾವು ಯಾವುದನ್ನೂ ಮಾಡಲಿಲ್ಲ ಎಂದರು. ಆದರೆ ಆತನ ಕೋಪ ಇಷ್ಟಕ್ಕೆ ತಗ್ಗಲಿಲ್ಲ. ಇವರನ್ನು ಸಿಕ್ಕಾಪಟ್ಟೆ ಬೈದು ಕಾಲಿನಿಂದ ಒದ್ದು ಹೋದ. ಇದನ್ನು ಶ‍್ರೀರಾಮಕೃಷ್ಣರ ಸೋದರಳಿಯ ಹೃದಯ ಎಂಬುವನು ನೋಡಿದ್ದ. ಶ‍್ರೀರಾಮ ಕೃಷ್ಣರು ಹೃದಯನಿಗೆ, ಈ ಸುದ್ದಿಯನ್ನು ಮಥುರನಾಥನಿಗೆ ಹೇಳಬೇಡ, ಹೇಳಿದರೆ ಆ ಬಡ ಬ್ರಾಹ್ಮಣನ ಜೀವನೋಪಾಯಕ್ಕೆ ಸೊನ್ನೆ ಆಗುವುದು ಎಂದು ಹೇಳಿದರು ಮತ್ತು ಹೃದಯನ ಕೈಯಿಂದ ತಾನು ಇದನ್ನು ಮಥುರನಿಗೆ ಹೇಳುವುದಿಲ್ಲ ಎಂಬ ಭಾಷೆಯನ್ನು ತೆಗೆದುಕೊಂಡರಂತೆ! ಶ‍್ರೀರಾಮಕೃಷ್ಣರಿಗೆ ಯಾವ ವೈರಿಗಳೂ ಇರಲಿಲ್ಲ. ಆದರೆ ಆ ಪುರೋಹಿತ ಶ‍್ರೀರಾಮಕೃಷ್ಣರನ್ನು ತನ್ನ ಬದ್ಧವೈರಿ ಎಂದು ಭಾವಿಸಿದ್ದ. ಆದರೆ ಶ‍್ರೀರಾಮಕೃಷ್ಣರಿಗಾದರೊ ಎಲ್ಲರ ಮೇಲೂ ಅನುಕಂಪ. ಶತ್ರು ಮಿತ್ರರನ್ನು ಸಮಾನ ನೋಡುವುದೆಂದರೆ ಇದೇ.

ಅವನು ಸಮಸ್ತ ಕರ್ಮಗಳನ್ನು ಬಿಟ್ಟಿರುತ್ತಾನೆ. ಅವನು ಪೂರ್ಣಾತ್ಮ. ಅವನಿಗೆ ಕರ್ಮ ಮಾಡಿದರೆ ಯಾವ ಲಾಭವೂ ಇಲ್ಲ. ಮಾಡದೆ ಇದ್ದರೆ ಯಾವ ನಷ್ಟವೂ ಇಲ್ಲ. ಆದರೂ ಇಂತಹವರು ಕೂಡಾ ಪ್ರಪಂಚದಲ್ಲಿ ಇರುವ ಪರಿಯಂತರ ಜನಗಳಿಗೆ ಬೋಧನೆ ಮಾಡುತ್ತಾರೆ, ಅವರನ್ನು ಸನ್ಮಾರ್ಗಿಗಳ ನ್ನಾಗಿ ಮಾಡಲು ಯತ್ನಿಸುತ್ತಾರೆ. ಆದರೆ ಹಾಗೆ ಮಾಡುವಾಗ ಸ್ವಲ್ಪವೂ ಆಸಕ್ತರಲ್ಲ. ಯಾವ ಸಮಯದಲ್ಲಿ ಬೇಕಾದರೆ ಅವರು ಇವುಗಳಿಂದ ಕೈತೊಳೆದುಕೊಳ್ಳಬಲ್ಲರು. ಎಳ್ಳಷ್ಟೂ ಅವರು ಕರ್ಮಕ್ಕೆ ಆಸಕ್ತರಲ್ಲ, ಕರ್ಮಫಲಕ್ಕೆ ಆಸಕ್ತರಲ್ಲ.

\begin{verse}
ಮಾಂ ಚ ಯೋಽವ್ಯಭಿಚಾರೇಣ ಭಕ್ತಿಯೋಗೇನ ಸೇವತೇ~।\\ಸ ಗುಣಾನ್ ಸಮತೀತ್ಯೈತಾನ್ ಬ್ರಹ್ಮಭೂಯಾಯ ಕಲ್ಪತೇ \versenum{॥ ೨೬~॥}
\end{verse}

{\small ಯಾರು ನನ್ನನ್ನು ಐಕಾಂತಿಕ ಭಕ್ತಿಯಿಂದ ಸೇವಿಸುತ್ತಾನೆಯೋ ಅವನು ಈ ಗುಣಗಳನ್ನು ದಾಟಿ ಬ್ರಹ್ಮಭಾವವನ್ನು ಹೊಂದಲು ಸಮರ್ಥನಾಗುತ್ತಾನೆ.}

ಗುಣಾತೀತನ ಸ್ಥಿತಿಯನ್ನು ಹೇಗೆ ಪಡೆಯುವುದು ಎಂದು ಅರ್ಜುನ ಕೇಳಿದ್ದ. ಈಗ ಅದಕ್ಕೆ ಶ‍್ರೀಕೃಷ್ಣ ಉತ್ತರವನ್ನು ಕೊಡುತ್ತಿರುವನು. ಒಂದು ಕ್ಷಿಪಣಿ ಭೂಮಿಯ ಸುತ್ತಲೂ ಸುಮಾರು ಹದಿನೆಂಟು ಸಾವಿರ ಮೈಲಿ ವೇಗದಲ್ಲಿ ಸುತ್ತುತ್ತಿರುವುದು. ಅದು ಹಾಗೆ ಸುತ್ತುವುದಕ್ಕಿಂತ ಮುಂಚೆ ಭೂಮಿಯ ಸೆಳೆತದಿಂದ ತಪ್ಪಿಸಿಕೊಂಡು ಹೋಗಬೇಕು. ಹೋಗಬೇಕಾದರೆ ಅಂತಹ ವೇಗವನ್ನು ವಿಜ್ಞಾನಿ ಆ ಯಂತ್ರದಲ್ಲಿ ಉತ್ಪತ್ತಿ ಮಾಡಬೇಕು. ಆಗ ಮಾತ್ರ ಅದು ತಪ್ಪಿಸಿಕೊಂಡು ಹೋಗಿ ಎಂದೆಂದಿಗೂ ಭೂಮಿಯ ಸುತ್ತಲೂ ಸುತ್ತುತ್ತಿರುವುದು. ಅದರಂತೆಯೇ ಜೀವಿ ಈ ಇಂದ್ರಿಯ ಪ್ರಪಂಚದ ಸೆಳೆತದಿಂದ ಮುಂಚೆ ತಪ್ಪಿಸಿಕೊಂಡು ಹೋಗುವುದಕ್ಕೆ ತನ್ನಲ್ಲಿ ಮದ್ದನ್ನು ತುಂಬಬೇಕಾಗಿದೆ. ಆ ಮದ್ದೇ ಭಗವಂತನ ಮೇಲೆ ಐಕಾಂತಿಕ ಭಕ್ತಿ. ಈಗ ನಾವು ಹಲವಾರು ಉದ್ದೇಶಗಳನ್ನು ಇಟ್ಟು ಕೊಂಡು ಭಗವಂತನನ್ನು ಪ್ರೀತಿಸುತ್ತಿದ್ದೇವೆ. ನಮಗೆ ಅವನು ಬೇಕಾಗಿಲ್ಲ. ಅವನ ಉಗ್ರಾಣದಲ್ಲಿರುವ ಹಲವು ಅದ್ಭುತ ವಸ್ತುಗಳು ಬೇಕು. ಅವನ ಹತ್ತಿರ ಹೋಗಿ ಏನಾದರೂ ಲೌಕಿಕ ವಸ್ತುಗಳನ್ನು ಯಾಚಿಸುತ್ತೇವೆ. ಈ ಜೀವನದಲ್ಲಿ ಏನಾದರೂ ಎಡವಟ್ಟಾದರೆ ಅದನ್ನು ರಿಪೇರಿ ಮಾಡಿಕೊಡು ಎಂದು ಕೇಳುತ್ತೇವೆ. ಇಲ್ಲದ ಲೌಕಿಕ ವಸ್ತುಗಳು ಸಿಕ್ಕಬೇಕು. ಇರುವುದು ಇನ್ನೂ ವೃದ್ಧಿಯಾಗಬೇಕು. ಅದಕ್ಕಾಗಿ ನಮಗೆ ದೇವರು ಬೇಕು. ಆದರೆ ಐಕಾಂತಿಕ ಭಕ್ತಿ ಎಂದರೆ,ಅವನು ಕೇವಲ ಭಗವಂತನನ್ನು ಮಾತ್ರ ಪ್ರೀತಿಸುತ್ತಾನೆ. ಅವನಿಗೆ ಬೇರೆ ಇನ್ನು ಯಾವುದೂ ಬೇಕಾಗಿಲ್ಲ. ಅವನು ನನಗೆ ಮುಕ್ತಿಯೂ ಬೇಡ, ನಿನ್ನನ್ನು ಪ್ರೀತಿಸುತ್ತಿದ್ದರೆ ಸಾಕು ಎಂದು ಕೇಳಿಕೊಳ್ಳುತ್ತಾನೆ. ಅವನಿಗೆ ಅಷ್ಟೈಶ್ವರ್ಯಗಳೂ ಬೇಡ, ಕೀರ್ತಿ ಬೇಡ, ಗೌರವ ಬೇಡ, ಕವಿತ್ವ, ಪಾಂಡಿತ್ಯ ಯಾವುದೂ ಬೇಡ. ನದಿ ಹೇಗೆ ಸಾಗರಕ್ಕೆ ಹಗಲು ರಾತ್ರಿ ಹರಿದುಕೊಂಡು ಹೋಗಿ ಅದರಲ್ಲಿ ಒಂದಾಗುತ್ತದೆಯೋ ಹಾಗೆ ಭಕ್ತ ಭಗವಂತನ ಕಡೆಗೆ ಹರಿದುಕೊಂಡು ಹೋಗಬಯಸುವನು. ಇದೇ ಐಕಾಂತಿಕ ಭಕ್ತಿ. ಇದಿದ್ದರೆ ಮನುಷ್ಯ ತನ್ನನ್ನು ಈ ಪ್ರಪಂಚಕ್ಕೆ ಬಿಗಿದ ತ್ರಿಗುಣಗಳನ್ನೆಲ್ಲ ಕಿತ್ತುಕೊಂಡು ಹೋಗಿ ಸದಾ ಭಗವಂತನ ಸುತ್ತಲೂ ಪ್ರದಕ್ಷಿಣೆ ಮಾಡುತ್ತಿರುವನು. ಅವನಿಗೆ ಇದೇ ಆನಂದ, ಇದನ್ನು ಮೀರಿದ ಆನಂದವಿಲ್ಲ.

\begin{verse}
ಬ್ರಹ್ಮಣೋ ಹಿ ಪ್ರತಿಷ್ಠಾಹಮಮೃತಸ್ಯಾವ್ಯಯಸ್ಯ ಚ~।\\ಶಾಶ್ವತಸ್ಯ ಚ ಧರ್ಮಸ್ಯ ಸುಖಸ್ಯೈಕಾಂತಿಕಸ್ಯ ಚ \versenum{॥ ೨೭~॥}
\end{verse}

{\small ಏಕೆಂದರೆ ನಾನು ಅವ್ಯಯವಾಗಿಯೂ ಅಮೃತವಾಗಿಯೂ ಇರುವ ಬ್ರಹ್ಮಕ್ಕೂ ಶಾಶ್ವತವಾದ ಧರ್ಮಕ್ಕೂ ಐಕಾಂತಿಕವಾದ ಸುಖಕ್ಕೂ ಆಧಾರವಾಗಿದ್ದೇನೆ.}

ಇಲ್ಲಿ ಶ‍್ರೀಕೃಷ್ಣ ಪರಮಾತ್ಮನೊಂದಿಗೆ ತಾದಾತ್ಮ್ಯಭಾವವನ್ನು ಪಡೆದು ‘ನಾನು’ ಎಂದು ಉಪ ಯೋಗಿಸುತ್ತಾನೆ. ಅವನೇ ಅವ್ಯಯವಾಗಿರುವ, ಅಮೃತವಾಗಿರುವ ಬ್ರಹ್ಮಕ್ಕೆ ಆಧಾರ. ಅವ್ಯಯ ಎಂದರೆ ಬದಲಾವಣೆ ಇಲ್ಲದವನು. ಈ ಪ್ರಪಂಚದಲ್ಲಿ ಭಗವಂತನನ್ನು ಬಿಟ್ಟು ಉಳಿದುದೆಲ್ಲಾ ಬದಲಾವಣೆಯ ಚಕ್ರಕ್ಕೆ ಸಿಕ್ಕಿರುತ್ತದೆ. ಅವನು ಚಕ್ರದ ಮಧ್ಯದಲ್ಲಿ ಇರುವ ಗೂಟದಂತೆ. ಚಕ್ರವೆಲ್ಲ ಉರುಳುತ್ತಿದೆ. ಆದರೆ ಮಧ್ಯದಲ್ಲಿರುವ ಗೂಟವಾದರೊ ಸ್ಥಿರವಾಗಿದೆ. ಈ ನಾಮರೂಪದ ಚಕ್ರ ಸುತ್ತುತ್ತಿದೆ ಬಿಡುವಿಲ್ಲದೆ. ಆದರೆ ಹಿಂದೆ ಯಾವ ಬದಲಾವಣೆಗೂ ನಿಲುಕದೆ, ಆದರೆ ಬದಲಾವಣೆ ಗಳಿಗೆಲ್ಲಾ ಆಧಾರವಾಗಿರುವವನು ಆ ಪರಮಾತ್ಮನೊಬ್ಬನೆ.

ಅವನು ಅಮೃತ ಸ್ವರೂಪ. ಯಾವಾಗ ಜೀವಿ ಅವನನ್ನು ರುಚಿ ನೋಡುವನೊ, ಅವನು ಅಮೃತನಾಗುತ್ತಾನೆ. ಇನ್ನು ಮೇಲೆ ಜನನ ಮರಣದ ಚಕ್ರಕ್ಕೆ ಸಿಕ್ಕುವುದಿಲ್ಲ. ಅವನು ತ್ರಿಕಾಲಕ್ಕೂ ನಿತ್ಯ ಸಾಕ್ಷಿ. ಅವನು ಶಾಶ್ವತವಾದ ಧರ್ಮಕ್ಕೆ ಮೂಲಾಧಾರ. ಈ ಜೀವನದಲ್ಲಿ ಯಾವ ಧರ್ಮಗಳನ್ನು ತೆಗೆದುಕೊಂಡರೂ ಅದರಲ್ಲಿ ಎರಡು ಭಾಗಗಳನ್ನು ನೋಡುತ್ತೇವೆ. ಒಂದು ಹೊರಗಿನ ಕವಚ ಧರ್ಮದಿಂದ ಧರ್ಮಕ್ಕೆ ವ್ಯತ್ಯಾಸವಾಗುತ್ತಿರುತ್ತದೆ. ಒಳಗಿರುವ ತಿರುಳಾದರೂ ಎಲ್ಲದರಲ್ಲಿಯೂ ಒಂದೇ.

ಸಗುಣವನ್ನು ಒಪ್ಪುವ ಧರ್ಮಗಳು ಅದನ್ನು ಒಂದು ವ್ಯಕ್ತಿ ಎಂದು ಕರೆಯುತ್ತಾರೆ. ನಿರ್ಗುಣ ವನ್ನು ಮಾತ್ರ ಒಪ್ಪುವವರು, ಅದನ್ನು ಒಂದು ಸ್ಥಿತಿ ಎನ್ನುತ್ತಾರೆ. ವ್ಯಕ್ತಿ ಮತ್ತುಸ್ಥಿತಿ ಎರಡೂ ಒಂದೇ ನಾಣ್ಯದ ಎರಡು ಮುಖಗಳು. ಒಂದು ಮೇಲು ಮತ್ತೊಂದು ಕೀಳು ಎನ್ನುವ ಗೋಜಿಗೆ ಕೈಹಾಕ ಬೇಕಾಗಿಲ್ಲ. ನಮಗೆ ಯಾವುದು ಹಿಡಿಸುವುದೋ ಅದನ್ನು ತೆಗೆದುಕೊಳ್ಳೋಣ. ಯಾವಾಗ ನಮಗೆ ಅದರಲ್ಲಿ ಯಾವುದಾದರೂ ಒಂದು ಸಿಕ್ಕಿದರೂ, ಮತ್ತೊಂದರ ಅನುಭವ ಆಗುವುದು. ಸಗುಣ, ನಿರ್ಗುಣ ಎನ್ನುವುದು ನಾಣ್ಯದ ಎರಡು ಕಡೆಯಂತೆ ಎಂದು ಶ‍್ರೀರಾಮಕೃಷ್ಣರು ಹೇಳುತ್ತಿದ್ದರು.

ಇವನೇ ಐಕಾಂತಿಕ ಸುಖಕ್ಕೆ ಆಧಾರ, ಈ ಪ್ರಪಂಚದಲ್ಲಿ ಪರಮಾತ್ಮನನ್ನು ರುಚಿ ನೋಡಿದ ಮನುಷ್ಯನಿಗೆ, ಇಂದ್ರಿಯ ಸುಖ ಕಾಕಪಿಷ್ಟದಂತೆ ಕಾಣುವುದು. ಅವನ ಮೂಲಕ ಬರುವ ಪರಮಾ ನಂದವನ್ನು ಅನುಭವಿಸಿದವನಿಗೆ, ಇಂದ್ರಿಯದ ಮೂಲಕ ಸಿಕ್ಕುವುದು ತಂಗಳಿನ ಕೂಳಿನಂತೆ. ಅವನಿನ್ನು ಮೇಲೆ ಅದಕ್ಕೆ ಕೈಯೊಡ್ಡನು. ಅವನು ಎಂದೆಂದಿಗೂ ತೃಪ್ತನಾಗಿ ಹೋಗುತ್ತಾನೆ. ಅವನಿಗೆ ಈ ಪ್ರಪಂಚದಿಂದ ಇನ್ನು ಏನೂ ಬೇಕಾಗಿಲ್ಲ.

