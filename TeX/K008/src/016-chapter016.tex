
\chapter{ದೈವಾಸುರ ಸಂಪದ್ವಿಭಾಗಯೋಗ}

ಶ‍್ರೀಕೃಷ್ಣ ಅರ್ಜುನನಿಗೆ ಹೇಳುತ್ತಾನೆ:

\begin{verse}
ಅಭಯಂ ಸತ್ತ್ವಸಂಶುದ್ಧಿರ್ಜ್ಞಾನಯೋಗವ್ಯವಸ್ಥಿತಿಃ~।\\ದಾನಂ ದಮಶ್ಚ ಯಜ್ಞಶ್ಚ ಸ್ವಾಧ್ಯಾಯಸ್ತಪ ಆರ್ಜವಮ್ \versenum{॥ ೧~॥}
\end{verse}

\begin{verse}
ಅಹಿಂಸಾ ಸತ್ಯಮಕ್ರೋಧಸ್ತ್ಯಾಗಃ ಶಾಂತಿರಪೈಶುನಮ್~।\\ದಯಾ ಭೂತೇಶ್ವಲೋಲುಪ್ತ್ವಂ ಮಾರ್ದವಂ ಹ್ರೀರಚಾಪಲಮ್ \versenum{॥ ೨~॥}
\end{verse}

\begin{verse}
ತೇಜಃ ಕ್ಷಮಾ ಧೃತಿಃ ಶೌಚಮದ್ರೋಹೋ ನಾತಿಮಾನಿತಾ~।\\ಭವಂತಿ ಸಂಪದಂ ದೈವೀಮಭಿಜಾತಸ್ಯ ಭಾರತ \versenum{॥ ೩~॥}
\end{verse}

{\small ಅರ್ಜುನ, ನಿರ್ಭಯತೆ, ಸತ್ತ್ವಸಂಶುದ್ಧಿ, ಜ್ಞಾನಯೋಗವ್ಯವಸ್ಥಿತಿ, ದಾನ, ಯಜ್ಞ, ವೇದಾಧ್ಯಯನ, ತಪಸ್ಸು, ಸರಳತೆ, ಅಹಿಂಸೆ, ಸತ್ಯ, ಅಕ್ರೋಧ, ತ್ಯಾಗ, ಶಾಂತಿ, ಚಾಡಿಯನ್ನು ಹೇಳದಿರುವುದು, ಭೂತದಯೆ, ಇಂದ್ರಿಯ ನಿಗ್ರಹ, ಮೃದುಸ್ವಭಾವ, ನಾಚಿಕೆ, ಚಪಲವಿಲ್ಲದೆ ಇರುವುದು, ತೇಜಸ್ಸು, ಕ್ಷಮೆ, ಧೈರ್ಯ, ಶೌಚ, ದ್ರೋಹ ಮಾಡದೆ ಇರುವುದು, ಅಭಿಮಾನವಿಲ್ಲದೆ ಇರುವುದು–ಇವು ದೈವೀಸಂಪತ್ತಿನ ಗುಣಗಳು.}

ಶ‍್ರೀಕೃಷ್ಣ ಈ ಅಧ್ಯಾಯದಲ್ಲಿ ದೈವೀ ಗುಣಗಳು ಮತ್ತು ಆಸುರೀ ಗುಣಗಳು ಇವುಗಳನ್ನು ಹೇಳುತ್ತಾನೆ. ದೈವೀಗುಣ ನಮ್ಮನ್ನು ಸಂಸಾರದಿಂದ ಬಿಡುಗಡೆ ಮಾಡುವುದು. ಆಸುರೀ ಗುಣ ನಮ್ಮನ್ನು ಸಂಸಾರದಲ್ಲಿ ಕಟ್ಟಿಹಾಕುವುದು. ಒಬ್ಬ ಯೋಧ ಯುದ್ಧಕ್ಕೆ ಹೋಗುವುದಕ್ಕೆ ಮುಂಚೆ ತರಬೇತನ್ನು ತೆಗೆದುಕೊಂಡು ಹೋಗುತ್ತಾನೆ. ಎಂತೆಂತಹ ಶಸ್ತ್ರಾಸ್ತ್ರಗಳನ್ನು ಯುದ್ಧದಲ್ಲಿ ಉಪ ಯೋಗಿಸಬೇಕು, ಮತ್ತೆ ಅವುಗಳಿಂದ ಹೇಗೆ ಪಾರಾಗಬೇಕು ಎಂಬುದನ್ನು ಚೆನ್ನಾಗಿ ತಿಳಿದುಕೊಂಡಿರ ಬೇಕು. ಇಲ್ಲದೇ ಇದ್ದರೆ ಅವನು ಪ್ರಥಮ ಬುಲೆಟ್ಟಿಗೇ ಆಹುತಿಯಾಗುತ್ತಾನೆ. ಅದರಂತೆಯೇ ಭಗವಂತನೆಡೆಗೆ ಹೊರಡುವುದು ಒಂದು ದೊಡ್ಡ ಶತ್ರುಸೇನೆಯ ಮೂಲಕ ಹೋಗುವಂತೆ. ನಾವು ನಮ್ಮನ್ನು ಪುನಃ ಸಂಸಾರಕ್ಕೆ ಕಟ್ಟಿಹಾಕಲು ಬರುವ ಶತ್ರುಗಳಿಂದೆಲ್ಲ ಪಾರಾಗಬೇಕು.

ಸತ್ತ್ವಗುಣಗಳಲ್ಲಿ ಮೊದಲು ಹೇಳುವುದೇ ನಿರ್ಭಯತೆ. ಆಧ್ಯಾತ್ಮಿಕ ಜೀವನ ಹೇಡಿಗಲ್ಲ. ಅದು ವೀರಾಧಿವೀರನಿಗೆ. ಅವನು ಯಾವುದಕ್ಕೂ ಅಂಜದವನಾಗಿರಬೇಕು. ಯಾರು ಈ ಸಂಸಾರಕ್ಕೆ ಅಂಟಿಕೊಂಡಿರುವರೋ ಅವರು ಹೇಡಿಗಳು. ಎಲ್ಲಿ ತಮಗೆ ಪ್ರಿಯವಾಗಿರುವುದು ಬಿಟ್ಟುಹೋಗು ವುದೋ ಎಂಬ ಅಂಜಿಕೆ ಒಂದು ಕಡೆ, ಅಪ್ರಿಯವಾಗಿರುವುದೆಲ್ಲಿ ಬರುವುದೊ ಎಂಬ ಅಂಜಿಕೆ ಒಂದು ಕಡೆ ಬಾಧಿಸುತ್ತಿರುವುದು ಅವರನ್ನು. ಆದರೆ ದೇವರ ಕಡೆ ಹೊರಟಿರುವವನು, ಪ್ರಪಂಚಕ್ಕೆ ವಿರೋಧ ವಾಗಿ ಹೊರಟಿರುವನು. ಇಲ್ಲಿ ಏನು ಬಂದರೂ ಅವನು ಅಂಜುವುದಿಲ್ಲ. ವೀರ ಸಾಯುವುದೊಂದೇ ಬಾರಿ. ಹೇಡಿಯಾದರೋ ಅಂಜಿದಾಗಲೆಲ್ಲ ಸಾಯುತ್ತಿರುವನು. ಈ ಪ್ರಪಂಚದ ಕೀರ್ತಿ ಐಶ್ವರ್ಯ ಗಳೇ ಹೇಡಿಗೆ ಸಿಕ್ಕಲಾರವು. ಇನ್ನು ದೈವೀ ಸಂಪತ್ತಿಗೆ ಸೇರಿದ ಗುಣಗಳು ಅವನಿಗೆ ಹೇಗೆ ಸಿಕ್ಕುವುವು?\\ಸಂಸಾರದ ಪಂಜರವನ್ನು ತನ್ನ ನಖಗಳಿಂದ ಕಿತ್ತುಕೊಂಡು ಬರುವ ಕೇಸರಿಯಂತೆ ನಿರ್ಭೀತನು, ಭಗವಂತನೆಡೆಗೆ ಹೊರಟಿರುವವನು.

ಅವನಲ್ಲಿ ಶುದ್ಧವಾಗಿರುವ ಸತ್ತ್ವಗುಣವಿದೆ. ಸತ್ತ್ವಗುಣ ಇದ್ದಕ್ಕೆ ಇದ್ದಂತೆಯೇ ನಮಗೆ ಬರಲಾರದು. ನಮ್ಮ ಆಹಾರ ಶುದ್ಧವಾದರೆ ಸತ್ತ್ವಗುಣ ವೃದ್ಧಿಯಾಗಬೇಕಾದರೆ. ಆಹಾರ ಸ್ಥೂಲ; ಅದರಿಂದ ಉತ್ಪತ್ತಿಯಾದ ಮಾನಸಿಕ ಶಕ್ತಿ ಸೂಕ್ಷ್ಮ. ಒಂದು ಶುದ್ಧವಾಗಿದ್ದರೆ ಮಾತ್ರ ಮತ್ತೊಂದನ್ನು ಶುದ್ಧಮಾಡಿಕೊಳ್ಳಬೇಕಾದರೆ. ನಾವು ಏನನ್ನು ತೆಗೆದುಕೊಳ್ಳುತ್ತೇವೆಯೊ ಅದು ಧಾರ್ಮಿಕವಾಗಿ ಸಂಪಾದನೆ ಮಾಡಿದ್ದಾಗಿರಬೇಕು. ಅನ್ಯಾಯ ಅಧರ್ಮ ಸುಳ್ಳು ತಟವಟ ಮುಂತಾದುವುಗಳಿಂದ ಸಂಪಾದಿಸಿ ಅದನ್ನು ಎಷ್ಟೇ ಮಡಿಯಿಂದ ಮಾಡಿದರೂ ಅದೇನೂ ಒಳ್ಳೆಯ ಆಹಾರವಾಗುವುದಿಲ್ಲ. ಆಹಾರ ಎಂದರೆ ನಾವು ಬರೀ ಬಾಯಿ ಮೂಲಕ ತಿನ್ನುವುದಕ್ಕೆ ಮಾತ್ರ ಅನ್ವಯಿಸುವುದಿಲ್ಲ. ಕಣ್ಣು ಕಿವಿ, ಮೂಗು ಇವುಗಳ ಮೂಲಕ ನಾವು ಪಡೆಯುವ ವೇದನೆಗಳು ಕೂಡ ಪವಿತ್ರವಾಗಿರಬೇಕು. ಇವುಗಳೆಲ್ಲ ಸೇರಿ ಆಹಾರವಾಗುವುದು. ಯಾವಾಗ ಒಳ್ಳೆಯ ಆಹಾರವಿರುವುದೋ ಆಗ ಅದರಿಂದ ಉಂಟಾಗುವ ಆಲೋಚನೆಗಳು ಕೂಡ ಪವಿತ್ರವಾಗಿರುವುವು. ಈ ಆಲೋಚನೆಗಳು ನಮ್ಮನ್ನು ಸಂಸಾರದಿಂದ ಪಾರುಮಾಡವ ಆಲೋಚನೆಗಳಾಗುವುವು.

ಜ್ಞಾನಯೋಗವ್ಯವಸ್ಥಿತಿ: ಶಾಸ್ತ್ರಾದಿಗಳನ್ನು ನಾವು ಓದಬೇಕು, ಯಾರು ಈ ಜೀವನದಲ್ಲಿ ಮುಂದುವರೆದಿರುವರೋ ಅವರ ಅನುಭವವನ್ನು ತೆಗೆದುಕೊಳ್ಳಬೇಕು. ಮತ್ತು ನಾವೇ ಅದರ ಆಧಾರದಮೇಲೆ ಚೆನ್ನಾಗಿ ವಿಚಾರ ಮಾಡಿ ಸತ್ಯವನ್ನು ತಿಳಿದುಕೊಳ್ಳಬೇಕು. ಜೀವನದಲ್ಲಿ ಬರೀ ತಿಳಿದುಕೊಂಡರೆ ಸಾಲದು. ತಿಳಿದುಕೊಂಡದ್ದನ್ನು ಅನುಷ್ಠಾನಕ್ಕೆ ತರಬೇಕು. ಆಗಲೇ ನಮಗೆ ಅವು ಸಹಾಯಕ್ಕೆ ಬರಬೇಕಾದರೆ. ಸುಮ್ಮನೆ ಭಾಂಗ್ ಭಾಂಗ್ ಎಂದರೆ ಮತ್ತು ಬರುವುದಿಲ್ಲ. ಅದನ್ನು ತಂದು ಅರೆದು ಕುಡಿಯಬೇಕು. ಆಗಲೆ ಮತ್ತು ಬರಬೇಕಾದರೆ ಎನ್ನುತ್ತಿದ್ದರು ಶ‍್ರೀರಾಮಕೃಷ್ಣರು. ಅದರಂತೆಯೆ ಯಾವ ಜ್ಞಾನವನ್ನು ನಾವು ಸಂಗ್ರಹಿಸಿಟ್ಟಿರುವೆವೊ ಜೀವನದಲ್ಲಿ ಅದನ್ನು ಅನುಷ್ಠಾನಕ್ಕೆ ತಂದಾಗಲೇ ಅದು ಯೋಗವಾಗಬೇಕಾದರೆ.

ಬರೀ ಬುದ್ಧಿಗೆ ಸಂಬಂಧಪಟ್ಟ ಗುಣಗಳು ಮಾತ್ರ ಇದ್ದರೆ ಸಾಲದು. ದಾನದ ಸ್ವಭಾವ ನಮ್ಮದಾಗಬೇಕು. ದೇವರು ಯಾವ ಒಳ್ಳೆಯದನ್ನು ನಮಗೆ ಕೊಟ್ಟಿರುವನೋ ಅದನ್ನು ಇತರರೊಂದಿಗೆ ಹಂಚಿಕೊಳ್ಳಬೇಕು. ಇಲ್ಲಿ ದಾನ ಎಂದರೆ ಬರೀ ಹಣವನ್ನು ಮಾತ್ರ ಇನ್ನೊಬ್ಬನಿಗೆ ಕೊಡುವುದಲ್ಲ. ದಾನಗಳಲ್ಲಿ ಹಲವು ಬಗೆಗಳಿವೆ. ದೇವರು ಕೊಟ್ಟರುವ ವಿದ್ಯೆಯನ್ನು ಹೇಳಿಕೊಡಬಹುದು, ಬಟ್ಟೆ ಕೊಡಬಹುದು, ಔಷಧಿ ಕೊಡಬಹುದು, ಊಟ ಕೊಡಬಹುದು. ಏನೂ ಕೊಡುವುದಕ್ಕೆ ಇಲ್ಲದಿದ್ದರೂ ಒಳ್ಳೆಯ ಮಾತನ್ನಾದರೂ ಮತ್ತೊಬ್ಬನಿಗೆ ಆಡಬಹುದು. ದಮ--ಇಂದ್ರಿಯಗಳ ಒಂದು ಸ್ವಭಾವವೆ ಅದಕ್ಕೆ ಸಂಬಂಧಪಟ್ಟ ವಿಷಯ ವಸ್ತುಗಳ ಕಡೆ ಹರಿದು ಹೋಗುವುದು. ಹೋಗುವುದಕ್ಕೆ ಮುಂಚೆ ಅದನ್ನು ತಡೆಗಟ್ಟುವನು. ಇದನ್ನೇ ದಮ ಎನ್ನುವುದು. ಹೋದಮೇಲೆ ಪಶ್ಚಾತ್ತಾಪ ಪಡುವುದಿಲ್ಲ. ಹೋಗುವುದಕ್ಕೆ ಮುಂಚೆಯೇ, ಹೋದರೆ ಏನು ಆಗುವುದು ಎಂಬುದನ್ನು ಮೊದಲೇ ಅರಿತು ಅದನ್ನು ನಿಗ್ರಹಿಸುವನು.

ಯಜ್ಞ, ಪೂಜೆ, ಪ್ರಾರ್ಥನೆ, ಹಲವು ವಿಧದ ವ್ರತನಿಯಮಗಳು ಇವುಗಳೆವೂ ಇದರಲ್ಲಿ ಸೇರಿವೆ. ಇವುಗಳನ್ನೆಲ್ಲ ಶ್ರದ್ಧೆಯಿಂದ ಮಾಡಿದರೆ ನಮ್ಮ ಚಿತ್ತವನ್ನು ಶುದ್ಧಿಮಾಡುವುದು. ಚಿತ್ತಶುದ್ಧವಿದ್ದರೆ, ಚಿತ್ತದ ಏಕಾಗ್ರತೆ ಸುಲಭವಾಗುವುದು.

ವೇದಾಧ್ಯಯನ: ಆಧ್ಯಾತ್ಮಿಕ ಮತ್ತು ತಾತ್ತ್ವಿಕ ವಿಷಯಗಳಿಗೆ ಸಂಬಂಧಪಟ್ಟ ವಿಷಯಗಳನ್ನು ಓದಬೇಕು. ಇದೇ ನಮ್ಮ ಜೀವನಕ್ಕೆ ಆಹಾರವನ್ನು ಒದಗಿಸುವುವು. ಅಧ್ಯಯನ ಎಂಬ ಪದ ಬಹಳ ಅರ್ಥಪೂರಿತವಾದುದು. ಇದು ಬರೀ ಓದುವುದಲ್ಲ. ಈ ಪದದಲ್ಲಿ ಓದುವುದು, ಆಲೋಚಿಸುವುದು ಇವುಗಳೆಲ್ಲ ಸೇರಿವೆ. ನಾವು ಮುಂಚೆ ಓದಬೇಕು. ನಮಗಿಂತ ಮುಂದಿನವರು ಹೋದ ದಾರಿಯನ್ನು ಇದು ತೋರುತ್ತದೆ. ಅವರು ಕಂಡ ಅನುಭವಗಳನ್ನು ಇದು ವಿವರಿಸುವುದು. ಇದನ್ನು ಓದಿ ಆದಮೇಲೆ ಚೆನ್ನಾಗಿ ಮನನ ಮಾಡಬೇಕು. ಹಲವಾರು ದೃಷ್ಟಿಕೋಣಗಳಿಂದ ಅದನ್ನು ತೂಗಬೇಕು, ಅಳೆಯಬೇಕು, ವಿರ್ಮರ್ಶಿಸಬೇಕು. ಆಗಲೇ ನಾವು ಅದನ್ನು ಚೆನ್ನಾಗಿ ಜೀರ್ಣಿಸಿಕೊಳ್ಳುತ್ತ ಬರುತ್ತೇವೆ.

ತಪಸ್ಸು: ಭಗವಂತನ ಮೇಲೆ ಮನಸ್ಸನ್ನು ಏಕಾಗ್ರತೆ ಮಾಡಬೇಕು. ಈ ಏಕಾಗ್ರತೆಗೇ ತಪಸ್ಸು ಎಂದು ಹೆಸರು. ಮನಸ್ಸಿನ ಏಕಾಗ್ರತೆಗೆ ಅದ್ಭುತವಾದ ಶಕ್ತಿ ಇದೆ. ಯಾವಾಗ ನಮ್ಮ ಮನಸ್ಸು ಭಗವಂತನಲ್ಲಿ ಏಕಾಗ್ರವಾಗುವುದೊ ಆಗ ಅವನೊಡನೆ ನಾವೊಂದು ಸಂಬಂಧ ಬೆಳೆಸುವೆವು. ಅವನಿಂದ ಎಷ್ಟು ಬೇಕಾದರೂ ನಾವು ಶಕ್ತಿಯನ್ನು ಹೀರಬಹುದು. ಇದು ಮನೆಯ ಮುಂದೆ ಹೋಗುತ್ತಿರುವ ವಿದ್ಯುತ್ ದೀಪದೊಂದಿಗೆ ಸಂಬಂಧವನ್ನು ಬೆಳೆಸಿಕೊಂಡಂತೆ. ಆಗ ಅದರ ಮೂಲಕ ನಾವು ಶಕ್ತಿಯನ್ನು ಪಡೆಯಲು ಸಾಧ್ಯವಾಗುವುದು.

ದೈವೀ ಸಂಪತ್ತಿನ ಮನುಷ್ಯನಲ್ಲಿ ನಾವು ಸರಳತೆಯನ್ನು ನೋಡುತ್ತೇವೆ. ಒಂದು ಸಣ್ಣ ಮಗುವಿನಂತೆ ಅವನ ಸ್ವಭಾವ. ಒಳಗೊಂದು ಹೊರಗೊಂದು ಇಲ್ಲ. ಅವನು ಯಾವುದನ್ನೂ ಬಚ್ಚಿಡುವುದಿಲ್ಲ. ಇಲ್ಲಿ ಯಾವ ತೋರಿಕೆಯೂ ಇಲ್ಲ. ಕಪಟವೂ ಇಲ್ಲ. ಥಳುಕಿಲ್ಲ. ಒಳಗೆ ಹೊರಗೆ ಎಲ್ಲಾ ಒಂದೇ. ಭಗವಂತ ಪ್ರತಿಬಿಂಬಿಸಬೇಕಾದರೆ ಇಂತಹ ಸರಳ ಹೃದಯ ಇರಬೇಕಾದರೆ ಮನಸ್ಸು ಶುದ್ಧವಾಗಿರಬೇಕು. ಅಲ್ಲಿ ಯಾವ ಆಸೆ ಆಕಾಂಕ್ಷೆಗಳ ದಾಂಧಲೆಯೂ ಇರಬಾರದು. ಜೀವನದಲ್ಲಿ ಶ್ರೇಷ್ಠ ಕಲೆ ಸರಳತೆ. ಅವನ ಮಾತು ಸರಳ, ಅವನ ಆಚರಣೆ ಸರಳ. ಅವನೇನಾದರೂ ಬರೆದರೆ ಅದೂ ಸರಳವಾಗಿಯೇ ಇರುವುದು. ಅದ್ಭುತ ಭಾವನೆಗಳನ್ನು ನಿರಾಡಂಬರವಾಗಿ ಹೊರಸೂಸುವನು. ಅದನ್ನು ಓದಿದೊಡನೆಯೇ ಹಿಂದಿರುವುದು ಕಾಣುವುದು. ಅಲ್ಲಿ ಯಾವ ಕ್ಲೇಶವೂ ಇಲ್ಲ. ಅವನ ಹೃದಯದಲ್ಲಿ ಭಾವನೆ ಮಿಂಚಿದಂತೆ ಹೊರಗೆ ಕಾಣುವುದು.

ಅವನು ಅಹಿಂಸಾವ್ರತಿ. ಯಾರಿಗೂ ಹಿಂಸೆಯನ್ನು ಕೊಡುವುದಿಲ್ಲ. ಇನ್ನೊಬ್ಬನನ್ನು ನೋಯಿಸಿ ಅವನು ಆನಂದಪಡುವವರ ಗುಂಪಿಗೆ ಸೇರಿಲ್ಲ. ಯಾವ ನೋವನ್ನಾದರೂ ತಾನು ಅನುಭವಿಸವನೆ ಹೊರತು ಇತರರಿಗೆ ಕೊಡಲು ಇಚ್ಛಿಸುವುದಿಲ್ಲ. ಅವನು ಶಸ್ತ್ರಗಳನ್ನು ಉಪಯೋಗಿಸಿ ಹಿಂಸೆ ಕೊಡುವುದಿಲ್ಲ. ಅಷ್ಟೇ ಅಲ್ಲ. ಒಂದು ಒರಟು ಮಾತಿನಿಂದಲೂ ಇನ್ನೊಬ್ಬನನ್ನು ನೋಯಿಸಲು ಇಚ್ಛಿಸುವುದಿಲ್ಲ.

ಅವನು ಯಾವಾಗಲೂ ಸತ್ಯವನ್ನೇ ಹೇಳುತ್ತಾನೆ. ಅನೇಕ ವೇಳೆ ನಾವು ಸುಳ್ಳನ್ನು ಹೇಳುವುದಕ್ಕೆ ಕಾರಣ ಒಂದು ಕೃತ್ಯದ ಪರಿಣಾಮದಿಂದ ತಪ್ಪಿಸಿಕೊಳ್ಳುವುದಕ್ಕಾಗಿ. ಆದರೆ ಸತ್ಯವನ್ನು ಯಾವಾಗಲೂ ಬಚ್ಚಿಡುವುದಕ್ಕೆ ಆಗುವುದಿಲ್ಲ. ಬಚ್ಚಿಟ್ಟರೆ ಒಳಗಿರುವುದು ಕೊಳೆತು ನಾರುವುದು. ಇನ್ನಾರೊ ನಮ್ಮನ್ನು ಕಂಡುಹಿಡಿಯಬೇಕಾಗಿಲ್ಲ. ನಾವೇ ಅಸ್ವಸ್ಥರಾಗುವೆವು. ನಮ್ಮ ಮನಸ್ಸಿನಲ್ಲಿಯೇ ಒಂದು ಕಳವಳ. ಯಾವಾಗ ಯಾರಿಂದ ನಾವು ಮಾಡಿರುವುದು ಬಯಲಾಗುವುದೋ ಎಂಬ ಭಯ ಯಾವಾಗಲೂ ನಮ್ಮನ್ನು ಕಾಡುತ್ತಿರುವುದು. ಇದರಿಂದ ನಮ್ಮ ಜೀವನಕ್ಕೆ ಆಗುವಷ್ಟು ಹಾನಿ ಇನ್ನಾವುದರಿಂದಲೂ ಆಗುವುದಿಲ್ಲ. ಸತ್ಯವನ್ನು ಹೇಳಿಬಿಡುವುದು ಯಾವಾಗಲೂ ಒಳ್ಳೆಯದು. ಅಪ್ರಿಯವಾಗಿರುವುದು ಮೊದಲು ಆಗಿಹೋಗುವುದು. ಅನಂತರ ಅದನ್ನು ಕುರಿತು ಚಿಂತಿಸಬೇಕಾಗಿಲ್ಲ.

ಅವನು ಕೋಪಗೊಳ್ಳುವುದಿಲ್ಲ. ಏಕೆಂದರೆ ಮನುಷ್ಯನಿಗೆ ಕೋಪವೇ ಪರಮ ಶತ್ರು. ಯಾವಾಗ ಕೋಪಕ್ಕೆ ತುತ್ತಾಗುವೆವೋ ಆಗ ಯುಕ್ತಾಯುಕ್ತಪರಿಜ್ಞಾನವನ್ನು ಕಳೆದುಕೊಂಡು ಮಾಡಬಾರದುದನ್ನು ಮಾಡಿಬಿಡುವೆವು. ಅನಂತರ ಎಷ್ಟು ಪಶ್ಚಾತ್ತಾಪಪಟ್ಟರೂ ಆದ ಅನಾಹುತವನ್ನು ನೇರಮಾಡಲು ಆಗುವುದಿಲ್ಲ. ಇವುಗಳನ್ನೆಲ್ಲ ಮುಂಚೆಯೇ ಕುರಿತು ಆಲೋಚಿಸುವವನು ಅವನು. ಕೋಪದಷ್ಟು ನಮ್ಮ ಶಕ್ತಿಯನ್ನು ವ್ಯಯಮಾಡುವುದು ಮತ್ತೊಂದಿಲ್ಲ.

ಅವನು ತ್ಯಾಗಮಾಡಿರುವನು. ಇರುವ ವಸ್ತುವಿಗೆ ಅಂಟಿಕೊಂಡಿಲ್ಲ. ಹೊಸ ವಸ್ತುಗಳು ಅವನಿಗೆ ಬೇಕಾಗಿಲ್ಲ. ತ್ಯಾಗದ ಮೂಲಕ ಆನಂದಿಸುವುದನ್ನು ಅವನು ಕಲಿತಿರುವನು. ನಾವು ಅನೇಕ ವೇಳೆ ಭಾವಿಸುತ್ತೇವೆ ಭೋಗದ ಮೂಲಕ ಮಾತ್ರ ಆನಂದಿಸುವುದು ಸಾಧ್ಯ ಎಂದು. ಭೋಗದ ಮೂಲಕ, ಭೋಗಿಸುವ ವಸ್ತುವಿಗೆ ದಾಸರಾಗುತ್ತೇವೆ. ಆ ಅನುಭವವನ್ನೇ ಪುನಃ ಪಡೆಯಲು ನಾವು ಏನು ಬೇಕಾದರೂ ಮಾಡಲು ಸಿದ್ಧರಾಗುತ್ತೇವೆ. ಯಾವಾಗ ಗುಲಾಮರಾಗುತ್ತೇವೆಯೊ ಆಗ ಎಲ್ಲಿದೆ ಆನಂದ? ತ್ಯಾಗಿ ಹಾಗಲ್ಲ. ಅವನು ತನ್ನನ್ನು ಕಟ್ಟಿಹಾಕುವ ವಸ್ತುವಿನ ಆಕರ್ಷಣೆಯಿಂದ ಪಾರಾಗಿರು ವನು. ಇನ್ನು ಮೇಲೆ ಅವನು ಅದರ ಕಾಟದಿಂದ ಪಾರಾಗಿರುವನು. ಅವನಿಗೆ ಗೊತ್ತಿದೆ ನಿಜವಾದ ಆನಂದ ಏನು ಎಂಬುದು. ಈಶಾವಾಸ್ಯ ಉಪನಿಷತ್ತು ತ್ಯಾಗದಿಂದ ಆನಂದಿಸು ಎಂದು ಹೇಳುವುದು.

ಅವನ ಮನಸ್ಸು ಪ್ರಶಾಂತವಾಗಿರುವುದು. ಹೊರಗಿನ ಸುದ್ದಿ ಸಮಾಚಾರಗಳು ಒಂದೂ ಅವನ ಸ್ವಾಸ್ಥ್ಯವನ್ನು ಕೆಡಿಸಲಾರವು. ಹಾಗೆಯೇ ಒಳಗಿನಿಂದ ಆಸೆ ಆಕಾಂಕ್ಷೆಗಳ ಗುಳ್ಳೆಗಳು ಎದ್ದು ಅವನ ಶಾಂತಿಗೆ ಭಂಗತಾರವು. ಬೆಣ್ಣೆ ಒಲೆ ಮೇಲೆ ಇರುವಾಗ ಗಲಾಟೆ ಮಾಡುವುದು. ಯಾವಾಗ ಅದರಲ್ಲಿ ನೀರಿನ ಅಂಶ ಹೋಗುವುದೋ ಆಗ ಶಾಂತವಾಗುವುದು. ಹಾಗೆಯೇ ನಮ್ಮ ಗಲಾಟೆಗೆಲ್ಲ ಕಾರಣ ನಮ್ಮ ಮನಸ್ಸಿನಲ್ಲಿರುವ ನೀರಿನ ಅಂಶ. ಆ ನೀರಿನ ಅಂಶವೇ ಪ್ರಪಂಚದ ಆಸೆ ಆಕಾಂಕ್ಷೆಗಳು. ಇವುಗಳೆಲ್ಲ ಇಂಗಿ ಹೋಗಿವೆ ದೈವೀ ಸಂಪತ್ತಿನ ಮನುಷ್ಯನಲ್ಲಿ.

ಅವನು ಇನ್ನೊಬ್ಬರ ಮೇಲೆ ಚಾಡಿ ಹೇಳುವುದಕ್ಕೆ ಹೋಗುವುದಿಲ್ಲ. ಒಬ್ಬನನ್ನು ಮತ್ತೊಬ್ಬನ ಮೇಲೆ ಎತ್ತಿ ಕಟ್ಟುವುದು, ಮೇಲಿರುವ ಮನುಷ್ಯನನ್ನು ಕೆಳಗೆ ತರುವುದಕ್ಕೆ ಮಾಡುವ ಹಲವು ಉಪಾಯಗಳು, ಇಂತಹವುಗಳೆಲ್ಲ ಅವನ ಸ್ವಭಾವಕ್ಕೆ ಹೊರತು. ಅವನು ಇನ್ನೊಬ್ಬನ ಮೇಲೆ ಏನನ್ನೂ ಹೇಳುವುದಕ್ಕೆ ಹೋಗುವುದಿಲ್ಲ. ಯಾವಾಗ ಒಬ್ಬ ಚಾಡಿ ಹೇಳುವನೊ, ಆಗ ಇನ್ನು ಯಾರೊ ಅವನ ಮೇಲೆಯೂ ಹೇಳುವರು. ಈ ಪ್ರಪಂಚದಲ್ಲಿ ಪ್ರತಿಯೊಬ್ಬನಿಗೂ ಅವನು ಮಾಡಿದ್ದು ಕಾದಿದೆ–ಅವನಿಗೆ ಶಿಕ್ಷೆ ವಿಧಿಸುವುದಕ್ಕೊ ಅಥವಾ ಬಹುಮಾನ ಕೊಡುವುದಕ್ಕೊ, ಮಾಡಿದ್ದುಣ್ಣೊ ಮಹರಾಯ ಎಂಬ ಗಾದೆಯಂತೆ. ಈ ಮರ್ಮವನ್ನು ಚೆನ್ನಾಗಿ ಅರಿತವನು ಅವನು.

ಅವನು ಎಲ್ಲರ ಮೇಲೆಯೂ ಕರುಣೆಯನ್ನು ಬೀರುವನು. ಅವನಿಗೆ ತನ್ನವರು ಪರಕೀಯರು ಎಂಬ ಭಾವನೆ ಇಲ್ಲ. ಎಲ್ಲರಿಗೂ ಮರುಗುವನು ಅವನು. ಭಗವಂತನ ಕಡೆಗೆ ಒಬ್ಬ ಹೋಗುತ್ತಿರು ವನು ಎಂಬುದಕ್ಕೆ ಒಂದು ಪ್ರಮಾಣವೆ ಇದು. ಎಷ್ಟು ಅವನ ಹೃದಯ ಇತರರಿಗೆ ಮರುಗುವುದೊ, ಅಷ್ಟು ಅವನು ದೇವರ ಸಮೀಪದಲ್ಲಿರುವನು. ಅವರು ಯಾವ ಜಾತಿ ಕುಲ ಗೋತ್ರಗಳಿಗೆ ಸೇರಿದ್ದರೂ ಚಿಂತೆಯಿಲ್ಲ, ಅವನು ಎಲ್ಲರಿಗೂ ಮರುಗುವನು. ದೇವರ ಹತ್ತಿರ ಹತ್ತಿರ ಹೋದಷ್ಟೂ ಎಲ್ಲರೂ ನಮ್ಮವರು ಎಂಬ ಭಾವ ನಮ್ಮಲ್ಲಿ ಜಾಗ್ರತವಾಗುತ್ತ ಬರುವುದು. ಅದಕ್ಕೇ ಸ್ವಾಮಿ ವಿವೇಕಾನಂದರು ಹೇಳುತ್ತಿದ್ದರು, ಯಾರ ಹೃದಯ ಮತ್ತೊಬ್ಬರಿಗೆ ಮರುಗುವುದೊ ಅವನು ಮಹಾತ್ಮ, ಇಲ್ಲದೇ ಇದ್ದರೆ ಅವನು ದುರಾತ್ಮ ಎಂದು.

ಅವನು ವಿಷಯ ಲಂಪಟನಲ್ಲ. ಇಲ್ಲದಾಗ ವಿಷಯ ವಸ್ತುಗಳಿಗೆ ಆಶಿಸುವುದಿಲ್ಲ. ಅದು ಎದುರಿಗೆ ಇದ್ದಾಗಲೂ ಅದರ ಬಲೆಗೆ ಬೀಳುವುದಿಲ್ಲ. ವಿಷಯ ವಸ್ತು ಎದುರಿಗೆ ಇರಬಹುದು. ಆದರೆ ಅವನು ತನ್ನ ಇಂದ್ರಿಯಗಳನ್ನು ಅದರ ಕಡೆ ಹರಿಸುವುದಿಲ್ಲ. ರಾಮನ ಕಡೆ ಹೋಗುವವನು ಕಾಮವನ್ನು ಬಿಟ್ಟಿರಬೇಕು. ಕಾಮವನ್ನು ಕಟ್ಟಿಕೊಂಡು ರಾಮನ ಕಡೆ ಹೋಗುವುದಕ್ಕೆ ಆಗುವುದಿಲ್ಲ.

ಅವನು ಮೃದು ಸ್ವಭಾವದವನು. ಯಾರಿಗೂ ಎಳ್ಳಷ್ಟೂ ನೋಯಿಸುವುದಿಲ್ಲ. ಆದರೆ ಅವನು ದುರ್ಬಲನಲ್ಲ. ಒಳಗೆ ಉಕ್ಕಿಗಿಂತ ಗಟ್ಚಿ. ಹೊರಗೆ ಕುಸುಮಕ್ಕಿಂತ ಕೋಮಲ. ಅವನ ಮೃದುತ್ವವನ್ನು ನೋಡಿ ಯಾರು ಬೇಕಾದರೂ ಅವನನ್ನು ಹಿಂಡ ಬಹುದು ಎಂದು ಭಾವಿಸಬಾರದು. ಅವನು ಮೃದು ಸರಳ; ಆದರೆ ಪೆದ್ದನಲ್ಲ. ಅವನಷ್ಟು ಜಾಣ ಇನ್ನಿಲ್ಲ. ಅವನಷ್ಟು ಧೀರ ಇನ್ನಿಲ್ಲ.

ಅವನು ಕೆಟ್ಟದ್ದನ್ನು ಮಾಡುವುದಕ್ಕೆ ನಾಚುವನು. ಇತರರು ಅವನನ್ನು ಛೀಮಾರಿ ಮಾಡ ಬೇಕಾಗಿಲ್ಲ. ಮಾಡುವುದಕ್ಕೆ ಮುಂಚೆ ಒಡಂಬಡುವುದಿಲ್ಲ. ಈ ಹೀನ ಕೃತ್ಯವನ್ನು ಮಾಡಿದರೆ ನಾಳೆ ಇತರರಿಗೆ ಮುಖವನ್ನು ಹೇಗೆ ತೋರಿಸಿಕೊಳ್ಳುವುದು ಎಂದು ಮುಂಚೆಯೇ ಆಲೋಚಿಸುವನು. ಅವನು ತನ್ನಲ್ಲಿರುವ ಒಳ್ಳೆಯದನ್ನು ಕೂಡ ಇತರರೊಂದಿಗೆ ಹೇಳಿ ಕೊಳ್ಳುವುದಕ್ಕೆ ನಾಚುವನು. ಬಡಾಯಿ ಕೊಚ್ಚಿಕೊಳ್ಳುವವರ ಗುಂಪಿಗೆ ಸೇರಿದವನಲ್ಲ. ಎಲೆಯ ಮರೆಯ ಹಿಂದೆ ಇರುವ ಕುಸಮದಂತೆ ಅವನು. ಅದರ ಪರಿಮಳ ಅಗೋಚರವಾಗಿ ವ್ಯಕ್ತವಾಗುವುದು.

ಅವನ ಮನಸ್ಸಿನಲ್ಲಿ ಚಪಲವಿಲ್ಲ, ಆಸೆಗಳಿಲ್ಲ. ಇದು ಹೇಗೋ ಅದು ಹೇಗೋ ಎಂಬ ಕುತೂಹಲ ಮನಸ್ಸನ್ನು ಚಂಚಲಗೊಳಿಸುವುದಿಲ್ಲ. ಬುದ್ಧಿ ಸ್ಥಿರವಾಗಿದೆ; ಏಕಾಗ್ರವಾಗಿದೆ. ಚಪಲ ಚಿತ್ತನಿಗೆ ಇದು ಸಾಧ್ಯವಿಲ್ಲ.

ಅವನು ತೇಜಸ್ಸಿನಿಂದ ಕೂಡಿರುವನು. ಅವರ್ಣನೀಯ ಕಾಂತಿ ಅವನಲ್ಲಿ ವ್ಯಕ್ತವಾಗುತ್ತಿದೆ. ಅವನು ಭಗವದ್​ಸಾಂನಿಧ್ಯದಲ್ಲಿರುವನು. ಚಿಮಣಿ, ಹಿಂದೆ ಇರುವ ಕಾಂತಿಯನ್ನು ಹೇಗೆ ವ್ಯಕ್ತಗೊಳಿಸುತ್ತಿರು ವುದೊ ಹಾಗೆ ಅವನ ವ್ಯಕ್ತಿತ್ವ ತೇಜಸ್ಸನ್ನು ಪ್ರಕಟಗೊಳಿಸುತ್ತಿರುವುದು. ಅವನ ಮಾತುಕತೆ ನಡತೆ ಇವುಗಳಲ್ಲೆಲ್ಲ ಆ ಕಾಂತಿ ಸ್ಪಂದಿಸುವುದನ್ನು ನೋಡುತ್ತೇವೆ. ಗಂಧದ ಕಾರ್ಖಾನೆಯಿಂದ ಬಂದವನ ಬಟ್ಟೆಗಳೆಲ್ಲ ಹೇಗೆ ಘಮಘಮಿಸುತ್ತಿರುವುವೋ ಹಾಗೆ ಇವನ ವ್ಯಕ್ತಿತ್ವದಲ್ಲೆಲ್ಲ ತೇಜಸ್ಸು ಪ್ರಕಟವಾಗು ತ್ತಿರುವುದು.

ಕ್ಷಮಾಗುಣ ಅವನಲ್ಲಿ ನೆಲೆಸಿರುವುದು. ಯಾರು ಏನು ತಪ್ಪನ್ನು ಮಾಡಲಿ ಅದನ್ನು ಕ್ಷಮಿಸುವನು. ಅವರು ಮಾಡಿದ ತಪ್ಪನ್ನು ಮೆಲಕು ಹಾಕುತ್ತ ಇರುವುದಿಲ್ಲ. ಅದನ್ನು ಬಡ್ಡಿಸಹಿತ ಹಿಂತಿರುಗಿ ಕೊಡುವುದಕ್ಕೆ ಅವಕಾಶ ಬಂದರೂ ಅದನ್ನು ಮಾಡುವುದಿಲ್ಲ. ಪೆಟ್ಟಿಗೆ ಪೆಟ್ಟು ಮೃಗೀಯ ನೀತಿ. ಪ್ರಪಂಚದಲ್ಲಿ ಯಾರೂ ಎಡವದೆ ಇರುವುದಿಲ್ಲ. ಎಲ್ಲರೂ ಒಂದಲ್ಲ ಒಂದು ಸಾರಿ ಎಡವುತ್ತಾರೆ. ಇನ್ನೊಬ್ಬರನ್ನು ಯಾವಾಗಲೂ ಉದಾರ ದೃಷ್ಟಿಯಿಂದ ನೋಡುತ್ತಾನೆ.

ಹೊರಗಿನಿಂದ ನೋಡಿದರೆ ಮೃದು ಸ್ವಭಾವದವನು ಅವನು. ಆದರೆ ಅವನಲ್ಲಿರುವಷ್ಟು ಧೈರ್ಯ ಇನ್ನಾರಲ್ಲಿಯೂ ಇರವುದಿಲ್ಲ. ಜೀವನದಲ್ಲಿ ಗುರಿ ಎಡೆಗೆ ಸಾಗುವಾಗ ಎಂತಹ ಕಷ್ಟಗಳು ಬರಲಿ, ಬೆನ್ನು ತೋರಿಸುವುದಿಲ್ಲ. ಅವುಗಳ ಮೂಲಕ ಗುರಿಯೆಡೆಗೆ ಸಾಗುತ್ತಾನೆ. ಮಿಥ್ಯದೊಡನೆ ರಾಜಿ ಮಾಡಿಕೊಳ್ಳುವವನಲ್ಲ. ಸೋಲನ್ನು ಒಪ್ಪಿಕೊಳ್ಳುವವನಲ್ಲ. ಜೀವನದಲ್ಲಿ ಎಂತಹ ಪ್ರಸಂಗದ ಲ್ಲಿಯೂ ಧೈರ್ಯಗೆಡುವುದಿಲ್ಲ. ಭರವಸೆಯೇ ಮೂರ್ತಿವೆತ್ತಂತೆ ಇರುವನು.

ಅವನು ಬಾಹ್ಯದಲ್ಲಿ ಶುಚಿಯಾಗಿರುವನು, ಅಂತರದಲ್ಲಿ ಶುಚಿಯಾಗಿರುವನು. ಅವನ ಬಟ್ಟೆ ಬರೆ ಊಟ ಉಪಚಾರ ಇವುಗಳೆಲ್ಲ ಶುಚಿಯಾಗಿರುವುವು. ಶೋಕಿಯಲ್ಲ, ಶುಚಿ ಇವನ ಗುರಿ. ಶುಚಿಯಾ ಗಿರುವುದಕ್ಕೆ ಹೆಚ್ಚು ದ್ರವ್ಯ ವೆಚ್ಚವಾಗುವುದಿಲ್ಲ. ಅದಕ್ಕೆ ಸ್ವಲ್ಪ ಶ್ರಮ ತೆಗೆದುಕೊಳ್ಳಬೇಕಷ್ಟೆ. ಹಾಗೆಯೇ ಅವನು ಆಂತರ್ಯದಲ್ಲಿ ಯಾವ ಹೀನ ಆಲೋಚನೆಗೂ ಅವಕಾಶ ಕೊಡುವುದಿಲ್ಲ. ಬಾಹ್ಯ ಮಡಿ ಸುಲಭ. ಆಂತರಿಕ ಮಡಿಯೇ ಕಷ್ಟ. ನಮ್ಮ ಮನಸ್ಸನ್ನು ಯಾವ ಹೀನ ಲೌಕಿಕ ಆಸೆಯೂ ಮುಟ್ಟದಂತೆ ನೋಡಿಕೊಳ್ಳಬೇಕು. ಇಂಗ್ಲೀಷಿನಲ್ಲಿ\enginline{Cleanliness is next to Godliness }ಎಂಬ ನುಡಿ ಇದೆ. ಇಲ್ಲಿ \enginline{cleanliness }ಎಂದರೆ ಬರೀ ಬಾಹ್ಯ ಶುಚಿ ಮಾತ್ರವಲ್ಲ, ಆಂತರಿಕ ಶುಚಿ ಕೂಡ ಸೇರಿದೆ. ಶುಚಿಯಾಗಿರುವ ಮನಸ್ಸು ಭಗವಂತನನ್ನು ಚೆನ್ನಾಗಿ ಪ್ರತಿಬಿಂಬಿಸುವುದು.

ಅವನು ಇತರರಿಗೆ ದ್ರೋಹವನ್ನೆಸಗುವುದಿಲ್ಲ. ಇತರರು ಇವನಿಗೆ ದ್ರೋಹವನ್ನು ಮಾಡಬಹುದು. ಆದರೆ ಇವನ ಧರ್ಮವೇ ಬೇರೆ. ಯಾರು ಏನು ಮಾಡಿದರೂ ಒಳ್ಳೆಯದಲ್ಲದೆ ಇನ್ನೇನೂ ಇವನಲ್ಲಿ ವ್ಯಕ್ತವಾಗುವುದಿಲ್ಲ. ಚಂದನದ ಮರ, ಕೊಡಲಿ ಹಿಡಿದುಕೊಂಡು ತನ್ನನ್ನು ಕತ್ತರಿಸುವುದಕ್ಕೆ ಬಂದವ ನಿಗೂ ತನ್ನ ಸುಗಂಧವನ್ನು ಕೊಡುವುದು. ಹಾಗೆಯೇ ದೈವೀ ಸಂಪತ್ತಿನ ಮನುಷೃ.

ಅವನು ಅಭಿಮಾನ ಶೂನ್ಯ. ತನ್ನಲ್ಲಿ ಒಳ್ಳೆಯ ಗುಣಗಳು ಇರಬಹುದು. ಆದರೆ ಎಂದಿಗೂ ಇವುಗಳ ವಿಷಯದಲ್ಲಿ ಹೆಮ್ಮೆ ಪಡುವುದಿಲ್ಲ. ಇವುಗಳಿರುವುದು ತೋರಿಸಿಕೊಳ್ಳುವುದಕ್ಕಲ್ಲ ಎಂಬು ದನ್ನು ಚೆನ್ನಾಗಿ ಅರಿತಿರುವನು ಅವನು. ಇವು ತಮ್ಮ ಪರಾಗದ ಕಂಪನ್ನು ಸುತ್ತಲೂ ಕಸ್ತೂರಿಯಂತೆ ಬೀರಬೇಕೇ ಹೊರತು, ನಾನಿಲ್ಲಿರುವೆನು ಎಂದು ಅರಚಿಕೊಳ್ಳಬೇಕಾಗಿಲ್ಲ. ದೀಪ ನಾನಿಲ್ಲಿರುವೆನು ಎನ್ನುವುದೆ? ಸುಮ್ಮನೆ ಸುತ್ತಲೂ ಕಾಂತಿಯನ್ನು ಚೆಲ್ಲುವುದು. ಹೂವು, ಗಿಡದಲ್ಲಿ ನಾನಿಲ್ಲಿರುವೆ ಎಂದು ಒರಲುವುದೆ? ಮೌನವಾಗಿ ಸೌರಭವನ್ನು ಹರಡುವುದು. ಹಾಗೆಯೇ ದೈವೀ ಸಂಪತ್ತಿನ ಮನುಷ್ಯ. ಅವನ ಜೀವನವೇ ಒಂದು ಸುಂದರ ಚಾರಿತ್ರದ ಹೂದೋಟ. ಆ ಹೂವುಗಳು ತಮ್ಮ ಪರಿಮಳವನ್ನು ದಿಕ್ಕುದಿಕ್ಕಿಗೆ ಸೂಸುತ್ತ ಇರುತ್ತವೆ.

\begin{verse}
ದಂಭೋ ದರ್ಪೋಽತಿಮಾನಶ್ಚ ಕ್ರೋಧಃ ಪಾರುಷ್ಯಮೇವ ಚ~।\\ಅಜ್ಞಾನಂ ಚಾಭಿಜಾತಸ್ಯ ಪಾರ್ಥ ಸಂಪದಮಾಸುರೀಮ್ \versenum{॥ ೪~॥}
\end{verse}

{\small ಅರ್ಜುನ, ದಂಭ, ದರ್ಪ, ಅಹಂಕಾರ, ಕ್ರೋಧ ಒರಟಾದ ಮಾತು ಮತ್ತು ಅಜ್ಞಾನ ಇವು ಆಸುರೀ ಸಂಪತ್ತಿಗೆ ಸೇರಿದ ಗುಣಗಳು.}

ರಜೋಗುಣಿಯ ಅಥವಾ ಆಸುರೀ ಸ್ವಭಾವದ ಮನುಷ್ಯನ ಗುಣಗಳು ಇವು. ಅವನು ಢಂಬಾ ಚಾರಿ. ತೋರಿಕೆ ಅವನಲ್ಲಿ ಹೆಚ್ಚು. ಧಾರ್ಮಿಕ ವಿಷಯಗಳಲ್ಲಿ ತನ್ನ ಭಕ್ತಿ ಭಾವ ಮುಂತಾದುವು ಗಳನ್ನೆಲ್ಲ ಹೊರಗೆ ವ್ಯಕ್ತಪಡಿಸುವನು. ಒಂದು ಸ್ವಲ್ಪ ಇದ್ದರೂ ಅದೇನೋ ತುಂಬಾ ಹೆಚ್ಚಾಗಿ ಅವನಲ್ಲಿ ಇರುವಂತೆ ತೋರಿಸಿಕೊಳ್ಳುವನು. ಅವನು ಒಂದು ಸ್ವಲ್ಪ ಹೊತ್ತು ಧ್ಯಾನ ಮಾಡಿದರೂ ಅದು ಎಲ್ಲರಿಗೂ ಗೊತ್ತಾಗಬೇಕು. ಯಾರೂ ನೋಡದೆ ಇದ್ದರೆ ಇವನು ತಾನು ಮಾಡಿದ್ದು ವ್ಯರ್ಥ ಎಂದು ಭಾವಿಸುತ್ತಾನೆ. ಶ‍್ರೀರಾಮಕೃಷ್ಣರ ಬಳಿಗೆ ಇಂತಹ ಒಬ್ಬ ಭಕ್ತ ಬರುತ್ತಿದ್ದ. ಆತ ಪಂಚವಟಿ ಯಲ್ಲಿ ಒಂದು ಕೃಷ್ಣಾಜಿನದ ಮೇಲೆ ಕಣ್ಣುಮುಚ್ಚಿ ಧ್ಯಾನಕ್ಕೆ ಕುಳಿತುಕೊಳ್ಳುತ್ತಿದ್ದ. ಎಲ್ಲರೂ ಬಂದು ಹೋಗುವ ಸ್ಥಳ ಅದು. ಅವನು ಧ್ಯಾನ ಮಾಡುವುದು ಎಲ್ಲರಿಗೂ ಗೊತ್ತಾಗುವುದು. ಅನಂತರ ತಾನು ಕುಳಿತುಕೊಳ್ಳುತ್ತಿದ್ದ ಕೃಷ್ಣಾಜಿನವನ್ನು ಶ‍್ರೀರಾಮಕೃಷ್ಣರ ಕೋಣೆಯೊಳಗೆ ಎಲ್ಲರಿಗೂ ಕಾಣುವಂತೆ ಇಡುತ್ತಿದ್ದ. ಏಕೆಂದರೆ ಅವರು ಕೋಣೆಗೆ ಬಂದವರು ಇದು ಯಾರದು ಎಂದು ಕೇಳಿದಾಗ, ಅದರ ಮೇಲೆ ಕುಳಿತುಕೊಳ್ಳುವವನ ಹೆಸರು ಹೇಳಬೇಕಾಗುವುದು. ಅವನಿಲ್ಲದಾಗಲೂ ಅವನ ವಿಷಯದಲ್ಲಿ ಜಾಹಿರಾತು ಆಗುತ್ತಿರುವುದು.

ಅವನಲ್ಲಿ ಲೌಕಿಕಕ್ಕೆ ಸಂಬಂಧಪಟ್ಟಿದ್ದು ಏನಿದ್ದರೂ ಅದನ್ನು ಮೆರೆಸುತ್ತಿರುವನು. ಐಶ್ವರ್ಯ ಇರಬಹುದು, ಅಧಿಕಾರ ಇರಬಹುದು, ದೊಡ್ಡ ಹುದ್ದೆಯಲ್ಲಿರುವ ನಂಟರೋ ಸ್ನೇಹಿತರೊ ಯಾರಾ ದರೂ ಇರಬಹುದು. ಅದನ್ನೆಲ್ಲ ಪದೇ ಪದೇ ಹೇಳಿಕೊಳ್ಳುತ್ತಿರುವನು. ದರ್ಪ ಒಂದು ಸುರೆಯಂತೆ. ಅದನ್ನು ಕುಡಿದು ಅಮಲೇರಿರುವನು. ಒಳಗೆ ಖಾಲಿ, ಹೊರಗೆಲ್ಲ ಆಟೋಪ. ಒಂದು ಸ್ವಲ್ಪಕ್ಕೇ ತಲೆ ತಿರುಗುತ್ತದೆ ಅವನಿಗೆ.

ಅಹಂಕಾರಿ ಅವನು. ಇದನ್ನು ಹೊರಗೆ ತೋರಿಸಿಕೊಳ್ಳುವುದರ ಜೊತೆಗೆ ತನ್ನಲ್ಲಿ ಇವುಗಳೆಲ್ಲ ಇದೆ ಎಂಬುದನ್ನು ಚೆನ್ನಾಗಿ ಅರಿತಿರುವನು. ತನ್ನ ಸಮಾನ ಇಲ್ಲ ಈ ಪ್ರಪಂಚದಲ್ಲಿ ಎಂದು ಭಾವಿಸುವನು. ಅಲ್ಪವಿದ್ಯ ಮಹಾಗರ್ವ ಎಂಬ ಮಾತು ಇವನಿಗೆ ಚೆನ್ನಾಗಿ ಅನ್ವಯಿಸುವುದು. ಪಾಪ! ಈ ಜೀವಿ ತನ್ನ ಅಹಂಕಾರದ ಕೋಡನ್ನು ಈ ಪ್ರಪಂಚದ ಬಂಡೆಗೆ ತಾಕಿಸಿಲ್ಲ ಇನ್ನೂ. ಅನಂತರ ಅದಕ್ಕೆ ಗೊತ್ತಾಗುವುದು ಈ ಪ್ರಪಂಚವೆಂಬ ಬಂಡೆ ದುರಹಂಕಾರಿಯ ಕೋಡಿಗೆ ಜಗ್ಗುವುದಿಲ್ಲ ಎಂಬುದು. ಬೇಕಾದಷ್ಟು ಬುದ್ಧಿ ಕಲಿತ ಮೇಲೆ ಈ ಮಹಾನೀತಿ ಅರ್ಥವಾಗುವುದು. ಇವನಿನ್ನೂ ಪ್ರಥಮ ಘಟ್ಟದಲ್ಲಿರುವನು.

ತನ್ನ ಸ್ವಾರ್ಥತೆಗೆ ಭಂಗ ಬಂದೊಡನೆಯೆ ಇತರರ ಮೇಲೆ ತಿರುಗಿ ಬೀಳುತ್ತಾನೆ. ಯುಕ್ತಾಯುಕ್ತಾ ಪರಿಜ್ಞಾನವನ್ನು ಕಳೆದುಕೊಳ್ಳುತ್ತಾನೆ. ತನ್ನ ದುಃಖಕ್ಕೆ ತಾನೇ ಕಾರಣನಾಗುತ್ತಾನೆ.

ಅವನ ಮಾತಿನಲ್ಲಿ ನಯ ವಿನಯವಿಲ್ಲ. ಇನ್ನೊಬ್ಬನನ್ನು ನೋಯಿಸುವುದಕ್ಕೆ ಸ್ವಲ್ಪವೂ ಅನು ಮಾನಿಸುವುದಿಲ್ಲ. ನಾನು ಅವನಿಗೆ ಹೀಗೆಂದೆ, ಹಾಗೆಂದೆ ಅಂತಹ ಬರೆ ಹಾಕಿದೆ ಎಂದು ಅನಂತರ ಹೆಮ್ಮೆ ಕೊಚ್ಚಿಕೊಳ್ಳುವುದು ಬೇರೆ ಇತರರ ಹತ್ತಿರ. ಯಾವುದಕ್ಕೆ ನಾಚಬೇಕೋ ಅದಕ್ಕೆ ಹೆಮ್ಮೆ ಕೊಚ್ಚಿಕೊಳ್ಳುತ್ತಾನೆ. ಮಾತಿನಿಂದ ಇತರರನ್ನು ಚುಚ್ಚುವುದು ಸುಲಭ. ಆದರೆ ಆ ಚುಚ್ಚು ಮಾತಿ ನಿಂದಾದ ಗಾಯ ಮಾಗಲು ಬಹಳ ದಿನಗಳು ಬೇಕು. ಒಂದು ವೇಳೆ ಮಾಗಿದರೂ ಮಚ್ಚೆ ನಿಲ್ಲುವುದು. ಅದು ಜೀವಾವಧಿ ಇರುವುದು. ನಾಲಗೆಯನ್ನು ಕತ್ತಿಯಂತೆ ಉಪಯೋಗಿಸಿದ ಫಲ ಇದು. ಇವನ ನಾಲಗೆಯಲ್ಲಿ ಅಮೃತವಲ್ಲ ತೊಟ್ಟಿಕ್ಕುವುದು, ಚೇಳಿನ ಬಾಲದಲ್ಲಿರುವ ವಿಷ ಇರುವುದು. ಇದರಿಂದ ಕುಟುಕುತ್ತಾನೆ.

ಅವನು ಅಜ್ಞಾನದಲ್ಲಿ ತೊಳಲುತ್ತಿರುವನು. ಯಾವುದು ಸರಿ ಯಾವುದು ತಪ್ಪು ಎಂಬುವುದು ಗೊತ್ತಿಲ್ಲ. ಮುಂದೇನಾಗುವುದು ಎಂಬುದನ್ನು ಕುರಿತು ಆಲೋಚಿಸುವುದಿಲ್ಲ. ತನ್ನ ಮುಂದಿರುವ ಗೇಣುದ್ದವೇ ಅವನಿಗೆ ಕಾಣುವುದು. ಅದರ ಹಿಂದೆ ಏನಿದೆ ಎಂಬುದು ಅವನಿಗೆ ಗೊತ್ತಿಲ್ಲ.

\begin{verse}
ದೈವೀಸಂಪದ್ವಿಮೋಕ್ಷಾಯ ನಿಬಂಧಾಯಾಸುರೀ ಮತಾ~।\\ಮಾ ಶುಚಃ ಸಂಪದಂ ದೈವೀಮಭಿಜಾತೋಽಸಿ ಪಾಂಡವ \versenum{॥ ೫~॥}
\end{verse}

{\small ದೈವೀ ಸಂಪತ್ತು ಮೋಕ್ಷದಾಯಕ, ಆಸುರೀ ಸಂಪತ್ತು ಬಂಧನದಲ್ಲಿ ತೊಡಗಿಸುವುದು. ಅರ್ಜುನ, ವಿಷಾದಿಸ ಬೇಡ. ನೀನು ದೈವೀಸಂಪತ್ತನ್ನು ಪಡೆದು ಹುಟ್ಟಿದ್ದೀಯೆ.}

ಯಾರಲ್ಲಿ ದೈವೀ ಸಂಪತ್ತು ಇದೆಯೋ ಅವರನ್ನು ಅದು ಮೋಕ್ಷಕ್ಕೆ ಕರೆದೊಯ್ಯುವುದು. ಯಾರಲ್ಲಿ ಆಸುರೀ ಸಂಪತ್ತು ಇದೆಯೋ ಅದು ಅವರನ್ನು ಸಂಸಾರದ ಗೋಜಿನಲ್ಲಿ ಸಿಕ್ಕಿಸುವುದು. ಅವನ ಬಂಧನ ಹೆಚ್ಟಾಗುತ್ತ ಬರುವುದು. ಏನೇನನ್ನೊ ಮುಂಚೆ ಬಿತ್ತುತ್ತಾನೆ. ಆಗ ಸಂತೋಷಪಡುತ್ತಾನೆ. ಅನಂತರ ಅದೆಲ್ಲ ಕೊಯ್ಲಿಗೆ ಬಂದಾಗ ವ್ಯಥೆಪಡುತ್ತಾನೆ. ಆದರೆ ಬಿಡುವುದಕ್ಕೆ ಆಗುವುದಿಲ್ಲ. ತಾನು ಬಿತ್ತಿದ್ದನ್ನು ತಾನೆ ಕೊಯ್ಯಬೇಕು, ಬಡಿಯಬೇಕು, ಕಣಜದಲ್ಲಿ ತುಂಬಬೇಕು. ಆಗ ವ್ಯಥೆಪಡುವನು.

ಅರ್ಜುನ ಏನೋ ತಾನು ಆಸುರೀ ಪ್ರವೃತ್ತಿಗೆ ಹುಟ್ಟಿದವನೇನೊ ಎಂದು ಸಂಶಯಪಡಬಹುದು. ಅದಕ್ಕಾಗಿಯೇ ಶ‍್ರೀಕೃಷ್ಣ ತಕ್ಷಣವೇ ಅವನನ್ನು ಸಮಾಧಾನ ಮಾಡುತ್ತಾನೆ. ನೀನು ದೈವೀ ಸ್ವಭಾವ ದಿಂದ ಹುಟ್ಟಿದ್ದೀಯೆ ಎನ್ನುತ್ತಾನೆ. ಅರ್ಜುನನಲ್ಲಿ ಕುರುಕ್ಷೇತ್ರಕ್ಕೆ ಬಂದಾಗಿನಿಂದಲೂ ಒಳ್ಳೆಯ ಗುಣವನ್ನೇ ನೋಡುತ್ತೇವೆ. ಗುರು ಹಿರಿಯರ ಮೇಲೆ ಅವನಿಗೆ ಪೂಜ್ಯಭಾವ ಇದೆ. ಅವರನ್ನು ಬಾಣದಿಂದ ನೋಯಿಸಲು ಇಚ್ಛಿಸುವುದಿಲ್ಲ. ಅವರನ್ನು ಕೊಲ್ಲುವುದಕ್ಕೆ ಮನಸ್ಸು ಬರುವುದಿಲ್ಲ. ರಾಜ್ಯದ ಮೇಲೆ ಅಭಿಲಾಷೆ ಇಲ್ಲ. ಕುರಕ್ಷೇತ್ರದ ಮಹಾಕೊಲೆಯಾದ ಮೇಲೆ ಸಮಾಜ ಎಂತಹ ದುರವಸ್ಥೆಗೆ ಬರುವುದು, ವರ್ಣಾಶ್ರಮದ ಗತಿ ಏನಾಗುವುದು, ಇವುಗಳನ್ನೆಲ್ಲ ಕುರಿತು ಚಿಂತಿಸುವಾಗ ತನ್ನ ಸ್ವಾರ್ಥವನ್ನು ಬದಿಗೊಡ್ಡುವನು. ಅರ್ಜುನ ಇಂತಹ ಸಂದಿಗ್ಧ ಪರಿಸ್ಥಿತಿಯಲ್ಲಿ ತನ್ನ ಕರ್ತವ್ಯ ವೇನು ಎಂಬುದನ್ನು ಮರೆತಿದ್ದೇ ಒಂದು ಲೋಪ ಅವನ ಗುಣದಲ್ಲಿ. ಆದರೆ ಇದು ಮಹಾದೋಷ ವಲ್ಲ. ಅರ್ಜುನನಿಗೇ ಇದು ದೋಷ ಎಂಬುದು ಗೊತ್ತಾಗಿದೆ. ಅದಕ್ಕಾಗಿಯೇ ಶ‍್ರೀಕೃಷ್ಣನಿಗೆ ‘ಕಾರ್ಪಣ್ಯದೋಷದಿಂದ ಹತಸ್ವಭಾವದವನಾಗಿದ್ದೇನೆ, ನೀನು ನಿಶ್ಚಯಿಸಿ ಹೇಳು ನನಗೆ ಯಾವುದು ಶ್ರೇಯಸ್ಕರವೊ ಅದನ್ನು; ಅದನ್ನು ಮಾಡಲು ನಾನು ಸಿದ್ಧನಾಗಿದ್ದೇನೆ’ ಎಂದು ಭಗವಂತನಲ್ಲಿ ಶರಣಾಗಿ ತನ್ನ ಜೀವನದ ಜವಾಬ್ದಾರಿಯನ್ನೆಲ್ಲ ಅವನಿಗೆ ವಹಿಸಿರುವುದು ಒಂದು ದೊಡ್ಡ ದೈವೀಗುಣ ಅರ್ಜುನನಲ್ಲಿ.

\begin{verse}
ದ್ವೌ ಭೂತಸರ್ಗೌ ಲೋಕೇಽಸ್ಮಿನ್ ದೈವ ಆಸುರ ಏವ ಚ~।\\ದೈವೋ ವಿಸ್ತರಶಃ ಪ್ರೋಕ್ತ ಆಸುರಂ ಪಾರ್ಥ ಮೇ ಶೃಣು \versenum{॥ ೬~॥}
\end{verse}

{\small ಅರ್ಜುನ, ಈ ಲೋಕದಲ್ಲಿ ದೈವೀ ಮತ್ತು ಆಸುರೀ ದೃಷ್ಟಿ ಎಂದು ಎರಡು ವಿಧಗಳು ಇವೆ. ದೈವೀ ದೃಷ್ಟಿಯನ್ನು ವಿಸ್ತಾರವಾಗಿ ವಿವರಿಸಿ ಆಯಿತು. ಇನ್ನು ಆಸುರೀ ವರ್ಣನೆಯನ್ನು ಕೇಳು.}

ಈ ಪ್ರಪಂಚದಲ್ಲಿ ಎರಡು ಸ್ವಭಾವದ ಜನರು ಇರುವರು. ಒಬ್ಬರು ದೈವೀ ಸ್ವಭಾವಕ್ಕೆ ಸೇರಿದವರು. ಮತ್ತೊಬ್ಬರು ಆಸುರೀ ಸ್ವಭಾವಕ್ಕೆ ಸೇರಿದವರು. ಯಾವಾಗಲೂ ಇದರಲ್ಲಿ ಎರಡನೆ ಸ್ವಭಾವಕ್ಕೆ ಸೇರಿದ ಜನರೇ ಜಾಸ್ತಿ. ಮೊದಲನೇ ಸ್ವಭಾವಕ್ಕೆ ಸೇರಿದವರ ಸಂಖ್ಯೆ ಯಾವಾಗಲೂ ವಿರಳವೇ. ಈ ಎರಡು ಗುಣಗಳನ್ನು ನಾವು ಹುಟ್ಟಿ ಆದಮೇಲೆ ಆರಿಸಿಕೊಳ್ಳವುದಿಲ್ಲ. ಹುಟ್ಟುವಾಗಲೇ ಇವುಗಳಲ್ಲಿ ಯಾವುದಾದರೂ ಒಂದು ಸ್ವಭಾವದೊಡನೆ ನಾವು ಹುಟ್ಟುತ್ತೇವೆ. ಅನಂತರ ಆ ಬೀಜಕ್ಕೆ ತಕ್ಕಂತೆ ಬೆಳೆಯುತ್ತಾ ಹೋಗುತ್ತೇವೆ. ಯಾವ ವಾತಾವರಣದಲ್ಲಿದ್ದರೂ ನಮಗೆ ಬೇಕಾದುದನ್ನು ಮಾತ್ರ ಹೀರಿಕೊಂಡು, ಬೇಡವಾದುದನ್ನು ಹಾಗೆಯೇ ಬಿಡುತ್ತೇವೆ.

ದೈವೀ ಸ್ವಭಾವವನ್ನು ಮೊದಲನೆ ಮೂರು ಶ್ಲೋಕಗಳಲ್ಲಿ ಹೇಳಿದ್ದಾಯಿತು. ಇದೇ ಪ್ರಥಮ ಬಾರಿ ಅಲ್ಲ ಅದು ಇಲ್ಲಿ ಬಂದಿರುವುದು. ಹಲವು ವೇಳೆ ಅದು ಬೇರೆ ಬೇರೆ ಹೆಸರಲ್ಲಿ ಬಂದಿದೆ; ಸ್ಥಿತಪ್ರಜ್ಞನ ಸ್ಥಿತಿಯನ್ನು ವಿವರಿಸುವಾಗ ಬಂದಿದೆ. ಸತ್ವಗುಣವನ್ನು ವಿವರಿಸುವಾಗ ಬಂದಿದೆ, ಗುಣಾತೀತನ ಸ್ಥಿತಿಯನ್ನು ವಿವರಿಸುವಾಗ ಬಂದಿದೆ, ಆದರೆ ಪುನರುಕ್ತಿ ದೋಷವಲ್ಲ ಗೀತೆಯಂತಹ ಶಾಸ್ತ್ರದಲ್ಲಿ. ಇದೊಂದು ಶ‍್ರೀಕೃಷ್ಣ-ಅರ್ಜುನರ ಸಂವಾದ ಗ್ರಂಥ. ಬೇರೆ ಬೇರೆ ಪ್ರಶ್ನೆಗಳನ್ನು ಕೇಳುತ್ತಿರುವಾಗ ಒಂದೇ ಪ್ರಶ್ನೆ ಹಲವು ವೇಳೆ ಬೇರೆ ಬೇರೆ ಆಕಾರವನ್ನು ತಾಳಿ ಏಳುವುದು. ಅದಕ್ಕೆ ಉತ್ತರವೂ ಕೂಡ ಹಾಗೆಯೆ. ಈ ಭಾವನೆಗಳು ಒಂದಲ್ಲದಿದ್ದರೆ ಮತ್ತೊಂದು ಸಲವಾದರೂ ನಮ್ಮ ಮನಸ್ಸಿನಲ್ಲಿ ಇಳಿಯುತ್ತವೆ.

ಇನ್ನು ಮೇಲೆ ಆಸುರೀಸ್ವಭಾವವನ್ನು ವಿವರಿಸುತ್ತೇನೆ ಎನ್ನುವನು. ಇದನ್ನು ಕೂಡ ಸಂಗ್ರಹವಾಗಿ ಇದೇ ಅಧ್ಯಾಯದ ನಾಲ್ಕನೆ ಶ್ಲೋಕದಲ್ಲಿ ದಂಭ, ದರ್ಪ, ಅಹಂಕಾರ ಮುಂತಾದ ಗುಣಗಳಿಂದ ವಿವರಿಸಿರುವನು. ಆದರೆ ಅದನ್ನೇ ಮುಂದೆ ವಿಸ್ತಾರವಾಗಿ ವಿವರಿಸುವನು. ಅದನ್ನು ತಿಳಿದುಕೊಂಡಿರು ವುದು ಮೇಲು. ಅನುಷ್ಠಾನ ಮಾಡುವುದಕ್ಕಲ್ಲ, ಅದನ್ನು ಮಾಡದೆ ಇರುವುದಕ್ಕೆ. ಅನೇಕ ವೇಳೆ ನಮಗೆ ಗೊತ್ತಿಲ್ಲದೆ ಇವುಗಳೆಲ್ಲ ನಮ್ಮಲ್ಲಿ ಅಂಟಿಕೊಂಡಿರುತ್ತವೆ. ಇದನ್ನು ತಿಳಿದುಕೊಂಡಾಗ ಇದನ್ನು ಕಿತ್ತುಹಾಕುವುದಕ್ಕೆ ನಮಗೆ ಸಹಾಯವಾಗುತ್ತದೆ. ಜೀವನದಲ್ಲಿ ನಾವು ದೈವೀ ಸಂಪತ್ತನ್ನು ಕಷ್ಟಪಟ್ಟು ಅಭ್ಯಾಸ ಮಾಡಬೇಕು. ಆಸುರೀಸಂಪತ್ತನ್ನು ಅದಕ್ಕಿಂತ ಹೆಚ್ಚು ಕಷ್ಟಪಟ್ಟು ಬಿಡುತ್ತ ಬರಬೇಕು.

\begin{verse}
ಪ್ರವೃತ್ತಿಂ ಚ ನಿವೃತ್ತಿಂ ಚ ಜನಾ ನ ವಿದುರಾಸುರಾಃ~।\\ನ ಶೌಚಂ ನಾಪಿ ಚಾಚಾರೋ ನ ಸತ್ಯಂ ತೇಷು ವಿದ್ಯತೇ \versenum{॥ ೭~॥}
\end{verse}

{\small ಆಸುರೀ ಸ್ವಭಾವದ ಜನಕ್ಕೆ ಪ್ರವೃತ್ತಿ ಎಂದರೇನು, ನಿವೃತ್ತಿ ಎಂದರೇನು ಎಂಬುದು ಗೊತ್ತಿಲ್ಲ. ಅವರಲ್ಲಿ ಶೌಚವಿಲ್ಲ, ಆಚಾರವಿಲ್ಲ, ಸತ್ಯವೂ ಇಲ್ಲ.}

ನಮ್ಮ ಶಾಸ್ತ್ರಗಳು ಎರಡು ಮಾರ್ಗವನ್ನು ನಮ್ಮ ಮುಂದೆ ಇಡುತ್ತವೆ. ಅದೇ ಪ್ರವೃತ್ತಿಮಾರ್ಗ ಮತ್ತು ನಿವೃತ್ತಿ ಮಾರ್ಗ. ಒಂದು ಮಾಡುವುದು, ಮತ್ತೊಂದು ಬಿಡುವುದು. ಅವೆರಡನ್ನೂ ಶಾಸ್ತ್ರೀಯವಾಗಿ ತಿಳಿದುಕೊಂಡಿರಬೇಕು. ಪ್ರಪಂಚವನ್ನು ಅನುಭವಿಸಬೇಕೆಂಬ ಆಸೆ ಬಲವಾಗಿದ್ದರೆ ಬಿಡು ಎನ್ನುವುದಿಲ್ಲ. ಅನುಭವಿಸು, ಆದರೆ ಅದನ್ನು ಧಾರ್ಮಿಕವಾಗಿ ಅನುಭವಿಸು ಎಂದು ಹೇಳು ವುದು. ಅದಕ್ಕಾಗಿ ಮಾನವನ ಮುಂದೆ ನಾಲ್ಕು ಪುರುಷಾರ್ಥಗಳನ್ನು ಇಟ್ಟಿರುವರು. ಅದೇ ಧರ್ಮ, ಅರ್ಥ, ಕಾಮ, ಮೋಕ್ಷ. ಇಲ್ಲಿ ಕಾಮ ಮತ್ತು ಅರ್ಥವನ್ನು ಧಾರ್ಮಿಕವಾಗಿ ಸಂಪಾದಿಸಬೇಕು, ಧರ್ಮಕ್ಕೆ ವಿರೋಧವಾಗಿ ಹೋಗಿ ಅದನ್ನು ಅನುಭವಿಸಕೂಡದು. ನಮ್ಮಲ್ಲಿ ಹಲವು ಆಸೆಗಳಿವೆ. ಅವನ್ನು ನ್ಯಾಯವಾಗಿ ಒಂದು ವಸ್ತುವನ್ನು ಪಡೆದು ಅನುಭವಿಸಿ ಅದರಿಂದ ಬರುವ ಪರಿಣಾಮ ಗಳಿಗೂ ಸಿದ್ಧರಾಗಿರಬೇಕು. ಒಬ್ಬನಿಗೆ ಕಾಮವಿದೆ. ಹೇಗೊ ಮೃಗದಂತೆ ಅದನ್ನು ತೃಪ್ತಿಮಾಡಿಕೊಳ್ಳದೆ ಒಂದು ವ್ಯಕ್ತಿಯನ್ನು ಧಾರ್ಮಿಕವಾಗಿ ಕೂಡಿ, ನಮ್ಮ ಚಪಲವನ್ನು ತೃಪ್ತಿಪಡಿಸಿಕೊಂಡು ಅದರಿಂದ ಬರುವ ಪರಿಣಾಮಗಳನ್ನೂ ಅನುಭವಿಸಲು ಅಣಿಯಾಗಬೇಕು. ಹಣ ಸಂಪಾದಿಸುವುದಕ್ಕೆ ಆಸೆ ಇದ್ದರೆ, ಅಡ್ಡ ಹಾದಿಯನ್ನು ಹಿಡಿಯದೆ ಯಾವುದಾದರೂ ಕಸಬನ್ನು ಹಿಡಿದು ಕಷ್ಟಪಟ್ಟು ಕೆಲಸ ಮಾಡಿ ಹಣ ಸಂಪಾದನೆ ಮಾಡಿ ಅನುಭವಿಸಲಿ. ಈ ದಾರಿಯಲ್ಲಿ ಶ‍್ರೀಮಂತನಾಗುವುದಕ್ಕೆ ಸಾಧ್ಯ. ಆದರೆ ಮನುಷ್ಯನಿಗೆ ತಾಳ್ಮೆಯಿಲ್ಲ. ತಕ್ಷಣ ಶ‍್ರೀಮಂತನಾಗಬೇಕೆಂದು ಬಯಸಿ ಅಡ್ಡಹಾದಿಗಳನ್ನೇ ಹಿಡಿಯು ವನು. ಅಧರ್ಮವನ್ನು ಆಶ್ರಯಿಸಿ ಹಣ ಗಳಿಸುವನು. ಆದರೆ ಜೀವನದಲ್ಲಿ ನ್ಯಾಯವಾಗಿ ಸಂಪಾದನೆ ಮಾಡಿರುವುದನ್ನೇ ಅರಗಿಸಿಕೊಳ್ಳುವುದು ಕಷ್ಟ. ಇನ್ನು ಅನ್ಯಾಯವಾಗಿ ಸಂಪಾದನೆ ಮಾಡಿರುವುದನ್ನು ಅನುಭವಿಸಬೇಕಾದರೆ ಇನ್ನೂ ಕಷ್ಟ. ಯಾವ ಸಮಯದಲ್ಲಿ ಸಿಕ್ಕಿಕೊಳ್ಳುತ್ತೇನೆಯೋ ಎಂಬ ಅಂಜಿಕೆ ಕಳವಳ ಯಾವಾಗಲೂ ಕಾಡುತ್ತಿರುವುದು. ಕಳ್ಳರಂತೆ ರೋಗದಂತೆ ಆಕಸ್ಮಿಕದಂತೆ, ಕೋರ್ಟು ಕಚೇರಿಯ ಮೂಲಕ ಅನ್ಯಾಯದ ದುಡ್ಡು ಬಹುಪಾಲು ಹೋಗುವುದು. ಇಂತಹ ಆಸ್ತಿಯನ್ನು ಮಕ್ಕಳಿಗೆ ಬಿಟ್ಟಾಗ ಅವರು ದುರ್ಮಾರ್ಗಿಗಳಾಗುವರು. ಇದು ಅನ್ಯಾಯದ ಹಾದಿ. ಇದನ್ನು ಅನುಸರಿಸಕೂಡದು. ಆಸುರೀ ಸ್ವಭಾವದ ಮನುಷ್ಯನಿಗೆ ಹೇಗೆ ಅನುಭವಿಸಬೇಕು, ಅದರ ರಹಸ್ಯವೇನು ಎಂಬುದು ಗೊತ್ತಿಲ್ಲ. ಅವನಿಗೆ ಆತುರ. ವಿಚಾರ ಮಾಡುವುದಕ್ಕೆ ಅವನಿಗೆ ತಾಳ್ಮೆಯಿಲ್ಲ. ಅನುಭವಿ ಸುವುದೊಂದು ಕಲೆ. ಅದನ್ನು ಚೆನ್ನಾಗಿ ಅನುಭವಿಸಬೇಕಾದರೆ ಅದರ ಮರ್ಮವನ್ನು ಅರಿತಿರಬೇಕು. ಹಲಸಿನ ಹಣ್ಣನ್ನು ಬಿಡಿಸಬೇಕಾದರೆ ಕೈಗೆ ಎಣ್ಣೆ ಸವರಿಕೊಂಡು ಬಿಡಿಸಿ ತಿನ್ನು ಎಂದು ಹೇಳುತ್ತಿದ್ದರು ಶ‍್ರೀರಾಮಕೃಷ್ಣರು. ಆದರೆ ನಾವು ಕೈಗೆ ಏನೂ ಹಚ್ಚಿಕೊಳ್ಳುವುದಿಲ್ಲ. ಸುಮ್ಮನೆ ಹೆಚ್ಚುವೆವು, ತಿನ್ನುವೆವು. ಕೈಗೆ ಬಾಯಿಗೆಲ್ಲಾ ಅಂಟು ಮೆತ್ತಿಕೊಳ್ಳುವುದು. ತಿಂದಾದ ಮೇಲೂ ಈ ಅಂಟು ನಮ್ಮನ್ನು ಬಿಡುವುದಿಲ್ಲ. ಅವರಿಗೆ ಪ್ರವೃತ್ತಿ ಗೊತ್ತಿಲ್ಲ, ಎಂದರೆ ಹೇಗೆ ಪುರುಷಾರ್ಥಗಳನ್ನು ಸಂಪಾದಿಸಬೇಕು ಎನ್ನುವುದು ಗೊತ್ತಿಲ್ಲ.

ಅವನಿಗೆ ನಿವೃತ್ತಿಯೂ ಗೊತ್ತಿಲ್ಲ. ಜೀವನದಲ್ಲಿ ಕೆಲವನ್ನು ಎಲ್ಲರೂ ಬಿಡಬೇಕು. ಯಾವುದನ್ನು ಬಿಡಬೇಕು, ಯಾವುದನ್ನು ಮಾಡಬೇಕು ಎಂಬುದು ಅವನಿಗೆ ಗೊತ್ತಿಲ್ಲ. ಅದೊಂದೇ ಅಲ್ಲ, ಅವನು ಅನೇಕ ವೇಳೆ ಬಿಡಬೇಕಾದುದನ್ನು ಮಾಡುವನು, ಮಾಡಬೇಕಾದುದನ್ನು ಬಿಡುವನು. ಕೇವಲ ಸುಖವನ್ನು ಭೋಗಿಸುವ ಆತುರ ಒಂದೆ ಅವನ ಮುಂದಿರುವುದು. ನಾಯಿಯ ಮುಂದೆ ಏನೋ ಇದೆ. ಅದು ಸ್ವಲ್ಪವೂ ಆಲೋಚಿಸುವುದಿಲ್ಲ. ಗಬಗಬ ಎಂದು ತಿಂದುಹಾಕುವುದು. ನಿಧಾನ ಮಾಡಿದರೆ ಇತರ ನಾಯಿಗಳು ಬಂದು ಎಲ್ಲಿ ತಿಂದುಹಾಕುವುವೋ ಎಂಬ ಭಯ ಅದಕ್ಕೆ. ಆಸುರೀ ಸ್ವಭಾವದ ಮನುಷ್ಯನೂ ಹಾಗೆಯೆ.

ಅವನಲ್ಲಿ ಶುಚಿ ಇಲ್ಲ. ಹೊರಗೆ ಬಟ್ಟೆ ಬರೆ ಮುಂತಾದುವೆಲ್ಲ ಶುಚಿಯಾಗಿರುವುದಿಲ್ಲ. ತಿನ್ನುವ ವಿಷಯದಲ್ಲಿ ಶುಚಿಯಾಗಿರುವುದಿಲ್ಲ. ದೇಹ ಶುಚಿಯಾಗಿರಬೇಕು, ಇದಕ್ಕಿಂತ ಹೆಚ್ಚಾಗಿ ಮನಸ್ಸು ಶುಚಿಯಾಗಿರಬೇಕು. ಇವೆರಡನ್ನೂ ಅವನು ಅಲ್ಲಗಳೆಯುವನು.

ಅವನಲ್ಲಿ ಒಳ್ಳೆಯ ಆಚಾರವೂ ಇಲ್ಲ, ನಡತೆಯೂ ಇಲ್ಲ. ಅವನು ಮಾತುಕತೆ ನಡವಳಿಕೆ ಇವುಗಳನ್ನು ಶುದ್ಧವಾಗಿಟ್ಟುಕೊಂಡಿರುವುದಿಲ್ಲ. ಒಳಗೊಂದು ಹೊರಗೊಂದು, ಹೇಳುವುದೊಂದು ಮಾಡುವುದೊಂದು. ಜೀವನದಲ್ಲಿ ಅವನ ಮಾತಿಗೂ ನಡತೆಗೂ ತಾಳೆಯಾಗುವುದಿಲ್ಲ.

ಅವನಲ್ಲಿ ಸತ್ಯವೂ ಇಲ್ಲ. ಸುಮ್ಮನೇ ಮಾತನಾಡುತ್ತಾನೆ. ಸತ್ಯವನ್ನೇ ಹೇಳಬೇಕು ಎಂಬ ನಿಯಮವನ್ನು ಅವನು ಅನುಸರಿಸುವುದಿಲ್ಲ. ಅವನ ಮಾತಿನಲ್ಲಿ ಮುಚ್ಚು, ಮರೆ, ಸುಳ್ಳು ಇವೇ ತುಂಬಿರುತ್ತವೆ. ಹೇಗೋ ಒಂದು ಪರಿಸ್ಥಿತಿಯಿಂದ ಅವನು ತಪ್ಪಿಸಿಕೊಳ್ಳುವುದಕ್ಕೆ ಏನು ಮಾಡ ಬೇಕೋ ಅದನ್ನು ಮಾಡುವನು. ಅದರಿಂದ ಪಾರಾಗುವುದಿಲ್ಲ. ಅದನ್ನು ಮುಂದೂಡುವನು. ಸಾಲ ಕೊಡಬೇಕಾದಾಗ, ವಸೂಲಿ ಮಾಡುವುದಕ್ಕೆ ಬಂದರೆ, ನಾಳೆ ನಿಮ್ಮ ಮನೆಗೇ ಬಂದು ಕೊಟ್ಟು ಹೋಗುತ್ತೇನೆ ಎನ್ನುವನು. ಆ ನಾಳೆಯಂತೂ ಬರುವ ಹಾಗಿಲ್ಲ!

\begin{verse}
ಅಸತ್ಯಮಪ್ರತಿಷ್ಠಂ ತೇ ಜಗದಾಹುರನೀಶ್ವರಮ್~।\\ಅಪರಸ್ಪರಸಂಭೂತಂ ಕಿಮನ್ಯತ್ ಕಾಮಹೈತುಕಮ್ \versenum{॥ ೮~॥}
\end{verse}

{\small ಈ ಜಗತ್ತು ಅಸತ್ಯ. ಇದಕ್ಕೆ ಆಧಾರವಿಲ್ಲ. ದೇವರೂ ಇಲ್ಲ. ಕೆಲವು ಸ್ತ್ರೀ ಪುರುಷರ ಕಾಮದಿಂದ ಇದು ಹುಟ್ಟಿದೆ ಎನ್ನುತ್ತಾನೆ. ಕಾಮವೊಂದೇ ಸೃಷ್ಟಿಗೆ ಕಾರಣ ಎನ್ನುವರು ಅವರು.}

ಜಡವಾದಿಗಳ ತತ್ತ್ವ ಇದು. ಈ ಪ್ರಪಂಚದಲ್ಲಿ ಎಲ್ಲರಿಗೂ ಒಂದು ತತ್ತ ್ವ ಇದೆ. ನಿರೀಶ್ವರವಾದಿ, ಅಥವಾ ಚಾರ್ವಾಕ ಎಂದರೆ ಅವನಿಗೆ ತತ್ತ್ವ ಇಲ್ಲ ಎಂದು ಅರ್ಥವಲ್ಲ. ಅವನದೇ ಒಂದು ಬೇರೆ ತತ್ತ್ವ. ಅವನೂ ಕೂಡ ನಾವೆಲ್ಲ ಉಪಯೋಗಿಸುವ ವಿಚಾರವನ್ನೇ ತೆಗೆದುಕೊಳ್ಳುತ್ತಾನೆ. ಅದರ ಮೂಲಕವೇ ವಿವರಿಸುತ್ತಾನೆ. ವಿಚಾರ ಎಲ್ಲೊ ಕೆಲವು ಭಗವದ್ಭಕ್ತರಿಗೆ ಮಾತ್ರ ಮೀಸಲಲ್ಲ. ಅವರಿಗಿಂತ ಹೆಚ್ಚಾಗಿ ದೇವರನ್ನು ನಂಬದವರು ವಿಚಾರವನ್ನು ಉಪಯೋಗಿಸುತ್ತಾರೆ. ಭಗವಂತನನ್ನು ನಂಬುವವನಿಗೆ, ಅವನನ್ನು ಅನುಭವಿಸುವವನಿಗೆ ವಿಚಾರವೇ ಬೇಕಿಲ್ಲ. ಅದು ಅನಾವಶ್ಯಕ. ಇತರರಿಗೆ ತಾನೆ ಅದು ಅತ್ಯಂತ ಆವಶ್ಯಕವಾಗಿರುವುದು.

ಆಸುರೀ ಸ್ವಭಾವದ ಮನುಷ್ಯ ಹೇಳುತ್ತಾನೆ, ಈ ಪ್ರಪಂಚ ಅಸತ್ಯ ಎಂದು. ಇಲ್ಲಿ ಅಸತ್ಯ ಎಂಬುದನ್ನು ದೊಡ್ಡ ವೇದಾಂತಿಗಳಂತೆ ಉಪಯೋಗಿಸುವುದಿಲ್ಲ. ಈ ಪ್ರಪಂಚ ನಮಗೆ ಕಾಣು ತ್ತಿರುವ ರೀತಿ ಸ್ಥಿರವಲ್ಲ. ದೊಡ್ಡ ಪಂಚಭೂತಗಳ ಮಿಶ್ರದಿಂದ ಬದಲಾವಣೆ ಆಗುತ್ತಿದೆ. ಹಾಗೆ ಬದಲಾವಣೆಗೆ ಸಿಕ್ಕಿ ಹೋಗುವಾಗ ಕೆಲವು ಗ್ರಹಗಳು ತಂಪಾಗಿ, ನಮ್ಮ ಭೂಮಿಯಂತಹ ಗ್ರಹವಾಗಿ, ಹಲವು ಕೋಟಿ ವರ್ಷಗಳಾದ ಮೇಲೆ ಸಸ್ಯ, ಪ್ರಾಣಿ, ಮನುಷ್ಯ ಎಲ್ಲ ಬರುವುವು. ಕೆಲಕಾಲದ ಮೇಲೆ ವಾತಾವರಣ ಬದಲಾವಣೆಯಾಗಿ ಇವುಗಳೆಲ್ಲ ನಿರ್ನಾಮವಾಗಿಬಿಡುವುವು. ಈ ಭೂಮಿಯ ಆಯುಸ್ಸಿನ ಮಧ್ಯಭಾಗದಲ್ಲಿ ಎಲ್ಲೊ ಕೆಲವು ದಿನ, ತಿರೆ ಹಸಿರಿನಿಂದ ಕೂಡಿರುವುದು, ಹಕ್ಕಿಗಳು ಹಾಡುವುವು, ಪ್ರಾಣಿಗಳು ಸಂಚರಿಸುವುವು, ಮನುಷ್ಯನೇಳುವನು, ಅವನಾಡುವನು. ಕೆಲವು ಕಾಲದ ಮೇಲೆ ಭೂಮಿ, ಪುನಃ ಬದಲಾಯಿಸುವುದು. ಈ ಚೇತನವೆಲ್ಲ ನಿರ್ನಾಮವಾಗಿ ಹೋಗುವುದು. ಬರೀ ಜಡವೊಂದೇ ಈ ಭೂಮಿಯಲ್ಲಿ ಸತ್ಯ. ಚೇತನ ಆಕಸ್ಮಿಕ.

ಪ್ರತಿಷ್ಠೆ ಇಲ್ಲದ್ದು. ಈ ಜಗತ್ತಿಗೆ ಯಾವ ಆಧಾರವೂ ಇಲ್ಲ. ಪುತು ಧರ್ಮ ಮುಂತಾದವುಗಳಾ ವುವೂ ಇಲ್ಲ. ಇದು ಬರೀ ಜಡ, ಸುಮ್ಮನೆ ಸುತ್ತುತ್ತಿದೆ. ಏಕೆ ಸುತ್ತುತ್ತಿದೆ ಎಂದರೆ, ಇದು ಪ್ರಕೃತಿಯ ಸ್ವಭಾವ, ಅದಕ್ಕಾಗಿ ಸುತ್ತುತ್ತಿದೆ. ಇದನ್ನು ಯಾವ ದೇವರೂ ಸೃಷ್ಟಿಸಬೇಕಾಗಿಲ್ಲ. ಇದು ತನಗೆ ತಾನೇ ಇದೆ. ದೇವರು ಧರ್ಮ ಕರ್ಮ ಇವೆಲ್ಲ ಅನಾವಶ್ಯಕ. ಇವುಗಳೆಲ್ಲ ಮನುಷ್ಯನ ಭ್ರಾಂತಿ. ಮನುಷ್ಯನ ಮನುಸ್ಸು ತೆಪ್ಪಗಿರಲಾರದು. ಏನಾದರೂ ಜಾಲವನ್ನು ನೆಯ್ಯುವುದು. ಅನಂತರ ಪ್ರಪಂಚದ ಮೇಲೆ ಆರೋಪ ಮಾಡುವುದು.

ಈ ಪ್ರಪಂಚವನ್ನು ಆಳುವ ದೇವರು ಯಾರೂ ಇಲ್ಲ. ಇದು ತನಗೆ ತಾನೇ ಇದೆ. ಪ್ರತಿ ಯೊಂದನ್ನೂ ಯಾವನೋ ಒಬ್ಬ ಸೃಷ್ಟಿಸಿರಬೇಕು ಎಂಬುದು ಮನುಷ್ಯನ ಒಂದು ದೌರ್ಬಲ್ಯ. ಕಲ್ಲು ಮಣ್ಣು ಗಾಳಿ ಬೆಳಕು ಇವೆಲ್ಲ ಬಿದ್ದಿದೆ, ವಿಕಾಸವಾಗುತ್ತ ಆಗುತ್ತ ಕೆಲವು ಕಾಲದ ಮೇಲೆ ಸಸ್ಯ, ಪಶು, ಪಕ್ಷಿ, ಮನುಷ್ಯ ಇವರೆಲ್ಲ ಏಳುತ್ತಾರೆ. ತಮ್ಮ ಕಾಯದ ಬಯಕೆಯನ್ನು ತೃಪ್ತಿಪಡಿಸಿಕೊಳ್ಳವುದಕ್ಕಾಗಿ, ಸ್ತ್ರೀ ಪುರುಷರು ಒಬ್ಬರನ್ನೊಬ್ಬರು ಸಂಧಿಸುತ್ತಾರೆ. ಅದರಿಂದ ಮಕ್ಕಳು ಮರಿಗಳಾಗುತ್ತವೆ. ಇಲ್ಲಿ ದೇವರನ್ನು ತರುವುದು ಅನಾವಶ್ಯಕ. ದೇವರಿಲ್ಲದೆ ಇವುಗಳನ್ನೆಲ್ಲ ವಿವರಿಸಬಹುದು. ಮನುಷ್ಯ ದುರ್ಬಲನಾಗಿರುವಾಗ, ಇನ್ನೂ ಮೌಢ್ಯತೆಯಲ್ಲಿ ಸಿಕ್ಕಿ ನರಳುತ್ತಿರುವಾಗ, ಅವನು ದೇವರು ಎಂಬ ಊರುಗೋಲು ಹಿಡಿದಿದ್ದ. ಆದರೆ ಪ್ರಾಪ್ತ ವಯಸ್ಕನಾದ ಮೇಲೆ ಅದು ಅನಾವಶ್ಯಕವಾಗುವುದು. ದೇವರನ್ನು ನೋಡಿದವರಾರೂ ಇಲ್ಲ. ಇದೆಲ್ಲ ಪಿತ್ತ ನೆತ್ತಿಗೆ ಏರಿದಾಗ ಆಡುವ ಮಾತುಗಳು. ಇದು ಒಂದು ವಿಧವಾದ ಸನ್ನಿ. ಮೆದುಳಿನ ದೌರ್ಬಲ್ಯದ ಚಿಹ್ನೆಗಳು. ನಾನಿದ್ದೇನೆ ನೀರಿನ ಗುಳ್ಳೆಯಂತೆ. ನನ್ನ ಹಿಂದಿಲ್ಲ, ಮುಂದಿಲ್ಲ, ಇರುವಾಗ ಸುತ್ತಮುತ್ತಲ ವಸ್ತುವನ್ನು ಪಡೆದುಕೊಂಡು ಸುಖಪಡೋಣ. ಅನಂತರ ನಾವಿಲ್ಲ. ತಡ ಮಾಡಿದರೆ ಎದುರಿಗೆ ಇರುವ ವಸ್ತುವೂ ನುಸುಳಿ ಹೋಗುವುದು. ಇದೇ ಅವನ ತತ್ತ್ವ.

\begin{verse}
ಏತಾಂ ದೃಷ್ಟಿಮವಷ್ಟಭ್ಯ ನಷ್ಟಾತ್ಮಾನೋಽಲ್ಪಬುದ್ಧಯಃ~।\\ಪ್ರಭವಂತ್ಯುಗ್ರಕರ್ಮಾಣಃ ಕ್ಷಯಾಯ ಜಗತೋಽಹಿತಾಃ \versenum{॥ ೯~॥}
\end{verse}

{\small ಈ ಮತವನ್ನು ಆಶ್ರಯಿಸಿಕೊಂಡು ನಷ್ಟಾತ್ಮರೂ, ಅಲ್ಪಬುದ್ಧಿಗಳೂ, ಕ್ರೂರ ಕರ್ಮಿಗಳೂ ಲೋಕ ಶತ್ರುಗಳೂ ಆದ ಆಸುರೀ ಸ್ವಭಾವದ ಜನರು ಜಗತ್ತಿನ ನಾಶಕ್ಕೆ ಕಾರಣರಾಗುತ್ತಾರೆ.}

ಯಾರು ಜೀವನದಲ್ಲಿ ಈ ದೃಷ್ಟಿಯನ್ನು ಸ್ವೀಕರಿಸಿರುವರೊ ಅವರು ತಮ್ಮ ಆತ್ಮನನ್ನು ಕಳೆದುಕೊಳ್ಳುತ್ತಾರೆ. ಇದನ್ನು ಅಕ್ಷರಶಃ ತೆಗೆದುಕೊಳ್ಳಕೂಡದು. ಆತ್ಮನನ್ನು ಯಾರೂ ಕಳೆದುಕೊಳ್ಳ ಲಾಗುವುದಿಲ್ಲ. ಅದು ಎಂದೆಂದಿಗೂ ಇರುವುದು. ನಮಗೆ ಗೊತ್ತಿಲ್ಲದೆ ಇದ್ದರೂ ಅದು ಇದ್ದೇ ತೀರುವುದು. ಆದರೆ ಯಾರು ಈ ದೃಷ್ಟಿಯನ್ನು ಅನುಸರಿಸುವರೊ ಅವರಿಗೆ ಆತ್ಮ ದೊರಕುವುದಿಲ್ಲ. ಆತ್ಮನನ್ನು ಕಾಣದಂತೆ ಮಾಡಿರುವ ದೇಹ ಇಂದ್ರಿಯ ಇವುಗಳು ದೊರಕುತ್ತವೆ. ಇದೇ ಸತ್ಯವೆಂದು ಭಾವಿಸಿ ಇದನ್ನು ತೃಪ್ತಿಪಡಿಸಲು ಯತ್ನಿಸುವರು. ಇದರಿಂದ ಆತ್ಮವಿದ್ದರೂ ಅದು ಕಾಣುವುದಿಲ್ಲ. ಅವರ ಪಾಲಿಗೆ ಅದು ಇಲ್ಲದಂತೆಯೆ.

ಆಸುರೀ ಸ್ವಭಾವದ ಜನರು ಅಲ್ಪ ಬುದ್ಧಿಗಳು. ಅವರು ತಮ್ಮ ಇಂದ್ರಿಯಕ್ಕೆ ಯಾವುದು ಕಾಣುವುದೋ ಅದನ್ನು ಮಾತ್ರ ತೆಗೆದುಕೊಳ್ಳುತ್ತಾರೆ. ಅವರಿಗೆ ಇನ್ನೂ ತಮ್ಮ ಇಂದ್ರಿಯಗಳ ಮಿತಿ ಗೊತ್ತಾಗಿಲ್ಲ. ಈ ಪ್ರಪಂಚದಲ್ಲಿ ನಮ್ಮ ಇಂದ್ರಿಯವೆಂಬ ಬಲೆ ಎಲ್ಲವನ್ನೂ ಹಿಡಿಯಲಾರದು. ಎಲ್ಲೋ ಒಂದು ಸ್ವಲ್ಪ ಅದರ ಬಲೆಗೆ ಬೀಳುವುದು. ಉಳಿದವು ಅದನ್ನು ಲೆಕ್ಕಿಸುವುದೇ ಇಲ್ಲ. ಇವರ ದೃಷ್ಟಿ ತಾತ್ಕಾಲಿಕ. ತಮಗೆ ಈಗ ಯಾವುದು ತೃಪ್ತಿ ಕೊಡುವುದೊ ಅದೊಂದೆ ಸತ್ಯ ಎಂದು ಭಾವಿಸುವರು. ಅದು ಅನಂತರ ಯಾವ ಪ್ರತಿಕ್ರಿಯೆಯನ್ನು ತರುವುದು ಎಂಬುದನ್ನು ಕುರಿತು ಆಲೋಚಿಸುವುದಿಲ್ಲ. ತಮ್ಮ ಮುಂದಿನ ಮಾರುದ್ದ ಮಾತ್ರ ಅವರಿಗೆ ಕಾಣುವುದು.

ಇವರು ಕ್ರೂರಕರ್ಮಿಗಳು. ತಮ್ಮ ಇಂದ್ರಿಯವನ್ನು ತೃಪ್ತಿಪಡಿಸಿಕೊಳ್ಳುವುದಕ್ಕೆ ಇತರರಿಗೆ ಯಾವ ಕಷ್ಟನಷ್ಟಗಳನ್ನಾದರೂ ಕೊಡಲು ಸಿದ್ಧವಾಗಿರುವರು. ಯಾವಾಗ ಅವರು ದೇವರನ್ನು ನಂಬುವು ದಿಲ್ಲವೋ, ಕರ್ಮದಲ್ಲಿ ನಂಬುವುದಿಲ್ಲವೋ, ಅವರನ್ನು ಯಾವುದೂ ನಿಗ್ರಹಿಸಲಾರದು. ಸಮಾಜ ಸೃಷ್ಟಿಸುವ ಪೋಲೀಸು ನ್ಯಾಯಾಲಯ ಇವುಗಳೇ ಸಾಲವು ಮಾನವನನ್ನು ದಂಡಿಸುವುದಕ್ಕೆ. ಇವುಗಳ ಕಣ್ಣಿಗೆ ಮಣ್ಣೆರಚುವುದು ಸುಲಭ. ಇಂತಹ ವ್ಯಕ್ತಿಗಳು ಲೋಕಕಂಟಕರಾಗುತ್ತಾರೆ. ಸಮಾಜಕ್ಕೆ ವಿರೋಧವಾಗಿ ಹೋಗುತ್ತಾರೆ. ತಮ್ಮ ಸ್ವಾರ್ಥ ತಮ್ಮ ಸುಖ ಒಂದೇ ಅವರಿಗೆ ಜೀವನದಲ್ಲಿ ಬೇಕಾಗಿರುವುದು. ಯಾವಾಗ ಸಮಾಜಕ್ಕೆ ಕೊಡದೇ ಅದರಿಂದ ಬರುವುದನ್ನೆಲ್ಲ ಹೀರಲು ಯತ್ನಿಸು ವರೊ ಆಗ ಅಂತಹ ಸಮಾಜ ಬೇಗ ಹಾಳಾಗುವುದು. ನಿಜವಾಗಿಯೂ ಲೋಕ ಶತ್ರುಗಳು ಇಂತಹವರೆ. ದೇಹದಲ್ಲಿ ಕ್ಯಾನ್ಸರ್ ಕ್ರಿಮಿ ಹೋಗೊ ಹಾಗೆ. ಕ್ಯಾನ್ಸರ್ ಕ್ರಿಮಿ ದೇಹದಲ್ಲಿರುವುದನ್ನೆಲ್ಲ ಹೀರುವುದು. ಕೊನೆಗೆ ದೇಹ ನಾಶವಾದರೆ ಇದಕ್ಕೆ ಆಹಾರವನ್ನು ಸರಬರಾಜು ಮಾಡುವವರಿಲ್ಲದೆ ಇದೇ ನಾಶವಾಗುವುದು. ಇದು ತನ್ನ ನಾಶವನ್ನು ತಾನೇ ಮಾಡಿಕೊಂಡಿತು. ಹಾಗೆಯೇ ಆಸುರೀ ಸ್ವಭಾವದ ಜನರು. ದೊಡ್ಡ ಸಮಾಜದಲ್ಲಿ ಇಂತಹ ಕೆಲವು ವ್ಯಕ್ತಿಗಳಿದ್ದರೆ ಹೇಗೋ ಸಹಿಸಿಕೊಂಡು ಹೋಗುವುದು. ಆದರೆ ಇಂತಹವರೇ ಹೆಚ್ಚಾಗಿದ್ದರೆ ಬಹಳ ಬೇಗ ನಾಶವಾಗುವುದು ಆ ದೇಶ. ಒಂದು ದೇಶದಲ್ಲಿರುವವರೆಲ್ಲ ಕಳ್ಳರಾದರೆ ಕದಿಯುವುದು ಯಾರ ಮನೆಯನ್ನು? ಎಲ್ಲರೂ ಲೂಟಿ ಮಾಡಲು ಶುರು ಮಾಡಿದರೆ ಇನ್ನು ಸಂಪಾದನೆ ಮಾಡುವವರು ಯಾರು?

\begin{verse}
ಕಾಮಮಾಶ್ರಿತ್ಯ ದುಷ್ಪೂರಂ ದಂಭಮಾನಮದಾನ್ವಿತಾಃ~।\\ಮೋಹಾದ್ಗೃಹೀತ್ವಾ ಸದ್ಗ್ರಾಹಾನ್ ಪ್ರವರ್ತಂತೇಽಶುಚಿವ್ರತಾಃ \versenum{॥ ೧೦~॥}
\end{verse}

{\small ತೃಪ್ತಿಪಡಿಲಾಗದ ಕಾಮವನ್ನು ಆಶ್ರಯಿಸಿ ದಂಭ ಮಾನ ಮದಗಳಿಂದ ಕೂಡಿದವರಾಗಿ ಮಿಥ್ಯಾವಸ್ತುಗಳನ್ನು ಹಿಡಿದು ದುಷ್ಟ ಇಚ್ಛೆಗಳನ್ನು ಆಶ್ರಯಿಸಿ ವರ್ತಿಸುತ್ತಾರೆ.}

ಆಸುರೀ ಪ್ರವೃತ್ತಿಯವರು ಎಂದಿಗೂ ಯಾರೂ ತೃಪ್ತಿಪಡಿಸುವುದಕ್ಕೆ ಆಗಲಾರದ ಆಸೆಯನ್ನು ತೃಪ್ತಿಪಡಿಸಲೆತ್ನಿಸುವರು. ಅದಕ್ಕೆ ಎಷ್ಟನ್ನು ಕೊಟ್ಟರೂ ಇನ್ನೂ ಕೊಡು ಎಂದು ಹೇಳುವುದೇ ಹೊರತು ಅದೆಂದಿಗೂ ಸಾಕು ಎನ್ನುವುದಿಲ್ಲ. ಅದನ್ನು ಹೆಚ್ಚು ತೃಪ್ತಿಪಡಿಸಿದಷ್ಟೂ ಅದಕ್ಕೆ ಸಂಬಂಧಪಟ್ಟ ನಮ್ಮಲ್ಲಿರುವ ವಾಸನೆ ವೃದ್ಧಿಯಾಗುವುದೇ ಹೊರತು ಅದೆಂದಿಗೂ ಕುಗ್ಗುವುದಿಲ್ಲ. ಇದನ್ನು ಅವರು ಆಲೋಚನೆ ಮಾಡುವುದಿಲ್ಲ. ಮುಂದಾಲೋಚನೆ, ದೀರ್ಘಾಲೋಚನೆ ಇವು ಅವರ ಪ್ರವೃತ್ತಿಗೆ ವಿರೋಧ. ಪ್ರಾಣಿಗಳ ಪ್ರತಿಕ್ರಿಯೆ ಹೇಗೋ ಹಾಗೆ ಅವರು. ಇಂದ್ರಿಯದ ಮುಂದೆ ಏನಿದ್ದರೆ ಅದನ್ನು ಅನುಭವಿಸಬೇಕೆಂಬುದು ಅದರ ಪ್ರಥಮ ಪ್ರತಿಕ್ರಿಯೆ. ಅದನ್ನೇ ಅವರು ಅನುಸರಿಸುವುದು.

ಅವನಿಗೆ ಧರ್ಮವೇ ಇಲ್ಲ. ಒಂದು ವೇಳೆ ಇದ್ದರೂ ಅದು ಢಂಬಾಚಾರಕ್ಕೆ, ಇತರರಿಗೆ ತೋರುವುದುಕ್ಕೇ ಹೊರತು ತನ್ನ ಆತ್ಮತೃಪ್ತಿಗಲ್ಲ. ಅವನು ಧರ್ಮಾತ್ಮನ ಸೋಗನ್ನು ಹಾಕುತ್ತಾನೆಯೆ ಹೊರತು ಧರ್ಮಾತ್ಮನ ಭಾವನೆಗಳು ಅವನಲ್ಲಿ ಇರುವುದಿಲ್ಲ. ಅವನು ಯಾರನ್ನು ಬೇಕಾದರೂ ಹಂಗಿಸುತ್ತಾನೆ, ಅವಮಾನ ಮಾಡುತ್ತಾನೆ, ಕೋಟಲೆ ಕೊಡುತ್ತಾನೆ. ಅವನಿಗೆ ಯಾರಾದರೂ ಸ್ವಲ್ಪ ಇದನ್ನು ಮಾಡಿದರೂ ಸಹಿಸುವುದಿಲ್ಲ. ತನ್ನಲ್ಲಿ ಒಳ್ಳೆಯ ಗುಣಗಳು ಇಲ್ಲದೆ ಇದ್ದರೂ ಇವೆ ಎಂದು ಭಾವಿಸುತ್ತಾನೆ. ಇರುವ ಕೆಲವು ಗುಣಗಳನ್ನು ಮೈಕ್ರಾಸ್ಕೋಪಿನ ಕೆಳಗೆ ಇಟ್ಟು ಅದು ಪೆಡಂಭೂತವಾಗಿ ಕಾಣುವಂತೆ ಮಾಡಿಕೊಳ್ಳುತ್ತಾನೆ. ತನ್ನ ಸಮಾನ ಇಲ್ಲ ಎಂದು ಆತ್ಮಶ್ಲಾಘನೆ ಮಾಡಿಕೊಳ್ಳುತ್ತಾನೆ. ತನ್ನಲ್ಲಿರುವ ಸ್ವಲ್ಪ ಹಣಕ್ಕೆ, ಅಧಿಕಾರಕ್ಕೆ, ರೂಪಕ್ಕೆ, ಯೌವನಕ್ಕೆ ಮಾರುಹೋಗುತ್ತಾನೆ. ಕಳ್ಳುಕುಡಿದು ಮನಬಂದಂತೆ ಒರಲುವವರಂತೆ ಆಗುತ್ತಾನೆ.

ಇವರು ಹಿಡಿದುಕೊಳ್ಳುವುದು ಮೋಹದ ವಸ್ತುಗಳು. ಅವು ಇವತ್ತು ಇದ್ದು ನಾಳೆ ಮಾಯ ವಾಗುವುವು. ತಿನ್ನುವಾಗ ಚೆನ್ನಾಗಿರುವುದು, ಅನಂತರ ಸಂಕಟಪಡಬೇಕಾಗುವುದು. ಒಂದು ವಸ್ತು ಇಲ್ಲ. ಆದರೂ ಇದೆ ಎಂದು ಭಾವಿಸುವರು. ವಿಕಾರವಾಗಿದೆ, ಆದರೂ ಸುಂದರವಾಗಿದೆ ಎಂದು ಭಾವಿಸುವರು. ಮೀನು ಹಿಡಿಯುವುದಕ್ಕೆ ಗಾಳದ ಕೊನೆಗೆ ಸಿಕ್ಕಿಸಿರುವ ಹುಳದ ಆಸೆಗೆ ಮೀನು ಧಾವಿಸಿ ಅದನ್ನು ತಿಂದು ಪ್ರಾಣ ಬಿಡುವುದು. ಹಾಗೆಯೇ ಇವರು ಯಾವುದನ್ನು ಹಿಡಿಯಬಾರದೊ ಅಂತಹ ಅನ್ಯಾಯ ಅಧರ್ಮದ ಕಡೆಗೆ ಹೋಗಿ ಮಾಡಬಾರದುದನ್ನು ಮಾಡುತ್ತಾರೆ. ಜೀವನದಲ್ಲಿ ಯಾರು ಮಾಡಬಾರದುದನ್ನು ಮಾಡುತ್ತಾರೆಯೊ ಅವರಿಗೆ ಆಗಬಾರದುದೇ ಆಗುವುದು.

ಅವರು ದುಷ್ಟ ಇಚ್ಛೆಗಳನ್ನು ಆಶ್ರಯಿಸಿ ವರ್ತಿಸುತ್ತಾರೆ. ಅವರು ಮಾಡುವುದೆಲ್ಲ ಅನ್ಯಾಯ ಅಧರ್ಮದ್ದೆ. ಆತ್ಮಹಾನಿ ಲೋಕಹಾನಿಕರವಾದ ಕೆಲಸಗಳೇ ಅವರು ದಿನಬೆಳಗಾದರೆ ಮಾಡುವುದು. ಜೀವನದಲ್ಲಿ ಎಷ್ಟು ಅನುಭವಿಸಿದರೂ ಬುದ್ಧಿ ಕಲಿಯುವುದಿಲ್ಲ. ಎಷ್ಟು ಬುದ್ಧಿವಾದ ಹೇಳಿದರೂ ಕೇಳುವುದಿಲ್ಲ. ಎಂತಹ ಒಂದು ಉತ್ತಮ ವಾತಾವರಣದಲ್ಲಿದ್ದರೂ ನೀಚಕೃತ್ಯಗಳು ಮತ್ತು ನೀಚ ಆಲೋಚನೆಗಳೇ ಅವರಲ್ಲಿ ಬರುವುವು.

\begin{verse}
ಚಿಂತಾಮಪರಿಮೇಯಾಂ ಚ ಪ್ರಲಯಾಂತಾಮುಪಾಶ್ರಿತಾಃ~।\\ಕಾಮೋಪಭೋಗಪರಮಾ ಏತಾವದಿತಿ ನಿಶ್ಚಿತಾಃ \versenum{॥ ೧೧~॥}
\end{verse}

\begin{verse}
ಆಶಾಪಾಶಶತೈರ್ಬದ್ಧಾಃ ಕಾಮಕ್ರೋಧಪರಾಯಣಾಃ~।\\ಈಹಂತೇ ಕಾಮಭೋಗಾರ್ಥಮನ್ಯಾಯೇನಾರ್ಥ ಸಂಚಯಾನ್ \versenum{॥ ೧೨~॥}
\end{verse}

{\small ಪ್ರಳಯದವರೆಗೆ ಮುಗಿಯದಿರುವ ಅಪರಿಮಿತ ಚಿಂತೆಯನ್ನು ಆಶ್ರಯಿಸಿ, ಕಾಮಗಳ ಪರಮಭೋಗಿಗಳಾಗಿ, ಭೋಗವೇ ಸರ್ವಸ್ವ ಎಂದು ನಿಶ್ಚಯಿಸಿ, ನೂರಾರು ಆಸೆಗಳ ಬಲೆಯಲ್ಲಿ ಸಿಕ್ಕಿ, ಕಾಮಕ್ರೋಧ ಪರಾಯಣರಾಗಿ ಅನ್ಯಾಯದಿಂದ ಧನಸಂಗ್ರಹ ಮಾಡಬಯಸುತ್ತಾರೆ.}

ಆಸುರೀ ಪ್ರವೃತ್ತಿಯ ಮನುಷ್ಯ ಆಶ್ರಯಿಸುವುದು ಅಪರಿಮಿತ ಚಿಂತೆಯನ್ನು. ಅವನ ಚಿಂತೆಗೆ ಒಂದು ಕೊನೆಯಿಲ್ಲ. ಎಷ್ಟೆಷ್ಟು ತೃಪ್ತಿ ಪಡಿಸುತ್ತಾನೆಯೋ ಅಷ್ಟೂ ಅಷ್ಟು ಅತೃಪ್ತಿ ಹೆಚ್ಚುತ್ತಿರುವುದು. ಇದು ಶ‍್ರೀರಾಮಕೃಷ್ಣರು ಹೇಳುವ ಯಕ್ಷನ ಹಣದ ಕೊಪ್ಪರಿಗೆಯಂತೆ. ನಾವು ಆ ಮಾಯಾ ಕೊಪ್ಪರಿಗೆಗೆ ಎಷ್ಟು ಹಾಕಿದರೂ ಅದೇನು ಭರ್ತಿಯಾಗುವುದಿಲ್ಲ. ಅದು ಯಾವಾಗಲೂ ಇನ್ನೂ ಬೇಕು ಎನ್ನುತ್ತಿರುವುದೇ ಹೊರತು ಸಾಕು ಎನ್ನುವುದಿಲ್ಲ. ಬದುಕಿರುವವರೆಗೆ ಅದನ್ನು ತೃಪ್ತಿಪಡಿಸಬಹುದು. ಆದರೂ ಅದೇನು ನಿಲ್ಲುವುದಿಲ್ಲ. ಹೊಸ ಹೊಸ ಬಯಕೆಗಳು ಏಳುತ್ತಿರುವುವು.

ಅವರ ಆದರ್ಶವೇ ಕಾಮಗಳನ್ನು ಭೋಗಿಸುವುದು. ಯಾವ ಯಾವ ಲೌಕಿಕ ಆಸೆಗಳು ಮನಸ್ಸಿನಲ್ಲಿ ಏಳುವುವೋ ಅವುಗಳನ್ನು ತೃಪ್ತಿಪಡಿಸಬೇಕು, ಆ ವಸ್ತುವನ್ನು ಪಡೆಯಬೇಕು. ನ್ಯಾಯವೋ ಅನ್ಯಾ ಯವೋ ಚಿಂತೆಯಿಲ್ಲ. ಇನ್ನೊಬ್ಬನಿಗೆ ಅದರಿಂದ ವ್ಯಥೆ ಮತ್ತು ನಷ್ಟವಾದರೂ ಚಿಂತೆಯಿಲ್ಲ. ಅವರ ಉದ್ದೇಶ ಈಡೇರಬೇಕು. ಅದರಿಂದ ಯಾರಿಗೆ ಏನಾದರೂ ಅದನ್ನು ಗಮನಿಸುವುದಿಲ್ಲ. ಸುಖವನ್ನು ಅನುಭವಿಸುವುದೇ ಅವರ ಧರ್ಮ, ಸುಖವನ್ನು ಅನುಭವಿಸುವುದೇ ಅವರ ಜೀವನದ ಉದ್ದೇಶ. ಭೋಗವೇ ಅವರಿಗೆ ಸರ್ವಸ್ವ. ಮನಸ್ಸಿನ ಚಪಲ ಏನೇನು ಕೇಳುವುದೋ ಅದನ್ನು ಕೊಡಲು ಅವರು ಬದುಕಿರುವರು. ಇದಕ್ಕಿಂತ ಅತೀತವಾಗಿರುವುದನ್ನು ಅವರ ಕಣ್ಣು ಕಾಣದು, ಬುದ್ಧಿ ಗ್ರಹಿಸದು. ಯಾರಾದರೂ ಹೇಳಿದರೂ ಅದನ್ನು ನಂಬುವುದಿಲ್ಲ. ಜೀವನದಲ್ಲಿ ಭೋಗವಿಲ್ಲದ ಬಾಳು ಅವರಿಗೆ ಸ್ಮಶಾನ ಸದೃಶವಾಗುವುದು. ಒಂದು ಭವ್ಯ ಆದರ್ಶಕ್ಕೆ ಹೋರಾಡುವುದು, ಇತರರ ಸುಖಶಾಂತಿಗೆ ಪ್ರಯತ್ನಿಸುವುದು, ಇವುಗಳನ್ನು ಅವರು ಅರ್ಥಮಾಡಿಕೊಳ್ಳುವ ಸ್ಥಿತಿಯಲ್ಲಿ ಇಲ್ಲ. ಅವರ ಇಂದ್ರಿಯದ ಮಾರಿಯೇ ಅವರ ಆರಾಧನೆಯ ಇಷ್ಟದೈವ. ಅದು ಕೇಳಿದ ಮೃಗವನ್ನು ಬಲಿ ಕೊಡುವುದೇ ಅವರ ಧರ್ಮ. 

ಅವನು ನೂರಾರು ಆಸೆಗಳ ಬಲೆಯಲ್ಲಿ ಸಿಕ್ಕಿಕೊಳ್ಳುತ್ತಾನೆ. ಪ್ರತಿಯೊಂದು ಬಯಕೆಯನ್ನು ತೃಪ್ತಿ ಮಾಡುತ್ತ ಬಂದಾಗ ಅವುಗಳೆಲ್ಲ ಒಂದೊಂದು ಬಲೆಯಾಗುವುವು. ಹೆಚ್ಚನ್ನು ಸಂಗ್ರಹಿಸಿದಂತೆ ಅದರ ಜವಾಬ್ದಾರಿ ಹೆಚ್ಚುವುದು. ಕೂತ ಕಡೆ ನಿಂತ ಕಡೆ ಯಾವುದು ಏನಾಗಿದೆಯೊ ಎಂಬ ಚಿಂತೆ ಯಾವಾಗಲೂ ಅವನನ್ನು ಬಾಧಿಸುತ್ತಿರುವುದು. ಅದರಿಂದ ಮುಂಚೆ ಎಷ್ಟು ಸುಖ ಬಂತೊ ಅನಂತರ ಅದೇ ವಸ್ತುವಿನಿಂದ ಸುಖಕ್ಕೆ ಹತ್ತು ಪಾಲು ದುಃಖ ಬರುವುದು. ಇದು ಮುಂಚೆ ಅವನಿಗೆ ಗೊತ್ತಾಗುವುದಿಲ್ಲ.

ಅವರು ದೇವರಲ್ಲಿ ಪರಾಯಣರಾಗಿರುವುದಿಲ್ಲ. ಕಾಮ ಮತ್ತು ಕ್ರೋಧವೇ ಅವರ ಪರಾಯಣ ವಸ್ತು. ಆಸೆಯ ಬುದ್ಬುದಗಳು ಒಂದಾದ ಮೇಲೊಂದು ಏಳುತ್ತಿರುವುವು. ಅವುಗಳನ್ನು ಮೆಲಕು ಹಾಕುವುದು ಮತ್ತು ಅವುಗಳನ್ನು ತೃಪ್ತಿಪಡಿಸುವುದು ಇದರಲ್ಲೇ ನಿರತರು. ಯಾವುದಾದರೂ ತನಗೂ ತನ್ನ ಭೋಗವಸ್ತುವಿಗೂ ಮಧ್ಯೆ ಬಂದು ನಿಂತಿತು ಎಂದರೆ ಕ್ರೋಧ ಭುಗಿಲೆಂದು ಹೆಡೆ ಎತ್ತಿ ನಿಲ್ಲುವುದು. ಕಾಮ ಮತ್ತು ಕ್ರೋಧ–ಒಂದು ವಸ್ತು ಮತ್ತೊಂದು ಅದರ ನೆರಳಿನಂತೆ. ಒಂದು ಇದ್ದರೆ ಮತ್ತೊಂದು ಹಾಜರ್, ಬೇಡವೆಂದರೂ ಬಿಡುವುದಿಲ್ಲ.

ತಮ್ಮ ಮನಸ್ಸಿನಲ್ಲಿ ಏಳುವ ಆಸೆ ಆಕಾಂಕ್ಷೆಗಳನ್ನು ತೃಪ್ತಿಪಡಿಸಬೇಕಾಗಿರುವುದರಿಂದ ಹೇಗೊ ಧನವನ್ನು ಸಂಪಾದಿಸಲು ಯತ್ನಿಸುವರು. ಸುಳ್ಳೋ, ಕಳ್ಳತನವೊ, ಕೊಲೆಯೋ, ನೋಟನ್ನು ಪ್ರಿಂಟು ಮಾಡಿಯೋ ಅಂತೂ ದ್ರವ್ಯಾರ್ಜನೆ ಮುಖ್ಯ ಅವರ ಬಯಕೆ ಈಡೇರಬೇಕಾದರೆ. ಹಣವನ್ನು ಸಂಗ್ರಹಿಸುತ್ತ ಹೋಗುವರು, ಎಲ್ಲ ಕಡೆಯಿಂದ ದೋಚುತ್ತ ಹೋಗುವರು. ಅದಕ್ಕಾಗಿ ಏನು ಬೇಕಾದರೂ ಮಾಡಲು ಅವರು ಸಿದ್ಧ.

\begin{verse}
ಇದಮದ್ಯ ಮಯಾ ಲಬ್ಧಮಿದಂ ಪ್ರಾಪ್ಸ್ಯೇ ಮನೋರಥಮ್~।\\ಇದಮಸ್ತೀದಮಪಿ ಮೇ ಭವಿಷ್ಯತಿ ಪುನರ್ಧನಮ್ \versenum{॥ ೧೩~॥}
\end{verse}

{\small ಇಂದು ಇದು ನನ್ನಿಂದ ಹೊಂದಲ್ಪಟ್ಟಿತು. ಈ ಬಯಕೆಯನ್ನು ಈಡೇರಿಸಿಕೊಳ್ಳುತ್ತೇನೆ. ಈ ಧನ ನನಗೆ ಇದೆ. ಇನ್ನೊಬ್ಬರ ಧನ ಕೂಡ ನನಗೆ ಬರುವುದು.}

ಆಸುರೀ ಸ್ವಭಾವದ ಮನುಷ್ಯ ದಿನದಿನಕ್ಕೆ ತಾನು ಗಳಿಸುತ್ತಿರುವುದನ್ನು ಲೆಕ್ಕಾಚಾರ ಹಾಕಿಕೊಂಡು ಸಂತೋಷಪಡುತ್ತಿರುವನು. ಪರವಾಯಿಲ್ಲ ನನ್ನ ಜೀವನ ವ್ಯರ್ಥವಾಗಲಿಲ್ಲ. ನಾನು ಸಾಕಷ್ಟು ಗಳಿಸಿರುವೆನು ಮತ್ತು ಇನ್ನೂ ಗಳಿಸುತ್ತೇನೆ ಎಂದು ತನಗೆ ತಾನೇ ಸಮಾಧಾನವನ್ನು ಹೇಳಿಕೊಳ್ಳುವನು. ಇಂದು ಇದು ನನ್ನಿಂದ ಹೊಂದಲ್ಪಟ್ಟಿತು. ಇಲ್ಲಿ ತನ್ನ ಶ್ರಮಕ್ಕೆ ಎಲ್ಲೂ ಇಲ್ಲದ ಬೆಲೆಯನ್ನು ಕಟ್ಟುವನು. ಇವತ್ತಿನವರೆಗೆ ತನ್ನ ಹತ್ತಿರ ಏನೇನು ಧನಕನಕಾದಿಗಳು ಇದೆಯೋ, ಅವನ್ನೆಲ್ಲ ತಾನು ಸಂಪಾದಿಸಿದ್ದು ಎಂದು ಭಾವಿಸುವನು. ಇನ್ನೊಬ್ಬನಿಗೆ ಅನ್ಯಾಯ ಮಾಡಿರಬಹುದು, ಅವನನ್ನು ನೋಯಿಸಿರಬಹುದು, ಕಳ್ಳತನ ಮಾಡಿರಬಹುದು. ಆದರೂ ಇದೂ ಕೂಡ ಒಂದು ವಿಧವಾದ ಕಷ್ಟ ತಾನೆ? ಅದಕ್ಕೆ ತಾನು ಹೆಮ್ಮೆ ತಾಳುವನು.

ಯಾವ ಬಯಕೆಯನ್ನು ಈಡೇರಿಸಿಕೊಳ್ಳಬೇಕಾದರೂ ಹಣ ಮುಖ್ಯ. ಯಾವಾಗ ಹಣ ಬರುವುದೊ, ಇನ್ನು ಮೇಲೆ ಮನಸ್ಸಿನಲ್ಲಿರುವ ಬಯಕೆಗಳನ್ನು ತೃಪ್ತಿಪಡಿಸಿಕೊಳ್ಳುತ್ತಾನೆ. ಆ ಬಯಕೆಯ ದೀಪಕ್ಕೆ ಇನ್ನೊಂದು ಸ್ವಲ್ಪ ಎಣ್ಣೆಯನ್ನು ಹಾಕುತ್ತಾನೆ. ಅದಕ್ಕೆ ಮತ್ತಷ್ಟು ಜೀವದಾನವಾಗುವುದು. ಅದು ಬರೀ ಆಸೆಯಂತೆಯೇ ಇಲ್ಲ. ಆಸೆಯನ್ನು ತೃಪ್ತಿಪಡಿಸಿರುವೆ ಎಂಬ ಸಮಾಧಾನವಿದೆ. ಆದರೆ ಈಗ ತೃಪ್ತಿಪಡಿಸಿದ ಆಸೆಗಳೇ ಮರಿಹಾಕಿ ಇವನನ್ನು ಕಚ್ಚಲು ಬರುವ ಮಹಾ ಘಟಸರ್ಪಗಳಾಗುತ್ತವೆ ಎಂಬುದನ್ನು ಇವನು ಅರಿಯ. ಇವನದೆಲ್ಲ ತಾತ್ಕಾಲಿಕ ತೃಪ್ತಿ ಮತ್ತು ಸಂತೋಷ. ನಾಳೆ, ಮುಂದೆ, ಇವುಗಳನ್ನೇ ಕುರಿತು ಆಲೋಚಿಸುತ್ತಿದ್ದರೆ ಇನ್ನು ಸುಖ ಯಾವಾಗ ಪಡುವುದು ಎನ್ನುತ್ತಾನೆ.

ಈಗ ನಮ್ಮಲ್ಲಿ ಈ ಧನ ಇದೆ, ಮುಂದೆ ಇನ್ನೊಂದು ಬರುವುದು. ಈಗ ಇರುವುದರಲ್ಲಿ ಒಂದು ಸಂತೋಷವಿದೆ. ಜೊತೆಗೆ ಇಷ್ಟೇ ಅಲ್ಲ ಇರುವುದು. ಇನ್ನೂ ಇದು ವೃದ್ಧಿಯಾಗುವುದು. ಇನ್ನೂ ಯಾರು ಯಾರದನ್ನೋ ಇದಕ್ಕೆ ಸೇರಿಸುತ್ತಾನೆ. ಹಣ ಜಮೀನು, ಮನೆ ಇವುಗಳನ್ನೆಲ್ಲ ಇತರರಿಂದ ಕಸಿದುಕೊಳ್ಳುವ ಪ್ರಯತ್ನ ಮಾಡುತ್ತಿರುವನು. ಅದು ಕೂಡ ಮುಂದೆ ಫಲಿಸುವುದರಲ್ಲಿ ಸಂದೇಹ ವಿಲ್ಲ ಎಂದು ಭಾವಿಸುವನು. ಯಾವುದನ್ನು ಇಲ್ಲೇ ಬಿಟ್ಟುಹೋಗಬೇಕೊ ಅಂತಹ ಹಣ ಮನೆ ಹೊಲ ಗದ್ದೆ, ಹೆಂಗಸರು ಮುಂತಾದುವುಗಳನ್ನು ಸಂಗ್ರಹಿಸುವುದರಲ್ಲಿ ಇವನಿಗೊಂದು ಆನಂದ. ಇದೇ ಇವನ ಜೀವನಾಧಾರ. ಇವುಗಳಿಲ್ಲದೇ ಇದ್ದರೆ ಬಾಳು ಬರಡಾಗುವುದು. ಇವನು ಕೊಬ್ಬಿದಷ್ಟೂ ಪ್ರಕೃತಿಗೇ ಪ್ರಯೋಜನ ಎನ್ನುವುದನ್ನು ಅವನು ಭಾವಿಸುವುದಿಲ್ಲ. ಕುರಿ ಕೊಬ್ಬಿದಷ್ಟೂ ಕುರುಬನಿಗೇ ಲಾಭ ಎಂಬ ಗಾದೆ ಇದೆ. ಅದರಂತೆ ಇವನು ಕುರಿ. ತಿಂದು ಸಂಗ್ರಹಿಸಿ ಕೊಬ್ಬುತ್ತಿರುವನು. ಹಾಗೆ ತಿನ್ನುವಾಗ ತಾನು ತಿನ್ನುತ್ತಿರುವೆ, ತಾನು ಸಂಗ್ರಹಿಸುತ್ತಿರುವೆ ಎಂದು ಭಾವಿಸುವುದು ಕುರಿ. ಆದರೆ ಅದಕ್ಕೆ ಗೊತ್ತಿಲ್ಲ ತಾನು ಕೊಬ್ಬುವುದು ಇನ್ನಾರಿಗೊ ಎಂಬುದು.

\begin{verse}
ಅಸೌ ಮಯಾ ಹತಃ ಶತ್ರುರ್ಹನಿಷ್ಯೇ ಚಾಪರಾನಪಿ~।\\ಈಶ್ವರೋಽಹಮಹಂ ಭೋಗೀ ಸಿದ್ಧೋಽಹಂ ಬಲವಾನ್ ಸುಖೀ \versenum{॥ ೧೪~॥}
\end{verse}

{\small ಈ ಶತ್ರು ನನ್ನಿಂದ ಕೊಲ್ಲಲ್ಪಟ್ಟನು. ಇತರ ಶತ್ರುಗಳನ್ನು ಕೊಲ್ಲುವೆನು. ನಾನು ಒಡೆಯ, ಭೋಗಿ, ಸಿದ್ಧ, ಬಲವಂತ ಮತ್ತು ಸುಖಿ.}

ಆಸೆಯ ಹಾದಿಯಲ್ಲಿ ಶತ್ರುಗಳು ಬಹಳ. ಏಕೆಂದರೆ ಎಲ್ಲರಿಗೂ ಅದರ ಮೇಲೆ ಕಣ್ಣು. ಹಲವು ಜನ ಒಂದು ವಸ್ತುವನ್ನು ಪಡೆಯಬೇಕೆಂದು ಯತ್ನಿಸುತ್ತಿರುವರು. ಅವರೆಲ್ಲ ಇವನಿಗೆ ಶತ್ರುಗಳೇ. ಇವನ ಆಸೆಗೆ ಕಂಟಕನಾದ ಒಬ್ಬ ಶತ್ರುವನ್ನು ಹೇಗೋ ಕೊಂದ. ಅದಕ್ಕೆ ಸಂತೋಷ ಪಡುತ್ತಾನೆ. ಇನ್ನೂ ಎಷ್ಟೋ ಇತರ ಶತ್ರುಗಳು ಇದ್ದಾರೆ. ಅವರನ್ನೆಲ್ಲ ಕ್ರಮೇಣ ನಿರ್ನಾಮ ಮಾಡುತ್ತೇನೆ ಎಂದು ಭಾವಿಸುತ್ತಾನೆ. ತನ್ನಲ್ಲಿರುವ ಕುಯುಕ್ತಿ, ಹಣ ಮುಂತಾದುವುಗಳಿಂದ ತನ್ನ ಮಾರ್ಗದಲ್ಲಿರುವ ಇತರ ಶತ್ರುಗಳನ್ನು ನಿವಾರಿಸುತ್ತೇನೆ ಎಂಬ ಭರವಸೆ ಇವನಿಗೆ ಇದೆ.

ತನ್ನಲ್ಲಿ ಈಗ ಏನೇನು ಇದೆಯೊ, ಆಳು, ಜಮೀನು, ಮನೆ, ಹಣ, ಅಧಿಕಾರ ಮುಂತಾದುವುಗಳಿಗೆ ತಾನು ಒಡೆಯ, ಅಧಿಪತಿ ಎಂದು ಭಾವಿಸುತ್ತಾನೆ. ಇವೆಲ್ಲ ನನಗೆ ಸೇರಿದ್ದು, ಯಾರೂ ಇದನ್ನು ಮುಟ್ಟ ಕೂಡದು, ಕಿತ್ತುಕೊಳ್ಳಕೂಡದು ಎಂದು ಅದರ ಸುತ್ತಲೂ ಬೇಲಿಯನ್ನು ಕಟ್ಟುತ್ತಾನೆ. ಬೆಳಗಿನಿಂದ ಸಾಯಂಕಾಲದ ತನಕ ಇದರ ರಕ್ಷಣೆಯೇ ಅವನ ತಲೆಗೆ ತಟ್ಟಿದ್ದು. ಆದರೂ ಒಡೆಯನೆಂಬ ಅಭಿಮಾನದಿಂದ ಇದನ್ನೆಲ್ಲ ಮೆರೆಯುತ್ತಾನೆ. ತನ್ನನ್ನು ಈ ಸ್ಥಿತಿಗೆ ತಂದಿರುವುದು ಹಲವಾರು ವಸ್ತುಗಳು ಮತ್ತು ಘಟನೆಗಳು ಎಂಬುದನ್ನು ಕುರಿತು ಅವನು ಯೋಚಿಸುವುದೇ ಇಲ್ಲ. ತನ್ನ ಸ್ವಂತ ಪ್ರಯತ್ನದಿಂದ ಈ ಸ್ಥಿತಿಗೆ ಏರಿರುವೆ ಎಂದು ಭಾವಿಸುತ್ತಾನೆ.

ತಾನು ಭೋಗಿ, ತನ್ನ ಹತ್ತಿರ ಇರುವುದನ್ನೆಲ್ಲ ಅನುಭವಿಸುವನು. ಹಾಗೆ ಅನುಭವಿಸುವುದಕ್ಕೆ ತನಗೆ ಅಧಿಕಾರ ಇದೆ. ಅವೆಲ್ಲ ನನಗೆ ಸೇರಿದ್ದು. ನಾನು ಇವನ್ನು ಮನದಣಿಯ ಅನುಭವಿಸುತ್ತೇನೆ ಎಂದು ಭಾವಿಸುವನು.

ನಾನು ಸಿದ್ಧ ಎಂದು ಭಾವಿಸುವನು. ಏನನ್ನು ಪಡೆಯಬೇಕೆಂದು ಸಂಕಲ್ಪ ಮಾಡಿಕೊಂಡಿದ್ದೆನೊ ಅದನ್ನು ಪಡೆದೆ. ನನ್ನ ಆಸೆ ನಿಷ್ಫಲವಾಗಲಿಲ್ಲ, ನಾನು ಗುರಿಯನ್ನು ಮುಟ್ಟಿದೆ ಎಂದು ಭಾವಿಸುವನು. ಪಾಪ, ಅವನಿಗೆ ಗೊತ್ತಿಲ್ಲ ಇದು ಬಹಳ ಕಾಲ ಇರುವುದಿಲ್ಲ ಎಂಬುದು. ಇನ್ನೂ ಹಲವಾರು ಏರಬೇಕಾದ ಶಿಖರಗಳಿವೆ. ಆದರೆ ಒಂದನ್ನು ಹತ್ತಿರುವಾಗ ಅದರಲ್ಲೇ ಕೃತಕೃತ್ಯನಾಗಿರುವನು. ದೃಷ್ಟಿಯನ್ನು ಸ್ವಲ್ಪ ಹೊತ್ತಿನ ಮೇಲೆ ಸುತ್ತ ಹೊರಳಿಸಿದರೆ ಆಗ ಗೊತ್ತಾಗುವುದು.

ತಾನು ಬಲವಂತ, ಒಂದು ವಸ್ತುವನ್ನು ಪಡೆಯುವುದಕ್ಕೆ ಇದ್ದ ಹೋರಾಟದಲ್ಲಿ ಎಲ್ಲರನ್ನೂ ಸದೆ ಬಡಿದು ತಾನು ಅದನ್ನು ಸಂಪಾದಿಸಿದನಲ್ಲ ಅದಕ್ಕೇ ಈ ಹೆಮ್ಮೆ. ಒಂದು ಬಲವಾದ ನಾಯಿ ಬಂದು ಇತರ ನಾಯಿಗಳನ್ನು ಆಚೆಗೆ ಓಡಿಸಿ, ಅಲ್ಲಿ ಬಿದ್ದಿದ್ದ ಎಂಜಲನ್ನು ತಾನು ತಿನ್ನುವಂತೆ. ಆದರೆ ಅದಕ್ಕಿಂತ ಬಲವಾದ ನಾಯಿ ಇದೆ ಎಂದು ಅದಕ್ಕೆ ಈಗ ಗೊತ್ತಿಲ್ಲ. ಪ್ರತಿಯೊಬ್ಬ ಕಳ್ಳನೂ, ಮೋಸಗಾರನೂ, ತನಗಿಂತ ಮೀರಿದವನನ್ನು ಸಂಧಿಸುವವರೆಗೆ ತನ್ನ ಸಮಾನ ಇಲ್ಲ ಎಂದು ಹೆಮ್ಮೆ ಕೊಚ್ಚಿ ಕೊಳ್ಳುವರು.

ತಾನು ಸುಖಪುರುಷ ಎಂದು ಭಾವಿಸುತ್ತಾನೆ. ತಿನ್ನುವುದಕ್ಕೆ ಉಣ್ಣುವುದಕ್ಕೆ ಬೇಕಾದಷ್ಟು ಇದೆ. ಮಲಗುವುದಕ್ಕೆ, ವಿಹರಿಸುವುದಕ್ಕೆ ದೊಡ್ಡ ಬಂಗಲೆ ಇದೆ. ಹೊರಗೆ ಹೋಗುವುದಕ್ಕೆ ಕಾರು, ಆಳು, ಕಾಳು, ಸಿಬ್ಬಂದಿ ಎಲ್ಲ ಇರುವುದರಿಂದ ತಾನು ಸುಖಿ ಎಂದು ಭಾವಿಸುತ್ತಾನೆ. ಈ ಸುಖ ಇನ್ನು ಮುಂದಿನ ಕ್ಷಣ ಏನಾಗುವುದು ಎಂಬುದನ್ನು ಅರಿಯ. ಮಹಲಿಂಗರಂಗ ತನ್ನ ಅನುಭವಾಮೃತದಲ್ಲಿ ಇದನ್ನು ಚೆನ್ನಾಗಿ ವಿವರಿಸುತ್ತಾನೆ: ಕಪ್ಪೆಯೊಂದು ನಾಗರಹಾವಿನ ಹೆಡೆಯ ನೆರಳಿನಲ್ಲಿ ತಂಪಾಗಿದ್ದಂತೆ ಇದು. ಮೇಕೆಯನ್ನು ಮಾರಿಗೆ ಬಲಿಕೊಡಲು ಎಳೆದುಕೊಂಡು ಹೋಗುತ್ತಿದ್ದಾಗ, ಕಟ್ಟಿದ್ದ ಎಳೆ ಎಲೆಯ ತೋರಣವನ್ನು ತಿನ್ನುವಂತೆ ಇದು. ಅವಕ್ಕೆ ಗೊತ್ತಿಲ್ಲ ಮುಂದೆ ಏನಾಗುವುದು ಎಂಬುದು. ಯಾವ ಒಂದು ಭೂಕಂಪವಾಗಿ ಇವನ ಸುಖದ ಮನೆಯನ್ನು ಧೂಳೀಸಮ ಮಾಡುವುದೊ, ಇತರ ಶತ್ರುಗಳು ಇವನ ಮೇಲೆ ಬಿದ್ದು ಇವನು ಈಗ ಹೆಮ್ಮೆ ಕೊಚ್ಚಿಕೊಳ್ಳುವುದನ್ನೆಲ್ಲ ನಿರ್ನಾಮ ಮಾಡುವರೊ, ಇವುಗಳಾವುದರ ಸುಳಿವೂ ಇವನಿಗೆ ಈಗ ಕಾಣುತ್ತಿಲ್ಲ. ಒಂದು ವೇಳೆ ಕಂಡರೂ ಅದು ಹಾಗೆ ಆಗಲಾರದು ಎಂದು ದಬ್ಬಿ ಬಿಡುತ್ತಾನೆ.

\begin{verse}
ಆಢ್ಯೋಽಭಿಜನವಾನಸ್ಮಿ ಕೋಽನ್ಯೋಽಸ್ತಿ ಸದೃಶೋ ಮಯಾ~।\\ಯಕ್ಷ್ಯೇ ದಾಸ್ಯಾಮಿ ಮೋದಿಷ್ಯ ಇತ್ಯಜ್ಞಾನವಿಮೋಹಿತಾಃ \versenum{॥ ೧೫~॥}
\end{verse}

\begin{verse}
ಅನೇಕಚಿತ್ತವಿಭ್ರಾಂತಾ ಮೋಹಜಾಲಸಮಾವೃತಾಃ~।\\ಪ್ರಸಕ್ತಾಃ ಕಾಮಭೋಗೇಷು ಪತಂತಿ ನರಕೇಽಶುಚೌ \versenum{॥ ೧೬~॥}
\end{verse}

{\small ನಾನು ಹಣವಂತ ಮತ್ತು ಕುಲೀನ. ನನಗೆ ಸಮ ಯಾರಿದ್ದಾರೆ? ನಾನು ಯಾಗ ಮಾಡುತ್ತೇನೆ, ದಾನ ಮಾಡು ತ್ತೇನೆ, ಸಂತೋಷಪಡುತ್ತೇನೆ–ಎಂದು ಅಜ್ಞಾನದಿಂದ ಮೋಹಿತರಾಗಿ, ಅನೇಕ ಚಿಂತೆಗಳಿಂದ ಭ್ರಾಂತರಾಗಿ, ಮೋಹಜಾಲದಿಂದ ಆವರಿಸಲ್ಪಟ್ಟವರಾಗಿ, ಕಾಮಭೋಗಗಳಲ್ಲಿ ಆಸಕ್ತರಾಗಿ ಅಶುಚಿಯಾದ ನರಕದಲ್ಲಿ ಬೀಳು ತ್ತಾರೆ.}

ಆಸುರೀ ಸ್ವಭಾವದವರನ್ನು ಮತ್ತಷ್ಟು ಸುಂದರವಾಗಿ ಚಿತ್ರಿಸುತ್ತಾನೆ. ಆತ ಯಾವಾಗಲೂ ಎಲ್ಲ ವಿಧದಲ್ಲಿಯೂ ತನಗಿಂತ ಕೆಳಗೆ ಇರುವವರೊಡನೆ ಹೋಲಿಸಿಕೊಳ್ಳುತ್ತಾನೆ. ಆಗಲೆ ಹೆಮ್ಮೆ ಪಡುವು ದಕ್ಕೆ ಸಾಧ್ಯ. ಅವನೇನಾದರೂ ತನಗಿಂತ ಮೇಲಿರುವವರೊಡನೆ ಹೋಲಿಸಿಕೊಂಡರೆ ನಾಚಿಕೆಯಿಂದ ತಲೆ ತಗ್ಗಿಸಬೇಕಾಗುವುದು. ಆದರೆ ಅವನು ಅದನ್ನು ಮಾಡುವುದಕ್ಕೆ ಹೋಗುವುದಿಲ್ಲ.

ತಾನು ಹಣವಂತ ಎಂದು ಭಾವಿಸುತ್ತಾನೆ. ಅವನ ಹತ್ತಿರ ಸ್ವಲ್ಪ ದುಡ್ಡು, ಮನೆ, ಜಮೀನು ಮುಂತಾದವು ಇರಬಹುದು. ಯಾರಿಗೆ ಇದಿಲ್ಲವೋ, ಅಥವಾ ಇವನಿಗಿಂತ ಕಡಿಮೆ ಇದೆಯೊ ಅವರೊಡನೆ ಹೋಲಿಸಿಕೊಂಡು ತಾನು ಶ‍್ರೀಮಂತನೆಂದು ಭಾವಿಸುತ್ತಾನೆ.

ಬರೀ ಹಣವಂತ ಮಾತ್ರವಲ್ಲ, ಗೌರವಸ್ಥ ವಂಶದಿಂದ ಬಂದವನು ಎಂಬ ಹೆಮ್ಮೆ ಬೇರೆ. ತಮ್ಮ ಪೂರ್ವಿಕರು ತುಂಬಾ ಪ್ರಖ್ಯಾತರು. ದೇಶದಲ್ಲಿ ಆಚಂದ್ರಾರ್ಕವಾದ ಕೀರ್ತಿಯನ್ನು ಬಿಟ್ಟಿರುವರು, ಅಂತಹ ಕುಲದಲ್ಲಿ ಹುಟ್ಟುವುದು ಎಲ್ಲರಿಗೂ ಸಾಧ್ಯವಿಲ್ಲ. ಎಲ್ಲೊ ಅಪೂರ್ವವಾದ ಕೆಲವು ವ್ಯಕ್ತಿಗಳಿಗೆ ಮಾತ್ರ ಲಭ್ಯ ಇದು. ಇಂತಹ ಅಲಭ್ಯ ತನಗೆ ಸಿದ್ಧಿಸಿದೆ ಎಂದು ಹೆಮ್ಮೆ ಪಡುತ್ತಾನೆ. ಹಿಂದಿನವರು ದೊಡ್ಡವರಾಗಿದ್ದರೇನಂತೆ. ತನಗೆ ಏನಾದರೂ ಆ ದೊಡ್ಡತನ ಬರುವುದೆ? ಆ ದೊಡ್ಡತನವನ್ನು ನಾವು ಸಂಪಾದಿಸಬೇಕು ಎಂದು ಭಾವಿಸುವುದಿಲ್ಲ. ಅಂತಹ ಕುಲದಲ್ಲಿ ಹುಟ್ಟಿದ್ದಕ್ಕೇ ಹೆಮ್ಮೆ ಪಡುವನು. ಅನೇಕ ವೇಳೆ ಪಂಡಿತ ಪುತ್ರ ಎಂಬ ಗಾದೆ ಇವರಿಗೆ ಅನ್ವಯಿಸುವುದು. ಇದು ಶುದ್ಧ ಶುಂಠಿಕಾಯಿ ಎಂದು ಹೇಳಿಕೊಳ್ಳುವುದಕ್ಕೆ ಮತ್ತೊಂದು ಹೆಸರು.

ತನ್ನ ಸಮಾನ ಯಾರೂ ಇಲ್ಲ ಎಂದು ಅವನು ಭಾವಿಸುವನು. ಐಶ್ವರ್ಯಕ್ಕೆ ಒಳ್ಳೆಯ ಮನೆತನಕ್ಕೆ ಹೆಸರಾಂತ ವಂಶ ತನ್ನದು ಎಂದು ಭಾವಿಸುವನು. ಇತರರು ಇದನ್ನು ಹೇಳುವುದಿಲ್ಲ. ಇವನೇ ಹೇಳಿಕೊಳ್ಳುವನು.

ತಾನು ಸಾಧಾರಣ ವ್ಯಕ್ತಿಯಲ್ಲ. ಹಲವಾರು ಒಳ್ಳೊಳ್ಳೆಯ ಕೆಲಸವನ್ನು ಮಾಡುತ್ತಿರುವವನು ಎಂದು ಭಾವಿಸುತ್ತಾನೆ. ಯಾಗ ಯಜ್ಞಗಳನ್ನು ಮಾಡುತ್ತಾನೆ. ಬೇಕಾದಷ್ಟು ದುಡ್ಡು ಖರ್ಚು ಮಾಡುತ್ತಾನೆ. ಈ ದುಡ್ಡೆಲ್ಲ ವೆಚ್ಚವಾಗುವುದು ಜಾಹಿರಾತಿಗೆ. ಇಲ್ಲಿ ಭಕ್ತಿ ಮುಕ್ತಿ ಮುಂತಾದುವುಗಳು ಯಾವುದೂ ಇರುವುದಿಲ್ಲ. ಇವೆಲ್ಲ ಬರೀ ಅಟ್ಟಹಾಸ, ತೋರಿಕೆ. ಕೆಲವು ವೇಳೆ ಯಾರಿಗಾದರೂ ದಾನ ಮಾಡುತ್ತಾನೆ. ಅದರ ಹಿಂದೆ ಹೆಸರು ಕೀರ್ತಿ ಇದೆ. ಇವನು ಕೊಟ್ಟಿದ್ದು ಪೇಪರಿನಲ್ಲಿ ಬರಬೇಕು. ಜನಗಳಿಗೆ ಗೊತ್ತಾಗಬೇಕು, ಸರ್ಕಾರದಿಂದ ಧರ್ಮರತ್ನಾಕರ, ಧರ್ಮಪರಾಯಣ ಮುಂತಾದ ಬಿರುದು ಗಳನ್ನು ಗಳಿಸುವುದಕ್ಕೆ ಇವೆಲ್ಲ ಸಾಧನಗಳು. ಇವನು ಮಾಡುವ ದಾನ ಇವನನ್ನು ಮೆರೆಸಲು ಇರುವ ವಾಹನ. ಅದು ಇವನನ್ನು ಹೊತ್ತುಕೊಂಡು ಊರಿನಲ್ಲೆಲ್ಲ ಜಾಹಿರಾತು ಮಾಡಬೇಕು. ನಾನು ಮೋದಪಡುತ್ತೇನೆ, ಸುಖಪಡುತ್ತೇನೆ, ಜೀವನವನ್ನು ಕಾರ್ಪಣ್ಯದಿಂದ ಕಳೆಯುತ್ತಿಲ್ಲ ಎಂದು ಭಾವಿಸು ವನು. ಚೆನ್ನಾಗಿ ತಿನ್ನುತ್ತಾನೆ, ಕುಡಿಯುತ್ತಾನೆ, ಬಟ್ಟೆಬರೆಗಳನ್ನು ಧರಿಸುತ್ತಾನೆ. ಸಿನಿಮಾ, ನಾಟಕ, ನೃತ್ಯ ಇವುಗಳಲ್ಲಿ ಮುಖ್ಯ ಪ್ರೇಕ್ಷಕನಾಗಿರುತ್ತಾನೆ. ಯಾವ ಒಂದು ಹೊಸ ತರಹ ಬಟ್ಟೆ ಬರಲಿ, ಶೋಕಿ ಬರಲಿ, ಇವನು ಅದರ ಬಲೆಗೆ ಮೊದಲು ಬೀಳುತ್ತಾನೆ. ಪಂಚೇಂದ್ರಿಯಗಳಿಗೆ ಪ್ರಿಯವಾದುದನ್ನು ಒದಗೀಸುವುದಕ್ಕೇ ತನ್ನಲ್ಲಿರುವ ದುಡ್ಡನ್ನೆಲ್ಲ ಖರ್ಚು ಮಾಡುವನು.

ಅವನು ಹೇಗಿದ್ದಾನೆ ಎಂಬುದನ್ನು ವಿವರಿಸುತ್ತಾನೆ. ಅಜ್ಞಾನದಿಂದ ಮೋಹಿತರಾಗಿದ್ದಾನೆ. ಈಗ ಏನು ಅನುಭವಿಸುತ್ತಾನೋ ಅದೇ ಪರಮ ಸತ್ಯವೆಂದು ಭಾವಿಸುತ್ತಾನೆ. ಇದರ ಹಿಂದೆ ಇರುವ ಕಷ್ಟ, ದುಃಖ ಅವನಿಗೆ ಈಗ ಗೊತ್ತಾಗುವುದಿಲ್ಲ. ಈಗಿನಂತೆಯೇ ಎಂದೆಂದಿಗೂ ಇರುವೆ ಎಂದು ತಿಳಿಯುವನು. ಅವನ ಬುದ್ಧಿ ತುಕ್ಕು ಹಿಡಿದಿದೆ. ಕೆಲಸ ಮಾಡುವುದಿಲ್ಲ. ಅನೇಕ ಚಿಂತೆಗಳು ಅವನನ್ನು ಆವರಿಸುವುವು. ಹಾಳು ಜೀವನ ನಡೆಸುತ್ತಿರುವನು. ಅನ್ಯಾಯ ಅಧರ್ಮದಿಂದ ಹಣ ಸಂಪಾದನೆ ಮಾಡಿದ್ದು. ಅವುಗಳೆಲ್ಲ ಪ್ರತಿಕ್ರಿಯೆಯನ್ನು ಅವನ ಮೇಲೆ ಬೀರುತ್ತವೆ. ಒಂದೊಂದೂ ಒಂದೊಂದು ಎಡವಟ್ಟಾಗುತ್ತಿದೆ. ಹೊರಗೆ ತನ್ನ ಸಮಾನ ಇಲ್ಲ ಎಂದು ಜಂಭ ಕೊಚ್ಚಿಕೊಂಡರೂ ಒಳಗೆಲ್ಲಾ ಟೊಳ್ಳು. ಕುಟ್ಟೆ ಹಿಡಿದು ಹೋದ ಸೌದೆಯಂತೆ ಇರುವನು. ಹೊರಗೆ ಚೆನ್ನಾಗಿರುವುದು. ಒಳಗೆ ಬರೀ ಪುಡಿ. ಹಾಗೆಯೇ ಇವನ ಬಾಳು. ಮೋಹಜಾಲದಿಂದ ಆವರಿಸಲ್ಪಟ್ಟಿರುವರು. ನೊಣ ಜೇಡರ ಬಲೆಗೆ ಬಿದ್ದಂತೆ. ತಕ್ಷಣವೇ ಜೇಡರ ಹುಳ ಬಂದು ಅದನ್ನು ತನ್ನ ತಂತುವಿನಿಂದ ಬಿಗಿದಿಡುವುದು. ಅದು ಸ್ವಲ್ಪವೂ ಅಳ್ಳಾಡುವುದಕ್ಕೆ ಆಗುವುದಿಲ್ಲ. ಆ ನೊಣ ಜೇಡನ ಬಾಯಿಗೆ ಆಹಾರ. ಆಸುರೀ ಸ್ವಭಾವದವರು ಅದಕ್ಕಿಂತ ಕೇಡು. ಹಲವು ತಂತುಗಳನ್ನು ತಾವೇ ನೆಯ್ದು ಅದರಲ್ಲಿ ಬದ್ಧರಾಗುವರು. ಇನ್ನಾರೋ ಹೊರಗಿನವರು ಇವರನ್ನು ಕಟ್ಟಿಹಾಕುವುದಿಲ್ಲ. ತಾವು ನೆಯ್ದ ಜಾಲದಲ್ಲಿ ತಾವೇ ಸಿಕ್ಕಿಕೊಂಡು ಪ್ರಾಣ ಬಿಡುವರು. ಸಾಯುವತನಕ ಕಾಮ ಮತ್ತು ಭೋಗದ ಕೆಸರಿನಲ್ಲಿ ಬಿದ್ದಿರುವರು. ಅದರಿಂದ ಮೇಲಕ್ಕೆ ಬರಲು ಅವರು ಇಚ್ಛಿಸುವುದಿಲ್ಲ. ಚರಂಡಿಯಲ್ಲಿ ಹುಟ್ಟಿದ ಹುಳಕ್ಕೆ ಅದೇ ನೈಜವಾದ ವಾತಾವರಣ. ಅದನ್ನು ಬಿಟ್ಟುಬರುವುದಕ್ಕೆ ಅದು ಇಚ್ಛಿಸುವುದಿಲ್ಲ. ಇರುವಾಗ ಅವರ ಬಾಳು ವಿಷಯೇಂದ್ರಿಯ ಕೆಸರಿನಲ್ಲಿ ಸಿಕ್ಕಿಕೊಂಡು ನರಳುವುದು. ಕಾಲವಾದ ಮೇಲೆ ಅಂತಹ ಅಥವಾ ಅದಕ್ಕಿಂತ ಶೋಚನೀಯವಾದ ಲೋಕಕ್ಕೆ ಹೋಗುತ್ತಾರೆ.

\begin{verse}
ಆತ್ಮಸಂಭಾವಿತಾಃ ಸ್ತಬ್ಧಾ ಧನಮಾನಮದಾನ್ವಿತಾಃ~।\\ಯಜಂತೇ ನಾಮಯಜ್ಞೆ ೈಸ್ತೇ ದಂಭೇನಾವಿಧಿಪೂರ್ವಕಮ್ \versenum{॥ ೧೭~॥}
\end{verse}

{\small ಆತ್ಮಶ್ಲಾಘನೆ ಮಾಡಿಕೊಳ್ಳುವವರು, ದುರಹಂಕಾರಿಗಳು, ಧನಮಾನಗಳ ಮದದಲ್ಲಿ ಮುಳುಗಿದವರು, ವಿಧಿ ಯನ್ನು ಅನುಸರಿಸದೆ ಕೇವಲ ಹೆಸರಿಗೆ ಮಾತ್ರ ಯಜ್ಞವನ್ನು ಮಾಡುತ್ತಾರೆ.}

ತನ್ನ ಸಮಾನ ಇಲ್ಲ ಎಂದು ಮೆರೆಯುವರು. ಇತರರಿಂದ ತಾವು ಕಲಿಯುವುದೇನಿಲ್ಲ ಎಂದು ತಿಳಿಯುವರು. ತಮ್ಮಲ್ಲಿರುವ ಗುಣಗಳನ್ನು ತಾವೇ ಹೊಗಳಿಕೊಳ್ಳುತ್ತಿರುವರು. ಅವರು ಸ್ತಬ್ಧರು. ಗುರುಹಿರಿಯರು ಇವರುಗಳಿಗೆ ಅವರು ಗೌರವವನ್ನು ತೋರುವ ಗುಂಪಿಗೆ ಸೇರಿಲ್ಲ, ಬಾಗಿದರೆ ಎಲ್ಲಿ ತಮ್ಮ ಮಾನಕ್ಕೆ ಕುಂದು ಬರುವುದೊ ಎಂದು. ಅಂತಹ ಸ್ವಭಾವವುಳ್ಳವನ ಪ್ರಗತಿ ಕೊನೆ ಗೊಂಡಂತೆಯೆ. ಯಾವಾಗ ನಮ್ಮಲ್ಲಿ ದೈನ್ಯತೆ ಇದೆಯೊ, ತಿಳಿದುಕೊಳ್ಳಬೇಕೆಂದು ಆಸೆಯಿದೆಯೊ, ಆಗ ಜ್ಞಾನಕ್ಕೆ ಅವಕಾಶ ಸಿಕ್ಕುವುದು. ಯಾವಾಗ ತನಗೆಲ್ಲ ಗೊತ್ತಿದೆ ಎಂದು ಭಾವಿಸುವನೋ ಅವನು ಜ್ಞಾನ ಪ್ರವೇಶಕ್ಕೆ ಬಾಗಿಲನ್ನು ಹಾಕಿಬಿಟ್ಟಿರುವನು.

ಸುರೆಯನ್ನು ಮಿತಿಮೀರಿ ಕುಡಿದವರಂತೆ ಇವರು ಪ್ರಜ್ಞೆ ತಪ್ಪಿರುವರು. ಇವರಿಗೆ ಸುರೆಯ ಮಾದಕ ನೆತ್ತಿಗೆ ಏರಿದೆ. ಸಾಧಾರಣ ಸುರೆಯ ಮಾದಕ ಬೇಗ ಇಳಿಯುವುದು. ಆದರೆ ಈ ಆಸುರೀ ಸ್ವಭಾವದವರ ಸುರೆಯಾದರೋ ಬೇಗ ಮತ್ತು ಇಳಿಯುವ ಸುರೆಯಲ್ಲ. ದೀರ್ಘಕಾಲ ಮತ್ತಿನಲ್ಲಿ ಕೆಡಹುವುದು. ಆ ಸುರೆಯೆ ಧನ. ಈ ಮದದಿಂದ ಮೆರೆಯುವನು. ಮಾನ ಎಂದರೆ ತನ್ನ ಸಮಾನ ಗೌರವಸ್ಥರಿಲ್ಲ ಎಂದು ಭಾವಿಸುವನು. ಈ ಸುರೆಯ ಮತ್ತು ಇಳಿಯಬೇಕಾದರೆ, ಇವುಗಳಿಂದ ಅನಾಹುತ ಆದ ಮೇಲೆಯೇ, ಮುಂಚೆ ಇಳಿಯುವುದಿಲ್ಲ.

ಅವರು ಮಾಡುವ ಯಾಗ ಯಜ್ಞಗಳು, ಪೂಜೆ ವ್ರತಗಳು ಎಲ್ಲಾ ಜನರನ್ನು ಆಕರ್ಷಿಸುವುದಕ್ಕೆ. ಜನರು ಇದನ್ನು ಕೊಂಡಾಡಬೇಕು. ಅದಕ್ಕಾಗಿ ಮೆರವಣಿಗೆ ಉತ್ಸವ ತಾಳ ಮೇಳಗಳು. ಶಾಸ್ತ್ರದಲ್ಲಿ ಬರುವ ಯಾವ ವಿಧಿನಿಯಮವನ್ನೂ ಅವರು ಅನುಸರಿಸುವುದಿಲ್ಲ. ಅವರ ಶಾಸ್ತ್ರವೇ ಜನರ ಮೆಚ್ಚುಗೆ. ಅದಕ್ಕೆ ಅವರು ಏನು ಬೇಕಾದರೂ ಮಾಡುವರು. ಗಣಪತಿ ಉತ್ಸವ ಎನ್ನುತ್ತಾರೆ. ಅಲ್ಲಿ ಯಾವಳೋ ಒಬ್ಬ ಸಿನಿಮಾ ನಟಿಯಿಂದ ನೃತ್ಯ. ಗಣಪತಿ ಪೂಜೆಗೂ ಇದಕ್ಕೂ ಸಂಬಂಧವಿಲ್ಲ. ಅವರಿಗೆ ಒಂದು ಗೂಟ ಬೇಕು, ಜನರಂಜನೆಯ ಕಾರ್ಯಕ್ರಮಗಳನ್ನು ತಗುಲಿ ಹಾಕುವುದಕ್ಕೆ. ಅದಕ್ಕಾಗಿ ಯಾವು ದಾದರೂ ದೇವರ, ಮಹಾತ್ಮನ, ಗೂಟವನ್ನು ಹೊಡೆದು, ತಮ್ಮ ಮನರಂಜನೆಯ ಗಂಟನ್ನು ನೇತು ಹಾಕುತ್ತಾರೆ. ಇದು ಹೆಸರಿಗೆ ಮಾತ್ರ ಪೂಜೆ. ಇಲ್ಲಿ ಯಾವ ಭಾವವೂ ಇಲ್ಲ, ಭಕ್ತಿಯೂ ಇಲ್ಲ. ಯೋಗ್ಯರಿಗೆ ಪುರಸ್ಕಾರವೂ ಇಲ್ಲ, ದಾನ ಧರ್ಮವೂ ಇಲ್ಲ.

\begin{verse}
ಅಹಂಕಾರಂ ಬಲಂ ದರ್ಪಂ ಕಾಮಂ ಕ್ರೋಧಂ ಚ ಸಂಶ್ರಿತಾಃ~।\\ಮಾಮಾತ್ಮಪರದೇಹೇಷು ಪ್ರದ್ವಿಷಂತೋಽಭ್ಯಸೂಯಕಾಃ \versenum{॥ ೧೮~॥}
\end{verse}

{\small ಅಹಂಕಾರ ಬಲ ದರ್ಪ ಕಾಮ ಕ್ರೋಧ-ಇವುಗಳನ್ನು ಆಶ್ರಯಿಸಿ, ತಮ್ಮ ಮತ್ತು ಇತರ ದೇಹದಲ್ಲಿರುವ ನನ್ನನ್ನು ದ್ವೇಷಿಸುತ್ತ ಅಸೂಯೆಯುಳ್ಳವರಾಗಿರುತ್ತಾರೆ.}

ಆಸುರೀ ಸ್ವಭಾವದವರು ಆಶ್ರಯಿಸುವುದೇ ತಮ್ಮ ಹೀನವೃತ್ತಿಗಳನ್ನು ಪ್ರಚೋದಿಸುವ ಗುಣ ಗಳನ್ನು. ಮನುಷ್ಯ ತಾನಿರುವಂತೆ ತನ್ನ ಪರಿವಾರವನ್ನು ಆರಿಸಿಕೊಳ್ಳುತ್ತಾನೆ. ಇವನ ಪರಿವಾರದಲ್ಲಿರು ವವರಲ್ಲಿ ಮೊದಲನೆಯವನೇ ಅಹಂಕಾರ. ತನ್ನ ಸಮಾನ ಯಾರೂ ಇಲ್ಲ ಎಂದು ಮೆರೆಯುವನು. ವಿದ್ಯೆಯಲ್ಲಿ, ರೂಪಿನಲ್ಲಿ, ಗುಣದಲ್ಲಿ, ಐಶ್ವರ್ಯದಲ್ಲಿ, ಬಲದಲ್ಲಿ ತಾನೇ ಎಲ್ಲರಿಗಿಂತ ಮೇಲು ಎಂದು ಭಾವಿಸುವನು. ತಾನು ಬಲವಂತ, ಆ ಬಲದಿಂದ ಈ ಪ್ರಪಂಚದಲ್ಲಿ ಏನನ್ನು ಬೇಕಾದರೂ ಸಾಧಿಸುವೆನು ಎಂದು ಭಾವಿಸುತ್ತಾನೆ. ದೇಹದ ಬಲ ಮಾತ್ರ ಅಲ್ಲ, ಧನದ ಬಲ, ಅಧಿಕಾರದ ಬಲ, ತನ್ನ ಸ್ಥಾನದ ಬಲ ಎಲ್ಲವನ್ನೂ ವಿನಿಯೋಗಿಸಿ ತನಗೆ ಬೇಕಾದುದನ್ನು ಸಾಧಿಸಲು ಯತ್ನಿಸುವನು.

ಆಸುರೀ ಸ್ವಭಾವದವರು ಆಶ್ರಯಿಸುವುದೇ ತಮ್ಮ ಹೀನವೃತ್ತಿಗಳನ್ನು ಪ್ರಚೋದಿಸುವ ಗುಣ ಗಳನ್ನು. ಮನುಷ್ಯ ತಾನಿರುವಂತೆ ತನ್ನ ಪರಿವಾರವನ್ನು ಆರಿಸಿಕೊಳ್ಳುತ್ತಾನೆ. ಇವನ ಪರಿವಾರದಲ್ಲಿರು ವವರಲ್ಲಿ ಮೊದಲನೆಯವನೇ ಅಹಂಕಾರ. ತನ್ನ ಸಮಾನ ಯಾರೂ ಇಲ್ಲ ಎಂದು ಮೆರೆಯುವನು. ವಿದ್ಯೆಯಲ್ಲಿ, ರೂಪಿನಲ್ಲಿ, ಗುಣದಲ್ಲಿ, ಐಶ್ವರ್ಯದಲ್ಲಿ, ಬಲದಲ್ಲಿ ತಾನೇ ಎಲ್ಲರಿಗಿಂತ ಮೇಲು ಎಂದು ಭಾವಿಸುವನು. ತಾನು ಬಲವಂತ, ಆ ಬಲದಿಂದ ಈ ಪ್ರಪಂಚದಲ್ಲಿ ಏನನ್ನು ಬೇಕಾದರೂ ಸಾಧಿಸುವೆನು ಎಂದು ಭಾವಿಸುತ್ತಾನೆ. ದೇಹದ ಬಲ ಮಾತ್ರ ಅಲ್ಲ, ಧನದ ಬಲ, ಅಧಿಕಾರದ ಬಲ, ತನ್ನ ಸ್ಥಾನದ ಬಲ ಎಲ್ಲವನ್ನೂ ವಿನಿಯೋಗಿಸಿ ತನಗೆ ಬೇಕಾದುದನ್ನು ಸಾಧಿಸಲು ಯತ್ನಿಸುವನು.

ಯಾರನ್ನೂ ಗಣನೆಗೇ ತರುವುದಿಲ್ಲ. ಅವರನ್ನು ಅಣಕಿಸುವನು, ಹಂಗಿಸುವನು, ಅವರಿಗೆ ಅಗೌರವ ತೋರುವನು. ಇದಕ್ಕೆ ನಾಚುವುದಿಲ್ಲ, ಹೆಮ್ಮೆ ತಾಳುವನು. ಅದನ್ನು ಎಲ್ಲರ ಹತ್ತಿರವೂ ಹೇಳಿಕೊಳ್ಳು ವನು. ಎಲ್ಲಾ ವಿಧವಾದ ಆಸೆಗಳೂ ಇಲ್ಲಿ ಸಂಗ್ರಹವಾಗುತ್ತ ಬರುವುವು. ಆಸೆ ಯಾವಾಗ ಸಂಗ್ರಹವಾಗುತ್ತ ಬರುವುದೊ, ಅದನ್ನು ತೃಪ್ತಿ ಪಡಿಸಿಕೊಳ್ಳುವುದಕ್ಕಾಗಿ ಕರ್ಮದಲ್ಲಿ ನಿರತನಾಗು ವನು. ಏನು ಆತಂಕ ಬಂದರೂ ಅದರ ಮೇಲೆ ಉರಿದು ಬೀಳುವನು. ಕಾಮದ ನೆರಳೆ ಕೋಪ. ಒಂದು ಇದ್ದರೆ ಮತ್ತೊಂದೂ ಇರಬೇಕು.

ಅವರು ತಮ್ಮ ಮತ್ತು ಇತರ ದೇಹಗಳಲ್ಲಿರುವ ನನ್ನನ್ನು ದ್ವೇಷಿಸುತ್ತಾರೆ. ಪರಮಾತ್ಮ ಎಲ್ಲರ ಹೃದಯದಲ್ಲಿಯೂ ಇರುವನು. ಆಸುರೀ ಸ್ವಭಾವದವರ ಹೃದಯದಲ್ಲಿಯೂ ಇರುವನು. ಕೆಲವು ವೇಳೆ ಭಗವಂತ ಅವರಿಗೆ ಎಚ್ಚರಿಕೆ ಕೊಡುವನು. ಅವರು ಹಿಡಿದಿರುವ ದಾರಿ ಶ್ರೇಯಸ್ಕರವಲ್ಲ ಎಂದು ಹೇಳುವನು. ಅದಕ್ಕೆ ಅವರು ಕಿವಿಗೊಡುವುದಿಲ್ಲ. ಇತರರು ಏನಾದರೂ ಒಳ್ಳೆಯದನ್ನು ಹೇಳಿದರೆ ಅವರಿಗೆ ಇದರಷ್ಟು ಕಹಿಯಾಗಿರುವುದು ಇನ್ನಿಲ್ಲ. ಅದನ್ನು ಗಮನಿಸುವುದೇ ಇಲ್ಲ.

ಯಾರಾದರೂ ಜೀವನದಲ್ಲಿ ಒಳ್ಳೆಯ ಆದರ್ಶವನ್ನು ಇಟ್ಟುಕೊಂಡು ಬಾಳುತ್ತಿದ್ದರೆ, ಜನ ಅವನನ್ನು ಗೌರವಿಸುತ್ತಿದ್ದರೆ, ಅವನನ್ನು ಕಂಡರೆ ಆಸುರೀ ಸ್ವಭಾವದವನಿಗೆ ಆಗುವುದಿಲ್ಲ. ಯಾವಾಗ ಅವನನ್ನು ಕೆಳಗೆ ಇಳಿಸೇನು, ತಮ್ಮಂತೆಯೆ ಮಾಡಿಯೇನು ಎಂದು ಹೊಂಚು ಹಾಕುತ್ತಿರವನೆ ಹೊರತು, ತನಗೆ ಸಾಧ್ಯವಾಗದುದು ಅವನಿಗಾದರೂ ಆಯಿತಲ್ಲ ಎಂದು ಗೌರವಿಸುವುದಕ್ಕೆ ಹೋಗು ವುದಿಲ್ಲ. ಇವನು ಅವರನ್ನು ದ್ವೇಷಿಸುತ್ತಾನೆ, ಗೂಬೆ ಬೆಳಕನ್ನು ದ್ವೇಷಿಸುವಂತೆ.

\begin{verse}
ತಾನಹಂ ದ್ವಿಷತಃ ಕ್ರೂರಾನ್ ಸಂಸಾರೇಷು ನರಾಧಮಾನ್~।\\ಕ್ಷಿಪಾಮ್ಯಜಸ್ರಮಶುಭಾನಾಸುರೀಷ್ವೇವ ಯೋನಿಷು \versenum{॥ ೧೯~॥}
\end{verse}

{\small ನನ್ನನ್ನು ದ್ವೇಷಿಸುತ್ತಿರುವ ಕ್ರೂರರೂ ಪಾಪಕರ್ಮಗಳನ್ನು ಮಾಡುವವರೂ ಆದ ನರಾಧಮರನ್ನು ಸಂಸಾರದಲ್ಲಿ ಆಸುರೀ ಯೋನಿಯಲ್ಲಿ ಯಾವಾಗಲೂ ಹಾಕುತ್ತೇನೆ.}

ಆಸುರೀ ಸ್ವಭಾವದ ಜನ ಕ್ರೂರಿಗಳು. ಎಳ್ಳಷ್ಟೂ ಅವರಿಗೆ ದಯವಿಲ್ಲ ಇತರರ ಮೇಲೆ. ತಮ್ಮ ಸ್ವಾರ್ಥ ಸಾಧನೆಗೆ ಇತರರಿಗೆ ಏನು ತೊಂದರೆ ಬೇಕಾದರೂ ಕೊಡುತ್ತಾರೆ, ಎಷ್ಚು ನೋವನ್ನು ಬೇಕಾದರೂ ಕೊಡುತ್ತಾರೆ. ಯಾವಾಗಲೂ ಪಾಪಕರ್ಮಗಳೇ ಅವರು ಮಾಡುತ್ತಿರುವುದು. ದೇವರಿಗೆ ವಿಮುಖರಾಗಿ ಹೋಗುವರು. ಇಂದ್ರಿಯ ಪ್ರಪಂಚದಲ್ಲಿ ಆಳ ಆಳಕ್ಕೆ ಹೋಗುವುದಕ್ಕೆ ಏನು ಬೇಕೊ ಅದನ್ನು ಮಾಡುವರು. ಅವರ ದೃಷ್ಟಿ ಸಂಕುಚಿತವಾಗುತ್ತ ಬರುವುದು. ಚಿತ್ತ ಕಶ್ಮಲದಿಂದ ಕೂಡಿರು ವುದು. ನಿತ್ಯ ಅನಿತ್ಯವನ್ನು ವಿಮರ್ಶಿಸಲಾರರು. ಅವರು ದೇವರನ್ನು ದ್ವೇಷಿಸುತ್ತಾರೆ. ಅವನ ಕಡೆ ಹೋಗುವುದಕ್ಕೆ ಬಯಸುವುದಿಲ್ಲ. ಮನಸ್ಸಿನಲ್ಲಿ ಯಾವಾಗಲಾದರೊಮ್ಮೆ ಒಳ್ಳೆಯ ಭಾವನೆ ಎದ್ದರೆ, ಅದು ಮತ್ತೊಮ್ಮೆ ಬರದಂತೆ ಮಾಡುತ್ತಾರೆ. ಏಕೆಂದರೆ ಆ ಭಾವನೆ ಮನಸ್ಸಿನಲ್ಲಿದ್ದರೆ ಇಂದ್ರಿಯ ಪ್ರಪಂಚದಲ್ಲಿ ಸ್ವೇಚ್ಛೆಯಿಂದ ವಿಹರಿಸುವುದಕ್ಕೆ ಆಗುವುದಿಲ್ಲ. ದೇವರನ್ನು ದ್ವೇಷಿಸುವುದಕ್ಕೆ ಕಾರಣ ಅದು ತಮ್ಮ ಮೃಗೀಯ ಜೀವನಕ್ಕೆ ಅಡ್ಡಿ ಬರುವುದು ಎಂಬುದು. ಇಂತಹ ಜನ ನರಾಧಮರು, ಮನುಷ್ಯರಲ್ಲಿ ನಿಕೃಷ್ಟರು. ದೇವರು ಕೊಟ್ಟ ವಿಚಾರ ಶಕ್ತಿಯನ್ನು ಉಪಯೋಗಿಸುವುದಿಲ್ಲ, ಇತರರಿಂದ ಬುದ್ಧಿಯನ್ನು ಕಲಿಯುವುದಿಲ್ಲ. ಇಂದ್ರಿಯ ಲೋಲುಪತೆಯೊಂದೇ ಅವರಿಗೆ ಇರುವುದು. ಇಂತಹ ನರಾಧಮರು ಪ್ರಾಣಿಗಿಂತ ಕೀಳು. ಪ್ರಾಣಿಗಳು ವಿಚಾರ ಮಾಡಲಾರವು. ಏಕೆಂದರೆ ಅವಕ್ಕೆ ಆ ಶಕ್ತಿ ಇಲ್ಲ. ಆದರೆ ಮನುಷ್ಯನಿಗೆ ಆ ಶಕ್ತಿ ಇದೆ. ಆದರೂ ಅದನ್ನು ಉಪಯೋಗಿಸದೆ ಇದ್ದರೆ ಅವನು ಕ್ಷಮಾರ್ಹನಲ್ಲ. ಆದಕಾರಣವೇ ದೇವರು ಇಂತಹ ವ್ಯಕ್ತಿಗಳನ್ನು ಆಸುರೀ ಮನೆಯಲ್ಲಿಯೇ ಹುಟ್ಟು ವಂತೆ ಮಾಡುತ್ತಾನೆ. ಇವರು ಅಂಥ ಕೆಲಸವನ್ನೇ ಮಾಡುವರು. ಅಂತಹ ಪರಿಣಾಮವನ್ನೇ ಅನುಭವಿಸಬೇಕಾಗಿದೆ. ಅಂತಹ ಆಸುರೀ ವಾತಾವರಣದಲ್ಲಿ ಬೆಳೆದರೆ ಮೊದಮೊದಲು ಸುಖ ಅನುಭವಿಸುವರು, ಬೇಕಾದಷ್ಟು ಇಂದ್ರಿಯ ಪ್ರಪಂಚದಲ್ಲಿ ಓಲಾಡುವರು. ಅನಂತರ ಅದರ ಪರಿಣಾಮಗಳು ರಣಹದ್ದಿನಂತೆ ಅವರನ್ನು ಕಿತ್ತು ತಿನ್ನುವುವು. ಆಸುರೀ ಸ್ವಭಾವದ ಜನರಿಗೆ ಮೊದಲು ಸುಖ, ಆಮೇಲೆ ದುಃಖ ಬೇಡವೆಂದರೂ ಕಾದು ಕುಳಿತಿದೆ.

\begin{verse}
ಆಸುರೀಂ ಯೋನಿಮಾಪನ್ನಾ ಮೂಢಾ ಜನ್ಮನಿ ಜನ್ಮನಿ~।\\ಮಾಮಪ್ರಾಪ್ಯೈವ ಕೌಂತೇಯ ತತೋ ಯಾಂತ್ಯಧಮಾಂ ಗತಿಮ್ \versenum{॥ ೨ಂ~॥}
\end{verse}

{\small ಅರ್ಜುನ, ಪ್ರತಿಜನ್ಮದಲ್ಲಿಯೂ ಆಸುರೀ ಯೋನಿಯನ್ನು ಹೊಂದಿದ ಮೂಢರು, ನನ್ನನ್ನು ಪಡೆಯದೆ, ಅದಕ್ಕಿಂತ ಕೀಳಾದ ಗತಿಯನ್ನು ಪಡೆಯುತ್ತಾರೆ.}

ಮೂಢರು ಪ್ರತಿ ಜನ್ಮದಲ್ಲಿಯೂ ಆಸುರೀ ಮನೆಯಲ್ಲಿಯೇ ಹುಟ್ಟುತ್ತಾರೆ. ಏಕೆಂದರೆ ಅವರು ಮಾಡಿರುವ ಕರ್ಮ ಅದೇ. ಫೇಲಾದ ಹುಡುಗ ಪ್ರತಿವರ್ಷವೂ ಅದೇ ಕ್ಲಾಸಿನಲ್ಲಿ ಓದಬೇಕಾಗುವಂತೆ ಇವರು. ಅದರಿಂದ ಮೇಲೆ ಬರುವುದಕ್ಕೆ ಅವರು ಏನನ್ನೂ ಮಾಡುವುದಿಲ್ಲ. ಇಂತಹ ಜನ ಭಗವಂತನನ್ನು ಪಡೆಯುವುದಿಲ್ಲ. ಭಗವಂತ ಎಲ್ಲರಲ್ಲಿಯೂ ಇದ್ದಾನೆ. ಆದರೆ ಇವರಿಗೆ ಅವನು ಬೇಕಾಗಿಲ್ಲ. ಅವನನ್ನು ಪಡೆಯುವುದಕ್ಕೆ ಬೇಕಾದ ಪ್ರಯತ್ನವನ್ನು ಮಾಡುವುದಿಲ್ಲ. ಎತ್ತು ಗಾಣದ ಸುತ್ತಲೂ ಸುತ್ತುವಂತೆ, ಆಸುರೀ ಸ್ವಭಾವದವರು ಇಂದ್ರಿಯದ ಗಾಣದ ಸುತ್ತಲೂ ಸುತ್ತುತ್ತಿರು ವರು. ಅವರಿಗೆ ಉತ್ತಮ ಜನ್ಮದ ರುಚಿಯ ಆವಶ್ಯಕತೆ ಇನ್ನೂ ಹುಟ್ಟಿಲ್ಲ. ಇದ್ದಕಡೆಯೇ ಆದರೂ ಇರುತ್ತಾರೆಯೇ ಎಂದರೆ, ಯಾವಾಗ ಉತ್ತಮನಾಗಲು ಪ್ರಯತ್ನಮಾಡುವುದಿಲ್ಲವೊ, ಅಧಮ ಮತ್ತೂ ಅಧಮನಾಗಿ ಹೋಗುವನು. ಏನನ್ನೊ ತಪ್ಪನ್ನು ಮಾಡಿದ್ದಕ್ಕೆ ಒಬ್ಬನನ್ನು ಜೈಲಿಗೆ ಹಾಕಿದ್ದಾರೆ. ಆ ಜೈಲಿನಲ್ಲಿಯೂ ಅವನು ನಿಯಮಗಳನ್ನು ಪಾಲಿಸುವುದಿಲ್ಲ. ಅಲ್ಲಿ ಏನು ಸಿಕ್ಕಿದರೆ ಅದನ್ನು ಕದಿಯುತ್ತಾನೆ. ಸಿಕ್ಕಿದ ಸಾಮಾನನ್ನು ಮುರಿದು ಹಾಕುತ್ತಾನೆ. ಕೆಲಸ ಕೊಟ್ಟರೆ ಮಾಡುವುದಿಲ್ಲ. ಸಮಯ ಸಿಕ್ಕಿದರೆ ಓಡಿಹೋಗಲು ಯತ್ನಿಸುತ್ತಾನೆ. ಇಂತಹ ತಪ್ಪಿತಸ್ಥನು ಅನುಭವಿಸಬೇಕಾದ ಶಿಕ್ಷೆಯ ಜೊತೆಗೆ ಜೈಲಿನಲ್ಲಿ ಮಾಡಿದ ಅಪರಾಧಗಳಿಗೆ ಮತ್ತಷ್ಟು ಹೆಚ್ಚು ದಿನ ಜೈಲುವಾಸವನ್ನು ಅನುಭವಿಸಬೇಕಾಗುವುದು. ಅವನಿಗೆ ಜೈಲಿನಲ್ಲಿ ಸಿಕ್ಕುವ ಸೌಲಭ್ಯಗಳೂ ಕಡಿಮೆಯಾಗುವುವು. ಅದ ರಂತೆಯೇ ಪಾಪ ಜನ್ಮದಲ್ಲಿ ಹುಟ್ಟಿರುವವನು, ಸ್ವಲ್ಪವಾದರೂ ಉತ್ತಮನಾಗಲು ಪ್ರಯತ್ನ ಪಡದೇ ಇದ್ದರೆ, ಅವನು ಮತ್ತೂ ಆಳಕ್ಕೆ ಹೋಗುವನು.

ಆದರೆ ಜೀವನದಲ್ಲಿ ಎಂದೆಂದಿಗೂ ಒಬ್ಬ ಹಾಗೆಯೇ ಕೆಳಕೆಳಕ್ಕೆ ಹೋಗುತ್ತಿರುವುದಕ್ಕೆ ಆಗುವು ದಿಲ್ಲ. ಎಷ್ಟೇ ಕೆಳಗೆ ಹೋದರೂ ಒಂದಲ್ಲ ಒಂದು ಸಲ ತಳ ಮುಟ್ಟುವೆವು. ಅನಂತರ ಮೇಲೆ ಬರಬೇಕಾಗುವುದು. ಆದರೆ ಹಾಗೆ ಬರುವುದಕ್ಕೆ ತುಂಬಾ ಕಾಲ ಹಿಡಿಯುವುದು ಇಂತಹ ಜನಗಳಿಗೆ.

\begin{verse}
ತ್ರಿವಿಧಂ ನರಕಸ್ಯೇದಂ ದ್ವಾರಂ ನಾಶನಮಾತ್ಮನಃ~।\\ಕಾಮಃ ಕ್ರೋಧಸ್ತಥಾ ಲೋಭಸ್ತಸ್ಮಾದೇತತ್ತ್ರಯಂ ತ್ಯಜೇತ್ \versenum{॥ ೨೧~॥}
\end{verse}

{\small ಕಾಮ ಕ್ರೋಧ ಮತ್ತು ಲೋಭ--ಇವು ಮೂರು ನರಕದ ದ್ವಾರಗಳು. ಇದು ಆತ್ಮನಾಶಕ್ಕೆ ಕಾರಣ. ಆದುದರಿಂದ ಈ ಮೂರನ್ನೂ ಬಿಡಬೇಕು.}

ನಮ್ಮಲ್ಲೆಲ್ಲ ಮುಕ್ತರಾಗುವುದಕ್ಕೆ ಬೀಜ ಇದೆ, ಬದ್ಧರಾಗುವುದಕ್ಕೆ ತಕ್ಕ ಬೀಜವೂ ಇದೆ. ನಾವು ಯಾವುದನ್ನು ಬಿತ್ತಿ ಕೃಷಿ ಮಾಡಿದರೆ ಅದರಂತೆ ಆಗುತ್ತೇವೆ. ನರಕಕ್ಕೆ ನಮ್ಮನ್ನು ಒಯ್ಯುವುದೇ ಈ ಮೂರು ಗುಣಗಳು. ನರಕವೆಂದರೆ ಬೇರೆ ಯಾವುದೋ ನರಕಲೋಕವೇ ಇದೆ ಎಂತಲೇ ಭಾವಿಸ ಬೇಕಾಗಿಲ್ಲ. ಈ ಪ್ರಪಂಚದಲ್ಲೆ ನರಕವೂ ಇದೆ, ಸ್ವರ್ಗವೂ ಇದೆ. ನಾವು ಯಾವುದಕ್ಕೆ ಯೋಗ್ಯರೋ ನಮಗೆ ಅದು ಸಿಕ್ಕುವುದು. ಕಾಮ, ಕ್ರೋಧ, ಲೋಭ ಇವು ನಮ್ಮನ್ನು ಅತ್ಯಂತ ಹೀನವಾದ ವಾತಾವರಣಕ್ಕೆ ಒಯ್ಯುವುವು, ನಮ್ಮನ್ನು ಮೃಗಗಳಿಗಿಂತ ಕೀಳಾಗಿ ಮಾಡುವುವು. ಕಾಮ ಎಂಬುದು ಒಬ್ಬನಲ್ಲಿ ಯಾವಾಗ ವೃದ್ಧಿಯಾಗುವುದೊ ಅದನ್ನು ತೃಪ್ತಿಪಡಿಸಬೇಕು. ಕ್ರಮೇಣ ಅವನು ವಿಷಯ ವಸ್ತುಗಳನ್ನು ಪಡೆಯಲು ಯತ್ನಿಸುವನು. ಅವನ್ನು ಪಡೆಯುವುದಕ್ಕೆ ಸುಳ್ಳೋ, ಕಳ್ಳತನವೋ ಕೊಲೆಯೋ ಏನನ್ನು ಬೇಕಾದರೂ ಮಾಡುವನು. ಆ ವಸ್ತುವನ್ನು ಹೇಗೋ ಪಡೆದು ಅನುಭವಿಸುತ್ತಾನೆ. ಅಂತಹ ಇನ್ನೂ ವಿವಿಧ ಅನುಭವಗಳು ಬೇಕೆಂದು ಅವನ ಮನಸ್ಸು ಹೇಳುವುದು. ಎಷ್ಟು ಕೊಟ್ಟರೂ ತೃಪ್ತಿ ಇಲ್ಲ. ಹಿಂದಿನ ಆಸೆ ಹೊಸ ಹೊಸ ರೂಪದಲ್ಲಿ ಬರುವುದು. ಅದನ್ನು ತೃಪ್ತಿ ಪಡಿಸುವುದಕ್ಕೆ ಅವನ ಬಾಳೆಲ್ಲ ವ್ಯರ್ಥವಾಗುವುದು. ಅದನ್ನು ಬೆಂಬಿಡದೆ ಇರುವುದೇ ಕ್ರೋಧ. ಎಲ್ಲಿ ಕಾಮವಿದೆಯೊ ಅಲ್ಲಿ ಕ್ರೋಧವೂ ಇರುವುದು. ಕಾಮಕ್ಕೆ ಅಡ್ಡಿಯಾಗಿ ಯಾವುದು ಬರಲಿ ಅದರ ಮೇಲೆ ನಮಗೆ ಕೋಪ. ಅದನ್ನು ನಿವಾರಿಸಲು ದಯಾದಾಕ್ಷಿಣ್ಯ ನೋಡುವುದಿಲ್ಲ, ಒಳ್ಳೆಯದು ಕೆಟ್ಟದ್ದು ನೋಡುವು ದಿಲ್ಲ. ದಳ್ಳುರಿ ತನ್ನ ಹತ್ತಿರ ಏನು ಬಂದರೆ ಅದನ್ನೆಲ್ಲ ಧ್ವಂಸ ಮಾಡುವಂತೆ ಕ್ರೋಧ ಮಾಡುವುದು. ಅನಂತರವೇ ಲೋಭದಿಂದ ಪ್ರೇರಿತರಾಗಿ ಹಲವು ವಸ್ತುಗಳನ್ನು ಸಂಗ್ರಹಿಸಿಟ್ಟುಕೊಳ್ಳುತ್ತೇವೆ. ಅದನ್ನು ಇನ್ನೊಬ್ಬರಿಗೆ ಕೊಡುವುದಿಲ್ಲ. ಈತನಿಗೆ ಗೊತ್ತಿರುವುದು ಒಂದೇ, ಅದೇ ಕೂಡಿಡುವುದು. ಇನ್ನೂ ಹೆಚ್ಚು ಹೆಚ್ಚಾಗಿ ಹೊರಗಿನಿಂದ ಬಂದು ಶೇಖರವಾಗುತ್ತಿರಬೇಕು. ಹೆಚ್ಚು ಹಣ ಬೇಕು, ಜಮೀನು ಬೇಕು, ಮನೆ ಬೇಕು, ಕೀರ್ತಿ ಬೇಕು, ಅಧಿಕಾರ ಬೇಕು. ಇದರಲ್ಲಿ ಒಂದು ಸ್ವಲ್ಲ ಖೋತಾ ಆಗುವ ಪರಿಸ್ಥಿತಿ ಬಂದರೂ ಇವನು ವಿಹ್ವಲನಾಗಿ ಹೋಗುವನು. ಇವನಂತೂ ಕೈಯೆತ್ತಿ ಇನ್ನೊಬ್ಬ ನಿಗೆ ದಾನ ಧರ್ಮ ಮಾಡುವವನಲ್ಲ. ಇವನಲ್ಲಿರುವುದನ್ನು ಕಸಿಯಬೇಕಾದರೆ ದೇವರೇ ಒಂದು ದೊಡ್ಡ ವಿಪತ್ತನ್ನು ಕಳುಹಿಸಬೇಕು. ಈ ಮೂರು ಸ್ವಭಾವ ನಮ್ಮಲ್ಲಿರುವ ಪರಮಾತ್ಮನನ್ನು ಕಾಣದಂತೆ ಮಾಡುವುವು. ಅವನ ಕಾಂತಿಯ ಮೇಲೆ ಈ ತೆರೆಯ ಬಟ್ಟೆಯನ್ನು ಕಟ್ಟುವವರು ನಾವೇ. ನಾವು ನಮ್ಮ ನಾಶಕ್ಕೆ ಕಾರಣರಾಗುತ್ತೇವೆ. ಯಾವಾಗ ದೇವರ ಕಡೆ ಹೋಗುವುದಿಲ್ಲವೋ, ಕೇವಲ ಇಂದ್ರಿಯ ಚಪಲ ತೀರಿಸಿಕೊಳ್ಳುವುದರಲ್ಲೆ ನಮ್ಮ ಬಾಳೆಲ್ಲ ವ್ಯರ್ಥವಾಗುವುದೋ, ಆಗ ಮೃಗ ಸದೃಶರಾಗುವೆವು. ಪ್ರಾಣಿಗಳಿಗೆ ಬುದ್ಧಿಯಿಲ್ಲ. ಅವು ನಾಲ್ಕು ಕಾಲಮೇಲೆ ಹೋಗುವುವು. ನಮಗೆ ಬುದ್ಧಿ ಇದೆ. ಆದರೆ ಅದನ್ನು ಉಪಯೋಗಿಸಿಕೊಂಡು ಪಾರಾಗುವುದಿಲ್ಲ. ಒಂದು ವೇಳೆ ಅದನ್ನು ಉಪಯೋಗಿಸಿಕೊಂಡರೆ, ಅದು ಇನ್ನೊಬ್ಬನನ್ನು ಹಾಳುಮಾಡುವುದಕ್ಕೆ ದುರುಪಯೋಗಪಡಿಸಿಕೊಳ್ಳು ವೆವು. ನಾವು ನರಮೃಗಗಳಾಗುತ್ತೇವೆ. ಶ‍್ರೀಕೃಷ್ಣ, ಈ ಮೂರು ಗುಣಗಳನ್ನು ಬಿಡಬೇಕು, ಇವೇ ನಮ್ಮನ್ನು ಹೀನಲೋಕಕ್ಕೆ ಒಯ್ಯುವುವು ಎನ್ನುತ್ತಾನೆ. ಈ ಗುಣಗಳು ಎಲ್ಲರಲ್ಲಿಯೂ ಒಂದು ಪ್ರಮಾಣದಲ್ಲಿವೆ. ಅವನ್ನು ಕಡಿಮೆಮಾಡಿ ಯೋಗ್ಯ ಗುಣಗಳನ್ನು ನಮ್ಮ ಜೀವನದಲ್ಲಿ ರೂಢಿಸಬೇಕು.

\begin{verse}
ಏತೈರ್ವಿಮುಕ್ತಃ ಕೌಂತೇಯ ತಮೋದ್ವಾರೈಸ್ತ್ರಿಭಿರ್ನರಃ~।\\ಆಚರತ್ಯಾತ್ಮನಃ ಶ್ರೇಯಸ್ತತೋ ಯಾತಿ ಪರಾಂ ಗತಿಮ್ \versenum{॥ ೨೨~॥}
\end{verse}

{\small ಅರ್ಜುನ, ಈ ಮೂರು ತಮೋದ್ವಾರಗಳಿಂದ ಬಿಡಲ್ಪಟ್ಟ ಮನುಷ್ಯ, ತನ್ನ ಶ್ರೇಯಸ್ಸನ್ನು ಮಾಡಿಕೊಳ್ಳುತ್ತಾನೆ. ಅದರಿಂದ ಪರಮ ಗತಿಯನ್ನು ಪಡೆಯುತ್ತಾನೆ.}

ಕಾಮ, ಕ್ರೋಧ, ಲೋಭ ಈ ಮೂರು ಗುಣಗಳೇ ನಮ್ಮನ್ನು ಸಂಸಾರಕ್ಕೆ ಕಟ್ಟಿಹಾಕುವುವು. ಇತರ ದುರ್ಗುಣಗಳೆಲ್ಲ ಇದರ ಮರಿಗಳು. ಯಾವಾಗ ಮೂರು ಮಹಾದುರ್ಗುಣಗಳನ್ನು ನಾವು ನಿಗ್ರಹಿಸು ತ್ತೇವೆಯೊ, ಉಳಿದವುಗಳೆಲ್ಲ ತಮಗೆ ತಾವೇ ನಾಶವಾಗುತ್ತವೆ. ನಾವೊಂದು ಕೊಂಬೆಯನ್ನು ಕತ್ತರಿಸಿ ಹಾಕಿದರೆ, ಅದರಲ್ಲಿರುವ ಎಲೆಗಳೆಲ್ಲ ಹೇಗೆ ಕ್ರಮೇಣ ಒಣಗಿ ಹೋಗುವುವೋ ಹಾಗೆ ಇತರ ದುರ್ಗುಣಗಳೆಲ್ಲ ನಾಶವಾಗುತ್ತವೆ. ಈ ಗುಣಗಳೇ ಎಲ್ಲಾ ಜೀವಿಗಳನ್ನು ಒಂದು ಪ್ರಮಾಣದಲ್ಲಿ ಮುತ್ತಿವೆ. ಕೆಲವರಲ್ಲಿ ಜಾಸ್ತಿ ಇದೆ, ಮತ್ತೆ ಕೆಲವರಲ್ಲಿ ಬಹಳ ಕಡಿಮೆ ಇದೆ. ಹೆಚ್ಚಾಗಿ ಇರುವವರನ್ನು ಆಸುರೀ ಪ್ರವೃತ್ತಿಯವರು ಎಂದು ಹೇಳುತ್ತೇವೆ. ಕಡಿಮೆ ಇರುವವರನ್ನು ದೈವೀ ಪ್ರವೃತ್ತಿಯವರು ಎಂದು ಹೇಳುತ್ತೇವೆ. ಆಸುರೀ ಪ್ರವೃತ್ತಿಯನ್ನು ನಾವು ಪ್ರಯತ್ನ ಪೂರ್ವಕ ಕಡಿಮೆ ಮಾಡಿಕೊಳ್ಳ ಬೇಕು, ದೈವೀ ಸ್ವಭಾವವನ್ನು ವೃದ್ಧಿಮಾಡಿಕೊಳ್ಳಬೇಕು. ಇದು ಪ್ರಯತ್ನದಿಂದ ಸಾಧ್ಯ. ಎಂತಹ ಆಸುರೀ ಸ್ವಭಾವದ ಮನುಷ್ಯನಾದರೂ ತನ್ನಲ್ಲಿ ಆ ಪ್ರವೃತ್ತಿಗಳು ಇವೆ, ಅವುಗಳಿಂದ ನಾನು ಪಾರಾಗಬೇಕೆಂದು ಸಾಕಷ್ಟು ಪ್ರಯತ್ನಪಟ್ಟರೆ ಅದರಿಂದ ಪಾರಾಗಲು ಸಾಧ್ಯ. ಅವು ಒಂದೇ ಸಲ ನಮ್ಮನ್ನು ಬಿಡುವುದಿಲ್ಲ. ಕ್ರಮೇಣ ಅವುಗಳಿಂದ ಪಾರಾಗಬೇಕಾಗಿದೆ. ಆಗಲೆ ನಮ್ಮ ಶ್ರೇಯಸ್ಸಿಗೆ ನಾವು ಕಾರಣಭೂತರಾಗುತ್ತೇವೆ.

ದೈವೀ ಪ್ರಕೃತಿಯನ್ನು ನಾವು ಹೆಚ್ಚು ಹೆಚ್ಚು ರೂಢಿಸಿಕೊಳ್ಳಬೇಕು. ಇದರಿಂದಲೇ ನಮಗೆ ಪರಮ ಗತಿ ಪ್ರಾಪ್ತವಾಗುವುದು. ಪರಮ ಗತಿ ಎಂದರೆ ಭಗವಂತನ ಸಾಂನಿಧ್ಯವನ್ನು ಮುಟ್ಟುವುದು. ಇನ್ನು ಮೇಲೆ ಬದ್ಧಜೀವಿಗಳಾಗಿ ನಾವು ಪ್ರಪಂಚಕ್ಕೆ ಹಿಂತಿರುಗುವುದಿಲ್ಲ. ಹೋದರೆ ಒಂದೇ ಸಲ ಹೋಗುವೆವು. ಹೋಗಿಬಿಟ್ಟು ಬರುತ್ತೇನೆ ಎಂದು ಹೇಳಿ ಪ್ರಪಂಚವನ್ನು ಬಿಡುವುದಿಲ್ಲ. ನಾವು ಸಂಸಾರದ ಹುರಿವ ಬಾಂಡಲೆಯಿಂದ ಹೋಗುವುದೇ ಪರಮಗತಿ.

\begin{verse}
ಯಃ ಶಾಸ್ತ್ರವಿಧಿಮುತ್ಸೃಜ್ಯ ವರ್ತತೇ ಕಾಮಕಾರತಃ~।\\ನ ಸ ಸಿದ್ಧಿಮವಾಪ್ನೋತಿ ನ ಸುಖಂ ನ ಪರಾಂ ಗತಿಮ್ \versenum{॥ ೨೩~॥}
\end{verse}

{\small ಯಾರು ಶಾಸ್ತ್ರವಿಧಿಯನ್ನು ಬಿಟ್ಟು, ಸ್ವೇಚ್ಛಾನುಸಾರ ಭೋಗದಲ್ಲಿ ಮಗ್ನನಾಗಿರುವನೋ ಅವನಿಗೆ ಸಿದ್ಧಿಯೂ ಆಗದು, ಸುಖವೂ ಸಿಕ್ಕದು, ಉತ್ತಮಗತಿಯೂ ಬಾರದು.}

ಶಾಸ್ತ್ರ ಒಂದು ಕೈ ಮರದಂತೆ, ಭಗವಂತನ ಕಡೆಗೆ ಹೋಗಬೇಕಾದರೆ ದಾರಿ ಯಾವುದು ಎಂಬುದನ್ನು ಹೇಳುವುದು. ಈ ಪ್ರಪಂಚದಲ್ಲೆ ಸುಖವಾಗಿರಬೇಕಾದರೆ ಅದಕ್ಕೂ ಒಂದು ದಾರಿಯಿದೆ. ಅದೇ ಅರ್ಥ ಮತ್ತು ಕಾಮಗಳನ್ನು ಧಾರ್ಮಿಕವಾಗಿ ಪಡೆದುಕೊಂಡು ಅನುಭವಿಸುವುದು, ಕೊನೆಗೆ ಮೋಕ್ಷಕ್ಕೆ ಅಣಿಯಾಗುವುದು. ಇದನ್ನೆಲ್ಲ ಶಾಸ್ತ್ರದಲ್ಲಿ ವಿವರಿಸುವರು. ಯಾವ ಧರ್ಮಕ್ಕೇ ಸೇರಿದ ಶಾಸ್ತ್ರವಾಗಲಿ ಬೇರೆ ಬೇರೆ ಭಾಷೆಯಲ್ಲಿ ಒಂದೇ ಸತ್ಯವನ್ನು ಬೋಧಿಸುವುದು: ದೇವರನ್ನು ಪ್ರೀತಿಸ ಬೇಕು, ಈ ಪ್ರಪಂಚವನ್ನು ತ್ಯಜಿಸಬೇಕು. ಈ ಪ್ರಪಂಚಕ್ಕೆ ನಮ್ಮನ್ನು ಬಂಧಿಸುವ ಸಂಬಂಧದಿಂದ ದೂರವಾಗಿರಬೇಕು ಎಂದು ಹೇಳುವುದು. ಯಾವ ಧರ್ಮವನ್ನು ತೆಗೆದುಕೊಂಡರೂ ನೀನು ಸ್ವೇಚ್ಛಾ ಚಾರಿಯಾಗು ಎನ್ನುವುದಿಲ್ಲ. ನಿನ್ನ ಆಸೆ ಆಕಾಂಕ್ಷೆಗಳನ್ನು ನಿಗ್ರಹಿಸು ಎನ್ನುವುದು. ಆದರೆ ಯಾರು ಶಾಸ್ತ್ರವನ್ನು ಅಲ್ಲಗಳೆಯುವರೊ ಮತ್ತು ಅದಕ್ಕೆ ವಿರೋಧವಾಗಿ ಕಾಮನೆಗಳನ್ನು ತೃಪ್ತಿಪಡಿಸುವುದ ರಲ್ಲಿ ಮಗ್ನರಾಗಿರುವರೋ ಅವರಿಗೆ ಸಿದ್ಧಿಯಾಗುವುದಿಲ್ಲ, ಎಂದರೆ ಜೀವನದಲ್ಲಿ ಒಂದು ಭವ್ಯ ಆದರ್ಶ ಅವರಿಗೆ ದೊರಕುವುದಿಲ್ಲ. ಸುಮ್ಮನೆ ಹುಟ್ಟುವರು, ಸಿಕ್ಕಿದುದನ್ನು ತಿಂದು ಅನುಭವಿಸುವರು, ಒಂದು ದಿನ ಕಂತೆ ಒಗೆಯವರು. ಪ್ರಾಣಿಯಂತೆ ಇವರ ಬಾಳು. ಕೇವಲ ಇಂದ್ರಿಯ ಸುಖಾಭಿಲಾಷೆ ಯಲ್ಲಿಯೇ ಮುಳುಗಿರುವರು.

ಅವನು ಸುಖವನ್ನು ಹೊಂದುವುದಿಲ್ಲ. ಅವನು ಎಂದಿಗೂ ಶ್ರೇಯಸ್ಸಿನ ಕಡೆ ಗಮನವನ್ನು ಕೊಡುವುದಿಲ್ಲ. ಇವನ ಗುರಿಯೆಲ್ಲ ಪ್ರೇಯಸ್. ತಾತ್ಕಾಲಿಕದಲ್ಲಿ ಸುಖ ಸಿಕ್ಕಬೇಕು, ಇಂದ್ರಿಯಕ್ಕೆ ಅದು ಪ್ರಿಯವಾಗಿರಬೇಕು. ಇದರಿಂದ ಯಾರಿಗೆ ಏನಾದರೂ ಚಿಂತೆಯಿಲ್ಲ. ಯಾವಾಗ ಮಿತಿಮೀರಿ ಇಂದ್ರಿಯಗಳನ್ನು ತಣಿಸುವುದಕ್ಕೆ ಯತ್ನಿಸುವನೋ, ಅವನು ಸ್ವಲ್ಪ ಕಾಲದ ಮೇಲೆ ಹಲವಾರು ರೋಗ ರುಜಿನಗಳಿಗೆ ತುತ್ತಾಗುವನು. ದಿನ ಬೆಳಗಾದರೆ ಯಾವುದಾದರೂ ಒಂದು ರೋಗ ಕಾಡುತ್ತಿರುವುದು. ತನ್ನ ಇಂದ್ರಿಯ ದಾಹವನ್ನು ತಣಿಸುವುದಕ್ಕಾಗಿ ಪ್ರಕೃತಿಯ ನಿಯಮಗಳನ್ನೆಲ್ಲ, ಉಲ್ಲಂಘಿಸುವನು. ಆದರೆ ಪ್ರಕೃತಿ ಇವನನ್ನು ಬಿಡುವುದಿಲ್ಲ. ಚಕ್ರಬಡ್ಡಿ ಸುಸ್ತಿಬಡ್ಡಿಯನ್ನು ವಸೂಲಿಮಾಡುವುದು. ಅವನು ಅದಕ್ಕಾಗಿ ಸಂಕಟಪಡಬೇಕು, ದುಃಖಪಡಬೇಕು. ಅದರಂತೆಯೇ ಅವನು ತನ್ನ ಸುಖ ಸಾಧನೆಗೆ ಎಷ್ಟೋ ಜನಕ್ಕೆ ಕಷ್ಟ ಕೊಟ್ಟಿರುವನು. ಅವುಗಳೆಲ್ಲ ಇವನ ಮೇಲೆ ತಿರುಗಿ ಬರುತ್ತಿರುವುವು. ಜೀವನದಲ್ಲಿ ನಾವು ಮತ್ತೊಬ್ಬನಿಗೆ ಕಷ್ಟ ಕೊಟ್ಟಿದ್ದರೆ, ಅನ್ಯಾಯ ಮಾಡಿದ್ದರೆ, ಸುಖವಾಗಿರುವುದಕ್ಕೆ ಆಗುವುದಿಲ್ಲ. ನಾನು ಏನನ್ನು ಹೊರಗೆ ಎಸೆದಿರುವೆನೊ ಅದೇ ನನ್ನ ಮೇಲೆ ಹಿಂತಿರುಗುವುದು ಮಹಾಸ್ತ್ರದಂತೆ. ನಾನು ಗೋಡೆಗೆ ಚಂಡನ್ನು ಎಸೆದರೆ ಗೋಡೆ ನನಗೆ ಪ್ರತಿ ಎಸೆಯುವುದು. ಅದರಂತೆಯೇ ಕರ್ಮನಿಯಮ. ನಾನು ಇತರರಿಗೆ ಮಾಡಿದ್ದು ನನಗೆ ಬರುವುದು. ವಿಳಂಬವನ್ನು ನೋಡಿ ನನಗೆ ಅದು ಬರಲಾರದು ಎಂದು ಭಾವಿಸುವೆನು. ಇದೊಂದು ನಮ್ಮ ಭ್ರಮೆ. ನಾವು ಮರೆತರೂ ಪ್ರಕೃತಿ, ಭಗವಂತನ ಭಂಟ, ಅದನ್ನು ಮರೆಯುವುದಿಲ್ಲ. ಅದು ನನ್ನ ಸುಖದ ಸೋಪಿನ ಗುಳ್ಳೆಯನ್ನು ಊದಿ ಒಡೆಯುವುದು. ಪಡಬಾರದ ಯಾತನೆಗಳನ್ನೆಲ್ಲ ಅನಂತರ ಪಡಬೇಕಾಗುವುದು.

ಇಂತಹ ಮನುಷ್ಯ ಜೀವನದ ಪರಮಗತಿಯಾದ ಪರಮಾತ್ಮನನ್ನು ಮುಟ್ಟುವುದಿಲ್ಲ. ಸಂಸಾರ ಚಕ್ರದಲ್ಲಿ ಸಿಕ್ಕಿ ನರಳುವನು. ನಾವು ಬದುಕಿರುವ ಪರಿಯಂತರ, ದೇಹ ಇಂದ್ರಿಯಗಳನ್ನು ಸತ್ಯವೆಂದು ಅವನ್ನು ಪೋಷಿಸಿದೆವು. ಅದರ ಹಿಂದೆ ಇರುವ ಪರಮಾತ್ಮನ ವಾಣಿಗೆ ಕಿವಿಗೊಡಲಿಲ್ಲ. ದೇವರ ಕಡೆ ಕರೆದುಕೊಂಡು ಹೋಗುವ ಪುಣ್ಯಕರ್ಮಗಳನ್ನು ಮಾಡಲಿಲ್ಲ. ಪುನಃ ಪುನಃ ಪ್ರಪಂಚಕ್ಕೆ ಹಿಂತಿರುಗಿ ಬರುವ ಪಾಪ ಕಾರ್ಯಗಳನ್ನೇ ಮಾಡಿದೆವು. ಸಾಲ ಮಾಡಿಕೊಂಡು ಹೊರಟರೆ ಅದನ್ನು ತೀರಿಸಲು ಹಿಂತಿರುಗಿ ಬರಬೇಕು. ಅನೇಕ ವೇಳೆ ನಾವು ಪುನಃ ಬಂದ ಮೇಲೆ ಸಾಲವನ್ನು ತೀರಿಸುವುದಿರಲಿ, ಇನ್ನೂ ಹೆಚ್ಚು ಸಾಲವನ್ನು ಮಾಡಿಕೊಳ್ಳುತ್ತೇವೆ.

\begin{verse}
ತಸ್ಮಾಚ್ಛಾಸ್ತ್ರಂ ಪ್ರಮಾಣಂ ತೇ ಕಾರ್ಯಾಕಾರ್ಯವ್ಯವಸ್ಥಿತೌ~।\\ಜ್ಞಾತ್ವಾ ಶಾಸ್ತ್ರವಿಧಾನೋಕ್ತಂ ಕರ್ಮ ಕರ್ತುಮಿಹಾರ್ಹಸಿ \versenum{॥ ೨೪~॥}
\end{verse}

{\small ಆದಕಾರಣ ಕಾರ್ಯ ಅಕಾರ್ಯಗಳ ನಿರ್ಣಯ ಮಾಡುವಾಗ ನೀನು ಶಾಸ್ತ್ರವನ್ನು ಪ್ರಮಾಣವೆಂದು ಭಾವಿಸಬೇಕು. ಶಾಸ್ತ್ರವಿಧಿಯನ್ನು ಅರಿತು ಈ ಲೋಕದಲ್ಲಿ ಕರ್ಮವನ್ನು ಮಾಡಲು ಅರ್ಹನಾಗುತ್ತೀಯೆ.}

ಏನನ್ನು ಮಾಡಬೇಕು, ಏನನ್ನು ಬಿಡಬೇಕು ಎಂಬ ವಿಷಯದಲ್ಲಿ ನಾವು ಶಾಸ್ತ್ರವನ್ನು ಪ್ರಮಾಣ ವನ್ನಾಗಿ ತೆಗೆದುಕೊಳ್ಳಬೇಕು. ಅಲ್ಲಿ ಪ್ರತಿಯೊಂದು ವರ್ಣಕ್ಕೆ ಸೇರಿದವರು, ಮತ್ತು ಪ್ರತಿಯೊಂದು ಆಶ್ರಮಕ್ಕೆ ಸೇರಿದವರು ಯಾವ ಯಾವ ಕಾರ್ಯಗಳನ್ನು ಮಾಡಬೇಕು ಎಂಬುದನ್ನು ವಿಧಿಯ ಮೂಲಕ ವಿವರಿಸುವರು. ಅದರಂತೆಯೇ ಏನನ್ನು ಬಿಡಬೇಕು ಎಂಬುದನ್ನು ನಿಷೇಧದ ಮೂಲಕ ಬೇಡ ಎನ್ನುವರು. ನಾವು ಎಲ್ಲವನ್ನು ನಿರ್ಧರಿಸುವ ಮಟ್ಟಕ್ಕೆ ಏರಿಲ್ಲ. ಕೇವಲ ತತ್ಕಾಲದಲ್ಲಿ ನಮಗೆ ಯಾವುದು ಪ್ರಿಯವಾಗುವುದೊ ಅದನ್ನು ಮಾಡಬಾರದು. ಅದು ಶಾಸ್ತ್ರಕ್ಕೆ ವಿರೋಧವಾಗಿದ್ದರೆ ಅದನ್ನು ಬಿಡಬೇಕು. ಯಾರು ಶಾಸ್ತ್ರವನ್ನು ಬರೆದರೋ ಅವರು ನಮ್ಮನ್ನು ತಪ್ಪುದಾರಿಗೆ ಎಳೆಯುವುದಕ್ಕೆ ಇದನ್ನು ಬರೆಯಲಿಲ್ಲ. ಅವರು ತುಂಬಾ ಮೇಧಾವಿಗಳು, ಮತ್ತು ಅನುಭಾವಿಗಳು. ಅವರಿಗೆ ತಮ್ಮ ಸ್ವಾರ್ಥವನ್ನು ಬಿಟ್ಟು ಬಹುಜನರಿಗೆ ಯಾವುದು ಹಿತವಾಗುವುದು, ಬಹುಜನರಿಗೆ ಯಾವುದು ಸುಖವಾಗುವುದು ಎಂಬುದನ್ನು ನಿಷ್ಕರ್ಷಿಸಬಲ್ಲ ಶಕ್ತಿ ಇತ್ತು. ಅವರ ಅನಂತರ ಬರುವವರಿಗೆ ಸಹಾಯವಾಗಲಿ ಎಂದು ತಮ್ಮ ಅನುಭವವನ್ನು ಶಾಸ್ತ್ರದಲ್ಲಿ ಹೇಳಿರುವರು. ನಾವು ನಮ್ಮ ಸ್ವಂತ ಅನುಭವದಿಂದ ಅದನ್ನು ಮೀರಿ ಹೋಗಲು ಶಕ್ತಿ ಬರುವ ತನಕ ಅದನ್ನು ಅನುಸರಿಸುವುದು ಮೇಲು. ಜೀವನದಲ್ಲಿ ಎಲ್ಲಾ ಕಾರ್ಯಕ್ಷೇತ್ರದಲ್ಲಿಯೂ ನಮಗಿಂತ ಹಿಂದೆ ಜನರು ಸಾಧಿಸಿದ ಅನುಭವಗಳನ್ನು ಸ್ವೀಕರಿಸುತ್ತೇವೆ. ಅದು ರಸಾಯನಶಾಸ್ತ್ರ ಆಗಬಹುದು, ಭೌತಿಕ ಶಾಸ್ತ್ರ ಆಗಬಹುದು, ಯಾವುದಾದರೂ ಚಿಂತೆಯಿಲ್ಲ–ಮೊದಲು ಅಲ್ಲಿ ಪಾರಂಗತರಾಗಬೇಕಾಗಿದೆ. ಅವರಿಗಿಂತ ಹೆಚ್ಚು ಬೆಳೆದಾಗ ಅದನ್ನು ಮೀರಿ ಹೋಗಬಹುದು. ಅದಕ್ಕಿಂತ ಮುಂಚೆ ಅಲ್ಲ.

ಅರ್ಜುನ ಯುದ್ಧ ಬಿಡಬೇಕು ಎಂದು ಆಲೋಚಿಸುತ್ತಿದ್ದ. ಆದರೆ ಕ್ಷತ್ರಿಯನ ಕರ್ತವ್ಯ, ಧರ್ಮ ಯುದ್ಧದಲ್ಲಿ ಭಾಗವಹಿಸಬೇಕಾದುದು. ಶಾಸ್ತ್ರ ಇದನ್ನು ಕ್ಷತ್ರಿಯನ ಕರ್ತವ್ಯದಲ್ಲಿ ಹೇಳುವುದು. ಇವನು ದೊಡ್ಡ ಸಾತ್ವಿಕ ವ್ಯಕ್ತಿಯಾಗಿ ಜ್ಞಾನಿಯಾಗಿದ್ದರೆ ಅದು ಬೇರೆ ಮಾತು. ಆಗ ಮಾಡುವ ಕೆಲಸದಿಂದ ಇವನಿಗೆ ಲಾಭವಿಲ್ಲ, ಬಿಟ್ಟರೆ ನಷ್ಟವಿಲ್ಲ. ಆದರೆ ಅರ್ಜುನ ಆ ಸ್ಥಿತಿಗೆ ಏರಿಲ್ಲ. ಅವನು ತನ್ನ ಕರ್ತವ್ಯವನ್ನು ಮಾಡಬೇಕಾಗಿದೆ. ಆದರೆ ಆ ಕೆಲಸವನ್ನು ಮಾಡುವಾಗ ಬೇಕಾದರೆ ದೃಷ್ಟಿಯನ್ನು ಬದಲಾಯಿಸಿಕೊಳ್ಳಬಹುದು. ಅದು ಇವನ ಕೈಯಲ್ಲಿದೆ. ಶ‍್ರೀಕೃಷ್ಣ ಅರ್ಜುನನಿಗೆ ಬೋಧಿಸಿದ್ದೂ ಇದನ್ನೇ. ಜಯವೋ ಅಪಜಯವೋ ನೀನು ಮಾಡುವ ಕೆಲಸವನ್ನು ಮಾಡು. ಫಲದ ಮೇಲೆ ಆಸಕ್ತಿ ಇಟ್ಟುಕೊಳ್ಳಬೇಡ, ಭಗವಂತನ ಕೈಯಲ್ಲಿ ಒಂದು ನಿಮಿತ್ತವಾಗು ಎನ್ನುವನು. ಇವುಗಳೆಲ್ಲ ಅರ್ಜುನ ರೂಢಿಸಿಕೊಳ್ಳಬೇಕಾದ ದೃಷ್ಟಿಯನ್ನು ಹೇಳುತ್ತವೆಯೆ ಹೊರತು ಕೆಲಸವನ್ನು ಬಿಡು ಎನ್ನುವುದಿಲ್ಲ.

