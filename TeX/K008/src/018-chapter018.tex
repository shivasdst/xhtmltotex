
\chapter{ಮೋಕ್ಷ ಸಂನ್ಯಾಸಯೋಗ}

ಅರ್ಜುನ ಶ‍್ರೀಕೃಷ್ಣನನ್ನು ಕೇಳುತ್ತಾನೆ.

\begin{verse}
ಸಂನ್ಯಾಸಸ್ಯ ಮಹಾಬಾಹೋ ತತ್ತ್ವಮಿಚ್ಛಾಮಿ ವೇದಿತುಮ್~।\\ತ್ಯಾಗಸ್ಯ ಚ ಹೃಷೀಕೇಶ ಪೃಥಕ್ಕೇಶಿನಿಷೂದನ \versenum{॥ ೧~॥}
\end{verse}

{\small ಹೃಷೀಕೇಶ, ಸಂನ್ಯಾಸ ಮತ್ತು ತ್ಯಾಗದ ತತ್ತ್ವವನ್ನು ಬೇರೆಬೇರೆಯಾಗಿ ತಿಳಿದುಕೊಳ್ಳಲು ಬಯಸುತ್ತೇನೆ.}

ಈ ಅಧ್ಯಾಯದಲ್ಲಿ ಗೀತೆಯಲ್ಲಿ ಇದುವರೆಗೆ ಹೇಳಿದ ಭಾವನೆಗಳನ್ನು ಉಪಸಂಹಾರ ಮಾಡು ತ್ತಾನೆ. ಆದಕಾರಣ ಇದುವರೆಗೆ ಬಂದ ಭಾವನೆಗಳೆಲ್ಲ ಪುನಃ ಬಂದುಹೋಗುವುವು. ಅರ್ಜುನನ ಮನಸ್ಸಿನಲ್ಲಿ ಪುನಃ ಸಂನ್ಯಾಸ ಮತ್ತು ತ್ಯಾಗದ ವಿಷಯದಲ್ಲಿ ಸಂದೇಹ ತಲೆದೋರುವುದು. ಅದಕ್ಕಾಗಿ ಪ್ರಶ್ನೆ ಮಾಡುತ್ತಾನೆ. ಇಲ್ಲಿ ಸಂನ್ಯಾಸ ಎಂದರೆ ಕೆಲಸ ಬಿಟ್ಟು ಸುಮ್ಮನಿರುವುದು, ತ್ಯಾಗ ಎಂದರೆ ಕೆಲಸ ಮಾಡಬೇಕು ಮತ್ತು ಅದರ ಫಲಗಳಿಗೆ ಆಸಕ್ತನಾಗಬಾರದು. ಈಗಲೂ ಸಾಧ್ಯವಾದರೆ ಯುದ್ಧದಿಂದ ತಪ್ಪಿಸಿಕೊಂಡು ಹೋಗಲು ಏನಾದರೂ ಸಾಧ್ಯವಿದೆಯೇ ಎಂದು ಚಿಂತಿಸಿ ಈ ಪ್ರಶ್ನೆಯನ್ನು ಕೇಳುತ್ತಾನೆ. ಈ ಪ್ರಶ್ನೆಯನ್ನು ಸಂನ್ಯಾಸಯೋಗವೆಂಬ ಐದನೆಯ ಅಧ್ಯಾಯದಲ್ಲಿ ಆಗಲೇ ಕೇಳಿರುವನು. ಕರ್ಮಸಂನ್ಯಾಸ ಮತ್ತು ಕರ್ಮಯೋಗ, ಇವುಗಳೆರಡರಲ್ಲಿ ಯಾವುದು ಮೇಲು ಎಂದು ಕೇಳಿದಾಗ ಶ‍್ರೀಕೃಷ್ಣ ಎರಡೂ ಒಂದೇ ಗುರಿಯೆಡೆಗೆ ಒಯ್ಯುವುವು, ಆದರೆ ಕರ್ಮಸಂನ್ಯಾಸ ಎಲ್ಲರಿಗೂ ಮೊದಲೇ ಸಿಕ್ಕುವುದಿಲ್ಲ. ಮಾನವರಲ್ಲಿ ಹಲವಾರು ಸಂಸ್ಕಾರಗಳಿವೆ. ಅವುಗಳೆಲ್ಲ ಕ್ಷಯವಾಗದ ಹೊರತು ಕರ್ಮವನ್ನು ಬಿಡುವುದಕ್ಕೆ ಆಗುವುದಿಲ್ಲ. ನಾವು ಬಿಟ್ಟರೂ ಬೇರೊಂದು ವಿಧವಾಗಿ ಇದು ನಮ್ಮನ್ನು ಹಿಡಿಯುವುದು. ಆದಕಾರಣ ಕರ್ಮಯೋಗದ ಮೂಲಕ ಹೋಗುವುದು ಮೇಲು ಎಂದು ಹಿಂದೆಯೇ ಹೇಳಿದ್ದ. ಆದರೂ ಗೀತೆಯ ಉಪದೇಶ ಇನ್ನೇನು ಕೊನೆಗಾಣುತ್ತಿದೆ. ಈಗ ಪುನಃ ಇನ್ನೊಮ್ಮೆ ಅದನ್ನು ಚೆನ್ನಾಗಿ ತಿಳಿದುಕೊಳ್ಳೋಣ ಎಂದು ಕೇಳುತ್ತಾನೆ. ಏಕೆಂದರೆ ಅರ್ಜುನನ ಮನಸ್ಸಿನಲ್ಲಿ ಇದ್ದ ಎರಡು ಮುಖ್ಯವಾದ ಸಮಸ್ಯೆಯೇ ಕೆಲಸವನ್ನು ಮಾಡಬೇಕೆ, ಬಿಡಬೇಕೆ ಎಂಬುದು. ಬಿಡುವುದಕ್ಕೆ ಸಾಧ್ಯವಾದರೆ ಅದನ್ನು ಈಗಲಾದರೂ ಬಿಡುವುದಕ್ಕೆ ಸಿದ್ಧವಾಗಿರು ವನು. ಗುರುಹಿರಿಯರ ಕೊಲೆಯನ್ನು ಅವನು ಇಚ್ಛಿಸುವುದಿಲ್ಲ. ಆದರೆ ಅರ್ಜುನ ಎಷ್ಟು ತಪ್ಪಿಸಿಕೊಳ್ಳು ವುದಕ್ಕೆ ಯತ್ನಿಸಿದರೂ ಕೃಷ್ಣ ಅವನನ್ನು ಹಾಗೆಯೇ ಬಿಡುವಂತಿಲ್ಲ. ರಣನದಿಯಲ್ಲಿ ಸ್ನಾನವನ್ನು ಮಾಡಿಸಿಯೇ ಮಾಡಿಸುತ್ತಾನೆ.

ಶ‍್ರೀಕೃಷ್ಣ ಹೇಳುತ್ತಾನೆ:

\begin{verse}
ಕಾಮ್ಯಾನಾಂ ಕರ್ಮಣಾಂ ನ್ಯಾಸಂ ಸಂನ್ಯಾಸಂ ಕವಯೋ ವಿದುಃ~।\\ಸರ್ವಕರ್ಮಫಲತ್ಯಾಗಂ ಪ್ರಾಹುಸ್ತ್ಯಾಗಂ ವಿಚಕ್ಷಣಾಃ \versenum{॥ ೨~॥}
\end{verse}

{\small ಕೆಲವು ಬುದ್ಧಿವಂತರು, ಕಾಮ್ಯ ಕರ್ಮಗಳ ತ್ಯಾಗವನ್ನು ಸಂನ್ಯಾಸ ಎಂದು ಕರೆಯುತ್ತಾರೆ. ಮತ್ತೆ ಕೆಲವು ಜ್ಞಾನಿಗಳು ಸರ್ವ ಕರ್ಮಗಳ ಫಲತ್ಯಾಗವನ್ನು ತ್ಯಾಗ ಎಂದು ಕರೆಯುತ್ತಾರೆ.}

ಇಲ್ಲಿ ಶ‍್ರೀಕೃಷ್ಣ ತನ್ನ ಅಭಿಪ್ರಾಯವನ್ನು ಕೊಡುವುದಕ್ಕೆ ಮುಂಚೆ, ಆಗ ಪ್ರಚಲಿತವಾದ ಎರಡು ಅಭಿಪ್ರಾಯಗಳನ್ನು ಕೊಡುತ್ತಾನೆ. ಅನಂತರ ತ್ಯಾಗ ಎಂದರೆ ತನ್ನ ಅಭಿಪ್ರಾಯ ಏನು ಎಂಬುದನ್ನು ವಿವರಿಸುತ್ತಾನೆ. ಎಂದಿನಂತೆ ಶ‍್ರೀಕೃಷ್ಣನ ಅಭಿಪ್ರಾಯ ಯಾವಾಗಲೂ ಪೂರ್ಣಗ್ರಾಹಿಯಾದದ್ದು. ಭಿನ್ನ ಭಿನ್ನ ಅಭಿಪ್ರಾಯಗಳಿಗೆಲ್ಲ ಒಂದು ಸ್ಥಾನವನ್ನು ಕೊಟ್ಟು ಬೇರೊಂದು ದೃಷ್ಟಿಯಿಂದ ಅವುಗಳಲ್ಲಿ ಸಾಮರಸ್ಯ ಬರುವಂತೆ ಮಾಡುತ್ತಾನೆ.

ಕೆಲವರು ಕಾಮ್ಯಕರ್ಮಗಳನ್ನು ಬಿಡುವುದನ್ನು ಸಂನ್ಯಾಸ ಎಂದು ಹೇಳುತ್ತಾರೆ. ಕಾಮನೆಗಾಗಿ ಪ್ರಪಂಚದ ಗೋಜಿಗೆ ಸಿಕ್ಕಿಕೊಳ್ಳಬಾರದು. ಆದಕಾರಣ ಯಾವ ಕರ್ಮ ಕಾಮನೆಯನ್ನು ಇಟ್ಟು ಕೊಂಡಿರುವುದೋ, ಆ ಕರ್ಮಗಳನ್ನು ಬಿಡುತ್ತಾನೆ. ಪುತ್ರಕಾಮೇಷ್ಠಿ, ರಾಜಸೂಯ, ಅಶ್ವಮೇಧ ಮುಂತಾದ ಹಲವು ಯಾಗಯಜ್ಞಗಳನ್ನು ಮಾಡುವುದನ್ನು ಬಿಡುತ್ತಾನೆ. ಇವುಗಳೆಲ್ಲ ಇಹಲೋಕದಲ್ಲಿ ಚೆನ್ನಾಗಿ ಬಾಳುವುದಕ್ಕೆ ಆಗಿದೆ. ಸಂನ್ಯಾಸಿ ಇವುಗಳನ್ನು ತ್ಯಜಿಸುತ್ತಾನೆ. ಅದೊಂದೇ ಅಲ್ಲ. ಈ ಪ್ರಪಂಚದಲ್ಲಿ, ಕೀರ್ತಿ, ಲಾಭ, ಅಧಿಕಾರ, ವೈರತ್ವ ಇವುಗಳಿಗಾಗಿ ಮಾಡುವ ಕರ್ಮವನ್ನು ಅವನು ಬಿಡುತ್ತಾನೆ. ಅವನಿಗೆ ಇವುಗಳಾವುವೂ ಬೇಕಾಗಿಲ್ಲ. ಇವುಗಳೆಲ್ಲ ಸತ್ತ್ವವಿಲ್ಲದುವು, ನೀರುಗುಳ್ಳೆ ಗಳಂತೆ. ನಮ್ಮನ್ನು ಈ ಪ್ರಪಂಚಕ್ಕೆ ಬಿಗಿಯುವ ಗೂಟಗಳು ಇವು ಎಂಬುದನ್ನು ಚೆನ್ನಾಗಿ ಅರಿತಿರುವನು.

ಆದರೆ ಅವನು ನಿತ್ಯಕರ್ಮವನ್ನು ಮಾಡುತ್ತಾನೆ. ನಿತ್ಯಕರ್ಮದ ಮೇಲೆಯೇ ಸೃಷ್ಟಿ ನಿಂತಿರುವುದು. ಪ್ರತಿಯೊಂದು ವರ್ಣದವರು, ಆಶ್ರಮದವರು, ಸಮಷ್ಟಿಯ ಹಿತಕ್ಕೆ ತಮ್ಮ ಪಾಲಿನ ಕರ್ತವ್ಯವನ್ನು ಧಾರೆ ಎರೆಯಬೇಕಾಗಿದೆ. ಆಗಲೆ ಸಮಾಜ ಭದ್ರವಾಗಿರಬೇಕಾದರೆ. ತನಗೆ ಫಲ ಬೇಕಾಗದೆ ಇದ್ದರೂ ಆ ಕೆಲಸವನ್ನು ಮಾಡಬೇಕಾಗಿದೆ. ಏಕೆಂದರೆ ಇತರರಿಗೆ ಅದು ಅತ್ಯಂತ ಆವಶ್ಯಕವಾಗಿ ಬೇಕಾಗಿದೆ. ರೈತ ಬೆಳೆಯನ್ನು ಬೆಳೆಯಬೇಕಾಗಿದೆ. ಇತರರು ಅದರ ಆಧಾರದ ಮೇಲೆ ಜೀವಿಸಬೇಕಾಗಿದೆ. ವೈಶ್ಯ ಒಂದು ಕಡೆ ಬೆಳೆದುದನ್ನು ಮತ್ತೊಂದು ಕಡೆ ರವಾನಿಸಬೇಕಾಗಿದೆ. ಕ್ಷತ್ರಿಯ ಹೊರಗಿನಿಂದ ಶತ್ರುಗಳು ಬರದಂತೆ ಒಳಗೆ ಅನ್ಯಾಯ ಅಧರ್ಮಗಳಾಗದಂತೆ ನೋಡಿಕೊಳ್ಳಬೇಕಾಗಿದೆ. ಬ್ರಾಹ್ಮಣ ತನ್ನ ಜ್ಞಾನವನ್ನು ಇತರರಿಗೆ ಕೊಡಬೇಕಾಗಿದೆ. ಅವನಿಗೆ ಕೀರ್ತಿ ಬೇಡ, ಲಾಭ ಬೇಡ, ದ್ವೇಷ ಬೇಡ ಎಂದರೆ ಅವನು ಮಾಡುವ ಕರ್ಮವನ್ನು ಬಿಡುವುದಿಲ್ಲ. ಆಗ ದೊಡ್ಡದೊಂದು ಲೋಪವಾಗುವುದು. ಇವನು ತಪ್ಪಿತಸ್ಥನಾಗುತ್ತಾನೆ.

ಇನ್ನೊಂದು ಗುಂಪಿನವರು ಇದ್ದಾರೆ. ಅವರು ನೀವು ಯಾವ ಕೆಲಸವನ್ನಾದರೂ ಮಾಡಿ; ಆದರೆ ಅದರ ಹಿಂದೆ ಇರುವ ಪ್ರತಿಫಲಾಪೇಕ್ಷೆಯನ್ನು ಬಿಡಿ ಎನ್ನುತ್ತಾರೆ. ಇಲ್ಲಿ ಅದು ಕಾಮ್ಯಕರ್ಮ ಆಗಬಹುದು, ಯಾವುದನ್ನು ಮಾಡಿದರೂ ಚಿಂತೆಯಿಲ್ಲ. ಅದರಿಂದ ನಮಗೆ ಯಾವ ಫಲ ಬೇಡ ಎಂದರೆ ಅದೇ ನಿಜವಾದ ತ್ಯಾಗ ಎನ್ನುವರು. ಇಲ್ಲಿ ಅವರು ಕರ್ಮವನ್ನು ಚೆನ್ನಾಗಿ ವಿಭಜನೆ ಮಾಡಿರುವರು. ಕರ್ಮ ನಮ್ಮನ್ನು ಬಂಧನಕ್ಕೆ ಗುರಿ ಮಾಡುವುದೇ? ಕರ್ಮ ಅಲ್ಲ ಕಟ್ಟಿ ಹಾಕುವುದು. ಅದರ ಹಿಂದೆ ಇರುವ ಫಲಾಪೇಕ್ಷೆ. ಅದು ಚೇಳಿನ ಬಾಲದಲ್ಲಿರುವ ವಿಷದ ಕೊಂಡಿಯಂತೆ. ಹಾವಿನ ಬಾಯಿಯಲ್ಲಿರುವ ವಿಷದ ಹಲ್ಲಿನಂತೆ. ಯಾವಾಗ ಒಬ್ಬ ಅದನ್ನು ತೆಗೆದು ಹಾಕುವನೋ, ಅವನು ಎಷ್ಟು ಬೇಕಾದರೂ ಕರ್ಮ ಮಾಡಬಹುದು, ಅದರಿಂದ ಬಾಧಿತನಾಗುವುದಿಲ್ಲ. ಯಾವಾಗ ಒಬ್ಬ ಫಲದ ಅಪೇಕ್ಷೆಯನ್ನು ಬಿಡುತ್ತಾನೆಯೊ ಅವನು ಕಾಮ್ಯಕರ್ಮವನ್ನು ಹೇಗೆ ಮಾಡುತ್ತಾನೆ? ಸ್ವರ್ಗಕ್ಕೆ ಹೋಗಬೇಕಾಗಿಲ್ಲ, ಇಲ್ಲಿ ಏನೂ ಬೇಕಾಗಿಲ್ಲ ಎನ್ನುವವನು ರಾಜಸೂಯ ಪುತ್ರಕಾಮೇಷ್ಠಿ ಯಾಗ ಗಳನ್ನು ಮಾಡುವುದೇ ಇಲ್ಲ. ಆತನು ಮಾಡುವ ಕರ್ಮವೇ, ಆತ್ಮಕಲ್ಯಾಣಕ್ಕೆ ಕಾರಣವಾದುದು, ಲೋಕಕಲ್ಯಾಣಕ್ಕೆ ಕಾರಣವಾದುದು. ಅವನೆಂದಿಗೂ ಸುಮ್ಮನಿರುವುದಿಲ್ಲ. ಎಲ್ಲಿಯವರೆಗೆ ಒಬ್ಬ ಬದುಕಿರುವನೊ, ಅಲ್ಲಿಯವರೆಗೆ ಒಬ್ಬ ಏನಾದರೂ ಕೆಲಸವನ್ನು ಮಾಡುತ್ತಲೇ ಇರಬೇಕಾಗಿದೆ. ಮಾಡುವ ಕೆಲಸದಲ್ಲಿ, ಮತ್ತು ಅದರ ಹಿಂದಿರುವ ದೃಷ್ಟಿಯಲ್ಲಿ ಮಾತ್ರ ವ್ಯತ್ಯಾಸ ಮಾಡಿಕೊಳ್ಳ ಬಹುದು. ಆದಕಾರಣ ಅವನು ತನ್ನ ಪಾಲಿಗೆ ಬಂದ ಕರ್ತವ್ಯಗಳನ್ನು ಮಾಡುತ್ತಾನೆ. ಪ್ರತಿಫಲವನ್ನು ಬಯಸುವುದಿಲ್ಲ. ಒಬ್ಬ ಕಾಮ್ಯಕರ್ಮವನ್ನು ಬಿಡುತ್ತಾನೆ, ಇನ್ನೊಬ್ಬ ಫಲಾಪೇಕ್ಷೆಯನ್ನೆಲ್ಲ ಬಿಡು ತ್ತಾನೆ. ಎರಡು ಕಡೆಯೂ ಬಿಡುವುದು ಸಾಮಾನ್ಯವಾಗಿರುವುದು.

\begin{verse}
ತ್ಯಾಜ್ಯಂ ದೋಷವದಿತ್ಯೇಕೇ ಕರ್ಮ ಪ್ರಾಹುರ್ಮನೀಷಿಣಃ~।\\ಯಜ್ಞದಾನತಪಃಕರ್ಮ ನ ತ್ಯಾಜ್ಯಮಿತಿ ಚಾಪರೇ \versenum{॥ ೩~॥}
\end{verse}

{\small ಕರ್ಮ ದೋಷಪೂರಿತವಾಗಿರುವುದರಿಂದ ಅದನ್ನು ಬಿಡಬೇಕೆಂದು ಕೆಲವು ವಿದ್ವಾಂಸರು ಹೇಳುತ್ತಾರೆ. ಯಜ್ಞ ದಾನ ತಪಸ್ಸು ಮುಂತಾದ ಕರ್ಮಗಳನ್ನು ಬಿಡಕೂಡದೆಂದು ಕೆಲವರು ಹೇಳುತ್ತಾರೆ.}

ಹಿಂದಿನಿಂದಲೂ ಕರ್ಮವನ್ನು ಬಿಡಬೇಕು, ಮತ್ತು ಕರ್ಮವನ್ನು ಮಾಡಬೇಕು ಎಂಬ ಎರಡು ಅಭಿಪ್ರಾಯಗಳು ಬಂದಿವೆ. ಇಬ್ಬರೂ ಕೂಡ ತಮ್ಮದೇ ಸರಿ ಎಂದು ಸಾಧಿಸುವುದಕ್ಕೆ ಕಾರಣಗಳನ್ನು ಕೊಡುವರು. ಬಿಡಬೇಕು ಎನ್ನುವವರು ಹೇಳುವುದೇ ಹೀಗೆ: ಎಲ್ಲಾ ಕರ್ಮವೂ ಮನುಷ್ಯವನ್ನು ಬಂಧಿಸುವುದು. ಒಳ್ಳೆಯ ಕರ್ಮ ಒಂದು ರೀತಿ ಬಂಧಿಸುತ್ತದೆ. ಒಂದು ಚಿನ್ನದ ಸರಪಣಿ. ಅದನ್ನು ಒಳ್ಳೆಯದು ಎನ್ನುತ್ತೇವೆ. ಇನ್ನೊಂದು ಕಬ್ಬಿಣದ ಸರಪಣಿ. ಅದನ್ನು ಕೆಟ್ಟದ್ದು ಎನ್ನುತ್ತೇವೆ. ಆದಕಾರಣವೇ ಯಾವ ವಿಧವಾದ ಕರ್ಮವನ್ನು ಕೂಡಾ ಮಾಡದೆ ಇರುವುದೇ ಮೇಲು. ಒಳ್ಳೆಯ ಉದ್ದೇಶವನ್ನು ಇಟ್ಟುಕೊಂಡು ಮಾಡುತ್ತಿದ್ದರೂ, ಕಾಲಕ್ರಮೇಣ ನಮಗೆ ಗೊತ್ತಾಗದೆ, ಅದಕ್ಕೆ ಆಸಕ್ತರಾಗುತ್ತ ಬರುತ್ತೇವೆ. ಕೊನೆಗೆ ಚೆನ್ನಾಗಿ ಅದಕ್ಕೆ ಆಸಕ್ತರಾದಾಗ ಮಾತ್ರ ನಮಗೆ ಅದು ಅರಿವಾಗುವುದು. ಆಗ ಇನ್ನು ತಪ್ಪಿಸಿಕೊಳ್ಳುವುದಕ್ಕೆ ಆಗುವುದಿಲ್ಲ. ಕರ್ಮದ ಜಾರುಗುಪ್ಪೆಯ ಮೇಲೆ ಒಂದು ಸಾರಿ ಜಾರುವುದಕ್ಕೆ ಶುರುವಾದರೆ, ಮಧ್ಯ ಎಲ್ಲಿಯೂ ನಿಲ್ಲುವುದಕ್ಕೆ ಆಗುವುದಿಲ್ಲ. ಆದಕಾರಣವೇ ಕರ್ಮದ ಗೋಜಿಗೆ ಹೋಗದೆ ಇರುವುದೇ ಮೇಲು ಎಂಬುದು ಅವರ ವಾದ.

ಮತ್ತೊಬ್ಬರು ಹೂಡುವ ವಾದ ಇದು. ಒಬ್ಬ ತನಗೆ ಇಚ್ಛೆ ಬಂದರೆ ಕಾಮ್ಯಕರ್ಮಗಳನ್ನು ಬಿಡಬಹುದು. ಆದರೆ ಎಲ್ಲಿಯವರೆಗೆ ಒಬ್ಬ ಸಮಾಜದಲ್ಲಿರುವನೊ, ಅವನು ಸಮಾಜದಿಂದ ಕಂಡಂತೆ ಕಾಣದಂತೆ ಎಲ್ಲಾ ಸಹಾಯವನ್ನು ಪಡೆಯುತ್ತಿರುವನು. ಅವನು ಊಟ ಬಟ್ಟೆಯನ್ನು ಸಮಾಜದಿಂದ ಪಡೆಯುತ್ತಾನೆ. ಅದರಂತೆಯೇ ಅವನಿಗೆ ಕಳ್ಳಕಾಕರ ಕಾಟದಿಂದ ರಕ್ಷಣೆ ಸಿಕ್ಕುವುದು. ತನ್ನ ಜೀವನ ಸುಸೂತ್ರವಾಗಿ ನಡೆಯುವುದಕ್ಕೆ ಬೇಕಾಗುವುದೆಲ್ಲವನ್ನೂ ತೆಗೆದುಕೊಂಡು, ಅನಂತರ ನಾನೇನೂ ಕೊಡುವುದಿಲ್ಲ ಎಂದರೆ ಆಗದು. ಪ್ರಕೃತಿಯಿಂದ ನಾವು ಯಾವಾಗ ಒಂದು ಕೈಯಿಂದ ತೆಗೆದುಕೊಳ್ಳುತ್ತೇವೆಯೋ, ಆಗ ಕೊಡಬೇಕಾಗುವುದು. ಕೊಡುವಾಗ ಫಲಾಪೇಕ್ಷೆ ಇಲ್ಲದೆ ಕೊಡಲಿ. ಫಲಾಪೇಕ್ಷೆ ಒಬ್ಬನನ್ನು ಕಟ್ಟಿಹಾಕುವುದು, ಕರ್ಮವಲ್ಲ ಎನ್ನುವನು. ಕರ್ಮವನ್ನು ಯಾರೂ ಸಂಪೂರ್ಣ ವಾಗಿ ಬಿಟ್ಟಿರುವುದಕ್ಕೆ ಆಗುವುದಿಲ್ಲ. ಯಾರು ಕರ್ಮವನ್ನು ಬಿಟ್ಟಿರುವೆವು ಎನ್ನುತ್ತಾರೆಯೊ, ಅವರು ಎಷ್ಟು ಕಾಲವನ್ನು ಧ್ಯಾನದಲ್ಲಿ ಕಳೆಯುವುದಕ್ಕೆ ಸಾಧ್ಯ? ಉಳಿದ ಎಲ್ಲವನ್ನೂ ಅವರು ಹೇಗೆ ಉಪಯೋಗ ಮಾಡುವರು? ಕೆಲವರು ತಪಸ್ಸಿಗೆ ಹೃಷಿಕೇಶ, ಉತ್ತರಕಾಶಿ ಮುಂತಾದ ಸ್ಥಳಗಳಿಗೆ ಹೋಗುತ್ತಾರೆ. ಛತ್ರದ ಹತ್ತಿರ ಭಿಕ್ಷೆ ಪಡೆಯುವುದಕ್ಕೆ ಗಂಟೆಗಟ್ಟಲೆ ಕಳೆಯುತ್ತಾರೆ. ನೀರು ಕಾಯಿಸಿಕೊಳ್ಳುವುದಕ್ಕೆ, ಪುರಲೆ ಆಯುವುದಕ್ಕೆ, ಹಲ್ಲುಜ್ಜುವುದು, ಇತರರೊಡನೆ ಮಾತುಕತೆಯಾಡು ವುದು ಇದರಲ್ಲೆ ಬಹುಕಾಲ ಹೋಗುವುದು. ಏನಾದರೂ ಕೆಲಸ ಮಾಡಿ ಎಂದರೆ ಅದು ಬಂಧನಕ್ಕೆ ಸಿಕ್ಕಿ ಬೀಳಿಸುವುದು ಎನ್ನುವರು. ಬಿಟ್ಟಿ ಛತ್ರದಲ್ಲಿ ಭೋಜನ ಸ್ವೀಕರಿಸುವುದು ನಮ್ಮನ್ನು ಬಂಧನಕ್ಕೆ ಗುರಿ ಮಾಡುವುದಿಲ್ಲವೆ? ಆದಕಾರಣವೆ ಹಲವು ವಿಧವಾದ ಯಜ್ಞ, ಹಲವು ವಿಧವಾದ ದಾನ ಮತ್ತು ಕಾಯಿಕ ವಾಚಿಕ ಮಾನಸಿಕ ತಪಸ್ಸು ಇವುಗಳನ್ನು ಫಲಾಪೇಕ್ಷೆ ಬಿಟ್ಟು ಮಾಡಬೇಕು. ಇದು ಸಮಾಜದಲ್ಲಿ ಪ್ರತಿಯೊಬ್ಬರಿಗೂ ಅನ್ವಯಿಸುವುದು. ನಾವು ನಮ್ಮ ಯೋಗ್ಯತಾನುಸಾರ ಕೊಟ್ಟು ಪುಣದಿಂದ ವಿಮುಕ್ತರಾಗಬೇಕು.

\begin{verse}
ನಿಶ್ಚಯಂ ಶೃಣು ಮೇ ತತ್ರ ತ್ಯಾಗೇ ಭರತಸತ್ತಮ~।\\ತ್ಯಾಗೋ ಹಿ ಪುರುಷವ್ಯಾಘ್ರ ತ್ರಿವಿಧಃ ಸಂಪ್ರಕೀರ್ತಿತಃ \versenum{॥ ೪~॥}
\end{verse}

{\small ಅರ್ಜುನ, ತ್ಯಾಗದ ವಿಷಯದಲ್ಲಿ ನನ್ನ ದೃಢ ನಿಶ್ಚಯವನ್ನು ಕೇಳು. ತ್ಯಾಗ ಮೂರು ಬಗೆ ಎಂದು ಶಾಸ್ತ್ರದಲ್ಲಿ ಹೇಳಿದೆ.}

ಶ‍್ರೀಕೃಷ್ಣ ಎರಡು ಅಭಿಪ್ರಾಯಗಳ ಹಿಂದೆಯೂ ಹೋಗುತ್ತಾನೆ. ಇಬ್ಬರೂ ಕೂಡಾ ಬಿಡಬೇಕು ಎಂಬುದನ್ನು ಒಪ್ಪಿಕೊಳ್ಳುತ್ತಾರೆ. ಕಾಮ್ಯಕರ್ಮವನ್ನು ಬಿಡಬೇಕು ಎಂದು ಒಬ್ಬರು ಹೇಳುತ್ತಾರೆ, ಕರ್ಮದ ಫಲವನ್ನು ಬಿಡಬೇಕು ಎಂದು ಮತ್ತೊಬ್ಬರು ಹೇಳುತ್ತಾರೆ. ಕಾಮ್ಯಕರ್ಮವನ್ನು ಬಿಟ್ಟವನು ನಿಯತ ಕರ್ಮವನ್ನು ಮಾಡಲೇ ಬೇಕಾಗಿದೆ. ಅನಿಯತ ಕರ್ಮದಿಂದ ಬರುವ ಫಲಗಳನ್ನು ತ್ಯಾಗ ಮಾಡಬೇಕು. ಇಲ್ಲದೆ ಇದ್ದರೆ ಅದು ಅವನನ್ನು ಸಂಸಾರಕ್ಕೆ ಕಟ್ಟಿಹಾಕುವುದು. ಎರಡನೆ ಮನುಷ್ಯ ಯಾವುದನ್ನೂ ಬಿಡುವುದಿಲ್ಲ. ಆದರೂ ಫಲವನ್ನು ತ್ಯಜಿಸಬೇಕು ಎಂಬುದನ್ನು ಒಪ್ಪಿಕೊಂಡ ಹಾಗಾಯಿತು.

ಶ‍್ರೀಕೃಷ್ಣ ಈ ಬಿಡುವುದನ್ನು ಕೂಡಾ ಮತ್ತೂ ವಿಭಜನೆ ಮಾಡುತ್ತಾನೆ. ಬಿಡಬೇಕಾದುದೇನೊ ಸರಿ. ಆದರೆ ಅವನು ಯಾವ ದೃಷ್ಟಿಯಿಂದ ಬಿಡುತ್ತಾನೆ ಎಂಬುದನ್ನು ಗಮನಿಸಬೇಕು. ಅದು ಅತ್ಯಂತ ಮುಖ್ಯ. ಆ ದೃಷ್ಟಿ ಸರಿಯಾಗಿದ್ದರೆ ಮಾತ್ರ ಅದಕ್ಕೆ ಪುಣ್ಯ ದೊರಕುತ್ತದೆ, ಇಲ್ಲದೆ ಇದ್ದರೆ ಇಲ್ಲ. ಕೆಲವು ವೇಳೆ ನನಗೆ ಒಂದು ವಸ್ತು ಸಿಕ್ಕುವ ಸಂಭವ ಇಲ್ಲ. ಅದಕ್ಕಾಗಿ ಬಿಡುತ್ತೇನೆ. ನರಿ ದ್ರಾಕ್ಷಿ ಹಣ್ಣಿಗೆ ನೆಗೆದು ನೆಗೆದು ಎಟುಕದಾಗ ಅದೆಲ್ಲ ಬರೀ ಹುಳಿ ಎಂದು ಬಿಟ್ಟಂತೆ ಆಗುವುದು. ಮತ್ತೆ ಕೆಲವು ವೇಳೆ ಅದನ್ನು ಪಡೆಯುವುದಕ್ಕೆ ತುಂಬಾ ಕಷ್ಟ ಪಡಬೇಕು. ಅದಕ್ಕಾಗಿ ಬೇಡವೇ ಬೇಡ ಎನ್ನುತ್ತೇವೆ. ಇವುಗಳ ಹಿಂದೆಲ್ಲ ಇರುವುದು ಅಭಾವ ವೈರಾಗ್ಯ. ಇಲ್ಲ, ಅದಕ್ಕೆ ಬಿಟ್ಟಿದ್ದೇವೆ. ಬಂದರೆ ಅದನ್ನು ಅನುಭವಿಸುವುದಕ್ಕೆ ಮನಸ್ಸು ಸಿದ್ಧವಾಗಿರಬಹುದು. ಆದಕಾರಣ ಬಿಡುವುದೇ ದೊಡ್ಡದಲ್ಲ, ಯಾತಕ್ಕಾಗಿ ಬಿಡುತ್ತಾರೆ ಎಂಬುದನ್ನು ಗಮನಿಸಬೇಕಾಗಿದೆ.

\begin{verse}
ಯಜ್ಞದಾನತಪಃಕರ್ಮ ನ ತ್ಯಾಜ್ಯಂ ಕಾರ್ಯಮೇವ ತತ್~।\\ಯಜ್ಞೋ ದಾನಂ ತಪಶ್ಚೈವ ಪಾವನಾನಿ ಮನೀಷಿಣಾಮ್ \versenum{॥ ೫~॥}
\end{verse}

{\small ಯಜ್ಞ ದಾನ ತಪಸ್ಸು ಎಂಬ ಕರ್ಮಗಳನ್ನು ಬಿಡಕೂಡದು. ಅದನ್ನು ಮಾಡಲೇಬೇಕು. ಯಜ್ಞ ದಾನ ಮತ್ತು ತಪಸ್ಸು ವಿವೇಕಿಗಳನ್ನು ಕೂಡ ಪಾವನಗೊಳಿಸುವುದು.}

ಶ‍್ರೀಕೃಷ್ಣ ಇಲ್ಲಿ ತನ್ನ ಅಭಿಪ್ರಾಯವನ್ನು ಕೊಡುತ್ತಾನೆ. ಯಾರು ಸಮಾಜದಲ್ಲಿ ಅದಕ್ಕೆ ಒಂದು ಅಂಗವಾಗಿರುವರೋ ಅವರೆಲ್ಲ ಆ ಕಾರ್ಯಗಳನ್ನು ಮಾಡಬೇಕು. ನಾವೆಲ್ಲ ಒಂದು ಕಡೆಯಿಂದ ತೆಗೆದುಕೊಳ್ಳುತ್ತೇವೆ, ಮತ್ತೊಂದು ಕಡೆಯಿಂದ ಕೊಡಬೇಕಾಗಿದೆ. ಕೊಟ್ಟು ತೆಗೆದುಕೊಳ್ಳುವ ನಿಯಮದ ಮೇಲೆ ಬ್ರಹ್ಮಾಂಡ ಪಿಂಡಾಂಡಗಳೆಲ್ಲಾ ನಿಂತಿರುವುದು. ಯಾವಾಗ ಬರೀ ತೆಗೆದುಕೊಳ್ಳು ತ್ತಾನೊ ಕೊಡುವುದಿಲ್ಲವೊ ಅವನು ಪಾಪಿಯಾಗುತ್ತಾನೆ, ಸಮಾಜಘಾತುಕನಾಗುತ್ತಾನೆ. ಸಾಗರ ನದಿಗಳಿಂದ ಹಗಲು ರಾತ್ರಿ ನೀರನ್ನು ತೆಗೆದುಕೊಳ್ಳುವುದು. ಪುನಃ ಮಳೆಯಂತೆ ಕಳುಹಿಸುವುದು. ಬರೀ ತೆಗೆದುಕೊಳ್ಳುತ್ತಲೇ ಇದ್ದರೆ, ಕೊಡದೇ ಇದ್ದರೆ, ಸಾಗರ ಉಕ್ಕಿ ಹರಿದು ಭೂಮಿಯನ್ನೆಲ್ಲ ಆವರಿಸುವುದು. ಸರಕಾರ ಒಂದು ಕಡೆಯಿಂದ ಸುಂಕ ಕಂದಾಯ ಮೊದಲಾದುವುಗಳಿಂದ ಹಣವನ್ನು ಸಂಗ್ರಹಿಸಿ, ಮತ್ತೊಂದು ಕಡೆ ಅದನ್ನು ದೇಶದಲ್ಲಿ ಎಲ್ಲರಿಗೂ ಎಲ್ಲಾ ಸೌಕರ್ಯಗಳನ್ನು ಒದಗಿಸು ವುದಕ್ಕೂ ಖರ್ಚು ಮಾಡುತ್ತದೆ. ನಮ್ಮ ದೇಹದಲ್ಲೇ ಹೃದಯ ಎಲ್ಲಾ ಕಡೆಯಿಂದ ಬರುವ ರಕ್ತವನ್ನು ಒತ್ತಿ, ದೇಹದ ಎಲ್ಲಾ ಕಡೆಗೆ ಪುನಃ ಹೋಗುವಂತೆ ಮಾಡುತ್ತಿದೆ. ನಮ್ಮ ದೇಹದಲ್ಲೇ ಪ್ರತಿಯೊಂದು ಅಂಗಾಂಗವು ಸಮಷ್ಟಿಗೆ ದುಡಿಯುತ್ತಿದೆ. ಒಟ್ಟು ದೇಹವಾದರೂ ಪ್ರತಿಯೊಂದು ಅಂಗಾಂಗವನ್ನೂ ಗಮನಿಸುವುದು. ಯಾವುದು ಒಟ್ಟನ್ನು ಗಮನಿಸುವುದಿಲ್ಲವೊ, ತನಗೆ ತಾನೇ ಬಾಳುವುದೋ ಅದು ದೇಹಕ್ಕೆ ಮೃತ್ಯು ತರುವುದು. ಕ್ಯಾನ್ಸರ್ ಸೆಲ್ಲುಗಳು ದೇಹದಿಂದ ಪೋಷಕದ್ರವ್ಯಗಳನ್ನೆಲ್ಲಾ ಹೀರಿ, ತಾನು ಬೆಳೆದು ವೃದ್ಧಿಯಾಗುವುದೇ ಹೊರತು, ದೇಹವನ್ನು ಗಮನಿಸುವುದಿಲ್ಲ. ಅದಕ್ಕೇ ಆ ದೇಹಕ್ಕೆ ಮೃತ್ಯು ಸಂಭವಿಸುವುದು. ಹಾಗೆಯೇ ಯಾರು ಕೇವಲ ತೆಗೆದುಕೊಳ್ಳುವ ಕಡೆ ಮಾತ್ರ ಗಮನ ಕೊಟ್ಟು, ಕೊಡುವ ಸಮಯ ಬಂದಾಗ, ಅದೆಲ್ಲ ನಮ್ಮನ್ನು ಬಂಧಿಸುವುದು ಎಂದರೆ ಅದು ಆಗುವುದಿಲ್ಲ. ಕೊಡುವುದಕ್ಕೆ ಇಚ್ಛೆ ಇಲ್ಲದವನು ತೆಗೆದುಕೊಳ್ಳಲೂ ಕೂಡದು.

ಆದಕಾರಣವೇ ಶ‍್ರೀಕೃಷ್ಣ ಯಜ್ಞ, ದಾನ, ತಪಸ್ಸು ಇವುಗಳನ್ನು ಎಲ್ಲರೂ ಮಾಡಬೇಕು, ಯಾರೂ ಬಿಡಕೂಡದು ಎನ್ನುವನು. ದೇವತೆಗಳಿಗೆ ನಾವು ಯಜ್ಞವನ್ನು ಮಾಡಬೇಕು. ಅವರನ್ನು ಪ್ರಾರ್ಥಿಸ ಬೇಕು. ಅವರಿಂದ ನಮಗೆ ಮಳೆ ಬೆಳೆ ಎಲ್ಲಾ ಆಗುತ್ತಿದೆ, ಅವರ ಸಾಲದಲ್ಲಿದ್ದೇವೆ. ಅದನ್ನು ಅವರಿಗೆ ಮಾಡುವ ಯಜ್ಞದಿಂದ ತೀರಿಸಬೇಕು. ನಾವು ನಮ್ಮ ಸಮಾಜದಲ್ಲಿ ಇರುವ ಇತರರಿಗೆ ನಮಗೆ ಸಾಧ್ಯವಾದುದನ್ನು ದಾನ ಮಾಡಬೇಕು. ಎಲ್ಲರಿಗೂ ಏನನ್ನಾದರೂ ಕೊಡಲು ಸಾಧ್ಯ. ಒಬ್ಬ ಹಣ ಕೊಡಬಹುದು. ಇನ್ನೊಬ್ಬ ವಿದ್ಯೆ ಕೊಡಬಹುದು. ಬಟ್ಟೆ ಕೊಡಬಹುದು. ಔಷಧಿ ಕೊಡಬಹುದು, ಅನ್ನ ಕೊಡಬಹುದು, ಶ್ರಮವನ್ನು ದಾನ ಮಾಡಬಹುದು. ಕೊಡುವಾಗ ಇದು ಭಗವದರ್ಪಿತವಾಗಲಿ ಎಂದು ಕೊಡಬಹುದು. ಕೊಡುವುದೇ ನಮ್ಮ ಹೃದಯವನ್ನು ವಿಕಾಸ ಮಾಡುವುದು, ನಮ್ಮ ಸ್ವಾರ್ಥವನ್ನು ತೆಗೆಯುವುದು, ನಮ್ಮ ಮನಸ್ಸಿನ ಕಿಲುಬನ್ನು ತೊಳೆಯುವುದು. ತಪಸ್ಸನ್ನು ಮಾಡ ಬೇಕು. ಇದನ್ನು ಮಾಡುವುದಕ್ಕೆ ಮನುಷ್ಯನಿಗೊಬ್ಬನಿಗೇ ಸಾಧ್ಯ. ಮನಸ್ಸನ್ನು ಕೇಂದ್ರೀಕರಿಸಿ ತಿಳಿದು ಕೊಳ್ಳಬೇಕೆಂಬ ಬಾಹ್ಯವಸ್ತುವಿನ ಕಡೆ ಬೀರಿದರೆ, ಎಲ್ಲಾ ಭೌತಿಕ ಜ್ಞಾನಗಳೂ ನಮಗೆ ಬರುವುವು. ಇಂದಿನ ವಿಜ್ಞಾನಿ ಇಂತಹ ತಪಸ್ಸನ್ನು ಮಾಡುತ್ತಿರುವನು. ಅದರಿಂದಲೇ ಪ್ರಕೃತಿಯ ಅಂತರಾಳದಲ್ಲಿ ಹುದುಗಿರುವ ಹಲವಾರು ಸೂಕ್ಷ್ಮನಿಯಮಗಳನ್ನು ಕಂಡುಹಿಡಿದು ಅದ್ಭುತವಾದ ವಸ್ತುಗಳನ್ನು ತಯಾರಿಸುತ್ತಿರುವನು. ಯಾವಾಗ ತಪಸ್ಸನ್ನು ಅಂತರ್ಮುಖ ಮಾಡುವೆವೋ ಆಗ ಮಾನವನಿಗೆ ಅತೀಂದ್ರಿಯ ಶಕ್ತಿಯ ಪರಿಚಯವಾಗುವುದು. ನಮಗೆ ನಮ್ಮ ನೈಜಸ್ಥಿತಿ ಅರಿವಾಗುವುದು. ಆಗ ಪರಮಾತ್ಮನನ್ನು ಅರಿತುಕೊಳ್ಳುತ್ತೇವೆ, ಮುಕ್ತಿಯನ್ನು ಪಡೆಯುತ್ತೇವೆ, ಸಚ್ಚಿದಾನಂದ ಸುಖಕ್ಕೆ ನಾವು ಭಾಗಿಗಳಾಗುತ್ತೇವೆ.

ಇವು ಮೂರು ಬಯಕೆಗಳನ್ನು ಪಾವನಗೊಳಿಸುವುವು. ನಮ್ಮ ಮನಸ್ಸಿನ ಕೊಳೆಯನ್ನು ತೊಳೆಯು ವುವು. ಜೀವನದಲ್ಲಿ ಚಿತ್ತಶುದ್ಧಿ ಅತ್ಯಂತ ಆವಶ್ಯಕ. ನಾವು ಯಾವಾಗ ಫಲಾಪೇಕ್ಷೆಯನ್ನು ಬಿಟ್ಟು ಕರ್ಮವನ್ನು ಮಾಡುತ್ತೇವೆಯೋ, ಆಗ ಅದು ನಮ್ಮನ್ನು ಅತ್ಯಂತ ಉತ್ತಮ ಸ್ಥಿತಿಗೆ ಒಯ್ಯುವುದು.

\begin{verse}
ಏತಾನ್ಯಪಿ ತು ಕರ್ಮಾಣಿ ಸಂಗಂ ತ್ಯಕ್ತ್ವಾ ಫಲಾನಿ ಚ~।\\ಕರ್ತವ್ಯಾನೀತಿ ಮೇ ಪಾರ್ಥ ನಿಶ್ಚಿತಂ ಮತಮುತ್ತಮಮ್ \versenum{॥ ೬~॥}
\end{verse}

{\small ಪಾರ್ಥ, ಆಸಕ್ತಿಯನ್ನು ಮತ್ತು ಫಲವನ್ನು ಬಿಟ್ಟು ಇವು ಕರ್ತವ್ಯ ಎಂದು ಈ ಕರ್ಮಗಳನ್ನು ಮಾಡಬೇಕು. ಇದು ನನ್ನ ನಿಶ್ಚಿತವಾದ ಮತ್ತು ಉತ್ತಮವಾದ ಅಭಿಪ್ರಾಯ.}

ಶ‍್ರೀಕೃಷ್ಣ ಇಲ್ಲಿ ತನ್ನ ಅಭಿಪ್ರಾಯಗಳನ್ನು ಕೊಡುತ್ತಾನೆ. ಇದು ಶ್ರೇಷ್ಠವಾದ ಅಭಿಪ್ರಾಯ. ಕರ್ಮವನ್ನು ಮಾಡಲೇಬೇಕು, ಬಿಡುವುದಕ್ಕೆ ಆಗುವುದಿಲ್ಲ. ಅದನ್ನು ಮಾಡುವುದು ಮಾತ್ರವಲ್ಲ, ಅದನ್ನು ಸರಿಯಾದ ರೀತಿಯಲ್ಲಿ ಮಾಡಬೇಕು. ಆ ರೀತಿಯ ಆಸಕ್ತಿಯನ್ನು ಬಿಡಬೇಕು. ಆ ಕೆಲಸ ನಿಯತಕರ್ಮವಾಗಿರಬಹುದು, ಯಜ್ಞದಾನತಪಸ್ಸು ಮುಂತಾದ ಪವಿತ್ರ ಕರ್ಮಗಳೇ ಆಗಿರಬಹುದು. ಅದನ್ನು ಕೂಡ ನಾನು ಮಾಡುತ್ತಿರುವೆನು ಎಂಬ ಅಹಂಕಾರ ಮತ್ತು ಮಮಕಾರ ಇರಕೂಡದು. ಅದರ ಹಿಂದೆ ಫಲಾಪೇಕ್ಷೆ ಇರಕೂಡದು. ಯಾವಾಗ ನಾವು ಅದರ ಫಲಕ್ಕೆ ಕೈ ಒಡ್ಡುತ್ತೇವೆಯೋ ಆಗ ನಾವು ಬದ್ಧರಾಗುತ್ತೇವೆ, ವ್ಯಥೆಗೀಡಾಗುತ್ತೇವೆ. ಫಲಾಪೇಕ್ಷೆಯೇ ಕರ್ಮದಲ್ಲಿರುವ ವಿಷ. ನಾವು ಮೊದಲು ಅದನ್ನು ತ್ಯಜಿಸಿ ಕರ್ಮ ಮಾಡಿದರೆ, ಅದು ನಮ್ಮನ್ನು ಉದ್ಧರಿಸುವ ಶಕ್ತಿಯಾಗುವುದು; ಸಂಸಾರದಲ್ಲಿ ಈಜುವುದಕ್ಕೆ ಸೋರೆ ಬರುಡೆಯ ಆಶ್ರಯವಾದಂತಾಗುವುದು. ಯಾವಾಗ ಫಲಾಪೇಕ್ಷೆ ಇರವುದೊ, ಆಗ ದೊಡ್ಡದೊಂದು ಕಲ್ಲುಚಪ್ಪಡಿಯಂತೆ ನಮ್ಮನ್ನು ನೀರಿನಲ್ಲಿ ಮುಳುಗಿಸುವುದು.

\begin{verse}
ನಿಯತಸ್ಯ ತು ಸಂನ್ಯಾಸಃ ಕರ್ಮಣೋ ನೋಪಪದ್ಯತೇ~।\\ಮೋಹಾತ್ತಸ್ಯ ಪರಿತ್ಯಾಗಸ್ತಾಮಸಃ ಪರಿಕೀರ್ತಿತಃ \versenum{॥ ೭~॥}
\end{verse}

{\small ನಿಯತಕರ್ಮವನ್ನು ಬಿಡುವುದು ಯುಕ್ತವಲ್ಲ. ಮೋಹದಿಂದ ಅದನ್ನು ಬಿಟ್ಟರೆ ಅದನ್ನು ತಾಮಸಿಕ ಎಂದು ಹೇಳುತ್ತಾರೆ.}

ನಿಯತ ಕರ್ಮವನ್ನು ಮಾಡಲೇ ಬೇಕು ಎನ್ನುವನು ಶ‍್ರೀಕೃಷ್ಣ. ಅವನ ಗುರಿಯೇ ಧರ್ಮರಕ್ಷಣೆ ಮತ್ತು ಹಾಗೆ ರಕ್ಷಿಸುವಾಗ ಯಾರು ಅದಕ್ಕೆ ವಿರೋಧವಾದರೂ ಅವರನ್ನು ಅಡಗಿಸಬೇಕು. ಯಾವಾಗ ಕ್ಷತ್ರಿಯ ತನ್ನ ಕಣ್ಣೆದುರಿಗೆ ಆಗುತ್ತಿರುವ ಅನ್ಯಾಯವನ್ನು ನೋಡಿಯೂ ಸುಮ್ಮನಿರುತ್ತಾನೆಯೊ ಆಗ ಸಮಾಜದಲ್ಲಿ ಬಿಗಿ ತಪ್ಪುವುದು. ಯಾರು ಬೇಕಾದರೂ ಅನ್ಯಾಯ ಮಾಡಿ ತಪ್ಪಿಸಿಕೊಳ್ಳಬಹುದು ಎಂಬ ಭಾವ ಬೇರೂರುವುದು. ಇದಕ್ಕೆ ಅವಕಾಶವನ್ನು ಕೊಡುತ್ತಾನೆ ಕ್ಷತ್ರಿಯ ತನ್ನ ಕರ್ತವ್ಯವನ್ನು ಮಾಡದಾಗ. ಇದು ಬ್ರಾಹ್ಮಣನಿಗೆ ವೈಶ್ಯನಿಗೆ ಮತ್ತು ಶೂದ್ರನಿಗೆ ಅನ್ವಯಿಸುವುದು. ಸಮಾಜದಲ್ಲಿ ನ್ಯಾಯ ಧರ್ಮ ಮುಂತಾದುವೆಲ್ಲ ಇರಬೇಕಾದರೆ ಕ್ಷತ್ರಿಯನ ಶಕ್ತಿ ಎಷ್ಟು ಆವಶ್ಯಕವೊ, ಅಷ್ಟೇ ಆವಶ್ಯಕ ಸಮಾಜದಲ್ಲಿ ವಿದ್ಯೆ ಇರಬೇಕಾದರೆ. ಬ್ರಾಹ್ಮಣ ಅದನ್ನು ಕೊಡಬೇಕಾಗಿರುವುದು. ಶೂದ್ರ ತನ್ನ ದುಡಿಮೆಯಿಂದ ಆಹಾರ ಸಾಮಾನುಗಳನ್ನು ಸೃಷ್ಟಿಸುತ್ತಾನೆ. ಸಮಾಜಕ್ಕೆ ಇದು ಅತ್ಯಾವಶ್ಯಕ. ಹಾಗೆಯೇ ಒಂದು ಕಡೆ ತಯಾರಾದುದನ್ನು ಇತರ ಎಲ್ಲಾ ಕಡೆಗೂ ಹಂಚುವ ವೈಶ್ಯನ ಕೆಲಸ ಅಷ್ಟೇ ಮುಖ್ಯ. ಇವೆಲ್ಲ ಆಯಾ ವರ್ಣದವರು ಮಾಡಲೇ ಬೇಕಾದ ಕರ್ಮಗಳು. ಇವುಗಳನ್ನು ಎಂದಿಗೂ ಬಿಡುವುದಕ್ಕೆ ಆಗುವುದಿಲ್ಲ.

ಮೋಹದಿಂದ ಅದನ್ನು ಬಿಟ್ಟರೆ ಅದು ತಾಮಸಿಕ. ಅಜ್ಞಾನದಿಂದ, ಮನುಷ್ಯ ತಾನು ಬಿಟ್ಟರೆ ಏನು ಪರಿಣಾಮವಾಗುವುದು ಸಮಾಜದಮೇಲೆ ಎಂಬುದನ್ನು ಕುರಿತು ಯೋಚಿಸುವುದೇ ಇಲ್ಲ. ತಾನು ಕೆಲಸ ಮಾಡುವುದನ್ನು ಬಿಟ್ಟರೆ ತನಗೆ ಮಾತ್ರ ನಷ್ಟ ಎಂದು ಅವನು ಭಾವಿಸುತ್ತಾನೆ. ಆದರೆ ಇದು ಅವನಿಗೆ ಮಾತ್ರ ನಷ್ಟವಲ್ಲ. ಇತರರಿಗೂ ಕಷ್ಟವನ್ನು ಕೊಡುವುದು. ಒಬ್ಬ ರೈತ ತಾನು ಮಾಡುವ ಕೆಲಸವನ್ನು ಬಿಟ್ಟರೆ, ತನಗೆ ಬೆಳೆ ಇಲ್ಲ ನಿಜ. ಆದರೆ ಆ ಬೆಳೆಯನ್ನು ತಿನ್ನುತ್ತಿದ್ದ ಇತರರಿಗೂ ಇಲ್ಲದೇ ಹೋಗುವುದು.

\begin{verse}
ದುಃಖಮಿತ್ಯೇವ ಯತ್ಕರ್ಮ ಕಾಯಕ್ಲೇಶಭಯಾತ್ತ್ಯಜೇತ್~।\\ಸ ಕೃತ್ವಾ ರಾಜಸಂ ತ್ಯಾಗಂ ನೈವ ತ್ಯಾಗಫಲಂ ಲಭೇತ್ \versenum{॥ ೮~॥}
\end{verse}

{\small ಶರೀರಕ್ಕೆ ಕಷ್ಟ ಎಂಬ ಭಯದಿಂದ ದುಃಖದಾಯಕ ಎಂಬ ತಿಳಿವಳಿಕೆಯಿಂದ ಮಾಡಿದ ಕರ್ಮತ್ಯಾಗ ರಾಜಸಿಕ ತ್ಯಾಗ. ಇದರಿಂದ ಅವನಿಗೆ ತ್ಯಾಗದ ಫಲ ಲಭಿಸುವುದಿಲ್ಲ.}

ಕಷ್ಟ ಎಂದು ಬಿಡುವುದು ರಾಜಸಿಕ. ಒಬ್ಬನಿಗೆ ಅದರಿಂದ ಬರುವ ಫಲದ ಮೇಲೆ ಇಚ್ಛೆ ಇಲ್ಲ ಎಂದಲ್ಲ. ಅದನ್ನು ಪಡೆಯಬೇಕಾದರೆ ಬಹುಶಃ ಕಷ್ಟ ಪಡಬೇಕು. ಐ.ಎ.ಎಸ್ ಪರೀಕ್ಷೆಗೆ ಕುಳಿತು ಕೊಂಡು ಪಾಸು ಮಾಡುವುದು ಕಷ್ಟ. ಅದಕ್ಕಾಗಿ ಅಷ್ಟೊಂದು ಓದಬೇಕು. ಮತ್ತು ಅಲ್ಲಿ ಅಷ್ಟೊಂದು ಸ್ಪರ್ಧೆ ಇದೆ. ಯಾರಿಗೆ ಗೊತ್ತು, ಅದರಲ್ಲಿ ತೇರ್ಗಡೆಯಾಗುತ್ತೇನೆಯೋ ಇಲ್ಲವೋ ಎಂದು? ಅದಕ್ಕಾಗಿ ನನಗೆ ಅದರ ಮೇಲೆ ಇಚ್ಛೆ ಇಲ್ಲ ಎಂದರೆ ಅದು ರಾಜಸಿಕ. ಒಂದು ರುಚಿಕರವಾದ ಅಡಿಗೆಯನ್ನು ಮಾಡಿಕೊಂಡು ಊಟ ಮಾಡಬೇಕಾದರೆ ತುಂಬಾ ಕಷ್ಟ ಪಡಬೇಕು. ಆ ಕಷ್ಟಪಡುವು ದಾದರೂ ಏತಕ್ಕೆ ಎಂದು ಬಿಡುವನು. ಅವನಿಗೇನು ಆ ವಸ್ತುಗಳ ಮೇಲೆ ಆಸೆ ಇಲ್ಲದೆ ಇಲ್ಲ. ಆದರೆ ಅದಕ್ಕೆ ಕಷ್ಟಪಡಬೇಕು. ಯಾರಾದರೂ ನಾವು ಕಷ್ಟಪಡದೆ ಅದನ್ನು ಕೊಟ್ಟರೆ ಅನುಭವಿಸಲು ಸಿದ್ಧ. ಆದರೆ ಕಷ್ಟಪಡಬೇಕಾಗಿ ಬಂದಾಗ ನಮಗೆ ವೈರಾಗ್ಯ ಬರುವುದು.

ಮತ್ತು ಅದರಿಂದ ದುಃಖವಾಗುವುದು. ಕೆಲವು ವೇಳೆ ನಾವು ನಮ್ಮ ಕೆಲಸವನ್ನು ಮಾಡುವಾಗ ಇತರರನ್ನು ನೋಯಿಸಬೇಕಾಗುವುದು. ಅವರೊಡನೆ ನಿಷ್ಠುರವನ್ನು ಕಟ್ಟಿಕೊಳ್ಳಬೇಕಾಗುವುದು. ಈ ತಾಪತ್ರಯವೆಲ್ಲ ಏತಕ್ಕೆ. ಆ ಕೆಲಸವನ್ನು ಮಾಡುವುದೇ ಬೇಡ ಎಂದು ಬಿಟ್ಟುಬಿಡುವೆವು. ಇಲ್ಲಿ ಅರ್ಜುನನಲ್ಲಿಯೂ ಇದೇ ದೋಷ ಇದೆ. ಯುದ್ಧ ಮಾಡುವುದರಿಂದ ಗುರುಹಿರಿಯರನ್ನೆಲ್ಲಾ ಕೊಲ್ಲಬೇಕಾಗುವುದು. ಇದರಿಂದ ದುಃಖಪ್ರಾಪ್ತವಾಗುವುದು. ಅದಕ್ಕಾಗಿ ಯುದ್ಧವನ್ನು ಮಾಡದೇ ಇದ್ದರೆ ಇವುಗಳಾವುವೂ ಬರುವುದಿಲ್ಲ ಎಂದು ಭಾವಿಸುವನು.

\begin{verse}
ಕಾರ್ಯಮಿತ್ಯೇವ ಯತ್ಕರ್ಮ ನಿಯತಂ ಕ್ರಿಯತೇಽಜುRನ~।\\ಸಂಗಂ ತ್ಯಕ್ತ್ವಾ ಫಲಂ ಚೈವ ಸ ತ್ಯಾಗಃ ಸಾತ್ತ್ವಿಕೋ ಮತಃ \versenum{॥ ೯~॥}
\end{verse}

{\small ಅರ್ಜುನ, ಇದು ಮಾಡಲೇಬೇಕಾದುದು ಎಂದು ಅರಿತು, ಆ ನಿಯತ ಕರ್ಮವನ್ನು ಸಂಗ ಮತ್ತು ಫಲಗಳನ್ನು ತ್ಯಜಿಸಿ ಮಾಡಿದರೆ, ಅದು ಸಾತ್ತ್ವಿಕ ತ್ಯಾಗ. }

ಶ‍್ರೀಕೃಷ್ಣ ಇಲ್ಲಿ ತ್ಯಾಗ ಎನ್ನುವುದಕ್ಕೆ ಬೇರೆ ಅರ್ಥವನ್ನು ಕೊಡುವನು. ಕರ್ಮವನ್ನು ಬಿಡುವುದಲ್ಲ ತ್ಯಾಗ. ನಿಯತಕರ್ಮ ಮಾಡಲೇಬೇಕಾಗಿದೆ. ಅದನ್ನು ಬಿಡುವುದಕ್ಕೆ ನಮಗೆ ಅಧಿಕಾರವಿಲ್ಲ. ಬಿಡುವುದು ಒಂದು ಮಹಾಪಾಪ. ಇದರಿಂದ ಆಗುವ ನಷ್ಟ ನನಗೆ ಮಾತ್ರವಲ್ಲ. ಸಮಾಜ ನನ್ನ ದುಡಿತದ ಮೇಲೆ ನಿಂತಿದೆ. ನನಗೆ ಬೇಡದೇ ಇದ್ದರೂ ಆ ಕೆಲಸ ನಾವು ಮಾಡಬೇಕಾಗಿದೆ. ಅರ್ಜುನ, ನನಗೆ ಜಯ ಬೇಕಾಗಿಲ್ಲ ಎಂದು ಹೇಳಿದರೂ, ಯುದ್ಧವನ್ನು ಮಾಡಬೇಕು, ಏಕೆಂದರೆ ಬಿಡುವುದು ಮಹಾಪಾಪ. ಇದು ಅಧರ್ಮಕ್ಕೆ ಪ್ರೋತ್ಸಾಹ ಕೊಟ್ಟಂತೆ.

ನಿಯತಕರ್ಮ ಎಂದರೆ ಪ್ರತಿಯೊಂದು ವ್ಯಕ್ತಿಯಿಂದಲೂ ಸಮಾಜ ಯಾವ ಕರ್ತವ್ಯವನ್ನು ನಿರೀಕ್ಷಿಸುವುದೋ ಅದು. ಎಲ್ಲರೂ ಮಾಡುವುದನ್ನು ಮಾಡಿದರೆ ಒಟ್ಟು ಸಮಾಜಕ್ಕೆ ಶಕ್ತಿ ಬರುವುದು. ಹಾಗಾದರೆ ನಮಗೆ ಏನನ್ನು ಬಿಡುವುದಕ್ಕೆ ಅಧಿಕಾರವಿದೆ ಎಂದರೆ, ಆ ಕರ್ಮದಿಂದ ಬರುವ ಫಲಕ್ಕೆ ನಾವು ಕೈಯೊಡ್ಡದೆ ಇರಬಹುದು. ಬೇಕಾದರೆ ಇನ್ನು ಯಾರಾದರೂ ಅದನ್ನು ಅನುಭವಿಸಲಿ ಎಂದು ಬಿಡಬಹುದು. ನಾನು ತಿನ್ನದೆ ಇರಬಹುದು. ಬೆಳೆಸುವುದು ನನ್ನ ಕರ್ತವ್ಯ. ಅನಂತರ ನನಗೆ ಬೇಡದಿದ್ದರೆ ಅದನ್ನು ಮತ್ತಾರಾದರೂ ತೆಗೆದುಕೊಂಡು ಅನುಭವಿಸಲಿ, ಬಾಧಕವಿಲ್ಲ. ಫಲವನ್ನು ಬಿಡುವುದಕ್ಕೆ ಮಾತ್ರ ನಮಗೆ ಅಧಿಕಾರ. ಮಾಡುವ ಕೆಲಸವನ್ನು ಬಿಡುವುದಕ್ಕೆ ಅಧಿಕಾರವಿಲ್ಲ. ಸಂಗವನ್ನು ಬಿಟ್ಟು ಮಾಡಬೇಕು. ಅದೇ ಕರ್ಮಯೋಗದ ರಹಸ್ಯ. ಇದು ನನ್ನದು, ನಾನು ಮಾಡುತ್ತಿರುವೆ, ಎಂದು ಭಾವಿಸಬಾರದು. ಇದನ್ನು ಬಿಡುವುದಕ್ಕೆ ಆಗುವುದಿಲ್ಲ, ಆದಕಾರಣ ನಾನು ಮಾಡುತ್ತೇನೆ, ನಾನು ಮಾಡಲೇಬೇಕಾಗಿರುವುದರಿಂದ ಮಾಡುತ್ತೇನೆ. ಈ ದೃಷ್ಟಿ ಅವನದಾಗಿರಬೇಕು. ಇಲ್ಲಿ ಅವನು ಬಿಡುವುದು ಸಂಗವನ್ನು, ಈ ಕರ್ಮದಿಂದ ಬರುವ ಫಲವನ್ನು. ನಿಜವಾದ ತ್ಯಾಗ ಎಂದರೆ ಇದೇ ಶ‍್ರೀಕೃಷ್ಣನ ದೃಷ್ಟಿಯಲ್ಲಿ.

\begin{verse}
ನ ದ್ವೇಷ್ಟ್ಯಕುಶಲಂ ಕರ್ಮ ಕುಶಲೇ ನಾನುಷಜ್ಜತೇ~।\\ತ್ಯಾಗೀ ಸತ್ತ್ವಸಮಾವಿಷ್ಟೋ ಮೇಧಾವೀ ಛಿನ್ನಸಂಶಯಃ \versenum{॥ ೧೦~॥}
\end{verse}

{\small ಮೇಧಾವಿಯೂ, ಸಂಶಯ ರಹಿತನೂ ಆದ ಸಾತ್ತ್ವಿಕ ತ್ಯಾಗಿ, ಅಹಿತ ಕರ್ಮವನ್ನು ದ್ವೇಷಿಸುವುದಿಲ್ಲ. ಹಿತವಾದ ಕರ್ಮದಲ್ಲಿ ಆಸಕ್ತನೂ ಆಗುವುದಿಲ್ಲ.}

ಮೇಧಾವಿಯ ಬುದ್ಧಿ ಹರಿತವಾಗಿರುವುದು. ತನ್ನ ಸುತ್ತಲೂ ಎಂತಹ ಮನೋಹರವಾದ ನಾಮರೂಪಗಳಿದ್ದರೂ ಅದರ ಹಿಂದೆ ಹೋಗಿ ವಸ್ತುವಿನ ಯಥಾರ್ಥ ನೀತಿಯನ್ನು ತಿಳಿದುಕೊಳ್ಳ ಬಹುದು. ಯಾವುದು ಸತ್ಯ, ಯಾವುದು ಅಸತ್ಯ, ಯಾವುದು ಸರಿ, ಯಾವುದು ತಪ್ಪು ಎಂಬುದನ್ನು ಕ್ಷಣಾರ್ಧದಲ್ಲಿ ತಿಳಿದುಕೊಳ್ಳಬಲ್ಲದು.

ಅವನು ಸಂಶಯರಹಿತ. ಇನ್ನೂ ಅನುಮಾನದಲ್ಲಿ ತೊಳಲುತ್ತಿಲ್ಲ ಅವನ ಬುದ್ಧಿ. ಅವನು ಅನುಮಾನದ ಮೋಡಗಳಿಂದ ಮೇಲೆದ್ದಿರುವನು. ಎಲ್ಲಿಯವರೆಗೆ ಒಬ್ಬನಿಗೆ ಅನುಭವ ಆಗಿಲ್ಲವೋ ಅಲ್ಲಿಯವರೆಗೆ ಅವನು ಎಷ್ಟೋ ಬುದ್ಧಿವಂತನಾಗಿ ಒಂದು ಸರಿ ಎಂದು ತಿಳಿದುಕೊಂಡಿದ್ದರೂ ಸಂದೇಹದಿಂದ ಪಾರಾಗಲಾರ. ಅನುಭವವೊಂದೇ ಸಂದೇಹಕ್ಕೆ ಮದ್ದು. ಸಾತ್ತ್ವಿಕ ತ್ಯಾಗಿಯಲ್ಲಿ ನಾವು ಇದನ್ನು ನೋಡುತ್ತೇವೆ. ಅವನಿಗೆ ವಿಷಯ ವಸ್ತುಗಳು ಇಂದ್ರಿಯಕ್ಕೆ ಎಷ್ಟು ಸತ್ಯವೋ, ಅದಕ್ಕಿಂತ ಹೆಚ್ಚು ಅತೀಂದ್ರಿಯವಾದ ಆಧ್ಯಾತ್ಮಿಕ ಅನುಭವ ಸತ್ಯ.

ಅವನು ಅಹಿತ ಕರ್ಮವನ್ನು ದ್ವೇಷಿಸುವುದಿಲ್ಲ. ನಮ್ಮ ಗೌರವ ಕುಶಲ ಕರ್ಮವೇ ಅಕುಶಲ ಕರ್ಮವೇ ಅದರ ಮೇಲೆ ನಿಂತಿಲ್ಲ. ಯಾವ ದೃಷ್ಟಿಯಿಂದ ಮಾಡುತ್ತೇವೆಯೋ ಅದರ ಮೇಲೆ ನಿಂತಿದೆ. ಹಲವಾರು ಅಕುಶಲ ಕರ್ಮಗಳು ನಮ್ಮ ಜೀವನಕ್ಕೆ ಅತ್ಯಾವಶ್ಯಕ. ಅದು ಒಬ್ಬನ ಪಾಲಿಗೆ ಬಂದರೆ ಅದಕ್ಕಾಗಿ ಗೊಣಗಾಡುವುದಿಲ್ಲ. ಯಾರಾದರೂ ಇದನ್ನು ಮಾಡಬೇಕಾಗಿದೆ. ಇದು ನನ್ನ ಪಾಲಿಗೆ ಬಂದಿದೆ. ಇದನ್ನು ಮಾಡುತ್ತೇನೆ, ಎನ್ನುವನು ಅವನು. ಮಹಾಭಾರತದಲ್ಲಿ ಬರುವ ವ್ಯಾಧ, ಮಾಂಸವನ್ನು ಬಿಕರಿ ಮಾಡುತ್ತಿದ್ದ, ಆದರೂ ಅವನು ಪರಮಜ್ಞಾನಿಯಾಗಿದ್ದ. ನಮ್ಮ ದೇಶದಲ್ಲೆ ಹಲವು ಭಕ್ತರು ಜ್ಞಾನಿಗಳು ಅಹಿತವಾದ ಕರ್ಮವನ್ನು ಮಾಡುತ್ತಿದ್ದರು. ಕಬೀರ ನೆಯ್ಗೆಯವನು, ಗೋರ ಕುಂಬಾರರವನು. ಹಾಗೆ ಹಲವರು ಜೀವನದ ಹಲವು ಕಾರ್ಯಕ್ಷೇತ್ರಗಳಿಂದ ಬರುವರು. ಒಂದು ಊರಿನಲ್ಲಿ ಪೋಲಿಸನವರು ಇರುವರು. ಊರನ್ನೆಲ್ಲ ಚೊಕ್ಕಟವಾಗಿ ಇಟ್ಟಿರುವ ಜಾಡಮಾಲಿಗಳು ಇರುವರು. ಶಸ್ತ್ರಚಿಕಿತ್ಸಕ, ರೋಗಿಗೆ ನೋವು ಕೊಡಬೇಕು, ಅವನ ಕೀವು ರಕ್ತ ಇವುಗಳನ್ನೆಲ್ಲ ಮುಟ್ಟಬೇಕು. ನ್ಯಾಯಾಧಿಪತಿ ಒಬ್ಬ ಕಳ್ಳನಿಗೆ ಸೆರೆಮನೆಯ ಶಿಕ್ಷೆಯನ್ನು ಕೊಡಬೇಕಾಗಿದೆ, ಕೊಲೆ ಮಾಡಿದವನನ್ನು ಗಲ್ಲಿಗೆ ಹಾಕಬೇಕಾಗಿದೆ. ಇವುಗಳೆಲ್ಲ ಅಹಿತ ಕರ್ಮಗಳು. ಆದರೆ ಸಾತ್ತ್ವಿಕ ತ್ಯಾಗಿ ತನ್ನ ಪಾಲಿಗೆ ಬರುವ ಯಾವ ಅಹಿತ ಕರ್ಮವೇ ಆಗಲಿ ಗೊಣಗಾಡದೆ ಇದೂ ಕೂಡ ಭಗವಂತನಿಗೆ ಮಾಡುವ ಸೇವೆ ಎಂದು ಮಾಡುವನು.

ನಮಗೆ ಹಿತವಾದ ಪ್ರಿಯವಾದ ಕರ್ಮ ಬಂದಾಗ ಅನೇಕರು ಕುಣಿದಾಡುವರು. ಏಕೆಂದರೆ ನಮಗೆ ಅದನ್ನು ಮಾಡುವುದಕ್ಕೆ ಇಚ್ಛೆ ಮತ್ತು ಯಾರಿಂದಲೂ ನಿಷ್ಠುರವನ್ನು ಕಟ್ಟಿಕೊಳ್ಳಬೇಕಾಗಿಲ್ಲ. ಎಲ್ಲರ ಹತ್ತಿರವೂ ಚೆನ್ನಾಗಿ ಮಾತನಾಡಿಕೊಂಡು ಇರುವರು. ಹಿತವಾದ ಕರ್ಮವನ್ನು ಯಾರು ಬೇಕಾದರೂ ಮಾಡುವರು. ಆದರೆ ಅದಕ್ಕೆ ಅಂಟಿಕೊಳ್ಳುವರು. ಅದರಿಂದ ಬರುವ ಫಲಕ್ಕೆ ಕೈಯೊಡ್ಡುವರು. ಅದನ್ನು ಬಿಡು ಎಂದರೆ ಬಂತು ಪ್ರಾಣಸಂಕಟ ಅವರಿಗೆ. ಆದರೆ ಸಾತ್ತ್ವಿಕ ತ್ಯಾಗಿ, ಹಿತವಾದ ಕರ್ಮವನ್ನು ಮಾಡುತ್ತಿದ್ದರೂ ಅದಕ್ಕೆ ಆಸಕ್ತನಾಗಿರುವುದಿಲ್ಲ.

\begin{verse}
ನ ಹಿ ದೇಹಭೃತಾ ಶಕ್ಯಂ ತ್ಯಕ್ತುಂ ಕರ್ಮಾಣ್ಯಶೇಷತಃ~।\\ಯಸ್ತು ಕರ್ಮಫಲತ್ಯಾಗೀ ಸ ತ್ಯಾಗೀತ್ಯಭಿಧೀಯತೇ \versenum{॥ ೧೧~॥}
\end{verse}

{\small ದೇಹಧಾರಿಗೆ ಕರ್ಮವನ್ನು ಸಂಪೂರ್ಣ ಬಿಡಲು ಸಾಧ್ಯವಿಲ್ಲ. ಆದರೆ ಕರ್ಮಫಲವನ್ನು ತ್ಯಾಗಮಾಡಿದವನು ತ್ಯಾಗಿ ಎನಿಸಿಕೊಳ್ಳುತ್ತಾನೆ.}

ಎಲ್ಲಿಯವರೆಗೆ ನಾವೊಂದು ದೇಹಕ್ಕೆ ಅಂಟಿಕೊಂಡಿರುವೆವೋ ಅಲ್ಲಿಯವರೆಗೆ ನಾವು ಉಸಿರಾಡ ಬೇಕು, ನಿದ್ರೆ ಮಾಡಬೇಕು, ತಿನ್ನಬೇಕು, ಓಡಾಡಬೇಕು. ಈ ಕ್ರಿಯೆಗಳನ್ನು ಬಿಡುವುದಕ್ಕೆ ಆಗುವುದಿಲ್ಲ. ಮಾಡಲೇಬೇಕು. ನಾವು ಸುಮ್ಮನೆ ತಿಂದುಕೊಂಡು ಇರುವುದಕ್ಕೆ ಆಗುವುದಿಲ್ಲ. ನಾವು ತಿನ್ನುವುದನ್ನು ಸಂಪಾದನೆ ಮಾಡಬೇಕು. ಎಲ್ಲಿಯವರೆಗೆ ದೇಹಭಾವ ನಮ್ಮಲ್ಲಿರುವುದೋ ಅಲ್ಲಿಯವರೆಗೆ ನಾವು ಕರ್ಮವನ್ನು ಮಾಡಲೇಬೇಕಾಗಿದೆ. ಪೂರ್ಣರಾಗುವುದಕ್ಕೆ ಮುಂಚೆ ಕರ್ಮವನ್ನು ಬಿಟ್ಟರೆ ನಾವು ತಿಳಿಯದೆ ಮತ್ತಾವುದೊ ಕರ್ಮಕ್ಕೆ ಒಳಗಾಗಿ ಮಾಡುತ್ತಿರುವೆವು. ಆದರೆ ಅದು ಕರ್ಮ ಎಂದು ನಮಗೆ ಗೊತ್ತಾಗುವುದಿಲ್ಲ.

ಫಲಾಪೇಕ್ಷೆ ಇಲ್ಲದೆ ನಮ್ಮ ಪಾಲಿಗೆ ಬಂದ ಕರ್ಮವನ್ನು ಮಾಡುತ್ತ ಹೋಗೋಣ. ಯಾವಾಗ ನಾವು ಕರ್ಮವನ್ನು ಮೀರುವ ಸ್ಥಿತಿಗೆ ಬರುವೆವೋ, ಆಗ ನಾವು ಕರ್ಮವನ್ನು ಬಿಡಬೇಕಾಗಿಲ್ಲ. ಕರ್ಮ ತನಗೆ ತಾನೆ ಬಿದ್ದು ಹೋಗುವುದು. ಹಣ್ಣು ಚೆನ್ನಾಗಿ ಮಾಗಿದ ಮೇಲೆ ಮರದಿಂದ ಉದುರಿ ಹೋಗುವುದು. ಶ‍್ರೀರಾಮಕೃಷ್ಣರು ಮತ್ತೊಂದು ಉದಾಹರಣೆಯನ್ನು ಕೊಡುತ್ತಿದ್ದರು. ಬಿದ್ದರೆ ಗಾಯವಾಗಿ ಅಲ್ಲಿ ಹೊಕ್ಕು ಕಟ್ಟಿಕೊಳ್ಳುವುದು. ಗಾಯ ವಾಸಿ ಆದ ಮೇಲೆ ಹೊಕ್ಕು ತಾನೇ ಬಿದ್ದು ಹೋಗುವುದು. ಆದರೆ ಗಾಯ ವಾಸಿ ಆಗುವುದಕ್ಕೆ ಮುಂಚೆ ನಾವು ಕಿತ್ತು ಹಾಕಿದರೆ, ಮತ್ತೊಂದು ಹೊಕ್ಕು ಕಟ್ಟಿಕೊಳ್ಳುವುದು. ಅದರಂತೆಯೇ ಕರ್ಮ.

\begin{verse}
ಅನಿಷ್ಟಮಿಷ್ಟಂ ಮಿಶ್ರಂ ಚ ತ್ರಿವಿಧಂ ಕರ್ಮಣಃ ಫಲಮ್~।\\ಭವತ್ಯತ್ಯಾಗಿನಾಂ ಪ್ರೇತ್ಯ ನ ತು ಸಂನ್ಯಾಸಿನಾಂ ಕ್ವಚಿತ್ \versenum{॥ ೧೨~॥}
\end{verse}

{\small ಯಾರು ತ್ಯಾಗ ಮಾಡುವುದಿಲ್ಲವೊ ಅವರಿಗೆ ಕರ್ಮಫಲ, ಅನಿಷ್ಟ, ಇಷ್ಟ ಮತ್ತು ಮಿಶ್ರ ಎಂದು ಮೂರು ವಿಧವಾಗಿ ಬರುವುದು. ಆದರೆ ತ್ಯಾಗಿಗೆ ಆ ಫಲ ಎಂದಿಗೂ ಆಗುವುದಿಲ್ಲ.}

ಯಾರು ಫಲಾಪೇಕ್ಷೆಯಿಂದ ಕರ್ಮಗಳನ್ನು ಮಾಡುತ್ತಾರೊ ಅವರಿಗೆ ಫಲ ಬರುವುದು. ಆದರೆ ಆ ಫಲ ಯಾವಾಗಲೂ ಪ್ರಿಯವಾದ ಫಲವಾಗುವುದಿಲ್ಲ. ಮೊದಲನೆಯದೆ, ಕೆಟ್ಟ ಕೆಲಸಗಳನ್ನು ಮಾಡಿದ್ದರೆ ನಾವು ಕಾಲವಾದ ಮೇಲೆ ಪ್ರಾಣಿಗಳ ಜನ್ಮದಲ್ಲಿ ಹುಟ್ಟಬಹುದು, ಫಲಾಪೇಕ್ಷೆಯಿಂದ ಒಳ್ಳೆಯ ಕೆಲಸವನ್ನು ಮಾಡಿದ್ದರೆ ಸ್ವರ್ಗ ಮುಂತಾದ ಲೋಕಗಳಲ್ಲಿ ಕೆಲವು ಕಾಲ ಇರುವ ಪುಣ್ಯ ಸಂಪಾದಿಸುತ್ತೇವೆ. ಮಿಶ್ರಫಲವಾದರೆ ಮನುಷ್ಯ ಜನ್ಮಕ್ಕೆ ಬರುತ್ತೇವೆ. ಇಲ್ಲಿ ಸುಖ ದುಃಖ ಎರಡೂ ಮಿಶ್ರವಾಗಿದೆ. ಇದನ್ನು ಪ್ರಾಣಿ ವರ್ಗ, ಸ್ವರ್ಗಲೋಕ ಮುಂತಾದುವುಗಳಲ್ಲದೆ, ಕೇವಲ ಮನುಷ್ಯ ಲೋಕದ ದೃಷ್ಟಿಯಿಂದಲೇ ವಿವರಿಸಬಹುದು. ಈ ಮನುಷ್ಯಲೋಕದಲ್ಲೆ ಇಷ್ಟ, ಅನಿಷ್ಟ, ಮಿಶ್ರ ಇವುಗಳೆಲ್ಲ ಇವೆ. ಅನಿಷ್ಟವೇ, ನಿತ್ಯ ತಾಪತ್ರಯದ ಮನೆಯಲ್ಲಿ ಹುಟ್ಟುವುದು. ತಿನ್ನುವುದಕ್ಕೆ ಇಲ್ಲ, ಹೊದೆಯುವುದಕ್ಕೆ ಇಲ್ಲ, ಕೊಳೆ ದಾರಿದ್ರ್ಯ ಅಜ್ಞಾನ ಕಾಡುತ್ತಿರುವುದು. ರೋಗ ರುಜಿನ ದಾರಿದ್ರ್ಯ ಅಜ್ಞಾನ ಈ ವಾತಾವರಣದಲ್ಲಿ ಹುಟ್ಟುವುದೇ ಅನಿಷ್ಟ. ಇಷ್ಟವೇ ದೊಡ್ಡ ಶ‍್ರೀಮಂತರ ಮನೆಯಲ್ಲಿ ಹುಟ್ಟುವುದು, ಬೇಕಾದ ಸುಖ ಸೌಕರ್ಯಗಳಿವೆ ಅಲ್ಲಿ. ಸುತ್ತಮುತ್ತಲೂ ಇರುವ ಜನ ವಿದ್ಯಾವಂತರು, ಸುಸಂಸ್ಕೃತರು. ಅವರಿಗೆ ದೈಹಿಕವಾಗಿ ಯಾವುದಕ್ಕೂ ಕೊರತೆ ಇಲ್ಲ. ಏನನ್ನು ಬೇಕಾದರೂ ಅನುಭವಿಸಲು ಅವಕಾಶವಿದೆ. ಮಿಶ್ರವೇ ಮಧ್ಯ ಕುಲದಲ್ಲಿ ಹುಟ್ಟುವುದು. ಇಲ್ಲಿಯೇ ಮನುಷ್ಯನಿಗೆ ಸುಖ ದುಃಖಗಳ ದ್ವಂದ್ವ ಅನುಭವ ಇತರ ಎರಡು ಕಡೆಗಳಿಗಿಂತ ಹೆಚ್ಚಾಗಿ ಆಗುವುದು. ಜೀವಿ ವಿಕಾಸಕ್ಕೆ ಇದು ಸರಿಯಾದ ವಾತಾವರಣ. ಇಲ್ಲಿ ಅಷ್ಟೊಂದು ಐಶ್ವರ್ಯವೂ ಇಲ್ಲ, ಕಡು ಬಡತನವೂ ಇಲ್ಲ. ಒಬ್ಬ ವಿಚಾರ ಮಾಡುವುದಕ್ಕೆ, ಪ್ರಪಂಚವನ್ನು ತಿಳಿದುಕೊಳ್ಳುವುದಕ್ಕೆ ಇದು ತುಂಬಾ ಸಹಾಯ ಮಾಡುವುದು. ತುಂಬಾ ಶ‍್ರೀಮಂತನಾದರೆ, ಇಂದ್ರಿಯ ಸುಖದಲ್ಲಿ ಮುಳುಗಿ ಹೋಗುತ್ತಾನೆ. ಜೀವನದ ಗಾಢ ವಿಷಯಗಳನ್ನು ಕುರಿತು ಚಿಂತಿಸಲು ಅವನಿಗೆ ಅವಕಾಶವೇ ಇರುವುದಿಲ್ಲ. ಒಂದು ಇಂದ್ರಿಯದ ತೃಪ್ತಿ ಆದ ಮೇಲೆ ಮತ್ತೊಂದು ಇಂದ್ರಿಯ ಹಾತೊರೆಯುವುದು. ಕಡುಬಡತನದಲ್ಲಿ ಯಾದರೂ, ದಿನ ಬೆಳಗಾದರೆ ಹೊಟ್ಟೆ ಬಟ್ಟೆಯ ಪ್ರಶ್ನೆಯೇ ಅವನ ಕಾಲವನ್ನೆಲ್ಲಾ ಹೀರುವುದು. ಅವನಿಗೆ ಜೀವನದಲ್ಲಿ ಆಲೋಚಿಸುವುದಕ್ಕೆ ಸಮಯವೇ ಇರುವುದಿಲ್ಲ.

ಆದರೆ ಯಾರು ತ್ಯಾಗಿ ಆಗಿರುವನೋ ಕರ್ಮಗಳನ್ನು ಯಾವ ಪ್ರತಿಫಲಾಕಾಂಕ್ಷೆಯೂ ಇಲ್ಲದೆ ಮಾಡುತ್ತಿರುವನೋ, ಅವನು ಜೀವನದಲ್ಲಿ ಬದುಕಿರುವಾಗಲೇ ಮುಕ್ತ. ಅವನೆಲ್ಲಿಗೂ ಹೋಗು ವುದೂ ಇಲ್ಲ, ಬರುವುದೂ ಇಲ್ಲ. ಈ ದೇಹದಿಂದ ಹೋದ ಮೇಲೆ, ಮಡಕೆಯನ್ನು ಒಡೆದರೆ, ಹೇಗೆ ಅದರೊಳಗೆ ಇರುವ ಆಕಾಶ ಮಹಾಕಾಶದಲ್ಲಿ ಒಂದಾಗುವುದೊ ಹಾಗೆ ಒಂದಾಗುತ್ತಾನೆ ಭಗವಂತ ನೊಡನೆ. ಅವನನ್ನು ಇಷ್ಟ ಅನಿಷ್ಟ ಮಿಶ್ರದ ಯಾವ ಸರಪಣಿಯೂ ಕಟ್ಟಿಹಾಕಲಾರದು.

\begin{verse}
ಪಂಚೈತಾನಿ ಮಹಾಬಾಹೋ ಕಾರಣಾನಿ ನಿಬೋಧ ಮೇ~।\\ಸಾಂಖ್ಯೇ ಕೃತಾಂತೇ ಪ್ರೋಕ್ತಾನಿ ಸಿದ್ಧಯೇ ಸರ್ವಕರ್ಮಣಾಮ್ \versenum{॥ ೧೩~॥}
\end{verse}

{\small ಅರ್ಜುನ, ಕರ್ಮಗಳ ಸಿದ್ಧಿಗಾಗಿ ಈ ಐದು ಕಾರಣಗಳನ್ನು ಸಾಂಖ್ಯ ಶಾಸ್ತ್ರ ಹೇಳಿದೆ. ಅದನ್ನು ನನ್ನಿಂದ ತಿಳಿದುಕೊ.}

ಶ‍್ರೀಕೃಷ್ಣ ಇಲ್ಲಿ ಪರಿಣಾಮ ಕ್ರಿಯೆಯನ್ನು ಮತ್ತಷ್ಚು ವಿಭಜನೆ ಮಾಡುವನು. ನಾನು ಮಾಡಿದ ಕೆಲಸಕ್ಕೆ ಬರುವ ಪರಿಣಾಮಕ್ಕೆಲ್ಲ ನಾನೇ ಕಾರಣನಾದರೆ, ಅದರ ಪ್ರತಿಫಲವನ್ನೆಲ್ಲ ನಾನೇ ಆಶಿಸುವುದು ನ್ಯಾಯವಾಗಿದೆ. ಆದರೆ ಆ ಕ್ರಿಯೆ ಆಗುವುದಕ್ಕೆ ಐದು ಕಾರಣಗಳಿವೆ ಎನ್ನುತ್ತಾನೆ. ಐದು ಕಾರಣಗಳೂ ಸರಿ ಹೊಂದಿಕೊಂಡರೆ ಮಾತ್ರ ಆ ಕಾರ್ಯ ಆಗಬಹುದು. ಐದರಲ್ಲಿ ಯಾವುದಾದರೂ ಒಂದು ಸಹಕರಿಸದೆ ಇದ್ದರೂ ಕೆಲಸ ಪೂರ್ಣವಾಗುವುದಿಲ್ಲ. ಇದನ್ನೆ ಸಾಂಖ್ಯ ಶಾಸ್ತ್ರ ಹೇಳುವುದು. ಒಂದು ಫಲದ ಮೇಲೆ ನಮಗೆ ಇರುವ ಆಸೆ ಹೋಗಬೇಕಾದರೆ, ಆ ಫಲ ಹೇಗಾಯಿತು, ಯಾರು ಯಾರು ಎಷ್ಟೆಷ್ಟು ದುಡಿದಿರುವರು ಅದಕ್ಕೆ ಎಂಬುದನ್ನು ತಿಳಿದುಕೊಳ್ಳಬೇಕು. ಅನಂತರ ಯಾರು ಯಾರು ಇದಕ್ಕೆ ಕಂಡಂತೆ ಕಾಣದಂತೆ ದುಡಿದಿರುವರೋ ಅವರಿಗೆಲ್ಲಾ ಅವರವರ ಪಾಲನ್ನು ಕೊಟ್ಟು ಉಳಿದುದನ್ನು ನಾನು ಅನುಭವಿಸಬೇಕು. ಆದರೆ ಫಲಾಕಾಂಕ್ಷಿತನಲ್ಲಿ ಈ ತಾಳ್ಮೆಯಿಲ್ಲ, ಇದನ್ನು ಅಷ್ಟು ಆಳವಾಗಿ ವಿಚಾರಿಸಲೂ ಅವನು ಹೋಗುವುದಿಲ್ಲ. ಅವನಿಗೆ ಅನುಭವಿಸುವ ಆತುರ. ಅದಕ್ಕೆ ಹಲಸಿನ ಹಣ್ಣನ್ನು ತಿನ್ನವನು. ಜೊತೆ ಜೊತೆಗೆ ಕೈಗೆ ಬಾಯಿಗೆ ಅಂಟೂ ತಾಕುವುದು. ತ್ಯಾಗಿ ಅನಾಸಕ್ತಿಯ ತೈಲವನ್ನು ಕೈಗೆ ತಾಕಿಸಿಕೊಂಡಿರುವನು. ಒಂದು ಫಲಕ್ಕೆ ಯಾರು ಯಾರು ಕಾರಣಕರ್ತರು ಎಂಬು ದನ್ನು ವಿಚಾರದ ಮೂಲಕ ಅರಿಯುವನು. ಚಿನ್ನದ ವ್ಯಾಪಾರಿ ಒಡವೆಯನ್ನು ಹೇಗೆ ಒರೆಗಲ್ಲಿಗೆ ತಿಕ್ಕಿ ನೋಡುವನೋ ಹಾಗೆ ತ್ಯಾಗಿ ವಿಮರ್ಶೆಯ ಒರೆಗಲ್ಲಿಗೆ ಒಂದು ಪರಿಣಾಮವನ್ನು ತಿಕ್ಕಿ ನೋಡುವನು. ಅದರ ಸತ್ಯವನ್ನು ತಿಳಿದ ಮೇಲೆಯೇ ಅದರ ಮೇಲಿನ ವ್ಯಾಮೋಹ ನಮ್ಮನ್ನು ಬಿಡಬೇಕಾದರೆ.

\begin{verse}
ಅಧಿಷ್ಠಾನಂ ತಥಾ ಕರ್ತಾ ಕರಣಂ ಚ ಪೃಥಗ್ವಿಧಮ್~।\\ವಿವಿಧಾಶ್ಚ ಪೃಥಕ್ಚೇಷ್ಟಾ ದೈವಂ ಚೈವಾತ್ರ ಪಂಚಮಮ್ \versenum{॥ ೧೪~॥}
\end{verse}

{\small ಅಧಿಷ್ಠಾನ ಕರ್ತೃ ಅನೇಕ ಬಗೆಯ ಕರಣ ವಿಧವಿಧವಾದ ಬೇರೆ ಬೇರೆ ಕಾರ್ಯಗಳು ಮತ್ತು ಐದನೆಯದೆ ದೈವ.}

ಒಂದು ಕೆಲಸವಾಗಬೇಕಾದರೆ ಈ ಐದು ಒಟ್ಟು ಸೇರಬೇಕು. ಅನಂತರವೇ ಆ ಕೆಲಸ ಆಗ ಬೇಕಾದರೆ. ಮೊದಲನೆಯದೆ ಅಧಿಷ್ಠಾನ. ಯಾವುದರಿಂದ ವಸ್ತುವನ್ನು ಮಾಡುತ್ತೇವೆಯೋ ಅದು. ಒಂದು ಮೇಜು ಮಾಡಬೇಕಾದರೆ ಅದಕ್ಕೆ ಮರ ಅಧಿಷ್ಠಾನ. ಒಂದು ಮಡಕೆಯನ್ನು ಮಾಡಬೇಕಾದರೆ, ಜೇಡಿಮಣ್ಣು ಅದಕ್ಕೆ ಅಧಿಷ್ಠಾನ. ಬಟ್ಟೆಯನ್ನು ಮಾಡಬೇಕಾದರೆ, ಹತ್ತಿ ಅದಕ್ಕೆ ಅಧಿಷ್ಠಾನ. ಎರಡನೆ ಯವನೆ ಕರ್ತೃ. ಮಾಡತಕ್ಕ ಮನುಷ್ಯ, ಅವನ ದೇಹ, ಬುದ್ಧಿ, ಇಂದ್ರಿಯಗಳೆಲ್ಲವೂ ಇದರಲ್ಲಿ ಸೇರಿವೆ. ಮೂರನೆಯದೆ ಕರಣ ಎಂದರೆ ಇತರ ಸಲಕರಣೆಗಳು. ಮರ ಒಂದು ಟೇಬಲ್ ಆಗಬೇಕಾದರೆ ಆ ಕೆಲಸ ಮಾಡುವುದಕ್ಕೆ ಹಲವು ವಸ್ತುಗಳು ಬೇಕಾಗುತ್ತವೆ. ಗರಗಸ, ಉಳಿ, ಹತ್ತರಿ, ಸುತ್ತಿಗೆ, ಮೊಳೆ, ಮುಂತಾದುವುಗಳೆಲ್ಲವೂ ಇದರಲ್ಲಿ ಸೇರಿವೆ. ಜೇಡಿಮಣ್ಣಲ್ಲಿ ಮಡಕೆ ಆಗಬೇಕಾದರೆ, ಅದಕ್ಕೆ ಒಂದು ಚಕ್ರ ಬೇಕು. ಹತ್ತಿ ಬಟ್ಟೆಯ ರೂಪವನ್ನು ಧರಿಸುವುದಕ್ಕೆ ಮುಂಚೆ ಹಲವು ರೂಪಗಳ ಮೂಲಕ ಹೋಗಬೇಕಾದರೆ ಅದಕ್ಕೆ ಹಲವು ವಸ್ತುಗಳು ಬೇಕಾಗಿವೆ. ಹತ್ತಿಯನ್ನು ಬಿಡಿಸುವ ಯಂತ್ರ, ಅದರಿಂದ ನೂಲನ್ನು ಮಾಡುವ ಯಂತ್ರ ಇವುಗಳೆಲ್ಲ ಬೇಕು. ನಾಲ್ಕನೆಯದೆ ಅದಕ್ಕೆ ಸಂಬಂಧಪಟ್ಟ ವಿವಿಧ ಕ್ರಿಯೆಗಳು. ಬಡಗಿ ಟೇಬಲ್ ಮಾಡುವುದಕ್ಕೆ ಮೊದಲು ಮರವನ್ನು ಗರಗಸದಲ್ಲಿ ಕತ್ತರಿಸುತ್ತಾನೆ. ಅನಂತರ ಮರದ ಕಾಲುಗಳನ್ನು ಹತ್ತರಿ ಹಿಡಿದು ನಯ ಮಾಡುತ್ತಾನೆ. ಅದನ್ನು ಜೋಡಿಸುತ್ತಾನೆ. ಮೊಳೆ ಹೊಡೆದೊ ಅಥವಾ ಬೇರೆ ರೀತಿಯಲ್ಲೊ ಜೋಡಿಸುತ್ತಾನೆ. ಅನಂತರ ಅದಕ್ಕೆ ಪಾಲಿಶ್ ಕೊಡುತ್ತಾನೆ. ಅನಂತರವೇ ಮರ ಒಂದು ಟೇಬಲ್ ರೂಪವನ್ನು ಧರಿಸುವುದು. ಜೇಡಿಮಣ್ಣು ಮಡಕೆ ಆಗಬೇಕಾದರೆ ಮೊದಲು ಮಣ್ಣನ್ನು ನೀರಿನಲ್ಲಿ ನೆನೆಹಾಕಿ ಹದ ಮಾಡಿಕೊಳ್ಳಬೇಕು. ಮಡಕೆ ಆದ ಮೇಲೆ ಅದನ್ನು ಬಿಸಿಲಿನಲ್ಲಿ ಒಣಗಿಸಬೇಕು. ಕೊನೆಗೆ ಆವಿಗೆಯಲ್ಲಿ ಬೇಯಿಸಬೇಕು. ಅನಂತರವೇ ಅದೊಂದು ಮಡಕೆ ಆಗಬೇಕಾದರೆ. ಹಾಗೆ ನೂಲು ಬಟ್ಟೆ ಆಗಬೇಕಾದರೆ ಯಂತ್ರದಲ್ಲಿ ಹಲವು ರೂಪಗಳ ಮೂಲಕ ಹೋಗಬೇಕಾಗಿದೆ. ದೈವ ಐದನೆಯದು. ಈ ದೈವದ ಸಹಕಾರವಿಲ್ಲದೆ ಇದ್ದರೆ ನಾವು ಏನನ್ನೂ ಮಾಡಲಾಗುವುದಿಲ್ಲ. ಟೇಬಲ್ ಮಾಡುವುದಕ್ಕೆ ಮರದ ತುಂಡನ್ನೆಲ್ಲ ಇಟ್ಟಿರುತ್ತಾನೆ. ಇದ್ದಕ್ಕೆ ಇದ್ದಂತೆ ಬೆಂಕಿ ಅದಕ್ಕೆ ತಗುಲುವುದು. ಅದೆಲ್ಲ ಸುಟ್ಟು ಹೋಗುವುದು. ಜೇಡಿಮಣ್ಣು ಕುಡಿಕೆಯಾಗಿ ಒಣಗಲು ಇಟ್ಟಿರುತ್ತಾನೆ. ದೊಡ್ಡದಾಗಿ ಮಳೆಬರುವುದು, ಅದೆಲ್ಲ ಕರಗಿಹೋಗ ಬಹುದು. ಯಾವುದೊ ದನವೊ, ಆಡುಗಳ ಮಂದೆಯೋ ಅದರ ಮೇಲೆ ಹೋಗಿ ಎಲ್ಲವನ್ನೂ ಒಡೆದುಹಾಕಿ ಬಿಡಬಹುದು. ಆವಿಗೆಯಲ್ಲಿ ಇಟ್ಟಿರುವಾಗ ಜೋರಾಗಿ ಮಳೆ ಬಂದು ಆವಿಗೆ ಬೆಂಕಿಯೇ ಆರಿಹೋಗಬಹುದು. ಅಂತೂ ಐದನೆಯದಾದ ದೈವ ಸಹಕರಿಸಿದರೆ ಮಾತ್ರ ಸಾಧ್ಯ. ಹೊಲ ಉತ್ತಿರುವನು. ಒಳ್ಳೆ ಬೀಜ ಬಿತ್ತಿರುವನು, ಸ್ವಲ್ಪ ಪೈರೂ ಮೇಲಕ್ಕೆ ಬಂದಿದೆ. ಕಳೆ ಕಿತ್ತಿರುವನು. ಮಳೆ ಬರುವುದಿಲ್ಲ. ಆಗ ಪೈರೆಲ್ಲ ಒಣಗಿ ಹೋಗುವುದು. ಮನುಷ್ಯ ತಾನು ಮಾಡಬೇಕಾದುದನ್ನೆಲ್ಲಾ ಮಾಡಿ ಐದನೆಯದಾದ ದೇವರು ಮಳೆ ಕೊಟ್ಟರೆ ಮಾತ್ರ ಅವನಿಗೆ ಬೆಳೆ. ಯಾವಾಗ ಅವನು ಸಹಕರಿಸುವು ದಿಲ್ಲವೊ, ಇವನು ಮಾಡಿದ ಕೆಲಸವೆಲ್ಲ ವ್ಯರ್ಥವಾಗುವುದು.

\begin{verse}
ಶರೀರವಾಙ್ಮನೋಭಿರ್ಯತ್ಕರ್ಮ ಪ್ರಾರಭತೇ ನರಃ~।\\ನ್ಯಾಯ್ಯಂ ವಾ ವಿಪರೀತಂ ವಾ ಪಂಚೈತೇ ತಸ್ಯ ಹೇತವಃ \versenum{॥ ೧೫~॥}
\end{verse}

{\small ಶರೀರ ವಾಕ್ ಮನಸ್ಸು ಇವುಗಳಿಂದ ಮನುಷ್ಯ, ಧರ್ಮವಾದ ಅಥವಾ ಅಧರ್ಮವಾದ ಯಾವ ಕೆಲಸವನ್ನು ಮಾಡುತ್ತಾನೆಯೋ ಅದಕ್ಕೆ ಈ ಐದು ಕಾರಣಗಳಿವೆ.}

ಯಾವ ಒಳ್ಳೆಯ ಅಥವಾ ಕೆಟ್ಟ ಕೆಲಸವನ್ನು ಮಾಡಬೇಕಾದರೂ ಈ ಐದೂ ಇರಬೇಕು. ನಾಲ್ಕೇ ಇದ್ದು ಐದನೆಯದು ಇಲ್ಲದೆ ಇದ್ದರೆ ಕೆಲಸ ಆಗುವ ಹಾಗಿಲ್ಲ. ದೇವರು ಒಳ್ಳೆಯ ಕೆಲಸಕ್ಕೇನೊ ಸಹಾಯ ಮಾಡುತ್ತಾನೆ. ಕೆಟ್ಟ ಕೆಲಸಕ್ಕೆ ಹೇಗೆ ಸಹಾಯ ಮಾಡುತ್ತಾನೆ ಎಂದು ನಾವು ಆಶ್ಚರ್ಯ ಪಡಬಹುದು. ದೇವರು ಕೆಟ್ಟದ್ದಕ್ಕೆ ಸಹಾಯ ಮಾಡುವುದಿಲ್ಲ. ದೇವರನ್ನು ಮನುಷ್ಯ ಕೆಟ್ಟದ್ದಕ್ಕೆ ಉಪಯೋಗಿಸಿಕೊಳ್ಳುತ್ತಾನೆ. ದೀಪ ಉರಿಯುತ್ತಿದ್ದರೆ, ಶ‍್ರೀರಾಮಕೃಷ್ಣರು ಹೇಳುವಂತೆ, ಒಬ್ಬ ಗೀತೆಯನ್ನಾದರೂ ಓದಬಹುದು, ಒಂದು ಸುಳ್ಳು ಅರ್ಜಿಯನ್ನಾದರೂ ಬರೆಯಬಹುದು. ಮಳೆ ಬಿದ್ದರೆ ಯಾರು ಚೆನ್ನಾಗಿ ಉತ್ತು ಬಿತ್ತಿರುವರೊ ಅವರಿಗೆ ಒಳ್ಳೆಯ ಬೆಳೆ ಸಿಕ್ಕುವುದು. ಕಾಡು ನೆಲದ ಮೇಲೆ ಬಿದ್ದರೆ, ಅಲ್ಲಿ ಯಾವ ಯಾವ ಬೀಜಗಳಿವೆಯೋ ಅವು ಕೂಡ ಹುಲುಸಾಗಿ ಬೆಳೆಯುತ್ತವೆ. ಮಳೆ ಪಕ್ಷಪಾತವಿಲ್ಲದೆ ಬೀಳುವುದು. ಅದನ್ನು ಉಪಯೋಗಪಡಿಸಿಕೊಳ್ಳುವುದು ನಮ್ಮ ಕೈಯಲ್ಲಿದೆ. ಬೆಂಕಿಯಂತೆ ದೇವರು. ಅದರಿಂದ ಒಂದು ಮನೆಯನ್ನು ಬೇಕಾದರೆ ಸುಡಬಹುದು, ಅಥವಾ ಅಡಿಗೆಯನ್ನು ಬೇಕಾದರೂ ಮಾಡಬಹುದು.

\begin{verse}
ತತ್ರೈವಂ ಸತಿ ಕರ್ತಾರಮಾತ್ಮಾನಂ ಕೇವಲಂ ತು ಯಃ~।\\ಪಶ್ಯತ್ಯಕೃತಬುದ್ಧಿತ್ವಾನ್ನ ಸ ಪಶ್ಯತಿ ದುರ್ಮತಿಃ \versenum{॥ ೧೬~॥}
\end{verse}

{\small ನಿಜಾಂಶ ಹೀಗಿರುವಾಗ ಯಾರು ಬುದ್ಧಿ ಇಲ್ಲದೆ ಕೇವಲ ಆತ್ಮವನ್ನು ಕರ್ತೃವೆಂದು ಭಾವಿಸುತ್ತಾನೆಯೊ ಆ ಹೀನ ಬುದ್ಧಿಯುಳ್ಳವನು ಸರಿಯಾಗಿ ನೋಡುವುದಿಲ್ಲ.}

ನಿಜವಾಗಿ ಎಲ್ಲಾಸೇರಿ ಒಂದು ಕೆಲಸವಾಗಿದೆ. ಅದಕ್ಕೆ ಯಾರೂ ಒಬ್ಬನೇ ಬರುವ ಫಲವನ್ನೆಲ್ಲ ತೆಗೆದುಕೊಳ್ಳುವುದಕ್ಕೆ ಆಗುವುದಿಲ್ಲ. ಚೆಂಡಾಟದಲ್ಲಿ ಕೊನೆಗೆ ಒಬ್ಬ ಗೋಲು ಹೊಡೆಯಬಹುದು. ಆದರೆ ಅವನಿಗೆ ಅಲ್ಲಿಯ ತನಕ ಚೆಂಡನ್ನು ತೆಗೆದುಕೊಂಡು ಹೋಗಿ ಕೊಟ್ಟವರು ಅವನ ಕಡೆಯವರು. ಗೋಲನ್ನು ಹೊಡೆದ ಕೀರ್ತಿ ಅವನಿಗೆ ಬರಬಹುದು. ಆದರೆ ಅದು ಆತನ ಕಡೆಯವರಿಗೆಲ್ಲ ಸೇರುವುದು. ಅವರೆಲ್ಲ ಇವನ ಹಿಂದೆ ನಿಂತು ಸಹಕರಿಸದೆ ಇದ್ದರೆ ಈ ಕೆಲಸ ಆಗುತ್ತಿರಲಿಲ್ಲ. ಮನುಷ್ಯ ಒಂದು ಕೆಲಸವನ್ನು ಸರಿಯಾಗಿ ವಿಭಜನೆ ಮಾಡಿದರೆ ಇದು ಗೊತ್ತಾಗುವುದು ಆ ಕೆಲಸ ಹೇಗೆ ಒಂದು ಮಿಶ್ರ ಎಂಬುದು. ಅದನ್ನು ವಿಭಜನೆ ಮಾಡಬೇಕಾದರೆ ಬುದ್ಧಿ ಹರಿತವಾಗಿರಬೇಕು. ಒಬ್ಬ ಶಾಸ್ತ್ರಾದಿ ಗಳನ್ನು ಓದಿರಬೇಕು. ಇಲ್ಲವೇ ತಾನೇ ಅದರಲ್ಲಿರುವುದನ್ನು ಚೆನ್ನಾಗಿ ವಿಮರ್ಶೆಮಾಡಿ ತಿಳಿದು ಕೊಳ್ಳುವ ಯೋಗ್ಯತೆ ಇರಬೇಕು. ಈ ಜೀವನದಲ್ಲಿ ಅನುಭಾವಿಗಳು ಯಾರೋ ಅವರನ್ನು ಕೇಳಿ ತಿಳಿದುಕೊಳ್ಳಲು ಯತ್ನಿಸಬೇಕು. ಇವುಗಳಿಲ್ಲದೆ ಒಬ್ಬ ಬುದ್ಧಿವಂತನಾಗಲಾರ, ಕೇವಲ ತನ್ನ ಜಡ ಕಣ್ಣಿಗೆ ಹೇಗೆ ತೋರುತ್ತದೆಯೋ ಅದನ್ನೇ ಸತ್ಯವೆಂದು ಭ್ರಮಿಸುತ್ತಾನೆ. ಅವನು ಆ ಕೆಲಸದ ಹಿಂದೆ ಹೋಗುವುದಿಲ್ಲ. ನಾನು ಈ ಬೆಳೆಯನ್ನು ಬೆಳೆದೆ; ಅದೆಲ್ಲ ನನ್ನ ಶ್ರಮದಿಂದ ಆದದ್ದು ಎಂದು ಭಾವಿಸುವನು. ಆದರೆ ಜೀವನದಲ್ಲಿ ಸತ್ಯ ಬೇರೆ, ತೋರಿಕೆ ಬೇರೆ. ನಾವು ಮಾಡುವ ಕೆಲಸವನ್ನೆಲ್ಲ ಮಾಡಿದರೂ ದೈವ ಸಹಾಯವಿಲ್ಲದೆ ಇದ್ದರೆ ಅದೆಲ್ಲ ನಿರರ್ಥಕವಾಗುವುದು.

ಹೀನಬುದ್ಧಿಯುಳ್ಳವನು ಒಂದು ಘಟನೆಯನ್ನು ಸರಿಯಾಗಿ ನೋಡುವ ಸ್ಥಿತಿಯಲ್ಲಿ ಇಲ್ಲ. ಇತರರಿಗೆ ಹೋಗುವ ಕೀರ್ತಿಯನ್ನೆಲ್ಲ ತಾನೆ ಬಾಚಿಕೊಳ್ಳಲು ಯತ್ನಿಸುವನು. ಈ ಜೀವನದಲ್ಲಿ ಯಾವುದು ನಮಗೆ ಸೇರಿದೆಯೊ ಅದನ್ನು ತಿಂದು ಅರಗಿಸಿಕೊಳ್ಳುವುದೇ ಕಷ್ಟ. ಇನ್ನು ನನಗೆ ಸೇರದೇ ಇರುವುದನ್ನು ನಾನು ತಿಂದರೆ, ಪಡಬಾರದ ಯಾತನೆ ಪಡಬೇಕು. ಅದರಿಂದ ನನ್ನ ದೇಹಾರೋಗ್ಯ ಕೆಡುವುದು. ಬೇಕಾದಷ್ಟು ಔಷಧಿ ತೆಗೆದುಕೊಳ್ಳಬೇಕು, ದುಡ್ಡು ಖರ್ಚು ಮಾಡಬೇಕು. ನನಗೆ ಸಲ್ಲದ ಕೀರ್ತಿಯನ್ನು ತೆಗೆದುಕೊಳ್ಳುವುದೂ ಹೀಗೆ. ಅದಕ್ಕೆ ಬೇಕಾದಷ್ಟು ಕಷ್ಟನಷ್ಟಗಳನ್ನು ಪಡಬೇಕಾಗಿದೆ.

\begin{verse}
ಯಸ್ಯ ನಾಹಂಕೃತೋ ಭಾವೋ ಬುದ್ಧಿರ್ಯಸ್ಯ ನ ಲಿಪ್ಯತೇ~।\\ಹತ್ವಾಪಿ ಸ ಇಮಾಂಲ್ಲೋಕಾನ್ನ ಹಂತಿ ನ ನಿಬಧ್ಯತೇ \versenum{॥ ೧೭~॥}
\end{verse}

{\small ಯಾರಿಗೆ ನಾನು ಕರ್ತೃ ಎಂಬ ಭಾವ ಇರುವುದಿಲ್ಲವೋ, ಯಾರ ಬುದ್ಧಿ ಮಲಿನವಾಗಿಲ್ಲವೊ, ಅವನು ಈ ಲೋಕವನ್ನೇ ಕೊಂದರೂ ಕೊಲ್ಲುವುದಿಲ್ಲ, ಬಂಧನದಲ್ಲಿ ಬೀಳುವುದಿಲ್ಲ.}

ಒಬ್ಬ ಕೆಲಸ ಮಾಡುತ್ತಿರುವಾಗ ಇದರಲ್ಲಿ ತನ್ನ ಭಾಗ ಎಷ್ಟು ದೇವರ ಭಾಗ ಎಷ್ಟು ಎಂಬುದನ್ನು ಚೆನ್ನಾಗಿ ತಿಳಿದುಕೊಂಡಿರಬೇಕು. ಯಾವಾಗ ವಿಭಜನೆ ಮಾಡುತ್ತಾನೋ ಆಗ ಅವನಿಗೆ ಗೊತ್ತಾಗು ವುದು, ನಾನು ಈ ಕೆಲಸವನ್ನು ಮಾಡುತ್ತಿಲ್ಲ, ನಾನೊಬ್ಬ ನಿಮಿತ್ತ, ಇದರ ಹಿಂದೆ ದೇವರೇ ಈ ಕೆಲಸವನ್ನು ಮಾಡಿಸುತ್ತಿರುವನು ಎಂದು. ಶ‍್ರೀಕೃಷ್ಣನೇ ಅರ್ಜುನನಿಗೆ ಹಿಂದಿನ ಅಧ್ಯಾಯದಲ್ಲಿ “ನನ್ನ ಕೈಯಲ್ಲಿ ನಿಮಿತ್ತವಾಗು.”, “ನನ್ನನ್ನು ಸ್ಮರಿಸುತ್ತ ಯುದ್ಧವನ್ನು ಮಾಡು” ಎಂದು ಹೇಳುವನು.

ಬುದ್ಧಿ ಮಲಿನತೆ ಹೋದರೆ ವಸ್ತುವಿನ ಯಥಾರ್ಥ ಸ್ಥಿತಿ ನಮಗೆ ಗೊತ್ತಾಗುವುದು. ಆಗ ಅಜ್ಞಾನಕ್ಕೆ ಒಳಗಾಗುವುದಿಲ್ಲ. ಅವನು ಕೇವಲ ಕರ್ತವ್ಯ ದೃಷ್ಟಿಯಿಂದ ಕೆಲಸ ಮಾಡುವನು. ಆ ಕರ್ತವ್ಯ ನಿರ್ವಹಣೆಯಿಂದ ಪಾಪಪುಣ್ಯಗಳು ಬಂದರೆ ಅದಕ್ಕೆ ನಾನು ಹೊಣೆಯಲ್ಲ ಎಂದು ಭಾವಿಸುವನು. ಅವನು ಈ ಲೋಕವನ್ನು ಕೊಂದರೂ ಕೊಲ್ಲುವುದಿಲ್ಲ. ಎಂದರೆ ಕೊಂದ ಪಾಪ ಅವನಿಗೆ ಬರುವುದಿಲ್ಲ. ಯುದ್ಧದಲ್ಲಿ ಸಿಪಾಯಿ ವೈರಿಗಳ ಪಕ್ಷದಲ್ಲಿ ಹಲವರನ್ನು ಕೊಲ್ಲುತ್ತಾನೆ. ಅದಕ್ಕಾಗಿ ಅವನನ್ನು ಮೆಚ್ಚುತ್ತಾರೆ. ಇಲ್ಲಿ ಕೊಲೆ ಒಂದು ದೃಷ್ಟಿಯಿಂದ ಪಾಪ. ಆದರೆ ದೇಶದ ರಕ್ಷಣೆಯ ದೃಷ್ಟಿಯಿಂದ ಅದು ಅತ್ಯಂತ ಆವಶ್ಯಕವಾಗಿ ಮಾಡಬೇಕಾದ ಕರ್ತವ್ಯ.

ಅವನಿಗೆ ಯಾವ ಬಂಧನವೂ ಇಲ್ಲ. ಅವನು ತಾನು ಮಾಡಿದ ಕೆಲಸದ ಪರಿಣಾಮಕ್ಕೆ ಸಿಕ್ಕಿ ತೊಂದರೆ ಪಡುವುದಿಲ್ಲ. ಯಾರೂ ಯುದ್ಧದಲ್ಲಿ ಶತ್ರುವನ್ನು ಕೊಂದಿದ್ದಕ್ಕೆ ಅವನಿಗೆ ಶಿಕ್ಷೆ ವಿಧಿಸುವು ದಿಲ್ಲ.

ಅರ್ಜುನನ ಮನಸ್ಸಿನಲ್ಲಿ ಬಾಧಿಸುತ್ತಿದ್ದ ಒಂದು ಸಮಸ್ಯೆ, ಯುದ್ಧದಲ್ಲಿ ಗುರು ಹಿರಿಯರನ್ನು ಕೊಲ್ಲಬೇಕಾಗುವುದಲ್ಲ, ಅದರ ಪಾಪ ನನಗೆ ಬರುವುದಲ್ಲ ಎಂಬುದು. ಅದಕ್ಕೆ ಉತ್ತರವಾಗಿಯೇ ಶ‍್ರೀಕೃಷ್ಣನು ಇದನ್ನು ಹೇಳುತ್ತಾನೆ. ಒಬ್ಬ ತಾನು ಫಲಾಪೇಕ್ಷೆಯನ್ನು ಬಿಟ್ಟು ಅನಾಸಕ್ತನಾಗಿ ನಿಂತರೆ, ಅವನು ಯಾವ ಸ್ವಾರ್ಥಕಾರ್ಯವನ್ನೂ ಮಾಡುವುದಿಲ್ಲ. ಯಾವ ಕೆಟ್ಟ ಕೆಲಸವೂ ಅವನಿಂದ ಆಗುವು ದಿಲ್ಲ. ಇಂತಹ ವ್ಯಕ್ತಿಗಳನ್ನು ದೇವರೇ ಒಂದು ನಿಮಿತ್ತವಾಗಿ ಆರಿಸಿಕೊಂಡು ತನ್ನ ಕೆಲಸವನ್ನು ಮಾಡುವನು. ಹೊರಗಿನಿಂದ ನೋಡಿದರೆ ಆ ವ್ಯಕ್ತಿ ಕೆಲಸ ಮಾಡುತ್ತಿರುವಂತೆ ಕಾಣುವುದು. ಆದರೆ ಸರಿಯಾಗಿ ಆ ವ್ಯಕ್ತಿಯನ್ನು ವಿಭಜನೆ ಮಾಡಿದರೆ, ಆ ವ್ಯಕ್ತಿ ಕೇವಲ ಭಗವಂತನ ಕೈಯಲ್ಲಿ ಒಂದು ನಿಮಿತ್ತ. ಆ ಕೆಲಸ ಮಾಡುತ್ತಿರುವವನು ದೇವರೆ. ಇದರ ಪಾಪ ಪುಣ್ಯಗಳೆಲ್ಲ ಅವನಿಗೆ ಸೇರಿದ್ದು. ಮಹಾಕವಿಯೊಬ್ಬ ಒಂದು ಲೇಖನಿಯ ಮೂಲಕ ಒಂದು ಅಮರ ಕಾವ್ಯವನ್ನೇ ಬರೆಯುತ್ತಾನೆ. ಆದರೆ ಆ ಲೇಖನಿಗೆ ಆ ಕೀರ್ತಿ ಸಲ್ಲುವುದೇ? ಹಾಗೆಯೇ ಭಗವಂತನೇ ಒಂದು ವ್ಯಕ್ತಿಯ ಹಿಂದೆ ನಿಂತುಕೊಂಡು ಲೋಕಕಲ್ಯಾಣ ಕಾರ್ಯವನ್ನು ಮಾಡುತ್ತಿರುವನು.\enginline{}

ಕೆಲವು ವೇಳೆ ನಾವು ಅಜ್ಞಾನ ಮತ್ತು ಸ್ವಾರ್ಥದಲ್ಲಿ ಮುಳುಗಿದ್ದರೂ ನಾನೇನು ಮಾಡುವುದು, ದೇವರೇ ಈ ಕೆಲಸವನ್ನು ಮಾಡಿಸಿಬಿಟ್ಟ ಎಂದು ಹೇಳಿ ತಪ್ಪಿಸಿಕೊಳ್ಳುವುದಕ್ಕೆ ಯತ್ನಿಸಬಹುದು. ಆದರೆ ಅವನು ನಿಜವಾಗಿ, ತನ್ನ ಹಿಂದೆ ಇರುವವನು ದೇವರೆ ಎಂದು ಅನುಭವಿಸಿದ್ದರೆ ಅವನಿಂದ ಯಾವ ಸಮಯದಲ್ಲಿಯೂ ಕೆಟ್ಟ ಕೆಲಸ ಆಗಲಾರದು. ಶ‍್ರೀರಾಮಕೃಷ್ಣರು ನುರಿತ ನೃತ್ಯಗಾರ ಎಂದಿಗೂ ತಪ್ಪು ಹೆಜ್ಜೆ ಇಡುವುದಿಲ್ಲ, ಎಂದು ಹೇಳುತ್ತಾರೆ. ಕೆಟ್ಟ ಕೆಲಸವನ್ನು ನಾವು ಮಾಡಿ ದೇವರನ್ನು ಹೊಣೆ ಮಾಡಿದರೆ ತಪ್ಪಾಗುವುದು. ದೇವರ ಆಲೋಚನೆ ಬಂದರೆ ಕೆಟ್ಟ ಕೆಲಸ ಆಗುವುದಿಲ್ಲ. ಕೆಟ್ಟ ಕೆಲಸದ ಆಲೋಚನೆ ಬಂದರೆ ದೇವರ ಜ್ಞಾಪಕವೇ ಇರುವುದಿಲ್ಲ. ನಾವು ಮಾಡಿದ ಕೆಟ್ಟದ್ದನ್ನು ಸಮರ್ಥಿಸಿ ಕೊಳ್ಳುವಾಗ ಅವನನ್ನು ಜ್ಞಾಪಿಸಿಕೊಳ್ಳುತ್ತೇವೆ.

\begin{verse}
ಜ್ಞಾನಂ ಜ್ಞೇಯಂ ಪರಿಜ್ಞಾತಾ ತ್ರಿವಿಧಾ ಕರ್ಮಚೋದನಾ~। \\ಕರಣಂ ಕರ್ಮ ಕರ್ತೇತಿ ತ್ರಿವಿಧಃ ಕರ್ಮಸಂಗ್ರಹಃ \versenum{॥ ೧೮~॥}
\end{verse}

{\small ಕರ್ಮ ಪ್ರೇರಣೆಯಲ್ಲಿ ಜ್ಞಾನ, ಜ್ಞೇಯ, ಪರಿಜ್ಞಾತೃಗಳೆಂದು ಮೂರು ವಿಧ. ಕರ್ಮದ ಅಂಗ ಮೂರು ಬಗೆಯಾಗಿದೆ. ಅದೇ ಇಂದ್ರಿಯ, ಕ್ರಿಯೆ, ಕರ್ತೃ.}

ಮನುಷ್ಯ ಒಂದು ಆಸೆಯಿಂದ ಪ್ರಚೋದಿತನಾದಾಗ ಕರ್ಮವನ್ನು ಮಾಡುತ್ತಾನೆ. ಆ ಪ್ರಚೋದನೆ ಯನ್ನು ವಿಭಜನೆ ಮಾಡಿದರೆ ಅಲ್ಲಿ ಮೂರು ಭಾಗವಿದೆ. ಮೊದಲನೆಯದೆ ಪರಿಜ್ಞಾತೃ. ಅವನು ತಿಳಿದುಕೊಳ್ಳಬಯಸುವನು. ಅನಂತರ ಆ ವಸ್ತು ಇವನ ಮೇಲೆ ತನ್ನ ಪರಿಣಾಮವನ್ನು ಬಿಡುತ್ತದೆ. ಅದು ಜ್ಞಾನ. ಒಂದು ಕ್ಯಾಮರಾ ಇದೆ. ಅದರಲ್ಲಿ ಲೆನ್ಸ್ ಇದೆ. ಇದು ಜ್ಞಾತೃ ಇದ್ದ ಹಾಗೆ. ಅದು ತನ್ನಿಂದ ಹೊರಗೆ ಬರುವ ಬೆಳಕನ್ನು ಹಿಡಿಯುವುದು. ಹೊರಗಿರುವ ವಸ್ತುವೇ ಜ್ಞೇಯವೆಂದು ತಿಳಿದುಕೊಳ್ಳಬೇಕಾಗುವುದು. ಹೊರಗಿನ ವಸ್ತುವಿನ ಛಾಯೆ ಒಳಗೆ ಇರುವ ಫಿಲಂನ ಮೇಲೆ ಬೀಳುವುದು. ಆಗ ಅಲ್ಲಿ ಒಂದು ಚಿತ್ರವಾಗುವುದು. ಅದು ಜ್ಞಾನ. ನಾನು ಜ್ಞಾತೃ, ನನ್ನಿಂದ ಹೊರಗಡೆ ಇರುವ ವಸ್ತು ಜ್ಞೇಯ, ಅದು ನನ್ನ ಮನಸ್ಸಿನ ಮೇಲೆ ಬಿಡುವ ಪರಿಣಾಮವೇ ಜ್ಞಾನ. ನಾವೊಂದು ಸೂಕ್ಷ್ಮಯಂತ್ರ ಹೇಗೆ ಕೆಲಸ ಮಾಡುತ್ತಿದೆ ಎಂಬುದನ್ನು ತಿಳಿದುಕೊಳ್ಳಬೇಕಾದರೆ, ಅದನ್ನು ಬಿಚ್ಚಿ, ಪ್ರತಿಯೊಂದು ಭಾಗವೂ ಯಾವ ಕೆಲಸವನ್ನು ಮಾಡುತ್ತದೆ ಎಂಬುದನ್ನು ಕಂಡುಹಿಡಿಯಬೇಕು. ಆಗಲೆ ಆ ಯಂತ್ರದ ಸೂಕ್ಷ್ಮ ನಮಗೆ ಗೊತ್ತಾಗಬೇಕಾದರೆ. ಅದರಂತೆಯೇ, ಮನಸ್ಸಿನಲ್ಲಿ ಹಲವು ರೀತಿಯ ಜ್ಞಾನಗಳಿವೆ. ಆ ಜ್ಞಾನ ಹೇಗೆ ಉಂಟಾಯಿತು ಎಂಬುದನ್ನು ತಿಳಿದುಕೊಳ್ಳಬೇಕು. ಆಗಲೇ ಅವುಗಳ ಮೇಲೆ ನಮಗೆ ಒಂದು ಸ್ವಾಧೀನ ಬರುವುದು.

ಕೆಲಸವನ್ನು ಮಾಡಿಸುವುದರಲ್ಲಿ ಕೂಡ ಮೂರು ಭಾಗಗಳಿವೆ. ಕರ್ತೃ ಅದನ್ನು ಮಾಡುವನು. ಇವನೇ ಜ್ಞಾತೃವೂ ಆಗಿರುವನು. ಮುಂಚೆ ಇದನ್ನು ತಿಳಿದುಕೊಳ್ಳುತ್ತಾನೆ. ಅನಂತರ ಇವನ ಕೆಲಸ ಮಾಡುತ್ತಾನೆ. ಎರಡನೆಯ ಭಾಗವೇ ಕರಣ. ಅವನು ಇಂದ್ರಿಯಗಳ ಸಹಾಯ ತೆಗೆದುಕೊಳ್ಳುತ್ತಾನೆ. ಇವೇ ಅವನಿಗೆ ಸಹಾಯಕವಾಗಿರುವ ಕರಣಗಳು. ಅನಂತರ ಅವುಗಳ ಮೂಲಕ ಕೆಲಸ ಮಾಡುತ್ತಾನೆ, ನೋಡುತ್ತಾನೆ, ಮೂಸುತ್ತಾನೆ, ಕೇಳುತ್ತಾನೆ, ಮುಟ್ಟುತ್ತಾನೆ. ಇವುಗಳ ಮೂಲಕ ತನ್ನ ಮನಸ್ಸಿನ ಚಪಲವನ್ನು ತೀರಿಸಿಕೊಳ್ಳುಲು ಯತ್ನಿಸುವನು.

\begin{verse}
ಜ್ಞಾನಂ ಕರ್ಮ ಚ ಕರ್ತಾ ಚ ತ್ರಿಧೈವ ಗುಣಭೇದತಃ~।\\ಪ್ರೋಚ್ಯತೇ ಗುಣಸಂಖ್ಯಾನೇ ಯಥಾವಚ್ಛೃಣು ತಾನ್ಯಪಿ \versenum{॥ ೧೯~॥}
\end{verse}

{\small ಜ್ಞಾನ, ಕರ್ಮ, ಕರ್ತೃಗಳು, ಗುಣಭೇದಾನುಸಾರವಾಗಿ ಮೂರು ಬಗೆಯಾಗಿವೆ. ಈ ಗುಣಗಣನೆಯಲ್ಲಿ ಯಾವ ಬಗೆಯ ವರ್ಣನೆ ಇದೆಯೋ ಅದನ್ನು ಕೇಳು.}

ಕರ್ಮ ಪ್ರಚೋದನೆಗೆ ಕಾರಣವಾಗಿರುವುದರಲ್ಲಿ ಜ್ಞಾನ ಮತ್ತು ಕರ್ತೃ ಇವೆ. ಕರ್ಮ ಮಾಡುವುದ ರಲ್ಲಿ ಕರ್ಮ ಮತ್ತು ಕರ್ತೃ ಇವೆ. ಇವೆರಡರಲ್ಲಿಯೂ ಸಾಮಾನ್ಯವಾಗಿರುವುದು ಕರ್ತೃ. ಈ ಕರ್ತೃ ಮೊದಲು ತಿಳಿದುಕೊಳ್ಳುತ್ತಾನೆ. ಅನಂತರ ಕರ್ಮವನ್ನು ಮಾಡುತ್ತಾನೆ. ತಿಳಿದುಕೊಳ್ಳುವುದು ಸೂಕ್ಷ್ಮ. ಅದೊಂದು ಬೌದ್ಧಿಕವಾಗಿರುವುದು. ತಿಳಿದುಕೊಂಡದ್ದನ್ನು ಅನುಷ್ಠಾನಕ್ಕೆ ತರುವುದೇ ಕ್ರಿಯೆ. ಇದು ಸ್ಥೂಲವಾಗಿರುವುದು.

ಜ್ಞಾನ, ಕರ್ಮ, ಮತ್ತು ಕರ್ತೃಗಳಲ್ಲಿ ಸಾತ್ತಿ ್ವಕ ರಾಜಸಿಕ ಮತ್ತು ತಾಮಸಿಕ ಎಂದು ಮೂರು ಬಗೆಯಾಗಿವೆ.

\begin{verse}
ಸರ್ವಭೂತೇಷು ಯೇನೈಕಂ ಭಾವಮವ್ಯಯಮೀಕ್ಷತೇ~।\\ಅವಿಭಕ್ತಂ ವಿಭಕ್ತೇಷು ತಜ್ಜ್ಞಾನಂ ವಿದ್ಧಿ ಸಾತ್ತ್ವಿಕಮ್ \versenum{॥ ೨0~॥}
\end{verse}

{\small ಯಾವುದರ ಮೂಲಕ ಮನುಷ್ಯ ಸಕಲ ಭೂತಗಳಲ್ಲಿಯೂ ಒಂದೇ ಅವಿನಾಶಿಯಾದ ಭಾವವನ್ನೂ, ವೈವಿಧ್ಯತೆ ಗಳಲ್ಲಿ ಏಕತ್ವವನ್ನು ನೋಡುತ್ತಾನೆಯೊ ಅದು ಸಾತ್ತ್ವಿಕ ಜ್ಞಾನ.}

ಸಕಲ ಭೂತಗಳ ವೈವಿಧ್ಯತೆಗಳಲ್ಲಿ ಅವಿನಾಶಿಯಾದ ಒಂದನ್ನು ನೋಡುವುದು ಸಾತ್ತಿ ್ವಕ ಜ್ಞಾನ. ಭೂತಗಳಲ್ಲಿ ಹೊರಗಿನ ನಾಮರೂಪ ವೇಷಗಳಿವೆ, ಒಳಗಿನ ಚೈತನ್ಯ ಇದೆ. ನಾಮರೂಪಗಳು ಯಾವಾಗಲೂ ಬದಲಾಯಿಸುತ್ತಿರುವುವು. ಅವುಗಳ ಹಿಂದೆ ಇರುವ ಪರಮಾತ್ಮ ವಸ್ತು ಎಂದಿಗೂ ಬದಲಾಯಿಸುವುದಿಲ್ಲ ಮತ್ತು ನಾಶವಾಗುವುದಿಲ್ಲ. ಹಲವಾರು ಮಡಿಕೆಕುಡಿಕೆಗಳು ಇರುತ್ತವೆ. ಅವುಗಳ ಒಳಗೆಯೂ ಆಕಾಶವಿದೆ. ಆ ಒಳಗಿರುವ ಆಕಾಶವೇ ಹೊರಗಿರುವ ಆಕಾಶದ ಭಾಗ. ಮಹಾಕಾಶವೇ ಘಟಾಕಾಶವಾಗುವುದು. ಯಾರು ಕೇವಲ ಮಡಕೆಯ ನಾಮರೂಪ ದೃಷ್ಟಿಯಿಂದ ನೋಡುತ್ತಾರೊ ಅವರು ಒಳಗಿರುವ ಆಕಾಶವನ್ನು ಗಮನಿಸುವುದಿಲ್ಲ. ಯಾರು ಆಕಾಶದ ದೃಷ್ಟಿ ಯಿಂದ ನೋಡುತ್ತಾರೆಯೊ ಅವರು ಮಡಕೆಕುಡಿಕೆಯ ನಾಮರೂಪಗಳನ್ನು ಗಮನಿಸುವುದಿಲ್ಲ. ಸಾತ್ತ್ವಿಕ ಜ್ಞಾನಿ, ಒಬ್ಬನೇ ಪರಮಾತ್ಮ ಸಕಲ ಜೀವರಾಶಿಗಳಲ್ಲಿ ಮತ್ತು ಜಗತ್ತಿನಲ್ಲಿ ಅವಿನಾಶಿಯಾ ಗಿರುವುದನ್ನು ನೋಡುತ್ತಾನೆ. ಅವನ ದೃಷ್ಟಿಯೆಲ್ಲ ಪರಮಾತ್ಮನ ಕಡೆಗೆ ಇರುವುದು. ಅವನನ್ನು ಮುಚ್ಚುವಂತೆ ಮಾಡುವ ನಾಮರೂಪಗಳ ಮೇಲೆ ಇಲ್ಲ. ಇದೇ ಪರಮಾತ್ಮದೃಷ್ಟಿ.

\begin{verse}
ಪೃಥಕ್ತ್ವೇನ ತು ಯಜ್ಜ್ಞಾನಂ ನಾನಾಭಾವಾನ್ ಪೃಥಗ್ವಿಧಾನ್~।\\ವೇತ್ತಿ ಸರ್ವೇಷು ಭೂತೇಷು ತಜ್ಜ್ಞಾನಂ ವಿದ್ಧಿ ರಾಜಸಮ್ \versenum{॥ ೨೧~॥}
\end{verse}

{\small ಯಾವ ಜ್ಞಾನ ಸಮಸ್ತ ಪ್ರಾಣಿಗಳಲ್ಲಿಯೂ ಭಿನ್ನ ಲಕ್ಷಣವುಳ್ಳ ನಾನಾ ಆತ್ಮಗಳನ್ನೇ ಬೇರೆ ಬೇರೆಯಾಗಿಯೇ ತಿಳಿಯುತ್ತದೆಯೋ ಆ ಜ್ಞಾನವನ್ನು ರಾಜಸ ಎಂದು ತಿಳಿ.}

ಈ ಶ್ಲೋಕದಲ್ಲಿ ಜೀವಾತ್ಮನ ದೃಷ್ಟಿಯಿಂದ ಎಲ್ಲವನ್ನೂ ನೋಡುವುದು. ಒಂದು ಜೀವಿಗೆ ಒಂದು ವ್ಯಕ್ತಿತ್ವವಿದೆ. ಅದು ಒಳ್ಳೆಯದು ಕೆಟ್ಟದ್ದು ಇಂತಹ ಎಷ್ಟೋ ಸ್ವಭಾವಗಳು ಒಂದು ಕಂತೆಯಂತೆ. ಯಾವಾಗ ನಾವು ಈ ದೃಷ್ಟಿಯಿಂದ ನೋಡುತ್ತೇವೆಯೋ, ಆಗ ಒಬ್ಬನಂತೆ ಮತ್ತೊಬ್ಬನಿಲ್ಲ. ಪ್ರತಿಯೊಬ್ಬರಲ್ಲಿಯೂ ಕರಗಿ ಹೋಗದ ವೈವಿಧ್ಯತೆಗಳಿವೆ. ಯಾವಾಗ ನಾವು ಹಲವನ್ನೇ ನೋಡುತ್ತೇವೆಯೋ, ಹಲವು ಸತ್ಯವಾಗಿ ನಮಗೆ ಕಾಣುವುದೊ ಆಗ ಕೆಲವನ್ನು ಪ್ರೀತಿಸುತ್ತೇವೆ, ಕೆಲವನ್ನು ದ್ವೇಷಿಸುತ್ತೇವೆ. ಮತ್ತೆ ಕೆಲವನ್ನು ಉದಾಸೀನ ಭಾವದಿಂದ ನೋಡುತ್ತೇವೆ. ಯಾವಾಗ ನಮಗೆ ಒಂದು ವಸ್ತುವಿನ ಮೇಲೆ ಪ್ರೀತಿ ದ್ವೇಷಗಳಿವೆಯೋ, ಆಗ ನಮ್ಮ ಮನಸ್ಸು ಶಾಂತವಾಗಿರುವುದಕ್ಕೆ ಆಗುವುದಿಲ್ಲ. ಅಲ್ಲೋಲಕಲ್ಲೋಲಗಳು ಆಗುತ್ತಲೇ ಇರುತ್ತವೆ. ಅಲ್ಲಿ ಪ್ರಶಾಂತಿ ಇರುವುದಿಲ್ಲ. ಅದೇ ರಾಜಸಿಕ ಜ್ಞಾನ.

\begin{verse}
ಯತ್ತು ಕೃತ್ಸ್ನವದೇಕಸ್ಮಿನ್ ಕಾರ್ಯೇ ಸಕ್ತಮಹೈತುಕಮ್~।\\ಅತತ್ತ್ವಾರ್ಥವದಲ್ಪಂ ಚ ತತ್ತಾಮಸುದಾಹೃತಮ್ \versenum{॥ ೨೨~॥}
\end{verse}

{\small ಆದರೆ ಯಾವ ಜ್ಞಾನ ಒಂದೇ ಜ್ಞಾನದಲ್ಲಿ ಸಂಪೂರ್ಣವಾಗಿರವಂತೆ ಆಸಕ್ತವಾಗಿರುವುದೋ, ಹೇತು ಶೂನ್ಯವೋ, ಸತ್ಯವಲ್ಲವೋ, ತುಚ್ಛವೋ, ಅದು ತಾಮಸವೆಂದು ಹೇಳಲ್ಪಟ್ಟಿದೆ.}

ದೇಹಾತ್ಮ ಭಾವವೇ ತಾಮಸಿಕ ದೃಷ್ಟಿ. ಇವರಿಗೆ ಜೀವ, ಈಶ್ವರ ಇವೆರಡೂ ಅನುಭವಕ್ಕೆ ಗೋಚರಿಸದ ವಸ್ತುಗಳು. ಎದುಗಿರುವ ಆವರ ದೇಹವೇ ಸತ್ಯವಾಗಿದೆ. ಸತ್ಯ ಅದನ್ನು ಮೀರಿಲ್ಲ. ಆ ದೇಹ ಹಿಂದೆ ಇರಲಿಲ್ಲ, ಮುಂದೆ ಇರುವುದಿಲ್ಲ. ಈಗ ಮಾತ್ರ ಇದೆ. ಈಗ ಅದು ಎಲ್ಲಕ್ಕಿಂತ ಹೆಚ್ಚು ಸತ್ಯವಾಗಿ ಕಾಣುತ್ತಿದೆ. ಇದನ್ನು ಮೀರಿ ಅದರ ದೃಷ್ಟಿ ಹೋಗುವುದಿಲ್ಲ. ಅವರಿಗೆ ಹಿಂದಿನ ಜನ್ಮ, ಮುಂದಿನ ಜನ್ಮ, ಕರ್ಮ, ಈಶ್ವರ ಮುಂತಾದುವುಗಳೆಲ್ಲಾ ಅರ್ಥವಾಗದ ಮಾತುಗಳು.

ಇದರಲ್ಲೆ ಅದು ಆಸಕ್ತವಾಗಿದೆ. ತಮ್ಮ ದೇಹಕ್ಕೆ ಸಂಬಂಧಪಟ್ಟ ನೆಂಟರಿಷ್ಟರಲ್ಲಿ, ಮನೆ ಮತ್ತು ವಸ್ತುಗಳಲ್ಲಿ ಆಸಕ್ತವಾಗಿದೆ. ದೇಹೊಂದೇ ಅದಕ್ಕೆ ಅತ್ಯಂತ ಮುಖ್ಯ ವಸ್ತು. ಆದಕಾರಣವೇ ಅದರಲ್ಲಿಆಸಕ್ತವಾಗಿದೆ. ಜೀವನದಲ್ಲಿ ಯಾವುದನ್ನಾದರೂ ಒಬ್ಬ ಹಿಡಿದುಕೊಳ್ಳಬೇಕಾಗಿದೆ. ಅದೇ ನಮಗೆ ಆಧಾರ. ಸಾತ್ವಿಕ, ದೇವರನ್ನು ಹಿಡಿದುಕೊಳ್ಳುತ್ತಾನೆ. ರಾಜಸಿಕ, ಜೀವವನ್ನು ಹಿಡಿದುಕೊಳ್ಳುತ್ತಾನೆ. ತಾಮಸಿಕ, ದೇಹವನ್ನು ಹಿಡಿದುಕೊಳ್ಳುತ್ತಾನೆ.

ಯಾವ ದೇಹವನ್ನು ಸತ್ಯವೆಂದು ಹಿಡಿದುಕೊಂಡಿರುವನೋ ಅದು ಹೇತು ಶೂನ್ಯ. ಅದನ್ನು ಸಮರ್ಥಿಸುವುದಕ್ಕೆ ಆಗುವುದಿಲ್ಲ. ಅದು ಹಿಂದೆ ಇರಲಿಲ್ಲ, ನಾವು ಸತ್ತು ಹೋದರೆ ದೇಹ ಕೊಳೆತು ನಾಶವಾಗುವುದು. ದೇಹ ತನಗೆ ತಾನೆ ಜೀವಂತವಾಗಿರಲಾರದು. ಜೀವಕ್ಕೂ ದೇಹಕ್ಕೂ ಸಂಬಂಧವಿದ್ದರೆ ಮಾತ್ರ ದೇಹ ಬದುಕಿರುವುದು. ಯಾವಾಗ ಸಂಬಂಧ ಕಡಿದುಹೋಯಿತೊ, ಆಗ ದೇಹ ನಾಶವಾಗುವುದು. ಆ ಜೀವ ಯಾವುದು, ಎಲ್ಲಿಂದ ಬಂತು, ಎಲ್ಲಿಗೆ ಹೋಗುತ್ತದೆ ಎಂಬುದನ್ನು ಇವನುವಿವರಿಸಲಾರ. ಇದು ಸತ್ಯವಲ್ಲ. ಈ ದೇಹ ಜೀವದ ಗೂಡು. ಎಂದರೆ ಯಾವಾಗಲೂ ಇರಲಿಲ್ಲ. ಒಂದು ಕಾಲದಲ್ಲಿ ಹುಟ್ಟಿತು. ಹಲವು ಬದಲಾವಣೆಗಳ ಮೂಲಕ ಸಾಗಿ ಈಗ ಒಂದು ಸ್ಥಿತಿಗೆ ಬಂದು ನಿಂತಿದೆ. ಇನ್ನು ಸ್ವಲ್ಪ ಕಾಲದ ಮೇಲೆ ನಾಶವಾಗಿ ಹೋಗುವುದು. ಇದು ಕೇವಲ ತೋರಿಕೆ. ಮತ್ತಾರಿಗೊ ಈ ದೇಹ ಇದೆ. ಅವನಾರು ಎಂಬುದು ತಾಮಸಿಕ ಜ್ಞಾನಿಗೆ ಅರಿವೇ ಇಲ್ಲ. ಇದು ಒಂದು ಬೆಳಗುತ್ತಿರುವ ವಿದ್ಯುತ್ ಬಲ್ಬನ್ನು ತೆಗೆದುಕೊಂಡು, ಬಲ್ಬೇ ಉರಿಯುತ್ತಿದೆ ಎಂಬಂತೆ. ನಿಜವಾಗಿ ಬಲ್ಬಲ್ಲ ಉರಿಯುತ್ತಿರುವುದು. ವಿದ್ಯುತ್ ಶಕ್ತಿ ಬಲ್ಬಿನ ಮೂಲಕ ಉರಿಯುತ್ತಿದೆ ಅಷ್ಟೇ. ಬಲ್ಬು ಒಂದು ಮಧ್ಯವರ್ತಿ. ಸತ್ಯ ಅದರ ಹಿಂದೆ ಇರುವ ವಿದ್ಯುತ್ ಶಕ್ತಿ.

ದೇಹವೇ ಸರ್ವಸ್ವ ಎಂಬುದು ತುಚ್ಛವಾದುದು. ಜೀವವನ್ನು ಕಳೆದರೆ ಈ ದೇಹಕ್ಕೆ ಬೆಲೆಯಿಲ್ಲ. ಎಷ್ಟೇ ಗಟ್ಟಿಮುಟ್ಟಾದ ದೇಹವಾದರೂ ಕೊಳೆತು ನಾರುವುದು. ಬದುಕಿರುವಾಗ ಎಷ್ಟೇ ಪ್ರಿಯವಾಗಿ ದ್ದರೂ ಅದೊಂದು ಶವವಾಗಿ ಕೊಳೆತು ನಾರುತ್ತಿರುವಾಗ ಅದಕ್ಕೆ ಮುಂಚೆ ಸರ್ವಸ್ವವನ್ನು ಅರ್ಪಣೆ ಮಾಡಿದವನಾವನೂ ಅದರ ಹತ್ತಿರ ಸುಳಿಯುವುದಿಲ್ಲ. ನೀರು ಗುಳ್ಳೆಯಂತೆ ಕೆಲವು ಕಾಲವಿದ್ದು ಕೊನೆಗೆ ಬೊಗಸೆ ಬೂದಿಯಾಗುವ ಈ ದೇಹವೇ ಸರ್ವಸ್ವ ಎಂದು ಭ್ರಮಿಸುವುದೇ ತಾಮಸಿಕ ಜ್ಞಾನ.

\begin{verse}
ನಿಯತಂ ಸಂಗರಹಿತಮರಾಗದ್ವೇಷತಃ ಕೃತಮ್~।\\ಅಫಲಪ್ರೇಪ್ಸುನಾ ಕರ್ಮ ಯತ್ತತ್ಸಾತ್ತ್ವಿಕಮುಚ್ಯತೇ \versenum{॥ ೨೩~॥}
\end{verse}

{\small ನಿಯತವಾದ ಸಂಗರಹಿತವಾದ ಫಲಾಪೇಕ್ಷೆ ಇಲ್ಲದವನಿಂದ ರಾಗ ದ್ವೇಷವಿಲ್ಲದೆ ಮಾಡಲ್ಪಟ್ಟ ಕರ್ಮವನ್ನು ಸಾತ್ತ್ವಿಕ ಕರ್ಮ ಎಂದು ಹೇಳುತ್ತಾರೆ.}

ಸಾತ್ತಿ ್ವಕವಾಗಿ ಮಾಡುವ ಕರ್ಮದ ಸ್ವಭಾವ ಎಂತಹುದು ಎಂಬುದನ್ನು ವಿವರಿಸುತ್ತಾನೆ. ಅದು ನಿಯತವಾದ ಕರ್ಮ ಆಗಿರಬೇಕು. ತನ್ನ ವರ್ಣಕ್ಕೆ ಸೇರಿದ ಮಾಡಲೇ ಬೇಕಾದ ಕರ್ಮವಾಗಿರಬೇಕು. ಅವನು ಮತ್ತಾರದೊ ಕರ್ಮವನ್ನು ಮಾಡುವುದಿಲ್ಲ. ಫಲ ಬರದೆ ಹೋಗುವುದಿಲ್ಲ, ಬರುವುದು. ಫಲಕ್ಕಾಗಿ ಮಾಡಿದರೆ ಎಷ್ಟು ಬರುವುದೊ ಅದಕ್ಕೆ ಹತ್ತರಷ್ಟು ಫಲ ಅದರ ಇಚ್ಛೆಯಿಲ್ಲದೇ ಇದ್ದರೆ ಬರುವುದು. ಆದರೆ ಆ ಫಲವನ್ನು ಭಗವದರ್ಪಣೆ ಮಾಡಿಬಿಡುತ್ತಾನೆ. ಆ ಕೆಲಸ ಮಾಡಿದಾಗ ಇವನನ್ನೆ ಜನ ಬೇಕಾದಷ್ಟು ಕೊಂಡಾಡಬಹುದು. ಏಕೆಂದರೆ ಜನರಿಗೆ ಇವನೊಬ್ಬನೇ ಕಾಣುವುದು. ಆದರೆ ಇವನಿಗೆ ಗೊತ್ತಿದೆ ಯಾರು ಈ ಕೆಲಸವನ್ನು ಮಾಡಿಸಿದರು ಎಂಬುದು. ಸೂತ್ರದ ಗೊಂಬೆಆಟದಲ್ಲಿ, ಆ ಗೊಂಬೆಯ ತಲೆ ಕೈಕಾಲು ಇವುಗಳಿಗೆಲ್ಲ ಹಗ್ಗವನ್ನು ಕಟ್ಟಿ, ಮೇಲಿನಿಂದ ಅದನ್ನು ಆಯಾ ದಾರಗಳನ್ನು ಎಳೆದು ಆಡಿಸುವನು. ಜನ, ಗೊಂಬೆಗಳು ಎಷ್ಟು ಚೆನ್ನಾಗಿ ಕುಣಿಯುತ್ತವೆ ಎಂದು ಹೊಗಳುವರು. ಆದರೆ ಕೀರ್ತಿ ನಿಜವಾಗಿ ಸಲ್ಲಬೇಕಾಗಿರುವುದು ಸೂತ್ರದ ಗೊಂಬೆಗಲ್ಲ, ಅದನ್ನು ಕುಣಿಸುತ್ತಿರುವವನಿಗೆ.

ಕೆಲಸವನ್ನು ಮಾಡುವಾಗ ಅವನಲ್ಲಿ ಆ ಕೆಲಸ ಪ್ರಿಯವಾಗಿ ಇದ್ದರೆ ಅದರ ಮೇಲೆ ರಾಗ ಎಂದರೆ ಆಕರ್ಷಣೆಯೂ ಇಲ್ಲ. ಆ ಕೆಲಸವನ್ನು ತಾನು ಮಾಡಬೇಕೆಂದು ಇಚ್ಛಿಸುವುದಿಲ್ಲ. ಅದು ತನ್ನ ಪಾಲಿಗೆ ಬಂದಿದೆ. ಅದನ್ನು ಮಾಡಲೇ ಬೇಕಾಗಿರುವುದರಿಂದ ಮಾಡಿ ಹಾಕುತ್ತಾನೆ. ಹಾಗೆಯೇ ಅಪ್ರಿಯ ವಾಗಿರುವುದು ಬಂದಾಗ ಏತಕ್ಕಾದರೂ ನನ್ನ ಪಾಲಿಗೆ ಈ ಕೆಲಸ ಬಂದಿತೋ ಎಂದು ಗೊಣಗಾಡು ವುದೂ ಇಲ್ಲ. ಕೆಲಸ ಮಾಡುವಾಗ ಒಮ್ಮೆ ಪ್ರಿಯವಾಗಿರುವುದು ಬರುವುದು, ಒಮ್ಮೆ ಅಪ್ರಿಯವಾಗಿ ರುವುದು ಬರುವುದು. ಎರಡನ್ನೂ ಅವನು ಅನಾಸಕ್ತನಾಗಿ ಮಾಡುವನು. ಇದೇ ಸಾತ್ತಿ ್ವಕವಾಗಿ ಕರ್ಮವನ್ನು ಮಾಡುವ ವಿಧಾನ.

\begin{verse}
ಯತ್ತು ಕಾಮೇಪ್ಸುನಾ ಕರ್ಮ ಸಾಹಂಕಾರೇಣ ವಾ ಪುನಃ~।\\ಕ್ರಿಯತೇ ಬಹುಲಾಯಾಸಂ ತದ್ರಾಜಸಮುದಾಹೃತಮ್ \versenum{॥ ೨೪~॥}
\end{verse}

{\small ಆದರೆ ಫಲಾಕಾಂಕ್ಷೆಯುಳ್ಳವನು, ಅಥವಾ ಅಹಂಕಾರಿಯು ಬಹು ಆಯಾಸದಿಂದ ಕೂಡಿ ಮಾಡುವ ಕರ್ಮವನ್ನು ರಾಜಸ ಎಂದು ಹೇಳುತ್ತಾರೆ.}

ರಾಜಸಿಕ ಕರ್ಮಿ ಫಲದ ಮೇಲೆ ಕಣ್ಣಿಟ್ಟುಕೊಂಡೇ ಕೆಲಸ ಮಾಡುತ್ತಾನೆ. ಅವನಿಗೆ ಆ ಫಲ ಬೇಕಾಗಿದೆ. ಅದರಿಂದ ಬರುವ ಕೀರ್ತಿ ಅದರಿಂದ ಆಗುವ ಲಾಭ ಮುಂತಾದುವುಗಳೆಲ್ಲ ಅವನಿಗೆ ಬೇಕಾಗಿರುವುದರಿಂದಲೇ ಅವನು ಕರ್ಮ ಮಾಡುತ್ತಾನೆ. ಫಲಾಪೇಕ್ಷೆ ಬಿಡು ಎಂದರೆ, ಇನ್ನು ನಾನು ಕರ್ಮವನ್ನು ಏತಕ್ಕೆ ಮಾಡಬೇಕು ಎನ್ನುತ್ತಾನೆ. ಕರ್ಮದ ಉದ್ದೇಶವೇ ಅದರ ಹಣ್ಣನ್ನು ತಾನು ತಿನ್ನುವುದು.

ಅವನು ಕೆಲಸ ಮಾಡುತ್ತಿರುವಾಗ, ಇದನ್ನು ನಾನು ಮಾಡುತ್ತಿರುವುದು ಎಂದು ತಾದಾತ್ಮ್ಯಭಾವ ವನ್ನು ಕೆಲಸದೊಂದಿಗೆ ಪಡೆದಿರುವನು. ಆ ಕೆಲಸ ಮಾಡಿ, ಆದ ಮೇಲೆಯೂ ಅದನ್ನು ತಾನೆ ಹೊಗಳಿಕೊಳ್ಳುತ್ತಿರುವನು. ಇಂತಹ ಅದ್ಭುತವಾದ ಕೆಲಸವನ್ನು ಸಾಧಿಸಿದೆ ನಾನು ಎಂದು ಎಲ್ಲರೆದು ರಿಗೂ ಹೇಳಿಕೊಳ್ಳುತ್ತಿರುವನು.

ಅವನು ಕೆಲಸ ಮಾಡುವಾಗ ತುಂಬಾ ಆಯಾಸವನ್ನು ನೋಡುತ್ತೇವೆ. ಅವನು ತನ್ನ ಆಧಾರದ ಮೇಲೆಯೇ ನಿಂತು ಕೆಲಸ ಮಾಡಬೇಕಾಗಿದೆ. ಭಿನ್ನ ಭಿನ್ನ ಪ್ರಕೃತಿಯವರೊಡನೆ ಬೆರೆಯಬೇಕಾಗಿದೆ. ಆಗ ಎಷ್ಟೋ ಫರ್ಷಣೆ ಮತ್ತು ಭಿನ್ನಾಭಿಪ್ರಾಯಗಳು ಬರುತ್ತವೆ. ಈತನಲ್ಲಿ ಅಧಿಕಾರ ಲಾಲಸೆ ಇದೆ. ಎಲ್ಲರ ಮೇಲೆಯೂ ಅದನ್ನು ಚಲಾಯಿಸುತ್ತಿರುವನು. ಅವನು ಮಾಡುವ ಕೆಲಸದಲ್ಲಿ ಗಲಾಟೆ ಜಾಸ್ತಿ. ಪುರೋಹಿತರ ಮಂತ್ರಕ್ಕಿಂತ ಉಗುಳೇ ಜಾಸ್ತಿ ಎನ್ನುವಂತೆ. ನಿಜವಾಗಿ ನಮ್ಮ ಹಿಂದೆ ನಿಂತು ಮಾಡಿಸುತ್ತಿರುವವನು ದೇವರು. ಅವನನ್ನು ನೆಚ್ಚಿದರೆ ಯಾವ ಆಯಾಸವೂ ಇಲ್ಲದೆ ಕೆಲಸ ಸುಸೂತ್ರವಾಗಿ ನೆರೆವೇರಿ ಹೋಗುವುದು. ಯಾವಾಗ ದೇವರನ್ನು ಮರೆಯುವೆವೋ, ಆಗ ಆ ಭಾರವನ್ನು ವಹಿಸಬೇಕಾಗುವುದು. ರೈಲಿನಲ್ಲಿ ಕುಳಿತುಕೊಂಡು ಹೋಗುತ್ತಿರುವಾಗ ಸಾಮಾನನ್ನು ಕೆಳಗೋ, ಮೇಲೋ ಇಡುವ ಬದಲು ತನ್ನ ತೊಡೆಯ ಮೇಲೆ ಅದನ್ನು ಹೊತ್ತುಕೊಂಡಿರುವಂತೆ. ನನ್ನ ತೊಡೆಯ ಮೇಲೆ ಆ ಸಾಮಾನಿರುವಾಗಲೂ ಹೊರುವುದು ರೈಲೆ. ಆದರೆ ನಾನು ವೃಥಾ ಅಜ್ಞಾನದಿಂದ ಅದನ್ನು ಹೊತ್ತು ವ್ಯಥೆ ಪಡೆಯುವೆನು. ಕೊಡ ನೀರನಲ್ಲಿರುವಾಗ ಅದರ ಭಾರ ಕಾಣುವುದಿಲ್ಲ. ನೀರಿನಿಂದ ಮೇಲಕ್ಕೆ ಬಂದಾಗ ಅದು ಭಾರವಾಗುವುದು. ಹಾಗೆಯೇ ದೇವರಲ್ಲಿ ನಾವಿದ್ದು ಕೆಲಸ ಮಾಡಿದರೆ, ನಮಗೆ ಆಯಾಸವಿಲ್ಲ, ಶ್ರಮವಿಲ್ಲ, ಕೆಲಸ ಸುಸೂತ್ರವಾಗಿ ಸಾಗುವುದು. ಯಾವಾಗ ಅವನನ್ನು ಮರೆಯುವೆವೊ, ನಾವೇ ಈ ಕೆಲಸ ಮಾಡುತ್ತಿರುವವರು ಎಂದು ಭಾವಿಸು ವೆವೊ, ಆಗ ರೈಲಿನಲ್ಲಿ ಕುಳಿತುವನು ಸಾಮಾನನ್ನು ಹೊತ್ತಂತೆ ಆಗುವುದು.

\begin{verse}
ಅನುಬಂಧಂ ಕ್ಷಯಂ ಹಿಂಸಾಮನವೇಕ್ಷ್ಯ ಚ ಪೌರುಷಮ್~।\\ಮೋಹಾದಾರಭ್ಯತೇ ಕರ್ಮ ಯತ್ತತ್ತಾಮಸಮುಚ್ಯತೇ \versenum{॥ ೨೫~॥}
\end{verse}

{\small ಮನುಷ್ಯ ಯಾವ ಕೆಲಸವನ್ನು, ಅದರ ಪರಿಣಾಮ ಹಾನಿ ಹಿಂಸೆ ತನ್ನ ಸಾಮರ್ಥ್ಯಗಳ ವಿಚಾರ ಮಾಡದೆ ಮೋಹವಶನಾಗಿ ಪ್ರಾರಂಭಿಸುವನೋ ಅದು ತಾಮಸ ಎಂದು ಹೇಳುತ್ತಾರೆ.}

ತಾಮಸಿ ಕೆಲಸ ಮಾಡುವಾಗ ಆ ಕೆಲಸದಿಂದ ಎಂತಹ ಪರಿಣಾಮ ಆಗಬಹುದು ಎಂಬುದನ್ನು ವಿಚಾರಿಸುವುದಕ್ಕೆ ಹೋಗುವುದಿಲ್ಲ. ಇದು ತನಗೇ ಸಂಕಟ ತರಬಹುದು, ನೆರೆಹೊರೆಯವರಿಗೆಲ್ಲ ವಿಪತ್ತನ್ನು ತರಬಹುದು. ಕಪಿ ಅರ್ಧ ಸೀಳಿದ ತೊಲೆಯ ಮಧ್ಯೆ ಇಟ್ಟ ಬೆಣೆಯನ್ನು ತೆಗೆದು ಬಾಲವನ್ನು ಅದರ ಮಧ್ಯದಲ್ಲಿ ಸಿಕ್ಕಿಸಿಕೊಂಡಂತೆ ಆಗುವುದು. ಯಾವುದೊ ಉದ್ಯಮಕ್ಕೆ ಕೈಹಾಕುತ್ತಾನೆ. ಪಾಪರ್ ಆಗುತ್ತಾನೆ. ಮುಂಚೆಯೇ ಆಲೋಚನೆ ಮಾಡುವುದಿಲ್ಲ. ಆ ಕೆಲಸ ಸರಿಯಾಗಿರುವುದೇ ಇಲ್ಲವೇ ಎಂದು.

ಅವನು ಮಾಡುವ ಕೆಲಸದಿಂದ ಹಾನಿಯೇ, ನಷ್ಟವೇ ಜಾಸ್ತಿ. ದೊಡ್ಡ ಒಂದು ಮನೆ ಕಟ್ಟುವುದಕ್ಕೆ ಯೋಜನೆ ತಯಾರು ಮಾಡುತ್ತಾನೆ. ಇರುವ ದುಡ್ಡನ್ನೆಲ್ಲ ಖರ್ಚು ಮಾಡಿ ಒಂದು ತಳಪಾಯ ಹಾಕುತ್ತಾನೆ. ಅಲ್ಲಿಗೇ ನಿಲ್ಲುವುದು ಅದು. ದುಡ್ಡೂ ಇಲ್ಲ, ಮನೆಯೂ ಇಲ್ಲ. ಅವನ ಅನೇಕ ಕೆಲಸಗಳು ಈ ಗುಂಪಿಗೆ ಸೇರಿದವು. ಅವನು ತನ್ನ ಕೆಲಸವನ್ನು ಮಾಡುವುದಕ್ಕಾಗಿ ಇತರರಿಗೆ ತೊಂದರೆ ಕೊಡಬೇಕಾಗಿದೆ. ಎಲ್ಲರಿಂದಲೂ ಅಸಮಾಧಾನವನ್ನು ಕಟ್ಟಿಕೊಳ್ಳಬೇಕು. ಯಾರಿಗೂ ಸರಿಯಾಗಿ ಕೂಲಿ ಕೊಡುವುದಿಲ್ಲ. ಯಾರಿಂದ ಸಾಲ ತಂದಿರುತ್ತಾನೆಯೊ, ಅವರಿಗೆ ಅಸಲು ಕೊಡುವುದಕ್ಕೆ ಆಗುವುದಿಲ್ಲ. ಬಡ್ಡಿಯನ್ನೂ ಕೊಡುವುದಕ್ಕೆ ಆಗುವುದಿಲ್ಲ. ಒಂದು ಸಲ ಇವನಿಗೆ ಸಹಾಯ ಮಾಡಿರುವವನು ಇನ್ನೊಂದು ಸಲ ಇವನಿಗೆ ಸಹಾಯ ಮಾಡಲು ಮುಂದೆ ಬರುವುದಿಲ್ಲ. ಒಂದು ಸಲ ಕೆಲಸಕ್ಕೆ ಬಂದವನು ಇನ್ನೊಂದು ಸಲ ಕೆಲಸಕ್ಕೆ ಬರುವುದಿಲ್ಲ. ತಾನು ಕಷ್ಟ ಅನುಭವಿಸುವುದು ಮಾತ್ರವಲ್ಲ, ಸುತ್ತಲೂ ಇರುವವರನ್ನೆಲ್ಲ ಕಷ್ಟಕ್ಕೆ ಸಿಕ್ಕಿಸುತ್ತಾನೆ.

ಕೆಲಸ ಮಾಡುವುದಕ್ಕೆ ತನಗೆ ಯೋಗ್ಯತೆ ಇದೆಯೇ ಇಲ್ಲವೇ ಸ್ವಲ್ಪವೂ ಅದನ್ನು ಕುರಿತು ಯೋಚಿಸುವುದಿಲ್ಲ. ಕೈಯಲ್ಲಿ ಕಾಸೇ ಇಲ್ಲ. ದೊಡ್ಡ ಉದ್ಯಮಕ್ಕೆ ಕೈಹಾಕುತ್ತಾನೆ. ತನಗೆ ಯೋಗ್ಯತೆಯೇ ಇಲ್ಲ. ಒಂದು ದೊಡ್ಡ ಪುಸ್ತಕ ಬರೆಯಲು ಪ್ರಯತ್ನ ಮಾಡುತ್ತಾನೆ, ದೊಡ್ಡ ಉಪನ್ಯಾಸ ಕೊಡಲು ಹೋಗುತ್ತಾನೆ. ಯುದ್ಧಕ್ಕೆ ತರಬೇತಿ ತೆಗೆದುಕೊಂಡಿಲ್ಲ, ಯುದ್ಧಕ್ಕೆ ಹೋಗುತ್ತಾನೆ. ಇದು ಈಜನ್ನೇ ಕಲಿತಿಲ್ಲ, ನೀರಿಗೆ ಈಜಲು ಬಿದ್ದಂತೆ ಇದೆ.

ಅವನು ಕೆಲಸ ಮಾಡುವಾಗ ಮೋಹದಿಂದ ಪ್ರಾರಂಭ ಮಾಡುತ್ತಾನೆ. ಯಾವುದನ್ನೊ ಮಾಡ ಬೇಕೆಂದು ಆಸೆ ಇದೆ. ಆದರೆ ಅದನ್ನು ಸಾಧಿಸಲು ಯೋಗ್ಯತೆ ಇಲ್ಲ. ಆ ಕೆಲಸದ ಸಾಧಕಬಾಧಕಗಳೇನು ಎಂಬುದನ್ನು ಸ್ವಲ್ಪವೂ ವಿಚಾರಿಸುವುದಿಲ್ಲ.

\begin{verse}
ಮುಕ್ತಸಂಗೋಽನಹಂವಾದೀ ಧೃತ್ಯುತ್ಸಾಹಸಮನ್ವಿತಃ~।\\ಸಿದ್ಧ್ಯಸಿದ್ಧ್ಯೊರ್|ನಿರ್ವಿಕಾರಃ ಕರ್ತಾ ಸಾತ್ತ್ವಿಕ ಉಚ್ಯತೇ \versenum{॥ ೨೬~॥}
\end{verse}

{\small ಫಲಸಂಗರಹಿತನೂ, ಅನಹಂಕಾರಿಯೂ, ಧೃತಿ ಉತ್ಸಾಹಗಳಿಂದ ಕೂಡಿದವನೂ, ಸಿದ್ಧಿ ಅಸಿದ್ಧಿ ಇವುಗಳಲ್ಲಿ ನಿರ್ವಿಕಾರನೂ ಆದ ಕರ್ತೃವನ್ನು ಸಾತ್ತ್ವಿಕ ಎಂದು ಹೇಳುತ್ತಾರೆ.}

ಸಾತ್ತ್ವಿಕ ಕರ್ತೃವಿನ ಸ್ವಭಾವವನ್ನು ಇಲ್ಲಿ ಹೇಳುತ್ತಾನೆ. ಅವನು ಕೆಲಸ ಮಾಡುವಾಗ ಫಲಕ್ಕೆ ಅಂಟಿಕೊಂಡಿರುವುದಿಲ್ಲ. ಅವನಲ್ಲಿ ಕೆಲಸ ಮಾಡುವಾಗ ಅದರಲ್ಲಿ ಆಸಕ್ತಿಯೂ, ಬಿಡಬೇಕಾಗಿ ಬಂದಾಗ ವ್ಯಥೆಯೂ ಇಲ್ಲ. ನಾನು ಇದನ್ನು ಮಾಡುತ್ತಿರುವೆ ಎಂಬ ಅಭಿಮಾನವಿಲ್ಲ. ಈ ಗುಣಗಳನ್ನೆಲ್ಲ ಸಾತ್ತ್ವಿಕ ಕರ್ಮದಲ್ಲಿಯೂ ನೋಡುತ್ತೇವೆ. ಇದಕ್ಕಿಂತ ಮುಂದೆ ಹೋಗುವನು ಇಲ್ಲಿ.

ಸಾತ್ತ್ವಿಕ ಕರ್ತೃವಿನಲ್ಲಿ ಧೃತಿ ಇದೆ. ಧೈರ್ಯವಿದೆ. ತನ್ನ ಪಾಲಿಗೆ ಬಂದ ಕೆಲಸವನ್ನು ಮಾಡುವಾಗ ಯಾವ ಅಡ್ಡಿ ಆತಂಕಗಳು ಬರಲಿ ಅದಕ್ಕೆ ಅಂಜಿ ಕುಳಿತುಕೊಳ್ಳುವವನಲ್ಲ. ಅವುಗಳೊಂದಿಗೆ ಹೋರಾಡಿ ಪಾರಾಗುತ್ತಾನೆ. ಯಾವ ಕಷ್ಟ ಎದುರಿಸಿದರೂ ಅವನು ಸೋಲನ್ನು ಒಪ್ಪಿಕೊಳ್ಳುವವನಲ್ಲ. ಅವನಿಗೆ ತನ್ನ ಶಕ್ತಿಗಿಂತ ಮಿಗಿಲಾದುದು ಹಿಂದೆ ಇದೆ ಎಂಬುದು ಚೆನ್ನಾಗಿ ಅರ್ಥವಾಗಿರುವುದು. ಅದೇ ಭಗವಂತನ ಶಕ್ತಿ. ಆ ಭಗವಂತನ ಶಕ್ತಿಗೆ ಯಾವುದೂ ಅಸಾಧ್ಯವಲ್ಲ ಎಂಬುದು ಅವನಿಗೆ ಗೊತ್ತಿದೆ.

ಅವನು ಉತ್ಸಾಹದಿಂದ ತುಂಬಿ ತುಳುಕಾಡುತ್ತಿರುವನು. ಅಯ್ಯೋ ಮಾಡಬೇಕಲ್ಲ ಎಂಬ ಗೋಳಾಗಲಿ, ಏನೋ ಮಾಡಬೇಕಾಗಿದೆ ಎಂಬ ಕಾಟಾಚಾರವಾಗಲಿ ಇಲ್ಲ. ಅದನ್ನು ಮಾಡುವಾಗ ಸಂತೋಷದಿಂದ ಮಾಡುವನು. ಅವನೊಡನೆ ಕೆಲಸ ಮಾಡುವವರಿಗೂ ಆ ಸಂತೋಷ ಜಾಡ್ಯದಂತೆ ಹರಡುವುದು. ಯಾವಾಗ ಭಗವಂತನ ಕೈಯಲ್ಲಿ ತಾನಿರುವೆನು, ಭಗವಂತ ಸಚ್ಚಿದಾನಂದ ಸ್ವರೂಪ ಎಂದು ತಿಳಿದಿರುವನೊ ಅವನಲ್ಲಿ ಅಳು ಮೋರೆ ಹೇಗೆ ಇರಬಲ್ಲುದು?

ಕೆಲಸ ಕೈಗೂಡಿದರೆ ಅವನು ಕುಣಿದಾಡುವುದೂ ಇಲ್ಲ, ಕೈಗೂಡದೆ ಇದ್ದರೆ ನಿರಾಶನೂ ಆಗುವು ದಿಲ್ಲ. ದೇವರು ಎಲ್ಲದರ ಮೂಲಕವೂ ಕೆಲಸ ಮಾಡುವನು. ಜಯ ಎಷ್ಟು ಮುಖ್ಯವೋ ಜೀವನದಲ್ಲಿ ಸೋಲೂ ಅಷ್ಟೇ ಮುಖ್ಯ. ಇದನ್ನು ಚೆನ್ನಾಗಿ ಅವನು ಅರಿತಿರುವನು.

\begin{verse}
ರಾಗೀ ಕರ್ಮಫಲಪ್ರೇಪ್ಸುರ್ಲುಬ್ಧೋ ಹಿಂಸಾತ್ಮಕೋಽಶುಚಿಃ~।\\ಹರ್ಷಶೋಕಾನ್ವಿತಃ ಕರ್ತಾ ರಾಜಸಃ ಪರಿಕೀರ್ತಿತಃ \versenum{॥ ೨೭~॥}
\end{verse}

{\small ರಾಗವುಳ್ಳವನೂ, ಕರ್ಮ ಫಲಾಸಕ್ತನೂ, ಲೋಭಿಯೂ, ಹಿಂಸಾ ಸ್ವಭಾವವುಳ್ಳವನೂ, ಅಶುಚಿಯೂ, ಹರ್ಷ ಶೋಕಗಳಿಂದ ಕೂಡಿದವನೂ ಆದ ಕರ್ತೃವು ರಾಜಸಿಕ ಎಂದು ಹೇಳುತ್ತಾರೆ.}

ರಾಜಸಿಕ ಕರ್ತೃ ರಾಗದಿಂದ ಕೂಡಿದವನು. ಅವನಿಗೆ ವಸ್ತುವಿನ ಮೇಲೆ ಆಸಕ್ತಿ ಇದೆ. ಅದೊಂದೇ ಅಲ್ಲ, ಅವನ ಮನಸ್ಸಿನಲ್ಲಿ ಹಲವಾರು ಆಸೆ ಆಕಾಂಕ್ಷೆಗಳಿವೆ. ಯಾವಾಗಲೂ ಇವುಗಳಿಂದ ಕುದಿಯು ತ್ತಿರುವನು. ಯಾವ ಕೆಲಸವನ್ನು ಮಾಡಿದರೂ ಫಲಾಪೇಕ್ಷೆಯಿಂದಲೇ ಮಾಡುವನು. ಫಲಕ್ಕೆ ಕಟ್ಟು ಬೀಳುವನು. ಯಾವಾಗ ಸಿಹಿಯಾದ ಫಲ ಬರುವುದೊ ಆಗ ಚಪ್ಪರಿಸುವನು. ಕಹಿಯಾದ ಫಲ ಬಂದರೆ ಗೊಣಗಾಡುವನು. ಫಲಾಪೇಕ್ಷಿಗಳ ಸ್ವಭಾವವೇ ಹೀಗೆ. ಆ ಫಲ ನಮ್ಮನ್ನು ಕುಣಿಸುವುದು. ನಾಯಿಗೆ ಏನಾದರೂ ತಿಂಡಿಯನ್ನು ಮೇಲಕ್ಕೆ ಹಿಡಿದರೆ, ಅದನ್ನು ತಿನ್ನಲು ನೆಗೆಯುವುದು. ಅದನ್ನು ಎತ್ತ ಹಿಡಿದರೆ ಅತ್ತ ಹಿಂದೆ ಓಡುವುದು. ನಾವು ಹಾಗೆಯೇ ಫಲಕ್ಕೆ ದಾಸರಾಗಿ ಅದು ಹೇಳಿದಂತೆ ಕೇಳುತ್ತೇವೆ.

ಅವನು ಲೋಭಿ. ಯಾರಿಗೂ ಏನನ್ನೂ ಕೊಡ. ಒಂದುಸಲ ಬಂದಿತು ಎಂದರೆ ಆಯಿತು, ಅವನ ಕಪಿಮುಷ್ಠಿಯಿಂದ ತಪ್ಪಿಸಿಕೊಂಡು ಹೋಗುವ ಹಾಗಿಲ್ಲ. ಕೂಡಿಹಾಕುವುದೊಂದೇ ಅವನಿಗೆ ಗೊತ್ತಿರು ವುದು. ಕಳೆದುಕೊಳ್ಳುವ ಪ್ರಸಂಗ ಬಂದರೆ ಅವನಿಗೆ ಪ್ರಾಣ ಹೋದಂತೆ ಆಗುವುದು. ಬರೀ ತನ್ನಲ್ಲಿರುವುದನ್ನು ಇತರರಿಗೆ ಕೊಡದೆ ಇರುವುದು ಮಾತ್ರವಲ್ಲ, ಇತರರಲ್ಲಿ ಇರುವುದರ ಮೇಲೆ ಕಣ್ಣನ್ನು ಹಾಕಿರುವನು. ಎಂದಾದರೂ ಹೇಗಾದರೂ ಅದನ್ನು ದೋಚಲು ಹವಣಿಸುತ್ತಿರುವನು.

ಅವನು ತನ್ನ ಇಚ್ಛೆಯನ್ನು ನೆರವೇರಿಸಿಕೊಳ್ಳುವುದಕ್ಕೆ ಯಾರಿಗಾದರೂ ಹಿಂಸೆಯನ್ನು ಕೊಡಲು ಸಿದ್ಧನಾಗಿರುವನು. ಅವನ ಸ್ನೇಹಿತನಾಗಿರಬಹುದು, ಬಂಧು ಬಳಗದವರಾಗಿರಬಹುದು, ತಂದೆ ತಾಯಿಗಳಾಗಿರಬಹುದು. ಯಾರನ್ನೂ ಗಮನಿಸುವುದಿಲ್ಲ. ತನ್ನ ಸ್ವಾರ್ಥ ಈಡೇರಬೇಕು. ಅದಕ್ಕೆ ಯಾರು ಅಡ್ಡಿಯಾಗಿರುವರೋ ಅವರನ್ನು ನಿವಾರಿಸಿಕೊಳ್ಳುವನು. ಬಲಾತ್ಕಾರದಿಂದಲೋ ನಿಂದೆ ಯಿಂದಲೋ ಅವರನ್ನು ನೋಯಿಸಿ ತನ್ನ ಉದ್ದೇಶವನ್ನು ಸಾಧಿಸಿಕೊಳ್ಳುವನು. ಇನ್ನೊಬ್ಬನ ಗೋಳಿನ ಮೇಲೆ ನಮ್ಮ ಸುಖವನ್ನು ಕಟ್ಟಿಕೊಂಡರೆ ಅದು ಬಹಳಕಾಲ ಹಾಗಿರುವುದಿಲ್ಲ. ಅದು ಬಿರುಕುಬಿಟ್ಟು, ಕುಸಿದು ಹೋಗುವುದು ಎಂಬುದು ಆ ಅಲ್ಪಮತಿಗೆ ಇನ್ನೂ ತಿಳಿಯದು.

ಅವನ ಮಾರ್ಗ ಅಶುಚಿಯಾದುದು. ಒಳ್ಳೆಯ ಉದ್ದೇಶ ಸಾಧನೆಗೂ ಅವನು ಒಳ್ಳೆಯ ಮಾರ್ಗವನ್ನು ಆರಿಸಿಕೊಳ್ಳುವುದಿಲ್ಲ. ಸುಳ್ಳು ಲಂಚ ವಂಚನೆಗಳನ್ನೆಲ್ಲ ಉಪಯೋಗಿಸುವನು. ಹರ್ಷ ಶೋಕಗಳಿಂದ ತಾಡಿತನಾಗುತ್ತಿರುವನು. ಏನಾದರೂ ಚೆನ್ನಾಗಿರುವುದು ಬಂದರೆ ನನ್ನ ಸಮಾನ ಇಲ್ಲ ಎಂದು ಕುಣಿದಾಡುವನು. ಆತಂಕಗಳು ಬಂದಾಗ ಕುಗ್ಗಿ ಹೋಗುವನು. ಅಲೆಯ ಮೇಲೆ ಇರುವ ಹುಲ್ಲಿನೆಸಳಿನಂತೆ ಅವನ ಮನಸ್ಸು. ಅಲೆ ಮೇಲಕ್ಕೆ ಎದ್ದಾಗ ಹುಲ್ಲಿನೆಸಳು ಮೇಲಕ್ಕೇಳುವುದು. ಅದು ಕೆಳಕ್ಕೆ ಬಿದ್ದಾಗ ಮನಸ್ಸು ಕೆಳಕ್ಕೆ ಬರುವುದು.

\begin{verse}
ಅಯುಕ್ತಃ ಪ್ರಾಕೃತಃ ಸ್ತಬ್ಧಃ ಶಠೋ ನೈಷ್ಕೃತಿಕೋಽಲಸಃ~।\\ವಿಷಾದೀ ದೀರ್ಘಸೂತ್ರೀ ಚ ಕರ್ತಾ ತಾಮಸ ಉಚ್ಯತೇ \versenum{॥ ೨೮~॥}
\end{verse}

{\small ಅಯುಕ್ತ, ಅಸಂಸ್ಕಾರಿ, ವಂಚನಪರ, ಜಿದ್ದುಳ್ಳವನು, ಸೋಮಾರಿ, ವಿಷಾದಪರ ಮತ್ತು ದೀರ್ಘಸೂತ್ರಿ ಆದ ಕರ್ತೃ ತಾಮಸಿಕ ಎಂದು ಹೇಳುತ್ತಾರೆ.}

ತಾಮಸಿಕ ಕರ್ತೃ ಅಯುಕ್ತ. ತನ್ನ ಇಂದ್ರಿಯಗಳನ್ನೆಲ್ಲಾ ನಿಗ್ರಹಿಸಿಲ್ಲ. ಒಳ್ಳೆಯ ಕೆಲಸಗಳನ್ನು ಮಾಡುವುದಕ್ಕೆ ಯೋಗ್ಯನಲ್ಲ. ಏನನ್ನು ಮಾಡಬೇಕೊ ಅದನ್ನೇ ಮನಸ್ಸಿನಲ್ಲಿ ಸ್ಪಷ್ಟವಾಗಿ ಕಲ್ಪಿಸಿ ಕೊಂಡಿಲ್ಲ. ಅದರ ಸಾಧಕಬಾಧಕಗಳನ್ನು ಕುರಿತು ಯೋಚಿಸಿಲ್ಲ.

ಅಸಂಸ್ಕಾರಿಯಲ್ಲಿ ಉತ್ತಮ ಮಾತಾಗಲೀ ನಡತೆಯಾಗಲೀ ಇಲ್ಲ. ಅನಾಗರಿಕನ ಗುಂಪಿಗೆ ಸೇರಿದವನು. ಮನಸ್ಸಿಗೆ ಬಂದದ್ದು ಮಾತಾಡುವನು. ನಯ ನಾಜೂಕಿಲ್ಲ ಯಾವುದರಲ್ಲಿಯೂ. ಸ್ತಬ್ಧ, ಯಾರನ್ನೂ ಬಾಗಿ ಗೌರವಿಸುವುದಿಲ್ಲ. ಇನ್ನೊಬ್ಬನಿಗೆ ಸಮಯ ಸಿಕ್ಕಿದಾಗ ಮೋಸ ಮಾಡುವುದಕ್ಕೆ ಸಿದ್ಧನಾಗಿರುವನು. ಅವನ ಮಾತಿಗೂ ನಡವಳಿಕೆಗೂ ಸಂಬಂಧವಿಲ್ಲ. ಒಳಗೆ ಒಂದು ಹೊರಗೆ ಒಂದು. ಜಿದ್ದುಳ್ಳವನು. ಯಾರಾದರೂ ಅವನಿಗೆ ಅಹಿತವನ್ನು ಮಾಡಿದ್ದರೆ, ಅದನ್ನು ಎಂದಿಗೂ ಮರೆಯುವುದಿಲ್ಲ. ಸಮಯ ಸಿಕ್ಕಿದಾಗ ತನ್ನ ಸೇಡನ್ನು ತೀರಿಸಿಕೊಳ್ಳುವನು. ಕೆಲಸವನ್ನು ಸಾಧ್ಯ ವಾದಷ್ಟು ಮುಂದೂಡುವನು. ಒಂದು ವೇಳೆ ಕೆಲಸಕ್ಕೆ ಪ್ರಾರಂಭಿಸಿದರೂ ಅದು ಎಂದು ಮುಗಿಯು ವುದು ಎಂದು ಹೇಳುವ ಹಾಗಿಲ್ಲ. ವಿಷಾದಿಸಿ ಅಯ್ಯೋ ಯಾತಕ್ಕಾದರೂ ಪ್ರಾರಂಭಿಸಿದೆನೋ, ಮಾಡದೆ ಇದ್ದರೆ ಚೆನ್ನಾಗಿತ್ತು, ಇದು ಹಾಗೆ ಆಗಬಾರದಾಗಿತ್ತು, ಹೀಗೆ ಆಗಬಾರದಾಗಿತ್ತು ಎಂದು ಯಾವಾಗಲೂ ಗೊಣಗುತ್ತಿರುವನು. ದೀರ್ಘಸೂತ್ರಿ ಎಂದರೆ ಮಂದಸ್ವಭಾವವುಳ್ಳವನು. ಕಷ್ಟಪಟ್ಟು ಕೆಲಸ ಮಾಡುವುದಿಲ್ಲ. ಒಂದು ನಿರ್ಣಯಕ್ಕೆ ಬರಬೇಕಾದರೆ ಬಹಳ ಕಾಲ ಹಿಡಿಯುವುದು.

\begin{verse}
ಬುದ್ಧೇರ್ಭೇದಂ ಧೃತೇಶ್ಚೈವ ಗುಣತಸ್ತ್ರಿವಿಧಂ ಶೃಣು~।\\ಪ್ರೋಚ್ಯಮಾನಮಶೇಷೇಣ ಪೃಥಕ್ತ್ವೇನ ಧನಂಜಯ \versenum{॥ ೨೯~॥}
\end{verse}

{\small ಅರ್ಜುನ, ಬುದ್ಧಿ ಮತ್ತು ಧೈರ್ಯವು ಗುಣಾನುಸಾರ ಮೂರು ಬಗೆಯಾಗಿವೆ. ಅದನ್ನು ಸಂಪೂರ್ಣವಾಗಿ ಬೇರೆ ಬೇರೆ ಹೇಳುತ್ತೇನೆ ಕೇಳು.}

ಇಲ್ಲಿ ಮೂರು ಬಗೆಯಾಗಿದೆ ಎಂದರೆ ಸಾತ್ತ್ವಿಕ, ರಾಜಸಿಕ ಮತ್ತು ತಾಮಸಿಕವಾಗಿ ಮೂರು ಭಾಗವಾಗಿದೆ.

\begin{verse}
ಪ್ರವೃತ್ತಿಂ ಚ ನಿವೃತ್ತಿಂ ಚ ಕಾರ್ಯಾಕಾರ್ಯೇ ಭಯಾಭಯೇ~।\\ಬಂಧಂ ಮೋಕ್ಷಂ ಚ ಯಾ ವೇತ್ತಿ ಬುದ್ಧಿಃ ಸಾ ಪಾರ್ಥ ಸಾತ್ತ್ವಿಕೀ \versenum{॥ ೩೦~॥}
\end{verse}

{\small ಅರ್ಜುನ, ಪ್ರವೃತ್ತಿ ನಿವೃತ್ತಿಗಳನ್ನು ಕಾರ್ಯ ಅಕಾರ್ಯಗಳನ್ನು ಭಯಾಭಯಗಳನ್ನು ಬಂಧಮೋಕ್ಷಗಳನ್ನು ಯಾವ ಬುದ್ಧಿ ತಿಳಿದುಕೊಳ್ಳುವುದೊ ಅದು ಸಾತ್ತ್ವಿಕ.}

ಸಾತ್ತ್ವಿಕಬುದ್ಧಿ ಮುಂಚೆಯೇ ಎಲ್ಲವನ್ನೂ ಗ್ರಹಿಸುವುದು. ಒಂದು ಕೆಲಸವನ್ನು ಆರಂಭ ಮಾಡಿದ ಮೇಲೆ ಅದನ್ನು ವಿಚಾರಿಸುವುದಕ್ಕೆ ಪ್ರಾರಂಭಿಸುವುದಿಲ್ಲ. ಪ್ರವೃತ್ತಿ ಎಂದರೆ ಯಾವುದು ನಮ್ಮನ್ನು ಸಂಸಾರಕ್ಕೆ ಕಟ್ಟಿಹಾಕುವುದು, ಯಾವುದು ನಮ್ಮನ್ನು ಪಂಚೇಂದ್ರಿಯಗಳ ಕಡೆ ಸೆಳೆಯುವುದು ಎಂಬುದು. ನಿವೃತ್ತಿ ಎಂದರೆ ಯಾವುದನ್ನು ಬಿಡಬೇಕೆಂದು ಮುಂಚೆಯೇ ಅರಿತುಕೊಂಡು ಬಿಟ್ಟಿರುವುದು. ಪ್ರವೃತ್ತಿಯ ಕೆಲಸವನ್ನು ಮಾಡುವುದಿಲ್ಲ, ನಿವೃತ್ತಿಯ ಕೆಲಸವನ್ನು ಮಾಡುತ್ತಾನೆ ಸಾತ್ತ್ವಿಕ. ಒಂದು ನಮ್ಮನ್ನು ಸಂಸಾರದ ಗೋಜಿನಲ್ಲಿ ಸಿಕ್ಕಿಸುವುದು. ಇನ್ನೊಂದು ಸಂಸಾರದಿಂದ ಪಾರು ಮಾಡುವುದು.

ಸಾತ್ತ್ವಿಕಬುದ್ಧಿಗೆ ಯಾವುದು ಕಾರ್ಯ, ಯಾವುದು ಅಕಾರ್ಯವೆಂಬುದು ಗೊತ್ತಿದೆ. ಯಾವ ಕಾರ್ಯವನ್ನು ಮಾಡಬೇಕೆಂಬುದನ್ನು ನಿರ್ಧರಿಸುವುದೇ ಕಾರ್ಯ. ಒಬ್ಬ ಒಂದು ವರ್ಣದಲ್ಲಿ ಒಂದು ಕೆಲಸವನ್ನು ಇದುವರೆಗೆ ಮಾಡುತ್ತಿರುವನು. ಇದು ನಿಯತ ಕಾರ್ಯ. ಇದನ್ನು ಮಾಡಬೇಕಾಗಿದೆ. ಸಮಾಜ ಇದರ ಆಧಾರದ ಮೇಲೆ ನಿಂತಿದೆ. ಅಕಾರ್ಯ ಎಂದರೆ ನಾನು ಒಂದು ಕೆಲಸಕ್ಕೆ ತರಬೇತು ಹೊಂದಿಲ್ಲ. ಆದರೂ ಮಾಡುತ್ತೇನೆ. ಯಾವ ಕಾಲದಲ್ಲಿ ಅವನು ಅದನ್ನು ಮಾಡಬೇಕೋ ಅದನ್ನು ಬಿಟ್ಟು ಬೇರೊಂದನ್ನು ಮಾಡುವುದು. ಇಲ್ಲಿ ಅರ್ಜುನ ಯುದ್ಧವನ್ನು ಮಾಡಬೇಕಾಗಿದೆ. ಇದು ಅವನ ನಿಯತ ಕಾರ್ಯ. ಅವನು ಕ್ಷತ್ರಿಯ. ಇತರರು ಅವನಿಗೆ ಮೋಸ ಮಾಡಿರುವರು. ಅದಕ್ಕಾಗಿ ಧರ್ಮಯುದ್ಧ ವನ್ನು ಮಾಡಬೇಕು. ಅದನ್ನು ಮಾಡದೆ ಬಿಡುವುದು ಅಕಾರ್ಯ. ಇಲ್ಲಿ ಯಾವಾಗ ಅವನು ಯುದ್ಧವನ್ನು ಮಾಡುವುದಿಲ್ಲವೋ ಅದರಿಂದ ಅವನು ಅಧರ್ಮಕ್ಕೆ ಪ್ರೋತ್ಸಾಹ ಕೊಟ್ಟಂತೆ. ಪಾಂಡವರಿಗೆ ರಾಜ್ಯ ಬರದೆ ಹೋದರೆ ಚಿಂತೆಯಿಲ್ಲ. ಅದು ಗೌಣ. ಆದರೆ ದೇಶದಲ್ಲಿ ಅಧರ್ಮ, ಅನ್ಯಾಯ ಗೆಲ್ಲುವುದಕ್ಕೆ ಸಹಾಯ ಮಾಡುವುದು ಮಹಾಪಾಪ, ಅಕಾರ್ಯ, ಮಾಡಬಾರದ ಕೆಲಸ. ಅರ್ಜುನ ತಿಳಿಯದೆ ಆ ಕೆಲಸವನ್ನು ಮಾಡುವುದಕ್ಕೆ ಉದ್ಯುಕ್ತನಾಗಿರುವನು.

ಸಾತ್ತ್ವಿಕನಿಗೆ, ಯಾವುದಕ್ಕೆ ಭಯಪಡಬೇಕು, ಯಾವುದಕ್ಕೆ ಭಯಪಡಬಾರದು ಎಂಬುದು ಗೊತ್ತಿದೆ. ಆಸಕ್ತಿಯಿಂದ ಕೆಲಸ ಮಾಡಿದರೆ ಅನಂತರ ನಾವು ಅದಕ್ಕೆ ನಿತ್ಯ ಒದ್ದಾಡಬೇಕಾಗುವುದು. ಅನ್ಯಾಯ, ಸುಳ್ಳು, ಅಧರ್ಮ ಇವುಗಳಿಗೆ ಅವನು ಅಂಜುವನು. ಪ್ರಾರಂಭದಲ್ಲಿ ಇವು ಹಿತಕರವಾಗಿರು ವುವು. ಅನಂತರ ಅದರಿಂದ ಬರುವ ಪರಿಣಾಮ ಭಯಾನಕವಾಗಿರುವುದು. ಇದನ್ನು ಮೊದಲೇ ಅರಿತಿರುವನು. ಅಂತಯೇ ಯಾವ ಕಾಲದಲ್ಲಿ ನಿರ್ಭಯನಾಗಿರಬೇಕು ಎಂಬುದನ್ನು ಅರಿತಿರುವನು. ಒಂದು ಕೆಲಸವನ್ನು ಸರಿಯಾಗಿ ವಿಮರ್ಶಿಸಿ ಅದು ಒಳ್ಳೆಯದು, ಅದನ್ನು ಮಾಡಬೇಕಾಗಿದೆ ಎಂದು ನಿರ್ಣಯಕ್ಕೆ ಬಂದು, ಅದರಲ್ಲಿ ನಿರತನಾಗಿರುವಾಗ ಹಲವಾರು ಅಡಚಣೆಗಳು ಬರಬಹುದು. ಆದರೆ ನಿರ್ಭಯನು ಇವುಗಳಾವುದಕ್ಕೂ ಅಂಜುವುದಿಲ್ಲ. ಇವುಗಳೊಂದಿಗೆ ಛಲದಿಂದ ಹೋರಾಡುವನು. ಆತಂಕಗಳು ಹೆಚ್ಚಾದಂತೆ ಅವುಗಳನ್ನು ಎದುರಿಸುವ ಧೈರ್ಯವೂ ಹೆಚ್ಚುತ್ತಾ ಹೋಗುವುದು. ಜೀವನ ದಲ್ಲಿ ಇವನನ್ನು ತಡೆಯಲು ಬರುವ ಯಾವ ವ್ಯಕ್ತಿಗೂ, ಬಾಹ್ಯಘಟನೆಗೂ ಇವನು ಶರಣಾಗುವವ ನಲ್ಲ.

ಯಾವುದು ಬಂಧನಕಾರಿ, ಯಾವುದು ಮೋಕ್ಷಕಾರಿ ಎಂಬುದನ್ನು ಈತನು ಅರಿತಿರುವನು. ಅನೇಕ ವೇಳೆ ನಮಗೆ ಯಾವುದು ಬಂಧನಕಾರಿ ಎಂಬುದು ತಿಳಿಯುವುದಿಲ್ಲ. ಅಷ್ಟು ನಿಧಾನವಾಗಿ ಜಾರುವೆವು. ನಾವು ಜಾರುತ್ತಿರುವ ಅರಿವೇ ನಮಗುಂಟಾಗುವುದಿಲ್ಲ. ಸ್ವಲ್ಪ ಕಾಲದ ಮೇಲೆಯೆ ನಮಗೆ ಗೊತ್ತಾಗು ವುದು, ನಾವೆಷ್ಟು ಕೆಳಗೆ ಬಿದ್ದಿರುವೆವು ಎಂಬುದು. ನಮ್ಮನ್ನು ಬಂಧನಕ್ಕೆ ಬೀಳಿಸುವುದೆಲ್ಲ ನಮ್ಮನ್ನು ಉದ್ಧರಿಸುವುದಕ್ಕೆ ಬಂದಂತೆ ಕಾಣುವುದು. ವೇಷ ಬದಲಾಯಿಸಿಕೊಂಡು ಬರುವುದು. ಆದರೆ ಸಾತ್ತ್ವಿಕಬುದ್ಧಿ ತಕ್ಷಣವೇ ನಿರ್ಧರಿಸಬಲ್ಲುದು, ಇಲ್ಲಿ ಬಿಡುಗಡೆಯಲ್ಲ ಇರುವುದು, ಬಂಧನ ಎಂಬುದು. ಕೆಲವರು ಆನೆಯನ್ನು ಹಿಡಿಯುವುದಕ್ಕೆ ಕಾಡಿನಲ್ಲಿ ಈ ಉಪಾಯವನ್ನು ಮಾಡುವರು. ಹಳ್ಳವನ್ನು ತೋಡಿ ಅದರ ಮೇಲೆ ತೆಳ್ಳಗೆ ಇರುವ ಬಿದರನ್ನು ಹರಡಿ ಮೇಲೆ ಮಣ್ಣನ್ನು ಹಾಕಿ ಅದರ ಮೇಲೆ ಹುಲ್ಲು ಬೆಳೆಸುವರು. ಆ ಹುಲ್ಲು ಇತರ ಕಡೆಗಳಿಗಿಂತ ಚೆನ್ನಾಗಿ ಬೆಳೆದಿರುವುದು. ಆನೆ ಹುಲ್ಲನ್ನು ತಿನ್ನಲು ಬಂದು ಹಳ್ಳಕ್ಕೆ ಬೀಳುವುದು. ಆದರೆ ಸಾತ್ತ್ವಿಕಬುದ್ಧಿಯವನು, ಏನೋ ಇಲ್ಲಿರಬೇಕು, ಅದರ ಹತ್ತಿರ ಹೋಗಬಾರದು ಎಂದು ಸುತ್ತುಮುತ್ತಲನ್ನು ಪರೀಕ್ಷಿಸಿ ತಿಳಿದುಕೊಳ್ಳು ವನು.

ಬರೀ ಯಾವುದು ಬಂಧನವೋ ಅದನ್ನು ತಿಳಿದುಕೊಂಡರೆ ಸಾಲದು. ಯಾವುದು ನಮಗೆ ಮುಕ್ತಿಯನ್ನು ಕೊಡುವುದು, ಅದಕ್ಕೆ ನಾವು ಏನೇನನ್ನು ಮಾಡಬೇಕು ಅದನ್ನೆಲ್ಲಾ ಅವನು ತಿಳಿದಿರು ವನು. ಅದಕ್ಕೆ ಶಾಸ್ತ್ರದ ಸಹಾಯ ಬೇಕು, ಗುರುವಿನ ಸಹಾಯ ಬೇಕು, ಸ್ವಪ್ರಯತ್ನ ಮಾಡಬೇಕು. ಆಗಲೇ ಮುಕ್ತಿ ಸಿಗಬಲ್ಲದು. ಇವುಗಳನ್ನೆಲ್ಲಾ ಅವನ ಬುದ್ಧಿ ಚೆನ್ನಾಗಿ ತಿಳಿದಿರುವುದು.

\begin{verse}
ಯಯಾ ಧರ್ಮಮಧರ್ಮಂ ಚ ಕಾರ್ಯಂ ಚಾಕಾರ್ಯಮೇವ ಚ~।\\ಅಯಥಾವತ್ ಪ್ರಜಾನಾತಿ ಬುದ್ಧಿಃ ಸಾ ಪಾರ್ಥ ರಾಜಸೀ \versenum{॥ ೩೧~॥}
\end{verse}

{\small ಅರ್ಜುನ, ಧರ್ಮ ಅಧರ್ಮಗಳನ್ನು, ಕಾರ್ಯ ಅಕಾರ್ಯಗಳನ್ನು ಯಾವುದರಿಂದ ಯಥಾರ್ಥವಾಗಿ ತಿಳಿಯುವು ದಿಲ್ಲವೊ, ಆ ಬುದ್ಧಿ ರಾಜಸಿಕ.}

ರಾಜಸಿಕ ಬುದ್ಧಿಗೆ ಧರ್ಮ ಯಾವುದು, ಅಧರ್ಮ ಯಾವುದು ಎಂಬುದು ಗೊತ್ತಾಗುವುದಿಲ್ಲ. ಯಾವುದು ಒಳ್ಳೆಯದು ಅದೂ ಗೊತ್ತಾಗುವುದಿಲ್ಲ, ಕೆಟ್ಟದ್ದೂ ತಿಳಿಯುವುದಿಲ್ಲ. ಒಂದನ್ನು ಮತ್ತೊಂದು ಎಂದು ಭಾವಿಸುತ್ತಾನೆ. ಅನೇಕ ವೇಳೆ ಪ್ರೇಯಸ್ಸನ್ನು ಧರ್ಮ ಎಂದು ಭಾವಿಸುತ್ತಾನೆ. ಏಕೆಂದರೆ ಅದು ಮೊದಲು ಸಿಹಿಯಾಗಿ ಕಾಣುವುದು. ಶ್ರೇಯಸ್ಸನ್ನು ಅಧರ್ಮ ಎಂದು ಭಾವಿಸುತ್ತಾನೆ. ಏಕೆಂದರೆ ಅದು ಮಾಡಲು ಕಷ್ಟವಾಗಿರುವುದು. ಒಬ್ಬನ ಬುದ್ಧಿ ತಿಳಿಯಾಗಿದ್ದರೆ ಮಾತ್ರ ಇದನ್ನು ತಿಳಿದುಕೊಳ್ಳಬಹುದು. ಕದಡಿ ಹೋದ ಬುದ್ಧಿಯಲ್ಲಿ ನಾವು ಇದನ್ನು ತಿಳಿದುಕೊಳ್ಳುವುದಕ್ಕೆ ಆಗುವು ದಿಲ್ಲ. ಅದು ನಮಗೆ ವಸ್ತುವಿನ ಯಥಾರ್ಥ ಸ್ಥಿತಿಯನ್ನು ತಿಳಿಸಲಾರದು.

ಅದರಂತೆಯೇ ತನ್ನ ಕರ್ತವ್ಯ ಯಾವುದು, ಯಾವುದನ್ನು ಮಾಡಬೇಕು ಎಂಬುದು ಗೊತ್ತಿರುವು ದಿಲ್ಲ. ಯಾವುದು ತನ್ನ ಕರ್ತವ್ಯವಲ್ಲ, ಯಾವುದನ್ನು ತಾನು ಮಾಡಬಾರದು ಎಂಬುದೂ ಗೊತ್ತಿಲ್ಲ. ಅರ್ಜುನ ಕ್ಷತ್ರಿಯನಾಗಿದ್ದಾನೆ. ಅವನ ಕರ್ತವ್ಯ ಧರ್ಮಯುದ್ಧವನ್ನು ಮಾಡುವುದು. ಅದನ್ನು ಮಾಡದೆ, ಅನ್ಯಾಯ ಅಧರ್ಮಗಳನ್ನು ಎದುರಿಸದೆ, ಜಯದಿಂದೇನು ಲಾಭದಿಂದೇನು ಎಂದು ಭಾವಿಸಿಕೊಂಡು ಸುಮ್ಮನೆ ಇರುವುದು ಅಕಾರ್ಯ.

\begin{verse}
ಅಧರ್ಮಂ ಧರ್ಮಮಿತಿ ಯಾ ಮನ್ಯತೇ ತಮಸಾವೃತಾ~।\\ಸರ್ವಾರ್ಥಾನ್ ವಿಪರೀತಾಂಶ್ಚ ಬುದ್ಧಿಃ ಸಾ ಪಾರ್ಥ ತಾಮಸೀ \versenum{॥ ೩೨~॥}
\end{verse}

{\small ಅರ್ಜುನ, ಯಾವುದು ತಮಸ್ಸಿನಿಂದ ಆವೃತವಾಗಿ, ಅಧರ್ಮವನ್ನು ಧರ್ಮವೆಂದು ತಿಳಿಯುವುದೊ, ಸರ್ವ ಅರ್ಥಗಳನ್ನು ವಿಪರೀತವಾಗಿ ತಿಳಿಯುವುದೊ ಆ ಬುದ್ಧಿ ತಾಮಸಿಕ.}

ತಾಮಸಿಯ ಒಂದು ಸ್ವಭಾವವೇ ತಪ್ಪನ್ನು ಸರಿ ಎಂದು ಭಾವಿಸುವುದು. ಅವನದು ಯಾವಾಗಲೂ ಕೊಂಕು ದೃಷ್ಟಿ. ಯಾವಾಗಲೂ ಮಾಡಬಾರದುದನ್ನೇ ಸರಿ ಎಂದು ಭಾವಿಸುತ್ತಾನೆ. ಅದನ್ನೇ ಸಾಧಿಸುತ್ತಾನೆ, ಅದನ್ನೇ ಮಾಡುತ್ತಾನೆ. ಯಾವುದನ್ನು ಮಾಡಬೇಕೊ ಅದನ್ನು ಬಿಡುತ್ತಾನೆ. ರಾಜಸಿಕ ನಿಗೆ ಇದನ್ನು ತಿಳಿಸಿದರೆ ಅವನು ಮಾಡುತ್ತಿರುವುದು ತಪ್ಪು ಎಂದು ತಿಳಿದುಕೊಳ್ಳುವ ಸ್ಥಿತಿಯಲ್ಲಿರು ತ್ತಾನೆ. ಆದರೆ ತಾಮಸಿಕ ಹಾಗಲ್ಲ. ಯಾರು ಹೇಳಿದರೂ ಅವನಿಗೆ ಗೊತ್ತಾಗುವ ಹಾಗಿಲ್ಲ. ಅವನು ತಿಳಿದುಕೊಳ್ಳುವುದಕ್ಕೆ ಸಿದ್ಧನಾಗಿಲ್ಲ. ಅಂತಹವನಿಗೆ ತಿಳಿಸುವುದು ಕಷ್ಟ.

ಅವನು ಸರ್ವ ಅರ್ಥಗಳನ್ನೂ ವಿಪರೀತ ದೃಷ್ಟಿಯಲ್ಲಿ ನೋಡುತ್ತಾನೆ. ಯಾವುದನ್ನೂ ಹೇಗೆ ಇದೆಯೋ ಹಾಗೆ ಸ್ವೀಕರಿಸುವುದಿಲ್ಲ. ಅದರ ಹಿಂದೆಲ್ಲ ದುರುದ್ದೇಶಗಳನ್ನು ಕಲ್ಪನೆ ಮಾಡುತ್ತಾನೆ. ಅವನ ಮನಸ್ಸು ಗಲೀಜಾಗಿದೆ. ಆದಕಾರಣ ಅವನು ನೋಡುವ ದೃಷ್ಚಿಯೆಲ್ಲ ಅದರಿಂದಲೇ ತುಂಬಿದೆ. ಬಣ್ಣದ ಕನ್ನಡಕದ ಮೂಲಕ ನೋಡಿದರೆ, ಎಲ್ಲದಕ್ಕೂ ಬಣ್ಣ ಹಾಕಿದಂತೆ ಕಾಣುವುದು. ಹಾಗೆಯೇ ಇವನು ಇರುವುದನ್ನು ಬಿಡುವನು, ಇಲ್ಲದೆ ಇರುವುದನ್ನು ಆರೋಪಮಾಡುವನು.

\begin{verse}
ಧೃತ್ಯಾ ಯಯಾ ಧಾರಯತೇ ಮನಃಪ್ರಾಣೇಂದ್ರಿಯಕ್ರಿಯಾಃ~।\\ಯೋಗೇನಾವ್ಯಭಿಚಾರಿಣ್ಯಾ ಧೃತಿಃ ಸಾ ಪಾರ್ಥ ಸಾತ್ತ್ವಿಕೀ \versenum{॥ ೩೩~॥}
\end{verse}

{\small ಅರ್ಜುನ, ಯೋಗದ ಮೂಲಕ ಅವ್ಯಭಿಚಾರವಾದ ಧೃತಿಯಿಂದ, ಮನಸ್ಸು ಪ್ರಾಣ ಇಂದ್ರಿಯ ಇವುಗಳ ಕ್ರಿಯೆಯನ್ನು ನಿಗ್ರಹಿಸುವ ಧೃತಿ ಸಾತ್ತ್ವಿಕ.}

ಧೃತಿ ಎಂದರೆ ಧೈರ್ಯ. ಇದು ದೈಹಿಕವಾದುದಲ್ಲ. ಈ ಧೈರ್ಯ ಮಾನಸಿಕವಾದುದು. ಅವನ ಧೈರ್ಯ ಅವ್ಯಭಿಚಾರವಾದುದು. ಯೋಗ್ಯವಾದ ವಸ್ತುವಿನ ಕಡೆ ಮಾತ್ರ ಹೋಗುವುದು. ಅಯೋಗ್ಯವಾದೆಡೆ ಹೋಗುವುದಿಲ್ಲ. ಇವನು ಮನಸ್ಸಿನ ಕ್ರಿಯೆಯನ್ನು ನಿಗ್ರಹಿಸುತ್ತಾನೆ. ಮನಸ್ಸಿನ ಸ್ವಭಾವವೇ ಹಲ ವಾರು ವಸ್ತುಗಳನ್ನು ಕುರಿತು ಚಿಂತಿಸುವುದು. ಸಾತ್ತ್ವಿಕ ಮನಸ್ಸು ಒಳ್ಳೆಯ ಆಲೋಚನೆಗಳ ಕಲ್ಪನೆಗಳನ್ನು ಮಾತ್ರ ಮಾಡಲು ಅವಕಾಶ ಕೊಡುವುದು. ಹೀನ ವಿಷಯಗಳನ್ನು ಚಿಂತಿಸುವುದನ್ನು ತಡೆಯುವುದು. ಪ್ರಪಂಚದ ಕಡೆ ಹೋಗುವುದನ್ನು ತಡೆದು ಪರಮಾರ್ಥದ ಕಡೆ ಹೋಗುವಂತೆ ಮಾಡುವನು. 

ಪ್ರಾಣಕ್ರಿಯೆಯೆ ನಮ್ಮ ದೇಹದ ಕೆಲಸ ಕಾರ್ಯಗಳಿಗೆಲ್ಲಾ ಕಾರಣ. ಪ್ರಾಣ ಅಪಾನ ಸಮಾನ ಉದಾನ ವ್ಯಾನ ಇವುಗಳ ಸಹಾಯದಿಂದಲೇ ನಾವು ಉಸಿರಾಡುವುದು, ರಕ್ತ ಚಲನೆ ನಡೆಯುವುದು, ನಾವು ತಿಂದದ್ದನ್ನು ಅರಗಿಸಿಕೊಳ್ಳುವುದು, ದೇಹದ ಕಶ್ಮಲಗಳನ್ನೆಲ್ಲಾ ಹೊರಗೆ ದೂಡುವುದು. ಅವನು ಪ್ರಾಣದ ಪ್ರತಿಯೊಂದು ಕ್ರಿಯೆಯನ್ನೂ ತನ್ನ ವಶಕ್ಕೆ ತೆಗೆದುಕೊಂಡಿರುವನು. ಅವು ಇಚ್ಛಿಸಿದಂತೆ ಇವನು ಮಾಡುವುದಿಲ್ಲ. ಇವನು ಇಚ್ಛಿಸಿದಂತೆ ಅವು ಕೆಲಸ ಮಾಡಬೇಕು. ಇವನು ಇಂದ್ರಿಯಗಳನ್ನು ವಿಷಯ ವಸ್ತುಗಳ ಕಡೆ ಹೋಗದಂತೆ ತಡೆಗಟ್ಟುತ್ತಾನೆ. ಇದೊಂದು ದೊಡ್ಡ ಸಾಹಸ. ಕಾರನ್ನು ಎಳೆದು ನಿಲ್ಲಿಸುವುದಕ್ಕಿಂತ ಕಷ್ಟ ಇದು. ವಿಷಯ ವಸ್ತುಗಳ ಮಧ್ಯದಲ್ಲಿ ಇವನ ಇಂದ್ರಿಯಗಳು ಸಂಪೂರ್ಣವಾಗಿ ಇವನ ವಶವಾಗಿವೆ.

\begin{verse}
ಯಯಾ ಧರ್ಮಕಾಮಾರ್ಥಾನ್ ಧೃತ್ಯಾ ಧಾರಯತೇಽಜುRನ~।\\ಪ್ರಸಂಗೇನ ಫಲಾಕಾಂಕ್ಷೀ ಧೃತಿಃ ಸಾ ಪಾರ್ಥ ರಾಜಸೀ \versenum{॥ ೩೪~॥}
\end{verse}

{\small ಅರ್ಜುನ, ಫಲಾಕಾಂಕ್ಷಿಯು ಅತ್ಯಂತ ಸಂಗದ ಮೂಲಕ ಯಾವ ಧೃತಿಯಿಂದ ಧರ್ಮ ಅರ್ಥ ಕಾಮಗಳನ್ನು ಹಿಡಿಯುತ್ತಾನೆಯೋ ಆ ಧೃತಿ ರಾಜಸಿಕ.}

ರಾಜಸಿಕನಲ್ಲಿಯೂ ಧೈರ್ಯವಿದೆ. ಹಿಡಿದುದನ್ನು ಬಿಡುವುದಿಲ್ಲ. ಆದರೆ ಅದೆಲ್ಲ ಲೌಕಿಕ ವಸ್ತುಗಳ ಕಡೆ ತಿರುಗಿದೆ. ಅಲ್ಲಿ ಅವನು ತನ್ನ ಪೌರುಷವನ್ನು ತೋರಿಸುವನು. ಅವನು ಫಲಾಕಾಂಕ್ಷಿ. ಫಲಕ್ಕಾಗಿಯೇ ಕೆಲಸ ಮಾಡುವುದು. ಧರ್ಮಕ್ಕೆ ಸಂಬಂಧಪಟ್ಟ ಹಲವು ಕ್ರಿಯೆಗಳನ್ನು ಮಾಡುತ್ತಾನೆ. ಅದು ದೇವರಿಗೆ ಸಂಬಂಧಪಟ್ಟದ್ದು ಇರಬಹುದು, ಸಮಾಜಕ್ಕೆ ಸಂಬಂಧಪಟ್ಟದ್ದು ಇರಬಹುದು. ಇವುಗಳನ್ನೆಲ್ಲ ಮಾಡುವಾಗ ಯಾವ ಎಡರು ತೊಡರುಗಳು ಬಂದರೂ ಬಿಡದೆ ಮುಂದುವರಿಸುತ್ತಾನೆ. ಇವನು ಒಂದು ಯಾಗವನ್ನೋ ಯಜ್ಞವನ್ನೋ ಮಾಡಬಹುದು. ಆಸ್ಪತ್ರೆಯನ್ನೋ ಛತ್ರವನ್ನೋ ಕಟ್ಟಿರಬಹುದು. ಇದರ ಹಿಂದೆಲ್ಲ ಅವನಿಗೆ ಕೀರ್ತಿಯ ಆಸೆ ಇದೆ.

ಹಣವನ್ನು ಸಂಪಾದನೆ ಮಾಡಬೇಕೆಂಬ ಆಸೆ ಇದೆ. ಆ ಆಸೆಯನ್ನು ತೃಪ್ತಿಪಡಿಸಿಕೊಳ್ಳಬೇಕಾದರೆ ಸುಮ್ಮನೆ ಕುಳಿತುಕೊಂಡರೆ ಆಗುವುದಿಲ್ಲ. ಯಾವುದಾದರೂ ಉದ್ಯಮಕ್ಕೆ ಕೈಹಾಕಬೇಕು. ಅವುಗಳಲ್ಲಿ ಎಷ್ಟೋ ತೊಂದರೆಗಳು ಬರುವುವು. ಆದರೆ ಅವನು ಇವುಗಳಾವುದಕ್ಕೂ ಅಂಜುವವನಲ್ಲ. ಎಷ್ಟೇ ಕಷ್ಟವಾಗಲಿ, ನಷ್ಟವಾಗಲಿ, ಮುಂದೆ ಬರುವುದಕ್ಕೆ ಯತ್ನಿಸುತ್ತಾನೆ. ಇವನಲ್ಲಿ ಒಂದು ಸಾಹಸ ಇದೆ. ಅದನ್ನು ಹಣ ಸಂಪಾದನೆಗೆ ಉಪಯೋಗಿಸುತ್ತಿರುವನು.

ಅದರಂತೆಯೆ ಅವನಲ್ಲಿ ಹಲವು ಕಾಮನೆಗಳಿವೆ. ಅವುಗಳನ್ನು ಪಡೆಯಬೇಕಾಗಿದೆ. ಅವುಗಳ ಕೈಯಿಂದ ಅನೇಕ ವೇಳೆ ಚೇಳಿನಿಂದ ಕುಟುಕಿಸಿಕೊಂಡಂತೆ ಕುಟುಕಿಸಿಕೊಳ್ಳಬೇಕು. ಇವನು ಅವು ಗಳಾವುದಕ್ಕೂ ಜಗ್ಗುವುದಿಲ್ಲ. ಮುಳ್ಳಿನ ಗಿಡದಲ್ಲಿರುವ ಗುಲಾಬಿಯನ್ನು ಕೀಳಬೇಕಾದರೆ ಕೆಲವು ವೇಳೆ ಮುಳ್ಳು ತಾಗುವುದು. ಅದಕ್ಕೆ ಅಂಜಿದರೆ ಅವನಿಗೆ ಹೂವು ಸಿಗಲಾರದು. ಹಾಗೆಯೇ ರಜೋಧೈರ್ಯ ಇವುಗಳನ್ನೆಲ್ಲಾ ಸಹಿಸುವುದಕ್ಕೆ ಸಿದ್ಧವಾಗಿದೆ. ನಾವು ಕಾಮಿಸುವ ವಸ್ತುಗಳು ಸುಲಭವಾಗಿ ನಮ್ಮ ವಶವಾಗುವುದಿಲ್ಲ. ಅದನ್ನು ಹಿಡಿಯುವುದಕ್ಕೆ ಹೋದರೆ ಅದು ಒದೆಯುತ್ತದೆ, ಕಚ್ಚುತ್ತದೆ, ಬೊಗುಳು ತ್ತದೆ. ಯಾರಿಗೆ ಇವುಗಳನ್ನೆಲ್ಲಾ ಸಹಿಸುವ ಧೈರ್ಯವಿರುವುದೊ ಅವನಿಗೆ ಸಿಕ್ಕಬಲ್ಲದು.

\begin{verse}
ಯಯಾ ಸ್ವಪ್ನಂ ಭಯಂ ಶೋಕಂ ವಿಷಾದಂ ಮದಮೇವ ಚ~।\\ನ ವಿಮುಂಚತಿ ದುರ್ಮೇಧಾ ಧೃತಿಃ ಸಾ ಪಾರ್ಥ ತಾಮಸೀ \versenum{॥ ೩೫~॥}
\end{verse}

{\small ಅರ್ಜುನ, ದುರ್ಬುದ್ಧಿಯುಳ್ಳವನು ಯಾವ ಧೈರ್ಯದಿಂದ ನಿದ್ರೆ, ಭಯ, ಶೋಕ, ವಿಷಾದ, ಮದ ಇವುಗಳನ್ನು ಬಿಡುವುದಿಲ್ಲವೊ ಆ ಧೃತಿ ತಾಮಸಿಕ.}

ತಮೋಧೈರ್ಯದಲ್ಲಿ ದುರ್ಬುದ್ಧಿಯನ್ನು ನೋಡುತ್ತೇವೆ. ಅವನ ಬುದ್ಧಿ ನೇರವಾಗಿ ಯಾವಾಗಲೂ ಕೆಲಸ ಮಾಡುವುದಿಲ್ಲ. ಅದು ಯಾವಾಗಲೂ ಡೊಂಕು ಡೊಂಕಾಗೇ ಕೆಲಸ ಮಾಡುವುದು. ಅವನು ಕೆಲವು ವಿಷಯಗಳನ್ನು ಧೈರ್ಯದಿಂದ ಹಿಡಿದುಕೊಂಡಿರುತ್ತಾನೆ. ಅದರಿಂದ ಅವನಿಗೆ ಒಳ್ಳೆಯದೇನೂ ಆಗುವುದಿಲ್ಲ. ಆದರೂ ಅವನು ಆ ಸ್ವಭಾವವನ್ನು ಬಿಡುವುದಿಲ್ಲ.

ನಿದ್ರೆಯ ಮೇಲೆ ಅವನಿಗೆ ಅಂತಹ ಪ್ರೀತಿ, ಅಂತಹ ಆಸಕ್ತಿ. ಕಾಲಕ್ಕೆ ಸರಿಯಾಗಿ ಏಳುವ ಸ್ವಭಾವ ಅವನದಲ್ಲ. ಅದರಿಂದ ಆಫೀಸಿಗೆ ಹೊತ್ತಾಗಿ ಹೋಗಬಹುದು, ರೈಲು ಬಸ್ಸು ತಪ್ಪಬಹುದು. ಇನ್ನೂ ಏನೋನೋ ಅನಾಹುತಗಳು ಆಗಬಹುದು. ಆದರೂ ನಿದ್ರೆ ಅವನನ್ನು ಮೆಟ್ಟಿಕೊಂಡಿರುವುದು. ಬಡಪೆಟ್ಟಗೆ ಅವನು ಅದರಿಂದ ಪಾರಾಗಲಾರ. ಬೆಳಗ್ಗೆ ಮುಂಚೆ ಏಳಬೇಕೆಂದು ಗಡಿಯಾರಕ್ಕೆ ಅಲಾರಂ ಕೊಟ್ಟು ಮಲಗುತ್ತಾನೆ. ಅಲಾರಂ ಹೊಡೆಯುವುದಕ್ಕೆ ಪ್ರಾರಂಭಿಸಿದರೆ, ಅವನು ತನ್ನ ಕೈಯಿಂದಲೇ ಅದನ್ನು ನಿಲ್ಲಿಸಿ ನಿದ್ರೆಮಾಡುತ್ತಾನೆ.

ಭಯ ಅವನನ್ನು ಬಿಡುವುದಿಲ್ಲ. ಭಯಕ್ಕೆ ಕಾರಣ ಅವನು ಮಾಡಬಾರದುದನ್ನು ಏನೋ ಮಾಡಿರುವನು. ಸುಳ್ಳು ಹೇಳಿರಬಹುದು, ಯಾರಿಗೊ ಮೋಸಮಾಡಿರಬಹುದು. ಯಾವ ಸಮಯ ದಲ್ಲಿ ಸಿಕ್ಕಿ ಹಾಕಿಕೊಳ್ಳುವೆನೊ ಎಂಬ ಭಯ ಯಾವಾಗಲೂ ಅವನನ್ನು ಕಾಡುತ್ತಿರುವುದು. ಎಷ್ಟೋ ವೇಳೆ ಸಿಕ್ಕಿ ಹಾಕಿಕೊಂಡೂ ಇರುವನು. ಆದರೆ ಆ ಭಯಕ್ಕೆ ಕಾರಣವಾದ ಕೆಲಸವನ್ನು ಮಾತ್ರ ಇವನು ಮಾಡುವುದನ್ನು ಬಿಡುವುದಿಲ್ಲ.

ಶೋಕ ಪಡುವುದು ಇವನ ಸ್ವಭಾವ. ಯಾರ ಜೀವನದಲ್ಲಿ ಶೋಕಕ್ಕೆ ಕಾರಣವಿರುವುದಿಲ್ಲ? ಆದರೆ ಇವನು ಅದನ್ನೇ ಮೆಲುಕು ಹಾಕುತ್ತಿರುವನು. ಎಂದಿಗೂ ಮರೆಯುವುದಿಲ್ಲ. ತಾನು ಪಡಬಾರದ ಯಾತನೆಯನ್ನೆಲ್ಲ ಪಟ್ಟಿರುವೆನು, ಬೇಕಾದಷ್ಟು ದುಃಖ ಅನುಭವಿಸುತ್ತಿರುವೆನು ಎಂದು ಕಂಡವರ ಹತ್ತಿರವೆಲ್ಲ ಹೇಳಿಕೊಳ್ಳುವನು ಮತ್ತು ಮನಸ್ಸಿನಲ್ಲಿ ಪದೇ ಪದೇ ಅದನ್ನೇ ಕುರಿತು ಚಿಂತಿಸು ತ್ತಿರುವನು. ಇವನದು ಒಂದು ಅಸ್ವಾಭಾವಿಕವಾದ ಗುಣ. ಕೆಲವರಿಗೆ ಕಹಿ ಎಂದರೆ ಇಷ್ಟ. ಅದರ ಹೆಸರನ್ನು ಕೇಳಿದರೆ ಬಾಯಿಯಲ್ಲಿ ನೀರೂರುವುದು. ಈ ತಮೋಗುಣಿಯೂ ಹಾಗೆ ಶೋಕವನ್ನು ಮೆಲ್ಲುವನು. ಆದರೆ ಆ ಶೋಕಕ್ಕೆ ಕಾರಣವನ್ನು ಬಿಡುವುದಿಲ್ಲ. ಮುಂದಿನ ಶೋಕಕ್ಕೆ ಈಗ ಕೆಲಸ ಮಾಡುತ್ತಿರುವನು. ತುಂಬಾ ಖಾರ ಇದನ್ನು ತಿನ್ನಬಾರದಾಗಿತ್ತು ಎಂದು ಹೇಳುತ್ತಾ, ಮತ್ತೆ ಖಾರವಾದ ಅಡಿಗೆಯನ್ನು ಮಾಡುವಂತೆ ಇವನು.

ಯಾವಾಗಲೂ ವಿಷಾದಪಡುತ್ತಿರುವನು. ಜೀವನದಲ್ಲಿ ಎಷ್ಟೋ ಅಚಾತುರ್ಯಗಳು ಆಗಿರುತ್ತವೆ. ಆದರೆ ಅವನ್ನೇ ಕುರಿತು ಚಿಂತಿಸುತ್ತಿದ್ದರೆ, ಆಗಿಹೋಗಿದ್ದು ನೇರವಾಗುವುದಿಲ್ಲ. ಕೆಲಸ ಆಗಿ ಹೋಗಿದೆ. ಅಯ್ಯೊ ಹಾಗೆ ಮಾಡಬಾರದಾಗಿತ್ತು ಎಂದು ಈಗ ವ್ಯಥೆಪಡುತ್ತಿರುವನು. ಬಾಯಿಯಲ್ಲಿ ಹಿಂದೆ ಹಾಗೆ ಮಾಡಬಾರದಾಗಿತ್ತು ಎಂದು ಹೇಳಿಕೊಳ್ಳುತ್ತಿರುವನೆ ಹೊರತು, ಈಗ ಆ ಕೆಲಸ ಮಾಡುವುದನ್ನು ಬಿಡುವುದಿಲ್ಲ. ಮುಂದೆ ವಿಷಾದಪಡಬೇಕಾದ ಕೆಲಸಕ್ಕೆ ಈಗ ಬೀಜ ನೆಡುತ್ತಿರುವನು. ಎಷ್ಟಾದರೂ ಇವನಿಗೆ ಬುದ್ಧಿ ಬರುವುದಿಲ್ಲ.

ಸಾಲದ್ದಕ್ಕೆ, ಈ ಮನುಷ್ಯನನ್ನು ಮದ ಬೇರೆ ಮೆಟ್ಟಿಕೊಂಡಿದೆ. ನನ್ನ ಸಮಾನ ಇಲ್ಲ ಕೆಲಸ ಮಾಡವವರು ಎಂದು ತಿಳಿದಿರುವನು. ಇವನ ಯೋಗ್ಯತೆ ಎಷ್ಟೋ ವೇಳೆ ಗೊತ್ತಾಗಿದೆ. ಆದರೆ ಅವನು ಇದನ್ನು ಬಿಡುವುದಿಲ್ಲ. ಹುಲಿ ಹೇಗೆ ತನ್ನ ಮಚ್ಚೆಯನ್ನು ಬಿಡಲಾರದೊ ಹಾಗೆಯೇ ತಮೋಗುಣಿ ಮೇಲಿನ ಸ್ವಭಾವಗಳನ್ನು ಆಭರಣದಂತೆ ಹಾಕಿಕೊಂಡಿರುವನು. ಅದರಿಂದ ಏನೂ ಪ್ರಯೋಜನವಿಲ್ಲ. ಆದರೂ ಆತ ಭಂಡ, ಅದನ್ನು ಬಿಡುವುದಿಲ್ಲ.

\begin{verse}
ಸುಖಂ ತ್ವಿದಾನೀಂ ತ್ರಿವಿಧಂ ಶೃಣು ಮೇ ಭರತರ್ಷಭ~।\\ಅಭ್ಯಾಸಾದ್ರಮತೇ ಯತ್ರ ದುಃಖಾಂತಂ ಚ ನಿಗಚ್ಛತಿ \versenum{॥ ೩೬~॥}
\end{verse}

\begin{verse}
ಯತ್ತದಗ್ರೇ ವಿಷಮಿವ ಪರಿಣಾಮೇಽಮೃತೋಪಮಮ್~।\\ತತ್ಸುಖಂ ಸಾತ್ತ್ವಿಕಂ ಪ್ರೋಕ್ತಮಾತ್ಮಬುದ್ಧಿಪ್ರಸಾದಜಮ್ \versenum{॥ ೩೬~॥}
\end{verse}

{\small ಅರ್ಜುನ, ಈಗ ಮೂರು ವಿಧವಾದ ಸುಖವನ್ನೂ ಕೇಳು. ದೀರ್ಘಾಭ್ಯಾಸದ ಮೂಲಕ ಯಾವ ಸುಖದಲ್ಲಿ ರಮಿಸುತ್ತಾನೆಯೋ ಮತ್ತು ದುಃಖದ ಅಂತ್ಯವನ್ನು ಹೊಂದುತ್ತಾನೆಯೊ, ಯಾವ ಸುಖ ಮೊದಲು ವಿಷದಂತೆ ಕೊನೆಯಲ್ಲಿ ಅಮೃತದಂತೆ ಇರುವುದೊ--ಅಂತಹ ಆತ್ಮಬುದ್ಧಿಯ ಪ್ರಸಾದದಿಂದುಂಟಾದ ಸುಖವನ್ನು ಸಾತ್ತ್ವಿಕವೆಂದು ಹೇಳುತ್ತಾರೆ.}

ಮನುಷ್ಯ ಬದುಕಿರುವುದೇ ಸುಖ ಪಡುವುದಕ್ಕೆ. ಅದಿಲ್ಲದೆ ಇದ್ದರೆ ಜೀವನದಲ್ಲಿ ಯಾರಿಗೂ ರುಚಿ ಇರುವುದಿಲ್ಲ. ಈಗ ಸುಖ ಅನುಭವಿಸಬೇಕು, ಈಗ ಅದು ಇಲ್ಲದೆ ಇದ್ದರೆ ಮುಂದಿನ ಸುಖಕ್ಕಾದರೂ ಈಗಿನಿಂದ ಕಷ್ಟಪಡಬೇಕು. ಅಂತು ಮನುಷ್ಯ ಸುಖಕ್ಕಾಗಿ ದುಡಿಯುತ್ತಾನೆ. ಇಂತಹ ಸುಖದಲ್ಲಿಯೂ ಸಾತ್ತ್ವಿಕ, ರಾಜಸಿಕ ಮತ್ತು ತಾಮಸಿಕ ಸುಖ ಎಂದು ಮೂರು ಬಗೆ ಇದೆ. ಸಾತ್ತ್ವಿಕ ಸುಖಕ್ಕೆ ದೀರ್ಘ ಅಭ್ಯಾಸ ಬೇಕು. ಅದು ಸುಮ್ಮನೆ ಸಿಕ್ಕುವುದಿಲ್ಲ. ಅದು ಸ್ವಾಭಾವಿಕವಾಗಿ ನಮ್ಮ ಜೊತೆಯಲ್ಲಿಯೇ ಬರುವುದು ಅಪರೂಪ. ಮಗು ಹುಟ್ಟುತ್ತಲೆ ಕುಡಿಯುತ್ತದೆ, ತಿನ್ನುತ್ತದೆ. ಅದಕ್ಕೆ ಯಾರೂ ಅದನ್ನು ಕಲಿಸಬೇಕಾಗಿಲ್ಲ. ಅದಕ್ಕೆ ಪ್ರಯತ್ನವನ್ನೂ ಮಾಡಬೇಕಾಗಿಲ್ಲ. ಹಾಗಿಲ್ಲ ಸಾತ್ತ್ವಿಕ ಸುಖ. ಅದಕ್ಕೆ ತುಂಬಾ ಪ್ರಯತ್ನಪಡಬೇಕು. ಸಾತ್ತ್ವಿಕ ಸುಖ ಬರಬೇಕಾದರೆ ನಾವು ಮೊದಲು ಇಂದ್ರಿಯಗಳನ್ನು ನಿಗ್ರಹಿಸಬೇಕು, ಚಿತ್ತವನ್ನು ಶುದ್ಧಿ ಮಾಡಿಕೊಳ್ಳಬೇಕು. ಅನಂತರ ಅದನ್ನು ಏಕಾಗ್ರ ಮಾಡಬೇಕು. ಅನಂತರ ಅದನ್ನು ಪರಮಾತ್ಮನ ಮೇಲೆ ತಿರುಗಿಸಬೇಕು. ಇದಕ್ಕೆಲ್ಲ ಬೇಕಾದಷ್ಟು ಪ್ರಯತ್ನ ಆವಶ್ಯಕ. ಮನಸ್ಸು ಮತ್ತು ಇಂದ್ರಿಯವನ್ನು ನಿಗ್ರಹಿಸುವುದು ಮೊದಮೊದಲು ನೀರಸವಾದ ಹೋರಾಟ. ಅವುಗಳ ಸ್ವಭಾವ ಯಾವಾಗಲೂ ಪ್ರಪಂಚದ ಕಡೆ ಹೋಗುವುದು. ಅದನ್ನು ದೇವರ ಕಡೆ ತಿರುಗಿಸಬೇಕಾಗಿದೆ. ಒಂದು ಹೊಸ ಸ್ವಭಾವವನ್ನೇ ಮನಸ್ಸಿನಲ್ಲಿ ಹುಟ್ಟಿಸಿಕೊಳ್ಳಬೇಕಾಗಿದೆ. ಸದ್ ಅಭ್ಯಾಸಗಳು ಬೆಳೆಯಬೇಕಾದರೆ ನಿಧಾನ. ಅದಕ್ಕೆ ತುಂಬ ಅಭ್ಯಾಸ ಬೇಕು. ಒಂದು ಚೆನ್ನಾಗಿರುವ ಬೆಳೆಯನ್ನು ತೆಗೆಯಬೇಕಾದರೆ ಅದಕ್ಕೆ ಎಷ್ಟೊಂದು ಕಷ್ಟಪಡಬೇಕು. ನೆಲವನ್ನು ಉತ್ತಬೇಕು, ಗೊಬ್ಬರ ಹಾಕಬೇಕು, ಬಿತ್ತಬೇಕು, ಕಳೆಗಳನ್ನು ಕೀಳಬೇಕು. ಬರೀ ಕಳೆಗಳನ್ನು ಬೆಳೆಯಬೇಕಾದರೆ ಏನೂ ಪ್ರಯತ್ನ ಮಾಡಬೇಕಾಗಿಲ್ಲ. ಅದು ತನಗೆ ತಾನೆ ಹುಲುಸಾಗಿ ಬೆಳೆಯುವುದು. ಅದಕ್ಕೆ ಯಾರೂ ಉಳಬೇಕಾಗಿಲ್ಲ, ಬೀಜವನ್ನು ಬಿತ್ತಬೇಕಾಗಿಲ್ಲ. ಅದೆಲ್ಲ ಆಗಲೇ ನೆಲದಲ್ಲಿದೆ. ಅದು ಸ್ವಲ್ಪ ನೆಲದಲ್ಲಿ ತನುವೇರಿದರೆ ಚಿಗುರುವುದು. ಆದರೆ ಒಳ್ಳೆ ಬೆಳೆ ಹಾಗಲ್ಲ. ಅದಕ್ಕಾಗಿ ಬಹಳ ಪ್ರಯತ್ನ ಪಡಬೇಕು. ಹಾಗೆಯೇ ಮೊದಲು ನಮ್ಮ ಇಂದ್ರಿಯ ಮತ್ತು ಮನಸ್ಸನ್ನು ನಿಗ್ರಹಿಸುವುದರ ಮೇಲೆ ನಿಂತಿದೆ ಸಾತ್ತ್ವಿಕ ಸುಖ.

ಸಾತ್ತ್ವಿಕ ಸುಖ ಭದ್ರವಾದ ತಳಹದಿಯ ಮೇಲೆ ನಿಂತಿದೆ. ಬಂಡೆಯ ಮೇಲೆ ಮನೆ ಕಟ್ಟಿದಂತಿದೆ. ಯಾವ ಗಾಳಿಯೂ ಮಳೆಯೂ ಅದನ್ನು ಅಲ್ಲಾಡಿಸಲಾರದು. ಈ ಸುಖದಿಂದ ದುಃಖ ನಾಶವಾಗು ವುದು. ನಮ್ಮ ದುಃಖಕ್ಕೆಲ್ಲ ಕಾರಣ, ಅಜ್ಞಾನ ಮತ್ತು ನಮ್ಮಲ್ಲಿರುವ ತಾತ್ಕಾಲಿಕ ಇಂದ್ರಿಯ ಚಪಲ. ಯಾವಾಗ ಈ ಪ್ರಪಂಚದಲ್ಲಿ ದೇವರು ಒಬ್ಬನೇ ಸತ್ಯ ಎಂದು ತಿಳಿದುಕೊಳ್ಳುತ್ತಾನೆಯೋ, ಅವನನ್ನು ಪಡೆಯಲು ಮನಸ್ಸನ್ನು ಶುದ್ಧಿ ಮಾಡಿ ಅಂತರ್ಮುಖ ಮಾಡುತ್ತಾನೆಯೊ ಆಗ ಅವನು ಇಂದ್ರಿಯ ಸುಖದ ಮೇಲೆ ಮನಸ್ಸನ್ನು ಹಾಕುವುದಿಲ್ಲ. ಕಾರಣವೇ ನಿರ್ಮೂಲವಾದಮೇಲೆ ಇನ್ನು ದುಃಖವೆಲ್ಲಿ ಬರುವುದು? ಜೀವನದಲ್ಲಿ ಯಾವುದೂ ಅವನಿಗೆ ದುಃಖವನ್ನು ತರಲಾರದು. ಏಕೆಂದರೆ ಅವನು ಈ ಪ್ರಪಂಚದ ನಶ್ವರತೆಯನ್ನು ಮೊದಲಿನಿಂದಲೂ ಬಲ್ಲ. ಅವನ ಮನಸ್ಸು ಅದಕ್ಕೆ ಅಣಿಯಾಗಿರುವುದು. ಒಂದು ಘಟನೆ ಆದಮೇಲೆ ಅದರ ಶಮನಕ್ಕೆ ಅವನು ಅಲ್ಲಿ ಇಲ್ಲಿ ಓಡಿ ಹೋಗುವುದಿಲ್ಲ. ಈ ಪ್ರಪಂಚದ ಸ್ವಭಾವವೇ ಇದು ಎಂದು, ಈ ಜೀವನದಲ್ಲಿ ಎಲ್ಲವನ್ನೂ ಧೀರನಾಗಿ ಎದುರಿಸಬಲ್ಲ. ಅವನು ಪ್ರಪಂಚದಲ್ಲಿ ತನಗೆ ಆಗುವುದನ್ನು ಬದಲಾಯಿಸುವುದಿಲ್ಲ. ಆಗುವುದು ತನ್ನ ಮೇಲೆ ಪರಿಣಾಮವನ್ನು ಬಿಡದಂತೆ ತನ್ನನ್ನು ತಿದ್ದುಕೊಂಡಿರುತ್ತಾನೆ. ಈ ರಹಸ್ಯವನ್ನು ಬಲ್ಲವನು ಸಾತ್ತ್ವಿಕ. ದುಃಖದ ಅವಸಾನ ಎಂದರೆ ಅವನಿಗೆ ದುಃಖವನ್ನು ತರುವ ಪ್ರಸಂಗಗಳೇ ಬರುವುದಿಲ್ಲ ಎಂದಲ್ಲ. ಅವನಿಗೂ ಸಾಧಾರಣ ಮನುಷ್ಯನನ್ನು ಯಾವ ಪ್ರಸಂಗಗಳು ಹೃದಯವನ್ನು ಹಿಂಡುವುವೋ ಅವುಗಳೆಲ್ಲ ಆಗುವುವು. ಆದರೆ ಅವನು ಅದಕ್ಕೆ ಒಳಗಾಗುವುದಿಲ್ಲ. ಹೇಗೆ ಕಾಲರಾ ಇನಾಕ್ಯುಲೇಷನ್ ಹಾಕಿಕೊಂಡರೆ ಅನಂತರ ಕಾಲರಾ ಬಂದರೂ ಅವನಿಗೆ ಏನೂ ಮಾಡಲಾರದೊ, ಹಾಗೆ ಇವನು.

ಸಾತ್ತ್ವಿಕ ಸುಖ ಮೊದಲು ವಿಷದಂತೆ ಪ್ರಾರಂಭವಾಗುವುದು. ಅನಂತರ ಅದು ಅಮೃತದಂತೆ ಇರುವುದು. ನೆಲ್ಲಿಕಾಯಿ ಬಾಯಿಯಲ್ಲಿ ಹಾಕಿಕೊಂಡಾಗ ಒಗರು, ಅದನ್ನು ಬಾಯಲ್ಲಿ ಅಗಿದು ನೀರನ್ನು ಕುಡಿದರೆ, ಆ ನೀರು ಸಿಹಿಯಾಗಿ ಕಾಣುವುದು. ಮೊದಲು ಒಗರು, ಆಮೇಲೆ ಸಿಹಿ. ಹಾಗೆಯೆ ಸಾತ್ತ್ವಿಕ ಸುಖ. ಮೊದಮೊದಲು ಇದನ್ನು ಇನ್ನೂ ಅಭ್ಯಾಸಮಾಡುತ್ತಿರುವಾಗ ತುಂಬಾ ಕಷ್ಟ ಪಡಬೇಕು. ಧ್ಯಾನ ಮಾಡುವುದಕ್ಕೆ ಪ್ರಯತ್ನ ಮಾಡಿದಾಗ ಮೊದಮೊದಲು ಮನಸ್ಸು ಎಷ್ಟು ಹೋರಾಡಬೇಕಾಗುವುದು? ಧ್ಯಾನಕ್ಕೆ ಕುಳಿತಾಗಲೆ, ಎಲ್ಲೂ ಇಲ್ಲದ ಆತಂಕಗಳೆಲ್ಲ ಬಂದು ಮನಸ್ಸನ್ನು ಮುತ್ತುವುವು. ಆಗ ಅನ್ನಿಸುವುದು, ಇದಕ್ಕೆ ಮುಂಚೆ ಮನಸ್ಸು ಶಾಂತವಾಗಿತ್ತು. ಈಗ ಎಲ್ಲೂ ಇಲ್ಲದ ಗಲಿಬಿಲಿ ಪ್ರಾರಂಭವಾಗಿದೆಯಲ್ಲ ಎಂದು. ಆದರೆ ಬಿಡದೆ ನಾವು ಹೋರಾಡಬೇಕು. ಆಗಲೆ ಅದು ಕ್ರಮೇಣ ಏಕಾಗ್ರವಾಗುತ್ತ ಬಂದು, ಭಗವಂತನಲ್ಲಿ ನೆಲೆಸುವುದು. ಅನಂತರ ಅವನಿಗೆ ಬರುವ ಆನಂದಕ್ಕೆ ಎಣೆಯಿಲ್ಲ. ಇಂದ್ರಿಯದ ಮೂಲಕ ಬರುವ ಯಾವ ಆನಂದವೂ ಅದಕ್ಕೆ ಎಣೆಯಿಲ್ಲ. ಧ್ಯಾನಾವಸ್ಥೆಯಲ್ಲಿ ಮನಸ್ಸು ಸಚ್ಚಿದಾನಂದ ಸ್ವರೂಪನಾದ ಭಗವಂತನ ಅಗ್ನಿಕುಂಡದಲ್ಲಿರುವುದು. ಅದರಿಂದ ಹೊರಗೆ ಬಂದಮೇಲೂ ಆ ಕಾವು ಬಹುಕಾಲ ಅದರಲ್ಲಿ ಇರುವುದು.

ಈ ಸುಖ ಇಂದ್ರಿಯ ಮತ್ತು ಮನಸ್ಸನ್ನು ನಿಗ್ರಹಿಸಿದುದರ ಫಲ. ಇಂದ್ರಿಯ ಸುಖ ಬಾಹ್ಯವಸ್ತು ವನ್ನು ಆಶ್ರಯಿಸಿದೆ. ನಮ್ಮ ಇಂದ್ರಿಯಗಳಿಗೆ ಅವಕ್ಕೆ ಪ್ರಿಯವಾದ ವಿಷಯವಸ್ತುಗಳ ಸಂಬಂಧ ವಾದಾಗ ಅದಕ್ಕೆ ಸುಖ. ಅದಿಲ್ಲದೆ ಇದ್ದರೆ ದುಃಖ. ಇಂದ್ರಿಯದ ಹಿಂದೆ ಮನಸ್ಸು ಇದರೊಂದಿಗೆ ತಾದಾತ್ಮ್ಯ ಭಾವವನ್ನು ಪಡೆದುಕೊಂಡು ಆ ಸುಖವನ್ನು ಅನುಭವಿಸುವುದು. ಆದರೆ ಸಾತ್ತ್ವಿಕ ಸುಖ ಹೊರಗಿನ ಸಂಪರ್ಕದಿಂದ ಬರುವುದಿಲ್ಲ. ಮನಸ್ಸು ಮತ್ತು ಇಂದ್ರಿಯಗಳನ್ನು ನಿಗ್ರಹಿಸಿದವನ ಹೃದಯದಿಂದ ಸಾತ್ತ್ವಿಕ ಸುಖ ಚಿಲುಮೆಯಂತೆ ಜಿನುಗುವುದು. ಇದು ಬರುವುದು ಭಗವಂತನ ಸ್ಪರ್ಶದಿಂದ. ಇದಕ್ಕೆ ಯಾವ ಬಾಹ್ಯ ಆಸರೆಯೂ ಬೇಕಿಲ್ಲ, ಆಧಾರವೂ ಬೇಕಿಲ್ಲ.

\begin{verse}
ವಿಷಯೇಂದ್ರಿಯಸಂಯೋಗಾದ್ಯತ್ತದಗ್ರೇಽಮೃತೋಪಮಮ್~।\\ಪರಿಣಾಮೇ ವಿಷಮಿವ ತತ್ಸುಖಂ ರಾಜಸಂ ಸ್ಮೃತಮ್ \versenum{॥ ೩೮~॥}
\end{verse}

{\small ವಿಷಯ ಇಂದ್ರಿಯಗಳ ಸಂಯೋಗದಿಂದ ಯಾವ ಸುಖ ಮೊದಲು ಅಮೃತದಂತೆ, ಕೊನೆಗೆ ವಿಷದಂತೆ ಇರುವುದೊ ಆ ಸುಖ ರಾಜಸಿಕ.}

ರಾಜಸಿಕ ಸುಖಕ್ಕೆ ಬಾಹ್ಯವಸ್ತುವಿನ ಆಧಾರ ಬೇಕು. ಇಂದ್ರಿಯ, ಅದಕ್ಕೆ ಸಂಬಂಧಪಟ್ಟ ವಿಷಯ ವಸ್ತುವಿನ ಸಂಪರ್ಕವನ್ನು ಪಡೆದಾಗಲೆ ಸುಖವೆಂಬ ವೇದನೆ ಆಗಬೇಕಾದರೆ. ಒಲೆ ಉರಿಯುತ್ತಿದ್ದರೆ ಮೇಲಿರುವುದು ಹೇಗೆ ಕುದಿಯುತ್ತಿರುವುದೋ ಹಾಗೆ ವಿಷಯ ಸುಖ.

ಈ ಸುಖ ಮೊದಮೊದಲು ಆನಂದ. ಅದನ್ನು ತಿನ್ನುವಾಗ ಚಪ್ಪರಿಸುವಾಗ ಆಗುವ ಸುಖವನ್ನು ಯಾರೂ ಹೇಳಿಕೊಡಬೇಕಾಗಿಲ್ಲ. ಅದು ತನಗೆ ತಾನೆ ಬರುವುದು. ರುಚಿಕರವಾದುದನ್ನು ತಿಂದು ಆನಂದಿಸುವುದನ್ನು ಯಾರೂ ಹೇಳಿಕೊಡಬೇಕಾಗಿಲ್ಲ. ಇಂದ್ರಿಯಕ್ಕೆ ಅದರ ವಿಷಯದ ಮೇಲೆ ಆಕರ್ಷಣೆ ಸ್ವಾಭಾವಿಕವಾಗಿರುವುದು. ಯಾವ ಪ್ರಯತ್ನವೂ ಇಲ್ಲದೆ, ಎಲ್ಲರೂ ಈ ಸುಖವನ್ನು ಅನುಭವಿಸಬಹುದು. ಆದರೆ ಕೊನೆಗೆ ಅದು ನಮಗೆ ದುಃಖವನ್ನು ತರುವುದು. ನಾವು ಪ್ರತಿಯೊಂದು ಸಲ ಇಂದ್ರಿಯ ಸುಖವನ್ನು ಅನುಭವಿಸಿದಾಗಲೂ, ಆ ವಸ್ತುವಿಗೆ ದಾಸರಾಗುತ್ತೇವೆ. ಆ ವಸ್ತುವನ್ನು ಪಡೆಯುವುದಕ್ಕೆ ಏನನ್ನು ಬೇಕಾದರೂ ಮಾಡುವೆವು. ನಮ್ಮ ಸರ್ವಸ್ವವನ್ನೇ ಬೇಕಾದರೆ ಅದಕ್ಕೆ ಮಾರಿಕೊಳ್ಳುತ್ತೇವೆ. ದೊಡ್ಡ ದೊಡ್ಡ ವಿಷಯ ವಸ್ತುಗಳನ್ನು ಬಿಟ್ಟು ಬಿಡೋಣ. ಕೇವಲ ಸಣ್ಣ ದುರಭ್ಯಾಸಗಳಿಗೆ ತುತ್ತಾದವನ ಪಾಡನ್ನು ನೋಡಿದರೆ ನಮಗೆ ಮರುಕ ಬರುವುದು. ಒಂದು ಚುಟಿಕೆ ನಶ್ಯ ಇಲ್ಲದೇ ಇದ್ದರೆ ಯಾರ ಯಾರನ್ನೋ ತಿರುಪೆ ಬೇಡುತ್ತಾನೆ. ಒಂದು ಸಿಗರೇಟಿಗೆ, ಬೀಡಿಗೆ, ಒಂದು ಕಪ್ಪು ಕಾಫಿಗೆ ನಾವು ಎಷ್ಟು ದೊಡ್ಡ ದಾಸರು! ನಾವು ತಿಳಿದುಕೊಂಡಿದ್ದೇವೆ, ಅದನ್ನು ಅನುಭವಿಸುತ್ತಿದ್ದೇವೆ ಎಂದು. ಆದರೆ ಇಷ್ಟು ದೊಡ್ಡ ಮನುಷ್ಯ ಅಷ್ಟು ಅಲ್ಪ ವಸ್ತುವಿಗೆ ದಾಸನಾಗಿ ಅದಕ್ಕೆ ಎಂತಹ ಅವಮಾನವನ್ನಾದರೂ ಸಹಿಸಲು ಸಿದ್ಧನಾಗುವನು!

ಇನ್ನು ದೊಡ್ಡದೊಡ್ಡ ವಿಷಯಗಳಿಗೆ ದಾಸರಾದವರ ಪಾಡನ್ನು ಹೇಳಲೇಬೇಕಾಗಿಲ್ಲ. ನಮ್ಮ ಪ್ರಪಂಚದ ಸುತ್ತ ಅದರ ನಿದರ್ಶನಗಳು ಹೇರಳವಾಗಿವೆ. ಮೀನು, ಗಾಳದ ಕೊನೆಯಲ್ಲಿರುವ ಹುಳುವನ್ನು ತಿನ್ನುವಾಗ ಆನಂದಪಡುವುದು. ಈ ಆನಂದವೇ ಅದಕ್ಕೆ ಮೃತ್ಯು. ಗಾಳ ಗಂಟಲಲ್ಲಿ ಸಿಕ್ಕಿಕೊಳ್ಳುವುದು. ಹಾಗೆಯೆ ವಿಷಯ ವಸ್ತುವಿನ ಗಾಳ. ಒಮ್ಮೆ ಅದರ ಪಾಶಕ್ಕೆ ಬಿತ್ತು ಎಂದರೆ ತಪ್ಪಿಸಿಕೊಂಡು ಬರುವುದು ಕಷ್ಟ. ನಮ್ಮನ್ನು ಜೀವಸಹಿತ ಹಿಂಡುವುದು. ಕೆಲವು ರಾಕ್ಷಸೀಯ ಸ್ವಭಾವದ ಹೂವುಗಳಿವೆ. ಯಾವುದಾದರೂ ಕ್ರಿಮಿಕೀಟ ಅದರೊಳಗೆ ಬಂದರೆ, ತಕ್ಷಣವೇ ಹೂವಿನ ರೇಕುಗಳು ಮುಚ್ಚಿಕೊಳ್ಳುವುವು. ಆ ಕೀಟ ಅದರಲ್ಲಿ ಬಂಧಿಯಾಗುವುದು. ಆ ಹೂವು ತನ್ನಿಂದ ಒಂದು ಬಗೆಯ ದ್ರವ್ಯವನ್ನು ಸೃಜಿಸಿ ಆ ಕೀಟವನ್ನು ಕರಗಿಸಿ ಅದರ ಸಾರವನ್ನೆಲ್ಲ ಹೀರಿಬಿಡುವುದು. ಹಾಗೆ ಇದೆ ವಿಷಯ ಕುಸುಮ. ಅದು ನೋಡುವುದಕ್ಕೆ ಮನೋಹರ. ಅದರೊಳಗೆ ಬಿತ್ತು ಎಂದರೆ ಆಯಿತು, ನಾವು ನಿರ್ನಾಮವಾಗುತ್ತೇವೆ.

\begin{verse}
ಯದಗ್ರೇ ಚಾನುಬಂಧೇ ಚ ಸುಖಂ ಮೋಹನಮಾತ್ಮನಃ~।\\ನಿದ್ರಾಲಸ್ಯಪ್ರಮಾದೋತ್ಥಂ ತತ್ತಾಮಸಮುದಾಹೃತಮ್ \versenum{॥ ೩೯~॥}
\end{verse}

{\small ಯಾವ ಸುಖ ಮೊದಲಲ್ಲಿ ಮತ್ತು ಕೊನೆಯಲ್ಲಿ ಆತ್ಮನನ್ನು ಮೋಹಗೊಳಿಸುವುದೊ, ನಿದ್ರೆ, ಆಲಸ್ಯ, ಪ್ರಮಾದದಿಂದ ಉಂಟಾದುದೊ ಅದನ್ನು ತಾಮಸಿಕ ಸುಖ ಎಂದು ಹೇಳುತ್ತಾರೆ.}

ಸಾತ್ತ್ವಿಕ ಸುಖದಲ್ಲಿ ಮೊದಲು ಕಹಿ, ಅನಂತರ ಸಿಹಿ; ರಾಜಸಿಕ ಸುಖದಲ್ಲಿ ಮೊದಲು ಸಿಹಿ ಅನಂತರ ಕಹಿ. ತಾಮಸಿಕ ಸುಖದಲ್ಲಾದರೊ ಮೊದಲೂ ಕಹಿಯೆ, ಕೊನೆಗೂ ಕಹಿಯೇ. ಅವರಿಗೆ ಅದೇ ಇಷ್ಟ. ರಾಜಸಿಕ, ವ್ಯಥೆಪಟ್ಟು ನಾನು ಹಾಗೆ ಮಾಡಬಾರದಾಗಿತ್ತು ಎಂದು ಪಶ್ಚಾತ್ತಾಪವನ್ನಾ ದರೂ ಪಡುತ್ತಾನೆ. ಆದರೆ ತಾಮಸಿಕನಿಗೆ ಅದೇನೂ ಇಲ್ಲ. ದುಃಖಪಡುತ್ತಾನೆ, ಆದರೂ ಆ ದುಃಖಕ್ಕೆ ಕಾರಣವಾಗಿರುವುದನ್ನೇ ಮಾಡುತ್ತಾನೆ. ಈ ಸುಖದ ಮೇಲೆಯೇ ಅವನಿಗೆ ಆಕರ್ಷಣೆ. ಈ ಸುಖ ವಾದರೊ ವಸ್ತುವಿನ ನೈಜ ಸ್ವಭಾವವನ್ನು ಎಂದಿಗೂ ವ್ಯಕ್ತಪಡಿಸುವುದಿಲ್ಲ.

ಇದಕ್ಕೆ ಕಷ್ಟವನ್ನೇ ಪಡಬೇಕಾಗಿಲ್ಲ. ಯಾವ ದುರಭ್ಯಾಸಗಳು ಹಿಂದಿನಿಂದ ಬಂದಿವೆಯೋ ಅವನ್ನು ಮುಂದುವರಿಸಿದರೆ ಸಾಕಾಗಿದೆ. ನಿದ್ರೆ ಎಂದರೆ ಅವನಿಗೆ ತುಂಬಾ ಇಷ್ಟ. ಹೊತ್ತಾಗಿ ಎದ್ದರೆ ಎಷ್ಟೋ ಕೆಲಸಗಳು ಕೆಟ್ಟು ಹೋಗುತ್ತವೆ. ಅವುಗಳನ್ನು ಗಮನಿಸುವುದಿಲ್ಲ. ಎಷ್ಚು ಚೆನ್ನಾಗಿ ನಿದ್ರೆ ಮಾಡಿದೆ ಎನ್ನುತ್ತಾನೆಯೆ ಹೊರತು, ಈ ಹಾಳು ನಿದ್ರೆಯಿಂದ ಕೆಲಸ ಕೆಟ್ಟುಹೋಯಿತು ಎಂದು ಯೋಚಿಸುವು ದಿಲ್ಲ.ಇವನು ಆಲಸ್ಯ ಜೀವಿ. ಎಂದಿಗೂ ಮಾಡಬೇಕಾದುದನ್ನು ಮಾಡುವುದಿಲ್ಲ. ಏನಾದರೂ ನೆವ ಹೇಳಿ ಮುಂದೆ ಹಾಕುತ್ತಿರುವನು. ಇದರಿಂದ ವ್ಯಥೆಪಡುವುದಿಲ್ಲ. ಇದಕ್ಕೆ ಹೆಮ್ಮೆಪಡುವನು. ನಾನು ಹೇಗೆ ಮೋಸ ಮಾಡಿದೆ ಎಂದು ಸಂತೋಷಪಡುವನು. ಪ್ರಮಾದ ಎಂದರೆ ತಮ್ಮ ಕೆಲಸವನ್ನು ಮಾಡದೆ, ಮತ್ತೊಬ್ಬರ ಕೆಲಸವನ್ನು ತಪ್ಪು ಮಾಡುವರು. ಮಾಡಬಾರದ ಕೆಲಸವನ್ನು ಮಾಡುವರು, ಮಾಡುವ ಕೆಲಸವನ್ನು ಬಿಡುವರು. ರಜೋಗುಣಿ ತಪ್ಪಿನಿಂದ ತಿದ್ದಿಕೊಳ್ಳುವನು. ತಪ್ಪು ಮಾಡಿದ್ದಕ್ಕೆ ವ್ಯಥೆಪಡುವನು. ಆದರೆ ಈ ತಮೋಗುಣಿ ತಿದ್ದುಕೊಳ್ಳುವವನಲ್ಲ, ವ್ಯಥೆಯನ್ನು ಪಡುವುದಿಲ್ಲ. ಪುನಃ ಪುನಃ ಹಿಂದಿನ ತಪ್ಪನ್ನೇ ಮಾಡುತ್ತಿರುವನು.

\begin{verse}
ನ ತದಸ್ತಿ ಪೃಥಿವ್ಯಾಂ ವಾ ದಿವಿ ದೇವೇಷು ವಾ ಪುನಃ~।\\ಸತ್ತ್ವಂ ಪ್ರಕೃತಿಜೈರ್ಮುಕ್ತಂ ಯದೇಭಿಃ ಸ್ಯಾತ್ತ್ರಿಭಿರ್ಗುಣೈಃ \versenum{॥ ೪೦~॥}
\end{verse}

{\small ಪ್ರಕೃತಿಯಿಂದ ಹುಟ್ಟಿದ ಈ ಮೂರು ಗುಣಗಳಿಂದ ಮುಕ್ತವಾಗಿರುವ ಪ್ರಾಣಿ ಸಮೂಹವು ಈ ಪ್ರಪಂಚ ದಲ್ಲಾಗಲಿ, ಸ್ವರ್ಗದಲ್ಲಾಗಲಿ, ದೇವತೆಗಳಲ್ಲಾಗಲಿ ಇಲ್ಲ.}

ಮನಷ್ಯರಾಗಲಿ, ದೇವತೆಗಳಾಗಲಿ, ದೇವರಾಗಲಿ, ದೇಶ ಕಾಲ ನಿಮಿತ್ತದ ಜಗತ್ತಿಗೆ ಬಂದೊಡನೆ ಈ ಗುಣಗಳಲ್ಲಿ ಯಾವುದಾದರೂ ಅವನ ಸ್ವಭಾವಕ್ಕೆ ಅನುಗುಣವಾಗಿ ಅಂಟಿಕೊಳ್ಳುವುದು. ಇದೇನೋ ಅಕಸ್ಮಾತ್ ಅಂಟಿಕೊಳ್ಳುವುದಿಲ್ಲ. ಹುಟ್ಟುವುದಕ್ಕೆ ಮುಂಚೆಯೇ ಹಿಂದಿನ ಜನ್ಮದಲ್ಲಿ ಅವನು ಕರ್ಮಗಳನ್ನು ಮಾಡಿ, ಮೂರು ಗುಣಗಳಲ್ಲಿ ಯಾವುದಾದರೂ ಒಂದಕ್ಕೇ ಮೀಸಲಾಗಿ ಬರುವನು. ಒಮ್ಮೆ ಅವನು ಈ ಗುಣದ ವೃತ್ತಕ್ಕೆ ಬಿದ್ದರೆ ತಪ್ಪಿಸಿಕೊಂಡು ಹೋಗುವುದು ಕಷ್ಟ. ಆದರೆ ಈ ವೃತ್ತದಲ್ಲಿ ವಿಕಾಸ ಆಗುವುದಕ್ಕೆ ಸಾಧ್ಯ. ಅದು ನಮ್ಮ ಸ್ವಪ್ರಯತ್ನದ ಮೇಲೆ ನಿಂತಿದೆ. ತಮೋಗುಣಿ ಬೇಕಾದಷ್ಟು ಪೆಟ್ಟು ತಿಂದು ಅಜ್ಞಾನದಲ್ಲಿ ತೊಳಲಿ, ಮುಂಚೆ ಏನನ್ನು ಕಲಿತುಕೊಳ್ಳುವು ದಿಲ್ಲವೆಂದು ಹಟ ಮಾಡಿದ್ದಿದ್ದನೊ ಅದನ್ನು ಸ್ವಲ್ಪ ಸ್ವಲ್ಪವಾಗಿ ಕಲಿಯುತ್ತ ಬರುವನು. ಕ್ರಮೇಣ ಅವನು ರಜೋಗುಣದ ಮೆಟ್ಟಿಲಿಗೆ ಬರುವನು. ಅಲ್ಲಿಂದ ಅವನು ಹೋರಾಡಿ ಕ್ರಮೇಣ ಸತ್ತ್ವಗುಣ ದಲ್ಲಿ ಇರುವಾಗ, ಅವನು ಫಲಾಪೇಕ್ಷೆಯಿಲ್ಲದೆ ಅನಾಸಕ್ತನಾಗಿ ಕರ್ಮವನ್ನು ಮಾಡಿಕೊಂಡು ಹೋಗುತ್ತಿದ್ದರೆ, ಅವನು ಸತ್ತ್ವಗುಣದಲ್ಲಿದ್ದರೂ ಆ ಗುಣಕ್ಕೂ ಅತೀತನಾಗುವನು. ಅದೇ ಗುಣಾ ತೀತನ ಸ್ಥಿತಿ. ಸತ್ಯಕ್ಕೆ ಬಹಳ ಸಮೀಪವಾದ ಸ್ಥಿತಿ ಇದು. ತಮಸ್ಸಿನಿಂದ ರಜೊಗುಣದ ಮೂಲಕ ಸತ್ತ್ವಕ್ಕೆ ಬಂದು, ಇದಕ್ಕೂ ಅತೀತನಾಗಿ ಹೋಗುವನು. ಯಾವಾಗ ಅತೀತನಾಗುವನೊ ಅವನು ತನ್ನ ವ್ಯಕ್ತಿತ್ವಕ್ಕೆ ಅತೀತನಾಗಿ ಹೋಗಿ ಪರಮಾತ್ಮನಲ್ಲಿ ಒಂದಾಗಬೇಕಾಗಿ ಬರುವುದು.

ಈ ಗುಣಗಳು ನೀರಿನ ಮೂರು ಅವಸ್ಥೆಯಂತೆ. ನೀರು ನೀರ್ಗಲ್ಲಾಗಿ, ಜಡವಾಗಿ ಒಂದು ಕಡೆ ಗಟ್ಟಿಗಟ್ಟಿಯಾಗಿ ಬಿದ್ದಿರುವುದು. ಇದು ತಮೋಗುಣ. ಆ ನೀರು ಕರಗಿ ವೇಗದಿಂದ ಬಂಡೆಯಿಂದ ಬಂಡೆಗೆ ಬಿದ್ದು ಭೋರ್ಗರೆದು ಹರಿದುಕೊಂಡು ಹೋಗುವುದು ರಾಜಸಿಕ ಸ್ಥಿತಿ. ಇದೇ ನೀರು ಆವಿಯಾಗಿ ಕೆಳಗಿನಿಂದ ಮೇಲಕ್ಕೆ ಹೋಗುವಾಗ ತನ್ನ ಕಶ್ಮಲವನ್ನೆಲ್ಲ ಇಲ್ಲೇ ಬಿಟ್ಟೇಳುವುದು. ಇದು ಸಾತ್ತ್ವಿಕ ಸ್ಥಿತಿ.

\begin{verse}
ಬ್ರಾಹ್ಮಣಕ್ಷತ್ರಿಯವಿಶಾಂ ಶೂದ್ರಾಣಾಂ ಚ ಪರಂತಪ~।\\ಕರ್ಮಾಣಿ ಪ್ರವಿಭಕ್ತಾನಿ ಸ್ವಭಾವಪ್ರಭವೈರ್ಗುಣೈಃ \versenum{॥ ೪೧~॥}
\end{verse}

{\small ಅರ್ಜುನ, ಬ್ರಾಹ್ಮಣ, ಕ್ಷತ್ರಿಯ, ವೈಶ್ಯ ಮತ್ತು ಶೂದ್ರರ ಕರ್ಮಗಳು ಸ್ವಭಾವದಿಂದ ಹುಟ್ಟಿದ ಗುಣಗಳಿಂದ ವಿಭಾಗಿಸಲ್ಪಟ್ಟಿವೆ.}

ನಾಲ್ಕು ವರ್ಣಗಳು ಎಲ್ಲಾ ಸಮಾಜದಲ್ಲಿಯೂ ಇವೆ. ಈ ವರ್ಣಗಳಿಲ್ಲದೆ ಇದ್ದರೆ ಆ ಸಮಾಜ ಅನಾಯಕವಾಗುವುದು. ನಾವು ಕರೆಯುವಂತೆ ಅವರು ಕರೆಯದೆ ಇರಬಹುದು ಮತ್ತು ಬರೀ ಜಾತಿಯಿಂದಲೇ ಅದನ್ನು ನಿಷ್ಕರ್ಷಿಸದೆ ಇರಬಹುದು. ಎಲ್ಲಾ ದೇಶದಲ್ಲಿಯೂ ಬ್ರಾಹ್ಮಣ, ಎಂದರೆ ಆಲೋಚನಾ ಜೀವಿ, ದೀರ್ಘ ವಿಚಾರ ಮಾಡುವ ತಾತ್ವಿಕ ಇರಬೇಕು. ಅವರೇ ಸಮಾಜಕ್ಕೆ ಮಾರ್ಗ ದರ್ಶಕರು. ಎರಡನೆಯವರೆ ಆಳುವವರು \enginline{(Administrators)}. ಇವರು ಇಲ್ಲದೇ ಇದ್ದರೆ ಯಾವ ಸಮಾಜವೂ ಸುವ್ಯವಸ್ಥಿತವಾಗಿರುವುದಿಲ್ಲ. ಸರ್ಕಾರದ ಹುದ್ದೆಯಲ್ಲಿರುವವರೆಲ್ಲ ಸುಮಾರು ಈ ಗುಂಪಿನ ಕೆಳಗೆ ಬರುವರು. ಪೋಲೀಸು, ಕೋರ್ಟು ಕಛೇರಿ ಮಿಲಿಟರಿ ಇವುಗಳೆಲ್ಲ ಅದರಲ್ಲಿವೆ. ಮೂರನೆಯವರೆ, ವ್ಯಾಪಾರ ಮಾಡುವ ವ್ಯೆಶ್ಯರು. ನಾಲ್ಕನೆಯವರೇ ಊಳಿಗದವರು, ಶ್ರಮ ಜೀವಿ ಗಳು. ಇವರ ಶ್ರಮದಿಂದಲೆ ಫ್ಯಾಕ್ಟರಿ ಮುಂತಾದುವುಗಳೆಲ್ಲ ಕೆಲಸ ಮಾಡಬೇಕಾದರೆ.

ಆದರೆ ಭರತಖಂಡದಲ್ಲಿ ಒಂದು ದೋಷ ತಲೆದೋರಿತು. ಮೊದಲು ವರ್ಣವನ್ನು ಸ್ವಭಾವದಿಂದ ನಿರ್ಣಯಿಸುತ್ತಿದ್ದರು. ಪ್ರತಿಯೊಬ್ಬನೂ ಹುಟ್ಟವಾಗಲೇ ಯಾವುದಾದರೊಂದು ಸ್ವಭಾವದೊಡನೆ ಹುಟ್ಟುತ್ತಾನೆ. ಅದಕ್ಕೆ ತಕ್ಕಂತೆ ಅವನು ಕಸುಬನ್ನು ಆರಿಸಿಕೊಳ್ಳುತ್ತಿದ್ದ. ಆದರೆ ಕಾಲಕ್ರಮೇಣ, ಆ ವರ್ಣದಲ್ಲಿ ಹುಟ್ಟಿದರೆ ಸಾಕು, ಅವನನ್ನು ಆ ಹೆಸರಿನಿಂದ ಕರೆಯುತ್ತಿದ್ದರು. ಅವನಲ್ಲಿ ಆ ಗುಣಗಳಿಲ್ಲದೇ ಇದ್ದರೂ ಆ ವರ್ಣದ ಚೀಟಿಯೊಂದು ಇರುತ್ತಿತ್ತು. ಶ‍್ರೀಕೃಷ್ಣ, ಎಲ್ಲಿಯೂ ಜಾತಿಯಿಂದಲೇ ಇದನ್ನು ನಿಷ್ಕರ್ಷಿಸುತ್ತಾರೆ ಎಂದು ಹೇಳುವುದಿಲ್ಲ. ನಾಲ್ಕನೆ ಅಧ್ಯಾಯದಲ್ಲಿ ಹದಿಮೂರನೆ ಶ್ಲೋಕದಲ್ಲಿ ವಿವರಿಸುವಾಗಲೇ ಶ‍್ರೀಕೃಷ್ಣ ಅವರ ಗುಣ ಮತ್ತು ಕರ್ಮಕ್ಕೆ ತಕ್ಕಂತೆ ಅವರ ವರ್ಣವನ್ನು ನಾನು ಸೃಷ್ಟಿಸಿದ್ದೇನೆ ಎನ್ನುವನು. ಆದಕಾರಣ ನಾವು ಈಗ ಯಾವ ವರ್ಣದಲ್ಲಿ ಹುಟ್ಟಿರುವೆವು ಎಂಬುದಕ್ಕೆ ಹೆಮ್ಮೆ ಪಡುವುದಕ್ಕಿಂತ ಹೆಚ್ಚಾಗಿ ಯಾವ ವರ್ಣದ ಕಸುಬನ್ನು ನಾನು ತೆಗೆದುಕೊಂಡಿರುವೆ, ಅದಕ್ಕೆ ಏನೇನನ್ನು ನಾನು ಅಭ್ಯಾಸ ಮಾಡಬೇಕು ಎಂಬುದನ್ನು ತಿಳಿದುಕೊಳ್ಳ ಬೇಕು. ಕರ್ಮಯೋಗದ ದೃಷ್ಟಿಯಿಂದ ನೋಡಿದರೆ, ಈ ವರ್ಣದಲ್ಲಿ ಯಾವುದೊಂದೂ ಮೇಲಲ್ಲ, ಯಾವುದೊಂದೂ ಕೀಳಲ್ಲ. ಪ್ರತಿಯೊಬ್ಬನೂ ತನ್ನ ಕೆಲಸವನ್ನು ತಾನು ಸರಿಯಾಗಿ ಮಾಡಿದರೆ, ಒಬ್ಬನಿಗೆ ಜಾಸ್ತಿ, ಮತ್ತೊಬ್ಬನಿಗೆ ಕಡಿಮೆ ಒಳ್ಳೆಯದಾಗುವುದಿಲ್ಲ. ಎಲ್ಲರಿಗೂ ಒಂದೇ ಸಮನಾಗಿ ಒಳ್ಳೆಯದಾಗುವುದು. ಇಲ್ಲಿ ನಾವು ಏನು ಮಾಡುತ್ತೇವೆ ಅದಲ್ಲ ಮುಖ್ಯ, ಅದನ್ನು ಯಾವ ದೃಷ್ಟಿಯಿಂದ ಮಾಡುತ್ತೇವೆ ಎಂಬುದು ಅತಿ ಮುಖ್ಯ.

\begin{verse}
ಶಮೋ ದಮಸ್ತಪಃ ಶೌಚಂ ಕ್ಷಾಂತಿರಾರ್ಜವಮೇವ ಚ~।\\ಜ್ಞಾನಂ ವಿಜ್ಞಾನಮಾಸ್ತಿಕ್ಯಂ ಬ್ರಹ್ಮಕರ್ಮ ಸ್ವಭಾವಜಮ್ \versenum{॥ ೪೨~॥}
\end{verse}

{\small ಶಮ ದಮ ತಪಸ್ಸು ಶೌಚ ಕ್ಷಮೆ ಆರ್ಜವ ಜ್ಞಾನ ವಿಜ್ಞಾನ ಮತ್ತು ಆಸ್ತಿಕ ಭಾವ--ಇವು ಸ್ವಭಾವತಃ ಇರುವ ಬ್ರಾಹ್ಮಣನ ಕರ್ಮ.}

ಬ್ರಾಹ್ಮಣನ ಸ್ವಭಾವದ ಗುಣಗಳನ್ನು ಹೇಳುತ್ತಾನೆ. ಇವನಲ್ಲಿ ಶಮ ಇರಬೇಕು. ಮನಸ್ಸನ್ನು ಅವನು ನಿಗ್ರಹಿಸಬೇಕು. ಅದು ಇಚ್ಛೆ ಪಟ್ಟ ಕಡೆ ಅಲೆದಾಡುವುದಕ್ಕೆ ಅವಕಾಶ ಕೊಡಕೂಡದು. ತಾನು ಯಾವುದರಮೇಲೆ ಮನಸ್ಸನ್ನು ಹಾಕುತ್ತಾನೆಯೋ ಅದನ್ನು ಮಾತ್ರ ಕುರಿತು ಚಿಂತಿಸಬೇಕು. ಮನಸ್ಸು ಸಂಪೂರ್ಣವಾಗಿ ಅವನ ಹತೋಟಿಯಲ್ಲಿರುವುದು. ದಮ ಎಂದರೆ ಸ್ಥೂಲವಾಗಿ ಇಂದ್ರಿಯಗಳನ್ನು ಆಯಾ ವಸ್ತುಗಳ ಕಡೆ ಹೋಗದಂತೆ ತಡೆಗಟ್ಟುವುದು. ಅವನಲ್ಲಿ ಶೌಚವಿರಬೇಕು. ಮೊದಲು ಬಾಹ್ಯ ಶೌಚವಿರಬೇಕು. ಅವನು ಉಡುವ ಬಟ್ಟೆ ಶುಚಿಯಾಗಿರಬೇಕು, ತಿನ್ನುವ ಆಹಾರ ಶುಚಿಯಾಗಿರಬೇಕು, ದೇಹವನ್ನು ಸ್ನಾನಾದಿಗಳಿಂದ ಶುಚಿಯಾಗಿಟ್ಟಿರಬೇಕು. ಅದರಂತೆಯೇ ಮನಸ್ಸನ್ನು ಶುಚಿಯಾಗಿಟ್ಟಿರು ವುದು ಬಾಹ್ಯಶೌಚಕ್ಕಿಂತ ಮುಖ್ಯ. ಮನಸ್ಸನ್ನು ಕೊಳೆ ಮಾಡುವ ಭಾವನೆಗಳು ಬರದಂತೆ ನೋಡಿಕೊಳ್ಳ ಬೇಕು. ಅಸೂಯೆ ಇರಕೂಡದು, ಸುಳ್ಳು ಹೇಳಕೂಡದು, ಶಾಂತನಾಗಿರಬೇಕು.

ಜೀವನದಲ್ಲಿ ತನಗೆ ಯಾರು ಎಂತಹ ಕೆಟ್ಟದ್ದನ್ನು ಮಾಡಿದರೂ ಕ್ಷಮಿಸುವ ಸ್ವಭಾವ ನಮ್ಮದಾಗಿರ ಬೇಕು. ತಿಳಿಯದೆ ಇರುವಾಗ, ಅಜ್ಞಾನದಲ್ಲಿ ಯಾರೊ ನಮಗೆ ವ್ಯಥೆ ಕೊಟ್ಟಿರಬಹುದು. ಅದನ್ನೇ ಕುರಿತು ಚಿಂತಿಸುತ್ತಿದ್ದರೆ ನಮ್ಮ ಮನಸ್ಸೆಲ್ಲ ಕೊಳೆಯಿಂದ ತುಂಬುವುದು. ನಾವು ಇದಕ್ಕೆ ಬದ್ಧರಾಗು ತ್ತೇವೆ. ಇದನ್ನು ಮೀರಿ ಹೋಗುವುದಿಲ್ಲ. ಯಾವಾಗ ಒಮ್ಮೆ ಕ್ಷಮಿಸುತ್ತಾನೆಯೋ, ಇವನು ಅದರಿಂದ ಪಾರಾಗುತ್ತಾನೆ. ಎರಡನೆಯದಾಗಿ ಯಾರು ಇವನಿಗೆ ಅನ್ಯಾಯ ಮಾಡಿದರೂ ಅವನಲ್ಲಿ ಪಶ್ಚಾತ್ತಾಪ ಬರುವಂತೆ ಮಾಡುವುದು. ಪೆಟ್ಟಿಗೆ ಪೆಟ್ಟು ಬುದ್ಧಿ ಕಲಿಸುವುದು ಎಂದು ನಾವು ಭಾವಿಸುತ್ತೇವೆ. ಎಂದಿಗೂ ಇಲ್ಲ. ಅದನ್ನು ಹಿಂತಿರುಗಿ ಬಡ್ಡಿ ಸಹಿತ ಕೊಡಲು ಅವನು ಹೊಂಚುತ್ತಿರುವನು. ಅದನ್ನು ನಿಲ್ಲಿಸಬೇಕಾದರೆ ಯಾರಾದರೂ ಕ್ಷಮಿಸಬೇಕು. ಆಗ ಮಾತ್ರ ದ್ವೇಷದ ಉರಿ ತಗ್ಗುವುದು. ಒಬ್ಬ ಸ್ವಭಾವತಃ ಬ್ರಾಹ್ಮಣನಾದರೆ, ಅವನು ಕ್ಷಮಿಸದ ಯಾವ ಮಹಾಪರಾಧವೂ ಇಲ್ಲ.

ಅವನಲ್ಲಿ ಆರ್ಜವದ ಗುಣವನ್ನು ನೋಡುತ್ತೇವೆ. ಅವನಲ್ಲಿ ಒಳಗೊಂದು ಹೊರಗೊಂದು ಇಲ್ಲ. ಅವನು ಖಂಡಿತವಾದಿ. ಬಚ್ಚಿಡುವುದಕ್ಕೆ ಹೋಗುವುದಿಲ್ಲ. ಅವನೇನೊ ಅದನ್ನು ಹೊರಗೆಯೇ ತೋರಿಸುಕೊಳ್ಳುವನು. ಅವನು ಜ್ಞಾನಿ. ಪ್ರಪಂಚದಲ್ಲಿ ಯಾವುದು ಸತ್ಯ ಮತ್ತು ಯಾವುದು ಮಿಥ್ಯ ಎಂಬುದನ್ನು ಚೆನ್ನಾಗಿ ವಿಚಾರ ಮಾಡಿ ತಿಳಿದುಕೊಂಡಿರುವನು. ಜೊತೆಗೆ ಅವನು ವಿಜ್ಞಾನಿಯೂ ಆಗಿರುವನು. ತಾನು ಏನನ್ನು ಸರಿ ಎಂದು ತಿಳಿದುಕೊಂಡಿರುವನೊ ಅದನ್ನು ಜೀವನದಲ್ಲಿ ಅನುಷ್ಠಾನಕ್ಕೆ ತಂದಿರುವನು, ಅದನ್ನು ಆನುಭವಿಸಿರುವನು. ಜ್ಞಾನ ಸಿದ್ಧಾಂತ, ವಿಜ್ಞಾನ ಅನುಭವ. ಅವನಲ್ಲಿ ಇವುಗಳೆರಡೂ ಸಮತೂಕದಲ್ಲಿವೆ.

ಎಲ್ಲಕ್ಕಿಂತ ಹೆಚ್ಚಾಗಿ ಅವನಲ್ಲಿ ಆಸ್ತಿಕ ಭಾವ ಇದೆ. ಈ ಪ್ರಪಂಚದ ಹಿಂದೆಲ್ಲ ಎಲ್ಲವನ್ನು ನಡೆಸುತ್ತಿರುವ ಪರಮಾತ್ಮನೊಬ್ಬನು ಇರುವನು. ಅವನಿಚ್ಛೆಯಂತೆ ಎಲ್ಲವೂ ಸಾಗುತ್ತಿದೆ. ಉರುಳುತ್ತಿರುವ ಚಕ್ರಕ್ಕೆ ಮಧ್ಯದ ನಾಭಿ ಹೇಗೊ ಹಾಗೆ ಪರಮಾತ್ಮ ಸೃಷ್ಟಿ ಚಕ್ರಕ್ಕೆ. ಅವನ ಸುತ್ತಲೂ ಸುತ್ತುತ್ತಿದೆ ಇದು. ಅವನು ಸರ್ವಾಂತರ್ಯಾಮಿ, ಸರ್ವಜ್ಞ, ಸರ್ವಶಕ್ತ, ಆಹೇತುಕ ಕೃಪಾಸಿಂಧು, ತನ್ನನ್ನು ನೆಚ್ಚಿದವರನ್ನು ಎಂದಿಗೂ ಕೈ ಬಿಡದವನು. ಇಂತಹ ಭಗವಂತನೆಂಬ ಧೃವತಾರೆ ಸದಾ ಇವನ ಮಾರ್ಗದರ್ಶಿ.

\begin{verse}
ಶೌರ್ಯಂ ತೇಜೋ ಧೃತಿರ್ದಾಕ್ಷ್ಯಂ ಯುದ್ಧೇ ಚಾಪ್ಯಪಲಾಯನಮ್~।\\ದಾನಮೀಶ್ವರಭಾವಶ್ಚ ಕ್ಷಾತ್ರಂ ಕರ್ಮ ಸ್ವಭಾವಜಮ್ \versenum{॥ ೪೩~॥}
\end{verse}

{\small ಶೌರ್ಯ, ತೇಜಸ್ಸು ಧೈರ್ಯ ಕಾರ್ಯದಕ್ಷತೆ, ಯುದ್ಧದಲ್ಲಿ ಹಿಂತೆಗೆಯದೆ ಇರುವುದು, ದಾನ, ಈಶ್ವರಭಾವ–ಇವು ಸ್ವಭಾವದಿಂದ ಉಂಟಾದ ಕ್ಷತ್ರಿಯನ ಕರ್ಮ.}

ಕ್ಷತ್ರಿಯ ಸಮರಾಂಗಣದಲ್ಲಿ ವೀರ. ಸಾವು, ಸೋಲು, ನಷ್ಟ ಯಾವುದಕ್ಕೂ ಅಂಜದವನು. ಎದೆಯನ್ನು ತೋರಿಸುವುದು ಕ್ಷತ್ರಿಯನ ಸ್ವಭಾವ. ಎಂದಿಗೂ ಅವನು ಬೆನ್ನು ತೋರಿಸುವುದಿಲ್ಲ. ಕ್ಷಾತ್ರ ತೇಜಸ್ಸು ಅವನಲ್ಲಿ ಹೊಳೆಯುತ್ತಿರುವುದು. ಮಾತನ್ನೆ ಆಡಬೇಕಾಗಿಲ್ಲ, ಆ ತೇಜಸ್ಸೆ ಇವನಾರು ಎಂಬುದನ್ನು ವ್ಯಕ್ತಪಡಿಸುವುದು. ಇದು ಮಾತಿಗೆ ನಿಲುಕದ ಪದ, ಆದರೆ ಎಲ್ಲರಿಗೂ ಅನುಭವ ವೇದ್ಯ. ಅವನಲ್ಲಿ ಸದಾ ಧೈರ್ಯ ತುಂಬಿ ತುಳುಕುತ್ತಿದೆ. ಅವನು ಜೀವನದ ಎಂತಹ ಪ್ರಸಂಗ ದಲ್ಲಿಯೂ ಎದೆಗೆಡುವುದಿಲ್ಲ. ಎಲ್ಲವನ್ನು ದಿಟ್ಟತನದಿಂದ ಎದುರಿಸುವನು. ಧೈರ್ಯ ಒಂದು ಪಂಜಿನಂತೆ ಅವನಲ್ಲಿ ಉರಿಯುತ್ತಿರುವುದು. ಅವನ ಸಾನ್ನಿಧ್ಯವೇ ಅಧೈರ್ಯವನ್ನು ದೂಡುವುದು. ಅವನು ಕಾರ್ಯದಕ್ಷ. ಒಂದು ಕೆಲಸವನ್ನು ತೆಗೆದುಕೊಂಡರೆ ಅದನ್ನು ಚೆನ್ನಾಗಿ ನಿರ್ವಹಿಸುವನು. ಒಂದು ಮಹಾ ಕೆಲಸವನ್ನು ಸಾಧಿಸಬೇಕಾದರೆ ಹಲವರ ಸಹಕಾರ ಬೇಕಾಗಿದೆ. ಹಲವು ಸ್ವಭಾವದ ಜನರಿರು ತ್ತಾರೆ. ಒಬ್ಬೊಬ್ಬರಲ್ಲಿ ಒಂದೊಂದು ಒಳ್ಳೆಯದು, ಒಂದೊಂದು ಕೆಟ್ಟದ್ದು ಇದೆ. ಇಂತಹ ಜನರೆಲ್ಲರ ಸಹಕಾರದಿಂದ ಅವನು ಕೆಲಸವನ್ನು ಮಾಡುವನು. ಬಗೆಬಗೆಯ ಜನರನ್ನು ಒಲಿಸಿಕೊಳ್ಳು ವುದು ಅವನಿಗೆ ಗೊತ್ತಿದೆ. ಅವರಲ್ಲಿರುವ ಅತ್ಯಂತ ಉತ್ತಮವಾದ ಗುಣ ವ್ಯಕ್ತವಾಗುವುದಕ್ಕೆ ಅವನು ಅವಕಾಶ ಕೊಡುವನು. ಕೆಲಸವನ್ನು ಮಾಡುತ್ತಿರುವಾಗಲೂ ಹಲವಾರು ಅಡ್ಡಿ ಆತಂಕಗಳು ಬರುವುವು. ಅವುಗಳನ್ನೆಲ್ಲ ಎದುರಿಸಿ ತನ್ನ ಗುರಿಯನ್ನು ಸಾಧಿಸುವ ಛಲ ಅವನಲ್ಲಿದೆ. ಅವನು ಎಂದಿಗೂ ಯುದ್ಧದಲ್ಲಿ ಬೆನ್ನು ತೋರುವವನಲ್ಲ. ಮಡಿಯುವನು ಇಲ್ಲವೆ ಜಯಿಸುವನು.

ಔದಾರ್ಯತೆ ಅವನ ಛಾಯೆಯಂತಿರುವುದು. ಎಲ್ಲಾ ವಿದ್ಯೆಗಳಿಗೂ ಅವನು ಪ್ರೋತ್ಸಾಹವನ್ನು ನೀಡುವನು. ಅವನನ್ನು ಬೇಡುವುದಕ್ಕೆ ಬಂದ ಮನುಷ್ಯ ನಿರಾಶನಾಗಿ ಹೋಗುವುದಿಲ್ಲ. ತನ್ನಲ್ಲಿರುವ ತನಕ ದಾನ ಮಾಡುವನು. ಹಾಗೆ ಕೊಡುವಾಗ ತನ್ನ ಗೌರವಕ್ಕೆ ತಕ್ಕಂತೆ ಕೊಡುವನು.

ಅವನಲ್ಲಿ ಈಶ್ವರೀ ಭಾವ ಇರುವುದು. ಆಳುವುದಕ್ಕೆ ಅವನು ಹುಟ್ಟಿರುವನು. ಅವನು ಎಲ್ಲಿ ಹೋದರೂ ಆ ಸ್ವಭಾವವನ್ನು ನೋಡುತ್ತೇವೆ. ಆಳುವವನ ಗುಣಗಳೆಲ್ಲ ಅವನಲ್ಲಿರುವುವು. ಅವನು ನೀಡುವ ಆಜ್ಞೆಯನ್ನು ಮರುಮಾತಿಲ್ಲದೆ ಇತರರು ನಿರ್ವಹಿಸುವರು. ಏಕೆಂದರೆ ಅವನ ವ್ಯಕ್ತಿತ್ವದಲ್ಲಿ ಅಂತಹ ಶಕ್ತಿ ಇದೆ, ಮಾತಿನಲ್ಲಿ ಅನುಭವ ಇದೆ.

\begin{verse}
ಕೃಷಿಗೌರಕ್ಷ್ಯವಾಣಿಜ್ಯಂ ವೈಶ್ಯಕರ್ಮ ಸ್ವಭಾವಜಮ್~।\\ಪರಿಚರ್ಯಾತ್ಮಕಂ ಕರ್ಮ ಶೂದ್ರಸ್ಯಾಪಿ ಸ್ವಭಾವಜಮ್ \versenum{॥ ೪೪~॥}
\end{verse}

{\small ಕೃಷಿ ಗೋರಕ್ಷಣೆ ಮತ್ತು ವಾಣಿಜ್ಯ ಇವು ಸ್ವಭಾವತಃ ಬಂದ ವೈಶ್ಯನ ಗುಣಗಳು. ಪರಿಚಾರಿಕೆ ಶೂದ್ರನಿಗೆ ಸ್ವಭಾವತಃ ಬರುವುದು.}

ವೈಶ್ಯನ ಕರ್ಮವೇ ಬೇಸಾಯ, ಗೋರಕ್ಷಣೆ ಮತ್ತು ವ್ಯಾಪಾರ. ಅವನು ವಸ್ತುವನ್ನು ಉತ್ಪತ್ತಿ ಮಾಡಬೇಕು ಮತ್ತು ಹಂಚಬೇಕು. ಉತ್ಪತ್ತಿ ಮಾಡುವಾಗ ಆಗಿನ ಕಾಲದಲ್ಲಿ ಬೇಸಾಯ ಒಂದೇ ಇದ್ದದ್ದು. ಆದರೆ ನಾವು ಈಗಿನ ದೃಷ್ಟಿಯಿಂದ ನೋಡಿದರೆ, ಬರೀ ಹೊಲ ಗದ್ದೆಯಲ್ಲಿ ಏನು ತಯಾರಾಗುತ್ತಿದೆಯೊ ಅದೇ ಸಾಲದು. ನಮ್ಮ ಜೀವನಕ್ಕೆ ಹಲವಾರು ಫ್ಯಾಕ್ಟರಿಗಳಲ್ಲಿ ತಯಾರಾಗು ತ್ತಿರುವ ವಸ್ತುಗಳೆಲ್ಲ ಬೇಕು. ಈಗಿನ ದೃಷ್ಟಿಯಿಂದ ನೋಡಿದರೆ ಅದರ ಅರ್ಥವನ್ನು ವಿಕಾಸ ಮಾಡಬೇಕಾಗಿದೆ. ಅದರಂತೆಯೇ ಗೋರಕ್ಷಣೆ. ಇದರಿಂದಲೇ ನಮಗೆ ಆವಶ್ಯಕವಾದ ಹಾಲು ದೊರಕಬೇಕಾದರೆ. ವಾಣಿಜ್ಯವೇ ವ್ಯಾಪಾರ. ಇರುವ ಕಡೆಯಿಂದ ಇಲ್ಲದ ಕಡೆ ಹಂಚುವುದು, ತಯಾರು ಮಾಡುವ ಕಡೆಯಿಂದ, ಅದನ್ನು ಉಪಯೋಗಿಸುವ ಜನಗಳಿಗೆ ಸಿಕ್ಕುವಂತೆ ಮಾಡುವುದು, ಮತ್ತು ಲೇವಾದೇವಿ ಇವುಗಳೆಲ್ಲ ಸೇರಿವೆ.

ನಾಲ್ಕನೆ ವರ್ಗದವರೇ ಶೂದ್ರರು, ಎಂದರೆ ಶ್ರಮಜೀವಿಗಳು. ಇವರ ಕೆಲಸವೇ ಪರಿಚಾರಕರದು. ಹೊಲದಲ್ಲಿ, ಗದ್ದೆಯಲ್ಲಿ, ಫ್ಯಾಕ್ಟರಿಗಳಲ್ಲಿ, ನಗರಗಳಲ್ಲಿ ಅತ್ಯಂತ ಆವಶ್ಯಕವಾದ ಕೆಲಸಗಳನ್ನು ಅವರು ಮಾಡುತ್ತಿರುವರು. ಅವರ ದುಡಿತದಿಂದಲೇ ಆಹಾರ ಉತ್ಪತ್ತಿಯಾಗುವುದು, ಯಂತ್ರ ಸಾಮಗ್ರಿಗಳು ಉತ್ಪತ್ತಿಯಾಗುವುವು. ನಗರಗಳು ನಿರ್ಮಲವಾಗಿರುವುವು. ಕೂಲಿ, ಗಾಡಿ ಹೊಡೆಯುವವನು, ಆಳು ಇವರುಗಳು ಸಮಾಜದ ಕೆಳಮಟ್ಟದಲ್ಲಿರಬಹುದು. ಆದರೆ ಇವರ ಶ್ರಮದ ಆಧಾರದ ಮೇಲೆ ಉಳಿದವರೆಲ್ಲ ಇರುವರು. ಇಲ್ಲಿ ಒಂದು ವರ್ಣ ಮುಖ್ಯ, ಮತ್ತೊಂದು ವರ್ಣ ಅಮುಖ್ಯ ಎಂದು ನೋಡುವ ಪ್ರಶ್ನೆಯೇ ಏಳುವುದಿಲ್ಲ. ಒಂದರಷ್ಟೇ ಮತ್ತೊಂದು ಮುಖ್ಯ. ವರ್ಣಗಳಲ್ಲಿ ಅನ್ಯೋನ್ಯ ಆಶ್ರಯವನ್ನು ನೋಡುತ್ತೇವೆ. ಒಂದಿದ್ದರೆ ಮತ್ತೊಂದು ಸರಿಯಾಗಿ ಕೆಲಸ ಮಾಡಬೇಕಾದರೆ. ಗಡಿಯಾರದಲ್ಲಿರುವ ಚಕ್ರಗಳಂತೆ. ಯಾವ ಸಣ್ಣ ಚಕ್ರ ನಿಂತರೂ ಇಡೀ ಗಡಿಯಾರ ನಿಲ್ಲುವುದು. ಸಮಾಜಯಂತ್ರವೂ ಹಾಗೆಯೆ. ಒಂದು ದೊಡ್ಡ ನಗರದಲ್ಲಿ ಕಸಗುಡಿಸುವವರೆಲ್ಲ ಎರಡು ದಿನ ಕೆಲಸ ಮಾಡುವುದನ್ನು ನಿಲ್ಲಿಸಿದರೆ, ನರಕಸದೃಶವಾಗುವುದು. ಆಗ ಗೊತ್ತಾಗುವುದು ಅವರ ಬೆಲೆ.

\begin{verse}
ಸ್ವೇ ಸ್ವೇ ಕರ್ಮಣ್ಯಭಿರತಃ ಸಂಸಿದ್ಧಿಂ ಲಭತೇ ನರಃ~।\\ಸ್ವಕರ್ಮನಿರತಃ ಸಿದ್ಧಿಂ ಯಥಾ ವಿಂದತಿ ತಚ್ಛೃಣು \versenum{॥ ೪೫~॥}
\end{verse}

{\small ಸ್ವಕರ್ಮದಲ್ಲಿ ನಿರತನಾದ ಮನುಷ್ಯ, ಮೋಕ್ಷವನ್ನು ಹೊಂದುತ್ತಾನೆ. ಸ್ವಕರ್ಮದಲ್ಲಿ ನಿರತನಾದವನು ಹೇಗೆ ಮೋಕ್ಷವನ್ನು ಪಡೆಯುತ್ತಾನೋ ಕೇಳು.}

ಪ್ರತಿಯೊಬ್ಬರೂ ತಮ್ಮ ಸ್ವಕರ್ಮದ ಮೂಲಕವೇ ಮುಕ್ತಿಯನ್ನು ಪಡೆಯುತ್ತಾರೆ. ಒಬ್ಬ ತಾನು ಮಾಡುವ ಕೆಲಸವನ್ನು ಬಿಟ್ಟು ಇನ್ನೊಬ್ಬನ ಕೆಲಸಕ್ಕೆ ಕೈಹಾಕಬೇಕಾಗಿಲ್ಲ. ತಾನು ಮಾಡುತ್ತಿರುವುದನ್ನೇ ಫಲಾಪೇಕ್ಷೆಯಿಲ್ಲದೆ, ಅನಾಸಕ್ತನಾಗಿ ಮಾಡಿದರೆ ಸಾಕು. ಯಾವ ಕರ್ಮವೂ ಮೇಲಲ್ಲ, ಯಾವ ಕರ್ಮವೂ ಕೀಳಲ್ಲ. ಪ್ರತಿಯೊಂದೂ ಅದರದರ ದೃಷ್ಟಿಯಿಂದ ಮುಖ್ಯವೇ. ಎಲ್ಲರೂ ತಮ್ಮ ತಮ್ಮ ಕೆಲಸವನ್ನು ಮಾಡಿದರೇ ಒಟ್ಟು ಸಮಾಜ ಭದ್ರವಾಗಿರಬೇಕಾದರೆ. ಇದು ನಮ್ಮ ದೇಹದಲ್ಲಿರುವ ಅಗಾಂಗಗಳಂತೆ. ಮೆದುಳು ಎಷ್ಟು ಮುಖ್ಯವೋ, ಹೃದಯವೂ ಅಷ್ಟೇ ಮುಖ್ಯ. ಹೊಟ್ಟೆ ಅಷ್ಟೇ ಮುಖ್ಯ, ರಕ್ತ ಚಲನೆ ಮತ್ತು ಇತರ ಕ್ರಿಯೆಗಳು ಅಷ್ಟೇ ಮುಖ್ಯ. ಪ್ರತಿಯೊಂದು ಇರುವುದು ಒಟ್ಟು ದೇಹದ ಭದ್ರತೆಗೆ. ಅದರಲ್ಲಿ ಯಾವುದಾದರೂ ಒಂದು ದೋಷದಿಂದ ಕೂಡಿದರೂ ಒಟ್ಟಿನ ಭದ್ರತೆಗೆ ಭಂಗ ಬರುವುದು.

ಯಾವಾಗ ಒಬ್ಬ ತನ್ನ ಕರ್ತವ್ಯವನ್ನು ಫಲಾಪೇಕ್ಷೆಯಿಲ್ಲದೆ ಮಾಡುವನೊ ಆಗ ಅದು ಅವನ ಚಿತ್ತವನ್ನು ಶುದ್ಧಿ ಮಾಡುವುದು. ಇಲ್ಲಿ ಫಲಾಪೇಕ್ಷೆ ಇಲ್ಲದೆ ಮಾಡುವುದು ಎಂಬುದರ ರಹಸ್ಯವನ್ನು ತಿಳಿದುಕೊಳ್ಳಬೇಕು. ಫಲವೇ ಬೇಡ ಎಂದಲ್ಲ. ಇವನಿಗೆ ಆ ಫಲ ಬೇಕಾಗಿಲ್ಲ. ಅದನ್ನು ಭಗವಂತನಿಗೆ ಅರ್ಪಣೆ ಮಾಡುವನು. ಯಾವಾಗ ಭಗವಂತನಿಗೆ ಅರ್ಪಣೆ ಮಾಡುವ ದೃಷ್ಟಿಯಿಂದ ಕೆಲಸ ಮಾಡುವನೊ, ಆಗ ಅಚ್ಚುಕಟ್ಟಾಗಿ ಮಾಡುವನು. ಅಲ್ಲಿ ಅಸಡ್ಡೆ ಅಜಾಗರೂಕತೆ ಯಾವುದೂ ಇರಲಾರದು. ಜನರನ್ನು ಮೆಚ್ಚಿಸುವುದಕ್ಕೆ ಅವನು ಮಾಡುವುದಿಲ್ಲ, ಭಗವಂತನಿಗೆ ಅರ್ಪಿಸಲು ಅವನು ಮಾಡುವನು. ಅದರಿಂದ ಅತ್ಯಂತ ಶ್ರೇಷ್ಠವಾದ ಪ್ರತಿಫಲವೇ ಬರುವುದು. ಆದರೆ ಅವನು ಅದಕ್ಕೆ ಕೈಯೊಡ್ಡುವುದಿಲ್ಲ. ಕೈಯೊಡ್ಡಿದರೆ ಅದು ಅವನನ್ನು ಬಂಧಿಸುವುದು. ಅವನಿಗೆ ಬೇರೆ ರೀತಿ ಈ ಅನಾಸಕ್ತಿಯ ಕರ್ಮ ಸಹಾಯ ಮಾಡುವುದು. ಅದೇ ಅವನ ಚಿತ್ತವನ್ನು ಶುದ್ಧ ಮಾಡುವುದು. ಎಲ್ಲಾ ಸಾಧನೆಯಲ್ಲಿಯೂ ಚಿತ್ತ ಶುದ್ಧವಾಗುವುದು. ಆಗ ಅದರ ಮೂಲಕ ಹಿಂದಿರುವ ಪರಮಾತ್ಮನ ದರ್ಶನವಾಗುವುದು. ಸಾಕ್ಷಾತ್ಕಾರ ಎಂದರೆ ಎಲ್ಲಿಯೊ ಹೊರಗಡೆ ದೇವರು ಬಂದು ನಿಂತುಕೊಳ್ಳುವು ದಲ್ಲ. ದೇವರು ಸರ್ವಾಂತರ್ಯಾಮಿಯಂತೆ ಆಗಲೆ ಎಲ್ಲರಲ್ಲಿಯೂ ಇರುವನು. ಸಾಕ್ಷಾತ್ಕಾರ ಎಂದರೆ ಒಳಗಿರುವುದು ಮೇಲೆದ್ದು ಬರುವುದು. ಮೊಸರಿನಲ್ಲಿ ಬೆಣ್ಣೆ ಇದೆ. ಕಡೆದಾಗ ಮೇಲೆದ್ದು ಬರುವುದು. ಹೀಗೆ ಕಡೆಯುವುದನ್ನೇ ಅನಾಸಕ್ತಿಯೋಗ ಎಂದು ಹೇಳುವುದು.

\begin{verse}
ಯತಃ ಪ್ರವೃತ್ತಿರ್ಭೂತಾನಾಂ ಯೇನ ಸರ್ವಮಿದಂ ತತಮ್​।\\ಸ್ವಕರ್ಮಣಾ ತಮಭ್ಯರ್ಚ್ಯ ಸಿದ್ಧಿಂ ವಿಂದತಿ ಮಾನವಃ \versenum{॥ ೪೬~॥}
\end{verse}

{\small ಯಾರಿಂದ ಪ್ರಾಣಿಗಳೆಲ್ಲವೂ ಬಂದಿವೆಯೊ, ಯಾರಿಂದ ಎಲ್ಲವೂ ವ್ಯಾಪ್ತವಾಗಿರುವುದೊ, ಅವನನ್ನು ಸ್ವ ಕರ್ಮದಿಂದ ಪೂಜಿಸುವ ಪುರುಷ ಮೋಕ್ಷವನ್ನು ಪಡೆಯುತ್ತಾನೆ.}

ನಮ್ಮ ಎದುರಿಗೆ ಇರುವ ವಸ್ತುಗಳೆಲ್ಲ ಭಗವಂತನಿಂದ ಬಂದಿವೆ. ಅಗ್ನಿಕುಂಡದಿಂದ ಕಿಡಿಗಳು ಹೇಗೆ ಏಳುತ್ತವೆಯೋ, ಜೇಡ ತನ್ನ ದೇಹದಿಂದ ತಂತುವನ್ನು ಹೇಗೆ ತರುವುದೋ ಹಾಗೆ ಭಗವಂತ ನಿಂದ ಈ ಬ್ರಹ್ಮಾಂಡವೆಲ್ಲವೂ ಬಂದಿದೆ. ಬಂದಾದ ಮೇಲೆ ಅವನು ಇವುಗಳನ್ನೆಲ್ಲ ವ್ಯಾಪಿಸಿ ಕೊಂಡಿರುವನು. ಇವುಗಳಿಂದ ಬೇರೆ ಇಲ್ಲ. ಆಕಾಶ ಹೇಗೆ ಎಲ್ಲದರ ಒಳಗೆ ಮತ್ತು ಹೊರಗೆ ಇದೆಯೋ ಹಾಗೆ ಭಗವಂತ ಸರ್ವವನ್ನೂ ವ್ಯಾಪಿಸಿಕೊಂಡಿರುವನು. ಅವನು ಸರ್ವವನ್ನು ವ್ಯಾಪಿಸಿ ಕೊಂಡಿದ್ದರೆ ನಮ್ಮ ಕಣ್ಣಿಗೆ ಕಾಣಬೇಕಲ್ಲ, ಏತಕ್ಕೆ ಕಾಣುತ್ತಿಲ್ಲ?? ಅವನಲ್ಲದ ವಸ್ತುಗಳೇ ಕಾಣುತ್ತವೆ ನಮ್ಮ ಎದುರಿಗೆ. ಅವನಲ್ಲದೆ ಇರುವ ವಸ್ತುಗಳೆಲ್ಲ ತೆರೆಗಳು, ಆ ತೆರೆ ನಾಮರೂಪದಿಂದ ಆಗಿರು ವುದು. ಯಾವಾಗ ಅದನ್ನು ಸರಿಸುತ್ತೇವೆಯೋ ಅದರ ಹಿಂದೆ ಅವನೇ ಇರುವನು. ನಮ್ಮ ಅಜ್ಞಾನದಿಂದ ಎದುರಿಗೆ ಇರುವ ನಾಮರೂಪವನ್ನು ಮಾತ್ರ ನೋಡುತ್ತಿದ್ದೇವೆ. ನಮ್ಮ ಅಜ್ಞಾನ ಯಾವಾಗ ಹೋಗುವುದೊ ಆಗ ಎದುರಿಗೆ ಇರುವ ವಸ್ತುವಿನ ಹಿಂದೆಲ್ಲ ಅವನ ಇರುವು ವೇದ್ಯವಾಗುವುದು.

ಆ ಪರಮಾತ್ಮನನ್ನು ಸ್ವಕರ್ಮದಿಂದ ಆರಾಧಿಸಿ ಮುಕ್ತಿಯನ್ನು ಪಡೆಯುತ್ತಾನೆ. ಇದೊಂದು ಅತ್ಯಂತ ಸುಂದರವಾದ ಭಾವನೆ. ನಾವು ದೇವರನ್ನು ಬಿಲ್ವ, ತುಳಸಿ ಮೊದಲಾದುವುಗಳಿಂದ ಅರ್ಚನೆ ಮಾಡುತ್ತೇವೆ. ಆದರೆ ದೇವರಿಗೆ ಅತ್ಯಂತ ಪ್ರಿಯವಾದ ಅರ್ಚನೆಯೇ ನಾವು ಯಾವ ಸ್ಥಿತಿಯಲ್ಲಿ ದ್ದೇವೆಯೊ, ನಮ್ಮ ಪಾಲಿಗೆ ಯಾವ ಕರ್ಮ ಬಂದಿದೆಯೊ ಅದನ್ನು ಇದು ಭಗವದರ್ಪಿತವಾಗಲಿ ಎಂದು ಪೂಜೆಯಂತೆ ಮಾಡುವುದು. ನಾವು ಮಾಡುವ ಕರ್ಮವನ್ನು ಒಂದು ಪರಮ ಪವಿತ್ರವಾದ ಪೂಜಾಸ್ಥಾನಕ್ಕೆ ಏರಿಸುವನು ಶ‍್ರೀಕೃಷ್ಣ. ಒಬ್ಬೊಬ್ಬ ಒಂದೊಂದು ರೀತಿ ಭಗವಂತನನ್ನು ಪೂಜೆ ಮಾಡಬಹುದು ಎನ್ನುತ್ತಾನೆ. ಬ್ರಾಹ್ಮಣ ತನ್ನ ತಪಸ್ಸು ಮತ್ತು ಜ್ಞಾನದಿಂದ ಪೂಜೆ ಮಾಡುತ್ತಾನೆ. ಕ್ಷತ್ರಿಯ ತನ್ನ ಶೌರ್ಯ ತೇಜಸ್ಸು ಪರಾಕ್ರಮಗಳಿಂದ ಪೂಜಿಸುತ್ತಾನೆ. ವೈಶ್ಯ ಹಂಚುವುದರ ಮೂಲಕ ಅವನನ್ನು ಪೂಜಿಸುತ್ತಾನೆ. ಶೂದ್ರ ತನ್ನ ಶ್ರಮದ ದುಡಿತದಿಂದ ಅವನನ್ನು ಪೂಜಿಸುತ್ತಾನೆ. ಪೂಜಾದೃಷ್ಟಿಯಿಂದ ಮಾಡಿದರೆ, ಎಲ್ಲರ ಮನಸ್ಸಿನ ಮೇಲೆಯೂ ಒಂದೇ ಪರಿಣಾಮವನ್ನು ಬಿಡುವುದು. ಅದೇ ಅವರ ಚಿತ್ತ ಶುದ್ಧವಾಗುವುದು. ಒಬ್ಬನಿಗೆ ಜಾಸ್ತಿಯಾಗುವುದಿಲ್ಲ,ಮತ್ತೊಬ್ಬನಿಗೆ ಕಡಿಮೆಯಾಗುವುದಿಲ್ಲ. ಭಗವಂತನ ದೃಷ್ಟಿಯಲ್ಲಿ ಎಲ್ಲರೂ ಪೂಜೆಯನ್ನು ಮಾಡುತ್ತಿರುವರು. ಯಾವಾಗ ಪೂಜಾಭಾವನೆಯನ್ನು ಮರೆಯುತ್ತೇವೆಯೊ, ಒಬ್ಬ ಬ್ರಾಹ್ಮಣನ ಕರ್ಮವನ್ನೆ ಮಾಡು ತ್ತಿರಲಿ, ಕ್ಷತ್ರಿಯನ ಕರ್ಮವನ್ನೆ ಮಾಡುತ್ತಿರಲಿ, ವೈಶ್ಯನ ಕರ್ಮವನ್ನೆ ಮಾಡುತ್ತಿರಲಿ, ಶೂದ್ರನ ಕರ್ಮವನ್ನೆ ಮಾಡುತ್ತಿರಲಿ, ಅವನು ಬಂಧನಕ್ಕೆ ಒಳಗಾಗುವನು. ಯಾವಾಗ ಪೂಜಾಭಾವನೆಯನ್ನು ಮರೆಯುವುದಿಲ್ಲವೊ ಆಗ ಒಬ್ಬನಷ್ಟೇ ಮತ್ತೊಬ್ಬ ಶ್ರೇಷ್ಠ.

\begin{verse}
ಶ್ರೇಯಾನ್ ಸ್ವಧರ್ಮೋ ವಿಗುಣಃ ಪರಧರ್ಮಾತ್ಸ್ವನುಷ್ಠಿತಾತ್~।\\ಸ್ವಭಾವನಿಯತಂ ಕರ್ಮ ಕುರ್ವನ್ನಾಪ್ನೋತಿ ಕಿಲ್ಬಿಷಮ್ \versenum{॥ ೪೭~॥}
\end{verse}

{\small ಚೆನ್ನಾಗಿ ಮಾಡಿದ ಪರಧರ್ಮಕ್ಕಿಂತ, ಚೆನ್ನಾಗಿಲ್ಲದೆ ಇರುವ ಸ್ವಧರ್ಮವೇ ಶ್ರೇಯಸ್ಕರ. ಸ್ವಭಾವಕ್ಕೆ ಅನುರೂಪ ವಾದ ಕರ್ಮ ಮಾಡುವವನಿಗೆ ಪಾಪ ಅಂಟುವುದಿಲ್ಲ.}

ನಾವು ಮೊದಲು ಇನ್ನೊಬ್ಬನ ಕೆಲಸವನ್ನು ಚೆನ್ನಾಗಿ ಮಾಡಬಹುದು. ಆದರೆ ಹಾಗೆಯೇ ಎಂದೆಂದಿಗೂ ಮಾಡುತ್ತೇವೆ ಎಂದು ಹೇಳುವುದಕ್ಕೆ ಆಗುವುದಿಲ್ಲ. ಆ ಕೆಲಸವನ್ನು ಮಾಡುತ್ತಿರುವಾಗ ಹಲವಾರು ಸಮಸ್ಯೆಗಳು ಏಳುತ್ತವೆ. ಅದನ್ನೆಲ್ಲ ಎದುರಿಸುವುದಕ್ಕೆ ನಾನು ತರಬೇತನ್ನು ತೆಗೆದು ಕೊಂಡಿಲ್ಲ. ನಾನು ಮತ್ತೊಬ್ಬನ ಕೆಲಸವನ್ನು ಚೆನ್ನಾಗಿ ಮಾಡಬಹುದು. ಆದರೆ ನನ್ನ ಕೆಲಸವನ್ನು ಹಾಗೆಯೇ ಬಿಡುವೆನು. ಆಗ ಅದು ನಷ್ಟವಾಗುವುದು. ನಾನು ಯಾವಾಗ ಇನ್ನೊಬ್ಬನ ಕೆಲಸವನ್ನು ಮಾಡುತ್ತೇನೆಯೋ ಇನ್ನೊಬ್ಬ ಆ ಕೆಲಸವನ್ನು ಮಾಡುತ್ತಿದ್ದವನು ಮಾಡುವುದಿಲ್ಲ. ಇದರಿಂದ ದೊಡ್ಡದೊಂದು ನಷ್ಟವಾಗುವುದು. ನಾನು ಮಾಡುತ್ತಿದ್ದ ಕೆಲಸ ನಿಂತು ಹೋಯಿತು. ಇನ್ನೊಬ್ಬ ಮಾಡುತ್ತಿದ್ದ ಕೆಲಸಕ್ಕೆ ಅಡ್ಡಿ ತಂದೆ.

ಚೆನ್ನಾಗಿ ಮಾಡದೇ ಇದ್ದರೂ ಚಿಂತೆಯಿಲ್ಲ, ನನ್ನ ಸ್ವಭಾವಕ್ಕೆ ತಕ್ಕಂತಹ ಕೆಲಸವನ್ನು ಮಾಡುವುದು ಮೇಲು. ಮೊದಮೊದಲು ನಾನು ಕೆಲಸವನ್ನು ಚೆನ್ನಾಗಿ ಮಾಡದೆ ಇರಬಹುದು. ಆದರೆ ಕ್ರಮೇಣ ಅದರಲ್ಲಿ ಪಾಂಡಿತ್ಯವನ್ನು ಪಡೆಯುವೆನು. ಇದರಿಂದ ನನ್ನ ಕೆಲಸ ನಿಲ್ಲಲಿಲ್ಲ, ನಾನು ಇನ್ನೊಬ್ಬನ ಕೆಲಸಕ್ಕೆ ಆತಂಕವನ್ನು ತಂದೊಡ್ಡಲಿಲ್ಲ.

ನನ್ನ ಪಾಲಿಗೆ ಬಂದ ಕೆಲಸ ತುಂಬಾ ಕಷ್ಟವಾಗಿದೆ ಅಥವಾ ಅಪ್ರಿಯವಾಗಿದೆ. ಅದಕ್ಕೇ ನಾನು ಆ ಕೆಲಸವನ್ನು ಬಿಟ್ಟು ಪ್ರಿಯವಾದ ಮತ್ತೊಬ್ಬನ ಕೆಲಸವನ್ನು ತೆಗೆದುಕೊಳ್ಳುವುದು. ಆದರೆ ನನ್ನ ಪಾಲಿಗೆ ಬಂದ ಕರ್ತವ್ಯವನ್ನು ಸರಿಯಾಗಿ ನಿರ್ವಹಿಸಿದರೆ, ಅದು ಅಪ್ರಿಯವಾಗಬಹುದು, ಅದರಿಂದ ಇತರರಿಗೆ ಕೆಲವು ವೇಳೆ ನೋವು ಕಷ್ಟ ಮುಂತಾದುವು ಆದರೂ ಅದರಿಂದ ನನಗೆ ಏನೂ ಕೆಟ್ಟದಾಗಲಾರದು. ನಾನು ನನ್ನ ಸ್ವಾರ್ಥಕ್ಕಾಗಿ ಮತ್ತೊಬ್ಬನಿಗೆ ತೊಂದರೆ ಕೊಡಲಿಲ್ಲ. ಸಮಾಜದ ದೃಷ್ಟಿಯಿಂದ ಅದನ್ನು ಮಾಡಬೇಕಾಯಿತು. ಕೆಲವು ವೇಳೆ ಕಳ್ಳನಿಗೆ ಶಿಕ್ಷೆ ಕೊಡಬೇಕಾಗಿದೆ. ಅವನನ್ನೇ ನೆಚ್ಚಿದ್ದ ಹೆಂಡತಿ ಮಕ್ಕಳಿಗೆ ಕಷ್ಟವಾಗಬಹುದು. ಆದರೆ ನಾನು ಇದಕ್ಕಾಗಿ ಕಳ್ಳನಿಗೆ ಶಿಕ್ಷೆ ಕೊಡುವುದನ್ನು ಬಿಡುವುದಕ್ಕೆ ಆಗುವುದಿಲ್ಲ. ವೈದ್ಯ ಶಸ್ತ್ರ ಚಿಕಿತ್ಸೆ ಮಾಡುವಾಗ ರೋಗಿಗೆ ವ್ಯಥೆ ಕೊಡುತ್ತಾನೆ. ಕೆಲವು ವೇಳೆ ಮನುಷ್ಯರಲ್ಲಿರುವ ನ್ಯೂನತೆಯನ್ನು ಎತ್ತಿತೋರುವಾಗ ಅವರು ತುಂಬಾ ವ್ಯಥೆ ಪಡುವರು. ಆದರೆ ಇವೆಲ್ಲ ಸಮಾಜದಲ್ಲಿ ಮಾಡಬೇಕಾದ ಅನಿವಾರ್ಯ ಕಾರ್ಯಗಳು. ಯಾವಾಗಲೂ ಪ್ರಿಯವಾದ ಕೆಲಸಗಳೇ ನಮ್ಮ ಪಾಲಿಗೆ ಬರುವುದಿಲ್ಲ. ಕೆಲವು ವೇಳೆ ಅಪ್ರಿಯವಾದ ಕೆಲಸಗಳೂ ನಮಗೆ ಬರುವುವು. ಆದರೆ ಅದನ್ನು ಮಾಡುವಾಗ ನಮ್ಮಲ್ಲಿ ಯಾವ ಸ್ವಾರ್ಥವೂ ಇಲ್ಲದೇ ಇದ್ದರೆ ನಾವು ಅದಕ್ಕೆ ವ್ಯಥೆ ಪಡಬೇಕಾಗಿಲ್ಲ.

\begin{verse}
ಸಹಜಂ ಕರ್ಮ ಕೌಂತೇಯ ಸದೋಷಮಪಿ ನ ತ್ಯಜೇತ್~।\\ಸರ್ವಾರಂಭಾ ಹಿ ದೋಷೇಣ ಧೂಮೇನಾಗ್ನಿರಿವಾವೃತಾಃ \versenum{॥ ೪೮~॥}
\end{verse}

{\small ಅರ್ಜುನ, ಸಹಜವಾಗಿ ಪ್ರಾಪ್ತವಾದ ಕರ್ಮ ದೋಷದಿಂದ ಕೂಡಿದ್ದರೂ ಅದನ್ನು ಬಿಡಕೂಡದು. ಏಕೆಂದರೆ ಅಗ್ನಿಯೊಂದಿಗೆ ಹೊಗೆ ಇರುವಂತೆ ಸಕಲ ಕರ್ಮಗಳೊಂದಿಗೆ ದೋಷಗಳು ಇರುತ್ತವೆ.}

ನನ್ನ ಪಾಲಿಗೆ ಬಂದ ಕರ್ಮವನ್ನು ಮಾಡುವಾಗ ಕೆಲವು ದೋಷಗಳು ಇರಬಹುದು. ಆದರೆ ಅದನ್ನು ಬಿಟ್ಟು ಬೇರೆ ಕರ್ಮವನ್ನು ತೆಗೆದುಕೊಂಡರೆ, ಅಲ್ಲಿ ದೋಷವಿರುವುದಿಲ್ಲ ಎಂದು ಹೇಗೆ ಹೇಳಬಹುದು? ಒಂದು ಕೆಲಸಕ್ಕೆ ಇಷ್ಟೊಂದು ತರಬೇತನ್ನು ತೆಗೆದುಕೊಂಡು ಮಾಡುವಾಗಲೆ, ಇಷ್ಟು ದೋಷಗಳು ಇರುವಾಗ, ಯಾವ ತರಬೇತೂ ಇಲ್ಲದೆ ಇನ್ನೊಬ್ಬನ ಕೆಲಸಕ್ಕೆ ಕೈಹಾಕಿದರೆ ಅದನ್ನು ಹೇಗೆ ಚೆನ್ನಾಗಿ ಮಾಡುವುದು? ಏನೋ ಒಂದು ವೇಳೆ ಅದು ಸುಲಭವಾದಾಗ ಅದನ್ನು ಸರಿಯಾಗಿ ಮಾಡಬಹುದು. ಅನಂತರ ಜಟಿಲ ಸಮಸ್ಯೆಗಳು ಉದ್ಭವಿಸಿದಾಗ ಅದನ್ನು ಎದುರಿಸುವುದಕ್ಕೆ ನಾವು ಯೋಗ್ಯತೆಯನ್ನು ಸಂಪಾದಿಸಿಕೊಂಡಿಲ್ಲ.

ಎಲ್ಲಾ ಕರ್ಮಗಳಲ್ಲಿಯೂ ಪ್ರಾರಂಭದಲ್ಲಿ ದೋಷಗಳು ಇದ್ದೇ ಇರುತ್ತವೆ. ಬೆಂಕಿ ಉರಿಯುವು ದಕ್ಕೆ ಮುಂಚೆ ಹೊಗೆಯಾಡುವುದು. ಅನಂತರ ಚೆನ್ನಾಗಿ ಉರಿಯುವುದು. ಅದರಂತೆಯೇ ಮುಂಚೆ ಮುಂಚೆ ಕೆಲಸವನ್ನು ಮಾಡುತ್ತಿರುವಾಗ ನ್ಯೂನತೆಗಳು ಅನಿವಾರ್ಯ. ಅದು ಕ್ರಮೇಣ ಅನುಭವದಿಂದ ಕಡಿಮೆಯಾಗುವುದು. ಈಗ ತಾನೆ ಮೆಡಿಕಲ್ ಕಾಲೇಜಿನಿಂದ ಬಂದ ವೈದ್ಯ ಮೊದಮೊದಲು ಕೆಲಸ ಮಾಡುವಾಗ ಕೆಲವು ತಪ್ಪುಗಳನ್ನು ಮಾಡುತ್ತಾನೆ. ಅದಕ್ಕೆ ಅಂಜಿ ಅವನು ಆ ಕೆಲಸವನ್ನು ಬಿಟ್ಟುಬಿಟ್ಟರೆ ವೈದ್ಯಕೀಯ ವೃತ್ತಿಯಲ್ಲಿ ಮುಂದುವರಿಯುವುದಕ್ಕೇ ಆಗುವುದಿಲ್ಲ. ತಪ್ಪಿನಿಂದ ಕಲಿತುಕೊಂಡು ಕ್ರಮೇಣ ಅದನ್ನು ಮಾಡದಂತೆ ನೋಡಿಕೊಳ್ಳಬೇಕು. ಎಲ್ಲಾ ವೃತ್ತಿಯಲ್ಲಿಯೂ ಪ್ರಾರಂಭದಲ್ಲಿ ತಪ್ಪುಗಳನ್ನು ಮಾಡದವರೇ ಇಲ್ಲ. ತಪ್ಪಿನ ಮೂಲಕವಾಗಿಯೇ ನಾವು ಸರಿಯಾಗಿ ಮಾಡುವುದನ್ನು ಕಲಿಯಬೇಕು. ತಪ್ಪು ಮಾಡುವುದಕ್ಕೆ ಅಂಜಿದರೆ ನಾವು ಯಾವ ಮಾರ್ಗದಲ್ಲಿಯೂ ಮುಂದುವರಿಯುವುದಕ್ಕೆ ಆಗುವುದಿಲ್ಲ. ತಪ್ಪೇ ಜಯಕ್ಕೆ ಮೆಟ್ಟಲು. ಅಂದರೆ ನಾವು ಬೇಕೆಂದೇ ತಪ್ಪು ಮಾಡಬಾರದು. ಕೆಲಸ ಮಾಡುವಾಗ ತಪ್ಪಿದರೆ ಅದಕ್ಕಾಗಿ ಸುಮ್ಮನೆ ಕೊರಗುತ್ತಿದ್ದರೆ ಪ್ರಯೋಜನ ವಿಲ್ಲ.

\begin{verse}
ಅಸಕ್ತಬುದ್ಧಿಃ ಸರ್ವತ್ರ ಜಿತಾತ್ಮಾ ವಿಗತಸ್ಪೃಹಃ~।\\ನೈಷ್ಕರ್ಮ್ಯಸಿದ್ಧಿಂ ಪರಮಾಂ ಸಂನ್ಯಾಸೇನಾಧಿಗಚ್ಛತಿ \versenum{॥ ೪೯~॥}
\end{verse}

{\small ಯಾವುದರಲ್ಲಿಯೂ ಆಸಕ್ತಿ ಇಡದವನು, ಮನಸ್ಸನ್ನು ಜಯಿಸಿದವನು, ಆಸೆಯಿಲ್ಲದವನು, ಸಂನ್ಯಾಸದ ಮೂಲಕ ನೈಷ್ಕರ್ಮ್ಯ ಸಿದ್ಧಿಯನ್ನು ಹೊಂದುತ್ತಾನೆ.}

ಕರ್ಮವನ್ನು ಸರಿಯಾದ ರೀತಿಯಲ್ಲಿ ಮಾಡುತ್ತಾ ಹೋದರೆ ಕೊನೆಗೆ ಕರ್ಮ ನಮ್ಮಿಂದ ಬಿದ್ದುಹೋಗುವ ಸ್ಥಿತಿ ಬರುತ್ತದೆ. ಅದೇ ನೈಷ್ಕರ್ಮ್ಯ ಸಿದ್ಧಿ. ಕರ್ಮವನ್ನು ಸರಿಯಾದ ರೀತಿಯಲ್ಲಿ ಮಾಡುವುದು ಎಂದರೆ ಏನು ಎಂಬುದನ್ನು ವಿವರಿಸುವನು. ಆಸಕ್ತಿಯನ್ನು ಇಡದೆ ಇರುವುದು. ಕರ್ಮದಿಂದ ಬರುವ ಫಲದ ಮೇಲೆ ಅವನಿಗೆ ಆಸಕ್ತಿ ಇಲ್ಲ. ಕರ್ಮವನ್ನು ಮಾಡುವುದಕ್ಕೆ ಮಾತ್ರಾ ನನಗೆ ಅಧಿಕಾರ, ಅದರಿಂದ ಬರುವ ಫಲಕ್ಕೆ ಅಲ್ಲ ಎಂಬುದನ್ನು ಅವನು ಚೆನ್ನಾಗಿ ಅರಿತಿರುವನು. ಅವನು ತನ್ನ ಮನಸ್ಸನ್ನು ಚೆನ್ನಾಗಿ ನಿಗ್ರಹಿಸಿರುವನು. ಅದು ಸಿಕ್ಕಿದ ಕಡೆ ಹೋಗುವುದಕ್ಕೆ ಆಗುವುದಿಲ್ಲ. ಯಾವಾಗಲೂ ಭಗವಂತನ ಕಡೆಗೆ ಹೋಗುವಂತೆ ಮಾಡಿಕೊಂಡಿರುವನು. ಅವನು ಯಾವ ಕೆಲಸ ವನ್ನು ಮಾಡುತ್ತಿದ್ದರೂ ಮನಸ್ಸು ವಿಷಯ ವಸ್ತುಗಳನ್ನು ಬಿಟ್ಟು ಭಗವಂತನನ್ನು ಚಿಂತಿಸುತ್ತಿರು ವುದು. ಅವನು ಮನಸ್ಸಿನಲ್ಲಿ ಯಾವ ಆಸೆ ಆಕಾಂಕ್ಷೆಗಳನ್ನೂ ಇಟ್ಟುಕೊಂಡಿಲ್ಲ. ಅವನ ಬಾಳೆಲ್ಲಾ ಭಗವದಿಚ್ಛೆಗೆ ಅಧೀನ. ಅವನು ಸಂನ್ಯಾಸದ ಮೂಲಕ ಕೆಲಸವನ್ನು ಮಾಡದೇ ಇರುವ ಸ್ಥಿತಿಗೆ ಬಂದಿದ್ದಾನೆ.

ನೈಷ್ಕರ್ಮ್ಯ ಸಿದ್ಧಿಸಿದೆ ಎಂದರೆ ಅವನು ಬದುಕಿರುವ ಪರ್ಯಂತರ ಏನನ್ನೂ ಕೆಲಸ ಮಾಡದೆ ಕಳೆಯುತ್ತಾನೆಯೆ? ಇಲ್ಲ; ಕೆಲಸ ಮಾಡುತ್ತಾನೆ. ಆದರೆ ಅದರಿಂದ ತಾನು ಏನನ್ನೋ ಪಡೆಯಬೇಕೆಂ ದಲ್ಲ. ಕೆಲಸ ಮಾಡದೇ ಇದ್ದರೆ ಅವನಿಗೆ ನಷ್ಟವಿಲ್ಲ. ಕೆಲಸ ಮಾಡಿದರೆ ಅವನಿಗೆ ಲಾಭವಿಲ್ಲ. ಆದರೂ ಉಸಿರಾಡುವವರೆಗೆ ಲೋಕಕಲ್ಯಾಣಕ್ಕಾಗಿ ಯಾವುದಾದರೂ ಕೆಲಸದಲ್ಲಿ ನಿರತನಾಗಿರಬಹುದು. ನಿರತ ನಾಗಿರಲೇಬೇಕೆಂಬ ನಿಯಮವಿಲ್ಲ. ಇನ್ನು ಮೇಲೆ ಅವನ ಖುಷಿ. ಮಾಡಬೇಕೆನಿಸಿದರೆ ಮಾಡುತ್ತಾನೆ, ಇಲ್ಲದೇ ಇದ್ದರೆ ಇಲ್ಲ. ಅವನು ಎಲ್ಲಾ ಕಟ್ಟು ಕಾಯಿದೆಗಳಿಗೆ ಅತೀತ, ಶಾಸ್ತ್ರಗಳಿಗೆ ಅತೀತ. ಅವನು ಮುಕ್ತಾತ್ಮ. ಅವನನ್ನು ಬಂಧಿಸುವುದಾವುದೂ ಇಲ್ಲ.

\begin{verse}
ಸಿದ್ಧಿಂ ಪ್ರಾಪ್ತೋ ಯಥಾ ಬ್ರಹ್ಮ ತಥಾಪ್ನೋತಿ ನಿಬೋಧ ಮೇ~।\\ಸಮಾಸೇನೈವ ಕೌಂತೇಯ ನಿಷ್ಠಾ ಜ್ಞಾನಸ್ಯ ಯಾ ಪರಾ \versenum{॥ ೫ಂ~॥}
\end{verse}

{\small ಅರ್ಜುನ, ಸಿದ್ಧಿಯನ್ನು ಹೊಂದಿದವನು ಜ್ಞಾನದ ಪರಮ ನಿಷ್ಠೆಯಾದ ಬ್ರಹ್ಮವನ್ನು ಹೇಗೆ ಹೊಂದುವನು ಅದನ್ನು ಸಂಕ್ಷೇಪವಾಗಿ ನನ್ನಿಂದ ತಿಳಿದುಕೊ.}

ಇಲ್ಲಿ ಸಿದ್ಧಿಯನ್ನು ಹೊಂದಿದವನು ಎಂದರೆ ಪರಮಾತ್ಮನ ಸಾಕ್ಷಾತ್ಕಾರವನ್ನು ಪಡೆದವನು ಎಂದಲ್ಲ. ಅವನನ್ನು ಪಡೆಯುವುದಕ್ಕೆ ಬೇಕಾಗುವ ಯೋಗ್ಯತೆಯನ್ನು ಪಡೆದವನು ಎಂದರ್ಥ. ಸ್ವಕರ್ಮದಿಂದ ಈಶ್ವರನನ್ನು ಪೂಜಿಸಿ ಮನಸ್ಸನ್ನು ಅಣಿಮಾಡಿಕೊಂಡವನು.

\begin{verse}
ಬುದ್ಧ್ಯಾ ವಿಶುದ್ಧಯಾ ಯುಕ್ತೋ ಧೃತ್ಯಾತ್ಮಾನಂ ನಿಯಮ್ಯ ಚ~। \\ಶಬ್ದಾದೀನ್ವಿಷಯಾಂಸ್ತ್ಯಕ್ತ್ವಾ ರಾಗದ್ವೇಷೌ ವ್ಯುದಸ್ಯ ಚ \versenum{॥ ೫೧~॥}
\end{verse}

\begin{verse}
ವಿವಿಕ್ತಸೇವೀ ಲಘ್ವಾಶೀ ಯತವಾಕ್ಕಾಯಮಾನಸಃ~।\\ಧ್ಯಾನಯೋಗಪರೋ ನಿತ್ಯಂ ವೈರಾಗ್ಯಂ ಸಮುಪಾಶ್ರಿತಃ \versenum{॥ ೫೨~॥}
\end{verse}

\begin{verse}
ಅಹಂಕಾರಂ ಬಲಂ ದರ್ಪಂ ಕಾಮಂ ಕ್ರೋಧಂ ಪರಿಗ್ರಹಮ್~।\\ವಿಮುಚ್ಯ ನಿರ್ಮಮಃ ಶಾಂತೋ ಬ್ರಹ್ಮಭೂಯಾಯ ಕಲ್ಪತೇ \versenum{॥ ೫೩~॥}
\end{verse}

{\small ವಿಶುದ್ಧವಾದ ಬುದ್ಧಿಯಿಂದ ಕೂಡಿ, ಧೃತಿಯಿಂದ ತನ್ನನ್ನು ನಿಗ್ರಹಿಸಿ, ಶಬ್ದಾದಿ ವಿಷಯಗಳನ್ನು ಬಿಟ್ಟು, ರಾಗ ದ್ವೇಷಗಳನ್ನು ನಿರಾಕರಿಸಿ, ನಿರ್ಜನ ಪ್ರದೇಶದಲ್ಲಿ ವಾಸ ಮಾಡುತ್ತಾ, ಮಿತಾಹಾರಿಯಾಗಿ, ವಾಕ್ ಶರೀರ ಮನಸ್ಸು ಇವುಗಳನ್ನು ನಿಗ್ರಹಿಸಿ, ಯಾವಾಗಲೂ ಧ್ಯಾನಯೋಗದಲ್ಲಿ ನಿರತನಾಗಿ, ವೈರಾಗ್ಯಪರನಾಗಿ, ಅಹಂಕಾರ, ಬಲ ದರ್ಪ ಕಾಮ ಕ್ರೋಧ ಪರಿಗ್ರಹ ಇವುಗಳನ್ನು ತ್ಯಜಿಸಿ, ಮಮತೆ ಇಲ್ಲದೆ ಶಾಂತನಾಗಿ ಬ್ರಹ್ಮ ಭಾವವನ್ನು ಹೊಂದಲು ಯೋಗ್ಯನಾಗುತ್ತಾನೆ.}

ಅವನು ತನ್ನ ಬುದ್ಧಿಯನ್ನು ಪರಿಶುದ್ಧಗೊಳಿಸಬೇಕು. ಬುದ್ಧಿಯ ಕೊಳೆಯನ್ನೆಲ್ಲ ತೆಗೆಯಬೇಕು. ಬುದ್ಧಿ ಕೇವಲ ಸತ್ಯವನ್ನು ತಿಳಿದುಕೊಳ್ಳಬೇಕೆಂದು ಇಚ್ಛಿಸಬೇಕೇ ಹೊರತು ತನಗೆ ಪ್ರಿಯವಾದುದನ್ನು ಸತ್ಯವೆಂದು ಸಮರ್ಥಿಸಕೂಡದು. ಲಾಯರು ಬುದ್ಧಿವಂತ. ಆದರೆ ಅವನ ದೃಷ್ಟಿ ಸತ್ಯವನ್ನು ತಿಳಿದುಕೊಳ್ಳುವುದಲ್ಲ. ತನ್ನ ಕಕ್ಷಿ ಏನನ್ನು ಹೇಳುತ್ತಾನೋ, ಅದನ್ನು ನ್ಯಾಯ ಎಂದು ನ್ಯಾಯಾಲಯ ದಲ್ಲಿ ಸಮರ್ಥಿಸುವುದು. ಆದರೆ ನ್ಯಾಯಾಧಿಪತಿಗೆ ಯಾವ ಕಕ್ಷಿಯ ಮೇಲೂ ಮನಸ್ಸಿಲ್ಲ. ಅವನು ಸತ್ಯವನ್ನು ಮಾತ್ರ ತಿಳಿದುಕೊಳ್ಳಬಯಸುವನು. ಹಾಗೆಯೇ ದೇವರೆಡೆಗೆ ಹೋಗುವವನ ಬುದ್ಧಿ ಸತ್ಯದ ವಿನಹ ಬೇರಾವ ಕಡೆಗೂ ಹೋಗಕೂಡದು. ಹಾಗೆ ಹೋಗಬೇಕಾದರೆ ಅದು ಪರಿಶುದ್ಧವಾಗಿರಬೇಕು.

ತನ್ನ ಇಂದ್ರಿಯ ಮತ್ತು ಮನಸ್ಸುಗಳನ್ನೆಲ್ಲಾ ಚೆನ್ನಾಗಿ ತನ್ನ ಸ್ವಾಧೀನಕ್ಕೆ ತಂದಿರುವನು. ವಿಷಯ ವಸ್ತುವಿನ ಕಡೆ ಅದನ್ನು ಬಿಡುವುದಿಲ್ಲ. ಇದಕ್ಕೆ ಅದ್ಭುತವಾದ ಮಾನಸಿಕ ಶಕ್ತಿ ಬೇಕು. ಚಿತ್ತ ಸಂಪೂರ್ಣ ಇವನ ವಶವಾಗಿ ಇವನು ಹೇಳಿದತ್ತ ಹೋಗಬೇಕು. ಈ ಚಿತ್ತಸ್ವಾಧೀನ ಅತ್ಯಂತ ಮುಖ್ಯವಾದುದು ದೇವರ ಕಡೆ ಹೋಗುವವನಿಗೆ.

ಅವನು ರಾಗ, ದ್ವೇಷಗಳನ್ನು ನಿರಾಕರಿಸಿರುವವನು. ಗುರಿಯೆಡೆಗೆ ಹೋಗುವವನನ್ನು ಕಟ್ಟಿಹಾಕುವ ಗೂಟಗಳು ಇವು. ರಾಗ, ದ್ವೇಷ ಎರಡೂ ಗೂಟವೇ. ಒಂದು ಪ್ರಿಯವಾಗಿರುವುದು, ಮತ್ತೊಂದು ಅಪ್ರಿಯವಾಗಿರುವುದು. ಆದರೆ ಎರಡೂ ನಮ್ಮ ಮನಸ್ಸಿನ ಶಕ್ತಿ ವ್ಯಯವಾಗುವುದಕ್ಕೆ ಇರುವ ರಂಧ್ರಗಳು. ಇವು ದೇವರನ್ನು ಮರೆಸುವುವು. ಅವನಲ್ಲದ ವಸ್ತುವಿನಿಂದ ಮನಸ್ಸನ್ನು ತುಂಬಿಕೊಳ್ಳು ವಂತೆ ಮಾಡುವುವು. ಯಾವಾಗ ತೇಲುವ ತೆಪ್ಪದಲ್ಲಿ ತೂತಾಗುವುದೊ ಆಗ ಹೊರಗಿನ ನೀರು ಅದರೊಳಗೆ ಕ್ರಮೇಣ ನುಗ್ಗಿ ದೋಣಿಯನ್ನು ಮುಳುಗಿಸುವುದು. ಹಾಗೆಯೇ ರಾಗ ದ್ವೇಷಗಳು ರಂಧ್ರಗಳು. ಆ ರಂಧ್ರದ ಮೂಲಕ ಪ್ರಾಪಂಚಿಕ ವಸ್ತುಗಳು ಬಂದು ತುಂಬಿಕೊಂಡು ಜೀವನದ ದೋಣಿಯನ್ನು ಸಂಸಾರದಲ್ಲಿ ಮುಳುಗಿಸುವುವು. ಪ್ರಯಾಣ ಹೊರಡುವುದಕ್ಕೆ ಮುಂಚೆ ಈ ಎರಡು ರಂಧ್ರಗಳನ್ನೂ ಅವನು ಮುಚ್ಚುವನು.

ಶಬ್ದ, ಸ್ಪರ್ಶ, ರೂಪ, ರಸ, ಗಂಧ ಇವುಗಳನ್ನು ಬಂಧನದಲ್ಲಿರುವಾಗ ಬೇಕಾದಷ್ಟು ಅನುಭವಿಸಿ ದ್ದಾನೆ. ಇವುಗಳಿಂದ ತಾತ್ಕಾಲಿಕ ಸಮಾಧಾನವಷ್ಟೆ. ಎಂದಿಗೂ ತೃಪ್ತಿ ಬರಲಿಲ್ಲ. ಎಷ್ಟು ಅನುಭವಿಸಿ ದರೂ ಇನ್ನೂ ಅನುಭವಿಸಬೇಕೆಂದಿರುವುದೇ ಹೊರತು ಸಾಕು ಎನ್ನುವುದಿಲ್ಲ. ಪ್ರತಿಯೊಂದು ಸುಖ ಅನುಭವಿಸಿದಾಗಲೂ ಇದು ಹೊಸ ಸಂಸ್ಕಾರವನ್ನು ಮನಸ್ಸಿನ ಮೇಲೆ ಬಿಡುವುದು. ಆ ಸಂಸ್ಕಾರ ನಮ್ಮ ಅನುಭವದಿಂದ ಬಲವಾಗುತ್ತಾ ಬರುವುದೇ ಹೊರತು ಅದೆಂದಿಗೂ ಕ್ಷೀಣವಾಗುವುದಿಲ್ಲ. ಅದನ್ನು ಉಪವಾಸದಿಂದ ಸಾಯಿಸಬೇಕೇ ಹೊರತು ಕೇಳಿದ್ದನ್ನೆಲ್ಲ ಕೊಡುತ್ತಿದ್ದರೆ ಅದು ಕೊಬ್ಬು ವುದು. ಇನ್ನು ಮೇಲೆ ಅವನು ವಿಷಯ ವಸ್ತುವಿಗೆ ವಿಮುಖನಾಗಿದ್ದಾನೆ. ದೇವರ ಕಡೆಗೆ ಹೋಗಬೇಕಾ ದರೆ ವಿಷಯ ವಸ್ತುವಿಗೆ ಬೆನ್ನು ತೋರಬೇಕು. ರಾಮ ಮತ್ತು ಕಾಮ ಎರಡನ್ನೂ ಒಂದು ಕಡೆ ಸೇರಿಸುವುದಕ್ಕೆ ಆಗುವುದಿಲ್ಲ. ಒಂದು ಇದ್ದರೆ ಮತ್ತೊಂದು ಇರುವುದಕ್ಕೆ ಆಗುವುದಿಲ್ಲ.

ಅವನು ಹೆಚ್ಚು ನಿರ್ಜನ ಪ್ರದೇಶವನ್ನು ಪ್ರೀತಿಸುವನು. ಅಲ್ಲಿ ಇವನ ಚಿತ್ತ ಸ್ವಾಸ್ಥ್ಯಕ್ಕೆ ಭಂಗ ತರುವ ಯಾವ ಗಲಾಟೆಯೂ ಇರುವುದಿಲ್ಲ. ನಿರಾತಂಕವಾಗಿ ಏಕಾಗ್ರತೆಯಿಂದ ಭಗವಂತನನ್ನು ಚಿಂತಿಸುವುದಕ್ಕೆ ಸಾಧ್ಯವಾಗುವುದು. ಎಂತಹ ಒಳ್ಳೆಯ ಸಹವಾಸವಾದರೂ ಯಾವಾಗಲೂ ಒಳ್ಳೆಯ ದಲ್ಲ. ಅವರೊಡನೆ ಇರುವಾಗ ಏನನ್ನಾದರೂ ಮಾತನಾಡಬೇಕಾಗುವುದು. ಮಾತನಾಡದೆ ಇದ್ದರೆ ಅವರು ಅನ್ಯಥಾ ಭಾವಿಸುವರು. ಆಗ ಮನಸ್ಸಿನ ಏಕಾಗ್ರತೆಗೆ ಭಂಗ ಬರುವುದು. ಆದಕಾರಣವೇ ಅವನು ಏಕಾಂಗಿಯಾಗಿರಲು ಆಶಿಸುವನು.

ಅವನು ಮಿತಾಹಾರಿ. ಯೋಗ ಜೀವನಕ್ಕೆ ಇದು ಅತ್ಯಾವಶ್ಯಕ. ನಾವು ಹಿಂದಿನ ಧ್ಯಾನಯೋಗದಲ್ಲೇ ಇದನ್ನು ನೋಡಿರುವೆವು. ಯಾರು ಹೆಚ್ಚು ಊಟ ಮಾಡುವರೋ ಅಥವಾ ಯಾರು ಕಡಿಮೆ ಊಟ ಮಾಡುವರೋ ಅವರಿಬ್ಬರಿಗೂ ಯೋಗ ಸಾಧ್ಯವಿಲ್ಲ. ಅತಿ ಊಟ ಮಾಡುವವನು ಸೋಮಾರಿಯಾಗು ತ್ತಾನೆ. ಅವನ ಶಕ್ತಿಯ ಬಹುಭಾಗ ತಿಂದ ಆಹಾರವನ್ನು ಅರಗಿಸುವುದಕ್ಕೆ ಬೇಕಾಗುವುದು. ಇನ್ನು ತಲೆ ಕಡೆ ಹೋಗುವುದಕ್ಕೆ ಹೆಚ್ಚು ಉಳಿಯುವುದಿಲ್ಲ. ತುಂಬಾ ಕಡಿಮೆ ಊಟ ಮಾಡುವವನ ಸ್ಥಿತಿ ಮತ್ತೂ ಶೋಚನೀಯವಾಗುವುದು. ಅಂಗಾಂಗಗಳು ಕ್ಷೀಣಿಸುವುವು, ಮನಸ್ಸು ದುರ್ಬಲವಾಗು ವುದು. ಮನಸ್ಸು ದುರ್ಬಲವಾದಾಗ ಯಾವ ಯಾವ ಚಲ್ಲಾಪಿಲ್ಲಿಯೊ ಮೆದುಳಿನಲ್ಲಿ ಆಗುತ್ತಿರುವು ದನ್ನು ಇವನು ಅತೀಂದ್ರಿಯ ಆಧ್ಯಾತ್ಮಿಕ ಅನುಭವ ಎಂದು ಭಾವಿಸುವನು. ರೇಡಿಯೋ ಕೆಟ್ಟುಹೋಗಿ ದ್ದರೆ ಕೆಲವು ವೇಳೆ ಅದರ ಮೂಲಕ ಏನೇನೋ ಶಬ್ದಗಳು ಬರುತ್ತವೆ. ಅದನ್ನೇ ಒಂದು ಅದ್ಭುತ ಸಂಗೀತ ಎಂದು ಭಾವಿಸಿದಂತೆ. ಮೊದಲು ದೇಹ ಗಟ್ಟಿಮುಟ್ಟಾಗಿರಬೇಕು. ಅದರ ಹಿಂದೆ ಶುದ್ಧವಾದ, ಬಲವಾದ ಮನಸ್ಸು ಮತ್ತು ಬುದ್ಧಿಗಳಿರಬೇಕು. ಇದಕ್ಕೆ ಮಿತಾಹಾರ ಅತ್ಯಾವಶ್ಯಕ.

ಮನಸ್ಸಿನ ಉದ್ವೇಗಕ್ಕೆಲ್ಲಾ ಕಾರಣ, ಮಾತು, ಶರೀರ ಮತ್ತು ಮನಸ್ಸಿನ ಚಂಚಲತೆ. ಅವನು ಇವುಗಳನ್ನು ನಿಗ್ರಹಿಸುವನು. ಮನಸ್ಸಿಗೆ ಬಂದದ್ದನ್ನೆಲ್ಲಾ ಮಾತನಾಡಿದರೆ ಅನಂತರ, ಎಂತಹ ಅಚಾತುರ್ಯವನ್ನು ನಾನು ಮಾಡಿದೆ ಎಂದು ಅದನ್ನೇ ಕುರಿತು ಪರಿತಪಿಸಬೇಕಾಗುವುದು. ಶರೀರವನ್ನು ಸುಮ್ಮನೆ ದಂಡಿಸಿದರೆ, ಅದರಿಂದಲೇ ಹಲವು ಬಗೆಯ ರೋಗಾದಿಗಳೆಲ್ಲ ಬರುತ್ತವೆ. ಇನ್ನು ಮನಸ್ಸು, ಅದರ ಚಂಚಲತೆಯೇ ನಮ್ಮ ಎಲ್ಲಾ ದುರವಸ್ಥೆಗೆ ಕಾರಣ. ಈ ಮೂರನ್ನು ಮುಂಚೆಯೇ ನಿಗ್ರಹಿಸುವನು.

ಅವನು ಯಾವಾಗಲೂ ಧ್ಯಾನಯೋಗದಲ್ಲಿ ನಿರತನಾಗಿರುತ್ತಾನೆ. ಧ್ಯಾನಕ್ಕೆ ಕುಳಿತಾಗ ಮಾತ್ರ ಅವನು ಧ್ಯಾನಮಾಡಿ ಇತರ ಕಾಲದಲ್ಲಿ ಬಾಹ್ಯ ವಸ್ತುಗಳಲ್ಲಿ ಮುಳುಗುವುದಿಲ್ಲ. ಇತರ ಕಾಲ ದಲ್ಲಿಯೂ ಮನಸ್ಸಿನ ಒಂದು ಭಾಗ ಧ್ಯಾನಾವಸ್ಥೆಯಲ್ಲಿಯೇ ಇರುವುದು. ಕೆಲವು ವೇಳೆ ಅದು ಎದ್ದು ಕಾಣುತ್ತದೆ, ಮತ್ತೆ ಕೆಲವು ವೇಳೆ ಮರಳಿನ ಕೆಳಗೆ ಇರುವ ನೀರಿನಂತೆ ಅಂತರ್ಮುಖವಾಗಿ ಭಗವಂತನನ್ನು ಕುರಿತು ಚಿಂತಿಸುತ್ತಿರುವುದು. ಧ್ಯಾನ ಮಾಡುವುದು ಅವನ ಮನಸ್ಸಿನ ಒಂದು ಸ್ವಭಾವವಾಗಿದೆ. ಕುಳಿತಿರಲಿ, ನಿಂತಿರಲಿ, ಏನಾದರೂ ಕೆಲಸ ಮಾಡುತ್ತಿರಲಿ, ಮನಸ್ಸಿನ ಬಹುಭಾಗ ಭಗವಂತನ ಕಡೆಗೆ ಹರಿಯುತ್ತಿರುವುದು.

ಅವನು ವೈರಾಗ್ಯವನ್ನು ಆಶ್ರಯಿಸಿರುವನು. ಈ ಪ್ರಪಂಚದಲ್ಲಿ ಎಲ್ಲಾ ಭಯದಿಂದ ಕೂಡಿದೆ. ವೈರಾಗ್ಯ ಒಂದರಲ್ಲೇ ಅಭಯ ಇರುವುದು. ನಾವು ಸಂಗ್ರಹಿಸುತ್ತಾ ಹೋದಷ್ಟು ಭಾರ ಹೆಚ್ಚುವುದು, ಜವಾಬ್ದಾರಿ ಹೆಚ್ಚುವುದು. ನಾವು ದೇವರ ಕಡೆಗೆ ಹೋಗಬೇಕಾದರೆ ಹಗುರವಾಗಬೇಕು. ಈ ಪ್ರಪಂಚದ ಹೆಣಭಾರವನ್ನು ಒದರಬೇಕು. ನಮಗೆ ಅತ್ಯಂತ ಆವಶ್ಯಕವಾದುದನ್ನು ದೇವರು ಒದಗಿಸು ವನು ಮತ್ತು ದೇವರು ರಕ್ಷಿಸುವನು ಎಂಬ ಭಾಷೆಯನ್ನೇ ಭಕ್ತನಿಗೆ ಕೊಟ್ಟಿರುವನು. ನಾವೀಗ ಹೊರುತ್ತಿರುವ ಪ್ರಯೋಜನವಿಲ್ಲದ ಭಾರವೇ ಅಹಂಕಾರ, ಬಲ, ದರ್ಪ, ಕಾಮ, ಕ್ರೋಧ ಮತ್ತು ಪರಿಗ್ರಹ ಎಂಬವುಗಳು. ಇವು ನಮಗೆ ಶಾಂತಿಯನ್ನು ಕೊಡಲಾರವು, ಸುಖವನ್ನು ಕೊಡಲಾರವು. ಇವೆಲ್ಲ ಪ್ರಯೋಜನವಿಲ್ಲದ ವಸ್ತುಗಳು, ಭಗವಂತನ ಕಡೆಗೆ ಹೋಗುವ ದೃಷ್ಟಿಯಿಂದ. ಈ ಭಾರದಿಂದ ಪಾರಾದರೆ ನಾವು ಹಗುರವಾಗಿ ದೇವರೆಡೆಗೆ ಬೇಗ ಹೋಗಬಹುದು. ಅಹಂಕಾರ, ‘ನನ್ನ ಸಮಾನ ಇಲ್ಲ’ಎಂದು ಮೆರೆಯುವುದು. ಯಾವುದರಲ್ಲಿ ಇವನಿಗೆ ಸಮಾನವಿಲ್ಲವೋ ತಿಳಿಯದು. ವಿದ್ಯೆ, ಶಕ್ತಿ, ಐಶ್ವರ್ಯ, ಅಧಿಕಾರ ಎಲ್ಲದರಲ್ಲಿಯೂ ಇವನನ್ನು ಮೀರಿದವರು ಇರುವರು. ಆದರೆ ಈ ಅಲ್ಪಮತಿ ತನ್ನಿಂದ ಕೆಳಗೆ ಇರುವವರೊಡನೆ ಹೋಲಿಸಿ ತನ್ನ ಸಮಾನ ಯಾರೂ ಇಲ್ಲ ಎಂದು ಭಾವಿಸುವನು. ಒಬ್ಬ ಬಡ ಗುಮಾಸ್ತ ಕೆಲಸವಿಲ್ಲದ ಮನುಷ್ಯನನ್ನು ನೋಡಿ ತಾನು ಅವನಿಗಿಂತ ಮೇಲು ಎಂದು ಹೆಮ್ಮೆಪಟ್ಟಂತೆ. ನಮ್ಮ ಅಹಂಕಾರ ಕುಗ್ಗಬೇಕಾದರೆ ನಮಗಿಂತ ಮೇಲಿರುವವರನ್ನು ನೋಡಬೇಕು. ಆಗ ನಮ್ಮಲ್ಲಿ ದೈನ್ಯತೆ ಮೂಡುವುದು. ನಮಗಿಂತ ಕೆಳಗಿರುವವರನ್ನು ನೋಡಿದರೆ ನಮ್ಮ ಸಮಾನ ಇಲ್ಲ ಎಂಬ ಅಹಂಕಾರ ಹುಟ್ಟುವುದು. ಈ ಪ್ರಪಂಚದಲ್ಲಿ ಎರಡಕ್ಕೂ ಅವಕಾಶವಿದೆ. ದೀನರಾಗಬಹುದು ಅಥವಾ ದುರಹಂಕಾರಿಗಳಾಗಬಹುದು. ಬ್ರಹ್ಮಜ್ಞಾನವನ್ನು ಪಡೆಯಬೇಕೆಂದು ಆಶಿಸುವವನು ಮೊದಲನೆಯದನ್ನು ತೆಗೆದುಕೊಳ್ಳುತ್ತಾನೆ.

ಇದರಂತೆಯೇ ಬಲ ಮತ್ತು ದರ್ಪ. ಈವಾಗ ನನಗೆ ಶಕ್ತಿ ಇದೆ. ನನ್ನ ಸಮಾನ ಇಲ್ಲ ಎಂದು ಮೆರೆಯುವೆನು. ಆದರೆ ಭೀಮಸೇನ, ಗಾಮಾಗಳಂಥವರು ದುರ್ಬಲರಾಗುತ್ತಾರೆ. ಈಗ ಹೆಮ್ಮೆಪಟ್ಟರೆ ಆ ಸಮಯದಲ್ಲಿ ನಮ್ಮ ಗತಿಯೇನು? ದರ್ಪ ಕೂಡಾ ಹಾಗೆಯೇ. ನನ್ನಲ್ಲಿ ಈಗ ಸ್ವಲ್ಪ ಅಧಿಕಾರ ಇರುವುದು ಅಥವಾ ಐಶ್ವರ್ಯ ಇರುವುದು. ಅದಕ್ಕಾಗಿ ‘ನನ್ನ ಸಮಾನ ಇಲ್ಲ’ ಎಂದು ಮೆರೆಯು ತ್ತೇನೆ. ಈ ಅಧಿಕಾರ ಇಂದು ಇದ್ದದ್ದು ನಾಳೆ ಹೋಗುವುದು. ಆಗ ಏನಿದೆ ನನಗೆ ಹೆಮ್ಮೆ ಪಡುವುದಕ್ಕೆ? ಅಧಿಕಾರದಲ್ಲಿದ್ದಾಗ ‘ನಾನು ಹಾಗೆ ಮಾಡಿದೆ, ಹೀಗೆ ಮಾಡಿದೆ’ ಎಂದು ಪ್ರತಾಪವನ್ನು ಕೊಚ್ಚಿ ಕೊಳ್ಳುವುದಷ್ಟೆ. ಲಕ್ಷ್ಮಿಯಷ್ಟು ಚಂಚಲ ಮತ್ತೊಂದಿಲ್ಲ. ಅದು ಹೇಗೋ ಬಂತು ಅಥವಾ ಆಗಲೇ ಇರುವ ಕಡೆ ನಾವು ಹುಟ್ಟುತ್ತೇವೆ. ಸ್ವಲ್ಪ ಕಾಲವಾದ ಮೇಲೆ ಅದು ಹೇಳದೆ ಕೇಳದೆ ಹಿಮ ಕರಗಿ ಹೋಗುವಂತೆ ಹೋಗುವುದು. ಇದನ್ನೆಲ್ಲ ಮುಂಚೆಯೇ ಅರಿತು ಜ್ಞಾನಿಯಾದವನು ತ್ಯಜಿಸುವನು.

ಕಾಮವನ್ನು ತ್ಯಜಿಸುತ್ತಾನೆ. ಕಾಮವೇ ನಮ್ಮನ್ನು ವಿಷಯ ವಸ್ತುವಿಗೆ ಕಟ್ಟಿಹಾಕುವುದು. ಎತ್ತು ಗೂಟದ ಸುತ್ತ ಸುತ್ತುವಂತೆ ನಾವು ಕಾಮದ ಸುತ್ತ ಸುತ್ತುತ್ತಿರಬೇಕಾಗುವುದು. ಅದನ್ನು ತೃಪ್ತಿ ಪಡಿಸಿದಷ್ಟು ಬಲವಾಗುವುದು. ಇನ್ನೂ ಬೇಕು ಎಂದು ಕೇಳುವುದೇ ಹೊರತು ಸಾಕು ಎನ್ನುವುದಿಲ್ಲ. ನಿಗ್ರಹಿಸುವುದರಿಂದ ಮಾತ್ರ ಅದನ್ನು ಸುಮ್ಮನಿರಿಸಬಹುದೆ ಹೊರತು ಅದು ಕೇಳಿದ್ದನ್ನೆಲ್ಲಾ ಕೊಟ್ಟು ಪೂರೈಸುವುದಕ್ಕಾಗುವುದಿಲ್ಲ.

ಕ್ರೋಧ ನಮ್ಮ ವಿವೇಚನಾ ಶಕ್ತಿಯನ್ನೆಲ್ಲಾ ಅಪಹರಿಸುವುದು. ಕೋಪ ನಮ್ಮನ್ನು ಮೆಟ್ಟಿತು ಎಂದರೆ ಯಾವುದು ಸರಿ ಯಾವುದು ತಪ್ಪು ಎಂಬುದೆಲ್ಲವೂ ಮರೆತುಹೋಗುವುದು. ಆಗ ಮಾಡಬಾರದುದನ್ನು ಮಾಡುವೆವು. ಅನಂತರ ಪಶ್ಚಾತ್ತಾಪ ಪಡುವೆವು. ಒಂದು ಮನೆಗೆ ಬೆಂಕಿ ಹಾಕಿ ಪಶ್ಚಾತ್ತಾಪ ಪಟ್ಟರೆ ಬೆಂಕಿ ನಂದಿಹೋಗುವುದಿಲ್ಲ. ಆದಕಾರಣವೇ ಕೋಪವನ್ನು ತ್ಯಜಿಸುವನು.

ಪರಿಗ್ರಹ ಎಂಬುದು ಸೂಕ್ಷ್ಮ ಬಂಧನ. ನಮಗೆ ಗೊತ್ತಿಲ್ಲದೆ ಇನ್ನೊಬ್ಬರಿಗೆ ನಮ್ಮನ್ನು ದಾಸರನ್ನಾಗಿ ಮಾಡುವುದು. ಯಾವಾಗ ನಾವು ಮತ್ತೊಬ್ಬರಿಂದ ಏನನ್ನಾದರೂ ಸ್ವೀಕರಿಸುತ್ತೇವೆಯೋ ಆಗ ಅವನ ಹಂಗಿಗೆ ಒಳಗಾಗುತ್ತೇವೆ. ಅವನಿಗೆ ವಿರೋಧವಾಗಿ ಹೋಗುವುದಕ್ಕೆ ಆಗುವುದಿಲ್ಲ. ಅವನು ಹೇಳಿದಂತೆ ಕೇಳಬೇಕಾಗುವುದು. ಇದನ್ನರಿತೇ ದೇವರೆಡೆಗೆ ಹೋಗುವವನು ಯಾರಿಗೂ ಕೈಯೊಡ್ಡುವುದಿಲ್ಲ.

ಆತನಲ್ಲಿ ತನ್ನದು ಎಂಬ ಭಾವವಿಲ್ಲ. ಎಲ್ಲ ದೇವರದು, ದೇವರು ಕೊಟ್ಟಿದ್ದು ಎಂಬ ಭಾವವೊಂದೇ ಇರುವುದು. ಇಂತಹ ಮನುಷ್ಯನ ಮನಸ್ಸು ಸದಾ ಶಾಂತವಾಗಿರುವುದು. ಪ್ರಪಂಚದ ಆಂದೋಳನಗಳಾವುವೂ ಅಲ್ಲಿ ಪ್ರವೇಶ ಮಾಡುವುದಕ್ಕಾಗುವುದಿಲ್ಲ. ಈ ಪ್ರಪಂಚದಲ್ಲಿದ್ದರೂ ಅವನು ಈ ಪ್ರಪಂಚಕ್ಕೆ ಸೇರಿದವನಲ್ಲ, ಭಗವಂತನಿಗೆ ಸೇರಿದವನು. ಇವನು ಬ್ರಹ್ಮಭಾವವನ್ನು ಹೊಂದಲು ಯೋಗ್ಯನಾಗಿರುತ್ತಾನೆ. ಇಷ್ಟು ಅಣಿಯಾಗಿದ್ದರೆ, ಹೇಗೆ ಇದ್ದಲು ಬೆಂಕಿಯ ಮಧ್ಯೆ ಇದ್ದರೆ ತಾನೂ ಕೂಡಾ ಧಗಧಗಿಸುವ ಕೆಂಡವೇ ಆಗುವುದೋ, ಹಾಗೆ ಬ್ರಹ್ಮಭಾವವನ್ನು ಪಡೆಯುವನು.

\begin{verse}
ಬ್ರಹ್ಮಭೂತಃ ಪ್ರಸನ್ನಾತ್ಮಾ ನ ಶೋಚತಿ ನ ಕಾಂಕ್ಷತಿ~।\\ಸಮಃ ಸರ್ವೇಷು ಭೂತೇಷು ಮದ್ಭಕ್ತಿಂ ಲಭತೇ ಪರಾಮ್ \versenum{॥ ೫೪~॥}
\end{verse}

{\small ಬ್ರಹ್ಮಭಾವವನ್ನು ಹೊಂದಿರುವವನು, ಪ್ರಸನ್ನಾತ್ಮನಾದವನು ಶೋಕಿಸುವುದಿಲ್ಲ. ಸರ್ವಭೂತಗಳನ್ನು ಸಮಭಾವ ದಿಂದ ನೋಡುತ್ತಾನೆ. ನನ್ನಲ್ಲಿ ಪರಮ ಭಕ್ತಿಯನ್ನು ಹೊಂದುತ್ತಾನೆ.}

ಬ್ರಹ್ಮಭಾವವನ್ನು ಹೊಂದಿದವನ ಮನಸ್ಸು ಪ್ರಸನ್ನವಾಗಿರುವುದು, ಶಾಂತವಾಗಿರುವುದು. ಅಲ್ಲಿ ಉದ್ವೇಗಕ್ಕೆ ಆಸ್ಪದವೇ ಇಲ್ಲ. ಅವನು ಯಾವುದಕ್ಕೂ ಶೋಕಿಸುವುದಿಲ್ಲ. ಈ ಪ್ರಪಂಚದಲ್ಲಿ ಎಷ್ಟೋ ದುಃಖಕರವಾದ ಪ್ರಸಂಗಗಳು ಬರಬಹುದು. ಹತ್ತಿರದ ಸಂಬಂಧಿಕರು ಕಾಲವಾಗಬಹುದು. ದೊಡ್ಡ ವಿಪತ್ತೇ ಬರಬಹುದು. ಆದರೆ ಅವನು ಶೋಕಿಸುವುದಿಲ್ಲ. ಎಂದರೆ ಅವನ ಹೃದಯ ಕಲ್ಲಿನಂತಾಗುವು ದಿಲ್ಲ. ಹೊರಗೆ ಅವನು ಅನುಕಂಪೆಯನ್ನು ತೋರುತ್ತಾನೆ. ಆದರೆ ಒಳಗೆ ತಿಳಿದುಕೊಂಡಿರುವನು: ಸಂಸಾರದಲ್ಲಿ ಈ ಅಲೆಗಳೆಲ್ಲಾ ಎದ್ದು ಬೀಳುತ್ತಿರುವುವು, ಇವು ಅನಿವಾರ್ಯ, ಇವುಗಳ ಸ್ವಭಾವವೇ ಬಂದು ಹೋಗುವುದು, ಬರುವಾಗಲೂ ಭಗವದಿಚ್ಛೆಯಿಂದ ಬರುವುವು, ಹೋಗುವಾಗಲೂ ಅವನ ಇಚ್ಛೆಯಿಂದಲೇ ಹೋಗುವುವು ಎಂದು ತನ್ನ ಮನಸ್ಸಿಗೆ ಹಾಕಿಕೊಳ್ಳವುದಿಲ್ಲ.

ಅವನು ಯಾವುದನ್ನೂ ಬಯಸುವುದಿಲ್ಲ. ಅವನ ಬಯಕೆಗಳೆಲ್ಲಾ ನಿಂತು ಹೋಗಿವೆ. ತನಗೆ ಅದಾಗಬೇಕು, ಇದಾಗಬೇಕು ಎಂಬ ಆಸೆಯೇ ಇಲ್ಲ. ದೇವರು ಮಾಡಿಸಿದ್ದು ಆಗಲಿ, ಅವನು ಕೊಟ್ಟಿದ್ದನ್ನು ಸ್ವೀಕರಿಸುತ್ತೇನೆ ಎಂದು ತನ್ನ ಇಚ್ಛೆಯನ್ನು ಭಗವಂತನಿಗೆ ಅರ್ಪಣೆ ಮಾಡಿಬಿಟ್ಟಿರು ವನು. ಅವನು ಸರ್ವಭೂತಗಳನ್ನು ಸಮಭಾವದಿಂದ ನೋಡುತ್ತಾನೆ. ಹೊರಗೆ ನೋಡಿದರೆ ವೈವಿಧ್ಯತೆ ಗಳಿವೆ. ಆದರೆ ಇವೆಲ್ಲಾ ಉಪಾಧಿಗಳು, ವೇಷಗಳು, ಇವುಗಳ ಹಿಂದೆಲ್ಲಾ ಭಗವಂತನೇ ಇರುವುದು. ಜೇಡಿಮಣ್ಣು ವಿನಾಯಕನಂತೆ ಇದೆ, ಮಡಕೆಯಂತೆ ಇದೆ, ಹಣತೆಯಂತೆ ಇದೆ. ಒಂದೊಂದೂ ಒಂದು ಪ್ರಯೋಜನಕ್ಕೆ ಬರುತ್ತದೆ. ಆದರೆ ಜೇಡಿಮಣ್ಣಿನ ದೃಷ್ಟಿಯಿಂದ ನೋಡಿದಾಗ ಎಲ್ಲವೂ ಒಂದೇ. ಅದರಂತೆಯೇ ಇವನು ನೋಡುವುದು ಪಾರಮಾರ್ಥಿಕ ದೃಷ್ಟಿಯಿಂದ.

ಇಂತಹ ವ್ಯಕ್ತಿಗೆ ಭಗವಂತನ ಮೇಲೆ ಪರಮ ಭಕ್ತಿ ಉಂಟಾಗುವುದು. ಅವನು ಪ್ರೀತಿಗಾಗಿ ದೇವರನ್ನು ಪ್ರೀತಿಸುವನು. ಅವನಿಂದ ಲೌಕಿಕವಾದುದಾವುದನ್ನೂ ಆಶಿಸನು. ಅವನು ಭಗವತ್ ಪ್ರೇಮದಿಂದ ತುಂಬಿ ತುಳುಕಾಡುತ್ತಿರುವನು. ಒಂದು ಸ್ಪಂಜನ್ನು ನೀರಿನಲ್ಲಿ ಅದ್ದಿದರೆ ಅದು ಹೇಗೆ ನೀರನ್ನು ಹೀರಿಕೊಳ್ಳುವುದೊ, ಅದರೊಳಗೆ ಇರುವ ರಂಧ್ರದಲ್ಲೆಲ್ಲಾ ನೀರೇ ಓತಪ್ರೋತವಾಗಿರು ವುದೊ ಹಾಗೆ ಭಕ್ತಿ—ಅಮೃತದಿಂದ ತುಂಬಿ ತುಳುಕುತ್ತಿರುವುದು ಅವನ ಹೃದಯ.

\begin{verse}
ಭಕ್ತ್ಯಾ ಮಾಮಭಿಜಾನಾತಿ ಯಾವಾನ್ಯಶ್ಚಾಸ್ಮಿ ತತ್ತ್ವತಃ~।\\ತತೋ ಮಾಂ ತತ್ತ್ವತೋ ಜ್ಞಾತ್ವಾ ವಿಶತೇ ತದನಂತರಮ್ \versenum{॥ ೫೫~॥}
\end{verse}

{\small ಭಕ್ತಿಯ ಮೂಲಕ ನಾನು ಯಾರು ಮತ್ತು ಹೇಗಿದ್ದೇನೆ ಎಂಬುದನ್ನು ಯಥಾರ್ಥವಾಗಿ ಅರಿಯುತ್ತಾನೆ. ಈ ಮೂಲಕ ನನ್ನನ್ನು ತತ್ತ್ವತಃ ತಿಳಿದುಕೊಂಡು ನನ್ನನ್ನು ಪ್ರವೇಶಿಸುತ್ತಾನೆ.}

ಭಗವಂತನನ್ನು ತಿಳಿದುಕೊಳ್ಳುವುದಕ್ಕೆ ಭಕ್ತಿ ಅತ್ಯಂತ ಸುಲಭವಾದ ಸಾಧನ. ಅದರ ಮೂಲಕ ಅವನ ನಿಜಸ್ವರೂಪವನ್ನು ಅರಿಯಬಹುದು. ಅವನೇ ಸರ್ವವ್ಯಾಪಿ, ಸರ್ವಜ್ಞ, ಸರ್ವಶಕ್ತ ಎಂಬುದ ನ್ನೆಲ್ಲಾ ಅರಿಯುವನು. ಅವನು ಹೇಗೆ ಈ ಪ್ರಪಂಚದಲ್ಲಿದ್ದಾನೆ ಎಂಬುದನ್ನರಿಯುವನು. ದೇವರು ನಮ್ಮಿಂದ ಬೇರೆಯಾಗಿ ಯಾವುದೋ ಲೋಕದಲ್ಲೊ ಸ್ಥಳದಲ್ಲೊ ಇಲ್ಲ. ಅವನು ಈ ಪ್ರಪಂಚ ದಲ್ಲೆಲ್ಲಾ ಓತಪ್ರೋತನಾಗಿದ್ದಾನೆ. ಆಕಾಶ ಹೇಗೆ ಎಲ್ಲವನ್ನೂ ವ್ಯಾಪಿಸಿಕೊಂಡಿದೆಯೋ ಅದರಂತೆ ವ್ಯಾಪಿಸಿರುವನು. ಮಜ್ಜಿಗೆಯಲ್ಲಿರುವ ಬೆಣ್ಣೆಯಂತೆ, ಎಳ್ಳಿನ ಬೀಜದಲ್ಲಿರುವ ಎಣ್ಣೆಯಂತೆ ಈ ಪ್ರಪಂಚದಲ್ಲಿ ಹಾಸುಹೊಕ್ಕಾಗಿದ್ದಾನೆ. ಇಂತಹ ಭಗವಂತನನ್ನು ಯಥಾರ್ಥವಾಗಿ ಅರಿಯುವನು. ಶ‍್ರೀರಾಮಕೃಷ್ಣರಂದಂತೆ ಹಾಲನ್ನು ಕೇಳಿದವರು ಇರುವರು, ನೋಡಿದವರೂ ಇರುವರು. ಅದನ್ನು ಕುಡಿದವರೂ ಇರುವರು. ಅದರಂತೆ ಭಗವಂತನ ಹೆಸರನ್ನು ಕೇಳಿದವರೂ ಇರುವರು, ಯಾವಾಗಲೋ ಒಮ್ಮೆ ಮಿಂಚಿನಂತೆ ಅವರ ಮನಸ್ಸು ತಿಳಿಯಾದಾಗ ಅವನನ್ನು ನೋಡಿದವರು ಕೆಲವರು. ಆದರೆ ಯಾರು ಭಗವಂತನಲ್ಲೆ ಅನುಗಾಲ ವಿಹರಿಸುವರೊ ಅವರಿಗೆ ಮಾತ್ರ ಅವನು ಹೇಗಿರುವನೋ ಹಾಗೆಯೇ ತಿಳಿದುಕೊಳ್ಳಲು ಸಾಧ್ಯ.

ಯಾವಾಗ ಅವನ ನೈಜ ಸ್ವಭಾವವನ್ನು ಅರಿತಿರುವರೊ ಆಗ ಅವನನ್ನು ಪ್ರವೇಶಿಸುವರು, ಎಂದರೆ ಅದೇ ಆಗುವರು. ಯಾರು ಬ್ರಹ್ಮವನ್ನು ಅನುಭವಿಸುವರೋ ಪ್ರವೇಶಿಸುವರೋ ಅವರು ಬ್ರಹ್ಮವೇ ಆಗುತ್ತಾರೆ. ಸೌದೆ, ಉರಿಯುವ ಒಲೆಗೆ ಬಿದ್ದರೆ ಅದೂ ಉರಿಯುವ ಕೊಳ್ಳಿಯಾಗುವಂತೆ. ನದಿ, ಸಾಗರಕ್ಕೆ ಬಿದ್ದರೆ ಅದೂ ಸಾಗರವಾಗುವಂತೆ.

\begin{verse}
ಸರ್ವಕರ್ಮಾಣ್ಯಪಿ ಸದಾ ಕುರ್ವಾಣೋ ಮದ್ವ್ಯಪಾಶ್ರಯಃ~।\\ಮತ್ಪ್ರಸಾದಾದವಾಪ್ನೋತಿ ಶಾಶ್ವತಂ ಪದಮವ್ಯಯಮ್ \versenum{॥ ೫೬~॥}
\end{verse}

{\small ಯಾವಾಗಲೂ ಎಲ್ಲಾ ಕರ್ಮಗಳನ್ನೂ ಮಾಡುತ್ತಿದ್ದರೂ, ನನ್ನನ್ನೇ ಆಶ್ರಯಿಸುವವನು ನನ್ನ ಪ್ರಸಾದದಿಂದ ಶಾಶ್ವತವೂ ಅವ್ಯಯವೂ ಆದ ಪದವನ್ನು ಹೊಂದುತ್ತಾನೆ.}

ಭಗವಂತನನ್ನು ತತ್ತ್ವತಃ ತಿಳಿದವನಾದರೂ ಅವನು ಎಲ್ಲಿಯವರೆಗೆ ಈ ಪ್ರಪಂಚದಲ್ಲಿರುವನೊ ಅಲ್ಲಿಯವರೆಗೆ ತನ್ನ ಪಾಲಿಗೆ ಬಂದ ಕರ್ತವ್ಯವನ್ನು ಮಾಡುತ್ತಿರುವನು. ಅವನು ಅಚ್ಚುಕಟ್ಟಾಗಿ ಭಗವದರ್ಪಣ ಭಾವದಿಂದ ಮಾಡುತ್ತಿರುವನು. ಶ‍್ರೀರಾಮಕೃಷ್ಣರು ಒಂದು ಉದಾಹರಣೆಯನ್ನು ಕೊಡುವರು. ಯಾರನ್ನೊ ಯಾವುದೊ ತಪ್ಪಿಗೆ ಜೈಲಿಗೆ ಹಾಕಿದ್ದರು. ಅಲ್ಲಿಂದ ಬಿಡುಗಡೆಯಾದ ಮೇಲೆ ಅವನೇನು ತಕಥೈ ಎಂದು ಕುಣಿದಾಡುವನೆ! ಹಿಂದೆ ಯಾವ ವೃತ್ತಿಯ ಮೂಲಕ ಜೀವನೋ ಪಾಯವನ್ನು ಮಾಡುತ್ತಿದ್ದನೋ ಅದೇ ಕೆಲಸವನ್ನು ಮಾಡುವನು. ಹಾಗೆಯೇ ಮುಕ್ತನೂ ಹಿಂದೆ ಮಾಡುತ್ತಿದ್ದ ಕೆಲಸವನ್ನೇ ಮಾಡುವನು. ಆದರೆ ಹಾಗೆ ಮಾಡುವಾಗ ಯಾವಾಗಲೂ ಭಗವಂತನನ್ನೇ ಆಶ್ರಯಿಸುವನು. ಅವನ ಕೈಯನ್ನು ಹಿಡಿದಿರುವನು. ಅವನನ್ನು ಬಿಟ್ಟಿರುವುದಿಲ್ಲ. ಭಗವಂತನ ದಯೆಯಿಂದ ಅವನಿಗೆ ಶಾಶ್ವತವಾದ ಪದವಿ ದೊರಕುವುದು. ಇನ್ನು ಮೇಲೆ ಅವನು ಸಂಸಾರ ಚಕ್ರಕ್ಕೆ ಸಿಕ್ಕುವುದಿಲ್ಲ. ಕಾಲ ದೇಶ ನಿಮಿತ್ತದ ಆಚೆ ಹೋಗುವನು. ಅವನು ಯಾವ ಬದಲಾವಣೆಗೂ ಸಿಕ್ಕುವುದಿಲ್ಲ.

\begin{verse}
ಚೇತಸಾ ಸರ್ವಕರ್ಮಾಣಿ ಮಯಿ ಸಂನ್ಯಸ್ಯ ಮತ್ಪರಃ~।\\ಬುದ್ಧಿಯೋಗಮುಪಾಶ್ರಿತ್ಯ ಮಚ್ಚಿತ್ತಃ ಸತತಂ ಭವ \versenum{॥ ೫೭~॥}
\end{verse}

{\small ಮನಸ್ಸಿನಲ್ಲಿ ಸರ್ವ ಕರ್ಮಗಳನ್ನೂ ನನಗರ್ಪಿಸಿ, ಮತ್ಪರಾಯಣನಾಗಿ, ಬುದ್ಧಿಯೋಗವನ್ನು ಆಶ್ರಯಿಸಿ ಯಾವಾಗಲೂ ನನ್ನಲ್ಲೇ ಚಿತ್ತವುಳ್ಳವನಾಗಿರು.}

ಕೆಲಸ ಮಾಡುವಾಗ ಮಾಡುವುದೆಲ್ಲಾ ಅವನಿಗೆ ಸೇರಿದ್ದು ಎಂದು ಭಾವಿಸಬೇಕು. ಯಾವಾಗ ಅವನನ್ನು ಎದುರಿಗೆ ಇಟ್ಟುಕೊಂಡು ನಾನು ಕೆಲಸ ಮಾಡುತ್ತಿರುವೆನೊ ಆಗ ಶ್ರೇಷ್ಠವಾದ ಕೆಲಸ ಮಾತ್ರ ಸಾಧ್ಯ. ಅವನ ಕೆಲಸದಲ್ಲಿ ಯಾವ ಸ್ವಾರ್ಥವೂ ಇರುವುದಿಲ್ಲ, ದುರುದ್ದೇಶಗಳೂ ಇರುವು ದಿಲ್ಲ. ಭಗವಂತನೆಂಬ ಬೆಂಕಿ ಈ ಕ್ರಿಮಿಗಳನ್ನೆಲ್ಲಾ ನಾಶ ಮಾಡುವುದು.

ಅವನು ಕೆಲಸವನ್ನು ಮಾಡುತ್ತಿರುವಾಗಲೂ ಸದಾ ಭಗವಂತನನ್ನೇ ಕುರಿತು ಚಿಂತಿಸುತ್ತಿರುವನು. ಅವನ ಮನಸ್ಸು ಯಾವಾಗಲೂ ಅವನ ಕಡೆಗೇ ತಿರುಗುವುದು. ಹೇಗೆ ಎಣ್ಣೆಯನ್ನು ಒಂದು ಪಾತ್ರೆಯಿಂದ ಮತ್ತೊಂದು ಪಾತ್ರೆಗೆ ಹಾಕುವಾಗ ಅಖಂಡವಾಗಿ ಸುರಿಯುವುದೊ ಹಾಗೆಯೇ ಸ್ವಲ್ಪವೂ ಬಿಡದೆ ಅವನು ಭಗವಂತನನ್ನು ಚಿಂತಿಸುತ್ತಿರುವನು. ಅವನ ಕಾರ್ಯಕ್ಕೆ ಭಗವಂತನ ಚಿಂತನೆ ಎಂಬುದು ಹಿನ್ನೆಲೆಯ ಸಂಗೀತದಂತೆ ಇರುವುದು.

ಅವನು ಬುದ್ಧಿಯೋಗವನ್ನು ಆಶ್ರಯಿಸಿರುವನು. ಎಂದರೆ ಬುದ್ಧಿಯನ್ನು ಅವನೆಡೆಗೆ ತಿರುಗಿಸಿರು ವನು. ಅವನನ್ನು ತಿಳಿದುಕೊಳ್ಳುವುದಕ್ಕೆ ಬುದ್ಧಿಯನ್ನು ಉಪಯೋಗಿಸಿರುವನು. ಬುದ್ಧಿ ಯೋಗವಾಗ ಬೇಕಾದರೆ ದೇವರೆಡೆಗೆ ಒಯ್ಯಬೇಕು. ಇಲ್ಲದೇ ಇದ್ದರೆ ಆ ಬುದ್ಧಿಯಿಂದ ಲೌಕಿಕ ವಸ್ತುಗಳನ್ನು ತಿಳಿದುಕೊಳ್ಳುತ್ತೇವೆ. ಕೀರ್ತಿ, ಹಣ, ಅಧಿಕಾರ ಮುಂತಾದುವುಗಳನ್ನು ಸಂಪಾದಿಸುತ್ತೇವೆ. ಅದು ಬುದ್ಧಿಯೋಗವಾಗಿಲ್ಲ. ವಿಷಯ ಪ್ರಪಂಚದಲ್ಲಿ ಅದು ಬುದ್ಧಿಯ ಕಸರತ್ತಾಗಿದೆ. ಭಕ್ತನಾದರೊ ಬುದ್ಧಿಯೆಂಬ ದೀಪವನ್ನು ದೇವರೆಡೆಗೆ ತಿರುಗಿಸುವನು. ಅವನಾರು, ಅವನು ಹೇಗಿರುವನು, ಅವನನ್ನು ನಾವು ಪಡೆಯಬೇಕಾದರೆ ಏನನ್ನು ಮಾಡಬೇಕು, ಇವುಗಳನ್ನೆಲ್ಲ ಬುದ್ಧಿಯ ಮೂಲಕ ತಿಳಿದುಕೊಳ್ಳು ವನು.

ಅನಂತರ ಅವನು ಯಾವಾಗಲೂ ದೇವರಲ್ಲೇ ಮನಸ್ಸುಳ್ಳವನಾಗಿರುವನು. ಅವನನ್ನು ಎಂದಿಗೂ ಮರೆಯುವುದಿಲ್ಲ. ಯಾವ ಪರಿಸ್ಥಿತಿಯಲ್ಲಿದ್ದರೂ ಅವನ ಮನಸ್ಸು ಮಾತ್ರ ದೇವರೆಡೆಗೆ ತಿರುಗಿರು ವುದು. ಉತ್ತರಮುಖಿಯನ್ನು ಹೇಗೆ ಹಿಡಿದರೂ ಅದು ಉತ್ತರಕ್ಕೆ ತಿರುಗಿರುವಂತೆ ಅವನ ಚಿತ್ತವಾ ಗಿರುವುದು.

\begin{verse}
ಮಚ್ಚಿತ್ತಃ ಸರ್ವದುರ್ಗಾಣಿ ಮತ್ಪ್ರಸಾದಾತ್ತರಿಷ್ಯಸಿ~।\\ಅಥ ಚೇತ್ತ್ವಮಹಂಕಾರಾನ್ನ ಶ್ರೋಷ್ಯಸಿ ವಿನಂಕ್ಷ್ಯಸಿ \versenum{॥ ೫೮~॥}
\end{verse}

{\small ನನ್ನಲ್ಲಿ ಮನಸ್ಸನ್ನು ಇಟ್ಟರೆ ನನ್ನ ಅನುಗ್ರಹದಿಂದ ಎಲ್ಲ ವಿಪತ್ತುಗಳಿಂದ ಪಾರಾಗುವೆ. ನೀನು ಅಹಂಕಾರದಿಂದ ನನ್ನ ಮಾತನ್ನು ಕೇಳದೆ ಹೋದರೆ ನಾಶವಾಗುತ್ತೀಯೆ.}

ಭಗವಂತನ ಮೇಲೆ ಮನಸ್ಸನ್ನು ಇಟ್ಟರೆ ಅವನ ಪವಿತ್ರತೆ, ಶಕ್ತಿ ಜ್ಞಾನಗಳನ್ನೆಲ್ಲಾ ಹೀರಿಕೊಂಡು ನಾವು ಬೆಳೆಯಬಹುದು. ಮನೆ ಮುಂದೆ ಹೋಗುವ ವಿದ್ಯುತ್ ತಂತಿಗೆ ತಾಗಿಸಿ ನಾವೊಂದು ತಂತಿಯನ್ನು ಎಳೆದುಕೊಂಡರೆ, ಆ ಶಕ್ತಿಯಿಂದ ನಮ್ಮ ಮನೆಯಲ್ಲಿ ದೀಪ ಬೆಳಗಿಸಬಹುದು. ಅದರಂತೆಯೇ ಭಗವಂತನ ದಿವ್ಯಶಕ್ತಿ ಎಲ್ಲಾ ಜೀವಿಗಳ ಮನೆಯ ಮುಂದೆಯೂ ಹೋಗುತ್ತಿದೆ. ಯಾರು ಅವನೊಡನೆ ಸಂಬಂಧ ಬೆಳೆಸುವರೊ ಅವರು ಪ್ರಯೋಜನ ಪಡೆಯುವರು, ಸಂಬಂಧ ಬೆಳೆಸದವರು ಪ್ರಯೋಜನ ಪಡೆಯುವುದಿಲ್ಲ.

ಅವನ ಅನುಗ್ರಹದಿಂದ ಎಲ್ಲಾ ಕಷ್ಟಗಳಿಂದ ಪಾರಾಗುತ್ತೇವೆ. ಈ ಸಂಸಾರ ಸಾಗರದಲ್ಲಿ ನಾವೊಂದು ಚಿಕ್ಕ ದೋಣಿಯಲ್ಲಿ ಹೋಗುತ್ತಿದ್ದೇವೆ. ಭೀಕರ ಅಲೆಗಳು ಏಳುತ್ತಿವೆ ಸುತ್ತಲೂ. ಯಾವ ಅಲೆ ದೋಣಿಯ ಮೇಲೆ ಬಿದ್ದರೂ ದೋಣಿ ಸಮುದ್ರದ ಪಾಲಾಗುವುದು. ಆದರೆ ಯಾರು ಭಗವಂತನನ್ನು ನೆಚ್ಚಿರುವರೊ ಅವರು ಅಲೆಯ ಮೇಲೆ ತೇಲುವರು. ನಮ್ಮನ್ನು ಅಪ್ಪಳಿಸಲು ಬರುವ ಅಲೆ ತನ್ನ ನೆತ್ತಿಯ ಮೇಲೆ ದೋಣಿಯನ್ನು ಮೇಲೆತ್ತಿ ಅತ್ತ ಬಿಡುವುದು. ಭಗವಂತನನ್ನು ನೆಚ್ಚಿದವನಿಗೆ ಕಷ್ಟ ಕಡಿಮೆ ಎಂದಲ್ಲ. ಇತರರಿಗಿಂತಲೂ ಹೆಚ್ಚು ಅವನ ಪಾಲಿಗೆ ಕಷ್ಟಗಳು ಬರುವುವು. ಆದರೂ ಅವನು ದೇವರನ್ನು ನೆಚ್ಚಿರುವುದರಿಂದ ಪಾರಾಗುತ್ತಾನೆ. ಸುತ್ತಲೂ ನಿರಾಶೆ ಕವಿದಿರುವಾಗ ಆಸೆಯ ಕೈಯೊಂದು ಚಾಚುವುದು, ಅವನನ್ನು ಹಿಡಿದುಕೊಳ್ಳುವುದಕ್ಕೆ. ಅವನಿಗೆ ಕಷ್ಟಗಳು ಪರ್ವತದಂತೆ ಬರುತ್ತವೆ. ಮಂಜಿನಂತೆ ಹೊರಟು ಹೋಗುತ್ತವೆ. ಅವನು ಯಾವ ಸಮಯದಲ್ಲಿಯೂ ನೆಚ್ಚು ಗೆಡುವುದಿಲ್ಲ. ಎಂತಹ ಕಷ್ಟಗಳಿಗೂ, ಅಪಾಯಗಳಿಗೂ ಕುಗ್ಗಿ ಹೋಗುವುದಿಲ್ಲ. ಅವನು ಜೀವನದಲ್ಲಿ ಏಕಾಂಗಿಯಲ್ಲ; ದೇವರನ್ನು ನೆಚ್ಚಿ ಹೊರಟಿರುವನು. ಇವನನ್ನು ರಕ್ಷಿಸುವ ಭಾರ ದೇವರದು. ಅವನಾದರೊ ಎಂದಿಗೂ ತನ್ನನ್ನು ನೆಚ್ಚಿದವರನ್ನು ಕೈಬಿಡುವುದಿಲ್ಲ. “ನನ್ನ ಭಕ್ತ ನಾಶವಾಗುವುದಿಲ್ಲ” ಎಂಬ ಭಾಷೆಯನ್ನೇ ಶ‍್ರೀಕೃಷ್ಣ ಕೊಟ್ಟಿದ್ದಾನೆ. ಆ ಭಾಷೆಯನ್ನು ಅಕ್ಷರಶಃ ನಂಬುತ್ತಾನೆ ಭಕ್ತ. ಅದೊಂದೇ ಅಲ್ಲ. ಇದರ ಆಧಾರದ ಮೇಲೆಯೇ ಅವನು ಹೊರಟಿರುವುದು ಭಗವಂತನೆಂಬ ಗುರಿಯನ್ನು ಸೇರುವುದಕ್ಕೆ.

ಯಾವಾಗ ನೀನು ನನ್ನ ಮಾತನ್ನು ಕೇಳುವುದಿಲ್ಲವೊ, ಆಗ ಹಾಳಾಗುತ್ತೀಯೆ ಎಂದು ಅರ್ಜುನನಿಗೆ ಹೇಳುವನು. ನೀನು ಯುದ್ಧ ಮಾಡು. ಮಾಡುವಾಗ ನನ್ನ ಕೈಯಲ್ಲಿ ನಿಮಿತ್ತನಾಗು ಎಂದು ಶ‍್ರೀಕೃಷ್ಣ ಅರ್ಜುನನಿಗೆ ಹೇಳಿದ್ದು. ಅರ್ಜುನನೇನಾದರೂ ಹಾಗೆ ಯುದ್ಧ ಮಾಡದೆ ಮೊಂಡನಾಗಿ ನಿಂತರೆ ಶ‍್ರೀಕೃಷ್ಣ ಏನನ್ನು ಮಾಡಬೇಕೆಂದಿರುವನೊ ಅದಕ್ಕೆ ಇನ್ನಾರನ್ನೋ ನಿಮಿತ್ತವಾಗಿ ಆರಿಸಿಕೊಂಡು ತನ್ನ ಕೆಲಸವನ್ನು ಪೂರೈಸುತ್ತಾನೆ. ಅದನ್ನೇನು ಅಷ್ಚಕ್ಕೆ ನಿಲ್ಲಿಸುವುದಿಲ್ಲ. ಆದರೆ ನಷ್ಟವುಂಟಾಗುವುದು ಅರ್ಜುನನಿಗೆ. ಉದ್ಧಾರವಾಗುವ ಒಂದು ಅವಕಾಶವನ್ನು ಕಳೆದುಕೊಳ್ಳುವನು.

\begin{verse}
ಯದಹಂಕಾರಮಾಶ್ರಿತ್ಯ ನ ಯೋತ್ಸ್ಯ ಇತಿ ಮನ್ಯಸೇ~।\\ಮಿಥ್ಯೈಷ ವ್ಯವಸಾಯಸ್ತೇ ಪ್ರಕೃತಿಸ್ತ್ವಾಂ ನಿಯೋಕ್ಷ್ಯತಿ \versenum{॥ ೫೯~॥}
\end{verse}

{\small ಅಹಂಕಾರದಿಂದ ನಾನು ಯುದ್ಧ ಮಾಡುವುದಿಲ್ಲ ಎಂದು ಭಾವಿಸಿದರೆ ಅದು ಸುಳ್ಳು. ಏಕೆಂದರೆ ಸ್ವಾಭಾವವೇ ನಿನ್ನನ್ನು ಪ್ರೇರೇಪಿಸುವುದು.}

ಅಹಂಕಾರದಿಂದ ನಾನು ಯುದ್ಧಮಾಡುವುದಿಲ್ಲ ಎಂದು ಭಾವಿಸಿದರೆ ಅದು ಸುಳ್ಳು ಎನ್ನುತ್ತಾನೆ. ಅರ್ಜುನನಲ್ಲಿ ವೈರಾಗ್ಯ ತಾತ್ಕಾಲಿಕವಾಗಿ ಬಂದಿರುವುದು, ಅದು ಶಾಶ್ವತವಾಗಿರುವಂತಹುದಲ್ಲ. ಅದನ್ನು ನೆಚ್ಚಿಕೊಂಡರೆ ಆ ವೈರಾಗ್ಯ ಅವನನ್ನು ಹಳ್ಳಕ್ಕೆ ಕೆಡಹುವುದು. ಈಗಲೇನೋ ಯುದ್ಧ ಮಾಡುವುದಿಲ್ಲ ಎಂದು ಶಸ್ತ್ರ ಸಂನ್ಯಾಸ ತೊಟ್ಟು ಹೋಗಬಹುದು. ಕೆಲವು ಕಾಲದ ಮೇಲೆ ಯಾವಾಗ ಜನ ಆಡಿಕೊಳ್ಳುವುದಕ್ಕೆ ಪ್ರಾರಂಭಿಸುವರೊ, ಅದನ್ನು ಎದುರಿಸುವ ಸಾತ್ತಿ ್ವಕ ಶಕ್ತಿಯನ್ನು ಅರ್ಜುನ ಇನ್ನೂ ಸಂಪಾದಿಸಿಕೊಂಡಿಲ್ಲ. ಅವನು ಪುನಃ ರೊಚ್ಚಿದಾಗ ಯುದ್ಧಕ್ಕೆ ಹೋಗುವನು. ಆದರೆ ಈಗಲೇ ಅದನ್ನು ಮಾಡಿ ಹಾಕಿದರೆ ಅನಂತರ ಅದಕ್ಕೆ ಪಶ್ಚಾತ್ತಾಪ ಪಡಬೇಕಾಗಿಲ್ಲ. ಅರ್ಜುನನಿಗಿಂತ ಹೆಚ್ಚಾಗಿ ಅವನ ಮನಸ್ಸನ್ನು ಶ‍್ರೀಕೃಷ್ಣ ಚೆನ್ನಾಗಿ ಬಲ್ಲ. ರೋಗಿಗೆ ತನ್ನ ರೋಗ ಏನೆಂಬುದು ಗೊತ್ತಿಲ್ಲ. ಆದರೆ ಅವನನ್ನು ಪರೀಕ್ಷಿಸುವ ವೈದ್ಯನಿಗೆ ಈ ರೋಗದ ಸುಳಿವೆಲ್ಲಾ ಚೆನ್ನಾಗಿ ಗೊತ್ತಿದೆ. ರೋಗಿಯು ಅದನ್ನು ಅನುಭವಿಸುತ್ತಿರಬಹುದು. ಆದರೆ ಆ ರೋಗದ ಸ್ವಭಾವಗಳಾವುವೂ ಗೊತ್ತಿಲ್ಲ. ವೈದ್ಯನಿಗಾದರೊ, ಆ ರೋಗದ ಸುಳಿವೆಲ್ಲಾ ಗೊತ್ತಿದೆ. ಅದು ಹೇಗೆ ಬಂತು ಎಷ್ಟು ದಿವಸ ಇರುವುದು, ಹೇಗೆ ಹೋಗುತ್ತದೆ ಎಂಬುದನ್ನು ವೈದ್ಯ ಕ್ಷಣ ಮಾತ್ರದಲ್ಲಿ ತಿಳಿದುಕೊಳ್ಳಬಲ್ಲ. ಅದರಂತೆಯೇ ಭವವೈದ್ಯ ನಾದ ಶ‍್ರೀಕೃಷ್ಣ ಅರ್ಜುನನ ವೈರಾಗ್ಯ ಎಷ್ಟು ದಿನದ್ದು ಎಂಬುದನ್ನು ಚೆನ್ನಾಗಿ ಬಲ್ಲನು.

ನಾನೇನಾದರೂ ಯುದ್ಧವನ್ನು ಮಾಡುವುದಿಲ್ಲವೆಂದು ಈಗ ಹೇಳಿದರೂ ಸ್ವಭಾವವೇ ಅನಂತರ ಪ್ರೇರೇಪಿಸುವುದು. ಅರ್ಜುನ ಕ್ಷತ್ರಿಯ ವ್ಯಕ್ತಿ. ಮೊದಲಿನಿಂದಲೂ ಅದಕ್ಕೆ ಸಂಬಂಧಪಟ್ಟ ಸಂಸ್ಕಾರ ಗಳನ್ನು ಸಂಗ್ರಹಿಸಿರುವನು. ಅವನು ವೀರ, ಪರಾಕ್ರಮಶಾಲಿ, ಅಗೌರವವನ್ನು ತಾಳುವವನಲ್ಲ. ಚುಚ್ಚು ಮಾತನ್ನು ಸಹಿಸುವವನಲ್ಲ. ತಾತ್ಕಾಲಿಕವಾಗಿ ವೈರಾಗ್ಯ ಬಂದಿರುವುದಷ್ಟೆ. ರೋಗಿಗೆ ಶಸ್ತ್ರ ಚಿಕಿತ್ಸೆ ಮಾಡುವುದಕ್ಕೆ ಮುಂಚೆ ನೋವು ಕಾಣದಂತೆ ಮಾಡುವ ಇಂಜೆಕ್ಷನ್ ಕೊಡುವಂತಿದೆ. ಆಗ ಆ ಸ್ಥಳದಲ್ಲಿ ರೋಗಿಗೆ ನೋವು ಕಾಣುವುದಿಲ್ಲ. ಆದರೆ ಕೆಲವು ಕಾಲದ ಮೇಲೆ ಔಷಧಿ ಪ್ರಭಾವ ತಗ್ಗಿದಾಗ, ಯಾತನೆ ಆರಂಭವಾಗುವುದು. ಇದನ್ನು ಚೆನ್ನಾಗಿ ಬಲ್ಲ ಶ‍್ರೀಕೃಷ್ಣ. ಅರ್ಜುನ ತನ್ನ ಹಿಂದಿನ ಸ್ವಭಾವವನ್ನೆಲ್ಲ ಒಂದು ಪ್ರತಿಜ್ಞೆಯಿಂದ ಬದಲಾಯಿಸುವುದಕ್ಕೆ ಆಗುವುದಿಲ್ಲ. ಆ ಹೊಸ ಪ್ರತಿಜ್ಞೆಯ ಕಿಡಿ ಅವನ ಹಿಂದಿನ ಸ್ವಭಾವವನ್ನೆಲ್ಲ ದಹಿಸುವುದಕ್ಕೆ ಬದಲಾಗಿ ಆ ಕಿಡಿಯೇ ಆರಿ ಹೋಗುವುದು. ಏಕೆಂದರೆ ಅಲ್ಲಿ ಸ್ವಭಾವ ಬೆಂಕಿಗೆ ವಿರೋಧವಾಗಿರುವುದು.

\begin{verse}
ಸ್ವಭಾವಜೇನ ಕೌಂತೇಯ ನಿಬದ್ಧಃ ಸ್ವೇನ ಕರ್ಮಣಾ~।\\ಕರ್ತುಂ ನೇಚ್ಛಸಿ ಯನ್ಮೋಹಾತ್ ಕರಿಷ್ಯಸ್ಯವಶೋಽಪಿ ತತ್ \versenum{॥ ೬ಂ~॥}
\end{verse}

{\small ಅರ್ಜುನ, ಸ್ವಭಾವದಿಂದ ಹುಟ್ಟಿದ ನಿನ್ನ ಕರ್ಮದಿಂದ ಕಟ್ಟಲ್ಪಟ್ಟವನಾಗಿ ಮೋಹದಿಂದ ಯಾವುದನ್ನು ಮಾಡಲು ಇಚ್ಛಿಸುವುದಿಲ್ಲವೊ ಅದನ್ನು ಪರವಶನಾಗಿ ಮಾಡುವೆ.}

ಪ್ರತಿಯೊಬ್ಬನೂ ತನ್ನ ಸ್ವಭಾವದಿಂದ ಹುಟ್ಟಿದ ಕರ್ಮಕ್ಕೆ ಬಂಧಿಯಾಗಿದ್ದಾನೆ. ಆ ಸ್ವಭಾವ ವೆಂಬುದು ನಮ್ಮ ಹಿಂದೆ ಜನ್ಮ ಜನ್ಮಾಂತರಗಳಿಂದಲೂ ಬರುತ್ತಿದೆ ಮತ್ತು ಈಗ ಇದುವರೆಗೆ ಅದಕ್ಕೆ ಪ್ರೋತ್ಸಾಹವನ್ನು ಕೊಟ್ಟಿರುತ್ತೇವೆ. ನಾವು ಎಲ್ಲಿ ಹೋಗಲಿ ಅದು ನಮ್ಮ ದೇಹದ ಛಾಯೆಯಂತೆ ನಮ್ಮನ್ನು ಅನುಸರಿಸಿಕೊಂಡು ಹೋಗುವುದು. ಅದರಿಂದ ತಪ್ಪಿಸಿಕೊಳ್ಳುವುದಕ್ಕೆ ಆಗುವುದಿಲ್ಲ. ಅದರಂತೆ ಕೆಲಸ ಮಾಡಬೇಕು. ಹಾಗೆ ಕೆಲಸ ಮಾಡುವಾಗ ಫಲಾಪೇಕ್ಷೆಯಿಲ್ಲದೆ ಕೆಲಸ ಮಾಡಿದರೆ ಆ ಕರ್ಮ ಕ್ಷಯವಾಗುವುದು.

ಕರ್ಮ ಕ್ಷಯವಾಗುವುದಕ್ಕೆ ಅದೊಂದೆ ದಾರಿ. ಮಾಡಿ ಮುಗಿಸಬೇಕು. ಬಿಟ್ಟರೆ ಆಮೇಲೆ ಬಡ್ಡಿ ಸಮೇತ ತೀರಿಸಬೇಕಾಗುವುದು.

ಯಾವುದನ್ನು ಈಗ ಮೋಹದಿಂದ ಮಾಡಲು ಇಚ್ಛಿಸುವುದಿಲ್ಲವೋ ಅದನ್ನು ಪರವಶನಾಗಿ ಆಮೇಲೆ ಮಾಡಿಯೇ ಮಾಡುತ್ತೀಯೆ. ಈಗ ಅರ್ಜುನ ಗುರುಹಿರಿಯರ ಮೇಲಿನ ವ್ಯಾಮೋಹದಿಂದ, ‘ನಾನು ಯುದ್ಧವನ್ನು ಮಾಡುವುದಿಲ್ಲ’ ಎಂದು ಹೊರಟು ಹೋಗಬಹುದು. ಇವನ ಮನಸ್ಸಿಗೇನೊ ತಾತ್ಕಾಲಿಕವಾಗಿ ಸಮಾಧಾನವಾಗಬಹುದು, ‘ನಾನು ಅವರನ್ನು ಕೊಂದು ಪಾಪ ಕಟ್ಟಿಕೊಳ್ಳಲಿಲ್ಲವಲ್ಲ’ ಎಂದು. ಆದರೆ ಅರ್ಜುನ ಸುಮ್ಮನಿರುವಾಗ, ಜನರು ತಮ್ಮ ಮನಸ್ಸಿಗೆ ಬಂದಂತೆ ಆಡಿಕೊಳ್ಳುವರು. ಅದರಲ್ಲಿಯೂ ಕೌರವರಂತೂ ತುಂಬಾ ತುಚ್ಛವಾಗಿಯೇ ಅವನನ್ನು ಆಡಿಕೊಳ್ಳುವರು. ಅವರು, ಅರ್ಜುನ, ಕರುಣೆಯಿಂದ ಯುದ್ಧಮಾಡುವುದನ್ನು ಬಿಟ್ಟ ಎನ್ನದೆ, ಅಂಜಿಕೆಯಿಂದ ಬಿಟ್ಟ ಎನ್ನುತ್ತಾರೆ. ಮೊದಮೊದಲು ಇದನ್ನು ತಾತ್ಸಾರವಾಗಿ ಕಾಣಬಹುದು. ಆದರೆ ಜನರ ಟೀಕೆ ಬಲವಾದಂತೆ ಶಾಂತಿಯೂ ತಣ್ಣಗಾಗುವುದು. ಅದು ಕುದಿಯತೊಡಗುವುದು. ಆಮೇಲೆ ಅವನು ಕೋಪದಿಂದ ಪ್ರೇರಿತನಾಗಿ, “ನಾನು ಹೇಡಿಯಂತೆ ಬಿಟ್ಟು ಹೋಗಲಿಲ್ಲ. ನೋಡಿ ನಾನು ಈಗ ಯುದ್ಧಕ್ಕೆ ಸಿದ್ಧನಾಗಿದ್ದೇನೆ” ಎಂದು ಮುಂದೆ ಬರಬಹುದು. ಆಗ ಯುದ್ಧವನ್ನು ಮಾಡಲೇ ಬೇಕಾಗುವುದು. ಆಗ ಅದು ಹಾಸ್ಯಾಸ್ಪದವಾಗುವುದು. ಅದರ ಬದಲು ಈಗಲೇ ಯುದ್ಧವನ್ನು ಮಾಡಿಬಿಟ್ಟರೆ ಶ‍್ರೀಕೃಷ್ಣನ ಬುದ್ಧಿವಾದವನ್ನು ಮೀರಿದಂತೆಯೂ ಆಗಲಿಲ್ಲ, ತನ್ನ ಸ್ವಭಾವಕ್ಕೆ ಅನುಗುಣವಾಗಿಯೂ ವರ್ತಿಸಿ ದಂತಾಗುವುದು. ಕ್ಷತ್ರಿಯನ ಕರ್ತವ್ಯವನ್ನು ನೇರವೇರಿಸಿದಂತಾಗುವುದು. ಧರ್ಮಯುದ್ಧವನ್ನು ಮಾಡುವುದು ಕ್ಷತ್ರಿಯನ ಕರ್ತವ್ಯ. ಅದಕ್ಕೆ ಈಗೊಂದು ಆವಕಾಶ ಬಂದಿದೆ. ಅದನ್ನು ಬಿಟ್ಟು ಬಿಟ್ಟು ಅನಂತರ ಯಾವುದೋ ಕೋಪಕ್ಕೆ ತುತ್ತಾಗಿ ಮಾಡಿದಾಗ ಹಾಸ್ಯಾಸ್ಪದವಾಗುವುದು. ಈಗ ಆ ಯುದ್ಧಕ್ಕೆ ಇರುವ ಬೆಲೆ ಆಗ ಇರುವುದಿಲ್ಲ.

\begin{verse}
ಈಶ್ವರಃ ಸರ್ವಭೂತಾನಾಂ ಹೃದ್ದೇಶೇಽಜುRನ ತಿಷ್ಠತಿ~।\\ಭ್ರಾಮಯನ್ ಸರ್ವಭೂತಾನಿ ಯಂತ್ರಾರೂಢಾನಿ ಮಾಯಯಾ \versenum{॥ ೬೧~॥}
\end{verse}

{\small ಅರ್ಜುನ, ಸರ್ವಪ್ರಾಣಿಗಳನ್ನು ಯಂತ್ರಾರೂಢವಾದ ಬೊಂಬೆಯಂತೆ ಮಾಯೆಯಿಂದ ತಿರುಗಿಸುತ್ತ ಈಶ್ವರನು ಸರ್ವ ಜೀವಿಗಳ ಹೃದಯದಲ್ಲಿ ನೆಲೆಸಿರುತ್ತಾನೆ.}

ಭಗವಂತನೇ ಸರ್ವಜೀವರ ಹೃದಯದಲ್ಲಿ ನೆಲೆಸಿದ್ದಾನೆ. ಅವನು ತನ್ನ ಕೆಲಸವನ್ನು ಮಾನವರ ಮೂಲಕ ಮಾಡಿಸುತ್ತಿರುವನು. ಹೇಗೆ ತೊಗಲುಗೊಂಬೆಗಳು ದಾರ ಎಳೆದಂತೆ ತಿರುಗುವುವೋ ಹಾಗೆ ಮನುಷ್ಯ ಭಗವಂತನ ಇಚ್ಛೆಯಂತೆ ಕೆಲಸ ಮಾಡುತ್ತಾನೆ. ಅವನು ಎಲ್ಲರ ಹೃದಯದಲ್ಲಿಯೂ ಕರ್ತವ್ಯ, ಪ್ರೀತಿ ಮುಂತಾದ ಗುಣಗಳನ್ನೆಲ್ಲ ಇಟ್ಟು ತನ್ನ ಕೆಲಸವನ್ನು ಮಾಡಿಸುತ್ತಿರುವನು. ಮಕ್ಕಳಿಲ್ಲದವರು ಮಕ್ಕಳನ್ನು ಇಚ್ಛಿಸುವುದು ಆ ಮಾಯೆಯಿಂದ. ಮಕ್ಕಳನ್ನು ಪ್ರೀತಿಸುವುದು, ಅವರಿಗೆ ಕಷ್ಟಪಡುವುದು, ಅವರನ್ನು ಮುಂದೆ ತರುವುದು ಆ ಮಾಯೆಯಿಂದ. ಒಬ್ಬನಿಗೆ ದೇಶದ ಮೇಲೆ ಪ್ರೀತಿ ಇರುವುದು ಆ ಮಾಯೆಯಿಂದ. ಧರ್ಮದ ಮೇಲೆ ಪ್ರೀತಿ ಇರುವುದು ಆ ಮಾಯೆಯಿಂದ. ಪ್ರತಿಯೊಂದು ವರ್ಣದವರಿಗೂ ಆಯಾ ಕೆಲಸದಲ್ಲಿ ಆಕರ್ಷಣೆ ಇರುವುದು ಇದರಿಂದ. ಇದಕ್ಕೆ ವಶನಾಗಿ ಕರ್ಮವನ್ನು ಮಾಡುತ್ತಾನೆ. ಅಜ್ಞಾನದಿಂದ ನಾನು ಆ ಕೆಲಸವನ್ನು ಮಾಡುತ್ತೇನೆ ಎಂದು ತಿಳಿಯುವನು. ಭಗವಂತನ ಕೆಲಸ ಆಗುವುದಕ್ಕೆ ನಾವೆಲ್ಲ ಒಂದು ನಿಮಿತ್ತ. ನಾವು ನಿಮಿತ್ತ ಆಗುವುದಕ್ಕೆ ಹಿಂಜರಿದರೆ ಅವನ ಕೆಲಸ ನಿಂತು ಹೋಗುವುದಿಲ್ಲ, ಅವನು ಮತ್ತಾರನ್ನೋ ನಿಮಿತ್ತವಾಗಿ ಮಾಡಿಕೊಂಡು ತನ್ನ ಕೆಲಸವನ್ನು ಪೂರೈಸುತ್ತಾನೆ. ಆದರೆ ನಷ್ಟ, ನಾನು ಮಾಡುವುದಿಲ್ಲ ಎಂದನಲ್ಲ ಅವನಿಗೆ. ಅವನಾದರೊ ಆಮೇಲೆ ಆ ಕೆಲಸವನ್ನು ಮಾಡದೆ ಇರು ತ್ತಾನೆಯೆ? ಇಲ್ಲ. ಮಾಡಿಯೇ ಮಾಡುತ್ತಾನೆ. ಈಗ ಅದಕ್ಕೆ ಇರುವ ಪ್ರಾಮುಖ್ಯತೆ ಆಗ ಇರುವುದಿಲ್ಲ. ಊಟಕ್ಕೆ ಏಳೊ ಎಂದು ಮಗುವಿಗೆ ಹೇಳಿದಾಗ, ನಾನು ಊಟವನ್ನೇ ಮಾಡುವುದಿಲ್ಲ ಎಂದು ಹೇಳಿ, ಸ್ವಲ್ಪ ಹೊತ್ತಾದ ಮೇಲೆ ನನಗೆ ಹೊಟ್ಟೆ ಹಸಿವು, ಊಟ ಕೊಡು, ಎಂದು ಕೇಳುವುದು. ಆಗ ಎಲ್ಲರೂ ಹಾಸ್ಯ ಮಾಡುವರು. ಊಟ ಬೇಕಾಗಿಲ್ಲ, ಊಟ ಮಾಡದೆ ಸಾಯುತ್ತೇನೆ ಎಂದು ಹೇಳಿದೆಯಲ್ಲ. ಸಾಯಿ ನೋಡೋಣ ಎಂದು ಹಾಸ್ಯ ಮಾಡುವರು.

\begin{verse}
ತಮೇವ ಶರಣಂ ಗಚ್ಛ ಸರ್ವಭಾವೇನ ಭಾರತ~।\\ತತ್ಪ್ರಸಾದಾತ್ಪರಾಂ ಶಾಂತಿಂ ಸ್ಥಾನಂ ಪ್ರಾಪ್ಸ್ಯಸಿ ಶಾಶ್ವತಮ್ \versenum{॥ ೬೨~॥}
\end{verse}

{\small ಅರ್ಜುನ, ಎಲ್ಲಾ ರೀತಿಯಿಂದಲೂ ನೀನು ಅವನನ್ನೇ ಶರಣು ಹೊಂದು. ಅವನ ಅನುಗ್ರಹದಿಂದ ಪರಮಶಾಂತಿ ಯನ್ನೂ ಶಾಶ್ವತವಾದ ಪದವಿಯನ್ನೂ ಹೊಂದುವೆ.}

ಶ‍್ರೀಕೃಷ್ಣ ಅರ್ಜುನನಿಗೆ ಎಲ್ಲ ರೀತಿಯಿಂದಲೂ ಭಗವಂತನಲ್ಲಿ ಶರಣಾಗು ಎನ್ನುವನು. ಕೆಲಸ ವನ್ನು ತಮೋರೀತಿ ಮಾಡಬಹುದು. ಯಶಸ್ಸಿಗೆ, ಕೀರ್ತಿಗೆ ವಶನಾಗಿ, ರಜೋಗುಣಕ್ಕೆ ಬದ್ಧನಾಗಿ ಮಾಡಬಹುದು. ಅದೇ ಕೆಲಸವನ್ನು ನನಗೆ ಏನೂ ಬೇಕಾಗಿಲ್ಲ, ಇದೊಂದು ಧರ್ಮಯುದ್ಧ ನನ್ನ ಪಾಲಿಗೆ ಬಂದಿದೆ, ಅದನ್ನು ನಾನು ಮಾಡುತ್ತೇನೆ, ಇದರಿಂದ ನನಗೆ ಯಾವ ಫಲವೂ ಬೇಕಾಗಿಲ್ಲ, ಎಲ್ಲವೂ ಶ‍್ರೀಕೃಷ್ಣಾರ್ಪಣವಾಗಲಿ ಎಂಬ ದೃಷ್ಟಿಯಿಂದ ಮಾಡಬಹುದು. ನಾನು ಅವನಲ್ಲಿ ಶರಣಾಗಿ ದ್ದೇನೆ, ನಾನೊಂದು ನಿಮಿತ್ತ. ಕೆಲಸವನ್ನೆಲ್ಲಾ ಮಾಡಿಸುವಾತನು ಅವನೆ, ಎಂದು ಜವಾಬ್ದಾರಿಯನ್ನೆಲ್ಲ ಅವನಿಗೆ ವಹಿಸಬಿಡಬಹುದು.

ಯಾವಾಗ ಒಬ್ಬ ಭಗವಂತನಲ್ಲಿ ಸಂಪೂರ್ಣ ಶರಣಾಗತನಾಗುವನೊ ಆಗಲೆ ಅವನ ಕೃಪೆ ನಮ್ಮ ಮೇಲೆ ಬೀಳಬಲ್ಲದು. ಶರಣಾಗತಿ, ನಮಗೂ ಅವನಿಗೂ ನಡುವೆ ಇರುವ ಅಂತರವನ್ನು ನೀಗಿಸು ವುದು. ನಾನು ಎಂಬುದು ಒಂದು ಅಡಚಣೆ. ನಾನು ಎಂಬುದನ್ನು ಭಗವಂತನಡಿಯಲ್ಲಿ ಉರುಳಿಸಿ ದಾಗ ದೇವರ ಕೃಪೆ ದೊರಕುವುದು. ವಿದ್ಯುತ್ ಶಕ್ತಿ ಒಂದು ಕಡೆಯಿಂದ ಮತ್ತೊಂದು ಕಡೆಗೆ ಹರಿಯುವಾಗ ಅಡಚಣೆ ಇರಕೂಡದು. ಅಡಚಣೆ ಇದ್ದರೆ ಹರಿಯಲಾರದು. ಅಡಚಣೆ ತೊಲಗಿ ದೊಡನೆ ಹರಿಯಲಾರಂಭಿಸುವುದು. ಈ ಕ್ಷುದ್ರ ನಾನು ‘ನನ್ನ ಸಮಾನರಿಲ್ಲ’ ಎಂದು ಎದ್ದು ನಿಂತಿರುವಾಗ ದೇವರನ್ನು ಕಾಣದಂತೆ ಮಾಡುವುದು. ಅವನ ಶಕ್ತಿ ಬರದಂತೆ ತಡೆಯುವುದು. ಯಾವಾಗ ಅದು ಕೆಳಕ್ಕೆ ಬಾಗುವುದೊ ಆಗ ತೂಬನ್ನು ತೆರೆದರೆ ಕೆರೆಯಿಂದ ನೀರು ಹೊರಗೆ ಬರುವಂತೆ ಭಗವಂತನ ಕೃಪೆ ಇವನೆಡೆಗೆ ಹರಿಯುವುದು.

ಅವನ ಅನುಗ್ರಹದಿಂದ ಪರಮಶಾಂತಿ ಬರುವುದು. ಈಗ ಅರ್ಜುನ ಯುದ್ಧ ಮಾಡದೆ ಹೋದರೆ ಪರಮಶಾಂತಿ ಬರುವುದೆಂದು ಭಾವಿಸಿದ್ದಾನೆ. ಆದರೆ ಈಗಿನ ಶಾಂತಿ ಚಂಡಮಾರುತ ಬರುವುದಕ್ಕೆ ಮೊದಲಿರುವ ಶಾಂತಿಯಂತಿದೆ. ಆದರೆ ಅರ್ಜುನ ತನ್ನ ಪಾಲಿಗೆ ಬಂದ ಕರ್ತವ್ಯವನ್ನು ಮಾಡಿದರೆ, ಅನಂತರ ಅವನಿಗೆ ದೊರಕುವ ಶಾಂತಿ ಎಂದಿಗೂ ಅವನನ್ನು ಬಿಡದ ಶಾಂತಿ. ಆಗ ಅವನು ಭಗವಂತನ ಕೈಯಲ್ಲಿ ನಿಮಿತ್ತವಾಗಿ ಅಧರ್ಮದ ಕಳೆಯನ್ನು ಕಿತ್ತು ಧರ್ಮ ಸ್ಥಾಪನೆಗೆ ಸಹಾಯಕನಾಗುವನು. ಭಗವಂತ ತನ್ನ ಕೆಲಸವನ್ನು ಯಾರು ತಿಳಿದು ಮಾಡುತ್ತಾರೊ ಅವರಿಗೆ ತನ್ನಲ್ಲಿರುವುದನ್ನು ಕೊಡುತ್ತಾನೆ.ಅದೇ ಪರಮಶಾಂತಿ. ಈ ಪ್ರಪಂಚದ ಯಾವ ಘಟನೆಗಳೂ ಅವನ ಶಾಂತಿಗೆ ಭಂಗವನ್ನು ತರಲಾರವು. ಏಕೆಂದರೆ ಅವನ ಆಧಾರ ಭಗವಂತನಾಗುವನು, ಈ ಪ್ರಪಂಚವಲ್ಲ.

ಅವನು ಶಾಶ್ವತವಾದ ಸ್ಥಾನವನ್ನು ಕೊಡುತ್ತಾನೆ. ಅದೇ ಮುಕ್ತಿ. ಸೃಷ್ಟಿ ಸಮಯದಲ್ಲಿ ಹುಟ್ಟಬೇಕಾಗಿಲ್ಲ. ಮರಣ ಸಮಯದಲ್ಲಿ ಸಾಯಬೇಕಾಗಿಲ್ಲ. ಇಡೀ ಬ್ರಹ್ಮಾಂಡದ ಲೀಲೆಯನ್ನು ಸಾಕ್ಷಿಯಂತೆ ನಿಂತು ನೋಡಬಲ್ಲ. ಸಚ್ಚಿದಾನಂದ ಸ್ವರೂಪನಾದ ಭಗವಂತನಲ್ಲಿ ಒಂದಾಗಬಲ್ಲ.

\begin{verse}
ಇತಿ ತೇ ಜ್ಞಾನಮಾಖ್ಯಾತಂ ಗುಹ್ಯಾದ್ಗುಹ್ಯತರಂ ಮಯಾ~।\\ವಿಮೃಶ್ಯೈತದಶೇಷೇಣ ಯಥೇಚ್ಛಸಿ ತಥಾ ಕುರು \versenum{॥ ೬೩~॥}
\end{verse}

{\small ಗುಹ್ಯಕ್ಕಿಂತಲೂ ಗುಹ್ಯವಾದ ಜ್ಞಾನವನ್ನು ನಿನಗೆ ಹೇಳಿದೆ. ಇದನ್ನು ಸಂಪೂರ್ಣವಿಚಾರಮಾಡಿ ನಿನಗೆ ಹೇಗೆ ತೋರುವುದೋ ಹಾಗೆ ಮಾಡು.}

ಶ‍್ರೀಕೃಷ್ಣ ಗೀತೆಯ ಸಂದೇಶವನ್ನು ಮುಗಿಸುತ್ತಾ ಬರುವನು. ಅತ್ಯಂತ ಗಹನವಾದ ಭಕ್ತಿ, ಕರ್ಮ, ಜ್ಞಾನ ಮತ್ತು ಯೋಗಕ್ಕೆ ಸಂಬಂಧಪಟ್ಟ ವಿಷಯಗಳನ್ನು ಇಲ್ಲಿ ಹೇಳಿರುವನು. ಇದು ಸಾಕ್ಷಾತ್ ಪರಮಾತ್ಮನ ಬಾಯಿಂದಲೇ ಬಂದಿರುವುದು. ಆದರೂ ಅರ್ಜುನನಿಗೆ ಇದನ್ನು ಸಂಪೂರ್ಣ ವಿಚಾರಮಾಡಿ ನಿನಗೆ ತೋರಿದಂತೆ ಮಾಡು ಎನ್ನುವನು. ಶ‍್ರೀಕೃಷ್ಣ ತಾನು ಏನು ಹೇಳಿರುವೆನೊ ಅದನ್ನು ವಿಚಾರ ಮಾಡದೆ ತೆಗೆದುಕೊ ಎನ್ನುವುದಿಲ್ಲ. ಅದನ್ನು ವಿಚಾರ ಮಾಡು ಎನ್ನವನು. ನಾವು ಅದನ್ನು ಚೆನ್ನಾಗಿ ತಿಳಿದುಕೊಂಡು ಮಾಡಿದರೆ ನಮಗೆ ಹೆಚ್ಚು ಪ್ರಯೋಜನವಾಗುವುದು. ದೇವರು ನಮ್ಮಲ್ಲಿ ವಿಚಾರ ಶಕ್ತಿಯನ್ನು ಕೊಟ್ಟಿರುವನು. ಅದನ್ನು ಉಪಯೋಗಿಸಿಕೊಂಡು ಹರಿತ ಮಾಡಬೇಕು. ಯಾವಾಗ ನಾವು ಅದನ್ನು ಉಪಯೋಗಿಸದೆ ಹಾಗೆಯೇ ಬಿಡುವೆವೋ ಆಗ ಬುದ್ಧಿ ತುಕ್ಕು ಹಿಡಿದು ಹೋಗವುದು.

ನಿನಗೆ ತೋರಿದಂತೆ ಮಾಡು ಎನ್ನುವನು. ನಾವು ಹೇಳಿದಂತೆ ಮಾಡೇತೀರಬೇಕು ಎನ್ನುವುದಿಲ್ಲ. ಗೀತೆಯಲ್ಲಿ ಜ್ಞಾನ, ಭಕ್ತಿ, ಕರ್ಮ ಎಲ್ಲಾ ಬೋಧನೆಗಳೂ ಇವೆ. ಪ್ರತಿಯೊಬ್ಬನೂ ತನಗೆ ಯಾವುದು ಹಿಡಿಸುವುದೊ ಅದನ್ನು ತೆಗೆದುಕೊಳ್ಳಬೇಕಾಗಿದೆ. ಇನ್ನೊಬ್ಬನು ಅದನ್ನು ಆರಿಸಿಕೊಡಲಾಗುವುದಿಲ್ಲ. ಪ್ರತಿಯೊಬ್ಬನೂ ತನ್ನ ಯೋಗ್ಯತೆ ಮತ್ತು ಅಭಿರುಚಿಗೆ ತಕ್ಕಂತೆ ಆರಿಸಿಕೊಳ್ಳಬೇಕಾಗುವುದು.

\begin{verse}
ಸರ್ವಗುಹ್ಯತಮಂ ಭೂಚಃ ಶೃಣು ಮೇ ಪರಮಂ ವಚಃ~।\\ಇಷ್ಟೋಽಸಿ ಮೇ ದೃಢಮಿತಿ ತತೋ ವಕ್ಷ್ಯಾಮಿ ತೇ ಹಿತಮ್ \versenum{॥ ೬೪~॥}
\end{verse}

{\small ಸರ್ವ ಗುಹ್ಯತಮವಾದ ನನ್ನ ವಚನವನ್ನು ಪುನಃ ಕೇಳು. ನೀನು ನನಗೆ ಅತ್ಯಂತ ಪ್ರಿಯನಾಗಿರುವುದರಿಂದ ನಿನಗೆ ಒಳ್ಳೆಯದನ್ನು ಹೇಳುತ್ತೇನೆ.}

ಈ ಗುಹ್ಯವಾದ ಮಾತನ್ನು ಪುನಃ ನನ್ನಿಂದ ಕೇಳು ಎನ್ನುತ್ತಾನೆ. ಇಲ್ಲಿ ಏನನ್ನು ಹೇಳುತ್ತಾನೆಯೋ ಅದನ್ನು ಅನೇಕ ವೇಳೆ ಬೇರೆ ಬೇರೆ ಸನ್ನಿವೇಶಗಳಲ್ಲಿ ಹೇಳಿರುವನು. ಆದರೂ ಕೊನೆಗೆ ಮತ್ತೊಮ್ಮೆ ಹೇಳಿಬಿಡುತ್ತೇನೆ ಎನ್ನುತ್ತಾನೆ. ಕೊನೆಗಾದರೂ ಅದು ಮನಸ್ಸಿನಲ್ಲಿ ನಿಲ್ಲಲಿ ಎಂದು. ಇಲ್ಲಿ ಪುನರುಕ್ತಿ ಒಂದು ದೋಷವಲ್ಲ. ಜೀವನದಲ್ಲಿ ನಾವು ಇದನ್ನು ಪದೇ ಪದೇ ಕೇಳುತ್ತಿರಬೇಕು. ಒಂದಲ್ಲ ಒಂದು ಸಲ ನಮ್ಮ ಮನಸ್ಸಿಗೆ ತಾಕುವುದು ಮತ್ತು ಯಾವುದನ್ನು ಕೊನೆಗೆ ಹೇಳಿ ಮುಗಿಸುತ್ತಾನೆಯೋ ಅದು ಮನಸ್ಸಿನಲ್ಲಿಯೇ ಕೊರೆಯುತ್ತಿರುವುದು. ಎಲ್ಲಾ ಭಾವನೆಗಳಿಗಿಂತ ಹೆಚ್ಚಾಗಿ ಇದು ಮನಸ್ಸಿನ ಮೇಲೆ ಪರಿಣಾಮವನ್ನು ಉಂಟು ಮಾಡುವುದು.

ನೀನು ನನಗೆ ಪ್ರಿಯನಾದುದರಿಂದ ಹಿತವಚನವನ್ನು ಹೇಳುತ್ತೇನೆ ಎನ್ನುತ್ತಾನೆ. ದೇವರು ತನಗೆ ಯಾರು ಪ್ರಿಯರೊ ಅವರಿಗೆ ಸರ್ವ ಶ್ರೇಷ್ಠವಾದುದನ್ನು ಕೊಡುವನು. ಅರ್ಜುನ ಶ‍್ರೀಕೃಷ್ಣನ ಪ್ರಾಣಸಖನಾಗಿದ್ದ ಇದುವರೆಗೆ. ಗೀತೋಪದೇಶ ಪ್ರಾರಂಭವಾದ ಮೇಲೆ ಅವನ ಶಿಷ್ಯನಾಗಿ ಅವನಲ್ಲಿ ಶರಣಾಗಿದ್ದಾನೆ. ಯಾರು ಭಗವಂತನಲ್ಲಿ ಶರಣಾಗುವರೋ ಅವರನ್ನು ಕಂಡರೆ ದೇವರಿಗೆ ಪ್ರೀತಿ. ಏಕೆಂದರೆ ಅವರು ಎಲ್ಲರನ್ನೂ ಬಿಟ್ಟು ತನ್ನಡೆಗೆ ಬಂದಿದ್ದಾರೆ. ಎಲ್ಲಾ ಜವಾಬ್ದಾರಿಯನ್ನೂ ತನ್ನೆಡೆಗೆ ಹಾಕಿದ್ದಾರೆ. ಆದಕಾರಣ ಯಾವುದರಿಂದ ಇವರು ಉದ್ಧಾರವಾಗುತ್ತಾರೊ ಅದನ್ನು ಹೇಳಬೇಕಾಗಿದೆ. ಜೀವನದಲ್ಲಿ ಹಲವಾರು ಒಳ್ಳೆಯ ವಿಷಯಗಳಿವೆ. ಆದರೆ ಎಲ್ಲವೂ ಎಲ್ಲರಿಗೂ ಒಳ್ಳೆಯದಲ್ಲ. ಏಕೆಂದರೆ ಕೆಲವನ್ನು ಅವರು ತಿಳಿದುಕೊಳ್ಳಲಾರರು. ತಪ್ಪು ತಿಳಿದುಕೊಂಡರೆ ಅಪಾಯ ಜಾಸ್ತಿ. ಶ‍್ರೀರಾಮಕೃಷ್ಣರು ಶಿಷ್ಯನ ಯೋಗ್ಯತೆಯನ್ನು ನೋಡಿ ಉಪದೇಶ ಮಾಡಬೇಕು ಎನ್ನುತ್ತಾರೆ. ಹಾಲೇನೋ ಒಳ್ಳೆಯದು. ಆದರೆ ಅದನ್ನು ಒಬ್ಬೊಬ್ಬರಿಗೆ ಒಂದೊಂದು ರೀತಿ ಕೊಡಬೇಕಾಗಿದೆ. ಮಗುವಿಗೆ ನೀರು ಹಾಕಿ ಕೊಡುತ್ತಾರೆ. ಪೈಲ್ವಾನ್ ಹಾಲನ್ನು ಕುದುಸಿ ಬಾಸುಂದಿ ಮಾಡಿ ತಿನ್ನುತ್ತಾನೆ. ಟೈಫಾಯಿಡ್ ರೋಗಿಗೆ ಹಾಲನ್ನು ಒಡೆದು ಗಟ್ಟಿಯನ್ನೆಲ್ಲಾ ಹೊರಗೆ ಹಾಕಿ, ಅದರ ನೀರನ್ನು ಮಾತ್ರ ಕೊಡಬೇಕು. ಆಮಶಂಕೆ ಮುಂತಾದ ರೋಗಿಗಳಿಗೆ ಹಾಲು ಕೆಟ್ಟದ್ದು. ಅದನ್ನು ಮೊಸರು ಮಾಡಿ ಬೆಣ್ಣೆ ತೆಗೆದು ಬರೀ ಮಜ್ಜಿಗೆ ಕೊಡಬೇಕು. ಇಲ್ಲಿ ಹಾಲೇನೊ ಒಂದೇ. ಆದರೆ ಒಂದೊಂದು ವ್ಯಕ್ತಿಗೆ ಯೋಗ್ಯತಾನುಸಾರ ಕೊಡುತ್ತಾರೆ. ಅದರಂತೆಯೇ ಗೀತಾಬೋಧನೆಯಲ್ಲಿ ಬರೀ ಜ್ಞಾನವೂ ಇದೆ, ಧ್ಯಾನವೂ ಇದೆ, ಕರ್ಮವೂ ಇದೆ, ಭಕ್ತಿಯೂ ಇದೆ ಮತ್ತು ಇದನ್ನೆಲ್ಲಾ ಒಂದೊಂದು ಪ್ರಮಾಣದಲ್ಲಿ ಮಿಶ್ರಮಾಡಿದ ಬೋಧನೆಯೂ ಇದೆ. ಇವುಗಳಲ್ಸಿ ಅರ್ಜುನನಿಗೆ ಯಾವುದು ಹಿಡಿಸುವುದು ಎಂಬುದು ಶ‍್ರೀಕೃಷ್ಣನಿಗೆ ಗೊತ್ತಿದೆ. ಅದಕ್ಕಾಗಿಯೇ ಅದನ್ನು ಮತ್ತೊಮ್ಮೆ ಹೇಳುತ್ತಿರುವನು.

\begin{verse}
ಮನ್ಮನಾ ಭವ ಮದ್ಭಕ್ತೋ ಮದ್ಯಾಜೀ ಮಾಂ ನಮಸ್ಕುರು~।\\ಮಾಮೇವೈಷ್ಯಸಿ ಸತ್ಯಂ ತೇ ಪ್ರತಿಜಾನೇ ಪ್ರಿಯೋಽಸಿ ಮೇ \versenum{॥ ೬೫~॥}
\end{verse}

{\small ನನ್ನಲ್ಲಿ ಮನಸ್ಸುಳ್ಳವನಾಗು. ನನ್ನ ಭಕ್ತನಾಗು. ನನ್ನನ್ನೇ ಆರಾಧಿಸುವವನಾಗು. ನನ್ನನ್ನು ನಮಸ್ಕರಿಸು. ನನ್ನನ್ನೇ ಸೇರುವೆ, ಸತ್ಯವಾಗಿ ಪ್ರತಿಜ್ಞೆ ಮಾಡುತ್ತೇನೆ; ಏಕೆಂದರೆ ನೀನು ನನಗೆ ಪ್ರಿಯನಾಗಿರುವೆ.}

ಜೀವನದ ಪ್ರಯಾಣದಲ್ಲಿ ಮನುಷ್ಯನಿಗೆ ಒಂದು ಊರುಗೋಲು ಬೇಕು. ಭಗವಂತನ ಯಾವು ದಾದರೊಂದು ಇಷ್ಟದಲ್ಲಿ ಮನಸ್ಸನ್ನು ಇಟ್ಟು ಅದನ್ನು ಆಧಾರವಾಗಿಟ್ಟುಕೊಂಡು ಮುಂದುವರಿ ಎಂದು ಹೇಳುತ್ತಾನೆ. ಅಥವಾ ನಾವು ಇಲ್ಲಿ ಶ‍್ರೀಕೃಷ್ಣ ‘ನಾನು’ ಎಂದು ಉಪಯೋಗಿಸುವಾಗ ದೇವರಲ್ಲಿ ಮನಸ್ಸಿಡು, ದೇವರ ಭಕ್ತನಾಗು, ಅವನನ್ನು ಆರಾಧಿಸು ಎಂದು ಬೇಕಾದರೂ ತೆಗೆದುಕೊಳ್ಳಬಹುದು. ಎಲ್ಲರೂ ಶ‍್ರೀಕೃಷ್ಣ ಎಂಬ ವ್ಯಕ್ತಿಯನ್ನೇ ಆಧಾರವಾಗಿಟ್ಟುಕೊಳ್ಳಬೇಕಾಗಿಲ್ಲ. ಅವನನ್ನು ಯಾವ ಹೆಸರಿನಿಂದ ಕರೆದರೂ ಎಲ್ಲವೂ ಅವನಿಗೇ ಸೇರುವುದು. ಇಲ್ಲಿ ಶ‍್ರೀಕೃಷ್ಣನೂ ಒಂದೇ ಪರಮಾತ್ಮನೂ ಒಂದೇ ಎಂಬ ದೃಷ್ಟಿಯಿಂದ ಹೇಳುತ್ತಿರುವನು.

ನಾನು ಎಂಬ ಪದದ ಹಿಂದೆಲ್ಲಾ ಭಗವಂತ ಎಂದು ತಂದರೆ, ಅದು ಎಲ್ಲರಿಗೂ ಅನ್ವಯಿಸು ವುದು. ಅದು ಹಿಂದೂವಿಗೂ, ಕ್ರೈಸ್ತನಿಗೂ, ಮಹಮ್ಮದೀಯನಿಗೂ ದೇವರನ್ನು ನಂಬುವ ಎಲ್ಲರಿಗೂ ಅನ್ವಯಿಸುವುದು. ನನ್ನಲ್ಲಿ ಮನಸ್ಸುಳ್ಳವನಾಗು ಎನ್ನುತ್ತಾನೆ. ಮನಸ್ಸನ್ನೆಲ್ಲಾ ಭಗವಂತನಿಂದ ತುಂಬ ಬೇಕು. ಹೇಗೆ ಕೊಡವನ್ನು ಬಾವಿಗೆ ಬಿಟ್ಟು ಅದನ್ನು ನೀರಿನಿಂದ ತುಂಬುವೆವೊ ಹಾಗೆ ನಮ್ಮ ಮನಸ್ಸನ್ನು ವಿಷಯವಸ್ತುಗಳಿಂದ ಖಾಲಿ ಮಾಡಿ ಭಗವಂತನಿಂದ ತುಂಬಬೇಕು.

ನನ್ನ ಭಕ್ತನಾಗು ಎನ್ನುತ್ತಾನೆ. ನಾವು ಭಗವಂತನ ಭಕ್ತರಾಗಬೇಕು. ಈ ಪ್ರಪಂಚದಲ್ಲಿ ಭಗವಂತ ನಲ್ಲದ, ಅವನನ್ನು ಮರೆಸುವ ಎಷ್ಚೊ ಕೆಲಸಕ್ಕೆ ಬಾರದ ವಸ್ತುಗಳಿವೆ. ಹೆಂಡತಿ, ಮನೆ, ಮಕ್ಕಳು, ಕೀರ್ತಿ, ಐಶ್ವರ್ಯ ಮುಂತಾದುವುಗಳೆಲ್ಲ ಇವೆ. ನಾವು ಇವುಗಳ ಭಕ್ತರಾಗಕೂಡದು. ನಮ್ಮ ಜೀವನದಲ್ಲಿ ಭಗವಂತನನ್ನು ಪರಮ ಗುರಿಯನ್ನಾಗಿ ಆರಿಸಿಕೊಳ್ಳಬೇಕು.

ನನ್ನನ್ನು ಆರಾಧಿಸು ಎನ್ನುತ್ತಾನೆ. ಭಗವಂತನನ್ನು ಪೂಜಿಸಬೇಕು, ಪ್ರೀತಿಸಬೇಕು. ಕೊಂಡಾಡ ಬೇಕು. ಅವನು ನಮಗೆ ಕೊಟ್ಚಿರುವ ಎಲ್ಲಾ ಇಂದ್ರಿಯಗಳಿಂದಲೂ ಅವನ ಪೂಜೆ ಮಾಡಬೇಕು. ಆಗ ಆ ಇಂದ್ರಿಯಗಳು ಸಾರ್ಥಕವಾಗುವುವು. ಕೈ ಕಾಲುಗಳನ್ನು ಭಗವಂತನ ಕೆಲಸಕ್ಕೆ ಉಪಯೋಗಿಸ ಬೇಕು. ಕಿವಿಯಿಂದ ಅವನ ಮಹಿಮೆಯನ್ನು ಕೇಳಬೇಕು, ಬಾಯಿಂದ ಅವನ ಮಹಿಮೆಯನ್ನು ಕೊಂಡಾಡಬೇಕು, ಕಣ್ಣಿನಿಂದ ಭಗವಂತನನ್ನು ಎಲ್ಲಾ ರೂಪಗಳ ಹಿಂದೆಯೂ ನೋಡಬೇಕು. ಮನಸ್ಸಿನಿಂದ ಅವನನ್ನು ತುಂಬಿಕೊಳ್ಳಬೇಕು. ಆಗ ಕರ್ಮೇಂದ್ರಿಯ ಜ್ಞಾನೇಂದ್ರಿಯಗಳೆಲ್ಲಾ ಸಾರ್ಥಕ ವಾಗುವುವು.

ನನಗೆ ನಮಸ್ಕಾರ ಮಾಡು ಎನ್ನುತ್ತಾನೆ. ಎಂದರೆ ನಾವು ಜೀವನದಲ್ಲಿ ಎಷ್ಟೋ ಕೆಲಸಕ್ಕೆ ಬಾರದ ಮನುಷ್ಯರಿಗೆಲ್ಲಾ ತಲೆ ತಗ್ಗಿಸುತ್ತೇವೆ, ಅವರನ್ನು ಹೊಗಳುತ್ತೇವೆ. ಇಲ್ಲಿ ಭಗವಂತನಿಗೆ ನಮಸ್ಕಾರ ಮಾಡು ಎನ್ನುತ್ತಾನೆ. ಅಹಂಕಾರದಿಂದ ನಿಂತಿರುವ ಮನುಷ್ಯ ವಿನೀತನಾಗಿ ಬಾಗಬೇಕು. ನಲ್ಲಿಯ ಕೆಳಗೆ ಪಾತ್ರೆ ಇದ್ದರೆ ಅದರಲ್ಲಿ ನೀರು ತುಂಬುವುದು. ನಲ್ಲಿಯ ಮೇಲಿದ್ದರೆ ನೀರು ಬರಲಾರದು. ಇಲ್ಲಿ ಭಗವಂತನ ಸಮೀಪದಲ್ಲಿ ದೈನ್ಯತೆಯಿಂದಿರಬೇಕು ಎಂದರ್ಥ.

ನೀನು ಯಾವಾಗ ಇದನ್ನೆಲ್ಲಾ ಮಾಡುತ್ತೀಯೋ ಆವಾಗ ನನ್ನನ್ನೇ ಸೇರುವೆ ಎನ್ನುತ್ತಾನೆ. ಜೀವನದಲ್ಲಿ ಯಾರು ಯಾವುದನ್ನು ಪ್ರೀತಿಸುತ್ತಾರೊ ಅದನ್ನು ಸೇರುತ್ತಾನೆ. ಯಕ್ಷ, ರಾಕ್ಷಸ, ಪಿತೃ ಮುಂತಾದುವನ್ನು ಪ್ರೀತಿಸುವವರು ಅವರನ್ನು ಸೇರುತ್ತಾರೆ. ನನ್ನ ಭಕ್ತ ನನ್ನನ್ನು ಸೇರುತ್ತಾನೆ ಎಂದು ಹಿಂದೆ ಹೇಳಿರುವನು.

ನೀನು ನನ್ನನ್ನೇ ಸೇರುತ್ತೀಯೆ ಎಂದು ಪ್ರತಿಜ್ಞೆ ಮಾಡುತ್ತಾನೆ. ಇದನ್ನು ಕೇವಲ ಅಲಂಕಾರಿಕ ಭಾಷೆ, ಉಪಚಾರದ ಮಾತು ಎಂದು ಅರ್ಜುನ ಭಾವಿಸದೆ ಇರಲಿ ಎಂದು. ಇದು ಪರಮ ಸತ್ಯ. ಇದರಲ್ಲಿ ಎಳ್ಳಷ್ಟೂ ಸಂದೇಹವಿಲ್ಲ. ಇರುವಾಗ ನಮ್ಮ ಮನಸ್ಸನ್ನು ಅವನಿಂದ ತುಂಬಿದ್ದರೆ ಕೊನೆಗೆ ನಾವು ಅವನಲ್ಲಿಗೇ ಹೋಗುತ್ತೇವೆ. ಭಗವಂತನಿಗೆ ಅವನು ತುಂಬಾ ಪ್ರಿಯನಾದುದರಿಂದ ಈ ಸತ್ಯವನ್ನು ಹೇಳುತ್ತಾನೆ.

\begin{verse}
ಸರ್ವಧರ್ಮಾನ್ ಪರಿತ್ಯಜ್ಯ ಮಾಮೇಕಂ ಶರಣಂ ವ್ರಜ~।\\ಅಹಂ ತ್ವಾ ಸರ್ವಪಾಪೇಭ್ಯೋ ಮೋಕ್ಷಯಿಷ್ಯಾಮಿ ಮಾ ಶುಚಃ \versenum{॥೬೬~॥}
\end{verse}

{\small ಸರ್ವ ಧರ್ಮಗಳನ್ನು ಪರಿತ್ಯಾಗ ಮಾಡಿ ನನ್ನೊಬ್ಬನನ್ನೇ ಶರಣು ಹೊಂದು. ನಾನು ನಿನ್ನನ್ನು ಸರ್ವ ಪಾಪಗಳಿಂದ ಪಾರು ಮಾಡಿಸುವೆನು; ಶೋಕಿಸಬೇಡ.}

ಶ‍್ರೀಕೃಷ್ಣನ ಗೀತಾ ಸಂದೇಶ ಅತ್ಯುನ್ನತ ಮಟ್ಟವನ್ನು ಸೇರುವುದು ಇಲ್ಲಿ. ಅರ್ಜುನ ಒಂದು ಧರ್ಮಸಂಕಟದಲ್ಲಿದ್ದ. ಯುದ್ಧವನ್ನು ಮಾಡಬೇಕೆ, ಬಿಡಬೇಕೆ ಎಂಬುದೇ ಅದು. ಮಾಡುವುದಕ್ಕೂ ಕಾರಣ ಕೊಡಬಹುದು, ಬಿಡುವುದಕ್ಕೂ ಕಾರಣ ಕೊಡಬಹುದು. ಎರಡರಲ್ಲೂ ಯಾವುದನ್ನು ಹಿಡಿಯಬೇಕೆಂಬುದೇ ಪ್ರಶ್ನೆ. ಅದಕ್ಕೆ, ಶ‍್ರೀಕೃಷ್ಣ, ಎಲ್ಲಾ ಹೊರೆ ಹೊಣೆಗಳನ್ನು ಬಿಟ್ಟು ನನ್ನಲ್ಲಿ ಶರಣಾಗು ಎನ್ನುತ್ತಾನೆ. ಭಗವಂತ, ಅವನಿಗೆ ಯಾವುದು ಶ್ರೇಯಸ್ಕರವೋ ಅದನ್ನು ಹೇಳುತ್ತಾನೆ. ಅವನಲ್ಲಿ ಶರಣಾದರೆ ಅವನು ನಮ್ಮನ್ನು ತಪ್ಪು ದಾರಿಗೆ ಒಯ್ಯುವುದಿಲ್ಲ. ಅವನು ಯಾವಾಗಲೂ ಸರಿಯಾದ ದಾರಿಯಲ್ಲೇ ನಡೆಸುತ್ತಾನೆ. ಅವನನ್ನು ಮರೆತಾಗಲೇ ನಾವು ತಪ್ಪು ಹೆಜ್ಜೆ ಇಡುವುದು. ಅವನನ್ನು ಚಿಂತಿಸುತ್ತಿರುವಾಗ, ಅವನಲ್ಲಿ ಶರಣಾಗಿರುವಾಗ, ನಾವು ಪ್ರಯತ್ನಪಟ್ಟರೂ ತಪ್ಪು ಹೆಜ್ಜೆ ಇಡುವುದಕ್ಕಾಗುವುದಿಲ್ಲ. ನಮ್ಮನ್ನು ಸರ್ವ ಪಾಪಗಳಿಂದಲೂ ಪಾರು ಮಾಡುತ್ತೇನೆ ಎನ್ನುತ್ತಾನೆ. ಅರ್ಜುನ ತನಗೆ ಗುರುಹಿರಿಯರ, ಬ್ರಾಹ್ಮಣರ ಕೊಲೆಯ ಪಾಪವೆಲ್ಲ ಸುತ್ತಿಕೊಳ್ಳುವುದು ಎಂದು ಭಾವಿಸಿದ್ದ. ಆದರೆ ಯಾವಾಗ ದೇವರಲ್ಲಿ ಶರಣಾಗುತ್ತಾನೆಯೋ, ಅವನ ಕೈಯಲ್ಲಿ ನಿಮಿತ್ತವಾಗು ವನೊ, ಆಗ ಪಾಪ ಯಾವುದೂ ಅವನಿಗೆ ಬರುವುದಿಲ್ಲ. ಭಗವಂತನ ಕಾರ್ಯಕ್ಕೆ ಅವನೊಂದು ನಿಮಿತ್ತವಾಗುತ್ತಾನೆ. ನಿಜವಾಗಿ ಅವನ ಮೂಲಕ ಕೆಲಸ ಮಾಡುತ್ತಿರುವವನು ಭಗವಂತನೆ. ಅವ ನೆಂದಿಗೂ ತಪ್ಪನ್ನು ಮಾಡುವುದಿಲ್ಲ. ಅವನಿಗೆ ಯಾವ ಆಸೆ ಆಕಾಂಕ್ಷೆಗಳೂ ಇಲ್ಲ. ಲೋಕಕಲ್ಯಾಣ ಒಂದೇ ಅವನ ಗುರಿ. ಅದಕ್ಕೆ ಆತಂಕವಾಗಿರುವುದನ್ನೆಲ್ಲಾ ದಬ್ಬಿ ಆಚೆಗೆ ಎಸೆಯುವನು. ರೈತ ಕಳೆಯನ್ನು ಕಿತ್ತು ಬೆಳೆಯನ್ನು ಹೇಗೆ ಉಳಿಸುತ್ತಾನೆಯೋ ಹಾಗೆ.

ಇನ್ನು ನೀನು ಶೋಕಿಸಬೇಡ ಎನ್ನುತ್ತಾನೆ. ಶೋಕದ ಕಾರಣವನ್ನು ಶ‍್ರೀಕೃಷ್ಣ ನಿವಾರಿಸುತ್ತಾನೆ. ಸ್ವಾರ್ಥತೆ ಇದ್ದರೆ, ಅಜ್ಞಾನ ಇದ್ದರೆ, ಶೋಕ. ಯಾವಾಗ ಭಗವಂತನ ವಾಣಿಯಿಂದ ಈ ಮೋಡಗಳು ಚದುರಿ ಹೋಗುವುವೊ ಅನಂತರ ಸತ್ಯಸೂರ್ಯ ಪ್ರಕಾಶಿಸಿ ವಸ್ತುವಿನ ಯಥಾರ್ಥ ಜ್ಞಾನ ಲಭಿಸುತ್ತದೆ.

\begin{verse}
ಇದಂ ತೇ ನಾತಪಸ್ಕಾಯ ನಾಭಕ್ತಾಯ ಕದಾಚನ~।\\ನ ಚಾಶುಶ್ರೂಷವೇ ವಾಚ್ಯಂ ನ ಚ ಮಾಂ ಯೋಽಭ್ಯಸೂಯತಿ \versenum{॥ ೬೭~॥}
\end{verse}

{\small ಇದನ್ನು ನಾನು ನಿನಗೆ ಹೇಳುತ್ತೇನೆ. ತಪೋರಹಿತನಿಗೆ, ಭಕ್ತನಲ್ಲದವನಿಗೆ ಇದನ್ನು ಹೇಳಕೂಡದು. ಶುಶ್ರೂಷೆ ಮಾಡದವನಿಗೆ ಎಂದಿಗೂ ತಿಳಿಸಬಾರದು. ನನ್ನನ್ನು ದ್ವೇಷಿಸುವವನಿಗೆ ಹೇಳಬಾರದು.}

ಇದನ್ನು ಸಿಕ್ಕಿ ಸಿಕ್ಕದವರಿಗೆಲ್ಲ ಹೇಳಬಾರದು. ಯಾರಿಗೆ ಇದನ್ನು ತಿಳಿದುಕೊಳ್ಳುವ ಯೋಗ್ಯತೆ ಇದೆಯೊ ಅಂತಹವರಿಗೆ ಮಾತ್ರ ಹೇಳಬೇಕು. ಗಹನವಾದ ತತ್ತ್ವವನ್ನು ಸರಿಯಾಗಿ ತಿಳಿದುಕೊಳ್ಳ ಲಾರದವರಿಗೆ ಹೇಳಿದರೆ ಅವರು ಅದನ್ನು ತಪ್ಪು ತಿಳಿದುಕೊಳ್ಳುವರು. ಅದರಿಂದ ಅವರಿಗೆ ಮಾತ್ರ ಅಲ್ಲ ಕೆಟ್ಟದ್ದು, ಲೋಕಹಾನಿಯೂ ಆಗುವುದು. ಗೀತೆಯಲ್ಲಿ ಶ‍್ರೀಕೃಷ್ಣ ಎರಡನೇ ಅಧ್ಯಾಯದಲ್ಲಿ ಆತ್ಮ ಯಾರನ್ನೂ ಕೊಲ್ಲುವುದಿಲ್ಲ, ಕೊಲ್ಲಿಸುವುದೂ ಇಲ್ಲ. ಯಾರಾದರೂ ತಾನೇ ಕೊಂದೆ ಎಂದು ಭಾವಿಸಿದರೆ, ಕೊಲ್ಲಿಸಿಕೊಂಡೆ ಎಂದು ಭಾವಿಸಿದರೆ, ಇಬ್ಬರೂ ಅಜ್ಞಾನಿಗಳು–ಎನ್ನುತ್ತಾನೆ. ಇಂತಹ ಬೋಧನೆಗಳನ್ನು ಅಪಾರ್ಥ ಮಾಡಿಕೊಂಡರೆ, ಹಿಂಸೆಗೆ ಸಹಾಯವಾಗುವುದು. ಆದಕಾರಣವೇ ಮೊದಲು ಅವನ ಯೋಗ್ಯತೆ ನೋಡಬೇಕು.

ಕಲಿತುಕೊಳ್ಳುವವನಿಗೆ ಇರಬೇಕಾದ ಯೋಗ್ಯತೆಗಳಲ್ಲೆಲ್ಲಾ ಮೊದಲನೆಯದೆ ತಪಸ್ಸಿನಿಂದ ಕೂಡಿರ ಬೇಕಾದ್ದು. ಯಾವಾಗ ತಪಸ್ಸಿನಿಂದ ಮನಸ್ಸನ್ನು ಶುದ್ಧಿ ಮಾಡಿಕೊಂಡಿಲ್ಲವೊ, ಅದನ್ನು ಏಕಾಗ್ರ ಮಾಡಿಕೊಂಡಿಲ್ಲವೊ ಅಂತಹ ಮನುಷ್ಯ ಗೀತೆಯ ಸೂಕ್ಷ್ಮ ಸಂದೇಶಗಳನ್ನು ಗ್ರಹಿಸಲಾರ. ಅದರಲ್ಲಿ ಒಂದು ಇದ್ದರೆ ಮತ್ತೊಂದನ್ನು ತಿಳಿಕೊಳ್ಳುವರು. ಅವನು ಭಕ್ತನಾಗಿರಬೇಕು. ಆಗ ಭಗವಂತನ ಸಂದೇಶವನ್ನು ಗೌರವದಿಂದ, ಪೂಜ್ಯ ದೃಷ್ಟಿಯಿಂದ ತೆಗೆದುಕೊಳ್ಳುತ್ತಾನೆ. ಯಾವನು ಭಕ್ತನಲ್ಲವೊ ಅಂತಹವನು ಸಂದೇಶವನ್ನು ಅಸಡ್ಡೆಯಿಂದ ನೋಡುತ್ತಾನೆ. ಇದರಿಂದ ಅವನು ಪ್ರಯೋಜನ ಪಡೆಯುವುದಿಲ್ಲ. ಮಂಗನ ಕೈಗೆ ಮಾಣಿಕ್ಯ ಕೊಟ್ಟಂತಾಗುವುದು. ಅವನಿಗೆ ಹೇಳಿದ್ದೂ ವೃಥಾ ಕಾಲಹರಣ ಮತ್ತು ಕಂಠಶೋಷಣೆಯಾಗುವುದು. ಗುರುವಿಗೆ, ದೇವರಿಗೆ, ಹಿರಿಯರಿಗೆ ಶುಶ್ರೂಷೆ ಮಾಡದವನಿಗೆ ಹೇಳಕೂಡದು ಎನ್ನುತ್ತಾನೆ. ಶುಶ್ರೂಷೆಯಿಂದ ನಾವು ದೈನ್ಯತೆಯನ್ನು ಪಡೆಯುತ್ತೇವೆ. ಹೇಳುವವನ ಮೇಲೆ ನಮಗೆ ಗೌರವಬುದ್ಧಿ ಹುಟ್ಟುವುದು. ಸ್ವೀಕರಿಸುವುದಕ್ಕೆ ನಮ್ಮ ಮನಸ್ಸು ಅಣಿಯಾಗುವುದು. ಆಗ ಹೇಳಿದರೆ ಉತ್ತ ಹೊಲಕ್ಕೆ ಬಿತ್ತಿದಂತೆ ಆಗುವುದು. ಇಲ್ಲದೇ ಇದ್ದರೆ ದಾರಿಯಲ್ಲಿ ಬೀಜವನ್ನು ಚೆಲ್ಲಿದಂತಾಗುವುದು. ಇರುವೆಯೋ, ಹಕ್ಕಿಯೊ ಅದನ್ನು ಕೊಂಡುಹೋಗು ವುದು. ಅದರಿಂದ ಮತ್ತಾವ ಪ್ರಯೋಜನವೂ ಆಗುವುದಿಲ್ಲ.

ಭಗವಂತನನ್ನು ದ್ವೇಷಿಸುವವನಿಗೆ ಇವುಗಳನ್ನು ಹೇಳಕೂಡದು. ಅವನು ಇವನ್ನು ನಿಕೃಷ್ಟ ದೃಷ್ಟಿಯಿಂದ ನೋಡುತ್ತಾನೆ. ಇಲ್ಲಿರುವ ಸಂದೇಶದ ಸಾರವೇ, ಚೆನ್ನಾಗಿ ಕರ್ಮಮಾಡಿದರೆ ಹೇಗೆ ಅವನೆಡೆಗೆ ಹೋಗಬಹುದು ಎಂಬುದನ್ನು ವಿವರಿಸುವುದು. ಯಾರು ದೇವರನ್ನು ದ್ವೇಷಿಸುವನೊ ಅವನಿಗೆ ದೇವರೆಡೆಗೆ ಹೋಗುವ ಸಂದೇಶದಿಂದ ಏನು ಪ್ರಯೋಜನ? ಇವನು ಅದರ ಪ್ರಯೋಜನ ಪಡೆಯಲಾರ. ಯಾವಾಗ ಒಬ್ಬ ಮನುಷ್ಯ ಒಂದು ವ್ಯಕ್ತಿಯನ್ನು ದ್ವೇಷಿಸುತ್ತಾನೆಯೋ ಆಗ ಅವನು ಮಾಡುವುದರಲ್ಲಿ ಹೇಳುವುದರಲ್ಲಿ ತಪ್ಪನ್ನೇ ಕಂಡುಹಿಡಿಯುವನು. ಅವನಿರುವುದು ಒಳ್ಳೆಯದನ್ನು ಸ್ವೀಕರಿಸಿ ಉದ್ಧಾರವಾಗುವುದಕ್ಕಲ್ಲ. ಕೊಟ್ಟಿದ್ದನ್ನು ಹರಿದು ಹಾಕುವನು. ಇಂತಹವನಿಗೆ ಕೊಟ್ಟು ಪ್ರಯೋಜನವಿಲ್ಲ ಎನ್ನುತ್ತಾನೆ.

\begin{verse}
ಯ ಇದಂ ಪರಮಂ ಗುಹ್ಯಂ ಮದ್ಭಕ್ತೇಷ್ವಭಿಧಾಸ್ಯತಿ~।\\ಭಕ್ತಿಂ ಮಯಿ ಪರಾಂ ಕೃತ್ವಾ ಮಾಮೇವೈಷ್ಯತ್ಯಸಂಶಯಃ \versenum{॥ ೬೮~॥}
\end{verse}

{\small ಯಾರು ಈ ಪರಮ ರಹಸ್ಯವನ್ನು ನನ್ನ ಭಕ್ತರಿಗೆ ಹೇಳುತ್ತಾನೆಯೋ ಅವನು ನನ್ನಲ್ಲಿ ಪರಮ ಭಕ್ತಿಯುಳ್ಳವನಾಗಿ ನಿಸ್ಸಂದೇಹವಾಗಿ ನನ್ನನ್ನೇ ಹೊಂದುತ್ತಾನೆ.}

ಅತ್ಯಂತ ರಹಸ್ಯವಾದ ಬೋಧನೆಯನ್ನು ಯೋಗ್ಯರಾದವರಿಗೆ ಹಂಚುವುದು ತಿಳಿದಿರುವವನ ಒಂದು ಕರ್ತವ್ಯ. ಅದನ್ನು ತಾನು ಮೊದಲು ಚೆನ್ನಾಗಿ ತಿಳಿದುಕೊಳ್ಳಬೇಕು. ಅನಂತರ ಯೋಗ್ಯರಾದ ಭಕ್ತರಲ್ಲಿ ಅದನ್ನು ಪ್ರಚಾರ ಮಾಡಬೇಕು. ನಾನು ತಿಳಿದುಕೊಂಡಿರುವುದು ನನ್ನಲ್ಲೇ ಕೊನೆಗೊಳ್ಳ ಬಾರದು. ಅದು ನನ್ನ ಮೂಲಕ ಯೋಗ್ಯ ಹೃದಯಕ್ಕೆಲ್ಲಾ ಹೋಗಬೇಕು. ನನಗೆ ಯಾವ ಶಾಂತಿ ಆನಂದಗಳು ಅದರಿಂದ ಬಂದಿವೆಯೋ ಅದನ್ನು ಇತರ ಭಕ್ತರೊಡನೆಯೂ ಹಂಚಿಕೊಳ್ಳಬೇಕು. ಹೀಗೆ ಹಂಚಿಕೊಳ್ಳುವುದರಿಂದ ನಮ್ಮ ಆನಂದವೇ ಜಾಸ್ತಿಯಾಗುವುದು. ಯಾವಾಗಲೂ ಆನಂದವನ್ನು ಹಂಚಿಕೊಂಡಾಗ ಜಾಸ್ತಿಯಾಗುವುದು. ಅದರಲ್ಲಿಯೂ ಭಗವದಾನಂದದಷ್ಟು ಆನಂದವನ್ನು ಕೊಡು ವುದು ಮತ್ತೊಂದಿಲ್ಲ. ಅದನ್ನು ನಮ್ಮ ಆನಂದಕ್ಕಾಗಿಯೇ ಇತರರಿಗೆ ಕೊಡಬೇಕು. ಹೀಗೆ ಕೊಡು ವುದು ತಿಳಿದವನ ಒಂದು ಕರ್ತವ್ಯವೂ ಆಗಿದೆ. ನಾವು ಕೊಡದೇ ಇದ್ದರೆ ಅದು ಜನರಿಗೆ ಪ್ರಚಾರವಾಗುವುದಕ್ಕೆ ನಾವೊಂದು ನಿಮಿತ್ತವಾಗಬೇಕು.

ಯಾರು ಸಂದೇಶವನ್ನು ಇತರರಿಗೆ ಹೇಳುತ್ತಾನೆಯೋ ಅಂಥವನನ್ನು ಕಂಡರೆ ದೇವರಿಗೆ ಪ್ರೀತಿ. ಯಾಕೆಂದರೆ ಅವನು ಭಗವಂತನ ವಾಣಿ ಜನರಲ್ಲೆಲ್ಲ ಹರಡುವಂತೆ ಮಾಡಿ ಅವರನ್ನು ದೇವರೆಡೆಗೆ ಒಯ್ಯುವಂತೆ ಮಾಡುವನು. ದೇವರಿಗೆ ಅವನನ್ನು ಹೊಗಳುವವರು ಬೇಕಾಗಿಲ್ಲ. ಅವನ ಕೆಲಸವನ್ನು ಮಾಡುವವರು ಬೇಕು. ಯಾರು ಈ ಪವಿತ್ರವಾದ ಗೀತಾ ಸಂದೇಶವನ್ನು ಇತರರಿಗೆ ಹೇಳುತ್ತಾರೆಯೋ ಅಂತಹವರಿಗೆ ಪರಾಭಕ್ತಿಯನ್ನು ಅನುಗ್ರಹಿಸುತ್ತಾನೆ. ಪರಾಭಕ್ತಿ ಒಂದು ಅಪೂರ್ವ ವಸ್ತು. ಶ‍್ರೀರಾಮ ಕೃಷ್ಣರು, ದೇವರು ಬೇಕಾದರೆ ಜ್ಞಾನವನ್ನು ಕೊಡುತ್ತಾನೆ; ಪರಾಭಕ್ತಿಯನ್ನು ಕೊಡುವುದಕ್ಕೆ ತುಂಬಾ ಚೌಕಾಸಿ ಮಾಡುತ್ತಾನೆ ಎನ್ನುತ್ತಿದ್ದರು. ಪರಾಭಕ್ತಿ ಎಂದರೆ ಕೇವಲ ಭಕ್ತಿಗಾಗಿ ಭಕ್ತಿ. ಅವನಿಗೆ ದೇವರಿಂದ ಲೌಕಿಕವಾದುದೇನೂ ಬೇಕಾಗಿಲ್ಲ. ಕೇವಲ ಅವನನ್ನು ಪ್ರೀತಿಸಬೇಕು, ಸೇವಿಸಬೇಕು ಎಂದು ಮಾತ್ರ ಆಶಿಸುವನು. ಇಂತಹ ಭಕ್ತ ನಿಸ್ಸಂಶಯವಾಗಿ ಭಗವಂತನನ್ನೇ ಸೇರುತ್ತಾನೆ.

\begin{verse}
ನ ಚ ತಸ್ಮಾನ್ಮನುಷ್ಯೇಷು ಕಶ್ಚಿನ್ಮೇ ಪ್ರಿಯಕೃತ್ತಮಃ~।\\ಭವಿತಾ ನ ಚ ಮೇ ತಸ್ಮಾದನ್ಯಃ ಪ್ರಿಯತರೋ ಭುವಿ \versenum{॥ ೬೯~॥}
\end{verse}

{\small ಮನುಷ್ಯರಲ್ಲಿ ಅವನಗಿಂತಲೂ ನನಗೆ ಪ್ರಿಯವನ್ನುಂಟುಮಾಡುವವರು ಯಾರೂ ಇಲ್ಲ. ಭೂಮಿಯಲ್ಲಿ ಅವನಿಗಿಂತ ನನಗೆ ಪ್ರಿಯಕರನಾದವನು ಬೇರೆ ಯಾರೂ ಹುಟ್ಟುವುದಿಲ್ಲ.}

ಯಾರು ಭಗವಂತನ ಸಂದೇಶವನ್ನು ಜನರಿಗೆ ಪ್ರಚಾರಮಾಡುತ್ತಾನೆಯೋ ಅವನೇ ದೇವರಿಗೆ ಅತ್ಯಂತ ಪ್ರೀತಿಪಾತ್ರನು. ಜನರು ಉದ್ಧಾರವಾಗುವುದಕ್ಕೆ ಇಲ್ಲಿ ಮಾರ್ಗವಿದೆ. ಗೀತೆ ಸಕಲ ವೇದಗಳ ಸಾರ. ಸಕಲ ಪುರುಷಾರ್ಥವೂ ಸಿದ್ಧಿಸುವುವು ಇದನ್ನು ನಮ್ಮ ಜೀವನದಲ್ಲಿ ಅನುಷ್ಠಾನ ಮಾಡಿದರೆ. ಇದರಿಂದ ಇಹ ಪರ ಎರಡೂ ದೊರಕುವುವು. ಇದರ ಸಂದೇಶದಿಂದ ನನ್ನ ಜೀವನವನ್ನು ತಿದ್ದಿಕೊಂಡರೆ ಮಾತ್ರ ನಾನು ಒಳ್ಳೆಯವನಾಗುತ್ತೇನೆ, ದೇಶ ಒಳ್ಳೆಯದಾಗುತ್ತದೆ. ಇರುವಾಗ ಇಲ್ಲಿ ನಾನು ಶಾಂತಿ ಅನುಭವಿಸುತ್ತೇನೆ. ಹೋಗುವಾಗ ಶಾಂತಿಯಿಂದ ಹೋಗುತ್ತೇನೆ. ಈ ಕೆಲಸವನ್ನೆಲ್ಲಾ ಮಾಡುವವರು ಗೀತಾ ಸಂದೇಶವನ್ನು ಜನರಲ್ಲಿ ಹರಡುವರು. ಅವರು ಭಗವಂತನ ಕೆಲಸವನ್ನು ಮಾಡುತ್ತಿರುವರು. ಚದುರಿ ಹೋದ, ವಿಯೋಗದಲ್ಲಿರುವ ಮನುಷ್ಯರನ್ನು ಭಗವಂತನೆಡೆಗೆ ಪುನಃ ಬರುವಂತೆ ಮಾಡುತ್ತಿರುವರು. ಆದಕಾರಣವೇ ಇಂತಹ ವ್ಯಕ್ತಿಗಳನ್ನು ಕಂಡರೆ ಭಗವಂತನಿಗೆ ಪ್ರೀತಿ.

ಭೂಮಿಯಲ್ಲಿ ಅವನಿಗಿಂತ ಮೇಲಾದವನು ಇನ್ನು ಯಾರೂ ಹುಟ್ಟುವುದಿಲ್ಲ ಎನ್ನುತ್ತಾನೆ. ಈ ಸೃಷ್ಟಿಯಲ್ಲಿ ಶ್ರೇಷ್ಠವ್ಯಕ್ತಿ ಐಶ್ವರ್ಯವಂತನಲ್ಲ, ಅಧಿಕಾರಿಯಲ್ಲ, ಪ್ರಚಂಡ ವಿದ್ಯಾವಂತನೂ ಅಲ್ಲ. ಯಾರು ಭಗವಂತನನ್ನು ಅನುಭವಿಸುತ್ತಿರುವನೊ ಮತ್ತು ಅವನನ್ನು ಪಡೆಯುವುದು ಹೇಗೆ ಎಂಬು ದನ್ನು ಇತರರಿಗೆ ವಿವರಿಸುತ್ತಿರುವನೊ ಅವನು. ಅವನೇ ಸೃಷ್ಟಿಯ ಅತ್ಯಂತ ಶ್ರೇಷ್ಠ ವ್ಯಕ್ತಿ. ಸಂಸಾರದ ಗಾಡಾಂಧಕಾರದಲ್ಲಿ ಸಂಚಾರ ಮಾಡುತ್ತಿರುವಾಗ ದಾರಿಯನ್ನು ತೋರುವ ದೀಪಗಳಿವರು. ಭಗವಂತ ನನ್ನು ತಾತ್ತ್ವಿಕವಾಗಿ ನೋಡಿದರೆ ಸರ್ವಾಂತರ್ಯಾಮಿ. ಎಲ್ಲಾ ಕಡೆಯಲ್ಲೂ ಅವನಿರುವನು. ಆದರೆ ಇಂತಹ ವ್ಯಕ್ತಿಗಳಲ್ಲಿ ಅವನು ಹೆಚ್ಚಾಗಿ ಮೈದೋರಿರುವನು. ಇಂತಹ ವ್ಯಕ್ತಿಗಳೇ ಹಸುವಿನ ಹಾಲಿರುವ ಕೆಚ್ಚಲು. ಹಾಲು ಹಸುವಿನಲ್ಲಿದೆ. ಆದರೆ ಅದರ ಕೊಂಬು, ಕಿವಿ, ಬಾಲ ಹಿಂಡಿದರೆ ಹಾಲು ಬರುವುದಿಲ್ಲ. ಅದರ ಕೆಚ್ಚಲನ್ನು ಹಿಂಡಬೇಕು. ಆಗ ಹಾಲು ದೊರೆಯುವುದು. ಹಾಗೆಯೇ ಭಗವಂತ ನನ್ನು ತಿಳಿಯಬೇಕಾದರೆ ಇಂತಹ ವ್ಯಕ್ತಿಗಳ ಮೂಲಕ ತಿಳಿಯಬೇಕು. ವಿಕಾಸದ ಏಣಿಯ ತುತ್ತ ತುದಿಯಲ್ಲಿ ನಿಂತ ವ್ಯಕ್ತಿಗಳಿವರು. ಕೀಟದಿಂದ ಪ್ರಾರಂಭವಾಗಿ ಖಗವಾಗಿ ಮೃಗವಾಗಿ ನರನಾಗಿ ವಿಷಯ ಕಾನನದಲ್ಲಿ ಸುತ್ತಾಡಿ ಬಳಲಿ ಬೇಸತ್ತು, ಈ ಪ್ರಪಂಚದಲ್ಲಿ ಪರಮಾತ್ಮನೊಬ್ಬನೇ ಸತ್ಯ ಎಂದು ತಿಳಿದುಕೊಂಡು, ಅವನನ್ನು ಅನುಭವಿಸಿ, ತಾನು ಬದುಕಿರುವ ಪರ್ಯಂತರವೂ ಇತರರಿಗೆ ಹೇಳಿ ಕಣ್ಣು ಮುಚ್ಚಿಕೊಳ್ಳುವ ವ್ಯಕ್ತಿ ಭಗವಂತನ ದೃಷ್ಟಿಯಲ್ಲಿ ಶ್ರೇಷ್ಠ ತಮವಾದ ವಸ್ತು. ಸೃಷ್ಟಿಯ ತರುವಿನಲ್ಲಿ ಬಿಟ್ಟ ಹಲವು ಪುಷ್ಪಗಳಲ್ಲಿ ಸರ್ವಶ್ರೇಷ್ಠವಾದ ಪುಷ್ಪ ಇಂತಹ ವ್ಯಕ್ತಿ. ಈ ಪುಷ್ಪ ಗೀತಾ ಸಂದೇಶದ ಪರಿಮಳವನ್ನು ದೆಸೆದೆಸೆಗೆ ಬೀರುತ್ತಿದೆ. ನಾವು ನಮ್ಮ ಗೀತೆಯ ದೃಷ್ಟಿಯಿಂದಲೇ ನೋಡುವುದಿಲ್ಲ. ಇದರ ಅರ್ಥವನ್ನು ಮತ್ತೂ ವಿಶಾಲ ಮಾಡಬಹುದು. ಎಲ್ಲಾ ಧರ್ಮದಲ್ಲಿರು ವುದೂ ಭಗವಂತನ ವಾಣಿಯೆ. ಯಾರು ಅದನ್ನು ತಾವು ಅನುಭವಿಸಿ ಇತರರಿಗೆ ಹೇಳುತ್ತಿರುವರೋ ಅವರೆಲ್ಲಾ ಶ‍್ರೀಕೃಷ್ಣ ಪರಮಾತ್ಮನ ಕೆಲಸವನ್ನೇ ಮಾಡುತ್ತಿರುವರು. ಅವರೆಲ್ಲರೂ ಭಗವಂತನಿಗೆ ಪ್ರಿಯರಾದವರೆ. ಅವರೆಲ್ಲರೂ ಪ್ರಪಂಚದಲ್ಲಿ ಶ್ರೇಷ್ಠವ್ಯಕ್ತಿಗಳೇ.

\begin{verse}
ಅಧ್ಯೇಷ್ಯತೇ ಚ ಯ ಇಮಂ ಧರ್ಮ್ಯಂ ಸಂವಾದಮಾವಯೋಃ~।\\ಜ್ಞಾನಯಜ್ಞೇನ ತೇನಾಹಮಿಷ್ಟಃ ಸ್ಯಾಮಿತಿ ಮೇ ಮತಿಃ \versenum{॥ ೭ಂ~॥}
\end{verse}

{\small ಧರ್ಮದಿಂದ ಕೂಡಿದ ಈ ಸಂವಾದವನ್ನು ಯಾರು ಅಧ್ಯಯನ ಮಾಡುವರೊ ಅವರು ಜ್ಞಾನಯಜ್ಞದ ಮೂಲಕ ನನ್ನನ್ನು ಭಜಿಸುತ್ತಾರೆ ಎಂಬುದೇ ನನ್ನ ಮತ.}

ಭಗವದ್ಗೀತೆಯನ್ನು ಯಾರು ಅಧ್ಯಯನ ಮಾಡುತ್ತಾರೆಯೋ ಅವರು ಜ್ಞಾನಯಜ್ಞವನ್ನು ಮಾಡುತ್ತಾ ಇದ್ದಾರೆ. ಇಲ್ಲಿ ಅಧ್ಯಯನ ಎಂದರೆ ಸುಮ್ಮನೆ ಓದುವುದು ಮಾತ್ರ ಅಲ್ಲ. ಅದರಲ್ಲಿರುವ ಭಾವನೆಗಳನ್ನು ಮೆಲಕು ಹಾಕುವುದು, ಧ್ಯಾನಿಸುವುದು ಎಲ್ಲವೂ ಸೇರಿವೆ. ಗೀತೆ ಎಂದಿಗೂ ಹಳತಾಗು ವುದಿಲ್ಲ. ನಾವು ಹೆಚ್ಚು ಹೆಚ್ಚು ಓದಿದಂತೆ ಆಳ ಆಳಕ್ಕೆ ಇಳಿಯುತ್ತೇವೆ. ಇದು ಭಗವಂತನಷ್ಟೇ ಆಳವಾಗಿರುವುದು. ಅವನಷ್ಟೇ ವಿಶಾಲವಾಗಿರುವುದು. ಹಲವಾರು ವರುಷಗಳಿಂದ ಗೀತೆಯನ್ನು ಓದುತ್ತಿದ್ದರೂ ಪ್ರತಿಯೊಂದು ಸಲ ಅದನ್ನು ಓದಿದಾಗಲೂ ಹೊಸ ಹೊಸ ಭಾವನೆಗಳು ಸ್ಫುರಿಸು ವುವು. ಶ‍್ರೀಕೃಷ್ಣನೇ ಹಲವು ಯಜ್ಞಗಳಿವೆ ಎಂಬುದನ್ನು ಹಿಂದಿನ ಅಧ್ಯಾಯಗಳಲ್ಲಿ ವಿವರಿಸುವನು. ಅದರಲ್ಲಿ ಜ್ಞಾನಯಜ್ಞ ಸರ್ವಶ್ರೇಷ್ಠ ಮತ್ತು ಭಗವಂತನಿಗೆ ತುಂಬಾ ಮೆಚ್ಚುಗೆಯಾದುದು. ಯಾರು ಗೀತೆಯನ್ನು ಅಧ್ಯಯನ ಮಾಡುತ್ತಾರೊ ಅವರೊಂದು ಜ್ಞಾನಯಜ್ಞವನ್ನು ಮಾಡುತ್ತಿರುವರು. ಭಗವಂತ ಇದರಿಂದ ಸುಪ್ರೀತನಾಗುವನು.

\begin{verse}
ಶ್ರದ್ಧಾವಾನನಸೂಯಶ್ಚ ಶೃಣುಯಾದಪಿ ಯೋ ನರಃ~।\\ಸೋಽಪಿ ಮುಕ್ತಃ ಶುಭಾಂಲ್ಲೋಕಾನ್ ಪ್ರಾಪ್ನುಯಾತ್ ಪುಣ್ಯಕರ್ಮಣಾಮ್ \versenum{॥ ೭೧~॥}
\end{verse}

{\small ಶ್ರದ್ಧಾವಂತನೂ ಅಸೂಯೆ ಇಲ್ಲದವನೂ ಆದ ಯಾರು ಇದನ್ನು ಕೇಳುವನೊ ಅವನು ಕೂಡ ಮುಕ್ತನಾಗಿ ಪುಣ್ಯ ಕರ್ಮಗಳ ಶುಭವಾದ ಲೋಕವನ್ನು ಹೊಂದುತ್ತಾನೆ.}

ಯಾರು ಇದನ್ನು ಕೇಳುತ್ತಾರೊ ಅವರು ಕೂಡಾ ಒಂದು ಶುಭಕರ್ಮವನ್ನು ಮಾಡುತ್ತಾರೆ. ಹೆಚ್ಚು ಪ್ರತಿಫಲವನ್ನು ಕೊಡಬೇಕಾದರೆ ಅದಕ್ಕೆ ಕೇಳುವವನು ಶ್ರದ್ಧಾವಂತನಾಗಿರಬೇಕು. ಇದೊಂದು ಪವಿತ್ರ ವಿಷಯ, ಮಾನವನ ಉದ್ಧಾರಕ್ಕೆ ಭಗವಂತನೇ ಹೇಳುತ್ತಿದ್ದಾನೆ ಎಂಬ ಶ್ರದ್ಧೆಯಿಂದ ಅದನ್ನು ಕೇಳಬೇಕು. ಮನಸ್ಸು ಅಸೂಯೆಯಿಂದ ಪಾರಾಗಿರಬೇಕು. ಜೀವನದಲ್ಲಿ ಇನ್ನೊಬ್ಬನನ್ನು ನೋಡಿ ನನ್ನಲ್ಲಿ ಇಲ್ಲದಿರುವ ಐಶ್ವರ್ಯ ಅಧಿಕಾರ ಕೀರ್ತಿ ಮುಂತಾದುವುಗಳು ಇದ್ದರೆ ಅದಕ್ಕೆ ಹೊಟ್ಟೆಕಿಚ್ಚು ಪಡದೆ ಇರಬೇಕು. ಜೀವನದಲ್ಲಿ ಪ್ರತಿಯೊಬ್ಬನೂ ತನ್ನ ಕರ್ಮಾನುಸಾರ ಪಡೆಯುತ್ತಾನೆ ಎಂದು ಇರುವವನಿಗೆ ಔದಾರ್ಯತೆಯನ್ನು ತೋರಬೇಕು. ಯಾವಾಗ ಇಂತಹ ಮನಸ್ಸಿನಿಂದ ಕೇಳುತ್ತಾನೊ ಆಗ ಅವನಲ್ಲಿ ಒಳ್ಳೆಯ ಸಂಸ್ಕಾರ ಉಂಟಾಗುವುದು. ದಿನ ಬೆಳಗ್ಗೆದ್ದರೆ ಹಾಳು ಮೂಳು ಕೇಳುತ್ತಾ ಇರುತ್ತೇವೆ. ಅದರ ಬದಲು ಆಧ್ಯಾತ್ಮಿಕ ಸತ್ಯಗಳನ್ನೊಳಗೊಂಡ ಗೀತೆಯನ್ನು ಕೇಳಿದರೆ ಅದು ನಮ್ಮ ಪರಲೋಕಯಾತ್ರೆಗೆ ಬುತ್ತಿಯಾಗುವುದು. ಮುಂದಿನ ಜನ್ಮದಲ್ಲಿ ಪುಣ್ಯವಂತರ ಮನೆಯಲ್ಲಿ ಹುಟ್ಟು ತ್ತೇವೆ. ನಮ್ಮ ಆಧ್ಯಾತ್ಮಿಕ ಜೀವನ ಸುಲಭವಾಗಿ ಅಂತಹ ವಾತಾವರಣದಲ್ಲಿ ವಿಕಾಸವಾಗುವುದು.

\begin{verse}
ಕಚ್ಚಿದೇತಚ್ಛ್ರುತಂ ಪಾರ್ಥ ತ್ವಯೈಕಾಗ್ರೇಣ ಚೇತಸಾ~।\\ಕಚ್ಚಿದಜ್ಞಾನಸಂಮೋಹಃ ಪ್ರನಷ್ಟಸ್ತೇ ಧನಂಜಯ \versenum{॥ ೭೨~॥}
\end{verse}

{\small ಪಾರ್ಥ, ನೀನು ಇದನ್ನು ಏಕಾಗ್ರಚಿತ್ತದಿಂದ ಕೇಳಿದೆಯಾ? ಅಜ್ಞಾನದಿಂದ ಉಂಟಾದ ಮೋಹ ನಾಶವಾಯಿತೆ?}

ಶ‍್ರೀಕೃಷ್ಣ ಅರ್ಜುನನಿಗೆ ಹೇಳಬೇಕಾದುದನ್ನೆಲ್ಲ ಹೇಳಿದ. ಈಗ ಕೇಳುತ್ತಾನೆ, ಚೆನ್ನಾಗಿ ಅರ್ಥ ವಾಯಿತೆ, ನಿನ್ನ ಮೋಹ ನಾಶವಾಯಿತೆ ಎಂದು. ಆದರ್ಶ ಗುರುವಿನ ಮಾರ್ಗ ಇದು. ಗುರುವೇನೊ ತನಗೆ ಗೊತ್ತಿರುವುದನ್ನೆಲ್ಲ ಹೇಳಬಹುದು. ಅಲ್ಲಿಗೆ ಅವನ ಕೆಲಸ ಮುಗಿಯಲಿಲ್ಲ. ಅದನ್ನೆಲ್ಲ ಚೆನ್ನಾಗಿ ಅರ್ಥ ಮಾಡಿಕೊಂಡೆಯಾ ಎಂದು ಕೇಳುತ್ತಾನೆ. ವೈದ್ಯ ರೋಗಿಗೆ ಔಷಧಿಯನ್ನೇನೊ ಕೊಡುತ್ತಾನೆ. ಅಲ್ಲಿಗೆ ಕೊನೆಗಾಣಲಿಲ್ಲ ಅವನ ಕೆಲಸವೆಲ್ಲ. ರೋಗದಿಂದ ಪಾರಾದೆಯಾ ಎಂದು ಕೇಳುತ್ತಾನೆ. ಔಷಧಿ ಕೊಡುವ ಉದ್ದೇಶ ರೋಗಿಯನ್ನು ಗುಣ ಮಾಡುವುದು. ಅಂತೆಯೇ ಶ‍್ರೀಕೃಷ್ಣ ಅರ್ಜುನನ್ನು ಕೇಳುತ್ತಾನೆ, ನಿನಗೆ ಚೆನ್ನಾಗಿ ಅರ್ಥವಾಗಿದೆಯೆ ಎಂದು. ಕೆಲವರು ಏನೋ ನಾಚಿಕೆಯಿಂದ ಗೊತ್ತಾ ಗದೇ ಇದ್ದರೂ ಗೊತ್ತಾಗಿದೆ ಎಂದು ಹೇಳುತ್ತಾರೆ. ಅದಕ್ಕೇ ಶ‍್ರೀಕೃಷ್ಣ ಕೇಳುವುದು. ಅವನಿಗೆ ಗೊತ್ತಾಗದೆ ಇದ್ದರೆ, ಯಾವುದು ಗೊತ್ತಾಗಲಿಲ್ಲವೆಂದು ಕೇಳಿ, ಪುನಃ ಆ ವಿಷಯಗಳನ್ನು ಮತ್ತೂ ಸ್ಪಷ್ಟವಾಗಿ ಮಾಡುವುದಕ್ಕೆ ಸಿದ್ಧನಾಗಿರುವನು. ಗುರು ಶಿಷ್ಯನಿಗೆ ಜ್ಞಾನವನ್ನು ನೀಡುವುದಕ್ಕೆ ಎಷ್ಟು ತೊಂದರೆಯನ್ನಾದರೂ ತಡೆದುಕೊಳ್ಳಲು ಸಿದ್ಧನಾಗಿರುವನು.

ಅರ್ಜುನ ಹೇಳುತ್ತಾನೆ:

\begin{verse}
ನಷ್ಟೋ ಮೋಹಃ ಸ್ಮೃತಿರ್ಲಬ್ಧಾ ತ್ವತ್ಪ್ರಸಾದಾನ್ಮಯಾಚ್ಯುತ~।\\ಸ್ಥಿತೋಽಸ್ಮಿ ಗತಸಂದೇಹಃ ಕರಿಷ್ಯೇ ವಚನಂ ತವ \versenum{॥ ೭೩~॥}
\end{verse}

{\small ಅಚ್ಯುತ, ನಿನ್ನ ಅನುಗ್ರಹದಿಂದ ನನ್ನ ಮೋಹ ನಾಶವಾಯಿತು. ನನಗೆ ಅರಿವು ಬಂದಿದೆ. ಸಂದೇಹ ನಿವಾರಣೆ ಆಗಿದೆ. ನಿನ್ನ ಉಪದೇಶದಂತೆ ನಡೆಯುತ್ತೇನೆ.}

ಅರ್ಜುನ ತನ್ನ ಮೋಹ ನಾಶವಾಯಿತು ಎನ್ನುತ್ತಾನೆ. ಇವರೆಲ್ಲ ನನ್ನ ಬಂಧು ಬಾಂಧವರು, ನನ್ನವರು ಎಂಬ ಮೋಹ ದೂರವಾಯಿತು. ಇವರೆಲ್ಲ ಈಗ ಅಧರ್ಮದ ಕಳೆಗಳು, ಚೆನ್ನಾಗಿ ಕೊಬ್ಬಿ ಬೆಳೆದಿರುವರು ಎಂಬುದು ಅರ್ಥವಾಯಿತು. ನನಗೆ ಅರಿವು ಬಂದಿದೆ ಎಂದರೆ ಜ್ಞಾನೋದಯವಾಗಿದೆ. ಪರಿಸ್ಥಿತಿಯನ್ನು ಯಥಾರ್ಥವಾಗಿ ನೋಡುತ್ತಿದ್ದೇನೆ. ಅದು ಹೇಗಿದೆಯೋ ಹಾಗೆ ಕಾಣುತ್ತಿದೆ. ಮೋಹದ ಅಂಜನ ಹೋದರೆ ವಸ್ತುವಿನ ನೈಜಸ್ಥಿತಿ ವ್ಯಕ್ತವಾಗುವುದು. ಶ‍್ರೀಕೃಷ್ಣ ತನ್ನ ಬೋಧನೆ ಯಿಂದ ಅಜ್ಞಾನವನ್ನು ನಾಶಮಾಡುತ್ತಾನೆ. ಸಂದೇಹ ನಿವಾರಣೆ ಆಗಿದೆ. ಯುದ್ಧದಲ್ಲಿ ನಾವೇ ಗೆಲ್ಲುತ್ತೇವೆಯೋ ಇವರೇ ಗೆಲ್ಲುತ್ತಾರೆಯೋ ಎಂಬ ಸಂದೇಹ ಇತ್ತು. ಅರ್ಜುನನು ವಿಶ್ವರೂಪದಲ್ಲಿ ಯಾರು ಸೋಲುತ್ತಾರೆ ಎಂಬುದನ್ನು ಸ್ಪಷ್ಟವಾಗಿ ನೋಡಿದ. ಗುರು ಹಿರಿಯರನ್ನು ಯುದ್ಧದಲ್ಲಿ ಗೆಲ್ಲುವುದು ಧರ್ಮವೇ ಅಧರ್ಮವೇ, ಈ ಕೃತ್ಯಗಳನ್ನು ಮಾಡಿದರೆ ನನಗೆ ಪಾಪ ಬರುವುದಿಲ್ಲವೆ ಎಂಬ ಸಂದೇಹಗಳೆಲ್ಲ ಇದ್ದುವು. ಶ‍್ರೀಕೃಷ್ಣ ಅದನ್ನೆಲ್ಲಾ ನಿವಾರಿಸುತ್ತಾನೆ. ಇನ್ನು ನಿನ್ನ ಉಪದೇಶ ದಂತೆ ನಡೆಯುತ್ತೇನೆ ಎಂದು ಹೇಳುತ್ತಾನೆ. ಅರ್ಜುನ ಯುದ್ಧ ಮಾಡುವುದಕ್ಕೆ ಸಿದ್ಧನಾಗುತ್ತಾನೆ. ಧರ್ಮ ಅಧರ್ಮಗಳನ್ನು ಬದಿಗೊಡ್ಡುತ್ತಾನೆ. ಎಲ್ಲವನ್ನೂ ಮಾಡಿಸುತ್ತಿರುವವನು ಭಗವಂತ, ನಾನೊಂದು ನಿಮಿತ್ತ, ಅವನಾಣತಿಯಂತೆ ನಡೆಯುತ್ತೇನೆ ಎಂದು ನಿರ್ಣಯಿಸುತ್ತಾನೆ. ಗೀತೆ ಪ್ರಾರಂಭವಾದದ್ದೇ ಅರ್ಜುನನನ್ನು ಯುದ್ಧೋನ್ಮಖನನ್ನಾಗಿ ಮಾಡುವುದಕ್ಕೆ. ಯಾವಾಗ ಅವನು ಅಣಿಯಾಗುತ್ತಾನೊ ಶ‍್ರೀಕೃಷ್ಣನ ಸಂದೇಶವೂ ನಿಲ್ಲುವುದು. ಅರ್ಜುನನ ಕರ್ಮಕ್ಕೆ ಒಂದು ದೊಡ್ಡ ತತ್ತ್ವದ ತಳಹದಿಯನ್ನು ನಿರ್ಮಿಸುತ್ತಾನೆ. ಅರ್ಜುನ ಮಾಡುವುದು ಬರಿಯ ಯುದ್ಧವಲ್ಲ. ಅದು ಒಂದು ಧರ್ಮ, ಕರ್ತವ್ಯ ನಿರ್ವಹಣೆ. ಅದನ್ನು ನಿಜವಾಗಿ ಮಾಡುತ್ತಿರುವವನು ಪರಮಾತ್ಮನೆ; ಅವನ ಕೈಯಲ್ಲಿ ಬರೀ ನಿಮಿತ್ತ ಅರ್ಜುನ. ಶ‍್ರೀರಾಮಕೃಷ್ಣರು ಒಂದು ಉದಾಹರಣೆ ಕೊಡುತ್ತಾರೆ. ಕಾದ ತುಪ್ಪಕ್ಕೆ ಏನಾದರೂ ಕರಿಯಲು ಹಸಿಯದನ್ನು ಹಾಕಿದರೆ, ಅದು ಬೇಯುವವರೆಗೆ ತುಪ್ಪ ಶಬ್ದ ಮಾಡುತ್ತಲೇ ಇರುವುದು. ಯಾವಾಗ ಅದು ಚೆನ್ನಾಗಿ ಬೇಯುವುದೋ ಆಗ ಶಬ್ದ ನಿಲ್ಲುವುದು. ಹಾಗೆಯೇ ಅರ್ಜುನನ ಅಜ್ಞಾನದ ಹಸಿ ಶ‍್ರೀಕೃಷ್ಣನ ಬೋಧನೆಯಲ್ಲಿ ಬೆಂದು ಈಗ ಸಿದ್ಧವಾಗಿದೆ. ಇನ್ನು ಅದನ್ನು ಬಾಂಡಲೆಯಿಂದ ಕೆಳಗೆ ಇಳಿಸುವನು.

ಸಂಜಯ ಹೇಳುತ್ತಾನೆ:

\begin{verse}
ಇತ್ಯಹಂ ವಾಸುದೇವಸ್ಯ ಪಾರ್ಥಸ್ಯ ಚ ಮಹಾತ್ಮನಃ~।\\ಸಂವಾದಮಿಮಮಶ್ರೌಷಮದ್ಭುತಂ ರೋಮಹರ್ಷಣಮ್ \versenum{॥ ೭೪~॥}
\end{verse}

{\small ನಾನು ಶ‍್ರೀಕೃಷ್ಣನ ಮತ್ತು ಮಹಾತ್ಮನಾದ ಅರ್ಜುನನ ಅದ್ಭುತವಾದ ರೋಮಾಂಚಕಾರಿಯಾದ ಈ ಸಂವಾದ ವನ್ನು ಕೇಳಿದೆನು.}

ಸಂಜಯ ಅರ್ಜುನನ್ನು ಮಹಾತ್ಮನೆಂದು ಕರೆಯುತ್ತಾನೆ. ಶ‍್ರೀಕೃಷ್ಣನಿಂದ ಗೀತಾ ಎಂಬ ಅಮೃತ ಹರಿಯುವುದಕ್ಕೆ ಅರ್ಜುನ ನಿಮಿತ್ತನಾದ. ಹಾಗೆ ನಿಮಿತ್ತವಾಗಬೇಕಾದರೆ ಎಲ್ಲರಿಗೂ ಸಾಧ್ಯವಿಲ್ಲ. ನಮ್ಮ ಅಹಂಕಾರವೆಲ್ಲ ಬರಿದಾದರೆ ಮಾತ್ರ ದೇವರು ನಮ್ಮನ್ನು ನಿಮಿತ್ತವನ್ನಾಗಿ ಆರಿಸಿಕೊಳ್ಳುವನು. ಎಲ್ಲಿಯವರೆಗೆ ಅಹಂಕಾರ ತುಂಬಿದೆಯೋ ಅಲ್ಲಿಯವರೆಗೆ ದೇವರು ಕೆಲಸ ಮಾಡುವುದಿಲ್ಲ. ಅಹಂಕಾರ ಒಂದು ಬಾಗಿಲಿನಿಂದ ಹೋದೊಡನೆಯೇ ದೇವರು ಮತ್ತೊಂದು ಬಾಗಿಲಿನಿಂದ ಬರುವನು. ಅರ್ಜುನನಿಗೆ ಮತ್ತೊಂದು ದೃಷ್ಟಿಯಿಂದಲೂ ಮಹಾತ್ಮ ಎಂಬ ಬಿರುದು ಸಲ್ಲುವುದು. ಭಗೀರಥ ಹೇಗೆ ಸ್ವರ್ಗಲೋಕದಿಂದ ಗಂಗೆಯನ್ನು ಭೂಮಿಗೆ ತಂದನೋ ಅಂದಿನಿಂದ ಇಂದಿನವರೆಗೆ ಅದು ಹರಿಯುತ್ತ ಅದರಲ್ಲಿ ಮೀಯಲು ಬಂದವರ ಪಾಪಗಳನ್ನೆಲ್ಲ ತೊಳೆಯುತ್ತಿರುವುದೊ ಹಾಗೆ ಅರ್ಜುನ ಗೀತೆಯೆಂಬ ಗಂಗೆ ಭಗವಂತನ ಬಾಯಿಯ ಮೂಲಕ ಧರೆಗೆ ಇಳಿದು ಬರುವುದಕ್ಕೆ ಕಾರಣವಾದ. ಅದು ಮುಂಚೆ ಬಂದದ್ದು ಅವನ ಶೋಕ ಮತ್ತು ಅನುಮಾನಗಳನ್ನು ಪರಿಹರಿಸು ವುದಕ್ಕೆ. ಆದರೆ ಅದು ಇಡೀ ಮಾನವ ಕೋಟಿಯ ಶೋಕ, ತಾಪಗಳನ್ನು ಕಳೆಯುವುದಕ್ಕೆ ಈಗಲೂ ಹರಿಯುತ್ತಿದೆ. ಇಂತಹ ಗೀತಾಭಗವತಿ ನಮ್ಮ ಭರತಖಂಡದಲ್ಲಿ ಹರಿಯುವುದಕ್ಕೆ ಮೂಲ ಕಾರಣ ನಾದವನು ಅರ್ಜುನ. ಇದಕ್ಕಿಂತ ಮಹತ್ತಾದ ಕೆಲಸ ಯಾವುದಿದೆ.

ಈ ಸಂವಾದ ರೂಪದಲ್ಲಿರುವ ಸಂದೇಶ ಅದ್ಭುತವಾದುದು. ಆಧ್ಯಾತ್ಮಿಕ ಜೀವನದ ಚರಮ ಸತ್ಯವನ್ನು ಇಲ್ಲಿ ನೋಡುತ್ತೇವೆ. ಇಷ್ಟು ಆಳವಾಗಿ ಇಷ್ಟು ವಿಶಾಲವಾಗಿರುವ ಸಂದೇಶ ಮತ್ತೊಂದು ಇಲ್ಲ. ಇಲ್ಲಿ ಎಲ್ಲರಿಗೂ ಸ್ಥಾನವಿದೆ. ತತ್ತ್ವದ ಸೋಪಾನ ಪಂಕ್ತಿಯನ್ನೇ ನಾವು ಇಲ್ಲಿ ನೋಡುತ್ತೇವೆ. ಸರ್ವ ಯೋಗಗಳ ಭಾವಗಳ ತತ್ತ್ವಗಳ ಸಂಗಮ ಸ್ಥಳ ಇದು. ಇಂತಹ ಮಹಾಸತ್ಯವನ್ನು ಶ‍್ರೀಕೃಷ್ಣ ಎಷ್ಟು ಸರಳವಾಗಿ ಹೇಳುತ್ತಾನೆ. ಸ್ವಲ್ಪ ಬುದ್ಧಿ ಇರುವ ಯಾರಿಗಾದರೂ ಇದು ಅರ್ಥವಾಗುವುದು. ಪೂರ್ತಿ ಅರ್ಥವಾಗಬೇಕಿಲ್ಲ. ಸ್ವಲ್ಪವಾದರೆ ಸಾಕು, ನಮ್ಮ ಬಾಯಾರಿಕೆ ಹಿಂಗುವುದು. ಶ‍್ರೀರಾಮ ಕೃಷ್ಣರು ಹೇಳುವಂತೆ ನಾವು ಭಾವಿಯಲ್ಲಿ ಎಷ್ಟು ನೀರಿದೆ ಎಂದು ತಿಳಿಯಬೇಕಾಗಿಲ್ಲ. ಅದರಿಂದ ಒಂದೆರಡು ಲೋಟ ಕುಡಿದರೆ ಸಾಕು, ನಮ್ಮ ದಾಹ ಶಮನವಾಗುವುದು. ಹಾಗೆಯೆ ಗೀತೆಯ ಒಂದೆರಡು ಗುಟುಕು ಸಾಕು ನಮ್ಮ ಭವರೋಗದ ದಾಹವನ್ನು ಶಮನ ಮಾಡಲು.

ಇದು ರೋಮಾಂಚಕಾರಿಯಾದ ಸಂವಾದ ಎನ್ನುತ್ತಾನೆ ಸಂಜಯ. ಸಂಜಯ ಕುರುಕ್ಷೇತ್ರಕ್ಕೆ ದೂರವಾಗಿದ್ದರೂ ತನ್ನ ಕಣ್ಣಿನಿಂದ ಶ‍್ರೀಕೃಷ್ಣ ಪರಮಾತ್ಮನನ್ನು ಕಂಡ. ಅವನ ದಿವ್ಯವಾಣಿಯನ್ನು ಕಿವಿಯಾರೆ ಕೇಳಿದ. ಇಂತಹ ಅಭೂತ ಪೂರ್ವವಾದ ಮಾತನ್ನು ಕೇಳಿದ. ಇದೊಂದು ಅವರ್ಣನೀಯ ಅನುಭವ. ಆಪಾದ ಮಸ್ತಕದವರೆಗೆ ರೋಮ ನಿಮಿರಿ ನಿಲ್ಲುವುದು.

\begin{verse}
ವ್ಯಾಸಪ್ರಸಾದುಚ್ಛ್ರತವಾನಿಮಂ ಗುಹ್ಯಮಹಂ ಪರಮ್~।\\ಯೋಗಂ ಯೋಗೇಶ್ವರಾತ್ ಕೃಷ್ಣಾತ್ ಸಾಕ್ಷಾತ್ ಕಥಯತಃ ಸ್ವಯಮ್ \versenum{॥ ೭೫~॥}
\end{verse}

{\small ವ್ಯಾಸರ ಅನುಗ್ರಹದಿಂದ ಶ್ರೇಷ್ಠವಾದ, ಗುಹ್ಯವಾದ ಈ ಯೋಗವನ್ನು ಸಾಕ್ಷಾತ್ತಾಗಿ ತಾನೇ ಹೇಳುತ್ತಿರುವ ಯೋಗೇಶ್ವರನಾದ ಶ‍್ರೀಕೃಷ್ಣನ ಬಾಯಿಂದ ಕೇಳಿದೆನು.}

ಸಂಜಯ ಸಾಕ್ಷಾತ್ ಶ‍್ರೀಕೃಷ್ಣನ ಬಾಯಿಂದ ಕೇಳುವುದಕ್ಕೆ ವ್ಯಾಸರೇ ಕಾರಣ. ಯುದ್ಧ ಪ್ರಾರಂಭ ದಲ್ಲಿ ವ್ಯಾಸರು ಧೃತರಾಷ್ಟ್ರನ ಸಮೀಪಕ್ಕೆ ಬಂದು ನಿನಗೆ ಯುದ್ಧವನ್ನು ನೋಡಬೇಕೆಂದು ಇಚ್ಛೆ ಯಾದರೆ ನಾನು ದಿವ್ಯ ಕಣ್ಣುಗಳನ್ನು ಕೊಡುತ್ತೇನೆ ಎನ್ನುತ್ತಾರೆ. ಆಗ ಧೃತರಾಷ್ಟ್ರನು ಆ ಕೊಲೆಯನ್ನು ತನ್ನ ಕಣ್ಣಿನಿಂದ ನೋಡ ಬಯಸಲಿಲ್ಲ. ಆಗ ವ್ಯಾಸರು ಸಂಜಯನಿಗೆ ಆ ದಿವ್ಯ ದೃಷ್ಟಿಯನ್ನು ಕೊಡುತ್ತಾರೆ. ಇದರ ಮೂಲಕ ದೂರದ ಕುರುಕ್ಷೇತ್ರದ ಯುದ್ಧದಲ್ಲಿ ಆಗುತ್ತಿರುವುದನ್ನು ಅವನು ನೋಡುವುದಕ್ಕೆ ಮತ್ತು ಕೇಳುವುದಕ್ಕೆ ಸಾಧ್ಯವಾಯಿತು. ಸಾಕ್ಷಾತ್ ಭಗವಂತನ ವಾಣಿಯನ್ನು ಇವನು ಹತ್ತಿರ ನಿಂತುಕೊಂಡು ಕೇಳುತ್ತಿರುವಂತೆ ಅನುಭವಿಸಿರುವನು.

\begin{verse}
ರಾಜನ್ ಸಂಸ್ಮೃತ್ಯ ಸಂಸ್ಮೃತ್ಯ ಸಂವಾದಮಿಮಮದ್ಭುತಮ್~।\\ಕೇಶವಾರ್ಜುನಯೋಃ ಪುಣ್ಯಂ ಹೃಷ್ಯಾಮಿ ಚ ಮುಹುರ್ಮುಹುಃ \versenum{॥ ೭೬~॥}
\end{verse}

{\small ಧೃತರಾಷ್ಟ್ರ, ಕೇಶವಾರ್ಜುನರ ಈ ಅದ್ಭುತವಾದ, ಪುಣ್ಯವಾದ ಸಂವಾದವನ್ನು ಸ್ಮರಿಸಿಕೊಂಡು ಪುನಃ ಪುನಃ ನಾನು ಹರ್ಷಪಡುತ್ತಿರುವೆನು.}

ಕೇಶವಾರ್ಜುನರ ಸಂವಾದವನ್ನು ಕೇಳಿದಾಗ ಆಗುವ ಆನಂದ ಸ್ವಲ್ಪ ಕಾಲದ ಮೇಲೆ ಹಳೆಯದಾಗು ವುದಿಲ್ಲ. ಅದನ್ನು ಎಷ್ಟು ಸಲ ಸ್ಮರಿಸಿಕೊಂಡರೂ ಮನಸ್ಸಿಗೆ ತೃಪ್ತಿಯಿಲ್ಲ. ಪುನಃ ಪುನಃ ಅದನ್ನೇ ಚಿಂತಿಸಿಕೊಳ್ಳಬೇಕೆನ್ನುವುದು. ಅಂತಹ ಆಕರ್ಷಣೆ ಇದೆ ಅದರಲ್ಲಿ. ಅದು ಅದ್ಭುತವಾದ ಸಂವಾದ. ಜ್ಞಾನ, ಭಕ್ತಿ, ಯೋಗಗಳ ಸಮನ್ವಯವನ್ನು ಇಲ್ಲಿ ನೋಡುತ್ತೇವೆ. ಇದು ಪರಮ ಪವಿತ್ರವಾದದ್ದು. ಕೇಳಿದರೆ ನಮ್ಮ ಮನಸ್ಸಿನಲ್ಲಿ ಒಳ್ಳೆಯ ಸಂಸ್ಕಾರವನ್ನು ಬಿಡುವುದು. ನಮ್ಮ ಕೊಳೆಯನ್ನೆಲ್ಲಾ ತೊಳೆಯುವುದು.

\begin{verse}
ತಚ್ಚ ಸಂಸ್ಮೃತ್ಯ ಸಂಸ್ಮೃತ್ಯ ರೂಪಮತ್ಯದ್ಭುತಂ ಹರೇಃ~।\\ವಿಸ್ಮಯೋ ಮೇ ಮಹಾನ್ ರಾಜನ್ ಹೃಷ್ಯಾಮಿ ಚ ಪುನಃ ಪುನಃ \versenum{॥ ೭೭~॥}
\end{verse}

{\small ಧೃತರಾಷ್ಟ್ರ, ಹರಿಯ ಅದ್ಭು‡ತವಾದ ಆ ರೂಪವನ್ನು ನೆನೆನೆನೆದು ನನಗೆ ಮಹಾವಿಸ್ಮಯವಾಗುತ್ತಿದೆ ಮತ್ತು ಪುನಃ ಪುನಃ ಹರ್ಷಪಡುತ್ತಿದ್ದೇನೆ.}

ಭಗವಂತನ ವಿಶ್ವರೂಪವನ್ನು ನೋಡಿ, ಅದನ್ನು ನೆನೆದರೇನೆ ಆಶ್ಚರ್ಯವಾಗುತ್ತಿದೆ. ಅಲ್ಲಿ ಮುಂದೆ ಕೌರವರಿಗೆ ಏನು ಕಾದಿದೆ ಎಂಬುದು ನಿಸ್ಸಂದೇಹವಾಗಿ ಕಾಣುತ್ತಿತ್ತು. ಆ ಭಗವಂತನ ವಿಶ್ವರೂಪ ಒಂದು ಅಪೂರ್ವ ದೃಶ್ಯ. ಹೊರಗೆ ಭಯಾನಕವಾದ ವಿಶ್ವವನ್ನು ಸಂಹಾರ ಮಾಡುವ ದೃಶ್ಯ. ಅದರ ಹಿಂದುಗಡೆಯೇ ಭಗವಂತನ ಭಕ್ತರು ಪ್ರಿಯತಮನಾದ ತಂದೆಯು ಅಡಗಿರುವುದನ್ನು ಕಾಣುತ್ತಿ ದ್ದರು. ಭಯ ಮತ್ತು ವಿಸ್ಮಯಗಳೆರಡನ್ನೂ ಉದ್ರೇಕಿಸುವಂತಹ ದೃಶ್ಯ ಅದು. ಅದನ್ನು ನೆನೆದು ಪುನಃ ಪುನಃ ಹರ್ಷಪಡುತ್ತಿದ್ದೇನೆ ಎನ್ನುತ್ತಾನೆ ಸಂಜಯ. ಸಂಜಯ ಭಕ್ತ. ವಿಶ್ವರೂಪದ ಭಯಾನಕ ದರ್ಶನದ ಹಿಂದೆ ಅವಿತಿರುವ ಪ್ರೇಮಮಯ ಮೂರ್ತಿಯನ್ನು ಅವನು ನೋಡಬಲ್ಲವನಾಗಿದ್ದ. ಅದನ್ನೇ ನೋಡಿ, ನೆನೆದು ಆನಂದ ಪಡುತ್ತಿರುವನು.

\begin{verse}
ಯತ್ರ ಯೋಗೇಶ್ವರಃ ಕೃಷ್ಣೋ ಯತ್ರ ಪಾರ್ಥೋ ಧನುರ್ಧರಃ~।\\ತತ್ರ ಶ‍್ರೀರ್ವಿಜಯೋ ಭೂತಿರ್ಧ್ರುವಾ ನೀತಿರ್ಮತಿರ್ಮಮ \versenum{॥ ೭೮~॥}
\end{verse}

{\small ಎಲ್ಲಿ ಯೋಗೇಶ್ವರನಾದ ಶ‍್ರೀಕೃಷ್ಣನಿರುವನೊ, ಎಲ್ಲಿ ಧನುರ್ಧಾರಿಯಾದ ಪಾರ್ಥನಿರುವನೊ ಅಲ್ಲಿ ಶ‍್ರೀ, ವಿಜಯ, ಐಶ್ವರ್ಯ, ಶಾಶ್ವತವಾದ ಧರ್ಮ ಇವುಗಳು ನೆಲೆಸಿರುತ್ತವೆ ಎಂಬುದೇ ನನ್ನ ಮತ.}

ಕೌರವರಿಗೆ ಮುಂದೆ ಯುದ್ಧದಲ್ಲಿ ಏನಾಗಬಹುದು ಎಂಬುದನ್ನು ಸೂಚಿಸುತ್ತಾನೆ. ಎಲ್ಲಿ ಯೋಗೇಶ್ವರನಾದ ಶ‍್ರೀಕೃಷ್ಣನಿರುವನೋ ಎಲ್ಲವನ್ನೂ ತಿಳಿದುಕೊಂಡ ಪರಮ ಜ್ಞಾನಿಯಾದ ಭಗ ವಂತನೇ ಗುರುವಿನಂತೆ ಆಜ್ಞಾಪಿಸುತ್ತಿರುವನೋ, ಅವನು ಹೇಳುತ್ತಿರುವುದನ್ನು ಅನುಷ್ಠಾನಕ್ಕೆ ತರುವು ದಕ್ಕೆ ಸಿದ್ಧನಾದ ಪಾರ್ಥನಿರುವನೊ ಅಲ್ಲಿ ಶ‍್ರೀ ಎಂದರೆ ಮಂಗಳ ಇರುವುದು. ಮುಂಚೆ ಶೋಕದ ಮಂಜು ಕವಿದಿರಬಹುದು. ಆದರೆ ಇವೆಲ್ಲಾ ಸೂರ್ಯೋದಯದ ಎದುರಿಗೆ ಬಾಗಿ ಸರಿಯುವುದು. ಮಂಗಳವೊಂದೇ ಕೊನೆಗೆ ನಿಲ್ಲವುದು. ಅಲ್ಲಿ ವಿಜಯ ಸುನಿಶ್ಚಿತ. ವಿಜಯಕ್ಕೆ ಬೇಕಾದಷ್ಟು ಆತಂಕಗಳಿರಬಹುದು. ಆದರೆ ಭಗವಂತ ಅಸಾಧ್ಯವನ್ನು ಸಾಧ್ಯವನ್ನಾಗಿ ಮಾಡುತ್ತಾನೆ. ಅವನನ್ನು ನೆಚ್ಚಿದವನಿಗೆ ಸೋಲೆಂಬುದಿಲ್ಲ. ಅವನು ಹಿಡಿಯುವುದೆಲ್ಲಾ ಜಯದಲ್ಲಿ ಪರ್ಯವಸಾನವಾಗುವುದು. ಜಯದೊಡನೆ ಈ ಲೋಕದ ಸಂಪತ್ತೂ ಕೂಡ ಬರುವುದು. ಅದನ್ನು ಬೇಕಾದರೆ ಒಬ್ಬ ತೆಗೆದುಕೊಳ್ಳ ಬಹುದು ಅಥವಾ ಬಿಡಬಹುದು. ಅದಂತೂ ಜಯದ ನೆರಳಿನಂತೆ ಬರುವುದು. ಪ್ರಪಂಚದಲ್ಲಿ ಶಾಶ್ವತವಾದ ನೀತಿ ನೆಲೆಸುವುದು. ಆಸುರೀ ಶಕ್ತಿಗಳು ಈಗ ಎಷ್ಟೇ ಬಲವಾಗಿದ್ದರೂ ಕೊನೆಗೆ ಅವು ನಾಶವಾಗುವುವು. ಈ ಪ್ರಪಂಚದಲ್ಲಿ ಭಗವಂತ ಅವತರಿಸುವುದೇ ದುರ್ಜನರನ್ನು ನಾಶ ಮಾಡುವು ದಕ್ಕೆ, ಧರ್ಮವನ್ನು ಸ್ಥಾಪನೆ ಮಾಡುವುದಕ್ಕೆ. ಸಮಾಜಕ್ಕೆ ಕಲ್ಯಾಣಕಾರಿಯಾದ ನೀತಿಯೊಂದೇ ಕೊನೆಗೆ ನಿಲ್ಲುವುದು.

