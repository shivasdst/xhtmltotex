
\chapter{ಜ್ಞಾನ–ವಿಜ್ಞಾನಯೋಗ}

ಶ‍್ರೀಕೃಷ್ಣ ಅರ್ಜುನನಿಗೆ ಹೇಳುತ್ತಾನೆ.

\begin{shloka}
ಮಯ್ಯಾಸಕ್ತಮನಾಃ ಪಾರ್ಥ ಯೋಗಂ ಯುಂಜನ್ ಮದಾಶ್ರಯಃ~।\\ಅಸಂಶಯಂ ಸಮಗ್ರಂ ಮಾಂ ಯಥಾ ಜ್ಞಾಸ್ಯಸಿ ತಚ್ಛೃಣು \hfill॥ ೧~॥
\end{shloka}

\begin{artha}
ಅರ್ಜುನ! ನನ್ನಲ್ಲಿ ಮನಸ್ಸನ್ನಿಟ್ಟು, ನನ್ನಲ್ಲಿ ಆಶ್ರಯವನ್ನು ಹೊಂದಿ, ಯೋಗ ಸಾಧನೆಯನ್ನು ಮಾಡುತ್ತ, ನೀನು ನನ್ನನ್ನು ನಿಸ್ಸಂಶಯವಾಗಿಯೂ ಸಂಪೂರ್ಣವಾಗಿಯೂ ಹೇಗೆ ಅರಿಯಬಲ್ಲೆಯೊ ಕೇಳು.
\end{artha}

ಶ‍್ರೀಕೃಷ್ಣ ಈ ಅಧ್ಯಾಯದಲ್ಲಿ ಈಶ್ವರತತ್ತ್ವ ಮತ್ತು ಭಕ್ತಿಯನ್ನು ಹೇಳುತ್ತಾನೆ. ಭಗವಂತನನ್ನು ಅರಿಯಬೇಕಾದರೆ ಮನಸ್ಸನ್ನೆಲ್ಲ ದೇವರಲ್ಲಿಡಬೇಕು. ಇತರ ವಸ್ತುಗಳ ಮೇಲೆಯೂ ಮನಸ್ಸಿಟ್ಟು ದೇವರನ್ನೂ ತಿಳಿದುಕೊಳ್ಳುತ್ತೇನೆ ಎಂದರೆ, ಅವನನ್ನು ಚೆನ್ನಾಗಿ ತಿಳಿದುಕೊಳ್ಳುವುದಕ್ಕೆ ಆಗುವುದಿಲ್ಲ. ಹತ್ತರಲ್ಲಿ ಹನ್ನೊಂದರಂತೆ ದೇವರನ್ನು ನೋಡುವುದು, ದೇವರಲ್ಲಿ ಮನಸ್ಸನ್ನು ಇಟ್ಟಂತೆ ಅಲ್ಲ. ಅವನಿಗಾಗಿ ಎಲ್ಲವೂ ಎಂಬ ದೃಷ್ಟಿ ಬರಬೇಕು. ಆಗಲೆ ಮನಸ್ಸು ಅವನ ಕಡೆ ಹೋಗಬೇಕಾದರೆ. ದೇವರು ಅಸೂಯಾಪರನಾದ ವ್ಯಕ್ತಿ ಎಂದು ಒಬ್ಬ ಭಕ್ತ ಹೇಳುತ್ತಾನೆ. ಅವನಿಗೆ ತನ್ನನ್ನು ಪ್ರೀತಿಸುವವನು, ಇನ್ನಾವುದನ್ನೂ ಪ್ರೀತಿಸಬಾರದು. ಅವನೊಬ್ಬನಿಗೇ ಮುಡುಪಾಗಿರಬೇಕು ಮನಸ್ಸು. ನಾವು ನಮ್ಮ ಸರ್ವಸ್ವವನ್ನು ಅವನಿಗೆ ಕೊಟ್ಟರೆ, ಅವನು ತನ್ನ ಸರ್ವಸ್ವವನ್ನೂ ನಮಗೆ ಕೊಡುವನು.

ಭಕ್ತ ಭಗವಂತನಲ್ಲಿಯೇ ಆಶ್ರಯವನ್ನು ಪಡೆಯಬೇಕು. ಇತರ ವಸ್ತುಗಳಲ್ಲಿ ಆಶ್ರಯ ಪಡೆಯ ಬಾರದು. ಅವನಿಗೆ ಕಷ್ಟಕಾಲದಲ್ಲಿ ದೇವರೊಬ್ಬನೇ ತನ್ನವನು ಎಂಬುದು ಗೊತ್ತಿದೆ. ಸ್ನೇಹಿತರಲ್ಲ, ಬಂಧೂ ಬಳಗ ಅಲ್ಲ, ಹಣ ಅಧಿಕಾರ ಅಲ್ಲ. ಅವನು ಯಾವಾಗಲೂ ದೇವರನ್ನೇ ನೆಚ್ಚಿರುವನು. ಗೀತೆಯನ್ನೆ ನಮ್ಮ ಜೀವನದಲ್ಲಿ ಅನುಷ್ಠಾನ ಮಾಡುತ್ತಿದ್ದ ಗಾಂಧೀಜಿ ಲೈಫ್ ಇನ್ಷುರೆನ್ಸ್ ಮಾಡಿರಲಿಲ್ಲವಂತೆ. ಏತಕ್ಕೆ ಎಂದು ಕೇಳಿದಾಗ, ದೇವರನ್ನು ನೆಚ್ಚುವುದು ಮತ್ತು ಜೀವವಿಮೆ ಮಾಡುವುದು ಎರಡೂ ಒಟ್ಟಿಗೆ ಹೋಗುವುದಿಲ್ಲ ಎನ್ನುತ್ತಾರೆ. ಅದರಂತೆ ಅವರ ಜೀವನದಲ್ಲಿ ಮತ್ತೊಂದು ನಿದರ್ಶನವನ್ನು ನೋಡುತ್ತೇವೆ. ಒಬ್ಬ ಸಾಧು, ಪಂಡಿತ ಮದನಮೋಹನ ಮಾಲವೀಯ ಅವರಿಗೆ ಕಾಯಕಲ್ಪಚಿಕಿತ್ಸೆಯನ್ನು ಮಾಡಿ, ಗಾಂಧೀಜಿ ಅವರ ಬಳಿಗೆ ಬಂದು ಆ ಚಿಕಿತ್ಸೆಯನ್ನು ಗಾಂಧೀಜಿ ಯವರಿಗೂ ಉಚಿತವಾಗಿ ಮಾಡುತ್ತೇನೆ, ದಯವಿಟ್ಟು ಅವಕಾಶಕೊಡಿ ಎಂದು ಕೇಳುತ್ತಾನೆ. ಆಗ ಗಾಂಧೀಜಿ, ಆತನಿಗೆ ಹೇಳುತ್ತಾರೆ: ದೇವರು ನನ್ನ ದೇಹವನ್ನು ಎಲ್ಲಿಯವರೆಗೆ ಉಪಯೋಗಿಸಿಕೊಳ್ಳು ವನೊ ಅವನ ಕೆಲಸ ಮಾಡುವುದಕ್ಕೆ ಅಲ್ಲಿಯವರೆಗೆ ಅದಕ್ಕೆ ಬೇಕಾದ ಆರೋಗ್ಯವನ್ನು ಅವನು ಕೊಡುತ್ತಾನೆ. ಅವನಿಗೆ ಇದರಿಂದ ಮಾಡಬೇಕಾದ ಕೆಲಸ ಆಯಿತು ಎಂದು ಗೊತ್ತಾದರೆ ದೊಡ್ಡ ದೊಂದು ವೈದ್ಯರ ಪಡೆಯೇ ಇರಬಹುದು, ಆದರೂ ಅವನು ಆ ಜೀವವನ್ನು ಹರಣ ಮಾಡುವನು ಎಂದರು. ನಿಜವಾಗಿ ಭಗವಂತನಲ್ಲಿ ಶರಣಾದವನ ದೃಷ್ಟಿ ಇದು.

ಅವನು ಯಾವಾಗಲೂ ಯೋಗಸಾಧನೆ ಮಾಡುತ್ತಿರುವನು. ಭಗವಂತನ ಕಡೆ ತನ್ನ ಮನಸ್ಸ\-ನ್ನೆಲ್ಲ ಹರಿಸುತ್ತಿರುವನು. ಯಾವುದಾದರೂ ಕೆಲಸದಲ್ಲಿ ನಿರತನಾಗಿರುವಾಗಲೂ ಅವನ ಮನಸ್ಸಿನ ಒಂದು ಭಾಗ ಸದಾ ದೇವರ ಕಡೆ ತಿರುಗುವುದು. ಕಮಲ ಸೂರ್ಯನ ಕಡೆ ತಿರುಗುವಂತೆ, ಸಾಗರದ ಕಡೆಗೆ ನದಿ ಹರಿದುಹೋಗುವಂತೆ, ಮಧ್ಯೆ ಏನು ಆತಂಕಗಳು ಬಂದರೂ ಮನಸ್ಸನ್ನು ದೇವರ ಕಡೆ ಕಳಿಸುತ್ತಿರುವನು. ಅವನೇ ಯೋಗಿ. ಇದನ್ನು ಮೊದಲು ಪ್ರಯತ್ನಪೂರ್ವಕ ಅಭ್ಯಾಸ ಮಾಡಬೇಕು. ಅಭ್ಯಾಸ ಯಾವಾಗ ಆಳಕ್ಕೆ ಇಳಿಯುತ್ತ ಬರುವುದೋ ಆಗ ಅದು ನಮ್ಮ ಸ್ವಭಾವ ಆಗುತ್ತ ಬರುವುದು. ಅನೈಚ್ಛಿಕವಾಗಿಯೂ ಮನಸ್ಸು ಭಗವಂತನ ಕಡೆ ತಿರುಗುವುದು.

ನನ್ನನ್ನು ನಿಸ್ಸಂಶಯವಾಗಿ ಹೇಗೆ ತಿಳಿದುಕೊಳ್ಳಬಲ್ಲೆಯೊ ಅದನ್ನು ಕೇಳು ಅನ್ನುತ್ತಾನೆ. ಬರೀ ವಿಚಾರ ವೇದಿಕೆಯ ಮೇಲೆ ನಿಂತು, ಬುದ್ಧಿಯ ಕಸರತ್ತಿನಿಂದಲೇ ದೇವರನ್ನು ತಿಳಿಯಬಯಸುತ್ತೇನೆ ಎಂದರೆ, ಈಗ ತಿಳಿದಂತೆ ಕಾಣುವುದು, ಅನಂತರ ಸಂದೇಹ ಪುನಃ ಆವರಿಸುವುದು. ಒಂದು ಸಂದೇಹ ಹೋಗುವುದು, ಮತ್ತೊಂದು ಸಂದೇಹ ಬರುವುದು. ಸಂಶಯದಿಂದ ಪಾರಾಗಬೇಕಾದರೆ ಪ್ರತ್ಯಕ್ಷ ನಾವು ಅದನ್ನು ಅನುಭವಿಸಿರಬೇಕು. ಬರೀ ಬುದ್ಧಿಯಿಂದ ಇದು ಸರಿ ಇರಬೇಕು ಎಂದು ತಿಳಿಯಲಾಗುವುದಿಲ್ಲ. ಬರೀ ಬುದ್ಧಿ ನಮ್ಮನ್ನು ಹೆಚ್ಚು ದೂರ ಕರೆದುಕೊಂಡು ಹೋಗಲಾರದು. ಸ್ವಲ್ಪ ಸಂಶಯದ ಗಾಳಿ ಬೀಸಿತು ಎಂದರೆ ತರಗೆಲೆಯಂತೆ ಹಾರಿ ಹೋಗುವುದು ಬುದ್ಧಿ. ಅನುಭವ ಸಿಕ್ಕಿದರೆ ಮಾತ್ರ ದೊಡ್ಡದೊಂದು ಕಲ್ಲುಬಂಡೆಯಂತೆ ಎಂತಹ ಸಂಶಯದ ಬಿರುಗಾಳಿಯನ್ನಾದರೂ ಸಹಿಸಬಲ್ಲದು. ಸಂಶಯ ರೋಗದಿಂದ ಪಾರಾಗಬೇಕಾದರೆ ಇರುವುದು ಒಂದೇ ಔಷಧ. ಅದೇ ಅನುಭವ. 

ಮತ್ತು ಅವನನ್ನು ಸಂಪೂರ್ಣವಾಗಿ ತಿಳಿಯಬೇಕು. ಮನುಷ್ಯ ಯಾವುದೊ ಒಂದು ದೃಷ್ಟಿಕೋನದಿಂದ ಪರಮಾತ್ಮನನ್ನು ತಿಳಿಯುತ್ತಾನೆ. ಇದು ಅಂಶದೃಷ್ಟಿ. ಎಲ್ಲಿಯವರೆಗೆ ಅಂಶದೃಷ್ಟಿ ಮಾತ್ರ ಅವನಿಗೆ ಇರುವುದೊ ಅಲ್ಲಿಯವರೆಗೆ ತನ್ನದೇ ಸರಿ ಎಂದು ಭಾವಿಸುವನು. ಆದರೆ ಪೂರ್ಣದೃಷ್ಟಿ ಯುಳ್ಳವನು ಪರಮಾತ್ಮನನ್ನು ಎಲ್ಲಾ ದೃಷ್ಟಿಕೋನಗಳಿಂದಲೂ ನೋಡಿರುವನು. ಅವನನ್ನು ಹಲವು ತತ್ತ್ವದೃಷ್ಟಿಕೋನಗಳ ಮೂಲಕ ನೋಡಿರುವನು. ಹಲವು ಭಾವನೆಗಳ ಮೂಲಕ ಪ್ರೀತಿಸಿರುವನು. ಹಲವು ಧರ್ಮಗಳ ಮೂಲಕ ಹೋದರೆ ಸಿಕ್ಕುವವನೊಬ್ಬನೇ ಎನ್ನುವನು. ಪೂರ್ಣದೃಷ್ಟಿ ಮಾತ್ರ ಎಲ್ಲಾ ಅಂಶ ದೃಷ್ಟಿಗಳನ್ನೂ ಒಳಗೊಳ್ಳುವುದು. ಅದು ಎಲ್ಲದಕ್ಕೂ ಒಂದು ಸ್ಥಾನವನ್ನು ಕೊಡುವುದು, ಇದು ಮಾತ್ರ ಸರಿ ಎಂದು ಹೇಳುವುದಿಲ್ಲ. ನಿನಗೆ ಈಗ ಇದು ಹೇಗೆ ಸರಿಯೋ, ಅದರಂತೆಯೇ ಬೇರೆ ದೃಷ್ಟಿಯಿಂದ ಇನ್ನೊಬ್ಬನಿಗೆ ಅದು ಸರಿ. ಹಲವು ಸತ್ಯಗಳಿಲ್ಲ, ಇರುವುದೊಂದೇ. ಅದನ್ನು ಮನುಷ್ಯ ಭಿನ್ನ ಭಿನ್ನ ದೃಷ್ಟಿಕೋಣಗಳಿಂದ ನೋಡುತ್ತಿರುವನು ಎನ್ನುವನು. ಅವನು ಯಾವುದನ್ನೂ ನಿರಾಕರಿಸುವುದಿಲ್ಲ. ಎಲ್ಲವನ್ನೂ ಸ್ವೀಕರಿಸುವನು. ಸಾಗರ ಹೇಗೆ ಎಲ್ಲಾ ನದಿಗಳನ್ನೂ ಸ್ವೀಕರಿಸುವುದೋ, ಹಾಗೆ ಎಲ್ಲಾ ತತ್ತ್ವಗಳು, ಧರ್ಮಗಳು, ಭಾವಗಳು ಪರಮಾತ್ಮನಲ್ಲಿ ಐಕ್ಯವಾಗುವುವು. ಇಂತಹ ಒಂದು ಅನುಭವವನ್ನು ಪಡೆಯುವುದು ಹೇಗೆ ಎಂಬುದನ್ನು ಶ‍್ರೀಕೃಷ್ಣ ಅರ್ಜುನನಿಗೆ ಇನ್ನು ಮುಂದೆ ಹೇಳುತ್ತಾನೆ.

\begin{shloka}
ಜ್ಞಾನಂ ತೇಽಹಂ ಸವಿಜ್ಞಾನಮಿದಂ ವಕ್ಷ್ಯಾಮ್ಯಶೇಷತಃ~।\\ಯಜ್ಜ್ಞಾತ್ವಾ ನೇಹ ಭೂಯೋನ್ಯಜ್ಜ್ಞಾತವ್ಯಮವಶಿಷ್ಯತೇ \hfill॥ ೨~॥
\end{shloka}

\begin{artha}
ಯಾವುದನ್ನು ತಿಳಿದುಕೊಂಡಮೇಲೆ, ಪುನಃ ತಿಳಿದುಕೊಳ್ಳಬೇಕಾಗಿರುವುದು ಯಾವುದೂ ಇಲ್ಲವೊ, ಅಂತಹ ವಿಜ್ಞಾನ ಸಹಿತವಾದ ಜ್ಞಾನವನ್ನು ನಿನಗೆ ನಿಶ್ಶೇಷವಾಗಿ ಹೇಳುತ್ತೇನೆ.
\end{artha}

ಯಾವುದನ್ನು ತಿಳಿದುಕೊಂಡಮೇಲೆ ಇನ್ನು ತಿಳಿದುಕೊಳ್ಳಬೇಕಾಗಿರುವುದು ಯಾವುದೂ\break ಇಲ್ಲವೊ, ಎಂದರೆ ಪರಮಾತ್ಮನನ್ನು ತಿಳಿದುಕೊಂಡಮೇಲೆ ಇನ್ನು ಉಳಿದವೆಲ್ಲವೂ ವೇದ್ಯವಾಗುವುದು. ಈ ಪ್ರಪಂಚದ ಚರಾಚರ ವಸ್ತುವಿನ ಹಿಂದೆ ಸತ್ಯವಾಗಿರುವವನು ಅವನು. ಮಿಕ್ಕಿರುವ ಜ್ಞಾನಗಳಾದರೊ ಆ ಸತ್ಯದ ಯಾವುದೋ ಒಂದು ದೃಷ್ಟಿಕೋಣದ ಅಸ್ಪಷ್ಟಭಾಗ. ಹಾಗಾದರೆ ದೇವರನ್ನು ತಿಳಿದವನಿಗೆ ಫಿಸಿಕ್ಸ್, ಕೆಮಿಸ್ಟ್ರಿ, ಅಸ್ಟ್ರಾನಮಿ, ಚರಿತ್ರೆ, ಭೂಗೋಳ ಇವೆಲ್ಲ ಗೊತ್ತಾಗು\-ವುದೆ? ಇವುಗಳ ಸತ್ಯವೆಲ್ಲ ಅವನಿಗೆ ಹೊಳೆಯುವುದೆ? ಎಂದರೆ ಇತರ ಜ್ಞಾನಗಳ ವಿವರಗಳು\break ಅವನಿಗೆ ಗೋಚರಿಸುವುದಿಲ್ಲ. ಅದನ್ನು ಕೂಡ ಮನಸ್ಸು ಮಾಡಿದರೆ ಬಹಳ ಅಲ್ಪ ಕಾಲದಲ್ಲಿ ತಿಳಿದುಕೊಳ್ಳಬಲ್ಲ. ಆದರೆ ಎಲ್ಲಾ ಜ್ಞಾನಗಳ ಹಿಂದೆ ಇರುವ ಏಕ ಮಾತ್ರ ಅವಿಕಾರಿಯಾದ ಸತ್ಯವನ್ನು ಅವನು ಬಲ್ಲನು. ಅದೇ ಪರಮಾತ್ಮ. ಎಲ್ಲಾ ಶಕ್ತಿಯ ಹಿಂದೆ, ಎಲ್ಲಾ ವಸ್ತುವಿನ ಹಿಂದೆ, ಎಲ್ಲಾ ನಾಮರೂಪಗಳ ಹಿಂದೆ ಇರುವುದೊಂದೇ ಪರಮಾತ್ಮಶಕ್ತಿ. ಈ ಪ್ರಪಂಚವೆಲ್ಲ ಅದರ ಆವಿರ್ಭಾವ. ಈ ಪ್ರಪಂಚದಲ್ಲಿ ಪರಮಾತ್ಮನನ್ನು ಅರಿಯುವುದೊಂದೇ ಪರಾವಿದ್ಯೆ. ಉಳಿದುವುಗಳೆಲ್ಲ ಅಪರವಿದ್ಯೆ.

ಅಂತಹ ವಿಜ್ಞಾನ ಸಹಿತವಾದ ಜ್ಞಾನವನ್ನು ನಾನು ನಿನಗೆ ಹೇಳುತ್ತೇನೆ ಎನ್ನುವನು ಶ‍್ರೀಕೃಷ್ಣ. ಬರಿಯ ಜ್ಞಾನ ಎಂದರೆ ಬೌದ್ಧಿಕವಾಗಿ ತಿಳಿದುಕೊಳ್ಳುವುದು. ವಿಜ್ಞಾನ ಎಂದರೆ ಅನುಷ್ಠಾನಕ್ಕೆ ತಂದದ್ದು, ಅನುಭವದಿಂದ ಕೂಡಿದ್ದು. ವ್ಯವಹಾರದಲ್ಲಿ ಬರದೆ ಬರೀ ಊಹೆಯಂತೆ, ಸಿದ್ಧಾಂತದಂತೆ, ಪಾಂಡಿತ್ಯದಂತೆ ಹಾರಾಡುತ್ತಿರುವುದಲ್ಲ. ಇವೆಲ್ಲ ಬರೀ ಜ್ಞಾನ. ಇಲ್ಲಿ ವಿಜ್ಞಾನ ಎಂದರೆ ನನ್ನ ಅಂತರಾಳ ಅದನ್ನು ಮನಗಂಡಿದೆ, ಅನುಭವಿಸಿದೆ. ಇನ್ನು ಮೇಲೆ ಆ ವಸ್ತು ‘ಅಂತೆ’ ‘ಕಂತೆ’ಯಲ್ಲ. ಸತ್ಯಸ್ಯ ಸತ್ಯ. ಅನುಭವವೇ ಒಂದು ವಸ್ತುವಿನ ಅಸ್ತಿತ್ವಕ್ಕೆ ಶ್ರೇಷ್ಠ ಪ್ರಮಾಣ. ಉಳಿದವುಗಳೆಲ್ಲ ತಾತ್ಕಾಲಿಕ. ಅನುಭವ ಬರುವವರೆಗೆ ಮಾತ್ರ ಅದಕ್ಕೆ ಪುರಸ್ಕಾರ. ಅನುಭವದ ಮುಂದೆ ಯಾವುದೂ ನಿಲ್ಲಲಾರದು. ಒಂದು ತೊಲ ಅನುಭವ ಸಾವಿರಾರು ಟನ್ನು ಸಿದ್ಧಾಂತಗಳಿಗಿಂತ ಮೇಲು. ಒಂದು ಚೂರು ಅನುಭವ ಪಡೆದವನ ಮಾತಿನಲ್ಲಿ ಅಂತಹ ಅದ್ಭುತ ಶಕ್ತಿ ಇದೆ, ಆಕರ್ಷಣೆ ಇದೆ. ಆ ಮಾತು ಒರಟಾಗಿರಬಹುದು. ಆದರೆ ಶಕ್ತಿಯಿಂದ ಸ್ಪಂದಿಸುತ್ತದೆ. ಅದು ಕೇಳುವವನ ಮೇಲೆ ಅದ್ಭುತ ಪರಿಣಾಮವನ್ನು ಉಂಟು ಮಾಡಬಲ್ಲದು. ಇಂತಹ ಜ್ಞಾನವನ್ನು ಸ್ವಲ್ಪವೂ ಬಿಡದೆ ನಿನಗೆ ಹೇಳುತ್ತೇನೆ ಎನ್ನುತ್ತಾನೆ. ಎಲ್ಲವನ್ನೂ ಹೇಳಿದರೆ ಇನ್ನು ನನ್ನ ದೊಡ್ಡಸ್ತಿಕೆಗೆ ಎಲ್ಲಿ ಚ್ಯುತಿ ಬರುವುದೊ ಎಂದು ಕೊಸರುವುದಿಲ್ಲ. ತನಗೆ ತಿಳಿದಿರುವುದನ್ನೆಲ್ಲ ಉದಾರವಾಗಿ ಹೇಳುತ್ತಾನೆ. ಉದಾರವಾಗಿ ಮಾತ್ರ ಹೇಳುವುದಲ್ಲ. ವಿವರವಾಗಿ ಹೇಳುತ್ತಾನೆ. ಅದಕ್ಕೆ ಸಂಬಂಧಪಟ್ಟ ವಿಷಯವನ್ನೆಲ್ಲ ಹೇಳುತ್ತಾನೆ. 

\begin{shloka}
ಮನುಷ್ಯಾಣಾಂ ಸಹಸ್ರೇಷು ಕಶ್ಚಿದ್ಯತತಿ ಸಿದ್ಧಯೇ~।\\ಯತತಾಮಪಿ ಸಿದ್ಧಾನಾಂ ಕಶ್ಚಿನ್ಮಾಂ ವೇತ್ತಿ ತತ್ತ್ವತಃ \hfill॥ ೩~॥
\end{shloka}

\begin{artha}
ಸಹಸ್ರಾರು ಮನುಷ್ಯರಲ್ಲಿ ಯಾರೊ ಒಬ್ಬ ಸಿದ್ಧಿಗಾಗಿ ಪ್ರಯತ್ನಿಸುತ್ತಾನೆ. ಹಾಗೆ ಪ್ರಯತ್ನಿಸುವ ಸಿದ್ಧರಲ್ಲಿ ಕೂಡ ಯಾವನೊ ಒಬ್ಬ ನನ್ನನ್ನು ಯಥಾರ್ಥವಾಗಿ ಅರಿಯುತ್ತಾನೆ.
\end{artha}

ಈ ಪ್ರಪಂಚದಲ್ಲಿ ದೇವರ ಕಡೆ ಹೋಗುವವರ ಸಂಖ್ಯೆ ಅತಿ ವಿರಳ. ಎಲ್ಲಾ ಜಗದ ಸಂತೆಯ ಕಡೆ ಹೋಗುವವರೇ ಹೆಚ್ಚು. ದೇವರು ಸೂಕ್ಷ್ಮ. ಕಣ್ಣಿಗೆ ಕಾಣದುದು, ಅತೀಂದ್ರಿಯ ವಸ್ತು, ಬಲು ಬೇಗ ಸಿಕ್ಕುವುದಿಲ್ಲ. ಇಂದ್ರಿಯ ವಸ್ತುವಾದರೊ ನಮ್ಮ ಕಣ್ಣೆದುರಿಗೆ ಬಿದ್ದಿದೆ. ಅದನ್ನು ನಾವು ತೆಗೆದುಕೊಳ್ಳಬೇಕು ಅಷ್ಟೆ. ಅದಕ್ಕೆ ಅಷ್ಟೊಂದು ಕಷ್ಟಪಡಬೇಕಾಗಿಲ್ಲ. ಕಷ್ಟಪಡಬೇಕಾದರೂ ಸ್ವಲ್ಪ ಮಾತ್ರ. ಪ್ರಯತ್ನ ಮಾಡಿದೊಡನೆಯೇ ಅದು ನಮಗೆ ದೊರಕುವುದು. ಎಲ್ಲಕ್ಕಿಂತ ಹೆಚ್ಚಾಗಿ, ವಿಷಯ ವಸ್ತುವನ್ನು ಅನುಭವಿಸಬೇಕೆಂಬ ಆಸೆ ಇನ್ನೂ ನಮ್ಮಲ್ಲಿದೆ. ಪರಮಾತ್ಮನನ್ನು ಅನುಭವಿಸಬೇಕೆಂಬ ಆಸೆ ಇನ್ನೂ ನಮ್ಮಲ್ಲಿ ಹುಟ್ಟಿಲ್ಲ. ಎದುರಿಗೆ ಕಾಣುವ ಮನೋಹರವಾದ ಇಂದ್ರಿಯ ಸುಖಗಳು ತಾತ್ಕಾಲಿಕ, ಕ್ಷಣಿಕ, ಇವು ನಮಗೆ ಪರಮಶಾಂತಿಯನ್ನು ಕೊಡಲಾರವು ಎಂಬುದು ಇನ್ನೂ ನಮಗೆ ಹೃದ್​ಗತವಾಗಿಲ್ಲ. ಈ ಪ್ರಪಂಚದ ಮೇಲಿನ ವ್ಯಾಮೋಹ ಹೋದಲ್ಲದೆ ಯಾರೂ ದೇವರ ಕಡೆ ತಿರುಗುವುದಿಲ್ಲ. ಒಂದು ವೇಳೆ ದೇವರ ಹತ್ತಿರ ಅಕಸ್ಮಾತ್ ಬಂದರೂ\break ಅವನಿಂದ ಲೌಕಿಕವಾದ ಏನನ್ನೊ ವಸೂಲಿ ಮಾಡುವುದಕ್ಕೆ ಬರುವನೆ ಹೊರತು, ಅವನಿಗೆ\break ದೇವರು ಬೇಕಾಗಿಲ್ಲ. ಆದ ಕಾರಣವೇ ಸಹಸ್ರಾರು ಜನ ಪ್ರಪಂಚದ ಕಡೆ ಧಾವಿಸುತ್ತಿದ್ದರೆ ದೇವರ ಕಡೆ ಹೋಗುವವರು ಒಂಟಿವರೆ ಮಾತ್ರ. ಈ ಹಾದಿಯಲ್ಲಿ ಯಾವ ಗಲಾಟೆಯೂ ಇಲ್ಲ. ಇಲ್ಲಿ ಪ್ರಯಾಣಿಕರು ಬಹಳ ಅಲ್ಪ.

ಸಹಸ್ರಾರು ಜನರಲ್ಲಿ ಒಬ್ಬನೊ ಇಬ್ಬರೊ ಇತ್ತ ತಿರುಗುವರು. ಹೀಗೆ ತಿರುಗಿದವರಲ್ಲಿಯೂ ಎಷ್ಟೋ ಜನ ಸ್ವಲ್ಪ ಕಾಲವಾದಮೇಲೆ ಇದು ನಮ್ಮ ಕೈಯಿಂದ ಸಾಧ್ಯವಿಲ್ಲ ಎಂದು ಹಿಂತಿರುಗುವವರೆ. ಇನ್ನುಳಿದವರು ದಾರಿಯಲ್ಲಿ ಯಾವುದಾದರೂ ಅಡ್ಡಹಾದಿಯನ್ನು ಹಿಡಿದು ಅಲ್ಲಿ ಸಿಕ್ಕುವ ಕೆಲವು ಕೆಲಸಕ್ಕೆ ಬಾರದ ಚೂರುಪಾರುಗಳನ್ನು ಆಯ್ದು ಅದನ್ನು ಸಾಮಾನ್ಯ ಜನರಿಗೆ ತೋರಿ ಅದರಿಂದ ಕೀರ್ತಿ ಗೌರವಗಳನ್ನು ಪಡೆಯುವರು. ಇನ್ನು ಸ್ವಲ್ಪ ಮುಂದೆ ಹೋಗಿ ಪರಮಾತ್ಮನ ತಾತ್ಕಾಲಿಕ ಅನುಭವಗಳನ್ನು ಪಡೆದು ಅದೇ ಸರ್ವಸ್ವ ಎಂದು ಭಾವಿಸಿ ಅಲ್ಲೆ ನಿಂತವರು ಕೆಲವರು. ಸಾವಿರಾರು ಜನ ಹೀಗೆ ಪ್ರಯತ್ನ ಪಡುವವರಲ್ಲಿ ದೇವರನ್ನು ಯಥಾರ್ಥವಾಗಿ\break ತಿಳಿದುಕೊಳ್ಳುವವರು ಅಪರೂಪ. ಯಥಾರ್ಥವಾಗಿ ಎಂದರೆ ಅವನು ಹೇಗೆ ಇರುವನೊ ಹಾಗೆ ತಿಳಿದುಕೊಳ್ಳುವುದು. ನಾವು ಅದನ್ನು ಹಾಗೆ ತಿಳಿದುಕೊಳ್ಳಬೇಕಾದರೆ, ನಾವು ಏನನ್ನು ಆರೋಪ\-ಮಾಡಿರುವೆವೊ ಅವುಗಳನ್ನೆಲ್ಲ ತೆಗೆಯಬೇಕು. ಆಗ ಮಾತ್ರ ವಸ್ತುವಿನ ಯಥಾರ್ಥ ಸ್ಥಿತಿ ನಮಗೆ ಅರಿವಾಗುವುದು. ನಾವು ಬಿಸಿಲಿನಲ್ಲಿ ಹೋಗುವಾಗ ಕೆಲವು ವೇಳೆ ಅದರ ಝಳದಿಂದ ಪಾರಾಗುವುದಕ್ಕೆ ಬಣ್ಣದ ಕನ್ನಡಕವನ್ನು ಉಪ ಯೋಗಿಸುತ್ತೇವೆ. ಆಗ ಹೊರಗೆಲ್ಲ ಆ ಗಾಜಿನ ಬಣ್ಣ ಕಾಣುವುದು. ಅದರಂತೆಯೇ ನಾವು ಅನೇಕವೇಳೆ ನಮಗೆ ಪ್ರಿಯವಾದ ರೀತಿ ದೇವರನ್ನು ಕಾಣಬೇಕು ಎಂದು ಬಯಸಿ ಆರೋಪದ ಮೂಲಕ ನೋಡುತ್ತೇವೆ. ಅವನು ಹೇಗಿರುವನೊ ಹಾಗೆ ಅನುಭವಿಸಬೇಕಾದರೆ ನಾವು ಆರೋಪ ಮಾಡಿರುವುದನ್ನು ತೆಗೆದರೆ ಮಾತ್ರ ಸಾಧ್ಯ. ದೇವರನ್ನು ಯಥಾರ್ಥವಾಗಿ ತಿಳಿದುಕೊಳ್ಳಬಯಸುವವನು ಮಹಾವೀರ. ಇಂತಹವರ ಸಂಖ್ಯೆ ಈ ಪ್ರಪಂಚದಲ್ಲಿ ಯಾವಾಗಲೂ ಅತಿವಿರಳ.

\begin{shloka}
ಭೂಮಿರಾಪೋಽನಲೋ ವಾಯುಃ ಖಂ ಮನೋ ಬುದ್ಧಿರೇವ ಚ~।\\ಅಹಂಕಾರ ಇತೀಯಂ ಮೇ ಭಿನ್ನಾ ಪ್ರಕೃತಿರಷ್ಟಧಾ \hfill॥ ೪~॥
\end{shloka}

\begin{artha}
ಭೂಮಿ, ನೀರು, ಅಗ್ನಿ, ವಾಯು, ಆಕಾಶ, ಮನಸ್ಸು, ಬುದ್ಧಿ ಮತ್ತು ಅಹಂಕಾರ ಎಂದು ಈ ನನ್ನ ಪ್ರಕೃತಿ ಎಂಟು ವಿಧವಾಗಿ ಭಿನ್ನವಾಗಿದೆ.
\end{artha}

ದೇವರ ಪ್ರಕೃತಿಯೇ ಈ ಸೃಷ್ಟಿಗೆ ಕಚ್ಚ ಸಾಮಾನು. ಅದರಿಂದ ಈ ಪ್ರಪಂಚದಲ್ಲಿ ನಾವು ಏನು ನೋಡುತ್ತಿರುವೆವೊ ಅಂತಹ ಸ್ಥೂಲ ಸೂಕ್ಷ್ಮ ವಸ್ತುಗಳೆಲ್ಲ ಆಗಿವೆ. ಇಲ್ಲಿ ಪಂಚಭೂತಗಳು ಸ್ಥೂಲವಾಗಿರುವುವು. ಇವುಗಳಿಂದಲೇ ಎಲ್ಲಾ ಆಕಾರಗಳೂ ಆಗುವುವು. ಹೇಗೆ ಜೇಡಿಮಣ್ಣಿನಿಂದ ಕುಂಬಾರ ಹಲವು ಮಡಿಕೆ ಕುಡಿಕೆಗಳನ್ನು ಮಾಡುತ್ತಾನೆಯೋ ಹಾಗೆ. ಈ ಸ್ಥೂಲದ ಹಿಂದೆ ಇರುವುದೇ ಸೂಕ್ಷ್ಮವಾದ ವ್ಯಕ್ತಿತ್ವ. ಇದೂ ಕೂಡ ಪ್ರಕೃತಿಗೆ ಸಂಬಂಧಪಟ್ಟಿದ್ದೆ. ಆದರೆ ಬಹು ಸೂಕ್ಷ್ಮ. ಇದೇ ಮನಸ್ಸು. ಇದು ಹೊರಗಿನ ವಸ್ತುವಿನ ವೇದನೆಯನ್ನು ಅರಿಯುವುದು. ಇದು ಮನುಷ್ಯ, ಇದು ಮರ, ಬೆಕ್ಕು ಬೆಟ್ಟ ನದಿ ಎಂದು ಇದರ ಸಹಾಯದಿಂದಲೇ ಹೇಳುವುದು. ಬುದ್ಧಿ ಎಂಬುದು ಬಹಳ ಸೂಕ್ಷ್ಮವಾದ ನಿರ್ಣಯಕ್ಕೆ ಬರುವ ಶಕ್ತಿ. ಹಲವು ಘಟನೆಗಳನ್ನು ತೆಗೆದುಕೊಂಡು ಅದರ ಅಂತರಾಳದಲ್ಲಿರುವ ನಿಯಮವನ್ನು ಕಂಡುಹಿಡಿಯುವುದು. ಅದಕ್ಕೆ ಸಿಕ್ಕುವುದು ಒಂದು ಚೂರು. ಅದರಿಂದ ಅದು ಒಟ್ಟು ಹೇಗಿದೆ ಎಂದು ನಿರ್ಧರಿಸುವುದು. ಅದಕ್ಕೆ ಸಿಕ್ಕಿದ ಚೂರು ಅನುಭವದಿಂದ ಯಾವುದೋ ಕಾಲದಲ್ಲಿ, ಯಾವುದೋ ದೇಶದಲ್ಲಿ ಆದ ಘಟನೆಗಳನ್ನು ನಿರ್ಧರಿಸಬಲ್ಲುದು. ತಿಳಿದುದರ ಸಹಾಯದಿಂದ ತಿಳಿಯದ ಕಡೆ ಹಾರುವುದು ಬುದ್ಧಿ. ಸೃಷ್ಟಿರಾಶಿಯಲ್ಲಿ ಮನುಷ್ಯನಲ್ಲಿ, ಬುದ್ಧಿಶಕ್ತಿ ತನ್ನ ಪರಾಕಾಷ್ಠೆಯನ್ನು ಮುಟ್ಟಿದೆ. ಸೂರ್ಯನ ಸುತ್ತಲೂ ಸುತ್ತುತ್ತಿರುವ ಈ ಕ್ಷುದ್ರಗ್ರಹದಲ್ಲಿ, ಐವತ್ತು ಅರವತ್ತು ವರುಷ ಬಾಳಿ ನಾಶವಾಗುವ ಮನುಷ್ಯನಲ್ಲಿ, ಎಂತಹ ಬುದ್ಧಿಶಕ್ತಿ ಹುದುಗಿದೆ! ಇಲ್ಲಿಂದ ಅವನ ಬುದ್ಧಿ ಕೋಟ್ಯಂತರ ಮೈಲಿಗಳ ದೂರದಲ್ಲಿ ಏನಿದೆ ಎಂಬುದನ್ನು ಕಂಡುಹಿಡಿಯಬಲ್ಲುದು. ಲಕ್ಷಾಂತರ ವರುಷಗಳು ಆದಮೇಲೆ ಹೇಗಿರಬಲ್ಲದು ಎಂಬುದನ್ನು ಕಲ್ಪಿಸಿಕೊಳ್ಳಬಹುದು. ಭೂಗರ್ಭದಲ್ಲಿ ಏನಿದೆ, ಆಕಾಶದಲ್ಲಿ ಏನಿದೆ ಇವುಗಳನ್ನು ಹುಡುಕಿ ತರಬಲ್ಲುದು ಬುದ್ಧಿ ಎಂಬ ಬೇಟೆಯ ನಾಯಿ. ಅನಂತರವೇ ಇದನ್ನು ಉಪಯೋಗಿಸುವ ಅಹಂಕಾರ ಇರುವುದು. ನಾನು ಇಂತಹ ವ್ಯಕ್ತಿ, ನನಗೆ ಇಂತಹ ಜ್ಞಾನವಿದೆ, ನಾನು ಇಂತಹ ಐಶ್ವರ್ಯವಂತ, ಮಹಿಮಾವಂತ ಎಂದು ಅಜ್ಞಾನದಿಂದ ಹೆಮ್ಮೆ ಕೊಚ್ಚಿಕೊಳ್ಳುವುದು. ಇವುಗಳೆಲ್ಲ ಭಿನ್ನ ಭಿನ್ನವಾಗಿದೆ. ಒಂದರಿಂದ ಮತ್ತೊಂದನ್ನು ಬೇರ್ಪಡಿಸಬಹುದು.

\begin{shloka}
ಅಪರೇಯಮಿತಸ್ತ್ವನ್ಯಾಂ ಪ್ರಕೃತಿಂ ವಿದ್ಧಿ ಮೇ ಪರಾಮ್~।\\ಜೀವಭೂತಾಂ ಮಹಾಬಾಹೋ ಯಯೇದಂ ಧಾರ್ಯತೇ ಜಗತ್ \hfill॥ ೫~॥
\end{shloka}

\begin{artha}
ಅರ್ಜುನ ಇದು ಅಪರಾಪ್ರಕೃತಿ. ಇದಕ್ಕಿಂತ ಭಿನ್ನವೂ ಜೀವರೂಪವೂ ಆದ ನನ್ನ ಪರಾಪ್ರಕೃತಿಯನ್ನು ತಿಳಿದುಕೊ. ಅದರಿಂದ ಈ ಜಗತ್ತು ಧರಿಸಲ್ಪಟ್ಟಿರುವುದು.
\end{artha}

ಹೋದ ಶ್ಲೋಕದಲ್ಲಿ ವಿವರಿಸುವುದು ಅಪರ ಪ್ರಕೃತಿ. ಅವನ ಸ್ಥೂಲಶಕ್ತಿ ಮತ್ತು ದ್ರವ್ಯಗಳನ್ನು ನೋಡುತ್ತೇವೆ. ಇದನ್ನೆ ಇಂಗ್ಲಿಷ್​ನಲ್ಲಿ \enginline{Matter} ಮತ್ತು \enginline{Energy} ಎಂದು ಹೇಳುತ್ತಾರೆ. ಈ ಶ್ಲೋಕದಲ್ಲಿ ಅವನ ಪರಾಪ್ರಕೃತಿಯನ್ನು ಹೇಳುತ್ತಾನೆ. ಇದು ಚಿತ್​ರೂಪವಾಗಿದೆ. ಇದು ಅಪರಾ ಪ್ರಕೃತಿಗಿಂತ ಬೇರೆಯಾಗಿದೆ. ಚೈತನ್ಯಾತ್ಮಕವಾಗಿರುವುದರಿಂದ ಇದು ಶಕ್ತಿ ಮತ್ತು ದ್ರವ್ಯಗಳಿಂದ ಬೇರೆಯಾಗಿದೆ. ಈ ಚೈತನ್ಯಾಂಶದಿಂದ ಪ್ರಪಂಚ ಧರಿಸಲ್ಪಟ್ಟಿದೆ. ಇಡೀ ಪ್ರಪಂಚವೇ ಅವನ ಕೋಟು ಇದ್ದಂತೆ. ಇದು ತನಗೆ ತಾನೇ ಬೇರೆ ಇಲ್ಲ. ಇದನ್ನು ಭಗವಂತ ಧರಿಸಿರುವನು. ಎಂದರೆ ಅದರ ಹಿಂದೆ ಅವನು ಸ್ಪಂದಿಸುತ್ತಿರುವನು. ಪರಾ ಅಪರಾ ಪ್ರಕೃತಿಗಳೆರಡನ್ನೂ ದೇವರು\break ಧರಿಸಿರುವುದರಿಂದ ಇದು ಅವನಿಗಾಗಿ ಇದೆ. ಇದರ ಹಿಂದೆಲ್ಲ ಅವನೇ ವ್ಯಾಪಿಸಿರುವನು. ಈ ಪ್ರಪಂಚದಲ್ಲಿ ತುಂಬಾ ಕೆಳ ಮಟ್ಟದಲ್ಲಿ ದ್ರವ್ಯವನ್ನು ನೋಡುತ್ತೇವೆ. ಇದು ಘನರೂಪದಲ್ಲಿ ದ್ರವರೂಪದಲ್ಲಿ ಅನಿಲರೂಪದಲ್ಲಿದೆ. ಚೈತನ್ಯ ಇಲ್ಲಿ ನಿದ್ರಿಸುತ್ತಿದೆ. ಇದಕ್ಕಿಂತ ಮೇಲಿನ ಮಟ್ಟದ್ದೆ ಗಿಡಮರ ತರುಲತೆಗಳು. ಇಲ್ಲಿ ಚೈತನ್ಯ ಕನಸು ಕಾಣುತ್ತಿದೆ. ಇದಕ್ಕಿಂತ ಮೇಲಿರುವುದೇ ಪಶುಪಕ್ಷಿ ಜೀವಜಂತುಗಳು. ಇಲ್ಲಿ ಚೈತನ್ಯ ಆಗತಾನೆ ಎಚ್ಚೆತ್ತಿದೆ. ಮನುಷ್ಯನಲ್ಲಿ ಚೆನ್ನಾಗಿ ಜಾಗ್ರತವಾಗಿದೆ. ಪ್ರಪಂಚವನ್ನೆಲ್ಲ ವ್ಯಾಪಿಸಿರುವ ಚೈತನ್ಯ ತರತಮದಂತೆ ಮಧ್ಯವರ್ತಿಯ ಮೂಲಕ ಕಾಣಿಸಿಕೊಳ್ಳುತ್ತಿದೆ. ಈ ಪ್ರಪಂಚದಲ್ಲಿ ಚೈತನ್ಯವಿಲ್ಲದ ಸ್ಥಳವೇ ಇಲ್ಲ ಎಂದಂತೆ ಆಯಿತು. ಎಲ್ಲಿ ನಮಗೆ ತೋರುವುದಿಲ್ಲವೊ ಅಲ್ಲಿಯೂ ಕೂಡ ಅದು ಸುಪ್ತಾವಸ್ಥೆಯಲ್ಲಿ ಇದೆ.

\begin{shloka}
ಏತದ್ಯೋನೀನಿ ಭೂತಾನಿ ಸರ್ವಾಣೀತ್ಯುಪಧಾರಯ~।\\ಅಹಂ ಕೃತ್ಸ್ನಸ್ಯ ಜಗತಃ ಪ್ರಭವಃ ಪ್ರಲಯಸ್ತಥಾ \hfill॥ ೬~॥
\end{shloka}

\begin{artha}
ಇವೆರಡೂ ಸಕಲ ಉತ್ಪತ್ತಿಗೆ ಕಾರಣ ಎಂದು ತಿಳಿದುಕೊ. ಸಮಗ್ರ ಜಗತ್ತಿನ ಉತ್ಪತ್ತಿ ಲಯಗಳಿಗೆ ಕಾರಣ ನಾನೇ.
\end{artha}

ಈ ಪ್ರಪಂಚದಲ್ಲಿ ನಾವು ಎನನ್ನು ನೋಡುತ್ತೇವೆಯೊ ಅವುಗಳೆಲ್ಲ ಅದು ದ್ರವ್ಯರೂಪದಲ್ಲಿರಲಿ, ಸಸ್ಯರೂಪದಲ್ಲಿರಲಿ, ಪ್ರಾಣಿರೂಪದಲ್ಲಿರಲಿ ಅವುಗಳೆಲ್ಲ ಬಂದಿರುವುದು, ಭಗವಂತನ ಪರಾ ಅಪರಾ ಪ್ರಕೃತಿಯ ಸಂಯೋಗದಿಂದ. ಪಂಚಭೂತಗಳು, ಮನಸ್ಸು ಬುದ್ಧಿ ಅಹಂಕಾರ ಮತ್ತು ಪ್ರಪಂಚ\-ವನ್ನೆಲ್ಲ ವ್ಯಾಪಿಸಿರುವ ಚೈತನ್ಯಾಂಶ ಇವೇ ಭಗವಂತನ ಆಲಯ ಎಂದಂತೆ ಆಯಿತು. ಇಲ್ಲೆಲ್ಲ ಅವನೇ ಇರುವನು. ಅದೊಂದು ಸಣ್ಣಕಣವಾಗಬಹುದು, ಅಥವಾ ಈ ಪೃಥ್ವಿಯಂತಹ ದೊಡ್ಡ\break ಗ್ರಹವಾಗಬಹುದು, ಸಣ್ಣ ಕೀಟವಾಗಬಹುದು, ವಿಚಾರ ಪ್ರವರ್ಧಮಾನಕ್ಕೆ ಬಂದ ಮನುಷ್ಯನಾಗಿರಬಹುದು, ಇದರ ಹಿಂದೆಲ್ಲ ಪರಮಾತ್ಮನೇ ವ್ಯಾಪಿಸಿರುವನು.

ಹಿಂದೆ ವ್ಯಾಪಿಸಿರುವ ಪರಮಾತ್ಮನೇ ಈ ಪ್ರಪಂಚವನ್ನು ಸೃಷ್ಟಿಸುತ್ತಾನೆ, ಪಾಲಿಸುತ್ತಾನೆ, ಪುನಃ ಸಂಹಾರ ಮಾಡುತ್ತಾನೆ. ಈ ಮೂರು ಕ್ರಿಯೆಗಳ ಹಿಂದೆಯೂ ಇರುವವನು ಅವನೊಬ್ಬನೆ. ಅವನು ಸೃಷ್ಟಿಯ ಹಿಂದೆ ಎಷ್ಟು ಇರುವನೊ, ಪಾಲನದ ಹಿಂದೆ ಅಷ್ಟೇ ಇರುವನು, ಸಂಹಾರದ ಹಿಂದೆಯೂ ಅಷ್ಟೇ ಇರುವನು. ಈ ಮೂರು ಕ್ರಿಯೆಗಳೂ ಪಕ್ಷಪಾತವಿಲ್ಲದೆ ಆಗುತ್ತಿರುವುದು ಅವನಿಂದಲೇ.

\begin{shloka}
ಮತ್ತಃ ಪರತರಂ ನಾನ್ಯತ್ ಕಿಂಚಿದಸ್ತಿ ಧನಂಜಯ~।\\ಮಯಿ ಸರ್ವಮಿದಂ ಪ್ರೋತಂ ಸೂತ್ರೇ ಮಣಿಗಣಾ ಇವ \hfill॥ ೭~॥
\end{shloka}

\begin{artha}
ಅರ್ಜುನ ನನಗಿಂತ ಶ್ರೇಷ್ಠವಾಗಿರುವುದು ಯಾವುದೂ ಇಲ್ಲ. ದಾರದಲ್ಲಿ ಹೇಗೆ ಮಣಿಗಳು ಪೋಣಿಸಲ್ಪಟ್ಟಿರುತ್ತವೆಯೋ ಹಾಗೆ ಇವೆಲ್ಲವೂ ನನ್ನಲ್ಲಿ ಪೋಣಿಸಲ್ಪಟ್ಟಿವೆ.
\end{artha}

ಶ‍್ರೀಕೃಷ್ಣ, ತನಗಿಂತ ಶ್ರೇಷ್ಠವಾಗಿರುವುದು ಯಾವುದೂ ಇಲ್ಲ ಎನ್ನುತ್ತಾನೆ. ಈ ಪ್ರಪಂಚದಲ್ಲಿ ಕಾಣುವ ಯಾವುದನ್ನು ವಿಭಜನೆ ಮಾಡಿದರೂ ಕೊನೆಗೆ ನಾವು ದೇವರನ್ನು ಎಡವುತ್ತೇವೆ. ನಮ್ಮ ಕಣ್ಣೆದುರಿಗೆ ಕಾಣುವ ವಸ್ತುಗಳೆಲ್ಲ ಪಂಚಭೂತಗಳಿಂದಾಗಿವೆ. ಈ ಪಂಚಭೂತಗಳಿಗೆ ಮೂಲ ಅದರ ತನ್ಮಾತ್ರ, ಅಂದರೆ ಸೂಕ್ಷ್ಮಪಂಚಭೂತಗಳು. ಅದಕ್ಕೂ ಹಿಂದೆ ಹೋದರೆ, ದ್ರವ್ಯಗಳು ಸೂಕ್ಷ್ಮ ಸೂಕ್ಷ್ಮವಾಗುತ್ತ ಕೊನೆಗೆ ಮೂಲವಾದ ಭಗವಂತನಿಗೆ ಒಯ್ಯುವುದು. ಅದರಂತೆಯೆ ಯಾವ ಶಕ್ತಿಯನ್ನು ತೆಗೆದುಕೊಂಡರೂ, ಅದು ಬೆಳಕಾಗಿರಬಹುದು, ವಿದ್ಯುತ್ ಆಗಿರಬಹುದು, ಮ್ಯಾಗ್ನೆಟಿಕ್ ಶಕ್ತಿ ಆಗಿರಬಹುದು, ಒಂದನ್ನು ಮತ್ತೊಂದಕ್ಕೆ ಪರಿವರ್ತನಗೊಳಿಸಲು ಸಾಧ್ಯವಾಗಬಹುದು, ಆ ಶಕ್ತಿ ಮತ್ತೆ ಸೂಕ್ಷ್ಮವಾಗಿ ಮೂಲವಾದ ಪರಮಾತ್ಮನ ಸಮೀಪಕ್ಕೆ ಒಯ್ಯುವುದು. ಅದರಂತೆಯೇ ಜೀವರಾಶಿಗಳಲ್ಲಿರುವ ಚೈತನ್ಯಾಂಶ. ಈ ಚೈತನ್ಯಗಳನ್ನು ವಿಭಜನೆ ಮಾಡುತ್ತಾ ಹೋದರೆ ಅದರ ಮೂಲವಾದ ಪರಮಾತ್ಮನ ಬಳಿಗೆ ಹೋಗುವೆವು. ಮನೆಯೊಳಗೆ ಉರಿಯುವ ವಿದ್ಯುತ್ ದೀಪಕ್ಕೆ ಶಕ್ತಿ ಸರಬರಾಜಾಗುವುದು ಆ ನಗರದ ಕೇಂದ್ರ ವಿದ್ಯುತ್ ಆಫೀಸಿನಿಂದ. ಅಲ್ಲಿಗೆ ಬರುವುದು, ಎಲ್ಲಿಂದ ವಿದ್ಯುತ್​ಶಕ್ತಿ ತಯಾರಾಗುವುದೊ ಅಲ್ಲಿಂದ. ಪ್ರತಿಯೊಂದು ವಸ್ತುವನ್ನೇ ಆಗಲಿ, ಶಕ್ತಿಯನ್ನೇ ಆಗಲಿ, ಚೈತನ್ಯವನ್ನೇ ಆಗಲಿ, ಕೊನೆಯ ತನಕ ವಿಭಜನೆ ಮಾಡಿಕೊಂಡು ಹೋದರೆ ಅದು ದೇವರ ಕಡೆ ನಮ್ಮನ್ನು ಒಯ್ಯುವುದು. ಯಾವುದರಲ್ಲಿಯೂ ಮೇಲೆಮೇಲೆ ತೇಲುತ್ತಿರುವವನು, ಆಳಕ್ಕೆ ಹೋಗದವನು ನಾಸ್ತಿಕನಾಗಿರಬಹುದು. ಆದರೆ ಯಾವುದಾದರೂ ವಸ್ತುವಾಗಲಿ ಅದರ ಅಂತರಾಳಕ್ಕೆ ಹೋಗಲು ಯತ್ನಿಸಿದರೆ ಅದರ ಹಿಂದೆ ಪರಮಾತ್ಮನ ಶಕ್ತಿ ಹರಿಯುತ್ತಿರುವುದು ವ್ಯಕ್ತವಾಗುವುದು.

ಈ ಪ್ರಪಂಚದಲ್ಲಿರುವುದೆಲ್ಲ ಮಣಿಗಳಂತೆ. ಇವುಗಳ ಅಂತರಾಳದಲ್ಲೆಲ್ಲ ದಾರದಂತೆ ಹರಿಯು\-ತ್ತಿರುವವನು ಅವನೆ. ನಮ್ಮ ಕಣ್ಣೆದುರಿಗೆ ಕಾಣುವ ವೈವಿಧ್ಯತೆಗಳಿಂದ ಕೂಡಿದ ವಸ್ತುವನ್ನೆಲ್ಲ ಒಂದು ದಾರದಿಂದ ಪೋಣಿಸಿರುವನು. ಇದರ ಹಿಂದೆ ಒಂದು ಉದ್ದೇಶವನ್ನು ನೋಡುತ್ತೇವೆ. ಒಂದು ಮತ್ತೊಂದಕ್ಕಾಗಿ ಇರುವುದನ್ನು ನೋಡುತ್ತೇವೆ. ಇದನ್ನೆಲ್ಲ ಹೀಗೆ ಮಾಡಿರುವವನೆ\break ದೇವರು. ಯಾವುದೂ ತನಗೆ ತಾನೇ ಇಲ್ಲ. ಯಾವ ಒಂದು ಸಣ್ಣ ಘಟನೆ ಆದರೂ ಅದಕ್ಕೂ ಮತ್ತೊಂದಕ್ಕೂ ಏನೊ ಒಂದು ಸಂಬಂಧವಿದೆ. ಮಳೆ ತನ್ನ ಪಾಡಿಗೆ ತಾನು ಬೀಳುವುದು. ಅದರಿಂದ ಬೆಳೆ, ಅದರಿಂದ ಪಶು ಪಕ್ಷಿ ಮನುಷ್ಯರಿಗೆ ಆಹಾರ, ಅದರಿಂದ ಶಕ್ತಿ, ಅದರಿಂದ ವೃದ್ಧಿ, ಅದರಿಂದ ಪ್ರೀತಿ, ಅದಕ್ಕಾಗಿ ಕಾದಾಟ, ಇವುಗಳೆಲ್ಲ ಒಂದನ್ನೊಂದು ಅನುಸರಿಸಿ ಬರುವುದನ್ನು ನೋಡುತ್ತೇವೆ. ರಂಗಭೂಮಿಯಲ್ಲಿ ದೃಶ್ಯವಾದಮೇಲೆ ದೃಶ್ಯಗಳು ಬರುತ್ತವೆ. ಒಂದಕ್ಕೂ ಮತ್ತೊಂದಕ್ಕೂ ಒಂದು ಸಂಬಂಧವನ್ನು ನೋಡುತ್ತೇವೆ. ಇದರ ಹಿಂದೆ ದಾರದಂತೆ ಇರುವುದೇ ಭಗವತ್ ಸೂತ್ರ. ಹೊರಗಿನಿಂದ ನೋಡಿದರೆ ನಮಗೆ ಮಣಿಗಳೊಂದೇ ಕಾಣುವುದು. ಆದರೆ ಒಂದು ಮಣಿಗೂ ಮತ್ತೊಂದು ಮಣಿಗೂ ಇರುವ ಸಂಬಂಧವನ್ನು ನೋಡಿದಾಗಲೆ ಹಿಂದೆ ಒಂದು ದಾರ ಇರಬೇಕೆಂದು ಭಾವಿಸಿಕೊಳ್ಳಲು ಸಾಧ್ಯ. ದೇವರು ಎಲ್ಲೋ ದೂರದ ಲೋಕದಲ್ಲಿ ಮುಗಿಲುಗಳಾಚೆ ಇಲ್ಲ. ಪ್ರಪಂಚದ ವಸ್ತುಗಳ ಅಂತರಾಳದಲ್ಲಿಯೂ, ಅದು ಜಡವಾಗಲಿ ಚೇತನವಾಗಲಿ, ಎಲ್ಲಾ ಕಡೆಯೂ ಗುಪ್ತಗಾಮಿನಿಯಂತೆ ಹರಿಯುತ್ತಿರುವನು. ಪ್ರಕೃತಿಯ ಬಾಗಿಲನ್ನು ತಟ್ಟಿದಂತೆ ಅದರ ಹಿಂದೆ ಇರುವ ಪರಮ ಪುರುಷನ ಅಸ್ತಿತ್ವ ನಮಗೆ ವೇದ್ಯವಾಗುವುದು.

\begin{shloka}
ರಸೋಽಹಮಪ್ಸು ಕೌಂತೇಯ ಪ್ರಭಾಸ್ಮಿ ಶಶಿಸೂರ್ಯಯೋಃ~।\\ಪ್ರಣವಃ ಸರ್ವವೇದೇಷು ಶಬ್ದಃ ಖೇ ಪೌರುಷಂ ನೃಷು \hfill॥ ೮~॥
\end{shloka}

\begin{artha}
ಕೌಂತೇಯ, ನಾನು ನೀರಿನಲ್ಲಿ ರಸ, ನಾನು ಸೂರ್ಯ ಚಂದ್ರರಲ್ಲಿರುವ ತೇಜಸ್ಸು, ನಾನು ಸರ್ವವೇದಗಳ ಓಂಕಾರ, ನಾನೇ ಆಕಾಶದಲ್ಲಿ ಶಬ್ದ, ಪುರುಷರಲ್ಲಿ ಪರಾಕ್ರಮ.
\end{artha}

ಭಗವಂತ ಯಾವ ರೀತಿ ಈ ಪ್ರಪಂಚದಲ್ಲಿ ಹಾಸುಹೋಕ್ಕಾಗಿರುವನು ಎಂಬುದನ್ನು ವಿವರಿಸುತ್ತಾನೆ. ಅವನೇ ಪ್ರಕೃತಿಯ ಧರ್ಮಗಳ ಹಿಂದೆಲ್ಲ ಇರುವುದು. ನೀರಿನಲ್ಲಿ ರಸ ಎನ್ನುವನು. ನಾವೆಲ್ಲಕುಡಿಯುವ ನೀರಿನಲ್ಲಿ ರಸದಂತೆ ಇರುವನು. ದ್ರವರೂಪದಲ್ಲಿ ಇರುವನು. ನಮ್ಮ ಬಾಳಿಗೆ ನೀರು ಅತ್ಯವಶ್ಯಕ. ಮೀನಿನಂತೆ ನಾವು ನೀರಿನಲ್ಲಿ ಇಲ್ಲದೇ ಇರಬಹುದು. ಆ ನೀರನ್ನೇ ನಾವು ಹೋತ್ತು ತಿರುಗುತ್ತಿರುವೆವು. ನಮ್ಮ ದೇಹವೆಲ್ಲ ಶೇಕಡ ತೊಂಬತ್ತರಷ್ಟು ನೀರಿನಿಂದಲೇ ಆಗಿದೆ. ಪಂಚಭೂತಗಳಲ್ಲಿ ನೀರಿನ ಅಂಶವೇ ಜಾಸ್ತಿ ನಮ್ಮ ದೇಹದಲ್ಲಿ.

ಸೂರ್ಯ ಮತ್ತು ಚಂದ್ರರಲ್ಲಿರುವ ಬೆಳಗುವ ಶಕ್ತಿಯ ಹಿಂದೆ ಇರುವುದೇ ಅವನ ಶಕ್ತಿ. ನಮ್ಮ ರಂಗಭೂಮಿಗೆ ಇರುವ ದೀಪದಂತಿದ್ದಾರೆ ಸೂರ್ಯ ಚಂದ್ರರು. ಅವರು ಬೆಳಗಿದರೆ ನಮಗೆ ಕಾಂತಿ, ಶಾಖ, ಮಳೆ ಬೆಳೆ. ನಾವು ಉಪಯೋಗಿಸುವ ಶಕ್ತಿಯ ಹಿಂದೆಲ್ಲ ಇರುವುದೇ ಸಂಗ್ರಹಿಸಿಟ್ಟ ಸೂರ್ಯನ ಶಕ್ತಿ. ನಾವು ಉಪಯೋಗಿಸುವ ವಿದ್ಯುತ್ ಶಕ್ತಿಯ ಹಿಂದೆ ಸೂರ್ಯಶಕ್ತಿ ಇದೆ. ಅದು ಮೇಲಿನಿಂದ ಕೆಳಗೆ ಬೀಳುವ ನೀರಿನಿಂದ ತಯಾರಾದುದು. ಆ ನೀರನ್ನು ಮೇಲಕ್ಕೆ ತಳ್ಳಿದ್ದು ಸೂರ್ಯನ ಕಾಂತಿ. ಸೂರ್ಯ ಸಾಗರದ ಮೇಲೆ ತಪಿಸಿದಾಗಲೆ ಅಲ್ಲಿರುವ ನೀರು ಆವಿಯಂತೆ ಮೇಲೇರುವುದು. ಸೌದೆ ಅಥವಾ ಕಲ್ಲಿದ್ದಲು ಉರಿಯುವುದರಿಂದ ತಯಾರಾದ ವಿದ್ಯುತ್​ಶಕ್ತಿಯಾದರೆ, ಆ ಕಲ್ಲಿದ್ದಲು ಮತ್ತು ಸೌದೆಗೆ ಕಾರಣ ಸೂರ್ಯನ ಕಾಂತಿ. ಸೂರ್ಯ ಬೆಳಗಿದರೇ ಸಸ್ಯರಾಶಿ ಬೆಳೆಯಬೇಕಾದರೆ, ಸಸ್ಯರಾಶಿ ಬೆಳೆದರೆ ಊಟ ಬಟ್ಟೆ ಇವುಗಳೆಲ್ಲ ಸಿಕ್ಕಬೇಕಾ\-ದರೆ. ಪ್ರತಿಯೊಬ್ಬರೂ ಒಂದಲ್ಲ ಒಂದು ರೀತಿ ಸೂರ್ಯನಿಂದ ಬಂದ ಶಕ್ತಿಯನ್ನು ಹೀರುತ್ತಿರುವರು. ಪರೋಕ್ಷವಾಗಿ ಭಗವಂತನನ್ನು ನಾವು ಹೀರುತ್ತಿರುವೆವು. ಅವನೇ ಬಾಳಿಗೆ ಆಧಾರ.

ವೇದಗಳಲ್ಲಿ ಓಂಕಾರ ಎನ್ನುತ್ತಾನೆ. ವೇದಗಳ ರಾಶಿಯನ್ನೆಲ್ಲ ಒಂದು ಅಕ್ಷರದಲ್ಲಿ ಸಂಗ್ರಹಿಸ\-ಬಹುದು. ಅದೇ ಓಂಕಾರ. ಅದರಿಂದ ಎಲ್ಲಾ ಅಕ್ಷರಗಳು ಬಂದಿವೆ, ಅಕ್ಷರಗಳಿಂದ ಎಲ್ಲಾ ನಾಮರೂಪಗಳೂ ಬಂದಿವೆ. ಓಂಕಾರವೆಂಬುದು ವ್ಯಕ್ತ ಮತ್ತು ಅವ್ಯಕ್ತಕ್ಕೆ ಸೇತುವೆಯಂತಿದೆ. ಸಗುಣ ಮತ್ತು ನಿರ್ಗುಣ ಬ್ರಹ್ಮಕ್ಕೆ ಅದೊಂದು ಚಿಹ್ನೆಯಾಗಿದೆ. ಇದನ್ನೆ ನಾವು ಶಬ್ದಬ್ರಹ್ಮ ಎನ್ನುತ್ತೇವೆ. ಪರಬ್ರಹ್ಮನಿಗೆ ಅತ್ಯಂತ ಸಮೀಪದಲ್ಲಿರುವುದೇ ಓಂಕಾರ. ಪರಬ್ರಹ್ಮನ ಶಕ್ತಿ ಅಲ್ಲಿ ಸ್ಪಂದಿಸುತ್ತಿದೆ.

ಆಕಾಶ ಎಂಬ ಪಾತ್ರೆಯಲ್ಲಿ ತುಂಬಿರುವುದೇ ಅನಾಹತ ಧ್ವನಿಯಾದ ಶಬ್ದ. ಅದೂ ಕೂಡ ಪರಬ್ರಹ್ಮನಿಗೆ ಒಂದು ಪ್ರತೀಕ. ಅದರ ಹಿಂದೆ ಇರುವವನು ಅವನೇ.

ಪುರುಷರಲ್ಲಿ ಪೌರಷವಾಗಿದ್ದೇನೆ ಎನ್ನುತ್ತಾನೆ. ಎಲ್ಲಿ ಮನುಷ್ಯರು ಸಾಹಸವನ್ನು ವ್ಯಕ್ತಪಡಿಸು ವರೋ, ಉತ್ಪಾದನೆಯನ್ನು ಮಾಡುವರೋ, ಇರುವ ಆತಂಕಗಳೊಂದಿಗೆ ಹೋರಾಡುತ್ತಿರುವರೋ, ಅಸಾಧ್ಯವನ್ನು ಸಾಧಿಸುವರೋ, ಎಲ್ಲಿ ಜೀವಿಗಳೆದೆಯಲ್ಲಿ ಅದಮ್ಯ ಉತ್ಸಾಹ ನೋಡುವೆವೊ ಛಲ ನೋಡುವೆವೊ, ಪ್ರತಿಜ್ಞೆ ನೋಡುವೆವೊ, ಅದರ ಹಿಂದೆಲ್ಲ ಭಗವಂತನ ಶಕ್ತಿಯೇ ವ್ಯಕ್ತವಾಗುತ್ತಿದೆ. ಅವನ ಪೌರುಷವೇ, ಅವನ ಶಕ್ತಿಯೇ ಯೋಗ್ಯ ವ್ಯಕ್ತಿಗಳನ್ನು ಆರಿಸಿಕೊಂಡು ಅವರ ಮೂಲಕ ತನ್ನ ಕೆಲಸವನ್ನು ಮಾಡುತ್ತಿರುವುದು. ಎಂತಹ ಸುಂದರವಾದ ಭಗವಂತನ ಭಾವನೆಯನ್ನು ನಾವಿಲ್ಲಿ ನೋಡುತ್ತೇವೆ! ಈ ಪ್ರಪಂಚ ದೇವರಿಲ್ಲದ ಒಂದು ಹಾಳೂರಲ್ಲ. ಇಲ್ಲಿರುವ ಪ್ರತಿಯೊಂದರ ಹಿಂದೆಯೂ ಅವನು ಸ್ಪಂದಿಸುತ್ತಿರುವನು. ಕಣ್ಣು ಬಿಟ್ಟು ನೋಡಿದೆಡೆಯಲ್ಲೆಲ್ಲ ಅವನೇ ತಾಂಡವ ವಾಡುತ್ತಿರುವವನು. ಆಕಾಶವೆಂಬ ಪಾತ್ರೆಯಲ್ಲಿ ಶಬ್ದದಂತೆ ತುಂಬಿರುವವನು ಅವನೇ. ಗಾಳಿಯಂತೆ ಬೀಸುವವನು ಅವನು. ನೀರಿನಂತೆ ಹರಿಯುವವನು ಅವನು. ವೇದಗಳಲ್ಲಿ ಜ್ಞಾನ ಸ್ವರೂಪನವನು. ಈ ಪೃಥ್ವಿಯಲ್ಲಿ ಎಲ್ಲಾ ಶ್ರಮದ ಹಿಂದೆ ಇರುವ ಪೌರಷವೂ ಅವನೇ. ಆ ಪೌರುಷ ಹಲವು ರೂಪಗಳನ್ನು ತಾಳಿ ಬೆಳಗುತ್ತಿರಬಹುದು. ಭೂಮಿಯಿಂದ ಬೆಳೆಯನ್ನು ತೆಗೆಯುವ ರೈತನ ಶ್ರಮವಾಗಬಹುದು, ದೇಶವನ್ನು ಅಪಾಯದಿಂದ ಅನಾಯಕತೆಯಿಂದ ರಕ್ಷಿಸುವ ಕ್ಷತ್ರಿಯನ ಶ್ರಮವಾಗಬಹುದು, ಬ್ರಾಹ್ಮಣನ ತಪಸ್ಸು ಮತ್ತು ಜ್ಞಾನದ ಹಿಂದೆ ಇರುವ ಶ್ರಮವಾಗಬಹುದು. ಈ ಶ್ರಮ ಎಲ್ಲರ ಕಣ್ ಮನಗಳನ್ನು ಸೆಳೆಯುವಂತಹ ರೀತಿಯಲ್ಲಿ ಅತ್ಯಂತ ಉನ್ನತ ಮಟ್ಟದಲ್ಲಿ ವ್ಯಕ್ತ\-ವಾದಾಗ ಅದರ ಹಿಂದೆ ಭಗವಂತನ ಆವಿರ್ಭಾವನೆಯನ್ನೇ ನೋಡುತ್ತೇವೆ. ನಮಗಲ್ಲಿ ಬರುವ ದೇವರ ಭಾವನೆ ಮಾನವನಂತಿರುವ ಒಂದು ವ್ಯಕ್ತಿಯಲ್ಲ, ಸರ್ವಾಂತರ್ಯಾಮಿಯಾದ ವಿಶ್ವಚೈತನ್ಯ.

\begin{shloka}
ಪುಣ್ಯೋ ಗಂಧಃ ಪೃಥಿವ್ಯಾಂ ಚ ತೇಜಶ್ಚಾಸ್ಮಿ ವಿಭಾವಸೌ~।\\ಜೀವನಂ ಸರ್ವಭೂತೇಷು ತಪಶ್ಚಾಸ್ಮಿ ತಪಸ್ವಿಷು \hfill॥ ೯~॥
\end{shloka}

\begin{artha}
ಪೃಥ್ವಿಯಲ್ಲಿ ಪುಣ್ಯವಾದ ಗಂಧವೂ, ಅಗ್ನಿಯಲ್ಲಿ ತೇಜಸ್ಸು, ಸರ್ವಪ್ರಾಣಿಗಳಲ್ಲಿ ಜೀವನವೂ ತಪಸ್ವಿಗಳ ತಪಸ್ಸೂ ನಾನೇ ಆಗಿರುವೆನು. 
\end{artha}

ಎಲ್ಲಾ ವಿಧವಾದ ವಾಸನೆಗಳೂ ಪೃಥ್ವಿಯಿಂದ ಬರುತ್ತವೆ. ಆಯಾ ವಸ್ತುವಿನಿಂದ ಸಣ್ಣ ಸಣ್ಣ ಧೂಳೀಕಣಗಳು ಸುತ್ತಲೂ ವ್ಯಾಪಿಸಿರುತ್ತದೆ. ಅವು ನಮ್ಮ ಮೂಗಿನಲ್ಲಿರುವ ವಾಸನೆಗಳಿಗೆ ಸಂಬಂಧ ಪಟ್ಟ ನರಗಳಿಗೆ ತಾಕಿದಾಗ ಇದು ಇಂತಹ ವಾಸನೆ ಎಂದು ಹೇಳುತ್ತೇವೆ. ವಸ್ತುಗಳಿಂದ, ಅದರಲ್ಲಿಯೂ ಪೃಥ್ವಿ ಮತ್ತು ಅದರಲ್ಲಿ ಬೆಳೆದ ಹೂವು ಮುಂತಾದವುಗಳಿಂದ ಬರುತ್ತಿರುವ ವಾಸನೆಗಳಲ್ಲಿ ಪುಣ್ಯವಾಸನೆಯ ಹಿಂದೆ ಅವನಿರುವನು. ಹಾಗಾದರೆ ದುರ್ವಾಸನೆಯ ಹಿಂದೆ ಅವನಿ ಲ್ಲವೆ? ಇಲ್ಲದೆ ಇದ್ದರೆ ಅವನ ಸರ್ವವ್ಯಾಪ್ತಿತ್ವಕ್ಕೆ ಧಕ್ಕೆ ಬಂತಲ್ಲ, ಅವನು ಪಕ್ಷಪಾತಿಯಾದನಲ್ಲ ಎಂದು ನಾವು ಭಾವಿಸಬಹುದು. ಅವನು ಸರ್ವವ್ಯಾಪ್ತಿತ್ವದ ದೃಷ್ಟಿಯಿಂದ ಎಲ್ಲಾ ಕಡೆಯಲ್ಲಿಯೂ ಇರುವನು. ಆದರೆ ಎಲ್ಲಾ ವಸ್ತುಗಳೂ ಒಂದೇ ಸಮನಾಗಿ ಅವನನ್ನು ಪ್ರತಿಬಿಂಬಿಸುವುದಿಲ್ಲ. ಪುಣ್ಯವಾಸನೆಯಲ್ಲಿ ಪರಮಾತ್ಮನ ಆವಿರ್ಭಾವ ಜಾಸ್ತಿಯಾಗಿ ನಮಗೆ ಕಾಣುವುದು. ಅದಕ್ಕೆ ಅವನು ಅಲ್ಲಿರುವನು ಎಂದರೆ, ವ್ಯಾವಹಾರಿಕ ದೃಷ್ಟಿಯಿಂದ ಅಲ್ಲಿ ಇತರ ಕಡೆಗಳಿಗಿಂತ ಹೆಚ್ಚಾಗಿ ಪ್ರತಿ ಬಿಂಬಿಸುವನು. ಒಂದು ಫ್ಯಾಕ್ಟರಿ ಚಿಮಣಿಯ ಮೂಲಕ ಬರುವ ಹೊಗೆಯ ಹಿಂದೆಯೂ\break ದೇವರು ಇರುವನು. ಒಂದು ಪವಿತ್ರ ಯಜ್ಞವನ್ನು ಮಾಡುತ್ತಿರುವ ಹೋಮದ ಧೂಮದಲ್ಲಿಯೂ ಅವನೇ ಇರುವನು. ಆದರೆ ಹೋಮದ ಧೂಮದ ಪರಿಮಳದಲ್ಲಿ ಅವನು ಹೆಚ್ಚಾಗಿ ಕಾಣುವನು. ಎಲ್ಲ ಹೂವುಗಳಲ್ಲಿಯೂ ಒಂದು ಬಗೆಯ ವಾಸನೆ ಇದೆ. ಅವುಗಳಲ್ಲಿ ಯಾವುದರಲ್ಲಿ ಪರಿಮಳ ಹೆಚ್ಚಾಗಿರುವುದೊ ಅಂತಹ ಮಲ್ಲಿಗೆ ಗುಲಾಬಿ ಜಾಜಿ ಚಂಪಕ ಮುಂತಾದ ಪುಷ್ಪಗಳಲ್ಲಿ ಅವನು ಹೆಚ್ಚು ಇರುವಂತೆ ಕಾಣುವುದು. ಅದಕ್ಕೇ ಆ ಪುಷ್ಪಗಳು ದೇವರಿಗೆ ಪ್ರಿಯವಾದ ವಾಸನೆ ಎಂದು ನಮ್ಮ ಮನಸ್ಸಿನಲ್ಲಿ ಅವನ ಭಾವನೆ ಮೂಡುವುದು. ಆ ವಾಸನೆಯನ್ನು ಸೇವಿಸಿದಾಗ, ಅವನಿಂದ\break ಬರುವ ವಾಸನೆಯನ್ನು ನಾವು ಸೇವಿಸುತ್ತಿರುವೆವು ಎಂಬ ಭಾವನೆ ಬರುವುದು. ಇವುಗಳ\break ಸಹಾಯದಿಂದ ಮನಸ್ಸು ದೇವರ ಕಡೆ ತಿರುಗಲು ಸಹಾಯವಾಗುವುದು.

ಬೆಂಕಿಯ ಉರಿಯುವ ಶಕ್ತಿಯ ಹಿಂದೆ ನಾನೆ ಇರುವೆನು ಎನ್ನುವನು. ಮನುಷ್ಯ ಅಗ್ನಿಯ ಉಪಯೋಗವನ್ನು ಯಾವಾಗ ಕಂಡುಹಿಡಿದನೊ ಗೊತ್ತಿಲ್ಲ. ಅಂದಿನಿಂದ ಅಗ್ನಿಯಷ್ಟು ನಮ್ಮ ಜೀವನದಲ್ಲಿ ಬಳಕೆಯಲ್ಲಿರುವುದು ಮತ್ತಾವುದೂ ಇಲ್ಲ. ಆದಕಾರಣವೆ ಆ ಅಗ್ನಿಗೆ ಒಂದು ದೇವರ ಪೀಠವನ್ನು ಕೊಟ್ಟಿರುವನು ಹಿಂದು. ಅವನನ್ನು ಅಗ್ನಿದೇವ ಎಂದು ಕರೆಯುತ್ತೇವೆ. ಅದಿಲ್ಲದೆ ಇದ್ದರೆ ನಮ್ಮ ಆಹಾರವನ್ನು ಬೇಯಿಸಿ ತಿನ್ನುವುದಕ್ಕೆ ಆಗುವುದಿಲ್ಲ. ಬರೀ ಕಟ್ಟಿಗೆಯ ಬೆಂಕಿ ಮಾತ್ರವಲ್ಲ. ಈಗ ನಾವು ಫ್ಯಾಕ್ಟರಿಗಳಲ್ಲಿ ವಿದ್ಯುತ್​ಶಕ್ತಿ ಉಪಯೋಗಿಸಿ ಅಗ್ನಿ ತಯಾರುಮಾಡುತ್ತೇವೆ. ಎಣ್ಣೆ ಮತ್ತು ಗ್ಯಾಸುಗಳನ್ನು ಉರಿಸಿ ಅಗ್ನಿ ತಯಾರುಮಾಡುತ್ತೇವೆ. ನಮ್ಮ ಫ್ಯಾಕ್ಟರಿ ನಡೆಯುವುದೇ ಪ್ರತ್ಯಕ್ಷವಾದ ಅಥವಾ ಪರೋಕ್ಷವಾದ ಅಗ್ನಿದೇವನ ದಯೆಯಿಂದ. ನಮ್ಮ ಮನೆಯಲ್ಲಿರುವ ಅಡಿಗೆಕೋಣೆಗೆ ಜೀವ ಬರುವುದೇ ಅಗ್ನಿದೇವನಿಂದ. ಅವನಿಂದ ನಮಗೆ ರಾತ್ರಿಯಲ್ಲಿ ಕತ್ತಲ ಪರಿಹಾರ. ಅಗ್ನಿ ಎಲ್ಲವನ್ನೂ ದಹಿಸುವುದು. ಅದರ ಮೂಲರೂಪಕ್ಕೆ ತೆಗೆದುಕೊಂಡು ಹೋಗುವುದು. ಎಂತಹ ಸುಂದರವಾದ ನಾಮರೂಪಗಳಾದರೂ ಅಗ್ನಿ ಅದನ್ನೆಲ್ಲ ಕಬಳಿಸಿ ಒಂದು ಬೊಗಸೆ ಬೂದಿಯನ್ನು ಬಿಡುವನು. ಅಗ್ನಿ ನಮ್ಮ ಸಂಸ್ಕೃತಿಯಲ್ಲಿ ಬಹಳ ಪವಿತ್ರವಾದುದು. ಗೃಹಸ್ಥರ ಮನೆಯಲ್ಲಿ ಹಿಂದಿನ ಕಾಲದಲ್ಲಿ ಅದನ್ನು ಹಗಲುರಾತ್ರಿ ರಕ್ಷಿಸುತ್ತಿದ್ದರು. ಮನುಷ್ಯ ಈ ದೇಹವನ್ನು ತ್ಯಜಿಸಿದಾಗ ಈ ದೇಹವನ್ನು ಬೊಗಸೆ ಬೂದಿ ಮಾಡುವುದೂ ಇದೇ ಅಗ್ನಿ. ಬದುಕಿರುವ ತನಕ ನಿತ್ಯಗೆಳೆಯನಾಗಿ, ಹೋಗುವಾಗಲೂ ದೇಹತ್ಯಾಗಕ್ಕೆ ಸಹಾಯ ಮಾಡುವವನೇ ಈ ಅಗ್ನಿದೇವ. 

ಭಗವಂತನೇ ಸರ್ವಪ್ರಾಣಿಗಳಲ್ಲಿ ಜೀವನವೂ ಆಗಿದ್ದಾನೆ. ಅವನೇ ಚೇತನಾತ್ಮನಂತೆ ಇರುವನು. ಒಂದು ದೇಹ ಬದುಕಿರುವುದು, ಅದರ ಹಿಂದೆ ಜೀವ ಇರುವುದರಿಂದ. ಆ ಜೀವ ಇದ್ದರೆ ಪಶುಪಕ್ಷಿ ಮನುಷ್ಯರೆಲ್ಲರೂ ತಮ್ಮ ತಮ್ಮ ಕೆಲಸಗಳನ್ನು ಮಾಡುವರು. ಅದಿರುವುದರಿಂದ ಸುತ್ತಲಿರುವ ಆತಂಕಗಳೊಡನೆ ಚೇತನ ಹೋರಾಡುವುದು. ಅದು ಹೋದೊಡನೆ ದೇಹ ಕೊಳೆಯಲು\break ಮೊದಲಾಗುವುದು. ಅನೇಕವೇಳೆ ಅದು ಪ್ರಕೃತಿಯನ್ನು ವಿರೋಧಿಸುವುದು, ಅದರೊಡನೆ ಸರಸವಾಡುವುದು, ಅದಕ್ಕೆ ಚಕ್ಕಳಗುಳಿಯನ್ನು ಕೊಡುವುದು. ಇಷ್ಟು ದೊಡ್ಡ ಪ್ರಕೃತಿ ಚೇತನಕ್ಕೆ ದಾಸ. ಆ ಚೇತನದ ಹಿಂದೆ ಬೆಳಗುವ ಶಕ್ತಿಯೇ ಪರಮಾತ್ಮ.

ತಪಸ್ವಿಗಳಲ್ಲಿರುವ ತಪೋಶಕ್ತಿಯೂ ನಾನೇ ಎನ್ನುವನು. ತಪಸ್ಸು ಎಂದರೆ ಮನಸ್ಸನ್ನು ಏಕಾಗ್ರ ಮಾಡಿ ಅದನ್ನು ಒಂದು ವಸ್ತುವಿನ ಕಡೆ ಹರಿಸುವುದು, ಅದರ ರಹಸ್ಯವನ್ನು ತಿಳಿಯಲು ಯತ್ನಿಸುವುದು. ಪ್ರಪಂಚದ ರಹಸ್ಯವನ್ನು ಭೇದಿಸಲು ಮನುಷ್ಯನ ಕೈಯಲ್ಲಿರುವ ಅತ್ಯಮೋಘವಾದ ಅಸ್ತ್ರವೇ ತಪಸ್ಸು. ನಮಗೆ ಲೌಕಿಕ ಜ್ಞಾನವಾಗಲೀ ಪಾರಮಾರ್ಥಿಕ ಜ್ಞಾನವಾಗಲೀ ಬೇಕಾದರೆ ತಪಸ್ಸು ಮಾಡಬೇಕು. ಜ್ಞಾನ ಬಿಟ್ಟಿ ನಮಗೆ ಸಿಕ್ಕುವುದಿಲ್ಲ. ಅದಕ್ಕೆ ತಪಸ್ಸು ಮಾಡಿದರೇ ಅದು ನಮಗೆ ಸಿಕ್ಕಬೇಕಾದರೆ. ತಪಸ್ಸು ಎಂದರೆ ತನ್ಮಯನಾಗಿ ಒಂದು ವಸ್ತುವನ್ನು ಕುರಿತು ಚಿಂತಿಸುವುದು. ಒಬ್ಬ ಅದಕ್ಕೆ ಖಯಾಲಿಯಾಗಿ ಹೋಗುವ ತನಕ ಅವನು ಅದನ್ನು ಕುರಿತು ಚಿಂತಿಸುತ್ತಿರುವನು. ಆಹಾರ ನಿದ್ರೆಗಳನ್ನು ಅದಕ್ಕಾಗಿ ಬಿಡುವನು. ಆರ್ಕಿಮಿಡೀಸ್ ಎಂಬ ಗ್ರೀಕ್ ವಿಜ್ಞಾನಿಗೆ ಒಂದು ಕಿರೀಟ ಕೊಟ್ಟು ಇದರಲ್ಲಿ ಚಿನ್ನವಲ್ಲದ ಲೋಹ ಏನಾದರೂ ಬೆರೆತಿದೆಯೆ, ಅದನ್ನು ಕಿರೀಟವನ್ನು ಕರಗಿಸದೆ ಹೇಳಬೇಕು ಎಂದು ಆಜ್ಞಾಪಿಸಿದಾಗ, ಅದನ್ನು ಹೇಗೆ ತಿಳಿಯುವುದಕ್ಕೆ ಸಾಧ್ಯ ಎಂಬುದರಲ್ಲೇ ತನ್ಮಯನಾಗುತ್ತಾನೆ. ಒಮ್ಮೆ ಬೆತ್ತಲೆಯಾಗಿ ನೀರಿಗೆ ಸ್ನಾನಕ್ಕೆ ಇಳಿದಾಗ, ಆ ನೀರು ಇವನ ಗಾತ್ರಕ್ಕೆ ಮೇಲೆ ಬಂದಾಗ, ಆತ ತಾನು ದಿಗಂಬರ ಎಂಬುದನ್ನು ಕೂಡ ಗಮನಿಸದೆ ನಾನದನ್ನು ಕಂಡುಹಿಡಿದೆ ಎಂದು ಜನಸಂದಣಿಯಲ್ಲಿ ಕಿರುಚಿಕೊಳ್ಳುತ್ತಾ ಹೋಗುವನು. ಇವನೇ \enginline{Principle of Density}ಯ ನಿಯಮವನ್ನು ಕಂಡುಹಿಡಿದವನು. ಜೆ. ಸಿ. ಬೋಸ್ ಸಸ್ಯಜೀವನದ ಮೇಲೆ ಪ್ರಯೋಗ ಮಾಡುತ್ತಿದ್ದಾಗ ಹಲವು ದಿನಗಳು ತನ್ನ ಪ್ರಯೋಗಶಾಲೆಯನ್ನು ಬಿಟ್ಟುಹೋಗುತ್ತಿರಲಿಲ್ಲವಂತೆ. ಊಟ ಮತ್ತು ನಿದ್ರೆಯನ್ನು ಕೂಡ ಬಿಟ್ಟು ತನ್ನ ಪ್ರಯೋಗದಲ್ಲಿ ತಲ್ಲೀನನಾಗಿರುತ್ತಿದ್ದ. ಅನಂತರವೆ ಸಸ್ಯಜೀವನದ ಹಿಂದೆ ಇರುವ ಚೈತನ್ಯದ ಪ್ರತಿಕ್ರಿಯೆಯನ್ನು ಕಂಡುಹಿಡಿದನು. ಇತ್ತೀಚೆಗೆ ಬಂದ ಅಮೆರಿಕಾ ದೇಶದ ಎಡಿಸನ್ ಎಂಬ ವಿಜ್ಞಾನಿ ವಿದ್ಯುತ್ ಬಲ್ಬುಗಳನ್ನು ಕಂಡುಹಿಡಿಯುವಾಗ ಇಪ್ಪತ್ತುನಾಲ್ಕು ಗಂಟೆಯಲ್ಲಿ ಇಪ್ಪತ್ತು ಮೂರು ಗಂಟೆಗಳು ಕೆಲಸ ಮಾಡುತ್ತಿದ್ದ ತನ್ನ ಪ್ರಯೋಗಶಾಲೆಯಲ್ಲಿ. ಇದೆಲ್ಲ ಪ್ರಕೃತಿಯ ರಹಸ್ಯವನ್ನು ಅರಿಯುವುದಕ್ಕೆ ಮಾಡಿದ ತಪಸ್ಸು. ವಿಜ್ಞಾನಿಗಳು ಮಾಡಿದ ತಪಸ್ಸಿನಿಂದಲೆ ಮನುಷ್ಯ ತನ್ನ ಸುತ್ತಲಿರುವ ಪ್ರಕೃತಿಯ ಮೇಲೆ ಅಂತಹ ಸ್ವಾಮಿತ್ವವನ್ನು ಪಡೆದಿರುವುದು. ಅದರಲ್ಲಿರುವ ರಹಸ್ಯವಾದ ಕೆಲವು ನಿಯಮಗಳಲ್ಲಿ ಕೆಲವನ್ನು ತಿಳಿದುಕೊಂಡಿರುವುದು. ಯಾರಲ್ಲಿ ಆ ತಪಸ್ಸು ಅಂತರ್ಮುಖವಾಗುವುದೊ ಅವನೇ ಯೋಗಿ. ಅವನು ಈ ಪ್ರಪಂಚದಲ್ಲಿ ತನ್ನ ಹಿಂದೆ, ಇಡೀ ಬ್ರಹ್ಮಾಂಡವನ್ನೆಲ್ಲ ವ್ಯಾಪಿಸಿರುವ ಭಗವತ್ ತತ್ತ್ವ ತಿಳಿಯಲು ಯತ್ನಿಸುವನು. ಈ ತಪಸ್ಸನ್ನು ಮಾಡುವಾಗ ಅವನು ಶರೀರವನ್ನು ಮರೆಯುವನು. ನಿದ್ರಾಹಾರಗಳನ್ನು ಮರೆಯುವನು. ತದೇಕ ಧ್ಯಾನದಲ್ಲಿ ತಲ್ಲೀನನಾಗುವನು. ಆಗಲೇ ಪರಮಾತ್ಮ ಅವನಿಗೆ ಗೋಚರಿಸಬೇಕಾದರೆ. ಪಂಚಭೂತಗಳಲ್ಲಿ ಹುದುಗಿರುವ ನಿಯಮಗಳನ್ನು ಕಂಡುಹಿಡಿಯುವುದಕ್ಕೆ ಮಾಡಿದ ತಪಸ್ಸಾಗಲಿ ಪರಮಾತ್ಮನನ್ನು ಕಂಡುಹಿಡಿಯುವುದಕ್ಕೆ ಮಾಡಿದ ತಪಸ್ಸಾಗಲಿ, ಅದರ ಹಿಂದೆ ಕೆಲಸ ಮಾಡುತ್ತಿರುವುದೇ ಭಗವಂತನ ಶಕ್ತಿ.

\begin{shloka}
ಬೀಜಂ ಮಾಂ ಸರ್ವಭೂತಾನಾಂ ವಿದ್ಧಿ ಪಾರ್ಥ ಸನಾತನಮ್~।\\ಬುದ್ಧಿರ್ಬುದ್ಧಿಮತಾಮಸ್ಮಿ ತೇಜಸ್ತೇಜಸ್ವಿನಾಮಹಮ್ \hfill॥ ೧೦~॥
\end{shloka}

\begin{artha}
ಪಾರ್ಥ, ಸಕಲ ಜೀವಿಗಳ ಸನಾತನ ಕಾರಣ ನಾನು ಎಂದು ತಿಳಿ. ಬುದ್ಧಿವಂತರ ಬುದ್ಧಿ, ತೇಜಸ್ವಿಗಳ ತೇಜಸ್ ನಾನು ಎಂದು ತಿಳಿ.
\end{artha}

ಹಿಂದೂಗಳಾದ ನಮ್ಮಲ್ಲಿ ಒಂದು ಸೃಷ್ಟಿ ಸಿದ್ಧಾಂತವಿದೆ. ಅದೇ ಸೃಷ್ಟಿ ಎಂಬುದು ಅಲೆಯಂತೆ ಇದೆ ಎಂಬುದು. ಒಂದು ಅಲೆ ಏಳುವುದು, ನಂತರ ಬೀಳುವುದು. ಇನ್ನು ಸ್ವಲ್ಪ ಕಾಲವಾದಮೇಲೆ ಪುನಃ ಏಳುವುದು, ಬೀಳುವುದು. ಸೃಷ್ಟಿ ಎಂಬುದು ಹೀಗೆ ಅನಾದಿ ಅನಂತ. ಇಲ್ಲಿ ಸೃಷ್ಟಿ ಎಂದರೆ ಏನಾದರೂ ಹೊಸದಾಗಿ ತಯಾರಾದುದು, ಹಿಂದೆ ಇರಲೇ ಇಲ್ಲ ಎಂಬ ಅರ್ಥದಲ್ಲಿ ನಾವು ತೆಗೆದುಕೊಳ್ಳುವುದಿಲ್ಲ. ನಾವು ಈಗ ನೋಡಿರುವಂತಹ ಸೃಷ್ಟಿ ಎಷ್ಟೋಸಲ ಆಗಿದೆ, ಮುಂದೆ ಆಗಲೂ ಇರುವುದು. ಈ ಸೃಷ್ಟಿಯಲ್ಲಿ ಯಾವ ಯಾವ ವೈವಿಧ್ಯತೆಗಳನ್ನು ನೋಡುವೆವೊ ಆ ಜೀವರಾಶಿಗಳ ಬೀಜವೆಲ್ಲ ಭಗವಂತನಿಂದಲೇ ಬಂದಿದೆ. ಈ ಸೃಷ್ಟಿ ನಾಶವಾದರೂ, ಈ ವೈವಿಧ್ಯತೆಗಳ ಕಾರಣವೆಲ್ಲ ಸೂಕ್ಷ್ಮಾವಸ್ಥೆಯಲ್ಲಿ ಭಗವಂತನಲ್ಲಿದೆ. ಅವನು ಪುನಃ ಬೀಜವನ್ನು ನೆಟ್ಟಾಗ ಪುನಃ ಸೃಷ್ಟಿ ಪ್ರಾರಂಭವಾಗಿ ಬಗೆಬಗೆಯ ಜೀವರಾಶಿಗಳೆಲ್ಲ ಪುನಃ ವಿಕಾಸವಾಗುವುದು. ಶ‍್ರೀರಾಮಕೃಷ್ಣರು ಇದನ್ನು ಸಣ್ಣ ಉಪಮಾನದ ಮೂಲಕ ವಿವರಿಸುವರು. ಮನೆಯಲ್ಲಿರುವ ಅಜ್ಜಿ ಹಲವು ಗಿಡಗಳನ್ನು ಹಾಕಿ ಅವುಗಳೆಲ್ಲ ಬಲಿತು ಒಣಗಿಹೋದಮೇಲೆ, ಅವುಗಳ ಕೆಲವು ಬೀಜಗಳನ್ನು ಸಂಗ್ರಹಿಸಿ ಮನೆಯ ಮಡಿಕೆಯಲ್ಲಿಡುವಳು. ಪುನಃ ಗಿಡ ನೆಡುವ ಸಮಯ ಬಂದಾಗ ಅವಳು ಇಟ್ಟಿರುವ ಬೀಜವನ್ನು ನೆಲದಲ್ಲಿ ಹಾಕುವಳು. ಹಾಗೆಯೆ ಭಗವಂತ ಪ್ರಳಯ ಸಮಯದಲ್ಲಿ ಪ್ರತಿಯೊಂದು ಜೀವರಾಶಿಯ ಬೀಜವನ್ನು ತನ್ನಲ್ಲಿ ಇಟ್ಟುಕೊಂಡಿರುವನು. ಪುನಃ ಸೃಷ್ಟಿ ಪ್ರಾರಂಭವಾದಾಗ ಅದನ್ನು ನೆಡುವನು. ಇದು ಸನಾತನವಾದ ಬೀಜ. ಎಂದಿಗೂ ಹಾಳಾಗದ ಬೀಜ.

ಅವನು ಬುದ್ಧಿವಂತರಲ್ಲಿ ಬುದ್ಧಿ ಆಗಿರುವನು. ಯಾರನ್ನು ತುಂಬಾ ಜಾಣ ಎನ್ನುವೆವೊ ಅವನಲ್ಲಿರುವ ಜಾಣತನ ಬಂದುದು ಭಗವಂತನ ಮೂಲದಿಂದ. ಎಲ್ಲಾ ಬುದ್ಧಿಯ ಹಿಂದೆಯೂ ಅವನಿರುವನು. ಆದರೆ ಎಲ್ಲಿ ಆ ಬುದ್ಧಿ ಒಳ್ಳೆಯ ಕಡೆ ತಿರುಗಿದೆಯೊ ಅಲ್ಲಿ ಅವನು ಹೆಚ್ಚು ಇರುವನು. ಈ ಜೀವನದಲ್ಲಿ ದೊಡ್ಡ ಬುದ್ಧಿವಂತಿಕೆಯೇ ಸತ್ಯವನ್ನು ತಿಳಿದುಕೊಳ್ಳುವುದು. ಈ ಪ್ರಪಂಚ ಮರಳು ಮಿಶ್ರವಾದ ಸಕ್ಕರೆಯಂತಿದೆ. ಸುಳ್ಳು ನಿಜ ಎರಡೂ ಮಿಶ್ರವಾದಂತಿದೆ. ಅನೇಕವೇಳೆ ಸತ್ಯಕ್ಕಿಂತ ಸುಳ್ಳೇ ಹೆಚ್ಚಾಗಿ ಸತ್ಯದಂತೆ ಕಾಣುತ್ತಿದೆ. ಇರುವೆಯಮುಂದೆ ಮರಳು ಸಕ್ಕರೆಯನ್ನು ಬೆರಸಿರುವುದನ್ನು ಇಟ್ಟರೆ, ಆ ಇರುವೆಯಾದರೋ ಸಕ್ಕರೆಯನ್ನು ಆರಿಸಿಕೊಂಡು ಮರಳನ್ನು ಹಾಗೇ ಬಿಡುವುದು. ಬುದ್ಧಿವಂತರೂ ಹಾಗೆಯೇ ಪ್ರಪಂಚದಲ್ಲಿ ಮಿಥ್ಯ ಎಷ್ಟು ಆಕರ್ಷಣೀಯವಾಗಿ ಕಂಡರೂ ಅದನ್ನು ತ್ಯಜಿಸಿ, ಸತ್ಯವನ್ನು ಆರಿಸಿಕೊಳ್ಳುವರು.

ತೇಜಸ್ವಿಗಳ ತೇಜಸ್ಸು ನಾನು ಎನ್ನುವನು. ತೇಜಸ್ಸು ಎಂದರೆ ಒಂದು ವ್ಯಕ್ತಿಯಲ್ಲಿರುವ ಆಕರ್ಷಣೆ. ಅದನ್ನು ಮಾತಿನಿಂದ ವಿವರಿಸುವುದಕ್ಕೆ ಆಗುವುದಿಲ್ಲ. ಆದರೆ ನಾವು ಯಾವಾಗ ಅವರೆದರುರಿಗೆ ಇರುವೆವೋ ಆಗ ನಮಗೆ ಇಚ್ಛೆ ಇಲ್ಲದೇ ಇದ್ದರೂ ಅದಕ್ಕೆ ಗೌರವ ತೋರುತ್ತೇವೆ.ಅಂತಹ ಒಂದು ಅದ್ಭುತವಾದ ಪ್ರಭಾವ ಇದೆ ಅವರ ಸುತ್ತಲೂ ಇರುವವರ ಮೇಲೆ. ಇದು ಕೇವಲ ಬಾಹ್ಯ ಆಕರ್ಷಣೆ ಅಲ್ಲ. ಇದೊಂದು ಅತೀಂದ್ರಿಯ ಆಂತರಿಕ ಆಕರ್ಷಣೆ. ಮಹಾಮಹಿಮರಲ್ಲಿ ಇದನ್ನು ನಾವು ನೋಡುತ್ತೇವೆ. ಯಾರೂ ಅವರ ವಿಷಯವನ್ನು ಜಾಹಿರಾತು ಮಾಡಿರುವುದಿಲ್ಲ. ನಮಗೆ ಅವರ ವಿಷಯ ಏನೂ ಗೊತ್ತಿಲ. ಆದರೂ ಅವರ ಸಾನ್ನಿಧ್ಯದಲ್ಲಿ ಒಂದು ವಾತವರಣ ಸ್ಪಂದಿಸುತ್ತಿರುವುದು. ಅವರಲ್ಲಿರುವ ದೊಡ್ಡತನಕ್ಕೆ ನಾವು ಮಣಿಯುವೆವು. ಅವರೊಂದು ವಿದ್ಯುತ್ ಬಲ್ಬಿನಂತೆ.ಅವರಲ್ಲಿ ಬೆಳಗುತ್ತಿರುವ ಶಕ್ತಿ ಭಗವಂತನೆಂಬ ಜೋಗದಜಲಪಾತದಲ್ಲಿ ತಯಾರಾಗಿರುವುದು.

\begin{shloka}
ಬಲಂ ಬಲವತಾಂ ಚಾಹಂ ಕಾಮರಾಗವಿವರ್ಜಿತಮ್~।\\ಧರ್ಮಾವಿರುದ್ಧೋ ಭೂತೇಷು ಕಾಮೋಽಸ್ಮಿ ಭರತರ್ಷಭ \hfill॥ ೧೧~॥
\end{shloka}

\begin{artha}
ಅರ್ಜುನ, ಬಲವಂತರಲ್ಲಿರುವ ಕಾಮರಾಗಗಳಿಲ್ಲದ ಬಲವೂ, ಭೂತಗಳಲ್ಲಿರುವ ಧರ್ಮ ವಿರುದ್ಧ\-ವಲ್ಲದ ಕಾಮವೂ ನಾನು ಆಗಿರುವೆನು.
\end{artha}

ಈ ಪ್ರಪಂಚದಲ್ಲಿ ಬಲಪ್ರಯೋಗವೆಲ್ಲ ಕೆಟ್ಟದ್ದಲ್ಲ. ಅದನ್ನು ಕೆಟ್ಟದ್ದಕ್ಕೆ ಬೇಕಾದರೆ ಉಪ ಯೋಗಿಸಿಕೊಳ್ಳಬಹುದು ಅಥವಾ ಒಳ್ಳೆಯದಕ್ಕೆ ಬೇಕಾದರೆ ಉಪಯೋಗಿಸಿಕೊಳ್ಳಬಹುದು. ಅದು ಬೆಂಕಿಯಂತೆ. ಬೇಕಾದರೆ ಇನ್ನೊಬ್ಬನ ಮನೆಯನ್ನು ಸುಟ್ಟುಹಾಕಬಹುದು ಅಥವಾ ಅದರಿಂದ ನಾವು ಅಡಿಗೆ ಮಾಡಿಕೊಳ್ಳಬಹುದು. ಹಾಗೆಯೇ ಅನೇಕ ವ್ಯಕ್ತಿಗಳಿಗೆ ಹಲವು ಬಗೆಯ ಬಲಗಳಿವೆ. ಅವನ್ನು ಹಲವರು ತಮ್ಮ ಸ್ವಾರ್ಥಕ್ಕೆ ಉಪಯೋಗಿಸಿಕೊಳ್ಳುವರು. ದೇಹದ ಬಲವಿದೆ, ಅದರಿಂದ ಸಜ್ಜನರನ್ನು ರಕ್ಷಿಸಬಹುದು ಅಥವಾ ದೇಶದಲ್ಲಿ ಪ್ರಜೆಗಳಿಗೆ ಒಳ್ಳೆಯದಾಗಲಿ ಎಂದು ಉಪಯೋಗಿಸಬಹುದು. ಯಾವಾಗ ಬಲದ ಹಿಂದೆ ಕಾಮರಾಗಗಳಿಲ್ಲವೊ ಅದು ಶುದ್ಧವಾದ ಬಲವಾಗುವುದು. ಅದನ್ನು ಒಳ್ಳೆಯದಕ್ಕೆ ಮಾತ್ರ ಸಮಾಜ ಕಲ್ಯಾಣಕ್ಕೆ ಮಾತ್ರ ಉಪಯೋಗಿಸುತ್ತಾನೆ. ಇಲ್ಲಿ ಕಾಮ ಎಂದರೆ ತನಗೆ ಇನ್ನೂ ಬರದೇ ಇರುವ ವಸ್ತುಗಳನ್ನು ಗಳಿಸಬೇಕು ಎಂಬ ಆಸೆ. ಅಧಿಕಾರ, ಕೀರ್ತಿ, ಲಾಭ ಐಶ್ವರ್ಯ ಮುಂತಾದುವುಗಳ ಕಡೆ ಗಮನ ಕೊಡದೆ ಇರಬೇಕು. ಇಂತಹ ಬಲವಂತ ತನ್ನಲ್ಲಿರುವ ಎಲ್ಲವನ್ನೂ ಸಮಾಜಕ್ಕೆ ಮರುಮಾತಾಡದೆ ಅರ್ಪಿಸಲು ಸಿದ್ಧನಾಗಿರುವನು. ಇವನಿಗೆ ಬರದಿರುವ ವಸ್ತುವಿನ ಮೇಲೆ ಹಂಬಲವಿಲ್ಲ. ಇರುವುದನ್ನಾದರೂ ರಕ್ಷಿಸಿಕೊಂಡಿರಬೇಕು ಎಂದು ಕಾತರನೂ ಆಗಿಲ್ಲ. ತನ್ನಲ್ಲಿರುವ ಶಕ್ತಿಯನ್ನು ಬಹು ಜನರ ಹಿತಕ್ಕೆ ಬಹು ಜನರ ಸುಖಕ್ಕೆ ಯಾರು ಅರ್ಪಣೆ ಮಾಡಲು ಸಿದ್ಧರಾಗಿರುವರೊ ಅಂತಹ ಬಲವಂತರ ಹಿಂದೆ ಇರುವ ಶಕ್ತಿಯೇ ಭಗವಂತನದಾಗಿದೆ.

ಧರ್ಮಕ್ಕೆ ವಿರುದ್ಧವಲ್ಲದ ಆಸೆಗಳು ಎಂದರೆ, ನನ್ನಲ್ಲಿರುವ ಆಸೆಯನ್ನು ತೃಪ್ತಿ ಪಡಿಸಿಕೊಳ್ಳುವುದರಿಂದ ಇತರರಿಗೆ ಯಾವ ಹಾನಿಯೂ ಆಗದೆ ಇರುವುದು. ಇನ್ನೊಬ್ಬನಿಗೆ ಹೋಗುತ್ತಿದ್ದ ಅನ್ನವನ್ನು ಕಿತ್ತುಕೊಂಡು ನಾನು ಅದನ್ನು ತಿನ್ನುವುದಿಲ್ಲ. ದೇಹವಿರಬೇಕು, ಅನ್ನಬಟ್ಟೆಗಳಿರಬೇಕು, ಮನೆಮಠಗಳಿರಬೇಕು. ಜೀವ ಕರ್ಮವನ್ನು ಸವೆಸಬೇಕಾದರೂ ಬದುಕಿರಬೇಕು. ಅದಕ್ಕಾಗಿ ಅವನಲ್ಲಿ ಇನ್ನೊಬ್ಬರಿಗೆ ಹಾನಿಯಾಗದ ಕೆಲವು ನಿರ್ದೋಷವಾದ ಆಸೆಗಳಿರಬಹುದು. ಇದರಿಂದ ಯಾವ ಅಪಾಯವೂ ಇಲ್ಲ. ಮತ್ತೆ ಕೆಲವು ವೇಳೆ ಮಹಾವ್ಯಕ್ತಿಯಲ್ಲಿ ವಿಶ್ವಕಾರುಣ್ಯದಿಂದ ಪ್ರೇರಿತವಾದ ಹಲವು ಆಸೆಗಳಿರಬಹುದು. ತಾವು ಬದುಕಿರುವವರೆಗೆ ಇತರರಿಗೆ ಒಳ್ಳೆಯದನ್ನು ಮಾಡಬೇಕೆಂಬ ಆಸೆ ಇದೆ. ಎಷ್ಟು ಸಲ ಬೇಕಾದರೂ ನಾನು ಜನ್ಮವೆತ್ತಿ ಬರಲು ಸಿದ್ಧವಾಗಿರುವೆ, ಪ್ರಪಂಚದಲ್ಲಿ ಎಲ್ಲ ಜೀವಿಗಳೂ ಭಗವಂತನೆಡೆಗೆ ಹೋಗಲು ನಾನು ನೆರವಾಗಬೇಕೆಂಬ ಆಸೆ ಸ್ವಾಮಿ ವಿವೇಕಾನಂದರಲ್ಲಿತ್ತು. ಲೋಕ ಕಲ್ಯಾಣವೂ ಒಂದು ದೃಷ್ಟಿಯಿಂದ ನೋಡಿದರೆ ಆಸೆಯೆ. ಆದರೆ ಆ ಭಾವನೆಯ ಹಿಂದೆ ಅಲ್ಪವ್ಯಕ್ತಿತ್ವವಲ್ಲ ಇರುವುದು, ಭಗವಂತನ ಭೂಮವ್ಯಕ್ತಿತ್ವ ಇವನನ್ನು ಒಂದು ನಿಮಿತ್ತವಾಗಿ ಮಾಡಿಕೊಂಡು ಕೆಲಸಮಾಡುತ್ತಿದೆ. ವ್ಯಕ್ತಿಗಳು ಬರೀ ನಿಮಿತ್ತ. ಇವುಗಳ ಹಿಂದೆ ನಿಜವಾಗಿ ಕೆಲಸ ಮಾಡುತ್ತಿರುವುದು ಭಗವಂತನ ಶಕ್ತಿಯೇ.

\begin{shloka}
ಯೇ ಚೈವ ಸಾತ್ತ್ವಿಕಾ ಭಾವಾ ರಾಜಸಾಸ್ತಾಮಸಾಶ್ಚ ಯೇ~।\\ಮತ್ತ ಏವೇತಿ ತಾನ್ ವಿದ್ಧಿ ನ ತ್ವಹಂ ತೇಷು ತೇ ಮಯಿ \hfill॥ ೧೨~॥
\end{shloka}

\begin{artha}
ಸಾತ್ವಿಕ, ರಾಜಸಿಕ, ತಾಮಸಿಕ ಭಾವಗಳೆಲ್ಲ ನನ್ನಿಂದಲೆ ಹುಟ್ಟಿದವು ಎಂದು ತಿಳಿ. ಆದರೆ ನಾನು ಅವುಗಳಲ್ಲಿ ಇಲ್ಲ. ಅವು ನನ್ನಲ್ಲಿವೆ.
\end{artha}

ತಮಸ್ಸು, ರಜಸ್ಸು, ಸತ್ತ್ವ ಈ ಮೂರು ಗುಣಗಳೂ ದೇವರಿಂದ ಬಂದವು. ಒಂದು ದೃಷ್ಟಿಯಿಂದ ಇವೆಲ್ಲ ಒಂದೇ ಗುಣ. ಅವುಗಳನ್ನು ಬೇರೆ ಬೇರೆ ಮಾಡಿದರೆ ಮೂರು ಭಾಗವಾಗಿ ಮಾಡಬಹುದು. ಇವು ಒಂದೇ ಗುಣದ ಮೂರು ಬೇರೆ ಬೇರೆ ಸ್ವಭಾವ ಎಂತಲೂ ಹೇಳಬಹುದು. ನೀರು ಮೂರು ರೂಪದಲ್ಲಿದೆ. ಘನೀಭೂತವಾಗಿ ಕಲ್ಲುಬಂಡೆಯಂತೆ ಹಿಮರಾಶಿಯಾಗಿ ಬಿದ್ದಿದೆ. ಇದಕ್ಕೆ ಚಲನೆ ಬಹಳ ನಿಧಾನ. ಶುದ್ಧ ತಮಸ್ಸು ಇದು. ಯಾವಾಗ ಹಿಮ ಕರಗಿ ನೀರಾಗುವುದೊ ಆಗ ಹರಿಯಲು ಮೊದಲಾಗುವುದು. ಮೇಲಿನಿಂದ ಕೆಳಗೆ ಹರಿದುಕೊಂಡು ಹೋಗುವುದು. ನೀರನ್ನು ಚೆನ್ನಾಗಿ ಮರಳಿಸಿದಾಗ, ಅಥವಾ ತೀಕ್ಷ ್ಣವಾದ ಸೂರ್ಯನ ರಶ್ಮಿ ಅದರ ಮೇಲೆ ಬಿದ್ದಾಗ ಅದು ಆವಿಯಾಗಿ ಕೆಳಗಿನಿಂದ ಮೇಲೆ ಹೋಗುವುದು. ಯಾವ ನೀರು ದ್ರವರೂಪದಲ್ಲಿರುವಾಗ ಮೇಲಿನಿಂದ ಕೆಳಕ್ಕೆ ಬರುವುದೊ, ಅದೇ ನೀರು ಆವಿಯಾದಾಗ ಕೆಳಗಿನಿಂದ ಮೇಲೆ ಹೋಗುವುದು. ಹಾಗೆಯೆ ಗುಣಗಳು. ತಮೋಗುಣಿ ಜಡ. ಅವನು ಚಲಿಸುವುದು ಬಹಳ ನಿಧಾನ. ನಿದ್ರಾ ಆಲಸ್ಯ ಅಜ್ಞಾನ ಮೋಹ ತುಂಬಿ ತುಳುಕಾಡುತ್ತಿವೆ. ರಜೋಗುಣಿಯ ಮನಸ್ಸು ಪಾದರಸದಂತೆ ಚಂಚಲ. ಯಾವಾಗಲೂ ಒಂದರಿಂದ ಮತ್ತೊಂದಕ್ಕೆ ನೆಗೆಯುತ್ತಿರುವುದು. ಬೇಕಾದಷ್ಟು ಆಸೆ ಆಕಾಂಕ್ಷೆಗಳು ಕುದಿಯುತ್ತಿವೆ ಒಳಗೆ. ಅವನ್ನು ತೃಪ್ತಿಪಡಿಸಿ ಕೊಳ್ಳಲು ಧಾವಿಸುವನು ಹೊರಗೆ. ಹಾಗೆ ಪ್ರಯತ್ನಪಡುವಾಗ, ಒಮ್ಮೆ ಸಂತೋಷ ಪಡುವನು, ಒಮ್ಮೆ ದುಃಖ ಪಡುವನು. ಹಲವಾರು ಪೆಟ್ಟುಗಳು ಬೀಳುವುವು. ಬಿದ್ದಾಗ ಸ್ವಲ್ಪಕಾಲ ತೆಪ್ಪಗಿರುವನು. ಪುನಃ ಶುರುಮಾಡುವನು ವಿಷಯಸುಖದ ಬೇಟೆಗೆ. ಅನಂತರ ಬರುವವನೆ ಸಾತ್ತ್ವಿಕ ಪ್ರಕೃತಿಯ ಮನುಷ್ಯ. ಅವನಲ್ಲಿ ಜ್ಞಾನವಿದೆ, ಶಾಂತಿ ಇದೆ, ಸಮಾಧಾನವಿದೆ, ಗಾಂಭೀರ್ಯವಿದೆ. ಇಲ್ಲಿ ಚೈತನ್ಯ ವಿಷಯವಸ್ತುಗಳನ್ನು ಬಿಟ್ಟು ಭಗವಂತನ ಕಡೆಗೆ ಹಾರುತ್ತಿರುವುದು. ಈ ಗುಣಗಳು ಬಂದಿರುವುದು ಭಗವಂತನಿಂದಲೇ. ಮೊದಲು ಕಟ್ಟಿಹಾಕಿ ಅನಂತರ ಕಟ್ಟನ್ನು ಬಿಡಿಸುವುದು. ಆದರೆ ಅವುಗಳಲ್ಲಿ ನಾನು ಇಲ್ಲಾ ಎನ್ನುತ್ತಾನೆ. ಭಗವಂತನಾದರೋ ಈ ಮೂರು ಗುಣಗಳಿಗೆ ಬಂಧಿಯಲ್ಲ. ಇವನು ಅವುಗಳ ಯಾವ ಹಂಗಿಗೂ ಸಿಕ್ಕಿಲ್ಲ. ಆದರೆ ಈ ಗುಣಗಳಾದರೊ ತಾವು ಬೇರೆ ಇರುವುದಕ್ಕೆ ಆಗುವುದಿಲ್ಲ. ಭಗವಂತನ ಆಧಾರದ ಮೇಲೆಯೇ ಇರಬೇಕಾಗಿದೆ. ಅಲೆಗೆ ಸಾಗರದ ಆಧಾರಬೇಕು. ಸಾಗರಕ್ಕೆ ಅಲೆ ಬೇಕಿಲ್ಲ. ಹಾಗೆಯೇ ಭಗವಂತನಿಗೆ ಗುಣಗಳ ಆಧಾರ ಬೇಕಾಗಿಲ್ಲ. ಅವನು ಇವುಗಳನ್ನು ಅತಿಕ್ರಮಿಸಿರುವನು.

\begin{shloka}
 ತ್ರಿಭಿರ್ಗುಣಮಯೈರ್ಭಾವೈರೇಭಿಃ ಸರ್ವಮಿದಂ ಜಗತ್~।\\ಮೋಹಿತಂ ನಾಭಿಜಾನಾತಿ ಮಾಮೇಭ್ಯಃ ಪರಮವ್ಯಯಮ್ \hfill॥ ೧೩~॥
\end{shloka}

\begin{artha}
ತ್ರಿಗುಣಮಯವಾದ ಭಾವಗಳಿಂದ ಇಡೀ ಜಗತ್ತು ಮೋಹಿತವಾಗಿದೆ. ಆದಕಾರಣವೇ ಅವುಗಳಿಗಿಂತಲೂ ವಿಲಕ್ಷಣನೂ ಅವ್ಯಯನೂ ಆದ ನನ್ನನ್ನು ಅದು ತಿಳಿದುಕೊಳ್ಳಲಾರದು.
\end{artha}

ಪ್ರಪಂಚದಲ್ಲಿರುವವರೆಲ್ಲ ಮೂರು ಗುಣಗಳ ಬಲೆಯಲ್ಲಿ ಸಿಕ್ಕಿಕೊಂಡು ನರಳುತ್ತಿರುವರು. ಈ ಮೂರು ಗುಣಗಳಲ್ಲಿ ಒಂದು ಗುಣಕ್ಕಿಂತ ಮತ್ತೊಂದು ಮೇಲು ನಿಜ. ತಮಸ್ಸಿಗಿಂತ ರಜಸ್ಸು ಮೇಲು. ರಜಸ್ಸಿಗಿಂತ ಸತ್ತ್ವ ಮೇಲು. ಆದರೆ ವಸ್ತುವಿನ ನೈಜ ಸ್ಥಿತಿ ನಾವು ಈ ಗುಣಗಳ ಆವರಣದೊಳಗೆ ಇರುವತನಕ ಗೊತ್ತಾಗುವುದಿಲ್ಲ. ಗುಣಗಳು ಎಂಬುದು ಒಂದು ಗೂಟ ಇದ್ದ ಹಾಗೆ. ಕಟ್ಟಿ ಹಾಕಿದ ದನ ಅದರ ಸುತ್ತಲೂ ಬರಬಹುದೇ ಹೊರತು ಅದನ್ನು ಕಿತ್ತುಕೊಂಡು ಹೋಗಲು ಅದಕ್ಕೆ ಸಾಧ್ಯವಿಲ್ಲ. ಶ‍್ರೀರಾಮಕೃಷ್ಣರು ಈ ಮೂರು ಗುಣಗಳನ್ನು ಮೂರು ಕಳ್ಳರಿಗೆ ಹೋಲಿಸುತ್ತಾರೆ. ಒಬ್ಬ ಪ್ರಯಾಣಿಕ ಒಂದು ಕಾಡಿನಲ್ಲಿ ಹೋಗುತ್ತಿದ್ದಾಗ ಕಳ್ಳರು ಅವನ ಮೇಲೆ ಬಿದ್ದು ಅವನ ಹತ್ತಿರ ಇದ್ದುದನ್ನೆಲ್ಲ ಅಪಹರಿಸಿದರು. ಇವನನ್ನು ಹಾಗೆ ಬಿಟ್ಟರೆ ಇವನು ಊರಿಗೆ ಹೋಗಿ ಹೇಳುತ್ತಾನೆ; ಅದಕ್ಕಾಗಿ ಇವನನ್ನು ಕೊಂದುಬಿಡೋಣ ಎಂದು ಹೇಳಿದ ಒಬ್ಬ ಕಳ್ಳ. ಅವನೇ ತಮೋಗುಣಿ. ಮತ್ತೊಬ್ಬ, ಪಾಪ, ಅವನನ್ನು ಏಕೆ ಕೊಲ್ಲಬೇಕು, ಅವನ ಕೈಕಾಲು ಕಟ್ಟಿಹಾಕಿ ಎಂದ. ಇವನೇ ರಜೋಗುಣಿ. ಇವರೆಲ್ಲ ಮುಂದೆ ಹೋದಮೇಲೆ ಇವರಲ್ಲಿ ಒಬ್ಬ ಕಳ್ಳ, ಅವರಿಂದ ತಪ್ಪಿಸಿಕೊಂಡು ಈ ಪ್ರಯಾಣಿಕನ ಬಳಿ ಬಂದು ಕಟ್ಟಿದ್ದ ಕೈಕಾಲುಗಳನ್ನು ಬಿಚ್ಚಿ ಹಾಕಿದ. ಆತನಿಗೆ ಊರಿಗೆ ಸೇರಲು ದಾರಿ ತೋರಿ, ದೂರದಲ್ಲಿ ಊರು ಕಾಣುವವರೆಗೂ ಹೋಗಿ, ಇನ್ನು ಮೇಲೆ ನೀನು ನಡೆದುಕೊಂಡು ಹೋಗಿ ಊರು ಸೇರು ಎಂದ. ಪ್ರಯಾಣಿಕ ಇವನಿಗೆ ಕೃತಜ್ಞತೆಯನ್ನು ತೋರುವುದಕ್ಕಾಗಿ, ನನ್ನ ಮನೆಗೆ ಬಾ, ನೀನು ನನಗೆ ಇಷ್ಟೊಂದು ಉಪಕಾರ ಮಾಡಿರುವೆ, ನಾನು ನಿನಗೆ ಏನಾದರೂ ಕೊಡ ಬೇಕೆಂದಿರುವೆ ಎನ್ನುತ್ತಾನೆ. ಅದಕ್ಕೆ ಆ ಕಳ್ಳ, ನಾನು ಸ್ವಲ್ಪ ಒಳ್ಳೆಯವನಿರಬಹುದು; ಆದರೆ ನಾನು ಆ ಕಳ್ಳರ ಗುಂಪಿಗೆ ಸೇರಿದವನು. ಊರಿಗೆ ಬರುವುದಕ್ಕಾಗುವುದಿಲ್ಲ. ಬಂದರೆ ಪೋಲಿಸಿನವರು ನನ್ನನ್ನು ಹಿಡಿದುಕೊಂಡುಬಿಡುವರು ಎನ್ನುತ್ತಾನೆ. ಇವನೇ ಸತ್ತ್ವಗುಣಿ. ಮೂರೂ ಕಟ್ಟಿ ಹಾಕುವುವು.

ಪರಮಾತ್ಮನಾದರೋ ಈ ಗುಣಗಳಿಗಿಂತ ಬೇರೆ. ಅವನಿಗೆ ಗುಣದಲ್ಲಿರುವುದೆಲ್ಲ ಕಾಣುವುದು. ಆದರೆ ಗುಣದೊಳಗೆ ಇರುವವರಿಗೆ ಪರಮಾತ್ಮ ಗೊತ್ತಾಗುವುದಿಲ್ಲ. ಶ‍್ರೀರಾಮಕೃಷ್ಣರು ತೆರೆಯ ಹಿಂದಿರುವ ಹೆಂಗಸರು ಒಳಗಿನಿಂದ ಕಂಡಿಯ ಮೂಲಕ ಎಲ್ಲವನ್ನೂ ನೋಡುತ್ತಾರೆ. ಆದರೆ ಹೊರಗೆ ಇರುವವರಿಗೆ ತೆರೆಯ ಹಿಂದಿರುವವರು ಕಾಣುವುದಿಲ್ಲ ಎಂಬ ಉದಾಹರಣೆಯನ್ನು ಕೊಡುತ್ತಾರೆ. ಅವನು ಗುಣಕ್ಕೆ ಅತೀತ. ಗುಣದಲ್ಲಿರುವವರು ಅವನನ್ನು ತಿಳಿದುಕೊಳ್ಳಲಾರರು ಮತ್ತು ಅವನು ಅವ್ಯಯ, ಯಾವ ಬದಲಾವಣೆಯೂ ಇಲ್ಲದವನು. ಈ ಪ್ರಪಂಚದಲ್ಲಿರುವುದೆಲ್ಲ ದೇಶಕಾಲ ನಿಮಿತ್ತದಲ್ಲಿ ಬದ್ಧವಾಗಿದೆ. ಬದ್ಧವಾಗಿರುವುದೆಲ್ಲ ಬದಲಾವಣೆ ಆಗುತ್ತಿದೆ. ಇಲ್ಲಿ ಯಾವುದೂ ಬದಲಾವಣೆ ಆಗದೇ ಇರಲಾರದು. ಕೆಲವು ಬೇಗ ಬದಲಾವಣೆ ಆಗುವುವು, ಮತ್ತೆ ಕೆಲವು ನಿಧಾನವಾಗಿ ಬದಲಾಯಿಸುವುವು. ಅಂತೂ ಎಲ್ಲಾ ಬದಲಾಯಿಸುತ್ತಿವೆ. ಪರಮಾತ್ಮನಿಗಾ\-ದರೊ ಈ ಬದಲಾವಣೆ ಇಲ್ಲ. ಅವನು ಬದಲಾವಣೆಗೆಲ್ಲ ಆಧಾರ. ಕುಂಬಾರನ ಚಕ್ರ ಒಂದು ಅಡಿಗಲ್ಲಿನ ಮೇಲೆ ಕುಳಿತು ತಿರುಗುತ್ತಿದೆ. ಆದರೆ ಅಡಿಗಲ್ಲಾದರೊ ಯಾವ ಬದಲಾವಣೆಯೂ ಇಲ್ಲದೆ ಇದೆ. ನೀರಿನ ಮೇಲೆ ಇರುವ ಮೀನು ನೆಲದಮೇಲೆ ಇರುವುದನ್ನು ತಿಳಿದುಕೊಳ್ಳಲಾರದು. ಅದು ನೀರನ್ನು ಬಿಟ್ಟು ನೆಲಕ್ಕೆ ಬರುವಂತೆಯೇ ಇಲ್ಲ. ಹೀಗಿರುವಾಗ ಅದು ತಿಳಿದುಕೊಳ್ಳುವುದು ಹೇಗೆ? ನಾವೆಲ್ಲ ಹಾಗೆ ಗುಣದ ನೀರಿನಲ್ಲಿ ವಾಸಿಸುವ ಮೀನುಗಳಂತೆ. ಅದನ್ನು ಬಿಟ್ಟು ಬಂದಲ್ಲದೆ, ನೆಲದಮೇಲೆ ಇರುವುದನ್ನು ತಿಳಿದುಕೊಳ್ಳಲಾರೆವು. ನಾವು ಅದನ್ನು ಬಿಟ್ಟು ಬರುವಂತೆಯೇ ಇಲ್ಲ.

\begin{shloka}
ದೈವೀ ಹ್ಯೇಷಾ ಗುಣಮಯೀ ಮಮ ಮಾಯಾ ದುರತ್ಯಯಾ~।\\ಮಾಮೇವ ಯೇ ಪ್ರಪದ್ಯಂತೇ ಮಾಯಾಮೇತಾಂ ತರಂತಿ ತೇ \hfill॥ ೧೪~॥
\end{shloka}

\begin{artha}
ಈ ಗುಣಮಯವಾದ ನನ್ನ ದೈವೀ ಮಾಯೆಯಿಂದ ಪಾರಾಗುವುದು ದುಸ್ತರ. ಆದರೆ ನನಗೆ ಶರಣಾಗತರಾದವರು ಈ ಮಾಯೆಯಿಂದ ಪಾರಾಗುತ್ತಾರೆ.
\end{artha}

ಈ ಮಾಯೆ ದೇವರದು. ಇದರಿಂದ ಪಾರಾಗುವುದು ಬಹಳ ಕಷ್ಟ. ಈ ಗುಣಗಳೆಂಬ ದಾರದ ಮೂಲಕ ನಮ್ಮನ್ನು ಈ ಪ್ರಪಂಚಕ್ಕೆ ಬಿಗಿದಿರುವನು. ಈ ಪ್ರಪಂಚದಲ್ಲಿ ಎಲ್ಲಾ ಅದಕ್ಕೆ ಬದ್ಧವಾಗಿದೆ. ಹೇಗೆ ಭೂಮಿ ತನ್ನ ಆಕರ್ಷಣ ಶಕ್ತಿಯಿಂದ ತನ್ನ ಮೇಲಿರುವುದನ್ನೆಲ್ಲ ಬಿಗಿದಿರುವುದೊ ಹಾಗೆ ಈ ಮಾಯೆ ನಮ್ಮನ್ನು ಬಂಧಿಸಿದೆ. ಕೆಲವು ವೇಳೆ ನಾವು ಭೂಮಿಯಿಂದ ನೆಗೆಯಲು ಯತ್ನಿಸಬಹುದು. ಆದರೆ ತಕ್ಷಣವೇ ಭೂಮಿಯ ಮೇಲೆ ಬಂದು ಬೀಳುತ್ತೇವೆ. ಎಲ್ಲಿಯವರೆಗೆ ಮಾಯೆ ಹೇಳಿದಂತೆ ಕೇಳಿಕೊಂಡಿರುವೆವೊ ಅಲ್ಲಿಯವರೆಗೆ ಅದು ನಮ್ಮನ್ನು ಅಷ್ಟು ಕಾಡುವುದಿಲ್ಲ. ಅವಳಿಗೆ ಗೊತ್ತಿದೆ, ನಾವು ನಿತ್ಯದಾಸರು ಎಂದು. ಆದರೆ ಯಾವಾಗ ನಾವು ಅವಳ ಪಾಶದಿಂದ ನುಸುಳಿಕೊಂಡು ಹೋಗುವುದಕ್ಕೆ ಪ್ರಯತ್ನಿಸುತ್ತೇವೆಯೊ ಆಗ ಬಲವಾದ ಸರಪಳಿಯನ್ನು ಅವಳು ಹಾಕಿ ಬಿಗಿದಂತೆ ಕಾಣುವುದು, ತಪ್ಪಿಸಿಕೊಂಡು ಹೋಗದೆ ಇರಲಿ ಎಂದು. ನಮ್ಮ ಕೆಲವು ಅಭ್ಯಾಸಗಳನ್ನು ಬಿಡ ಬೇಕೆಂದು ಯತ್ನಿಸಿದಾಗ ಇದು ವೇದ್ಯವಾಗುವುದು. ಯಾವುದನ್ನು ಬಿಡಬೇಕೆಂದಿರುವೆವೊ ಅದನ್ನು ಎಂದಿಗಿಂತ ಹೆಚ್ಚಾಗಿ ಅವತ್ತು ಮಾಡಿರುವೆವು. ಮತ್ತು ಒಂದು ಪಾಶದಿಂದ ಕಿತ್ತುಕೊಂಡು ಹೋದರೆ ಇನ್ನೊಂದು ಪಾಶ ಕಾದಿದೆ ನಮ್ಮನ್ನು ಬಿಗಿಯುವುದಕ್ಕೆ. ಒಂದು ವೇಷದಲ್ಲಿ ನಮ್ಮ ಮನೋದೌರ್ಬಲ್ಯವನ್ನು ಕಂಡುಹಿಡಿದು ಆಚೆಗೆ ಅಟ್ಟಿದರೆ, ಹೊರಗೆ ಹೋಗಿ ಬೇರೊಂದು ವೇಷವನ್ನು ಅದು ಹಾಕಿಕೊಂಡು ಬರುವುದು. ಅದು ನಮಗೆ ಗೊತ್ತೇ ಆಗುವುದಿಲ್ಲ, ನಾವು ಕಳುಹಿಸಿದ ಹಳೇ ಶತ್ರು ಎಂಬುದು. ಸ್ವಲ್ಪ ಕಾಲ ಅದರ ಬಲೆಗೆ ಬಿದ್ದಾದಮೇಲೆಯೇ ಅದು ನಮಗೆ ಅರಿವಾಗಬೇಕಾದರೆ.

ಎಷ್ಟೇ ಕಷ್ಟವಾದರೂ ಅದರಿಂದ ಪಾರಾಗುವುದಕ್ಕೆ ಒಂದು ದಾರಿ ಇದೆ ಎನ್ನುವನು. ಅದೇ ಯಾರು ಮಾಯಾಧೀಶನೋ ಅವನಲ್ಲಿ ಶರಣಾಗುವುದು, ದೇವರೇ! ದಯವಿಟ್ಟು ನನ್ನನ್ನು ಕಾಡಬೇಡ ಎಂದು ಬೇಡುವುದು, ಮೊರೆ ಇಡುವುದು. ಅವನು, ಕಾಡುವ ಮಾಯೆಗೆ ಸುಮ್ಮನಿರು ಎಂದರೆ ಅದು ಸುಮ್ಮನಾಗುವುದು. ನಾವು ಅದರೊಡನೆ ಎಷ್ಟು ಸೆಣಸಿದರೂ ಅದು ನಮ್ಮನ್ನು ಬಿಡುವುದಿಲ್ಲ. ಮನೆಯ ಮುಂದೆ ಇರುವ ದೊಡ್ಡದೊಂದು ನಾಯಿ ಒಳಗೆ ಯಾರನ್ನೂ ಬಿಡುವುದಿಲ್ಲ. ಅದು ಬಗುಳಿ ಕಚ್ಚಿಹಾಕುವುದು. ಹೊರಗೆ ನಿಂತು ಮನೆಯ ಯಜಮಾನನನ್ನು ಕರೆದು ನಾನು ನಿಮ್ಮೊಡನೆ ಮಾತನಾಡ ಬೇಕಾಗಿದೆ, ದಯವಿಟ್ಟು ನಿಮ್ಮ ನಾಯನ್ನು ಸುಮ್ಮನಿರಿಸಿ ಎಂದು ಕೇಳಿಕೊಂಡರೆ, ಯಜಮಾನ ಬಂದು ಆಜ್ಞೆ ಮಾಡಿದರೆ, ಆ ನಾಯಿ ಸುಮ್ಮನೆ ಗದ್ದಲವಿಲ್ಲದೆ ಅವನ ಮಾತನ್ನು ಕೇಳುವುದು. ಆದರೆ ಅದು ಅಪರಿಚಿತನ ಮಾತನ್ನು ಕೇಳುವುದಿಲ್ಲ. ಹಾಗೆಯೇ ಭಗವಂತನಲ್ಲಿ ನಾವು ಶರಣಾದರೆ ನಮ್ಮನ್ನು ಅವನು ತನ್ನ ಮಾಯೆಯ ಉಪಟಳದಿಂದ ಪಾರುಮಾಡುವನು. ಮುಂಚೆ ತಪ್ಪಿಸಿಕೊಂಡು ಹೋಗುವುದಕ್ಕೆ ಒಂದು ಉಪಾಯವನ್ನು ತಕ್ಷಣವೇ ಭಗವಂತ ಹೇಳುವನು. ಅವನಲ್ಲಿ ಶರಣಾದರೆ, ಅವನಿಚ್ಛೆ ಪಟ್ಟರೆ ಬಾಗಿಲುಗಳೆಲ್ಲ ತಾವಾಗಿ ತೆರೆಯುವುವು. ಶ‍್ರೀಕೃಷ್ಣ ಸೆರೆಮನೆಯಲ್ಲಿ ದೇವಕಿ ವಸುದೇವರಿಗೆ ಹುಟ್ಟಿದ. ಅವನನ್ನು ಸೆರೆಮನೆಯಿಂದ ವಸುದೇವ ತೆಗೆದುಕೊಂಡು ಹೋಗುವಾಗ, ಹಾಕಿದ್ದ ಬೀಗ ತಾನಾಗಿ ಕಳಚಿ ಬಿತ್ತು. ಕಾಯುತ್ತಿದ್ದ ಪಹರೆಯವರಿಗೆ ನಿದ್ದೆ ಬಂತು. ತುಂಬು ಯಮುನೆ ದಾರಿಕೊಟ್ಟಳು. ಅಸಾಧ್ಯವಾಗಿರುವುದೆಲ್ಲ ಸಾಧ್ಯವಾಯಿತು. ಯಾವಾಗ ಭಗವಂತ ಇಚ್ಛಿಸುವನೊ ಆಗ ಎಲ್ಲವೂ ಅವನಿಚ್ಛೆಗೆ ಬಾಗಬೇಕು. ನಾವು ಅವನಲ್ಲಿ ಶರಣಾಗತರಾಗಬೇಕು. ಆಗ ಮಾತ್ರ ನಾವು ಬೇಡುವುದನ್ನು ಅವನು ಕೊಡುವನು.

\begin{shloka}
ನ ಮಾಂ ದುಷ್ಕೃತಿನೋ ಮೂಢಾಃ ಪ್ರಪದ್ಯಂತೇ ನರಾಧಮಾಃ~।\\ಮಾಯಯಾಪಹೃತಜ್ಞಾನಾ ಆಸುರಂ ಭಾವಮಾಶ್ರಿತಾಃ \hfill॥ ೧೫~॥
\end{shloka}

\begin{artha}
ದುರಾಚಾರಿಗಳು, ಮೂಢರು, ಅಧಮ ಮನುಷ್ಯರು ನನಗೆ ಶರಣಾಗರು. ಅವರು ಆಸುರೀ ಸ್ವಭಾವದವರು. ಮಾಯೆಯಿಂದ ಅವರ ಜ್ಞಾನ ಹರಣವಾಗಿದೆ. 
\end{artha}

ಮಾಯೆಯನ್ನು ದಾಟುವುದಕ್ಕೆ ಅಸಾಧ್ಯ ಎಂದು ಹೇಳಿ ಅದನ್ನು ದಾಟಲು ಒಂದು ಉಪಾಯವನ್ನು ಕೂಡ ಶ‍್ರೀಕೃಷ್ಣ ಹೇಳುತ್ತಾನೆ. ಉಪಾಯ ಇದ್ದರೆ ಏತಕ್ಕೆ ಆ ಮನುಷ್ಯ ಪಾರಾಗುವುದಕ್ಕೆ ಯತ್ನಿಸುವುದಿಲ್ಲ ಎಂದು ನಾವು ಆಶ್ಚರ್ಯಪಡಬಹುದು. ಆದರೆ ನರಾಧಮರಲ್ಲಿ ಇರುವ ಸ್ವಭಾವ ಭಗವಂತನಲ್ಲಿ ಶರಣಾಗುವುದಕ್ಕೆ ಬಿಡುವುದಿಲ್ಲ. ಅವನಲ್ಲಿ ಶರಣಾದರೆ ಉದ್ಧಾರವಾಗುತ್ತೇವೆ. ಶರಣಾಗಲು ಹೋಗದೆ ಇದ್ದರೆ ಏನು ಮಾಡುವುದು? ನಮ್ಮನ್ನು ದೇವರೆಡೆಗೆ ಹೋಗಗೊಡದೆ ಇರುವುದೇ ದುರಾಚಾರ. ಮನುಷ್ಯರಿಗೆ ಕೆಟ್ಟಕೆಲಸ ಮಾಡುವುದರಲ್ಲಿಯೇ ಒಂದು ಆನಂದ. ಅದನ್ನು ಬಿಟ್ಟು ಬರಲು ಮನಸ್ಸಿಲ್ಲ. ಅವನ ಸ್ವಭಾವ ಯಾವಾಗಲೂ ಮಾಡಬಾರದ್ದನ್ನು ಮಾಡುವುದು, ಮಾಡುವುದನ್ನು ಬಿಡುವುದು. ಕೆಟ್ಟ ಮಾತುಕತೆ ನಡವಳಿಕೆ ಅವನಿಗೆ ಒಗ್ಗಿಹೋಗಿದೆ. ಅದನ್ನು ಬಿಟ್ಟು, ಬೇರೆ ಉತ್ತಮ ವಾತಾವರಣಕ್ಕೆ ಬಾ ಎಂದರೆ ಅವನಿಗಾಗುವುದಿಲ್ಲ. ಚರಂಡಿ\-ಯಲ್ಲಿರುವ ಹುಳಕ್ಕೆ ಅಲ್ಲೇ ಇರಲು ಆಸೆ. ಅದನ್ನು ಬಿಟ್ಟಬರಲು ಅದು ಯತ್ನಿಸುವುದಿಲ್ಲ. 

ಮೂಢರು, ಅಜ್ಞಾನಿಗಳು ಇವರಿಗೆ ಮುಂಚೆಯೇ ಗೊತ್ತಾಗುವುದಿಲ್ಲ, ಕೆಟ್ಟ ದಾರಿಯಲ್ಲಿ ಹೋದರೆ ಏನಾಗುವುದು ಎಂಬುದು. ಒಂದುವೇಳೆ ಪೆಟ್ಟು ತಿಂದರೂ ಅದರಿಂದ ಬುದ್ಧಿ ಕಲಿಯು\-ವವರಲ್ಲ ಅವರು. ಅವರ ಬುದ್ಧಿ ಇನ್ನೂ ಅಷ್ಟು ವಿಕಾಸವಾಗಿಲ್ಲ. ಇತರರಿಗೆ ಆಗಿರುವುದರಿಂದ ಬುದ್ಧಿ ಕಲಿಯುವುದಿಲ್ಲ ಅಥವಾ ತಮಗೇ ಆದರೂ ಅದರಿಂದ ಬುದ್ಧಿ ಕಲಿಯುವುದಿಲ್ಲ.

ಅಧಮ ತುಂಬಾ ಕೆಳಮಟ್ಟದಲ್ಲಿರುವ ಮನುಷ್ಯ. ಅವನು ಪ್ರಾಣಿಗಳಂತೆ. ತಾತ್ಕಾಲಿಕ ತೃಪ್ತಿಯನ್ನು ಕೊಡುವುದನ್ನು ಮಾತ್ರ ಮಾಡುವನು. ಅವನ ರುಚಿ ಯಾವಾಗಲೂ ಕೆಳಮಟ್ಟದ್ದು.\break ದೇವರು, ಸತ್ಯ, ಧರ್ಮ, ಕರ್ಮ ಮುಂತಾದವುಗಳು ಅವನ ಮನಸ್ಸಿಗೆ ಹತ್ತುವುದೇ ಇಲ್ಲ.\break ಯಾರಾದರೂ ಅದನ್ನು ಹೇಳಿದರೆ ಕಿವಿಯ ಮೇಲೆ ಹಾಕಿಕೊಳ್ಳುವುದೇ ಇಲ್ಲ. ಹಿಂದೆ ಕಲ್ಲಿರುವ ಗೋಡೆಗೆ ಮೊಳೆ ಹೊಡೆದಂತೆ ಇದು. ಮೊಳೆಯೇ ಬಗ್ಗಿಹೋಗುವುದು, ಕಲ್ಲಿನೊಳಗೆ ಹೋಗುವುದಿಲ್ಲ. ಈ ಪ್ರಪಂಚದಲ್ಲಿ ಇದೇ ಒಂದು ವಿಚಿತ್ರ. ಕುಡಿಯುವುದಕ್ಕೆ ಅಮೃತವಿದೆ, ಅದನ್ನು ಬಿಟ್ಟು ಚರಂಡಿಯ ನೀರು ಕುಡಿಯುವನು. ತಪ್ಪಿಸಿಕೊಂಡು ಹೋಗುವುದಕ್ಕೆ ಬಾಗಿಲು ತೆಗೆದಿದೆ. ಅದರ ಹತ್ತಿರವೇ ಹೋಗುವುದಿಲ್ಲ ಮನುಷ್ಯ. ತನ್ನ ಅದ್ಭುತವಾದ ಮಾಯೆಯಿಂದ ಈ ಪ್ರಪಂಚವನ್ನು ಸೃಷ್ಟಿಸಿ, ಇದರಿಂದ ತಪ್ಪಿಸಿಕೊಂಡು ಹೋಗುವುದಕ್ಕೆ ಕೆಲವು ಬಾಗಿಲುಗಳನ್ನು ಇಟ್ಟಿರುವನು. ಆದರೂ ಆ ಬಾಗಿಲುಗಳನ್ನು ಬಡಿಯುವವರು ಅಪರೂಪ. 

ಭಗವಂತನ ಮಾಯೆಯಲ್ಲಿ ಎರಡು ಸ್ವಭಾವಗಳಿವೆ. ಒಂದು ಆಸುರೀ ಸ್ವಭಾವ, ಮತ್ತೊಂದು ದೈವೀಸ್ವಭಾವ. ದೈವೀಸ್ವಭಾವ ನಮ್ಮನ್ನು ಪ್ರಪಂಚದಿಂದ ಬಿಡಿಸುವುದು. ಆಸುರೀಸ್ವಭಾವ ನಮ್ಮನ್ನು ಪ್ರಪಂಚಕ್ಕೆ ಕಟ್ಟಿಹಾಕುವುದು. ಮೇಲಿನ ಮೂರು ಬಗೆಯ ಮನುಷ್ಯರು ಆಸುರೀಸ್ವಭಾವವನ್ನು ಆಶ್ರಯಿಸಿದವರು. ಆಸುರೀ ಸ್ವಭಾವದವರ ಜ್ಞಾನ ಅಪಹರಿಸಲ್ಪಟ್ಟಿದೆ. ಮುಂದೇನಾಗುವುದು, ನಾವು ಇಂತಹ ಕೆಟ್ಟಕೆಲಸ ಮಾಡಿದರೆ ಅದರಿಂದ ಬರುವ ಪರಿಣಾಮ ಎಷ್ಟು ಭೀಕರವಾಗುವುದು ಎಂಬುದನ್ನು ಪರ್ಯಾಲೋಚಿಸುವುದೇ ಇಲ್ಲ.

\begin{shloka}
ಚತುರ್ವಿಧಾ ಭಜಂತೇ ಮಾಂ ಜನಾಃ ಸುಕೃತಿನೋಽಜುRನ~।\\ಆರ್ತೋ ಜಿಜ್ಞಾಸುರರ್ಥಾರ್ಥೀ ಜ್ಞಾನೀ ಚ ಭರತರ್ಷಭ \hfill॥ ೧೬~॥
\end{shloka}

\begin{artha}
ಅರ್ಜುನ, ಆರ್ತ, ಜಿಜ್ಞಾಸು, ಅರ್ಥಾರ್ಥಿ ಜ್ಞಾನಿ ಎಂಬ ನಾಲ್ಕು ಬಗೆಯ ಸದಾಚಾರಿಗಳಾದ ಜನರು ನನ್ನನ್ನು ಭಜಿಸುತ್ತಾರೆ.
\end{artha}

ಹಿಂದಿನ ಶ್ಲೋಕದಲ್ಲಿ ಯಾರು ದುರಾಚಾರಿಗಳು ಎಂಬುದನ್ನು ವಿವರಿಸಿ ಇಲ್ಲಿ ಸದಾಚಾರಿ\-ಗಳು ಯಾರು ಎಂಬುದನ್ನು ಹೇಳುತ್ತಾನೆ. ದುರಾಚಾರಿಗೆ ದೇವರ ಆವಶ್ಯಕತೆ ಇನ್ನೂ ಬಂದಿಲ್ಲ. ಅವನಿಗೆ ದೇವರು ಬೇಕಾಗಿಲ್ಲ. ಸುತ್ತಮುತ್ತಲಿನ ವಸ್ತುಗಳೇ ಸಾಕು ಅವನ ಜೀವನಕ್ಕೆ. ಸದಾಚಾರಿಗೆ ದೇವರ ಆವಶ್ಯಕತೆ ಜಾಗ್ರತವಾಗಿದೆ. ಹಸಿದರೆ ಆಹಾರವನ್ನು ಹೇಗೆ ಹುಡುಕಿಕೊಂಡು ಹೋಗುವನೊ, ಬಾಯಾರಿದರೆ ನೀರನ್ನು ಹೇಗೆ ಹುಡುಕಿಕೊಂಡು ಹೋಗುವನೊ ಹಾಗೆ ಸದಾಚಾರಿಗಳು ದೇವರನ್ನು ಹುಡುಕಿಕೊಂಡು ಹೋಗುವರು. ಹಾಗೆ ದೇವರ ಬಳಿಗೆ ಹೋಗುವವರಲ್ಲಿ ಒಬ್ಬೊಬ್ಬರು ಒಂದೊಂದು ಉದ್ದೇಶದಿಂದ ಪ್ರೇರಿತರಾಗಿರುವರು. ಮೊದಲನೆಯವನೆ ಆರ್ತ, ಈ ಪ್ರಪಂಚದಲ್ಲಿ ಸಂಕಟಕ್ಕೆ ಸಿಕ್ಕಿ ನರಳುತ್ತಿರುವವನು. ಈ ಪ್ರಪಂಚದ ವಸ್ತುಗಳು ಅವನ ಸಂಕಟವನ್ನು ಪರಿಹರಿಸಲಾರವು. ಆ ಸಂಕಟ ಹಲವು ಕಾರಣಗಳಿಂದ ಆಗಿರಬಹುದು. ತಾನು ನೆಚ್ಚಿದವರನ್ನು, ಪ್ರೀತಿಸುವವರನ್ನು ಕಳೆದುಕೊಂಡಿರಬಹುದು. ಆ ತೆರವನ್ನು ಈ ಪ್ರಪಂಚದಲ್ಲಿ ಯಾವುದೂ ಭರ್ತಿಮಾಡಲಾರದು. ಯಾವುದೊ ಅಂಜಿಕೆ ಕಳವಳ ಕಾಡುತ್ತಿದೆ. ಮುಂದೆ ಯಾವ ಸಮಯದಲ್ಲಿ ಇನ್ನು ಇಂತಹ ಕಷ್ಟವನ್ನು ಜಾರಿಮಾಡುವುದಕ್ಕೆ ದೂತರು ಸಮನ್ ತರುವರೊ ಎಂದು ಅಂಜುತ್ತಿರುವನು. ಇಂತಹ ಸಮಯದಲ್ಲಿ ಮಾನವ ಸಹಾಯ ನಮ್ಮನ್ನು ಬಹಳ ಮೇಲಕ್ಕೆ ಎತ್ತಲಾರದು. ಇನ್ನು ಯಾವ ಕಾಲದಲ್ಲಿ ದೇವರು ಬೇಕಾಗದೆ ಇದ್ದರೂ, ಸಂಕಟಗಳು ಬಂದು ನಮ್ಮನ್ನು ಕಾಡುವಾಗ ವೆಂಕಟರಮಣನ ಆವಶ್ಯಕತೆ ಬರುವುದು. ನಮ್ಮ ಸಂಕಟದ ಅರ್ಜಿಯನ್ನು ತೆಗೆದುಕೊಂಡು ಹೋಗಿ ದೇವರಮನೆಯ ಬಾಗಿಲಿನಲ್ಲಿ ನಿಂತು ಅವನನ್ನು ಕರೆಯುತ್ತೇವೆ. ಅವನು ಹೊರಗೆ ಬಂದಾಗ ನಮ್ಮ ಹವಾಲನ್ನು ಅರ್ಪಿಸುತ್ತೇವೆ. ನಮ್ಮ ಸಂಕಟಗಳನ್ನು ಅವನು ನಿವಾರಿಸಬೇಕು. ದೇವರು ಎಂಬುವನು ಒಬ್ಬನಿರುವನು, ಅವನಿಗೆ ನಮ್ಮ ಸಂಕಟವನ್ನು ಪರಿಹರಿಸುವ ಶಕ್ತಿ ಇದೆ ಎಂದು ಆ ಭಕ್ತ ಬಂದಿರುವನು. ಇಲ್ಲಿ ಇನ್ನೂ ಭಗವಂತನಮೇಲೆ ಪ್ರೀತಿ ಹುಟ್ಟಬೇಕಾಗಿದೆ. ಈ ಪ್ರಪಂಚದ ನೋವಿನ ಯಾತನೆಯಿಂದ ಪಾರಾಗಲು ಹೋಗಿರುವನೇ ಹೊರತು ದೇವರನ್ನು ಪ್ರೀತಿಸುವುದಕ್ಕಲ್ಲ. ರೋಗಿ ಪ್ರಖ್ಯಾತ ವೈದ್ಯನ ಬಳಿ ಬರುತ್ತಾನೆ. ತನ್ನ ಖಾಯಿಲೆಯನ್ನು ಗುಣ ಮಾಡು ಎಂದು ಕೇಳಿಕೊಳ್ಳುತ್ತಾನೆ. ಅವನೇನು ವೈದ್ಯನನ್ನು ಪ್ರೀತಿಸುವುದಿಲ್ಲ. ವೈದ್ಯನಿಗಿಂತ ಹೆಚ್ಚಾಗಿ ವೈದ್ಯ ಕೊಡುವ ಮದ್ದಿನ ಮೇಲೆಯೇ ಇವನ ಆಸಕ್ತಿಯೆಲ್ಲ.

ದೇವರ ಬಳಿಗೆ ಬರುವ ಎರಡನೆಯವನೇ ಜಿಜ್ಞಾಸು. ಇವನು ದೇವರು ಎಂದರೆ ಏನು ಎಂದು ತಿಳಿದುಕೊಳ್ಳಬೇಕೆಂದು ಬರುವನು. ಇವನು ತತ್ತ್ವಜ್ಞಾನಿ. ಪ್ರಪಂಚವನ್ನೆಲ್ಲ ವಿಚಾರದ ಒಂದರಿಯಿಂದ ಆಡಿರುವನು. ಇವನ ಕುತೂಹಲ ಬೌದ್ಧಿಕವಾಗಿರುವುದು. ದೇವರ ಸ್ವಭಾವಗಳೇನು, ಪ್ರಪಂಚಕ್ಕೂ ಇವನಿಗೂ ಇರುವ ಸಂಬಂಧ ಎಂತಹುದು, ಜೀವರಾಶಿಗಳಿಗೂ ಇವನಿಗೂ ಇರುವ ಸಂಬಂಧವೆಂತಹುದು, ಎಂಬುದನ್ನು ಅರಿಯಲು ಬಂದಿರುವನು. ಜೀವನದಲ್ಲಿ ಸರ್ವಶ್ರೇಷ್ಠವಾದುದು ಯಾವುದು ಎಂಬುದನ್ನು ಹುಡುಕುತ್ತಿರುವಾಗ ಎಲ್ಲಾ ವಸ್ತುಗಳನ್ನೂ ವಿಮರ್ಶಿಸಿ ಆದ\-ಮೇಲೆ, ಕೊನೆಗೆ ಸಿಕ್ಕುವುದೇ ಪರಮಾತ್ಮ. ಅದನ್ನು ಮೀರಿರುವುದು ಮತ್ತಾವುದೂ ಇಲ್ಲ. ವಿಚಾರದ ಹಾದಿಯಲ್ಲಿ ನಡೆಯುವಾಗ ಸಿಕ್ಕುವ ಕೊನೆಯ ಸ್ಥಳ ದೇವರೆಂಬುದು. ಇನ್ನು ಅದನ್ನು ಅತಿಕ್ರಮಿಸಿ ಹೋಗುವುದಕ್ಕೆ ಆಗುವುದಿಲ್ಲ. ಯತ್ನಿಸಿದರೆ ಇವನು ತನ್ನತನವನ್ನೇ ಕಳೆದುಕೊಳ್ಳುವನು. ಶ‍್ರೀರಾಮಕೃಷ್ಣರು ಹೇಳುವಂತೆ ಉಪ್ಪಿನ ಗೊಂಬೆ ಸಮುದ್ರದ ಆಳವನ್ನು ನೋಡಲು ಹೋದಂತೆ ಆಗುವುದು.

ಮೂರನೆಯವನೆ ಅರ್ಥಾರ್ಥಿ. ಜೀವನದಲ್ಲಿ ಅವನಿಗೆ ಹಣ ಬೇಕಾಗಿದೆ. ಅದಕ್ಕೇ ದೇವರಿಗೆ ಹರಕೆ ಹೊರುವನು. ಲಕ್ಷ್ಮಿ, ಕುಬೇರರು ಅವನ ಮನೆಯ ಮೂಲೆಯಲ್ಲಿ ಬಿದ್ದಿರುವರೆಂಬದನ್ನು ಕೇಳಿರುವನು. ಅವನು ಕಲ್ಪವೃಕ್ಷ ಎಂಬುದನ್ನು ಕೇಳಿರುವನು. ಯಾರು ಏನು ಕೇಳಿದರೆ ಇಲ್ಲವೆನ್ನುವವನಲ್ಲ ದೇವರು. ಬರೀ ಕೈಯಲ್ಲಿ ಕಳುಹಿಸುವುದಿಲ್ಲ ಎಂಬುದನ್ನು ಅರಿತು ಭಿಕ್ಷಕರಂತೆ ಭಿಕ್ಷಾಪಾತ್ರೆಯನ್ನು ಹಿಡಿದುಕೊಂಡು ಅವನ ಮನೆಯ ಬಾಗಲಿಗೆ ಹೋಗಿ ನಮ್ಮ ಶಂಖವನ್ನು ಊದುತ್ತೇವೆ. ಅವನು ಹೊರಗೆ ಬಂದರೆ ನಮ್ಮ ಭಿಕ್ಷಾಪಾತ್ರೆಯನ್ನು ತೋರುತ್ತೇವೆ. ಅವನಿಗಿರುವ ಐಶ್ವರ್ಯರಾಶಿಯಲ್ಲಿ ಚೂರು ಪಾರುಬೇಕು ನಮಗೆ. ದೇವರು ಬೇಡ. ದೇವರನ್ನು ತುಂಬಾ ದೊಡ್ಡವನು ಎಂದು ನೋಡುತ್ತಾನೆ. ಮಹಾ ಐಶ್ವರ್ಯವಂತನೆಂದು ಭಾವಿಸುತ್ತಾನೆ. ಇವನಿಗೆ ದೇವರ ಖಜಾನಿಯ ಮೇಲೆ ಕಣ್ಣು. ಇವನಿಗೆ ಬೇಕಾಗಿರುವುದು ಪುಡಿಕಾಸು, ಭಕ್ತಿಯಲ್ಲ ಮುಕ್ತಿಯಲ್ಲ.

ನಾಲ್ಕನೆಯವನೇ ಜ್ಞಾನಿ, ಭಗವತ್ ತತ್ತ ್ವವನ್ನು ಅರಿತವನು, ಅವನನ್ನು ಅನುಭವಿಸಿದವನು. ಅವನು ದೇವರ ಬಳಿಗೆ ಬರುತ್ತಾನೆ. ಅವನು ದೇವರಿಂದ ಏನನ್ನೂ ಯಾಚಿಸುವುದಿಲ್ಲ. ಅವನು ದೇವರೊಬ್ಬನೇ ಪರಮಪವಿತ್ರ ವಸ್ತು, ಪ್ರೀತಿಸುವುದಕ್ಕೆ ಯೋಗ್ಯವಾದ ವಸ್ತು, ಅದೊಂದೇ ಸಾರ, ಮಿಕ್ಕಿರುವುದೆಲ್ಲಾ ನಿಸ್ಸಾರ ಎಂಬುದನ್ನು ಅರಿತಿರುವನು. ದೇವರ ಅಷ್ಟೈಶ್ವರ್ಯಗಳ ಕಡೆ ಅವನ ಕಣ್ಣಿಲ್ಲ. ಅವನ ಶಕ್ತಿಯನ್ನು ನೋಡಿ ಇವನು ಚಕಿತನಾಗುವುದಿಲ್ಲ. ಮಗು ತಾಯಿಯನ್ನು ಪ್ರೀತಿಸುವಂತೆ ಪ್ರೀತಿಸುತ್ತಾನೆ. ಇವನ ಸಂಬಂಧ ತುಂಬಾ ನಿಕಟವಾದುದು. ಇವನಿಗೆ ದೇವರೊಬ್ಬನೇ ಬೇಕು. ದೇವರ ಹತ್ತಿರ ಇರುವುದಾವುದೂ ಬೇಡ.

\begin{shloka}
ತೇಷಾಂ ಜ್ಞಾನೀ ನಿತ್ಯಯುಕ್ತ ಏಕಭಕ್ತಿರ್ವಿಶಿಷ್ಯತೇ~।\\ಪ್ರಿಯೋ ಹಿ ಜ್ಞಾನಿನೋಽತ್ಯರ್ಥಮಹಂ ಸ ಚ ಮಮ ಪ್ರಿಯಃ \hfill॥ ೧೭~॥
\end{shloka}

\begin{artha}
ಅವರಲ್ಲಿ ನಿತ್ಯಯುಕ್ತನೂ, ಏಕ ಭಕ್ತಿಯುಳ್ಳವನೂ ಆದ ಜ್ಞಾನಿಯೇ ಶ್ರೇಷ್ಠ. ನಾನು ಜ್ಞಾನಿಗೆ ಅತ್ಯಂತ ಪ್ರಿಯ ಮತ್ತು ಅವನು ನನಗೆ ಪ್ರಿಯ.
\end{artha}

ಅವನೆಡೆಗೆ ಬರುವ ಈ ನಾಲ್ಕು ಜನರಲ್ಲಿ ಜ್ಞಾನಿಯೇ ಪರಮಾತ್ಮನಿಗೆ ಪರಮ ಪ್ರಿಯವಾಗಿರು\-ವವನು. ಏಕೆಂದರೆ ಅವನು ದೇವರಿಂದ ಏನನ್ನೂ ವಸೂಲಿ ಮಾಡುವುದಕ್ಕೆ ಬರುವುದಿಲ್ಲ. ಕೇವಲ ಪ್ರೀತಿಯಿಂದ ಪ್ರೇರಿತನಾಗಿ ಬರುತ್ತಾನೆ. ಅವನ ಪ್ರೀತಿ, ಪತಂಗಕ್ಕೆ ಬೆಂಕಿಯ ಮೇಲೆ ಇರುವ ಪ್ರೀತಿಯಂತೆ. ಪತಂಗ ದೂರದಲ್ಲಿ ಎಲ್ಲೋ ಕತ್ತಲಲ್ಲಿ ಹಾರಾಡುತ್ತಿರುವುದು. ಯಾವಾಗ ಅದಕ್ಕೆ ಉರಿಯುತ್ತಿರುವ ಬೆಂಕಿ ಕಾಣುವುದೊ ಅತ್ತ ಕಡೆ ಧಾವಿಸುವುದು. ಆ ಉರಿಯ ಮೇಲೆ ಅಂಥ ಆಕರ್ಷಣೆ ಅದಕ್ಕೆ. ಆ ಉರಿಯ ಹತ್ತಿರ ಬಂದು ಹಾರಾಡಿ ಕೊನೆಗೆ ಅದರಲ್ಲಿ ಬಿದ್ದು ಸಾಯುವುದು. ತನ್ನನ್ನು ಇಲ್ಲದಂತೆ ಮಾಡಿಕೊಳ್ಳುವುದು. ಅದಕ್ಕೆ ಇದರಲ್ಲೇ ಒಂದು ಆನಂದ. ಹಾಗೆಯೇ ಭಗವತ್ ತತ್ತ್ವವನ್ನು ಅರಿತ ಜ್ಞಾನಿ. ಇನ್ನು ಮೇಲೆ ಅವನು ಈ ಸಂಸಾರದ ಅಂಧಕಾರದಲ್ಲಿರಲಾರ. ಅಲ್ಲಿಂದ ಭಗವಂತನೆಂಬ ಪ್ರೇಮಶಿಖೆಯೆಡೆಗೆ ಬಂದು ಅದರಲ್ಲಿ ಬೀಳುವನು.

ಇಂತಹ ಭಕ್ತನ ಪ್ರೀತಿ ಯಾವಾಗಲೂ ಭಗವಂತನ ಕಡೆಗೆ ನಿಷ್ಠೆಯುಳ್ಳದ್ದು. ಅವನು ಸುಖವನ್ನು ಕೊಟ್ಟಾಗ ಮಾತ್ರವಲ್ಲ, ಅವನು ದುಃಖವನ್ನು ಕೊಟ್ಟರೂ ಅವನೆಡೆಗೆ ಹೋಗುವುದು. ಸೋಲು ಬರಲಿ, ಕಷ್ಟಕಾರ್ಪಣ್ಯಗಳ ಸಮೂಹವೇ ಬರಲಿ, ದೇವರನ್ನು ಪ್ರೀತಿಸುವುದನ್ನು ಮಾತ್ರ ಬಿಡುವುದಿಲ್ಲ. ಉಳಿದವರು ಹಾಗಲ್ಲ. ತಮಗೆ ಬೇಕಾದ ಹಣವೋ ಕಾಸೋ ಆರೋಗ್ಯವೋ ಸಿಕ್ಕಿತು ಎಂದರೆ ದೇವರಿಗೆ ಧನ್ಯವಾದವನ್ನು ಅರ್ಪಿಸಿ ಮರೆಯುವರು. ಪುನಃ ಇನ್ನೊಂದು ಸಲ ಅವರು ದೇವರನ್ನು ಕುರಿತು ಯೋಚಿಸಬೇಕಾದರೆ ಪ್ರಪಂಚದಿಂದ ಏನಾದರೂ ಗಂಡಾಂತರ ಬರಬೇಕು. ಆಗಲೇ ದೇವರ ಜ್ಞಾಪಕ ಬರುವುದು. ಖಾಯಿಲೆ ಬಿದ್ದರೆ ತಾನೆ ವೈದ್ಯನ ಜ್ಞಾಪಕ ಬರುವುದು, ಹಾಗೆ. ಆದರೆ ಜ್ಞಾನಿಯಾದರೊ ಹಗಲು ರಾತ್ರಿ ನದಿ ಹೇಗೆ ಸಾಗರಕ್ಕೆ ತನ್ನನ್ನು ಅರ್ಪಿಸಿಕೊಳ್ಳುತ್ತಿರುವುದೊ ಹಾಗೆ ಭಗವಂತನನ್ನು ಕುರಿತು ಚಿಂತಿಸುತ್ತಿರುವನು. ದೇವರು ಕೋಪದಿಂದ ಬಡಿದರೂ ಅವನೆಡೆಗೆ ಹೋಗುವನು. ಪ್ರೇಮದಿಂದ ತಕ್ಕೈಸಿದರೂ ಅವನೆಡೆಗೆ ಹೋಗುವನು. ಸಣ್ಣ ಮಗು ತೊಡೆಯ ಮೇಲೆ ಹಾಲು ಕುಡಿಯುತ್ತಿರುವಾಗ ಅರಿಯದೆ ಮೊಲೆಯನ್ನು ಕಚ್ಚುವುದು. ತಾಯಿ ಕೋಪದಿಂದ ಅದನ್ನು ಆಚೆಗೆ ನೂಕುವಳು. ಕೆಲವು ವೇಳೆ ನೋವು ತಾಳಲಾರದೆ ಒಂದು ಏಟನ್ನೂ ಕೊಡುವಳು. ಆ ಮಗುವಾದರೊ ಪುನಃ ತಾಯಿಯ ಕಡೆಗೆ ತೆವಳುತ್ತಾ ಬರುವುದು. ಬಡಿದ ಕೈಗಳನ್ನೇ ಹಿಡಿದುಕೊಳ್ಳುವುದು. ಮಗು ತಾಯಿಯನ್ನು ಬಿಟ್ಟು ಇರಲಾರದು. ಹಾಗೆಯೇ ಭಕ್ತ.

ಅವನ ಭಕ್ತಿ ಏಕಮುಖವಾಗಿರುವುದು. ಅದನ್ನು ಅವ್ಯಭಿಚಾರಿಣೀ ಭಕ್ತಿ ಎಂದು ಕರೆಯುತ್ತಾರೆ. ಅವನಿಂದ ಏನನ್ನೂ ಪಡೆಯಬೇಕೆಂದು ಪ್ರೀತಿಸುವುದು ಪ್ರೀತಿಯಲ್ಲ. ಅದೊಂದು ನಟನೆ. ಬರುವುದು ಬಂದುಬಿಟ್ಟರೆ ಆ ಪ್ರೀತಿಯ ಸೋಗು ಹೊರಟುಹೋಗುವುದು. ಆದರೆ ಜ್ಞಾನಿಯಾದರೂ ಮನಸ್ಸಿನ ಏಕಾಗ್ರತೆಯಿಂದ ಭಗವಂತನನ್ನು ಪ್ರೀತಿಸುವನು. ಅದು ಕವಲೊಡೆದಿಲ್ಲ, ಹಲವು ಶಾಖೆಗಳಿಲ್ಲ. ಧರೆಯ ಮೇಲೆ ಬಿದ್ದ ಹನಿಗೆ ಗೊತ್ತಾಗಿದೆ ತನ್ನ ತೌರುಮನೆ ಮಹಾಸಾಗರವೆಂದು. ಅದನ್ನು ಸೇರಲು ಒಂದು ನದಿಯೊಡನೆ ಹಗಲು ರಾತ್ರಿ ಪ್ರಯಾಣ ಮಾಡುತ್ತಿರುವುದು. ಅಂತಹ ಏಕಮುಖ, ಭಕ್ತನದು.

ಜ್ಞಾನಿಗೆ ಅತ್ಯಂತ ಪ್ರಿಯನಾದವನು ದೇವರು. ಜ್ಞಾನಿಯಾದರೊ ಪ್ರಪಂಚವನ್ನೆಲ್ಲ ಚೆನ್ನಾಗಿ ಪರೀಕ್ಷೆ ಮಾಡಿ ನೋಡಿರುವನು. ಇಲ್ಲಿರುವ ಪ್ರೀತಿ, ಇಲ್ಲಿ ನಮಗಿರುವ ಐಶ್ವರ್ಯ, ನಮ್ಮ ಬುದ್ಧಿವಂತಿಕೆ ಎಲ್ಲದರ ಮಿತಿಯನ್ನೂ ಅರಿತಿರುವನು. ಸಾಂತವಸ್ತುವನ್ನು ಪ್ರೀತಿಸಿದರೆ ನಾವು ಮುಕ್ತರಾಗೆವು, ಭೂಮವಾದ ಅನಂತಾತ್ಮನಾದ ಪರಮಾತ್ಮನೊಬ್ಬನೆ ಪ್ರೀತಿಸುವುದಕ್ಕೆ ಯೋಗ್ಯವಾದ ವಸ್ತು ಎಂಬುದನ್ನು ಅರಿತಿರುವನು. ಆ ಒಂದನ್ನು ಪ್ರೀತಿಸಿದರೆ ಪ್ರಪಂಚದಲ್ಲಿ ಎಲ್ಲವನ್ನೂ ಪ್ರೀತಿಸಿದಂತೆ ಆಗುವುದು ಎಂಬುದನ್ನು ಅರಿತಿರುವನು. ಮರದ ಬೇರಿಗೆ ನೀರು ಹಾಕಿದರೆ ಅದರ ಎಲೆ ಎಲೆ, ರೆಂಬೆ ರೆಂಬೆಗೆ ಹೇಗೆ ನೀರು ಹೋಗುವುದೊ ಹಾಗೆ. ದೇವರೇ ಪ್ರಪಂಚದ ತಾಯಿ ಬೇರು. ಅದಕ್ಕೆ ನೀರು ಹಾಕಿದರೆ ಎಲ್ಲಾ ಕಡೆಯೂ ಹರಿಯುವುದು. ದೇವರನ್ನು ಪ್ರೀತಿಸುನ ಜ್ಞಾನಿ ಪ್ರಪಂಚವನ್ನು ಧಿಕ್ಕರಿಸುವುದಿಲ್ಲ. ಅದನ್ನು ಪ್ರೀತಿಸುತ್ತಾನೆ. ಏಕೆಂದರೆ ದೇವರು ಅದರ ಹಿಂದೆ ಅವನಿಗೆ ಕಾಣುವನು. ಆದರೆ ದೇವರನ್ನು ಮರೆತು ಲೌಕಿಕ ವಸ್ತುವನ್ನು ಪ್ರೀತಿಸುವವನಾದರೊ, ಅದಲ್ಲದೆ ಬೇರೆ ಯಾವು ದನ್ನೂ ಪ್ರೀತಿಸಲಾರನು. ತಾನು ಪ್ರೀತಿಸುವ ವಸ್ತು ಯಾವಾಗ ಪ್ರೀತಿಯನ್ನು ಕೊಡುವುದಿಲ್ಲವೋ ಅಥವಾ ಬೇರೆಯವರನ್ನು ಅದು ಪ್ರೀತಿಸುವುದೊ ಆಗ ಅದನ್ನು ಸಹಿಸಲಾರನು. ತನ್ನ ಪ್ರಾಣದ ಪ್ರಾಣ ಎಂದು ಹೇಳಿಕೊಳ್ಳುತ್ತಿದ್ದ ಪ್ರಿಯತಮೆಯನ್ನು ಕೊಲ್ಲಲು ಸಿದ್ಧನಾಗಿರುವನು. ಆದರೆ ಭಗವಂತನನ್ನು ಪ್ರೀತಿಸುವವನು ಹಾಗಲ್ಲ. ಅವನು ಪ್ರಪಂಚದ ಪ್ರೀತಿಯ ಎಂತಹ ನಿರಾಶೆಯನ್ನಾದರೂ ಸಹಿಸಬಲ್ಲ. ಅವನಿಗೆ ಲೌಕಿಕ ಪ್ರೀತಿಗಳು ಗೌಣ. ಅವನ ಪರಮ ಪ್ರೀತಿಗೆ ಪಾತ್ರನಾಗಿರುವುದೇ ಪರಮಾತ್ಮ. ಉಳಿದ ವಸ್ತುಗಳು ಅದನ್ನು ಪ್ರತಿಬಿಂಬಿಸುವುದರಿಂದ ಅದನ್ನು ಪ್ರೀತಿಸುತ್ತಾನೆ. ಅದು ಪ್ರತಿಬಿಂಬಿಸದೇ ಇದ್ದರೆ ಅವನು ಕೋಪಗೊಳ್ಳುವುದಿಲ್ಲ, ಉದಾಸೀನನಾಗುತ್ತಾನೆ, ಅಷ್ಟೆ. ಅದನ್ನು ಕೊಲ್ಲುವುದಕ್ಕೆ ಹೋಗುವುದಿಲ್ಲ. ಅದಕ್ಕೆ ಶಾಪ ಕೊಡುವುದಿಲ್ಲ. ಜ್ಞಾನಿಗೆ ಪರಮಾತ್ಮನ ಮೇಲೆ ಇರುವ ಏಕಮುಖವಾಗಿರುವ ಪ್ರೀತಿಗೆ ಈ ಪ್ರಪಂಚದ ಯಾವ ಘಟನೆಗಳೂ ಭಂಗ ತಾರವು.

ಭಗವಂತನಿಗೆ ಇಂತಹ ಜ್ಞಾನಿ ಅತ್ಯಂತ ಪ್ರಿಯ. ತನ್ನ ಬಳಿಗೆ ಬರುವ ಭಕ್ತರ ಉದ್ದೇಶಗಳೆಲ್ಲ ಅವನಿಗೆ ಗೊತ್ತಿದೆ. ಈ ಪ್ರಪಂಚದಲ್ಲಿ ತಾನು ಸೃಷ್ಟಿಸಿದ ವಸ್ತುಗಳಲ್ಲಿ ಅತ್ಯಂತ ಶ್ರೇಷ್ಠವಾದುದೇ ತನ್ನನ್ನು ಪ್ರೀತಿಸುವ ಜ್ಞಾನಿ ಎಂಬುದು ಗೊತ್ತಿದೆ ದೇವರಿಗೆ. ಇದೊಂದೇ ಪರಮಾತ್ಮನೆಂಬ ಸೂರ್ಯನನ್ನು ಚೆನ್ನಾಗಿ ಪ್ರತಿಬಿಂಬಿಸುವ ಕನ್ನಡಿ. ದೇವರಿಗೆ ಜ್ಞಾನಿಯ ಮೇಲೆ ಏನೂ ಪಕ್ಷಪಾತವಿಲ್ಲ. ಸೃಷ್ಟಿಸಿದ ಎಲ್ಲಾ ವಸ್ತುಗಳ ಹಿಂದೆಯೂ ಅವನು ಒಂದೇ ಸಮನಾಗಿ ಇರುವನು. ಆದರೆ ಎಲ್ಲಾ ವಸ್ತುಗಳೂ ಒಂದೇ ಸಮನಾಗಿ ಅದನ್ನು ಪ್ರತಿಬಿಂಬಿಸುತ್ತಿಲ್ಲ. ಸೂರ್ಯನ ಬೆಳಕು ಎಲ್ಲಾ ಕಡೆಯೂ ಒಂದೇ ಸಮನಾಗಿ ಬೀಳುತ್ತಿದೆ. ಆದರೆ ಎಲ್ಲಾ ವಸ್ತುಗಳೂ ಒಂದೇ ಸಮನಾಗಿ ಪ್ರತಿಬಿಂಬಿಸುತ್ತಿಲ್ಲ. ನೆಲ ಗೋಡೆ, ನೀರು, ಬಟ್ಟೆ ಇವುಗಳ ಮೇಲೆ ಬಿದ್ದಾಗ ಒಂದು ರೀತಿ ಪ್ರತಿಬಿಂಬಿಸುವುದು. ಆದರೆ ಶುದ್ಧವಾದ ಕನ್ನಡಿ ಸೂರ್ಯನ ಬೆಳಕು ಹೇಗಿದೆಯೊ ಹಾಗೆ ಅದನ್ನು ಪ್ರತಿಬಿಂಬಿಸುವುದು. ಹಾಗೆಯೆ ಜ್ಞಾನಿ ಶುದ್ಧವಾದ ಕನ್ನಡಿಯಂತೆ. ಶ‍್ರೀರಾಮಕೃಷ್ಣರು ಭಗವಂತನಿರುವ ಪಡಸಾಲೆ \enginline{(Drawing Room)}ಭಕ್ತ ಎನ್ನುತ್ತಿದ್ದರು. ಮನೆಯಲ್ಲಿ ಯಜಮಾನ ಎಲ್ಲಿ ಬೇಕಾದರೂ ಇರಬಹುದು. ಆದರೆ ಅವನು ಹೆಚ್ಚಾಗಿ ಇರುವುದು ಪಡಸಾಲೆಯಲ್ಲಿ. ಹಸುವಿನಲ್ಲಿ ಹಾಲಿದೆ. ಹಾಲು ಬೇಕಾದರೆ, ಕೊಂಬು, ಕಿವಿ ಹಿಂಡಿದರೆ ಹಾಲು ತೊಟ್ಟಿಕ್ಕುವುದಿಲ್ಲ. ಅದರ ಕೆಚ್ಚಲನ್ನು ಹಿಂಡಬೇಕು. ಹಾಗೆಯೇ ಭಕ್ತ ಭಗವಂತನ ಕೆಚ್ಚಲು. ಅದರಲ್ಲಿ ಹಾಲಿದೆ. ಮಿಕ್ಕಿರುವ ಕಡೆಯೂ ಹಾಲಿದೆ. ಆದರೆ ಹಾಲು ಸಿಕ್ಕಲಾರದು. ಬೇರೆ ಬೇರೆ ಉದ್ದೇಶಗಳಿಂದ ದೇವರ ಹತ್ತಿರ ಬರುವವರ ಹಿಂದೆಯೂ ದೇವರೇ ಇರುವನು. ಆದರೆ ಅದು ಭಗವಂತನ ಕೆಚ್ಚಲಲ್ಲ. ಪರಮ ಭಕ್ತನಿಗೆ ಶ್ರೇಷ್ಠವಾದ ಸಿರಿ ದೇವರು ಹೇಗೊ, ಹಾಗೆಯೆ ದೇವರಿಗೆ ಪರಮ ಆಪ್ತವಾಗಿರುವುದು, ಭಕ್ತನ ಪಾದಧೂಳಿ. ಭಗವಂತನ ಸೃಷ್ಟಿಯಲ್ಲಿ ಆಕಾಶವನ್ನು ಭೇದಿಸುವ ಪರ್ವತಗಳಿವೆ. ಸಾಗರದಂತಿರುವ ನದಿಗಳಿವೆ. ಮೃತ್ಯುವನ್ನು ಹಲ್ಲಿನಲ್ಲಿ ಬಾಲದ ತುದಿಯಲ್ಲಿ ಇಟ್ಟುಕೊಂಡಿರುವ ಹಾವು ಚೇಳುಗಳಿವೆ. ಎಂತಹ ಜೀವಜಂತುಗಳನ್ನು ಸಿಗಿದು ಬಿಡುವ ನಖದಾಡೆಗಳನ್ನು ಪಡೆದ ಹುಲಿ ಸಿಂಹಗಳಿವೆ. ಹಲವು ಅದ್ಭುತ ಕಾರ್ಯಗಳನ್ನು ಮಾಡಬಲ್ಲ ಮಾನವ ನಿರುವನು. ಈ ಮಾನವರಲ್ಲಿ ಭಗವಂತನನ್ನು ಕೇವಲ ಪ್ರೀತಿಗಾಗಿ ಪ್ರೀತಿಸುವವನೇ ಶ್ರೇಷ್ಠ. ತನ್ನಿಂದ ಬಂದ ಮಹಾ ಕಾವ್ಯಕ್ಕೆ ಕವಿ ತಾನೇ ಮಣಿಯುವಂತೆ ಭಗವಂತ ಪರಮ ಭಕ್ತನನ್ನು ನೋಡಿದಾಗ ಸಾರ್ಥಕವಾಯಿತು ನನ್ನ ಸೃಷ್ಟಿ ಎಂದು ತಾನೇ ತಲೆದೂಗುವನು.

\begin{shloka}
ಉದಾರಾಃ ಸರ್ವ ಏವೈತೇ ಜ್ಞಾನೀ ತ್ವಾತ್ಮೈವ ಮೇ ಮತಮ್~।\\ಆಸ್ಥಿತಃ ಸ ಹಿ ಯುಕ್ತಾತ್ಮಾ ಮಾಮೇವಾನುತ್ತಮಾಂ ಗತಿಮ್ \hfill॥ ೧೮~॥
\end{shloka}

\begin{artha}
ಈ ಭಕ್ತರೆಲ್ಲರೂ ಉತ್ತಮರೇ ನಿಜ. ಆದರೆ ಜ್ಞಾನಿ ನನ್ನ ಆತ್ಮ. ಏಕೆಂದರೆ ನನ್ನನ್ನು ಪಡೆಯುವುದಕ್ಕಿಂತ ಶ್ರೇಷ್ಠಗತಿ ಮತ್ತೊಂದಿಲ್ಲವೆಂದು ತಿಳಿದು ಆ ಯೋಗಿ ನನ್ನ ಆಶ್ರಯವನ್ನೇ ಪಡೆಯುತ್ತಾನೆ.
\end{artha}

ಈ ನಾಲ್ಕು ಬಗೆಯ ಭಕ್ತರೆಲ್ಲರೂ ಉತ್ತಮರೇ ಎನ್ನುವನು ಶ‍್ರೀಕೃಷ್ಣ. ಏಕೆಂದರೆ ಬೇರೆ ಕೆಲಸಕ್ಕೆ ಬಾರದ ಮನುಷ್ಯನ ಹತ್ತಿರ ಹೋಗಿ ಬೇಡುವುದಕ್ಕಿಂತ ದೇವರ ಸಮೀಪಕ್ಕೆ ಬಂದಿರುವರಲ್ಲ ಇವರು. ಯಾವುದಾದರೂ ಉಪಾಯದಿಂದಲೇ ಬರಲಿ, ಅಂತೂ ತನ್ನೆಡೆಗೆ ಬಂದರಲ್ಲ ಎಂದು ಹೊಗಳುತ್ತಾನೆ. ಇವತ್ತು ದೇವರಿಂದ ಏನನ್ನೊ ವಸೂಲಿ ಮಾಡಬೇಕೆಂದು ಬಂದಿರಬಹುದು. ಆದರೆ ಇವರ ಉದ್ದೇಶ ಕ್ರಮೇಣ ಪರಿಶುದ್ಧವಾಗುತ್ತ ಬಂದು, ಕೊನೆಗೆ ದೇವರಿಗಾಗಿ, ಅವನ ಪ್ರೇಮವನ್ನು ರುಚಿ ನೋಡುವುದಕ್ಕಾಗಿ ಬರುವರು ಎಂಬುದನ್ನು ಅರಿತಿರುವನು. ಜ್ಞಾನಿಯ ವಿನಹ ಉಳಿದವರಲ್ಲಿ ಇನ್ನೂ ಲೌಕಿಕ ವಾಸನೆ ಇದೆ, ಗಣಿಯಿಂದ ತೆಗೆದ ಚಿನ್ನದ ಅದುರಿನಲ್ಲಿ ಚಿನ್ನವಲ್ಲದೆ ಇರುವ ಮಿಶ್ರ ವಸ್ತುಗಳಿರುವಂತೆ. ಆದರೆ ಚಿನ್ನವನ್ನು ಶುದ್ಧ ಮಾಡುವ ಮಹಾಯಂತ್ರದ ಮೂಲಕವಾಗಿ ಹೋಗುವಾಗ ಮಿಶ್ರವೆಲ್ಲ ಬಿಟ್ಟುಬಿಡುವುದು. ಈ ಮಿಶ್ರವನ್ನು ಕಳೆದುಕೊಳ್ಳಬೇಕಾದರೆ ದೇವರ ಹತ್ತಿರ ಹೋಗಬೇಕು. ಆಗಲೇ ಕ್ರಮೇಣ ಶುದ್ಧಿಯಾಗುತ್ತ ಬರುವನು. ಕಬ್ಬಿಣದ ಚೂರಿಗೂ ಆಯ ಸ್ಕಾಂತಕ್ಕೂ ಇರುವ ಆಕರ್ಷಣೆಯಂತೆ ಇದು. ಇವರಿಬ್ಬರ ಮಧ್ಯದಲ್ಲಿ ಒಂದು ಸ್ವಾಭಾವಿಕ ಆಕರ್ಷಣೆ ಇದೆ. ಆ ಜ್ಞಾನಿ ಯುಕ್ತಾತ್ಮ. ತನ್ನ ದೇಹ ಮನಸ್ಸು ಇಂದ್ರಿಯಗಳನ್ನೆಲ್ಲ ಪ್ರಪಂಚದ ಕಡೆ ಹರಿದು ಹೋಗುವುದನ್ನು ತಡೆಗಟ್ಟಿ ದೇವರ ಕಡೆ ಬಿಟ್ಟಿರುವನು. ಪ್ರಪಂಚದಲ್ಲಿ ದೇವರನ್ನು ಸೇರುವುದೇ ಅತಿ ಉತ್ತಮ ಸ್ಥಿತಿ ಎಂಬುದನ್ನು ಅವನು ಅರಿತಿರುವನು. ಅವನು ಬ್ರಹ್ಮಪದವಿಯನ್ನೂ ನಿಕೃಷ್ಟವಾಗಿ ಕಾಣುವನು. ಇಂದ್ರಭೋಗವನ್ನು ನಿರಾಕರಿಸುವನು. ಅವನಿಗೆ\break ಪರಮಾತ್ಮನೊಬ್ಬನೇ ಬೇಕು. ಏಕೆಂದರೆ ಇದೊಂದೇ ಪರಮ ಸತ್ಯ ಎಂಬುದನ್ನು ಅವನು ಅರಿತಿರುವನು. ಈ ಪ್ರಪಂಚದ ಆಶ್ರಯಗಳನ್ನೆಲ್ಲ ಪರಿತ್ಯಜಿಸಿ ಭಗವಂತನಲ್ಲಿ ಆಶ್ರಯವನ್ನು ಪಡೆದಿರುವನು.

\begin{shloka}
ಬಹೂನಾಂ ಜನ್ಮನಾಮಂತೇ ಜ್ಞಾನವಾನ್ ಮಾಂ ಪ್ರಪದ್ಯತೇ~।\\ವಾಸುದೇವಃ ಸರ್ವಮಿತಿ ಸ ಮಹಾತ್ಮಾ ಸುದುರ್ಲಭಃ \hfill॥ ೧೯~॥
\end{shloka}

\begin{artha}
ಅನೇಕ ಜನ್ಮಗಳ ನಂತರ ಕೊನೆಯಲ್ಲಿ ಜ್ಞಾನಿ ನನ್ನನ್ನು ಪಡೆಯುತ್ತಾನೆ. ಸಕಲವೂ ವಾಸುದೇವಮಯ ಎಂಬುದನ್ನು ಅರಿತ ಮಹಾತ್ಮ ಬಹು ದುರ್ಲಭ.
\end{artha}

ಜ್ಞಾನಿ ಈಗ ದೇವರಲ್ಲಿ ಆಶ್ರಯ ಪಡೆಯುವುದಕ್ಕೆ ಎಲ್ಲವನ್ನು ಬಿಟ್ಟು ಅವನ ಬಳಿಗೆ ಬಂದಿರಬಹುದು. ಆದರೆ ಅವನಿನ್ನೂ ಪೂರ್ಣಾತ್ಮನಾಗಿಲ್ಲ. ಅವನಿನ್ನೂ ಮಾಗಬೇಕು. ಆಧ್ಯಾತ್ಮಿಕ ಜೀವನದಲ್ಲಿ ಪರಮಗುರಿಯನ್ನು ಮುಟ್ಟಬೇಕಾದರೆ ಸತತ ಸಾಧನೆ ಮಾಡಬೇಕು. ಅದೆಲ್ಲೊ ಒಂದು ಜನ್ಮದಲ್ಲಿ ಪೂರೈಸುವ ಸಾಧನೆಯಲ್ಲ. ನಾವು ಎಷ್ಟೊ ಜನ್ಮಗಳ ಹಿಂದಿನಿಂದ ಸಂಸಾರದಲ್ಲಿ ಈಜಾಡಿ ಸಂಸ್ಕಾರಗಳನ್ನು ಮನಸ್ಸಿನಲ್ಲಿ ಸಂಗ್ರಹಿಸಿಕೊಂಡಿರುವೆವು. ಇನ್ನು ಮೇಲೆ ಉತ್ತಮ ಸಾಧನೆ ಮಾಡಿ ಆ ವಾಸನೆಯ ಸಾಲವನ್ನೆಲ್ಲ ತೀರಿಸಬೇಕಾಗಿದೆ. ಹುಟ್ಟಿದ ಮಗು ಕಾಲಮೇಲೆ ನಿಂತು ನಡೆಯಬೇಕಾದರೆ ಎಷ್ಟು ಪ್ರಯತ್ನ ಪಡಬೇಕಾಗುವುದು! ಅದರಂತೆಯೆ ನಮ್ಮ ಹಳೆಯ ವಾಸನೆಗಳನ್ನೆಲ್ಲ ಕೊಡವಿಕೊಳ್ಳ ಬೇಕಾದರೆ ಹಲವು ಜನ್ಮಗಳ ಸಾಧನೆಯಾದರೂ ಮಾಡಬೇಕು. ಒಂದೇ ಜನ್ಮದಲ್ಲಿ ಸಿಕ್ಕಲಿಲ್ಲ ಎಂದರೆ ನಾವು ಈಗ ಮಾಡಿರುವುದಾವುದೂ ವ್ಯರ್ಥವಾಗುವುದಿಲ್ಲ. ನಾವು ಮಾಡಿರುವ ಪ್ರತಿಯೊಂದು ಉತ್ತಮ ಆಲೋಚನೆಯೂ, ಉತ್ತಮ ಕರ್ಮವೂ ನಮ್ಮ ಮನಸ್ಸಿನಲ್ಲಿ ಉತ್ತಮ ಸಂಸ್ಕಾರವಾಗಿ ನಮ್ಮನ್ನು ಕಾಯಲು ಸಂಗ್ರಹವಾಗುತ್ತಿದೆ. ಕಾಲಕ್ರಮೇಣ ಇವುಗಳ ಮೊತ್ತ ಹೆಚ್ಚಾಗುತ್ತ ಬಂದು ಯಾವುದೋ ಒಂದು ಜನ್ಮದಲ್ಲಿ ನಾವು ಅಜ್ಞಾನದಿಂದ ಸಂಪೂರ್ಣವಾಗಿ ಪಾರಾಗುತ್ತೇವೆ. ನಾವು ಗುರಿಯನ್ನು ಮುಟ್ಟಿದೆವೆ ಇಲ್ಲವೆ ಎನ್ನುವುದಲ್ಲ ಪ್ರಶ್ನೆ. ನಾವು ಗುರಿ ಕಡೆಗೆ ಹೋಗುವುದಕ್ಕೆ ಸತತ ಪ್ರಯತ್ನ ನಡೆಸಿದ್ದೇವೆ ಎಂಬುದು ಮುಖ್ಯವಾಗಿರುವುದು. ನಾವು ಇರುವ ತನಕ ಪ್ರಯತ್ನ ನಡೆಸಿದ್ದರೆ ಶಾಂತವಾಗಿ ಕಣ್ಣುಮುಚ್ಚಿಕೊಳ್ಳಬಹುದು. ಮುಂದೆಯೂ ಹೀಗೆಯೇ ಹೋರಾಡುವ ಸಂಸ್ಕಾರ ನಮ್ಮ ಜೀವನದಲ್ಲಿ ಎಡಬಿಡದೆ ಇರುವುದು ಎಂಬ ದೊಡ್ಡ ಭರವಸೆ ಇದೆ.

ಗುರಿಯನ್ನು ಮುಟ್ಟಿದಾಗ ವಾಸುದೇವನೇ ಸರ್ವವೂ ಎಂಬ ಭಾವನೆ ಆಗುವುದು. ಎಲ್ಲವೂ ದೇವರಿಂದ ತುಂಬಿ ತುಳುಕಾಡುತ್ತಿದೆ. ಅವನಿಲ್ಲದ ಸ್ಥಳವಿಲ್ಲ, ಅವನಿಲ್ಲದ ಕಾಲವಿಲ್ಲ. ಎಲ್ಲಾ ನಾಮರೂಪಗಳ ಹಿಂದೆ, ಈ ವಿಶ್ವದಲ್ಲಿರುವ ಅನಂತ ಜೀವರಾಶಿಗಳ ಅಲೆಯ ಹಿಂದೆ ಇರುವುದೇ ಭಗವಂತ. ಇದೇ ಆಧ್ಯಾತ್ಮಿಕ ಜೀವನದಲ್ಲಿ ಚರಮ ಅನುಭವ. ಎಲ್ಲಾ ಕಡೆಯಲ್ಲಿಯೂ ಅವನೇ, ತನ್ನ ಹೃದಯಾಂತರಾಳದಲ್ಲಿಯೂ ಅವನೇ ಇರುವುದನ್ನು ಅನುಭವಿಸುವನು.

ಇಂತಹ ಮಹಾತ್ಮರು ದುರ್ಲಭ ಎನ್ನುವನು ಶ‍್ರೀಕೃಷ್ಣ. ಸೃಷ್ಟಿಯಲ್ಲಿ ಇಂತಹ ಪೂರ್ಣ ಅನುಭವವನ್ನು ಪಡೆದುಕೊಂಡವರು ಬಹಳ ವಿರಳ. ವಿರಳ ಎಂದರೆ ಇಲ್ಲವೆ ಇಲ್ಲ ಎಂದಲ್ಲ. ಇಂತಹವರು ಇರುವರು. ಆದರೆ ಅಲ್ಪ ಪ್ರಮಾಣದಲ್ಲಿರಬಹುದು. ಇವರೇ ಪ್ರಪಂಚದ ಸಾರ, ಸೃಷ್ಟಿಯ ಶ್ರೇಷ್ಠತಮ ಪುಷ್ಪ. ಪಿರಮಿಡ್ಡಿನ ಕೆಳಭಾಗ ಯಾವಾಗಲೂ ವಿಸ್ತಾರವಾಗಿರುವುದು. ಆದರೆ ಅದರ ಮೇಲೆ ಚೂಪಿನಲ್ಲಿ ಒಬ್ಬನೊ ಇಬ್ಬರೊ ನಿಲ್ಲುವುದಕ್ಕೆ ಮಾತ್ರ ಸ್ಥಳ ಇರುವುದು. ಹಾಗೆ ಎಲ್ಲಾ ವಾಸುದೇವಮಯ ಎಂದು ಅರಿಯುವನು ಜ್ಞಾನಿ. ಆ ಒಂದೆರಡು ಅಸಾಧಾರಣ ವ್ಯಕ್ತಿಗಳನ್ನು ಸೃಷ್ಟಿಸುವುದಕ್ಕೆ ಸೃಷ್ಟಿ ಕೋಟ್ಯಂತರ ಜೀವಿಗಳನ್ನು ಉಪಯೋಗಿಸಿಕೊಳ್ಳುತ್ತಿದೆ. ಒಂದು ಔನ್ಸ್ ಚಿನ್ನ ತಯಾರಾಗಬೇಕಾದರೆ ನೂರಾರು ಟನ್ನುಗಳಷ್ಟು ಚಿನ್ನದ ಅದುರಿನ ಕಲ್ಲುಗಳನ್ನು ಕುಟ್ಟಿ ಪುಡಿ ಪುಡಿ ಮಾಡಿ ಹಲವಾರು ರಸಾಯನಿಕ ದ್ರವ್ಯಗಳೊಂದಿಗೆ ಸೇರಿಸಿ ಕೆಲವು ಪ್ರಯೋಗಗಳನ್ನು ಮಾಡ ಬೇಕಾಗಿದೆ. ಅನಂತರವೇ ಯಂತ್ರದ ಮತ್ತೊಂದು ಕಡೆಯಿಂದ ಪೂರ್ಣತೆಯ ಸೀಲನ್ನು ಬೆನ್ನ ಮೇಲೆ ಹೊತ್ತ ಅಪರಂಜಿ ಚಿನ್ನ ಬರಬೇಕಾದರೆ. ಸಂಖ್ಯೆಯಲ್ಲ ಮುಖ್ಯ ಇಲ್ಲಿ. ಪೂರ್ಣತೆಯನ್ನು ಮುಟ್ಟಿದ ಜೀವಿಗಳು ಎಲ್ಲೊ ಕೆಲವು ಮಂದಿ ಇರಬಹುದು. ಆದರೆ ಇಡಿ ಮಾನವಕೋಟಿಯ ಮೇಲೆ ಅವರು ತಮ್ಮ ಪ್ರಭಾವವನ್ನು ಬೀರಬಲ್ಲರು. ಹೀಗಾಗುವುದು ಸಾಧ್ಯ, ಇದೇ ನಮ್ಮ ಪರಮಗುರಿ, ಮಾನವ ಕೋಟಿಯ ವಿಕಾಸದ ತುತ್ತತುದಿ ಎಂಬುದನ್ನು ಅವರ ಜೀವನವೇ ಅನುಗಾಲವೂ ಸಾರುತ್ತಿದೆ.

\begin{shloka}
ಕಾಮೈಸ್ತೈಸ್ತೈರ್ಹೃತಜ್ಞಾನಾಃ ಪ್ಪಪದ್ಯಂತೇಽನ್ಯದೇವತಾಃ~।\\ತಂ ತಂ ನಿಯಮಮಾಸ್ಥಾಯ ಪ್ರಕೃತ್ಯಾ ನಿಯತಾಃ ಸ್ವಯಾ \hfill॥ ೨೦~॥
\end{shloka}

\begin{artha}
ಅನೇಕ ಕಾಮನೆಗಳಿಂದ ಯಾರ ಜ್ಞಾನ ನಷ್ಟವಾಗಿದೆಯೋ ಅವರು ತಮ್ಮ ಪ್ರಕೃತಿಗೆ ತಕ್ಕಂತೆ, ಭಿನ್ನ ಭಿನ್ನ ಆಶ್ರಯ ಪಡೆದು ಬೇರೆ ಬೇರೆ ದೇವರುಗಳಲ್ಲಿ ಶರಣು ಹೋಗುತ್ತಾರೆ.
\end{artha}

ಎಲ್ಲರೂ ವಾಸುದೇವನೇ ಸರ್ವ ಎಂಬ ಗುರಿಯ ಕಡೆಗೆ ಹೋಗದೆ ಇರುವುದಕ್ಕೆ ಕಾರಣಗಳನ್ನು ಕೊಡುತ್ತಾನೆ ಶ‍್ರೀಕೃಷ್ಣ. ಅವರಲ್ಲಿ ಹಲವು ಕಾಮನೆಗಳಿವೆ. ಅವುಗಳನ್ನು ತೃಪ್ತಿಪಡಿಸಿಕೊಳ್ಳಬೇಕೆಂಬ ಆಸೆ ಇದೆ. ಅದನ್ನೆಲ್ಲ ಬಿಟ್ಟು ದೇವರೆಡೆಗೆ ಹೋಗಿ ಎಂದರೆ ಅವರು ಕೇಳುವುದಿಲ್ಲ. ಅನುಭವಿಸಿಯೆ ಮುಂದಕ್ಕೆ ಹೋಗಬೇಕಾಗಿದೆ. ಅದಕ್ಕಾಗಿ ಅವರು ಬೇರೆ ಬೇರೆ ದೇವರುಗಳಿಗೆ ಹೋಗುತ್ತಾರೆ. ಯಾವುದೊ ಒಂದು ದೊಡ್ಡ ಕೆಲಸ ಹಿಡಿದಿರುವನು. ಅದು ಸುಸೂತ್ರವಾಗಿ ನಡೆಯಬೇಕಾದರೆ ವಿನಾಯಕನ ವರ ಬೇಕು ಎಂದು ತಿಳಿಯುತ್ತಾನೆ. ಏಕೆಂದರೆ ಅವನೇ ಎಲ್ಲಾ ವಿಘ್ನಗಳಿಗೆ ಒಡೆಯ. ಎಲ್ಲಾ ವಿಘ್ನಗಳು ತಲೆಬಾಗುವುವು ಅವನಿಗೆ. ಅದಕ್ಕಾಗಿ ವಿಶೇಷ ಪೂಜೆ ಸಲ್ಲಿಸುತ್ತಾನೆ ಅವನಿಗೆ.\break ಹರಕೆಹೊರುತ್ತಾನೆ, ಉಪವಾಸವಿರುತ್ತಾನೆ, ಹೋಮ ಮುಂತಾದುವುಗಳನ್ನು ಮಾಡಿಸುತ್ತಾನೆ. ಮತ್ತೊಬ್ಬನಿಗೆ ವಿದ್ಯೆ ಬೇಕಾಗಿದೆ, ವಿದ್ಯೆಗೆ ಅಧಿದೇವತೆ ಸರಸ್ವತಿ. ಅವಳ ಉಪಾಸನೆ ಮಾಡುತ್ತಾನೆ. ಅವಳ ವರ ಪಡೆದು ಪ್ರಖ್ಯಾತ ಪಂಡಿತನಾಗಿ ಕವಿಯಾಗಿ ಇರಬೇಕೆಂದು ಆಸೆ. ಮತ್ತೊಬ್ಬನಿಗೆ ಐಶ್ವರ್ಯದ ಮೇಲೆ ಇಚ್ಛೆ. ಅದಕ್ಕಾಗಿ ಲಕ್ಷ್ಮಿಪೂಜೆ ಮಾಡುತ್ತಾನೆ. ಅದಕ್ಕೆ ಸಂಬಂಧಪಟ್ಟ ವ್ರತ ಕೈಗೊಳ್ಳುವನು. ಲಕ್ಷ್ಮಿ ಪೂಜೆ, ಲಕ್ಷ್ಮಿ ಸಹಸ್ರನಾಮ ಪಾರಾಯಣ ಮುಂತಾದುವುಗಳನ್ನು ಮಾಡುತ್ತಾನೆ. ಒಬ್ಬನಿಗೆ ಜಯ ಬೇಕು. ಅವನು ಭವಾನಿಯನ್ನು ಪೂಜಿಸುತ್ತಾನೆ. ಮನುಷ್ಯರಲ್ಲಿ ಹಲವು ಕಾಮನೆಗಳಿವೆ. ಆ ಕಾಮನೆಗಳನ್ನು ಈಡೇರಿಸುವುದಕ್ಕೆ ಹಲವು ದೇವ ದೇವತೆಗಳೂ ಇರುವರು. ಕಾಮಿ ಆಯಾ ದೇವತೆಗಳಲ್ಲಿ ಶರಣಾಗುತ್ತಾನೆ.

\begin{shloka}
ಯೋ ಯೋ ಯಾಂ ಯಾಂ ತನುಂ ಭಕ್ತಃ ಶ್ರದ್ಧಯಾರ್ಚಿತುಮಿಚ್ಛತಿ~।\\ತಸ್ಯ ತಸ್ಯಾಚಲಾಂ ಶ್ರದ್ಧಾಂ ತಾಮೇವ ನಿದಧಾಮ್ಯಹಮ್ \hfill॥ ೨೧~॥
\end{shloka}

\begin{artha}
ಯಾವ ಯಾವ ಭಕ್ತನು ಯಾವ ಯಾವ ದೇವತೆಗಳನ್ನು ಶ್ರದ್ಧೆಯಿಂದ ಪೂಜಿಸಲು ಇಚ್ಛಿಸುತ್ತಾನೆಯೋ ಅವನಿಗೆ ಆಯಾ ರೂಪದಲ್ಲಿಯೇ ಭಕ್ತಿಯನ್ನು ಸ್ಥಿರಗೊಳಿಸುತ್ತೇನೆ.
\end{artha}

ಇಲ್ಲಿ ಶ‍್ರೀಕೃಷ್ಣ ತನ್ನನ್ನು ಬಿಟ್ಟು ಬೇರೆ ಬೇರೆ ದೇವತೆಗಳ ಬಳಿಗೆ ಕಾಮನೆಗಳನ್ನು ಈಡೇರಿಸಿ ಕೊಳ್ಳಲು ಹೋಗುವವರನ್ನು ತಿರಸ್ಕರಿಸುವುದಿಲ್ಲ. ತಾನೊಬ್ಬನೆ ನಿಜವಾದ ದೇವರು, ಉಳಿದವರೆಲ್ಲ ಸುಳ್ಳು ದೇವರು ಎಂದು ಹೇಳುವುದಿಲ್ಲ. ಏಕೆಂದರೆ ಎಲ್ಲಾ ದೇವರಂತೆ ಇರುವವನು ಇವನೆ, ಎಲ್ಲಾ ದೇವರ ಮೂಲಕ ಕೆಲಸ ಮಾಡುವನು ಇವನೇ. ಆದರೆ ಒಬ್ಬೊಬ್ಬನಲ್ಲಿ ಒಂದೊಂದು ಅಭಿಲಾಷೆ ಯಿದೆ. ಅದನ್ನು ಖಂಡಿಸುವುದಿಲ್ಲ. ತಮ್ಮ ಬೇಡಿಕೆಗಳನ್ನು ಸಲ್ಲಿಸುವ ಬೇರೆ ಬೇರೆ ದೇವರುಗಳ ಮೇಲೆಯೆ ಅವರಿಗೆ ಹೆಚ್ಚು ಶ್ರದ್ಧೆ ಹುಟ್ಟುವಂತೆ ಮಾಡುತ್ತಾನೆ. ಯಾವಾಗ ಶ್ರದ್ಧೆ ಹುಟ್ಟುತ್ತಾ ಹೋಗುವುದೋ ಆಯಾ ದೇವ ದೇವಿಯರ ಮೇಲೆ ಪ್ರೇಮ ಹೆಚ್ಚುತ್ತಾ ಹೋಗುವುದೊ, ಆಗ ತಾನು ಲೌಕಿಕವಾದ ವಸ್ತುಗಳನ್ನು ಕೇಳುವುದನ್ನು ಬಿಟ್ಟು ಶಾಶ್ವತವಾದ ಜ್ಞಾನ ಭಕ್ತಿ ವೈರಾಗ್ಯಗಳನ್ನು ಬೇಡುತ್ತಾನೆ. ಈಗ ಬೇರೆ ಬೇರೆಯಂತೆ ಕಂಡರೂ ಆಧ್ಯಾತ್ಮಿಕ ಜೀವನದಲ್ಲಿ ಮುಂದುವರಿದಂತೆ, ಒಂದೇ ಪರಮಾತ್ಮ ಹಲವು ನಾಮಗಳನ್ನು ಧರಿಸಿ, ಹಲವು ಪಾತ್ರಗಳಲ್ಲಿ ನಿರತವಾಗಿರುವುದು ಗೊತ್ತಾಗುವುದು.

ಶ‍್ರೀಕೃಷ್ಣ ಎಲ್ಲಿಯೂ ಯಾರನ್ನೂ ಖಂಡಿಸುವುದಕ್ಕೆ ಹೋಗುವುದಿಲ್ಲ. ಯಾರ ಶ್ರದ್ಧೆಗೂ ಭಂಗ ತರುವುದಿಲ್ಲ. ಮನುಷ್ಯ ಎಲ್ಲಿರುವನೊ ಅಲ್ಲಿಂದ ಮುಂದಕ್ಕೆ ಮೇಲಕ್ಕೆ ಒಯ್ಯುವನು. ಅವನ ಶ್ರದ್ಧೆ ಮತ್ತೂ ಆಳವಾಗಬೇಕು, ಪ್ರಪಂಚದಲ್ಲಿರುವ ದೇವ ದೇವತೆಗಳನ್ನೆಲ್ಲ ಒಳಕೊಳ್ಳುವಷ್ಟು ವಿಸ್ತಾರವಾಗಬೇಕು. ಅದಕ್ಕೆ ಏನು ಬೇಕೊ ಅದನ್ನೆಲ್ಲ ಮಾಡುವನು. ಎಲ್ಲಿಯೂ ತಾತ್ಸಾರವಿಲ್ಲ. ಕುಹುಕ ಇಲ್ಲ, ನಿಂದೆಯಿಲ್ಲ. ಒಬ್ಬ ಯಾವುದನ್ನಾದರೂ ಪೂಜಿಸುತ್ತಿರಲಿ–ಕಲ್ಲು ಮಣ್ಣು ಲೋಹ, ಮಾರಮ್ಮ ಮಸಣಮ್ಮ, ಶಿವ ಪಾರ್ವತಿ ವಿಷ್ಣು ರಾಮ, ಎಲ್ಲರೂ ಕರೆಯುತ್ತಿರುವುದು ಅವರನ್ನೇ ಎಂಬುದನ್ನು ಅವನು ಚೆನ್ನಾಗಿ ಅರಿತವನು. ಮನೆಯಲ್ಲಿ ಎಳೆಯ ಮಗುವೊಂದಿದೆ. ಅದಕ್ಕೆ ಅಪ್ಪ ಎಂದು ಕರೆಯುವುದಕ್ಕೆ ಬರುವುದಿಲ್ಲ. ಸುಮ್ಮನೆ ‘ಪ’ ಎನ್ನುವುದು. ತಂದೆ ಬಂದು ಎತ್ತಿಕೊಳ್ಳುವುದಿಲ್ಲವೆ? ಹಾಗೆಯೆ ಯಾವ ದೇವರ ಮೂಲಕ ಕರೆಯಲಿ, ಯಾವ ಉದ್ದೇಶದಿಂದಲೆ ಕರೆಯಲಿ, ಎಲ್ಲರಿಗೂ ಓ ಕೊಡುವನು ದೇವರು. ಬೇಕಾಗಿರುವುದು ನಮ್ಮ ಕರೆಯ ಹಿಂದೆ ಇರುವ ಶ್ರದ್ಧೆ. ಆ ಶ್ರದ್ಧೆ ಮೊದಲು ಅಲ್ಪವಾಗಿದ್ದರೂ ದೇವರೇ ಅದನ್ನು ಕ್ರಮೇಣ ವೃದ್ಧಿ ಮಾಡುವನು.

\begin{shloka}
ಸ ತಯಾ ಶ್ರದ್ಧಯಾ ಯುಕ್ತಸ್ತಸ್ಯಾರಾಧನಮೀಹತೇ~।\\ಲಭತೇ ಚ ತತಃ ಕಾಮಾನ್ ಮಯೈವ ವಿಹಿತಾನ್ ಹಿ ತಾನ್ \hfill॥ ೨೨~॥ 
\end{shloka}

\begin{artha}
ಅವನು ಶ್ರದ್ಧಾಪೂರ್ವಕ ಆಯಾ ಸ್ವರೂಪದ ಆರಾಧನೆ ಮಾಡಿ, ಅದರ ಮೂಲಕ ನನ್ನಿಂದ ನಿರ್ಮಿತವಾದ, ಹಾಗೂ ತನಗೆ ಬೇಕಾದ ಕಾಮಫಲಗಳನ್ನು ಹೊಂದುತ್ತಾನೆ.
\end{artha}

ಯಾವಾಗ ಭಕ್ತಿ ಶ್ರದ್ಧೆಯಿಂದ ಆಯಾ ದೇವತೆಗಳನ್ನು ಬೇಡುತ್ತಾನೆಯೋ ಭಗವಂತನೇ ಅವನ ಕೋರಿಕೆಗಳನ್ನು ಈ ದೇವತೆಗಳ ಮೂಲಕ ಈಡೇರಿಸುತ್ತಾನೆ. ಒಂದೇ ಭಗವತ್ ಶಕ್ತಿ ಎಲ್ಲಾ ದೇವ ದೇವಿಯರ ಮೂಲಕವಾಗಿಯೂ ಕೆಲಸ ಮಾಡುತ್ತಿರುವುದು. ಉಪಾಸನೆ ಮಾಡುವವನಿಗೆ ಇದು ಗೊತ್ತಿ ಲ್ಲದೆ ಇರಬಹುದು. ಆದರೆ ದೇವರಿಗೆ ಇದು ಗೊತ್ತಿದೆ. ದೇವರಾದರೊ ಉದಾರ ದೃಷ್ಟಿಯಿಂದ ನೋಡುತ್ತಾನೆ. ಪಾಪ, ಜೀವಿಗಳು ಆಸೆ ಆಕಾಂಕ್ಷೆಗಳಿಂದ ಕೂಡಿರುತ್ತಾರೆ. ಎಲ್ಲರಿಗೂ ಅದು ಮಿಥ್ಯ, ಪ್ರಯೋಜನವಿಲ್ಲ, ದೇವರೊಬ್ಬನೇ ಸತ್ಯ ಎಂಬುದು ತಿಳಿದಿಲ್ಲ. ಒಂದು ವೇಳೆ ಬೌದ್ಧಿಕವಾಗಿ ತಿಳಿದಿದ್ದರೂ, ಮನಸ್ಸಿನ ಚಪಲ ಬಿಟ್ಟುಹೋಗಿಲ್ಲ. ಅದನ್ನು ರುಚಿ ನೋಡಲಿ, ಅನಂತರವೆ ಅದರ ಮೇಲೆ ಜುಗುಪ್ಸೆ ಹುಟ್ಟುಬೇಕಾದರೆ ಎಂದು ಅವರ ಬೇಡಿಕೆಗಳನ್ನು ಈಡೇರಿಸುವನು.

\begin{shloka}
ಅಂತವತ್ತು ಫಲಂ ತೇಷಾಂ ತದ್ಭವತ್ಯಲ್ಪಮೇಧಸಾಮ್~।\\ದೇವಾನ್ ದೇವಯಜೋ ಯಾಂತಿ ಮದ್ಭಕ್ತಾ ಯಾಂತಿ ಮಾಮಪಿ \hfill॥ ೨೩~॥
\end{shloka}

\begin{artha}
ಆದರೆ ಅಲ್ಪ ಬುದ್ಧಿಗಳಿಗೆ ದೊರೆಯುವ ಫಲ ನಾಶವುಳ್ಳದ್ದು. ದೇವತೆಗಳನ್ನು, ಭಜಿಸುವವರು ದೇವತೆಗಳನ್ನು ಪಡೆಯುತ್ತಾರೆ. ನನ್ನನ್ನು ಭಜಿಸುವವರು ನನ್ನನ್ನು ಪಡೆಯುತ್ತಾರೆ.
\end{artha}

ದೇವರೇನೋ ಇವರು ಕೇಳಿದ್ದನ್ನು ಕೊಡುತ್ತಾನೆ. ಇದನ್ನು ಕೇಳುವವರು ಅಲ್ಪ ಬುದ್ಧಿಗಳು. ಅವರಿಗೆ ಜೀವನದಲ್ಲಿ ಯಾವುದು ಶ್ರೇಷ್ಠವಾದುದು, ಯಾವುದು ಗೌಣವಾದುದು ಎಂಬುದು\break ಗೊತ್ತಿಲ್ಲ. ವಜ್ರದ ಗಣಿಯ ಹತ್ತಿರ ಕುಳಿತು ವಜ್ರವನ್ನು ಹುಡುಕುವುದನ್ನು ಬಿಟ್ಟು ಗಾಜಿನ ಚೂರನ್ನು ಹುಡುಕುತ್ತಿರುವರು. ಇದು ದೊಡ್ಡ ದೊರೆಯ ಸನ್ನಿಧಿಯಲ್ಲಿ ನಿಂತುಕೊಂಡಾಗ ನಿನಗೆ ಏನು ಬೇಕು ಕೇಳು ಎಂದರೆ, ಒಂದೆರೆಡು ಬದನೆಕಾಯನ್ನು ಕೇಳುವಂತೆ. ಅವನು ಬೇಕಾದರೆ ನಮಗೆ ಒಂದು ಜಮೀನ್​ದಾರಿಯನ್ನೇ ಕೊಡಬಲ್ಲ. ಅದನ್ನು ಬಿಟ್ಟು ಕೆಲಸಕ್ಕೆ ಬಾರದ ವಸ್ತುಗಳನ್ನು ಕೇಳುತ್ತೇವೆ.

ಅಲ್ಪ ಬುದ್ಧಿಗಳಿಗೆ ಯಾವುದನ್ನು ಭಗವಂತನಿಂದ ಬೇಡುತ್ತೇವೆಯೊ ಅದು ನಾಶವುಳೃದ್ದು ಎಂಬುದು ಗೊತ್ತಿಲ್ಲ. ನಾವು ಕೇಳುವ ಪ್ರಿಯವಾದ ವಸ್ತುವಿನ ಹಿಂದೆ ಕೇಳದ ಅಪ್ರಿಯವಾದುದು ಬರುವುದು. ಅದನ್ನು ನಾವು ತ್ಯಜಿಸುವುದಕ್ಕೆ ಆಗುವುದಿಲ್ಲ. ಒಂದಕ್ಕೆ ಕೈ ಒಡ್ಡಿದರೆ ಮತ್ತೊಂದನ್ನೂ ಸ್ವೀಕರಿಸಲೇಬೇಕಾಗಿದೆ. ಪ್ರಿಯ ಅಪ್ರಿಯ ಬೇರೆ ಬೇರೆ ಅಲ್ಲ. ಅವೆರಡೂ ಒಂದೇ ನಾಣ್ಯದ ಎರಡು ಭಾಗ. ಎಂತಹ ಪ್ರಿಯವಾದ ವಸ್ತುವನ್ನೇ ನಾವು ಪಡೆದರೂ, ಅದು ನಮ್ಮನ್ನು ಬಿಟ್ಟು ಹೋಗುವುದು, ಅಥವಾ ನಾವು ಅದನ್ನು ಬಿಟ್ಟು ಹೋಗುವೆವು. ಯಾರೂ ಇದನ್ನು ತಪ್ಪಿಸಿಕೊಳ್ಳುವುದಕ್ಕೆ ಆಗುವುದಿಲ್ಲ. ಪ್ರಿಯವಾದುದನ್ನು ಚೀಪುತ್ತಿರುವಾಗಲೇ ಅದರ ಹಿಂದೆ ಇರುವ ಅಪ್ರಿಯ ಕಹಿ ತೋರುವುದು. ಅದನ್ನು ಬಿಸಾಡುವುದಕ್ಕೂ ಆಗುವುದಿಲ್ಲ. ಅದನ್ನು ಗೊಣಗಾಡಿಕೊಂಡು ನುಂಗಲೇ ಬೇಕಾಗುವುದು.

ಇತರ ದೇವತೆಗಳನ್ನು ಆರಾಧಿಸುವವರು ಆ ದೇವತೆಗಳನ್ನು ಹೊಂದುತ್ತಾರೆ. ಅವರು ಕಳೆದುಕೊಂಡ ವಸ್ತುಗಳನ್ನು ಭರ್ತಿ ಮಾಡಬೇಕಾಗಿದೆ. ಅದಕ್ಕಾಗಿಯೇ ಪದೇ ಪದೇ ಅಲ್ಲಿಗೆ ಹೋಗುತ್ತಿರ ಬೇಕಾಗಿದೆ. ಅಲ್ಲಿಗೆ ಹೋಗುವುದೂ ಕೂಡ ಏನೂ ಆ ದೇವರ ಮೇಲಿರುವ ಪ್ರೀತಿಯಿಂದಲ್ಲ. ತಮ್ಮ ಭೋಗವಸ್ತುಗಳು ರಿಪೇರಿಗೆ ಬಂದಿವೆ. ಅದಕ್ಕೆ \enginline{Spare Parts} ಪ್ರಪಂಚದ ಗುಜರಿಯಲ್ಲಿ ಸಿಕ್ಕುವು ದಿಲ್ಲ. ಅದಕ್ಕಾಗಿಯೇ ದೇವರ ಬಳಿಗೆ ಹೋಗುತ್ತಾರೆ. ಆದರೆ ಭಗವಂತನನ್ನು ಪ್ರೀತಿಗಾಗಿ ಪ್ರೀತಿಸುವ ಭಕ್ತನೂ ಭಗವಂತನೆಡೆಗೆ ಹೋಗುತ್ತಾನೆ. ಅವನಿಂದ ಏನನ್ನೂ ಯಾಚಿಸುವುದಕ್ಕೆ ಹೋಗುವುದಿಲ್ಲ. ಅವನಿಗೆ ದೇವರೇ ಬೇಕಾಗಿರುವುದು, ಅವನ ಖಜಾನಿಯೊಳಗೆ ಇರುವ ಚೂರುಪಾರುಗಳಲ್ಲ.

\begin{shloka}
ಅವ್ಯಕ್ತಂ ವ್ಯಕ್ತಿಮಾಪನ್ನಂ ಮನ್ಯಂತೇ ಮಾಮಬುದ್ಧಯಃ~।\\ಪರಂ ಭಾವಮಜಾನಂತೋ ಮಮಾವ್ಯಯಮನುತ್ತಮಮ್ \hfill॥ ೨೪~॥
\end{shloka}

\begin{artha}
ಅವ್ಯಯವೂ ಅನುಪಮವೂ ಆದ ನನ್ನ ಪರಭಾವವನ್ನು ಅರಿಯದ ಅವಿವೇಕಿಗಳು ಅವ್ಯಕ್ತವಾಗಿದ್ದು ವ್ಯಕ್ತ ಗೊಂಡವನೆಂದು ನನ್ನನ್ನು ಎಣಿಸುತ್ತಾರೆ.
\end{artha}

ಶ‍್ರೀಕೃಷ್ಣನಿಗೆ ಎರಡು ವ್ಯಕ್ತಿತ್ವವಿದೆ. ಒಂದು ಹೊರಗಿನದು, ಸಾಧಾರಣ ಮನುಷ್ಯನಂತೆ ಅವನು ದೇವಕಿ ವಸುದೇವರಿಗೆ ಪುತ್ರನಾಗಿ, ದ್ವಾರಕೆಗೆ ರಾಜನಾಗಿ, ಪಾಂಡವರಿಗೆ ಸ್ನೇಹಿತನಾಗಿದ್ದಾನೆ. ಅವನೊಬ್ಬ ದೊಡ್ಡ ತತ್ತ್ವಜ್ಞಾನಿ, ಮಹಾ ಕರ್ಮಯೋಗಿ. ಅವನು ಚಾರಿತ್ರಿಕ ವ್ಯಕ್ತಿ, ಕೆಲವು ವರುಷಗಳ ಹಿಂದೆ ಹುಟ್ಟಿ, ಕೆಲವು ಕಾಲವಾದ ಮೇಲೆ ಪ್ರಪಂಚವನ್ನು ಬಿಟ್ಟು ಹೋಗುವನು. ಅವಿವೇಕಿಗಳು ನೋಡುವುದು ಇದನ್ನೆ. ಅವರು ತಮ್ಮಂತೆಯೇ ಶ‍್ರೀಕೃಷ್ಣ, ಆದರೆ ತಮಗಿಂತ ಸ್ವಲ್ಪ ಮೇಲಿರಬಹುದು ಅಷ್ಟೆ ಎಂದು ಭಾವಿಸುವರು. ಎಲ್ಲರೂ ಒಂದೇ ಕ್ಲಾಸಿನಲ್ಲಿ ಓದುತ್ತಿರುವರು. ಒಬ್ಬನಿಗೆ ಸ್ವಲ್ಪ ಜಾಸ್ತಿ ನಂಬರು ಬಂದಿದೆ, ಇನ್ನೊಬ್ಬನಿಗೆ ಸ್ವಲ್ಪ ಕಡಿಮೆ ನಂಬರ್ ಬಂದಿದೆ ಅಷ್ಟೇ. ಅವನು ನಮ್ಮಂತೆ ಹುಟ್ಟುವುದಕ್ಕೆ ಮುಂಚೆ ಅವ್ಯಕ್ತನಾಗಿದ್ದ. ಈಗ ವ್ಯಕ್ತನಾಗಿರುವನು, ಅನಂತರ ಪುನಃ ಅವ್ಯಕ್ತನಾಗುತ್ತಾನೆ ಎಂದು ತಮ್ಮಂತೆಯೇ ಅವನನ್ನು ನೋಡುತ್ತಾರೆ. ಆದರೆ ಶ‍್ರೀಕೃಷ್ಣನಿಗೆ ಆ ಚಾರಿತ್ರಿಕ ವ್ಯಕ್ತಿತ್ವದ ಹಿಂದೆ ಇನ್ನೊಂದು ವ್ಯಕ್ತಿತ್ವವಿದೆ. ಅದೇ ಭೂಮ ವ್ಯಕ್ತಿತ್ವ. ಅದು ಅವ್ಯಯವಾದುದು. ಯಾವ ಬದಲಾವಣೆಗೂ ಸಿಲುಕದುದು. ಅವನು ಸಾಕ್ಷಾತ್ ಭಗವಂತ. ಇವನಿಗೆ ಸಮನಾದವರು, ಇವನನ್ನು ಮೀರಿದವರು, ಜಗದಲ್ಲಿ ಯಾರೂ ಇಲ್ಲ. ಇವನೊಂದು ಆಕಾರವನ್ನು ಧರಿಸಿ ಅವತಾರವನ್ನು ಮಾಡಿದ್ದರೂ ತನ್ನ ಹಿಂದಿನದನ್ನು ಮರೆತಿಲ್ಲ. ಅವನು ಪ್ರಪಂಚಕ್ಕೆ ನಮ್ಮಂತೆ ಕರ್ಮ ಸಮೆಸಲು ಬಂದವನಲ್ಲ. ನಾವು ಬರದೇ ಇರುವುದಕ್ಕೆ ಆಗುವುದಿಲ್ಲ. ನಮ್ಮ ವಾಸನೆಗಳೇ ಪ್ರಪಂಚಕ್ಕೆ ನಮ್ಮನ್ನು ಎಳೆದುಕೊಂಡು ಬರುವುವು. ಆದರೆ ಪೂರ್ಣಾತ್ಮನಾದ ಭಗವಂತನಿಗೆ ಕಾಮನೆಗಳಾವುವೂ ಇಲ್ಲ. ಮಾನವಕೋಟಿಯ ಮೇಲಿನ ಅನುಕಂಪೆಯಿಂದ ಅವನು ಬರುವನು. ಇಲ್ಲಿ ಬಂದಾದ ಮೇಲೂ, ಮಾನವಾಕಾರವನ್ನು ಧರಿಸಿದ ಮೇಲೆಯೂ, ಅವನು ತನ್ನ ದೈವತ್ವವನ್ನು ಕ್ಷಣ ಕಾಲವೂ ಮರೆತಿಲ್ಲ. ಇದನ್ನು ಮೂಢರು ಅರಿಯರು. ಅವತಾರದ ಹಿಂದೆ ಇರುವ ದೈವತ್ವವನ್ನು ಅರಿಯುವುದು ಎಲ್ಲರಿಗೂ ಸಾಧ್ಯವಿಲ್ಲ. ಅದನ್ನು ಅರಿಯಬೇಕಾದರೆ ಭಗವದಂಶ ನಮ್ಮಲ್ಲಿಯೇ ಜಾಗ್ರತವಾಗಿರಬೇಕು. ಆಗ ಮಾತ್ರ ಸಾಧ್ಯ. ಇಲ್ಲದೇ ಇದ್ದರೆ ನಾವು ಹೇಗೆ ಇರುವೆವೊ ಹಾಗೆಯೆ ಇನ್ನೊಬ್ಬರನ್ನೂ ಅಳೆಯುವೆವು. ಅವತಾರವನ್ನೂ ನಾವು ಅಳೆಯುವುದು, ನಮ್ಮ ಸೇರು ಪಾವು ಚಟಾಕಿನ ಮೂಲಕ! ಶ‍್ರೀರಾಮ ಹುಟ್ಟಿ ಕೆಲಸ ಮಾಡಿದ ಕಾಲದಲ್ಲಿ\break ಅವನನ್ನು ಅವತಾರವೆಂದು ತಿಳಿದವರು ಎಲ್ಲೊ ಅಲ್ಪ ಮಂದಿ. ಶ‍್ರೀಕೃಷ್ಣ ಜನ್ಮಧಾರಣೆ ಮಾಡಿ ಕರ್ಮದಲ್ಲಿ ನಿರತನಾದಾಗ, ನಮ್ಮಂತೆಯೇ ಅವನು ಟೀಕೆ ನಿಂದೆಗಳನ್ನು ಕೇಳಬೇಕಾಗಿ ಬಂತು. ಆದರೆ ಈಗ ಜನಸಾಧಾರಣರೆಲ್ಲ ಅವನನ್ನು ಅವತಾರವೆಂದು ಒಪ್ಪುತ್ತಾರೆ. ಏಕೆಂದರೆ ಅದು ಸ್ಥಾಪಿತವಾಗಿ ಹೋಗಿದೆ. ಒಂದು ವಸ್ತು ಸ್ಥಾಪಿತವಾಗಿ ಹೋದಮೇಲೆ ಅದನ್ನು ನಂಬುವುದು ಸುಲಭ. ಆದರೆ ಅದಕ್ಕಿಂತ ಮುಂಚೆಯೇ ಅದರಲ್ಲಿರುವ ದೈವತ್ವಕ್ಕೆ ಮಣಿಯಬೇಕಾದರೆ ಆ ಜೀವಿ ನಿಜವಾಗಿಯೂ ಆಧ್ಯಾತ್ಮಿಕ ಜೀವನದಲ್ಲಿ ಬಹಳ ಮುಂದುವರಿದು ಹೋಗಿರಬೇಕು.

\begin{shloka}
ನಾಹಂ ಪ್ರಕಾಶಃ ಸರ್ವಸ್ಯ ಯೋಗಮಾಯಾಸಮಾವೃತಃ~।\\ಮೂಢೋಽಯಂ ನಾಭಿಜಾನಾತಿ ಲೋಕೋ\\ ಮಾಮಜಮವ್ಯಯಮ್ \hfill॥ ೨೫~॥
\end{shloka}

\begin{artha}
ಯೋಗಮಾಯೆಯಿಂದ ಆವೃತನಾದ ನಾನು ಎಲ್ಲರಿಗೂ ಕಾಣುವುದಿಲ್ಲ. ಈ ಮೂಢ ಜನ, ಅಜನೂ ಅವ್ಯಯನೂ ಆದ ನನ್ನನ್ನು ಅರಿಯರು.
\end{artha}

ದೇವರು ಧರೆಗೆ ಇಳಿದು ಬರುವಾಗ ಯೋಗಮಾಯೆಯಿಂದ ಆವೃತನಾಗಿ ಬರುತ್ತಾನೆ. ಜನರಿಗೆ ಅದರ ಹಿಂದೆ ಇರುವವನು ಕಾಣುವುದಿಲ್ಲ. ಆದರೆ ಅದರ ಹಿಂದೆ ಇರುವವರಿಗೆ ಎಲ್ಲವೂ ಕಾಣುವುದು. ಹೊರಗೆ ಇರುವವರಿಗೆ ತೆರೆಯ ಹಿಂದಿರುವ ವ್ಯಕ್ತಿ ಕಾಣುವುದಿಲ್ಲ. ಯೋಗಮಾಯೆ ಭಗವಂತನನ್ನು ಸರ್ವಸಾಮಾನ್ಯನಂತೆ ಕಾಣುವಂತೆ ಮಾಡುವುದು. ಆದರೆ ದೇವರು ತನ್ನ ದೇವರತನವನ್ನು ಎಂದಿಗೂ ಮರೆತಿರುವುದಿಲ್ಲ. ಹೊರಗೆ ಆರೋಪ ಮಾಡಿಕೊಂಡು ವೇಷ ಹಾಕಿಕೊಂಡವನು, ತನ್ನ ಪಾತ್ರಕ್ಕೆ ತಕ್ಕಂತೆ ಅಭಿನಯಿಸುತ್ತಿದ್ದರೂ, ತಾನು ಯಾರು ಎಂಬುದನ್ನು ಕ್ಷಣಕಾಲವೂ ಮರೆತಿರುವುದಿಲ್ಲ. ಅವನು ಈ ವೇಷವನ್ನು ಯಾವಾಗ ಬೇಕಾದರೂ ಕಳಚಿ ಇಡಬಹುದು. ಆದರೆ ಜೀವಿಗಳಿಗಾದರೊ ಅಂಟಿಕೊಂಡಿರುವುದು ಕರ್ಮಮಾಯೆ. ನಾವು ಅದನ್ನು ಕಳಚಿ ಇಡುವುದಕ್ಕಾಗುವುದಿಲ್ಲ ಮತ್ತು ವೇಷ ಹಾಕಿಕೊಂಡೊಡನೆಯೇ ಹಿಂದಿನದೆಲ್ಲ ಮರೆತುಹೋಗುವುದು. ಇಲ್ಲಿ ಇಬ್ಬರೂ ವೇಷ ಹಾಕಿಕೊಂಡು ಬಂದಿರುವರು. ದೇವರು ವೇಷ ಹಾಕಿಕೊಂಡರೆ ತಾನು ನಿಜವಾಗಿ ಯಾರು ಎಂಬುದನ್ನು ಮರೆಯುವುದಿಲ್ಲ. ಜೀವಿಯೂ ತನ್ನ ನೈಜಸ್ವಭಾವವನ್ನು ಮರೆತು, ದೇಹ ಬುದ್ಧಿ ಇಂದ್ರಿಯಗಳ ಉಪಾಧಿಯನ್ನು ತೆಗೆದುಕೊಂಡು ಅದರಿಂದ ಅಲಂಕರಿಸಿಕೊಂಡು ವೇಷ ಹಾಕಿಕೊಳ್ಳುವನು. ಹಾಕಿದ ವೇಷದಲ್ಲೆ ಅವನು ತನ್ಮಯನಾಗು\-ವನು. ಹಿಂದಿನದೆಲ್ಲ ಅವನಿಗೆ ಮರೆತೇ ಹೋಗುವುದು. ವೇಷವನ್ನು ಇಚ್ಛಿಸಿದಾಗ ತೆಗೆದುಹಾಕಲೂ ಆಗುವುದಿಲ್ಲ. ಅದೇ ಭಗವಂತನ ವೇಷಕ್ಕೂ ನಮ್ಮ ವೇಷಕ್ಕೂ ಇರುವ ವ್ಯತ್ಯಾಸ.

ಯೋಗಮಾಯೆಯಲ್ಲಿ ದೇವರು ಎಲ್ಲರಿಗೂ ವ್ಯಕ್ತನಲ್ಲ ಎಂದು ಹೇಳುವನು. ಯಾರಿಗೂ ಕಾಣುವುದಿಲ್ಲ ಎಂದಲ್ಲ. ಅವನನ್ನು ಗುರುತು ಹಿಡಿಯುವವರು ಬಹಳ ಅಲ್ಪ ಮಂದಿ. ವಜ್ರಪಡಿ ವ್ಯಾಪಾರಿ ಮಾತ್ರ ವಜ್ರ ಯಾವುದು ಗಾಜು ಯಾವುದು ಎಂಬುದನ್ನು ಕಂಡುಹಿಡಿಯಬಲ್ಲ. ಬದನೆಕಾಯಿ ವ್ಯಾಪಾರಿ ಹೇಗೆ ಅದನ್ನು ಕಂಡುಹಿಡಿಯಬಲ್ಲ ಎನ್ನುವರು ಶ‍್ರೀರಾಮಕೃಷ್ಣರು. ಈ ಜೀವನದಲ್ಲಿ ಯಾವುದನ್ನು ಅರಿತುಕೊಳ್ಳಬೇಕಾದರೂ ಯೋಗ್ಯತೆಯನ್ನು ಸಂಪಾದಿಸಿಕೊಂಡಿರಬೇಕು. ವೈದ್ಯನಿಗೆ ಮಾತ್ರ ರೋಗಿಯ ನಾಡಿ ಗೊತ್ತಾಗುವುದು. ಸಾಧಾರಣ ಮನುಷ್ಯನಿಗೆ ಗೊತ್ತಾಗುವುದಿಲ್ಲ. ಹಾಗೆಯೆ ಭಗವತ್ ತತ್ತ್ವವನ್ನು ತಿಳಿದುಕೊಳ್ಳುವುದಕ್ಕೆ ಒಬ್ಬ ಸಾಧನೆ ಮಾಡಿರಬೇಕು. ಸಾಧನೆ ಮಾಡಿ ಜೀವನದಲ್ಲಿ ಬಹಳ ಮುಂದುವರಿದಿದ್ದರೆ ಮಾತ್ರ ಇದು ಗೊತ್ತಾಗುವುದು. ಆದರೆ ಈ ಲೋಕದಲ್ಲಿ ಬಹುಪಾಲು ಜನರು ವ್ಯವಹಾರದಲ್ಲೆ ಮುಳುಗಿದವರು. ಅವರಿಗೆ ಲೌಕಿಕ ವಸ್ತುಗಳೇ ಸತ್ಯ. ಅವನ್ನು ಮೀರಿದ ಒಂದು ಅತೀಂದ್ರಿಯ ಅನುಭವ ಸಾಧ್ಯ ಎಂಬುದನ್ನು ಕಂಡು ಅರಿಯರು. ಕಣ್ಣೆದುರಿಗೆ ಹುಟ್ಟಿ ಸಾಯುವುದು ಮಾತ್ರ ಸತ್ಯ ಅವರ ಪಾಲಿಗೆ. ಹುಟ್ಟಿದವರ ಹಿಂದೆ ಯಾವುದಿದೆ, ಸಾಯುವುದರ ಮುಂದೆ ಯಾವುದಿದೆ ಎಂಬ ಪ್ರಶ್ನೆಯನ್ನು ಹಾಕುವುದಿಲ್ಲ. ಭಗವಂತನಾದರೊ ಜನನ ಮರಣ ಬದಲಾವಣೆಗಳಿಗೆ ಅತೀತ. ಅವನನ್ನು ಜನಸಾಧಾರಣರು ತಿಳಿದುಕೊಳ್ಳಲಾರರು.

\begin{shloka}
ವೇದಾಹಂ ಸಮತೀತಾನಿ ವರ್ತಮಾನಾನಿ ಚಾರ್ಜುನ~।\\ಭವಿಷ್ಯಾಣಿ ಚ ಭೂತಾನಿ ಮಾಂ ತು ವೇದ ನ ಕಶ್ಚನ \hfill॥ ೨೬~॥
\end{shloka}

\begin{artha}
ಅರ್ಜುನ, ಹಿಂದೆ ಆಗಿಹೋದ, ಈಗಿರುವ, ಮುಂದೆ ಆಗುವ ಸಕಲ ಭೂತಗಳನ್ನೂ ನಾನು ಬಲ್ಲೆ. ಆದರೆ ನನ್ನನ್ನು ಯಾರೂ ಅರಿತಿಲ್ಲ.
\end{artha}

ಪರಮೇಶ್ವರನಿಗೂ ಜೀವರಾಶಿಗಳಿಗೂ ಇರುವ ವ್ಯತ್ಯಾಸವನ್ನು ಇಲ್ಲಿ ಶ‍್ರೀಕೃಷ್ಣ ಹೇಳುತ್ತಾನೆ. ಭಗವಂತನಿಗೆ ಪ್ರತಿಯೊಂದು ಪ್ರಾಣಿಯು ಹಿಂದೆ ಏನಾಗಿತ್ತು ಎಂಬುದು ಗೊತ್ತಿದೆ. ಮುಂದೆ ಏನಾಗುವುದು ಎಂಬುದು ಗೊತ್ತಿದೆ. ಈಗ ಏನಾಗಿದೆ ಎಂಬುದು ಗೊತ್ತಿದೆ. ಏಕೆಂದರೆ ಅವನು ಸೃಷ್ಟಿಸಿದವನು, ಪಾಲಿಸುತ್ತಿರುವವನು, ಕೊನೆಗೆ ಸೆಳೆದುಕೊಳ್ಳುವವನು.. ಅವನು ಈ ಸೃಷ್ಟಿಯ ನಾಟಕವನ್ನು ಎಷ್ಟು ವೇಳೆ ಸೂತ್ರಧಾರನಂತೆ ಆಡಿಸಿರುವನು! ಆಟವನ್ನು ಆಡಿಸುವವನಿಗೆ ಯಾರು ಯಾರು ಪಾತ್ರಧಾರಿಗಳು, ಅವರು ಏನು ಏನು ಮಾಡಬೇಕಾಗಿದೆ ಮತ್ತುಅದು ಎತ್ತ ಸಾಗುವುದು ಮತ್ತು ಅದು ಹೇಗೆ ಪರ್ಯವಸಾನವಾಗುವುದು ಎಂಬುದು ಗೊತ್ತಿದೆ. ಅದನ್ನೇ ಕುಳಿತು ನೋಡುವವನಿಗೆ ವರ್ತಮಾನ, ಇನ್ನೊಬ್ಬನಿಗೆ ಭವಿಷ್ಯವಾಗಿರಬಹುದು. ಹಲವು ಜ್ಯೋತಿವತ್ಸರ\-ಗಳಾಚೆ ಇರುವ ಒಂದು ನಕ್ಷತ್ರ ಲೋಕದಲ್ಲಿ ಒಂದು ಘಟನೆ ನಡೆಯುವುದು ಎಂದು ಇಟ್ಟು\-ಕೊಳ್ಳೋಣ. ಅದು ಒಂದು ವರುಷದ ಹಿಂದೆ ಆಯಿತು. ಅವರಿಗೆ ಈ ಘಟನೆ ಭೂತಕಾಲದ್ದು. ಆ ಘಟನೆ ಬೆಳಕಿನ ವೇಗದಲ್ಲಿ ಚಲಿಸುತ್ತ ಸುಮಾರು ಒಂದು ಜ್ಯೋತಿವತ್ಸರದ ದೂರದಲ್ಲಿರುವ ಈ ಗ್ರಹಕ್ಕೆ ಈಗ ಮುಟ್ಟುತ್ತದೆ. ಅದನ್ನು ನೋಡುವವರಿಗೆ ಅದು ವರ್ತಮಾನ. ನಮ್ಮಿಂದ ಇನ್ನೂ ದೂರದಲ್ಲಿರುವವರಿಗೆ ಆ ಘಟನೆ ಇನ್ನೂ ಆಗಿಲ್ಲ ಎಂದರೆ ಭವಿಷ್ಯವಾಯಿತು. ಕಾಲದಲ್ಲಿರುವವರು ಯಾವುದೋ ಒಂದು ಸ್ಥಳಕ್ಕೆತಮ್ಮನ್ನು ಬಿಗಿದುಕೊಂಡು, ಇದು ಭೂತ, ವರ್ತಮಾನ, ಭವಿಷ್ಯ ಎಂದು ಹೇಳುತ್ತಾರೆ. ಹಿಂದಿರುವ ಸೂತ್ರಧಾರನ ಪರಿಚಯ ಆಗುವುದಿಲ್ಲ. ನಾವು ಕಾಲದಲ್ಲಿ ಹಿಂದೆ ಈಗ ಮುಂದೆ ಎಂಬ ಭಾಗಗಳನ್ನು ನೋಡುತ್ತೇವೆ. ಇದು ಕೃತಕ. ಒಂದೇ ಒಂದು ಘಟನೆ ಒಬ್ಬನಿಗೆ ಭೂತ, ಮತ್ತೊಬ್ಬನಿಗೆ ವರ್ತಮಾನ, ಇನ್ನೊಬ್ಬನಿಗೆ ಭವಿಷ್ಯವಾಗಿರಬಹುದು. ಹಲವು ಜ್ಯೋತಿವತ್ಸರಗಳಾಚೆ ಇರುವ ಒಂದು ನಕ್ಷತ್ರಲೋಕದಲ್ಲಿ ಒಂದು ಘಟನೆ ನಡೆಯುವುದು ಎಂದು ಇಟ್ಟುಕೊಳ್ಳೋಣ. ಅದು ಒಂದು ವರುಷದ ಹಿಂದೆ ಆಯಿತು. ಅವರಿಗೆ ಈ ಘಟನೆ ಭೂತಕಾಲದ್ದು. ಆ ಘಟನೆ ಬೆಳಕಿನ ವೇಗದಲ್ಲಿ ಚಲಿಸುತ್ತ ಸುಮಾರು ಒಂದು ಜ್ಯೋತಿವತ್ಸರದ ದೂರದಲ್ಲಿರುವ ಈ ಗ್ರಹಕ್ಕೆ ಈಗ ಮುಟ್ಟುತ್ತದೆ. ಅದನ್ನು ನೋಡುವವರಿಗೆ ಅದು ವರ್ತಮಾನ. ನಮ್ಮಿಂದ ಇನ್ನೂ ದೂರದಲ್ಲಿರುವವರಿಗೆ ಆ ಘಟನೆ ಇನ್ನೂ ಆಗಿಲ್ಲ ಎಂದರೆ ಭವಿಷ್ಯವಾಯಿತು. ಕಾಲದಲ್ಲಿರುವವರು ಯಾವುದೋ ಒಂದು ಸ್ಥಳಕ್ಕೆ ತಮ್ಮನ್ನು ಬಿಗಿದುಕೊಂಡು, ಇದು ಭೂತ, ವರ್ತಮಾನ, ಭವಿಷ್ಯ ಎಂದು ಹೇಳುತ್ತಾರೆ. ಆದರೆ ಯಾರು ದೇಶ ಕಾಲಾತೀತನೊ ಅವನು ಎಲ್ಲವನ್ನೂ ಏಕಕಾಲದಲ್ಲಿ ನೋಡುವನು, ಈಗಿರುವ ಪ್ರಾಣಿಯ ಹಿಂದೆ ಮುಂದೆ ಎಲ್ಲವೂ ವೇದ್ಯವಾಗುವುದು. ಭಗವಂತ ದೇಶಕಾಲನಿಮಿತ್ತಕ್ಕೆ ಅತೀತನಾಗಿರುವುದರಿಂದ ಅವನಿಗೆ ಎಲ್ಲವನ್ನೂ ತಿಳಿದುಕೊಳ್ಳುವುದಕ್ಕೆ ಸಾಧ್ಯ. ಆದರೆ ಸಾಧಾರಣ ಜೀವರಾಶಿಗಳಿಗಾದರೋ, ಈಗ ಆಗುತ್ತಿರುವುದೊಂದೇ ಗೊತ್ತಿರುವುದು. ಹಿಂದಿನ ಜನ್ಮದಲ್ಲಿ ನಾವು ಏನೇನು ಮಾಡಿದೆವು, ಎಲ್ಲೆಲ್ಲಿ ಹುಟ್ಟಿದೆವು, ಬೆಳೆದೆವು, ಅನುಭವಿಸಿದೆವು ಎಂಬುದು ಯಾವುದೂ ಗೊತ್ತಿಲ್ಲ. ಎಲ್ಲಾ ಮರೆವಿನಲ್ಲಿ ಮುಳುಗಿಹೋಗಿದೆ. ನಮ್ಮ ಭವಿಷ್ಯವನ್ನು ನಾವು ಇನ್ನೂ ರೂಪಿಸಿಕೊಳ್ಳ ಬೇಕಾಗಿರುವುದರಿಂದ ಅದು ಹೇಗಿದೆ ಎಂಬುದನ್ನು ಹೇಗೆ ಊಹಿಸುವುದಕ್ಕೆ ಸಾಧ್ಯ ಎನ್ನುವೆವು ನಾವು. ಅಜ್ಞಾನಿಗೆ ಮುಂದಿನದು ಕಾಣದು. ಆದರೆ ಭಗವಂತನಿಗೆ ಹೊಸದೇನೂ ಅಲ್ಲ. ಅವನು ಯಾವ ಮಾರ್ಗದಲ್ಲಿ ನಡೆಯುತ್ತಾನೆ ಹೇಗೆ ಮಾಡುತ್ತಾನೆ ಎಂಬುದೆಲ್ಲ ಆಗಲೆ ನಿರ್ಧಾರವಾಗಿದೆ. ಪ್ರೊಜೆಕ್ಟರಿನ ಮೂಲಕ ಬೀಳುವ ಫಿಲಂ ರೋಲಿನಲ್ಲಿ ಆಗಲೇ ಕಥೆ ತಯಾರಾಗಿ ಜೋಡಿಸಿದೆ. ಜೋಡಿಸಿದ್ದೇ ತೆರೆಯ ಮೇಲೆ ಬೀಳುವುದು. ಹೊಸದೇನು ಇಲ್ಲ. ದೇವರು ಸೃಷ್ಟಿಯ ಸಿನಿಮಾದ ಡೈರೆಕ್ಟರ್. ಈಗ ನಾವು ಏನನ್ನು ನೋಡುತ್ತಿರುವೆವೋ ಅದು ಆಗಲೆ ರೋಲಿನಲ್ಲಿ ಫಿಲಂನಿಂದ ಬೀಳುತ್ತಿದೆ. ಮುಂದೇನಾಗುವುದು ಎಂಬುದು ಅದನ್ನು ಜೋಡಿಸಿದ ಡೈರೆಕ್ಟರಿಗೆ ಮೊದಲೇ ವೇದ್ಯ.

ದೇವರಿಗೆ ನಮ್ಮದೆಲ್ಲ ಗೊತ್ತು. ನಮ್ಮ ಕಥೆಯನ್ನೆಲ್ಲ ಬರೆದವನು ಅವನು. ಅವನು ಬರೆದಂತೆ ಬಾಳುತ್ತಿರುವೆವು ಈಗ. ಆದರೆ ಜೀವರಾಶಿಗಳಿಗಾದರೊ ಈಗ ತೆರೆಯಮೇಲೆ ಬೀಳುತ್ತಿರುವು\-ದೊಂದೇ ಕಾಣುವುದು. ಅದನ್ನು ಮೀರಿ ನೋಡಲಾರರು ಅವರು. ಆ ಶಕ್ತಿಯನ್ನು ಇನ್ನೂ ಸಂಪಾದಿಸಿಕೊಂಡಿಲ್ಲ. ಸಂಪಾದಿಸಿಕೊಂಡರೆ ನೋಡಬಹುದು. ಯಾವ ಯೋಗಿಗೆ ಜಾತಿಸ್ಮರಣೆ ಉಂಟಾಗುವುದೊ ಆತ ತನ್ನ ಹಿಂದಿನದನ್ನು ತಿಳಿದುಕೊಳ್ಳಬಲ್ಲ. ಆದರೆ ದೇವರನ್ನು ತಿಳಿದುಕೊಳ್ಳ\-ಬೇಕಾದರೆ ಇದನ್ನೂ ಮೀರಿದ ಸಾಧನೆಯನ್ನು ಮಾಡಬೇಕು.

\begin{shloka}
ಇಚ್ಛಾದ್ವೇಷಸಮುತ್ಥೇನ ದ್ವಂದ್ವಮೋಹೇನ ಭಾರತ~।\\ಸರ್ವಭೂತಾನಿ ಸಂಮೋಹಂ ಸರ್ಗೇ ಯಾಂತಿ ಪರಂತಪ \hfill॥ ೨೭~॥
\end{shloka}

\begin{artha}
ಅರ್ಜುನ, ಇಚ್ಛಾ ದ್ವೇಷಗಳಿಂದ ಹುಟ್ಟಿದ ದ್ವಂದ್ವಗಳ ಮೋಹದಿಂದ ಸರ್ವ ಪ್ರಾಣಿಗಳು ಹುಟ್ಟುವ ಕಾಲದಲ್ಲಿ ಭ್ರಾಂತಿಯನ್ನು ಹೊಂದುತ್ತವೆ.
\end{artha}

ಹುಟ್ಟುವಾಗಲೇ ನಮ್ಮನ್ನು ಭ್ರಾಂತಿ ಆವರಿಸುವುದು. ಆದಕಾರಣವೇ ಹಿಂದಿನದನ್ನು ಮುಂದಿನ\-ದನ್ನು ನಾವು ಅರಿಯಲಾರೆವು. ಕನಸು ಕಾಣುವುದಕ್ಕೆ ಮುಂಚೆ ಜಾಗ್ರತಾವಸ್ಥೆಯಲ್ಲಿರುವಾಗ ನಾವು ಯಾರು ಎಲ್ಲಿರುವೆವು ಎಂಬುದೆಲ್ಲ ಜ್ಞಾಪಕವಿರುವುದು. ಕನಸಿಗೆ ಬೀಳುತ್ತಲೇ ನೈಜಸ್ಥಿತಿ ಮರೆಯುವುದು. ತಾತ್ಕಾಲಿಕ ಕನಸೇ ಸರ್ವ ಸತ್ಯವಾಗಿ ಕಾಣುವುದು. ಹಾಗೆಯೆ ಜೀವರಾಶಿ ಪ್ರಪಂಚಕ್ಕೆ ಬಂದೊಡನೆಯೆ, ಹುಟ್ಟುವುದಕ್ಕೆ ಮುಂಚೆ, ಈ ದೇಹವನ್ನು ತ್ಯಜಿಸಿ ಆದಮೇಲೆ ನಮ್ಮ ನೈಜಸ್ಥಿತಿ ಯಾವುದು ಎಂಬುದನ್ನು ಮರೆತು, ತಾತ್ಕಾಲಿಕ ಸ್ಥೂಲದೇಹ ಧಾರಣೆಮಾಡಿದ ಈಗಿನ ಸ್ಥಿತಿಯೇ ತಮ್ಮ ಸರ್ವಸ್ವ ಎಂದು ಭಾವಿಸುತ್ತವೆ.

ನಾವು ಭ್ರಾಂತಿಯಲ್ಲಿ ಬೀಳುವುದಕ್ಕೆ ಕಾರಣವನ್ನು ಕೊಡುತ್ತಾನೆ. ನಾವು ಬರುವಾಗಲೇ ಹಲವು ಸಂಸ್ಕಾರಗಳನ್ನು ತರುತ್ತೇವೆ. ಅದಕ್ಕೆ ತಕ್ಕಂತೆ ನಾವು ಕೆಲಸಗಳನ್ನು ಮಾಡುತ್ತೇವೆ. ಕೆಲಸಕ್ಕೆ ಫಲವನ್ನು ಅನುಭವಿಸುತ್ತೇವೆ. ಕೆಲವನ್ನು ಕಂಡರೆ ಇಷ್ಟ, ಮತ್ತೆ ಕೆಲವನ್ನು ಕಂಡರೆ ನಮಗೆ ಅನಿಷ್ಟ. ನಮಗೆ ಸುಖ ಬೇಕು, ಕಷ್ಟ ಬೇಡ. ಇಂದ್ರಿಯ ಸುಖ ಬೇಕು, ಆಧ್ಯಾತ್ಮಿಕ ಸುಖ ಬೇಡ. ಈ ಭಾವನೆಗಳಿಂದಲೇ ನಾವು ಹುಟ್ಟುವಾಗ, ಮೊದಲು ಮನಸ್ಸು ಖಾಲಿಯಾಗಿದ್ದು, ಅನಂತರ ನಮ್ಮ ತಂದೆತಾಯಿಗಳಿಂದ ಅಥವಾ ವಾತಾವರಣದ ಮೂಲಕ ನಾವು ಹೀಗೆ ಆದೆವು ಎಂದು ಹೇಳುವುದಕ್ಕೆ ಆಗುವುದಿಲ್ಲ. ಇದು ನಾವೇ ತಂದದ್ದು. ಇನ್ನೊಬ್ಬರು ನಮ್ಮ ಮೇಲೆ ಹೇರಿದ್ದಲ್ಲ. ತಂದೆತಾಯಿಗಳಿಂದ ನಾವು ಈ ಸ್ವಭಾವವನ್ನು ಪಡೆಯುವ ಹಾಗಿದ್ದರೆ, ಸ್ವಭಾವದಲ್ಲಿ ಗುಣದಲ್ಲಿ ನಾವು ಅವರಂತೆ ಆಗಬೇಕಾಗಿತ್ತು. ಆದರೆ ನಮ್ಮ ನಿತ್ಯ ಜೀವನದಲ್ಲಿ ನಮ್ಮ ಅಪ್ಪ ಅಮ್ಮಂದಿರ ಸ್ವಭಾವವೇ ಒಂದು, ನಾವಾಗಿರುವುದು ಮತ್ತೊಂದು. ಒಳ್ಳೆಯ ತಂದೆತಾಯಿಗಳಿಗೆ ಕೆಟ್ಟ ಮಕ್ಕಳು ಹುಟ್ಟುತ್ತಾರೆ. ಕೆಟ್ಟ ತಂದೆತಾಯಿಯರಿಗೆ ಒಳ್ಳೆಯ ಮಕ್ಕಳು ಕೆಲವು ವೇಳೆ ಹುಟ್ಟುತ್ತಾರೆ. ಇದಕ್ಕೆ ಕಾರಣವೇನು? ಏನೋ ಅಕಸ್ಮಾತ್ ಎಂದು ಹೇಳಬಹುದು. ಆದರೆ ಜೀವನದಲ್ಲಿ ಅಕಸ್ಮಾತ್ ಒಂದು ಕಾರಣವಲ್ಲ. ನಮಗೆ ಕಾರಣ ಗೊತ್ತಿಲ್ಲ ಎಂಬ ಅಜ್ಞಾನವನ್ನು ಮುಚ್ಚುವುದಕ್ಕೆ ನಾವು ಉಪಯೋಗಿಸುವ ಉಪಾಯದ ಪದ. ಇನ್ನೊಂದು ಪಂಗಡದವರು ಇರುವರು. ಅವರು ಹೊರಗಿನ ವಾತಾವರಣದ ಮೇಲೆಯೇ ಎಲ್ಲಾ ಭಾರವನ್ನು ಹೇರುವರು. \enginline{Environment }ಎಂದರೆ ಸನ್ನಿವೇಶ ಹೊಸದಾಗಿ ಏನನ್ನೂ ಮಾಡಲಾರದು. ಆಗಲೇ ಜೀವಿಯಲ್ಲಿ ಏನಿದೆಯೊ ಅದನ್ನು ಹೊರಸೆಳೆಯುವುದನ್ನು ಮಾತ್ರ ಮಾಡುವುದು. ಒಂದು ಬೀಜ ತಾನು ಯಾವ ಗಿಡವಾಗಬೇಕು ಎಂಬುದು ಆಗಲೆ ನಿರ್ಧಾರಿತವಾಗಿದೆ. ಹೊರಗಿನ ನೆಲ ಗಾಳಿ ಬೆಳಕು ನೀರು ಗೊಬ್ಬರ ಇವೆಲ್ಲ ಬೀಜದಲ್ಲಿರುವುದನ್ನು ಬೇಗ ಹುಲುಸಾಗಿ ಹೊರಗೆ ಬರುವಂತೆ ಮಾಡ ಬಹುದೇ ಹೊರತು, ಬೀಜವನ್ನೆ ಬದಲಾಯಿಸುವುದಕ್ಕೆ ಆಗುವುದಿಲ್ಲ. ಜೀವರಾಶಿಗಳು ಹುಟ್ಟುವಾಗಲೇ ಪ್ರಪಂಚವನ್ನು ಅನುಭವಿಸಬೇಕು ಎಂಬ ಸಂಸ್ಕಾರದೊಡನೆ ಹುಟ್ಟುವರು. ಅದರಂತೆ ಜೀವನದ ಘಟನೆಗಳಿಗೆ ಅವರಲ್ಲಿ ಪ್ರತಿಕ್ರಿಯೆಯನ್ನು ನೋಡುವೆವು. ಆಕರ್ಷಣೆಗಳಿಂದ ಬೀಸಿ ಕರೆಯುತ್ತಿದೆ ವಿಷಯ ಪ್ರಪಂಚ. ಅದರೊಳಗೆ ಧುಮುಕುವನು. ಈ ಪ್ರಪಂಚದ ಹಿಂದೆ ಇರುವ ಸೂತ್ರಧಾರನನ್ನು ತಿಳಿದುಕೊಳ್ಳಬೇಕೆಂಬ ಆಸೆ ಜೀವಿಗಳ ಹೃದಯದಲ್ಲಿರುವುದು ಬಹಳ ಅಪರೂಪ. ಅವರನ್ನು ಒಂದು ವಿನಾಯತಿ \enginline{(exception)} ಎಂತಲೇ ಹೇಳಬಹುದು.

\begin{shloka}
ಯೇಷಾಂ ತ್ವಂತಗತಂ ಪಾಪಂ ಜನಾನಾಂ ಪುಣ್ಯಕರ್ಮಣಾಮ್~।\\ತೇ ದ್ವಂದ್ವಮೋಹನಿರ್ಮುಕ್ತಾ ಭಜಂತೇ ಮಾಂ ದೃಢವ್ರತಾಃ \hfill॥ ೨೮~॥
\end{shloka}

\begin{artha}
ಆದರೆ ಪುಣ್ಯಕರ್ಮಗಳನ್ನು ಮಾಡುವ ಯಾವ ಜನರ ಪಾಪವು ಕ್ಷೀಣವಾಗಿರುವುದೋ, ಅವರು ಸುಖದುಃಖಗಳೆಂಬ ದ್ವಂದ್ವಗಳ ಮೋಹದಿಂದ ಬಿಡುಗಡೆಯನ್ನು ಹೊಂದಿ ದೃಢವ್ರತರಾಗಿ ನನ್ನನ್ನು ಭಜಿಸುತ್ತಾರೆ.
\end{artha}

ಹುಟ್ಟುವಾಗಲೆ ದ್ವಂದ್ವಗಳಿಂದ ಮುತ್ತಲ್ಪಟ್ಟ ವ್ಯಕ್ತಿ ಪಾರಾಗುವುದಕ್ಕೆ ಒಂದು ಮಾರ್ಗವನ್ನೂ ಕೂಡ ತೋರುತ್ತಾನೆ. ಪರಿಸ್ಥಿತಿ ನಿರಾಶೆಯಿಂದಲೇ ಕೂಡಿಲ್ಲ. ಈ ಗಾಢಾಂಧಕಾರದಲ್ಲೂ ಬೆಳಕಿನ ಗತಿಯೊಂದಿದೆ. ನಾವು ನಮ್ಮಲ್ಲಿರುವ ಹೀನ ಸಂಸ್ಕಾರಗಳನ್ನು ಕಡಮೆ ಮಾಡಿಕೊಳ್ಳಬೇಕು. ಆಗಲೇ ಪುಣ್ಯ ವೃದ್ಧಿಯಾಗುವುದು; ಮನಸ್ಸು ದೇವರ ಕಡೆ ಹೋಗುವುದು. ನದಿಯ ಮೇಲೆ ತೇಲುತ್ತಿರುವ ದೋಣಿಯಲ್ಲಿ ತೂತು ಬಿದ್ದಿದೆ. ನೀರು ತುಂಬಿಕೊಳ್ಳುತ್ತಾ ಭಾರವಾಗಿ ಮುಳುಗುತ್ತಿದೆ. ನಾವು ಮೊದಲು ಯಾವುದಾದರೂ ಪಾತ್ರೆಯಿಂದ ನೀರನ್ನು ಖಾಲಿ ಮಾಡುತ್ತಾ ಹೋಗಬೇಕು. ಆಗ ದೋಣಿ ಕ್ರಮೇಣ ಹಗುರವಾಗಿ, ತೇಲುವುದು. ಅದರೊಳಗೆ ನೀರು ಬರುವ ವೇಗಕ್ಕಿಂತ ಬೇಗಬೇಗ ನೀರನ್ನು ಎತ್ತಿ ಹಾಕಿದರೆ ದೋಣಿಯನ್ನು ಸುರಕ್ಷಿತವಾಗಿ ಕರೆಯನ್ನು ಮುಟ್ಟಿಸಬಹುದು. ಹಾಗೆಯೇ ನಮ್ಮ ಹೀನ ಸಂಸ್ಕಾರದ ರಂಧ್ರದಿಂದ ಪಾಪ ನುಗ್ಗುತ್ತಿದೆ. ನಾವು ಆ ನುಗ್ಗಿದ ಪಾಪವನ್ನು ಪುಣ್ಯಕರ್ಮಗಳಿಂದ ಕ್ಷೀಣಮಾಡಿಕೊಳ್ಳಬೇಕು. ಒಳ್ಳೆಯ ಕೆಲಸ ನಮ್ಮ ಮನಸ್ಸನ್ನು ಶುದ್ಧ ಮಾಡುವುದು. ಮನಸ್ಸು ಶುದ್ಧವಾದಂತೆ ಅದು ದೇವರ ಕಡೆ ತಿರುಗುವುದು. ಸುಮ್ಮನೆ ಈಗಿರುವ ಪಾಪಗಳ ಭಾರಕ್ಕೆ ಸೊರಗಿ ನಿರಾಶರಾಗಿ ಬೀಳುವುದಲ್ಲ. ಪಾಪದ ಭಾರ ಯಾರಲ್ಲಿ ಇಲ್ಲ? ಎಲ್ಲರಲ್ಲಿಯೂ ಇದೆ. ಕೆಲವರಲ್ಲಿ ಕಡಮೆ ಇದೆ; ಮತ್ತೆ ಕೆಲವರಲ್ಲಿ ಜಾಸ್ತಿ ಇದೆ. ಯಾರಲ್ಲಿ ಈಗ ಕಡಮೆ ಇದೆಯೋ ಅವರು ಕೂಡ ಹಿಂದೆ ಜಾಸ್ತಿ ಇದ್ದವರೇ, ಒಳ್ಳೆಯ ಕರ್ಮಗಳನ್ನು ಮಾಡಿ ತಮ್ಮ ಮನಸ್ಸನ್ನು ಹಗುರ ಮಾಡಿಕೊಂಡವರು. ಪ್ರತಿಯೊಂದು ಉತ್ತಮ ಆಲೋಚನೆ ಮತ್ತು ಕೆಲಸವೂ ನಮ್ಮ ಜೀವನದಲ್ಲಿ ಉತ್ತಮ ಸಂಸ್ಕಾರಗಳನ್ನು ಬಿಡುವುದು. ಇವುಗಳ ಮೊತ್ತ ಕ್ರಮೇಣ ಹೆಚ್ಚಾಗುವುದು, ಹನಿಗೂಡಿದರೆ ಹಳ್ಳವಾಗುವಂತೆ.

ಈಗ ಉತ್ತಮ ಕೆಲಸಗಳನ್ನು ಮಾಡುವುದು, ಉತ್ತಮ ಆಲೋಚನೆಗಳನ್ನು ಮಾಡುವುದು, ಇದೊಂದೇ ದಾರಿ ನಾವು ದ್ವಂದ್ವಗಳ ಮೋಹದಿಂದ ಪಾರಾಗಬೇಕಾದರೆ. ನಮಗೆಲ್ಲ ಸಿಹಿ ಅನುಭವ ಬೇಕು, ಕಹಿ ಅನುಭವ ಬೇಡ–ಎಂಬ ಜಾಡ್ಯ ಮೆಟ್ಟಿದೆ. ಆದರೆ ಈ ಸಂಸಾರಕ್ಕೆ ಬಂದಮೇಲೆ ಬರೀ ಸಿಹಿ ಅನುಭವವೇ ಹೇಗೆ ಸಿಕ್ಕುವುದು? ಸಿಕ್ಕಿದ ಸಿಹಿ ಅನುಭವವನ್ನೇ ತಿಂದರೂ ಅದು ನಮ್ಮ ದೇಹಕ್ಕೆ ಕೆಟ್ಟದ್ದು. ಅದರಿಂದ ಪಾರಾಗಬೇಕಾದರೂ ಕಹಿ ಔಷಧಿಯನ್ನು ತಿನ್ನಲೇ ಬೇಕು. ಸಾಧಕ ಕ್ರಮೇಣ ಈ ದ್ವಂದ್ವಗಳ ಕಡೆ ಮನಸ್ಸನ್ನು ಹಾಕುವುದಿಲ್ಲ. ಅವನು ಹಾಕದೇ ಇದ್ದರೆ ಇವು ಬರುವುದಿಲ್ಲ ಎಂದಲ್ಲ. ಇವು ಬರುತ್ತವೆ ಹೋಗುತ್ತವೆ. ಆದರೆ ಅವನು ಇದನ್ನು ಉದಾಸೀನನಾಗಿ ನೋಡುತ್ತಾನೆ. ಇವುಗಳ ಮೇಲೆ ಮನಸ್ಸನ್ನು ಇಡುವುದಿಲ್ಲ. ನಮ್ಮ ಮಾನಸಿಕ ಶಕ್ತಿಯ ಬಹುಪಾಲು ಈ ಬೇಕು ಬೇಡ ಎಂಬ ಬಿಲದ ಮೂಲಕ ವ್ಯಯವಾಗುತ್ತಿದೆ. ಯಾವಾಗ ಇವುಗಳ ಕಡೆ ಅನಾಸಕ್ತರಾಗುತ್ತೇವೆಯೋ ಅಷ್ಟೊಂದು ಮಾನಸಿಕ ಶಕ್ತಿ ನಮ್ಮ ಕರಗತವಾಗುವುದು. ಅದನ್ನೆಲ್ಲ ನಾವು ದೇವರ ಕಡೆ ಹರಿಸಬಹುದು.

ದ್ವಂದ್ವಗಳಿಂದ ಪಾರಾದಮೇಲೆಯೇ ನಾವು ಭಗವಂತನನ್ನು ದೃಢವಾಗಿ ಭಜಿಸಬೇಕಾದರೆ. ಅದಕ್ಕಿಂತ ಮುಂಚೆ ನಾವು ಹೋಗುವುದು ಅನಿಷ್ಟ ನಿವೃತ್ತಿಗಾಗಿ, ಸುಖಪ್ರಾಪ್ತಿಗಾಗಿ. ಏನೋ ಕೆಲವು ಸುಖವಾದ ಅನುಭವಗಳು ಆದಾಗ ದೇವರು ನಮಗೆ ಒಳ್ಳೆಯದನ್ನು ಮಾಡಿದನೆಂದು ಅವನನ್ನು ಹೊಗಳಿ, ಕೆಲವು ಕಾಲದ ಮೇಲೆ ಬಂದ ಸುಖಾನುಭವದಲ್ಲಿ ತಲ್ಲೀನರಾಗುತ್ತೇವೆ. ದೇವರನ್ನು ಮರೆಯುತ್ತೇವೆ. ಪುನಃ ಅವನ ಜ್ಞಾಪಕ ಬರುವುದು, ಜೀವನದಲ್ಲಿ ದುಃಖದ ಸಿಡಿಲು ಬಡಿದಾಗಲೇ. ಆಗ ಗೊಣಗಾಡುತ್ತಾ, ಏತಕ್ಕೆ ಇವುಗಳನ್ನೆಲ್ಲಾ ಕೊಟ್ಟೆ? ದಯವಿಟ್ಟು ಅವುಗಳನ್ನು ನಿವಾರಿಸು ಎಂದು ಬೇಡುತ್ತೇವೆ. ಮುಂಚೆ ಅವನು ಯಾವುದೋ ಒಂದು ಲೌಕಿಕ ಉದ್ದೇಶದಿಂದ ಹೋಗುತ್ತಾನೆ. ಅದೂ ಪ್ರತಿದಿನ ಹೋಗುವುದಿಲ್ಲ. ಆವಶ್ಯಕತೆ ಬಿದ್ದಾಗ ಹೋಗುತ್ತಾನೆ. ಅವನಿಗೆ ದೇವರ ಮೇಲೆ ಇರುವ ಭಕ್ತಿಗೆ ಅಮಾವಾಸ್ಯೆ ಪೂರ್ಣಿಮೆಗಳಿವೆ. ಕೆಲವು ವೇಳೆ ಅದು ಕ್ಷಯಿಸುತ್ತಾ ಬರುವುದು. ಕೊನೆಗೆ ಅಮಾವಾಸ್ಯೆಯನ್ನು ಮುಟ್ಟುವುದು. ದೃಢವ್ರತಿ ಯಾರು ಎಂದರೆ ಇಂತಹ ಏರಿಳಿತಗಳು, ವೃದ್ಧಿಕ್ಷಯಗಳು ಇಲ್ಲದವನು. ಜೀವನದಲ್ಲಿ ಒಂದೇಸಮನಾಗಿ ದೇವರ ಕಡೆ ಅವನ ಮನಸ್ಸು ಹೋಗುತ್ತಿರುವುದು. ಅವನು ಭಗವಂತನನ್ನು ಭಜಿಸುತ್ತಾನೆ. ಪ್ರೀತಿಯಿಂದ ತುಂಬಿ ತುಳುಕಾಡುವ ಹೃದಯದಿಂದ ಭಗವಂತನನ್ನು ಚಿಂತಿಸುತ್ತಾನೆ. ಅವನಿಂದ ಲೌಕಿಕವಾದ ಏನನ್ನೂ ಆಶಿಸುವುದಿಲ್ಲ. ಎಂತಹ ಕಷ್ಟನಷ್ಟಗಳು ಬಂದರೂ ದೇವರನ್ನು ಮರೆಯದೆ, ಗೊಣಗಾಡದೆ ಅವನನ್ನೇ ಚಿಂತಿಸುತ್ತಿರುವನು.

\begin{shloka}
ಜರಾಮರಣಮೋಕ್ಷಾಯ ಮಾಮಾಶ್ರಿತ್ಯ ಯತಂತಿ ಯೇ~।\\ತೇ ಬ್ರಹ್ಮ ತದ್ವಿದುಃ ಕೃತ್ಸ್ನಮಧ್ಯಾತ್ಮಂ ಕರ್ಮ ಚಾಖಿಲಂ \hfill॥ ೨೯~॥
\end{shloka}

\begin{artha}
ನನ್ನ ಆಶ್ರಯವನ್ನು ಹೊಂದಿ ಜರಾಮರಣಗಳಿಂದ ಮುಕ್ತರಾಗಲು ಯಾರು ಪ್ರಯತ್ನಿಸುವರೊ ಅವರು ಪೂರ್ಣ ಬ್ರಹ್ಮವನ್ನೂ ಅಧ್ಯಾತ್ಮವನ್ನೂ, ಕರ್ಮವನ್ನೂ ತಿಳಿಯುತ್ತಾರೆ.
\end{artha}

ಅನುಗಾಲವೂ ಭಗವಂತನನ್ನು ಚಿಂತಿಸುವ ಭಕ್ತರು ಭಗವಂತನ ಆಶ್ರಯವನ್ನು ಪಡೆಯುತ್ತಾರೆ. ಜೀವನದಲ್ಲಿ ಅವನನ್ನು ಬಲವಾಗಿ ಹಿಡಿದುಕೊಂಡರೇನೇ ಈ ಪ್ರಪಂಚಕ್ಕೆ ನಾವು ಬರುವುದನ್ನು ತಪ್ಪಿಸಿಕೊಳ್ಳಬೇಕಾದರೆ. ಈ ಪ್ರಪಂಚ ಮುಪ್ಪು-ಸಾವುಗಳಿಂದ ಕೂಡಿದೆ. ಬರೀ ಜನನ ವೃದ್ಧಿ ಮಾತ್ರ ಅಲ್ಲ ಇರುವುದು. ಅದರ ಮುಂದಿನ ಭಾಗವೇ ಮುಪ್ಪು. ಇಂದ್ರಿಯಗಳು ಕುಗ್ಗುವುವು. ತಮ್ಮ ಪಟುತ್ವವನ್ನು ಕಳೆದುಕೊಳ್ಳುವುವು. ಹಲವಾರು ರೋಗರುಜಿನಗಳು ಧಾಳಿಯಿಡುವುವು. ನಮಗೆ ನಾವೇ ಭಾರವಾಗುತ್ತೇವೆ, ಇತರರಿಗೊಂದು ಭಾರವಾಗುತ್ತೇವೆ. ಜೊತೆಗೆ ಇದು ಯೌವನದಲ್ಲಿ ಬಿತ್ತಿದ ಬೆಳೆಯನ್ನು ಕೊಯ್ಯುವ ಸಮಯ. ಒಂದೊಂದು ಒಂದೊಂದು ವಿಧವಾಗಿದೆ. ಇವನು ತನ್ನೊಂದಿಗೆ ಬಾಳಿದ, ಬದುಕಿದ ಜನರ ಸಾವನ್ನು ನೋಡಬೇಕಾಗಿದೆ. ತಾನೇ ಸುತ್ತಲೂ ಅನುಭವಿಸಿದ ಸ್ಥಳದಿಂದ, ನೆಂಟರಿಷ್ಟರಿಂದ ಅಗಲಬೇಕಾಗಿದೆ. ಜೀವನಕ್ಕೆ ಎಷ್ಟು ಸಲ ನಾವು ಬಂದಿರುವೆವು, ಎಷ್ಟೊಂದು ಸುಖದುಃಖಗಳನ್ನು ಅನುಭವಿಸಿರುವೆವು. ಈ ದ್ವಂದ್ವಗಳ ಚಕ್ರದಲ್ಲಿ ಸಿಕ್ಕಿ ಜರ್ಝರಿತರಾಗಿರುವೆವು. ಈ ಸಂಸಾರವೆಂಬ ತೀಕ್ಷ್ಣ ಸೂರ್ಯನ ತಾಪದಲ್ಲಿ ಜನ್ಮಜನ್ಮಗಳು ಬೆಂದಿರುವೆವು. ಇದರಿಂದ ಪಾರಾಗುವುದಕ್ಕೆ ಭಗವಂತನ ಅಡಿದಾವರೆಗಳನ್ನು ಆಶಿಸುವುದು ಜೀವ. ಅವನನ್ನು ಬಿಗಿಯಾಗಿ ಹಿಡಿದುಕೊಳ್ಳುವುದಕ್ಕೆ, ಪುನಃ ಈ ಪ್ರಪಂಚದ ವ್ಯಾಮೋಹಕ್ಕೆ ತುತ್ತಾಗದೇ ಇರಲು ಅವನು ಪ್ರಯತ್ನಿಸುವನು. ಒಂದೇ ಸಲ ಇಚ್ಛೆ ಕೈಗೂಡುವುದಿಲ್ಲ. ಅವನಿಂದ ಕಿತ್ತುಕೊಂಡು ಹೋದ ಮನಸ್ಸನ್ನು ಪುನಃಪುನಃ ಹಿಡಿದು ತಂದು ಅವನ ಪಾದಪದ್ಮಗಳಲ್ಲಿ ಕಟ್ಟಿಹಾಕಬೇಕು. ಸಂಸಾರವನ್ನು ಮೇಯುತ್ತಿದ್ದ ದನಕ್ಕೆ ಭಗವಚ್ಚಿಂತನೆಯ, ಭಜನೆಯ ಮೇವಿನ ರುಚಿ ಅಭ್ಯಾಸವಾಗಬೇಕು. ಈ ಅಭ್ಯಾಸವನ್ನು ರೂಢಿಸಿಕೊಳ್ಳುವುದಕ್ಕಾಗಿ ಬಿಡದೆ ಯತ್ನಿಸಬೇಕು.

ಯಾವಾಗ ಭಗವಂತನಲ್ಲಿ ಸಂಪೂರ್ಣ ಶರಣಾಗುತ್ತೇವೆಯೊ, ಆಗ ಭಗವಂತನನ್ನು ಸಂಪೂರ್ಣ ತಿಳಿಯುವೆವು. ಅದಕ್ಕಿಂತ ಮುಂಚೆ ಅವನಿಂದ ದೂರದಲ್ಲಿ ನಿಂತುಕೊಂಡು ನಮ್ಮ ಊಹೆ, ವಿಚಾರಗಳ ಮೂಲಕ ಚಿತ್ರಿಸಿಕೊಳ್ಳುತ್ತಿದ್ದೆವು. ಅವನು ಹಾಗಿರಬಹುದು, ಹೀಗಿರಬಹುದು ಎಂದು ಕಲ್ಪಿಸಿಕೊಳ್ಳುತ್ತಿದ್ದೆವು. ಆದರೆ ಯಾವಾಗ ಅವನ ಆಶ್ರಯವನ್ನು ಪಡೆಯುತ್ತೇವೆಯೊ, ಅವನ ಹತ್ತಿರವೇ ಇರುತ್ತೇವೆಯೊ, ಆಗ ಪ್ರತ್ಯಕ್ಷವಾಗಿ ಅವನೇನು ಎಂಬುದನ್ನು ಅರಿಯುತ್ತೇವೆ. ಜೀವನದಲ್ಲಿ ಪ್ರತ್ಯಕ್ಷವೊಂದೇ ದೊಡ್ಡ ಪ್ರಮಾಣ. ಆಗ ಅವನ ಪಾಲಿಗೆ ಶಾಸ್ತ್ರಾದಿಗಳೆಲ್ಲ ಬರೀ ಕೈಮರವಾಗಿ ದ್ದುವು, ಅವು ಭಗವಂತನ ಕಡೆ ತೋರಿಸುತ್ತಿದ್ದುವು ಎಂದು ಗೊತ್ತಾಗುವುದು. ಅವನೇನೋ ಅದನ್ನು ನಾವು ಅನುಭವಿಸಬೇಕು. ಆ ಇಂದ್ರಿಯಾತೀತ ಅನುಭವ ನಮಗೆ ಮತ್ತಾವುದರಿಂದಲೂ ಸಿಕ್ಕುವು ದಿಲ್ಲ. ಈ ಸ್ಥಿತಿಗೆ ಬಂದಾಗ ಅವನು ದೇವರನ್ನು ರುಚಿ ನೋಡುತ್ತಾನೆ. ಅವನಲ್ಲೇ ಬಾಳುತ್ತಾನೆ. ಅವನಲ್ಲೇ ವ್ಯವಹರಿಸುತ್ತಾನೆ.

ಅವನು ಅಧ್ಯಾತ್ಮವನ್ನು ತಿಳಿಯುತ್ತಾನೆ. ತಾನಾರು, ತನ್ನ ನಿಜವಾದ ಸ್ವರೂಪವೇನು ಎಂಬು\-ದನ್ನು ಅರಿಯುತ್ತಾನೆ. ಆತ ಮುಂಚೆ ಸಂಸಾರಕ್ಕೆ ಬಂದು ಹೋಗುತ್ತಿದ್ದಾಗ, ತನ್ನ ಮೇಲೆ ಹಾಕಿಕೊಂಡ ಮಿಥ್ಯಾವೇಷದಿಂದ ತನ್ನನ್ನೇ ಮರೆತಿದ್ದ. ಸುಳ್ಳು ನಾನನ್ನು ನಾನು ಎಂದು ಭಾವಿಸಿದ್ದ. ನಿಜವಾದ ನಾನು ಅವನಿಗೆ ಪತ್ತೆಯೇ ಇರಲಿಲ್ಲ. ಭಗವಂತನನ್ನು ನೋಡಿದಮೇಲೆ, ತಾನಾರು, ಭಗವಂತನಿಗೂ ತನಗೂ ಏನು ಸಂಬಂಧವಿದೆ, ಎಂಬುದನ್ನು ಅರಿಯುತ್ತಾನೆ. ಇಲ್ಲಿ ನಾವು ದ್ವೈತವೋ, ವಿಶಿಷ್ಟಾ ದ್ವೈತವೋ, ಅದ್ವೈತವೋ, ಯಾವುದನ್ನೂ ಸರಿ ತಪ್ಪು ಎಂದು ಗುದ್ದಾಡಬೇಕಾಗಿಲ್ಲ. ಯಾರಿಗೆ ಯಾವ ದೃಷ್ಟಿಯ ಮೂಲಕ ಅವನನ್ನು ಅನುಭವಿಸಲು ಆಸೆಯೋ ಅದನ್ನು ಹಿಡಿಯಲಿ. ಕೆಲವರು, ನಾವು ಅನುಭವಿಸುತ್ತೇವೆ, ನಾವೊಂದು ಪಾತ್ರೆ, ಅವನಿಂದ ಹೊರಗೆ ಇರುವವರು ಎಂದು ಭಾವಿಸಬಹುದು. ಮತ್ತೆ ಕೆಲವರು ನಾವೊಂದು ಅಲೆಯಂತೆ, ಆ ಭಗವಂತ ಅನಂತ ಸಾಗರ ಎನ್ನುವರು. ಇನ್ನು ಕೆಲವರು ಅಲೆಗೆ ಬೇರೆ ವ್ಯಕ್ತಿತ್ವ ಎಲ್ಲಿದೆ? ಅಲೆಯಂತೆ ಕಾಣುತ್ತಿರುವುದೂ ಸಾಗರವೇ ಎನ್ನಬಹುದು. ಈ ದೃಷ್ಟಿ ವೈವಿಧ್ಯತೆಯೆಲ್ಲಾ ತಾತ್ತ್ವಿಕ ರಂಗದಲ್ಲಿ. ಅನುಭವದ ಭೂಮಿಕೆಯಲ್ಲಿ ಈ ಮನಸ್ತಾಪಕ್ಕೆ ಕಾರಣವೇ ಇಲ್ಲ. ಅವರಲ್ಲಿ ಒಂದು ಸಾಮರಸ್ಯವನ್ನು ನೋಡುವೆವು. ಒಂದು ಸರಿ, ಮತ್ತೊಂದು ತಪ್ಪು ಎಂಬುವ ಗೋಜಿಗೆ ಹೋಗದೆ ಅವನನ್ನು ಎಷ್ಟು ವಿಧ ಅನುಭವಿಸಬಹುದೋ ಅಷ್ಟು ವಿಧವೂ ಅನುಭವಿಸಿ ಯಾವುದಾದರೂ ಒಂದರ ಮೇಲೆ ಹೆಚ್ಚುಕಾಲ ನಿಂತಿರುವರು. ಶ‍್ರೀರಾಮಕೃಷ್ಣರು ದ್ವೈತ, ಅದ್ವೈತ, ವಿಶಿಷ್ಟಾದ್ವೈತ ಭಾವನೆಗಳನ್ನೆಲ್ಲಾ ರುಚಿ ನೋಡಿ, ಕೊನೆಗೆ ತಾವು ಹೆಚ್ಚು ಕಾಲ ಭಕ್ತನ ಹೊಸಲ ಮೇಲೆ ನಿಂತು ಬೋಧಿಸಿದರು. ಏಕೆಂದರೆ, ಈ ದಾರಿಯಲ್ಲಿ ನಡೆಯುವ ಜನರೇ ಈ ಪ್ರಪಂಚದಲ್ಲಿ ಹೆಚ್ಚು ಮಂದಿ ಯಾವಾಗಲೂ.

ಅವನು ಕರ್ಮಗಳನ್ನೆಲ್ಲಾ ತಿಳಿಯುತ್ತಾನೆ. ಕರ್ಮದ ರಹಸ್ಯವನ್ನೆಲ್ಲಾ ತಿಳಿಯುತ್ತಾನೆ. ಈ ಜೀವನದಲ್ಲಿ ಕರ್ಮ ಹೇಗೆ ಯಾವ ಕಡೆ ನಮ್ಮನ್ನು ಒಯ್ಯುವುದೋ ಅದನ್ನು ಹೇಳುವುದಕ್ಕೆ ಎಲ್ಲರಿಗೂ ಸಾಧ್ಯವಿಲ್ಲ. ನಮ್ಮನ್ನು ಪ್ರಪಂಚಕ್ಕೆ ಬಿಗಿಯುವುದೂ ಕರ್ಮವೇ, ಬಿಗಿದ ಕಟ್ಟನ್ನು ಬಿಡಿಸುವುದೂ ಕರ್ಮವೇ. ಹೇಗೆ ಮಾಡಿದರೆ ನಮ್ಮನ್ನು ಬಿಡಿಸುವುದು, ಹೇಗೆ ಮಾಡಿದರೆ ನಮ್ಮನ್ನು ಬಂಧಿಸುವುದು –ಎಂಬುದನ್ನು ಅವನು ಚೆನ್ನಾಗಿ ಅರಿತಿರುತ್ತಾನೆ. ಯಾವುದನ್ನು ಮಾಡಬೇಕು, ಯಾವುದನ್ನು ಬಿಡಬೇಕು ಎಂಬುದನ್ನು ಚೆನ್ನಾಗಿ ಗ್ರಹಿಸಬಲ್ಲ. ಅನಾಸಕ್ತನಾಗಿರುವವನಿಗೆ ಮಾತ್ರ ಇದನ್ನು ನಿರ್ಣಯಿಸಲು ಸಾಧ್ಯ. ಕರ್ಮದಷ್ಟು ಸ್ಥೂಲವಾಗಿರುವುದು, ಸುಲಭವಾಗಿರುವುದು ಮತ್ತಾವುದೂ ಇಲ್ಲ. ಕರ್ಮ ಮಾಡುವುದು ನಮ್ಮ ಕಣ್ಣಿಗೆ ಕಾಣುವುದು, ಇತರರ ಕಣ್ಣಿಗೆ ಕಾಣುವುದು. ತೆಪ್ಪಗೆ ಇರುವುದಕ್ಕೆ ಯಾರ ಕೈಯಲ್ಲಿಯೂ ಸಾಧ್ಯವಿಲ್ಲ. ಏನನ್ನಾದರೂ ಮಾಡುವುದು ಸುಲಭ. ಅದಕ್ಕಾಗಿ ಕರ್ಮವನ್ನು ಮಾಡುತ್ತಾ ಇರುತ್ತೇವೆ. ಆದರೆ ಕರ್ಮದ ಗತಿಯಷ್ಟು ಗಹನವಾಗಿರುವುದು ಮತ್ತಾವುದೂ ಇಲ್ಲ. ಅದು ಎಲ್ಲಿಗೆ ನಮ್ಮನ್ನು ಒಯ್ಯುವುದೋ ಅದನ್ನು ಊಹಿಸುವುದಕ್ಕೆ ದೊಡ್ಡ ಜ್ಞಾನಿಗಳಿಗೇ ಕಷ್ಟ. ಆದರೆ ಭಗವಂತನಲ್ಲಿ ಆಶ್ರಯ ಪಡೆದವನು ಇದನ್ನೆಲ್ಲ ತಿಳಿಯುತ್ತಾನೆ.

\begin{shloka}
ಸಾಧಿಭೂತಾಧಿದೈವಂ ಮಾಂ ಸಾಧಿಯಜ್ಞಂ ಚ ಯೇ ವಿದುಃ~।\\ಪ್ರಯಾಣಕಾಲೇಽಪಿ ಚ ಮಾಂ ತೇ ವಿದುರ್ಯುಕ್ತಚೇತಸಃ \hfill॥ ೩೦~॥
\end{shloka}

\begin{artha}
ಅಧಿಭೂತ, ಅಧಿದೈವ, ಅಧಿಯಜ್ಞರೂಪದಲ್ಲಿ ನನ್ನನ್ನು ತಿಳಿದವರು ಸಮತ್ವವನ್ನೈದಿ, ಮರಣ ಸಮಯದಲ್ಲಿ ಕೂಡ ನನ್ನನ್ನು ತಿಳಿಯುತ್ತಾರೆ.
\end{artha}

ಶ‍್ರೀಕೃಷ್ಣ ಇಲ್ಲಿ ಬರುವ ಮೂರು ಪದಗಳನ್ನೂ ಮತ್ತು ಇನ್ನು ಕೆಲವನ್ನೂ ವಿವರವಾಗಿ ಮುಂದಿನ ಅಧ್ಯಾಯದಲ್ಲಿ ವಿವರಿಸುತ್ತಾನೆ. ಅದರ ಸಂಕ್ಷೇಪ ಅರ್ಥವನ್ನು ಇಲ್ಲಿ ಕೊಡುತ್ತೇವೆ. ಯಾರು ಭಗವಂತನಲ್ಲಿ ಶರಣಾಗತರಾಗಿ, ದೃಢ ಭಾವದಿಂದ ಭಜಿಸುವರೋ, ಅವರಿಗೆ ಈ ಮೂರು ಅತ್ಯಂತ ಮುಖ್ಯವಾದ ತಾತ್ತ್ವಿಕ ವಿಷಯಗಳು ಚೆನ್ನಾಗಿ ಗೊತ್ತಾಗುವುವು. ಅದೇ ಅಧಿಭೂತ, ಅಧಿದೈವ ಮತ್ತು ಅಧಿಯಜ್ಞ. ಎಲ್ಲಾ ಭೂತಗಳ ಹಿಂದೆಯೂ ಯಾವುದಿದೆಯೋ, ಯಾವುವು ಮಾರ್ಪಾಡಾಗಿ ವೈವಿಧ್ಯ ರೂಪಗಳನ್ನು ಧರಿಸುವುವೋ ಅಂತಹ ಅಧಿಭೂತ ಅವನಿಗೆ\break ಗೊತ್ತಾಗುವುದು. ಇದೇ ಸೃಷ್ಟಿ ಅರಮನೆಯನ್ನು ಕಟ್ಟುವುದಕ್ಕೆ ಉಪಯೋಗಿಸುವ ಇಟ್ಟಿಗೆ. ಅಧಿದೈವ ಎಂದರೆ ಎಲ್ಲಾ ದೇವದೇವತೆಗಳ ಹಿಂದೆ ನಮ್ಮ ಕೋರಿಕೆಗಳನ್ನು ನೆರವೇರಿಸುವವನು ಒಬ್ಬ ದೇವರಿರುವನು. ಇತರ ದೇವತೆಗಳೆಲ್ಲಾ ಒಂದೇ ದೇವರಿಗೆ ಕೊಟ್ಟಿರುವ ಬೇರೆ ಬೇರೆ ಹೆಸರುಗಳು. ಪ್ರಪಂಚವನ್ನೆಲ್ಲಾ ಆಳುತ್ತಿರುವ ಒಬ್ಬ ಪರಮೇಶ್ವರನನ್ನು ಅವನು ಅರಿಯುವನು. ಅಧಿಯಜ್ಞ–ಎಂದರೆ ಯಜ್ಞಕ್ಕೆ ಸಂಬಂಧಪಟ್ಟುದೆಲ್ಲಾ, ಎಲ್ಲಕ್ಕಿಂತ ಹೆಚ್ಚಾಗಿ, ಯಜ್ಞದ ಸಾರ ಅವನಿಗೆ ಗೊತ್ತಾಗುವುದು. ಯಜ್ಞರೂಪದಿಂದ ಕರ್ಮ ಮಾಡಿದರೆ ನಮ್ಮನ್ನು ಬಂಧಿಸುವುದಿಲ್ಲ. ಯಜ್ಞ ಮಾಡಿ ಆದಮೇಲೆ ಮಿಗುವುದೇ ಅಮೃತ. ಯಾರು ಭುಂಜಿಸುವರೋ ಅವರು ಅಮೃತವನ್ನೇ ಭುಂಜಿಸುತ್ತಾರೆ. ಎಲ್ಲಾ ಕರ್ಮಗಳು ಅದನ್ನು ಯಜ್ಞ ದೃಷ್ಟಿಯಿಂದ ಮಾಡಿದರೆ ಕೊನೆಗೆ ಶ್ರೇಷ್ಠಜ್ಞಾನದಲ್ಲಿ ಪರ್ಯವಸಾನವಾಗುತ್ತವೆ ಎಂಬುದನ್ನೆಲ್ಲಾ ಅರಿಯುತ್ತಾನೆ. ಈ ಪ್ರಪಂಚ, ಈಶ್ವರ, ಅವನನ್ನು ಪಡೆಯುವ ವಿಧಾನ ಇವುಗಳೆಲ್ಲವೂ ಭಗವಂತನನ್ನು ಆಶ್ರಯಿಸಿರುವವನಿಗೆ ಗೊತ್ತಾಗುವುದು.

ಭಗವಂತನನ್ನು ಸದಾ ಚಿಂತಿಸುತ್ತಾ, ತನ್ನ ಇಂದ್ರಿಯಗಳನ್ನೆಲ್ಲಾ ನಿಗ್ರಹಿಸಿ, ಮನಸ್ಸನ್ನು ದೇವರ ಕಡೆ ಹಾಕಿರುವವನು ಮರಣಕಾಲದಲ್ಲಿಯೂ ಅವನನ್ನೇ ಅರಿಯುತ್ತಾನೆ. ಮರಣಕಾಲದಲ್ಲಿ ಮನಸ್ಸು ಚೆನ್ನಾಗಿ ಭಗವಂತನನ್ನು ಚಿಂತಿಸುವ ಸ್ಥಿತಿಯಲ್ಲಿ ಇರಬೇಕು. ಆಗಲೇ ನಾವು ಸತ್ತರೆ ಸದ್ಗತಿ. ಸಾಯುವ ಕಾಲದಲ್ಲಿ ನಾವು ಏನನ್ನು ಚಿಂತಿಸುತ್ತಿರುವೆವೋ ಹಾಗೆಯೇ ಅನಂತರ ಆಗುತ್ತೇವೆ. ಸಾಯುವ ಸಮಯದಲ್ಲಿ ನಮ್ಮನ್ನು ಕೈಬಿಟ್ಟರೆ, ಆ ಜ್ಞಾನ ಇನ್ನೂ ನಮ್ಮದಾಗಿಲ್ಲ, ಆ ಜ್ಞಾನವನ್ನು ಪಡೆದು ಇನ್ನೂ ಪ್ರಯೋಜನ ಹೊಂದಿಲ್ಲ. ಪುಸ್ತಕಗಳನ್ನೆಲ್ಲಾ ತಿಳಿದುಕೊಂಡು ಪರೀಕ್ಷೆಯ\break ಸಮಯದಲ್ಲಿ ನಾವು ಊಹಿಸಿದ್ದ ಪ್ರಶ್ನೆಯೇ ಬಂದಿರುವಾಗ ಉತ್ತರವೆಲ್ಲಾ ಮರೆತುಹೋದರೆ ನಾವು ಓದಿದ್ದು ಏನು ಪ್ರಯೋಜನವಾಯಿತು? ನಾವು ಕಲಿತಿದ್ದು ಪರೀಕ್ಷಾ ಸಮಯದಲ್ಲಿ ನಮ್ಮದಾಗಬೇಕು. ಇಲ್ಲಿ ಭಗವಂತ ನಮಗೆ ಆ ಭರವಸೆಯನ್ನು ಕೊಡುವನು. ಬದುಕಿರುವಾಗ ನಮ್ಮ ಮನಸ್ಸು ದೇವರನ್ನು ಚಿಂತಿಸುತ್ತಿದ್ದರೆ, ಅಂತ್ಯಕಾಲದಲ್ಲಿಯೂ ಅದೇ ನಮ್ಮಲ್ಲಿ ಬಹಳ ಬಲವಾಗಿರುವ ಆಲೋಚನೆಯಾಗಿ ಅದನ್ನು ಎಂದಿಗೂ ಮರೆಯುವುದಿಲ್ಲ. ಅವನನ್ನೇ ಕುರಿತು ಚಿಂತಿಸುತ್ತಾ ‘ಹಾವು ತನ್ನ ಪರೆಯನ್ನು ಬಿಡುವಂತೆ’ ಜೀವಿ ಈ ಕರ್ಮದೇಹವನ್ನು ತ್ಯಜಿಸುವನು.

