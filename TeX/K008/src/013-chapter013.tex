
\chapter{ಕ್ಷೇತ್ರಕ್ಷೇತ್ರಜ್ಞಯೋಗ}

\begin{verse}
ಇದಂ ಶರೀರಂ ಕೌಂತೇಯ ಕ್ಷೇತ್ರಮಿತ್ಯಭಿಧೀಯತೇ~।\\ಏತದ್ಯೋ ವೇತ್ತಿ ತಂ ಪ್ರಾಹುಃ ಕ್ಷೇತ್ರಜ್ಞ ಇತಿ ತದ್ವಿದಃ \versenum{॥ ೧~॥}
\end{verse}

{\small ಅರ್ಜುನ, ಈ ಶರೀರವನ್ನು ಕ್ಷೇತ್ರವೆಂದು ಹೇಳುತ್ತಾರೆ. ಯಾರಿಗೆ ಇದು ಗೊತ್ತೊ ಅವನನ್ನು ಕ್ಷೇತ್ರಜ್ಞ ಎಂದು ಕ್ಷೇತ್ರಕ್ಷೇತ್ರಜ್ಞರ ಸ್ವರೂಪವನ್ನು ತಿಳಿದವರು ಹೇಳುತ್ತಾರೆ.}

ಕ್ಷೇತ್ರ ಮತ್ತು ಕ್ಷೇತ್ರಜ್ಞರ ಸ್ವರೂಪವನ್ನು ತಿಳಿದವರು ನಿಜವಾದ ತತ್ತ್ವಜ್ಞಾನಿಗಳು. ನಾವು ಈ ದೇಹದಲ್ಲಿ ವಾಸಮಾಡುತ್ತಿರುವೆವು. ಈ ದೇಹ ಎಂಬುದು ಒಂದು ಮನೆಯಂತೆ. ಇದರಲ್ಲಿ ವಾಸಮಾಡುವವನು ಜೀವ. ಈ ದೇಹವೇ ಅದಲ್ಲ. ಈ ದೇಹದಲ್ಲಿ ವಾಸಮಾಡುವವನು ಅವನು. ಈ ದೇಹವನ್ನು ಕ್ಷೇತ್ರ ಎಂತಲೂ ಹೇಳುತ್ತಾರೆ. ಕ್ಷೇತ್ರ ಎಂದರೆ ಹೊಲ. ಹೇಗೆ ಹೊಲದಲ್ಲಿ ರೈತ ತನಗೆ ಬೇಕಾದ ಬೆಳೆಯನ್ನು ತೆಗೆಯುತ್ತಾನೆಯೊ ಹಾಗೆಯೇ ಜೀವ ಈ ದೇಹವೆಂಬ ಕ್ಷೇತ್ರದಲ್ಲಿ ತನಗೆ ಬೇಕಾದ ಬೆಳೆಯನ್ನು ತೆಗೆಯುತ್ತಾನೆ. ಈ ದೇಹದಲ್ಲಿ ನಾವು ಸುಖ ದುಃಖಗಳೆಂಬ ಫಲವನ್ನು ಬೆಳೆಯುತ್ತೇವೆ.

ಯಾರು ಇದನ್ನು ತಿಳಿದುಕೊಳ್ಳಬಲ್ಲನೊ ಅವನೇ ಕ್ಷೇತ್ರಜ್ಞ. ಈ ದೇಹವೆನ್ನುವುದು ಜೀವಿಯ ಒಂದು ಹೊಲ. ಇದನ್ನು ಉಪಯೋಗಿಸುವವನೇ ಇದನ್ನು ಅರಿತವನು. ಈ ದೇಹವೇ ನಾವು ಎಂಬ ತಾದಾತ್ಮ್ಯಭಾವ ಬಂದುಹೋಗಿರುವುದು. ಆದರೆ ತತ್ತ್ವಜ್ಞಾನಿ ಇದನ್ನು ವಿಭಜನೆ ಮಾಡಿರುತ್ತಾನೆ. ಅವನಲ್ಲಿ ತಾದಾತ್ಮ್ಯಭಾವ ಇರುವುದಿಲ್ಲ.

\begin{verse}
ಕ್ಷೇತ್ರಜ್ಞಂ ಚಾಪಿ ಮಾಂ ವಿದ್ಧಿ ಸರ್ವಕ್ಷೇತ್ರೇಷು ಭಾರತ~।\\ಕ್ಷೇತ್ರಕ್ಷೇತ್ರಜ್ಞಯೋರ್ಜ್ಞಾನಂ ಯತ್ತಜ್ಜ್ಞಾನಂ ಮತಂ ಮಮ \versenum{॥ ೨~॥}
\end{verse}

{\small ಅರ್ಜುನ, ಎಲ್ಲಾ ಕ್ಷೇತ್ರಗಳಲ್ಲಿಯೂ ಇರುವ ಕ್ಷೇತ್ರಜ್ಞ ನಾನು ಎಂದು ತಿಳಿ. ಕ್ಷೇತ್ರಕ್ಷೇತ್ರಜ್ಞರ ಯಾವ ಜ್ಞಾನ ಉಂಟೋ ಅದೇ ಜ್ಞಾನವೆಂದು ನನ್ನ ಮತ.}

ಈ ದೇಹದಲ್ಲಿ ನಾವು ಇಬ್ಬರನ್ನು ನೋಡುತ್ತೇವೆ. ಒಬ್ಬ ಜೀವಾತ್ಮ. ಆತ ತನ್ನ ಪಾಪಪುಣ್ಯಗಳನ್ನು ಅನುಭವಿಸುವುದು ಈ ದೇಹದಲ್ಲಿ. ಮತ್ತೊಬ್ಬ ಪರಮಾತ್ಮ. ಅವನು ಎಲ್ಲಾ ಜೀವರಾಶಿಗಳಿಗೂ ಹಿನ್ನೆಲೆಯಂತೆ ಇರುವನು. ಸಮುದ್ರದಲ್ಲಿ ಅಲೆಯನ್ನು ನೋಡುತ್ತೇನೆ. ಹಲವು ಅಲೆಗಳಿಗೆಲ್ಲಾ ಸಾಮಾನ್ಯವಾದ ಹಿನ್ನೆಲೆಯೆ ಸಾಗರ. ಅದರಂತೆಯೇ ಈ ದೇಹದಲ್ಲಿ ಇದನ್ನು ಉಪಯೋಗಿಸುವ ಜೀವಿ ಇರುವನು. ಈ ಜೀವಿಗಳಿಗೆಲ್ಲ ಒಡೆಯನೇ ದೇವರು. ಕ್ಷೇತ್ರ ಎಂದರೆ ಈ ದೇಹ. ಇದು ತುಂಬ ಸ್ಥೂಲವಾಗಿರುವುದು. ಇದು ಪಂಚಭೂತಗಳಿಂದ ಆಗಿರುವುದು. ಇದು ಹುಟ್ಟುವುದಕ್ಕೆ ಹಿಂದೆ ಇರಲಿಲ್ಲ. ಸತ್ತುಹೋದರೆ ಆ ಮೇಲೆ ಪಂಚಭೂತಗಳಲ್ಲಿ ಒಂದಾಗಿ ಹೋಗುವುದು. ಈ ದೇಹವನ್ನು ಉಪಯೋಗಿಸುವವನೇ ಜೀವ. ಇದರಲ್ಲಿಯೇ ಅವನು ಪಾಪ ಪುಣ್ಯಗಳನ್ನು ಬಿತ್ತುವುದು ಮತ್ತು ಅನುಭವಿಸುವುದು. ಈ ಜೀವಿಯ ಹಿಂದೆ ಈಶ್ವರ ಇರುವನು. ಅವನು ದೇಹ ಮನಸ್ಸು ಬುದ್ಧಿ ಇಂದ್ರಿಯ ಇವುಗಳಲ್ಲಿ ಆಗುವ ಬದಲಾವಣೆಗಳನ್ನೆಲ್ಲಾ ನೋಡುತ್ತಿರುವನೆ ಹೊರತು ಇದರಿಂದ ಬಾಧಿತನಾಗುವುದಿಲ್ಲ. ತತ್ತ್ವಶಾಸ್ತ್ರ ಈ ಮೂರನ್ನೂ ತಿಳಿದುಕೊಳ್ಳಲು ಯತ್ನಿಸುವುದು, ಒಂದಕ್ಕೂ ಮತ್ತೊಂದಕ್ಕೂ ಇರುವ ಸಂಬಂಧವನ್ನು ಅರಿಯಲು ಯತ್ನಿಸುವುದು.

\begin{verse}
ತತ್ ಕ್ಷೋತ್ರಂ ಯಚ್ಚ ಯೈದೃಕ್ಚಯದ್ವಿಕಾರಿ ಯತತಶ್ಚ ಯತ್​।\\ಸ ಚ ಯೋ ಯತ್ಪಭಾವಶ್ಚ ತತ್ಸಮಾಸೇನ ಮೇ ಶೃಣು \versenum{॥ ೩~॥}
\end{verse}

{\small ಆ ಕ್ಷೇತ್ರ ಯಾವುದು, ಹೇಗಿದೆ, ಅದರ ವಿಕಾರವೆಂಥದ್ದು, ಎಲ್ಲಿಂದ ಬಂತು, ಕ್ಷೇತ್ರಜ್ಞ ಯಾರು, ಅವನ ಶಕ್ತಿ ಎಂಥದ್ದು ಅದನ್ನು ಸಂಕ್ಷೇಪವಾಗಿ ಕೇಳು.}

ಇಲ್ಲಿಂದ ಶ‍್ರೀಕೃಷ್ಣ ತತ್ತ ್ವಕ್ಕೆ ಸಂಬಂಧಪಟ್ಟ ವಿಷಯಗಳನ್ನು ತಿಳಿಸುತ್ತಾನೆ. ಯಾವುದನ್ನು ಕ್ಷೇತ್ರ ಎಂದು ಹೇಳುತ್ತಾರೆ, ಅದು ಹೇಗಿದೆ ಎಂದರೆ ಅದರ ಸ್ವಭಾವಗಳೇನು, ಅದರ ಗುಣಗಳೇನು, ಅದರ ವಿಕಾರಗಳೇನು ಎಂದರೆ ಅದು ಯಾವ ಯಾವ ರೂಪುಗಳನ್ನು ಧರಿಸುತ್ತದೆ, ಅದು ಯಾವುದರಿಂದ ಆಗುವುದು ಎಂದರೆ ಅದಕ್ಕೂ ಅದರ ಹಿಂದಿನದಕ್ಕೂ ಇರುವ ಕಾರ್ಯಕಾರಣ ಸಂಬಂಧಗಳಾವುವು –ಇವುಗಳನ್ನೆಲ್ಲಾ ಸಂಕ್ಷೇಪವಾಗಿ ವಿವರಿಸುತ್ತಾನೆ.

ಅವನು ಯಾರು, ಈ ಕ್ಷೇತ್ರವನ್ನು ಉಪಯೋಗಿಸುವವನ ಸ್ವಭಾವ ಎಂತಹುದು ಅವುಗಳನ್ನು ಹೇಳುತ್ತಾನೆ. ಈ ದೇಹ, ಇದನ್ನು ಉಪಯೋಗಿಸುವ ಜೀವ, ದೇಹ ಮತ್ತು ಜೀವಗಳೆರಡಕ್ಕೂ ಒಡೆಯನಾದ ಈಶ್ವರ, ಈ ಮೂರು ತತ್ತ್ವಪ್ರಪಂಚದಲ್ಲಿ ಬಹಳ ಮುಖ್ಯವಾದ ವಿಷಯಗಳು.

\begin{verse}
ಪುಷಿಭಿರ್ಬಹುಧಾ ಗೀತಂ ಛಂದೋಭಿರ್ವಿವಿಧೈಃ ಪೃಥಕ್~।\\ಬ್ರಹ್ಮಸೂತ್ರಪದೈಶ್ಚೈವ ಹೇತುಮದ್ಭಿರ್ವಿನಿಶ್ಚಿತೈಃ \versenum{॥ ೪~॥}
\end{verse}

{\small ಕ್ಷೇತ್ರ ಕ್ಷೇತ್ರಜ್ಞರ ಸ್ವರೂಪವನ್ನು ಪುಷಿಗಳು ಬಹು ಪ್ರಕಾರವಾಗಿ ಹಾಡಿದ್ದಾರೆ. ವಿವಿಧವಾದ ಛಂದಸ್ಸುಗಳು ಬೇರೆ ಬೇರೆಯಾಗಿ ಹಾಡಿವೆ. ಹೇತುಗಳಿಂದ ಕೂಡಿರುವ ಮತ್ತು ನಿಶ್ಚಯರೂಪವಾಗಿರುವ ಬ್ರಹ್ಮಸೂತ್ರಪದಗಳು ಹೇಳಿವೆ.}

ಪುಷಿಗಳು ಎಂದರೆ ಸತ್ಯವನ್ನು ಸಾಕ್ಷಾತ್ಕಾರ ಮಾಡಿಕೊಂಡ ವ್ಯಕ್ತಿಗಳು. ಇವರನ್ನು ಮಂತ್ರ ದ್ರಷ್ಟಾರರು ಎನ್ನುತ್ತಾರೆ. ಇವರು ಪ್ರಪಂಚದಲ್ಲಿ ಸದಾ ಜಾಗೃತವಾಗಿರುವ ಆಧ್ಯಾತ್ಮಿಕ ಸತ್ಯಗಳನ್ನು ಮೊದಲು ಮನಗಂಡವರು. ಅದನ್ನು ತಯಾರು ಮಾಡಿದವರು ಎಂದಲ್ಲ. ಯಾರೂ ಆ ಸತ್ಯಗಳನ್ನು ತಯಾರು ಮಾಡಿಲ್ಲ. ಆಧ್ಯಾತ್ಮಿಕ ಸತ್ಯಗಳು ಈ ಪ್ರಪಂಚದಲ್ಲಿ ಸದಾ ಇರುತ್ತವೆ. ಆದರೆ ಯಾವಾಗ ಒಬ್ಬ ವ್ಯಕ್ತಿ ತನ್ನ ಮನಸ್ಸನ್ನು ಶುದ್ಧ ಮಾಡಿಕೊಂಡು ಅದನ್ನು ಸತ್ಯದ ಮೇಲೆ ಕೇಂದ್ರೀಕರಿಸುತ್ತಾನೆಯೊ ಅವನಿಗೆ ಅದು ಅನುಭವ ಆಗುವುದು. ಯಾವುದನ್ನು ಅನುಭವಿಸುತ್ತಾನೆಯೊ ಅದನ್ನು ಕವಿ ಹಾಡುವಂತೆ ಹಾಡುತ್ತಾನೆ. ತನ್ನ ಅನುಭವಕ್ಕೆ ಒಂದು ಶಬ್ದರೂಪವನ್ನು ಕೊಡುತ್ತಾನೆ. ಆ ಶಬ್ದರೂಪ ಒಂದು ಪ್ರಾಸದ ಅಲಂಕಾರವನ್ನು ಹಾಕಿಕೊಂಡು ವ್ಯಕ್ತವಾಗುತ್ತದೆ. ವೇದಗಳಲ್ಲಿ ಹಲವು ಬಗೆಯ ಛಂದಸ್ಸುಗಳಿವೆ. ಒಂದೊಂದು ಮಂತ್ರವೂ ಒಂದೊಂದು ಛಂದಸ್ಸಿನ ಆಕಾರವನ್ನು ಧರಿಸಿ ವ್ಯಕ್ತ ವಾಗುವುದು.

ಬ್ರಹ್ಮಸೂತ್ರ ಎಂಬುದನ್ನು ವ್ಯಾಸರು ಉಪನಿಷತ್ತೆಂಬ ಹೂದೋಟದಿಂದ ಪುಷ್ಪಗಳನ್ನು ಕಿತ್ತು ತಂದು ಅದನ್ನು ಒಂದು ಹೇತುಶಾಸ್ತ್ರವನ್ನಾಗಿ ಮಾಡಿ ಮಾಲೆಯಂತೆ ಕಟ್ಟಿದ್ದಾರೆ. ಇಲ್ಲಿ ಜೀವ, ಈಶ್ವರ, ಜಗತ್, ಇವುಗಳಿಗೆ ಇರುವ ಸಂಬಂಧ, ಸೃಷ್ಟಿ ಜನ್ಮ ಕರ್ಮ ಮುಂತಾದವುಗಳೆ ಬರುವುವು. ನಾವು ವೇದಾಂತ ತತ್ತ್ವವನ್ನು ತಿಳಿದುಕೊಳ್ಳಬೇಕಾದರೆ, ಉಪನಿಷತ್ತು ಮತ್ತು ಬ್ರಹ್ಮಸೂತ್ರಗಳು ಅತ್ಯಂತ ಮುಖ್ಯವಾದ ಶಾಸ್ತ್ರಗಳು. ಅನಂತರ ಹಿಂದೂ ದರ್ಶನದ ಮೇಲೆ ಯಾರು ಏನನ್ನು ಹೇಳಬೇಕಾದರೂ ಅವರೆಲ್ಲ ಉಪನಿಷತ್​ಗಳು ಮತ್ತು ಬ್ರಹ್ಮಸೂತ್ರದ ಆಧಾರದ ಮೇಲೆ ಹೇಳುವರು. ಇದೇ ತಳಪಾಯ ಇದ್ದ ಹಾಗೆ. ಇದರ ಆಧಾರದ ಮೇಲೆ ಮಾತ್ರ ತತ್ತ್ವಶಾಸ್ತ್ರಗಳು ಊರ್ಜಿತವಾಗಿ ನಿಲ್ಲಬಲ್ಲವು. ಉಪನಿಷತ್ತಿನಲ್ಲಿ ಭಾವದ ಗೊಂಚಲಿದೆ. ಅದನ್ನು ತಾತ್ತ್ವಿಕವಾಗಿ ಪೋಣಿಸಿರುವುದೇ ಬ್ರಹ್ಮಸೂತ್ರ. ಇದನ್ನೇ ಹೇತುಶಾಸ್ತ್ರ ಎಂದು ಕೂಡ ಹೇಳುತ್ತಾರೆ. ಕಾರ್ಯಕಾರಣ ಸಂಬಂಧದ ರೀತಿ ಇದನ್ನು ಜೋಡಿಸಿರುವರು.

\begin{verse}
ಮಹಾಭೂತಾನ್ಯಹಂಕಾರೋ ಬುದ್ಧಿರವ್ಯಕ್ತಮೇವ ಚ~।\\ಇಂದ್ರಿಯಾಣಿ ದಶೈಕಂ ಚ ಪಂಚ ಚೇಂದ್ರಿಯಗೋಚರಾಃ \versenum{॥ ೫~॥}
\end{verse}

\begin{verse}
ಇಚ್ಛಾ ದ್ವೇಷಃ ಸುಖಂ ದುಃಖಂ ಸಂಘಾತಶ್ಚೇತನಾ ಧೃತಿಃ~।\\ಏತತ್​ಕ್ಷೇತ್ರಂ ಸಮಾಸೇನ ಸವಿಕಾರಮುದಾಹೃತಮ್ \versenum{॥ ೬~॥}
\end{verse}

{\small ಪಂಚ ಮಹಾಭೂತಗಳು, ಅಹಂಕಾರ ಬುದ್ಧಿ ಅವ್ಯಕ್ತ ಹತ್ತು ಇಂದ್ರಿಯಗಳು ಮತ್ತು ಮನಸ್ಸು, ಐದು ಇಂದ್ರಿಯ ವಿಷಯಗಳು, ಇಚ್ಛೆ, ದ್ವೇಷ ಸುಖ ದುಃಖ ಸಂಘಾತ, ಚೇತನಾ, ಧೃತಿ ಇವು ಕ್ಷೇತ್ರ. ಇವು ವಿಕಾರದಿಂದ ಕೂಡಿವೆ. ಇದು ಸಂಕ್ಷೇಪವಾಗಿ ಹೇಳಲ್ಪಟ್ಟಿತು.}

ಇಲ್ಲಿ ಕ್ಷೇತ್ರದ ವಿವರಣೆಯನ್ನು ಕೊಟ್ಟಿದೆ. ತುಂಬಾ ಸ್ಥೂಲವಾಗಿರುವುದೇ ಪಂಚಭೂತಗಳು. ಇದರಿಂದಲೇ ನಾಮರೂಪಗಳೆಲ್ಲ ಆಗುವುದು. ನಮ್ಮ ದೇಹದಿಂದ ಹಿಡಿದು, ಸೂರ್ಯ, ಚಂದ್ರ ನಕ್ಷತ್ರಗಳವರೆಗೆ ಎಲ್ಲಾ ವಸ್ತುಗಳು ಆಗುವುದು ಈ ಪಂಚಭೂತಗಳಿಂದ. ಅದೇ ಪೃಥ್ವಿ, ನೀರು, ಗಾಳಿ, ಬೆಂಕಿ, ಆಕಾಶ ಆಗಿರುವುದು. ಅನಂತರವೇ ಅದಕ್ಕಿಂತ ಸ್ವಲ್ಪ ಸೂಕ್ಷ್ಮವಾಗಿರುವುದಕ್ಕೆ ಬರುತ್ತೇವೆ. ಅದೇ ಅಹಂಕಾರ, ನಮ್ಮಲ್ಲೆಲ್ಲ ಇರುವ ನಾನು ಎಂಬ ಪ್ರತ್ಯಯ. ಇದೊಂದು ಗೂಟದಂತೆ, ಇದಕ್ಕೆ ನಮ್ಮ ದೇಹ ಇಂದ್ರಿಯ ಮನಸ್ಸು ಬುದ್ಧಿ ಎಂಬ ಉಪಾಧಿಗಳ ಗಂಟನ್ನೆಲ್ಲ ತಗಲಿಹಾಕುವೆವು. ಈ ಉಪಾಧಿಗಳೆಲ್ಲ ನಮಗಿಂತ ಹೊರಗಿವೆ. ಆದರೆ ಅವನ್ನು ಉಪಯೋಗಿಸುತ್ತ ಇದೇ ನಾವು ಎಂಬ ತಾದಾತ್ಮ್ಯ ಭಾವ ಬಂದು ಹೋಗುವುದು. ಅನಂತರವೇ ಅಹಂಕಾರಕ್ಕೆ ಬಹಳ ಹತ್ತಿರದಲ್ಲಿರುವ ಅದರ ಬಯಕೆಗಳನ್ನೆಲ್ಲ ಈಡೇರಿಸುವುದರಲ್ಲಿ ನಿರತವಾದ ಬುದ್ಧಿ ಇದೆ. ಇದು ವಿಚಾರ ಮಾಡುವುದು. ತಿಳಿದಿರುವುದರಿಂದ ತಿಳಿಯದ ಕಡೆಗೆ ನೆಗೆಯುವುದು. ಯಾವುದನ್ನು ಹೇಗೆ ಪಡೆಯಬಹುದು ಎಂಬುದಕ್ಕೆ ಚತುರೋಪಾಯಗಳನ್ನೆಲ್ಲ ಕುರಿತು ಯೋಚಿಸುವುದು. ಮನುಷ್ಯ ನಲ್ಲಿರುವ ಅತ್ಯಂತ ಶಕ್ತಿಯುತವಾದುದೇ ಇದು. ಇಲ್ಲಿ ಪ್ರಾಣಿಗಳಿಗೆ ತಮ್ಮನ್ನು ರಕ್ಷಿಸಿಕೊಳ್ಳುವುದಕ್ಕೆ ಮತ್ತೊಂದು ಪ್ರಾಣಿಯನ್ನು ನಾಶಮಾಡುವುದಕ್ಕೆ, ವಿಷ, ದಂಷ್ಟ್ರ, ನಖ, ಸೊಂಡಿಲು ಮುಂತಾದುವು ಗಳನ್ನು ದೇವರು ಕೊಟ್ಟಿರುವನು. ಮನುಷ್ಯನಿಗಾದರೊ ಬುದ್ಧಿಶಕ್ತಿಯನ್ನು ಕೊಟ್ಟಿರುವನು. ಇದರ ಸಹಾಯದಿಂದ ಮನುಷ್ಯ ಅದ್ಭುತವಾದುದನ್ನು ಸಾಧಿಸಬಲ್ಲ. ಇದರ ಸಹಾಯದಿಂದ ಸಮುದ್ರದ ಆಳಕ್ಕೆ ಹೋಗುತ್ತಾನೆ. ಆಕಾಶದ ಎತ್ತರಕ್ಕೆ ಹಾರುತ್ತಾನೆ. ಚತುಷ್ಪಾದಿಗಳನ್ನು ಮೀರಿ ಓಡುತ್ತಾನೆ. ಅವನು ಬುದ್ಧಿಶಕ್ತಿಯಿಂದ ಕಂಡುಹಿಡಿದ ಧ್ವಂಸ ಮತ್ತು ಉಪಕಾರಾತ್ಮಕವಾಗಿರುವ ಕೆಲಸಗಳ ಮುಂದೆ ಪ್ರಕೃತಿಯೇ ನಾಚುವುದು. ಅವನ ಕೈಬೆರಳ ಸನ್ನೆಗೆ ಜ್ವಾಲಾಮುಖಿ ಚಂಡಮಾರುತಗಳೇಳು ವಂತೆ ಮಾಡಬಲ್ಲ. ಸಹಸ್ರಾರು ಜನರನ್ನು ನಿಮಿಷಾರ್ಧದಲ್ಲಿ ಕೊಲ್ಲಬಲ್ಲ, ಮತ್ತು ಕಾಯಬಲ್ಲ. ಈ ಬುದ್ಧಿಶಕ್ತಿಯೊಂದೇ ಮನುಷ್ಯನನ್ನು ಇತರ ಪ್ರಾಣಿಗಳ ಗುಂಪಿನಿಂದ ಬೇರೆ ಮಾಡಿರುವುದು.\enginline{}

ಅನಂತರವೇ ಅವ್ಯಕ್ತವೆಂಬುದಿದೆ. ಎಲ್ಲಾ ಜೀವರಾಶಿಗಳಿಗೂ ಬುದ್ಧಿ ಬರಬೇಕಾದರೆ ಅದಕ್ಕೊಂದು ಮೂಲ ಇರಬೇಕು. ನಮ್ಮ ದೇಹ ಆಗುವುದಕ್ಕೆ ಪಂಚಭೂತಗಳು ಹೇಗೆ ಮೂಲವೋ ಹಾಗೆ. ಆ ಮೂಲ ಅಹಂಕಾರವೇ ಅವ್ಯಕ್ತ. ಇದರಿಂದ ಪ್ರತಿ ಜೀವಿಯೂ ತನ್ನ ಕರ್ಮಾನುಸಾರ ತನ್ನ ಬುದ್ಧಿಯನ್ನು ಪೋಷಿಸುವ ದ್ರವ್ಯವನ್ನು ಹೀರುತ್ತಾನೆ. ಬುದ್ಧಿ ಕೆಲಸ ಮಾಡಬೇಕಾದರೆ, ಅದಕ್ಕೆ ಸಹಾಯಕವಾಗಿ ಹತ್ತು ಇಂದ್ರಿಯಗಳು, ಮನಸ್ಸು ಮತ್ತು ವಿಷಯ ವಸ್ತುಗಳು ಇರಬೇಕಾಗಿದೆ. ಹತ್ತು ಇಂದ್ರಿಯಗಳೇ ಐದು ಕರ್ಮೇಂದ್ರಿಯಗಳು ಐದು ಜ್ಞಾನೇಂದ್ರಿಯಗಳು. ಕರ್ಮೇಂದ್ರಿಯಗಳಿಂದ ಕರ್ಮ ಮಾಡುವನು. ಅವೇ ಕೈ, ಕಾಲು, ಬಾಯಿ, ಎರಡು ಗುಹ್ಯೇಂದ್ರಿಯಗಳು. ಐದು ಜ್ಞಾನೇಂದ್ರಿಯ ಗಳಿಂದ ಹೊರಗಿನ ಪ್ರಪಂಚ ಹೇಗಿದೆ ಎಂದು ತಿಳಿಯುತ್ತಾನೆ. ಇದು ಮನೆ ಒಳಗೆ ಇದ್ದುಕೊಂಡು ಬಾಗಿಲು, ಕಿಟಕಿ ಮತ್ತು ಗವಾಕ್ಷಗಳ ಮೂಲಕ ಹೊರಗೆ ಏನಿದೆ ಎಂಬುದನ್ನು ತಿಳಿದುಕೊಳ್ಳುವಂತೆ. ಕಣ್ಣುಗಳಿಂದ ವಸ್ತುವಿನ ರೂಪವನ್ನು ತಿಳಿದುಕೊಳ್ಳುತ್ತೇವೆ. ಕಿವಿಯಿಂದ ಬರುವ ಶಬ್ದವನ್ನು ಹೇಗಿದೆ, ಅದು ಯಾರಿಂದ ಬರುತ್ತಿದೆ, ಅದು ಪ್ರಿಯವಾಗಿರುವುದೇ, ಅಪ್ರಿಯವಾಗಿರುವುದೇ ಎಂಬುದನ್ನು ನಿಶ್ಚಯಿಸುತ್ತೇವೆ. ಮೂಗು ಹೊರಗಿನ ವಸ್ತುಗಳ ಮೂಲಕ ಬರುವ ವಾಸನೆಯನ್ನು ಹೀರುತ್ತದೆ. ಅದನ್ನು ಹಲವು ರೀತಿ ವಿಭಾಗಮಾಡುವುದು. ನಾಲಗೆಯಿಂದ ಒಂದು ವಸ್ತುವನ್ನು ರುಚಿ ನೋಡು ತ್ತೇವೆ, ಅದರ ರಸಗಳನ್ನೆಲ್ಲ ತಿಳಿದುಕೊಳ್ಳುತ್ತೇವೆ. ಸ್ಪರ್ಶದಿಂದ ಒಂದು ವಸ್ತು ಗಟ್ಟಿಯಾಗಿದೆಯೆ, ಮೃದುವಾಗಿದೆಯೆ, ಒರಟಾಗಿದೆಯೆ, ಶೀತವಾಗಿದೆಯೆ, ಉಷ್ಣವಾಗಿದೆಯೆ, ಎಲ್ಲವನ್ನೂ ತಿಳಿದುಕೊಳ್ಳು ತ್ತೇವೆ. ಐದು ವಿಷಯ ವಸ್ತುಗಳು ಬೇರೆ ಹೊರಗೆ ಇವೆ. ನಮ್ಮ ಜ್ಞಾನೇಂದ್ರಿಯ ಕೆಲಸ ಮಾಡ ಬೇಕಾದರೆ ಅದಕ್ಕೆ ಹೊರಗೆ ಈ ಗುಣಗಳುಳ್ಳ ವಸ್ತುಗಳಿರಬೇಕು–ಅವೇ ಶಬ್ದ ಸ್ಪರ್ಶ ರೂಪ ರಸ ಗಂಧ.

ಅನಂತರ ಮನಸ್ಸಿನಲ್ಲಿ ಬಾಹ್ಯವಸ್ತುವಿನೊಂದಿಗೆ ಪ್ರತಿಕ್ರಿಯಾರೂಪವಾಗಿ, ಏಳುವ ಗುಣಗಳಿವೆ. ಇವೇ ಇಚ್ಛೆ ಮತ್ತು ದ್ವೇಷಗಳು. ಕೆಲವು ವೇದನೆಗಳು ಪ್ರಿಯವಾಗಿರುತ್ತವೆ. ಅವನ್ನು ಪ್ರೀತಿಸುತ್ತೇವೆ. ಕೆಲವು ವೇದನೆಗಳು ಅಪ್ರಿಯವಾಗಿರುತ್ತವೆ. ತತ್​ಕ್ಷಣವೇ ಅವನ್ನು ತಿರಸ್ಕರಿಸುತ್ತೇವೆ. ಬೆಂಕಿಯನ್ನು ಮುಟ್ಟಿದೊಡನೆ ಕೈ ಅದನ್ನು ತಿರಸ್ಕರಿಸುವುದು. ಸುಖ ಮತ್ತು ದುಃಖವೂ ಕೂಡ ಎಲ್ಲೊ ಹೊರಗೆ ಬಿದ್ದಿರುವುದಿಲ್ಲ. ಇದು ನನ್ನ ಮನಸ್ಸಿನ ಪ್ರತಿಕ್ರಿಯೆ. ಒಂದು ಪರೀಕ್ಷೆಯಲ್ಲಿ ಪಾಸು ಆಗಿದ್ದೇನೆ ಎಂಬ ಸುದ್ದಿಯನ್ನು ಕೇಳಿ ಸಂತೋಷಪಡುತ್ತೇನೆ. ಯಾರೊ ನನಗೆ ಬೇಕಾಗಿರುವವರು ಕಾಲವಾದರು ಎಂಬುದನ್ನು ಕೇಳಿ ವ್ಯಥೆಪಡುತ್ತೇನೆ.

ಸಂಘಾತ, ಎಂದರೆ ಹೊರಗಿನ ವಿಷಯ ವಸ್ತುವಿನೊಂದಿಗೆ ಸಂಬಂಧವನ್ನು ಕಲ್ಪಿಸಿ ಕೊಳ್ಳುವುದಕ್ಕೆ ಹೋಗುತ್ತೇವೆ. ಆಗಲೇ ನನ್ನಲ್ಲಿ ಒಂದು ಪ್ರತಿಕ್ರಿಯೆ ಆಗಬೇಕಾದರೆ. ಸುಮ್ಮನೆ ಹೊರಗೆ ವಿಷಯ ವಸ್ತು ಇದ್ದು, ನನ್ನಲ್ಲಿ ಅದನ್ನು ಹಿಡಿಯುವ ಇಂದ್ರಿಯ ಇದ್ದರೆ ಮಾತ್ರ ಸಾಲದು. ನಾನು ಅದರೊಂದಿಗೆ ಸಂಬಂಧ ಬೆಳಸಬೇಕು. ಆಗಲೆ ನನಗೊಂದು ಅನುಭವ ಆಗಬೇಕಾದರೆ. ಆಗಲೇ ಚೇತನ ಇಂದ್ರಿಯದ ಹಿಂದೆ ಇದ್ದುಕೊಂಡು ಅದನ್ನು ಅನುಭವಿಸುವುದು. ಧೃತಿ ಎಂದರೆ ಅನುಭವ ವನ್ನು ಧೈರ್ಯವಾಗಿ ನಿಂತು ಅನುಭವಿಸುವುದು. ಆ ಧೃತಿ ಇಲ್ಲದಿದ್ದರೆ ನಾವು ಅದರ ವೇಗಕ್ಕೆ ಕೊಚ್ಚಿಕೊಂಡು ಹೋಗುತ್ತೇವೆ. ಒಂದು ವಸ್ತುವನ್ನು ಅನುಭವಿಸಬೇಕಾದರೂ ಅದರಿಂದ ಬರುವ ವೇದನೆಗಳನ್ನು ಎದುರಿಸಬೇಕು.

ಆ ಕ್ಷೇತ್ರ ವಿಕಾರದಿಂದ ಕೂಡಿದೆ. ಈ ದೇಹ, ಮನಸ್ಸು, ಬುದ್ಧಿ, ಇಂದ್ರಿಯ, ವಿಷಯಗಳು ಎಲ್ಲವೂ ಬದಲಾಯಿಸುತ್ತಿವೆ. ಒಂದು ದಿನ ಇದ್ದ ಹಾಗೆ ಮತ್ತೊಂದು ದಿನ ಇರುವುದಿಲ್ಲ. ನಮ್ಮ ಇಂದ್ರಿಯಗಳು ಬಾಲ್ಯದಲ್ಲಿ ಯೌವನದಲ್ಲಿ ಬಹಳ ಬಲಿಷ್ಠವಾಗಿದ್ದು ವಯಸ್ಸಾದಂತೆಲ್ಲ ದುರ್ಬಲ ವಾಗುತ್ತ ಬರುವುವು. ಅಂತೂ ಒಂದು ಸಲ ಇವೆಲ್ಲ ನಾಶವಾಗುವುದು. ಎಲ್ಲಾ ನಾಶದ ಕಡೆ ನಿಧಾನವಾಗಿ ಹೋಗುತ್ತಿದೆ. ಕೆಲವು ವೇಗವಾಗಿ ಹೋಗುತ್ತಿವೆ, ಮತ್ತೆ ಕೆಲವು ನಿಧಾನವಾಗಿ ಹೋಗುತ್ತಿವೆ. ಯಾವುದೂ ಮಧ್ಯ ನಿಲ್ಲುವಂತಿಲ್ಲ. ನಾಶದ ಕಡೆ ಹೋಗಬೇಕು. ಈ ಬದಲಾಯಿಸು ತ್ತಿರುವ ಸೂಕ್ಷ್ಮ ಮತ್ತು ಸ್ಥೂಲ ವಸ್ತುಗಳ ಸುಳಿಯನ್ನೇ ನಾವು ಜೀವ ಎನ್ನುತ್ತೇವೆ.

\begin{verse}
ಅಮಾನಿತ್ವಮಂದಭಿತ್ವಮಹಿಂಸಾ ಕ್ಷಾಂತಿರಾರ್ಜವಮ್~।\\ಆಚಾರ್ಯೋಪಾಸನಂ ಶೌಚಂ ಸ್ಥೈರ್ಯಮಾತ್ಮವಿನಿಗ್ರಹಃ \versenum{॥ ೭~॥}
\end{verse}

{\small ಮಾನವನ್ನು ಭಾವಿಸದೆ ಇರುವುದು, ಜಂಭವಿಲ್ಲದೆ ಇರುವುದು, ಅಹಿಂಸೆ, ಕ್ಷಾಂತಿ, ಆರ್ಜವ, ಆಚಾರ್ಯೋ ಪಾಸನೆ, ಶೌಚ, ಸ್ಥೈರ್ಯ, ಆತ್ಮನಿಗ್ರಹ–}

ಇಲ್ಲಿಂದ ಹನ್ನೊಂದನೆ ಶ್ಲೋಕದವರೆಗೆ, ಜ್ಞಾನಿಯ ಲಕ್ಷಣವನ್ನು ಹೇಳುವನು. ಇಲ್ಲಿ ಜ್ಞಾನಿ ಎಂದರೆ ಪರಮಾತ್ಮನ ವಿಷಯವನ್ನು ಶಾಸ್ತ್ರ ಮುಖೇನ ತಿಳಿದವನು ಎಂದು ಭಾವಿಸಬಾರದು. ಇಲ್ಲಿ ಬರುವ ಜ್ಞಾನದ ವಿವರಣೆಗೂ ಪಾಂಡಿತ್ಯಕ್ಕೂ ಯಾವ ಸಂಬಂಧವೂ ಇರುವಂತೆ ತೋರುವುದಿಲ್ಲ. ಯಾರು ಆತ್ಮ ಅನಾತ್ಮಗಳನ್ನು ಬೇರ್ಪಡಿಸಿ ಪರಮಾತ್ಮನನ್ನು ಅನುಭವಿಸುವುದಕ್ಕೆ ಯತ್ನಿಸುತ್ತಿರ ವನೊ ಅವನು ಜ್ಞಾನಿ ಎನ್ನುವನು. ಮುಂದೆ ಬರುವ ಗುಣಗಳಿದ್ದರೆ ಅವನು ಜ್ಞಾನವನ್ನು ಪಡೆಯಲು ಯೋಗ್ಯನಾಗುತ್ತಾನೆ. ನಮಗೆಲ್ಲ ಜ್ಞಾನವನ್ನು ಪಡೆಯಬೇಕೆಂಬ ಆಸೆ ಇರಬಹುದು. ಆದರೆ ಒಂದು ವಸ್ತುವನ್ನು ಪಡೆಯಲು ಆಸೆಯೊಂದೇ ಸಾಲದು. ಅದಕ್ಕೆ ಯೋಗ್ಯತೆಯನ್ನು ಪಡೆದುಕೊಂಡಿರಬೇಕು. ಆ ಯೋಗ್ಯತೆ ನಮ್ಮಲ್ಲಿ ಈಗ ಇಲ್ಲದೆ ಇದ್ದರೆ ಅದನ್ನು ಪಡೆಯಲು ಪ್ರಯತ್ನಿಸಬೇಕು. ಇಚ್ಛಿಸಿದ ಒಡನೆಯೆ ಅವೇನು ನಮಗೆ ಬಂದು ಬಿಡುವುದಿಲ್ಲ. ಅದಕ್ಕಾಗಿ ಪ್ರಯತ್ನಿಸಬೇಕು.

ಅವನಲ್ಲಿ ಅಮಾನ್ವಿತ ಇರುವುದು. ಆತ ತನ್ನ ಸಮಾನ ಇಲ್ಲ ಎಂದು ಮೆರೆಯುವುದಿಲ್ಲ. ನಮ್ಮಲ್ಲಿರುವ ಅಧಿಕಾರ, ಪಾಂಡಿತ್ಯ ಮುಂತಾದುವುಗಳ ವಿಷಯದಲ್ಲಿ ಹೆಮ್ಮೆ ಕೊಚ್ಚಿಕೊಳ್ಳುವುದಕ್ಕೆ ಏನಿದೆ ಸರಿಯಾಗಿ ನೋಡಿದರೆ? ಇವು ತಾತ್ಕಾಲಿಕವಾಗಿ ನಮ್ಮಲ್ಲಿವೆ. ಇದನ್ನು ಇತರರರಿಗೆ ಕೊಡುವು ದಕ್ಕೆ ನನ್ನಲ್ಲಿ ಇಟ್ಟಿರುವನು ದೇವರು ಎಂದು ಭಾವಿಸುವನು. ಯಾರಾದರೂ ಒಬ್ಬರಲ್ಲಿ ಇಡಬೇಕಾಗಿದೆ ದೇವರು ಅದನ್ನು ಹಂಚುವುದಕ್ಕೆ ಮುಂಚೆ. ಅದಕ್ಕಾಗಿ ನನ್ನಲ್ಲಿ ಇಟ್ಟಿರುವನು. ಅವನ ಕೆಲಸವಾಗು ವುದಕ್ಕೆ ನನ್ನಲ್ಲಿದೆ, ಕೇವಲ ನನ್ನ ಭೋಗಕ್ಕೆ ಅಲ್ಲ ಎಂಬುದನ್ನು ಚೆನ್ನಾಗಿ ತಿಳಿದಿರುವನು. ಒಂದು ವೇಳೆ ನನ್ನಲ್ಲಿ ಅಧಿಕಾರ, ಐಶ್ವರ್ಯ, ಪಾಂಡಿತ್ಯ ಇದೆ ಎಂದು ಭಾವಿಸಿದರೂ, ಎಷ್ಟು ಇದೆ ಅದು. ಅದಕ್ಕಿಂತ ಹೆಚ್ಚು ಇರುವವನೊಂದಿಗೆ ಹೋಲಿಸಿಕೊಂಡರೆ ನಾವು ನಾಚಬೇಕಾಗುವುದು. ಆದರೆ ನಾವು ಆ ರೀತಿ ಮಾಡುವುದಿಲ್ಲ. ಯಾರಿಗೆ ನಮಗಿಂತ ಕಡಿಮೆ ಇದೆಯೊ ಅವರೊಂದಿಗೆ ಹೋಲಿಸಿಕೊಂಡು ನಾವು ಸಂತೋಷಪಡುತ್ತೇವೆ. ಶ‍್ರೀರಾಮಕೃಷ್ಣರು ನಿಗರ್ವವನ್ನು ಕಲಿಯಬೇಕಾದರೆ ಈ ಉದಾಹರಣೆ ಯನ್ನು ಜ್ಞಾಪಕದಲ್ಲಿ ಇಟ್ಟುಕೋ ಎನ್ನುತ್ತಾರೆ. ರಾತ್ರಿ ಮಿಂಚು ಹುಳು ಗಾಢಾಂಧಕಾರದಲ್ಲಿ ತಾರಾಡುತ್ತಿದ್ದಾಗ ನಾನೇ ಜಗತ್ತಿಗೆ ಬೆಳಕು ಕೊಡುತ್ತಿರುವೆ ಎಂದು ಭಾವಿಸಬಹುದು. ಚಂದ್ರ ಹುಟ್ಟಿದೊಡನೆ, ಮಿಂಚಿನ ಕಾಂತಿಯನ್ನು ಕೇಳುವವರಿಲ್ಲ. ಆಗ ಚಂದ್ರ ತಾನೆ ಜಗತ್ತನ್ನು ಬೆಳಗು ತ್ತಿರುವೆ ಎಂದು ಭಾವಿಸುತ್ತಾನೆ. ಬೆಳಗಾಯಿತು, ಪೂರ್ವ ದಿಕ್ಕಿನಲ್ಲಿ ಸೂರ್ಯ ಏಳುತ್ತಾನೆ. ಇನ್ನು ಮೇಲೆ ಚಂದ್ರನನ್ನು ಯಾರೂ ಕೇಳುವಂತಿಲ್ಲ. ಹಾಗೆಯೆ ನಮಗೆ ಇರುವವನಿಗಿಂತ ಜಾಸ್ತಿ ಇರುವವ ನೊಂದಿಗೆ ಹೋಲಿಸಿಕೊಂಡಾಗ ನಮ್ಮಲ್ಲಿರುವುದು ಎಷ್ಟು ಅಲ್ಪ ಎಂದು ಗೊತ್ತಾಗುವುದು. ನಿಗರ್ವಿಯಾದ ಮನುಷ್ಯ ತನ್ನಲ್ಲಿರುವ ವಿಷಯಗಳನ್ನು ತಾನು ಹೇಳಿಕೊಳ್ಳುವುದಿಲ್ಲ, ಅದನ್ನು ಜಾಹಿರಾತು ಪಡಿಸುವುದಕ್ಕೆ ಹೋಗುವುದಿಲ್ಲ. ಅದನ್ನು ಹೇಳಿಕೊಳ್ಳಲೇಬೇಕೆ ಮತ್ತೊಬ್ಬನಿಗೆ ಗೊತ್ತಾ ಗುವುದಕ್ಕೆ? ಮಲ್ಲಿಗೆ ತಾನಿರುವೆ ಇಲ್ಲಿ, ನನಗೆ ಕಂಪಿದೆ ಎಂದು ಹೇಳಿಕೊಳ್ಳಲೇ ಬೇಕಾಗಿಲ್ಲ. ಅದು ಕಾಣದೆ ಎಲೆಯ ಮರೆಯ ಹಿಂದಿದ್ದರೂ ತನ್ನ ಪರಿಮಳದಿಂದ ಎಲ್ಲರನ್ನೂ ಆಕರ್ಷಿಸುವುದು. ಹಾಗೆಯೇ ಅಮಾನಿ.

ಅವನಲ್ಲಿ ಢಂಬಾಚಾರವಿಲ್ಲ. ಕೀಳುಮಟ್ಟದಲ್ಲಿರುವವನು ತನ್ನಲ್ಲಿರುವ ಸ್ವಲ್ಪ ಧಾರ್ಮಿಕ ಪಿಪಾಸೆ ಯನ್ನು ಹೊರಗೆ ತೋರ್ಪಡಿಸಿಕೊಳ್ಳುವುದಕ್ಕೆ ಉಪಯೋಗಿಸಿಕೊಳ್ಳುವನು. ಇಂತಹ ಪ್ರವೃತ್ತಿಯವನು ಸ್ವಲ್ಪ ಧ್ಯಾನ ಮಾಡುವನು, ಸುತ್ತಲೂ ಯಾರಾದರೂ ಅದನ್ನು ನೋಡದೆ ಇದ್ದರೆ ಮಾಡಿದ್ದು ವ್ಯರ್ಥ ಎಂದು ಭಾವಿಸುವನು. ಅವನಲ್ಲಿ ಬೇಕಾದಷ್ಟು ಬಾಹ್ಯ ಆಡಂಬರಗಳೂ ಇರುತ್ತವೆ. ನಾಮ, ಮುದ್ರೆ, ವಿಭೂತಿ, ಮುಂತಾದುವುಗಳನ್ನು ಬಲವಾಗಿ ಬಳಿದುಕೊಳ್ಳುತ್ತಾನೆ. ಚೆನ್ನಾಗಿರುವ ರೇಷ್ಮೆ ಜರತಾರಿ ಬಟ್ಟೆ ಬರೆಗಳನ್ನು ಹಾಕಿಕೊಳ್ಳುತ್ತಾನೆ. ಕೊರಳಲ್ಲಿ ತುಳಸಿಮಣಿ, ರುದ್ರಾಕ್ಷಿ ಸರ ಇವುಗಳೆಲ್ಲ ಇರುವುವು. ಅವನಲ್ಲಿ ಇರುವುದೆಲ್ಲ ಹೊರಗಿನ ಮೆರವಣಿಗೆಗೆ. ಆದರೆ ಢಂಬಾಚಾರವಿಲ್ಲದವನು, ತನ್ನ ಸಾತ್ತ್ವಿಕ ಗುಣಗಳನ್ನು ಮತ್ತು ದೈವಭಕ್ತಿಯನ್ನು ಸಾಧ್ಯವಾದಷ್ಟು ಗೋಪ್ಯವಾಗಿಡುವನು. ಅದನ್ನು ವ್ಯಕ್ತಪಡಿಸು ವುದು ಏನೋ ಒಂದು ಅಪಚಾರವನ್ನು ಮಾಡುವಂತೆ ಎಂದು ಭಾವಿಸುವನು.

ಅವನು ಇತರರಿಗೆ ಹಿಂಸೆಯನ್ನು ಕೊಡುವುದಿಲ್ಲ. ತಾನು ಏನುಬೇಕಾದರೂ ಸಹಿಸುವನೆ ಹೊರತು ಇತರರಿಗೆ ಸಂಕಟವನ್ನು ಕೊಡುವವನು ಅವನಲ್ಲ. ಇತರರು ಅವನಿಗೆ ಕಷ್ಟವನ್ನು ಸಂಕಟವನ್ನು ತಂದೊಡ್ಡಿದರೂ ಅವನ್ನು ಅನುಭವಿಸುವನು. ಇದಕ್ಕಾಗಿ ಅವರ ಮೇಲೆ ಅವನು ಮನಸ್ಸಿನಲ್ಲಿ ಯಾವ ಕೆಟ್ಟ ಭಾವನೆಯನ್ನೂ ಕಲ್ಪಿಸಿಕೊಳ್ಳುವುದಿಲ್ಲ. ಎಲ್ಲಾ ದೇವರ ಇಚ್ಛೆಯಂತೆ ಮಾಡುವನು. ತನ್ನ ಕೆಲವು ಲೋಪ ದೋಷಗಳನ್ನು ನೇರ ಮಾಡಲು ದೇವರು ಅವರ ಮೂಲಕ ನನಗೆ ಈ ಕಷ್ಟವನ್ನು ಕೊಟ್ಟಿರಬಹುದು ಎಂದು ಭಾವಿಸುವನೆ ಹೊರತು, ಅವರನ್ನೇ ದೂರುವುದಕ್ಕೆ ಹೋಗುವುದಿಲ್ಲ. 

ಅವನಲ್ಲಿ ಆರ್ಜವ ಮನೆಮಾಡಿಕೊಂಡಿರುವುದು. ಅವನಲ್ಲಿ ಒಳಗೆ ಒಂದು, ಹೊರಗೆ ಒಂದು ಇಲ್ಲ. ಗ್ಲಾಸಿನ ಬೀರು ಹಿಂದೆ ಇಟ್ಟಿರುವುದೆಲ್ಲಾ ಹೇಗೆ ಕಾಣುವುದೊ ಹಾಗೆ ಅವನಲ್ಲಿ ಪಾರದರ್ಶಿಕತೆ ಇದೆ. ಅವನೇನು ಎಂಬುದನ್ನು ತತ್​ಕ್ಷಣವೇ ಹೇಳಿಬಿಡಬಹುದು. ಇತರರಲ್ಲಿರುವ ಕೆಟ್ಟದನ್ನು ನೋಡಿ ಆಡಿಕೊಳ್ಳುವುದೂ ಇಲ್ಲ. ಪಾಪ, ಇವೆಲ್ಲ ಇನ್ನೂ ಕಾಯಿ, ಅದಕ್ಕೆ ಹುಳಿಯಾಗಿದೆ, ಕೊನೆಗೆ ಎಲ್ಲಾ ಸಿಹಿಯಾಗುವುದು ಎಂಬ ಉದಾರ ದೃಷ್ಟಿಯಿಂದ ನೋಡುತ್ತಾನೆ. ಇಲ್ಲದ ಒಳ್ಳೆಯ ಗುಣಗಳನ್ನು ಒಂದು ವ್ಯಕ್ತಿಯಲ್ಲಿ ಆರೋಪಮಾಡಿ ಅವನನ್ನು ಇಂದ್ರ ಚಂದ್ರ ದೇವೆಂದ್ರ ಎಂದು ಹೊಗಳುವುದಕ್ಕೆ ಹೋಗುವುದಿಲ್ಲ. ಅವನಿಗೆ ಯಾರಿಂದಲೂ ಏನೂ ಬೇಕಾಗಿಲ್ಲ. ಅದಕ್ಕಾಗಿ ಅವನು ಹೊಗಳಿಕೆಯ ಲಂಚವನ್ನು ಕೊಡುವುದಕ್ಕೆ ಹೋಗುವುದಿಲ್ಲ.

ಅವನಲ್ಲಿ ಆಚಾರ್ಯೋಪಾಸನೆಯನ್ನು ನೋಡುತ್ತೇವೆ. ಗುರುಗಳು, ಹಿರಿಯರು ಇವರನ್ನು ಗೌರವದಿಂದ ನೋಡುತ್ತಾನೆ. ಎಲ್ಲಾ ಕಡೆಯಲ್ಲಿಯೂ ಗುರು ಹಿರಿಯರಿಗೆ ಗೌರವ ತೋರುವುದು ನಮ್ಮಲ್ಲಿ ಹಿಂದಿನಿಂದ ಬಂದ ಒಂದು ಒಳ್ಳೆಯ ಗುಣ. ಆದರೆ ಆಧ್ಯಾತ್ಮಿಕ ಜೀವನದಲ್ಲಿ ಇದು ಮತ್ತೂ ವಿಶೇಷವಾಗಿದೆ. ಗುರುಸೇವೆಯಿಂದ ಶಿಷ್ಯ ಏನನ್ನು ಬೇಕಾದರೂ ಪಡೆಯಬಹುದು. ತನ್ನ ಮುಕ್ತಿಯನ್ನು ಕೂಡ ಕೇವಲ ಗುರು ಸೇವೆಯಿಂದಲೇ ಪಡೆದ ಶಿಷ್ಯರಿದ್ದಾರೆ. ಅಂತಹ ಮಹಾನ್ ಗುರುವಿನ ಸೇವೆ ಮಾಡುವಾಗ ಅವರ ಮನಸ್ಸಿನ ನಿಕಟ ಪರಿಚಯ ನಮಗೆ ಆಗುವುದು. ಸದ್ಗುರು ಶಿಷ್ಯ ಜಾಗ್ರತನಾಗಲಿ ಎಂದು ಆಶಿಸಿದರೆ ಶಿಷ್ಯ ಉದ್ಧಾರವಾಗುವುದರಲ್ಲಿ ಸಂದೇಹವಿಲ್ಲ.

ಅವನಲ್ಲಿ ಶೌಚವನ್ನು ನೋಡುತ್ತೇವೆ. ಅವನು ಹಾಕಿಕೊಳ್ಳುವ ಬಟ್ಟೆ ಬರೆ ಸರಳವಾಗಿರುತ್ತದೆ. ಆದರೆ ಅದು ಕೊಳೆಯಾಗಿರುವುದಿಲ್ಲ, ಶುಭ್ರವಾಗಿರುವುದು, ಸರಳವಾಗಿರುವುದು. ಹರಿದ ಬಟ್ಟೆ ಕೊಳಕು ಬಟ್ಟೆಯನ್ನು ಹಾಕಿಕೊಳ್ಳುವುದಿಲ್ಲ. ಹೊರಗಿನ ವೇಷ ಭೂಷಣಗಳಲ್ಲಿ ಹೇಗೆ ಶುಭ್ರವಾಗಿ ಸರಳವಾಗಿರುವನೊ ಹಾಗೆಯೆ ಅವನ ಆಂತರಿಕ ಜೀವನದಲ್ಲಿಯೂ. ಅವನು ಮೈಲಿಗೆಯಾಗುವ ಕೆಲಸ ಮಾಡುವುದಿಲ್ಲ, ಅಂತಹ ಆಲೋಚನೆಯನ್ನು ಮನಸ್ಸಿನಲ್ಲೂ ಮಾಡುವುದಿಲ್ಲ. ಕೊಳೆಬಟ್ಟೆಯನ್ನು ಉಡುವುದು ಎಷ್ಟು ಕೆಟ್ಟದ್ದೊ ಕೊಳೆಮನಸ್ಸು ಅದಕ್ಕಿಂತಲೂ ಕೆಟ್ಟದ್ದು. ಇವನ ಒಳಗೆ ಹೊರಗೆ ಎಲ್ಲಾ ಶುಭ್ರವಾಗಿರುವುದು.

ಇವನು ದೇವರ ಕಡೆ ಹೋಗುವಾಗ ಸ್ಥಿರವಾಗಿರುವನು. ಅವನಲ್ಲಿರುವ ವೈರಾಗ್ಯ ಮತ್ತು ಭಕ್ತಿ ತಾತ್ಕಾಲಿಕವಾದುದಲ್ಲ. ಗುರಿ ಸೇರುವವರೆಗೆ ಆತ ಒಂದೇ ಸಮನಾಗಿ ಸಾಧನೆ ಮಾಡಿಕೊಂಡು ಹೋಗುವನು. ಎಷ್ಟೇ ಆತಂಕಗಳು ಬರಲಿ ತಾನು ಮಾಡುವುದನ್ನು ಬಿಡುವುದಿಲ್ಲ, ಹಿಂದೆ ಹೋಗು ವುದಿಲ್ಲ ಅಡ್ಡಹಾದಿ ಹಿಡಿಯುವುದಿಲ್ಲ. ಅವನು ಆತ್ಮವನ್ನು ನಿಗ್ರಹಿಸುವನು. ಮನಸ್ಸನ್ನು ಚೆನ್ನಾಗಿ ತನ್ನ ವಶದಲ್ಲಿ ಇಟ್ಟುಕೊಂಡಿರುವನು. ಇಂದ್ರಿಯಗಳನ್ನು ಹೇಳಿದಂತೆ ಕೇಳುವಂತೆ ಮಾಡಿ ಕೊಂಡಿರು ವನು.

\begin{verse}
ಇಂದ್ರಿಯಾರ್ಥೇಷು ವೈರಾಗ್ಯಮನಹಂಕಾರ ಏವ ಚ~।\\ಜನ್ಮಮೃತ್ಯುಜರಾವ್ಯಾಧಿದುಃಖದೋಷಾನುದರ್ಶನಮ್ \versenum{॥ ೮~॥}
\end{verse}

{\small ಇಂದ್ರಿಯಗಳ ವಿಷಯದಲ್ಲಿ ವೈರಾಗ್ಯ. ಅಹಂಕಾರವಿಲ್ಲದಿರುವುದು, ಜನ್ಮ, ಮೃತ್ಯು ಜರಾ, ವ್ಯಾಧಿ,—ಇವುಗಳಲ್ಲಿ ಒಂದು ದೋಷವನ್ನು ನೋಡುವುದು–}

ಇಂದ್ರಿಯಕ್ಕೆ ಸಂಬಂಧಪಟ್ಟ ವಿಷಯವನ್ನು ನೋಡಿದೊಡನೆ ಅದನ್ನು ಅವನು ಅನುಸರಿಸಿ ಹೋದರೆ, ಎಂತಹ ಪರಿಸ್ಥಿತಿಗೆ ಬರುವನು ಎಂಬುದನ್ನು ಮುಂಚೆಯೇ ಊಹಿಸಬಲ್ಲ ಅವನು. ಇಂದ್ರಿಯ ವಿಷಯವನ್ನು ಅನುಭವಿಸಿ ತೃಪ್ತರಾದವರು ಯಾರೂ ಇಲ್ಲ. ಅದನ್ನು ಅನುಭವಿಸಿದರೆ ಪುನಃ ಅನುಭವಿಸಬೇಕೆಂಬ ಆಸೆಗೆ ತುತ್ತಾಗುತ್ತೇವೆ, ಅದರ ಗುಲಾಮರಾಗುತ್ತೇವೆ. ಗೂಟಕ್ಕೆ ಕಟ್ಟಿಸಿಕೊಂಡವರಂತೆ ಆಗುತ್ತೇವೆ ನಾವು. ಅದರಿಂದ ಕಿತ್ತುಕೊಂಡು ಬರುವಂತೆ ಇಲ್ಲ. ವಿಷಯದ ಗುಹೆಗೆ ಹೋದವರೇ ಹೆಚ್ಚು. ಹಿಂದಿರುಗಿ ಬಂದವರು ಅಪರೂಪ ಅಲ್ಲಿಂದ. ಅಲ್ಲಿರುವ ಸಿಂಹ ಎಲ್ಲರನ್ನೂ ಕಬಳಿಸಿಬಿಡುವುದು.

ಅವನಲ್ಲಿ ಅಹಂಕಾರವಿರುವುದಿಲ್ಲ. ಯಾವುದಕ್ಕೂ ಹೆಮ್ಮೆ ತಾಳುವುದಿಲ್ಲ. ಪಾಂಡಿತ್ಯ, ಐಶ್ವರ್ಯ, ಅಧಿಕಾರ, ಯಾವುದಾದರೂ ಇರಲಿ, ಅವನು ನೋಡುವುದು ಭಗವತ್ ಸಾಕ್ಷಾತ್ಕಾರ ದೃಷ್ಟಿಯಿಂದ. ಇವುಗಳಾವುವೂ ಅವನಿಗೆ ಸಹಾಯಮಾಡಲಾರವು. ಅದಕ್ಕೆ ಬೇಕಾದರೆ ಆತಂಕವನ್ನು ತಂದೊಡ್ಡ ಬಲ್ಲವೆ ಹೊರತು ಸಹಾಯಮಾಡಲಾರವು. ಒಂದುವೇಳೆ ಆಧ್ಯಾತ್ಮಿಕ ದೃಷ್ಟಿಯಿಂದ ಇಲ್ಲದೇ ಇದ್ದರೂ ಬರೀ ಐಶ್ವರ್ಯ ಅಧಿಕಾರ ಪಾಂಡಿತ್ಯಕ್ಕೆ ಹೆಮ್ಮೆ ಕೊಚ್ಚಿಕೊಳ್ಳುವಂತಹುದು ಏನಿದೆ? ಈ ಜೀವನದಲ್ಲಿ ಒಬ್ಬನಿಗಿಂತ ಮತ್ತೊಬ್ಬ ಪಂಡಿತನಿರುವನು, ಅವನೊಂದಿಗೆ ಹೋಲಿಸಿಕೊಂಡರೆ ನಮಗೆ ತಿಳಿದಿರುವುದೆಲ್ಲ ಕೆಲಸಕ್ಕೆ ಬಾರದವು. ಅದಕ್ಕೆ ನಾವು ಅಷ್ಟೊಂದು ಆನಂದ ಪಡುವುದೇನು? ಸಮುದ್ರದ ಒಳಗಿರುವ ಮುತ್ತು ರತ್ನ ಸಿಕ್ಕುವುದಕ್ಕೆ ಶ್ರಮಪಡಬೇಕೆ ಹೊರತು ಜೇಬಿನಲ್ಲಿರುವ ಕಪ್ಪೆಚಿಪ್ಪುಗಳಿಗೆ ಕೃತಾರ್ಥರಾಗಿ ಹೋಗಕೂಡದು. ಹೀಗೆ ಆಗುವವನೂ ಅಲ್ಲ ಜ್ಞಾನಿ.

ಅವನು ಜನ್ಮ ಮೃತ್ಯು ಜರೆ ವ್ಯಾಧಿ ಇವುಗಳಲ್ಲಿ ದೋಷವನ್ನು ನೋಡುತ್ತಾನೆ. ಇವುಗಳು ಅವನನ್ನು ಮರುಳುಗೊಳಿಸಲಾರವು. ಅವನು ಜನ್ಮದಲ್ಲಿಯೇ ದೋಷವನ್ನು ನೋಡುತ್ತಾನೆ. ಏನೊ ಹುಟ್ಟಿದೆ, ಬೇಕಾದಷ್ಟು ಸಂತೋಷಪಡಬಹುದು ಎಂದು ಭಾವಿಸುವುದಿಲ್ಲ. ಪುನಃ ಬಂದೆನಲ್ಲ, ಇನ್ನು ನನಗೆ ಏನೇನು ಕಾದಿವೆಯೊ ಎಂದು ಅನುಮಾನಿಸುವುದು ಅವನ ಪ್ರಕೃತಿ. ಸುಖವೆಂಬುದು ಪ್ರಕೃತಿ ನಮ್ಮನ್ನು ತನ್ನ ಬಲೆಯಲ್ಲಿ ಕೆಡವಲು ಹಾಕಿರುವ ಆಸೆಯ ತಿಂಡಿ. ಅದನ್ನು ತಿನ್ನಲು ಹೋದೊಡನೆಯೆ ಹಿಂದಿನಿಂದ ಬಾಗಿಲು ಮುಚ್ಚಿಕೊಳ್ಳುವುದು. ನಾವು ಚೀಪಿದ ಪ್ರತಿಯೊಂದು ಸುಖವೂ ತನಗೆ ಸಂಬಂಧಪಟ್ಟ ದುಃಖವನ್ನು ಬಿಟ್ಟು ಹೋಗಿದೆ. ನಾವು ಎಷ್ಟು ಜೋಪಾನವಾಗಿ ವರ್ತಿಸಿದರೂ ನಮಗೆ ಅರಿವಾಗದೆ ಸಂಸಾರದ ಗೋಜಿಗೆ ಸಿಕ್ಕಿಕೊಳ್ಳುತ್ತೇವೆ. ಇದೊಂದು ಮಸಿಯಿಂದ ತುಂಬಿದ ಕೋಣೆ ಯಲ್ಲಿ ವಾಸಮಾಡುವಂತೆ. ಎಷ್ಟೇ ಜೋಪಾನವಾಗಿದ್ದರೂ ಕೂತಲ್ಲಿಯೋ ನಿಂತಲ್ಲಿಯೋ ಮಸಿ ತಾಕು ವುದು ನಮ್ಮ ಬಟ್ಟೆಗೆ. ಸಂಸಾರಕ್ಕೆ ಬಂದರೆ ಸಂತೋಷದಿಂದ ಕುಣಿಯುವುದೇನೂ ಇಲ್ಲ. ಹೆಚ್ಚು ಮಸಿ ತಾಗದೆ ವರ್ತಿಸುವುದು ಹೇಗೆ ಎಂಬ ಜವಾಬ್ದಾರಿಯಲ್ಲೇ ಮುಳುಗಿರಬೇಕಾಗಿದೆ.

ಮೃತ್ಯು ಬಂದು ಇನ್ನು ಆದರೂ ನಮ್ಮನ್ನು ಜೀವನದ ಕಷ್ಟಗಳಿಂದ ಬಿಡಗಡೆ ಮಾಡುವುದು ಎಂದು ಭಾವಿಸುವುದಕ್ಕೆ ಆಗುವುದಿಲ್ಲ. ಇದೇ ಕೊನೆಯ ಮೃತ್ಯುವಾದರೆ ಪರವಾಗಿಲ್ಲ. ಆದರೆ ಈ ಮೃತ್ಯು ಮತ್ತೊಂದು ಜನ್ಮಕ್ಕೆ ಬೀಜವಾದರೆ, ಇನ್ನು ಎಲ್ಲೋ ಹುಟ್ಟಿ ಮತ್ತಷ್ಟು ಸುಖದುಃಖಗಳನ್ನು ಅನುಭವಿಸಬೇಕಾಗುವುದು! ಸಂಸಾರದಲ್ಲಿ ಆಸಕ್ತಿ ಕುಗ್ಗದೆ ಇದ್ದರೂ ಮೃತ್ಯು ಬರುವುದು, ನಾವೆಲ್ಲ ರನ್ನೂ ಬಿಟ್ಟು ಹೋಗಬೇಕಾಗುವುದು ಎಂಬ ಆಲೋಚನೆಯೊಂದೇ ಸಾಕು ನಮ್ಮ ಜೀವನವೆಲ್ಲ ಸಪ್ಪೆ ಮಾಡಲು.

ಜರೆ ಎಂಬ ದುಃಖ ಬೇರೆ ಇರುವುದು. ಇದರಷ್ಟು ಘೋರವಾದ ದುಃಖ ಮತ್ತೊಂದಿಲ್ಲ. ಒಂದು ಕಾಲದಲ್ಲಿ ನಮ್ಮ ದೇಹದಾರ್ಢ್ಯ, ನಮ್ಮ ಯೌವನ ಸೌಂದರ್ಯಕ್ಕೆ ಸಂತೋಷ ಪಟ್ಟವರು ನಾವು. ಅದನ್ನು ನೋಡಿ ಆನಂದಪಟ್ಟೆವು. ಈಗ ಅವೆಲ್ಲ ಗತಕಾಲದ ಚರಿತ್ರೆ ಆಗಿವೆ. ನಮ್ಮ ದೇಹದಾರ್ಢ್ಯ ಉಡುಗಿದೆ. ಹೆಮ್ಮೆ ಕೊಚ್ಚಿಕೊಳ್ಳುತ್ತಿದ್ದ ಸೌಂದರ್ಯ ಬಾಡಿಹೋಗಿದೆ. ಏಳಲಾರೆವು, ಕುಳಿತಕೊಳ್ಳ ಲಾರೆವು. ನಾವು ನಮ್ಮ ಮಕ್ಕಳು ಮರಿಗಳಿಗೆ ಬೇಡವಾಗಿದ್ದೇವೆ. ನಮ್ಮಿಂದ ಅವರಿಗೆ ಇನ್ನು ಏನೂ ಸಿಕ್ಕುವಂತಿಲ್ಲ. ಬಹುಶಃ ನಾವು ಹೋದರೆ ಅವರಿಗೆ ಲಾಭ. ಇಂತಹ ಪರಿಸ್ಥಿತಿಯಲ್ಲಿ ಜೀವನ ತಳ್ಳುವುದು ನರಕ ರುಚಿ ನೋಡಿದಂತೆ.

ಇನ್ನು ದೇಹವನ್ನು ಹೊತ್ತವರಿಗೆಲ್ಲ ವ್ಯಾಧಿ ಎಂಬುದೊಂದು ಇದೆ. ಒಬ್ಬೊಬ್ಬನಿಗೆ ಒಂದೊಂದು ಕಾಲದಲ್ಲಿ ಒಂದೊಂದು ವ್ಯಾಧಿ ಹಿಡಿಯುವುದು. ಕೆಲವಂತೂ ನಮ್ಮ ಜೊತೆಯಲ್ಲಿಯೇ ಹೋಗು ವುವು. ಏನು ಮಾಡಿದರೂ, ಎಲ್ಲಿಗೆ ಹೋದರೂ, ಯಾವ ವೈದ್ಯನ ಸಹಾಯ ಪಡೆದರೂ ಅವುಗಳಿಂದ ತಪ್ಪಿಸಿಕೊಳ್ಳುವಂತಿಲ್ಲ. ಅನುಭವಿಸುವ ನಮಗೆ ಅದು ಯಾತನಾಮಯ, ನೋಡುವವರಿಗೆ ಅದು ಯಾತನಾಮಯ. ದಾನಕ್ಕೆ ಧರ್ಮಕ್ಕೆ ಕೊಡದಿದ್ದರೂ,ಇರುವ ಆಸ್ತಿಯನ್ನೆಲ್ಲ, ವ್ಯಾಧಿಯ ನಿವಾರಣೆಗೆ ಖರ್ಚುಮಾಡಲು ಸಿದ್ಧರಾಗಿರುವರು.

ದೇಹದ ಮೂಲಕ ಸ್ವಲ್ಪ ಸುಖಪಟ್ಟಿದ್ದಕ್ಕೆ ಅಷ್ಟೊಂದು ವ್ಯಾಧಿಗಳನ್ನು ಅನುಭವಿಸಬೇಕಾಗಿದೆ. ವ್ಯಾಧಿ ಕಾಡುವಾಗ ಅಸಹಾಯಕ ಸ್ಥಿತಿಗೆ ಬರುತ್ತೇವೆ ನಾವು. ಪರಾಧೀನತೆ ಏನು ಎಂಬುದು ಆಗ ಗೊತ್ತಾಗುವುದು. ಎಷ್ಟಿದ್ದರೇನು? ಪ್ರತಿಯೊಂದಕ್ಕೂ ಮತ್ತೊಬ್ಬರನ್ನು ಗೋಗರೆಯಬೇಕು.

ದುಃಖವೆಂಬ ಮತ್ತೊಂದು ಶತ್ರುವಿದೆ ಜೀವನದಲ್ಲಿ ಕಾಡುವುದಕ್ಕೆ. ವ್ಯಾಧಿಗಳಲ್ಲಿ ಹೇಗೆ ಬಹು ವಿಧಗಳಿವೆಯೊ ಹಾಗೆಯೆ ದುಃಖದಲ್ಲಿ ಬಹು ವಿಧಗಳಿವೆ. ಇದು ಯಾರಿಗೂ ಬಿಟ್ಟಿದ್ದಲ್ಲ. ಪ್ರತಿ ಯೊಬ್ಬನೂ ಯಾವುದಕ್ಕಾದರೂ ಕೊರಗುತ್ತಿರುವನು. ಒಂದು ಇದ್ದರೆ ಮತ್ತೊಂದು ಇಲ್ಲ. ಎಲ್ಲ ಇದೆ ಎಂದು ಯಾರು ಹೇಳಬಲ್ಲರು? ಈ ದುಃಖ ಬಡವನನ್ನು ಬಿಟ್ಟಿಲ್ಲ, ಪಂಡಿತನನ್ನು ಬಿಟ್ಟಿಲ್ಲ, ಪಾಮರನನ್ನು ಬಿಟ್ಟಿಲ್ಲ. ಒಬ್ಬೊಬ್ಬರನ್ನು ಒಂದೊಂದು ಬಗೆಯಾಗಿ ಕಾಡುತ್ತಿದೆ.

ಇವುಗಳೆಲ್ಲ ಇರುವ ಜೀವನದಲ್ಲಿ ವಿಚಾರಪರನಿಗೆ ದೋಷವೇ ಕಾಣುವುದು. ಈ ಸತ್ಯವನ್ನು ಮರೆಮಾಚುವುದಕ್ಕೆ ಬಳಿದ ಬಣ್ಣ–ಸುಖವೆಂಬುದು. ಒರೆಗೆ ತಿಕ್ಕಿದೊಡನೆ ಹಾಕಿರುವ ಬಣ್ಣ ಗೊತ್ತಾಗಿ ಬಿಡುವುದು.

\begin{verse}
ಅಸಕ್ತಿರನಭಿಷ್ವಂಗಃ ಪುತ್ರದಾರಗೃಹಾದಿಷು~।\\ನಿತ್ಯಂ ಚ ಸಮಚಿತ್ತತ್ವಮಿಷ್ಟಾನಿಷ್ಟೋಪಪತ್ತಿಷು \versenum{॥ ೯~॥}
\end{verse}

{\small ಹೆಂಡತಿ, ಮನೆ, ಮಕ್ಕಳು, ಮೊದಲಾದುವುಗಳಲ್ಲಿ ಆಸಕ್ತಿ, ಮತ್ತು ತನ್ಮಯತೆ ಇಲ್ಲದಿರುವುದು, ಇಷ್ಟ ಅನಿಷ್ಟಗಳು ಪ್ರಾಪ್ತವಾದಾಗ ಯಾವಾಗಲೂ ಸಮಚಿತ್ತನಾಗಿರುವುದು–}

ಸಂಸಾರದಲ್ಲಿರುತ್ತಾನೆ, ಆದರೆ ಸಂಸಾರಕ್ಕೆ ಅಂಟಿಕೊಂಡಿರುವುದಿಲ್ಲ. ನೀರಿನಮೇಲೆ ಕಮಲದ ಎಲೆ ಇದೆ. ಆದರೆ ನೀರಿಗೆ ಅಂಟಿಕೊಂಡಿಲ್ಲ. ಅವನು ಮನಸ್ಸಿನಲ್ಲಿ ಇವುಗಳಿಂದ ಕೈತೊಳೆದು ಕೊಂಡಿರುವನು. ಈ ಸಂಬಂಧ ತಾತ್ಕಾಲಿಕ. ಈ ಜನ್ಮದಲ್ಲಿ ಈ ದೇಹಕ್ಕೆ ಸಂಬಂಧಪಟ್ಟವರು ಇವರು. ನಾವು ಬರುವುದಕ್ಕೆ ಮುಂಚೆ ಇವರ ಸಂಬಂಧವಿರಲಿಲ್ಲ. ಸತ್ತ ಮೇಲೆ ಇವರಾರೂ ನಮ್ಮ ಹಿಂದೆ ಬರುವವರಲ್ಲ. ತನ್ನ ಕರ್ತವ್ಯವನ್ನೇನೋ ಇವರಿಗೆ ಮಾಡುತ್ತಾನೆ. ಆದರೆ ಇವರಿಗೆ ಕಟ್ಟಿಕೊಂಡಿರು ವುದಿಲ್ಲ. ಅವನ ಪ್ರೀತಿಯಲ್ಲಿ ತನ್ಮಯತೆ ಇಲ್ಲ. ತನ್ಮಯತೆ ಎಂದರೆ ಅದರಲ್ಲಿ ಮುಳುಗಿಹೋಗಿರು ವುದು. ಅವನು ಹರಿವ ನೀರಿನಲ್ಲಿ ಸ್ನಾನಮಾಡುತ್ತಾನೆ. ಆದರೆ ಪ್ರವಾಹದಿಂದ ಕೊಚ್ಚಿಕೊಂಡು ಹೋಗದಂತೆ ನೋಡಿಕೊಳ್ಳುತ್ತಾನೆ.

ಜೀವನದಲ್ಲಿ ಒಬ್ಬನಿಗೆ ಬೇಕಾದುದು ಬರುವುದು; ಬೇಡವಾದುದೂ ಬರುವುದು. ಅವನು ಒಂದಕ್ಕೆ ಹಿಗ್ಗುವುದಿಲ್ಲ, ಮತ್ತೊಂದಕ್ಕೆ ಕುಗ್ಗುವುದಿಲ್ಲ. ಇವೆರಡನ್ನೂ ಎದುರಿಸುವುದಕ್ಕೆ ಮನಸ್ಸು ಸಿದ್ಧವಾಗಿರು ವುದು. ಇವುಗಳ ಸೆಳೆತಕ್ಕೆ ಅವನು ಬೀಳುವುದಿಲ್ಲ.

\begin{verse}
ಮಯಿ ಚಾನನ್ಯಯೋಗೇನ ಭಕ್ತಿರವ್ಯಭಿಚಾರಿಣೀ~।\\ವಿವಿಕ್ತದೇಶಸೇವಿತ್ವಮರತಿರ್ಜನಸಂಸದಿ \versenum{॥ ೧೦~॥}
\end{verse}

{\small ನನ್ನಲ್ಲಿ ಅನನ್ಯಯೋಗದ ಮೂಲಕ ಏಕನಿಷ್ಠೆಯಾದ ಭಕ್ತಿ, ಏಕಾಂತ ಸ್ಥಳದ ಸೇವನೆ, ಜನಸಮೂಹದಲ್ಲಿ ಅರುಚಿ–}

ಅವನಲ್ಲಿ ಭಗವಂತನಮೇಲೆ ಅನನ್ಯವಾದ ಭಕ್ತಿ ಇರುವುದು. ಒಂದೇ ಸಮನಾಗಿ ಭಗವಂತನನ್ನು ಚಿಂತಿಸುತ್ತಿರುವನು. ಬಿಟ್ಟು ಬಿಟ್ಟು ಭಗವಂತನನ್ನು ಚಿಂತಿಸುವುದಿಲ್ಲ. ಅವನ ಭಕ್ತಿಜೀವನಕ್ಕೆ ರಜವೇ ಇಲ್ಲ. ಉತ್ತರಮುಖಿ ಯಾವಾಗಲೂ ಉತ್ತರ ದಿಕ್ಕನ್ನು ಹೇಗೆ ತೋರಿಸುತ್ತಿರುವುದೋ ಹಾಗೆ ಅವನ ಮನಸ್ಸು ಸದಾ ದೇವರ ಕಡೆ ತಿರುಗಿರುವುದು.

ಅವನ ಭಕ್ತಿ ಏಕನಿಷ್ಠೆಯಿಂದ ಕೂಡಿದೆ. ಮತ್ತಾವುದನ್ನೂ ದೇವರಿಂದ ಪಡೆಯಬೇಕು ಎಂದು ಅವನು ದೇವರನ್ನು ಆಶಿಸುವುದಿಲ್ಲ. ಕೇವಲ ಪ್ರೀತಿಗಾಗಿ ಅವನನ್ನು ಆಶಿಸುವನು. ದೇವರು ಏನನ್ನು ಬೇಕಾದರೂ ಕೊಡಬಲ್ಲ ಎಂಬುದು ಅವನಿಗೆ ಗೊತ್ತಿದೆ. ಆದರೂ ಅವನು ಏನನ್ನೂ ಕೇಳುವುದಕ್ಕೆ ಹೋಗುವುದಿಲ್ಲ. ಬರೀ ಶುದ್ಧ ಭಕ್ತಿ ಮಾತ್ರ ಕೊಡು ಎಂದು ಬೇಡುತ್ತಾನೆ.

ಅವನು ಏಕಾಂತ ಸ್ಥಳವನ್ನು ಪ್ರೀತಿಸುವನು. ನಿಸ್ಸಂದೇಹವಾಗಿ ಭಗವಂತನನ್ನು ಚಿಂತಿಸುವುದಕ್ಕೆ ಏಕಾಂತ ಸ್ಥಳದಷ್ಟು ಶ್ರೇಷ್ಠವಾಗಿರುವುದು ಮತ್ತಾವುದು ಇಲ್ಲ. ಅಲ್ಲಿ ನಮ್ಮನ್ನು ಯಾರೂ ನೋಡುತ್ತಿಲ್ಲ. ಯಾರಿಗೂ ಸಂಕೊಚಪಡಬೇಕಾಗಿಲ್ಲ. ನಮ್ಮ ಹೃದಯವನ್ನೆಲ್ಲಾ ದೇವರ ಹತ್ತಿರ ಬಿಚ್ಚಬಹುದು. ನಮ್ಮ ಮನಸ್ಸನ್ನೆಲ್ಲಾ ನಾವು ಚೆನ್ನಾಗಿ ವಿಭಜನೆಮಾಡಿಕೊಳ್ಳಬಹುದು ಅಲ್ಲಿ. ಬಾಹ್ಯದ ಯಾವ ಸುದ್ದಿ ಸಮಾಚಾರವೂ ಮನಸ್ಸನ್ನು ಮುಟ್ಟಿ ಅದನ್ನು ಆಚೆಗೆ ಸೆಳೆಯಲಾರದು ಅಲ್ಲಿ.

ಅವನಿಗೆ ಜನಜಂಗುಳಿಯನ್ನು ಕಂಡರೆ ಆಗದು. ಮಂದಿಗಳು ಯಾವಾಗಲೂ ಲೌಕಿಕ ವಿಷಯ ಗಳನ್ನೇ ಮಾತನಾಡುವುದು. ಅವರ ಮಾತುಕತೆ ವೇಷಭೂಷಣಗಳೆಲ್ಲಾ ಸಾಂಸಾರಿಕತೆಯಿಂದ ತುಂಬಿ ತುಳುಕುತ್ತಿರುವುದು. ಅವನಿಗೆ ಬೇಡದೆ ಇದ್ದರೂ ಈ ಮಾತುಕತೆಗಳನ್ನು ಕೇಳಬೇಕಾದಾಗ ತುಂಬಾ ವ್ಯಥೆಯಾಗುವುದು.

ಅವನು ಜನಸಂಗವನ್ನೆಲ್ಲಾ ದೂರುತ್ತಾನೆ ಎಂದಲ್ಲ, ಯಾರು ಭಗವಂತನ ವಿಷಯವನ್ನು ಮಾತನಾಡುತ್ತಾರೆಯೋ ಅವರನ್ನು ಪ್ರೀತಿಸುತ್ತಾನೆ. ಅವರ ಸಂಗವನ್ನು ಕೆಲವುವೇಳೆ ಆಶಿಸುತ್ತಾನೆ. ಆದರೆ ಯಾವಾಗಲೂ ಅಲ್ಲ. ದೇವನೆಡೆಗೆ ಹೋಗುವ ಯಾತ್ರಿಕ ಒಂಟಿ. ಅವನು ಜೊತೆ ಕಟ್ಟಿಕೊಂಡು ಹೋಗುವುದಿಲ್ಲ. ಎಂತಹ ಒಳ್ಳೆಯ ಜೊತೆಯಾದರೂ ಯಾವಾಗಲೂ ಇದ್ದರೆ ಅದೊಂದು ಬಂಧನ ವಾಗುವುದು. ಭಕ್ತ ಏಕಾಂತ ಪ್ರಿಯ, ನಿರ್ಜನಪ್ರಿಯ.\enginline{}

\begin{verse}
ಅಧ್ಯಾತ್ಮಜ್ಞಾನನಿತ್ಯತ್ವಂ ತತ್ತ್ವಜ್ಞಾನಾರ್ಥದರ್ಶಮ್~।\\ಏತಜ್ಜ್ಞಾನಮಿತಿ ಪ್ರೋಕ್ತಮಜ್ಞಾನಂ ಯದತೋಽನ್ಯಥಾ \versenum{॥ ೧೧~॥}
\end{verse}

{\small ಅಧ್ಯಾತ್ಮಜ್ಞಾನದಲ್ಲಿ ನಿಷ್ಠೆ, ತತ್ತ್ವಜ್ಞಾನದ ಅರ್ಥದಮೇಲೆ ಆಲೋಚನೆ ಇವನ್ನೇ ಜ್ಞಾನ ಎಂದು ಹೇಳುವುದು. ಯಾವುದು ಇದಕ್ಕಿಂತ ಬೇರೆ ಆಗಿರುವುದೋ ಅದು ಅಜ್ಞಾನ.}

ಆಧ್ಯಾತ್ಮಿಕ ಜ್ಞಾನದಲ್ಲಿ ಶ್ರದ್ಧೆಯುಳ್ಳವನು ಅವನು. ಅದಕ್ಕೆ ಸಂಬಂಧಪಟ್ಟ ಕೆಲಸಗಳನ್ನು ಬಿಡದೆ ಮಾಡುತ್ತಿರುವನು. ಅದಕ್ಕೆ ಸಂಬಂಧಪಟ್ಟ ಪುಸ್ತಕಗಳನ್ನು ಓದುವನು. ಪ್ರಾರ್ಥನೆ ಮಾಡುವನು. ಧ್ಯಾನ ಜಪತಪಾದಿಗಳನ್ನು ಮಾಡುತ್ತಿರುವನು. ಅದಕ್ಕೆ ಸಂಬಂಧಪಟ್ಟ ಮಾತುಕತೆಗಳಲ್ಲಿ ಅವನು ನಿರತನಾಗಿರುವನು. ಒಂದು ದಿನವೂ ತಪ್ಪದೇ ಇವುಗಳನ್ನೆಲ್ಲಾ ಮಾಡುವನು.

ತತ್ತ್ವಜ್ಞಾನದ ಅರ್ಥದಮೇಲೆ ಆಲೋಚನೆಮಾಡುವನು. ಆಧ್ಯಾತ್ಮಿಕ ಜೀವನಕ್ಕೆ ಸಂಬಂಧಪಟ್ಟ ವಿಷಯಗಳನ್ನು ಒಮ್ಮೆ ಓದಿಬಿಟ್ಟಕೆ ಸಾಲದು. ಓದಿರುವುದನ್ನು ಚೆನ್ನಾಗಿ ತಿಳಿದುಕೊಳ್ಳಬೇಕು. ಅದನ್ನು ನಮ್ಮದನ್ನಾಗಿ ಮಾಡಿಕೊಳ್ಳಬೇಕು. ಹಾಗೆ ಮಾಡಬೇಕಾದರೆ ಆ ಭಾವನೆಯನ್ನೇ ಕುರಿತು ಮೆಲುಕು ಹಾಕುತ್ತಿರಬೇಕು. ಇದೇ ಮನನ. ಮೊದ ಮೊದಲು ನಮಗೆ ಅರ್ಥವಾಗದುದು ಕ್ರಮೇಣ ಮನನದಿಂದ ಅರ್ಥವಾಗತೊಡಗುವುದು. ಇದು ಗಟ್ಟಿಯಾದ ಕಾಳನ್ನು ನೀರಿನಲ್ಲಿ ಹಾಕಿ ನೆನೆಸಿದಂತೆ, ನೆನೆದಮೇಲೆ ಕಾಳು ಮೃದುವಾಗುವುದು. ಅದರಂತೆಯೇ ಮನನದ ಜಲದಲ್ಲಿ ಯಾವಾಗ ಒಂದು ವಿಷಯವನ್ನು ನೆನೆಸುತ್ತೇವೆಯೋ ಆಗ ಅದು ಕ್ರಮೇಣ ಅರ್ಥವಾಗುತ್ತಾ ಬರುವುದು.

ಈ ಮೇಲಿನ ಗುಣಗಳಿದ್ದರೆ ಆತನಿಗೆ ಜ್ಞಾನ ಬಂದಿದೆ ಎಂದು ಅರ್ಥ. ಮೇಲಿನ ಗುಣಗಳಿಲ್ಲದೆ ಕೆಲವು ವಿಷಯಗಳನ್ನು ಮಾತ್ರ ತಿಳಿದುಕೊಂಡರೆ ಅದು ಜ್ಞಾನವಾಗುವುದಿಲ್ಲ. ನಾವು ತಿಳಿದುಕೊಂಡಿರು ವುದು ನಮ್ಮ ಜೀವನದಲ್ಲಿ ಪ್ರವೇಶಿಸಿರಬೇಕು. ಅದು ನನ್ನಲ್ಲಿ, ನಾನು ಅದರಲ್ಲಿ ಮುಳುಗಿರಬೇಕು. ಉಪ್ಪಿನಕಾಯಿಯನ್ನು ರಸದಲ್ಲಿ ಹಾಕಿಟ್ಟರೆ ಹೇಗೆ ಕೆಲವು ಕಾಲವಾದಮೇಲೆ ರಸದಲ್ಲಿರುವುದೆಲ್ಲಾ ಕಾಯೊಳಗೆ ಪ್ರವೇಶಿಸುವುದೋ, ಹಾಗೆಯೇ ನಾವು ತಿಳಿದುಕೊಂಡದ್ದು ನಮ್ಮ ಜೀವನಕ್ಕೆ ಪ್ರವೇಶಿಸಿ ನಮ್ಮ ಬಾಳಿನ ಅಂಶವಾಗಿರಬೇಕು. ಆಧ್ಯಾತ್ಮಿಕ ಜೀವನ ಅನುಷ್ಠಾನ ಪ್ರಧಾನವಾಗಿ ಇರುವುದು. ನಮಗೆ ಏನೇನು ಗೊತ್ತಿದೆ ಅದಲ್ಲ ಮುಖ್ಯ. ನಾವು ಏನನ್ನು ಅನುಭವಿಸಿದ್ದೇವೆ ಅದು ಮುಖ್ಯ. ನನ್ನ ಅನುಭವಕ್ಕೆ ಸಿಕ್ಕಿರುವಷ್ಟೆ ನನ್ನದು.\enginline{}

\begin{verse}
ಜ್ಞೇಯಂ ಯತ್ತತ್ಪ್ರವಕ್ಷ್ಯಾಮಿ ಯಜ್ಜ್ಞಾತ್ವಾಮೃತಮಶ್ನುತೇ~।\\ಅನಾದಿಮತ್ಪರಂ ಬ್ರಹ್ಮ ನ ಸತ್ತನ್ನಾಸದುಚ್ಯತೇ \versenum{॥ ೧೨~॥}
\end{verse}

{\small ಯಾವುದು ಜ್ಞೇಯವೋ, ಯಾವುದನ್ನು ತಿಳಿದುಕೊಂಡು ಮನುಷ್ಯ ಅಮೃತತ್ತ್ವವನ್ನು ಪಡೆಯುತ್ತಾನೋ ಅದನ್ನು ಹೇಳುತ್ತೇನೆ. ಅದು ಅನಾದಿ ಪರಬ್ರಹ್ಮ. ಅದು ಸತ್ ಅಲ್ಲ. ಅಸತ್ ಅಲ್ಲ ಎಂದು ಹೇಳಲ್ಪಟ್ಟಿದೆ.}

ಜ್ಞಾನವನ್ನು ಹೇಳಿಯಾದ ಮೇಲೆ ಜ್ಞೇಯದ ವಿಷಯ, ಯಾವ ವಿಷಯನ್ನು ತಿಳಿದುಕೊಳ್ಳಬೇಕೋ ಅದರ ಸ್ವಭಾವ. ಇಲ್ಲಿ ಶ‍್ರೀಕೃಷ್ಣ ಜ್ಞಾನ ಎಂಬುದಕ್ಕೆ ಒಂದು ಹೊಸ ವಿವರಣೆಯನ್ನು ಕೊಡುತ್ತಾನೆ. ನಾವು ಸಾಧಾರಣವಾಗಿ ತಿಳಿದುಕೊಂಡಿರುವ ಜ್ಞಾನದ ಅರ್ಥಕ್ಕೂ ಇಲ್ಲಿ ಬರುವ ಜ್ಞಾನದ ಅರ್ಥಕ್ಕೂ ಬಹಳ ವ್ಯತ್ಯಾಸವಿದೆ. ಶ‍್ರೀಕೃಷ್ಣ ಕೊಡುವ ಗುಣಗಳಲ್ಲಿ ಎಲ್ಲಾ ಉದಾತ್ತವಾದ ನೀತಿಗಳು ಯಾರ ಲ್ಲಿದೆಯೋ ಅವನು ಜ್ಞಾನಿ. ಅಹಂಕಾರವಿಲ್ಲದಿರುವುದು, ಡಂಭಾಚಾರವಿಲ್ಲದಿರುವುದು, ಅಹಿಂಸೆ, ಕ್ಷಾಂತಿ, ಆರ್ಜವ, ಆಚಾರ್ಯೋಪಾಸನೆ, ಶೌಚ, ಸ್ಥೈರ್ಯ, ಆತ್ಮನಿಗ್ರಹ ಮುಂತಾದ ಗುಣಗಳೇ ತಳಹದಿ ಆಧ್ಯಾತ್ಮಿಕ ಅನುಭವವನ್ನು ಪಡೆಯಬೇಕಾದರೆ. ಈ ಗುಣಗಳಿಲ್ಲದೇ ಇದ್ದರೆ ನಮಗೆ ಆಧ್ಯಾತ್ಮಿಕ ಅನುಭವ ಬರಲಾರದು. ಅಕಸ್ಮಾತ್ತಾಗಿ ಬಂದರೂ ಅದು ಇರಲಾರದು. ಅನುಭವ ಎಂಬುದು ಮೇಲಿರುವ ಕಟ್ಟಡ. ಅದನ್ನು ಹೊರುವುದೇ ಅದರ ಕೆಳಗೆ ಇರುವ ತಳಪಾಯ. ತಳಪಾಯ ಬಲವಾಗಿ ಇದ್ದರೇನೆ ಮೇಲಿನ ಕಟ್ಟಡ ಭದ್ರವಾಗಿ ನಿಲ್ಲುವುದು. ಆಧ್ಯಾತ್ಮಿಕ ಜೀವನದಲ್ಲಿ ನೀತಿಯ ಜೀವನ ಅತಿ ಮುಖ್ಯ. ಬೇರೆ ಕಾರ್ಯಕ್ಷೇತ್ರಗಳಲ್ಲಾದರೆ ಅವನ ಜೀವನ ಹೇಗೆ ಬೇಕಾದರೂ ಇರಬಹುದು. ಆದರೆ ಅವನು ಅವನಿಗೇ ಸಂಬಂಧಪಟ್ಟ ವಿದ್ಯೆಯಲ್ಲಿ ಪ್ರಚಂಡನಾಗಿರಬಹುದು. ಹೊಲಸು ಜೀವನ ನಡೆಸಬಹುದು. ಆದರೂ ಚೆನ್ನಾದ ಸಂಗೀತಗಾರನಾಗಿರಬಹುದು, ಚಿತ್ರಗಾರನಾಗಿರ ಬಹುದು, ಸಾಹಿತಿಯಾಗಿರಬಹುದು, ವಿಜ್ಞಾನಿಯಾಗಿರಬಹುದು. ಇಲ್ಲಿ ಅವನ ಜೀವನ ಹೇಗೆ ಎಂದು ಪ್ರಶ್ನೆ ಹಾಕುವುದಿಲ್ಲ. ಅವನೇನೆಂದು ಮಾತ್ರ ಕೇಳುತ್ತೇವೆ. ಆಧ್ಯಾತ್ಮಿಕ ಜೀವನದಲ್ಲಾದರೊ, ಮೊದಲು ಅವನ ಜೀವನ ಏನು, ಅನಂತರ ಅವನಿಗೆ ಏನು ಅನುಭವ ಸಿಕ್ಕಿದೆ ಎಂದು ಕೇಳುತ್ತೇವೆ. ಅವನ ಜೀವನವೇ ಅನೈತಿಕವಾಗಿದ್ದರೆ, ಅಲ್ಲಿ ಯಾವ ಆಧ್ಯಾತ್ಮಿಕ ಅನುಭವಗಳೂ ಬರಲಾರವು. ಆದ ಕಾರಣವೇ ನೀತಿಗೆ ಅಷ್ಟೊಂದು ಪ್ರಾಮುಖ್ಯತೆ.

ಜ್ಞೇಯ ಎಂದರೆ ಯಾವುದನ್ನು ತಿಳಿದುಕೊಂಡರೆ ಮನುಷ್ಯ ಮುಕ್ತನಾಗುತ್ತಾನೆಯೊ ಅದು. ಇಲ್ಲಿ ತಿಳಿದುಕೊಳ್ಳುವುದು, ಅದರಂತಾಗುವುದು, ಎರಡೂ ಒಟ್ಟಿಗೆ ಹೋಗುವುದು. ಅದರಂತಾಗದೆ ನಾನು ತಿಳಿದುಕೊಂಡಿದ್ದೇನೆ ಎಂದರೆ ಅದಕ್ಕೆ ಬೆಲೆಯಿಲ್ಲ. ಅದನ್ನು ಯಾರೂ ಮಾನ್ಯಮಾಡುವುದಿಲ್ಲ. ಯಾವಾಗ ಪರಮಾತ್ಮನನ್ನು ಒಬ್ಬ ತಿಳಿದುಕೊಳ್ಳತ್ತಾನೆಯೋ, ಅವನು ಇನ್ನು ಮೇಲೆ ಅಜ್ಞಾನಿಯಾಗಿರು ವುದಕ್ಕೆ ಆಗುವುದಿಲ್ಲ. ಅವನಲ್ಲಿ ಇನ್ನು ಮೇಲೆ ವಾಸನೆಗಳು ಇರಲಾರವು, ಸಂದೇಹ ಇರಲಾರವು. ದೊಡ್ಡದೊಂದು ಕಾಡುಗಿಚ್ಚಿಗೆ ಒಂದು ಹಸಿಯ ಸೌದೆಯ ತುಂಡನ್ನು ಹಾಕಿದರೆ, ಸ್ವಲ್ಪ ಹೊತ್ತಿ ನಲ್ಲಿಯೇ ಅದು ಹತ್ತಿಕೊಳ್ಳುವುದು, ಉರಿದುಹೋಗುವುದು. ತನ್ನ ಹಿಂದಿನ ಸ್ವರೂಪವನ್ನು ಅದನ್ನು ರಕ್ಷಿಸಿಕೊಂಡು ಇರುವುದಕ್ಕೆ ಆಗುವುದಿಲ್ಲ. ಆದಕಾರಣವೇ ನಮ್ಮ ಶಾಸ್ತ್ರಗಳು ಯಾರು ಬ್ರಹ್ಮವನ್ನು ಅರಿಯುತ್ತಾನೆಯೋ ಅವನು ಬ್ರಹ್ಮನೇ ಆಗುತ್ತಾನೆ ಎಂದು ಸಾರುವುವು. ಅದನ್ನು ತಿಳಿದುಕೊಂಡ ಮೇಲೆ ನಾವು ಆವಿಗೆಯಲ್ಲಿ ಬೆಂದ ಮಡಕೆಯಂತೆ ಆಗುತ್ತೇವೆ. ಅದನ್ನು ಪುನಃ ಕುಂಬಾರ ಚಕ್ರದ ಮೇಲೆ ಇಡಲಾಗುವುದಿಲ್ಲ. ಅದನ್ನು ಆಚೆಗೆ ಬಿಸುಡಬೇಕು. ಜ್ಞೇಯವಸ್ತುವನ್ನು ತಿಳಿದುಕೊಂಡಾದ ಮೇಲೆ ಅವನಿನ್ನು ಸಂಸಾರಚಕ್ರಕ್ಕೆ ಸಿಕ್ಕುವುದಿಲ್ಲ. ಅವನು ಹುರಿದ ಬೀಜದಂತೆ ಆಗುತ್ತಾನೆ.

ಆ ಜ್ಞೇಯವಸ್ತು ಯಾವುದೆಂದರೆ, ಅದೇ ಅನಾದಿ ಪರಬ್ರಹ್ಮ. ಆ ಪರಬ್ರಹ್ಮ ವಸ್ತುವಿಗೆ ಒಂದು ಆದಿಯಿಲ್ಲ. ದೇಶಕಾಲ ನಿಮಿತ್ತದಲ್ಲಿ ಒಂದು ಆದಿ ಅಂತ್ಯ ಎಂಬುದು ಇದೆ. ಆದರೆ ಯಾವುದು ದೇಶಕಾಲಕ್ಕೆ ಅತೀತವಾಗಿದೆಯೊ ಅಲ್ಲಿ ಆದಿಗೆ ಅರ್ಥವಿಲ್ಲ. ನಾವು ಯಾವಾಗ ಒಂದು ಆದಿಯನ್ನು ಒಪ್ಪಿಕೊಳ್ಳುತ್ತೇವೆಯೊ, ಆಗ ನಮ್ಮ ಮನಸ್ಸು ಸುಮ್ಮನೆ ಇರುವುದಿಲ್ಲ. ಆದಿಗೆ ಆದಿ ಯಾವುದು ಎಂದು ಕೇಳುವುದು. ಅದಕ್ಕೆ ಒಂದು ಆದಿಯನ್ನು ಒಪ್ಪಿಕೊಂಡರೆ, ಅದಕ್ಕೆ ಆದಿ ಯಾವುದು ಎಂದು ಕೇಳುತ್ತಾ ಹೋಗುವುದು. ಈ ಬ್ರಹ್ಮವಸ್ತುವೇ ಆದಿ. ಅದಕ್ಕಿಂತ ಬೇರೆ ಯಾವುದೂ ಆದಿಯಿಲ್ಲ. ಮುಂಚೆ ಅದು ಅನಂತರ ದೇಶಕಾಲ ನಿಮಿತ್ತದಲ್ಲಿ ನಾವು ನೋಡುವುದೆಲ್ಲವೂ ಆಗಿದೆ.

ಅದನ್ನು ವಿವರಿಸುವುದಕ್ಕೆ ಶ‍್ರೀಕೃಷ್ಣ ಎರಡು ಗುಣವಾಚಕಗಳನ್ನು ಉಪಯೋಗಿಸುವನು. ಆ ಗುಣವಾಚಕಗಳೂ ಪರಸ್ಪರ ವಿರೋಧವಾಗಿರುವುವು. ಪರಬ್ರಹ್ಮನನ್ನು ವಿವರಿಸುವುದಕ್ಕೆ ಇದೊಂದೇ ಹತ್ತಿರದ ಭಾಷೆ. ಅದನ್ನು ಸತ್ ಎಂದು ಹೇಳುವುದಕ್ಕೆ ಆಗುವುದಿಲ್ಲ. ಅದು ನಮ್ಮ ಎದುರಿಗೆ ಕಾಣುವ ದೃಶ್ಯವಸ್ತುಗಳಂತೆ ಇಲ್ಲ. ದೃಶ್ಯವಸ್ತುಗಳ ಬಗ್ಗೆ ಇಲ್ಲಿದೆ, ಅಲ್ಲಿದೆ, ಹಾಗಿದೆ, ಹೀಗಿದೆ ಎಂದು ನಾವು ಹೇಳಬಹುದು. ಆದರೆ ಪರಬ್ರಹ್ಮನ ವಿಷಯದಲ್ಲಿ ಹಾಗೆ ಹೇಳುವುದಕ್ಕೆ ಆಗುವುದಿಲ್ಲ. ಅದು ಕಣ್ಣಿಗೆ ಕಾಣುವುದಾವುದೂ ಅಲ್ಲ. ಕಣ್ಣಿಗೆ ಕಾಣವುದೆಲ್ಲಾ ನಾಮರೂಪದಿಂದ ಆಗಿದೆ. ನಾಮರೂಪದಿಂದ ಆಗಿರುವುದೆಲ್ಲಾ ಒಂದು ಕಾಲದಲ್ಲಿ ಇರಲಿಲ್ಲ, ಈಗ ಇದೆ, ಮುಂದೆ ಇರುವುದಿಲ್ಲ. ಇವೆಲ್ಲಾ ಪರಸ್ಪರ ವಸ್ತುಗಳ ಸಂಯೋಗದಿಂದ ಆಗಿವೆ. ಇರುವಾಗಲೂ ಒಂದೇ ಸಮನಾಗಿರುವುದಿಲ್ಲ. ಕ್ಷಣಕ್ಷಣಕ್ಕೂ ಬದಲಾಯಿಸುತ್ತಾ ಇರವುವು. ಈ ಗುಂಪಿಗೆ ಸೇರಿದುದಲ್ಲ ಪರಬ್ರಹ್ಮ ವಸ್ತು.

ಇನ್ನು ಅದನ್ನು ಅಸತ್, ಇಲ್ಲ ಎಂದು ಹೇಳೋಣವೇ ಎಂದರೆ, ಹಾಗೆ ಹೇಳುವುದಕ್ಕೂ ಆಗುವುದಿಲ್ಲ. ಅದು ಇರಲೇ ಬೇಕು. ಆದರೆ ನಮಗೆ ಈಗ ಕಾಣುತ್ತಿರುವ ಯಾವ ವಸ್ತುವಿನಂತೆಯೂ ಅಲ್ಲ. ಅದು ನಿತ್ಯಸಾಕ್ಷಿಯಾಗಿದೆ. ದೃಶ್ಯಪ್ರಪಂಚದ ಮಹಾ ಸೌಧ ನಿಂತಿರುವುದೇ ಅದರ ಆಧಾರದ ಮೇಲೆ. ಮೇಲಿರುವುದನ್ನು ನೋಡುತ್ತೇವೆ. ಆಧಾರವನ್ನು ನೋಡುವುದಿಲ್ಲ. ಆದರೆ ಇಲ್ಲ ಎನ್ನಲಾಗು ವುದೆ? ಅದಿಲ್ಲದೇ ಇದ್ದರೆ ಈ ದೃಶ್ಯಪ್ರಪಂಚವೇ ಇರುತ್ತಿರಲಿಲ್ಲ. ಅದು ಇದೆ. ಅದಿದ್ದರೆ, ನಮಗೆ ಈಗ ವಸ್ತುಗಳು ಹೇಗೆ ಕಾಣುತ್ತಿವೆಯೊ ಹಾಗೆ ಇಲ್ಲ. ದೇಶ ಕಾಲ ನಿಮಿತ್ತದಲ್ಲಿ, ನಾಮರೂಪವನ್ನು ಧರಿಸಿ, ಯಾವುದೋ ಒಂದು ಘಟನೆಯಂತೆಯೋ, ವ್ಯಕ್ತಿಯಂತೆಯೋ, ವಸ್ತುವಿನಂತೆಯೋ ಇಲ್ಲ.

\begin{verse}
ಸರ್ವತಃಪಾಣಿಪಾದಂ ತತ್ಸರ್ವತೋಽಕ್ಷಿಶಿರೋಮುಖಮ್~।\\ಸರ್ವತಶ್ಶ್ರುತಿಮಲ್ಲೋಕೇ ಸರ್ವಮಾವೃತ್ಯ ತಿಷ್ಠತಿ \versenum{॥ ೧೩~॥}
\end{verse}

{\small ಅದು ಎಲ್ಲೆಲ್ಲಿಯೂ ಕೈಕಾಲುಗಳುಳ್ಳದ್ದು, ಎಲ್ಲೆಲ್ಲಿಯೂ ಕಣ್ಣು ತಲೆ ಬಾಯಿಗಳುಳ್ಳದ್ದು, ಎಲ್ಲೆಲ್ಲಿಯೂ ಕಿವಿಗಳುಳ್ಳದ್ಧು, ಲೋಕದಲ್ಲಿ ಎಲ್ಲವನ್ನೂ ಆವರಿಸಿಕೊಂಡಿದೆ.}

ಪರಸ್ಪರ ವಿರೋಧ ಭಾವಗಳನ್ನು ವಿವರಿಸಲು ಯತ್ನಿಸಿದ್ದಾಯಿತು ಹಿಂದಿನ ಶ್ಲೋಕದಲ್ಲಿ. ಇಲ್ಲಿ ನಮ್ಮ ಬುದ್ಧಿ ಅಷ್ಟೇ ಕಕ್ಕಾಬಿಕ್ಕಿಯಾಗಬೇಕು ಅದನ್ನು ಕೇಳಿದರೆ, ಆ ವಿವರಣೆಯನ್ನು ಕೊಡುತ್ತಾನೆ. ಇಂದ್ರಿಯಾತೀತವಾದ ಅನುಭವವನ್ನು ಇಂದ್ರಿಯದ ಭಾಷೆಯಲ್ಲಿಡಬೇಕಾದಾಗ ಬರುವ ತೊಡಕು ಗಳು ಇವು. ಇಲ್ಲಿ ಪರಸ್ಪರ ವಿರೋಧವಾಗಿರುವವುಗಳೆಲ್ಲಾ ಸಂಧಿಸುವುವು. ಒಂದು ಮತ್ತೊಂದನ್ನು ಅಲ್ಲಗಳೆಯುವುದಿಲ್ಲ. ಒಂದರ ಜೊತೆಯಲ್ಲಿಯೇ ಮತ್ತೊಂದು ಇರುವಂತೆ ಕಾಣುವುದು.

ಅದಕ್ಕೆ ಕೈಕಾಲುಗಳು ಎಲ್ಲೆಲ್ಲಿಯೂ ಇದೆ. ಅದೆಲ್ಲೊ ಒಂದು ಸ್ಥಳದಲ್ಲಿ, ಒಂದು ಕಾಲದಲ್ಲಿ, ಒಂದು ಕೆಲಸವನ್ನು ಮಾಡುವಂತಹುದಲ್ಲ. ಎಲ್ಲಾ ಕಡೆಗಳಲ್ಲಿಯೂ ಇದೆ. ಎಲ್ಲಾ ಕೆಲಸಗಳನ್ನು ಮಾಡುತ್ತಿದೆ. ಬ್ರಹ್ಮಾಂಡದ ಚಲನವಲನಗಳ ಹಿಂದೆಲ್ಲಾ ಅದೇ ಇರುವುದು. ಎಲ್ಲೆಲ್ಲಿಯೂ ಅದಕ್ಕೆ ಕಣ್ಣುಗಳಿವೆ ಎಂದರೆ, ಎಲ್ಲೆಲ್ಲಿ ಏನೇನು ಆಗುತ್ತಿದೆ ಎಂಬುದನ್ನು ಗಮನಿಸುತ್ತಿದೆ. ಅದು ನೋಡದ ಘಟನೆಯೇ ಇಲ್ಲ. ಅದು ಎಲ್ಲಾ ಕಡೆಯಲ್ಲಿಯೂ ತಲೆ ಬಾಯಿಗಳುಳ್ಳದ್ದು. ಮಾತನಾಡುತ್ತಿರುವುದರ ಹಿಂದೆ, ತಿನ್ನುತ್ತಿರುವುದರ ಹಿಂದೆ, ಯೋಚಿಸುತ್ತಿರುವುದರ ಹಿಂದೆಲ್ಲಾ ಅದೇ ಇರುವುದು. ವಿಶ್ವರೂಪ ದಲ್ಲಿ ಒಂದು ರೀತಿ ಅವನನ್ನು ಭಯಾನಕವಾಗಿ ಚಿತ್ರಿಸಲಾಯಿತು. ಇಲ್ಲಿ ಹೃದಯಕ್ಕೆ ಭಯವಲ್ಲ, ಬುದ್ಧಿಗೆ ಗ್ರಹಿಸುವುದಕ್ಕೆ ಆಗುವುದಿಲ್ಲ, ಅದು ತತ್ತರಿಸುವುದು, ಅಂತಹ ಭಾಷೆಯಲ್ಲಿ ಪರಬ್ರಹ್ಮನನ್ನು ಚಿತ್ರಿಸಲಾಗಿದೆ. ನಮ್ಮ ಬುದ್ಧಿಯಾದರೊ ಸಾಂತವನ್ನು ಗ್ರಹಿಸುವುದು, ಅಳೆಯುವುದು, ತೂಗುವುದು. ಇದರಿಂದ ಅನಂತವನ್ನು ಗ್ರಹಿಸುವುದಕ್ಕೆ ಆಗುವುದಿಲ್ಲ. ಅನಂತ ಸಾಂತದಲ್ಲಿ ಕಂಡಾಗ ನಮಗೆ ಹೀಗೆ ತೋರುವುದು.

ಅದು ಎಂದರೆ ಪರಬ್ರಹ್ಮ, ಎಲ್ಲವನ್ನೂ ಆವರಿಸಿಕೊಂಡಿದೆ. ಆಕಾಶ ಹೇಗೆ ಕಾಣುವ ಚರಾಚರ ವಸ್ತುಗಳನ್ನೆಲ್ಲಾ ಆವರಿಸಿಕೊಂಡಿದೆಯೊ, ಹಾಗೆ ಪರಬ್ರಹ್ಮ ವಸ್ತುಗಳನ್ನೆಲ್ಲಾ ಆವರಿಸಿಕೊಂಡಿದೆ. ಎಲ್ಲದರ ಒಳಗೆ ಮತ್ತು ಹೊರಗೆ ಅದು ಇದೆ. ಮತ್ತು ಎಲ್ಲವನ್ನೂ ಅತಿಕ್ರಮಿಸಿಯೂ ಇದೆ.

\begin{verse}
ಸರ್ವೇಂದ್ರಿಯಗುಣಾಭಾಸಂ ಸರ್ವೇಂದ್ರಿಯವಿವರ್ಜಿತಮ್~।\\ಅಸಕ್ತಂ ಸರ್ವಭೃಚ್ಚೈವ ನಿರ್ಗುಣಂ ಗುಣಭೋಕ್ತೃ ಚ \versenum{॥ ೧೪~॥}
\end{verse}

{\small ಅದು ಎಲ್ಲಾ ಇಂದ್ರಿಯಗಳ ಗುಣದಿಂದ ಬೆಳಗುತ್ತಿದೆ, ಸರ್ವೇಂದ್ರಿಯ ರಹಿತವಾಗಿದೆ. ಅಲ್ಲಿ ಯಾವ ಆಸಕ್ತಿಯೂ ಇಲ್ಲ. ಅದು ಎಲ್ಲಕ್ಕೂ ಆಧಾರ. ನಿರ್ಗುಣ ಮತ್ತು ಗುಣಗಳ ಭೋಕ್ಚ.}

ಬ್ರಹ್ಮ ಎಲ್ಲಾ ಇಂದ್ರಿಯಗಳ ಗುಣದಿಂದ ಬೆಳಗುತ್ತಿದೆ. ಬ್ರಹ್ಮ ಸರ್ವವ್ಯಾಪಿ. ಎಲ್ಲರಲ್ಲಿಯೂ ತುಂಬಿದೆ. ಅದು ನಮ್ಮ ಜ್ಞಾನೇಂದ್ರಿಯ, ಕರ್ಮೇಂದ್ರಿಯ, ಬುದ್ಧಿ, ಮನಸ್ಸು, ಅಹಂಕಾರಗಳ ಹಿಂದೆಲ್ಲಾ ಬೆಳಗುತ್ತಿದೆ, ಬೆಳಗುವಾಗ ಆಯಾ ಇಂದ್ರಿಯಗಳೇ ಬೆಳಗುತ್ತಿವೆಯೇನೊ ಎಂದು ನಾವು ತಪ್ಪು ಭಾವಿಸುತ್ತೇವೆ. ವಿದ್ಯುಚ್ಛಕ್ತಿ ಹಲವು ಬಲ್ಬುಗಳಲ್ಲಿ ಪ್ರಕಾಶಿಸುವಾಗ ಆಯಾ ಬಲ್ಬಿಗೆ ಹಾಕಿರುವ ಬಣ್ಣದಂತೆ ಕಾಣುತ್ತದೆ. ವಿದ್ಯುತ್ತಿಗೆ ಯಾವ ಬಣ್ಣವೂ ಇಲ್ಲ. ಆದರೆ ಪ್ರತಿಯೊಂದು ವಸ್ತುವಿನ ಬಣ್ಣವನ್ನೂ ತೆಗೆದುಕೊಂಡಂತೆ ಕಾಣುವುದು. ಇದು ‘ಬೃಂದಾವನ’ದ ನೀರಿಗೆ ಬಣ್ಣ ಬರುವಂತೆ. ಕಾವೇರಿಯ ನೀರು ಮೆಟ್ಟಿಲುಗಳ ಮೇಲಿಂದ ಬೀಳುತ್ತದೆ. ಆಗೊಂದು ಬಣ್ಣದ ಬಲ್ಬಿನ ಮೂಲಕ ಬೆಳಕನ್ನು ಅದರಮೇಲೆ ಬಿಡುವರು. ಕೃತಕ ಚಿಲುಮೆಗಳಲ್ಲಿ ನೀರು ಹಲವು ಆಕಾರಗಳನ್ನು ಧರಿಸಿ ಬೀಳುವಾಗ ಅದರಮೇಲೆ ಬಣ್ಣಬಣ್ಣದ ಕಾಂತಿಯನ್ನು ಬಿಡುವರು. ಇದರಿಂದ ನೀರಿಗೆ ಆ ಬಣ್ಣ ಬಂದಂತೆ ನಮಗೆ ಭ್ರಾಂತಿಯಾಗುವುದು; ನೀರಿಗೆ ಯಾವ ಬಣ್ಣವೂ ಇಲ್ಲ. ಅದು ತನ್ನ ಮೇಲೆ ಬೀಳುವ ಬಣ್ಣವನ್ನು ಪ್ರತಿಬಿಂಬಿಸುತ್ತದೆ ಅಷ್ಟೆ. ಅದರಂತೆಯೇ ಜ್ಞಾನೇಂದ್ರಿಯ, ಕರ್ಮೇಂದ್ರಿಯ ಮುಂತಾದ ಬಲ್ಬುಗಳ ಹಿಂದೆ ಪರಮಾತ್ಮನಿರುವುದರಿಂದ, ಈ ಬಲ್ಬುಗಳೇ ಪ್ರಕಾಶವಾದಂತೆ ಕಾಣುವುದು. ಬರೀ ಬಲ್ಬಿಗೆ ಬೆಳಗಲು ಶಕ್ತಿಯಿಲ್ಲ. ಹಿಂದೆ ವಿದ್ಯುಚ್ಛಕ್ತಿ ಸಂಪರ್ಕವಿರುವುದರಿಂದ ಅದು ಬೆಳಗುತ್ತಿದೆ. ಆ ಸಂಪರ್ಕ ಕಡಿದೊಡನೆಯೆ ಬೆಳಕು ನಿಲ್ಲುವುದು.

ಅದು ಸರ್ವೇಂದ್ರಿಯ ರಹಿತವಾಗಿದೆ. ಬ್ರಹ್ಮ ಈ ವಸ್ತುಗಳ ಮೂಲಕ ಬೆಳಗುತ್ತಿದ್ದರೂ ಅದು ಯಾವ ಇಂದ್ರಿಯಕ್ಕೂ ಅಂಟಿಕೊಂಡಿಲ್ಲ. ಯಾವ ಯಾವ ಮಧ್ಯವರ್ತಿ ಮೂಲಕ ಬೆಳಗುವುದೊ ಆಗ ಆಯಾ ಗುಣಗಳನ್ನು ಪಡೆದುಕೊಂಡಿರುವಂತೆ ಕಂಡರೂ ಅದಕ್ಕೆ ಯಾವ ಬಣ್ಣವೂ ಇಲ್ಲ, ವಿದ್ಯುಚ್ಛಕ್ತಿ ಬಣ್ಣದ ಬಲ್ಬುಗಳ ಮೂಲಕವಾಗಿ ಬೆಳಗಿದಾಗಲೂ, ಬಣ್ಣಗಳಿಗೆ ಅದು ಅಂಟಿಕೊಂಡಿಲ್ಲ. ಬಣ್ಣಬಣ್ಣದ ಬುಡ್ಡಿಗಳಿಗೆ ನೀರನ್ನು ಹಾಕಿದರೆ ಆ ನೀರು ಆ ಬುಡ್ಡಿಯ ಆಕಾರವನ್ನು ತಾಳಿದಂತೆ ಕಾಣುವುದು. ಆದರೆ ಆ ನೀರನ್ನು ಹೊರಗೆ ಚೆಲ್ಲಿದರೆ ಅದಕ್ಕೆ ಯಾವ ಆಕಾರವಾಗಲಿ, ಬಣ್ಣವಾಗಲಿ, ಬಂದಿಲ್ಲವೆಂಬುದು ನಮಗೆ ಗೊತ್ತಾಗುವುದು.

ಅವನು ಇಂದ್ರಿಯಗಳ ಮೂಲಕ ಕೆಲಸ ಮಾಡುತ್ತಿರುವನು. ಆದರೆ ಅವನು ಇಂದ್ರಿಯಗಳಿಗೆ ಅಂಟಿಕೊಂಡಿಲ್ಲ, ಮತ್ತು ಅವುಗಳಿಂದ ಬರುವ ಫಲಗಳಿಗೂ ಆಸಕ್ತನಲ್ಲ. ಇಂದ್ರಿಯದ ಮೂಲಕ ಬೆಳಗುವುದು ಬೇರೆ, ಇಂದ್ರಿಯಕ್ಕೆ ಅಂಟಿಕೊಳ್ಳುವುದು ಬೇರೆ. ವಿದ್ಯುಚ್ಛಕ್ತಿ ಬಲ್ಬಿನ ಮೂಲಕ ಬೆಳಗುವುದು. ಆದರೆ ಬಲ್ಬಿಗೆ ಅಂಟಿಕೊಂಡಿಲ್ಲ. ಅದನ್ನು ಆರಿಸಿದೊಡನೆ ಸ್ವಲ್ಪವೂ ಅಲ್ಲಿ ವಿದ್ಯುಚ್ಛಕ್ತಿ ಇರುವುದನ್ನು ನೋಡುವುದಿಲ್ಲ. ಅದರಿಂದ ಬೆಳಕು ಬರುತ್ತದೆ. ಆ ಬೆಳಕಿನಿಂದ ಒಬ್ಬೊಬ್ಬ ಒಂದೊಂದು ಕೆಲಸವನ್ನು ಮಾಡುವನು, ಒಬ್ಬ ದೀಪದ ಬೆಳಕಿನಲ್ಲಿ ಭಾಗವತ ಓದುವನು, ಇನ್ನೊಬ್ಬ ಅದೇ ಬೆಳಕಿನಲ್ಲಿ ಸುಳ್ಳು ಅರ್ಜಿ ಬರೆಯುವನು. ಇದರಿಂದ ದೀಪವನ್ನು ದೂರುವುದಕ್ಕೆ ಆಗುವುದಿಲ್ಲ.

ಇದು ಯಾವುದಕ್ಕೂ ಅಂಟಿಕೊಳ್ಳದೇ ಇದ್ದರೂ ಇದೇ ಎಲ್ಲಕ್ಕೂ ಆಧಾರ. ಇದಿದ್ದರೆ ಕರ್ಮೇಂ ದ್ರಿಯ, ಜ್ಞಾನೇಂದ್ರಿಯ, ಮನಸ್ಸು, ಬುದ್ಧಿ ಇವೆಲ್ಲವೂ ಕೆಲಸ ಮಾಡಬೇಕಾದರೆ. ಮಡಕೆ ಕುಡಿಕೆಗಳೆಲ್ಲ ಇರಬೇಕಾದರೆ ಮೊದಲು ಜೇಡಿಮಣ್ಣು ಇರಬೇಕು, ಅನಂತರ ಮಡಿಕೆ ಕುಡಿಕೆಗಳೆಲ್ಲ ಅದರಿಂದ ಆಗುವುದು. ಜೇಡಿಮಣ್ಣು ಇಲ್ಲದೆ ಮಡಕೆ ಕುಡಿಕೆ ಹೇಗೆ ಆಗಬಲ್ಲುದು? ಮೊದಲು ನೂಲು ಅನಂತರ ಅದರಿಂದ ಆದ ಬಗೆಬಗೆಯ ಆಕಾರದ ಬಟ್ಟೆಗಳು. ನೂಲೇ ಇಲ್ಲದೆ ಬಟ್ಟೆ ಹೇಗೆ ಇರಬಲ್ಲುದು? ಅದರಂತೆಯೇ ಮೊದಲು ಬ್ರಹ್ಮ, ಆಮೇಲೆ ಈ ಜಗತ್ತು. ಅವನಿದ್ದರೆ ಈ ಜಗತ್, ಅವನಿಲ್ಲದೆ ಈ ಜಗತ್ತು ಇರಲಾರದು. ನಾನಿದ್ದರೆ ತಾನೇ ಕನಸು ಇರಬಲ್ಲುದು. ನನ್ನನ್ನು ಬಿಟ್ಟು ಕನಸು ಇರಲಾರದು. ಸಾಗರ ಇದ್ದರೆ ತಾನೆ, ಸಾಗರದ ಮೇಲೆ ಏಳುವ ಅಲೆ, ನೊರೆ, ಗುಳ್ಳೆ ಇವುಗಳೆಲ್ಲ ಇರಬಲ್ಲುವು. ಅದರಂತೆಯೇ ಬ್ರಹ್ಮ ಈ ಪ್ರಪಂಚಕ್ಕೆ ಉಪಾದಾನ ಕಾರಣವೂ ನಿಮಿತ್ತ ಕಾರಣವೂ ಎರಡೂ ಆಗಿರುವನು. ಅವನು ಯಾವ ಗುಣಕ್ಕೂ ಅಂಟಿಕೊಂಡಿಲ್ಲ. ಈ ಪ್ರಪಂಚದಲ್ಲೆಲ್ಲ ಮೂರು ಗುಣಗಳು ಆವರಿಸಿಕೊಂಡಿವೆ. ಅವೇ ಸತ್ತ್ವ, ರಜಸ್ಸು ಮತ್ತು ತಮಸ್ಸು. ಬ್ರಹ್ಮ ಯಾವ ಗುಣಕ್ಕೂ ಬದ್ಧನಾಗ ದವನು. ಅವನಿಂದ ಈ ಗುಣಗಳು ಬಂದಿವೆ. ಆದರೆ ಅವನಾದರೋ ಇವುಗಳಾವುದಕ್ಕೂ ಬದ್ಧನಾಗದೆ ಇರುವನು. ಆದರೂ ಅವನು ಗುಣಗಳ ಭೋಕ್ತೃ. ಎಂದರೆ ಅದನ್ನು ಅನುಭವಿಸುವಂತೆ ಕಾಣುವನು. ವಿದ್ಯುಚ್ಛಕ್ತಿ ಬಣ್ಣಬಣ್ಣದ ಬಲ್ಬಿನಲ್ಲಿ ಬೆಳಗಿದಾಗ, ಆಯಾ ಬಣ್ಣವನ್ನೇ ತೆಗೆದುಕೊಂಡಿರುವಂತೆ ನಮಗೆ ಕಾಣುವುದು. ನೀರನ್ನು ಬೇರೆ ಬೇರೆ ಬಾಟಲುಗಳಿಗೆ ಹಾಕಿದಾಗ, ಆ ಬಾಟಲಿನ ಬಣ್ಣ ಮತ್ತು ಆಕಾರವನ್ನು ಪಡೆದಂತೆ ಕಾಣುವುದು. ಇಲ್ಲಿ ಗುಣಗಳನ್ನು ಅವನು ಅಳೆಯುತ್ತಿರುವನು. ದಾಸನಾಗಿ ಅದಕ್ಕೆ ಅಂಟಿಕೊಂಡಿಲ್ಲ. ಯಜಮಾನನಾಗಿ ಸಾಕ್ಷಿಯಂತೆ ಅವುಗಳನ್ನು ಅರಿತುಕೊಳ್ಳುತ್ತಿರುವನು. ಆದರೆ ಅವನು ಯಾವ ಕಾಲದಲ್ಲಿಯೂ ಇದರಿಂದ ಬಾಧಿತನಾಗುವುದಿಲ್ಲ.

\begin{verse}
ಬಹಿರಂತಶ್ಚ ಭೂತಾನಾಮಚರಂ ಚರಮೇವ ಚ~।\\ಸೂಕ್ಷ್ಮತ್ವಾತ್ ತದವಿಜ್ಞೇಯಂ ದೂರಸ್ಥಂ ಚಾಂತಿಕೇ ಚ ತತ್ \versenum{॥ ೧೫~॥}
\end{verse}

{\small ಅದು ಸಮಸ್ತ ಪ್ರಾಣಿಗಳ ಒಳಗೂ ಹೊರಗೂ ಇರುವುದು. ಚರಾಚರ ಸ್ವರೂಪವಾಗಿದೆ, ಸೂಕ್ಷ್ಮಸ್ವರೂಪ ವಾಗಿರುವುದರಿಂದ ಅರಿಯಲು ಅಸಾಧ್ಯವಾಗಿರುವುದು. ಅದು ದೂರದಲ್ಲಿರುವುದು, ಸಮೀಪದಲ್ಲಿಯೂ ಇರುವುದು.}

ಅದು ಎಲ್ಲದರ ಒಳಗೆ ಮತ್ತು ಹೊರಗೆ ಇದೆ. ಇದಕ್ಕೆ ಒಂದು ಉದಾಹರಣೆ ನಮಗೆ ಬೇಕಾದರೆ ಆಕಾಶ. ಆಕಾಶದಲ್ಲಿ ಬ್ರಹ್ಮಾಂಡಗಳೆಲ್ಲ ತೇಲುತ್ತಿದೆ. ಅದೇ ಆಕಾಶ, ಅತ್ಯಂತ ಸಣ್ಣದರ ಒಳಗೂ ಇದೆ. ಅತಿ ದೊಡ್ಡ ನಿಹಾರಿಕೆ ಕೂಡ ಆಕಾಶದಲ್ಲಿದೆ. ಅತ್ಯಂತ ಸಣ್ಣದು ಮತ್ತು ಅತ್ಯಂತ ದೊಡ್ಡದಲ್ಲಿರುವುದು ಮಾತ್ರವಲ್ಲ, ಇದನ್ನು ಅತಿಕ್ರಮಿಸಿದೆ. ನಾವು ಯಾವುದನ್ನು ಬ್ರಹ್ಮಾಂಡ ಎಂದು ಹೇಳುತ್ತೇವೆಯೊ ಅದು ಬ್ರಹ್ಮನ ಯಾವುದೋ ಸಣ್ಣ ಅಂಶ. ಇನ್ನೂ ಬಹುಪಾಲು ಇದಕ್ಕೆ ಅತೀತವಾಗಿದೆ. ನಮಗೆ ತಿಳಿದಿರುವ ಜ್ಞಾನವೆಲ್ಲ ಅಲ್ಪ. ಅವನು ಇದನ್ನು ಅತಿಕ್ರಮಿಸಿರುವನು. ಈ ಪ್ರಪಂಚದಲ್ಲಿ ಅವನು ಓತಪ್ರೋತನಾಗಿದ್ದಾನೆ. ಆದರೆ ಅವನು ಇಲ್ಲಿಯೇ ಖಾಲಿಯಾಗಿಹೋಗಿಲ್ಲ. ಅವನ ಯಾವುದೋ ಒಂದು ಭಾಗ ಈ ಬ್ರಹ್ಮಾಂಡವಾಗಿರುವುದು. ಸಮುದ್ರದ ನೀರಿನಲ್ಲಿ ಯಾವುದೋ ಸಣ್ಣ ಅಂಶ ನೀರ್ಗಲ್ಲಿನಂತೆ \enginline{(ice berg)} ತೇಲುತ್ತಿದೆ. ಅದರ ಬಹು ಅಂಶ ನೀರ್ಗಲ್ಲಿಗೆ ಅತೀತವಾಗಿದೆ.

ಅದು ಚರಾಚರ ಸ್ವರೂಪವಾಗಿದೆ. ಈ ಪ್ರಪಂಚದಲ್ಲಿ ನಾವು ಏನನ್ನು ನೋಡುತ್ತೇವೆಯೋ ಅದೆಲ್ಲ ಆಗಿರುವುದು ಅದರಿಂದಲೇ. ದೊಡ್ಡದೊಡ್ಡ ನಕ್ಷತ್ರಗಳು ಗ್ರಹಗಳು ಸೂರ್ಯಚಂದ್ರರು ಆಗ ಬಹುದು, ಅಣುರೇಣು ತೃಣಕಾಷ್ಟ ಆಗಬಹುದು, ಅದೆಲ್ಲ ಆಗಿರುವುದು ಅದರಿಂದಲೆ. ಅದು ಚರ ಮತ್ತು ಅಚರವಾಗಿರುವ ವಸ್ತುಗಳು ಮಾತ್ರವಲ್ಲ, ಅದೇ ಜಡ ಮತ್ತು ಚೇತನದಂತೆ ರೂಪುತಾಳಿದೆ. ಈ ಜಗತ್ತಿನಂತೆ ಜಡಸ್ವರೂಪವನ್ನು ತಳೆದಿರುವುದು ಅದೇ, ಚೇತನದಂತೆ, ಹಲವು ಪಶುಪಕ್ಷಿ ಮನುಷ್ಯರಂತೆ ರೂಪನ್ನು ಧರಿಸುವುದರ ಹಿಂದೆಯೂ ಅದೇ ಇರುವುದು. ಜೀವ ಮತ್ತು ಜಗತ್ತೆಂದು ಯಾವುದನ್ನು ಹೇಳುವೆವೊ ಅದಾಗಿರುವುದು ಆ ಬ್ರಹ್ಮನಿಂದ. ಅದು ಸೂಕ್ಷ್ಮವಾಗಿರುವುದರಿಂದ ತಿಳಿಯಲು ಕಷ್ಟ. ನಾವು ಒಂದು ವಸ್ತುವನ್ನು ತಿಳಿದುಕೊಳ್ಳಬೇಕಾದರೆ ಕೆಲವು ಸಾಧನಗಳನ್ನು ಉಪಯೋಗಿಸುತ್ತೇವೆ. ಅದೇ ಪಂಚೇಂದ್ರಿಯಗಳು, ಮನಸ್ಸು ಮತ್ತು ಬುದ್ಧಿ ಇವುಗಳು. ಯಾವುದು ಇವುಗಳ ಒಳಗೆ ಬೀಳುವುವೊ ಅದನ್ನು ತಿಳಿದುಕೊಳ್ಳುತ್ತೇವೆ, ಯಾವುದು ಇದನ್ನು ಗಣನೆಗೆ ತರುವುದಿಲ್ಲವೊ ಇದರ ಬಲೆಗೆ ಬೀಳುವುದಿಲ್ಲವೊ ಅದನ್ನು ನಾವು ತಿಳಿದುಕೊಳ್ಳಲಾರೆವು. ದೊಡ್ಡ ಬಲೆಯನ್ನು ಒಡ್ಡಿರುತ್ತೇವೆ. ಹಾರಾಡುತ್ತಿರುವ ಹಕ್ಕಿ ಚಿಟ್ಟೆ ಮತ್ತು ಕಸಕಡ್ಡಿ ತರಗೆಲೆಗಳು ಬೇಕಾದರೆ ಅದರಲ್ಲಿ ಸಿಕ್ಕಿಕೊಳ್ಳುವುವು. ಆದರೆ ಗಾಳಿಯನ್ನು ಹಿಡಿಯಬಲ್ಲದೆ? ಅದು ಒಂದು ಕಡೆಯಿಂದ ಮತ್ತೊಂದು ಕಡೆಗೆ ಹೊರಟುಹೋಗುವುದು. ನಾವು ಅದನ್ನು ಹಿಡಿಯುವುದಕ್ಕೆ ಆಗುವುದಿಲ್ಲ. ಆದರೆ ಅದು ಇಲ್ಲ ಎಂದು ಅರ್ಥವಲ್ಲ. ನಾವು ಅದನ್ನು ಗ್ರಹಿಸಲಾರೆವು ಎಂದು ಒಪ್ಪಿಕೊಳ್ಳುತ್ತೇವೆ. ನಮ್ಮ ಇಂದ್ರಿಯಗಳಿಗೆ ಒಂದು ವಸ್ತು ಗೊತ್ತಾಗಬೇಕಾದರೆ, ಅದಕ್ಕೆ ಶಬ್ದ, ಸ್ಪರ್ಶ, ರೂಪ, ರಸ, ಗಂಧ, ಇರಬೇಕು. ಇವಿಲ್ಲದೇ ಇದ್ದರೆ ಅದನ್ನು ಅರ್ಥಮಾಡಿಕೊಳ್ಳಲಾರೆವು.

ಅದರಂತೆಯೆ ನಮ್ಮ ಬುದ್ಧಿ ದೇಶ ಕಾಲ ನಿಮಿತ್ತದ ಕ್ಷೇತ್ರದಲ್ಲಿ ಕೆಲಸ ಮಾಡುವುದು. ಅಲ್ಲಿ ಅದು ಪ್ರತಿಯೊಂದಕ್ಕೂ ಕಾರ್ಯಕಾರಣ ಸಂಬಂಧ ಏರ್ಪಡಿಸುವುದು. ಕಾರ್ಯಕಾರಣ ಸಂಬಂಧ ವಿದ್ದರೇನೆ ಅದು ತಿಳಿಯಬಲ್ಲದು, ಅದನ್ನು ಮೀರಿತು ಎಂದರೆ ಬುದ್ಧಿ ನಿಸ್ಸಹಾಯಕವಾಗುವುದು. ನನ್ನ ಕೈಯಲ್ಲಿ ಸಾಧ್ಯವಿಲ್ಲ ಇದು ಎನ್ನವುದು. ಬುದ್ಧಿ ಗ್ರಹಿಸುವುದಕ್ಕೆ ಸಾಧ್ಯವಾಗದ ವಸ್ತು ಪ್ರಪಂಚ ದಲ್ಲಿಇಲ್ಲ ಎಂದಲ್ಲ. ಬಹುಪಾಲು ನಮ್ಮ ಬುದ್ಧಿಗೆ ನಿಲುಕದುದೆ ಇರುವುದು. ಎಲ್ಲೊ ಒಂದು ಸ್ಪಲ್ಪವನ್ನು ಮಾತ್ರ ಬುದ್ಧಿ ತಿಳಿದುಕೊಳ್ಳಬಲ್ಲದು. ನಮ್ಮಲ್ಲಿರುವುದು ಸೇರು ಪಾವು, ಚಟಾಕು. ಇದರಿಂದ ಕೆಲವು ವಸ್ತುಗಳನ್ನು ಇಷ್ಟು ಇದೆ ಎಂದು ಹೇಳುತ್ತೇವೆ. ಆದರೆ ಈ ಸಾಧನದಿಂದ ಆಕಾಶವನ್ನು ಅಳೆಯಲಾಗುವುದೆ? ಅವುಗಳೆನ್ನೆಲ್ಲಾ ಅಳೆಯಬೇಕಾದರೆ ಸೂಕ್ಷ್ಮ ಯಂತ್ರಗಳು ಬೇಕು, ಅದರಿಂದ ಮಾತ್ರ ಸಾಧ್ಯ. ಅದರಂತೆಯೇ ಸಾಧಾರಣ ಜನರ ಬುದ್ಧಿ ಇನ್ನೂ ಜಡಾವಸ್ಥೆಯಲ್ಲಿರುವ ಕೆಲವು ವಿಷಯಗಳನ್ನು ಮಾತ್ರ ತಿಳಿದುಕೊಳ್ಳಬಲ್ಲದು. ಅದಿನ್ನೂ ಪ್ರಬುದ್ಧವಾಗಿಲ್ಲ. ಈಗಿನ ಸ್ಥಿತಿಯಲ್ಲಿ ಅದು ಪರಬ್ರಹ್ಮನನ್ನು ಅರಿಯಲಾರದು.

ಅದು ದೂರದಲ್ಲಿದೆ. ಎಂದರೆ ಅಜ್ಞಾನಿಗಳಿಗೆ ಬಹಳ ದೂರದಲ್ಲಿದೆ. ಅವರಿಗೆ ನಿಲುಕುವ ವಸ್ತುವಲ್ಲ ಅದು. ದೇವರು ಎಂದೊಡನೆ ಆಕಾಶದ ಕಡೆ ನೋಡುವರು ಹಲವರು, ಏನೋ ದೇವರು ಮೋಡಗಳಾಚೆ ಇರುವಂತೆ. ಅಜ್ಞಾನಿಗಳ ಕೈಗೆ ಎಟುಕದ ವಸ್ತು ಅದು. ಆದರೆ ಅದು ಅತ್ಯಂತ ಸಮೀಪದಲ್ಲಿಯೇ ಇದೆ ಜ್ಞಾನಿಗಳಿಗೆ. ಇದರಷ್ಟು ಹತ್ತಿರವಾಗಿರುವುದು ಮತ್ತಾವುದೂ ಇಲ್ಲ. ಅವನಷ್ಟು ಸಮೀಪದ ವಸ್ತು ಮತ್ತಾವುದೂ ಇಲ್ಲ. ಮೊದಲು ಅವನು. ಅನಂತರ ಜೀವ ಮತ್ತು ಜಗತ್. ಮೊದಲು ಅವನು, ಅನಂತರ ನಾನು, ನನ್ನ ದೇಹ, ಮನಸ್ಸು, ಬುದ್ಧಿ, ಇಂದ್ರಿಯ, ದೇಹಗಳು. ಅವನೇ ಇಲ್ಲದೆ ನಾವು ಹೇಗೆ ಬರುತ್ತೇವೆ? ಸಾಗರವೇ ಇಲ್ಲದೆ ಅಲೆ ಹೇಗೆ ಬರುತ್ತದೆ? ಅಲೆಯಲ್ಲಿ ಕೂಡ ಸತ್ಯವಾಗಿರುವುದು ಯಾವುದು? ಸಾಗರ ತಾನೆ. ಅಲೆ ಸಾಗರವನ್ನು ತಿಳಿಯಬೇಕಾದರೂ ಎಲ್ಲೊ ಹೊರಗೆ ಹೋಗ ಬೇಕಾಗಿಲ್ಲ; ತನ್ನೊಳಗೆ ಮುಳುಗಿದರೆ ಸಾಕು, ತಾನಾರು ಎಂಬುದನ್ನು ಅರಿತರೆ ಸಾಕು. ಸಾಗರವನ್ನು ಪ್ರತ್ಯೇಕವಾಗಿ ಅರಿಯಬೇಕಾಗಿಲ್ಲ. ನಾನಾರು ಎಂಬ ಜ್ಞಾನವೇ ಸಾಗರ ಯಾವುದು ಎಂಬ ಜ್ಞಾನಕ್ಕೆ ಒಯ್ಯುವುದು.

\begin{verse}
ಅವಿಭಕ್ತಂ ಚ ಭೂತೇಷು ವಿಭಕ್ತಮಿವ ಚ ಸ್ಥಿತಮ್~।\\ಭೂತಭರ್ತೃ ಚ ತಜ್ಜ್ಞೇಯಂ ಗ್ರಸಿಷ್ಣು ಪ್ರಭವಿಷ್ಣು ಚ \versenum{॥೧೬॥}
\end{verse}

{\small ಅದು ವಿಭಾಗವಿಲ್ಲದೆ ಇರುವುದು. ಪ್ರಾಣಿಗಳಲ್ಲಿ ವಿಭಾಗವಾದಂತೆ ಕಾಣುತ್ತಿರುವುದು. ಆ ಜ್ಞೇಯವು ಸಮಸ್ತ ಭೂತಗಳನ್ನು ಪಾಲಿಸುವುದು, ಸಂಹಾರ ಮಾಡುವುದು ಮತ್ತು ಸೃಷ್ಟಿಸುವುದು.}

ಜ್ಞೇಯವಸ್ತು ಅನಂತವಾದುದು. ಅನಂತವನ್ನು ಯಾರೂ ಭಾಗಮಾಡಲಾರರು. ಒಂದು ವೇಳೆ ನಾವು ಭಾಗ ಮಾಡಿದೆವು ಎಂದು ಭಾವಿಸಿದರೆ, ಪ್ರತಿಯೊಂದು ಭಾಗವೂ ಅನಂತವೇ ಆಗುವುದು. ಆಗ ಹಲವು ಅನಂತಗಳು ಹೇಗೆ ಇರಲಿಕ್ಕೆ ಸಾಧ್ಯ? ಇದು ಹಾಸ್ಯಾಸ್ಪದವಾಗಿ ಕಾಣುವುದು. ಒಂದು ಸೊನ್ನೆಯನ್ನು ಅರ್ಧ, ಕಾಲು, ಮುಕ್ಕಾಲು ಮಾಡಿದರೆ, ಅವುಗಳೆಲ್ಲ ಅಷ್ಟು ಸೊನ್ನೆಗಳಾಗುತ್ತವೆ ಹೊರತು, ಸೊನ್ನೆಯ ಅಷ್ಟು ಭಾಗವಾಗುವುದಿಲ್ಲ.

ಅದು ವಿಭಾಗವಾಗದೆ ಇದ್ದರೂ ವಿಭಾಗವಾದಂತೆ ನಮಗೆ ಕಾಣುತ್ತಿದೆ. ಜೀವರಾಶಿಗಳನ್ನು ನೋಡಿದಾಗ ಒಂದು ಪರಬ್ರಹ್ಮ ಇಷ್ಟೊಂದು ಜೀವರಾಶಿಗಳಾಗಿ ವಿಭಾಗವಾದಂತೆ ಕಾಣುತ್ತಿದೆ. ಆದರೆ ವ್ಯಾವಹಾರಿಕವಾಗಿ ನೋಡಿದರೆ ಆದಂತೆ ಕಾಣುತ್ತಿದೆ. ಆಕಾಶ ಅಖಂಡವಾದುದು. ಯಾರೂ ಅದನ್ನು ವಿಭಾಗ ಮಾಡಿಲ್ಲ. ಆದರೆ ಹಲವು ಮನೆ ಮಡಿಕೆ ಕುಡಿಕೆಗಳನ್ನು ನೋಡಿದಾಗ ಆಕಾಶ ವಿಭಾಗವಾದಂತೆ ನಮಗೆ ಕಾಣುವುದು. ಸೂರ್ಯ ಒಂದು. ಬೆಳಗ್ಗೆ ನಾವು ಹೊರಗೆ ಹೊರಟರೆ ಪ್ರತಿಯೊಂದು ಹಿಮಮಣಿಯೂ ಒಂದೊಂದು ಸೂರ್ಯನನ್ನೇ ಪ್ರತಿಬಿಂಬಿಸುತ್ತಿರುವುದು. ಅಷ್ಟೊಂದು ಸೂರ್ಯರಿರುವರೆ? ಅಥವಾ ಒಂದು ಸೂರ್ಯ ಅಷ್ಟೊಂದು ಭಾಗವಾಗಿ ಹೋಗಿರುವನೆ? ಇಲ್ಲ. ಆದರೆ ಆದಂತೆ ಕಾಣುತ್ತಿದೆ. ಇದೇ ವ್ಯಾವಹಾರಿಕ ಸತ್ಯಕ್ಕೂ ಪಾರಮಾರ್ಥಿಕ ಸತ್ಯಕ್ಕೂ ಇರುವ ವ್ಯತ್ಯಾಸ. ಸೂರ್ಯ ಹುಟ್ಟುವುದೂ ಇಲ್ಲ, ಮುಳುಗುವುದೂ ಇಲ್ಲ. ಖಗೋಳ ದೃಷ್ಟಿಯಿಂದ ನೋಡಿದಾಗ ಅವನು ಸುಮ್ಮನೆ ಒಂದು ಕಡೆ ಇರುವನು. ಭೂಮಿ ಅವನ ಸುತ್ತಲೂ ಸುತ್ತುತ್ತಿದೆ. ಭೂಮಿಯ ಮೇಲೆ ಇರುವವರಿಗೆ ಸೂರ್ಯ ಹುಟ್ಟಿದಂತೆ ಮುಳುಗಿದಂತೆ ಕಾಣುತ್ತಿರುವನು. ನಿಜವಾಗಿ ಅವನು ಹುಟ್ಟುತ್ತಲೂ ಇಲ್ಲ, ಮುಳುಗುತ್ತಲೂ ಇಲ್ಲ. ಆದರೆ ಹಾಗೆ ಆಗುವಂತೆ ನಾವು ನೋಡುವೆವು, ಒಂದು ಅಖಂಡ ಪರಬ್ರಹ್ಮ ನಿಜವಾಗಿ ಭಾಗವಾಗಿಲ್ಲ. ಆದರೆ ವ್ಯಾವಹಾರಿಕ ದೃಷ್ಟಿಯಿಂದ ನೋಡಿದರೆ ಅದು ಭಾಗವಾದಂತೆ ಕಾಣುವುದು.

ಆ ಜ್ಞೇಯ ವಸ್ತುವೇ ಸಮಸ್ತ ಪ್ರಪಂಚವನ್ನೂ ಸೃಷ್ಟಿಸುವುದು. ಇದೆಲ್ಲ ಬಂದಿರುವುದು ಪರಬ್ರಹ್ಮ ನಿಂದ. ನಾವು ಏನೇನು ನೋಡುವೆವೊ ಅದೆಲ್ಲ ಒಂದರಿಂದಲೇ ಬಂದಿದೆ, ಒಂದರಿಂದಲೇ ಆಗಿದೆ. ಅದನ್ನು ಪಾಲಿಸುವುದೂ ಅದೇ. ಪ್ರತಿಯೊಂದು ವಸ್ತು ಬೆಳೆಯುವುದಕ್ಕೂ ಏನೇನು ಬೇಕೊ ಅದನ್ನೆಲ್ಲ ಒದಗಿಸುತ್ತಿದೆ. ಯಾರೂ ಆಹಾರವಿಲ್ಲದೇ ಹೋಗುವುದಿಲ್ಲ. ಪ್ರತಿಯೊಂದರ ಬೆಳವಣಿಗೆಗೂ ದೇವರು ಅಣಿ ಮಾಡಿರುವನು. ಜೀವರಾಶಿಗಳು ಹುಟ್ಟುವಾಗ ಎಷ್ಟು ನಿಸ್ಸಹಾಯರಾಗಿರುತ್ತವೆ? ಆದರೆ ಅವರಿಗೆ ಆಹಾರ ಕೊಟ್ಟು ಕಷ್ಟದಿಂದ ತಪ್ಪಿಸಿ ಮುಂದೆ ಬರುವಂತೆ ಮಾಡುವುದಕ್ಕೆ ಎಷ್ಟೊಂದು ಶಕ್ತಿಗಳು ಕೆಲಸ ಮಾಡುತ್ತಿವೆ. ಅದರ ಹಿಂದೆಲ್ಲ ಪರಬ್ರಹ್ಮನ ಶಕ್ತಿಯೇ ಇರುವುದು. ಹಾಗೆಯೇ ಜೀವರಾಶಿ ಗಳನ್ನು ನಾಶಮಾಡುವುದಕ್ಕೂ ಕೂಡ ಅಷ್ಟೇ ಶಕ್ತಿಗಳು ಎಲ್ಲಾ ಕಡೆಯಲ್ಲಿಯೂ ಕೆಲಸ ಮಾಡುತ್ತಿರು ವುವು. ನಮ್ಮ ದೇಹವನ್ನು ನಾಶ ಮಾಡುವ ಶಕ್ತಿ ನಮ್ಮಲ್ಲಿಯೇ ಬೆಳೆಯುತ್ತಿದೆ. ಜೊತೆಗೆ ನಮ್ಮನ್ನು ನಾಶಮಾಡುವುದಕ್ಕೆ ಹೊರಗಡೆ ಬೇಕಾದಷ್ಟು ಕ್ರಿಮಿಕೀಟಗಳಿವೆ, ಉಪದ್ರವಗಳಿವೆ, ಸಾಂಕ್ರಾಮಿಕ ಜಾಡ್ಯಗಳಿವೆ, ಯುದ್ಧವಿದೆ, ಭೂಕಂಪ ಜ್ವಾಲಾಮುಖಿಗಳಿವೆ. ಸೃಷ್ಟಿಯ ಕೆಲಸಕ್ಕೂ ಸಂಹಾರದ ಕೆಲಸಕ್ಕೂ ಒಂದು ನಿಕಟ ಸಂಬಂಧವಿರುವಂತೆ ಕಾಣುವುದು. ಸಮುದ್ರಕ್ಕೆ ಹಗಲು ರಾತ್ರಿ ಎಲ್ಲಾ ನದಿಗಳಿಂದ ನೀರು ಬಂದು ಬೀಳುತ್ತಿರುತ್ತದೆ. ಆದರೆ ಎಷ್ಟು ಬರುವುದೋ ಅಷ್ಟು ಆವಿಯಾಗಿ ಹೋಗುವುದು. ಅದಕ್ಕೆ ಸಮುದ್ರ ಹೆಚ್ಚಾಗುವುದೂ ಇಲ್ಲ, ಕಡಿಮೆಯಾಗುವುದೂ ಇಲ್ಲ. ಇಲ್ಲಿ ಹೇಗೆ ಅನ್ಯೋನ್ಯ ಸಂಬಂಧವಿದೆಯೋ, ಹಾಗೆಯೇ ಸೃಷ್ಟಿ ಸ್ಥಿತಿ ಪ್ರಳಯಗಳಲ್ಲಿ ಒಂದಕ್ಕೂ ಮತ್ತೊಂದಕ್ಕೂ ಸಂಬಂಧವಿದೆ. ಎಲ್ಲದರ ಹಿಂದೆಯೂ ಪಕ್ಷಪಾತವಿಲ್ಲದೆ ಭಗವಂತನ ಶಕ್ತಿ ಕೆಲಸ ಮಾಡುತ್ತಿದೆ.

\begin{verse}
ಜ್ಯೋತಿಷಾಮಪಿ ತಜ್ಜ್ಯೋತಿಸ್ತಮಸಃ ಪರಮುಚ್ಯತೇ~।\\ಜ್ಞಾನಂ ಜ್ಞೇಯಂ ಜ್ಞಾನಗಮ್ಯಂ ಹೃದಿ ಸರ್ವಸ್ಯ ವಿಷ್ಠಿತಮ್ \versenum{॥ ೧೭~॥}
\end{verse}

{\small ಅದು ಜ್ಯೋತಿಗೆ ಜ್ಯೋತಿ, ತಮಸ್ಸಿನ ಆಚೆ ಇದೆ ಎಂದು ಹೇಳುವರು. ಅದು ಜ್ಞಾನ ಜ್ಞೇಯ ಜ್ಞಾನಗಮ್ಯ ಮತ್ತು ಎಲ್ಲರ ಹೃದಯದಲ್ಲಿಯೂ ಪ್ರತಿಷ್ಠಿತವಾಗಿದೆ.}

ಅದು ಜ್ಯೋತಿಗೆ ಜ್ಯೋತಿ, ಎಂದರೆ ಜೀವರಿಗೆ ಚೈತನ್ಯವನ್ನು ಕೊಡುವುದು. ಅದು ಜೀವಿಗಳಲ್ಲಿ ಚೈತನ್ಯವನ್ನು ನೀಡುವುದರಿಂದ, ನಾವು ಬದುಕಿದ್ದೇವೆ, ವಿಚಾರ ಮಾಡುತ್ತೇವೆ, ಕೆಲಸ ಮಾಡುತ್ತೇವೆ. ದೊಡ್ಡ ಒಂದು ಯಂತ್ರ ಕೆಲಸ ಮಾಡಬೇಕಾದರೆ ವಿದ್ಯುಚ್ಛಕ್ತಿ ಬಂದು ಮೊದಲು ಗಾಲಿಯನ್ನು ತಿರುಗಿಸಬೇಕು. ಅನಂತರ ಬೃಹತ್ ಯಂತ್ರದ ಸಣ್ಣ ಪುಟ್ಟಗಾಲಿಗಳೆಲ್ಲ ಕೆಲಸ ಮಾಡಲು ಪ್ರಾರಂಭ ಮಾಡುವುವು.

ಅದು ತಮಸ್ಸಿನ ಆಚೆ ಇದೆ. ಅಜ್ಞಾನದ ಆಚೆ ಇದೆ. ಇನ್ನೂ ಅಜ್ಞಾನದಲ್ಲಿರುವವರಿಗೆ ಅದು ಕಾಣದು. ಹತ್ತಿರಕ್ಕೆ ಅದು ಹತ್ತಿರವಾಗಿದ್ದರೂ, ಅಜ್ಞಾನ ಯಾವಾಗ ನಮ್ಮನ್ನು ಆವರಿಸುವುದೊ ಆಗ ಅದನ್ನು ಗ್ರಹಿಸುವುದಕ್ಕೆ ಆಗುವುದಿಲ್ಲ. ನಾವು ನಿಧಿಯ ಮೇಲೆಯೇ ಇರಬಹುದು. ಆದರೆ ಅದು ನಮಗೆ ಗೊತ್ತಿಲ್ಲದೆ ಇದ್ದರೆ ನಿಧಿ ಇದ್ದರೂ ಏನೂ ಪ್ರಯೋಜನವಾಗುವುದಿಲ್ಲ.

ಅದೇ ಜೀವಿಗಳಲ್ಲಿ ಜ್ಞಾನದಂತೆ ಇರುವುದು. ಅಮಾನಿತ್ವ, ಢಂಬಾಚಾರವಿಲ್ಲದೆ ಇರುವುದು, ಅಹಿಂಸೆ, ಶಾಂತಿ, ಆರ್ಜವ, ಆಚಾರ್ಯೋಪಾಸನೆ, ಶೌಚ, ಸ್ಥೈರ್ಯ, ಆತ್ಮನಿಗ್ರಹ, ವೈರಾಗ್ಯ, ನಿರಹಂಕಾರ, ಏಕನಿಷ್ಠವಾದ ಭಕ್ತಿ, ನಿರ್ಜನಪ್ರಿಯತೆ, ಆತ್ಮಜ್ಞಾನದಲ್ಲಿ ನಿಷ್ಠೆ ಮುಂತಾದುವುಗಳಲ್ಲಿ ಅದೇ ಇರುವುದು. ಸಕ್ಕರೆ ಪಾಕ ಹಲವು ಅಚ್ಚುಗಳನ್ನು ಪ್ರವೇಶಿಸಿ ಹಲವು ಆಕಾರಗಳಾಗಿವೆ. ಆದರೆ ಅದರ ಹಿಂದೆಲ್ಲ ಇರುವುದು ಸಕ್ಕರೆ ಪಾಕವೆ. ಹಾಗೆಯೆ ಈ ಗುಣಗಳ ಹಿಂದೆಲ್ಲ ಇರುವುದು ಪರಬ್ರಹ್ಮವಸ್ತುವೆ. ಅದೇ ಜ್ಞೇಯ ವಸ್ತುವಿನಂತೆ ಇರುವುದು. ಅದು ಇಂದ್ರಿಯದಲ್ಲಿದೆ ಮತ್ತು ಅದಕ್ಕೆ ಅತೀತವಾಗಿದೆ. ಗುಣಾತೀತ, ಆದರೂ ಗುಣವನ್ನು ಅನುಭವಿಸುತ್ತದೆ. ಅದು ಎಲ್ಲಾ ಪ್ರಾಣಿಗಳ ಒಳಗೆ ಮತ್ತು ಹೊರಗೆ ಇದೆ. ಬುದ್ಧಿ ಅದನ್ನು ಗ್ರಹಿಸುವುದಕ್ಕೆ ಆಗುವುದಿಲ್ಲ. ವಿಭಾಗವಾಗಿಲ್ಲ. ಆದರೂ ಅದರಂತೆ ಕಾಣುತ್ತಿದೆ. ಅದೇ ಎಲ್ಲಾ ವಸ್ತುಗಳ ಸೃಷ್ಟಿ ಪಾಲಕ ಮತ್ತು ಸಂಹಾರಕ.

ಜ್ಞಾನ ಎಂದರೆ ಹಿಂದೆ ಹೇಳಿದ ಗುಣಗಳೆಲ್ಲಾ ಯಾರಲ್ಲಿ ಇದೆಯೊ, ಅವನಿಗೆ ಪರಬ್ರಹ್ಮ ಗಮ್ಯವಸ್ತು, ಸಿಕ್ಕತಕ್ಕ ವಸ್ತು ಆಗಿದೆ. ಈ ಸರ್ವವ್ಯಾಪಿಯಾದ ಪರಬ್ರಹ್ಮ ಎಲ್ಲರ ಹೃದಯದಲ್ಲಿ ಇರುವನು. ಇವನು ನಮ್ಮಿಂದ ಬೇರೆ ಇಲ್ಲ. ನಾವು ಯಾವುದನ್ನು ನಾನು ಎಂದು ಹೇಳುತ್ತೇವೋ, ಅದೊಂದು ಪರಮಾತ್ಮನೆಂಬ ಗಂಗೆಯಲ್ಲಿ ಮುಳಗುವುದಕ್ಕೆ ಇರುವ ಘಾಟಿನಂತೆ. ಅವನನ್ನು ಹುಡುಕಲು ನಾವು ಇನ್ನೆಲ್ಲೊ ಹೋಗಬೇಕಾಗಿಲ್ಲ. ನಮ್ಮ ಹಿಂದೆ ಹೋದರೆ ಸಾಕು, ಆ ಸರ್ವವ್ಯಾಪಿಯಾದವನು ಅಲ್ಲಿರುವುದು ನಮಗೆ ಗೋಚರವಾಗುವುದು. ಆದರೆ ಪ್ರಪಂಚದ ವಿಚಿತ್ರ ಇದು. ನಾವು, ನಾವೆಂಬ ಅಹಂಕಾರದ ಹಿಂದೆ ಇರುವ ಪರಮಾತ್ಮನ ಕಡೆ ಧಾವಿಸುವುದಿಲ್ಲ. ನಾನೆಂಬುದೇ ಹೊಸಿಲು. ಅದರ ಮುಂದೆ ದೃಶ್ಯಪ್ರಪಂಚ ಬೆಳಗುತ್ತಿದೆ, ಅದರ ಹಿಂದೆ ಪರಂಜ್ಯೋತಿ ಸ್ವರೂಪನಾದ ಪರಮಾತ್ಮ ಬೆಳಗುತ್ತಿರುವನು. ನಾವು ಹಿಂದಿರುಗಿ ನೋಡಬೇಕು, ಆಗ ಅವನು ಸದಾ ಕಾಲದಲ್ಲಿಯೂ ಅಲ್ಲಿಯೇ ಇದ್ದಾನೆ ಎಂಬುದು ಗೊತ್ತಾಗುವುದು.

\begin{verse}
ಇತಿ ಕ್ಷೇತ್ರಂ ತಥಾ ಜ್ಞಾನಂ ಜ್ಞೇಯಂ ಚೋಕ್ತಂ ಸಮಾಸತಃ~।\\ಮದ್ಭಕ್ತ ಏತದ್ವಿಜ್ಞಾಯ ಮದ್ಭಾವಾಯೋಪಪದ್ಯತೇ \versenum{॥ ೧೮~॥}
\end{verse}

{\small ಹೀಗೆ ಕ್ಷೇತ್ರ, ಜ್ಞಾನ ಮತ್ತು ಜ್ಞೇಯ ಇವುಗಳನ್ನು ಸಂಕ್ಷೇಪವಾಗಿ ಹೇಳಿದ್ದಾಯಿತು. ನನ್ನ ಭಕ್ತ ಇವನ್ನು ತಿಳಿದುಕೊಂಡು ನನ್ನ ಭಾವವನ್ನು ಹೊಂದಲು ಅರ್ಹನಾಗುತ್ತಾನೆ.}

ತಾತ್ತ್ವಿಕ ಕ್ಷೇತ್ರದಲ್ಲಿ ಅತ್ಯಂತ ಮುಖ್ಯವಾದ ಭಾವನೆಗಳು ಇದುವರೆಗೆ ಶ‍್ರೀಕೃಷ್ಣ ವಿವರಿಸಿದ್ದು. ಮೊದಲನೆಯದು ಕ್ಷೇತ್ರ, ಅಂದರೆ ಮನೆ. ಈ ಮನೆಯ ಹಿತ್ತಲಿನಲ್ಲಿ ಜೀವಿ ಹಲವಾರು ಬೆಳೆಗಳನ್ನು ಉತ್ತುವುದು, ಬಿತ್ತುವುದು, ಕೊಯ್ಯುವುದು. ಈ ಕ್ಷೇತ್ರದ ಲಕ್ಷಣಗಳನ್ನೆಲ್ಲ ಹೇಳುತ್ತಾನೆ. ಅನಂತರವೇ ಅದನ್ನು ತಿಳಿದುಕೊಳ್ಳುವವನೇ ಕ್ಷೇತ್ರಜ್ಞ. ಇವನೇ ಜೀವಾತ್ಮ. ಅನಂತರ ಯಾರು ದೇವರನ್ನು ತಿಳಿದುಕೊಳ್ಳಬೇಕೆಂದು ಬಯಸುವನೊ ಆ ಜ್ಞಾನಿಗೆ ಯಾವ ಲಕ್ಷಣಗಳಿರಬೇಕು ಎಂಬುದನ್ನು ವಿವರಿಸುತ್ತಾನೆ. ಒಬ್ಬ ಪರಮಾತ್ಮನನ್ನು ಪಡೆದುಕೊಳ್ಳಬೇಕಾದರೆ ಅವನಿಗೆ ಕೆಲವು ಯೋಗ್ಯತೆಗಳಿರ ಬೇಕು. ಆಗಲೆ ಅವನನ್ನು ನಾವು ಗ್ರಹಿಸುವುದಕ್ಕೆ ಸಾಧ್ಯ. ಆ ಯೋಗ್ಯತೆಗಳು ಇಲ್ಲದೆ ಇದ್ದರೆ ನಾವು ಅವನನ್ನು ಗ್ರಹಿಸುವುದಕ್ಕೆ ಆಗುವುದಿಲ್ಲ. ಜೀವನದಲ್ಲಿ ಬರೀ ಆಸೆಯೇ ಸಾಲದು ಒಂದು ವಸ್ತುವನ್ನು ಪಡೆಯಬೇಕಾದರೆ. ಅದಕ್ಕೆ ಯೋಗ್ಯತೆಯನ್ನು ಪಡೆದಿರಬೇಕು. ಒಂದು ಕಾರನ್ನು ಕೊಂಡುಕೊಳ್ಳಬೇಕು ಎಂದು ಪೇಟೆಗೆ ಹೋದರೆ ಅದಕ್ಕೆ ಸಾಕಷ್ಟು ದುಡ್ಡನ್ನು ತೆಗೆದುಕೊಂಡು ಹೋಗಬೇಕು. ಜೇಬಿನ ತುಂಬ ಕೆಲವು ರೂಪಾಯಿಗಳ ಚಿಲ್ಲರೆ ಕಾಸನ್ನು ತೆಗೆದುಕೊಂಡು ಹೋದರೆ ಸಿಕ್ಕುವುದಿಲ್ಲ. ಆಟದ ಕಾರು ಕೂಡ ಈಗಿನ ಕಾಲದಲ್ಲಿ ಆ ದುಡ್ಡಿಗೆ ಸಿಕ್ಕುವುದು ಕಷ್ಟ. ಇನ್ನು ನಿಜವಾದ ಕಾರು ಸಿಕ್ಕೀತೇ? ಭಗವಂತನನ್ನು ಪಡೆಯುವುದು ಎಂಬುದು ಈ ಪ್ರಪಂಚದಲ್ಲಿ ಅತಿಮಾನುಷ ಸಾಹಸ. ಈ ಸಾಹಸ ಯಾತ್ರೆಗೆ ನಮ್ಮ ಬುತ್ತಿಯಲ್ಲಿ ಏನೇನಿದೆ ಎಂಬುದನ್ನು ನಾವು ಪರಿಗಣಿಸಬೇಕು. ಏನೇನು ಇರಬೇಕು ಎಂಬುದನ್ನು ಜ್ಞಾನದ ಲಕ್ಷಣದಲ್ಲಿ ಶ‍್ರೀಕೃಷ್ಣ ಹೇಳಿರುವನು. ಅನಂತರವೇ ಜ್ಞೇಯವಸ್ತು, ಪರಮಾತ್ಮ. ವಿವರಿಸುವುದಕ್ಕೆ ಆಗದುದನ್ನು ಶ‍್ರೀಕೃಷ್ಣ ಚಮತ್ಕಾರವಾದ ಭಾಷೆಯಲ್ಲಿ ವಿವರಿಸಲು ಯತ್ನಿಸುವನು. ಇದಕ್ಕಿಂತ ಹೆಚ್ಚು ಯಾರೂ ಜಯಶಾಲಿಗಳಾಗಿಲ್ಲ. ಬಹುಶಃ ಅದು ಹಾಗಿರಬಹುದು ಎಂಬ ಗ್ರಹಿಕೆ ನಮ್ಮ ಮನಸ್ಸಿನಲ್ಲಿ ಮೂಡಲು ಏನು ಮಾಡಬೇಕು ಅದನ್ನು ಮಾಡುವನು.

ಅನಂತರವೇ ನನ್ನ ಭಕ್ತ ಇದನ್ನು ತಿಳಿದುಕೊಂಡು ನನ್ನ ಭಾವವನ್ನು ಹೊಂದಲು ಅರ್ಹನಾಗು ತ್ತಾನೆ ಎನ್ನುವನು. ಅವನ ಭಾವ ಎಂದರೆ, ಅವನಂತಾಗುವುದು. ಇಲ್ಲಿ ಮೂರು ಸಿದ್ಧಾಂತವಾದಿಗಳೂ ತಮಗೆ ತೋರಿದುದನ್ನು ಹೇಳುತ್ತಾರೆ: ಬ್ರಹ್ಮವನ್ನು ತಿಳಿದವನು ಬ್ರಹ್ಮವೇ ಆಗುತ್ತಾನೆ. ಈ ಜೀವಿಯೇ ಬ್ರಹ್ಮವಾಗುತ್ತಾನೆ ಎಂಬುವರು ಅದ್ವೈತಿಗಳು. ನಾವೆಲ್ಲ ವ್ಯಷ್ಟಿಗಳು, ಅವನು ಸಮಷ್ಟಿ. ನಾವೆಲ್ಲ ಅಲೆಗಳು, ಅವನು ಅಲೆಗಳ ತೌರೂರಾದ ಸಾಗರ. ಅವನು ಪೂರ್ಣ, ನಾವು ಅವನ ಅಂಗಗಳಾಗುತ್ತೇವೆ ಎಂದು ವಿಶಿಷ್ಟಾದ್ವೈತಿಗಳು ಹೇಳುವರು. ದ್ವೈತಿಗಳಾದರೊ ಜೀವಿ ಪರಿಶುದ್ಧನಾಗಿ ಭಗವಂತನಂತೆ ಕಾಣುತ್ತಾನೆ. ಆದರೆ ಅವನು ಹಾಗೆ ಆಗುವುದಿಲ್ಲ. ಅವನು ಯಾವಾಗಲೂ ಬೇರೆಯಾಗಿಯೇ ಇರುತ್ತಾನೆ ಭಗವಂತನ ಸನ್ನಿಧಿಯಲ್ಲಿ ಎನ್ನುವರು. ಗೀತೆ ಒಂದು ತತ್ತ್ವ ಸರಿ, ಮತ್ತೊಂದು ತಪ್ಪು ಎಂದು ಸಾರುವುದಕ್ಕೆ ಹೋಗುವುದಿಲ್ಲ. ಯಾರು ಯಾವ ಭಾವದಿಂದಲಾದರೂ ತೆಗೆದುಕೊಳ್ಳಲಿ, ಎಲ್ಲರೂ ಮುಕ್ತರಾಗುತ್ತಾರೆ, ಇದು ಸಾಮಾನ್ಯವಾದ ಅರ್ಥ. ದೇವರಿಗೂ ನಮಗೂ ಇರುವ ಸಂಬಂಧದಲ್ಲಿ ವ್ಯತ್ಯಾಸವಾಗಬಹುದು. ಆದರೆ ಎಲ್ಲರೂ ನಿರ್ವಾಣ ಪಡೆಯುತ್ತಾರೆ. ಇನ್ನೊಮ್ಮೆ ಕರ್ಮ ಸವೆಸಲು ಈ ಪ್ರಪಂಚಕ್ಕೆ ಬರುವುದಿಲ್ಲ. ತೆವಳುತ್ತಿರುವ ಕೀಟವೊಂದು ಭ್ರಮರದ ಗೂಡಿಗೆ ಹೋಗಿ ಅಲ್ಲಿ ಅನುಗಾಲವೂ ಭ್ರಮರವನ್ನು ಚಿಂತಿಸುತ್ತಾ ಇದ್ದು ಕೊನೆಗೆ ಇದು ತನ್ನ ಕೀಟಾವಸ್ಥೆ ಯನ್ನು ಕಳೆದುಕೊಂಡು ಭ್ರಮರವೇ ಆಗುವುದು. ಹಾಗೆಯೇ ಜೀವ ಭಗವಂತನ ಸ್ವರೂಪವನ್ನು ಪಡೆಯುತ್ತಾನೆ. ಅಗ್ನಿಕುಂಡದ ಮಧ್ಯದಲ್ಲಿ ಬಿದ್ದಿರುವ ಇದ್ದಲು ಕ್ರಮೇಣ ತಾನೂ ಬೆಂಕಿಯೇ ಆಗುವಂತೆ ಇದು.

\begin{verse}
ಪ್ರಕೃತಿಂ ಪುರುಷಂ ಚೈವ ವಿದ್ಧ್ಯನಾದೀ ಉಭಾವಪಿ~।\\ವಿಕಾರಾಂಶ್ಚ ಗುಣಾಂಶ್ಚೈವ ವಿದ್ಧಿ ಪ್ರಕೃತಿಸಂಭವಾನ್ \versenum{॥ ೧೯~॥}
\end{verse}

{\small ಪ್ರಕೃತಿ ಮತ್ತು ಪುರುಷ ಇವೆರಡೂ ಅನಾದಿಗಳು ಎಂದು ತಿಳಿ. ವಿಕಾರಗಳು ಮತ್ತು ಗುಣಗಳು ಪ್ರಕೃತಿಯಿಂದ ಹುಟ್ಟಿದವುಗಳು ಎಂದು ತಿಳಿ.}

ಇಲ್ಲಿ ಶ‍್ರೀಕೃಷ್ಣ ಇನ್ನೆರಡು ಹೊಸ ಪದಗಳನ್ನು ತರುತ್ತಾನೆ. ಆದರೆ ಅದರ ಅರ್ಥ ಹೆಚ್ಚು ಕಡಿಮೆ ಕ್ಷೇತ್ರ ಮತ್ತು ಕ್ಷೇತ್ರಜ್ಞಕ್ಕೆ ಸೇರಿದ್ದು. ಆದರೆ ಅದಕ್ಕಿಂತ ವ್ಯಾಪಕತೆಯಲ್ಲಿ ಹೆಚ್ಚು ವಿಶಾಲವಾಗಿದೆ. ಇಲ್ಲಿ ಪ್ರಕೃತಿ ಮತ್ತು ಪುರುಷ ಎಂದರೆ, ಜೀವಿ ಮತ್ತು ಜಗತ್ ಎರಡೂ ಅನಾದಿ ಎನ್ನುವನು. ಒಂದು ಪ್ರಕೃತಿ, ಜೀವ ವಿಕಾಸವಾಗುವುದಕ್ಕೆ ಇರುವ ರಂಗಭೂಮಿ. ಇಲ್ಲಿ ಹೊರಗಿನ ಪಂಚಭೂತಗಳು ಇವೆ. ಅದಕ್ಕಿಂತ ಸೂಕ್ಷ್ಮವಾದ ಮನಸ್ಸು, ಬುದ್ಧಿ, ಇಂದ್ರಿಯಗಳಿವೆ. ಇವುಗಳೂ ಕೂಡ ಪ್ರಕೃತಿಯ ಉಗ್ರಾಣದಲ್ಲಿವೆ. ಜೀವ ಅವುಗಳಿಂದ ತನ್ನ ಕರ್ಮಾನುಸಾರ ಹೀರಿಕೊಂಡು ಬೆಳೆಯುವುದು. ಈ ಪ್ರಕೃತಿ ಅನಾದಿ ಎನ್ನುವನು. ಅದಕ್ಕೆ ಒಂದು ಆದಿ ಇದ್ದರೆ ಮುಂಚೆ ಏನಾಗಿತ್ತು, ಯಾವುದರಿಂದ ಅದು ಬಂತು, ಎಂದು ಕೇಳಬೇಕಾಗುವುದು. ಇಲ್ಲಿ ಸೃಷ್ಟಿ ಎಂಬ ಪದವನ್ನು ಉಪಯೋಗಿಸುವರು. ಈ ಬ್ರಹ್ಮಾಂಡ ಸೃಷ್ಟಿಯಾಯಿತು ಎಂದರೆ, ಹಿಂದೆ ಇರಲಿಲ್ಲ, ಶೂನ್ಯದಿಂದ ಆಯಿತು ಎಂದು ಹೇಳುವುದಿಲ್ಲ. ಶೂನ್ಯದಿಂದ ಏನೂ ಆಗಲಾರದು. ಅದು ಎಲ್ಲಾ ತರ್ಕಕ್ಕೂ ವಿರೋಧ. ಅದು ಹಿಂದೆ ಸೂಕ್ಷ್ಮಾವಸ್ಥೆಯಲ್ಲಿತ್ತು. ಈಗ ಸ್ಥೂಲ ಅವಸ್ಥೆಯನ್ನು ತಾಳಿದೆ. ಇಂತಹ ಸೃಷ್ಟಿಗಳು ಹಿಂದೆ ಎಷ್ಟೋ ಸಲ ಆಗಿವೆ ಮತ್ತು ಎಷ್ಟೋ ಸಲ ಆಗುವುದಿದೆ. ಇದಕ್ಕೆ ಸೃಷ್ಟಿಯೇ ಆದಿಯಲ್ಲ, ಪ್ರಳಯವೇ ಅಂತ್ಯವಲ್ಲ. ಇವುಗಳೆಲ್ಲ ಅವಸ್ಥೆಯ ಭೇದವಷ್ಟೆ. ಸಮುದ್ರದಲ್ಲಿ ಒಂದು ಅಲೆ ಈಗ ಎದ್ದಿತು ಎನ್ನುತ್ತೇವೆ, ಹಾಗಾದರೆ ಅದು ಎಲ್ಲಿತ್ತು ಎಂದು ಕೇಳಿದರೆ, ಸೂಕ್ಷ್ಮರೂಪದಲ್ಲಿ ಸಾಗರದಲ್ಲಿತ್ತು. ಅದು ಹಿಂದೆ ಎಷ್ಟೋ ವೇಳೆ ಎದ್ದಿದೆ ಮತ್ತು ಎಷ್ಟೋ ವೇಳೆ ಬಿದ್ದಿದೆ. ಹಾಗೆಯೆ ಸೃಷ್ಟಿ ಕೂಡ.

ಅದರಂತೆಯೇ ಪುರುಷ ಅಂದರೆ ಪ್ರಕೃತಿಯಲ್ಲಿ ಜೀವಿಸುವ ಜೀವನೂ ಕೂಡ ಅನಾದಿ. ಅವನನ್ನೂ ದೇವರು ಯಾವುದೊ ಒಂದು ಕಾಲದಲ್ಲಿ ಸೃಷ್ಟಿ ಮಾಡಿದ, ಹಿಂದೆ ಇರಲಿಲ್ಲ ಎನ್ನುವುದಕ್ಕೆ ಆಗುವುದಿಲ್ಲ. ಯಾವಾಗ ಹಿಂದೆ ಇರಲಿಲ್ಲವೊ, ಈಗ ಹೊಸದಾಗಿ ಬಂದ ಎನ್ನುವೆವೊ, ಆಗ ಹೊಸದಾಗಿ ಯಾವುದರಿಂದ ದೇವರು ಮಾಡಿದ ಎಂದು ಕೇಳಬೇಕಾಗುವುದು. ಯಾವಾಗ ಜೀವರಿಗೆಲ್ಲ ಒಂದು ಆದಿಯನ್ನು ಒಪ್ಪುತ್ತೇವೆಯೊ, ಆಗ ದೇವರು ಪ್ರಾರಂಭದಲ್ಲಿ ಎಲ್ಲರನ್ನೂ ಚೆನ್ನಾಗಿ ಮಾಡಬೇಕಾಗಿತ್ತು. ಏತಕ್ಕೆ ಕೆಲವರನ್ನು ಬುದ್ಧಿವಂತರನ್ನಾಗಿ ಮಾಡುತ್ತಾನೆ, ಕೆಲವರನ್ನು ದಡ್ಡರನ್ನಾಗಿ ಮಾಡುತ್ತಾನೆ, ಅವನಿಗೆ ಏತಕ್ಕೆ ಕೆಲವರನ್ನು ಕಂಡರೆ ಆಗದು, ಕೆಲವರನ್ನು ಕಂಡರೆ ಪ್ರಾಣ. ಅವನು ಪಕ್ಷಪಾತಿ ಆಗುತ್ತಾನೆ. ಆದಕಾರಣವೆ ಜೀವರು ಯಾವಾಗಲೂ ಇದ್ದರು.

ಸೃಷ್ಟಿಕಾಲದಲ್ಲಿಯೇ ದೇವರು ಜೀವರನ್ನು ಹೊಸದಾಗಿ ತಯಾರು ಮಾಡಿ ಕಳುಹಿಸಲಿಲ್ಲ. ಯಾರು ಸೂಕ್ಷ್ಮಾವಸ್ಥೆಯಲ್ಲಿದ್ದರೊ ಅವರು ಸ್ಥೂಲಾವಸ್ಥೆಗೆ ಮಾತ್ರ ಬಂದರು. ಜೀವಿಗಳಲ್ಲಿರುವ ವೈವಿಧ್ಯತೆ ಗಳಿಗೆಲ್ಲ ಅವರೇ ಹೊಣೆ. ತಮ್ಮ ಕರ್ಮಾನುಸಾರ ಅವರಿಗೆ ದೇಹ, ಮನಸ್ಸು, ಬುದ್ಧಿ, ಇಂದ್ರಿಯಗಳು ಬರುತ್ತವೆ, ಹೋಗುತ್ತವೆ. ಅದಕ್ಕಾಗಿ ದೇವರನ್ನು ದೂರಿ ಪ್ರಯೋಜನವಿಲ್ಲ. ಈ ಸ್ವಾತಂತ್ರ್ಯವನ್ನು ದೇವರು ಜೀವರಿಗೆ ಕೊಟ್ಟಿದ್ದಾನೆ. ಇದನ್ನು ತಪ್ಪಾಗಿ ಉಪಯೋಗಿಸಿಕೊಂಡರೆ ಜೀವಿಗಳು ವ್ಯಥೆಪಡ ಬೇಕಾಗುವುದು.

ವಿಕಾರ ಎಂದರೆ ಜೀವಿಗಳು ಉಪಯೋಗಿಸುವ ದೇಹ, ಮನಸ್ಸು, ಬುದ್ಧಿ, ಇಂದ್ರಿಯಗಳಲ್ಲಿ ಆಗಿರುವ ಬದಲಾವಣೆಗಳೆಲ್ಲ ಪ್ರಕೃತಿಯಿಂದ ಆದದ್ದು. ಜೀವಿ ಅದನ್ನು ತನ್ನ ಕರ್ಮಾನುಸಾರ ಉಪಯೋಗಿಸುವುದು. ಇವುಗಳೆಲ್ಲಾ ಕಚ್ಚಾಸಾಮಾನಿನಂತೆ ಹೊರಗೆ ಇವೆ. ಒಂದು ಕಟ್ಟಡಕ್ಕೆ ಬೇಕಾದ ಸಾಮಾನುಗಳೆಲ್ಲ ಆಗಲೆ ಹೊರಗಡೆ ಇವೆ. ಆದರೆ ಒಂದೊಂದು ಒಂದೊಂದು ಕಡೆ ಇದೆ. ಇಟ್ಟಿಗೆ, ಅದಕ್ಕೆ ಬೇಕಾಗುವ ಮಣ್ಣು ಬೇರೆಲ್ಲೊ ಬಂಡೆಯ ರೂಪದಲ್ಲಿತ್ತು. ಗಾರೆಗೆ ಬೇಕಾಗುವ ಮರಳು ಮತ್ತು ಸುಣ್ಣ ಬೇರೆಲ್ಲೊಇತ್ತು. ಅದಕ್ಕೆ ಉಪಯೋಗಿಸುವ ಮರ ಕಬ್ಬಿಣ ಮುಂತಾದವುಗಳೆಲ್ಲ ಒಂದೊಂದೂ ಒಂದೊಂದು ಕಡೆ ಆಗಲೆ ಇವೆ. ಮನೆ ಕಟ್ಟಿಸುವವನು ತನ್ನಲ್ಲಿರುವ ದುಡ್ಡು ಮತ್ತು ತನ್ನ ಅಭಿರುಚಿ ಇದಕ್ಕೆ ಅನುಗುಣವಾಗಿ ಕಟ್ಟಿಸುತ್ತಾನೆ. ನಾಮರೂಪದ ದೃಷ್ಟಿಯಿಂದ ಈ ಮನೆ ಒಂದು ಸೃಷ್ಟಿಯಾಗಿರಬಹುದೆ ಹೊರತು, ಈ ಮನೆ ಯಾವ ವಸ್ತುವಿನಿಂದ ಆಯಿತು? ಅದರ ಪ್ರತಿಯೊಂದು ಭಾಗವೂ ಆಗಲೆ ಬೇರೆಲ್ಲೊ ಇತ್ತು. ಅವನ್ನು ಒಟ್ಟಿಗೆ ಸೇರಿಸಿದಾಗ ಅದಕ್ಕೊಂದು ಬೇರೆ ಹೆಸರು ಬಂತು, ಆಕಾರ ಬಂತು.

ಪ್ರಕೃತಿಯಲ್ಲಿ ಎಲ್ಲಾ ಇದೆ. ಒಳ್ಳೆಯದು ಕೆಟ್ಟದ್ದು, ಪ್ರಯೋಜಕವಾಗಿರುವುದು, ಅಪ್ರಯೋಜಕ ವಾಗಿರುವುದು, ಇವುಗಳೆಲ್ಲ ಇವೆ. ಆದರೆ ಜೀವರು ತಮಗೆ ಬೇಕಾದ ಯಾವುದನ್ನೋ ಆರಿಸಿಕೊಂಡು ತಾವು ವಾಸಮಾಡುವ ದೇಹ ಇಂದ್ರಿಯ ಮನಸ್ಸು ಮತ್ತು ಬುದ್ಧಿಯಿಂದ ಕೂಡಿದ ಮನೆಯನ್ನು ಕಟ್ಟಿಕೊಳ್ಳುತ್ತಾರೆ. ಪ್ರತಿಯೊಂದು ಜೀವವೂ ಖಾಲಿಯಾಗಿ ಬರುವುದಿಲ್ಲ. ಅದರಲ್ಲಿ ಆಗಲೇ ಹಿಂದಿನ ವಾಸನೆಗಳಿವೆ. ಆದರೆ ಅವುಗಳೆಲ್ಲ ಬೀಜರೂಪದಲ್ಲಿವೆ. ಬೀಜವನ್ನು ನೆಲಕ್ಕೆ ಹಾಕಿದರೆ ಪ್ರತಿಯೊಂದು ಬೀಜವೂ ತನ್ನ ಗುಣಕ್ಕೆ ಬೇಕಾಗುವ ಸಾರವಸ್ತುಗಳನ್ನು ನೆಲದಿಂದ ಹೀರುವುದು. ಹಾಗಲಕಾಯಿಯ ಬೀಜವೂ ಒಂದೇ ಕಡೆ ಇದೆ; ಕಬ್ಬಿನ ಸಸಿಯೂ ಒಂದೇ ಕಡೆ ಬೆಳೆಯುತ್ತಿದೆ. ಕಬ್ಬು ಸಿಹಿಯಾಗುವುದು, ಹಾಗಲಕಾಯಿ ಕಹಿ ಆಗುವುದು. ನೆಲಕ್ಕೆ ಅವುಗಳ ಮೇಲೆ ಏನಾದರೂ ಪಕ್ಷಪಾತವೆ? ಇಲ್ಲ, ನೆಲದಲ್ಲಿ ಎಲ್ಲವೂ ಇದೆ, ಯಾವ ಬೀಜ ಏನನ್ನು ಕೇಳುವುದೊ ಅದನ್ನು ಕೊಡುವುದು.

\begin{verse}
ಕಾರ್ಯಕರಣಕರ್ತೃತ್ವೇ ಹೇತುಃ ಪ್ರಕೃತಿರುಚ್ಯತೇ~।\\ಪುರುಷಃ ಸುಖದುಃಖಾನಾಂ ಭೋಕ್ತೃತ್ವೇ ಹೇತುರುಚ್ಯತೇ \versenum{॥ ೨ಂ~॥}
\end{verse}

{\small ಪ್ರಕೃತಿಯು ಕಾರ್ಯ ಮತ್ತು ಕರಣಗಳ ಉತ್ಪತ್ತಿಗೆ ಮೂಲ ಎನ್ನುತ್ತಾರೆ. ಪುರುಷ ಸುಖ ದುಃಖಗಳ ಅನುಭವಕ್ಕೆ ಕಾರಣ.}

ಹೇಗೆ ಒಂದು ಮನೆ ಮತ್ತು ಅದರಲ್ಲಿ ವಾಸಿಸುವವರು ಇದ್ದಾರೊ ಹಾಗೆಯೇ ಈ ದೇಹ ಮತ್ತು ಇದರಲ್ಲಿ ವಾಸಿಸುವವನು ಬೇರೆ ಬೇರೆ. ಪ್ರಕೃತಿ ಸ್ಥೂಲವಾಗಿರುವುದನ್ನು ಒದಗಿಸುವುದು. ಅದ ರಿಂದಲೇ ಕಾರ್ಯ ಎಂದರೆ ದೇಹ ಮತ್ತು ಅದು ಕೆಲಸ ಮಾಡುವ ಜ್ಞಾನೇಂದ್ರಿಯ ಕರ್ಮೇಂದ್ರಿಯಗಳು ಬರುವುವು. ಅದರಲ್ಲಿ ವಾಸ ಮಾಡುವ ಪುರುಷನೇ ಜೀವ. ತನಗೆ ಬರುವ ವೇದನೆಗಳಿಂದ ಸುಖ ಮತ್ತು ದುಃಖಗಳನ್ನು ಅನುಭವಿಸುವನು. ಸುಖ ದುಃಖವೆಂಬುದು ಜೀವಿಯ ಪ್ರತಿಕ್ರಿಯೆ. ಅದೆಲ್ಲೊ ಹೊರಗೆ ಬಿದ್ದಿಲ್ಲ. ಯಾವ ಒಂದು ವೇದನೆಯನ್ನೇ ಆಗಲಿ ಬರೀ ಸುಖ ಅಥವಾ ದುಃಖ ಎಂದು ಹೇಳಲಾಗುವುದಿಲ್ಲ. ಎಲ್ಲಾ ಜೀವಿಯು ಅದನ್ನು ನೋಡುವುದರ ಮೇಲಿದೆ. ಅದರಂತೆಯೇ ಪಾಪ ಪುಣ್ಯಗಳು ಕೂಡ.

\begin{verse}
ಪುರುಷಃ ಪ್ರಕೃತಿಸ್ಥೋ ಹಿ ಭುಂಕ್ತೇ ಪ್ರಕೃತಿಜಾನ್ ಗುಣಾನ್~।\\ಕಾರಣಂ ಗುಣಸಂಗೋಽಸ್ಯ ಸದಸದ್ಯೋನಿಜನ್ಮಸು \versenum{॥ ೨೧~॥}
\end{verse}

{\small ಪುರುಷ ಪ್ರಕೃತಿಯಿಂದ ಹುಟ್ಟಿದ ಗುಣಗಳನ್ನು ಅನುಭವಿಸುತ್ತಿರುವನು. ಅವನು ಒಳ್ಳೆಯ ಅಥವಾ ಕೆಟ್ಟ ಯೋನಿಯಲ್ಲಿ ಹುಟ್ಟುವುದಕ್ಕೆ ಗುಣಸಂಗವೇ ಕಾರಣ.}

ಜೀವ ದೇಹದಲ್ಲಿದ್ದುಕೊಂಡು ಸುಖ ದುಃಖಗಳೆಂಬ ಬೆಳೆಗಳನ್ನು ಅನುಭವಿಸುತ್ತಿರುವನು. ಇದನ್ನು ಅನುಭವಿಸುತ್ತಿರುವುದು ಪ್ರಕೃತಿಯಲ್ಲ. ಪ್ರಕೃತಿ ಅದನ್ನು ಕೊಡುವುದು. ಜೀವ ಅದನ್ನು ಅನುಭವಿಸುವುದು. ಅದರಲ್ಲಿ ಕೆಲವು ಸಿಹಿ ಮತ್ತು ಕೆಲವು ಕಹಿ, ಇದಕ್ಕೆಲ್ಲ ಕಾರಣ ಅವನು ಎಂತಹ ಬೀಜ ನೆಡುತ್ತಾನೆಯೋ ಅದರ ಮೇಲಿದೆ.

ಪ್ರತಿಯೊಂದು ಜೀವವೂ ಒಂದು ಕಡೆ ಹುಟ್ಟಬೇಕಾದರೆ ಅದಕ್ಕೆ ಒಂದು ಕಾರಣ ಇದೆ. ಅಕಸ್ಮಾತ್ತಾಗಿ ಅವನು ಅಲ್ಲಿ ಹುಟ್ಟುವುದಿಲ್ಲ. ಅದರ ಮೇಲೆ ಮಮತೆ ಮತ್ತು ವ್ಯಾಮೋಹ ಹಿಂದಿನಿಂದ ಇದೆ. ಅದರ ಮೂಲಕ ಬರುವ ಅನುಭವಕ್ಕಾಗಿ ಅವನು ಆಸೆ ಪಟ್ಟಿರುವನು. ಅವನೇನು ದುಃಖವನ್ನೇ ಬೇಕೆಂದು ಇಚ್ಛಿಸುವುದಿಲ್ಲ. ಅವನು ಸುಖವನ್ನು ಆಶಿಸುತ್ತಾನೆ. ಆದರೆ ಸುಖವನ್ನು ಚೀಪಿ ಆದಮೇಲೆ ಹಿಂದೆ ದುಃಖವಿರುವುದು. ಮಾವಿನ ಹಣ್ಣನ್ನು ತಿಂದಾದ ಮೇಲೆ ಓಟೆ ಸಿಕ್ಕುವುದು; ನಾವು ಬೇಕಾದರೆ ಓಟೆಯನ್ನು ಆಚೆಗೆ ಎಸೆಯಬಹುದು. ಆದರೆ ಇಲ್ಲಿ ಹಾಗೆ ಎಸೆಯುವುದಕ್ಕೆ ಆಗುವುದಿಲ್ಲ. ಮೇಲಿರುವ ಸುಖವನ್ನು ತಿಂದರೆ ಹಿಂದಿರುವ ದುಃಖವನ್ನು ತಿನ್ನಲೇಬೇಕಾಗಿದೆ. ನಾವು ಒಂದು ಕಡೆ ಹುಟ್ಟಬೇಕಾದರೆ ಅಲ್ಲಿಯ ಜನ ಮತ್ತು ಸುತ್ತಮುತ್ತಲಿನ ವಾತಾವರಣದ ಮೇಲೆ ನಮಗೆ ಇರುವ ಆಸಕ್ತಿಯೇ ಕಾರಣ. ಈ ಆಸಕ್ತಿ ನಮ್ಮನ್ನು ಅಲ್ಲಿಗೆ ಒಯ್ಯುವುದು.

\begin{verse}
ಉಪದ್ರಷ್ಟಾಽನುಮಂತಾ ಚ ಭರ್ತಾ ಭೋಕ್ತಾ ಮಹೇಶ್ವರಃ~।\\ಪರಮಾತ್ಮೇತಿ ಚಾಪ್ಯುಕ್ತೋ ದೇಹೇಽಸ್ಮಿನ್ ಪುರುಷಃ ಪರಃ \versenum{॥ ೨೨~॥}
\end{verse}

{\small ಈ ದೇಹದಲ್ಲಿ ಪರಮಪುರುಷನು ಸಾಕ್ಷಿ, ಅನುಮೋದಿಸುವವನು, ಭರ್ತೃ ಭೋಕ್ತೃ ಮಹೇಶ್ವರ, ಮತ್ತು ಪರಮಾತ್ಮ ಎಂದು ಹೇಳಲ್ಪಡುವನು.}

ಈ ದೇಹದಲ್ಲಿ ಇಬ್ಬರು ಇದ್ದಾರೆ. ಒಬ್ಬ ಪುರುಷ, ಜೀವಾತ್ಮ ಈ ದೇಹದಲ್ಲಿ ಮಾತ್ರ ಇರುವನು. ಮತ್ತೊಬ್ಬ ಪರಮಪುರುಷ. ಅವನು ಎಲ್ಲಾ ಜೀವಗಳಿಗೂ ಸಾಮಾನ್ಯವಾದ ಹಿನ್ನೆಲೆ ಆಗಿದ್ದಾನೆ. ಸಾಗರದಲ್ಲಿ ಅಲೆಗಳಿವೆ, ಒಂದು ಅಲೆ ಮತ್ತೊಂದು ಅಲೆಗಿಂತ ಬೇರೆ. ಆದರೆ ಎಲ್ಲಾ ಅಲೆಗಳಿಗೂ ಸರ್ವಸಾಮಾನ್ಯವಾದ ಸಾಗರವಿದೆ. ಅದರಂತೆಯೇ ಪರಮಪುರುಷ ಸಾಧಾರಣ ಪುರುಷನ ಹಿನ್ನೆಲೆ ಯಾಗಿದ್ದಾನೆ.

ಪರಮಪುರುಷ ಇಲ್ಲಿ ಹೇಗಿದ್ದಾನೆ ಎಂಬುದನ್ನು ಹೇಳುವನು. ಅವನು ಸಾಕ್ಷಿಯಂತೆ ಇದ್ದಾನೆ, ಅವನು ಎಲ್ಲವನ್ನೂ ನೋಡುತ್ತಿರುವನು. ಆದರೆ ಯಾವುದರಿಂದಲೂ ಬಾಧಿತನಾಗುವುದಿಲ್ಲ. ದೀಪದ ಬೆಳಕು ಇದೆ. ಅದರಲ್ಲಿ ಒಬ್ಬೊಬ್ಬರು ಒಂದೊಂದು ಕೆಲಸವನ್ನು ಮಾಡುತ್ತಿರುವರು. ಆ ದೀಪ ವಾದರೊ ಎಲ್ಲವನ್ನೂ ನೋಡುತ್ತಿದೆ.

ಅವನು ಯಾವುದನ್ನು ಮಾಡಬೇಕಾದರೂ ಒಪ್ಪಿಗೆ ಕೊಡುತ್ತಾನೆ. ಆದಕಾರಣ ತಪ್ಪೆಲ್ಲಾ ಅವನದೇ ಎಂದು ಹೇಳಲಾಗುವುದಿಲ್ಲ. ಜೀವಿಗಳಿಗೆ ಒಂದು ಮಿತಿಯಲ್ಲಿ ದೇವರು ಸ್ವಾತಂತ್ರ್ಯ ಕೊಟ್ಟಿರುವನು. ಅವರು ತಮಗೆ ತೋಚಿದ ರೀತಿಯಲ್ಲಿ ಕೆಲಸ ಮಾಡುತ್ತಾರೆ. ದೇವರು ಹಾಗೆ ಮಾಡು, ಹೀಗೆ ಮಾಡು ಎಂದು ಹೇಳುವುದಿಲ್ಲ. ಆದರೆ ಅವರು ಹೇಗೆ ಕರ್ಮ ಮಾಡುತ್ತಾರೋ ಅದರಂತೆ ಅನುಭವಿಸ ಬೇಕಾಗಿದೆ. ಅದೂ ಪ್ರಕೃತಿ ನಿಯಮ. ಒಬ್ಬ ಮೆಣಸಿನ ಕಾಯನ್ನು ಬೇಕಾದರೆ ಅಗಿಯಬಹುದು. ಕಬ್ಬಿನ ಜಲ್ಲೆಯನ್ನು ಬೇಕಾದರೆ ಅಗಿಯಬಹುದು. ಒಂದರಿಂದ ಬಾಯಿ ಉರಿಯುತ್ತದೆ. ಮತ್ತೊಂದರಿಂದ ಬಾಯಿ ಸಿಹಿ ಆಗುತ್ತದೆ. ಜೀವಿಗಳು ಹೇಗೆ ಬೇಕಾದರೂ ಮಾಡಬಹುದು. ಆದರೆ ಅವರು ಮಾಡಿದ್ದಕ್ಕೆ ತಕ್ಕದ್ದನ್ನು ಅನುಭವಿಸಬೇಕಾಗಿದೆ.

ಅವನು ಭರ್ತೃ ಎಂದರೆ ಎಲ್ಲವನ್ನೂ ಹೊರುವನು. ಎಂದರೆ ಎಲ್ಲದಕ್ಕೂ ಆಧಾರ ಅವನೆ. ನಮ್ಮ ದೇಹ, ಮನಸ್ಸು, ಇಂದ್ರಿಯ, ಬುದ್ಧಿ, ಅಹಂಕಾರ ಇವುಗಳೆಲ್ಲ ಬಂದಿರುವುದು, ಇವುಗಳೆಲ್ಲ ಇರುವುದು ಅವನಿಂದ. ನಾನು ರೈಲಿನಲ್ಲಿ ಕುಳಿತುಕೊಂಡು ಹೋಗುತ್ತಿರುವೆನು. ಮಗುವನ್ನು ತೊಡೆಯ ಮೇಲೆ ಕೂಡಿಸಿಕೊಳ್ಳುವೆನು. ಸಾಕಾದಾಗ ಪಕ್ಕದ ಸೀಟಿನ ಮೇಲೆ ಕುಳ್ಳಿರಿಸುವೆನು. ಆದರೆ ಅದು ನನ್ನ ತೊಡೆಯ ಮೇಲೆ ಕುಳಿತಾಗಲೂ, ಪಕ್ಕದಲ್ಲಿ ಕುಳಿತಾಗಲೂ, ಯಾವಾಗಲೂ ಹೊರುತ್ತಿ ದ್ದುದು ರೈಲೆ, ಬೇರಲ್ಲ. ನನ್ನ ದೃಷ್ಟಿಯಿಂದ ವ್ಯತ್ಯಾಸವಾಗಬಹುದು. ಆದರೆ ರೈಲಿನ ದೃಷ್ಟಿಯಿಂದ ಯಾವ ವ್ಯತ್ಯಾಸವೂ ಎಳೆಯುವ ಭಾರದಲ್ಲಿ ಆಗಲಿಲ್ಲ. ಸ್ವಾರ್ಥಿ, ಇದೆಲ್ಲ ತನ್ನದು ಎಂದು ತನ್ನ ತೊಡೆಯ ಮೇಲೆ ಇಟ್ಟುಕೊಂಡಿರುವಾಗಲೂ ದೇವರೇ ಅದನ್ನು ಹೊರುವನು. ಸ್ವಲ್ಪ ಅನಾಸಕ್ತನಾಗಿ ದೂರದಲ್ಲಿಟ್ಟರೂ ದೇವರೆ ಅದನ್ನು ಹೊರುವನು.

ಅವನು ಭೋಕ್ತೃ ಎಂದರೆ ಇದನ್ನೆಲ್ಲ ಅನುಭವಿಸುವವನು. ಅವನು ಹೇಗೆ ಅನುಭವಿಸುತ್ತಾನೆ, ಪ್ರೇಕ್ಷಕನಂತೆ ಇದನ್ನು ನೋಡಿ ಅನುಭವಿಸುವನು. ಇದೆಲ್ಲಾ ನನ್ನದು ಎಂದು ನೋಡಿದರೆ ಅನು ಭವಿಸುವುದಕ್ಕಿಂತ ಹೆಚ್ಚಾಗಿ ಸಂಕಟಪಡುವುದೇ ಹೆಚ್ಚಾಗಿದೆ. ಒಂದು ಫುಟ್​ಬಾಲ್ ಮ್ಯಾಚ್ ನೋಡುವುದಕ್ಕೆ ಹೋದರೆ, ಆಟವನ್ನು ಯಾರು ಚೆನ್ನಾಗಿ ನೋಡಿ ಮೆಚ್ಚುವರು? ಯಾರ ಪಕ್ಷವನ್ನೂ ಯಾರು ತೆಗೆದುಕೊಂಡಿಲ್ಲವೋ ಅವನು. ಪಕ್ಷಪಾತಿಗಳು ತಮ್ಮ ಕಡೆಯವರು ಗೆದ್ದರೆ ಸಂತೋಷ ಪಡುವರು, ತಮ್ಮ ಕಡೆಯವರು ಸೋತರೆ ದುಃಖ ಪಡುವರು. ಆದರೆ ಯಾರು ಯಾರ ಕಡೆಗೂ ಸೇರಿಲ್ಲವೋ ಆತ ನಮ್ಮೆಲ್ಲರಿಗಿಂತ ಚೆನ್ನಾಗಿ ನೋಡಿ ಸಂತೋಷ ಪಡುವನು. ಏಕೆಂದರೆ ಯಾರ ಕಡೆ ಗೆದ್ದರೂ ಅವನಿಗೆ ಸಂತೋಷವೇ. ಒಂದು ವಸ್ತುವನ್ನು ನೋಡಿ ಪರಮಶ್ರೇಷ್ಠ ಆನಂದವನ್ನು ಪಡೆಯುವ ರೀತಿಯೇ ಇದು. ಇದೊಂದು ಲೀಲೆ ಭಗವಂತನಿಗೆ. ಇದನ್ನು ನಿಷ್ಪಕ್ಷಪಾತಿಯಾದ ಪ್ರೇಕ್ಷಕನು ಹೇಗೆ ನೋಡಿ ಆನಂದಿಸುತ್ತಾನೆಯೋ ಹಾಗೆ ನೋಡುತ್ತಾನೆ ದೇವರು. ಇದನ್ನೇ ದೇವರ ಲೀಲೆ ಎಂಬುದು. ಅವನು ಏನಾದರೂ ಯಾರ ಪಕ್ಷವನ್ನಾದರೂ ಹಿಡಿದರೆ ಇದು ಲೀಲೆ ಆಗುವುದಿಲ್ಲ. ಅವನು ಯಾರ ಕಡೆಗೂ ವಾಲಬಾರದು, ಯಾರ ಪಕ್ಷವನ್ನೂ ಹಿಡಿಯಬಾರದು. ಈ ದೃಷ್ಟಿಯಿಂದ ಅವನು ಅನುಭವಿಸುತ್ತಾನೆ. ಆದರೆ ಸಾಧಾರಣ ಜೀವರು ಅನುಭವಿಸುವಾಗಲಾದರೋ, ಅವರು ಪಕ್ಷಪಾತವಿಲ್ಲದೆ ಯಾವುದನ್ನು ಮಾಡಲಾರರು, ಯಾವುದನ್ನೂ ನೋಡಲಾರರು. ಅದಕ್ಕೆ ನಾವು ಎಷ್ಟು ಸುಖಪಡುತ್ತೇವೆಯೋ ಅಷ್ಟು ದುಃಖವನ್ನೂ ಪಡಬೇಕಾಗಿದೆ. ಆದರೆ ನಿಷ್ಪಕ್ಷಪಾತಿಗೆ ಮಾತ್ರ ಎಲ್ಲರಿಗಿಂತ ಆನಂದ ಸಿಕ್ಕಬೇಕಾದರೆ.

ಆ ಪರಮಪುರುಷ ಮಹೇಶ್ವರ. ಎಲ್ಲಾ ಜೀವರಾಶಿಗಳಿಗೂ ಮತ್ತು ಪ್ರಕೃತಿಗೂ ಒಡೆಯನಾಗಿ ದ್ದಾನೆ. ಯಾರೂ ಅವನ ಆಜ್ಞೆಯನ್ನು ಮೀರಿ ಹೋಗಲಾರರು. ಜಡವಾಗಲಿ ಚೇತನವಾಗಲಿ, ಅವನ ಆಜ್ಞೆಗೆ ಬಾಗಿ ನಡೆಯಬೇಕು. ನಾವು ಮಾಡಿದುದಕ್ಕೆ ತಕ್ಕ ಪರಿಣಾಮವನ್ನು ಅನುಭವಿಸಿಯೇ ತೀರಬೇಕು. ಕರ್ಮ ಸಿದ್ಧಾಂತದಂತೆ ಇದೇ ನಮ್ಮನ್ನು ಬಂದು ಕಾಡುವುದು. ಯಾರೂ ಅವನ ಕಣ್ಣಿಗೆ ಮಣ್ಣನ್ನು ಹಾಕುವುದಕ್ಕೆ ಆಗುವುದಿಲ್ಲ. ನಾವು ಮಾಡಿದಂತೆ ಅನುಭವಿಸಿಯೇ ತೀರಬೇಕು. ಏಕೆಂದರೆ ಇದು ಮಹೇಶ್ವರನ ಆಜ್ಞೆ.

ಅವನು ಪರಮಪುರುಷ, ಶ್ರೇಷ್ಠವಾದ ಪುರುಷ. ಅವನು ಯಾವ ಕರ್ಮದ ಬೆಲೆಗೂ ಸಿಲುಕಿಲ್ಲ. ಅವನಲ್ಲಿಅಜ್ಞಾನವಿಲ್ಲ. ಪೂರ್ಣಾತ್ಮ ಅವನು; ಸರ್ವಜ್ಞ ಅವನು. ಮಾಯೆಯನ್ನು ಆಳುವವನು ಅವನು. ಮಾಯೆಯ ವಶನಲ್ಲ. ಇಲ್ಲಿ ಯಾವುದೂ ಅವನನ್ನು ಬಂಧಿಸಲಾರದು. ಎಲ್ಲದಕ್ಕೂ ಅತೀತ ಅವನು.

\begin{verse}
ಯ ಏವಂ ವೇತ್ತಿ ಪುರುಷಂ ಪ್ರಕೃತಿಂ ಚ ಗುಣೈಃ ಸಹ~।\\ಸರ್ವಥಾ ವರ್ತಮಾನೋಽಪಿ ನ ಸ ಭೂಯೋಽಭಿಜಾಯತೇ\versenum{॥೨೩~॥}
\end{verse}

{\small ಯಾರು ಹೀಗೆ ಪುರುಷನನ್ನೂ, ಗುಣಗಳೊಂದಿಗೆ ಪ್ರಕೃತಿಯನ್ನೂ ಅರಿತುಕೊಳ್ಳುವನೊ ಅವನು ಹೇಗಿದ್ದರೂ ಪುನಃ ಹುಟ್ಟುವುದಿಲ್ಲ.}

ನಮ್ಮ ಭ್ರಾಂತಿಜೀವನದಲ್ಲಿ ಪುರುಷ, ಪರಮಪುರುಷ ಮತ್ತು ಪ್ರಕೃತಿ ಇವುಗಳೆಲ್ಲೆ ಕಲಸು ಮೇಲೋಗರ ಆಗಿಹೋಗಿವೆ. ಯಾವುದು ಎಲ್ಲಿ ಪ್ರಾರಂಭವಾಗುತ್ತದೆ ಮತ್ತು ಎಲ್ಲಿ ಕೊನೆಗೊಳ್ಳು ತ್ತದೆ, ಅವುಗಳ ಗುಣಗಳೇನು ಮತ್ತು ಮೇರೆಗಳೇನು ಎಂಬುದು ನಮಗೆ ಗೊತ್ತಿರುವುದೇ ಇಲ್ಲ. ಆದರೆ ಜ್ಞಾನಿಯಾದವನು ಮೊದಲು ಮಾಡುವ ಕೆಲಸವೇ ಇದು. ಅವನು ಮೊದಲು ನಾನಾರು ಮತ್ತು ಜಗತ್ತಿನ ಅಂಶಗಳಾವುವು ಎಂಬುದನ್ನು ವಿಮರ್ಶಿಸುವನು. ನಾವು ಯಾವುದನ್ನು ನಾನು ಎಂದು ಕರೆಯುತ್ತಿರುವೆನೊ ಅದರಲ್ಲಿ ಮೂರು ವಸ್ತುಗಳಿವೆ. ಅದರಲ್ಲಿ ಅತ್ಯಂತ ಹೊರಗಿನದೆ ಪ್ರಕೃತಿಯಿಂದ ಆದ ದೇಹ. ಈ ಪ್ರಂಚೇಂದ್ರಿಯ, ಬುದ್ಧಿ, ಮನಸ್ಸು ಇವುಗಳೆಲ್ಲ ಪಂಚಭೂತಗಳು ಮತ್ತು ಸತ್ತ್ವ, ರಜಸ್ಸು, ತಮಸ್ಸು ಎಂಬ ಮೂರು ಗುಣಗಳಿಂದ ಆದುವುಗಳು. ಇವು ನಾನಲ್ಲ, ನಾನು ಉಪಯೋಗಿ ಸುವ ವಸ್ತುಗಳು. ನಾವು ಇವುಗಳನ್ನು ಉಪಯೋಗಿಸುತ್ತಾ ಉಪಯೋಗಿಸುತ್ತಾ ಇವುಗಳಲ್ಲಿ ಆಸಕ್ತರಾಗಿ ಇವು ನಾವು ಎಂದು ಭ್ರಮಿಸುತ್ತೇವೆ. ಇವುಗಳ ಗುಣಗಳನ್ನೇ ನಮ್ಮ ಮೇಲೆ ಆರೋಪ ಮಾಡಿಕೊಂಡು ಅದಕ್ಕಾಗಿ ಸಂತೋಷ ಪಡುತ್ತೇವೆ ಅಥವಾ ದುಃಖ ಪಡುತ್ತೇವೆ. ದೇಹ ದಪ್ಪವಾಗಿ ದ್ದರೆ ನಾನು ದಪ್ಪವಾಗಿದ್ದೇನೆ ಎಂದು ಭಾವಿಸುತ್ತೇನೆ, ಅದಕ್ಕೆ ರೋಗ ರುಜಿನಗಳು ಬಂದರೆ ನನಗೆ ಅದೆಲ್ಲ ಬಂತು ಎಂದು ಭಾವಿಸುತ್ತೇನೆ. ದೃಶ್ಯಗುಣಗಳನ್ನೆಲ್ಲ ದೃಗ್​ಮೇಲೆ ಆರೋಪ ಮಾಡಿಕೊಂಡು, ದೃಗ್ ಸಂಪೂರ್ಣವಾಗಿ ಮರೆತು ದೃಶ್ಯದಲ್ಲಿ ನಿರತರಾಗುತ್ತೇವೆ, ಮಗು ಆಟದ ಸಾಮಾನಿನಲ್ಲಿ ತಲ್ಲೀನವಾಗಿರುವಂತೆ.

ಅನಂತರವೇ ನಾನೆಂಬುದು ಬರುವುದು. ಯಾವುದು ಈ ದೇಹ ಮನಸ್ಸು, ಬುದ್ಧಿಗಳನ್ನು ತಿಳಿದುಕೊಳ್ಳುವುದೊ, ಯಾವುದು ಇವುಗಳಿಗೆಲ್ಲ ಸಾಕ್ಷಿಯಂತೆ ಇರುವುದೊ ಅದೇ ಪುರುಷ. ಅದು ಒಂದು ಹಕ್ಕಿ ಗೂಡಿನಲ್ಲಿರುವಂತೆ ಬಂದು ಈ ದೇಹದಲ್ಲಿ ಇದ್ದು ಹೋಗುವುದು, ಅದು ಹಿಂದೆ ಇಂತಹ ಹಲವು ದೇಹವೆಂಬ ಗೂಡುಗಳಲ್ಲಿ ಬಂದು ಹೋಗಿರುವುದು. ಈ ದೇಹ, ಇಂದ್ರಿಯ, ಮನಸ್ಸು ಇವುಗಳ ಮೂಲಕ ಬೇಕಾದಷ್ಟು ಸುಖ ದುಃಖ, ಲಾಭ ನಷ್ಟಗಳನ್ನು ಅನುಭವಿಸಿದೆ. ಈ ಕ್ಷೇತ್ರದಲ್ಲಿ ಜೀವ ಬೇಕಾದಷ್ಟು ಉತ್ತಿದೆ, ಬಿತ್ತಿದೆ, ಕೊಯ್ದಿದೆ. ಪ್ರತಿಸಲವೂ ಹಿಂದೆ ಇರುವ ದುಃಖ, ದುಃಖದ ಹಿಂದೆ ಇರುವ ಸುಖದ ಬೆಳೆಯನ್ನು ಕೊಯ್ದಿದೆ. ಪ್ರತಿಸಲವೂ ದುಃಖವಿಲ್ಲದ ಸುಖ ಸಿಕ್ಕುವುದೇನೊ ಎಂದು ನಿರೀಕ್ಷಿಸಿದೆ. ಆದರೆ ಒಂದನ್ನು ಬಿಟ್ಟು ಮತ್ತೊಂದು ಇರಲಾರದು. ಒಂದರ ಹಿಂದೆ ಮತ್ತೊಂದು ನೆರಳಿನಂತೆ ಹೋಗುತ್ತಿರುವುದು. ಒಂದು ಬೇಕಾದರೆ ಮತ್ತೊಂದನ್ನು ಒಪ್ಪಿ ಕೊಳ್ಳಬೇಕು; ಜನನ ಬೇಕಾದರೆ ಮರಣಕ್ಕೆ ಸಿದ್ಧರಾಗಿರಬೇಕು. ಇಲ್ಲಿ ತೃಪ್ತಿ ಸಿಕ್ಕದೆ ಜೀವ ಅಂತರ್ ಮುಖವಾಗಿ ಹುಡುಕಲಾರಂಭಿಸುವುದು.

ಅಂತರ್ಮುಖದ ಅನ್ವೇಷಣೆಯಿಂದಲೇ ಆಧ್ಯಾತ್ಮಿಕ ಜೀವನ ಪ್ರಾರಂಭವಾಗುವುದು. ಈ ಜೀವನ ದಲ್ಲಿ ನಾನೆಂಬುದಕ್ಕಿಂತ ಶ್ರೇಷ್ಠವಾಗಿರುವುದು ಯಾವುದು, ಅದನ್ನು ಕಂಡು ಹಿಡಿಯಬೇಕಾದರೆ ಏನು ಮಾಡಬೇಕು ಎಂದು ಯೋಚಿಸುತ್ತಿರುವಾಗಲೆ, ಯಾವುದು ನನ್ನನ್ನು ನನ್ನದಲ್ಲದ ವಸ್ತುವಿಗೆ ಬಿಗಿದಿದೆಯೊ ಅದರಿಂದ ಮೊದಲು ಬಿಡಿಸಿಕೊಳ್ಳಬೇಕಾಗಿದೆ ಎಂಬುದು ಹೊಳೆಯುವುದು. ಆಗಲೆ ಒಬ್ಬ ಏಳನೆಯ ಶ್ಲೋಕದಿಂದ ಹನ್ನೊಂದನೇ ಶ್ಲೋಕದವರೆಗೆ ಇರುವ ಗುಣಗಳನ್ನು ತನ್ನ ಜೀವನದಲ್ಲಿ ರೂಢಿಸಿಕೊಳ್ಳಬೇಕಾಗಿ ಬರುವುದು. ಅನಂತರ ನನ್ನ ಹಿಂದೆ ಇರುವ ನಮ್ಮೆಲ್ಲರಿಗೆ ನಿತ್ಯ ಸಾಮಾನ್ಯ ಹಿನ್ನೆಲೆಯಂತೆ ಇರುವ ಪುರುಷನ ವಿಷಯವನ್ನು ಹನ್ನೆರಡನೆ ಶ್ಲೋಕದಿಂದ ಹದಿನಾರನೆಯ ಶ್ಲೋಕದವರೆಗೆ ಹೇಳಿದ್ದಾಯಿತು. ಆ ಪರಮಪುರುಷ ಅಥವಾ ಪರಮಾತ್ಮ ನಮ್ಮಲ್ಲೆಲ್ಲಾ ಓತಪ್ರೋತ ವಾಗಿದ್ದಾನೆ. ಅವನಿಲ್ಲದ ಸ್ಥಳವೇ ಇಲ್ಲ. ಅವನಿಲ್ಲದ ಕಾಲವೇ ಇಲ್ಲ. ಜೀವ ಜಗತ್ತುಗಳಿಗೆ ಮೂಲ ಕಾರಣನಾದವನೇ ಪರಮಪುರುಷ. ಅವನು ನಮ್ಮ ಹಿಂದೆಯೇ ಇದ್ದಾನೆ ಎಂದು ನಾವು ಅವನನ್ನು ಚಿಂತಿಸತೊಡಗಿ ಅವನಲ್ಲೆ ತನ್ಮಯರಾಗುತ್ತಾ ಬಂದರೆ ನಾವು ಕೂಡ ಅವನಂತೆ ಆಗುತ್ತೇವೆ. ಆಧ್ಯಾತ್ಮಿಕ ಜೀವನ ಎಂದರೆ ನನ್ನ ಮುಂದೆ ಇರುವುದನ್ನು ಬಿಟ್ಟು ನನ್ನ ಹಿಂದೆ ಹೋಗಬೇಕಾಗಿದೆ. ಅಲ್ಲಿರುವ ಪರಮಪುರುಷನನ್ನು ತಿಳಿದಾದಮೇಲೆ ನಾವು ಮುಂಚೆ ಯಾವುದನ್ನು ಬಿಟ್ಟಿರುವೆವೊ ಆ ನಾಮರೂಪಗಳ ಹಿಂದೆಲ್ಲಾ ಅವನೇ ನಿಜವಾಗಿ ಸತ್ಯವಾಗಿರುವವನು ಎಂಬುದು ಅರ್ಥವಾಗುವುದು. ಅಜ್ಞಾನದಲ್ಲಿ ಕೇವಲ ನಮಗೆ ಕಾಣುವ ನಾಮರೂಪಗಳು ಮಾತ್ರ ಸತ್ಯ ಎಂದು ಭಾವಿಸಿ ಅದರಲ್ಲೆ ಉನ್ಮತ್ತಾರಾಗಿರುತ್ತೇವೆ. ಜ್ಞಾನ ಬಂದಾಗ ನಾಮರೂಪವನ್ನು ನೋಡುತ್ತೇವೆ. ಆದರೆ ಅದಕ್ಕೆ ಪರವಶರಾಗುವುದಿಲ್ಲ. ನಿಜವಾಗಿ ಸತ್ಯವಾಗಿರುವುದು ನಾಮರೂಪದ ಹಿಂದೆ ಯಾವ ಬದಲಾವಣೆಗೂ ಸಿಲುಕದ ಪರಮಾತ್ಮನೊಬ್ಬನೆ. ಅಜ್ಞಾನದಲ್ಲಿ ಅವನನ್ನು ಕಾಣದಂತೆ ಮಾಡಿಕೊಂಡವರು ನಾವೆ. ಹೊರಗೆ ಪಂಚಭೂತಗಳಿಂದ ಆದ ವಿಕಾರಗಳನ್ನು ಸತ್ಯ ಎಂದು ಭಾವಿಸಿದ್ದೆವು. ಒಳಗೆ ನನ್ನ ದೇಹ ಮನಸ್ಸು ಬುದ್ಧಿ ಎಂಬುದನ್ನು ಮಾತ್ರ ಸತ್ಯವೆಂದು ಭಾವಿಸಿದ್ದೆವು. ಜ್ಞಾನ ಬಂದಾಗ ಹಿಂದೆ ಹೇಳಿದ ವಸ್ತುಗಳೆಲ್ಲ ಇರುವುದು, ಆದರೆ ನೋಡುವ ದೃಷ್ಟಿ ವ್ಯತ್ಯಾಸವಾಗುವುದು. ಇವುಗಳೆಲ್ಲ ಇರುವುದು ಬರೀ ನಾಮರೂಪಗಳು, ಬರೀ ಹೊಟ್ಟು. ಅದರ ಹಿಂದೆ ಇರುವ ಪರಮಪುರುಷನೊಬ್ಬನೇ ಸತ್ಯ ವೆಂದು ಕಾಣುವುದು.

ಇವುಗಳನ್ನೆಲ್ಲ ಯಾವಾಗ ಒಬ್ಬ ಚೆನ್ನಾಗಿ ತಿಳಿದುಕೊಂಡು ತಾನು ಮತ್ತು ತಾನಿರುವ ಕ್ಷೇತ್ರ ಇವುಗಳಿಗಿಂತ ಶ್ರೇಷ್ಠವಾದ ಪರಮಪುರುಷನನ್ನು ಸಾಕ್ಷಾತ್ಕಾರ ಮಾಡಿಕೊಳ್ಳುತ್ತಾನೆಯೊ, ಅವನು ಹೇಗೆ ಇದ್ದರೂ ಪುನಃ ಹುಟ್ಟುವುದಿಲ್ಲ. ಎಂದರೆ ಬದ್ಧನಾಗಿ ಪ್ರಪಂಚಕ್ಕೆ ತನ್ನ ಕರ್ಮದ ಸಾಲವನ್ನು ತೀರಿಸಲು ಬರುವುದಿಲ್ಲ. ಅವನು ಒಂದೇ ಸಲ ಮುಕ್ತನಾಗಿ ಹೋಗುತ್ತಾನೆ. ಹೋದವನು ಹೋಗಿಯೇ ಬಿಡುತ್ತಾನೆ, ಹಿಂತಿರುಗಿ ಬರುವುದಿಲ್ಲ. ಅವನು ದೇವನೆಡೆಗೆ ಹೋಗುತ್ತಾನೆ. ಅವನಿನ್ನು ಪ್ರಪಂಚಕ್ಕೆ ಬರುವುದಿಲ್ಲ. ಅವನು ಹೋಗುವಾಗ ಹೇಗೆ ಬೇಕಾದರೂ ಇರಬಹುದು ಎಂದು ಶ‍್ರೀಕೃಷ್ಣ ಹೇಳುತ್ತಾನೆ. ಮುಕ್ತರಾದವರೆಲ್ಲ ಒಂದೇ ವೃತ್ತಿಯಲ್ಲಿರಬೇಕು ಎಂದು ಹೇಳುವುದಿಲ್ಲ. ಅಥವಾ ಅವರು ಮಾಡುತ್ತಿರುವ ವೃತ್ತಿಯನ್ನು ಬಿಡುವುದೂ ಇಲ್ಲ. ತಮ್ಮ ಪಾಲಿಗೆ ಯಾವ ಕೆಲಸ ಬಂದಿರುವುದೋ ಅದನ್ನು ಅನಾಸಕ್ತರಾಗಿ ಮಾಡುತ್ತಾ ಇರುವರು. ಆದರೆ ಅವರಲ್ಲಿ ಪರಮ ಪುರುಷನ ಅನುಭವ ಸದಾಕಾಲದಲ್ಲಿಯೂ ಇರುವುದು. ಮಹಾಭಾರತದಲ್ಲಿ ವ್ಯಾಧನಿಗೆ ಜ್ಞಾನ ಬಂದ ಮೇಲೆ ತನ್ನ ವೃತ್ತಿಯನ್ನು ಬಿಡುವುದಿಲ್ಲ. ಆದರೂ ಅವನು ಧರ್ಮವ್ಯಾಧನಾಗುತ್ತಾನೆ. ಅಲ್ಲಿ ಬರುವ ಜ್ಞಾನಿಯಾದ ಗೃಹಿಣಿ ಮನೆ ಬಿಟ್ಟು ಹೋಗಿಬಿಡುವುದಿಲ್ಲ. ಅಲ್ಲಿಯೇ ಜ್ಞಾನಿಯಾಗಿ ಆ ಜ್ಯೋತಿಯನ್ನು ಸುತ್ತಲೂ ಬೆಳಗುತ್ತಿರುವಳು. ಕಬೀರ ನೆಯ್ಗೆ ಕೆಲಸ ಬಿಡಲಿಲ್ಲ. ಗೋರಕುಂಬಾರ ತನ್ನ ಕೆಲಸ ಬಿಡಲಿಲ್ಲ. ಹಾಗೆಯೇ ಅನೇಕ ಭಕ್ತರು ಮತ್ತು ಜ್ಞಾನಿಗಳು ಹಿಂದೆ ತಾವು ಯಾವ ಕೆಲಸವನ್ನು ಮಾಡುತ್ತಿದ್ದರೊ ಅದನ್ನೇ ಮಾಡುತ್ತಿರುವರು. ಆದರೂ ಅವರು ಮುಕ್ತಪುರುಷರು. ಅವರು ಮಾಡುವ ಕರ್ಮ ಅವರನ್ನು ಬಂಧನಕ್ಕೆ ಒಳಗುಮಾಡುವುದಿಲ್ಲ. ಏಕೆಂದರೆ ಅದರ ಹಿಂದೆ ಯಾವ ಫಲಾಪೇಕ್ಷೆ ಇಲ್ಲ.

\begin{verse}
ಧ್ಯಾನೇನಾತ್ಮನಿ ಪಶ್ಯಂತಿ ಕೇಚಿದಾತ್ಮಾನಮಾತ್ಮನಾ~।\\ಅನ್ಯೇ ಸಾಂಖ್ಯೇನ ಯೋಗೇನ ಕರ್ಮಯೋಗೇನ ಚಾಪರೇ \versenum{॥ ೨೪~॥}
\end{verse}

{\small ಕೆಲವರು ಧ್ಯಾನದ ಮೂಲಕ (ಪರಮ)ಆತ್ಮನನ್ನು ಆತ್ಮನಲ್ಲಿ ಆತ್ಮನ ಮೂಲಕ ನೋಡುತ್ತಾರೆ. ಮತ್ತೆ ಕೆಲವರು ಜ್ಞಾನಯೋಗದಿಂದಲೂ ಕರ್ಮಯೋಗದಿಂದಲೂ ಅವನನ್ನು ನೋಡುತ್ತಾರೆ.}

ಶ‍್ರೀಕೃಷ್ಣ ಇಲ್ಲಿ ಪರಮಾತ್ಮನ ಸಾಕ್ಷಾತ್ಕಾರದ ಮೂರು ಮಾರ್ಗಗಳನ್ನು ವಿವರಿಸುತ್ತಾನೆ. ಮೊದಲ ನೆಯದು ಧ್ಯಾನ ಮಾರ್ಗ. ಅಲ್ಲಿ ಪರಮಾತ್ಮನನ್ನು ತನ್ನ ಹೃದಯಯಾಂತರಾಳದಲ್ಲಿ ಶುದ್ಧವಾದ ಮನಸ್ಸಿನಿಂದ ನೋಡುತ್ತಾನೆ; ಇದೇ ಧ್ಯಾನಮಾರ್ಗ. ಇಲ್ಲಿ ಚಿತ್ತಶುದ್ಧಿಯೇ ಅತ್ಯಂತ ಮುಖ್ಯವಾಗಿ ಬೇಕಾಗಿರುವುದು. ಎಲ್ಲಿ ಚಿತ್ತ ಶುದ್ಧವಾಗಿದೆಯೋ, ಅಲ್ಲಿ ಆ ಚಿತ್ತ ಪರಮಾತ್ಮನನ್ನು ಪ್ರತಿಬಿಂಬಿಸು ವುದು. ಸಾಧಾರಾಣವಾಗಿ ಚಿತ್ತ ಯಾವಾಗಲೂ ಕ್ಷೋಭೆಗೊಂಡಿರುವುದು. ಪಂಚೇಂದ್ರಿಯಗಳ ಗವಾಕ್ಷದ ಮೂಲಕ ಯಾವಾಗಲೂ ಸುದ್ದಿ ಸಮಾಚಾರಗಳು ಚಿತ್ತಕ್ಕೆ ಬಂದು ಬೀಳುತ್ತ ಇರುತ್ತವೆ. ಈ ಯೋಗದಲ್ಲಿ ಆ ಗವಾಕ್ಷಗಳನ್ನು ಮುಚ್ಚುತ್ತಾನೆ. ಮನಸ್ಸನ್ನು ಅಂತರ್ಮುಖ ಮಾಡುತ್ತಾನೆ. ಆಗ ಹಿಂದಿರುವ ಪರಮಾತ್ಮನ ಸಾಕ್ಷಾತ್ಕಾರವಾಗುತ್ತದೆ.

ಎರಡನೆಯ ಮಾರ್ಗವೇ ಜ್ಞಾನದ ಮೂಲಕ ಅವನನ್ನು ಅರಿತು, ಅವನಲ್ಲಿ ಒಂದಾಗುವುದು. ವಿಚಾರ ಎಲ್ಲವನ್ನೂ ವಿಭಜನೆಮಾಡುವುದು. ಮನಮೋಹಕವಾದ ನಾಮರೂಪಗಳ ಬಲೆಗೆ ಅದು ಬೀಳುವುದಿಲ್ಲ. ಎಷ್ಟೇ ಸುಂದರವಾಗಿರಲಿ, ಮನೋಹರವಾಗಿರಲಿ, ಇದು ಸತ್ಯವೆ ಎಂದು ಪ್ರಶ್ನಿಸು ವನು. ಸತ್ಯವಾಗಿದ್ದರೆ, ಅದು ಹಿಂದೆ, ಈಗ, ಮುಂದೆ ಎಂದೆಂದಿಗೂ ಇರಬೇಕು. ಹಾಗೆ ಇರುವು ದಾವುದು ಈ ಪ್ರಪಂಚದಲ್ಲಿ ಎಂದು ತರ್ಕಿಸುತ್ತಾನೆ. ಬಾಹ್ಯ ವಸ್ತುಗಳೆಲ್ಲ ನಮಗೆ ಕಾಣುವುದು ನಾಮರೂಪದ ಬಣ್ಣದ ಮೂಲಕ. ಈ ನಾಮರೂಪದ ಹಿಂದೆ ಹೋದರೆ ಏನಿದೆ ಎಂದು ಪ್ರಶ್ನಿಸು ವನು. ನಾವು ಯಾವುದನ್ನು ನೋಡುತ್ತೇವೆಯೋ ಅದು ದೇಶಕಾಲನಿಮಿತ್ತದ ತೆರೆಯಮೇಲೆ ಕಾಣುವ ಒಂದು ಘಟನೆ ಮಾತ್ರ. ಈ ದೇಶಕಾಲ ನಿಮಿತ್ತ ಇರುವುದೆಲ್ಲಿ? ನಮಗಿಂತ ಹೊರಗೆ ಏನಾದರೂ ಬಿದ್ದಿದೆಯೆ ಅಥವಾ ಪ್ರತಿಯೊಂದು ವಸ್ತುವನ್ನೂ ನೋಡುವಾಗ ಈ ದೇಶಕಾಲನಿಮಿತ್ತವನ್ನು ನಾವು ಆರೋಪಮಾಡಿ ಅದರಮೇಲೆ ಈ ಘಟನೆಯನ್ನು ಚಿತ್ರಿಸುತ್ತೇವೆಯೆ? ನಾವು ಆರೋಪಮಾಡಿರುವು ದನ್ನು ತೆಗೆದರೆ ಯಾವುದು ಅದರ ಆಧಾರದ ಮೇಲಿರುವುದೋ ಅದು ಮಂಗಮಾಯವಾಗುವುದು. ಇದೊಂದು ನಾವು ಕನಸುಕಾಣುವಂತೆ. ನಾವು ನಮ್ಮ ಚಿತ್ತವೃತ್ತಿಯಿಂದಲೇ ಕನಸನ್ನು ತಯಾರುಮಾಡಿ ಅಲ್ಲಿ ಹಲವು ವಸ್ತುಗಳನ್ನು ನೋಡುತ್ತೇವೆ. ನಾನು ಕನಸಿನಲ್ಲಿರುವವರೆಗೆ ಅದೆಲ್ಲ ನನ್ನಿಂದ ಹೊರಗೆ ಇರುವ ಸತ್ಯದಂತೆ ಕಾಣುತ್ತಿರುತ್ತವೆ. ಆದರೆ ಎದ್ದೊಡನೆ ಅವೆಲ್ಲಾ ನಮ್ಮ ಸೃಷ್ಚಿ ಎಂದು ಗೊತ್ತಾಗುವುದು. ದೇಶಕಾಲ ನಿಮಿತ್ತವೆಲ್ಲ ಆರೋಪ. ಅದರ ಹಿಂದೆ ಹೋದರೆ, ಪರಬ್ರಹ್ಮವೊಂದೇ ಸತ್ಯವಾಗಿರುವುದು. ದೇಶಕಾಲ ನಿಮಿತ್ತ ಎಂಬ ತ್ರಿಕೋಣದ ಗಾಜು \enginline{(Prism)}ಬಿಳಿ ಬೆಳಕನ್ನು ಹೇಗೆ ಕವಲೊಡೆಯುವಂತೆ ಮಾಡುವುದೋ ಹಾಗೆ ಅಖಂಡ ಪರಬ್ರಹ್ಮನನ್ನು, ಜೀವ, ಜಗತ್ತು, ಈಶ್ವರ ನಂತೆ ವಿಭಾಗಮಾಡುವುದು. ದೇಶಕಾಲನಿಮಿತ್ತವನ್ನು ತೆಗೆದುಹಾಕಿದರೆ ಇರುವುದೊಂದೆ ಅಖಂಡ ಪರಬ್ರಹ್ಮ. ಹಾಗೆಯೆ ನಾನು ಒಂದು ಸಾಂತ ವ್ಯಕ್ತಿ ಎಂದು ಭಾವಿಸಿರುವೆನು. ಈ ನಾನು ಎಂಬುದಕ್ಕೆ ಹಲವಾರು ಉಪಾಧಿಗಳ ವೇಷವನ್ನೆಲ್ಲಾ ಹಾಕಿ ಅದನ್ನು ಒಂದರಿಂದ ಮತ್ತೊಂದು ಬೇರೆ ಆಗಿರು ವಂತೆ, ದೇವರಿಂದ ಅದು ಬೇರೆಯಾಗಿರುವಂತೆ ಮಾಡಿರುವೆನು. ವಿಚಾರವನ್ನು ಉಪಯೋಗಿಸುತ್ತ ಬಂದರೆ, ಹಾಕಿದ ಉಪಾಧಿಯೆಲ್ಲ ಜಾರಿಹೋಗುವುದು. ಮಣ್ಣಿನಿಂದ ಮಾಡಿದ ಗಣಪತಿ ನೀರಿನಿಂದ ಅಭಿಷೇಕಮಾಡುವುದಕ್ಕೆ ಪ್ರಾರಂಭಿಸಿದರೆ ಸ್ವಲ್ಪಹೊತ್ತಿನಲ್ಲಿ ಕರಗಿ ಹೋಗುವುದು. ಅವನ ನಾಮ ರೂಪಗಳೆಲ್ಲ ಮಾಯವಾಗುವುದು. ಬರೀ ಜೇಡಿಮಣ್ಣು ಮಾತ್ರ ಉಳಿಯುವುದು. ಉಪಾಧಿಗಳನ್ನು ತೆಗೆದುಹಾಕಿದರೆ ಉಳಿಯುವುದು ಒಂದೇ ಅಖಂಡಬ್ರಹ್ಮ. ಆಕಾಶ ಅಖಂಡವಾಗಿ ಎಲ್ಲಾ ಕಡೆ ಯಲ್ಲಿಯೂ ವ್ಯಾಪಿಸಿದೆ. ನಾವು ಒಂದು ಸಣ್ಣ ಕುಡಿಕೆಮಾಡಿ ಅದರಲ್ಲಿ ಆಕಾಶವನ್ನು ಬೇರೆ ಮಾಡು ವೆವು. ಆ ಕುಡಿಕೆಗೆ ಆಕಾರ ಕೊಡುವೆವು, ಬಣ್ಣ ಬಳಿಯುವೆವು. ಒಂದು ಹೆಸರಿನಿಂದ ಕರೆಯುವೆವು. ಅಖಂಡ ಆಕಾಶದಿಂದ ಅದನ್ನು ಕೃತಕವಾಗಿ ಬೇರೆಮಾಡುವೆವು. ಆದರೆ ಆಕಾಶ ಎಂದಾದರೂ ಬೇರೆ ಆಗಿದೆಯೆ? ಇಲ್ಲ. ಅದು ಬೇರೆ ಆಗಿದೆ ಎಂದು ಭಾವಿಸಿದಾಗಲೂ ಒಂದೇ ಅಖಂಡ ಆಕಾಶವಾಗಿತ್ತು. ಅಥವಾ ಶ‍್ರೀರಾಮಕೃಷ್ಣರು ಕೊಡುತ್ತಿದ್ದ ಮತ್ತೊಂದು ಉಪಮಾನವನ್ನು ಉಪಯೋಗಿಸಬಹುದು. ಅದೇ ಉಪ್ಪಿನಗೊಂಬೆ ಸಮುದ್ರದ ಆಳವನ್ನು ನೋಡಲು ಹೋದಂತೆ ಎಂಬುದು. ಉಪ್ಪಿನಗೊಂಬೆ ಆಗಿರುವುದು ಸಮುದ್ರದ ನೀರಿನಿಂದ. ಅದು ಸಮುದ್ರಕ್ಕೆ ಇಳಿದರೆ ಸ್ವಲ್ಪದೂರ ಹೋಗುವುದರೊಳಗೆ ಎಲ್ಲಾ ಕರಗಿಹೋಗುವುದು. ಯಾವುದರಿಂದ ಆಗಿದೆಯೊ ಅದರಲ್ಲಿ ಕರಗಿಹೋಗುವುದು, ನಾನು ಎಂಬುದು ಬ್ರಹ್ಮನಲ್ಲಿ ಕರಗಿ ಒಂದಾಗಿಹೋಗುವುದು. ಇದು ವಿಚಾರದ ಮಾರ್ಗ. ವಿಚಾರ ಮಾಡುತ್ತ, ಮಾಡುತ್ತ ಮಧ್ಯದಲ್ಲಿ ಎಲ್ಲಿಯೂ ನಿಲ್ಲದೆಹೋದರೆ ಕೊನೆಗೆ ವಿಚಾರಮಾಡುವವನೂ ಕರಗಿಹೋಗುವನು. ಅವನು ಪರಬ್ರಹ್ಮನಲ್ಲಿ ಒಂದಾಗಿಬಿಡುವನು.

ಮೂರನೆಯ ಮಾರ್ಗವೇ ಕರ್ಮ, ಅದರ ಮೂಲಕವೂ ಪರಬ್ರಹ್ಮನ ಸಾಕ್ಷಾತ್ಕಾರವಾಗುವುದು. ಮುಂಚೆ ಅವನು ಜ್ಞಾನ ಪಡೆಯಬೇಕೆಂದು ಕರ್ಮಮಾಡುವುದಿಲ್ಲ. ಸುಮ್ಮನೆ ಫಲಾಪೇಕ್ಷೆಯಿಲ್ಲದೆ, ಇದು ಭಗವದರ್ಪಿತವಾಗಲಿ ಎಂದು ಯಜ್ಞದೃಷ್ಟಿಯಿಂದ ಕರ್ಮ ಮಾಡಿಕೊಂಡು ಹೋಗುವನು. ಅನಂತರ ಅವನ ಚಿತ್ತ ಇದರಿಂದ ಶುದ್ಧವಾಗುವುದು. ಯಾವಾಗ ಚಿತ್ತಶುದ್ಧವಾಗುವುದೋ, ಆಗ ಅವನು ವಸ್ತುವಿನ ಯಥಾರ್ಥ ಸ್ಥಿತಿಯನ್ನು ತಿಳಿಯುತ್ತಾನೆ. ಅವನು ತಿಳಿಯಬೇಕೆಂದು ಮನಸ್ಸುಮಾಡ ಬೇಕಾಗಿಲ್ಲ. ಅದು ತನಗೆ ತಾನೆ ಗೊತ್ತಾವುದು. ನಾನು ಕಣ್ಣು ಬಿಟ್ಟು ನೋಡಿದರೆ, ನನ್ನ ಕಣ್ಣೆದುರಿಗೆ ಇರುವುದು ಹೇಗೆ ಕಾಣುವುದೋ ಹಾಗೆ. ಜಗತ್ತು ಜೀವ ಇವುಗಳ ಸತ್ಯಸ್ಥಿತಿ ಇವನಿಗೆ ಅರಿವಾಗುತ್ತಾ ಬರುವುದು. ಇಲ್ಲಿ ನಾವು, ಮುಖ್ಯವಾಗಿ ಗಮನಿಸಬೇಕಾಗಿರುವುದು, ಕರ್ಮವನ್ನು. ಕರ್ಮ ಕರ್ಮಯೋಗವಾಗಬೇಕು. ಫಲಾಪೇಕ್ಷೆ ಇರಕೂಡದು. ಕರ್ಮಮಾಡುವುದಕ್ಕೆ ಮಾತ್ರ ನನಗೆ ಅಧಿಕಾರ; ಅದರಿಂದ ಬರುವ ಫಲಗಳಿಗಲ್ಲ ಎಂಬುದನ್ನು ಅವನು ಯಾವಾಗಲೂ ತಿಳಿದಿರಬೇಕು. ಅವನ ಬುದ್ಧಿ ಸಮತ್ವದಿಂದ ಕೂಡಿರಬೇಕು. ಹಿಡಿದ ಕೆಲಸ ಸಿದ್ಧಿಸಲಿ ಬಿಡಲಿ, ಜನ ಹೊಗಳಲಿ, ತೆಗಳಲಿ, ಸುಖ ಬರಲಿ, ದುಃಖ ಬರಲಿ, ಎಲ್ಲಾ ದ್ವಂದ್ವ ಅನುಭವಗಳಲ್ಲಿ ಅವನು ವಿಚಲಿತನಾಗದೆ ಇರುವನು. ಹಾಗಿದ್ದರೂ ಕೂಡ ಒಬ್ಬನಿಗೆ ಪರಮಾತ್ಮನ ಸಾಕ್ಷಾತ್ಕಾರವಾಗುವುದು. ಈ ಮಾರ್ಗದಲ್ಲಿ ಒಂದು ಮಾರ್ಗ ಮೇಲು, ಮತ್ತೊಂದು ಕೀಳು ಎಂದು ಹೇಳುವ ಗೋಜಿಗೆ ಶ‍್ರೀಕೃಷ್ಣ ಹೋಗುವುದಿಲ್ಲ. ಪ್ರತಿಯೊಬ್ಬನು ತನ್ನ ತನ್ನ ಸಂಸ್ಕಾರಕ್ಕೆ ತಕ್ಕಂತೆ ತನಗೆ ಸರಿದೋರುವ ಮಾರ್ಗವನ್ನು ಹಿಡಿಯಬೇಕಷ್ಟೆ. ಮತ್ತೊಬ್ಬರದನ್ನು ದೂರುವುದಕ್ಕೆ ಹೋಗಬೇಕಾಗಿಲ್ಲ. ನಮ್ಮ ಶಕ್ತಿಯನ್ನು ವ್ರಯಮಾಡದೆ ನಾವು ಹಿಡಿಯುವ ಮಾರ್ಗದಲ್ಲಿ ಮುಂದುವರಿದರೆ ಎಲ್ಲರೂ ಸಾಕ್ಷಾತ್ಕಾರವನ್ನು ಪಡೆಯುತ್ತಾರೆ.

\begin{verse}
ಅನ್ಯೇ ತ್ವೇವಮಜಾನಂತಃ ಶ್ರುತ್ವಾಽನ್ಯೇಭ್ಯ ಉಪಾಸತೇ \enginline{।\\}ತೇಽಪಿ ಚಾತಿತರಂತ್ಯೇವ ಮೃತ್ಯುಂ ಶ್ರುತಿಪರಾಯಣಾಃ \versenum{॥ ೨೫~॥}
\end{verse}

{\small ಆದರೆ ಇನ್ನು ಕೆಲವರು ಹೀಗೆ ತಿಳಿದುಕೊಳ್ಳದೆ ಇತರರಿಂದ ಕೇಳಿ ಅದನ್ನು ಅನುಷ್ಠಾನ ಮಾಡುತ್ತಾರೆ. ಹಾಗೆ ಕೇಳಿ ಮಾಡುವವರು ಸಹ ಮೃತ್ಯುವನ್ನು ನಿಜವಾಗಿ ದಾಟುವರು.}

ಇನ್ನು ಕೆಲವರಿಗೆ ತಾವೇ ಸ್ವತಂತ್ರವಾಗಿ ವಿಚಾರ ಮಾಡುವುದಕ್ಕೆ ಆಗುವುದಿಲ್ಲ. ಧ್ಯಾನಕ್ಕೆ ಮನಸ್ಸನ್ನು ಏಕಾಗ್ರ ಮಾಡುವುದಕ್ಕೂ ಆಗುವುದಿಲ್ಲ. ನಿಸ್ಸಂಗರಾಗಿ ಕರ್ಮವನ್ನು ಮಾಡಲೂ ಸಾಧ್ಯವಿಲ್ಲ. ಆದರೆ ಯಾರು ಯಾವುದಾದರೂ ಹಾದಿಯನ್ನು ಹಿಡಿದು ಸಾಧನೆ ಮಾಡಿ ಬೆಳಕನ್ನು ಕಂಡಿರುವರೋ ಅಂತಹವರು ಹೇಳುವ ಬುದ್ಧಿವಾದವನ್ನು ಶ್ರದ್ಧೆಯಿಂದ ಆಚರಿಸುವರು. ಅವರಿಗೆ ತಾವೆ ವಿಚಾರ ಮಾಡುವ ಶಕ್ತಿ ಇಲ್ಲದೆ ಇದ್ದರೂ ಇತರರು ವಿಚಾರ ಮಾಡಿ ತಿಳಿದುಕೊಂಡಿರುವುದನ್ನು ಕೇಳಿ ಅದನ್ನು ಜೀವನದಲ್ಲಿ ಅನುಷ್ಠಾನ ಮಾಡುವರು. ನಾವೊಂದು ರೋಗದಿಂದ ನರಳುತ್ತಿರುವೆವು. ನಮಗೆ ರೋಗ ಹೇಗೆ ಬಂತೊ ಗೊತ್ತಿಲ್ಲ. ಅದನ್ನು ಗುಣಮಾಡಿಕೊಳ್ಳಬೇಕಾದರೆ ಏನು ಮಾಡಬೇಕೊ ಗೊತ್ತಿಲ್ಲ. ಒಬ್ಬ ನುರಿತ ವೈದ್ಯನಿಗೆ ನಮ್ಮ ರೋಗವನ್ನು ಕುರಿತು ಹೇಳುತ್ತೇವೆ. ಅವನು ಅದಕ್ಕೆ ಔಷಧಿ ಬರೆದುಕೊಡುತ್ತಾನೆ. ನಾವು ಅದನ್ನು ಶ್ರದ್ಧೆಯಿಂದ ತೆಗೆದುಕೊಳ್ಳುತ್ತೇವೆ. ನಮಗೆ ರೋಗ ಹೇಗೆ ಬಂತೊ ಗೊತ್ತಿಲ್ಲ. ಈ ಔಷಧಿ ಹೇಗೆ ರೋಗವನ್ನು ಹೋಗಲಾಡಿಸುವುದೊ, ಗೊತ್ತಿಲ್ಲ. ಆದರೂ ನಾವು ರೋಗದಿಂದ ಪಾರಾಗುತ್ತೇವೆ. ಹಾಗೆಯೇ ಶ್ರದ್ಧೆ ಭಕ್ತಿಯಿಂದ ನಾವು ಇತರರು ಹೇಳಿದುದನ್ನು ನಂಬಿ ಹೊರಟರೂ ಗುರಿ ಸೇರುವುದರಲ್ಲಿ ಸಂದೇಹವಿಲ್ಲ.

ಜೀವನದಲ್ಲಿ ಇನ್ನೊಬ್ಬರಿಂದ ಕೇಳಿ ಅನುಷ್ಠಾನಕ್ಕೆ ತರುವವರನ್ನೇ ಶ‍್ರೀಕೃಷ್ಣ ಶ್ರುತಿಪಾರಾಯಣ ಎನ್ನುವುದು. ನಾವು ಭಗವಂತನಲ್ಲಿಟ್ಟಿರುವ ಶ್ರದ್ಧೆಯೇ ಸಾಕು ನಮ್ಮನ್ನು ಉದ್ಧರಿಸುವುದಕ್ಕೆ. ಇಂತಹವರು ಕೂಡ ಮೃತ್ಯು ಸಂಸಾರ ಸಾಗರವನ್ನು ಖಂಡಿತ ದಾಟುತ್ತಾರೆ ಎಂದು ಭರವಸೆ ಕೊಡುವನು. ಶಾಸ್ತ್ರಾದಿಗಳನ್ನು ತಿಳಿದವನಿಗೆ ಧ್ಯಾನಿಗೆ ಮತ್ತು ಕರ್ಮಿಗೆ ಮಾತ್ರ ಅಲ್ಲ ಭಗವಂತ ಮೀಸಲಾಗಿರುವುದು. ಏನೂ ಗೊತ್ತಿಲ್ಲದೆ ಇದ್ದರೂ ಶ್ರದ್ಧೆಯ ಸೋರೆಬುರುಡೆ ಒಂದಿದ್ದರೆ ಸಾಕು ಸಂಸಾರ ಸಾಗರವನ್ನು ಈಜಿಕೊಂಡು ದಾಟುವುದಕ್ಕೆ ಎಂದು ಹೇಳುತ್ತಾನೆ. ಇದರಲ್ಲಿ ಎಳ್ಳಷ್ಟೂ ಸಂದೇಹವಿಲ್ಲ ಎನ್ನುವನು. ಒಂದು ವೇಳೆ ಅವನ ಕೈಯಲ್ಲಿ ಸಾಧ್ಯವಾಗದೆ ಇದ್ದರೆ ಎಂಬ ಅನುಮಾನ ನಮ್ಮಲ್ಲಿ ಬರುವುದು. ಜೀವನದಲ್ಲಿ, ಜ್ಞಾನವೊ, ಧ್ಯಾನವೊ, ಕರ್ಮವೊ ಎಲ್ಲವೂ ದೇವರೆಡೆಗೆ ಒಯ್ಯುವುದು. ಆದರೆ ಕೆಲವು ವೇಳೆ ಆ ಮಾರ್ಗದಲ್ಲಿ ಅಡ್ಡಹಾದಿಗಳೂ ಇವೆ. ನಾವು ಗುರಿಯನ್ನು ಮರೆತು ಎಲ್ಲೆಲ್ಲೊ ಹೋಗಬಹುದು. ಅದು ಎಲ್ಲಾ ಕಡೆಯೂ ಆಗುವುದು. ಬರೀ ನಂಬಿ ಹೊರಟವನಿಗೆ ಮಾತ್ರವಲ್ಲ. ಆದರೆ ಗುರಿ ಸೇರಬೇಕು ಎಂದು ಹೊರಟ ಮನುಷ್ಯ ತಪ್ಪು ದಾರಿ ಹಿಡಿದರೂ ಸ್ವಲ್ಪ ಹೊತ್ತಾದ ಮೇಲೆ, ದಾರಿಯನ್ನು, ವಿಚಾರಿಸುತ್ತಿದ್ದರೆ, ಯಾರಾದರೂ ಅವನನ್ನು ಸರಿಯಾದ ದಾರಿಗೆ ಬಿಟ್ಟೇಬಿಡುವರು. ನಾವು ಎಂದೆಂದಿಗೂ ತಪ್ಪುವುದಿಲ್ಲ. ಉದ್ದೇಶ ಸರಿಯಾಗಿ ದ್ದರೆ ಕೆಲವು ವೇಳೆ ಎಡವಬಹುದು, ಸಂದುಗೊಂದುಗಳಲ್ಲಿ ನುಗ್ಗಬಹುದು, ಆದರೆ ಕೊನೆಗೆ ಗುರಿ ಸೇರುವುದರಲ್ಲಿ ಸಂದೇಹವಿಲ್ಲ.

\begin{verse}
ಯಾವತ್ ಸಂಜಾಯತೇ ಕಿಂಚಿತ್ ಸತ್ತ್ವಂ ಸ್ಥಾವರಜಂಗಮಮ್~।\\ಕ್ಷೇತ್ರಕ್ಷೇತ್ರಜ್ಞ ಸಂಯೋಗಾತ್ ತದ್ವಿದ್ಧಿ ಭರತರ್ಷಭ \versenum{॥ ೨೬~॥}
\end{verse}

{\small ಅರ್ಜುನ, ಚಲಿಸುವ ಮತ್ತು ಚಲಿಸದಿರುವ ಯಾವುದಾದರೂ ಒಂದು ಪ್ರಾಣಿ ಉತ್ಪನ್ನವಾಗುವುದೊ ಅದು ಕ್ಷೇತ್ರಕ್ಷೆತ್ರಜ್ಞರ ಸಂಯೋಗದಿಂದ ಆಗಿದೆ ಎಂದು ತಿಳಿ.}

ಈ ಪ್ರಪಂಚದಲ್ಲಿ ಕ್ಷೇತ್ರ ಮತ್ತು ಕ್ಷೇತ್ರಜ್ಞನ ಸಂಘಾತವಿಲ್ಲದೆ, ಏನೂ ಜನಿಸುವಂತೆ ಇಲ್ಲ. ಎಲ್ಲಾ ಕಡೆಯಲ್ಲಿಯೂ ಕ್ಷೇತ್ರಜ್ಞ ಇರುವನು. ಆದರೆ ಅವನಿಗೆ ಅದು ಅರಿವಿಲ್ಲ. ಮಾನವರಲ್ಲಿ ಕೂಡ ಎಷ್ಟು ಜನಕ್ಕೆ ಇದರ ಅರಿವಿರುವುದು. ನಾವೊಂದು ದೇಹವನ್ನು ಹೊತ್ತು ನಡೆಯುತ್ತಿರುವೆವು. ಅದರಲ್ಲಿ ಏನೇನಿದೆ, ಅದು ಏನು ಕೆಲಸ ಮಾಡುತ್ತಿದೆ ಎಂಬುದು ಎಷ್ಟು ಜನಕ್ಕೆ ಗೊತ್ತು? ಅದರಂತೆಯೇ ಜೀವ ಹಲವು ಸ್ಥಿತಿಗಳಲ್ಲಿದೆ. ತರುಲತೆ, ಪಶುಪಕ್ಷಿ, ಪ್ರಾಣಿ, ಕ್ರಿಮಿಕೀಟ ಎಲ್ಲಾ ಕಡೆಯೂ ಹಲವು ದೇಹಗಳ ಮೂಲಕ ಜೀವ ಕೆಲಸ ಮಾಡುತ್ತಿದೆ. ಎಲ್ಲಿ ಮತ್ತೊಂದು ಉತ್ಪತ್ತಿಯಾಗ ಬೇಕೊ ಅಲ್ಲಿ ಕ್ಷೇತ್ರಜ್ಞ ಕ್ಷೇತ್ರದಿಂದ ಸಾಮಾನನ್ನು ತೆಗೆದುಕೊಂಡು ತನ್ನ ಕರ್ಮ ಸಮೆಸಲು ಗೂಡನ್ನು ಕಟ್ಟಿಕೊಳ್ಳಬೇಕು. ಚೇತನ ಜಡದ ಸಹಾಯದಿಂದ ದೇಹ ಮುಂತಾದವುಗಳನ್ನು ಸೃಷ್ಟಿಸಿಕೊಳ್ಳುವುದು.

ಚೇತನ ಸರ್ವಾಂತರ್ಯಾಮಿಯಾಗಿದೆ. ಆದರೆ ಎಲ್ಲಾ ಕಡೆಯೂ ಒಂದೇ ರೂಪವನ್ನು ಧರಿಸಬೇಕಾ ಗಿಲ್ಲ. ವಾತಾವರಣಕ್ಕೆ ತಕ್ಕಂತೆ ಅದು ರೂಪವನ್ನು ಧರಿಸುವುದು. ವಾತಾವರಣ ಬದಲಾಯಿಸಿದರೆ ಚೇತನ ಇರುವುದಕ್ಕೆ ಆಗುವುದಿಲ್ಲ ಎಂದಲ್ಲ. ಮೀನು ನೀರಿದ್ದರೆ ಮಾತ್ರ ಯಾರಾದರೂ ಬಾಳಲು ಸಾಧ್ಯ ಎಂದು ಭಾವಿಸುವುದು. ಆದರೆ ನೆಲದ ಮೇಲೆ ಗಾಳಿಯಲ್ಲಿ ಜೀವಜಂತುಗಳಿಲ್ಲವೆ? ಹಾಗೆಯೇ ಮನುಷ್ಯನೂ ಕೂಡ, ತಾನು ಯಾವ ವಾತಾವರಣದಲ್ಲಿರುವನೊ ಅಂತಹ ವಾತಾವರಣ ಇದ್ದರೆ ಮಾತ್ರ ಜೀವ ಜಂತುಗಳು ಇರಲು ಸಾಧ್ಯ ಎಂದು ತಪ್ಪು ಭಾವಿಸುವನು. ಬೇರೆ ಬೇರೆ ಗ್ರಹಗಳಲ್ಲಿ ನಮ್ಮಂತೆ ವಾತಾವರಣ ಇಲ್ಲದೇ ಇದ್ದರೆ ಜೀವ ಇಲ್ಲವೇ ಇಲ್ಲ ಎಂದಲ್ಲ. ಅಲ್ಲಿ ಬೇರೆ ಬೇರೆ ರೂಪದಲ್ಲಿದೆ. ಒಂದು ಹಿಡಿ ಮಣ್ಣಿನಲ್ಲಿ, ಒಂದು ಉದ್ಧರಣೆ ನೀರಿನಲ್ಲಿ, ಒಂದು ಅಂಗುಲ ಆಕಾಶದಲ್ಲಿ ಎಷ್ಟೊಂದು ಜೀವ ಜಂತುಗಳು ನಮಗೆ ಸೂಕ್ಷ್ಮದರ್ಶಕ ಯಂತ್ರದಿಂದ ನೋಡಿದಾಗ ಕಾಣುವುದು. ಇವುಗಳೆಲ್ಲ ಆಗಿರುವುದು ಚೇತನ ಮತ್ತು ಜಡದ ಸಂಘಾತದಿಂದ.

\begin{verse}
ಸಮಂ ಸರ್ವೇಷು ಭೂತೇಷು ತಿಷ್ಠಂತಂ ಪರಮೇಶ್ವರಮ್~।\\ವಿನಶ್ಯತ್ಸ್ವವಿನಶ್ಯಂತಂ ಯಃ ಪಶ್ಯತಿ ಸ ಪಶ್ಯತಿ \versenum{॥ ೨೭~॥}
\end{verse}

{\small ಸಮಸ್ತ ಪ್ರಾಣಿಗಳಲ್ಲಿ ಸಮನಾಗಿರುವವನೂ, ನಶ್ವರ ವಸ್ತುವಿನಲ್ಲಿ ಅವಿನಾಶಿಯಾಗಿರುವವನೂ ಆದ ಪರಮೇಶ್ವರನನ್ನು ಯಾರು ನೋಡುತ್ತಾರೆಯೋ ಅವರೇ ನೋಡುತ್ತಾರೆ.}

ಪ್ರಕೃತಿ ಮತ್ತು ಪುರುಷರ ವಿಷಯವನ್ನು ಹೇಳಿ ಆಯಿತು. ಇನ್ನು ಮೇಲೆ ಪರಮೇಶ್ವರನ ವಿಷಯವನ್ನು ಹೇಳುತ್ತಾನೆ. ಅವನು ಸಮಸ್ತ ಪ್ರಾಣಿಗಳಲ್ಲಿ ಒಂದೇ ಸಮನಾಗಿದ್ದಾನೆ. ಒಳ್ಳೆಯವನು, ಕೆಟ್ಟವನು, ಜಡ, ಚೇತನ ಎಲ್ಲದರ ಹಿಂದೆಯೂ ಒಂದೇ ಸಮನಾಗಿರುವನು. ಆದರೆ ಸಾಧಾರಣ ಮನುಷ್ಯರಿಗೆ ಒಂದೇ ಸಮನಾಗಿ ಕಾಣುವುದಿಲ್ಲ. ಸೂರ್ಯ ಎಲ್ಲಾ ಕಡೆಯೂ ಬೆಳಗುತ್ತಿರುವನು. ಆದರೆ ಎಲ್ಲಾ ವಸ್ತುಗಳೂ ಒಂದೇ ಸಮನಾಗಿ ಅವನನ್ನು ಪ್ರತಿಬಿಂಬಿಸುತ್ತಿಲ್ಲ. ಕನ್ನಡಿ ಚೆನ್ನಾಗಿ ಪ್ರತಿಬಿಂಬಿಸುವುದು. ನೀರು ಒಂದು ಸ್ವಲ್ಪ ಕಡಿಮೆ ಪ್ರತಿಬಿಂಬಿಸುವುದು. ಹಾಗೆಯೆ ಬಟ್ಟೆ, ನೆಲ ಮುಂತಾದುವು. ಅದರ ಪ್ರತಿಬಿಂಬದಲ್ಲಿ ತಾರತಮ್ಯ ನೋಡುತ್ತೇವೆ. ಹಾಗೆಯೆ ಭಗವಂತ ಎಲ್ಲಾ ಕಡೆಯಲ್ಲಿಯೂ ಇರುವನು. ಯಾರು ಭಗವದ್ಭಕ್ತರೊ, ಮಹಾಜ್ಞಾನಿಗಳೋ ಅವರ ಮೂಲಕ ಸ್ವಲ್ಪ ಚೆನ್ನಾಗಿ ಕಾಣುವನು. ಅವರು ಭಗವಂತನನ್ನು ನೋಡುವುದಕ್ಕೆ ಇರುವ ರಂಧ್ರಗಳಂತೆ. ಮಿಕ್ಕಿರು ವವರೆಲ್ಲ ಆ ರಂಧ್ರವನ್ನು ಮುಚ್ಚಿಕೊಂಡಿರುವರು. ಅಂತೂ ದೇವರಿಗೆ ಪಕ್ಷಪಾತವಿಲ್ಲ. ಎಲ್ಲಾ ಕಡೆಯೂ ಇರುವನು. ಆದರೆ ನಮಗೆ ಅವನು ಎಲ್ಲಾ ಕಡೆಯಲ್ಲಿಯೂ ಒಂದೇ ಸಮನಾಗಿ ಕಾಣುವು ದಿಲ್ಲ.

ಅವನು ಹೇಗೆ ಇದ್ದಾನೆ ಎಂಬುದನ್ನು ವಿವರಿಸುತ್ತಾನೆ. ನಶ್ವರವಾದ ವಸ್ತುವಿನಲ್ಲಿ ನಶ್ವರವಾಗದೆ ಇದ್ದಾನೆ. ಮನುಷ್ಯಜೀವನ ಬದಲಾವಣೆಯಾಗುತ್ತಿದೆ. ಬಾಲ್ಯದಿಂದ ವೃದ್ಧಾಪ್ಯದವರೆಗೆ ಹಲವು ಅವಸ್ಥೆಗಳ ಮೂಲಕ ಹೋಗುತ್ತಿದೆ. ದೇಹ ಬದಲಾಯಿಸುತ್ತಿದೆ. ಇಂದ್ರಿಯ ಬದಲಾಯಿಸುತ್ತಿದೆ. ಮನಸ್ಸು ಬುದ್ಧಿ ಬದಲಾಯಿಸುತ್ತಿವೆ. ಆದರೆ ಇವುಗಳ ಹಿಂದೆಲ್ಲಾ, ಈ ವಿಕಾರಗಳಾವುದಕ್ಕೂ ಸಿಲುಕದೆ ಇದೆ ಪರಮಾತ್ಮವಸ್ತು. ಇವುಗಳು ನಾಶವಾಗಬಹುದು, ಆದರೂ ಅದೇನೂ ನಾಶವಾಗುವುದಿಲ್ಲ. ಗಡಿಗೆಯೊಡೆಯಬಹುದು. ಆದರೆ ಅದರೊಳಗೆ ಇರುವ ಆಕಾಶವನ್ನು ಧ್ವಂಸಮಾಡಲಾಗುವುದೆ? ರೇಡಿಯೋ ಒಡೆದು ಹಾಕಬಹುದು. ಆದರೆ ರೇಡಿಯೋ ಮೂಲಕ ವ್ಯಕ್ತವಾಗುವ ಗಾನಸ್ಪಂದನವನ್ನು ನಾವು ನಾಶಮಾಡುವುದಕ್ಕೆ ಆಗುವುದೆ? ಪರಮಾತ್ಮ ನಾಶದ ಮಧ್ಯದಲ್ಲಿ ನಾಶವಾಗದೆ ಇರುವನು.

ಯಾರು ಅವನನ್ನು ನೋಡುತ್ತಾನೆಯೋ ಅವನೇ ಸರಿಯಾಗಿ ನೋಡುತ್ತಾನೆ. ಅವನೇ ಜ್ಞಾನಿ. ಇತರರೆಲ್ಲ ಅದರ ಮೇಲೆ ಅದನ್ನು ಕಾಣದಂತೆ ಹಾಕಿರುವ ಉಪಾಧಿಗಳನ್ನು ಮಾತ್ರ ನೋಡುತ್ತಿರು ವರು. ಜ್ಞಾನಿ ಎಲ್ಲಾ ನಾಮರೂಪಗಳ ಹಿಂದೆ ಇರುವ ಪರಮಾತ್ಮನನ್ನು ನೋಡುತ್ತಾನೆ. ಈ ಪ್ರಪಂಚದಲ್ಲಿ ಎಲ್ಲಾ ವಸ್ತುಗಳ ಹಿಂದೆ ಇರುವ ಏಕ ಮಾತ್ರ ಸತ್ಯವಾದ ವಸ್ತುವೇ ಪರಮಾತ್ಮ.

\begin{verse}
ಸಮಂ ಪಶ್ಯನ್ ಹಿ ಸರ್ವತ್ರ ಸಮವಸ್ಥಿತಮೀಶ್ವರಮ್~।\\ನ ಹಿನಸ್ತ್ಯಾತ್ಮನಾತ್ಮಾನಂ ತತೋ ಯಾತಿ ಪರಾಂ ಗತಿಮ್ \versenum{॥ ೨೮~॥}
\end{verse}

{\small ಏಕೆಂದರೆ ಎಲ್ಲಾ ಕಡೆಯಲ್ಲಿಯೂ ಸಮನಾಗಿರುವ ಈಶ್ವರನನ್ನು ಸಮನಾಗಿ ನೋಡುತ್ತಿರುವವನು ತನ್ನಿಂದ ತನ್ನನ್ನು ನಾಶಮಾಡಿಕೊಳ್ಳುವುದಿಲ್ಲ. ಅದರಿಂದ ಅವನು ಪರಮಗತಿಯನ್ನು ಪಡೆಯುತ್ತಾನೆ.}

ಎಲ್ಲಾ ಕಡೆಯೂ ಪರಮೇಶ್ವರನು ಸಮನಾಗಿದ್ದಾನೆ. ಅವನು ನಮ್ಮ ಕಣ್ಣಿಗೆ ಕೆಲವು ಕಡೆ ಹೆಚ್ಚಾಗಿ ಇರುವಂತೆ ಕಾಣಬಹುದು, ಕೆಲವು ಕಡೆ ಕಡಿಮೆ ಇರುವಂತೆ ಕಾಣಬಹುದು. ಆದರೆ ಅವನೇನೋ ಸಮನಾಗಿರುವನು. ಎಲ್ಲಾ ನಾಮರೂಪಗಳ ಹಿಂದೆಯೂ ಇರುವವನು ಅವನೇ. ಜಡಚೇತನದ ಹಿಂದೆಯೂ ಇರುವವನು ಅವನೇ. ಪಾಪಿ ಪುಣ್ಯವಂತ, ಬುದ್ಧಿವಂತ ದಡ್ಡ ಎಲ್ಲರ ಹಿಂದೆಯೂ ಅವನೇ ಇರುವನು. ಒಂದು ಸಣ್ಣ ತುಳಸೀಗಿಡ ಮತ್ತೊಂದು ದೊಡ್ಡ ತುಳಸೀಗಿಡದಂತೆ. ಒಂದರಷ್ಟೇ ಪವಿತ್ರ ಮತ್ತೊಂದು.

ಆದರೆ ಸಮನಾಗಿರುವುದೊಂದು, ಸಮನಾಗಿ ನೋಡುವುದು ಮತ್ತೊಂದು. ತಾತ್ತ್ವಿಕದೃಷ್ಟಿ ಯಿಂದ ಪ್ರಪಂಚ ಅವನಿಂದ ಬಂದಿರುವುದರಿಂದ ಅವನೇನೊ ಸಮನಾಗಿರಲೇ ಬೇಕಾಗಿದೆ. ಸಕ್ಕರೆಯಿಂದ ಗೊಂಬೆಗಳನ್ನು ಮಾಡಿ ಒಂದೊಂದಕ್ಕೂ ಒಂದೊಂದು ಹೆಸರು ಮತ್ತು ಆಕಾರಗಳನ್ನು ಕೊಟ್ಟರೂ, ಅವುಗಳ ಹಿಂದೆಲ್ಲಾ ಸಕ್ಕರೆ ಒಂದೆ ಸಮನಾಗಿ ವ್ಯಾಪಿಸಿದೆ. ಒಂದು ಕಡೆ ಹೆಚ್ಚಾಗಿ ಸಿಹಿ ಇಲ್ಲ, ಮತ್ತೊಂದು ಕಡೆ ಕಡಿಮೆ ಸಿಹಿ ಇಲ್ಲ. ಎಲ್ಲಿಯವರೆಗೆ ನಾಮರೂಪಗಳಿಗೆ ನಾವು ಬೆಲೆಯನ್ನು ಕೊಡುತ್ತೇವೆಯೋ ಅಲ್ಲಿಯವರೆಗೆ ಅದನ್ನು ಮೀರಿಹೋಗುವುದಕ್ಕೆ ಆಗುವುದಿಲ್ಲ. ಯಾವಾಗ ನಾಮ ರೂಪವನ್ನು ಮೀರಿ ಹೋಗುತ್ತೇವೆಯೋ ಆಗ ಹೇಗೆ ಸಕ್ಕರೆಯೊಂದೇ ಎಲ್ಲಾ ಗೊಂಬೆಗಳ ಹಿಂದೆಯೂ ಇರುವುದೋ ಹಾಗೆಯೇ ಪರಮಾತ್ಮನೇ ಎಲ್ಲದರ ಹಿಂದೆ ಇರುವನು. ಅದನ್ನು ಸಮನಾಗಿ ನೋಡಬೇಕಾದರೆ ಒಬ್ಬ ತನ್ನ ದೇಹ, ಮನಸ್ಸು, ಬುದ್ಧಿ ಇಂದ್ರಿಯದ ಬಿಲದಿಂದ ಹೊರಗೆ ಬಂದಿರಬೇಕು. ಅವನಲ್ಲಿ ಯಾವ ವಾಸನೆಗಳೂ ಇರಕೂಡದು. ನಾವು ಉಪಾಧಿಯ ಅಂಜನವನ್ನು ಹಾಕಿಕೊಂಡು ನೋಡುವುದರಿಂದ, ಅದು ಪರಮಾತ್ಮನನ್ನು ಮರೆಸುವುದು. ಅದರ ಮೇಲೆ ಆರೋಪಿತ ವಾಗಿರುವ ನಾಮರೂಪವನ್ನೆ ತೋರುವುದು. ನಾವು ಯಾವಾಗ ಉಪಾಧಿಯ ಅಂಜನವನ್ನು ತೆಗೆಯು ತ್ತೇವೆಯೋ ಆಗ ವಸ್ತುವಿನ ನೈಜಸ್ಥಿತಿಯಾದ ಪರಮಾತ್ಮ ಅಲ್ಲಿ ಕಾಣುತ್ತಾನೆ. ಇಲ್ಲಿ ಆಗುವ ಬದಲಾವಣೆಯೆಲ್ಲ ನಾನು ನೋಡುವ ದೃಷ್ಟಿಯಲ್ಲಿ. ಹೊರಗಿನದು ಹಾಗೆಯೇ ಇರುವುದು. ಅಲ್ಲಿ ವ್ಯತ್ಯಾಸವೇನೂ ಆಗುವುದಿಲ್ಲ. ಆದರೆ ನಾವು ನೋಡುವ ದೃಷ್ಟಿಯಲ್ಲಿ ಕಾಣುವುದು. ರೇಡಿಯೋ ಕೇಳುತ್ತಿರುವಾಗ ಸ್ವಲ್ಪ ಅತ್ತ ಇತ್ತ ತಿರುಗಿಸಿದರೂ ಸಾಕು ನಮಗೆ ಬೇರೆ ಊರಿನ ಸಂಗೀತ ಬರುವುದು. ಒಂದು ವ್ಯಕ್ತಿಯನ್ನು ನೋಡಿದಾಗ, ಅವನನ್ನು ದೇಹ ಮತ್ತು ಇಂದ್ರಿಯದೃಷ್ಟಿಯಿಂದ ಭಾವಿಸಿದರೆ, ಅದಕ್ಕೆ ಸಂಬಂಧಪಟ್ಟ ಭಾವನೆಗಳೇ ನಮಗೆ ಬರುವುವು. ಯಾವಾಗ ಪರಮಾತ್ಮದೃಷ್ಟಿಯಿಂದ ನೋಡುತ್ತೇವೆಯೋ ಆಗ ನಮ್ಮ ಮನಸ್ಸಿನಲ್ಲಿ ಏಳುವ ಭಾವನೆಗಳೇ ಬೇರೆಯಾಗುವುವು.

ಆಗ ಅವನು ತನ್ನಿಂದ ನಾಶಮಾಡಿಕೊಳ್ಳುವುದಿಲ್ಲ. ಒಂದು ವಸ್ತುವನ್ನು ಪರಮಾತ್ಮನ ದೃಷ್ಟಿ ಯಿಂದ ನೋಡುವುದು, ಅಥವಾ ಅಲ್ಪವಾದ ಯಾವುದೋ ಗಂಡಸೊ ಹೆಂಗಸೊ ಎಂದು ನೋಡು ವುದು ನಮ್ಮ ಕೈಯಲ್ಲಿದೆ! ಸಾಧಾರಣ ಮನುಷ್ಯ ಅದನ್ನು ಯಾವುದೋ ಗಂಡಸು, ಹೆಂಗಸು ಅದಕ್ಕೆ ಆ ಗುಣಗಳಿವೆ ಎಂದು ಭಾವಿಸುತ್ತಾನೆ. ಆಗ ಕೆಲವು ವೇಳೆ ಅದಕ್ಕೆ ಆಕರ್ಷಿತನಾಗುತ್ತಾನೆ. ಮತ್ತು ಕೆಲವು ವೇಳೆ ನಮಗೆ ಅದನ್ನು ಕಂಡರೆ ಆಗುವುದಿಲ್ಲ. ಪ್ರೀತಿ ಮತ್ತು ದ್ವೇಷ ಎರಡರಿಂದಲೂ ವಸ್ತುವಿಗೆ ದಾಸರಾಗುತ್ತೇವೆ. ಅದು ಹೇಳಿದಂತೆ ಕೇಳುತ್ತೇವೆ. ನಾವು ಪಟ್ಟ ಆನಂದಕ್ಕೆ ಹತ್ತರಷ್ಟು ದುಃಖವನ್ನು ಅನುಭವಿಸಬೇಕಾಗುವುದು. ಅದರಂತೆಯೇ ದ್ವೇಷ. ಯಾವಾಗ ಅದನ್ನು ಕಂಡರೆ ಆಗುವುದಿಲ್ಲವೋ ಅದಕ್ಕೆ ಸಂಬಂಧಪಟ್ಟ ಹೀನ ಆಲೋಚನೆಗಳೆಲ್ಲ ನನ್ನಲ್ಲಿ ಏಳುತ್ತಿರುವುದು. ಯಾವನೊ ಒಬ್ಬನು ಕಳ್ಳನಾಗಿರಬಹುದು. ಆದರೆ ಅವನ ಕೆಟ್ಟತನವನ್ನು ನಾನು ಆಲೋಚಿಸುತ್ತಿದ್ದರೆ ನನ್ನ ಮನಸ್ಸೂ ಗಲೀಜಾಗುವುದು. ಹೊರಗಿನ ಕಸ ನನ್ನ ಮನೆ ಬಳಿಗೆ ಬರುವಂತೆ ಮಾಡಿಕೊಳ್ಳುವೆನು.

ಅದಕ್ಕೆ ಜ್ಞಾನಿಯಾದವನು ಪ್ರತಿಯೊಂದು ವಸ್ತುವಿನ ಹಿಂದೆಯೂ ಪರಮಾತ್ಮನ ಭಾವನೆ ತಂದು ಅದನ್ನು ಆ ದೃಷ್ಟಿಯಿಂದ ನೋಡುವನು. ವ್ಯವಹಾರದಲ್ಲಿರುವಾಗ ಅವನು ಯಾವ ಉಪಾಧಿಯನ್ನು ತೊಟ್ಟಿರುವನೊ ಆ ದೃಷ್ಟಿಯಿಂದ ವ್ಯವಹಾರ ನಡೆಸುತ್ತಿದ್ದರೂ ಅವನಿಗೆ ತನ್ನ ಮನಸ್ಸಿನೊಳಗೆ ಚೆನ್ನಾಗಿ ಗೊತ್ತಿದೆ ಯಾವುದು ಸತ್ಯ ಎಂದು. ಅವನು ತನ್ನನ್ನು ನಾಶಮಾಡಿಕೊಳ್ಳುವುದಿಲ್ಲ. ಅಂದರೆ ವ್ಯಕ್ತಿಯನ್ನು ವ್ಯಕ್ತಿಯಂತೆ ನೋಡಿ ಬಂಧನಕ್ಕೆ ಬೀಳುವುದಿಲ್ಲ. ಆಸಕ್ತಿ ಮತ್ತು ದ್ವೇಷಗಳಿಂದ ಪೀಡಿತನಾಗುವುದಿಲ್ಲ. ಎಲ್ಲರ ಹಿಂದೆಯೂ ಪರಮಾತ್ಮನನ್ನು ನೋಡುತ್ತಾನೆ. ಹೇಗೆ ಹಲವು ಹಿಮಮಣಿಗಳು ಒಂದೇ ಸೂರ್ಯನನ್ನು ಪ್ರತಿಬಿಂಬಿಸುವುವೊ ಹಾಗೆಯೇ ಎಲ್ಲ ಜೀವರಾಶಿಗಳೂ ಪರಮಾತ್ಮನನ್ನು ತಮ್ಮ ಯೋಗ್ಯತಾನುಸಾರ ಪ್ರತಿಬಿಂಬಿಸುತ್ತಿವೆ. ಆದರೆ ಜ್ಞಾನಿಯಾದರೋ ಎಲ್ಲ ಕಡೆಯೂ ಒಂದೇ ಸಮನಾಗಿ ಅವನು ಇರುವುದನ್ನೆ ನೋಡುವನು.

ಅವನು ಪರಮಗತಿಯನ್ನು ಹೊಂದುತ್ತಾನೆ. ಇರುವಾಗಲೇ ಎಲ್ಲಾಕಡೆಯೂ ಪರಮಾತ್ಮನೇ ಎಲ್ಲಾ ಉಪಾಧಿಗಳ ಹಿಂದೆಯೂ ಬೆಳಗುತ್ತಿರುವುದನ್ನು ನೋಡಿದಾಗ ಅವನು ಯಾವ ಬಂಧನಕ್ಕೂ ಒಳಗಾಗುವುದಿಲ್ಲ. ಅವನಲ್ಲಿ ಯಾವ ಸಂಶಯವೂ ಇಲ್ಲ, ಸಂಸ್ಕಾರವೂ ಇಲ್ಲ, ವಾಸನೆಯೂ ಇಲ್ಲ. ಅವನು ಇರುವಾಗಲೇ ಮುಕ್ತ; ಹುರಿದ ಬೀಜದಂತೆ ಇರುವನು. ಇನ್ನದು ಮೊಳೆಯಲಾರದು.

\begin{verse}
ಪ್ರಕೃತ್ಯೈವ ಚ ಕರ್ಮಾಣಿ ಕ್ರಿಯಮಾಣಾನಿ ಸರ್ವಶಃ~।\\ಯಃ ಪಶ್ಯತಿ ತಥಾತ್ಮಾನಮಕರ್ತಾರಂ ಸ ಪಶ್ಯತಿ \versenum{॥ ೨೯~॥}
\end{verse}

{\small ಸರ್ವ ಪ್ರಕಾರದಿಂದಲೂ ಪ್ರಕೃತಿಯೇ ಕರ್ಮಗಳನ್ನೆಲ್ಲ ಮಾಡುವುದು ಮತ್ತು ತಾನು ಅಕರ್ತೃ ಎಂದು ಯಾವನು ನೋಡುತ್ತಾನೆಯೊ ಅವನೇ ನೋಡುತ್ತಾನೆ.}

ಹೇಗೆ ಜ್ಞಾನಿ ಬಾಹ್ಯ ವಸ್ತುವಿನಮೇಲೆ ಆರೋಪಿತವಾಗಿರುವ ನಾಮರೂಪಗಳನ್ನು ತೂರಿಹೋಗಿ ಹಿಂದಿರುವ ಪರಮಾತ್ಮನನ್ನು ನೋಡುತ್ತಾನೆಯೋ, ಹಾಗೆಯೆ ಅವನು ತನ್ನನ್ನು ವಿಭಜನೆ ಮಾಡಿ ಕೊಂಡಿರುವನು. ಅನೇಕವೇಳೆ ಅಜ್ಞಾನದಲ್ಲಿ ಅನಾತ್ಮದೊಂದಿಗೆ ಬೆರತುಹೋಗಿ ತಾನೆ ಮಾಡುತ್ತಿರು ವುದು, ಅನುಭವಿಸುತ್ತಿರುವುದು ಎಂದು ಭ್ರಮಿಸುತ್ತಾನೆ. ಯಾವಾಗ ತನ್ನನ್ನೇ ವಿಭಜನೆ ಮಾಡಿಕೊಳ್ಳು ತ್ತಾನೊ ಆಗ ಅವನಿಗೆ ಗೊತ್ತಾಗುವುದು, ತಾನು ಕೇವಲ ಸಾಕ್ಷಿ, ಇಲ್ಲಿ ಕೆಲಸಗಳೆನ್ನೆಲ್ಲ ಮಾಡುತ್ತಿರು ವುದು ದೇಹ ಮನಸ್ಸು ಬುದ್ಧಿ ಇಂದ್ರಿಯಗಳು. ಇವುಗಳು ಹೊರಗಿನ ವಸ್ತುವಿನ ಸಂಬಂಧ ಮಾಡಿಕೊಂಡು ಕೆಲಸ ಮಾಡುತ್ತಿವೆ. ಆತ್ಮನಾದರೊ ಇದನ್ನು ನೋಡುತ್ತಿರುವನೇ ಹೊರತು ಮಾಡುತ್ತಿರುವವನಲ್ಲ. ಯಾವಾಗ ಇವನು ಆತ್ಮನನ್ನು ಅನಾತ್ಮದಿಂದ ಬೇರ್ಪಡಿಸುತ್ತಾನೆಯೊ ಅವನು ಮುಕ್ತನಾಗುತ್ತಾನೆ. ನಾವು ಅಜ್ಞಾನದಲ್ಲಿರುವಾಗ ಅನಾತ್ಮನೊಂದಿಗೆ ಸಂಪರ್ಕ ಪಡೆದು ಕೊಂಡು ಅದರ ಗುಣಧರ್ಮಗಳೆ ನಮಗೆ ಬಂದಂತೆ ತಿಳಿದುಕೊಳ್ಳುವೆವು. ಯಾವಾಗ ಅದರಿಂದ ಪಾರಾಗುತ್ತೇವೆಯೋ ಆಗ ನಾವು ಅದರ ಕರ್ಮಗಳಿಂದ ಬಾಧಿತರಾಗುವುದಿಲ್ಲ.

ಹಾಗಾದರೆ ಹಲವು ಕೆಟ್ಟ ಕೆಲಸಗಳನ್ನೆಲ್ಲ ಮಾಡುತ್ತ ನಾನು ಬರೀ ಸಾಕ್ಷಿ, ನನಗೂ ಅದಕ್ಕೂ ಏನೂ ಸಂಬಂಧವಿಲ್ಲ ಎಂದು ಹೇಳಬಹುದು. ಯಾರು ಸಂಬಂಧವಿಲ್ಲ ಎಂದು ಹೇಳಿಕೊಳ್ಳುತ್ತಾನೆಯೋ, ಅವನ ಮೂಲಕ ಯಾವ ಕೆಟ್ಟ ಕೆಲಸವೂ ಆಗುವಂತೆ ಇಲ್ಲ. ಏಕೆಂದರೆ ನಾನು ಬೇರೆ ಎಂಬ ಜ್ಞಾನ ಚೆನ್ನಾಗಿ ಬರಬೇಕಾದರೆ ಅವನು ತನ್ನ ಪೂರ್ವ ವಾಸನೆಗಳೆಲ್ಲದರಿಂದ ಪಾರಾದ ಮೇಲೆ ಮಾತ್ರ. ಈ ಸ್ಥಿತಿಗೆ ಬಂದವನಲ್ಲಿ ಕೆಲವು ಉತ್ತಮ ಸಂಸ್ಕಾರಗಳು ಮಾತ್ರ ಉಳಿದುಕೊಂಡಿರುತ್ತವೆ. ಅವು ತಮ್ಮ ವೇಗ ಇರುವವರೆಗೆ ಚಲಿಸಿಕೊಂಡು ಹೋಗುತ್ತಿದ್ದು ಆಮೇಲೆ ಕೆಳಗೆ ಬೀಳುವುವು.

ಜೀವಾತ್ಮ ತಾನು ಅಕರ್ತೃ ಎಂದು ನೋಡುವ ಮಟ್ಟವನ್ನು ಮುಟ್ಟಬೇಕಾದರೆ ಬಹಳ ಕಷ್ಟ. ಅದು ವಿಕಾಸದ ತುತ್ತ ತುದಿಯ ಮೆಟ್ಟಲು. ಅದರ ಮೇಲೆ ನಿಂತಾಗ ಮಾತ್ರ ತಾನು ಕೇವಲ ಸಾಕ್ಷಿ, ಎಲ್ಲ ಕರ್ಮಗಳು ಪ್ರಕೃತಿಯ ಮೂಲಕವಾಗಿ ಆಗುತ್ತಿವೆ ಎಂದು ತೋರಬೇಕಾದರೆ. ಆ ಸಮಯದಲ್ಲಿ ಹಾವು ಹೇಗೆ ತನ್ನ ಪೊರೆಯನ್ನು ಬಿಟ್ಟು ಹೊರಗೆ ಬರುವುದೊ ಹಾಗೆ ಜ್ಞಾನಿ ತನ್ನ ದೇಹ ಇಂದ್ರಿಯ ಮನಸ್ಸು ಬುದ್ಧಿಯ ಪೊರೆಯನ್ನು ಬಿಟ್ಟು ಸಾಕ್ಷಿಯಂತೆ ನೋಡಬಲ್ಲ ಸ್ಥಿತಿಗೆ ಬರುವನು.

\begin{verse}
ಯದಾ ಭೂತಪೃಥಗ್ಭಾವಮೇಕಸ್ಥಮನುಪಶ್ಯತಿ~।\\ತತ ಏವ ಚ ವಿಸ್ತಾರಂ ಬ್ರಹ್ಮ ಸಂಪದ್ಯತೇ ತದಾ \versenum{॥ ೩೦~॥}
\end{verse}

{\small ಪ್ರಾಣಿಗಳ ಪೃಥಗ್ಭಾವ ಒಂದೇ ಆತ್ಮನಲ್ಲಿರುವುದೆಂದು, ಮತ್ತು ಅದರಿಂದಲೇ ಇವುಗಳೆಲ್ಲ ವಿಕಾಸವನ್ನು ಹೊಂದಿದೆ ಎಂದು ಯಾವಾಗ ನೋಡುತ್ತಾನೆಯೊ ಆಗ ಬ್ರಹ್ಮನನ್ನು ನೋಡುತ್ತಾನೆ.}

ಜೀವಾತ್ಮರುಗಳೆಲ್ಲ ಪರಮಾತ್ಮನಿಂದ ಸಿಡಿದು ಬಂದವರು. ಉರಿಯುತ್ತಿರುವ ಅಗ್ನಿಕುಂಡದಿಂದ ಸಿಡಿಯುವ ಕಿಡಿಗಳಂತೆ ಜೀವಾತ್ಮರೆಲ್ಲರೂ ಒಂದೇ ಪರಬ್ರಹ್ಮದ ಮೂಲದಿಂದ ಬಂದವರು. ಶರಾವತಿಯಲ್ಲಿ ಉತ್ಪತ್ತಿಯಾಗುವ ವಿದ್ಯುಚ್ಛಕ್ತಿ ಮನೆಮನೆಗೆ ಹೋಗಿ ಬಲ್ಬಿನಲ್ಲಿ ಬೆಳಗುವಂತೆ, ಒಂದೇ ಪರಮಾತ್ಮನ ಮೂಲದಿಂದ ಬಂದು ಈ ಜೀವರಾಶಿಗಳೆಲ್ಲಾ ಬೆಳಗುತ್ತಿವೆ. ಸೂರ್ಯ ಒಂದೇ, ಅವನಿಂದ ಅನಂತ ಕಿರಣಗಳು ಎಲ್ಲಾಕಡೆಯೂ ಪಸರಿಸುತ್ತಿವೆ. ಕಿರಣಗಳು ಬೇರೆ ಬೇರೆ ಆಗಿವೆ. ಆದರೆ ಅವುಗಳೆಲ್ಲ ಬಂದಿರುವುದು ಒಂದೇ ಮೂಲದಿಂದ. ಕಿರಣಗಳ ನಿಜವಾದ ಅಸ್ತಿತ್ವ ಸೂರ್ಯ ನಲ್ಲಿದೆ.

ಒಂದೇ ಪರಮಾತ್ಮನ ಮೂಲದಿಂದ ಹೊರಟು ಹಲವಾರು ಆಕಾರಗಳನ್ನು ಪಡೆದುಕೊಂಡು ವಿಕಾಸವಾಗಿ ಕೊನೆಗೆ ಎಲ್ಲ ಆಕಾರವನ್ನೂ ಬಿಟ್ಟು ಬಂದಲ್ಲಿಗೆ ಸೇರುತ್ತವೆ. ವಿಕಾಸ ಎಂಬುದು ಆತ್ಮನ ದೃಷ್ಟಿಯಿಂದ ಅಲ್ಲ. ಅದು ತೊಟ್ಟಿರುವ ಉಪಾಧಿಗಳಲ್ಲಿ ಮಾತ್ರ. ಕೆಲವು ಉಪಾಧಿಗಳು ಪಾರದರ್ಶಕ ವಾಗಿವೆ. ಹಿಂದೆ ಏನಿದೆಯೋ ಅದನ್ನು ಚೆನ್ನಾಗಿ ತೋರುವುವು. ಮತ್ತೆ ಕೆಲವು ಹಿಂದೆ ಇರುವುದನ್ನು ಕಾಣದಂತೆ ಮರೆಸುವುವು. ಅಥವಾ ಅಲ್ಲಿ ಒಂದು ಇದ್ದರೆ ಇಲ್ಲಿ ಒಂದು ರೀತಿ ಕಾಣುವಂತೆ ಮಾಡುವುದು. ವಿಕಾಸ ಹೊಂದುವಾಗ ಅತ್ಯಂತ ಕೀಳು ಪ್ರಾಣಿಯಿಂದ ಪ್ರಬುದ್ಧನಾದ ಮಾನವನವರೆಗೆ ಹಲವು ಮೆಟ್ಟಲುಗಳನ್ನು ಹತ್ತಿಕೊಂಡು ಹೋಗಬೇಕು. ಮೇಲೆ ಮೇಲೆ ಹೋದಂತೆ ಅದು ಹೆಚ್ಚು ಹೆಚ್ಚಾಗಿ ಹಿಂದಿರುವ ಪರಮಾತ್ಮನನ್ನು ವ್ಯಕ್ತಪಡಿಸುವುದು. ಸಾಗರದಿಂದ ಮೋಡ ಒಂದು ಏಳು ವುದು. ದೂರದಲ್ಲಿ ಎಲ್ಲಿಯೋ ಮಳೆಯಾಗಿ ಬೀಳುವುದು. ಅದು ಹನಿಯಾಗಿ ಹರಿದು, ಹಳ್ಳ ತೊರೆ ನದಿಯನ್ನು ಸೇರಿ ಕೊನೆಗೆ ಸಾಗರವನ್ನು ಸೇರಿ ತನ್ನ ಪ್ರಯಾಣವನ್ನು ಮುಗಿಸುವುದು.

ಜ್ಞಾನಿಯಾದವನಿಗೆ ಪ್ರತಿಯೊಂದು ಜೀವರಾಶಿಯೂ ಪರಮಾತ್ಮನನ್ನು ನೋಡುವುದಕ್ಕೆ ಇರುವ ರಂಧ್ರದಂತಿದೆ. ವಿಕಾಸವೆನ್ನುವುದು ಆ ರಂಧ್ರದಲ್ಲಿ ಮಾತ್ರ. ಯಾವಾಗ ರಂಧ್ರ ಸಣ್ಣದಾಗಿದೆಯೋ ಅಲ್ಲಿ ಕಾಣುವುದು ಅಲ್ಪ. ವಿಸ್ತಾರವಾಗುತ್ತ ಬಂದಂತೆ ಹೆಚ್ಚು ಹೆಚ್ಚು ವಿವರಗಳು ಕಾಣುತ್ತವೆ. ಕೊನೆಗೆ ಇರುವ ಆತಂಕವೆಲ್ಲ ಹೋದಮೇಲೆ, ವಸ್ತು ಹೇಗೆ ಇದೆಯೋ ಹಾಗೆ ಕಾಣುವುದು. ಇಲ್ಲಿ ವಿಕಾಸವಾಗುತ್ತಿರುವುದು ರಂಧ್ರ, ಅದರ ಮೂಲಕ ಕಾಣುವ ದೃಶ್ಯವಲ್ಲ. ಜ್ಞಾನಿ ನೋಡುವುದು ಪರಮಾತ್ಮನ ದೃಷ್ಟಿಯಿಂದ. ಈ ಜೀವರಾಶಿಗಳೆಲ್ಲ ಬಂದಿರುವುದೆಲ್ಲಿಂದ? ಪರಮಾತ್ಮನಿಂದ. ಹೊರಟಿರುವುದು ಪರಮಾತ್ಮನೆಡೆಗೆ. ಹಲವಾರು ದೇಹಗಳನ್ನು ಧರಿಸಿ ಅದರ ಮೂಲಕ ಬರುವ ಸುಖ ದುಃಖಗಳೆನ್ನೆಲ್ಲಾ ಅನುಭವಿಸಿ ಅನಂತವನ್ನು ಸಾಂತದ ಮೂಲಕ ವ್ಯಕ್ತಪಡಿಸಲು ಸಾಧ್ಯವಿಲ್ಲ ಎಂದು ಕೊನೆಗೆ ಉಪಾಧಿಗಳನ್ನೆಲ್ಲ ಬಿಟ್ಟು ಪರಮಾತ್ಮನಲ್ಲಿ ಮುಳುಗುವುದು. ವಿಕಾಸದ ದೃಷ್ಟಿಯಿಂದ ನೋಡಿದಾಗ ನಾಮರೂಪಗಳು ಬದಲಾಯಿಸುತ್ತಿದ್ದರೂ ಜ್ಞಾನಿ ನೋಡುವುದು ವಿಕಾಸದ ಹಿಂದೆ ಇರುವ ಪರಮಾತ್ಮನ ದೃಷ್ಚಿಯಿಂದ. ಅವನು ವಿಕಾಸದ ಕಾಲದಲ್ಲಿಯೂ, ಎಲ್ಲದರ ಹಿಂದೆಯೂ ಒಂದನ್ನೆ ನೋಡುತ್ತಿರುವನು. ಅದೇ ಬ್ರಹ್ಮಸಾಕ್ಷಾತ್ಕಾರ.

\begin{verse}
ಅನಾದಿತ್ವಾನ್ನಿರ್ಗುಣತ್ವಾತ್ಪರಮಾತ್ಮಾಯಮವ್ಯಯಃ~।\\ಶರೀರಸ್ಥೋಽಪಿ ಕೌಂತೇಯ ನ ಕರೋತಿ ನ ಲಿಪ್ಯತೇ \versenum{॥ ೩೧~॥}
\end{verse}

{\small ಅರ್ಜುನ, ಅನಾದಿಯಾಗಿರುವುದರಿಂದಲೂ, ನಿರ್ಗುಣನಾಗಿರುವುದರಿಂದಲೂ, ಈ ಅವ್ಯಯನಾದ ಪರಮಾತ್ಮ ಶರೀರದಲ್ಲಿದ್ದರೂ ಯಾವ ಕರ್ಮವನ್ನೂ ಮಾಡುವುದಿಲ್ಲ ಮತ್ತು ಲಿಪ್ತನಾಗುವುದಿಲ್ಲ.}

ವಿಕಾಸವಾಗುತ್ತಿರುವ ಜೀವನದಲ್ಲಿ ಸದಾ ಕಾಲದಲ್ಲಿಯೂ ಪರಮೇಶ್ವರನು ಬೆಳಗುತ್ತಿರುವನು. ಅವನು ಅನಾದಿ. ಯಾವಾಗಲೂ ಇರುವವನು. ಅವನಿಲ್ಲದ ಕಾಲವೇ ಇಲ್ಲ. ಕಾಲವೆಂಬ ಘಟನೆ ಬರುವುದೇ ಅವನಿಂದ. ಅವನು ನಿರ್ಗುಣವಾಗಿ ಇರುವನು. ಸತ್ತ್ವ, ರಜಸ್ಸು, ತಮಸ್ಸು, ಯಾವ ಗುಣಗಳ ಪ್ರಭಾವಕ್ಕೂ ಅವನು ಸಿಕ್ಕಿಲ್ಲ. ಇವೆಲ್ಲ ಇವೆ. ಆದರೆ ಅವನನ್ನು ಬಂಧಿಸಲಾರದೆ ಇವೆ. ಆ ಪರಮೇಶ್ವರನು ಅವ್ಯಯ, ಯಾವ ಬದಲಾವಣೆಗೂ ಸಿಲುಕಿಲ್ಲ, ಇವುಗಳನ್ನೆಲ್ಲ ಕೇವಲ ಸಾಕ್ಷಿಯಂತೆ ನಿಂತು ನೋಡುತ್ತಿರುವನು. ಸೂರ್ಯ ಬೆಳಗುತ್ತಿರುವನು. ಆ ಬೆಳಕಿನಲ್ಲಿ ಹಲವು ಜನ ಹಲವು ಕೆಲಸಗಳನ್ನು ಮಾಡುತ್ತಿರುವರು. ಸೂರ್ಯ ಇವುಗಳಾವುದರಿಂದಲೂ ಬದಲಾಯಿಸುವುದಿಲ್ಲ. ಈ ಪರಮಾತ್ಮ ಶರೀರದಲ್ಲಿ ಜೀವಾತ್ಮನಿಗೆ ಹಿನ್ನೆಲೆಯಂತೆ ಸದಾ ಇರುವನು. ಇಲ್ಲಿ ಆಗುತ್ತಿರುವ ಕೆಲಸ, ಆಲೋಚನೆ ಎಲ್ಲವನ್ನೂ ನೋಡುತ್ತಿರುವನು. ಆದರೆ ಅವನು ಅದಕ್ಕೆ ಪ್ರೋತ್ಸಾಹವನ್ನು ಕೊಡುತ್ತಿಲ್ಲ, ಅದರಂತೆಯೇ ಯಾವ ಆತಂಕವನ್ನೂ ತಂದೊಡ್ಡುವುದಿಲ್ಲ. ಜೀವಿ ತನ್ನ ಕರ್ಮಾನುಸಾರ ಪಾಪ ಪುಣ್ಯವನ್ನು ಮಾಡುವುದು. ಆದರೆ ಪರಮಾತ್ಮನಾದರೊ ಇವುಗಳಿಂದ ಬಾಧಿತನಾಗುವುದಿಲ್ಲ. ಕರ್ಮದಿಂದ ಬರುವ ಫಲಾಫಲಗಳಿಂದ ಅವನು ಕುಗ್ಗುವುದೂ ಇಲ್ಲ, ಹಿಗ್ಗುವುದೂ ಇಲ್ಲ. ಅವನು ಕರ್ಮಕ್ಕೆ ಆಸಕ್ತನಲ್ಲ, ಅದರಿಂದ ಬರುವ ಫಲಕ್ಕೆ ಆಸಕ್ತನಲ್ಲ.

\begin{verse}
ಯಥಾ ಸರ್ವಗತಂ ಸೌಕ್ಷ್ಮ್ಯಾದಾಕಾಶಂ ನೋಪಲಿಪ್ಯತೇ~।\\ಸರ್ವತ್ರಾವಸ್ಥಿತೋ ದೇಹೇ ತಥಾತ್ಮಾ ನೋಪಲಿಪ್ಯತೇ \versenum{॥ ೩೨~॥}
\end{verse}

{\small ಸರ್ವವ್ಯಾಪಿಯಾದ ಆಕಾಶ ಸೂಕ್ಷ್ಮವಾಗಿರುವುದರಿಂದ ಹೇಗೆ ಲಿಪ್ತವಾಗಿರುವುದಿಲ್ಲವೋ ಹಾಗೆಯೇ ಎಲ್ಲಾ ದೇಹದಲ್ಲಿರುವ ಆತ್ಮ ಲಿಪ್ತನಾಗುವುದಿಲ್ಲ.}

ಪರಮಾತ್ಮ ಎಲ್ಲಾ ಜೀವರಾಶಿಗಳಲ್ಲೂ ಇದ್ದರೂ ಹೇಗೆ ಲಿಪ್ತನಾಗುವುದಿಲ್ಲವೋ ಅದಕ್ಕೆ ಒಂದು ಉದಾಹರಣೆಯನ್ನು ಕೊಡುವನು. ಆಕಾಶ ಸರ್ವವ್ಯಾಪಿಯಾಗಿ ಎಲ್ಲ ಕಡೆಯಲ್ಲಿಯೂ ಇದೆ. ಒಳ್ಳೆಯ ಪಾತ್ರಗಳಿವೆ, ಕೆಟ್ಟ ಪಾತ್ರಗಳಿವೆ. ಆಕಾಶ ಅದರಿಂದ ಬಾಧಿತವಾಗುವುದಿಲ್ಲ. ಹಲವಾರು ಫ್ಯಾಕ್ಟರಿ ಗಳಿಂದ ಬೇಕಾದಷ್ಟು ಗಾಳಿಗಳು ಬಂದು ಆಕಾಶಕ್ಕೆ ಬೀಳುತ್ತಿದ್ದರೂ ಇದರಿಂದ ಆಕಾಶ ಬಾಧಿತ ವಾಗುವುದಿಲ್ಲ. ಮನೆಯಲ್ಲಿರುವ ಆಕಾಶವೂ ಒಂದೇ; ದೇವಸ್ಥಾನದಲ್ಲಿರುವ ಆಕಾಶವೂ ಒಂದೇ. ಇದರಲ್ಲಿ ಯಾವ ವಿಧವಾದ ವ್ಯತ್ಯಾಸವೂ ಇಲ್ಲ. ಪಾಪಾತ್ಮನ ಹಿಂದೆ ಪರಮಾತ್ಮನಿರುವನು. ಪುಣ್ಯಾತ್ಮನ ಹಿಂದೆ ಪರಮಾತ್ಮನಿರುವನು. ಅದರಂತೆಯೇ ಚರಂಡಿಯ ನೀರು ಸೂರ್ಯನನ್ನು ಪ್ರತಿಬಿಂಬಿಸುತ್ತಿದೆ. ಶುದ್ಧವಾದ ಕೊಳದ ನೀರೂ ಪ್ರತಿಬಿಂಬಿಸುತ್ತಿದೆ. ಇದರಿಂದ ಸೂರ್ಯನಿಗೆ ಏನೂ ಆಗುವುದಿಲ್ಲ.

\begin{verse}
ಯಥಾ ಪ್ರಕಾಶಯತ್ಯೇಕಃ ಕೃತ್ಸ್ನಂ ಲೋಕಮಿಮಂ ರವಿಃ~।\\ಕ್ಷೇತ್ರಂ ಕ್ಷೇತ್ರೀ ತಥಾ ಕೃತ್ಸ್ನಂ ಪ್ರಕಾಶಯತಿ ಭಾರತ \versenum{॥ ೩೩~॥}
\end{verse}

{\small ಅರ್ಜುನ, ಒಬ್ಬನೇ ಸೂರ್ಯ ಈ ಲೋಕವನ್ನೆಲ್ಲ ಹೇಗೆ ಪ್ರಕಾಶಿಸುತ್ತಿರುವನೋ ಹಾಗೆ ಕ್ಷೇತ್ರಜ್ಞನು ಸಮಸ್ತ ಕ್ಷೇತ್ರವನ್ನೂ ಪ್ರಕಾಶಿಸುತ್ತಿರುವನು.}

ಒಬ್ಬನೇ ಸೂರ್ಯ ಈ ಪ್ರಪಂಚವನ್ನೆಲ್ಲ ಬೆಳುಗುವನು. ಎಲ್ಲದರ ಮೇಲೂ ಪಕ್ಷಪಾತವಿಲ್ಲದೆ ಬೀಳುವನು. ಬಡವನ ಗುಡಿಸಿಲಿನ ಮೇಲೆ ಬೀಳುವನು. ಅರಸನ ಅರಮನೆಯ ಮೇಲೆ ಬೀಳುವನು. ನೀರಿನ ಮೇಲೆ, ನೆಲದ ಮೇಲೆ, ಗ್ಲಾಸಿನ ಮೇಲೆ ಬೀಳುವನು. ಕೆಲವು ಅವನನ್ನು ಮಂದವಾಗಿ ಪ್ರತಿಬಿಂಬಿಸುತ್ತವೆ. ಮತ್ತೆ ಕೆಲವು ಅವನನ್ನು ಪ್ರತಿಬಿಂಬಿಸಲೇ ಆರವು. ಆದರೂ ಸೂರ್ಯನಿಗೇನೂ ಪಕ್ಷಪಾತವಿಲ್ಲ. ಅವನೂ ಕೊಡುವುದನ್ನು ಎಲ್ಲರಿಗೂ ಸಮವಾಗಿ ಕೊಡುತ್ತಾನೆ. ಆದರೆ ಅವು ಒಂದೇ ಸಮನಾಗಿ ಪ್ರತಿಬಿಂಬಿಸುವುದಿಲ್ಲ.

ಹಾಗೆಯೇ ಪರಮಾತ್ಮನೆಂಬ ಸರ್ವ ಸಾಮಾನ್ಯವಾದ ಕ್ಷೇತ್ರಜ್ಞನು, ಇಲ್ಲಿ ಸಮಸ್ತ ಭೂತ ಗಳಲ್ಲಿಯೂ ಪ್ರಕಾಶಿಸುತ್ತಿರುವನು. ಇದು ವಿದ್ಯುಚ್ಛಕ್ತಿ ಸೊನ್ನೆ ಕ್ಯಾಂಡಲ್ ಬಲ್ಬಿನಿಂದ ಸಾವಿರಾರು ಕ್ಯಾಂಡಲ್ ಬಲ್ಬಿನವರೆಗೆ ಇರುವ ಬಲ್ಬುಗಳಲ್ಲಿ ಬೆಳುಗುತ್ತಿರುವಂತೆ. ಪ್ರತಿಯೊಂದು ಜೀವರಾಶಿಯೂ ತನ್ನ ಯೋಗ್ಯತಾನುಸಾರ ಪರಮಾತ್ಮನನ್ನು ವ್ಯಕ್ತಗೊಳಿಸುತ್ತದೆ. ಉಪಾಧಿಗಳು ಹೆಚ್ಚಾಗಿದ್ದರೆ ಅವನು ಮಂದವಾಗಿ ಕಾಣುತ್ತಾನೆ. ಉಪಾಧಿಗಳು ತೆಳ್ಳಗಿದ್ದರೆ, ಪಾರದರ್ಶಕವಾಗಿದ್ದರೆ, ಅವನ ಕಾಂತಿ ಚೆನ್ನಾಗಿ ಕಾಣುವುದು. ಎಲ್ಲಾ ಕಾಂತಿಯ ಮೂಲವೂ ಒಂದೇ. ಅದೇ ಪರಮಾತ್ಮ. ಎಲ್ಲಾ ಜೀವಿಗಳೆಂಬ ಕ್ಷೇತ್ರಕ್ಕೆ ನೀರು ಬರುವುದು, ಪರಮಾತ್ಮನೆಂಬ ಮಹಾಸಾಗರದಿಂದ. ಮೈಸೂರಿನಲ್ಲಿ ಮನೆಗಳಲ್ಲಿ ನಲ್ಲಿ ಇವೆ; ಆ ನೀರೆಲ್ಲ ಕಾವೇರಿಯಿಂದಲೇ ಬಂದುದು. ಮನೆಗಳಲ್ಲಿ ವಿದ್ಯುತ್​ದೀಪಗಳಿವೆ. ಅದಕ್ಕೆ ಶಕ್ತಿಯೆಲ್ಲ ಶರಾವತಿಯ ವಿದ್ಯುತ್ ಆಗಾರದಿಂದಲೇ ಬಂದುದು. ಅತ್ಯಂತ ಸೂಕ್ಷ್ಮ ಕ್ರಿಮಿಕೀಟದಿಂದ ಹಿಡಿದು ವಿಕಾಸದ ಏಣಿಯಲ್ಲಿ ತುತ್ತತುದಿಯನ್ನು ಮುಟ್ಟಿದ ಶ್ರೇಷ್ಠ ಮಾನವನವರೆಗೆ ಎಲ್ಲಾ ಕಡೆಯಲ್ಲಿ ಸರ್ವಸಾಮಾನ್ಯ ಬೆಳುಗುವವನೆ ಪರಮಾತ್ಮ.

\begin{verse}
ಕ್ಷೇತ್ರಕ್ಷೇತ್ರಜ್ಞಯೋರೇವಮಂತರಂ ಜ್ಞಾನಚಕ್ಷುಷಾ~।\\ಭೂತಪ್ರಕೃತಿಮೋಕ್ಷಂ ಚ ಯೇ ವಿದುರ್ಯಾಂತಿ ತೇ ಪರಮ್ \versenum{॥ ೩೪~॥}
\end{verse}

{\small ಕ್ಷೇತ್ರಕ್ಷೇತ್ರಜ್ಞರ ಭೇದಗಳನ್ನು ಮತ್ತು ಭೂತಗಳ ಪ್ರಕೃತಿಯ ಮೋಕ್ಷವನ್ನು ಯಾರು ಜ್ಞಾನದೃಷ್ಟಿಯಿಂದ ತಿಳಿದುಕೊಳ್ಳುವರೋ ಅವರು ಬ್ರಹ್ಮನನ್ನು ಹೊಂದುತ್ತಾರೆ.}

ಬ್ರಹ್ಮಸಾಕ್ಷಾತ್ಕಾರವನ್ನು ಯಾರು ಪಡೆಯಬಲ್ಲರು ಎಂಬುದಕ್ಕೆ ಕೆಲವು ಯೋಗ್ಯತೆಗಳನ್ನು ಹೇಳು ತ್ತಾನೆ. ಯಾರಿಗೆ ಅದು ಇದೆಯೋ ಅವರು ಪಡೆಯುತ್ತಾರೆ. ದೇವರು ಯಾವುದೋ ಒಂದು ಧರ್ಮಕ್ಕೊ ಜಾತಿಗೊ ವರ್ಣಕ್ಕೊ ಆಶ್ರಮಕ್ಕೊ ಮೀಸಲಿಲ್ಲ.

ಕ್ಷೇತ್ರ ಮತ್ತು ಕ್ಷೇತ್ರಜ್ಞರನ್ನು ಬಿಡಿಸಬಲ್ಲ ಶಕ್ತಿ ಅವನಲ್ಲಿರಬೇಕು. ಮರಳು ಸಕ್ಕರೆ ಹೇಗೆ ಬೆರತುಹೋಗಿದೆಯೋ ಹಾಗೆ ಕ್ಷೇತ್ರ ಮತ್ತು ಕ್ಷೇತ್ರಜ್ಞರು ಬೆರೆತುಹೋದಂತೆ ಕಾಣುವರು. ಅದನ್ನು ಜ್ಞಾನಿ ಪ್ರತ್ಯೇಕಿಸಬಲ್ಲ. ನಮ್ಮಲ್ಲಿಯೇ ಕ್ಷೇತ್ರ ಇದೆ. ಅದೇ ದೇಹ, ಮನಸ್ಸು, ಬುದ್ಧಿ, ಇಂದ್ರಿಯಗಳು. ಇದನ್ನೇ ಮುಕ್ಕಾಲುಪಾಲು ಜನ ತಾವು ಎಂದು ಭಾವಿಸುವರು. ಜ್ಞಾನಿಯಾದರೋ ಇದರ ಹಿಂದೆ ತೂರಿಹೋಗುವನು. ಎಕ್ಸ್​ರೇಗಳು ಮನುಷ್ಯನ ದೇಹವನ್ನು ತೂರಿಹೋಗಿ ಹಿಂದಿರುವ ಅಸ್ಥಿಪಂಜರ ವನ್ನು ತೋರುವುವು. ನಮ್ಮ ಜಡಗಣ್ಣಿಗೆ ಹೊರಗಿನ ಮಾಂಸದ ದೇಹ ಮಾತ್ರ ಕಾಣುವುದು. ಹಿಂದಿರುವ ಮೂಳೆಗಳು ಕಾಣುವುದಿಲ್ಲ. ಅದರಂತೆಯೇ ಜ್ಞಾನಿ ಕ್ಷೇತ್ರವನ್ನು ತೂರಿಹೋಗಿ ಅದನ್ನು ಪ್ರತ್ಯೇಕಿಸುತ್ತಾನೆ. ತನ್ನ ದೇಹದಲ್ಲಿ ತನಗೆ ಸಾಕ್ಷಿಯಾಗಿ ನಿಂತು ನೋಡುವವನು ಜೀವಾತ್ಮ. ಈ ಬ್ರಹ್ಮಾಂಡದಲ್ಲಿ ಎಲ್ಲಾ ಜೀವರಾಶಿಗಳಿಗೂ ಸಾಕ್ಷಿಯಾಗಿರುವವನು ಪರಮಾತ್ಮ. ಇದನ್ನು ತತ್ ಕ್ಷಣವೇ ಸಿಪ್ಪೆಯಂತೆ ಸುಲಿಯಬಲ್ಲವನೇ ಜ್ಞಾನಿ.

ಅಂತಹ ಜ್ಞಾನಿ, ಭೂತಗಳು ಪ್ರಕೃತಿಯಿಂದ ಹೇಗೆ ಬಿಡುಗಡೆ ಆಗುತ್ತವೆ ಎಂಬುದನ್ನು ಅರಿಯ ಬಲ್ಲ. ಮೋಡ ಸೂರ್ಯನನ್ನು ಮುಚ್ಚಿ ಅದರ ಕಾಂತಿ ನಮಗೆ ಬರದಂತೆ ಮಾಡುವುದು. ಹಾಗೆಯೆ ಪ್ರಕೃತಿ ಎಂಬ ಮೋಡದಲ್ಲಿರುವ ತ್ರಿಗುಣಗಳು ಪಂಚಭೂತಗಳು, ಇಂದ್ರಿಯ, ಮನಸ್ಸು ಬುದ್ಧಿ ಇವುಗಳೆಲ್ಲ ಕ್ಷೇತ್ರಜ್ಞನನ್ನು ಮುಚ್ಚಿವೆ. ವ್ಯಷ್ಟಿಯ ಕ್ಷೇತ್ರಜ್ಞನೇ ಜೀವಾತ್ಮ, ಸಮಷ್ಟಿಯ ಕ್ಷೇತ್ರಜ್ಞನೇ ಪರಮಾತ್ಮ. ಎರಡರ ಜ್ಞಾನವನ್ನು ಮುಚ್ಚಿರುವುದೇ ಪ್ರಕೃತಿಯ ಮೋಡ. ಯಾವಾಗ ಮೋಡ ಸರಿಯುವುದೋ ಆಗ ಆತ್ಮದರ್ಶನ ಅವನಿಗೆ ಆಗುವುದು. ತನ್ನ ನೈಜ ಸ್ವಭಾವ ಏನು ಎನ್ನುವುದು ಅರ್ಥವಾಗುವುದು.

