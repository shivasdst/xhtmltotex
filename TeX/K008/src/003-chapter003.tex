
\chapter{ಕರ್ಮಯೋಗ}

ಸ್ಥಿತಪ್ರಜ್ಞನ ಲಕ್ಷಣವನ್ನು ಕೇಳಿ ಆದಮೇಲೆ ಅರ್ಜುನ ಶ್ರೀಕೃಷ್ಣನನ್ನು ಹೀಗೆ ಕೇಳುತ್ತಾನೆ.

\begin{verse}
ಜ್ಯಾಯಸೀ ಚೇತ್ ಕರ್ಮಣಸ್ತೇ ಮತಾ ಬುದ್ಧಿರ್ಜನಾರ್ದನ ।\\ತತ್ಕಿಂ ಕರ್ಮಣಿ ಘೋರೇ ಮಾಂ ನಿಯೋಜಯಸಿ ಕೇಶವ \versenum{॥ ೧ ॥}
\end{verse}

{\small ಜನಾರ್ದನ, ಕರ್ಮಕ್ಕಿಂತ ಬುದ್ಧಿಯೇ ಹೆಚ್ಚಿನದೆಂದು ನಿನ್ನ ಅಭಿಪ್ರಾಯವಾದರೆ, ಕೇಶವ, ನನ್ನನ್ನು ಘೋರವಾದ ಕರ್ಮದಲ್ಲಿ ಏಕೆ ನಿರತವಾಗುವಂತೆ ಮಾಡುತ್ತಿರುವೆ?}

ಶ್ರೀಕೃಷ್ಣ, ಹಿಂದಿನ ಅಧ್ಯಾಯದಲ್ಲಿ ಕರ್ಮಕ್ಕಿಂತ ಅದನ್ನು ಸಮತ್ವದಿಂದ ಮಾಡುವ ಬುದ್ಧಿಯಲ್ಲಿ ಶರಣಾಗು ಎಂದ. ಕೆಲಸವನ್ನೇ ಬಿಟ್ಟು ಸಮತ್ವಬುದ್ಧಿಯಲ್ಲಿ ಶರಣಾಗು ಎನ್ನಲಿಲ್ಲ. ಕೆಲಸ ಮಾಡು ತ್ತಿರುವಾಗ ಕೆಲವು ವೇಳೆ ಜಯ, ಕೆಲವು ವೇಳೆ ಅಪಜಯ ಬರಬಹುದು. ಕೆಲವರು ಹೊಗಳಬಹುದು ಮತ್ತೆ ಕೆಲವರು ತೆಗಳಬಹುದು. ಆದರೆ ಅದರ ಮೇಲೆ ಗಮನ ಕೊಡದೆ ಅದನ್ನು ಒಂದೇ ಸಮನಾಗಿ ನೋಡುವುದನ್ನು ಕಲಿತ ದೃಷ್ಟಿಯಲ್ಲಿ ಶರಣಾಗು ಎಂದು ಶ್ರೀಕೃಷ್ಣ ಹೇಳಿದ್ದ. ಇಲ್ಲಿ ಅರ್ಜುನ ಅದನ್ನು ಗ್ರಹಿಸಲಿಲ್ಲ. ಕರ್ಮವನ್ನೇ ಬಿಟ್ಟು ಸಮತ್ವದೃಷ್ಟಿಯಿಂದ ಶರಣಾದರೆ ಸಾಕಲ್ಲ ಎಂದು ಭಾವಿಸಿದನು. ಹೇಗಾದರೂ ಮಾಡಿ ಅವನಿಗೆ ಕಹಿಯಾಗಿರುವಂತಹ ಕರ್ಮದಿಂದ ತಪ್ಪಿಸಿಕೊಳ್ಳಬೇಕೆಂದು ಆಸೆ. ಅದಕ್ಕಾಗಿ ಈ ಪ್ರಶ್ನೆಯನ್ನು ಹಾಕುತ್ತಿರುವನು.

\begin{verse}
ವ್ಯಾಮಿಶ್ರೇಣೇವ ವಾಕ್ಯೇನ ಬುದ್ಧಿಂ ಮೋಹಯಸೀವ ಮೇ ।\\ತದೇಕಂ ವದ ನಿಶ್ಚಿತ್ಯ ಯೇನ ಶ್ರೇಯೋಽಹಮಾಪ್ನುಯಾಮ್ \versenum{॥ ೨ ॥}
\end{verse}

{\small ಮಿಶ್ರವಾದ ಅರ್ಥವುಳ್ಳ ಮಾತಿನಿಂದ ನನ್ನ ಬುದ್ಧಿಯನ್ನು ಭ್ರಾಂತಿಗೊಳಿಸುತ್ತಿರುವಂತೆ ಕಾಣುತ್ತದೆ. ಆದುದರಿಂದ ಯಾವುದೊಂದರಿಂದ ನಾನು ಶ್ರೇಯಸ್ಸನ್ನು ಪಡೆಯಬಹುದೊ ಅದನ್ನು ನಿಶ್ಚಯಿಸಿ ಹೇಳು.}

ಮಿಶ್ರವಾದ ನಿನ್ನ ಮಾತಿನಿಂದ ನನ್ನನ್ನು ಭ್ರಾಂತಿಗೊಳಿಸುವಂತೆ ಇದೆ ಎನ್ನುತ್ತಾನೆ. ಕಾರ್ಯವನ್ನು ಮಾಡುವುದು ಮೇಲು ಎಂದು ಒಂದು ಸಲ ಹೇಳುತ್ತಾನೆ. ಸಮತ್ವದೃಷ್ಟಿಯಿಂದ ನೋಡುವುದೇ ಶ್ರೇಷ್ಠ ಎಂದು ಒಂದು ಸಲ ಹೇಳುತ್ತಾನೆ. ಶ್ರೀಕೃಷ್ಣ ಎಲ್ಲಿಯೂ ಅರ್ಜುನನಿಗೆ ಕರ್ಮವನ್ನು ಬಿಡು ಎಂದು ಹೇಳುವುದಿಲ್ಲ. ಅದನ್ನು ಹಲವಾರು ದೃಷ್ಟಿಕೋನದಿಂದ ಮಾಡಬಹುದು ಎಂದು ಹೇಳುತ್ತಾನೆ. ಆ ದೃಷ್ಟಿಕೋನಗಳಲ್ಲಿ ಸಮತ್ವದೃಷ್ಟಿಯೊಂದು ಶ್ರೇಷ್ಠವಾದ ನಿಲುವು. ಕರ್ಮವನ್ನೇ ಮಾಡದೆ ಇದ್ದರೆ ಇವನಿಗೆ ಸಮತ್ವ ಪ್ರಾಪ್ತವಾಗಿದೆಯೇ ಇಲ್ಲವೆ ಎಂಬುದನ್ನು ತಿಳಿದುಕೊಳ್ಳುವುದು ಹೇಗೆ? ನನಗೆ ಜಗಳ ಕಾಯುವುದಕ್ಕೆ ಯಾರೂ ಸಿಕ್ಕದೆ ಇರುವಾಗ ನಾನು ಜಗಳ ಕಾಯುವುದಿಲ್ಲ, ಸಮಾಧಾನದಿಂದ ಇದ್ದೇನೆ ಎಂದು ಹೇಳಿಕೊಂಡಂತೆ. ಕೆಲಸ ಮಾಡುವಾಗ ಸಮತ್ವಬುದ್ಧಿಯಿಂದ ಚ್ಯುತನಾಗುವ ಅವಕಾಶಗಳು ಹಲವು ಬಂದರೂ ಯಾರು ಬೀಳುವುದಿಲ್ಲವೋ ಅವನಿಗೆ ಅದು ಸಿದ್ಧಿಸಿದೆ ಎಂದು ಹೇಳಬಹುದು. ಈ ಸೂಕ್ಷ್ಮವನ್ನು ಅರ್ಜುನ ಇಲ್ಲಿ ತಿಳಿದಿಲ್ಲ ಎಂದು ತೋರುವುದು.

ಇವೆರಡರಲ್ಲೂ ಯಾವುದು ಮುಖ್ಯವೋ ಎಂದರೆ ಕರ್ಮ ಮಾಡುವುದೆ ಬಿಡುವುದೆ, ಅದರಲ್ಲಿ ಅರ್ಜುನನಿಗೆ ಯಾವುದು ಶ್ರೇಯಸ್ಕರವೋ ಅದನ್ನು ಅವನೇ ತಿಳಿದುಕೊಳ್ಳುವ ಸ್ಥಿತಿಯಲ್ಲೂ ಇಲ್ಲ. ಅದಕ್ಕಾಗಿ ಅದನ್ನು ಶ್ರೀಕೃಷ್ಣನಿಗೆ ನಿಷ್ಕರ್ಷಿಸಿ ಹೇಳು ಎಂದು ಕೇಳಿಕೊಳ್ಳುವನು. ಅರ್ಜುನನಿಗೆ ತನ್ನ ಬುದ್ಧಿಯ ಮಿತಿ ಈಗ ಚೆನ್ನಾಗಿ ಅರಿವಾಗುವುದು. ಅದು ಸೂಕ್ಷ್ಮವಾದ ವಿಷಯವನ್ನು ಗ್ರಹಿಸುವಂತೆ ಇಲ್ಲ. ಅದನ್ನು ನಿಷ್ಕರ್ಷಿಸಿ ಶ್ರೀಕೃಷ್ಣ ಹೇಳಬೇಕು. ರೋಗಿಗೆ ಯಾವ ಔಷಧಿ ಒಳ್ಳೆಯದು ಎಂದು ಗೊತ್ತಿಲ್ಲ. ನುರಿತ ವೈದ್ಯನಿಗೆ ಅದನ್ನು ಹೇಳುವಂತೆ ಕೇಳಿಕೊಳ್ಳುವನು. ಒಬ್ಬ ತನಗೆ ಎಲ್ಲಾ ಗೊತ್ತಿದೆ ಎಂದರೆ ನಾನು ಹೇಳುವುದಕ್ಕೆ ಹೋಗುವುದಿಲ್ಲ. ಯಾವಾಗ ನನಗೆ ಏನೂ ಗೊತ್ತಿಲ್ಲ ಎಂದು ಶಿಷ್ಯ ಹೇಳುವನೋ ಆಗಲೆ ಗುರು ಅವನನ್ನು ತುಂಬಲು ತನ್ನ ಬೋಧನೆಯನ್ನು ಹರಿಸುತ್ತಾನೆ. ಆಗಲೇ ಶ್ರೀಕೃಷ್ಣ ಅರ್ಜುನನ ಸಂದೇಹ ನಿವೃತ್ತಿಗೆ ಬೋಧನೆ ಮಾಡುವನು.

\begin{verse}
ಲೋಕೇಽಸ್ಮಿನ್ ದ್ವಿವಿಧಾ ನಿಷ್ಠಾ ಪುರಾ ಪ್ರೋಕ್ತಾ ಮಯಾನಘ ।\\ಜ್ಞಾನಯೋಗೇನ ಸಾಂಖ್ಯಾನಾಂ ಕರ್ಮಯೋಗೇನ ಯೋಗಿನಾಮ್ \versenum{॥ ೩ ॥}
\end{verse}

{\small ಪಾಪರಹಿತನೇ, ಈ ಲೋಕದಲ್ಲಿ ಎರಡು ಬಗೆಯ ನಿಷ್ಠೆಯನ್ನು ನಾನು ಪೂರ್ವದಲ್ಲಿ ಹೇಳಿದ್ದೇನೆ. ಅದೇ ಜ್ಞಾನಯೋಗದಿಂದ ಸಾಂಖ್ಯರಿಗೆ ಕರ್ಮಯೋಗದಿಂದ ಕರ್ಮಿಗಳಿಗೆ.}

ಪೂರ್ವದಲ್ಲಿ ಎಂದರೆ ಸೃಷ್ಟಿಯ ಪ್ರಾರಂಭದಲ್ಲಿಯೇ ಭಗವಂತನು ಮಾನವರ ಮುಂದೆ ಎರಡು ಮಾರ್ಗಗಳನ್ನು ಇಟ್ಟಿರುವನು. ಅದೇ ಜ್ಞಾನನಿಷ್ಠೆ ಮತ್ತು ಕರ್ಮನಿಷ್ಠೆ. ಇದನ್ನೇ ಆಧುನಿಕ ಕಾಲದ ಭಾವನೆಯಲ್ಲಿ ಹೇಳಬೇಕಾದರೆ ಒಂದು ಅಂತರ್ಮುಖ ಮತ್ತೊಂದು ಬಹಿರ್ಮುಖ. ಇಲ್ಲಿ ಜ್ಞಾನ ಮತ್ತು ಕರ್ಮ ಒಂದನ್ನು ಬಿಟ್ಟು, ಮತ್ತೊಂದು ಇಲ್ಲ. ಒಂದಿರುವ ಕಡೆ ಮತ್ತೊಂದು ಇರುವುದು. ಆದರೆ ಜ್ಞಾನನಿಷ್ಠೆಯಲ್ಲಿ ಮುಂದೆ ಜ್ಞಾನ ಹಿಂದೆ ಕರ್ಮ, ಕರ್ಮನಿಷ್ಠೆಯಲ್ಲಿ ಮುಂದೆ ಕರ್ಮ ಹಿಂದೆ ಜ್ಞಾನ. ಒಂದು ಮತ್ತೊಂದಕ್ಕೆ ಪೂರಕವಾಗಿದೆಯೇ ಹೊರತು ಒಂದು ಮತ್ತೊಂದನ್ನು ಬಿಟ್ಟು ಇಲ್ಲ. ಈ ಪ್ರಪಂಚದಲ್ಲಿ ಎರಡು ಬಗೆಯ ಜನರಿದ್ದಾರೆ. ಅವರಿಬ್ಬರಿಗೂ ಒಂದೇ ಮಾರ್ಗವನ್ನು ಕೊಡುವು ದಕ್ಕೆ ಆಗುವುದಿಲ್ಲ. ಪ್ರತಿಯೊಬ್ಬನಿಗೂ ಅವನವನು ತಿಳಿದುಕೊಳ್ಳಬಲ್ಲಂತಹ ಮಾರ್ಗವನ್ನು ಮುಂದಿಡಬೇಕಾಗಿದೆ. ಅಂತರ್ಮುಖಿ ವಸ್ತುವನ್ನು ತಿಳಿಯಬಯಸುವನು. ಅವನದು ವಿಚಾರ ಪ್ರಧಾನವಾದ ಮಾರ್ಗ. ಹೊರಗೆ ಇರುವ ದೃಶ್ಯಪ್ರಪಂಚವನ್ನು ಅವನು ತಿಳಿಯಬಯಸುವನು. ಅದರ ಹಿಂದೆ ಏನಿದೆ ಅದನ್ನು ತಿಳಿಯಬಯಸುವನು. ತನ್ನನ್ನು ವಿಭಜನೆ ಮಾಡಿ ತಾನು ಹೇಗೆ ಆಗಿದ್ದೇನೆ, ಯಾವ ಯಾವ ಭಾಗಗಳಿವೆ ನನ್ನಲ್ಲಿ ಎಂದು ಕಾರನ್ನೊ ರೇಡಿಯೋವನ್ನೊ ಬಿಚ್ಚಿನೋಡುವ ಯಾಂತ್ರಿಕನಂತೆ ವಿಚಾರಪರ. ಯಾವುದನ್ನೂ ಅದು ಕಾಣುವಂತೆ ತೆಗೆದುಕೊಳ್ಳುವುದಿಲ್ಲ. ವಿಚಾರದ ಒರೆಗಲ್ಲಿನ ಮೇಲೆ ತಿಕ್ಕಿ ಪರೀಕ್ಷೆಮಾಡಿ ಸತ್ಯವನ್ನು ತೆಗೆದುಕೊಳ್ಳುವನು. ಮಿಥ್ಯವನ್ನು ಬಿಸುಡುವನು. ಸತ್ಯವನ್ನು ತಿಳಿದಾದಮೇಲೆ ಅವನೇನು ಸುಮ್ಮನೆ ಇರುವುದಿಲ್ಲ. ಅದನ್ನು ಎಲ್ಲರಿಗೂ ಹೇಳುತ್ತಾನೆ. ಅವರ ಅಜ್ಞಾನವನ್ನು ನಿವಾರಣೆ ಮಾಡುತ್ತಾನೆ. ಹಾಗೆ ಮಾಡುವುದೇ ಜೀವನದಲ್ಲಿ ಒಂದು ದೊಡ್ಡ ಯಜ್ಞವಾಗುವುದು. ಇದೂ ಕೂಡ ಒಂದು ವಿಧವಾದ ಕರ್ಮವೇ ಆಯಿತು. ಇಲ್ಲಿ ಜ್ಞಾನ ಮುಂದೆ ಅದರ ನೆರಳು ಕರ್ಮ ಆಯಿತು. ಅದರಂತೆಯೇ ಒಬ್ಬ ಕರ್ಮವಾದಿ. ಮೊದಲು ವಿಚಾರದಿಂದ ಪ್ರಯೋಜನ ಇಲ್ಲ. ಅವನೊಂದು ಪ್ರಪಂಚದಲ್ಲಿರುವನು. ಆ ಪ್ರಪಂಚ ಸರಿಯಾಗಿರಬೇಕಾದರೆ ಅಲ್ಲಿ ಪ್ರತಿಯೊಬ್ಬನೂ ತನ್ನ ಕರ್ತವ್ಯವನ್ನು ಮಾಡಿಕೊಂಡು ಹೋಗಬೇಕು. ತನ್ನ ಪಾಲಿಗೆ ಬಂದ ಕರ್ತವ್ಯ ನಿರ್ವಹಣೆಯೇ ಅವನಿಗೆ ಮುಖ್ಯವಾಗಿ ಕಾಣುವುದು. ಆದರೆ ಅವನು ಕರ್ಮವನ್ನು ಮಾಡುತ್ತ, ಅವನ ಚಿತ್ತಶುದ್ಧಿಯಾದ ಮೇಲೆ, ನಾನಲ್ಲ ಮಾಡುವವನು, ನಾನೊಂದು ನಿಮಿತ್ತ, ಎಲ್ಲವನ್ನು ಮಾಡಿಸುವವನು ಒಬ್ಬನಿರುವನು ಎಂಬುದನ್ನು ಅರಿಯುವನು. ಅವನಿಗಾಗಿ ಕರ್ಮ, ಅವನಿಗೆ ಅದರಿಂದ ಬರುವ ಫಲಗಳೆಲ್ಲವೂ. ದೊಡ್ಡ ಜಮೀನ್​ದಾರ ತನ್ನ ಜಮೀನಿನಲ್ಲಿ ಹಲವರನ್ನು ಇಟ್ಟುಕೊಂಡು ಕೆಲಸ ಮಾಡಿಸಿಕೊಳ್ಳುವನು. ಕೆಲಸವಾದಮೇಲೆ ಕೂಲಿ ಕೊಟ್ಟು ಕಳುಹಿಸುವನು. ಆ ಕೆಲಸದಿಂದ ಬರುವ ಫಲಕ್ಕೆ ಯಜಮಾನ ಒಡೆಯನೆ ಹೊರತು ಆಳುಗಳಲ್ಲ. ಆಳುಗಳೇನು ಬಿಟ್ಟಿ ದುಡಿಯುವುದಿಲ್ಲ. ಅವರಿಗೆ ಕೂಲಿ ಸಿಕ್ಕುವುದು. ಈ ಪ್ರಪಂಚದಲ್ಲಿ ಎಲ್ಲದರ ಮೂಲಕವಾಗಿಯೂ ಕೆಲಸ ಮಾಡುತ್ತಿರುವ ಮತ್ತು ಎಲ್ಲದಕ್ಕೂ ಅತೀತವಾದ ಆ ಪರಾತ್ಪರ ತತ್ತ್ವವನ್ನು ತಿಳಿದುಕೊಳ್ಳುತ್ತಾ ಹೋಗುವನು. ಅವನು ಇದನ್ನು ತಿಳಿದುಕೊಳ್ಳಬೇಕು ಎಂಬ ಉದ್ದೇಶದಿಂದ ಕೆಲಸ ಮಾಡಲಿಲ್ಲ. ಸುಮ್ಮನೆ ಕೆಲಸವನ್ನು ಮಾಡುತ್ತ ಹೋದನು. ಅದರ ಪರಿಣಾಮವಾಗಿ ಇವನು ಕೇಳದೆ ಹೋದರೂ ಈ ಜ್ಞಾನ ಬಂತು. ಇಲ್ಲಿ ಕರ್ತವ್ಯ ನಿರ್ವಹಣೆ ಮೊದಲು, ಅನಂತರ ಅದರ ನೆರಳಿನಂತೆ ಜ್ಞಾನ ಬರುವುದು.

ಈ ಸಂಸಾರದಿಂದ ತಪ್ಪಿಸಿಕೊಂಡು ಹೋಗುವುದಕ್ಕೆ ದೇವರು ಆದಿಯಲ್ಲೇ ಈ ಎರಡು ಬಾಗಿಲುಗಳನ್ನು ಮಾಡಿರುವನು.

\begin{verse}
ನ ಕರ್ಮಣಾಮನಾರಂಭಾನ್ನೈಷ್ಕರ್ಮ್ಯಂ ಪುರುಷೋಽಶ್ನುತೇ ।\\ನ ಚ ಸಂನ್ಯಸನಾದೇವ ಸಿದ್ಧಿಂ ಸಮಧಿಗಚ್ಛತಿ \versenum{॥ ೪ ॥}
\end{verse}

{\small ಕರ್ಮಗಳನ್ನು ಆರಂಭಿಸದೇ ಇರುವುದರಿಂದ ಪುರುಷನು ನೈಷ್ಕರ್ಮವನ್ನು ಪಡೆಯುವುದಿಲ್ಲ. ಬರಿಯ ಕರ್ಮ ತ್ಯಾಗದಿಂದಲೇ ಅವನು ಸಿದ್ಧಿಯನ್ನು ಪಡೆಯುವುದಿಲ್ಲ.}

ಕೆಲಸವನ್ನು ಪ್ರಾರಂಭದಲ್ಲಿ ಎಲ್ಲರೂ ಮಾಡಲೇಬೇಕಾಗಿದೆ. ನಮ್ಮಲ್ಲಿ ಹಲವು ವಾಸನೆಗಳಿವೆ. ಆಸೆ ಆಕಾಂಕ್ಷೆಗಳಿವೆ. ಅವುಗಳನ್ನು ಕ್ಷಯ ಮಾಡಿಕೊಳ್ಳಬೇಕಾದರೆ ಕರ್ಮವನ್ನು ಮಾಡಬೇಕು. ಮುಂಚೆ ಪ್ರಚಂಡ ಕರ್ಮದ ಮೂಲಕ ಹೋಗಬೇಕು. ಅನಂತರ ಕರ್ಮ ತಾನಾಗಿ ಬಿದ್ದುಹೋಗುವುದು. ಆದರೆ ನಾವೇ ಬೀಳಿಸುವುದಕ್ಕೆ ಯತ್ನಿಸಿದರೆ ಒಂದು ಕೆಲಸ ಹೋಗುವುದು ಮತ್ತೊಂದು ಕೆಲಸ ಬರುವುದು. ಶ್ರೀ ರಾಮಕೃಷ್ಣರು ಒಂದು ಉದಾಹರಣೆಯನ್ನು ಕೊಡುತ್ತಿದ್ದರು. ಯಾರಾದರೂ ಬಿದ್ದು ಗಾಯ ಮಾಡಿಕೊಂಡರೆ ಗಾಯದ ಸ್ಥಳದಲ್ಲಿ ಒಂದು ಹೊಕ್ಕು ಕಟ್ಟಿಕೊಳ್ಳುವುದು. ಗಾಯ ಮಾಗಿದ ಮೇಲೆ ಅದು ಬಿದ್ದುಹೋಗುವುದು. ಆದರೆ ಮಾಗುವುದಕ್ಕೆ ಮುಂಚೆ ನಾವು ಅದನ್ನು ಕಿತ್ತರೆ ಇನ್ನೊಂದು ಹೊಕ್ಕು ಕಟ್ಟಿಕೊಳ್ಳುವುದು. ಅದರಂತೆಯೇ ತೆಂಗಿನ ಮರ ಬೆಳೆಯುತ್ತಾ ಹೋದಂತೆ ಕೆಳಗಿನ ಗರಿಗಳು ಬಿದ್ದುಹೋಗುವುವು. ಆಧ್ಯಾತ್ಮಿಕ ಜೀವನದಲ್ಲಿ ಒಬ್ಬನು ಮುಂದುವರಿಯುತ್ತ ಹೋದಂತೆ ಅನಾವ ಶ್ಯಕವಾದ ಕರ್ಮಗಳು ಅವನಿಂದ ಬಿದ್ದುಹೋಗುವುವು.

ಅನೇಕ ವೇಳೆ ನಾವು ಕೆಲಸ ಬಿಟ್ಟರೇನೆ ಕೆಲಸ ಮಾಡದೆ ಇರುವ ಸ್ಥಿತಿಗೆ ಬರುತ್ತೇವೆ ಎಂದು ಭಾವಿಸುತ್ತೇವೆ. ಇದೊಂದು ಭ್ರಾಂತಿ. ಕೆಲಸ ಬಿಡುವುದಲ್ಲ ಮುಖ್ಯ. ಅದರ ಹಿಂದೆ ಇರುವ ಆಸೆಯನ್ನು ಬಿಡಬೇಕು. ಆಸೆಯನ್ನು ಬಿಟ್ಟಿಲ್ಲ. ಆದರೆ ಅದನ್ನು ತೃಪ್ತಿಪಡಿಸಿಕೊಳ್ಳಲು ಈಗ ಯತ್ನಿಸುತ್ತಿಲ್ಲ. ಹಾಗಾದರೆ ಅವನ ಮನಸ್ಸೇನೂ ಶಾಂತಿಯಿಂದ ಇರುವುದಿಲ್ಲ. ಮುಂಚೆ ಪ್ರಚಂಡ ಕರ್ಮ ಮಾಡಿ, ಅನಂತರ ನಾವು ಅದರಿಂದ ನಿವೃತ್ತಿವೇತನ ಪಡೆಯಬಹುದು. ಕೆಲಸವನ್ನೇ ಮಾಡದೆ ಹೋದರೆ ನಿವೃತ್ತಿವೇತನ ಹೇಗೆ ಬರುವುದು?

ಕಾಲಕ್ಕೆ ಮುಂಚೆ ಕರ್ಮವನ್ನು ಬಿಟ್ಟರೆ ಅದರಿಂದ ನಾವು ಪಕ್ವವಾಗುವುದಿಲ್ಲ. ಮಾವಿನ ಗಿಡದಿಂದ ಎಳೆಯ ಕಾಯನ್ನು ಕಿತ್ತು ಇಟ್ಟಂತೆ ಇದು. ಇದು ಹಣ್ಣಂತೆ ಕಾಣುವುದೇ ಹೊರತು, ಹಣ್ಣಿನ ರುಚಿಯಾಗಲಿ, ವಾಸನೆಯಾಗಲಿ ಇರುವುದಿಲ್ಲ. ಅದೇ ಬಣ್ಣಕ್ಕೆ ಬಂದ ಬಲಿತ ಕಾಯನ್ನು ಕಿತ್ತು ಇಟ್ಟರೆ ಅದು ಚೆನ್ನಾದ ಹಣ್ಣಾಗುವುದು. ಕರ್ಮದ ಮೂಲಕ ನಮ್ಮ ಚಿತ್ತಶುದ್ಧಿಗೆ ಬೇಕಾದುದನ್ನೆಲ್ಲಾ ಹೀರಿಕೊಂಡು ಬೆಳೆಯಬೇಕು. ಅದರಿಂದ ಬರುವುದನ್ನೆಲ್ಲಾ ಮೊದಲು ಸ್ವೀಕರಿಸಬೇಕು. ಅನಂತರ ಅದು ತಾನಾಗಿ ಮಾಗಿ ಗಿಡದಿಂದ ಕೆಳಗೆ ಬೀಳುವುದು. ಹೀಚೂ ಅದರಿಂದ ಉದುರುವುದು. ಆದರೆ ಇವೆರಡಕ್ಕೂ ಎಷ್ಟೊಂದು ವ್ಯತ್ಯಾಸ. ಕಾದ ಎಣ್ಣೆಗೆ ಏನನ್ನಾದರೂ ಕರಿಯಲು ಹಾಕುತ್ತೇವೆ. ಅದು ತಕ್ಷಣವೇ ಶಬ್ದ ಮಾಡುವುದು. ಬೆಂದಾದ ಮೇಲೆ ತೆಪ್ಪಗಾಗುವುದು. ಆದರೆ ಎಣ್ಣೆಗೆ ಬೀಳುವುದಕ್ಕೆ ಮುಂಚೆಯೂ ಅದು ತೆಪ್ಪಗಿರುವುದು. ಎರಡು ಸ್ಥಿತಿಯೂ ಒಂದೇ ಏನು? ಒಂದು ಕರ್ಮವನ್ನು ಮಾಡಿಲ್ಲ. ಮತ್ತೊಂದು ಕರ್ಮವನ್ನು ಮೀರಿದೆ. ಕರ್ಮವನ್ನು ಮೀರುವ ಸ್ಥಿತಿಗೆ ಹೋಗಬೇಕಾದರೆ ಕರ್ಮದ ಮೂಲಕವಾಗಿ ಮಾತ್ರ ಹೋಗಬೇಕೇ ಹೊರತು ಅದಕ್ಕೆ ವಿಮುಖರಾದರೆ ಸಾಧ್ಯವಿಲ್ಲ.

\begin{verse}
ನ ಹಿ ಕಶ್ಚಿತ್ ಕ್ಷಣಮಪಿ ಜಾತು ತಿಷ್ಠತ್ಯಕರ್ಮಕೃತ್ ।\\ಕಾರ್ಯತೇ ಹ್ಯವಶಃ ಕರ್ಮ ಸರ್ವಃ ಪ್ರಕೃತಿಜೈರ್ಗುಣೈಃ \versenum{॥ ೫ ॥}
\end{verse}

{\small ಯಾರೂ ಒಂದು ಕ್ಷಣವೂ ಕೆಲಸ ಮಾಡದೆ ಇರುವುದಕ್ಕೆ ಆಗುವುದಿಲ್ಲ. ಪ್ರತಿಯೊಬ್ಬನೂ ತನ್ನ ಪ್ರಕೃತಿಯಿಂದ ಹುಟ್ಟುವ ಗುಣಗಳಿಂದ ಪರವಶನಾಗಿ ಕರ್ಮವನ್ನು ಮಾಡುತ್ತಲೇ ಇರುವನು.}

ಹಲವರು ಅನೇಕವೇಳೆ ಕೆಲಸದಲ್ಲಿ ನಿರತರಾಗಿರುವಾಗ, ಅದರಿಂದ ಬೇಜಾರಾಗಿ ಹಾಯಾಗಿರ ಬೇಕೆಂದು ಭಾವಿಸುತ್ತಾರೆ. ಆದರೆ ಅವರೇನಾದರೂ ರಜ ತೆಗೆದುಕೊಂಡರೆ ಮನೆಯಲ್ಲಿ ಮಾತ್ರ ಹಾಯಾಗಿ ಇರುವುದಕ್ಕೆ ಆಗುವುದಿಲ್ಲ. ಆಫೀಸಿನ ಕೆಲಸ ಮಾಡುವುದಿಲ್ಲ. ಆದರೆ ಮನೆಯಲ್ಲೇ ಏನೇನೋ ಕೆಲಸವನ್ನು ಹುಡುಕಿಕೊಂಡು ಹೋಗುತ್ತಾರೆ. ಸದ್ಯಕ್ಕೆ ಇವನ ರಜ ಮುಗಿದರೆ ಸಾಕು, ಈತ ಆಫೀಸಿಗೆ ಹೋದರೆ ಸಾಕು ಎನ್ನಿಸಿಬಿಡುತ್ತದೆ ಮನೆಯವರಿಗೆ. ಏಕೆಂದರೆ ಸುಮ್ಮನೆ ಇರುವುದಕ್ಕೆ ಆಗುವುದಿಲ್ಲ. ಸುಮ್ಮನೆ ಕುಳಿತುಕೊಂಡಿರುವುದನ್ನು ಅವರು ಅಭ್ಯಾಸ ಮಾಡಲಿಲ್ಲ. ಜೀವನದಲ್ಲಿ ಕೆಲಸಕ್ಕೆ ಎಷ್ಟು ತರಬೇತು ಬೇಕೋ ಸುಮ್ಮನೆ ಕುಳಿತುಕೊಂಡಿರುವುದಕ್ಕೆ ಅದಕ್ಕಿಂತ ಹೆಚ್ಚು ತರಬೇತು ಬೇಕು. ಅದನ್ನು ಎಷ್ಚು ಜನ ಪಡೆದಿರುತ್ತಾರೆ? ಏಕೆಂದರೆ ಸುಮ್ಮನೆ ಇರುವುದಕ್ಕೆ ಆಗುವುದಿಲ್ಲ; ಅದಕ್ಕಾಗಿ ಏನನ್ನಾದರೂ ಮಾಡುತ್ತಾ ಇರುವನು. ಈ ಪ್ರಪಂಚದಲ್ಲಿ ಎಲ್ಲವೂ ಚಲಿಸುತ್ತಿದೆ. ಯಾವುದೂ ಸುಮ್ಮನೆ ಇಲ್ಲ. ಸಣ್ಣ ಕಣದಿಂದ ಹಿಡಿದು ಈ ಭೂಮಿ, ನಕ್ಷತ್ರ, ಸೂರ್ಯ, ಚಂದ್ರ ಯಾವುದನ್ನು ತೆಗೆದುಕೊಂಡರೂ ಎಲ್ಲವೂ ಅದ್ಭುತ ವೇಗದಲ್ಲಿ ಚಲಿಸುವುದನ್ನು ನೋಡುತ್ತೇವೆ. ಅವೆಲ್ಲ ಏಕೆ ಚಲಿಸುತ್ತಿವೆ? ಚಲಿಸುವುದು ಸುಮ್ಮನೆ ಇರುವುದಕ್ಕಿಂತ ಸುಲಭ; ಅದಕ್ಕಾಗಿ ಎಲ್ಲವೂ ಚಲಿಸುತ್ತಿವೆ ಎನ್ನುತ್ತಾನೆ ಐನ್​ಸ್ಟೀನ್ ಎಂಬ ವಿಜ್ಞಾನಿ.

ಕೆಲವು ವೇಳೆ ನಾವು ಒಂದು ಕೆಲಸ ಮಾಡಿ ಬೇಜಾರಾಗುತ್ತದೆ. ಅದನ್ನು ಬಿಟ್ಟು ಬೇರೆ ಕೆಲಸ ಮಾಡುತ್ತೇವೆ. ಅಂತೂ ಕೆಲಸ ಮಾಡದೆ ಇರುವುದಕ್ಕೆ ಆಗುವುದಿಲ್ಲ. ಸುಮ್ಮನೆ ಇರಗೊಡಿಸದು ನಮ್ಮಲ್ಲಿರುವ ಪ್ರಕೃತಿ. ಪ್ರಕೃತಿ ಮೂರು ಗುಣಗಳಿಂದ ಆಗಿದೆ. ಅವೇ ತಮಸ್ಸು, ರಜಸ್ಸು, ಸತ್ತ್ವ. ಈ ಮೂರು ಗುಣಗಳೂ ಕೂಡ ಒಬ್ಬನನ್ನು ಕಾರ್ಯೋನ್ಮುಖನನ್ನಾಗಿ ಮಾಡುತ್ತದೆ. ಒಂದಕ್ಕಿಂತ ಮತ್ತೊಂದು ಕೆಲಸ ಒಳ್ಳೆಯದಿರಬಹುದು. ಆದರೆ ಎಲ್ಲವೂ ಕೆಲಸವೇ. ಎಲ್ಲವೂ ಗುಣದ ಕ್ಷೇತ್ರದಲ್ಲಿಯೇ ಇವೆ. ಮೊದಲನೆಯದೆ ತಮೋಗುಣ. ಇಲ್ಲಿ ನಿದ್ರೆ, ಆಲಸ್ಯ, ಮೋಹ, ಮರೆವು, ಅಸಡ್ಡೆ ಇವುಗಳೆಲ್ಲ ಇವೆ. ಯಾವಾಗಲೂ ಒಬ್ಬ ನಿದ್ರೆ ಮಾಡುವುದಕ್ಕೆ ಆಗುವುದಿಲ್ಲ. ಸುಮ್ಮನೆ ಏನನ್ನೋ ಮಾಡುತ್ತಾ ಇರುತ್ತಾನೆ. ಇದರಿಂದ ಅವನಿಗೆ ಪ್ರಯೋಜನವಿಲ್ಲ. ಇತರರಿಗೆ ಪ್ರಯೋಜನ ವಿಲ್ಲ. ಅನಾವಶ್ಯಕವಾಗಿರುವುದನ್ನು ಮಾಡುತ್ತಾನೆ. ಒಬ್ಬ ಈ ಗುಂಪಿಗೆ ಸೇರಿದವನು ಬೆಳಗ್ಗೆ ಮನೆಯಿಂದ ತಿಂಡಿ ತಿಂದುಕೊಂಡು ಹೋಗುತ್ತಿದ್ದ. ಆ ಊರಿನಲ್ಲಿ ಕುಂಬಾರಕೇರಿಯಲ್ಲಿ ಒಂದು ಆವಿಗೆಯ ಮನೆ. ಅಲ್ಲಿ ಒಡೆದುಹೋದ ಮಡಿಕೆಗಳನ್ನೆಲ್ಲಾ ರಾಶಿಯಾಗಿ ಇಡುತ್ತಿದ್ದರು. ಈತ ಒಂದು ಕೋಲಿನಿಂದ ಅವುಗಳನ್ನೆಲ್ಲಾ ಒಡೆದು ಸುಸ್ತಾಗಿ ಬೆವರು ಸುರಿಸುತ್ತ ಊಟದ ಹೊತ್ತಿಗೆ ಬರುತ್ತಿದ್ದ. ಇವತ್ತು ತುಂಬಾ ಕೆಲಸ ಮಾಡಿದೆ ಎಂದು ಅವನು ಭಾವಿಸುತ್ತಾನೆ! ನಮ್ಮಲ್ಲಿ ಅನೇಕರು ಈ ಮಡಕೆ ಒಡೆಯುವವರ ಗುಂಪಿಗೆ ಸೇರಿದವರು. ಸುಮ್ಮನೆ ಕಾಲವನ್ನು ಕೊಲ್ಲುತ್ತೇವೆಯೇ ಹೊರತು ಅದರಿಂದ ಒಳ್ಳೆಯ ಪ್ರಯೋಜನವಿಲ್ಲ. ನನಗಲ್ಲ, ಮನೆಗಲ್ಲ, ದೇಶಕ್ಕಲ್ಲ. ದೇಶದಲ್ಲಿರುವ ಜೊಳ್ಳು ಜನಾಂಗ ಇದು. ಕೆಲಸವನ್ನೇನೋ ಮಾಡುತ್ತಾನೆ. ಯೋಗ್ಯವಾದುದನ್ನು ಮಾಡುವುದಿಲ್ಲ. ಸಕಾಲದಲ್ಲಿ ಮಾಡುವುದಿಲ್ಲ.

ಎರಡನೆಯ ಗುಂಪಿನವನೆ ರಜೋಗುಣಿ. ಅವನು ಪಾದರಸದಂತೆ ಚುರುಕು. ಯಾವಾಗಲೂ ಏನನ್ನಾದರೂ ಮಾಡುತ್ತ ಇರುವನು. ಇದರಿಂದ ಅವನಿಗೆ ಪ್ರಯೋಜನವಾಗಬೇಕು, ಲಾಭ ಬರ ಬೇಕು, ಇದಕ್ಕಾಗಿ ಹಗಲು ರಾತ್ರಿ ದುಡಿಯುತ್ತ ಇರುವನು. ಅವನು ತೆಪ್ಪಗೆ ಇರುವ ಸಮಯವೇ ಇಲ್ಲ. ಸ್ವಾರ್ಥವೋ, ಪರಹಿತವೋ ಅಂತೂ ಯಾವುದಾದರೂ ಉದ್ದೇಶದಿಂದ ಏನನ್ನಾದರೂ ಮಾಡುತ್ತಲೇ ಇರುತ್ತಾನೆ. ಹೀಗೆ ಕೆಲಸ ಮಾಡುವಾಗ ಅನೇಕ ವೇಳೆ ವ್ಯಥೆಯನ್ನು ಅನುಭವಿಸುತ್ತಾನೆ. ಕಷ್ಟವನ್ನೂ ಅನುಭವಿಸುವನು. ಸ್ವಲ್ಪ ಕಾಲ ನಾನು ಇನ್ನೇನನ್ನೂ ಮಾಡುವುದಿಲ್ಲ ಎಂದು ಸಂಕಲ್ಪ ಮಾಡುತ್ತಾನೆ. ತನಗೆ ಗೊತ್ತಾಗದೆ ಅವನು ಆಗಲೆ ಇನ್ನಾವುದೋ ಒಂದರಲ್ಲಿ ಮುಳುಗಿರುವನು. ಆನೆಯ ಮೇಲೆ ಕುಳಿತುಕೊಂಡ ಮಾಹುತ ಅದನ್ನು ಅಂಕುಶದಿಂದ ಹೇಗೆ ತಿವಿಯುತ್ತಿರುವನೋ ಹಾಗೆ ರಜೋಗುಣ ಪ್ರವೃತ್ತಿ ಇವನನ್ನು ತಿವಿಯುತ್ತಿರುವುದು. ಇವನು ಸುಮ್ಮನೆ ಇರಲಾರ. ಇನ್ನೊಬ್ಬರು ಕಚಗುಳಿ ಇಟ್ಟರೆ ನಾವು ನಗಲೇಬೇಕಾಗುವುದು. ಅದರಂತೆಯೇ ರಜೋಗುಣ. ಇವನ ಮೂಲಕ ಕೆಲಸ ಮಾಡಿಸುತ್ತಿರುವುದು.

ಮೂರನೆಯವನು ಇರುವನು. ಇವನೇ ಸತ್ತ್ವಗುಣಿ. ಗುಣಗಳಲ್ಲೆಲ್ಲಾ ಶ್ರೇಷ್ಠವಾದುದು ಇದು. ಅವನು ಮೇಧಾವಿ, ಒಳ್ಳೆಯ ಆಲೋಚನೆಗಳನ್ನು ಮಾಡುತ್ತಾನೆ. ಒಳ್ಳೆಯ ಭಾವನೆಗಳು ಅವನಲ್ಲಿ ಇವೆ. ಭಗವತ್​ಭಕ್ತ, ಈ ಪ್ರಪಂಚದ ಮೇಲೆ ಅನುಕಂಪ. ಇಲ್ಲಿರುವ ಜನರನ್ನು ಕಂಡರೆ ಪ್ರೀತಿ, ಕಷ್ಟದಲ್ಲಿರುವವರನ್ನು ಕಂಡರೆ ಹೃದಯ ಕರಗುವುದು. ಆನಂದಪಡುವವರನ್ನು ನೋಡಿ ಅವರ ಆನಂದದಲ್ಲಿ ಭಾಗಿಯಾಗುವನು. ನನಗೆ ಬರಲಿಲ್ಲ ಎಂದು ಕರುಬುವುದಿಲ್ಲ. ದುಃಖದಲ್ಲಿರುವವರನ್ನು ನೋಡಿ ಅವರಿಗೆ ಸಾಂತ್ವನ ಹೇಳುತ್ತಾನೆ. ತಾನು ಬದುಕಿರುವವರೆಗೆ ಇತರರಿಗೆ ಒಳ್ಳೆಯದನ್ನು ಮಾಡುವನು. ಹೆಚ್ಚಾಗಿ ಭಗವಂತನಿಗೆ, ಅವನ ಮಕ್ಕಳಿಗೆ ತನ್ನಲ್ಲಿರುವುದನ್ನೆಲ್ಲಾ ಕೊಡಬೇಕು, ಅತ್ಯಂತ ಕಡಿಮೆ ಸ್ವೀಕರಿಸಬೇಕು ಎಂಬ ನಿಯಮದ ಮೇಲೆ ಅವನು ಜೀವನವನ್ನು ರೂಢಿಸುವನು. ಇಂತಹ ವ್ಯಕ್ತಿಯೂ ಕೂಡ ಆತ್ಮಕಲ್ಯಾಣ ಮತ್ತು ಪರಹಿತ ಇವುಗಳಿಂದ ಪ್ರೇರಿತನಾಗಿ ಯಾವುದಾದರೂ ಒಂದು ಕೆಲಸದಲ್ಲಿ ನಿರತನಾಗಿರುವನು. ಅವನು ಮಾಡುವ ಕೆಲಸದ ಹಿಂದೆ ಉದ್ವೇಗವಾಗಲಿ, ಗಡಿಬಿಡಿಯಾಗಲಿ ಇಲ್ಲ. ಕೀರ್ತಿ ಅಧಿಕಾರ ಲಾಭದ ಆಸೆ ಇಲ್ಲ. ಭಗವತ್ ಪ್ರೀತ್ಯರ್ಥವಾಗಿ ತನ್ನ ಕಾಲವನ್ನು ಕಳೆಯುತ್ತಿರುವನು. ಆದರೂ ಇದೂ ಕೂಡ ಕಾರ್ಯದೃಷ್ಟಿಯಿಂದ ನೋಡಿದರೆ ಕರ್ಮವೆ. ತಮೋಗುಣಿ ಕೆಲಸಕ್ಕೆ ಬರದೆ ಇರುವುದನ್ನು ಮಾಡುತ್ತಾನೆ. ರಜೋಗುಣಿ ಸ್ವಾರ್ಥದಿಂದ ಮಾಡು ತ್ತಾನೆ. ಸತ್ತ್ವಗುಣಿ ನಿಃಸ್ವಾರ್ಥನಾಗಿ ಮಾಡುತ್ತಾನೆ.

\begin{verse}
ಕರ್ಮೇಂದ್ರಿಯಾಣಿ ಸಂಯಮ್ಯ ಯ ಆಸ್ತೇ ಮನಸಾ ಸ್ಮರನ್ ।\\ಇಂದ್ರಿಯಾರ್ಥಾನ್ ವಿಮೂಢಾತ್ಮಾ ಮಿಥ್ಯಾಚಾರಃ ಸ ಉಚ್ಯತೇ \versenum{॥ ೬ ॥}
\end{verse}

{\small ಯಾರು ಕರ್ಮೇಂದ್ರಿಯಗಳನ್ನು ನಿಗ್ರಹಿಸಿ ಮನಸ್ಸಿನಲ್ಲಿ ಇಂದ್ರಿಯ ವಿಷಯವನ್ನು ಸ್ಮರಿಸುತ್ತಿರುವನೋ ಅವನು ವಿಮೂಢಾತ್ಮ ಮತ್ತು ಮಿಥ್ಯಾಚಾರಿ ಎನಿಸುವನು.}

ಒಬ್ಬ ಕರ್ಮೇಂದ್ರಿಯಗಳನ್ನು ನಿಗ್ರಹಿಸುತ್ತಾನೆ. ಆದರೆ ಮನಸ್ಸಿನ ಒಳಗೆ ಅದಕ್ಕೆ ಸಂಬಂಧಪಟ್ಟ ಆಸೆಗಳನ್ನು ಮೆಲ್ಲುತ್ತಿರುವನು. ಇಂಥವನು ವಿಮೂಢಾತ್ಮ. ಏಕೆಂದರೆ ಇವನ ಮನಸ್ಸು ಆ ಕೆಲಸ ಮಾಡುತ್ತ ಇದೆ. ಇದಕ್ಕೆ ಸಂಬಂಧಪಟ್ಟ ಸಂಸ್ಕಾರ, ಮನಸ್ಸಿನಲ್ಲಿ ಉತ್ಪತ್ತಿ ಆಗುತ್ತದೆ. ಇವನು ಮಾಡುವುದು ಹೊರಗೆ ಕಾಣಿಸುವುದಿಲ್ಲ. ಪೋಲೀಸಿನವನಾಗಲಿ, ಜಡ್ಜಿಯಾಗಲಿ ಹಿಡಿಯುವಂತೆ ಇಲ್ಲ. ಆದರೆ ಸೂಕ್ಷ ್ಮವಾಗಿ ಆ ಕರ್ಮವನ್ನು ಮಾಡುತ್ತಿರುವನು. ಅದನ್ನು ನಿಗ್ರಹಿಸದೆ ಹೋದರೆ ಸ್ಥೂಲ ವಾಗಿಯೂ ಆ ಕಾರ್ಯವನ್ನು ಮಾಡುವ ಸ್ಥಿತಿಗೆ ಇವನ ಮನಸ್ಸು ಬರುವುದು. ಮನಸ್ಸಿನಲ್ಲಿ ಮಾಡಿದರೆ ಪಾಪ ಇಲ್ಲ, ಹೊರಗೆ ಮಾಡಿದರೆ ಮಾತ್ರ ಪಾಪ ಎಂದು ಭಾವಿಸುವುದು ತಪ್ಪು. ಇದು ಬಾಹ್ಯ ಪ್ರಪಂಚಕ್ಕೆ ಅನ್ವಯಿಸುವುದು. ಆದರೆ ಆಧ್ಯಾತ್ಮಿಕ ದೃಷ್ಟಿಯಿಂದ ಮನಸ್ಸಿನಲ್ಲಿ ಮಾಡುವುದೂ ಒಂದು ಮೈಲಿಗೆಯೆ. ಇದು ಮೈಲಿಗೆ ಎಂದು ಭಾವಿಸದೆ ಇದ್ದರೆ ಅವನು ಮೂಢ. ಒಬ್ಬ ಮಾಡುವ ತಪ್ಪು ಸಮಾಜಕ್ಕೆ ಗೊತ್ತಾಗಿದೆ. ಮತ್ತೊಬ್ಬ ಮಾಡುವ ತಪ್ಪು ಸಮಾಜಕ್ಕೆ ಗೊತ್ತಾಗಿಲ್ಲ. ಆದರೆ ದೇವರ ಎದುರಿಗೆ ಇಬ್ಬರೂ ತಪ್ಪಿತಸ್ಥರೇ. ಹಿಂದಿನ ಕಾಲದಲ್ಲಿ ಹೆಬ್ರು ದೇಶದಲ್ಲಿ ಒಂದು ಆಚಾರವಿತ್ತು. ಅಲ್ಲಿ ಯಾವ ಹೆಂಗಸಾದರೂ ಕಳಂಕ ಜೀವನವನ್ನು ನಡೆಸಿದರೆ ಅದು ಜನಗಳಿಗೆ ಗೊತ್ತಾದರೆ, ಅವಳನ್ನು ಎಲ್ಲರೂ ಕಾಣುವ ಕಡೆ ನಿಲ್ಲಿಸಿ ಎಲ್ಲರೂ ತಮ್ಮ ಕೈಯಿಂದ ಕಲ್ಲನ್ನು ತೆಗೆದುಕೊಂಡು ಅವಳನ್ನು ಹೊಡೆಯುವುದು ಆ ದೇಶದ ಆಚಾರವಾಗಿತ್ತು. ಒಂದು ಸಲ ಒಬ್ಬ ತಬ್ಬಲಿ ಹುಡುಗಿಯನ್ನು ಜನರೆಲ್ಲಾ ಸೇರಿಕೊಂಡು ಹೊಡೆಯುತ್ತಿದ್ದರು. ಕ್ರೆ ೈಸ್ತ ಆ ದಾರಿಯಲ್ಲಿ ಹೋಗುತ್ತಿದ್ದನು. ಅವನು ಈ ಸಮಾಚಾರವನ್ನು ಕೇಳಿದ. ಆತ ತಕ್ಷಣವೇ ಅಲ್ಲಿ ನೆರೆದ ಮಂದಿಗೆ ಹೀಗೆ ಹೇಳಿದ: ನಿಮ್ಮಲ್ಲಿ ಯಾರು ಮನಸ್ಸಿನಲ್ಲಿಯೂ ಒಂದು ಅಪರಾಧವನ್ನು ಮಾಡಿಲ್ಲವೋ ಅವನು ಮುಂದೆ ಬರಲಿ ಕಲ್ಲನ್ನು ಎಸೆಯುವುದಕ್ಕೆ. ಎಲ್ಲರೂ ಹಿಂದೆ ಸರಿದು ನಿಂತರು. ಯಾರಿಗೂ ಧೈರ್ಯವಿರಲಿಲ್ಲ ತಾವು ಮನಸ್ಸಿನಲ್ಲಿಯೂ ಒಂದು ಅಪರಾಧವನ್ನು ಮಾಡಿಲ್ಲ ಎಂದು ಹೇಳಿಕೊಳ್ಳುವುದಕ್ಕೆ. ಅನೇಕರು ಬಾಹ್ಯದಲ್ಲಿಯೇ ಎಷ್ಟೋ ತಪ್ಪನ್ನು ಮಾಡಿರುತ್ತಾರೆ. ಆದರೆ ಅವರು ಸಿಕ್ಕಿಕೊಂಡಿರುವುದಿಲ್ಲ. ತಪ್ಪನ್ನು ಮಾಡಿದವರನ್ನೆಲ್ಲ ಹಿಡಿದು ಸಮಾಜದಲ್ಲಿ ಶಿಕ್ಷಿಸುತ್ತಿಲ್ಲ. ಹಾಗೆ ತಪ್ಪು ಮಾಡಿದವರಲ್ಲಿ ಎಲ್ಲೋ ಕೆಲವರು ಸಿಕ್ಕಿಕೊಳ್ಳುವರು.

ಒಳಗೆ ಒಂದು ಹೊರಗೆ ಒಂದು ಇರುವವರನ್ನು ಶ್ರೀಕೃಷ್ಣ ಮಿಥ್ಯಾಚಾರಿ ಎನ್ನುವನು. ಎಲ್ಲಾ ಹೊಲಸು ಯೋಚನೆಗಳು ಮನಸ್ಸಿನಲ್ಲಿದ್ದರೂ ಕೂಡ ದೊಡ್ಡ ಸಾಧು ಪುರುಷನಂತೆ ಸೋಗು ಹಾಕುತ್ತಿರುವನು. ಅವನು ಹೊರಗಿನ ಸಮಾಜವನ್ನು ವಂಚಿಸುವನು. ಅದಕ್ಕಿಂತ ಹೆಚ್ಚಾಗಿ ತನ್ನನ್ನು ವಂಚಿಸಿಕೊಳ್ಳುತ್ತಿರುವನು. ಪಾಪಿಗಾದರೂ ದೇವರ ಕಡೆ ಹೋಗುವುದಕ್ಕೆ ಸಾಧ್ಯ. ಆದರೆ ಇಂತಹ ಮಿಥ್ಯಾಚಾರಿಗೆ ಅಸಾಧ್ಯ. ಶ್ರೀಕೃಷ್ಣ ಇಲ್ಲಿ ಹೇಳುವುದು ಮನಸ್ಸಿನ ಆಸೆಗಳನ್ನು ಯಾವಾಗ ಗೆಲ್ಲುವುದಕ್ಕೆ ಆಗುವುದಿಲ್ಲವೋ ಅದನ್ನು ಬಾಹ್ಯ ರೀತಿಯಲ್ಲಿ ಧಾರ್ಮಿಕವಾಗಿ ಮಾಡಿ ತೃಪ್ತಿಪಡಿಸಿಕೊಳ್ಳಲಿ ಎಂದು. ಅದೇನೂ ಪಾಪವಲ್ಲ. ಅದಕ್ಕೇ ನಮ್ಮ ಶಾಸ್ತ್ರಗಳು, ಧರ್ಮ, ಅರ್ಥ, ಕಾಮ, ಮೋಕ್ಷಗಳನ್ನು ಪುರುಷಾರ್ಥ ಎಂದು ಸಾರುವುದು. ಹಣದ ಮೇಲೆ ಆಸೆ ಇದ್ದರೆ ಕಷ್ಟಪಟ್ಟು ಸಂಪಾದನೆ ಮಾಡಿ ಅನುಭವಿಸಲಿ. ಆದರೆ ಸುಳ್ಳು ಹೇಳದೆ ಮೋಸ ಮಾಡದೆ ಇರಲಿ. ಅದರಂತೆಯೇ ಕಾಮದ ಬಯಕೆ ಇದ್ದರೆ, ಅದನ್ನು ಸ್ವಾಭಾವಿಕ ರೀತಿಯಲ್ಲಿ ತೃಪ್ತಿಪಡಿಸಿಕೊಳ್ಳಲಿ. ಹಾಗೆ ಮಾಡುವಾಗ ಸಮಾಜಕ್ಕೆ ವಿರೋಧವಾದ ರೀತಿಯಲ್ಲಿ ಹೋಗಬಾರದು. ಯಾವಾಗಲೂ ಅಧರ್ಮವನ್ನು ಆಶ್ರಯಿಸಬಾರದು. ಧರ್ಮದ ಮೂಲಕವಾಗಿಯೇ ಕಾಮ ಮತ್ತು ಅರ್ಥವನ್ನು ತೃಪ್ತಿಪಡಿಸಿಕೊಳ್ಳಬಹುದು. ಈ ಅನು ಭವದ ಯಂತ್ರದ ಮೂಲಕ ಹೋದ ಮೇಲೆಯೇ ಈ ಕಿಲುಬು ಸುಟ್ಟುಹೋಗುವುದು. ಅದಿದ್ದರೂ ಇಲ್ಲ ಎಂದು ಹೇಳಿಕೊಂಡು ಮನಸ್ಸಿನಲ್ಲಿಯೇ ಅದನ್ನು ಕಬಳಿಸುತ್ತಿದ್ದರೆ ಈ ಆಸೆಯಿಂದ ಪಾರಾಗು ವುದಕ್ಕೆ ಆಗುವುದಿಲ್ಲ. ಆಸೆಯನ್ನು ನಿಗ್ರಹಿಸುವ ಸಾತ್ತ್ವಿಕ ಶಕ್ತಿ ಇದ್ದರೆ ಹಾಗೆ ಮಾಡಲಿ. ಅವರು ಸರ್ವೋತ್ಕೃಷ್ಟ. ಹಾಗೆ ಮಾಡಲು ಸಾಧ್ಯವಿಲ್ಲದೆ ಇದ್ದರೆ, ಅದನ್ನು ಧಾರ್ಮಿಕವಾಗಿ ತೃಪ್ತಿಪಡಿಸಿ ಕೊಳ್ಳಲಿ. ಇವನು ಮಧ್ಯಮ. ಅದನ್ನು ತೃಪ್ತಿಪಡಿಸಿಕೊಳ್ಳಲು ಅಧರ್ಮದ ಹಾದಿಯನ್ನು ಹಿಡಿಯು ವವನು ಅಧಮ. ಕೇವಲ ಮನಸ್ಸಿನಲ್ಲಿಯೇ ಮೆಲ್ಲುವವನು ಅಧಮಾಧಮನು. ಅವನಿನ್ನೂ ಆ ಗುಹೆಯಿಂದ ಹೊರಗೆ ಬರಬೇಕಾಗಿದೆ.

\begin{verse}
ಯಸ್ತ್ವಿಂದ್ರಿಯಾಣಿ ಮನಸಾ ನಿಯಮ್ಯಾರಭತೇಽಜುRನ ।\\ಕರ್ಮೇಂದ್ರಿಯೈಃ ಕರ್ಮಯೋಗಮಸಕ್ತಃ ಸ ವಿಶಿಷ್ಯತೇ \versenum{॥ ೭ ॥}
\end{verse}

{\small ಆದರೆ ಅರ್ಜುನ, ಯಾರು ಇಂದ್ರಿಯವನ್ನು ಮನಸ್ಸಿನಿಂದ ತಡೆದು, ಕರ್ಮೇಂದ್ರಿಯಗಳ ಮೂಲಕ ಕರ್ಮವನ್ನು ಅನಾಸಕ್ತನಾಗಿ ಮಾಡುತ್ತಿರುವನೋ ಅವನು ಅತಿಶಯನು.}

ಎರಡನೆಯವನು ಆಸೆಯನ್ನು ಮನಸ್ಸಿನಲ್ಲಿಯೇ ಗೆಲ್ಲುತ್ತಾನೆ. ಆದರೆ ಕರ್ಮೇಂದ್ರಿಯಗಳ ಮೂಲಕ ಆವಶ್ಯಕವಾದ ಕರ್ಮಗಳನ್ನು ಮಾಡುತ್ತಾನೆ. ಹಾಗೆ ಕರ್ಮ ಮಾಡುವಾಗ ಅನಾಸಕ್ತ ನಾಗಿರುವನು. ಬರುವ ಫಲಾಫಲಗಳಿಗೆ ಅಂಟಿಕೊಂಡಿರುವುದಿಲ್ಲ. ಇವನು ಕೆಲಸವನ್ನು ಮಾಡುತ್ತ ಇದ್ದರೂ ಅದು ಮನಸ್ಸಿನ ಮೇಲೆ ಯಾವ ಪರಿಣಾಮವನ್ನೂ ಉಂಟುಮಾಡುವುದಿಲ್ಲ. ಇವನಲ್ಲಿ ಒಳಗೆ ಒಂದು ಹೊರಗೆ ಒಂದು ಇಲ್ಲ. ಇವನು ಒಳಗೆ ಒಂದು ಹೊರಗೆ ಒಂದು ಇರುವ ಮಿಥ್ಯಾಚಾರಿಗಿಂತ ಶ್ರೇಷ್ಠ. ಇವನು ಮಾಡುವ ಕರ್ಮ ಕರ್ಮಯೋಗವಾಗಿರುವುದು. ಬರೀ ಕರ್ಮ ಮಾಡಿದರೆ ಅದು ನಮ್ಮನ್ನು ಫಲಗಳಿಗೆ ಕಟ್ಟಿಹಾಕುವುದು. ಆದರೆ ಅದನ್ನು ಯೋಗದೃಷ್ಟಿಯಿಂದ ಮಾಡಿದರೆ ಯಾವ ಬಂಧನವೂ ಇರುವುದಿಲ್ಲ. ಯೋಗವೆಂಬ ಬೆಂಕಿ ಕರ್ಮದಲ್ಲಿರುವ ನಮ್ಮನ್ನು ಕಟ್ಟಿಹಾಕುವ ಸ್ವಭಾವವನ್ನು ನಾಶಮಾಡುವುದು. ಬೆಂಕಿಯ ಮೇಲೆ ಹುರಿದ ಬೀಜ ಹೇಗೆ ಚಿಗುರುವ ಸ್ವಭಾವವನ್ನು ಕಳೆದು ಕೊಳ್ಳುವುದೋ ಹಾಗೆ ಆಗುವುದು.

\begin{verse}
ನಿಯತಂ ಕುರು ಕರ್ಮ ತ್ವಂ ಕರ್ಮ ಜ್ಯಾಯೋ ಹ್ಯಕರ್ಮಣಃ ।\\ಶರೀರಯಾತ್ರಾಪಿ ಚ ತೇ ನ ಪ್ರಸಿಧ್ಯೇದಕರ್ಮಣಃ \versenum{॥ ೮ ॥}
\end{verse}

{\small ನೀನು ನಿಯತವಾದ ಕರ್ಮವನ್ನು ಮಾಡು. ಏಕೆಂದರೆ ಕರ್ಮ ಅಕರ್ಮಕ್ಕಿಂತ ಮೇಲು. ಅಕರ್ಮಿಯಾಗಿದ್ದರೆ ಜೀವನದಲ್ಲಿ ಬದುಕಿರುವುದೂ ಕಷ್ಟವಾಗುವುದು.}

ನೀನು ನಿಯತವಾದ ಕರ್ಮವನ್ನು ಮಾಡಬೇಕು ಎಂದು ಶ್ರೀಕೃಷ್ಣ ಅರ್ಜುನನಿಗೆ ಹೇಳುತ್ತಾನೆ. ಎರಡು ಬಗೆಯ ಕರ್ಮಗಳಿವೆ. ಒಂದು ನಿಯತಕರ್ಮ. ಮತ್ತೊಂದು ಕಾಮ್ಯಕರ್ಮ. ಬೇಕಾದರೆ ನಾವು ಕಾಮ್ಯಕರ್ಮವನ್ನು ತ್ಯಜಿಸಬಹುದು. ಸ್ವರ್ಗಕ್ಕೆ ಹೋಗುವುದಕ್ಕೆ, ಮಕ್ಕಳನ್ನು ಪಡೆಯುವುದಕ್ಕೆ ಮುಂತಾದುವುಗಳಿಗೆ ಬೇಕಾದಷ್ಟು ಯಾಗ ಯಜ್ಞಗಳಿವೆ. ಮತ್ತು ಇಹಲೋಕದಲ್ಲಿ ನಾನು ಕೀರ್ತಿ ಪಡೆಯಬೇಕಾದರೆ, ಅಧಿಕಾರವನ್ನು ಪಡೆಯಬೇಕಾದರೆ, ಅದಕ್ಕೆಂದೇ ಬೇಕಾದಷ್ಟು ಕರ್ಮಗಳಿವೆ. ಅದನ್ನು ಬಿಟ್ಟರೆ ನನಗೆ ಮಾತ್ರ ಅದರಿಂದ ನಷ್ಟ. ಅದರಿಂದ ಸಮಾಜಕ್ಕೆ ಹಾನಿಯಿಲ್ಲ. ನಿಯತಕರ್ಮ ಪ್ರತಿ ವರ್ಣ ಮತ್ತು ಆಶ್ರಮದವರು ಮಾಡಲೇಬೇಕಾದ ಕರ್ಮಗಳು. ಅದನ್ನು ಬಿಡುವುದಕ್ಕೆ ಆಗುವುದಿಲ್ಲ. ಇಡೀ ಸಮಾಜ ನಿಂತಿರುವುದೇ ಇದರ ಮೇಲೆ. ಇದರಿಂದ ನಮಗೆ ಯಾವ ಲಾಭ ಆಗದೇ ಇದ್ದರೂ ಈ ಕರ್ತವ್ಯಗಳನ್ನು ನಾವು ನೆರವೇರಿಸಬೇಕು. ಯಾವಾಗ ನಾವೊಂದು ಸಮಾಜದಲ್ಲಿ ಹುಟ್ಟಿರುವೆವೋ ಅದರಿಂದ ನಮಗೆ ಬರುವುದನ್ನೆಲ್ಲಾ ಹೀರಿಕೊಂಡು ನಾವು ಬೆಳೆದಿರುವೆವು. ಈಗ ನಾವು ಆ ಸಮಾಜ ಚೆನ್ನಾಗಿರುವುದಕ್ಕೆ ಏನು ಬೇಕೊ ಅದನ್ನು ಮಾಡುವುದು ನಮ್ಮ ಒಂದು ಮುಖ್ಯ ಕರ್ತವ್ಯ. ಯಾವಾಗ ನಾವು ಇದನ್ನು ಮಾಡುವುದಿಲ್ಲವೋ ಆಗ ನಾವು ಮಹಾಪಾಪ ಮಾಡಿದಂತೆ ಆಗುವುದು.

ಅಕರ್ಮಕ್ಕಿಂತ ಕರ್ಮ ಉತ್ತಮ ಎನ್ನುವನು. ನಮ್ಮಲ್ಲಿ ಬೇಕಾದಷ್ಟು ವಾಸನೆಗಳು ಇವೆ. ಅವು ಕ್ಷಯವಾದಲ್ಲದೆ ಚಿತ್ತಶುದ್ಧಿಯಾಗುವುದಿಲ್ಲ. ಮುಂಚೆ ನಮ್ಮ ಒಳಗಿರುವ ಆಸೆಗಳು ಮೇಲೇಳಬೇಕು. ಅನಂತರ ಆ ವಾಸನೆಗಳಿಗೆ ವಿರೋಧವಾಗಿರುವ ಕರ್ಮಗಳನ್ನು ಮಾಡಬೇಕು. ಆಗಲೆ ಕರ್ಮದಿಂದ ಪಾರಾಗಲು ಸಾಧ್ಯ. ಆಸೆ ಮೇಲೇಳಬೇಕಾದರೆ ಅದನ್ನು ಕರ್ಮದಿಂದ ಕಡೆಯಬೇಕು. ಮೊಸರಿನಲ್ಲಿ ಬೆಣ್ಣೆ ಇದೆ. ಆ ಬೆಣ್ಣೆ ಮೊಸರಿನ ಮೇಲೆ ಏಳಬೇಕಾದರೆ ಕಡೆಯುವ ಕೋಲಿನಿಂದ ಕಡೆಯಬೇಕು. ಆಗ ಮಾತ್ರ ಮೇಲೆ ತೇಲುವುದು. ಅನಂತರ ಅದನ್ನು ತೆಗೆದುಹಾಕಬಹುದು. ಅದರಂತೆಯೇ ನಾವು ಕರ್ಮವನ್ನು ಮಾಡಿದಾಗಲೆ ನಮ್ಮಲ್ಲಿರುವ ನ್ಯೂನತೆಗಳು ಮೇಲೆದ್ದು ಬರಬೇಕಾದರೆ. ಅನಂತರ ಅಲ್ಲಿ ಬಂಧನಕ್ಕೆ ಒಳಗುಮಾಡುವ ಪ್ರತಿಯೊಂದು ವಾಸನೆಗೂ ಅದಕ್ಕೆ ವಿರೋಧವಾದ ಒಳ್ಳೆಯ ಭಾವನೆ ಯಿಂದ ಕರ್ಮ ಮಾಡಬೇಕು. ಇದು ಹೊರಗಿನಿಂದ ಒಂದು ಮುಳ್ಳನ್ನು ತೆಗೆದುಕೊಂಡು ಕಾಲಿನ ಲ್ಲಿರುವ ಮತ್ತೊಂದು ಮುಳ್ಳನ್ನು ತೆಗೆದುಹಾಕುವಂತೆ.

ಒಳ್ಳೆಯ ದೃಷ್ಟಿಯಿಂದ ಕರ್ಮವನ್ನು ಮಾಡುವುದಕ್ಕೆ ಸಾಧ್ಯವಾಗದೇ ಇದ್ದರೂ ಚಿಂತೆಯಿಲ್ಲ. ಸುಮ್ಮನೆ ಇರುವುದಕ್ಕಿಂತ ಲಾಭಕ್ಕೋ, ಕೀರ್ತಿಗೋ, ಅಂತೂ ಯಾವುದಾದರೂ ಒಂದು ಉದ್ದೇಶ ದಿಂದ ಮಾಡುವುದಾದರೂ ಮೇಲು. ತಮೋಗುಣದಲ್ಲಿರುವವನು ಒಂದೇ ಸಲ ಸತ್ತ್ವದೃಷ್ಟಿಯಿಂದ ಕೆಲಸವನ್ನು ಮಾಡಲು ಕಲಿಯಲು ಆಗುವುದಿಲ್ಲ. ರಜೋಗುಣದಿಂದಲಾದರೂ ಪ್ರೇರೇಪಿತನಾಗಿ ಕರ್ಮ ಮಾಡಲಿ. ಇದರಿಂದ ಅವನಿಗೆ ಪ್ರಯೋಜನ. ಜೊತೆಗೆ ಇದರಿಂದ ಸಮಾಜಕ್ಕೆ ಲಾಭ ವಾಗುವುದು. ಸಮಷ್ಟಿಜೀವನವೇ ಸಮಾಜ. ಪ್ರತಿಯೊಬ್ಬನೂ ಸಮಷ್ಟಿಜೀವನಕ್ಕೆ ತನ್ನ ಪಾಲಿನ ಕರ್ಮವನ್ನು ಧಾರೆಯೆರೆಯಬೇಕು. ನಿಃಸ್ವಾರ್ಥನಾಗಿ ಮಾಡಲು ಸಾಧ್ಯವಾಗದೇ ಇದ್ದರೆ ಸ್ವಾರ್ಥನಾಗಿ ಆದರೂ ಮಾಡಲಿ. ಒಬ್ಬ ರೈತ ಬೆಳೆಯುತ್ತಾನೆ. ಇದನ್ನು ಅವನು ಸ್ವಾರ್ಥ ದೃಷ್ಟಿಯಿಂದ ಮಾಡ ಬಹುದು. ಆದರೇನು, ಇದರಿಂದ ಬೆಳೆಯದೆ ಇರುವವರಿಗೆ ಕಾಳು ಸಿಕ್ಕಿದಂತೆ ಆಗಲಿಲ್ಲವೆ? ಸ್ವಾರ್ಥದ ದೃಷ್ಟಿಯಿಂದ ನಾವು ಮಾಡುವುದೂ ಸಮಷ್ಟಿಯ ಹಿತಕ್ಕೆ ಆಗುವುದು. ಯಾವನು ಸ್ವಾರ್ಥದೃಷ್ಟಿಯನ್ನು ಬಿಟ್ಟು ಮಾಡುತ್ತಾನೆಯೋ ಅದು ಶ್ರೇಷ್ಠ. ಸ್ವಾರ್ಥದೃಷ್ಟಿಯಿಂದ ಮಾಡುವುದು ಮಧ್ಯಮ. ಏನನ್ನೂ ಬೆಳೆಸದೆ ಕಾಲಹರಣ ಮಾಡುವ ರೈತ ಅಧಮ. ಇದನ್ನು ಜೀವನದಲ್ಲಿ ಎಲ್ಲಾ ಕಾರ್ಯಕ್ಷೇತ್ರಕ್ಕೂ ಅನ್ವಯಿಸಬಹುದು.

ಕರ್ಮ ಮಾಡದೆ ಇದ್ದರೆ ಜೀವನದಲ್ಲಿ ಬಾಳುವುದೇ ಕಷ್ಟವಾಗುವುದು. ಕರ್ಮ ಮಾಡದೆ ಇದ್ದರೆ ನಮಗೆ ಊಟಕ್ಕೆ ಕೊಡುವವರು ಯಾರು? ನಾನು ಭಿಕ್ಷೆ ಬೇಡಬಹುದು, ಭಿಕ್ಷೆ ಬೇಡುವುದೂ ಒಂದು ಕರ್ಮವಾಯಿತು. ಯಾರೋ ನನಗೆ ಊಟ ಹಾಕಿದರೆಂದು ಭಾವಿಸೋಣ. ಅದನ್ನು ಚೆನ್ನಾಗಿ ಅರಗಿಸಿಕೊಳ್ಳಬೇಕಾದರೆ ಏನಾದರೂ ಕಷ್ಟಪಟ್ಟು ಕೆಲಸ ಮಾಡಬೇಕು. ಏನೂ ಕೆಲಸ ಮಾಡದೆ, ಆಟವಾಡದೆ ದೇಹವನ್ನು ದಂಡಿಸದೇ ಚೆನ್ನಾಗಿ ಪುಷ್ಕಳ ಭೋಜನವನ್ನು ಮಾಡಿಕೊಂಡಿದ್ದರೆ ಇಪ್ಪ ತ್ತೆಂಟು ರೋಗಗಳಿಂದ ನರಳಬೇಕಾಗುವುದು. ದೇಹ ರೋಗದಿಂದ ನರಳುವುದೂ ಕೂಡ ಒಂದು ವಿಧವಾದ ಕರ್ಮವೆ. ಸಾರ್ಥಕವಾದ ಕರ್ಮವನ್ನು ಮಾಡುವುದನ್ನು ಬಿಟ್ಟು ಈಗ ನರಳಬೇಕಾದ ಕರ್ಮವನ್ನು ಕಟ್ಟಿಕೊಂಡಂತೆ ಆಯಿತು. ಅಂತೂ ನಾವು ಕರ್ಮದಿಂದ ಪಾರಾಗುವಂತೆ ಇಲ್ಲ. ಪಾರಾದೆ ಎಂದು ಭಾವಿಸಿದಾಗಲೂ ಮತ್ತೊಂದು ಕರ್ಮಕ್ಕೆ ಸಿಕ್ಕಿಕೊಂಡು ನರಳುವೆವು.

\begin{verse}
ಯಜ್ಞಾರ್ಥಾತ್ ಕರ್ಮಣೋಽನ್ಯತ್ರ ಲೋಕೋಽಯಂ ಕರ್ಮಬಂಧನಃ ।\\ತದರ್ಥಂ ಕರ್ಮ ಕೌಂತೇಯ ಮುಕ್ತಸಂಗಃ ಸಮಾಚರ \versenum{॥ ೯ ॥}
\end{verse}

{\small ಯಜ್ಞಾರ್ಥವಾಗಿ ಮಾಡಿದ ಕರ್ಮ ವಿನಃ ಉಳಿದ ಕರ್ಮವೆಲ್ಲ ಬಂಧನಕ್ಕೆ ಕಾರಣವಾಗುವುದು. ಆದಕಾರಣ ಕೌಂತೇಯ, ಸಂಗವನ್ನು ಬಿಟ್ಟು ಕರ್ಮವನ್ನು ಸರಿಯಾಗಿ ಮಾಡು.}

ಯಜ್ಞಕ್ಕಾಗಿ ಮಾಡಿದ ಕರ್ಮ ಎಂದರೆ ಇದು ಭಗವದರ್ಪಿತವಾಗಲಿ ಎಂದು ಮಾಡಿದ ಕರ್ಮ. ಈ ದೃಷ್ಟಿಯಿಂದ ಮಾಡಿದ ಕರ್ಮ ನಮ್ಮನ್ನು ಬಂಧಿಸುವುದಿಲ್ಲ. ದೇವರು ನಮಗೆ ಏನನ್ನು ಕೊಟ್ಟಿರುವನೋ ಅದು ದೇಹದ ದುಡಿಮೆ ಇರಬಹುದು, ಜಾಣತನದ ಕೆಲಸ ಇರಬಹುದು, ಕಲಾಭಿರುಚಿ ಇರಬಹುದು, ದೊಡ್ಡ ಬೋಧನೆಯ ಕೆಲಸ ಇರಬಹುದು, ಇವನ್ನೆಲ್ಲಾ ನಾವೊಂದು ಯಜ್ಞರೂಪಕ್ಕೆ ಪರಿವರ್ತಿಸಬೇಕು. ಪ್ರತಿಯೊಂದು ಲೌಕಿಕ ಕೆಲಸವನ್ನೂ ಪೂಜೆಯಂತೆ ಮಾಡಲು ಸಾಧ್ಯ. ಯಾವಾಗ ನಾವು ಯಜ್ಞದೃಷ್ಟಿಯಿಂದ ಮಾಡುತ್ತೇವೆಯೋ ಆಗ ಒಂದು ಕೆಲಸ ಮೇಲಲ್ಲ, ಮತ್ತೊಂದು ಕೆಲಸ ಕೀಳಲ್ಲ. ಒಂದರಷ್ಟೇ ಮತ್ತೊಂದು ಮುಖ್ಯ. ಕೇವಲ ಕೀರ್ತಿ, ಹೆಸರು, ಅಧಿಕಾರ ಸಂಪಾದನೆಯ ದೃಷ್ಟಿಯಿಂದ ಒಂದರಿಂದ ಹೆಚ್ಚು ವರಮಾನ ಮತ್ತೊಂದರಿಂದ ಕಡಿಮೆ ವರಮಾನ ಬರಬಹುದು. ಆದರೆ ಸಮಷ್ಟಿಯ ಹಿತದ ದೃಷ್ಟಿಯಿಂದ ಒಬ್ಬನಷ್ಟೇ ಮತ್ತೊಬ್ಬ ಮುಖ್ಯ. ಅಲ್ಲಿ ಯಾರೂ ನಗಣ್ಯರಲ್ಲ. ಒಂದು ದೇವಸ್ಥಾನದಲ್ಲಿ ಕಸ ಗುಡಿಸುವವನು ಎಷ್ಟು ಮುಖ್ಯವೋ, ದೇವರಿಗೆ ಚಾಮರ ಹಾಕುವವನು ಅಷ್ಟೇ ಮುಖ್ಯ. ದೇವರಿಗೆ ಪೂಜೆ ಮಾಡುವವನು, ಅವನಿಗೆ ನೈವೇದ್ಯ ಕೊಡುವವನು, ಅವನ ಕೊರಳಿನಲ್ಲಿ ಹಾಕಲು ಮಾಲೆ ಮಾಡುವವನು ಎಲ್ಲರೂ ಒಬ್ಬನೇ ದೇವರ ಕೆಲಸ ಮಾಡುತ್ತಿರುವರು. ಯಾರ ಕಣ್ಣಿಗೂ ಬೀಳದೆ ಆ ಗುಡಿಯ ಕಸವನ್ನು ಗುಡಿಸಿ ಅದನ್ನು ನಿರ್ಮಲ ಮಾಡುತ್ತಿರುವ ವ್ಯಕ್ತಿ, ಸದಾ ಕಣ್ಣಿಗೆ ಬೀಳುವ ಪೂಜಾರಿಯಷ್ಟೇ ಮುಖ್ಯ. ಯಾವನು ಕಾರ್ಯವನ್ನು ಮಾಡುವಾಗ ಭಗವಂತನನ್ನು ತನ್ನೆದುರಿಗೆ ಇಟ್ಟುಕೊಂಡು ತಾನು ಅವನಿಗೆ ಮಾಡುತ್ತಿರುವ ಸೇವೆ ಈ ಕರ್ಮ ಎಂದು ಮಾಡುವನೋ ಅವನು ಉದ್ಧಾರವಾಗುತ್ತಾನೆ. ಆ ಕರ್ಮವೇ ಅದ್ಭುತ ಪರಿಣಾಮ ವನ್ನು ಬೀರುತ್ತದೆ ಜನಸಮುದಾಯದ ಮೇಲೆ.

ಯಾವಾಗ ನಾವು ಯಜ್ಞದೃಷ್ಟಿಯನ್ನು ಮರೆಯುತ್ತೇವೆಯೋ ಆಗ ನಾವು ಸ್ವಾರ್ಥಕ್ಕೆ ಇಳಿಯು ತ್ತೇವೆ. ಲಾಭಕ್ಕೆ, ಯಶಸ್ಸಿಗೆ, ಕೀರ್ತಿಗೆ ಕರ್ಮ ಮಾಡುತ್ತೇವೆ. ಅನಂತರ ಅದರ ಪಾಶಕ್ಕೆ ಸಿಕ್ಕಿ ನರಳುತ್ತೇವೆ. ಒಳ್ಳೆಯ ಕೆಲಸ ಮಾಡಿದರೂ ಅದರ ಫಲಾಪೇಕ್ಷೆಗೆ ಸಿಕ್ಕಿದರೆ ವ್ಯಥೆಪಡುತ್ತೇವೆ. ಫಲಾಪೇಕ್ಷೆ ಎನ್ನುವುದು ಒಂದು ಬಗೆಯ ಅಂಟು. ಅದರಿಂದ ಪಾರಾಗಬೇಕಾದರೆ ಕೈಗೆ ಯಜ್ಞದೃಷ್ಟಿಯ ಎಣ್ಣೆಯನ್ನು ಸವರಿಕೊಂಡಿರಬೇಕು.

ಆದಕಾರಣವೇ ಶ್ರೀಕೃಷ್ಣ ಅರ್ಜುನನಿಗೆ ಮಾಡುವ ಕೆಲಸವನ್ನು ಬಿಡಬೇಡ, ಆಸಕ್ತಿಯನ್ನು ಮಾತ್ರ ಬಿಡು, ಅನಂತರ ಕೆಲಸವನ್ನು ಸರಿಯಾಗಿ ಮಾಡು ಎನ್ನುತ್ತಾನೆ. ಅನಾಸಕ್ತಿಯಿಂದ ಕೆಲಸ ಮಾಡುವುದು ಎಂದರೆ ಏನೊ ಕಾಟಾಚಾರಕ್ಕೆ ಉದಾಸೀನವಾಗಿ ಮಾಡುವ ಕರ್ಮವಲ್ಲ. ನಿಜ, ಅವನಿಗೆ ಫಲದ ಮೇಲೆ ಅಪೇಕ್ಷೆ ಇಲ್ಲ. ಆದರೆ ಅವನು ಕರ್ಮ ಮಾಡುವಾಗ ಈ ಕರ್ಮವೇ ನನ್ನನ್ನು ಮೋಕ್ಷಕ್ಕೆ ಒಯ್ಯುವುದು ಎಂಬ ದೃಷ್ಟಿಯಿಂದ ಮಾಡುವನು. ಅವನಿಗೆ ಗುರಿ ಬೇರೆ ಅಲ್ಲ, ದಾರಿ ಬೇರೆ ಅಲ್ಲ. ಗುರಿಗೆ ಕೊಡುವ ಪ್ರಾಮುಖ್ಯತೆಯನ್ನು ದಾರಿಗೆ ಕೊಡುವನು. ಅನಾಸಕ್ತಿ ಎಂಬುದು ಒಂದು ದೃಷ್ಟಿಯಿಂದ ಆಸಕ್ತಿಯ ಪರಾಕಾಷ್ಠೆ. ಅಲ್ಲಿ ತನಗಾಗಿ ಇಲ್ಲ. ಎಲ್ಲ ದೇವರಿಗಾಗಿ ಮಾಡುವನು. ತನ್ನ ಸ್ವಂತ ಲಾಭ ಮತ್ತು ಪ್ರಯೋಜನಕ್ಕೆ ಎಷ್ಟು ಮುತುವರ್ಜಿಯಿಂದ ಕೆಲಸ ಮಾಡುವನೋ ಅದಕ್ಕಿಂತ ನೂರು ಪಾಲು ಹೆಚ್ಚು ಮುತುವರ್ಜಿಯಿಂದ ಕೆಲಸ ಮಾಡುವನು. ಆದರೆ ಅವನಿಂದ ಏನೂ ಹಿಂತಿರುಗಿ ಬಯಸುವುದಿಲ್ಲ. ಹಿಂತಿರುಗಿ ಬಯಸುವವರು ಕೃಪಣರು. ಜೀವನದಲ್ಲಿ ಕೊಡುವುದ ರಲ್ಲಿಯೂ ಒಂದು ಆನಂದವಿದೆ. ಭಗವಂತನಿಗೆ ಕೊಡುವುದರಲ್ಲಿ ಒಂದು ಭೂಮಾನಂದವಿದೆ. ಯಜ್ಞದೃಷ್ಟಿಯಿಂದ ಮಾಡುವವನಿಗೆ ಈ ರುಚಿ ಗೊತ್ತಿದೆ. ಅಯ್ಯೋ, ಕೊಡಬೇಕಲ್ಲ, ಮಾಡಬೇಕಲ್ಲ ಎಂದು ಗೊಣಗುವುದಿಲ್ಲ ಅವನು. ಅವನಿಗಾಗಿ ಮಾಡುವುದಕ್ಕೆ, ಅವನಿಗೆ ಕೊಡುವುದಕ್ಕೆ ಒಂದು ಅವಕಾಶವನ್ನು ಅವನು ನನಗೆ ದಯಪಾಲಿಸಿದನಲ್ಲ ಎಂದು ಎಂದೆಂದಿಗೂ ಧನ್ಯವಾದವನ್ನು ಅವನು ಭಗವಂತನಿಗೆ ಅರ್ಪಿಸುವನು. ಭಗವಂತನ ಕೆಲಸ ನಮ್ಮಂತಹ ಯಃಕಶ್ಚಿತ್ ವ್ಯಕ್ತಿಗಳಿಂದ ನಿಂತು ಹೋಗುವುದಿಲ್ಲ. ಅವನು ನಾನಲ್ಲದೇ ಇದ್ದರೆ ಇನ್ನಾರನ್ನೊ ನಿಮಿತ್ತವಾಗಿಟ್ಟುಕೊಂಡು ಕೆಲಸ ಮಾಡುತ್ತಾನೆ. ಆದರೆ ಅವನು ನನ್ನನ್ನು ತೆಗೆದುಕೊಂಡಿದ್ದಕ್ಕೆ ನಾನು ಧನ್ಯವಾದವನ್ನು ಅರ್ಪಿಸ ಬೇಕಾಗಿದೆ.

\begin{verse}
ಸಹಯಜ್ಞಾಃ ಪ್ರಜಾಃ ಸೃಷ್ಟ್ವಾ ಪುರೋವಾಚ ಪ್ರಜಾಪತಿಃ ।\\ಅನೇನ ಪ್ರಸವಿಷ್ಯಧ್ವಮೇಷ ವೋಽಸ್ತಿಷ್ಟಕಾಮಧುಕ್ \versenum{॥ ೧೦ ॥}
\end{verse}

{\small ಪೂರ್ವದಲ್ಲಿ ಪ್ರಜಾಪತಿ ಯಜ್ಞಗಳೊಡನೆ ಪ್ರಜೆಗಳನ್ನು ಸೃಷ್ಟಿಸಿ “ಇದರಿಂದ ನೀವು ಅಭಿವೃದ್ಧಿಯನ್ನು ಹೊಂದಿ. ಇದು ನಿಮಗೆ ಇಷ್ಟವನ್ನು ಕೊಡುವ ಕಾಮಧೇನುವಾಗಲಿ” ಎಂದನು.}

ಪೂರ್ವದಲ್ಲಿ ಎಂದರೆ ಸೃಷ್ಟಿಯ ಆದಿಯಲ್ಲಿ. ಹಿಂದೂಗಳಾದ ನಾವು ಯಾವುದೋ ಒಂದು ಸೃಷ್ಟಿಯನ್ನು ನಂಬುವುದಿಲ್ಲ. ಹಲವಾರು ಸೃಷ್ಟಿಗಳು ಹಿಂದೆ ಆಗಿವೆ, ಹಲವು ಸೃಷ್ಟಿಗಳು ಮುಂದೆ ಆಗುವುದರಲ್ಲಿವೆ. ಸೃಷ್ಟಿಯೆನ್ನುವುದು ಸಾಗರದಲ್ಲಿ ಏಳುವ ಒಂದು ಅಲೆಯಂತೆ. ಬೀಳುವುದು ಏಳುವುದು. ಅದರಂತೆಯೇ ನಾವಿರುವ ಈ ಸೃಷ್ಟಿಯ ಪ್ರಾರಂಭದಲ್ಲಿ ಪ್ರಜಾಪತಿ ಎಂದರೆ ಭಗವಂತನು ಪ್ರಜೆಗಳನ್ನು ಸೃಷ್ಟಿ ಮಾಡಿದಾಗ ಜೊತೆ ಜೊತೆಯಲ್ಲಿಯೇ ಯಜ್ಞವನ್ನು ಸೃಷ್ಟಿ ಮಾಡಿದನು. ಯಜ್ಞ ಎಂದರೆ ಸಮಷ್ಟಿಯ ಜೀವನಕ್ಕೆ ವ್ಯಷ್ಟಿ ಮಾಡುವ ಕ್ರಿಯೆಗಳು. ನಮ್ಮ ಸಮಾಜ ನಿಂತಿರುವುದೇ ಈ ಆಧಾರದ ಮೇಲೆ. ಒಂದು ಸಂಸಾರ ನಿಂತಿರುವುದು, ಊರು ಇರುವುದು, ದೇಶ ಇರುವುದು ಈ ಆಧಾರದಮೇಲೆ. ಮನೆಯ ಯಜಮಾನ ಸಂಸಾರ ನಿರ್ವಹಣೆಗೆ ಕಷ್ಟಪಡಬೇಕು. ಅದನ್ನು ಚೆನ್ನಾಗಿ ಇಟ್ಟಿರುವುದಕ್ಕೆ ಅದರಲ್ಲಿರುವ ಪ್ರತಿಯೊಬ್ಬರಿಗೂ ಏನೇನು ಬೇಕೋ ಅದನ್ನು ಒದಗಿಸುವುದಕ್ಕೆ ದುಡಿಯಬೇಕು. ಅದರಂತೆಯೇ ಊರು ಮತ್ತು ದೇಶ. ಸಮಷ್ಟಿಗೆ ನಾವು ಕೊಟ್ಟರೆ ನಮಗೆ ಅದು ಬರುವುದು. ಅದನ್ನು ಬಲ ಮಾಡಿದರೆ ನಮಗೆ ಬಲ ಬರುವುದು. ಯಜ್ಞ ಎಂದರೆ ನಮ್ಮ ಪಾಲಿಗೆ ಬಂದ ಕರ್ತವ್ಯಗಳನ್ನು ಮಾಡುವುದು. ನಮಗೆ ಮನೆಯಿಂದ, ಊರಿನಿಂದ ದೇಶದಿಂದ ಏನೇನು ಬರಬೇಕಾಗಿದೆ ಎಂದು ವಸೂಲಿ ಮಾಡುವುದಕ್ಕೆ ಕಾತರನಾಗಿರುವುದಿಲ್ಲ. ನಾವು ಏನು ಕೊಡಬೇಕಾಗಿದೆ, ನಾವು ಏನು ಮಾಡಬೇಕಾಗಿದೆ ಇತರರಿಗೆ ಎಂದು ನೋಡುವ ದೃಷ್ಟಿಗೆ ವಿರೋಧ ವಾಗಿದೆ ಇದು. ಮುಕ್ಕಾಲು ಪಾಲು ಮಾನವರಿಗೆ ಬರುವುದರ ಮೇಲೆ ಆಸೆ, ಕೊಡುವುದರ ಮೇಲೆ ಅಲ್ಲ. ಅದಕ್ಕೆ ನಾವು ವ್ಯಥೆಯನ್ನು ಅನುಭವಿಸಬೇಕಾಗುವುದು. ಒಬ್ಬ ಶ್ರೀಮಂತನ ಅವಸಾನ ಕಾಲ ಪ್ರಾಪ್ತವಾಯಿತು. ಅವನು ತನ್ನ ಮಕ್ಕಳಿಗೆಲ್ಲಾ ಹೇಳಿದನು. ನಾನು ಉಳಿಯುವಂತೆ ಇಲ್ಲ. ನೀವು ಕಾಗದ ಮಸಿಕಡ್ಡಿ ತೆಗೆದುಕೊಂಡು ಬನ್ನಿ. ಯಾರು ಯಾರು ನನಗೆ ಎಷ್ಟು ಕೊಡಬೇಕಾಗಿದೆ ಎಂಬುದನ್ನು ಹೇಳುತ್ತೇನೆ, ಅದನ್ನು ಬರೆದು ಇಟ್ಟುಕೊಳ್ಳಿ. ಮಕ್ಕಳಿಗೆಲ್ಲಾ ಕಿವಿ ಚುರುಕಾಯಿತು. ಹತ್ತಿರ ಬಂದು ಕುಳಿತುಕೊಂಡು ಯಾರು ಯಾರಿಂದ ಎಷ್ಟೆಷ್ಟು ಸಾವಿರ ರೂಪಾಯಿಗಳು ಬರಬೇಕಾಗಿದೆ ಎಂಬುದನ್ನು ಬಹಳ ಉತ್ಸಾಹದಿಂದ ಬರೆಯತೊಡಗಿದರು. ಅನಂತರ ವರ್ತಕ ಹೇಳಿದ:“ಇನ್ನು ಮೇಲೆ ನಾನು ಯಾರಿಗೆ ಎಷ್ಟು ಕೊಡಬೇಕಾಗಿದೆ ಎಂಬುದನ್ನು ಹೇಳುತ್ತೇನೆ. ಅದನ್ನು ಬರೆದು ಇಟ್ಟುಕೊಳ್ಳಿ.” ಮಕ್ಕಳಿಗೆ ಈ ಮಾತನ್ನು ಕೇಳಿದಾಗ ಅಪ್ಪನಿಗೆ ಈಗ ಬುದ್ಧಿ ಸ್ಥಿಮಿತವಿಲ್ಲ ಎಂದರು. ಏಕೆಂದರೆ ಈಗ ಕೊಡಬೇಕಾದ ಪ್ರಸಂಗ ಬಂತು. ಅದರಂತೆಯೇ ನಾವೆಲ್ಲ. ಮಾಡುವುದನ್ನು ಮಾಡದೆ ನಮಗೆ ಬರಬೇಕಾಗಿರುವುದು ಹೇಗೆ ಬರುವುದು? ಸೃಷ್ಟಿ ನಿಂತಿರುವುದೇ ಕೊಟ್ಟು ತೆಗೆದುಕೊಳ್ಳುವ ಆಧಾರದ ಮೇಲೆ. ಯಾವಾಗ ಇಲ್ಲಿ ಅನಾಯಕವಾಗುವುದೋ ಆಗ ಸಮಾಜದಲ್ಲಿ ಬಿಗಿ ತಪ್ಪುವುದು. ಪ್ರಕೃತಿಯಲ್ಲಿ ಈ ನಿಯಮ ಜಾರಿಯಲ್ಲಿರುವುದನ್ನು ನೋಡುತ್ತೇವೆ. ಸಮುದ್ರ ಹಗಲು ರಾತ್ರಿ ತನಗೆ ಬೀಳುವ ನದಿಗಳಿಂದ ನೀರನ್ನು ತೆಗೆದುಕೊಂಡು ಮಳೆಯಂತೆ ಹಿಂತಿರುಗಿ ಕಳುಹಿಸುತ್ತಿರುವುದು. ತನಗೆ ಎಷ್ಟು ಬರುವುದೋ ಅಷ್ಟನ್ನು ಹಿಂತಿರುಗಿ ಕಳುಹಿಸುವುದು. ಜಾಸ್ತಿಯನ್ನೂ ಕಳುಹಿಸುವುದಿಲ್ಲ, ಕಡಿಮೆಯನ್ನೂ ಕಳುಹಿಸುವುದಿಲ್ಲ. ಸಮುದ್ರ ಉಕ್ಕಿ ಹರಿಯುವುದನ್ನು ಕಂಡಿಲ್ಲ, ಸಮುದ್ರ ಬತ್ತಿ ಹೋಗುವುದನ್ನೂ ಕಂಡಿಲ್ಲ. ದಿನ ಬೆಳಗಾದರೆ ನೀರು ಒಂದೇ ಸಮನಾಗಿರುವುದು. ಬೇಕಾದಷ್ಟು ಪ್ರವಾಹ ಬರಬಹುದು. ಮಳೆಯೇ ಬರದೆ ಹಲವು ನದಿಗಳು ಬತ್ತಿಹೋಗಬಹುದು. ಆದರೆ ಸಮುದ್ರದಲ್ಲಿ ಯಾವ ಬದಲಾವಣೆಯನ್ನೂ ನೋಡುವುದಿಲ್ಲ. ಇದೇ ನಿಯಮವನ್ನು ಸರ್ಕಾರದಲ್ಲಿ ನೋಡುವೆವು. ಒಂದು ಸರ್ಕಾರ ಹಲವು ರೀತಿಯ ತೆರಿಗೆಯನ್ನು ಹಾಕುವುದು. ಆ ದುಡ್ಡು ಬಂದಮೇಲೆ ಎಷ್ಟು ಬಂತೋ ಅದೆಲ್ಲ ಖರ್ಚು ಆಗುವುದು. ಸರಿಯಾದ ಬಡ್ಜೆಟ್ ಎಂದರೆ ಆದಾಯ ವೆಚ್ಚ ಸಮತೂಕವಾಗಿರಬೇಕು. ಸರ್ಕಾರ ಒಂದು ಕಡೆಯಿಂದ ತೆಗೆದುಕೊಳ್ಳುವುದು. ಪುನಃ ತೆಗೆದುಕೊಂಡು ದನ್ನು ಹಿಂತಿರುಗಿ ಕೊಡುವುದು. ನಮ್ಮ ದೇಹದಲ್ಲಿರುವ ಹೃದಯದಲ್ಲಿಯೇ ಆ ನಿಯಮ ಜಾರಿಯಲ್ಲಿ ರುವುದನ್ನು ನೋಡುತ್ತೇವೆ. ದೇಹದ ಹಲವು ಭಾಗಗಳಿಂದ ರಕ್ತ ಹೃದಯಕ್ಕೆ ಬರುವುದು. ಆ ಹೃದಯವಾದರೋ ಪುನಃ ಆ ರಕ್ತ ಬೇರೆ ಬೇರೆ ಕಡೆ ಸಂಚರಿಸುವಂತೆ ಒತ್ತಿ ಕಳುಹಿಸುವುದು.

ಇಂತಹ ಯಜ್ಞರೂಪವಾದ ಕ್ರಿಯೆಯಿಂದ ನೀವು ವೃದ್ಧಿಯಾಗಿರಿ ಎಂದು ಶ್ರೀಕೃಷ್ಣ ಸಾರುತ್ತಾನೆ. ಎಲ್ಲ ಬೆಳೆಯಬೇಕು, ವಿಕಾಸವಾಗಬೇಕು. ಅದಕ್ಕಾಗಿ ಈ ಭೂಮಿಗೆ ಬಂದಿರುವುದು. ಆದರೆ ವಿಕಾಸವಾದಾಗ, ವೃದ್ಧಿಯಾದಾಗ, ಈ ಯಜ್ಞದೃಷ್ಟಿಯ ಆಧಾರದ ಮೇಲೆ ಇರಬೇಕು.

ಈ ಯಜ್ಞ ನಿಮ್ಮ ಇಷ್ಟವನ್ನು ಪೂರೈಸುವ ಕಾಮಧೇನುವಾಗಲಿ ಎನ್ನುವನು ಶ್ರೀಕೃಷ್ಣ. ನಮಗೆ ಏನು ಬೇಕೋ ಅದನ್ನು ಕೊಡುವುದು ಈ ಯಜ್ಞ. ನಾವು ಮಾಡಬೇಕಾಗಿರುವುದನ್ನು ಮಾಡಿದರೆ, ಜೀವನದಲ್ಲಿ ನಮಗೆ ಬರಬೇಕಾಗಿರುವುದು ಬಂದೇ ಬರುವುದು. ವಿಧಿನಿಯಮ ಇದು. ಆಕರ್ಷಣ, ವಿದ್ಯುತ್​ಶಕ್ತಿ, ಬೆಳಕು, ಶಾಖ ಮುಂತಾದುವುಗಳ ನಿಯಮ ಹೇಗೋ ಹಾಗೆಯೇ ಈ ನಿಯಮ ಮಾನವ ಕೋಟಿಯಲ್ಲಿ ಜಾಗೃತವಾಗಿದೆ. ಶ್ರೀಕೃಷ್ಣ ಬಯಕೆಯನ್ನು ಇಟ್ಟುಕೊಳ್ಳಬೇಡಿ ಎನ್ನುವುದಿಲ್ಲ. ಯಜ್ಞ ರೀತಿ ಆ ಬಯಕೆಯನ್ನು ಈಡೇರಿಸಿಕೊಳ್ಳಿ ಎನ್ನುವನು.

\begin{verse}
ದೇವಾನ್ ಭಾವಯತಾನೇನ ತೇ ದೇವಾ ಭಾವಯಂತು ವಃ ।\\ಪರಸ್ಪರಂ ಭಾವಯಂತಃ ಶ್ರೇಯಃ ಪರಮವಾಪ್ಸ್ಯಥ \versenum{॥ ೧೧ ॥}
\end{verse}

{\small ಯಜ್ಞದ ಮೂಲಕ ನೀವು ದೇವತೆಗಳನ್ನು ತೃಪ್ತಿಪಡಿಸಿ. ಆ ದೇವತೆಗಳು ನಿಮ್ಮನ್ನು ತೃಪ್ತಿಪಡಿಸುತ್ತಾರೆ ಪರಸ್ಪರ ಭಾವಿಸುತ್ತಿದ್ದರೆ ಪರಮ ಶ್ರೇಯಸ್ಸನ್ನು ಹೊಂದುತ್ತೀರಿ.}

ನಾವು ದೇವರನ್ನು ಕೇಳಬೇಕಾದರೆ ಯಜ್ಞದ ಮೂಲಕ ಕೇಳಬೇಕು. ಕೊಟ್ಟು ಕೇಳಬೇಕು. ಹಾಗೆಯೇ ಕೇಳುವುದಲ್ಲ. ನಾವು ಗರೀಬರು. ದೇವರಿಗೆ ಎಲ್ಲವೂ ಇದೆ. ನಾವು ಕೊಡುವುದು ಏನನ್ನು? ಅವನಾದರೂ ನಮ್ಮಿಂದ ಏತಕ್ಕೆ ನಿರೀಕ್ಷಿಸಬೇಕು ಎಂದು ಆಲೋಚಿಸಬಹುದು. ನಮ್ಮಲ್ಲಿ ಏನಿ ದೆಯೋ ಅದನ್ನು ಕೊಡೋಣ. ನಮ್ಮಲ್ಲಿ ಏನೂ ಇಲ್ಲ ಎಂದಲ್ಲ. ನಮ್ಮಲ್ಲಿರುವ ಅಲ್ಪವನ್ನು ದೇವರಿಗೆ ಕೊಟ್ಟುಬಿಡೋಣ. ಅವನು ಅಲ್ಪವನ್ನು ತೆಗೆದುಕೊಂಡು ಭೂಮವನ್ನಾಗಿ ಮಾಡುತ್ತಾನೆ. ಅವನೇನೂ ನಮ್ಮಿಂದ ನಿರೀಕ್ಷಿಸುವುದಿಲ್ಲ. ನಾವು ಕೊಡುವುದಕ್ಕೆ ಮುಂಚೆಯೇ ಇದೆಲ್ಲಾ ಅವನದು. ನಾವು ಏನನ್ನು ಕೊಡುವುದು ಎಂದು ಹೇಳುತ್ತೇವೆಯೋ ಅದು ಒಂದೇ ಕೋಟಿನಲ್ಲಿ ಒಂದು ಜೇಬಿನಿಂದ ತೆಗೆದು ಇನ್ನೊಂದು ಜೇಬಿಗೆ ಹಾಕಿದಂತೆ. ಈ ಕೊಡುವುದು ನಮ್ಮ ಜೀವನಕ್ಕೆ ಇರುವ ಒಂದು ಅವಕಾಶ. ನಮ್ಮ ಬೆಳವಣಿಗೆಗೆ, ಸ್ವಾರ್ಥದ ಕೊಳೆಯನ್ನು ಕಳೆದುಕೊಳ್ಳುವುದಕ್ಕೆ ಒಂದು ಅವಕಾಶ ಎಂಬ ದೃಷ್ಟಿಯಿಂದ ನೋಡಬೇಕು. ನಾವು ದೇವರಿಗೆ ಕೊಟ್ಟದ್ದು ಯಾವುದೂ ಹಿಂತಿರುಗಿ ಬರದೆ ಇರುವುದಿಲ್ಲ. ನಾವು ಕೊಟ್ಟಿದ್ದೆಲ್ಲಾ ಒಂದಾಗಿ, ನೂರಾಗಿ, ಸಾವಿರವಾಗಿ ಹಿಂತಿರುಗುವುದು. ರೈತ ಒಂದು ಕಾಳನ್ನು ನೆಲಕ್ಕೆ ಹಾಕುತ್ತಾನೆ. ಭೂಮಿದೇವಿ ಅದನ್ನು ತಿಂದುಬಿಡುವಳೆ? ಆ ಕಾಳಿಗೆ ಬದಲು ಅವಳು ಸಾವಿರಾರು ಕಾಳುಗಳಿರುವ ತುಂಬಿದ ತೆನೆಯನ್ನು ಕೊಡುವಳು. ನಾವು ದೇವರ ಹೆಸರಿನಲ್ಲಿ ಮತ್ತೊಬ್ಬರಿಗೆ ಮಾಡಿರುವುದೆಲ್ಲಾ–ಅದು ನಮ್ಮ ಶ್ರಮದಾನವಿರಬಹುದು, ಅನ್ನದಾನ ವಿರಬಹುದು, ವಿದ್ಯಾದಾನವಿರಬಹುದು, ಯಾವುದೂ ನಿರರ್ಥಕವಾಗುವುದಿಲ್ಲ. ನಾವು ಕೊಟ್ಟಿದ್ದಕ್ಕಿಂತ ಹೆಚ್ಚು ನಮಗೆ ಬಂದೇ ಬರುವುದು. ಸಾಧಾರಣ ಯಜಮಾನನೇ ಕೂಲಿಯ ಕೈಯಲ್ಲಿ ಕೆಲಸ ಮಾಡಿಸಿಕೊಂಡು ಹಾಗೆಯೇ ಕಳುಹಿಸುವುದಿಲ್ಲ. ಕೂಲಿ ಕೊಟ್ಟು ಕಳುಹಿಸುವನು. ಹೀಗಿರುವಾಗ ದೇವರ ಹೆಸರಿನಲ್ಲಿ ನಾವು ಏನನ್ನಾದರೂ ಕೊಟ್ಟರೆ ಅವನು ನಾವು ಕೊಟ್ಟಿದ್ದನ್ನು ಸುಮ್ಮನೆ ತೆಗೆದುಕೊಂಡು ಹೋಗಿಬಿಡುವುದಿಲ್ಲ. ಒಂದನ್ನು ಕೊಟ್ಟರೆ ಹತ್ತನ್ನು ಹಿಂತಿರುಗಿ ಕೊಡುತ್ತಾನೆ. ನಾವು ನಮ್ಮ ಪಾಲಿನ ಕರ್ತವ್ಯವನ್ನು ದೇವರಿಗೆ ಯಜ್ಞರೂಪದಂತೆ ಮಾಡಿದರೆ, ದೇವರು ನಮಗೆ ಬೇಕಾದು ದನ್ನು ವರರೂಪದಲ್ಲಿ ಕರುಣಿಸುವನು. ಈ ಅನ್ಯೋನ್ಯ ನಿಯಮದ ಮೇಲೆ ಸೃಷ್ಟಿ ನಿಂತಿದೆ.

ಇದರಿಂದ ನೀವು ಪರಮ ಶ್ರೇಯಸ್ಸನ್ನು ಪಡೆಯುತ್ತೀರಿ ಎನ್ನುವನು ಶ್ರೀಕೃಷ್ಣ. ಶ್ರೇಯಸ್ಸಿನಲ್ಲೆಲ್ಲಾ ಪರಮ ಶ್ರೇಯಸ್ಸೇ ಮುಕ್ತಿ. ಇದಕ್ಕಿಂತ ಮಿಗಿಲಾದುದು ಯಾವುದೂ ಇಲ್ಲ. ನಾನು ಮಾಡುವ ಕರ್ತವ್ಯ ಶ್ರೀಕೃಷ್ಣನಿಗೆ ಅರ್ಪಿತವಾಗಲಿ ಎಂಬ ದೃಷ್ಟಿಯಿಂದ ಮಾಡಿದರೆ, ನನ್ನ ಚಿತ್ತ ಶುದ್ಧಿಯಾಗಿ ಆ ಭಗವಂತನ ಜ್ಯೋತಿ ಅಲ್ಲಿ ಬೆಳಗುವುದು. ನಮ್ಮ ಅಜ್ಞಾನ ನಾಶವಾಗುವುದು. ನಮ್ಮ ಬಂಧನಗಳು ಕಳಚಿಬೀಳುವುವು. ನಮ್ಮ ದುಃಖ ದುರಿತಗಳು ಕೊನೆಗಾಣುವುವು. ಈ ಪ್ರಪಂಚವನ್ನು ಬಿಟ್ಟು ಹೋಗುವಾಗ ಹಿಂತಿರುಗಿ ಬರದ ರೀತಿಯಲ್ಲಿ ಹೋಗುತ್ತೇವೆ.

\begin{verse}
ಇಷ್ಟಾನ್ ಭೋಗಾನ್ ಹಿ ವೋ ದೇವಾ ದಾಸ್ಯಂತೇ ಯಜ್ಞಭಾವಿತಾಃ ।\\ತೈರ್ದತ್ತಾನಪ್ರದಾಯೈಭ್ಯೋ ಯೋ ಭುಂಕ್ತೇ ಸ್ತೇನ ಏವ ಸಃ \versenum{॥ ೧೨ ॥}
\end{verse}

{\small ಯಜ್ಞದಿಂದ ತೃಪ್ತರಾದ ದೇವತೆಗಳು ನಿಮಗೆ ಇಷ್ಟವಾದ ಭೋಗಗಳನ್ನು ನೀಡುತ್ತಾರೆ. ಅವರು ಕೊಟ್ಟಿದ್ದನ್ನು ಅವರಿಗೆ ಕೊಡದೆ ಯಾರು ತಾನೇ ಅನುಭವಿಸುವನೋ ಅವನು ಕಳ್ಳ.}

ಯಾವಾಗ ನಾವು ದೇವರನ್ನು ಯಜ್ಞದ ಮೂಲಕ ಪ್ರಾರ್ಥಿಸುತ್ತೇವೆಯೋ ಆಗ ದೇವರು ನಾವು ಕೇಳಿದುದನ್ನು ಅನುಗ್ರಹಿಸುವನು. ಕೇಳಿದರೆ ದೇವರು ಎಂದಿಗೂ ಇಲ್ಲ ಎನ್ನುವುದಿಲ್ಲ. ನಾವು ಎಷ್ಟಕ್ಕೆ ಯೋಗ್ಯರೋ ಅಷ್ಟನ್ನು ಕೊಟ್ಟೇ ಕೊಡುವನು. ಅವನು ಕೊಟ್ಟದ್ದನ್ನು ಪುನಃ ದೇವರಿಗೆ ನೈವೇದ್ಯ ಮಾಡಿ ನಾವು ಅದನ್ನು ಅನುಭವಿಸಬೇಕು. ಎಲ್ಲಾ ಅವನ ದಯೆಯಿಂದ ಬಂದದ್ದು. ಕೆರೆಯ ನೀರನ್ನು ಕೆರೆಗೆ ಚೆಲ್ಲಿ ಧನ್ಯರಾಗಿರೋ ಎಂದು ದಾಸರು ಹಾಡುವಂತೆ. ಇದೆಲ್ಲ ಅವನಿಂದ ಬಂದುದು, ಎಂಬ ಭಾವನೆ ನಮ್ಮಲ್ಲಿ ಇರಬೇಕು. ನಾವು ಅವನಿಗೆ ಕೊಟ್ಟರೆ ಅವನೇನೂ ಅದನ್ನೆಲ್ಲ ತಿಂದುಬಿಡುವುದಿಲ್ಲ. ನಾವು ಕೊಟ್ಟದ್ದನ್ನು ಮತ್ತೂ ಪವಿತ್ರ ಮಾಡಿ ಹಿಂತಿರುಗಿ ಕೊಡುವನು. ನಾವು ಮುಂಚೆ ಆಹಾರವನ್ನು ಅವನ ಮುಂದೆ ಇಟ್ಟೆವು. ಅವನು ಅದನ್ನು ಪ್ರಸಾದ ಮಾಡಿ ನಮಗೆ ಹಿಂತಿರುಗಿ ಕೊಡುತ್ತಾನೆ. ಆ ಪ್ರಸಾದವನ್ನು ನಾವು ಸ್ವೀಕರಿಸಿದರೆ ನಮ್ಮ ಚಿತ್ತಶುದ್ಧಿಯಾಗುವುದು. ಯಾವನು ಕೊಡದೆ ತಿನ್ನು ತ್ತಾನೆಯೋ ಅವನನ್ನು ಕಳ್ಳ ಎನ್ನುವನು. ಕಳ್ಳ ಎಂದರೆ ತನಗೆ ಸೇರಿಲ್ಲದ ವಸ್ತುವನ್ನು ತೆಗೆದುಕೊಳ್ಳುವ ವನು. ದೇವರೆ ಎಲ್ಲಕ್ಕೂ ಒಡೆಯ. ಅವನು ನಮಗೆ ಅನುಗ್ರಹಿಸಿ ಕೊಟ್ಟಿರುವನು. ನಾವು ಯಾವಾಗ ಅವನಿಗೆ ಅರ್ಪಣೆಮಾಡಿ ಅನುಭವಿಸುತ್ತೇವೆಯೋ ಆಗ ಆಹಾರದಲ್ಲಿರುವ ದೋಷ ನಾಶವಾಗುವುದು. ಅವನು ಕೊಟ್ಟ ವಸ್ತು ನಮ್ಮ ಹತ್ತಿರ ಇರುವಾಗಲೂ ಎಲ್ಲಕ್ಕೂ ಒಡೆಯ ಅವನೆ. ನಾವು ಕೇವಲ ಟ್ರಸ್ಟಿಗಳು ಎಂಬ ಭಾವ ಇರಬೇಕು. ನಮ್ಮಲ್ಲಿ ಯಾವ ಅಧಿಕಾರವಿರಲಿ, ಐಶ್ವರ್ಯವಿರಲಿ, ಪಾಂಡಿತ್ಯ ವಿರಲಿ ಇದೆಲ್ಲಾ ಅವನದು, ತನ್ನ ಕೆಲಸವನ್ನು ಮಾಡುವುದಕ್ಕೆ ಅವನು ನನ್ನ ಹತ್ತಿರ ಇಟ್ಟಿರುವನು ಎಂಬ ಭಾವವನ್ನು ನಾವು ರೂಢಿಸಬೇಕು. ದೇವರು ಕೆಲವು ವೇಳೆ, ಅಧಿಕಾರ, ಪಾಂಡಿತ್ಯ ಮುಂತಾದು ವನ್ನೆಲ್ಲ ಎಲ್ಲರಿಗೂ ಹಂಚುವುದಕ್ಕೆ ಮುಂಚೆ ಒಂದು ಕಡೆ ಸಂಗ್ರಹಿಸಿ ಅನಂತರ ಎಲ್ಲ ಕಡೆಗೂ ಕಳುಹಿಸುವನು. ನಮ್ಮ ಹೃದಯದಲ್ಲಿ ರಕ್ತ ಎಲ್ಲ ಕಡೆಯಿಂದಲೂ ಬರುವುದು. ಇದು ಬರಿ ತನಗಾಗಿ ಎಂದು ಭಾವಿಸಿದರೆ ಹೃದಯ ನಾಶವಾಗುವುದು. ಬಂದುದನ್ನು ಒತ್ತಿ ಹಿಂತಿರುಗಿ ಬೇರೆ ಬೇರೆ ಕಡೆಗೆ ಕಳುಹಿಸಬೇಕು. ಒಂದೂರಿನಲ್ಲಿ ಮನೆಯ ನಲ್ಲಿಗಳಿಗೆ ನೀರನ್ನು ಒದಗಿಸಲು ಸ್ವಲ್ಪ ಎತ್ತರದಲ್ಲಿ ನೀರನ್ನು ಸಂಗ್ರಹಿಸುವ ದೊಡ್ಡ ತೊಟ್ಟಿಯನ್ನು ಕಟ್ಟುವರು. ಆ ತೊಟ್ಟಿ ಏನಾದರೂ ಆ ನೀರೆಲ್ಲ ತನಗೆ ಎಂದು ಭಾವಿಸಿದರೆ ಅದು ತಪ್ಪು. ತನ್ನಲ್ಲಿಗೆ ಬರುವುದು ಬೇರೆ ಬೇರೆ ಕಡೆ ಹೋಗಲು ಸರಾಗವಾಗುವುದಕ್ಕೆ ಎಂದು ತಿಳಿದರೆ ಸರಿ. ಅದರಂತೆಯೇ ದೇವರು ಹಂಚುವುದಕ್ಕೆ ಮುಂಚೆ ಕೆಲವು ಕಡೆಸಂಗ್ರಹಿಸಿ ಅನಂತರ ಹಂಚುವ ಕೆಲಸವನ್ನು ಮಾಡುವನು. ಭಗವಂತನೇ ಎಲ್ಲದಕ್ಕೂ ಯಜ ಮಾನ. ಅವನು ನಮ್ಮಲ್ಲಿ ಏನು ಕೊಟ್ಟಿರುವನೋ ಅದೆಲ್ಲ ಅವನದು, ಅವನ ಕೆಲಸವನ್ನು ಮಾಡುವುದಕ್ಕೆ ಇರುವುದು ಎಂದು ಭಾವಿಸಿದರೆ ಧನ್ಯರಾಗುತ್ತೇವೆ. ಯಾವಾಗ ದೇವರನ್ನು ಮರೆತು ಇದಕ್ಕೆಲ್ಲ ನಾವೇ ಒಡೆಯರು ಎಂಬ ಮುದ್ರೆಯನ್ನು ಒತ್ತುತ್ತೇವೆಯೊ ಆಗ ನಾವು ತಪ್ಪಿತಸ್ಥರು, ಕಳ್ಳರು. ಅದಕ್ಕೆ ಬರುವ ಶಿಕ್ಷೆಯನ್ನು ಅನುಭವಿಸಲೇಬೇಕಾಗುವುದು. ಇದೇ ಒಂದು ವಸ್ತುವನ್ನು ಅನುಭವಿಸುವ ರಹಸ್ಯ ಮರ್ಮ. ಅನುಭವಿಸುವವನಿಗೆ ಈ ಮರ್ಮ ಗೊತ್ತಿರುವುದಿಲ್ಲ. ಅದಕ್ಕೇ ದುಃಖ ಪಡುವುದು.

\begin{verse}
ಯಜ್ಞಶಿಷ್ಟಾಶಿನಃ ಸಂತೋ ಮುಚ್ಯಂತೇ ಸರ್ವಕಿಲ್ಬಿಷೈಃ ।\\ಭುಂಜತೇ ತೇ ತ್ವಘಂ ಪಾಪಾ ಯೇ ಪಚಂತ್ಯಾತ್ಮಕಾರಣಾತ್ \versenum{॥ ೧೩ ॥}
\end{verse}

{\small ಯಜ್ಞವನ್ನು ಮಾಡಿ ಆದಮೇಲೆ ಮಿಗುವುದನ್ನು ಅನುಭವಿಸುವವರು ಯೋಗ್ಯರು. ಅವರು ಎಲ್ಲಾ ದೋಷ ಗಳಿಂದಲೂ ಪಾರಾಗುತ್ತಾರೆ. ಆದರೆ ಯಾವ ಪಾಪಿಗಳು ತಮಗಾಗಿ ಅಡುಗೆಯನ್ನು ಮಾಡಿಕೊಳ್ಳುತ್ತಾರೊ ಅವರು ಪಾಪವನ್ನೇ ಉಣ್ಣುತ್ತಾರೆ.}

ಯಜ್ಞವನ್ನು ಮಾಡಿ ಆದಮೇಲೆ ಏನು ಉಳಿಯುವುದೋ ಅದೇ ಪವಿತ್ರವಾದುದು. ಇಲ್ಲಿ ಆಹಾರವನ್ನು ಮಾತ್ರ ದೇವರಿಗೆ ಕೊಡು ಎಂದು ಹೇಳುವುದಿಲ್ಲ. ಇದು ತುಂಬಾ ಸಣ್ಣ ಅರ್ಥ. ನಾವು ಯಾವುದನ್ನು ಪಡೆದಿರುವೆವೋ ಅದು ಅಧಿಕಾರವಾಗಬಹುದು, ಪಾಂಡಿತ್ಯವಾಗಬಹುದು, ಅಥವಾ ಇನ್ನು ಹಲವಾರು ವಸ್ತುಗಳಾಗಬಹುದು. ಯಾವಾಗ ನಾವು ಅದನ್ನು ಭಗವಂತನಿಗೆ ನಮ್ಮ ಕಣ್ಣೆದುರಿಗೆ ಇರುವ ಅವನ ಮಕ್ಕಳ ಮೂಲಕ ಕೊಡುವೆವೋ ಆಗ ಅವನಿಗೆ ಸೇರುವುದು. ನಾವು ದೇವರಿಗೆ ಪ್ರತ್ಯಕ್ಷ ಕೊಡುವುದಕ್ಕೆ ಆಗದೆ ಇರಬಹುದು. ಆದರೆ ಪ್ರತಿಯೊಬ್ಬ ಜೀವರಾಶಿಯೂ ನಾವು ಕೊಟ್ಟದ್ದನ್ನು ಅವನಿಗೆ ತಲುಪಿಸಲು ಇರುವ ಒಂದು ಅಂಚೆಯ ಪೆಟ್ಟಿಗೆಯಂತೆ. ನಾವು ಕಾಗದವನ್ನು ದಾರಿಯಲ್ಲಿರುವ ಯಾವ ಪೆಟ್ಟಿಗೆಗೆ ಹಾಕಿದರೂ ಅದರ ಮುಖ್ಯ ಕಛೇರಿಗೆ ಹೋಗಿ ಅಲ್ಲಿಂದ ವಿಳಾಸ ಇರುವ ಕಡೆಗೆ ಹೋಗುವುದು. ಅದರಂತೆಯೇ ದೇವರು ಜೀವರಾಶಿಗಳ ಮೂಲಕ ಅರ್ಪಿಸಿದ್ದನ್ನು ಸ್ವೀಕರಿಸುವನು. ದೇವರಿಗೆ ಕೊಟ್ಟಿದ್ದಾವುದೂ ಹೊರಟು ಹೋಗಿ ಬಿಡುವುದಿಲ್ಲ. ಅದು ವೃದ್ಧಿಯಾಗು ವುದು. ನಮ್ಮಲ್ಲಿರುವ ವಿದ್ಯೆಯನ್ನು ಕೊಟ್ಟರೆ ಆ ವಿದ್ಯೆ ಜಾಸ್ತಿಯಾಗುವುದು. ನಮ್ಮಲ್ಲಿರುವ ಸಂಗೀತವನ್ನು ಮತ್ತೊಬ್ಬನಿಗೆ ಕೊಟ್ಟರೆ ಅದು ನನ್ನಲ್ಲಿ ಮತ್ತೂ ಹೆಚ್ಚಾಗಿ ವೃದ್ಧಿಯಾಗುವುದು. ಕೊಟ್ಟಾದ ಮೇಲೆ ಉಳಿಯುವುದೇ ಶ್ರೇಷ್ಠವಾದುದು. ಮುಂಚೆ ನಾವು ಓದುವಾಗ ಪರೀಕ್ಷೆಯಲ್ಲಿ ಪಾಸು ಮಾಡಲು ಓದುತ್ತೇವೆ. ಪಾಸೆಲ್ಲವನ್ನೂ ಮಾಡಿ ಆದಮೇಲೆ ನಾನು ಏನನ್ನು ಕಲಿತೆನೋ ಅದನ್ನು ಹೇಳಿಕೊಡುವುದಕ್ಕೆ ಓದುತ್ತೇನೆ. ಆಗ ಪಾಸಾಗಲು ಓದಿದಷ್ಟು ಸಾಲದು. ಇನ್ನೊಬ್ಬನಿಗೆ ಹೇಳಿ ಕೊಡುವಾಗ ಹೇಳಿಕೊಡುವ ವಸ್ತುವನ್ನು ಆಮೂಲಾಗ್ರವಾಗಿ ತಿಳಿದುಕೊಂಡಿರಬೇಕು. ಹೇಳುವಾಗ ನಮ್ಮ ಹಲವು ಸಂದೇಹಗಳು ನಿವಾರಣೆ ಆಗುವುವು, ಭಾವನೆ ಸ್ಪಷ್ಟವಾಗಿ ನಿಲ್ಲುವುದು. ನಾವು ಕಲಿತಿದ್ದನ್ನು ಇನ್ನೊಬ್ಬನಿಗೆ ಪಾಠ ಹೇಳಿದ ಮೇಲೆ, ಆ ವಿಷಯವನ್ನು ನಾವು ಮತ್ತೂ ಚೆನ್ನಾಗಿ ತಿಳಿದುಕೊಳ್ಳುತ್ತೇವೆ. ಮುಂಚೆ ಬಗ್ಗಡವಾಗಿದ್ದು, ಕೊಟ್ಟಾದ ಮೇಲೆ ತಿಳಿಯಾಗುವುದು.

ಯಾವಾಗ ಅವನು ಯಜ್ಞವನ್ನು ಮಾಡುತ್ತಾನೆಯೋ, ಆಗ ಆ ವಸ್ತುವನ್ನು ಸಂಗ್ರಹಿಸುವುದಕ್ಕೆ ಮಾಡಿದ ಪಾಪಗಳಿಂದೆಲ್ಲಾ ಪಾರಾಗುತ್ತಾನೆ. ಈ ಜೀವನದಲ್ಲಿ ಯಾವ ಕೆಲಸವನ್ನು ಮಾಡಬೇಕಾದರೂ ಅದರಿಂದ ಅವರಿಗೆ ದುಃಖವಾಗುವುದು, ಕಷ್ಟವಾಗುವುದು. ಆದರೆ ಯಜ್ಞಕ್ಕಾಗಿ ಇದನ್ನು ಮಾಡಿದರೆ ನಾವು ಆ ದೋಷದಿಂದ ಪಾರಾಗುತ್ತೇವೆ. ಯಜ್ಞದೃಷ್ಟಿ ಎನ್ನುವುದು ಜ್ವಾಲೆಯಂತೆ ಪಾಪವನ್ನೆಲ್ಲಾ ಭಸ್ಮೀಭೂತ ಮಾಡುವುದು. ಶಸ್ತ್ರಚಿಕಿತ್ಸೆಯನ್ನು ಮಾಡುವುದಕ್ಕೆ ಮುಂಚೆ, ಅದಕ್ಕೆ ಸಂಬಂಧಪಟ್ಟ ಕರಣಗಳನ್ನೆಲ್ಲಾ ಬಿಸಿನೀರಿನಲ್ಲಿಟ್ಟು ಚೆನ್ನಾಗಿ ಕುದಿಸುವರು. ಆಗ ಅದಕ್ಕೆ ಅಂಟಿಕೊಂಡಿರುವ ಸೂಕ್ಷ್ಮ ವಾದ ಕ್ರಿಮಿಗಳು ನಾಶವಾಗುತ್ತವೆ. ಅದರಂತೆ ನಾವೊಂದು ಕೆಲಸ ಮಾಡುವಾಗ ಅನಿವಾರ್ಯವಾಗಿ ಹಲವು ಲೋಪದೋಷಗಳು ತಲೆದೋರಿರಬಹುದು. ಆದರೆ ಯಜ್ಞದ ಮೂಲಕ ಅವೆಲ್ಲಾ ಪರಿಶುದ್ಧ ವಾಗಿ ಹೋಗುತ್ತವೆ.

ಆದರೆ ನಾವು ಮಾಡುವ ಕರ್ಮಕ್ಕೆ ಯಾವಾಗ ಯಜ್ಞದ ಬೆಂಕಿ ತಾಕುವುದಿಲ್ಲವೋ ಕೇವಲ ಸ್ವಾರ್ಥಕ್ಕಾಗಿ ಆ ಕೆಲಸವನ್ನು ಮಾಡುತ್ತೇವೆಯೋ ಆಗ ಅದರಲ್ಲಿರುವ ಪಾಪದ ಕ್ರಿಮಿಗಳೆಲ್ಲಾ ನಮ್ಮ ಮೇಲೆ ತಮ್ಮ ಪ್ರಭಾವವನ್ನು ಬೀರುತ್ತವೆ. ಸ್ವಾರ್ಥವೇ ಪಾಪ. ದೇವರಿಗಾಗಿ ಮಾಡುವ ನಿಃಸ್ವಾರ್ಥ ಕ್ರಿಯೆಯೇ ಪುಣ್ಯ. ಇದೇ ಯಜ್ಞ. ಬೆಂಕಿಯುರಿಸಿ, ಮಂತ್ರ ಹೇಳಿ, ಅದಕ್ಕೆ ತುಪ್ಪ ಬಟ್ಟೆ ಬರೆ ಮುಂತಾದುವನ್ನು ಕೊಡುವುದು ಒಂದು ಸಣ್ಣ ಯಜ್ಞ. ಆದರೆ ನಿಜವಾದ ಯಜ್ಞವೆ ನಮ್ಮ ಪಾಲಿಗೆ ಬರುವ ಕರ್ಮವನ್ನೆಲ್ಲಾ ವಿರಾಟ್ ಪುರುಷನಿಗೆ ಅರ್ಪಿತವಾಗಲೆಂದು ಮಾಡುವುದು.

\begin{verse}
ಅನ್ನಾದ್ಭವಂತಿ ಭೂತಾನಿ ಪರ್ಜನ್ಯಾದನ್ನಸಂಭವಃ ।\\ಯಜ್ಞಾದ್ಭವತಿ ಪರ್ಜನ್ಯೋ ಯಜ್ಞಃ ಕರ್ಮಸಮುದ್ಭವಃ \versenum{॥ ೧೪ ॥}
\end{verse}

\begin{verse}
ಕರ್ಮಬ್ರಹ್ಮೋದ್ಭವಂ ವಿದ್ಧಿ ಬ್ರಹ್ಮಾಕ್ಷರಸಮುದ್ಭವಮ್ ।\\ತಸ್ಮಾತ್ ಸರ್ವಗತಂ ಬ್ರಹ್ಮ ನಿತ್ಯಂ ಯಜ್ಞೇ ಪ್ರತಿಷ್ಠಿತಮ್ \versenum{॥ ೧೫ ॥}
\end{verse}

{\small ಅನ್ನದಿಂದ ಪ್ರಾಣಿಗಳು ಹುಟ್ಟುತ್ತವೆ. ಮಳೆಯಿಂದ ಅನ್ನವಾಗುವುದು. ಯಜ್ಞದಿಂದ ಮಳೆಯಾಗುವುದು. ಕರ್ಮದಿಂದ ಯಜ್ಞ ಆಗುವುದು. ಕರ್ಮ ಬ್ರಹ್ಮದಿಂದ ಉಂಟಾಯಿತು. ಬ್ರಹ್ಮ ಅಕ್ಷರದಿಂದ ಆಯಿತು. ಸರ್ವವ್ಯಾಪಿಯಾಗಿರುವ ಬ್ರಹ್ಮ ಸದಾ ಯಜ್ಞದಲ್ಲಿ ನೆಲೆಸಿದೆ.}

ಆಹಾರವನ್ನು ತಿಂದು ಪ್ರಾಣಿಗಳು ಬೆಳೆಯುವುದನ್ನು ನೋಡುತ್ತೇವೆ. ಆಹಾರ ಬೆಳೆಯಬೇಕಾದರೆ ಮಳೆ ಬರಬೇಕು. ಮಳೆ ಇಲ್ಲದೆ ಇದ್ದರೆ ಯಾರಿಗೂ ಆಹಾರ ಲಭಿಸುವುದಿಲ್ಲ. ಇಡೀ ಪ್ರಾಣಿ ಪ್ರಪಂಚವೇ ತನ್ನ ಆಹಾರಕ್ಕೆ ಮಳೆಯನ್ನು ಎದುರು ನೋಡಬೇಕಾಗಿದೆ. ಈ ಮಳೆ ಯಜ್ಞದಿಂದ ಆಗುವುದು ಎನ್ನುವನು ಶ್ರೀಕೃಷ್ಣ. ಮಳೆಗೂ ಯಜ್ಞಕ್ಕೂ ಏನು ಸಂಬಂಧವಿದೆ ಎಂದು ಭಾವಿಸ ಬಹುದು. ಆದರೆ ಪ್ರಕೃತಿಗೂ ಜೀವಕ್ಕೂ ಒಂದು ಸಂಬಂಧವಿದೆ. ಯಾರು ತಮ್ಮ ಪಾಲಿಗೆ ಬಂದ ಕರ್ತವ್ಯಗಳನ್ನು ನಿರ್ವಂಚನೆಯಿಂದ ಸಮಷ್ಟಿಯ ಹಿತಕ್ಕೆ ಅರ್ಪಿಸುವರೋ ಅವರೆಲ್ಲಾ ಯಜ್ಞ ಮಾಡುತ್ತಿರುವರು. ಎಲ್ಲಿ ಇಂತಹ ಜನ ಹೆಚ್ಚಾಗಿ ಇರುವರೋ ಅಲ್ಲಿ ಕಾಲಕಾಲಕ್ಕೆ ಮಳೆ ಬೀಳುವುದು. ಬೆಳೆಯಾಗುವುದು. ಇದು ಪ್ರಕೃತಿ ನಿಯಮ. ಮಳೆ ಬರದೆ ಇದ್ದರೆ ಬಾಹ್ಯ ಪ್ರಕೃತಿಯಲ್ಲಿ ಎಲ್ಲಿಯೋ ಒಂದು ಕಡೆ ಅಡಚಣೆ ಇದೆ ಎಂದು ಭಾವಿಸುವೆವು. ಆ ಅಡಚಣೆ ಬಾಹ್ಯದಲ್ಲಿಇಲ್ಲ. ನಾವು ಮಾಡುವ ಕರ್ತವ್ಯವನ್ನು ಸರಿಯಾಗಿ ಮಾಡಿರುವೆವೆ, ಕೊಡುವುದನ್ನು ಸರಿಯಾಗಿ ಕೊಟ್ಟಿರುವೆವೆ ಎಂದು ಪ್ರಶ್ನೆ ಹಾಕಿಕೊಳ್ಳಬೇಕು. ತಪ್ಪನ್ನು ಹೊರಗೆ ಹುಡುಕುವುದಲ್ಲ. ಅದನ್ನು ನಮ್ಮಲ್ಲಿಯೇ ಶೋಧಿಸಿಕೊಳ್ಳ ಬೇಕಾಗಿದೆ. ಈ ಯಜ್ಞ ಹೇಗೆ ಆಯಿತು, ಪ್ರತಿಯೊಂದು ವರ್ಣ ಮತ್ತು ಆಶ್ರಮಕ್ಕೆ ಸೇರಿದವರು ತಮ್ಮ ಪಾಲಿನ ಕರ್ತವ್ಯವನ್ನು ಇದು ಭಗವದರ್ಪಿತವಾಗಲಿ ಎಂದು ಮಾಡುವುದರಿಂದ. ದೇವರನ್ನು ಮರೆತರೆ, ಸ್ವಾರ್ಥವೇ ನಮ್ಮ ಮನಸ್ಸಿನಲ್ಲಿದ್ದರೆ, ಕೇವಲ ಲಾಭವೇ ನಮ್ಮ ಉದ್ದೇಶವಾದರೆ, ನಾವು ಮಾಡುವುದೆಲ್ಲ ಬರೀ ಕರ್ಮವಾಗುವುದು. ಅದು ಯಜ್ಞವಾಗುವುದಿಲ್ಲ. ನಾವು ಮಾಡುವುದನ್ನೆಲ್ಲಾ ಯಜ್ಞಕ್ಕೆ ಪರಿವರ್ತಿಸಬೇಕೆಂದು ಶ್ರೀಕೃಷ್ಣ ಹೇಳುವನು. ಪ್ರತಿಯೊಂದು ಕರ್ಮವೂ ಒಂದು ಪೂಜೆ ಯಾಗಬಲ್ಲದು, ಅದರ ಮುಂದೆ ದೇವರನ್ನು ತಂದಾಗ. ಹಾಗೆ ಮಾಡಿದ ಕರ್ಮವೇ ಆದರ್ಶಕರ್ಮ. ಇದೇ ನಮ್ಮನ್ನು ಬಂಧನದಿಂದ ಪಾರು ಮಾಡುವುದು. ಯಾವಾಗ ಈ ದೃಷ್ಟಿಯನ್ನು ಮರೆಯುವೆವೋ ಆಗ ನಾವು ಮಾಡುವುದು ಕೇವಲ ನಮ್ಮ ಸ್ವಾರ್ಥವನ್ನು ತೃಪ್ತಿಪಡಿಸಿಕೊಳ್ಳುವುದಕ್ಕಾಗಿ ಆಗುವುದು. ಆಗ ನಾವು ಕರ್ಮಜಾಲದಲ್ಲಿ ಸಿಕ್ಕಿಕೊಳ್ಳುವೆವು.

ಈ ಕರ್ಮ ಯಾರಿಂದ ಜಾರಿಗೆ ಬಂತು ಎಂದು ಕೇಳಿದರೆ ಸೃಷ್ಟಿಕರ್ತನೆ ಇದಕ್ಕೆ ಕಾರಣ. ಅವನು ಪ್ರಪಂಚವನ್ನು ಸೃಷ್ಟಿ ಮಾಡಿದಾಗ ಪ್ರತಿಯೊಂದಕ್ಕೂ ಒಂದೊಂದು ಸ್ವಭಾವ ಮತ್ತು ಕರ್ಮವನ್ನು ಕೊಟ್ಟಿರುವನು. ಅದು ಈ ಕರ್ಮದ ಮೂಲಕವಾಗಿಯೇ ಭಗವಂತನನ್ನು ಉಪಾಸನೆ ಮಾಡಬೇಕು. ಈ ಸೃಷ್ಟಿ ನಿಂತಿರುವುದೇ ಅನ್ಯೋನ್ಯ ಕರ್ಮದ ಆಶ್ರಯದ ಮೇಲೆ. ಈ ವಿರಾಟ್ ಯಜ್ಞಕ್ಕೆ ಎಲ್ಲರೂ ತಮತಮಗೆ ಏನು ಸಾಧ್ಯವೋ ಅದನ್ನು ಕೊಡುತ್ತಿದ್ದರೆ ತಾನೆ ಎಲ್ಲರಿಗೂ ಯಾವುದು ಬೇಕೊ ಅದು ದೊರಕುವುದು. ಕೊಡದೆ ಕೇಳಿದರೆ ಹೇಗೆ ಸಿಕ್ಕುವುದು? ಬ್ಯಾಂಕಿನಲ್ಲಿ ದುಡ್ಡನ್ನೇ ಹಾಕಿಲ್ಲ, ಆದರೆ ಚೆಕ್ಕನ್ನು ತೆಗೆದುಕೊಂಡು ಹೋಗುತ್ತೇನೆ ಅಲ್ಲಿಂದ ದುಡ್ಡನ್ನು ತರಲು! ಪ್ರತಿಯೊಬ್ಬನೂ ಅವನು ಮಾಡಬೇಕಾದ ಕೆಲಸವನ್ನು ಮಾಡಿದರೆ ಸಿಕ್ಕಬೇಕಾದುದು ಸಿಕ್ಕುವುದು.

ಇಲ್ಲಿ ಯಾವುದನ್ನು ಬ್ರಹ್ಮ ಎನ್ನುತ್ತೇವೆಯೋ ಅವನು ಸಗುಣ ಬ್ರಹ್ಮ, ಈಶ್ವರ. ಅವನನ್ನು ಯಾವ ಹೆಸರಿನಿಂದ ಬೇಕಾದರೂ ಕರೆಯಬಹುದು. ಅವನೇ ಸೃಷ್ಟಿಯ ಕರ್ತೃ, ಪಾಲಕ, ಸಂಹಾರಕ. ಜೀವರಾಶಿಗಳಿಗೆ ಅವರವರ ಕರ್ಮಾನುಸಾರ ಫಲಗಳನ್ನು ಕೊಡುವವನು. ಈಶ್ವರ, ಜೀವ, ಜಗತ್ತು ಇವು ಮೂರೂ ಏಕಕಾಲದಲ್ಲಿ ಇರುತ್ತವೆ. ಒಂದು ಇದ್ದರೆ ಮತ್ತೊಂದು ಇರಲೇ ಬೇಕು. ಒಂದಿಲ್ಲದೆ ಮತ್ತೊಂದು ಇರಲಾರದು. ಈ ಸಗುಣ ಬಂದಿರುವುದೇ ನಿರ್ಗುಣ ಬ್ರಹ್ಮನಿಂದ. ನಿರ್ಗುಣ ದೇಶ ಕಾಲ ನಿಮಿತ್ತಾತೀತವಾಗಿರುವುದು. ಅದನ್ನು ಗ್ರಹಿಸುವುದಕ್ಕೆ ದೇಶ ಕಾಲ ನಿಮಿತ್ತದಲ್ಲಿ ಕೆಲಸ ಮಾಡುವ ಸಾಂತ ಬುದ್ಧಿಗೆ ಸಾಧ್ಯವಿಲ್ಲ. ಸಗುಣ ಬ್ರಹ್ಮ ಪರಮೇಶ್ವರನಂತೆ ಈ ಪ್ರಪಂಚವನ್ನೆಲ್ಲಾ ವ್ಯಾಪಿಸಿಕೊಂಡಿರುವನು. ಒಳಗೆ ಹೊರಗೆಲ್ಲಾ ಓತಪ್ರೋತನಾಗಿರುವನು. ಈ ಸೃಷ್ಟಿ ನಿಂತಿರುವುದೇ ಯಜ್ಞದ ಆಧಾರದ ಮೇಲೆ, ಕೊಡುವುದರ ಮೇಲೆ. ಈ ಪ್ರಪಂಚದಲ್ಲಿ ಎಲ್ಲಾ ಕೊಡುತ್ತಿವೆ. ಸಸ್ಯ, ಪ್ರಾಣಿ, ಪಕ್ಷಿ, ಮನುಷ್ಯ. ಕೆಲವರು ಕೊಡುವಾಗ ಸಂತೋಷದಿಂದ ಕೊಡುತ್ತಿರುವರು. ಹಲವರು ಅಯ್ಯೋ ಕೊಡಬೇಕಲ್ಲಾ ಎಂದು ಗೊಣಗಾಡಿಕೊಂಡು ಕೊಡುವರು. ಕೊಡದೆ ಯಾರಾದರೂ ಭಂಡರಾದರೆ ದೇವರು ಅವರಿಂದ ಇನ್ನು ಬೇರೆಬೇರೆ ವಿಧದಿಂದ ಕಿತ್ತುಕೊಳ್ಳುವನು. ಸಮಷ್ಟಿಯ ವ್ಯಾಪಾರ ಮುಂದುವರಿಯಬೇಕಾದರೆ ವ್ಯಷ್ಟಿಯು ಕೊಡುವುದನ್ನು ಕೊಡಬೇಕು. ಸಮುದ್ರ ಮಳೆ ಕರೆಯುವ ಮೋಡವನ್ನು ಕಳುಹಿಸಬೇಕಾದರೆ ಹಗಲು ರಾತ್ರಿ ನದಿಗಳು ಅದಕ್ಕೆ ನೀರನ್ನು ಒದಗಿಸು ತ್ತಿರಬೇಕು. ಭೂಮ ಒಂದು ರೀತಿ ಕೊಡುತ್ತಿದೆ. ಅಲ್ಪ ಒಂದು ರೀತಿ ಕೊಡುತ್ತಿದೆ. ಈ ಕೊಡುತ್ತಿರುವ ನಿಯಮದ ಮೇಲೆಯೇ ಸೃಷ್ಟಿ ಬದುಕಿರುವುದು.

\begin{verse}
ಏವಂ ಪ್ರವರ್ತಿತಂ ಚಕ್ರಂ ನಾನುವರ್ತಯತೀಹ ಯಃ ।\\ಅಘಾಯುರಿಂದ್ರಿಯಾರಾಮೋ ಮೋಘಂ ಪಾರ್ಥ ಸ ಜೀವತಿ \versenum{॥ ೧೬ ॥}
\end{verse}

{\small ಪಾರ್ಥ, ಹೀಗೆ ಪ್ರವರ್ತಿತವಾಗಿರುವ ಜಗಚ್ಚಕ್ರವನ್ನು ಯಾರು ಇಲ್ಲಿ ಅನುಕರಿಸುವುದಿಲ್ಲವೋ ಅವನು ಪಾಪಿಯು, ಇಂದ್ರಿಯ ಸುಖಾಭಿಲಾಷಿಯೂ ಆಗಿ ಅವನ ಬಾಳು ವ್ಯರ್ಥವಾಗುವುದು.}

ಪ್ರಪಂಚ ನಿಂತಿರುವುದೇ ಯಜ್ಞದ ಆಧಾರದ ಮೇಲೆ. ಪಂಚಭೂತಗಳೂ, ತರುಲತೆಗಳೂ, ಪಶುಪಕ್ಷಿಗಳು, ಮಾನವರೆಲ್ಲರೂ ಇದರಲ್ಲಿ ಸಹಕರಿಸಬೇಕು. ತಮ್ಮ ಪಾಲಿನದನ್ನು ಇದಕ್ಕೆ ಅರ್ಪಿಸ ಬೇಕು. ನಾವು ದೇಹಕ್ಕೆ ಆಹಾರವನ್ನು ಕೊಡುತ್ತೇವೆ. ದೇಹ ಆಹಾರವನ್ನು ಅರಗಿಸಿಕೊಂಡು ಅದನ್ನು ದೇಹದ ಅಂಗೋಪಾಂಗಗಳಿಗೆ ಬೇಕಾದ ವಸ್ತುವನ್ನಾಗಿ ಮಾರ್ಪಡಿಸುತ್ತದೆ. ಅದರಂತೆಯೇ ಪ್ರತಿ ಯೊಬ್ಬನೂ ತನ್ನ ಕೊಡುಗೆಯನ್ನು ಕೊಟ್ಟರೆ ಸಮಷ್ಟಿ ಬಾಳುವೆ ಅದನ್ನೆಲ್ಲಾ ತೆಗೆದುಕೊಂಡು ಪ್ರತಿಯೊಬ್ಬನಿಗೂ ಏನೇನು ಬೇಕೋ ಅದನ್ನು ತಯಾರು ಮಾಡಿಕೊಡುವುದು. ಪ್ರತಿಯೊಬ್ಬನೂ ಇದರೊಂದಿಗೆ ಸಹಕರಿಸಬೇಕು. ಸಸ್ಯ ಪ್ರಾಣಿವರ್ಗಗಳು ತಮಗೆ ತಿಳಿಯದೆ ಈ ಸಮಷ್ಟಿ ಬಾಳುವೆಗೆ ಅರ್ಪಣ ಮಾಡುತ್ತವೆ. ವಿಚಾರಪರನಾಗಿರುವ ಮನುಷ್ಯ ಇದನ್ನು ತಿಳಿದು ಮಾಡಬೇಕಾಗಿದೆ. ಯಾವಾಗ ಕೊಡುವುದನ್ನು ನಿಲ್ಲಿಸಿ ಬರುವುದರ ಕಡೆ ಮಾತ್ರ ಗಮನ ಕೊಡುವನೊ ಅವನನ್ನು ಪ್ರಕೃತಿ ಬೇಗ ಆಚೆಗೆ ಎಸೆಯುವುದು.

ಯಜ್ಞದ ನಿಯಮಕ್ಕೆ ವಿರೋಧವಾಗಿ ಹೋಗುವವನೇ ಪಾಪಿ. ಅವನು ಕೇವಲ ತನ್ನನ್ನು ಮಾತ್ರ ಚಿಂತಿಸುತ್ತ ಇರುವನು. ಹೊರಗಿನಿಂದ ತನ್ನ ತೃಪ್ತಿಗೆ ಏನು ಬೇಕೋ ಅದನ್ನು ಹೀರುವುದರಲ್ಲಿ ಮಾತ್ರ ಅವನು ನಿರತನಾಗಿರುವನು. ಇಂತಹ ಸ್ವಾರ್ಥಿಯೇ ಪಾಪಿ. ಸಮಷ್ಟಿಯ ಜೀವನಕ್ಕೆ ದುಡಿಯದವನ ಬಾಳು ಸಂಕುಚಿತವಾಗುವುದು. ಅಂಥವನು ಇಂದ್ರಿಯಾರಾಮನಾಗುತ್ತಾನೆ. ಕೇವಲ ತನ್ನ ಇಂದ್ರಿಯ ಸುಖವನ್ನು ಮಾತ್ರ ಆಲೋಚಿಸುತ್ತಾನೆ. ಪ್ರತಿಯೊಂದು ಇಂದ್ರಿಯ ಸುಖವೂ ನಮ್ಮನ್ನು ಬಾಹ್ಯ ಸಂವೇದನೆಗೆ ಕಟ್ಟಿಹಾಕುವುದು. ನಾವೊಂದು ಮೃಗಸದೃಶರಾಗುವೆವು. ಅದೊಂದೇ ಅಲ್ಲ. ಅದಕ್ಕಿಂತ ಕೀಳಾಗುವೆವು ನಾವು. ಪ್ರಾಣಿಗಳಿಗೆ ಯಾವ ಉತ್ತಮ ಆಲೋಚನೆಯನ್ನೂ ಮಾಡುವುದಕ್ಕೆ ಆಗುವು ದಿಲ್ಲ. ಇಂದ್ರಿಯದಲ್ಲಿ ಅವು ಬದ್ಧವಾಗಿವೆ. ಆದಕಾರಣ ಅದಕ್ಕಾಗಿ ಬಾಳಿ ಸಾಯಬೇಕಾಗುವುದು. ಆದರೆ ಮನುಷ್ಯನಿಗಾದರೋ ಪ್ರಾಣಿಗಳಲ್ಲಿ ಇಲ್ಲದ ವಿಚಾರವನ್ನು ದೇವರು ಕೊಟ್ಟು ಕಳುಹಿಸಿರುವನು. ಅದರಿಂದ ಪೂರ್ವಾಪರವನ್ನು ತಿಳಿದುಕೊಳ್ಳುವುದು ಸಾಧ್ಯವಾಗುವುದು. ಪ್ರಾಣಿಗಳಿಗೆ ಯಾವುದು ಕಾಣದೋ ಅದು ಮನುಷ್ಯನಿಗೆ ಕಾಣುವುದು. ಆದರೂ ದೇವರು ಕೊಟ್ಟ ವಿಚಾರಶಕ್ತಿಯನ್ನು ಇಂದ್ರಿಯಾತೀತವಾದ ವಿಷಯಗಳನ್ನು ತಿಳಿದುಕೊಳ್ಳುವುದಕ್ಕೆ ಉಪಯೋಗಿಸದೆ, ಕೇವಲ ವಿಷಯ ಸುಖ ಸಂಗ್ರಹಕ್ಕೆ ಮಾತ್ರ ಬಳಸಿದರೆ ಅಂತಹ ಅನರ್ಘ್ಯ ಶಕ್ತಿಯನ್ನು ನಾವು ವ್ಯರ್ಥಮಾಡಿಕೊಳ್ಳುವೆವು.

ಮನುಷ್ಯನ ಬಾಳೇ ವ್ಯರ್ಥವಾಗುವುದು ಆಗ. ದೇವರು ಕೊಟ್ಟ ವಿಚಾರವನ್ನು ಉಪಯೋಗಿಸಿ ಸೃಷ್ಟಿಯ ರಹಸ್ಯವನ್ನು ಅರಿತು ಭವಬಂಧನದಿಂದ ಪಾರಾಗುವುದನ್ನು ಬಿಟ್ಟು, ಆತ ಇಂದ್ರಿಯ ಸುಖಕ್ಕೆ ಗುಲಾಮನಾಗಿ ಪದೇಪದೇ ಈ ಸಂಸಾರ ಚಕ್ರಕ್ಕೆ ಸಿಕ್ಕುತ್ತಿರುವನು.

\begin{verse}
ಯಸ್ತ್ವಾತ್ಮರತಿರೇವ ಸ್ಯಾದಾತ್ಮತೃಪ್ತಶ್ಚ ಮಾನವಃ ।\\ಆತ್ಮನ್ಯೇವ ಚ ಸಂತುಷ್ಟಸ್ತಸ್ಯ ಕಾರ್ಯಂ ನ ವಿದ್ಯತೇ \versenum{॥ ೧೭ ॥}
\end{verse}

{\small ಯಾರು ಆತ್ಮರತಿಯಾಗಿರುವನೋ ಆತ್ಮತೃಪ್ತನಾಗಿರುವನೋ ಆತ್ಮದಲ್ಲಿಯೇ ಸಂತೋಷ ಕಾಣುವನೋ ಅವನಿಗೆ ಯಾವ ಕರ್ತವ್ಯವೂ ಇಲ್ಲ.}

ಒಬ್ಬ ಸಂಸಾರದ ವಿಕಾಸದ ಏಣಿಯಲ್ಲಿ ತುತ್ತತುದಿಗೆ ಏರಿರುವನು. ಅವನು ಬಾಹ್ಯ ಪ್ರಪಂಚ ದಿಂದ ಏನನ್ನೂ ಇಚ್ಛಿಸುವುದಿಲ್ಲ. ಅವನು ತನ್ನ ಅಂತರಾಳದಲ್ಲಿರುವ ಪರಮಾತ್ಮನ ಚಿಂತನೆ ಯಲ್ಲಿಯೇ ಮುಳುಗಿಹೋಗಿದ್ದಾನೆ. ಅವನಿಗೆ ವಿಷಯ ಇಂದ್ರಿಯದ ಮರಳುಕಾಡಿನಲ್ಲಿ ಸಿಕ್ಕುವ ಯಾವ ಸುಖವೂ ಬೇಕಾಗಿಲ್ಲ. ಏಕೆಂದರೆ ಇದು ಅನಂತರ ಯಾವ ರೀತಿ ಪರ್ಯವಸಾರವಾಗುವುದು ಎಂಬುದನ್ನು ಅವನು ಚೆನ್ನಾಗಿ ಬಲ್ಲನು. ಅವನು ಅದಕ್ಕಾಗಿ ಹೊರಗಿನದನ್ನು ತನ್ನ ಸುಖಕ್ಕಾಗಿ ಹುಡುಕಿಕೊಂಡೂ ಹೋಗುವುದಿಲ್ಲ. ಅವನು ತನ್ನಲ್ಲಿರುವ ಪರಮಾತ್ಮನಲ್ಲಿಯೇ ತೃಪ್ತನಾಗಿದ್ದಾನೆ. ಯಾರು ಹೊರಗಡೆಯಿಂದ ತೆಗೆದುಕೊಳ್ಳುವರೊ ಅವರು ಕೊಡಬೇಕಾಗಿದೆ. ಕೊಡದೆ ತೆಗೆದು ಕೊಂಡರೆ, ಅದನ್ನು ಪುನಃ ನಮ್ಮಿಂದ ವಸೂಲಿ ಮಾಡಲಾಗುವುದು. ತೆಗೆದುಕೊಳ್ಳುವುದು ಸಾಲ ಮಾಡಿ ದಂತೆ. ಅದನ್ನು ಕೊಟ್ಟೇ ತೀರಿಸಬೇಕು. ಸಾಲಮಾಡಿ ತಪ್ಪಿಸಿಕೊಂಡು ಹೋಗುವುದಕ್ಕೆ ಆಗುವುದಿಲ್ಲ. ಬಾಹ್ಯ ಪ್ರಪಂಚದಲ್ಲಿ ಮಾಡಿದ ಸಾಲದಿಂದ ತಪ್ಪಿಸಿಕೊಳ್ಳುವುದೇ ಕಷ್ಟ. ಅದರಲ್ಲಿ ಪ್ರಕೃತಿಯ ಕೈಯಿಂದ ನಮ್ಮ ಸುಖಕ್ಕೆ ಬರಬೇಕಾಗಿರುವುದನ್ನು ತೆಗೆದುಕೊಂಡು ನಾವು ಏನನ್ನೂ ಕೊಡುವುದಿಲ್ಲ ಎಂದರೆ ಆಗುವುದಿಲ್ಲ. ಆದರೆ ಆತ್ಮತೃಪ್ತ ಆತ್ಮರತಿ ಯಾವುದಕ್ಕೂ ಕೈಯೊಡ್ಡಿಲ್ಲ. ಅವನ ಬಾಳಿಗೆ ಬಾಹ್ಯ ಪ್ರಪಂಚದಿಂದ ಬರುವ ವಸ್ತುಗಳು ಅನಾವಶ್ಯಕ. ಯಾವಾಗ ಅವನು ತೆಗೆದುಕೊಂಡಿಲ್ಲವೋ ಆಗ ಅವನು ಏನನ್ನೂ ಕೊಡಬೇಕಾಗಿಲ್ಲ. ಕೊಡು ಎಂದು ಯಾರೂ ಅವನನ್ನು ಬಲಾತ್ಕಾರ ಮಾಡಲಾಗುವುದಿಲ್ಲ. ಆದರೆ ಇವನು ಸುಮ್ಮನೆ ಇರುವುದಿಲ್ಲ. ಏನನ್ನಾದರೂ ಕೊಡುತ್ತಲೇ ಇರು ತ್ತಾನೆ. ಬಲಾತ್ಕಾರಕ್ಕೆ ತುತ್ತಾಗಿ ಕೊಡುವುದಿಲ್ಲ. ಸ್ವೇಚ್ಛೆಯಿಂದ ಕೊಡುವನು. ಎಂತಹ ಅನರ್ಘ್ಯ ರತ್ನಗಳನ್ನು ಮಾನವನ ಮುಂದೆ ಎಸೆಯುತ್ತಾನೆ! ದನದಂತೆ ವಿಷಯಸುಖವನ್ನು ಮೇಯುತ್ತಿರುವ ಜೀವಿಗಳಿಗೆ ಇಂದ್ರೀಯಾತೀತ ಅನುಭವದ ಸವಿಯನ್ನು ತೋರುವನು. ಮರ್ತ್ಯ ಅಮೃತನಾಗುವ ವಿದ್ಯೆಯನ್ನು ನೀಡುವನು. ಹೀಗೆ ಕೊಡುವಾಗ ಅವನು ಏನನ್ನೂ ಇಚ್ಛಿಸುವುದಿಲ್ಲ. ಏನನ್ನೂ ಪ್ರತಿಬಯಸುವುದೂ ಇಲ್ಲ. ತನ್ನಂತೆ ಎಲ್ಲರೂ ಅಮೃತತ್ವಕ್ಕೆ ಭಾಗಿಗಳಾಗಿ ಎಂದು ಕೊಡುವನು. ಆದರೆ ಇದು ಕರ್ತವ್ಯವೆಂದು ಭಾವಿಸುವುದಿಲ್ಲ. ಕರ್ತವ್ಯ ಎಂದ ತಕ್ಷಣವೇ ನಾವು ಅದನ್ನು ಮಾಡಲೇ ಬೇಕು, ತಪ್ಪಿಸಿಕೊಂಡು ಹೋಗಲು ಸಾಧ್ಯವೇ ಇಲ್ಲ ಎಂಬ ಭಾವನೆ ಬರುವುದು. ಕರ್ತವ್ಯ ಎಂದೊಡನೆ ಗಾಡಿಗೆ ಕಟ್ಟಿದ ಎತ್ತು ಜ್ಞಾಪಕಕ್ಕೆ ಬರುವುದು. ಅದು ಎಳೆಯಲೇ ಬೇಕು, ಇಲ್ಲದೆ ಇದ್ದರೆ ಸವಾರನ ಚಾವುಟಿ ಏಟು ಕಾದಿದೆ. ಮುಕ್ತನನ್ನು ಯಾರೂ ಬಲಾತ್ಕಾರ ಮಾಡುವುದಕ್ಕೆ ಆಗುವುದಿಲ್ಲ. ಅವನ ಖುಷಿ, ಕೊಟ್ಟರೆ ಕೊಡುತ್ತಾನೆ ಇಲ್ಲದೆ ಇದ್ದರೆ ಇಲ್ಲ. ಕೆಲವು ವೇಳೆ ವಿಶ್ವಾನುಕಂಪದಿಂದ ಅವನೇ ಕೆಲವು ಕೆಲಸಗಳನ್ನು ಮಾಡುತ್ತಾನೆ. ಮತ್ತೆ ಕೆಲವು ವೇಳೆ ಎಲ್ಲಾ ಕಡೆಯಲ್ಲಿಯೂ ಎಲ್ಲಾ ನಾಮರೂಪ ಗಳ ಹಿಂದೆಯೂ ವಾಸುದೇವನನ್ನು ಕಾಣುವಾಗ, ಅವನು ತನ್ನ ಹೃದಯಾಂತರಾಳದಲ್ಲಿಯೂ ಮಂಡಿಸಿರುವಾಗ, ಕೊಡುವವನು ಯಾರು, ತೆಗೆದುಕೊಳ್ಳುವವನು ಯಾರು ಎಂದು ಯಾವ ಕೆಲಸಕ್ಕೂ ಕೈ ಹಾಕುವುದಿಲ್ಲ.

\begin{verse}
ನೈವ ತಸ್ಯ ಕೃತೇನಾರ್ಥೋ ನಾಕೃತೇನೇಹ ಕಶ್ಚನ ।\\ನ ಚಾಸ್ಯ ಸರ್ವಭೂತೇಷು ಕಶ್ಚಿದರ್ಥವ್ಯಪಾಶ್ರಯಃ \versenum{॥ ೧೮ ॥}
\end{verse}

{\small ಇಲ್ಲಿ ಕರ್ಮ ಮಾಡಿದರೆ ಇವನಿಗೆ ಯಾವ ಲಾಭವೂ ಇಲ್ಲ. ಮಾಡದೆ ಇದ್ದರೆ ಯಾವ ನಷ್ಟವೂ ಇಲ್ಲ. ಇವನಿಗೆ ಯಾರಿಂದಲೂ ಯಾವ ಪ್ರಯೋಜನದ ಸಂಬಂಧವೂ ಇಲ್ಲ.}

ಆತ್ಮತೃಪ್ತನಾಗಿರುವವನಿಗೆ ಹೊರಗೆ ಕೆಲಸ ಮಾಡಿದರೆ ಅವನಿಗೆ ಯಾವ ಹೊಸ ಅನುಭವವೂ ಆಗುವಂತಿಲ್ಲ. ತಾನು ಆಗಲೆ ಪೂರ್ಣ ಸ್ಥಿತಿಯನ್ನು ಪಡೆದುಕೊಂಡಿರುವನು. ಅದನ್ನು ಹೆಚ್ಚು ಪೂರ್ಣನಾಗಿ ಮಾಡುವುದಕ್ಕೆ ಆಗುವುದಿಲ್ಲ. ಒಂದು ಪಾತ್ರೆ ಆಗಲೆ ತುಂಬಿದೆ. ಇನ್ನು ಹೆಚ್ಚನ್ನು ಸುರಿದರೆ ಅದೆಲ್ಲಾ ಚೆಲ್ಲಿ ಹೋಗುವುದು. ಅದರಿಂದ ಏನೂ ಪ್ರಯೋಜನವಾಗುವುದಿಲ್ಲ. ಸಾಧಾರಣ ಮನುಷ್ಯ ಲೌಕಿಕವಾದ ದೃಷ್ಟಿಯಿಂದ ಮಾಡಿದಾಗ ಅದು ಅವನ ಕಾಮನೆಗಳನ್ನು ಈಡೇರಿಸುವುದು. ಆದರೆ ಆತ್ಮತೃಪ್ತನಿಗೆ ಯಾವ ಕಾಮನೆಗಳೂ ಇಲ್ಲ.

ಕೆಲಸ ಮಾಡದೆ ಇದ್ದರೆ ನಷ್ಟವಿಲ್ಲ ಆತ್ಮತೃಪ್ತನಿಗೆ. ಅವನು ಪೂರ್ಣತೆಯನ್ನು ಮುಟ್ಟಿರುವನು. ಇನ್ನು ಮೇಲೆ ಯಾರೂ ಅವನಿಂದ ಅದನ್ನು ಕಿತ್ತುಕೊಳ್ಳುವುದಕ್ಕೆ ಆಗುವುದಿಲ್ಲ. ಒಬ್ಬ ಓದಿ ಪಾಸು ಮಾಡಿ ಡಿಗ್ರಿ ತೆಗೆದುಕೊಂಡಿರುವನು. ಅನಂತರ ಯಾರೂ ಅವನಿಂದ ಹಿಂದಕ್ಕೆ ಆ ಡಿಗ್ರಿಯನ್ನು ತೆಗೆದುಕೊಳ್ಳಲಾರರು. ಒಂದು ಹೀನ ಲೋಹ ಪರುಷ ಶಿಲೆಗೆ ತಾಕಿ ಹೊನ್ನಾಗಿದೆ. ಅನಂತರ ಅದು ಆ ಸಂಗವನ್ನು ಬಿಟ್ಟರೆ ತನ್ನ ಹಿಂದಿನ ಸ್ಥಿತಿಗೆ ಬರುವುದಿಲ್ಲ. ಇನ್ನು ಮೇಲೆ ಅವನು ಕರ್ಮಾದಿಗಳನ್ನು ಮಾಡದೆ ಇದ್ದರೆ ನಷ್ಟವಿಲ್ಲ.

ಇವನು ಪ್ರಪಂಚದಲ್ಲಿರುವಾಗ ಎಲ್ಲರನ್ನೂ ಪ್ರೀತಿಸುತ್ತಾನೆ. ಹಲವರು ಇವನಿಗೆ ಹಲವು ವಸ್ತುಗಳನ್ನು ಕೊಡುತ್ತಾರೆ. ಇವನು ಕೆಲವು ವೇಳೆ ಇವನ್ನು ಸ್ವೀಕರಿಸಬಹುದು. ಸ್ವೀಕರಿಸದೆ ಇದ್ದರೆ ಕೊಡುವವನು ವ್ಯಥೆಪಡುವನು ಎಂದು ಕೆಲವು ವೇಳೆ ತೆಗೆದುಕೊಳ್ಳುವನು. ಆದರೆ ಅವನು ಯಾರಿಗೂ ಬದ್ಧನಲ್ಲ. ಯಾವ ವಸ್ತುನಿನಲ್ಲೂ ಆಸಕ್ತನಲ್ಲ. ಇವನು ಪ್ರಪಂಚದಲ್ಲಿ ತಾನು ಯಾರ ಹಂಗೂ ಇಲ್ಲದೆ ಜೀವಿಸಬಹುದು ಎಂದು ಹೇಳಬಲ್ಲ. ಇವನ ಹೃದಯದಲ್ಲಿ ಯಾವ ವಾಂಛನೆಗಳೂ ಇಲ್ಲ. ಇನ್ನು ಅವನ್ನು ತೃಪ್ತಿಪಡಿಸುವವನ ಕಾಲನ್ನು ಏತಕ್ಕೆ ಕಟ್ಟಬೇಕು ಅವನು?

\begin{verse}
ತಸ್ಮಾದಸಕ್ತಃ ಸತತಂ ಕಾರ್ಯಂ ಕರ್ಮ ಸಮಾಚರ ।\\ಅಸಕ್ತೋ ಹ್ಯಾಚರನ್ ಕರ್ಮ ಪರಮಾಪ್ನೋತಿ ಪೂರುಷಃ \versenum{॥ ೧೯ ॥}
\end{verse}

{\small ಆದಕಾರಣ ಅನಾಸಕ್ತನಾಗಿ ಯಾವಾಗಲೂ ನಿಯತವಾದ ಕರ್ಮವನ್ನು ಮಾಡು. ಏಕೆಂದರೆ ಅನಾಸಕ್ತನಾದ ಮನುಷ್ಯ ಕರ್ಮವನ್ನು ಮಾಡುತ್ತ ಮುಕ್ತಿಯನ್ನು ಪಡೆಯುತ್ತಾನೆ.}

ಶ್ರೀಕೃಷ್ಣ ಅರ್ಜುನನಿಗೆ ಆದಕಾರಣ ನೀನು ಯುದ್ಧವನ್ನು ಮಾಡಬೇಕು ಎನ್ನುತ್ತಾನೆ. ಎಂದರೆ ಅರ್ಜುನ ಆತ್ಮತೃಪ್ತನಾಗಿ, ಆತ್ಮರತಿಯಾಗಿ ಇರುವವನಲ್ಲ. ಅವನು ಹೊರಗಿನಿಂದ ಬರುವ ವಸ್ತುವಿಗೆ ಕಾಯಬೇಕಾಗಿದೆ. ಅದರಿಂದ ತನ್ನ ಜೀವನವನ್ನು ಪೋಷಿಸಿಕೊಳ್ಳಬೇಕಾಗಿದೆ. ಎಲ್ಲವನ್ನೂ ಬಲ್ಲ ಸಂನ್ಯಾಸಿಯಂತೆ ಜ್ಞಾನಿಯಂತೆ ಇರುವುದಕ್ಕೆ ಆಗುವುದಿಲ್ಲ.

ಯಾವಾಗಲೂ ನಿಯತವಾದ ಕರ್ಮವನ್ನು ಮಾಡು ಎನ್ನುತ್ತಾನೆ. ಕ್ಷತ್ರಿಯನಿಗೆ ಧರ್ಮಕ್ಕಾಗಿ ಯುದ್ಧವನ್ನು ಮಾಡುವುದು ನಿಯತ ಕರ್ಮ, ಅದು ಕಾಮ್ಯ ಕರ್ಮವಲ್ಲ. ಕಾಮ್ಯ ಕರ್ಮವಾದರೆ ತನಗೆ ಬೇಕಾದರೆ ಮಾಡಬಹುದು, ಇಲ್ಲದೆ ಇದ್ದರೆ ಬಿಡಬಹುದಾಗಿತ್ತು. ಕ್ಷತ್ರಿಯನ ಅತ್ಯಂತ ಆವಶ್ಯಕ ವಾಗಿರುವ ಕರ್ಮ ಇದು. ಯಾವಾಗಲೂ ಆ ಕೆಲಸವನ್ನು ಮಾಡಲು ಅರ್ಜುನ ಸಿದ್ಧನಾಗಿರಬೇಕು. ಅದರಲ್ಲಿಯೂ ಈಗ ಕೌರವರ ಕಡೆಯವರು ಧರ್ಮಕ್ಕೆ ಬಗ್ಗದೆ ಕೇವಲ ದುಂಡಾವರ್ತಿಯಿಂದಲೇ ಪಾಂಡವರಿಗೆ ಕೊಡಬೇಕಾಗಿರುವುದನ್ನು ಕೊಡದೆ ಇರುವಾಗ ಎಂದಿಗೂ ಅವರಿಗೆ ಬೆನ್ನನ್ನು ತೋರ ಕೂಡದು. ಹಾಗೆ ಬೆನ್ನನ್ನು ತೋರುವುದು ಅಧರ್ಮಕ್ಕೆ ಸಹಾಯ ಮಾಡಿದಂತೆ.

ಆ ಕೆಲಸವನ್ನು ಬೇಕಾದರೆ ಫಲಾಪೇಕ್ಷೆಯಿಂದ ಮಾಡಬಹುದು. ಇಲ್ಲದೆ ಅನಾಸಕ್ತನಾಗಿ ಮಾಡ ಬಹುದು. ಇವೆರಡರಲ್ಲಿ ಅರ್ಜುನ ಯಾವ ದೃಷ್ಟಿಯಿಂದಲಾದರೂ ಅಂತೂ ಆ ಕೆಲಸವನ್ನು ಮಾಡಬೇಕು; ಬಿಡುವುದಕ್ಕೆ ಆಗುವುದಿಲ್ಲ. ಅನಾಸಕ್ತಿಯಿಂದ ಮಾಡುವುದು ಮೇಲು ಎಂದು ಶ್ರೀಕೃಷ್ಣ ಸಾರುತ್ತಾನೆ. ಯಾವಾಗ ಅನಾಸಕ್ತನಾಗಿ ಕರ್ಮವನ್ನು ಮಾಡುತ್ತಾನೆಯೋ ಆಗ ಅವನು ಬದ್ಧನಾಗು ವುದಿಲ್ಲ. ಕೇವಲ ಅನಾಸಕ್ತನಾಗಿ ತನ್ನ ಪಾಲಿಗೆ ಬಂದ ಕರ್ಮವನ್ನು ಮಾಡುತ್ತಿದ್ದರೇ ಸಾಕು, ಅದರಿಂದ ಮುಕ್ತಿ ದೊರೆಯುವುದು ಎನ್ನುವನು. ಇದೊಂದು ಗೀತೆಯಲ್ಲಿ ಬರುವ ಅಸದೃಶವಾದ ಸಂದೇಶ. ಇದಕ್ಕೆ ಮುಂಚೆ ಎಲ್ಲಿಯೂ ಕೇವಲ ಕರ್ಮದಿಂದಲೇ ಒಬ್ಬ ಮುಕ್ತನಾಗುತ್ತಾನೆ ಎಂದು ಸಾರಿಲ್ಲ. ಕರ್ಮಕ್ಕೆ ಒಂದು ಗೌಣ ಸ್ಥಾನವನ್ನು ಕೊಟ್ಟಿದ್ದಾರೆ. ಅದು ಚಿತ್ತ ಶುದ್ಧಿಯನ್ನು ಮಾಡುವುದು. ಚಿತ್ತ ಶುದ್ಧಿ ಜ್ಞಾನಕ್ಕೆ ಮತ್ತು ಭಕ್ತಿಗೆ ಸಹಕಾರಿ ಎಂದು ಹಿಂದಿನವರು ಹೇಳುತ್ತಾರೆಯೇ ಹೊರತು ಕರ್ಮದಿಂದಲೇ ಪ್ರತ್ಯಕ್ಷವಾಗಿ ಮುಕ್ತಿ ಸಿಕ್ಕುವುದು ಎಂದು ಹೇಳುವುದಿಲ್ಲ. ಆದರೆ ಶ್ರೀಕೃಷ್ಣ ಇಲ್ಲಿ ಕೇವಲ ಕರ್ಮವನ್ನು ಹೇಳುವುದಿಲ್ಲ. ಆ ಕರ್ಮ ಫಲಾಪೇಕ್ಷೆ ಇಲ್ಲದೆ ಮಾಡಿದ್ದಾಗಿರಬೇಕು. ಅದು ನಿಯತ ಕರ್ಮವಾಗಿರಬೇಕು. ಅದು ಹೀಗೆ ಇದ್ದರೆ ನೇರ ಇದರಿಂದಲೇ ಅವನಿಗೆ ಜ್ಞಾನ ಪ್ರಾಪ್ತವಾಗು ವುದು, ಇದರಿಂದಲೇ ಅವನು ಮುಕ್ತನಾಗಿ ಹೋಗುವನು.

\begin{verse}
ಕರ್ಮಣೈವ ಹಿ ಸಂಸಿದ್ಧಿಮಾಸ್ಥಿತಾ ಜನಕಾದಯಃ ।\\ಲೋಕಸಂಗ್ರಹಮೇವಾಪಿ ಸಂಪಶ್ಯನ್ ಕರ್ತುಮರ್ಹಸಿ \versenum{॥ ೨೦ ॥}
\end{verse}

{\small ಕೇವಲ ಕರ್ಮಾನುಷ್ಠಾನದಿಂದಲೇ ಜನಕನೇ ಮುಂತಾದವರು ಪೂರ್ಣತೆಯನ್ನು ಪಡೆದರು. ಲೋಕಸಂಗ್ರಹದ ದೃಷ್ಟಿಯಿಂದಲಾದರೂ ನೀನು ಕರ್ಮವನ್ನು ಮಾಡಬೇಕು.}

ಶ್ರೀಕೃಷ್ಣ ಈ ಮಾರ್ಗವನ್ನು ಹೇಳುವುದು ಇದೇ ಪ್ರಥಮ ಎಂದು ಹೇಳುವುದಿಲ್ಲ. ಈ ದಾರಿ ಹಿಂದಿನಿಂದ ಇತ್ತು. ಈ ದಾರಿಯಲ್ಲಿ ಕೆಲವರು ಹೋಗಿದ್ದಾರೆ. ಅವರೇ ಜನಕ, ಅಶ್ವಪತಿ ಮುಂತಾದವರು. ಅವರು ತಮ್ಮ ಪಾಲಿಗೆ ಬಂದ ಕರ್ಮವನ್ನು ಅನಾಸಕ್ತರಾಗಿ ಮಾಡಿಯೇ ಮುಕ್ತರಾದರು. ಅರ್ಜುನನು ಕೂಡ ಇವರನ್ನು ಅನುಸರಿಸಬಹುದು.

ಇಲ್ಲಿಂದ ಶ್ರೀಕೃಷ್ಣ ಲೋಕಸಂಗ್ರಹದ ದೃಷ್ಟಿಯಿಂದ ನೋಡಿದರೂ ನೀನು ಕರ್ಮವನ್ನು ಮಾಡ ಬೇಕಾಗುವುದು ಎನ್ನುವನು. ಅರ್ಜುನ ತನಗೇನೂ ಲಾಭ ಬೇಡ ಎಂದು ಕರ್ಮವನ್ನು ಬಿಡಬಹುದು. ತನಗೆ ಲಾಭ ಬೇಡದೆ ಇದ್ದರೂ ಆ ಕರ್ಮವನ್ನು ಬಿಡಬಾರದು. ತನಗೆ ಲಾಭ ಬೇಡದೆ ಇದ್ದರೂ ಆ ಕರ್ಮವನ್ನು ಪ್ರಪಂಚದಲ್ಲಿ ಇರುವ ಸಜ್ಜನರಿಗೆ ಒಳ್ಳೆಯದಾಗಲಿ, ಅಧರ್ಮ ತಗ್ಗಲಿ ಎಂಬ ದೃಷ್ಟಿಯಿಂದಲಾದರೂ ಮಾಡಬೇಕು. ಯಾವಾಗ ಒಬ್ಬ ಮನುಷ್ಯ ಪ್ರಪಂಚದಲ್ಲಿರುವನೋ, ಆ ಪ್ರಪಂಚವನ್ನು ನೋಡುತ್ತಿರುವನೋ, ಅಲ್ಲಿಯವರೆಗಾದರೂ ಅದು ಸತ್ಯ. ಅಲ್ಲಿರುವವರೆಗೆ ಅದನ್ನು ಸಾಧ್ಯವಾದಷ್ಟು ಚೆನ್ನಾಗಿ ಇಡಲು ಪ್ರಯತ್ನಿಸಬೇಕು. ಅದು ಇಲ್ಲಿರುವ ಪ್ರತಿಯೊಬ್ಬರ ಕರ್ತವ್ಯ. ನಮ್ಮಲ್ಲಿ ಸ್ವಾರ್ಥ ಇದೆ. ಅದಕ್ಕಾಗಿ ಕರ್ಮ ಮಾಡುತ್ತಿರುವೆವು. ಇದು ನಮ್ಮನ್ನು ಬಂಧನಕ್ಕೆ ಗುರಿ ಮಾಡುವುದು. ಎಲ್ಲಿಯವರೆಗೆ ರಜೋ ಪ್ರವೃತ್ತಿ ನಮ್ಮಲ್ಲಿ ಇದೆಯೋ ಅಲ್ಲಿಯವರೆಗೆ ಕರ್ಮಮಾಡದೆ ಇರುವುದಕ್ಕೆ ಆಗುವುದಿಲ್ಲ. ಸ್ವಾರ್ಥತೆಯಿಂದ ಪಾರಾಗಬೇಕಾದರೆ ಮತ್ತೊಂದನ್ನು ಹಿಡಿಯಬೇಕು. ನಾವಿರುವ ಏಣಿಯ ಮೆಟ್ಟಲನ್ನೇ ಬಿಡಬೇಕಾದರೆ, ಕೆಳಗೆ ಬರಬೇಕು, ಇಲ್ಲವೆ ಮೇಲೆ ಹೋಗಬೇಕು. ಕೆಳಗೆ ಬಂದರೆ ಮತ್ತೆ ಮೃಗೀಯ ಸ್ಥಿತಿಗೆ ಬರುತ್ತೇವೆ. ಅದಕ್ಕಾಗಿ ಮೇಲಕ್ಕೆ ಏರಬೇಕು. ಅದೇ ಲೋಕಕಲ್ಯಾಣ ದೃಷ್ಟಿ.

\begin{verse}
ಯದ್ಯದಾಚರತಿ ಶ್ರೇಷ್ಠಸ್ತತ್ತದೇವೇತರೋ ಜನಃ ।\\ಸ ಯತ್ಪ್ರಮಾಣಂ ಕುರುತೇ ಲೋಕಸ್ತದನುವರ್ತತೇ \versenum{॥ ೨೧ ॥}
\end{verse}

{\small ಶ್ರೇಷ್ಠರು ಯಾವುದನ್ನು ಆಚರಿಸುತ್ತಾರೆಯೋ ಇತರರು ಅದನ್ನೇ ಅನುಸರಿಸುತ್ತಾರೆ. ಅವನು ಯಾವುದನ್ನು ಆದರ್ಶವೆಂದು ಭಾವಿಸುತ್ತಾನೆಯೋ ಇತರರೂ ಕೂಡ ಅದನ್ನೇ ಸ್ವೀಕರಿಸುತ್ತಾರೆ.}

ಜನರಲ್ಲಿ ಯಾರು ಮುಂದಾಳಾಗಿದ್ದಾನೆಯೋ ನಾಯಕನೆಂದು ಪರಿಗಣಿಸಲ್ಪಟ್ಟಿರುವನೋ ಅವನು ಮಾಡಿದುದನ್ನೇ ಇತರರು ಅನುಸರಿಸುತ್ತಾರೆ. ಮನುಷ್ಯನಲ್ಲಿ ಈ ಅನುಕರಿಸುವ ಗುಣ ಇರುವುದು. ಏಕೆಂದರೆ ಅದು ಸುಲಭ. ನಾವೇ ಒಂದು ನಿರ್ಣಯಕ್ಕೆ ಬರಬೇಕಾದರೆ ಅದು ಅತಿ ಪ್ರಯಾಸದ ಕೆಲಸ. ಅದಕ್ಕಾಗಿ ತುಂಬಾ ಯೋಚಿಸಬೇಕು. ಮನಸ್ಸಿನಲ್ಲಿ ದೊಡ್ಡ ದಾಂಧಲೆ ಏಳುವುದು. ಇಷ್ಟೊಂದು ಕಷ್ಟಪಟ್ಟು ಒಂದು ನಿರ್ಣಯಕ್ಕೆ ಬಂದರೂ, ಯಾರಿಗೆ ಗೊತ್ತು ಒಂದು ವೇಳೆ ನಾನು ತಪ್ಪಾಗಿ ಇದ್ದರೂ ಇರಬಹುದು ಎಂಬ ಅನುಮಾನ ಯಾವಾಗಲೂ ನಮ್ಮನ್ನು ಕಾಡುತ್ತ ಇರುವುದು. ಅದಕ್ಕಾಗಿಯೇ ಒಬ್ಬ ದೊಡ್ಡವನನ್ನು ಅನುಸರಿಸಿದರೆ ಅವನು ಜವಾಬ್ದಾರಿಯ ಮನುಷ್ಯ, ನನಗಿಂತ ಚೆನ್ನಾಗಿ ಆ ವಿಷಯವನ್ನು ಕುರಿತು ಆಲೋಚಿಸಿರುವನು. ಅವನು ಎಂದಿಗೂ ತಪ್ಪುವುದಿಲ್ಲ. ಒಂದು ವೇಳೆ ಅವನು ತಪ್ಪಿದರೂ ಅವನ ಜೊತೆಯಲ್ಲಿಯೇ ತಪ್ಪು ಮಾಡುತ್ತೇವೆ. ಅವನಿಗೆ ಆದದ್ದೇ ನಮಗೆ ಆಗುವುದು ಎಂದು ಭಾವಿಸುತ್ತೇವೆ.

ದೊಡ್ಡ ವ್ಯಕ್ತಿ ಯಾವುದನ್ನು ಪ್ರಮಾಣ, ಆದರ್ಶ ಎಂದು ಸ್ವೀಕರಿಸುತ್ತಾನೆಯೋ ಅದನ್ನು ಇತರರೆಲ್ಲರೂ ಸ್ವೀಕರಿಸುತ್ತಾರೆ. ಆ ದೊಡ್ಡ ವ್ಯಕ್ತಿ ಜನ ಏತಕ್ಕೆ ನನ್ನನ್ನು ಅನುಸರಿಸಬೇಕು, ಅವರವರ ದಾರಿಯನ್ನು ಅವರು ಹಿಡಿಯಲಿ, ನಾನು ನನ್ನ ದಾರಿಯನ್ನು ಹಿಡಿಯುತ್ತೇನೆ ಎನ್ನಲಾಗುವುದಿಲ್ಲ. ತಾನು ಇಚ್ಛೆಪಡದೆ ಇದ್ದರೂ ಜನ ಇವನನ್ನು ಅನುಸರಿಸುತ್ತಾರೆ. ನಾಳೆ ಅವರು ತಪ್ಪು ದಾರಿಗೆ ಹೋದರೆ ಇವನೇ ಅದಕ್ಕೆ ಜವಾಬ್ದಾರನಾಗುತ್ತಾನೆ. ಇವನು ಜವಾಬ್ದಾರಿಯನ್ನು ಒದರಿ ಬಿಡುವುದಕ್ಕೆ ಆಗುವುದಿಲ್ಲ. ಇವನು ಅದನ್ನು ಒದರಿದರೂ ಜವಾಬ್ದಾರಿ ಮಾತ್ರ ಇವನನ್ನು ಬಿಡದು. ಇಂತಹ ಒಂದು ವ್ಯಕ್ತಿ ರೈಲ್ವೆ ಇಂಜನ್ನಿನಂತೆ. ಬೇಡವೆಂದರೂ ಗಾಡಿಗಳು ಅದಕ್ಕೆ ತಗಲಿಹಾಕಿಕೊಳ್ಳುವುವು. ಇಂಜಿನ್ ಹೋದ ಕಡೆ ಗಾಡಿಗಳೂ ಹೋಗುವುವು.

\begin{verse}
ನ ಮೇ ಪಾರ್ಥಾಸ್ತಿ ಕರ್ತವ್ಯಂ ತ್ರಿಷು ಲೋಕೇಷು ಕಿಂಚನ ।\\ನಾನವಾಪ್ತಮವಾಪ್ತವ್ಯಂ ವರ್ತ ಏವ ಚ ಕರ್ಮಣಿ \versenum{॥ ೨೨ ॥}
\end{verse}

{\small ಪಾರ್ಥ, ಮೂರು ಲೋಕಗಳಲ್ಲಿಯೂ ನನಗೆ ಮಾಡಬೇಕಾದ ಕರ್ತವ್ಯವಿಲ್ಲ. ನಾನು ಪಡೆಯದೆ ಇರುವುದು, ಇನ್ನು ಮೇಲೆ ಪಡೆಯಬೇಕಾಗಿರುವುದು ಯಾವುದೂ ಇಲ್ಲ. ಆದರೂ ನಾನು ಕರ್ಮದಲ್ಲಿ ನಿರತನಾಗಿರುವೆನು.}

ಶ್ರೀಕೃಷ್ಣ ತನ್ನ ಜೀವನವನ್ನೇ ಇದಕ್ಕೆ ಉದಾಹರಣೆಯಾಗಿ ಕೊಡುವನು. ಉದಾಹರಣೆ ಇಲ್ಲದಿದ್ದರೆ ಸಾಧಾರಣ ಮನುಷ್ಯ ನಂಬುವುದಿಲ್ಲ. ಅವನ ಬುದ್ಧಿಗೆ ಇದು ಹಿಡಿಯುವುದಿಲ್ಲ. ಶ್ರೀಕೃಷ್ಣ ಅವತಾರ, ಅವನು ಸತ್ಯಸಂಕಲ್ಪ ಪೂರ್ಣಕಾಮ. ಅವನಿಗೆ ಯಾವ ಕೊರತೆಯೂ ಇಲ್ಲ. ಅವನು ಕರ್ಮವನ್ನು ಮಾಡಿ ನಮ್ಮಂತೆ ಪುಣ್ಯ ಸಂಪಾದಿಸಿ ಪಾಪದ ಸಾಲವನ್ನು ತೀರಿಸಬೇಕಾಗಿಲ್ಲ. ಕರ್ತವ್ಯವೆಂದರೆ ಮಾಡಲೇಬೇಕಾಗಿರುವುದು. ಅದರಿಂದ ತಪ್ಪಿಸಿಕೊಂಡು ಹೋಗುವುದಕ್ಕೆ ಆಗುವುದಿಲ್ಲ. ಹಾಗೆ ತಪ್ಪಿಸಿಕೊಳ್ಳುವುದು ಅಕ್ಷಮ್ಯ. ಆದರೆ ಶ್ರೀಕೃಷ್ಣ ಆ ಗುಂಪಿಗೆ ಸೇರಿಲ್ಲ. ಆದರೂ ಅವನು ಕರ್ಮದಲ್ಲಿ ನಿರತನಾಗಿರುವನು. ಕಂಸನನ್ನು ಕೊಂದಾದ ಮೇಲೆ ಶ್ರೀಕೃಷ್ಣನು ಹಲವು ಯುದ್ಧಗಳನ್ನು ಮಾಡಿದನು. ಆದರೆ ಎಂದಿಗೂ ಅವರ ರಾಜ್ಯವನ್ನು ತೆಗೆದುಕೊಳ್ಳಲಿಲ್ಲ. ಅದಕ್ಕೆ ಉತ್ತರಾಧಿಕಾರಿಗಳನ್ನು ಹುಡುಕಿ ಅವರನ್ನು ಕುಳ್ಳಿರಿಸಿದನು. ಕರ್ಮ ಮಾಡುತ್ತಿದ್ದರೂ ಫಲದ ಮೇಲೆ ಅಪೇಕ್ಷೆ ಇರುತ್ತಿರಲಿಲ್ಲ. ಹಾಗೆ ಕರ್ಮ ಮಾಡುವುದರಿಂದ ಲೋಕೋಪಕಾರವಾಗುವುದು ಎಂಬ ದೃಷ್ಟಿ ಒಂದೇ ಅವನಲ್ಲಿತ್ತು. ಈಗ ಕೌರವ ಪಾಂಡವರ ಯುದ್ಧವಾಗುತ್ತಿರುವಾಗ ಶ್ರೀಕೃಷ್ಣನಿಗೆ ಯಾರಿಂದಲೂ ಏನೂ ಲಾಭವಿಲ್ಲ. ಆದರೆ ಅವನು ಈ ಯುದ್ಧವನ್ನು ನೋಡುತ್ತಿರುವುದು ಧರ್ಮಕ್ಕೂ ಅಧರ್ಮಕ್ಕೂ ಈಗ ಒಂದು ಘರ್ಷಣೆ ಆಗುತ್ತಿದೆ ಎಂಬ ದೃಷ್ಟಿಯಿಂದ. ಅವನು ಧರ್ಮದ ಕಡೆ ವಾಲುವನು. ತಾನೇ ಕೈ ಹಿಡಿದು ಯುದ್ಧಮಾಡದೆ ಇದ್ದರೂ ಅರ್ಜುನನ ಸಾರಥಿಯಾಗುವನು. ಅರ್ಜುನ ಯುದ್ಧ ಮಾಡುವುದಿಲ್ಲ ವೆಂದರೂ ಅವನ ಮೂಲಕ ಮಾಡಿಸಲು ಯತ್ನಿಸುವನು.

\begin{verse}
ಯದಿ ಹ್ಯಹಂ ನ ವರ್ತೇಯಂ ಜಾತು ಕರ್ಮಣ್ಯತಂದ್ರಿತಃ ।\\ಮಮ ವರ್ತ್ಮಾನುವರ್ತಂತೇ ಮನುಷ್ಯಾಃ ಪಾರ್ಥ ಸರ್ವಶಃ \versenum{॥ ೨೩ ॥}
\end{verse}

{\small ಪಾರ್ಥ, ನಾನೇನಾದರೂ ಅಜಾಗರೂಕನಾಗಿ ಯಾವಾಗಲಾದರೂ ಕರ್ಮದಲ್ಲಿ ನಿರತನಾಗದಿದ್ದರೆ, ಮನುಷ್ಯರು ಎಲ್ಲಾ ವಿಧದಿಂದಲೂ ನನ್ನನ್ನೇ ಅನುಸರಿಸುತ್ತಾರೆ.}

ಶ್ರೀಕೃಷ್ಣ ತನ್ನ ಪಾಲಿಗೆ ಬಂದ ಕರ್ಮವನ್ನು ಯಾವಾಗ ಬಿಡುತ್ತಾನೆಯೋ ಆಗ ಉಳಿದವರು, ಶ್ರೀಕೃಷ್ಣನಂಥವನೇ ಮಾಡಲಿಲ್ಲ, ನಾನು ಏತಕ್ಕೆ ಮಾಡಬೇಕು? ಎನ್ನುತ್ತಾರೆ. ಶ್ರೀಕೃಷ್ಣನಿಗೆ ಇದು ಆವಶ್ಯಕವಿಲ್ಲ, ಅದಕ್ಕಾಗಿ ಅವನು ಮಾಡಲಿಲ್ಲ. ಆದರೆ ನಾವು ಆ ಸ್ಥಿತಿಯಲ್ಲಿ ಇಲ್ಲ, ನಾವು ಇಲ್ಲಿ ಅವನನ್ನು ಅನುಸರಿಸಬಾರದು ಎಂದು ಯೋಚಿಸುವುದಿಲ್ಲ. ದೊಡ್ಡವನು ಮಾಡಿದಂತೆ ಅನುಕರಿಸಲು ಯತ್ನಿಸುವರು. ಜನ ಕುರಿಯ ಮಂದೆಯಂತೆ. ಏನೂ ಆಲೋಚನೆ ಮಾಡದೆ ಮುಂದಿನವರನ್ನು ಅನುಸರಿಸುವರು. ಆದಕಾರಣವೇ ಮುಂದೆ ಇರುವವನಿಗೆ ದೊಡ್ಡ ಜವಾಬ್ದಾರಿ ಇದೆ. ಆ ಜವಾಬ್ದಾರಿ ನಾನಾಗಿ ತೆಗೆದುಕೊಂಡದ್ದಲ್ಲ. ಅದಾಗಿ ನನಗೆ ಬೇಡ ಎಂದರೂ ಬಂದದ್ದು. ನಾನಾಗಿಯೇ ಅದನ್ನು ಇಚ್ಛೆಪಟ್ಟು ತೆಗೆದುಕೊಳ್ಳಲಿಲ್ಲ ಅಥವಾ ನಾನು ಬೇಡವೆಂದರೂ ಅದನ್ನು ನಾನು ಬಿಟ್ಟುಬಿಡುವುದಕ್ಕೆ ಆಗುವುದಿಲ್ಲ. ಒಬ್ಬ ಸಮಾಜವನ್ನು ಬಿಟ್ಟು ನಿರ್ಜನ ಅರಣ್ಯದಲ್ಲಿಯೋ ಗುಹೆಯಲ್ಲಿಯೋ ಇದ್ದರೆ ತನಗೆ ತೋಚಿದುದನ್ನು ಮಾಡಬಹುದು. ಸುತ್ತಮುತ್ತ ಇರುವ ಜನ ನಮ್ಮನ್ನು ನೋಡುತ್ತಿರುವಾಗ ಹಾಗೆ ಮಾಡಲು ಆಗುವುದಿಲ್ಲ. ನಾನು ಮಾಡುವ ಕೆಲಸದಿಂದ ಸುತ್ತಮುತ್ತ ಇರುವ ಜನರ ಮೇಲೆ ಯಾವ ಪರಿಣಾಮ ಉಂಟಾಗುವುದು ಎಂಬ ದೃಷ್ಟಿಯಿಂದ ನೋಡುತ್ತಿರಬೇಕಾಗುವುದು. ಶ್ರೀರಾಮ ಕೃಷ್ಣರ ಜೀವನದಲ್ಲಿ ಇದನ್ನು ನೋಡುತ್ತೇವೆ. ಅವರು ಅದ್ವೈತ ಸಾಧನೆ ಮಾಡಿ ನಿರ್ವಿಕಲ್ಪ ಸಮಾಧಿಯನ್ನು ಪಡೆದರು. ಎಲ್ಲಾ ಕಡೆಯಲ್ಲಿಯೂ ಒಬ್ಬನೇ ಇರುವುದು. ಅವನಲ್ಲದೆ ಬೇರೆ ಅವರಿಗೆ ಏನೂ ಕಾಣುತ್ತಿರಲಿಲ್ಲ. ಆದರೂ ಅವರು ವಿಗ್ರಹಾರಾಧನೆಯನ್ನು ಬಿಡಲಿಲ್ಲ. ಗರ್ಭಗುಡಿಗೆ ಹೋಗಿ ದೇವಿಯ ಎದುರಿಗೆ ಭಕ್ತಿಯಿಂದ ಬಾಗುತ್ತಿದ್ದರು. ಸಗುಣದೇವರನ್ನು ಯಾವಾಗಲೂ ಒಂದು ಊರುಗೋಲಿನಂತೆ ಇಟ್ಟುಕೊಂಡಿದ್ದರು. ತಾವು ಇದ್ದ ದೇವಸ್ಥಾನದಲ್ಲಿ ಕಾಲಕಾಲಕ್ಕೆ ಆಗುತ್ತಿದ್ದ ಪೂಜಾ ನೈವೇದ್ಯಾದಿಗಳಲ್ಲಿ ಒಂದು ಚೂರೂ ತಪ್ಪದಂತೆ ನೋಡಿಕೊಳ್ಳುತ್ತಿದ್ದರು. ಇದರಿಂದ ಅವರಿಗೇನೂ ಪ್ರಯೋಜನವಾಗುತ್ತಿರಲಿಲ್ಲ. ಆದರೆ ಅವರೇನಾದರೂ ಇದಕ್ಕೆ ಅಜಾಗರೂಕತೆಯನ್ನು ತೋರಿದರು ಎಂದರೆ ಸುತ್ತಲೂ ಇರುವವರು ಅನುಕರಿಸುತ್ತಿದ್ದರು. ಅವರು ಮತ್ತೊಂದು ನಿದರ್ಶನ ವನ್ನು ಕೊಡುವರು. ಬೋಧಿಸುವವನು ತಾನು ಬೋಧಿಸುವುದಕ್ಕೆ ಉದಾಹರಣೆಯಾಗಿರಬೇಕು. ಇಲ್ಲದೆ ಇದ್ದರೆ ಅವನ ಮಾತಿಗೆ ಬೆಲೆಯಿಲ್ಲ. ಒಬ್ಬ ತಂದೆ ತನ್ನ ಮಗುವನ್ನು ಕರೆದುಕೊಂಡು ವೈದ್ಯನ ಬಳಿಗೆ ಹೋದ. ವೈದ್ಯ ಆ ಮಗುವನ್ನು ನೋಡಿ ನಾಳೆ ಬಾ, ಏನು ಮಾಡಬೇಕು ಎನ್ನುವುದನ್ನು ಹೇಳುತ್ತೇನೆ ಎಂದು ಕಳುಹಿಸಿದ. ಮಾರನೆ ದಿನ ಬಂದಾಗ ಆ ಮಗುವಿನ ತಂದೆಗೆ ಈ ಹುಡುಗನಿಗೆ ಬೆಲ್ಲ ಮುಂತಾದುವನ್ನು ಕೊಡಬಾರದು. ಇದನ್ನು ಮಾಡಿದರೆ ಸಾಕು ಖಾಯಿಲೆ ಗುಣವಾಗುವುದು ಎಂದು ಹೇಳಿ ಕಳುಹಿಸಿದ. ಹತ್ತಿರ ಇದ್ದವನು, ಆ ಮಗು ಹೋದಮೇಲೆ, ನೀವು ನಿನ್ನೆಯೇ ಇದನ್ನು ಹೇಳಿಬಿಡಬಹುದಾಗಿತ್ತಲ್ಲ, ಇವತ್ತು ಕೂಡ ಆತ ಬರುವಂತೆ ಏತಕ್ಕೆ ಮಾಡಿದಿರಿ? ಎಂದು ಕೇಳಿದ. ಅದಕ್ಕೆ ವೈದ್ಯನು ಹೀಗೆ ಹೇಳಿದನು: ನೋಡು, ನಿನ್ನೆ ನಾನಿದ್ದ ಸ್ಥಳದಲ್ಲಿ ಬೇಕಾದಷ್ಟು ಬೆಲ್ಲದಚ್ಚುಗಳ ಪಿಂಡಿ ಇದ್ದವು. ಬೆಲ್ಲ ತಿನ್ನಬಾರದು, ಕೆಟ್ಟದ್ದು ಎಂದು ಹೇಳಿದರೆ ಆ ಮಗು ನನ್ನ ಸುತ್ತಲೂ ಇರುವ ಬೆಲ್ಲವನ್ನು ನೋಡಿ, ಇವರು ಅದನ್ನು ಬೇಕಾದಷ್ಟು ತಿನ್ನಬಹುದು, ನನಗೆ ಮಾತ್ರ ಬೇಡ ಎನ್ನುತ್ತಾರೆ, ಕೆಟ್ಟದ್ದಾದರೆ ಎಲ್ಲರಿಗೂ ಕೆಟ್ಟದ್ದಾಗಬೇಕಲ್ಲ ಎಂದು ನಾನು ಹೇಳಿದ ಮಾತನ್ನು ನಂಬುತ್ತಿರಲಿಲ್ಲ. ಆದರೆ ಇವತ್ತು ಆ ಬೆಲ್ಲದ ಪಿಂಡಿಗಳನ್ನೆಲ್ಲ ಆ ಮಗುವಿಗೆ ಕಾಣದೆ ಬೇರೆ ಕಡೆ ಇರಿಸಿದ್ದೇನೆ. ಈಗ ಆ ಆಲೋಚನೆ ಅದರಲ್ಲಿ ಬರುವುದಕ್ಕೆ ಅವಕಾಶವಿಲ್ಲ ಎನ್ನುತ್ತಾನೆ. ಇಲ್ಲಿ ಮಗು ತನ್ನ ಪ್ರಕೃತಿಗೆ ಬೆಲ್ಲ ಒಗ್ಗುವುದಿಲ್ಲ ಅದಕ್ಕೆ ನಾನು ಅದನ್ನು ತಿನ್ನಬಾರದು, ಇನ್ನೊಬ್ಬನ ಪ್ರಕೃತಿಗೆ ಒಗ್ಗುವುದು, ಅವನು ಬೇಕಾದರೆ ತಿನ್ನಬಹುದು ಎಂದು ಆಲೋಚಿಸುವುದಿಲ್ಲ. ಯಾರು ಹೇಗೆ ಮಾಡಿದರೆ ಹಾಗೆ ಮಾಡುವುದು ಮಗುವಿನ ಸ್ವಭಾವ. ಅದರಂತೆಯೇ ಬಹುಪಾಲು ಮಾನವರು ವಯಸ್ಸಿನಲ್ಲಿ ದೊಡ್ಡವ ರಾಗಿರಬಹುದು, ಬುದ್ಧಿಯ ವಿಕಾಸದಲ್ಲಿ ಆ ಮಗುವಿಗಿಂತಲೂ ಮೀರಿಹೋಗಿಲ್ಲ.

ಜೀವನದಲ್ಲಿ ಯಾವಾಗಲೂ ಒಬ್ಬ ದೊಡ್ಡ ಮನುಷ್ಯನಲ್ಲಿರುವ ಒಳ್ಳೆಯದನ್ನು ಹೇಗೆ ಅನುಕರಿ ಸಲು ಜನ ಸಿದ್ಧರಾಗಿರುವರೊ ಹಾಗೆಯೇ ಅವರಲ್ಲಿರುವ ಕೆಲವು ಲೋಪದೋಷಗಳನ್ನೂ ಒಳ್ಳೆಯದ ಕ್ಕಿಂತ ಹೆಚ್ಚಾಗಿ ಅನುಕರಿಸುವರು. ಏಕೆಂದರೆ ಕೆಟ್ಟದ್ದನ್ನು ಅನುಕರಿಸುವುದು ಸುಲಭ. ಮುಂದೆ ಇರುವವನು ಇದಕ್ಕೆ ಅವಕಾಶವನ್ನು ಕೊಡಬಾರದು.

\begin{verse}
ಉತ್ಸೀದೇಯುರಿಮೇ ಲೋಕಾ ನ ಕುರ್ಯಾಂ ಕರ್ಮ ಚೇದಹಮ್ ।\\ಸಂಕರಸ್ಯ ಚ ಕರ್ತಾ ಸ್ಯಾಮುಪಹನ್ಯಾಮಿಮಾಃ ಪ್ರಜಾಃ \versenum{॥ ೨೪ ॥}
\end{verse}

{\small ನಾನು ಕರ್ಮವನ್ನು ಮಾಡದೆ ಇದ್ದರೆ ಈ ಲೋಕ ನಾಶವಾಗುವುದು. ನಾನೇ ಈ ವರ್ಣಸಂಕರಕ್ಕೆ ಕಾರಣ ನಾಗುವೆನು. ಈ ಜನರನ್ನು ನಾನೇ ನಾಶ ಮಾಡಿದಂತಾಗುವುದು.}

ಯಾವಾಗ ಶ್ರೀಕೃಷ್ಣನಂತಹ ಮುಂದಾಳು ತನಗೆ ಏನೂ ಕರ್ಮದಿಂದ ಬರಬೇಕಾಗಿಲ್ಲ ಎಂದು ಕರ್ಮ ಬಿಟ್ಟರೆ, ಸುತ್ತಲೂ ಇರುವ ಜನ, ಅವನಿಗೆ ಆವಶ್ಯಕವಿಲ್ಲ ಅದಕ್ಕೆ ಅವನು ಮಾಡಲಿಲ್ಲ, ಆದರೆ ನಮಗೆ ಆವಶ್ಯಕವಿದೆ, ನಾವು ಬಿಡುವುದಕ್ಕೆ ಆಗುವುದಿಲ್ಲ ಎಂದು ಯೋಚಿಸುವುದಿಲ್ಲ. ಅವನೇ ಮಾಡಲಿಲ್ಲ, ಅವನು ನಮಗಿಂತ ಚೆನ್ನಾಗಿ ವಿಚಾರ ಮಾಡಬಲ್ಲ–ಯಾವುದು ಸರಿ, ಯಾವುದು ತಪ್ಪು ಎಂದು. ಅಂಥವನೇ ಮಾಡದೆ ಇದ್ದರೆ ಇನ್ನು ನಾವೇಕೆ ಮಾಡಬೇಕು ಎಂದು ತಮ್ಮ ಪಾಲಿನ ಕೆಲಸವನ್ನು ಮಾಡುವುದನ್ನು ಬಿಡುವರು. ತಮ್ಮ ತಮ್ಮ ವೃತ್ತಿಗೆ ಸೇರಿದ ಕಾರ್ಯವನ್ನು ಮಾಡಿದರೆ ಸಮಾಜ ಭದ್ರವಾಗಿರುವುದು. ಆಗ ಸಮಾಜದಲ್ಲಿ, ಶಿಸ್ತು, ನ್ಯಾಯ ಮತ್ತು ಬೇಕಾದಷ್ಟು ಉತ್ಪಾದನೆ ಇವೆಲ್ಲ ಇರುವುದು. ಯಾವಾಗ ಜನ ಸಮಷ್ಟಿಜೀವನಕ್ಕೆ ಮಾಡುವುದನ್ನು ನಿಲ್ಲಿಸುವರೋ ಆಗ ಸಮಷ್ಟಿ ಜೀವನ ಇತರರಿಗೆ ಏನು ಬೇಕೋ ಅದನ್ನು ಹೇಗೆ ಕೊಡಬಲ್ಲುದು? ಜನ ತಾವು ಮಾಡುವ ಕೆಲಸವನ್ನು ಬಿಡುತ್ತಾರೆ. ಇದರಿಂದ ಅಪಾರ ಹಾನಿಯಾಗುವುದು. ಬ್ರಾಹ್ಮಣ, ಕ್ಷತ್ರಿಯ, ವೈಶ್ಯ, ಶೂದ್ರ ಯಾರೂ ತಮ್ಮ ತಮ್ಮ ಕೆಲಸವನ್ನು ಮಾಡುವುದಿಲ್ಲ. ಶುದ್ಧ ಸೋಮಾರಿಗಳಾಗಿ ಕಾಲ ಕಳೆಯುವುದಕ್ಕೆ ಆಗುವುದಿಲ್ಲ. ನಮ್ಮಲ್ಲಿ ರಜೋಗುಣವಿದೆ. ಅದು ನಮ್ಮನ್ನು ಸುಮ್ಮನೆ ಬಿಡುವುದಿಲ್ಲ. ಆದಕಾರಣ ಇನ್ನೊಬ್ಬರ ಕೆಲಸಕ್ಕೆ ಕೈಹಾಕುತ್ತೇವೆ. ಅದರ ನೂತನತೆ ನಮ್ಮನ್ನು ಆಕರ್ಷಿಸುವುದು. ಒಬ್ಬ ಇನ್ನೊಬ್ಬನ ಕೆಲಸವನ್ನು ಮಾಡಲು ಪ್ರಾರಂಭಿಸುವನು. ಹುಟ್ಟಿದಾರಭ್ಯ ಒಂದು ಕೆಲಸಕ್ಕೆ ತರಬೇತನ್ನು ತೆಗೆದು ಕೊಂಡು ಅದನ್ನು ತಲೆತಲಾಂತರಗಳಿಂದಲೂ ಮಾಡಿಕೊಂಡು ಬರುತ್ತಿದ್ದರೂ ನಮ್ಮಲ್ಲಿ ಆ ಕೆಲಸವನ್ನು ಮಾಡುವಾಗ ಎಷ್ಟೋ ಲೋಪದೋಷಗಳು ತೋರುವುವು. ಹೀಗಿರುವಾಗ ಸುಮ್ಮನೆ ಇನ್ನೊಬ್ಬನ ಕೆಲಸಕ್ಕೆ ಕೈಹಾಕಿದರೆ ಅದನ್ನು ಚೆನ್ನಾಗಿ ಮಾಡುವುದು ಹೇಗೆ? ಇದರಿಂದ ನಷ್ಟ ಆ ವ್ಯಕ್ತಿಗೆ ಮಾತ್ರ ಅಲ್ಲ. ವ್ಯಕ್ತಿಗಳು ಕೊಡುವುದರ ಮೇಲೆ ನಿಂತ ಸಮಾಜವೇ ಕುಸಿದು ಬೀಳುವುದು. ಒಂದು ಸುವ್ಯವಸ್ಥಿತ ಸಮಾಜಕ್ಕೆ ವರ್ಣ ಆಶ್ರಮ ಧರ್ಮಗಳ ಪರಿಪಾಲನೆಯೇ ತಳಹದಿ. ಅದೇ ಕುಸಿದು ಬಿದ್ದರೆ ಮೇಲಿರುವ ಕಟ್ಟಡ ಕುಸಿದು ಬೀಳುವುದಕ್ಕೆ ಎಷ್ಟು ಕಾಲ ಹಿಡಿದೀತು? ಆದುದರಿಂದಲೇ ಯಾರು ಮುಂದಾಳಾಗಿರುವನೋ ಅವನು ಒಂದು ಕೆಲಸವನ್ನು ಮಾಡುವಾಗ ದೀರ್ಘವಾಗಿ ಆಲೋಚನೆ ಮಾಡಬೇಕು. ತಾನು ಮಾಡುವುದನ್ನು ಜನ ಅನುಸರಿಸಿದರೆ ಜನ ಉದ್ಧಾರವಾಗುವರೆ ಹಾಳಾಗುವರೆ ಎಂಬುದನ್ನು ಯಾವಾಗಲೂ ಚಿಂತಿಸುತ್ತಿರಬೇಕು. ಇವನನ್ನು ಅನುಸರಿಸಿ ಅವರು ಬಾಳಿದರೆ ಪಾಪ ಇವನಿಗೆ ತಟ್ಟುವುದು. ಒಂದು ವ್ಯಕ್ತಿಗೆ ಅನ್ಯಾಯ ಮಾಡಿದರೇನೆ ಸಮಾಜ ಅಷ್ಟೊಂದು ಉಗ್ರವಾಗಿ ಅದಕ್ಕೆ ಕಾರಣಕರ್ತನಾದವನನ್ನು ಕಟುವಾಗಿ ಟೀಕಿಸುವುದು. ಹೀಗಿರುವಾಗ ಸಂಸಾರಗಳ ಮೊತ್ತವಾದ ಸಮಾಜವನ್ನೇ ನಿರ್ನಾಮ ಮಾಡಿದವನು ಇನ್ನಾವ ಶಿಕ್ಷೆಗೆ ತುತ್ತಾಗುವನು? ಲೋಕಪಾಲಕನಾದ ಶ್ರೀಕೃಷ್ಣನೇ ಲೋಕಸಂಹಾರ ಮಾಡಿದಂತೆ ಆಗುವುದು. ಆಟದ ಕೊನೆಯಲ್ಲಿ ಸಂಹಾರ ಬರುವುದು ನಿಜ. ಆದರೆ ಮುಂಚೆಯೇ ಸಂಹಾರವಾದರೆ ಇನ್ನು ಲೋಕಲೀಲೆಯ ಆಟ ಸಾಗುವುದೆಂತು?

\begin{verse}
ಸಕ್ತಾಃ ಕರ್ಮಣ್ಯವಿದ್ವಾಂಸೋ ಯಥಾ ಕುರ್ವಂತಿ ಭಾರತ ।\\ಕುರ್ಯಾದ್ವಿದ್ವಾಂಸ್ತಥಾಸಕ್ತಶ್ಚಿಕೀರ್ಷುರ್ಲೋಕಸಂಗ್ರಹಮ್ \versenum{॥ ೨೫ ॥}
\end{verse}

{\small ಅರ್ಜುನ, ಅಜ್ಞಾನಿಗಳು ಕರ್ಮದಲ್ಲಿ ಆಸಕ್ತರಾಗಿ ಹೇಗೆ ಕರ್ಮವನ್ನು ಮಾಡುತ್ತಾರೆಯೋ ಹಾಗೆಯೇ ಜ್ಞಾನಿಯು, ಅನಾಸಕ್ತನಾಗಿ ಲೋಕಸಂಗ್ರಹದ ದೃಷ್ಟಿಯಿಂದ ಕರ್ಮವನ್ನು ಮಾಡಬೇಕು.}

ಅಜ್ಞಾನಿ ತಕ್ಷಣವೇ ತನಗೆ ದೊರಕುವ ತಾತ್ಕಾಲಿಕ ಲಾಭಕ್ಕೆ, ಸುಖಕ್ಕೆ ಕೆಲಸ ಮಾಡುತ್ತಾನೆ. ಅವನಿಗೆ ಫಲ ಬೇಕು, ಅದಕ್ಕೆ ಕೆಲಸ ಮಾಡುತ್ತಾನೆ. ಒಬ್ಬ ಜ್ಞಾನಿಯಾಗಿದ್ದಾನೆ. ಅವನಿಗೆ ಲಾಭ ಕೀರ್ತಿ ಇವುಗಳ ಕ್ಷಣಿಕತೆ ಚೆನ್ನಾಗಿ ಅರ್ಥವಾಗಿದೆ. ಅವನಿಗೆ ಇದಾವುದೂ ಬೇಕಾಗಿಲ್ಲ. ಆದರೂ ಅವನು ಕೆಲಸವನ್ನು ಮಾಡುವುದನ್ನು ಬಿಡಕೂಡದು. ಅವನಿಗೆ ಫಲ ಬೇಡದೆ ಇದ್ದರೆ ಚಿಂತೆಯಿಲ್ಲ. ಅನಾಸಕ್ತನಾಗಿ ತನ್ನ ಪಾಲಿನ ಕರ್ತವ್ಯವನ್ನು ನಿರ್ವಹಿಸಲಿ. ಅದರಿಂದ ಲೋಕಕ್ಕೆ ಒಳ್ಳೆಯದಾಗುವುದು. ಯಾವುದು ಲೋಕಕ್ಕೆ ಒಳ್ಳೆಯದಾಗುವುದೋ ಅದು ಆತ್ಮ ಕಲ್ಯಾಣಕಾರಿಯಾಗುತ್ತದೆ. ತನಗೆ ಫಲ ಬೇಡ ಎಂದರೆ ಕರ್ಮವನ್ನು ಬಿಡುವುದಕ್ಕೆ ಆಗುವುದಿಲ್ಲ. ಬೇಕಾದರೆ ಬಂದ ಫಲವನ್ನು ಭಗವದರ್ಪಣ ಮಾಡಲಿ.

\begin{verse}
ನ ಬುದ್ಧಿಭೇದಂ ಜನಯೇದಜ್ಞಾನಾಂ ಕರ್ಮಸಂಗಿನಾಮ್ ।\\ಜೋಷಯೇತ್ ಸರ್ವಕರ್ಮಾಣಿ ವಿದ್ವಾನ್ ಯುಕ್ತಃ ಸಮಾಚರನ್ \versenum{॥ ೨೬ ॥}
\end{verse}

{\small ಕರ್ಮದಲ್ಲಿ ಆಸಕ್ತರಾದ ಅಜ್ಞಾನಿಗಳಲ್ಲಿ ಬುದ್ಧಿಭೇದವನ್ನು ಉಂಟುಮಾಡಬಾರದು. ಜ್ಞಾನಿಯು ಸರಿಯಾಗಿ ತನ್ನ ಪಾಲಿಗೆ ಬಂದ ಕರ್ಮಗಳನ್ನು ಮಾಡಿ ಅಜ್ಞಾನಿಗಳ ಕೈಯಿಂದಲೂ ಮಾಡಿಸಬೇಕು.}

ತಿಳಿದವನು, ಫಲಾಪೇಕ್ಷೆಯಿಂದ ಅಜ್ಞಾನಿಗಳು ಕರ್ಮವನ್ನು ಮಾಡುತ್ತಿದ್ದರೆ ಅದನ್ನು ನಿಲ್ಲಿಸ ಕೂಡದು. ಅದನ್ನು ಇನ್ನೂ ಉತ್ತಮ ದೃಷ್ಟಿಯಿಂದ ಮಾಡಿ ಎನ್ನಬೇಕು. ಅದನ್ನು ನಿಲ್ಲಿಸಿಬಿಡುವುದು ಸುಲಭ. ಅದನ್ನು ಉತ್ತಮ ದೃಷ್ಟಿಯಿಂದ ಮಾಡಿ ಎಂದು ಹೇಳುವುದಕ್ಕೆ ಸಹಾನುಭೂತಿ ಬೇಕು. ಮನುಷ್ಯ ಯಾವಾಗಲೂ ಸುಳ್ಳಿನಿಂದ ಸತ್ಯದ ಕಡೆ ಪ್ರಯಾಣ ಮಾಡುತ್ತಿಲ್ಲ. ಸಣ್ಣ ಸತ್ಯದಿಂದ ದೊಡ್ಡ ಸತ್ಯದ ಕಡೆ ಪ್ರಯಾಣ ಮಾಡುತ್ತಿರುವನು. ಅವನು ಇನ್ನೂ ಉತ್ತಮ ದೃಷ್ಟಿಯಿಂದ ಕೆಲಸ ಮಾಡುವುದಕ್ಕೆ ಸಾಧ್ಯವಿಲ್ಲದೇ ಇದ್ದರೆ ಹಿಂದೆ ಯಾವ ದೃಷ್ಟಿಯಿಂದ ಮಾಡುತ್ತಿದ್ದನೋ ಅದ ರಂತೆಯೇ ಮಾಡಲಿ ಎಂದು ಬಿಡಬೇಕು. ಅವನು ಕೆಲಸ ಮಾಡುತ್ತಿದ್ದರೆ ಅದರಿಂದ ಸಮಾಜಕ್ಕೆ ಏನಾದರೂ ಸಿಕ್ಕುವುದು. ಕೆಲಸವನ್ನು ಬಿಟ್ಟರೆ ಅದರಿಂದ ಸಮಾಜಕ್ಕೆ ಬರುವುದೂ ನಿಲ್ಲುವುದು, ಇವನಿಗೆ ಬರುವುದೂ ನಿಲ್ಲುವುದು.

ಸ್ವಾರ್ಥಕ್ಕೋ ಕೀರ್ತಿಗೋ ಅಂತು ಯಾವುದಾದರೂ ದೃಷ್ಟಿಯಿಂದ ಕರ್ಮವನ್ನು ಮಾಡಿಕೊಂಡು ಹೋಗಲಿ. ಹಾಗೆ ಮಾಡುವುದು ಮಾಡದೆ ಇರುವುದಕ್ಕಿಂತ ಮೇಲು. ಇಂದು ಸ್ವಾರ್ಥದಿಂದ ಕೆಲಸ ಮಾಡುವುದಕ್ಕೆ ಪ್ರಾರಂಭಿಸುವನು, ಕಾಲ ಕ್ರಮೇಣ ಅನುಭವ ಗಳಿಸಿಕೊಂಡು ನಿಃಸ್ವಾರ್ಥನಾಗಿ ಕೆಲಸ ಮಾಡುವುದನ್ನು ಕಲಿಯುವನು. ಕೆಲಸ ಮಾಡುವುದನ್ನೇ ಬಿಟ್ಟರೆ, ನಿಃಸ್ವಾರ್ಥನಾಗಿ ಕೆಲಸ ಮಾಡುವು ದನ್ನು ಅವನು ಯಾವಾಗ ಕಲಿಯುತ್ತಾನೆ? ಒಂದು ಮಾವಿನ ಕಾಯಿ ಎಳೆಯದಾಗಿರುವಾಗ ಹುಳಿ ಯಾಗಿರುವುದು. ಅದು ಮಾಗುತ್ತ ಮಾಗುತ್ತ ಬಂದರೆ ಕ್ರಮೇಣ ಆ ಹುಳಿಯೆಲ್ಲ ಹೋಗಿ ಸಿಹಿಯಾಗುವುದು. ಈಗ ಇರುವ ಹುಳಿಯೆ ಕಾಲಕ್ರಮೇಣ ಸಿಹಿಯಾಗಬೇಕಾದರೆ.

ಒಂದೇ ಸತ್ಯವನ್ನು ಎಲ್ಲರಿಗೂ ಹೇಳಬೇಕಾಗಿ ಬಂದರೂ ವ್ಯಕ್ತಿಗೆ ತಕ್ಕಂತೆ ಅದನ್ನು ಬದಲಾಯಿಸಿ ಕೊಡಬೇಕು. ಶ್ರೀರಾಮಕೃಷ್ಣರು ಹಾಲು ಎಲ್ಲರಿಗೂ ಬೇಕು; ಆದರೆ ಅದನ್ನು ಒಂದೇ ರೀತಿ ಕೊಡುವುದಕ್ಕೆ ಆಗುವುದಿಲ್ಲ. ಅವರವರ ಪ್ರಕೃತಿಗೆ ಅನುಸಾರವಾಗಿ ಅದನ್ನು ಕೊಡಬೇಕು ಎನ್ನು ತ್ತಿದ್ದರು. ಎಳೆಯ ಮಗುವಿಗೆ ಹಾಲಿಗೆ ನೀರು ಹಾಕಿ ಕೊಡಬೇಕು. ಪೈಲ್ವಾನ್ ಹಾಲನ್ನು ಇಂಗಿಸಿ ಬಾಸುಂದಿ ಮಾಡಿ ತೆಗೆದುಕೊಳ್ಳುತ್ತಾನೆ. ಟೈಫಾಯಿಡ್ ರೋಗಿಗೆ ಹಾಲನ್ನು ಒಡೆದು ಗಟ್ಟಿಯ ಭಾಗವನ್ನು ಬಿಟ್ಟು ನೀರಿನ ಭಾಗವನ್ನು ಕೊಡುವರು. ಅಂತೂ ಹಾಲಿನ ಅಂಶ ರೋಗಿಗೆ ಸೇರಬೇಕು. ಆಗಲೆ ಅವನಿಗೆ ಶಕ್ತಿ ಬರಬೇಕಾದರೆ. ಶ್ರೀಕೃಷ್ಣನೇ ಅರ್ಜುನನಿಗೆ ಯುದ್ಧವನ್ನು ಹಲವು ದೃಷ್ಟಿಕೋಣ ಗಳಿಂದ ಮಾಡು ಎನ್ನುವನು. ಒಂದು ವೇದಾಂತದ ದೃಷ್ಟಿ. ಅಲ್ಲಿ ಕೊಲ್ಲುವವರಿಲ್ಲ ಕೊಲ್ಲಿಸಿಕೊಳ್ಳು ವವರಿಲ್ಲ. ಎರಡನೆಯದೆ ಕ್ಷತ್ರಿಯನ ಕರ್ತವ್ಯ ದೃಷ್ಟಿ. ಮೂರನೆಯದೆ ಲೋಕಾಪವಾದ ದೃಷ್ಟಿ; ಅನಂತರ ಕೀರ್ತಿಯ ದೃಷ್ಟಿ, ಕೇವಲ ಸ್ವಾರ್ಥದ ದೃಷ್ಟಿ ಕೊನೆಗೆ. ಅಂತೂ ಯಾವುದಾದರೂ ಉದ್ದೇಶದಿಂದ ಕೆಲಸ ಮಾಡು ಎಂದು ಹೇಳಬೇಕು.

ನಾವು ಮಾಡುವ ಕೆಲಸ ಪ್ರಾರಂಭದಲ್ಲಿ ಸ್ವಾರ್ಥತೆ, ಲಾಭ, ಅಜ್ಞಾನಗಳಿಂದ ಕೂಡಿದ್ದರೂ, ಕರ್ಮದ ಒಲೆಯ ಮೇಲೆ ಕಾಯುತ್ತ ಕಾಯುತ್ತ ಈ ದೋಷಗಳೆಲ್ಲಾ ಕ್ರಮೇಣ ಹೋಗುವುವು. ಆದಕಾರಣವೆ ಜ್ಞಾನಿ ತನ್ನ ಪಾಲಿಗೆ ಬಂದ ಕರ್ಮವನ್ನು ಮಾಡಬೇಕು. ಅಜ್ಞಾನಿಗೆ ಯಾವುದಾದರೂ ಒಂದು ದೃಷ್ಟಿಯಿಂದ ಕರ್ಮವನ್ನು ಮಾಡುವಂತೆ ಪ್ರೋತ್ಸಾಹ ಕೊಡಬೇಕು. ಆಗಲೆ ಆ ವ್ಯಕ್ತಿಗೆ ಒಳ್ಳೆಯ ದೃಷ್ಟಿಯಿಂದ ಕರ್ಮ ಮಾಡುವ ಸ್ಥಿತಿ ಕ್ರಮೇಣ ಪ್ರಾಪ್ತವಾಗುವುದು. ಸಮಾಜಕ್ಕೂ ಯಾವ ವಿಧವಾದ ನಷ್ಟವೂ ಆಗುವುದಿಲ್ಲ. ಇನ್ನೊಬ್ಬರ ಕೈಯಿಂದ ಕರ್ಮಗಳನ್ನು ಮಾಡಿಸಬೇಕಾದರೆ ಬುದ್ಧಿವಾದ ಹೇಳಿದರೆ ಸಾಲದು. ನಾವು ಅವರ ಎದುರಿಗೆ ಮಾಡಬೇಕು. ಆಗ ಅವರು ನಮ್ಮಂತೆ ಮಾಡುತ್ತಾರೆ. ಕೆಲವು ವೇಳೆ ಜಟಕಾ ಗಾಡಿಯಲ್ಲಿ ಹೋಗುತ್ತಿರುವಾಗ ಮೊಂಡುತನದಿಂದ ಕುದುರೆ ನಿಲ್ಲುವುದು. ಹಿಂದಿನಿಂದ ಒಂದು ಗಾಡಿ ಬಂದು ಅದರ ಮುಂದೆ ಹೋದಾಗ ಅದೂ ಆ ಗಾಡಿಯನ್ನೇ ಅನುಸರಿಸುವುದು. ಮನುಷ್ಯನಿಗೆ ಎರಡು ಕಾಲುಗಳೇ ಇದ್ದರೂ ಹಲವು ವಿಧದಲ್ಲಿ ಅವನು ನಾಲ್ಕು ಕಾಲಿನ ಪ್ರಾಣಿಗಳನ್ನೇ ಹೋಲುತ್ತಾನೆ. ಜ್ಞಾನಿ ಅಜ್ಞಾನಿಗೆ ಜೊತೆ ಕೊಟ್ಟು ಮುಂದಕ್ಕೆ ಕರೆದುಕೊಂಡು ಹೋಗಬೇಕಾಗಿದೆ.

\begin{verse}
ಪ್ರಕೃತೇಃ ಕ್ರಿಯಮಾಣಾನಿ ಗುಣೈಃ ಕರ್ಮಾಣಿ ಸರ್ವಶಃ ।\\ಅಹಂಕಾರವಿಮೂಢಾತ್ಮಾ ಕರ್ತಾಹಮಿತಿ ಮನ್ಯತೇ \versenum{॥ ೨೭ ॥}
\end{verse}

{\small ಪ್ರಕೃತಿಯ ಗುಣಗಳಿಂದಲೇ ಎಲ್ಲಾ ಬಗೆಯ ಕರ್ಮಗಳೂ ಆಗುತ್ತಿವೆ. ಅಹಂಕಾರದಿಂದ ಮೂಢನಾದವನು ನಾನು ಮಾಡುತ್ತೇನೆ ಎಂದು ಭಾವಿಸುತ್ತಾನೆ.}

ಚೇತನ ಸಾಕ್ಷಿ, ಅವನಿಂದ ಹೊರಗಡೆ ಇರುವುದೇ ಪ್ರಕೃತಿ. ಇಲ್ಲಿ ಹೊರಗಡೆ ಎಂದರೆ ಈ ಚೇತನಕ್ಕೆ ಹೊರಗಡೆ ಇರುವುದು. ದೇಹ ಮನಸ್ಸು ಇಂದ್ರಿಯ ಇವುಗಳೆಲ್ಲ ಚೇತನಕ್ಕೆ ಹೊರಗಿನದು. ಇದ ರಂತೆಯೇ ದೇಶ ಕಾಲ ನಿಮಿತ್ತ ಪ್ರಪಂಚದಲ್ಲಿ ಪಂಚಭೂತಗಳಿಂದಾದ ವಸ್ತುಗಳು. ಇಲ್ಲೆಲ್ಲ ಮೂರು ಗುಣಗಳು ಇರುವುದನ್ನು ನೋಡುವೆವು. ಇವೇ ತಮಸ್ಸು, ರಜಸ್ಸು, ಸತ್ತ್ವ. ಎಲ್ಲಾ ಕೆಲಸಗಳೂ ಅದು ಕೆಟ್ಟದ್ದಾಗಿರಲಿ, ಒಳ್ಳೆಯದು ಮತ್ತು ಕೆಟ್ಟದ್ದುದರ ಮಿಶ್ರವಾಗಿರಲಿ, ಒಳ್ಳೆಯ ದಾಗಿರಲಿ ಇವುಗಳೆಲ್ಲ ತಮಸ್ಸು ರಜಸ್ಸು ಮತ್ತು ಸತ್ತ್ವಗುಣದಿಂದ ಆಗಿವೆ. ಆಯಾ ಗುಣಗಳಿಗೆ ತಕ್ಕಂತೆ ಮನುಷ್ಯ ಕರ್ಮ ಮಾಡುತ್ತಿರುವನು. ಸಂಪಿಗೆ ಗಿಡದಲ್ಲಿ ಒಂದು ವಿಧವಾದ ಪರಿಮಳ, ಗುಲಾಬಿಯಲ್ಲಿ ಒಂದು ವಿಧವಾದ ಪರಿಮಳ, ಗೊಬ್ಬಳಿ ಗಿಡದಲ್ಲಿ ಒಂದು ವಿಧದ ಪರಿಮಳ ಇದೆ. ಅದು ತನ್ನಲ್ಲಿರುವುದನ್ನು ಮಾತ್ರ ವ್ಯಕ್ತಗೊಳಿಸಬಲ್ಲುದು. ಇಲ್ಲದುದನ್ನು ತರುವುದಕ್ಕೆ ಆಗುವುದಿಲ್ಲ. ಅದರಂತೆಯೇ ಮನುಷ್ಯ ಹುಟ್ಟುವಾಗಲೇ ತಮೋಗುಣಕ್ಕೋ, ರಜೋಗುಣಕ್ಕೋ ಸತ್ತ್ವಗುಣಕ್ಕೋ ವಶವಾಗಿ ಹುಟ್ಟುವನು. ಆಯಾ ಗುಣಕ್ಕೆ ತಕ್ಕಂತೆ ಕರ್ಮಗಳನ್ನು ಮಾಡಿಕೊಂಡು ಹೋಗುವನು.

ತಾನು ಮಾಡುತ್ತೇನೆ ಎಂದು ಅಹಂಕಾರದಿಂದ ಹೇಳುವನು. ಮಾಡುವುದನ್ನು ಬಿಡುವುದಕ್ಕೆ ಸ್ವಾತಂತ್ರ್ಯವಿದ್ದರೆ ತಾನೆ ಹಾಗೆ ಹೇಳಬಹುದು. ಅವನು ತನ್ನ ಪ್ರಕೃತಿಗೆ ಅನುಗುಣವಾಗಿ ಆ ಕೆಲಸವನ್ನೇ ಮಾಡಲೇ ಬೇಕಾಗಿದೆ. ಬಿಡುವುದಕ್ಕೆ ಆಗುವುದಿಲ್ಲ. ಇವನು ಬಿಟ್ಟರೆ ಪ್ರಕೃತಿ ಸುಮ್ಮನೆ ಇರುವುದಿಲ್ಲ. ಚಾವಟಿ ಏಟಿನಿಂದ ಹೊಡೆದು ಇವನ ಕೈಯಿಂದ ಮಾಡಿಸುವುದು. ಗಾಡಿಯ ಕುದುರೆ ಹೋಗುವುದಿಲ್ಲ ಎಂದು ಮೊಂಡಾಟ ಮಾಡಿದರೆ, ಹೊಡೆಯುವ ಸಾಹೇಬನೇನೂ ಬಿಡುವುದಿಲ್ಲ. ಹೊಡೆಯುತ್ತಾನೆ, ಗಾಡಿಯನ್ನು ನೂಕುತ್ತಾನೆ, ತಳ್ಳುತ್ತಾನೆ. ಅಂತೂ ಮುಂದುವರಿಯುವಂತೆ ಮಾಡುತ್ತಾನೆ.

\begin{verse}
ತತ್ತ್ವವಿತ್ತು ಮಹಾಬಾಹೋ ಗುಣಕರ್ಮ ವಿಭಾಗಯೋಃ ।\\ಗುಣಾ ಗುಣೇಶು ವರ್ತಂತ ಇತಿ ಮತ್ವಾ ನ ಸಜ್ಜತೇ \versenum{॥ ೨೮ ॥}
\end{verse}

{\small ಅರ್ಜುನ, ಗುಣ ಕರ್ಮಗಳ ವಿಭಾಗಗಳನ್ನು ತಿಳಿದ ತತ್ತ್ವಜ್ಞಾನಿ, ಗುಣಗಳು ಗುಣಗಳಲ್ಲಿ ಪ್ರವರ್ತಿಸುತ್ತಿವೆ ಎಂದು ತಿಳಿದು ತಾನು ಆಸಕ್ತನಾಗುವುದಿಲ್ಲ.}

ತತ್ತ್ವಜ್ಞಾನಿ ಎಂದರೆ ವಸ್ತು ಹೇಗೆ ಇದೆಯೋ ಹಾಗೆ ತಿಳಿದುಕೊಳ್ಳತಕ್ಕವನು. ಅದು ನಮಗೆ ಹೇಗೆ ತೋರುತ್ತದೆಯೋ ಅದನ್ನೇ ಸತ್ಯವೆಂದು ಭಾವಿಸುವುದಿಲ್ಲ. ಈ ಪ್ರಪಂಚದಲ್ಲಿ ಸತ್ಯ ಬೇರೆ, ತೋರಿಕೆ ಅಥವಾ ಮಿಥ್ಯಾ ಬೇರೆ. ಇವೆರಡು ಹೇಗೋ ಕಲಸುಮೇಲೋಗರವಾಗಿ ಬಿಟ್ಟಿದೆ. ಸತ್ಯ ಕಾಣುತ್ತಿಲ್ಲ. ಮಿಥ್ಯ ಮಾತ್ರ ಕಾಣಿಸುವಂತೆ ಇದೆ. ಸಕ್ಕರೆ ಮತ್ತು ಚೆನ್ನಾಗಿರುವ ಬಿಳಿ ಮರಳನ್ನು ಬೆರಸಿದರೆ ಹೇಗೊ ಹಾಗಾಗಿದೆ. ಇರುವೆಯ ಮುಂದೆ ಅದನ್ನು ಎಸೆದರೆ ಮರಳ ಕಣವನ್ನು ಬಿಡುವುದು, ಸಕ್ಕರೆ ಕಣವನ್ನು ಮಾತ್ರ ಆರಿಸಿಕೊಳ್ಳುವುದು. ತತ್ತ್ವಜ್ಞಾನಿ ಇಂತಹವನು.

ಕರ್ಮವನ್ನು ವಿಭಜನೆ ಮಾಡಿದರೆ ಅದರಲ್ಲಿ ಹೊರಗಿನ ಭಾಗ ಮತ್ತು ಒಳಗಿನ ಭಾಗ, ಎರಡನ್ನು ನೋಡುತ್ತೇವೆ. ಹೊರಗೆ ಇರುವುದು ಕರ್ಮ, ಆ ಕರ್ಮದ ಹಿಂದೆ ಇರುವುದು ಗುಣ. ಈ ಗುಣ ಮೂರು ವಿಧವಾಗಿದೆ. ಈ ಗುಣಕ್ಕೆ ತಕ್ಕ ಕರ್ಮ ಪ್ರತಿಯೊಬ್ಬ ವ್ಯಕ್ತಿಯ ಮೂಲಕವೂ ಆಗುತ್ತಿರು ವುದು. ಅಜ್ಞಾನಿಯಲ್ಲಿ ಇವೆರಡೂ ಮಿಶ್ರವಾಗಿ ಹೋಗಿವೆ. ಜ್ಞಾನಿಯಾದರೋ ಇದನ್ನು ವಿಭಜನೆ ಮಾಡಿ, ಗುಣವನ್ನು ಬೇರೆ ಕರ್ಮವನ್ನು ಬೇರೆ ಇಡುವನು.

ಗುಣಗಳು ಗುಣಗಳಲ್ಲಿ ಪ್ರವರ್ತಿಸುತ್ತಿವೆ ಎಂದು ತಿಳಿದು ಅವನು ಆಸಕ್ತನಾಗುವುದಿಲ್ಲ. ಸತ್ತ್ವಗುಣದಿಂದ ಪ್ರೇರಿತವಾದ ಕೆಲಸ ಈ ಪ್ರಪಂಚದಲ್ಲಿ ಒಳ್ಳೆಯ ಕೆಲಸದಲ್ಲಿ ಪರ್ಯವಸಾನವಾಗು ತ್ತದೆ. ರಜೋಗುಣದಿಂದ ಪ್ರೇರಿತವಾದ ಕೆಲಸ, ಮಿಶ್ರವಾದ ಕೆಲಸದಲ್ಲಿ ಪರ್ಯವಸಾನವಾಗುವುದು. ತಮೋಗುಣದಿಂದ ಪ್ರೇರಿತವಾದುವು ಅಜ್ಞಾನ, ಮರವು, ತಪ್ಪು–ಈ ಕೆಲಸಗಳಲ್ಲಿ ಪರ್ಯವಸಾನ ವಾಗುವುವು. ಗುಣಗಳು ಮೇಲಿನಿಂದ ಬಿದ್ದು ವಿಷಯ ವಸ್ತುವಿನ ಕಡೆ ಹರಿದು ಹೋಗುತ್ತಿವೆ. ಚೇತನ ವಾದರೋ ಇದನ್ನು ನೋಡುವವನು. ಅಜ್ಞಾನಿ ಗುಣಗಳಲ್ಲಿ ತಾದಾತ್ಮ ್ಯ ಭಾವವನ್ನು ಪಡೆದು ತಾನು ಮಾಡುತ್ತೇನೆ, ತಾನು ಅನುಭವಿಸುತ್ತೇನೆ ಎಂದು ಭಾವಿಸುತ್ತಾನೆ. ಜ್ಞಾನಿಯಾದರೊ ನಾನು ಸಾಕ್ಷಿ, ನನ್ನ ಎದುರಿಗೆ ಇವು ಆಗುತ್ತಿವೆ; ನಾನು ಬೇರೆ ಅದೇ ಬೇರೆ ಎಂದು ತಿಳಿದು ಅನಾಸಕ್ತನಾಗುವನು. ಯಾವಾಗ ಒಬ್ಬ ಹೊರಗಡೆ ಇರುವ ವಿಷಯಗಳು ಮತ್ತು ಒಳಗಡೆ ಇರುವ ಗುಣ ಇವುಗಳಿಗೆ ಸಾಕ್ಷಿಯಾಗಿ ನಿಲ್ಲಬಲ್ಲನೋ ಅವನು ಇನ್ನು ಮೇಲೆ ಅದರಿಂದ ಬಾಧಿತನಾಗುವುದಿಲ್ಲ. ನಾವು ಬಾಹ್ಯ ಆಕರ್ಷಣೆಯಲ್ಲಿ ಮುಳುಗಿ ಹೋಗುತ್ತೇವೆ. ನಮ್ಮದುಃಖ ದುರಿತಗಳಿಗೆಲ್ಲಾ ಅದೇ ಕಾರಣ. ನಾವು ಯಾವಾಗ ಆ ಸೆಳೆತದಿಂದ ಹೊರಗೆ ಬಂದು ನಿಲ್ಲುತ್ತೇವೆಯೋ ಆಗ ಮುಕ್ತಾತ್ಮರು.

\begin{verse}
ಪ್ರಕೃತೇರ್ಗುಣಸಂಮೂಢಾಃ ಸಜ್ಜಂತೇ ಗುಣಕರ್ಮಸು \eng{।\\}ತಾನಕೃತ್ಸ್ನವಿದೋ ಮಂದಾನ್ ಕೃತ್ಸ್ನವಿನ್ನ ವಿಚಾಲಯೇತ್ \versenum{॥ ೨೯ ॥}
\end{verse}

{\small ಪ್ರಕೃತಿಯ ಗುಣಗಳಿಂದ ಮೋಹಿತರಾದವರು ಗುಣ ಮತ್ತು ಕರ್ಮಗಳಲ್ಲಿ ಆಸಕ್ತರಾಗುತ್ತಾರೆ. ಆತ್ಮಜ್ಞನು ಅಲ್ಪಜ್ಞರಾದ ಮಂದಮತಿಗಳನ್ನು ಅಲುಗಾಡಿಸಬಾರದು.}

ಯಾರು ಗುಣ ಮತ್ತು ಕರ್ಮಗಳನ್ನು ವಿಭಜನೆ ಮಾಡಲಾರರೋ ಅವರು ಅದರಿಂದ ಬರುವ ಫಲಾಫಲಗಳಿಗೆ ವಶರಾಗುತ್ತಾರೆ. ಏನಾದರೂ ಒಳ್ಳೆಯದಾದರೆ ಸಂತೋಷಪಡುತ್ತಾರೆ, ಮೆರೆದಾಡು ತ್ತಾರೆ. ಕೆಟ್ಟದ್ದಾದರೆ ಕೊರಗುತ್ತಾರೆ. ಗುಣದೊಂದಿಗೆ ತಾದಾತ್ಮ್ಯ ಭಾವವನ್ನು ಪಡೆದವನ ಸ್ಥಿತಿಯೇ ಇದು. ಆದರೆ ಜ್ಞಾನಿ ಅದನ್ನು ನೋಡಿದಾಗ ಹಂಗಿಸಬಾರದು. ಹಾಗೆಮಾಡಿದರೆ ಅವನು ಕೆಲಸ ಮಾಡುವುದನ್ನೇ ಬಿಡುತ್ತಾನೆ. ಕೆಲಸ ಮಾಡು, ಮತ್ತೂ ಚೆನ್ನಾಗಿ ಮಾಡು ಎಂದು ಅವನನ್ನು ಸರಿಯಾದ ದಾರಿಗೆ ತಿರುಗಿಸಬೇಕೇ ಹೊರತು ಹಿಂತಿರುಗಿ ಹೋಗುವಂತೆ ಮಾಡಬಾರದು. ಎಲ್ಲರಲ್ಲಿಯೂ ಪ್ರಾರಂಭದಲ್ಲಿಯೇ ಅನಾಸಕ್ತಿಯ ಬುದ್ಧಿ ಸಮತ್ವ ಮುಂತಾದವು ಉದಿಸುವುದಿಲ್ಲ. ಇದೆಲ್ಲ ಕರ್ಮಗಿಡ ಬೆಳೆದುಕೊಂಡು ಹೋದಮೇಲೆ ಅದರಲ್ಲಿ ಬಿಡುವ ಒಳ್ಳೆಯ ಹೂವು ಹಣ್ಣುಗಳು ಹೇಗೊ ಹಾಗೆ. ಜ್ಞಾನಿಯಾದವನು, ಅಜ್ಞಾನಿಗೆ ಬೆಳೆಯುವುದಕ್ಕೆ ಸಾಕಷ್ಟು ಕಾಲ ಕೊಟ್ಟು ಕ್ರಮೇಣ ಅವನನ್ನು ತಿದ್ದಬೇಕು. ಜೀವಿ ಆಸಕ್ತಿಯಿಂದ ಅನಾಸಕ್ತಿಯ ಕಡೆ ಪ್ರಯಾಣ ಮಾಡುತ್ತಿರುವನು. ಮಧ್ಯ ಎಲ್ಲೂ ನಿಲ್ಲದೆ ಇದ್ದರೆ ಅನಾಸಕ್ತಿಯ ಗುರಿಯನ್ನು ಎಲ್ಲರೂ ಒಂದಲ್ಲ ಒಂದು ಸಮಯದಲ್ಲಿ ಸೇರಿಯೇ ಸೇರುವರು.

\begin{verse}
ಮಯಿ ಸರ್ವಾಣಿ ಕರ್ಮಾಣಿ ಸಂನ್ಯಸ್ಯಾಧ್ಯಾತ್ಮಚೇತಸಾ ।\\ನಿರಾಶೀರ್ನಿರ್ಮಮೋ ಭೂತ್ವಾ ಯುಧ್ಯಸ್ವ ವಿಗತಜ್ವರಃ \versenum{॥ ೩೦ ॥}
\end{verse}

{\small ನನ್ನಲ್ಲಿ ಸರ್ವ ಕರ್ಮಗಳನ್ನು ಅಧ್ಯಾತ್ಮ ದೃಷ್ಟಿಯ ಮೂಲಕ ಅರ್ಪಣೆ ಮಾಡಿ, ಆಶೆ ಇಲ್ಲದೆ, ಮಮತೆ ಇಲ್ಲದೆ ಉದ್ವೇಗವಿಲ್ಲದೆ ಯುದ್ಧವನ್ನು ಮಾಡು.}

ಶ್ರೀಕೃಷ್ಣ ಅನಾಸಕ್ತನಾಗಿ ಕರ್ಮವನ್ನು ಮಾಡುವುದಕ್ಕೆ ಇಲ್ಲಿ ಒಂದು ಊರುಗೋಲನ್ನು ಕೊಡು ವನು. ಅದೇ ಭಗವಂತನನ್ನು ನಮ್ಮ ಮುಂದೆ ಇಟ್ಟುಕೊಂಡು, ಅವನಿಗೆ ಅರ್ಪಣೆ ಮಾಡುವ ದೃಷ್ಟಿಯಿಂದ ಮಾಡುವುದು. ಇಲ್ಲಿ ‘ಎಲ್ಲಾ ಕರ್ಮಗಳನ್ನೂ’ ಎನ್ನುತ್ತಾನೆ. ನಮ್ಮ ಪಾಲಿಗೆ ಬರುವ ಕರ್ಮಗಳೆಲ್ಲ, ಎಂತಹ ಕರ್ಮ ಬೇಕಾದರೂ ಆಗಬಹುದು, ಅದನ್ನೆಲ್ಲಾ ವಿರಾಟ್ ರೂಪಿಯಾಗಿರುವ ಭಗವಂತನಿಗೆ ಅರ್ಪಣೆ ಮಾಡಬೇಕು. ಯಾವಾಗ ನಾವು ಅರ್ಪಿತ ಭಾವದಿಂದ ಕರ್ಮ ಮಾಡುವೆವೋ ಆಗ ಮಾಡುವುದೆಲ್ಲ ಪವಿತ್ರವಾಗುವುದು, ಮಾಡುವುದೆಲ್ಲ ಪೂಜೆಯಾಗುವುದು. ಹಾಗೆ ಮಾಡುವಾಗ ಫಲಾಪೇಕ್ಷೆಯನ್ನು ಬಿಡುವನು. ಜನರ ನಿಂದೆ, ಟೀಕೆ, ಹೊಗಳಿಕೆಗಳಿಗೆ ಮನಗೊಡನು. ಭಗವಂತ ನೊಬ್ಬನೆ ಧ್ರುವತಾರೆಯಂತೆ ಸದಾ ಅವನ ಮುಂದಿರುವನು. ಕೆಲಸದ ಹಿಂದೆ ಅವನಿಗೆ ಮಮತೆ ಇರುವುದಿಲ್ಲ. ಇದು ನನ್ನ ಕೆಲಸ ಎಂದು ಭಾವಿಸುವುದಿಲ್ಲ. ಇದು ಅವನ ಕೆಲಸ, ನಾನೊಬ್ಬ ಅದನ್ನು ಮಾಡುವುದಕ್ಕೆ ದೇವರು ಆರಿಸಿಕೊಂಡಿರುವ ನಿಮಿತ್ತ ಎಂದು ಮಾತ್ರ ಭಾವಿಸುವನು. ಯಾವಾಗ ಇವನು ಅದಕ್ಕೆ ಅಂಟಿಕೊಂಡಿರುವುದಿಲ್ಲವೋ ಆಗ ಯಾವಾಗ ಬೇಕಾದರೂ ದೇವರು ಇಚ್ಛಿಸಿದರೆ ಅದನ್ನು ಬಿಡುವುದಕ್ಕೆ ಸಿದ್ಧನಾಗಿರುತ್ತಾನೆ. ಹಾಗೆ ಅವನು ಕೆಲಸ ಮಾಡುವಾಗ ಯಾವ ಉದ್ವೇಗ ವಾಗಲಿ, ಬಿಸಿಯಾಗಲಿ, ತಾಪವಾಗಲಿ ಇರುವುದಿಲ್ಲ. ಯಾವ ಗಾಡಿಯ ಚಕ್ರಗಳಿಗೆ ಎಣ್ಣೆಯನ್ನು ಸವರಿಲ್ಲವೋ ಅದು ಬೇಕಾದಷ್ಟು ಶಬ್ದ ಮಾಡುವುದು. ಆದರೆ ಅದಕ್ಕೆ ಚೆನ್ನಾಗಿ ಎಣ್ಣೆ ಸವರಿದ್ದರೆ ಶಬ್ದ ಮಾಡುವುದಿಲ್ಲ. ಭಕ್ತ ತಾನು ಮಾಡುವ ಕೆಲಸದ ಹಿಂದೆ, ಇದು ಭಗವದರ್ಪಿತವಾಗಲಿ ಎಂಬ ತೈಲವನ್ನು ಸವರುವನು. ಅವನು ಅಷ್ಟೊಂದು ಕೆಲಸ ಮಾಡುತ್ತಿರುವನು. ಆದರೂ ಎಳ್ಳಷ್ಟೂ ಅವನಲ್ಲಿ ಉದ್ವೇಗವಿರುವುದಿಲ್ಲ. ದೇವರನ್ನು ಮರೆತಾಗಲೇ ಈ ಉದ್ವೇಗ, ಘರ್ಷಣೆ ಗಲಾಟೆಯೆಲ್ಲ.

\begin{verse}
ಯೇ ಮೇ ಮತಮಿದಂ ನಿತ್ಯಮನುತಿಷ್ಠಂತಿ ಮಾನವಾಃ ।\\ಶ್ರದ್ಧಾವಂತೋಽನಸೂಯಂತೋ ಮುಚ್ಯಂತೇ ತೇಽಪಿ ಕರ್ಮಭಿಃ \versenum{॥ ೩೧ ॥}
\end{verse}

{\small ಯಾರು ಶ್ರದ್ಧೆಯಿಂದ ಅಸೂಯೆಯನ್ನು ತೊರೆದು ಸದಾ ಕರ್ಮವನ್ನು ಮಾಡುವರೋ ಅವರೂ ಕೂಡ ಕರ್ಮಗಳಿಂದ ಬಿಡುಗಡೆ ಹೊಂದುತ್ತಾರೆ.}

ಯಾರು ಶ್ರೀಕೃಷ್ಣ ಹಿಂದೆ ಹೇಳಿದ ಶ್ಲೋಕದ ಭಾವದಲ್ಲಿ ಶ್ರದ್ಧೆ ಇಟ್ಟು ಕರ್ಮ ಮಾಡುವರೋ ಅಂಥವರೂ ಕೂಡ ಕರ್ಮ ಪಾಶಗಳಿಂದ ಬಿಡುಗಡೆ ಹೊಂದುವರು. ಶ್ರದ್ಧೆಯೆಂಬ ಪದ ಧ್ವನಿಪೂರಿತ ವಾದುದು. ಯಾರು, ಆ ಮಾತು ಸತ್ಯ, ಸಾಕ್ಷಾತ್ ಭಗವಂತನ ಬಾಯಿಂದಲೇ ಬಂದಿದೆ, ಅದು ನಮ್ಮನ್ನು ಉದ್ಧಾರಮಾಡುವುದು ಎಂದು ನಂಬುತ್ತಾರೋ ಅವರು ಉದ್ಧಾರವಾಗುವರು. ಈ ಒಂದು ತತ್ತ್ವ ಅಗಾಧವಾದ ಆಧ್ಯಾತ್ಮಿಕ ನಿಯಮ. ಅರ್ಪಣ ಭಾವದಿಂದ ಮಾಡಿದ ಫಲಗಳನ್ನು ದೇವರು ಸ್ವೀಕರಿಸಿ, ನಮ್ಮ ಮನಸ್ಸನ್ನು ಶುದ್ಧಮಾಡಿ, ನಮಗೆ ಜ್ಞಾನ ನೀಡಿ, ನಮ್ಮನ್ನು ಬಂಧನದಿಂದ ಪಾರು ಮಾಡುವನು. ಅವನನ್ನು ನಂಬಿ ನಡೆದರೆ ಅವನು ನಮ್ಮನ್ನು ತೊರೆಯುವುದಿಲ್ಲ.

ಭಗವಂತನಲ್ಲಿ ಇಡುವ ಶ್ರದ್ಧೆಯ ಜೊತೆಗೆ ಭಕ್ತರಲ್ಲಿ ಯಾವ ವ್ಯಕ್ತಿಯ ಮೇಲೆಯೂ ಧರ್ಮದ ಮೇಲೆಯೂ ಅಸೂಯೆ ಇರಕೂಡದು, ಅಸೂಯೆ ಇಲ್ಲದಿರುವುದು ಎಂಬುದಕ್ಕೆ ಪರೋತ್ಕರ್ಷ ಸಹಿಷ್ಣುತೆ ಎನ್ನುವರು. ಇನ್ನೊಬ್ಬ ಮುಂದುವರಿದು ಹೋಗುತ್ತಿದ್ದರೆ ಅವನನ್ನು ನೋಡಿ ಕರುಬು ವುದಲ್ಲ, ದೇವರು ಅವನಿಗೆ ಒಳ್ಳೆಯದನ್ನು ಮಾಡಲಿ, ಅವನು ಹೋಗಿ ಗುರಿ ಸೇರಲಿ ಎಂದು ಹರಸಬೇಕು. ನನಗೆ ಇಲ್ಲದುದು ಇನ್ನೊಬ್ಬನಿಗೆ ಬಂದಾಗ ಅಸೂಯೆ ಪಡಕೂಡದು. ಅವನಿಗಾದರೂ ಬಂತಲ್ಲ ಎಂದು ಅವನ ಆನಂದದಲ್ಲಿ ಭಾಗಿಯಾಗುವಂತಹ ಉದಾರ ಬುದ್ಧಿ ಇರಬೇಕು. ಭಗವಂತನೆಡೆಗೆ ಒಬ್ಬ ಹೋಗುತ್ತಿರುವನು ಎಂಬುದಕ್ಕೆ ಒಂದು ಲಕ್ಷಣವೇ ಅವನಲ್ಲಿ ಸಣ್ಣತನವಿಲ್ಲದೆ ಇರುವುದು. ಅವನು ಯಾರನ್ನು ಕಂಡರೂ ಹೊಟ್ಟೆಕಿಚ್ಚು ಪಡುವುದಿಲ್ಲ. ಇದೊಂದು ಹೃದಯದ ಹಿರಿಮೆಯ ಹೆಗ್ಗುರುತು.

ಯಾರು ಹೀಗೆ ಅರ್ಪಿತ ಭಾವದಿಂದ ಕರ್ಮ ಮಾಡುವರೋ ಅಂತಹವರೂ ಕೂಡ ಬಂಧನದಿಂದ ಪಾರಾಗುವರು ಎಂದು ಸಾರುವನು ಶ್ರೀಕೃಷ್ಣ. ಹೀಗೆ ಅರ್ಪಿತಭಾವದಿಂದ ಕೆಲಸ ಮಾಡುವವನಿಗೆ ಶ್ರದ್ಧೆಯೊಂದಿದೆ ಅಷ್ಟೆ. ಅವನಿಗೆ ಶಾಸ್ತ್ರ ಗೊತ್ತಿಲ್ಲದೆ ಇರಬಹುದು, ಜ್ಞಾನ ಗೊತ್ತಿಲ್ಲದೆ ಇರಬಹುದು. ಅವನು ಯಾವ ವರ್ಣಕ್ಕಾಗಲಿ ಆಶ್ರಮಕ್ಕಾಗಲಿ ಸೇರಿರಬಹುದು. ಆದರೂ ಚಿಂತೆಯಿಲ್ಲ. ಅವನು ಉದ್ಧಾರವಾಗುವುದರಲ್ಲಿ ಸಂದೇಹವಿಲ್ಲ. ಶ್ರದ್ಧೆಯೊಂದಿದ್ದರೆ ಸಾಕು, ಇನ್ನು ಏನೇನು ಆವಶ್ಯಕವೋ ಅದನ್ನೆಲ್ಲಾ ದೇವರೆ ಅವನಿಗೆ ಒದಗಿಸಿ ಕೊಡುವನು. ವೈದ್ಯ ರೋಗಿಯೊಬ್ಬನಿಗೆ ಯಾವುದೋ ಖಾಯಿಲೆಗೆ ಎಂದು ಔಷಧಿ ಕೊಡುತ್ತಾನೆ. ರೋಗಿಯಾದರೊ ದಡ್ಡ. ಅವನಿಗೆ ಈ ಖಾಯಿಲೆ ಹೇಗೆ ಬಂತೊ ಗೊತ್ತಿಲ್ಲ. ವೈದ್ಯ ಕೊಡುವ ಔಷಧಿ ತನ್ನ ದೇಹದ ಮೇಲೆ ಯಾವ ಪರಿಣಾಮವನ್ನು ಹೇಗೆ ಉಂಟುಮಾಡುವುದು, ಇದಾವುದೂ ಗೊತ್ತಿಲ್ಲ. ಆದರೂ ವೈದ್ಯನ ಔಷಧಿಯನ್ನು ಶ್ರದ್ಧೆಯಿಂದ ತೆಗೆದುಕೊಂಡರೆ ಅವನು ರೋಗದಿಂದ ಪಾರಾಗುವನು. ಆ ಔಷಧಿ ರೋಗಿಯನ್ನು, ನಿನಗೆ ಆರೋಗ್ಯ ಶಾಸ್ತ್ರ ಗೊತ್ತಿದೆಯೆ, ಅನಾಟಮಿ ಗೊತ್ತಿದೆಯೆ, ಈ ಖಾಯಿಲೆ ಯಾವುದರಿಂದ ಹೇಗೆ ಆಗಿದೆ ಎನ್ನುವುದು ಗೊತ್ತಿದೆಯೆ ಎಂದು ಕೇಳಿ ಅದಕ್ಕೆ ಅವನಿಂದ ಸಮರ್ಪಕವಾದ ಉತ್ತರ ಬಂದಮೇಲೆ ತನ್ನ ಕೆಲಸವನ್ನು ಪ್ರಾರಂಭಿಸುವುದಿಲ್ಲ. ಅದು ಹೊಟ್ಟೆಗೆ ಹೋದಮೇಲೆ ಅದು ತನ್ನ ಕೆಲಸವನ್ನು ಯಾರ ಮೇಲಾಗಲಿ ಮಾಡಿಯೇ ಮಾಡುವುದು.

ಇಂತಹ ಒಂದು ಶ್ರದ್ಧೆ ಭಗವಂತನ ಮಾತಿನಲ್ಲಿ ಇರಬೇಕು. ಅನೇಕ ವೇಳೆ ಇದು ಅದ್ಭುತವಾದ ಪರಿಣಾಮವನ್ನುಂಟುಮಾಡುವುದು. ಒಬ್ಬ ವೇದ ಶಾಸ್ತ್ರಗಳನ್ನು ತಿಳಿದಿರಬಹುದು. ಆದರೆ ಅದನ್ನು ತಾನೆ ಹೃತ್ಪೂರ್ವಕವಾಗಿ ನಂಬುವುದಿಲ್ಲ. ಇನ್ನೊಬ್ಬನಿಗೆ ಬೇಕಾದರೆ ಅದನ್ನು ಬೋಧನೆ ಮಾಡುವನು. ಇಂತಹ ಪಂಡಿತನಿಗಿಂತ ನಿರಕ್ಷರಕುಕ್ಷಿಯಲ್ಲಿರುವ ಶ್ರದ್ಧೆ ಅದ್ಭತವಾದ ಕೆಲಸವನ್ನು ಮಾಡುವುದು. ಶ್ರೀರಾಮಕೃಷ್ಣರು ಇದನ್ನು ವಿವರಿಸುವುದಕ್ಕೆ ಒಂದು ಸಣ್ಣ ಕಥೆಯನ್ನು ಹೇಳುವರು. ಒಬ್ಬ ಬ್ರಾಹ್ಮಣ ಪಂಡಿತನ ಮನೆಗೆ ಹಾಲು ಮಾರುವ ಹೆಂಗಸೊಬ್ಬಳು ಪ್ರತಿದಿನ ಬೆಳಗ್ಗೆ ಹಾಲನ್ನು ತಂದುಕೊಡು ತ್ತಿದ್ದಳು. ಇವಳು ಬರಬೇಕಾದ ದಾರಿಯಲ್ಲಿ ಒಂದು ನದಿ ಇತ್ತು. ಒಂದು ದಿನ, ಹಿಂದಿನ ರಾತ್ರಿ ಜಾಸ್ತಿ ಮಳೆ ಬಂದದ್ದರಿಂದ, ನಡೆದು ದಾಟಲು ಆಗದೆ ದೋಣಿಗೆ ಕಾಯಬೇಕಾಯಿತು. ಬ್ರಾಹ್ಮಣನ ಮನೆಗೆ ಅಂದು ಹಾಲನ್ನು ತರಲು ಹೊತ್ತಾಯಿತು. ಆತ ಏತಕ್ಕೆ ಇಂದು ಹೊತ್ತಾಗಿ ಬಂದೆ ಎಂದು ಕೇಳಿದ. ಆಕೆ ದೋಣಿಗಾಗಿ ಕಾಯಬೇಕಾಗಿ ಬಂದುದರಿಂದ ಹೊತ್ತಾಯಿತು ಎಂದಳು. ಈ ಬ್ರಾಹ್ಮಣ ಶಾಸ್ತ್ರ ಬಲ್ಲವನು ತಾನೆ? ಆತ, ಜನ ದೇವರ ಹೆಸರನ್ನು ಹೇಳಿಕೊಂಡು ಭವಸಾಗರವನ್ನು ದಾಟುತ್ತಾರೆ. ನೀನು ಮಾರುದ್ದ ನದಿಯನ್ನು ದಾಟಲಾರೆಯ ಎಂದು ಹೇಳಿದ. ಆ ಹಾಲು ಮಾರುವ ಹೆಂಗಸು ಇದರಲ್ಲಿ ನಂಬಿದಳು. ಮಾರನೆಯ ದಿನ ನದಿತುಂಬಿ ಹರಿಯುತ್ತಿದ್ದರೂ ಹೊತ್ತಿಗೆ ಸರಿಯಾಗಿ ಹಾಲನ್ನು ತಂದು ಕೊಟ್ಟಳು. ಈ ಪಂಡಿತನು ಕೇಳಿದನು, ನೆನ್ನೆಯೂ ಮಳೆ ಬಂದಿತ್ತಲ್ಲ ಆದರೂ ನೀನು ಬೇಗ ಬಂದೆಯಲ್ಲ ಇಂದು ಎಂದು. ಆಕೆ, ನೆನ್ನೆ ನೀವೇ ಹೇಳಿದಂತೆ, ಭಗವಂತನ ಹೆಸರನ್ನು ಹೇಳಿಕೊಂಡು ನದಿ ದಾಟಿದೆ ಎಂದಳು. ಆ ಪಂಡಿತನಿಗೆ ಇದನ್ನು ನಂಬಲು ಆಗಲಿಲ್ಲ. ತಾನು ಹೇಳಿದ್ದರಲ್ಲಿಯೇ ತನಗೆ ನಂಬಿಕೆ ಇಲ್ಲ. ಅದನ್ನು ನಾನು ನೋಡಿದಲ್ಲದೆ ನಂಬುವುದಿಲ್ಲ ಎಂದನು. ಹಾಲಿನವಳ ಹಿಂದೆ ನದೀತೀರಕ್ಕೆ ಹೋದನು. ಅವಳು ಪಂಡಿತನಿಗೆ ಕಾಣುವಂತೆ ಭಗವಂತನ ಹೆಸರನ್ನು ತೆಗೆದುಕೊಂಡು ನದಿಯ ಮೇಲೆ ನಡೆದುಕೊಂಡು ಹೋದಳು. ಪಂಡಿತನಿಗೆ ದೇವರ ಹೆಸರು ತೇಲಿಸುವುದು ಎಂಬುದು ಆಗ ಗೊತ್ತಾಯಿತು. ತಾನೂ ನಡೆಯಲು ನದಿಯ ಮೇಲೆ ಹೊರಟ. ದೇವರ ಹೆಸರನ್ನು ಹೇಳಿ ಹೋಗುತ್ತಿದ್ದಂತೆ ಪಂಚೆ ಒದ್ದೆಯಾಗದಿರಲಿ ಎಂದು ಮೇಲಕ್ಕೆ ಎತ್ತುತ್ತ ಹೋದ. ಈತ ಬಟ್ಟೆಯನ್ನು ಮೇಲಕ್ಕೆ ಎತ್ತುತ್ತಿದ್ದಂತೆ ನೀರು ಮೇಲುಮೇಲಕ್ಕೆ ಬಂತು. ಅವನು ಮುಂದೆ ಹೋಗುವ ಹೆಂಗಸಿಗೆ ನಾನು ಮುಳುಗುತ್ತಿದ್ದೇನಲ್ಲಾ ಎಂದು ಅರಚಿಕೊಂಡ. ಆ ಹೆಂಗಸಾದರೋ, ದೇವರ ಹೆಸರನ್ನು ಹೇಳಿ ಪಂಚೆಯನ್ನೂ ಮೇಲಕ್ಕೆತ್ತುತ್ತಿದ್ದರೆ ಮುಳುಗದೆ ಇನ್ನೇನು ಎಂದಳು. ಎಂದರೆ ಆ ಹೆಂಗಸಿಗೆ ಅಂತಹ ಶ್ರದ್ಧೆ ಇತ್ತು. ಅವಳು ಸರಾಗವಾಗಿ ನಡೆದಳು. ಪಂಡಿತನಿಗೆ ವಿದ್ಯೆ ಇತ್ತು ಬುದ್ಧಿ ಇತ್ತು. ಆದರೆ ತನ್ನಲ್ಲಿರುವ ಸರಕಿನಲ್ಲಿಯೇ ಅವನಿಗೆ ವಿಶ್ವಾಸವಿಲ್ಲ. ಭಗವಂತನ ಮಾತಿನಲ್ಲಿ ವಿಶ್ವಾಸ, ಶ್ರದ್ಧೆಯೊಂದಿದ್ದರೆ ಸಾಕು, ಯಾವ ಶಾಸ್ತ್ರಗಳನ್ನೂ ವೇದವೇದಾಂತಗಳನ್ನೂ ಅವನು ಓದಿಲ್ಲದೆ ಇರಬಹುದು. ಆದರೂ ದೇವರು ಅವನನ್ನು ಮೇಲೆತ್ತುವನು. ಎಲ್ಲವನ್ನೂ ಓದಿರಬಹುದು, ತಿಳಿದುಕೊಂಡಿರಬಹುದು. ಅದರಲ್ಲಿ ನಂಬಿಕೆ ಇಲ್ಲದೆ ಇದ್ದರೆ ಅದರಿಂದ ಏನೂ ಪ್ರಯೋಜನವಿಲ್ಲ. ಎಷ್ಟನ್ನು ನಂಬಿರುವೆವೊ ಅದು ಮಾತ್ರ ನಮ್ಮದು. ಏನೇನನ್ನು ತಿಳಿದುಕೊಂಡಿರು ವೆವೋ ಇವೆಲ್ಲಾ ಪರೀಕ್ಷಾ ಸಮಯದಲ್ಲಿ ನಮ್ಮ ಕೈಬಿಡುವುವು. ಈ ಶ್ರದ್ಧೆ ಎಷ್ಟು ಅಮೋಘ ವಾದುದು, ಉಜ್ವಲವಾದುದು ಎಂಬುದನ್ನು ಶ್ರೀಕೃಷ್ಣ ಸಾರುತ್ತಾನೆ.

\begin{verse}
ಯೇ ತ್ವೇತದಭ್ಯಸೂಯಂತೋ ನಾನುತಿಷ್ಠಂತಿ ಮೇ ಮತಮ್ ।\\ಸರ್ವಜ್ಞಾನವಿಮೂಢಾಂಸ್ತಾನ್ ವಿದ್ಧಿ ನಷ್ಟಾನಚೇತಸಃ \versenum{॥ ೩೨ ॥}
\end{verse}

{\small ಆದರೆ ಯಾರು ಅಸೂಯೆಯುಳ್ಳವರೋ, ನನ್ನ ಈ ಮತವನ್ನು ಅನುಷ್ಠಾನ ಮಾಡುವುದಿಲ್ಲವೋ ಅವರನ್ನು ಸರ್ವಜ್ಞಾನ ವಿಮೂಢರೆಂದೂ, ಹಾಳಾದವರೆಂದೂ, ಬುದ್ಧಿ ಇಲ್ಲದವರೆಂದೂ ತಿಳಿದುಕೊ.}

ಯಾರ ಹೃದಯ ಅಸೂಯಾ ಬುದ್ಧಿಯಿಂದ ತುಂಬಿ ತುಳುಕಾಡುತ್ತಿದೆಯೋ, ಇನ್ನೊಬ್ಬರ ಉತ್ಕರ್ಷೆಯನ್ನು ನೋಡಿ ಸಹಿಸಲಾರನೋ ಅಂತಹವನು ಬೇಕಾದಷ್ಟು ತಿಳಿದುಕೊಂಡಿರಬಹುದು, ಆದರೆ ಅದರಿಂದ ಇವನಿಗೆ ಏನೂ ಪ್ರಯೋಜನವಾಗುವುದಿಲ್ಲ. ತಿಳಿದುಕೊಳ್ಳುವುದೊಂದೇ ಸಾಲದು ಉದ್ಧಾರವಾಗುವುದಕ್ಕೆ. ತಿಳಿದುಕೊಂಡಿರುವುದನ್ನು ಅನುಷ್ಠಾನಕ್ಕೆ ತರಬೇಕು. ಅದೊಂದೇ ನಮ್ಮನ್ನು ಉದ್ಧಾರ ಮಾಡುವುದು. ಬ್ಯಾಂಕಿನಲ್ಲಿರುವ ಅಕೌಂಟೆಂಟ್​ಗೆ ಯಾರು ಯಾರು ಎಷ್ಟು ದುಡ್ಡು ಇಟ್ಟಿರುವರು ಎಂಬುದು ಗೊತ್ತಿದೆ. ಅವರು ಬಂದು ಕೇಳಿದಾಗ ಅವರ ಅಕೌಂಟಿನಲ್ಲಿರುವಂತೆ ಕಂತೆ ನೋಟುಗಳನ್ನು ಕೊಡುತ್ತಾನೆ. ಆದರೆ ಇವನಿಗೆ ಇದಾವುದಾದರೂ ಸೇರಿದೆಯೇನು? ತಿಂಗಳ ಕೊನೆ ಯಲ್ಲಿ ಬರುವ ಸಂಬಳವೊಂದೆ ಇವನದು, ತಿಂಗಳೆಲ್ಲ ಎಣಿಸುವ ಲಕ್ಷಾಂತರ ರೂಪಾಯಿ ಇವನದಲ್ಲ. ಒಬ್ಬ ಅಡಿಗೆಯವನಿಗೆ ಯಾವ ಯಾವ ಭಕ್ಷ್ಯಗಳನ್ನು ಹೇಗೆ ಮಾಡುವುದು ಎಂಬುದು ಗೊತ್ತಿದೆ. ಆದರೆ ಹೊಟ್ಟೆ ಹಸಿದಾಗ ಗೊತ್ತಿರುವುದನ್ನು ವ್ಯವಹಾರಕ್ಕೆ ಇಳಿಸಬೇಕು. ಆಗಲೆ ಅದರಿಂದ ನಮಗೆ ಪ್ರಯೋಜನವಾಗಬೇಕಾದರೆ.

ಭಗವಂತನನ್ನು ನಂಬಿದರೆ ಅವನು ನಮ್ಮನ್ನು ಉದ್ಧಾರ ಮಾಡುವನು ಎಂಬುದರಲ್ಲಿ ನಂಬಿಕೆ ಇಲ್ಲದೆ ಇದ್ದರೆ, ಅವನು ಬೇಕಾದಷ್ಟು ಶಾಸ್ತ್ರವನ್ನು ತಿಳಿದುಕೊಂಡಿರಬಹುದು. ಅದರಿಂದ ಆದ ಪ್ರಯೋಜನವೇನು? ಆ ಪಂಡಿತನೂ ಬೇಕಾದಷ್ಟು ತಿಳಿದುಕೊಂಡಿದ್ದ. ಆದರೇನು, ಅವನು ನದಿ ಯನ್ನು ದಾಟಿದನೆ! ಅವರು ಹಾಳಾದವರು, ಎಂದರೆ ಗೊತ್ತಿದ್ದರೂ ಅದರಿಂದ ಏನೂ ಪ್ರಯೋಜನ ಪಡೆಯದವರು. ಅವರು ಅಶ್ರದ್ಧೆಯ, ಅಸೂಯೆಯ ಹಳ್ಳದಲ್ಲಿ ಬಿದ್ದಿದ್ದಾರೆ, ಮೇಲೇಳಲಾರರು.

ಬುದ್ಧಿ ಇಲ್ಲದವರು ಎಂದು ಶ್ರೀಕೃಷ್ಣ ಅಂತಹವರನ್ನು ಕರೆಯುತ್ತಾನೆ. ಹಲವಾರು ಪುಸ್ತಕಗಳಿಂದ ಹಲವು ವಿಷಯಗಳನ್ನು ಸಂಗ್ರಹಿಸಿರುವನು. ಬೇಕಾದರೆ ಅವನೇ ಒಂದು ರೆಫರೆನ್ಸ್ ಪುಸ್ತಕ ಆಗಬಹುದು. ಆದರೆ ಅದರಿಂದ ಸಾಧಿಸಿದ್ದು ಏನು ಈ ಮನುಷ್ಯ? ಕತ್ತೆ ಒಂದು ಮೂಟೆ ಸಕ್ಕರೆ ಹೊರುವುದು. ಆದರೆ ಅದಕ್ಕೆ ತಿನ್ನುವುದಕ್ಕೆ ಸಿಕ್ಕುವುದು ತರಗೆಲೆ ಕಾಗದ ಕಸ ಅಲ್ಲದೆ ಬೇರಲ್ಲ. ಎಲ್ಲಾ ವಿದ್ಯೆಯ ಸಾರವೇ ಶ್ರದ್ಧೆ. ಆ ಶ್ರದ್ಧೆಯೇ ಇಲ್ಲದೆ ಉಳಿದುದನ್ನೆಲ್ಲ ಸಂಗ್ರಹಿಸಿ ಪ್ರಯೋಜನವಿಲ್ಲ. ಇಂತಹವರನ್ನು ಬೇಕಾದರೆ \eng{learned fools} ಎಂದು ಕರೆಯಬಹುದು. ಸಾಧಾರಣ ಮನುಷ್ಯ ಕಲಿಯದೆ ದಡ್ಡ, ಇವನು ಎಲ್ಲಾ ಕಲಿತರೂ ದಡ್ಡ. ಎಷ್ಟನ್ನು ಅನುಷ್ಠಾನಕ್ಕೆ ತಂದಿರುವೆವೋ ಅದು ಮಾತ್ರ ನಮ್ಮದು. ಉಳಿದುದೆಲ್ಲ ಅನ್ಯರದು. ಆಧ್ಯಾತ್ಮಿಕ ಜೀವನದಲ್ಲಿ ಅನುಷ್ಠಾನಕ್ಕೇ ಬೆಲೆ. ಸುಮ್ಮನೆ ತಿಳಿದುಕೊಂಡ ವಿಷಯಗಳ ಸರಕಿಗಲ್ಲ.

\begin{verse}
ಸದೃಶಂ ಚೇಷ್ಟತೇ ಸ್ವಸ್ಯಾಃ ಪ್ರಕೃತೇರ್ಜ್ಞಾನವಾನಪಿ ।\\ಪ್ರಕೃತಿಂ ಯಾಂತಿ ಭೂತಾನಿ ನಿಗ್ರಹಃ ಕಿಂ ಕರಿಷ್ಯತಿ \versenum{॥ ೩೩ ॥}
\end{verse}

{\small ತಿಳಿದವನು ಕೂಡ ತನ್ನ ಪ್ರಕೃತಿಗೆ ಸಮವಾಗಿ ವರ್ತಿಸುತ್ತಾನೆ. ಪ್ರಾಣಿಗಳು ತಮ್ಮ ಪ್ರಕೃತಿಯನ್ನು ಹೊಂದುತ್ತವೆ. ನಿಗ್ರಹ ಏನು ಮಾಡೀತು?}

ತಿಳಿದವನು ಎಂದರೆ ಒಬ್ಬ ಮನುಷ್ಯ ಬೌದ್ಧಿಕವಾಗಿ ಯಾವುದು ಸರಿ ಯಾವುದು ತಪ್ಪು ಎಂಬುದನ್ನು ಅರಿತಿರಬಹುದು. ಆದರೆ ಸುಮ್ಮನೆ ಅರಿತಿದ್ದರೆ ಮಾತ್ರ ಸಾಲದು ಈ ಜೀವನದಲ್ಲಿ. ಅದರಂತೆ ಕೆಲಸ ಮಾಡಬೇಕಾದರೆ ಬೇಕಾದ ಸಾಮಾನುಗಳನ್ನು ಕೂಡಿಹಾಕಿಕೊಂಡಿರಬೇಕು. ಆಗ ಮಾತ್ರ ನಾವು ತಿಳಿದಂತೆ ಮಾಡುತ್ತೇವೆ. ಅದಿಲ್ಲದೆ ಬರೀ ತಿಳಿವಳಿಕೆ ಮಾತ್ರ ಇದ್ದು, ಅದಕ್ಕೆ ವಿರೋಧವಾಗಿ ಕೆಲಸ ಮಾಡುವ ಸ್ವಭಾವವನ್ನು ಪಡೆದಿದ್ದರೆ, ಆ ಕೆಲಸವನ್ನು ಬಿಡುವುದಕ್ಕೆ ಆಗುವು ದಿಲ್ಲ. ಆ ಸಮಯ ಬಂದಾಗ ಆ ಕೆಲಸ ಮಾಡಿ ಹಾಕುತ್ತಾನೆ. ಏಕೆಂದರೆ ಸಂಸ್ಕಾರ ಅವನನ್ನು ಮಾಡುವಂತೆ ಪ್ರೇರೇಪಿಸುತ್ತ ಇರುವುದು. ಇದು ನೂರು ಜನ ಒಬ್ಬನನ್ನು ನೂಕುತ್ತಿರುವಂತೆ. ಅವನ ತಿಳಿವಳಿಕೆಯಾದರೊ ಬೇಡ ಎನ್ನುವುದು. ಒಬ್ಬ ನೂರು ಜನ ನೂಕುವುದನ್ನು ಎಷ್ಟು ಕಾಲ ತಡೆದಾನು? ಸ್ವಲ್ಪ ಕಾಲ ಮಾಡುವುದಿಲ್ಲ, ಮಾಡುವುದಿಲ್ಲ ಎಂದು ಹೇಳುತ್ತಿರುವನು. ಒತ್ತಡ ಜಾಸ್ತಿ ಆದಾಗ ಮಾಡಿಹಾಕುವನು. ನಾವು ಹಿಂದಿನಿಂದ ಸಂಗ್ರಹಿಸಿದ ಸ್ವಭಾವಕ್ಕೆ ನಮ್ಮ ಮೇಲೆ ಅದ್ಭುತವಾದ ಪ್ರಭಾವವಿದೆ.

ಪ್ರಾಣಿಗಳು ತಮ್ಮ ತಮ್ಮ ಪ್ರಕೃತಿಯಂತೆ ಎಂದರೆ ಸ್ವಭಾವದಂತೆ ಕೆಲಸ ಮಾಡುತ್ತವೆ. ಆ ಸ್ವಭಾವ ನಾವು ಹಿಂದಿನಿಂದ ಕೂಡಿಹಾಕಿಕೊಂಡು ಬಂದಿರುವುದು. ಅದನ್ನು ಅಷ್ಟು ಸುಲಭವಾಗಿ ಮಾರ್ಪಡಿಸು ವುದಕ್ಕೆ ಆಗುವುದಿಲ್ಲ. ಒಂದು ಸ್ಕ್ರೂ ಮೊಳೆಯನ್ನು ಮರದೊಳಗೆ ಎಂಟು ಹತ್ತು ಸುತ್ತು ಕಳುಹಿಸಿರು ತ್ತೇವೆ. ಸುಮ್ಮನೆ ಒಂದು ಸಲ ಅದನ್ನು ಕಿತ್ತು ಎಳೆದರೆ ಬಂದು ಬಿಡುವುದಿಲ್ಲ. ಅದನ್ನು ಒಂದು ಸ್ಕ್ರೂಡ್ರೈವರಿನಿಂದ ಅಷ್ಟು ಸಾರಿ ಹಿಂದಕ್ಕೆ ತಿರುಗಿಸಿದರೆ ತಾನೆ ಬರುವುದು.

ನಮ್ಮ ಸಂಸ್ಕಾರವೆಲ್ಲ ಒಂದು ರೀತಿ ಇದ್ದರೆ, ನಾವು ಆ ಕೆಲಸವನ್ನು ಮಾಡುವುದಿಲ್ಲ ಎಂದರೆ ಮಾತ್ರ ಸಾಲದು. ಕೆಲವು ವೇಳೆ ಆ ಕೆಲಸದ ಪರಿಣಾಮ ನಮಗೆ ಚೆನ್ನಾಗಿ ತಾಕಿ ಇನ್ನೊಂದು ಸಲ ನಾನು ಆ ಕೆಲಸ ಮಾಡುವುದಿಲ್ಲ ಎಂದು ಶಪಥ ಮಾಡುವೆವು. ಆದರೆ ಆ ಸ್ವಭಾವಕ್ಕೆ ವಿರೋಧವಾದ ಸಂಸ್ಕಾರಗಳನ್ನು ಅಷ್ಟೇ ಬಲವಾಗಿ ನಮ್ಮಲ್ಲಿ ಉತ್ಪತ್ತಿ ಮಾಡಿಕೊಂಡಿರುವುದಿಲ್ಲ. ಆದಕಾರಣ ಪರೀಕ್ಷಾ ಸಮಯ ಬಂದಾಗ ಬಲವಾದ ಹಳೆಯ ಸಂಸ್ಕಾರ ಗೆಲ್ಲುವುದು. ಅತಿ ದುರ್ಬಲವಾದ ಈಗತಾನೆ ಸಂಗ್ರಹಿಸುತ್ತಿರುವ ಅದಕ್ಕೆ ವಿರೋಧವಾದ ಸಂಸ್ಕಾರ ಸೋಲುವುದು.

ಹಾಗಾದರೆ ಯಾವಾಗಲೂ ಹಿಂದಿನ ಸಂಸ್ಕಾರಗಳಿಗೆ ಒಬ್ಬ ತುತ್ತಾಗಿ ಕೆಟ್ಟದ್ದನ್ನೇ ಮಾಡುತ್ತಿದ್ದರೆ, ಅವನು ಅದಕ್ಕೆ ವಿರೋಧವಾದ ಒಳ್ಳೆಯದನ್ನು ಮಾಡುವ ಶಕ್ತಿಯನ್ನು ಕೂಡಿಡುವುದು ಯಾವಾಗ? ಈಗಿನಿಂದಲೆ ಅದನ್ನು ಮಾಡಲು ಯತ್ನಿಸಬೇಕು. ನಮ್ಮಲ್ಲಿ ಬಲವಾದ ಹಿಂದಿನದನ್ನು ತಡೆದು ನಿಂತು ಅದಕ್ಕೆ ವಿರೋಧವಾದ ಕೆಲಸ ಮಾಡಬಲ್ಲಂತಹ ಸತ್ ಸಂಸ್ಕಾರಗಳನ್ನು ಕೂಡಿಹಾಕಿಕೊಳ್ಳುತ್ತಾ ಹೋಗಬೇಕು. ಅದು ಬಹಳ ಬೇಗ ಸಿದ್ಧಿಸುವುದಿಲ್ಲ ನಿಜ. ಆದರೆ ಅದಕ್ಕೆ ವಿರೋಧವಾಗಿ ನಾವು ಮಾಡಿದ ಆಲೋಚನೆ, ಭಾವನೆ, ಕಲ್ಪನೆ ಇವುಗಳೆಲ್ಲಾ ಕ್ರಮೇಣ ನಮ್ಮ ಸ್ವಭಾವವಾಗುತ್ತಾ ಬರುವುದು. ನಾವು ಮಾಡಿದ ಪ್ರಯತ್ನ ಯಾವುದೂ ನಿಷ್ಫಲವಾಗುವುದಿಲ್ಲ. ನಮಗೆ ಅಷ್ಟು ಶಕ್ತಿಯನ್ನು ನೀಡಿ ಹೋಗುವುದು. ಕ್ರಮೇಣ ಇವುಗಳೆಲ್ಲಾ ಒಟ್ಟಿಗೆ ಸೇರಿ ಒಂದು ಪ್ರಬಲವಾದ ಶಕ್ತಿಯಾಗುವುದು. ಅನಂತರವೇ ಹಿಂದಿನದನ್ನು ತಡೆಗಟ್ಟಲು ಸಾಧ್ಯವಾಗುವುದು.

ಇಲ್ಲಿ ಶ್ರೀಕೃಷ್ಣ ನಿಗ್ರಹ ಏನು ಮಾಡಬಲ್ಲುದು, ಹಿಂದಿನಿಂದ ಹೇಗೆ ಆಗುವುದೋ ಹಾಗೆಯೇ ಆಗುವುದು ಎಂದು ಹೇಳಿರುವುದರಿಂದ ನಾವು ಹಿಂದಿನದಕ್ಕೆ ಗುಲಾಮರಾಗಿಯೇ ಯಾವಾಗಲೂ ಇರಬೇಕೆಂದು ಅಲ್ಲ. ಬರೀ ತಡೆಯುತ್ತೇನೆ ಎಂಬ ಇಚ್ಛಾಶಕ್ತಿಯಿಂದಲೇ ಈ ಕೆಲಸ ಆಗುವುದಿಲ್ಲ. ಆ ಇಚ್ಛಾಶಕ್ತಿಗೆ ಬಲವನ್ನು ಕೊಡುವುದು ನಾವು ಈಗ ರೂಢಿಸಿಕೊಳ್ಳುವ ಹೊಸ ಸಂಸ್ಕಾರ. ಹೊಸ ಸಂಸ್ಕಾರ ಹಳೆಯದರಷ್ಟೇ ಬಲವಾದಾಗ ಮಾತ್ರ ಅದನ್ನು ಹೊಡೆದುಹಾಕಬಹುದು.

\begin{verse}
ಇಂದ್ರಿಯಸ್ಯೇಂದ್ರಿಯಸ್ಯಾರ್ಥೇ ರಾಗದ್ವೇಷೌ ವ್ಯವಸ್ಥಿತೌ ।\\ತಯೋರ್ನ ವಶಮಾಗಚ್ಛೇತ್ ತೌ ಹ್ಯಸ್ಯ ಪರಿಪಂಥಿನೌ \versenum{॥ ೩೪ ॥}
\end{verse}

{\small ಎಲ್ಲಾ ಇಂದ್ರಿಯಗಳಿಗೆ ವಿಷಯ ವಸ್ತುಗಳ ಮೇಲೆ ರಾಗದ್ವೇಷಗಳು ಇವೆ. ಇವೆರಡಕ್ಕೂ ಅವನು ವಶನಾಗ ಬಾರದು. ಏಕೆಂದರೆ ಇವು ಅವನ ಶತ್ರುಗಳು.}

ನಾವು ಉತ್ತಮ ಸಂಸ್ಕಾರವನ್ನು ಕ್ರಮೇಣ ಹೇಗೆ ಉತ್ಪತ್ತಿ ಮಾಡಬೇಕು ಎನ್ನುವುದನ್ನು ಇಲ್ಲಿ ಹೇಳುತ್ತಾನೆ. ನಮ್ಮಲ್ಲಿ ಇಂದ್ರಿಯಗಳಿವೆ. ಅವು ಎರಡು ಭಾಗವಾಗಿವೆ. ಒಂದು ಜ್ಞಾನೇಂದ್ರಿಯ ಮತ್ತೊಂದು ಕರ್ಮೇಂದ್ರಿಯ. ಜ್ಞಾನೇಂದ್ರಿಯದ ಮೂಲಕ ಹೊರಗಿನ ಜಗತ್ತನ್ನು ತಿಳಿದುಕೊಳ್ಳು ತ್ತೇವೆ. ಅದೇ ಕಣ್ಣು, ಕಿವಿ, ಮೂಗು, ನಾಲಿಗೆ, ಚರ್ಮ ಇವುಗಳು. ಇವುಗಳಿಂದ ರೂಪ, ರಸ, ಶಬ್ದ, ಸ್ಪರ್ಶ, ಗಂಧ ಇವುಗಳನ್ನು ತಿಳಿದುಕೊಳ್ಳುತ್ತೇವೆ. ಅನಂತರ ನಾವು ಕೆಲಸ ಮಾಡುವ ಐದು ಇಂದ್ರಿಯಗಳಿವೆ. ಬಾಯಿ, ಕೈ, ಕಾಲು, ಜನನೇಂದ್ರಿಯಗಳು. ಇವುಗಳ ಮೂಲಕ ಮಾತು ಕೆಲಸ ಮಾಡುವುದು, ನಡೆಯುವುದು, ವಿಸರ್ಜಿಸುವುದು ಈ ಕೆಲಸಗಳನ್ನು ಮಾಡುತ್ತೇವೆ. ಇವುಗಳ ಹಿಂದೆ ಚಿತ್ತ, ಮನಸ್ಸು, ಬುದ್ಧಿ ಎಂಬ ಸೂಕ್ಷ್ಮವಸ್ತುಗಳಿವೆ. ಇವುಗಳಿಗೆಲ್ಲಾ ಇವುಗಳಿಗೆ ಪ್ರಿಯವಾಗಿರುವ ಬಾಹ್ಯವಸ್ತುವಿನ ಮೇಲೆ ಪ್ರೀತಿ. ಅದರ ಕಡೆ ಧಾವಿಸುತ್ತ ಇರುತ್ತವೆ. ಅಪ್ರಿಯವಾಗಿರುವ ವಸ್ತುವನ್ನು ಕಂಡರೆ ಆಗುವುದಿಲ್ಲ. ಅದರಿಂದ ಹಿಂತಿರುಗಿ ಬರುತ್ತವೆ. ಅದನ್ನು ಕಂಡರೆ ದ್ವೇಷಿಸುತ್ತವೆ. ಒಂದು ರಾಗ, ಮತ್ತೊಂದು ದ್ವೇಷ,ಇವೆರಡೂ ನಮ್ಮನ್ನು ಆ ವಸ್ತುವಿಗೆ ಕಟ್ಟಿಹಾಕುವುದು. ನಾವು ಈಗ ಲೇನೋ ದ್ವೇಷ ನಮಗೆ ಬೇಕಾಗಿಲ್ಲವಲ್ಲ, ಅದನ್ನು ಕಂಡರೆ ಆಗುವುದಿಲ್ಲವಲ್ಲ, ಅದು ಹೇಗೆ ನಮ್ಮನ್ನು ಕಟ್ಟಿಹಾಕುವುದು ಎಂದು ಭಾವಿಸಬಹುದು. ಯಾವಾಗ ನಾವು ಒಂದು ವಸ್ತುವನ್ನು ದ್ವೇಷಿಸು ತ್ತೇವೆಯೋ ಅದು ನಮ್ಮ ಮನಸ್ಸಿನ ಮುಂದೆ ಇರುತ್ತದೆ. ಅದನ್ನೇ ಕುರಿತು ಚಿಂತಿಸುತ್ತಿರುತ್ತೇವೆ. ಆದರೆ ಪ್ರಿಯವಾದ ಭಾವನೆಯಿಂದಲ್ಲ, ಅಪ್ರಿಯವಾದ ಭಾವನೆಯಿಂದ. ಆದರೂ ಅದು ನಮ್ಮ ಮನಸ್ಸಿನ ಮೇಲೆ ದೊಡ್ಡ ಪ್ರಭಾವವನ್ನು ಬಿಡುವುದು. ಒಮ್ಮೆ ಶ್ರೀರಾಮಕೃಷ್ಣರ ಶಿಷ್ಯನಾದ ಹರಿ ಎಂಬುವನು ಶ್ರೀರಾಮಕೃಷ್ಣರ ಹತ್ತಿರ ಮಾತನಾಡುತ್ತಿದ್ದಾಗ, ನನಗೆ ಹೆಂಗಸರನ್ನು ಕಂಡರೆ ಆಗುವು ದಿಲ್ಲ. ಒಂದು ಹೆಣ್ಣುಮಗು ನನ್ನ ಬಳಿಗೆ ತೆವಳುತ್ತಾ ಬಂದರೂ ಅದನ್ನು ದ್ವೇಷಿಸುತ್ತೇನೆ ಎಂದನು. ಶ್ರೀರಾಮಕೃಷ್ಣರು ಆಗ ನಕ್ಕರು. ಜೋಕೆ, ದ್ವೇಷಿಸಬೇಡ, ಪ್ರೀತಿಸುವುದಕ್ಕಿಂತ ಬೇಗ ದ್ವೇಷಿಸುವುದರ ಬಲೆಗೆ ಬೀಳುವೆ ಎಂದರು. ಹಾಗಾದರೆ ನಾವೇನು ಮಾಡಬೇಕು? ಎಲ್ಲದರ ಹಿಂದೆಯೂ ಇರುವ ದೇವತ್ವಕ್ಕೆ ಮಣಿಯಬೇಕು. ಮುಂದಿರುವ ಉಪಾಧಿಗೆ ಉದಾಸೀನರಾಗಿರಬೇಕು. ವಿಷಯ ವಸ್ತುವಿನ ಕಡೆ ಹೋಗುವುದನ್ನು ಬಿಡಬೇಕು. ದ್ವೇಷದಿಂದ ಅದರ ಬಳಿಯಿಂದ ಓಡಿಹೋಗುವುದನ್ನು ಬಿಡಬೇಕು. ನಿಜವಾಗಿ ನಮ್ಮನ್ನು ಕಟ್ಟಿಹಾಕುವುದೇ ದ್ವೇಷ ಮತ್ತು ಪ್ರೀತಿ. ಇವೆರಡೂ ಬಂಧನಗಳೇ. ಪ್ರೀತಿ ಪ್ರಿಯವಾದ ಬಂಧನ. ದ್ವೇಷ ಅಪ್ರಿಯವಾದ ಬಂಧನ. ನಾವು ಪ್ರೀತಿಸಿದರೆ ಪ್ರೀತಿಸುವ ವಸ್ತುವಿಗೆ ದಾಸರಾಗುತ್ತೇವೆ. ದ್ವೇಷಿಸಿದರೆ ಆ ವಸ್ತು ಪಿಶಾಚಿಯಂತೆ ಬೇಡವೆಂದರೂ ನಮ್ಮನ್ನು ಮೆಟ್ಟಿಕೊಳ್ಳು ವುದು. ಇವೆರಡೂ ನಮ್ಮನ್ನು ಕಾಡುವುವು. ಜ್ಞಾನಿಯಾದವನು, ಇವುಗಳಿಂದ ನಿಸ್ಸಂಗನಾಗಿರುವನು. ಇವುಗಳಿಂದ ನಾವು ಓಡಿಹೋಗುವುದಕ್ಕೆ ಆಗುವುದಿಲ್ಲ. ಎಲ್ಲಿಗೆ ಹೋದರೂ ನಮ್ಮ ನೆರಳಿನಂತೆ ಅದು ಬೆನ್ನು ಹಿಡಿಯುವುದು. ಹಾಗಾದರೆ ಪಾರಾಗುವುದು ಹೇಗೆ? ಅದನ್ನು ಗಮನಿಸದೆ ಇರುವುದ ರಿಂದ, ಮನಸ್ಸನ್ನು ಅತ್ತ ಹರಿಯಬಿಡದೆ ಇರುವುದರಿಂದ ಅದರ ಮೇಲೆ ಅನಾದರವನ್ನು ಉಂಟು ಮಾಡಿಕೊಳ್ಳುವೆವು. ಇಷ್ಟು ಕಾಲ ನಾವು ಅದನ್ನು ಮುದ್ದಿಸುತ್ತಿದ್ದೆವು. ಅದಕ್ಕೇ ಅವು ನಮ್ಮ ಹತ್ತಿರ ಬಂದು ತಮ್ಮನ್ನು ಪ್ರೀತಿಸಲಿ ಎಂದು ನಿಲ್ಲುವುವು. ಯಾವಾಗ ನಾವು ಅದರ ಕಡೆ ಗಮನ ಕೊಡುವುದಿಲ್ಲವೋ, ಕ್ರಮೇಣ ಅವು ಇದ್ದರೂ ನಮ್ಮ ಪಾಲಿಗೆ ಇಲ್ಲದಂತೆ ಆಗುವುದು.

\begin{verse}
ಶ್ರೇಯಾನ್ ಸ್ವಧರ್ಮೋ ವಿಗುಣಃ ಪರಧರ್ಮಾತ್ ಸ್ವನುಷ್ಠಿತಾತ್ ।\\ಸ್ವಧರ್ಮೇ ನಿಧನಂ ಶ್ರೇಯಃ ಪರಧರ್ಮೋ ಭಯಾವಹಃ \versenum{॥ ೩೫ ॥}
\end{verse}

{\small ಚೆನ್ನಾಗಿ ಮಾಡದ ಸ್ವಧರ್ಮ, ಚೆನ್ನಾಗಿ ಮಾಡಿದ ಪರಧರ್ಮಕ್ಕಿಂತ ಮೇಲು. ಸ್ವಧರ್ಮದಲ್ಲಿ ಮರಣ ಲೇಸು. ಪರಧರ್ಮ ಭಯದಿಂದ ಕೂಡಿದೆ.}

ಇಲ್ಲಿ ಮೊದಲು ಯಾವುದು ಸ್ವಧರ್ಮ ಎಂಬುದನ್ನು ಕಂಡಿಹಿಡಿಯಬೇಕಾಗಿದೆ. ಪ್ರತಿಯೊಬ್ಬನೂ ಒಂದೊಂದು ಸ್ವಭಾವದಿಂದ ಹುಟ್ಟುತ್ತಾನೆ. ಜೀವನದಲ್ಲಿ ಅದಕ್ಕೆ ತಕ್ಕ ತರಬೇತನ್ನು ತೆಗೆದುಕೊಂಡು ಆಯಾ ಕೆಲಸಗಳನ್ನು ಮಾಡುತ್ತಾನೆ. ಅಂದರೆ ಯಾವುದರ ಮೇಲೆ ಅಭಿರುಚಿ ಇರುವುದು, ಯಾವುದ ಕ್ಕಾಗಿ ಇವನು ತರಬೇತನ್ನು ತೆಗೆದುಕೊಂಡಿರುವನೋ ಮತ್ತು ಇದುವರೆಗೆ ಯಾವ ಕೆಲಸವನ್ನು ಮಾಡುತ್ತಿರುವನೋ ಅದೆಲ್ಲ ಸೇರಿಕೊಂಡು ಅವನ ಧರ್ಮ ಆಗುವುದು. ಅರ್ಜುನ ಕ್ಷತ್ರಿಯನ ಕುಲದಲ್ಲಿ ಹುಟ್ಟಿದ್ದಾನೆ. ಕ್ಷತ್ರಿಯನ ತರಬೇತನ್ನು ತೆಗೆದುಕೊಂಡಿದ್ದಾನೆ, ಇದುವರೆಗೆ ಅವನು ಕ್ಷತ್ರಿಯನ ಪಾತ್ರವನ್ನು ನಿರ್ವಹಿಸುತ್ತಿರುವನು.

ಕೆಲವು ವೇಳೆ ಹೊಸ ಹೊಸದರಲ್ಲಿ ತನ್ನ ಪಾಲಿಗೆ ಬಂದ ಧರ್ಮವನ್ನು ಎಂದರೆ ಕರ್ತವ್ಯದ ಹೊರೆ ಹೊಣೆಗಳನ್ನು ಚೆನ್ನಾಗಿ ನಿರ್ವಹಿಸುವುದಕ್ಕೆ ಆಗದೆ ಇರಬಹುದು. ಆದರೆ ಅದರಲ್ಲಿಯೇ ಮುಂದು ವರಿದರೆ ಕ್ರಮೇಣ ಅವನು ಆ ಕೆಲಸದಲ್ಲಿ ಕುಶಲಿ ಆಗುವನು. ಅವನು ಮುಂಚೆ ಅಷ್ಟು ಚೆನ್ನಾಗಿ ಮಾಡದೆ ಇದ್ದರೂ ಕ್ರಮೇಣ ಚೆನ್ನಾಗಿ ಮಾಡುವುದರಿಂದ ವ್ಯಕ್ತಿಗೆ ಮತ್ತು ಸಮಾಜಕ್ಕೆ ಯಾವ ನಷ್ಟವೂ ಆಗುವುದಿಲ್ಲ. ಒಬ್ಬ ಶಾಲೆಯ ಉಪಾಧ್ಯಾಯನಾಗಿ ಬಂದಿದ್ದಾನೆ, ಕಲಿತಿದ್ದಾನೆ, ತರಬೇತು ತೆಗೆದುಕೊಂಡು ಹೊಸದಾಗಿ ಉಪಾಧ್ಯಾಯ ವೃತ್ತಿಯನ್ನು ಪ್ರಾರಂಭಿಸಿದ್ದಾನೆ. ಸೇರಿದ ಹೊಸದರಲ್ಲೇ ಅವನು ಆದರ್ಶ ಉಪಾಧ್ಯಾಯನಾಗದೆ ಹೋಗಬಹುದು. ಆದರೆ ಕ್ರಮೇಣ ಆ ವೃತ್ತಿಯಲ್ಲಿ ಮುಂದುವರಿದರೆ ಅವನೊಬ್ಬ ಮಾದರಿಯ ಉಪಾಧ್ಯಾಯನಾಗುತ್ತಾನೆ. ಇದರಿಂದ ಸಮಾಜಕ್ಕೆ ಒಬ್ಬ ಉಪಾಧ್ಯಾಯ ಸಿಕ್ಕಿದ. ಅನಂತರ ಉಪಾಧ್ಯಾಯನ ವೃತ್ತಿಯನ್ನು ತೆಗೆದುಕೊಂಡ ವ್ಯಕ್ತಿ ಕೂಡ ಆ ಮಾರ್ಗದಲ್ಲಿ ಮುಂದುವರಿದು ಮೇಲೇರುವುದಕ್ಕೆ ಸಾಧ್ಯವಾಯಿತು. ಇಲ್ಲಿ ಅವನು ತನ್ನ ಸ್ವಭಾವಕ್ಕೆ ಅನುಗುಣವಾಗಿ ಹೋಗುತ್ತಿರುವನು.

ಅದೇ ಉಪಾಧ್ಯಾಯ ಯಾವುದೋ ಕಾರಣದಿಂದ ಚೆನ್ನಾಗಿ ಪಾಠ ಹೇಳುವುದಕ್ಕೆ ಬರದಾಗ, ಹುಡುಗರು ತಮಾಷೆ ಮಾಡಿದರೆ, ಈ ಕೆಲಸ ನನಗೆ ಒಗ್ಗುವುದಿಲ್ಲ ಎಂದುಬಿಟ್ಟರೆ, ಅವನು ತನ್ನ ಸ್ವಭಾವಕ್ಕೆ ವಿರೋಧವಾಗಿ ಹೋಗುತ್ತಾನೆ. ಮುಂದೆ ಒಬ್ಬ ಆದರ್ಶ ಉಪಾಧ್ಯಾಯನಾಗುವ ಅವಕಾಶ ವನ್ನು ಕಳೆದುಕೊಳ್ಳುತ್ತಾನೆ. ಅದೇ ಶಾಲೆಯಲ್ಲಿ ಬೆಲ್ಲು ಹೊಡೆಯುವ ಆಳಿನ ಕೆಲಸವನ್ನು ನಾನು ಚೆನ್ನಾಗಿ ಮಾಡುತ್ತೇನೆ ಎಂದು ಉಪಾಧ್ಯಾಯನ ವೃತ್ತಿಯನ್ನು ಬಿಡಲಾಗುವುದೆ? ಪ್ರತಿಯೊಬ್ಬನೂ ತನ್ನ ತನ್ನ ರೀತಿಯಲ್ಲಿ ಕರ್ಮವನ್ನು ಕ್ಷಯಮಾಡಿಕೊಳ್ಳಬೇಕಾಗಿದೆ. ಒಂದು ಸಲ ಅದನ್ನು ಚೆನ್ನಾಗಿ ವಿಮರ್ಶಿಸಿ ತೆಗೆದುಕೊಂಡ ಮೇಲೆ ದಾರಿಯಲ್ಲಿ ಯಾವ ಅಡಚಣೆಗಳು ಬಂದರೂ ಅದನ್ನು ಬಿಡದೆ ಮುಂದುವರಿಸಬೇಕು. ಇದೇ ಉದ್ಧಾರದ ಹಾದಿ.

ನಮ್ಮ ಸ್ವಭಾವಕ್ಕೆ ತಕ್ಕ ಕರ್ಮವನ್ನು ಮಾಡುತ್ತ ನಾವು ಗುರಿಯನ್ನು ಮುಟ್ಟದೆ ದಾರಿಯಲ್ಲೇ ಸಾಯಬಹುದು. ಆದರೆ ಅದರಿಂದ ನಮಗೆ ನಷ್ಟವಿಲ್ಲ. ಒಂದೇ ಜನ್ಮದಲ್ಲಿ ನಾವು ಗುರಿಯೆಡೆಗೆ ಹೋಗುವುದಕ್ಕಾಗುವುದಿಲ್ಲ. ಅದಕ್ಕಾಗಿ ದೇವರು ಹಲವು ಜನ್ಮಗಳನ್ನು ನಮಗೆ ಕೊಟ್ಟಿರುವನು. ಈ ಜನ್ಮದಲ್ಲಿ ಎಲ್ಲಿ ಬಿಟ್ಟಿರುವೆವೋ ಮುಂದಿನ ಜನ್ಮದಲ್ಲಿ ಅಲ್ಲಿಂದ ಮುಂದಕ್ಕೆ ಹೋಗಬಹುದು. ಈ ಜನ್ಮದಲ್ಲಿ ಪಡೆದದ್ದಾವುದೂ ನಷ್ಟವಾಗುವುದಿಲ್ಲ.

ಪರಧರ್ಮ ಭಯಂಕರ ಎನ್ನುವನು. ಅನೇಕವೇಳೆ ನಾವು ಇನ್ನೊಬ್ಬರ ಕೆಲಸಕ್ಕೆ ಕೈಹಾಕುವುದಕ್ಕೆ ಕಾರಣ, ಅದರಲ್ಲಿ ಸುಲಭವಾಗಿ ಮುಂದುವರೆಯಬಹುದು ಎಂದು ಭಾವಿಸುವುದರಿಂದ. ಆದರೆ ಎಲ್ಲಿಯೂ ಮನುಷ್ಯ ಸುಲಭವಾಗಿ ಮುಂದುವರಿಯುವುದಕ್ಕಾಗುವುದಿಲ್ಲ. ಯಾವಾಗ ಅಲ್ಲಿ ಅಡ್ಡಿ ಆತಂಕಗಳು ಬರುವುವೋ ಅದನ್ನು ಎದುರಿಸುವಷ್ಟು ಸಾಹಸ ಅವನಲ್ಲಿ ಇರುವುದಿಲ್ಲ. ಏಕೆಂದರೆ ಅವನು ಇದಕ್ಕೆ ಹೊಸಬ. ಅಲ್ಲಿ ಕಷ್ಟ ನಷ್ಟಗಳನ್ನು ಎದುರಿಸುವ ತರಬೇತನ್ನು ತೆಗೆದುಕೊಂಡಿಲ್ಲ. ಎಲ್ಲಾ ಸರಿಯಾಗಿ ಅವನು ಇಚ್ಛಿಸಿದಂತೆ ಆಗುತ್ತಿದ್ದರೆ ಪರವಾಗಿಲ್ಲ. ಆದರೆ ಜೀವನ ವಿಚಿತ್ರ ವಾದುದು. ನಾವು ಇಚ್ಛಿಸಿದಂತೆಯೇ ಎಲ್ಲಾ ಸಾಗುವುದಿಲ್ಲ. ನಾವು ಕಟ್ಟಿದ ಸವಿಗನಸುಗಳು ಪುಡಿಪುಡಿಯಾಗಿ ಹೋಗುವ ಕಾಲ ಬರುವುದು. ಆಗ ಅನ್ನಿಸುವುದು, ಹಿಂದಿದ್ದುದನ್ನೇ ಇಟ್ಟುಕೊಂಡಿ ದ್ದರೆ ಮೇಲಾಗುತ್ತಿತ್ತು ಎಂದು. ಆದಕಾರಣವೇ ಶ್ರೀಕೃಷ್ಣ ಹೇಳುವುದು, ಜೀವನದ ಯಾವ ಕಾರ್ಯಕ್ಷೇತ್ರದಲ್ಲಿ ಇದುವರೆಗೆ ನಿಮ್ಮ ಸ್ವಭಾವಕ್ಕೆ ಅನುಗುಣವಾದ ಯಾವ ಕೆಲಸವನ್ನು ಮಾಡು ತ್ತಿದ್ದಿರೋ ಅದನ್ನೇ ಮಾಡಿಕೊಂಡು ಹೋಗಿ; ಬೇಕಾದರೆ ಅದನ್ನು ಉತ್ತಮ ದೃಷ್ಟಿಯಿಂದ ಮಾಡುವುದನ್ನು ಕಲಿಯಿರಿ ಎನ್ನುವನು.

ಅರ್ಜುನ ಇಲ್ಲಿ ಕ್ಷತ್ರಿಯನಾಗಿದ್ದಾನೆ. ಈಗ ಅವನು ಯುದ್ಧ ಮಾಡುವುದಿಲ್ಲ, ಆ ದಾರಿ ನನಗೆ ಬೇಕಾಗಿಲ್ಲ, ರಾಜ್ಯದಿಂದೇನು, ಭೋಗದಿಂದೇನು? ಎಂಬ ವಿರಕ್ತನ ಹಾದಿಯನ್ನು ಹಿಡಿಯಬೇಕೆಂದು ಬಯಸುವನು. ತತ್ಕಾಲಕ್ಕೆ ಯುದ್ಧ ಕಹಿಯಾಗಿ ಕಾಣಿಸುತ್ತದೆ. ಏಕೆಂದರೆ ಇಲ್ಲಿ ಕೊಲೆ ಹಿಂಸೆ ಇವುಗಳೆಲ್ಲ ಇವೆ. ವಿರಕ್ತನ ಗುರಿ ಪ್ರಿಯವಾಗಿ ಕಾಣುತ್ತಿದೆ. ಏಕೆಂದರೆ ಅಲ್ಲಿ ಯಾವ ಯುದ್ಧವೂ ಇಲ್ಲ, ಯಾವ ಕೊಲೆಯೂ ಇಲ್ಲ ಎಂದು ಭಾವಿಸುವನು. ಅರ್ಜುನ ವಿರಕ್ತನ ದಾರಿಯಲ್ಲಿ ಹೋಗಲು ತರಬೇತನ್ನು ತೆಗೆದುಕೊಂಡಿಲ್ಲ. ಮುಂದೆ ಆ ಮಾರ್ಗದಲ್ಲಿ ಏನೇನು ಅಡಚಣೆಗಳು ಬರುತ್ತವೆ ಎಂಬುದು ಗೊತ್ತಿಲ್ಲ. ಅಲ್ಲಿ ಆಂತರಿಕ ಜಗತ್ತಿನ ಹೋರಾಟ ಪ್ರಾರಂಭವಾದರೆ, ನಮ್ಮ ಮನಸ್ಸಿನ ಲ್ಲಿರುವ ಪ್ರಲೋಭನೆಗಳೊಂದಿಗೆ ಯುದ್ಧ ಪ್ರಾರಂಭವಾದಾಗ ಅರ್ಜುನ ಭಾವಿಸುತ್ತಾನೆ, ಕುರುಕ್ಷೇತ್ರ ದಲ್ಲಿ ಗುರುಹಿರಿಯರನ್ನು ಶಸ್ತ್ರಪ್ರಯೋಗದಿಂದ ಕೆಡಹುವುದು ಈ ಮಾನಸಿಕ ಹೋರಾಟಕ್ಕಿಂತ ಮೇಲು ಎಂದು. ಅರ್ಜುನನಿಗೆ ವಿರಕ್ತಿ ಜೀವನದ ಕಾವು ತಾಕಿಲ್ಲ. ಅದಕ್ಕೆ ಭಾವಿಸುತ್ತಿದ್ದಾನೆ ಅಲ್ಲಿ ಎಲ್ಲಾ ಚೆನ್ನಾಗಿದೆ ಎಂದು. ಆದಕಾರಣವೇ ನಮ್ಮ ತಾತ್ಕಾಲಿಕ ಉದ್ವೇಗವನ್ನು ಅನುಸರಿಸಿ ನಾವು ಇದುವರೆಗೆ ನಡೆದ ದಾರಿಯನ್ನು ಬಿಟ್ಟು ಅನ್ಯರ ದಾರಿಗೆ ಹೋಗಬಾರದೆಂದು ಹೇಳುತ್ತಾನೆ. ಆಗಲೆ ಅರ್ಜುನ ಕೇಳುತ್ತಾನೆ:

\begin{verse}
ಅಥ ಕೇನ ಪ್ರಯುಕ್ತೋಽಯಂ ಪಾಪಂ ಚರತಿ ಪೂರುಷಃ ।\\ಅನಿಚ್ಛನ್ನಪಿ ವಾರ್ಷ್ಣೇಯ ಬಲಾದಿವ ನಿಯೋಜಿತಃ \versenum{॥ ೩೬ ॥}
\end{verse}

{\small ಕೃಷ್ಣ, ಹಾಗಾದರೆ ಈ ಮನುಷ್ಯ ಇಚ್ಛಿಸದೇ ಇದ್ದರೂ ಏತರಿಂದ ಪ್ರೇರಿತನಾಗಿ ಬಲಾತ್ಕಾರದಿಂದ ಕಟ್ಟು ಮಾಡಲ್ಪಟ್ಟಂತೆ ಪಾಪವನ್ನು ಮಾಡುತ್ತಿರುವನು?}

ಅರ್ಜುನನು ಇಲ್ಲಿ ಈ ಪ್ರಶ್ನೆಯನ್ನು ನಮ್ಮ ಪರವಾಗಿ ಕೇಳುತ್ತಿರುವಂತೆ ಇದೆ. ಅನೇಕ ವೇಳೆ ನಮಗೆ ಒಂದು ಕೆಲಸವನ್ನು ಮಾಡುವುದಕ್ಕೆ ಇಚ್ಛೆ ಇಲ್ಲ. ಆದರೆ ಆ ಸಮಯ ಬಂತು ಎಂದರೆ ನಾವು ಇಚ್ಛಿಸದೆ ಇದ್ದರೂ, ಯಾರೋ ಬಲಾತ್ಕಾರವಾಗಿ ನಮ್ಮ ಕೈಯಿಂದ ಕೆಲಸವನ್ನು ಮಾಡುವಂತೆ ಪ್ರೇರೇಪಿಸುವಂತೆ ಇದೆ. ಅದನ್ನು ನಿಗ್ರಹಿಸುವುದಕ್ಕೆ ಆಗುವುದಿಲ್ಲ. ನಾವು ಸ್ವತಂತ್ರರಲ್ಲ. ಮತ್ತಾರೊ ನಮ್ಮನ್ನು ಆಳುತ್ತಿರುವಂತೆ ಇದೆ. ಸರ್ಕಸ್ಸಿನಲ್ಲಿ ಹುಲಿ, ಸಿಂಹ, ಆನೆಗಳು, ಯಜಮಾನ ಹೇಳಿದ ಹಾಗೆ ಮಾಡುವಂತೆ, ಯಾರೋ ನಮ್ಮ ಮೇಲೆ ಅಧಿಕಾರವನ್ನು ಚಲಾಯಿಸಿ, ಇಚ್ಛೆ ಇಲ್ಲದೆ ಇದ್ದರೂ ನಮ್ಮ ಕೈಯಿಂದ ಆ ಕೆಲಸವನ್ನು ಮಾಡಿಸಿಬಿಡುವರು.

ನಮಗೆಲ್ಲ ಯಾವುದು ಸರಿ ಎಂಬುದು ಗೊತ್ತಿದೆ; ಆದರೆ ಅದನ್ನು ಮಾಡುವುದಿಲ್ಲ. ಯಾವುದು ತಪ್ಪು ಎಂಬುದು ಗೊತ್ತಿದೆ; ಅದನ್ನು ಬಿಡುವುದಿಲ್ಲ. ಗೊತ್ತಿಲ್ಲದೆ ತಪ್ಪನ್ನು ಮಾಡುವವರು ಬಹಳ ಅಪರೂಪ. ಅದು ಅವನಿಗೆ ಗೊತ್ತಿದೆ ತಪ್ಪು ಎಂದು. ಸುತ್ತಮುತ್ತ ಇರುವವರೆಲ್ಲ ತಪ್ಪು ಎಂದು ಹೇಳುತ್ತಾರೆ. ಆದರೂ ಆ ಕೆಲಸ ಮಾಡುವುದನ್ನು ಅವನು ಬಿಡುವುದಿಲ್ಲ. ಮಹಾಭಾರತದಲ್ಲಿ ಶ್ರೀಕೃಷ್ಣ ಸಂಧಿಗೆ ಒಪ್ಪಿಕೊ ಎಂದು ದುರ್ಯೋಧನನಿಗೆ ಹೇಳಿದಾಗ ಆ ದುರ್ಯೋಧನ ಹೀಗೆ ಹೇಳುತ್ತಾನೆ: “ನನಗೆ ಧರ್ಮ ಯಾವುದು ಗೊತ್ತಿದೆ, ಅದನ್ನು ಮಾಡಲಾರೆ. ಅಧರ್ಮ ಯಾವುದು ಗೊತ್ತಿದೆ, ಅದನ್ನು ಬಿಡಲಾರೆ.” ನಾವೆಲ್ಲ ಆ ಪರಿಸ್ಥಿತಿಯಲ್ಲಿಇರುವೆವು.

ಆದರೆ ನಮ್ಮ ಮನಸ್ಸನ್ನು ವಿಭಜನೆ ಮಾಡಿದರೆ ಒಂದು ಕೆಟ್ಟ ಕೆಲಸವನ್ನು ಮಾಡಬಾರದು ಎಂಬ ಇಚ್ಛೆಯೇನೋ ಇದೆ ಎಂಬುದು ಗೊತ್ತಾಗುವುದು. ಆದರೆ ಆ ಇಚ್ಛೆ ದುರ್ಬಲವಾದುದು. ಹಿಂದಿನಿಂದ ಆ ಕೆಲಸವನ್ನು ಮಾಡಿ ಸಂಪಾದಿಸಿಕೊಂಡಿರುವ ಸಂಸ್ಕಾರ ಬಹಳ ಬಲವಾದುದು. ಅದನ್ನು ಒಂದೇ ಸಲ ಕಿತ್ತುಹಾಕುವುದಕ್ಕೆ ಆಗುವುದಿಲ್ಲ. ಆ ಕೆಲಸವನ್ನು ಮಾಡುವುದಕ್ಕೆ ಅವಕಾಶ ಬರುವುದಕ್ಕೆ ಮುಂಚೆ ನಾನು ಈ ಸಲ ಅದನ್ನು ಮಾಡುವುದಿಲ್ಲ ಎಂದು ನನಗೆ ನಾನೇ ಹೇಳಿಕೊಳ್ಳುತ್ತಿರುತ್ತೇನೆ. ಆದರೆ ಸಮಯ ಬಂದಾಗ ಮೊದಮೊದಲು ಇಲ್ಲ, ಇಲ್ಲ ಎಂದು ಹೇಳುತ್ತಿದ್ದರೂ ಕ್ರಮೇಣ ಅದಕ್ಕೆ ಒಪ್ಪಿಗೆ ಕೊಟ್ಟು ಮಾಡಿಹಾಕುವೆವು. ಏಕೆಂದರೆ ಹಿಂದಿನ ಸ್ವಭಾವ ಅಷ್ಟು ಬಲವಾಗಿದೆ ನಮ್ಮ ಮೇಲೆ. ಆ ಸ್ವಭಾವ ಹೇಗೆ ಆಯಿತು? ಯಾರೊ ಹೊರಗಿನವರು ನಮ್ಮಲ್ಲಿ ಅದನ್ನು ಇಡಲಿಲ್ಲ. ಒಂದು ಕಾಲದಲ್ಲಿ ಹಿಂದೆ ನಾವೇ ಆ ಕೆಲಸವನ್ನು ಕಾಮಿಸಿ ಅದರಿಂದ ಪ್ರೇರೇಪಿತರಾಗಿ ಹಲವು ವೇಳೆ ಮಾಡಿಕೊಂಡು ಈಗ ಮನಸ್ಸಿಗೆ ಅದು ಬೇಡವಾದರೂ ಮಾಡಲೇಬೇಕಾದ ಸ್ಥಿತಿಗೆ ಬಂದಿರುವೆವು.

ಶ್ರೀಕೃಷ್ಣ ಅದಕ್ಕೆ ಹೇಳುತ್ತಾನೆ:

\begin{verse}
ಕಾಮ ಏಷ ಕ್ರೋಧ ಏಷ ರಜೋಗುಣಸಮುದ್ಭವಃ ।\\ಮಹಾಶನೋ ಮಹಾಪಾಪ್ಮಾ ವಿದ್ಧ್ಯೇನಮಿಹ ವೈರಿಣಮ್ \versenum{॥ ೩೭ ॥}
\end{verse}

{\small ಇದೇ ಕಾಮ, ಕ್ರೋಧ. ಇದು ರಜೋಗುಣದಿಂದ ಹುಟ್ಟಿರುವುದು. ಎಷ್ಟಾದರೂ ತಿನ್ನುವುದು. ಮಹಾ ಪಾಪಿ. ಇದೇ ಇಲ್ಲಿ ವೈರಿ ಎಂಬುದನ್ನು ತಿಳಿ.}

ಮನುಷ್ಯನಿಗೆ ವಿಧಿ ಇಲ್ಲದೇ ಮಾಡುವುದಕ್ಕೆ ಕಾರಣವೇ ಅವನಿಗೆ ಇರುವ ರಾಗದ್ವೇಷಗಳು. ಇದನ್ನು ಅವನು ಕೂಡಿಹಾಕಿಕೊಂಡಿರುವನು. ಇನ್ನು ಪರಿಣಾಮ ಎಂದರೆ ಅದು ಕಾರ್ಯದಲ್ಲಿ ಪರ್ಯವಸಾನವಾಗುವುದು ಆಗಲೇಬೇಕು. ಕಾರಣವನ್ನು ಆಚೆಗೆ ಓಡಿಸದೆ, ಪರಿಣಾಮವನ್ನು ಆಚೆಗೆ ಓಡಿಸುವುದಕ್ಕೆ ಆಗುವುದಿಲ್ಲ. ನಾವೆಲ್ಲ ರಾಗ ದ್ವೇಷಗಳ ಕೈಯಲ್ಲಿ ಬಂಧಿಗಳು. ನಾವು ಸ್ವತಂತ್ರರು, ಯಾರ ಮಾತನ್ನೂ ಕೇಳುವುದಿಲ್ಲ, ನಮ್ಮ ಇಚ್ಛೆ ಬಂದಂತೆ ಬಾಳುತ್ತೇವೆ ಎಂದು ಹೆಮ್ಮೆ ಕೊಚ್ಚಿಕೊಳ್ಳುವೆವು. ಆದರೆ ಹೊರಗಡೆ ಮಾತ್ರ ಯಾರ ಮಾತನ್ನೂ ಕೇಳುವುದಿಲ್ಲ. ನಮ್ಮ ಕಾಮಕ್ರೋಧಗಳು ಹೇಳಿದಂತೆ ಕೇಳುತ್ತೇವೆ. ಅವೇನು ಕೆಲಸ ಮಾಡು ಎಂದರೆ ಮಾಡುತ್ತೇವೆ. ನಿಜವಾಗಿ ನಮ್ಮನ್ನು ಆಳುವವರು ಯಾರು ಎಂದರೆ ನಮ್ಮಲ್ಲಿರುವ ಕಾಮ ಕ್ರೋಧಗಳು. ಇವೇ ಅಂಕುಶದಿಂದ ನಮ್ಮನ್ನು ತಿವಿಯುತ್ತಿವೆ. ಅವು ತಿವಿದತ್ತ ನಾವು ಹೋಗುತ್ತೇವೆ. ಆನೆ ನೋಡಲು ಅಷ್ಟು ದೊಡ್ಡದು, ಅಷ್ಟೊಂದು ಪರಾಕ್ರಮಶಾಲಿ, ಬೇಕಾದರೆ ಮನುಷ್ಯನನ್ನು ಸಿಗಿದುಹಾಕಬಲ್ಲುದು. ಆದರೆ ಮಾಹುತ ಅದರ ಮೇಲೆ ಕುಳಿತು ಅಂಕುಶದಿಂದ ತಿವಿದ ಎಂದರೆ ಅವನು ಹೇಳಿದುದನ್ನು ಮಾಡಿಹಾಕುವುದು. ಇಂತಹ ದೊಡ್ಡ ಜಂತುವನ್ನು ಆಳುವನು ಮಾಹುತ. ಹಾಗೆಯೇ ನಮ್ಮಲ್ಲಿ ಅಂತಹ ಮಾಹುತ ಯಾರಾಗಿದ್ದಾರೆ ಎಂದರೆ ಅವರೇ ನಮ್ಮಲ್ಲಿರುವ ರಾಗ ದ್ವೇಷಗಳು.

ನಮಗೆ ಯಾರನ್ನೋ ಕಂಡರೆ ಇಚ್ಛೆ. ಅದರಂತೆ ಮಾಡಬೇಕು. ಆ ವಸ್ತುವನ್ನು ಸಂಪಾದಿಸಬೇಕು. ಅದನ್ನು ನನ್ನ ಹತ್ತಿರ ಇಟ್ಟುಕೊಳ್ಳಬೇಕು. ಅದನ್ನು ನಾನು ಪದೇಪದೇ ಅನುಭವಿಸುತ್ತಿರಬೇಕು, ಎಂಬ ಇಚ್ಛೆ ಇದೆ. ಅದನ್ನು ತೃಪ್ತಿಪಡಿಸಿಕೊಳ್ಳುವುದಕ್ಕಾಗಿ ನಮ್ಮಲ್ಲಿರುವ ಬುದ್ಧಿವಂತಿಕೆ, ಜಾಣತನ, ಶಕ್ತಿ, ವಿದ್ಯೆ, ಪಾಂಡಿತ್ಯ, ಐಶ್ವರ್ಯ ಎಲ್ಲವನ್ನೂ ತೆರಲು ಸಿದ್ಧ. ಅದರಂತೆಯೇ ನಮಗೆ ಯಾರನ್ನೊ ಕಂಡರೆ ಆಗುವುದಿಲ್ಲ, ಅವರನ್ನು ಹಾಳುಮಾಡಬೇಕು. ಅದಕ್ಕಾಗಿ ನಮ್ಮಲ್ಲಿ ಇರುವುದನ್ನು ವಿನಿಯೋಗಿಸುವೆವು. ಕಾಮಕ್ರೋಧವೆ ನಮಗೆ ಹಾಕಿರುವ ಮೂಗುದಾರಗಳು; ಅವು ಎಳೆದತ್ತ ನಾವು ಹೋಗುತ್ತೇವೆ.

ಈ ಕಾಮ ಕ್ರೋಧ ಎಲ್ಲಿಂದ ಬಂತು? ಇದು ರಜೋಗುಣದಿಂದ ಹುಟ್ಟಿರುವುದು. ಈ ರಜೋಗುಣ ನಮ್ಮಲ್ಲಿರುವುದು. ಈ ಗುಣದ ಶಾಖೋಪಶಾಖೆಗಳು ನಾವು ಮಾಡುವ ಕೆಲಸ. ಎಲ್ಲಿಯವರೆಗೆ ಈ ಗುಣ ನಮ್ಮಲ್ಲಿರುವುದೋ ಅಲ್ಲಿಯವರೆಗೆ ಎಷ್ಟು ಸಲ ರೆಂಬೆಕೊಂಬೆಗಳನ್ನು ಕಡಿದುಹಾಕುತ್ತಿದದರೂ ಪುನಃ ಪುನಃ ಅವು ಚಿಗುರುವುವು. ರಜೋಗುಣದ ಸ್ವಭಾವದವನು ಎಂದಿಗೂ ತೆಪ್ಪಗಿರುವುದಿಲ್ಲ. ಏನಾದರೂ ಮಾಡುತ್ತಲೇ ಇರುವನು. ಒಳಗೆ ಆಸೆ ಹರಿಯುತ್ತಿದೆ. ಅವನು ಸುಮ್ಮನಿರಬೇಕೆಂದು ಇಚ್ಛಿಸಿದರೂ ಇರಗೊಡದು ಅದು.

ಇದು ಎಷ್ಟನ್ನಾದರೂ ಕಬಳಿಸುವುದು. ತೃಪ್ತಿ ಮಾತ್ರ ಇಲ್ಲ. ಬರುವ ತೃಪ್ತಿ ಎಲ್ಲ ತಾತ್ಕಾಲಿಕ. ಆ ಸಮಯಕ್ಕೆ ಹೆಚ್ಚು ಎಂದರೆ ಆ ದಿನಕ್ಕೆ ಮಾತ್ರ ನಮಗೆ ಅದು ಬೇಡ. ಕೆಲವು ಕಾಲವಾದ ಮೇಲೆ ಪುನಃ ಅದನ್ನು ನಾವು ಕೇಳುತ್ತೇವೆ. ಎಷ್ಟು ಸಲ ಇದೇ ಕೊನೆ, ಇನ್ನು ಮೇಲೆ ಮಾಡುವುದಿಲ್ಲ, ಎಂದು ಕೆಲವು ಕೆಲಸಗಳನ್ನು ಮಾಡಿರುವೆವು. ನಾವು ಬಿಟ್ಟಿರುವೆವೆ? ಬಿಟ್ಟಿದ್ದರೆ ನಾವು ಇಷ್ಟು ಹೊತ್ತಿಗೆ ಹೊಸ ಮನುಷ್ಯರೇ ಆಗುತ್ತಿದ್ದೆವು. ನಾವು ತೃಪ್ತಿಪಡಿಸುವುದರಿಂದ ಆಸೆ ನಿಲ್ಲುವುದಿಲ್ಲ. ಹಾಗೆ ತೃಪ್ತಿಪಡಿಸು ವುದು ಉರಿಯುತ್ತಿರುವ ಆಸೆಗೆ ಇನ್ನಷ್ಟು ಸೌದೆಯನ್ನು ಹಾಕಿದಂತೆ.

ಈ ಆಸೆ ಎಂಬುದು ಮಹಾಪಾಪಿ. ಆಸೆ ಎಂಬ ಪಿಶಾಚಿ ನಮ್ಮನ್ನು ಮೆಟ್ಟಿಕೊಂಡಿತು ಎಂದರೆ ನಾವು ಅದನ್ನು ತೃಪ್ತಿಪಡಿಸಿಕೊಳ್ಳುವುದಕ್ಕಾಗಿ ಏನನ್ನು ಬೇಕಾದರೂ ಮಾಡುವೆವು. ಎಂತಹ ಅವ ಮಾನದ ಕೆಲಸವನ್ನಾಗಲಿ, ಅಧರ್ಮದ ಕೆಲಸವನ್ನಾಗಲಿ ಮಾಡುವೆವು. ಮುಂದೇನಾಗುವುದು ಎಂಬುದು ಸಂಪೂರ್ಣ ಮರೆತು ಹೋಗುವುದು. ತತ್ಕಾಲದಲ್ಲಿ ಅದನ್ನು ತೃಪ್ತಿಪಡಿಸಿಕೊಳ್ಳುವು ದೊಂದೇ ನಮಗೆ ಕಾಣುತ್ತಿರುವುದು. ರಾಮಾಯಣದಲ್ಲಿ ಅಯೋಧ್ಯಾಕಾಂಡದಲ್ಲಿ ಬರುವ ಕಥೆಯಲ್ಲಿ ಕೈಕೇಯಿ ಕೋಪದ ಗೃಹದಲ್ಲಿದ್ದಳು. ದಶರಥ ಅಲ್ಲಿಗೆ ಹೋಗಿ, ತನ್ನ ಪ್ರಿಯಳಾದ ಕಿರಿಯ ಹೆಂಡತಿಗೆ ಯಾರು ಏನುಮಾಡಿದರು ಹೇಳು, ನೀನು ಏನನ್ನು ಹೇಳಿದರೆ ಅದನ್ನು ನಾನು ಮಾಡಲು ಸಿದ್ಧನಾಗಿರು ವೆನು. ಯಾವ ಅಧರ್ಮಿಗಳನ್ನು ಬಂಧನದಿಂದ ಬಿಡಿಸಬೇಕೆಂದು ಹೇಳಿದರೂ, ಯಾವ ಧರ್ಮಾತ್ಮ ರನ್ನು ಶಿಕ್ಷಿಸೆಂದು ಹೇಳಿದರೂ ನಾನು ಅದನ್ನು ಮಾಡುವುದಕ್ಕೆ ಸಿದ್ಧನಾಗಿದ್ದೇನೆ ಎನ್ನುವನು. ಆ ಕಾಮುಕ ಮುದುಕ ನ್ಯಾಯ ಅನ್ಯಾಯ, ಧರ್ಮ ಅಧರ್ಮ ಎಲ್ಲವನ್ನೂ ತನ್ನ ಹೆಂಡತಿಯ ಪದತಲ ದಲ್ಲಿಟ್ಟು ಅವಳು ಹೇಳಿದಂತೆ ಮಾಡಲು ಸಿದ್ಧನಾಗಿರುವನು. ನಾವೆಲ್ಲಾ ಅದೇ ಪರಿಸ್ಥಿತಿಯಲ್ಲಿರುವೆವು. ನಮ್ಮಲ್ಲಿರುವ ಕಾಮ ಕ್ರೋಧಗಳೇ ನಮ್ಮ ಪ್ರೀತಿಗೆ ಪಾತ್ರವಾದವುಗಳು. ಅವು ಹೇಳಿದಂತೆ ಮಾಡಿ ಹಾಕುತ್ತೇವೆ. ಅನಂತರ ಪಡಬಾರದ ಯಾತನೆಯನ್ನು ಅನುಭವಿಸುತ್ತೇವೆ.

ಇಲ್ಲಿ ಶ್ರೀಕೃಷ್ಣ ಇವೇ ನಿನ್ನ ಶತ್ರು ಎಂದು ಹೇಳುತ್ತಾನೆ. ಜೀವನದಲ್ಲಿ ದೇವರ ಕಡೆಗೆ ಹೋಗುವುದನ್ನು ತಡೆದು, ಸಂಸಾರದ ಗಾಣಕ್ಕೆ ಕಟ್ಟಿ ನಿತ್ಯವೂ ನಮ್ಮನ್ನು ಅದರಲ್ಲಿ ಸುತ್ತಿಸುತ್ತಿರು ವುದು ಈ ಕಾಮಕ್ರೋಧ. ಈ ಒಳಗಿರುವ ಶತ್ರುವಿನಷ್ಟು ಅನಾಹುತವನ್ನು ಯಾವ ಬಾಹ್ಯ ಶತ್ರುವೂ ಮಾಡಲಾರದು. ಇಂತಹ ದೊಡ್ಡ ಶತ್ರುವಾದರೂ ನಮಗೆ ಅವನು ಈಗ ಪ್ರಿಯಮಿತ್ರ, ಹಿತೈಷಿ ಆಗಿರುವನು. ಅವನು ಹೇಳಿದಂತೆ ನಾವು ಕುಣಿಯುತ್ತಿರುವೆವು. ನಾವು ಪಾರಾಗಬೇಕಾದರೆ, ನಮ್ಮ ಪರಿಸ್ಥಿತಿಯನ್ನು ಸರಿಯಾಗಿ ತಿಳಿದುಕೊಳ್ಳಬೇಕು. ನಿಜವಾಗಿ ನಮ್ಮ ಶತ್ರುಗಳಾರು, ಮಿತ್ರರಾರು ಎಂಬುದನ್ನು ಅರಿತುಕೊಳ್ಳಬೇಕು. ಶತ್ರುಗಳನ್ನು ಹೊರಗಟ್ಟಬೇಕು. ಮಿತ್ರರನ್ನು ಒಳಗೆ ಬರಮಾಡಿ ಕೊಳ್ಳಬೇಕು. ಹಿಂದಿನ ಹೀನ ಸಂಸ್ಕಾರಗಳೇ ನಮ್ಮಲ್ಲಿ ಬಹಳ ದಿನಗಳಿಂದ ಬಂದ, ನಮ್ಮ ಆಪ್ತ ಗೆಳೆಯರು. ಅವರ ಕೈಗೆ ಸಿಕ್ಕಿ ನಾವು ಪಡಬಾರದ ದುಃಖಕಷ್ಟಗಳನ್ನೆಲ್ಲಾ ಪಟ್ಟಿದ್ದೇವೆ. ಅವನ್ನು ಆಚೆಗೆ ಕಳುಹಿಸಬೇಕು. ಜೊತೆಜೊತೆಯಲ್ಲಿಯೇ ಉತ್ತಮ ಸಂಸ್ಕಾರಗಳನ್ನು ರೂಢಿಸಿಕೊಳ್ಳಬೇಕು. ಹೀನಕಳೆ ಯನ್ನು ಕೀಳಬೇಕು. ಅದರ ಸ್ಥಳದಲ್ಲಿ ಉತ್ತಮ ಬೀಜವನ್ನು ನೆಡಬೇಕು.

\begin{verse}
ಧೂಮೇನಾವ್ರಿಯತೇ ವಹ್ನಿರ್ಯಥಾದರ್ಶೋ ಮಲೇನ ಚ ।\\ಯಥೋಲ್ಬೇನಾವೃತೋ ಗರ್ಭಸ್ತಥಾ ತೇನೇದಮಾವೃತಮ್ \versenum{॥ ೩೮ ॥}
\end{verse}

{\small ಬೆಂಕಿ ಹೊಗೆಯಿಂದ ಮುಚ್ಚಿಕೊಂಡಿರುವಂತೆ, ಕನ್ನಡಿ ಕೊಳೆಯಿಂದ ಮಾಸಲಾಗುವಂತೆ, ಗರ್ಭ ಜರಾಯುವೆಂಬ ಪೊರೆಯಿಂದ ಮುಚ್ಚಿರುವಂತೆ, ಅಜ್ಞಾನದಿಂದ ಜ್ಞಾನ ಮುಚ್ಚಲ್ಪಟ್ಟಿದೆ.}

ಎಲ್ಲ ಜೀವರಾಶಿಗಳಲ್ಲಿಯೂ ನೈಜವಾದ ಜ್ಞಾನವಿದೆ. ಆದರೆ ಅಜ್ಞಾನದಿಂದ ತ್ರಿಗುಣಗಳ ಮಾಯೆ ಯಿಂದ ಮುಚ್ಚಲ್ಪಟ್ಟಿದೆ. ಯಾವಾಗ ಮುಚ್ಚಿರುವುದನ್ನು ನಿವಾರಿಸುತ್ತೇವೆಯೋ ಆಗ ಜ್ಞಾನ ತಾನೇ ವ್ಯಕ್ತವಾಗುತ್ತದೆ. ಮುಂಚೆಯೂ ಅದು ಅಲ್ಲಿ ಇತ್ತು. ಆದರೆ ಅದರ ಸುತ್ತ ಇದ್ದ ಆತಂಕದಿಂದ ಕಾಣುತ್ತಿರಲಿಲ್ಲ. ಆತಂಕ ಸರಿದೊಡನೆಯೆ ಆಗಲೆ ಅಲ್ಲಿರುವುದು ಕಾಣುವುದು.

ಈ ಆವರಣದಲ್ಲಿ ಸಾತ್ತ್ವಿಕಗುಣದ ಆವರಣದಂತೆ ಇರುವುದೇ ಬೆಂಕಿಯ ಸುತ್ತಲೂ ಇರುವ ಹೊಗೆ. ಬೆಂಕಿ ನಮಗೆ ಕಾಣುತ್ತಿಲ್ಲ. ಆದರೆ ಹೊಗೆ ಇರುವುದರಿಂದ ಒಳಗೆ ಬೆಂಕಿ ಇದೆ ಎಂದು ನಿರ್ವಿವಾದವಾಗಿ ಊಹಿಸಬಹುದು. ಒಳಗಿರುವ ಬೆಂಕಿ ಚೆನ್ನಾಗಿ ಹೊತ್ತಿಕೊಂಡು ಉರಿಯಲು ಮೊದಲುಮಾಡಿದರೆ ಹೊಗೆ ತಾನಾಗಿ ಹೋಗುವುದು. ಹಿಂದೆ ಇರುವ ಬೆಂಕಿ ಕಾಣುವುದು. ಎರಡನೆಯದೆ ಕನ್ನಡಿಯ ಮೇಲೆ ಕುಳಿತ ಕೊಳೆ. ಸುಮ್ಮನೇ ಕಾಲವಿಳಂಬ ಮಾಡಿದರೆ ಆ ಕೊಳೆ ಹೊರಟುಹೋಗಿಬಿಡುವುದಿಲ್ಲ. ನಾವು ತೊಂದರೆ ತೆಗೆದುಕೊಂಡು ಒರಸಿದರೇನೇ ಆ ಕೊಳೆ ಹೋಗ ಬೇಕಾದರೆ, ಅನಂತರ ಅದು ಪ್ರತಿಬಿಂಬಿಸಬೇಕಾದರೆ. ಇಲ್ಲಿ ನಾವು ಸ್ವಲ್ಪ ಕಷ್ಟಪಡಬೇಕಾಗಿದೆ. ಮೂರನೆಯದೆ ತಾಯಿಯ ಗರ್ಭದಲ್ಲಿ ಮಗುವು ಜರಾಯುವೆಂಬ ಪೊರೆಯಿಂದ ಮುಚ್ಚಲ್ಪಟ್ಟಿರು ವುದು. ಸುಮ್ಮನೆ ನಾವು ತೊಂದರೆ ತೆಗೆದುಕೊಂಡರೆ ಬೇಗ ಮಗು ಬೆಳೆಯುವುದಿಲ್ಲ. ಅದಕ್ಕೆ ಎಷ್ಟು ಕಾಲ ಬೇಕೋ ಅಷ್ಟು ಆಗಬೇಕು. ಆಗಲೆ ಆ ಪೊರೆ ಬಿದ್ದುಹೋಗಬೇಕಾದರೆ. ಒಂದು ಕುರು ಏಳುತ್ತಿದೆ. ಮುಂಚೆ ಅದು ಮಾಗಬೇಕು. ಅನಂತರ ಮಾತ್ರ ಶಸ್ತ್ರಚಿಕಿತ್ಸೆ ಮಾಡಬಹುದು. ಮುಂಚೆಯೇ ಮಾಡುವುದಕ್ಕೆ ಆಗುವುದಿಲ್ಲ. ಈ ಉದಾಹರಣೆಗಳ ಮೂಲಕ ನಾವು ಹುಡುಕುವ ವಸ್ತು ಆಗಲೆ ಇದೆ, ಅದು ನಮಗೆ ದೊರಕಬೇಕಾದರೆ ನಾವು ತೊಂದರೆ ತೆಗೆದುಕೊಳ್ಳಬೇಕು, ಕಾಲಾವಕಾಶವಾಗಬೇಕು. ಆಗ ಅದು ನಮಗೆ ದೊರಕುತ್ತದೆ ಎಂದು ಹೇಳುತ್ತಾನೆ.

\begin{verse}
ಆವೃತಂ ಜ್ಞಾನಮೇತೇನ ಜ್ಞಾನಿನೋ ನಿತ್ಯವೈರಿಣಾ ।\\ಕಾಮರೂಪೇಣ ಕೌಂತೇಯ ದುಷ್ಪೂರೇಣಾನಲೇನ ಚ \versenum{॥ ೩೯ ॥}
\end{verse}

{\small ಅರ್ಜುನ, ಜ್ಞಾನಿಗೆ ನಿತ್ಯವೈರಿಯಾಗಿರುವ, ತೃಪ್ತಿಪಡಿಸುವುದಕ್ಕೆ ಅಸಾಧ್ಯವಾದ ಕಾಮದಿಂದ ಜ್ಞಾನ ಮುಚ್ಚಲ್ಪಟ್ಟಿದೆ.}

ಮನುಷ್ಯನಲ್ಲಿರುವ ಜ್ಞಾನವನ್ನು ಮುಚ್ಚಿಡುವುದೇ ಅವನಲ್ಲಿರುವ ಕಾಮ. ಎಲ್ಲಿ ಕಾಮವಿರುವುದೋ ಅಲ್ಲಿ ರಾಮ ಕಾಣಿಸಲಾರ. ನಮ್ಮನ್ನು ಮುತ್ತಿರುವ ಕಾಮದ ಸ್ವಭಾವವಾದರೋ ಅದು ಜ್ಞಾನಕ್ಕೆ ನಿತ್ಯವೈರಿ. ತಾತ್ಕಾಲಿಕ ವಸ್ತುವಿನ ಕಡೆ ಯಾವಾಗ ಹೋಗುತ್ತೇವೆಯೋ ಆಗ ನಿತ್ಯವನ್ನು ಮರೆಯು ತ್ತೇವೆ. ಶ್ರೀರಾಮಕೃಷ್ಣರು ಒಂದು ಉದಾಹರಣೆ ಕೊಡುತ್ತಿದ್ದರು. ವರ್ತಕರು ತಮ್ಮ ದಿನಸಿ ಮಳಿಗೆಗಳಿಗೆ ಹೆಗ್ಗಣ ನುಗ್ಗಿ ಲೂಟಿ ಮಾಡದೆ ಇರಲಿ ಎಂದು ಅದು ಬರುವ ದಾರಿಯಲ್ಲೇ ಸ್ವಲ್ಪ ಪುರಿಯನ್ನೋ ಏನನ್ನೋ ಚೆಲ್ಲುವರು. ಆ ಹೆಗ್ಗಣಗಳು ಅದನ್ನು ತಿಂದು ಹೋಗುವುವು. ದಿನಸಿ ಇರುವ ಉಗ್ರಾಣದ ಕಡೆ ಹೋಗುವುದಿಲ್ಲ. ಅದರಂತೆಯೇ ನಿತ್ಯವಾದ ಆತ್ಮವಸ್ತುವಿನ ದಾರಿಯಲ್ಲಿ ಸುಲಭ ವಾಗಿ ಸಿಕ್ಕತಕ್ಕ ಇಂದ್ರಿಯಗಳಿಗೆ ಸಂತೋಷವನ್ನು ಕೊಡುವ ವಿಷಯವಸ್ತುಗಳು ಬಿದ್ದಿವೆ. ಯಾವಾಗ ಅವನ್ನು ಜೀವಿ ಕಬಳಿಸುತ್ತಾನೆಯೋ ಅದರ ಬಲೆಗೆ ಬೀಳುತ್ತಾನೆ. ಅದರಿಂದ ತಪ್ಪಿಸಿಕೊಂಡು ಬರುವುದು ಬಹಳ ಕಷ್ಟ. ಮೀನು ಗಾಳದ ಕೊನೆಯಲ್ಲಿರುವ ಹುಳುವನ್ನು ತಿಂದೊಡನೆಯೇ, ಆ ಗಾಳ ಕುತ್ತಿಗೆಗೆ ಸಿಕ್ಕಿಕೊಳ್ಳುವುದು. ಅದರಂತೆಯೇ ವಿಷಯ ವಸ್ತುಗಳು. ನಾವು ಅದಕ್ಕೆ ಎಷ್ಟು ಕೊಟ್ಟರೂ ಅದಕ್ಕೇನೂ ತೃಪ್ತಿಯಾಗುವುದಿಲ್ಲ. ಪುರಾಣದಲ್ಲಿ ಬರುವ ಯಯಾತಿ ತನ್ನ ಯೌವನವನ್ನು ಅನುಭವಿಸಿ ತೃಪ್ತಿ ಬಾರದೆ, ತನ್ನ ಮಗನ ಯೌವನವನ್ನು ಸಾಲವಾಗಿ ತೆಗೆದುಕೊಂಡು ಅದರ ಮೂಲಕವೂ ಅನುಭವಿಸುವನು. ಆದರೂ ತೃಪ್ತಿ ಬರಲಿಲ್ಲ. ಆಗ ಅವನು ಕಾಮವನ್ನು ಭೋಗಿಸಿ ತೃಪ್ತಿಪಡಿಸುತ್ತೇನೆ ಎಂಬುದು ಆಗಲೇ ಉರಿಯುತ್ತಿರುವ ಬೆಂಕಿಗೆ ತುಪ್ಪವನ್ನು ಹಾಕಿ ಆರಿಸಲೆತ್ನಿಸಿದಂತೆ ಎನ್ನುವನು. ಆರಿಸುವ ಬದಲು ಅದು ಇನ್ನೂ ಹೆಚ್ಚು ಹೊತ್ತಿಕೊಳ್ಳುವುದು. ತೃಪ್ತಿಪಡಿಸುತ್ತ ಹೋದಂತೆಲ್ಲಾ ಆ ಸಂಸ್ಕಾರ ಮತ್ತೂ ಬಲವಾಗುತ್ತ ಬರುವುದು. ಜ್ಞಾನ ವೃದ್ಧಿಯಾದರೆ ಮಾತ್ರ ಅದು ಅಜ್ಞಾನದ ಗರ್ಭದಿಂದ ಸೀಳಿಕೊಂಡು ಬರಬಹುದು. ಜ್ಞಾನವನ್ನು ವೃದ್ಧಿ ಮಾಡಿಕೊಳ್ಳಬೇಕಾದರೆ ಅದಕ್ಕೆ ವಿರೋಧಿಯಾದ ಕಾಮದಿಂದ ದೂರವಿರಬೇಕು. ಆಗ ಮಾತ್ರ ಸಾಧ್ಯವಾಗುವುದು.

\begin{verse}
ಇಂದ್ರಿಯಾಣಿ ಮನೋ ಬುದ್ಧಿರಸ್ಯಾಧಿಷ್ಠಾನಮುಚ್ಯತೇ ।\\ಏತೈರ್ವಿಮೋಹಯತ್ಯೇಷ ಜ್ಞಾನಮಾವೃತ್ಯ ದೇಹಿನಮ್ \versenum{॥ ೪ಂ ॥}
\end{verse}

{\small ಇಂದ್ರಿಯ, ಮನಸ್ಸು, ಬುದ್ಧಿ ಇವೇ ಕಾಮಕ್ಕೆ ಆಶ್ರಯಸ್ಥಾನಗಳು. ಕಾಮ ಇವುಗಳ ಮೂಲಕ ಮುತ್ತಿ ಜೀವಿಯನ್ನು ಮೋಹಗೊಳಿಸುವುದು.}

ಕಾಮಗಳು ವಿಷಕ್ರಿಮಿಗಳಂತೆ. ಅವೆಲ್ಲ ವೃದ್ಧಿಯಾಗುವುದಕ್ಕೆ ಕೆಲವು ಸ್ಥಳಗಳಿವೆ. ಅವೇ ಇಂದ್ರಿಯ, ಮನಸ್ಸು ಮತ್ತು ಬುದ್ಧಿ. ಮಲೇರಿಯಾ ತರುವ ಸೊಳ್ಳೆಗಳು ಪಾಚಿಯ ನೀರಿನಲ್ಲಿ ಹೇಗೆ ವೃದ್ಧಿಯಾಗು ವುದೋ ಹಾಗೆಯೇ ಕಾಮದ ವಿಷಕ್ರಿಮಿಗಳು ವೃದ್ಧಿಯಾಗುವ ಕೆಸರಿನ ಗುಂಡಿಗಳು ಇವು. ತುಂಬಾ ಸ್ಥೂಲವಾಗಿರುವುದೇ ನಮ್ಮ ದೇಹದಲ್ಲಿರುವ ಇಂದ್ರಿಯಗಳು. ಅದು ಬಾಹ್ಯ ವಿಷಯವಸ್ತುವಿನ ಸಂಪರ್ಕದಿಂದ ಪ್ರಿಯವಾದ ಯಾವುದನ್ನಾದರೂ ಅನುಭವಿಸಿದರೆ ಸಾಕು, ಪುನಃ ಪುನಃ ಆ ಅನುಭವ ವನ್ನು ಕೋರುತ್ತ ಇರುತ್ತದೆ, ಆ ವಿಷಯವಸ್ತುವಿನ ಸುತ್ತಲೂ ಸುತ್ತುತ್ತಿರುವಂತೆ ಕಾಣುವುದು. ದನವನ್ನು ಒಂದು ಗೂಟಕ್ಕೆ ಕಟ್ಟಿಹಾಕಿದರೆ ಅದು ಆ ಗೂಟದ ಸುತ್ತಲೂ ಹೇಗೆ ಸುತ್ತುತ್ತಿರುವುದೋ ಅದರಂತೆಯೇ ಇಂದ್ರಿಯ ಅದಕ್ಕೆ ಸಂಬಂಧಪಟ್ಟ ವಿಷಯವಸ್ತುವಿನ ದಾಸನಾಗಿ ಅದರಿಂದ ಸಿಕ್ಕುವ ಕೂಳಿಗೆ ಕಾದು ಕುಳಿತಿರುತ್ತದೆ. ತೃಪ್ತಿಪಡಿಸಿದರೆ ಅದೇನೂ ಬಿಟ್ಟುಹೋಗುವುದಿಲ್ಲ. ಪುನಃ ಕೇಳುವುದು ಬಲವಾಗುವುದು. ಕೊಡದೇ ಇದ್ದರೆ ಕಾಡುವುದು. ಆ ಕಾಟದಿಂದ ಪಾರಾಗುವುದಕ್ಕೇ ಕೊಡುವುದು. ಅಂತೂ ಕೊಡುತ್ತ ಹೋಗುತ್ತಿರುವೆವು. ಹಳೆಯ ಸಾಲವನ್ನು ತೀರಿಸುವುದಕ್ಕೆ ಹೊಸ ಸಾಲ. ಆ ಹೊಸ ಸಾಲವೂ ಹೆಚ್ಚು ಬಡ್ಡಿ ಕೊಟ್ಟು ತೆಗೆದುಕೊಂಡುದು. ಅಂತೂ ನಾವು ಯಾವಾಗಲೂ ಸಾಲಗಾರರ ಪುತ್ರರೇ ಆಗಬೇಕಾಗುವುದು.

ದೇಹದಲ್ಲಿರುವ ಇಂದ್ರಿಯಕ್ಕಿಂತ ಸೂಕ್ಷ್ಮವಾಗಿರುವುದು ಮನಸ್ಸು. ಯಾವಾಗ ಸ್ಥೂಲವಾಗಿ ನಾವು ಒಂದು ವಸ್ತುವನ್ನು ಅನುಭವಿಸುವೆವೋ ಅದು ನಮ್ಮ ಮನಸ್ಸಿನಲ್ಲಿ ಸಂಸ್ಕಾರವನ್ನು ಬಿಡುವುದು. ಆ ಅನುಭವವನ್ನು ಕುರಿತು ಅದು ಪುನಃ ಪುನಃ ಚಪ್ಪರಿಸುತ್ತಿರುವುದು. ಸ್ಥೂಲವಾಗಿ ಅದನ್ನು ಪುನಃ ಅನುಭವಿಸುವಂತೆ ದೇಹೇಂದ್ರಿಯಗಳಿಗೆ ಪ್ರೇರೇಪಿಸುತ್ತಿರುವುದು. ಮನಸ್ಸು ವಿಷಯವಸ್ತುವನ್ನು ಕುರಿತು ಚಿಂತಿಸುತ್ತಾ ಚಿಂತಿಸುತ್ತಾ ಕೊನೆಗೆ ಮೈಲಿಗೆ ಆಗುವುದು. ಅದಕ್ಕೆ ಸಂಬಂಧಪಟ್ಟ ಭಾವನೆ ಗಳಿಂದ ಮನಸ್ಸು ತುಂಬುತ್ತಾ ಹೋಗುವುದು. ಬುದ್ಧಿ ಮನಸ್ಸಿಗಿಂತ ಸೂಕ್ಷ್ಮವಾಗಿರುವುದು. ಮನಸ್ಸಿನಲ್ಲಿರುವ ಸಂಸ್ಕಾರ ಬಲವಾದರೆ ಅದನ್ನು ತೃಪ್ತಿಪಡಿಸುವುದಕ್ಕೆ ಉಪಾಯಗಳನ್ನು ಹುಡುಕು ತ್ತಿರುವುದು. ಸತ್ಯವನ್ನು ತಿಳಿದುಕೊಳ್ಳುವುದಕ್ಕೆ ಬದಲಾಗಿ ಕಾಮವನ್ನು ತೃಪ್ತಿಪಡಿಸಿಕೊಳ್ಳುವುದಕ್ಕೆ ತನ್ನ ಜಾಣತನವನ್ನೆಲ್ಲಾ ಉಪಯೋಗಿಸುವುದು. ಇದು ಒಬ್ಬ ಲಾಯರ್ ಬುದ್ಧಿಯಂತೆ ಆಗುವುದು. ತನ್ನ ಕಕ್ಷಿ ಗೆಲ್ಲುವುದಕ್ಕೆ ಅವನು ವಾದಿಸುವನು. ಏಕೆಂದರೆ ಅವನು ಇವನಿಗೆ ಹಣ ಕೊಡುವನು. ಅವನಿಗೆ ಸತ್ಯ ತಿಳಿದುಕೊಳ್ಳಲು ಆಸೆಯಿಲ್ಲ. ಗೆಲ್ಲುವುದೇ ಅವನ ಗುರಿ. ಕೇಸನ್ನು ಗೆದ್ದಮೇಲೆಯೂ ಲಾಯರಿಗೆ ತನ್ನ ಕಕ್ಷಿ ನಿಜವೇ ಸುಳ್ಳೇ ಎಂಬುದು ಗೊತ್ತಾಗುವುದಿಲ್ಲ.

ಇಂದ್ರಿಯ, ಮನಸ್ಸು, ಬುದ್ಧಿ ಇವು ಸತ್ಯವನ್ನು ತಿಳಿದುಕೊಳ್ಳುವುದನ್ನು ಬಿಟ್ಟು, ತಮಗೆ ಸುಖ ಕೊಡುವುದು ಯಾವುದು, ಅದನ್ನು ಹೇಗೆ ಸಂಪಾದನೆ ಮಾಡಬೇಕು ಎನ್ನುವುದರಲ್ಲಿ ನಿರತವಾಗಿ ರುತ್ತವೆ.

\begin{verse}
ತಸ್ಮಾತ್ತ್ವಮಿಂದ್ರಿಯಾಣ್ಯಾದೌ ನಿಯಮ್ಯ ಭರತರ್ಷಭ ।\\ಪಾಪ್ಮಾನಂ ಪ್ರಜಹಿ ಹ್ಯೇನಂ ಜ್ಞಾನವಿಜ್ಞಾನನಾಶನಮ್ \versenum{॥ ೪೧ ॥}
\end{verse}

{\small ಅರ್ಜುನ, ಆದಕಾರಣ ನೀನು ಮೊದಲು ಇಂದ್ರಿಯಗಳನ್ನು ನಿಗ್ರಹಿಸಿ, ಜ್ಞಾನ ಮತ್ತು ವಿಜ್ಞಾನವನ್ನು ನಾಶ ಮಾಡುವ ಈ ಪಾಪಿಯನ್ನು ಗೆಲ್ಲು.}

ಮೊದಲು ಇಂದ್ರಿಯಗಳ ಬಾಗಿಲನ್ನು ಹಾಕಬೇಕು ಎನ್ನುತ್ತಾನೆ. ಈ ಹೆಬ್ಬಾಗಿಲಿನ ಮೂಲಕವೇ ಹೊಸಹೊಸ ಸಂಸ್ಕಾರಗಳು ಒಳಗೆ ನುಗ್ಗಬೇಕಾದರೆ. ನಾವು ಮೊದಲು ಹೊರಗಿನಿಂದ ಬರುವುದನ್ನು ನಿಲ್ಲಿಸಬೇಕು. ಅನಂತರ ಆಗಲೇ ಒಳಗೆ ಇರುವುದನ್ನು ಬದಲಾಯಿಸುತ್ತಾ ಬರಬೇಕಾಗುವುದು. ಸ್ಥೂಲವಾಗಿ ಇರುವ ಇಂದ್ರಿಯವನ್ನು ನಿಗ್ರಹಿಸುವುದು ಸುಲಭ. ಸೂಕ್ಷ್ಮವಾದ ಮನಸ್ಸನ್ನು ನಿಗ್ರಹಿಸು ವುದು ಕಷ್ಟ. ಆದಕಾರಣವೇ ಮೊದಲು ಸ್ಥೂಲದಿಂದ ಪ್ರಾರಂಭಿಸಬೇಕು. ಸೂಕ್ಷ್ಮಕ್ಕೆ ಸಾರವನ್ನು ಒದಗಿಸುವುದೇ ಸ್ಥೂಲ ಅನುಭವಗಳು. ಯಾವಾಗ ಹೊಸದಾಗಿ ಬರುವ ಸ್ಥೂಲ ಅನುಭವಗಳನ್ನು ನಿಲ್ಲಿಸುತ್ತೇವೆಯೋ ಒಳಗಿರುವ ಆಸೆ ಕ್ರಮೇಣ ಒಣಗಿಹೋಗುತ್ತ ಬರುವುದು. ಅದಕ್ಕೆ ಮೊದಲನೆ ಹೆಜ್ಜೆಯೇ ಇಂದ್ರಿಯ ನಿಗ್ರಹ.

ಈ ಕಾಮವೇ ಜ್ಞಾನ ಮತ್ತು ವಿಜ್ಞಾನವನ್ನು ನಾಶಮಾಡುವುದು. ಜ್ಞಾನವೆಂದರೆ ಒಂದು ಸರಿಯೆ ತಪ್ಪೆ ಎಂದು ಬೌದ್ಧಿಕವಾಗಿ ತಿಳಿದುಕೊಳ್ಳುವುದು. ವಿಜ್ಞಾನವೆಂದರೆ ಅದನ್ನು ಅನುಭವಿಸುವುದು. ಒಂದು ಸಿದ್ಧಾಂತ, \eng{Theory}, ಮತ್ತೊಂದು \eng{Experience}. ಕಾಮ ಒಂದು ವಸ್ತುವಿನ ನೈಜ ಸ್ಥಿತಿಯನ್ನು ತಿಳಿದುಕೊಳ್ಳುವುದಕ್ಕೆ ನಮ್ಮನ್ನು ಬಿಡುವುದಿಲ್ಲ. ವಸ್ತುವಿನ ನೈಜ ಸ್ಥಿತಿ ಗೊತ್ತಾದರೆ ಅದರ ಮೇಲೆ ನಮಗೆ ವ್ಯಾಮೋಹ ಇರುವುದಿಲ್ಲ. ಕಾಮ ಅದಕ್ಕೆ ಆಸ್ಪದ ಕೊಡುವುದಿಲ್ಲ. ಇನ್ನು ಇಂದ್ರಿಯಾತೀತ ಅನುಭವವಾದರೊ ಕಾಮಕ್ಕೆ ದಾಸನಾಗಿರುವವನಿಗೆ ಹೇಗೆ ದೊರಕುವುದು? ಮೊದಲು ಕಾಮದಿಂದ ಪಾರಾಗಬೇಕು. ಅನಂತರವೇ ಜ್ಞಾನ ಮತ್ತು ಅನುಭವ ಸಾಧ್ಯ.

\begin{verse}
ಇಂದ್ರಿಯಾಣಿ ಪರಾಣ್ಯಾಹುರಿಂದ್ರಿಯೇಭ್ಯಃ ಪರಂ ಮನಃ ।\\ಮನಸಸ್ತು ಪರಾ ಬುದ್ಧಿರ್ಯೋ ಬುದ್ಧೇಃ ಪರತಸ್ತು ಸಃ \versenum{॥ ೪೨ ॥}
\end{verse}

{\small ಇಂದ್ರಿಯಗಳನ್ನು ಹೆಚ್ಚಿನವು ಎಂದು ಹೇಳುತ್ತಾರೆ. ಇಂದ್ರಿಯಗಳಿಗಿಂತ ಮನಸ್ಸು ಹೆಚ್ಚಿನದು. ಮನಸ್ಸಿಗಿಂತ ಬುದ್ಧಿ ಹೆಚ್ಚಿನದು. ಆದರೆ ಯಾರು ಬುದ್ಧಿಗಿಂತ ಹೆಚ್ಚಿನವನೋ ಅವನು ಆತ್ಮ.}

ಇಂದ್ರಿಯಗಳು ಹೆಚ್ಚಿನವು. ದೇಹದಿಂದ ಹೊರಗೆ ಇರುವ ವಸ್ತುಗಳೊಂದಿಗೆ ಸಂಬಂಧವನ್ನು ಕಲ್ಪಿಸಿಕೊಳ್ಳುವುದಕ್ಕೆ ಇರುವ ಬಾಗಿಲುಗಳೇ ಇವು. ಮನುಷ್ಯ ಸುಮಾರು ಆರು ಅಡಿಯಷ್ಟು ಎತ್ತರವಿರುವವನಾದರೂ ಅವನು ಯಾವುದೋ ಕೆಲವು ಇಂಚುಗಳ ಸ್ಥಳಗಳಲ್ಲಿ ಮಾತ್ರ ಇರುತ್ತಾನೆ. ಒಬ್ಬನಿಗೆ ನೂರಾರು ಕೋಣೆಗಳಿರುವ ಮನೆ ಇದ್ದರೂ ಅವನು ಎಲ್ಲಾ ಕೋಣೆಗಳಲ್ಲೂ ಏಕಕಾಲದಲ್ಲಿ ಇರುವುದಕ್ಕೆ ಆಗುವುದಿಲ್ಲ. ಯಾವುದಾದರೂ ಒಂದು ಕೋಣೆಯಲ್ಲಿ ಅವನು ಇರುತ್ತಾನೆ. ಅದ ರಂತೆಯೇ ನೋಡುವಾಗ ಒಂದು ಕಡೆ, ಕೇಳುವಾಗ ಒಂದು ಕಡೆ, ಮೂಸುವಾಗ ಒಂದು ಕಡೆ, ಮುಟ್ಟುವಾಗ ಒಂದು ಕಡೆ ಇರುತ್ತಾನೆ. ಆಗ ಅದರಲ್ಲಿಯೇ ನಿರತನಾಗಿರುವನು. ಈ ಇಂದ್ರಿಯಗಳು ವಿಷಯವಸ್ತುವಿನೆಡೆಗೆ ಧಾವಿಸುವುವು. ಮೇಲಿನಿಂದ ನೀರು ಕೆಳಗೆ ಹೇಗೆ ಹರಿದು ಹೋಗುವುದೋ ಹಾಗೆಯೇ ಇಂದ್ರಿಯಗಳು ವಿಷಯವಸ್ತುವಿನ ಕಡೆ ಹರಿದು ಹೋಗುವುವು. ಮುಕ್ಕಾಲುಪಾಲು ಜನ ಇಂದ್ರಿಯದ ಜಗಲಿಯ ಮೇಲೆ ಕಾಲ ಕಳೆಯುತ್ತಾರೆ–ಮನೆಯ ಒಳಗೆ ಇರುವವರಿಗೆ ಬೇಜಾರಾಗಿ ಹೊರಗೆ ಬಂದು ಜಗಲಿಯ ಮೇಲೆ ಬೀದಿಯಲ್ಲಿ ಆಗುತ್ತಿರುವುದನ್ನು ಕುಳಿತು ನೋಡುವಂತೆ. ಬಾಹ್ಯ ಪ್ರಕೃತಿಯವರೇ ಬಹಳ. ಅವರ ಸುಖ ಮತ್ತು ಸಂತೋಷಕ್ಕೆ ಯಾವಾಗಲೂ ಹೊರಗಿನಿಂದ ವಿಷಯವಸ್ತುಗಳು ಬೀಳುತ್ತಿರಬೇಕು. ಉರಿಯುತ್ತಿರುವ ಆಸೆಯ ಜ್ವಾಲೆಗೆ ಸೌದೆಯನ್ನು ಒದಗಿಸು ತ್ತಿರುವುದೇ ನಾವೀಗ ಬಾಹ್ಯ ಪ್ರಪಂಚದಿಂದ ಇಂದ್ರಿಯದ ಮೂಲಕ ಅನುಭವಿಸುತ್ತಿರುವುದು.

ಮನಸ್ಸು ಇಂದ್ರಿಯಕ್ಕಿಂತ ಹೆಚ್ಚಿನದು. ಇಂದ್ರಿಯಕ್ಕೆ ಪ್ರಾಣ ಕೊಡುವುದು ಅದರ ಹಿಂದೆ ಇರುವ ಮನಸ್ಸು. ಇಂದ್ರಿಯದ ಹಿಂದೆ ಮನಸ್ಸಿದ್ದರೇನೇ ಇಂದ್ರಿಯ ಕೆಲಸ ಮಾಡಬೇಕಾದರೆ. ಮನಸ್ಸು ಮತ್ತಾವುದನ್ನೋ ದೀರ್ಘವಾಗಿ ಯೋಚಿಸುತ್ತಿರುವಾಗ, ಏನು ಶಬ್ದವಾಗಲಿ, ಏನು ನೋಟವಾಗಲಿ, ಕಾಣುವುದಿಲ್ಲ. ಅದೊಂದೇ ಅಲ್ಲ. ಹೊರಗೆ ವಿಷಯ ವಸ್ತುಗಳು ಸಿಕ್ಕದೇ ಇದ್ದರೆ ಮನಸ್ಸು ತನ್ನ ನೆನಪಿನ ಉಗ್ರಾಣದಿಂದ ಹಿಂದೆ ಆದ ಅನುಭವಗಳ ಸರಕನ್ನು ತೆಗೆದುಕೊಂಡು ಅನುಭವಿಸುವುದು.

ಅನಂತರ ಮನಸ್ಸಿಗಿಂತ ಸೂಕ್ಷ್ಮವಾದುದು ಇದೆ. ಅದೇ ಬುದ್ಧಿ. ಅದು ತನಗೆ ಬಂದ ವಿಷಯಗಳ ಅನುಭವಗಳನ್ನೆಲ್ಲಾ ಜೋಡಿಸುವುದು. ಅದರ ಹಿಂದೆ ಇರುವ ನಿಯಮವನ್ನು ಕಂಡುಹಿಡಿಯುವುದು. ಒಂದು ವಸ್ತು ಹೇಗೆ ಆಗುವುದು, ಅದನ್ನು ಪಡೆಯಬೇಕಾದರೆ ಏನೇನು ಮಾಡಬೇಕು, ದಾರಿಯ ಲ್ಲಿರುವ ಆತಂಕಗಳು ಯಾವುವು, ಅವುಗಳಿಂದ ಪಾರಾಗುವುದು ಹೇಗೆ ಇವುಗಳನ್ನು ಚೆನ್ನಾಗಿ ವಿಚಾರಮಾಡುವುದು. ಆದರೆ ಬುದ್ಧಿ ತಾನೆ ಯಜಮಾನನಲ್ಲ. ಮತ್ತಾರಿಗೋ ಆಳು. ಅವನು ಹೇಳಿದಂತೆ ಕೇಳುವುದು. ಬುದ್ಧಿ ಯಜಮಾನನಿಗೆ ಇರುವ ಬುದ್ಧಿವಂತನಾದ ಆಳಿನಂತೆ. ಆ ಯಜಮಾನನೇ ಆತ್ಮ. ಅವನ ಅಧೀನ ಬುದ್ಧಿ, ಮನಸ್ಸು, ಇಂದ್ರಿಯ, ದೇಹಗಳೆಲ್ಲಾ. ಅವನಿಗಾಗಿ ಇವು ದುಡಿಯುತ್ತಿವೆ.

\begin{verse}
ಏವಂ ಬುದ್ಧೇಃ ಪರಂ ಬುದ್ಧ್ವಾ ಸಂಸ್ತಭ್ಯಾತ್ಮಾನಮಾತ್ಮನಾ ।\\ಜಹಿ ಶತ್ರುಂ ಮಹಾಬಾಹೋ ಕಾಮರೂಪಂ ದುರಾಸದಮ್ \versenum{॥ ೪೩ ॥}
\end{verse}

{\small ಅರ್ಜುನ, ಬುದ್ಧಿಗಿಂತ ಹೆಚ್ಚಿನದನ್ನು ತಿಳಿದುಕೊಂಡು, ಮನಸ್ಸನ್ನು ಮನಸ್ಸಿನಿಂದ ಗೆದ್ದು ಕಾಮರೂಪವೂ, ದುರ್ಜಯವೂ ಆದ ಶತ್ರುವನ್ನು ನಾಶಮಾಡು.}

ನಾವು ದೇಹದ ಹೊರವಲಯದಲ್ಲಿರುವೆವು. ಅಂತರ್ಮುಖರಾಗಿ ಇನ್ನು ಮೇಲೆ ಒಳಗೆ ಹೋಗ ಬೇಕಾಗಿದೆ. ಅಲ್ಲಿ ಬುದ್ಧಿಯ ಹಿಂದೆ ಇರುವ ಅಂತಃಪುರದಲ್ಲಿ ಆತ್ಮನಿರುವನು. ನಾವು ಅವನನ್ನು ತಿಳಿದುಕೊಳ್ಳಬೇಕು. ಅವನಿಗೆ ಯಾವ ದೇಹವೂ ಇಲ್ಲ, ಅವನು ಹುಟ್ಟಿಲ್ಲ, ಸಾಯುವುದಿಲ್ಲ. ಎಂದೆಂದಿಗೂ ಇರುವವನು ಅವನೊಬ್ಬನೆ. ಹುಟ್ಟಿ ನಾಶವಾಗುವ ದೇಹಾದಿಗಳನ್ನು ನಾವು ಎಂದು ಭಾವಿಸಿ ಅದಕ್ಕಾಗಿ ದುಡಿಯುತ್ತಿರುವೆವು. ನಮ್ಮದಲ್ಲದುದನ್ನು ನಾವು ಎಂದು ಭಾವಿಸಿರುವೆವು. ಯಾವುದು ನಿಜವಾಗಿ ನಾನೋ ಅದನ್ನು ಪ್ರಶ್ನಿಸುವುದೇ ಇಲ್ಲ. ಜೀವನದಲ್ಲಿ ಇಷ್ಟು ವಿಚಿತ್ರ ವಾಗಿರುವುದು ಇನ್ನೊಂದು ಇಲ್ಲ. ನಾನು ಯಾರೋ ಅದು ಗೊತ್ತಿಲ್ಲ. ನಾನು ಯಾರಲ್ಲವೋ ಅದೆಲ್ಲಾ ನನಗೆ ಗೊತ್ತು. ನಾವು ಯಾವುದನ್ನು ಮುಖ್ಯವಾಗಿ ಮುಂಚೆ ತಿಳಿದುಕೊಳ್ಳಬೇಕೋ ಅದಕ್ಕೆ ಗಮನವನ್ನೇ ಕೊಡುವುದಿಲ್ಲ. ಯಾವುದು ಗೌಣವೋ, ಯಾವುದನ್ನು ಇಲ್ಲೇ ಹರಿದ ಬಟ್ಟೆಯಂತೆ ಎಸೆದು ಹೋಗುವೆವೋ ಅದರ ರಕ್ಷಣೆಗಾಗಿ ಹಗಲಿರುಳು ದುಡಿಯುತ್ತಿರುವೆವು. ಆದಕಾರಣವೇ ಜೀವನದಲ್ಲಿ ಮುಖ್ಯವಾಗಿ ನಾವು ತಿಳಿದುಕೊಳ್ಳಬೇಕಾಗಿರುವುದು ಆತ್ಮವಸ್ತುವನ್ನು. ಮುಂಚೆ ಆತ್ಮ, ಅನಂತರ ಅದಕ್ಕೆ ಹಾಕಿರುವ ಉಪಾಧಿಗಳು, ದೇಹ ಇಂದ್ರಿಯ ಮನಸ್ಸು ಬುದ್ಧಿಗಳು.

ನಾವು ಇಲ್ಲಿ ಮಾಡಬೇಕಾದ ಒಂದು ದೊಡ್ಡ ಕೆಲಸವೇ ಮನಸ್ಸಿನಿಂದ ಮನಸ್ಸನ್ನು ಗೆಲ್ಲಬೇಕು. ಒಳ್ಳೆಯ ಮನಸ್ಸಿನಿಂದ ಕೆಟ್ಟ ಮನಸ್ಸನ್ನು ಗೆಲ್ಲಬೇಕು. ಉತ್ತಮ ಸಂಸ್ಕಾರದಿಂದ ಅಧಮ ಸಂಸ್ಕಾರ ಗಳನ್ನು ನಾಶಮಾಡಬೇಕು. ಮಾಡಿರುವ ಸಾಲವನ್ನು ಹೊಸದಾಗಿ ಸಂಪಾದನೆ ಮಾಡಿ ತೀರಿಸಬೇಕಾಗಿದೆ. ಇಲ್ಲಿ ಬೇರೆಯವರು ಯಾರೂ ನಮ್ಮ ಸಹಾಯಕ್ಕೆ ಬರಲಾರರು. ನಮ್ಮ ಹೀನ ಸಂಸ್ಕಾರಗಳನ್ನು ಅದಕ್ಕೆ ವಿರೋಧವಾದ ಉತ್ತಮ ಸಂಸ್ಕಾರಗಳಿಂದ ಗೆಲ್ಲಬೇಕಾಗಿದೆ. ನಮ್ಮಲ್ಲಿ ನಮ್ಮನ್ನು ಧ್ವಂಸ ಮಾಡುವ ಶಕ್ತಿಯೂ ಇದೆ, ನಮ್ಮನ್ನು ಉದ್ಧಾರ ಮಾಡುವ ಶಕ್ತಿಯೂ ಇದೆ. ಇದುವರೆಗೆ ಧ್ವಂಸ ಶಕ್ತಿಯ ಅನುಭವ ನಮಗೆ ಆಗಿದೆ. ಅದನ್ನು ಬೇಕಾದಷ್ಟು ಉಪಯೋಗಿಸಿರುವೆವು. ಹಲವು ಹೀನ ಸಂಸ್ಕಾರಗಳನ್ನು ಅದರ ಮೂಲಕ ಪಡೆದುಕೊಂಡಿರುವೆವು. ನಾವು ಬಂಧನಕ್ಕೆ ಬೀಳುವುದಕ್ಕೆ ಇದೇ ಕಾರಣ. ಈಗ ನಾವು ಅವುಗಳ ಹಿಡಿತದಿಂದ ಪಾರಾಗಬೇಕಾಗಿದೆ. ಅದಕ್ಕಾಗಿ ನಾವೆಲ್ಲಿಗೂ ಹುಡುಕಿ ಕೊಂಡು ಹೋಗಬೇಕಾಗಿಲ್ಲ. ನಮ್ಮಲ್ಲಿರುವ ಉತ್ತಮ ಆಸೆ, ಆಕಾಂಕ್ಷೆ, ಇವುಗಳ ಸಹಾಯವನ್ನು ತೆಗೆದುಕೊಂಡು ಪ್ರಯತ್ನಮಾಡಿ ಹಿಂದಿನ ಹಿಡಿತದಿಂದ ಪಾರಾಗಬೇಕಾಗಿದೆ. ಕಾಡಿನಲ್ಲಿ ಆನೆಗಳನ್ನು ಹಿಡಿಯುತ್ತಾರೆ. ಅನಂತರ ಅದಕ್ಕೆ ಹಗ್ಗವನ್ನು ಹಾಕಿ ಕಟ್ಟುವುದಕ್ಕೆ ಸಾಕಿದ ಆನೆಗಳನ್ನು ಉಪಯೋಗಿ ಸುತ್ತಾರೆ. ಸಾಕಿದ ಆನೆ ಕಾಡಾನೆಯನ್ನು ಹಗ್ಗದಿಂದ ಕಟ್ಟುವುದು. ಅದರಂತೆಯೇ ನಾವು ನಮ್ಮಲ್ಲಿರುವ ಕೆಲವು ಉತ್ತಮ ಸಂಸ್ಕಾರಗಳನ್ನು ವೃದ್ಧಿಗೊಳಿಸಬೇಕಾಗಿದೆ. ಅವುಗಳ ಮೂಲಕವೇ ನಾವು ನಮ್ಮ ಹೀನ ಸಂಸ್ಕಾರಗಳನ್ನು ನಾಶಮಾಡಲು ಸಾಧ್ಯ. ನಾವು ಒಂದು ಸ್ಲೇಟಿನ ಮೇಲೆ ಅಶ್ಲೀಲವಾಗಿರು ವುದನ್ನು ನಮ್ಮ ಕೈಯಿಂದ ಬರೆದಿರುವೆವು. ಈಗ ಅದೇ ಕೈಯಿಂದ ಅದನ್ನು ಒರಸಿ ಉತ್ತಮವಾದ ವಿಷಯಗಳನ್ನು ಅಲ್ಲಿ ಬರೆಯಬೇಕಾಗಿದೆ.

ಉತ್ತಮ ಸಂಸ್ಕಾರಗಳನ್ನು ರೂಢಿಸುವ ಪ್ರಯತ್ನದಿಂದಲೇ ನಮ್ಮ ಶತ್ರುವನ್ನು ನಾಶಮಾಡಬೇಕು. ನಮ್ಮ ಶತ್ರುವಾದರೋ ಕಾಮರೂಪಿ. ನನಗೆ ಅದು ಬೇಕು, ಇದು ಬೇಕು, ಎಂದು ವಿಷಯ ವಸ್ತುಗಳನ್ನು ಬೇಡುತ್ತಿರುವನು. ಕೊಡದೆ ಇದ್ದರೆ ಕಾಡುವನು. ಆ ಕಾಟ ಕೆಲವು ವೇಳೆ ಯಮಯಾತನೆ ಗಿಂತ ಭಯಂಕರವಾಗಿರುವುದು. ಅನೇಕ ವೇಳೆ ನಾವು ಅದಕ್ಕೆ ಕೊಡುವುದು ಯಮಕಾಟದಿಂದ ಪಾರಾಗುವುದಕ್ಕೆ. ಆದರೆ ಅವನು ನಮ್ಮನ್ನು ಕಾಡುವುದನ್ನು ಬಿಡುವನೋ? ಇಲ್ಲ. ಪುನಃ ಬರುವನು. ಮತ್ತೂ ಬಲವಾಗಿ ಬರುವನು, ನಾವು ಹಿಂದೆ ಕೊಟ್ಟುದನ್ನು ತಿಂದಾದಮೇಲೆ.

ಈ ಕಾಮರೂಪಿಯಾದ ಶತ್ರು, ದುರ್ಜಯ. ಗೆಲ್ಲುವುದಕ್ಕೆ ಬಹಳ ಕಷ್ಟ. ಒಂದು ರೂಪಿನಿಂದ ಅವನನ್ನು ಕಂಡುಹಿಡಿದು ಹೊರಗೆ ಅಟ್ಟಿದರೆ, ಬೇರೊಂದು ರೂಪವನ್ನು ಧರಿಸಿ ಬರುವನು. ಬಾಗಿಲಿನಿಂದ ಅವನನ್ನು ಹೊರಗೆ ನೂಕಿ ಬಾಗಿಲು ಹಾಕಿದರೆ ಕಿಟಿಕಿಯಿಂದ ಬರುವನು. ಕಿಟಿಕಿ ಹಾಕಿದರೆ ಗವಾಕ್ಷದ ಮೂಲಕ ಬರುವನು. ಹೇಗೆ ಹಿಂದೆ ಅಸುರರು ತಮಗೆ ಬೇಕಾದ ಆಕಾರವನ್ನು ಧರಿಸಬಲ್ಲರು ಎಂದು ಕೇಳಿದ್ದೆವೊ ಅದರಂತೆ ನಮ್ಮ ಕಾಮ. ಇದು ತುಂಬ ಕೆಟ್ಟದ್ದು. ಇದರಿಂದ ನಾವು ಹಾಳಾಗುತ್ತೇವೆ. ಇದನ್ನು ಆಚೆಗೆ ಅಟ್ಟಬೇಕು ಎಂದು ಕಳಿಸಿದರೆ, ಅದು ತುಂಬಾ ಒಳ್ಳೆಯ ವೇಷ ಹಾಕಿಕೊಂಡು ಬರುವುದು. ಮಾರೀಚ ತನ್ನ ರಾಕ್ಷಸನ ಆಕಾರವನ್ನು ತೊರೆದು ಸುಂದರವಾದ ಜಿಂಕೆಯಂತೆ ಸೀತೆಯ ಮುಂದೆ ಬಂದು ನಿಂತನು. ಅಂತಹ ಶತ್ರುವನ್ನು ಕಂಡುಹಿಡಿಯಬೇಕಾದರೆ ನಮ್ಮ ಬುದ್ಧಿವಂತಿಕೆಯನ್ನೆಲ್ಲ ಉಪಯೋಗಿಸಬೇಕು. ಅದೊಂದೇ ಅಲ್ಲ, ಅವನನ್ನು ಕಂಡುಹಿಡಿದ ಮೇಲೂ ಅವನನ್ನು ಕೊಂದೆ ಎಂದು ಭಾವಿಸಿದರೆ, ಆ ಬೂದಿಯಿಂದ ಪುನಃ ಹುಟ್ಟುವನು. ಹಳೆಯ ಆಸೆಯ ಅವಶೇಷದಿಂದ ಅವನು ಮೇಲೇಳುವನು. ಅವನನ್ನು ಪದೇ ಪದೇ ಆಚೆಗೆ ಅಟ್ಟುತ್ತಿರಬೇಕು. ಅವನನ್ನು ಅಸಡ್ಡೆಯಿಂದ ನೋಡುವುದು, ಆಚೆಗೆ ಅಟ್ಟುವುದು ನಮ್ಮ ಸ್ವಭಾವ ಆಗಬೇಕು. ದೀರ್ಘ ಅಭ್ಯಾಸ ಮತ್ತು ಶ್ರಮದ ಪರಿಣಾಮವಾಗಿಯೇ ಅವನು ನಮ್ಮನ್ನು ಬಿಡಬೇಕಾದರೆ. ದುರ್ಜಯ ಎಂದರೆ ಅವನನ್ನು ಜಯಿಸುವುದಕ್ಕೇ ಆಗುವುದಿಲ್ಲ ಎಂದಲ್ಲ. ಅವನನ್ನು ಜಯಿಸುವುದು ಅಷ್ಟು ಸುಲಭವಲ್ಲ. ಅದಕ್ಕಾಗಿ ಕಷ್ಟಪಡಬೇಕು, ಸತತ ಪ್ರಯತ್ನ ಪಡಬೇಕು. ಮರಳಿ ಯತ್ನವ ಮಾಡು, ಮರಳಿ ಯತ್ನವ ಮಾಡು ಎಂಬ ನುಡಿ ಇತರ ಕ್ಷೇತ್ರಗಳಲ್ಲಿ ಹೇಗೆ ಸತ್ಯವೋ ಅದಕ್ಕಿಂತ ಹೆಚ್ಚು ಸತ್ಯ ಈ ಆಧ್ಯಾತ್ಮಿಕ ಜೀವನದ ಹೋರಾಟದಲ್ಲಿ.

