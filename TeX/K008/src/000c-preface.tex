
\chapter*{ಪ್ರಸ್ತಾವನೆ}

ಹಲವಾರು ವರುಷಗಳಿಂದ ಜನರಿಗೆ ಭಗವದ್ಗೀತೆಯನ್ನು ವಿವರಿಸುತ್ತಿದ್ದೆ. ಅನೇಕ ವೇಳೆ ಇದನ್ನು ಕೇಳಿದವರು ನೀವು ಇದನ್ನು ಒಂದು ಪುಸ್ತಕರೂಪಕ್ಕೆ ತೆಗೆದುಕೊಂಡು ಬಂದರೆ ಮೇಲಾಗುವುದು; ಕೇಳಿ ಅನೇಕವೇಳೆ ಮರೆಯುವೆವು; ಪುನಃ ಅದನ್ನು ಜ್ಞಾಪಿಸಿಕೊಳ್ಳುವುದಕ್ಕೆ ಒಂದು ಪುಸ್ತಕರೂಪದಲ್ಲಿ ಇದ್ದರೆ ಸಹಾಯವಾಗುವುದು ಎಂದು ಹೇಳುತ್ತಿದ್ದರು. ಅವರ ಕೋರಿಕೆಯಂತೆ ಮಾತಿನಲ್ಲೆ ಕೊನೆ ಗಾಣುತ್ತಿದ್ದ ವಿಷಯಗಳನ್ನು ಪುಸ್ತಕರೂಪದಲ್ಲಿ ಪ್ರಕಟಿಸಿದೆ.

ಭಗವದ್ಗೀತೆಯ ಮೇಲೆ ಹಿಂದಿನಿಂದಲೂ ಅನೇಕ ಜನ ಭಾಷ್ಯಗಳನ್ನು ಬರೆದಿರುವರು. ನಮ್ಮ ಮತ್ತಾವ ಗ್ರಂಥದ ಮೇಲೂ ಅಷ್ಟೊಂದು ಸಾಹಿತ್ಯ ಬೆಳೆದಿಲ್ಲ. ಇದಕ್ಕೆ ಕಾರಣ ಗೀತೆಗೆ ಜನರ ಮೇಲಿರುವ ಪ್ರಭಾವ. ಯಾರು ಏನನ್ನು ಹೇಳಬೇಕಾದರೂ ಗೀತೆಯ ಮೂಲಕ ಹೇಳಿದರೆ ಜನ ಅದನ್ನು ಪುರಸ್ಕರಿಸುತ್ತಾರೆ, ಇಲ್ಲದೆ ಇದ್ದರೆ ಇಲ್ಲ. ಗೀತೆಯಲ್ಲಿ ಎರಡು ಭಾಗದ ಸತ್ಯಗಳಿವೆ. ಒಂದು-ಆ ಕಾಲ ಮತ್ತು ಆ ದೇಶಕ್ಕೆ ಮಾತ್ರ ಅನ್ವಯಿಸುವುದು. ಮತ್ತೊಂದು ಎಲ್ಲಾ ಕಾಲ ದೇಶ ಜನಾಂಗಕ್ಕೂ ಅನ್ವಯಿಸುವ ಸನಾತನ ತತ್ತ್ವ. ಈ ಎರಡನೆಯ ದೃಷ್ಟಿಯಿಂದ ಗೀತೆ ಎಂದಿಗೂ ಹಳೆಯದಾಗುವುದಿಲ್ಲ. ಆದಕಾರಣವೇ ಅಷ್ಟೊಂದು ಮೇಧಾವಿಗಳು ಹಿಂದಿನಿಂದಲೂ ವಿದ್ವತ್​ಪೂರ್ಣವಾದ ಭಾಷ್ಯಗಳನ್ನು ಬರೆದಿರುವುದು. ನಮಗೆ ಸಿಕ್ಕಿರುವುದರಲ್ಲಿ ಶಂಕರಾಚಾರ್ಯರ ಭಾಷ್ಯವೇ ಅತ್ಯಂತ ಪುರಾತನವಾದುದು. ಬಹುಶಃ ಅದಕ್ಕಿಂತ ಪುರಾತನವಾದುದು ಇದ್ದಿರಬಹುದು. ಆದರೆ ಅದು ನಮಗೆ ಸಿಕ್ಕಿಲ್ಲ. ಅನಂತರ ಶ್ರೀ ರಾಮಾನುಜಾಚಾರ್ಯರು, ಶ್ರೀ ಮಧ್ವಾಚಾರ್ಯರು, ನಿಂಬಾರ್ಕ, ವಲ್ಲಭ, ಶ್ರೀಧರ ಮುಂತಾದ ಪ್ರಮುಖರು ಭಾಷ್ಯವನ್ನು ಬರೆದು ತಮ್ಮ ತಮ್ಮ ಸಿದ್ಧಾಂತಗಳನ್ನು ಸಮರ್ಥಿಸಿರುವರು. ಇತ್ತೀಚೆಗೆ ಕರ್ಮರಂಗದಲ್ಲಿ ಹೆಸರಾಂತ ಮೂರು ವ್ಯಕ್ತಿಗಳಿಂದ ಮೂರು ಭಾಷ್ಯಗಳು ಬಂದಿವೆ. ಇವರೆಲ್ಲ ತಮ್ಮ ಜೀವನದ ಚಟುವಟಿಕೆಗೆ ಗೀತೆಯಿಂದ ಸಾರವನ್ನು ಹೀರಿದವರು. ಅವರೇ ರಾಜಕೀಯ ಕ್ಷೇತ್ರದಲ್ಲಿ ಪ್ರಖ್ಯಾತರಾದ ಬಾಲಗಂಗಾಧರ ತಿಲಕರು, ಮಹಾತ್ಮ ಗಾಂಧೀಜಿ ಮತ್ತು ಪಾಂಡಿಚೆರಿಯ ಅರವಿಂದ ಘೋಷರು. ಇವುಗಳಲ್ಲದೆ ದೇಶಭಾಷೆಯಲ್ಲಿ ಬೇಕಾದಷ್ಟು ಗೀತೆಯ ಮೇಲೆ ಪುಸ್ತಕಗಳು ಬಂದಿವೆ. ಗೀತೆ ಆತ್ಮನ ವಿಕಾಸಕ್ಕೆ ಸಂಬಂಧಪಟ್ಟ ನಿತ್ಯಶಾಸ್ತ್ರ. ಅದರ ಮೇಲೆ ಎಷ್ಟು ಗ್ರಂಥಗಳನ್ನು ಬರೆದರೂ ಅದೇನು ಹಳತಾಗುವುದಿಲ್ಲ. ಒಬ್ಬೊಬ್ಬನೂ ತನ್ನ ತನ್ನ ಸಂಸ್ಕಾರ ಮತ್ತು ವಿಕಾಸಕ್ಕೆ ತಕ್ಕಂತೆ ಗೀತೆಯನ್ನು ವಿವರಿಸಲು ಸಾಧ್ಯ. ಅಲ್ಲಿ ಎಲ್ಲರಿಗೂ ಸ್ಥಳವಿದೆ–ಹಿಂದಿನ ಭಾಷ್ಯಕಾರರಿಗೆ, ಈಗಿನ ವರಿಗೆ, ಮುಂದಿನವರಿಗೆ. ಏಕೆಂದರೆ ಅಲ್ಲಿರುವುದು ಅನಂತ ಸತ್ಯ, ಬಹುಮುಖದ ಸತ್ಯ. ಅದನ್ನು ಎಷ್ಟು ದೃಷ್ಟಿಕೋಣಗಳಿಂದ ಬೇಕಾದರೂ ನೋಡಬಹುದು. ಹಿಂದಿನಿಂದ ಸೂರ್ಯೋದಯ, ಅಸ್ತಮಯವನ್ನು ಎಷ್ಟು ಕವಿಗಳು ಬಣ್ಣಿಸಿರುವರು, ಚಿತ್ರಕಾರರು ಚಿತ್ರಿಸಿರುವರು. ಇಂದಿಗೂ ಅದನ್ನು ಬೇರೆ ಬೇರೆ ದೃಷ್ಟಿಕೋಣಗಳಿಂದ ಬಣ್ಣಿಸುವರು, ಚಿತ್ರಿಸುವರು. ಒಂದರಂತೆ ಮತ್ತೊಂದು ಇಲ್ಲ. ಪ್ರತಿಯೊಂದರಲ್ಲಿಯೂ ಒಂದು ನವ್ಯತೆಯನ್ನು ನೋಡುತ್ತೇವೆ. ಹಾಗೆಯೇ ಗೀತೆ. ಅದು ಎಲ್ಲರ ಹೃದಯನಾಡಿಯನ್ನು ಮಿಡಿಯುವುದು. ಮಿಡಿದಾಗ ಒಬ್ಬೊಬ್ಬನಿಂದ ಒಂದೊಂದು ಭಾವಧಾರೆ ಹೊರಹೊಮ್ಮುವುದು. ಸರಿಯಾಗಿ ನೋಡಿದರೆ ಒಂದು ಮತ್ತೊಂದಕ್ಕೆ ವಿರೋಧವಾಗಿಲ್ಲ; ಒಂದು ಮತ್ತೊಂದಕ್ಕೆ ಪೋಷಕವಾಗಿರುವುದು.

ನಾನು ಗೀತೆಯನ್ನು ಸಿದ್ಧಾಂತದ ದೃಷ್ಟಿಗಿಂತ ಹೆಚ್ಚಾಗಿ ಸಾಧಕನ ದೃಷ್ಟಿಯಿಂದ ನೋಡುತ್ತೇನೆ. ಸಾಧಕನಿಗೆ ಯಾವುಯಾವುದು ಪೂರ್ಣ ಸತ್ಯದೆಡೆಗೆ ಹೋಗುವುದಕ್ಕೆ ಅಡ್ಡಿಯಾಗಿದೆ, ಅವುಗಳಿಂದ ಪಾರಾಗುವುದು ಹೇಗೆ ಎಂಬುದನ್ನು ಶ್ರೀಕೃಷ್ಣ ಗೀತೆಯ ಬೋಧೆಯಲ್ಲೆಲ್ಲ ಬೇಕಾದಷ್ಟು ಹೇಳಿರುವನು. ಸಾಧಕನ ದೃಷ್ಟಿ ಎಲ್ಲಾ ಸಿದ್ಧಾಂತದವರಿಗೂ ಅನ್ವಯಿಸುವುದು. ಎಲ್ಲರೂ ಸಂಧಿಸುವ ಒಂದು ಸಾಮಾನ್ಯ ಭೂಮಿಕೆ ಎಂದರೆ ಇದೇ. ಯಾರು ಯಾವ ತತ್ತ್ವವನ್ನಾದರೂ ತೆಗೆದುಕೊಂಡಿರಲಿ, ಯಾವ ಯೋಗವನ್ನಾದರೂ ತೆಗೆದುಕೊಂಡಿರಲಿ, ಯಾವ ಭಾವದ ಮೂಲಕವಾಗಿ ಆದರೂ ಭಗವಂತನನ್ನು ಪ್ರೀತಿಸುತ್ತಿರಲಿ, ಅಲ್ಲಿ ಇರುವ ಸ್ಥಳದಲ್ಲೆ ಇರದೆ ಮುಂದುವರಿದು ಹೋಗಬೇಕಾದರೆ ಏನು ಮಾಡಬೇಕು ಎಂಬುದನ್ನು ಹೇಳುವುದು. ಭಗವದ್ಗೀತೆ ಎಲ್ಲಾ ಸಾಧಕರಿಗೂ ಒಂದು ಊರುಗೋಲು ಆಗಬಲ್ಲದು; ಅವರ ಕೈದೀವಿಗೆ ಆಗಬಲ್ಲದು.

ನಾನು ಇದನ್ನು ಬರೆಯುವಾಗ ಸಾಧಾರಣ ಕುಳಿತುಕೊಂಡು ಮಾತನಾಡುವ ಶೈಲಿಯ ಸಮೀಪ ದಲ್ಲೆ ಇದ್ದೇನೆ. ಶ್ರೋತೃಗಳಿಗೆ ಮಾತನಾಡುತ್ತಿರುವಾಗ ಹೇಗೆ ಹೇಳುತ್ತಿದ್ದೆನೊ, ಯಾವಯಾವ ಉದಾಹರಣೆಗಳನ್ನು ಕೊಡುತ್ತಿದ್ದೆನೊ, ಅದನ್ನೆಲ್ಲಾ ಇದರಲ್ಲಿ ಬಳಸಿಕೊಂಡಿರುವೆನು. ಗ್ರಂಥದ ಉದ್ದಕ್ಕೂ ಸಾಧ್ಯವಾದೆಡೆಯಲ್ಲೆಲ್ಲಾ ಶ್ರೀರಾಮಕೃಷ್ಣರ ಉಪದೇಶಗಳನ್ನು ಉದಹರಿಸಿದ್ದೇನೆ. ಶ್ರೀ ರಾಮಕೃಷ್ಣರು ಗೀತೆಯಂತೆ ಬಾಳಿ ಬದುಕಿದವರು. ಅವರ ಜೀವನ ಮತ್ತು ಉಪದೇಶವೇ ಗೀತೆಗೆ ಬರೆದ ಸಚೇತನ ಭಾಷ್ಯದಂತಿದೆ. ಶ್ರೀಕೃಷ್ಣನ ಪವಿತ್ರವದನದಿಂದ ಭವಜೀವಿಗಳ ಅಜ್ಞಾನಮೋಚನೆ ಗೆಂದು ಧರೆಗೆ ಇಳಿದುಬಂದ ಗಂಗೆ ಭಗವದ್ಗೀತೆ. ಅದನ್ನು ಹೇಳುವಾಗ ನಾನು ಆ ತೀರ್ಥದಲ್ಲಿ ಮೀಯುತ್ತಿದ್ದೆ, ಕೇಳುತ್ತಿದ್ದವರು ಮೀಯುತ್ತಿದ್ದರು. ಈಗ ಇದರಲ್ಲಿ ಓದುವವರೂ ಮೀಯುವರು. ಇದನ್ನು ಹೇಳುವುದು, ಕೇಳುವುದು, ಓದುವುದು ಎಲ್ಲವೂ ಒಂದು ಜ್ಞಾನತಪಸ್ಸು, ಒಂದು ಜ್ಞಾನ ಯಜ್ಞ, ಒಂದು ಪೂಜೆ. ಅದರಲ್ಲಿ ಎಲ್ಲರೂ ಭಾಗಿಗಳು.

\titleauthor{ಸೋಮನಾಥಾನಂದ}

