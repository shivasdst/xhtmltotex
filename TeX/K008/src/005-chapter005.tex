
\chapter{ಸಂನ್ಯಾಸಯೋಗ}

ಅರ್ಜುನ ಶ‍್ರೀಕೃಷ್ಣನನ್ನು ಹೀಗೆ ಕೇಳುತ್ತಾನೆ:

\begin{verse}
ಸಂನ್ಯಾಸಂ ಕರ್ಮಣಾಂ ಕೃಷ್ಣ ಪುನರ್ಯೋಗಂ ಚ ಶಂಸಸಿ~।\\ಯಚ್ಛ್ರೇಯ ಏತಯೋರೇಕಂ ತನ್ಮೇ ಬ್ರೂಹಿ ಸುನಿಶ್ಚಿತಮ್ \versenum{॥ ೧~॥}
\end{verse}

{\small ಶ‍್ರೀಕೃಷ್ಣ, ನೀನು ಕರ್ಮಸಂನ್ಯಾಸವನ್ನು ಮತ್ತು ಕರ್ಮಯೋಗವನ್ನು ಹೇಳುತ್ತೀಯೆ. ಇವೆರಡರಲ್ಲಿ ಯಾವುದು ಶ್ರೇಯಸ್ಕರವೊ ಅದನ್ನು ನಿಶ್ಚಯಿಸಿ ಹೇಳು.}

ಇಲ್ಲಿ ಅರ್ಜುನ ಕರ್ಮ ವಿಕರ್ಮ ಅಕರ್ಮ ಎಂಬುದನ್ನು ಚೆನ್ನಾಗಿ ಅರ್ಥ ಮಾಡಿಕೊಂಡಿಲ್ಲ. ಅದಕ್ಕಾಗಿಯೇ ಈ ಪ್ರಶ್ನೆಯನ್ನು ಹಾಕುತ್ತಿರುವನು. ಅರ್ಜುನನ ಮನಸ್ಸಿನಲ್ಲಿ ಕರ್ಮಸಂನ್ಯಾಸ ಮತ್ತು ಕರ್ಮಯೋಗ ಪರಸ್ಪರ ವಿರೋಧವಾಗಿ ಇರುವಂತೆ ಕಾಣುವುದು. ಯಾವುದನ್ನು ಮಾಡಬೇಕು ಯಾವುದನ್ನು ಬಿಡಬೇಕು ಎಂಬ ವಿಷಯದಲ್ಲಿ ದೊಡ್ಡ ಸಮಸ್ಯೆ ತಲೆದೋರಿದೆ. ಅವನಿಗೆ ಯಾವುದು ಶ್ರೇಯಸ್ಕರವೊ ಅದನ್ನು ನಿರ್ಧರಿಸುವ ಯೋಗ್ಯತೆ ಇಲ್ಲ. ಅನೇಕವೇಳೆ ನಾವು ಪ್ರೇಯಸ್ಸನ್ನು ತೆಗೆದುಕೊಳ್ಳುತ್ತೇವೆ. ಶ್ರೇಯಸ್ಸನ್ನು ನಿರಾಕರಿಸುತ್ತೇವೆ. ನಮಗೆ ಬೇಕಾಗಿರುವುದು ಕಹಿಯಲ್ಲ, ಸಿಹಿ. ಮುಂದೇನು ಹಿತವಾಗಬಹುದು ಎಂಬುದಲ್ಲ. ತತ್ಕಾಲದಲ್ಲಿ ನಮಗೆ ಹಿತವಾಗಿ ಕಾಣಬೇಕು. ಸಾಧಾರಣ ಮನುಷ್ಯನ ದೃಷ್ಟಿ ಇದು. ಆದರೆ ಅರ್ಜುನನಿಗೆ ಈ ದೃಷ್ಟಿಯಲ್ಲಿ ಅಪಾಯವಿದೆ ಎಂಬುದು ಚೆನ್ನಾಗಿ ಗೊತ್ತು. ಆದಕಾರಣವೇ ತನಗೆ ಯಾವುದು ಒಳ್ಳೆಯದೆಂದು ಶ‍್ರೀಕೃಷ್ಣ ಭಾವಿಸುತ್ತಾನೆಯೋ ಅದನ್ನು ನಿಷ್ಕರ್ಷಿಸಿ ಹೇಳು ಎಂದು ಜವಾಬ್ದಾರಿಯನ್ನು ಅವನಿಗೆ ಬಿಡುತ್ತಾನೆ. ರೋಗಿಗೆ ಗೊತ್ತಾಗುವು ದಿಲ್ಲ ಯಾವುದು ಒಳ್ಳೆಯ ಔಷಧಿ, ಯಾವುದು ಕೆಟ್ಟ ಔಷಧಿ ಎಂಬುದು. ವೈದ್ಯನಿಗೆ ಮಾತ್ರ ಅದು ಗೊತ್ತು. ರೋಗಿ ತನಗೆ ಏನು ಬೇಕು ಅದನ್ನು ವೈದ್ಯನಿಗೆ ಹೇಳದೆ, ವೈದ್ಯನಿಗೆ ಚಿಕಿತ್ಸೆಯ ಸ್ವಾತಂತ್ರ್ಯ ಕೊಟ್ಟರೆ ಬೇಗ ಮೇಲಾಗಬಹುದು. ಅದಕ್ಕೆ ಶ‍್ರೀಕೃಷ್ಣ ಹೀಗೆ ಹೇಳುತ್ತಾನೆ.

\begin{verse}
ಸಂನ್ಯಾಸಃ ಕರ್ಮಯೋಗಶ್ಚ ನಿಃಶ್ರೇಯಸಕರಾವುಭೌ~।\\ತಯೋಸ್ತು ಕರ್ಮಸಂನ್ಯಾಸಾತ್ ಕರ್ಮಯೋಗೋ ವಿಶಿಷ್ಯತೇ \versenum{॥ ೨~॥}
\end{verse}

{\small ಕರ್ಮತ್ಯಾಗ ಕರ್ಮಯೋಗ ಎರಡೂ ಒಬ್ಬನಿಗೆ ಮುಕ್ತಿಯನ್ನು ಕೊಡುವುವು. ಆದರೆ ಕರ್ಮತ್ಯಾಗಕ್ಕಿಂತ ಕರ್ಮಯೋಗವೇ ಶ್ರೇಷ್ಠ.}

ಕರ್ಮತ್ಯಾಗ ಮತ್ತು ಕರ್ಮಯೋಗ ಇವೆರಡರ ಗುರಿ ಒಂದೇ. ಮಾರ್ಗಗಳು ಮಾತ್ರ ಬೇರೆಬೇರೆ. ಚಾಮುಂಡಿ ಬೆಟ್ಟಕ್ಕೆ ಮೆಟ್ಟಲಿನ ಮೂಲಕ ಬೇಕಾದರೆ ಹತ್ತಿಕೊಂಡು ಹೋಗಬಹುದು. ಅದಕ್ಕೆ ಶಕ್ತಿ ಇಲ್ಲದೆ ಇದ್ದರೆ ಬಸ್ಸಿನಲ್ಲಿ ಬೇಕಾದರೂ ಹೋಗಿ ಬೆಟ್ಟದ ಮೇಲಿರುವ ದೇವಸ್ಥಾನದ ಬಾಗಿಲಿನ ಮುಂದೆಯೇ ಇಳಿಯಬಹುದು. ಬೆಟ್ಟ ಹತ್ತಿ ಹೋಗಬೇಕಾದರೆ ಶರೀರದಾರ್ಢ್ಯ ಬಲವಾಗಿರಬೇಕು, ಹೃದಯ ಬಲವಾಗಿರಬೇಕು. ಒಂದು ಕಷ್ಟದ ಹಾದಿ ಮತ್ತೊಂದು ಸುಲಭದ ಹಾದಿ. ಆದರೆ ಎರಡರ ಉದ್ದೇಶವೂ ಒಂದೇ.

ಕರ್ಮತ್ಯಾಗಿ ಸಂನ್ಯಾಸಿ. ಅವನು ಅತ್ಯಾಶ್ರಮಿ. ಸಮಾಜಕ್ಕೆ ದೇಶಕ್ಕೆ ಗೃಹಕ್ಕೆ ಸಂಬಂಧಪಟ್ಟ ಕೆಲಸಗಳನ್ನೆಲ್ಲ ಮೊದಲೆ ಬಿಟ್ಟುಬಿಟ್ಟಿರುತ್ತಾನೆ. ಅವನು ಯಾವುದಕ್ಕೂ ಬದ್ಧನಲ್ಲ. ಅವನು ಯಾರಿಂ ದಲೂ ಏನನ್ನೂ ತೆಗೆದುಕೊಳ್ಳುವುದಿಲ್ಲ ಮತ್ತು ಅವನು ಇತರರಿಗೆ ಏನಾದರೂ ಕೊಡಬೇಕಾಗಿದೆ ಎಂದು ಯಾರೂ ಅವನನ್ನು ಬಲಾತ್ಕರಿಸಲಾರರು. ಸಾಲ ತೆಗೆದುಕೊಂಡರೆ ಸಾಲ ತೀರಿಸುವ ಜವಾಬ್ದಾರಿ. ಆದರೆ ಯಾರು ಸಾಲಕ್ಕೆ ಕೈಒಡ್ಡುವುದೇ ಇಲ್ಲವೋ ಅವನು ಯಾರಿಗೆ ತೀರಿಸಬೇಕು? ಆದರೆ ಅವನಿಗೆ ಮನಸ್ಸು ಬಂದರೆ ತನ್ನಲ್ಲಿರುವ ಜ್ಞಾನವನ್ನು ಬೇಕಾದರೆ ಇತರರಿಗೆ ದಾನ ಮಾಡು ತ್ತಾನೆ. ಅವನೇ ಸ್ವೇಚ್ಛೆಯಿಂದ ಮಾಡುತ್ತಾನೆ. ಅವನಿಗೆ ಮಾಡಲೇಬೇಕು ಎಂಬ ಯಾವ ನಿಯಮವೂ ಇಲ್ಲ. ಮಾಡದೇ ಇದ್ದರೆ ಯಾರೂ ಅವನನ್ನು ನಿರ್ಬಂಧಿಸುವ ಹಾಗಿಲ್ಲ.

ಕರ್ಮಯೋಗಿ ಹಾಗಲ್ಲ. ಅವನು ಸಮಾಜದಲ್ಲಿರುವನು. ಅವನಿಗೊಂದು ಮನೆ ಇದೆ, ಅವನಿ ಗೊಂದು ಸಮಾಜವಿದೆ, ಅವನಿಗೊಂದು ದೇಶವಿದೆ. ಇದಕ್ಕೆ ಸಂಬಂಧವುಳ್ಳ ಕರ್ತವ್ಯಗಳೆಲ್ಲ ಅವನ ಪಾಲಿಗೆ ಇವೆ. ಅವನು ಇವುಗಳಿಂದ ಕೈತೊಳೆದುಕೊಂಡಿಲ್ಲ. ಇವುಗಳಿಂದ ಅವನು ಸಾಲ ತೆಗೆದು ಕೊಂಡು ತನ್ನ ಬಾಳನ್ನು ಕಟ್ಟಿಕೊಂಡಿರುವನು. ಅವನು ಈಗ ಅದನ್ನು ಕೊಡಬೇಕಾಗಿದೆ. ಕೊಡದೆ ಇದ್ದರೆ ಬಿಡುವುದಿಲ್ಲ. ಅವನು ಬೇಕಾದರೆ ಗೊಣಗಾಡುತ್ತ ಕೊಡಬಹುದು, ಅಥವಾ ಬೇಕಾದರೆ ಹಸನ್ಮುಖದಿಂದ ಸಧ್ಯಕ್ಕೆ ಸಾಲ ತೀರುತ್ತದಲ್ಲ ಸಾಕಪ್ಪ ಎಂಬ ದೃಷ್ಟಿಯಿಂದ ಕೊಡಬಹುದು. ಅವನು ಬೇಕಾದರೆ ಲಾಭಕ್ಕೆ, ಕೀರ್ತಿಗೆ, ಅಧಿಕಾರಕ್ಕೆ ಕೆಲಸವನ್ನು ಮಾಡಬಹುದು ಮತ್ತು ಅದರಿಂದ ಬರುವ ಪರಿಣಾಮಗಳಿಗೆ ಬದ್ಧನಾಗಬಹುದು. ಇಲ್ಲದೇ ಇದ್ದರೆ ಫಲಾಪೇಕ್ಷೆಯನ್ನು ಬಿಟ್ಟು ಯಜ್ಞದೃಷ್ಟಿಯಿಂದ ತನ್ನ ಪಾಲಿನ ಕರ್ತವ್ಯವನ್ನು ಸಮತ್ವದಿಂದ ಮಾಡಿ ಹಾಕಬಹುದು. ಹೀಗೆ ಮಾಡಿದರೆ ಅವನು ಕರ್ಮದಿಂದ ಬದ್ಧನಾಗುವುದಿಲ್ಲ. ಮಾಡುವ ಕರ್ಮವೆ ಅವನ ಕರ್ಮಪಾಶವನ್ನು ನಿರ್ಮೂಲ ಮಾಡುವುದು. ಒಂದು ರೀತಿ ಕರ್ಮ ಮಾಡಿದರೆ ಅದು ನಮ್ಮನ್ನು ಕಟ್ಟಿಹಾಕುವುದು. ಒಂದು ದಾರವನ್ನು ಉಂಡೆಗೆ ಸುತ್ತುವಂತೆ ಇದು. ಇನ್ನೊಂದು ರೀತಿ ಕೆಲಸ ಮಾಡಿದರೆ ಅದು ಬಿಚ್ಚಿಕೊಳ್ಳು ವುದು. ಉಂಡೆಯ ದಾರವನ್ನು ಬಿಚ್ಚುವಂತೆ ಇದು. ಸುತ್ತುವುದೂ ಕರ್ಮವೇ, ಬಿಚ್ಚುವುದೂ ಕರ್ಮವೇ. ಕರ್ಮಯೋಗಿ ಬಿಚ್ಚುವ ರೀತಿಯಲ್ಲಿ ಕರ್ಮ ಮಾಡುತ್ತಾನೆ. ಫಲಾಪೇಕ್ಷೆಯಿಂದ ಮಾಡು ವುದು ಸುತ್ತುವ ರೀತಿಯಲ್ಲಿ ಕರ್ಮವನ್ನು ಮಾಡಿದಂತೆ.

ಕರ್ಮವನ್ನು ಬಿಡುವುದಕ್ಕಿಂತ ಕರ್ಮವನ್ನು ಅನಾಸಕ್ತಿಯಿಂದ ಮಾಡುವುದು ಸುಲಭ ಎನ್ನುತ್ತಾನೆ. ನಮ್ಮಲ್ಲಿ ರಜೋಗುಣ ತುಂಬಿ ತುಳುಕಾಡುತ್ತಿದೆ. ಕರ್ಮದಿಂದ ಆ ಗುಣವನ್ನು ಕಾಯಿಸಬೇಕು. ಅನಂತರ ಅದು ಸತ್ತ್ವಗುಣಕ್ಕೆ ತಿರುಗಬೇಕಾದರೆ. ಇದು ಬೆಣ್ಣೆಗೂ ತುಪ್ಪಕ್ಕೂ ಇರುವ ವ್ಯತ್ಯಾಸದಂತೆ. ಬೆಣ್ಣೆಯಲ್ಲಿ ನೀರಿನ ಅಂಶ ಇದೆ. ಅದನ್ನು ಆಚೆಗೆ ತೆಗೆಯಬೇಕಾದರೆ ಬೆಂಕಿಯ ಮೇಲೆ ಇಟ್ಟು ಕಾಯಿಸಬೇಕು. ಆಗಲೆ ಶಬ್ದಮಾಡಿಕೊಂಡು ಕಾಯಲು ಮೊದಲಾಗುವುದು. ನೀರೆಲ್ಲ ಹೋದಮೇಲೆ ತೆಪ್ಪಗಾಗಿ ಮೌನವಾಗುವುದು. ಅದೇ ತುಪ್ಪ, ಸತ್ತ್ವಗುಣ. ಬೆಣ್ಣೆ ಬೆಂಕಿಯ ಮೇಲೆ ಕಾದಲ್ಲದೆ ಅದರಲ್ಲಿರುವ ನೀರು ಹೋಗುವುದಿಲ್ಲ. ಅದರಂತೆಯೇ ರಜೋಗುಣ ಇರುವ ಮನುಷ್ಯ ಕರ್ಮ ಮಾಡಬೇಕು. ಅನಂತರವೇ ಅವನಲ್ಲಿರುವ ರಜೋಗುಣ ಕ್ರಮೇಣ ಕಡಮೆಯಾಗುತ್ತ ಬಂದಂತೆ ಸತ್ತ್ವಗುಣಕ್ಕೆ ತಿರುಗುತ್ತಾನೆ.

ಕರ್ಮ ಮಾಡುವುದು ಸುಲಭದ ಮಾರ್ಗ. ನಾವು ಅನೇಕವೇಳೆ ಭಾವಿಸಬಹುದು, ಕರ್ಮ ಮಾಡದೇ ಇರುವುದು ಸುಲಭ ಎಂದು. ಕರ್ಮಕ್ಕೆ ಕಷ್ಟಪಡಬೇಕು. ಕರ್ಮಮಾಡದೆ ಇರುವುದಕ್ಕೆ ಕಷ್ಟವೇನೂ ಪಡಬೇಕಾಗಿಲ್ಲವಲ್ಲ ಎನ್ನಬಹುದು. ನಮ್ಮ ಮನಸ್ಸಿನ ಪ್ರವೃತ್ತಿ ಬಾಹ್ಯಮುಖವಾಗಿ ಇರುವಾಗ, ನಾವು ತೆಪ್ಪಗೆ ಕುಳಿತುಕೊಂಡರೂ, ಬಹಳಕಾಲ ಹಾಗೆ ಕೂಡಿರಿಸದು. ಕೆಲವು ಕಾಲವಾದ ಮೇಲೆ ನಮ್ಮನ್ನು ಕರ್ಮಕ್ಕೆ ನೂಕುತ್ತದೆ. ಆಗ ನಿರ್ವಾಹವಿಲ್ಲದೆ ಕರ್ಮ ಮಾಡುವುದನ್ನು ಕಲಿಯಬೇಕಾಗುವುದು. ಇದನ್ನು ಅರಿತು ಮುಂಚೆಯೇ ಕರ್ಮ ಮಾಡುವುದನ್ನು ಕಲಿಯುವುದು ಎಷ್ಟೋ ಮೇಲು.

ನಾವು ಯಾವಾಗಲೂ ನಮ್ಮ ಪ್ರಕೃತಿಗೆ ವಿರೋಧವಾಗಿ ಹೋಗಲು ಆಗುವುದಿಲ್ಲ. ಪ್ರಕೃತಿಗೆ ಅನುಸಾರವಾಗಿ ಹೋಗುವಾಗ ಅಲ್ಲಿ ಘರ್ಷಣೆ ಇರುವುದಿಲ್ಲ, ಬಲಾತ್ಕಾರ ಇರುವುದಿಲ್ಲ. ಸ್ವಲ್ಪ ಕಾಲ ಹೆಚ್ಚು ತೆಗೆದುಕೊಳ್ಳಬಹುದು ಗುರಿ ಸೇರುವುದಕ್ಕೆ. ಆದರೇನಂತೆ, ಭೇದಿಸಲಾಗದ ಆತಂಕದೊಡನೆ ಹೋರಾಡಿ ಕಾಲುಮುರಿದು ಕುಳಿತುಕೊಳ್ಳುವುದಕ್ಕಿಂತ, ಸ್ವಲ್ಪ ನಿಧಾನವಾಗಿ ಬಳಸಿಕೊಂಡು ಗುರಿ ಯನ್ನು ಸೇರುವುದು ಮೇಲು. ನದಿ ಹರಿಯುವಾಗ ಈ ಮಾರ್ಗವನ್ನು ಅನುಸರಿಸುವುದು. ತನ್ನ ಎದುರಿಗೆ ಇರುವ ಆತಂಕವನ್ನು ತೂರಿಕೊಂಡು ಹೋಗುವುದಿಲ್ಲ. ಅದನ್ನು ಬಳಸಿಕೊಂಡು ಹೋಗು ವುದು. ಹೀಗೆಯೇ ನಮ್ಮ ಪ್ರಕೃತಿಗೆ ಅನುಸಾರವಾಗಿ ಹೋಗುವುದು ಸುಲಭ ಎನ್ನುವನು ಶ‍್ರೀಕೃಷ್ಣ.

\begin{verse}
ಜ್ಞೇಯಃ ಸ ನಿತ್ಯಸಂನ್ಯಾಸೀ ಯೋ ನ ದ್ವೇಷ್ಟಿ ನ ಕಾಂಕ್ಷತಿ~।\\ನಿರ್ದ್ವಂದ್ವೋ ಹಿ ಮಹಾಬಾಹೋ ಸುಖಂ ಬಂಧಾತ್ ಪ್ರಮುಚ್ಯತೇ \versenum{॥ ೩~॥}
\end{verse}

{\small ಯಾರು ದ್ವೇಷಿಸುವುದಿಲ್ಲವೋ ಅವನು ನಿತ್ಯಸಂನ್ಯಾಸಿ ಎಂದು ತಿಳಿ. ಮಹಾಬಾಹುವೆ, ಏಕೆಂದರೆ ದ್ವಂದ್ವವಿಲ್ಲ ದವನು ಸುಖವಾಗಿ ಬಂಧನದಿಂದ ಪಾರಾಗುತ್ತಾನೆ.}

ಶ‍್ರೀಕೃಷ್ಣ ಇಲ್ಲಿ ಕೆಲಸ ಬಿಡುವುದೇ ಆಗಲೀ ಮಾಡುವುದೇ ಆಗಲೀ ಅದಲ್ಲ ಮುಖ್ಯ; ಅದರ ಹಿಂದೆ ಇರುವ ಉದ್ದೇಶಗಳು ನಮ್ಮ ಬಂಧನ-ಮೋಕ್ಷಕ್ಕೆ ಕಾರಣ ಎಂಬುದನ್ನು ಚೆನ್ನಾಗಿ ವಿವರಿಸು ವನು. ಒಬ್ಬ ಕರ್ಮ ಮಾಡುತ್ತಿರಬಹುದು. ಆದರೆ ಅದರ ಹಿಂದೆ ಆಸಕ್ತಿ ಇಲ್ಲದೇ ಇದ್ದರೆ, ಕರ್ಮದಿಂದ ಬರುವ ಫಲ ಇವನನ್ನು ಕಟ್ಟಿಹಾಕಲಾರವು. ಮತ್ತೊಬ್ಬ ಕರ್ಮವನ್ನು ಮಾಡದೇ ಇರಬಹುದು. ಆದರೂ ಬೇಕಾದಷ್ಟು ಆಸೆಯ ಬುದ್ಬುದಗಳು ಮನಸ್ಸಿನಲ್ಲಿ ಏಳುತ್ತಿರಬಹುದು. ಕೆಲಸ ಮಾಡದೇ ಇದ್ದರೂ ಅವನ ಮನಸ್ಸೆಲ್ಲ ಅಲ್ಲೋಲಕಲ್ಲೋಲವಾಗುವುದು. ಮತ್ತೊಬ್ಬ ಕೆಲಸ ಮಾಡಿಯೂ ಅದರ ಬಂಧನಕ್ಕೆ ಬೀಳದೆ, ಕೆಲಸ ಮಾಡದವನಿಗಿಂತ ಮೇಲಾಗುವನು. ಕೆಲಸ ಮಾಡದೆ ಇರುವವನಲ್ಲಿ ಆಸೆಗಳು ಇದೆಯೆ ಇಲ್ಲವೆ ಅದನ್ನು ನಾವು ಗಮನಿಸಬೇಕಾಗಿದೆ.

ಇಲ್ಲಿ ನಿತ್ಯಸಂನ್ಯಾಸಿ ಎಂಬುದಕ್ಕೆ ಶ‍್ರೀಕೃಷ್ಣ ಒಂದು ಶ್ರೇಷ್ಠವಾದ ಅರ್ಥವನ್ನು ಕೊಡುವನು. ಹೇಗೆ ಯಜ್ಞ ಎಂಬುದಕ್ಕೆ ಶ‍್ರೀಕೃಷ್ಣ ಒಂದು ಹೊಸ ಅರ್ಥವನ್ನು ಕೊಡುವನೊ, ಪ್ರತಿಯೊಂದು ಕೆಲಸವನ್ನೂ ಯಜ್ಞರೂಪಕ್ಕೆ ಏರಿಸಲು ಸಾಧ್ಯ ಎಂಬುದನ್ನು ಹೇಳುವನೊ, ಅದರಂತೆಯೆ ಅವನ ಸಂನ್ಯಾಸ ಎಂಬ ಪದ. ಮನೆ ಹೆಂಡತಿ ಮಕ್ಕಳು ತೊರೆದು, ಬಟ್ಟೆ ಮತ್ತು ಹೆಸರನ್ನು ಬದಲಾಯಿಸಿಕೊಂಡು ದಂಡ ಕಮಂಡಲು ಹಿಡಿದುಕೊಂಡು ಪರಿವ್ರಾಜಕನಾಗಿ ಅಲೆಯುವವನನ್ನು ಸಂನ್ಯಾಸಿ ಎಂದು ಕರೆಯುತ್ತೇವೆ. ಶ‍್ರೀಕೃಷ್ಣ ವೇಷಕ್ಕೆ ಅಷ್ಟು ಬೆಲೆಯನ್ನೇ ಕೊಡುವುದಿಲ್ಲ, ವೇಷದ ಹಿಂದೆ ಅವನ ಮನಸ್ಸು ಯಾವ ಬಣ್ಣದ್ದು ಎಂಬುದರ ಆಧಾರದ ಮೇಲೆ ಒಬ್ಬ ಸಂನ್ಯಾಸಿಯೆ ಅಲ್ಲವೆ ಎಂಬುದನ್ನು ನಿಷ್ಕರ್ಷಿಸ ಬೇಕಾಗುವುದು. ಯಾರು ಸಂನ್ಯಾಸಿಯಾಗಿರುವನೊ ಅವನು ಯಾರನ್ನೂ ದ್ವೇಷಿಸುವುದಿಲ್ಲ. ದ್ವೇಷ ನಮ್ಮ ಮನಸ್​ಶಕ್ತಿಯನ್ನು ವ್ರಯಮಾಡುವ ಬಿಲ. ಮುಂಚೆ ಅದನ್ನು ನಾವು ಮುಚ್ಚಬೇಕು. ಯಾವಾಗ ಒಂದನ್ನು ದ್ವೇಷಿಸುವೆವೊ ಆಗ ಯಾವಾಗಲೂ ಅದನ್ನೇ ಕುರಿತು ಚಿಂತಿಸುತ್ತ ಇರಬೇಕಾಗುವುದು. ಅದರಿಂದ ತಪ್ಪಿಸಿಕೊಂಡು ಬರುವುದಕ್ಕೆ ಆಗುವುದಿಲ್ಲ. ಸಂನ್ಯಾಸಿ ದ್ವೇಷ ಮಾಡುವುದಿಲ್ಲ. ಅದನ್ನು ಉದಾಸೀನಭಾವದಿಂದ ನೋಡುವನು. ಅದರಂತೆಯೇ ಅವನು ಯಾವ ವಸ್ತುವನ್ನೂ ಬಯಸುವು ದಿಲ್ಲ. ಬಯಕೆ ಕೂಡ ನಮ್ಮನ್ನು ಬಯಸುವ ವಸ್ತುವಿಗೆ ಕಟ್ಟಿಹಾಕುವುದು. ಯಾವಾಗಲೂ ಅದನ್ನೇ ಕುರಿತು ಚಿಂತಿಸಬೇಕಾಗುವುದು. ಆ ಬಯಕೆಯಾದರೊ ಆ ವಸ್ತುವನ್ನು ಕೊಟ್ಟರೆ ಆರಿಹೋಗುವ ಉರಿಯಲ್ಲ. ನಾವು ತೃಪ್ತಿಪಡಿಸುವುದಕ್ಕೆ ಯತ್ನಿಸುವುದೆಲ್ಲ ಉರಿಗೆ ಒಂದೆರಡು ಸೌದೆಯನ್ನು ಜಾಸ್ತಿ ಹಾಕಿದಂತೆ. ಇದು ನಮ್ಮ ಸಂಸ್ಕಾರಗಳನ್ನೆಲ್ಲ ಮತ್ತೂ ಹೆಚ್ಚಿಸುವುದು. ಯಾರು ಅಪ್ರಿಯವಾಗಿರು ವುದು ಬಂದರೆ ಚಿಂತಿಸುವುದಿಲ್ಲವೊ, ಪ್ರಿಯವಾಗಿರುವುದನ್ನು ಕುರಿತು ಮೆಲಕುಹಾಕುವುದಿಲ್ಲವೊ ಅವನೇ ನಿತ್ಯಸಂನ್ಯಾಸಿ.

ಯಾರಿಗೆ ದ್ವಂದ್ವಗಳಿಲ್ಲವೊ ಅವನು ಸುಖವಾಗಿ ಸಂಸಾರದಿಂದ ಪಾರಾಗುತ್ತಾನೆ. ನಮ್ಮನ್ನು ಪ್ರಪಂಚಕ್ಕೆ ಬಿಗಿದಿರುವುದೆ ಈ ದ್ವಂದ್ವ ಅನುಭವಗಳು. ಒಂದು ಬೇಕು, ಮತ್ತೊಂದು ಬೇಡ. ಪ್ರಿಯವಾಗಿರುವುದು ಬೇಕು. ಅದು ನನ್ನ ಹತ್ತಿರ ಇದ್ದರೆ ಬಿಟ್ಟುಹೋಗಕೂಡದು. ಜೀವನದಲ್ಲಿ ಪ್ರಿಯ ಅಪ್ರಿಯಗಳೆರಡೂ ಒಂದೇ ನಾಣ್ಯದ ಎರಡು ಭಾಗಗಳು ಎಂಬುದನ್ನು ಕಲಿಯಲು ಬಹಳ ಅನುಭವ ಬೇಕು. ನಾವು ಒಂದಕ್ಕೆ ಕೈ ಒಡ್ಡಿದರೆ ಇನ್ನೊಂದಕ್ಕೂ ಕೈಯೊಡ್ಡಬೇಕು. ಸುಖವಾಗಿರುವುದನ್ನು ಚಪ್ಪರಿಸಿ ಅಪ್ರಿಯವಾಗಿರುವುದು ಬಂದಾಗ ಅದನ್ನು ಆಚೆಗೆ ಬಿಸಾಡುವುದಕ್ಕೆ ಆಗುವುದಿಲ್ಲ. ನಾವೇನೊ ಬಿಸಾಡಲು ಯತ್ನಿಸುತ್ತೇವೆ. ಆದರೆ ಪ್ರಕೃತಿ ನಮ್ಮ ಗಂಟಲಿನ ಒಳಗೆ ತುರುಕುವುದು. ಅಪ್ರಿಯ ಬೇಡವಾದರೆ, ಅದರ ಮತ್ತರ್ಧ ಭಾಗವಾದ ಪ್ರಿಯವನ್ನೂ ಬಿಡಬೇಕಾಗುವುದು. ಸಂನ್ಯಾಸಿ ಯಾದವನು ಸಂಸಾರವನ್ನು ಬಿಟ್ಟರೂ ಈ ಪ್ರಪಂಚ ಅವನಿಗೆ ದ್ವಂದ್ವ ಅನುಭವಗಳನ್ನು ಕೊಡುವು ದನ್ನು ಬಿಡುವುದೆ? ಅದು ಕೊಟ್ಟೇ ಕೊಡುವುದು. ಕೆಲವರು ಅವನನ್ನು ಟೀಕಿಸುವರು, ಕೆಲವರು ಅವನನ್ನು ಹೊಗಳುವರು. ಅವನಿಗೂ ಸನ್ಮಾನ ಮಾಡುವವರು ಕೆಲವರು, ಅಗೌರವ ತೋರಿಸುವವರು ಕೆಲವರು. ಆದರೆ ಸಂನ್ಯಾಸಿ ಅದನ್ನು ಗಮನಿಸುವುದಿಲ್ಲ. ಅದರ ಮೇಲೆ ಮನಸ್ಸಿಟ್ಟರೆ ಅದು ತನ್ನ ಪ್ರಭಾವವನ್ನು ಬೀರುವುದು. ಅದನ್ನು ಗಮನಿಸದೇ ಇದ್ದರೆ ಅದು ಬರುವುದು, ಹೋಗುವುದು. ಸೇತುವೆಯ ಒಂದು ಕಡೆಯಿಂದ ನೀರು ಬರುವುದು, ಮತ್ತೊಂದು ಕಡೆಯಿಂದ ಹರಿದುಹೋಗು ವುದು.

\begin{verse}
ಸಾಂಖ್ಯಯೋಗೌ ಪೃಥಗ್ಬಾಲಾಃ ಪ್ರವದಂತಿ ನ ಪಂಡಿತಾಃ~।\\ಏಕಮಪ್ಯಾಸ್ಥಿತಃ ಸಮ್ಯಗುಭಯೋರ್ವಿಂದತೇ ಫಲಮ್ \versenum{॥ ೪~॥}
\end{verse}

{\small ಸಾಂಖ್ಯಯೋಗ ಮತ್ತು ಕರ್ಮಯೋಗ ಬೇರೆ ಬೇರೆ ಎಂದು ಅಜ್ಞಾನಿಗಳು ಹೇಳುತ್ತಾರೆ. ಪಂಡಿತರು ಹಾಗೆ ಹೇಳುವುದಿಲ್ಲ. ಇವುಗಳಲ್ಲಿ ಯಾವುದಾದರೂ ಒಂದನ್ನು ಚೆನ್ನಾಗಿ ಅನುಷ್ಠಾನ ಮಾಡಿದರೂ ಎರಡರ ಫಲವನ್ನೂ ಪಡೆಯುತ್ತಾನೆ.}

ಇಲ್ಲಿ ಸಾಂಖ್ಯಯೋಗ ಎಂದರೆ ಜ್ಞಾನಯೋಗ. ಜ್ಞಾನಯೋಗ ಬೇರೆ, ಕರ್ಮಯೋಗ ಬೇರೆ, ಎಂದರೆ ಬೇರೆ ಬೇರೆ ಗುರಿಯೆಡೆಗೆ ಒಯ್ಯುವುವು ಎಂದು ಭಾವಿಸಿದರೆ ತಪ್ಪು. ಇವೆರಡೂ ಬೇರೆ ಬೇರೆ ದಾರಿಗಳು ನಿಜ. ಆದರೆ ಎರಡರ ಗುರಿಯೂ ಒಂದೇ. ಜ್ಞಾನಿಯ ಮಾರ್ಗ ವಿಚಾರದ ಮಾರ್ಗ. ವಿಚಾರವೂ ಕೂಡ ಒಂದು ವಿಧವಾದ ಕರ್ಮವೇ. ಅವನು ನಿತ್ಯ ಅನಿತ್ಯವನ್ನು ಯಾವಾಗಲೂ ವಿಮರ್ಶಿಸುತ್ತಿರಬೇಕು. ಇದು ಕೈಕಾಲುಗಳ ಮೂಲಕ ಎಲ್ಲರಿಗೂ ಕಾಣುವಂತೆ ಮಾಡುವ ಕೆಲಸವಲ್ಲ. ಬುದ್ಧಿಯ ಮೂಲಕ ಮಾಡುವುದು, ಸೂಕ್ಷ್ಮಕರ್ಮ. ಸತ್ಯವನ್ನು ಕಾಣದಂತೆ ಮಾಡಿರುವ ಮಿಥ್ಯದ ಉಪಾಧಿಗಳನ್ನು ತೆಗೆದುಹಾಕಿದರೆ ಸತ್ಯ ಗೋಚರಿಸುವುದು. ಹೀಗೆ ಮಾಡುವುದು ಜ್ಞಾನಮಾರ್ಗ. ಕರ್ಮಯೋಗ ಫಲಾಪೇಕ್ಷೆ ಇಲ್ಲದೆ ಕೆಲಸವನ್ನು ಮಾಡುತ್ತ ಹೋದಂತೆ ನಮ್ಮ ಚಿತ್ತವನ್ನು ಶುದ್ಧಿಮಾಡುವುದು. ಚಿತ್ತಶುದ್ಧಿಯಾದರೆ ಆಗಲೇ ಎಲ್ಲದರಲ್ಲಿಯೂ ಇರುವ ಸ್ವಯಂಜ್ಯೋತಿ ಸ್ವರೂಪನಾದ ಪರಮಾತ್ಮನ ಸಾಕ್ಷಾತ್ಕಾರವಾಗುವುದು.

ಎರಡೂ ಬೇರೆ ಬೇರೆ, ಅಥವಾ ಒಂದು ಮೇಲು ಮತ್ತೊಂದು ಕೀಳು ಎನ್ನುವವರು ಹೋಲಿಸು ವಾಗ ಈ ತಪ್ಪನ್ನು ಮಾಡುವರು. ಉತ್ತಮ ಜ್ಞಾನವನ್ನು ಫಲಾಪೇಕ್ಷೆಯಿಂದ ಮಾಡುವ ಕರ್ಮ ದೊಂದಿಗೆ ಹೋಲಿಸಿ, ಜ್ಞಾನವೆ ಶ್ರೇಷ್ಠವೆನ್ನುವರು. ಹಾಗೆಯೇ ಉತ್ತಮ ಕರ್ಮಯೋಗದ ದೃಷ್ಟಿ ಯಿಂದ ಮಾಡಿದ ಕರ್ಮ ಮತ್ತು ಅದರಿಂದ ಮನಸ್ಸಿನ ಮೇಲೆ ಆದ ಪರಿಣಾಮವನ್ನು, ಅನುಷ್ಠಾನಕ್ಕೆ ತರದ ಬರೀ ಸಿದ್ಧಾಂತಗಳನ್ನು ಓದುತ್ತಿರುವ ಪಂಡಿತನೊಂದಿಗೆ ಹೋಲಿಸಿ ಕರ್ಮಯೋಗ ಮೇಲೆನ್ನು ವರು. ನಾವು ಹೋಲಿಸಬೇಕಾದರೆ ಎರಡು ಶ್ರೇಷ್ಠವನ್ನೂ ತೆಗೆದುಕೊಳ್ಳಬೇಕು. ಅಲ್ಲಿ ಒಂದು ಮೇಲಲ್ಲ, ಮತ್ತೊಂದು ಕೀಳಲ್ಲ. ಒಂದರಷ್ಟೇ ಮತ್ತೊಂದು ಶ್ರೇಷ್ಠ. ಮಿಥ್ಯಾಮೋಹದ ಆಕರ್ಷಣೆಗೆ ಬೀಳದಿರುವುದು ದೊಡ್ಡ ಸಾಹಸ. ಜ್ಞಾನಿ ಇದನ್ನು ವಿಚಾರದ ಮೂಲಕ ಮಾಡುವನು. ಕರ್ಮಕುಶಲಿ ಕೆಲಸ ಮಾಡಿದರೂ ಅದರ ಗೋಜಿಗೆ ಸಿಕ್ಕದ ರೀತಿ ಮಾಡುವನು. ಇದೇನು ಜ್ಞಾನಿಯ ಸಾಹಸಕ್ಕಿಂತ ಕಡಮೆಯಲ್ಲ. ಯಾರಾದರೂ ಒಂದರಲ್ಲಿ ಮುಂದುವರಿದರೆ ಅವರಿಗೆ ಮತ್ತೊಂದೂ ಬರುವುದು. ಜ್ಞಾನ ಮತ್ತು ಕರ್ಮ ಎಂಬುದು ಕತ್ತಲೆ ಬೆಳಕಿನಂತಲ್ಲ. ಒಂದು ಮುಂದೆ, ಮತ್ತೊಂದು ಹಿಂದೆ, ಅಷ್ಟೇ. ಎಲ್ಲಿ ಜ್ಞಾನವಿದೆಯೊ, ಅಲ್ಲಿ ವಿಚಾರ ಪ್ರಧಾನ. ಆದರೆ ಆತನಲ್ಲಿ ಅನಾಸಕ್ತಿಯ ಕರ್ಮ, ಲೋಕಸಂಗ್ರಹ ದೃಷ್ಟಿ, ಯಜ್ಞ ದೃಷ್ಟಿ ಇಲ್ಲ ಎಂದಲ್ಲ. ಇದೆಲ್ಲ ಅವನ ಜ್ಞಾನವನ್ನು ಅನುಸರಿಸಿಕೊಂಡು ಹೋಗುವುದು. ರೈಲ್ವೆ ಎಂಜಿನ್ ಮುಂದೆ ಹೋಗುತ್ತಿದ್ದರೆ ಅದಕ್ಕೆ ತಗುಲಿಹಾಕಿದ ಗಾಡಿಗಳೆಲ್ಲ ಹೇಗೆ ಹಿಂದೆ ಹೋಗುತ್ತಿರುವುವೋ ಹಾಗೆ. ಅದರಂತೆಯೇ ಯಾರಲ್ಲಿ ಕರ್ಮ ಮುಂದಾಗಿದೆಯೋ ಅವನಲ್ಲಿ ಜ್ಞಾನವಿಲ್ಲದೆ ಇಲ್ಲ. ಅವನೇನು ಪಂಡಿತರಂತೆ ಹಲವು ಶಾಸ್ತ್ರಗಳನ್ನು ಉದಾಹರಿಸದೆ ಇರಬಹುದು. ನಿಜವಾದ ಸತ್ಯ ಯಾವುದು ಎಂಬುದು ಅವನಿಗೆ ಕ್ಷಣದಲ್ಲಿ ಹೊಳೆಯುವುದು. ಏಕೆಂದರೆ ಅವನಿಗೆ ಚಿತ್ತ ಶುದ್ಧಿಯಾಗಿದೆ. ಅವನಿಗೆ ಯಾವ ಆಸೆ ಆಕಾಂಕ್ಷೆಯೂ ಇಲ್ಲ. ಅವನು ಭಗವಂತನ ಕೈಯಲ್ಲಿ ಒಂದು ನಿಮಿತ್ತವಾಗಿರುವನು.

ಯಾವುದೋ ಒಂದು ಮಾರ್ಗ ಮೇಲಲ್ಲ, ಮತ್ತೊಂದು ಕೀಳಲ್ಲ. ಪ್ರತಿಯೊಬ್ಬರೂ ತಮ್ಮ ತಮ್ಮ ಪ್ರಕೃತಿಗೆ ಅನುಸಾರವಾಗಿರುವ ಮಾರ್ಗವನ್ನು ಹಿಡಿದುಕೊಂಡು ಹೊರಟರೆ, ಎಲ್ಲರೂ ನಿಸ್ಸಂಶಯ ವಾಗಿ ಒಂದೇ ಗುರಿಯಲ್ಲಿ ಸಂಧಿಸುವರು. ಅದನ್ನು ಮಾಡುವುದು ಬಿಟ್ಟು ಕೇವಲ ದಾರಿಗೆ ಕಚ್ಚಾಡುವೆವು.

\begin{verse}
ಯತ್ ಸಾಂಖ್ಯೈಃ ಪ್ರಾಪ್ಯತೇ ಸ್ಥಾನಂ ತದ್ಯೋಗೈರಪಿ ಗಮ್ಯತೇ~।\\ಏಕಂ ಸಾಂಖ್ಯಂ ಚ ಯೋಗಂ ಚ ಯಃ ಪಶ್ಯತಿ ಸ ಪಶ್ಯತಿ \versenum{॥ ೫~॥}
\end{verse}

{\small ಜ್ಞಾನನಿಷ್ಠರು ಯಾವ ಗುರಿಯನ್ನು ಮುಟ್ಟುತ್ತಾರೊ, ಅದನ್ನೇ ಕರ್ಮಯೋಗಿಗಳು ಕೂಡ ಮುಟ್ಟುತ್ತಾರೆ. ಜ್ಞಾನಯೋಗವನ್ನು ಮತ್ತು ಕರ್ಮಯೋಗವನ್ನು ಯಾರು ಒಂದೇ ಎಂದು ನೋಡಬಲ್ಲನೊ ಅವನೇ ನಿಜವಾಗಿ ನೋಡುವನು.}

ಹಿಂದಿನ ಶ್ಲೋಕದಲ್ಲಿ ಯಾವ ಭಾವನೆಯನ್ನು ಹೇಳುತ್ತಾನೋ ಅದನ್ನೇ ಇಲ್ಲಿ ಮತ್ತೂ ವಿವರಿಸುವನು. ಜ್ಞಾನಯೋಗಿಯ ಗುರಿ ಕರ್ಮಯೋಗಿಯ ಗುರಿ ಎರಡೂ ಒಂದೇ. ಒಂದೇ ಗುರಿ ಸೇರಬೇಕಾದರೆ ಒಂದೇ ಮಾರ್ಗವಿಲ್ಲ. ಹಲವು ಮಾರ್ಗಗಳಿವೆ. ಕೆಲವರದು ವಿಚಾರಪ್ರಧಾನವಾದ ದಾರಿ. ಅವರು ಪ್ರತಿಯೊಂದನ್ನೂ ತಿಳಿದುಕೊಳ್ಳಲು ಯತ್ನಿಸುವರು. ಅದು ಹೇಗೆ? ಏತಕ್ಕೆ? ಎಂಬ ಪ್ರಶ್ನೆಯನ್ನು ಹಾಕಿ ಎಷ್ಟು ಸಾಧ್ಯವಾಗುವುದೊ ಅಷ್ಟನ್ನು ತಿಳಿದುಕೊಳ್ಳುವರು. ಅವರಿಗೆ ತಿಳಿದುಕೊಳ್ಳದೆ ಇದ್ದರೆ ಸಮಾಧಾನವಾಗುವುದಿಲ್ಲ. ನಮ್ಮ ಕಣ್ಣಿಗೆ ಈ ಪ್ರಪಂಚ ಒಂದು ರೀತಿಯಾಗಿ ಕಾಣುತ್ತಿದೆ, ನಮ್ಮ ವ್ಯಕ್ತಿತ್ವ ಒಂದು ರೀತಿ ಕಾಣುತ್ತಿದೆ. ಅದನ್ನು ವಿಭಜನೆ ಮಾಡಬೇಕು. ಹೊರಗೆ ಇರುವುದೆಲ್ಲ ನಾಮರೂಪದಿಂದ ಆಗಿದೆ. ನಾಮರೂಪವನ್ನು ತೆಗೆದುಹಾಕಿದರೆ, ಇರುವುದು ಒಂದೇ, ಅಖಂಡ ಸಚ್ಚಿದಾನಂದವಾಗಿರುವುದು. ಅದನ್ನು ಕಾಣದಂತೆ ಮಾಡಿರುವುದೇ ನಾಮರೂಪದ ತೆರೆ. ಒಂದು ಅಖಂಡ ಒಡೆದುಹೋಗಿ ಹಲವು ಸಾಂತ ವಸ್ತುಗಳಾಗಿ ಮಾರ್ಪಾಡಾಗಿದೆ ಎಂಬ ಭ್ರಾಂತಿಯನ್ನು ಹುಟ್ಟಿಸುತ್ತಿದೆ. ಹೊರಗೆ ಇರುವುದು ಒಂದೇ ಅಖಂಡ ಆಕಾಶ. ಮಣ್ಣಿನಿಂದ ಮಡಕೆ ಕುಡಿಕೆಗಳನ್ನು ಮಾಡಿ ಅದರೊಳಗೆ ಇರುವ ಆಕಾಶ ಸಾಂತವಾದುದೆಂದು ನಾವು ಭಾವಿಸುವೆವು. ನಾವೇನೊ ಕೃತಕವಾಗಿ ಒಡೆದಿರಬಹುದು. ಆದರೆ ನಿಜವಾಗಿ ಆಕಾಶವನ್ನು ವಿಭಾಗಿಸಿರುವೆವೆ? ಅದರಂತೆಯೆ ನಮ್ಮ ವ್ಯಕ್ತಿತ್ವ. ಅಖಂಡವಾದ ಸರ್ವವ್ಯಾಪಿಯಾದ ನಾನು ಆಕಾಶದಂತೆ. ಅದಕ್ಕೆ ನಾನು ಇಂತಹವನು ಎಂಬ ಅಹಂಕಾರ, ಮನಸ್ಸು, ಬುದ್ಧಿ, ಇಂದ್ರಿಯ, ದೇಹ ಎಂಬ ಉಪಾಧಿಗಳನ್ನು ತೊಡಿಸಿ ಅದನ್ನು ಸಾಂತವಾಗಿ ಮಾಡಿಹಾಕುವೆವು. ನಮ್ಮ ಅನಂತತ್ವಕ್ಕೆ ಸಾಂತತ್ವವನ್ನು ತಂದೊಡ್ಡುವೆವು. ನಾವೇ ಈ ಮೇರೆಯನ್ನು ಕಲ್ಪಿಸಿಕೊಂಡವರು. ಅನಂತರ ನಾವು ಇದನ್ನು ಕಲ್ಪಿಸಿಕೊಂಡವರು ಎಂಬುದನ್ನು ಮರೆಯುವೆವು. ನಾನೇ ಒಂದು ಕನಸನ್ನು ಸೃಷ್ಟಿ ಮಾಡಿ ಅದರೊಳಗೆ ಪ್ರವೇಶಿಸುವೆನು. ಅನಂತರ ಇದು ನನ್ನ ಕನಸು, ಇದನ್ನು ಸೃಷ್ಟಿ ಮಾಡಿದವನು ನಾನು ಎಂಬುದನ್ನು ಮರೆತು ಅದರಲ್ಲಿ ನನ್ನಿಂದ ಹೊರಗಡೆ ಇರುವ ಮತ್ತಾರದೋ ಪ್ರಪಂಚದಲ್ಲಿ ನಿಸ್ಸಹಾಯಕನಾಗಿ ಅಲೆದಾಡುತ್ತಿರುವಂತೆ ಭಾವಿಸು ವೆನು. ಇದಕ್ಕೆಲ್ಲ ಕಾರಣ ನಾನೇ. ವಿಚಾರದಿಂದ ಇದನ್ನೆಲ್ಲ ತಿಳಿದುಕೊಳ್ಳುತ್ತ ಬರುವೆವು. ಮಿಥ್ಯವಾದ ಉಪಾಧಿಗಳ ಗೋಜಿನಿಂದ ಬಿಡಿಸಿಕೊಂಡು ಬರುವುದು ಆತ್ಮ.

ಕರ್ಮಯೋಗಿ ತಿಳಿದುಕೊಳ್ಳುವುದಕ್ಕೆ ಹೋಗುವುದಿಲ್ಲ. ಅವನು ಏನನ್ನಾದರೂ ಮಾಡುವನು. ತಿಳಿವಳಿಕೆ ಅನಂತರ ಬರುವುದು. ತನ್ನ ಕಣ್ಣೆದುರಿಗೆ ವಿಶ್ವವಿದೆ. ಅದರಲ್ಲಿ ತಾನಿರುವನು, ಮತ್ತು ತನ್ನ ಹಿಂದೆ, ಪ್ರಪಂಚದ ಹಿಂದೆ ಎಲ್ಲವನ್ನೂ ಆಳುತ್ತಿರುವ ಭಗವಂತನೊಬ್ಬನು ಇರುವನು. ಕರ್ಮ ಯೋಗಿ ತನ್ನ ಪಾಲಿಗೆ ಬಂದ ಕೆಲಸವನ್ನು ಮಾಡಿಕೊಂಡು ಹೋಗುವನು. ಮೊದಲು ಹೊರಗಿರುವ ಪ್ರಪಂಚವನ್ನು ನೇರ ಮಾಡಬೇಕೆಂದು ಪ್ರಯತ್ನಿಸುವನು. ಬಹಳ ಕಾಲದ ಮೇಲೆ ಗೊತ್ತಾಗುವುದು, ಅದೇನು ನೇರವಾಗುವ ಹಾಗಿಲ್ಲ, ಅದು ನಾಯಿಬಾಲದಂತೆ ಯಾವಾಗಲೂ ಡೊಂಕೆ. ಒಂದು ಕಡೆ ಡೊಂಕು ಹೋಗುವುದು, ಮತ್ತೊಂದು ಕಡೆ ಡೊಂಕು ಬರುವುದು. ಇದೊಂದು ದೇಹದಲ್ಲಿರುವ ವಾತರೋಗದಂತೆ ಎಂದು ಸ್ವಾಮಿ ವಿವೇಕಾನಂದರು ಹೇಳುತ್ತಿದ್ದರು. ಸೊಂಟ ಬಿಡುವುದು, ಕತ್ತಿಗೆ ಹಿಡಿಯುವುದು, ಕತ್ತು ಬಿಡುವುದು, ಎದೆಗೆ ಹಿಡಿಯುವುದು. ಅಂತೂ ಒಂದಲ್ಲ ಒಂದು ಕಡೆ ಈ ದೇಹದಲ್ಲಿ ಸೇರಿಕೊಂಡು ಯಾತನೆಯನ್ನು ಕೊಡುತ್ತಿರುವುದು. ಕೆಲವು ವೇಳೆ ಸ್ವಲ್ಪ ಕಾಲ ಬಿಟ್ಟು ಹೋದಂತೆ ಇರಬಹುದು, ಆದರೆ ಅದು ಪುನಃ ಬರುವುದಕ್ಕೆ. ನಾನೇನೊ ಇದು ಬಿಟ್ಟುಹೋಯಿತು ಎಂದು ಸಂತೋಷಪಡಬಹುದು. ಇದೊಂದು ನಮ್ಮ ಭ್ರಮೆ ಎಂಬುದು ಅನಂತರ ಗೊತ್ತಾಗುವುದು. ಅದಕ್ಕಿಂತ ಮೇಲಿನ ಮೆಟ್ಟಲು ಈ ಪ್ರಪಂಚವನ್ನು ನೇರ ಮಾಡುವುದಲ್ಲ ನಮ್ಮ ಗುರಿ, ಈ ಪ್ರಪಂಚವನ್ನು ನೇರ ಮಾಡುವುದೊಂದು ನಿಮಿತ್ತ. ಇದರಿಂದ ನೇರವಾಗುವುದು ನಾವೆ. ನಮ್ಮ ಲ್ಲಿರುವ ಸ್ವಾರ್ಥತೆ ಅಜ್ಞಾನ ಇವುಗಳಿಂದ ಕ್ರಮೇಣ ದೂರವಾಗುತ್ತ ಬರುವೆವು. ಇದರಿಂದ ಪ್ರಪಂಚಕ್ಕಿಂತ ನಮಗೇ ಪ್ರಯೋಜನ. ಇದು ಎರಡನೇ ಮೆಟ್ಟಿಲು. ಮೂರನೆಯದೆ, ತಾನು ನೇರವಾದ ಮೇಲೆ ಎಲ್ಲಾ ಕಡೆಯಲ್ಲಿಯೂ ದೇವರೆ ವಿಹರಿಸುತ್ತಿರುವುದು, ಎಲ್ಲವೂ ಅವನೇ ಆಗಿರುವುದು ಗೊತ್ತಾಗುವುದು. ಆಗ ಅವನು ದೇವರ ಕೈಯಲ್ಲಿ ನಿಮಿತ್ತವಾಗಿ, ಅವನು ಏನು ಮಾಡಿಸುವನೊ ಅದನ್ನು ಮಾಡುತ್ತ ಪ್ರತಿಫಲಾಪೇಕ್ಷೆಯಿಲ್ಲದೆ ಕಾಲಕಳೆಯುವನು. ಇಲ್ಲಿಯೂ ಸರ್ವವ್ಯಾಪಿಯಾದ ಭಗವಂತನನ್ನು ಅರಿತು, ತನ್ನ ವ್ಯಕ್ತಿತ್ವವನ್ನು ಅದರಲ್ಲಿ ಲೀನಗೊಳಿಸುವನು.

ಜ್ಞಾನಯೋಗ ಮತ್ತು ಕರ್ಮಯೋಗ ಒಂದೇ ಎಂದು ಗೊತ್ತಾಗುವುದು ಆ ದಾರಿಯಲ್ಲಿ ಗುರಿಯನ್ನು ಮುಟ್ಟಿದಾಗ. ಅಲ್ಲಿ ಒಂದು ಮತ್ತೊಂದರಲ್ಲಿ ಕರಗುವುದನ್ನು ನೋಡುವೆವು. ಜ್ಞಾನದ ಮೂಲಕ ಗುರಿಯನ್ನು ಮುಟ್ಟಿದವನು ಸುಮ್ಮನೆ ಇರುವುದಿಲ್ಲ. ಲೋಕಸಂಗ್ರಹಕ್ಕೆ ಅವನು ಸಮೆಯು ತ್ತಿರುವನು. ಅದರಂತೆಯೇ ಅನಾಸಕ್ತನಾಗಿ ಕರ್ಮ ಮಾಡುವವನಿಗೆ ಪರಮಾತ್ಮ ಜ್ಞಾನ ಬೇಡ ವೆಂದರೂ ದೊರಕುವುದು. ಅವನೇನು ಎಂಬುದನ್ನು ತಿಳಿದುಕೊಳ್ಳುವನು. ಒಲೆಯ ಸಮೀಪಕ್ಕೆ ಹೋದರೆ ನಮಗೆ ಬೇಡದೆ ಇದ್ದರೂ ಕಾವು ಬರುವಂತೆ ಇದು. ಸ್ವಾರ್ಥ, ಲಾಭ, ಐಶ್ವರ್ಯ, ಕೀರ್ತಿ ಇವುಗಳನ್ನೆಲ್ಲ ಬಿಟ್ಟು ಕೇವಲ ಯಜ್ಞದೃಷ್ಟಿಯಿಂದ ಕೆಲಸ ಮಾಡುತ್ತಿದ್ದರೆ ಅವನು ಜೀವನದ ಯಾವ ಕಸುಬನ್ನು ಮಾಡುತ್ತಿರಲಿ, ಯಾವ ವರ್ಣದಲ್ಲಾದರೂ ಇರಲಿ, ಆಶ್ರಮದಲ್ಲಾದರೂ ಇರಲಿ, ಅವನು ಭಗವಂತನೆಂಬ ಮೂಲಚೈತನ್ಯದ ಒಲೆಯ ಹತ್ತಿರ ಹತ್ತಿರ ಹೋಗುತ್ತಾನೆ. ಆ ಮೂಲದಿಂದಲೇ ಬಂದ ಕಾಂತಿಯಲ್ಲಿ ಜೀವರಾಶಿಗಳೆಲ್ಲ ಬೆಳಗುತ್ತಿರುವುದು ಎಂಬುದು ಗೊತ್ತಾಗುವುದು. ಒಂದರಲ್ಲಿ ಮತ್ತೊಂದನ್ನು ನೋಡುವವನೆ ನಿಜವಾಗಿ ನೋಡುವವನು. ಒಂದು ಮತ್ತೊಂದಕ್ಕೆ ವಿರೋಧ, ಅಥವಾ ಒಂದು ಹೆಚ್ಚು ಮತ್ತೊಂದು ಕಡಮೆ ಎಂದು ವಾದಿಸುತ್ತಿರುವವರು ಇನ್ನೂ ತುದಿ ಮುಟ್ಟಿಲ್ಲ. ಮಾರ್ಗದಲ್ಲಿ ಎಲ್ಲೆಲ್ಲೊ ನಿಂತುಕೊಂಡು ಕಾದಾಡುತ್ತಿರುವರು.

\begin{verse}
ಸಂನ್ಯಾಸಸ್ತು ಮಹಾಬಾಹೋ ದುಃಖಮಾಪು ್ತಮಯೋಗತಃ~।\\ಯೋಗಯುಕ್ತೋ ಮುನಿರ್ಬ್ರಹ್ಮ ನ ಚಿರೇಣಾಧಿಗಚ್ಛತಿ \versenum{॥ ೬~॥}
\end{verse}

{\small ಮಹಾಬಾಹುವೆ, ಕರ್ಮಯೋಗದ ಸಹಾಯವಿಲ್ಲದೆ ಸಂನ್ಯಾಸವನ್ನು ಹೊಂದುವುದು ಕಷ್ಟ. ಯೋಗದಿಂದ ಕೂಡಿದ ಮುನಿ ಬೇಗ ಬ್ರಹ್ಮನನ್ನು ಹೊಂದುತ್ತಾನೆ.}

ಕರ್ಮಯೋಗವಿಲ್ಲದೆ ಸಂನ್ಯಾಸ ಪ್ರಾಪ್ತವಾಗುವುದಿಲ್ಲ. ಮನುಷ್ಯನಲ್ಲಿ ತಮೋ ರಜೋಗುಣ ಗಳಿರುತ್ತವೆ. ಅವುಗಳನ್ನು ಕರ್ಮದ ಮೂಲಕ ಶುದ್ಧಮಾಡಿದಲ್ಲದೆ ಒಂದೇ ಸಲ ಸತ್ತ್ವಕ್ಕೆ ಏರಿ ಹೋಗಲು ಸಾಧ್ಯವಿಲ್ಲ. ಬಹಿರ್ಮುಖತೆ ನಮ್ಮ ಸ್ವಭಾವವಾಗಿದೆ. ಅದನ್ನು ಸುಮ್ಮನೆ ಹೊರಗೆ ಹೋಗದಂತೆ ತಡೆಗಟ್ಟಿದರೆ ಅದು ತೆಪ್ಪಗಾಗುವುದಿಲ್ಲ. ಕುದುರೆಯನ್ನು ಒಂದೇ ಕಡೆ ಕಟ್ಟಿಹಾಕಿದ್ದರೆ ಅದು ಬೇಜಾರಾಗಿ ನೆಗೆದಾಡುವುದು. ಬಯಲಲ್ಲಿ ಬಿಟ್ಟರೆ ಅದರ ಕಾಲುಗಳು ಸೋಲುವ ಪರಿಯಂತರ ಓಡಾಡುವುದು. ಅನಂತರ ಸಾಕಾಗಿ ಬಳಲಿ ಒಂದುಕಡೆ ನಿಲ್ಲುವುದು. ಅದರಂತೆಯೇ ಮನಸ್ಸೂ ಕೂಡ. ನಮ್ಮಲ್ಲಿ ಹಲವು ಕೆಟ್ಟ ಸಂಸ್ಕಾರಗಳಿವೆ. ಅದನ್ನು ಮುಂಚೆ ಒಳ್ಳೆಯ ಸಂಸ್ಕಾರಗಳಿಂದ ಬದಲಾಯಿಸ ಬೇಕು. ಅನಂತರವೇ ಶುದ್ಧವಾದ ಸತ್ತ್ವಗುಣ ನಮ್ಮಲ್ಲಿ ನಿಲ್ಲುವುದು.

ಯಾರು ಕೆಲಸಗಳನ್ನು ಮಾಡಿ ಚಿತ್ತವನ್ನು ಶುದ್ಧಮಾಡಿಕೊಂಡಿರುವನೊ, ಅವನ ಮನಸ್ಸು ಸ್ತಿಮಿತಕ್ಕೆ ಬಂದಿದೆ. ಅವನು ಮನಸ್ಸನ್ನು ಏಕಾಗ್ರ ಮಾಡಬಲ್ಲ. ಅವನಲ್ಲಿ ಯಾವ ಉದ್ವೇಗವನ್ನೂ ನೋಡುವುದಿಲ್ಲ. ಅವನು ಇಂದ್ರಿಯವನ್ನು ಬಿಗಿ ಹಿಡಿದಿರುವನು. ಇಂತಹ ಮನುಷ್ಯನು ಬ್ರಹ್ಮ ಸಾಕ್ಷಾತ್ಕಾರವನ್ನು ಬೇಗ ಪಡೆಯುವನು. ಇಂತಹ ಮನುಷ್ಯನಲ್ಲಿ ಗುರಿಯ ಕಡೆಗೆ ಹೋಗುವ ಸಿದ್ಧತೆಗಳೆಲ್ಲ ಇವೆ. ಇವುಗಳನ್ನು ಅಣಿಮಾಡಿಕೊಳ್ಳದೆ, ಬರೀ ಉದ್ವೇಗದಿಂದ ಕರ್ಮವನ್ನು ಬಿಟ್ಟು, ಜ್ಞಾನಮಾರ್ಗದಲ್ಲಿ ನಡೆಯಬೇಕೆಂದು ಇಚ್ಛಿಸಿದರೆ ನಾವು ಬಹುದೂರ ಆ ಮಾರ್ಗದಲ್ಲಿ ನಡೆಯ ಲಾರೆವು.\\

\begin{verse}
ಯೋಗಯುಕ್ತೋ ವಿಶುದ್ಧಾತ್ಮಾ ವಿಜಿತಾತ್ಮಾ ಜಿತೇಂದ್ರಿಯಃ~।\\ಸರ್ವಭೂತಾತ್ಮಭೂತಾತ್ಮಾ ಕುರ್ವನ್ನಪಿ ನ ಲಿಪ್ಯತೇ \versenum{॥ ೭~॥}
\end{verse}

ಯೋಗದಿಂದ ಕೂಡಿದವನೂ, ವಿಶುದ್ಧಾತ್ಮನೂ, ವಿಜಿತಾತ್ಮನೂ, ಜಿತೇಂದ್ರಿಯನೂ, ಸರ್ವಭೂತಗಳ ಆತ್ಮವನ್ನು ತನ್ನ ಆತ್ಮವೆಂದು ಭಾವಿಸಿರುವವನೂ, ಕರ್ಮವನ್ನು ಮಾಡುತ್ತಿದ್ದರೂ ಲಿಪ್ತನಾಗುವುದಿಲ್ಲ.

ಯಾರು ಯೋಗದಿಂದ ಕೂಡಿರುವವನೊ, ದೇವರೆಡೆಗೆ ಹೋಗುವುದಕ್ಕೆ ಮನಸ್ಸನ್ನು ತಿರುಗಿಸಿರುವನೊ, ಅವನು ವಿಶುದ್ಧಾತ್ಮನಾಗಿರುವನು. ಅಂದರೆ ಶುದ್ಧ ಸತ್ವಗುಣವನ್ನು ತನ್ನಲ್ಲಿ ಇಟ್ಟುಕೊಂಡಿರುವನು. ತನ್ನಲ್ಲಿರುವ ತಮಸ್ ರಜೋ ಗುಣಗಳನ್ನೆಲ್ಲ ಪರಿವರ್ತಿಸಿಕೊಂಡು ಅದನ್ನು ಸತ್ವದ ರೂಪಕ್ಕೆ ತಿರುಗಿಸಿರುವನು. ಬೆಣ್ಣೆಯನ್ನು ಕಾಯಿಸಿ, ಅದರಲ್ಲಿರುವ ನೀರಿನ ಅಂಶವನ್ನು ಓಡಿಸಿ, ಅದನ್ನು ತುಪ್ಪ ಮಾಡಿರುವನು. ನೀರಿನ ಅಂಶವೇ ಫಲಾಪೇಕ್ಷೆ ಮುಂತಾದುವು. ಅವನು ಮನಸ್ಸಿನಿಂದ ಅವುಗಳನ್ನೆಲ್ಲ ಓಡಿಸಿರುವನು. ಅವನು ವಿಜಿತಾತ್ಮನಾಗಿರುವನು. ದೇಹವನ್ನು ವಶದಲ್ಲಿಟ್ಟುಕೊಂಡಿರುವನು. ಅವನು ದೇಹ ಹೇಳಿದಂತೆ ಕೇಳುವುದಿಲ್ಲ. ದೇಹ ಅವನು ಹೇಳಿದಂತೆ ಕೇಳುವಂತೆ ಮಾಡಿಕೊಂಡಿರುವನು. ಅವನು ಜಿತೇಂದ್ರಿಯನಾಗಿರುವನು. ಇಂದ್ರಿಯಗಳನ್ನು ನಿಗ್ರಹಿಸಿರುವನು. ಅವುಗಳನ್ನು ಸಂಪೂರ್ಣ ಧ್ವಂಸಮಾಡಿರುವುದಿಲ್ಲ. ತಾನು ಹೇಳಿದಂತೆ ಕೇಳುವಂತೆ ಮಾಡಿಕೊಂಡಿರುವನು. ಅವು ಇನ್ನುಮೇಲೆ ವಿಷಯ ಪ್ರಪಂಚಕ್ಕೆ ಓಡಿಹೋಗಿ ಹೀನ ಸಂಸ್ಕಾರಗಳನ್ನು ಗಳಿಸಲಾರವು. ಇನ್ನುಮೇಲೆ ಉತ್ತಮ ಸಂಸ್ಕಾರಗಳನ್ನು ಪಡೆಯುವಂತಹ ಕೆಲಸಗಳನ್ನು ಮಾತ್ರ ಮಾಡುತ್ತವೆ.

ಅವನು ಸರ್ವಭೂತಗಳ ಆತ್ಮವೇ ತನ್ನಾತ್ಮ ಎಂದು ನೋಡುವ ಸ್ಥಿತಿಗೆ ಬಂದಿದ್ದಾನೆ. ತನ್ನಲ್ಲಿರುವುದೇ, ಎಲ್ಲರಲ್ಲಿಯೂ ಇರುವುದು. ಯಾವ ವಿದ್ಯುತ್​ಶಕ್ತಿ ನನ್ನ ಕೋಣೆಯೊಳಗೆ ಇರುವ ಬಲ್ಬಿನಲ್ಲಿ ಬೆಳಗುತ್ತಿದೆಯೋ, ಅದೇ ವಿದ್ಯುತ್​ಶಕ್ತಿಯೇ ಎಲ್ಲರ ಮನೆಯಲ್ಲಿರುವ ಬಲ್ಬಿನೊಳಗೆ ಬೆಳಗುತ್ತಿರುವುದು. ಅವನು ಎಲ್ಲರನ್ನೂ ಪ್ರೀತಿಸುತ್ತಾನೆ. ಏಕೆಂದರೆ ತಾನೇ ಎಲ್ಲವೂ ಆಗಿರುವುದರಿಂದ. ಇನ್ನೊಬ್ಬನಿಗೆ ತೊಂದರೆ ಕೊಟ್ಟರೆ ತನಗೆ ಅದನ್ನು ಕೊಟ್ಟಂತೆ ಆಗುವುದು. ಅವನು ಎಲ್ಲರ ಆನಂದದಲ್ಲಿಯೂ ಭಾಗಿ. ಅವನ ಹೃದಯ ಎಲ್ಲರ ದುಃಖಕ್ಕೆ ಮಿಡಿಯುವುದು. ಅವನ ಹೃದಯ ವಿಶಾಲವಾಗಿದೆ. ಎಲ್ಲವನ್ನೂ ವ್ಯಾಪಿಸಿದೆ. ಅವನಲ್ಲಿ ಇತರರ ಮೇಲೆ ದ್ವೇಷ ಅಸೂಯೆಯನ್ನು ನೋಡುವುದಿಲ್ಲ. ಎಲ್ಲರಿಗೂ ಒಳ್ಳೆಯದನ್ನು ಆಶಿಸುವುದು ಅವನ ಹೃದಯ. ಭಗವಂತನೆಂಬ ಕೇಂದ್ರದ ಕಡೆಗೆ ಹತ್ತಿರ ಹತ್ತಿರ ಹೋದಂತೆಲ್ಲಾ, ನನಗೂ ಇತರರಿಗೂ ಇರುವ ದೂರ ಕಡಿಮೆಯಾಗುವುದು. ಕೊನೆಗೆ ಅವನು ಎಲ್ಲರನ್ನು ತನ್ನಂತೆ ನೋಡುವ ಉದಾರ ಹೃದಯಿ ಆಗುವನು. ಅವನು ಇನ್ನು ಮೇಲೆ ಯಾವುದೊ ಒಂದು ಸಣ್ಣ ಜಾತಿಯ ದೇಶದ ಪಂಜರದಲ್ಲಿರಲಾರನು. ಅವನು ಎಲ್ಲರೆದುರಿಗೆ ಒಂದಾಗಿ ನಿಲ್ಲುವನು. ಇಂತಹ ಏಕತ್ವದ ಭಾವನೆಯಿಂದ ಕೂಡಿರುವನು, ಏನು ಕರ್ಮವನ್ನು ಮಾಡುತ್ತಿದ್ದರೂ ಅದರಿಂದ ಬಾಧಿತನಾಗುವುದಿಲ್ಲ. ಏಕೆಂದರೆ ಇಂತಹ ಜೀವಿಯಲ್ಲಿ ಸ್ವಾರ್ಥವು ಲವಲೇಶವೂ ಇರದು. ಅವನು ಲಾಭಕ್ಕಾಗಿ, ಕೀರ್ತಿಗಾಗಿ ಯಾವ ಕರ್ಮವನ್ನೂ ಮಾಡುವುದಿಲ್ಲ. ಅವನು ಕರ್ಮ ಮಾಡಿದರೆ ಲೋಕಸಂಗ್ರಹಕ್ಕಾಗಿ ಮಾಡುತ್ತಾನೆ, ಭಗವಂತನಿಗೆ ಮಾಡುವ ಒಂದು ಕೈಂಕರ್ಯದಂತೆ ಮಾಡುವನು. ಇವನ ಹೃದಯದಲ್ಲಿ ಪ್ರಪಂಚಕ್ಕೆ ಕಟ್ಟಿಹಾಕುವ ಸಂಸ್ಕಾರಗಳೇ ಇರುವುದಿಲ್ಲ. ಇದೊಂದು ಒಣಗಿಹೋದ ಮರ. ಅದನ್ನು ಎಷ್ಟು ನೀರು ಎರೆದು ಆರೈಕೆ ಮಾಡಿದರೂ ಚಿಗುರುವುದಿಲ್ಲ. ಇವನೊಂದು ಚೆನ್ನಾಗಿ ಬೆಂದ ಬೀಜ. ಅದನ್ನು ನೆಲದಲ್ಲಿ ಹಾಕಿದರೆ ಪುನಃ ಬೆಳೆಯುವುದಿಲ್ಲ. ತನ್ನಲ್ಲಿರುವ ಸತ್​ಸಂಸ್ಕಾರಗಳ ಶೇಷ ಇರುವತನಕ ಕೆಲಸ ಮಾಡುತ್ತಿರುವನು. ಅದು ಖರ್ಚಾದರೆ ಕೆಳಗೆ ಬೀಳುವುದು. ಇಂತಹ ಜೀವಿ ಕೆಲಸದ ಸುಂಟರಗಾಳಿಯಲ್ಲಿ ನಿರತನಾಗಿದ್ದರೂ ನಿರ್ಲಿಪ್ತ.

\begin{verse}
ನೈವ ಕಿಂಚಿತ್ ಕರೋಮೀತಿ ಯುಕ್ತೋ ಮನ್ಯೇತ ತತ್ತ್ವವಿತ್~।\\ಪಶ್ಯನ್ ಶೃಣ್ವನ್​ಸ್ಪೃಶನ್ ಜಿಘ್ರನ್ನಶ್ನನ್ ಗಚ್ಛನ್ ಸ್ವಪನ್ ಶ್ವಸನ್ \versenum{॥ ೮~॥}
\end{verse}

\begin{verse}
ಪ್ರಲಪನ್ ವಿಸೃಜನ್ ಗೃಹ್ಣನ್ನುನ್ಮಿಷನ್ನಿಮಿಷನ್ನಪಿ~।\\ಇಂದ್ರಿಯಾಣೀಂದ್ರಿಯಾರ್ಥೇಷು ವರ್ತಂತ ಇತಿ ಧಾರಯನ್ \versenum{॥ ೯~॥}
\end{verse}

{\small ಯುಕ್ತನಾದ ತತ್ತ್ವಜ್ಞಾನಿ ನೋಡುವಾಗಲೂ, ಕೇಳುವಾಗಲೂ, ಮುಟ್ಟುವಾಗಲೂ, ಮೂಸುವಾಗಲೂ, ನಡೆಯು ವಾಗಲೂ, ನಿದ್ರಿಸುವಾಗಲೂ, ಉಸಿರು ಬಿಡುವಾಗಲೂ, ಮಾತನಾಡುವಾಗಲೂ, ಮಲಮೂತ್ರ ವಿಸರ್ಜಿಸು ವಾಗಲೂ, ತೆಗೆದುಕೊಳ್ಳುವಾಗಲೂ, ಕಣ್ಣು ತೆರೆಯುವಾಗಲೂ, ಮುಚ್ಚುವಾಗಲೂ, ಇಂದ್ರಿಯಗಳು ತಮ್ಮ ವಿಷಯದಲ್ಲಿ ಪ್ರವರ್ತಿಸುತ್ತಿವೆ ಎಂದು ನಿಶ್ಚಯಿಸಿ, ತಾನು ಏನನ್ನೂ ಮಾಡುತ್ತಿಲ್ಲ ಎಂದು ತಿಳಿಯಬೇಕು.}

ಹಿಂದಿನ ಶ್ಲೋಕದಲ್ಲಿ ಕರ್ಮಯೋಗಿ ಕೆಲಸ ಮಾಡುತ್ತಿದ್ದರೂ ಅದಕ್ಕೆ ಅಂಟಿಕೊಂಡಿಲ್ಲ ಎಂಬುದನ್ನು ಹೇಳಿದನು. ಇಲ್ಲಿನ ಎರಡು ಶ್ಲೋಕಗಳಲ್ಲಿ ಜ್ಞಾನಿಯಾದವನು ಎಂತಹ ಸಾಕ್ಷೀಭಾವ ವನ್ನು ತಾಳುತ್ತಾನೆ ತಾನು ಕೆಲಸ ಮಾಡುವಾಗ ಎಂಬುದನ್ನು ಹೇಳುತ್ತಾನೆ. ಅವನ ಜ್ಞಾನೇಂದ್ರಿಯ ಕರ್ಮೇಂದ್ರಿಯಗಳೆಲ್ಲ ಕೆಲಸ ಮಾಡುತ್ತಿರುತ್ತವೆ. ಅದನ್ನು ನೋಡಿದಾಗ, ತನ್ನಿಂದ ಹೊರಗೆ ಇರುವ ಇಂದ್ರಿಯಗಳು, ಅದರ ಹೊರಗೆ ಇರುವ ವಿಷಯ ವಸ್ತುವಿನ ಕಡೆ ಹೋಗುತ್ತಿವೆ, ತಾನು ಏನನ್ನೂ ಮಾಡುತ್ತಿಲ್ಲ ಎಂದು ಭಾವಿಸುತ್ತಾನೆ. ಇಲ್ಲಿಯೂ ಕರ್ಮದ ಹಿಂದುಗಡೆ ಒಂದು ಅನಾಸಕ್ತಿಯನ್ನು ನೋಡುತ್ತೇವೆ. ಆದರೆ ಅವನು ಇದನ್ನು ಅಭ್ಯಾಸ ಮಾಡುವುದು ಬೇರೆ ದೃಷ್ಟಿಯಿಂದ. ಅವನು ಮಾಡುವ ಕ್ರಿಯೆಗಳ ಹಿಂದುಗಡೆ ತಾದಾತ್ಮ್ಯಭಾವವಿಲ್ಲ. ನಾನೇ ಇದನ್ನೆಲ್ಲ ನೋಡುತ್ತಿರುವುದು, ಮಾಡುತ್ತಿರುವುದು ಎಂದು ತಲ್ಲೀನನಾಗುವುದಿಲ್ಲ. ಇಂದ್ರಿಯದ ಮೂಲಕ ಕೆಲಸ ಆಗುತ್ತಿರು ವಾಗಲೂ ಅವನು ಇದರ ಹಿಂದೆ ನಿಂತಂತೆ ಇರುವನು. ಅವನು ನೋಡುತ್ತಾನೆ, ಕಣ್ಣಿನಲ್ಲಿ ಆಸಕ್ತಿಯಿಲ್ಲ. ಅದು ಅಲ್ಲೇ ಅಂಟಿಹೋಗುವುದಿಲ್ಲ. ಅದು ಸುಂದರವಾಗಿದೆ, ಕುರೂಪವಾಗಿದೆ ಎಂದು ಅರಿಯಬಲ್ಲ. ಅದರ ಅಂದಕ್ಕೆ ಮನಸೋಲುವುದೂ ಇಲ್ಲ, ವಿಕಾರಕ್ಕೆ ಜುಗುಪ್ಸೆಯೂ ಇಲ್ಲ. ಕೇಳುತ್ತಾನೆ. ಯಾವುದು ಹೇಗೆ ಎಂದು ವಿಮರ್ಶಿಸಬಲ್ಲ. ಆದರೆ ಅವನು ಕೇಳುವುದಕ್ಕೆ ದಾಸನಲ್ಲ. ಅದನ್ನು ಹುಡುಕಿಕೊಂಡು ಹೋಗುವುದಿಲ್ಲ. ಸಿಕ್ಕಿದಾಗ ಅದರೊಡನೆ ಕಟ್ಟಿಹಾಕಿಕೊಳ್ಳುವುದೂ ಇಲ್ಲ. ಮುಟ್ಟುತ್ತಾನೆ. ಅದು ಮೃದುವಾಗಿದೆಯೆ, ಗಟ್ಟಿಯಾಗಿದೆಯೆ, ತಂಪಾಗಿದೆಯೆ, ಶಾಖವಾಗಿದೆಯೆ ಎಂಬುದನ್ನು ಅರಿಯುತ್ತಾನೆ. ಆದರೆ ಆ ವೇದನೆಗಳಿಗೆ ದಾಸನಲ್ಲ. ಆ ವೇದನೆಗಳು ಅವನಲ್ಲಿ ಯಾವ ಸಂಸ್ಕಾರಗಳನ್ನೂ ಬಿಡಲಾರವು. ಇದರಂತೆಯೇ ಮೂಸುವಾಗಲೂ ತಿನ್ನುವಾಗಲೂ ಕೂಡ. ಸುಗಂಧ ದುರ್ಗಂಧಗಳನ್ನು ತಿಳಿಯುತ್ತಾನೆ. ಉಪ್ಪು, ಹುಳಿ, ಖಾರ ಯಾವುದಕ್ಕೆ ಹೆಚ್ಚು, ಕಡಮೆ ಎಂಬುದನ್ನೆಲ್ಲ ತಿಳಿದುಕೊಳ್ಳಬಲ್ಲ. ಆದರೆ ಅವನು ರುಚಿಗೆ ದಾಸನಲ್ಲ. ಅವನ ಇಂದ್ರಿಯಗಳು, ಸಾಧಾರಣ ಮನುಷ್ಯನ ಇಂದ್ರಿಯಗಳಿಗಿಂತ ಸೂಕ್ಷ್ಮವಾಗಿವೆ. ಜ್ಞಾನಿಯಲ್ಲಿ ಅದೆಲ್ಲ ತುಕ್ಕು ಹಿಡಿದುಹೋಗಿಲ್ಲ. ಆದರೆ ಅವನು ಅದಕ್ಕೆ ದಾಸನಲ್ಲ. ಅವುಗಳಿಂದ ಹೊರಗೆ ನಿಂತು ತನಗೆ ಬೇಕಾದಾಗ ಅದನ್ನು ಬಳಸುತ್ತಾನೆ, ಬೇಡವಾದಾಗ ಅದನ್ನು ಹಾಗೆಯೇ ಇಟ್ಟಿರುತ್ತಾನೆ. ತಾನು ಯಾವಾಗಲೂ ಅದರಿಂದ ಹೊರಗೆ, ತಾನು ಅದಲ್ಲ ಎಂಬುದನ್ನು ಸರ್ವದಾ ತಿಳಿದುಕೊಂಡಿರುವನು. ನಮ್ಮ ಎದುರಿಗೆ ಇರುವ ಗಿಡ, ಮರ, ಗೋಡೆ ಹೇಗೆ ನಮ್ಮಿಂದ ಹೊರಗೆ ಇವೆಯೋ ಹಾಗೆ ಜ್ಞಾನಿ ತನ್ನ ಕರ್ಮೇಂದ್ರಿಯ, ಜ್ಞಾನೇಂದ್ರಿಯದಿಂದ ಹೊರಗೆ ಸಾಕ್ಷಿಯಂತೆ ನೋಡಬಲ್ಲ. ಆಗ ಅವನು ಆ ಕ್ರಿಯೆಗೆ ಬದ್ಧನಾಗುವು ದಿಲ್ಲ. ಅದರಿಂದ ಬರುವ ಫಲಕ್ಕೂ ಆಸಕ್ತನಲ್ಲ. ಹಾವು ಹೇಗೆ ಪೊರೆಯನ್ನು ಕಳಚಿಕೊಂಡು ಹೊರಗೆ ಬರುವುದೊ ಹಾಗೆ ಅವನು ಇಂದ್ರಿಯಗಳ ಪೊರೆಯನ್ನು ಬಿಡಿಸಿಕೊಂಡು ಅದರಿಂದ ಹೊರಗೆ ಬಂದು ಸಾಕ್ಷಿಯಂತೆ ಅದನ್ನು ನೋಡುತ್ತಿರುವನು.

\begin{verse}
ಬ್ರಹ್ಮಣ್ಯಾಧಾಯ ಕರ್ಮಾಣಿ ಸಂಗಂ ತ್ಯಕ್ತ್ವಾ ಕರೋತಿ ಯಃ~।\\ಲಿಪ್ಯತೇ ನ ಸ ಪಾಪೇನ ಪದ್ಮಪತ್ರಮಿವಾಂಭಸಾ \versenum{॥ ೧೦~॥}
\end{verse}

{\small ಯಾರು ಬ್ರಹ್ಮನಲ್ಲಿ ಕರ್ಮಗಳನ್ನು ಇಟ್ಟು, ಸಂಗವನ್ನು ಬಿಟ್ಟು, ಕರ್ಮವನ್ನು ಮಾಡುತ್ತಿರುವನೋ, ಅವನಿಗೆ ತಾವರೆಯ ಎಲೆಗೆ ನೀರಿನ ಲೇಪವಿಲ್ಲದಿರುವಂತೆ ಪಾಪದ ಲೇಪವಿರುವುದಿಲ್ಲ.}

ಶ‍್ರೀಕೃಷ್ಣ, ಮಾಡುತ್ತಿರುವ ಕರ್ಮದ ಉಪಟಳಕ್ಕೆ ಬೀಳದಂತೆ ನೋಡಿಕೊಳ್ಳುವುದಕ್ಕೆ ಹಲವಾರು ಉಪಾಯಗಳನ್ನು ಕೊಡುತ್ತಾನೆ. ಮೊದಲನೆಯದೇ ಅನಾಸಕ್ತನಾಗಿರುವುದು. ಇದು ಕರ್ಮಯೋಗದ ದೃಷ್ಟಿ. ಇಲ್ಲಿ ಕೆಲಸ ಮಾಡುವುದಕ್ಕೆ ಮಾತ್ರ ನಮಗೆ ಅಧಿಕಾರ. ಅದರಿಂದ ಬರುವ ಫಲಗಳಿಗೆ ಅಲ್ಲ. ಎರಡನೆಯದೇ ಜ್ಞಾನದೃಷ್ಟಿ. ಇಂದ್ರಿಯಗಳು ಅದಕ್ಕೆ ಸಂಬಂಧಪಟ್ಟ ವಿಷಯವಸ್ತುಗಳೊಡನೆ ಸೇರಿ ಕರ್ಮ ಆಗುತ್ತಿದೆ. ನಾನು ಕೇವಲ ಸಾಕ್ಷಿ ಎಂಬ ಭಾವವನ್ನು ರೂಢಿಸುವುದು. ಮುಂದೆ ಹೇಳುವುದೇ ಭಕ್ತಿಯ ದೃಷ್ಟಿ, ಭಾವದ ಹಾದಿಯದು. ಅವನು ಕರ್ಮಗಳನ್ನು ಬ್ರಹ್ಮದಲ್ಲಿ ಮಾಡುತ್ತಾನೆ ಎಂದರೆ ಭಗವಂತನಿಗಾಗಿ ಮಾಡುತ್ತಾನೆ. ಇವನ ಕರ್ಮವೆಲ್ಲ ಕೈಂಕರ್ಯ, ಭಗವಂತನಿಗೆ ಮಾಡುತ್ತಿರುವುದು. ಅದರಿಂದ ಬರುವ ಫಲ ಅವನಿಗೆ ಬೇಕಾಗಿಲ್ಲ. ದೇವರಿಗಾಗಿ ಒಂದು ಕೆಲಸ ಮಾಡುವುದೇ ಆನಂದ. ಇದಕ್ಕೆ ಮಿಗಿಲಾದ ಆನಂದ ಬೇರೊಂದಿಲ್ಲ ಅವನ ಪಾಲಿಗೆ. ಮಾಡುವಾಗ ಅದಕ್ಕೆ ಅಂಟಿಕೊಂಡಿರು ವುದಿಲ್ಲ. ಏಕೆಂದರೆ ಅವನಿಗೆ ಗೊತ್ತಿದೆ, ಭಗವಂತನ ಕೆಲಸ ತನ್ನಂತಹ ಅಲ್ಪ ವ್ಯಕ್ತಿ ಮಾಡುವುದರ ಮೇಲೆ ನಿಂತಿಲ್ಲ ಎಂಬುದು. ದೇವರು ಅವಕಾಶ ಕೊಟ್ಟರೆ ಮಾಡುತ್ತಾನೆ, ಯಾವಾಗ ಅವನನ್ನು ತೆಗೆಯುವನೋ, ಆಗ ಅವನು ಪೂರ್ತಿಯಾಗುವವರೆಗೆ ದೇವರು ತನ್ನನ್ನೇ ಉಪಯೋಗಿಸಬೇಕು ಎಂದು ಗೊಣಗಾಡುವುದಿಲ್ಲ. ದೇವರಿಗೆ ತನಗಿಂತ ಚೆನ್ನಾಗಿ ಗೊತ್ತಿದೆ ಎಂಬುದನ್ನು ಅವನು ಎಂದಿಗೂ ಮರೆಯುವುದಿಲ್ಲ. ಎಲ್ಲ ಜೀವರೂ ಅವನ ನಿಮಿತ್ತವೆ. ಯಾರು ಯಾರಿಂದ ಯಾವಯಾವ ಕೆಲಸ ಎಷ್ಟು ಆಗಬೇಕಾಗಿದೆಯೋ ಅದು ಅವನಿಗೆ ಗೊತ್ತು. ಅವನಿಚ್ಛೆ ನೆರವೇರಲಿ ಎಂದು ಶಾಂತಿಯಿಂದ ಇರುತ್ತಾನೆ.

ಅವನು ಕರ್ಮಪ್ರಪಂಚದಲ್ಲಿ ಇರುತ್ತಾನೆ. ಅವನು ಕೂಡ ಕರ್ಮಗಳನ್ನು ಮಾಡುತ್ತಿರುವನು. ಆದರೆ ಅದಕ್ಕೆ ಅಂಟಿಕೊಂಡಿರುವುದಿಲ್ಲ ಎಂಬುದಕ್ಕೆ ಈ ಸುಂದರ ಉಪಮಾನವನ್ನು ಕೊಡುವನು. ನೀರು ಕಮಲದ ಎಲೆಯ ಮೇಲೆ ಇರುವುದು. ಆದರೆ ಅದಕ್ಕೆ ಅಂಟಿಕೊಂಡಿರುವುದಿಲ್ಲ. ಶ‍್ರೀರಾಮ ಕೃಷ್ಣರು ಇದೇ ಭಾವನೆಯನ್ನು ತಮ್ಮದೇ ಆದ ಮತ್ತೊಂದು ಉದಾಹರಣೆಯ ಮೂಲಕ ವಿವರಿಸು ತ್ತಾರೆ. ಕರ್ಮನದಿಯ ಮೇಲೆ ದೋಣಿ ತೇಲುತ್ತಿರುವುದು, ಆದರೆ ಆ ನೀರು ದೋಣಿಯೊಳಗೆ ಬರದಂತೆ ನೋಡಿಕೊಳ್ಳುವನು.

ಯಾವಾಗ ಒಬ್ಬ ಭಗವದರ್ಪಣ ಭಾವದಿಂದ ಕರ್ಮವನ್ನು ಮಾಡುತ್ತಾನೋ ಅವನನ್ನು ಯಾವ ಪಾಪಗಳೂ ಸೋಕುವುದಿಲ್ಲ. ಏಕೆಂದರೆ ಅವನು ಕರ್ಮಗಳನ್ನು ತನಗಾಗಿ ಮಾಡುತ್ತಿಲ್ಲ. ದೇವರಿಗಾಗಿ ಕೆಲಸವನ್ನು ಮಾಡುವಾಗ ಕೆಲವು ವೇಳೆ ಅನಿವಾರ್ಯವಾಗಿ ಇತರರಿಗೆ ತೊಂದರೆ ಕೊಡಬೇಕಾದ ಪ್ರಸಂಗಗಳು ಬರಬಹುದು. ವೈದ್ಯ ಶಸ್ತ್ರಚಿಕಿತ್ಸೆ ಮಾಡುವಾಗ ರೋಗಿಗೆ ನೋವನ್ನು ಕೊಡುತ್ತಾನೆ. ನ್ಯಾಯಾಧಿಪತಿ ಕಳ್ಳನಿಗೆ ಶಿಕ್ಷೆ ವಿಧಿಸುತ್ತಾನೆ. ಯೋಧ ಶತ್ರುವನ್ನು ಸಂಹರಿಸುತ್ತಾನೆ. ದೇವರು ಕೆಲವು ವೇಳೆ ಅಧರ್ಮದ ಕಳೆಯನ್ನು ಕೀಳುವುದಕ್ಕೆ ತನಗೆ ಅರ್ಪಿತರಾದ ಭಕ್ತರನ್ನು ನಿಮಿತ್ತ ಮಾಡಿಕೊಂಡು ಆ ಕೆಲಸವನ್ನು ಮಾಡುತ್ತಾನೆ. ಆದರೆ ಆ ನಿಮಿತ್ತಕ್ಕೆ ಪಾಪ ಅಂಟುವುದಿಲ್ಲ. ಅವನು ಅದನ್ನು ತನಗಾಗಿ ಮಾಡುವುದಿಲ್ಲ. ಅವನಿಗೆ ತಾನು ಮಾಡುವ ಕರ್ಮದ ಫಲದ ಮೇಲೆ ಸ್ವಲ್ಪವೂ ಆಸಕ್ತಿ ಇಲ್ಲ.

\begin{verse}
ಕಾಯೇನ ಮನಸಾ ಬುದ್ಧ್ಯಾ ಕೇವಲೈರಿಂದ್ರಿಯೈರಪಿ~।\\ಯೋಗಿನಃ ಕರ್ಮ ಕುರ್ವಂತಿ ಸಂಗಂ ತ್ಯಕ್ತ್ವಾತ್ಮಶುದ್ಧಯೇ \versenum{॥ ೧೧~॥}
\end{verse}

{\small ಕೇವಲ ದೇಹ, ಮನಸ್ಸು, ಬುದ್ಧಿ, ಇಂದ್ರಿಯಗಳಿಂದ ಯೋಗಿಗಳು ಸಂಗವನ್ನು ಬಿಟ್ಟು ಆತ್ಮಶುದ್ಧಿಗಾಗಿ ಕರ್ಮವನ್ನು ಮಾಡುತ್ತಾರೆ.}

ಇಲ್ಲಿ ಕರ್ಮಮಾಡುವುದಕ್ಕೆ ಮತ್ತೊಂದು ದೃಷ್ಟಿಯನ್ನು ಶ‍್ರೀಕೃಷ್ಣ ತರುತ್ತಾನೆ. ಸಂಪೂರ್ಣವಾಗಿ ಫಲಾಸಕ್ತನಾಗದಿರಲು ಒಬ್ಬನಿಗೆ ಸಾಧ್ಯವಿಲ್ಲದೇ ಇರಬಹುದು, ತನ್ನ ಅಹಂಕಾರವನ್ನು ಭಗವಂತನಿಗೆ ಅರ್ಪಿಸಿ ಒಂದು ನಿಮಿತ್ತವಾಗುವ ಸ್ಥಿತಿಗೆ ಒಬ್ಬ ಬಂದಿಲ್ಲದೇ ಇರಬಹುದು. ಅವನು ಇನ್ನೂ ಸಾಧನಾವಸ್ಥೆಯಲ್ಲಿರುವನು. ಅಂತಹವನು ಕರ್ಮ ಮಾಡುವುದಕ್ಕೆ ಒಂದು ದೃಷ್ಟಿಯನ್ನು ಕೊಡು ತ್ತಾನೆ. ಅದೇ ಚಿತ್ತಶುದ್ಧಿಗಾಗಿ ಕರ್ಮ ಮಾಡುವುದು. ಅವನು ತನ್ನಿಂದ ಹೊರಗಡೆ ಇರುವ ಪ್ರಪಂಚವನ್ನು ಬದಲಾಯಿಸಬೇಕೆಂದಾಗಲಿ, ಅದನ್ನು ಉತ್ತಮ ಸ್ಥಿತಿಗೆ ತರಬೇಕೆಂದಾಗಲಿ ಮಾಡುವು ದಿಲ್ಲ. ಅನಾಸಕ್ತನಾಗಿ ಕರ್ಮವನ್ನು ಮಾಡಿಕೊಂಡು ಹೋದರೆ ತನಗೇ ಮೇಲೆಂದು ಅವನಿಗೆ ಗೊತ್ತಿದೆ. ಇದು ಅವನ ಚಿತ್ತವನ್ನು ಶುದ್ಧಿ ಮಾಡುವುದು. ಚಿತ್ತಶುದ್ಧಿಯೇ ಅವನ ಗುರಿ. ಚಿತ್ತಶುದ್ಧಿಯೇ ಅವನು ನಿರೀಕ್ಷಿಸುವ ಫಲ. ಅವನು ತನ್ನ ಅಧೀನದಲ್ಲಿರುವ ದೇಹ, ಮನಸ್ಸು, ಬುದ್ಧಿ, ಇಂದ್ರಿಯಗಳ ಮೂಲಕ ಕೆಲಸ ಮಾಡುವನು. ಇಲ್ಲಿ ಶ‍್ರೀಕೃಷ್ಣ ಕೇವಲ ದೇಹ, ಮನಸ್ಸು, ಬುದ್ಧಿ, ಇಂದ್ರಿಯ ಎಂದು ಉಪಯೋಗಿಸುವನು. ಅದಕ್ಕಿಂತ ಹೆಚ್ಚಿಲ್ಲ. ಅವನು ಆಸಕ್ತಿಯಿಂದ ಮಾಡುವುದಿಲ್ಲ. ಕೆಲಸಕ್ಕೆ ಅವನು ತನ್ನ ಸರ್ವಸ್ವವನ್ನೂ ಅರ್ಪಿಸಿ ಅದರಲ್ಲೇ ತನ್ಮಯನಾಗಿ ಹೋಗುವುದಿಲ್ಲ. ದೇಹ ಮತ್ತು ಇಂದ್ರಿಯ ಗಳ ಮೂಲಕ ಹಲವು ಕೆಲಸಗಳನ್ನು ಮಾಡುತ್ತಾನೆ. ಕೆಲಸವನ್ನು ಮಾಡಿಹಾಕುತ್ತಾನೆ ಅಷ್ಟೆ. ಅದರಿಂದ ಏನು ಪ್ರಯೋಜನವಾಗುವುದು ಎಂಬುದನ್ನು ಗಮನಿಸುವುದೇ ಇಲ್ಲ. ಮನಸ್ಸಿನಿಂದ ಕಲ್ಪನೆ, ಸ್ಮೃತಿ ಇವುಗಳನ್ನೆಲ್ಲ ಚಿಂತಿಸುತ್ತಾನೆ. ಆದರೆ ಕೆಲಸಕ್ಕೆ ಎಷ್ಟು ಬೇಕೋ ಅಷ್ಟು. ಸುಮ್ಮನೆ ಅದನ್ನು ಕುರಿತು ಚಿಂತಿಸುತ್ತಾ ಅದರ ಸುತ್ತಲೂ ಸುತ್ತುತ್ತಿರುವುದಿಲ್ಲ. ಬುದ್ಧಿಯಿಂದ ಅಮೋಘವಾದ ಕೆಲಸವನ್ನು ಮಾಡುವನು. ತರ್ಕಿಸುವನು, ವಿಚಾರ ಮಾಡುವನು, ಒಂದು ಊಹೆಗೆ ಬರುವನು. ಇದರಿಂದ ಹೊಸ ಹೊಸ ನಿಯಮಗಳನ್ನು ಅವನು ಕಂಡುಹಿಡಿಯಬಹುದು. ದೊಡ್ಡ ವಿಜ್ಞಾನಿಯಾಗಬಹುದು. ಆದರೆ ಅದರಿಂದ ಫಲಗಳಿಗೆ ಅಂಟಿಕೊಂಡಿರುವುದಿಲ್ಲ. ಇದನ್ನೆಲ್ಲ ನಾನು ಸಾಧಿಸಿದವನು, ನನ್ನ ಬುದ್ಧಿ ಅತ್ಯಂತ ಪ್ರಚಂಡವಾದುದು ಎಂದು ಹೊಗಳಿಸಿಕೊಳ್ಳಬೇಕು ಎಂಬ ಆಸೆ ಇಲ್ಲ. ಹಾಗಾದರೆ ಅವನ ಯೋಗ್ಯತೆ ಅವನಿಗೆ ಗೊತ್ತಿಲ್ಲವೆ ಎಂದರೆ, ಗೊತ್ತಿದೆ, ಆದರೆ ಅದರಿಂದ ಅವನ ತಲೆ ತಿರುಗಿ ಹೋಗುವುದಿಲ್ಲ. ಸಾಧಿಸಿರುವುದನ್ನು ಸಾಧಿಸಬೇಕಾಗಿರುವುದರೊಂದಿಗೆ ಹೋಲಿಸಿ ನೋಡಿದರೆ ನಾವು ಸಾಧಿಸಿದ್ದು ಎಷ್ಟು ಅಲ್ಪ ಎಂದು ಗೊತ್ತಾಗುವುದು. ಆಕರ್ಷಣ ಸಿದ್ಧಾಂತವನ್ನು ಕಂಡುಹಿಡಿದ ನ್ಯೂಟನ್ನನನ್ನು ಯಾರೋ, ಇವನೊಬ್ಬ ಮಹಾ ವಿಜ್ಞಾನಿ, ಪ್ರಪಂಚದಲ್ಲೇ ಅದ್ಭುತವಾದ ವೈಜ್ಞಾನಿಕ ರಹಸ್ಯವನ್ನು ಕಂಡುಹಿಡಿದಿರುವನು ಎಂದು ಹೊಗಳುತ್ತಿದ್ದರು. ಅದನ್ನು ಕೇಳಿ ನ್ಯೂಟನ್ ಈ ಒಂದು ಉದಾಹರಣೆಯನ್ನು ಕೊಡುತ್ತಾನೆ:“ಸಮುದ್ರದ ತೀರದಲ್ಲಿ ಬಂದು ಬೀಳುವ ಕಪ್ಪೆ ಚಿಪ್ಪುಗಳನ್ನು ಕೈಯಲ್ಲಿ ತೆಗೆದುಕೊಂಡು ಆಡುತ್ತಿರುವ ಹುಡುಗನಂತೆ ನಾನು. ನನ್ನ ಕೈಯಲ್ಲಿರುವ ಕಪ್ಪೆಚಿಪ್ಪುಗಳಿ ಗಿಂತ ಅನರ್ಘ್ಯವಾದ ಮುತ್ತುರತ್ನಗಳೆಲ್ಲ ಸಮುದ್ರದ ಒಳಗೆ ಬಿದ್ದಿವೆ.” ಎಂತಹ ನಿಗರ್ವವನ್ನು ನೋಡುತ್ತೇವೆ ಇಂತಹ ದೊಡ್ಡ ವಿಜ್ಞಾನಿಯಲ್ಲಿ!

ಕೇವಲ ಚಿತ್ತಶುದ್ಧಿಯನ್ನೇ ಗುರಿಯನ್ನಾಗಿ ಇಟ್ಟುಕೊಂಡರೆ ಅಂತಹವರಿಂದ ಪ್ರಪಂಚಕ್ಕೆ ಅಷ್ಟೊಂದು ಒಳ್ಳೆಯದು ಆಗದೇ ಇರಬಹುದು ಎಂದು ಭಾವಿಸಬಹುದು. ಪ್ರಪಂಚಕ್ಕೆ ಒಳ್ಳೆಯದನ್ನು ಮಾಡಬೇಕೆಂಬ ಉದ್ದೇಶದಿಂದ ಮಾಡಿದರೆ ಮಾತ್ರ ಪ್ರಪಂಚಕ್ಕೆ ಒಳ್ಳೆಯದಾಗುವುದು; ಆ ಉದ್ದೇಶ ವನ್ನು ಮರೆತು ನಮ್ಮ ಚಿತ್ತಶುದ್ಧಿಯನ್ನೇ ಆದರ್ಶವಾಗಿಟ್ಟುಕೊಂಡರೆ, ಪ್ರಪಂಚ ಹಾಳಾದರೂ, ಅದರಿಂದ ಅವನ ಚಿತ್ತಶುದ್ಧಿಗೆ ಸಹಾಯ ಆದರೆ ಅದನ್ನು ಅವನು ಮಾಡಬಹುದಲ್ಲ ಎಂದು ತರ್ಕಿಸಬಹುದು. ಹೊರಗೆ ನಷ್ಟ ಮಾಡಿದರೆ ತನಗೆ ಶ್ರೇಯಸ್ ಆಗಲಾದರು. ಚಿತ್ತಶುದ್ಧಿಯ ಆದರ್ಶವನ್ನು ಇಟ್ಟುಕೊಂಡು ಕರ್ಮ ಮಾಡಿದರೆ ಅವನು ಜೀವನದಲ್ಲಿ ಶ್ರೇಷ್ಠವಾದ ಕರ್ಮವನ್ನು ಮಾಡುತ್ತಾನೆ. ಅವನು ಮಾಡಿದ ಕೆಲಸ ನಿಜವಾಗಿಯೂ ಲೋಕಸಂಗ್ರಹವಾದ ಕೆಲಸ. ಇದರಲ್ಲಿ ಯಾವ ಒಂದು ಸಂಶಯವೂ ಇಲ್ಲ. ಚಿತ್ತಶುದ್ಧಿಯ ಆದರ್ಶವನ್ನು ಇಟ್ಟುಕೊಂಡಿದ್ದರೆ ಲೋಕಹಾನಿ ಅವನಿಂದ ಸಾಧ್ಯವೇ ಇಲ್ಲ. ಲೋಕೋದ್ಧಾರವೇ ಅವನಿಂದ ಆಗುವುದು. ಇದು ಲೋಕೋದ್ಧಾರಕ್ಕಾಗಿ ಎಂಬ ದೊಡ್ಡ ಮಾತನ್ನು ಆಡುವುದಿಲ್ಲ. ಆತ್ಮೋದ್ಧಾರಕ್ಕಾಗಿ ಎಂದು ದೈನ್ಯದಿಂದ ಹೇಳುವನು.

\begin{verse}
ಯುಕ್ತಃ ಕರ್ಮಫಲಂ ತ್ಯಕ್ತ್ವಾ ಶಾಂತಿಮಾಪ್ನೋತಿ ನೈಷ್ಠಿಕೀಮ್~।\\ಅಯುಕ್ತಃ ಕಾಮಕಾರೇಣ ಫಲೇ ಸಕ್ತೋ ನಿಬಧ್ಯತೇ \versenum{॥ ೧೨~॥}
\end{verse}

{\small ಯೋಗಿಯು ಕರ್ಮಫಲವನ್ನು ಬಿಟ್ಟು ನಿಷ್ಠಾರೂಪವಾದ ಶಾಂತಿಯನ್ನು ಪಡೆಯುತ್ತಾನೆ. ಯುಕ್ತನಾಗದವನು ಕಾಮಪ್ರೇರಣೆಯಿಂದ ಫಲದಲ್ಲಿ ಆಸಕ್ತನಾಗಿ ಕಟ್ಟು ಬೀಳುತ್ತಾನೆ.}

ಕರ್ಮಫಲದಲ್ಲಿ ಆಸಕ್ತಿಯನ್ನಿಡದೆ ಕೆಲಸ ಮಾಡುವವನು ಶಾಂತಿಯನ್ನು ಪಡೆಯುತ್ತಾನೆ. ಆ ಶಾಂತಿ ಇದ್ದಕ್ಕೆ ಇದ್ದಂತೆ ಬಂದುದಲ್ಲ. ಅದನ್ನು ಅವನು ನಿಷ್ಠೆಯ ಜೀವನದಿಂದ ಪಡೆದಿರುವನು. ಒಮ್ಮೆ ಯಾವಾಗ ಅದನ್ನು ಪಡೆಯುವನೋ ಅದು ಇವನನ್ನು ಇನ್ನುಮೇಲೆ ಬಿಟ್ಟುಹೋಗುವುದಿಲ್ಲ. ಜೀವನದಲ್ಲಿ ಎಂದೆಂದಿಗೂ ಅವನ ಸ್ವಭಾವವಾಗಿ ಉಳಿಯುವುದು. ಅವನ ಶಾಂತಿಗೆ ಈ ಪ್ರಪಂಚ ದಿಂದ ಬೇಕಾದಷ್ಟು ಆತಂಕಗಳು ಬಂದು ಒದಗುತ್ತಿದ್ದರೂ ಅವನು ಇದರಿಂದ ಬಾಧಿತನಾಗುವುದಿಲ್ಲ. ಅವನ ಶಾಂತಿಗೆ ಭದ್ರವಾದ ತಳಪಾಯ ಸಿಕ್ಕಿದೆ. ಎಷ್ಟೇ ಮಳೆ ಬರಲಿ, ಗಾಳಿ ಬೀಸಲಿ ಅದು ಕೊಚ್ಚಿಕೊಂಡು ಹೋಗುವುದಿಲ್ಲ.

ಯುಕ್ತನಾಗದವನು, ತನ್ನ ಮನಸ್ಸನ್ನು ಇನ್ನೂ ನಿಗ್ರಹಿಸದವನು, ಮನದಲ್ಲಿ ಏಳುವ ಹಲವು ಆಕಾಂಕ್ಷೆಗಳಿಗೆ ತುತ್ತಾಗುತ್ತಾನೆ. ಯಾವಾಗ ಫಲಕ್ಕೆ ಕೈಯೊಡ್ಡುತ್ತಾನೆಯೋ ಅದಕ್ಕೆ ದಾಸನಾಗುತ್ತಾನೆ. ಅದು ಹೇಳಿದಂತೆ ಕೇಳಬೇಕಾಗುವುದು. ಕರ್ಮದ ಗೋಜಿನಿಂದ ಅವನು ಪಾರಾಗಲಾರ. ಅವನಿಗೆ ಶಾಶ್ವತವಾದ ಶಾಂತಿ ದೊರಕಲಾರದು. ಯಾರಾದರೂ ಅವನನ್ನು ಟೀಕಿಸಿದರೆ ಸಾಕು, ಉದ್ವಿಗ್ನನಾಗು ವನು, ಹೊಗಳಿದರೆ ಉಬ್ಬಿಹೋಗುವನು. ಲಾಭ ಸಿಕ್ಕಿದರೆ ಕುಣಿದಾಡುವನು, ನಷ್ಟ ಬಂದರೆ ಪೇಚಾಡುವನು. ಇವನ ಜೀವನ ಒಂದು ಬಡ ಜೋಪಡಿಯಂತೆ. ಮಳೆ ಬಂದರೆ ನೀರೆಲ್ಲ ಒಳಗಿರುವುದು. ಸ್ವಲ್ಪ ಜೋರಾಗಿ ಗಾಳಿ ಬೀಸಿದರೆ ಸಾಕು, ಹಾರಿಹೋಗುವುದು. ಪ್ರಪಂಚವನ್ನು ಎದುರಿಸಿ ನಿಲ್ಲಲಾರ.

\begin{verse}
ಸರ್ವಕರ್ಮಾಣಿ ಮನಸಾ ಸಂನ್ಯಸ್ಯಾಸ್ತೇ ಸುಖಂ ವಶೀ~।\\ನವದ್ವಾರೇ ಪುರೇ ದೇಹೀ ನೈವ ಕುರ್ವನ್ ನ ಕಾರಯನ್ \versenum{॥ ೧೩~॥}
\end{verse}

{\small ಸರ್ವಕರ್ಮಗಳನ್ನು ಮನಸ್ಸಿನಿಂದ ತ್ಯಜಿಸಿ, ಜಿತೇಂದ್ರಿಯನಾಗಿ ನವದ್ವಾರಗಳುಳ್ಳ ಪುರದಲ್ಲಿ ಏನನ್ನೂ ಮಾಡದೆ ಮತ್ತು ಮಾಡಿಸದೆ ಸುಖವಾಗಿ ಇರುತ್ತಾನೆ.}

ಕರ್ಮಯೋಗಿ ಸರ್ವಕರ್ಮಗಳನ್ನು ಮನಸ್ಸಿನಿಂದ ತ್ಯಜಿಸುತ್ತಾನೆ. ಅಂದರೆ ಒಳಗೆ ಅವನಿಗೆ ಆ ಕರ್ಮದ ಮೇಲೆ ಯಾವ ಆಸಕ್ತಿಯೂ ಇಲ್ಲ. ಅವನು ಮಾಡುವ ಕರ್ಮ ಅಕರ್ಮಕ್ಕೆ ಸಮನಾದುದು. ಅದರಿಂದ ಅವನನ್ನು ಬಂಧಿಸುವ ಯಾವುದೂ ಉತ್ಪತ್ತಿಯಾಗುವುದಿಲ್ಲ.

ಅವನು ದೇಹದಲ್ಲಿರುವಾಗ ಜಿತೇಂದ್ರಿಯನಾಗಿರುವನು. ಇಂದ್ರಿಯಗಳನ್ನು ಉಪಯೋಗಿಸು ತ್ತಿದ್ದರೂ ಅದಕ್ಕೆ ವಶನಾಗಿರುವುದಿಲ್ಲ. ತಾನು ಹೇಳಿದಂತೆ ಇಂದ್ರಿಯಗಳು ಕೇಳುವಂತೆ ಮಾಡಿ ಕೊಂಡಿರುವನು. ಇಂದ್ರಿಯಗಳನ್ನು ಬಂಧನಕ್ಕೆ ಬೀಳದ ರೀತಿ ಉಪಯೋಗಿಸುವನು. ಅವನು ಈ ದೇಹವೆಂಬ ಒಂಬತ್ತು ಬಾಗಿಲುಗಳುಳ್ಳ ಪುರದಲ್ಲಿ ಸುಖವಾಗಿರುತ್ತಾನೆ. ಆ ಸುಖಕ್ಕೆ ಕಾರಣ ಇವನು ಏನನ್ನೂ ಸ್ವಾರ್ಥಕ್ಕೆ ಮಾಡದೇ ಇರುವುದು, ಮತ್ತುಇತರರಿಂದ ತನಗಾಗಿ ಏನನ್ನೂ ಮಾಡಿಸದೇ ಇರುವುದು. ಯಾವ ಸ್ವಾರ್ಥಕ್ಕೆ ತುತ್ತಾಗಿ, ತಾತ್ಕಾಲಿಕ ಸುಖದಿಂದ ಪ್ರೇರೇಪಿತನಾಗಿ ಒಂದು ಕ್ರಿಯೆಯನ್ನು ಮಾಡುತ್ತಾನೆಯೊ ಆಗ ಅದಕ್ಕೆ ಕಟ್ಟುಬೀಳುತ್ತಾನೆ. ಅವನು ಅದನ್ನು ಪುನಃ ಪುನಃ ಮಾಡುತ್ತ ಇರಬೇಕಾಗುವುದು. ಇದು ಅವನ ಅಭ್ಯಾಸವಾಗುವುದು. ಅಭ್ಯಾಸದಷ್ಟು ಭಯಂಕರವಾದ ಬಂಧನ ಮತ್ತಾವುದೂ ಇಲ್ಲ. ಇದು ಹೊರಗಿನ ಎಲ್ಲಾ ಸೆರೆಮನೆಗಳಿಗಿಂತಲೂ ಘೋರವಾಗಿರು ವುದು. ಅದರಲ್ಲಿ ದುರಭ್ಯಾಸದಷ್ಟು ಬಲವಾದ ಬಂಧನ ಮತ್ತೊಂದಿಲ್ಲ.

ಕೆಲವು ವೇಳೆ ನಾವೇ ಆ ಕೆಲಸವನ್ನು ಮಾಡುವುದಿಲ್ಲ. ಏಕೆಂದರೆ ಎಲ್ಲಿ ಸಿಕ್ಕಿಬೀಳುವೆವೋ ಎಂಬ ಅಂಜಿಕೆ. ಮತ್ತು ಅದನ್ನು ಮಾಡುವ ಶಕ್ತಿಯೂ ಇಲ್ಲ. ಅದಕ್ಕಾಗಿ ನಾವು ಮತ್ತೊಬ್ಬನಿಂದ ಆ ಕೆಲಸವನ್ನು ಮಾಡಿಸಿಕೊಳ್ಳುತ್ತೇವೆ. ಅವನಿಗೆ ಏನಾದರೂ ಲಂಚವನ್ನು ಕೊಟ್ಟು ನನಗಾಗಿ ಅವನು ಕೆಲಸವನ್ನು ಮಾಡಿಕೊಡಬೇಕೆಂದು ಕೇಳಿಕೊಳ್ಳುತ್ತೇನೆ. ನಾನೇ ಮಾಡುವುದು ತಪ್ಪಾದರೆ ಅದನ್ನು ನಾನು ಮಾಡದೆ ಮತ್ತೊಬ್ಬನ ಕೈಯಿಂದ ಮಾಡಿಸಿಕೊಳ್ಳುವುದು, ಅದರಷ್ಟೇ ಅಥವಾ ಅದಕ್ಕಿಂತ ದೊಡ್ಡ ತಪ್ಪು. ಕೊಲೆ ಮಾಡುವುದು ಎಷ್ಟು ತಪ್ಪೋ, ಕೊಲೆ ಮಾಡುವಂತೆ ಸಲಹೆ ಕೊಡುವುದು, ಅದಕ್ಕೆ ಸಹಾಯ ಮಾಡುವುದು, ಕೊಲೆಗಿಂತ ಭೀಕರವಾಗಿರುವುದು.

ಯಾವಾಗ ಯೋಗಿ ಏನನ್ನೂ ಮಾಡುವುದಿಲ್ಲವೋ, ಬಂಧನಕ್ಕೆ ಕಟ್ಟಿಹಾಕುವಂತಹ ಸಂಸ್ಕಾರಗಳು ಅವನಲ್ಲಿ ಇಲ್ಲದೇ ಇರುವುದರಿಂದ, ಅದರ ಪರಿಣಾಮಕ್ಕೆ ಅವನು ಒಳಗಾಗುವುದಿಲ್ಲ. ಯಾವುದನ್ನು ತಾನು ಮಾಡುವುದಕ್ಕೆ ಇಚ್ಛೆಪಡುವುದಿಲ್ಲವೋ ಅದನ್ನು ಅವನು ಇತರರ ಕೈಯಿಂದ ಎಂದೂ ಮಾಡಿಸನು. ತನ್ನ ಸುಖಕ್ಕಾಗಿ ಇತರರಿಗೆ ಅವನು ಕಷ್ಟವನ್ನು ಕೊಡಬಯಸುವುದಿಲ್ಲ. ಅವನು ಪರಮಸುಖಿ. ಸಿಕ್ಕಿದುದರಲ್ಲಿ ತೃಪ್ತನಾಗಿರುವನು. ಅಯ್ಯೊ ಇಷ್ಟೆ ಬಂತಲ್ಲ, ಇನ್ನೂ ಹೆಚ್ಚು ಬರಬೇಕಾಗಿತ್ತು ಎಂದು ಆಸೆ ಪಡುವುದಿಲ್ಲ. ಅವನ ಸುಖ ಇಂದ್ರಿಯ ವಿಷಯಳ ಆಧಾರದ ಮೇಲೆ ಇಲ್ಲ. ಅನ್ಯಾಶ್ರಯದ ಸುಖವಿದ್ದರೆ, ಆ ವಸ್ತುಇಲ್ಲದೇ ಇದ್ದರೆ ಆ ಸುಖವೂ ಇರುವುದಿಲ್ಲ. ಯೋಗಿಯಾದರೋ ಆತ್ಮಾರಾಮ. ಅವನಿಗೆ ಹೊರಗಿನಿಂದ ಏನೂ ಬೇಕಾಗಿಲ್ಲ. ಯಾವುದು ಬಂದರೂ ಒಂದೇ, ಯಾವುದು ಹೋದರೂ ಒಂದೇ, ಯಾವ ಬಾಹ್ಯ ಘಟನೆಯೂ ಅವನ ಅಂತಃಸುಖಕ್ಕೆ ಧಕ್ಕೆಯನ್ನು ತರಲಾರದು.

\begin{verse}
ನ ಕರ್ತೃತ್ವಂ ನ ಕರ್ಮಾಣಿ ಲೋಕಸ್ಯ ಸೃಜತಿ ಪ್ರಭುಃ~।\\ನ ಕರ್ಮಫಲಸಂಯೋಗಂ ಸ್ವಭಾವಸ್ತು ಪ್ರವರ್ತತೇ \versenum{॥ ೧೪~॥}
\end{verse}

{\small ಪ್ರಭುವು ಜನರಿಗೆ ಕರ್ತೃತ್ವವನ್ನಾಗಲಿ, ಕರ್ಮವನ್ನಾಗಲಿ ಸೃಷ್ಟಿಸಿಲ್ಲ, ಕರ್ಮಫಲಸಂಯೋಗವನ್ನೂ ಸೃಷ್ಟಿಸಿಲ್ಲ. ಸ್ವಭಾವವೇ ಕೆಲಸ ಮಾಡುತ್ತಿರುವುದು.}

ದೇವರು ಜೀವಿಗೆ, ಇಂತಹ ಕೆಲಸವನ್ನು ಮಾಡು ಎಂದು ಅದರ ಕರ್ತೃತ್ವವನ್ನಾಗಲಿ, ಕರ್ಮ ವನ್ನಾಗಲಿ ಸೃಷ್ಟಿಸಿಲ್ಲ. ಹಾಗೆ ಅವನು ಏನಾದರೂ ಮಾಡಿದರೆ ಪಕ್ಷಪಾತಿ ಆಗುತ್ತಾನೆ. ಅವನು ಕೆಲವರಿಗೆ ಒಳ್ಳೆಯ ಕೆಲಸವನ್ನು ಮಾಡುವಂತೆ ಪ್ರೇರೇಪಿಸುತ್ತಾನೆ ಎಂದರೆ, ದೇವರೊ ಒಬ್ಬರ ಕಣ್ಣಿಗೆ ಬೆಣ್ಣೆ, ಮತ್ತೊಬ್ಬರ ಕಣ್ಣಿಗೆ ಸುಣ್ಣ ಏತಕ್ಕೆ ಇಡುತ್ತಾನೆ ಎಂದು ಪ್ರಶ್ನಿಸಬೇಕಾಗುವುದು. ಅವನು ಕರ್ಮಫಲದಾತ. ಜೀವರು ಯಾವಯಾವ ಕೆಲಸವನ್ನು ಮಾಡುವರೊ ಅವರವರಿಗೆ ತಕ್ಕ ಫಲಗಳನ್ನು ಕೊಡುತ್ತಾನೆ. ದೇವರು ಎಲ್ಲರಿಗೂ ಕೆಲಸ ಮಾಡುವುದಕ್ಕೆ ಒಂದು ಅವಕಾಶವನ್ನೇನೋ ಕಲ್ಪಿಸುವನು. ಆದರೆ ಮಾಡುವ ಒಳ್ಳೆಯ ಮತ್ತು ಕೆಟ್ಟ ಕೆಲಸಗಳಿಗೆ ಅವನಲ್ಲ ಕಾರಣ. ಶ‍್ರೀರಾಮಕೃಷ್ಣರು, ಒಬ್ಬ ಒಂದು ದೀಪದ ಬೆಳಕಿನಲ್ಲಿ ಗೀತಾ ಪಾರಾಯಣ ಮಾಡಬಹುದು, ಮತ್ತೊಬ್ಬ ಅದೇ ಬೆಳಕಿನಲ್ಲಿ ಕುಳಿತುಕೊಂಡು ಒಂದು ಸುಳ್ಳು ಅರ್ಜಿಯನ್ನು ಬರೆಯಬಹುದು, ಅದಕ್ಕೆ ದೀಪವನ್ನು ದೂರುವುದಕ್ಕೆ ಆಗುವುದಿಲ್ಲ ಎನ್ನುತ್ತಿದ್ದರು. ಮಳೆ ಪಕ್ಷಪಾತವಿಲ್ಲದೆ ಬೀಳುವುದು. ಯಾರು ಜಮೀನನ್ನು ಉತ್ತಿ ಬಿತ್ತಿರುವರೋ ಅಲ್ಲಿ ಚೆನ್ನಾಗಿ ಬೆಳೆ ಬರುವುದು. ಯಾರು ಉತ್ತಿಲ್ಲವೋ, ಅಲ್ಲಿರುವ ಮುಳ್ಳಿನ ಬೀಜಗಳೇ ಹುಲುಸಾಗಿ ಬರುವುವು. ಇದಕ್ಕೆ ಮಳೆಯನ್ನು ಬೈದು ಪ್ರಯೋಜನವೇನು? ಅದರಂತೆಯೇ ಭಗವತ್​ಕೃಪೆ ಎಲ್ಲರ ಮೇಲೂ ಒಂದೇ ಸಮನಾಗಿಬೀಳುತ್ತದೆ. ಆದರೆ ಪ್ರತಿಯೊಬ್ಬನೂ ತನ್ನ ಸ್ವಭಾವಕ್ಕೆ ತಕ್ಕಂತೆ ಅದನ್ನು ಉಪಯೋಗಿಸಿಕೊಳ್ಳುತ್ತಿರುವನು.

ಇಲ್ಲಿ ನಾವು ಮಾಡುವ ಒಳ್ಳೆಯ ಮತ್ತು ಕೆಟ್ಟ ಕೆಲಸಗಳನ್ನು ನಿಷ್ಕರ್ಷಿಸುವುದು ಯಾವುದು ಎಂದರೆ ಅದೇ ನಮ್ಮಲ್ಲಿರುವ ಸ್ವಭಾವ. ಆ ಸ್ವಭಾವವನ್ನು ಯಾರೋ ನಾವು ಹುಟ್ಟಿದಾಗ ನಮ್ಮಲ್ಲಿಟ್ಟು ಕಳುಹಿಸಲಿಲ್ಲ. ನಾವೇ ಆ ಸ್ವಭಾವವನ್ನು ತಯಾರುಮಾಡಿಕೊಂಡವರು. ಹಲವು ರೀತಿ ಆಲೋಚನೆ ಮಾಡಿ, ಮನನ ಮಾಡಿ, ಒಂದು ಕೆಲಸವನ್ನು ಮಾಡಿ, ಅದರಿಂದ ನಮ್ಮ ಸ್ವಭಾವ ಎಂಬುದು ಆಗಿದೆ. ನಾವು ಆ ಸ್ವಭಾವದ ಕೈಯಲ್ಲಿ ಸಿಕ್ಕಿಬಿದ್ದು ಅದು ಹೇಳಿದಂತೆ ಕೇಳುತ್ತೇವೆ. ಸ್ವಭಾವವೇ ಮನುಷ್ಯನ ಮುಕ್ಕಾಲುಪಾಲು. ಅದರ ಅಧೀನ ನಾವೆಲ್ಲ. ನಾವು ಬದಲಾಯಿಸಬೇಕಾದರೆ ಆ ಸ್ವಭಾವವನ್ನು ಬದಲಾಯಿಸಬೇಕು. ಆ ಮುಕ್ಕಾಲುಪಾಲು ಸ್ವಭಾವವನ್ನು ಎಲ್ಲೋ ಅಲ್ಪಭಾಗವಾಗಿರುವ ಈಗಿನ ಪರಿಸ್ಥಿತಿ ಹೇಗೆ ಬದಲಾಯಿಸಬಲ್ಲುದು? ಎಂದೆಂದಿಗೂ ನಾವು ಅದು ಹೇಳಿದಂತೆ ಕೇಳಿಕೊಂಡು ಹೆಚ್ಚುಹೆಚ್ಚಾಗಿ ಅದು ಪ್ರೇರೇಪಿಸಿದಂತೆ ಮಾಡುತ್ತಿದ್ದರೆ, ಅದರಿಂದ ತಪ್ಪಿಸಿಕೊಂಡು ಬರುವುದಕ್ಕೆ ದಾರಿಯೇ ಇಲ್ಲವೇ ಎಂದು ಹತಾಶರಾಗುತ್ತೇವೆ. ಆ ಸ್ವಭಾವವನ್ನೂ ಕೂಡ ಈಗಿನಿಂದ ಪ್ರಯತ್ನ ಪಡುತ್ತಿದ್ದರೆ ಕ್ರಮೇಣ ಬದಲಾಯಿಸಬಹುದು. ಆದರೆ ಸುಲಭವಾಗಿ ಆಗುವ ಕೆಲಸವಲ್ಲ ಇದು. ಬಹಳ ಆಳಕ್ಕೆ ಹೋಗಿ ಕುಳಿತಿರುವುದು ಈ ಸ್ವಭಾವ. ಇದರೊಡನೆ ಸತತ ಹೋರಾಟವಾಡುತ್ತಿದ್ದರೆ ಮಾತ್ರ ಕ್ರಮೇಣ ನಾವು ಉತ್ತಮ ಸ್ವಭಾವವನ್ನು ರೂಢಿಸಬಹುದು.

ಇಲ್ಲಿ ಶ‍್ರೀಕೃಷ್ಣ ನಮ್ಮ ಪರಿಸ್ಥಿತಿಗೆ, ಯಾವುದೋ ದೇವರನ್ನು ದೂರುವುದಿಲ್ಲ, ಅದೃಷ್ಟವನ್ನು ಹಳಿಯುವುದಿಲ್ಲ. ನಮ್ಮ ಸಮಾಜವನ್ನಾಗಲಿ, ತಂದೆತಾಯಿಗಳನ್ನಾಗಲಿ, ದೂರುವುದಿಲ್ಲ. ಏನೋ ಅಕಸ್ಮಾತ್ ಹೀಗೆ ಆಯಿತು ಎಂತಲೂ ಹೇಳುವುದಿಲ್ಲ. ಇದು ನಮ್ಮಿಂದಲೇ ಆಗಿದೆ. ನಾವು ಏನನ್ನು ಬಿತ್ತಿದ್ದೆವೋ ಅದನ್ನು ಕುಯ್ಯಬೇಕಾಗಿದೆ. ಬಿತ್ತಿದಾಗ ಇಂತಹ ಕಹಿ ಫಲ ಬರುತ್ತದೆ ಎಂದು ಭಾವಿಸಿರಲಿಲ್ಲ. ಆದರೆ ಈಗ ಅದು ಗೊತ್ತಾಗಿದೆ. ಮುಂದೆ ಹೀಗೆ ಮಾಡದೆ ಇರುವ ಎಂದು ಪ್ರತಿಜ್ಞೆ ಮಾಡಿ, ಅದರಂತೆ ನಡೆದರೆ ಮಾತ್ರ ಭವ್ಯವಾದ ಭವಿಷ್ಯವನ್ನು ನಾವು ರೂಪಿಸಿಕೊಳ್ಳಬಹುದು. ನಮ್ಮ ಭವಿಷ್ಯದ ಶಿಲ್ಪಿಗಳು ಯಾವುದೋ ನಮ್ಮಿಂದ ಹೊರಗೆ ಇರುವ ಅದೃಷ್ಟವಲ್ಲ, ನಾವೆ.

\begin{verse}
ನಾದತ್ತೇ ಕಸ್ಯಚಿತ್ ಪಾಪಂ ನ ಚೈವ ಸುಕೃತಂ ವಿಭುಃ~।\\ಅಜ್ಞಾನೇನಾವೃತಂ ಜ್ಞಾನಂ ತೇನ ಮುಹ್ಯಂತಿ ಜಂತವಃ \versenum{॥ ೧೫~॥}
\end{verse}

{\small ವಿಭು ಯಾರೊಬ್ಬರ ಪಾಪವನ್ನಾಗಲೀ ಪುಣ್ಯವನ್ನಾಗಲೀ ತೆಗೆದುಕೊಳ್ಳುವುದಿಲ್ಲ. ಜ್ಞಾನವು ಅಜ್ಞಾನದಿಂದ ಮುಚ್ಚಲ್ಪಟ್ಟಿದೆ. ಆದುದರಿಂದ ಪ್ರಾಣಿಗಳು ಮೋಹಗೊಳ್ಳುತ್ತಿವೆ.}

ದೇವರು ಜೀವಿಗಳಲ್ಲಿ ಪಾಪ ಪುಣ್ಯಗಳನ್ನು ಇಟ್ಟಿದ್ದರೆ ಅದಕ್ಕೆ ಅವನು ಜವಾಬ್ದಾರನಾಗಬೇಕಾಗು ವುದು. ಆಗ ಅವನನ್ನು ಪಕ್ಷಪಾತಿಯೆಂದು ದೂರಬಹುದು. ಜನ್ಮಜನ್ಮಾಂತರದಿಂದ ಸಂಗ್ರಹಿಸಿ ಕೊಂಡು ಬಂದಿರುವ ಆ ಬೀಜಗಳ ವಿಕಾಸಕ್ಕೆ ಮಾತ್ರ ಅವನು ಅವಕಾಶವನ್ನು ಕೊಡುವನು. ನಾವು ಬೆಂಕಿಯಲ್ಲಿ ಅಡಿಗೆಯನ್ನಾದರೂ ಮಾಡಬಹುದು. ಅಥವಾ ಕೈಯನ್ನು ಬೇಕಾದರೆ ಸುಟ್ಟುಕೊಳ್ಳ ಬಹುದು. ಅದಕ್ಕೆ ಬೆಂಕಿ ಹೊಣೆಯಲ್ಲ.

ಎಲ್ಲಾ ಜೀವರಾಶಿಗಳಲ್ಲಿಯೂ ದೇವರು ಜ್ಞಾನದ ಹಣತೆಯನ್ನು ಇಟ್ಟಿರುವನು. ನಾವೇ ಅದನ್ನು ಕಾಣದಂತೆ ಅಜ್ಞಾನದ ತೆರೆಗಳಿಂದ ಮುಚ್ಚಿರುವೆವು. ಮನುಷ್ಯನು ದೇವರು ಕೊಟ್ಟ ಬೆಳಕನ್ನು ಕಾಣದಂತೆ ಮಾಡಿಕೊಂಡು ಅಜ್ಞಾನದಲ್ಲಿ ತೆವಳುತ್ತಿರುವನು. ಎದುರಿಗೆ ಸೂರ್ಯನಿರುವನು ಆಕಾಶದಲ್ಲಿ. ನಾವು ಕೈಗಳಿಂದ ಕಣ್ಣನ್ನು ಮುಚ್ಚಿಕೊಂಡು ಅದು ನಮಗೆ ಕಾಣುತ್ತಿಲ್ಲ ಎಂದು ಹೇಳುತ್ತಿರುವೆವು. ನಮ್ಮಲ್ಲಿ ಈಗಿರುವ ಅಜ್ಞಾನಕ್ಕೂ ನಾವು ದೇವರನ್ನು ದೂರಬೇಕಾಗಿಲ್ಲ. ಅದನ್ನು ನಾವೇ ಮಾಡಿಕೊಂಡ ವರು. ಯಾವುದನ್ನು ನಾವು ಮಾಡಿಕೊಂಡಿರುವೆವೊ ಅದರ ಮೇಲೆ ನಮಗೆ ಸ್ವಾಧೀನವಿದೆ. ಅದನ್ನು ಬೇಕಾದರೆ ಬೇರೆ ರೀತಿ ಮಾಡಿಕೊಳ್ಳಬಹುದು. ಒಂದು ರೀತಿ ಮಾಡಿಕೊಂಡು ಅಜ್ಞಾನದಲ್ಲಿ ನರಳುತ್ತಿರುವೆವು. ಇನ್ನೊಂದು ರೀತಿ ಮಾಡಿಕೊಂಡು ಅಜ್ಞಾನದಿಂದ ಬಿಡಿಸಿಕೊಂಡು ಬರಬಹುದು. ರೇಶ್ಮೆಯ ಹುಳು ತನ್ನಿಂದಲೇ ನೂಲನ್ನು ನೆಯ್ದು ಅದರ ಗೂಡಿನಲ್ಲಿ ಅದೇ ಬಂದಿಯಾಗುವುದು. ಹೊರಗಿನವರು ಅದನ್ನು ಕೂಡಿಹಾಕುವುದಿಲ್ಲ. ಅದು ಒಳಗೆ ಇದ್ದು ಬೇಜಾರಾಗಿ ತಾನೇ ಬೇಕಾದರೆ ಗೂಡಿನಿಂದ ಸೀಳಿಕೊಂಡು ಹೊರಗೆ ಬರಬಹುದು. ಯಾವ ಬಂಧನವನ್ನು ನಾವೇ ನೆಯ್ದುಕೊಂಡಿರು ವೆವೊ ಅದನ್ನು ನಾವೇ ಬಿಚ್ಚಿಕೊಳ್ಳಲೂ ಸಾಧ್ಯ. ಹಾಗೆ ಬಿಚ್ಚಿಕೊಳ್ಳಲು ಏನೇನು ಸೌಲಭ್ಯಗಳು ಬೇಕೊ ದೇವರು ಅದನ್ನೆಲ್ಲ ಕೊಡುವನು. ಬಿಚ್ಚಿಕೊಳ್ಳಬೇಕೆಂಬ ಆಸೆ ನಮ್ಮಲ್ಲಿ ತೀವ್ರವಾಗಿರಬೇಕು. ಆಗ ಉಳಿದವುಗಳೆಲ್ಲವೂ ಒದಗಿಬರುವುವು. ಹಾಗೆ ಒದಗಿ ಬಂದುದನ್ನು ಉಪಯೋಗಿಸಿಕೊಳ್ಳುವುದು ನಮ್ಮಲ್ಲಿದೆ.

\begin{verse}
ಜ್ಞಾನೇನ ತು ತದಜ್ಞಾನಂ ಯೇಷಾಂ ನಾಶಿತಮಾತ್ಮನಃ~।\\ತೇಷಾಮಾದಿತ್ಯವಜ್ಜ್ಞಾನಂ ಪ್ರಕಾಶಯತಿ ತತ್ಪರಮ್ \versenum{॥ ೧೬~॥}
\end{verse}

{\small ಆದರೆ ಯಾರ ಅಜ್ಞಾನ ಆತ್ಮಜ್ಞಾನದ ಮೂಲಕ ನಾಶವಾಗಿದೆಯೊ, ಅವರ ಆತ್ಮಜ್ಞಾನವು ಸೂರ್ಯನ ಹಾಗೆ ಪರಬ್ರಹ್ಮವನ್ನು ಪ್ರಕಾಶಿಸುವುದು.}

ನಮ್ಮ ಸುತ್ತಲಿರುವ, ನಮ್ಮನ್ನು ಮುತ್ತಿರುವ, ಅಜ್ಞಾನವನ್ನು ನಾವು ತೊಳೆದುಕೊಳ್ಳಬೇಕಾಗಿದೆ. ಯಾವಾಗ ನಾವು ಅದನ್ನು ತೊಳೆದುಕೊಳ್ಳುತ್ತೇವೆಯೊ ಆಗ ಜ್ಞಾನ ತನ್ನ ಸ್ವಭಾವಕ್ಕೆ ತಕ್ಕಂತೆ ಬೆಳಗುವುದು. ಆ ಬೆಳಕಿನ ಮೂಲಕ ಪರಮಾತ್ಮನನ್ನು ತಿಳಿದುಕೊಳ್ಳುತ್ತೇವೆ. ಮುಂಚೆ ನಾವು ಮಾಡಬೇಕಾಗಿರುವುದು ನಮ್ಮ ಅಜ್ಞಾನದಿಂದ ಪಾರಾಗುವುದು. ನಮ್ಮ ಬುದ್ಧಿ ಯಾವಾಗ ತಿಳಿ ಆಗುವುದೊ, ಅದಕ್ಕೆ ಆಗ ಒಂದು ಹೊಸ ಶಕ್ತಿ ಬಂದಂತೆ ಆಗುವುದು. ಅದನ್ನು ಯಾವುದರ ಮೇಲೆ ಬೀರಿದರೆ, ಅದು ತನ್ನ ರಹಸ್ಯವನ್ನು ನಮಗೆ ಕೊಡುವುದು. ಕೇಂದ್ರೀಕರಿಸಿದ ಪರಿಶುದ್ಧವಾದ ಜ್ಞಾನ, ಸೂರ್ಯನಂತೆ. ಅದರ ಬೆಳಕಿನ ಎದುರಿಗೆ ಯಾವ ಅಜ್ಞಾನವೂ ಇರಲಾರದು. ಕತ್ತಲಲ್ಲಿ ನಾವು ವಸ್ತುವನ್ನು, ಅದು ಒಂದಾಗಿದ್ದರೆ ಮತ್ತೊಂದು ಎಂದು ಭ್ರಮಿಸಿದ್ದೆವು. ಈಗ ಬೆಳಕಿನಲ್ಲಿ ವಸ್ತುವನ್ನು ಹೇಗಿದೆಯೋ ಹಾಗೆ ನೋಡುವೆವು. ಅದು ಯಾವ ರಹಸ್ಯವನ್ನು ಬೇಕಾದರೂ ಭೇದಿಸಿಕೊಂಡು ಹೋಗಬಹುದು. ನಾವು ಒಂದು ದೇಹಕ್ಕೆ ಬದ್ಧರಾಗಿ, ಅದರ ಇಂದ್ರಿಯ ಮನಸ್ಸಿನೊಂದಿಗೆ ತಾದಾತ್ಮ್ಯಭಾವವನ್ನು ಹೊಂದಿ ದೇಹಧರ್ಮವನ್ನೆಲ್ಲ ನಾವು ಆರೋಪಮಾಡಿಕೊಂಡು ನರಳುತ್ತಿರು ವುದು ಅಜ್ಞಾನದಿಂದ. ಜ್ಞಾನಜ್ಯೋತಿ ನಮ್ಮನ್ನು ಇದರಿಂದ ಪಾರುಮಾಡುವುದು. ನಿಜವಾಗಿ ನಾನು ಯಾರು ಎಂಬುದನ್ನು ತೋರುವುದು. ಅದರಂತೆಯೆ ನನ್ನಿಂದ ಹೊರಗೆ ಇರುವ ಪ್ರಪಂಚದ ನೈಜಸ್ಥಿತಿ ನಮಗೆ ಅರ್ಥವಾಗುತ್ತಿಲ್ಲ. ಅದು ನಾಮರೂಪದಿಂದ ಆರೋಪಿತವಾಗಿದೆ. ನಾವು ವ್ಯವಹಾರ ನಡೆಸುತ್ತಿರುವುದು ನಾಮರೂಪದೊಡನೆ, ಇಲ್ಲಿ ನಮಗೆ ಒಂದು ಬೇಕು, ಮತ್ತೊಂದು ಬೇಡ. ಆದರೆ ನಾಮರೂಪವನ್ನು ತೂರಿ, ಹಿಂದೆ ಇರುವ ಶಕ್ತಿಯನ್ನು ನೋಡಬಲ್ಲೆವಾದರೆ ವಸ್ತುವಿನ ರಹಸ್ಯ ನಮಗೆ ಗೊತ್ತಾಗುವುದು. ಯಾವಾಗ ನಮ್ಮ ಮನಸ್ಸನ್ನು ಅಂತರ್ಮುಖ ಮಾಡುವೆವೊ, ಆಗ ಅಂತರ್ಮುಖ ವಾದ ಮನಸ್ಸಿನ ಕೇಂದ್ರೀಕೃತ ಶಕ್ತಿ, ಪರಮಾತ್ಮನ ಸ್ವಭಾವವನ್ನು ನಮಗೆ ಅರಿಯುವಂತೆ ಮಾಡು ವುದು. ಇಲ್ಲಿ ಜೀವಿ ತನ್ನಲ್ಲಿರುವ ಬೆಳಕಿನಿಂದ ಪರಬ್ರಹ್ಮನನ್ನು ಬೆಳಗುತ್ತಾನೆಂದು ನಾವು ಅಕ್ಷರಶಃ ತೆಗೆದುಕೊಳ್ಳಬೇಕಾಗಿಲ್ಲ. ಪರಬ್ರಹ್ಮನೇ ಸ್ವಯಂಪ್ರಕಾಶ ಸ್ವರೂಪನು. ಅವನು ಬೆಳಗುತ್ತಿರುವುದ ರಿಂದ ನಾವೆಲ್ಲ ಬೆಳಗುತ್ತೇವೆ. ಅವನು ಕೊಟ್ಟಿರುವ ಬೆಳಕಿನಿಂದ ನಾವು ಬೆಳಗುವುದು. ಸೂರ್ಯನನ್ನು ನೋಡಲು ಟಾರ್ಚ್​ಲೈಟಿನ ಆವಶ್ಯಕತೆ ಬೇಕಾಗಿಲ್ಲ. ನಾವು ಯಾವುದೊ ಕತ್ತಲೆ ಕೋಣೆಯಲ್ಲಿ ಸೇರಿಕೊಂಡಿರುವುದರಿಂದ, ಅದರಿಂದ ಹೊರಗೆ ಬರಲು ಟಾರ್ಚ್​ಲೈಟ್ ಬೇಕು. ಹೊರಗೆ ಬಂದ ಮೇಲೆ ಜಗತ್ತನ್ನೆಲ್ಲ ಬೆಳಗುವ ರವಿಯ ದರ್ಶನವಾಗುವುದು. ಆತ್ಮಜ್ಞಾನ ಕ್ರಮೇಣ ನಮ್ಮನ್ನು ಪರಮಾತ್ಮನ ಕಡೆಗೆ ಒಯ್ಯುವುದು ಹೀಗೆ.

\begin{verse}
ತದ್ಬುದ್ಧಯಸ್ತದಾತ್ಮಾನಸ್ತನ್ನಿಷ್ಠಾಸ್ತತ್ಪರಾಯಣಾಃ~।\\ಗಚ್ಛಂತ್ಯಪುನರಾವೃತ್ತಿಂ ಜ್ಞಾನನಿರ್ಧೂತಕಲ್ಮಷಾಃ \versenum{॥ ೧೭~॥}
\end{verse}

{\small ತದ್​ಬುದ್ಧಿಯುಳ್ಳವರಾಗಿ, ತದಾತ್ಮರಾಗಿ, ತನ್ನಿಷ್ಠರಾಗಿ, ತತ್ಪರಾಯಣರಾಗಿ ಜ್ಞಾನದಿಂದ ಕಲ್ಮಷಗಳನ್ನೆಲ್ಲ ತೊಳೆದುಕೊಂಡು ಮತ್ತೆ ಹಿಂತಿರುಗದಿರುವಿಕೆಯನ್ನು ಪಡೆಯುತ್ತಾರೆ.}

ಅವರ ಬುದ್ಧಿ ಪರಮಾತ್ಮನಲ್ಲಿಯೇ ನಿರತವಾಗಿದೆ. ಅದಕ್ಕೆ ಅವನ ಇರುವಿಕೆಯಲ್ಲಿ ಯಾವ ಸಂದೇಹವೂ ಇಲ್ಲ. ಏಕೆಂದರೆ ಅವರು ಅವನನ್ನು ಅನುಭವಿಸಿರುವರು. ಇನ್ನುಮೇಲೆ ಅಲ್ಲಿ ಸಂದೇಹಕ್ಕೆ ಆಸ್ಪದವಿರುವುದಿಲ್ಲ. ಬುದ್ಧಿ ಎಲ್ಲಾ ಸಂದೇಹಗಳಿಂದಲೂ ಪಾರಾದ ಸ್ಥಿತಿ. ಅವರ ಬುದ್ಧಿ ಉತ್ತರಮುಖಿ ಹೇಗೆ ಯಾವಾಗಲೂ ಉತ್ತರದಿಕ್ಕನ್ನೇ ತೋರುತ್ತಿರುವುದೋ ಹಾಗೆ ಯಾವಾಗಲೂ ದೇವರ ಕಡೆಗೆ ತಿರುಗುವುದು, ಅವನ ಸಮರ್ಥನೆ ಮಾಡುವುದು, ಅವನಿಲ್ಲದ ವಸ್ತುವನ್ನು ನಿರಾಕರಿಸು ವುದು. ಇನ್ನುಮೇಲೆ ಅವರ ಬುದ್ಧಿ ತಪ್ಪು ಹಾದಿಗೆ ಎಳೆಯಲಾರದು.

ತದಾತ್ಮ್ಯ, ಅವನ ಮನಸ್ಸೆಲ್ಲ ಪರಮಾತ್ಮನಿಂದ ತುಂಬಿ ತುಳುಕಾಡುತ್ತಿರುವುದು. ನಾವೊಂದು ಸ್ಪಂಜನ್ನು ನೀರಿನಲ್ಲಿ ಹಾಕಿದರೆ ಅದು ಹೇಗೆ ನೀರನ್ನು ತನ್ನ ಎಲ್ಲಾ ರಂಧ್ರಗಳಿಂದ ಹೀರಿಕೊಳ್ಳು ವುದೊ, ಹಾಗೆ ಅವನ ಮನಸ್ಸು ಭಗವಂತನಿಂದ ತುಂಬಿ ತುಳುಕುತ್ತಿದೆ. ಹಾಗಿರುವಾಗ ಪರಮಾತ್ಮನಿಗೆ ಕುರಿತದ್ದನ್ನು ನೋಡುತ್ತಾನೆ. ಜಿಲೇಬಿಯನ್ನು ತುಪ್ಪದಲ್ಲಿ ಕರಿದು ಸಕ್ಕರೆ ಪಾಕದಲ್ಲಿ ಅದ್ದಿದರೆ ಅದು ಪಾಕವನ್ನು ಹೇಗೆ ಹೀರಿಕೊಳ್ಳುವುದೊ, ಹಾಗೆ ಪರಮಾತ್ಮನನ್ನು ಅರಿತವನ ವ್ಯಕ್ತಿತ್ವ ಆಗುವುದು. ಅವನ ಒಳಗೆ ಹೊರಗೆ ಎಲ್ಲಾ ಅವನೇ ಆಗಿರುವನು.

ಅದರಲ್ಲಿಯೇ ನಿಷ್ಠನಾಗಿದ್ದಾನೆ. ಪರಮಾತ್ಮನ ಆಧಾರದ ಮೇಲೆ ತನ್ನ ಜೀವನವನ್ನು ರೂಪಿಸಿ ಕೊಳ್ಳುವನು. ಯಾವಾಗಲೂ ಏನನ್ನು ಮಾಡುತ್ತಿದ್ದರೂ ಪರಮಾತ್ಮನನ್ನು ಮರೆಯುವುದಿಲ್ಲ. ಅವನು ಮಾಡುವ ಕ್ರಿಯೆಗಳೆಲ್ಲ ಭಗವಂತನನ್ನು ಸೇರುವುದಕ್ಕೆ, ಅವನೊಡನೆ ಬಾಳುವುದಕ್ಕೆ, ಬದುಕುವುದಕ್ಕೆ.

ಅದರಲ್ಲಿಯೇ ಪರಾಯಣನಾಗಿದ್ದಾನೆ. ಅಂದರೆ, ದೇವರಲ್ಲಿಯೇ ಮುಳುಗಿರುವನು. ದೇವರನ್ನೇ ಸಂಪೂರ್ಣ ಆಶ್ರಯಿಸಿರುವನು. ಅವನಿಗೆ ಇನ್ನು ಬೇರೆ ಆಶ್ರಯ ಬೇಕಿಲ್ಲ. ಜೀವನದ ಪರಮ ಗಂತವ್ಯ ಪರಮಾತ್ಮ. ಅವನನ್ನು ಪಡೆದಮೇಲೆ ಮತ್ತಾವುದೂ ಅವನಿಗೆ ಬೇಕೆನಿಸುವುದಿಲ್ಲ. ಅವನು ಪರ ಮಾತ್ಮನ ಚಿಂತೆಯಲ್ಲಿಯೇ ಮುಳುಗಿ ಹೋಗಿದ್ದಾನೆ. ಅವನ ವ್ಯಕ್ತಿತ್ವವೆಲ್ಲ ಅವನನ್ನು ಸ್ಪಂದಿಸು ತ್ತಿದೆ. ಅವನ ಹಳೆಯದೆಲ್ಲ ಹೋಗಿದೆ, ಅಲ್ಲೆಲ್ಲ ದೇವರೇ ಬಂದು ನೆಲೆಸಿರುವನು.

ಜ್ಞಾನದಿಂದ ತನ್ನ ಕಲ್ಮಷವನ್ನು ತೊಳೆದುಕೊಂಡವನು, ಹೋದರೆ ಹಿಂತಿರುಗದ ರೀತಿ ಹೋಗು ತ್ತಾನೆ ಎಂದರೆ ಮೋಕ್ಷವನ್ನು ಪಡೆಯುತ್ತಾನೆ. ನಾವು ಪ್ರಪಂಚಕ್ಕೆ ಬರಲು ಕಾರಣ ನಮ್ಮಲ್ಲಿರುವ ಆಸೆ ಆಕಾಂಕ್ಷೆಗಳು. ಯಾವಾಗ ಜ್ಞಾನದ ಆವಿಗೆಯಲ್ಲಿ ಹಸಿ ಬೆಂದುಹೋಗುವುದೊ ಇನ್ನುಮೇಲೆ ಅವನು ಬೇರೊಂದು ಜನ್ಮವನ್ನು ಎತ್ತಬೇಕಾಗಿಲ್ಲ. ಏಕೆಂದರೆ ಅವನು ಕೃತಕೃತ್ಯನಾಗಿದ್ದಾನೆ, ಪುಣ್ಯಾತ್ಮ ನಾಗಿದ್ದಾನೆ. ಪ್ರಪಂಚಕ್ಕೆ ಕೊಡಬೇಕಾದ ಸಾಲ ಅವನಲ್ಲಿ ಇನ್ನಾವುದೂ ಇಲ್ಲ. ಅವನು ಪುಣಮುಕ್ತ ನಾಗಿ ಹೋಗುತ್ತಾನೆ.

ಬದ್ಧಜೀವಿಗಳೆಲ್ಲ ಪ್ರಪಂಚವನ್ನು ಬಿಟ್ಟುಹೋಗುತ್ತಾರೆ. ಅಯ್ಯೊ ಬಿಡಬೇಕಲ್ಲ ಎಂದು ಬಿಡು ತ್ತಾರೆ. ಅವರ ದೃಷ್ಟಿಯೆಲ್ಲ ಇಲ್ಲಿಯೇ; ಅವರ ಪುಣ ತೀರಿಲ್ಲ. ಅದನ್ನು ತೀರಿಸಲು ಪುನಃ ಈ ಪ್ರಪಂಚಕ್ಕೆ ಬಂದು ತಮ್ಮ ಸಂಸ್ಕಾರಕ್ಕೆ ತಕ್ಕಂತೆ ಎಲ್ಲೊ ಜನ್ಮ ಧರಿಸುವರು. ಆದರೆ ಮುಕ್ತಜೀವಿ, ಬದುಕಿರುವಾಗಲೆ, ಎಲ್ಲರಿಂದ ರಜ ತೆಗೆದುಕೊಂಡಿರುವನು. ಇನ್ನು ಮೃತ್ಯುಮುಖದಲ್ಲಿ ಪರಮಾತ್ಮ ನಲ್ಲದೆ ಬೇರೆ ಚಿಂತನೆಯೇ ಇಲ್ಲ ಅವನಲ್ಲಿ. ಆಸೆ ಆಕಾಂಕ್ಷೆಗಳು ಭಸ್ಮವಾಗಿ ಹೋಗಿವೆ. ಹನಿಯೊಂದು ಸಾಗರಕ್ಕೆ ಬೀಳುವಂತೆ ಅವನು ಮುಕ್ತನಾಗಿ ಹೋಗುವನು.

\begin{verse}
ವಿದ್ಯಾವಿನಯಸಂಪನ್ನೇ ಬ್ರಾಹ್ಮಣೇ ಗವಿ ಹಸ್ತಿನಿ~।\\ಶುನಿ ಚೈವ ಶ್ವಪಾಕೇ ಚ ಪಂಡಿತಾಃ ಸಮದರ್ಶಿನಃ \versenum{॥ ೧೮~॥}
\end{verse}

{\small ವಿದ್ಯಾವಿನಯಗಳಿಂದ ಕೂಡಿದ ಬ್ರಾಹ್ಮಣನಲ್ಲಿ, ಗೋವಿನಲ್ಲಿ, ಆನೆಯಲ್ಲಿ, ನಾಯಿಯಲ್ಲಿ ಮತ್ತು ಚಂಡಾಲ ನಲ್ಲಿಯೂ ಪಂಡಿತರು ಸಮದರ್ಶಿಗಳಾಗಿರುವರು.}

ಪಂಡಿತರು ತಮ್ಮ ಕಣ್ಣೆದುರಿಗೆ ಕಾಣುವ ಉಪಾಧಿಗಳನ್ನು ತೂರಿ, ಅದರ ಹಿಂದೆ ಇರುವ ಪರಮಾತ್ಮ ವಸ್ತುವನ್ನು ನೋಡುವರು. ಅಜ್ಞರು ಕೇವಲ ಉಪಾಧಿಗಳನ್ನು ಮಾತ್ರ ನೋಡುವರು. ಅಜ್ಞಾನದಲ್ಲಿ ನಾಮರೂಪಗಳ ವಿಜೃಂಭಣೆ. ಇಲ್ಲಿಯೇ ಒಂದು ಹೆಚ್ಚು, ಮತ್ತೊಂದು ಕಡಮೆ, ಒಂದು ಪವಿತ್ರ ಮತ್ತೊಂದು ಅಪವಿತ್ರ, ಒಂದು ಪ್ರಾಣಿ ಮತ್ತೊಂದು ಮನುಷ್ಯ ಎಂದು ನೋಡುವುದು. ಪಂಡಿತ ಎಂದರೆ ಜ್ಞಾನಿಗೆ ಪ್ರಪಂಚವೇ ಬದಲಾಯಿಸಿದೆ. ಎಲ್ಲಾ ಭಗವನ್ಮಯವಾಗಿ ಕಾಣುವುದು. ಸಕ್ಕರೆಯಿಂದ ಹಲವು ಆಕಾರಗಳನ್ನು ಮಾಡಿ ಅಂಗಡಿಯಲ್ಲಿ ಸಂಕ್ರಾಂತಿಯ ಸಮಯದಲ್ಲಿ ಮಾರು ವರು. ಅವುಗಳಲ್ಲಿ ದೇವರು, ಮನುಷ್ಯ, ತುಳಸೀಕಟ್ಟೆ, ಹಲವು ಪ್ರಾಣಿಗಳು ಇವುಗಳ ಆಕಾರವೆಲ್ಲ ಇರುವುದು. ನಾವು ನಾಮರೂಪದ ದೃಷ್ಟಿಯಿಂದ ನೋಡಿದಾಗ ಒಂದಕ್ಕೂ ಮತ್ತೊಂದಕ್ಕೂ ಭಿನ್ನತೆ ಇದ್ದೇ ಇದೆ. ಆದರೆ ಸಕ್ಕರೆಯ ದೃಷ್ಟಿಯಿಂದ ನೋಡಿದಾಗ ಎಲ್ಲವೂ ಒಂದೇ. ಜ್ಞಾನಿಯದು ಸಕ್ಕರೆ ದೃಷ್ಟಿ. ಒಳ್ಳೆಯ ಗುಣಗಳಿಂದ ಕೂಡಿರುವ ಬ್ರಾಹ್ಮಣ ಒಂದು ಕಡೆ, ಮತ್ತೊಂದು ಕಡೆ ಎಲ್ಲಾ ಅನಾಚಾರಕ್ಕೂ ಹೆಸರಾಂತ ಚಂಡಾಲ ಇರುವನು. ಜ್ಞಾನಿ, ಬ್ರಾಹ್ಮಣ ಮತ್ತು ಚಂಡಾಲನ ಹಿಂದೆ ಇರುವ ಒಂದೇ ವಸ್ತುವನ್ನು ನೋಡುತ್ತಾನೆ. ಅದರಂತೆಯೆ ಪ್ರಾಣಿಗಳಲ್ಲಿ. ಹಿಂದೂಗಳಿಗೆ ಪರಮ ಪವಿತ್ರವಾದ ಗೋವು, ಮತ್ತೊಂದು ಎಲ್ಲಾ ಗಲೀಜುಗಳನ್ನೂ ತಿನ್ನುವ ನಾಯಿ. ಅದನ್ನು ಮುಟ್ಟಿದರೆ ಸ್ನಾನಮಾಡಬೇಕಾಗುವುದು. ಅದರ ಹಿಂದೆಯೂ ಒಂದನ್ನೇ ನೋಡುತ್ತಾನೆ. ಆನೆಯಾದರೊ ಬಹಳ ಭಾರಿ ಪ್ರಾಣಿ. ಅಷ್ಟೊಂದು ದೊಡ್ಡ ಪ್ರಾಣಿಯಾದರೂ ಅದರ ಹಿಂದೆ ಇರುವ ಪರಮಾತ್ಮ ವಸ್ತು ಒಂದೇ.

ಹಾಗಾದರೆ ಜ್ಞಾನಿ ಒಂದೇ ಪರಮಾತ್ಮನನ್ನು ನೋಡಿದರೆ, ಒಂದೇ ರೀತಿ ಸನ್ಮಾನಿಸುತ್ತಾನೆಯೆ? ಸಾಧು, ಕಳ್ಳ, ಹುಲಿ, ಹಸು ಎಲ್ಲರನ್ನೂ ಒಂದೇ ದೃಷ್ಟಿಯಿಂದ ನೋಡುವನೆ? ಶ‍್ರೀರಾಮಕೃಷ್ಣರು ಮನೋಜ್ಞವಾಗಿ ಬಣ್ಣಿಸುವರು. ನಾರಾಯಣನ ದೃಷ್ಟಿಯಿಂದ ಸಾಧುವಿನ ಹಿಂದೆ ಕಳ್ಳನ ಹಿಂದೆ ಇರುವವನು ಒಬ್ಬನೇ. ಆದರೆ ನಾರಾಯಣನ ದೃಷ್ಟಿಯಿಂದ ಸಾಧುವೇಷ ಹಾಕಿಕೊಂಡಿರುವಾಗ ಒಂದು ರೀತಿ ವ್ಯವಹರಿಸುತ್ತಾನೆ; ಚೋರನಾಗಿರುವಾಗ ಒಂದು ರೀತಿ ವ್ಯವಹರಿಸುತ್ತಾನೆ. ಸಮನಾಗಿ ನೋಡುವ ಜ್ಞಾನಿ ವ್ಯವಹಾರದಲ್ಲಿ ಮೂಢನಾಗುವುದಿಲ್ಲ. ಹಿಂದಿರುವ ನಾರಾಯಣನಿಗೆ ನಮಸ್ಕರಿಸಿ ದರೂ, ಮುಂದಿರುವ ಅದರ ಉಪಾಧಿಯಾದ ಚೋರನ ಹತ್ತಿರ ಜಾಗ್ರತನಾಗಿರುವನು. ವ್ಯಾಘ್ರ ದಲ್ಲಿಯೂ ನಾರಾಯಣನೇ ಇರುವನು. ಆದರೆ ಅದಕ್ಕೆ ನಾವು ದೂರದಲ್ಲಿ ಪಂಜರದ ಹೊರಗಡೆ ಯಿಂದಲೇ ನಮಸ್ಕಾರ ಹಾಕಬೇಕು. ಒಳಗೆ ಹೋಗಿ ಏನು ನಾರಾಯಣ ಎಂದರೆ, ಅದರ ಉಪಾಧಿ ತೊಂದರೆಯನ್ನು ಕೊಡಬಹುದು. ಪಾರಮಾರ್ಥಿಕ ದೃಷ್ಟಿಯಿಂದ ಎಲ್ಲದರ ಹಿಂದೆಯೂ ಒಂದನ್ನೇ ನೋಡಿದರೂ ವ್ಯಾವಹಾರಿಕ ದೃಷ್ಟಿಯಿಂದ ಅದರ ಮೇಲೆ ಆರೋಪಿತವಾದ ಉಪಾಧಿಗಳ ಅರಿವು ಅವನಿಗೆ ಆಗುವುದು. ಆದರೆ ಅದಕ್ಕೆ ಮೋಸ ಬೀಳುವುದಿಲ್ಲ.

\begin{verse}
ಇಹೈವ ತೈರ್ಜಿತಃ ಸರ್ಗೋ ಯೇಷಾಂ ಸಾಮ್ಯೇ ಸ್ಥಿತಂ ಮನಃ~।\\ನಿರ್ದೋಷಂ ಹಿ ಸಮಂ ಬ್ರಹ್ಮ ತಸ್ಮಾದ್ಬ್ರಹ್ಮಣಿ ತೇ ಸ್ಥಿತಾಃ \versenum{॥ ೧೯~॥}
\end{verse}

{\small ಯಾರ ಮನಸ್ಸು ಸಾಮ್ಯದಲ್ಲಿ ನೆಲೆಸಿರುವುದೊ ಅವರು ಇಲ್ಲಿಯೇ ಸಂಸಾರವನ್ನು ಜಯಿಸುತ್ತಾರೆ. ಏಕೆಂದರೆ ಬ್ರಹ್ಮ ದೋಷರಹಿತವಾಗಿಯೂ ಸಮವಾಗಿಯೂ ಇದೆ. ಆದುದರಿಂದ ಅವರು ಬ್ರಹ್ಮದಲ್ಲಿಯೇ ನೆಲಸಿರುವರು.}

ಸಾಮ್ಯದಲ್ಲಿ ನೆಲೆಸಿರುವವರು ಇಲ್ಲಿಯೇ ಸಂಸಾರದಿಂದ ಪಾರಾಗಿದ್ದಾರೆ. ಸಾಮ್ಯದೃಷ್ಟಿ ನಮಗೆ ಬರಬೇಕಾದರೆ, ಎಲ್ಲಾ ನಾಮರೂಪಗಳ ಹಿಂದೆ, ಅಖಂಡವಾದ ಅವಿಭಾಜ್ಯವಾದ ಬ್ರಹ್ಮತತ್ವವನ್ನು ನೋಡಬೇಕು. ಆ ಭೂಮಾನುಭೂತಿ ನಮಗಾದರೆ, ಅದರ ಮುಂದೆ ತಾತ್ಕಾಲಿಕವಾಗಿ ಯಾವುದೊ ನಾಮರೂಪದ ಮೋಡ ಮುಚ್ಚಿದರೆ ಅದನ್ನು ಗಣನೆಗೆ ತರುವುದಿಲ್ಲ. ಆ ಮೋಡ ಜ್ಞಾನಿಗೆ ಸರಿದುಹೋಗಿದೆ. ಅವನು ಇನ್ನುಮೇಲೆ ಇಂದ್ರಿಯ ವಿಷಯಗಳ ಕೋಟಲೆಗೆ ಸಿಲುಕುವುದಿಲ್ಲ. ನಾವು ಸರ್ವವ್ಯಾಪಿಯಾದ ಒಂದನ್ನು ನೋಡಬೇಕಾದರೆ, ದೇಹ ಮನಸ್ಸು ಬುದ್ಧಿ ಇಂದ್ರಿಯಗಳನ್ನು ಮೀರಿಹೋಗಬೇಕು. ಅವುಗಳ ಆಕರ್ಷಣೆಗೆ ಅವನು ಸಿಲುಕುವುದಿಲ್ಲ. ಇಂತಹ ಸ್ಥಿತಿಗೆ ಬಂದ ಮನುಷ್ಯ ಈ ಪ್ರಪಂಚದಲ್ಲಿರುವಾಗಲೇ ಅದೊಂದನ್ನೇ ಎಲ್ಲಾ ವಸ್ತುಗಳ ಹಿಂದೆ ನೋಡುತ್ತಿರುವನು. ಈ ಸಂಸಾರದ ಹಿಂದೆ ಹೋಗಿರುವನು. ಈ ಸಂಸಾರದ ಜಳ್ಳನ್ನು ಭೇದಿಸಿರುವನು. ಮುಕ್ತಿ ಎಂದರೆ ಬದುಕಿರುವಾಗ ಇಲ್ಲದ್ದು ಸತ್ತಮೇಲೆ ಬರುವ ಅವಸ್ಥೆಯಲ್ಲ. ಬದುಕಿರುವಾಗಲೂ ಯಾವುದು ಇದೆಯೋ, ಅದೇ ಸತ್ತಮೇಲೆಯೂ ಇರುವುದು. ಇರುವಾಗ ಸರ್ವವ್ಯಾಪಿಯಾದ ಭಗವದ್ ದರ್ಶನ ಆಗಿದ್ದರೆ, ಕಾಲವಾದಮೇಲೂ ಅದೇ ಸ್ಥಿತಿಯಲ್ಲಿರುವನು. ಇರುವಾಗ ಸುಳ್ಳನೊ ಕಳ್ಳನೊ ಆಗಿ, ಯಾವುದೊ ಒಂದು ಪುಣ್ಯ ಕೆಲಸವನ್ನು ಮಾಡಿದನೆಂದು ಅವನಿಗೆ ಮುಕ್ತಿ ಇದ್ದಕ್ಕೆ ಇದ್ದಂತೆ ಬರುವುದಿಲ್ಲ. ಅವರು ಮಾಡಿದ ಪುಣ್ಯ ಕೆಲಸಕ್ಕೆ ಒಂದು ಪುಣ್ಯಲೋಕವೊ, ಒಂದು ಪುಣ್ಯಜನ್ಮವೊ ತಾತ್ಕಾಲಿಕವಾಗಿ ಪ್ರಾಪ್ತವಾಗಬಹುದು. ಇದು ಮುಕ್ತಿಯಲ್ಲ, ಬಿಡುಗಡೆಯಲ್ಲ. ಅವನು ತಾತ್ಕಾಲಿಕ ವಾಗಿ ಪಡೆದ ಸ್ಥಿತಿಯನ್ನು ಅನುಭವಿಸಿ ಆದಮೇಲೆ ಪುನಃ ಮನುಷ್ಯನಾಗಿ ಹುಟ್ಟಿ ಮನಸ್ಸನ್ನು ಸರ್ವವ್ಯಾಪಿಯಾದ ಭಗವಂತನ ಮೇಲೆ ಹರಿಸುವುದನ್ನು ಅಭ್ಯಾಸಮಾಡಿ, ಅದರಲ್ಲಿ ಯಶಸ್ವಿ ಆದರೆ ಮಾತ್ರ ಸಂಸಾರದಿಂದ ಪಾರಾಗುವನು. ಇಲ್ಲದೇ ಇದ್ದರೆ ಇಲ್ಲ.

ದೋಷ ಬ್ರಹ್ಮದಲ್ಲಿ ಇಲ್ಲ. ದೋಷ ಇರುವುದು ಉಪಾಧಿಯಲ್ಲಿ. ಒಂದೇ ವಿದ್ಯುತ್​ಶಕ್ತಿ ಎಲ್ಲಾ ಬಲ್ಬುಗಳ ಮೂಲಕ ಪ್ರಕಾಶಿಸುತ್ತಿದೆ. ಕೆಲವು ಬಲ್ಬು ಹಳದಿ, ಕೆಲವು ನೀಲಿ ಹೀಗಿವೆ. ಕೆಲವು ಸಾವಿರ ಕ್ಯಾಂಡಲ್ ಕಾಂತಿಯದು, ಮತ್ತೆ ಕೆಲವು ಜೀರೊ ಕ್ಯಾಂಡಲ್ ಕಾಂತಿಯದು. ಇದರಿಂದ ವಿದ್ಯುತ್ ಶಕ್ತಿಗೆ ಬಣ್ಣ ಬಂತೆ? ವಿದ್ಯುತ್ ಒಂದೇ ಸಮನಾಗಿದೆ. ಯಾವುದರ ಮೂಲಕ ಅದು ವ್ಯಕ್ತವಾಗುವುದೊ ಅದರ ಗುಣದೋಷಗಳು ಇವು. ಬೃಂದಾವನದಲ್ಲಿ ನೀರು ಹರಿದುಕೊಂಡು ಹೋಗುವಾಗ, ಒಂದೊಂದು ಕಡೆ ಅದರ ಮೇಲೆ ಒಂದೊಂದು ಬಗೆಯ ಬೆಳಕನ್ನು ಬಿಡುತ್ತಾರೆ. ಆಗ ನೀರೇ ಆ ಬಣ್ಣವನ್ನು ಧರಿಸಿದಂತೆ ನಮಗೆ ತೋರುವುದು. ಇದೊಂದು ನಮ್ಮ ಭ್ರಮೆ. ನೀರಿಗೆ ಯಾವ ದೋಷವೂ ತಾಕಿಲ್ಲ.

ಬ್ರಹ್ಮ ಎಲ್ಲಾ ಕಡೆಯೂ ಸಮನಾಗಿ ನಿಷ್ಪಕ್ಷಪಾತವಾಗಿ ಇರುವನು. ಒಂದು ಕಡೆ ಜಾಸ್ತಿ ಇಲ್ಲ. ಮತ್ತೊಂದು ಕಡೆ ಕಡಮೆಯಿಲ್ಲ. ಸೂರ್ಯನ ಬೆಳಕು ಅರಸನ ಮನೆಯ ಮೇಲೆ ಬೀಳುವುದು, ಬಡವನ ಗುಡಿಸಲಮೇಲೆ ಬೀಳುವುದು. ಅದು ಚರಂಡಿಯಮೇಲೆ ಬೀಳುವುದು. ಅದಕ್ಕೆ ಒಂದು ಹೆಚ್ಚಲ್ಲ ಮತ್ತೊಂದು ಕಡಮೆಯಲ್ಲ. ಉಪಾಧಿಗಳು ಬ್ರಹ್ಮನ ಕಾಂತಿಯನ್ನು ಪ್ರತಿಬಿಂಬಿಸುವುದರಲ್ಲಿ ಹೆಚ್ಚು ಕಡಮೆಯನ್ನು ನಾವು ನೋಡಿದರೂ ಇದು ಉಪಾಧಿಯ ದೋಷವೇ ಹೊರತು ಬ್ರಹ್ಮನ ದೋಷವಲ್ಲ. ಸೂರ್ಯನ ಬೆಳಕು ನೆಲದ ಮೇಲೆ ಬೀಳುವುದು, ನೀರಿನ ಮೇಲೆ ಬೀಳುವುದು. ಕನ್ನಡಿಯ ಮೇಲೆ ಬೀಳುವುದು. ಇಲ್ಲಿ ಬೆಳಕಿನ ಪ್ರತಿಬಿಂಬವನ್ನು ಒಂದು ಮತ್ತೊಂದಕ್ಕಿಂತ ಹೆಚ್ಚು ವ್ಯಕ್ತಪಡಿಸುತ್ತಿದ್ದರೂ ಇದು ಸೂರ್ಯನ ತಪ್ಪಲ್ಲ. ಅದರಂತೆಯೇ ದೇವರು ಎಲ್ಲರಲ್ಲಿಯೂ ಒಂದೇ ಸಮನಾಗಿರುವನು. ಆವಿರ್ಭಾವದ ದೃಷ್ಟಿಯಿಂದ ನೋಡಿದಾಗ ಸಾಧುವಿನಲ್ಲಿ ಹೆಚ್ಚಾಗಿಯೂ ಕಳ್ಳನಲ್ಲಿ ತುಂಬಾ ಕಡಮೆಯಾಗಿಯೂ ಇರುವುದು ಕಾಣುವುದು.

ಪಂಡಿತರು ಅಥವಾ ಜ್ಞಾನಿಗಳು ಆವಿರ್ಭಾವದ ದೃಷ್ಟಿಯಿಂದ ನೋಡುವುದಿಲ್ಲ. ಅದನ್ನು ಮಾಡುವವರು ಸಾಧಾರಣ ಮನುಷ್ಯರು. ಯಾವುದು ವ್ಯಕ್ತವಾಗಿದೆಯೊ ಅದಕ್ಕೆ ಮಾತ್ರ ಪುರಸ್ಕಾರ, ಬಹುಮಾನ; ಯಾವುದು ಸುಪ್ತವಾಗಿದೆಯೊ ಅದನ್ನು ಅವರು ಗಣನೆಗೆ ತೆಗೆದುಕೊಳ್ಳುವುದಿಲ್ಲ. ಜ್ಞಾನಿಯಾದರೊ ಉಪಾಧಿಗಳ ಹಿಂದೆ ಹೋಗಿ ಒಂದೇ ಸಮನಾಗಿ ಎಲ್ಲಕಡೆಯಲ್ಲೂ ಪಸರಿಸಿರುವ ಬ್ರಹ್ಮದ ಮೇಲೆ ಮನಸ್ಸನ್ನು ಇಟ್ಟಿರುವನು.

\begin{verse}
ನ ಪ್ರಹೃಷ್ಯೇತ್ ಪ್ರಿಯಂ ಪ್ರಾಪ್ಯ ನೋದ್ವಿಜೇತ್ ಪ್ರಾಪ್ಯ ಚಾಪ್ರಿಯಂ~।\\ಸ್ಥಿರಬುದ್ಧಿರಸಂಮೂಢೋ ಬ್ರಹ್ಮವಿದ್ಬ್ರಹ್ಮಣಿ ಸ್ಥಿತಃ \versenum{॥ ೨೦~॥}
\end{verse}

{\small ಸ್ಥಿರಬುದ್ಧಿಯುಳ್ಳವನು, ಮೋಹವಿಲ್ಲದವನು, ಬ್ರಹ್ಮದಲ್ಲಿಯೇ ನೆಲಸಿರುವ ಬ್ರಹ್ಮಜ್ಞಾನಿ, ಪ್ರಿಯವಾದುದನ್ನು ಪಡೆದಾಗ ಹಿಗ್ಗಕೂಡದು. ಅಪ್ರಿಯವಾದುದನ್ನು ಪಡೆದಾಗ ಕುಗ್ಗಕೂಡದು.}

ಬ್ರಹ್ಮಜ್ಞಾನಿಯಲ್ಲಿ ನಾವು ಈ ಗುಣಗಳನ್ನು ಕಾಣುತ್ತೇವೆ. ಅವನ ಬುದ್ಧಿ ಸ್ಥಿರವಾಗಿದೆ. ಅವನು ಮೋಹದಿಂದ ಪಾರಾಗಿರುವನು. ಅವನು ಬ್ರಹ್ಮಾನುಭವದ ಮೇಲೆಯೆ ನೆಲಸಿರುವನು. ಅವನು ಇನ್ನುಮೇಲೆ ಬ್ರಹ್ಮನ ವಿಷಯದಲ್ಲಿ ಅನುಮಾನಿಸುವುದಿಲ್ಲ. ಈ ಅನುಮಾನ, ಊಹೆ, ಅಪನಂಬಿಕೆ ಇವುಗಳೆಲ್ಲ ಇನ್ನೂ ಪ್ರಾರಂಭದ ಸ್ಥಿತಿ. ಇವುಗಳನ್ನೆಲ್ಲ ಅವನು ದಾಟಿಹೋಗಿರುವನು. ಆ ಬುದ್ಧಿಗೆ ಅನುಭವದ ಆಧಾರ ಸಿಕ್ಕಿದೆ. ಅದಕ್ಕೇ ಅದು ಚಂಚಲವಾಗುವುದಿಲ್ಲ. ಯಾವ ಬುದ್ಧಿಗೆ ಅನುಭವದ ಆಧಾರ ಸಿಕ್ಕಿಲ್ಲವೊ, ಅದಕ್ಕೆ ಸ್ವಲ್ಪ ಬಿರುಸಾದ ಸಂಶಯದ ಗಾಳಿ ಬೀಸಿದರೆ ಸಾಕು, ಅವನು ಕಟ್ಟಿಕೊಂಡಿರುವ ನೆಮ್ಮದಿಯ ಗುಡಿಸಲು ಕಿತ್ತುಹೋಗುವುದು. ಅನುಭವದ ಮೇಲೆ ನಿಂತ ಬುದ್ಧಿ ಯಾದರೋ ಒಳ್ಳೆಯ ಬಂಡೆಯ ಮೇಲೆ ಕಟ್ಟಿದ ಕಟ್ಟಡ. ಎಷ್ಟು ಗಾಳಿ ಬೀಸಲಿ, ಮಳೆ ಬೀಳಲಿ ಅದರಿಂದ ಏನೂ ಆಗುವುದಿಲ್ಲ. ಅದನ್ನು ಮುತ್ತಿರುವ ಸ್ವಲ್ಪ ಕೊಳೆ ಹೋಗಿ ಹಿಂದಿಗಿಂತ ಮತ್ತೂ ಕಾಂತಿಯಿಂದ ಶೋಭಿಸುವುದು.

ಅವನಲ್ಲಿ ಮೋಹವಿಲ್ಲ. ಸತ್ಯಸ್ಯಸತ್ಯವಾದ ಪರಮಾತ್ಮನನ್ನು ಅವನು ಕಂಡಿರುವನು, ಅನುಭವಿ ಸಿರುವನು. ಅವನು ಇನ್ನುಮೇಲೆ ಪ್ರಪಂಚದ ಎಂತಹ ಮೋಹಕವಾದ ವಸ್ತುವಾಗಲಿ ಅದರ ಆಕರ್ಷಣೆಗೆ ಬೀಳನು. ಬಿದ್ದರೆ ಆ ಮನುಷ್ಯನಿಗೆ ಇನ್ನೂ ಬ್ರಹ್ಮಾನುಭವ ಸಿಕ್ಕಿಲ್ಲ. ರಾಮ, ಕಾಮ ಎರಡೂ ಒಟ್ಟಿಗೆ ಹೋಗಲಾರದು. ಒಂದಿದ್ದರೆ ಮತ್ತೊಂದಿಲ್ಲ. ರಾಮನನ್ನು ರುಚಿ ನೋಡಿದವನಿಗೆ ಈ ಪ್ರಪಂಚ ಹೊಲಸು. ಇದರ ಕಡೆ ಅವನು ಗಮನ ಕೊಡನು. ಇಲ್ಲಿ ಗಮನ ಕೊಟ್ಟರೆ ಅವನ ಜ್ಞಾನದಲ್ಲಿ ಏನೋ ದೋಷವಿದೆ. ಅನುಭಾವಿಗೆ ಈ ದೋಷ ನಿರ್ಮೂಲವಾಗಿದೆ. ಅವನು ಸದಾ ಬ್ರಹ್ಮದಲ್ಲಿಯೇ ಇರುವನು. ಒಂದು ಸಮಯ ಅದನ್ನು ಜ್ಞಾಪಿಸಿಕೊಳ್ಳುವುದು, ಮತ್ತೊಂದು ಸಮಯ ಅದನ್ನು ಮರೆಯುವುದಿಲ್ಲ. ಎಲ್ಲಾ ಅವಸ್ಥೆಗಳಲ್ಲಿಯೂ ಜ್ಞಾನ ಅವನನ್ನು ಬೆಂಬಿಡದೆ ಇರುವುದು. ಈ ವ್ಯವಹಾರ ಭೂಮಿಕೆಯಲ್ಲಿರುವಾಗಲೂ, ಅವನ ಮನಸ್ಸು ಪರಮಾರ್ಥದಲ್ಲಿ ಮುಳುಗಿರುತ್ತದೆ.

ಬ್ರಹ್ಮಾನುಭವದ ಮೇಲೆ ನಿಂತಿರುವ ಜೀವಿ ಪ್ರಿಯವಾದದು ಬಂದರೆ, ಅದಕ್ಕೆ ಹಿಗ್ಗುವುದಿಲ್ಲ. ಅದು ತಾತ್ಕಾಲಿಕ, ಕ್ಷಣಿಕ, ಅದೊಂದು ಆರೋಪ ಎಂಬುದನ್ನು ಚೆನ್ನಾಗಿ ಮನಗಂಡಿರುವನು. ಅವನು ಅದಕ್ಕೆ ಆಸಕ್ತನಾಗುವುದಿಲ್ಲ. ಅದು ಬರುವುದು, ಹೋಗುವುದು. ಹಾಗೆಯೇ ಅಪ್ರಿಯವಾದುದು ಬಂದರೆ ಅದರ ಭಾರಕ್ಕೆ ಕುಗ್ಗುವುದಿಲ್ಲ. ಅವನಿಗೆ ಗೊತ್ತಿದೆ ಇದೂ ಕ್ಷಣಿಕ ತಾತ್ಕಾಲಿಕ ಎಂಬುದು. ಇವೆಲ್ಲ ವೇಷ, ಸುಳ್ಳು. ಕೆಲವು ಪ್ರಿಯವಾಗಿರಬಹುದು, ಮತ್ತೆ ಕೆಲವು ಅಪ್ರಿಯವಾಗಿರಬಹುದು. ಜ್ಞಾನಿಯ ದೃಷ್ಟಿ, ಪ್ರಿಯ ಅಪ್ರಿಯದ ಹಿಂದೆ ಇರುವ ಸನಾತನ ಸತ್ಯದ ಕಡೆಗೆ.

\begin{verse}
ಬಾಹ್ಯಸ್ಪರ್ಶೇಷ್ವಸಕ್ತಾತ್ಮಾ ವಿಂದತ್ಯಾತ್ಮನಿ ಯತ್ಸುಖಮ್~।\\ಸ ಬ್ರಹ್ಮಯೋಗಯುಕ್ತಾತ್ಮಾ ಸುಖಮಕ್ಷಯಮಶ್ನುತೇ \versenum{॥ ೨೧~॥}
\end{verse}

{\small ಶಬ್ದಾದಿ ವಿಷಯಗಳಲ್ಲಿ ಆಸಕ್ತಿ ಇಲ್ಲದವನು ತನ್ನಲ್ಲಿಯೇ ಸುಖವನ್ನು ಪಡೆಯುತ್ತಾನೆ. ಬ್ರಹ್ಮಯೋಗದಿಂದ ಕೂಡಿದ ಅವನು ಅಕ್ಷಯವಾದ ಸುಖವನ್ನು ಪಡೆಯುತ್ತಾನೆ.}

ಯೋಗಿಯ ಸುಖವೆಲ್ಲ ತನ್ನಲ್ಲಿಯೇ ಇರುವುದು. ಅವನು ಬಾಹ್ಯ ವಿಷಯಗಳನ್ನು ಬೆನ್ನಟ್ಟಿ ಹೋಗುವುದಿಲ್ಲ ತನ್ನ ಸುಖಕ್ಕೆ. ಯಾವ ಸುಖ ಬಾಹ್ಯವಸ್ತುವಿನ ಆಸರೆಯ ಮೇಲೆ ನಿಂತಿದೆಯೋ, ಅದು ಅನ್ಯಾಶ್ರಯದ್ದು. ಅದಿಲ್ಲದೇ ಇದ್ದರೆ ನಮಗೆ ಸುಖವಿಲ್ಲ. ಒಂದು ಸಲ ಅದನ್ನು ನಾವು ಅನುಭವಿಸಿದರೆ ಪದೇ ಪದೇ ಅದನ್ನು ಅನುಭವಿಸಬೇಕೆಂದು ಹಾತೊರೆಯುವುದು ಮನಸ್ಸು. ಅದಕ್ಕಾಗಿ ಎಷ್ಟು ಕಷ್ಟವಾದರೂ ಪಡಲು ಸಿದ್ಧ. ನಾವು ಹೊರಗಿನಿಂದ ಬರುವ ವೇದನೆಗಳಿಗೆ ಗುಲಾಮರಾಗು ವೆವು. ಅದು ಏನು ಬೆಲೆ ಕೇಳುವುದೊ ಅದನ್ನು ಕೊಡುವೆವು. ಯೋಗಿ ಹೊರಗೆ ಹೋಗುವ ತನ್ನ ಇಂದ್ರಿಯಗಳನ್ನೆಲ್ಲ ನಿಗ್ರಹಿಸಿ, ಅದರ ಮೂಲಕ ವ್ಯರ್ಥವಾಗಿ ಹೋಗುತ್ತಿದ್ದ ಮಾನಸಿಕ ಶಕ್ತಿಯನ್ನೆಲ್ಲ ಅಂತರ್ಮುಖ ಮಾಡಿ, ಅದನ್ನು ಪರಮಾತ್ಮನ ಕಡೆ ಹರಿಸುವನು. ಆ ಪರಮಾತ್ಮನ ಸಂಬಂಧದಿಂದ ಇವನಿಗೆ ಸುಖ ಬರುವುದು. ಆ ಸುಖವೊ ತನ್ನಲ್ಲಿಯೇ ಜಿನುಗುತ್ತಿರುವುದು. ಅದಕ್ಕೆ ಬಾಹ್ಯವಸ್ತುಗಳ ಸಂಪರ್ಕವೇ ಬೇಕಿಲ್ಲ. ಅದನ್ನು ಎಷ್ಟು ಅನುಭವಿಸಿದರೂ ಜುಗುಪ್ಸೆಯಾಗುವುದಿಲ್ಲ. ಅದನ್ನು ಇನ್ನೂ ಹೆಚ್ಚುಹೊತ್ತು ಅನುಭವಿಸುತ್ತಿರುವ ಎನ್ನಿಸುವುದು. ಇಂದ್ರಿಯ ಸುಖ ಹಾಗಲ್ಲ. ನಿಮಗೆ ಎಷ್ಟೇ ಪ್ರಿಯವಾದ ಅನುಭವ ಆದರೂ ಸ್ವಲ್ಪಕಾಲವಾದ ಮೇಲೆ ಬೇಜಾರಾಗುವುದು. ನನಗೆ ತುಂಬಾ ಪ್ರಿಯವಾಗಿರುವುದನ್ನು ಎಷ್ಟು ತಿಂದೇನು? ಅದೊಂದು ಮಿತಿ ಮೀರಿದರೆ ನಮಗೆ ವಾಂತಿ ಬರುವುದು. ಮುಂಚೆ ಯಾವುದನ್ನು ನೆನೆಸಿಕೊಂಡರೆ ಬಾಯಲ್ಲಿ ನೀರು ಬರುತ್ತಿತ್ತೊ, ಈಗ ಅದನ್ನು ನೆನೆಸಿಕೊಂಡರೆ ವಾಕರಿಕೆ ಬರುವುದು. ಎಷ್ಚೇ ಪ್ರಿಯವಾದ ಹಾಡಾದರೂ ಕೇಳುತ್ತಿದ್ದರೆ ಬೇಜಾರಾಗುವುದು. ಸುಂದರ ವಾದ ನೋಟ ನೋಡುತ್ತಾ, ನೋಡುತ್ತಾ ಬೇಜಾರಾಗುವುದು. ಎಲ್ಲಾ ಹೊಸದರಲ್ಲಿ ಅದರಿಂದ ಬರುವ ಆನಂದ. ಸ್ವಲ್ಪ ಕಾಲವಾದಮೇಲೆ ಅದರಿಂದ ಆನಂದವೇನೂ ಬರುವುದಿಲ್ಲ. ಅಭ್ಯಾಸವಾಗಿರು ವುದರಿಂದ, ಅದನ್ನು ಮಾಡಲೇ ಬೇಕಾಗಿರುವುದರಿಂದ ಮಾಡಿಹಾಕುವೆವು. ಭಗವಂತನ ಚಿಂತನೆ ಯಿಂದ ಬರುವ ಆನಂದ ಹಾಗಲ್ಲ. ನಾವು ಅದನ್ನು ಎಷ್ಟು ಅನುಭವಿಸಿದರೂ ಹಳತಾಗುವುದಿಲ್ಲ, ಬೇಜಾರಾಗುವುದಿಲ್ಲ, ಸಾಕು ಎನಿಸುವುದಿಲ್ಲ. ಇದು ಯಾವ ವಿಧವಾದ ಕೆಟ್ಟ ಪರಿಣಾಮವನ್ನು ನಮ್ಮ ಮೇಲೆ ಬಿಡುವುದಿಲ್ಲ. ಇಂದ್ರಿಯ ಸುಖ ಹಾಗಲ್ಲ. ಅದು ಅತ್ಯುನ್ನತ ಸುಖದ ಶಿಖರಕ್ಕೆ ಮುಟ್ಟಿದಾಗಲೆ, ಅಯ್ಯೋ ಇದನ್ನು ಏತಕ್ಕೆ ಅನುಭವಿಸಿದೆನೊ, ಎಂಬ ಪಶ್ಚಾತ್ತಾಪ ಪ್ರಾರಂಭವಾಗುವುದು. ಈ ಗೋಳಿನ ಎದುರಿಗೆ ಈಗತಾನೆ ಅನುಭವಿಸಿದ ಸುಖವೆಲ್ಲ ಮಾಯವಾಗುವುದು. ಯೋಗಿಯ ಸುಖ ಇಂಥದಲ್ಲ. ಅವನಿಗೆ ಬರುವ ಸುಖ ಅಕ್ಷಯವಾದುದು. ತನ್ನ ಇಂದ್ರಿಯಗಳನ್ನು ಗೆದ್ದು, ಮನಸ್ಸನ್ನು ಶುದ್ಧಗೊಳಿಸಿ, ಏಕಾಗ್ರಮಾಡಿ ಪರಮಾತ್ಮನ ಕಡೆ ಕಳಿಸಿದರೆ ಅದರಿಂದ ಸಿಗುವ ಸುಖಕ್ಕೆ ಒಂದು ಅಂತ್ಯವಿಲ್ಲ. ಅವನು ಈ ಪ್ರಪಂಚದ ಹಂಗಿಲ್ಲದೆ ಪರಮಾತ್ಮನ ಮೂಲಕ ಬರುವ ಇಂದ್ರಿಯಾತೀತ ಸುಖವನ್ನು ಹೀರುತ್ತಿರಬಹುದು.

\begin{verse}
ಯೇ ಹಿ ಸಂಸ್ಪರ್ಶಜಾ ಭೋಗಾ ದುಃಖಯೋನಯ ಏವ ತೇ~।\\ಆದ್ಯಂತವಂತಃ ಕೌಂತೇಯ ನ ತೇಷು ರಮತೇ ಬುಧಃ \versenum{॥ ೨೨~॥}
\end{verse}

{\small ಕೌಂತೇಯ, ಯಾವ ಸುಖಗಳು ವಿಷಯ ಮತ್ತು ಇಂದ್ರಿಯಗಳ ಸಂಯೋಗದಿಂದ ಆಗುವುದೋ, ಅವು ದುಃಖಕ್ಕೆ ಕಾರಣ. ಅವುಗಳಿಗೆಲ್ಲ ಒಂದು ಆದಿ ಮತ್ತು ಅಂತ್ಯವಿದೆ. ಜ್ಞಾನಿ ಅವುಗಳಲ್ಲಿ ರಮಿಸುವುದಿಲ್ಲ.}

ವಿಷಯ ಹೊರಗಡೆ ಇದೆ. ಇಂದ್ರಿಯ ನಮ್ಮಲ್ಲಿದೆ. ನಮಗೆ ಹೊರಗಿನಿಂದ ಸುಖ ಬೇಕಾದರೆ, ಒಳಗೆ ಇರುವ ಇಂದ್ರಿಯಕ್ಕೂ ಹೊರಗೆ ಇರುವ ವಿಷಯಕ್ಕೂ ಸಂಬಂಧ ಏರ್ಪಡಬೇಕು. ಆಗಲೆ ಸುಖ ಎನ್ನುವ ಸಂವೇದನೆ ನಮಗೆ ಆಗುವುದು. ಸುಂದರವಾದ ವಸ್ತುವನ್ನು ಕಣ್ಣು ನೋಡಬೇಕು, ಇಂಪಾದ ಗಾನವನ್ನು ಕಿವಿ ಕೇಳಬೇಕು, ಸ್ಪರ್ಶಕ್ಕೆ ಪ್ರಿಯವಾದುದನ್ನು ಅಂಗಾಗಗಳು ಮುಟ್ಟಬೇಕು, ರುಚಿಕರವಾದುದನ್ನು ನಾವು ತಿನ್ನಬೇಕು, ಆಗಲೆ ಸುಖ.

ಹೊರಗಿನಿಂದ ಬರುವ ಸುಖಗಳೆಲ್ಲ ದುಃಖಕ್ಕೆ ಕಾರಣ. ಒಂದು ಸಲ ಅದನ್ನು ಅನುಭವಿಸಿದರೆ, ಪದೇ ಪದೇ ಅದನ್ನು ಅನುಭವಿಸಬೇಕೆಂಬ ಅಭ್ಯಾಸ ನಮ್ಮಲ್ಲಿ ಉಂಟಾಗುವುದು. ಅಭ್ಯಾಸದ ಕೈಗೆ ಸಿಕ್ಕಿಕೊಂಡರೆ ಇನ್ನು ಬಿಡಿಸಿಕೊಳ್ಳುವಂತೆ ಇಲ್ಲ. ಬೇರೆ ನರಕ ಬೇಕಾಗಿಲ್ಲ. ಆ ಅಭ್ಯಾಸದ ಕೈಗೆ ಸಿಕ್ಕಿ ನರಳುವವನನ್ನು ನೋಡಿದರೆ ಅದೇನು ಎಂಬುದು ಗೊತ್ತಾಗುವುದು. ಅಫೀಮನ್ನು ಸೇವಿಸುವ ಅಭ್ಯಾಸವಿರುವವನಿಗೆ, ಆ ಸಮಯಕ್ಕೆ ಅದನ್ನು ತೆಗೆದುಕೊಳ್ಳಲಿಲ್ಲ ಎಂದರೆ ಅವನ ಅಂಗೋಪಾಂಗ ಗಳು ಅದಕ್ಕೆ ಕಾಡುವುವು. ಆ ಯಾತನೆಯನ್ನು ಸಹಿಸಲಾರದೆ ಅವನು ಅಫೀಮನ್ನು ಸೇವಿಸುವನು. ಮುಂಚೆ ಅದರಿಂದ ಬರುವ ಸುಖಕ್ಕೆ ತೆಗೆದುಕೊಂಡ. ಇನ್ನುಮೇಲೆ ಅದನ್ನು ತೆಗೆದುಕೊಳ್ಳದೆ ಇದ್ದರೆ ತನ್ನಲ್ಲಾಗುವ ತೀವ್ರ ಯಾತನೆಯಿಂದ ಪಾರಾಗಲು ತೆಗೆದುಕೊಳ್ಳುವನು. ಹೀಗೆ ಯಾವುದಾದರೂ ಒಂದು ಸಲ ಅಭ್ಯಾಸವಾದರೆ ಸಾಕು ಅದನ್ನು ಮಾಡಲು ನಾವು ಏನನ್ನು ಬೇಕಾದರೂ ಅನುಭವಿಸಲು ಸಿದ್ಧ. ಒಂದು ಚುಟಿಕೆ ನಶ್ಯಕ್ಕೆ ಒಂದು ಮೈಲಿ ಹೋಗಿಬರುವರು. ಒಂದು ಕಪ್ಪು ಕಾಫಿಗೆ ಬೇಕಾದರೆ ಕ್ಯೂ ನಿಂತುಕೊಳ್ಳುವರು. ಕುಡಿಯುವ ಅಭ್ಯಾಸ ಇರುವವನಿಗಂತಲೂ ಆ ಸಮಯ ಬಂದರೆ ಕುಡಿಯುವುದಕ್ಕಾಗಿ ಏನನ್ನು ಬೇಕಾದರೂ ಮಾಡುವನು. ದುಡ್ಡಿಲ್ಲದಿದ್ದರೆ ಕದಿಯುವನು, ತನ್ನಲ್ಲಿರು ವುದನ್ನು ಮಾರಿಹಾಕುವನು. ಹೆಂಡತಿಯ ತಾಳಿಯನ್ನಾದರೂ ಮಾರಿ ಕುಡಿಯುವವರೆಗೆ ಹೋಗುವನು. ಹೆಂಡದ ಬದಲು ಇನ್ನೇನು ಸಿಕ್ಕುವುದೋ ಅದನ್ನು ಕುಡಿದು ತನ್ನನ್ನು ಹಾಳುಮಾಡಿಕೊಳ್ಳುವನು. ತನ್ನನ್ನು ನೆಚ್ಚಿ ಬಂದವರನ್ನು ಕೈಬಿಡುವನು. ಜೂಜು ಒಂದು ಅಭ್ಯಾಸ. ಮುಂಚೆ ಸಂತೋಷಕ್ಕೆ ಆಡುವರು. ಅನಂತರ ಅಭ್ಯಾಸವಾಗಿ ಆಡುವುದಕ್ಕೆ ಜತೆಗಾರರನ್ನು ಹುಡುಕಿಕೊಂಡು ಹೋಗುವರು. ಆಟದಲ್ಲಿ ಎಷ್ಟು ಸೋತರೂ ನಾನು ಮುಂದೆ ಗೆಲ್ಲುತ್ತೇನೆ ಎಂಬ ಹುಚ್ಚಂತೂ ಬಿಡದು. ವಿಷಯವಸ್ತುವಿಗೆ ನಮ್ಮನ್ನು ಮಾರಿಕೊಂಡ ಅಭ್ಯಾಸಗಳೆಲ್ಲ ನಮ್ಮನ್ನು ದುಃಖದ ಗುಂಡಿಗೆ ನೂಕು ತ್ತವೆ. ನಾವು ಅಲ್ಲಿಂದ ಗುಲಾಮರಾಗಿಯೇ ಏಳುವುದು. ಸ್ವತಂತ್ರರಾಗಿ ಪಾರಾಗುವ ಭರವಸೆಯೆ ಇಲ್ಲ. ಆನೆಯನ್ನು ಹಿಡಿಯುವುದು ಹೀಗೆ. ಒಂದು ಹಳ್ಳವನ್ನು ತೋಡಿ, ಅದರ ಮೇಲೆ ಬಿದಿರು ಚಾಪೆಯನ್ನು ಮುಚ್ಚಿ, ಮಣ್ಣನ್ನು ಹರಡಿ ಏನಾದರೂ ಬೀಜ ಬಿತ್ತಿ ಬೆಳೆ ಬೆಳೆಯುವರು. ಆನೆ ಅದನ್ನು ತಿನ್ನಲು ಬಂದೊಡನೆಯೆ ಅದರ ಭಾರಕ್ಕೆ ಚಾಪೆ ಕುಸಿಯುವುದು, ಹಳ್ಳದೊಳಕ್ಕೆ ಬೀಳುವುದು. ಇನ್ನು ಏನು ಮಾಡಿದರೂ ಮೇಲೆ ಬರುವಂತೆ ಇಲ್ಲ. ಮನುಷ್ಯ ಇದಕ್ಕೆ ಕಾದಿದ್ದು ಅದಕ್ಕೆ ಸರಪಳಿ ಹಾಕಿ ಕಟ್ಟಿ, ಸಾಕಿದ ಆನೆಗಳ ಮೂಲಕ ಅದನ್ನು ಪಳಗಿಸಿ ಬದುಕಿರುವ ತನಕ ಅದನ್ನು ತನ್ನ ಗುಲಾಮನನ್ನಾಗಿ ಮಾಡಿಕೊಳ್ಳುವನು. ವಿಷಯವಸ್ತುವನ್ನು ಅರಸಿ ಹೋದರೆ ನಮ್ಮ ಪಾಡು ಇದು. ನಾವು ವಿದ್ಯಾವಂತ ರಾಗಿರಬಹುದು, ಬಲಾಢ್ಯರಾಗಿರಬಹುದು, ಆದರೇನಂತೆ, ವಿಷಯವಸ್ತುವಿನ ಆಸೆಗೆ ಇದನ್ನೆಲ್ಲ ಬಲಿದಾನ ಮಾಡುವೆವು. ಇಂದ್ರಿಯ ಸುಖಕ್ಕೆಲ್ಲ ಒಂದು ಆದಿ ಅಂತ್ಯವಿದೆ. ಅದು ಮುಂಚೆ ಇರುವುದಿಲ್ಲ. ಅದರ ಮೇಲೆ ನಮಗೆ ಆಸೆ ಹುಟ್ಟುವುದು. ಅದನ್ನು ಪಡೆಯಬೇಕೆಂದು ಮನಸ್ಸು ಬಲಾತ್ಕರಿಸುವುದು. ಅಂತೂ ಹೇಗೋ ಅದನ್ನು ಪಡೆಯಲು ಯತ್ನಿಸುವೆವು. ನ್ಯಾಯವೊ ಅನ್ಯಾಯವೊ ಅದನ್ನು ಪರಿಗಣಿಸುವುದಿಲ್ಲ. ಆ ವಸ್ತುವನ್ನು ಅನುಭವಿಸುತ್ತಿರುವಾಗ ಸುಖದ ಆದಿ ಪ್ರಾರಂಭವಾಗು ವುದು. ಅನುಭವಿಸುತ್ತ ಅನುಭವಿಸುತ್ತ ಇದ್ದಂತೆ ಸುಖ ತೀರುವುದು. ಅದನ್ನು ಇನ್ನೂ ಹೆಚ್ಚು ಅನುಭವಿಸುವುದಕ್ಕೆ ಆಗುವುದಿಲ್ಲ. ಸ್ವಲ್ಪ ಕಾಲವಾದಮೇಲೆ, ಪುನಃ ಆ ವಸ್ತುವಿನ ಮೇಲೆ ಬಯಕೆ ಚಿಗುರುವುದು. ಅದನ್ನು ಸಂಗ್ರಸಿಸುತ್ತಿರುವುದು, ಅನುಭವಿಸುತ್ತಿರುವುದು, ಹೀಗೆ ಸಾಗುತ್ತಿರುವುದು. ಗಾಣಕ್ಕೆ ಕಟ್ಟಿದ ಎತ್ತು ಹೇಗೆ ಗಾಣದ ಸುತ್ತಲೂ ಹೋಗುತ್ತಿರುವುದೋ ಹಾಗೆ ನಾವೂ ಅಭ್ಯಾಸದ ಗಾಣದ ಸುತ್ತ ಹೋಗುತ್ತಿರುವೆವು.

ಜ್ಞಾನಿ ಇವುಗಳಲ್ಲಿ ರಮಿಸುವುದಿಲ್ಲ. ಏಕೆಂದರೆ ಅವನು ಈ ವಿಷಯವಸ್ತುಗಳು ಎಂತಹ ದುಃಸ್ಥಿತಿಗೆ ಒಯ್ಯುವುವು ಎಂಬುದನ್ನು ತಕ್ಷಣವೆ ಊಹಿಸಬಲ್ಲ. ನಾವು ಯಾವುದನ್ನು ಸುಖ ಎನ್ನುತ್ತೇವೆಯೊ ಅದು ಗಾಳದ ಕೊನೆಯಲ್ಲಿರುವ ಹುಳು. ಮೀನು ಅದನ್ನು ತಿನ್ನಲು ಹೋದರೆ ತನ್ನ ಪ್ರಾಣವನ್ನು ಆಹುತಿಯಾಗಿ ಕೊಡಬೇಕು. ಇದನ್ನು ಜ್ಞಾನಿ ಇತರರ ಅನುಭವದಿಂದ ಕಲಿಯುತ್ತಾನೆ. ತನ್ನ ಬುದ್ಧಿಬಲದಿಂದಲೆ ಅದನ್ನು ತಿಳಿಯುತ್ತಾನೆ. ಮುಂಚೆ ವಿಷಯವಸ್ತುವಿನ ಕಡೆ ಇಂದ್ರಿಯ ಜಾರುವುದಕ್ಕೆ ಅವಕಾಶ ಕೊಡುವುದಿಲ್ಲ. ಒಂದು ಸಲ ಜಾರಿದರೆ ಆಯಿತು, ಮಧ್ಯೆ ಎಲ್ಲಿಯೂ ನಿಲ್ಲುವಂತೆಯೆ ಇಲ್ಲ. ಚೆಂಡೊಂದು ಮೇಲಿನ ಮೆಟ್ಟಲಿನಿಂದ ಉರುಳುವಂತೆ ಇದು. ಒಂದು ಸಲ ಉರುಳಿತು ಎಂದರೆ ಮಧ್ಯದಲ್ಲಿ ಎಲ್ಲಿಯೂ ನಿಲ್ಲುವುದಕ್ಕೆ ಆಗುವುದಿಲ್ಲ. ಉರುಳುತ್ತಾ ಉರುಳುತ್ತಾ ವೇಗ ಹೆಚ್ಚಿ ತಳ ಮುಟ್ಟುವವರೆಗೆ ಹೋಗುವುದು. ಜ್ಞಾನಿ ಇವುಗಳನ್ನೆಲ್ಲ ಮುಂಚೆಯೆ ಆಲೋಚಿಸ ಬಲ್ಲ. ಅದಕ್ಕೆ ಅವನು ಸುಖಕ್ಕಾಗಿ ಹೊರಗೆ ಅಲೆದಾಡುವುದಿಲ್ಲ. ತನ್ನಲ್ಲಿಯೇ ಚಿಲುಮೆಯನ್ನು ತೋಡುವನು. ಅದರೊಳಗಿಂದ ಜಿನುಗುವುದು ಪರಮಾತ್ಮನ ಆನಂದದ ಬುಗ್ಗೆ. ಅದನ್ನು ಪಾನ ಮಾಡುವನು. ಸರ್ವಸಂಶಯಗಳನ್ನು ನಿವಾರಿಸುವುದು ಇದು. ಭವರೋಗವನ್ನು ಕೊನೆಗಾಣಿಸುವುದು ಇದು. ಇನ್ನವನು ಮರ್ತ್ಯನಲ್ಲ, ಅಮೃತಾತ್ಮನಾಗುವನು.

\begin{verse}
ಶಕ್ನೋತೀಹೈವ ಯಃ ಸೋಢುಂ ಪ್ರಾಕ್ ಶರೀರವಿಮೋಕ್ಷಣಾತ್~।\\ಕಾಮಕ್ರೋಧೋದ್ಭವಂ ವೇಗಂ ಸ ಯುಕ್ತಃ ಸ ಸುಖೀ ನರಃ \versenum{॥ ೨೩~॥}
\end{verse}

{\small ಶರೀರವನ್ನು ಬಿಡುವುದಕ್ಕೆ ಮುಂಚೆ, ಯಾರು ಕಾಮಕ್ರೋಧಗಳಿಂದ ಉಂಟಾದ ವೇಗವನ್ನು ಸಹಿಸಿಕೊಳ್ಳಲು ಶಕ್ತನಾಗುವನೋ ಅವನೇ ಯೋಗಿ, ಅವನೇ ಸುಖಿ.}

ಕಾಲವಾದಮೇಲೆ ಕಾಮಾದಿಗಳನ್ನು ಜಯಿಸುವುದಕ್ಕೆ ಆಗುವುದಿಲ್ಲ. ಸಾಯುವುದಕ್ಕೆ ಮುಂಚೆ ಅದನ್ನು ಮಾಡಬೇಕಾಗಿದೆ. ಕೆರೆಯಲ್ಲಿ ಮೀನನ್ನು ಹಿಡಿಯಬೇಕಾದರೆ ನೀರು ತುಂಬಿದಾಗ ಪ್ರಯತ್ನಿಸ ಬೇಕು. ಕೆರೆಯೆಲ್ಲ ಬತ್ತಿಹೋಗಿ ಅಂಗಳ ಖಾಲಿಯಾಗಿರುವಾಗ ಯಾವ ಮೀನು ಅಲ್ಲಿರುವುದು ಹಿಡಿಯುವುದಕ್ಕೆ? ಇಂದ್ರಿಯಗಳನ್ನು ವಿಷಯವಸ್ತು ಯಾವಾಗಲೂ ಸೆಳೆಯುತ್ತಿರುವಂತೆ ಕಾಣು ವುದು. ಅದು ಭರದಿಂದ ಅತ್ತ ಸಾಗುವುದು, ವಿಷಯವಸ್ತುವಿನ ಕಡೆ ಹೋಗುವುದು. ಹುಡುಗರು ಜಾರುಗುಪ್ಪೆಯಲ್ಲಿ ಮೇಲಿನಿಂದ ಕೆಳಕ್ಕೆ ಜಾರುವುದು ಸರಾಗ. ಅದಕ್ಕೆ ವಿರೋಧವಾಗಿ ಹೋಗಬೇಕಾ ದರೆ ಕಷ್ಟ. ವಿಷಯವಸ್ತು ಯಾವ ವೇಗದಿಂದ ಇಂದ್ರಿಯಗಳನ್ನು ಸೆಳೆಯುತ್ತಿದೆಯೊ ಅದಕ್ಕಿಂತ ಹೆಚ್ಚು ವೇಗದಿಂದ ಮನಸ್ಸನ್ನು ದೇವರ ಕಡೆಗೆ ಸೆಳೆದರೆ ಮಾತ್ರ ಸಾಧ್ಯ. ಅದು ಕಷ್ಟವೇನೊ ನಿಜ. ಆದರೆ ಹೋರಾಟದಿಂದ ಸಾಧ್ಯವಾಗುವುದು. ವಿಮಾನ ನೆಲದ ಮೇಲೆ ಇರುವುದು. ಅದು ಆಕಾಶಕ್ಕೆ ಏರಬೇಕಾದರೆ, ನೆಲ ಅದನ್ನು ಎಷ್ಟು ವೇಗದಿಂದ ಸೆಳೆಯುತ್ತದೆಯೊ ಅದಕ್ಕಿಂತ ಹೆಚ್ಚು ವೇಗವನ್ನು ವ್ಯತಿರಿಕ್ತವಾಗಿ ಉತ್ಪತ್ತಿಮಾಡಿಕೊಂಡರೆ ಮಾತ್ರ ಸಾಧ್ಯ. ಆ ಶಕ್ತಿಯನ್ನು ಉತ್ಪತ್ತಿ ಮಾಡಿಕೊಳ್ಳು ವುದಕ್ಕೆ ಅದರಲ್ಲಿ ಶಕ್ತಿ ಇರುವುದರಿಂದ ಅದು ನೆಲವನ್ನು ಬಿಟ್ಟು ಮೇಲೇರುವುದು. ಅದರಂತೆಯೆ ಇಂದ್ರಿಯ ಪ್ರಪಂಚಕ್ಕೆ ವಿರೋಧವಾಗಿ ಭಗವಂತನ ಮೇಲೆ ಅಭೀಪ್ಸೆ ಬಲವಾಗಿದ್ದರೆ ಮಾತ್ರ ದೇವರೆಡೆಗೆ ಹಾರಬಹುದು. ಯೋಗಿ ಈ ಶಕ್ತಿಯನ್ನು ಉತ್ಪತ್ತಿ ಮಾಡಿಕೊಂಡು ಇಂದ್ರಿಯಾತೀತ ಅನುಭವದೆಡೆಗೆ ಹಾರುವನು.

ಶರೀರ ಬೀಳುವುದಕ್ಕೆ ಮುಂಚೆಯೇ ಅಂದರೆ ಇಳಿ ವಯಸ್ಸಿನಲ್ಲಿ ಸಾಯುವುದಕ್ಕೆ ಸ್ವಲ್ಪ ಮುಂಚೆ ಇದನ್ನು ಅಭ್ಯಾಸ ಮಾಡಬೇಕೆಂದಲ್ಲ. ಇಳಿ ವಯಸ್ಸಿನಲ್ಲಿ ಇಂದ್ರಿಯಗಳೆಲ್ಲ ದುರ್ಬಲವಾಗಿ ಹೋಗಿವೆ. ಅದನ್ನು ನಿಗ್ರಹಿಸುತ್ತೇನೆ ಎಂದರೆ ಅದಕ್ಕೆ ಅರ್ಥವೇ ಇಲ್ಲ. ಕಟ್ಟೆಯಲ್ಲಿರುವ ನೀರೆಲ್ಲ ಹರಿದು ಹೋದಮೇಲೆ ಇನ್ನು ತೂಬನ್ನು ಮುಚ್ಚಿ ಪ್ರಯೋಜನವೇನು? ನೀರು ಕಟ್ಟೆಗೆ ವೇಗವಾಗಿ ಹರಿದು ಬರುತ್ತಿರುವಾಗ ತೂಬನ್ನು ಮುಚ್ಚಬೇಕು. ಆಗ ನೀರನ್ನು ಸಂಗ್ರಹಿಸುತ್ತೇವೆ. ಅದನ್ನು ಉಪಯುಕ್ತವಾದ ರೀತಿಯಲ್ಲಿ ಉಪಯೋಗಿಸಲು ಸಾಧ್ಯವಾಗುವುದು. ಯೌವನದಲ್ಲಿ, ಇನ್ನೂ ಇಂದ್ರಿಯಗಳು ಬಲವಾಗಿರುವಾಗ ಇಂದ್ರಿಯಗಳನ್ನು ನಿಗ್ರಹಿಸುವುದನ್ನು ಅಭ್ಯಾಸ ಮಾಡಬೇಕು. ಬದುಕಿರುವಾಗ ಅದು ನಮಗೆ ಸಿದ್ಧಿಸಿದ್ದರೆ ಸೋತಮೇಲೂ ಪುನರ್ಜನ್ಮದಲ್ಲಿಯೂ ನಾವು ಆ ನಿಗ್ರಹದೊಡನೆ ಹುಟ್ಟುತ್ತೇವೆ. ಈಗ ಇಲ್ಲದೆ ಮುಂದೆ ನಮಗೆ ಸಿಕ್ಕುವುದು ಎಂದರೆ ಬೀಜವನ್ನು ಬಿತ್ತಲು ಬೆಳೆಯನ್ನು ಕೊಯ್ಯಲು ಕುಡುಗೋಲನ್ನು ತೆಗೆದುಕೊಂಡು ಹೋದಂತೆ.

ಯೋಗಿ ಯೌವನದಲ್ಲಿಯೆ, ಇಂದ್ರಿಯಗಳು ಬಲವಾಗಿರುವಾಗಲೆ ಅವುಗಳನ್ನು ನಿಗ್ರಹಿಸುವನು. ಇಂದ್ರಿಯಗಳಿಗೆ ಎಷ್ಟು ಕೊಟ್ಟರೂ ಇನ್ನೂ ಬೇಡ ಎಂದು ಹೇಳುವುದಿಲ್ಲ. ನಾವು ಅವುಗಳು ಕೇಳುವುದನ್ನು ಕೊಡುವುದರಲ್ಲೆ ನಮ್ಮ ಜೀವನವೆಲ್ಲ ಕಳೆಯುವುದು. ನಾವು ಅದರ ಗುಲಾಮರಾಗುವೆವು. ಯೋಗಿಯಾದರೊ ಪರಮ ಸುಖಿ. ಅವನು ಹೊರಗೆ ವಿಷಯವಸ್ತುಗಳಲ್ಲಿ ಅದನ್ನು ಹುಡುಕಲು ಹೋಗುವುದಿಲ್ಲ. ತನ್ನಲ್ಲಿಯೇ ಸದಾ ಕಾಲವೂ ಹರಿಯುತ್ತಿರುವ ಅಮೃತ ಚಿಲುಮೆಯನ್ನು ಪಾನಮಾಡಿ ಸುಖಿಯಾಗಿರುವನು.

\begin{verse}
ಯೋಽಂತಃಸುಖೋಽಂತರಾಮಾಸ್ತಥಾರ್ಂಜ್ಯೋತಿರೇವ ಯಃ~।\\ಸ ಯೋಗೀ ಬ್ರಹ್ಮನಿರ್ವಾಣಂ ಬ್ರಹ್ಮಭೂತೋಽಧಿಗಚ್ಛತಿ \versenum{॥ ೨೪~॥}
\end{verse}

{\small ಯಾರು ಅಂತಸ್ಸುಖಿಯಾಗಿರುವನೊ, ಅಂತರರಾಮನಾಗಿರುವನೊ, ಅಂತರ್ಜ್ಯೋತಿಯಾಗಿರುವನೊ ಆ ಯೋಗಿಯು ಬ್ರಹ್ಮಭೂತನಾಗಿ ನಿರ್ವಾಣವನ್ನು ಪಡೆಯುವನು.}

ಸಾಯುವುದಕ್ಕೆ ಮುಂಚೆ ಈ ಗುಣಗಳು ಒಬ್ಬನಲ್ಲಿದ್ದರೆ ಅವನು ಸತ್ತಮೇಲೆ ಮುಕ್ತನಾಗುತ್ತಾನೆ. ಇಲ್ಲದೆ ಇದ್ದರೆ ಅವನು ಹಿಂತಿರುಗಿ ಬರಬೇಕಾಗುವುದು. ಅವನ ಸುಖ ಆತ್ಮನಲ್ಲಿಯೇ ಇದೆ. ಅವನು ಸುಖವನ್ನು ಆರಿಸಿಕೊಂಡು ಹೊರಗೆ ಹೋಗುವುದಿಲ್ಲ. ಅವನ ಸುಖ ವಿಷಯವಸ್ತುಗಳ ಆಧಾರದ ಮೇಲೆ ಇಲ್ಲ. ಇಂದ್ರಿಯ ವೇದನೆಯ ಮೂಲಕವೂ ಇಲ್ಲ. ಅವನು ತನ್ನಲ್ಲಿರುವ ಪಂಚಕೋಶಗಳೆಂಬ ತೆರೆಯನ್ನು ಸರಿಸಿ ಅಂತರಾಳಕ್ಕೆ ಹೋಗಿರುವನು. ಅಲ್ಲಿರುವ ಪರಮಾತ್ಮನನ್ನು ಕುರಿತು ಚಿಂತಿಸುವನು, ಅವನನ್ನು ಅನುಭವಿಸುವನು, ಅವನನ್ನು ಸವಿಯುವನು. ಇದೇ ಅವನ ಸುಖ. ಈ ಸುಖಕ್ಕೆ ಮಿಗಿಲಾದ ಸುಖವಿಲ್ಲ. ವಿಷಯಸುಖದ ಪ್ರಪಂಚದಲ್ಲಿ ಹೊರಗಿನ ಯಾವ ವಸ್ತುವಿನ ಹಂಗು ಇಲ್ಲದೆ ತನ್ನ ಪಾಡಿಗೆ ತಾನು ಇದ್ದರೂ ಅವನಷ್ಟು ಸುಖಪುರುಷ ಇನ್ನಾರೂ ಇಲ್ಲ.

ಅವನ ಆರಾಮ ಎಲ್ಲಾ ಒಳಗೆ ಇರುವುದು. ಅವನು ಅದಕ್ಕಾಗಿ ಹೊರಗೆ ದೃಶ್ಯಗಳನ್ನು ಹುಡುಕಿಕೊಂಡು ಹೋಗಬೇಕಾಗಿಲ್ಲ. ಅನೇಕ ವೇಳೆ ಆಫೀಸಿನಲ್ಲಿ ಕೆಲಸ ಮಾಡಿ ಸಾಕಾಗಿ ತಾನು ಹಾಯಾಗಿ ರಜಾ ತೆಗೆದುಕೊಂಡು ಇರಬೇಕೆಂದು ವ್ಯಕ್ತಿ ಬಯಸುವನು. ರಜವನ್ನು ತೆಗೆದುಕೊಳ್ಳು ವನು. ಮನೆಯಲ್ಲಿಯೂ ಇರುವನು. ಹಾಯಾಗಿ ಮಾತ್ರ ಇರುವುದಿಲ್ಲ. ಇಲ್ಲಿ ಇಂದ್ರಿಯಗಳು ಅವನನ್ನು ಸುಮ್ಮನೆ ಬಿಡುವುವೇ? ಯಾವಾಗಲೂ ಎಲ್ಲೆಲ್ಲೊ ಅವನನ್ನು ನೂಕುತ್ತಿರುವುವು. ಬಹುಶಃ ಆಫೀಸಿಗಿಂತ ಹೆಚ್ಚು ತರಲೆ ತಾಪತ್ರಯವನ್ನೆಲ್ಲಾ ಮನೆಯಲ್ಲಿರುವಾಗ ಹಚ್ಚಿಕೊಳ್ಳುವನು. ಇಂದ್ರಿಯ ಗುಲಾಮನಾಗಿ ವಿರಾಮ ಸಿಕ್ಕುವುದಿಲ್ಲ. ಅವನು ಎಲ್ಲಿದ್ದರೂ ಜೀತ ಮಾಡಲೇಬೇಕು. ನಿಜವಾಗಿ ವಿರಾಮ ಸಿಕ್ಕುವುದು ಇಂದ್ರಿಯ ಜಯಿಗೆ. ಅವನಷ್ಟು ಆರಾಮವಾಗಿ ಯಾರಿರುವರು? ಅವನು ಇಂದ್ರಿಯ ವಸ್ತುಗಳು ಕರೆದತ್ತ ಹೋಗುವುದಿಲ್ಲ.

ಹಾಗೆಯೇ ಯೋಗಿಯ ಬೆಳಕು, ಜ್ಞಾನವೆಲ್ಲ ಒಳಗೆ ಇರುವುದು. ಪರಾತ್ಪರ ವಸ್ತುವನ್ನು ಕಾಣುವುದು, ಅದರ ಬೆಳಕಿನಲ್ಲಿ ಎಲ್ಲವನ್ನೂ ನೋಡುವುದೇ ಜ್ಞಾನ. ಈ ಜ್ಞಾನ ಯಾವಾಗಲೂ ಹೊರಗಿನಿಂದ ಬರುವುದಿಲ್ಲ. ಹೊರಗೆ ನಮಗೆ ಸಿಕ್ಕುವುದು ಒಳಗೆ ಇರುವುದನ್ನು ಪಡೆದುಕೊಳ್ಳು ವುದಕ್ಕೆ ಸಹಾಯ ಮಾತ್ರ. ಯಾರೂ ಜ್ಞಾನವನ್ನು ನಮಗೆ ಕೈತುತ್ತಿನಂತೆ ಹೊರಗೆ ಬಡಿಸಲಾರರು. ನಾವೇ ಅದನ್ನು ಮಾಡಿಕೊಳ್ಳಬೇಕು ಎಂಬುದನ್ನು ಹೊರಗಿನಿಂದ ಪಡೆಯಬಹುದು. ಆಧ್ಯಾತ್ಮಿಕಅ ವಿಷಯಗಳನ್ನು ವಿವರಿಸುವ ಪುಸ್ತಕಗಳನ್ನು ಓದುವುದು ನಮಗೆ ಸಹಾಯ ಮಾಡುವುದು. ಆದರೆ ಅದೇ ಸಾಲದು. ನಾವು ಸಾಧನೆ ಮಾಡಿ ಒಳಗೆ ಅದನ್ನು ತಿಳಿದುಕೊಳ್ಳಬೇಕು. ಆ ದಿವ್ಯಜ್ಯೋತಿ ಎಲ್ಲಾ ಜೀವಿಗಳ ಹೃದಯದ ಅಂತರಾಳದಲ್ಲಿಯೂ ಇದೆ. ಆದರೆ ನಾವು ಅಜ್ಞಾನದ ತೆರೆಗಳಿಂದ ಅದನ್ನು ಮುಚ್ಚಿ ಅಯ್ಯೋ ಕತ್ತಲು ಎಂದು ಅಳುತ್ತಿರುವೆವು. ಯಾವಾಗ ತೆರೆಯನ್ನು ಅತ್ತ ಸರಿಸುವೆವೋ ವೇದೋಪನಿಷತ್ತಿನ ಉಗಮ ಸ್ಥಾನ ನಮ್ಮಲ್ಲಿಯೇ ಇರುವುದು ಗೊತ್ತಾಗುವುದು.

ಯಾರು ಅಂತಸ್ಸುಖ ಅಂತರಾರಾಮ ಅಂತರ್ಜ್ಯೋತಿಯನ್ನು ಪಡೆದಿರುವನೋ ಅವನು ಬ್ರಹ್ಮ ಸ್ವರೂಪವನ್ನು ಪಡೆದು ಮುಕ್ತನಾಗುತ್ತಾನೆ. ಹೇಗೆ ಇದ್ದಲೊಂದು ಬೆಂಕಿಯ ಮಧ್ಯ ಇದ್ದರೆ ಅದು ಕೂಡ ಧಗಧಗಿಸುವ ಬೆಂಕಿಯಂತೆ ಆಗುವುದೋ ಹಾಗೆ ಯಾವುದನ್ನು ನಾವು ಕುರಿತು ಚಿಂತಿಸು ತ್ತಿರುವೆವೊ ಅದರಂತೆಯೆ ಆಗುವೆವು. ಆದಕಾರಣ ಶಾಸ್ತ್ರಗಳು, ಬ್ರಹ್ಮನನ್ನು ಚಿಂತಿಸುತ್ತಿದ್ದರೆ ಬ್ರಹ್ಮನೇ ಆಗುವನು ಎಂದು ಸಾರುವುವು.

\begin{verse}
ಲಭಂತೇ ಬ್ರಹ್ಮನಿರ್ವಾಣಮೃಷಯಃ ಕ್ಷೀಣಕಲ್ಮಷಾಃ~।\\ಛಿನ್ನದ್ವ್ಯೆಧಾ ಯತಾತ್ಮಾನಃ ಸರ್ವಭೂತಹಿತೇ ರತಾಃ \versenum{॥ ೨೫~॥}
\end{verse}

{\small ಕ್ಷೀಣಕಲ್ಮಶರಾಗಿ, ದ್ವಂದ್ವರಹಿತರಾಗಿ, ಜಿತೇಂದ್ರಿಯರೂ, ಸರ್ವಭೂತಹಿತದಲ್ಲಿ ನಿರತರಾಗಿರುವವರೂ ಆದ ಪುಷಿಗಳು ಬ್ರಹ್ಮನಿರ್ವಾಣವನ್ನು ಪಡೆಯುವರು.}

ಕಲ್ಮಷವಿಲ್ಲದವರು ಎಂದರೆ ಅವರು ಕೊಳೆಯನ್ನು ಎಂದೆಂದಿಗೂ ಶುದ್ಧಮಾಡಿಕೊಂಡಿರುವರು. ಸಾಧಾರಣ ಮನುಷ್ಯನಿಗೆ ಅಂದಂದಿನ ಕೊಳೆಯನ್ನು ಅಂದಂದು ಶುದ್ಧಿ ಮಾಡಿಕೊಳ್ಳುವುದೇ ದುಸ್ತರವಾಗಿರುತ್ತದೆ. ಪುನಃ ಮಾರನೆ ದಿನದ ಹೊತ್ತಿಗೆ ಆಗಲೇ ಆ ಕೊಳೆ ಸೇರಿಕೊಂಡಿರುವುದು. ಇದು ಇಂದು ಸ್ನಾನ ಮಾಡಿದಂತೆ. ನಾಳೆ ನಾವು ಪುನಃ ಸ್ನಾನಮಾಡಬೇಕು. ಇಲ್ಲದೇ ಇದ್ದರೆ ನಾಳೆಯ ಕೊಳೆ ಹೋಗುವುದಿಲ್ಲ. ಇಲ್ಲಿ ಯೋಗಿ ಒಂದೇ ಸಲ ತಾನು ಕಲ್ಮಷದಿಂದ ಪಾರಾಗಿದ್ದಾನೆ. ಇನ್ನು ಮೇಲೆ ಅವನನ್ನು ಯಾವ ಕಲ್ಮಷವೂ ಮುಟ್ಟುವುದಿಲ್ಲ. ಅಜ್ಞಾನದ ಕಿಲುಬು ಇನ್ನು ಮೇಲೆ ಆ ಜ್ಞಾನಕ್ಕೆ ಅಂಟುವುದಿಲ್ಲ. ಹಿತ್ತಾಳೆ ತಾಮ್ರದ ಒಡವೆಗೆ ಚಿನ್ನದ ಮುಲಾಮನ್ನು ಹಾಕಿದ್ದರೆ ಆ ಮುಲಾಮು ಕೆಲವು ಕಾಲದಮೇಲೆ ಬಿಟ್ಟು ಹೋಗುವುದು. ಆದರೆ ಅದು ಚಿನ್ನದ್ದೇ ಆಗಿದ್ದರೆ, ಅದರ ಸ್ವಭಾವ ಎಂದಿಗೂ ಬಿಟ್ಟು ಹೋಗುವುದಿಲ್ಲ. ಯೋಗಿ ಹಾಗೆ ಅಪರಂಜಿಯ ಚಿನ್ನದ ಪಾತ್ರೆಯಂತೆ. ಅದಕ್ಕೆ ಯಾವ ತೊಳೆಯುವುದು ಬೆಳಗುವುದು ಬೇಕಾಗಿಲ್ಲ. ಅದರ ನೈಜ ಸ್ವಭಾವವೇ ಶುಭ್ರತೆ.

ಅವರು ದ್ವಂದ್ವದಿಂದ ಪಾರಾಗಿದ್ದಾರೆ. ಲಾಭ, ನಷ್ಟ, ಜಯ, ಅಪಜಯ ನಿಂದಾಸ್ತುತಿ ಮುಂತಾದ ದ್ವಂದ್ವಗಳು ಅವರನ್ನು ಇನ್ನು ಮೇಲೆ ಅಂಜಿಸಲಾರವು. ಅವರು ದ್ವಂದ್ವವನ್ನು ನೋಡುವುದಿಲ್ಲ. ದ್ವಂದ್ವದ ಹಿಂದೆ ಇರುವ ಸಾಮ್ಯವನ್ನು ನೋಡುತ್ತಾರೆ. ದ್ವಂದ್ವವನ್ನು ನೋಡಿದರೆ ಅದರಿಂದ ಆಗುವ ಪ್ರತಿಕ್ರಿಯೆಗೆ ಒಬ್ಬನು ಬದ್ಧನಾಗುತ್ತಾನೆ. ಯಾರು ದ್ವಂದ್ವದ ಮೇಲೆ ಮನಸ್ಸನ್ನೇ ಹಾಕುವುದಿಲ್ಲವೋ ಅವನಿಗೆ ಇವು ಏನನ್ನೂ ಮಾಡಲಾರವು. ನಾಮರೂಪದ ದೃಷ್ಟಿಯಿಂದ ನೋಡಿದರೆ, ನಾನಾತ್ವ ಮತ್ತು ದ್ವಂದ್ವಗಳು. ಯಾರು ನಾಮರೂಪಗಳನ್ನು ಭೇದಿಸಿ ಹೋಗವರೋ ಅವರು ಎಲ್ಲಾ ಕಡೆಯಲ್ಲೂ ಒಂದನ್ನೇ ನೋಡುತ್ತಾರೆ.

ಅವರು ಜಿತೇಂದ್ರಿಯರು. ತಮ್ಮಇಂದ್ರಿಯಗಳನ್ನು ನಿಗ್ರಹಿಸಿದವರು. ವಿಷಯ ವಸ್ತುವಿನ ಕಡೆ ಹೋಗುವುದನ್ನು ತಡೆದು ಆ ಮನೋಶಕ್ತಿಯನ್ನು ಪರಮಾತ್ಮನ ಕಡೆ ಹೋಗುವಂತೆ ಮಾಡುವರು. ಇಂದ್ರಿಯನಿಗ್ರಹ ಜಯಕಾರಿಯಾಗಬೇಕಾದರೆ, ಬರೀ ವಿಷಯ ವಸ್ತುವಿನ ಕಡೆ ಹೋಗುವ ಮನಸ್ಸನ್ನು ತಡೆದರೆ ಸಾಲದು. ಮನಸ್ಸನ್ನು ದೇವರ ಕಡೆ ಹರಿಸಬೇಕು. ಮನಸ್ಸಿಗೆ ಯಾವುದಾದರೂ ಒಂದು ನೆಲೆ ಬೇಕು. ಒಂದು ಕಡೆ ಹೋಗಬೇಡ ಎಂದರೆ ಸಾಲದು. ಎಲ್ಲಿ ಹೋಗಬೇಕು ಎಂಬುದನ್ನು ಹೇಳಬೇಕು. ಒಂದು ಕಡೆ ನೀರು ಹರಿದುಕೊಂಡು ಹೋಗುತ್ತಿದೆ. ಅಲ್ಲಿ ಕಟ್ಟೆಹಾಕಿ, ಆ ನೀರನ್ನು ಬೇರೆ ಕಡೆ ಹೋಗುವಂತೆ ಮಾಡಬೇಕು. ಇಂದ್ರಿಯವನ್ನು ಯಶಸ್ವಿಯಾಗಿ ನಿಗ್ರಹಿಸಿದ್ದೇನೆ ಎಂದರೆ ಅದು ಬಹಳ ಕಾಲ ಹಾಗೆ ಇರಲಾರದು. ನಿಜವಾದ ಈ ರಹಸ್ಯವನ್ನು ಅರಿತವನು ಅದನ್ನು ಒಂದು ಕಡೆ ಮುಚ್ಚುತ್ತಾನೆ, ಮತ್ತೊಂದು ಕಡೆ ತೆಗೆಯುತ್ತಾನೆ. ಇಂದ್ರಿಯದ ಕಡೆ ಹೋಗುವುದನ್ನು ಮುಚ್ಚುತ್ತಾನೆ, ಪರಮಾತ್ಮನ ಕಡೆ ಹರಿದುಹೋಗುವುದಕ್ಕೆ ಬಾಗಿಲನ್ನು ತೆಗೆಯುತ್ತಾನೆ.

ಇಂತಹ ಯೋಗಿಗಳು ಬದುಕಿರುವ ಪರಿಯಂತರ ಸರ್ವಭೂತಗಳ ಹಿತದಲ್ಲಿ ತನ್ಮಯರಾಗಿರು ವರು. ಅವರು ಎಲ್ಲರ ಕಣ್ಣು ಮುಂದೆ ಒಳ್ಳೆಯ ಕೆಲಸವನ್ನು ಯಾವಾಗಲೂ ಮಾಡುವುದು ನಮಗೆ ತೋರುವುದಿಲ್ಲ. ಆದರೆ ಯೋಗಿ ತಾನೊಂದು ಗುಹೆಯಲ್ಲಿರಲಿ ಅಲ್ಲಿಯೂ ಎಲ್ಲರಿಗೂ ಒಳ್ಳೆಯ ದನ್ನು ಕೋರುವನು. ತಾನು ಕಂಡ, ಅನುಭವಿಸಿದ ಶಾಂತಿಯನ್ನು ಎಲ್ಲರಿಗೂ ಯಾವ ಜಾತೀ ಕೋಮಿನ ಭಾವನೆಯೂ ಇಲ್ಲದೆ ನೀಡುತ್ತಿರುವನು. ಎಲ್ಲರೂ ಕಷ್ಟದಿಂದ ಪಾರಾಗಬೇಕು, ಎಲ್ಲರೂ ಬಂಧನದಿಂದ ಪಾರಾಗಬೇಕು ಎಂಬುದೊಂದೆ ಅವನ ಬಯಕೆ. ದೇವರ ಸಮೀಪಕ್ಕೆ ಹೋಗುತ್ತಿರು ವವನ ಹೃದಯ ವಿಶಾಲವಾಗುತ್ತಾ ಬರುವುದು. ಇಡೀ ಮಾನವನಕೋಟಿಗೆ ಅವನ ಹೃದಯ ಮರುಗು ವುದು. ಅವನಿಗೆ ಎಲ್ಲರೂ ಸಹೋದರರಾಗುವರು. ಏಕೆಂದರೆ ಎಲ್ಲರೂ ಭಗವಂತನ ಮಕ್ಕಳು. ಪ್ರಪಂಚದಲ್ಲಿರುವುದನ್ನೆಲ್ಲ ಬಾಚಿ ತಬ್ಬುವಂತಹ ಪ್ರೇಮ ಅವನದಾಗುವುದು. ಏಕೆಂದರೆ ಅವನು ಎಲ್ಲಡೆಯಲ್ಲೂ ಭಗವಂತನನ್ನು ನೋಡುವನು.

\begin{verse}
ಕಾಮಕ್ರೋಧವಿಯುಕ್ತಾನಾಂ ಯತೀನಾಂ ಯತಚೇತಸಾಮ್~।\\ಅಭಿತೋ ಬ್ರಹ್ಮನಿರ್ವಾಣಂ ವರ್ತತೇ ವಿದಿತಾತ್ಮನಾಮ್ \versenum{॥ ೨೬~॥}
\end{verse}

{\small ಕಾಮಕ್ರೋಧಗಳಿಂದ ಪಾರಾಗಿ, ಮನಸ್ಸನ್ನು ತಮ್ಮ ಸ್ವಾಧೀನದಲ್ಲಿಟ್ಟುಕೊಂಡು ಆತ್ಮವನ್ನು ಅರಿತಿರುವ ಯೋಗಿಗಳಿಗೆ ಎಲ್ಲೆಲ್ಲಿಯೂ ಬ್ರಹ್ಮನಿರ್ವಾಣ ಇರುವುದು.}

ಇಲ್ಲಿ ಯತಿ ಯಾರು ಎಂಬುದನ್ನು ಶ‍್ರೀಕೃಷ್ಣ ವಿವರಿಸುವನು, ಮತ್ತು ಅವನಿಗೆ ಏನು ದೊರಕು ವುದು ಎಂಬದನ್ನು ಹೇಳುವನು. ಯತಿ ಮುಂಚೆ ಕಾಮ ಕ್ರೋಧಗಳಿಂದ ಪಾರಾಗಬೇಕು. ನಮ್ಮನ್ನು ಈ ಸಂಸಾರವೆಂಬ ಗೂಟಕ್ಕೆ ಕಟ್ಟಿ ಹಾಕಿರುವ ದೊಡ್ಡ ಸರಪಳಿಯೇ ಇದು. ಕಾಮವೆಂಬುದು ನಮ್ಮನ್ನು ಒಂದು ವಿಷಯವಸ್ತುವಿಗೆ ಕರೆದುಕೊಂಡುಹೋಗಿ ಅದರಿಂದ ಬರುವ ಅನುಭವಕ್ಕೆ ನಮ್ಮನ್ನು ಕಟ್ಟಿ ಹಾಕುವುದು. ನಮ್ಮಲ್ಲಿ ಒಂದು ಆಸೆಯಲ್ಲ ಇರುವುದು, ಹಲವು ಆಸೆಗಳಿವೆ. ಎಲ್ಲಕ್ಕೂ ದಾಸರಾಗು ವೆವು. ಇಂದ್ರಿಯದ ಸೆರೆಮನೆಯಲ್ಲಿ ಬಂಧಿತರಾಗುವೆವು. ಪ್ರಾಣಿಗಳು ಹೇಗೆ ಪಂಜರದಲ್ಲಿ ಬಂಧಿ ಗಳೊ ಹಾಗೆ ಇರುವೆವು. ಇಂದ್ರಿಯದ ಬಾಗಿಲಿನ ಮೂಲಕ ನಮಗೆ ಆಯಾ ಕಾಲಕ್ಕೆ ಸಿಕ್ಕುವ ಚೂರುರೊಟ್ಟಿ ಮೂಳೆಗಳಲ್ಲದೆ ಬೇರೆ ಇಲ್ಲ. ಯಾವುದಾದರೂ ವಸ್ತು ನಾವು ಆಶಿಸುವ ವಸ್ತುವಿನ ಮಧ್ಯ ಆತಂಕವಾಗಿ ಬಂದರೆ ಅದರ ಮೇಲೆ ರೇಗುವೆವು. ಕೋಪ ಬಂದರೆ ನಮ್ಮ ವಿವೇಚನಾ ಶಕ್ತಿಯೆಲ್ಲ ಕದಡಿಹೋಗುವುದು. ಆ ಸಮಯದಲ್ಲಿ ಏನಾದರೂ ಅನಾಹುತ ಮಾಡುವೆವು. ಹಲವು ದಿನಗಳು ಮಾಡಿದ ಒಂದು ಕೆಲಸವನ್ನು ಒಂದು ದಿನದಲ್ಲಿ ಹಾಳುಮಾಡಿಕೊಳ್ಳಬಹುದು. ರಿಪೇರಿ ಮಾಡದ ರೀತಿಯಲ್ಲಿ ಅದಕ್ಕೆ ಬೆಂಕಿ ಇಡುವೆವು. ಯತಿ ಮುಂಚೆಯೇ ಈ ಕಾಮಕ್ರೋಧದ ಬಲೆಗೆ ಬೀಳುವುದಿಲ್ಲ. ಅವನಿಗೆ ಪರಬ್ರಹ್ಮನ ಹೊರತು ಈ ಜಗತ್ತಿನಲ್ಲಿ ಇನ್ನಾವ ಮೋಹವೂ ಇಲ್ಲ. ಎಲ್ಲಾ ಮೋಹಗಳನ್ನೂ ಅವನು ಅದಕ್ಕಾಗಿ ಬಲಿಕೊಡುವನು. ಅವನು ಪ್ರಶಾಂತ ಚಿತ್ತ. ಕೋಪದ ಮೂಲಕ ಅವನು ತನ್ನ ಶಕ್ತಿಯನ್ನು ವ್ಯಯ ಮಾಡಿಕೊಳ್ಳುವುದಿಲ್ಲ. ಕಾಮವಿದ್ದರೇ ಕೋಪ. ಕಾಮವೇ ಬೇರು, ಕೋಪ ಅದರಿಂದ ಹುಟ್ಟಿರುವುದು. ಕಾಮವೇ ಇಲ್ಲದವನಿಗೆ ಇನ್ನು ಅದರ ಫಲವೆಲ್ಲಿ ಬರುವುದು?

ಯತಿ ತನ್ನ ಮನಸ್ಸನ್ನು ನಿಗ್ರಹಿಸಿದವನು. ಅವನು ಸಿಕ್ಕಿದಂತೆ ಮನಸ್ಸನ್ನು ಹರಿಬಿಡುವುದಿಲ್ಲ. ಅವನು ಅದನ್ನು ಕೇಂದ್ರೀಕರಿಸಿ, ಪರಮಾತ್ಮವಸ್ತುವಿನ ಕಡೆ ಹರಿಸುವನು. ಇನ್ನು ಮೇಲೆ ಪ್ರಿಯವಾದ ವೈಷಯಿಕ ವಸ್ತುಗಳು ಅವನಿಗೆ ಚಕ್ಕಳಗುಳಿಯನ್ನು ಇಟ್ಟು ನಗಿಸಲಾರವು. ಅಪ್ರಿಯವಾದ ವಿಷಯಗಳು ಅವನನ್ನು ಬಾಧಿಸಲಾರವು. ಮನೋನಿಗ್ರಹದಿಂದ ಅದ್ಭುತವಾದ ಶಕ್ತಿ ಅವನಿಗೆ ಕರಗತವಾಗುವುದು. ಅದನ್ನು ಅವನು ಬ್ರಹ್ಮನ ಕಡೆ ತಿರುಗಿಸುವನು. ಆಗ ಅವನಿಗೆ ಬ್ರಹ್ಮದರ್ಶನವಾಗುವುದು. ಯಾವಾಗ ಮನಸ್ಸನ್ನು ಅಂತರ್ಮುಖ ಮಾಡುವನೊ, ಆ ನಿಗ್ರಹಿಸಿದ ಶಕ್ತಿ ಎಲ್ಲಾ ಕೋಶಗಳನ್ನು ತೂರಿಹೋಗಿ, ಎಲ್ಲದರ ಹಿಂದೆ ಇರುವ ಬ್ರಹ್ಮದರ್ಶನವಾಗುವುದು. ಯಾವಾಗ ಒಬ್ಬ ತನ್ನೊಳಗೆ ಅದನ್ನು ತಂದಿರುವನೊ ಆಗ ಪ್ರಪಂಚದಲ್ಲಿ ಎಲ್ಲಾ ಕಡೆಯಲ್ಲಿಯೂ ಎಲ್ಲಾ ನಾಮರೂಪಗಳ ಹಿಂದೆಯೂ ಅವನಿಗೆ ಅದೇ ಕಾಣುವುದು. ಅವನು ಪ್ರಪಂಚವನ್ನು ನೋಡುವ ದೃಷ್ಟಿಯೇ ಬದಲಾಯಿಸುವುದು. ಬದ್ಧ ಜೀವಿ ಹೊರಗೆ ಕಾಣುವ ನಾಮರೂಪ ನೋಡುತ್ತಾನೆ. ಮುಕ್ತಜೀವಿ ನಾಮರೂಪಗಳ ಹಿಂದೆ ಇರುವ ಪರಮಾತ್ಮ ವಸ್ತುವನ್ನು ನೋಡುತ್ತಾನೆ. ದೇಹಾಭಿಮಾನ ಹೋಗಿ, ಪರಮಾತ್ಮನನ್ನು ತನ್ನ ಅಂತರಾಳದಲ್ಲಿ ಅರಿತಮೇಲೆ, ಹೊರಗೆ ದೃಶ್ಯ ಪ್ರಪಂಚದಲ್ಲಿ ಎತ್ತ ನೋಡಿದರೂ ಅವನಿಗೆ ಬ್ರಹ್ಮದರ್ಶನವೇ ಆಗುವುದು. ಮಡಕೆಯು ಮಣ್ಣಿನಿಂದ ಆಗಿದೆ, ಬಟ್ಟೆಯು ನೂಲಿನಿಂದ ಆಗಿದೆ, ಸಕ್ಕರೆಯ ಪಾಕದಿಂದ ಮಾಡಿದ ಆಕಾರಗಳ ಹಿಂದೆಲ್ಲಾ ಸಕ್ಕರೆಯೇ ಇದೆ. ಜ್ಞಾನಿ ಮತ್ತು ಅಜ್ಞಾನಿ ಇಬ್ಬರೂ ಒಂದೇ ವಸ್ತುವನ್ನು ನೋಡುವರು. ಜ್ಞಾನಿ ಅದನ್ನು ಬ್ರಹ್ಮ ಎನ್ನುವನು. ಅಜ್ಞಾನಿ ಅದನ್ನು ಯಾವುದೊ ತುಚ್ಛವಾದ ನಾಮವೆನ್ನುವನು. ನಾವು ನೋಡುವ ದೃಷ್ಟಿಯನ್ನು ಬದಲಾಯಿಸಿದರೆ ಇಡೀ ಪ್ರಪಂಚವೇ ಬದಲಾಯಿಸುವುದು. ಯತಿ ತಾನು ನೋಡುವ ದೃಷ್ಟಿಯಲ್ಲಿ ಒಂದು ಬದಲಾವಣೆಯನ್ನು ಮಾಡಿಕೊಂಡಿರುವನು. ಯಾವುದನ್ನು ನೋಡಿದರೂ ಅವನಿಗೆ ಬ್ರಹ್ಮದರ್ಶನವೇ ಆಗುವುದು.

\begin{verse}
ಸ್ಪರ್ಶಾನ್ ಕೃತ್ವಾ ಬಹಿರ್ಬಾಹ್ಯಾಂಶ್ಚಕ್ಷುಶ್ಚೈವಾಂತರೇ ಭ್ರುವೋಃ~।\\ಪ್ರಾಣಾಪಾನೌ ಸಮೌ ಕೃತ್ವಾ ನಾಸಾಭ್ಯಂತರಚಾರಿಣೌ \versenum{॥ ೨೭~॥}
\end{verse}

\begin{verse}
ಯತೇಂದ್ರಿಯಮನೋಬುದ್ಧಿರ್ಮುನಿರ್ಮೋಕ್ಷಪರಾಯಣಃ~।\\ವಿಗತೇಚ್ಛಾಭಯಕ್ರೋಧೋ ಯಃ ಸದಾ ಮುಕ್ತ ಏವ ಸಃ \versenum{॥ ೨೮~॥}
\end{verse}

{\small ಹೊರಗಿನ ವಿಷಯಗಳನ್ನು ಬಿಟ್ಟು, ದೃಷ್ಟಿಯನ್ನು ಹುಬ್ಬುಗಳ ನಡುವೆ ಇಟ್ಟು, ಮೂಗಿನ ಮೂಲಕ ಉಸಿರಾಡುವ ಪ್ರಾಣಾಯಾಮಗಳನ್ನು ಸಮಮಾಡಿ, ಇಂದ್ರಿಯ ಮನಸ್ಸು, ಬುದ್ಧಿಗಳನ್ನು ನಿಗ್ರಹಿಸಿ, ಮೋಕ್ಷಪರಾಯಣನಾಗಿ ಇಚ್ಛೆ ಭಯ ಕ್ರೋಧವನ್ನು ಬಿಟ್ಟ ಮುನಿ ಯಾವಾಗಲೂ ಮುಕ್ತನೆ.}

ಮುನಿ ಯಾವಾಗಲೂ ಮುಕ್ತನೆ. ಅವನೇನು ಪ್ರಾಣ ಬಿಟ್ಟಮೇಲೆ ಮುಕ್ತಿಯ ಅವಸ್ಥೆಯನ್ನು ಪಡೆಯುವುದಿಲ್ಲ. ಮುಕ್ತಿ ಎಂಬುದು ಯಾವುದೋ ಲೋಕಕ್ಕೆ ಹೋಗುವುದು ಅಲ್ಲ. ಅದು ಪ್ರಪಂಚವನ್ನು ನೋಡುವ ಒಂದು ದೃಷ್ಟಿ. ಅದು ನಾವಿಲ್ಲಿರುವಾಗ ಬಂದಿದ್ದರೆ ಬಿಟ್ಟುಹೋಗು ವಾಗಲೂ ನಮ್ಮಲ್ಲಿಯೇ ಇರುವುದು. ಪರಮಾತ್ಮನನ್ನು ನೋಡುವವನ ದೃಷ್ಟಿಯಲ್ಲಿ ಈ ವಿಶೇಷ ವನ್ನು ಗಮನಿಸುತ್ತೇವೆ. ಅವನು ಹೊರಗಿನ ವಿಷಯ ವಸ್ತುಗಳ ಮೇಲೆ ವ್ಯಾಮೋಹವನ್ನು ಬಿಟ್ಟರು ವನು. ಅವನು ವಿಷಯ ವಸ್ತುಗಳನ್ನು ತೆಗೆದುಕೊಳ್ಳುವುದೇ ಇಲ್ಲ ಎಂದಲ್ಲ. ದೇಹಧಾರಣೆಗೆ ಆವಶ್ಯಕವಾದುದನ್ನು ಸ್ವೀಕರಿಸುವನು. ವಿಷಯ ವಸ್ತುಗಳನ್ನು ಅನುಭವಿಸಬೇಕೆಂದಲ್ಲ. ಅವನು ನಿರ್ಲಿಪ್ತನಾಗಿ ಅವುಗಳೊಂದಿಗೆ ವ್ಯವಹಾರ ನಡೆಸುತ್ತಿರುವನು. ಅವನು ಎಂದೂ ಅದಕ್ಕೆ ಅಂಟಿ ಕೊಂಡಿರುವುದಿಲ್ಲ, ಆಸಕ್ತನಲ್ಲ, ಬದ್ಧನಲ್ಲ.

ಅವನ ದೃಷ್ಟಿಯನ್ನು ಹುಬ್ಬುಗಳ ನಡುವೆ ಇಟ್ಟಿರುವನು. ಎಂದರೆ ಮನಸ್ಸನ್ನು ಯಾವಾಗಲೂ ಅಂತರ್ಮುಖ ಮಾಡಿ ಪರಮಾತ್ಮನ ಮೇಲೆ ಏಕಾಗ್ರಮಾಡಿರುವನು. ಮನಸ್ಸು ಹೀಗಾದಾಗ ಅವನ ದೃಷ್ಟಿಯಲ್ಲಿ ಒಂದು ವ್ಯತ್ಯಾಸ ತೋರುವುದು. ಅವನು ವಿಷಯ ಪ್ರಪಂಚದಲ್ಲಿದ್ದರೂ ಅವನ ದೃಷ್ಟಿ ಅದರ ಮೇಲಿಲ್ಲ, ಅದರ ಹಿಂದೆ ಇರುವ ಪರಮಾತ್ಮ ವಸ್ತುವಿನ ಕಡೆ.

ಅವನು ಉಸಿರು ಬಿಡುವುದು ತೆಗೆದುಕೊಳ್ಳುವುದರಲ್ಲಿ ಒಂದು ಸಮತ್ವವನ್ನು ನೋಡುವೆವು. ಯಾವಾಗ ಒಬ್ಬ ಕೋಪಗೊಳ್ಳುವನೋ, ಇನ್ನು ಬೇರೆ ಉದ್ವೇಗಕ್ಕಾದರೂ ಸಿಕ್ಕಿಕೊಳ್ಳುವನೊ, ಆಗ ದೊಡ್ಡ ತಿದಿಯಂತೆ ಉಸಿರಾಡುವುದನ್ನು ನೋಡುತ್ತೇವೆ. ಸಮಾಧಾನವಾಗಿರುವ ಪ್ರಶಾಂತಚಿತ್ತನಾದ ಯೋಗಿಯ ಉಸಿರಾಟ ನಿಧಾನವಾಗಿರುವುದು. ಒಬ್ಬ ಬಿಡುವ ಉಸಿರಿನಿಂದಲೆ ಒಬ್ಬನ ಮನಸ್ಸನ್ನು ಊಹಿಸಬಹುದು. ಅವನು ಯಾವಾಗಲೂ ಮೋಕ್ಷಪರಾಯಣ. ಇಂದ್ರಿಯ ಸುಖವನ್ನು ಗಳಿಸುವುದ ಕ್ಕಾಗಲಿ, ಗಳಿಸಿದುದನ್ನು ರಕ್ಷಿಸುವುದಕ್ಕಾಗಲಿ ಗಡಿಬಿಡಿಯಾಗಿರುವ ವ್ಯಕ್ತಿಯಲ್ಲ. ಅವನ ಆದರ್ಶ ಈ ಪ್ರಪಂಚವಲ್ಲ. ಅವನ ಪಾಲಿಗೆ ಈ ಪ್ರಪಂಚ ಒಂದು ಕಸದ ಬುಟ್ಟಿ. ಇದರಲ್ಲಿ ಪಡೆಯ ಬೇಕಾಗಿರುವುದು, ಅನುಭವಿಸಬೇಕಾಗಿರುವುದು ಯಾವುದೂ ಇಲ್ಲ. ಅವನ ಮನಸ್ಸು ಯಾವಾಗಲೂ ಭಗವಂತನ ಕಡೆ ಹರಿಯುತ್ತಿರುವುದು. ಏಕೆಂದರೆ ಅವನು ಮುಕ್ತಿಕಾಮಿ. ನದಿ ಹೇಗೆ ಸಾಗರಮುಖ ವಾಗಿ ಹರಿಯುವುದೊ, ಉತ್ತರಮುಖಿ ಯಾವಾಗಲೂ ಉತ್ತರದಿಕ್ಕನ್ನೇ ಸೂಚಿಸುತ್ತಿರುವುದೊ, ಹಾಗೆ ಮುನಿಯ ಮನಸ್ಸು ಯಾವಾಗಲೂ ಪರಮಾತ್ಮನ ಕಡೆ ಹರಿಯುತ್ತಿರುವುದು. ಅವನು ಈ ಪ್ರಪಂಚ ವನ್ನು ನೋಡುತ್ತಿದ್ದರೂ ನೋಡುತ್ತಿಲ್ಲ, ಕೇಳುತ್ತಿದ್ದರೂ ಕೇಳುತ್ತಿಲ್ಲ, ಕೆಲಸ ಮಾಡುತ್ತಿದ್ದರೂ ಮಾಡುತ್ತಿಲ್ಲ. ಅವನು ಮನಸ್ಸಿಲ್ಲದ ಮನಸ್ಸಿನಲ್ಲಿರುವನು ಈ ಪ್ರಪಂಚದಲ್ಲಿ.

ಅವನು ಇಂದ್ರಿಯ ಮನಸ್ಸು ಬುದ್ಧಿಗಳನ್ನು ನಿಗ್ರಹಿಸಿರುವನು.ಅವನು ಇಂದ್ರಿಯಕ್ಕೆ ದಾಸನಲ್ಲ. ಇಂದ್ರಿಯದ ಮೂಲಕ ವಿಷಯ ವಸ್ತವಿನ ಕಡೆ ಸರಿದು ಅಲ್ಲಿ ಸಿಕ್ಕಿಕೊಳ್ಳುವುದಿಲ್ಲ. ಮನಸ್ಸು ತನಗೆ ತೋಚಿದ ವೈಷಯಿಕ ವಸ್ತುವನ್ನೆಲ್ಲ ಕುರಿತು ಚಿಂತಿಸುವುದಕ್ಕಾಗುವುದಿಲ್ಲ. ಅದು ಪರಮಾತ್ಮನಿಗೆ ಮೀಸಲಾಗಿದೆ, ಅದನ್ನು ಪ್ರಪಂಚಕ್ಕೆ ಕೊಟ್ಟು ಎಂಜಲು ಮಾಡಿಲ್ಲ. ಅದರಂತೆಯೇ ಅವನ ಬುದ್ಧಿಯನ್ನು ಪರಮಾತ್ಮನನ್ನು ತಿಳಿದುಕೊಳ್ಳುವುದಕ್ಕೆ ಉಪಯೋಗಿಸುತ್ತಿರುವನು. ಅದರಿಂದ ಬಾಹ್ಯ ಪ್ರಪಂಚದಲ್ಲಿ ಏನನ್ನಾದರು ಪಡೆಯುವುದಕ್ಕೆ ಯತ್ನಿಸುತ್ತಿಲ್ಲ. ಮುನಿ ತುಂಬಾ ಬುದ್ಧಿವಂತ. ಆದರೆ ಅವನು ಆ ಬುದ್ಧಿಯನ್ನು ಪ್ರಪಂಚದ ಸಂತೆಯಲ್ಲಿ ಸಿಕ್ಕುವ ಕ್ಷುದ್ರ ವಸ್ತುವನ್ನು ಕೊಳ್ಳಲು ಖರ್ಚು ಮಾಡುವುದಿಲ್ಲ. ಅದನ್ನು ಸಂತೆಯಲ್ಲಿ ಸಿಕ್ಕದ ಪರಮಾತ್ಮ ವಸ್ತುವನ್ನು ಅರಿಯಲು ಮೀಸಲಾಗಿಟ್ಟಿರು ವನು. ಗೆಲ್ಲುವುದು ಎಂದರೆ ಇದೇ, ಬುದ್ಧಿಯನ್ನು ಸೇವಕನನ್ನಾಗಿ ಮಾಡಿಕೊಳ್ಳುವುದು.

ಅವನು ಇಚ್ಛೆ, ಭಯ, ಕ್ರೋಧವನ್ನು ಬಿಟ್ಟಿರುವನು. ಯಾವ ಪ್ರಾಪಂಚಿಕ ಆಸೆ ಆಕಾಂಕ್ಷೆಗಳೂ ಅವನಲ್ಲಿ ಇಲ್ಲ. ಇವೆಲ್ಲ ಮರೀಚಿಕೆ ಎಂಬುದು ಅವನಿಗೆ ಚೆನ್ನಾಗಿ ವೇದ್ಯವಾಗಿದೆ. ಅವನು ಯಾವುದಕ್ಕೂ ಅಂಜುವುದಿಲ್ಲ. ಈ ಪ್ರಪಂಚದಲ್ಲಿ ಏನಾದರೂ ಅಂಜುವುದಿಲ್ಲ. ಯಾರೂ ಅವನನ್ನು ಅಂಜಿಸಲಾರರು. ಅವನು ಅಂಜಿಕೆಗೇ ಮೃತ್ಯುವಾದ ಭಗವಂತನ ಕೈಯನ್ನು ಹಿಡಿದುಕೊಂಡಿರುವನು. ಈ ಪ್ರಪಂಚದಲ್ಲಿ ಭಗವತ್ಸಾಕ್ಷಾತ್ಕಾರ ಮಾಡಿಕೊಂಡವನೆ ನಿರ್ಭೀತ. ಯಾವ ಸಾರ್ವಭೌಮನೂ ಅಲ್ಲ. ಯಾವ ಕೋಟ್ಯಾಧೀಶ್ವರನೂ ಅಲ್ಲ. ಎಲ್ಲರೂ ಒಂದಲ್ಲದೇ ಇದ್ದರೆ ಮತ್ತೊಂದಕ್ಕೆ ಅಂಜಬಹುದು. ಸಾರ್ವಭೌಮನಿಗೆ ಬೇಕಾದಷ್ಟು ಸೇನೆ ಇದೆ. ಯಾರೂ ಅವನನ್ನು ಸೋಲಿಸುವಂತೆ ಇಲ್ಲ. ಆದರೆ ಅವನು ಮೃತ್ಯುವನ್ನು ವಂಚಿಸಬಲ್ಲನೆ? ಅಪಕೀರ್ತಿಯನ್ನು ತಳ್ಳಬಲ್ಲನೆ? ರೋಗರುಜಿನಗಳನ್ನು, ಶೋಕವನ್ನು ಓಡಿಸಬಲ್ಲನೆ? ಮುನಿ ಇದನ್ನೆಲ್ಲ ಮಾಡುವನು. ಆದರೆ ಬಾಹ್ಯಬಲ ಯಾವುದೂ ಅವನಲ್ಲಿಲ್ಲ. ಅವನಷ್ಟು ದುರ್ಬಲ ಯಾರೂ ಇಲ್ಲ. ಆದರೆ ಈ ಪ್ರಪಂಚದಲ್ಲಿ ಯಾವುದೂ ಅವನನ್ನು ಅಂಜಿಸಲಾರದು. ಕ್ರೋಧವನ್ನು ಬಿಟ್ಟವನವನು. ವೈಷಯಿಕ ಪ್ರಪಂಚದಲ್ಲಿ ಆಸಕ್ತಿ ಇದ್ದರೆ ಅದನ್ನು ಸಂಗ್ರಹಿಸುವಾಗ ಅದಕ್ಕೆ ಆತಂಕವಾದವುಗಳ ಮೇಲೆ ನಮಗೆ ಕೋಪ ಬರುವುದು. ಆದರೆ ಮುನಿಗೆ ಇವೆಲ್ಲ ಕೆಲಸಕ್ಕೆ ಬಾರದ ಕಪ್ಪೆಚಿಪ್ಪುಗಳು. ಇದನ್ನು ಆರಿಸುವವರು ಏನೂ ಅರಿಯದ ಮಕ್ಕಳು. ಇದನ್ನು ಮಕ್ಕಳಿಗೆ ಬಿಡುವನು. ಮುನಿಯಾದರೂ ತನ್ನೊಳಗೇ ಇರುವ ಮಾಣಿಕ್ಯವನ್ನು ಕಂಡು ಹಿಡಿದಿರುವನು. ಅವನಿಗೆ ಇನ್ನು ಮೇಲೆ ಯಾರಮೇಲೂ ಕೋಪವಿಲ್ಲ.

ಇಂತಹ ಮುನಿ ಬದುಕಿರುವಾಗಲೇ, ಅವನು ಯಾವ ಕಾರ್ಯಕ್ಷೇತ್ರದಲ್ಲಿರಲಿ, ಯಾವ ಕೆಲಸವನ್ನು ಮಾಡುತ್ತಿರಲಿ, ಅವನ ಮನಸ್ಸು ಮಾತ್ರ ಮುಕ್ತಾವಸ್ಥೆಯಲ್ಲಿಯೇ ಇರುವುದು.

\begin{verse}
ಭೋಕ್ತಾರಂ ಯಜ್ಞ ತಪಸಾಂ ಸರ್ವಲೋಕಮಹೇಶ್ವರಮ್​।\\ಸುಹೃದಂ ಸರ್ವಭೂತಾನಾಂ ಜ್ಞಾತ್ವಾ ಮಾಂ ಶಾಂತಿಮೃಚ್ಛತಿ \versenum{॥ ೨೯~॥}
\end{verse}

{\small ಯಜ್ಞ ತಪಸ್ಸುಗಳ ಭೋಕ್ತೃವೆಂದೂ, ಸರ್ವಲೋಕಗಳ ಮಹೇಶ್ವರನೆಂದೂ, ಸರ್ವಪ್ರಾಣಿಗಳ ಸ್ನೇಹಿತನೆಂದೂ ನನ್ನನ್ನು ಅರಿತುಕೊಂಡ ಯೋಗಿ ಶಾಂತಿಯನ್ನು ಪಡೆಯುತ್ತಾನೆ.}

ನಾವು ಮಾಡುವ ಯಜ್ಞ, ತಪಸ್ಸು, ಅದು ಯಾವ ವಿಧವಾದರೂ ಆಗಿರಲಿ, ಅದೆಲ್ಲ ಕೊನೆಗೆ ಭಗವಂತನಿಗೆ ಅರ್ಪಿತ. ಅವನು ಅದನ್ನು ಸ್ವೀಕರಿಸಿ, ನಮಗೆ ಯೋಗ್ಯವಾದ ಫಲ ಕೊಡುವನು. ಅವನು ಯಾವುದನ್ನೂ ಗಣನೆಗೆ ತೆಗೆದುಕೊಳ್ಳದೆ ಇರುವುದಿಲ್ಲ. ನಾವು ಭಗವಂತನಿಗೆ ಕೊಡುವುದಾವುದೂ ನಿರರ್ಥಕವಾಗುವುದಿಲ್ಲ. 

ಅವನು ಸರ್ವಲೋಕ ಮಹೇಶ್ವರ. ನಾವು ಯಾವ ಭೂಮಿ ಎಂಬ ಗ್ರಹದ ಮೇಲೆ ವಾಸಿಸು ತ್ತಿರುವೆವೊ ಅದು ಬ್ರಹ್ಮಾಂಡದಲ್ಲಿ ಒಂದು ಸಣ್ಣ ಮರಳು ಕಣದಂತೆ. ಈ ಪ್ರಪಂಚದಲ್ಲಿ ಎಲ್ಲಾ ಕಡೆಯೂ ಅವನ ನಿಯಮ ಆಳುತ್ತಿದೆ. ಸಣ್ಣದು-ದೊಡ್ಡದು, ಜಡ-ಚೇತನ ಎಲ್ಲಕ್ಕೂ ಒಡೆಯನವನು. ಅವನ ಇಚ್ಛೆ ಇಲ್ಲದೆ ಒಂದು ಸಣ್ಣ ಮರಳು ಕಣ ಚಲಿಸುವಂತೆ ಇಲ್ಲ. ಭೂಮಿ ಸುತ್ತುವುದು ಅವನಿಂದ, ಸೂರ್ಯ ಬೆಳಗುವುದು ಅವನಿಂದ, ಮೃತ್ಯು ಕೂಡ ಬರುವುದು ಅವನಿಂದಲೇ.

ಅವನು ಜೀವರಾಶಿಗಳಿಗೆಲ್ಲ ಒಳ್ಳೆಯ ಸ್ನೇಹಿತ. ಯಾವ ಪ್ರತಿಫಲಾಪೇಕ್ಷೆಯೂ ಇಲ್ಲದೆ ಜೀವಿಗಳಿಗೆ ಉಪಕಾರ ಮಾಡುವವನು. ಅವನು ಜೀವ ವಿಕಾಸವಾಗುವುದಕ್ಕೆ ಅವಕಾಶವನ್ನು ಕಲ್ಪಿಸುವನು. ಇರುವ ಆತಂಕಗಳನ್ನು ನಿವಾರಿಸುವನು, ಸಂದೇಹಗಳನ್ನು ಪರಿಹರಿಸುವನು. ಎಂತಹ ದುಃಖ ಸಂಕಟ ಗಳಲ್ಲಿಯೂ ನಮ್ಮನ್ನು ಕೈಬಿಡದೆ ನಡೆಸುವನು. ಈ ಪ್ರಪಂಚದಲ್ಲಿ ಎಂದೆಂದಿಗೂ ನಮ್ಮ ನೆರವಿ ಗಿರುವ ಶಕ್ತಿ ಅವನೊಬ್ಬನೇ. ಇದನ್ನು ಯಾರು ಅರಿತಿರುವನೊ ಅವನು ಪರಮಶಾಂತಿಯನ್ನು ಪಡೆಯುವನು. ನನ್ನ ಗತಿ ಏನು, ನಡುನೀರಿನಲ್ಲಿ ನನ್ನ ಕೈ ಬಿಡುವನೆ ಎಂದು ಇನ್ನುಮೇಲೆ ತಲ್ಲಣಿಸುವುದಿಲ್ಲ. ಮಗುವೊಂದು ಅಪ್ಪನ ಕೈಯನ್ನು ಹಿಡಿದುಕೊಂಡು ಹೋದರೆ ಹೇಗೆ ಸುರಕ್ಷಿತ ವಾಗಿ ಗುರಿಯನ್ನು ಸೇರುವುದೋ ಹಾಗೆ ನಂಬಿದ ಭಕ್ತ ಪರಂಧಾಮವನ್ನು ಸೇರುವನು.

