
\chapter{ಅರ್ಜುನ ವಿಷಾದಯೋಗ}

ಭಗವದ್ಗೀತೆಯಲ್ಲಿ ಪ್ರತಿಯೊಂದು ಅಧ್ಯಾಯದ ಕೊನೆಯಲ್ಲಿಯೂ ಅದನ್ನು ಒಂದು ಯೋಗ ವೆಂದು ಹೇಳುವರು. ಬೇರೆ ಬೇರೆ ಅಧ್ಯಾಯಗಳಿಗೆ ಸಾಂಖ್ಯಯೋಗ, ಕರ್ಮಯೋಗ, ಜ್ಞಾನಯೋಗ ಮುಂತಾದುವು ಅನ್ವಯಿಸಬಹುದು. ಈ ಮೊದಲನೇ ಅಧ್ಯಾಯಕ್ಕೆ ಏತಕ್ಕೆ ವಿಷಾದಯೋಗವೆಂದು ಹೆಸರನ್ನು ಕೊಟ್ಟಿರುವರು ಎಂದು ನಮಗೆ ಆಶ್ಚರ್ಯವಾಗಬಹುದು. ಯೋಗ ಎಂದರೆ ಒಂದುಗೂಡಿ ಸುವುದು. ಜೀವಾತ್ಮನನ್ನು ಪರಮಾತ್ಮನೊಡನೆ ಒಂದುಗೂಡಿಸುವುದಕ್ಕೆ ಯೋಗವೆಂದು ಹೇಳುತ್ತೇವೆ. ನಾವು ಈಗ ವಿಯೋಗದಲ್ಲಿರುವೆವು. ದೇವರನ್ನು ಮರೆಯುತ್ತಿರುವೆವು. ಸುತ್ತಮುತ್ತಲಿರುವ ಪ್ರಾಪಂಚಿಕ ವಸ್ತುಗಳನ್ನೇ ಸತ್ಯವೆಂದು ಭ್ರಮಿಸಿ ಅದರಲ್ಲಿ ತಲ್ಲೀನರಾಗಿರುವೆವು. ಯೋಗ ಎಂಬುದು ಬಂದು ನಮ್ಮನ್ನು ದೇವರೊಡನೆ ಸೇರಿಸುವುದು. ವಿಷಾದ ಹೇಗೆ ನಮ್ಮನ್ನು ಸೇರಿಸುವುದು? ವಿಷಾದವೇ ಸೇರಿಸುವುದಿಲ್ಲ. ವಿಷಾದ ಜೀವಿಯ ಮನಸ್ಸನ್ನು ದೇವರೆಡೆಗೆ ಹೋಗಲು ಅಣಿಮಾಡುವುದು.

ವಿಷಾದಕ್ಕೆ ಜೀವನದಲ್ಲಿ ಒಂದು ದೊಡ್ಡ ಸ್ಥಾನವಿದೆ. ಸಾಧಾರಣವಾಗಿ ಮನುಷ್ಯ ಕೆಳಗಿನ ಭೂಮಿಕೆಯಲ್ಲಿ ಕೆಲಸ ಮಾಡುತ್ತಿರುವನು. ಮೇಲಿನ ಭೂಮಿಕೆಗೆ ಹೋಗಬೇಕಾದಾಗ ಯಾತನೆಯ ಬೆಲೆಯನ್ನು ಕೊಟ್ಟಲ್ಲದೆ ಸಾಧ್ಯವಿಲ್ಲ. ಆ ಯಾತನೆ ನಮ್ಮ ಇಡೀ ಜೀವನಕ್ಕೆ ಸಿಡಿಲಿನ ಪೆಟ್ಟಿನಂತೆ ಬೀಳುವುದು. ನಮ್ಮನ್ನು ಅಲ್ಲೋಲ ಕಲ್ಲೋಲ ಮಾಡುವುದು ಮೊದಲು. ಅನಂತರವೇ ಅದು ನಮ್ಮ ಜೀವನದಲ್ಲಿ ಸರಿಹೊಂದಿಕೊಳ್ಳುವುದು. ಅರ್ಜುನ ಯುದ್ಧವನ್ನು ತನ್ನ ಕೇವಲ ವ್ಯಷ್ಟಿದೃಷ್ಟಿಯಿಂದ ನೋಡುತ್ತಿದ್ದ. ಅವನು ಸಮಷ್ಟಿಯ ದೃಷ್ಟಿಯ ಕಡೆ ಹೋಗಬೇಕಾಗಿದೆ. ಜೀವನದಲ್ಲಿ ಹೊಸದಕ್ಕೆ ಹೊಂದಿಕೊಳ್ಳುವಾಗ ಯಾವಾಗಲೂ ಯಾತನೆ ಆಗುವುದು. ಇದುವರೆಗೆ ಒಂದು ಕಡೆ ಬೆಳೆದ ಒಂದು ಸಸಿಯನ್ನು ಕಿತ್ತು ಬೇರೆ ಕಡೆ ಹಾಕಿದರೆ ಅದು ಒಂದೆರಡು ದಿನ ಬಾಡಿಹೋಗುವುದು. ಹಳೆಯ ಎಲೆಗಳೆಲ್ಲಾ ಉದುರಿ ಹೋಗುವುದು. ಅನಂತರ ಹೊಸ ನೆಲದಲ್ಲಿ ಬೇರು ಬಿಟ್ಟಾದ ಮೇಲೆ ಪುನಃ ಚೇತರಿಸಿಕೊಳ್ಳುವುದು. ಹೊಸ ಎಲೆಗಳು ಚಿಗುರುವುವು. ಕುರುಕ್ಷೇತ್ರದ ಯುದ್ಧವನ್ನು ಕೇವಲ ತನ್ನ ಲಾಭದ ದೃಷ್ಟಿಯಿಂದ ಮಾತ್ರ ನೋಡುತ್ತಿದ್ದ ಅರ್ಜುನನಿಗೆ ತೀವ್ರ ಯಾತನೆ ಆಗುವುದು. ತನಗೆ ಜಯ ಸಿಕ್ಕಬೇಕಾದರೆ ವಿರೋಧಪಕ್ಷದಲ್ಲಿರುವ ಗುರುಹಿರಿಯರನ್ನು ಬಂಧುಬಾಂಧವರನ್ನು ನಿರ್ನಾಮ ಮಾಡಿದಲ್ಲದೆ ಸಾಧ್ಯವಾಗುವುದಿಲ್ಲ. ಆಗ ವ್ಯಾಮೋಹ ವಿಷಾದದ ಹೆಸರಿನಲ್ಲಿ ಅರ್ಜುನನ ಮನಸ್ಸನ್ನು ಧಾಳಿ ಇಟ್ಟಿತು. ಗಾಂಢೀವ ಕೈಯಿಂದ ಜಾರಿತು. ಕಣ್ಣು ಹನಿಗೂಡಿತು, ಅಂಗಾಂಗಗಳೆಲ್ಲಾ ಶೋಕ ದಿಂದ ಕಾಯತೊಡಗಿತು. ಆ ಸಮಯದಲ್ಲಿ ಏನು ಮಾಡಬೇಕೆಂದು ಕಾಣದಾದ. ಶ್ರೀಕೃಷ್ಣ ಅರ್ಜುನನಿಗೆ ತಾನು ಮಾಡುವ ಕೆಲಸವನ್ನು ಬೇರೆ ದೃಷ್ಟಿಯಿಂದ ಹೇಗೆ ಮಾಡಬೇಕೆಂಬುದನ್ನು ತೋರುವನು. ಭಗವತ್ ಗೀತೆಯ ಮಳೆ ಪ್ರಾರಂಭವಾಗಬೇಕಾದರೆ ವಿಷಾದ ಮೋಡದಂತೆ ಇದೆ. ವಿಷಾದದ ಮೋಡ ಅರ್ಜುನನನ್ನು ಕವಿಯದೇ ಇದ್ದಿದ್ದರೆ ಭಗವಂತನ ವಾಣಿಯ ಹನಿ ಬೀಳು ತ್ತಿರಲಿಲ್ಲ. ಆದಕಾರಣ ವಿಷಾದ ಒಂದು ಹಿನ್ನೆಲೆಯನ್ನು ಕಲ್ಪಿಸುವುದು.

ಧೃತರಾಷ್ಟ್ರ ಸಂಜಯನನ್ನು ಕುರಿತು ಕೇಳುತ್ತಾನೆ,

\begin{verse}
ಧರ್ಮಕ್ಷೇತ್ರೇ ಕುರುಕ್ಷೇತ್ರೇ ಸಮವೇತಾ ಯುಯುತ್ಸವಃ ।\\ಮಾಮಕಾಃ ಪಾಂಡವಾಶ್ಚೈವ ಕಿಮಕುರ್ವತ ಸಂಜಯ\num{॥ ೧ ॥}
\end{verse}

{\small ಸಂಜಯ, ಧರ್ಮಕ್ಷೇತ್ರವಾದ ಕುರುಕ್ಷೇತ್ರದಲ್ಲಿ ಯುದ್ಧಕ್ಕಾಗಿ ನೆರೆದ ನಮ್ಮವರು ಮತ್ತು ಪಾಂಡವರು ಏನನ್ನು ಮಾಡಿದರು?}

ಕುರುಕ್ಷೇತ್ರದ ಯುದ್ಧಕ್ಕೆ ಮುಂಚೆ ವ್ಯಾಸರು ಧೃತರಾಷ್ಟ್ರನ ಸಮೀಪಕ್ಕೆ ಬಂದು ಜನ್ಮಾಂಧನಾದ ಅವನಿಗೆ ಯುದ್ಧವನ್ನು ನೋಡುವುದಕ್ಕೆ ಬೇಕಾದರೆ ಕಣ್ಣುಗಳನ್ನು ನಾನು ಕೊಡುತ್ತೇನೆ ಎಂದು ಹೇಳಿದರು. ಆದರೆ ಧೃತರಾಷ್ಟ್ರನಿಗೆ ಕೊಲೆಯನ್ನು ನೋಡುವುದಕ್ಕಾಗಿ ಕಣ್ಣು ಬೇಕಾಗಿರಲಿಲ್ಲ. ಮಕ್ಕಳ ಬಾಲಲೀಲೆಯನ್ನು ನೋಡಲಿಲ್ಲ ಕಣ್ಣು. ಅವರ ಶಕ್ತಿ ಪರಾಕ್ರಮ ನೋಡಲಿಲ್ಲ. ಈಗ ಸಮರಾಂಗಣ ದಲ್ಲಿ ಒಬ್ಬರು ಮತ್ತೊಬ್ಬರನ್ನು ಕೊಲ್ಲುವುದನ್ನು ನೋಡುವುದಕ್ಕೆ ಕಣ್ಣನ್ನು ಆಶಿಸಲಿಲ್ಲ. ಆಗ ವ್ಯಾಸರು ಧೃತರಾಷ್ಟ್ರನ ಸ್ನೇಹಿತನಾದ ಸಂಜಯನಿಗೆ ದಿವ್ಯ ಚಕ್ಷುಸ್ಸನ್ನು ಅನುಗ್ರಹಿಸುವರು. ಆತ ಯುದ್ಧ ಭೂಮಿಗೆ ದೂರದಲ್ಲಿರುವ ಅರಮನೆಯಲ್ಲಿ ಧೃತರಾಷ್ಟ್ರನ ಸಮೀಪದಲ್ಲಿ ಕುಳಿತಿದ್ದರೂ, ಯುದ್ಧ ಭೂಮಿಯಲ್ಲಿ ಎಲ್ಲೆಲ್ಲಿ ಏನು ಆಗುತ್ತಿದೆಯೊ ಅದನ್ನು ನೋಡಬಲ್ಲ ಶಕ್ತಿಯನ್ನು ಅವನಿಗೆ ಅನುಗ್ರಹಿಸುವರು. ಕೇವಲ ಹೊರಗೆ ಆಗುವುದನ್ನು ನೋಡುವುದು ಮಾತ್ರವಲ್ಲ ಅವರವರ ಮನಸ್ಸಿ ನಲ್ಲಿ ಏನೇನು ಆಗುತ್ತಿದೆ ಎಂಬುದನ್ನು ತಿಳಿದುಕೊಳ್ಳಬಲ್ಲ ಶಕ್ತಿಯನ್ನು ಕೊಡುವರು. ದೂರದಲ್ಲಿ ಆಗುತ್ತಿರುವ ಕ್ರಿಕೆಟ್ ಮ್ಯಾಚನ್ನು ನಮ್ಮ ರೂಮಿನಲ್ಲಿ ಕುಳಿತು ಟಿ.ವಿ. ಮೂಲಕ ನೋಡುವಂತೆ ಇದು. ಆದಕಾರಣವೆ ಧೃತರಾಷ್ಟ್ರ ಸಂಜಯನನ್ನು ಕೇಳುತ್ತಾನೆ.

ಯುದ್ಧ ಮಾಡುವುದಕ್ಕೆ ಕೌರವರು ಪಾಂಡವರು ಧರ್ಮಕ್ಷೇತ್ರವಾದ ಕುರುಕ್ಷೇತ್ರದಲ್ಲಿ ನೆರೆದರು. ಕುರುಕ್ಷೇತ್ರ ಹಿಂದಿನಿಂದಲೂ ತಪಃಕ್ಷೇತ್ರ, ಬ್ರಹ್ಮಪುಷಿ ದೇಶ ಎಂದು ಪ್ರಖ್ಯಾತವಾಗಿತ್ತು. ಮುನಿಗಳು ಪುಷಿಗಳು ಹಿಂದೆ ತಪಸ್ಸನ್ನು ಮಾಡಲು ಆ ಸ್ಥಳವನ್ನು ಆರಿಸಿಕೊಳ್ಳುತ್ತಿದ್ದರು. ಈಗ ಕೊಲೆಗೆ ಆರಿಸಿಕೊಂಡಿರುವರು. ಹಿಂದಿನಿಂದಲೂ ಅದು ತಪಃಕ್ಷೇತ್ರವಾದುದರಿಂದ ತನ್ನ ಸಾತ್ವಿಕ ವಾತಾವರಣ ವನ್ನು ಅಲ್ಲಿ ನೆರೆದವರ ಮೇಲೆ ಬೀರಿ ಅವರ ಮನಸ್ಸನ್ನು ಬದಲಾಯಿಸಬಹುದೇನೊ ಎಂದು ಭಾವಿಸಿರಬೇಕು. ಅನೇಕವೇಳೆ ವಾತಾವರಣ ತನ್ನ ಪ್ರಭಾವವನ್ನು ವ್ಯಕ್ತಿಯ ಮೇಲೆ ಬೀರುವುದು. ಆದರೆ ಹಾಗೆ ಬೀರಬೇಕಾದರೆ ನಮ್ಮ ಮನಸ್ಸು ಅದಕ್ಕೆ ಅಣಿಯಾಗಿರಬೇಕು. ಇಲ್ಲದೇ ಇದ್ದರೆ ಬರೀ ವಾತಾವರಣ ನಮ್ಮನ್ನು ಏನೂ ಮಾಡಲಾರದು. ದುರ್ಯೋಧನ ಮೊದಲಿನಿಂದಲೂ ಒಳ್ಳೆಯ ಪ್ರಭಾವಕ್ಕೆ ಜಗ್ಗುವವನಲ್ಲ. ಅವನ ಮನಸ್ಸಿನ ರೀತಿಯೇ ಬೇರೆ. ದುರ್ಯೋಧನನಿಗೆ ಒಳ್ಳೆಯ ಬುದ್ಧಿ ವಾದವನ್ನು ಹೇಳುವವರಿಗೆ ಬರಗಾಲವಿರಲಿಲ್ಲ. ಇವನ ಅಪ್ಪ, ಅಮ್ಮ, ಭೀಷ್ಮ, ದ್ರೋಣ, ವಿದುರ ಮುಂತಾದವರೆಲ್ಲ ಬೇಕಾದಷ್ಟು ಬುದ್ಧಿವಾದವನ್ನು ಹೇಳಿದರು. ಅದೆಲ್ಲ ಬಂಡೆಯ ಮೇಲೆ ನೀರು ಹರಿದಂತೆ ಆಯಿತು. ಕೊನೆಗೆ ಸಾಕ್ಷಾತ್ ಶ್ರೀಕೃಷ್ಣನೇ ಸಂಧಿಗೆ ಒಪ್ಪಿಸಲು ಬಂದನು. ತನ್ನ ವಿಶ್ವರೂಪವನ್ನು ತೋರಿದನು. ಅದರಿಂದ ಅವನೇನು ಬೆರಗಾಗಿಹೋದನೆ? ಅವನೊಬ್ಬ ಮಾಯಾವಿ, ತಪ್ಪಿಸಿಕೊಳ್ಳುವುದಕ್ಕೆ ಹೂಡಿರುವ ಆಟ ಇದು, ಅವನನ್ನು ಹಿಡಿದು ಕಟ್ಟಿ ಎಂದನು. ಮೇರು ಸದೃಶ ವ್ಯಕ್ತಿಯಂತಹ ಶ್ರೀಕೃಷ್ಣನ ಸಾನ್ನಿಧ್ಯದಲ್ಲಿದ್ದರೂ ಅದಕ್ಕೆ ಜಗ್ಗದವನು, ಯಾವುದೋ ಕಾಲದಲ್ಲಿ ಯಾರೋ ಮಾಡಿದ ತಪಸ್ಸಿನ ಪ್ರಭಾವಕ್ಕೆ ಸಿಕ್ಕಿ ಬದಲಾವಣೆ ಆಗಿಬಿಡುವನೆ? ಆದರೂ ಮನುಷ್ಯನಿಗೆ ಒಂದು ಚಪಲ, ಹಂಬಲ. ದಾರುಣ ಯಾತನೆಯಿಂದ ನರಳುತ್ತಿರುವವನಿಗೆ ಮಹಾವೈದ್ಯರೆಲ್ಲಾ ಕೈಬಿಟ್ಟಿರುವಾಗ ಯಾರೋ ಒಬ್ಬ ಅಳಲೆಕಾಯಿ ಪಂಡಿತ ಬಂದು ತಾನು ಗುಣಮಾಡುತ್ತೇನೆ ಎಂದಾಗ, ಒಂದು ವೇಳೆ ಇವನ ಔಷಧದಿಂದ ತನ್ನ ವ್ಯಾಧಿ ಗುಣ ಆದರೂ ಆಗಬಹುದು ಎಂಬ ಆಸೆ ಹೇಗೆ ಚಿಗುರುತ್ತದೆಯೋ ಹಾಗೆ ಈ ವಾತಾವರಣ ಸ್ವಲ್ಪ ಬುದ್ಧಿ ಕಲಿಸಬಹುದೇನೊ ತನ್ನ ಮಗನಿಗೆ ಎಂದು ಅಂಧವೃದ್ಧ ಆಶಿಸುವನು.

ಅವನು ಕೇಳುವುದಕ್ಕೆ ಮತ್ತೊಂದು ಕಾರಣವೇ, ಕೌರವರೇನೊ ಭಂಡರು. ಯಾರ ಮಾತನ್ನೂ ಕೇಳುವವರಲ್ಲ. ಆದರೆ ಪಾಂಡವರು ಮೊದಲಿನಿಂದಲೂ ಕೊಲೆಗೆ ಹಿಂಸೆಗೆ ಹಿಂದೆಗೆಯುವವರು. ಅವರು ಮನಸ್ಸು ಮಾಡಿದ್ದರೆ ಬೇರೆಲ್ಲಿಯಾದರೂ ತಮ್ಮ ಅಸೀಮ ಸಾಹಸದಿಂದ ಒಂದು ಚಕ್ರಾಧಿ kಪತ್ಯವನ್ನು ಕಟ್ಟಬಹುದಾಗಿತ್ತು. ಹಾಗೇನಾದರೂ ಮಾಡುವರೆ ಎಂಬ ಅನುಮಾನವೂ ಇದ್ದಿರಬಹುದು. ಪಾಂಡವರ ಸ್ವಭಾವವನ್ನು ಬಹಳ ಕಾಲದಿಂದಲೂ ಬಲ್ಲ ಧೃತರಾಷ್ಟ್ರ ಹೀಗೆ ಆಲೋಚನೆ ಮಾಡಿದ್ದ ರಲ್ಲಿ ಯಾವ ಒಂದು ಉತ್ಪ್ರೇಕ್ಷೆಯೂ ಇರಲಾರದು. ಪಾಂಡವರು, ತಮಗೆ ಅರ್ಧ ರಾಜ್ಯ ಬೇಕಾಗಿಲ್ಲ, ಐದು ಕುಗ್ರಾಮಗಳನ್ನು ಕೊಟ್ಟರೆ ಸಾಕು ಎನ್ನುವಷ್ಟು ಕೆಳಗೆ ಇಳಿದಿದ್ದರು.

ಸಂಜಯ ಹೇಳುತ್ತಾನೆ:

\begin{verse}
ದೃಷ್ಟ್ವಾ ತು ಪಾಂಡವಾನೀಕಂ ವ್ಯೂಢಂ ದುಯೋಧನಸ್ತದಾ ।\\ಆಚಾರ್ಯಮುಪಸಂಗಮ್ಯ ರಾಜಾ ವಚನಮಬ್ರವೀತ್ \num{॥ ೨ ॥}
\end{verse}

{\small ವ್ಯೂಹಾಕಾರವಾಗಿ ರಚಿಸಿದ್ದ ಪಾಂಡವರ ಸೇನೆಯನ್ನು ನೋಡಿ ರಾಜನಾದ ದುರ್ಯೋಧನನು ದ್ರೋಣಾಚಾರ್ಯರ ಬಳಿಗೆ ಬಂದು ಹೀಗೆ ಹೇಳುತ್ತಾನೆ.}

ದುರ್ಯೋಧನ ಯುದ್ಧಕ್ಕೆ ಮುಂಚೆ ದ್ರೋಣಾಚಾರ್ಯರ ಬಳಿಗೆ ಬರುತ್ತಾನೆ. ಏಕೆಂದರೆ ಮೊದಲಿ ನಿಂದಲೂ ದ್ರೋಣಾಚಾರ್ಯರು ತನ್ನ ಪಕ್ಷದಲ್ಲಿ ಮನಸ್ಸೆಲ್ಲ ಇಟ್ಟು ಯುದ್ಧಮಾಡುವರೆ ಎಂಬ ಸಂದೇಹವಿತ್ತು. ದ್ರೋಣಾಚಾರ್ಯರು ಕೌರವರು ಪಾಂಡವರು ಇಬ್ಬರಿಗೂ ಗುರುಗಳು. ಅರ್ಜುನ ಅವರ ಮೆಚ್ಚಿನ ಶಿಷ್ಯ. ಇದರಲ್ಲಿ ಯಾವ ಮುಚ್ಚುಮರೆಯೂ ಇರಲಿಲ್ಲ. ಅರ್ಜುನನನ್ನು ಲೋಕೈಕ ಬಿಲ್ಲಾಳನ್ನಾಗಿ ಮಾಡಲು ತಮ್ಮ ವಿದ್ಯೆಯನ್ನೆಲ್ಲ ನಿರ್ವಂಚನೆಯಿಂದ ಶಿಷ್ಯನಿಗೆ ದಾನಮಾಡಿದ್ದರು. ಅಜ್ಞಾತನಾದ ಏಕಲವ್ಯ ಅರ್ಜುನನನ್ನು ಮೀರಿ ಹೋಗಿರುವುದನ್ನು ಕೇಳಿ ಅವನಿಂದ ಬೆರಳನ್ನೇ ದಾನವಾಗಿ ಪಡೆದು ಅರ್ಜುನನಿಗೆ ಸಮಸ್ಪರ್ಧಿಯೇ ಇಲ್ಲದಂತೆ ಮಾಡಿದ್ದರು. ತನ್ನ ಸ್ವಂತ ಮಗ ಅಶ್ವತ್ಥಾಮನಿಗಿಂತ ತಮ್ಮ ಶಿಷ್ಯನಾದ ಅರ್ಜುನನನ್ನು ಪ್ರೀತಿಸುತ್ತಿದ್ದರು. ಇದನ್ನೆಲ್ಲ ಅರಿತು ದುರ್ಯೋಧನ ದ್ರೋಣಾಚಾರ್ಯನ ಸಮೀಪಕ್ಕೆ ಬಂದು ಯುದ್ಧಕ್ಕೆ ಮುಂಚೆ ಅವರು ಮನಸ್ಸಿಟ್ಟು ತನ್ನ ಪರವಾಗಿ ಯುದ್ಧ ಮಾಡುವಂತೆ ಅವರನ್ನು ಪ್ರಚೋದಿಸಬೇಕೆಂದು ಹೀಗೆ ಹೇಳುತ್ತಾನೆ.

\begin{verse}
ಪಶ್ಯೈತಾಂ ಪಾಂಡುಪುತ್ರಾಣಾಮಾಚಾರ್ಯ ಮಹತೀಂ ಚಮೂಮ್ ।\\ವ್ಯೂಢಾಂ ದ್ರುಪದಪುತ್ರೇಣ ತವ ಶಿಷ್ಯೇಣ ಧೀಮತಾ \num{॥ ೩ ॥}
\end{verse}

{\small ಎಲೈ ಆಚಾರ್ಯನೇ, ಬುದ್ಧಿವಂತನಾದ ನಿನ್ನ ಶಿಷ್ಯನಾದ ದ್ರುಪದ ಪುತ್ರನಿಂದ ವ್ಯೂಹಾಕಾರದಲ್ಲಿ ರಚಿಸಿರುವ ಪಾಂಡುಪುತ್ರರ ಈ ದೊಡ್ಡ ಸೇನೆಯನ್ನು ನೋಡು.}

ದುರ್ಯೋಧನ ಗುರುಗಳ ಬಳಿಗೆ ಬಂದು ದೃಷ್ಟದ್ಯುಮ್ನನಿಂದ ರಚಿಸಲ್ಪಟ್ಟ ಸೇನೆಯನ್ನು ನೋಡಿ ಎನ್ನುವನು. ದೃಷ್ಟದ್ಯುಮ್ನ ಎಂಬುದಕ್ಕೆ ಹಲವು ಗುಣವಾಚಕಗಳನ್ನು ಪೋಣಿಸುವನು. ದುರ್ಯೋಧನ ಪೋಣಿಸುವ ಗುಣವಾಚಕಗಳ ಹಿಂದೆಲ್ಲ ಒಂದು ದುರುದ್ದೇಶವಿದೆ. ಅದೇ ದ್ರೋಣನಿಗೂ ದ್ರುಪದ ನಿಗೂ ಹಿಂದೆ ಇದ್ದ ವೈರ. ದುರ್ಯೋಧನ ಆ ವೈರದ ಕುಡಿಯನ್ನು ತನ್ನ ಮಾತಿನಿಂದ ಮೀಟುವನು. ಹಿಂದೆ ದ್ರೋಣ ಮತ್ತು ದ್ರುಪದ ಒಂದೇ ಗುರುವಿನ ಹತ್ತಿರ ವಿದ್ಯೆಯನ್ನು ಕಲಿತವರು, ಇಬ್ಬರೂ ಗುರುಭಾಯಿಗಳು. ಆದರೆ ಕಾಲಗತಿಯಲ್ಲಿ ಮುಂದೆ ದ್ರುಪದ ಪಾಂಚಾಲ ದೇಶಕ್ಕೆ ರಾಜನಾಗುವನು. ಬ್ರಾಹ್ಮಣ ದ್ರೋಣ ಹೊಟ್ಟೆಗಿಲ್ಲದೆ ಅಲೆದಾಡುತ್ತಿದ್ದನು. ತನ್ನ ಸ್ನೇಹಿತ ದ್ರುಪದ ರಾಜನಾಗಿರುವನು, ಅವನ ಬಳಿಗೆ ಹೋದರೆ ಅವನು ತನಗೆ ಏನಾದರೂ ಸಹಾಯ ಮಾಡಬಹುದೆಂದು ಅವನ ಸಮೀಪಕ್ಕೆ ಹೋಗುವನು. ದ್ರುಪದ ದ್ರೋಣನನ್ನು ಗುರುತು ಹಿಡಿದರೂ ಈ ತಿರುಪೆಯವನನ್ನು ತನ್ನ ಸ್ನೇಹಿತ ನೆಂದು ಎಲ್ಲರೆದುರಿಗೆ ಒಪ್ಪಿಕೊಳ್ಳುವುದಕ್ಕೆ ನಾಚಿ, ಇವನಾರೊ ತನಗೆ ಗೊತ್ತಿಲ್ಲವೆಂದು ಅವಮಾನ ಮಾಡಿ ಕಳುಹಿಸುವನು. ದ್ರೋಣ ಆಗ ಶಪಥ ಮಾಡುವನು, ತನ್ನ ಶಿಷ್ಯನಿಂದ ದ್ರುಪದನ ಮಾನವನ್ನು ಭಂಗಿಸುತ್ತೇನೆ ಎಂದು. ಕಾಲಕ್ರಮೇಣ ಅರ್ಜುನನನ್ನು ಹೆಸರಾಂತ ಬಿಲ್ಲುಗಾರನನ್ನಾಗಿ ಮಾಡಿ ಅವನು ಗುರುದಕ್ಷಿಣೆ ಕೊಡುವ ಪ್ರಸಂಗ ಬಂದಾಗ ದ್ರುಪದನನ್ನು ಹಿಡಿದು ತಾನು ಮಲಗುವ ಮಂಚಕ್ಕೆ ಕಟ್ಟಿಸಿ ಎದ್ದಾಗ ಅವನನ್ನು ಅರಿಯದವನಂತೆ ಒದೆಯುವನು. ಆಗ ದ್ರುಪದ ಶಪಥಮಾಡುವನು, ದ್ರೋಣನನ್ನು ಕೊಲ್ಲುವಂತಹ ಒಬ್ಬ ಮಗನನ್ನು ನಾನು ಪಡೆಯುತ್ತೇನೆ ಎಂದು. ಆ ಮಗನೇ ದೃಷ್ಟದ್ಯುಮ್ನ, ದ್ರೌಪದಿಯ ಸಹೋದರ, ದ್ರೋಣನ ತಲೆಯನ್ನು ಕತ್ತರಿಸಲು ಅಸ್ತ್ರವನ್ನು ಮಸೆಯು ತ್ತಿರುವವನು.

ಈ ದೃಷ್ಟದ್ಯುಮ್ನ ಈಗ ಕಡುವೈರಿಯಾಗಿದ್ದರೂ ಹಿಂದೆ ದ್ರೋಣಾಚಾರ್ಯರಿಂದಲೇ ಬಿಲ್ಲಿನ ವಿದ್ಯೆಯನ್ನು ಕಲಿತದ್ದು. ಅದಕ್ಕೇ ದುರ್ಯೋಧನ, ದೃಷ್ಟದ್ಯುಮ್ನ ನಿನ್ನ ಶಿಷ್ಯನೇ ಎಂದು ಚುಚ್ಚು ಮಾತಿನಿಂದ ತಿವಿಯುವನು. ಗುರುವಿಗೆ ಆ ಶಿಷ್ಯ ತೋರುತ್ತಿರುವ ಗೌರವ ಇದು. ಗುರುವಿಗೆ ತಿರುಮಂತ್ರ ಹೇಳುವನು. ಇವನು ಬುದ್ಧಿವಂತ, ಇವನನ್ನು ಉಪೇಕ್ಷೆ ಮಾಡುವಂತೆ ಇಲ್ಲ ಎನ್ನುವನು. ದುರ್ಯೋಧನ ಆಡುತ್ತಿರುವ ಮಾತೆಲ್ಲ ದ್ರೋಣನ ಹೃದಯದಲ್ಲಿರುವ ಮಾತ್ಸರ್ಯದ ಕಿಡಿಯನ್ನು ಊದಿ ಅದಕ್ಕೆ ಮತ್ತಷ್ಟು ಸೌದೆಯನ್ನು ಹಾಕಿ ಅದನ್ನು ದಳ್ಳುರಿಯನ್ನಾಗಿ ಮಾಡುವ ಪ್ರಯತ್ನ.

\begin{verse}
ಅತ್ರ ಶೂರಾ ಮಹೇಷ್ವಾಸಾ ಭೀಮಾರ್ಜುನಸಮಾ ಯುಧಿ ।\\ಯುಯುಧಾನೋ ವಿರಾಟಶ್ಚ ದ್ರುಪದಶ್ಚ ಮಹಾರಥಃ \num{॥ ೪ ॥}
\end{verse}

\begin{verse}
ದೃಷ್ಟಕೇತುಶ್ಚೇಕಿತಾನಃ ಕಾಶೀರಾಜಶ್ಚ ವೀರ್ಯವಾನ್ ।\\ಪುರುಜಿತ್ ಕುಂತಿಭೋಜಶ್ಚ ಶೈಬ್ಯಶ್ಚ ನರಪುಂಗವಃ \num{॥ ೫ ॥}
\end{verse}

\begin{verse}
ಯುಧಾಮನ್ಯುಶ್ಚ ವಿಕ್ರಾಂತ ಉತ್ತಮೌಜಾಶ್ಚ ವೀರ್ಯವಾನ್ ।\\ಸೌಭದ್ರೋ ದ್ರೌಪದೇಯಾಶ್ಚ ಸರ್ವ ಏವ ಮಹಾರಥಾಃ \num{॥ ೬ ॥}
\end{verse}

{\small ಪಾಂಡವರ ಸೇನೆಯಲ್ಲಿ ದೊಡ್ಡ ಬಿಲ್ಗಾರರು, ಯುದ್ಧದಲ್ಲಿ ಭೀಮಾರ್ಜುನರಿಗೆ ಸರಿಸಮಾನರೂ ಆಗಿರುವ ಶೂರನಾದ ಯುಯುಧಾನ, ವಿರಾಟ, ಮಹಾರಥನಾದ ದ್ರುಪದ, ದೃಷ್ಟಕೇತು, ಚೇಕಿತಾನ, ವೀರ್ಯವಂತನಾದ ಕಾಶೀರಾಜ, ಪುರಜಿತ್ತು, ಕುಂತಿಭೋಜ, ನರಶ್ರೇಷ್ಠನಾದ ಶೈಭ್ಯ, ವಿಕ್ರಮಿಯಾದ ಯುಧಾಮನ್ಯು, ವೀರ್ಯ ವಂತನಾದ ಉತ್ತಮೌಜಸ್ಸು–ಇವರು ಮತ್ತು ಸುಭದ್ರೆಯ ಮಗನೂ ದ್ರೌಪದಿಯ ಮಕ್ಕಳೂ ಇರುವರು. ಇವರೆಲ್ಲರೂ ಮಹಾರಥರೇ.}

ದುರ್ಯೋಧನ ಇಲ್ಲಿ ಪಾಂಡವರ ಪಕ್ಷದಲ್ಲಿರುವ ಮುಖ್ಯ ವೀರರ ಹೆಸರುಗಳನ್ನು ಕೊಡುವನು. ಇವರು ಯಾರನ್ನೂ ಉಪೇಕ್ಷೆ ಮಾಡುವಂತಹವರಲ್ಲ. ಅವರೆಲ್ಲ ಪರಾಕ್ರಮದಲ್ಲಿ ಭೀಮಾರ್ಜುನರಿಗೆ ಸರಿಸಮಾನರು. ಎಂದರೆ ಇಲ್ಲಿ ಅವರೊಡನೆ ಮಾಡಬೇಕಾದ ಯುದ್ಧ ಮಕ್ಕಳಾಟವಲ್ಲ, ಉಳಿವು ಅಳಿವು ಇದರ ಮೇಲೆ ನಿಂತಿರುವುದು. ಇಲ್ಲಿ ಸ್ವಲ್ಪವಾದರೂ ಆಲಸ್ಯಕ್ಕೆ ಉದಾಸೀನತೆಗೆ ಎಡೆಯಿಲ್ಲ.

ಯದುಕುಲದ ಶಿಬಿ ಎಂಬುವನ ಮಗನಾದ ಸತ್ಯಕನ ಮಗನಾದ್ದರಿಂದ ಸಾತ್ಯಕಿ ಯುಯುಧಾನ ಎಂದು ಪ್ರಸಿದ್ಧನಾದನು. ಇವನೇ ಮುಂದೆ ಯುದ್ಧದಲ್ಲಿ ಭೂರಿಶ್ರವಸ್ಸಿನ ಮತ್ತು ಕೃತವರ್ಮನ ತಲೆಯನ್ನು ಕತ್ತರಿಸಿದನು. ದ್ರೋಣಾಚಾರ್ಯರ ನೂರು ಬಿಲ್ಲುಗಳನ್ನು ತುಂಡು ಮಾಡಿದನು. ದ್ರುಪದನ ವಿಷಯವನ್ನು ನಾವು ಆಗಲೆ ನೋಡಿರುವೆವು. ದ್ರೋಣನ ಬದ್ಧವೈರಿ ಅವನು. ದೃಷ್ಟಕೇತು ಎನ್ನುವ ವನೇ ಶಿಶುಪಾಲನ ಮಗ. ಮುಂದೆ ದ್ರೋಣಾಚಾರ್ಯರು ಇವನನ್ನು ಕೊಂದರು. ಚೇಕಿತಾನ ಎಂಬುವನು ವೃಷ್ಣಿ ವಂಶದ ಪ್ರಖ್ಯಾತನಾದ ವೀರ. ಪುರುಜಿತ್ ಎನ್ನುವವನೇ ಕುಂತಿಯ ಸಹೋದರ. ಇವನನ್ನು ಅನಂತರ ದ್ರೋಣರು ಕೊಂದರು. ಕುಂತಿಭೋಜ ಎಂಬುವನು ಕುಂತಿಯ ಸಾಕುತಂದೆ. ಶೈಬ್ಯನೆನ್ನು ವವನು ಆಗಿನ ಕಾಲದ ಪ್ರಖ್ಯಾತ ವೀರರಲ್ಲಿ ಒಬ್ಬ. ಯುಧಾಮನ್ಯು ಮತ್ತು ಉತ್ತಮೌಜಸ್ ಎನ್ನುವವರು ಪಾಂಚಾಲದೇಶಕ್ಕೆ ಸೇರಿದವರು. ಇವರು ಅರ್ಜುನನ ಚಕ್ರರಕ್ಷಕರಾಗಿದ್ದರು. ಮುಂದೆ ಅಶ್ವತ್ಥಾಮನಿಂದ ಹತರಾದರು. ಸುಭದ್ರೆಯ ಮಗನೆ ವೀರಾಧಿವೀರನಾದ ಅಭಿಮನ್ಯು. ಚಕ್ರವ್ಯೂಹ ವನ್ನು ಪ್ರವೇಶಿಸಿ ವೀರಾಧಿವೀರರನ್ನು ಕೊಂದು ತಾನು ಸಮರದಲ್ಲಿ ಹತನಾದನು. ದ್ರೌಪದಿಯ ಮಕ್ಕಳೇ ಪ್ರತಿವಿಂದ್ಯ, ಸುತಸೋಮ, ಶ್ರುತಕೀರ್ತಿ, ಶತಾನೀಕ, ಶುಕ್ರವರ್ಮ ಎಂಬುವರು. ಅಶ್ವತ್ಥಾಮ ಅನಂತರ ಇವರನ್ನೆಲ್ಲಾ ಕೊಂದನು. ಇವರನ್ನೆಲ್ಲ ಮಹಾರಥರು ಎಂದು ಕರೆಯುವರು. ಹಿಂದಿನ ಕಾಲದ ಯುದ್ಧದಲ್ಲಿ ಕೆಳಗೆ ಬರುವ ಹೆಸರುಗಳನ್ನು ಉಪಯೋಗಿಸುತ್ತಾರೆ: ಯಾರು ಯುದ್ಧದಲ್ಲಿದ್ದು ಕೊಂಡು ಹತ್ತುಸಾವಿರ ಪದಾತಿಗಳೊಡನೆ ಯುದ್ಧಮಾಡುವನೋ ಅವನನ್ನು ಮಹಾರಥಿಯೆಂದು ಕರೆಯುವರು. ಯಾರು ತನ್ನಂತೆ ಇರುವ ಮಹಾರಥಿಯೊಡನೆ ಯುದ್ಧಮಾಡುವರೋ ಅವರನ್ನು ಸಮರಥ ಎಂದು ಕರೆಯುವರು. ಯಾರು ಹಲವು ಸಮರಥರೊಡನೆ ಏಕಕಾಲದಲ್ಲಿ ಯುದ್ಧ ಮಾಡಬಲ್ಲವನಾಗಿದ್ದಾನೊ ಅವನನ್ನು ಅತಿರಥ ಎಂದು ಕರೆಯುತ್ತಿದ್ದರು.

ಹಿಂದಿನ ಕಾಲದ ಯುದ್ಧ ಈಗಿನ ಕಾಲದ ಯುದ್ಧದಂತೆ ಆಗಿರಲಿಲ್ಲ. ಅದೊಂದು ಭಯಾನಕವಾದ ಕ್ರೀಡೆಯಂತೆ ಇತ್ತು. ಇಲ್ಲಿ ಪ್ರತಿಯೊಬ್ಬರೂ ಹಲವಾರು ನಿಯಮಗಳನ್ನು ಪಾಲಿಸಬೇಕಾಗಿತ್ತು. ಯಾರು ಅದಕ್ಕೆ ವಿರುದ್ಧವಾಗಿ ಹೋಗುವರೋ ಅವರನ್ನು ತಿರಸ್ಕಾರ ದೃಷ್ಟಿಯಿಂದ ನೋಡುತ್ತಿದ್ದರು.

ಪಾಂಡವರ ಪಕ್ಷದ ಪ್ರಮುಖ ವೀರರನ್ನು ಹೇಳಿದ ಮೇಲೆ ದುರ್ಯೋಧನ ತನ್ನ ಕಡೆಯೇನೂ ಅವರಿಗಿಂತ ಕಡಿಮೆಯಿಲ್ಲ; ಇಲ್ಲಿ ಪ್ರಖ್ಯಾತ ವೀರಾಧಿವೀರರು ಇರುವರು, ಎದೆಗುಂದಬೇಕಾಗಿಲ್ಲ, ಸೋಲಿನ ಮನೋಭಾವವನ್ನು ತಾಳಬೇಕಾಗಿಲ್ಲ ಎಂದು ತನ್ನ ಕಡೆಯವರ ವಿಷಯವನ್ನು ಹೇಳುತ್ತಾನೆ.

\begin{verse}
ಅಸ್ಮಾಕಂ ತು ವಿಶಿಷ್ಟಾ ಯೇ ತಾನ್ನಿಬೋಧ ದ್ವಿಜೋತ್ತಮ ।\\ನಾಯಕಾ ಮಮ ಸೈನ್ಯಸ್ಯ ಸಂಜ್ಞಾರ್ಥಂ ತಾನ್ ಬ್ರವೀಮಿ ತೇ \num{॥ ೭ ॥}
\end{verse}

{\small ದ್ವಿಜೋತ್ತಮರೆ, ನಮ್ಮ ಪಕ್ಷದಲ್ಲಿ ಕೂಡ ಯಾರು ಮುಖ್ಯರಾದ ಸೇನಾಪತಿಗಳೊ ಅವರನ್ನು ತಿಳಿದುಕೊಳ್ಳಿ. ನಿಮ್ಮ ಜ್ಞಾಪಕಕ್ಕೋಸ್ಕರ ಅವರ ಹೆಸರುಗಳನ್ನು ಹೇಳುತ್ತೇನೆ.}

ದುರ್ಯೋಧನ ದ್ರೋಣಾಚಾರ್ಯರಿಗೆ ಧೈರ್ಯವನ್ನು ಕೊಡುವುದಕ್ಕಾಗಿ ತಮ್ಮ ಪಡೆ ಕೂಡ ಪಾಂಡವ ರಿಗೆ ಕಡಮೆ ಇಲ್ಲ. ಇಲ್ಲಿಯೂ ಪ್ರಖ್ಯಾತ ವೀರರು ಇರುವರು. ಅವರೆಲ್ಲ ಆಗಲೇ ದ್ರೋಣಾಚಾರ್ಯರಿಗೆ ಗೊತ್ತಿದೆ. ಆದರೆ ಪುನಃ ಅವರಿಗೆ ಜ್ಞಾಪಿಸುವುದಕ್ಕಾಗಿ ತಮ್ಮ ಕಡೆಯಲ್ಲಿ ಎಂತಹ ಅಪ್ರತಿಮ ವೀರರು ಇರುವರು ಎಂಬುದನ್ನು ಯೋಚಿಸಿದಾಗ ಒಂದು ಧೈರ್ಯ ಬರುವುದೆಂದು ಸೂಚನೆ ಕೊಡುವುದಕ್ಕಾಗಿ ಹೇಳುತ್ತಾನೆ.

\begin{verse}
ಭವಾನ್ ಭೀಷ್ಮಶ್ಚ ಕರ್ಣಶ್ಚ ಕೃಪಶ್ಚ ಸಮಿತಿಂಜಯಃ ।\\ಅಶ್ವತ್ಥಾಮಾ ವಿಕರ್ಣಶ್ಚ ಸೌಮದತ್ತಿರ್ಜಯದ್ರಥಃ \num{॥ ೮ ॥}
\end{verse}

{\small ನೀವು,ಭೀಷ್ಮ, ಕರ್ಣ, ಸಮರಜಯಿಯಾದ ಕೃಪ, ಅಶ್ವತ್ಥಾಮ, ವಿಕರ್ಣ, ಸೋಮದತ್ತನ ಮಗನಾದ ಭೂರಿ ಶ್ರವಸ್ಸು ಮತ್ತು ಜಯದ್ರಥ.}

ದುರ್ಯೋಧನ ಮೊದಲು ದ್ರೋಣಾಚಾರ್ಯರ ಹೆಸರನ್ನು ಹೇಳುತ್ತಾನೆ. ಅವನಿಗೆ ಸೇನಾನಿಯ ಪಟ್ಟ ಬರಲಿಲ್ಲವೆಂಬ ಅತೃಪ್ತಿ ಇದ್ದರೆ ಸಮಾಧಾನವಾಗಲಿ ಎಂದು. ಅನಂತರವೆ ಸಮರಜಯಿಯಾಗಿರು ವಂತಹ ಕೃಪಾಚಾರ್ಯರು, ದ್ರೋಣರ ಹತ್ತಿರದ ನೆಂಟರು. ಅನಂತರ ದ್ರೋಣಾಚಾರ್ಯರ ಮಗನಾದ ಅಶ್ವತ್ಥಾಮನ ಹೆಸರನ್ನು ಹೇಳುತ್ತಾನೆ. ದ್ರೋಣಾಚಾರ್ಯರನ್ನು ಅವರ ನೆಂಟರನ್ನು ಅವರ ಮಗನನ್ನು ಮುಂಚೆಯೇ ಹೊಗಳುತ್ತಾನೆ. ಹೊಗಳಿಕೆ ಒಂದು ವಿಧವಾದ ಲಂಚ. ಇದಕ್ಕೆ ವಶವಾಗದವರು ಬಹಳ ಅಪರೂಪ. ಹೆತ್ತವರಿಗಂತೂ ತಮ್ಮನ್ನು ಹೊಗಳುವಾಗ ಆಗುವ ಆನಂದಕ್ಕಿಂತ ಹೆಚ್ಚು ಅವರ ಮಕ್ಕಳನ್ನು ಹೊಗಳಿದರೆ ಆಗುವುದು. ದುರ್ಯೋಧನನ ಬಾಯಿಂದಲೆ ತನ್ನ ಮಗ ಪ್ರಖ್ಯಾತನಾದ ವೀರ ಎಂದು ಹೊಗಳಿಸಿಕೊಳ್ಳುವನು. ವಿಕರ್ಣ ಎಂಬುವನು ಧೃತರಾಷ್ಟ್ರನಿಗೆ ಮೂರನೆಯ ಮಗ, ದುರ್ಯೋಧನನ ತಮ್ಮ. ಸೌಮದತ್ತಿ ಎಂಬುವನು ಬಾಹ್ಲೀಕ ರಾಜನಾದ ಸೋಮದತ್ತನ ಮಗ. ಪಂಜಾಬಿನ ಒಂದು ಭಾಗಕ್ಕೆ ಆಗ ಇದ್ದ ಹೆಸರು ಇದು.

\begin{verse}
ಅನ್ಯೇ ಚ ಬಹವಃ ಶೂರಾ ಮದರ್ಥೇ ತ್ಯಕ್ತಜೀವಿತಾಃ ।\\ನಾನಾಶಸ್ತ್ರಪ್ರಹರಣಾಃ ಸರ್ವೇ ಯುದ್ಧವಿಶಾರದಾಃ \num{॥ ೯ ॥}
\end{verse}

{\small ಅನೇಕ ಜನ ಶೂರರು ನನಗಾಗಿ ಪ್ರಾಣತ್ಯಾಗ ಮಾಡಲು ಕೃತಸಂಕಲ್ಪರಾಗಿ ಬಂದಿದ್ದಾರೆ. ಇವರೆಲ್ಲರೂ ಬಹುಬಗೆಯ ಶಸ್ತ್ರಾಸ್ತ್ರಗಳ ಬಳಕೆಯ ನಿಪುಣರು ಮತ್ತು ಯುದ್ಧನೀತಿಯಲ್ಲಿ ನಿಪುಣರು.}

ದುರ್ಯೋಧನ ಅನೇಕ ಜನ ನನಗೋಸ್ಕರ ಪ್ರಾಣ ನೀಡಲು ಬಂದಿದ್ದಾರೆ ಎನ್ನುವನು. ಅವನ ಸಹಾಯಕ್ಕೆ ಬಂದವರು ಯಾರೂ ಯುದ್ಧಕ್ಕೆ ಬೆನ್ನು ತೋರಿಸಿ ಹೋಗುವವರಲ್ಲ. ಜೀವವಿರುವವರೆಗೆ ಯುದ್ಧ ಮಾಡುವರು. ದುರ್ಯೋಧನನಲ್ಲಿ ಮೊಂಡತನ, ಛಲ, ಅಭಿಮಾನ ಮುಂತಾದ ದುರ್ಗುಣಗಳು ಬೇಕಾದಷ್ಟು ಇದ್ದುವು. ಆದರೆ ಅವನು ಮತ್ತೊಬ್ಬರ ಸ್ನೇಹವನ್ನು ಸಂಪಾದಿಸುವುದರಲ್ಲಿ ನಿಪುಣ. ಈ ಕಲೆ ಅವನಲ್ಲಿತ್ತು. ಅದೂ ಎಂತಹ ಸ್ನೇಹಿತರು! ಪ್ರಾಣ ಹೋದರೂ ಇವನನ್ನು ತ್ಯಜಿಸತಕ್ಕವರಲ್ಲ ಮತ್ತು ಇವನ ಸಹಾಯಕ್ಕೆ ಬಂದವರು ಕಳಪೆಯ ಜನರಲ್ಲ. ಹಲವು ಬಗೆಯ ಶಸ್ತ್ರಾಸ್ತ್ರಗಳಲ್ಲಿ ನಿಪುಣರಾದ ದಕ್ಷರು. ಇದೆಲ್ಲ ದ್ರೋಣಾಚಾರ್ಯರಿಗೆ ಏನಾದರೂ ಅಂಜಿಕೆಯ ಮೋಡ ಮುತ್ತಿದ್ದರೆ ಅದನ್ನು ಆಚೆಗೆ ಓಡಿಸುವಂತೆ ಮಾಡುವುದಕ್ಕಾಗಿ.

\begin{verse}
ಅಪರ್ಯಾಪ್ತಂ ತದಸ್ಮಾಕಂ ಬಲಂ ಭೀಷ್ಮಾಭಿರಕ್ಷಿತಮ್ ।\\ಪರ್ಯಾಪ್ತಂ ತ್ವಿದಮೇತೇಷಾಂ ಬಲಂ ಭೀಮಾಭಿರಕ್ಷಿತಮ್\num{॥ ೧೦ ॥}
\end{verse}

{\small ಭೀಷ್ಮನಿಂದ ರಕ್ಷಿತವಾದ ನಮ್ಮ ಸೇನೆ ಅಪರಿಮಿತವಾಗಿದೆ. ಆದರೆ ಭೀಮನಿಂದ ರಕ್ಷಿತವಾದ ಪಾಂಡವರ ಸೇನೆಯಾದರೊ ಪರಿಮಿತವಾಗಿದೆ.}

ದುರ್ಯೋಧನ ತನ್ನ ಕಡೆ ಇರುವ ಹನ್ನೊಂದು ಅಕ್ಷೋಹಿಣಿ ಸೈನ್ಯ ಮತ್ತು ಅದರಲ್ಲಿರುವ ಅತಿರಥ ಮಹಾರಥರು ಇವರನ್ನೆಲ್ಲ ನೋಡಿ ತಾವು ಗೆಲ್ಲುವುದರಲ್ಲಿ ಯಾವ ಸಂದೇಹವೂ ಇಲ್ಲ ಎಂದು ಭಾವಿಸುವನು. ದ್ರೋಣಾಚಾರ್ಯರಿಗೆ ಮತ್ತೂ ಧೈರ್ಯ ಕೊಡುವುದಕ್ಕಾಗಿ ಇದು. ಪಾಂಡವರ ಸೇನೆ ಯಾದರೋ ಕೌರವರ ಸೇನೆಗೆ ಹೋಲಿಸಿ ನೋಡಿದರೆ ಸಂಖ್ಯೆಯಲ್ಲಿ ಸಣ್ಣದು.

\begin{verse}
ಅಯನೇಷು ಚ ಸರ್ವೇಷು ಯಥಾಭಾಗಮವಸ್ಥಿತಾಃ ।\\ಭೀಷ್ಮಮೇವಾಭಿರಕ್ಷಂತು ಭವಂತಃ ಸರ್ವ ಏವ ಹಿ \num{॥ ೧೧ ॥}
\end{verse}

{\small ನೀವೆಲ್ಲ ಸೇನೆಯ ವ್ಯೂಹ ದ್ವಾರಗಳಲ್ಲಿ ನಿಮ್ಮ ನಿಮ್ಮ ಭಾಗಗಳಲ್ಲಿ ಇದ್ದುಕೊಂಡು ಭೀಷ್ಮರನ್ನು ರಕ್ಷಿಸಬೇಕು.}

ಮೊದಲಿನಿಂದಲೂ ಕರ್ಣನಿಗೂ ಭೀಷ್ಮರಿಗೂ ಆಗುತ್ತಿರಲಿಲ್ಲ. ದ್ರೋಣಾಚಾರ್ಯರ ಪ್ರೀತಿಯೆಲ್ಲ ಪಾಂಡವರ ಕಡೆ ವಾಲುತ್ತಿದ್ದುದನ್ನು ಕೌರವ ಬಲ್ಲ. ಸೇನೆಯಲ್ಲಿ ಯಾವ ವಿಧವಾದ ಅಶಿಸ್ತು ಕೂಡ ಬರದಂತೆ ನೋಡಿಕೊಳ್ಳಬೇಕೆಂದು ಹೇಳುವನು. ಭೀಷ್ಮ ಕೌರವ ಸೇನೆಯ ಸೇನಾನಿ. ಅವನು ಇಡೀ ಹನ್ನೊಂದು ಅಕ್ಷೋಹಿಣಿ ಕೌರವ ಸೇನೆ ಎಂಬ ದೇಹಕ್ಕೆ ಮೆದುಳು ಇದ್ದಂತೆ. ಅವನಿಗೇನಾದರೂ ಅಪಾಯವಾದರೆ ಸೇನೆಯೆಲ್ಲಾ ಚೆಲ್ಲಾಪಿಲ್ಲಿಯಾಗುವುದು. ಆದಕಾರಣ ನೀವುಗಳೆಲ್ಲ ಭೀಷ್ಮರನ್ನು ರಕ್ಷಿಸಿ ಎಂದು ಹೇಳುವನು.

ತಸ್ಯ ಸಂಜನಯನ್ ಹರ್ಷಂ ಕುರುವೃದ್ಧಃ ಪಿತಾಮಹಃ ।\\ಸಿಂಹನಾದಂ ವಿನದ್ಯೋಚ್ಚೈಃ ಶಂಖಂ ದಧ್ಮೌ ಪ್ರತಾಪವಾನ್ \num{॥ ೧೨ ॥}

{\small ಕುರುಕುಲಕ್ಕೆ ಪಿತಾಮಹನೂ ಪ್ರತಾಪಶಾಲಿಯೂ ಆದ ಭೀಷ್ಮನು ದುರ್ಯೋಧನನಿಗೆ ಹರ್ಷವನ್ನು ಉಂಟುಮಾಡು ವುದಕ್ಕಾಗಿ ಸಿಂಹನಾದವನ್ನು ಮಾಡಿ ಶಂಖವನ್ನು ಊದಿದನು.}

ದುರ್ಯೋಧನ ದ್ರೋಣಾಚಾರ್ಯರ ಸಮೀಪಕ್ಕೆ ಹೋಗಿ ಅವರನ್ನು ಮನಸ್ಸಿಟ್ಟು ಯುದ್ಧ ಮಾಡ ಬೇಕೆಂದು ಕೋರಿಕೊಳ್ಳುವುದನ್ನು ಭೀಷ್ಮನು ನೋಡಿದನು. ಆತನು ತುಂಬಾ ವ್ಯಾಕುಲದಲ್ಲಿರ ಬಹುದೆಂದು ಅವನ ಮನಸ್ಸಿನ ದುಗುಡವನ್ನು ಪರಿಹರಿಸುವುದಕ್ಕಾಗಿ ತಾನು ಸೇನೆಯ ಸಮೇತ ಯುದ್ಧಮಾಡುವುದಕ್ಕೆ ಅಣಿಯಾಗಿರುವೆನು ಎಂಬುದನ್ನು ಸೂಚಿಸುವುದಕ್ಕಾಗಿ ಸಿಂಹನಾದವನ್ನು ಮಾಡಿ ಶಂಖವನ್ನು ಊದಿದನು. ಹಿಂದಿನ ಕಾಲದಲ್ಲಿ ಯುದ್ಧಕ್ಕೆ ಮುಂಚೆ ನಾವು ಯುದ್ಧಮಾಡಲು ಪ್ರಾರಂಭಿಸುತ್ತಿದ್ದೇವೆ ಎಂಬುದನ್ನು ಸೂಚಿಸುತ್ತಿದ್ದರು ಎಂದು ಕಾಣುವುದು. ಈಗಿನ ಕಾಲದಲ್ಲಿ ಹೇಗೆ ಫುಟ್​ಬಾಲ್ ಕ್ರಿಕೆಟ್ ಆಟಗಳು ಪ್ರಾರಂಭವಾಗುವಾಗ ಎರಡೂ ಕಡೆಯವರು ತಾವು ಅದಕ್ಕೆ ಸಿದ್ಧವಾಗಿದ್ದೇವೆ ಎಂದು ಸೂಚಿಸುತ್ತಾರೆಯೋ ಹಾಗೆ. ಇಲ್ಲಿಯೂ ಕೂಡ ಕೌರವರು ಮುಂಚೆ ನಾವು ಯುದ್ಧಮಾಡಲು ಪ್ರಾರಂಭಿಸಿದ್ದೇವೆ ಎಂಬುದನ್ನು ನೋಡುತ್ತೇವೆ.

\begin{verse}
ತತಃ ಶಂಖಾಶ್ಚ ಭೇರ್ಯಶ್ಚ ಪಣವಾನಕಗೋಮುಖಾಃ ।\\ಸಹಸೈವಾಭ್ಯಹನ್ಯಂತ ಸ ಶಬ್ದಸ್ತುಮುಲೋ ಭವತ್ \num{॥ ೧೩ ॥}
\end{verse}

{\small ಅನಂತರ ಶಂಖ, ಭೇರಿ, ಡೋಲು, ಮೃದಂಗ, ಗೋಮುಖಾದಿ ವಾದ್ಯಗಳು ಒಂದೇಸಾರಿ ಮೊಳಗಿದುವು. ರಣಶಬ್ದ ಭಯಂಕರವಾಗಿತ್ತು.}

ಯಾವಾಗ ಭೀಷ್ಮಾಚಾರ್ಯರು ತಾವು ಯುದ್ಧಕ್ಕೆ ಅಣಿಯಾಗಿದ್ದೇವೆ ಎಂಬುದನ್ನು ಸೂಚಿಸಿದರೊ ಆಗ ಅವರ ಕಡೆಯ ಪ್ರಮುಖ ವೀರರೆಲ್ಲ ತಮ್ಮ ಒಪ್ಪಿಗೆಯನ್ನು ಸೂಚಿಸುವುದಕ್ಕಾಗಿ ತಮ್ಮ ತಮ್ಮ ಶಂಖಗಳನ್ನು ಊದಿದರು. ಸೈನ್ಯದಲ್ಲಿ ರಣೋತ್ಸಾಹವನ್ನು ಉಂಟುಮಾಡುವ ವಾದ್ಯಗಳೂ ಕೂಡ ಮೊಳಗಿದುವು, ಈಗಿನ ಮಿಲಿಟರಿಯವರು ತಮ್ಮ ತಮ್ಮ ಬ್ಯಾಂಡನ್ನು ಬಾರಿಸುವಂತೆ. ಯಾವಾಗ ಕೌರವರ ಕಡೆಯವರು ತಾವು ಯುದ್ಧ ಮಾಡುವುದಕ್ಕೆ ಸಿದ್ಧರಾಗಿರುವೆವು ಎಂಬುದನ್ನು ಮೊದಲು ಸೂಚಿಸಿದರೊ ಆಗ ಪಾಂಡವರ ಕಡೆಯವರು ತಮ್ಮ ಶಂಖಗಳನ್ನು ಊದಿ ತಾವು ಕೂಡ ಅವರ ಸವಾಲನ್ನು ಸಮರಾಂಗಣದಲ್ಲಿ ಎದುರಿಸುವುದಕ್ಕೆ ಸಿದ್ಧರಾಗಿರುವೆವು ಎಂಬುದನ್ನು ಸೂಚಿಸಿದರು.

\begin{verse}
ತತಃ ಶ್ವೇತೈರ್ಹಯೈರ್ಯುಕ್ತೇ ಮಹತಿ ಸ್ಯಂದನೇ ಸ್ಥಿತೌ ।\\ಮಾಧವಃ ಪಾಂಡವಶ್ಚೈವ ದಿವ್ಯೌ ಶಂಖೌ ಪ್ರದಧ್ಮತುಃ \num{॥ ೧೪ ॥}
\end{verse}

\begin{verse}
ಪಾಂಚಜನ್ಯಂ ಹೃಷೀಕೇಶೋ ದೇವದತ್ತಂ ಧನಂಜಯಃ ।\\ಪೌಂಡ್ರಂ ದಧ್ಮೌ ಮಹಾಶಂಖಂ ಭೀಮಕರ್ಮಾ ವೃಕೋದರಃ \num{॥ ೧೫ ॥}
\end{verse}

\begin{verse}
ಅನಂತವಿಜಯಂ ರಾಜಾ ಕುಂತೀಪುತ್ರೋ ಯುಧಿಷ್ಠಿರಃ ।\\ನಕುಲಃ ಸಹದೇವಶ್ಚ ಸುಘೋಷಮಣಿಪುಷ್ಪಕೌ \num{॥ ೧೬ ॥}
\end{verse}

{\small ಅನಂತರ ಬಿಳಿಯ ಕುದುರೆಗಳಿಂದ ಕಟ್ಟಿದ ಮಹಾರಥದಲ್ಲಿ ಕುಳಿತ ಕೃಷ್ಣಾರ್ಜುನರು ತಮ್ಮ ತಮ್ಮ ದಿವ್ಯವಾದ ಶಂಖಗಳನ್ನು ಊದಿದರು. ಕೃಷ್ಣ ಪಾಂಚಜನ್ಯವನ್ನು ಅರ್ಜುನ ದೇವದತ್ತವನ್ನು ಕರ್ಮಪಟುವಾದ ಭೀಮನು ಪೌಂಡ್ರಕವನ್ನು ಊದಿದರು. ಕುಂತೀಪುತ್ರನಾದ ರಾಜ ಯುಧಿಷ್ಠಿರನು ಅನಂತರ ವಿಜಯವನ್ನು, ನಕುಲ ಸಹದೇವರು ಸುಘೋಷ ಮಣಿಪುಷ್ಪಗಳೆಂಬ ತಮ್ಮ ತಮ್ಮ ಶಂಖಗಳನ್ನು ಊದಿದರು.}

\begin{verse}
ಕಾಶ್ಯಶ್ಚ ಪರಮೇಷ್ವಾಸಃ ಶಿಖಂಡೀ ಚ ಮಹಾರಥಃ ।\\ಧೃಷ್ಟದ್ಯುಮ್ನೋ ವಿರಾಟಶ್ಚ ಸಾತ್ಯಶ್ಚಾಪರಾಜಿತಃ \num{॥ ೧೭ ॥}
\end{verse}

\begin{verse}
ದ್ರುಪದೋ ದ್ರೌಪದೇಯಾಶ್ಚ ಸರ್ವಶಃ ಪೃಥಿವೀಪತೇ ।\\ಸೌಭದ್ರಶ್ಚ ಮಹಾಬಾಹುಃ ಶಂಖಾನ್ ದಧ್ಮುಃ ಪೃಥಕ್ ಪೃಥಕ್ \num{॥ ೧೮ ॥}
\end{verse}

\begin{verse}
ಸ ಘೋಷೋ ಧಾರ್ತರಾಷ್ಟ್ರಾಣಾಂ ಹೃದಯಾನಿ ವ್ಯದಾರಯತ್ ।\\ನಭಶ್ಚ ಪೃಥಿವೀಂ ಚೈವ ತುಮುಲೋ ವ್ಯನುನಾದಯನ್ \num{॥ ೧೯ ॥}
\end{verse}

{\small ಹೇ ರಾಜನೇ, ದೊಡ್ಡ ಬಿಲ್ಲುಗಾರನಾದ ಕಾಶಿರಾಜನು, ಮಹಾರಥರಾದ ಶಿಖಂಡಿ ಮತ್ತು ಧೃಷ್ಟದ್ಯುಮ್ನರು ವಿರಾಟ ಅಪರಾಜಿತನಾದ ಸಾತ್ಯಕಿ ದ್ರುಪದರಾಜ ಮತ್ತು ದ್ರೌಪದಿಯ ಮಕ್ಕಳಾದ ಉಪಪಾಂಡವರು ಮಹಾ ಬಾಹುವಾದ ಅಭಿಮನ್ಯು ಇವರೆಲ್ಲರೂ ತಮ್ಮ ತಮ್ಮ ಶಂಖಗಳನ್ನು ಊದಿದರು. ಆ ಭಯಂಕರವಾದ ಶಂಖಧ್ವನಿ, ಭೂಮಿ, ಆಕಾಶಗಳಲ್ಲಿ ಅನುರಣಿತವಾಗಿ ಕೌರವರ ಹೃದಯವನ್ನು ಬಿರಿಯುವಂತೆ ಮಾಡಿತು.}

ಈ ರಣ ನಿನಾದ ಆಕಾಶವನ್ನು ಸೀಳುವಂತೆ ಇತ್ತು. ಕೌರವರ ಹೃದಯವನ್ನು ಭೇದಿಸುವಂತೆ ಇತ್ತು. ಇನ್ನೇನು ಯುದ್ಧ ಶುರು ಆಗುವುದಕ್ಕೆ ಮುಂಚೆ ಅರ್ಜುನನು ಕೌರವರ ಕಡೆ ಯಾರು ಯಾರು ಸೇರಿದ್ದಾರೆ ಯುದ್ಧ ಮಾಡುವುದಕ್ಕೆ ಎಂಬುದನ್ನು ತಿಳಿದುಕೊಳ್ಳಲು ಇಚ್ಛಿಸಿದನು.

\begin{verse}
ಅಥ ವ್ಯವಸ್ಥಿತಾನ್ ದೃಷ್ಟ್ವಾ ಧಾರ್ತರಾಷ್ಟ್ರಾನ್ ಕಪಿಧ್ವಜಃ ।\\ಪ್ರವೃತ್ತೇ ಶಸ್ತ್ರಸಂಪಾತೇ ಧನುರುದ್ಯಮ್ಯ ಪಾಂಡವಃ \eng{॥ ೨ಂ ॥}\\ಹೃಷೀಕೇಶಂ ತದಾ ವಾಕ್ಯಮಿದಮಾಹ ಮಹೀಪತೇ ।
\end{verse}

{\small ಎಲೈ ರಾಜನೆ, ಆಗ ಯುದ್ಧಕ್ಕಾಗಿ ಸಿದ್ಧವಾಗಿರುವ ಕೌರವರನ್ನು ನೋಡಿ, ಕಪಿಧ್ವಜನಾದ ಅರ್ಜುನನು ಇನ್ನೇನು ಶಸ್ತ್ರ ಪ್ರಯೋಗಕ್ಕೆ ಸಮಯ ಸನ್ನಿಹಿತವಾಯಿತೆಂದು ತನ್ನ ಬಿಲ್ಲನ್ನು ಎತ್ತಿ ಹಿಡಿದು ಶ್ರೀಕೃಷ್ಣನಿಗೆ ಈ ಮಾತನ್ನು ಹೇಳುತ್ತಾನೆ:}

\begin{verse}
ಸೇನಯೋರುಭಯೋರ್ಮಧ್ಯೇ ರಥಂ ಸ್ಥಾಪಯ ಮೇsಚ್ಯುತ \eng{॥ ೨೧ ॥}\\ಯಾವದೇತಾನ್ ನಿರೀಕ್ಷೇsಹಂ ಯೋದ್ಧುಕಾಮಾನವಸ್ಥಿತಾನ್ ।
\end{verse}

\begin{verse}
ಕೈರ್ಮಯಾ ಸಹ ಯೋದ್ಧವ್ಯಮಸ್ಮಿನ್ ರಣಸಮುದ್ಯಮೇ \eng{॥ ೨೨ ॥}\\ಯೋತ್ಸ್ಯಮಾನಾನವೇಕ್ಷೇsಹಂ ಯ ಏತೇsತ್ರ ಸಮಾಗತಾಃ ।\\ಧಾರ್ತರಾಷ್ಟ್ರಸ್ಯ ದುರ್ಬುದ್ಧೇರ್ಯುದ್ಧೇ ಪ್ರಿಯಚಿಕೀರ್ಷವಃ \eng{॥ ೨೩ ॥}
\end{verse}

{\small ಎಲೈ ಅಚ್ಯುತ, ಈ ರಣದಲ್ಲಿ ನಾನು ಯಾರೊಡನೆ ಯುದ್ಧ ಮಾಡಬೇಕೆಂದು ಗೊತ್ತು ಮಾಡುವುದಕ್ಕಾಗಿ ನನ್ನ ರಥವನ್ನು ಎರಡು ಸೈನ್ಯಗಳ ಮಧ್ಯೆ ನಿಲ್ಲಿಸು. ಇಲ್ಲಿ ಯುದ್ಧಮಾಡಬೇಕೆಂದು ಬಂದಿರುವ ಇವರನ್ನು ನೋಡೋಣ. ದುರ್ಬುದ್ಧಿಯಾದ ದುರ್ಯೋಧನನಿಗೆ ಯುದ್ಧದಲ್ಲಿ ಪ್ರಿಯವನ್ನು ಉಂಟುಮಾಡಬೇಕೆಂದು ಯಾರು ಯಾರು ಕಾದಾಟಕ್ಕೆ ಅನುವಾಗಿ ಇಲ್ಲಿ ಬಂದಿರುವರೊ ಅವರನ್ನು ನಾನು ನೋಡಬೇಕು.}

ಸಂಜಯ ಧೃತರಾಷ್ಟ್ರನಿಗೆ ಹೇಳುತ್ತಾನೆ:

\begin{verse}
ಏವಮುಕ್ತೋ ಹೃಷೀಕೇಶೋ ಗುಡಾಕೇಶೇನ ಭಾರತ ।\\ಸೇನಯೋರುಭಯೋರ್ಮಧ್ಯೇ ಸ್ಥಾಪಯಿತ್ವಾ ರಥೋತ್ತಮಮ್ \eng{॥ ೨೪ ॥}
\end{verse}

\begin{verse}
ಭೀಷ್ಮದ್ರೋಣಪ್ರಮುಖತಃ ಸರ್ವೇಷಾಂ ಚ ಮಹೀಕ್ಷಿತಾಮ್ ।\\ಉವಾಚ ಪಾರ್ಥ ಪಶ್ಯೈತಾನ್ ಸಮವೇತಾನ್ ಕುರೂನಿತಿ \eng{॥ ೨೫ ॥}
\end{verse}

{\small ಅರ್ಜುನನು ಹಾಗೆ ಹೇಳಿದ ಮೇಲೆ ಶ್ರೀಕೃಷ್ಣ ರಥವನ್ನು ಎರಡು ಸೈನ್ಯಗಳ ಮಧ್ಯೆ ಭೀಷ್ಮ ದ್ರೋಣ ಮತ್ತು ಇತರ ರಾಜರ ಮುಂದೆ ತನ್ನ ಉತ್ತಮವಾದ ರಥವನ್ನು ನಿಲ್ಲಿಸಿದನು. ‘ಅರ್ಜುನ, ಯುದ್ಧಕ್ಕಾಗಿ ನೆರೆದಿರುವ ಸಮಸ್ತ ಕೌರವರನ್ನು ನೋಡು’ ಎಂದನು.}

ಅರ್ಜುನನಿಗೆ ಕೌರವ ಸೇನೆಯ ಬಲಾಬಲಗಳನ್ನು ತಿಳಿದುಕೊಳ್ಳಬೇಕೆಂಬ ಕುತೂಹಲವಾಯಿತು ಮತ್ತು ಯಾರು ಯಾರು ಅವರ ಕಡೆ ಸೇರಿದ್ದರೊ ಅವರನ್ನು ಕೂಡ ತಿಳಿದುಕೊಂಡಂತೆ ಆಗುವುದು ಎಂದು ಶ್ರೀಕೃಷ್ಣನಿಗೆ ತನ್ನ ರಥವನ್ನು ಕೌರವರ ಸೇನೆಗೆ ಎದುರಾಗಿ ಎರಡು ಸೈನ್ಯಗಳ ಮಧ್ಯದಲ್ಲಿ ನಿಲ್ಲಿಸು ಎಂದು ಹೇಳುತ್ತಾನೆ. ಶ್ರೀಕೃಷ್ಣನಾದರೋ ಅರ್ಜುನನ ಈ ಕುತೂಹಲ ಸ್ವಾಭಾವಿಕ ವಾದದ್ದೆಂದು ಅದನ್ನು ತೃಪ್ತಿಪಡಿಸಲು ರಥವನ್ನು ಅಲ್ಲಿ ನಿಲ್ಲಿಸುವನು.

ಇಲ್ಲಿ ಅರ್ಜುನನನ್ನು ಗುಡಾಕೇಶ ಎಂದರೆ ನಿದ್ರೆಯನ್ನು ಗೆದ್ದವನು ಎಂಬ ಗುಣವಾಚಕದಿಂದ ಕರೆಯುತ್ತಾನೆ. ಇಲ್ಲಿ ನಿದ್ರೆಯನ್ನು ಗೆದ್ದವನು ಎಂದರೆ ನಿದ್ರಿಸುತ್ತಲೇ ಇರಲಿಲ್ಲವೆಂಬ ಅರ್ಥವಲ್ಲ. ನಿದ್ರೆ ಅವನ ಸ್ವಾಧೀನವಾಗಿತ್ತು. ಯಾವಾಗ ಬೇಕಾದರೂ ನಿದ್ರಿಸುತ್ತಿದ್ದ ಮತ್ತು ಯಾವಾಗ ಬೇಕಾದರೂ ಜಾಗ್ರತನಾಗುತ್ತಿದ್ದ. ವೀರ ಯೋಧನಾದ ನೆಪೋಲಿಯನ್ ತಾನು ಯುದ್ಧರಂಗದಲ್ಲಿ ಕುದುರೆಯ ಮೇಲಿದ್ದಾಗಲೆ ತನ್ನ ನಿದ್ರೆಯನ್ನು ಪೂರೈಸುತ್ತಿದ್ದನಂತೆ! ಅದರಂತೆಯೇ ಎಷ್ಟು ಹೊತ್ತು ಬೇಕಾದರೂ ಎಚ್ಚೆತ್ತಿರುತ್ತಿದ್ದ ಎನ್ನುವರು.

\begin{verse}
ತತ್ರಾಪಶ್ಯತ್ ಸ್ಥಿತಾನ್ ಪಾರ್ಥ ಪಿತೄನಥ ಪಿತಾಮಹಾನ್ ।\\ಆಚಾರ್ಯಾನ್ಮಾತುಲಾನ್ ಭ್ರಾತೄನ್ ಪುತ್ರಾನ್ ಪೌತ್ರಾನ್ ಸಖೀಂಸ್ತಥಾ ।\\ಶ್ವಶುರಾನ್ ಸುಹೃದಶ್ಚೈವ ಸೇನಯೋರುಭಯೋರಪಿ \eng{॥ ೨೬ ॥}
\end{verse}

{\small ಅನಂತರ ಅರ್ಜುನ, ಅಲ್ಲಿ ಎರಡು ಸೇನೆಗಳಲ್ಲಿಯೂ ಇದ್ದ ಚಿಕ್ಕಪ್ಪ, ದೊಡ್ಡಪ್ಪ, ತಾತ, ಗುರು, ಸೋದರ ಮಾವಂದಿರು, ಅಣ್ಣತಮ್ಮಂದಿರು, ಮಕ್ಕಳು, ಮೊಮ್ಮಕ್ಕಳು, ಸ್ನೇಹಿತರು, ಮಾವಂದಿರು ಮತ್ತು ಪಾಂಡವರ ಮೇಲೆ ಪ್ರೀತಿಯುಳ್ಳ ಇತರರನ್ನೂ ನೋಡಿದನು.}

\begin{verse}
ತಾನ್ ಸಮೀಕ್ಷ್ಯ ಸ ಕೌಂತೇಯಃ ಸರ್ವಾನ್ ಬಂಧೂನವಸ್ಥಿತಾನ್ ।\\ಕೃಪಯಾ ಪರಯಾವಿಷ್ಟೋ ವಿಷೀದನ್ನಿದಮಬ್ರವೀತ್\eng{॥ ೨೭ ॥}
\end{verse}

{\small ಅರ್ಜುನ ಅಲ್ಲಿ ನೆರೆದಿದ್ದ ಬಂಧುಗಳನ್ನು ಚೆನ್ನಾಗಿ ನೋಡಿ ಅತಿಶಯವಾದ ಕೃಪೆಯಿಂದ ಕೂಡಿದವನಾಗಿ ವಿಷಾದದಿಂದ ಈ ಮಾತನ್ನು ಹೇಳುತ್ತಾನೆ.}

ಅರ್ಜುನ ಸಮರಾಂಗಣದಲ್ಲಿ ಶತ್ರುಪಕ್ಷದಲ್ಲಿ ನೆರೆದ ಬಂಧುಬಳಗದವರನ್ನು ಹಾಗೆ ನೋಡುವಾಗ ಅವನ ಮನಸ್ಸಿನಲ್ಲಿ ಸುಪ್ತವಾಗಿದ್ದ ಅವರ ಮೇಲಿನ ವ್ಯಾಮೋಹವೆಲ್ಲ ಮೇಲೇಳುವುದು. ಅರ್ಜುನನ ಮನಸ್ಸಿನಲ್ಲಿ ವಾಸನೆಗಳು ನಾಶವಾಗಿರಲಿಲ್ಲ. ಆ ವಾಸನೆಗಳನ್ನು ಮೇಲಕ್ಕೆ ಎಬ್ಬಿಸಲು ಕೆಲವು ಸನ್ನಿವೇಶಗಳು ಬೇಕಾಗಿವೆ. ಆ ಸನ್ನಿವೇಶ ದೊರೆತೊಡನೆ ಒಳಗಿರುವುದು ಮೇಲೇಳುವುದು. ಅದು ಮೇಲೆದ್ದ ಮೇಲೆ ನಮ್ಮನ್ನು ಅಪ್ಪಳಿಸಲು ಮೊದಲು ಮಾಡುವುದು. ಅವರನ್ನು ಎಷ್ಟು ಎಷ್ಟು ನೋಡುತ್ತಾನೊ ಅಷ್ಟು ಅಷ್ಟು ಆಳದಿಂದ ವ್ಯಾಮೋಹದ ಅಲೆಗಳು ಮೇಲೇಳುತ್ತವೆ. ಇವನನ್ನು ಅದರಲ್ಲಿ ಮುಳುಗಿಸಿಬಿಡುವುದು. ಅದು ಹೇಗಿದೆಯೊ ಹಾಗೆ ಬಂದರೆ ಅದನ್ನು ಕಂಡು ಹಿಡಿದು ಹೊರಗೆ ಅಟ್ಟುತ್ತೇವೆ. ಅದಕ್ಕಾಗಿ ಆ ದೌರ್ಬಲ್ಯಗಳು ವೇಷವನ್ನು ಮರೆಮಾಚಿಕೊಂಡು ಬರುತ್ತವೆ. ವ್ಯಾಮೋಹದಂತೆ ಬಂದರೆ ಅದು ಮನೋದೌರ್ಬಲ್ಯ ಎಂಬುದು ಗೊತ್ತಾಗುವುದು. ಅದಕ್ಕಾಗಿಯೇ ಅದು ಕೃಪೆಯಂತೆ ಬರುವುದು, ದಯೆಯಂತೆ ಬರುವುದು.

ಕೃಪೆ ತೋರಿಸಕೂಡದು ಎಂದು ಅಲ್ಲ. ಕೃಪೆಗೆ ಒಂದು ಸಮಯ ಇದೆ, ಸನ್ನಿವೇಶ ಇದೆ. ಕ್ಷತ್ರಿಯ ಯಾವಾಗ ಕರ್ತವ್ಯಕ್ಕೆ ನಿಲ್ಲುತ್ತಾನೋ ಆಗ ಅವನು ವಜ್ರದಂತೆ ಕಠೋರನಾಗಬೇಕು. ಎದುರಿಗೆ ಗುರುಗಳು ಹಿರಿಯರು ಬಂಧು ಬಾಂಧವರು ಇದ್ದಾರೆ ಎಂದು ಕರ್ತವ್ಯಕ್ಕೆ ಚ್ಯುತಿ ತರಬಾರದು. ಕುಸುಮದಂತೆ ಮೃದುವಾಗುವುದಕ್ಕೆ ಏನೂ ಇಲ್ಲ ಅಲ್ಲಿ. ಅನಿಷ್ಟದಂತೆ ಅದನ್ನು ಗುಡಿಸಿ ಆಚೆಗೆ ಎಸೆಯಬೇಕು. ಆದರೆ ಅರ್ಜುನ ಇದನ್ನು ಮಾಡಲಾರ. ವ್ಯಾಮೋಹದಿಂದ ಅವನ ಚಿತ್ತ ಕದಡಿ ಹೋಗುವುದು. ಮುಂದೆ ಅವನು ಆಡುವ ಮಾತಿನಲ್ಲೆಲ್ಲ, ಕೊಡುವ ಕಾರಣದಲ್ಲೆಲ್ಲ, ಈ ದೌರ್ಬಲ್ಯ ಹೊಕ್ಕಿರುವುದನ್ನು ನೋಡುತ್ತೇವೆ.

ಅರ್ಜುನ ಹೀಗೆ ಹೇಳುತ್ತಾನೆ:

\begin{verse}
ದೃಷ್ಟ್ವೇಮಂ ಸ್ವಜನಂ ಕೃಷ್ಣ ಯುಯುತ್ಸುಂ ಸಮುಪಸ್ಥಿತಮ್ ।\\ಸೀದಂತಿ ಮಮ ಗಾತ್ರಾಣಿ ಮುಖಂ ಚ ಪರಿಶುಷ್ಯತಿ \eng{॥ ೨೮ ॥}
\end{verse}

{\small ಶ್ರೀಕೃಷ್ಣ, ಯುದ್ಧಮಾಡಬೇಕೆಂದು ನಿಂತಿರುವ ಈ ಸ್ವಜನರನ್ನು ನೋಡಿ ನನ್ನ ಶರೀರ ಸೊರಗುತ್ತಿದೆ, ಬಾಯಿ ಒಣಗುತ್ತಿದೆ.}

ಯಾವಾಗ ಅರ್ಜುನನು ಎದುರಿಗೆ ನಿಂತಿರುವ ಬಂಧು ಬಾಂಧವರನ್ನು ನೋಡುವನೊ ಆಗ ಮನದಲ್ಲಿ ದಾರುಣ ವ್ಯಥೆ ಆಗುವುದು. ಮನಸ್ಸಿಗೆ ಶರೀರದ ಮೇಲೆ ದೊಡ್ಡ ಸ್ವಾಧೀನವಿದೆ. ದುಃಖ ಅಂಜಿಕೆ ಕಳವಳಗಳು ಮನಸ್ಸಿನಲ್ಲಿ ಆದಾಗ ಅವು ಹೃದಯದ ಮೇಲೆ ದೊಡ್ಡ ಪ್ರಭಾವವನ್ನು ಬೀರುವುವು. ಅನೇಕ ವೇಳೆ ಪ್ರಜ್ಞೆ ತಪ್ಪಿ ಬೀಳುವೆವು. ಅರ್ಜುನ ಪ್ರಜ್ಞೆ ತಪ್ಪಿ ಬೀಳುವಷ್ಟು ಹೋಗಲಿಲ್ಲ. ಅಂಗಾಂಗಗಳು ದುರ್ಬಲವಾಗುವುವು, ಬಾಯಿ ಒಣಗಿ ಹೋಗುವುದು.

\begin{verse}
ವೇಪಥುಶ್ಚ ಶರೀರೇ ಮೇ ರೋಮಹರ್ಷಶ್ಚ ಜಾಯತೇ ।\\ಗಾಂಡೀವಂ ಸ್ರಂಸತೇ ಹಸ್ತಾತ್ ತ್ವಕ್ಚೈವ ಪರಿದಹ್ಯತೇ\eng{॥ ೨೯ ॥}
\end{verse}

{\small ನನ್ನ ಮೈಯೆಲ್ಲ ನಡುಗುತ್ತಿದೆ, ಶರೀರ ರೋಮಾಂಚನವಾಗುತ್ತಿದೆ. ಗಾಂಡೀವ ಧನುಸ್ಸು ಕೈಯಿಂದ ಜಾರಿ ಬೀಳುತ್ತಿದೆ. ದೇಹದ ಚರ್ಮ ಸುಡುತ್ತಿದೆ.}

ಅಜ್ಞಾನದಿಂದ ಕೂಡಿದ ವ್ಯಾಮೋಹದ ಬಿರುಗಾಳಿ ಅರ್ಜುನನೆಂಬ ಭೀಮವೃಕ್ಷದ ಮೇಲೆ ಬೀಸುವುದು. ಈ ವೃಕ್ಷದ ಶಾಖೋಪಶಾಖೆಗಳೆಲ್ಲ ತರತರಗುಟ್ಟುವಂತೆ ಅರ್ಜುನನ ಶರೀರಾ ದ್ಯಂತವೂ ಕಂಪಿಸುವುದು. ಬೀಸುವ ಬಿರುಗಾಳಿಯ ಜೋರಿಗೆ ಆ ಭೀಮವೃಕ್ಷದ ಕೊಂಬೆಗಳು ಅಲ್ಲಾಡಿ ಬಿದ್ದುಬಿಡುವುದೇನೋ ಎಂಬ ಶಂಕೆ ಉಂಟಾಗುವುದು.

\begin{verse}
ನ ಚ ಶಕ್ನೋಮ್ಯವಸ್ಥಾತುಂ ಭ್ರಮತೀವ ಚ ಮೇ ಮನಃ ।\\ನಿಮಿತ್ತಾನಿ ಚ ಪಶ್ಯಾಮಿ ವಿಪರೀತಾನಿ ಕೇಶವ \eng{॥ ೩ಂ ॥}
\end{verse}

{\small ಕೇಶವ, ನನಗೆ ನಿಲ್ಲುವುದಕ್ಕೂ ತ್ರಾಣವಿಲ್ಲ. ನನ್ನ ಮನಸ್ಸು ಭ್ರಮೆಯಿಂದ ಸುತ್ತುತ್ತಿದೆ. ವಿರುದ್ಧವಾದ ಶಕುನಗಳನ್ನು ಬೇರೆ ನೋಡುತ್ತಿರುವೆನು.}

ಅರ್ಜುನನ ಮನಸ್ಸು ಕದಡಿ ಹೋಗಿದೆ. ಅವನು ಈಗಿನ ಸ್ಥಿತಿಯಲ್ಲಿ ಯೋಗ್ಯವಾಗಿ ನೋಡುವ ಸ್ಥಿತಿಯಲ್ಲಿ ಇಲ್ಲ. ಮನಸ್ಸು ಯಾವಾಗ ದುರ್ಬಲವಾಗುವುದೋ ಆಗ ಸಕಲ ಗ್ರಹಬಲ ನೀನೆ ಎಂದು ನೋಡುವ ದೃಷ್ಟಿ ಮರೆಯುವುದು. ಕೆಲಸಕ್ಕೆ ಬಾರದ ಶಕುನಗಳನ್ನೆಲ್ಲ ನಂಬುತ್ತ ಹೋಗುವುದು.

\begin{verse}
ನ ಚ ಶ್ರೇಯೋsನುಪಶ್ಯಾಮಿ ಹತ್ವಾ ಸ್ವಜನಮಾಹವೇ ।\\ನ ಕಾಂಕ್ಷೇ ವಿಜಯಂ ಕೃಷ್ಣ ನ ಚ ರಾಜ್ಯಂ ಸುಖಾನಿ ಚ \eng{॥ ೩೧ ॥}
\end{verse}

\begin{verse}
ಕಿಂ ನೋ ರಾಜ್ಯೇನ ಗೋವಿಂದ ಕಿಂ ಭೋಗ್ಯೆರ್ಜೀವಿತೇನ ವಾ \eng{॥ ೩೨ ॥}
\end{verse}

{\small ಯುದ್ಧದಲ್ಲಿ ಸ್ವಜನರನ್ನು ಕೊಂದರೆ ಯಾವ ಶ್ರೇಯಸ್ಸು ಆಗುವಂತೆ ಕಾಣೆ. ಕೃಷ್ಣ, ನನಗೆ ಜಯ ಬೇಕಿಲ್ಲ, ರಾಜ್ಯ ಬೇಡ, ಸುಖ ಬೇಡ. ಗೋವಿಂದ, ರಾಜ್ಯಭೋಗಗಳಿಂದಲೂ ಅಥವಾ ಜೀವಿಸುವುದರಿಂದಲೂ ಏನು ಪ್ರಯೋಜನ?}

ಇಲ್ಲಿ ಅರ್ಜುನನ ಬುದ್ಧಿ ಕದಡಿ ಹೋಗಿದೆ ಎಂಬುದಕ್ಕೆ ಅವನಾಡುವ ಮಾತುಗಳೇ ಸಾಕ್ಷಿ. ಇಲ್ಲಿ ಶ್ರೇಯಸ್ ಎಂಬ ಪದವನ್ನು ಉಪಯೋಗಿಸುತ್ತಿರುವನು. ಆದರೆ ಅರ್ಜುನನು ಇಲ್ಲಿ ಯುದ್ಧವನ್ನು ನೋಡುತ್ತಿರುವುದು ಪ್ರೇಯಸ್ಸಿನ ದೃಷ್ಟಿಯಿಂದ, ಒಂದನ್ನು ಮತ್ತೊಂದು ಎಂದು ಭಾವಿಸುವನು. ಶ್ರೇಯಸ್ಸು ಯಾವಾಗಲೂ ಕಷ್ಟದ ಹಾದಿ, ಕಂಟಕಮಯ, ಮೊದಮೊದಲಲ್ಲಿ ನಮಗೆ ತುಂಬಾ ಅಪ್ರಿಯವಾಗುವುದು. ಅಯ್ಯೋ ಇದನ್ನು ಮಾಡಬೇಕಲ್ಲ ಎಂದು ಮಾಡುವೆವು. ಇದು ರೋಗವಿರು ವಾಗ ಕಹಿಯಾದ ಔಷಧಿಯನ್ನು ತೆಗೆದುಕೊಳ್ಳುವಂತೆ. ಪ್ರೇಯಸ್ಸಿನ ಮಾರ್ಗದಲ್ಲಿ ಯಾವ ಘರ್ಷಣೆಯೂ ಇಲ್ಲ. ಗುರುಹಿರಿಯರು ಬಂಧು ಬಾಂಧವರು ಇವರನ್ನೆಲ್ಲ ಅರ್ಜುನನು ಎದುರಿಸು ತ್ತಿರುವಾಗ, ಧರ್ಮಕ್ಕಾಗಿ ಕರ್ತವ್ಯಕ್ಕಾಗಿ ಅವರೊಡನೆ ಮನಸ್ಸಿಗೆ ಅಪ್ರಿಯವಾದರೂ ಯುದ್ಧಮಾಡ ಬೇಕಾಗಿರುವುದು ಶ್ರೇಯಸ್ಸಿನ ದೃಷ್ಟಿಯಿಂದ. ಆ ಕೆಲಸವನ್ನು ಬಿಡುವುದು ಪ್ರೇಯಸ್ಸಿನ ಹಾದಿ. ಇಲ್ಲಿ ಮನಸ್ಸಿಗೆ ಯಾವ ನೋವು ಇಲ್ಲ, ಮೊದಮೊದಲಲ್ಲಿ. ಆದರೆ ದೂರದ ಪರಿಣಾಮ ಕಾದುಕೊಂಡಿದೆ. ಒಬ್ಬನ ಕಣ್ಣಿನಲ್ಲಿ ದುಃಖದಿಂದ ನೀರು ತುಂಬಿರುವುದು ಮುಂದೆ ಕಾಣುವುದಿಲ್ಲ. ಅದರಂತೆ ಅರ್ಜುನ ಈಗ ತನಗೆ ಬಂದ ವೈರಾಗ್ಯವೇ ಎಂದೆಂದಿಗೂ ಶಾಶ್ವತವಾಗಿರುವುದು ಎಂದು ಭಾವಿಸುವನು. ಒಬ್ಬನಿಗೆ ತನ್ನ ಸಮೀಪದ ಪ್ರಿಯನಾದ ನೆಂಟನೊಬ್ಬ ಕಾಲವಾಗಿರುವಾಗ, ಅವನಿಗೆ ಸೇರಿದ ಆಸ್ತಿಯನ್ನು ಹಂಚಿಕೊಳ್ಳುತ್ತಿರುವಾಗ, ಇವನ ಪಾಲಿಗೆ ಯಾವುದು ಬರಬೇಕೆಂದು ಕೇಳಲು ಬಂದರೆ, ಇವನು ಶೋಕಸಂತಪ್ತನಾಗಿರುವುದರಿಂದ “ಅಂತಹ ವ್ಯಕ್ತಿಯೇ ಕಾಲವಾದನಂತೆ, ಇನ್ನು ಅವನ ಆಸ್ತಿಯನ್ನು ಕಟ್ಟಿಕೊಂಡು ಏನು ಮಾಡುವುದು” ಎಂದು ಮಾತನಾಡಬಹುದು. ಆದರೆ ಶೋಕ ದೀರ್ಘಕಾಲ ಇರುವುದಿಲ್ಲ. ಅದು ಮಾಯವಾದ ಮೇಲೆ ಅಯ್ಯೋ ದುಃಖದ ಸಮಯದಲ್ಲಿ ಎಂತಹ ಅಚಾತುರ್ಯ ಮಾಡಿಕೊಂಡೆ ಎಂದು ಪರಿತಾಪ ಪಡುವನು. ಅರ್ಜುನ, ಜಯ ರಾಜ್ಯ ಇವನ್ನೆಲ್ಲ ಕಟ್ಟಿಕೊಂಡು ಏನು ಪ್ರಯೋಜನ? ಎನ್ನುವನು. ಜೀವನದಲ್ಲಿ ಬೆನ್ನು ತೋರಿಸಿದರೆ ಸಮಸ್ಯೆ ನಮ್ಮನ್ನು ಬಿಡುವುದಿಲ್ಲ. ಅದರೊಂದಿಗೆ ನಾವು ಹೋರಾಡಬೇಕು, ಮುಷ್ಟಿಯುದ್ಧ ಮಾಡಬೇಕು. ಆಗಲೇ ಅವುಗಳಿಂದ ಪಾರಾಗಬಹುದು. ಮಗುವಿಗೆ ಒಂದು ಲೆಕ್ಕವನ್ನು ಮಾಡಲು ಬರದೆ ಕೋಪದಿಂದ ಯಾವ ಸ್ಲೇಟಿನ ಮೇಲೆ ಆ ಲೆಕ್ಕವನ್ನು ಬರೆದಿರುವುದೋ ಅದನ್ನೇ ಎಸೆದು ಒಡೆದು ಹಾಕುವುದು. ಆದರೆ ಇದರಿಂದ ಲೆಕ್ಕದ ಉತ್ತರ ಬಂತೇನು?

\begin{verse}
ಯೇಷಾಮರ್ಥೇ ಕಾಂಕ್ಷಿತಂ ನೋ ರಾಜ್ಯಂ ಭೋಗಾಃ ಸುಖಾನಿ ಚ ।\\ತ ಇಮೇsವಸ್ಥಿತಾ ಯುದ್ಧೇ ಪ್ರಾಣಾಂಸ್ತ್ಯಕ್ತ್ವಾಧನಾನಿ ಚ \eng{॥ ೩೩ ॥}
\end{verse}

\begin{verse}
ಆಚಾರ್ಯಾಃ ಪಿತರಃ ಪುತ್ರಾಸ್ತಥೈವ ಚ ಪಿತಾಮಹಾಃ ।\\ಮಾತುಲಾಃ ಶ್ವಶುರಾಃ ಪೌತ್ರಾ ಶ್ಯಾಲಾಃ ಸಂಬಂಧಿನಸ್ತಥಾ \eng{॥ ೩೪ ॥}
\end{verse}

{\small ಯಾರಿಗಾಗಿ ನಮಗೆ ರಾಜ್ಯಭೋಗ ಸುಖಗಳು ಬೇಕೆನಿಸಬಹುದೊ ಅಂಥ ಇವರೇ ಪ್ರಾಣ ಮತ್ತು ಧನವನ್ನು ಬಿಟ್ಟು ಯುದ್ಧಕ್ಕೆ ನಿಂತಿರುವರು. ಗುರುಗಳು ಚಿಕ್ಕಪ್ಪ ದೊಡ್ಡಪ್ಪನವರು ಮಕ್ಕಳು ತಾಯಂದಿರು ಸೋದರ ಮಾವಂದಿರು ಮೊಮ್ಮಕ್ಕಳು ಭಾವಮೈದಂದಿರು ಮತ್ತು ಇತರ ನೆಂಟರುಗಳೆಲ್ಲ ಇರುವರು.}

ರಾಜ್ಯ ಭೋಗ ಸುಖ ಇವುಗಳನ್ನೆಲ್ಲ ಅನುಭವಿಸಬೇಕಾದರೆ ಸುತ್ತಮುತ್ತಲೂ ನೆಂಟರಿಷ್ಟರು ಪ್ರಜೆಗಳು ಇವರುಗಳೆಲ್ಲ ಇರಬೇಕು. ಅವರಿಗೆ ತೋರಿಸಿಕೊಂಡು ಆನಂದಿಸಬೇಕು. ಹಾಳೂರಿಗೆ ರಾಜನಾದರೆ ಬಂದ ಪ್ರಯೋಜನವೇನು? ನೋಡುವುದಕ್ಕೆ ಒಂದು ನರಪಿಳ್ಳೆ ಇಲ್ಲದೇ ಇದ್ದರೆ, ದೊಡ್ಡದೊಂದು ದರ್ಬಾರನ್ನು ಮಾಡಿದರೆ ಯಾವ ಸಂತೋಷ? ಕೌರವನಿಗೆ ಸಹಾಯಕ್ಕೆ ಬಂದ ವೀರಾಧಿವೀರರು ಸುಮ್ಮನೆ ಯುದ್ಧರಂಗದಿಂದ ಬೆನ್ನು ತೋರುವವರಲ್ಲ. ಅವರು ತಾವು ಸಾಯುವು ದಕ್ಕೆ ಮುಂಚೆ ಬೇಕಾದಷ್ಟು ಜನರನ್ನು ಕೊಂದು ತಾವು ಸಾಯುವರು. ಹೀಗಿರುವಾಗ ಒಂದು ವೇಳೆ ಜಯ ಪಾಂಡವರಿಗೆ ಲಭಿಸಿದರೂ ಅದು ಲಕ್ಷಾಂತರ ಜನರ ಸಾವಿನ ಮೇಲಿದೆ. ಅರ್ಜುನ ಇಲ್ಲಿ ಕೊಡುವ ಕಾರಣಗಳು ಕರುಣೆಯಿಂದ ಕೂಡಿರುವಂತೆ ಕಾಣುವುದು. ಆದರೆ ಅವನು ಇಲ್ಲಿ ಮಾಡು ತ್ತಿರುವ ತಪ್ಪು ತನ್ನ ಕರ್ತವ್ಯವನ್ನು ಮರೆತಿರುವುದು. ಕ್ಷತ್ರಿಯನಿಗೆ ಧರ್ಮಕ್ಕಾಗಿ ಯುದ್ಧ ಮಾಡು ವುದು ಮೊದಲು, ಉಳಿದವುಗಳೆಲ್ಲ ಗೌಣ. ಇಲ್ಲಿ ಗೌಣಕ್ಕೆ ಪ್ರಥಮ ಸ್ಥಾನವನ್ನು ಕೊಟ್ಟು ಮುಖ್ಯ ವಾಗಿರುವುದನ್ನೇ ಮರೆತಿರುವನು.

\begin{verse}
ಏತಾನ್ನ ಹಂತುಮಿಚ್ಛಾಮಿ ಘ್ನತೋsಪಿ ಮಧುಸೂದನ ।\\ಅಪಿ ತ್ರೈಲೋಕ್ಯರಾಜ್ಯಸ್ಯ ಹೇತೋಃ ಕಿಂ ನು ಮಹೀಕೃತೇ \eng{॥ ೩೫ ॥}
\end{verse}

{\small ಮಧುಸೂದನ, ಇವರು ನನ್ನನ್ನು ಕೊಂದರೂ (ಚಿಂತೆಯಿಲ್ಲ). ಮೂರು ಲೋಕದ ರಾಜ್ಯಗಳು ಬಂದರೂ ಇವರನ್ನು ನಾನು ಕೊಲ್ಲಲು ಇಚ್ಛಿಸುವುದಿಲ್ಲ. ಇನ್ನು ಈ ಭೂಮಿಗಾಗಿ ಹೇಳುವುದೇನಿದೆ?}

ಅರ್ಜುನ ಇಲ್ಲಿ ಮಾಡುತ್ತಿರುವ ತಪ್ಪೆ ಈ ಸಮಸ್ಯೆಯನ್ನು ಕೇವಲ ತನ್ನ ಲಾಭ ನಷ್ಟದ ದೃಷ್ಟಿಯಿಂದ ಮಾತ್ರ ನೋಡುತ್ತಿರುವುದು. ಅದನ್ನು ಮೀರಿದ ಒಂದು ದೃಷ್ಟಿ ಇದೆ. ಅಲ್ಲಿ ಇವನ ಲಾಭ ನಷ್ಟಗಳು ಗೌಣ. ದೇಶದಲ್ಲಿ ಸತ್ಯಕ್ಕೆ ಮತ್ತು ಧರ್ಮಕ್ಕೆ ಬೆಲೆ ಇದೆ ಎಂಬುದನ್ನು ಅವನು ತೋರಬೇಕು. ಹೀಗೆ ತೋರಬೇಕಾದರೆ ಯಾರು ಧರ್ಮಕ್ಕೆ ವಿರೋಧವಾಗಿ ಇರುವರೊ ಅವರಿಗೆ ಶಿಕ್ಷೆಯನ್ನು ಕೊಡಬೇಕು. ಶಿಕ್ಷೆ ಎರಡು ಕೆಲಸವನ್ನು ಮಾಡುವುದು. ಅನ್ಯಾಯ ಮಾಡಿದವನಿಗೆ ಶಿಕ್ಷೆ ಬಂದಾಗ ಅವನು ಪಶ್ಚಾತ್ತಾಪ ಪಟ್ಟು ತನ್ನನ್ನು ತಿದ್ದಿಕೊಳ್ಳಲು ಯತ್ನಿಸುವನು. ಇಂತಹವನಿಗೆ ವಿಧಿಸುವ ಶಿಕ್ಷೆಯಿಂದ ಜನರ ಮೇಲೂ ಒಂದು ಪರಿಣಾಮ ಉಂಟಾಗುವುದು. ಅನ್ಯಾಯದ ಹಾದಿ ಹಿಡಿಯಬಾರದು, ಹಿಡಿದರೆ ನಾವೂ ಕೂಡ ಅವರಂತೆಯೇ ವ್ಯಥೆ ಪಡಬೇಕಾಗುವುದು ಎಂದು ಗೊತ್ತಾಗುವುದು.

\begin{verse}
ನಿಹತ್ಯ ಧಾರ್ತರಾಷ್ಟ್ರಾನ್ನಃ ಕಾ ಪ್ರೀತಿಃ ಸ್ಯಾಜ್ಜನಾರ್ದನ ।\\ಪಾಪಮೇವಾಶ್ರಯೇದಸ್ಮಾನ್ ಹತ್ವೈತಾನಾತತಾಯಿನಃ \eng{॥ ೩೬ ॥}
\end{verse}

{\small ಜನಾರ್ದನ, ಕೌರವರ ಕಡೆಯವರನ್ನು ಕೊಂದರೆ ನಮಗೇನು ಸಂತೋಷವಾಗುವುದು? ಈ ಆತತಾಯಿಗಳನ್ನು ಕೊಂದರೆ ನಮಗೆ ಪಾಪವೇ ಬರುವುದು.}

ಅರ್ಜುನ ಇಲ್ಲಿ ಪಾಪದ ದೃಷ್ಟಿಯಿಂದ ಮಾತಾಡುವನು. ಗುರುಗಳು ಹಿರಿಯರು ಬ್ರಾಹ್ಮಣರು ಇವರನ್ನು ಕೊಂದರೆ ಪಾಪ ಬರುವುದೆಂದು ಸಾರುವುದು ನಮ್ಮ ಧರ್ಮ. ಆದರೆ ಅವರನ್ನು ಇಲ್ಲಿ ದಂಡಿಸದೆ ಬಿಟ್ಟರೆ ಅಧರ್ಮಕ್ಕೆ ಪ್ರೋತ್ಸಾಹ ಕೊಟ್ಟಂತೆ ಆಗುವುದು ಎಂಬುದನ್ನು ಅರ್ಜುನ ಮರೆಯುವನು. ಇಲ್ಲಿ ಕೌರವರು ಆತತಾಯಿಗಳು. ಆತತಾಯಿ ಎಂದರೆ ಮನೆಗೆ ಬೆಂಕಿ ಇಟ್ಟವರು, ವಿಷವನ್ನು ಹಾಕಿದವರು, ಶಸ್ತ್ರವನ್ನೆತ್ತಿ ಹೊಡೆಯುವುದಕ್ಕೆ ಬಂದಿರುವವರು, ಧನ ಭೂಮಿ ಮತ್ತು ಹೆಂಡತಿ ಇವರುಗಳನ್ನು ಅಪಹರಿಸಿರುವವರು. ಕೌರವರು ಇದನ್ನೆಲ್ಲ ಮಾಡಿರುವರು. ಇವರನ್ನು ಶಿಕ್ಷಿಸದೆ ಹೋದರೆ ಪಾಪ, ಶಿಕ್ಷಿಸಿದರೆ ಪಾಪವಿಲ್ಲ. ಅರ್ಜುನನ ಬುದ್ಧಿ ವಿಪರೀತವನ್ನು ನೋಡುತ್ತಿರು ವುದು. ತಪ್ಪನ್ನು ಸರಿಯೆಂದು ಭಾವಿಸುವನು, ಸರಿಯನ್ನು ತಪ್ಪೆಂದು ಭಾವಿಸುವನು.

ಇಂತಹ ಯಾತನೆಯನ್ನು ಕೊಡುವ ಕೆಲಸದಿಂದ ಏನು ಸಂತೋಷವಾಗುವುದು ಎಂದು ಕೇಳು ವನು. ಸಂತೋಷಕ್ಕಲ್ಲ ಯಾವಾಗಲೂ ಕೆಲಸ ಮಾಡುವುದು. ಕರ್ತವ್ಯ ಪರಾಯಣತೆಯ ದೃಷ್ಟಿಯಿಂದ ಅದನ್ನು ಮಾಡಬೇಕು. ಇಂತಹ ಪ್ರಸಂಗಗಳಲ್ಲಿ ನಮಗೆ ಸಂತೋಷವಾಗುವುದು ಬಹಳ ಅಪರೂಪ. ನ್ಯಾಯಾಧಿಪತಿ ಕೋರ್ಟಿನಲ್ಲಿ ಕುಳಿತುಕೊಂಡು ಬೆಳಗಿನಿಂದ ಸಂಜೆಯ ವರೆಗೆ ಸಮಾಜಕಂಟಕರಾದ ಹಲವರಿಗೆ ದಂಡನೆಯನ್ನು ವಿಧಿಸುತ್ತಿರುವನು. ಇದನ್ನು ಅವನು ಕೇವಲ ತನ್ನ ಸಂತೋಷ ದೃಷ್ಟಿ ಯಿಂದ ಮಾಡುವನೆ? ಇದು ಸಮಾಜಕ್ಕೆ ನಾನು ಮಾಡಬೇಕಾಗಿರುವ ಕರ್ತವ್ಯ ಎಂದು ಮಾತ್ರ ಮಾಡಲು ಸಾಧ್ಯ. ನಮ್ಮ ವೈಯಕ್ತಿಕ ಸುಖಸಂತೋಷವನ್ನು ಸಾರ್ವಜನಿಕ ಸುಖಕ್ಕಾಗಿ, ಬದಿಗೊಡ್ಡ ಬೇಕು. ಕ್ಷತ್ರಿಯ ಇದಕ್ಕಾಗಿ ತರಬೇತನ್ನು ತೆಗೆದುಕೊಂಡಿರುವನು. ಅರ್ಜುನ ಇದನ್ನು ಮರೆತಿರುವನು.

\begin{verse}
ತಸ್ಮಾನ್ನಾರ್ಹಾ ವಯಂ ಹಂತುಂ ಧಾರ್ತರಾಷ್ಟ್ರಾನ್ ಸ್ವಬಾಂಧವಾನ್ ।\\ಸ್ವಜನಂ ಹಿ ಕಥಂ ಹತ್ವಾ ಸುಖಿನಃ ಸ್ಯಾಮ ಮಾಧವ \eng{॥ ೩೭ ॥}
\end{verse}

{\small ಮಾಧವ, ಆದಕಾರಣ ಸ್ವಜನರಾದ ಕೌರವರನ್ನು ನಾವು ಕೊಲ್ಲುವುದು ಸರಿಯಲ್ಲ. ಏಕೆಂದರೆ ಸ್ವಜನರನ್ನು ಕೊಂದು ನಾವು ಹೇಗೆ ಸುಖಿಗಳಾದೇವು?}

ಸ್ವಜನರಾದ ತಪ್ಪಿತಸ್ಥರನ್ನು ನಾವು ಕೊಲ್ಲುವುದು ಸರಿಯಲ್ಲ ಎನ್ನುವನು. ಅದನ್ನು ಇನ್ನು ಬೇರೆ ಯಾರಾದರೂ ಮಾಡಿಕೊಂಡು ಹೋಗಲಿ. ಮನುಷ್ಯನಿಗೆ ಯಾವಾಗಲೂ ಇನ್ನೊಬ್ಬರಿಗೆ ಒಳ್ಳೆಯದನ್ನು ಹೇಳುವ ಆಸೆ. ಅಪ್ರಿಯವಾಗಿರುವುದು ಬಂದರೆ ಎಲ್ಲರೂ ಮತ್ತೊಬ್ಬರಿಗೆ ಆ ಕೆಲಸವನ್ನು ಅಂಟಿಸಿ ಬಿಡಲು ಯತ್ನಿಸುವರು. ಆದರೆ ಮಹಾಪುರುಷ ಬೇರೆ ಎರಕದಲ್ಲಿ ರೂಪುಗೊಂಡವನು. ಅವನು ಪ್ರಿಯ ಅಪ್ರಿಯಗಳನ್ನು ಗಮನಿಸುವುದಿಲ್ಲ. ಒಂದು ಕೆಲಸವನ್ನು ಮಾಡಬೇಕಾಗಿದೆ. ಅದನ್ನು ಮಾಡಿ ಹಾಕುವನು. ಇದರಿಂದ ತನಗೆ ವ್ಯಥೆಯಾದರೂ ಚಿಂತೆಯಿಲ್ಲ, ಇತರರು ಟೀಕಿಸಿದರೂ ಚಿಂತೆಯಿಲ್ಲ. ಕರ್ತವ್ಯನಿಷ್ಠೆಯೇ ಅವನ ಧರ್ಮ, ಅದಕ್ಕೆ ಏನನ್ನು ಬೇಕಾದರೂ ಬಲಿ ಕೊಡಲು ಸಿದ್ಧ.

\begin{verse}
ಯದ್ಯಪ್ಯೇತೇ ನ ಪಶ್ಯಂತಿ ಲೋಭೋಪಹತಚೇತಸಃ ।\\ಕುಲಕ್ಷಯಕೃತಂ ದೋಷಂ ಮಿತ್ರದ್ರೋಹೇ ಚ ಪಾತಕಮ್ \eng{॥ ೩೮ ॥}
\end{verse}

\begin{verse}
ಕಥಂ ನ ಜ್ಞೇಯಮಸ್ಮಾಭಿಃ ಪಾಪಾದಸ್ಮಾನ್ನಿವರ್ತಿತುಮ್ ।\\ಕುಲಕ್ಷಯಕೃತಂ ದೋಷಂ ಪ್ರಪಶ್ಯದ್ಭಿರ್ಜನಾರ್ದನ \eng{॥ ೩೯ ॥}
\end{verse}

{\small ಜನಾರ್ದನ, ಅತಿ ಆಸೆಯಿಂದ, ಅವಿವೇಕಿಗಳಾದ ಇವರು ಕುಲಕ್ಷಯದಿಂದಾಗುವ ದೋಷವನ್ನು ಮತ್ತು ಮಿತ್ರದ್ರೋಹದಲ್ಲಿರುವ ಪಾಪವನ್ನು ಕಾಣರು. ಕುಲನಾಶದಿಂದಾಗುವ ಕೇಡನ್ನು ಬಲ್ಲ ನಾವು ಈ ಪಾಪದಿಂದ ಹಿಂದೆ ಸರಿಯಬೇಕೆಂದು ಏತಕ್ಕೆ ಭಾವಿಸಬಾರದು?}

ಈ ಯುದ್ಧದಲ್ಲಿ ಒಬ್ಬರು ಇನ್ನೊಬ್ಬರನ್ನು ಕೊಂದು ಕ್ಷತ್ರಿಯ ಕುಲವೇ ನಾಶವಾಗುವುದೆಂದು ಅರ್ಜುನ ಭಾವಿಸುವನು. ಕ್ಷತ್ರಿಯ ಪ್ರಪಂಚದಲ್ಲಿರುವುದಕ್ಕೆ ಒಂದು ಕಾರಣವಿದೆ. ಯಾವಾಗ ಅವನು ಅದಕ್ಕೆ ವಿರೋಧವಾಗಿ ಹೋಗುವನೋ ಅಂತಹ ಕ್ಷತ್ರಿಯರು ಇದ್ದರೆಷ್ಟು, ಹೋದರೆಷ್ಟು ಸಮಾಜಕ್ಕೆ. ಅರ್ಜುನ, ನಾವಾದರೂ ಏತಕ್ಕೆ ಹಿಂದೆ ಸರಿಯಬಾರದು ಎಂದು ಕೇಳುವನು. ಹೀಗೆ ಹಿಂದೆ ಸರಿದರೆ ಎಂತಹ ಘೋರ ಪರಿಣಾಮ ಕಾದಿದೆ ಎಂಬುದನ್ನು ಸ್ವಲ್ಪವೂ ಅವನು ಗಮನಿಸುವುದಿಲ್ಲ. ಪಾಂಡವರು ಹಿಂದೆ ಸರಿದರೆ, ಸತ್ಯಕ್ಕೆ ಧರ್ಮಕ್ಕೆ ಇರುವ ಬೆಲೆಯೆಲ್ಲ ಹೋಗುವುದು. ಯಾರಿಗೆ ಪ್ರಪಂಚದಲ್ಲಿ ಹೆಚ್ಚು ಬಲವೋ ಅವನು ಮಾತ್ರ ಬದುಕಲು ಯೋಗ್ಯ ಎಂದಂತಾಗುವುದು. ದುಷ್ಟಪ್ರಾಣಿಗಳ ದಂಷ್ಟ್ರ ನಖ ದಾಡೆಗಳಿಗೆ ಬೆಲೆ. ಸತ್ಯ ಧರ್ಮ ನಿಃಸ್ವಾರ್ಥತೆ ಮುಂತಾದುವಕ್ಕೆ ಪ್ರಪಂಚದಲ್ಲಿ ಬೆಲೆಯೇ ಇಲ್ಲದಂತಾಗುವುದು.

\begin{verse}
ಕುಲಕ್ಷಯೇ ಪ್ರಣಶ್ಯಂತಿ ಕುಲಧರ್ಮಾ ಸನಾತನಾಃ ।\\ಧರ್ಮೇ ನಷ್ಟೇ ಕುಲಂ ಕೃತ್ಸ್ನಮಧರ್ಮೋsಭಿಭವತ್ಯುತ \eng{॥ ೪ಂ ॥}
\end{verse}

{\small ಕುಲಕ್ಷಯವಾದರೆ ಬಹುಕಾಲದಿಂದ ಬಂದಿರುವ ಕುಲಧರ್ಮ ನಾಶವಾಗುವುದು. ಧರ್ಮ ಇಲ್ಲದೇ ಇದ್ದರೆ ಕುಲವನ್ನೆಲ್ಲ ಅಧರ್ಮ ವ್ಯಾಪಿಸಿಕೊಳ್ಳುವುದು.}

ಕ್ಷತ್ರಿಯ ಸಮಾಜದಲ್ಲಿ ಎಲ್ಲಾ ಕುಲಧರ್ಮಗಳನ್ನು ರಕ್ಷಿಸುವನು. ವೀರ ಕ್ಷತ್ರಿಯನ ಬೇಲಿ ಬಲವಾಗಿದ್ದರೇನೆ ಬ್ರಾಹ್ಮಣ ಯಾವ ಕಂಟಕವೂ ಇಲ್ಲದೆ ಧ್ಯಾನ ಅಧ್ಯಯನ ತಪಸ್ಸಿನಲ್ಲಿ ನಿರತನಾಗಿರ ಬಹುದು. ವರ್ತಕ ಮತ್ತು ಶೂದ್ರರು ತಮ್ಮ ತಮ್ಮ ಕೆಲಸಕಾರ್ಯಗಳನ್ನು ಮಾಡಿಕೊಂಡು ಹೋಗುವರು. ಯಾವಾಗ ಕ್ಷತ್ರಿಯನ ಶಕ್ತಿ ಕ್ಷೀಣಿಸುವುದೋ ಆಗ ಸಮಾಜದಲ್ಲಿ ಬಿಗಿ ತಪ್ಪುವುದು.

\begin{verse}
ಅಧರ್ಮಾಭಿಭವಾತ್ ಕೃಷ್ಣ ಪ್ರದುಶ್ಯಂತಿ ಕುಲಸ್ತ್ರಿಯಃ ।\\ಸ್ತ್ರೀಷು ದುಷ್ಟಾಸು ವಾರ್ಷ್ಣೇಯ ಜಾಯತೇ ವರ್ಣಸಂಕರಃ \eng{॥ ೪೧ ॥}
\end{verse}

{\small ಕೃಷ್ಣ ಅಧರ್ಮ ವ್ಯಾಪಿಸಿಕೊಳ್ಳುವುದರಿಂದ ಕುಲಸ್ತ್ರೀಯರೆಲ್ಲ ಕೆಡುವರು. ವೃಷ್ಣಿ ಕುಲೋದ್ಭವನೆ, ಸ್ತ್ರೀಯರು ಕೆಟ್ಟರೆ ವರ್ಣಸಂಕರವುಂಟಾಗುವುದು.}

ಒಂದು ಭೀಕರ ಯುದ್ಧದಲ್ಲಿ ಅನೇಕ ಜನ ಪುರುಷರು ಯಾವಾಗ ನಾಶವಾಗುವರೋ ಆಗ ಸ್ತ್ರೀಯರಿಗೆ ಸಾಕಾದಷ್ಟು ರಕ್ಷಣೆ ದೊರಕುವುದಿಲ್ಲ. ಅವರು ಇತರರ ಪಾಲಾಗುವರು; ಇಲ್ಲವೆ ಜೀವನೋಪಾಯಕ್ಕಾಗಿ ಹಲವು ಉದ್ಯಮಗಳಿಗೆ ಕೈಹಾಕಬೇಕಾಗುವುದು. ಇದರಲ್ಲಿ ಹಲವು ಸಮಾಜಕ್ಕೆ ಘಾತುಕಗಳು. ಒಬ್ಬ ಪುರುಷ ಕೆಟ್ಟರೆ ಆ ಸಮಾಜಕ್ಕೆ ಅಷ್ಟೊಂದು ನಷ್ಟವಿಲ್ಲ. ಯಾವಾಗ ಒಬ್ಬ ಸ್ತ್ರೀ ಕೆಡುವಳೋ ಅದರ ಪರಿಣಾಮ ಸಮಾಜದ ಮೇಲೆ ಭೀಕರವಾಗುವುದು.

\begin{verse}
ಸಂಕರೋ ನರಕಾಯೈವ ಕುಲಘ್ನಾನಾಂ ಕುಲಸ್ಯ ಚ ।\\ಪತಂತಿ ಪಿತರೋ ಹ್ಯೇಷಾಂ ಲುಪ್ತಪಿಂಡೋದಕಕ್ರಿಯಾಃ \eng{॥ ೪೨ ॥}
\end{verse}

{\small ವರ್ಣಸಂಕರವು ಕುಲನಾಶ ಮಾಡಿದವರಿಗೆ ಮತ್ತು ಕುಲಕ್ಕೆ ನರಕವನ್ನುಂಟುಮಾಡುವುದು. ಏಕೆಂದರೆ ಇವರ ಪಿತೃಗಳು ಶ್ರಾದ್ಧತರ್ಪಣಗಳಿಲ್ಲದೆ (ಪಿತೃಲೋಕದಿಂದ ) ಕೆಳಗೆ ಬೀಳುವರು.}

ಹಿಂದುಗಳ ಸ್ಮೃತಿಯಲ್ಲಿ ಇವುಗಳನ್ನು ಹೇಳಿರುವರು. ಆದರೆ ಧರ್ಮ ರಕ್ಷಣೆಗಾಗಿ ಇವುಗಳು ಆದರೆ ಬಾಧಕವಿಲ್ಲ. ಉದ್ದೇಶವನ್ನು ಇಲ್ಲಿ ಗಮನಿಸಬೇಕಾಗುವುದು. ಅರ್ಜುನನ ಕಾರಣಗಳಲ್ಲಿರುವ ದೋಷವೇ ಇದು.

\begin{verse}
ದೋಷೈರೇತೈಃ ಕುಲಘ್ನಾನಾಂ ವರ್ಣಸಂಕರಕಾರಕೈಃ ।\\ಉತ್ಸಾದ್ಯಂತೇ ಜಾತಿಧರ್ಮಾಃ ಕುಲಧರ್ಮಾಶ್ಚ ಶಾಶ್ವತಾಃ \eng{॥ ೪೩ ॥}
\end{verse}

{\small ವರ್ಣಸಂಕರಕ್ಕೆ ಕಾರಣವಾದ ಕುಲನಾಶಕರ ದೋಷಗಳಿಂದ ಹಿಂದಿನಿಂದ ಬಂದಿರುವ ಜಾತಿಧರ್ಮಗಳು ಕುಲಧರ್ಮಗಳು ನಾಶವಾಗುವುವು.}

ವರ್ಣಸಂಕರದಿಂದ ಆಯಾ ವರ್ಣಕ್ಕೆ ಸೇರಿದ ಕರ್ತವ್ಯಗಳಿಗೆ ಲೋಪ ಬರುವುದು. ಬ್ರಾಹ್ಮಣ, ವೈಶ್ಯ, ಕ್ಷತ್ರಿಯ, ಶೂದ್ರರು ತಮ್ಮ ತಮ್ಮ ಕೆಲಸವನ್ನು ಬಿಟ್ಟು ಇತರರ ಕೆಲಸಕ್ಕೆ ಕೈಹಾಕುವರು. ಒಂದೇ ವರ್ಣದಲ್ಲಿ ಹಲವು ಕುಲ ಕಸುಬುಗಳನ್ನು ಮಾಡುತ್ತಿರುವ ಇತರರು ಕೂಡ ಹೀಗೆಯೇ ಮಾಡುವರು. ಇದರಿಂದ ಸಮಾಜದಲ್ಲೆಲ್ಲ ಅನಾಯಕತೆ ಪ್ರಾಪ್ತವಾಗುವುದು.

\begin{verse}
ಉತ್ಸನ್ನ ಕುಲಧರ್ಮಾಣಾಂ ಮನುಷ್ಯಾಣಾಂ ಜನಾರ್ದನ ।\\ನರಕೇ ನಿಯತಂ ವಾಸೋ ಭವತೀತ್ಯನುಶುಶ್ರುಮ \eng{॥ ೪೪ ॥}
\end{verse}

{\small ಜನಾರ್ದನ, ಕುಲಧರ್ಮವನ್ನು ನಾಶಮಾಡಿದ ಜನರಿಗೆ ತಪ್ಪದೆ ನರಕವಾಗುವುದು ಎಂದು ಕೇಳಿರುತ್ತೇವೆ.}

ಹಿಂದಿನ ಕಾಲದಲ್ಲಿ ಶಾಸ್ತ್ರದಲ್ಲಿ ಜನರಿಗೆ ತಮ್ಮ ತಮ್ಮ ಕೆಲಸವನ್ನು ಸರಿಯಾಗಿ ಮಾಡದೆ ಇದ್ದರೆ ನರಕವಾಗುವುದು ಎಂಬ ಭಯವನ್ನು ಇಡುತ್ತಿದ್ದರು. ಇದರಂತೆಯೇ ಇಲ್ಲಿ ಯಾರು ಚೆನ್ನಾಗಿ ಕೆಲಸ ಮಾಡುವರೋ ಅವರಿಗೆ ಸ್ವರ್ಗ ಪ್ರಾಪ್ತಿ ಎಂಬ ಆಸೆಯನ್ನು ಒಡ್ಡುತ್ತಿದ್ದರು. ಸಾಧಾರಣವಾಗಿ ಕೆಳಮಟ್ಟದ ಮನುಷ್ಯನಿಗೆ ನರಕದ ಭಯ ಮತ್ತು ಸ್ವರ್ಗಲೋಕದ ಸುಖ, ಇಲ್ಲಿ ಕೆಟ್ಟ ಕೆಲಸವನ್ನು ಮಾಡುವುದನ್ನು ತಡೆಯುವುದಕ್ಕೂ, ಒಳ್ಳೆಯ ಕೆಲಸವನ್ನು ಮಾಡಲು ಪ್ರೋತ್ಸಾಹಿಸುವುದಕ್ಕೂ ಇದನ್ನು ಮಾಡಿರುವರು. ಆದರೆ ಮನುಷ್ಯ ವಿಕಾಸದ ಏಣಿಯಲ್ಲಿ ಮುಂದುವರಿದರೆ ಹೊರಗಿನ ಆಸೆ ಭಯಗಳಿಂದ ಅವನು ಕರ್ತವ್ಯತತ್ಪರನಾಗುವುದಿಲ್ಲ. ಅವನನ್ನು ಹಾಗೆ ಮಾಡುವುದು ಅವನ ಪುಜುತ್ವ.

\begin{verse}
ಅಹೋ ಬತ ಮಹತ್ಪಾಪಂ ಕರ್ತುಂ ವ್ಯವಸಿತಾ ವಯಮ್ ।\\ಯದ್ರಾಜ್ಯಸುಖಲೋಭೇನ ಹಂತುಂ ಸ್ವಜನಮುದ್ಯತಾಃ \eng{॥ ೪೫ ॥}
\end{verse}

{\small ಅಯ್ಯೋ! ರಾಜ್ಯದ ಸುಖಲೋಭದಿಂದ ಸ್ವಜನರನ್ನು ಕೊಲ್ಲುವುದಕ್ಕೆ ಯತ್ನಿಸುತ್ತಿರುವೆವಲ್ಲ! ನಾವು ದೊಡ್ಡ ಪಾಪವನ್ನು ಮಾಡಲು ನಿಶ್ಚಯಿಸಿದಂತಾಗಿದೆ.}

ಸ್ವಜನರನ್ನು ಕೊಂದರೆ ಪಾಪವೆನ್ನುವರು. ಸ್ವಜನರನ್ನು ಕೊಲ್ಲುವಂತೆ ಮಾಡಿರುವುದಾವುದು? ಆಪ್ತ ಜನರು ಮಾಡಿದ ಅನ್ಯಾಯ ಅಧರ್ಮಗಳು. ಈ ಕೃತ್ಯಗಳನ್ನು ಸ್ವಜನರೇ ಮಾಡಲಿ ಪರ ಜನರೇ ಮಾಡಲಿ ಇಬ್ಬರೂ ದಂಡನೆಗೆ ಅರ್ಹರು. ಈ ಸೂಕ್ಷ್ಮ ದುಃಖತಾಡಿತ ಅರ್ಜುನನ ಬುದ್ಧಿಗೆ ಹೊಳೆಯದೆ ಇದೆ.

\begin{verse}
ಯದಿ ಮಾಮಪ್ರತೀಕಾರಮಶಸ್ತ್ರಂ ಶಸ್ತ್ರಪಾಣಯಃ ।\\ಧಾರ್ತರಾಷ್ಟ್ರಾ ರಣೇ ಹನ್ಯುಸ್ತನ್ಮೇ ಕ್ಷೇಮತರಂ ಭವೇತ್ \eng{॥ ೪೬ ॥}
\end{verse}

{\small ನಾನು ಶಸ್ತ್ರವನ್ನು ತ್ಯಜಿಸಿ ಯಾವ ಪ್ರತೀಕಾರವನ್ನೂ ಮಾಡದೆ ಇರುವೆನು. ಶಸ್ತ್ರಪಾಣಿಗಳಾದ ಕೌರವರು ನನ್ನನ್ನು ಕೊಂದರೆ ಅದೇ ಕ್ಷೇಮಕರವಾಗುವುದು.}

ಇಲ್ಲಿ ಅರ್ಜುನನು ಸಮಸ್ಯೆಯನ್ನು ಎದುರಿಸುವುದನ್ನು ಬಿಟ್ಟು ಸುಮ್ಮನೆ ಕುಳಿತುಕೊಳ್ಳಲು ಇಚ್ಛಿಸುವನು. ಹೀಗೆ ಇರುವಾಗ ಅವರೇನಾದರೂ ಇವನನ್ನು ಕೊಂದರೆ ಸಮಸ್ಯೆಯಿಂದ ಪಾರಾಗು ವೆನು ಎಂದು ಬಯಸುವನು. ಇದರಿಂದ ಅರ್ಜುನ ಅಧರ್ಮವನ್ನು ಎದುರಿಸಿದಂತೆ ಆಗುವುದಿಲ್ಲ. ಅದಕ್ಕೆ ಪ್ರೋತ್ಸಾಹ ಕೊಟ್ಟಂತೆ ಆಗುವುದು. ಅಧರ್ಮವನ್ನು ಮಾಡುವಂತೆ ಚಿತಾವಣೆ ಮಾಡುವುದೂ ಒಂದೇ; ಅದನ್ನು ಮಾಡುತ್ತಿರುವಾಗ ನಾವು ತೆಪ್ಪಗೆ ಇರುವುದೂ ಒಂದೇ ಎಂಬುದನ್ನು ಅರ್ಜುನ ಅರಿಯನು.

ಅರ್ಜುನ ಹೀಗೆ ಹೇಳಿದ ಮೇಲೆ ಸಂಜಯ ಮುಂದೆ ಬರುವಂತೆ ಮಾತನಾಡುವನು:

\begin{verse}
ಏವಮುಕ್ತ್ವಾರ್ಜುನಃ ಸಂಖ್ಯೇ ರಥೋಪಸ್ಥ ಉಪಾವಿಶತ್ ।\\ವಿಸೃಜ್ಯ ಸಶರಂ ಚಾಪಂ ಶೋಕಸಂವಿಗ್ನಮಾನಸಃ \eng{॥ ೪೭ ॥}
\end{verse}

{\small ಹೀಗೆಂದು ನುಡಿದ ಅರ್ಜುನ ಯುದ್ಧರಂಗದಲ್ಲಿ ದುಃಖದಿಂದ ಕಳವಳಪಡುತ್ತ ಬಿಲ್ಲುಬಾಣಗಳನ್ನು ಕೈಯಿಂದ ಬಿಟ್ಟು ರಥದ ಮುಂದುಗಡೆ ಕುಳಿತುಕೊಂಡನು.}

ಈ ಅಧ್ಯಾಯದ ಪ್ರಾರಂಭದಲ್ಲಿ ಎಷ್ಟೊಂದು ಉತ್ಸಾಹದಿಂದ ಅರ್ಜುನ ಶ್ರೀಕೃಷ್ಣನಿಗೆ ತನ್ನ ರಥವನ್ನು ಎರಡು ಸೈನ್ಯಗಳ ಮಧ್ಯೆ ನಿಲ್ಲಿಸು ಎಂದು ಹೇಳುವನು. ಕೊನೆಯಲ್ಲಿ ಈ ವ್ಯಾಮೋಹದಿಂದ ಬಂದ ದುಃಖದಿಂದ ತನ್ನ ಕರ್ತವ್ಯವನ್ನು ಮರೆತು ಬಿಲ್ಲು ಬಾಣಗಳನ್ನು ಕೈಯಿಂದ ಜಾರಿ ರಥದಲ್ಲಿ ತೆಪ್ಪಗೆ ಕುಳಿತುಕೊಂಡಿರುವುದನ್ನು ನೋಡುವೆವು. ಅವನು ಯುದ್ಧ ಮಾಡದೆ ಇರುವುದಕ್ಕೆ ಹಲವು ಕಾರಣಗಳನ್ನು ಕೊಡುವರು. ಇವನು ಕೊಡುವ ಕಾರಣಗಳೆಲ್ಲ ಅಲ್ಪ ದೃಷ್ಟಿಯಿಂದ ಪ್ರೇರಿತ ವಾದವುಗಳು. ಭೂಮದೃಷ್ಟಿಯಿಂದ ನೋಡುತ್ತಿಲ್ಲ. ಅಲ್ಪದೃಷ್ಟಿ ಸಂಕುಚಿತ. ಅದು ತಾತ್ಕಾಲಿಕ. ಸಮಸ್ಯೆಯನ್ನು ಮುಂದೂಡುತ್ತದೆಯೆ ಹೊರತು ಅದನ್ನು ಬಗೆಹರಿಸುವುದಿಲ್ಲ. ಅವನು ಕೊಡುವ ಕಾರಣಗಳೇ ಇವು: ಯುದ್ಧ ಮಾಡಿ ಹತ್ಯೆ ಮಾಡುವುದರಿಂದ ಶ್ರೇಯಸ್ಸು ಬರುವುದಿಲ್ಲ. ಇದರಿಂದ ಕುಲಕ್ಷಯವಾಗುವುದು, ಕುಲಧರ್ಮ ನಾಶವಾಗುವುದು, ಕುಲಸ್ತ್ರೀಯರು ಕೆಡುವರು. ವರ್ಣಸಂಕರ ವಾಗುವುದು. ಇದರಿಂದ ಪಿತೃಗಳಿಗೆ ಶ್ರಾದ್ಧತರ್ಪಣಾದಿಗಳು ನಿಂತುಹೋಗಿ ಅವರು ಪತಿತರಾಗು ವರು. ಇದನ್ನೆಲ್ಲ ಮಾಡಿದವರಿಗೆ ನರಕ ಪ್ರಾಪ್ತಿ. ಇದಾವುದನ್ನೂ ಮಾಡದೆ ಸುಮ್ಮನೆ ಕುಳಿತುಕೊಂಡರೆ ಹೆಚ್ಚು ಎಂದರೆ ಅವರು ಅರ್ಜುನನನ್ನು ಕೊಲ್ಲಬಹುದಷ್ಟೆ. ಇದರಿಂದ ಅರ್ಜುನನಿಗೇ ಒಳ್ಳೆಯದಾಗು ವುದು. ತಾನು ಅಪ್ರಿಯವಾದ ಕೆಲಸವನ್ನು ಮಾಡಿದಂತಾಗುವುದಿಲ್ಲ.

ಇದೇ ಅರ್ಜುನ ವಿಷಾದಯೋಗವೆಂಬ ಭಗವದ್ಗೀತೆಯ ಮೊದಲನೆಯ ಅಧ್ಯಾಯ. ಹಲವಾರು ಯುದ್ಧಗಳನ್ನು ಮಾಡಿ ಲೋಕೈಕ ವೀರನೆಂಬ ಬಿರುದನ್ನು ಗಳಿಸಿದ ಅರ್ಜುನನ ಮನಸ್ಸು ಸುಂಟರಗಾಳಿ ಯಲ್ಲಿ ತರಗೆಲೆ ಸಿಕ್ಕಿಕೊಂಡು ಸುತ್ತುತ್ತಿರುವಂತೆ ಆಗಿದೆ. ಈ ಪರಿಸ್ಥಿತಿಯನ್ನು ನೋಡುತ್ತಿರುವವನು ಶ್ರೀಕೃಷ್ಣ. ಅವನನ್ನು ಕಾರ್ಯೋನ್ಮುಖನನ್ನಾಗಿ ಮಾಡುವುದಕ್ಕೆ ತನ್ನ ಬೋಧನೆಯ ಸಂಜೀವಿನಿಯನ್ನು ಕೊಡಬೇಕಾಗಿದೆ.

