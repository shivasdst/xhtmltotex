
\chapter{ರಾಜವಿದ್ಯಾ ರಾಜಗುಹ್ಯಯೋಗ}

ಶ್ರೀಕೃಷ್ಣ ಅರ್ಜುನನಿಗೆ ಹೇಳುತ್ತಾನೆ:

\begin{verse}
ಇದಂ ತು ತೇ ಗುಹ್ಯತಮಂ ಪ್ರವಕ್ಷ್ಯಾಮ್ಯನಸೂಯವೇ ।\\ಜ್ಞಾನಂ ವಿಜ್ಞಾನಸಹಿತಂ ಯಜ್ಞಾತ್ವಾ ಮೋಕ್ಷ್ಯಸೇಽಶುಭಾತ್ \versenum{॥ ೧ ॥}
\end{verse}

{\small ಯಾವುದನ್ನು ತಿಳಿದುಕೊಂಡು ಸಂಸಾರದಿಂದ ಮುಕ್ತನಾಗುವೆಯೊ ಅಂತಹ ಗುಹ್ಯತಮವಾದ ವಿಜ್ಞಾನದಿಂದ ಕೂಡಿದ ಜ್ಞಾನವನ್ನು ಅಸೂಯಾರಹಿತನಾದ ನಿನಗೆ ಹೇಳುತ್ತೇನೆ.}

ಸಂಸಾರದಿಂದ ಮುಕ್ತನಾಗಬೇಕಾದರೆ ಭಗವಂತನನ್ನು ಸೇರಬೇಕು. ಈ ಪ್ರಪಂಚದ ಆಕರ್ಷಣೆಯ ಬಲೆಗೆ ಬೀಳಕೂಡದು. ಯಾವಾಗ ಒಮ್ಮೆ ಅದರ ಆಕರ್ಷಣೆಯ ಬಲೆಗೆ ಬೀಳುತ್ತೇವೆಯೋ ಪುನಃ ಪುನಃ ಅದರ ಬಲೆಗೆ ಬೀಳಬೇಕಾಗುವುದು. ಅದರಿಂದ ತಪ್ಪಿಸಿಕೊಂಡು ಹೋಗುವುದಕ್ಕೆ ಸಾಧ್ಯವಿಲ್ಲ. ನಮ್ಮ ಆಸೆಯನ್ನು ತೃಪ್ತಿಪಡಿಸಿ ಯಾರೂ ಅದರಿಂದ ಪಾರಾಗಿಲ್ಲ. ಆಸೆಗೆ ಒಂದು ಅಂತ್ಯವಿಲ್ಲ. ಅದನ್ನು ಒಂದು ಪ್ರಮಾಣದಲ್ಲಿ ತೃಪ್ತಿಪಡಿಸಿದರೆ ಮತ್ತೊಂದು ಪ್ರಮಾಣದಲ್ಲಿ ವೃದ್ಧಿಯಾಗುವುದು. ಪ್ರತಿಯೊಂದು ಸಲ ಅದನ್ನು ತೃಪ್ತಿ ಪಡಿಸಿದಾಗಲೂ ಇದೇ ಕೊನೆಯೆಂದು ತೃಪ್ತಿಪಡಿಸುವೆವು. ಆದರೆ ಇಂತಹ ಕೊನೆಗಳೆಷ್ಟೋ ಆಗಿ ಹೋಗಿವೆ. ಸುಮ್ಮನೆ ಆಸೆಯನ್ನು ಬಿಡುವುದಕ್ಕೆ ಆಗುವುದಿಲ್ಲ. ಒಂದನ್ನು ಬಿಡಬೇಕು, ಮತ್ತೊಂದನ್ನು ಹಿಡಿಯಬೇಕು. ಈ ಪ್ರಪಂಚವನ್ನು ಬಿಡಬೇಕು, ಆದರೆ ಜೊತೆ ಜೊತೆಯಲ್ಲಿಯೇ ದೇವರನ್ನು ಬಿಗಿಯಾಗಿ ಹಿಡಿಯಬೇಕು. ಈ ಪ್ರಪಂಚದ ಆಕರ್ಷಣೆಯನ್ನು ಬಿಟ್ಟು ದೇವರನ್ನು ಹಿಡಿಯದೇ ಇದ್ದರೆ ಬಹಳ ಕಾಲ ನಾವು ಇಂದ್ರಿಯ ವಸ್ತುಗಳಿಂದ ದೂರ ಇರಲು ಸಾಧ್ಯವಿಲ್ಲ. ಪ್ರಪಂಚದ ಬಿರುಗಾಳಿ ಬೀಸಿದೊಡನೆಯೆ ನಾವು ಕೊಚ್ಚಿಕೊಂಡು ಹೋಗುತ್ತೇವೆ. ಅದಕ್ಕಾಗಿಯೆ ದೇವರೆಂಬ ದೊಡ್ಡ ಆಧಾರವನ್ನು ಹಿಡಿದುಕೊಂಡು ಇರಬೇಕು. ಈ ಪ್ರಪಂಚದಲ್ಲಿ ದೇವರೆಂಬ ಆಧಾರಸ್ತಂಭ ಒಂದೇ ಕುಸಿದು ಬೀಳದೆ ಇರುವುದು. ಉಳಿದುದೆಲ್ಲಾ ನಮ್ಮನ್ನು ಪರೀಕ್ಷಾ ಸಮಯದಲ್ಲಿ ಕೈಬಿಡುವುದು ಎಂಬುದನ್ನು ತಿಳಿದುಕೊಂಡು ಅವನ ಕಡೆ ಮನಸ್ಸನ್ನು ಹಾಕಬೇಕು.

ಇದು ಗುಹ್ಯತಮವಾದ ವಿದ್ಯೆ. ತುಂಬಾ ಸೂಕ್ಷ್ಮವಾದದ್ದು ಮತ್ತು ಗಹನವಾದದ್ದು. ಇದನ್ನು ಗ್ರಹಿಸುವುದು ತುಂಬಾ ಕಷ್ಟ. ನಮ್ಮ ಇಂದ್ರಿಯ ಮತ್ತು ಮನಸ್ಸು ಯಾವಾಗಲೂ ನಮ್ಮಿಂದ ಹೊರಗೆ ಇರುವ ದೃಶ್ಯವಸ್ತುಗಳನ್ನು ತಿಳಿದುಕೊಳ್ಳುವುದಕ್ಕೆ ತರಬೇತಾಗಿದೆ. ಅಧ್ಯಾತ್ಮ ವಿದ್ಯೆಯಾದರೊ ಹೊರಗೆ ಹೋಗುವುದಲ್ಲ, ತನ್ನೊಳಗೇ ಮುಳುಗಬೇಕು, ತನ್ನನ್ನೇ ತಿಳಿದುಕೊಳ್ಳಬೇಕು. ಪರಮಾತ್ಮ ಎಲ್ಲೋ ಹೊರಗೆ ಇರುವವನಲ್ಲ. ಹೊರಗಿಗಿಂತ ಹೆಚ್ಚಾಗಿ ನಮ್ಮೊಳಗೆ ಇರುವನು. ನಮಗೆ ತುಂಬಾ ಸಮೀಪವಾಗಿರುವುದು ಹೊರಗಿನದಲ್ಲ. ಮೊದಲು ನಾನು, ಆಮೇಲೆ ಪ್ರಪಂಚ. ನನ್ನ ಹಿಂದೆಯೇ ನನಗೆ ಜೀವಾಳವಾಗಿರುವವನೇ ಪರಮಾತ್ಮ. ಇಲ್ಲಿ ಗುಹ್ಯ ಎಂಬುದನ್ನು ಯಾವುದೊ ಒಂದು ರಹಸ್ಯ, ಇತರರಿಗೆ ಇದನ್ನು ಹೇಳಬಾರದು, ಹೇಳಿದರೆ ಅದರ ಪ್ರಯೋಜನ ಹೊರಟುಹೋಗುವುದು ಎಂಬ ದೃಷ್ಟಿಯಲ್ಲಿ ಹೇಳಿಲ್ಲ. ಇದು ಸೂಕ್ಷ್ಮಾತಿಸೂಕ್ಷ್ಮ ಎಂಬ ದೃಷ್ಟಿಯಿಂದ ಹೇಳಿದೆ.

ಇದು ವಿಜ್ಞಾನದಿಂದ ಕೂಡಿದ ಜ್ಞಾನ. ಈ ಜ್ಞಾನಕ್ಕೆ ತಳಪಾಯವಾಗಿ ವಿಜ್ಞಾನವಿದೆ, ಎಂದರೆ ಅನುಭವ ಇದೆ. ಅನುಭವದೂರವಾದ ಬರಿಯ ಊಹೆ ಅಥವಾ ಕಲ್ಪನೆಯಲ್ಲ. ಇದನ್ನು ಯೋಗಿಗಳು ತಮ್ಮ ಹೃದಯದ ಪ್ರಯೋಗಶಾಲೆಯಲ್ಲಿ ಮೊದಲು ಕಂಡುಹಿಡಿದರು. ಅನಂತರ ಅದನ್ನು ಇತರರಿಗೆ ಹೇಳಿದರು. ನರೇಂದ್ರ ಶ್ರೀರಾಮಕೃಷ್ಣರನ್ನು, ನೀವು ದೇವರನ್ನು ಕಂಡಿರುವಿರಾ ಎಂದು ಪ್ರಶ್ನಿಸಿದಾಗ ಅವರು ಹೌದು, ನಾನು ಕಂಡಿರುವೆ, ನಾನು ನಿನ್ನನ್ನು ನೋಡುವುದಕ್ಕಿಂತ ಸ್ಪಷ್ಟವಾಗಿ ನೋಡುತ್ತಿರುವೆ ಎಂದು ಹೇಳಿ, ಅನಂತರ ನರೇಂದ್ರನಿಗೆ ಹೀಗೆ ಕೇಳುತ್ತಾರೆ: ಆದರೆ ಎಷ್ಟು ಜನರಿಗೆ ದೇವರು ಬೇಕು ಹೇಳು. ಪ್ರಪಂಚದಲ್ಲಿ ಮನುಷ್ಯನಿಗೆ ದೇವರಲ್ಲದ ಇತರ ವಸ್ತುಗಳೆಲ್ಲ ಬೇಕು. ಅದಕ್ಕಾಗಿ ಅವನು ಕಣ್ಣೀರಿನ ಕೋಡಿಯಲ್ಲಿ ಈಜಿಕೊಂಡು ಹೋಗಲು ಸಿದ್ಧನಾಗಿರುವನು. ಆದರೆ ದೇವರಿಗೆ ಎಷ್ಟು ಜನ ಹೀಗೆ ಮಾಡಲು ಸಿದ್ಧವಾಗಿರುವರು?

ಅಸೂಯಾರಹಿತನಾದ ಅರ್ಜುನನಿಗೆ ಹೇಳುತ್ತೇನೆ ಎನ್ನುವನು ಶ್ರೀಕೃಷ್ಣ. ಶಿಷ್ಯನಲ್ಲಿ ಅಸೂಯೆಯ ಸಣ್ಣತನ ಇರಕೂಡದು. ಅಸೂಯೆ ಇದ್ದರೆ ಭಗವಂತನ ಸಮೀಪಕ್ಕೆ ಹೋಗಲಾರೆವು. ಯಾರ ಮೇಲೆ ನಾವು ಅಸೂಯೆ ಪಡುತ್ತೇವೆಯೊ ಆ ಗೂಟಕ್ಕೇ ನಮ್ಮನ್ನು ಕಟ್ಟಿಹಾಕುವುದು. ಅದರಿಂದ ತಪ್ಪಿಸಿ ಕೊಂಡು ಹೋಗುವುದಕ್ಕೇ ಕಷ್ಟವಾಗುವುದು. ಒಳ್ಳೆಯದು ಎಲ್ಲಿದ್ದರೂ ಅದನ್ನು ನೋಡಿ ಮೆಚ್ಚಬೇಕು, ತಲೆದೂಗಬೇಕು. ಹೊಟ್ಟೆಕಿಚ್ಚು ಪಡಬಾರದು. ಅಸೂಯೆಯ ದಳ್ಳುರಿ ಒಳಗೆ ಉರಿಯುತ್ತಿದ್ದರೆ, ಹೊರಗೆ ಭಕ್ತನೆಂದು ನಟನೆ ಮಾಡುವುದು ಬಹಳ ಕಾಲ ನಡೆಯುವುದಿಲ್ಲ. ಒಂದು ಬೇಕಾದರೆ ನಾವು ಮತ್ತೊಂದನ್ನು ಬಿಡಲು ಸಿದ್ಧವಾಗಿರಬೇಕು. ಅಸೂಯೆ ಇಲ್ಲದ ನೆಲದಲ್ಲೇ ಭಕ್ತಿಬೀಜ ಚೆನ್ನಾಗಿ ಮೊಳೆಯುವುದು.

\begin{verse}
ರಾಜವಿದ್ಯಾ ರಾಜಗುಹ್ಯಂ ಪವಿತ್ರಮಿದಮುತ್ತಮಮ್ ।\\ಪ್ರತ್ಯಕ್ಷಾವಗಮಂ ಧರ್ಮ್ಯಂ ಸುಸುಖಂ ಕರ್ತುಮವ್ಯಯಮ್ \versenum{॥ ೨ ॥}
\end{verse}

{\small ಇದು ರಾಜವಿದ್ಯೆ. ಇದು ಗೂಢವಾಗಿರುವುದಕ್ಕೆಲ್ಲ ರಾಜ. ಈ ವಿದ್ಯೆ ಪವಿತ್ರ, ಪ್ರತ್ಯಕ್ಷ ಅನುಭವಕ್ಕೆ ಯೋಗ್ಯ. ಧರ್ಮದಿಂದ ಕೂಡಿರುವುದು. ಆಚರಿಸುವುದಕ್ಕೆ ಸುಲಭ ಮತ್ತು ಅವಿನಾಶಿಯಾದದ್ದು.}

ಇದು ವಿದ್ಯೆಗಳಲ್ಲಿ ರಾಜ. ಉಳಿದ ವಿದ್ಯೆಗಳೆಲ್ಲ ಅಲ್ಪ; ಈ ಪ್ರಪಂಚದ ಚೂರು ಪಾರು ವಸ್ತುಗಳನ್ನು ತಿಳಿದುಕೊಳ್ಳುವುದು. ವಸ್ತು ಯಾವುದರಿಂದ ಆಗಿದೆ, ಶಕ್ತಿ ಯಾವುದರಿಂದ ಆಗಿದೆ, ವಸ್ತುವಿಗೂ ಶಕ್ತಿಗೂ ಇರುವ ಸಂಬಂಧಗಳೇನು ಮುಂತಾದುವುಗಳನ್ನು ಅರಿಯಲೆತ್ನಿಸುವುದು ಭೌತಿಕ ವಿದ್ಯೆ. ಉಳಿದವುಗಳಾದರೊ ಮಾನವ, ಅವನ ಸಮಾಜ, ಅವನ ನೀತಿ, ಅವನ ಇತಿಹಾಸ, ಅವನ ಮನಸ್ಸು ಮುಂತಾದುವುಗಳನ್ನೇ ಕುರಿತದ್ದು. ಆದರೆ ಅಧ್ಯಾತ್ಮವಾದರೊ ಇವುಗಳೆಲ್ಲವನ್ನೂ ಮೀರಿ ದುದು. ಇದು ಪರಾವಿದ್ಯೆ. ಉಳಿದವುಗಳೆಲ್ಲ ಅಪರಾ ವಿದ್ಯೆ. ಯಾವಾಗ ಅಧ್ಯಾತ್ಮ ವಿದ್ಯೆಯ ಮೂಲಕ ಒಬ್ಬ ಪರಮಾತ್ಮನನ್ನು ಅರಿಯುವನೊ ಅವನು ಎಲ್ಲವನ್ನೂ ತಿಳಿದುಕೊಳ್ಳುವನು, ಇನ್ನು ಅವನಿಗೆ ತಿಳಿದುಕೊಳ್ಳಬೇಕಾಗಿರುವುದು ಏನೂ ಇರುವುದಿಲ್ಲ.

ಇದು ರಹಸ್ಯವಾಗಿರುವ ವಸ್ತುಗಳಲ್ಲೆಲ್ಲಾ ರಾಜ ಎಂದರೆ ಶ್ರೇಷ್ಠವಾದುದು. ಈ ಪ್ರಪಂಚದಲ್ಲಿ ಎಷ್ಟೋ ಭೌತಿಕ ನಿಯಮಗಳು ನಿಮಗೆ ಕಾಣದೆ ಪ್ರಕೃತಿಯ ಭಂಡಾರದಲ್ಲಿ ಹುದುಗಿಕೊಂಡಿವೆ. ಕೆಲವು ವೇಳೆ ಅವು ಮನುಷ್ಯನ ಅರಿವಿಗೆ ಬರುವುವು ಅಥವಾ ಮನುಷ್ಯ ಅವನ್ನು ಎಡವುತ್ತಾನೆ ಎಂದು ಬೇಕಾದರೂ ಹೇಳಬಹುದು. ಸಮುದ್ರದಲ್ಲಿ ಹಲವಾರು ಅನರ್ಘ್ಯ ವಸ್ತುಗಳ ಜೊತೆಗೆ ನೋಡಲು ಚೆನ್ನಾಗಿರುವ ಬೆಲೆ ಇಲ್ಲದ ಕಪ್ಪೆಚಿಪ್ಪುಗಳು ಕೂಡ ಇರುವುವು. ಅವು ದಡದ ಮೇಲೆ ಬಿದ್ದಾಗ ಹುಡುಗರು ಆ ಕಪ್ಪೆಚಿಪ್ಪುಗಳನ್ನು ಆಯ್ದುಕೊಂಡು ಅವುಗಳ ಆಕಾರ ಮತ್ತು ಬಣ್ಣಕ್ಕೆ ಮರುಳಾಗು ತ್ತಾರೆ. ಆದರೆ ಅನರ್ಘ್ಯವಾದ ಮುತ್ತು ರತ್ನಗಳು ಕಡಲಾಳದಲ್ಲಿ ಹುದುಗಿವೆ. ಅದು ಸುಮ್ಮನೆ ತೀರದ ಮೇಲೆ ಬಂದು ಬೀಳುವುದಿಲ್ಲ. ಅವನ್ನು ತೆಗೆಯುವುದಕ್ಕೆ ನಾವು ಕಷ್ಟಪಟ್ಟು ಮುಳುಗಬೇಕು. ಅದರಂತೆಯೇ ಆಧ್ಯಾತ್ಮಿಕ ಸತ್ಯಗಳು ಜೀವನದ ಅನುಭವದ ಕಡಲಿನಾಳದಲ್ಲಿ ಇವೆ. ನಾವು ಕಷ್ಟಪಟ್ಟು ಸಾಧನೆ ಮಾಡಿದಲ್ಲದೆ ನಮಗೆ ಅವು ದೊರಕುವುದಿಲ್ಲ.

ಇದರಷ್ಟು ನಮ್ಮನ್ನು ಪವಿತ್ರ ಮಾಡುವುದು ಮತ್ತೊಂದು ಇಲ್ಲ. ಉಳಿದವುಗಳೆಲ್ಲ ನಮ್ಮನ್ನು ತಾತ್ಕಾಲಿಕವಾಗಿ ಮಾತ್ರ ಶುದ್ಧಿ ಮಾಡುವುದು. ನಾವು ಚೆನ್ನಾಗಿ ಸೋಪು ಹಾಕಿಕೊಂಡು ಸ್ನಾನ ಮಾಡಿದರೆ ಕೆಲವು ಕಾಲ ಮಾತ್ರ ನಮ್ಮ ದೇಹ ಶುದ್ಧವಾಗಿರುವುದು. ಕ್ರಮೇಣ ಬೆವರು ಬಂದು ದೇಹವನ್ನೆಲ್ಲ ಆವರಿಸುವುದು. ಯಾವುದಾದರೊಂದು ಒಳ್ಳೆಯ ವಿಷಯವನ್ನು ಕೇಳಿದರೆ ತಾತ್ಕಾಲಿಕ ವಾಗಿ ಮಾತ್ರ ನಮ್ಮ ಮನಸ್ಸಿನ ಮೇಲೆ ಅದರ ಪ್ರಭಾವ ಇರುವುದು. ಅನಂತರ ಮನಸ್ಸಿನ ಮೈಲಿಗೆ ಎದ್ದು ಎಲ್ಲವನ್ನೂ ಆವರಿಸಿಕೊಳ್ಳುವುದು. ಆದರೆ ಒಮ್ಮೆ ಭಗವತ್ ದರ್ಶನದಿಂದ ಮನುಷ್ಯ ಪವಿತ್ರನಾದರೆ ಅವನಿನ್ನುಮೇಲೆ ಅಜ್ಞಾನಕ್ಕೆ ಸಿಕ್ಕುವುದಿಲ್ಲ. ಜನ್ಮಜನ್ಮಾಂತರಗಳಿಂದ ಕೂಡಿಹಾಕಿ ಕೊಂಡಿರುವ ಅಜ್ಞಾನವೆಲ್ಲ ಒಂದೇ ಸಲ ದಗ್ಧವಾಗಿ ಹೋಗುವುದು.

ಇದನ್ನು ಪ್ರತ್ಯಕ್ಷ ಅನುಭವಿಸಲು ಸಾಧ್ಯ. ಹೇಗೆ ವಿಜ್ಞಾನ ಪ್ರಪಂಚದಲ್ಲಿ ಪ್ರಯೋಗಶಾಲೆಗೆ ಹೋಗಿ ಕೆಲವು ಪ್ರಯೋಗಗಳನ್ನು ಮಾಡಿದರೆ ನಮಗೆ ಕೆಲವು ಅನುಭವಗಳು ಎದುರಿಗೆ ಆಗುತ್ತವೆಯೋ ಹಾಗೆಯೇ ಇಲ್ಲಿಯೂ ನಾವು ಸರಿಯಾಗಿ ಅನುಷ್ಠಾನ ಮಾಡಿದರೆ ಆಂತರಿಕ ಅನುಭವ ಆಗುವುದು. ಅನುಭವಕ್ಕೆ ಕಷ್ಟಪಡಬೇಕು. ಸುಮ್ಮನೆ ಅನುಭವ ಆಗಬೇಕು ಎಂದು ಇಚ್ಛಿಸಿದರೆ ಸಾಲದು. ಕಟ್ಟಿಗೆಯಲ್ಲಿ ಬೆಂಕಿ ಇದೆ ಎಂದು ನಮಗೆ ಗೊತ್ತಿದೆ. ಸುಮ್ಮನೆ ಬೆಂಕಿ ಇದೆ ಎಂದು ಹೇಳುತ್ತ ಕುಳಿತರೆ ಅಲ್ಲಿ ಬೆಂಕಿ ಆಗುವುದಿಲ್ಲ. ಅದಕ್ಕೆ ಬೆಂಕಿ ತಗುಲಿಸಬೇಕು. ಆಗಲೇ ಒಳಗೆ ಸುಪ್ತವಾಗಿರುವ ಬೆಂಕಿ ಉರಿಯುತ್ತ ಧಗಧಗಿಸಿ ಮೇಲೇಳುವುದು.

ಹಲವು ವಿಧವಾದ ಪರೋಪಕಾರಗಳನ್ನು ಮಾಡಿ ನಾವು ಸಂಪಾದಿಸುವ ಧರ್ಮ ಅಲ್ಪವಾದದ್ದು. ಆಧ್ಯಾತ್ಮಿಕ ಜೀವನದಲ್ಲಿ ನಾವು ಸಾಧನೆ ಮಾಡಿ ಪಡೆಯುವ ಅನುಭವದಿಂದ ಕೃತಕೃತ್ಯರಾಗಿ ಹೋಗುವೆವು. ಇದನ್ನು ಅನುಷ್ಠಾನ ಮಾಡುವುದು ಸುಲಭ ಎಂದು ಶ್ರೀಕೃಷ್ಣ ಹೇಳುವನು. ಸುಲಭ ಎಂದರೆ ಏನೂ ಪ್ರಯತ್ನವಿಲ್ಲದೆ ಅದು ಸಿಕ್ಕುವುದು ಎಂದಲ್ಲ. ಜೀವನದಲ್ಲಿ ಯಾವುದು ಬೇಕಾದರೂ ನಾವು ಕಷ್ಟಪಡಬೇಕು. ವಿದ್ಯಾರ್ಥಿ ಪರೀಕ್ಷೆಯಲ್ಲಿ ಪಾಸು ಮಾಡಬೇಕಾದರೆ ಚೆನ್ನಾಗಿ ಕಷ್ಟಪಟ್ಟು ಓದಬೇಕು. ಸಂಗೀತ ಕಲಿಯಬೇಕಾದರೆ ವರ್ಷಾನುಗಟ್ಟಲೆ ಚಳಿಗಾಲವೆನ್ನದೆ ಬಿಸಿಲುಗಾಲವೆನ್ನದೆ ಅಭ್ಯಾಸ ಮಾಡುತ್ತಿರಬೇಕು. ಏನಿಲ್ಲ ಒಬ್ಬ ದೊಡ್ಡ ಪೈಲ್ವಾನನಾಗಬೇಕಾದರೂ ಹಲವು ವರುಷಗಳ ಅಭ್ಯಾಸ ಬೇಕು. ಬರೀ ಹೊಟ್ಟೆಪಾಡಿಗೆ ಇಷ್ಟೊಂದು ಕಷ್ಟಪಡಬೇಕು ಜೀವನದಲ್ಲಿ. ಇದರೊಂದಿಗೆ ಹೋಲಿಸಿ ನೋಡಿದರೆ ದೇವರನ್ನು ಪಡೆಯುವುದು ಸುಲಭ. ಪ್ರಾಪಂಚಿಕ ವಸ್ತುಗಳನ್ನು ಪಡೆಯು ವುದಕ್ಕೆ ಎಷ್ಟು ಕಷ್ಟ ಪಡುತ್ತೇವೆಯೋ, ಅಲ್ಲಿ ಎಷ್ಟು ಛಲವನ್ನು ವ್ಯಕ್ತಪಡಿಸುವೆವೊ ಅಷ್ಟನ್ನು ವ್ಯಕ್ತಪಡಿಸಿದರೆ ಸಾಕು ದೇವರ ಕಡೆಗೆ. ಸಾಂಸಾರಿಕ ವಸ್ತುಗಳನ್ನು ಪಡೆಯಲು ನಾವು ಎಷ್ಟಾದರೂ ಕಷ್ಟಪಡುತ್ತೇವೆ. ಅಲ್ಲಿ ಎಷ್ಟು ನಿರಾಶೆ ಆದರೂ ನಾವೇನೂ ಆಸೆಯನ್ನು ಬಿಡುವುದಿಲ್ಲ. ಎಷ್ಟು ಸಲ ಕುಟುಕಿಸಿಕೊಂಡರೂ ಪುನಃ ಕುಟುಕಿಸಿಕೊಳ್ಳಲು ಸಿದ್ಧ. ಒದೆಸಿಕೊಂಡರೂ ಪುನಃ ಒದೆಸಿಕೊಳ್ಳಲು ಸಿದ್ಧ. ಇದಕ್ಕಿಂತ ಜಾಸ್ತಿ ಬೇಕಾಗಿಲ್ಲ. ಇಷ್ಟೇ ಛಲವನ್ನು, ಭಂಡತನವನ್ನು ದೇವರೆಡೆಗೆ ಹರಿಸಿದರೆ ಸಾಕು. ಸುಲಭ ಎಂದರೆ ಇದೇ ಅರ್ಥ.

ಈ ಅನುಭವ ಅವ್ಯಯವಾದದ್ದು. ಎಂದಿಗೂ ನಾಶವಾಗದುದು. ಒಮ್ಮೆ ಅದನ್ನು ಅನುಭವಿಸಿದರೆ ಎಂದೆಂದಿಗೂ ಅದು ನಮ್ಮದಾಗುವುದು. ಇನ್ನುಮೇಲೆ ಅದು ನಮ್ಮನ್ನು ಎಂದಿಗೂ ಬಿಡುವುದಿಲ್ಲ. ಒಮ್ಮೆ ಹೀನಲೋಹ ಸ್ಪರ್ಶಶಿಲೆಗೆ ತಾಕಿ ಚಿನ್ನವಾದರೆ, ಅದು ತಿಪ್ಪೆಯಲ್ಲಿ ಬಿದ್ದಿರಲಿ, ನಡುರಸ್ತೆಯಲ್ಲಿ ರಲಿ, ಸಂದೂಕದಲ್ಲಿಟ್ಟು ಬೀಗ ಹಾಕಿರಲಿ, ಅದು ಎಂದೆಂದಿಗೂ ಚಿನ್ನವಾಗಿಯೇ ಉಳಿಯುವುದು. ಒಮ್ಮೆ ಆ ಅನುಭವ ನಮಗೆ ಬಂದರೆ ನಾವು ಇನ್ನು ಮೇಲೆ ಹಿಂದಿನ ಮನುಷ್ಯರಾಗುವುದಿಲ್ಲ.

\begin{verse}
ಅಶ್ರದ್ಧಧಾನಾಃ ಪುರುಷಾ ಧರ್ಮಸ್ಯಾಸ್ಯ ಪರಂತಪ ।\\ಅಪ್ರಾಪ್ಯ ಮಾಂ ನಿವರ್ತಂತೇ ಮೃತ್ಯುಸಂಸಾರವರ್ತ್ಮನಿ \versenum{॥ ೩ ॥}
\end{verse}

{\small ಅರ್ಜುನ, ಯಾರಿಗೆ ಈ ಧರ್ಮದಲ್ಲಿ ಶ್ರದ್ಧೆ ಇಲ್ಲವೊ ಅವರು ನನ್ನನ್ನು ಪಡೆಯದೆ, ಮೃತ್ಯುಮಯವಾದ ಸಂಸಾರ ಮಾರ್ಗಕ್ಕೆ ಪುನಃ ಪುನಃ ಬರುತ್ತಾರೆ.}

ಯಾರು ದೇವರಲ್ಲಿ ನಂಬುವುದಿಲ್ಲವೊ, ಕೇವಲ ಈ ಮಾನುಷ ದೇಹ ಮತ್ತು ಇಂದ್ರಿಯಗಳ ಸುಖ ಇವುಗಳಲ್ಲಿ ಮಾತ್ರ ನಂಬುತ್ತಾರೊ ಅವರು, ಅದಕ್ಕೆ ಸಂಬಂಧಪಟ್ಟ ಕೆಲಸಗಳನ್ನು ಮಾಡಿ, ಅಂತಹ ಸಂಸಾರ ವಾಸನೆಗಳನ್ನು ಮಾತ್ರ ಕೂಡಿಹಾಕಿಕೊಳ್ಳುತ್ತಾರೆ. ಇಂತಹ ಜನ ಸತ್ತರೆ ಅವರು ದೇವರ ಸಮೀಪಕ್ಕೆ ಹೋಗುವುದಿಲ್ಲ. ಅವರು ಪುನಃ ಪುನಃ ಈ ಸಂಸಾರಕ್ಕೆ ಬರುತ್ತಾರೆ. ಈ ಸಂಸಾರ ಜನನ ಮರಣಗಳಿಂದ ತುಂಬಿದೆ. ಇಲ್ಲಿ ಹಲವು ಬಗೆಯ ಜನ್ಮಗಳನ್ನು ಪಡೆದು ಸುಖದುಃಖಗಳನ್ನು ಅನುಭವಿಸುತ್ತಿರುವರು.

ಜೈನ ಬುದ್ಧ ಮುಂತಾದ ಧರ್ಮಕ್ಕೆ ಸೇರಿದವರು ಕೂಡ ಒಂದು ದೇವರಲ್ಲಿ ನಂಬುವುದಿಲ್ಲ. ಆದರೆ ಅವರು ಬುದ್ಧನಂತಹ ಒಂದು ಸ್ಥಿತಿ, ತೀರ್ಥಂಕರರಂತಹ ಒಂದು ಸ್ಥಿತಿಯನ್ನು ನಂಬುವರು. ಯಾರು ಒಳ್ಳೆಯ ಜೀವನ ನಡೆಸಿಕೊಂಡು ಬುದ್ಧ ಅಥವಾ ತೀರ್ಥಂಕರರ ಸಂದೇಶಕ್ಕೆ ತಕ್ಕಂತೆ ತಮ್ಮ ಜೀವನವನ್ನು ಮಾರ್ಪಡಿಸಿಕೊಂಡಿರುವರೊ ಅವರು ಕೂಡ ಮುಕ್ತರಾಗುತ್ತಾರೆ. ಇವರು ಒಬ್ಬ ದೇವರನ್ನು ನಂಬದೆ ಇರಬಹುದು. ಆದರೆ ಒಬ್ಬ ದೇವರಂತೆ ಆಗುವ ಸ್ಥಿತಿಯನ್ನು ನಂಬುತ್ತಾರೆ. ಕರ್ಮವನ್ನು ನಂಬುತ್ತಾರೆ, ಪುನರ್ಜನ್ಮವನ್ನು ನಂಬುತ್ತಾರೆ. ಆದಕಾರಣ ಇವರು ಚಾರ್ವಾಕರಲ್ಲ, ಜಡವಾದಿಗಳಲ್ಲ. ಶ್ರೀಕೃಷ್ಣ ಏನನ್ನು ಹೇಳುವನೊ ಅದು ಇಂತಹ ಜನರನ್ನು ಕುರಿತದ್ದಲ್ಲ. ಆಧ್ಯಾತ್ಮಿಕ ಜೀವನದಲ್ಲಿ ಶ್ರದ್ಧೆ ಇಲ್ಲದವರು ಸಂಸಾರಕ್ಕೆ ಹಿಂತಿರುಗುತ್ತಾರೆ ಎಂದು ವಿಶಾಲ ದೃಷ್ಟಿಯಿಂದ ತೆಗೆದುಕೊಳ್ಳಬೇಕಾಗುವುದು.

\begin{verse}
ಮಯಾ ತತಮಿದಂ ಸರ್ವಂ ಜಗದವ್ಯಕ್ತಮೂರ್ತಿನಾ ।\\ಮತ್ಸಾó್ಥನಿ ಸರ್ವಭೂತಾನಿ ನ ಚಾಹಂ ತೇಷ್ವವಸ್ಥಿತಃ \versenum{॥ ೪ ॥}
\end{verse}

{\small ಅವ್ಯಕ್ತಮೂರ್ತಿಯಾದ ನನ್ನಿಂದ ಈ ಜಗತ್ತೆಲ್ಲವೂ ವ್ಯಾಪಿಸಲ್ಪಟ್ಟಿದೆ. ಎಲ್ಲಾ ಪ್ರಾಣಿಗಳೂ ನನ್ನಲ್ಲಿವೆ. ನಾನು ಅವುಗಳಲ್ಲಿಲ್ಲ.}

ಭಗವಂತ ಈ ಪ್ರಪಂಚವನ್ನೆಲ್ಲ ಅವ್ಯಕ್ತರೂಪದಿಂದ ವ್ಯಾಪಿಸಿಕೊಂಡಿರುವನು. ಆಕಾಶ ಹೇಗೆ ಎಲ್ಲಾ ಸಣ್ಣ ದೊಡ್ಡ ವಸ್ತುಗಳನ್ನು ವ್ಯಾಪಿಸಿಕೊಂಡಿದೆಯೋ ಹಾಗೆ ಪರಮಾತ್ಮ ವ್ಯಾಪಿಸಿಕೊಂಡಿರು ವನು. ಅವನು ಸೂಕ್ಷ್ಮ, ಇಂದ್ರಿಯಾತೀತ. ಅವನನ್ನು ನಾವು ಹೊರಗೆ ಕಾಣುವ ದೃಶ್ಯ ವಸ್ತುವಿನಂತೆ ಗ್ರಹಿಸುವುದಕ್ಕೆ ಆಗುವುದಿಲ್ಲ. ಆದರೆ ಅವನಿಲ್ಲದೆ ಈ ಪ್ರಪಂಚದಲ್ಲಿ ಯಾವುದೂ ಇಲ್ಲ. ಎಲ್ಲಾ ನಾಮರೂಪಗಳ ಹಿಂದೆಯೂ ಇರುವವನು ಅವನೆ. ಮಣ್ಣಿನಿಂದ ಮಡಕೆ ಕುಡಿಕೆಗಳನ್ನು ಮಾಡಿದರೆ ಜೇಡಿಮಣ್ಣು ಅದರಲ್ಲೆಲ್ಲಾ ವ್ಯಾಪಿಸಿಕೊಂಡಿರುವಂತೆ, ನೂಲಿನಿಂದ ಬಟ್ಟೆಯನ್ನು ಮಾಡಿದರೆ, ಆ ಬಟ್ಟೆಯಲ್ಲೆಲ್ಲಾ ನೂಲೇ ವ್ಯಾಪಿಸಿಕೊಂಡಿರುವಂತೆ, ನಾಮರೂಪಗಳ ಹಿಂದೆಲ್ಲ ಅವನೇ ವ್ಯಾಪಿಸಿ ಕೊಂಡಿರುವನು.

ಈ ಪ್ರಪಂಚದಲ್ಲಿ ನಾವು ಏನನ್ನು ನೋಡುವೆವೊ ಅವುಗಳೆಲ್ಲ ಪರಮಾತ್ಮನಲ್ಲೆ ಇರುವುವು. ಅವನನ್ನು ಬಿಟ್ಟು ಇದಕ್ಕೆ ಬೇರೆ ವ್ಯಕ್ತಿತ್ವ ಇಲ್ಲ. ಸಮುದ್ರದ ಮೇಲೆ ಇರುವ ಅಲೆ, ನೊರೆ, ಗುಳ್ಳೆ ಇವುಗಳೆಲ್ಲ ಸಮುದ್ರವಿದ್ದರೆ ಅದರಲ್ಲಿ ಮಾತ್ರ ಇರಲು ಸಾಧ್ಯ. ನಾನು ಒಂದು ಕನಸು ಕಾಣುತ್ತಿದ್ದರೆ, ನಾನು ಕನಸಿನಲ್ಲಿ ನೋಡುವುದೆಲ್ಲ ನನ್ನ ಚಿತ್ತದಲ್ಲಿದೆ. ನನ್ನ ಚಿತ್ತದಿಂದ ಮೇಲೆದ್ದ ಗುಳ್ಳೆಗಳೇ ಕನಸು. ನನ್ನ ಚಿತ್ತವಿಲ್ಲದೆ ಇದ್ದರೆ ಕನಸೇ ಇಲ್ಲ. ಹೀಗೆಯೇ ಬ್ರಹ್ಮಾಂಡವೆಲ್ಲಾ ಭಗವಂತನ ಕಲ್ಪನೆಯ ಜಗತ್ತು. ಅವನು ಅವುಗಳಲ್ಲೆಲ್ಲ ಹಾಸುಹೊಕ್ಕಾಗಿರುವನು. ಅವನಿಲ್ಲದ ವಸ್ತುವೇ ಇಲ್ಲ. ಯಾವುದನ್ನು ನಾವು ಬ್ರಹ್ಮಾಂಡ ಎನ್ನುವೆವೊ ಅದು ಅವನ ಮನಸ್ಸೆಂಬ ಸಾಗರದಿಂದ ಎದ್ದ ಗುಳ್ಳೆ. ಆ ಗುಳ್ಳೆಯಲ್ಲೆಲ್ಲ ಓತಪ್ರೋತವಾಗಿರುವವನೆ ಪರಮಾತ್ಮ.

ನಾನು ಅವುಗಳಲ್ಲಿ ಇಲ್ಲ ಎನ್ನುತ್ತಾನೆ. ಅವು ಇರಬೇಕಾದರೆ ಇವನ ಆಸರೆ ಬೇಕು. ಇವನು ಇರಬೇಕಾದರೆ ಅದರ ಆಸರೆ ಬೇಕಾಗಿಲ್ಲ. ಇವುಗಳಾವುವೂ ಇಲ್ಲದೆ ಅವನು ಇರಬಲ್ಲ. ಅಲೆ ನೊರೆ ತೆರೆ ಇವುಗಳಾವುವೂ ಸಾಗರ ಇರುವುದಕ್ಕೆ ಆವಶ್ಯಕವಿಲ್ಲ. ಇವುಗಳಿಲ್ಲದೆ ಸಾಗರ ಇರಬಲ್ಲುದು. ನಾನು ಅವುಗಳಲ್ಲಿ ಇಲ್ಲ ಎಂದರೆ ಭಗವಂತನ ಮನಸ್ಸು ಸೃಷ್ಟಿಯಲ್ಲಿ ಆಸಕ್ತವಾಗಿಲ್ಲ ಎಂದು ಹೇಳಬಹುದು. ಅವನು ಇದನ್ನು ಮಾಡಿದ, ಸೃಷ್ಟಿಗೆ ಅವನೇ ಕಾರಣ. ಆದರೂ ಅವನು ತಾನು ಸೃಷ್ಟಿಸಿದ ವಸ್ತುವಿನಲ್ಲಿ ಸಿಕ್ಕಿಕೊಂಡು ನರಳುತ್ತಿಲ್ಲ. ನಾವು ಏನಾದರೂ ಒಂದು ಸಣ್ಣದನ್ನು ಸೃಷ್ಟಿಸಿದರೆ ನಮ್ಮ ಮನಸ್ಸೆಲ್ಲ ಅದರಲ್ಲಿ. ನಾವು ಅದರಲ್ಲಿ ಸಿಕ್ಕಿಕೊಂಡು ಬಿಡುವೆವು. ಮನಸ್ಸನ್ನು ಅದರಿಂದ ಸೆಳೆಯ ಲಾರೆವು. ನಾವೊಂದು ಸಣ್ಣ ಪುಸ್ತಕ ಬರೆಯಲಿ, ಚಿತ್ರ ಬರೆಯಲಿ, ಒಂದು ಮಗುವನ್ನು ಸೃಷ್ಟಿಸಲಿ, ನಮ್ಮ ಒಂದು ಭಾಗ ಅದರಲ್ಲಿ ಸಿಕ್ಕಿಕೊಳ್ಳುವುದು. ಜನರು ಅವನ್ನು ಕೊಂಡಾಡಿದರೆ ನಮಗೆ ಸಂತೋಷ. ಅದನ್ನು ತೆಗಳಿದರೆ ದುಃಖ. ಏತಕ್ಕೆಂದರೆ ನಾವು ಆಸಕ್ತರು. ಆದರೆ ದೇವರು ಹಾಗಲ್ಲ. ಈ ಪ್ರಪಂಚದಲ್ಲಿ ಯಾವುದು ಇರಬೇಕಾದರೂ ದೇವರು ಬೇಕು. ಆದರೆ ದೇವರು ಯಾವುದಕ್ಕೂ ದಾಸನಲ್ಲ.

\begin{verse}
ನ ಚ ಮತ್ಸಾó್ಥನಿ ಭೂತಾನಿ ಪಶ್ಯ ಮೇ ಯೋಗಮೈಶ್ವರಮ್ ।\\ಭೂತಭೃನ್ನ ಚ ಭೂತಸ್ಥೋ ಮಮಾತ್ಮಾ ಭೂತಭಾವನಃ \versenum{॥ ೫ ॥}
\end{verse}

{\small ಪ್ರಾಣಿಗಳು ಕೂಡ ನನ್ನಲ್ಲಿ ಇಲ್ಲ. ನನ್ನ ಈಶ್ವರಸಂಬಂಧವಾದ ಯೋಗವನ್ನು ನೋಡು. ನಾನು ಭೂತಗಳನ್ನು ಧರಿಸಿದ್ದರೂ ಭೂತಗಳಲ್ಲಿ ಇಲ್ಲ. ಆದರೆ ಅವುಗಳ ಉತ್ಪತ್ತಿಗೆ ನಾನು ಕಾರಣ.}

ಹಿಂದಿನ ಶ್ಲೋಕದಲ್ಲಿ ಪ್ರಾಣಿಗಳೆಲ್ಲ ನನ್ನಲ್ಲಿವೆ ಎಂದು ಹೇಳಿ, ಇಲ್ಲಿ ಅವು ನನ್ನಲ್ಲಿ ಇಲ್ಲ ಎನ್ನುತ್ತಾನೆ. ಒಂದು ದೃಷ್ಟಿಯಿಂದ ನೋಡಿದರೆ ಇದು ಪರಸ್ಪರ ವಿರೋಧವಾಗಿ ಕಾಣುವುದು. ಒಂದು ನಿಜವಾದರೆ ಮತ್ತೊಂದು ಸುಳ್ಳಾಗಬೇಕು ಎಂದು ಭಾವಿಸುವೆವು. ಆದರೆ ವಿರೋಧಾಭಾಸಗಳೆಲ್ಲ ಭಗವಂತನಲ್ಲಿ ಸಂಧಿಸುವುವು. ಅವನ್ನು ನೋಡುವ ದೃಷ್ಟಿ ಮಾತ್ರ ವ್ಯತ್ಯಾಸವಾಗುವುದು. ಎಲ್ಲಾ ಪ್ರಾಣಿಗಳು ಪರಮಾತ್ಮನಲ್ಲಿವೆ. ಪರಮಾತ್ಮನನ್ನು ಬಿಟ್ಟರೆ ಅವು ಇರಲಾರವು. ಮೀನು ನೀರುಬಿಟ್ಟು ಹೇಗೆ ಇರಲಾರದೊ, ಅಲೆ ಸಾಗರ ಬಿಟ್ಟು ಹೇಗೆ ಇರಲಾರದೊ ಹಾಗೆ ಎಲ್ಲವೂ ಅವನಲ್ಲಿದೆ. ಇದನ್ನು ಬೇರೆ ದೃಷ್ಟಿಯಿಂದ ಹೇಳುತ್ತಿರುವನು. ಇವನ ಮನಸ್ಸು ಸೃಷ್ಟಿಯಮೇಲೆ ನಿಂತಿಲ್ಲ. ಇವನು ಸೃಷ್ಟಿಯಿಂದ ಏನನ್ನೂ ನಿರೀಕ್ಷಿಸುತ್ತಿಲ್ಲ, ಇವನಿಗೆ ಸೃಷ್ಟಿಯಿಂದ ಏನೂ ಆಗಬೇಕಾಗಿಲ್ಲ. ಇವನಿರು ವುದಕ್ಕೆ ಸೃಷ್ಟಿಯ ಆಧಾರ ಬೇಕಾಗಿಲ್ಲ. ಕನಸಿಗೆ ನಾನಿರಬೇಕು, ಆದರೆ ನಾನಿರಬೇಕಾದರೆ ಕನಸು ಇರಬೇಕಾಗಿಲ್ಲ.

ಇದನ್ನೇ ಶ್ರೀಕೃಷ್ಣ ತನ್ನ ಯೋಗೈಶ್ವರ್ಯ ಎನ್ನುತ್ತಾನೆ. ಅದ್ಭುತವಾದ ಅಘಟಿತವಾದ ಕಾರ್ಯಗಳು ಅವುಗಳ ಮೂಲಕ ಆಗಿಹೋಗುತ್ತವೆ. ಅವನಿದದ ನಾವೂ ಸಣ್ಣ ಪ್ರಮಾಣದಲ್ಲಿ ಸೃಷ್ಟಿಕರ್ತರೆ. ಆದರೆ ನಾವೊಂದು ಸಣ್ಣದೊಂದು ವಸ್ತುವನ್ನುಸೃಷ್ಟಿ ಮಾಡಿ ಎಂತಹ ತಾಪತ್ರಯಕ್ಕೆ ಸಿಕ್ಕಿಹಾಕಿಕೊಳ್ಳುತ್ತೇವೆ. ಭಗವಂತ ಈ ಭುವನವೆಂಬ ಅದ್ಭುತ ವಸ್ತುವನ್ನು ಸೃಷ್ಟಿಸಿದರೂ ಕೇವಲ ಸಾಕ್ಷಿಯಂತೆ ಅಸಂಗನಾಗಿ ಅದನ್ನು ನಿತು ನೋಡುತ್ತಿರುವನು. ವಾಸಮಾಡಲು ನಾವೊಂದು ಗೂಡನ್ನು ಕಟ್ಟಿಕೊಂಡರೆ ನಮಗೊಂದು ದೊಡ್ಡ ಅರಮನೆ ಅದು, ನಮ್ಮ ಮಗು ಒಂದು ಅದ್ಭುತ ಪ್ರತಿಭಾವಂತ, ನಾವು ಬರೆದ ಒಂದು ಸಣ್ಣಕವನವೊ, ಅಥೆಯೊ ಒಂದು ಅಮರಕಾವ್ಯ ಒಂದು ಭಾವಿಸುತ್ತೇವೆ. ಅಷ್ಟು ಮಾತ್ರವಲ್ಲ ನಾವು ಅದರಲ್ಲಿ ಕೂಡಿಹಾಕಿಕೊಳ್ಳುತ್ತೇವೆ. ನಾವು ಸೃಷ್ಟಿಸಿದ ವಸ್ತುವೇ ನಮಗೊಂದು ಸೆರೆಮನೆಯಾಗುವುದು. ಅದೊಂದು ಹಕ್ಕಿಯಗೂಡಿನಂತೆ ಇಲ್ಲ. ಭಗವಂತನ ಸೃಷ್ಟಿ ಎಂತಹ ಒಂದು ಮಹೋನ್ನತ ಕಾವ್ಯ. ತುದಿಮೊದಲಿಲ್ಲದ ಕಾವ್ಯ. ಇದೆಂತಹ ಒಂದು ಚಿತ್ರ, ಇದೆಂತಹ ಒಂದು ನಾಟಕ, ಇಂಎಂತಹ ಒಂದು ಹಾಡು, ಇಂಎಂತಹ ಒಂದಿ ಭಾವ. ಈ ಭೂಮವನ್ನು ಸೃಷ್ಟಿಸಿದವನು ಭಗವಂತ. ಆದರೆ ಎಳ್ಳಷ್ಟು ಇದರಮೇಲೆ ಆಸಕ್ತನಲ್ಲ. ಎಂತಹ ಒಂದು ಸಾಕ್ಷಾ ಭಾವ ಅಲ್ಲದೆ!

\begin{verse}
ಯಥಾಕಾಶಸ್ಥಿತೋ ನಿತ್ಯಂ ವಾಯುಃ ಸರ್ವತ್ರಗೋ ಮಹಾನ್ ।\\ತಥಾ ಸರ್ವಾಣಿ ಭೂತಾನಿ ಮತ್ಸಾó್ಥನೀತ್ಯುಪಧಾರಯ \versenum{॥ ೬ ॥}
\end{verse}

ಹೇಗೆ ಎಲ್ಲೆಲ್ಲಿಯೂ ಸಂಚರಿಸುವ ಮಹಾವಾಯು ಆಕಾಶದಲ್ಲಿದೆಯೊ ಹಾಗೆಯೇ ಎಲ್ಲಾ ಪ್ರಾಣಿಗಳೂ ನನ್ನಲ್ಲಿವೆ ಎಂಬುದನ್ನು ತಿಳಿದುಕೊ.

ಎಲ್ಲ ವಸ್ತುಗಳು ತನ್ನಲ್ಲಿ ಹೇಗಿವೆ ಎಂಬುದಕ್ಕೆ ಒಂದು ಉದಾಹರಣೆಯನ್ನು ಕೊಡುವನು. ಮಹಾವಾಯು ಎಲ್ಲಾ ಕಡೆಯಲ್ಲಿಯೂ ಸಂಚರಿಸುತ್ತಿರುವುದು ಹೇಗೆ ಆಕಾಶದಲ್ಲಿದೆಯೋ ಹಾಗೆ ಎಲ್ಲವೂ ಅವನಲ್ಲಿದೆ. ವಾಯು ಆಕಾಶದಲ್ಲಿ ಹಲವು ರೂಪಗಳನ್ನು ಧರಿಸಿ ಚಲಿಸುತ್ತಿದೆ. ಒಂದು ಕಡೆ ಚಂಡಮಾರುತದಂತೆ ಗಿಡಮನೆಗಳನ್ನು ಉರುಳಿಸುತ್ತ ಇದೆ. ಮತ್ತೊಂದು ಕಡೆ ತಂಗಾಳಿಯಂತೆ ಬೀಸುತ್ತಿದೆ. ಇನ್ನೊಂದು ಕಡೆ ಸುಂಟುರುಗಾಳಿಯಂತೆ ನೆಲದ ಮೇಲೆ ಇರುವುದನ್ನೆಲ್ಲ ಆಕಾಶಕ್ಕೆ ಎರಚಿ ಹೋಗುತ್ತಿದೆ. ಒಂದು ಕಡೆ ಪುಷ್ಪಗಳ ಪರಿಮಳವನ್ನು ಹೊತ್ತು ಬರುತ್ತದೆ. ಮತ್ತೊಂದು ಕಡೆ ತಿಪ್ಪೆ ಗುಂಡಿಯ ದುರ್ಗಂಧವನ್ನು ಹೊತ್ತು ಬರುತ್ತದೆ. ಆದರೆ ಆಕಾಶ ಇವುಗಳಾವುದರಿಂದಲೂ ಬಾಧಿದವಾಗುವುದಿಲ್ಲ. ಪುಣ್ಯವಂತರಿರುವರು, ಪಾಪಾತ್ಮರಿರುವರು, ಬುದ್ಧಿವಂತರಿರುವರು ದಡ್ಡರಿರುವರು. ಈ ಪ್ರಪಂಚದಲ್ಲಿ ಎಲ್ಲಾ ಬಗೆಯ ಜನರೂ ಇರುವರು. ಆದರೆ ದೇವರು ಇದರಿಂದ ಬಾಧಿತನಾಗುವುದಿಲ್ಲ. ಹಾವಿನ ಬಾಯಲ್ಲಿರುವ ವಿಷದಂತೆ. ಹಾವಿಗೆ ಇದರಿಂದ ಯಾವ ಬಾಧೆಯೂ ಇಲ್ಲ. ಎಲ್ಲಾ ಇರುವುದಕ್ಕೆ ದೇವರಿರಬೇಕು. ಆದರೆ ದೇವರು ಎಲ್ಲದರಿಂದ ತಾಪತ್ರಯಕ್ಕೆ ಸಿಕ್ಕಿಕೊಳ್ಳುವುದಿಲ್ಲ. ಒಂದು ಸಿನಿಮಾ ತೆರೆಯ ಮೇಲೆ ಏನೇನೊ ಗೊಂಬೆಗಳು ಬೀಳುತ್ತವೆ. ಆ ಗೊಂಬೆ ಬೀಳಬೇಕಾದರೆ ತೆರೆ ಆವಶ್ಯಕ. ಆದರೆ ಆ ಗೊಂಬೆಗಳಿಂದ ತೆರೆ ಕೆಟ್ಟು ಹೋಗುವುದೇ? ಸಿನಿಮಾದಲ್ಲಿ ಕಾಡುಗಿಚ್ಚು ಉರಿಯುವುದನ್ನು ತೋರಬಹುದು. ಆದರೆ ತೆರೆಗೆ ಅದರಿಂದ ಸ್ವಲ್ಪವೂ ಬೆಂಕಿ ತಾಕುವುದಿಲ್ಲ. ಅಲೆಗಳಿಂದ ತುಂಬಿದ ಮಹಾಸಾಗರವನ್ನೇ ತೆರೆಯ ಮೇಲೆ ತೋರಬಹುದು.ಅದರರಿಂದ ಅದು ಸ್ವಲ್ಪವೂ ಒದ್ದೆಯಾಗುವುದಿಲ್ಲ. ಬ್ರಹ್ಮಾಂಡದ ಸಿನಿಮ ಪರಮಾತ್ಮನೆಂಬ ತೆರೆಯ ಮೇಲೆ ಬೀಳುತ್ತಿದ್ದರೂ ಅವನು ಇದರಿಂದ ಬಾಧಿತನಾಗುವುದಿಲ್ಲ.

\begin{verse}
ಸರ್ವಭೂತಾನಿ ಕೌಂತೇಯ ಪ್ರಕೃತಿಂ ಯಾಂತಿ ಮಾಮಿಕಾಮ್ ।\\ಕಲ್ಪಕ್ಷಯೇ ಪುನಸ್ತಾನಿ ಕಲ್ಪಾದೌ ವಿಸೃಜಾಮ್ಯಹಮ್ \versenum{॥ ೭ ॥}
\end{verse}

ಅರ್ಜುನ, ಎಲ್ಲಾ ಪ್ರಾಣಿಗಳು ಪ್ರಳಯ ಕಾಲದಲ್ಲಿ ನನ್ನ ಪ್ರಕೃತಿಯನ್ನು ಸೇರುತ್ತವೆ. ಪುನಃ ಸೃಷ್ಟಿಯ ಕಾಲದಲ್ಲಿ ನಾನು ಅವುಗಳನ್ನು ಸೃಷ್ಟಿಸುತ್ತೇನೆ.

ಭಗವಂತನ ಪ್ರಕೃತಿ ತ್ರಿಗುಣಗಳಿಂದ ಆಗಿದೆ. ಅದೇ ಸತ್ವ, ರಜಸ್ ಮತ್ತು ತಮಸ್ ಎಂಬುವು. ನಅವು ನೋಡುವ ಈ ವಿಶ್ವವೆಲ್ಲಾ ಈ ಗುಣಗಳ ತರತಮದಿಂದ ಆದುದು. ಯಾವುದನ್ನು ಪ್ರಳಯ ಮತ್ತು ಸೃಷ್ಟಿ ಎನ್ನುತ್ತೇವೆಯೋ ಅದು ಸಾಗರದಲ್ಲಿರುವ ಅಲೆಗಳಂತೆ. ಎದ್ದಾಗ ಸೃಷ್ಟಿ, ಬಿದ್ದಾಗ ಪ್ರಳಯ. ಎಷ್ಟೊ ಪ್ರಳಯ ಸೃಷ್ಟಿಗಳು ಆಗಿವೆ ಹಿಂದೆ, ಆಗಲಿವೆ ಮುಂದೆ. ಪ್ರಳಯದಲ್ಲೂ ನಾಮರೂಪುಗಳ ಸ್ಥೂಲ ನಾಶವಾಗುವುದು. ಆದರೆ ಅವುಗೊಳ ಹಿಂದೆ ಇರುವ ಸ್ವಭಾವ ಹಾಗೆ ಸೂಕ್ಷ್ಮ ಅವಸ್ಥೆಯಲ್ಲಿ ಇರುವುದು. ಇವು ಪ್ರಕೃತಿಯ ಉಗ್ರಾಣದಲ್ಲಿರುವುವು. ದೊಡ್ಡಮರ ಹೇಗೆ ಸಣ್ಣಬೀಜದಲ್ಲಿರುವುದೊ ಹಾಗೆ ಕಾರಣಾವಸ್ಥೆಯಲ್ಲಿ ಪ್ರಪಂಚವೆಲ್ಲಾ ಸುಪ್ತಾವಸ್ಥೆಯಲ್ಲಿರುವುದು. ಯಾವಾಗ ಸೃಷ್ಟಿ ಪ್ರಾರಂಭವಾಗುವುದೋ ಆಗ ಉಗ್ರಾಣದಲ್ಲಿರುವ ಬೀಜಗಳನ್ನು ಪುನಃ ದೇವರು ಬಿತ್ತುವನು. ಈ ಬ್ರಹ್ಮಾಂಡವೆಂಬ ಸೃಷ್ಟಿ ಕ್ರಮೇಣ ಹುಲುಸಾಗಿ ಬೆಳೆಯುವುದು. ಇಲ್ಲಿ ಎರಡು ಕೆಲಸಗಳನ್ನು ಮಾಡುವನು ಅವನು. ಪ್ರಪಂಚವನ್ನು ಸೃಷ್ಟಿಸುತ್ತಾನೆ ಕಲ್ಪದ ಪ್ರಾರಂಭದಲ್ಲಿ. ನಿಷ್ಕರುಣಿಯಂತೆ ಎಲ್ಲವನ್ನೂ ಹಿಂದಕ್ಕೆ ಕಸಿದುಕೊಳ್ಳುತ್ತಾನೆ ಪ್ರಳಯಕಾಲದಲ್ಲಿ. ಮನೆಯ ಮುಂದೆ ಮಗು ಮಣ್ಣಿನ ಮನೆಯನ್ನು ಕಟ್ಟಿಕೊಂಡು ಆಡುವುದು. ಹೋಗುವ ಸಮಯ ಬಂದಾಗ ಅದರ ಮೇಲೆ ನಿಂತು ಎಲ್ಲವನ್ನೂ ಕೆಡವಿ ಹೊರಟುಹೋಗುವುದು. ಅದಕ್ಕೆ ಕಟ್ಟುವುದೂ ಒಂದು ಆಟವೇ, ಕೆಡಹುವುದೂ ಒಂದು ಆಟವೇ. ಹಾಗೆಯೇ ಬ್ರಹ್ಮಾಂಡದ ಸೃಷ್ಟಿ ಸಂಹಾರಗಳು ಭಗವಂತನಿಗೆ.

\begin{verse}
ಪ್ರಕೃತಿಂ ಸ್ವಾಮವಷ್ಟಭ್ಯ ವಿಸೃಜಾಮಿ ಪುನಃ ಪುನಃ ।\\ಭೂತಗ್ರಾಮಮಿಮಂ ಕೃತ್ಸ್ನಮವಶಂ ಪ್ರಕೃತೇವಶಾತ್ \versenum{॥ ೮ ॥}
\end{verse}

ನನ್ನ ಪ್ರಕೃತಿಯನ್ನು ಅಧಿಷ್ಠಾನ ಮಾಡಿಕೊಂಡು ಪ್ರಕೃತಿಯ ವಶವಾದ ಅಸ್ವತಂತ್ರವಾದ ಈ ಪ್ರಾಣಿವರ್ಗಗಳನ್ನೆಲ್ಲಾ ನಾನು ಪುನಃ ಪುನಃ ಸೃಷ್ಟಿಸುತ್ತಿರುವೆನು.

ಣಿಜ್ಞ್ಜಕಿಖಿಗಿ ಲವ್ಕಿವ್ ಪ್ ಅವುಗಳ ಮೂಲಕ ಆಗಿಹೋಗುತ್ತವೆ. ಅವನಿಂದ ಸೃಷ್ಟಿ ಹೊಮ್ಮುತ್ತದೆ. ಆದರೂ ಅವನು ಅದಕ್ಕೆ ಆಸಕ್ತನಲ್ಲ. ಒಂದು ದೃಷ್ಟಿಯಿಂದ ನಾವೂ ಸಣ್ಣ ಪ್ರಮಾಣದಲ್ಲಿ ಸೃಷ್ಟಿಕರ್ತರೆ. ಆದರೆ ನಾವೊಂದು ಸಣ್ಣದೊಂದು ವಸ್ತುವನ್ನು ಸೃಷ್ಟಿ ಮಾಡಿ ಎಂತಹ ತಾಪತ್ರಯಕ್ಕೆ ಸಿಕ್ಕಿಹಾಕಿಕೊಳ್ಳುತ್ತೇವೆ. ಭಗವಂತ ಈ ಭುವನವೆಂಬ ಅದ್ಭುತ ವಸ್ತುವನ್ನು ಸೃಷ್ಟಿಸಿದರೂ ಕೇವಲ ಸಾಕ್ಷಿಯಂತೆ ಅಸಂಗನಾಗಿ ಅದನ್ನು ನಿಂತು ನೋಡುತ್ತಿರುವನು. ವಾಸಮಾಡಲು ನಾವೊಂದು ಗೂಡನ್ನು ಕಟ್ಟಿಕೊಂಡರೆ ನಮಗೊಂದು ದೊಡ್ಡ ಅರಮನೆ ಅದು, ನಮ್ಮ ಮಗು ಒಂದು ಅದ್ಭುತ ಪ್ರತಿಭಾವಂತ, ನಾವು ಬರೆದ ಒಂದು ಸಣ್ಣಕವನವೊ, ಕಥೆಯೊ ಒಂದು ಅಮರಕಾವ್ಯ ಒಂದು ಭಾವಿಸುತ್ತೇವೆ. ಅಷ್ಟು ಮಾತ್ರವಲ್ಲ ನಾವು ಅದರಲ್ಲಿ ಕೂಡಿಹಾಕಿಕೊಳ್ಳುತ್ತೇವೆ. ನಾವು ಸೃಷ್ಟಿಸಿದ ವಸ್ತುವೇ ನಮಗೊಂದು ಸೆರೆಮನೆಯಾಗುವುದು. ಅದೊಂದು ಹಕ್ಕಿಯಗೂಡಿನಂತೆ ಇಲ್ಲ. ಭಗವಂತನ ಸೃಷ್ಟಿ ಎಂತಹ ಒಂದು ಮಹೋನ್ನತ ಕಾವ್ಯ! ತುದಿಮೊದಲಿಲ್ಲದ ಕಾವ್ಯ. ಇದೆಂತಹ ಒಂದು ಚಿತ್ರ, ಇದೆಂತಹ ಒಂದು ನಾಟಕ, ಇದೆಂತಹ ಒಂದು ಹಾಡು, ಇದೆಂತಹ ಒಂದು ಭಾವ! ಈ ಭೂಮವನ್ನು ಸೃಷ್ಟಿಸಿದವನು ಭಗವಂತ. ಆದರೆ ಎಳ್ಳಷ್ಟು ಇದರಮೇಲೆ ಆಸಕ್ತನಲ್ಲ. ಎಂತಹ ಒಂದು ಸಾಕ್ಷೀಭಾವ ಅಲ್ಲಿದೆ!

\begin{verse}
ಯಥಾಕಾಶಸ್ಥಿತೋ ನಿತ್ಯಂ ವಾಯುಃ ಸರ್ವತ್ರಗೋ ಮಹಾನ್ ।\\ತಥಾ ಸರ್ವಾಣಿ ಭೂತಾನಿ ಮತ್ಸಾó್ಥನೀತ್ಯುಪಧಾರಯ \versenum{॥ ೬ ॥}
\end{verse}

{\small ನನ್ನ ಪ್ರಕೃತಿಯನ್ನು ಅಧಿಷ್ಠಾನ ಮಾಡಿಕೊಂಡು ಪ್ರಕೃತಿಯ ವಶವಾದ ಅಸ್ವತಂತ್ರವಾದ ಈ ಪ್ರಾಣಿವರ್ಗ ಗಳನ್ನೆಲ್ಲಾ ನಾನು ಪುನಃ ಪುನಃ ಸೃಷ್ಟಿಸುತ್ತಿರುವೆನು.}

ಪ್ರಳಯದಲ್ಲಿ ಜೀವರಾಶಿಗಳೆಲ್ಲಾ ಸುಪ್ತಾವಸ್ಥೆಗೆ ಹೋಗುತ್ತವೆ. ಅಲ್ಲಿ ಅವು ಪರಮಾತ್ಮನ ಪ್ರಕೃತಿಯ ಉಗ್ರಾಣದಲ್ಲಿರುತ್ತವೆ. ಇದು ಒಂದು ಕೋಲ್ಡ್ ಸ್ಟೋರೇಜ್​ನಂತೆ. ಹಣ್ಣು ಹಂಪಲುಗಳು ಹೇಗೆ ಅಲ್ಲಿ ಎಷ್ಟು ದಿನಗಳು ಬೇಕಾದರೂ ಇರುವುವೋ ಹಾಗೆ. ಆದರೆ ಪ್ರಕೃತಿ ಜೀವಿಯಲ್ಲಿರುವ ಸಂಸ್ಕಾರಗಳನ್ನು ಮಾರ್ಪಡಿಸುವುದಕ್ಕೆ ಅವಕಾಶ ಕೊಡುವುದು. ಅಪೂರ್ಣತೆಗಳೆಲ್ಲಾ ಅಲ್ಲಿ ಇರುವುವು. ಅವು ನೇರವಾಗಬೇಕಾದರೆ ಜೀವರಾಶಿಗಳು ತಾವೇ ಹೊಸದಾಗಿ ಕರ್ಮಮಾಡಿ ಹಿಂದಿನದರಿಂದ ಪಾರಾಗಬೇಕಾಗಿದೆ. ಅವು ಪೂರ್ಣವಾಗುವುದಕ್ಕೆ ಮತ್ತೊಂದು ಸೃಷ್ಟಿಯ ಆವಶ್ಯಕತೆ ಇದೆ. ಅದಕ್ಕೆ ಭಗವಂತ ಪುನಃ ಸೃಷ್ಟಿಯನ್ನು ಮಾಡುತ್ತಾನೆ. ಹಾಗೆ ಸೃಷ್ಟಿಯನ್ನು ಮಾಡುವಾಗ ತನ್ನ ಪ್ರಕೃತಿಯ ಆಧಾರದಿಂದ ಮಾಡುತ್ತಾನೆ. ಪ್ರಕೃತಿಯೇ ಸ್ಥೂಲ ಸೂಕ್ಷ್ಮವಾದ ಶಕ್ತಿಯಂತೆ, ವಸ್ತುವಿನಂತೆ ಮತ್ತು ಗುಣಗಳಂತೆ ಇರುವುದು. ಈ ಪ್ರಕೃತಿಯೇ ಜೀವರಾಶಿಗಳು ಕರ್ಮ ಸವೆಸುವುದಕ್ಕೆ ರಂಗಭೂಮಿ ಯಾಗುವುದು. ಇಲ್ಲಿ ಪುನಃ ಯಾವುದಾದರೂ ಪಾತ್ರ ವಹಿಸಿಕೊಂಡು ಅಭಿನಯಿಸುತ್ತಾರೆ.

ಬದ್ಧರಾದ ಜೀವರಾಶಿಗಳು ಅಸ್ವತಂತ್ರರು. ತಮ್ಮ ಕರ್ಮಗಳನ್ನು ಸವೆಸಲು ಪುನಃ ಅವರು ಬರಲೇಬೇಕಾಗಿದೆ. ಫೇಲ್ ಆಗಿರುವ ವಿದ್ಯಾರ್ಥಿ ಮೇಲಿನ ಕ್ಲಾಸಿಗೆ ಹೋಗಬೇಕಾದರೆ ಪುನಃ ಪರೀಕ್ಷೆಗೆ ಕುಳಿತುಕೊಳ್ಳಬೇಕಾಗಿರುವಂತೆ ಇದು. ಆದರೆ ಇಲ್ಲಿ ಬೇಕಾದರೆ ನಾನು ಓದುವುದಿಲ್ಲ, ಮುಂದಿನ ಕ್ಲಾಸಿಗೂ ಹೋಗುವುದಿಲ್ಲ, ಎಂದು ನಮ್ಮ ವಿದ್ಯಾಭ್ಯಾಸವನ್ನು ನಿಲ್ಲಿಸಬಹುದು. ಆದರೆ ಪ್ರಕೃತಿಯ ವಶದಲ್ಲಿರುವ ನಾವು ಹಾಗೆ ಮಾಡುವುದಕ್ಕೆ ಆಗುವುದಿಲ್ಲ. ನಮಗೆ ಮುಕ್ತಿ ಬೇಡವೆಂದರೂ ಜನ್ಮವೆತ್ತಬೇಕು. ಮುಕ್ತಿ ಬೇಕಾದರಂತೂ ಪುಣ್ಯಕರ್ಮಗಳನ್ನು ಮಾಡುವುದಕ್ಕೆ ಜನ್ಮವೆತ್ತಲೇಬೇಕಾ ಗಿದೆ. ಪ್ರಕೃತಿ ಯಾರನ್ನೂ ಒಂದು ಕಡೆ ಸುಮ್ಮನೆ ಕುಳಿತುಕೊಂಡಿರುವುದಕ್ಕೆ ಬಿಡುವುದಿಲ್ಲ. ಅದು ಯಾವಾಗಲೂ ಸಂಚರಿಸುತ್ತಿರುವ ಕಾರ್ಖಾನೆಯಲ್ಲಿರುವ ಬೆಲ್ಟಿನಂತೆ. ನಮ್ಮನ್ನು ಅದರ ಮೇಲಿಟ್ಟು ಒಂದು ಕಡೆಯಿಂದ ಮತ್ತೊಂದೆಡೆಗೆ ರವಾನಿಸುವುದು. ಒಂದೊಂದು ಕಡೆ ಒಂದೊಂದು ರೀಪೇರಿ ಆಗಿ, ಕೆಲವು ಕಡೆ ಒಂದನ್ನು ಹೊಸದಾಗಿ ಸೇರಿಸಿಯೊ ಮತ್ತೊಂದೆಡೆ ಇನ್ನೊಂದನ್ನು ಕಳಚಿಯೊ ಅಂತೂ ಪೂರ್ಣಾತ್ಮರಾಗುವವರೆಗೆ ಅದು ನಮ್ಮನ್ನು ಒಂದೆಡೆಯಿಂದ ಇನ್ನೊಂದೆಡೆಗೆ ರವಾನಿಸು ತ್ತಿರುವುದು.

ಇದು ನನ್ನ ಪ್ರಕೃತಿ ಎಂದು ಹೇಳುತ್ತಾನೆ ಶ್ರೀಕೃಷ್ಣ. ಈ ಪ್ರಕೃತಿಯೆಲ್ಲಾ ಅವನದಾಗಿರುವುದರಿಂದ ಅದರ ಸ್ವಾಮಿ ಅವನು. ಇವನು ಹೇಳಿದಂತೆ ಪ್ರಕೃತಿ ಕೇಳುವುದು. ಬದ್ಧರಾದ ಜೀವರಾಶಿಗಳೆಲ್ಲ ಪ್ರಕೃತಿಯಲ್ಲಿಇವೆ. ಆದರೆ ಪ್ರಕೃತಿಯೇ ಭಗವಂತನಿಗೆ ದಾಸನಂತಿದೆ. ಇತರರ ಮೇಲೆ ತನ್ನ ಪ್ರಭಾವವನ್ನು ಬೀರಿದರೂ ಪ್ರಕೃತಿ ಪರಮಾತ್ಮನನ್ನು ಏನೂ ಮಾಡಲಾರದು. ಅದು ಹಾವಾಡಿಸು ವವನ ಕೊರಳಿನಲ್ಲಿರುವ ಹಾವಿನಂತೆ. ಹಾವಾಡಿಸುವವನು ಅವುಗಳಿಂದ ಏನು ಬೇಕಾದರೂ ಮಾಡುವನು. ಆದರೆ ಅದೇ ಹಾವು ಇತರರನ್ನು ಕಂಡರೆ ಬುಸುಗುಟ್ಟುವುದು.

\begin{verse}
ನ ಚ ಮಾಂ ತಾನಿ ಕರ್ಮಾಣಿ ನಿಬಧ್ನಂತಿ ಧನಂಜಯ ।\\ಉದಾಸೀನವದಾಸೀನಮಸಕ್ತಂ ತೇಷು ಕರ್ಮಸು \versenum{॥ ೯ ॥}
\end{verse}

{\small ಧನಂಜಯ, ಈ ಕರ್ಮಗಳು ನನ್ನನ್ನು ಬಂಧಿಸವು. ಏಕೆಂದರೆ ನಾನು ಅವುಗಳಲ್ಲಿ ಉದಾಸೀನನಾಗಿ, ಆಸಕ್ತಿ ರಹಿತನಾಗಿ ವರ್ತಿಸುತ್ತೇನೆ.}

ಈ ಸೃಷ್ಟಿಯನ್ನೆಲ್ಲಾ ದೇವರು ಮಾಡುತ್ತಿರುವುದರಿಂದ ಅವನಿಗೂ ಒಂದು ಉದ್ದೇಶವಿದೆ ಎಂದಂತಾಯಿತು. ಅವನಿಗೆ ಸಾಧಿಸುವುದಕ್ಕೆ ಏನೊ ಇದೆ ಎಂದರೆ ಅವನು ಅಪೂರ್ಣನಾಗುತ್ತಾನೆ. ಪೂರ್ಣಕಾಮನಾದವನು ಏನನ್ನು ಪಡೆಯಲು ಇದನ್ನೆಲ್ಲಾ ಮಾಡಬೇಕು? ನಮಗೆ ಒಂದು ಸಣ್ಣ ಬಯಕೆ ಇದೆ, ಅದಕ್ಕೆ ಪ್ರಪಂಚಕ್ಕೆ ಬಂದು ಹೋಗುತ್ತಿರುವುದು. ಅವನಿಗೆ ಒಂದು ದೊಡ್ಡ ಬಯಕೆ ಇದೆ, ಅದಕ್ಕಾಗಿ ಸೃಷ್ಟಿ ಸಂಹಾರ ಕಾರ್ಯದಲ್ಲಿ ಮಗ್ನನಾಗಿರುವುದು ಎಂದು ಭಾವಿಸಬಹುದು. ಅವನಿಗೆ ಪಡೆಯದೆ ಇರುವುದು ಯಾವುದೂ ಇಲ್ಲ, ಇವನ್ನು ಸುಮ್ಮನೆ ಮಾಡುತ್ತಾನೆ ಎನ್ನಬಹುದು. ಸುಮ್ಮನೆ ಮಾಡುವವನು ಹುಚ್ಚನೊಬ್ಬನೆ. ಅವನು ಕುಳಿತಿರಲಾರದೆ ಏನನ್ನಾದರೂ ಮಾಡುವನು. ಹಾಗೆಯೇ ದೇವರು ಸುಮ್ಮನೆ ಏನೇನೊ ಮಾಡುವನು. ಅದರಿಂದ ನರಳುವವರು ಜೀವರಾಶಿಗಳು. ಅಂತೂ ದೇವರು ಉದ್ದೇಶವನ್ನಿಟ್ಟುಕೊಂಡವನೆಂದರೆ ಅಪೂರ್ಣನಾಗುವನು. ಉದ್ದೇಶವಿಲ್ಲ ಎಂದರೆ ಹುಚ್ಚನೋ ಮರುಳನೋ ಯಾರೋ ಆಗುತ್ತಾನೆ. ಆದರೆ ದೇವರು ಈ ಸೃಷ್ಟಿಯನ್ನೆಲ್ಲಾ ಮಾಡಿದರೂ ಅವನು ಆಸಕ್ತಿಯಿಂದ ಮಾಡುತ್ತಿಲ್ಲ. ಅದರಿಂದ ಬರುವ ಫಲವನ್ನುಣ್ಣಲು ಮಾಡುತ್ತಿಲ್ಲ. ಅವನು ಉದಾಸೀನನಾಗಿ ಸೃಷ್ಟಿ ಮಾಡದೆ ಇದ್ದರೆ ಬದ್ಧಜೀವಿಗಳು ಮುಕ್ತರಾಗುವುದಕ್ಕೆ ಅವಕಾಶವೇ ಇರುವು ದಿಲ್ಲ. ಜೀವರಾಶಿಗಳ ಮೇಲಿನ ಕಾರುಣ್ಯದಿಂದಲೇ ಅವರ ಉದ್ಧಾರಕ್ಕಾಗಿ ಮತ್ತೊಂದು ಸೃಷ್ಟಿಯ ಅವಕಾಶವನ್ನು ಕೊಡುವನು. ಇಷ್ಟೊಂದು ಕರುಣೆ ಇದ್ದರೆ ಸುಮ್ಮನೆ ಅವನು ಏತಕ್ಕೆ ಒಂದೇ ಸಲ ಮುಕ್ತಿಯನ್ನು ಕೊಡಬಾರದು? ಕೊಟ್ಟು ಅವರನ್ನು ಪ್ರಕೃತಿಯ ಜೈಲಿನಿಂದ ಪಾರು ಮಾಡಬಾರದೇಕೆ!\\ಕಷ್ಟಪಟ್ಟು ಸಾಧನೆ ಮಾಡುವವನಿಗೂ ಮುಕ್ತಿ, ಏನೂ ಮಾಡದೆ ಸುಮ್ಮನೆ ಇಂದ್ರಿಯ ಪ್ರಪಂಚದಲ್ಲಿ ವಿಹರಿಸುವವನಿಗೂ ಮುಕ್ತಿ ಕೊಟ್ಟ ಎಂದರೆ ಕಷ್ಟಪಟ್ಟು ಓದುವವನೂ ಪಾಸು ಆಗುತ್ತಾನೆ. ಸುಮ್ಮನೆ ಓದದೆ ಹರಟೆ ಹೊಡೆಯುತ್ತಿರುವವನೂ ಪಾಸು ಆಗುತ್ತಾನೆ ಎಂದಾದರೆ ಯಾರೂ ಓದುವುದಕ್ಕೆ ಹೋಗುವುದಿಲ್ಲ. ಆಗಲೂ ದೇವರನ್ನು ದೂರುವುದನ್ನು ನಾವು ಬಿಡುವುದಿಲ್ಲ. ಆಗ ದೇವರನ್ನು ಪಕ್ಷಪಾತಿ ಎನ್ನುತ್ತೇವೆ. ದೇವರಿಗೂ ಸೃಷ್ಟಿಗೂ ಇರುವ ಸಂಬಂಧವನ್ನು ನಾವು ಹೇಗೆ ನೋಡಿದರೂ ಯಾವುದಾದರೂ ದೋಷ ಕಂಡೇ ಕಾಣುತ್ತೇವೆ. ಆ ದೋಷಗಳೆಲ್ಲ ಕಡಮೆ ಎನ್ನಬಹುದು. ಶ್ರೀಕೃಷ್ಣ ಕೊಡುವ ಕಾರಣದಲ್ಲಿ–ತನಗಾಗಿ ಸೃಷ್ಟಿ ಮಾಡುತ್ತಿಲ್ಲ ಜೀವರಾಶಿಗಳಿಗಾಗಿ ಸೃಷ್ಟಿ ಮಾಡುತ್ತೇನೆ. ಅದರಿಂದ ಅವನು ಕೊನೆಗೆ ಯಾವ ಫಲಕ್ಕೂ ಕೈ ಒಡ್ಡುವುದಿಲ್ಲ ಅಥವಾ ಅದನ್ನು ಮಾಡುತ್ತಿರು ವಾಗಲೂ ಆ ಕೆಲಸದಲ್ಲಿ ಆಸಕ್ತಿಯಿಲ್ಲ, ಉದಾಸೀನನಾಗಿ ಮಾಡುತ್ತಾನೆ.

\begin{verse}
ಮಯಾಧ್ಯಕ್ಷೇಣ ಪ್ರಕೃತಿಃ ಸೂಯತೇ ಸಚರಾಚರಮ್​।\\ಹೇತುನಾನೇನ ಕೌಂತೇಯ ಜಗದ್ವಿಪರಿವರ್ತತೇ \versenum{॥ ೧೦ ॥}
\end{verse}

{\small ಅರ್ಜುನ, ಅಧ್ಯಕ್ಷನಾದ ನನ್ನಿಂದ ಪ್ರಕೃತಿಯು ಚರಾಚರಾತ್ಮಕವಾದ ಜಗತ್ತನ್ನು ಉತ್ಪತ್ತಿ ಮಾಡುವುದು. ಈ ಕಾರಣದಿಂದ ಜಗತ್ತು ಚಕ್ರದಂತೆ ಸುತ್ತುತ್ತಿದೆ.}

ಪ್ರಕೃತಿ ಜಡ. ಅದು ಎಷ್ಟೇ ಅದ್ಭುತವಾಗಿದ್ದರೂ ಅದೊಂದು ಯಂತ್ರದಂತೆ. ಭಗವಂತನೆಂಬ ಚೈತನ್ಯ ಹಿಂದೆ ಇದ್ದರೇನೇ ಅದು ಕೆಲಸ ಮಾಡಬಲ್ಲುದು. ಭಗವಂತನೇ ಯಂತ್ರವನ್ನು ಚಲಿಸು ವವನು. ಅದನ್ನು ಒಂದು ಉದ್ದೇಶಕ್ಕಾಗಿ ನಿರ್ಮಿಸಿದ್ದಾನೆ. ಅದು ಬದ್ಧ ಜೀವಿಗಳನ್ನು ಒಂದು ಕಡೆಯಿಂದ ತೆಗೆದುಕೊಂಡು ಹಲವು ಅನುಭವಗಳ ಮೂಲಕ ಸಾಗಿ ಹೋಗುವಂತೆ ಮಾಡಿ ಪೂರ್ಣಾತ್ಮರನ್ನಾಗಿ ಮಾಡಿ ಆಚೆಗೆ ಎಸೆಯುತ್ತದೆ. ಇದೇ ಪ್ರಕೃತಿಯ ಕೆಲಸ. ಇದೇ ಬ್ರಹ್ಮಾಂಡವನ್ನು ಸೃಷ್ಟಿಸಿದೆ. ಅಂದರೆ, ಶೂನ್ಯದಿಂದ ಇದನ್ನು ತೆಗೆದುಕೊಂಡು ಬರಲಿಲ್ಲ. ಯಾವುದು ಸೂಕ್ಷ್ಮಾವಸ್ಥೆ ಯಲ್ಲಿತ್ತೋ ಅದನ್ನು ಸ್ಥೂಲರೂಪಕ್ಕೆ ತರುವುದು. ಇದು ಜೀವಿಗಳನ್ನು ಏನೂ ಹೊಸದಾಗಿ ಸೃಷ್ಟಿಸುವುದಿಲ್ಲ. ಯಾವ ಜೀವಿಗಳು ಪ್ರಳಯಕಾಲದಲ್ಲಿ ಅವ್ಯಕ್ತಾವಸ್ಥೆಯಲ್ಲಿರುವವೋ ಅವು ಪುನಃ ಸ್ಥೂಲರೂಪಕ್ಕೆ ಬರುವಂತೆ ಮಾಡುವುದು. ತಮ್ಮ ಕರ್ಮವನ್ನು ಸವೆಸಲು ಹಲವಾರು ಹೊಸ ಹೊಸ ವೇಷಗಳನ್ನು ಧರಿಸುವುವು, ನೂತನ ಅನುಭವಗಳನ್ನು ಪಡೆಯುವವು. ಪ್ರಕೃತಿಯನ್ನು ನಾವು ದೊಡ್ಡದೊಂದು ಶಾಲೆಗೆ ಹೋಲಿಸಬಹುದು. ಅಲ್ಲಿ ಹಲವಾರು ತರಗತಿಗಳಿವೆ. ಅಕ್ಷರಾಭ್ಯಾಸದಿಂದ ಹಿಡಿದು ಡಾಕ್ಟರೇಟ್ ಪದವಿಯನ್ನು ಪಡೆಯುವವರೆಗೆ ಅಲ್ಲಿ ಓದಬಹುದು. ಜೀವಿಗಳು ಕ್ಲಾಸಿನಿಂದ ಕ್ಲಾಸಿಗೆ ಹೋಗುತ್ತಿರುವುವು. ಇನ್ನು ಕಲಿಯುವುದೇನೂ ಇಲ್ಲ ಎಂದ ಮೇಲೆ ಶಾಲೆಯನ್ನು ಬಿಟ್ಟು ಹೋಗುವುವು. ಇಲ್ಲಿ ಶಾಲೆ ಮಕ್ಕಳನ್ನು ಸೃಷ್ಟಿ ಮಾಡುವುದಿಲ್ಲ. ಹೊರಗಿರುವ ಮಕ್ಕಳೆಲ್ಲಾ ಇಲ್ಲಿ ಕಲಿಯುವುದಕ್ಕೆ ಬರುತ್ತವೆ. ಅದರಂತೆಯೇ ಜೀವರಾಶಿಗಳು ಅನಾದಿ ಮತ್ತು ಅನಂತ. ಇವುಗಳೆಲ್ಲಾ ಸೃಷ್ಟಿಯ ಶಾಲೆ ಮೂಲಕ ಸಾಗಿ ಹೋಗುತ್ತವೆ.

ಈ ಜಗತ್ತು ಚಕ್ರದಂತೆ ಉರುಳುತ್ತಿದೆ ಎನ್ನುತ್ತಾನೆ. ಈ ವಿಶ್ವವೆಲ್ಲ ಉರುಳುತ್ತಿರುವ ಗೋಳಗಳ ಸಂತೆ. ಸೂಕ್ಷ್ಮಾತಿ ಸೂಕ್ಷ್ಮಕಣದಿಂದ ಹಿಡಿದು ದೊಡ್ಡ ದೊಡ್ಡ ಸೂರ್ಯರನ್ನೊಳಗೊಂಡ ನೀಹಾರಿಕೆವರೆಗೆ ಎಲ್ಲವೂ ಒಂದನ್ನೊಂದನ್ನು ಪ್ರದಕ್ಷಿಣೆ ಮಾಡುತ್ತಿವೆ. ಎಲ್ಲೂ ತೆಪ್ಪಗೆ ಯಾವುದೂ ಕುಳಿತಿಲ್ಲ. ಅದು ಚಲಿಸುತ್ತಿರುವುದು ನಮಗೆ ಕಾಣಿಸುವುದಿಲ್ಲ ಅಷ್ಟೆ. ಸುಮಾರು ನಿಮಿಷಕ್ಕೆ ಇಪ್ಪತ್ತು ಮೈಲಿ ವೇಗದಲ್ಲಿ ಸೂರ್ಯನ ಸುತ್ತಲೂ ಉರುಳುತ್ತಿರುವ ಈ ಭೂಮಿ ನಮ್ಮ ಕಣ್ಣಿಗೆ ತಟಸ್ಥವಾಗಿರುವಂತಿದೆ. ದೊಡ್ಡ ದೊಂದು ಫ್ಯಾಕ್ಟರಿಯಲ್ಲಿ ಅಲ್ಲಿರುವ ಸಣ್ಣ ದೊಡ್ಡ ಚಕ್ರಗಳೆಲ್ಲ ಹೇಗೆ ಉರುಳುತ್ತಿವೆಯೊ ಹಾಗೆ ಉರುಳುತ್ತಿವೆ ಈ ಬ್ರಹ್ಮಾಂಡದಲ್ಲಿ ಇರುವುದೆಲ್ಲಾ. ಸೃಷ್ಟಿಗೆ ಚಲನೆ ಅತಿ ಮುಖ್ಯ. ಚಲಿಸಿದರೇ ಪರಿಣಾಮಗಳು, ಬದಲಾವಣೆಗಳು ಆಗಬೇಕಾದರೆ. ಕಾಲದ ಭಾವನೆ ಬರಬೇಕಾದರೂ ಚಲಿಸು ತ್ತಿರಬೇಕು.

\begin{verse}
ಅವಜಾನಂತಿ ಮಾಂ ಮೂಢಾ ಮಾನುಷೀಂ ತನುಮಾಶ್ರಿತಮ್ ।\\ಪರಂ ಭಾವಮಜಾನಂತೋ ಮಮ ಭೂತಮಹೇಶ್ವರಮ್ \versenum{॥ ೧೧ ॥}
\end{verse}

{\small ಸರ್ವ ಭೂತಗಳಿಗೂ ಮಹೇಶ್ವರನಾದ ನನ್ನ ಪರಭಾವವನ್ನು ಅರಿಯದ ಮೂಢರು ಮನುಷ್ಯ ದೇಹವನ್ನು ಆಶ್ರಯಿಸುವ ನನ್ನನ್ನು ತಿರಸ್ಕರಿಸುತ್ತಾರೆ.}

ಶ್ರೀಕೃಷ್ಣನಿಗೆ ಎರಡು ವ್ಯಕ್ತಿತ್ವಗಳಿವೆ. ಒಂದು ಅವನು ದೇವಕಿ ವಸುದೇವನ ಪುತ್ರನಂತಿರುವನು. ಇದು ಚಾರಿತ್ರಿಕ ಘಟನೆ. ಕೆಲವು ವರ್ಷಗಳ ಹಿಂದೆ ಹುಟ್ಟಿದ್ದಾನೆ. ಇನ್ನು ಕೆಲವು ವರುಷಗಳಾದ ಮೇಲೆ ತೀರಿಹೋಗುತ್ತಾನೆ. ಅವನು ಇಲ್ಲಿ ಬದುಕಿರುವಾಗ ಕೆಲವು ಒಳ್ಳೆಯ ಕೆಲಸಗಳನ್ನು ಮಾಡಿದ್ದಾನೆ. ಇತರರು ಕೆಟ್ಟ ಕೆಲಸ ಎಂದು ಭಾವಿಸುವ ಕೆಲಸವನ್ನೂ ಮಾಡಿದ್ದಾನೆ. ಇದರ ಹಿಂದೆ ಶ್ರೀಕೃಷ್ಣನಲ್ಲಿ ಮತ್ತೊಂದು ವ್ಯಕ್ತಿತ್ವವಿದೆ. ಅದೇ ಪರಮಾತ್ಮನದು. ಅವನು ಎಂದೂ ಹುಟ್ಟಿದವನಲ್ಲ, ಸಾಯು ವವನೂ ಅಲ್ಲ. ಒಂದು ವ್ಯಕ್ತಿಯಂತೆ ಕಂಡರೂ ಅವನು ಅದರಲ್ಲಿ ಖಾಲಿಯಾಗಿ ಹೋಗಲಾರದವನು. ಅವನು ಸರ್ವಾಂತರ್ಯಾಮಿ, ಸರ್ವಜ್ಞ, ಸರ್ವಶಕ್ತ. ಇದು ನಾಟಕದಲ್ಲಿ ಪಾತ್ರ ಮಾಡುವವನಿಗೆ ಇರುವ ಎರಡು ವ್ಯಕ್ತಿತ್ವದಂತೆ. ಅವನು ಅಭಿನಯಿಸುತ್ತಿರುವ ವ್ಯಕ್ತಿತ್ವ ಒಂದು, ಮತ್ತು ಅದರ ಹಿಂದೆ ಇರುವ ಅವನ ನಿಜವಾದ ವ್ಯಕ್ತಿತ್ವ ಒಂದು. ಶ್ರೀಕೃಷ್ಣನಲ್ಲಿರುವ ಪರಮಾತ್ಮನ ವ್ಯಕ್ತಿತ್ವವನ್ನು ಎಲ್ಲರಿಗೂ ತಿಳಿದುಕೊಳ್ಳುವುದಕ್ಕೆ ಸಾಧ್ಯವಿರಲಿಲ್ಲ. ಏಕೆಂದರೆ ಪರಮಾತ್ಮ ಅವತಾರವಾಗಿ ಜನ್ಮವೆತ್ತುವಾಗ ಹೊರಗಿನಿಂದ ನೋಡಿದರೆ ಅವನು ಸಾಧಾರಣ ಮನುಷ್ಯನಂತೆಯೇ ಇರುವನು. ಅವರಂತೆಯೇ ಮಾತುಕತೆ ವ್ಯವಹಾರ ಎಲ್ಲವನ್ನೂ ಮಾಡುತ್ತಾನೆ. ಆದರೆ ಅದರ ಹಿಂದೆ ದೈವತ್ವ ಸದಾ ಪ್ರಕಟಿತ ವಾಗಿರುವುದು. ಅದನ್ನು ತಿಳಿದುಕೊಳ್ಳಬೇಕಾದರೆ ಒಬ್ಬ ಆಧ್ಯಾತ್ಮಿಕ ಜೀವನದಲ್ಲಿ ಬಹಳ ಮುಂದು ವರಿದು ಹೋಗಿರಬೇಕು. ಅಂತಹವನಿಗೆ ಅದು ಗೊತ್ತಾಗುವುದು. ಈಗ ಬೇಕಾದಷ್ಟು ಜನ ಕೃಷ್ಣನನ್ನು ಅವತಾರ ಎಂದು ಭಾವಿಸಬಹುದು. ಏಕೆಂದರೆ ದೈವತ್ವದ ಪೀಠ ಆ ವ್ಯಕ್ತಿಗೆ ಬಂದು ಹೋಗಿದೆ. ಆದರೆ ಶ್ರೀಕೃಷ್ಣನ ಕಾಲದಲ್ಲಿ ಅದಿನ್ನೂ ಬಂದಿರಲಿಲ್ಲ. ಎಲ್ಲೊ ಕೆಲವು ಜ್ಞಾನಿಗಳು ಪುಷಿಗಳು ಮಾತ್ರ ಶ್ರೀಕೃಷ್ಣನ ಮಹಿಮೆಯನ್ನು ಬಲ್ಲವರಾಗಿದ್ದರು. ಯುಧಿಷ್ಠಿರನ ರಾಜಸೂಯಯಾಗದಲ್ಲಿ ಅಗ್ರ ಪೂಜೆಗೆ ಶ್ರೀಕೃಷ್ಣನೇ ಯೋಗ್ಯ ಎಂದು ಭೀಷ್ಮರು ಸಾರುತ್ತಾರೆ. ಆದರೆ ಶಿಶುಪಾಲ ಎದ್ದು ಆ ತುಂಬಿದ ಸಭೆಯಲ್ಲಿ ಶ್ರೀಕೃಷ್ಣನನ್ನು ಟೀಕಿಸುತ್ತಾನೆ. ಆಗ ಭೀಷ್ಮರು ಶ್ರೀಕೃಷ್ಣ ಯಾರು ಎಂಬುದನ್ನು ಹೇಳಬೇಕಾಯಿತು. ಆದರೂ ಎಲ್ಲರೂ ಒಪ್ಪಿದಂತೆ ಕಾಣೆ. ಅಂತೂ ಧರ್ಮರಾಯನ ಇಚ್ಛೆಯಂತೆ ಅವನಿಗೆ ಅಗ್ರಪೂಜೆ ಸನ್ಮಾನವೇನೋ ನಡೆಯಿತು. ಪರಮಾತ್ಮ ಅವತಾರವಾಗಿ ಬಂದಾಗ ಅವನನ್ನು ಗುರುತು ಹಿಡಿಯಬೇಕಾದರೆ ನಮ್ಮಲ್ಲಿಯೂ ಪರಮಾತ್ಮನ ಅಂಶ ಜಾಗೃತವಾಗಿದ್ದರೆ ಮಾತ್ರ ಸಾಧ್ಯ. ವಜ್ರದ ವ್ಯಾಪಾರಿ ಮಾತ್ರ ವಜ್ರದ ಬೆಲೆ ತಿಳಿಯಬಲ್ಲ, ಬಳೆಯ ವ್ಯಾಪಾರಿ ವಜ್ರದ ಬೆಲೆಯನ್ನು ಹೇಗೆ ಅರಿಯುತ್ತಾನೆ? ಆದಕಾರಣವೇ ಪ್ರಪಂಚದಲ್ಲಿರುವ ಬಹುಪಾಲು ಜನರು ಅಜ್ಞರು, ಶ್ರೀಕೃಷ್ಣನನ್ನು ತಿರಸ್ಕರಿಸುತ್ತಿದ್ದರು.

\begin{verse}
ಮೋಘಾಶಾ ಮೋಘಕರ್ಮಾಣೋ ಮೋಘಜ್ಞಾನಾ ವಿಚೇತಸಃ ।\\ರಾಕ್ಷಸೀಮಾಸುರೀಂ ಚೈವ ಪ್ರಕೃತಿಂ ಮೋಹಿನೀಂ ಶ್ರಿತಾಃ \versenum{॥ ೧೨ ॥}
\end{verse}

{\small ವ್ಯರ್ಥ ಆಸೆ ಪಡುವವರು ವ್ಯರ್ಥ ಕೆಲಸ ಮಾಡುವವರು, ವ್ಯರ್ಥ ಜ್ಞಾನದ ಮೂಢ ಜನರು, ಮೋಹಾಸಕ್ತಿಯಲ್ಲಿ ಕೆಡುಹುವ ರಾಕ್ಷಸೀ ಆಸುರೀ ಪ್ರಕೃತಿಯ ಆಶ್ರಯವನ್ನು ಹೊಂದುತ್ತಾರೆ.}

ರಾಕ್ಷಸೀ ಮತ್ತು ಆಸುರೀ ಸ್ವಭಾವದ ಜನ ರಜಸ್ಸು ಮತ್ತು ತಮೋಗುಣಗಳನ್ನು ಆಶ್ರಯಿಸುತ್ತಾರೆ. ಅವರ ಬದುಕು ಹೇಗೆ ಎಂಬುದನ್ನು ಶ್ರೀಕೃಷ್ಣ ವಿವರಿಸುತ್ತಾನೆ. ಅವರ ಆಸೆ ವ್ಯರ್ಥ. ಅವರಿಗೆ ತಾತ್ಕಾಲಿಕ ಇಂದ್ರಿಯ ಸುಖ ಒಂದೇ ಸತ್ಯ. ಧರ್ಮ, ಕರ್ಮ, ದೇವರು ಮುಂತಾದುವುಗಳೆಲ್ಲಾ ಅವರ ಬಾಳಿಗೆ ಅನಾವಶ್ಯಕ. ಅವರದು ತಾತ್ಕಾಲಿಕ ದೃಷ್ಟಿ. ಸ್ಥೂಲವನ್ನು ಮಾತ್ರ ತಿಳಿದುಕೊಳ್ಳಬಲ್ಲರು. ಇಂದ್ರಿಯ ಮುಷ್ಟಿಗೆ ಸಿಲುಕಿದರೆ ಮಾತ್ರ ಸತ್ಯ, ಇಲ್ಲದೇ ಇದ್ದರೆ ಅದು ಸತ್ಯವಲ್ಲ. ಇಂತಹವರು ಆಸೆ ಪಡುವುದೆಲ್ಲ ಈ ಪ್ರಪಂಚದಲ್ಲಿ ಸುಖವಾಗಿರಬೇಕೆಂಬುದಕ್ಕೆ. ಅದಕ್ಕಾಗಿ ಪುಷ್ಟವಾದ ದೇಹ ಇಂದ್ರಿಯಗಳ ಮೂಲಕ ಅನುಭವಿಸುವುದಕ್ಕೆ ಬೇಕಾದಷ್ಟು ವಿಷಯ ವಸ್ತುಗಳು ಇರಬೇಕು. ತನ್ನ ಸುಖವೊಂದೇ ಮುಖ್ಯ. ಅನಂತರದ ಜೀವನದಲ್ಲಿ ಮತ್ತು ಕರ್ಮದಲ್ಲಿ ನಂಬಿಕೆ ಇಲ್ಲ. ಅವನು ಯಾವ ರೀತಿಯಿಂದಲಾದರೂ ಇಂದ್ರಿಯ ಚಪಲವನ್ನು ತೀರಿಸಿಕೊಳ್ಳಲು ಯತ್ನಿಸುವನು.

ಅವನ ಕರ್ಮ ವ್ಯರ್ಥವಾಗುವುದು. ಸ್ವಾರ್ಥ ಸಾಧನೆಗೆ ಅವನು ಇನ್ನೊಬ್ಬನಿಗೆ ಕಿರುಕುಳ ಕೊಡಲೂ ಹಿಂದು ಮುಂದು ನೋಡುವುದಿಲ್ಲ. ಇನ್ನೊಬ್ಬರ ಗೋಳಿನ ಮೇಲೆ, ಸಂಕಟದ ಮೇಲೆ ತನ್ನ ಸೌಖ್ಯದ ಮನೆಯನ್ನು ಕಟ್ಟುತ್ತಾನೆ. ತನಗೆ ಪ್ರಿಯವಾಗಿರುವುದನ್ನೆಲ್ಲಾ ಗಳಿಸಲು ಯತ್ನಿಸುವನು. ಕೆಲವನ್ನು ಸಂಪಾದಿಸಿಯೂ ಇರುವನು. ಆದರೆ ಈ ಕರ್ಮಗಳೆಲ್ಲಾ ವ್ಯರ್ಥವಾಗುವುವು. ಇವನಿಗೆ ಸಿಕ್ಕುವ ಚೂರು ಪಾರು ಸುಖದ ಜೊತೆಗೆ ದುಃಖವೂ ಸಿಕ್ಕುವುದು. ಈ ಜೀವನದಲ್ಲಿ ನಾವು ಕರ್ಮ ನಿಯಮಗಳನ್ನು ಒಪ್ಪಿಕೊಳ್ಳದೇ ಇದ್ದರೂ ಅದು ನಮ್ಮ ಮೇಲೆ ತನ್ನ ಶಕ್ತಿಯನ್ನು ಬೀರಿಯೇ ಬೀರುವುದು. ನಾವು ಭೂಮಿಯ ಆಕರ್ಷಣ ಸಿದ್ಧಾಂತವನ್ನು ಒಪ್ಪದೇ ಇರಬಹುದು. ಆದರೆ ಆಕರ್ಷಣ ಸಿದ್ಧಾಂತ ನಮ್ಮನ್ನು ಸೆಳೆಯುವದಿಲ್ಲವೇ? ಬೆಂಕಿ ಸುಡುವುದು ಎಂಬುದನ್ನು ನಾವು ನಂಬದೇ ಇದ್ದರೂ ಅದು ನಮ್ಮನ್ನು ಸುಡುವುದು ಬಿಡುವುದಿಲ್ಲ. ಅದರಂತೆಯೇ ಮನುಷ್ಯ ಕೇವಲ ತನ್ನ ಸುಖ ಸಾಧನೆ ಯೊಂದನ್ನೇ ಉದ್ದೇಶವಾಗಿಟ್ಟುಕೊಂಡು ಕರ್ಮಮಾಡಿದರೆ, ಅದರಿಂದ ಇತರರಿಗಾದ ಅನ್ಯಾಯ, ಗೋಳು ಇವುಗಳೆಲ್ಲಾ ತನ್ನ ಜೀವನವನ್ನು ಬೇಡವೆಂದರೂ ಧಾಳಿ ಇಡುವುದು. ನಮ್ಮ ಬಾಳಿನ ಹಾಲಿಗೆ ಹುಳಿ ಹಿಂಡುವುದು ಇದು. ನಮ್ಮ ಸ್ವಾರ್ಥ ಸಾಧಿಸಿಕೊಳ್ಳಬೇಕಾದರೂ ಇನ್ನೊಬ್ಬರ ಸ್ವಾರ್ಥಕ್ಕೆ ಅವಕಾಶ ಕೊಟ್ಟರೆ ತಾನೇ ಸಾಧ್ಯ. ಆಗ ಸ್ವಾರ್ಥ ವಿಸ್ತಾರವಾಗುವುದು. ಆದರೆ ಆಸುರೀ ಪ್ರವೃತ್ತಿಯ ಜನ ಬಡಪೆಟ್ಟಿಗೆ ಇದನ್ನು ಒಪ್ಪುವುದಿಲ್ಲ. ಅವರಿಗೆ ಬೇಕಾಗಿರುವುದು ತಾತ್ಕಾಲಿಕ ಪರಿಣಾಮಗಳು ಅಷ್ಟೆ. ದೀರ್ಘಕಾಲದ ಪರಿಣಾಮಗಳಲ್ಲ.

ಅವರ ಜ್ಞಾನ ವ್ಯರ್ಥ. ಇಂದ್ರಿಯಾತೀತವಾಗಿರುವುದನ್ನು, ಸೂಕ್ಷ್ಮವಾಗಿರುವುದನ್ನು, ಅನಂತರ ಆಗುವುದನ್ನು, ಅವರು ತಿಳಿದುಕೊಳ್ಳಲಾರರು. ಅವರ ಜ್ಞಾನ ಇನ್ನೂ ವಿಕಾಸವಾಗಬೇಕಾಗಿದೆ. ಬುದ್ಧಿ ಮೊನಚಾದರೆ, ಪರಿಶುದ್ಧವಾದರೆ ಮಾತ್ರ ಅದು ಸೂಕ್ಷ್ಮ ನಿಯಮಗಳನ್ನು ಅರ್ಥಮಾಡಿಕೊಳ್ಳ ಬಲ್ಲದು. ಕೇವಲ ಇಂದ್ರಿಯ ಸುಖವೇ ಸಾರ ಸರ್ವಸ್ವ ಎಂದು ಭಾವಿಸುವವನ ಚಿತ್ತ ಹೇಗೆ ಶುದ್ಧವಾಗಿರಬಲ್ಲದು? ಯಾವಾಗಲೂ ಸ್ವಾರ್ಥ ಸುಖವನ್ನು ಬೇಟೆ ಆಡುವುದರಲ್ಲಿ ನಿರತನಾದವನು ಮನಸ್ಸನ್ನು ಹೇಗೆ ಗಹನವಾದ ವಸ್ತುಗಳ ಮೇಲೆ ಕೇಂದ್ರೀಕರಿಸಬಲ್ಲನು? ಇವನಿಗೆ ಸಿಕ್ಕುವ ಜ್ಞಾನವೆಲ್ಲ ತೋರಿಕೆಯದು, ಬರೀ ಜೊಳ್ಳು. ಇವನು ಮೋಹದ ಬಲೆಯಲ್ಲಿ ಬಿದ್ದವನು. ಮೀನು ಗಾಳದ ಕೊನೆಯಲ್ಲಿ ಕಾಣುವ ಹುಳು ಒಂದೇ ಸತ್ಯವೆಂದು ಭ್ರಮಿಸಿ ಅದರೆಡೆಗೆ ಧಾವಿಸಿ ಅದನ್ನು ತಿನ್ನುವುದು. ಕೊನೆಗೆ ತನ್ನ ಪ್ರಾಣವನ್ನು ಅದಕ್ಕೆ ಬಲಿಕೊಡುವುದು. ಅದರಂತೆಯೇ ಅವಿವೇಕಿ ತನ್ನ ಇಂದ್ರಿಯಗಳಿಗೆ ತಾತ್ಕಾಲಿಕವಾಗಿ ತೃಪ್ತಿ ಕೊಡುವ ವೇದನೆಗಳಿಗೆ ಬಲಿಯಾಗಿ ಅದನ್ನು ಪಡೆದು ಅನುಭವಿಸುತ್ತಾನೆ. ಅನಂತರ ಆ ಅನುಭವಕ್ಕೆ ದಾಸನಾಗುತ್ತಾನೆ. ಅದಕ್ಕಾಗಿ ಏನನ್ನು ಬೇಕಾದರೂ ಮಾಡಲು ಸಿದ್ಧನಾಗಿರು ವನು.

\begin{verse}
ಮಹಾತ್ಮಾನಸ್ತು ಮಾಂ ಪಾರ್ಥ ದೈವೀಂ ಪ್ರಕೃತಿಮಾಶ್ರಿತಾಃ ।\\ಭಜಂತ್ಯನನ್ಯಮನಸೋ ಜ್ಞಾತ್ವಾ ಭೂತಾದಿಮವ್ಯಯಮ್ \versenum{॥ ೧೩ ॥}
\end{verse}

{\small ಅರ್ಜುನ, ದೈವೀ ಸ್ವಭಾವವನ್ನು ಆಶ್ರಯಿಸಿದ ಮಹಾತ್ಮರು, ಸರ್ವಭೂತಗಳಿಗೆ ಕಾರಣನೂ ಅವ್ಯಯನೂ ಆದ ನನ್ನನ್ನು ಅನನ್ಯಮನಸ್ಕರಾಗಿ ಭಜಿಸುತ್ತಾರೆ.}

ಪ್ರಕೃತಿಯಲ್ಲಿ ಆಸುರೀ ಸ್ವಭಾವ ಮತ್ತು ದೈವೀ ಸ್ವಭಾವ ಎರಡೂ ಇವೆ. ಆಸುರೀ ಸ್ವಭಾವ ಅಜ್ಞಾನದಿಂದ ಕೂಡಿದ್ದು. ಅದು ಪ್ರಪಂಚಕ್ಕೆ ನಮ್ಮನ್ನು ಕಟ್ಟಿಹಾಕುವುದು. ಕಣ್ಣ ಮುಂದೆ ಇರುವುದು, ಈಗಿರುವುದು ಇದೇ ಸತ್ಯವೆಂಬಂತೆ ನಮಗೆ ಕಾಣುವುದು. ದೈವೀ ಸ್ವಭಾವವಾದರೊ ನಮಗೆ ಹಾಕಿದ ಕಟ್ಟನ್ನು ಬಿಡಿಸುವುದು, ಗಹನವಾದ ಸೂಕ್ಷ್ಮ ಆಧ್ಯಾತ್ಮಿಕ ಸತ್ಯಗಳು ನಮಗೆ ಕಾಣುವಂತೆ ಮಾಡುವುದು. ಈ ಸ್ವಭಾವ ನಮ್ಮನ್ನು ಭಗವಂತನ ಕಡೆಗೆ ಕರೆದುಕೊಂಡು ಹೋಗುವುದು. ಮಹಾತ್ಮರು ಈ ಸ್ವಭಾವವನ್ನು ಆಶ್ರಯಿಸುವರು. ಅದೇ ಗಾಳಿದೋಣಿಯನ್ನು ನೂಕಿಕೊಂಡು ಹೋಗುವಂತೆ ಜೀವಿ ಗಳನ್ನು ಭಗವಂತನ ಕಡೆಗೆ ಕರೆದುಕೊಂಡು ಹೋಗುವುದು. ಈ ಪ್ರಪಂಚದಲ್ಲಿ ಯಾವಾಗಲೂ ಎರಡು ಶಕ್ತಿಗಳು ಕೆಲಸ ಮಾಡುತ್ತಿವೆ. ಈ ಪ್ರಪಂಚದಲ್ಲಿ ಒಂದು ಕಟ್ಟಿಹಾಕುವುದು ಮತ್ತೊಂದು ಬಿಡಿಸುವುದು. ನಾವು ಯಾವುದನ್ನು ಆರಿಸಿಕೊಳ್ಳುತ್ತೇವೆಯೊ ಅದು ತನ್ನ ಧರ್ಮವನ್ನು ನಮ್ಮ ಮೇಲೆ ಬೀರುವುದು. ಬೆಂಕಿ ಮುಟ್ಟಿದರೆ ಸುಡುವುದು. ಹಿಮದಗಡ್ಡೆ ಮುಟ್ಟಿದರೆ ಚಳಿಯಿಂದ ನಮ್ಮ ಕೈ ಕೊರೆಯುವುದು. ಅವಕ್ಕೇನೂ ನಮ್ಮ ಮೇಲೆ ಕೋಪವಿಲ್ಲ, ಪ್ರೀತಿಯೂ ಇಲ್ಲ. ತಮ್ಮ ತಮ್ಮ ಧರ್ಮಾನುಸಾರ ನಮ್ಮ ಮೇಲೆ ವ್ಯವಹರಿಸುವುವು. ಕಬ್ಬು ತಿಂದರೆ ಸಿಹಿಯಾಗಿರುವುದು. ಮೆಣಸಿನ ಕಾಯಿ ತಿಂದರೆ ಖಾರವೆನಿಸುವುದು. ಹಾಗೆಯೆ ಪ್ರಕೃತಿಯಲ್ಲಿರುವ ಆಸುರೀ ಗುಣ ಮತ್ತು ದೈವೀ ಗುಣ. ಅಜ್ಞಾನಿಗಳು ಆಸುರೀ ಗುಣವನ್ನು ಆರಿಸಿಕೊಂಡಿರುವರು. ಅದಕ್ಕನುಸಾರವಾಗಿ ಜೀವನವನ್ನು ಸಾಗಿಸುತ್ತಿರುವರು. ಮಹಾತ್ಮರು ದೈವೀ ಸ್ವಭಾವವನ್ನು ಆಶ್ರಯಿಸಿರುವರು. ಅವರಿಗೆ ಆಸುರೀ ಸ್ವಭಾವ ಏನು ಮಾಡುವುದು, ಯಾವ ರೀತಿ ಪರ್ಯವಸಾನವಾಗುವುದು ಎಂಬುದು ಗೊತ್ತು. ಆದಕಾರಣವೇ ಅದನ್ನು ಬಿಟ್ಟಿರುವರು. ದೈವೀ ಸ್ವಭಾವವನ್ನು ಆಶ್ರಯಿಸುವವರು ಜೀವನದಲ್ಲಿ ಉತ್ತಮವಾದ ಗುಣಗಳನ್ನು ರೂಢಿಸಿಕೊಂಡಿರುವರು. ಅವರ ಹೃದಯ ಶುದ್ಧವಾಗಿದೆ, ಚಿತ್ತ ಏಕಾಗ್ರವಾಗಿದೆ. ಈ ದೃಶ್ಯವಸ್ತುಗಳ ಹಿಂದೆ ಇರುವ ವಿಶ್ವಚೈತನ್ಯ ಹೇಗೆ ಕೆಲಸ ಮಾಡುತ್ತಿದೆ ಎಂಬುದು ಅವರ ಅರಿವಿಗೆ ಬಂದಿದೆ. ಭಗವಂತ ಈ ಪ್ರಪಂಚದಲ್ಲಿ ಕಾಣುವ ಎಲ್ಲಾ ವಸ್ತುಗಳಿಗೂ ಕಾರಣ. ಅವನನ್ನು ಮೀರಿ ಯಾವುದೂ ಹೋಗಲಾರದು ಎಂಬುದನ್ನು ಹೃದಯದಲ್ಲಿ ಅನುಭವಿಸುತ್ತಿರುವರು. ಈ ಪ್ರಪಂಚದಲ್ಲಿರುವ ಭೂತಗಳೆಲ್ಲಾ ಬಂದಿರುವುದು ಭಗವಂತನಿಂದ. ಅವನಿಂದ ಬಂದಿರುವ ವಸ್ತುಗಳೆಲ್ಲ ಬದಲಾಯಿಸುತ್ತಿವೆ. ಬದಲಾವಣೆ ಎಂಬ ಚಕ್ರದಲ್ಲಿ ಸಿಕ್ಕಿ ಸುತ್ತುತ್ತಿವೆ. ಆದರೆ ಅವನು ಮಾತ್ರ ಬದಲಾಯಿಸುವುದಿಲ್ಲ. ಇದು ಅಕ್ಕಸಾಲಿಗನ ಹತ್ತಿರ ಇರುವ ಅಡಿಗಲ್ಲಿನಂತೆ. ಅಕ್ಕಸಾಲಿಗ ಬೆಳಗ್ಗಿನಿಂದ ಸಾಯಂಕಾಲದ ವರೆಗೆ ಅಡಿಗಲ್ಲಿನ ಮೇಲೆ ಹಲವು ವಸ್ತುಗಳನ್ನು ಇಟ್ಟು ಬಡಿದು ಬೇರೆ ಬೇರೆ ಆಕಾರಗಳನ್ನು ಬಿಡಿಸುತ್ತಿರುವನು. ಆದರೆ ಅಡಿಗಲ್ಲಾದರೊ ಯಾವಾಗಲೂ ಹಾಗೆಯೇ ಇರುವುದು. ಬದಲಾವಣೆಗೆ ಆಧಾರ ಅಡಿಗಲ್ಲು. ಆದರೆ ಅದು ಬದಲಾಯಿಸುವುದಿಲ್ಲ. ಇದನ್ನು ದೈವೀ ಸ್ವಭಾವದಿಂದ ಕೂಡಿದ ಮಹಾತ್ಮರು ಚೆನ್ನಾಗಿ ತಿಳಿದುಕೊಂಡಿರುವರು. ಇದೆಲ್ಲಾ ಬರೀ ಪುಸ್ತಕ ಜ್ಞಾನವಲ್ಲ. ವಿಚಾರದಿಂದ ಸಮರ್ಥಿಸಿದ್ದು ಮಾತ್ರವಲ್ಲ. ತಮ್ಮ ಅನುಭವದ ಮುಷ್ಟಿಗೆ ಸಿಕ್ಕಿದ ಸತ್ಯ. ಇದನ್ನು ಅರಿತವರು ಭಗವಂತನನ್ನು ಅನನ್ಯ ಮನಸ್ಸಿನಿಂದ ಭಜಿಸುವರು. ಅವರು ಎಲ್ಲವನ್ನೂ ಬಿಟ್ಟು ಭಗವಂತನ ಮೇಲೆಯೇ ಸದಾ ಮನಸ್ಸನ್ನೆಲ್ಲ ಇಟ್ಟಿರುವರು. ಅವರ ಮನಸ್ಸು ನದಿ ಹೇಗೆ ಸಾಗರಕ್ಕೆ ಒಂದೇ ಸಮನಾಗಿ ಹರಿದುಕೊಂಡು ಹೋಗುತ್ತಿದೆಯೊ ಹಾಗೆ ಭಗವಂತನ ಕಡೆಗೆ ಹರಿದುಹೋಗು ತ್ತಿದೆ. ಕೆಲವು ಕಾಲ ಹರಿಯುವುದು, ಕೆಲವು ಕಾಲ ಹರಿಯದಿರುವುದಲ್ಲ. ಅದು ಸಾಧಾರಣ ಮನುಷ್ಯನ ಸ್ಥಿತಿ. ಆದರೆ ಅನನ್ಯ ಭಾವದಿಂದ ಕೂಡಿದವನು ಹಗಲೂ ರಾತ್ರಿ ಸುಖದಲ್ಲಿ ದುಃಖದಲ್ಲಿ ಲಾಭನಷ್ಟದಲ್ಲಿ, ಭಗವಂತನನ್ನು ಮರೆಯದೆ ಅವನನ್ನು ಕುರಿತು ಚಿಂತಿಸುತ್ತಿರುವನು.

ಹಾಗೆ ಅವನನ್ನು ಕುರಿತು ಚಿಂತಿಸುವಾಗ ಬರೇ ಯಾಂತ್ರಿಕವಾಗಿ ಅವನನ್ನು ಚಿಂತಿಸುವುದಿಲ್ಲ, ನೀರಸವಾಗಿರುವುದಿಲ್ಲ. ಆದರೆ ಅವನನ್ನು ಭಕ್ತಿಯಿಂದ ತುಂಬಿ ತುಳುಕಾಡುತ್ತ ನೆನೆಯುತ್ತಿರುವರು. ಅವರ ಹೃದಯ ಭಕ್ತಿ ಎಂಬ ಜೇನಿನಿಂದ ಕೂಡಿದೆ. ಈ ಜೀವನದಲ್ಲಿ ಭಗವಂತ ಅಮೃತಮಯ. ಅವನನ್ನು ಕುರಿತು ಚಿಂತಿಸುವಾಗ ಇವನ ಹೃದಯವು ಅಮೃತದಿಂದ ತೊಟ್ಟಿಕ್ಕುತ್ತಿರುವುದು.

\begin{verse}
ಸತತಂ ಕೀರ್ತಯಂತೋ ಮಾಂ ಯತಂತಶ್ಚ ದೃಢವ್ರತಾಃ ।\\ನಮಸ್ಯಂತಶ್ಚ ಮಾಂ ಭಕ್ತ್ಯಾ ನಿತ್ಯಯುಕ್ತಾ ಉಪಾಸತೇ \versenum{॥ ೧೪ ॥}
\end{verse}

{\small ದೃಢವ್ರತರು ಪ್ರಯತ್ನಶೀಲರು ನಿರಂತರವೂ ನನ್ನ ಕೀರ್ತನೆಯನ್ನು ಮಾಡುತ್ತಾರೆ. ಭಕ್ತಿಯಿಂದ ನನಗೆ ನಮಸ್ಕರಿಸುತ್ತಾರೆ. ನಿತ್ಯವೂ ನನ್ನನ್ನು ಧ್ಯಾನ ಮಾಡುತ್ತಾರೆ. ನನ್ನನ್ನೇ ಉಪಾಸನೆ ಮಾಡುತ್ತಾರೆ.}

ದೈವೀ ಸ್ವಭಾವವನ್ನು ಆಶ್ರಯಿಸಿದ ಭಕ್ತರು ದೃಢವ್ರತರು. ಭಗವಂತನನ್ನು ಪಡೆಯಬೇಕೆಂಬ ವ್ರತವನ್ನು ಕೈಕೊಂಡಿದ್ದರೆ ಯಾವುದನ್ನು ಕಳೆದುಕೊಂಡರೂ ಅವನನ್ನು ಬಿಡುವುದಿಲ್ಲ. ಜೀವನದಲ್ಲಿ ಯಾವುದಾದರೊಂದು ಆದರ್ಶವನ್ನು ಸಾಧಿಸಬೇಕಾದರೆ ದೃಢತೆ ಇರಬೇಕು. ಇಲ್ಲದೇ ಇದ್ದರೆ ನಾವು ಏನನ್ನು ಸಾಧಿಸಲೂ ಆಗುವುದಿಲ್ಲ. ಕಪಿಮುಷ್ಟಿಯಿಂದ ಹಿಡಿದುಕೊಳ್ಳಬೇಕು. ಅಷ್ಟು ಬಲವಾಗಿ ಆದರ್ಶವನ್ನು ಹಿಡಿದಿರಬೇಕು.

ಅವರು ಪ್ರಯತ್ನಶೀಲರೂ ಬೇರೆ ಆಗಿದ್ದಾರೆ. ಜೀವನದಲ್ಲಿ ಪ್ರಯತ್ನವಿಲ್ಲದೆ ಯಾವುದೂ ಸಿದ್ಧಿಸುವುದಿಲ್ಲ. ಅದರಲ್ಲಿಯೂ ಒಳ್ಳೆಯ ಸ್ವಭಾವವನ್ನು ನಾವು ರೂಢಿಸಿಕೊಳ್ಳುವಾಗ ಹಿಂದಿನ ಹೀನ ಸ್ವಭಾವ ಬಡಪೆಟ್ಟಿಗೆ ನಾಶವಾಗುವುದಿಲ್ಲ. ಪದೇ ಪದೇ ಬಂದು ನಮಗೆ ಕಿರುಕುಳ ಕೊಡುತ್ತ ಇರುತ್ತದೆ. ಅದು ಎಷ್ಟು ಸಲ ಆತಂಕವನ್ನು ತಂದೊಡ್ಡಿದರೂ ಬಿಡದೆ ನಾವು ಸತತ ಪ್ರಯತ್ನ ಮಾಡುತ್ತಿರಬೇಕು. ದೇವರ ಕಡೆ ನಮ್ಮ ಮನಸ್ಸನ್ನು ಕಳಿಸಬೇಕಾದರೆ ಪ್ರಯತ್ನಶೀಲರು ಒಂದೆರಡು ಸಲ ಪ್ರಯತ್ನ ಮಾಡಿ ಸಿದ್ಧಿಸದೇ ಇದ್ದರೆ ಬಿಡುವುದಿಲ್ಲ. ಪುನಃ ಪುನಃ ಪ್ರಯತ್ನ ಮಾಡುವರು. ಹೆಚ್ಚು ಛಲದಿಂದ ಪ್ರಯತ್ನ ಮಾಡುವರು. ಬಿಡದೆ ಮಾಡಿದ ಪ್ರಯತ್ನವೆ ನಮ್ಮಲ್ಲಿ ಒಂದು ಬಲವಾದ ಸ್ವಭಾವವಾಗಿ ನಮ್ಮನ್ನು ಮುಂದಕ್ಕೆ ಒಯ್ಯುವ ಶಕ್ತಿಯಾಗುವುದು.

ನಿತ್ಯವೂ ಭಕ್ತರು ಭಗವಂತನ ಕೀರ್ತನೆ ಮಾಡುತ್ತಿರುವರು. ಅವನನ್ನು ಕೊಂಡಾಡುತ್ತಿರುವರು. ಈ ಪ್ರಪಂಚವೆಲ್ಲ ಅವನ ಸೌಂದರ್ಯವನ್ನು, ಅವನ ಚಾತುರ್ಯವನ್ನು ಬಣ್ಣಿಸುತ್ತದೆ. ಇವರ ಬಾಯಿಂದ ಭಗವಂತನ ಮಾತುಕತೆಯಲ್ಲದೆ ಬೇರೆ ಯಾವುದೂ ಬರುವುದಿಲ್ಲ. ಹೃದಯವೆಂಬ ರೇಡಿಯೋ ಯಂತ್ರವನ್ನು ಭಗವಂತನೆಂಬ ಬ್ರಾಡ್​ಕಾಸ್ಟಿಂಗ್ ಸ್ಟೇಷನ್ನಿಗೆ ತಿರುಗಿಸುವರು. ಯಾವಾ ಗಲೂ ಅವನಿಗೆ ಸಂಬಂಧಪಟ್ಟ ಗಾನವೇ ಹೊರಹೊಮ್ಮುವುದು ಅವರ ಹೃದಯದಿಂದ. ಭಕ್ತಿಯಿಂದ ಭಗವಂತನಿಗೆ ತಮ್ಮ ಹೃದಯದಲ್ಲೇ ನಮಸ್ಕರಿಸುವರು. ಭಗವಂತನೆಂಬ ಭೂಮಾನುಭವ ಆದಾಗ ಅವರ ವ್ಯಕ್ತಿತ್ವ ಸ್ವಭಾವತಃ ಅದಕ್ಕೆ ಮಣಿಯುವುದು. ಮಣಿಯಬೇಕಾದರೆ ಅಹಂಕಾರ ಅಡಗಿರಬೇಕು. ಭಕ್ತನಲ್ಲಿ ಅಹಂಕಾರ ನಾಶವಾಗಿಲ್ಲ. ಅವನು ನಾಶ ಮಾಡಲೂ ಇಚ್ಛಿಸುವುದಿಲ್ಲ. ನಾನು ಎಂಬುದನ್ನು ಕೊಬ್ಬಿ ಮೆರೆಸುವುದಿಲ್ಲ. ಭಗವಂತನನ್ನು ಕೊಂಡಾಡುವುದಕ್ಕೆ, ಅವನ ಸೇವೆ ಮಾಡುವುದಕ್ಕೆ, ಅವನ ಭೃತ್ಯನಾಗುವುದಕ್ಕೆ ಈ ನಾನು ಎಂಬುದನ್ನು ಉಪಯೋಗಿಸುವನು.

ನಿತ್ಯವೂ ಭಗವಂತನನ್ನು ಧ್ಯಾನ ಮಾಡುತ್ತಾನೆ. ಧ್ಯಾನಾವಸ್ಥೆಯೇ ಅವನ ಸಹಜ ಸ್ವಭಾವ ವಾಗುವುದು. ಧ್ಯಾನಕ್ಕೆ ಆತ ಪ್ರಯತ್ನ ಪಡಬೇಕಾಗಿಲ್ಲ. ಆವಿ ಹೇಗೆ ಕೆಳಗಿನಿಂದ ಮೇಲಕ್ಕೆ ಹೋಗುತ್ತಿರುವುದೋ, ಹಾಗೆ ಇವನ ಮನಸ್ಸು ಭಗವಂತನ ಕಡೆಗೆ ಏಳುತ್ತಿರುವುದು. ಅವನು ಭಗವಂತನನ್ನು ಎಲ್ಲಾ ಇಂದ್ರಿಯಗಳಿಂದಲೂ ಪೂಜಿಸುತ್ತಾನೆ. ಕಣ್ಣಿಂದ ಅವನ ಅನುಪಮ ಸೌಂದರ್ಯವನ್ನು ನೋಡುತ್ತಾನೆ. ‘ಸಾರ್ಥಕವಾಯಿತು ದೇವರು ನನಗೆ ಕಣ್ಣು ಕೊಟ್ಟಿರುವುದು’ ಎಂದು ಧನ್ಯವಾದಗಳನ್ನು ಅರ್ಪಿಸುತ್ತಾನೆ. ಕಿವಿಯಿಂದ ಅವನ ಮಾತುಕತೆ ಕೇಳುತ್ತಾನೆ. ಕೈಕಾಲು ಗಳಿಂದ ಅವನ ಸೇವೆ ಮಾಡುತ್ತಾನೆ. ಬಾಯಿಂದ ಭಗವಂತನನ್ನು ಕೊಂಡಾಡುತ್ತಾನೆ. ಕಣ್ಣು ಬಿಟ್ಟು ಎದುರಿಗೆ ನೋಡಿದರೆ ವಿಶ್ವದಲ್ಲೆಲ್ಲಾ ಚರಾಚರ ವಸ್ತುಗಳಲ್ಲಿ ಅವನು ಓತಪ್ರೋತವಾಗಿರುವುದನ್ನು ಕಾಣುವನು. ಅಂತರ್ಮುಖನಾಗಿ ಧ್ಯಾನ ಮಾಡಿದರೆ ಒಳಗೆಲ್ಲಾ ಅವನೇ ತುಂಬಿ ತುಳುಕಾಡುತ್ತಿರು ವನು. ಅವನು ನೀರಿನಲ್ಲಿ ಮುಳುಗಿರುವ ಒಂದು ಪಾತ್ರೆಯಂತಾಗುತ್ತಾನೆ. ಪಾತ್ರೆಯ ಒಳಗೆ ನೀರು, ಸುತ್ತಲೂ ನೀರು. ಹಾಗೆಯೇ ಅವನು ಹೊರಗೆ ಬಾಹ್ಯದಲ್ಲಿ ಚರಾಚರ ವಸ್ತುವಿನ ಅಂತರಾಳದಲ್ಲೆಲ್ಲಾ ಭಗವಂತನನ್ನು ಕಾಣುತ್ತಾನೆ. ಅವನ ಜೀವನವೇ ಭಗವಂತನಿಗೆ ಅರ್ಪಿಸಿದ ಮಹಾಕಾವ್ಯದಂತಾಗು ವುದು.

\begin{verse}
ಜ್ಞಾನಯಜ್ಞೇನ ಚಾಪ್ಯನ್ಯೇ ಯಜಂತೋ ಮಾಮುಪಾಸತೇ ।\\ಏಕತ್ವೇನ ಪೃಥಕ್ತ್ವೇನ ಬಹುಧಾ ವಿಶ್ವತೋಮುಖಮ್ \versenum{॥ ೧೫ ॥}
\end{verse}

{\small ಮತ್ತೆ ಕೆಲವರು ಅದ್ವೈತ ರೂಪದಿಂದಲೋ ದ್ವೈತ ರೂಪದಿಂದಲೋ ಅಥವಾ ಅನೇಕ ರೂಪುಗಳಲ್ಲಿಯೋ ಸರ್ವತ್ರ ಇರುವ ನನ್ನನ್ನು ಜ್ಞಾನದ ಮೂಲಕ ಪೂಜೆಮಾಡುತ್ತಾರೆ.}

ಹಿಂದಿನ ಶ್ಲೋಕದಲ್ಲಿ ಭಾವಪ್ರಧಾನ ದೃಷ್ಟಿಯಿಂದ ಭಗವಂತನನ್ನು ನೋಡುವುದನ್ನು ವಿವರಿಸಿ ದನು. ಇಲ್ಲಿ ವಿಚಾರಪ್ರಧಾನದ ದೃಷ್ಟಿಯಿಂದ ಹೇಳುವನು. ಮೊದಲನೇ ಗುಂಪಿನವರು ಅದ್ವೈತ ದೃಷ್ಟಿಯಿಂದ ಅವನನ್ನು ನೋಡುವರು. ಅವನಿರುವವನೊಬ್ಬನೇ, ಎರಡಿಲ್ಲ. ಎಲ್ಲಾ ಕಡೆಯಲ್ಲೂ ನಾಮರೂಪಗಳನ್ನು ಭೇದಿಸಿ ಹೋದರೆ ಕಾಣುವುದೊಂದೇ ಪರಾತ್ಪರ ಸತ್ಯ. ನಾವು ಹಲವನ್ನು ನೋಡುವುದಕ್ಕೆ ಕಾರಣ, ಕಾಲದೇಶನಿಮಿತ್ತದ ತ್ರಿಕೋಣದ ಗ್ಲಾಸಿನ ಮೂಲಕ ನೋಡುವುದಾಗಿದೆ. ಯಾವಾಗ ಆ ತ್ರಿಕೋಣವನ್ನು ಅತಿಕ್ರಮಿಸುವೆವೋ ಆಗ ಕಾಣುವುದೊಂದೇ. ಅದರಂತೆಯೇ ಪರಬ್ರಹ್ಮ, ಜೀವ ಮತ್ತು ಜಗತ್ತು ಈಶ್ವರನಂತೆ ಕವಲೊಡೆದು ಕಾಣುವನು ದೇಶಕಾಲನಿಮಿತ್ತದ ಮೂಲಕ ನೋಡಿದಾಗ. ಯಾವಾಗ ಅದನ್ನು ಅತಿಕ್ರಮಿಸುವೆವೊ ಆಗ ಹಲವು ವ್ಯಕ್ತಿತ್ವಗಳು ಇರುವುದಕ್ಕೆ ಸಾಧ್ಯವಿಲ್ಲ. ಹಲವು ವ್ಯಕ್ತಿತ್ವಗಳೆಲ್ಲಾ ಅನಂತದ ಅಲೆಗಳಂತೆ ಕಾಣುವುವು. ಅಂತಹ ಸ್ಥಿತಿಯಲ್ಲಿ ಅವನು ಎಲ್ಲದರಲ್ಲಿಯೂ ಒಂದನ್ನೇ ನೋಡುವನು. ಎರಡನೆಯದು ಭಾವಪ್ರಧಾನ ವಾಯಿತು. ತಾನು ಬೇರೆ ಎಂಬ ವ್ಯಕ್ತಿತ್ವವನ್ನು ಕಾದಿರಿಸಿಕೊಂಡಿರುವನು. ಅವನು ಬ್ರಹ್ಮನಾಗ ಬಯಸುವುದಿಲ್ಲ. ಆದರೆ ಬ್ರಹ್ಮನ ರುಚಿ ನೋಡಬಯಸುವನು. ಹಲವು ವಿಧದಲ್ಲಿ ದೇವರನ್ನು ಪ್ರೀತಿಸುವನು. ಮೊದಲನೆಯವನು ಉಪ್ಪಿನ ಗೊಂಬೆಯಂತೆ ಸಮುದ್ರದಲ್ಲಿ ಕರಗಿಹೋಗಿ ಅವನೂ ಸಮುದ್ರವೇ ಆಗುವನು. ಎರಡನೆಯವನು ಒಂದು ಪಾತ್ರೆ ನೀರಿನಲ್ಲಿರುವಂತೆ ಇರುವನು. ಆ ಪಾತ್ರೆ ಒಳಗೆ ನೀರಿದೆ ಮತ್ತು ಆ ಪಾತ್ರೆಯ ಹೊರಗೆ ನೀರಿದೆ. ಒಳಗೆ ಹೊರಗೆಲ್ಲಾ ನೀರಿದ್ದರೂ ಆ ಪಾತ್ರೆ ಮಾತ್ರ ಎಂದಿಗೂ ನೀರಿನಲ್ಲಿ ಕರಗುವುದಿಲ್ಲ.

ಮೂರನೆಯವನಿಗೆ ಆ ಪರಮಾತ್ಮನೇ ಎಲ್ಲಾ ವಸ್ತುಗಳ ಅಂತರಾಳದಲ್ಲಿಯೂ ಇರುವಂತೆ ಕಾಣುವನು. ಅವನು ಎದುರಿಗಿರುವ ದೃಶ್ಯ ಪ್ರಪಂಚವನ್ನು ನಿರಾಕರಿಸುವುದಿಲ್ಲ. ಇದೂ ಕೂಡ ಆ ಪರಮಾತ್ಮನ ಆವಿರ್ಭಾವವೇ ಎಂದು ನೋಡುತ್ತಾನೆ. ಸಕ್ಕರೆಯಿಂದ ಹಲವು ಆಕಾರದ ವಿಗ್ರಹಗಳನ್ನು ಮಾಡುವರು. ಅದರಲ್ಲೆಲ್ಲ ನಾಮರೂಪಗಳು ಬೇರೆಯಾದರೂ ಒಂದೇ ಸಕ್ಕರೆ ಇರುವುದು. ಅದು ಮಾತ್ರ ಸತ್ಯ. ಜ್ಞಾನಿ ಈ ಸ್ಥಿತಿಯಲ್ಲಿ ‘ಸರ್ವಂ ಖಲ್ವಿದಂ ಬ್ರಹ್ಮ’, ಎಲ್ಲವೂ ಬ್ರಹ್ಮ ಎನ್ನುತ್ತಾನೆ. ಭಕ್ತ ಇದೇ ಸ್ಥಿತಿಯನ್ನು ‘ಸರ್ವವೂ ವಾಸುದೇವಮಯ’ ಎನ್ನುತ್ತಾನೆ. ಒಂದೇ ಸತ್ಯವನ್ನು ನೋಡುವ ಬೇರೆ ಬೇರೆ ದೃಷ್ಟಿಗಳು ಅಷ್ಟೆ. ಇವುಗಳಲ್ಲಿ ಯಾವ ವಿರೋಧವೂ ಇಲ್ಲ. ಒಬ್ಬನಿಗೆ ಯಾವುದು ಹಿಡಿಸುವುದೋ ಅದನ್ನು ತೆಗೆದುಕೊಳ್ಳಬಹುದು. ಇನ್ನೊಬ್ಬರ ದೃಷ್ಟಿಯನ್ನು ನೋಡಿದಾಗಲೂ ತನ್ನದೇ ಸರಿ, ಇನ್ನೊಬ್ಬನದು ತಪ್ಪು ಎಂದು ಹಳಿಯಬೇಕಾಗಿಲ್ಲ. ಯಾವ ರೀತಿಯಲ್ಲಿ ತನಗೆ ಭಗವಂತನನ್ನು ಅನುಭವಿಸಲು ಇಚ್ಛೆಯೋ ಹಾಗೆ ಅನುಭವಿಸಲಿ. ಆದರೆ ಇನ್ನೊಬ್ಬನಿಗೆ ಹೇಗೆ ಭಾವಿಸಲು ಇಚ್ಛೆಯೋ ಅದಕ್ಕೆ ಸ್ವಾತಂತ್ರ್ಯವನ್ನು ಕೊಡಲಿ. ಎಲ್ಲರನ್ನೂ ಒಂದೇ ಭಾವಕ್ಕೆ, ಒಂದೇ ದೃಷ್ಟಿಕೋನಕ್ಕೆ ತರಲಾಗು ವುದಿಲ್ಲ. ಯಾವಾಗಲೂ ವೈವಿಧ್ಯತೆಗಳು ಈ ಸೃಷ್ಟಿಯಲ್ಲಿ ಇದ್ದೇ ಇರುವುವು. ಆದರೆ ಈ ವೈವಿಧ್ಯತೆಯ ಹಿಂದೆ ಸಾಮರಸ್ಯ ಸದಾಕಾಲದಲ್ಲಿಯೂ ಇದೆ. ಈ ಪ್ರಪಂಚದಲ್ಲಿ ಒಂದೇ ಆಕಾರದ ಹೂವಿಲ್ಲ. ವಿವಿಧ ಆಕಾರಗಳಿವೆ ಅವುಗಳಲ್ಲಿ. ಒಂದೇ ಬಣ್ಣವಿಲ್ಲ. ಬಣ್ಣದ ಸಂತೆಯೇ ಅವುಗಳಲ್ಲಿದೆ. ಒಂದೇ ಪರಿಮಳವಿಲ್ಲ, ಎಲ್ಲಾ ವಿಧವಾದ ಪರಿಮಳವೂ ಇದೆ. ಆದರೆ ಈ ವೈವಿಧ್ಯತೆಯ ಪುಷ್ಪಗಳೆಲ್ಲಾ ಭಗವಂತ ಕೊರಳಿನಲ್ಲಿ ರಾಜಿಸುತ್ತಿರುವ ಹಾರ, ಅವನ ಕೈಯಲ್ಲಿರುವ ತುರಾಯಿ. ಯಾವಾಗ ಈ ದೃಷ್ಟಿ ಒಬ್ಬನಲ್ಲಿ ಬರುವುದೋ ಆಗ ಅವನು ಎಷ್ಟೊಂದು ವೈವಿಧ್ಯತೆಗಳನ್ನು ಬೇಕಾದರೂ ಸ್ವೀಕರಿಸು ವನು. ಅವು ಘರ್ಷಣೆ ಇಲ್ಲದೆ ಅವನಲ್ಲಿ ಇರಬಲ್ಲವು. ಏಕೆಂದರೆ ಅವನಿಗೆ ಇವುಗಳ ಸಾಮಾನ್ಯವಾದ ಹಿನ್ನೆಲೆಯ ಪರಿಚಯವಿದೆ.

\begin{verse}
ಅಹಂ ಕ್ರತುರಹಂ ಯಜ್ಞಃ ಸ್ವಧಾಹಮಹಮೌಷಧಮ್ ।\\ಮಂತ್ರೋಽಹಮಹಮೇವಾಜ್ಯಮಹಮಗ್ನಿರಹಮ್ ಹುತಮ್ \versenum{॥ ೧೬ ॥}
\end{verse}

{\small ನಾನೇ ಕ್ರತು, ಸ್ವಧಾ, ಔಷಧಿ, ಮಂತ್ರ, ಆಜ್ಯ, ಅಗ್ನಿ, ಹುತ ಎಲ್ಲವೂ ಆಗಿದ್ದೇನೆ.}

ಸರ್ವವ್ಯಾಪಿಯಾದ ಪರಮಾತ್ಮನೇ ಎಲ್ಲವೂ ಆಗಿದ್ದಾನೆ ಎಂದು ಹೇಳುತ್ತಾನೆ. ಹಿಂದಿನ ಕಾಲದ ವರಿಗೆ ಯಜ್ಞ ಒಂದು ಅತ್ಯಂತ ಪವಿತ್ರವಾದ ಭಗವಂತನ ಚಿಹ್ನೆಯಾಗಿತ್ತು. ಈ ಚಿಹ್ನೆಯ ಮೂಲಕವೇ ವಿವರಿಸಿದರೆ ಚೆನ್ನಾಗಿ ಅರ್ಥವಾಗುವುದೆಂದು ಹೇಳುತ್ತಾನೆ. ಜನಗಳಿಗೆ ಯಾವುದು ಆಗಲೇ ಗೊತ್ತಿ ದೆಯೋ ಆ ಸ್ಥೂಲವನ್ನು ತೆಗೆದುಕೊಂಡು ಸೂಕ್ಷ್ಮಭಾವಗಳು ವ್ಯಕ್ತವಾಗುವಂತೆ ಮಾಡುತ್ತಾನೆ. ಕ್ರತು ಎಂದರೆ ವೈಶ್ವದೇವಾದಿ ಮಹಾಯಜ್ಞಗಳು. ನಾನೇ ಪಿತೃಗಳಿಗೆ ಕೊಡುವ ಸ್ವಧಾ ಎಂಬ ಅನ್ನ. ನಾನೇ ಔಷಧ. ಪ್ರಾಣಿಗಳು ತಿನ್ನುವ ಭತ್ತ ಜವೆ ಮುಂತಾದ ಆಹಾರಗಳು. ನಾನೇ ಪಿತೃಗಳಿಗೂ ದೇವತೆ ಗಳಿಗೂ ಹವಿಸ್ಸನ್ನು ಕೊಡುವುದಕ್ಕೆ ಉಪಯೋಗಿಸುವ ಮಂತ್ರ. ನಾನೇ ಯಜ್ಞಕಾಲದಲ್ಲಿ ಉಪಯೋಗಿ ಸುವ ತುಪ್ಪ ಮುಂತಾದ ಆಜ್ಯ ವಸ್ತುಗಳು. ಯಾವುದರಲ್ಲಿ ಇವುಗಳ ಅರ್ಪಣೆ ಮಾಡುತ್ತಾರೊ ಆ ಅಗ್ನಿಯೂ ನಾನೇ. ನಾನೇ ಹುತ ಎಂದರೆ ಹೋಮಕರ್ಮ. ಇಲ್ಲಿ ಎಲ್ಲದರ ಹಿಂದೆಯೂ ಇರುವವನು ಅವನೆ. ಎಲ್ಲದರ ಹಿಂದೆಯೂ ಭಗವಂತನು ಓತಪ್ರೋತವಾಗಿರುವುದನ್ನು ನೋಡುತ್ತೇವೆ. ಇಲ್ಲಿ ಯಜ್ಞಸ್ವರೂಪನಾಗಿರುವನು ಪರಮಾತ್ಮನೆ. ಯಜ್ಞಕ್ಕಾಗಿ ಮಾಡುವ ಪ್ರತಿಯೊಂದು ಕ್ರಿಯೆಯ ಹಿಂದೆಯೂ ಇರುವವನೂ ಅವನೇ. ಈ ಸೃಷ್ಟಿಯೇ ಒಂದು ಭೂಮಯಜ್ಞ, ಹಗಲು ರಾತ್ರಿ ಇದು ಸಾಗುತ್ತಿದೆ. ಈ ಯಜ್ಞ ಅವನಿಂದ ಬಂದಿದೆ, ಅವನಲ್ಲಿ ನಡೆಯುತ್ತಿದೆ, ಕೊನೆಗೆ ಅವನಲ್ಲಿ ಕೊನೆಗೊಳ್ಳುತ್ತದೆ. ಕೊನೆಗೊಂಡುದು ಒಂದೇ ಸಲ ಕೊನೆಗೊಳ್ಳುವುದಿಲ್ಲ. ತಾತ್ಕಾಲಿಕವಾಗಿ ಮಾತ್ರ. ಈ ಯಜ್ಞ ನಿರಂತರ ಸಾಗುತ್ತಿರುವುದು. ಒಂದಾದ ಮೇಲೊಂದು ಅಲೆ ಸಾಗರದಲ್ಲಿ ಎದ್ದು ಬೀಳುವಂತೆ ಮಹಾಯಜ್ಞಗಳು ಬ್ರಹ್ಮನಲ್ಲಿ ಸಾಗುತ್ತಿವೆ.

\begin{verse}
ಪಿತಾಽಹಮಸ್ಯ ಜಗತೋ ಮಾತಾ ಧಾತಾ ಪಿತಾಮಹಃ ।\\ವೇದ್ಯಂ ಪವಿತ್ರಮೋಂಕಾರ ಪುಕ್ಸಾಮ ಯಜುರೇವ ಚ \versenum{॥ ೧೭ ॥}
\end{verse}

{\small ನಾನು ಜಗತ್ತಿಗೆ ತಂದೆ, ತಾಯಿ, ಧಾತೃ, ಪಿತಾಮಹ, ತಿಳಿಯಲು ಯೋಗ್ಯವಾದ ವಸ್ತು, ಪವಿತ್ರ, ಓಂಕಾರ, ಪುಕ್, ಸಾಮ, ಯಜಸ್ಸು.}

ಭಗವಂತನೇ ಜಗತ್ತಿನ ತಂದೆ. ನಮಗೆಲ್ಲಾ ತಂದೆ. ಮಗುವಿನ ತಂದೆ ಈ ಪ್ರಪಂಚದಲ್ಲಿ ಒಂದು ಭರವಸೆಯ ದ್ವೀಪ. ಅವನು ಮಗುವನ್ನು ಎಲ್ಲಾ ಕಷ್ಟಕಾರ್ಪಣ್ಯಗಳಿಂದಲೂ ಪಾರು ಮಾಡುತ್ತಾನೆ. ತನ್ನನ್ನು ನೆಚ್ಚಿದ ಮಗು ಮುಂದೆ ಬರುವುದಕ್ಕೆ ಏನು ಬೇಕೊ ಅದೆಲ್ಲವನ್ನೂ ಮಾಡುತ್ತಾನೆ. ತಂದೆಯ ಕೈಹಿಡಿದು ನಡೆದರೆ ಈ ಸಂಸಾರದ ದುರ್ಗಮವಾದ ಪಥದಲ್ಲಿ ಸುಲಭವಾಗಿ ಗುರಿಯನ್ನು ಸೇರ ಬಹುದು. ಅವನು ತಾಯಿಯೂ ಆಗಿದ್ದಾನೆ. ತಾಯಿಯಷ್ಟು ನಿಕಟವಾದ ವಸ್ತು ಈ ಪ್ರಪಂಚದಲ್ಲಿ ಮತ್ತಾವುದೂ ಇಲ್ಲ. ನಾವು ಎಲ್ಲವನ್ನೂ ತಂದೆಯ ಹತ್ತಿರ ಹೇಳಿಕೊಳ್ಳಲು ಆಗುವುದಿಲ್ಲ. ಆದರೆ ತಾಯಿಯ ಹತ್ತಿರ ನಿಸ್ಸಂಕೋಚದಿಂದ ಹೇಳಿಕೊಳ್ಳಬಹುದು. ಅವಳು ಯಾರನ್ನೂ ನಿರಾಕರಿಸು ವುದಿಲ್ಲ. ಒಂದು ಸಲ ‘ತಾಯಿ’ ಎಂದು ಕರೆದರೆ ಸಾಕು, ಅವಳು ನಾವು ಮಾಡಿರುವ ತಪ್ಪನ್ನೆಲ್ಲಾ ಕ್ಷಮಿಸುತ್ತಾಳೆ. ನಮ್ಮನ್ನು ಬೆಳೆಸುವುದರಲ್ಲಿ ತನ್ನ ಕಷ್ಟವನ್ನು ಮರೆಯುತ್ತಾಳೆ. ನಾವು ಏನು ಕೇಳಿದರೂ ಇಲ್ಲ ಎನ್ನುವುದಿಲ್ಲ ಅವಳು. ನಾವೆಷ್ಟು ಎಡವಿದರೂ, ಜಾರಿದರೂ ಸದಾ ನಮ್ಮನ್ನು ಕ್ಷಮಿಸಲು ಸಿದ್ಧಳಾಗಿರುವಳು. ಭಗವಂತ ದಯಾಮಯಿಯಾದ ತಾಯಿಯಲ್ಲದೇ ಇದ್ದರೆ ನಾವು ಒಂದು ಹೆಜ್ಜೆಯೂ ಮುಂದಿಡುವುದಕ್ಕೆ ಅವಕಾಶವಿರುತ್ತಿರಲಿಲ್ಲ.

ಅವನು ಧಾತೃ ಎಂದರೆ ನಮ್ಮ ಕರ್ಮಗಳಿಗೆ ಫಲವನ್ನು ಕೊಡುವವನು. ಜೀವನದಲ್ಲಿ ನಾವು ಏನನ್ನು ಬಿತ್ತಿರುವೆವೊ ಅದನ್ನು ಕುಯ್ಯಬೇಕಾಗಿದೆ. ನಾವು ಮಾಡಿರುವುದೇ ನಮಗೆ ಕಾದಿದೆ ಮುಂದಿನ ಬುತ್ತಿಯಾಗಿ. ಜೀವನದಲ್ಲಿ ಯಾವುದೂ ನಮಗೆ ಅಕಸ್ಮಾತ್ತಾಗಿ ಬರುವುದಿಲ್ಲ. ನಡೆಯುತ್ತಿರುವಾಗ ಅದೃಷ್ಟವನ್ನೋ ದುರದೃಷ್ಟವನ್ನೋ ಅಕಸ್ಮಾತ್ತಾಗಿ ಎಡಹುವುದಕ್ಕೆ ಆಗುವುದಿಲ್ಲ. ನಾವು ಮಾಡಿದ ಕಾರ್ಯಗಳನ್ನೆಲ್ಲಾ ಸದಾ ಕಾಲವೂ ಗಮನಿಸುತ್ತಿರುವನು ಅವನು. ಯಾರೂ ಅವನ ಕಣ್ಣಿಗೆ ಮಣ್ಣೆರಚು ವುದಕ್ಕೆ ಆಗುವುದಿಲ್ಲ. ನಾವು ಮಾಡಿದ್ದಕ್ಕೆ ತಕ್ಕ ಫಲ ಬರಲೇಬೇಕು. ಅವನು ಅದನ್ನು ಕೊಟ್ಟೇ ಕೊಡುತ್ತಾನೆ. ಈ ಪ್ರಪಂಚದಲ್ಲಿ ಹೇಗೆ ಹಲವಾರು ಭೌತಿಕ ನಿಯಮಗಳು ಚಿರಜಾಗ್ರತವಾಗಿ ಆಕರ್ಷಣೆಯಂತೆ, ವಿದ್ಯುತ್ತಿನಂತೆ, ಬೆಳಕು ಶಾಖದ ನಿಯಮಗಳಂತೆ ಚಲಾವಣೆಯಲ್ಲಿವೆಯೋ ಹಾಗೆಯೇ ಭಗವಂತನ ಕರ್ಮ ನಿಯಮ ಜಾಗ್ರತವಾಗಿದೆ. ಯಾವುದಾದರೂ ದೊಡ್ಡ ತಪ್ಪನ್ನು ಮಾಡಿ ಮನುಷ್ಯನು ಕೋರ್ಟು ಕಚೇರಿಯಿಂದ ಒಬ್ಬ ಬುದ್ಧಿವಂತನಾದ ಲಾಯರನ್ನಿಟ್ಟು ಪಾರಾಗುವುದು ಸುಲಭ. ಆದರೆ ಭಗವಂತನ ಕರ್ಮದ ಕೋರ್ಟಿನಲ್ಲಿ ನಾವೇನು ಮಾಡಿದರೂ ತಪ್ಪಿಸಿಕೊಂಡು ಬರುವುದಕ್ಕೆ ಆಗುವುದಿಲ್ಲ.

ಈ ಪ್ರಪಂಚಕ್ಕೆ ಪಿತಾಮಹ ಪರಮಾತ್ಮ. ಅವನೇ ಆದಿಪುರುಷ. ಎಲ್ಲಾ ಬಂದಿರುವುದು ಅವನಿಂದ. ಮೊದಲು ಅವನು, ಅನಂತರ ಈ ಪ್ರಪಂಚ, ಮತ್ತು ಇಲ್ಲಿರುವ ಬಂಧುಬಾಂಧವರು. ಅವನೇ ತಿಳಿಯಲು ಯೋಗ್ಯನಾದ ವ್ಯಕ್ತಿ ಮತ್ತು ವಸ್ತು. ಉಳಿದವುಗಳೆಲ್ಲಾ ಜೊಳ್ಳು, ಗೌಣವಸ್ತುಗಳು. ಯಾವಾಗ ಒಬ್ಬ ದೇವನನ್ನು ಅರಿಯುವನೋ ಆಗ ಅವನು ಈ ಪ್ರಪಂಚದಲ್ಲಿ ಎಲ್ಲವನ್ನೂ ಅರಿಯುವನು, ಎಲ್ಲಾ ವಸ್ತುಗಳ ಹಿಂದೆ, ಎಲ್ಲಾ ಶಕ್ತಿಗಳ ಹಿಂದೆ ಇರುವ ಏಕಮಾತ್ರ ವಸ್ತುವನ್ನು ಅರಿಯುವನು. ಇವನಷ್ಟು ಪವಿತ್ರವಾದುದು ಯಾವುದೂ ಇಲ್ಲ. ಒಮ್ಮೆ ಇವನು ನಮ್ಮನ್ನು ಪರಿಶುದ್ಧ ಮಾಡಿದ ಮೇಲೆ ನಾವಿನ್ನು ಅಜ್ಞಾನಕ್ಕೆ ಬೀಳುವ ಹಾಗಿಲ್ಲ. ಇವನ ಜ್ಞಾನದಿಂದ ತೊಳೆಯಲಾರದ ಯಾವ ಪಾಪದ ಕೆಸರೂ ಇಲ್ಲ. ಇವನೇ ಓಂಕಾರ ಎಂದರೆ ಸಂಕ್ಷಿಪ್ತ ಬ್ರಹ್ಮ. ಸಗುಣ ನಿರ್ಗುಣ ಗಳೆರಡು ನಾಣ್ಯದ ಎರಡು ಮುಖಗಳಂತೆ ಈ ಪ್ರಣವದಲ್ಲಿದೆ. ಮೂರು ವೇದಗಳಲ್ಲಿ ಪ್ರತಿಪಾದ್ಯ ನಾಗಿರುವವನು ಇವನೇ. ಬಹುಶಃ ನಾಲ್ಕನೆಯ ವೇದ ಗೀತೆಯನ್ನು ರಚಿಸಿದ ಅನಂತರ ಬಂದಿರಬೇಕು ಎಂದು ತೋರುವುದು. ಆದಕಾರಣವೇ ಅದನ್ನು ಇಲ್ಲಿ ಹೇಳಿಲ್ಲ.

\begin{verse}
ಗತಿರ್ಭರ್ತಾ ಪ್ರಭುಃ ಸಾಕ್ಷೀ ನಿವಾಸಃ ಶರಣಂ ಸುಹೃತ್ ।\\ಪ್ರಭವಃ ಪ್ರಲಯಃ ಸ್ಥಾನಂ ನಿಧಾನಂ ಬೀಜಮವ್ಯಯಮ್ \versenum{॥ ೧೮ ॥}
\end{verse}

{\small ಗತಿ, ಭರ್ತೃ, ಪ್ರಭು, ಸಾಕ್ಷಿ, ನಿವಾಸ, ಶರಣು, ಸುಹೃತ್, ಪ್ರಭವ, ಪ್ರಳಯ, ಸ್ಥಾನ, ನಿಧಾನ, ಅವ್ಯಯವಾದ ಬೀಜ ಇವೆಲ್ಲವೂ ನಾನೇ ಆಗಿದ್ದೇನೆ.}

ಈ ನಾಮರೂಪದ ತೆರೆಯ ಮರೆಯ ಹಿಂದೆ ಎಲ್ಲವೂ ಅವನೇ ಆಗಿದ್ದಾನೆ ಎಂಬುದನ್ನು ಹೇಳುತ್ತಾನೆ. ಅವನು ಗತಿ, ನಾವು ಕೊನೆಗೆ ಸೇರುವ ಸ್ಥಳ. ಸೇರಿದ ಮೇಲೆ ಈ ಸಂಸಾರ ಚಕ್ರಕ್ಕೆ ಬರದ ಸ್ಥಳ. ನಮ್ಮ ಈ ಜೀವನಯಾತ್ರೆ ಎಷ್ಟೊಂದು ಜನ್ಮಗಳ ಹಿಂದೆ ಆರಂಭವಾಯಿತೋ ಗೊತ್ತಿಲ್ಲ. ಆದರೆ ಸೇರುವ ಕೊನೆಯ ನಿಲ್ದಾಣ ಅವನು. ಅವನೇ ಭರ್ತೃ, ನಮ್ಮನ್ನು ಸಾಕಿ ಸಲಹುವವನು. ನಮಗೆ ಬೇಕಾದುದನ್ನೆಲ್ಲವನ್ನೂ ಕೊಡುವವನು. ಬೇಡದುದರಿಂದ ಪಾರುಮಾಡು ತ್ತಾನೆ. ಅವನಿಲ್ಲದೆ ನಾವು ಇಲ್ಲಿ ಬದುಕಲಾರೆವು. ಅವನೇ ಈ ಬ್ರಹ್ಮಾಂಡಕ್ಕೆಲ್ಲಾ ಪ್ರಭು, ಒಡೆಯ, ಶಾಸನಕರ್ತೃ. ಅವನು ಎಲ್ಲದಕ್ಕೂ ಒಂದು ಶಾಸನವನ್ನು ಮಾಡಿರುವನು. ಅದರಂತೆ ಎಲ್ಲಾ ನಡೆಯುತ್ತಿದೆ. ನೀರು ಹರಿಯುವುದು ಅವನಿಂದ, ಗಾಳಿ ಬೀಸುವುದು ಅವನಿಂದ, ಮಳೆಗರೆಯುವುದು ಅವನಿಂದ, ಸೂರ್ಯ ತಪಿಸುವುದೂ ಅವನಿಂದ, ನಮ್ಮ ಭೂಮಿ ಮತ್ತು ಇತರ ಗ್ರಹಗಳೆಲ್ಲಾ ಸೂರ್ಯನ ಸುತ್ತಲೂ ಸುತ್ತುತ್ತಿರುವುದು ಅವನಿಂದ. ಅವನೇ ನೈಸರ್ಗಿಕ ನಿಯಮದಂತೆ ಇರುವನು. ಅವನೇ ಕರ್ಮಸಿದ್ಧಾಂತದ ಹಿಂದೆ ಇರುವನು. ಅವನು ಪ್ರಪಂಚವನ್ನೆಲ್ಲ ಶಾಸನದಿಂದ ಬಿಗಿದಿರುವನು. ಅದನ್ನು ತಪ್ಪಿಸಿಕೊಳ್ಳುವುದಕ್ಕೆ ಯಾವುದಕ್ಕೂ ಸಾಧ್ಯವಿಲ್ಲ. ಕೆಲವು ವೇಳೆ ಸಣ್ಣ ಶಾಸನದ ಹಗ್ಗವನ್ನು ಕಿತ್ತುಕೊಂಡು ಹೋಗುತ್ತೇವೆ. ನಾವು ಸ್ವತಂತ್ರರು ಎಂದು ಭಾವಿಸುವೆವು. ಆದರೆ ನಾವು ಬೇರೊಂದು ಶಾಸನಕ್ಕೆ ಅಡಿಯಾಳು ಎಂಬುದು ನಮಗೆ ಗೊತ್ತೇ ಆಗುವುದಿಲ್ಲ. ಸ್ಪುಟ್ನಿಕ್ ಭೂಮಿಯ ಆಕರ್ಷಣೆ ಯಿಂದ ಕಿತ್ತುಕೊಂಡು ಹೋಗಬಹುದು. ಅದು ಪುನಃ ಭೂಮಿಗೆ ಪ್ರವೇಶಿಸದೆ ಇರಬಹುದು. ಭೂಮಿಯ ಸುತ್ತಲೂ ಅಥವಾ ಇನ್ನೂ ವೇಗವಾಗಿ ಚಲಿಸಿಕೊಂಡು ಹೋದರೆ ಸೂರ್ಯನ ಸುತ್ತಲೂ ಬೇರೊಂದು ಉಪಗ್ರಹದಂತೆ ಸುತ್ತಲೇ ಬೇಕಾಗಿದೆ. ಈ ದಾಸ್ಯಕ್ಕೆ ಕೊನೆಯಿಲ್ಲ. ಒಂದರಿಂದ ತಪ್ಪಿಸಿಕೊಂಡರೆ ನಮ್ಮನ್ನು ನುಂಗಲು ಬೇರೊಂದು ಆಗಲೇ ಬಾಯಿತೆರೆದುಕೊಂಡಿರುವುದು. ಇವುಗಳೆಲ್ಲ ಭಗವಂತನ ಕೈವಾಡವೆ.

ಅವನು ಈ ಬ್ರಹ್ಮಾಂಡದ ನಾಟಕವನ್ನು ಸಾಕ್ಷಿಯಂತೆ ನಿಂತು ನೋಡುತ್ತಿರುವನು. ಅವನಿಗೆ ಯಾವುದರ ಮೇಲೆಯೂ ಪಕ್ಷಪಾತವಿಲ್ಲ. ಯಾವುದನ್ನೂ ಪ್ರೀತಿಸುವುದೂ ಇಲ್ಲ, ದ್ವೇಷಿಸುವುದೂ ಇಲ್ಲ. ಸಂಪೂರ್ಣ ಅನಾಸಕ್ತಿ ಅವನದು. ಎಲ್ಲವನ್ನೂ ಮಾಡಿದವನು, ನಡೆಸುತ್ತಿರುವವನು. ಆದರೂ ತಾನು ಕೈಯನ್ನು ತೊಳೆದುಕೊಂಡಿರುವನು. ಅವನೇ ನಿವಾಸ, ನಮ್ಮಗಳ ವಾಸಸ್ಥಾನ. ಅವನ ತೊಡೆಯ ಮೇಲೆಯೇ ಮಕ್ಕಳು ತಾಯಿಯ ತೊಡೆಯ ಮೇಲೆ ಆಡುವಂತೆ ಆಡುತ್ತಿರುವೆವು. ಅಗಣಿತ ಅಲೆಗಳ ವಾಸಸ್ಥಾನವಾದ ಸಾಗರದಂತೆ ಜೀವರಾಶಿಗಳಿಗೆಲ್ಲಾ ವಾಸಸ್ಥಾನವಾಗಿರುವವನು ಅವನು. ಅವನು ಶರಣು, ಎಂದರೆ ರಕ್ಷಕ. ನಾವು ಯಾವಾಗ ಅವನಲ್ಲಿ ಶರಣಾಗುವೆವೊ ಆಗ ಎಲ್ಲಾ ಭೂತಗಳಿಂದ ಅಭಯ ಕೊಟ್ಟು ಸಲಹುವವನು ಅವನು. ತನ್ನನ್ನು ನೆಚ್ಚಿದವರನ್ನು ಎಂದೆಂದಿಗೂ ಕೈಬಿಡುವುದಿಲ್ಲ. ಕಾಪಾಡಿಯೇ ಕಾಪಾಡುತ್ತಾನೆ. ಅವನೇ ನಮ್ಮ ಹಿತೈಷಿ, ಒಳ್ಳೆಯ ಸ್ನೇಹಿತ. ಯಾವ ಸಮಯದಲ್ಲಿ ನಾವು ಹೇಗೆ ಮಾಡಬೇಕು ಎಂಬುದನ್ನು ಅಂತರ್ವಾಣಿಯ ಮೂಲಕ ನಮಗೆ ಹೇಳುತ್ತಿರುವನು. ನಮಗಿಂತ ಹೆಚ್ಚಾಗಿ ನಮ್ಮ ಭವಿಷ್ಯ ಶ್ರೇಯಸ್ಕರವಾಗಬೇಕೆಂದು ಬಯಸುವನು. ನಮಗೇ ಬಿಟ್ಟರೆ ನಾವು ತಾತ್ಕಾಲಿಕ ಪ್ರಿಯವಾದ ಹಾದಿಯನ್ನೇ ಹಿಡಿಯುತ್ತೇವೆ. ಅವನು ಇರುವುದರಿಂದ, ಅವನಲ್ಲಿ ಶರಣಾಗಿರುವುದರಿಂದ, ನಮ್ಮನ್ನು ಶ್ರೇಯಸ್ಸಿನ ಕಡೆಗೆ ಒಯ್ಯುವನು. ಅವನಿಂದ ಈ ಸೃಷ್ಟಿ ಬಂದಿದೆ. ಅವನಲ್ಲಿ ಈ ಸೃಷ್ಟಿ ಇದೆ. ಕೊನೆಗೆ ಅವನ ಬಳಿಗೆ ಪ್ರಳಯದಲ್ಲಿ ಹೋಗುವುದು. ಅಲೆ ಏಳುವುದಕ್ಕೆ ಮುಂಚೆ, ಎದ್ದಾಗ ಮತ್ತು ಬಿದ್ದಾಗ ಹೇಗೆ ಸಾಗರದಲ್ಲಿರುವುದೋ ಹಾಗೆಯೆ. ಅವನೇ ಎಲ್ಲದಕ್ಕೆ ಆಧಾರವಾಗಿರುವನು. ಆ ತಳಪಾಯದ ಮೇಲೆ ಈ ಬ್ರಹ್ಮಾಂಡವೆಲ್ಲ ನಿಂತಿದೆ. ಈ ಸೃಷ್ಟಿಯಲ್ಲಿ ನಾಶವಾಗದ ಬೀಜರೂಪದಿಂದ ಇರುವವನೂ ಅವನೇ. ಈ ಸೃಷ್ಟಿಯಲ್ಲಿರುವ ವೈವಿಧ್ಯತೆಗಳೆಲ್ಲಾ ಅವನಲ್ಲಿರುವ ಬೀಜಗಳು. ಎಷ್ಟು ಸಲ ಪ್ರಳಯವಾದರೂ ಆ ಬೀಜಗಳು ಬೇರೆ ನಾಶವಾಗುವುದಿಲ್ಲ. ಅವು ಕೇವಲ ಅವ್ಯಕ್ತ ಸ್ಥಿತಿಗೆ ಹೋಗುತ್ತವೆ. ಆಗ ಸುಪ್ತಾವಸ್ಥೆಯಲ್ಲಿರುತ್ತವೆ. ಪುನಃ ಸೃಷ್ಟಿಯಾದಾಗ ಹೊಸ ನಾಮರೂಪಗಳನ್ನು ಧರಿಸಿ ಬರುತ್ತವೆ.

\begin{verse}
ತಪಾಮ್ಯಹಮಹಂ ವರ್ಷಂ ನಿಗೃಹ್ಣಾಮ್ಯುತ್ಸೃಜಾಮಿ ಚ ।\\ಅಮೃತಂ ಚೈವ ಮೃತ್ಯುಶ್ಚ ಸದಸಚ್ಚಾಹಮರ್ಜುನ \versenum{॥ ೧೯ ॥}
\end{verse}

{\small ಅರ್ಜುನ, ಬಿಸಿಲಿನಂತೆ ತಪಿಸುತ್ತೇನೆ, ಮಳೆಯನ್ನು ಸುರಿಸುತ್ತೇನೆ. ಅದನ್ನು ತಡೆಯುತ್ತೇನೆ. ನಾನೇ ಅಮರತ್ವ ಮತ್ತು ಮೃತ್ಯು; ಸತ್, ಅಸತ್ ಗುಣಗಳೂ ನಾನೆ.}

ಭಗವಂತ ಪ್ರಕೃತಿಯಲ್ಲಿ ಹೇಗೆ ಸರ್ವವ್ಯಾಪಿಯಾಗಿ ಕೆಲಸ ಮಾಡುತ್ತಿರುವನು ಎಂಬುದನ್ನು ಹೇಳುತ್ತಾನೆ. ಬಿಸಿಲಿನಂತೆ ತಪಿಸುತ್ತಿರುವವನು ಅವನು. ಈ ಬಿಸಿಲೇ ನಮಗೆಲ್ಲಾ ಶಕ್ತಿಯನ್ನು ಒದಗಿಸುವುದು. ಸಸ್ಯಾದಿಗಳು ಬೆಳೆಯಬೇಕಾದರೆ ಸೂರ್ಯನ ಬೆಳಕು ಅದರ ಮೇಲೆ ಬೀಳಬೇಕು. ಆಗಲೇ ಅದು ಚುರುಕಾಗಿ ಕೆಲಸ ಮಾಡುವುದು. ಮರಗಿಡಗಳೆಲ್ಲಾ ಸೂರ್ಯನ ಶಕ್ತಿಯನ್ನು ಸಂಗ್ರಹಿಸಿ ಇಟ್ಟುಕೊಂಡು ಇರುವುವು. ಕಟ್ಟಿಗೆಯನ್ನು ಉರಿಸಿದಾಗ ನಮಗೆ ದೊರಕುವುದೇ ಅದು. ಕಲ್ಲಿದ್ದಲಿನಲ್ಲಿ ಹುದುಗಿರುವುದೇ ಅದು. ಅನಂತ ಸಾಗರದ ನೀರು ಆವಿಯಾಗಿ ಮೇಲೇಳುವುದೇ ಸೂರ್ಯ ತಪಿಸು ವುದರಿಂದ.

ಮಳೆಯನ್ನು ಸುರಿಸುತ್ತೇನೆ ಎನ್ನುವನು. ಮಳೆ ಬೀಳುವುದು ಅವನಿಂದ. ಮಳೆಯ ನೀರೇ ನಮ್ಮ ಜೀವನಾಧಾರ. ಬಾಳು ವಿಕಾಸವಾಗಬೇಕಾದರೆ ಮಳೆ ಬೀಳಬೇಕು. ಮಳೆ ಬಂದರೆ ಬೆಳೆ, ಬೆಳೆ ಇದ್ದರೆ ಪ್ರಾಣಿಗಳು ಮನುಷ್ಯ ಇವರೆಲ್ಲರೂ ತಮ್ಮ ಜೀವಗತಿಯನ್ನು ಅನುಸರಿಸಬೇಕಾದರೆ. ಹೇಗೆ ಮಳೆಯ ಹಿಂದೆ ಭಗವಂತನಿರುವನೋ ಹಾಗೆಯೇ ಅದನ್ನು ತಡೆಯುವುದರ ಹಿಂದೆಯೂ ಅವನೇ ಇರುವನು. ಮಳೆ ಜಾಸ್ತಿ ಆದರೂ ಕಷ್ಟ; ಮಳೆ ಬೀಳದೇ ಇದ್ದರೂ ಕಷ್ಟ. ಮನುಷ್ಯ ಅತಿವೃಷ್ಟಿ ಅನಾವೃಷ್ಟಿಗಳ ನಡುವೆ ಇರುತ್ತಾನೆ. ಸ್ವಲ್ಪ ಜಾಸ್ತಿ ಬಿದ್ದರೆ ಆಯಿತು ಅವನಾಟ; ಬೀಳುವುದು ಕಡಮೆಯಾದರೂ ಅವನಿಗೆ ವಿಪತ್ತು ಬರುವುದು. ಕೊಡುವುದರ ಹಿಂದೆ ಎಷ್ಟು ಅವನು ಕೆಲಸ ಮಾಡುತ್ತಿರುವನೋ ಅಷ್ಟೇ ತಡೆಯುವುದರ ಹಿಂದೆಯೂ ಕೆಲಸ ಮಾಡುತ್ತಿರುವನು. ಅವನೇ ಅಮರತ್ವ ಮತ್ತು ಮೃತ್ಯು. ಎರಡು ವಿರೋಧ ಗುಣಗಳಿಗೂ ಅವನೊಡೆಯ. ಯಾರು ಜೀವನವನ್ನು ಭಗವಂತನನ್ನು ಕಾಣುವುದಕ್ಕೆ ವಿನಿಯೋಗಿಸುವರೋ ಮತ್ತು ಅದರಲ್ಲಿ ಜಯಶೀಲರಾಗಿರುವರೋ ಅವರನ್ನು ಜನನ ಮರಣಗಳ ಕೋಟಲೆಗಳಿಂದ ತಪ್ಪಿಸಿ ಅವರಿಗೆ ಅಮರತ್ವವನ್ನು ಕೊಡುವನು. ಯಾರು ಕೇವಲ ಇಂದ್ರಿಯ ಸುಖಾಭಿಲಾಶೆಯಲ್ಲಿಯೇ ಮತ್ತರಾಗಿ ಇರುವರೋ ಅವರು ಮೃತ್ಯುವನ್ನು ಅನುಭವಿಸಲೇ ಬೇಕು. ಪುನಃ ಪುನಃ ಹುಟ್ಟಬೇಕು, ಸಾವಿನ ದವಡೆಗೆ ಸಿಕ್ಕಬೇಕು. ಯಾವ ಕೈಗಳು ಮುಕ್ತನನ್ನು ಬಿಡುಗಡೆ ಮಾಡುವುವೋ ಅದೇ ಕೈಗಳು ಬದ್ಧಜೀವಿಯನ್ನು ಪುನಃ ಪುನಃ ಸಂಸಾರದ ಒರಳುಕಲ್ಲಿಗೆ ಹಾಕಿ ಚೆನ್ನಾಗಿ ರುಬ್ಬುತ್ತಿದೆ, ಅವನನ್ನು ಮುಂದೆ ಸಿದ್ಧಮಾಡಲು.

ಅವನೇ ಸತ್​; ಎದುರಿಗೆ ಕಾಣುವ ಚರಾಚರಾತ್ಮಕವಾದ ವಿಶ್ವದ ಹಿಂದೆಲ್ಲ ಇರುವ ಏಕಮಾತ್ರ ಸತ್ಯ. ನಾವೊಂದು ಕನಸು ಕಂಡಾಗ ಕನಸಿನ ಪ್ರತಿಯೊಂದು ವಸ್ತುವಿನ ಹಿಂದೆಯೂ ನಾನೆ ಇರುವೆನು. ಅದರಂತೆಯೇ ಈ ಬ್ರಹ್ಮಾಂಡ ಭಗವಂತನ ಒಂದು ಕನಸಿನಂತೆ. ಈ ಕನಸಿನಲ್ಲೆಲ್ಲೆಲ್ಲೂ ಅವನ ಅಂಶವೇ ಇರುವುದು. ಜಡದಲ್ಲಿ, ಚೇತನದಲ್ಲಿ, ಒಳ್ಳೆಯದರಲ್ಲಿ, ಕೆಟ್ಟದರಲ್ಲಿ, ಸಣ್ಣದರಲ್ಲಿ, ದೊಡ್ಡದರಲ್ಲಿ ಇರುವುದೆಲ್ಲಾ ಅವನೆ. ಅವನೇ ಅಸತ್​; ಎಂದರೆ ಸುಳ್ಳಲ್ಲ, ಅವ್ಯಕ್ತ. ನಮಗೆ ಕಾಣುವುದು ವ್ಯಕ್ತ ಸ್ವರೂಪ. ಈ ವ್ಯಕ್ತ ಸ್ವರೂಪ ಬರುವುದಕ್ಕೆ ಮುಂಚೆ ಅವ್ಯಕ್ತ ಅವಸ್ಥೆಯಲ್ಲಿತ್ತು. ಅಲೆ ಏಳುವುದಕ್ಕೆ ಮುಂಚೆ ಸಾಗರದಲ್ಲಿ ಸುಪ್ತಾವಸ್ಥೆಯಲ್ಲಿತ್ತು. ಗಿಡ ವಿಕಾಸವಾಗುವುದಕ್ಕೆ ಮುಂಚೆ ಬೀಜದಲ್ಲಿ ಸೂಕ್ಷ್ಮಾವಸ್ಥೆಯಲ್ಲಿ ನಿದ್ರಿಸುತ್ತಿತ್ತು. ಕಾಣುವ ಸ್ಥೂಲದ ಹಿಂದೆ ಇರುವವನು ಅವನು. ಕಾಣದ ಸೂಕ್ಷ್ಮದ ಹಿಂದೆ ಇರುವವನೂ ಅವನೇ.

\begin{verse}
ತ್ರೈವಿದ್ಯಾ ಮಾಂ ಸೋಮಪಾಃ ಪೂತಪಾಪಾ ಯಜ್ಞೈರಿಷ್ಟ್ವಾ ಸ್ವರ್ಗತಿಂ ಪ್ರಾರ್ಥಯಂತೇ ।\\ತೇ ಪುಣ್ಯಮಾಸಾದ್ಯ ಸುರೇಂದ್ರಲೋಕಮಶ್ನಂತಿ ದಿವ್ಯಾನ್ ದಿವಿದೇವಭೋಗಾನ್ \versenum{॥ ೨ಂ ॥}
\end{verse}

{\small ಮೂರು ವೇದಗಳನ್ನು ಬಲ್ಲವರು ಯಜ್ಞದಿಂದ ನಮ್ಮನ್ನು ಪೂಜಿಸಿ ಸೋಮಪಾನ ಮಾಡಿ ಪಾಪಗಳನ್ನು ಕಳೆದುಕೊಂಡು ಸ್ವರ್ಗಲೋಕವನ್ನು ಪ್ರಾರ್ಥಿಸುತ್ತಾರೆ. ಅವರು ಪವಿತ್ರ ದೇವಲೋಕವನ್ನು ಹೊಂದಿ ಅಲ್ಲಿ ದಿವ್ಯವಾದ ಭೋಗಗಳನ್ನು ಅನುಭವಿಸುತ್ತಾರೆ.}

ವೇದಗಳಲ್ಲಿ ಕರ್ಮಕಾಂಡ ಮತ್ತು ಜ್ಞಾನಕಾಂಡಗಳೆಂದು ಎರಡು ಭಾಗಗಳಿವೆ. ಒಂದು ಪ್ರವೃತ್ತ ಜೀವಿಗಳಿಗೆ. ಇನ್ನೊಂದು ನಿವೃತ್ತ ಜೀವಿಗಳಿಗೆ. ಇಲ್ಲಿ ಯಾಗಯಜ್ಞಾದಿಗಳನ್ನು ಮಾಡುವವರು ಫಲಾಪೇಕ್ಷೆಯಿಂದ ಮಾಡುತ್ತಾರೆ. ಆ ಸಮಯದಲ್ಲಿ ದೇವತೆಗಳನ್ನು ಉಪಾಸನೆ ಮಾಡಿ ಸೋಮರಸ ವನ್ನು ಅವರಿಗೆ ನೈವೇದ್ಯ ಮಾಡಿ ಉಳಿದುದನ್ನು ಪ್ರಸಾದದಂತೆ ಪಾನಮಾಡುತ್ತಾರೆ, ಮತ್ತು ದೇವತೆಗಳನ್ನು ಸ್ವರ್ಗಲೋಕದಲ್ಲಿ ಭೋಗವನ್ನು ಅನುಭವಿಸುವುದಕ್ಕೆ ಪ್ರಾರ್ಥಿಸುತ್ತಾರೆ. ಶ್ರೀಕೃಷ್ಣ ಹೀಗೆ ಪ್ರಾರ್ಥನೆ ಮಾಡುವುದು ವ್ಯರ್ಥವಾಗುವುದು ಎಂದು ಹೇಳುವುದಿಲ್ಲ. ಸ್ವರ್ಗಾದಿ ಲೋಕಗಳು ಇಲ್ಲವೆಂತಲೂ ಹೇಳುವುದಿಲ್ಲ. ಪ್ರಾರ್ಥನೆ ಎಂದೂ ನಿಷ್ಪ್ರಯೋಜಕವಾಗುವುದಿಲ್ಲ. ದೇವರು ನಾವು ಕೇಳಿದುದನ್ನು ಕೊಡುತ್ತಾನೆ. ದೈವಭೋಗಗಳೂ ಇಲ್ಲವೆನ್ನುವುದಿಲ್ಲ. ಅದನ್ನೂ ಕೊಡುತ್ತಾನೆ. ಆದರೆ ಅವು ಶ್ರೇಷ್ಠ ವಸ್ತುವಲ್ಲ. ಆದರೆ ಮನುಷ್ಯನಲ್ಲಿ ಭೋಗವಾಸನೆ ಇದೆ. ಈ ದೇಹದಲ್ಲಿ ಹಲವು ಸುಖಗಳನ್ನು ಅನುಭವಿಸಿರುವನು. ಆದರೆ ಈ ಸ್ಥೂಲ ದೇಹಕ್ಕೆ ಒಂದು ಮಿತಿ ಇದೆ. ನಮಗೆ ಬೇಕಾದಷ್ಟನ್ನು ಈ ದೇಹದ ಮೂಲಕ ಅನುಭವಿಸಲಾಗುವುದಿಲ್ಲ. ಅದಕ್ಕೆ ಬೇರೊಂದು ಸೂಕ್ಷ್ಮವಾದ ದೇಹವನ್ನು ತೆಗೆದುಕೊಳ್ಳಬೇಕಾಗುವುದು. ಇದನ್ನು ಪ್ರಾರ್ಥಿಸುತ್ತಾರೆ. ದೇವರು, ಪಾಪ ಅನುಭವಿಸಲಿ ಎಂದು ಕೇಳಿದುದನ್ನು ಕೊಡುತ್ತಾನೆ. ದೇವಲೋಕವೆಂದರೆ ಅವರು ಬೇರೊಂದು ಲೋಕಕ್ಕೆ ಹೋಗಿ ಅನುಭವಿಸುತ್ತಾರೆಂದು ತಿಳಿದುಕೊಳ್ಳಬೇಕಾಗಿಲ್ಲ. ನಾವಿರುವ ಲೋಕದಲ್ಲೆ ನರಕ ಸ್ವರ್ಗಗಳೆರಡೂ ಇವೆ. ಶ್ರೀಮಂತರ ಕುಲದಲ್ಲಿ ಎಲ್ಲಾ ಭೋಗಗಳ ಮಧ್ಯೆ ಜನ್ಮವೆತ್ತುವುದು ದೇವಲೋಕದಲ್ಲಿ ಜನ್ಮವೆತ್ತಿದಂತೆ. ಹುಟ್ಟು ಬಡತನ ಕಿತ್ತು ತಿನ್ನುತ್ತಿರುವಾಗ ರೋಗರುಜಿನಗಳು ಕೊಳೆ ಕಸಗಳ ಮಧ್ಯೆ ಜನ್ಮವೆತ್ತುವುದು ಒಂದು ನರಕದಲ್ಲಿ ಹುಟ್ಟಿದಂತೆಯೇ. ಅಂತೂ ದೇವರು ನಾವು ಕೇಳಿದುದನ್ನು ಕೊಡುತ್ತಾನೆ. ಸ್ವರ್ಗದಲ್ಲಿ ಅನುಭವಿಸಬೇಕೆಂದರೆ ಅದೇ ಲೋಕಕ್ಕೆ ನಮ್ಮನ್ನು ಕಳುಹಿಸುತ್ತಾನೆ. ಈ ಪ್ರಪಂಚದ ಸ್ವರ್ಗದ ಮಧ್ಯದಲ್ಲಿ ಇರಬೇಕೆಂದರೆ ಆ ಸ್ಥಳವನ್ನು ಕೊಡುತ್ತಾನೆ. ಆದರೆ ಇವುಗಳೆಲ್ಲಾ ತಾತ್ಕಾಲಿಕ. ಎಂದೆಂದಿಗೂ ನಾವು ಅಲ್ಲಿಯೇ ಇರುವುದಕ್ಕೆ ಆಗುವುದಿಲ್ಲ. ಇಲ್ಲಿ ಕೆಲವು ಪುಣ್ಯಕರ್ಮ ಗಳು ಮತ್ತು ಪ್ರಾರ್ಥನೆಯಿಂದ ಅಲ್ಲಿ ಕೆಲವು ಕಾಲ ಇರುವ ಹಕ್ಕನ್ನು ಪಡೆಯುತ್ತೇವೆ.

\begin{verse}
ತೇ ತಂ ಭುಕ್ತ್ವಾ ಸ್ವರ್ಗಲೋಕಂ ವಿಶಾಲಂ ಕ್ಷೀಣೇ ಪುಣ್ಯೇ ಮರ್ತ್ಯಲೋಕಂ ವಿಶಂತಿ ।\\ಏವಂ ತ್ರಯೀಧರ್ಮಮನುಪ್ರಪನ್ನಾ ಗತಾಗತಂ ಕಾಮಕಾಮಾ ಲಭಂತೇ \versenum{॥ ೨೧ ॥}
\end{verse}

{\small ಆ ವಿಶಾಲವಾದ ಸ್ವರ್ಗಲೋಕಗಳನ್ನು ಅನುಭವಿಸಿ ಪುಣ್ಯಕ್ಷೀಣವಾದಮೇಲೆ ಪುನಃ ಮರ್ತ್ಯಲೋಕವನ್ನು ಹೊಂದುತ್ತಾರೆ. ಮೂರು ವೇದಗಳ ಅನುಸಾರವಾಗಿ ಕರ್ಮಮಾಡುವ ಇವರು ಫಲವನ್ನು ಬಯಸುವುದರಿಂದ ಜನನ ಮರಣವನ್ನು ಹೊಂದುತ್ತಾರೆ.}

ಇವರು ಸಂಪಾದಿಸಿದ ಪುಣ್ಯ ಸಾಂತವಾದುದು. ಇದರಿಂದ ಅವರು ಎಂದೆಂದಿಗೂ ಸ್ವರ್ಗಲೋಕ ದಲ್ಲಿ ಇರುವುದಕ್ಕೆ ಆಗುವುದಿಲ್ಲ. ನಾವು ತೆಗೆದುಕೊಂಡು ಹೋದ ಬುತ್ತಿ ತೀರಿದೊಡನೆ ಅಲ್ಲಿಂದ ಮರ್ತ್ಯಲೋಕಕ್ಕೆ ನಾವು ಪುನಃ ಉರುಳಬೇಕಾಗುವುದು. ನಾವೊಂದು ನಾಟಕ ನೋಡಲು ಟಿಕೇಟು ತೆಗೆದುಕೊಂಡು ಹೋಗಬೇಕು. ನಾಟಕ ಪೂರ್ತಿ ಆದೊಡನೆ ನಾವು ಅಲ್ಲೇ ಇರುವುದಕ್ಕೆ ಆಗುವುದಿಲ್ಲ. ಹಿಂತಿರುಗಿ ಮನೆಗೆ ಬರಬೇಕು. ಇನ್ನೊಮ್ಮೆ ನಾಟಕ ನೋಡಬೇಕೆಂದರೆ ಬೇರೆ ಟಿಕೀಟನ್ನು ತೆಗೆದುಕೊಳ್ಳ ಬೇಕು. ಹಿಂದಿನ ಟಿಕೀಟು ಅಂದಿಗೆ ತೀರಿಹೋಯಿತು. ಹಾಗೆಯೇ ಸ್ವರ್ಗಲೋಕದ ಅನುಭವ. ನಾವು ಎಷ್ಟನ್ನು ಪಡೆದುಕೊಂಡಿರುವೆವೊ ಅದು ತೀರಿದಮೇಲೆ ನಾವು ಅಲ್ಲಿ ಇನ್ನು ಮೇಲೆ ಇರುವುದಕ್ಕೆ ಆಗುವುದಿಲ್ಲ. ಅಲ್ಲಿಂದ ನಮ್ಮನ್ನು ಈ ಮರ್ತ್ಯಲೋಕಕ್ಕೆ ತಳ್ಳುವರು. ನಾವು ಮಳೆಯಲ್ಲಿ ಬಿದ್ದು, ಬೆಳೆಯಲ್ಲಿ ಸೇರಿ ಮನುಷ್ಯ ದೇಹವನ್ನೊ ಪ್ರಾಣಿದೇಹವನ್ನೊ ನಮ್ಮ ಕರ್ಮಾನುಸಾರ ಪಡೆಯ ಬೇಕಾಗಿದೆ.

ಕರ್ಮಕಾಂಡವನ್ನು ಮಾಡುವ ಮೂರು ವೇದಗಳನ್ನು ಬಲ್ಲವರ ಗತಿಯೂ ಇದೆ. ಇವರು ಮುಕ್ತಿಯನ್ನು ಆಶಿಸುವುದಿಲ್ಲ. ಸ್ವರ್ಗದಲ್ಲಿ ಭೋಗವನ್ನು ಆಶಿಸುತ್ತಾರೆ. ಯಾವಾಗ ಭೋಗವೇ ಇವರ ಗುರಿಯಾಗಿದೆಯೋ, ಪುನಃ ಈ ಪ್ರಪಂಚಕ್ಕೆ ಅವರು ಬರಬೇಕಾಗಿದೆ. ಇಲ್ಲಿ ಹಲವಾರು ಪುಣ್ಯ ಕೆಲಸವನ್ನು ಫಲಾಪೇಕ್ಷೆಯಿಂದ ಮಾಡಿ ಪುನಃ ಕೆಲವು ಕಾಲ ಸ್ವರ್ಗದಲ್ಲಿ ಇರುವ ಹಕ್ಕನ್ನು ಪಡೆಯುತ್ತಾರೆ. ಅಲ್ಲಿ ಅದು ಮುಗಿದಮೇಲೆ ಪುನಃ ಈ ಮರ್ತ್ಯಲೋಕದಲ್ಲಿಯೇ ಹೊಸ ಕರ್ಮ ಮಾಡುವುದಕ್ಕೆ ನಮಗೆ ಸ್ವಾತಂತ್ರ್ಯವಿರುವುದರಿಂದ ಸ್ವರ್ಗಲೋಕಕ್ಕೆ ಹೋಗುವ ಪುಣ್ಯಸಂಪಾದನೆಗೆ ಬರಬೇಕಾಗಿದೆ.

ಎಲ್ಲಿಯವರೆಗೆ ಭೋಗವೇ ನಮ್ಮ ಗುರಿಯಾಗಿದೆಯೋ, ಅದು ದೇವಲೋಕದಲ್ಲಿರಬಹುದು ಅಥವಾ ನಮ್ಮ ಈ ಕರ್ಮಲೋಕದಲ್ಲಿಯೇ ಇರಬಹುದು, ಅಲ್ಲಿಯವರೆಗೆ ನಾವು ಪ್ರಪಂಚಕ್ಕೆ ಹಲವು ವೇಳೆ ಬಂದು ಹೋಗುತ್ತಿರಬೇಕಾಗುವುದು. ಈ ಪ್ರಪಂಚಕ್ಕೆ ಒಂದು ಬಾಗಿಲಿನಿಂದ ಬರುತ್ತೇವೆ; ಮತ್ತೊಂದು ಬಾಗಿಲಿನಿಂದ ಹೋಗುತ್ತೇವೆ. ನಾಟಕ ಮಂದಿರದಲ್ಲಿ ರಂಗಭೂಮಿಯ ಮೇಲೆ ಪಾತ್ರಧಾರಿ ಒಂದು ಮೂಲೆಯಿಂದ ಬಂದು ಮತ್ತೊಂದು ಮೂಲೆಯ ಮೂಲಕ ಹೋಗುವಂತೆ ಇದು. ಬರುವುದು ಹೋಗುವುದು ಜೀವಿಯ ಪಾಡಾಗುವುದು. ಬರುವಾಗಲೂ ಇದೇ ಕೊನೆ ಎಂದು ಹೇಳುವ ಹಾಗಿಲ್ಲ; ಹೋಗುವಾಗಲೂ ಇದೇ ಕೊನೆ, ಇನ್ನುಮೇಲೆ ನಾನು ಹೋಗುವುದಿಲ್ಲ ಎಂತಲೂ ಹೇಳುವಹಾಗಿಲ್ಲ. ಭೋಗವಾಸನೆ ನಮ್ಮಲ್ಲಿ ಇರುವವರೆಗೆ ಗಾಣಕ್ಕೆ ಕಟ್ಟಿದ ಎತ್ತಿನಂತೆ ನಾವು ಸಂಸಾರದ ಗಾಣದ ಸುತ್ತಲೂ ಸುತ್ತುತ್ತಿರಬೇಕಾಗುವುದು. ಸತ್ತರೆ ಬಿಡುಗಡೆ ಏನಿಲ್ಲ ನಮಗೆ; ರಾತ್ರಿ ಎತ್ತನ್ನು ಒಳಗೆ ಕಟ್ಟುವಂತೆ ಅದು. ಪುನಃ ಬೆಳಗೆದ್ದಮೇಲೆ ನೊಗ ಕೊರಳಮೇಲೆ ಬೀಳುವುದು.

\begin{verse}
ಅನನ್ಯಾಶ್ಚಿಂತಯಂತೋ ಮಾಂ ಯೇ ಜನಾಃ ಪರ್ಯುಪಾಸತೇ ।\\ತೇಷಾಂ ನಿತ್ಯಾಭಿಯುಕ್ತಾನಾಂ ಯೋಗಕ್ಷೇಮಂ ವಹಾಮ್ಯಹಂ \versenum{॥ ೨೨ ॥}
\end{verse}

{\small ಯಾರು ಅನನ್ಯಭಾವದಿಂದ ನನ್ನನ್ನು ಚಿಂತಿಸುತ್ತಾ ನನ್ನನ್ನು ಭಜಿಸುತ್ತಾರೊ, ನನ್ನಲ್ಲಿಯೇ ನಿರತರಾಗಿರುವ ಅವರ ಯೋಗಕ್ಷೇಮವನ್ನು ನಾನು ವಹಿಸುತ್ತೇನೆ.}

ದೇವರನ್ನು ನಾವು ಏನನ್ನೂ ಕೇಳದೆ ಇದ್ದರೆ ಅವನು ನಮ್ಮನ್ನು ಮರೆಯಬಹುದು. ಅದಕ್ಕೆ ನಾವು ಮಧ್ಯೆಮಧ್ಯೆ ಅವನಿಗೆ ನಮಗೆ ಬೇಕಾದುದನ್ನು ಜ್ಞಾಪಿಸಬೇಕಾಗಿದೆ ಎಂದು ಭಾವಿಸುವೆವು. ನಾವು ಕೇಳದೇ ಇದ್ದರೂ ಅತ್ಯಂತ ಆವಶ್ಯಕವಾಗಿರುವುದನ್ನು ನಮಗೆ ದೇವರು ಕೊಡುವನು. ಕೇಳಿಸಿಕೊಂಡು ಕೊಡುವವನು ಅವನಲ್ಲ. ಕೇಳಿಸಿಕೊಳ್ಳದೆಯೇ ಕೊಡುವನು. ಎಳೆಯ ಮಗುವಿನಿಂದ ಕೇಳಿಸಿಕೊಂಡು ತಾಯಿ ಅದಕ್ಕೆ ಕಾಲಕಾಲಕ್ಕೆ ಆಹಾರಾದಿಗಳನ್ನು ಕೊಡುವುದಿಲ್ಲ. ಕೇಳಿಸಿಕೊಳ್ಳದೆಯೇ ಅದಕ್ಕೆ ಎಲ್ಲ ವನ್ನೂ ಮಾಡುತ್ತಾಳೆ. ಹಾಗೆಯೇ ಭಗವಂತ ಭಕ್ತರ ಸರ್ವಸ್ವವನ್ನೂ ನೋಡಿಕೊಳ್ಳುತ್ತಾನೆ. ಆದರೆ ಭಕ್ತ ನಿಜವಾಗಿಯೂ ಭಗವಂತನ ಭಕ್ತನಾಗಿರಬೇಕು. ಅವನ ಭಕ್ತಿ ಅನನ್ಯವಾಗಿರಬೇಕು. ಯಾವಾ ಗಲೂ ಭಗವಂತನನ್ನೇ ಚಿಂತಿಸುತ್ತಿರಬೇಕು. ಭಗವಂತನನ್ನು ಏಕಮಾತ್ರ ಭಕ್ತಿಯಿಂದ ಆರಾಧಿಸು ತ್ತಿರಬೇಕು. ಅವನು ದೇವರಿಂದ ಯಾವ ಲೌಕಿಕವಾದುದನ್ನೂ ಬೇಡುವುದಕ್ಕೆ ಹೋಗುವುದಿಲ್ಲ. ಅವನಿಗೆ ದೇವರ ಉಗ್ರಾಣದಲ್ಲಿರುವ ಅಷ್ಟೈಶ್ವರ್ಯ ಬೇಕಾಗಿಲ್ಲ. ಭಗವಂತನ ಪ್ರೀತಿಯೊಂದೇ ಬೇಕಾಗಿರುವುದು. ಅನನ್ಯ ಎಂದರೆ ಒಂದೇ ಸಮನಾಗಿ ಭಗವಂತನ ಕಡೆ ಹರಿಯುತ್ತಿರಬೇಕು. ಒಂದು ಕಾಲದಲ್ಲಿ ಅವನನ್ನು ಚಿಂತಿಸುವುದು, ಮತ್ತೊಂದು ಕಾಲದಲ್ಲಿ ಆತನನ್ನು ಚಿಂತಿಸದೇ ಇರುವುದು, ಹಾಗಲ್ಲ. ಲಾಭನಷ್ಟಗಳಲ್ಲಿ, ಸುಖದುಃಖಗಳಲ್ಲಿ, ಜನನಮರಣಗಳಲ್ಲಿ, ಒಂದೇ ಸಮನಾಗಿ ಭಕ್ತ ಭಗವಂತನನ್ನು ಆಶಿಸುತ್ತಿದ್ದರೆ ಅದು ಅನನ್ಯವಾಗುವುದು.

ಇಂತಹ ನಿತ್ಯಯುಕ್ತರಾದವರ ಯೋಗಕ್ಷೇಮವನ್ನು ದೇವರು ನೋಡಿಕೊಳ್ಳುತ್ತಾನೆ. ಇಂತಹ ಭಕ್ತರು ಪ್ರಾಪಂಚಿಕವಾದ ಏನನ್ನೂ ಮಾಡಲು ಸಾಧ್ಯವಿಲ್ಲ. ಅವರ ಮನಸ್ಸೆಲ್ಲಾ ಭಗವಂತನಿಂದ ತುಂಬಿ ತುಳುಕಾಡುತ್ತಿರುವುದು. ಯಾರು ತಮ್ಮ ಸರ್ವಸ್ವವನ್ನೂ ಭಗವಂತನಿಗೆ ಧಾರೆಯೆರೆದು ಅವನಲ್ಲಿಯೇ ಶರಣಾಗಿ, ಅವನನ್ನೇ ಪ್ರೀತಿಸುವರೋ ಅವರ ಜವಾಬ್ದಾರಿಯನ್ನು ದೇವರು ನೋಡಿಕೊಳ್ಳುವನು. ಅವನ ಹೆಸರಿನಲ್ಲಿ ನಾವು ನಿಂತರೆ ಅವನು ನಮ್ಮ ಕೈಹಿಡಿದು ನಡೆಸುವನು. ಅವನು ಭಕ್ತನ ಯೋಗ ಮತ್ತು ಕ್ಷೇಮ ಎರಡನ್ನೂ ನೋಡಿಕೊಳ್ಳುವೆನೆಂದು ಭರವಸೆ ಇತ್ತಿದ್ದಾನೆ. ಯೋಗ ಎಂದರೆ ಯಾವುದು ಅತ್ಯಂತ ಆವಶ್ಯಕವಾಗಿದೆಯೋ ಅವನ್ನೆಲ್ಲಾ ಅವನು ಅನುಗ್ರಹಿಸುವನು. ಕ್ಷೇಮ ಎಂದರೆ ನಮಗೆ ಅತ್ಯಂತ ಆವಶ್ಯಕವಾಗಿರುವುದನ್ನು ರಕ್ಷಿಸುವನು.

ದೇಹಕ್ಕೆ ಆರೋಗ್ಯ ಬೇಕು, ದೃಢತೆ ಬೇಕು. ಆಗ ಮಾತ್ರ ಸಾಧನೆ ಮಾಡುಲು ಸಾಧ್ಯವಾಗುವುದು. ಈ ದೇಹ ಎಂಬ ಯಂತ್ರ ಕೆಲಸ ಮಾಡಬೇಕಾದರೂ ಅದಕ್ಕೆ ಆಹಾರಾದಿಗಳು ಬೇಕು. ಇಲ್ಲದೆ ಇದ್ದರೆ ದುರ್ಬಲವಾಗಿ ಹಲವು ರೋಗರುಜಿನಗಳಿಗೆ ತುತ್ತಾಗುವುದು. ದೇವರಿಗೆ ನಾವು ಪರಿಪೂರ್ಣವಾಗಿ ಮನಸ್ಸನ್ನು ಅರ್ಪಿಸಿದರೆ, ನಮಗೆ ಏನೇನು ಬೇಕೋ ಅದನ್ನೆಲ್ಲಾ ಅವನು ಕೊಡುವನು. ಯಾರ ಮೂಲಕವಾಗಿಯೋ ಅದನ್ನು ಒದಗಿಸುವನು. ಹಲವು ಭಕ್ತರ ಜೀವನವೇ ಇದಕ್ಕೆ ಉದಾಹರಣೆ ಯಾಗಿದೆ. ಹೇಗೋ ಎಂತೋ ದೇವರು ಇವರಿಗೆ ಅತ್ಯಂತ ಆವಶ್ಯಕವಾಗಿ ಬೇಕಾದುದನ್ನು ಕೊಟ್ಟು ಹೋಗಿರುವನು. ಹಿಂದೆ ಮಾತ್ರವಲ್ಲ, ಎಂದೆಂದಿಗೂ ಈ ನಿಯಮ ಸತ್ಯ. ಹಾಗೆಯೇ ಯಾವುದನ್ನು ಅತ್ಯಂತ ಆವಶ್ಯಕವಾಗಿ ಸಂರಕ್ಷಿಸಬೇಕಾಗಿದೆಯೊ ಅದನ್ನು ಕೂಡಾ ಅವನೇ ಮಾಡುತ್ತಾನೆ. ಅವನು ಹೊರಗಿನ ರಕ್ಷೆಯನ್ನು ಬೇಡುವುದಕ್ಕೆ ಹೋಗುವುದಿಲ್ಲ. ಭಗವಂತನೇ ಅವನ ಏಕಮಾತ್ರ ರಕ್ಷೆ. ಅವನು ದೇವರನ್ನು ರಕ್ಷಿಸು ಎಂತಲೂ ಕೇಳುವುದಿಲ್ಲ. ಕೇಳದೆ ಇದ್ದರೂ ಅದನ್ನು ಮಾಡಿಯೇ ಮಾಡುತ್ತಾನೆ. ಭಗವಂತನನ್ನು ಸಂಪೂರ್ಣವಾಗಿ ಪ್ರೀತಿಸಿದರೆ ಅವನಲ್ಲಿ ಶರಣಾಗತನಾದರೆ, ಭಕ್ತನಿಗೆ ಯಾವ ಕೊರತೆಯೂ ಇರುವುದಿಲ್ಲ.

\begin{verse}
ಯೇಽಪ್ಯನ್ಯದೇವತಾ ಭಕ್ತಾ ಯಜಂತೇ ಶ್ರದ್ಧಯಾನ್ವಿತಾಃ ।\\ತೇಽಪಿ ಮಾಮೇವ ಕೌಂತೇಯ ಯಜಂತ್ಯವಿಧಿಪೂರ್ವಕಮ್ \versenum{॥ ೨೩ ॥}
\end{verse}

{\small ಅರ್ಜುನ, ಯಾರು ಶ್ರದ್ಧಾಪೂರ್ವಕ ಅನ್ಯ ದೇವತೆಗಳನ್ನು ಪೂಜಿಸುತ್ತಾರೊ ಅವರು ಕೂಡಾ ವಿಧಿವಿಹೀನವಾಗಿ ಪೂಜಿಸಿದರೂ ನನ್ನನ್ನೇ ಪೂಜಿಸುತ್ತಾರೆ.}

ಹಲವರಿಗೆ ದೇವರಿಗಾಗಿ ದೇವರು ಬೇಕಿಲ್ಲ. ಅವರಿಗೆ ಇನ್ನೂ ಹಲವು ಸಾಂಸಾರಿಕ ವಾಸನೆಗಳಿವೆ. ಅವರು ಪ್ರಾಪಂಚಿಕ ವಸ್ತುಗಳನ್ನು ಅನುಭವಿಸಬೇಕೆಂದು ಆಶಿಸುತ್ತಾರೆ. ಅದಕ್ಕಾಗಿ ತಮ್ಮ ಉದ್ದೇಶ ಸಾಧನೆಗೆ ಒಬ್ಬೊಬ್ಬ ದೇವತೆಗಳ ಹತ್ತಿರ ಹೋಗುತ್ತಾರೆ. ಐಶ್ವರ್ಯ ಬೇಕಾದವರು ಲಕ್ಷ್ಮಿಯ ಹತ್ತಿರ ಹೋಗುತ್ತಾರೆ. ಕಾಲರ, ಪ್ಲೇಗು, ಸಿಡುಬು ಮುಂತಾದುವು ಬರದೇ ಇರಲೆಂದು ಆಯಾ ದೇವತೆಗಳ ಹತ್ತಿರ ಹೋಗಿ ಬೇಡಿಕೊಳ್ಳುತ್ತಾರೆ. ಅವರು ತಾವು ಯಾವ ದೇವರನ್ನಾದರೂ ಪೂಜಿಸಲಿ, ಕೇಳಲಿ, ಅದರ ಹಿಂದೆ ಶ್ರದ್ಧೆ ಇದ್ದರೆ, ಅದನ್ನು ಸರ್ವವ್ಯಾಪಿಯಾದ ಭಗವಂತನೇ ನಡೆಸಿಕೊಡುವನು. ಅವರಿಗೆ ತಾವು ಮಾಡುವ ಕೆಲಸದಲ್ಲಿ ಮಾತ್ರ ಶ್ರದ್ಧೆ ಇರಬೇಕು. ಕಾಟಾಚಾರಕ್ಕಾಗಿ ಮಾಡಕೂಡದು. ಯಾವಾಗ ಶ್ರದ್ಧೆ ಇದೆಯೋ ಆಗ ದೇವರು ಅದನ್ನು ಲಾಲಿಸುತ್ತಾನೆ. ಯಾವುದೋ ಮಾರಮ್ಮನನ್ನೋ, ಶನಿಯನ್ನೊ, ಗಣಪತಿಯನ್ನೊ ಕೇಳುತ್ತಾನೆ. ನಾನೇಕೆ ಕೊಡಬೇಕು ಎನ್ನುವುದಿಲ್ಲ ದೇವರು. ಇವರೆಲ್ಲ ಭಗವಂತನ ಪರವಾಗಿ ಇರುವವರು. ಪ್ರತಿಯೊಬ್ಬನೂ ತನ್ನ ವಿಕಾಸಕ್ಕೆ ತಕ್ಕಂತಹ ಭಗವಂತನ ಭಾವನೆಯನ್ನು ಕಲ್ಪಿಸಿಕೊಂಡು ಅವನಲ್ಲಿ ಶರಣಾಗುತ್ತಾನೆ. ಎಲ್ಲ ದೇವ ದೇವಿಯರ ಹಿಂದೆ ಕೆಲಸ ಮಾಡುವುದು ಭಗವಂತನ ಶಕ್ತಿಯೆ. ನಮ್ಮ ಮನೆಯ ಮುಂದಿರುವ ಅಂಚೆ ಪೆಟ್ಟಿಗೆಗೆ ನಾವು ಕಾಗದವನ್ನು ಹಾಕುತ್ತೇವೆ. ದೊಡ್ಡ ಪೋಸ್ಟಾಫೀಸಿನವರು ತಮ್ಮಲ್ಲಿ ಹಾಕಿದ ಕಾಗದವನ್ನು ಮಾತ್ರ ವಿಲೇವಾರಿ ಮಾಡಿ, ಜನ ತಮ್ಮ ಮನೆ ಹತ್ತಿರ ಇರುವ ಪೋಸ್ಟ್ ಬಾಕ್ಸ್​ಗೆ ಹಾಕಿದ್ದರೆ ‘ನನಗೆ ಹಾಕಿಲ್ಲ’ ಎಂದು ಸುಮ್ಮನೆ ಇರುತ್ತಾರೆಯೆ? ರಸ್ತೆಯ ಅಂಚಿನ ಪೆಟ್ಟಿಗೆಯಲ್ಲಿರುವುದನ್ನು ಹೊತ್ತುಕೊಂಡು ಹೋಗುವುದಕ್ಕೂ ದೊಡ್ಡ ಪೋಸ್ಟಾಫೀಸಿನಿಂದ ನೌಕರ ಬರುತ್ತಾನೆ. ಹಾಗೆಯೇ ಬೇರೆ ಬೇರೆ ದೇವತೆಗಳು. ಅವರ ಹಿಂದೆ ಕೆಲಸ ಮಾಡುತ್ತಿರುವುದು ಒಂದೇ ದೇವರ ಶಕ್ತಿ. ನಿಜವಾಗಿ ಹಲವಾರು ದೇವತೆಗಳಿಲ್ಲ, ಒಂದೇ ದೇವರ ಹಲವು ರೂಪಗಳಿವೆ ಅಷ್ಟೆ.

ಲೌಕಿಕವಾದುದನ್ನು ಆಯಾ ದೇವತೆಗಳಿಂದ ಯಾಚಿಸುತ್ತಿರುವವನು ಅವಿಧಿಪೂರ್ವಕ ಮಾಡು ತ್ತಿರುವನು. ಎಂದರೆ ಅವರಿಗೆ ಭಗವಂತನ ಜ್ಞಾನವಿಲ್ಲ. ಅವನು ಸರ್ವವ್ಯಾಪಿ, ಎಲ್ಲ ಕಡೆ, ಎಲ್ಲಾ ದೇವದೇವತೆಗಳ ಮೂಲಕ ಕೆಲಸ ಮಾಡುತ್ತಿರುವನು ಎಂಬ ಜ್ಞಾನವಿಲ್ಲ ಅವನಿಗೆ. ಅವನಿಗೆ ಜ್ಞಾನವಿಲ್ಲದೇ ಇದ್ದರೆ ದೇವರು ಮಾಡುವ ಕೆಲಸವನ್ನು ಮಾಡದೆ ಬಿಡುತ್ತಾನೆಯೆ? ಯಾವುದನ್ನೊ ಬಾಯಿ ರುಚಿಗೆ ತಿನ್ನುತ್ತೇವೆ ನಾವು. ಕ್ರಮೇಣ ಅದು ನಾಲಿಗೆಯಿಂದ ಜಾರಿ, ಹೊಟ್ಟೆಗೆ ಹೋಗಿ ಅಲ್ಲಿ ಹಲವಾರು ರಸಗಳೊಂದಿಗೆ ಬೆರತು ಜೀರ್ಣವಾಗಿ ನಮ್ಮ ದೇಹದ ರಕ್ತ ಮಾಂಸವಾಗುವುದು. ನಾವು ಈ ಉದ್ದೇಶದಿಂದ ತಿನ್ನಲಿಲ್ಲ. ಬರೀ ಬಾಯಿ ಚಪಲಕ್ಕೆ ತಿಂದೆವು ಎಂದರೆ ಹೊಟ್ಟೆ ತಾನು ಮಾಡುವ ಕೆಲಸವನ್ನು ಮಾಡುವುದಿಲ್ಲವೆ? ತನ್ನ ದೇಹಕ್ಕೆ ಶಕ್ತಿ ಬರಲಿ ಎಂದು ತಿಂದರೂ ಹೊಟ್ಟೆ ಅದೇ ಕೆಲಸಮಾಡುವುದು. ಇದಾವುದು ಗೊತ್ತಿಲ್ಲದೆ ರುಚಿಯಾಸೆಗೆ ತಿಂದರೂ ಹೊಟ್ಟೆ ಅದೇ ಕೆಲಸ ಮಾಡುವುದು. ಹಾಗೆಯೇ ಸರ್ವವ್ಯಾಪಿಯಾದ ಭಗವಂತನಿಲ್ಲದ ಸ್ಥಳವಿಲ್ಲ. ಅವನು ಸರ್ವಶಕ್ತ ಎಂದು ಯಾವ ಹೆಸರೂ ಕೊಡದೆ ಚಿಂತಿಸಿದರೂ ಅವನಿಗೆ ಹೋಗುವುದು. ಈ ವೇದಾಂತ ಗೊತ್ತಿಲ್ಲದ ಹಳ್ಳಿಯವನೂ,‘ನಮ್ಮಮ್ಮ ಮಾರಮ್ಮ, ನನ್ನ ಪ್ರಾರ್ಥನೆ ಕೇಳುತ್ತಾಳೆ, ನಮಗೆ ಒಳ್ಳೆಯದನ್ನು ಮಾಡುತ್ತಾಳೆ’ ಎಂದು ಭಾವಿಸಿದರೂ ಅದನ್ನೆಲ್ಲಾ ನೆರೆವೇರಿಸಿ ಕೊಡುವವನೂ ಆ ಪರಮಾತ್ಮನೆ.

\begin{verse}
ಅಹಂ ಹಿ ಸರ್ವಯಜ್ಞಾನಾಂ ಭೋಕ್ತಾ ಚ ಪ್ರಭುರೇವ ಚ ।\\ನ ತು ಮಾಮಭಿಜಾನಂತಿ ತತ್ತ್ವೇನಾತಶ್ಚ್ಯವಂತಿ ತೇ \versenum{॥ ೨೪ ॥}
\end{verse}

{\small ಸರ್ವ ಯಜ್ಞಗಳ ಭೋಕ್ತೃವೂ, ಪ್ರಭುವೂ ನಾನೆ. ಆದರೆ ಅವರು ನನ್ನನ್ನು ಯಥಾರ್ಥವಾಗಿ ತಿಳಿಯರು. ಆದುದರಿಂದ ಅವರು ಚ್ಯುತರಾಗುತ್ತಾರೆ.}

ಭೋಗ ವಾಸನೆಯುಳ್ಳವರು, ಹಲವು ಯಾಗ ಯಜ್ಞಗಳನ್ನು ಮಾಡಿ ಬೇರೆ ಬೇರೆ ಪ್ರಾರ್ಥನೆ ಮಾಡಿದರೂ, ಕೊನೆಗೆ ಅದೆಲ್ಲವೂ ಸರ್ವ ದೇವೇಶನಾದ ಪರಮಾತ್ಮನಿಗೆ ಹೋಗುವುದು. ಇವರ ಪ್ರಾರ್ಥನೆ ಈಡೇರುವುದೂ ಕೂಡಾ ಆ ಒಂದೇ ಪರಮಾತ್ಮನಿಂದ. ಬೇರೆ ಬೇರೆ ದೇವರುಗಳೆಲ್ಲ ಆ ಒಂದೇ ಭಿನ್ನ ಭಿನ್ನ ಸ್ವರೂಪಗಳು ಅಷ್ಟೆ. ನಾವು ಮಾಡುವುದೆಲ್ಲಾ ಅವನಿಗೆ ಅರ್ಪಿತವಾಗುವುದು. ಎಲ್ಲಾ ಯಾಗ ಯಜ್ಞಗಳಿಗೂ ಒಡೆಯನು ಅವನು. ಆದರೆ ಅಜ್ಞರು ಯಥಾರ್ಥವಾಗಿ ಇದನ್ನು ತಿಳಿಯರು. ಬೇರೆ ಬೇರೆ ದೇವರು ಇವರು ಕೇಳಿದುದನ್ನು ಕೊಡುವುದಕ್ಕೆ ಸ್ವತಂತ್ರರಾಗಿರುವರೆಂದು ಭಾವಿಸಿದರೂ ಕೊಡುವವನು ಒಬ್ಬನೆ. ಅವನು ಹೇಗೆ ಇದ್ದಾನೆಯೊ ಹಾಗೆ ತಿಳಿದುಕೊಂಡರೆ ಇತರ ಕ್ಷುದ್ರದೇವತೆಗಳ ಬಳಿಗೆ ಹೋಗದೆ ಸರ್ವವ್ಯಾಪಿಯಾದ ಭಗವಂತನೆಡೆಗೆ ಬರುವನು. ದೇವರೊಬ್ಬನೆ ಈ ಜೀವನದಲ್ಲಿ ಪಡೆಯಲು ಯೋಗ್ಯವಾದುದು, ಉಳಿದವುಗಳೆಲ್ಲ ಕ್ಷಣಿಕ, ಕೆಲಸಕ್ಕೆ ಬಾರದವು ಎಂಬುದನ್ನು ಚೆನ್ನಾಗಿ ಅರಿತವನು ದೇವರನ್ನು ಲೌಕಿಕ ವಸ್ತುಗಳಿಗಾಗಿ ಹೇಗೆ ಕೇಳುತ್ತಾನೆ! ಇದು ವಜ್ರದ ಗಣಿಗೆ ಹೋಗಿ ಗಾಜಿನ ಚೂರನ್ನು ಕೇಳಿದಂತೆ. ಆದರೆ ಇದು ಅವರಿಗೆ ತಿಳಿಯದು. ಅಜ್ಞಾನದಲ್ಲಿದ್ದಾರೆ. ಅವರಿಗೆ ಬೆಲೆ ಬಾಳತಕ್ಕಂತಹ ವಸ್ತುಗಳೇ ಇಂದ್ರಿಯಕ್ಕೆ ಪ್ರಿಯವಾದ ವಿಷಯ ವಸ್ತುಗಳು. ಅದನ್ನೇ ದೇವರಿಂದ ಬೇಡುತ್ತಾರೆ. ದೇವರೂ ಕೂಡಾ ಅದನ್ನೆ ಕೊಡುತ್ತಾನೆ. ಅದೇ ನಾದರೂ ಕೆಟ್ಟರೆ, ಕಳೆದು ಹೋದರೆ, ದುಃಖಿತರಾಗಿ ಹಳೆಯ ಆಟದ ಸಾಮಾನಿನ ರಿಪೇರಿಗೊ, ಹೊಸ ಆಟದ ಸಾಮಾನನ್ನು ಪಡೆಯುವುದಕ್ಕೊ ಪುನಃ ದೇವರ ಹತ್ತಿರ ಬರುತ್ತಾರೆ. ಅವರಿಗೆ ಗೊತ್ತಿದೆ, ದೇವರ ಹತ್ತಿರ ಬಗೆಬಗೆಯ ಆಟದ ಸಾಮಾನುಗಳಿವೆ, ಅವನು ಯಾವುದನ್ನು ಕೇಳಿದರೂ ಕೊಡಬಲ್ಲ ಎಂಬುದು. ಆದರೆ ದೇವರಲ್ಲದೆ ಅನ್ಯವಸ್ತುಗಳನ್ನು ಕೇಳಿದರೆ ಆ ವಸ್ತುಗಳಿಗೆ ದಾಸನಾಗುತ್ತಾನೆ. ಅದಕ್ಕಾಗಿ ಪಡಬಾರದ ಕಷ್ಟ ಪಡಬೇಕಾಗುವುದು, ದುಃಖ ಅನುಭವಿಸಬೇಕಾಗುವುದು, ಎಂಬುದನ್ನು ಇನ್ನೂ ಅವನು ಅರಿಯಬೇಕಾಗಿದೆ. ಅವನು ದೇವರನ್ನು ಸೇರದೆ ಕೆಳಕ್ಕೆ ಬೀಳುವುದಕ್ಕೆ ಕಾರಣವೇ ದೇವರ ಮೇಲೆ ಇನ್ನೂ ಪ್ರೀತಿ ಉದಿಸದೇ ಇರುವುದು. ಅವನೇನೋ ದೇವರ ಮನೆ ಬಾಗಿಲಿಗೆ ಹೋಗುತ್ತಾನೆ. ಆದರೆ ಹೋಗಿ ಅಲ್ಲಿ ಕೇಳುವುದೇನು? ಭಿಕ್ಷುಕನಂತೆ ಒಂದು ಹಿಡಿ ಭಿಕ್ಷೆ. ದೇವರು ಅದನ್ನು ಕೊಟ್ಟು ಕಳುಹಿಸುತ್ತಾನೆ. ಆದರೆ ದೇವರನ್ನು ಪ್ರೀತಿಸಿದರೆ, ಅವನೇ ಬೇಕು ಬೇರೆ ಯಾವುದೂ ಅಲ್ಲ ಎಂಬುದನ್ನು ನಮ್ಮ ವ್ಯಾಕುಲತೆಯಲ್ಲಿ ವ್ಯಕ್ತಪಡಿಸಿದರೆ, ಅವನು ಪರಜ್ಞಾನ ಮತ್ತು ಪರಭಕ್ತಿ ಯನ್ನು ಕೊಡಲು ಸಿದ್ಧನಾಗಿರುವನು. ಆದರೆ ಅದನ್ನು ಬೇಡುವವರು ಬೇಕಲ್ಲ!

\begin{verse}
ಯಾಂತಿ ದೇವವ್ರತಾ ದೇವಾನ್ ಪಿತೄನ್ ಯಾಂತಿ ಪಿತೃವ್ರತಾಃ ।\\ಭೂತಾನಿ ಯಾಂತಿ ಭೂತೇಜ್ಯಾ ಯಾಂತಿ ಮದ್ಯಾಜಿನೋಽಪಿ ಮಾಮ್ \versenum{॥ ೨೫ ॥}
\end{verse}

{\small ದೇವತೆಗಳನ್ನು ಪೂಜೆ ಮಾಡುವವರಿಗೆ ದೇವಲೋಕ, ಪಿತೃಗಳನ್ನು ಪೂಜೆ ಮಾಡುವವರಿಗೆ ಪಿತೃಲೋಕ ಸಿಕ್ಕುವುದು. ಭೂತಪ್ರೇತಾದಿಗಳನ್ನು ಪೂಜಿಸುವವರಿಗೆ ಆಯಾ ಲೋಕಗಳು ಸಿಕ್ಕುತ್ತವೆ. ನನ್ನನ್ನು ಭಜಿಸುವವರು ನನ್ನನ್ನೇ ಹೊಂದುತ್ತಾರೆ.}

ಒಬ್ಬ ದೊಡ್ಡ ಶ್ರೀಮಂತನ ಮನೆಯಲ್ಲಿ ಮನೆಯ ಯಜಮಾನನೇ ಇರುವ ಕೋಣೆ ಇದೆ. ಅಲ್ಲಿ ಅಡಿಗೆಯವನು ಇರುವ ಸ್ಥಳವಿದೆ. ಅಲ್ಲಿ ಪರಿಚಾರಕರು ಇರುವ ಸ್ಥಳವಿದೆ. ಅಲ್ಲಿ ತೋಟಗಾರ ಇರುವ ಸ್ಥಳವೂ ಇದೆ. ಎಲ್ಲಾ ಮನೆ ಯಜಮಾನನಿಗೇ ಸೇರಿದ್ದು. ಆದರೆ ಯಾರು ಆ ಮನೆಯಲ್ಲಿ ಯಾರೊಡನೆ ಸ್ನೇಹ ಬೆಳೆಸಿದ್ದಾನೆಯೋ ಅವನ ಸಮೀಪಕ್ಕೆ ಹೋಗುತ್ತಾನೆ. ಅವನಿಗೆ ಅವನೇ ಪ್ರಿಯ. ಅಡಿಗೆಯವನ ಸ್ನೇಹಿತ ಅಡಿಗೆಯವನನ್ನು ಹುಡುಕಿಕೊಂಡು ಹೋಗುತ್ತಾನೆ. ತೋಟಗಾರನ ಸ್ನೇಹಿತ ತೋಟಗಾರನನ್ನು ಹುಡುಕಿಕೊಂಡು ಹೋಗುತ್ತಾನೆ. ಯಜಮಾನನ ಸ್ನೇಹಿತ ಯಜಮಾನನಿರುವ ದಿವಾನಖಾನೆಗೆ ಹೋಗುತ್ತಾನೆ. ಹಾಗೆಯೇ ಭಗವಂತನ ಮನೆಯಲ್ಲಿ ಬಗೆ ಬಗೆಯ ದೇವತೆಗಳೆಂಬ ಸ್ಥಾನಗಳಿವೆ. ಅವರು ಕೆಲಸದಲ್ಲಿರುವುದು, ಅವರು ಮಾಡುತ್ತಿರುವ ಕೆಲಸಗಳು ಎಲ್ಲಾ ಒಬ್ಬನೇ ದೇವನಿಗಾಗಿ. ಆ ಒಬ್ಬ ದೇವನ ಶಕ್ತಿಯೇ ಇವುಗಳ ಮೂಲಕ ಕೆಲಸ ಮಾಡಿದರೂ ಒಂದೊಂದು ಕಡೆ ಒಂದೊಂದು ವಿಧವಾಗಿರುವುದು. 

ಇಂದ್ರ ಮುಂತಾದ ದೇವತೆಗಳನ್ನು ಪ್ರೀತಿಸುವವರು ಇಂದ್ರ ಲೋಕಕ್ಕೆ ಹೋಗುತ್ತಾರೆ. ಅಲ್ಲಿರುವ ಭೋಗವಸ್ತುಗಳನ್ನು ಅನುಭವಿಸುತ್ತಾರೆ. ತಮ್ಮ ಪುಣ್ಯದ ಬುತ್ತಿ ತೀರಿದ ಮೇಲೆ ಪ್ರಪಂಚಕ್ಕೆ ಉರುಳುತ್ತಾರೆ. ಮತ್ತೆ ಕೆಲವರಿಗೆ ಗತಿಸಿಹೋದ ಪಿತೃಗಳ ಮೇಲೆ ತುಂಬಾ ಆಸಕ್ತಿ. ಆ ಪಿತೃಗಳೇ ನಾದರೂ ಬೇರೆ ಜನ್ಮವನ್ನು ಪಡೆಯದೆ ಪಿತೃ ಲೋಕದಲ್ಲಿದ್ದರೆ ಅಲ್ಲಿಗೆ ಹೋಗಿ ಕೆಲವು ಕಾಲ ಇರುತ್ತಾರೆ. ಆದರೆ ಎಂದೆಂದಿಗೂ ಅಲ್ಲಿ ಇರುವುದಕ್ಕೆ ಆಗುವುದಿಲ್ಲ. ಆ ಲೋಕದಿಂದಲೂ ಕೆಳಗೆ ಉರುಳಬೇಕಾಗಿದೆ. ಏಕೆಂದರೆ ಇವುಗಳೆಲ್ಲ ತಾತ್ಕಾಲಿಕ ಸ್ಥಳಗಳು, ಎಂದೆಂದಿಗೂ ಅಲ್ಲಿಯೇ ಬೇರು ಬಿಡುವುದಕ್ಕೆ ಆಗುವುದಿಲ್ಲ. ದಾರಿಯಲ್ಲಿ ಸಿಕ್ಕುವ ಮುಸಾಫಿರಖಾನೆಯಂತೆ ಒಂದೆರಡು ದಿನ ತಂಗಬಹುದು ಅಷ್ಟೆ. ಅನಂತರ ಗಂಟು ಮೂಟೆ ಕಟ್ಟಿಕೊಂಡು ಮುಂದೆ ಹೋಗಬೇಕಾಗಿದೆ. ಒಂದು ವೇಳೆ ಆ ಪಿತೃಗಳು ಈ ಪೃಥ್ವಿಯಲ್ಲಿಯೇ ತಮ್ಮ ಕರ್ಮ ಸವೆಸಲು ಜನ್ಮ ತಾಳಿದ್ದರೆ, ಅಂತಹ ಮನೆಯಲ್ಲಿ ಹುಟ್ಟಿ, ಆ ವಾತಾವರಣದಲ್ಲಿ ಬೆಳೆದು, ಅವರ ಪ್ರೀತಿಯ ಸವಿ ನೋಡುವರು. ಅದರ ಪುಣ ತೀರಿದ ಮೇಲೆ ಅಥವಾ ಬೇಜಾರಾದಮೇಲೆ ಆ ಸ್ಥಳ ಬಿಟ್ಟುಕೊಡುವರು. ನಾವೆಲ್ಲ ಈ ಪ್ರಪಂಚದಲ್ಲಿ ಯಾರಿಗೋ ಮಗನಾಗಿಯೋ ಮಗಳಾಗಿಯೋ ಹುಟ್ಟುವುದಕ್ಕೆ ಇದೇ ಕಾರಣ.

ಅನಂತರ ಅತೀಂದ್ರಿಯ ಶಕ್ತಿಯನ್ನು ಪಡೆದು ಕೆಲವು ಕಾಲ ಭೂತಾವಸ್ಥೆಯಲ್ಲಿರುವ ಕೆಲವು ಶಕ್ತಿಗಳು ಬೇರೆ ಇರುತ್ತವೆ. ಕೆಲವರು ಅದಕ್ಕೆ ಭಕ್ತರು. ಏಕೆಂದರೆ ಆ ಭೂತಗಳ ಸಹಾಯದಿಂದ ಭವಿಷ್ಯ ಹೇಳುವುದು, ಮಾಯ ಮಂತ್ರಗಳನ್ನು ಮಾಡುವುದು, ಕೆಲವು ಖಾಯಿಲೆಗಳನ್ನು ಗುಣಮಾಡು ವುದು ಮುಂತಾದ ವರಗಳನ್ನು ಪಡೆದು ಈ ಸಂಸಾರದಲ್ಲಿ ಅವರು ವ್ಯವಹರಿಸುತ್ತಿರುವರು. ಆ ಭೂತದ ದಯೆಯಿಂದ ಇವರಿಗೆ ಸ್ವಲ್ಪ ಹಣ, ಕೀರ್ತಿ ಹೆಸರು ಮುಂತಾದುವುಗಳೆಲ್ಲಾ ಲಭಿಸುತ್ತವೆ. ಅದಕ್ಕಾಗಿ ಅವರು ಭೂತಕ್ಕೆ ಭಕ್ತರು. ಅವರು ಭೂತದ ಸಮೀಪಕ್ಕೆ ಹೋಗುತ್ತಾರೆ.

ಆದರೆ ಯಾರು ದೇವರನ್ನು ಪ್ರೀತಿಸುತ್ತಾನೊ, ಯಾರಿಗೆ ಈ ಪ್ರಪಂಚದಲ್ಲಿ ಅವನಲ್ಲದೆ ಮತ್ತಾವುದು ಬೇಡವೊ, ಅಂತಹ ಪರಮಭಕ್ತ ಭಗವಂತನನ್ನು ಸೇರುತ್ತಾನೆ. ಯಾರಿಗೂ ದೇವರು, ನನ್ನೆಡೆಗೇ ಬನ್ನಿ, ಎಂದು ಬಲಾತ್ಕಾರ ಮಾಡುವುದಿಲ್ಲ. ಬೇಕಾದಷ್ಟು ಸ್ವಾತಂತ್ರ್ಯವನ್ನು ಕೊಟ್ಟಿದ್ದಾನೆ. ಅವರು ಕೇಳಿದ್ದನ್ನು ಕೊಡುತ್ತಾನೆ. ಇಚ್ಛೆ ಬಂದೆಡೆ ಹೋಗಲು ಬಿಡುತ್ತಾನೆ.

\begin{verse}
ಪತ್ರಂ ಪುಷ್ಪಂ ಫಲಂ ತೋಯಂ ಯೋ ಮೇ ಭಕ್ತ್ಯಾ ಪ್ರಯಚ್ಛತಿ ।\\ತದಹಂ ಭಕ್ತ್ಯುಪಹೃತಮಶ್ನಾಮಿ ಪ್ರಯತಾತ್ಮನಃ \versenum{॥ ೨೬ ॥}
\end{verse}

{\small ಯಾರು ಭಕ್ತಿಯಿಂದ ಎಲೆ, ಹೂ, ಹಣ್ಣು, ನೀರು ಯಾವುದನ್ನು ಕೊಡುವರೋ ಅಂತಹ ಶುದ್ಧ ಚಿತ್ತವುಳ್ಳವರು ಭಕ್ತಿಯಿಂದ ಸಮರ್ಪಿಸಿದ ಅದನ್ನು ನಾನು ಸ್ವೀಕರಿಸುತ್ತೇನೆ.}

ಭಕ್ತ ಏನನ್ನು ಕೊಡುತ್ತಾನೆಯೋ ಅದನ್ನು ದೇವರು ಸ್ವೀಕರಿಸುತ್ತಾನೆ. ಅವನೆಷ್ಟು ಕೊಡುತ್ತಾನೆ, ಏನು ಕೊಡುತ್ತಾನೆ ಎಂಬುದನ್ನು ನೋಡುವುದಿಲ್ಲ. ಅವನು ಯಾವ ದೃಷ್ಟಿಯಿಂದ ಕೊಡುವನು ಎಂಬುದನ್ನು ಮಾತ್ರ ನೋಡುತ್ತಾನೆ. ನಮ್ಮ ಮನಸ್ಸು ಶುದ್ಧವಾಗಿದ್ದರೆ, ನಾವು ಕೊಡುವುದನ್ನೆಲ್ಲಾ ಸ್ವೀಕರಿಸುವನು. ಈ ಪ್ರಪಂಚದಲ್ಲಿ ದೇವರಿಗೆ ಕೊಡುವುದು ಸುಲಭ. ದೇವರನ್ನು ತೃಪ್ತಿಪಡಿಸುವುದು ಸುಲಭ. ಅವನಿಗೆ ಕೊಡುವುದಕ್ಕೆ ನಮಗೆ ಗತಿ ಇಲ್ಲ ಎಂದು ಯಾವ ಭಿಕಾರಿಯು ಹೇಳುವುದಿಲ್ಲ. ನಮಗೆ ಕೊಡಲಾರದಿರುವುದನ್ನು ಅವನು ಕೇಳುವುದಿಲ್ಲ. ಒಂದು ದಳ ತುಳಸಿಯೋ ಬಿಲ್ವಪತ್ರೆಯೋ ಸಾಕು. ಒಂದು ಹೂವು ಹಣ್ಣು ಸಾಕು. ಒಂದು ಉದ್ಧರಣೆ ನೀರು ಸಾಕು. ಇದು ಅಲ್ಪ ಎಂದು ತಿರಸ್ಕರಿಸುವುದಿಲ್ಲ. ಅದನ್ನು ಸಂತೋಷದಿಂದ ಸ್ವೀಕರಿಸುತ್ತಾನೆ. ಪಾಂಡವರು ತಿಂದು ಮಿಕ್ಕ ಒಂದೆರಡು ಅಗಳು ಅನ್ನವನ್ನು ದ್ರೌಪದಿ ಕೊಡಲು ಅದನ್ನು ತಿಂದು ತೃಪ್ತನಾದ ಶ್ರೀಕೃಷ್ಣ. ವಿದುರ ಒಂದು ಸಣ್ಣ ಮಣ್ಣಿನ ಕುಡಿಕೆಯಲ್ಲಿ ಹಾಲನ್ನು ಕೊಡಲು ಶ್ರೀಕೃಷ್ಣ ಅದನ್ನು ಕುಡಿದು ತೃಪ್ತನಾಗಿ, ಆ ಕುಡಿಕೆಯನ್ನು ಬಗ್ಗಿಸಿದಾಗ ಕ್ಷೀರಸಾಗರವೇ ಅದರಿಂದ ಬಂದಂತೆ ಆಗಿ ಕೌರವನ ಅರಮನೆಯನ್ನು ಮುಳಗಿಸಲು ಹೋಯಿತಂತೆ! ಕುಚೇಲ ಶ್ರೀಕೃಷ್ಣನಿಗೆ ಒಣಗಿದ ಅವಲಕ್ಕಿಯನ್ನು ಕೊಡಲು ನಾಚಿದಾಗ, ಶ್ರೀಕೃಷ್ಣನೇ ಅದನ್ನು ಕುಚೇಲನಿಂದ ಕೇಳಿ ತೆಗೆದುಕೊಂಡು ತಿಂದು ಆನಂದ ಪಡುತ್ತಾನೆ. ಶಬರಿ ತಾನು ರುಚಿ ನೋಡಿ ಮಿಕ್ಕ ಹಣ್ಣನ್ನು ರಾಮನಿಗೆ ಇಟ್ಟಿದ್ದಳು. ಇದನ್ನೆಲ್ಲಾ ಭಗವಂತ ಸ್ವೀಕರಿಸಿದ. ನಾವು ಪ್ರೀತಿಯಿಂದ ಕೊಟ್ಟುದುದನ್ನೆಲ್ಲಾ ಅವನು ಸ್ವೀಕರಿಸಿ ಅದನ್ನು ಪವಿತ್ರ ಮಾಡಿ ಪ್ರಸಾದದಂತೆ ನಮಗೆ ಹಿಂದಕ್ಕೆ ಕೊಡುವನು. ಏಕೆಂದರೆ ಭಕ್ತರು ಅದನ್ನು ಭುಂಜಿಸಿ ಪವಿತ್ರರಾಗಲಿ ಎಂದು.

ಶ್ರೀಶಾರದಾದೇವಿ ಅವರು ಭಗವಂತನಿಗೆ ಅರ್ಪಣೆ ಮಾಡುತ್ತಿದ್ದರು. ಯಾರೊ ಭಕ್ತರು “ನೀವು ಕೊಡುವುದನ್ನು ಅವನು ತೆಗೆದುಕೊಳ್ಳುತ್ತಾನೆಯೆ” ಎಂದು ಕೇಳಿದಾಗ ಶ್ರೀ ಶಾರದಾದೇವಿ ಅವರು “ಹೌದು, ಅವನು ತೆಗೆದುಕೊಳ್ಳುತ್ತಾನೆ. ಮೂರು ವಿಧದಲ್ಲಿ ಅದನ್ನು ತೆಗೆದುಕೊಳ್ಳುತ್ತಾನೆ. ಇದನ್ನು ನಾನು ನೋಡಿದ್ದೇನೆ” ಎಂದರು. ಮೂರು ವಿಧದಗಳಾವುವು ಎಂಬುದನ್ನು ವಿವರಿಸುತ್ತಾರೆ. ಕೆಲವು ವೇಳೆ ಅವನು ನಾವು ಕೊಟ್ಟಿದ್ದರಲ್ಲಿ ಸ್ವಲ್ಪವನ್ನು ರುಚಿ ನೋಡುತ್ತಾನೆ. ಮತ್ತೆ ಕೆಲವು ವೇಳೆ, ಇದು ನನಗೆ ಬಂದಿತು ಎಂದು ಒಪ್ಪಿಗೆ ಕೊಡುತ್ತಾನೆ. ಮತ್ತೆ ಕೆಲವು ವೇಳೆ ಅವನ ಹಣೆಯಿಂದ ಒಂದು ಜ್ಯೋತಿ ಬಂದು ಅರ್ಪಿಸಿದುದನ್ನೆಲ್ಲಾ ಮುಟ್ಟಿ ಹೋಗುವುದು. ಹಿಂದೂಗಳು ತಾವು ಏನು ತಿನ್ನ ಬೇಕಾದರೂ ಅದನ್ನು ಮೊದಲು ದೇವರಿಗೆ ಅರ್ಪಿಸಿ ಅನಂತರ ತಿನ್ನುವರು. ದೇವರಿಗೆ ಅರ್ಪಣೆಮಾಡಿ ತಿಂದ ಮೇಲೆಯೇ ಅದು ನಿಜವಾದ ಭೋಜನ, ಇಲ್ಲದೇ ಇದ್ದರೆ ಅದು ಪೈಶಾಚಿಕ ಭೋಜನ ಎನ್ನುತ್ತಾನೆ ಮನು.

\begin{verse}
ಯತ್ಕರೋಷಿ ಯದಶ್ನಾಸಿ ಯಜ್ಜುಹೋಷಿ ದದಾಸಿ ಯತ್ ।\\ಯತ್ತಪಸ್ಯಸಿ ಕೌಂತೇಯ ತತ್ಕುರುಷ್ವ ಮದರ್ಪಣಮ್ \versenum{॥ ೨೭ ॥}
\end{verse}

{\small ಅರ್ಜುನ, ನೀನು ಯಾವ ಕೆಲಸವನ್ನು ಮಾಡುತ್ತೀಯೊ, ಏನನ್ನು ಊಟ ಮಾಡುತ್ತೀಯೊ, ಏನನ್ನು ಹೋಮ ಮಾಡುತ್ತೀಯೊ, ಏನನ್ನು ದಾನ ಮಾಡುತ್ತೀಯೊ, ಯಾವ ತಪಸ್ಸನ್ನು ಮಾಡುತ್ತೀಯೊ ಅದನ್ನೆಲ್ಲಾ ನನಗೆ ಅರ್ಪಿಸು.}

ನಾವು ಯಾವ ಕೆಲಸವನ್ನು ಮಾಡಿದರೂ, ದೇವರನ್ನೇ ನಮ್ಮ ಗುರಿಯಾಗಿ ಇಟ್ಟುಕೊಂಡಿರಬೇಕು. ಆಗ ನಾವು ಕೆಟ್ಟ ಕೆಲಸ ಮಾಡುವುದಕ್ಕೆ ಆಗುವುದಿಲ್ಲ. ಯಾವಾಗ ದೇವರನ್ನು ಗುರಿಯಾಗಿ ಇಟ್ಟುಕೊಳ್ಳುವೆವೊ ಆಗ ಅಧಿಕಾರಲಾಲಸೆ, ಹಣ, ಕೀರ್ತಿ ಇವುಗಳಿಗೆ ದಾಸರಾಗುವುದಿಲ್ಲ. ನಾವು ದೇವರನ್ನು ಮರೆತಾಗಲೇ ಈ ಹಳ್ಳಕ್ಕೆ ಬೀಳಬೇಕಾದರೆ. ಎಲ್ಲಿಯವರೆಗೆ ನಾವು ಅವನನ್ನು ಚೆನ್ನಾಗಿ ಜ್ಞಾಪಕದಲ್ಲಿ ಇಟ್ಟುಕೊಂಡಿರುವೆವೋ ಅಲ್ಲಿಯವರೆಗೆ ನಮ್ಮಿಂದ ಒಳ್ಳೆಯ ಕೆಲಸವೇ ಆಗುತ್ತ ಇರುವುದು.

ನಾವು ಯಾವ ಯಾಗ ಯಜ್ಞ ಮಾಡಿದರೂ ಭಗವಂತನ ಪ್ರೀತಿಗಾಗಿ ಅದರಿಂದ ಬರುವ ಪುಣ್ಯವನ್ನೆಲ್ಲ ಅರ್ಪಿಸಬೇಕು. ಕೇವಲ ಸ್ವರ್ಗಲೋಕಕ್ಕೆ ಹೋಗುವುದಕ್ಕೆ, ಅಥವಾ ಹಲವು ಲೌಕಿಕ ವಸ್ತುಗಳನ್ನು ಸಂಪಾದಿಸಿಕೊಳ್ಳುವುದಕ್ಕೆ ಆ ಪುಣ್ಯವನ್ನು ಖರ್ಚು ಮಾಡಬಾರದು. ಇದರಿಂದ ನಾವು, ಭಗವಂತನನ್ನು ಪಡೆಯಲಾರೆವು. ಭಗವಂತ ರಸ. ಅವನಲ್ಲಿರುವುದೆಲ್ಲ ರಸ. ನಾವು ಯಾವಾಗ ಭಗವಂತನನ್ನು ಮರೆಯುವೆವೊ ಆಗ ಕಸಕ್ಕೆ ಬೀಳುತ್ತೇವೆ. ಮರ್ತ್ಯ ಅಥವಾ ಸ್ವರ್ಗಲೋಕದ ಭೋಗಗಳೆಲ್ಲ ನಿಜವಾದ ಭಕ್ತನಿಗೆ ತಿಪ್ಪೆಗುಂಡಿಯಂತೆ ಕಾಣುವುದು. ಅವನು ಅದರ ಕಡೆ ಕಣ್ಣನ್ನೂ ಹಾಯಿಸುವುದಿಲ್ಲ. ಆದಕಾರಣವೇ ಯಾವುದಾದರೂ ದೊಡ್ಡ ಪುಣ್ಯ ಕೆಲಸವನ್ನೋ ಯಾಗ ಯಜ್ಞ ವನ್ನೋ ಮಾಡಿದರೆ “ಇತಿ ಶ್ರೀಕೃಷ್ಣಾರ್ಪಣಮಸ್ತು” “ಭಗವತ್ ಪ್ರೀತ್ಯರ್ಥಂ” ಎಂದು ಹೇಳುವುದು.

ನಾವು ಯಾವ ಆಹಾರವನ್ನು ಭುಂಜಿಸಬೇಕಾದರೂ ಮುಂಚೆ ಅದನ್ನು ದೇವರಿಗೆ ಅರ್ಪಿಸಬೇಕು. ದೇವರಿಗೆ ಅರ್ಪಿಸಬೇಕು ಎಂಬ ಭಾವ ನಮ್ಮಲ್ಲಿದ್ದರೆ ನಾವು ಅದನ್ನು ನ್ಯಾಯವಾಗಿ ಸಂಪಾದಿಸಿರ ಬೇಕು. ಅದನ್ನು ಶುಚಿಯಾಗಿ ಮಾಡಿರಬೇಕು. ಅದನ್ನು ಪ್ರೀತಿಯಿಂದ ಭಗವಂತನಿಗೆ ನೀಡಬೇಕು. ಆಗ ಅವನು ಅದನ್ನು ಸ್ವೀಕರಿಸುತ್ತಾನೆ. ಆಹಾರ ಭಗವಂತನ ಪ್ರಸಾದವಾದಾಗಲೇ ಶುದ್ಧವಾಗುವುದು. ಆ ಶುದ್ಧ ಆಹಾರವನ್ನು ತಿಂದರೆ ನಮ್ಮಲ್ಲಿ ಸತ್ತ್ವಗುಣ ವೃದ್ಧಿಯಾಗುವುದು. ಸತ್ವಗುಣ ವೃದ್ಧಿಯಾದರೆ ಏಕಾಗ್ರತೆ ಸಿದ್ಧಿಸುವುದು. ಭಗವಂತನನ್ನು ಪಡೆಯಬೇಕಾದರೆ ಏಕಾಗ್ರತೆ ಸಿದ್ಧಿಸಬೇಕು. ಒಂದು ಕಡೆಯಿಂದ ದೇಹವೆಂಬ ಯಂತ್ರಕ್ಕೆ ಆಹಾರ ಕೊಡುತ್ತೇವೆ. ಮತ್ತೊಂದು ಕಡೆಯಿಂದ ಅದು ಪವಿತ್ರ ಆಲೋಚನೆ ಎಂಬ ಶಕ್ತಿ ನೀಡುವುದು. ಆ ಶಕ್ತಿಯಿಂದ ನಾವು ಭಗವಂತನನ್ನು ಪಡೆಯಲು ಸಾಧ್ಯ.

ನಾವು ಮಾಡುವ ದಾನದ ಮುಂದೆಲ್ಲಾ ಭಗವಂತನನ್ನು ಇಟ್ಟುಕೊಂಡಿರಬೇಕು. ಅವನಿಗಾಗಿ ನಾವು ಮನುಷ್ಯನ ಮೂಲಕ ಕೊಡುತ್ತೇವೆ ಎಂಬ ಭಾವ ಇರಬೇಕು. ಆಗ ಬಂಧನಕ್ಕೆ ಬೀಳುವುದಿಲ್ಲ. ದೇವರೇ ಬಡವನಂತೆ, ರೋಗಿಯಂತೆ ಇರುವನು. ಇವರಿಗೆ ನಾವು ಕೊಡುವುದೆಲ್ಲಾ ಕೊನೆಗೆ ಭಗವಂತನಿಗೇ ಸಲ್ಲುವುದು. ನಮಗೆ ಕೊಡಲು ಅವನು ಹಲವು ಅವಕಾಶವನ್ನು ಒದಗಿಸುತ್ತಾನೆ. ಇಲ್ಲಿ ಸ್ವೀಕರಿಸಿ ದವನಲ್ಲ ಉದ್ಧಾರವಾಗುವುದು ಕೊಡುವವನು ಕೊಟ್ಟು ಉದ್ಧಾರವಾಗುತ್ತಾನೆ. ನಾನು ಕೊಡುವುದರ ಮೇಲೆ ಪ್ರಪಂಚ ನಿಂತಿಲ್ಲ. ಈ ಬ್ರಹ್ಮಾಂಡವನ್ನು ಸೃಷ್ಟಿಸಿದ ಪ್ರಭು, ಇಷ್ಟೊಂದು ಮಾಡಿದವನಿಗೆ ಇನ್ನು ಸ್ವಲ್ಪ ಮಾಡಲು ಕೈಲಾಗದೆ ನಮಗೆ ಮಾಡಿ ಎಂದು ಹೇಳಿಲ್ಲ. ನಮ್ಮ ಹೃದಯ ವಿಕಾಸವಾಗಲು ನಾವು ಕೊಡುವುದಕ್ಕೆ ಒಂದು ಅವಕಾಶವನ್ನು ಕಲ್ಪಿಸುವನು. ಈ ದೃಷ್ಟಿಯಿಂದ ಕೊಡಬೇಕು. ಆಗ ನಾವು ಕೊಟ್ಟಿದ್ದು ಸಾರ್ಥಕವಾಗುವುದು.

ಯಾವ ತಪಸ್ಸನ್ನು ಮಾಡಿದರೂ ಅದರ ಫಲವನ್ನೆಲ್ಲಾ ಭಗವಂತನಿಗೆ ಅರ್ಪಣೆ ಮಾಡುವುದನ್ನು ಕಲಿಯಬೇಕು. ನಾವು ಅವನಿಗೆ ಕೊಟ್ಟರೆ ನಮಗೆ ಏನೇನು ಆವಶ್ಯಕವಾಗಿರುವುದೊ ಅದನ್ನೆಲ್ಲಾ ಕೊಡುತ್ತಾನೆ. ನಮ್ಮಿಂದ ಕೇಳಿಸಿಕೊಂಡು ಕೊಡುವುದಿಲ್ಲ, ಕೇಳಿಸಿಕೊಳ್ಳದೆಯೇ ಕೊಡುತ್ತಾನೆ. ಯಾವಾಗ ದೇವರನ್ನು ಮರೆಯುತ್ತೇವೆಯೋ ಆಗ ತಪಸ್ಸಿನಿಂದ ನಾವು ನಾಶಮಾಡಿಕೊಳ್ಳುವ ವರವನ್ನು ಕೇಳಿಕೊಂಡು ನಾವೂ ನಾಶವಾಗಿ ಇತರರನ್ನೂ ನಾಶಮಾಡುತ್ತೇವೆ. ಪುರಾಣಗಳಲ್ಲಿ ಬರುವ ದಾನವರು ಕಠೋರ ತಪಸ್ವಿಗಳು. ರಾವಣ ಕುಂಭಕರ್ಣ ಮುಂತಾದ ಅಸುರರು ಕಡಮೆ ತಪಸ್ಸು ಮಾಡಿದವರಲ್ಲ. ದೇವರು ಅವರ ತಪಸ್ಸಿಗೆ ಒಲಿದು ಬಂದ. ಆಗ ಅವರು ಏನನ್ನು ಕೇಳಿದರು ದೇವರನ್ನು? “ನನಗೆ ಅದರಿಂದ ಮರಣ ಬೇಡ, ಇದರಿಂದ ಮರಣ ಬೇಡ” ಎಂದು ಬೇಡುತ್ತಾ ಹೋದರು. ಕೊನೆಗೆ ತಪಸ್ಸುಮಾಡಿ ಅವರು ದೇವರಿಂದ ಪಡೆದುಕೊಂಡ ವರವೇ ಲೋಕಕಂಟಕ ವಾಯಿತು. ಆದಕಾರಣವೇ ಇಂತಹ ಕ್ಷುದ್ರ ವಸ್ತುಗಳನ್ನು ಕೇಳದೆ, ಭಗವಂತನ ಪ್ರೀತಿಗೆ ಅದರಿಂದ ಬರುವ ಫಲವನ್ನೆಲ್ಲಾ ಅರ್ಪಿಸುವ. ಆಗ ನಾವು ಲೋಕಕಲ್ಯಾಣಕ್ಕೆ ಭಗವಂತನ ಕೈಯಲ್ಲಿ ನಿಮಿತ್ತವಾಗು ತ್ತೇವೆ.

\begin{verse}
ಶುಭಾಶುಭಫಲೈರೇವಂ ಮೋಕ್ಷ್ಯಸೇ ಕರ್ಮಬಂಧನೈಃ ।\\ಸಂನ್ಯಾಸಯೋಗಯುಕ್ತಾತ್ಮಾ ವಿಮುಕ್ತೋ ಮಾಮುಪೈಷ್ಯಸಿ \versenum{॥ ೨೮ ॥}
\end{verse}

{\small ಹೀಗೆ ಶುಭ, ಅಶುಭ ಫಲಗಳುಳ್ಳ ಕರ್ಮ ಬಂಧನದಿಂದ ಬಿಡುಗಡೆಯನ್ನು ಹೊಂದುತ್ತೀಯೆ. ಫಲತ್ಯಾಗರೂಪದ ಸಮತ್ವವನ್ನು ಪಡೆದು ಜನನ ಮರಣಗಳಿಂದ ಮುಕ್ತನಾಗಿ ನನ್ನನ್ನು ಸೇರುತ್ತೀಯೆ.}

ನಾವು ಯಾವ ಕೆಲಸವನ್ನು ಮಾಡಿದರೂ ಅದಕ್ಕೊಂದು ಪ್ರತಿಕ್ರಿಯೆ ಇದೆ. ಕೆಟ್ಟದ್ದರಿಂದ ಕೆಟ್ಟ ಪ್ರತಿಕ್ರಿಯೆ, ಒಳ್ಳೆಯದರಿಂದ ಒಳ್ಳೆಯ ಪ್ರತಿಕ್ರಿಯೆ ನಮ್ಮನ್ನು ಕಟ್ಟಿಹಾಕುವುದು. ಕೆಲಸವನ್ನು ಮಾಡಿಯೂ ಪ್ರತಿಕ್ರಿಯೆಗೆ ದಾಸರಾಗದ ರೀತಿಯಲ್ಲಿ ಕೆಲಸ ಮಾಡಬೇಕಾದರೆ, ದೇವರನ್ನು ಎದುರಿಗೆ ಇಟ್ಟುಕೊಂಡು ಅವನಿಗಾಗಿ ಮಾಡಬೇಕು. ಬರುವ ಪುಣ್ಯಾಪುಣ್ಯಗಳೆಲ್ಲ ನಿನಗೆ ಅರ್ಪಿತ ಎಂದು ಹೇಳಬೇಕು. ಆಗ ನಾವು ಮಾಡಿದ ಕರ್ಮದಿಂದ ಬಾಧಿತರಾಗುವುದಿಲ್ಲ. ಯಾವಾಗ ನಾವು ದೇವರನ್ನು ಮರೆತು ಫಲಾಪೇಕ್ಷೆಯಿಂದ ಕೆಲಸ ಮಾಡುತ್ತೇವೆಯೋ ಆಗ ಪರಿಣಾಮದ ಕೆಸರಿನಲ್ಲಿ ಸಿಕ್ಕಿಕೊಂಡು ನರಳಬೇಕಾಗುವುದು. ಅದಕ್ಕಾಗಿಯೇ ಶ್ರೀಕೃಷ್ಣ ಎಲ್ಲವನ್ನೂ ನನಗೆ ಅರ್ಪಿಸು ಎನ್ನುತ್ತಾನೆ.

ಇಂತಹ ಸಮತ್ವ ಯಾವಾಗ ನಮಗೆ ಪ್ರಾಪ್ತವಾಗುವುದೊ ಆಗಲೇ ನಾವು ಮುಕ್ತರಾಗುವುದು; ಜನನ ಮರಣಗಳಿಂದ ಪಾರಾಗುವುದು. ಎಲ್ಲಿಯವರೆಗೆ ಒಳ್ಳೆಯ ಹೆಸರಿನ ಹಿಂದೆ ಓಡಿಹೋಗಲು ಯತ್ನಿಸುವೆವೊ ಅಲ್ಲಿಯವರೆಗೆ ಅದು ಆಸೆ ತೋರಿಸುವುದು. ನಮ್ಮ ಕೈಗೆ ಸಿಕ್ಕಿದಂತೆ ಕಾಣುವುದು. ಆದರೆ ಕೈಗೆ ಸಿಕ್ಕುವುದಿಲ್ಲ. ಅದರಂತೆಯೇ ಯಾವಾಗ ಕೆಟ್ಟ ಹೆಸರಿನಿಂದ ಓಡಿ ಹೋಗಲು ಪ್ರಯತ್ನಿಸುತ್ತೇವೆಯೋ ಅದು ನಮಗಿಂತ ವೇಗವಾಗಿ ಓಡಿಬಂದು ನಮ್ಮನ್ನು ಹಿಡಿಯುವುದು. ಸಮತ್ವವನ್ನು ಪಡೆದಿರುವವನು ಯಾವುದನ್ನೂ ಆಶಿಸುವುದಿಲ್ಲ, ಯಾವುದಕ್ಕೂ ಅಂಜುವುದಿಲ್ಲ. ಅವನು ಬದುಕಿರುವಾಗಲೇ ಮುಕ್ತ. ಮುಕ್ತಿಗಾಗಿ ಸಾಯುವ ತನಕ ಕಾಯಬೇಕಾಗಿಲ್ಲ. ಸತ್ತ ಮೇಲೆ ಇಂತಹ ವ್ಯಕ್ತಿ ಪುನಃ ಸಂಸಾರಕ್ಕೆ ಬರುವುದಿಲ್ಲ. ಬರದ ರೀತಿಯಲ್ಲಿ ಹೋಗುವುದನ್ನು ಕಲಿತಿರುವನು ಇವನು.

\begin{verse}
ಸಮೋಽಹಂ ಸರ್ವಭೂತೇಷು ನ ಮೇ ದ್ವೇಷ್ಯೋಽಸ್ತಿ ನ ಪ್ರಿಯಃ ।\\ಯೇ ಭಜಂತಿ ತು ಮಾಂ ಭಕ್ತ್ಯಾ ಮಯಿ ತೇ ತೇಷು ಚಾಪ್ಯಹಮ್ \versenum{॥ ೨೯ ॥}
\end{verse}

{\small ನಾನು ಸರ್ವಭೂತಗಳಲ್ಲಿಯೂ ಸಮನಾಗಿರುತ್ತೇನೆ, ನನಗೆ ಶತ್ರುವೂ ಇಲ್ಲ ಮಿತ್ರನೂ ಇಲ್ಲ. ಆದರೆ ಭಕ್ತಿಪೂರ್ವಕ ಯಾರು ನನ್ನನ್ನು ಭಜಿಸುತ್ತಾರೆಯೋ ಅವರು ನನ್ನಲ್ಲಿ ಇರುವರು, ಮತ್ತು ನಾನು ಕೂಡ ಅವರಲ್ಲಿ ಇರುವೆನು.}

ಭಗವಂತ ಪಕ್ಷಪಾತವಿಲ್ಲದೆ ಎಲ್ಲದರಲ್ಲಿಯೂ ಒಂದೇ ಸಮನಾಗಿರುವನು. ಸರ್ವಾಂತರ್ಯಾಮಿ ಯಾಗಿರುವುದರಿಂದ ಎಲ್ಲಾ ಕಡೆಯಲ್ಲಿ ಇರಲೇಬೇಕಾಗಿದೆ. ಆದರೆ ಎಲ್ಲಾ ವಸ್ತುಗಳು ಒಂದೇ ಸಮನಾಗಿ ಅವನ ಕಾಂತಿಯನ್ನು ವ್ಯಕ್ತಗೊಳಿಸುತ್ತಿಲ್ಲ. ಸೂರ್ಯನ ಬೆಳಕು ನಿಷ್ಪಕ್ಷಪಾತವಾಗಿ ನೆಲ ನೀರು ಬಟ್ಟೆ ಇವುಗಳ ಮೇಲೆ ಬೀಳುವುದು. ಆದರೆ ಎಲ್ಲಾ ಒಂದೇ ಸಮನಾಗಿ ಅದರ ಕಾಂತಿಯನ್ನು ಪ್ರತಿಬಿಂಬಿಸುವುದಿಲ್ಲ. ನೆಲ ಒಂದು ಸ್ವಲ್ಪ ಅದನ್ನು ಪ್ರತಿಬಿಂಬಿಸುವುದು. ಬಟ್ಟೆ ಅದಕ್ಕಿಂತ ಹೆಚ್ಚು. ನೀರು ಅದಕ್ಕಿಂತ ಹೆಚ್ಚು. ಕನ್ನಡಿ ಶುಭ್ರವಾಗಿದ್ದರೆ ಎಲ್ಲದಕ್ಕಿಂತ ಹೆಚ್ಚಾಗಿ ಸೂರ್ಯನ ಕಾಂತಿಯನ್ನು ಪ್ರತಿಬಿಂಬಿಸುವುದು. ಆದರೆ ಇಲ್ಲಿ ಸೂರ್ಯನಿಗೆ ಕನ್ನಡಿಯ ಮೇಲೆ ಪಕ್ಷಪಾತವಿದೆ ಎಂದು ಹೇಳುವುದಕ್ಕೆ ಆಗುವುದಿಲ್ಲ. ಹಾಗೆಯೇ ಭಗವಂತ ಎಲ್ಲರ ಹೃದಯದಲ್ಲಿ ಇರುವನು. ಶ್ರೇಷ್ಠರಾದ ಜ್ಞಾನಿಗಳಲ್ಲಿ ಭಕ್ತರಲ್ಲಿ ಕರ್ಮಯೋಗಿಗಳಲ್ಲಿ ಅವನು ಕನ್ನಡಿಯಲ್ಲಿ ಬೆಳಕು ಪ್ರತಿಬಿಂಬಿಸುವಂತೆ ಪ್ರತಿಬಿಂಬಿಸು ತ್ತಾನೆ.

ಅವನಿಗೆ ಶತ್ರುವೂ ಇಲ್ಲ ಮಿತ್ರನೂ ಇಲ್ಲ. ಆದರೆ ಕೆಲವರು ಭಗವಂತನನ್ನು ತಮ್ಮ ಪರಮ ಮಿತ್ರನೆಂದು ಭಾವಿಸುತ್ತಾರೆ. ಅವನಿಗಾಗಿ ತಮ್ಮ ಬಾಳನ್ನು ಅರ್ಪಣೆ ಮಾಡುತ್ತಾರೆ. ಅವನನ್ನು ಪ್ರೀತಿಸುವುದು ಮತ್ತು ಸೇವಿಸುವುದೇ ಜೀವನದ ಅತ್ಯಂತ ಆನಂದವಾಗುವುದು. ಮತ್ತೆ ಕೆಲವರು ಇರುವರು. ಅವರು ದೇವರನ್ನು ದ್ವೇಷಿಸುವರು. ಅವನಿಗೆ ವಿರೋಧವಾಗಿ ಹೋಗುವರು. ಅಂತರ್ಯಾಮಿ ಯಾದ ಅವನು ಕಾಣಿಸದಂತೆ ಮಾಡುವುದಕ್ಕೆ ಏನುಬೇಕೊ ಅದನ್ನೆಲ್ಲಾ ಮಾಡುವರು. ಮೂರನೇ ಗುಂಪಿನ ಮನುಷ್ಯರು ಬೇರೆ ಇದ್ದಾರೆ. ಅವರ ಜೀವನಕ್ಕೆ ದೇವರು ಅನಾವಶ್ಯಕ. ಅವನಿದ್ದರೆ ಎಷ್ಟೋ ಹೋದರೂ ಅಷ್ಟೆ. ದೇವರನ್ನು ಈ ಮೂರು ಬಗೆ ಜನ ಒಬ್ಬೊಬ್ಬರು ಒಂದೊಂದು ದೃಷ್ಟಿಯಿಂದ ನೋಡಿದರೂ ದೇವರು ಮಾತ್ರ ಪಕ್ಷಪಾತವಿಲ್ಲದೇ ಒಂದೇ ಸಮನಾಗಿ ಎಲ್ಲರ ಮೇಲೂ ಕಂಗೊಳಿಸು ತ್ತಿರುವನು. ಶ್ರೀರಾಮಕೃಷ್ಣರು ದೀಪದ ಬೆಳಕಿನ ಉದಾಹರಣೆಯನ್ನು ಕೊಡುತ್ತಿದ್ದರು. ದೀಪ ಬೆಳಗುತ್ತಿದ್ದರೆ ಒಬ್ಬ ಎದ್ದು ಗೀತಾಪಾರಾಯಣ ಮಾಡುವನು, ಮತ್ತೊಬ್ಬ ಒಂದು ಸುಳ್ಳು ಅರ್ಜಿ ಬರೆಯುವನು. ಅದಕ್ಕೆ ದೀಪವನ್ನು ಹೊಗಳುವುದಕ್ಕೆ ಆಗುವುದಿಲ್ಲ, ತೆಗಳುವುದಕ್ಕೂ ಆಗುವುದಿಲ್ಲ. ಅವನು ಎಲ್ಲರಲ್ಲಿಯೂ ಒಂದೇ ಸಮನಾಗಿ ಬೆಳಗುತ್ತಿರುವನು.

ಆದರೆ ಯಾರು ಭಗವಂತನನ್ನು ಭಕ್ತಿ ಪೂರ್ವಕವಾಗಿ ಭಜಿಸುವರೊ ಅವರಲ್ಲಿ ಅವನು ಇತರ ಎಲ್ಲಾ ಕಡೆಗಿಂತ ಹೆಚ್ಚಾಗಿ ಕಾಣುವನು. ಭಕ್ತಿಯಿಂದ ಪೂಜಿಸುವವನು ಭಗವಂತನನ್ನು ತನ್ನ ಹೃದಯ ಮಂದಿರದಲ್ಲಿ ಪ್ರತಿಷ್ಠೆಮಾಡಿರುವನು, ಅವನಿಗಾಗಿ ತನ್ನ ಸರ್ವಸ್ವವನ್ನೂ ಅರ್ಪಣೆ ಮಾಡಿರುವನು. ಪ್ರೀತಿಸುವುದಕ್ಕೆ ಅವನು ಬದುಕಿರುವನು. ಪರಮ ಪ್ರೀತಿಯೇ ಅವನ ಧರ್ಮ. ಇಂತಹ ವ್ಯಕ್ತಿಯಲ್ಲಿ ದೇಹ ಮನಸ್ಸು ಬುದ್ಧಿ ಇಂದ್ರಿಯ ಅಹಂಕಾರಗಳೆಲ್ಲಾ ಚೆನ್ನಾಗಿ ಪರಿಶುದ್ಧವಾಗಿರುವುದರಿಂದ ಶುಭ್ರವಾದ ಗ್ಲಾಸಿನ ಚಿಮಣಿ ಮೂಲಕ ಕಾಂತಿ ವ್ಯಕ್ತವಾಗುವಂತೆ ಅವನು ವ್ಯಕ್ತವಾಗುತ್ತಿರುವನು. ಅವನಲ್ಲಿ ದೇವರು ಇರುವನು. ಅವನು ಭಗವಂತನನ್ನೇ ಇರಿಸುವುದಕ್ಕಾಗಿ ಎಲ್ಲವನ್ನೂ ಖಾಲಿಮಾಡಿ ಅವನಿಗೆ ಸ್ಥಳವನ್ನು ಅಣಿಮಾಡಿರುವನು. ಅವನ ಹೃದಯದಲ್ಲಿ ದೇವರು ಪಾತ್ರೆಯಲ್ಲಿ ನೀರು ಹೇಗಿದೆಯೋ ಹಾಗೆ ತುಂಬಿ ತುಳುಕಾಡುತ್ತಿರುವನು. ಅವನೊಳಗೆ ಮಾತ್ರ ದೇವರು ಇರುವುದಲ್ಲ, ದೇವರಲ್ಲಿ ಇವನು ತೇಲುತ್ತಿರುವನು. ಮೀನು ನೀರಿನಲ್ಲಿ ತೇಲುತ್ತಿರುವಂತೆ, ಹಕ್ಕಿ ಆಕಾಶದಲ್ಲಿ ಹಾರುತ್ತಿರುವಂತೆ, ಭಕ್ತ ಸದಾ ಅವನಲ್ಲೆ ಇರುವನು. ಭಕ್ತನ ಒಳಗೆ ಹೊರಗೆಲ್ಲ ಭಗವಂತ ಓತಪ್ರೋತನಾಗಿ ಪ್ರವೇಶಿಸಿರುವನು.

\begin{verse}
ಅಪಿ ಚೇತ್ಸುದುರಾಚಾರೋ ಭಜತೇ ಮಾಮನನ್ಯಭಾಕ್ ।\\ಸಾಧುರೇವ ಸ ಮಂತವ್ಯಃ ಸಮ್ಯಗ್ವ್ಯವಸಿತೋ ಹಿ ಸಃ \versenum{॥ ೩೦ ॥}
\end{verse}

{\small ಅತ್ಯಂತ ದುರಾಚಾರಿ ಕೂಡಾ ಅನನ್ಯ ಭಕ್ತಿಯಿಂದ ನನ್ನನ್ನು ಭಜನೆ ಮಾಡಿದರೆ ಅವನು ಸಾಧುವೆಂದೇ ತಿಳಿಯತಕ್ಕದ್ದು. ಏಕೆಂದರೆ ಈಗ ಅವನದು ಶುದ್ಧ ಸಂಕಲ್ಪವಾಗಿದೆ.}

ಭಗವಂತನ ಕಡೆಗೆ ಹೋಗುವುದಕ್ಕೆ ಯಾರೂ ಬಾಹಿರರಲ್ಲ. ಯಾರನ್ನು ನಾವು ಒಬ್ಬ ದುರಾಚಾರಿ ಎಂದು ನೋಡುತ್ತಿರುವೆವೋ, ಅವನಲ್ಲಿ ಹಲವು ಸತ್ ಸಂಸ್ಕಾರಗಳು ಸುಪ್ತಾವಸ್ಥೆಯಲ್ಲಿದ್ದು ಕ್ರಮೇಣ ಉತ್ತಮವಾಗಬಹುದು. ದುರ್ಜನ ಯಾವಾಗಲೂ ದುರ್ಜನನಾಗಿಯೇ ಇರುವುದಿಲ್ಲ, ಅವನು ಯಾವಾಗ ಸಜ್ಜನನಾಗುತ್ತಾನೆ ಎಂಬುದನ್ನು ಹೇಳುವುದಕ್ಕೆ ಆಗುವುದಿಲ್ಲ. ಆದಕಾರಣವೇ ನದೀ ಮೂಲ, ಪುಷಿ ಮೂಲ ತಿಳಿದುಕೊಳ್ಳುವುದಕ್ಕೆ ಯತ್ನಿಸಬೇಡ ಎನ್ನುತ್ತಾರೆ. ಒಂದು ದೃಷ್ಟಿ ಯಿಂದ ಅದರ ಮೂಲ ತಿಳಿದರೆ ಅದರ ಮೇಲೆ ಇರುವ ಗೌರವ ಕಡಮೆಯಾಗಬಹುದು ಎಂದು ಹಿಂದಿನವರು ಹಾಗೆ ಹೇಳಿರಬಹುದು. ಈಗ ಕವಿ ಮಹಾಪುಷಿ ಎಂದು ವಾಲ್ಮೀಕಿಯನ್ನು ಗೌರವಿಸು ತ್ತಿರಬಹುದು. ಆದರೆ ಆತ ಹಿಂದೆ ಕಾಡಿನಲ್ಲಿ ಸುಲಿಗೆ ಮಾಡುತ್ತಿದ್ದ ಡಕಾಯಿತ ಎಂದು ಗೊತ್ತಾದರೆ ಅವನನ್ನು ಸಾಧಾರಣ ಜನ ಗೌರವಿಸುವುದು ಕುಗ್ಗಬಹುದು. ಆದರೆ ಅದನ್ನೇ ಮತ್ತೊಂದು ದೃಷ್ಟಿಯಿಂದ ನೋಡಿದರೆ ಎಲ್ಲರಿಗೂ ಭರವಸೆ ಬರುವುದು. ಅಂತಹ ದುರಾಚಾರಿಯೇ ಈಗ ಇಂತಹ ಮಹಾತ್ಮನಾಗಿದ್ದರೆ, ನಾವೂ ಅವನಂತೆ ಆಗುವುದು ಸಾಧ್ಯ, ಎಂಬ ಧ್ಯೆರ್ಯವನ್ನು ಕೂಡಾ ತರುವುದು. ಈ ಜೀವನದಲ್ಲಿ ಯಾವಾಗಲೂ ಒಬ್ಬ ಈಗ ಇದ್ದ ಹಾಗೆಯೇ ಇರುವುದಕ್ಕೆ ಆಗುವುದಿಲ್ಲ. ವಿಕಾಸದ ಮಹಾ ಪ್ರವಾಹ ಪೂರ್ಣತೆಯ ಕಡೆಗೆ ಭರದಿಂದ ಸಾಗುತ್ತಿದೆ. ಆ ಪ್ರವಾಹದಲ್ಲಿ ಒಂದೊಂದು ಕಡೆ ಒಬ್ಬೊಬ್ಬರು ಇರುವರು. ಕೆಲವರು ಇನ್ನೇನು ಪೂರ್ಣತೆಯ ಸಾಗರವನ್ನು ಮುಟ್ಟುತ್ತಿರುವರು. ಇನ್ನು ಕೆಲವರು ಬಹಳ ಹಿಂದೆ ಇರುವರು. ಆದರೆ ಹಿಂದಿರುವುದು ಮುಂದೆ ಬರುವುದು, ಮಧ್ಯದಲ್ಲಿರುವುದು ತುತ್ತತುದಿಗೆ ಮುಟ್ಟುವುದು. ಅದರಂತೆಯೇ ಆಧ್ಯಾತ್ಮಿಕ ಜೀವನದಲ್ಲಿ ಹಿಂದೆ ಒಂದು ಕಾಲದಲ್ಲಿ ದುಷ್ಟರು, ದುರಾಚಾರಿಗಳು, ಸಮಾಜಕ್ಕೆ ಕಂಟಕಪ್ರಾಯರಾಗಿದ್ದವರು, ಅವರಲ್ಲಿ ಸುಪ್ತವಾಗಿರುವ ಯಾವುದೋ ಒಳ್ಳೆಯ ಗುಣ ಸಿಡಿದು ಮೇಲೆ ಬಂದು ಹಿಂದಿನದೆಲ್ಲ ಬದಲಾಗಿ ಅವರು ಸಂಪೂರ್ಣ ಹೊಸ ಮನುಷ್ಯರಾಗಿರುವುದನ್ನು ನೋಡುತ್ತೇವೆ. ಇತ್ತೀಚೆಗೆ ಶ್ರೀರಾಮಕೃಷ್ಣ ಪರಮಹಂಸರ ಶಿಷ್ಯರ ಜೀವನದಲ್ಲಿ ಅಂತಹ ಘಟನೆಗಳನ್ನು ನೋಡುತ್ತೇವೆ. ಗಿರೀಶಚಂದ್ರ ಘೋಷ್ ಎಂಬುವನು ಪರಮಹಂಸರ ಭಕ್ತನಾಗುತ್ತಾನೆ. ಆತ ಅವರ ಸಂಪರ್ಕಕ್ಕೆ, ಬರುವ ಮುಂಚೆ, ಸಮಾಜದ ಕಸದ ಬುಟ್ಟಿಯಲ್ಲಿ ಇದ್ದಂತಹ ವ್ಯಕ್ತಿಯಾಗಿದ್ದ. ಆತ ದೊಡ್ಡ ನಟನಾಗಿದ್ದ. ಆತನೆ ಒಂದು ಸಲ ಹೇಳುತ್ತಾನೆ “ನಾನು ಕುಡಿದ ಸರಾಯಿ ಬುಡ್ಡಿಗಳನ್ನು ಒಂದರ ಮೇಲೆ ಒಂದನ್ನು ಇಟ್ಟರೆ ಗೌರಿಶಂಕರ ಶಿಖರದಷ್ಟು ಎತ್ತರ ಹೋಗುವುದು. ನಾನು ಮಾಡದ ಪಾಪದ ಕೆಲಸವೇ ಇಲ್ಲ” ಎಂದು ಹೆಮ್ಮೆ ತಾಳುತ್ತಿದ್ದ. ಆ ಮನುಷ್ಯ ಪರಮಹಂಸರೆಂಬ ಸ್ಪರ್ಶಶಿಲೆಗೆ ತಾಕಿ ಸಂಪೂರ್ಣ ಬದಲಾಯಿಸಿ ಹೋದ. ದೇವರು ಅವನ ಬಳಿಗೆ ಬಂದ ಮೇಲೆ “ನೀನು ಹಿಂದೆ ಏನಾಗಿದ್ದೆ” ಎಂದು ಕೇಳುವುದಿಲ್ಲ. ಈಗ, ಇನ್ನು ಮೇಲೆ, ಹೇಗೆ ಇರುವೆ ಎಂಬುದನ್ನು ಮಾತ್ರ ನೋಡುತ್ತಾನೆ. ಎಲ್ಲರಿಗೂ ಒಂದು ಕೊಳಕು ಬಾಳಿನ ಭೂತಕಾಲವಿದೆ. ಆದರೆ ಕೆಲವರದು ಗೊತ್ತಾಗಿದೆ, ಮತ್ತೆ ಕೆಲವರದು ಗೊತ್ತಾಗಿಲ್ಲ. ಕೆಲವರು ಅದನ್ನು ಈಗ ಮಾಡಿದವರು, ಮತ್ತೆ ಕೆಲವರು ಬಹಳ ಹಿಂದೆ ಮಾಡಿದ್ದರು. ಅಂತೂ ನಾವೆಲ್ಲಾ ಅಪೂರ್ಣತೆಯಿಂದ ಬಂದಿದ್ದೇವೆ. ಆದರೆ ಇನ್ನು ಮೇಲೆ ಸತ್ಯ ಸಂಕಲ್ಪ ಮಾಡಿ ಭಗವಂತನನ್ನು ಅನನ್ಯ ಭಾವದಿಂದ ಭಜಿಸಿದರೆ ಅವನು ನಮ್ಮನ್ನು ಉದ್ಧಾರ ಮಾಡುತ್ತಾನೆ. ಭಗವಂತನೆಂಬುದು ಕಾಡ್ಗಿಚ್ಚು. ಅದು ನಮ್ಮ ಎಂತಹ ದುರಾಚಾರ ಎಂಬುವ ಪೊದೆಯನ್ನಾದರೂ ಭಸ್ಮೀಭೂತ ಮಾಡಬಲ್ಲದು. ಆದರೆ ಅವನ ಬಳಿಗೆ ಬರುವವನಲ್ಲಿ ಅನನ್ಯಭಕ್ತಿ ಇರಬೇಕು. ಆಗ ಅವನು ಮಾರ್ಪಾಡಾಗುತ್ತಾನೆ. ಧಗಧಗ ಉರಿಯುತ್ತಿರುವ ಬೆಂಕಿಗೆ ಯಾವುದೊ ಹಸಿ ಸೌದೆಯನ್ನು ಹಾಕಬಹುದು. ಸ್ವಲ್ಪ ಕಾಲದ ಮೇಲೆ ಅದೂ ಕೂಡ ಹತ್ತಿಕೊಂಡು ಉರಿಯುತ್ತದೆ. ಬೆಂಕಿಯ ಕಾವಿನಿಂದ ತನ್ನಲ್ಲಿರುವ ನೀರನ್ನು ಕಳೆದು ಕೊಂಡು ಇತರರಂತೆ ತಾನೂ ಧಗಧಗಿಸಿ ಉರಿಯ ತೊಡಗುವುದು. ಅದರಂತೆಯೇ ನಾವು ನಮ್ಮ ಕೊಳಕು ಬಾಳಿಗೆ ನಾಚಬೇಕಾಗಿಲ್ಲ. ಆದರೆ ಭಗವಂತನ ಸಮೀಪಕ್ಕೆ ಹೋದರೆ ಅವನು ನಮ್ಮ ಪಾಪದ ಕೆಸರನ್ನು ತೊಳೆದು ಪರಿಶುದ್ಧ ಮಾಡುವನು. ಅವನಿಗೆ ಜುಗುಪ್ಸೆಯಿಲ್ಲ, ಅವನಿಗೆ ಬೇಸರವಿಲ್ಲ, ಸದಾ ನಮ್ಮನ್ನು ಶುದ್ಧ ಮಾಡಲು ಅವನು ಕಾದಿರುವನು. ಆದರೆ ನಾವೇ ಅವನ ಬಳಿಗೆ ಹೋಗಲು ಅನುಮಾನಿಸುವವರು. ನಾವು ಯಾವಾಗ ಅನುಮಾನಿಸುವುದನ್ನು ಬಿಟ್ಟು ನಿರ್ಲಜ್ಜೆಯಿಂದ ಅವನಲ್ಲಿ ಶರಣಾಗುವೆವೋ ಆಗ ಅವನು ನಮ್ಮನ್ನು ಉದ್ಧಾರ ಮಾಡಿಯೇ ಮಾಡುತ್ತಾನೆ.

ಅಂತಹ ಮನುಷ್ಯನನ್ನು ಇನ್ನು ಮೇಲೆ ಸಾಧುವೆಂದೇ ತಿಳಿಯತಕ್ಕದ್ದು. ಅವನು ದೇವರ ಬಳಿಗೆ ಬಂದಿದ್ದಾನೆ. ಇನ್ನು ಮೇಲೆ ಅವನ ಹಿಂದಿನ ಸ್ವಭಾವ ಹಿಂದಿನಂತೆ ಇರುವುದಕ್ಕೆ ಆಗುವುದಿಲ್ಲ. ನಾಗರಹಾವು ಕಪ್ಪೆಯನ್ನು ಕಚ್ಚಿದರೆ, ಒಂದೆರಡು ಸಲ ಒಟಗುಟ್ಟುವುದರೊಳಗೆ ಸಾಯುವುದು. ಹಾಗೆಯೇ ನಾವು ದೇವರಿಗೆ ಅರ್ಪಣೆ ಮಾಡಿಕೊಂಡರೆ ನಮ್ಮ ಹಿಂದಿನದನ್ನೆಲ್ಲಾ ಬೇಗ ಬಿಡಿಸುವನು. ಅವನು ಸತ್ಯ ಸಂಕಲ್ಪ ಮಾಡಿದ್ದಾನೆ ಇನ್ನು ಮೇಲೆ ಹಿಂದಿನ ದಾರಿ ಹಿಡಿಯುವುದಿಲ್ಲ ಎಂದು ನಿಶ್ಚಯ ಮಾಡಿದ್ದಾನೆ. ಇನ್ನು ಮೇಲೆ ಅವನದು ಶುಭದ ಹಾದಿ, ಭರವಸೆಯ ಹಾದಿ. ದೇವರು ಅವನನ್ನು ಕೈಹಿಡಿದು ನಡೆಸುವನು, ಗುರಿ ಮುಟ್ಟಿಸುವನು.

\begin{verse}
ಕ್ಷಿಪ್ರಂ ಭವತಿ ಧರ್ಮತ್ಮಾ ಶಶ್ವಚ್ಛಾಂತಿಂ ನಿಗಚ್ಛತಿ ।\\ಕೌಂತೇಯ ಪ್ರತಿ ಜಾನೀಹಿ ನ ಮೇ ಭಕ್ತಃ ಪ್ರಣಶ್ಯತಿ \versenum{॥ ೩೧ ॥}
\end{verse}

{\small ಅರ್ಜುನ, ಅವನು ಬೇಗ ಧರ್ಮಾತ್ಮನಾಗುತ್ತಾನೆ. ಮತ್ತು ಶಾಶ್ವತವಾದ ಶಾಂತಿಯನ್ನು ಪಡೆಯುತ್ತಾನೆ. ನನ್ನ ಭಕ್ತ ಎಂದಿಗೂ ನಾಶವಾಗುವುದಿಲ್ಲ ಎಂಬುದನ್ನು ತಿಳಿದುಕೊ.}

ಯಾವನು ಭಗವಂತನ ಬಳಿಗೆ ಬಂದು ಅವನಲ್ಲಿ ಶರಣಾಗುತ್ತಾನೊ ಅವನು ಬೇಗ ಧರ್ಮಾತ್ಮ ನಾಗುತ್ತಾನೆ. ಅವನಲ್ಲಿ ಅಂತಹ ತೀವ್ರ ಅಭೀಪ್ಸೆ ಇದೆ. ಹೇಗೆ ಒಲೆಯ ಮೇಲಿಟ್ಟ ಹಸಿ ಸೌದೆ ಯಲ್ಲಿರುವ ನೀರೆಲ್ಲ ಒಣಗಿ ಹೋಗುವುದೊ ಹಾಗೆ ಭಗವಂತನ ಬಳಿಗೆ ಬಂದವನ ಪಾಪದ ಹಸಿಯೆಲ್ಲ ಒಣಗಿ ಹೋಗುವುದು. ಅವನು ಶಾಶ್ವತವಾದ ಶಾಂತಿಯನ್ನು ಪಡೆಯುತ್ತಾನೆ. ಅವನಿಗೆ ದೇವರ ಬಳಿಗೆ ಬರುವುದಕ್ಕೆ ಮುಂಚೆ ಸಿಕ್ಕುತ್ತಿದ್ದುದು ಇಂದ್ರಿಯದ ಮೂಲಕ ಬರುತ್ತಿದ್ದ ಸುಖ. ಪ್ರತಿಯೊಂದು ಸುಖದ ಮುಂದೆಯೂ ಅಸುಖ ಕಾದಿರುತ್ತಿತ್ತು. ಅವನಿಗೆ ಕಾದಿರುತ್ತಿದ್ದುದು ಅಶಾಂತಿ. ಬಿರುಗಾಳಿಯೊಂದು ಏಳುವುದಕ್ಕೆ ಮುಂಚೆ ಇರುವ ಪ್ರಶಾಂತಿಯಂತೆ ಅದು. ಯಾವಾಗ ಒಬ್ಬ, ದೇವರ ಪಾದಪದ್ಮಗಳನ್ನು ಮುಟ್ಟುತ್ತಾನೆಯೋ ಆಗ ಮಾತ್ರ ನಿಜವಾದ ಶಾಂತಿ ಏನು ಎಂಬುದು ಅರ್ಥವಾಗುವುದು. ಇನ್ನು ಮೇಲೆ ಪ್ರಪಂಚದಲ್ಲಿ ಏನಾದರೂ ಅವನ ಶಾಂತಿಗೆ ಭಂಗ ಬಾರದು. ಏಕೆಂದರೆ ಅವನ ಆಧಾರ ಈ ಪ್ರಪಂಚವಲ್ಲ, ಭಗವಂತ. ಅವನಿಂದ ಬರುವ ಶಾಂತಿಯೊಂದೇ ಶಾಶ್ವತ.

‘ನನ್ನ ಭಕ್ತ ನಾಶವಾಗುವುದಿಲ್ಲವೆಂದು ತಿಳಿದುಕೊ’ ಎಂದು ಶ್ರೀಕೃಷ್ಣ ಅರ್ಜುನನಿಗೆ ಭರವಸೆ ಯನ್ನು ಕೊಡುತ್ತಾನೆ. ಈ ಭರವಸೆ ಕೇವಲ ಅರ್ಜುನನಿಗೆ ಮಾತ್ರ ಕೊಟ್ಟುದುದಲ್ಲ. ಪ್ರತಿಯೊಬ್ಬ ಮಾನವ ಕೋಟಿಗೆ ಅರ್ಜುನನ ಮೂಲಕ ಕೊಟ್ಟ ಭರವಸೆ. ಒಬ್ಬ ದೇವರನ್ನು ನೆಚ್ಚಿ ಹೊರಟರೆ ಅವನು ಹಾಳಾಗಿ ಹೋಗುವಹಾಗಿದ್ದರೆ ಈ ಪ್ರಪಂಚದಲ್ಲಿ ಇನ್ನೆಲ್ಲಿ ತಂಗುವುದಕ್ಕೆ ಸ್ಥಳವಿದೆ! ಇಂದ್ರಿಯ ಪ್ರಪಂಚದ ಸುಖವನ್ನು ಬಿಟ್ಟು ದೇವರನ್ನು ನೆಚ್ಚಿ ಹೋದರೆ, ಕೆಲವು ವೇಳೆ ಮನಸ್ಸಿನಲ್ಲಿ ಅನುಮಾನ ಅಂಜಿಕೆಗಳು ಮೂಡಬಹುದು. ಯಾರಿಗೆ ಗೊತ್ತು, ಅವನು ಬರುತ್ತಾನೆಯೋ ಇಲ್ಲವೊ; ನಮ್ಮನ್ನು ಉದ್ಧಾರ ಮಾಡುತ್ತಾನೆಯೊ ಇಲ್ಲವೊ ಎಂಬ ಕಳವಳವೇಳುವುದು ಮನಸ್ಸಿನಲ್ಲಿ. ‘ನಡುನೀರಲಿ ಕೈಯ ಬಿಡುವರೇನೋ ರಂಗ!’ ಎಂದು ಹೇಳುವ ಪ್ರಸಂಗವೇಳುವುದು. ಆದರೆ ಇದು ಕೇವಲ ನಮ್ಮನ್ನು ಪರೀಕ್ಷಿಸುವುದಕ್ಕಾಗಿ. ನಾವು ಆಗಲೂ ಕಲ್ಲು ಮನಸ್ಸು ಮಾಡಿ ಅವನು ‘ನನ್ನನ್ನು ಉದ್ಧರಿಸಲಿ ಬಿಡಲಿ ನಾನಂತೂ ಬಿಟ್ಟುಹೋಗುವುದಿಲ್ಲ’ ಎಂದು ಮೊಂಡು ಹಿಡಿದರೆ ಅವನ ಕೃಪಾಹಸ್ತ ನಮ್ಮನ್ನು ಮೇಲೆತ್ತಲು ಎಲ್ಲಿಂದಲೋ ಬರುವುದು. ಆಗ ಗೊತ್ತಾಗುವುದು ದೇವರನ್ನು ನಂಬಿ ಕೆಟ್ಟವರಿಲ್ಲ ಎಂಬುದು.

ಇದನ್ನು ‘ಪ್ರತಿ ಜಾನೀಹಿ’ ತಿಳಿದುಕೊ ಎಂದು ಒತ್ತಿ ಹೇಳುತ್ತಾನೆ. ಇದೊಂದು ಗಾಢವಾದ, ಆಧ್ಯಾತ್ಮಿಕ ನಿಯಮ. ಅವನನ್ನು ನೆಚ್ಚಿದರೆ, ನಂಬಿದರೆ, ಅವನು ನಮ್ಮ ಕೈ ಬಿಡುವುದಿಲ್ಲ. ಈ ಪ್ರಪಂಚದಲ್ಲಿ ನಾವೆಷ್ಟು ದೇವರಲ್ಲದ ವಸ್ತುಗಳನ್ನು ನೆಚ್ಚಿದ್ದೇವೆ, ನಂಬಿದ್ದೇವೆ. ಅದರಿಂದ ಪಡ ಬಾರದ ಯಾತನೆಯನ್ನು ಪಟ್ಟಿದ್ದೇವೆ. ಬೇಕಾದಷ್ಟು ಅನುಭವಿಸಿದ್ದೇವೆ. ಆದರೂ ಅವುಗಳನ್ನು ಬಿಡುವುದಿಲ್ಲ. ಒಂದು ಹೋದರೆ ಮತ್ತೊಂದನ್ನು ಹಿಡಿಯುತ್ತೇವೆ. ನಾವು ಪ್ರಪಂಚ ವಸ್ತುಗಳನ್ನು ಎಷ್ಟು ನೆಚ್ಚುತ್ತೇವೋ ಅಷ್ಟು ದೇವರನ್ನು ನೆಚ್ಚಿದರೆ ಸಾಕು. ಅವನು ನಮ್ಮನ್ನು ಉದ್ಧರಿಸುತ್ತಾನೆ. ಏನಿಲ್ಲ, ಈ ಲೋಕದಲ್ಲಿಯೇ ಯಾರಾದರೂ ನಮ್ಮನ್ನು ನಂಬಿದರೆ ಅವರಿಗೆ ಮೋಸಮಾಡುವಂತಹ ಜನ ಬಹಳ ಕಡಮೆ. ಅವರು ನರಾಧಮರ ಗುಂಪಿಗೆ ಸೇರಿದವರು. ಉಳಿದವರು ಸಾಧ್ಯವಾದಷ್ಟಾದರೂ ನಂಬಿದವರಿಗೆ ಒಳ್ಳೆಯದನ್ನು ಮಾಡಲೆತ್ನಿಸುವರು. ದೇವರು ಇವರೆಲ್ಲರಿಗಿಂತ ಮೇಲಲ್ಲವೆ! ಅವನ ಹೆಸರಿನಲ್ಲಿ ನಾವು ಎಲ್ಲಾ ಬಿಟ್ಟು ಬಂದರೆ ಅವನು ಎಂದಿಗೂ ನಮ್ಮನ್ನು ಕೈಬಿಡುವುದಿಲ್ಲ. ಅವನನ್ನು ನೆಚ್ಚುವ ಧೈರ್ಯ ಇರಬೇಕು. ಆಗ ನಮ್ಮನ್ನು ಉದ್ಧರಿಸುವ ಶಕ್ತಿ ಕಾದಿದೆ ಎಂಬುದು ಗೊತ್ತಾಗುವುದು. ಇದೆಲ್ಲಾ ದ್ವಾಪರ ಯುಗದಲ್ಲಿ, ಅರ್ಜುನನಿಗೆ ಮಾತ್ರ ಸತ್ಯವಲ್ಲ. ಈ ಭರವಸೆ ಎಲ್ಲಾ ಕಾಲಕ್ಕೆ ಎಲ್ಲಾ ಯುಗಕ್ಕೆ ಎಲ್ಲಾ ಜೀವರಾಶಿಗಳಿಗೂ ದೇವರು ಕೊಟ್ಟ ಭರವಸೆ.

\begin{verse}
ಮಾಂ ಹಿ ಪಾರ್ಥ ವ್ಯಪಾಶ್ರಿತ್ಯ ಯೇಽಪಿ ಸ್ಯುಃ ಪಾಪಯೋನಯಃ ।\\ಸ್ತ್ರಿಯೋ ವೈಶ್ಯಾಸ್ತಥಾ ಶೂದ್ರಾಸ್ತೇಽಪಿ ಯಾಂತಿ ಪರಾಂ ಗತಿಮ್ \versenum{॥ ೩೨ ॥}
\end{verse}

{\small ಪಾರ್ಥ, ಪಾಪಯೋನಿಗಳು, ಹೆಂಗಸರು, ವೈಶ್ಯರು, ಶೂದ್ರರು ನನಗೆ ಶರಣಾದರೆ ಪರಮಗತಿಯನ್ನು ಹೊಂದು ತ್ತಾರೆ.}

ಇಲ್ಲಿ ಭಗವಂತ ಉದ್ಧಾರವಾಗುವುದಕ್ಕೆ ಎಲ್ಲರಿಗೂ ಅವಕಾಶ ಕಲ್ಪಿಸಿದ್ದಾನೆ. ದೇವರ ಬಳಿಗೆ ಹೋಗಲು ಉತ್ತಮ ಕುಲದವರಿಗೆ ಮತ್ತು ಉತ್ತಮ ವರ್ಣದವರಿಗೆ ಮಾತ್ರ ಅಲ್ಲ ಅಧಿಕಾರ ಇರುವುದು. ಎಲ್ಲರೂ ಭಗವಂತನ ಮಕ್ಕಳೇ. ಅಜ್ಞಾನದಲ್ಲಿರುವಾಗ ಮಣ್ಣಿನಲ್ಲಿ ಆಡಿ ಮನಸ್ಸನ್ನು ಕೊಳೆ ಮಾಡಿಕೊಂಡಿದ್ದೇವೆ. ಆದರೆ ಭಗವಂತನೆಡೆಗೆ ನಾವು ಹೋದರೆ ಅವನು ನಮ್ಮನ್ನು ಗೊಬ್ಬರದ ಗುಂಡಿಗೆ ಎಸೆಯುವುದಿಲ್ಲ. ಒಬ್ಬ ತಾಯಿ ತನ್ನ ಮಗುವನ್ನು ಹಾಗೆ ಮಾಡಲಾರಳು. ಕೊಳೆ ಮಾಡಿಕೊಂಡು ಬಂದರೆ, ಮೈ ತೊಳೆಯುತ್ತಾಳೆ. ಈ ಪ್ರೇಮವೆಲ್ಲ ಬಂದಿರುವುದು ಜಗನ್ಮಯಿ ಯಿಂದ. ಇನ್ನು ಅವಳ ಪ್ರೇಮ ಎಷ್ಟು ಅಪಾರವಾಗಿರಬೇಕು!

ಪಾಪಯೋನಿಜರಾದರೂ ಅವರ ಕೈ ಬಿಡುವುದಿಲ್ಲ. ಪುಣ್ಯಾತ್ಮರೆಲ್ಲರೂ ಸತ್ಕುಲದಲ್ಲಿಯೇ ಹುಟ್ಟುವವರಲ್ಲ. ಅನೇಕ ಮಾನವ ಅನರ್ಘ್ಯ ರತ್ನಗಳು ಹೀನ ತಂದೆ ತಾಯಿಗಳಿಗೆ ಹುಟ್ಟುತ್ತಾರೆ. ತಾಯಿ ತಂದೆ ಹೇಗಿದ್ದರೇನಂತೆ? ಈಗ ಭಗವಂತನೆಡೆಗೆ ಬಂದಿರುವವನ ಯೋಗ್ಯತೆ ಏನು ಎಂಬುದನ್ನು ಮಾತ್ರ ನೋಡುತ್ತಾನೆ ದೇವರು. ಕೆಸರಿನಲ್ಲಿ ಕಮಲ ಹುಟ್ಟುವುದಿಲ್ಲವೆ, ಆದರೆ ಅದನ್ನು ದೇವರಿಗೆ ಅರ್ಪಿಸುವುದಿಲ್ಲವೇ? ದೇವರು ಭಕ್ತನ ಗುಣವನ್ನು ನೋಡುತ್ತಾನೆಯೇ ಹೊರತು ಅವನ ಅಪ್ಪ ಯಾರು, ಅಮ್ಮ ಯಾರು ಎಂದು ಕೇಳುವುದಿಲ್ಲ. ಈ ಜಗದ ಸಂತೆಯಲ್ಲಿ ಮಾತ್ರ ಅದಕ್ಕೆ ಬೆಲೆ. ದೇವರು ಅವನೆಡೆಗೆ ಬರುವ ಭಕ್ತನನ್ನು ಮಾತ್ರ ಪರೀಕ್ಷಿಸುತ್ತಾನೆ. ಅವನಲ್ಲಿ ಒಳ್ಳೆಯ ಗುಣಗಳಿದ್ದರೆ ಅವನನ್ನು ಸ್ವೀಕರಿಸುತ್ತಾನೆ.

ಸ್ತ್ರೀಯರಿಗೂ ಕೂಡ ದೇವರೆಡೆಗೆ ಹೋಗುವುದಕ್ಕೆ ಪುರುಷರಿಗಿರುವಷ್ಟೇ ಅಧಿಕಾರವಿದೆ ಎಂದು ಸಾರುತ್ತಾನೆ. ಹಿಂದಿನ ಕಾಲದಲ್ಲಿ ಪುರುಷನಿಗೆ ಯಾವ ಸೌಲಭ್ಯಗಳಿತ್ತೊ ಅವು ಸ್ತ್ರೀಯರಿಗಿರಲಿಲ್ಲ. ಆದರೂ ಆಗಿನ ಕಾಲದಲ್ಲೆ ಶ್ರೀಕೃಷ್ಣ ಸ್ತ್ರೀಯರನ್ನು ನಿಕೃಷ್ಟ ದೃಷ್ಟಿಯಿಂದ ನೋಡದೆ ಭಗವಂತನೆಡೆಗೆ ಹೋಗುವುದಕ್ಕೆ ಅವರಿಗೂ ಸಮನಾದ ಅಧಿಕಾರವಿದೆ ಎನ್ನುತ್ತಾನೆ. ಇಪ್ಪತ್ತನೆಯ ಶತಮಾನದಲ್ಲಿರುವ ನಮಗೆ ಎಲ್ಲರಿಗೂ ಸಮಾನ ಅವಕಾಶಗಳಿವೆ. ಈಗಿನ ದೃಷ್ಟಿಯಿಂದ ನೋಡಿದರೆ ಇದರಲ್ಲೇನೂ ಅಂತಹ ವಿಶೇಷ ಕಾಣುವುದಿಲ್ಲ. ಆದರೆ ಹಿಂದಿನ ಕಾಲದಲ್ಲಿ ಸ್ತ್ರೀ ಪರತಂತ್ರಳು ಎಂದು ಭಾವಿಸಿದ ಸಮಾಜದ ಎದುರಿಗೆ ಶ್ರೀಕೃಷ್ಣ ಈ ಮಾತನ್ನು ಹೇಳುತ್ತಾನೆ.

ಇನ್ನು ವೈಶ್ಯರು ಮತ್ತು ಶೂದ್ರರು ಇವರನ್ನು ಬ್ರಾಹ್ಮಣವರ್ಗದವರೊಂದಿಗೆ ಹೋಲಿಸಿ ನೋಡಿ ದರೆ ಕೆಳಗೆ ಇರುವವರು. ದೇವರು, ಸಮಾಜದ ದೃಷ್ಟಿಯಲ್ಲಿ ಯಾರು ಮೇಲಿನವರು ಯಾರು ಕೆಳಗಿನವರು ಎಂದು ನೋಡುವುದಿಲ್ಲ. ಸಮಾಜ ಒಂದು ವ್ಯಕ್ತಿಯನ್ನು ವರ್ಗೀಕರಿಸುವುದೇ ಬೇರೆ. ದೇವರು ತನ್ನ ಭಕ್ತನನ್ನು ವರ್ಗೀಕರಿಸುವ ದೃಷ್ಟಿಯೇ ಬೇರೆ. ದೇವರು ಹೊರಗಿನದನ್ನು ನೋಡುವು ದಿಲ್ಲ. ವ್ಯಕ್ತಿಯ ಹೃದಯದಲ್ಲಿರುವ ಪವಿತ್ರ ಉದ್ದೇಶಗಳನ್ನು ನೋಡುತ್ತಾನೆ, ಸತತ ಪ್ರಯತ್ನವನ್ನು ನೋಡುತ್ತಾನೆ. ಅದಕ್ಕೆ ತಕ್ಕಂತೆ ಅವರನ್ನು ಆರಿಸಿಕೊಳ್ಳುತ್ತಾನೆ. ಸಮಾಜಬಾಹಿರರಿಗೂ ಅತಿ ನೀಚ ರಿಗೂ ಭಗವಂತನೆಡೆಗೆ ಹೋಗುವುದಕ್ಕೆ ಅಧಿಕಾರವಿರುವಾಗ ಸ್ತ್ರೀ, ವೈಶ್ಯ, ಶೂದ್ರರಿಗೆ ಆ ಅವಕಾಶವಿರು ವುದರಲ್ಲಿ ಆಶ್ಚರ್ಯವೇನಿದೆ! ಇಲ್ಲಿ ಶ್ರೀಕೃಷ್ಣ ಬರೀ ಹಿಂದೂಗಳ ದೃಷ್ಟಿಯಿಂದ ಮಾತನಾಡುವುದಿಲ್ಲ. ದೇವರೆಡೆಗೆ ಹೋಗುವುದಕ್ಕೆ, ಪರಮಗತಿಯನ್ನು ಪಡೆಯುವುದಕ್ಕೆ, ಎಲ್ಲರಿಗೂ ಬಾಗಿಲನ್ನು ತೆರೆಯು ವನು. ಗೀತೆಯಲ್ಲಿ ಬರುವ ಧರ್ಮ ಅತ್ಯಂತ ವಿಶಾಲವಾದ ಧರ್ಮ, ಎಲ್ಲ ಧರ್ಮಗಳನ್ನು ಬಾಚಿ ತಬ್ಬಿಕೊಳ್ಳುವಂತಹ ಧರ್ಮ. ಇಲ್ಲಿ ಬರುವ ದೇವರೂ ಅಷ್ಟೆ ಅನಂತವಾಗಿರುವನು. ಯಾವ ಹೆಸರಿನಿಂದ ಕರೆದರೂ ಅವನಿಗೇ ಹೋಗುವುದು. ಯಾರುಯಾರು ಅವನನ್ನು ಯಾವಯಾವ ಧರ್ಮದ ಆಶ್ರಯ ಪಡೆದಿದ್ದರೂ ಎಲ್ಲರಿಗೂ ಪರಮಗತಿಯನ್ನು ಎಂದರೆ ಮುಕ್ತಿಯನ್ನು ಕೊಡುತ್ತಾನೆ.

\begin{verse}
ಕಿಂ ಪುನರ್ಬ್ರಾಹ್ಮಣಾಃ ಪುಣ್ಯಾ ಭಕ್ತಾ ರಾಜರ್ಷಯಸ್ತಥಾ ।\\ಅನಿತ್ಯಮಸುಖಂ ಲೋಕಮಿಮಂ ಪ್ರಾಪ್ಯ ಭಜಸ್ವ ಮಾಮ್ \versenum{॥ ೩೩ ॥}
\end{verse}

{\small ಹೀಗಿರುವಾಗ ನನ್ನ ಭಕ್ತರಾದ ಪುಣ್ಯಶಾಲಿಗಳಾದ ಬ್ರಾಹ್ಮಣರ ರಾಜ ಪುಷಿಗಳ ವಿಚಾರ ಹೇಳುವುದೇನು! ಆದಕಾರಣ ಅನಿತ್ಯವೂ ಅಸುಖವೂ ಆದ ಈ ಲೋಕವನ್ನು ಪಡೆದಿರುವ ನೀನು ನನ್ನನ್ನು ಭಜಿಸು.}

ತುಂಬಾ ಕೆಳಮಟ್ಟದಲ್ಲಿರುವವರಿಗೇ ಭಗವಂತನೆಡೆಗೆ ಹೋಗುವುದಕ್ಕೆ ಸಾಧ್ಯವಿರುವಾಗ, ಯಾರು ಉತ್ತಮ ವಾತಾವರಣದಲ್ಲಿರುವರೊ ಅವರಿಗೆ ಮತ್ತೂ ಸುಲಭ ಸಾಧ್ಯವಾಗುವುದು ಅವನ ಕಡೆಗೆ ಹೋಗುವುದಕ್ಕೆ. ಈ ಸಂಸಾರ ಅನಿತ್ಯ. ಇಂದು ಇರುವುದು ನಾಳೆ ಇಲ್ಲ. ಧನ, ಯೌವನ, ಅಧಿಕಾರ ಎಲ್ಲ ನಮ್ಮನ್ನು ಬಿಟ್ಟುಹೋಗುವುದು. ಈ ಜೀವನದಲ್ಲಿ ಬರುವಾಗ ಹೇಗೆ ಬಂದೆವೋ ಹಾಗೆಯೇ ಹೋಗುವಾಗ. ಏನನ್ನು ಮಧ್ಯದಲ್ಲಿ ಶೇಖರಿಸಿ ಇಟ್ಟುಕೊಂಡೆವೊ ಅವುಗಳಾವುವೂ ಸಾಯುವಾಗ ನಮ್ಮ ಹಿಂದೆ ಬರುವುದಿಲ್ಲ. ಇರುವಾಗಲಾದರೂ ನಾವು ಅವನ್ನು ಅನುಭವಿಸುತ್ತಿರಬಹುದಲ್ಲ ಎಂದು ಭಾವಿಸಬಹುದು. ಆದರೆ ನಾವೇನೋ ಅವುಗಳನ್ನು ಅನುಭವಿಸುವುದಕ್ಕೆ ಸಿದ್ಧರಾಗಿರಬಹುದು. ಅವು ನಮ್ಮೊಡನೆ ಇರಲು ಸಿದ್ಧವಾಗಿಲ್ಲ. ಅದು ಹೇಗೋ ಬರುವುದು, ಹೇಗೊ ಹೋಗುವುದು. ನಮ್ಮನ್ನು ಹೇಳಿ ಕೇಳಿ ಬರುವುದಿಲ್ಲ. ಹೋಗುವಾಗ ನಾವು ಎಷ್ಟೇ ಕಪಿಮುಷ್ಠಿಯಿಂದ ಅದನ್ನು ಹಿಡಿದುಕೊಂಡಿ ದ್ದರೂ ಅದು ಬೆರಳುಗಳಿಂದ ನುಸುಳಿ ಹೋಗುವುದು. ಈ ಪ್ರಪಂಚದಲ್ಲಿ ಎಲ್ಲಾ ನೀರಗುಳ್ಳೆಯಂತೆ, ಕ್ಷಣಿಕ. ಒಂದು ನೀರ ಹನಿಯಿಂದ ನೀರುಗುಳ್ಳೆ ಏಳುವುದು, ಮತ್ತೊಂದು ನೀರ ಹನಿಯಿಂದ ನೀರಗುಳ್ಳೆ ಸಿಡಿಯುವುದು.

ಈ ಸಂಸಾರದಲ್ಲಿ ವಸ್ತುಗಳ ಎರಡನೆಯ ಗುಣವೇ ಅಸುಖ. ಯಾವ ಸುಖವಾಗಿರುವುದನ್ನು ಮೊದಲು ಅನುಭವಿಸಿದರೂ ಕೊನೆಗೆ ಅದು ಅಸುಖದಲ್ಲಿ ಪರ್ಯವಸಾನವಾಗುವುದು. ಯಾವುದನ್ನು ನಾವು ಸುಖ ಎನ್ನುತ್ತೇವೆಯೋ ಅದು ಅಸುಖದ ಮೇಲಿರುವ ತೆಳ್ಳನೆಯ ಸಿಹಿ. ಅದು ಚೀಪುತ್ತಿರು ವಾಗಲೇ ಸಿಹಿ ಹೋಗುವುದು, ಹಿಂದಿರುವ ಕಹಿ ಬರುವುದು. ಅದನ್ನು ಉಗುಳುವುದಕ್ಕೆ ಆಗುವುದಿಲ್ಲ. ಸಿಹಿ ಕಹಿ ಎರಡೂ ಒಂದೇ ವಸ್ತುವಿನ ಎರಡು ಭಾಗ. ನಾವು ಸಿಹಿ ಭಾಗ ತೆಗೆದುಕೊಂಡರೆ, ಕಹಿ ಭಾಗವನ್ನು ತೆಗೆದುಕೊಳ್ಳಬೇಕಾಗುವುದು. ಬೇಡ ಎಂದರೆ ಪ್ರಕೃತಿ ಬಿಡುವುದಿಲ್ಲ. ಬಾಯೊಳಗೆ ಹಾಕಿ ತುರುಕುವುದು. ಕಹಿ ಯಾರಿಗೆ ಬೇಡವೊ ಅವನು ಸಿಹಿಗೂ ಕೈ ಒಡ್ಡಬಾರದಾಗಿತ್ತು. ಯಾವಾಗ ಸಿಹಿಗೆ ಕೈಯೊಡ್ಡುತ್ತಾನೆಯೊ, ಆಗ ಕಹಿಯನ್ನು ತಿನ್ನಲೂ ಅವನು ಸಿದ್ಧವಾಗಿರಬೇಕು. ಹೆಂಡತಿ ಮಕ್ಕಳು, ಐಶ್ವರ್ಯ, ಅಧಿಕಾರ, ಕೀರ್ತಿ ಇವುಗಳಿಗೆಲ್ಲ ಚಂದ್ರನಿಗಿರುವಂತೆ ಒಂದು ಪೂರ್ಣಿಮೆಯ ಭಾಗವಿದೆ; ಮತ್ತೊಂದು ಅಮಾವಾಸ್ಯೆಯ ಭಾಗವಿದೆ. ಒಂದೇ ಮತ್ತೊಂದು ಆಗುತ್ತಿರುವುದು. ಈ ಸಂಸಾರದ ಲಕ್ಷಣವೇ ಇದು. 

ನೀನು ಇಲ್ಲಿ ಜನ್ಮವೆತ್ತಿರುವೆ ಎಂದರೆ ಮನುಷ್ಯನಾಗಿ ಹುಟ್ಟಿರುವೆ. ಈ ಪ್ರಪಂಚದಲ್ಲಿ ಮನುಷ್ಯ ನಾಗಿ ಹುಟ್ಟುವುದು ದೊಡ್ಡ ಒಂದು ವರ. ಲಕ್ಷಾಂತರ ಜೀವರಾಶಿಗಳಿವೆ. ಅವುಗಳಲ್ಲಿ ಯಾವುದೂ ಆಗದೆ ಮನುಷ್ಯನಾಗಿ ಹುಟ್ಟಿದುದಕ್ಕೆ ಎಷ್ಟೊ ಧನ್ಯವಾದಗಳನ್ನು ಅರ್ಪಿಸಬೇಕಾಗಿದೆ. ಏಕೆಂದರೆ ಮನುಷ್ಯ ಮಾತ್ರ ವಿಚಾರ ಮಾಡಬಲ್ಲ. ಇನ್ನು ಯಾವ ಪ್ರಾಣಿಯೂ ವಿಚಾರ ಮಾಡಲಾರದು. ಮನುಷ್ಯ ಮಾತ್ರ ಅನುಭವದಿಂದ ಬುದ್ಧಿ ಕಲಿಯಬಲ್ಲ. ಮನುಷ್ಯ ಮಾತ್ರ ಈಗಿನ ಮತ್ತು ಮುಂದಿನ ಸ್ಥಿತಿಗತಿಗಳನ್ನು ಊಹಿಸಬಲ್ಲ. ಇತರ ಪ್ರಾಣಿಗಳಾವುದಕ್ಕೂ ಸಾಧ್ಯವಿಲ್ಲ. ಇಂತಹ ಅಪೂರ್ವ ಮನುಷ್ಯ ಜನ್ಮ ನಮಗೆ ಮತ್ತೊಂದು ಸಲ ಬರುವುದೆಂದು ಹೇಳುವುದಕ್ಕೆ ಆಗುವುದಿಲ್ಲ. ಈಗ ಅದನ್ನು ಸರಿಯಾಗಿ ಉಪಯೋಗಿಸಿಕೊಂಡರೆ ಮುಕ್ತರಾಗಿ ಹೋಗಬಹುದು.

ಹಾಗೆಯೇ ಮುಕ್ತಿಯನ್ನು ಗಳಿಸಬೇಕಾದರೆ ಸುಲಭದ ಹಾದಿಯನ್ನು ಭಗವಂತ ತೋರಿರುವನು. ಅದೇ ಅವನನ್ನು ಭಜಿಸುವುದು. ಅವನನ್ನು ಚಿಂತಿಸುವುದು, ಧ್ಯಾನಿಸುವುದು, ಪ್ರಾರ್ಥಿಸುವುದು. ಭಗವಂತನಲ್ಲಿ ಮೊರೆಯಿಡಬೇಕು, ಮತ್ತೊಮ್ಮೆ ಈ ಸಂಸಾರಕ್ಕೆ ಬರದಂತೆ ನೋಡಿಕೊ ಎಂದು. ಆಗ ಅವನು ನಮ್ಮನ್ನು ಉದ್ಧಾರ ಮಾಡುತ್ತಾನೆ. ಈ ಸಂಸಾರ ಅಸುಖ, ಅನಿತ್ಯವಾಗಿಯೇ ಇರುವುದು. ಇದನ್ನೇನು ಅವನು ಬದಲಾವಣೆ ಮಾಡುವುದಿಲ್ಲ. ಆದರೆ ನಮ್ಮನ್ನು ಬದಲಾಯಿಸುತ್ತಾನೆ. ಈ ಅನಿತ್ಯ ಅಸುಖದ ಸಂಸಾರದ ಮೇಲಿನಿಂದ ನಮ್ಮ ಮನಸ್ಸನ್ನು ತೆಗೆದು ಹಾಕುತ್ತಾನೆ. ಇದರ ಮೇಲಿನ ವ್ಯಾಮೋಹವನ್ನು ಬಿಡಿಸುತ್ತಾನೆ.

\begin{verse}
ಮನ್ಮನಾ ಭವ ಮದ್ಭಕ್ತೋ ಮದ್ಯಾಜೀ ಮಾಂ ನಮಸ್ಕುರು ।\\ಮಾಮೇವೈಷ್ಯಸಿ ಯುಕ್ತ್ವೈವಮಾತ್ಮಾನಂ ಮತ್ಪರಾಯಣಃ \versenum{॥ ೩೪ ॥}
\end{verse}

{\small ನನ್ನಲ್ಲಿ ಮನಸ್ಸಿಡು, ನನ್ನ ಭಕ್ತನಾಗು, ನನ್ನನ್ನು ಪೂಜಿಸು, ನನ್ನನ್ನೇ ನಮಸ್ಕರಿಸು, ನನ್ನಲ್ಲಿ ಪರಾಯಣನಾಗಿ ನನ್ನೊಂದಿಗೆ ಆತ್ಮವನ್ನು ಸೇರಿಸು. ನೀನು ನನ್ನನ್ನು ಹೊಂದುತ್ತೀಯೆ.}

ಶ್ರೀಕೃಷ್ಣ ಇಲ್ಲಿ ನಾನು ಎಂಬುದನ್ನು ಉಪಯೋಗಿಸುತ್ತಾನೆ. ಈ ನಾನು ಒಂದು ಸಾಂತ ವ್ಯಕ್ತಿಗೆ ಅನ್ವಯಿಸುವುದಿಲ್ಲ. ಇಲ್ಲಿ ನಾನು ಎಂಬುದೂ ಒಂದೇ ಭಗವಂತ ಎಂಬುದೂ ಒಂದೇ. ಅವನನ್ನು ಶ್ರೀಕೃಷ್ಣ ಎಂತಲೇ ಕರೆಯಬೇಕಾಗಿಲ್ಲ. ಪ್ರತಿಯೊಬ್ಬ ಉಪಾಸಕನೂ ತನಗೆ ತೋರಿದ ರೀತಿಯಲ್ಲಿ ಅವನನ್ನು ಕರೆಯಬಹುದು. ಎಲ್ಲವೂ ಅವನಿಗೇ ಸೇರುವುದು.

ಭಗವಂತನಲ್ಲಿ ಮನಸ್ಸನ್ನು ಇಡು ಎನ್ನುತ್ತಾನೆ. ನಮ್ಮ ಮನಸ್ಸೆಂಬ ಪಾತ್ರೆಯನ್ನು ಅವನಲ್ಲಿ ಅದ್ದಬೇಕು. ಅವನಿಂದ ನಮ್ಮ ಮನಸ್ಸು ತುಂಬಿ ತುಳುಕುತ್ತಿರಬೇಕು. ಅಲ್ಲಿ ಅವನಲ್ಲದೆ ಬೇರೆ ಯಾವುದೂ ಇರಕೂಡದು. ಎರಡು ವಸ್ತುಗಳು ಏಕ ಕಾಲದಲ್ಲಿ ಒಂದು ಸ್ಥಳವನ್ನು ಆಕ್ರಮಿಸಿಕೊಳ್ಳು ವುದಿಲ್ಲ ಎಂಬ ಪ್ರಕೃತಿಯ ನಿಯಮವಿದೆ. ಎಲ್ಲಿ ದೇವರಿರುವನೊ ಅಲ್ಲಿ ಕಾಮಕಾಂಚನಗಳು ಇರಲಾರವು. ಎಲ್ಲಿ ಕಾಮಕಾಂಚನಗಳು ಇವೆಯೊ ಅಲ್ಲಿ ದೇವರು ಇರಲಾರನು. ಭಗವಂತನಿಗೆ ನಮ್ಮ ಮನಸ್ಸನ್ನು ಮುಡುಪಾಗಿಟ್ಟರೆ ಅಲ್ಲಿ ಬೇರೆ ಇನ್ನು ಯಾವುದೂ ಇರಲಾರದು.

ನಾವು ಭಗವಂತನ ಭಕ್ತರಾಗಬೇಕು, ಅವನ ಸೇವಕರಾಗಬೇಕು, ಅವನ ಆಣತಿಯನ್ನು ಪರಿಪಾಲಿಸು ವವರಾಗಬೇಕು. ಭಕ್ತನಿಗೆ ತುಂಬ ಪ್ರಿಯವಾಗಿರುವುದು ಭಗವಂತನ ಸೇವೆಯನ್ನು ಮಾಡುವುದು. ನಿಜವಾದ ಭಕ್ತ ಭಗವಂತನ ಸೇವೆಯಲ್ಲಿ ತನ್ನ ಬಾಳನ್ನು ಸಮೆಸುತ್ತಾನೆ. ತಾನು ತುಕ್ಕು ಹಿಡಿದು ಹೋಗುವುದಿಲ್ಲ.

ನಾವು ಭಗವಂತನನ್ನು ಪೂಜಿಸಬೇಕು; ಐಶ್ವರ್ಯವನ್ನಲ್ಲ, ಅಧಿಕಾರವನ್ನಲ್ಲ, ಕೀರ್ತಿಯನ್ನಲ್ಲ. ದೇವರಿಗಾಗಿ ದೇವರನ್ನು ಪೂಜಿಸಬೇಕು. ದೇವರ ಉಗ್ರಾಣದಿಂದ ಏನನ್ನೊ ವಸೂಲಿ ಮಾಡುವುದಕ್ಕೆ ಅವನನ್ನು ಪೂಜಿಸಕೂಡದು, ಅವನನ್ನು ಹೊಗಳಕೂಡದು. ಅವನಿಗೆ ನಮಸ್ಕರಿಸಬೇಕು. ಅವನೆದುರಿಗೆ ನಮ್ಮ ಅಹಂಕಾರವನ್ನು ಅರ್ಪಿಸಬೇಕು, ಬಾಗಿಸಬೇಕು. ಅಹಂಕಾರವನ್ನು ಮೆರೆಸುವುದು ಭಗವಂತ ನನ್ನು ನಮಸ್ಕರಿಸುವುದು ಒಟ್ಟಿಗೆ ಹೋಗುವುದಿಲ್ಲ. ಭಗವಂತ ಭೂಮ. ಭೂಮಕ್ಕೆ ಅಂತಹ ಆಕರ್ಷಣೆ ಇದೆ. ಅಲ್ಪ ತನಗೆ ತಾನೇ ಅದರೆದುರಿಗೆ ಮಣಿಯುವುದು, ಭೂಮವನ್ನು ನೋಡಿ, ಅನುಭವಿಸಿ, ಅದಕ್ಕೆ ಮಣಿಯದೆ ಇರುವುದಕ್ಕೆ ಆಗುವುದಿಲ್ಲ.

ಅವನಲ್ಲಿಯೇ ಪರಾಯಣನಾಗಿರಬೇಕು ಎಂದರೆ ಅನುಗಾಲವೂ ಮನಸ್ಸು ಅವನನ್ನೇ ಕುರಿತು ಚಿಂತಿಸುತ್ತಿರಬೇಕು, ಅವನೆಡೆಗೇ ಹರಿಯುತ್ತಿರಬೇಕು. ಕ್ಷಣಕಾಲವೂ ಅವನನ್ನು ಮರೆಯಕೂಡದು. ಬದುಕಿರುವಾಗ ಮನಸ್ಸು ಅಲ್ಲಿದ್ದರೆ ಕೊನೆಗಾಲದಲ್ಲಿಯೂ ಮನಸ್ಸು ಅವನಲ್ಲಿ ಲಯವಾಗುವುದು. ಅವನು ತನ್ನತನವನ್ನು ಕಳೆದುಕೊಂಡು ಭಗವಂತನಲ್ಲಿ ಒಂದಾಗುವನು. ಹನಿ ಸಾಗರಕ್ಕೆ ಬಿದ್ದಂತೆ, ಉಪ್ಪು ಅಥವಾ ಸಕ್ಕರೆ ನೀರಿನಲ್ಲಿ ಬಿದ್ದು ಕರಗಿಹೋದಂತೆ, ನಮ್ಮ ಅಲ್ಪತನ ಹೋಗುವುದು, ಭೂಮದೊಂದಿಗೆ ಒಂದಾಗುವುದು.

