
\chapter{ಧ್ಯಾನಶ್ಲೋಕಗಳು}

ಭಗವದ್ಗೀತೆಯನ್ನು ಪಾರಾಯಣ ಮಾಡುವುದಕ್ಕೆ ಮುಂಚೆ ಬರುವ ಧ್ಯಾನಶ್ಲೋಕಗಳನ್ನು ಓದುವುದು ರೂಢಿಯಾಗಿದೆ. ಇಲ್ಲಿ ಗೀತೆ, ಇದನ್ನು ಹೇಳುವ ಶ‍್ರೀಕೃಷ್ಣ, ಇದು ಇರುವ ಮಹಾಭಾರತ, ಮಹಾಭಾರತವನ್ನು ಜಗತ್ತಿಗೆ ಕೊಟ್ಟ ವ್ಯಾಸರು ಇವರನ್ನೆಲ್ಲಾ ಕೊಂಡಾಡಿರುವರು.

\begin{shloka}
ಪಾರ್ಥಾಯ ಪ್ರತಿಬೋಧಿತಾಂ ಭಗವತಾ ನಾರಾಯಣೇನ ಸ್ವಯಂ\\ವ್ಯಾಸೇನ ಗ್ರಥಿತಾಂ ಪುರಾಣಮುನಿನಾ ಮಧ್ಯೇ ಮಹಾಭಾರತಮ್~।\\ಅದ್ವೈತಾಮೃತವರ್ಷಿಣೀಂ ಭಗವತೀಂ ಅಷ್ಟಾದಶಾಧ್ಯಾಯಿನೀಂ\\ಅಂಬ ತ್ವಾಮನುಸಂದಧಾಮಿ ಭಗವದ್​ಗೀತೇ ಭವದ್ವೇಷಿಣೀಮ್\hfill॥ ೧~॥
\end{shloka}

\begin{artha}
ನಾರಾಯಣನೆ ಪಾರ್ಥನಿಗೆ ಇದನ್ನು ಬೋಧಿಸಿದನು. ಪುರಾಣ ಮುನಿಗಳಾದ ವ್ಯಾಸರು ಇದನ್ನು ಮಹಾಭಾರತದ ಮಧ್ಯದಲ್ಲಿ ಪೋಣಿಸಿರುವರು. ಹದಿನೆಂಟು ಅಧ್ಯಾಯಗಳುಳ್ಳ ಭಗವತಿಯು ಅದ್ವೈತ ಅಮೃತವನ್ನು ಮಳೆಗರೆಯು ತ್ತಿರುವಳು. ಸಂಸಾರವನ್ನು ನಾಶಮಾಡುವ ಭಗವದ್ಗೀತೆ\-ಯೆಂಬ ತಾಯಿಯನ್ನು ಧ್ಯಾನಿಸುತ್ತೇನೆ.
\end{artha}

ಭಗವದ್ಗೀತೆಯನ್ನು ಪಾರ್ಥನಿಗೆ ಬೋಧಿಸಿದವನು ಶ‍್ರೀಕೃಷ್ಣ. ಶ‍್ರೀಕೃಷ್ಣನು\break ಶ‍್ರೀಮನ್ನಾರಾಯಣನ ಅವತಾರವೆಂದು ಹಿಂದುಗಳಲ್ಲೆಲ್ಲ ಪ್ರಖ್ಯಾತನಾಗಿರುವನು. ಅವತಾರ ಮಾತ್ರವಲ್ಲ, ಅವತಾರಗಳಲ್ಲಿ ಅಂಶಾವತಾರಗಳಿವೆ, ಪೂರ್ಣಾವತಾರಗಳಿವೆ. ಶ‍್ರೀಕೃಷ್ಣನನ್ನು ಪೂರ್ಣಾವತಾರವೆಂದು ಕರೆಯುತ್ತಾರೆ. ಇವನದು ಬಹುಮುಖವಾದ ವ್ಯಕ್ತಿತ್ವ. ಇಂತಹ ಬಹು\-ಮುಖತೆ\-ಯನ್ನು ಬೇರೆ ಅವತಾರಗಳಲ್ಲಿ ಕಾಣುವುದು ಬಹಳ ಅಪರೂಪ. ಶ‍್ರೀಕೃಷ್ಣ ಆದರ್ಶ ಬಾಲಗೋಪಾಲ. ಶ‍್ರೀಕೃಷ್ಣ ನೋಡುವುದಕ್ಕೆ ಅಷ್ಟು ಮನೋಹರವಾಗಿದ್ದನು. ಭಾಗವತವನ್ನು ಬರೆದ ವ್ಯಾಸರು ಅವನನ್ನು ಸಾಕ್ಷಾತ್ ಮನ್ಮಥ ಎಂದು ಕರೆದಿರುವರು. ಅವನು ಗೋಕುಲ ಮತ್ತು ಬೃಂದಾವನದಲ್ಲಿ ಬೆಳೆದಾಗ ಅವನನ್ನು ಸಾಕಿ ಸಲಹಿದ ತಂದೆತಾಯಿಗಳಿಗೆ ಮಾತ್ರ ಆನಂದವನ್ನು ಕೊಟ್ಟವನಲ್ಲ, ಊರಿನ ಎಲ್ಲರೂ ಇವನನ್ನು ಪ್ರೀತಿಸುತ್ತಿದ್ದರು. ಅವನು ಎಲ್ಲರಿಗೂ ಬೇಕಾಗಿದ್ದನು. ಒಮ್ಮೆ ಶ‍್ರೀಕೃಷ್ಣನನ್ನು ನೋಡಿದರೆ ಆಕರ್ಷಣದ ಸುಳಿಯಲ್ಲಿ ಸಿಕ್ಕಿಕೊಂಡವರು ತಪ್ಪಿಸಿಕೊಳ್ಳುವಂತಿರಲಿಲ್ಲ; ಪುನಃಪುನಃ ಅವನನ್ನು ನೋಡಬೇಕೆಂದು ಬಯಸುತ್ತಿದ್ದರು. ಹಲವಾರು ಕಾರಣಗಳಿಂದ ಅವನನ್ನು ನೋಡಲು ಬರುತ್ತಿದ್ದರು. ಶ‍್ರೀಕೃಷ್ಣ ತನ್ನ ಓರಗೆಯವರಿಗೆ ಪ್ರಾಣವಾಗಿದ್ದ. ಹಿರಿಯರಿಗೆ ಶ‍್ರೀಕೃಷ್ಣನನ್ನು ಎಷ್ಟು ನೋಡಿದರೂ ತೃಪ್ತಿಯಿಲ್ಲ. ಗೋಪಿಯರಿಗೆ ಜೀವವೇ ಆಗಿದ್ದ ಶ‍್ರೀಕೃಷ್ಣ ಬೃಂದಾವನದಲ್ಲಿ ಇರುವವರೆಗೆ. ಶ‍್ರೀಕೃಷ್ಣ ಅಲ್ಲಿಂದ ಹೊರಟಮೇಲೆ ಬೇರೆ ವ್ಯಕ್ತಿಯಾಗುವನು, ಚತುರೋಪಾಯಗಳಲ್ಲಿ ನಿಪುಣನಾಗುವನು, ವೀರನಾಗುವನು, ರಾಜ\-ನಿರ್ಮಾಪಕನಾಗುವನು, ಬೇಕಾದಷ್ಟು ಶತ್ರುಗಳ ಮೇಲೆ ಯುದ್ಧವನ್ನು ಮಾಡಿದನು. ಆದರೂ ಫಲದ ಮೇಲೆ ಆಸಕ್ತಿ ಇಲ್ಲ. ಉಗ್ರಸೇನ ಬದುಕಿರುವವರೆಗೆ ಶ‍್ರೀಕೃಷ್ಣ ಅವನನ್ನೇ ಸಿಂಹಾಸನದ ಮೇಲೆ ಕೂಡಿಸಿದ್ದನು. ತಾನು ರಾಜನಾಗಲಿಲ್ಲ. ಪಾಂಡವರ ಅತಿ ನಿಕಟ ಗೆಳೆಯನಾಗುವನು. ಅದರಲ್ಲಿಯೂ ಅರ್ಜುನನ್ನು ಕಂಡರಂತೂ ಅತ್ಯಂತ ಪ್ರೀತಿ. ಶ‍್ರೀಕೃಷ್ಣ ಏನನ್ನು ಬೇಕಾದರೂ ಅವನಿಗೆ ಮಾಡಲು ಸಿದ್ಧನಾಗಿದ್ದನು. ಅವನು ಮಾಡುವ ಕೆಲಸದಲ್ಲಿ ಉಚ್ಚ ನೀಚ ಭಾವವೇ ಇಲ್ಲ. ಎಲ್ಲವನ್ನು ಅಷ್ಟೊಂದು ಆಸಕ್ತಿಯಿಂದ ಮಾಡುವನು. ಪಾಂಡವರು ರಾಜಸೂಯ ಯಾಗವನ್ನು ಮಾಡಿದಾಗ ಶ‍್ರೀಕೃಷ್ಣನಿಗೆ ಅತಿಥಿಗಳನ್ನು ಸ್ವಾಗತಿಸುವ ಕೆಲಸ ಬರುವುದು. ಯಾರ ಬೆರಳಸನ್ನೆಗೆ ರಾಜಾಧಿರಾಜರು ಸಿಂಹಾಸನದಿಂದ ಕೆಳಗೆ ಇಳಿದು ಬರುತ್ತಿದ್ದರೊ ಅಂತಹ ವ್ಯಕ್ತಿ ರಾಜರುಗಳಿಗೆ ಅರ್ಘ್ಯಪಾದ್ಯಗಳನ್ನು ಕೊಟ್ಟು ಉಚಿತ ಆಸನದಲ್ಲಿ ಕುಳ್ಳಿರಿಸುವನು. ಅದೇ ಕೃಷ್ಣನಿಗೆ ತುಂಬಿದ ಸಭೆಯಲ್ಲಿ ಅಗ್ರಪೂಜೆಯನ್ನು ಮಾಡಿದಾಗ ಅವನ ತಲೆತಿರುಗಿ ಹೋಗುವುದೂ ಇಲ್ಲ. ಅದೇ ಶ‍್ರೀಕೃಷ್ಣ ಈಗ ಕುರುಕ್ಷೇತ್ರದಲ್ಲಿ ಅರ್ಜುನನ ಸಾರಥಿಯಾಗಿರುವನು. ಈ ಸ್ಥಳದಲ್ಲಿಯೇ ತಾನೊಬ್ಬ ಮಹಾ ಗುರು, ತತ್ವಜ್ಞಾನಿ, ಆಚಾರ್ಯ, ಸತ್ಯವನ್ನು ಪೂರ್ಣವಾಗಿ ಸಾಕ್ಷಾತ್ಕಾರ ಮಾಡಿಕೊಂಡವನು ಎಂಬುದನ್ನು ತೋರುತ್ತಿರುವನು. ಇಂತಹ ಮಹಾನ್ ವ್ಯಕ್ತಿ ಮಾತನಾಡುವಾಗ ಆ ಮಾತಿನಲ್ಲಿ ಸಾಕ್ಷಾತ್ಕಾರದ ಕಿಡಿ ಸ್ಪಂದಿಸುತ್ತಿರುವುದು. ಅದು ಕೇಳಿದವರೆದೆಗೆ ತಾಕುವುದು. ಅವನ ಮಾತಿನಲ್ಲಿರುವ ಕಾಂತಿ ಮತ್ತು ಕಾವು ಅರ್ಜುನನಿಗೆ ಮಾತ್ರ ಆಗ ತಾಕಿದ್ದಲ್ಲ, ಈಗಲೂ ಯಾರು ಯಾರು ಗೀತೆಯನ್ನು ಓದುತ್ತಾರೊ ಅವರಿಗೆಲ್ಲ ಇದು ಸ್ವಯಂವೇದ್ಯ. ನಮ್ಮ ಸಂಸ್ಕಾರಕ್ಕೆ ತಕ್ಕಂತೆ ನಾವು ಗೀತೆಯಲ್ಲಿ ಭಗವಂತನನ್ನು ಕಾಣಬಹುದು.

ವ್ಯಾಸರು ಈ ಭಗವದ್ಗೀತೆಯನ್ನು ಮಹಾಭಾರತದ ಮಧ್ಯದಲ್ಲಿ ಪೋಣಿಸಿರುವರು. ದೊಡ್ಡ ದೊಂದು ಬಂಗಾರದ ಹಾರ, ಅದರ ಮಧ್ಯದಲ್ಲಿ ಒಂದು ಸುಂದರವಾದ ವಜ್ರದ ಪದಕವನ್ನು ಪೋಣಿಸಿರುವ ಹಾಗಿದೆ. ಮಹಾಭಾರತವೆಂಬುದು ಸುಂದರ ಮನೋಹರವಾದ ಕಥೆಗಳ ಗೊಂಚಲು. ಈ ಕಥೆಯೇ ಮಾನವ ಸಹಜವಾದ ಕಥೆ. ನಮ್ಮಂತೆಯೇ ಇವರು ಸುಖ ದುಃಖ ಲಾಭ ನಷ್ಟಗಳ ಕಣಿವೆಯಲ್ಲಿ ಸಾಗಿಹೋಗುತ್ತಿರುವರು. ಇವರನ್ನೂ ಹಲವಾರು ಸಮಸ್ಯೆಗಳು ಮುತ್ತುವುವು. ಇವರು ಇವುಗಳೊಂದಿಗೆ ಪಾರಾಗುವುದಕ್ಕೆ ಸಮಸ್ಯೆಯೊಂದಿಗೆ ಹೋರಾಡುವರು. ಬೆನ್ನು ತೋರಿಸಿ ಕಂಬಿ ಕೀಳುವವರಲ್ಲ. ಕಥೆಯ ಮಧ್ಯದಲ್ಲಿ ದೊಡ್ಡದೊಂದು ಅನರ್ಘ್ಯವಾದ ರತ್ನವಿದೆ. ತತ್ವವನ್ನು ಪೋಷಿಸುವುದಕ್ಕಾಗಿಯೇ ಇಲ್ಲಿ ಕಥೆಗಳು ಇರುವುದು.

ಇಲ್ಲಿ ಗೀತೆಯಲ್ಲಿ ಬರುವ ತತ್ತ್ವ ಅದ್ವೈತ ಅಮೃತ. ಪ್ರಪಂಚದಲ್ಲಿ ಭಗವಂತನೊಬ್ಬನೇ ಸತ್ಯ, ಉಳಿದವುಗಳೆಲ್ಲ ಮಿಥ್ಯ, ಅನಿತ್ಯ. ಅವನನ್ನು ಬಿಟ್ಟರೆ ಈ ಪ್ರಪಂಚ ದುಃಖಾಲಯ ಅಶಾಶ್ವತ. ಈ ಅಮೃತ ನಮಗೆ ಹಿಡಿದಿರುವ ಭವರೋಗವನ್ನು ಬಿಡಿಸುವುದಕ್ಕಾಗಿ ಇರುವುದು. ಈ ಸಂಸಾರದಲ್ಲಿರುವುದೇ ದುಃಖ. ಮಧ್ಯೆ ಮಧ್ಯೆ ಕೆಲವು ಸುಖದ ದ್ವೀಪಗಳು ಇರುವಂತೆ ಭಾಸವಾಗುವುದು. ಆದರೆ ದುಃಖಸಮುದ್ರದ ಭೀಮ ಅಲೆಗಳು ಸುಖದ್ವೀಪವನ್ನು ಕ್ರಮೇಣ ಆಪೋಶನ ತೆಗೆದುಕೊಳ್ಳುವುವು. ನಮ್ಮಲ್ಲಿರುವ ಸಾಂಸಾರಿಕತೆ ಮಾಯವಾಗಬೇಕಾದರೆ ಭಗವಂತನ ಮೇಲೆ ಭಕ್ತಿ ಉದ್ದೀಪನೆ ಆಗಬೇಕು, ಸಂಸಾರದ ಮೇಲೆ ಆಸಕ್ತಿ ತಗ್ಗಬೇಕು. ಈ ಸಂಸಾರ ನಾಮರೂಪಗಳಿಂದ\break ಎಷ್ಟೇ ಮನೋಹರವಾಗಿ ಕಂಡರೂ ಜೀವಿಗಳೆಂಬ ನೊಣಗಳನ್ನು ಹಿಡಿಯುವುದಕ್ಕಾಗಿ ಮಾಯೆ\-ಯೆಂಬ ಜೇಡ ನೆಯ್ದಿರುವ ಬಲೆ, ಎಂದು ಹೃದಯಕ್ಕೆ ತಾಕಬೇಕು. ಇದನ್ನು ಮಾಡುವುದು ಗೀತೆ.

ಈ ಭಗವದ್ಗೀತೆಯನ್ನು ತಾಯಿಯೆಂದು ಸಂಬೋಧಿಸುವರು. ಈ ತಾಯಿಯ ಅಂಗಾಂಗಗಳೇ ಹದಿನೆಂಟು ಅಧ್ಯಾಯಗಳು. ಆ ಗೀತಾಮಾತೆಯನ್ನು ಪ್ರಥಮದಲ್ಲಿ ಓದುಗರು ಧ್ಯಾನಿಸುವರು. ತಾಯಿ ಹೇಗೆ ತನ್ನ ರಕ್ತವನ್ನು ಎದೆಹಾಲು ಮಾಡಿ ಮಕ್ಕಳಿಗೆ ಕೊಟ್ಟು ಬೆಳೆಸುವಳೋ ಹಾಗೆ ಗೀತಾಮಾತೆ ತನ್ನೆಡೆಗೆ ಓದಲು ಬರುವ ಮಕ್ಕಳಿಗೆ ಜೀರ್ಣಿಸಿಕೊಳ್ಳಲು ಸುಲಭವಾದ ತತ್ವವನ್ನೇ ನೇರವಾಗಿ ಹೇಳುವಳು. ಇಲ್ಲಿ ಯಾವ ಬುದ್ಧಿಯ ಕಸರತ್ತು ಕೂಡ ಬೇಕಿಲ್ಲ, ಇದನ್ನು ತಿಳಿದುಕೊಳ್ಳಲು. ಮಗುವಿನಂತೆ ತಾಯೆಡೆಗೆ ಬಂದರೆ ಸಾಕು, ತಾಯಿಯ ಉಣಿಸು ಕಾದಿದೆ ನಮಗೆ ಕೊಡಲು.

\begin{shloka}
ನಮೋsಸ್ತು ತೇ ವ್ಯಾಸ ವಿಶಾಲಬುದ್ಧೇ ಪುಲ್ಲಾರವಿಂದಾಯತಪತ್ರನೇತ್ರ~।\\ಯೇನ ತ್ವಯಾ ಭಾರತತೈಲಪೂರ್ಣಃ ಪ್ರಜ್ವಾಲಿತೋ\hfill\break ಜ್ಞಾನಮಯಪ್ರದೀಪಃ\hfill॥ ೨~॥
\end{shloka}

\begin{artha}
ವಿಶಾಲ ಬುದ್ಧಿವಂತರಾದ ಕೆಂದಾವರೆಯಂತಿರುವ ಕಣ್ಣುಗಳುಳ್ಳ ವ್ಯಾಸರಿಗೆ ನಮಸ್ಕಾರ. ಭಾರತ ತೈಲದಿಂದ ಕೂಡಿದ ಜ್ಞಾನಮಯವಾದ ಪ್ರದೀಪ ನಿಮ್ಮಿಂದ ಹತ್ತಿಸಲ್ಪಟ್ಟಿತು.
\end{artha}

ಇಲ್ಲಿ ವ್ಯಾಸರಿಗೆ ಎರಡು ಗುಣವಾಚಕಗಳನ್ನು ಕೊಡುವರು. ಒಂದು ವಿಶಾಲವಾದ ಬುದ್ಧಿ. ಮತ್ತೊಂದು ಕೆಂದಾವರೆಯಂತೆ ವಿಕಸಿತವಾದ ಕಣ್ಣುಗಳು ಭಾವಕ್ಕೆ ಚಿಹ್ನೆಯಂತಿವೆ. ಮಹಾಭಾರತವನ್ನು ಜಗತ್ತಿಗೆ ಕೊಟ್ಟ ವ್ಯಾಸರ ಬುದ್ಧಿಯ ಪ್ರತಿಭೆಗೆ ಯಾರು ಅಚ್ಚರಿಗೊಳ್ಳದೆ ಇರುವರು? ಇಡೀ ಪ್ರಪಂಚದ ಸಾರವನ್ನು ಹಿಂಡಿದಂತಿದೆ ಮಹಾಭಾರತ. ಈ ಮಹಾಕಾವ್ಯವೇ ಒಂದು ಸಂಕ್ಷಿಪ್ತ ಬ್ರಹ್ಮಾಂಡ. ಎಂತೆಂತಹ ಪಾತ್ರಗಳು ಇಲ್ಲಿ ಸುಳಿಯುವರು, ಎಂತೆಂತಹ ಉದ್ದೇಶಗಳನ್ನು ಇಲ್ಲಿ ಸಾಧಿಸಲೆತ್ನಿಸುವರು. ಈ ಜಗತ್ತಿನ ಬಾಹ್ಯವನ್ನು ಎಷ್ಟು ಕೂಲಂಕಷವಾಗಿ ವ್ಯಾಸರು ಚಿತ್ರಿಸುವರೋ ಅಷ್ಟೇ ಅಥವಾ ಅದಕ್ಕಿಂತ ಆಳವಾಗಿ ಮಾನವನ ಮನಸ್ಸನ್ನು ವಿಭಜನೆ ಮಾಡುವರು. ರವಿ ಕಾಣದ್ದನ್ನು ಕವಿ ಕಂಡ ಎಂಬುದು ವ್ಯಾಸರಿಗೆ ಅಕ್ಷರಶಃ ಅನ್ವಯಿಸುವುದು. ರವಿಯಾದರೋ ಹೊರಗಿನದನ್ನು ನೋಡುವನು. ವ್ಯಾಸರಂತಹ ಪುಷಿಕವಿಗಳು ಮಾನವನ ಮನಸ್ಸನ್ನೆಲ್ಲ ಶೋಧಿಸುವರು. ಈ ಪ್ರಚಂಡ ಬುದ್ಧಿ ಕಿರಣದೆದುರಿಗೆ ಬಚ್ಚಿಟ್ಟುಕೊಳ್ಳಬಲ್ಲ ಮಾನವರಿಲ್ಲ.

ಎಷ್ಟು ವ್ಯಾಸರ ಬುದ್ಧಿ ಸೂಕ್ಷ್ಮವೋ ಅಷ್ಟೇ ಅವರ ಭಾವ ಆಳವಾದುದು, ವಿಶಾಲವಾದುದು. ಮಾನವನನ್ನು ಹಿಂಡುವ, ಅವನನ್ನು ತನ್ನ ಕೈಯಲ್ಲಿಟ್ಟುಕೊಂಡು ಕುಣಿಸುವ ಯಾವ ರಸವೂ ವ್ಯಾಸರ ಹೃದಯಕ್ಕೆ ಹೊರಗಲ್ಲ. ಮಹಾಭಾರತವನ್ನು ಬರೆದ ವ್ಯಾಸರೆಂಬ ವ್ಯಕ್ತಿ ಒಂದು ವಿರಾಟ್ ಚೈತನ್ಯ.

ಈ ವ್ಯಾಸರು ಮಹಾಭಾರತವೆಂಬ ಕಥೆಯ ಎಣ್ಣೆಯನ್ನು ಒಂದು ದೊಡ್ಡ ಪ್ರದೀಪಕ್ಕೆ ಹಾಕಿದರು. ಅಲ್ಲಿ ಬೇಕಾದಷ್ಟು ಕಥೆ ಉಪಕಥೆಗಳು ಎಲ್ಲರ ಕಣ್​ಮನಗಳನ್ನು ಸೆಳೆಯುತ್ತವೆ. ಆ ಕಥೆಯನ್ನು ಎಷ್ಟು ಸಾವಿರ ವರುಷಗಳ ಹಿಂದೆ ಅವರು ಬರೆದರೊ ಗೊತ್ತಿಲ್ಲ. ಈಗಲೂ ಪ್ರತಿಯೊಬ್ಬ ಭಾರತೀಯ ಆಬಾಲ ವೃದ್ಧರಿಗೆ ಇದು ಚಿರಪರಿಚಿತವಾಗಿದೆ, ಅವರ ನಾಡಿನಾಡಿಯಲ್ಲಿ ಪ್ರವಹಿಸಿದೆ. ಅವರ ಜೀವನದ ಆದರ್ಶವನ್ನು ರೂಪಿಸುತ್ತಿದೆ. ಜೀವನದಲ್ಲಿ ಜೀವಿ ಎದುರಿಸುವ ಸಮಸ್ಯೆಗೆಲ್ಲ ಮದ್ದು ಇದೆ ಇಲ್ಲಿ.

ಇಲ್ಲಿರುವುದೆಲ್ಲ ಬರೀ ರಸವತ್ತಾದ ಕಥೆ ಮಾತ್ರವಲ್ಲ. ಈ ಕಥೆಗಳನ್ನು ಮತ್ತಾವುದೋ ಒಂದನ್ನು ಪೋಷಿಸುವುದಕ್ಕಾಗಿ ನೆಯ್ದಿರುವುದು. ಅದೇ ಭವಜೀವಿಗಳಿಗೆ ದಾರಿ ಬೆಳಕಾಗಬಲ್ಲಂತಹ ಭಗವ ದ್ಗೀತಾ ಎಂಬ ತತ್ತ್ವದ ಬೆಳಕಿನ ಕುಡಿ. ಮಹಾಭಾರತದ ಕಥೆಗಳಲ್ಲಿ ನಾವು ನೋಡುವುದು ಅನುಷ್ಠಾನವನ್ನು. ಭಗವದ್ಗೀತೆಯಲ್ಲಿ ನಮಗೆ ಬರುವುದು ಆ ಅನುಷ್ಠಾನದ ಹಿಂದೆ ಇರುವ ಸತ್ಯ. ಒಂದು ಮತ್ತೊಂದಕ್ಕೆ ಪೋಷಕವಾಗಿದೆ. ಒಂದಿಲ್ಲದೆ ಮತ್ತೊಂದು ಇರಲಾರದು. ಎಣ್ಣೆಯಿಲ್ಲದೆ ದೀಪ ಉರಿಯಲಾರದು. ದೀಪ ಉರಿಯದೆ ಇದ್ದರೆ ಎಷ್ಟು ಎಣ್ಣೆಯಿದ್ದರೂ ಗಾಢಾಂಧಕಾರ. ಒಂದು ಅನುಷ್ಠಾನ, ಮತ್ತೊಂದು ಅನುಷ್ಠಾನದ ಹಿಂದೆ ಹಾಸುಹೊಕ್ಕಾಗಿರುವ ಸಿದ್ಧಾಂತ. ತತ್ವ ಎಲ್ಲರಿಗೂ ಹಿಡಿಸಲಾರದು. ಅದಕ್ಕಾಗಿ ತತ್ವ ಬೆರೆತ ಕಥೆ ಇದು. ಬರೀ ಅರೇಬಿಯನ್ ನೈಟ್ಸ್ ಮುಂತಾದ ಕಥೆಯಂತೆ ಅಲ್ಲ, ಬೇಜಾರು ಪರಿಹರಿಸುವುದಕ್ಕೆ ಮಾತ್ರವಲ್ಲ ಮಹಾಭಾರತ ಇರುವುದು, ಇದು ಜೀವನವನ್ನು ಪೋಷಿಸುವುದು, ಬಾಡಿದ ಚೇತನಕ್ಕೆ ಸ್ಫೂರ್ತಿ ಕೊಡುವುದು, ಭವಜೀವಿಗಳನ್ನು ಭಗವಂತನೆಡೆಗೆ ಕರೆದೊಯ್ಯುವ ದೇವದೂತರಂತಿವೆ ಇಲ್ಲಿ ಬರುವ ತತ್ತ್ವಗಳು. ಇಂತಹ ಮಹಾ ವ್ಯಕ್ತಿಗೆ ಬಾಗಿ ಎರಗುವುದು ನಾವು ಮಾಡಬೇಕಾದ ಮೊದಲನೆ ಕರ್ತವ್ಯ.

\begin{shloka}
ಪ್ರಪನ್ನಪಾರಿಜಾತಾಯ ತೋತ್ರವೇತ್ರೈಕಪಾಣಯೇ\\ಜ್ಞಾನಮುದ್ರಾಯ ಕೃಷ್ಣಾಯ ಗೀತಾಮೃತದುಹೇ ನಮಃ\hfill॥ ೩~॥
\end{shloka}

\begin{artha}
ಶ‍್ರೀಕೃಷ್ಣ ಶರಣ್ಯರಿಗೆ ಕಲ್ಪವೃಕ್ಷದಂತೆ ಇರುವನು. ಕೈಯಲ್ಲಿ ಒಂದು ಕೋಲು ಹಿಡಿದುಕೊಂಡಿರುವನು. ಜ್ಞಾನಮುದ್ರೆಯನ್ನು ಧರಿಸಿರುವನು. ಗೀತಾ ಅಮೃತವನ್ನು ಕರೆಯುತ್ತಿರುವನು. \hbox\bgroup ಇವನಿಗೆ ನಮಸ್ಕಾರ. \egroup
\end{artha}

ಇನ್ನು ಮಹಾಭಾರತದ ಆರಾಧ್ಯಮೂರ್ತಿಯಾದ ಶ‍್ರೀಕೃಷ್ಣನನ್ನು ಕೊಂಡಾಡುವರು. ಯಾರು ಅವನಲ್ಲಿ ಶರಣಾಗುವರೊ ಅವರಿಗೆ ಕಲ್ಪವೃಕ್ಷ. ಶರಣಾಗುವವನು ಹಿಂದೆ ಏನಾದರೂ ಆಗಿರಬಹುದು. ಮಾಡಬಾರದ ಪಾತಕಗಳನ್ನೆ ಮಾಡಿರಬಹುದು, ಆದರೂ ಒಮ್ಮೆ ಅವನಲ್ಲಿ ಶರಣಾದರೆ ಸಾಕು. ಅವರು ಕೇಳುವುದನ್ನೆಲ್ಲ ಕೊಡುವನು. ಈ ಮಾತನ್ನು ಕೇಳಿದಾಗ ಭಗವಂತನಿಂದ ವಸೂಲಿ ಮಾಡುವುದು ಎಷ್ಟು ಸುಲಭ ಎಂದು ಭಾವಿಸುವೆವು. ಆದರೆ ಆತನಲ್ಲಿ ಶರಣಾಗುವುದು ಅಷ್ಟು ಸುಲಭವಲ್ಲ. ನಮ್ಮ ಅಹಂಕಾರವನ್ನೆಲ್ಲ ಸಂಪೂರ್ಣ ಅವನಿಗೆ ಮಾರಿಕೊಳ್ಳಬೇಕು. ಆಗಲೇ ಅವನು ನಮಗೆ ಕೇಳಿದುದನ್ನೆಲ್ಲ ಕೊಡುವುದು. ಯಾವಾಗ ನಾವು ನಮ್ಮನ್ನು ಅವನಿಗೆ ಅರ್ಪಣೆ ಮಾಡಿಕೊಳ್ಳುವೆವೊ ಆಗ ಅವನು ನಮಗೆ ಕಲ್ಪತರುವಾಗುವನು. ನಮ್ಮನ್ನು ಅವನಿಗೆ ಮಾರಿಕೊಂಡ ಮೇಲೆ ನಾವು ಅವನನ್ನು ಏನು ಬೇಡುತ್ತೇವೆ? ಧನಕನಕ ವಸ್ತು ಅಧಿಕಾರ ಯಾವುದನ್ನೂ\break ಬೇಡುವುದಿಲ್ಲ. ಪರ ಮಾತ್ಮನ ಪ್ರೇಮವನ್ನು ರುಚಿ ನೋಡಿದಮೇಲೆ, ಅವನಿಗೆ ನಮ್ಮನ್ನು ಅರ್ಪಿಸಿಕೊಂಡ ಮೇಲೆ, ದೇವರು ನಾವು ಯಾವುದನ್ನು ಕೇಳಿದರೂ ಇಲ್ಲವೆನ್ನುವುದಿಲ್ಲ. ಭಕ್ತ ಭಕ್ತಿಮುಕ್ತಿಗಳನ್ನು ಮಾತ್ರ ಯಾಚಿಸುವನು, ಅವನ ಕಡೆಗೆ ಹೋಗುವುದಕ್ಕೆ ಇರುವ ಆತಂಕಗಳನ್ನು ನಿವಾರಿಸೆಂದು ಬೇಡುವನು. ಇನ್ನುಮೇಲೆ ದೇವರು ಅವನು ಯಾವುದನ್ನು ಕೇಳಿದರೂ ಇಲ್ಲವೆನ್ನುವುದಿಲ್ಲ. ಆದರೆ ನಿಜವಾದ ಭಕ್ತ ಲೌಕಿಕವಾದವುಗಳನ್ನು ಕೇಳುವುದಿಲ್ಲ. ಭಾಗವತದಲ್ಲಿ ನರಸಿಂಹ ಭಕ್ತ ಪ್ರಹ್ಲಾದನಿಗೆ ಪ್ರತ್ಯಕ್ಷವಾಗುತ್ತಾನೆ. ಭಗವಂತ ತನ್ನ ಭಕ್ತನ ಪ್ರೀತಿಗೆ ಮೆಚ್ಚಿ ಏನನ್ನಾದರೂ ವರವನ್ನು ಕೇಳು ಎಂದು ಪ್ರಹ್ಲಾದನಿಗೆ ಹೇಳುವನು. ಮೊದಲು ಪ್ರಹ್ಲಾದ, ನಿನ್ನನ್ನು ನೋಡಿದಮೇಲೆ ನನಗೇನೂ ಬೇಡ, ನಾನು ಕೃತಾರ್ಥನಾದೆ ಎನ್ನುವನು. ಆದರೂ ದೇವರು ನೀನು ಏನನ್ನಾದರೂ ನನ್ನಿಂದ ತೆಗೆದುಕೊಳ್ಳಬೇಕು ಎಂದು ಬಲಾತ್ಕರಿಸಿದಾಗ ಪ್ರಹ್ಲಾದ ಹೀಗೆ ಕೇಳಿಕೊಳ್ಳುತ್ತಾನೆ, “ಅವಿವೇಕಿಗಳು ವಿಷಯವಸ್ತುಗಳನ್ನು ಎಷ್ಟು ಉತ್ಕಟವಾಗಿ ಪ್ರೀತಿಸುವರೊ ಅಷ್ಟು ಉತ್ಕಟವಾಗಿ ನಿನ್ನ ಪಾದ\-ಕಮಲಗಳನ್ನು ನಾನು ಪ್ರೀತಿಸುವಂತೆ ಅನುಗ್ರಹಿಸು.”

ಶ‍್ರೀಕೃಷ್ಣ ತನ್ನ ಒಂದು ಕೈಯಲ್ಲಿ ದನ ಕಾಯುವವರ ಕೈಯಲ್ಲಿರುವಂತೆ ಕೋಲನ್ನು ಹಿಡಿದು\-ಕೊಂಡಿರುವನು. ಭವಜೀವಿಗಳೆಂಬ ದನಗಳನ್ನು ಕಾಯುತ್ತಿರುವವನು ಅವನು. ಬೇರೆ ಕಡೆ\break ಹೋದಾಗ ದಾರಿಗೆ ತರುವುದು ಕೋಲಿನ ಭಯ. ಕೋಲಿನಿಂದ ಹೆದರಿಸುವನು. ಕೆಲವು ವೇಳೆ ನಮ್ಮನ್ನು ಹೊಡೆಯಲೂ ಬಹುದು. ಆದರೆ ನಮಗೆ ಪೆಟ್ಟು ಕೊಟ್ಟು ನೋಯಿಸುವುದಕ್ಕಲ್ಲ. ನಮ್ಮಲ್ಲಿರುವ ನ್ಯೂನತೆ ಬಡಪೆಟ್ಟಿಗೆ ನಮ್ಮನ್ನು ಬಿಟ್ಟುಹೋಗದು. ಒಳ್ಳೆಯ ಮಾತು, ನಯ, ವಿನಯ ಎಲ್ಲವನ್ನೂ ಖರ್ಚುಮಾಡಿ ಆದಮೇಲೆ ದಾರಿಗೆ ಬರದೆ ಇದ್ದರೆ ನಮಗೆ ಒಳ್ಳೆಯದನ್ನು ಮಾಡುವುದಕ್ಕಾಗಿ ಏಟನ್ನು ತಾಕಿಸುವನು. ಕೆಲವು ವೇಳೆ ಭಂಡ ಕುದುರೆ ಗಾಡಿ ಮುಂದೆ ಹೋಗಬೇಕಾದರೆ ಗಾಡಿ ಹೊಡೆಯುವವನ ಚಾವಟಿ ಏಟಿನ ಬಿಸಿ ತಾಕಬೇಕು. ಅದಕ್ಕಾಗಿಯೆ ಶ‍್ರೀಕೃಷ್ಣ ಕೈಯಲ್ಲಿ ಕೋಲನ್ನು ಹಿಡಿದಿರುವನು. ಅವನ ಬೆರಳುಗಳಲ್ಲಿ ಜ್ಞಾನಮುದ್ರೆಯನ್ನು ನೋಡುತ್ತೇವೆ. ಜೀವನದಲ್ಲಿ ಪರಮಸತ್ಯವನ್ನು ತಿಳಿದವನು ಅವನು. ಅದನ್ನು ಹೇಳುವ ವಿಧಾನವನ್ನು ಬಲ್ಲವನು ಅವನು. ಗೀತೆಯಲ್ಲಿ ಬರುವ ಅತ್ಯಂತ ಗಹನವಾದ ಆಧ್ಯಾತ್ಮಿಕ ವಿಷಯಗಳನ್ನು ಎಷ್ಟು ಸುಲಭವಾಗಿ ಎಲ್ಲರ ಹೃದಯಕ್ಕೆ ತಾಕುವಂತೆ ಹೇಳುತ್ತಾನೆ ಶ‍್ರೀಕೃಷ್ಣ! ಬಲ್ಲವರಿಗೆಲ್ಲಾ ಅದನ್ನು ಹೇಳುವ ಕಲೆ ಗೊತ್ತಿರುವುದಿಲ್ಲ. ಶ‍್ರೀಕೃಷ್ಣ ಪರಮಸತ್ಯದ ಮೇಲೆ ನೆಲೆಸಿರುವನು, ಅದನ್ನು ಇತರರೂ ಪಡೆಯುವುದು ಹೇಗೆಂಬುದನ್ನು ತನ್ನದೇ ಆದ ರೀತಿಯಲ್ಲಿ ಹೇಳುವನು. ಜೀವನದಲ್ಲಿ ಅತಿ ಸುಲಭವಾಗಿರುವುದನ್ನು ಅತಿ ಕ್ಲಿಷ್ಟವಾದ ಭಾಷೆಯಲ್ಲಿ ಕೆಲವರು ಇಡುವರು. ಭಾವ ನಮಗೆ ಕಾಣದ ರೀತಿಯಲ್ಲಿ ಪಾಂಡಿತ್ಯ ಅದನ್ನು ರಕ್ಷಿಸುವುದು. ಆದರೆ ಶ‍್ರೀಕೃಷ್ಣನ ಭಾಷೆಯಾದರೋ ಜೀವನದ ಪರಮ ಸತ್ಯಗಳನ್ನು ಎಲ್ಲರೂ ಅರ್ಥಮಾಡಿಕೊಳ್ಳುವಂತೆ ಇರುವುದು. ಅದು ಹೃದಯದಿಂದ ಬಂದ ಭಾಷೆ. ಮತ್ತೊಂದು ಹೃದಯಕ್ಕೆ ನೇರವಾಗಿ ತಾಕುವುದು.

\newpage

ಶ‍್ರೀಕೃಷ್ಣ ಇಲ್ಲಿ ಏನು ಮಾಡುತ್ತಿರುವನು? ಅನಾವಶ್ಯಕವಾದ ಭಾವನೆಗಳನ್ನೆಲ್ಲ ಬಿಟ್ಟು ಯಾವುದು ಸಾರವೋ, ಭವಜೀವಿಗಳ ಅಜ್ಞಾನದ ರೋಗವನ್ನು ಗುಣಮಾಡುವುದೋ, ಅಂತಹ ಅಮೃತವನ್ನು ಕರೆಯುತ್ತಿರುವನು. ಈ ಹಾಲನ್ನು ಅರಗಿಸಿಕೊಳ್ಳುವುದು ಜೀವನದಲ್ಲಿ ಬಹಳ ಸುಲಭ. ನಮ್ಮ ಪೋಷಣೆಗೆ ಅತ್ಯಂತ ಆವಶ್ಯಕವಾದ ವಸ್ತುಗಳೆಲ್ಲ ಹೇಗೆ ಹಾಲಿನಲ್ಲಿದೆಯೋ ಅದರಂತೆಯೇ ಜೀವಿಯ ಆತ್ಮವಿಕಾಸಕ್ಕೆ ಬೇಕಾದ ಪೋಷಕ ದ್ರವ್ಯಗಳೆಲ್ಲ ಇಲ್ಲಿವೆ. ನಾವೆಲ್ಲ ವಯಸ್ಸಿನಲ್ಲಿ ಬಹಳ ಹಿರಿಯ ರಾಗಿರಬಹುದು. ಹಲವು ಮಕ್ಕಳ ತಾಯಿತಂದೆಗಳಾಗಿರಬಹುದು. ಆದರೆ ಆಧ್ಯಾತ್ಮಿಕ ಜೀವನ ದೃಷ್ಟಿಯಲ್ಲಿ ನಾವೆಲ್ಲ ಈಗತಾನೆ ಕಣ್ತೆರೆದ ಎಳೆ ಹಸುಳೆಗಳು. ಇಂತಹ ಮಕ್ಕಳಿಗೆ ಕೊಡುವುದಕ್ಕಾಗಿಯೆ ಶ‍್ರೀಕೃಷ್ಣ ಹಾಲನ್ನು ಕರೆಯುತ್ತಿರುವನು. ಇಂತಹ ಮಹಾಮಹಿಮನಿಗೆ ಮಾಡುವ ನಮಸ್ಕಾರದಿಂದಲೇ ಗೀತೆಯ ಅಧ್ಯಯನ ಪ್ರಾರಂಭವಾಗುವುದು.

\begin{shloka}
ಸರ್ವೋಪನಿಷದೋ ಗಾವೋ ದೋಗ್ಧಾ ಗೋಪಾಲನಂದನಃ~।\\ ಪಾರ್ಥೋ ವತ್ಸಃ ಸುಧೀರ್ಭೋಕ್ತಾ ದುಗ್ಧಂ ಗೀತಾಮೃತಂ ಮಹತ್\hfill॥ ೪~॥
\end{shloka}

\begin{artha}
ಉಪನಿಷತ್ತುಗಳೇ ಹಸುಗಳು, ಇದರ ಹಾಲನ್ನು ಕರೆಯುವವನೇ ಗೋಪಾಲನಂದನ. ಪಾರ್ಥನೇ ಕರು. ಮಹತ್ತಾದ ಗೀತಾಮೃತವೇ ಹಾಲು. ಇದನ್ನು ಪಾನ ಮಾಡುವವರೇ ಸಜ್ಜನರು.
\end{artha}

ಇಲ್ಲಿ ಉಪನಿಷತ್ತುಗಳನ್ನು ಹಸುಗಳಿಗೆ ಹೋಲಿಸುವರು. ಉಪನಿಷತ್ತಿನ ಸಾರವೇ ಅದರ ಹಾಲಿನಲ್ಲಿದೆ. ಉಪನಿಷತ್ತು ಸಾಧಾರಣ ಮನುಷ್ಯನಿಗೆ ಓದಿ ತಿಳಿದುಕೊಳ್ಳಲು ಕಷ್ಟ. ಜೊತೆಗೆ ಅದನ್ನು ಎಲ್ಲರೂ ಓದಕೂಡದೆಂದು ಅದರ ಸುತ್ತಲೂ ಹಿಂದಿನಿಂದ ಬೇಲಿಯನ್ನು ಬೇರೆ ಹಾಕಿದರು. ಏಕೆಂದರೆ ಇದನ್ನು ಓದಿ ತಿಳಿದುಕೊಳ್ಳಬೇಕಾದರೆ ಬುದ್ಧಿ ತುಂಬಾ ಸೂಕ್ಷ್ಮವಾಗಿರಬೇಕು, ಚಿತ್ತ ಪರಿಶುದ್ಧವಾಗಿರಬೇಕು. ಆಗ ಮಾತ್ರ ಅಲ್ಲಿರುವ ವಿಷಯವನ್ನು ತಿಳಿದುಕೊಳ್ಳಲು ಸಾಧ್ಯ. ಇಲ್ಲದೇ ಇದ್ದರೆ ಅಲ್ಲಿ ಒಂದು ಬರೆದಿದ್ದರೆ ನಾವೊಂದು ತಿಳಿದುಕೊಳ್ಳುತ್ತೇವೆ. ತತ್ವವನ್ನು ಸರಿಯಾಗಿ ತಿಳಿದುಕೊಳ್ಳದೆ ತಪ್ಪಾಗಿ ತಿಳಿದುಕೊಳ್ಳುತ್ತೇವೆ. ಅದನ್ನು ತಿಳಿದುಕೊಳ್ಳದೇ ಇದ್ದರೆ ಅಷ್ಟು ನಷ್ಟವಿಲ್ಲ, ನಾವು ಅಜ್ಞಾನಿಗಳಾಗಿ ಉಳಿಯುವೆವು ಅಷ್ಟೆ. ಆದರೆ ತಪ್ಪಾಗಿ ತಿಳಿದುಕೊಂಡರೆ ಅದರಿಂದ ನಮಗೆ ನಷ್ಟ, ಜೊತೆಗೆ ಅದನ್ನು ಇತರರಿಗೆ ಹೇಳಿದರೆ ಇತರರನ್ನೂ ಕೆಡಿಸುವೆವು. ಆದಕಾರಣವೇ ಅದನ್ನು ಓದಲು ಇಷ್ಟೊಂದು ಆತಂಕ ಕಲ್ಪಿಸಿದರು. ಭಗವದ್ಗೀತೆ ಬರುವುದು ಮಹಾಭಾರತದ ಮಧ್ಯದಲ್ಲಿ. ಮಹಾಭಾರತವನ್ನು ಪಂಚಮವೇದ ಎನ್ನುವರು. ವೇದವೇದಾಂತದಲ್ಲಿ ಅಡಗಿರುವ ಗಹನವಾದ ತತ್ವಗಳನ್ನೆಲ್ಲ ಸುಲಭವಾಗಿ ಸರ್ವ ಸಾಮಾನ್ಯರಿಗೆ ಅರ್ಥವಾಗುವಂತೆ ಹೇಳುವುದು. ಅದನ್ನು ಯಾರು ಬೇಕಾದರೂ ಓದಬಹುದು. ಉಪನಿಷತ್ತಿನಿಂದ ಆಗಬೇಕಾದ ಕೆಲಸವನ್ನು ಮಹಾಭಾರತ ಮತ್ತೂ ಚೆನ್ನಾಗಿ ಜನಸಾಮಾನ್ಯರಿಗೆ ಮಾಡುವುದು.

ಈ ಹಾಲನ್ನು ಕರೆಯುವವನು ಕೃಷ್ಣ. ಯಾರನ್ನು ಪೂರ್ಣಾವತಾರವೆಂದು ಕರೆಯುವೆವೋ ಅಂತಹ ಶ‍್ರೀಕೃಷ್ಣನೇ ಇಲ್ಲಿ ಉಪನಿಷತ್ತುಗಳಿಂದ ಎಲ್ಲರಿಗೂ ಏನು ಬೇಕೊ ಅದನ್ನು ಕೊಡುವನು. ಅವನಿಗೆ ಮನುಷ್ಯ ವಿಕಾಸದ ಏಣಿಯಲ್ಲಿ ಯಾವಯಾವ ಮೆಟ್ಟಿಲಿನಲ್ಲಿ ಇದ್ದಾನೆಂಬುದು ಚೆನ್ನಾಗಿ ತಿಳಿದಿರುವುದು. ದ್ವೈತ ಅದ್ವೈತ, ವಿಶಿಷ್ಟಾದ್ವೈತ ಮುಂತಾದ ತತ್ವಗಳ ಸಮನ್ವಯವೇ ಆ ಹಾಲು. ಭಕ್ತರು ಕರ್ಮಿಗಳು ಜ್ಞಾನಿಗಳು, ಅವರು ಯಾವ ಧರ್ಮಕ್ಕೆ ಸೇರಿದರೂ ಚಿಂತೆ ಇಲ್ಲ, ಅವರನ್ನೆಲ್ಲ ಅದು ಪೋಷಿಸುವುದು.

ಇಲ್ಲಿ ಕರುವೇ ಅರ್ಜುನ. ಹಾಲನ್ನು ಕರೆಯುವುದಕ್ಕೆ ಒಂದು ನಿಮಿತ್ತ ಇರಬೇಕು. ಅದೇ ಅರ್ಜುನನ ಸಂದೇಹ. ಅನುಮಾನ ಸಂದೇಹ ಇಲ್ಲದೇ ಇದ್ದರೆ ತಿಳಿದುಕೊಳ್ಳಲು ಆಸಕ್ತಿ ಇರುವುದಿಲ್ಲ. ಅನುಮಾನ ಸಂದೇಹ ಎಂಬ ಅಂಕುಶ ನಮ್ಮನ್ನು ತಿವಿಯುತ್ತಿದ್ದರೆ ಮಾತ್ರ ನಾವು ತಿಳಿದುಕೊಳ್ಳಲು ಮುಂದುವರಿ ಯುವುದು. ಅರ್ಜುನನು ಇಂತಹ ಒಂದು ಪರಿಸ್ಥಿತಿಯಲ್ಲಿ ಸಿಲುಕಿದ್ದಾನೆ. ಅದಕ್ಕಾಗಿ ಗೀತಾಮೃತವನ್ನು ಅವನಿಗೆ ಕೊಡಬೇಕಾಗಿದೆ. ಆದರೆ ಇಲ್ಲಿ ಅವನಿಗೆ ಮಾತ್ರ ಮೀಸಲಾಗಿಲ್ಲ ಅದು. ಭಗವಂತನು ಯಾವುದನ್ನು ಅವನಿಗೆ ನೀಡಿದನೊ ಅದು ಅಕ್ಷಯಪಾತ್ರೆಯಂತಿರುವ ಗೀತೆಯಲ್ಲಿ ಎಂದಿಗೂ ಕಡಮೆಯಾಗದೆ ಇನ್ನೂ ಉಳಿದಿದೆ. ಅದನ್ನು ಅನುಭವಿಸುವವರೇ ಸಜ್ಜನರು–ಎಂದರೆ ಪ್ರಪಂಚದಲ್ಲಿ ಅರ್ಜುನನಂತೆ ಸಂದೇಹಕ್ಕೆ ಸಿಕ್ಕಿ ದಾರಿ\-ಗಾಣದೆ ದಿಕ್ಕೆಟ್ಟವರು, ದಾರಿ ತೋರುವವನಿಗಾಗಿ ಕಾತರದಿಂದ ನಿಂತು ಬೇಡುತ್ತಿರುವರು. ಆಧ್ಯಾತ್ಮಿಕ ಪಿಪಾಸುಗಳೆಲ್ಲ ಅರ್ಜುನನ ಪರಿಸ್ಥಿತಿಯಲ್ಲಿ ಇರುವರು. ಶ‍್ರೀಕೃಷ್ಣನ ಮುಂದೆ ಅರ್ಜುನನು ಮಾತ್ರ ಅಲ್ಲ ಇರುವುದು. ಅರ್ಜುನನು ಅಲ್ಲಿ ನಮ್ಮೆಲ್ಲರ ಪ್ರತಿನಿಧಿಯಂತೆ ಇರುವನು. ಭಗವಂತ ಎಲ್ಲ ಮಾನವಕೋಟಿಗೆ ಇಲ್ಲಿ ಸಂದೇಶ ನೀಡುತ್ತಿರುವನು.

\begin{shloka}
ವಸುದೇವಸುತಂ ದೇವಂ ಕಂಸಚಾಣೂರಮರ್ದನಮ್~।\\ದೇವಕೀಪರಮಾನಂದಂ ಕೃಷ್ಣಂ ವಂದೇ ಜಗದ್ಗುರುಮ್\hfill॥ ೫~॥
\end{shloka}

\begin{artha}
ವಸುದೇವನ ಮಗನೂ ಕಂಸ ಚಾಣೂರರನ್ನು ಕೊಂದವನೂ, ದೇವಕಿಗೆ ಪರಮಾನಂದವನ್ನು ಉಂಟುಮಾಡಿ\-ದವನೂ, ಜಗದ್ಗುರುವೂ, ದೇವನೂ ಆದ ಕೃಷ್ಣನನ್ನು ನಮಿಸುತ್ತೇನೆ.
\end{artha}

ಈ ಶ್ಲೋಕದಲ್ಲಿ ಪುನಃ ಶ‍್ರೀಕೃಷ್ಣನನ್ನು ಕೊಂಡಾಡುವರು. ಅವನು ವಸುದೇವ ದೇವಕಿಯರಿಗೆ ಮಗ. ಅವರಿಗೆ ಪರಮಾನಂದವನ್ನು ಉಂಟುಮಾಡಿದವನು. ಒಬ್ಬ ಮಗುವಿನಿಂದ ಅಪ್ಪ ಅಮ್ಮ ಎನಿಸಿಕೊಂಡಾಗ ಆಗುವ ಆನಂದವನ್ನು ಹೆತ್ತವರೇ ಬಲ್ಲರು. ಆ ಮಗು ಸಾಧಾರಣ ಮಗುವಲ್ಲ. ಇಡೀ ಬ್ರಹ್ಮಾಂಡಕ್ಕೆ ಒಡೆಯನಾದವನು. ಅವನಿಗೆ ಜನ್ಮ ನೀಡುವುದು, ಅವನು ಮಾಡುವ ಲೀಲೆಗಳನ್ನು ನೋಡುವುದು ಹಲವು ಜನ್ಮಗಳ ತಪಸ್ಸಿನ ಮತ್ತು ಪುಣ್ಯದ ಪ್ರಭಾವ. ಶ‍್ರೀಕೃಷ್ಣನೇ ಧರ್ಮಕಂಟಕರಾದ ಕಂಸ ಚಾಣೂರರು ಮುಂತಾದವರನ್ನೆಲ್ಲ ಕೊಂದವನು. ಧರ್ಮಕಂಟಕರನ್ನು ನಾಶಮಾಡುವಾಗ ಶ‍್ರೀಕೃಷ್ಣ ಇವರು ತನ್ನವರು ಎಂಬ ದಯಾದಾಕ್ಷಿಣ್ಯವನ್ನೇ ತರುವುದಿಲ್ಲ. ತನ್ನ ಸೋದರಮಾವನಾದರೂ ಕಂಸನನ್ನು ಕೊಲ್ಲುವನು. ಅವನು ಅವತಾರವೆತ್ತಿದ್ದೇ ಧರ್ಮಸಂಸ್ಥಾಪನೆಗೆ. ಅದು ಆಗಬೇಕಾದರೆ ಅಧರ್ಮದ ಕಳೆಯನ್ನು ನಿರ್ದಾಕ್ಷಿಣ್ಯವಾಗಿ ಕೀಳಬೇಕು. ಶ‍್ರೀಕೃಷ್ಣ ಈ ಕೆಲಸವನ್ನು ಮಾಡುವಾಗ ಸ್ವಲ್ಪವೂ ಅಂಜುವುದಿಲ್ಲ, ಅಳುಕುವುದಿಲ್ಲ. ಇವನು ಜಗದ್ಗುರು, ಇಡೀ ಬ್ರಹ್ಮಾಂಡಕ್ಕೆ ಗುರು. ಜನ ಯಾವ ಹೆಸರಿನಲ್ಲಿ ಯಾವ ಭಾವದ ಮೂಲಕ ಆರಾಧಿಸಲಿ, ಅದೆಲ್ಲ ಇವನಿಗೇ ಸೇರುವುದು. ಎಲ್ಲಾ ಧರ್ಮಗಳೂ ಸೂತ್ರದಲ್ಲಿ ಮಣಿ ಹೇಗೆ ಪೋಣಿಸಲ್ಪಟ್ಟಿದೆಯೋ ಹಾಗೆ ಶ‍್ರೀಕೃಷ್ಣನಲ್ಲಿ ಪೋಣಿಸಲ್ಪಟ್ಟಿವೆ. ಇಂತಹ ದೇವನೂ ಪವಿತ್ರೋತ್ತಮನೂ ಆದವನನ್ನು ನಮಸ್ಕರಿಸುವರು.

\begin{shloka}
ಭೀಷ್ಮದ್ರೋಣತಟಾ ಜಯದ್ರಥಜಲಾ ಗಾಂಧಾರನೀಲೋತ್ಪಲಾ\\ಶಲ್ಯಗ್ರಾಹವತೀ ಕೃಪೇಣ ವಹನೀ ಕರ್ಣೇನ ವೇಲಾಕುಲಾ~।\\ಅಶ್ವತ್ಥಾಮವಿಕರ್ಣಘೋರಮಕರಾ ದುರ್ಯೋಧನಾವರ್ತಿನೀ\\ಸೋತ್ತೀರ್ಣಾ ಖಲು ಪಾಂಡವೈ ರಣನದೀ ಕೈವರ್ತಕಃ ಕೇಶವಃ \hfill॥ ೬~॥
\end{shloka}

\begin{artha}
ಭೀಷ್ಮದ್ರೋಣರೇ ದಡಗಳು, ಮಧ್ಯದಲ್ಲಿರುವ ನೀರೇ ಜಯದ್ರಥ, ಗಾಂಧಾರ ರಾಜನೇ ಮಧ್ಯದಲ್ಲಿ ನಿಮಿರಿ ನಿಂತಿರುವ ಕರಿಯ ಬಂಡೆ, ಶಲ್ಯನೇ ಮೊಸಳೆ, ಕೃಪನೇ ಪ್ರವಾಹ, ಕರ್ಣನೇ ಭಯಾನಕವಾದ ಅಪ್ಪಳಿಸುವ ಅಲೆ, ಅಶ್ವತ್ಥಾಮ ವಿಕರ್ಣರು ಭಯಾನಕವಾದ ಮಕರ ಮತ್ತು ದುರ್ಯೋಧನನೇ ಸುಳಿ. ಇಂತಹ ರಣನದಿಯನ್ನು ಪಾಂಡವರು ದಾಟಿದರು. ಇದನ್ನು ದಾಟಿಸಿದ ಅಂಬಿಗನೇ ಶ‍್ರೀಕೃಷ್ಣ.
\end{artha}

ಇಲ್ಲಿ ಕೌರವ ಸೇನೆಯನ್ನೇ ಒಂದು ರಣನದಿಯನ್ನಾಗಿ ಮಾಡಿರುವ ಉಪಮಾನವನ್ನು ಕೊಡುವರು. ಚಿತ್ತಾಕರ್ಷಕವಾಗಿದೆ ಇಲ್ಲಿ ಬರುವ ಚಿತ್ರ. ಈ ಮಹಾ ರಣನದಿ ಹನ್ನೊಂದು ಅಕ್ಷೋಹಿಣಿ ಸೈನ್ಯವನ್ನು ಕೂಡಿಕೊಂಡು ಭೋರ್ಗರೆದು ಹರಿಯುತ್ತಿದೆ. ಇದಕ್ಕೆ ಎರಡು ದಡಗಳು ಬೇಕಾಗಿದೆ. ಆ ದಡಗಳೇ ಕುರುಪಿತಾಮಹನಾದ ಭೀಷ್ಮ–ಇಚ್ಛಾಮರಣಿ, ಅಷ್ಟವಸುಗಳಲ್ಲಿ ಒಬ್ಬ, ಕಠೋರ ಬ್ರಹ್ಮಚಾರಿ. ಸಮರದಲ್ಲಿ ಇದುವರೆಗೂ ಇವನನ್ನು ಗೆದ್ದವರಿಲ್ಲ. ಮತ್ತೊಂದು ದಡವೇ ದ್ರೋಣ. ಆಗಿನ ಕಾಲದ ಹೆಸರಾಂತ ಧನುರ್ವಿದ್ಯಾಚಾರ್ಯ, ಕೌರವ ಪಾಂಡವರಿಗೆ ವಿದ್ಯೆಯನ್ನು ದಾನ\-ಮಾಡಿದವನು. ಆ ದಡಗಳ ಮಧ್ಯದಲ್ಲಿ ಹರಿಯುವ ಸೇನೆಗೆ ಒಡೆಯನಾಗಿರುವವನೇ ಜಯದ್ರಥ. ಆ ರಣನದಿಯಲ್ಲಿ ಮಧ್ಯೆ ಮೇಲೆದ್ದು ಬಂದವರನ್ನು ಅಪ್ಪಳಿಸಿ ಬಡಿಯುವುದಕ್ಕೆ ಇರುವ ಬಂಡೆಯೇ ಗಾಂಧಾರರಾಜ. ಶಲ್ಯ ಅಶ್ವತ್ಥಾಮರುಗಳೇ ನೀರಿನೊಳಗೆ ಇರುವ ಮೊಸಳೆಗಳು. ದೋಣಿಗಳನ್ನು ಮುಳುಗಿಸಲು ಮೇಲೆ ದ್ದಿರುವ ಭೀಕರವಾದ ಅಲೆಯೇ ಕರ್ಣ. ಒಳಗೆ ಸೆಳೆದುಕೊಳ್ಳುವುದಕ್ಕೆ ತಿರುಗುತ್ತಿರುವ ಸುಳಿಯೇ ದುರ್ಯೋಧನ. ಒಂದರಿಂದ ತಪ್ಪಿಸಿಕೊಂಡರೆ ಅದಕ್ಕಿಂತ ಬಲವಾಗಿರುವ ಅಪಾಯ ಕಾದಿದೆ ನದಿಯನ್ನು ದಾಟುವಾಗ. ಇಂತಹ ರಣನದಿಯನ್ನು ಪಾಂಡವರು ದಾಟಬೇಕಾಗಿದೆ. ಇವರಿಗೆ ಇರುವ ಸೈನ್ಯವಾದರೋ ಏಳು ಅಕ್ಷೋಹಿಣಿಯ ಪುಟ್ಟ ದೋಣಿ. ಆದರೆ ದೋಣಿ ಪುಟ್ಟದಾದರೇನು ಇದನ್ನು ನಡೆಸುವವನು ಕುಶಲಿಗಳಲ್ಲಿ ಕುಶಲಿಯಾದ ಶ‍್ರೀಕೃಷ್ಣ. ಎಲ್ಲಾ ಅಪಾಯಗಳಿಂದಲೂ ಪಾಂಡವರನ್ನು ನಿವಾರಿಸಿ ಕಡೆಹಾಯಿಸುವನು.

\begin{shloka}
ಪಾರಾಶರ್ಯವಚಃಸರೋಜಮಮಲಂ ಗೀತಾರ್ಥಗಂಧೋತ್ಕಟಂ\\ನಾನಾಖ್ಯಾನಕಕೇಸರಂ ಹರಿಕಥಾಸಂಬೋಧನಾಬೋಧಿತಮ್~।\\ಲೋಕೇಸಜ್ಜನಷಟ್ಪದೈರಹರಹಃ ಪೇಪೀಯಮಾನಂ ಮುದಾ\\ಭೂಯಾದ್ಭಾರತಪಂಕಜಂ ಕಲಿಮಲಪ್ರಧ್ವಂಸಿ ನಃ ಶ್ರೇಯಸೇ\hfill॥ ೭~॥
\end{shloka}

\begin{artha}
ಮಹಾಭಾರತ ಎಂಬ ಕಮಲ ಪರಾಶರರ ಮಗನಾದ ವ್ಯಾಸರ ಮಾತಿನ ಸರೋವರದಲ್ಲಿ ಹುಟ್ಟಿದೆ. ಆ ಕಮಲ ನಾನಾ ಕಥೆಗಳೆಂಬ ಕೇಸರಗಳಿಂದ ಕೂಡಿದೆ. ಹರಿಕಥೆ ಎಂಬ ಬೋಧನೆಯಿಂದ ಅರಳಿದೆ. ಸಜ್ಜನರು ಎಂಬ ಜೇನುನೊಣಗಳೇ ಸಂತೋಷದಿಂದ ಇದನ್ನು ಪ್ರತಿದಿನವೂ ಪಾನಮಾಡುತ್ತಿರುವರು. ಕಲಿ ಕಲ್ಮಷವನ್ನು ನಾಶಮಾಡುವ, ಗೀತಾರ್ಥದ ಪರಿಮಳದಿಂದ ಕೂಡಿದ ಈ ಕಮಲ ನಮಗೆ ಶ್ರೇಯಸ್ಸನ್ನು ಮಾಡಲಿ.
\end{artha}

ಇಲ್ಲಿ ಮತ್ತೊಂದು ಉಪಮಾನವನ್ನು ಕೊಡುವರು. ವ್ಯಾಸರ ವಾಣಿಯೇ ಒಂದು ಸರೋವರ, ಅದರಲ್ಲಿ ವಿಕಸಿತವಾದ ಕಮಲವೇ ಮಹಾಭಾರತ. ಆ ಕಮಲಕ್ಕೆ ಇರುವ ಕೇಸರಗಳೇ ಮಹಾಭಾರತದಲ್ಲಿ ಬರುವ ಕಥೋಪಕಥೆಗಳು. ಹರಿಯ ಕಥೆಯಿಂದಲೆ ಅದು ವಿಕಾಸವಾಗುವುದು. ಕಮಲ ಅರಳಬೇಕಾದರೆ ಮೇಲೆ ಸೂರ್ಯ ಮೂಡಬೇಕು, ಹಾಗೆಯೇ ಭಗವಂತನ ಕಥೆಯಿಂದಲೇ ಇದು ಅರಳುವುದು. ಇದು ಬರೀ ಕಥೆಯಲ್ಲ, ಭಗವದ್​ವಾಣಿಯನ್ನು ನಿರೂಪಿಸುವುದು. ಸತ್ಯವೊಂದೇ ಜಯಿಸುವುದು, ಅಸತ್ಯವಲ್ಲ ಎಂಬ ಸನಾತನ ತತ್ವವನ್ನು ಸಾರುವುದಕ್ಕಾಗಿಯೇ ಈ ಕಥೆ ಹುಟ್ಟಿರುವುದು. ಇದರಲ್ಲಿನ ಮಕರಂದವನ್ನು ಭೋಗಿಸುವ ದುಂಬಿಗಳೇ ಸಜ್ಜನರು. ದಿನದಿನವೂ ಇದನ್ನು ಮುತ್ತಿರುವುವು. ಈ ಮಧುಪಾನಮಾಡಿ ಆನಂದವನ್ನು ಪಡೆಯುವರು. ಇದರಿಂದ ಬರುವ ಗೀತಾರ್ಥವೆಂಬ ಪರಿಮಳ ಪಾಪದ ದುರ್ವಾಸನೆಯನ್ನು ಹೊರದೂಡುವುದು. ಇಂತಹ ಭಾರತ ಕಮಲ ನಮಗೆ ಮಂಗಳವನ್ನು ಮಾಡಲಿ.

\begin{shloka}
ಮೂಕಂ ಕರೋತಿ ವಾಚಾಲಂ ಪಂಗುಂ ಲಂಘಯತೇ ಗಿರಿಮ್~।\\ಯತ್ಕೃಪಾ ತಮಹಂ ವಂದೇ ಪರಮಾನಂದಮಾಧವಮ್\hfill॥ ೮~॥
\end{shloka}

\begin{artha}
ಯಾರ ಕೃಪೆ ಮೂಕನನ್ನು ವಾಚಾಳಿಯನ್ನಾಗಿ ಮಾಡುವುದೊ, ಕುಂಟನನ್ನು ಪರ್ವತವನ್ನು ಏರುವಂತೆ ಮಾಡುವುದೊ, ಅಂತಹ ಪರಮಾನಂದ ಸ್ವರೂಪನಾದ ಮಾಧವನನ್ನು ವಂದಿಸುತ್ತೇನೆ.
\end{artha}

ಭಗವಂತನನ್ನು ನೆಚ್ಚಿದರೆ ಅಸಾಧ್ಯವನ್ನು ಸಾಧ್ಯವನ್ನಾಗಿ ಮಾಡುವುದು ಎಂಬುದನ್ನು ತೋರುವುದು. ನಾವು ಸಾಧಿಸಬೇಕಾಗಿರುವುದೋ ದೊಡ್ಡದು. ಅದಕ್ಕೆ ನಮಗೆ ಇರುವ ಯೋಗ್ಯತೆ ಅತ್ಯಲ್ಪ. ಮಗು ಚಂದ್ರನಿಗೆ ಕೈಯೊಡ್ಡುವಂತಿದೆ ನಮ್ಮ ಪರಿಸ್ಥಿತಿ. ಆದರೆ ಭಗವಂತನಲ್ಲಿ ಶರಣಾದ ಭಕ್ತ ಕುಗ್ಗುವುದಿಲ್ಲ. ತನ್ನನ್ನು ಮೀರಿದ ಒಂದು ಶಕ್ತಿಯ ಅರಿವು ಅವನಿಗಿದೆ. ಜಗವನ್ನೆಲ್ಲ ಆಳುತ್ತಿದೆ ಅದು. ನಮ್ಮನ್ನು ಎಂದಿಗೂ ಕೈಬಿಡುವುದಿಲ್ಲ. ನಮ್ಮನ್ನು ಕೈಹಿಡಿಸಿ ನಡಸುವುದು ಅದು. ನಮಗೇ ಅನೇಕ ವೇಳೆ ಅಚ್ಚರಿಯಾಗಬಹುದು ನಾವು ಇಷ್ಟೊಂದನ್ನು ಹೇಗೆ ಸಾಧಿಸಿದೆವು ಎಂದು. ಲೋಕವ್ಯವಹಾರದ ದೃಷ್ಟಿಯಿಂದ ಒಂದು ಅಸಾಧ್ಯ. ಆದರೆ ಭಗವಂತನಿಗೆ ಅಸಾಧ್ಯವಾದುದು ಯಾವುದೂ ಇಲ್ಲ. ಎಲ್ಲ ನಿಯಮಗಳೂ ತಲೆ ಬಾಗುವುವು ಅವನೆದುರಿಗೆ. ಭಗವಂತನನ್ನು ನೆಚ್ಚಿದವನ ಸಹಾಯಕ್ಕೆ ಅವನ ಶಕ್ತಿಯೇ ಬರುವುದು. ಈ ಬ್ರಹ್ಮಾಂಡಕ್ಕೆ ಒಡೆಯನಾದವನು ಆನಂದ ಸ್ವರೂಪಿ. ಉಪನಿಷತ್ತು, ಈ ಜಗತ್ತು ಆನಂದದಿಂದ ಬಂತು, ಆನಂದದಲ್ಲಿದೆ, ಆನಂದಕ್ಕೆ ಹೋಗುವುದು ಎಂದು ಸಾರುವುದು. ಈ ಸೃಷ್ಟಿ ಸ್ಥಿತಿ ಲಯವೆಲ್ಲ ಆಗುವುದು ಆನಂದಕ್ಕಾಗಿ. ನಾವಿರುವುದೇ ಅವನ ಆನಂದಕ್ಕಾಗಿ, ಅಂತಹ ಆನಂದ ಸ್ವರೂಪಿಯನ್ನು ವಂದಿಸುವೆನು.

\begin{shloka}
ಯಂ ಬ್ರಹ್ಮಾ ವರುಣೇಂದ್ರರುದ್ರಮರುತಃ ಸ್ತುನ್ವಂತಿ ದಿವ್ಯೈಽಃ ಸ್ತವೈ-\\ವೆR|ದೈಃ ಸಾಂಗ ಪದಕ್ರಮೋಪನಿಷದೈರ್ಗಾಯಂತಿ ಯಂ ಸಾಮಗಾಃ~।\\ಧ್ಯಾನಾವಸ್ಥಿತತದ್ಗತೇನ ಮನಸಾ ಪಶ್ಯಂತಿ ಯಂ ಯೋಗಿನೋ \\ಯಸ್ಯಾಂತಂ ನ ವಿದುಃ ಸುರಾಸುರಗಣಾ ದೇವಾಯ ತಸ್ಮೈ ನಮಃ \hfill॥ ೯~॥
\end{shloka}

\begin{artha}
ಯಾರನ್ನು ಬ್ರಹ್ಮ ವರುಣ ಇಂದ್ರ ರುದ್ರ ಮರುದ್ಗಣಗಳು ದಿವ್ಯಸ್ತವಗಳಿಂದ ಸ್ತುತಿಸುತ್ತವೆಯೋ, ಯಾರನ್ನು ಸಾಮಗರು ಅಂಗ ಪದಕ್ರಮ ಉಪನಿಷತ್ತು ಇವುಗಳಿಂದ ಕೂಡಿದ ವೇದಗಳಿಂದ ಗಾನ ಮಾಡುತ್ತಾರೆಯೋ, ಯಾರನ್ನು ಯೋಗಿಗಳು ಧ್ಯಾನಿಸುವರೊ, ಯಾರ ಅಂತ್ಯವನ್ನು ಸುರಾಸುರ ಗಣಗಳು ತಿಳಿಯಲಾರವೊ ಅಂತಹ ದೇವನಿಗೆ ನಮಸ್ಕಾರ.
\end{artha}

ಇಲ್ಲಿ ದೇವರನ್ನು ಬ್ರಹ್ಮ ವರುಣ ಇಂದ್ರ ರುದ್ರ ಮುಂತಾದವರು ಸ್ತುತಿಸುವರು. ಭಗವಂತನನ್ನು ಸ್ತುತಿಸಬೇಕಾದರೆ ಅವನನ್ನು ತಿಳಿದುಕೊಳ್ಳುವ ಯೋಗ್ಯತೆ ಇರಬೇಕು, ಅವನ ಹತ್ತಿರ ಹೋಗಿರಬೇಕು. ಅಂತಹ ವ್ಯಕ್ತಿಗಳಿಗೆ ಮಾತ್ರ ಸಾಧ್ಯ. ಜೀವನದ ವಿಕಾಸದ ಏಣಿಯಲ್ಲಿ ತುತ್ತ ತುದಿಯನ್ನು ಮುಟ್ಟಿದಂತಹ ವ್ಯಕ್ತಿಗಳು ಸ್ತುತಿಸುತ್ತಿರುವರು. ಇದು ಬುದ್ಧಿಪ್ರಧಾನವಾದುದು. ಅನಂತರ\break ಅವನನ್ನು ಗಾನ ಮಾಡುವವರು ಬರುವರು. ಗಾನ ಭಾವದಿಂದ ಹೊರಹೊಮ್ಮುವುದು. ನಮ್ಮ ಹೃದಯ ತಂತಿಯನ್ನು ಭಗವಂತನ ಬೆರಳು ಮಿಡಿದಾಗ ಕಾವ್ಯ ಗಾನದಂತೆ ಹೊರಹೊಮ್ಮುವುದು. ತತ್ವವೂ ಇಲ್ಲಿ ಕಾವ್ಯವೇಷದಲ್ಲಿ ಬರುವುದು. ಉಪನಿಷತ್ತುಗಳೇ ಉದಾಹರಣೆ. ಇಲ್ಲಿ ಬರುವುದು ಒಂದು ದರ್ಶನ, ಅನುಭೂತಿ. ಇದು ಕಾವ್ಯಮಯವಾಗಿ ಮೂಡಿಬಂದಿದೆ. ಅನಂತರ ತತ್ವ ಬಂದು ಈ ಅನುಭವಗಳನ್ನು ತೆಗೆದುಕೊಂಡು ಇದನ್ನು ಒಂದು ಸಿದ್ಧಾಂತವನ್ನಾಗಿ ಮಾಡುವುದು.

ಮೂರನೆಯದೇ ಯೋಗಿಯ ಅನುಭವ. ಚಿತ್ತವನ್ನು ನಿರೋಧಿಸಿ ಅಂತರ್ಮುಖ ಮಾಡಿ ಪರಮಾತ್ಮನ ಮೇಲೆ ಕೇಂದ್ರೀಕರಿಸುವನು. ಅದರಲ್ಲಿ ತನ್ಮಯನಾಗುವನು. ತನ್ನ ವ್ಯಕ್ತಿತ್ವವನ್ನು ಅದರಲ್ಲಿ ಮುಳುಗಿಸುವನು.

ಆ ಭಗವಂತನ ಮಿತಿಯನ್ನು ಸುರ ಅಸುರರು ಯಾರೇ ಆಗಲಿ ತಿಳಿದಿಲ್ಲ. ಅದನ್ನು ತಿಳಿದುಕೊಳ್ಳುವುದಕ್ಕೆ ಹೋದರೆ ನಾನೆಂಬುದೇ ಪತ್ತೆ ಇರುವುದಿಲ್ಲ. ಶ‍್ರೀರಾಮಕೃಷ್ಣರು ಹೀಗೆ ಹೇಳುತ್ತಿದ್ದರು. ಇದು ಒಂದು ಉಪ್ಪಿನ ಗೊಂಬೆ ಸಾಗರದ ಆಳವನ್ನು ನೋಡಲು ಹೋದಂತಿದೆ. ಹೋದರೆ ಹಿಂತಿರುಗಿ ಬರುವಂತೆಯೇ ಇಲ್ಲ. ಇನ್ನು ಅದು ಎಷ್ಟು ಆಳ ಎಂದು ಹೇಳುವವರು ಯಾರು? ಹಾಗೆಯೇ ನಮ್ಮ ಸಾಂತಮತಿಗೆ ಅವನ ಆಳವನ್ನು ಪತ್ತೆ ಹಚ್ಚಲು ಅಸಾಧ್ಯ.

ಒಮ್ಮೆ ಒಬ್ಬ ಘನ ವಿದ್ವಾಂಸನಿದ್ದ. ಅವನು ಹಲವು ಶಾಸ್ತ್ರಗಳನ್ನು ಓದಿದ್ದ. ದೇವರ ವಿಷಯವಾಗಿ ಎಲ್ಲವನ್ನು ತಿಳಿದೊಕಂಡಿರುವೆ ಎಂದು ಅಹಂಕಾರಿ ಆಗಿದ್ದ. ದೇವರು ಪಂಡಿತನಿಗೆ ಅವನ ಪಾಂಡಿತ್ಯದ ಮಿತಿಯನ್ನು ತಿಳಿಸುವುದಕ್ಕಾಗಿ ಒಂದು ಉಪಾಯ ಹೂಡಿದ. ಆ ಪಂಡಿತ ಪ್ರತಿದಿನವೂ ಸಮುದ್ರ ತೀರದಲ್ಲಿ ಸಾಯಂಕಾಲ ಗಾಳಿಸೇವನೆಗೆ ಹೋಗುತ್ತಿದ್ದ. ಆ ತೀರದ ಸಮೀಪದಲ್ಲಿ ಒಂದು ಮಗು ಮರಳನ್ನು ಬಗೆದು ಸಮುದ್ರದ ನೀರನ್ನು ಕೈಯಲ್ಲಿ ಹಿಡಿದುಕೊಂಡು ಅದರಲ್ಲಿ ಹಾಕುತ್ತಿತ್ತು. ಪಂಡಿತ ಈ ಮಗುವಿನ ಆಟವನ್ನು ನೋಡಿ ಅಚ್ಚರಿಗೊಂಡು “ನೀನು ಏನು ಮಾಡುತ್ತಿರುವೆ ಮಗು!” ಎಂದು ಕೇಳಿದ. ಆ ಮಗು “ಸಮುದ್ರವನ್ನು ಈ ಹಳ್ಳದಲ್ಲಿ ಖಾಲಿ ಮಾಡಬೇಕೆಂದಿರುವೆ” ಎಂದಿತು. ಆ ಪಂಡಿತ “ಇಷ್ಟೊಂದು ದೊಡ್ಡ ಸಾಗರವನ್ನು ನೀನು ನಿನ್ನ ಪುಟ್ಟ ಕೈಗಳಲ್ಲಿ ತೋಡಿರುವ ಈ ಗುಂಡಿಯಲ್ಲಿ ಖಾಲಿ ಮಾಡಬಲ್ಲೆಯಾ?” ಎಂದು ಕೇಳಿದ. ಅದಕ್ಕೆ ಆ ಮಗು, “ನೀನು ನಿನ್ನ ಚಿಕ್ಕ ತಲೆಯಲ್ಲಿ ದೇವರ ವಿಷಯವನ್ನೆಲ್ಲ ಅಡಗಿಸಲು ಸಾಧ್ಯ ಎಂದು ಭಾವಿಸಿದರೆ, ನಾನು ಏತಕ್ಕೆ ಇದನ್ನು ಮಾಡಲು ಸಾಧ್ಯವಿಲ್ಲ?” ಎಂದು ಹೇಳಿ ಅದೃಶ್ಯವಾಯಿತು. ಆ ಪಂಡಿತನಿಗೆ ಜ್ಞಾನೋದಯ\-ವಾಯಿತು. ನಮ್ಮ ಬುದ್ಧಿಯ ಗಜಕಡ್ಡಿಯಿಂದ ದೇವರ ಆಳವನ್ನು ಯಾರೂ ತಿಳಿಯಲಾರರು.

ಆಳವನ್ನು ತಿಳಿಯುವ ಆವಶ್ಯಕತೆಯಾದರೂ ಏನಿದೆ? ಶ‍್ರೀರಾಮಕೃಷ್ಣರು ಮತ್ತೊಂದು ಉಪಮಾನದ ಮೂಲಕ ಇದನ್ನು ವಿಶದಪಡಿಸುವರು. ಬಾಯಾರಿ ದಾರಿಯಲ್ಲಿ ಬರುತ್ತಿರುವವನಿಗೆ ಒಂದು ಬಾವಿ ಕಾಣುವುದು. ಯಾರೋ ನೀರು ಸೇದುತ್ತಿದ್ದರು. ತನ್ನ ದಾಹವನ್ನು ಇಂಗಿಸಲು ಅವರಿಂದ ಒಂದು ಚೆಂಬಿನಲ್ಲಿ ಸ್ವಲ್ಪ ನೀರನ್ನು ತೆಗೆದುಕೊಂಡು ಬಾಯಾರಿಕೆಯಿಂದ ಪಾರಾಗಿ ಹೋಗುವನು. ಬಾವಿಯಲ್ಲಿ ಎಷ್ಟು ಸಾವಿರ ಗ್ಯಾಲನ್ನು ನೀರಿದೆ ಎಂಬ ಪ್ರಶ್ನೆಯನ್ನು ಕೇಳುವ ಪ್ರಸಂಗವೇ ಬರುವುದಿಲ್ಲ ಅವನಿಗೆ.

ಹೀಗೆಯೇ ಭಗವಂತನನ್ನು ಆಪಾದಮಸ್ತಕ ಪರಿಯಂತರ ನಾವು ತಿಳಿದುಕೊಳ್ಳಬೇಕಾಗಿಲ್ಲ. ಹಾಗೆ ಪ್ರಯತ್ನಪಟ್ಟರೆ ಯಾರೂ ಈ ಸಾಹಸದಲ್ಲಿ ಜಯಶೀಲರಾಗಲಾರರು. ಅವನು ನಮಗಿಂತ ಆಳ. ಅವನ ಸ್ವಲ್ಪ ಕೃಪೆ ಸಾಕು ನಮ್ಮನ್ನು ಸಂಸಾರ ಬಂಧನದಿಂದ ಪಾರು ಮಾಡಲು. ಇಂತಹ ಭಗವಂತನಿಗೆ ನಮಸ್ಕರಿಸುವುದರೊಂದಿಗೆ ಗೀತೆಯ ಅಧ್ಯಯನ ಪ್ರಾರಂಭವಾಗುವುದು.

