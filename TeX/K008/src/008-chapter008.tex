
\chapter{ಅಕ್ಷರಬ್ರಹ್ಮಯೋಗ}

\begin{verse}
ಕಿಂ ತದ್ಬ್ರಹ್ಮ ಕಿಮಧ್ಯಾತ್ಮಂ ಕಿಂ ಕರ್ಮ ಪುರುಷೋತ್ತಮ~।\\ಅಧಿಭೂತಂ ಚ ಕಿಂ ಪ್ರೋಕ್ತಮಧಿದೈವಂ ಕಿಮುಚ್ಯತೇ \versenum{॥ ೧~॥}
\end{verse}

\begin{verse}
ಅಧಿಯಜ್ಞಃ ಕಥಂ ಕೋಽತ್ರ ದೇಹೇಽಸ್ಮಿನ್ ಮಧುಸೂದನ~।\\ಪ್ರಯಾಣಕಾಲೇ ಚ ಕಥಂ ಜ್ಞೇಯೋಽಸಿ ನಿಯತಾತ್ಮಭಿಃ \versenum{॥ ೨~॥}
\end{verse}

ಅರ್ಜುನನು ಶ‍್ರೀಕೃಷ್ಣನನ್ನು ಹೀಗೆ ಕೇಳುತ್ತಾನೆ:

{\small ಪುರುಷೋತ್ತಮ! ಈ ಪರಬ್ರಹ್ಮದ ಸ್ವರೂಪ ಎಂತಹುದು? ಅಧ್ಯಾತ್ಮ ಎಂದರೇನು? ಕರ್ಮ ಎಂದರೇನು? ಅಧಿಭೂತ ಯಾವುದು? ಅಧಿದೈವ ಯಾವುದು?}

{\small ಮಧುಸೂದನ! ಈ ದೇಹದಲ್ಲಿ ಅಧಿಯಜ್ಞ ಎಂದರೇನು? ಅದು ಹೇಗಿದೆ? ಮರಣಸಮಯದಲ್ಲಿ ಯೋಗಿಗಳು ನಿನ್ನನ್ನು ಹೇಗೆ ಅರಿತುಕೊಳ್ಳುತ್ತಾರೆ?}

ಈ ಅಧ್ಯಾಯದಲ್ಲಿ ಈಶ್ವರನ ತತ್ತ್ವವನ್ನು ವಿವರಿಸುತ್ತಾನೆ. ಅರ್ಜುನ ಶ‍್ರೀಕೃಷ್ಣನಿಗೆ ಹಲವು ಪ್ರಶ್ನೆಗಳನ್ನು ಹಾಕುತ್ತಾನೆ. ನಮ್ಮ ಪರವಾಗಿಯೇ ಅವನು ಈ ಪ್ರಶ್ನೆಗಳನ್ನು ಹಾಕುತ್ತಿರುವಂತಿದೆ. ಅರ್ಜುನ ಒಂದು ದೃಷ್ಟಿಯಿಂದ ಮಾನವಕೋಟಿಯ ಪ್ರತಿನಿಧಿ. ಅವನು ಹಾಕುವ ಪ್ರಶ್ನೆಗಳಿಂದ ಅದಕ್ಕೆ ಸಂಬಂಧಪಟ್ಟ ವಿಷಯಗಳು ನಮಗೆ ಹೆಚ್ಚು ಅರ್ಥವಾಗುತ್ತವೆ. ಗುರು, ಪ್ರಶ್ನೆ ಕೇಳದೆ ಇದ್ದರೆ ಸಾಮಾನ್ಯವಾಗಿ ಹೇಳುತ್ತಾನೆ. ಆದರೆ ನಮಗೆ ಯಾವುದು ಬೇಕೋ ಅದನ್ನು ಹೇಳಬೇಕಾದರೆ ನಾವು ಅದನ್ನು ಕೇಳಬೇಕು. ಆಗ ಮಾತ್ರ ನಮ್ಮ ಮನಸ್ಸಿನ ಸಮಸ್ಯೆಗಳು ಅವನಿಗೆ ಅರ್ಥವಾಗಿ ಅದಕ್ಕೆ ತಕ್ಕ ಉತ್ತರವನ್ನು ಕೊಡುತ್ತಾನೆ. ಇಲ್ಲಿ ಬರುವ ತತ್ತ್ವಬೋಧನೆ ಶ‍್ರೀಕೃಷ್ಣಾರ್ಜುನರ ಸಂವಾದರೂಪದ್ದು. ಯಾವುದೋ ಒಂದು ಸಿದ್ಧಾಂತವನ್ನು ಮೆಟ್ಟಲು ಮೆಟ್ಟಲಾಗಿ ವಿವರಿಸಿಕೊಂಡು ಹೋಗುವ ಪ್ರಾಧ್ಯಾ ಪಕನಂತೆ ಶ‍್ರೀಕೃಷ್ಣ ಇಲ್ಲಿ ವಿವರಿಸುವುದಿಲ್ಲ. ತನಗೆ ಗೊತ್ತಿರುವುದನ್ನೆಲ್ಲಾ ಹೇಳಲು ಹೋಗುವುದಿಲ್ಲ ಶ‍್ರೀಕೃಷ್ಣ. ಅರ್ಜುನನಿಗೆ ಏನು ಬೇಕು, ಎಷ್ಟು ಬೇಕು, ಅದನ್ನು ಹೇಗೆ ಹೇಳಿದರೆ ಅರ್ಜುನನ ಮನಸ್ಸಿಗೆ ನಾಟುತ್ತದೆ, ಆ ಭಾಷೆಯಲ್ಲಿ ಹೇಳುವನು.

ಮೊದಲನೆಯದಾಗಿ, ಆ ಪರಬ್ರಹ್ಮನ ಸ್ವರೂಪ ಎಂತಹುದು ಎಂದು ಕೇಳುತ್ತಾನೆ. ಆ ಬ್ರಹ್ಮ ವಾದರೊ ಅನಂತ ಅಖಂಡ ಸಚ್ಚಿದಾನಂದ ಸ್ವರೂಪ. ಈ ಪ್ರಪಂಚ ಮತ್ತು ಜೀವರಾಶಿಗಳೆಲ್ಲಾ ಅದರಲ್ಲಿ ಹಾಸು ಹೊಕ್ಕಾಗಿದೆ. ಇದರ ವಿಷಯವನ್ನು ಇನ್ನೂ ಹೆಚ್ಚು ಕೇಳ ಬಯಸುವನು. ‘ಅಧ್ಯಾತ್ಮ’ ಎಂದರೆ ಈ ಜೀವಿಯ ಸ್ವರೂಪ ಎಂತಹುದು; ಅವನು ಇಲ್ಲಿ ಹೇಗಿದ್ದಾನೆ, ಈ ದೇಶ ಕಾಲ ನಿಮಿತ್ತದ ಪ್ರಪಂಚದಲ್ಲಿ ಹಲವು ದೇಹಗಳನ್ನು ಧರಿಸಿರುವ ಜೀವಿ ಯಾವ ರೀತಿಯಲ್ಲಿದ್ದಾನೆ. ಕರ್ಮ, ಅದು ನಮ್ಮನ್ನು ಕಟ್ಟಿಹಾಕುವುದು, ಬಿಡಿಸುವುದು–ಈ ಸ್ವಭಾವಗಳೆಲ್ಲಾ ಇದೆಯಲ್ಲಾ ಇದನ್ನು ಸಾಂಗವಾಗಿ ಅರಿಯಬಯಸುತ್ತಾನೆ. ‘ಅಧಿಭೂತ’–ಈ ಪಂಚಭೂತಗಳಿಂದ ಆದ ಪ್ರಪಂಚ, ಇದರ ಸ್ವರೂಪ ಎಂತಹುದು; ‘ಅಧಿದೈವ’ ಎಂದರೆ ಹಲವಾರು ದೇಹಗಳಲ್ಲೂ ಜೀವಧಾರಿ ಯಾವ ರೀತಿ ಇದ್ದಾನೆ?\\ಇವನ್ನೆಲ್ಲಾ ಅರಿಯಲು ಇಚ್ಛಿಸುವನು.

ಇಲ್ಲಿ ಅಧಿಯಜ್ಞದಂತೆ ಇರುವವನು ಯಾರು? ಪ್ರತಿ ಜೀವರಾಶಿಗಳ ಅಂತರಾಳದಲ್ಲಿಯೂ ನಾವು ಮಾಡುವ ಪ್ರತಿಯೊಂದು ಯಜ್ಞವನ್ನೂ ಸ್ವೀಕರಿಸುತ್ತಿರುವವನೇ ಪರಮೇಶ್ವರ. ಅವನು ಇಲ್ಲಿ ಹೇಗಿದ್ದಾನೆ? ಅಂತ್ಯಕಾಲದಲ್ಲಿ ಪರಮೇಶ್ವರನನ್ನು ಚಿಂತೆ ಮಾಡುತ್ತಿರುವವರು ಭಗವಂತನನ್ನು ಹೇಗೆ ತಿಳಿದುಕೊಳ್ಳುತ್ತಾರೆ? ಈ ಪ್ರಶ್ನೆಗಳ ಸುರಿಮಳೆಯನ್ನೇ ಹಾಕುತ್ತಾನೆ. ಶ‍್ರೀಕೃಷ್ಣ ಅವುಗಳಲ್ಲಿ ಒಂದೊಂದನ್ನಾಗಿ ಬಿಡಿಸಲು ಯತ್ನಿಸುತ್ತಾನೆ. ಇದರ ಪರಿಣಾಮವಾಗಿ ಅತ್ಯಂತ ಜಟಿಲವಾದ ಪ್ರಶ್ನೆಗಳಿಗೆ ಉತ್ತರ ದೊರಕಬಲ್ಲ ಒಂದು ಸುಂದರ ತತ್ತ ್ವಶಾಸ್ತ್ರವೇ ತಯಾರಾಗುವುದು. ಶ‍್ರೀಕೃಷ್ಣ ಅರ್ಜುನನಿಗೆ ಹೇಳುತ್ತಾನೆ:

\begin{verse}
ಅಕ್ಷರಂ ಬ್ರಹ್ಮ ಪರಮಂ ಸ್ವಭಾವೋಽಧ್ಯಾತ್ಮಮುಚ್ಯತೇ~।\\ಭೂತಭಾವೋದ್ಭವಕರೋ ವಿಸರ್ಗಃ ಕರ್ಮಸಂಜ್ಞಿತಃ \versenum{॥ ೩~॥}
\end{verse}

{\small ಯಾವುದು ಸರ್ವೋತ್ತಮವೊ ಅವಿನಾಶಿಯೊ ಅದು ಬ್ರಹ್ಮ. ಪ್ರಾಣಿ ಮಾತ್ರದಲ್ಲಿ ಯಾವುದು ತನ್ನ ಸ್ವಭಾವ ದೊಂದಿಗೆ ಇರುವುದೊ ಅದು ಅಧ್ಯಾತ್ಮ. ಪ್ರಾಣಿಗಳ ಉತ್ಪತ್ತಿ ಸ್ಥಿತಿಗಳಿಗೆ ಕಾರಣವಾದ ವ್ಯಾಪಾರವೆ ಕರ್ಮ.}

ಬ್ರಹ್ಮವೇ ಈ ವಿಶ್ವದಲ್ಲೆಲ್ಲಾ ಸರ್ವೋತ್ತಮವಾದುದು, ಸರ್ವ ಶ್ರೇಷ್ಠವಾದುದು. ಉಳಿದಿರುವು ದೆಲ್ಲ ಅದರ ಆಧಾರದ ಮೇಲೆ ಇರುವುದು. ಈ ಪ್ರಪಂಚದಲ್ಲಿ ಬ್ರಹ್ಮನ ವಿನಃ ಉಳಿದಿರುವುದೆಲ್ಲ ಹುಟ್ಟು-ಸಾವು ಚಕ್ರದಲ್ಲಿ ಸಿಕ್ಕಿ ನರಳುತ್ತಿರುವುವು. ಬ್ರಹ್ಮ ಮಾತ್ರ ಅದಕ್ಕೆ ಸಿಕ್ಕಿಲ್ಲ. ಅದು ಎಂದೆಂದೂ ಇರುವುದು. ಅದು ಇಲ್ಲದ ಕಾಲವೇ ಇಲ್ಲ. ಅದನ್ನು ಯಾವುದೂ ಇಲ್ಲದಂತೆ ಮಾಡಲಾಗುವುದಿಲ್ಲ. ಆಕಾಶ ಹೇಗೆ ಏನು ಮಾಡಿದರೂ ನಾಶವಾಗುವುದಿಲ್ಲವೋ ಹಾಗೆ ಬ್ರಹ್ಮ. ಆದರೆ ಆಕಾಶ ಜಡ. ಬ್ರಹ್ಮನಾದರೋ ಚೈತನ್ಯ ಸ್ವರೂಪ. ಅವನು ಸೃಷ್ಟಿಸಿದ ಈ ವಿಶ್ವದಲ್ಲೆಲ್ಲಾ ಹಾಸು ಹೊಕ್ಕಾಗಿರುವನು. ಆದರೆ ಇಲ್ಲಿಯೇ ಅವನು ಖರ್ಚಾಗಿ ಹೋಗಿಲ್ಲ. ಇದನ್ನು ಮೀರಿರುವನು. ಅವನ ಅನಂತದಲ್ಲಿ ಯಾವುದೋ ಅಂಶ ಈ ಪ್ರಪಂಚದ ರೂಪವನ್ನು ತಾಳುವುದು. ಈ ಪೃಥ್ವಿಯನ್ನೆಲ್ಲಾ ವ್ಯಾಪಿಸಿರುವ ಸಾಗರದಲ್ಲಿ ಯಾವುದೋ ಸ್ವಲ್ಪ ಭಾಗ ಘನೀಭೂತವಾಗಿ ನೀರ್ಗಲ್ಲಿನಂತೆ ನೀರಿನ ಮೇಲೆ ತೇಲುವಂತಿದೆ.

ಯಾವುದು ಜೀವಿಯಲ್ಲಿ ಸ್ವಭಾವದಂತೆ ಇದೆಯೊ ಅದೇ ಅಧ್ಯಾತ್ಮ. ಪ್ರತಿ ಜೀವರಾಶಿಯಲ್ಲಿಯೂ ಅದರ ವ್ಯಕ್ತಿತ್ವದ ಕೇಂದ್ರವೇ ಅದರ ಸ್ವಭಾವ. ನಾವೆಲ್ಲ ಕೆಲವು ಸ್ವಭಾವದೊಡನೆ ಹುಟ್ಟುವೆವು. ಸ್ವಭಾವವನ್ನು ನಾವು ಬದಲಾಯಿಸಿಕೊಳ್ಳಬಹುದು. ಆದರೆ ಯಾರೂ ಖಾಲಿ ಬರುವುದಿಲ್ಲ. ಸ್ವಭಾವ ದೊಡನೆ ಈ ಪ್ರಪಂಚಕ್ಕೆ ಬರುತ್ತೇವೆ. ಸ್ವಭಾವಕ್ಕೆ ತಕ್ಕಂತೆ ನಾವು ಬಾಳಿ ಬದುಕುತ್ತೇವೆ, ಕರ್ಮವನ್ನು ಮಾಡುತ್ತೇವೆ. ‘ಮನುಷ್ಯ’ಎಂದರೆ ಸ್ವಭಾವದ ಒಂದು ಕಂತೆಯಲ್ಲದೆ ಬೇರಲ್ಲ. ನಮ್ಮ ವ್ಯಕ್ತಿತ್ವದ ಸಾರವೇ ಸ್ವಭಾವ.

ಅನಂತರ ‘ಕರ್ಮ’ಎಂದರೆ ಏನು ಎಂಬುದನ್ನು ಹೇಳುತ್ತಾನೆ. ನಮ್ಮ ಉತ್ಪತ್ತಿ-ಸ್ಥಿತಿಗೆ ಕಾರಣ ವಾಗಿರುವುದೇ ಕರ್ಮ. ಕರ್ಮ ಮಾಡಿ ಸ್ವಭಾವವನ್ನು ಉತ್ಪತ್ತಿ ಮಾಡಿಕೊಳ್ಳುತ್ತೇವೆ. ಅನಂತರ ಆ ಸ್ವಭಾವಕ್ಕೆ ತಕ್ಕಂತೆ ಕರ್ಮ ಮಾಡುತ್ತಾ ಇರುತ್ತೇವೆ. ಈ ಕರ್ಮಕ್ಕೆ ಒಂದು ಆದಿಯಿಲ್ಲ. ಎಷ್ಟು ಜನ್ಮಗಳ ಹಿಂದೆ ಹೋದರೂ ಅದರ ಹಿಂದೆ ಕರ್ಮ ಇದೆ. ಮಾಡಿದ ಪ್ರತಿಯೊಂದು ಕರ್ಮವೂ ಸಂಸ್ಕಾರದಂತೆ ನಮ್ಮ ಮನಸ್ಸಿನಲ್ಲಿ ಬೀಜವಾಗುವುದು. ಆ ಬೀಜ ಎಂದಿಗೂ ನಾಶವಾಗುವುದಿಲ್ಲ. ನಮ್ಮ ಮನಸ್ಸಿನ ಉಗ್ರಾಣದಲ್ಲಿರುವುದು. ಅದಕ್ಕೆ ಸರಿಯಾದ ವಾತಾವರಣ ಸಿಕ್ಕಿದರೆ ಪುನಃ ಅದು ಕಾರ್ಯರೂಪದಲ್ಲಿ ವ್ಯಕ್ತವಾಗಿ, ಪುನಃ ಬೀಜ ಬಿಟ್ಟು ಸಂಸ್ಕಾರ ರೂಪದಲ್ಲಿ ಸಂಗ್ರಹವಾಗುವುದು. ನಾವು ನಾಶ ಮಾಡಿದ ಕರ್ಮಗಳ ವಾಸನೆ ಇರುವ ತನಕ ಅದನ್ನು ವೃದ್ಧಿ ಮಾಡಿಕೊಳ್ಳುವುದರ ಕಡೆಗೇ ನಮ್ಮ ಮನಸ್ಸು ಹೋಗುತ್ತಿರುವುದು. ಅದನ್ನು ಕಡಮೆ ಮಾಡಿಕೊಂಡು ವಾಸನಾ ಕ್ಷಯಕ್ಕೂ ಕರ್ಮ ಮಾಡಲೇಬೇಕು. ಕರ್ಮವಿಲ್ಲದೇ ಇದ್ದರೆ ವಾಸನೆಯೂ ಕ್ಷಯಿಸುವುದಿಲ್ಲ. ಒಂದು ದಾರವನ್ನು ಉಂಡೆಯಾಗಿ ಸುತ್ತವುದೂ ಕರ್ಮವೇ. ಇದೇ ಪ್ರವೃತ್ತಿ ಕರ್ಮ. ಸುತ್ತಿದ ಉಂಡೆಯನ್ನು ಬಿಡಿಸು ವುದೂ ಕರ್ಮವೇ. ಇದೇ ನಿವೃತ್ತಿ ಕರ್ಮ.

\begin{verse}
ಅಧಿಭೂತಂ ಕ್ಷರೋ ಭಾವಃ ಪುರುಷಶ್ಚಾಧಿದೈವತಮ್~।\\ಅಧಿಯಜ್ಞೋಽಹಮೇವಾತ್ರ ದೇಹೇ ದೇಹಭೃತಾಂ ವರ \versenum{॥ ೪~॥}
\end{verse}

{\small ದೇಹಧಾರಿಗಳಲ್ಲಿ ಶ್ರೇಷ್ಠನಾದ ಅರ್ಜುನ! ನಶ್ವರವಾದ ವಸ್ತುಗಳೇ ಅಧಿಭೂತ; ಅಲ್ಲಿರುವ ಜೀವಸ್ವರೂಪನೇ ಅಧಿದೈವ; ಈ ದೇಹದಲ್ಲಿ ನಾನೇ ಅಧಿಯಜ್ಞ.}

ನಶ್ವರವಾದ ವಸ್ತುವೇ ಅಧಿಭೂತ. ಈ ದೇಹ ಮತ್ತು ಈ ಪ್ರಪಂಚ ಎಲ್ಲಾ ಆಗಿರುವುದು ಈ ಜಡವಾದ ಪಂಚಭೂತಗಳಿಂದ. ಈ ಪಂಚಭೂತಗಳು ಒಂದೊಂದು ಕಡೆ ಒಂದೊಂದು ಆಕಾರವನ್ನು ತಾಳುವುವು. ಯಾವ ಆಕಾರವನ್ನು ಪಡೆದುಕೊಂಡರೂ ಅವು ಬದಲಾವಣೆ ಆಗುವುವು. ಬದಲಾವಣೆ ಆಗದ ಆಕಾರವೇ ಇಲ್ಲ. ಕೆಲವು ನಿಧಾನವಾಗಿ ಬದವಾವಣೆ ಆಗುವುವು. ಮತ್ತೆ ಕೆಲವು ಬೇಗ ಬೇಗ ಬದಲಾವಣೆ ಆಗುವುವು. ಈಗ ಭೂಮಿಯ ಮೇಲಿರುವ ಹಿಮಾಲಯ ಪರ್ವತ ಒಂದು ಕಾಲದಲ್ಲಿ ಕಡಲು ತಡಿಯಲ್ಲಿತ್ತು. ಈಗ ಸಮುದ್ರ ಇರುವ ಕಡೆ ಒಂದು ಕಾಲದಲ್ಲಿ ನೆಲವಿತ್ತು. ಈಗ ನೆಲ ಇರುವ ಕಡೆ ಒಂದು ಕಾಲದಲ್ಲಿ ಸಮುದ್ರವಿತ್ತು. ದೊಡ್ಡ ದೊಡ್ಡ ಬೆಟ್ಟಗಳು ಕೂಡ ಭೂಮಿ ಯಿಂದೆದ್ದಿವೆ. ಮಳೆ-ಬಿಸಿಲು-ಗಾಳಿ ಇವುಗಳ ಹೊಡತಕ್ಕೆ ಸಿಕ್ಕಿ ಕ್ರಮೇಣ ನಾಶವಾಗುತ್ತಿವೆ. ರೂಪವನ್ನು ಧರಿಸಿರುವುದು-ರೂಪವನ್ನು ಕಳೆದುಕೊಳ್ಳುವ ಕಡೆ ಧಾವಿಸುತ್ತಿವೆ. ಸೃಷ್ಟಿಯಾದ ಪ್ರತಿಯೊಂದು ವಸ್ತುವೂ ನಾಶವಾಗಲೇ ಬೇಕು. ನಾಶವಾಗದೆ ಯಾವುದೂ ಇರುವುದಕ್ಕೆ ಆಗುವುದಿಲ್ಲ. 

ಈ ದೇಹವೆಂಬ ಗೂಡಿನಲ್ಲಿ ಹಕ್ಕಿಯಂತೆ ಬಂದು ಕುಳಿತು, ಈ ದೇಹದ ಮೂಲಕ ಹಲವು ಕರ್ಮಗಳನ್ನು ಮಾಡಿ ಸಿಹಿ ಕಹಿ ಅನುಭವಗಳನ್ನುಂಡು, ದೇಹ ನಾಶವಾದ ಮೇಲೆ ಪುನಃ ಹಾರಿ ಹೋಗುವ ಹಕ್ಕಿಯಂತೆ ಇರುವವನೇ ಜೀವ. ಈ ದೇಹವೇ ಜೀವನಲ್ಲ. ಜೀವ ವಾಸಿಸುವ ಗೂಡು ಈ ದೇಹ. ಈ ಜೀವವನ್ನು ಪ್ರತಿಬಿಂಬಿಸುತ್ತಿರುವುದು ದೇಹ ತನ್ನ ಕ್ರಿಯೆಗಳ ಮೂಲಕ. ಜೀವ ಇದ್ದರೆ ಅಂಗಗಳೆಲ್ಲ ಕೆಲಸ ಮಾಡುವುವು. ಅದಿಲ್ಲದೇ ಇದ್ದರೆ ಅಂಗಗಳೆಲ್ಲಾ ಇರುವುವು, ಆದರೆ ಏನನ್ನೂ ಮಾಡಲಾಗುವುದಿಲ್ಲ. ದೊಡ್ಡದೊಂದು ಫ್ಯಾಕ್ಟರಿ ಕೆಲಸ ಮಾಡಬೇಕಾದರೆ, ಅದನ್ನು ಚಲಿಸುವಂತೆ ಮಾಡುವುದೇ ವಿದ್ಯುತ್​ಶಕ್ತಿ. ಯಾವಾಗ ವಿದ್ಯುತ್​ಶಕ್ತಿ ನಿಲ್ಲುವುದೋ, ಆ ಫ್ಯಾಕ್ಟರಿಯ ಯಂತ್ರ ಗಳೆಲ್ಲವೂ ಇವೆ, ಆದರೆ ನಿಶ್ಶಬ್ದ, ಏನನ್ನೂ ಮಾಡಲಾರವು. ಜೀವಕ್ಕೆ ದೇಹ ಸಂಬಂಧವಾದರೆ, ದೇಹ ವಿದ್ಯುತ್​ಶಕ್ತಿಗೆ ಹಾಕಿದ ಬಲ್ಬಿನಂತೆ ಹೊಳೆಯುವುದು. ಯಾವಾಗ ಜೀವ ಅದರಿಂದ ಬೇರೆ ಆಗುವುದೊ, ಆಗ ಬಲ್ಬಿದ್ದರೂ ಯಾವ ಪ್ರಯೋಜನಕ್ಕೂ ಬರುವುದಿಲ್ಲ.

ಈ ದೇಹದಲ್ಲಿ ಜೀವಿ ಇರುವನು. ಅದರಂತೆ ಎಲ್ಲಾ ಜೀವರಾಶಿಗಳ ಹಿಂದೆಯೂ ಸರ್ವಸಾಮಾನ್ಯ ವಾಗಿ ದೇವರು ಇರುವನು. ಇವನೇ ಅಧಿಯಜ್ಞನಂತೆ ಇರುವನು. ನಾವು ಮಾಡುವ ಯಜ್ಞಗಳೆಲ್ಲ ಅರ್ಪಿತವಾಗುವುದು ನಮ್ಮಲ್ಲಿರುವ ಅವನಿಗೆ. ನಾವು ಅವನಿಗೆ ಕೊಟ್ಟರೂ ಸೇರುವುದು, ಅವನಿಗೆ ಕೊಡದೆ ಇದ್ದರೂ ಸೇರುವುದು. ಆದಕಾರಣವೇ ನಾವು ಮಾಡುವ ಕರ್ಮಗಳೆಲ್ಲ ಜೋಪಾನವಾಗಿರ ಬೇಕು. ನಮ್ಮಅಂತರ್ಯಾಮಿಯಾಗಿ ಭಗವಂತ ಇರುವನು. ಎಲ್ಲವನ್ನೂ ಅವನು ಸ್ವೀಕರಿಸುತ್ತಿರುವನು. ಹೇಗೆ ಕಡಲಡಿಯಲ್ಲಿ ಬಾಡಬ ಬರುವ ನದಿಯ ನೀರನ್ನೆಲ್ಲಾ ಕುಡಿಯುವನೊ ಹಾಗೆ ನಮ್ಮ ಹಿಂದೆ ಇರುವ ದೇವರು ಯಜ್ಞಗಳನ್ನೆಲ್ಲ ಸ್ವೀಕರಿಸುತ್ತಿರುವನು. ನಾನಿರುವ ಕಡೆಯಲ್ಲೆಲ್ಲ ದೇವರಿರುವನು, ಅಲೆ ಇರುವ ಕಡೆಯಲ್ಲೆಲ್ಲ ಸಮುದ್ರ ಇರುವಂತೆ. ಅವನು, ನನಗೆ ಗೊತ್ತಿದ್ದರೂ ಇರುವನು, ಗೊತ್ತಿಲ್ಲದೇ ಇದ್ದರೂ ಇರವನು. ಅಜ್ಞಾನದಲ್ಲಿ ನಾನು ಅವನನ್ನು ಮರೆತಾಗ, ದುರಹಂಕಾರದಲ್ಲಿ ನಾನು ಅವನನ್ನು ಅಲ್ಲಗಳೆದಾಗ, ಸದಾಕಾಲದಲ್ಲಿಯೂ ಇರುವನು. ಅದರಂತೆಯೇ ನಾವು ಮಾಡುವು ದೆಲ್ಲ ಅವನಿಗೆ ಅರ್ಪಿತವಾಗುವುದು. ತಿಳಿದು ಕೊಟ್ಟರೂ ಅದು ಕೃಷ್ಣಾರ್ಪಿತ, ತಿಳಿಯದೇ ಕೊಟ್ಟರೂ ಅದನ್ನು ಕೃಷ್ಣ ತೆಗೆದುಕೊಂಡೇ ತೆಗೆದುಕೊಳ್ಳುವನು. ಆದಕಾರಣವೇ ನಾವು ಏನು ಕೆಲಸ ಮಾಡುತ್ತೇವೆ, ಏನು ಆಲೋಚನೆ ಮಾಡುತ್ತೇವೆ ಎಂಬ ವಿಷಯದಲ್ಲಿ ಜೋಪಾನವಾಗಿರಬೇಕು.

\begin{verse}
ಅಂತಕಾಲೇ ಚ ಮಾಮೇವ ಸ್ಮರನ್ ಮುಕ್ತ್ವಾ ಕಲೇವರಮ್~।\\ಯಃ ಪ್ರಯಾತಿ ಸ ಮದ್ಭಾವಂ ಯಾತಿ ನಾಸ್ತ್ಯತ್ರ ಸಂಶಯಃ \versenum{॥ ೫~॥}
\end{verse}

{\small ಮತ್ತು ಮರಣಕಾಲದಲ್ಲಿ ಯಾರು ನನ್ನನ್ನು ಸ್ಮರಣೆ ಮಾಡುತ್ತ ದೇಹವನ್ನು ಬಿಡುವನೊ ಅವನು ನನ್ನ ಸ್ವಭಾವವನ್ನು ಹೊಂದುತ್ತಾನೆ. ಇದರಲ್ಲಿ ಸಂಶಯ ಇಲ್ಲ.}

ಬದುಕಿರುವಾಗ ಯಾರು ಯಾವಾಗಲೂ ಚಿತ್ತವನ್ನು ಭಗವಂತನ ಕಡೆಗೆ ಹಾಕಿರುವರೊ ಅವರು ಅಂತ್ಯಕಾಲದಲ್ಲಿಯೂ ಭಗವಂತನನ್ನು ಚಿಂತಿಸುವುದರಲ್ಲಿ ಸಂದೇಹವಿಲ್ಲ. ಇರುವಾಗ ಮಾಡುವ ಆಲೋಚನೆಗಳಲ್ಲಿ ಬಲವಾಗಿರುವುದೇ ಅಂತ್ಯಕಾಲದಲ್ಲಿ ಬರಬೇಕಾದರೆ. ಅಂತೂ ಅಂತ್ಯಕಾಲದಲ್ಲಿ ಆಲೋಚನೆ ಮಾಡಬೇಕಾದರೆ ಈಗಿನಿಂದಲೂ ಏತಕ್ಕೆ, ಕೊನೆಗಾಲದಲ್ಲಿ ಅದನ್ನುಅಭ್ಯಾಸ ಮಾಡಿದರೆ ಸಾಕಲ್ಲ ಎಂದುಕೊಳ್ಳಬಹುದು ನಾವು. ಆದರೆ ಮನುಷ್ಯ ಯಾವಾಗ, ಹೇಗೆ ಸಾಯುತ್ತಾನೆಂದು ಹೇಳುವುದಕ್ಕೆ ಆಗುವುದಿಲ್ಲ. ಮೃತ್ಯುವಿನಷ್ಟು ಅನಿಶ್ಚಿತವಾಗಿರುವುದು ಯಾವುದೂ ಇಲ್ಲ. ಆದ ಕಾರಣವೇ ಈಗಿನಿಂದಲೂ ಭಗವಂತನನ್ನು ಕುರಿತು ಚಿಂತಿಸುವುದನ್ನು ಅಭ್ಯಾಸ ಮಾಡಬೇಕು. ಮೃತ್ಯು ಬಂದಾಗ, ಮುಂದೆ ಹೋಗಿ ಬಾ, ನಾನು ಇನ್ನೂ ಅಣಿಯಾಗಿಲ್ಲ ಎಂದರೆ ಕೇಳುವಂತಿಲ್ಲ ಅದು. ‘ನಿನ್ನನ್ನೇ ಕಾಯುತ್ತಿದ್ದೆ ನಾನು; ಅಣಿಯಾಗಿರುವೆನು’ ಎಂದು ಹೇಳಬೇಕು. ಹಾಗೆ ಹೇಳಬೇಕಾದರೆ, ಭಗವಂತನಲ್ಲಿ ಯಾರು ಆಶ್ರಯ ಪಡೆದಿದ್ದಾನೊ ಅವನಿಗೆ ಮಾತ್ರ ಸಾಧ್ಯ.

ಭಗವಂತನನ್ನು ಕುರಿತು ಚಿಂತಿಸುತ್ತ ದೇಹವನ್ನು ತ್ಯಜಿಸಿದರೆ ನಾವು ಅವನನ್ನೇ ಪಡೆಯುವುದರಲ್ಲಿ ಸಂದೇಹವಿಲ್ಲ. ಹನಿ ಸಾಗರಕ್ಕೆ ಸೇರಿದಂತೆ, ನಾವು ಅವನಲ್ಲಿ ಒಂದಾಗುವೆವು. ಅಗ್ನಿಕುಂಡದಿಂದ ಸಿಡಿದು ಮೇಲೆದ್ದ ಕಿಡಿಯೊಂದು ಪುನಃ ಅಗ್ನಿಕುಂಡಕ್ಕೆ ಬಿದ್ದಂತೆ, ಅವನ ಮಡಿಲಿನಲ್ಲಿ ನಾವು ಬೀಳುವೆವು. ಅಂತ್ಯಕಾಲದಲ್ಲಿ ಅವನನ್ನು ಕುರಿತು ಚಿಂತನೆ ಮಾಡುವುದು ಅವನ ಬಳಿಗೆ ಹೋಗುವು ದಕ್ಕೆ ಟಿಕೇಟನ್ನು ತೆಗೆದುಕೊಂಡಂತೆ. ಕೀಟ ಭ್ರಮರವನ್ನು ಕುರಿತು ಚಿಂತಿಸುತ್ತಿದ್ದರೆ ತನ್ನ ಕೀಟತ್ವವನ್ನು ತ್ಯಜಿಸಿ ಭ್ರಮರವಾಗುವಂತೆ, ನಾವು ಕೂಡ ಭಗವಂತನಂತೆ ಆಗುವೆವು. ಇದರಲ್ಲಿ ಯಾವ ಸಂದೇಹವೂ ಇಲ್ಲ ಎನ್ನುತ್ತಾನೆ ಶ‍್ರೀಕೃಷ್ಣ. ಯಾರಿಗೆ ಗೊತ್ತು? ಆದರೂ ಆಗಬಹುದು, ಇಲ್ಲದೇ ಇದ್ದರೂ ಇಲ್ಲ ಎಂಬ ಸಂದೇಹ ನಮ್ಮನ್ನು ಕಾಡಬಹುದು. ಇದನ್ನು ಪರಿಹರಿಸುವುದಕ್ಕಾಗಿ ಧೈರ್ಯದಿಂದ ಈ ಮಾತನ್ನು ಹೇಳುತ್ತಾನೆ. ಈ ಮಾತು ಆಧ್ಯಾತ್ಮಿಕ ಸತ್ಯ. ಹೇಗೆ ವೈಜ್ಞಾನಿಕ ಪ್ರಪಂಚದಲ್ಲಿ ನಿಯಮಗಳಿವೆಯೋ, ಪ್ರಯೋಗಶಾಲೆಯಲ್ಲಿ ಇದರ ಸತ್ಯವನ್ನು ಕಂಡುಹಿಡಿದಿರು ವರೊ, ಹಾಗೆಯೇ ಸಾಧುಸಜ್ಜನರು, ಯೋಗಿಗಳು ಶ‍್ರೀಕೃಷ್ಣ ಹೇಳಿದ ಸತ್ಯವನ್ನು ತಮ್ಮ ಜೀವನದ ಪ್ರಯೋಗಶಾಲೆಯಲ್ಲಿ ಪರೀಕ್ಷಿಸಿಕೊಂಡಿರುವರು. ಯಾರು ಬದುಕಿರುವಾಗ ಭಗವಂತನೆಂಬ ಧ್ರುವ ತಾರೆಯನ್ನು ಮನದ ಮುಂದಿಟ್ಟುಕೊಂಡು ಬಾಳುತ್ತಿರುವರೊ, ಅವರಿಗೆ, ಮರಣ ಇದ್ದಕ್ಕೆ ಇದ್ದಂತೆ, ನಿರೀಕ್ಷಿಸದೆ ಇರುವಾಗ ಬಂದರೂ ಅವರು ಅದಕ್ಕೆ ಸಿದ್ಧವಾಗಿರುವರು. ಗಾಂಧೀಜಿ ಪ್ರಾರ್ಥನಾಸಭೆಗೆ ಹೋಗುತ್ತಿದ್ದಾಗ ಗುಂಡಿನೇಟಿಗೆ ತುತ್ತಾದರು. ತಕ್ಷಣವೇ ಅವರ ಬಾಯಿಂದ ‘ಹೇ ರಾಮ್​’ ಎಂಬ ನುಡಿ ಹೊರಟು, ಪ್ರಾಣಪಕ್ಷಿ ಹಾರಿಹೋಯಿತು.

\begin{verse}
ಯಂ ಯಂ ವಾಪಿ ಸ್ಮರನ್ ಭಾವಂ ತ್ಯಜತ್ಯಂತೇ ಕಲೇವರಮ್~।\\ತಂ ತಮೇವೈತಿ ಕೌಂತೇಯ ಸದಾ ತದ್ಭಾವಭಾವಿತಃ \versenum{॥ ೬~॥}
\end{verse}

{\small ಅರ್ಜುನ, ಮರಣಕಾಲದಲ್ಲಿ ಯಾವುದನ್ನು ಚಿಂತಿಸುತ್ತ ದೇಹವನ್ನು ತ್ಯಜಿಸುವನೊ, ಅನುಗಾಲವೂ ಚಿಂತಿಸುತ್ತಿರುವ ಅದನ್ನೇ ಅವನು ಹೊಂದುವನು.}

ಮರಣಕಾಲದಲ್ಲಿ ಯಾವುದನ್ನು ಚಿಂತಿಸುತ್ತಿರುವನೊ ಅದನ್ನೆ ಹೊಂದುವನು. ಅವನಿಗೆ ಯಾವ ದೇವತೆಯ ಮೇಲೆ ಇಷ್ಟವೋ ಅದನ್ನೇ ಕುರಿತು ಚಿಂತಿಸುತ್ತಿದ್ದರೆ ಅದನ್ನೇ ಹೊಂದುವನು. ಈ ಪ್ರಪಂಚದ ಯಾವುದಾದರೂ ವ್ಯಕ್ತಿಯ ಮೇಲೆ ಆಸೆ ಬಲವಾಗಿದ್ದರೆ, ಅದೇ ವಾತಾವರಣದಲ್ಲಿ ಹುಟ್ಟುವನು. ಅಥವಾ ಹಲವು ವಸ್ತುಗಳ ಮೇಲೆ ನಮಗೆ ಆಸೆ ಇನ್ನೂ ಹಿಂಗಿಲ್ಲ. ಅವನ್ನು ಇನ್ನೂ ಅನುಭವಿಸಬೇಕೆಂಬ ಆಸಕ್ತಿ ಬಲವಾಗಿದೆ. ಆ ವಸ್ತುಗಳನ್ನು ಪಡೆದು ಅನುಭವಿಸುವ ಸ್ಥಳದಲ್ಲಿಯೇ ಅವನು ಹುಟ್ಟುವನು. ಈ ಪ್ರಪಂಚದ ವಸ್ತುಗಳ ಪ್ರೀತಿ ಬಲವಾಗಿದ್ದರೆ ನಾವು ಆ ಸ್ಥಳದಲ್ಲಿಯೇ ಹುಟ್ಟುತ್ತೇವೆ. ದ್ವೇಷ ಬಲವಾಗಿದ್ದರೂ ನಾವು ಅದನ್ನು ತೀರಿಸಿಕೊಳ್ಳುವುದಕ್ಕೆ ಆ ಸ್ಥಳದಲ್ಲಿ ಹುಟ್ಟುತ್ತೇವೆ. ಪ್ರೀತಿ ಹೇಗೆ ನಮ್ಮನ್ನು ಒಂದು ವಸ್ತುವಿಗೆ ಕಟ್ಟಿಹಾಕುವುದೊ, ದ್ವೇಷವೂ ಅದಕ್ಕಿಂತ ಬಲವಾಗಿ ದ್ವೇಷಿಸುವ ವಸ್ತುವಿಗೆ ನಮ್ಮನ್ನು ಕಟ್ಟಿಹಾಕುವುದು. ಪ್ರೀತಿಯ ಸಂಬಂಧಕ್ಕಿಂತ ದ್ವೇಷದ ಸಂಬಂಧ ಅತಿ ತೀವ್ರವಾದುದು. ಅದಕ್ಕಾಗಿಯೇ ಜಯ-ವಿಜಯರಿಗೆ ದೂರ್ವಾಸರು ಶಾಪ ಕೊಟ್ಟಾಗ, ಏಳು ಜನ್ಮಗಳು ದೇವರ ಮಿತ್ರರಾಗಿ ಹುಟ್ಟುವಿರೋ, ಮೂರು ಜನ್ಮಗಳು ದೇವರ ಶತ್ರುಗಳಾಗಿ ಹುಟ್ಟುವಿರೋ ಎಂದಾಗ, ಅವರು ಹತ್ತಿರದ ಹಾದಿಯನ್ನೇ ಹುಡುಕಿಕೊಂಡರು, ದ್ವೇಷದ ಮೂಲಕ. ಆದಕಾರಣವೇ ಮನಸ್ಸನ್ನು ಲೌಕಿಕ ಸಂಬಂಧಗಳಿಂದ ಬಿಡಿಸಿಕೊಂಡು ಭಗವಂತನ ಕಡೆಗೆ ಸದಾ ಹರಿಯುವಂತೆ ರೂಢಿಸಿಕೊಂಡಿರಬೇಕು.

\begin{verse}
ತಸ್ಮಾತ್ ಸರ್ವೇಷು ಕಾಲೇಷು ಮಾಮನುಸ್ಮರ ಯುಧ್ಯ ಚ~।\\ಮಯ್ಯರ್ಪಿತಮನೋಬುದ್ಧಿರ್ಮಾಮೇವೈಷ್ಯಸ್ಯಸಂಶಯಃ \versenum{॥ ೭~॥}
\end{verse}

{\small ಆದಕಾರಣ ನನ್ನನ್ನೇ ಅನುಗಾಲವೂ ಸ್ಮರಣೆಮಾಡುತ್ತಾ ಯುದ್ಧ ಮಾಡುತ್ತಿರು. ಹೀಗೆ ನನ್ನಲ್ಲಿ ಮನಸ್ಸು, ಬುದ್ಧಿಗಳನ್ನು ಅರ್ಪಿಸಿದರೆ ಸಂದೇಹವಿಲ್ಲದೆ ನನ್ನನ್ನು ಹೊಂದುತ್ತೀಯೆ.}

ಶ‍್ರೀಕೃಷ್ಣ ಅರ್ಜುನನಿಗೆ ಯಾವಾಗಲೂ ದೇವರನ್ನೇ ಚಿಂತಿಸುತ್ತ ಯುದ್ಧಮಾಡು ಎನ್ನುವನು. ಮೃತ್ಯು ಯಾವಾಗ ಯಾವ ರೂಪದಲ್ಲಿ ಬರುವುದೊ ಅದನ್ನು ಯಾರೂ ಹೇಳುವುದಕ್ಕೆ ಆಗುವುದಿಲ್ಲ. ಅದರಲ್ಲಿಯೂ ಯುದ್ಧಕ್ಷೇತ್ರದಲ್ಲಿ ಮೃತ್ಯುವಿನ ಸುರಿಮಳೆಯೇ ಸುತ್ತಲೂ ಇರುವಾಗ ನಾವು ಅದಕ್ಕೆ ಹೆಚ್ಚು ಅಣಿಯಾಗಿರಬೇಕು. ನಾನು ಈ ಕ್ಷಣದಲ್ಲಿಯೇ ಸಾಯಬಹುದು ಎಂಬ ಭಾವನೆ ಮನಸ್ಸಿನಲ್ಲಿರ ಬೇಕು. ಸತ್ತರೆ ಭಗವಂತನ ಸ್ಮರಣೆ ಮನದಲ್ಲಿರಬೇಕು. ಅವನ ಸ್ಮರಣೆಯ ಹೆಗಲನ್ನೇರಿ ಜೀವ ಹಾರಿಹೋಗಬೇಕು.

ಶ‍್ರೀಕೃಷ್ಣ ಬರೀ ‘ನನ್ನನ್ನು ಚಿಂತಿಸುತ್ತಿರು’ ಎಂದು ಮಾತ್ರ ಹೇಳುವುದಿಲ್ಲ. ಚಿಂತಿಸುತ್ತಾ ಸುಮ್ಮನೆ ಕುಳಿತುಕೊಳ್ಳುವುದಲ್ಲ. ನಮ್ಮ ಪಾಲಿಗೆ ಬಂದ ಕರ್ತವ್ಯವನ್ನು ಮಾಡುತ್ತಿರಬೇಕು. ಬಂದು ಕರೆದಾಗ ನಾವು ಹೋಗಲು ಸಿದ್ಧವಾಗಿರಬೇಕು. ಇದು ಬರೀ ಸಾವಿಗೆ ಅಣಿಯಾಗುವುದು ಮಾತ್ರವಲ್ಲ. ಅವನು ಬದುಕಿರುವುದಕ್ಕೂ ಅಣಿಯಾಗಿರುವನು. ಜೀವನದ ಮೇಲೆ ಜುಗುಪ್ಸೆ ಹುಟ್ಟಿ ಮೃತ್ಯುವನ್ನೇ ಎದುರು ನೋಡುತ್ತಿರುವವನಿಗೆ ಈ ಪ್ರಪಂಚದ ಯಾವ ಕಾರ್ಯದಲ್ಲಿ ತೊಡಗಿರುವುದಕ್ಕೂ ಉತ್ಸಾಹವಿಲ್ಲ. ಹಾಗೆಯೇ ಜೀವನದಲ್ಲಿ ಉತ್ಸಾಹ ಇರುವವನು ಸಾವಿನ ಯೋಚನೆಯನ್ನು ಮಾಡಲಾರ. ಅದರಷ್ಟು ಕಂಪನಕಾರಿ ಮತ್ತಾವುದೂ ಇಲ್ಲ. ಈ ಪ್ರಪಂಚಕ್ಕೆ, ಮಾಡುವ ಕೆಲಸಕ್ಕೆ ಅಂಟಿಕೊಂಡು ಇರುವನು, ಜಿಗಣೆಯಂತೆ. ಅದನ್ನು ಬಿಡಲು ಮನಸ್ಸಿಲ್ಲ. ಆದರೆ ಯೋಗಿಯಾದರೊ ಎರಡಕ್ಕೂ ಸಿದ್ಧವಾಗಿರು ವನು. ಇರುವ ತನಕ, ತುಕ್ಕು ಹಿಡಿದುಹೋಗುವುದಕ್ಕಿಂತ ಸವೆದುಹೋಗುವುದು ಮೇಲೆಂದು ಕರ್ಮದ ನೊಗಕ್ಕೆ ತನ್ನ ಹೆಗಲನ್ನು ಕೊಟ್ಟಿರುವನು. ಕೆಲಸದ ಮಧ್ಯದಲ್ಲಿ ಮೃತ್ಯುವಿನ ಕರೆ ಬಂದರೂ ಕ್ಷಣಾರ್ಧದಲ್ಲಿ ತನ್ನ ಕೈಯನ್ನು ತೊಳೆದುಕೊಳ್ಳುವುದಕ್ಕೆ ಸಿದ್ಧನಾಗಿರುವನು.

ನನಗೆ ಮನಸ್ಸು ಮತ್ತು ಬುದ್ಧಿಯನ್ನು ಅರ್ಪಿಸಿದರೆ ನನ್ನನ್ನೇ ಹೊಂದುವೆ ಎನ್ನುತ್ತಾನೆ. ನಮ್ಮ ಮನಸ್ಸು ಭಗವಂತನಲ್ಲೆ ಅದ್ದಿರಬೇಕು, ಅವನಲ್ಲೆ ಮುಳುಗಿರಬೇಕು. ಆಗ ಮನಸ್ಸು ಪವಿತ್ರವಾಗು ವುದು, ಪರಿಶುದ್ಧವಾಗುವುದು. ಪ್ರಪಂಚದ ಗಾಣಕ್ಕೆ ನಮ್ಮ ಮನಸ್ಸನ್ನು ಅರ್ಪಿಸಿದರೆ ಅದು ಅರೆದು ನಿಸ್ಸಾರ ಮಾಡುವುದು. ಭಗವಂತನಿಗೆ ಅರ್ಪಿಸಿದರೆ ಅದನ್ನು ಅಮೃತದಿಂದ ತುಂಬುವನು ಅವನು. ಹಾಗೆಯೆ ಬುದ್ಧಿಯನ್ನು ಭಗವಂತನ ಕಡೆ ಹರಿಸಿದರೆ, ಅವನ ಮಹಿಮೆ ಅರ್ಥವಾಗುತ್ತ ಬರುವುದು, ಜೀವನವನ್ನು ಶಾಂತಿಯಿಂದ ತುಂಬುವುದು. ಯಾವಾಗ ಪ್ರಪಂಚಕ್ಕೆ ಅರ್ಪಿಸುತ್ತೇವೆಯೋ ಅದರಿಂದ ಹೆಚ್ಚು ಲಾಭ, ಕೀರ್ತಿ ಮುಂತಾದುವು ಬರಬಹುದು; ಆದರೆ ಬುದ್ಧಿಯೆಲ್ಲಾ ಎಂಜಲಾಗುವುದು. ಇದರಿಂದ ಶಾಂತಿ ಸಮಾಧಾನ ಸಿಲುಕದು ನಮಗೆ.

ಭಗವಂತ, ಪ್ರಾಣಿಗಳಲ್ಲೆಲ್ಲ ಮನುಷ್ಯನಲ್ಲಿ ಮಾತ್ರ ಅತ್ಯಮೋಘವಾದ ಮನಸ್ಸು ಮತ್ತು ಬುದ್ಧಿಯನ್ನು ಇಟ್ಟು ಕಳುಹಿಸಿರುವನು. ನಾವು ಯಾವಾಗ ಅದನ್ನು ಭಗವಂತನಿಗೆ ಅರ್ಪಣೆ ಮಾಡು ತ್ತೇವೆಯೋ ಆಗ ಅದು ಸಾರ್ಥಕವಾಗುವುದು. ಪ್ರಪಂಚಕ್ಕೆ ಅರ್ಪಿಸಿದರೆ ವ್ಯರ್ಥವಾಗುವುದು. ಇದರಿಂದ ಸಂಸಾರದ ಗಾಣವನ್ನು ನಾವು ಬಡ ಎತ್ತುಗಳಂತೆ ಎಳೆಯಬೇಕಾಗುವುದು.

\begin{verse}
ಅಭ್ಯಾಸಯೋಗಯುಕ್ತೇನ ಚೇತಸಾ ನಾನ್ಯಗಾಮಿನಾ~।\\ಪರಮಂ ಪುರುಷಂ ದಿವ್ಯಂ ಯಾತಿ ಪಾರ್ಥಾನುಚಿಂತಯನ್ \versenum{॥ ೮~॥}
\end{verse}

{\small ಪಾರ್ಥ! ಅಭ್ಯಾಸಯೋಗದ ಮೂಲಕ ಚಿತ್ತವನ್ನು ಸ್ಥಿರಗೊಳಿಸಿ, ಅದು ಎತ್ತಲೂ ಚದುರದಂತೆ ಮಾಡಿ, ಏಕಾಗ್ರನಾಗುವವನು ದಿವ್ಯನಾದ ಪರಮಪುರುಷನನ್ನು ಹೊಂದುತ್ತಾನೆ.}

ನಮಗೆ ಭಗವಂತನನ್ನು ಕುರಿತು ಚಿಂತಿಸುವುದಕ್ಕೆ ಏಕಾಗ್ರತೆ ಇಲ್ಲ ಎಂದು ಹೇಳುವುದು ರೂಢಿ. ಇಲ್ಲದೆ ಇದ್ದರೆ ಅದನ್ನು ರೂಢಿಸಿಕೊಳ್ಳಬೇಕು. ಈ ಜೀವನದಲ್ಲಿ ಎಷ್ಟೋ ವಸ್ತುಗಳು ನಮಗೆ ಹೊಸದು. ಆದರೆ ಅದನ್ನು ಅಭ್ಯಾಸ ಮಾಡುವುದಿಲ್ಲವೆ ಮನುಷ್ಯ! ಅಭ್ಯಾಸ ಮಾಡುವಾಗ ಎಷ್ಟು ನೀರಸ ಮತ್ತು ಎಷ್ಟು ಕಷ್ಟವಾಗುವುದು. ಆದರೂ ಬಿಡದೆ ಅದನ್ನು ಮಾಡುತ್ತೇವೆ. ಏಕೆಂದರೆ ಅದರಲ್ಲಿ ಮುಂದುವರಿಯಬೇಕೆಂದು ನಮಗೆ ಆಸೆ. ಸಂಗೀತ ಕಲಿಯಬೇಕಾದರೆ ಎಷ್ಟೊಂದು ಅಭ್ಯಾಸ ಮಾಡಬೇಕು. ಚೆನ್ನಾಗಿ ಅಭ್ಯಾಸ ಮಾಡಿದಮೇಲೆ ಸಭೆಯಲ್ಲಿ ಕುಳಿತುಕೊಂಡು ಹಾಡುತ್ತಿದ್ದರೆ, ಜನ ಬೇಕಾದಷ್ಟು ಕರತಾಡನವನ್ನು ಮಾಡುವರು. ಆದರೆ ಈ ಮೆಚ್ಚುಗೆ ಪಡುವವರಿಗೆ ಸಂಗೀತಗಾರನ ಶ್ರಮದ ಪರಿಚಯವಿರುವುದಿಲ್ಲ. ಅವನನ್ನೆ ಕೇಳಿದಾಗ ಹೇಳುತ್ತಾನೆ, ಪ್ರತಿದಿನವೂ ತಪ್ಪದೆ ಅಭ್ಯಾಸ ಮಾಡುವುದು, ಅದರಲ್ಲಿಯೂ ಬೆಳಗ್ಗೆ ಹೊತ್ತಿಗೆ ಮುಂಚೆಯೇ ಎದ್ದು ಅಭ್ಯಾಸ ಮಾಡುವುದು –ಮಾಮೂಲು ಆ ವೃತ್ತಿಯಲ್ಲಿ. ಚಳಿಗಾಲ ಬಿಸಿಲುಕಾಲವೆನ್ನದೆ ಬಿಡದೆ ಸಾಧಿಸಿದರೆ ಮಾತ್ರ ಆ ವೃತ್ತಿಯಲ್ಲಿ ನಿಪುಣನಾಗಬಹುದು. ಏನಿಲ್ಲ–ನಾವು ಹೊಸದರಲ್ಲಿ ಸೈಕಲ್ಲನ್ನು ಕಲಿಯುವಾಗ ಎಷ್ಟು ಸಲ ಬೀಳಬೇಕು, ಏಳಬೇಕು! ಅನಂತರ ಪೆಡಲ್ ಒತ್ತುವುದು, ಬ್ರೇಕ್ ಹಾಕುವುದು ಇವುಗಳೆಲ್ಲಾ ಅನೈಚ್ಛಿಕವಾಗುವುವು. ಅನಂತರ ಸೈಕಲ್​ಮೇಲೆ ಹೋಗುತ್ತಿರುವಾಗ ಏನನ್ನಾದರೂ ಹಾಡಿಕೊಂಡು ಹೋಗುವನು. ಉಳಿದ ವಸ್ತುವಿಗೆಲ್ಲ ನಾವು ಬಿಡದೆ ಭಂಡರಂತೆ ಅಭ್ಯಾಸಮಾಡಿ ಕಲಿಯುತ್ತೇವೆ. ಧ್ಯಾನವು ಕೂಡ ಹಾಗೆ. ಎಲ್ಲೊ ಕೆಲವು ಅಪರೂಪ ವ್ಯಕ್ತಿಗಳಿಗೆ ಚಿತ್ತ ಏಕಾಗ್ರವಾಗುವುದು ಭಗವಂತನ ಮೇಲೆ. ಉಳಿದವರೆಲ್ಲ ಅದಕ್ಕೆ ಹೋರಾಡಬೇಕು. ಅದಕ್ಕೇ ಶ‍್ರೀಕೃಷ್ಣ ಅಭ್ಯಾಸ ಯೋಗ ಎಂದು ಹೇಳುತ್ತಾನೆ. ಬಿಡದೆ ನಾವು ಅಭ್ಯಾಸ ಮಾಡಬೇಕು. ಆಗಲೇ ಅದು ನಮ್ಮ ಸ್ವಭಾವವಾಗುವುದು. ಪ್ರಾರಂಭದಲ್ಲಿ ಮನಸ್ಸನ್ನು ಏಕಾಗ್ರಗೊಳಿಸುವುದು ತುಂಬಾ ಕಷ್ಟದ ಕೆಲಸ. ಬೇರೆ ಕಾಲಕ್ಕಿಂತ ನಾವು ಮನಸ್ಸನ್ನು ಏಕಾಗ್ರ ಮಾಡಬೇಕು ಎನ್ನುವ ಕಾಲದಲ್ಲಿ ಮನಸ್ಸು ಚಾಂಚಲ್ಯಕ್ಕೆ ಹೆಚ್ಚಾಗಿ ಒಳಗಾಗು ವುದು ಸಾಮಾನ್ಯ ಅನುಭವ. ಎಷ್ಟೇ ಚಂಚಲವಾದರೂ ನಾನೇನು ಬಿಡುವುದಿಲ್ಲ, ಅದನ್ನು ದೇವರ ಕಡೆ ತಿರುಗಿಸಿಯೇ ತಿರುಗಿಸುತ್ತೇನೆ, ನಾನು ಅದಕ್ಕಾಗಿ ಎಷ್ಟು ಬೇಕಾದರೂ ಹೋರಾಡಲು ಸಿದ್ಧನಾಗಿ ರುವೆ ಎಂಬ ಶಪಥವನ್ನು ಮಾಡಬೇಕು. ಅನಂತರವೇ ನಮ್ಮ ಮನಸ್ಸು ಕ್ರಮೇಣ ದಾರಿಗೆ ಬರಬೇಕಾದರೆ. ಜೀವನದಲ್ಲಿ ಯಾವುದೂ ಬಿಟ್ಟಿ ಸಿಕ್ಕುವುದಿಲ್ಲ. ಒಂದು ವೇಳೆ ಬಿಟ್ಟಿ ಸಿಕ್ಕಿದರೂ ಅದರ ಯೋಗ್ಯತೆ ನಮಗೆ ಗೊತ್ತಾಗುವುದಿಲ್ಲ. ಕೆಲಸಕ್ಕೆ ಬಾರದ ಕ್ಷುದ್ರ ವಸ್ತುಗಳನ್ನು ಕಲಿಯಲು ನಾವು ಎಷ್ಟು ಕಷ್ಟಪಡುತ್ತೇವೆ. ದೇವರನ್ನು ಚಿಂತಿಸಲು ಅಷ್ಟು ಕಷ್ಟ ಪಡದೆ ಇದ್ದರೆ ಅದು ನಮಗೆ ಹೇಗೆ ದೊರಕುವುದು? ಈ ವಸ್ತುವಿಗೆ ನಾವು ಎಷ್ಟು ಕಷ್ಟಪಟ್ಟರೂ ಕಡಮೆಯೆ, ಎಷ್ಟು ಬೆಲೆ ಕೊಟ್ಟರೂ ಕಡಮೆಯೆ. ಕೊನೆಗೆ ನಮಗೆ ಸಿಕ್ಕುವುದು ಪರಮ ಶಾಂತಿ; ನಮ್ಮನ್ನು ಜನನ ಮರಣಾತೀತರನ್ನಾಗಿ ಮಾಡುವುದು; ಸಾಕ್ಷಿಯಂತೆ ಭಗವಂತನ ಲೀಲೆಯನ್ನು ನೋಡಿ ಆನಂದಿಸಬಹುದು.

ಮನಸ್ಸನ್ನು ಪ್ರಪಂಚದ ಕಡೆಗಳಿಂದ ಸೆಳೆದು ಭಗವಂತನ ಕಡೆ ಮಾತ್ರ ಹರಿಸುತ್ತ ಧ್ಯಾನಿಸುತ್ತ ಪ್ರಾಣವನ್ನು ತ್ಯಜಿಸಿದರೆ ಅವನನ್ನೇ ಪಡೆಯುತ್ತಾನೆ. ಆದರೆ ಕೊನೆ ಗಳಿಗೆಯಲ್ಲಿ ನಮ್ಮ ಮನಸ್ಸು ಅಂತಹ ಪಕ್ವಸ್ಥಿತಿಗೆ ಬರಬೇಕಾದರೆ ಹಲವು ವರುಷಗಳ ಸಾಧನೆ ಅದಕ್ಕೆ ತಳಪಾಯವಾಗಿರಬೇಕು. ಆ ತಳಪಾಯ ಹಾಕುವ ಕೆಲಸಕ್ಕೆ ನಾವು ಎಷ್ಟು ಬೇಗ ಪ್ರಾರಂಭಿಸಿದರೆ ಅಷ್ಟು ಒಳ್ಳೆಯದು.

\begin{verse}
ಕವಿಂ ಪುರಾಣಮನುಶಾಸಿತಾರಮಣೋರಣೀಯಾಂಸಮನುಸ್ಮರೇದ್ಯಃ~।\\ಸರ್ವಸ್ಯ ಧಾತಾರಮಚಿಂತ್ಯರೂಪಮಾದಿತ್ಯವರ್ಣಂ ತಮಸಃ ಪರಸ್ತಾತ್ \versenum{॥ ೯~॥}
\end{verse}

\begin{verse}
ಪ್ರಯಾಣಕಾಲೇ ಮನಸಾಚಲೇನ ಭಕ್ತ್ಯಾ ಯುಕ್ತೋ ಯೋಗಬಲೇನ ಚೈವ~।\\ಭ್ರುವೋರ್ಮಧ್ಯೇ ಪ್ರಾಣಮಾವೇಶ್ಯ ಸಮ್ಯಕ್ ಸ ತಂ ಪರಂ ಪುರುಷಮುಪೈತಿ\\ ದಿವ್ಯಮ್ \versenum{॥ ೧೦~॥}
\end{verse}

{\small ಯಾರು ಮರಣ ಸಮಯದಲ್ಲಿ ಅಚಲವಾದ ಮನಸ್ಸಿನಿಂದ ಭಕ್ತಿಪೂರ್ಣವಾಗಿ, ಯೋಗಬಲದಿಂದ ಹುಬ್ಬಿನ ಮಧ್ಯೆ ಪ್ರಾಣವನ್ನು ಚೆನ್ನಾಗಿ ನಿಲ್ಲಿಸಿ, ಸರ್ವಜ್ಞ, ಪುರಾತನ, ನಿಯಂತ, ಸೂಕ್ಷ್ಮತಮ, ಸಕಲಾಧಾರ, ಅಚಿಂತ್ಯ, ಸೂರ್ಯನಂತೆ ತೇಜಸ್ವಿ, ಅಜ್ಞಾನರೂಪದ ಅಂಧಕಾರದಿಂದ ಬೇರೆಯಾದ ಸ್ವರೂಪವುಳ್ಳ ಅವನನ್ನು ಚೆನ್ನಾಗಿ ಸ್ಮರಿಸುವವನು ದಿವ್ಯವಾದ ಪರಮಪುರುಷನನ್ನು ಪಡೆಯುತ್ತಾನೆ.}

ಮರಣಕಾಲದಲ್ಲಿ ಮನಸ್ಸನ್ನು ಏಕಾಗ್ರ ಮಾಡಬೇಕು. ಅದನ್ನು ಭಕ್ತಿಯಲ್ಲಿ ಅದ್ದಿರಬೇಕು. ಎರಡು ಹುಬ್ಬಿನ ಮಧ್ಯೆ ಪ್ರಾಣವನ್ನು ನಿಲ್ಲಿಸಬೇಕು. ಮನಸ್ಸು ಆಗ ಅಂತರ್ಮುಖವಾಗುವುದು. ಆಗ ಭಗವಂತನಿಗೆ ಸಂಬಂಧಪಟ್ಟ ಭಾವನೆಗಳನ್ನು ಚಿಂತಿಸುತ್ತಿರಬೇಕು. ಭಗವದ್ಭಾವನೆಯನ್ನು ಚಿಂತಿಸು ವುದೇ ಮನಸ್ಸಿಗೆ ಆಹಾರ. ಇದೇ ಪರಲೋಕಕ್ಕೆ ಬುತ್ತಿ.

ಅವನು ಸರ್ವಜ್ಞ. ಅವನಿಗೆ ಗೊತ್ತಿಲ್ಲದ ವಿಷಯವಿಲ್ಲ, ಈ ಪ್ರಪಂಚವನ್ನೆಲ್ಲ ಸೃಷ್ಟಿಸಿದವನು. ಸೃಷ್ಟಿಸಿದವನಿಗೆ ಅದರಲ್ಲಿರುವ ಪ್ರತಿಯೊಂದು ವಸ್ತುವಿನ ಗುಣವೂ ಗೊತ್ತಿದೆ. ಅದನ್ನು ಏತಕ್ಕೆ ಮಾಡಿದ್ದೇನೆ, ಅದರ ಉದ್ದೇಶ ಏನು ಎಂಬುದನ್ನು ಸೃಷ್ಟಿಸಿದಾತನೊಬ್ಬನೇ ನಿರ್ವಿವಾದವಾಗಿ ಹೇಳ ಬಲ್ಲ. ನಾವೆಲ್ಲ ಅವನ ಸೃಷ್ಟಿಯಲ್ಲಿರುವವರು. ನಮಗೆ ಅದೆಲ್ಲ ಚೆನ್ನಾಗಿ ಹೊಳೆಯುವುದಿಲ್ಲ. ಒಂದು ಕಾರಿನಲ್ಲಿರುವ ನಟ್ಟೊ, ಬೋಲ್ಟೊ, ಈ ಕಾರಿಗೆ ನಾವು ಯಾವ ರೀತಿ ಸಹಾಯ ಮಾಡುತ್ತಿರುವೆವು ಎಂಬುದನ್ನು ಅರಿಯಲಾರವು. ಯಾರು ಅದನ್ನು ತಯಾರು ಮಾಡಿರುವನೋ ಅವನಿಗೆ ಮಾತ್ರ ಕಾರಿನಲ್ಲಿರುವ ಪ್ರತಿಯೊಂದು ಭಾಗದ ಕೆಲಸ ಗೊತ್ತಿದೆ, ಉದ್ದೇಶ ಗೊತ್ತಿದೆ. ಅದರಂತೆಯೇ ಭಗವಂತನೊಬ್ಬನೆ ಎಲ್ಲವನ್ನೂ ಬಲ್ಲವನು. ಯಾರನ್ನು ತುಂಬ ಬುದ್ಧಿವಂತ ಮನುಷ್ಯ ಎನ್ನುವೆವೋ ಅವನು ತನ್ನೆದುರಿಗಿರುವ ಚೂರು ಪಾರು ವಸ್ತುವನ್ನು ತಿಳಿಯಲು ಯತ್ನಿಸುವನು. ಅದನ್ನು ತಿಳಿಯುತ್ತ ಅವನಿಗೆ ಅರಿವಾಗುವುದು, ದೇವರು ಎಂತಹ ಸೂತ್ರದಿಂದ ಈ ವಿಶ್ವವನ್ನು ಬಂಧಿಸಿರುವನು ಎಂಬುದು. ಅದನ್ನು ತಿಳಿದುಕೊಳ್ಳುವಾಗಲೇ, ಆ ಸೂತ್ರದ ಪರಿಚಯವಾದರೇನೇ ನಾವು ಆಶ್ಚರ್ಯಚಕಿತ ರಾಗಿ ಮೂಕರಾಗುತ್ತೇವೆ. ಇನ್ನು ಯಾರು ಇದನ್ನೆಲ್ಲ ಮಾಡಿರುವನೊ ಅವನ ಜಾಣ್ಮೆ ಎಷ್ಟು ಅದ್ಭುತ! ಬ್ರಹ್ಮಾಂಡದ ಗೋಳಗಳನ್ನು ಅಂತರಿಕ್ಷದಲ್ಲಿ ಹೇಗೆ ಚಲಿಸುವಂತೆ ಮಾಡಿರುವನು ಅವನು. ನಾವಿರುವ ಭೂಮಿ ನಿಮಿಷಕ್ಕೆ ಸುಮಾರು ಇಪ್ಪತ್ತು ಮೈಲಿ ವೇಗದಲ್ಲಿ ಸೂರ್ಯನ ಸುತ್ತಲೂ ಸುತ್ತುತ್ತಿದೆ ಎಂಬ ಅರಿವೇ ನಮಗೆ ಆಗುವುದಿಲ್ಲ. ನಾವು ಹೊತ್ತಿರುವ ದೇಹವಾದರೊ, ಅದರಲ್ಲಿ ಎಷ್ಟೊಂದು ಅಂಗಗಳಿವೆ! ಅವುಗಳೆಲ್ಲ ಹೇಗೆ ಕೆಲಸ ಮಾಡಿಕೊಂಡು ಹೋಗುತ್ತಿವೆ! ನಮ್ಮನ್ನು ಹೇಳಿ ಕೇಳಿ, ನಮ್ಮಿಂದ ಅವು ಅಪ್ಪಣೆಯನ್ನು ಪಡೆದು ಕೆಲಸ ಮಾಡುತ್ತಿವೆಯೆ? ಇಲ್ಲ. ಹೃದಯ ನನ್ನ ಇಚ್ಛೆ ಇಲ್ಲದೆ ಬಡಿಯುವುದು. ನನ್ನ ಇಚ್ಛೆ ಇಲ್ಲದೆ ಉಸಿರಾಡುವೆನು. ನಾವು ತಿಂದುದು ನಮ್ಮ ಇಚ್ಛೆ ಇಲ್ಲದೆ ಜೀರ್ಣವಾಗುವುದು. ಅಲ್ಲಿ ಏನಾದರೂ ಅಡಚಣೆಯುಂಟಾದಾಗ ಮಾತ್ರ ಇದು ಏತಕ್ಕೆ ಹೀಗೆ ಆಯಿತು ಎಂದು ನಮ್ಮ ಮನಸ್ಸು ಅತ್ತ ಹೋಗುವುದು. ಎಲ್ಲಾ ಸರಿಯಾಗಿ ನಡೆಯುತ್ತಿದ್ದರೆ ನಮಗೇನೂ ಅರಿವಾಗುವುದಿಲ್ಲ. ನಮ್ಮದೇ ನಮಗೆ ಗೊತ್ತಿಲ್ಲ. ನಮ್ಮ ದೇಹದಷ್ಟು ಹತ್ತಿರ ಯಾವುದಿದೆ? ಅದರಷ್ಟು ನಿಕಟ ಯಾವುದಿದೆ? ಜಾಗ್ರತಾವಸ್ಥೆಯ ಯಾವ ಕಾಲದಲ್ಲಿಯೂ ನಾವು ಅದನ್ನು ಮರೆಯುವುದಿಲ್ಲ. ಆದರೂ ಅದರ ವಿಷಯ ನಮಗೆ ಗೊತ್ತಿರುವುದು ಅಲ್ಪ. ಆದರೆ ಭಗವಂತನೊಬ್ಬನೇ, ಕಣದಲ್ಲಿರುವುದನ್ನು, ಜೀವಾಣುವಿನಲ್ಲಿ ಇರುವುದನ್ನು, ಜಡವನ್ನು, ಚೇತನ ವನ್ನು, ಎಲ್ಲವನ್ನೂ ತಿಳಿಯಬೇಕೆಂದು ಪ್ರಯತ್ನಿಸುವುದೊಂದು ದಾರಿ. ಜೀವನದಲ್ಲಿ ನಮ್ಮ ಸುತ್ತ ಲಿರುವ ಚೂರುಪಾರು ವಸ್ತುಗಳನ್ನು ತಿಳಿಯುವುದಕ್ಕೆ ಸಮಯವಿಲ್ಲ. ಸ್ವಲ್ಪ ತಿಳಿಯುವುದರೊಳಗಾಗಿ ಮೃತ್ಯುವಿನ ಕರೆ ಬರುವುದು. ಎಲ್ಲವನ್ನು ಬಲ್ಲವನಲ್ಲಿ ಶರಣಾಗುವುದು ಸುಲಭವಾದ ಮಾರ್ಗ. ಅವನೇ ನಮಗೇನು ಬೇಕೊ ಅದನ್ನೆಲ್ಲ ಕಲಿಸಿ ಕೊಡುತ್ತಾನೆ. ನಾವು ಅಪೂರ್ಣರು; ಪಾತ್ರೆ ಖಾಲಿ. ಆದರೆ ಭಗವಂತನು ಪೂರ್ಣ. ನಮ್ಮ ಖಾಲಿಯನ್ನು ಅವನು ಭರ್ತಿ ಮಾಡುತ್ತಾನೆ. ನಮ್ಮ ಖಾಲಿ ಪಾತ್ರೆಯನ್ನು ನೀರಿನಲ್ಲಿ ಅದ್ದಿದರೆ, ನೀರು ಹೇಗೆ ಬಂದು ತುಂಬುವುದೊ ಹಾಗೆ ಆ ಸರ್ವಜ್ಞ ನಮ್ಮನ್ನೆಲ್ಲ ಪ್ರವೇಶ ಮಾಡುವನು. ಪಾತ್ರೆಯನ್ನು ಕೆರೆಯ ನೀರಿನ ಸಮೀಪದಲ್ಲಿಟ್ಟರೆ ಸಾಲದು. ಆ ನೀರಿನಲ್ಲಿ ಅದ್ದಬೇಕು. ಆಗ ಒಳಗಿರುವ ಗಾಳಿ ಹೋಗುವುದು. ಸುತ್ತಲಿರುವ ನೀರು ಪಾತ್ರೆಯನ್ನು ಸೇರುವುದು. ಅಲ್ಪ ಹೋಗುವುದು, ಸರ್ವಜ್ಞನಾದ ಪರಮಾತ್ಮ ಅದನ್ನು ಪ್ರವೇಶಿಸುವನು.

ಅವನು ಪುರಾತನ, ಅವನಷ್ಟು ಹಳೆಯವರು ಯಾರೂ ಇಲ್ಲ. ಏಕೆಂದರೆ ಮೊದಲು ಅವನಿಂದ ನಾವು ನೋಡುವ ಜಗತ್ತೆಲ್ಲ ಆಗಿದೆ. ಅವನು ವಿಶ್ವನಿಯಾಮಕ–ವಿಶ್ವವನ್ನು ಆಳುವವನು. ಸೃಷ್ಟಿಸಿದ ದೇವರು ಒಂದು ನಿಯಮದಿಂದ ಮತ್ತು ಒಂದು ಉದ್ದೇಶದಿಂದ ಎಲ್ಲವನ್ನೂ ಬಿಗಿದಿರುವನು. ಈ ಪ್ರಪಂಚದಲ್ಲಿ ಗಾಳಿ ಬೀಸುವುದವನಿಂದ; ಮಳೆ ಬರುವುದವನಿಂದ; ಸೂರ್ಯ ಚಂದ್ರರು ತಪಿಸುವುದು ಅವನಿಂದ. ಮೃತ್ಯು ಕೂಡ ಅವನ ಆಜ್ಞಾನುಸಾರವಾಗಿ ಬರುವುದು. ಆಂತರಿಕ ಪ್ರಪಂಚದಲ್ಲಿ ಜೀವರಾಶಿಗಳಿಗೆ ಕರ್ಮಕ್ಕೆ ತಕ್ಕ ಫಲ ಕೊಡುವವನು ಅವನೇ. ಅವನು ಈ ಸೃಷ್ಟಿಯನ್ನೆಲ್ಲ ತನ್ನ ನಿಯಮವೆಂಬ ವಜ್ರಮುಷ್ಠಿಯಲ್ಲಿ ಬಿಗಿದು ಹಿಡಿದಿರುವನು. ಅಣುವಿಗಿಂತ ಅಣು ಅವನು, ಸೂಕ್ಷ್ಮಾತಿ ಸೂಕ್ಷ್ಮನು. ನಾವು ಯಾವುದನ್ನು ಅತ್ಯಂತ ಸೂಕ್ಷ್ಮವೆಂದು ಭಾವಿಸುವೆವೊ ಆ ಸೂಕ್ಷ್ಮದಲ್ಲಿಯೂ ಅವನು ಇರುವುದರಿಂದ ಆ ಸೂಕ್ಷ್ಮಕ್ಕೆ ಸೂಕ್ಷ್ಮನಾಗಬೇಕಾಯಿತು. ಹಾಗೆಯೇ ಮಹತ್ತಿಗೆ ಮಹತ್ತು. ಅನಂತ ತಾರಾ ಖಚಿತ ಬ್ರಹ್ಮಾಂಡವೆಲ್ಲವೂ ಆಕಾಶದಲ್ಲಿದೆ. ಆಕಾಶ ಅವನಲ್ಲಿದೆ. ಯಾವುದನ್ನು ಬಹಳ ವಿಸ್ತಾರವಾಗಿದೆ ಎಂದು ಭಾವಿಸುವೆವೊ ಅದು ಅವನಲ್ಲಿದೆ. ಆದಕಾರಣ ಅವನು ಅದನ್ನು ಒಳಕೊಳ್ಳುವಷ್ಟು ದೊಡ್ಡವನಾಗಿರಬೇಕು.

ಅವನು ಎಲ್ಲಕ್ಕೂ ಧಾತೃ ಎಂದರೆ ಜೀವರಾಶಿಗಳಿಗೆ ಕರ್ಮಫಲವನ್ನು ಕೊಡುವವನು. ಪ್ರತಿ ಯೊಂದು ಪ್ರಾಣಿಗೂ ಕಾಲಕಾಲಕ್ಕೆ ತಕ್ಕ ಆಹಾರವನ್ನು ಕೊಡುವವನು ಅವನು. ಮನುಷ್ಯ ತಾನೆ ಬೆಳೆ ಬೆಳೆಸುತ್ತಾನೆ. ಆದರೂ ಆ ಬೆಳೆ ಬೆಳೆಯಬೇಕಾದರೆ ಮಳೆ ಬರಬೇಕು. ಮಳೆಯನ್ನು ತರುವುದಕ್ಕೆ ಆಗುವುದಿಲ್ಲ ಮನುಷ್ಯನಿಗೆ. ಕೆಲವು ವೇಳೆ ಪ್ರಯೋಗ ಶಾಲೆಯಲ್ಲಿ ಕೆಲವು ಹನಿಗಳನ್ನು ಅವನು ಉತ್ಪತ್ತಿ ಮಾಡಬಹುದಷ್ಟೆ. ಮನುಷ್ಯ ಇನ್ನೂ ತನ್ನ ಇಚ್ಛಾನುಸಾರ ಮಳೆ ಬೀಳುವಂತೆ ಮಾಡಲಾರ. ಆದರೆ ಪಶುಪಕ್ಷಿಗಳು, ತೃಣಕೀಟಗಳು ಅವುಗಳಿಗೆ ದೇವರು ಆಹಾರವನ್ನು ಹೇಗೆ ಒದಗಿಸುತ್ತಾನೆ! ಇದೊಂದು ಅದ್ಭುತ. ಎಂತಹ ಪುಕ್ಕಗಳನ್ನು ಕೊಟ್ಟಿದ್ದಾನೆ ಹಾರುವ ಹಕ್ಕಿಗಳಿಗೆ, ಎಂತಹ ಬಣ್ಣವನ್ನು ಕೊಟ್ಟಿದ್ದಾನೆ, ಎಂತಹ ನಕ್ಷೆಯನ್ನು ಬರೆದಿದ್ದಾನೆ ಅವುಗಳ ಮೇಲೆ! ಮನುಷ್ಯ ಇವುಗಳಿಂದ ಕಲಿಯಬೇಕು. ಯಾವುದೋ ಚಿಟ್ಟೆಯೊ ಹಕ್ಕಿಯೊ ಅವುಗಳ ಬೆನ್ನಮೇಲಿರುವ ನಕ್ಷೆಯನ್ನು ನಮ್ಮ ಬಟ್ಟೆಯ ಮೇಲೆ ಅನುಕರಿಸಿದಾಗ ಅದೊಂದು ದೊಡ್ಡ ಫ್ಯಾಶನ್ ಆಗುವುದು. ಅವು ಉಳುವುದಿಲ್ಲ, ಬಿತ್ತುವುದಿಲ್ಲ. ಆದರೂ ಎಂತಹ ಊಟ ಅವಕ್ಕೆ ಕಾದಿದೆ! ನಮ್ಮಂತೆ ಮನೆಯನ್ನು ಕಟ್ಟುವುದಿಲ್ಲ. ಆದರೂ ಎಂತಹ ಸ್ಥಳಗಳಿವೆ ಅವಕ್ಕೆ ವಾಸಮಾಡುವುದಕ್ಕೆ. ದೇವರೇ, ಸೃಷ್ಟಿಸಿದಾತನೆ ಇವುಗಳ ಜವಾಬ್ದಾರಿಯನ್ನೆಲ್ಲ ತೆಗೆದುಕೊಂಡಿರುವುದರಿಂದ ಎಲ್ಲವನ್ನೂ ಅಷ್ಟು ಚೆನ್ನಾಗಿ ಅಣಿಮಾಡಿರುವನು ಅವಕ್ಕೆ.

ಅವನ ರೂಪ ಅಚಿಂತ್ಯ. ನಾವು ಕಲ್ಪಿಸಿಕೊಳ್ಳುವುದು, ಆಲೋಚಿಸಿಕೊಳ್ಳುವುದು, ನಾವು ನೋಡು ವುದು ಇವುಗಳನ್ನೆಲ್ಲ ಅವನು ಮೀರಿರುವನು. ಇವುಗಳೆಲ್ಲ ಸಾಂತ ಪ್ರಪಂಚದಲ್ಲಿ ಇವೆ. ಇವುಗಳನ್ನೆಲ್ಲ ನಮ್ಮ ಪಂಚೇಂದ್ರಿಯಗಳು ಎಂಬ ಬಲೆಯಲ್ಲಿ ಹಿಡಿದಿರುವೆವು. ನಮ್ಮ ಚಿಂತನೆಯ ಬಲೆಯಲ್ಲಿ ಪರಮಾತ್ಮನನ್ನು ಹಿಡಿಯುವುದಕ್ಕೆ ಆಗುವುದಿಲ್ಲ. ಅವನು ಅತೀಂದ್ರಿಯ ವಸ್ತು. ಅತೀಂದ್ರಿಯದ ವಸ್ತುವನ್ನು ಇಂದ್ರಿಯದ ಪಾತ್ರೆಯಲ್ಲಿ ಅಳೆಯುವುದಕ್ಕೆ ಆಗುವುದಿಲ್ಲ. ನಾವು ಅದನ್ನು ಅನುಭವಿಸ ಬಹುದು. ಆದರೆ ಅನುಭವಿಸಿದ್ದನ್ನು ಇನ್ನೊಬ್ಬರಿಗೆ ಹೇಳಲಾರೆವು. ನಮ್ಮ ಭಾಷೆಯ ಮೂಲಕ ಅದನ್ನು ವಿವರಿಸಲು ಅಸಾಧ್ಯ. ಮಾತಿನಿಂದ ಅವನನ್ನು ವಿವರಿಸುವುದಕ್ಕೆ ಆಗುವುದಿಲ್ಲ ಎಂದು ಮಾತ್ರ ಹೇಳಬಹುದು.

ಅವನು ಆದಿತ್ಯನಂತೆ ಸ್ವಯಂಪ್ರಕಾಶಮಾನನು. ಸೂರ್ಯ ಯಾರಿಂದಲೂ ಬೆಳಕನ್ನು ಸಾಲವಾಗಿ ತೆಗೆದುಕೊಂಡು ಬೆಳಗುವುದಿಲ್ಲ. ಅವನೇ ಎಲ್ಲರಿಗೂ ಬೆಳಕನ್ನು ಕೊಡುತ್ತಾನೆ. ಅವನು ಸ್ವಯಂ ಜ್ಯೋತಿ ಸ್ವರೂಪ. ಅವನು ಅಜ್ಞಾನದ ಆಚೆ ಇರುವನು. ಅಜ್ಞಾನದಲ್ಲಿರುವವನು ಅವನನ್ನು ನೋಡಲಾರನು. ಅಜ್ಞಾನದಲ್ಲಿದ್ದರೆ ಅವನು ಇಲ್ಲ ಎಂದಲ್ಲ ಅರ್ಥ. ಅಜ್ಞಾನದಲ್ಲಿದ್ದರೆ ಅವನು ಕಾಣುವುದಿಲ್ಲ ಎಂದರ್ಥ. ನಾವು ಕಣ್ಣು ಮುಚ್ಚಿಕೊಂಡರೆ ಸೂರ್ಯ ಕಾಣುವುದಿಲ್ಲ. ಆದರೆ ಸೂರ್ಯ ಇಲ್ಲದೇ ಹೋಗುವುದಿಲ್ಲ.

ಮೇಲೆ ಹೇಳಿದ ಗುಣಗಳಿಂದ ಕೂಡಿದ ಪರಮಾತ್ಮನ ಮೇಲೆ ಯಾರು ಮನಸ್ಸನ್ನು ಕೇಂದ್ರೀ ಕರಿಸಿರುವನೊ ಅವನು ಆ ಪವಿತ್ರವಾದ ಪರಮಾತ್ಮನನ್ನು ಸೇರುತ್ತಾನೆ.

\begin{verse}
ಯದಕ್ಷರಂ ವೇದವಿದೋ ವದಂತಿ ವಿಶಂತಿ ಯದ್ಯತಯೋ ವೀತರಾಗಾಃ~।\\ಯದಿಚ್ಛಂತೋ ಬ್ರಹ್ಮಚರ್ಯಂ ಚರಂತಿ ತತ್ತೇ ಪದಂ ಸಂಗ್ರಹೇಣ ಪ್ರವಕ್ಷ್ಯೇ \versenum{॥ ೧೧~॥}
\end{verse}

{\small ವೇದವನ್ನು ಬಲ್ಲವರು ಯಾವುದನ್ನು ಅಕ್ಷರ ಎಂದು ಹೇಳುತ್ತಾರೆಯೊ, ವೀತರಾಗಿಗಳಾದ ಮುನಿಗಳು ಯಾವುದನ್ನು ಪ್ರವೇಶಿಸುವರೋ, ಯಾವುದನ್ನು ಪಡೆಯುವುದಕ್ಕಾಗಿ ಜನ ಬ್ರಹ್ಮಚರ್ಯ ವ್ರತವನ್ನು ಪರಿಪಾಲನೆ ಮಾಡುವರೊ, ಆ ಪದದ ಸಂಕ್ಷಿಪ್ತ ವರ್ಣನೆಯನ್ನು ನಾನು ನಿನಗೆ ಹೇಳುತ್ತೇನೆ.}

ಅಚಿಂತ್ಯ ರೂಪ ಎಂದು ಹಿಂದಿನ ಶ್ಲೋಕದಲ್ಲಿ ಹೇಳಿ, ಇಲ್ಲಿ ನಮ್ಮ ಮನಸ್ಸಿಗೆ ಸ್ವಲ್ಪ ಅದರ ಅಸ್ಪಷ್ಟ ರೂಪವನ್ನು ಕೊಡಲು ಯತ್ನಿಸುವನು. ಚಿಂತಿಸುವುದಕ್ಕೆ ಆಗುವುದಿಲ್ಲ, ಎಂದರೆ ಮನುಷ್ಯ ಸುಮ್ಮನೆ ಇರುವುದಿಲ್ಲ. ನಮ್ಮ ಚಿಂತನೆಯ ಯಂತ್ರ ಕೆಲಸ ಮಾಡಿಯೇ ಮಾಡುವುದು. ಅದಕ್ಕೆ ಆಹಾರವನ್ನು ಒದಗಿಸಬೇಕಾಗಿದೆ. ಅದು ಹಾಗೆಯೇ ಎಂದು ಹೇಳುವುದಕ್ಕೆ ಆಗುವುದಿಲ್ಲ. ಆದರೆ, ಅದರಂತೆ ಇರಬಹುದು, ಇದರಂತೆ ಇರಬಹುದು ಎಂದು ಕೆಲವು ಗುಣವಾಚಕಗಳ ಮೂಲಕ ನಮ್ಮ ಗ್ರಹಿಕೆಗೆ ಸ್ವಲ್ಪ ಬರುವಂತೆ ಮಾಡಲೆತ್ನಿಸುವನು.

ವೇದಗಳನ್ನು ಬಲ್ಲವರು ಬ್ರಹ್ಮನ ವಿಷಯದಲ್ಲಿ ಅವನನ್ನು ಅಕ್ಷರ ಎನ್ನುತ್ತಾರೆ. ಅವನು ನಾಶವಿಲ್ಲದವನು. ನಮ್ಮ ಕಣ್ಣೆದುರಿಗೆ ಕಾಣುವ ದೃಶ್ಯ ಪ್ರಪಂಚವೆ ಇಲ್ಲದೆ ಹೋಗಬಹುದು. ಆದರೆ ಅವನು ನಾಶವಾಗುವುದಿಲ್ಲ. ಈ ಪ್ರಪಂಚದಲ್ಲಿ ದೇಶ, ಕಾಲ, ನಿಮಿತ್ತದಲ್ಲಿರುವುದೆಲ್ಲ ನಾಶವಾಗು ವುದು. ಆದರೆ ಅದನ್ನು ಅತಿಕ್ರಮಿಸಿರುವವನು ದೇಶ ಕಾಲ ನಿಮಿತ್ತದ ಪ್ರಪಂಚದ ಧರ್ಮಗಳಿಂದ ಬಾಧಿತನಾಗುವುದಿಲ್ಲ.

ಎಲ್ಲಾ ಇಂದ್ರಿಯಗಳನ್ನೂ ನಿಗ್ರಹಿಸಿದ ಮುನಿಗಳು ವಿಷಯವಸ್ತುವಿನ ಕಡೆ ಗಮನವನ್ನೇ ಕೊಡುವುದಿಲ್ಲ. ಆದರೆ ಅವನು ಗಮನಿಸುವ ಒಂದು ವಸ್ತುವಿದೆ. ಅದೇ ಪರಬ್ರಹ್ಮ. ಅವನನ್ನು ಪಡೆಯಲು ಉಳಿದ ಎಲ್ಲವನ್ನೂ ತಿರಸ್ಕರಿಸಿರುವರು. ಅವನನ್ನು ರುಚಿ ನೋಡಲು ಉಳಿದ ಎಲ್ಲವನ್ನೂ ತಿರಸ್ಕರಿಸಿರುವರು. ಆ ಭೂಮವನ್ನು ಪಡೆಯುವುದಕ್ಕೆ ಅಲ್ಪಸ್ವಲ್ಪವನ್ನೆಲ್ಲ ನಿರಾಕರಿಸಿರುವರು.

ಬ್ರಹ್ಮಚರ್ಯಪಾಲನೆ ಮಾಡುವುದು ನಮ್ಮ ಶಕ್ತಿಯನ್ನು ನಿಗ್ರಹಿಸುವುದಕ್ಕಾಗಿ. ಆ ನಿಗ್ರಹಿಸಿದ ಶಕ್ತಿಯನ್ನು ನಾವು ದೇವರ ಕಡೆ ಕಳುಹಿಸಬೇಕಾಗಿದೆ. ಇಲ್ಲಿ ನಿಗ್ರಹಿಸಿದ ಶಕ್ತಿಯೇ ದೇವರ ಕಡೆಗೆ ನಮಗೆ ಹೋಗಲು ಸಹಾಯಕವಾಗಿರುವುದು. ಕೆಳಗೆ ಹೋಗುವ ಶಕ್ತಿಯನ್ನು ನಿಗ್ರಹಿಸಿದರೆ ಅದು ಮೇಲಕ್ಕೆ ಏಳುವುದು. ಇಂದ್ರಿಯ ವಿಷಯದ ಕಡೆ ಹರಿಯುವ ಶಕ್ತಿಯನ್ನು ನಿಗ್ರಹಿಸಿದಾಗ ಅದು ದೇವರ ಕಡೆ ಹೋಗುವುದು. ಸಾಧಕ ಶಕ್ತಿ ಸಂಗ್ರಹಿಸುವುದು ಬರೀ ಸಂಗ್ರಹಕ್ಕಲ್ಲ. ಅದರಿಂದ ಮತ್ತೇನನ್ನೋ ಪಡೆಯಲು ಕೂಡಿಡುವನು. ಆ ಪಡೆಯಬೇಕಾದ ವಸ್ತುವೇ ಪರಬ್ರಹ್ಮ. ಆ ಸ್ಥಿತಿಯ ಸಂಕ್ಷೇಪ ವರ್ಣನೆ ಕೊಡುತ್ತೇನೆ ಎನ್ನುತ್ತಾನೆ. ಇದೊಂದು ಭಾವಚಿತ್ರವಲ್ಲ \enginline{(Photo)}, ಮೂಲವಸ್ತು ವನ್ನು ಊಹಿಸಿಕೊಳ್ಳುವುದಕ್ಕೆ ಕೊಡುವ ಸಲಹೆಗಳು ಅಷ್ಟೆ.

\begin{verse}
ಸರ್ವದ್ವಾರಾಣಿ ಸಂಯಮ್ಯ ಮನೋ ಹೃದಿ ನಿರುಧ್ಯ ಚ~।\\ಮೂರ್ಧ್ನ್ಯಾಧಾಯಾತ್ಮನಃ ಪ್ರಾಣಮಾಸ್ಥಿತೋ ಯೋಗಧಾರಣಾಮ್ \versenum{॥ ೧೨~॥}
\end{verse}

\begin{verse}
ಓಮಿತ್ಯೇಕಾಕ್ಷರಂ ಬ್ರಹ್ಮ ವ್ಯಾಹರನ್ ಮಾಮನುಸ್ಮರನ್~।\\ಯಃ ಪ್ರಯಾತಿ ತ್ಯಜನ್ ದೇಹಂ ಸ ಯಾತಿ ಪರಮಾಂ ಗತಿಮ್ \versenum{॥ ೧೩~॥}
\end{verse}

{\small ಇಂದ್ರಿಯಗಳ ಸರ್ವದ್ವಾರಗಳನ್ನು ತಡೆದು, ಮನಸ್ಸನ್ನು ಹೃದಯದಲ್ಲಿ ಸ್ಥಿರಗೊಳಿಸಿ ಪ್ರಾಣವಾಯುವನ್ನು ಶಿರಕ್ಕೆ ಏರಿಸಿ, ಸಮಾಧಿಸ್ಥನಾಗಿ, ‘ಓಂ’ ಎಂಬ ಏಕಾಕ್ಷರ ಬ್ರಹ್ಮದ ಉಚ್ಚಾರಣೆ ಮಾಡುತ್ತಾ ನನ್ನನ್ನು ಸ್ಮರಿಸುತ್ತ ಯಾವ ಮಾನವ ದೇಹತ್ಯಾಗ ಮಾಡುವನೊ ಅವನಿಗೆ ಪರಮಗತಿ ಸಿಕ್ಕುತ್ತದೆ.}

ಸಂಯಮ ಎಂಬುದು ಇಂದ್ರಿಯಗಳನ್ನು ಸ್ಥೂಲವಾಗಿ ಮತ್ತು ಸೂಕ್ಷ್ಮವಾಗಿ ನಿಗ್ರಹಿಸುವುದಕ್ಕೆ ಅನ್ವಯಿಸುವುದು. ಮೊದಲು ಸ್ಥೂಲವಾಗಿ, ಇಂದ್ರಿಯಗಳು ಅದರ ವಿಷಯಗಳ ಕಡೆ ಹೋಗದಂತೆ ನೋಡಿಕೊಳ್ಳಬೇಕು. ಅನಂತರ ಮನಸ್ಸಿನಲ್ಲಿರುವ ಸೂಕ್ಷ್ಮ ವಾಸನೆಗಳನ್ನು ಅದಕ್ಕೆ ವಿರೋಧವಾದ ಭಾವನೆಗಳಿಂದ ತೊಳೆಯಬೇಕು. ಇಂದ್ರಿಯಗಳಲ್ಲಿ ಒಂದರಷ್ಟೇ ಮತ್ತೊಂದು ನಮ್ಮನ್ನು ಪತನಕ್ಕೆ ಒಯ್ಯುವುದು. ಒಂದು ಬಾಗಿಲನ್ನು ಮುಚ್ಚಿದರೆ ಸಾಲದು. ಮತ್ತೊಂದು ಬಾಗಿಲಿನ ಮೂಲಕ ಹೆಚ್ಚಾಗಿ ಹೋಗುವುದು. ಆದಕಾರಣವೇ ಹೊರಗೆ ಹೋಗುವ ಎಲ್ಲಾ ಬಾಗಿಲುಗಳನ್ನು ಮುಚ್ಚಬೇಕು.

ಮನಸ್ಸನ್ನು ಹೃದಯದಲ್ಲಿ ನೆಲೆಸುವಂತೆ ಮಾಡಬೇಕು. ಅದನ್ನು ಅಂತರ್ಮುಖ ಮಾಡಬೇಕು. ದೇವರೆಲ್ಲೊ ಹೊರಗೆ ಸಿಕ್ಕುವವನಲ್ಲ. ಅವನನ್ನು ತಡಕಬೇಕಾದರೆ ಒಳಗೆ ಮುಳುಗಬೇಕು. ಅದಕ್ಕೆ ಯೋಗಿಗಳು ಧ್ಯಾನ ಮಾಡುವಾಗ ಕಣ್ಣನ್ನು ಮುಚ್ಚಿಕೊಂಡಿರುವರು. ಅವರೆಲ್ಲೊ ಆಕಾಶದ ಕಡೆ ನೋಡುತ್ತಿರವವರಲ್ಲ. ಹೊರಗಿಗಿಂತ ಆಳವಾಗಿರುವುದು ಒಳಗೆ. ಹೊರಗಿನದು ಒಳಗಿನ ಎಲ್ಲೋ ಒಂದು ಸಣ್ಣ ಅಂಶ. ಅದಕ್ಕೆ ಯೋಗಿ ತನ್ನಲ್ಲಿ ಹಿಂದೆ ಹೋಗುವನು.

ಪ್ರಾಣವಾಯುವು ಉಳಿದ ಕಾಲದಲ್ಲಿ ದೇಹದ ಎಲ್ಲಾ ಕಡೆಯೂ ವ್ಯಾಪಿಸಿಕೊಂಡಿರುವುದು. ಅಂತ್ಯ ಕಾಲದಲ್ಲಿ ಯೋಗಿ ಅದನ್ನೆಲ್ಲ ಸೆಳೆದುಕೊಂಡು ಶಿರದಲ್ಲಿ ನಿಲ್ಲಿಸುತ್ತಾನೆ. ಆಕಾಶವಿಮಾನ ಮೇಲೆ ಹಾರಿಹೋಗುವುದಕ್ಕೆ ಮುಂಚೆ ವಿಮಾನ ನಿಲ್ದಾಣದಲ್ಲಿರುವಂತೆ, ಪ್ರಾಣ ಶಿರೋಮಧ್ಯದಲ್ಲಿರುವುದು ಈ ದೇಹವನ್ನು ಬಿಡುವುದಕ್ಕೆ ಮುಂಚೆ. ಆಗ ಓಂಕಾರವನ್ನು ಮನದಲ್ಲಿಯೇ ಉಚ್ಚರಿಸುತ್ತಿರುವನು. ಓಂಕಾರವೇ ಶಬ್ದ ಬ್ರಹ್ಮ. ಅದನ್ನು ಸಂಕ್ಷಿಪ್ತ ಬ್ರಹ್ಮವೆಂದೂ ಕರೆಯಬಹುದು. ಅದು ಸಗುಣ- ನಿರ್ಗುಣ ಉಪಾಸನೆಗಳೆರಡಕ್ಕೂ ಸೇತುವೆಯಂತೆ ಇರುವುದು. ಓಂಕಾರವನ್ನು ಉಚ್ಚರಿಸುತ್ತ, ಅದಕ್ಕೆ ಸಂಬಂಧಪಟ್ಟ ಭಾವನೆಯನ್ನು ಚಿಂತಿಸುತ್ತ, ಪ್ರಾಣವನ್ನು ತ್ಯಜಿಸಿದವನು ಶ್ರೇಷ್ಠವಾದ ಗತಿಯನ್ನು ಪಡೆಯುತ್ತಾನೆ. ಶ್ರೇಷ್ಠವಾದ ಗತಿ ಎಂದರೆ ಹಿಂದಿರುಗಿ ಬರದೆ ಹೋಗುವ ಸ್ಥಿತಿ. ಅವನು ಹುರಿದ ಬೀಜದಂತೆ ಆಗುತ್ತಾನೆ. ಸಂಸ್ಕಾರಗಳೆಲ್ಲ ದಗ್ಧವಾಗಿ ಹೋಗುವುವು. ಮತ್ತೊಮ್ಮೆ ಜನ್ಮವೆತ್ತಿ ಬರುವುದಕ್ಕೆ ಬೇಕಾದ ಆಸೆಯ ಮೊಳಕೆಯೇ ಅಲ್ಲಿರುವುದಿಲ್ಲ. ಅವನು ಮುಕ್ತನಾಗುತ್ತಾನೆ.

\begin{verse}
ಅನನ್ಯಚೇತಾಃ ಸತತಂ ಯೋ ಮಾಂ ಸ್ಮರತಿ ನಿತ್ಯಶಃ~।\\ತಸ್ಯಾಹಂ ಸುಲಭಃ ಪಾರ್ಥ ನಿತ್ಯಯುಕ್ತಸ್ಯ ಯೋಗಿನಃ \versenum{॥ ೧೪~॥}
\end{verse}

{\small ಅರ್ಜುನ, ಚಿತ್ತವನ್ನು ಮತ್ತೆಲ್ಲಿಯೂ ಇಡದೆ ಯಾರು ನಿತ್ಯ ನಿರಂತರ ನನ್ನ ಸ್ಮರಣೆಯನ್ನು ಮಾಡುತ್ತಾನೆಯೊ, ಆ ನಿತ್ಯಯುಕ್ತನಾದ ಯೋಗಿ ನನ್ನನ್ನು ಸಹಜವಾಗಿ ಪಡೆಯುತ್ತಾನೆ.}

ಅಂತ್ಯ ಕಾಲದಲ್ಲಿ ಭಗವಂತನನ್ನು ಸ್ಮರಿಸುತ್ತ ಈ ಪ್ರಪಂಚವನ್ನು ತ್ಯಜಿಸಬೇಕಾದರೆ, ಬದುಕಿರು ವಾಗ ಚಿತ್ತವನ್ನು ಭಗವಂತನ ಮೇಲೆ ಇಟ್ಟಿರಬೇಕು. ಅಲ್ಲಿ ದೇವರಲ್ಲದ ಅನ್ಯ ವಿಷಯಕ್ಕೆ ಅವಕಾಶ ಇರಕೂಡದು. ಅನ್ಯ ವಿಷಯವಿದ್ದರೂ ಅದು ದೇವರಿಗಾಗಿ ಇರಬೇಕು. ಜೀವನದ ದ್ವಂದ್ವ ಅನುಭವ ಗಳ ಮೂಲಕ ಹೋಗುತ್ತಿರುವಾಗ ಅವನನ್ನು ಮಾತ್ರ ಅನುಗಾಲವೂ ಚಿಂತಿಸುತ್ತಿರುವನು ಭಕ್ತ. ಎಂತಹ ದುಃಖ ಬಂದರೂ ಭಗವಂತನನ್ನು ಮರೆಯುವುದಿಲ್ಲ. ಎಂತಹ ಸುಖ ಬಂದರೂ ಅವನನ್ನು ಮರೆಯುವುದಿಲ್ಲ.

ನಿತ್ಯವೂ ಅವನನ್ನು ಚಿಂತಿಸುತ್ತಿರಬೇಕು. ಭಗವಂತನನ್ನು ಚಿಂತಿಸುವುದಕ್ಕೆ ರಜಾ ಇಲ್ಲ. ಜೀವನ ದಲ್ಲಿ ಆಫೀಸಿಗೆ ಹೋಗುವುದಕ್ಕೆ, ಶಾಲೆಗೆ ಹೋಗುವುದಕ್ಕೆ ರಜಾ ಇದೆ. ಆದರೆ ಭೂಮಿ ಸೂರ್ಯನ ಸುತ್ತಲೂ ಸುತ್ತುವುದಕ್ಕೆ ರಜಾ ಇಲ್ಲ. ಅದರಂತೆಯೇ ಯಾವಜ್ಜೀವನವೂ ಭಗವಂತನ ಕಡೆ ಮನಸ್ಸು, ನದಿ ಸಾಗರಕ್ಕೆ ಸೇರುವಂತೆ ಅವನ ಕಡೆಗೆ ತಿರುಗಿರಬೇಕು. ಭಗವಂತನ ಚಿಂತೆ ನಮ್ಮ ನಿತ್ಯ ವ್ಯವಹಾರಕ್ಕೇ ಹಿನ್ನೆಲೆಯಾಗಿರಬೇಕು.

ಹೀಗೆ ನಿತ್ಯವೂ ದೇವರೊಡನೆ ಕೂಡಿಕೊಂಡಿರುವ ಯೋಗಿಗೆ ಕೊನೆಗಾಲದಲ್ಲಿ ದೇವರಲ್ಲಿ ಸೇರುವುದು ಸುಲಭವಾಗುವುದು. ಇದನ್ನು ಮುಂಚೆ ಒಂದು ಬಲವಾದ ಅಭ್ಯಾಸವನ್ನಾಗಿ ಮಾಡಿ ಕೊಂಡಿದ್ದರೆ ಕೊನೆಗಾಲದಲ್ಲಿ ಸುಲಭವಾಗಿ ಅವನನ್ನು ಕುರಿತು ಚಿಂತಿಸಬಹುದು. ಬದುಕಿರುವಾಗ ಅಭ್ಯಾಸವಿಲ್ಲದೆ ಕೊನೆಗಳಿಗೆಯಲ್ಲಿ ಭಗವಂತನನ್ನು ಸ್ಮರಿಸುವುದಕ್ಕೆ ಆಗುವುದಿಲ್ಲ.

\begin{verse}
ಮಾಮುಪೇತ್ಯ ಪುನರ್ಜನ್ಮ ದುಃಖಾಲಯಮಶಾಶ್ವತಮ್~।\\ನಾಪ್ನುವಂತಿ ಮಹಾತ್ಮಾನಃ ಸಂಸಿದ್ಧಿಂ ಪರಮಾಂ ಗತಾಃ \versenum{॥ ೧೫~॥}
\end{verse}

{\small ನನ್ನನ್ನು ಸೇರಿ ಪರಮಗತಿಯನ್ನು ಹೊಂದಿರುವ ಮಹಾತ್ಮರು, ದುಃಖದ ಮನೆಯೂ, ಅಶಾಶ್ವತವೂ ಆದ ಪುನರ್ಜನ್ಮವನ್ನು ಮತ್ತೆ ಹೊಂದುವುದಿಲ್ಲ.}

ದೇವರನ್ನು ಮುಟ್ಟಿದವರು ಪ್ರಪಂಚಕ್ಕೆ ಇನ್ನು ಮೇಲೆ ಬರುವುದಿಲ್ಲ. ಆವಿಗೆಯ ಮನೆಯಲ್ಲಿ ಬೆಂದ ಮಡಕೆಯನ್ನು ಪುನಃ ಚಕ್ರದ ಮೇಲೆ ತರುವುದಕ್ಕೆ ಆಗುವುದಿಲ್ಲ. ದೇವರನ್ನು ಮುಟ್ಟಿದ ಜೀವಿ ಸಂಸಾರದ ಬಾಂಡಲೆಯಿಂದ ಹಾರಿಹೋದವನು. ಇದನ್ನು ಪರಮಗತಿ ಎನ್ನುತ್ತಾರೆ. ಯಾವ ಗತಿ ಯನ್ನು ಸೇರಿದ ಮೇಲೆ, ಇನ್ನು ಹೆಚ್ಚಾಗಿ ಪಡೆಯುವುದಕ್ಕೆ ಯಾವುದೂ ಇಲ್ಲವೊ ಮತ್ತು ನಾವು ಹಿಂದಕ್ಕೆ ಬರಲಾರೆವೊ ಅದು ಪರಮಗತಿ.

ಈ ಸಂಸಾರದ ಸ್ವಭಾವವನ್ನು ವಿವರಿಸುತ್ತಾನೆ. ಇದು ದುಃಖಾಲಯ. ಸಂಸಾರವೆಂದರೆ ದುಃಖದ ತೌರುಮನೆ. ಎಲ್ಲರಿಗೂ ಒಂದು ಬಗೆಯ ದುಃಖ ಕಾದಿದೆ ಪೀಡುಸುವುದಕ್ಕೆ. ದೇಹವನ್ನು ತೆಗೆದು ಕೊಂಡರೆ ಸಾಕು; ಒಬ್ಬೊಬ್ಬನಿಗೆ ಒಂದೊಂದು ಕೊರತೆ. ಬಡಪಾಯಿಗೆ ಊಟದ, ಬಟ್ಟೆಯ, ವಸತಿಯ ಚಿಂತೆ. ಭಾಗ್ಯವಂತನಿಗೆ ತನ್ನ ವ್ಯಾಪಾರದ, ಹುದ್ದೆಯ ಲಾಭದ ಚಿಂತೆ. ಒಂದಿರುವವನಿಗೆ ಮತ್ತೊಂದರ ಚಿಂತೆ. ದೇಹ ಬಲವಾಗಿದ್ದರೆ ತಿನ್ನುವುದಕ್ಕೆ ಏನು ಇಲ್ಲ. ತಿನ್ನುವುದಕ್ಕೆ ಸಮೃದ್ಧಿಯಾಗಿ ದ್ದರೆ, ಒಂದು ತಿಂದರೆ ಜಾಸ್ತಿ, ಒಂದು ತಿಂದರೆ ಕಡಿಮೆ. ಮಕ್ಕಳಿಲ್ಲದವರಿಗೆ ಒಂದು ಚಿಂತೆ. ಮಕ್ಕಳಿದ್ದವರಿಗೆ ಹಲವು ಚಿಂತೆ. ಒಂದು ಸರಿಯಾಗಿದ್ದರೆ ಮತ್ತೊಂದು ಸರಿಯಾಗಿಲ್ಲ. ಅನೇಕ ವೇಳೆ ನಾವೇನು ಭಾವಿಸುತ್ತೇವೆ ಎಂದರೆ ನಮ್ಮಷ್ಟು ದುಃಖಪಟ್ಟವರು ಇನ್ನು ಯಾರೂ ಇಲ್ಲ ಎಂದು. ಹಾಗೆಯೇ ಎಲ್ಲರೂ ಭಾವಿಸುತ್ತಾರೆ. ಆದರೆ ಎಲ್ಲರ ಮನೆಯ ದೋಸೆಯೂ ತೂತು ಎನ್ನುವಂತೆ, ಎಲ್ಲರ ಬಾಳಿನ ಕಥೆಯೂ ತೂತಿನಿಂದ ತುಂಬಿದೆ. ಈ ದುಃಖವಿಲ್ಲದೆ ಇದ್ದರೆ ಸಂಸಾರವೇ ಆಗುತ್ತಿರಲಿಲ್ಲ. ಆದರೆ ಇಲ್ಲಿ ದುಃಖದ ವೈವಿಧ್ಯತೆ ಇದೆ. ಒಂದಾದ ಮೇಲೆ ಮತ್ತೊಂದು ಕಾದಿದೆ. ಒಂದು ದುಃಖದಿಂದ ಪಾರಾಗಿ, ಸಧ್ಯಕ್ಕೆ ಈಗ ನೆಮ್ಮದಿಯಾಗಿರುವೆ ಎನ್ನುವ ಹೊತ್ತಿಗೆ ಮತ್ತೊಂದು ದುಃಖ ನಮ್ಮ ಮೇಲೆ ಬೀಳುವುದಕ್ಕೆ ಅಣಿಯಾಗುತ್ತದೆ. ಇದು ಸಮುದ್ರದ ಮೇಲಿರುವ ಅಲೆಯಂತೆ. ಒಂದು ಅಲೆ ಬೀಳುವುದು, ಮತ್ತೊಂದು ಅಲೆ ಏಳುವುದು. ಈ ದುಃಖವನ್ನು ದೇವರು ಬೇಕೆಂತಲೆ ಇಲ್ಲಿ ಇಟ್ಟಿದ್ದಾನೆ. ಇಲ್ಲದೇ ಇದ್ದರೆ ಜೀವ ಮಾಗುವುದು ಹೇಗೆ? ಈ ಪ್ರಪಂಚದ ದುಃಖ ನಮಗೆ ತಾಕಿದಾಗಲೆ, ಅದರ ಚುರುಕು ತಾಕಿ, ನಾವು ಇಲ್ಲಿಂದ ಪಾರಾಗಲು ಯತ್ನಿಸಬೇಕಾದರೆ. ಬರೀ ಸುಖವೇ ಇದ್ದರೆ ಯಾರು ಪ್ರಪಂಚವನ್ನು ಬಿಡಲು ಯತ್ನಿಸುವರು? ದೇವರ ಕಡೆಗೆ ಬರಲು ಯಾರು ತವಕಪಡುವರು? ಉರಿಯುತ್ತಿರುವ ಒಲೆ ಮೇಲೆ ತಾನೆ ಬೀಜವನ್ನು ಹುರಿಯಬೇಕಾದರೆ. ಹುರಿಯು ವಾಗ ಒಮ್ಮೆ ಬಾಂಡಲೆಯಲ್ಲಿ ಮೇಲುಭಾಗಕ್ಕೆ ಬರುವುದು. ಆಗ, ಬೀಜ ಹಾಯಾಗಿದೆ ಎಂದು ಭಾವಿಸುವುದು. ಆದರೆ ಹುರಿಯುವವನು ಮೊಗಚೆಯಕಾಯಿಯನ್ನು ತೆಗೆದುಕೊಂಡು ಮೇಲಿರುವು ದನ್ನು ಕೆಳಕ್ಕೆ ತರುತ್ತಾನೆ, ಕೆಳಗಿರುವುದನ್ನು ಮೇಲಕ್ಕೆ ತರುತ್ತಾನೆ.

ಇದರ ಮತ್ತೊಂದು ಸ್ವಭಾವವೇ ಅಶಾಶ್ವತ. ಈಗ ನನಗೆ ಏನಾದರೂ ಚೂರು ಒಳ್ಳೆಯದು ಸಿಕ್ಕಿದೆ ಎಂದು ಭಾವಿಸೋಣ. ಅದು ಎಂದೆಂದಿಗೂ ನನ್ನ ಹತ್ತಿರ ಇರುವುದೆ? ಇಲ್ಲ. ಸ್ವಲ್ಪ ಹೊತ್ತಿನಲ್ಲಿಯೇ ಅದು ನನ್ನನ್ನು ಬಿಟ್ಟು ಹೋಗುವುದು. ಯಾರು ನಮಗೆ ತುಂಬಾ ಬೇಕಾಗಿರುವರೊ, ತುಂಬಾ ಪ್ರಿಯರಾಗಿರುವರೊ, ಯಾರಿದ್ದರೆ ಜೀವನಕ್ಕೆ ನಮಗೆ ಒಂದು ಬೆಲೆ ಬರುವುದೊ ಅವರು ನಮ್ಮನ್ನು ಅಗಲಿ ಹೋಗುವರು. ಇಂದಿನ ಐಶ್ವರ್ಯ ನಾಳೆ ನಮ್ಮದಲ್ಲ. ಆರೋಗ್ಯ ನಾಳೆ ನಮ್ಮದಲ್ಲ. ಎಂತಹ ಘಟಾನುಘಟಿ ಬಲಿಷ್ಠರು ಮತ್ತು ಭೀಮನಂತೆ ಇದ್ದವರೆಲ್ಲ ಎಂತಹ ಶೋಚನೀಯ ರೋಗಕ್ಕೆ ತುತ್ತಾಗಿ ನರಳಿ ಅಂಗುಲ ಅಂಗುಲ ಸಾಯುತ್ತಾ ವ್ಯಥೆಪಡುವರು. ಶ‍್ರೀಮಂತರು, ನಮ್ಮ ಸಮಾನ ಇಲ್ಲ ಎಂದು ಮೆರೆದವರು, ಒಂದು ತುತ್ತು ಅನ್ನಕ್ಕೆ ಮನೆ ಮನೆ ಅಲೆದಿರುವರು. ಈ ಜೀವನದಲ್ಲಿ ಹೆಮ್ಮೆ ಪಡುವುದಕ್ಕೆ ಏನಿದೆ ನಮಗೆ? ಆದರೂ ಮನುಷ್ಯ ಮೂಢ. ಅನುಭವದಿಂದ ಬುದ್ಧಿ ಕಲಿಯಲಾರ. ಹಿಂದೆ ದುಃಖದ ಅನುಭವ ಇದೆ. ಮುಂದೆ ಅನುಭವಿಸಬೇಕಾದ ದುಃಖ ಸಿದ್ಧವಾಗುತ್ತಿದೆ ನಮ್ಮ ಮೇಲೆ ಬೀಳುವುದಕ್ಕೆ. ಆದರೂ ಇವುಗಳ ಮಧ್ಯದಲ್ಲಿ ಒಂದು ಚೂರು ಸುಖ ಸಿಕ್ಕಿತು ಎಂದರೆ ಮನುಷ್ಯ ಎಲ್ಲವನ್ನೂ ಮರೆತು ಆ ಸುಖವನ್ನು ನೆಕ್ಕುವನು. ಮಹಾಭಾರತದಲ್ಲಿ ಇದನ್ನು ವಿವರಿಸುವ ಒಂದು ಕಥೆ ಇದೆ. ಒಬ್ಬ ಕಾಡುದಾರಿಯಲ್ಲಿ ಹೋಗುತ್ತಿದ್ದವನು ಒಂದು ಹಾಳು ಭಾವಿಗೆ ಎಡವಿ ಬೀಳುತ್ತಾನೆ. ಬೀಳುವಾಗ ಮರದ ಒಂದು ಬೇರನ್ನು ಹಿಡಿದುಕೊಂಡು ಮಧ್ಯದಲ್ಲಿ ಜೋಲಾಡುತ್ತಿರುವನು. ಮೇಲೆ ಎರಡು ಹೆಗ್ಗಣಗಳು ಇವನು ಹಿಡಿದಿರುವ ಬೇರನ್ನು ಕತ್ತರಿಸುತ್ತಿವೆ. ಭಾವಿಯ ಒಳಗಡೆ ದೊಡ್ಡದೊಂದು ಹೆಬ್ಬಾವು ಇವನನ್ನು ನುಂಗಲು ಬಾಯನ್ನು ತೆರೆದುಕೊಂಡಿದೆ. ಆಗ ಇವನ ಪಕ್ಕದಲ್ಲಿ ಜೇನುಗೂಡೊಂದರಿಂದ ಜೇನುತುಪ್ಪ ತೊಟ್ಟಿಕ್ಕುವುದು. ಅದನ್ನು ಚೀಪಲು ನಾಲಗೆಯನ್ನು ನೀಡುವನು. ನಮಗೆ ಸಿಕ್ಕುವ ಸುಖವೂ ಹೀಗೆಯೆ. ಮಾರಿಗೆ ಬಲಿಗೆಂದೇ ಒಂದು ಕುರಿಯನ್ನು ಹೊಡೆದುಕೊಂಡು ಹೋಗುತ್ತಿರುವರು. ಇದರ ಬಲಿಗೆಂದೇ ದಾರಿಯಲ್ಲಿ ತೋರಣ ಕಟ್ಟಿದ್ದರೆ ಕುರಿ ಅದನ್ನು ತಿನ್ನುವುದು. ಸಾಯುವುದಕ್ಕೆ ಮುಂಚೆ ಸ್ವಲ್ಪ ಫಲಾಹಾರ! ಈ ಸಂಸಾರದ ದುಃಖದ ಪರಿಚಯ ಇರುವ ಮನುಷ್ಯ ಇನ್ನೊಮ್ಮೆ ಇಲ್ಲಿಗೆ ಬರಲು ಇಚ್ಛಿಸುವುದಿಲ್ಲ. ಹೋದರೆ ಬರದ ರೀತಿಯಲ್ಲಿ ಹೋಗುವುದಕ್ಕೆ ಯತ್ನಿಸುತ್ತಾನೆ. ಹಾಗೆ ಬರದ ರೀತಿಯಲ್ಲಿ ಹೋಗಬೇಕಾದರೆ, ಭಗವಂತನಿಂದ ನಮ್ಮ ಮನಸ್ಸನ್ನು ತುಂಬಿಸಿರಬೇಕು. ಇದು ಸೋರೆಯ ಬುರುಡೆ ನೀರಿನಲ್ಲಿ ತೇಲುವಂತೆ ನಮ್ಮನ್ನು ಸಂಸಾರದಲ್ಲಿ ತೇಲಿಸುವುದು ಮತ್ತು ಭಗವಂತನೆಡೆಗೆ ನಮ್ಮನ್ನು ಒಯ್ಯುವುದು.

\begin{verse}
ಆಬ್ರಹ್ಮಭುವನಾಲ್ಲೋಕಾಃ ಪುನರಾವರ್ತಿನೋಽಜುRನ~।\\ಮಾಮುಪೇತ್ಯ ತು ಕೌಂತೇಯ ಪುನರ್ಜನ್ಮ ನ ವಿದ್ಯತೇ \versenum{॥ ೧೬~॥}
\end{verse}

{\small ಅರ್ಜುನ! ಬ್ರಹ್ಮಲೋಕ ಮೊದಲ್ಗೊಂಡು ಎಲ್ಲಾ ಲೋಕಗಳು ಬಂದು ಹೋಗತಕ್ಕವುಗಳು. ಆದರೆ ನನ್ನನ್ನು ಸೇರಿದ ಬಳಿಕ ಮನುಷ್ಯ ಮತ್ತೆ ಜನ್ಮವೆತ್ತಬೇಕಾಗಿಲ್ಲ.}

ಹಿಂದಿನ ಕಾಲದಲ್ಲಿ, ಪುರಾಣದಲ್ಲಿ, ಈ ಭೂಮಿ ಮಧ್ಯದಲ್ಲಿರುವುದಾಗಿಯೂ, ಅದರ ಮೇಲೆ ಏಳು ಲೋಕಗಳು ಕೆಳಗೆ ಏಳು ಲೋಕಗಳು ಇರುವುವು ಎಂದೂ ಹೇಳುತ್ತಾರೆ. ಆದರೆ ಈಗಿನ ಖಗೋಳ ಶಾಸ್ತ್ರದ ಭೂಮ ಕಲ್ಪನೆಯ ಎದುರಿಗೆ ನಮ್ಮ ಪುರಾಣವೂ ಕುಬ್ಜವಾಗುವುದು. ಆಕಾಶದಲ್ಲಿ ರುವ ನಕ್ಷತ್ರಗಳೆಲ್ಲ ಒಂದೊಂದು ಸೂರ್ಯಲೋಕವಾಗಿವೆ. ಆ ಸೂರ್ಯರಿಗೆ ಎಷ್ಟೋ ಗ್ರಹಗಳಿರುವುವು. ಆ ಗ್ರಹಗಳಲ್ಲಿ ಕೆಲವದರಲ್ಲಾದರೂ ಮನುಷ್ಯರು ಇರುವುದಕ್ಕೆ ಸಾಧ್ಯವಾದರೆ ಲಕ್ಷಾಂತರ ಲೋಕಗಳು ಆಗುವುವು. ಆದರೆ ಎಲ್ಲಿಗೆ ಹೋದರೂ ನಾವು ಅಲ್ಲಿ ಕರ್ಮವನ್ನು ಅನುಭವಿಸಿ ಆದಮೇಲೆ ಪುನಃ ಇಲ್ಲಿಗೆ ಬರಬೇಕಾಗಿದೆ. ಇಲ್ಲಿ ಫಲಾಪೇಕ್ಷೆಯಿಂದ ಪುಣ್ಯಕೆಲಸಗಳನ್ನು ಮಾಡಿದರೆ ಬೇರೆ ಲೋಕಕ್ಕೆ ಹೋಗಿ ಪುಣ್ಯ ಅನುಭವಿಸುತ್ತಾನೆ. ಆದರೆ ಎಂದೆಂದಿಗೂ ಅನುಭವಿಸುತ್ತಿರುವುದಕ್ಕೆ ಆಗುವುದಿಲ್ಲ. ನಮ್ಮ ಪುಣ್ಯತೀರಿದ ಮೇಲೆ ನಮ್ಮನ್ನು ಅಲ್ಲಿಂದ ಕೆಳಗೆ ಅಟ್ಟುವರು. ನಾವೊಂದು ನಾಟಕ ನೋಡಲು ಹೋದರೆ ಅದು ಮುಗಿದಾದ ಮೇಲೆ ನಾವು ಅಲ್ಲೇ ಇರುವುದಕ್ಕಾಗುವುದಿಲ್ಲ. ಪುನಃ ಮನೆಗೆ ಬರಬೇಕು. ಅದರಂತೆ ಬೇರೆ ಬೇರೆ ಲೋಕಗಳು. ಹಾಗೆಯೆ ಏನಾದರೂ ಪಾಪ ಮಾಡಿದರೆ, ಪಾಪ ಅನುಭವಿಸುವುದಕ್ಕೆ ಆ ಲೋಕಕ್ಕೆ ಹೋಗಿ ಅಲ್ಲಿ ಅದನ್ನು ಅನುಭವಿಸಿ ಆದಮೇಲೆ ಪುನಃ ನಾವು ಹಿಂತಿರುಗಬೇಕು.

ಬರುವುದು-ಹೋಗುವುದರಿಂದ ತಪ್ಪಿಸಿಕೊಳ್ಳಬೇಕಾದರೆ ದೇವರನ್ನು ಸೇರಬೇಕು. ಆಗ ಮಾತ್ರ ಕರ್ಮ ತನ್ನ ಸಮನ್ನನ್ನು ನನ್ನ ಮೇಲೆ ಜಾರಿ ಮಾಡಲಾರದು. ನಾವು ಈ ಪ್ರಪಂಚಕ್ಕೆ ಅತೀತರಾಗಿ ಹೋಗುತ್ತೇವೆ, ಕಣ್ಣಾಮುಚ್ಚಾಲೆ ಆಟದಲ್ಲಿ ಅಡಗೂಲಜ್ಜಿಯನ್ನು ಮುಟ್ಟಿದವನು ಆಟದಿಂದ ಪಾರಾ ದಂತೆ, ಭಗವಂತನನ್ನು ಮುಟ್ಟಿದವನು ಸಂಸಾರದ ಆಟದಿಂದ ಪಾರಾಗುತ್ತಾನೆ.

\begin{verse}
ಸಹಸ್ರಯುಗಪರ್ಯಂತಮಹರ್ಯದ್ಬ್ರಹ್ಮಣೋ ವಿದುಃ~।\\ರಾತ್ರಿಂ ಯುಗಸಹಸ್ರಾಂತಾಂ ತೇಽಹೋರಾತ್ರವಿದೋ ಜನಾಃ \versenum{॥ ೧೭~॥}
\end{verse}

{\small ಬ್ರಹ್ಮನಿಗೆ ಸಾವಿರ ಯುಗ ಒಂದು ಹಗಲು. ಸಾವಿರ ಯುಗ ಒಂದು ರಾತ್ರಿ. ಇದನ್ನು ತಿಳಿದವರು ‘ಅಹೋ ರಾತ್ರವಿದರು’.}

ಇಲ್ಲಿ ಬ್ರಹ್ಮ ಎಂದರೆ ವೇದಾಂತದ ಪರಬ್ರಹ್ಮನಲ್ಲ. ಈ ಸೃಷ್ಟಿಯನ್ನು ಕೆಲವು ಕಾಲ ಆಳುವ ಒಂದು ಪದವಿಯನ್ನು ಅಲಂಕರಿಸಿರುವ ವ್ಯಕ್ತಿ. ನಾಲ್ಕು ಯುಗಗಳು ಸೇರಿದರೆ ಒಂದು ಮಹಾಯುಗ ವಾಗುತ್ತವೆ. ಇಂತಹ ಒಂದು ಸಾವಿರ ಮಹಾಯುಗವಾದರೆ ಬ್ರಹ್ಮನಿಗೆ ಒಂದು ರಾತ್ರಿ. ಇನ್ನೊಂದು ಸಾವಿರ ಮಹಾಯುಗವಾದರೆ ಅವನಿಗೆ ಒಂದು ಹಗಲು. ಬ್ರಹ್ಮ ಇಂತಹ ದಿನಗಳನ್ನು ಒಳಗೊಂಡು ಬರುತ್ತಾನೆ. ಇವೆಲ್ಲಾ ಕೆಲವು ಮಹಾನ್ ವ್ಯಕ್ತಿಗಳಿಗೆ ಸಿಕ್ಕುವ ಒಂದು ಪದವಿ.

ಬ್ರಹ್ಮನ ಆಯುಸ್ಸಿನೊಂದಿಗೆ ನಮ್ಮ ಆಯುಸ್ಸನ್ನು ಹೋಲಿಸಿಕೊಂಡರೆ ಅದೊಂದು ಕಣ್ ಮಿಟುಕು. ಆ ಕಣ್​ಮಿಟುಕಿನಲ್ಲಿ ನಮ್ಮ ಪಾಲಿಗೆ ಬರುವ ಸುಖದುಃಖ ಎನ್ನುವುದು ಎಂದರೆ, ನಮ್ಮ ಬಾಳು ಎಷ್ಟು ನಶ್ವರ, ಕ್ಷಣಿಕ, ಅದರಲ್ಲಿ ಉತ್ಸಾಹವನ್ನು ತಾಳಬಾರದು ಎಂದು ಭಾವಿಸುತ್ತಾನೆ. ನಿಜವಾಗಿ ಬ್ರಹ್ಮನ ಆಯುಸ್ಸಿನೊಡನೆ ನಮ್ಮದನ್ನು ಹೋಲಿಸಿಕೊಂಡಾಗ ನಮ್ಮದು ಎಷ್ಟು ಅಲ್ಪ ಎಂಬುದು ಅರಿವಾಗುವುದು. ಭಗವಂತನನ್ನು ಮುಟ್ಟಿದರೆ ನಾವು ಅನಂತಾತ್ಮನಲ್ಲಿ ಒಂದಾಗುತ್ತೇವೆ, ಭೂಮದಲ್ಲಿ ಒಂದಾಗುತ್ತೇವೆ. ಆ ಭೂಮ ನಮಗೆ ಬರಬೇಕಾದರೆ ಅಲ್ಪವನ್ನು ತ್ಯಜಿಸಬೇಕು. ಅಲ್ಪ ಹೋದೊಡನೆಯೇ ಭೂಮ ತಾನಿರುವೆ ಎಂದು ಹೇಳುವುದು. ಅಲೆಯ ಹಿಂದೆಯೇ ಅನಂತ ಸಾಗರವಿದೆ. ಆದರೆ ಅಲೆಯನ್ನು ನೋಡುತ್ತಿರುವಾಗ ನಮಗೆ ಸಾಗರ ಕಾಣುವುದಿಲ್ಲ. ಮುಕ್ತಿಯನ್ನು ಪಡೆಯುವುದು ಎಂದರೆ ಅನಂತಾತ್ಮನನ್ನು ಪಡೆಯುವುದು ಎಂದು ಅರ್ಥ.

\begin{verse}
ಅವ್ಯಕ್ತಾದ್ವ್ಯಕ್ತಯಃ ಸರ್ವಾಃ ಪ್ರಭವಂತ್ಯಹರಾಗಮೇ~।\\ರಾತ್ರ್ಯಾಗಮೇ ಪ್ರಲೀಯಂತೇ ತತ್ರೈವಾವ್ಯಕ್ತಸಂಜ್ಞಕೇ \versenum{॥ ೧೮~॥}
\end{verse}

{\small ಬ್ರಹ್ಮನ ಹಗಲು ಬಂದಾಗ ಅವ್ಯಕ್ತದಿಂದ ಎಲ್ಲವೂ ವ್ಯಕ್ತವಾಗುತ್ತವೆ. ರಾತ್ರಿಯಾದ ಕೂಡಲೆ ಅವು ಅವ್ಯಕ್ತದಲ್ಲಿ ಲೀನವಾಗುತ್ತವೆ.}

ಸೃಷ್ಟಿ ಹೇಗೆ ಆಗುವುದು ಎಂಬುದನ್ನು ವಿವರಿಸುವನು. ಈ ಸೃಷ್ಟಿ ಬ್ರಹ್ಮಾಂಡದಲ್ಲಿ ಹೇಗೆ ಆಗುತ್ತಿರುವುದೊ ಅದು ಪಿಂಡಾಂಡದಲ್ಲಿಯೂ ಆಗುತ್ತಿರುವುದು. ನಾವು ರಾತ್ರಿ ಮಲಗಿಕೊಳ್ಳುವಾಗ ನೋಡುವ ವಿಶ್ವವೆಲ್ಲ ಮಾಯವಾಗಿ ಹೋಗುವುದು. ಪುನಃ ನಾವು ಎಚ್ಚೆತ್ತಕೂಡಲೆ ಅದು ಬರುವುದು. ಅದರಂತೆಯೇ ಈ ಬ್ರಹ್ಮಾಂಡ. ಬ್ರಹ್ಮ ಎದ್ದಾಗ ಈ ವಿಶ್ವವೆಲ್ಲ ಜಾಗ್ರತವಾಗುವುದು. ಅವನು ಕಣ್ಣು ಮುಚ್ಚಿಕೊಂಡರೆ ಈ ವಿಶ್ವರೂಪವೆಲ್ಲ ಇಲ್ಲದೆ ಹೋಗುವುದು. ಹೇಗೆ ರಾತ್ರಿಯಾದೊಡನೆ ಅವ್ಯಕ್ತದ ಕತ್ತಲಿಗೆ ನಾಮರೂಪಗಳೆಲ್ಲ ಹೋಗಿ ಹಗಲಾದೊಡನೆ ನಾಮರೂಪಗಳೆಲ್ಲ ಪುನಃ ಅಲ್ಲಿಂದ ಬರುವುವೊ ಹಾಗೆ ಈ ಬ್ರಹ್ಮಾಂಡವೆಲ್ಲ ಬ್ರಹ್ಮನಿಂದ ಬಂದಿದೆ, ಪುನಃ ಬ್ರಹ್ಮನಲ್ಲಿಗೆ ಹೋಗುವುದು. ಇದನ್ನು ನಾವೇನೂ ಮಾಡಲಾರೆವು. ಸೃಷ್ಟಿ ಪ್ರಳಯಗಳೆಲ್ಲ ಅವನ ಕೈಯಲ್ಲಿವೆ.

\begin{verse}
ಭೂತಗ್ರಾಮಃ ಸ ಏವಾಯಂ ಭೂತ್ವಾ ಭೂತ್ವಾ ಪ್ರಲೀಯತೇ~।\\ರಾತ್ರ್ಯಾಗಮೇಽವಶಃ ಪಾರ್ಥ ಪ್ರಭವತ್ಯಹರಾಗಮೇ \versenum{॥ ೧೯~॥}
\end{verse}

{\small ಅರ್ಜುನ, ಈ ಪ್ರಾಣಿಗಳ ಸಮುದಾಯ ಹೀಗೆ ಹುಟ್ಟಿ ರಾತ್ರಿ ಆದೊಡನೆಯೆ ಲಯವಾಗಿ ಹೋಗುತ್ತವೆ; ಹಗಲಾಗುತ್ತಲೆ ಮತ್ತೆ ಹುಟ್ಟುತ್ತವೆ.}

ಬ್ರಹ್ಮನ ಹಗಲು ಬಂದೊಡನೆಯೆ ಈ ಪ್ರಾಣಿ ಸಮುದಾಯವೆಲ್ಲ ಹುಟ್ಟಿ ಬರುತ್ತವೆ. ಇದೇನು ಹೊಸ ಸೃಷ್ಟಿಯಲ್ಲ. ಇದೇ ಹಿಂದೆ ಎಷ್ಟೋ ಸಲ ಬಂದಿದೆ, ಮುಂದೆ ಬರುವುದಿದೆ. ಇದರ ಪರಿಚಯವಿಲ್ಲದವರು ಇದನ್ನೇ ಹೊಸದು ಎಂದು ಭಾವಿಸಬಹುದು. ಆದರೆ ಇದು ಸಮುದ್ರದ ಅಲೆಯಂತೆ ಮೇಲೆ ಏಳುವುದು, ಕೆಳಗೆ ಬೀಳುವುದು. ಈ ಅಲೆ ಹಿಂದೆ ಎಷ್ಟೋ ಸಲ ಎದ್ದಿದೆ, ಮುಂದೆ ಎಷ್ಟೋ ಸಲ ಏಳುವುದಿದೆ. ಪ್ರತಿಯೊಂದು ಸಲ ಸೃಷ್ಟಿಯಾದಾಗಲೂ ಜೀವರಾಶಿಗಳೆಲ್ಲ ಬರಲೇ ಬೇಕಾಗುವುದು–ಬದ್ಧಜೀವಿಗಳು, ಮುಕ್ತಜೀವಿಗಳಲ್ಲ. ಸುಮ್ಮನೇ ಕಾಲಕಳೆದರೆ ಜೀವಿ ಮುಕ್ತನಾಗು ವುದಿಲ್ಲ. ಮುಕ್ತನಾಗಬೇಕಾದರೆ ಅದಕ್ಕೆ ಪ್ರಯತ್ನ ಪಡಬೇಕು. ಸುಮ್ಮನೆ ಒಂದು ಕ್ಲಾಸಿನಲ್ಲಿ ಒಬ್ಬ ಇದ್ದಾನೆ ಎಂದು ಅವನನ್ನು ಮೇಲಿನ ತರಗತಿಗೆ ಹಾಕುವುದಕ್ಕೆ ಆಗುವುದಿಲ್ಲ. ಅವನು ಕಷ್ಟಪಟ್ಟು ಓದಿ ಕನಿಷ್ಠಾಂಶವನ್ನಾದರೂ ಗಳಿಸಬೇಕು. ಇಲ್ಲದೇ ಇದ್ದರೆ ಅವನು ಪುನಃ ಅದೇ ಕ್ಲಾಸಿನಲ್ಲಿ ಕುಳಿತುಕೊಳ್ಳಬೇಕಾಗಿದೆ. ಸುಮ್ಮನೆ ಪಕ್ಕದ ಕೋಣೆಯಲ್ಲೆ ಮೇಲಿನ ಕ್ಲಾಸಿದೆ ಎಂದು ಅಲ್ಲಿ ಹೋಗಿ ಕುಳಿತುಕೊಳ್ಳುವುದಕ್ಕೆ ಆಗುವುದಿಲ್ಲ. ಇಲ್ಲಿ ಬದ್ಧರಾದ ಜೀವರಾಶಿಗಳಿಗೆ ಸ್ವಾತಂತ್ರ್ಯವಿಲ್ಲ. ನಾವು ಬರುವುದಿಲ್ಲ ಎಂದರೂ ಯಾರೂ ಕೇಳುವುದಿಲ್ಲ. ಕರ್ಮದ ಹಗ್ಗ ನಮ್ಮ ಕುತ್ತಿಗೆಗೆ ಬಿಗಿದಿದೆ. ಪ್ರಕೃತಿ ನಮ್ಮನ್ನು ಎಳೆದುಕೊಂಡು ಬರುತ್ತದೆ. ಸರ್ಕಸ್ಸಿನಲ್ಲಿರುವ ಪ್ರಾಣಿಗಳು ಆಯಾಕಾಲಕ್ಕೆ ಮ್ಯಾನೇಜರ್ ಹೇಳಿದ ಹಾಗೆ ತಮ್ಮ ಬೋನಿನಿಂದ ಹೊರಗೆ ಬಂದು ಅವನು ಏನನ್ನು ಮಾಡಬೇಕೆಂದು ಹೇಳುವನೊ ಅದನ್ನು ಮಾಡಿ ಹೋಗಬೇಕಾಗಿದೆ. ಮಾಡುವುದಿಲ್ಲ ಎಂದು ಭಂಡತನ ಮಾಡಿದರೆ ಕಾದಿದೆ ಚಾವಟಿಯ ಏಟು. ಹಾಗೆಯೇ ಸೃಷ್ಟಿಯ ಸರ್ಕಸ್ಸಿನಲ್ಲಿ ಜೀವಿಗಳೆಂಬ ಪ್ರಾಣಿಗಳು ಅಭಿನಯಿಸಬೇಕಾ ಗಿದೆ. ಸರ್ಕಸ್ಸಿನಿಂದ ಓಡಿಹೋದರೆ ಮಾತ್ರ ಸ್ವಾತಂತ್ರ್ಯ. ಮುಕ್ತಜೀವಿಗಳೆಂದರೆ ಈ ಸೃಷ್ಟಿಯಿಂದ ಹೊರಗೆ ಹೋಗಿರುವವರು.

\begin{verse}
ಪರಸ್ತಸ್ಮಾತ್ತು ಭಾವೋಽನ್ಯೋಽವ್ಯಕ್ತೋಽವ್ಯಕ್ತಾತ್ ಸನಾತನಃ~।\\ಯಃ ಸ ಸರ್ವೇಷು ಭೂತೇಷು ನಶ್ಯತ್ಸು ನ ವಿನಶ್ಯತಿ \versenum{॥ ೨೦~॥}
\end{verse}

{\small ಈ ಅವ್ಯಕ್ತಕ್ಕೆ ಅತೀತವಾದ ಮತ್ತೊಂದು ಸನಾತನ ಅವ್ಯಕ್ತಭಾವವಿದೆ. ಸಕಲ ಪ್ರಾಣಿಗಳು ನಾಶವಾದರೂ ಈ ಸನಾತನ ಅವ್ಯಕ್ತಭಾವ ನಾಶವಾಗುವುದಿಲ್ಲ.}

ಇಲ್ಲಿ ಎರಡು ಅವ್ಯಕ್ತಗಳನ್ನು ಶ‍್ರೀಕೃಷ್ಣ ಹೇಳುತ್ತಾನೆ. ಒಂದು ಈ ಪ್ರಪಂಚಕ್ಕೆ ಬರುವುದಕ್ಕೆ ಮುಂಚೆ ಎಲ್ಲವೂ ಅವ್ಯಕ್ತ ಸ್ಥಿತಿಯಲ್ಲಿರುತ್ತವೆ. ಆದರೆ ಅವುಗಳಲ್ಲಿ ಅವಿದ್ಯೆ ಬೀಜರೂಪದಲ್ಲಿದೆ. ಯಾವಾಗ ಅವಕಾಶ ಸಿಕ್ಕುವುದೊ ಆಗ ಪುನಃ ಮೇಲೇಳುತ್ತದೆ. ಬೀಜದಲ್ಲಿ ಗಿಡ ಹೇಗೆ ಸುಪ್ತಾವಸ್ಥೆ ಯಲ್ಲಿದೆಯೊ ಹಾಗೆ. ನಾವು ಸುಷುಪ್ತಿಗೆ ಹೋದರೆ ಅವಿದ್ಯೆ ಬೀಜರೂಪದಲ್ಲಿರುತ್ತದೆ. ಆದರೆ ಅಲ್ಲಿಂದ ಎದ್ದು ಕನಸಿಗೊ, ಜಾಗ್ರತಾವಸ್ಥೆಗೊ ಬಂದರೆ ಸಾಕು, ಆಗಲೆ ಅವು ನಮ್ಮನ್ನು ಮೆಟ್ಟಿಕೊಳ್ಳು ತ್ತವೆ. ಕೆಲವು ವೇಳೆ ಜೇನುಗೂಡನ್ನು ಕೆಣಕಿ ನೀರಿನಲ್ಲಿ ಮುಳುಗುತ್ತೇವೆ ಅವುಗಳಿಂದ ಪಾರಾಗುವುದಕ್ಕೆ. ಆದರೆ ಅವು ನೀರಿನ ಮೇಲೆಯೇ ಹಾರಾಡುತ್ತಿರುತ್ತವೆ. ಯಾವಾಗ ನೀರಿನ ಮೇಲೆ ಮುಖ ಇಡುತ್ತೇವೋ ಆಗ ಅವು ನಮ್ಮನ್ನು ಕಚ್ಚಲು ಸಿದ್ಧವಾಗಿವೆ. ಇಲ್ಲಿ ಅವ್ಯಕ್ತ ಎಂಬುದು ಹಾಗಿದೆ. ನಮ್ಮ ಹಿಂದೆ ಮುಂದೆ ಕಾಡುವುದಕ್ಕೆ ಸಿದ್ಧವಾಗಿದೆ.

ಮತ್ತೊಂದು ಅವ್ಯಕ್ತವೇ ಸನಾತನವಾದುದು. ಅಲ್ಲಿ ಯಾವ ಸಂಸ್ಕಾರವೂ ಇಲ್ಲ; ಯಾವ ಅಜ್ಞಾನವೂ ಇಲ್ಲ. ಅದು ಯಾವಾಗಲೂ ಇರುವುದು. ಈ ಪ್ರಪಂಚವೆಲ್ಲ ಬರುವುದಕ್ಕೆ ಮುಂಚೆ ಅದು ಇತ್ತು. ಈ ಪ್ರಪಂಚವೆಲ್ಲ ನಾಶವಾದಮೇಲೂ ಅದು ಇರುವುದು. ಆ ಅವ್ಯಕ್ತ ಬೇರೆ. ಅದು ಭಗವಂತನಿಗೆ ಸೇರಿದ ಅವ್ಯಕ್ತ. ನಮ್ಮ ಅವ್ಯಕ್ತ ಸ್ಥಿತಿ ಅಥವಾ ನಮ್ಮ ಸಮುದಾಯವನ್ನೆಲ್ಲ ಸೇರಿದ ಪ್ರಳಯ ಕಾಲದ ಅಥವಾ ಸೃಷ್ಟಿಗೆ ಮುಂಚಿನ ಅವ್ಯಕ್ತ ಸ್ಥಿತಿ ಬೇರೆ. ಇವುಗಳೆಲ್ಲ ಈಗ ಕಾಣದೆ ಇದ್ದರೂ ಮುಂದೆ ಯಾವುದೋ ರೂಪದಲ್ಲಿ ಕಾಣಿಸಿಕೊಂಡು ನಮ್ಮ ಕರ್ಮವನ್ನು ಸವೆಸಬೇಕಾಗಿದೆ. ಭಗ ವಂತನ ಅವ್ಯಕ್ತ ಸ್ವರೂಪವಾದರೊ ಯಾವ ಕರ್ಮ ನಿಯಮಕ್ಕೂ ಒಳಪಟ್ಟಿಲ್ಲದ ಸ್ಥಿತಿ. ಅದೆಂದಿಗೂ ನಾಶವಾಗುವುದಿಲ್ಲ. ಮಡಕೆಕುಡಿಕೆಗಳನ್ನು ಒಡೆದು ಹಾಕಿದರೆ ಅದರೊಳಗೆ ಇರುವ ಆಕಾಶವನ್ನು ನಾವು ಹೇಗೆ ಒಡೆದುಹಾಕಲಾರೆವೊ ಹಾಗೆ ಪರಮಾತ್ಮನ ಅವ್ಯಕ್ತ ಸ್ಥಿತಿ.

\begin{verse}
ಅವ್ಯಕ್ತೋಽಕ್ಷರ ಇತ್ಯುಕ್ತಸ್ತಮಾಹುಃ ಪರಮಾಂ ಗತಿಮ್~।\\ಯಂ ಪ್ರಾಪ್ಯ ನ ನಿವರ್ತಂತೇ ತದ್ಧಾಮ ಪರಮಂ ಮಮ \versenum{॥ ೨೧~॥}
\end{verse}

{\small ಅವ್ಯಕ್ತವೂ ಅಕ್ಷರವೂ ಆದುದಕ್ಕೆ ಪರಮಗತಿ ಎನ್ನುತ್ತಾರೆ. ಯಾವುದನ್ನು ಪಡೆದಮೇಲೆ ಪುನರ್ಜನ್ಮ ಇಲ್ಲವೋ ಅದೇ ನನ್ನ ಪರಮಧಾಮ.}

ಪರಮಗತಿ ಯಾವುದು ಎಂಬುದನ್ನು ಇಲ್ಲಿ ಹೇಳುತ್ತಾನೆ. ಅದು ಅವ್ಯಕ್ತವಾಗಿದೆ. ಅಲ್ಲಿ ಯಾವ ಬದಲಾವಣೆಯೂ ಇಲ್ಲ. ಯಾವುದು ದೇಶ ಕಾಲ ನಿಮಿತ್ತಕ್ಕೆ ಸೇರಿದೆಯೊ ಅಲ್ಲಿ ಬದಲಾವಣೆಯನ್ನು ನೋಡುತ್ತೇವೆ. ಅಲ್ಲಿ ಪ್ರತಿಯೊಂದೂ ಹಿಂದೆ ಈಗಿನ ಸ್ಥಿತಿಯಲ್ಲಿ ಇರಲಿಲ್ಲ. ಅನಂತರ ಪ್ರತಿ ಕ್ಷಣವೂ ಬದಲಾಯಿಸಿ ನಾಶವಾಗುತ್ತವೆ. ವ್ಯಕ್ತ ಸ್ವರೂಪವನ್ನು ಧರಿಸಿದ ವಸ್ತುವಿನ ಗತಿಯೆಲ್ಲ ಇದೆ. ಅದು ಅನಂತ. ಸಾಂತದಲ್ಲಿ ಹೇಗೆ ಅದನ್ನು ಭರ್ತಿಮಾಡಲು ಸಾಧ್ಯ? ಅನಂತ ಸಾಗರವನ್ನು ಒಂದು ಸೇರಿನಲ್ಲಿ ತುಂಬುವಂತೆ ಇದು. ವ್ಯಕ್ತ, ಭಗವಂತನನ್ನು ಅಳೆಯಲಾರದು. ಬೇಕಾದರೆ ನಮ್ಮ ಮನಸ್ಸಿನ ಸೇರಿನಲ್ಲಿ ಪರಬ್ರಹ್ಮನ ಸ್ವಲ್ಪ ನೀರನ್ನು ತುಂಬಬಹುದು. ಆದರೆ ನಮ್ಮ ಮನಸ್ಸಿನಲ್ಲಿ ಅದನ್ನು ಖಾಲಿ ಮಾಡಲಾಗುವುದಿಲ್ಲ.

ಆ ಪರಬ್ರಹ್ಮನನ್ನು ಅಕ್ಷರ ಎಂದು ಸಾರುತ್ತಾನೆ. ಅದು ಎಂದಿಗೂ ನಾಶವಾಗುವುದಿಲ್ಲ. ನಾಶ ಎಂದರೆ ಹಲವಾರು ವಸ್ತುಗಳ ಸಂಯೋಗದಿಂದ ಸೇರಿ ಆದ ವಸ್ತುವಿಗೆ ಅನ್ವಯಿಸುತ್ತದೆ. ಅದು ನಾಶವಾಯಿತು ಎಂದರೆ ಪಂಚಭೂತಗಳಿಂದ ಆದುದು ಪುನಃ ಪಂಚಭೂತಗಳಿಗೆ ಹೋಗುವುದು. ಆದರೆ ಭಗವಂತ ಹಾಗೆ ಹಲವು ವಸ್ತುಗಳ ಸಂಯೋಗದಿಂದ ಆದವನಲ್ಲ. ಅವನು ಒಂದು; ಅಖಂಡ. ಅದು ನಾಶವಾಗುವುದಾದರೂ ಹೇಗೆ? ನಾಶ ಎನ್ನುವುದು ದೇಶ ಕಾಲ ನಿಮಿತ್ತದಲ್ಲಿ ಹರಿಯುವ ವಸ್ತುವಿಗೆ ಅನ್ವಯಿಸುವುದು. ದೇಶಕಾಲನಿಮಿತ್ತದ ಧರ್ಮವೇ ಜನನ ಮತ್ತು ಮರಣ. ಈ ಧರ್ಮ ಇದಕ್ಕೆ ಅತೀತವಾದ ವಸ್ತುವಿಗೆ ಅನ್ವಯಿಸುವುದಿಲ್ಲ.

ಇದನ್ನೇ ಪರಮಗತಿ ಎಂದು ಹೇಳುತ್ತಾನೆ. ಇದಕ್ಕಿಂತ ಮೀರಿದ ಸ್ಥಿತಿಯಿಲ್ಲ. ಯಾರು ಇದನ್ನು ಪಡೆಯುತ್ತಾರೊ ಅವರು ಸರ್ವಶ್ರೇಷ್ಠವಾದುದನ್ನು ಪಡೆಯುತ್ತಾರೆ. ಈ ಗತಿಯನ್ನು ಪಡೆದವನು ಪುನಃ ಸಂಸಾರದಲ್ಲಿ ನರಳುವುದಕ್ಕೆ ಬರುವುದಿಲ್ಲ. ಹೋದ ಅಜ್ಞಾನ ಪುನಃ ಅವನಿಗೆ ಬರುವುದಿಲ್ಲ. ಸುಟ್ಟುಹೋದ ವಾಸನೆಗಳು ಪುನಃ ಹುಟ್ಟಲಾರವು. ಹೇಗೆ ಒಣಗಿದ ಮರದ ತುಂಡನ್ನು ನೆಟ್ಟರೆ ಚಿಗುರುವುದಿಲ್ಲವೋ, ಬೆಂದ ಕಾಳನ್ನು ಬಿತ್ತಿದರೆ ಮೊಳಕೆ ಆಗುವುದಿಲ್ಲವೊ, ಹಾಗೆ ಮುಕ್ತನಾದ ಜೀವಿ ಪುನಃ ಹಿಂತಿರುಗಿ ಬರುವುದಿಲ್ಲ.

ಹಿಂತಿರುಗದ ಶ್ರೇಷ್ಠವಾದ ಸ್ಥಾನ ಪರಂಧಾಮ. ನಾವೆಲ್ಲ ಹಿಂತಿರುಗುವುದು ಪೂರ್ಣತೆಯನ್ನು ಪಡೆಯುವುದಕ್ಕೆ. ಈ ಸಂಸಾರದ ಉದ್ದೇಶವೇ ಜೀವಿಯನ್ನು ಪೂರ್ಣ ಮಾಡುವುದು. ಯಾವುದು ಪೂರ್ಣವಾಗಿಲ್ಲವೋ ಅದು ಪುನಃಪುನಃ ಬರುತ್ತಿರಬೇಕು. ಅನುಭವಗಳನ್ನು ಪಡೆಯಬೇಕು. ಈ ಸಂಸಾರದ ಕಣಿವೆ ಮೂಲಕ ಸಾಗಿ ಹೋಗುವಾಗ ಸುಖದುಃಖಗಳನ್ನು ಉಣ್ಣಬೇಕು. ಆದರೆ ಒಂದು ಸಲ ಪೂರ್ಣತೆ ಪಡೆದಮೇಲೆ ಅವನು ಪುನಃ ಇಲ್ಲಿಗೆ ಬರುವುದಿಲ್ಲ. ಒಂದು ಶಾಲೆಯಲ್ಲಿ ಓದುವ ಪಾಠವನ್ನೆಲ್ಲಾ ಓದಿ ಪಾಸು ಮಾಡಿದ ಮೇಲೆ ಅವನು ಇನ್ನು ಆ ಶಾಲೆಯಲ್ಲಿ ಕುಳಿತುಕೊಂಡು ಓದಬೇಕಾಗಿಲ್ಲ. ನಾವು ಹಿಟ್ಟು ಮಾಡುವುದಕ್ಕೆ ಅಕ್ಕಿಯನ್ನು ಮಿಲ್ಲಿಗೆ ತೆಗೆದುಕೊಂಡು ಹೋಗುತ್ತೇವೆ. ಚೆನ್ನಾಗಿ ಹಿಟ್ಟಾದ ಮೇಲೆ ಪುನಃ ಅದನ್ನು ಯಾರು ಮಿಲ್ಲಿಗೆ ಕೊಡುತ್ತಾರೆ? ಚಿನ್ನದ ಅದಿರಿನಲ್ಲಿರುವ ಚಿನ್ನವನ್ನು ಬೇರ್ಪಡಿಸಲು, ಆ ಅದಿರನ್ನು ಕುಟ್ಟುತ್ತಾರೆ, ಪುಡಿ ಮಾಡುತ್ತಾರೆ, ನೀರಿನೊಂದಿಗೆ ಬೆರೆಸುತ್ತಾರೆ. ಅಪರಂಜಿ ಚಿನ್ನವನ್ನು ಬೇರ್ಪಡಿಸುತ್ತಾರೆ. ಚಿನ್ನ ಅಪರಂಜಿ ಆದಮೇಲೆ ಅದನ್ನು ಸಂದೂಕದಲ್ಲಿಟ್ಟು ಬೀಗ ಹಾಕುತ್ತಾರೆ. ಆ ಚಿನ್ನವನ್ನು ಪುನಃ ಕುಟ್ಟುವುದು, ಪುಡಿ ಮಾಡುವುದು ಮುಂತಾದ ಬದಲಾವಣೆಯ ಚಕ್ರಕ್ಕೆ ಒಳಗು ಮಾಡುವುದಿಲ್ಲ. ಜೀವಿಗೆ ಮೆತ್ತಿದ್ದ ಉಪಾಧಿಗಳ ಕೆಸರೆಲ್ಲ ತೊಳೆದುಹೋದ ಮೇಲೆ, ಒಳಗಿರುವ ಸಂಸ್ಕಾರಗಳೆಂಬ ಆಸೆಗಳೆಲ್ಲ ಹಿಂಗಿ ಹೋದಮೇಲೆ, ಈ ಪ್ರಪಂಚಕ್ಕೆ ಅವನು ಪುನಃ ಬಾರನು.

\begin{verse}
ಪುರುಷಃ ಸ ಪರಃ ಪಾರ್ಥ ಭಕ್ತ್ಯಾ ಲಭ್ಯಸ್ತ್ವನನ್ಯಯಾ~।\\ಯಸ್ಯಾಂತಃಸ್ಥಾನಿ ಭೂತಾನಿ ಯೇನ ಸರ್ವಮಿದಂ ತತಮ್ \versenum{॥ ೨೨~॥}
\end{verse}

{\small ಪಾರ್ಥ! ಆ ಉತ್ತಮ ಪುರುಷನ ಸಾಕ್ಷಾತ್ಕಾರ ಅನನ್ಯ ಭಕ್ತಿಯಿಂದ ಲಭಿಸುವುದು. ಈ ಭೂತ ಮಾತ್ರವೆಲ್ಲ ಅವನಲ್ಲಿ ಇದೆ. ಇವುಗಳೆಲ್ಲವೂ ಅವನಿಂದ ಪರಿವ್ಯಾಪ್ತವಾಗಿವೆ.}

ಪರಮಾತ್ಮನನ್ನು ಅನನ್ಯ ಭಕ್ತಿಯಿಂದ ಪಡೆಯಬಹುದು. ಅವನನ್ನು ಪಡೆಯುವುದಕ್ಕೆ ನಾವು ಬೇಕಾದಷ್ಟು ಕರ್ಮ ಮಾಡಬೇಕಾಗಿಲ್ಲ. ನಮ್ಮ ವಿಚಾರದ ಒಂದರಿಯಿಂದ ಅವನನ್ನು ಜರಡಿ ಆಡಬೇಕಾಗಿಲ್ಲ. ಇಲ್ಲಿ ಬಹಳ ಸುಲಭವಾದ, ಸ್ವಾಭಾವಿಕವಾದ ಮಾರ್ಗವನ್ನು ಶ‍್ರೀಕೃಷ್ಣ ಹೇಳುವನು. ಇದು ಸುಲಭ. ಏಕೆಂದರೆ ಪ್ರೀತಿಸುವಷ್ಟು ಸುಲಭವಾದುದು ಮತ್ತೊಂದು ಇಲ್ಲ. ನಾವು ಯಾವಾಗ ಪ್ರೀತಿಸುತ್ತೇವೆಯೊ, ಆ ಪ್ರೀತಿಸುವ ವಸ್ತುಗಳ ಕಡೆಗೆ ಮನಸ್ಸು ಏಕಾಗ್ರವಾಗುತ್ತ ಬರುತ್ತದೆ. ನಾವು ಏಕಾಗ್ರತೆಗೆ ಅಷ್ಟೊಂದು ಕಷ್ಟವನ್ನು ಪಡಬೇಕಾಗಿಲ್ಲ. ಭಗವಂತನ ಮೇಲೆ ಪ್ರೀತಿಯನ್ನು ಹುಟ್ಟಿಸಿ ಕೊಳ್ಳಬೇಕು ಅಷ್ಟೆ. ಇದು ಸ್ವಾಭಾವಿಕವಾದುದು. ಪ್ರೀತಿಯಷ್ಟು ಸ್ವಾಭಾವಿಕವಾಗಿರುವುದು ಇನ್ನು ಯಾವುದೂ ಇಲ್ಲ. ಈ ಪ್ರೀತಿಯನ್ನು ಪಶು ಪಕ್ಷಿಗಳಲ್ಲಿ, ಮನುಷ್ಯರಲ್ಲಿ, ಎಲ್ಲರಲ್ಲಿಯೂ ನೋಡು ತ್ತೇವೆ. ಪ್ರೀತಿಯಿಲ್ಲದ ಎದೆಯಿಲ್ಲ. ಒಬ್ಬ ಚಾರ್ವಾಕನಾಗಿರಬಹುದು, ಜಡವಾದಿ ಆಗಿರಬಹುದು, ಸಂದೇಹವಾದಿ ಆಗಿರಬಹುದು. ಆದರೆ ಎಲ್ಲರೂ ಯಾವುದಾದರೂ ಒಂದು ವಸ್ತುವನ್ನು ಪ್ರೀತಿಸಿಯೇ ಪ್ರೀತಿಸುತ್ತಾರೆ. ಆದರೆ ಭಕ್ತನಾದರೊ ಆ ಪ್ರೀತಿಯನ್ನು ದೇವರ ಕಡೆಗೆ ಹರಿಸುತ್ತಾನೆ. ಅದರಿಂದ ಅವನ ಪ್ರೀತಿ ಪರಿಪೂರ್ಣವಾಗುವುದು. ದೇವರಿಲ್ಲದ ಅನ್ಯವಸ್ತುಗಳೆಲ್ಲ ಅಲ್ಪ. ಅವುಗಳನ್ನು ಪ್ರೀತಿಸಿ ಸುಖವಿಲ್ಲ, ಆನಂದವಿಲ್ಲ. ಅದರಿಂದ ಬರುವ ಸುಖ, ಆನಂದ ತಾತ್ಕಾಲಿಕ. ಅದರ ಹಿಂದೆಯೇ ಕಾದು ಕುಳಿತಿದೆ, ನಾವು ಅನುಭವಿಸಿದ ಒಂದಕ್ಕೆ ಹತ್ತರಷ್ಟು ದುಃಖ ನಮ್ಮ ಮೇಲೆ ಬೀಳುವುದಕ್ಕೆ. ಇದನ್ನೆಲ್ಲ ಅರಿತ ಭಕ್ತ ಭಗವಂತನೆಂಬ ಭೂಮವನ್ನು ಪ್ರೀತಿಸುತ್ತಾನೆ. ಅಲ್ಲಿ ಮಾತ್ರ ಸುಖ, ಆನಂದ. ಆದರೆ ಇದು ಇಂದ್ರಿಯಾತೀತವಾದುದು.

ಭಗವಂತನನ್ನು ಅನನ್ಯ ಭಕ್ತಿಯಿಂದ ಪ್ರೀತಿಸಬೇಕು ಎನ್ನುತ್ತಾನೆ. ಪ್ರೀತಿ ಅನನ್ಯವಾಗಿರಬೇಕು. ಯಾವಾಗಲೂ ಭಗವಂತನ ಕಡೆಗೆ ಹರಿಯುತ್ತಿರಬೇಕು. ನದಿ ಎಡೆಬಿಡದೆ ಸಾಗರಕ್ಕೆ ಹರಿಯುವಂತೆ, ಜೀವಿ ದೇವರೆಡೆಗೆ ಧಾವಿಸುತ್ತಿರಬೇಕು. ಮತ್ತು ಭಗವಂತನನ್ನು ಪ್ರೀತಿಗಾಗಿ ಪ್ರೀತಿಸುತ್ತಾನೆ, ಅನನ್ಯ ಭಕ್ತ. ಅವನಿಗೆ ದೇವರಿಂದ ಏನನ್ನೂ ಪಡೆಯಬೇಕೆಂಬ ಆಸೆಯಿಲ್ಲ. ಅವನಿಗೆ ದೇವರೊಬ್ಬನೇ ಬೇಕು; ಅವನ ಉಗ್ರಾಣದಲ್ಲಿರುವ ಯಾವ ಐಶ್ವರ್ಯವೂ ಅಲ್ಲ. ಇಂದ್ರ ಮತ್ತು ಬ್ರಹ್ಮನ ಪದವಿಯನ್ನೂ ನಿರಾಕರಿಸುವವನು ಭಕ್ತ ಭಗವಂತನ ಪ್ರೀತಿಗಾಗಿ.

ಆ ಭಗವಂತನಲ್ಲಿ ಈ ಪ್ರಾಣಿಗಳೆಲ್ಲ ಇವೆ. ಈ ಬ್ರಹ್ಮಾಂಡವೆಲ್ಲ ಅವನಲ್ಲಿದೆ. ನಾವು ನೋಡುವ ಬ್ರಹ್ಮಾಂಡವೆಲ್ಲ ಆಕಾಶದೊಳಗೆ ಹೇಗೆ ಇದೆಯೊ ಹಾಗೆ ಅವನಲ್ಲಿ ಈ ವಿಶ್ವವೆಲ್ಲ ಅಡಗಿದೆ. ಅವನನ್ನು ಪ್ರೀತಿಸಿದರೆ ಎಲ್ಲವನ್ನೂ ಪ್ರೀತಿಸಿದಂತೆ. ಗರ್ಭಿಣಿಯಾದ ತಾಯಿ ಊಟಮಾಡಿದರೆ, ಅವಳೊಳಗೆ ಇರುವ ಮಗುವಿಗೆ ಹೇಗೆ ಅದರ ಸಾರ ಹೋಗುವುದೊ ಹಾಗೆ, ನಾವು ಭಗವಂತನನ್ನು ಪ್ರೀತಿಸಿದರೆ ಎಲ್ಲರನ್ನೂ ಪ್ರೀತಿಸುತ್ತೇವೆ. ಏಕೆಂದರೆ ಇವರೆಲ್ಲ ಅವನಲ್ಲಿ ಇರುವವರು. ಭಗವಂತನಿಲ್ಲದ ಸ್ಥಳವಿಲ್ಲ. ಎಲ್ಲಾ ಕಡೆಯಲ್ಲಿಯೂ ವ್ಯಾಪಿಸಿಕೊಂಡಿರುವನು ಅವನು. ಪಾಪಿ-ಪುಣ್ಯವಂತ, ಬುದ್ಧಿವಂತ-ದಡ್ಡ ಎಲ್ಲರೂ ಅವನಲ್ಲಿ ಇರುವರು. ಇರುವುದು ಮಾತ್ರವಲ್ಲ, ಅವರನ್ನೆಲ್ಲ ವ್ಯಾಪಿಸಿ ಕೊಂಡಿರುವನು. ಅವರೊಳಗೆಲ್ಲ ಹರಿಯುತ್ತಿರುವನು. ಕಲ್ಲು ನೀರಿನಲ್ಲಿದ್ದರೆ ಮೇಲೆ ಮಾತ್ರ ನೀರು ಇರುವುದು. ಆದರೆ ನೀರು ಒಳಗೆ ವ್ಯಾಪಿಸಿಲ್ಲ. ಹಾಗಲ್ಲ ದೇವರು. ಅವರಲ್ಲಿ ಇರುವುದು ಮಾತ್ರವಲ್ಲ, ಅವರಲ್ಲಿ ಓತಪ್ರೋತನಾಗಿ ಹರಡಿರುವನು. ಎಳೆಯ ಮಾವಿನ ಕಾಯನ್ನು ಊರುಗಾಯಿ ಉಪ್ಪಿನ ಕಾಯಿ ಮಾಡುವೆವು. ಆ ಮಾವಿನ ಕಾಯನ್ನು ಮಸಾಲೆಯಲ್ಲಿ ಅದ್ದಿಡುವೆವು. ಆ ಮಸಾಲೆಯ ಉಪ್ಪು, ಹುಳಿ, ಖಾರ ಎಲ್ಲ ಕೆಲವು ಕಾಲವಾದಮೇಲೆ ಮಾವಿನ ಕಾಯಿಯ ಒಳಗೆ ಹೊರಗೆಲ್ಲವನ್ನು ವ್ಯಾಪಿಸಿಕೊಳ್ಳುವುದು. ಅದರಂತೆಯೇ ಜಗತ್ತು ಅವನಲ್ಲಿದೆ, ಪ್ರತಿಯೊಂದು ವಸ್ತುವಿನ ಒಳಗೂ ಅವನಿರುವನು.

\begin{verse}
ಯತ್ರ ಕಾಲೇ ತ್ವನಾವೃತ್ತಿಮಾವೃತ್ತಿಂ ಚೈವ ಯೋಗಿನಃ~।\\ಪ್ರಯಾತಾ ಯಾಂತಿ ತಂ ಕಾಲಂ ವಕ್ಷ್ಯಾಮಿ ಭರತರ್ಷಭ \versenum{॥ ೨೩~॥}
\end{verse}

{\small ಅರ್ಜುನ, ಯಾವ ಕಾಲದಲ್ಲಿ ಹೋದರೆ ಮೋಕ್ಷವನ್ನು ಪಡೆಯುವರೊ, ಮತ್ತು ಯಾವ ಸಮಯದಲ್ಲಿ ಹೋದರೆ ಅವರಿಗೆ ಪುನರ್ಜನ್ಮವಿರುವುದೊ ಆ ಸಮಯವನ್ನು ಹೇಳುತ್ತೇನೆ.}

ಇಲ್ಲಿ ಒಂದು ತೊಡಕಾದ ಶ್ಲೋಕ ಬಂದಿದೆ. ಯಾವ ಕಾಲದಲ್ಲಾದರೂ ಸರಿಯೆ, ಸಾಯುವಾಗ ಮನಸ್ಸು ಭಗವಂತನಿಂದ ತುಂಬಿ, ಸಂಪೂರ್ಣವಾಗಿ ಅವನನ್ನೇ ನೆಚ್ಚಿದ್ದರೆ ಅವನಲ್ಲಿ ಸೇರುತ್ತಾನೆ. ಇದರಲ್ಲಿ ಸಂಶಯವಿಲ್ಲ ಎಂದು ಹೇಳಿ, ಕೆಲವು ನಿರ್ದಿಷ್ಟವಾದ ಸಮಯವನ್ನು ಕೊಡುತ್ತಾನೆ. ನಾವು ಅದನ್ನು ಅಕ್ಷರಶಃ ತೆಗೆದುಕೊಂಡರೆ ಭಗವದ್ಗೀತೆಯ ಉದಾರ ಬೋಧನೆಗೆ ಕುಂದು ಬಂದಂತೆ ಆಗುವುದು. ಅಂತೂ ಮುಂದಿನ ಒಂದೆರಡು ಶ್ಲೋಕಗಳು ಹೇಗೆ ಬಂದಿವೆಯೊ ಗೊತ್ತಿಲ್ಲ. ಸಾಧ್ಯವಾದಷ್ಟು ಅದರ ಹಿಂದಿರುವ ಧ್ವನಿಯನ್ನು ನಾವು ಗಮನಿಸಬೇಕಾಗುವುದು.

\begin{verse}
ಅಗ್ನಿರ್ಜ್ಯೋತಿರಹಃ ಶುಕ್ಲಃ ಷಣ್ಮಾಸಾ ಉತ್ತರಾಯಣಮ್~।\\ತತ್ರ ಪ್ರಯಾತಾ ಗಚ್ಛಂತಿ ಬ್ರಹ್ಮ ಬ್ರಹ್ಮವಿದೋ ಜನಾಃ \versenum{॥ ೨೪~॥}
\end{verse}

{\small ಅಗ್ನಿ, ಜ್ಯೋತಿ, ಹಗಲು, ಶುಕ್ಲಪಕ್ಷ, ಉತ್ತರಾಯಣದ ಆರು ಮಾಸಗಳು ಈ ಮಾರ್ಗದಲ್ಲಿ ಪ್ರಯಾಣ ಮಾಡಿದ ಬ್ರಹ್ಮಜ್ಞಾನಿಗಳು ಬ್ರಹ್ಮವನ್ನು ಸೇರುತ್ತಾರೆ.}

ಮೊದಲು ಅಗ್ನಿ ಜ್ಯೋತಿ, ಅನಂತರ ಹಗಲಿಗೆ ಅಭಿಮಾನಿ ದೇವತೆ, ಶುಕ್ಲಪಕ್ಷದ ದೇವತೆ, ಉತ್ತರಾಯಣದ ಅಭಿಮಾನಿ ದೇವತೆಯ ಆರು ಮಾಸಗಳು–ಇವುಗಳಲ್ಲಿ ಮರಣವನ್ನೈದಿದ ಬ್ರಹ್ಮ ಜ್ಞಾನಿಗಳು ಹಿಂತಿರುಗುವುದಿಲ್ಲ, ಬ್ರಹ್ಮವನ್ನೇ ಸೇರುತ್ತಾರೆ. ಇವುಗಳಲ್ಲಿ ಬೆಳಕು ಮುಖ್ಯ. ಬೆಳಕು ಯಾವಾಗಲೂ ಜ್ಞಾನಕ್ಕೆ ಚಿಹ್ನೆ. ಕೇವಲ ಹೊರಗಡೆಯ ಬೆಳಕೇ ಮುಖ್ಯವಲ್ಲ. ಇದು ಅಂತರ್ಜ್ಯೋತಿಗೆ ಚಿಹ್ನೆ ಅಷ್ಟೆ. ಯಾವಾಗ ಅಂತರ್​ಜ್ಯೋತಿ ಹೆಚ್ಚಾಗಿ, ಪ್ರಪಂಚವೆಲ್ಲ ಪರಬ್ರಹ್ಮನಿಂದ ಓತ ಪ್ರೋತವಾಗಿರುವುದನ್ನು ಅನುಭವಿಸುತ್ತ, ಮರಣವನ್ನು ಹೊಂದುವವನು, ಹೋದರೆ ಹಿಂತಿರುಗದ ದಾರಿಯನ್ನು ಹಿಡಿಯುತ್ತಾನೆ ಎಂದು ತೆಗೆದುಕೊಳ್ಳಬಹುದು. ಸುಮ್ಮನೇ ಇಲ್ಲಿ ಹೇಳಿರುವ ಸಮಯ ದಲ್ಲಿ ತೀರಿಹೋಗುವವರಿಗೆಲ್ಲ ಇದು ಅನ್ವಯಿಸುವುದಿಲ್ಲ. ಹೀಗೆ ಹೋಗುವವರಲ್ಲಿ ಯಾರು ಬ್ರಹ್ಮಜ್ಞಾನಿಗಳಾಗಿರುವರೊ ಅವರಿಗೆ ಮಾತ್ರ ಅನ್ವಯಿಸುವುದು.

\begin{verse}
ಧೂಮೋ ರಾತ್ರಿಸ್ತಥಾ ಕೃಷ್ಣಃ ಷಣ್ಮಾಸಾ ದಕ್ಷಿಣಾಯನಮ್~।\\ತತ್ರ ಚಾಂದ್ರಮಸಂ ಜ್ಯೋತಿಃ ಯೋಗೀ ಪ್ರಾಪ್ಯ ನಿವರ್ತತೇ \versenum{॥ ೨೫~॥}
\end{verse}

{\small ಧೂಮ, ರಾತ್ರಿ, ಕೃಷ್ಣಪಕ್ಷದ ದಕ್ಷಿಣಾಯನದ ಆರು ಮಾಸಗಳು, ಈ ಮಾರ್ಗದಲ್ಲಿ ಪ್ರಯಾಣ ಮಾಡಿದ ಯೋಗಿ ಚಂದ್ರನ ಜ್ಯೋತಿಯನ್ನು ಸೇರಿ ಪುನಃ ಈ ಲೋಕಕ್ಕೆ ಬರುತ್ತಾನೆ.}

ಮೇಲೆ ಹೇಳಿದ ಕಾಲದಲ್ಲಿ ತೀರಿಹೋದವರಿಗೆಲ್ಲ ಇದು ಅನ್ವಯಿಸುವುದಿಲ್ಲ. ಯಾರು ಯೋಗಿ ಯಾಗಿರುವನೊ ಅವನಿಗೆ ಮಾತ್ರ ಇದು ಅನ್ವಯಿಸುವುದು, ಅದರಲ್ಲಿಯೂ ಯಾವ ಯೋಗಿಯಲ್ಲಿ ಪೂರ್ಣಜ್ಞಾನ ಇನ್ನೂ ಬಂದಿಲ್ಲವೋ ಅವನಿಗೆ ಇದು ಅನ್ವಯಿಸುವುದು. ಅವನಲ್ಲಿ ಇನ್ನೂ ಕೆಲವು ಕಾಮನೆಗಳು ಇವೆ. ಅವನ್ನು ಅನುಭವಿಸುವುದಕ್ಕಾಗಿ ಪುನಃ ಈ ಪ್ರಪಂಚಕ್ಕೆ ಬಂದು, ಸಾಧನೆಯನ್ನು ಪೂರೈಸಿ ಹಲವು ಜನ್ಮಗಳಾದ ಮೇಲೆ ಮುಕ್ತಿಯನ್ನು ಪಡೆಯುತ್ತಾನೆ.

\begin{verse}
ಶುಕ್ಲಕೃಷ್ಣೇ ಗತೀ ಹ್ಯೇತೇ ಜಗತಃ ಶಾಶ್ವತೇ ಮತೇ~।\\ಏಕಯಾ ಯಾತ್ಯನಾವೃತ್ತಿಮನ್ಯಯಾವರ್ತತೇ ಪುನಃ \versenum{॥ ೨೬~॥}
\end{verse}

{\small ಜಗತ್ತಿನಲ್ಲಿ ಶುಕ್ಲಪಕ್ಷ, ಕೃಷ್ಣಪಕ್ಷಗಳೆಂಬ ಎರಡು ಗತಿಗಳು ಶಾಶ್ವತವಾಗಿವೆ ಎಂದು ಹೇಳಲ್ಪಟ್ಟಿದೆ. ಇದರಲ್ಲಿ ಒಂದರಲ್ಲಿ ಹೋದರೆ ಹಿಂತಿರುಗುವುದಿಲ್ಲ; ಮತ್ತೊಂದರಲ್ಲಿ ಹೋದರೆ ಹಿಂತಿರುಗುತ್ತಾನೆ.}

ಈ ಪ್ರಪಂಚದಲ್ಲಿ ಬಹಳ ಹಿಂದಿನಿಂದಲೂ ಎರಡು ಮಾರ್ಗಗಳಿವೆ. ಒಬ್ಬನು-ಮುಕ್ತನಾಗಿ ಒಂದೇ ಸಲ ಈ ಸಂಸಾರದ ಬಾಂಡಲೆಯಿಂದ ಸಿಡಿದು ಹೋದ ಬೀಜದಂತೆ ನೆಗೆದು ಹೋಗುವನು. ಇವನು ಹಿಂತಿರುಗಿ ಬರುವುದಿಲ್ಲ. ಇವನು ಹಿಂತಿರುಗಿ ಬರುವ ಟಿಕೀಟಿಲ್ಲದೆ ಹೋಗುತ್ತಾನೆ. ಮತ್ತೊಬ್ಬ ಈ ಪ್ರಪಂಚವನ್ನು ಬಿಡುವಾಗಲೆ ಬರುವ ಟಿಕೀಟನ್ನು ತೆಗೆದುಕೊಂಡು ಹೋಗುವನು. ತನ್ನ ಕರ್ಮಾನು ಸಾರ ಯಾರದೊ ಮನೆಯಲ್ಲಿ ಹುಟ್ಟಿ ಇನ್ನೂ ಅವನು ಅನುಭವಿಸಬೇಕಾಗಿದೆ. ಈ ಜೀವದ ಕಾಯಿ ಮಾಗುವುದಕ್ಕೆ ಸಂಸಾರದ ಗುಡಾಣದಲ್ಲಿ ಇನ್ನೂ ಇರಬೇಕಾಗಿದೆ.

\begin{verse}
ನೈತೇ ಸೃತೀ ಪಾರ್ಥ ಜಾನನ್ ಯೋಗೀ ಮುಹ್ಯತಿ ಕಶ್ಚನ~।\\ತಸ್ಮಾತ್ ಸರ್ವೇಷು ಕಾಲೇಷು ಯೋಗಯುಕ್ತೋ ಭವಾರ್ಜುನ \versenum{॥ ೨೭~॥}
\end{verse}

{\small ಪಾರ್ಥ! ಈ ಎರಡು ಮಾರ್ಗಗಳನ್ನು ಅರಿತ ಯಾವ ಯೋಗಿಯೂ ಮೋಹವಶನಾಗುವುದಿಲ್ಲ. ಆದುದರಿಂದ ನೀನು ಯಾವಾಗಲೂ, ಎಲ್ಲಾ ಕಾಲದಲ್ಲಿಯೂ, ಯೋಗದಿಂದ ಕೂಡಿರಬೇಕು.}

ಯಾರು ಬರದೆ ಇರುವ ರೀತಿಯಲ್ಲಿ ಹೋಗಬೇಕೆಂದು ಆಶಿಸುವರೊ ಅವರು ತಮ್ಮ ಜೀವನ ವನ್ನು ಆ ರೀತಿ ರೂಢಿಸಿಕೊಳ್ಳಬೇಕಾಗಿದೆ. ತಮ್ಮ ಮನಸ್ಸನ್ನು ಪರಿಶುದ್ಧ ಮಾಡಿಕೊಳ್ಳಬೇಕು. ಅಲ್ಲಿ ಯಾವ ಆಸೆಯ ಸಂಸ್ಕಾರವೂ ಬೀಜರೂಪದಲ್ಲಿ ಇರದಂತೆ ನೋಡಿಕೊಳ್ಳಬೇಕು. ಅಂತ್ಯಕಾಲಕ್ಕೆ ಮನಸ್ಸು ಸದಾ ಅಣಿಯಾಗಿರಬೇಕು. ಆ ಸಮಯ ಬಂದಾಗ ನಾವು ಅಣಿಯಾಗುವುದಕ್ಕೆ ಹೊತ್ತಿರು ವುದಿಲ್ಲ. ಅದು, ಮನೆಗೆ ಬೆಂಕಿ ಬಿದ್ದಾಗ ಭಾವಿ ತೋಡಲು ಹೋದಂತೆ. ನಿಜವಾದ ಯೋಗಿ ಯಾವಾಗಲೂ ಯೋಗದಿಂದ ಕೂಡಿರುವನು. ಯಾವ ಸಮಯದಲ್ಲಿ ಮೃತ್ಯುವಿನ ಕರೆ ಬಂದರೂ ಅವನು ಗೊಣಗಾಡದೆ ಈ ಪ್ರಪಂಚವನ್ನು ತ್ಯಜಿಸಲು ಸಿದ್ಧನಾಗಿರುವನು. ಅವನು ಈ ಪ್ರಪಂಚಕ್ಕೆ ಅಂಟಿಕೊಂಡಿಲ್ಲ. ಇರುವಾಗಲೆ ಅವನು ಸತ್ತಂತಿರುವನು.

\begin{verse}
ವೇದೇಷು ಯಜ್ಞೇಷು ತಪಃಸು ಚೈವ ದಾನೇಷು ಯತ್ಪುಣ್ಯಫಲಂ ಪ್ರದಿಷ್ಟಮ್~।\\ಅತ್ಯೇತಿ ತತ್ಸರ್ವಮಿದಂ ವಿದಿತ್ವಾ ಯೋಗೀ ಪರಂ ಸ್ಥಾನಮುಪೈತಿ ಚಾದ್ಯಮ್ \versenum{॥ ೨೮~॥}
\end{verse}

{\small ಇದನ್ನು ಅರಿತ ಬಳಿಕ ವೇದ, ಯಜ್ಞ, ತಪಸ್ಸು, ದಾನಗಳಲ್ಲಿ ಹೇಳಿರುವ ಪುಣ್ಯಫಲಗಳನ್ನು ದಾಟಿ ಯೋಗಿ ಉತ್ತಮವಾದ ಆ ಸ್ಥಾನವನ್ನು ಹೊಂದುತ್ತಾನೆ. }

ಇದನ್ನು ಅರಿತವನು, ಎಂದರೆ ಈ ಅಧ್ಯಾಯದಲ್ಲಿ ಹೇಳಿರುವ ವಿಷಯಗಳನ್ನೆಲ್ಲ ತಿಳಿದುಕೊಂಡು ಮನಸ್ಸನ್ನು ಭಗವಂತನ ಕಡೆ ತಿರುಗಿಸಿ, ಅವನಲ್ಲೆ ತನ್ಮಯನಾಗಿ ಯಾರು ಪ್ರಪಂಚವನ್ನು ತೊರೆಯು ವನೊ ಅವನು ಎಂದು ಅರ್ಥ. ಈತ ವೇದವನ್ನು ಅಧ್ಯಯನ ಮಾಡಿದರೆ, ತಪಸ್ಸು ದಾನಾದಿಗಳನ್ನು ಮಾಡಿದರೆ ಬರುವ ಪುಣ್ಯಗಳೆಲ್ಲಕ್ಕಿಂತ ಶ್ರೇಷ್ಠವಾದುದನ್ನು ಪಡೆಯುತ್ತಾನೆ. ಯಜ್ಞ ತಪಸ್ಸು ದಾನ ಮುಂತಾದವುಗಳನ್ನು ಫಲಾಪೇಕ್ಷೆಯಿಂದ ಮಾಡಿದರೆ ಅದು ನಮ್ಮನ್ನು ಆಯಾ ಫಲಕ್ಕೆ ಕಟ್ಟಿಹಾಕು ವುದು. ನಾವು ಆ ಫಲದ ಸುತ್ತಲೂ, ಎತ್ತು ಗಾಣದ ಸುತ್ತ ಸುತ್ತುತ್ತಿರುವಂತೆ ಸುತ್ತುತ್ತಿರಬೇಕಾಗು ವುದು. ಅನನ್ಯ ಚಿತ್ತನಾಗಿ ಭಗವಂತನನ್ನು ಧ್ಯಾನಿಸುವ ಯೋಗಿ ಇವುಗಳನ್ನು ಮೀರಿಹೋಗುತ್ತಾನೆ. ಭಗವಂತನ ಪರಮಸ್ಥಾನವನ್ನು ಪಡೆಯುತ್ತಾನೆ. ಇನ್ನು ಯಾವ ಕರ್ಮ ಧರ್ಮ ನಿಯಮಗಳೂ ಇವನ ಮೇಲೆ ಪ್ರಪಂಚಕ್ಕೆ ಪುನಃ ಕರೆತರುವ ಸಮನ್ನನ್ನು ಜಾರಿಮಾಡಲಾರವು.

