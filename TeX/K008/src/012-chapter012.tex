
\chapter{ಭಕ್ತಿಯೋಗ}

ಅರ್ಜುನ ಕೃಷ್ಣನನ್ನು ಈ ರೀತಿ ಕೇಳುತ್ತಾನೆ:

\begin{shloka}
ಏವಂ ಸತತಯುಕ್ತಾ ಯೇ ಭಕ್ತಾಸ್ತ್ವಾಂ ಪರ್ಯುಪಾಸತೇ~।\\ಯೇ ಚಾಪ್ಯಕ್ಷರಮವ್ಯಕ್ತಂ ತೇಷಾಂ ಕೇ ಯೋಗವಿತ್ತಮಾಃ \hfill॥ ೧~॥
\end{shloka}

\begin{artha}
ಈ ರೀತಿ ಸತತವೂ ನಿರತರಾಗಿ ಯಾವ ಭಕ್ತರು ನಿನ್ನನ್ನು ಉಪಾಸನೆ ಮಾಡುತ್ತಾರೆಯೋ ಮತ್ತು ಯಾರು ಅವ್ಯಕ್ತವಾದ ಅಕ್ಷರವನ್ನು ಉಪಾಸನೆ ಮಾಡುತ್ತಾರೆಯೋ ಅವರಲ್ಲಿ ಯೋಗವನ್ನು ಚೆನ್ನಾಗಿ ತಿಳಿದವರು ಯಾರು?
\end{artha}

ಈ ರೀತಿಯಾಗಿ ಎಂದರೆ, ಯಾರು ಭಗವಂತನಿಗಾಗಿ ಕರ್ಮವನ್ನು ಮಾಡುತ್ತಾನೆಯೋ, ಅವನನ್ನೇ ಪರಮಗತಿಯೆಂದು ತಿಳಿಯುವನೊ, ಅವನ ಭಕ್ತನೂ ಸಂಗರಹಿತನೂ ಯಾರ ಮೇಲೂ ವೈರವಿಲ್ಲದವನೂ ಆಗಿರುವನೊ ಎಂದು. ಅವನು ಸತತವೂ ಭಗವಂತನಿಗೆ ಸಂಬಂಧಪಟ್ಟುದನ್ನು ಆಲೋಚಿಸುವನು ಮಾತುಕತೆಯಾಡುವನು, ಅವನನ್ನು ಪ್ರಾರ್ಥಿಸುವನು, ಅವನನ್ನು ಕೀರ್ತಿಸು\-ವನು ಮತ್ತು ಜೀವನದಲ್ಲಿ ತನ್ನ ಪಾಲಿಗೆ ಬಂದ ಕೆಲಸವನ್ನೆಲ್ಲಾ ಅದು ಲೌಕಿಕವಾಗಲಿ ಪರಮಾರ್ಥ\-ವಾಗಲಿ ಎಲ್ಲವನ್ನೂ ಭಗವದರ್ಪಣ ಭಾವದಿಂದ ಮಾಡುತ್ತಾ ಫಲವನ್ನೆಲ್ಲಾ ಅವನಿಗೆ ಅರ್ಪಿಸುವನು. ಅವನು ಯಾವಾಗಲೂ ಇದರಲ್ಲಿ ನಿರತನಾಗಿರುವನು. ಅವನನ್ನು ಕುರಿತು ಚಿಂತಿಸುವನು. ಇಲ್ಲವೆ ಅವನಿಗೆ ಸಂಬಂಧಪಟ್ಟ ಕೆಲಸವನ್ನು ಮಾಡುವನು. ಇದೇ ಉಪಾಸನೆ ಮಾಡುತ್ತಿರುವನು. ಇದೇ ಭಗವಂತನ ಹತ್ತಿರ ಹೋಗಲು ಮಾಡಬೇಕಾದ ಸಾಧನೆ. ಹೃದಯ ಅವನಿಂದ ತುಂಬಿರಬೇಕು. ಕೈಗಳು ಅವನಿಗೆ ಸಂಬಂಧಪಟ್ಟ ಕೆಲಸ ಮಾಡುತ್ತಿರಬೇಕು. ಇದು ಭಗವಂತನ ಸಗುಣೋಪಾಸನೆ. ದೇವರು ಸರ್ವ ವ್ಯಾಪಿಯಾಗಿ ಪ್ರಪಂಚವನ್ನೆಲ್ಲಾ ವ್ಯಾಪಿಸಿರುವನು. ಅವನು ಸರ್ವಶಕ್ತ, ಸರ್ವಜ್ಞ, ತನ್ನನ್ನು ನೆಚ್ಚಿದ ಭಕ್ತರನ್ನು ಕೈಬಿಡದೆ ಯಾವಾಗಲೂ ಅವರನ್ನು ರಕ್ಷಿಸುವನು–ಮುಂತಾದ ಗುಣಗಳನ್ನೆಲ್ಲಾ ದೇವರಲ್ಲಿ ಕಲ್ಪಿಸಿಕೊಂಡು ಸಾಧನೆಮಾಡುತ್ತಿರುವವನು ಮೇಲೋ ಅಥವಾ ಯಾವ ಗುಣಗಳಿಲ್ಲದೆ ಅವ್ಯಕ್ತವಾದ ಅಕ್ಷರವನ್ನು ಚಿಂತಿಸುವವನು ಮೇಲೊ ಎಂದು ಶ‍್ರೀಕೃಷ್ಣನನ್ನು ಕೇಳುತ್ತಾನೆ. ಒಂದು ಸಗುಣ ಉಪಾಸನೆ ಮತ್ತೊಂದು ನಿರ್ಗುಣ ಉಪಾಸನೆ; ಒಂದು ಭಕ್ತಿ ಮತ್ತು ಕರ್ಮದ ಮೂಲಕ ಭಗವಂತನನ್ನು ಚಿಂತಿಸುವುದು, ಮತ್ತೊಂದು ಬರೀ ಜ್ಞಾನದ ಮೂಲಕ ಭಗವಂತನ ಎಲ್ಲಾ ಗುಣಗಳನ್ನೂ ಮೀರಿ ಹೋಗಿ ಪರಬ್ರಹ್ಮನನ್ನು ಚಿಂತಿಸುವುದರಲ್ಲಿ ತತ್ಪರವಾಗಿರುವುದು. ಅರ್ಜುನನಿಗೆ ಭಕ್ತಿಯೇನೊ ಹಿಡಿಸುವುದು. ಆದರೆ ಶ‍್ರೀಕೃಷ್ಣ, ಭಕ್ತಿಗೆ ಭಗವಂತನಿಗಾಗಿ ಮಾಡುವ ಕರ್ಮವನ್ನು ತಗಲಿಹಾಕುವನು. ಒಂದನ್ನು ತೆಗೆದುಕೊಂಡರೆ ಮತ್ತೊಂದನ್ನು ತೆಗೆದುಕೊಳ್ಳಬೇಕಾಗಿದೆ. ಆದರೆ ಜ್ಞಾನದಲ್ಲಾದರೋ ಈ ಕರ್ಮದ ಗಲಾಟೆ ಇಲ್ಲ. ಅವನಿಗೆ ಯಾವ ಹೊರೆ ಹೊಣೆಗಳೂ ಇಲ್ಲ. ಹೊರಗೆ ಒಂದು ಜಗತ್ತು ಇದೆ ಎಂದು ನೋಡಿದರೆ ತಾನೇ ಅದಕ್ಕೆ ತಾನು ಏನಾದರೂ ಮಾಡ ಬೇಕಾಗಿರುವುದು? ಈ ಜಗತ್ತನ್ನೇ ಅಲ್ಲಗಳೆದರೆ, ಅದೇ ಒಂದು ಭ್ರಾಂತಿಯಾದರೆ, ಅದಕ್ಕೆ ಮಾಡುವ ಕೆಲಸವೂ ಒಂದು ಭ್ರಾಂತಿಯಾಗುವುದು. ಅರ್ಜುನ ಈಗಲೂ ಕರ್ಮದಿಂದ ಪಾರಾಗಲು ಜ್ಞಾನದ ಹಾದಿಯಲ್ಲಿ ಹೋದರೆ ಸಾಧ್ಯವಾಗುವ ಹಾಗಿದ್ದರೆ ನಾವು ಯಾತಕ್ಕೆ ಅದನ್ನು ಅನುಸರಿಸಬಾರದು ಎಂದು ಕೇಳುತ್ತಾನೆ. ಅದಕ್ಕೆ ಶ‍್ರೀಕೃಷ್ಣ ಹೀಗೆ ಹೇಳುತ್ತಾನೆ:

\begin{shloka}
ಮಯ್ಯಾವೇಶ್ಯ ಮನೋ ಯೇ ಮಾಂ ನಿತ್ಯಯುಕ್ತಾ ಉಪಾಸತೇ~।\\ಶ್ರದ್ಧಯಾ ಪರಯೋಪೇತಾಸ್ತೇ ಮೇ ಯುಕ್ತತಮಾ ಮತಾಃ \hfill॥ ೨~॥
\end{shloka}

\begin{artha}
ಯಾರು ನನ್ನಲ್ಲಿ ಮನಸ್ಸನ್ನು ಇಟ್ಟು, ನಿತ್ಯಯುಕ್ತರಾಗಿ, ಪರಮಶ್ರದ್ಧೆಯಿಂದ ನನ್ನನ್ನು ಉಪಾಸನೆ ಮಾಡುತ್ತಾರೆಯೋ ಅವರು ಶ್ರೇಷ್ಠರು ಎಂಬುದು ನನ್ನ ಮತ.
\end{artha}

ಶ‍್ರೀಕೃಷ್ಣ ಇಲ್ಲಿ ಸಗುಣೋಪಾಸನೆಯೇ ಮೇಲು ಎಂದು ಅರ್ಜುನನಿಗೆ ಹೇಳುತ್ತಾನೆ. ಯಾರು ನನ್ನಲ್ಲಿ ಮನಸ್ಸನ್ನು ಇಟ್ಟು ಕರ್ಮ ಮಾಡುವರೊ ಅವರು ಮೇಲು. ಬರೀ ಫಲ, ಕೀರ್ತಿ, ಲಾಭ ಮುಂತಾದುವುಗಳಿಗೆ ಕರ್ಮ ಮಾಡದೆ ಭಗವತ್ಪ್ರೀತಿಗಾಗಿ ಕೆಲಸ ಮಾಡುವವನು, ಜೀವನದಲ್ಲಿ ಸುಲಭವಾದ ಹಾದಿಯನ್ನು ಹಿಡಿಯುವನು. ನಿತ್ಯಯುಕ್ತರಾಗಿ ಎಂದರೆ ಬಿಡದೆ ಅವನ ಕೆಲಸವನ್ನು ಮಾಡುತ್ತಾ ಹೋಗುತ್ತಿರುವವರು. ಜೀವನದಲ್ಲಿ ಲಾಭ ಬರಲಿ, ನಷ್ಟ ಬರಲಿ, ಜನತೆ\-ಗಳಲಿ, ಹೊಗಳಲಿ, ಸುಖ ಬರಲಿ ಅಥವಾ ದುಃಖ ಬರಲಿ, ಯಾವಾಗಲೂ ಬಿಡದೆ ಅವನಿಗಾಗಿ ಕೆಲಸ ಮಾಡುವವನು, ಜೀವನದಲ್ಲಿ ಸುಲಭವಾದ ಹಾದಿಯನ್ನು ಹಿಡಿಯುವನು. ಅವನು ಶ್ರದ್ಧೆಯಿಂದ ಕೂಡಿರುವನು. ತಾನು ಯಾರಿಗಾಗಿ ಕೆಲಸ ಮಾಡತ್ತಿರುವನೊ ಅವನನ್ನು ಪ್ರೀತಿಸುತ್ತಾನೆ. ಅವನಿಗಾಗಿ ಅದ್ಭುತವಾದ ಶ್ರದ್ಧೆ ಇದೆ. ಅವನು ಕಾಟಾಚಾರಕ್ಕೆ ಮಾಡುವುದಿಲ್ಲ. ಮಾಡುವ ಕೆಲಸವನ್ನು ಪೂಜೆಯಂತೆ ಮಾಡುತ್ತಾನೆ. ಹೀಗೆ ಕೆಲಸ ಮಾಡುವವನು ಮೇಲು. ಇದಕ್ಕೆ ಹಲವು ಗುಣಗಳಿವೆ. ಇಲ್ಲಿ ಇಂದ್ರಿಯಗಳನ್ನು ಸಂಪೂರ್ಣವಾಗಿ ನಿಗ್ರಹಿಸುವುದಿಲ್ಲ. ಆದರೆ ಅದಕ್ಕೆ ಹಿಡಿದುಕೊಳ್ಳುವುದಕ್ಕೆ ಬೇರೊಂದು ವಸ್ತುವನ್ನು ಕೊಡುವುದು. ಇಲ್ಲಿ ಯಾವ ಬಲಾತ್ಕಾರವೂ ಇಲ್ಲ. ನಮ್ಮ ಸ್ವಭಾವಕ್ಕೆ ಅನುಗುಣವಾದ ಹಾದಿ ಇದು. ಇಂದ್ರಿಯಗಳನ್ನು ದಮನ ಮಾಡಬೇಕಾಗಿಲ್ಲ, ಅದನ್ನು ದೇವರ ಕಡೆ ತಿರುಗಿಸಬೇಕಾಗಿದೆ ಅಷ್ಟೆ. ನಮ್ಮಲ್ಲೆಲ್ಲಾ ರಜೋಗುಣವಿದೆ. ಅದನ್ನು ಸುಮ್ಮನೆ ಇಟ್ಟರೆ ಕಡಮೆಯಾಗುವುದಿಲ್ಲ. ರಜೋಗುಣ ಬೆಣ್ಣೆಯಲ್ಲಿರುವ ನೀರಿನಂತೆ. ಆ ನೀರನ್ನು ಓಡಿಸಬೇಕಾಗಿದೆ, ಒಲೆಯ ಮೇಲಿಟ್ಟು \break ಕಾಯಿಸಬೇಕು. ಆಗಲೇ ನೀರು ಹೊರಟುಹೋಗಬಲ್ಲುದು. ಹೋಗುವಾಗ ಬೇಕಾದಷ್ಟು ಸದ್ದುಮಾಡಿಕೊಂಡು ಹೋಗುವುದು. ನೀರೆಲ್ಲ ಹೋದ ಮೇಲೆ ತೆಪ್ಪಗಾಗುವುದು. ಅದರಂತೆಯೇ ರಜೋಗುಣವೇ ನಮ್ಮ ಮನಸ್ಸಿನಲ್ಲಿರುವ ನೀರು. ಅದನ್ನು ಕಡಮೆ ಮಾಡಬೇಕಾದರೆ ಕರ್ಮದ ಒಲೆಯ ಮೇಲಿಟ್ಟು ಕಾಯಿಸಬೇಕು. ನಾವು ಫಲಾಪೇಕ್ಷೆ ಬಿಟ್ಟು ಕೇವಲ ಭಗವಂತನ ಪ್ರೀತಿಗಾಗಿ ಕರ್ಮ ಮಾಡಿದರೆ ಅದು ನಮ್ಮನ್ನು ಬಂಧಿಸುವುದಿಲ್ಲ. ಅದರ ಬದಲು ನಮ್ಮ ಚಿತ್ತವನ್ನು ಶುದ್ಧಿ ಮಾಡಿ, ಭಗವಂತನನ್ನು ಕುರಿತು ಚಿಂತಿಸುವುದಕ್ಕೆ ಮನಸ್ಸನ್ನು ಹೆಚ್ಚು ಅಣಿಮಾಡುವುದು. ಇಲ್ಲಿ ನಮ್ಮ ಇಂದ್ರಿಯಗಳಿಗೂ ಒಂದು ಕೆಲಸ ಕೊಡುವೆವು. ಆದರೆ ಆ ಕೆಲಸ ನಮ್ಮನ್ನು ಬಂಧಿಸುವುದಿಲ್ಲ. ಕಣ್ಣಿದೆ ಅದರಿಂದ ನೋಡುತ್ತೇವೆ. ಭಗವಂತನ ವಿರಾಡ್ರೂಪವನ್ನು ನೋಡುವುದಕ್ಕೆ, ಅವನ ಸೌಂದರ್ಯವನ್ನು ಅನುಭವಿಸುವುದಕ್ಕೆ ಇದನ್ನು ಉಪಯೋಗಿಸುತ್ತೇವೆ. ಭಗವಂತನನ್ನು ಬಿಟ್ಟ ಆನಂದವಲ್ಲ, ಭಗವಂತನನ್ನು ಸೇವಿಸಿದ ಆನಂದ ಇದು. ಹಾಗೆಯೇ ಮಾತನಾಡುತ್ತಾನೆ ಭಕ್ತ. ಆದರೆ ಹಾಳು ಮೂಳನ್ನು ಅಲ್ಲ. ಭಗವಂತನಿಗೆ ಸಂಬಂಧಪಟ್ಟುದನ್ನು ಮಾತನಾಡುತ್ತಾನೆ. ಇದರಿಂದ ಅವನಿಗೆ ಆನಂದ. ಕೇಳುವವರಿಗೂ ಒಂದು ಆನಂದ. ಇದು ಕೇಳುವವನು ಮತ್ತು ಹೇಳುವವನು ಇಬ್ಬರ ಮೇಲೆಯೂ ಒಳ್ಳೆಯ ಸಂಸ್ಕಾರಗಳನ್ನು ಬಿಡುವುದು. ಸುಮ್ಮನೆ ಕುಳಿತಿರಲು ಯಾರ ಕೈಯಲ್ಲೂ ಸಾಧ್ಯವಿಲ್ಲ., ಇದರಷ್ಟು ಕಷ್ಚವಾಗಿರುವುದು ಮತ್ತೊಂದಿಲ್ಲ.

ಆದಕಾರಣ ಭಕ್ತ ಯಾವಾಗಲೂ ಏನನ್ನಾದರೂ ಮಾಡುತ್ತಿರುವನು. ಇತರ ಇಂದ್ರಿಯಗಳನ್ನು ಕೂಡ ದೇವರೆಡೆಗೆ ಹರಿಸುವನು. ಇಂದ್ರಿಯಗಳನ್ನು ನೀನು ಹೊರಗೆ ಹೋಗಬೇಡ ಎಂದು ತಡೆಗಟ್ಟುವುದು ಬಹಳ ಕಷ್ಟವಾದ ಮಾರ್ಗ. ನದಿ ಹರಿಯುತ್ತಿರುವಾಗ ಅದರ ಮಾರ್ಗದಲ್ಲಿ ಒಂದು ಕಾಲುವೆಯನ್ನು ತೋಡಿ ಆ ನೀರನ್ನು ಭಗವಂತನೆಂಬ ಹೊಲಗದ್ದೆಗೆ ಹಾಯಿಸಿದರೆ ನೀರು ವ್ಯರ್ಥವಾಗುವುದಿಲ್ಲ ಮತ್ತು ಅದು ಧ್ವಂಸಕಾರಿಯೂ ಆಗುವುದಿಲ್ಲ. ಇದನ್ನೇ ಭಕ್ತ ಮಾಡುವುದು. ಇದು ಸುಲಭ, ಅದಕ್ಕಾಗಿಯೇ ಮೇಲು.

\begin{shloka}
ಯೇ ತ್ವಕ್ಷರಮನಿರ್ದೇಶ್ಯಮವ್ಯಕ್ತಂ ಪರ್ಯುಪಾಸತೇ~।\\ಸರ್ವತ್ರಗಮಚಿಂತ್ಯಂ ಚ ಕೂಟಸ್ಥಮಚಲಂ ಧ್ರುವಮ್ \hfill॥ ೩~॥
\end{shloka}

\begin{shloka}
ಸಂನಿಯಮ್ಯೇಂದ್ರಿಯಗ್ರಾಮಂ ಸರ್ವತ್ರ ಸಮಬುದ್ಧಯಃ~।\\ತೇ ಪ್ರಾಪ್ನುವಂತಿ ಮಾಮೇವ ಸರ್ವಭೂತಹಿತೇ ರತಾಃ \hfill॥ ೪~॥
\end{shloka}

\begin{artha}
ಆದರೆ ಯಾರು ಇಂದ್ರಿಯಗಳನ್ನು ನಿಗ್ರಹಿಸಿ, ಎಲ್ಲಾ ಕಾಲದಲ್ಲಿಯೂ ಸಮಬುದ್ಧಿಯುಳ್ಳವರಾಗಿ ಸರ್ವ ಪ್ರಾಣಿಗಳ ಹಿತದಲ್ಲಿ ನಿರತರಾಗಿ, ನಿರ್ದೇಶಿಸಲು ಅಶಕ್ಯವೂ, ಅವ್ಯಕ್ತವೂ, ಸರ್ವವ್ಯಾಪಿಯೂ, ಚಿಂತಿಸಲು ಅಶಕ್ಯವೂ, ಕೂಟಸ್ಥವೂ, ಅಜವೂ, ಶಾಶ್ವತವೂ ಆದ ಅಕ್ಷರವನ್ನು ಉಪಾಸನೆ ಮಾಡು\-ತ್ತಾರೆಯೋ ಅವರು ನನ್ನನ್ನೇ ಹೊಂದುತ್ತಾರೆ.
\end{artha}

ಇಲ್ಲಿ ಶ‍್ರೀಕೃಷ್ಣ ನಿರ್ಗುಣ ಬ್ರಹ್ಮನ ಉಪಾಸನೆಯನ್ನು ಹೇಳಿ, ಈ ಉಪಾಸನೆಯಲ್ಲಿ ನಿರತರಾದ ವರೂ ಕೂಡಾ ನನ್ನನ್ನೇ ಸೇರುತ್ತಾರೆ ಎನ್ನವನು. ಚಾಮುಂಡಿಬೆಟ್ಟದ ಮೇಲೆ ಹತ್ತಲು ಹಲವು ವಿಧಗಳಿವೆ. ಮೆಟ್ಟಲನ್ನು ಹತ್ತಿಕೊಂಡು ಹೋಗಿ ಗುಡಿಯನ್ನು ಸೇರಬಹುದು. ರಸ್ತೆಯ ಮೂಲಕ ಕಾರಿನಲ್ಲಿ ಹೋಗಿ ಗುಡಿಯನ್ನು ಸೇರಬಹುದು. ದಾರಿಯೂ ಬೇಡ ಮೆಟ್ಟಲೂ ಬೇಡ. ನನ್ನ ಮೂಗಿನ ನೇರಕ್ಕೆ ಹತ್ತಿಕೊಂಡು ಹೋಗುತ್ತೇನೆ ಎಂದರೆ ಅವನಿಗೂ ಮೇಲಕ್ಕೆ ಹೋಗಲು ಸಾಧ್ಯ. ಆದರೆ ಅದು ಕಷ್ಟದ ಹಾದಿ; ಕಲ್ಲು ಮುಳ್ಳು ಕೊರಕಲು ಬಂಡೆಗಳನ್ನು ಹತ್ತಿಕೊಂಡು ಹೋಗಬೇಕು. ಕೆಲವು ಸಾಹಸ ಪ್ರಿಯರೂ ಇರುತ್ತಾರೆ, ಅವರಿಗೆ ಸುಲಭದ ಹಾದಿ ಹಿಡಿಸುವುದಿಲ್ಲ. ಅದು ಅವರಿಗೆ ಬಹಳ ಸಪ್ಪೆಯಾಗಿ ತೋರುವುದು. ಅವರಿಗೆ ಜೀವನದಲ್ಲಿ ಸಾಹಸ ಬೇಕು, ಕಷ್ಟಗಳನ್ನು ಎದುರಿಸಬೇಕು. ಆಗಲೇ ತೃಪ್ತಿ. ಅಂತಹ ಮನೋಭಾವದವರಿಗೂ ಒಂದು ದಾರಿ ಇದೆ. ನಿರ್ಗುಣೋಪಾಸನೆ ಈ ಗುಂಪಿಗೆ ಸೇರಿರುವುದು. ಈ ದಾರಿಯಲ್ಲಿ ಹೋಗುವವರಿಗೆ ಮುಂದಿನ ಗುಣಗಳು ಇರಬೇಕು. ಇದನ್ನು ಶ‍್ರೀಕೃಷ್ಣ ವಿವರಿಸುವನು.

ಅವನು ಇಂದ್ರಿಯಗಳನ್ನು ನಿಗ್ರಹಿಸುವನು. ಹೊರಗೆ ವಿಷಯವಸ್ತುಗಳ ಕಡೆಗೆ ಹೋಗುವುದಕ್ಕೆ ಅವಕಾಶವನ್ನೇ ಕೊಡುವುದಿಲ್ಲ. ಹೋಗುವುದಕ್ಕೆ ಬಿಟ್ಟರೆ ತಾನೆ ಅದನ್ನು ಆಮೇಲೆ ಹುಡುಕಿಕೊಂಡು ಹೋಗಿ ಚದುರಿರುವುದನ್ನೆಲ್ಲಾ ಹಿಂತಿರುಗಿ ತೆಗೆದುಕೊಂಡು ಬರುವ ತೊಂದರೆಯನ್ನು ತೆಗೆದುಕೊಳ್ಳ ಬೇಕಾಗುವುದು? ಬಿಡದೇ ಇದ್ದರೆ ಅದನ್ನು ಹುಡುಕಿಕೊಂಡು ಹೋಗುವ ಕಷ್ಟವೇ ತಪ್ಪುವುದು. ಇಂದ್ರಿಯಗಳು ಬಹಿರ್ಮುಖವಾಗುವುದಕ್ಕೆ ಅವನು ಬಿಡುವುದೇ ಇಲ್ಲ. ಆಯಾ ಕೇಂದ್ರಗಳಲ್ಲಿ ಇಂದ್ರಿಯಗಳನ್ನು ಅಲ್ಲಲ್ಲಿಯೇ ಕಟ್ಟಿಹಾಕುವನು. ಜೀವನದಲ್ಲಿ ಹೊರಗೆ ಹೋಗಲು ಸೆಳೆಯುತ್ತಿರುವ ಇಂದ್ರಿಯಗಳನ್ನು ನಿಗ್ರಹಿಸುವುದು ಬಹಳ ಕಷ್ಟ. ವೇಗವಾಗಿ ಹೋಗುತ್ತಿರುವ ರೈಲನ್ನು ನಿಲ್ಲಿಸುವುದು ಸುಲಭ. ದುಷ್ಟಮೃಗಗಳನ್ನು ನಿಗ್ರಹಿಸುವುದು ಸುಲಭ. ಆದರೆ ಇಂದ್ರಿಯಗಳನ್ನು ನಿಗ್ರಹಿಸಬೇಕಾದರೆ ಎಲ್ಲದಕ್ಕಿಂತ ಹೆಚ್ಚಾಗಿ ಇಚ್ಛಾಶಕ್ತಿ ಬೇಕಾಗುವುದು. ಜ್ಞಾನಿ ಇಂದ್ರಿಯ ತನ್ನ ಇಚ್ಛೆಗೆ ವಿರೋಧವಾಗಿ ಹೊರಗೆ ಹೋಗಲು ಅವಕಾಶವನ್ನೇ ಕೊಡುವುದಿಲ್ಲ. ಒಮ್ಮೆ ಕೊಟ್ಟರೆ ತಾನೇ ಪುನಃ ನಾವು ಅದಕ್ಕೆ ಸದರ ಕೊಡುತ್ತಿರಬೇಕಾಗುವುದು? ಮೊದಲೇ ತಡೆಗಟ್ಟಿದರೆ ಅನಂತರ ಸುಲಭವಾಗುವುದು.

ಎಲ್ಲಾ ಕಾಲದಲ್ಲಿಯೂ ಅವನು ಸಮಬುದ್ಧಿಯುಳ್ಳವನು. ಜೀವನದಲ್ಲಿ ಅವನು ಅತಿಗೆ ಅವಕಾಶ ಕೊಡುವುದಿಲ್ಲ. ನಮ್ಮ ಮಾನಸಿಕ ಶಕ್ತಿ ವ್ಯಯವಾಗುವುದು ಅತಿಯ ದ್ವಾರದ ಮೂಲಕವೇ. ದುಃಖ ಬಂದರೆ ಹತಾಶರಾಗುತ್ತೇವೆ. ಆಯಿತು ನಮ್ಮ ಕತೆ ಎಂದು ತಲೆ ಚಚ್ಚಿಕೊಳ್ಳುತ್ತೇವೆ. ಯಾರಾ\-ದರೂ ನಮ್ಮನ್ನು ಟೀಕಿಸಿದರೆ, ಅವರ ಮೇಲೆ ಕೋಪ, ಯಾರಾದರೂ ನಮ್ಮನ್ನು ಹೊಗಳಿದರೆ ನಮ್ಮಲ್ಲಿರುವ ಒಳ್ಳೆಯ ಗುಣಗಳಿಗೆ ನಾವೇ ಮನಸೋಲುತ್ತೇವೆ. ಲಾಭ ಬಂದರೆ ನಮ್ಮನ್ನು ಹಿಡಿಯುವವರು ಯಾರೂ ಇಲ್ಲ. ನಷ್ಟ ಬಂದರೆ ಅದರ ಭಾರದಲ್ಲಿ ಕುಗ್ಗಿಹೋಗುತ್ತೇವೆ. ಒಂದು ಅತಿಯಿಂದ ಇನ್ನೊಂದು ಅತಿಯ ಕಡೆಗೆ ಪ್ರಯಾಣ ಮಾಡುತ್ತಾ ಇರುತ್ತೇವೆ. ಜ್ಞಾನಿಯಾದರೊ ಯಾವಾಗಲೂ ಮಧ್ಯದಲ್ಲಿರುವನು. ಅವನು ನಗುವುದೂ ಇಲ್ಲ, ಅಳುವುದೂ ಇಲ್ಲ. ನನ್ನ ಸಮಾನ ಯಾರೂ ಇಲ್ಲ ಎಂದು ಮೆರೆಯುವುದೂ ಇಲ್ಲ. ಹತಾಶನಾಗಿ ಕುಗ್ಗುವುದೂ ಇಲ್ಲ. ಈ ದ್ವಂದ್ವ ಅನುಭವಗಳ ಮೂಲಕ ನಾವು ಸಾಗಿ ಹೋಗಬೇಕಾಗಿದೆ. ಇದೇ ನಮ್ಮ ಗುರಿಯಲ್ಲ. ಇವುಗಳೆಲ್ಲ ದಾರಿಯ ಇಕ್ಕೆಲಗಳಲ್ಲಿ ಸಿಗುವ ರೈಲ್ವೇ ನಿಲ್ದಾಣಗಳು. ಇವುಗಳನ್ನು ಅತಿಕ್ರಮಿಸಿ ಹೋಗಬೇಕಾಗಿದೆ, ನಾವು ಗುರಿಯನ್ನು ಸೇರಬೇಕಾದರೆ ಎಂಬುದು ಇವನಿಗೆ ಚೆನ್ನಾಗಿ ಗೊತ್ತು.

ಇವನು ಸರ್ವ ಪ್ರಾಣಿಗಳ ಹಿತದಲ್ಲೂ ನಿರತನಾಗಿದ್ದಾನೆ. ಏಕೆಂದರೆ ಪರಬ್ರಹ್ಮನೊಬ್ಬನೆ ಈ ಪ್ರಪಂಚದಲ್ಲಿ ಸತ್ಯ. ಆ ಪರಬ್ರಹ್ಮನೇ ಜೀವನದಲ್ಲಿ ಎಲ್ಲಕ್ಕೂ ಸಾಮಾನ್ಯ ಹಿನ್ನೆಲೆ ಆಗಿರುವನು. ನಾವು ಪರಬ್ರಹ್ಮನಲ್ಲಿ ಒಂದು ಎಂದು ಭಾವಿಸಿದರೆ ಈ ಪ್ರಪಂಚದಲ್ಲಿ ಅಣುರೇಣು ತೃಣಕಾಷ್ಠಗಳ\-ನ್ನೆಲ್ಲಾ ಪ್ರೀತಿಸಬೇಕಾಗಿದೆ. ಏಕೆಂದರೆ ಒಂದೇ ಎಲ್ಲಾ ಕಡೆಯಲ್ಲಿಯೂ ಇರುವುದು. ಭಗವಂತನನ್ನು ಉಪಾಸನೆ ಮಾಡುವವನು ಯಾರಿಗೂ ಎಳ್ಳಷ್ಟೂ ತೊಂದರೆ ಕೊಡುವುದಿಲ್ಲ. ಇನ್ನೊಬ್ಬನಿಗೆ ತೊಂದರೆ ಕೊಟ್ಟರೆ ಭಗವಂತನನ್ನೇ ಪೀಡಿಸಿದಂತೆ ಆಗುವುದು.

ಮೇಲಿನ ಕೆಲವು ಗುಣಗಳು ಅಕ್ಷರವನ್ನು ಉಪಾಸನೆ ಮಾಡುವವನದಾಯಿತು, ಇನ್ನು ಅಕ್ಷರ ಹೇಗಿದೆ ಎಂಬುದನ್ನು ವಿವರಿಸಲು ಯತ್ನಿಸುವನು. ಇದು ಬಣ್ಣನೆಗೆ ನಿಲುಕದ ವಸ್ತು. ಆದರೂ ಮನಸ್ಸು ಹಾಗೆಯೇ ಬಿಡುವುದಿಲ್ಲ. ಹಾಗೇ ಹೀಗೇ ಎಂಬ ಕೆಲವು ಉಪಮಾನಗಳು, ಗುಣವಾಚಕಗಳು ಇವುಗಳ ಕೈಮರದ ಮೂಲಕ ಅಚಿಂತ್ಯವನ್ನು ಚಿಂತಿಸುವುದಕ್ಕೆ ಯತ್ನಿಸುವುದು.

ಅಕ್ಷರವನ್ನು ನಿರ್ದೇಶಿಸಲು ಆಗುವುದಿಲ್ಲ. ನಿರ್ದೇಶ ಮಾಡುವುದು ಎಂದರೆ ತೋರುವುದು. ತೋರಬೇಕಾದರೆ ಆ ವಸ್ತುವಿಗೆ ಒಂದು ಆಕಾರ ಇರಬೇಕು, ಅದೊಂದು ಸ್ಥಳದಲ್ಲಿ ಇರಬೇಕು. ಆಗ ನೋಡಿ, “ಅಲ್ಲಿ ಆ ವಸ್ತು”ಎಂದು ತೋರುತ್ತೇವೆ. ತೋರುವ ವಸ್ತು ಸಾಂತವಾಗಿರಬೇಕು. ಅಕ್ಷರವನ್ನು ಹಾಗೆ ತೋರುವುದಕ್ಕೆ ಆಗುವುದಿಲ್ಲ. ಅದು ಕಾಲ ದೇಶದ ಎಲ್ಲೆಯಲ್ಲಿಲ್ಲ. ಕಾಲ ದೇಶವನ್ನು ಅದು ಅತಿಕ್ರಮಿಸಿದೆ. ಅದಕ್ಕೆ ನಾಮರೂಪಗಳಿಲ್ಲ. ನಾಮರೂಪಗಳೆಲ್ಲ ನಾಶವಾಗುವುವು. ಹಿಂದೆ ಇರಲಿಲ್ಲ, ಮುಂದೆ ಇರುವುದಿಲ್ಲ. ಈಗ ಮಾತ್ರ ಅದು ಕಾಣುತ್ತಿದೆ. ಕಾಣುತ್ತಿರುವಾಗಲೂ ಅದು ಬದಲಾಯಿಸುತ್ತಿದೆ. ಅಕ್ಷರ ಹೀಗಲ್ಲ. ಅದು ಇಲ್ಲದ ಕಾಲವೇ ಇಲ್ಲ, ಇಲ್ಲದ ದೇಶವೇ ಇಲ್ಲ. ಗುಣವನ್ನು ಯಾವಾಗ ಒಂದು ವಸ್ತುವಿಗೆ ಕೊಡುತ್ತೇವೆಯೋ ಅದನ್ನು ವ್ಯಾಪ್ತಿಯಲ್ಲಿ ಕಡಮೆ ಮಾಡುತ್ತೇವೆ. ಗುಣವಾಚಕಗಳನ್ನು ಹೇರಿದಂತೆಲ್ಲಾ ಅದರ ವ್ಯಾಪ್ತಿ ಕಿರಿದಾಗುತ್ತಾ ಬರುವುದು ಮತ್ತು ಒಳ್ಳೆಯ ಗುಣ ಮತ್ತೊಂದು ಕೆಟ್ಟ ಗುಣದಿಂದ ಬಾಧಿತವಾಗುವುದು.

ಅದು ಅವ್ಯಕ್ತ. ಕಣ್ಣಿಗೆ ಕಾಣದುದು. ಈಗ ಕಾಣದೆ ಇರಬಹುದು, ಮುಂದೆಯಾದರೂ ವ್ಯಕ್ತವಾಗ ಬಹುದೇನೊ ಎಂದು ಯೋಚಿಸಬಹುದು. ಅದು ದೇಶಕಾಲ ನಿಮಿತ್ತದ ಬಲೆಯೊಳಗೆ ಎಂದಿಗೂ ಬೀಳುವ ವಸ್ತುವೇ ಅಲ್ಲ. ಅದು ಇಂದ್ರಿಯಾತೀತವಾದುದು. ಅದು ಸರ್ವವ್ಯಾಪಿ. ಅದು ಇಲ್ಲದ ಸ್ಥಳವೇ ಇಲ್ಲ. ಆಕಾಶ ಹೇಗೆ ಸರ್ವವನ್ನೂ ವ್ಯಾಪಿಸಿಕೊಂಡಿರುವುದೋ, ಎಲ್ಲದರ ಒಳಗೆ ಹೊರಗೆ ಇದೆಯೋ ಹಾಗೆ ಅಕ್ಷರ. ಎಲ್ಲವನ್ನೂ ವ್ಯಾಪಿಸಿಕೊಂಡಿರುವುದು. ಅದೇ ಸಣ್ಣದರೊಳಗೆ ಇದೆ. ಅದೇ ಮಹತ್ತಿನೊಳಗೆ ಇದೆ ಮತ್ತು ಇವೆರಡೂ ಅದರಲ್ಲಿವೆ. ಅದನ್ನು ನಾವು ಚಿಂತಿಸುವುದಕ್ಕೆ ಆಗುವುದಿಲ್ಲ. ಚಿಂತನೆ ಎಂದರೆ ರೂಪ ಗುಣ ಇವುಗಳು ಇರುವ ವಸ್ತುವನ್ನು ಮನಸ್ಸು ನಮ್ಮ ಕಲ್ಪನೆಯ ಬಲೆಯಲ್ಲಿ ಬೆಸ್ತ ಮೀನನ್ನು ಹಿಡಿಯುವ ಹಾಗೆ ಹಿಡಿಯುವುದು. ಆದರೆ ಮೀನಿನ ಬಲೆ ಎಲ್ಲವನ್ನೂ ಹಿಡಿಯಲಾರದು. ಬರೀ ನೀರನ್ನು ಮೀನಿನ ಬಲೆಯಲ್ಲಿ\break ಹಿಡಿಯುವುದಕ್ಕೆ ಆಗುವುದಿಲ್ಲ. ಅದರಂತೆಯೇ ನಮ್ಮ ಮನಸ್ಸು ಕೆಲವು ಸ್ಥೂಲ ವಿಷಯಗಳನ್ನು ತನ್ನ ಜಾಲದಲ್ಲಿ ಹಿಡಿಯಬಹುದು. ಆದರೆ ಅಕ್ಷರವನ್ನು ಹಿಡಿಯಲಾರದು. ಏಕೆಂದರೆ ಅದಕ್ಕೆ ಆಕಾರವೇ ಇಲ್ಲ. ಗುಣವೇ ಇಲ್ಲ.

ಅದು ಕೂಟಸ್ಥ, ಸ್ವಲ್ಪವೂ ಬದಲಾಯಿಸದೆ ಇದ್ದ ಕಡೆಯೆ ಇರುವುದು. ಆದರೆ ಬದಲಾಯಿ\-ಸುವು\-ದಕ್ಕೆಲ್ಲಾ ಇದು ಆಧಾರವಾಗಿರುವುದು. ಕಮ್ಮಾರ ತನ್ನ ಅಡಿಗಲ್ಲಿನ ಮೇಲೆ ಹಲವು ವಸ್ತುಗಳನ್ನು ಇಟ್ಟು ಅದಕ್ಕೆ ಪೆಟ್ಟು ಕೊಟ್ಟು ಹಲವು ಆಕಾರಗಳನ್ನು ತಯಾರುಮಾಡಿ ಎಸೆಯುತ್ತಾನೆ. ಆದರೆ ಅಡಿಗಲ್ಲಾದರೋ ಒಂದೇ ಸಮನಾಗಿರುವುದು. ಬದಲಾಯಿಸುವುದಕ್ಕೆ ಸಾಕ್ಷಿಯಾಗಿರುವುದು ಕೂಟಸ್ಥ. ಅದು ಅಚಲ. ಚಲ ಎನ್ನುವುದು ಸಾಂತ ವಸ್ತುವಿಗೆ ಅನ್ವಯಿಸುವುದು. ಒಂದು ವಸ್ತುವು ಎಲ್ಲಾ ಕಡೆಯೂ ಇದ್ದರೆ, ಇಲ್ಲದ ಕಡೆಯೇ ಇರದೇ ಇದ್ದರೆ, ಇನ್ನು ಅದು ಚಲಿಸುವುದು ಹೇಗೆ? ಅದು ಶಾಶ್ವತವಾಗಿ ಸರ್ವವ್ಯಾಪಿಯಾಗಿ ಇರುವುದು. ಈ ಪ್ರಪಂಚದಲ್ಲಿ ಕಾಣುವ ದೃಶ್ಯವೆಲ್ಲಾ ಸರ್ವನಾಶವಾದರೂ ಅದೊಂದು ಶಾಶ್ವತವಾಗಿರುವುದು. ಸೃಷ್ಟಿಗೆ ಮುಂಚೆ ಅದು ಇತ್ತು. ಪ್ರಳಯವಾದ ಮೇಲೂ ಅದು ಇರುವುದು. ಯಾವಾಗಲೂ ಬದಲಾಯಿಸುತ್ತಿರುವುದೇ ಅಶಾಶ್ವತವಾದುದು.

\begin{shloka}
ಕ್ಲೇಶೋಽಧಿಕತರಸ್ತೇಷಾಮವ್ಯಕ್ತಾಸಕ್ತಚೇತಸಾಮ್~।\\ಅವ್ಯಕ್ತಾ ಹಿ ಗತಿರ್ದುಖಂ ದೇಹವದ್ಭಿರವಾಪ್ಯತೇ \hfill॥ ೫~॥
\end{shloka}

\begin{artha}
ಅವ್ಯಕ್ತದಲ್ಲಿ ಆಸಕ್ತರಾದವರಿಗೆ ಕ್ಲೇಶ ಹೆಚ್ಚು. ಏಕೆಂದರೆ ದೇಹಧಾರಿಗಳಿಗೆ ಅವ್ಯಕ್ತ ಉಪಾಸನೆ ಬಹಳ ಕಷ್ಟ.
\end{artha}

ಭಗವಂತನ ವ್ಯಕ್ತ ಮತ್ತು ಅವ್ಯಕ್ತ ಸ್ವರೂಪ ಎರಡೂ ಒಂದೇ ನಾಣ್ಯದ ಎರಡು ಕಡೆಗಳಂತೆ. ಆದರೆ ಒಂದು ಭಾಗವನ್ನು ತಿಳಿದುಕೊಳ್ಳುವುದು ಸುಲಭ, ಅನುಷ್ಠಾನ ಮಾಡುವುದು ಸುಲಭ. ಮತ್ತೊಂದು ಭಾಗ ಸಾಧಾರಣ ಜೀವಿಗಳಿಗೆ ಬಹಳ ಕಷ್ಟ. ಅಲ್ಲಿ ಅವರ ಹಿಡಿತಕ್ಕೇ ಏನೂ ಸಿಕ್ಕುವುದಿಲ್ಲ. ಗುಣಗಳಿಲ್ಲದ ದೇವರನ್ನು ಕೊಂಡಾಡುವುದು ಹೇಗೆ, ಪ್ರಾರ್ಥಿಸುವುದು ಹೇಗೆ? ಆಕಾರವಿಲ್ಲದ ದೇವರನ್ನು ಪೂಜೆ ಮಾಡುವುದು ಹೇಗೆ? ಅವನನ್ನು ಕುರಿತು ಮನಸ್ಸಿನಲ್ಲಿ ಕಲ್ಪಿಸಿಕೊಳ್ಳುವುದಕ್ಕೆ ಎಷ್ಟೋ ಕಷ್ಟವಾಗುವುದು. ಆದರೆ ಸಾಕಾರ ಮತ್ತು ಸಗುಣ ಉಪಾಸನೆಯಲ್ಲಿ ಮನಸ್ಸಿನ ದೌರ್ಬಲ್ಯವನ್ನು ಗಮನಕ್ಕೆ ತೆಗೆದುಕೊಂಡು ಆ ಎತ್ತರವನ್ನು ಒಂದೇ ಸಲ ನೆಗೆಯುವುದಕ್ಕೆ ಆಗುವುದಿಲ್ಲವೆಂದು ಅದಕ್ಕಾಗಿ ಮೆಟ್ಟಿಲುಗಳನ್ನು ಮಾಡಿರುವರು. ಒಂದಾದ ಮೇಲೆ ಒಂದನ್ನು ಹತ್ತಿ ಹೋಗಬಹುದು. ಅವನಿಗೆ ಒಂದು ಆಕಾರವನ್ನು ಕೊಡುವರು, ಗುಣವನ್ನು ಕೊಡುವರು. ಏಕೆಂದರೆ ನಾವು ಗುಣ ಮತ್ತು ಆಕಾರವಿಲ್ಲದ ವಸ್ತುವನ್ನು ಚಿಂತಿಸಲಾರೆವು. ನೀರಿಗೆ ಯಾವ ಆಕಾರವೂ ಇಲ್ಲ. ಅದನ್ನು ಒಂದು ಬುಡ್ಡಿಯಲ್ಲಿ ಹಾಕಿದರೆ ಅದು ಬುಡ್ಡಿಯ ಆಕಾರವನ್ನು ತಾಳುವುದು. ಅದಕ್ಕೆ ಬುಡ್ಡಿಯ ಬಣ್ಣ ಬರುವುದು. ಸಗುಣ ಉಪಾಸನೆಯೂ ನಿರ್ಗುಣ ಉಪಾಸನೆಯೇ. ಸಾಕಾರ ಉಪಾಸನೆಯೂ ನಿರಾಕಾರ ಉಪಾಸನೆಯೇ. ಸಮುದ್ರದ ಯಾವುದೋ ಒಂದು ಭಾಗ ಮಂಜಿನ ಗಡ್ಡೆಯಂತೆ ಮೇಲೆದ್ದರೆ, ಅಲ್ಲಿರುವುದೂ ಸಮುದ್ರದ ನೀರೆ. ಆದರೆ ನಾಮರೂಪಗುಣವಿಲ್ಲದೆ ಭಗವಂತನನ್ನು ಚಿಂತಿಸ ಬೇಕಾದರೆ, ನಮ್ಮ ಮನಸ್ಸು ಮುಂಚೆಯೇ ಪರಿಶುದ್ಧಿಯಾಗಿರಬೇಕು, ಬುದ್ಧಿ ಬಹಳ ಹರಿತವಾಗಿರಬೇಕು. ಆಗ ಮಾತ್ರ ಸಾಧ್ಯ.

ದೇಹಧಾರಿಗಳಿಗೆ ಅಕ್ಷರ ಬ್ರಹ್ಮನ ಉಪಾಸನೆ ಕಷ್ಟ ಎಂದು ಹೇಳುವನು. ಶ‍್ರೀಕೃಷ್ಣ\break ಯಾವುದು ಮೇಲು ಯಾವುದು ಕೀಳು ಎಂದು ಹೇಳುತ್ತಿಲ್ಲ. ನಾವು ಯಾವುದನ್ನು ಅರಗಿಸಿಕೊಳ್ಳುವ ರೀತಿಯಲ್ಲಿ ಇರುವೆವು ಎಂಬ ದೃಷ್ಟಿಯಲ್ಲಿ ಹೇಳುವನು. ಕಬ್ಬಿಣ ನಮ್ಮ ದೇಹಕ್ಕೆ ಬೇಕು. ಆದರೆ ಒಂದು ಮುದ್ದೆ ಕಬ್ಬಿಣವನ್ನು ನುಂಗಿದರೆ ನಮ್ಮ ದೇಹ ಅದನ್ನು ಅರಗಿಸಿಕೊಳ್ಳುವ ಸ್ಥಿತಿಯಲ್ಲಿಲ್ಲ. ಆ ಕಬ್ಬಿಣದ ಅಂಶವನ್ನು ನಮ್ಮ ದೇಹ ಸುಲಭವಾಗಿ ಅರಗಿಸಿಕೊಳ್ಳುವ ರೀತಿಯಲ್ಲಿ ಕೊಡಬೇಕು. ಶ‍್ರೀಕೃಷ್ಣ ನಮ್ಮ ಮನೋದಾರ್ಢ್ಯವನ್ನು ಗಮನಿಸುವನು. ಅದಕ್ಕೆ ತಕ್ಕ ಚಿಕಿತ್ಸೆಯನ್ನು ಮಾಡುವನು. ಮಾನವ ಈಗ ಯಾವ ಸ್ಥಿತಿಯಲ್ಲಿದ್ದಾನೆ, ಅದನ್ನು ಗಮನಕ್ಕೆ ತೆಗೆದುಕೊಳ್ಳಬೇಕು. ನಾವೀಗ ದೇಹಧಾರಿಗಳು. ದೇಹಕ್ಕೆ ಅಂಟಿಕೊಂಡಿದ್ದೇವೆ, ದೇಹವೇ ನಮ್ಮದು ಎಂದು ಭಾವಿಸಿದ್ದೇವೆ. ದೇಹದ ಧರ್ಮವೇ ನಮ್ಮ ಧರ್ಮ ಎಂದು ಭಾವಿಸಿದ್ದೇವೆ. ಅದರಂತೆಯೇ ಇಂದ್ರಿಯ ಬುದ್ಧಿ ಮೊದಲಾದವುಗಳೆಲ್ಲ. ನಿಜವಾದ ನಮ್ಮನ್ನು ಮರೆತಿರುವೆವು. ಅದರ ಮೇಲೆ ಆರೋಪಮಾಡಿದ ತಾತ್ಕಾಲಿಕವಾದ ವಸ್ತುವನ್ನು ನಮ್ಮದು ಎಂದು ಭ್ರಮಿಸಿರುವೆವು. ನಾವು ಒಂದೇ ಸಲ ದೇಹದಿಂದ ಕಿತ್ತುಕೊಂಡು ಹೋಗುವುದಕ್ಕೆ ಆಗುವುದಿಲ್ಲ. ಒಂದು ಮುಳ್ಳಿನ ಪೊದೆಗೆ ನಮ್ಮ ಬಟ್ಟೆ ತಾಕಿದೆ. ಅದನ್ನು ನಿಧಾನವಾಗಿ ನಿಂತುಕೊಂಡು ಒಂದೊಂದು ಮುಳ್ಳಿನಿಂದಲೂ ಬಿಡಿಸಿಕೊಳ್ಳಬೇಕು. ಸುಮ್ಮನೆ ರೇಗಿ ಬಟ್ಟೆಯನ್ನು ಎಳೆದರೆ ಅದು ಹರಿದುಹೋಗುವುದು. ನಾವು ಏನನ್ನು ಕುರಿತು ಚಿಂತಿಸಬೇಕಾದರೂ ಅದಕ್ಕೆ ನಾಮರೂಪಗಳಿರಬೇಕು. ಆಗ ಮಾತ್ರ ಚಿಂತಿಸಲು ಸಾಧ್ಯ. ನಾಮರೂಪಗಳಿಲ್ಲದೆ ಚಿಂತಿಸುವುದಕ್ಕೆ ನಮ್ಮ ಮನಸ್ಸಿಗೆ ಸಾಧ್ಯವಿಲ್ಲ. ರೈಲು ಹೋಗಬೇಕಾದರೆ ಅದಕ್ಕೆ ರೈಲ್ವೆ ಕಂಬಿ ಇರಬೇಕು. ರೈಲ್ವೆ ಕಂಬಿಗೆ ಏನಾದರೂ ಆದರೆ ರೈಲು ಬೀಳುವುದು. ಹಾಗೆಯೇ ಮನಸ್ಸು ಇನ್ನೂ ಶೈಶವಾವಸ್ಥೆಯಲ್ಲಿರುವಾಗ ಅದು ಹೇಗೆ ಚಿಂತಿಸಲು ಸಾಧ್ಯವೋ ಹಾಗೆ ಅದಕ್ಕೆ ಅವಕಾಶ ಕೊಡಬೇಕು. ನಿರಾಕಾರವನ್ನು ಚಿಂತಿಸುವುದಕ್ಕೆ ಪ್ರಯತ್ನಮಾಡಿ, ಹಳೆಯ ಆಕಾರಗಳನ್ನೆಲ್ಲಾ ತೆಗೆಯುವೆವು. ನಮಗೆ ಗೊತ್ತಾಗದೆ ಅವನಿಗೆ ಹೊಸ ಆಕಾರಗಳನ್ನು ಕೊಡುವೆವು. ಈ ಪ್ರಪಂಚದಲ್ಲೆಲ್ಲ ಆಗಿರುವುದು ಹೀಗೆಯೆ. ಯಾರು ತಾವು ಆಕಾರವಿಲ್ಲದ ದೇವರನ್ನು ಚಿಂತಿಸುತ್ತೇವೆ ಎಂದು ಭಾವಿಸುವರೊ ಅವರೆಲ್ಲ ದೇವರಿಗೆ ಕೊಟ್ಟ ಹಿಂದಿನ ಆಕಾರವನ್ನು ತೆಗೆದುಹಾಕಿ ಬೇರೊಂದು ಚಿಹ್ನೆಯ ಆಕಾರವನ್ನು ಕೊಡುವರು. ದೇವರನ್ನು ಕ್ರಿಸ್ತನಂತೆ ಚಿಂತಿಸುತ್ತಿದ್ದರೆ ಅದನ್ನು ಬಿಟ್ಟು ಒಂದು ಕ್ರಾಸಿನಂತೆ ಚಿಂತಿಸುವನು; ಇಲ್ಲವೆ ಚರ್ಚಿನಂತೆ ಚಿಂತಿಸುವನು. ಇದೊಂದು ಆಕಾರವಲ್ಲವೆ? ಅವರು ಇದು ಬರೀ ಚಿಹ್ನೆ, ದೇವರ ಆಕಾರವಲ್ಲ ಎನ್ನಬಹುದು. ಹಾಗೆಯೇ ಯಾವ ಭಕ್ತನೂ ತಾನು ಪೂಜಿಸುವ ಐದಾರು ಅಂಗುಲ ಉದ್ದವಿರುವ ಗೊಂಬೆಯೇ ದೇವರು ಎನ್ನುವುದಿಲ್ಲ. ಸರ್ವಾಂತರ್ಯಾಮಿಯಾದ ಭಗವಂತನನ್ನು ಚಿಂತಿಸಲು ಅದೊಂದು ಚಿಹ್ನೆ ಎನ್ನುವನು. ಬಳ್ಳಿ ಹಬ್ಬಿ ಹೋಗಬೇಕಾದರೆ ಅದಕ್ಕೆ ಒಂದು ಬೇಲಿಯ ಆಸರೆ ಬೇಕು. ಹಾಗೆಯೇ ಮನಸ್ಸು ಚಿಂತಿಸಬೇಕಾದರೆ ನಾಮರೂಪಗಳು ಬೇಕು. ಸ್ವಾಮಿ ವಿವೇಕಾನಂದರು ದೇವರು ಹೇಗಿ ದ್ದಾನೆಯೋ ಹಾಗೆ ಕುರಿತು ಚಿಂತಿಸಬೇಕಾದರೆ ಪರಮಹಂಸರಿಗೆ ಮಾತ್ರ ಸಾಧ್ಯ, ಉಳಿದವರೆಲ್ಲ ತಾವು ಹೇಗೋ ದೇವರೂ ಹಾಗೆಯೇ ಎಂದು ಮಾತ್ರ ಚಿಂತಿಸಬಹುದು ಎನ್ನುವರು. ಒಂದು ಕೋಣ ದೇವರನ್ನು ಚಿಂತಿಸುವುದಕ್ಕೆ ಸಾಧ್ಯವಾದರೆ ದೇವರನ್ನು ಒಂದು ದೊಡ್ಡ ಕೋಣವೆಂದು ಚಿಂತಿಸುವುದು. ಮನುಷ್ಯ ದೇವರನ್ನು ಕುರಿತು ಚಿಂತಿಸಬೇಕಾದರೆ ದೇವನೊಬ್ಬ ದೊಡ್ಡ ಮನುಷ್ಯ, ಮಾನವ ದೌರ್ಬಲ್ಯಗಳಿಲ್ಲದ ಮನುಷ್ಯ ಎಂದು ಭಾವಿಸಬಹುದು. ನಾವು ಎಲ್ಲಿಯವರೆಗೆ ಮಾನವರು ಎಂದು ಭಾವಿಸುವೆವೊ ಅಲ್ಲಿಯವರೆಗೆ ನಮ್ಮ ದೇವರೂ ಮಾನವರೇ ಆಗಬೇಕಾಗಿದೆ. ನಾವು ಮಾನವನ ಪಂಜರದಿಂದ ಸೀಳಿಕೊಂಡು ಹೋದರೆ, ಅದರ ಮಿತಿಯಿಂದ ಪಾರಾದರೆ, ಆಗ ಮಾತ್ರ ದೇವರು ಹೇಗಿರುವನೋ ಹಾಗೆ ಚಿಂತಿಸಬಹುದು. ಸಾಕಾರದಲ್ಲಿ ಸಗುಣದಲ್ಲಿ ದೇವರಿಗೊಂದು ಆಕಾರ ಕೊಡುವರು, ಅವನಿಗೆ ಕೆಲವು ಗುಣಗಳನ್ನು ಕೊಡುವರು. ಆಗ ಅವನನ್ನು ಗ್ರಹಿಸುವುದಕ್ಕೆ ಸುಲಭವಾಗುವುದು. ಮನಸ್ಸನ್ನು ಬಲಾತ್ಕಾರವಾಗಿ ಮೇಲಕ್ಕೆ ನೂಕಬೇಕಾಗಿಲ್ಲ. ಮನಸ್ಸು ಬೆಳೆಯಲು ಅವಕಾಶ ಕೊಟ್ಟರೆ ಅದು ತಾನೇ ಹಿಂದಿನದನ್ನು ಬಿಟ್ಟು ಮುಂದಿನದನ್ನು ತೆಗೆದುಕೊಳ್ಳುವುದು. ಗಾಯ ಮಾಗಿದಾಗ ಹೊಕ್ಕು ತಾನಾಗಿ ಬಿದ್ದು ಹೋಗುವುದು. ಮಾಗುವುದಕ್ಕೆ ಮುಂಚೆ ಕೋತಿಚೇಷ್ಟೆ ಮಾಡಿ ಸುಮ್ಮನಿರಲಾರದೆ ಅದನ್ನು ಕಿತ್ತುಹಾಕಿದರೆ, ಪುನಃ ಇನ್ನೊಂದು ಕಟ್ಟಿಕೊಳ್ಳುವುದು. ಗಾಯ ಮಾಗುವುದು ನಿಧಾನವಾಗುವುದು. ಶ‍್ರೀರಾಮಕೃಷ್ಣರು ಇನ್ನೊಂದು ಉಪಮಾನವನ್ನು ಕೊಡುತ್ತಿದ್ದರು. ತೆಂಗಿನ ಮರ ಬೆಳೆದು ಹೋಗುತ್ತಿದ್ದರೆ ಕೆಳಗೆ ಇರುವ ಗರಿಗಳೆಲ್ಲ ತಾವೇ ಬಿದ್ದು ಹೋಗುತ್ತವೆ. ಅದನ್ನು ಯಾರೂ ಬಲಾತ್ಕಾರವಾಗಿ ಕೀಳಬೇಕಾಗಿಲ್ಲ. ಹಾಗೆಯೇ ಆಧ್ಯಾತ್ಮಿಕ ಜೀವನದಲ್ಲಿ ವಿಕಾಸವಾಗುತ್ತ ಬಂದರೆ ಅನಾವಶ್ಯಕವಾಗಿರುವುದೆಲ್ಲ ತಮಗೆ ತಾವೆ ಬಿಟ್ಟು ಹೋಗುತ್ತವೆ. ಮೊದಲು ನಾವು ಆಧ್ಯಾತ್ಮಿಕ ಜೀವನವನ್ನು ಪೋಷಣೆ ಮಾಡುವುದಕ್ಕೆ ಗಮನ ಕೊಡಬೇಕು. ಅದಿಲ್ಲದೆ ಸುಮ್ಮನೆ ಗಿಡವನ್ನು ಕಸಿಮಾಡುತ್ತಾ ಹೋದರೆ ಅದು ಒಣಗಿಹೋಗುವುದು. ಆದಕಾರಣವೆ, ಶ‍್ರೀಕೃಷ್ಣ ಸಾಧಕರು ವ್ಯಕ್ತದಿಂದ ಅವ್ಯಕ್ತದ ಕಡೆಗೆ ಸಗುಣದಿಂದ ನಿರ್ಗುಣದ ಕಡೆಗೆ ಹೋಗುವುದು ಮೇಲು ಎನ್ನುವನು.

\begin{shloka}
ಯೇ ತು ಸರ್ವಾಣಿ ಕರ್ಮಾಣಿ ಮಯಿ ಸಂನ್ಯಸ್ಯ ಮತ್ಪರಾಃ~।\\ಅನನ್ಯೇನೈವ ಯೋಗೇನ ಮಾಂ ಧ್ಯಾಯಂತ ಉಪಾಸತೇ \hfill॥ ೬~॥
\end{shloka}

\begin{shloka}
ತೇಷಾಮಹಂ ಸಮುದ್ಧರ್ತಾ ಮೃತ್ಯುಸಂಸಾರಸಾಗರಾತ್~।\\ಭವಾಮಿ ನ ಚಿರಾತ್ ಪಾರ್ಥ ಮಯ್ಯಾವೇಶಿತಚೇತಸಾಮ್ \hfill॥ ೭~॥
\end{shloka}

\newpage

\begin{artha}
ಅರ್ಜುನ, ಯಾರು ಸಮಸ್ತ ಕರ್ಮಗಳನ್ನು ನನ್ನಲ್ಲಿ ಅರ್ಪಿಸಿ, ಮತ್ಪರರಾಗಿ ಅನನ್ಯವಾದ ಯೋಗ\-ದಿಂದಲೇ ನನ್ನನ್ನು ಧ್ಯಾನಿಸುತ್ತ ಉಪಾಸನೆ ಮಾಡುವರೊ, ನನ್ನಲ್ಲಿ ನೆಟ್ಟ ಮನಸ್ಸುಳ್ಳ ಅವರನ್ನು ನಾನು ಮೃತ್ಯುಸಂಸಾರ ಸಾಗರದಿಂದ ಶೀಘ್ರವಾಗಿ ಉದ್ಧಾರ ಮಾಡುತ್ತೇನೆ.
\end{artha}

ಶ‍್ರೀಕೃಷ್ಣ ಇಂದ್ರಿಯವನ್ನು ಸಂಪೂರ್ಣವಾಗಿ ನಿಗ್ರಹಿಸದೆ ಅದರ ಮೂಲಕ ಭಗವಂತನ ಕೆಲಸ ಮಾಡುವವನನ್ನು ಕುರಿತು ಹೇಳುತ್ತಾನೆ. ಎರಡೂ ಒಂದೇ ಗುರಿಗೇನೋ ಕೊಂಡೊಯ್ಯುವುದು. ಆದರೆ ಇಂದ್ರಿಯದ ಮೂಲಕ ದೇವರ ಕೆಲಸಮಾಡುವುದು ಇಂದ್ರಿಯವನ್ನು ನಿಗ್ರಹಿಸುವುದಕ್ಕಿಂತ ಸುಲಭದ ಹಾದಿ. ಯಾರು ಬೇಕಾದರೂ ಇದನ್ನು ಮಾಡಬಹುದು. ಈ ದಾರಿ ಸ್ವಲ್ಪ ದೂರ; ಆದರೆ ಕಡಿದಲ್ಲ; ನಡೆದುಕೊಂಡು ಹೋಗಿ ಗುರಿ ಮುಟ್ಟಬಹುದು. ಸ್ವಲ್ಪ ನಿಧಾನವಾಗಬಹುದು, ಅಷ್ಟೆ. ಆದರೆ ಕಷ್ಟ ಮತ್ತು ಅಪಾಯ ಕಡಮೆ.

ಸಮಸ್ತ ಕರ್ಮಗಳನ್ನು ಭಗವಂತನಿಗೆ ಅರ್ಪಿಸಿ ಕೆಲಸ ಮಾಡುವನು. ಇಲ್ಲಿ ಶ‍್ರೀಕೃಷ್ಣ ಎಲ್ಲಾ ಕರ್ಮಗಳನ್ನು ಎಂದು ಹೇಳುವನು. ಬರೀ ದೇವರ ಪೂಜೆ, ಪ್ರಾರ್ಥನೆ, ಧ್ಯಾನ ಇವುಗಳನ್ನು ಮಾತ್ರ ಹೇಳುವುದಿಲ್ಲ. ಒಬ್ಬ ತನ್ನ ಪಾಲಿಗೆ ಬರುವ ಕರ್ಮವನ್ನೆಲ್ಲಾ ಅದು ದೇವರಿಗೆ ಸಂಬಂಧಪಟ್ಟಿರಲಿ ಅಥವಾ ಲೌಕಿಕವಾಗಿರಲಿ, ದೇವರ ಸೇವೆಯಂತೆ ಅದನ್ನು ಮಾಡುತ್ತಾನೆ. ಯಾವಾಗ ದೇವರಿಗಾಗಿ ಮಾಡುತ್ತಾನೆಯೋ ಆಗ ಕೆಟ್ಟ ಕೆಲಸ ಮಾಡುವುದಕ್ಕೇ ಆಗುವುದಿಲ್ಲ. ನಮ್ಮ ಕರ್ತವ್ಯವನ್ನು ಅರೆಮನಸ್ಸಿನಿಂದ ಮಾಡುವುದಕ್ಕೆ ಆಗುವುದಿಲ್ಲ. ಅದನ್ನು ಅಚ್ಚುಕಟ್ಟಾಗಿ ಮಾಡಬೇಕಾಗುವುದು. ಹಾಗೇ ಕೆಲಸವನ್ನು ಮಾಡಿ ಫಲಾಪೇಕ್ಷೆಯನ್ನು ತೊರೆದರೆ, ಅದೇ ನಮ್ಮ ಮನಸ್ಸನ್ನು ಶುದ್ಧಿಗೊಳಿಸುವುದು. ಅದರಿಂದ ಮನಸ್ಸಿಗೆ ಏಕಾಗ್ರತೆ ಹೆಚ್ಚುವುದು. ಏಕಾಗ್ರವಾದ ಮನಸ್ಸಿನಿಂದ ಭಗವಂತನನ್ನು ಕುರಿತು ಚಿಂತಿಸುವುದಕ್ಕೆ ಬಹಳ ಸಹಾಯವಾಗುವುದು.

ಅವನು ಮತ್ಪರನಾಗಿರಬೇಕು. ದೇವರೇ ಶ್ರೇಷ್ಠ. ಅವನಿಗಿಂತ ಬೇರೆ ಯಾವುದೂ ಮಿಗಿಲಾಗಿಲ್ಲ ಎಂದು ಅರಿತಿರಬೇಕು. ಇವನು ಸೇರುವ ಗುರಿಯೇ ಅದು. ಅದನ್ನು ಬಿಟ್ಟು ಇವನು ಯಾವ ಅಡ್ಡಹಾದಿಯ ಕಡೆಗೂ ವಾಲಬಾರದು. ದೇವರ ಕಡೆ ನಡೆಯುವಾಗ ಕೆಲವು ವೇಳೆ ಹಲವು ಸಿದ್ಧಿಗಳು ಬರಬಹುದು. ಸಿದ್ಧಿಗಳು ಎಂದರೆ ಅತೀಂದ್ರಿಯ ಶಕ್ತಿಗಳು. ದೂರದಲ್ಲಿ ಏನಾಗುತ್ತದೆ ಎಂಬುದನ್ನು ನೋಡಬಹುದು ಮತ್ತು ಕೇಳಬಹುದು. ಒಬ್ಬನ ಮನಸ್ಸಿನಲ್ಲಿ ಏನಿದೆ ಎಂಬುದನ್ನು ಅರಿಯುವುದು, ಮತ್ತು ಹಲವು ವಸ್ತುಗಳನ್ನು ಬರಿಸುವುದು ಮಾಯವಾಗುವಂತೆ ಮಾಡುವುದು ಇತ್ಯಾದಿ. ಆದರೆ ನಿಜವಾದ ಭಕ್ತನಾದವನು ಇವುಗಳನ್ನೆಲ್ಲ ನಿರಾಕರಿಸುವನು. ಇವುಗಳೆಲ್ಲ ಭಗವದ್​ಭಕ್ತಿಯ ದೃಷ್ಟಿಯಿಂದ ಹೇಯವಾದ ವಸ್ತುಗಳು. ಇವು ದೇವರನ್ನು ಮರೆಸುವುವು. ಪ್ರಪಂಚಕ್ಕೆ ಇದನ್ನು ತೋರಿ ಜನರಿಂದ ಕೀರ್ತಿ ಗೌರವ ಮುಂತಾದುವುಗಳನ್ನು ಪಡೆದು ಇದರಲ್ಲಿ ನಿರತನಾಗುವಂತೆ ಮಾಡುವುವು. ಅವನು ಗುರಿಯನ್ನು ಮರೆಯುವನು, ಅಡ್ಡಹಾದಿಯನ್ನು ಹಿಡಿಯುವನು. ಆದರೆ ನಿಜವಾದ ಭಕ್ತ ಇದನ್ನು ಮೊದಲಲ್ಲಿಯೇ ಗ್ರಹಿಸುವನು. ಅವನು ಇತ್ತಕಡೆ ಸ್ವಲ್ಪವೂ ಗಮನ ಕೊಡುವುದಿಲ್ಲ.

ಅವನು ಅನನ್ಯಯೋಗದಲ್ಲಿ ನಿರತನಾಗಿರುವನು. ಅವನು ಯಾವಾಗಲೂ ಭಗವಂತನನ್ನು ಕುರಿತು ಚಿಂತಿಸುತ್ತಿರುವನು. ಆ ಚಿಂತನೆಗೆ ಒಂದು ರಜಾ ಇಲ್ಲ. ಈ ಜೀವನದಲ್ಲಿ ಕೆಲವು ವೇಳೆ ನಾವು ಕೆಲಸಮಾಡಿ ಬೇಜಾರಾದರೆ ರಜಾ ತೆಗೆದುಕೊಳ್ಳುತ್ತೇವೆ. ಹಾಗಲ್ಲ ಭಗವಂತನನ್ನು ಕುರಿತು ಚಿಂತಿಸುವುದು. ಯಾವಾಗಲೂ ಅವನು ಭಗವಂತನನ್ನು ಕುರಿತು ಚಿಂತಿಸುವುದರಲ್ಲಿ ಮಗ್ನನಾಗಿರುವನು. ಅವನ ಮನಸ್ಸು ಅವನನ್ನು ಚಿಂತಿಸುತ್ತಿರುವುದು. ಇಲ್ಲವೆ ಕೈಕಾಲುಗಳು ಅವನಿಗೆ ಸಂಬಂಧಪಟ್ಟ ಕೆಲಸವನ್ನು ಮಾಡುತ್ತಿರುವುವು. ಇದು ಅವನ ಸ್ವಭಾವವಾಗಿ ಹೋಗಿರುವುದು. ಹೃದಯ ನಮ್ಮ ಇಚ್ಛೆ ಇಲ್ಲದೆ ಬಡಿಯುತ್ತಿರುವುದು. ಶ್ವಾಸಕೋಶಗಳು ನಮ್ಮ ಇಚ್ಛೆ ಇಲ್ಲದೆ ಉಸಿರಾಡುವುವು. ಹಾಗೆಯೇ ಯಾವಾಗಲೂ ಭಗವಂತನನ್ನು ಕುರಿತು ಚಿಂತಿಸುವುದು ಇವನ ಸ್ವಭಾವವಾಗಿದೆ. ಉತ್ತರಮುಖಿ ಯಾವಾಗಲೂ ಉತ್ತರ ದಿಕ್ಕನ್ನು ತೋರುತ್ತಿರುವುದು. ಅದೇ ಅದರ ಸ್ವಭಾವವಾಗಿದೆ; ಮತ್ತು ಅದರ ಧರ್ಮ. ಅವನು ದೇವರಿಂದ ಏನನ್ನೂ ಯಾಚಿಸುವುದಿಲ್ಲ. ಪ್ರೀತಿಗಾಗಿ ಪ್ರೀತಿಸುವನು. ಅವನನ್ನು ಕುರಿತು ಚಿಂತಿಸಲು ಅವನಿಗೆ ಆಸೆ. ಅದಕ್ಕಾಗಿ ಅವನನ್ನು ಕುರಿತು ಚಿಂತಿಸುವನು. ಶಿವನ ಕೆಲಸವನ್ನು ಮಾಡಲು ಆಸೆ, ಅದಕ್ಕಾಗಿ ಅದನ್ನು ಮಾಡುವನು. ಇದಿಲ್ಲದೆ ಅವನಿಗೆ ಬೇರೆ ಏನೂ ಬೇಕಾಗಿಲ್ಲ.

ಅವನಲ್ಲಿ ನೆಟ್ಟ ಮನಸ್ಸುಳ್ಳವನು ಅವನು. ಅವನು ಯಾವ ಕೆಲಸ ಮಾಡುತ್ತಿರಲಿ ಮನಸ್ಸು ಮಾತ್ರ ಯಾವಾಗಲೂ ಭಗವಂತನ ಕಡೆ ಹರಿದುಹೋಗುತ್ತಿರುವುದು. ಅಲ್ಲಿ ದೇವರಲ್ಲದ ಇತರ ವಸ್ತುಗಳಿಗೆ ತಂಗಲು ಅವಕಾಶ ಇರುವುದಿಲ್ಲ. ಅವನು ಯಾವ ಕಷ್ಟವನ್ನೂ ಗಮನಿಸುವುದಿಲ್ಲ. ಯಾವ ಅಪಾಯವನ್ನು ಬೇಕಾದರೂ ಎದುರಿಸಬಲ್ಲ. ಏಕೆಂದರೆ ಅವನು ಭಗವಂತನಿಗೆ ಅರ್ಪಿತ\-ನಾದವನು. ಇವನ ಉತ್ಸಾಹ ಮತ್ತು ಸ್ಫೂರ್ತಿಯನ್ನು ಯಾವುದೂ ತಗ್ಗಿಸಲಾರದು.

ಯಾವಾಗ ಮನಸ್ಸನ್ನೆಲ್ಲ ತನ್ನಮೇಲೆ ಇಟ್ಟಿರುವನೋ ಆಗ ಶೀಘ್ರವಾಗಿ ಉದ್ಧಾರಮಾಡುತ್ತೇನೆ ಎನ್ನುತ್ತಾನೆ. ಏನೋ ಸತ್ತಮೇಲೆ ಉದ್ಧಾರ ಆಗುವುದಲ್ಲ. ಸತ್ತವರಾರೂ ನಮಗೆ ಬಂದು ಹೇಳಿಲ್ಲ ತಾವು ಹೇಗೆ ಉದ್ಧಾರವಾದೆವು ಎಂಬುದನ್ನು. ಇರುವಾಗಲೇ ಅವರು ಮುಕ್ತರಾಗುತ್ತಾರೆ, ಸಂಸಾರ ಬಂಧನದಿಂದ ಪಾರಾಗುತ್ತಾರೆ. ಶೀಘ್ರವಾಗಿ ಎಂದು ಶ‍್ರೀಕೃಷ್ಣ ಹೇಳುತ್ತಾನೆ. ಯಾವಾಗ ಉರಿಯುತ್ತಿರುವ ಒಲೆಗೆ ಹೊಸ ಸೌದೆಯನ್ನು ಹಾಕುತ್ತೇವೆಯೋ ಆಗ ಅದೂ ಕೂಡ ಉರಿಯಲು ಮೊದಲಾಗುವುದು. ಹಸಿಯಾಗಿದ್ದರೆ ಸ್ವಲ್ಪ ನಿಧಾನವಾಗಬಹುದು ಹತ್ತಿಕೊಳ್ಳುವುದಕ್ಕೆ. ಆ ಹಸಿ ಸೌದೆ ಧಗಧಗಿಸುವ ಒಲೆಯೊಳಗೆ ಇರುವುದರಿಂದ ಬೇಗ ಒಣಗಿಹೋಗುವುದು. ಭಗವಂತನಲ್ಲೆ ನೆಟ್ಟ ಮನಸ್ಸುಳ್ಳವನು, ಇನ್ನು ಯಾವ ಅಜ್ಞಾನದ ಕಾರ್ಯವನ್ನೂ ಮಾಡಲಾರ. ಇನ್ನುಮೇಲೆ ಅವನಲ್ಲಿ ಯಾವ ಬಯಕೆಗಳೂ ಇರುವುದಿಲ್ಲ. ಆಸೆ ಮತ್ತು ಅಜ್ಞಾನದ ಹಸಿಯೆಲ್ಲ ಒಣಗಿಹೋಗಿದೆ.

ಮುಕ್ತಿ ಎಂದರೆ ಏನು ಎಂಬುದನ್ನು ವಿವರಿಸುತ್ತಾನೆ. ಮೊದಲನೆಯದೇ ಅವನ ಮನಸ್ಸಿನ ಪಾತ್ರೆಯಲ್ಲಿ ಸದಾಕಾಲವೂ ಭಗವಂತನನ್ನು ಅನುಭವಿಸುತ್ತಿರುವನು. ಅಲ್ಲಿ ಅವನೇ ಇರುವನು. ವಿಷಯವಸ್ತುಗಳೆಲ್ಲ ಆ ಸ್ಥಳವನ್ನು ಬಿಟ್ಟುಹೋಗಿವೆ. ಎರಡನೆಯದೇ ಆ ಮನುಷ್ಯ ಕಾಲವಾದರೆ ಈ ಹುಟ್ಟು ಸಾವುಗಳಿಂದ ಕೂಡಿದ ಸಂಸಾರಕ್ಕೆ ಅವನು ಪುನಃ ಬರುವುದಿಲ್ಲ. ಅವನು ಹೋದರೆ ಒಂದೇ ಸಲ ಹಾರಿಹೋಗುವನು. ಬೇರೆ ಬಾಗಿಲಿನಿಂದ ಬರುವುದಕ್ಕೆ ಹೋಗುವುದಿಲ್ಲ. ಕರ್ಮದ ಸಾಲವಿದ್ದರೆ ತಾನೇ ತೀರಿಸಲು ಬರಬೇಕಾಗುವುದು. ಬದುಕಿರುವಾಗಲೇ ಅವನು ಪುಣದಿಂದ ಪಾರಾಗಿರುವನು. ಅವನಲ್ಲಿ ಯಾವ ವಾಸನೆಯೂ ಇಲ್ಲ. ಲೋಕದ ಯಾವ ಭೋಗಗಳೂ ಅವನಿಗೆ ಬೇಕಾಗಿಲ್ಲ. ಅವನಲ್ಲಿ ಯಾವ ಸಂಶಯವೂ ಇಲ್ಲ. ಹಗಲಿನಲ್ಲಿ ಸೂರ್ಯನನ್ನು ನೋಡುವಂತೆ ಅವನು ದೇವರನ್ನು ಅನುಭವಿಸುತ್ತಿರುವನು. ಅವನು ಯಾರಿಗೂ ಏನೂ ಕೊಡಬೇಕಾಗಿಲ್ಲ, ಏನೂ ಮಾಡಬೇಕಾಗಿಲ್ಲ. ಆತ ಪುಣಮುಕ್ತ ಮತ್ತು ಕೃತಾರ್ಥ. ಅವನು ಇನ್ನುಮೇಲೆ ಬರುವ ಆವಶ್ಯಕತೆಯಿಲ್ಲ.

\begin{shloka}
ಮಯ್ಯೇವ ಮನ ಆಧತ್ಸ್ವ ಮಯಿ ಬುದ್ಧಿಂ ನಿವೇಶಯ~।\\ನಿವಸಿಷ್ಯಸಿ ಮಯ್ಯೇವ ಅತ ಊರ್ಧ್ವಂ ನ ಸಂಶಯಃ \hfill॥ ೮~॥
\end{shloka}

\begin{artha}
ನನ್ನಲ್ಲಿಯೇ ಮನಸ್ಸನ್ನು ಇಡು. ನನ್ನಲ್ಲಿಯೇ ಬುದ್ಧಿಯನ್ನು ಸೇರಿಸು. ಅನಂತರ ನನ್ನಲ್ಲಿಯೇ ವಾಸಮಾಡುವೆ. ಇದರಲ್ಲಿ ಸಂಶಯವಿಲ್ಲ.
\end{artha}

ಮನಸ್ಸನ್ನು ಭಗವಂತನಲ್ಲಿ ಇಡಬೇಕು. ಮನಸ್ಸಿನ ಸ್ವಭಾವವೇ ಏನನ್ನಾದರೂ ಚಿತ್ರಿಸಿಕೊಳ್ಳುವುದು, ಸ್ಮರಿಸಿಕೊಳ್ಳುವುದು. ಅದು ಯಾವಾಗಲೂ ಸುಮ್ಮನೆ ಮಾತ್ರ ಇರುವುದಿಲ್ಲ. ಈಗ ದೇಹಕ್ಕೆ ಇಂದ್ರಿಯಕ್ಕೆ ಸಂಬಂಧಪಟ್ಟ ಹಲವಾರು ಅನುಭವಗಳನ್ನು ಚಿಂತಿಸುತ್ತಿರುವುದು. ಕಲ್ಪಿಸಿಕೊಳ್ಳದೇ ಇರುವುದಕ್ಕೆ ಆಗದೇ ಇದ್ದರೆ ದೇವರ ವಿಷಯವನ್ನು ಕಲ್ಪಿಸಿಕೊಳ್ಳುವ; ಅದಕ್ಕೆ ಸಂಬಂಧಪಟ್ಟ ಚಿತ್ರಗಳನ್ನು ಮನನ ಮಾಡುವ. ಅವನಿಗೆ ಸಂಬಂಧಪಟ್ಟ ಸ್ಮೃತಿಗಳನ್ನು ಮೆಲುಕು ಹಾಕುವ. ಅದು ಮನದಲ್ಲಿ ಉತ್ತಮ ಸಂಸ್ಕಾರಗಳನ್ನು ಬಿಡುವುದು.

ಅದರಂತೆಯೇ ಬುದ್ಧಿಯನ್ನು ಕೂಡಾ ಭಗವಂತನಿಗೆ ಅರ್ಪಣ ಮಾಡಬೇಕು. ಅನೇಕ ವೇಳೆ ನಾವು ತುಂಬಾ ಜಾಣರು ಮತ್ತು ಬುದ್ಧಿವಂತರು ಎಂದು ಭಾವಿಸುತ್ತೇವೆ. ಆದರೆ ಆ ಜಾಣತನ ಮತ್ತು ಬುದ್ಧಿವಂತಿಕೆಯನ್ನು ನಮ್ಮ ಹೀನ ಆಸೆಗಳನ್ನು ತೃಪ್ತಿಪಡಿಸಿಕೊಳ್ಳುವುದಕ್ಕೆ ಉಪಯೋಗಿಸಿ\-ಕೊಂಡಿರುವೆವು. ನಮ್ಮ ಬುದ್ಧಿಯ ದೀವಿಗೆಯಿಂದ ದೇವರನ್ನು ಹುಡುಕುವುದಕ್ಕೆ ಹೋಗುವುದಿಲ್ಲ. ಅದರ ಬದಲು ಲೌಕಿಕ ವಸ್ತುಗಳನ್ನು ಹುಡುಕಲು ಹೋಗುತ್ತೇವೆ. ನಮ್ಮಲ್ಲಿರುವುದು ಲಾಯರಿನ ಬುದ್ಧಿಯಂತೆ. ಲಾಯರು ಸತ್ಯವನ್ನು ತಿಳಿಯಬೇಕಾಗಿಲ್ಲ. ಅವನ ಕಕ್ಷಿ ಏನನ್ನು ಹೇಳುವನೋ ಅದನ್ನು ಸರಿ ಎಂದು ಕೋರ್ಟಿನಲ್ಲಿ ವಾದಿಸಬೇಕಾಗಿದೆ. ಕೇಸನ್ನು ಕೋರ್ಟಿನಲ್ಲಿ ಗೆಲ್ಲುವುದಕ್ಕೆ ತನ್ನ ಜಾಣತನವನ್ನೆಲ್ಲ ಉಪಯೋಗಿಸುವನೇ ಹೊರತು ಸತ್ಯಾಕಾಂಕ್ಷಿಯಾಗಿಲ್ಲ ಅವನು. ಅದರಂತೆಯೇ ನಮ್ಮ ಬಯಕೆಗಳನ್ನು ತೃಪ್ತಿಪಡಿಸಿಕೊಳ್ಳುವುದಕ್ಕೆ ಜ್ಞಾನದಲ್ಲಿರುವಾಗ ಬುದ್ಧಿಯನ್ನು ಉಪಯೋಗಿಸುತ್ತೇವೆ. ಆದರೆ ಭಕ್ತನಾದ ಮೇಲೆ ಆ ಬುದ್ಧಿಯಿಂದ ದೃಶ್ಯ ವಸ್ತುಗಳ ಕ್ಷಣಿಕತೆಯನ್ನು ಅರಿಯುವನು.

\newpage

ಯಾವಾಗ ಮನಸ್ಸು ಬುದ್ಧಿ ಎಂಬ ಇಬ್ಬರು ಪ್ರಬಲ ಸೇವಕರು ಭಗವಂತನ ಕಡೆಗೆ ಹೋಗು\-ತ್ತಾರೋ ಅವನನ್ನು ತಿಳಿಯಲು ಅವನನ್ನು ಪ್ರೀತಿಸಲು ಯತ್ನಿಸುತ್ತಾರೆಯೋ ಅನಂತರ ಅವನಲ್ಲಿಯೇ ನಾವು ವಾಸ ಮಾಡುವೆವು. ನಾವೆಲ್ಲಿರುವೆವು ಅಂದರೆ ನಮ್ಮ ಮನಸ್ಸು ಮತ್ತು ಬುದ್ಧಿ ಇರುವ ಕಡೆ. ಯಾವಾಗ ಭಗವಂತನಲ್ಲಿ ಇವೆರಡೂ ನಾಟಿರುವವೊ ಆಗ ಬದುಕಿರುವಾಗ ಅವನಲ್ಲಿ ಬಾಳುತ್ತಿರುವೆವು. ಅನಂತರ ಈ ಜೀವನಯಾತ್ರೆ ಪೂರೈಸಿದ ಮೇಲೂ ಅವನಲ್ಲಿಯೇ ನಾವು ಇರುವೆವು.

ಈ ವಿಷಯದಲ್ಲಿ ಸಂಶಯವಿಲ್ಲ ಎನ್ನುತ್ತಾನೆ ಶ‍್ರೀಕೃಷ್ಣ. ಏಕೆಂದರೆ ಅವನು ಇದ್ದಾನೆ. ಅವನು ನಮ್ಮನ್ನು ಕೈಬಿಡುವುದಿಲ್ಲ ಎಂಬುದೇ ಊರುಗೋಲು ನಮ್ಮ ಪ್ರಯಾಣಕ್ಕೆ. ಅನುಮಾನ ಪ್ರಕೃತಿಯುಳ್ಳವರು ನಾವು. ನಮಗೆ ಪದೇ ಪದೇ ಅವನು ಸತ್ಯ, ಅವನನ್ನು ನಂಬಿದರೆ ಅವನು ನಮ್ಮನ್ನು ತ್ಯಜಿಸುವುದಿಲ್ಲ ಎಂಬುದನ್ನು ಒತ್ತಿ ಒತ್ತಿ ಹೇಳಬೇಕಾಗಿದೆ. ಇಲ್ಲದೇ ಇದ್ದರೆ ನಾವು ಪ್ರಪಂಚದ ಕಡೆಗೆ ಜಾರಿಹೋಗುತ್ತೇವೆ.

\begin{shloka}
ಅಥ ಚಿತ್ತಂ ಸಮಾಧಾತುಂ ನ ಶಕ್ನೋಷಿ ಮಯಿ ಸ್ಥಿರಮ್~।\\ಅಭ್ಯಾಸಯೋಗೇನ ತತೋ ಮಾಮಿಚ್ಛಾಪ್ತುಂ ಧನಂಜಯ \hfill॥ ೯~॥
\end{shloka}

\begin{artha}
ಅರ್ಜುನ, ಒಂದು ವೇಳೆ ಚಿತ್ತವನ್ನು ನನ್ನಲ್ಲಿ ಸ್ಥಿರಗೊಳಿಸಲು ಸಾಧ್ಯವಾಗದೇ ಇದ್ದರೆ, ಅಭ್ಯಾಸ ಯೋಗದಿಂದ ನನ್ನನ್ನು ಪಡೆಯಲು ಯತ್ನಿಸು.
\end{artha}

ಒಂದೇ ಸಲ ನಮ್ಮ ಮನಸ್ಸನ್ನು ಅವನ ಪಾದಪದ್ಮಗಳಲ್ಲಿ ಕಟ್ಟಿಹಾಕಲು ಸಾಧ್ಯವಿಲ್ಲ. ನಾವೇನೋ ಕಟ್ಟಿಹಾಕುತ್ತೇವೆ. ಆದರೆ ಅದಕ್ಕೆ ಬಿಗಿದ ದಾರ ತುಂಬ ದುರ್ಬಲವಾಗಿದೆ. ಆ ಮನಸ್ಸಾದರೋ ಎಳೆದು ಕಿತ್ತುಕೊಂಡು ಹೋಗಿ ಪುನಃ ವಿಷಯವಸ್ತುವಿನ ಹೊಲದಲ್ಲಿ ಮೇಯುವುದು. ಇದು ನಮ್ಮ ಜೀವನದಲ್ಲಿ ವೇದ್ಯವಾದ ಅನುಭವ. ಹಾಗಿದ್ದರೆ ಏನು ಮಾಡಬೇಕು ಎಂದರೆ, ಶ‍್ರೀಕೃಷ್ಣ ಅಭ್ಯಾಸದಿಂದ ದೇವರ ಕಡೆ ಹೋಗುವಂತೆ ಮಾಡಬೇಕು ಎಂದು ಹೇಳುವನು.

ಇದೇ ಗೀತೆಯಲ್ಲಿ ಬರುವ ಮುಖ್ಯವಾದ ಅಭ್ಯಾಸಯೋಗ. ಒಂದೇ ಸಲ ಮನಸ್ಸನ್ನು ದೇವರಲ್ಲಿ ನಿಲ್ಲಿಸಬಲ್ಲ ವ್ಯಕ್ತಿಗಳು ಬಹಳ ಅಪರೂಪ ಪ್ರಪಂಚದಲ್ಲಿ. ಹಲವರಿಗೆ ದೇವರ ಮೇಲೆ ಮನಸ್ಸನ್ನು ನಿಲ್ಲಿಸಬೇಕೆಂದು ಇಚ್ಛೆಯಿದೆ. ನಾವು ಕೆಲವು ವೇಳೆ ಪ್ರಯತ್ನಮಾಡಿ ನಿರಾಶರಾಗಿ ಕೈಬಿಡುವೆವು. ದೇವರು ಈ ಜನ್ಮದಲ್ಲಿ ಅದನ್ನು ಪಡೆಯಲು ನಮ್ಮ ಹಣೆಯಲ್ಲಿ ಬರೆಯಲಿಲ್ಲ ಎಂದು ಹತಾಶರಾಗುವೆವು. ಆದರೆ ಯಾರೂ ದೇವರ ಮೇಲೆ ಒಂದೇ ಸಲ ಮನಸ್ಸನ್ನು ಹಾಕಿ ಯಶಸ್ವಿ ಆದವರಿಲ್ಲ. ಎಲ್ಲರೂ ಎಷ್ಟೋ ಸಲ ಸೋತಿದ್ದಾರೆ. ಆದರೆ ಅದರಿಂದ ನಿರಾಶರಾಗದೆ ಪುನಃ ಪುನಃ ಪ್ರಯತ್ನ ಮಾಡಿ ಕ್ರಮೇಣ ಅದರಲ್ಲಿ ಪಳಗುತ್ತಾ ಬಂದು ಮನಸ್ಸನ್ನು ದೇವರ ಮೇಲೆ ಸಂಪೂರ್ಣ ಹಾಕುವುದನ್ನು ಕಲಿತಿರುವರು. ಜೀವನದಲ್ಲಿ ಬಿಡದೆ ಅಭ್ಯಾಸ ಮಾಡಿದರೆ ಏನನ್ನು ಬೇಕಾದರೂ ಪಡೆಯಬಹುದು. ಅಭ್ಯಾಸವೇ ಒಬ್ಬನನ್ನು ಪರಿಪೂರ್ಣನನ್ನಾಗಿ ಮಾಡುವುದು. ಪ್ರತಿಯೊಂದು ಸಲ ಅಭ್ಯಾಸ ಮಾಡಿದಾಗಲೂ ನಾವು ಅದಕ್ಕೆ ಸಂಬಂಧಪಟ್ಟ ಉತ್ತಮ ಸಂಸ್ಕಾರಗಳನ್ನು ಮನಸ್ಸಿನಲ್ಲಿ ಅಭಿವೃದ್ಧಿಪಡಿಸಿಕೊಳ್ಳುವೆವು. ಕ್ರಮೇಣ ಈ ಉತ್ತಮ ಸಂಸ್ಕಾರದ ಮೊತ್ತ ವೃದ್ಧಿಯಾಗಿ ಹಿಂದಿನಿಂದ ಬಂದ ಹೀನ ಸಂಸ್ಕಾರ ದುರ್ಬಲವಾಗಿ ಕೊನೆಗೆ ನಮ್ಮನ್ನು ಬಿಟ್ಟುಹೋಗುವುದು. ಸರ್ಕಸ್ಸಿನಲ್ಲಿ ಅಭ್ಯಾಸಬಲದಿಂದ ದೇಹದ ಮೂಲಕ ಎಂತೆಂತಹ ಸಾಹಸ ಕೃತ್ಯಗಳನ್ನು ಮಾಡುವರು. ಇದನ್ನೆಲ್ಲ ಏನು ಅವರು ಹುಟ್ಟುತ್ತಲೇ ಕಲಿತುಕೊಂಡು ಬಂದರೆ? ಇಲ್ಲ, ಪ್ರತಿದಿನವೂ ಹಲವು ಗಂಟೆಗಳು ಬಿಡದೆ ಅಭ್ಯಾಸ ಮಾಡಿ, ಮಾಡಿ ಅದನ್ನು ಕಲಿತರು. ತಂತಿಯ ಮೇಲೆ ನಡೆಯುವರು. ಒಂದು ಚಕ್ರದ ಸೈಕಲ್ಲಿನ ಮೇಲೆ ಸವಾರಿ ಮಾಡುವರು. ಓಡುವ ಕುದುರೆಯ ಮೇಲೆ ನಿಂತುಕೊಂಡು ಹೋಗುವರು. ಅದರ ಮೇಲೆ ನೆಗೆಯುವರು, ಅದರ ಮೇಲೆ ಪಲ್ಟಿ ಹಾಕುವರು. ಇದನ್ನೆಲ್ಲ ಅವರು ಹಲವು ಕಾಲ ಬಿಡದೆ ಅಭ್ಯಾಸ ಮಾಡಿ ಪಡೆದರು. ಹಾಗೆಯೇ ಯೋಗಿ ಮನಸ್ಸನ್ನು ದೇವರ ಕಡೆ ಹರಿಸುವನು. ಮೊದಮೊದಲು ನಮ್ಮ ಮನಸ್ಸಿನ ದೌರ್ಬಲ್ಯ ಒಟ್ಟಿಗೆ ಸೇರಿ ಧಾಳಿ ಇಟ್ಟು ನಮ್ಮನ್ನು ದೇವರ ಕಡೆಯಿಂದ ಸೆಳೆದುಕೊಂಡು ಬರುತ್ತದೆ. ಆದರೆ ಬಿಡದೆ ನಾವು ಪದೇ ಪದೇ ಹೋರಾಡುತ್ತಿದ್ದರೆ ನಮ್ಮನ್ನು ಪ್ರಪಂಚಕ್ಕೆ ಸೆಳೆಯುವ ಶಕ್ತಿ ಕಡಮೆಯಾಗಿ ಭಗವಂತನೆಡೆ ಸೆಳೆಯುವ ಶಕ್ತಿ ವೃದ್ಧಿಯಾಗುವುದು. ಮನುಷ್ಯ ಎಂದರೆ ಅವನೊಂದು ಅಭ್ಯಾಸದ ಕಂತೆ. ಅಭ್ಯಾಸ ಇಲ್ಲದೇ ಇದ್ದರೆ ಅವನು ಮನುಷ್ಯನೇ ಅಲ್ಲ. ಇದುವರೆಗೆ ಒಂದು ದುರಭ್ಯಾಸದ ಕಂತೆ ಆಗಿದ್ದೆವು. ಇನ್ನುಮೇಲೆ ನಾವೊಂದು ಸದಭ್ಯಾಸದ ಕಂತೆ ಆಗೋಣ. ಹೋರಾಟವೇ ಜೀವನ. ಸೋಲನ್ನು ಒಪ್ಪಿಕೊಳ್ಳುವುದು ಮರಣ. ಬದುಕಿರುವವರೆಗೆ ಮನುಷ್ಯ ಹೋರಾಡಲೇ ಬೇಕಾಗಿದೆ. ಒಪ್ಪೊತ್ತು ಕೂಳಿಗೆ, ಗೇಣುದ್ದ ಬಟ್ಟೆಗೆ ಎಷ್ಟೊಂದು ಹೋರಾಡುತ್ತೇವೆ ನಾವು. ಎಷ್ಟೊಂದು ಅವಮಾನಗಳನ್ನು ಸಹಿಸುತ್ತೇವೆ ನಾವು. ಎಷ್ಟು ಸಲ ಸೋತರೂ ಭಂಡರಾಗಿ ಪುನಃ ಅದಕ್ಕೆ ಪ್ರಯತ್ನಿಸುತ್ತೇವೆ. ನಾವು ಅದರಂತೆಯೇ ದೇವರೆಡೆಗೆ ಹೋದಾಗ ಎಷ್ಟೋ ವೇಳೆ ಜಾರಬಹುದು, ಹಿಂದೆ ಬೀಳಬಹುದು, ಆದರೂ ಸೋಲನ್ನು ಒಪ್ಪಿಕೊಳ್ಳದೆ ಪುನಃ ಪುನಃ ಪ್ರಯತ್ನಿಸುತ್ತಿದ್ದರೆ, ಅದೇ ಒಂದು ಉತ್ತಮ ಸಂಸ್ಕಾರವಾಗಿ ನಮ್ಮಲ್ಲಿ ನಿಂತು ಸದಾ ಭಗವಂತನ ಕಡೆಗೆ ಹೋಗುವಂತೆ ಮನಸ್ಸನ್ನು ಪ್ರಚೋದಿಸುವುದು. ದೇವರೆಡೆಗೆ ಹೋಗುವಾಗ ನಾವು ಎಷ್ಟು ಗೆದ್ದಿರುವೆವು ಎಂಬುದರ ಮೇಲೆ ನಿಂತಿಲ್ಲ. ನಾವೆಷ್ಟು ಹೋರಾಡುವೆವು ಎಂಬುದು ಮುಖ್ಯ. ನನಗೆ ಅವನು ಸಿಕ್ಕಿಲ್ಲ, ಮನಸ್ಸು ಇನ್ನೂ ಹದವಾಗಿಲ್ಲ ಅವನನ್ನು ಪಡೆಯುವುದಕ್ಕೆ. ಆದರೆ ಅದಕ್ಕಾಗಿ ಪ್ರಯತ್ನಿಸಿದ್ದೇನೆ. ಕಾಲವನ್ನು ವ್ಯರ್ಥ ಮಾಡಿಲ್ಲ, ಹೋರಾಡುತ್ತಿದ್ದೇನೆ, ಎಂಬುದೇ ಒಂದು ದೊಡ್ಡ ಸಮಾಧಾನ. ಯಾರು ಅವನಿಗಾಗಿ ಸತತ ಶ್ರಮಿಸುತ್ತಿರುವನೋ ಅವನು ಭಗವಂತನ ಕಡೆ ಕ್ರಮೇಣ ಹೋಗುತ್ತಿರುವನು. ಒಂದೇ ಸಲಕ್ಕೆ ಗುರಿ ಮುಟ್ಟಿಲ್ಲ. ಆದರೆ ಗುರಿಯೆಡೆಗೆ ನಾವು ಮಾಡಿದ ಪ್ರತಿಯೊಂದು ಪ್ರಯತ್ನವೂ ಒಂದು ಹೆಜ್ಜೆ ಮುಂದೆ ಒಯ್ದಿದೆ. ಹೇಗೆ ಶ‍್ರೀಕೃಷ್ಣ ಗೀತೆಯಲ್ಲಿ ಕರ್ಮಯೋಗ, ಜ್ಞಾನಯೋಗ, ಭಕ್ತಿಯೋಗ ಎಂಬ ಮಾರ್ಗವನ್ನು ಹೇಳುತ್ತಾನೆಯೋ ಹಾಗೆಯೇ ಅಭ್ಯಾಸವೂ ಒಂದು ಯೋಗವೇ. ನಮಗೆ ಇನ್ನು ಯಾವುದಿಲ್ಲದೇ ಹೋದರೂ ಬಿಡದೆ ಅಭ್ಯಾಸ ಮಾಡುವುದನ್ನು ಕಲಿಯೋಣ. ಸುಮ್ಮನೆ ಇಲ್ಲ ಎಂದು ಗೋಳಿಡುವುದಲ್ಲ, ಇಲ್ಲದೆ ಇದ್ದರೆ ಅದನ್ನು ಅಭ್ಯಾಸಬಲದಿಂದ ಪಡೆಯುವುದಕ್ಕೆ ಪ್ರಯತ್ನಿಸೋಣ.

\begin{shloka}
ಅಭ್ಯಾಸೇಽಪ್ಯಸಮರ್ಥೋಽಸಿ ಮತ್ಕರ್ಮಪರಮೋ ಭವ~।\\ಮದರ್ಥಮಪಿ ಕರ್ಮಾಣಿ ಕುರ್ವನ್ ಸಿದ್ಧಿಮವಾಪ್ಸ್ಯಸಿ \hfill॥ ೧೦~॥
\end{shloka}

\begin{artha}
ಅಭ್ಯಾಸದಲ್ಲಿಯೂ ನೀನು ಸಮರ್ಥನಾಗದಿದ್ದರೆ, ನನಗಾಗಿ ಮಾಡುವ ಕರ್ಮದಲ್ಲಿ ನಿರತನಾಗು. ನನಗಾಗಿ ಕರ್ಮಗಳನ್ನು ಮಾಡುತ್ತಿದ್ದರೂ ಸಿದ್ಧಿಯನ್ನು ಪಡೆಯುವೆ.
\end{artha}

ಕೆಲವು ವೇಳೆ ಮನಸ್ಸಿನಿಂದ ಅಭ್ಯಾಸ ಮಾಡುವುದಕ್ಕೆ ಆಗುವುದಿಲ್ಲ. ಬಹಳ ನೀರಸವಾಗುವುದು. ಮನಸ್ಸಿಗೆ ಕೆಲವು ವೇಳೆ ದೇವರ ವಿಷಯ ಕಷಾಯದಂತೆ ಕಹಿಯಾಗುವುದು. ಆಗ ನಾವು ಅದನ್ನು ಬಿಟ್ಟು ಬೇರೆ ಬಗೆಯ ಸಾಧನೆಗೆ ಕೈಹಾಕಬೇಕು. ದೇವರಿಗೆ ಸಂಬಂಧಪಟ್ಟ ಕರ್ಮಗಳನ್ನು ಮಾಡಬೇಕು. ಇದಕ್ಕೆ ಅಷ್ಟೊಂದು ಏಕಾಗ್ರತೆ ಬೇಕಾಗಿಲ್ಲ. ಅವನ ಮಕ್ಕಳ ಸೇವೆಯಲ್ಲಿ ನಿರತನಾಗಿರಬಹುದು. ಅದು ಯಾವುದಾದರೂ ಇನ್ನೊಬ್ಬರಿಗೆ ಮಾಡುವ ಸೇವೆಯಾಗಿರಬಹುದು. ಅದರ ಹಿಂದೆ, ಇದು ಭಗವಂತನ ಕೆಲಸ ಎಂಬ ದೃಷ್ಟಿಯಿಂದ ಮಾಡಬಹುದು ಅಥವಾ ನಮ್ಮ ಮನೆಯಲ್ಲೇ ಧ್ಯಾನ ಮಾಡುವುದಕ್ಕೆ ಆಗದೆ ಇದ್ದರೆ ದೇವರ ಪೂಜೆಗೆ ಸಂಬಂಧಪಟ್ಟ ಇತರ ಕೆಲಸಗಳನ್ನು ಮಾಡಬಹುದು. ದೇವರಿಗೆ ಸಂಬಂಧಪಟ್ಟ ಗ್ರಂಥಗಳನ್ನು ಓದಬಹುದು. ಅದನ್ನು ಪಾರಾಯಣ ಮಾಡಬಹುದು. ಅದನ್ನು ಕುರಿತು ಇನ್ನೊಬ್ಬರಿಗೆ ಹೇಳಬಹುದು. ಅಂತೂ ಯಾವ ಕೆಲಸವೇ ಆಗಲಿ, ಅದು ಲೌಕಿಕವಾಗಿರಲಿ, ಪಾರ ಮಾರ್ಥಿಕವೇ ಆಗಿರಲಿ, ಎಲ್ಲವನ್ನೂ ಭಗವಂತನಿಗೆ ಸಂಬಂಧಪಟ್ಟದ್ದು ಎಂಬ ಭಾವದಿಂದ ಮಾಡಿದರೆ, ಅದು ನಮ್ಮನ್ನು ಶುದ್ಧ ಮಾಡುವುದು. ಕ್ರಮೇಣ ಉತ್ತಮ ಸಂಸ್ಕಾರಗಳನ್ನು ಸಂಗ್ರಹಿಸುತ್ತಾ ಹೋಗುತ್ತೇವೆ. ಈಗ ನಮ್ಮ ಮನಸ್ಸು ಏಕಾಗ್ರವಾಗದೆ ಹೋದರೂ ಅವನಿಗೆ ಸಂಬಂಧಪಟ್ಟ ಕೆಲಸ ಮಾಡುತ್ತಾ ಹೋದರೆ ಅನಂತರ ಮನಸ್ಸು ಶುದ್ಧವಾಗಿ ಮನಸ್ಸು ಏಕಾಗ್ರವಾಗುವುದು. ಆಗ ನಾವು ಭಗವಂತನನ್ನು ಪಡೆಯಲು ಸಾಧ್ಯವಾಗುವುದು. ಅಂತೂ, ಮನಸ್ಸನ್ನು ಒಂದಿಲ್ಲದಿದ್ದರೆ ಮತ್ತೊಂದು ಮಾರ್ಗದಲ್ಲಾದರೂ ದೇವರೆಡೆಗೆ ಹೋಗುವಂತೆ ಮಾಡುತ್ತಿರಬೇಕು. ಏನೋ ನನ್ನ ಕೈಯಲ್ಲಿ ಆಗುವುದಿಲ್ಲ ಎಂದು ಕೈಕಟ್ಟಿ ಕುಳಿತುಕೊಳ್ಳಬಾರದು.

\begin{shloka}
ಅಥೈತದಪ್ಯಶಕ್ತೋಽಸಿ ಕರ್ತುಂ ಮದ್ಯೋಗಮಾಶ್ರಿತಃ~।\\ಸರ್ವಕರ್ಮಫಲತ್ಯಾಗಂ ತತಃ ಕುರು ಯತಾತ್ಮವಾನ್ \hfill॥ ೧೧~॥
\end{shloka}

\begin{artha}
ಇದನ್ನು ಕೂಡಾ ಮಾಡಲು ಶಕ್ತಿ ಇಲ್ಲದೇ ಇದ್ದರೆ, ಯತ್ನಪೂರ್ವಕ ಸಕಲ ಕರ್ಮಗಳ ಫಲವನ್ನು ತ್ಯಾಗಮಾಡು.
\end{artha}

ಭಗವಂತನಿಗಾಗಿ ನಾನು ಕರ್ಮಗಳನ್ನು ಮಾಡುತ್ತಿದ್ದೇನೆ ಎಂಬ ಭಾವದಿಂದ ಕೆಲಸ ಮಾಡಲು ಸಾಧ್ಯವಿಲ್ಲದೇ ಹೋದಲ್ಲಿ, ಯಾವ ಕರ್ಮವನ್ನು ಮಾಡುತ್ತಿರುವೆಯೋ, ಆ ಕರ್ಮದಿಂದ ಬರುವ ಫಲವನ್ನು ಭಗವಂತನಿಗೆ ಅರ್ಪಿಸು ಎನ್ನುವನು. ಕರ್ಮವನ್ನು ಯಾವಾಗ ಫಲಾಪೇಕ್ಷೆಯಿಂದ ಮಾಡುತ್ತೇವೆಯೋ ಆಗ ಅದು ನಮ್ಮನ್ನು ಕಟ್ಟಿಹಾಕುವುದು. ಯಾವಾಗ ಫಲಾಪೇಕ್ಷೆಯನ್ನು ಬಿಡುತ್ತೇವೆಯೋ ಆಗ ನಾವು ಏನು ಕರ್ಮ ಮಾಡಿದರೂ ಬಂಧನಕ್ಕೆ ಬೀಳುವುದಿಲ್ಲ. ಫಲಕ್ಕೆಲ್ಲ ದೇವನೇ ಒಡೆಯ, ನಾನಲ್ಲ ಎಂಬ ದೃಷ್ಟಿಯಿಂದ ಕರ್ಮ ಮಾಡಿದರೆ ಕರ್ಮಬಂಧನದಿಂದ ನಾವು ಪಾರಾಗುತ್ತೇವೆ. ಒಂದೇ ಸಲ ಫಲಗಳನ್ನೆಲ್ಲಾ ಸಮರ್ಪಿಸುವಷ್ಟು ನಮ್ಮ ಮನಸ್ಸು ಸಿದ್ಧವಾಗಿಲ್ಲ. ಆದರೂ ಅದಕ್ಕೆ ನಾವು ಪ್ರಯತ್ನಿಸಬೇಕು. ಒಂದು ವೇಳೆಯಲ್ಲ, ಹಲವು ವೇಳೆ ಈ ಅರ್ಪಿತ ಭಾವದಿಂದ ಕೆಲಸ ಮಾಡಲು ಕಲಿಯಬೇಕು. ಈಗ ನಮಗೆ ಸಾಧ್ಯವಿಲ್ಲ, ಅದಕ್ಕಾಗಿ ಬಿಟ್ಟುಬಿಡಿ ಎಂದು ಶ‍್ರೀಕೃಷ್ಣ ಹೇಳುವುದಿಲ್ಲ. ಯಾವುದು ಸಾಧ್ಯವಿಲ್ಲವೋ ಅದನ್ನು ಪ್ರಯತ್ನದಿಂದ ಸಾಧ್ಯ ಮಾಡಿಕೊಳ್ಳಿ ಎನ್ನುತ್ತಾನೆ. ಮೇಲಿನ ಆದರ್ಶ ಸಾಧ್ಯವಿಲ್ಲದೇ ಇದ್ದರೆ ಅದಕ್ಕಿಂತ ಕೆಳಗಿನ ಆದರ್ಶ ಹಿಡಿಯೋಣ. ಅಂತೂ ಆದರ್ಶವಿಲ್ಲದೆ ಇರಕೂಡದು. ಆದರ್ಶ ಎಂದು ಹೇಳಿದೊಡನೆ ಅದು ಯಾವಾಗಲೂ ನಮ್ಮಿಂದ ಮೇಲಕ್ಕೆ ಇರುವುದು. ಅದು ತತ್​ಕ್ಷಣವೇ ನಮ್ಮ ಕೈಗೆ ಎಟುಕುವುದಿಲ್ಲ. ಎಟುಕುವಂತಿದ್ದರೆ ಅದು ಆದರ್ಶವೇ ಆಗುತ್ತಿರಲಿಲ್ಲ. ಅದಕ್ಕಾಗಿ ಶ್ರಮ ಪಡಬೇಕು. ಕ್ರಮೇಣ ಅದು ನಮಗೆ ಸಿದ್ಧಿಸುವುದು.

ಪ್ರಯತ್ನ, ಅಭ್ಯಾಸ ಇವೇ ಆಧ್ಯಾತ್ಮಿಕ ಜೀವನದಲ್ಲಿ ನಾವು ಮುಂದುವರಿಯಬೇಕಾದರೆ ಕೊಡ ಬೇಕಾಗಿರುವ ಬೆಲೆ. ಪ್ರಯತ್ನಮಾಡದೆ ನಾವು ಸುಮ್ಮನೆ ಕುಳಿತುಕೊಂಡಿರುವುದು, ಯಾರೊ ನಮ್ಮ ಜುಟ್ಟನ್ನು ಹಿಡಿದು ಮೇಲೆತ್ತಿಬಿಡುವುದು ಸಾಧ್ಯವಿಲ್ಲ. ದೇವರೂ ಕೂಡಾ ಸೋಮಾರಿಗಳಿಗೆ ಸಹಾಯ ಮಾಡುವುದಿಲ್ಲ. ಯಾರು ಪ್ರಯತ್ನಮಾಡುತ್ತಿರುವನೋ ಅವನಿಗೆ ಭಗವಂತನ ಸಹಾಯ ಬರುವುದು. ನಾವು ಪ್ರಯತ್ನಮಾಡಿ ಅವನು ಕಡೆಗೆ ಹೋಗಲು ತಪ್ಪು ಹೆಜ್ಜೆ ಇಟ್ಟರೂ ಅವನು ನಮ್ಮನ್ನು ಮೇಲೆತ್ತುವನು.

\begin{shloka}
ಶ್ರೇಯೋ ಹಿ ಜ್ಞಾನಮಭ್ಯಾಸಾಜ್ಜ್ಞಾನಾದ್ಧ್ಯಾನಂ ವಿಶಿಷ್ಯತೇ~।\\ಧ್ಯಾನಾತ್ ಕರ್ಮಫಲತ್ಯಾಗಸ್ತ್ಯಾಗಾಚ್ಛಾಂತಿರನಂತರಮ್ \hfill॥ ೧೨~॥
\end{shloka}

\begin{artha}
ಅಭ್ಯಾಸಕ್ಕಿಂತ ಜ್ಞಾನ ಮೇಲು. ಜ್ಞಾನಕ್ಕಿಂತ ಧ್ಯಾನ ಮೇಲು. ಧ್ಯಾನಕ್ಕಿಂತ ಕರ್ಮಫಲತ್ಯಾಗ ಮೇಲು. ತ್ಯಾಗದಿಂದ ಕೂಡಲೆ ಶಾಂತಿ ಲಭಿಸುವುದು.
\end{artha}

ಇಲ್ಲಿ ಮೆಟ್ಟಲು ಮೆಟ್ಟಲಾಗಿ ಯಾವುದಕ್ಕಿಂತ ಯಾವುದು ಮೇಲು ಎಂಬುದನ್ನು ಶ‍್ರೀಕೃಷ್ಣ ವಿವರಿಸುತ್ತಾನೆ. ಮನುಷ್ಯ ವಿಕಾಸದ ಏಣಿಯಲ್ಲಿ ಮೆಟ್ಟಲುಗಳನ್ನು ಒಂದಾದ ಮೇಲೊಂದರಂತೆ ಹತ್ತಿಕೊಂಡು ಹೋಗುತ್ತಿರುವನು. ಒಂದು ಮತ್ತೊಂದಕ್ಕೆ ನಮ್ಮನ್ನು ಬಿಡುವುದು. ಯಾವುದನ್ನೂ ನಾವು ಅಲ್ಲಗಳೆಯುವುದಕ್ಕೆ ಆಗುವುದಿಲ್ಲ. ಒಂದಿಲ್ಲದೆ ಮತ್ತೊಂದು ಇರಲಾರದು. ನಾವು ಯಾವ ಸ್ಥಿತಿಯಲ್ಲಿರುವೆವೋ ಅದರಲ್ಲಿ ಚೆನ್ನಾಗಿ ಬೆಳೆದರೆ ಅದಕ್ಕೆ ಮುಂದಿನದು ತೆರೆದಿರುವುದು. ಮುಂದೆ ಹೋಗುವಾಗ ಹಿಂದಿನದಕ್ಕೆ ಕೃತಜ್ಞತೆಯನ್ನು ಅರ್ಪಿಸಿ ಹೋಗುವನು. ಅದನ್ನು ದೂರಿ\break ಹೋಗಕೂಡದು. ಒಬ್ಬ ಹುಡುಗ ಒಂದು ತರಗತಿಯಲ್ಲಿ ಓದುತ್ತಿರುವನು. ಆ ತರಗತಿಯ ಪಾಠವನ್ನೆಲ್ಲ ಮುಗಿಸಿದಮೇಲೆ ಮೇಲಿನ ತರಗತಿಗೆ ಹೋಗುವನು. ಯಾವಾಗಲೂ ಮುಂದು\-ವರಿಯುವುದು ಸ್ವಾಭಾವಿಕವಾಗಿರಬೇಕು. ಮೊದಲು ಮಾವಿನ ಮರದಲ್ಲಿ ಹೂ ಬಿಡುವುದನ್ನು ನೋಡುತ್ತೇವೆ. ಅದು ಉದುರಿಹೋಗಿ ಮಾವಿನ ಮಿಡಿಯಾಗುವುದು. ಸ್ವಲ್ಪ ಕಾಲದಮೇಲೆ ಅದೊಂದು ಬಲಿತ ಕಾಯಾಗುವುದು. ಅನಂತರ ಬಣ್ಣ ಬದಲಾಯಿಸಿ ಹಣ್ಣಾಗುವುದು. ಹಣ್ಣಾದ ಮೇಲೆ ಮರದಿಂದ ತಾನೇ ಉದುರುವುದು. ಇಲ್ಲಿ ಪ್ರತಿಯೊಂದು ಹಿಂದಿನ ಸ್ಥಿತಿಯೂ ಮುಂದಿನದಕ್ಕೆ ಒಯ್ಯುವುದು.

ಬರೀ ಅಭ್ಯಾಸಕ್ಕಿಂತ ಜ್ಞಾನ ಮೇಲು. ಇಲ್ಲಿ ಅಭ್ಯಾಸ ಎಂದರೆ ಬರೀ ಯಂತ್ರದಂತೆ ಕೆಲಸಮಾಡಿಕೊಂಡು ಹೋಗುವುದು. ಆಧ್ಯಾತ್ಮಿಕ ಜೀವನದಲ್ಲಿ ನಾವು ಪ್ರಾರಂಭದಲ್ಲಿ ಹಿರಿಯರು ಹೇಳಿದಂತೆ ಪೂಜೆ ಪಾರಾಯಣ ವ್ರತ ಹೋಮ ಮುಂತಾದುವುಗಳನ್ನು ಮಾಡುತ್ತಾ ಇರುತ್ತೇವೆ. ಭಗವಂತನನ್ನೆ ಮೊದಲು ಪೂಜಿಸುವುದು ಹೇಗೆ? ಇಲ್ಲಿ ಒಬ್ಬನಿಗೆ ಮಂತ್ರ ಗೊತ್ತಿರಬಹುದು. ಅರ್ಥ ಗೊತ್ತಿರುವುದಿಲ್ಲ. ಯಾಂತ್ರಿಕವಾಗಿ ಮಾಡುತ್ತಾ ಹೋಗುವನು. ಇದೂ ಕೂಡ ಏನೂ ಮಾಡದಿರುವುದಕ್ಕಿಂತ ಮೇಲು. ಒಂದು ಸ್ಥಳದಲ್ಲಿ, ಒಂದು ಕಾಲದಲ್ಲಿ, ಭಗವಂತನ ಒಂದು ನಿರ್ದಿಷ್ಟ ಆಕಾರವನ್ನು ಪೂಜಿಸುವುದು, ಅದಕ್ಕೆ ಸಂಬಂಧ ಪಟ್ಟ ಪೂಜಾದಿಗಳನ್ನು ಮಾಡುವುದು ನಮ್ಮ ಮನಸ್ಸಿನಮೇಲೆ ಒಳ್ಳೆಯ ಸಂಸ್ಕಾರಗಳನ್ನು ಬಿಡುವುದು. ನಾವು ಉಚ್ಚರಿಸುವ ಮಂತ್ರದ ಅರ್ಥ ನಮಗೆ ಗೊತ್ತಿಲ್ಲದೇ ಇರಬಹುದು; ಮಾಡುವ ಪೂಜೆಯ ರಹಸ್ಯ ನಮಗೆ ಗೊತ್ತಿಲ್ಲದೇ ಇರಬಹುದು; ಆದರೇನು? ಅವು ತಮ್ಮ ಪ್ರಭಾವವನ್ನು ನಮ್ಮ ಜೀವನದಮೇಲೆ ಬಿಟ್ಟೇಬಿಡುವುವು. ಒಂದು ಬೀಜದಲ್ಲಿ ಯಾವ ಗಿಡ ಹುದುಗಿದೆಯೋ ಅದು ನಮಗೆ ಗೊತ್ತಿಲ್ಲ. ಆದರೆ ಅದನ್ನು ನೆಲಕ್ಕೆ ಹಾಕಿ ಆರೈಕೆಮಾಡುತ್ತ ಬಂದರೆ ಅದೊಂದು ಸಸಿಯಾದಾಗ ಅದು ಎಂತಹ ಮರ ಎಂಬುದು ಗೊತ್ತಾಗುವುದು. ಅಂತೆಯೇ ಅರ್ಥ ಗೊತ್ತಿಲ್ಲದೆ ಮಾಡಿದ ಆಚಾರಗಳೂ ಕೂಡ. ಪ್ರತಿಯೊಂದು ಆಚಾರದ ಹಿಂದೆಯೂ ಒಂದು ಗಹನ ಭಾವನೆ ಇದೆ. ಆ ಗಹನ ಭಾವನೆ ತನಗೆ ತಾನೇ ಇರಲಾರದು. ಅದಕ್ಕೆ ಆಸರೆಯಾಗಿ ಯಾವುದಾದರೂ ಚಿಹ್ನೆ ಬೇಕಾಗಿದೆ. ಅಕ್ಕಿಯನ್ನೇ ನಾವು ಊಟ ಮಾಡಿದರೂ ಅದು ಅಕ್ಕಿಯಂತೆ ನಮಗೆ ಸಿಕ್ಕುವುದಿಲ್ಲ. ಅದು ನಮಗೆ ಮೊದಲು ಭತ್ತದ ರೂಪದಲ್ಲಿ ದೊರಕುವುದು. ಅನಾವಶ್ಯಕವಾದ ಹೊಟ್ಟು ಅದನ್ನು ರಕ್ಷಿಸುವುದು. ನಾವು ತಿನ್ನುವಾಗ ಹೊಟ್ಟನ್ನು ಕುಟ್ಟಿ ತೆಗೆದು ಅದರ ಹಿಂದಿರುವ ಅಕ್ಕಿಯನ್ನು ಮಾತ್ರ ತೆಗೆದುಕೊಳ್ಳುತ್ತೇವೆ. ಆದರೆ ನಾವು ಅಕ್ಕಿಯನ್ನು ಬೆಳೆಸಬೇಕಾದರೆ ಅಕ್ಕಿಯನ್ನು ಬಿತ್ತಿದರೆ ಪ್ರಯೋಜನವಿಲ್ಲ. ಹೊಟ್ಟಿರುವ ಭತ್ತವನ್ನು ಮಾತ್ರ ಬಿತ್ತಬೇಕು. ಆಗಲೇ ಅದು ಮೊಳೆಯಬಲ್ಲದು. ಅದರಂತೆಯೇ ಮೇಲಿನ ಹೊಟ್ಟಿಗೆ ನಾವು ತಿನ್ನುವ ದೃಷ್ಟಿಯಿಂದ ಬೆಲೆಯಿಲ್ಲದೇಹೋದರೂ, ಅದನ್ನು ರಕ್ಷಿಸಬೇಕಾದರೆ, ಮುಂದಿನವರಿಗೆ ಕೊಡಬೇಕಾದರೆ, ಭತ್ತದ ಹೊಟ್ಟು ಆವಶ್ಯಕ. ಈ ಸುಂದರ ಉಪಮಾನವನ್ನು ಶ‍್ರೀರಾಮಕೃಷ್ಣರು ಕೊಡುತ್ತಿದ್ದರು. ಅವರು ಬಾಹ್ಯಾಚಾರಗಳನ್ನು ದೂರುತ್ತಿರಲಿಲ್ಲ. ಅದರಲ್ಲಿ ಒಂದು ಗಹನ ಭಾವನೆ ಇದೆ. ಕ್ರಮೇಣ ನಾವು ಶ್ರದ್ಧೆಯಿಟ್ಟು ಮಾಡಿದರೆ ಅದರ ಹಿಂದೆ ಇರುವ ಭಾವನೆ ನಮ್ಮ ಜೀವನದಮೇಲೆ ಪರಿಣಾಮವನ್ನು ಬಿಡುವುದರಲ್ಲಿ ಸಂದೇಹವಿಲ್ಲ.

ಬರಿಯ ಅಭ್ಯಾಸಕ್ಕಿಂತ ಅದರ ಹಿಂದಿರುವ ಜ್ಞಾನವನ್ನು ತಿಳಿದುಕೊಳ್ಳುವುದು ಮೇಲು. ಸುಮ್ಮನೆ ಯಂತ್ರದಂತೆ ಮಾಡುವುದಕ್ಕಿಂತ ನಾವು ಮಾಡುವ ಕ್ರಿಯೆಗಳ ಮತ್ತು ಉಚ್ಚರಿಸುವ ಮಂತ್ರಗಳ ಅರ್ಥವನ್ನು ತಿಳಿದುಕೊಂಡರೆ ಅದರಿಂದ ಹೆಚ್ಚು ಉಪಯೋಗವಾಗುವುದು. ಇಲ್ಲಿ ಶ‍್ರೀಕೃಷ್ಣ ಜ್ಞಾನ ಬಂದರೆ ಅಭ್ಯಾಸವನ್ನು ಬಿಟ್ಟುಬಿಡು ಎಂದು ಹೇಳುವುದಿಲ್ಲ. ಸುಮ್ಮನೆ ಮಾಡುವುದಕ್ಕಿಂತ ಅರ್ಥ ತಿಳಿದುಕೊಂಡು ಮಾಡಿದರೆ ಹೆಚ್ಚು ಪ್ರಯೋಜನ. ಅರ್ಥವನ್ನು ತಿಳಿದುಕೊಳ್ಳುತ್ತ ಹೋದರೆ ನಮಗೆ ಅದರಮೇಲೆ ಸ್ವಾಧೀನ ಬರುವುದು. ಬರೀ ಅಂಧ ಮೂಢನಂಬಿಕೆಗೆ ವಶವಾಗದೆ ಹಿಂದಿರುವ ಗಹನ ಭಾವನೆಗಳನ್ನು ತಿಳಿದುಕೊಳ್ಳುತ್ತೇವೆ. ಯಾವಾಗ ತಿಳಿದುಕೊಳ್ಳುತ್ತೇವೆಯೋ ಆಗ ಹಿಂದಿನ ಅಭ್ಯಾಸವನ್ನೇ ಭಾವಪೂರ್ವಕ ಮಾಡಬಹುದು. ಅದನ್ನು ಮತ್ತೂ ಅಚ್ಚುಕಟ್ಟಾಗಿ ಮಾಡುವನು. ಮುಂದಿನದು ಬಂತು ಎಂದು ಹಿಂದಿನದನ್ನು ಮರೆಯುವುದಿಲ್ಲ. ಹಿಂದಿನ ಜ್ಞಾನಕ್ಕೆ ಮತ್ತಷ್ಟು ಸೇರಿಸುತ್ತಾ ಹೋಗುತ್ತೇವೆಯೇ ಹೊರತು ಒಂದನ್ನು ಬಿಟ್ಟು ಮತ್ತೊಂದನ್ನು ಹಿಡಿಯುವುದಿಲ್ಲ. ಒಬ್ಬ ಹುಡುಗ ಮುಂದಿನ ತರಗತಿಗೆ ಹೋದರೆ ಹಿಂದಿನ ಪುಸ್ತಕಗಳನ್ನು ಓದದೆ ಇರಬಹುದು; ಆದರೆ ಅವನೇನು ಕಲಿತಿದ್ದನೋ ಅದು ಯಾವಾಗಲೂ ಅವನಲ್ಲಿರುವುದು. ಅದಕ್ಕೆ ಹೊಸದಾಗಿ ಸೇರಿಸುವನು. ಜ್ಞಾನವೆಂದರೆ ಒಂದನ್ನು ಬಿಟ್ಟು ಮತ್ತೊಂದನ್ನು ಸ್ವೀಕರಿಸುವುದಲ್ಲ,\break ಒಂದಕ್ಕೆ ಮತ್ತೊಂದನ್ನು ಸೇರಿಸುವುದು.

ಅನಂತರ ಬರಿಯ ಜ್ಞಾನಕ್ಕಿಂತ ಧ್ಯಾನ ಮೇಲು ಎನ್ನುತ್ತಾನೆ. ಒಂದು ವಸ್ತುವನ್ನು ಬೌದ್ಧಿಕವಾಗಿ ತಿಳಿದುಕೊಂಡಿರುವುದಕ್ಕಿಂತ, ಅದರಲ್ಲಿರುವ ಸಾರವಸ್ತುವನ್ನು ಧ್ಯಾನಮಾಡುವುದು ಮೇಲು. ದೇವರು ಎಂದರೆ ಸರ್ವಾಂತರ್ಯಾಮಿ, ಅನಂತ ಕಲ್ಯಾಣಗುಣಗಳೆಲ್ಲ ಅವನಲ್ಲಿವೆ; ಅವನು ಸರ್ವಜ್ಞ, ಸರ್ವಶಕ್ತ ಎಂದು ನಾವು ತಿಳಿದುಕೊಂಡಿರಬಹುದು. ಆದರೆ ಅಧ್ಯಾತ್ಮ ಜೀವನದಲ್ಲಿ ಬರೀ ತಿಳಿವಳಿಕೆಯೇ ನಮ್ಮನ್ನು ಉದ್ಧಾರಮಾಡಲಾರದು. ಆ ತಿಳಿವಳಿಕೆಯಿಂದ ನಾವು ಪ್ರಯೋಜನ ಪಡೆಯಬೇಕಾದರೆ ಅದರೊಂದಿಗೆ ಸಂಬಂಧ ಕಲ್ಪಿಸಿಕೊಳ್ಳಬೇಕು. ಸಂಬಂಧವನ್ನು ಕಲ್ಪಿಸಿಕೊಳ್ಳುವುದೇ ಧ್ಯಾನ. ನಮ್ಮ ಮನೆ ಮುಂದೆ ವಿದ್ಯುಚ್ಛಕ್ತಿಯ ತಂತಿ ಹೋಗುತ್ತಿದೆ. ಮನೆಯಲ್ಲಿ ವಿದ್ಯುಚ್ಛಕ್ತಿ ಇಲ್ಲ. ನನಗೆ ವಿದ್ಯುಚ್ಛಕ್ತಿ ಎಂದರೇನು? ಅದರ ಗುಣಗಳೇನು? ಅದರಿಂದ ಏನೇನು ಮಾಡಬಹುದು ಎಂಬುದೆಲ್ಲ ಗೊತ್ತಿದೆ. ಆದರೆ ಬರೀ ಜ್ಞಾನದಿಂದ ಬಂದದ್ದೇನು? ಆ ಜ್ಞಾನದಿಂದ ಪ್ರಯೋಜನ ನಾವು ಪಡೆಯಲೇಬೇಕಾದರೆ ಅದನ್ನು ಅನುಷ್ಠಾನಕ್ಕೆ ತರಬೇಕು. ಹೊರಗೆ ಹೋಗುವ ತಂತಿಯಿಂದ ನಾವು ಶಕ್ತಿಯನ್ನು ಮತ್ತೊಂದು ತಂತಿಯ ಸಹಾಯದಿಂದ ಮನೆಗೆ ತಂದರೆ ಆಗ ವಿದ್ಯುಚ್ಛಕ್ತಿಯ ಪ್ರಯೋಜನ ಪಡೆಯುತ್ತೇವೆ. ಅದರಂತೆಯೆ ಧ್ಯಾನ. ಭಗವಂತನನ್ನು ಕುರಿತು ಅವನ ಕಡೆಗೆ ಇಡೀ ಮನಸ್ಸನ್ನು ಹರಿಸಿ, ಅವನನ್ನು ಚಿಂತಿಸುವುದು. ಆಗ ಭಗವಂತನಲ್ಲಿರುವ ಶಕ್ತಿ, ಪವಿತ್ರತೆ, ಜ್ಞಾನ ಇವುಗಳೆಲ್ಲ ನಮ್ಮ ತಂತಿಯ ಯೋಗ್ಯಾತಾನುಸಾರ ನಮಗೆ ಇಳಿದುಬರುತ್ತವೆ. ನಾವು ಏನೇನನ್ನು ತಿಳಿದುಕೊಂಡಿರುವೆವೋ ಅದೆಲ್ಲ ನಮ್ಮದಲ್ಲ. ನಾವು ಎಷ್ಟನ್ನು ಅರಗಿಸಿಕೊಂಡಿರುವೆವೋ ಅದು ಮಾತ್ರ ನಮ್ಮದು. ಧ್ಯಾನ ಎಂದರೆ ದೇವರಿಗೆ ಸಂಬಂಧಪಟ್ಟ ಭಾವನೆಗಳನ್ನು ಜೀವನದಲ್ಲಿ ಅರಗಿಸಿಕೊಳ್ಳುವುದು. ಪಾಂಡಿತ್ಯ ನಮ್ಮನ್ನು ಧ್ಯಾನಕ್ಕೆ ಒಯ್ಯಬೇಕು. ಬರೀ ಪಾಂಡಿತ್ಯದಲ್ಲೇ ಮುಳುಗಿ ದ್ದರೆ ಸಾಲದು. ಬೇಕಾದಷ್ಟು ತಿಳಿದಿರಬಹುದು. ಆದರೆ ಜೀವನದಲ್ಲಿ ನಮ್ಮನ್ನು ಉದ್ಧಾರ ಮಾಡುವುದು ನಾವು ತಿಳಿದುಕೊಂಡಿರುವುದಲ್ಲ. ನಾವು ಅರಗಿಸಿಕೊಂಡಿರುವುದು, ಅದಷ್ಟೇ ನಮ್ಮದು. ಮಿಕ್ಕದ್ದೆಲ್ಲ ಅಲಂಕಾರಕ್ಕೆ, ಪ್ರಯೋಜನಕ್ಕಲ್ಲ.

ಧ್ಯಾನಕ್ಕಿಂತ ಮೇಲಿನ ಮೆಟ್ಟಲೇ ಕರ್ಮದಿಂದ ಬರುವ ಫಲಗಳನ್ನೆಲ್ಲಾ ಭಗವಂತನಿಗೆ ಅರ್ಪಿಸುವುದು. ಅದು ಎಂಥ ಫಲವಾಗಲಿ, ತಪಸ್ಸಿನಿಂದ ಬರುವ ಫಲವಾಗಲಿ, ಜ್ಞಾನದಿಂದ ಬರುವ ಫಲವಾಗಲಿ, ಸಮಾಜದಲ್ಲಿ ಯಾವುದಾದರೂ ಒಳ್ಳೆಯ ಕೆಲಸವನ್ನು ಮಾಡಿದುದರಿಂದ ಬಂದ ಫಲವಾಗಲಿ, ಎಲ್ಲವನ್ನೂ ಭಗವಂತನಿಗೆ ಅರ್ಪಣೆಮಾಡಬೇಕು. ಇದೇ ಎಲ್ಲಕ್ಕಿಂತ ಶ್ರೇಷ್ಠ. ಜೀವನದಲ್ಲಿ ನಾವು ಏನನ್ನು ಪಡೆದರೂ ಅದೆಲ್ಲಾ ವಾಸುದೇವನಿಗೆ ಅರ್ಪಿತವಾಗಬೇಕು. ನನಗೆ ಯಾವುದೂ ಇಲ್ಲ, ನನ್ನಲ್ಲಿರುವುದೆಲ್ಲ ಭಗವಂತನದು. ಒಂದು ಹೊಲದಲ್ಲಿ ಬೆಳೆದುದೆಲ್ಲ ಅದರ ಒಡೆಯನಿಗೆ ಸೇರಿದ್ದು. ಈ ದೇಹ, ಮನಸ್ಸು, ಬುದ್ಧಿ, ಇಂದ್ರಿಯ ಅಹಂಕಾರಗಳೆಲ್ಲ ಅವನ ಹೊಲ. ಇಲ್ಲಿ ಅವನು ಬಗೆಬಗೆಯ ಬೆಳೆಗಳನ್ನು ಬೆಳೆಸುತ್ತಾನೆ. ಏಕೆಂದರೆ ಅದನ್ನು ಇತರರಿಗೆ ಹಂಚುವುದಕ್ಕೆ. ಈ ಮರ್ಮವನ್ನು ತಿಳಿದವನೇ ಶ್ರೇಷ್ಠ. ನನಗೆ ದೊಡ್ಡ ವಿದ್ವತ್ ಇರಬಹುದು. ಅದರಿಂದ ಕೀರ್ತಿಗಳಿಸುವವನು ಸಾಮಾನ್ಯ ಮನುಷ್ಯ. ಭಕ್ತನಾದರೋ ಕೀರ್ತಿಯ ಕಡೆ ಮನಸ್ಸನ್ನು ಹಾಕುವುದಿಲ್ಲ. ದೇವರು ಈ ಪಾಂಡಿತ್ಯವನ್ನು ಇಲ್ಲಿ ಇಟ್ಟಿರುವನು, ಅದನ್ನು ಎಲ್ಲರಿಗೂ ಕೊಡುವಾಗ ಪ್ರತಿಫಲಾಪೇಕ್ಷೆಯನ್ನು ಗಮನಿಸದೆ ಕೊಡುವ. ಕೆರೆಯ ನೀರನ್ನು ಕೆರೆಗೆ ಚೆಲ್ಲಿ ಧನ್ಯರಾಗಬೇಕು ನಾವು. ದೇವರು ನಮಗೆ ಯಾವ ವಿದ್ಯೆಯನ್ನಾದರೂ ಕೊಟ್ಟಿರಲಿ, ಯಾವ ಕಲೆಯನ್ನಾದರೂ ಕೊಡಲಿ, ಇದು ಭಗವದರ್ಪಣೆ ಆಗಬೇಕು. ಅವನು ಸುಪ್ರೀತನಾಗುವುದು ಆಗ. ನಮ್ಮ ಅಲ್ಪವಿದ್ಯೆಯಿಂದ ನಮ್ಮನ್ನು ಮೆರೆಸಿಕೊಳ್ಳುವುದಲ್ಲ, ಭಗವಂತನನ್ನು ಮೆರೆಸಬೇಕು. ಆಗ ನಾನು ಕೃತಾರ್ಥನಾಗುತ್ತೇನೆ. ಮೊದಲಲ್ಲಿ ಇನ್ನೊಬ್ಬನನ್ನು ಸೋಲಿಸಬೇಕು, ಕೀರ್ತಿಗಳಿಸಬೇಕು, ನನ್ನಷ್ಟು ದೊಡ್ಡ ವಿದ್ಯಾವಂತನೇ ಇಲ್ಲ ಎನಿಸಿಕೊಳ್ಳಬೇಕು ಎಂಬ ಆಸೆಗಳು ನಮ್ಮನ್ನು ಆಯಾ ವೃತ್ತಿಯಲ್ಲಿ ಮೇಲಕ್ಕೆ ಒಯ್ಯುವುವು. ಅದರಲ್ಲಿ ಶಿಖರವನ್ನು ಮುಟ್ಟ ಬೇಕಾದರೆ, ಪರಾಕಾಷ್ಠೆಗೆ ಏರಬೇಕಾದರೆ, ಅಹಂಕಾರವನ್ನು ಮರೆಯಬೇಕು, ಇದು ಭಗವಂತನ ಸೇವೆ ಎಂಬ ದೃಷ್ಟಿಯಿಂದ ಮಾಡಬೇಕು. ಆಗ ಮಾತ್ರ ಸಾಧ್ಯ. ಎಲ್ಲಿಯವರೆಗೆ ನಾನು ಕೀರ್ತಿಗೆ ಲಾಭಕ್ಕೆ ಹಣಕ್ಕೆ ಕಟ್ಟಿಹಾಕಿಕೊಂಡಿರುವೆನೋ ಅಲ್ಲಿಯವರೆಗೆ ನಾನು ಭಗವಂತನ ಕಡೆಗೆ ಹೋಗಲಾರೆ. ಅವನ ಕಡೆಗೆ ಹಾರಲು ಹೊರಟರೆ ನನ್ನದೆಂಬ ಅಹಂಕಾರವನ್ನು ಮರೆಯಬೇಕಾಗಿದೆ. ಅಕ್ಬರನ ಆಸ್ಥಾನದಲ್ಲಿ ತಾನ್​ಸೇನನೆಂಬ ಸಂಗೀತ\-ಗಾರನಿದ್ದ. ಅವನು ಬಹಳ ಪ್ರಖ್ಯಾತ ಸಂಗೀತ\-ಗಾರನಾಗಿದ್ದ. ಅನೇಕ ಸಮಯಗಳಲ್ಲಿ ಆಸ್ಥಾನದಲ್ಲಿ ಹಾಡುತ್ತಿದ್ದಾಗ ಅದನ್ನು ಕೇಳಿ ಅಕ್ಬರ್ ಮಾರುಹೋಗುತ್ತಿದ್ದ. ಆತ ಮಧ್ಯೆ ಮಧ್ಯೆ\break ಆಸ್ಥಾನದಿಂದ ತಪ್ಪಿಸಿಕೊಂಡು ಒಬ್ಬನೇ ಹೋಗುತ್ತಿದ್ದ. ಅವನು ಎಲ್ಲಿಗೆ ಹೋಗತ್ತಾನೆ, ಏನು ಮಾಡುತ್ತಾನೆ ಎಂಬುದನ್ನು ನೋಡಲು ಅಕ್ಬರ್ ವೇಷ ಬದಲಾಯಿಸಿಕೊಂಡು ತಾನ್​ಸೇನ್ ಮತ್ತೊಮ್ಮೆ ಹೋಗುತ್ತಿದ್ದಾಗ ಹಿಂಬಾಲಿಸಿದ. ತಾನ್​ಸೇನ್ ಆಸ್ಥಾನವನ್ನು ಬಿಟ್ಟು ಹಲವು ಮೈಲಿ ದೂರ ಹೋಗಿ, ಒಂದು ಕಾಡನ್ನು ಪ್ರವೇಶಿಸಿದ. ಅಲ್ಲೊಂದು ಬಂಡೆಯಮೇಲೆ ಕುಳಿತ. ನಿರ್ಜನ ಪ್ರದೇಶ. ಯಾರೂ ಇಲ್ಲ ಎಂದು ಭಾವಿಸಿದ್ದ. ಅಲ್ಲಿ ಕುಳಿತು ಹಾಡಲು ಪ್ರಾರಂಭಿಸಿದ. ಎಲ್ಲವನ್ನೂ ಮರೆತು ಗಾನದಲ್ಲಿ ತಲ್ಲೀನನಾಗಿ ಹೋದ. ಅಕ್ಬರ್ ಅವನಿಗರಿಯದಂತೆ ಹಿಂದೆ ಕುಳಿತುಕೊಂಡು ಕೇಳಿದ. “ಅಬ್ಬ, ಏನು ಗಾನಸೌಂದರ್ಯ! ಒಂದು ದಿನವೂ ಅರಮನೆಯಲ್ಲಿ ಹೀಗೆ ಹಾಡಿರಲಿಲ್ಲ,” ಎಂದು ಭಾವಿಸಿದ. ಸ್ವಲ್ಪ ಕಾಲವಾದಮೇಲೆ ತಾನ್​ಸೇನ್ ಹೊರಡಲುದ್ಯುಕ್ತನಾದಾಗ ಅಕ್ಬರ್ ಕಾಣಿಸಿಕೊಂಡು ಕೇಳಿದ: “ತಾನ್​ಸೇನ್, ನಾನು ನಿನ್ನ ರಾಜ. ಸಂಬಳ ಕೊಟ್ಟು ನಿನ್ನನ್ನಿಟ್ಟುಕೊಂಡಿದ್ದೇನೆ. ಆದರೆ ನೀನು ಒಮ್ಮೆಯೂ ಇಷ್ಟು ಚೆನ್ನಾಗಿ ಆಸ್ಥಾನದಲ್ಲಿ ಹಾಡಲಿಲ್ಲವಲ್ಲ!” ಅದಕ್ಕೆ ತಾನ್​ಸೇನ್,\break “ನಾನಲ್ಲಿ ಈ ಜಗದ ರಾಜರೆದುರಿಗೆ ಹಾಡುತ್ತೇನೆ. ಅವನು ಕೊಡುವ ಸಂಬಳಕ್ಕೆ ಹಾಡುತ್ತೇನೆ. ಇಲ್ಲಿ ರಾಜಾಧಿರಾಜನಾದ ಭಗವಂತನಿಗೆ ಹಾಡುತ್ತೇನೆ. ಅವನು ಕೊಟ್ಟಿದ್ದನ್ನು ಅವನಿಗೆ ಅರ್ಪಿಸುತ್ತಿದ್ದೇನೆ. ಹೀಗೆ ಬೇರೆ ಕಡೆ ಹಾಡಲು ಸಾಧ್ಯವೆ?” ಎಂದ.

ಯಾವಾಗ ದೇವರು ನಮಗೆ ಕೊಟ್ಟದ್ದನ್ನು ಅವನಿಗೆ ಹಿಂತಿರುಗಿ ಕೊಡುತ್ತೇವೆಯೋ ಆಗಲೇ ನಮಗೆ ಪರಮಶಾಂತಿ ಲಭಿಸಬಲ್ಲದು. ಎಲ್ಲಿಯವರೆಗೆ ನಾನು ಅದನ್ನು ನನ್ನಲ್ಲಿಯೇ ಇಟ್ಟುಕೊಂಡಿರು ವೆನೋ ಅಲ್ಲಿಯವರೆಗೆ ಅದು ಸಾಲ ಮಾಡಿದಂತೆ. ಅದನ್ನು ತೀರಿಸುವುದು ಅದನ್ನು ಭಗವಂತನಿಗೆ ಹಿಂತಿರುಗಿ ಕೊಟ್ಟಾಗಲೇ. ಹಾಗೆ ಹಿಂದಿರುಗಿಸಿ ಕೊಡುವಾಗ ಕೆಲವರು ಭಗವಂತನಿಗೆ ಹಿಂತಿರುಗಿ ಕೊಡುತ್ತಾರೆ ಮತ್ತೆ ಕೆಲವರು ಅವನ ಮಕ್ಕಳ ಮೂಲಕ ಹಿಂತಿರುಗಿಸುತ್ತಾರೆ; ಆಗಲೇ ನಮ್ಮ ಜವಾಬ್ದಾರಿ ತೀರುವುದು. ಇದು ಒಬ್ಬ ಯಾವ ವಿದ್ಯೆಯಲ್ಲಾದರೂ ಸರಿಯೆ, ಪರಿಣತಿಯನ್ನು ಮುಟ್ಟಿದರೆ ಎಲ್ಲರಿಗೂ ಅನ್ವಯಿಸುವುದು. ದೊಡ್ಡ ಸಂಗೀತಗಾರ ತಾನು ಸಾಯುವುದಕ್ಕೆ ಮುಂಚೆ ತನಗೆ ಗೊತ್ತಿರುವುದನ್ನೆಲ್ಲ ಮತ್ತೊಬ್ಬನಿಗೆ ಕಲಿಸಿ ಕಣ್ಣುಮುಚ್ಚಿಕೊಳ್ಳಲು ಇಚ್ಛಿಸುವನು. ಚಿತ್ರಕಾರ ಯೋಗ್ಯ ಶಿಷ್ಯನನ್ನು ತರಬೇತು ಮಾಡಿ ಎಲ್ಲವನ್ನೂ ಅವನಿಗೆ ಕಲಿಸಿಕೊಟ್ಟು ತಾನು ಕಣ್ಮರೆಯಾಗುವನು. ಹಾಗೆಯೇ ಅಧ್ಯಾತ್ಮ ವಿದ್ಯೆಯಲ್ಲಿಯೂ. ಶ‍್ರೀರಾಮಕೃಷ್ಣರು ಎಲ್ಲಾ ಸಾಧನೆಯನ್ನು ಮಾಡಿ, ಭಗವಂತನನ್ನು ಎಲ್ಲಾ ದೃಷ್ಟಿಕೋನದ ಮೂಲಕ ನೋಡಿ, ಎಲ್ಲಾ ಭಾವಗಳ ಮೂಲಕ ಅವನನ್ನು ಪ್ರೀತಿಸಿ, ಎಲ್ಲಾ ಧರ್ಮಗಳನ್ನು ಅನುಷ್ಠಾನ ಮಾಡಿ, ಪರಮ ಸಿದ್ಧಿಯನ್ನು ಮುಟ್ಟಿದಾಗ ತಾಯಿಗೆ ಪ್ರಾರ್ಥನೆ ಸಲ್ಲಿಸುತ್ತಿದ್ದರು, “ತಾಯಿ, ನನ್ನ ಶಿಷ್ಯರನ್ನು ಬೇಗ ಇತ್ತ ಕಳಿಸು” ಎಂದು. ಯಾವಾಗ ಶಿಷ್ಯರು ಬರಲು ಉಪಕ್ರಮಿಸಿದರೋ, ಅವರಲ್ಲಿ ಮುಂದೆ ವಿವೇಕಾನಂದರಾದ ನರೇಂದ್ರನಾಥ ಬಂದನೋ, ಅವನನ್ನು ಚೆನ್ನಾಗಿ ಪರೀಕ್ಷಿಸಿ, ಅವನನ್ನು ಚೆನ್ನಾಗಿ ತರಬೇತು ಮಾಡಿ, ಒಂದು ದಿನ ತಾವು ಯೋಗಾವಸ್ಥೆಯಲ್ಲಿದ್ದಾಗ ಎಲ್ಲಾ ಅನುಭವಗಳನ್ನೂ ನರೇಂದ್ರನಿಗೆ ದಾನ ಮಾಡಿ– “ಇಂದಿನಿಂದ ನಾನು ಫಕೀರ, ನನ್ನದನ್ನೆಲ್ಲಾ ನಿನಗೆ ಕೊಟ್ಟೆ. ನನ್ನ ಜವಾಬ್ದಾರಿ ತೀರಿತು. ನಾನು ಶಾಂತಿಯಿಂದ ಕಣ್ಮುಚ್ಚಿಕೊಳ್ಳುತ್ತೇನೆ” ಎಂದರು. ಈ ಮಹಾತ್ಯಾಗವೇ ನಮ್ಮ ಸಾಕ್ಷಾತ್ಕಾರದ ಕಲಶ. ಯಾರು ಇದನ್ನು ಮಾಡಿರುವರೊ ಅವರಿಗೆ ಶಾಂತಿ. ಸುಮ್ಮನೆ ಅದನ್ನು ತಮ್ಮ ಕೀರ್ತಿ ಗೌರವ ಲಾಭಕ್ಕೆ ಮಾರುವವನಿಗಲ್ಲ.

\begin{shloka}
ಅದ್ವೇಷ್ಟಾ ಸರ್ವಭೂತಾನಾಂ ಮೈತ್ರಃ ಕರುಣ ಏವ ಚ~।\\ನಿರ್ಮಮೋ ನಿರಹಂಕಾರಃ ಸಮದುಃಖಸುಖಃ ಕ್ಷಮೀ \hfill॥ ೧೩~॥
\end{shloka}

\begin{shloka}
ಸಂತುಷ್ಟಃ ಸತತಂ ಯೋಗೀ ಯತಾತ್ಮಾ ದೃಢನಿಶ್ಚಯಃ~।\\ಮಯ್ಯರ್ಪಿತಮನೋಬುದ್ಧಿರ್ಯೋಮದ್ಭಕ್ತಃ ಸ ಮೇ ಪ್ರಿಯಃ \hfill॥ ೧೪~॥
\end{shloka}

\begin{artha}
ಯಾವನು ಸರ್ವ ಪ್ರಾಣಿಗಳನ್ನೂ ದ್ವೇಷಿಸುವುದಿಲ್ಲವೊ, ಮಿತ್ರಭಾವದಿಂದ ಕೂಡಿರುವನೋ, ಕರುಣೆಯುಳ್ಳವನೋ, ಅಹಂಕಾರ ಮಮಕಾರಗಳಿಲ್ಲದವನೋ, ಸುಖ ದುಃಖಗಳಲ್ಲಿ ಸಮನಾಗಿರು\-ವನೋ, ಕ್ಷಮಾಶೀಲನೊ, ಯಾವಾಗಲೂ ಸಂತುಷ್ಟನೊ, ಯೋಗಿಯೊ, ಯತಾತ್ಮನೋ, ದೃಢ\-ನಿಶ್ಚಯನೋ, ನನ್ನಲ್ಲಿ ಅರ್ಪಿಸಲ್ಪಟ್ಟ ಮನೋಬುದ್ಧಿಯುಳ್ಳವನೋ, ನನ್ನ ಭಕ್ತನೋ ಅವನು ನನಗೆ ಪ್ರಿಯನು.
\end{artha}

ಒಬ್ಬ ನಿರ್ಗುಣನನ್ನು ಉಪಾಸನೆ ಮಾಡಲಿ, ಅಥವಾ ಸಗುಣವನ್ನು ಉಪಾಸನೆ\break ಮಾಡಲಿ ದೇವರು ಮೆಚ್ಚುವುದು ಮುಂದೆ ಬರುವ ಗುಣಗಳುಳ್ಳವನನ್ನು. ಸುಮ್ಮನೆ ನಾನು ನಿರ್ಗುಣೋಪಾಸಕ ಎಂದರೆ ಅವನನ್ನು ಕಳಪೆ ಎಂದೂ ಭಾವಿಸುವುದಿಲ್ಲ. ಅವರಿಬ್ಬರಲ್ಲೂ ಒಳ್ಳೆಯ ಗುಣಗಳಿದ್ದರೆ ಅವರು ಅವನ ಪ್ರೀತಿಗೆ ಪಾತ್ರರು.

ಭಕ್ತ ಯಾರನ್ನೂ ದ್ವೇಷಿಸುವುದಿಲ್ಲ. ಭಕ್ತನಿಗೆ ದ್ವೇಷಿಗಳು ಯಾರೂ ಇಲ್ಲ. ತನ್ನನ್ನು ದ್ವೇಷಿಸುವವನನ್ನೂ ಕೂಡಾ ಅವನು ಪ್ರೀತಿಸುವನು. ಏಕೆಂದರೆ ದ್ವೇಷ ನಮ್ಮ ಮನಸ್ಸಿನ ಶಕ್ತಿಯನ್ನು ವ್ಯಯ ಮಾಡುವುದು. ನಾವು ಯಾವ ವ್ಯಕ್ತಿಯನ್ನು ದ್ವೇಷಿಸುತ್ತೇವೆಯೋ ಅವನಿಗಿಂತ ಹೆಚ್ಚಾಗಿ ನಮಗೇ ಅದರಿಂದ ಕೇಡು. ಭಕ್ತನಿಗೆ ಯಾರಾದರೂ ಕೇಡು ಮಾಡಿದರೂ ಅವನು ಅವರನ್ನು ದ್ವೇಷಿಸುವುದಿಲ್ಲ. ದೇವರೇ ಇವನ ಮೂಲಕ ನನ್ನನ್ನು ಶಿಕ್ಷಿಸಿದ ಎಂದು ಭಾವಿಸುವನು.

ಅವನು ದ್ವೇಷಿಸದೇ ಇದ್ದರೆ ಉದಾಸೀನನಾಗಿರಬಹುದಲ್ಲ ಎಂದು ನಾವು ಭಾವಿಸಬಹುದು. ಉದಾಸೀನತೆಯಲ್ಲಿ ಯಾವ ಒಂದು ಒಳ್ಳೆಯ ಭಾವವೂ ಇಲ್ಲ. ಅದು ದ್ವೇಷಕ್ಕಿಂತ ಮೇಲು ಅಷ್ಟೆ. ಆದರೆ ಅದಕ್ಕಿಂತ ಶ್ರೇಷ್ಠವಾಗಿರುವುದೇ ಮೈತ್ರಿ ಭಾವ. ಯಾರು ಭಗವಂತನನ್ನು ಪ್ರೀತಿಸುವನೋ ಅವನು ಪ್ರಪಂಚದಲ್ಲಿ ಎಲ್ಲರನ್ನೂ ಪ್ರೀತಿಸುವನು. ಏಕೆಂದರೆ ಅವನಿಗೆ ಎಲ್ಲರೂ ಸ್ನೇಹಿತರಾಗುವರು. ದೇವರನ್ನು ಪ್ರೀತಿಸುವುದು, ಇತರರನ್ನು ದ್ವೇಷಿಸುವುದು ಒಟ್ಟಿಗೆ ಹೋಗುವುದಕ್ಕೆ ಆಗುವುದಿಲ್ಲ. ಭಕ್ತ ಪರಿಚಿತರ ಮೇಲೆ ಮಾತ್ರ ಸ್ನೇಹವನ್ನು ತೋರುವುದಿಲ್ಲ. ಎಲ್ಲರ ಮೇಲೆಯೂ ಸ್ನೇಹವನ್ನು ತೋರುತ್ತಾನೆ. ದ್ವೇಷದ ಭಾವನೆ ಅವನಲ್ಲಿ ಇಲ್ಲವೇ ಇಲ್ಲ.

\newpage

ಅವನು ಇತರರ ಮೇಲೆ ಕರುಣೆಯುಳ್ಳವನು. ಭಗವಂತನ ಭಕ್ತನಾದರೆ ಅವನ ಹೃದಯ ಕಲ್ಲಾಗುವುದಿಲ್ಲ. ಅವನು ಎಲ್ಲರ ಕಷ್ಟ ಮತ್ತು ದುಃಖಗಳಲ್ಲಿಯೂ ಭಾಗಿಯಾಗುವನು. ಅವರೊಂದಿಗೆ ಸಹಾನುಭೂತಿಯನ್ನು ತೋರುತ್ತಾನೆ. ಅವರೊಂದಿಗೆ ಕಣ್ಣಿರನ್ನು ಸುರಿಸುತ್ತಾನೆ. ಅವರನ್ನು ದುಃಖದ ಪ್ರವಾಹದಿಂದ ಮೇಲೆತ್ತುತ್ತಾನೆ.

ಭಕ್ತರು ಅಹಂಕಾರ ಪಡುವುದಿಲ್ಲ. ದೇವರನ್ನು ಮರೆತಾಗ ನಾವು ನಮ್ಮಲ್ಲಿರುವ ಅಲ್ಪಕ್ಕೆ ಅಹಂಕಾರಪಡುತ್ತೇವೆ. ಆದರೆ ಯಾವಾಗ ಒಬ್ಬ ದೇವರನ್ನು ಸದಾ ಚಿಂತಿಸುತ್ತಿರುವನೋ ಅವನು ದೇವರೇ ತನ್ನಲ್ಲಿರುವುದನ್ನೆಲ್ಲಾ ಕೊಟ್ಟಿರುವನು, ತನ್ನ ಕೆಲಸ ಮಾಡಿಸಿಕೊಳ್ಳುವುದಕ್ಕೆ ಎಂದು ಭಾವಿಸು\-ವನು. ಅವನಲ್ಲಿರುವ ಜಾಣ್ಮೆಯಾಗಲೀ ಪಾಂಡಿತ್ಯವಾಗಲೀ ಒಳ್ಳೆಯ ಗುಣವಾಗಲೀ ದೇವರೆ ಅವನಿಗೆ ಕೊಟ್ಟಿದ್ದು. ಇದನ್ನು ತಾನು ಸಂಪಾದಿಸಿದ್ದು ಎಂದು ತಿಳಿಯುವುದಿಲ್ಲ. ಅದಕ್ಕೆ ಅಹಂಕಾರ ಪಡುವುದೇನಿದೆ? ಇದೊಂದು ಜವಾಬ್ದಾರಿ. ಇದನ್ನು ಸರಿಯಾಗಿ ನೋಡಿಕೊಳ್ಳಬೇಕಾಗಿದೆ ಅಷ್ಟೆ. ಅದರಂತೆಯೇ ಅವನಲ್ಲಿ ಯಾವ ಮಮಕಾರವೂ ಇಲ್ಲ. ಈ ಮನೆ, ಐಶ್ವರ್ಯ, ಹೆಂಡತಿ, ಮಕ್ಕಳು,ಆಸ್ತಿ ಇವೆಲ್ಲ ತನ್ನವು ಎಂದು ಭಾವಿಸುವುದಿಲ್ಲ. ಇವುಗಳ ರಕ್ಷಕ ತಾನು ಅಷ್ಟೆ. ಇವೆಲ್ಲ ಸೇರಿರುವುದು ಭಗವಂತನಿಗೆ. ಅವನು ಇದನ್ನು ಎಲ್ಲಿಯಾದರೂ ಇಡಬೇಕಾಗಿದೆ. ಅದಕ್ಕೆ ನನ್ನಲ್ಲಿ ಇಟ್ಟಿರುವನು, ಅವನಿಚ್ಛೆ ಬಂದಾಗ ಅದನ್ನು ತೆಗೆದುಕೊಂಡು ಹೋಗುವನು ಎಂದು ತಿಳಿದಿರುವನು. ಅವನು ಕೊಟ್ಟ, ಕೊಟ್ಟವನು ತೆಗೆದುಕೊಂಡು ಹೋದ, ಧನ್ಯವಾಗಲಿ ಅವನ ನಾಮ ಎಂದು ಕೊಂಡಾಡುವನು. ಒಂದೂರಿನಲ್ಲಿ ಒಬ್ಬ ಆದರ್ಶ ಭಕ್ತ ದಂಪತಿಗಳಿದ್ದರು. ಅವರಿಗೆ ಕೆಲವು ಮಕ್ಕಳಿದ್ದರು. ಗಂಡ ದೂರದೂರಿಗೆ ಯಾವುದೋ ಕೆಲಸದ ಮೇಲೆ ಹೋದ. ಆಮೇಲೆ ಮಕ್ಕಳಿಗೆ ಕಾಲರ ತಗುಲಿ ಎಲ್ಲರೂ ತೀರಿಹೋದರು. ಅವರ ಶವವನ್ನು ಬಟ್ಟೆ ಹೊದಿಸಿ ಮನೆಯಲ್ಲಿ ಹೆಂಡತಿ ಇಡುವಳು. ಗಂಡ ಬಂದು ಮನೆಯ ಬಾಗಿಲು ತಟ್ಟಿದ. ಹೆಂಡತಿ ಮನೆಬಾಗಿಲ ಸಮೀಪದಲ್ಲೇ ನಿಂತು ಗಂಡನನ್ನು ಹೀಗೆ ಕೇಳಿದಳು: “ಯಾರೋ ಒಬ್ಬರು ನಮಗೆ ಸಾಮಾನನ್ನು ಕೊಟ್ಟು ಹೋಗಿದ್ದರು. ಅದು ಕೆಲವು ಕಾಲ ನಮ್ಮಲ್ಲಿತ್ತು. ಅವರು ಬಂದು ತೆಗೆದುಕೊಂಡು ಹೋದರೆ ನೀವು ವ್ಯಥೆ ಪಡುವಿರಾ!” ಅದಕ್ಕೆ ಗಂಡ, “ಅವರ ಸಾಮಾನನ್ನು ಅವರು ಕೊಂಡುಹೋದರು. ಅದಕ್ಕೆ ವ್ಯಥೆ ಏತಕ್ಕೆ?” ಎಂದ. ಆಗ ಹೆಂಡತಿ ಮಕ್ಕಳ ಸಾವಿನ ವಿಷಯವನ್ನು ಹೇಳಿದಳು. ಗಂಡ, “ಮಕ್ಕಳನ್ನು ದೇವರು ಕೊಟ್ಟ, ದೇವರು ತೆಗೆದುಕೊಂಡು ಹೋದ” ಎನ್ನುತ್ತಾನೆ.

ಅವನು ಸುಖದುಃಖಗಳಲ್ಲೂ ಸಮನಾಗಿರುವನು. ಸುಖ ಬಂದರೆ ಹಿಗ್ಗಿಬಿಡುವುದಿಲ್ಲ. ದುಃಖ ಬಂದರೆ ಕುಗ್ಗಿಬಿಡುವುದಿಲ್ಲ. ಈ ಸಂಸಾರದಲ್ಲಿ ಒಂದರ ಹಿಂದೆ ಮತ್ತೊಂದು ಬರುತ್ತಿರುವುದು. ಯಾರಿಗಾದರೂ ಆಗಲೀ ಬರೀ ಸುಖವೇ ಬರುವುದಿಲ್ಲ, ಅಥವಾ ಬರೀ ದುಃಖವೇ ಬರುವುದಿಲ್ಲ. ರಾತ್ರಿಯಾದಮೇಲೆ ಹಗಲು, ಹಗಲಾದಮೇಲೆ ರಾತ್ರಿ ಹೇಗೆ ಒಂದಾದಮೇಲೊಂದು ಬರುತ್ತಿದೆಯೋ ಹಾಗೆ ಸುಖದುಃಖಗಳು ಬಂದುಹೋಗುತ್ತಿರುವುವು. ಭಕ್ತ ಇದನ್ನು ಸಮನಾಗಿ ಕಾಣುತ್ತಾನೆ.

ಅವನು ಕ್ಷಮಾಶೀಲ. ಯಾರು ಎಂತಹ ತಪ್ಪನ್ನು ಮಾಡಿದರೂ ಅವನು ಕ್ಷಮಿಸಲು ಸಿದ್ಧ\-ನಾಗಿರುವನು. ತಪ್ಪು ಮಾಡುವುದು ಮನುಷ್ಯನ ಸ್ವಭಾವ. ಅದನ್ನು ಕ್ಷಮಿಸುವುದು ಭಕ್ತನ ಸ್ವಭಾವ. ಇನ್ನೊಬ್ಬರು ಮಾಡಿದ ತಪ್ಪನ್ನೇ ಮೆಲಕು ಹಾಕುತ್ತಾ, ಅವರಿಗೆ ಬಡ್ಡಿ ಸಮೇತ ಕೊಡುವುದಕ್ಕೆ ಹೊಂಚು ಹಾಕುವವನಲ್ಲ ಭಕ್ತ. ಯಾವಾಗ ನಾವು ಇನ್ನೊಬ್ಬರ ತಪ್ಪನ್ನು ಮೆಲುಕು ಹಾಕುತ್ತಿರುವೆವೊ, ಅದು ನಮ್ಮ ಮನಸ್ಸನ್ನು ಹಾಳು ಮಾಡುವುದು. ಮೆಲುಕು ಹಾಕಬೇಕಾದರೆ ಇನ್ನೊಬ್ಬನಲ್ಲಿರುವ ಒಳ್ಳೆಯ ಗುಣಗಳನ್ನು ಮೆಲುಕು ಹಾಕೋಣ. ಇದರಿಂದ ನನಗೆ ಮೇಲಾಗುವುದು. ಉತ್ತಮ ಸಂಸ್ಕಾರಗಳನ್ನು ಇದು ಬಿಡುವುದು. ನಾವು ಕೆಟ್ಟದ್ದನ್ನು ಮಾಡುವುದೂ ಒಂದೇ, ಇನ್ನೊಬ್ಬರು ಮಾಡಿದ ಕೆಟ್ಟದ್ದನ್ನು ಚಿಂತಿಸುವುದೂ ಒಂದೇ. ಇದು ನಮ್ಮ ಮನಸ್ಸನ್ನು ಹಾಳುಮಾಡುವುದು. ಅಂತಹ ಸಮಯದಲ್ಲಿ ತಪ್ಪು ಮಾಡಿದವನನ್ನು ಉದಾರವಾಗಿ ನೋಡಿ ‘ಅಯ್ಯೊ ತಪ್ಪು ಮಾಡದವರು ಯಾರಿದ್ದಾರೆ?’ ಎಂದು ಕ್ಷಮಿಸಿಬಿಟ್ಟರೆ, ಅದು ನಮ್ಮನ್ನು ಮೇಲೆತ್ತುವುದು.

ಜೀವನದಲ್ಲಿ ಭಕ್ತ ಯಾವಾಗಲೂ ಸಂತುಷ್ಟ. ದೇವರು ಅವನನ್ನು ಯಾವ ಸ್ಥಿತಿಯಲ್ಲೇ ಇಟ್ಟಿರಲಿ, ಅವನಿಗೆ ಏನೇ ಕೊಡಲಿ, ಗೊಣಗಾಡುವುದಿಲ್ಲ ಅವನು. ನನಗೆ ಇನ್ನೂ ಹೆಚ್ಚು ಬರಬೇಕಾಗಿತ್ತು ಎನ್ನುವವನಲ್ಲ ಅವನು. ನಮ್ಮ ಆಸೆಗೆ ಒಂದು ಮಿತಿ ಎಲ್ಲಿದೆ? ಎಷ್ಟು ಕೊಟ್ಟರೂ ನಮಗೆ ಇನ್ನೂ ಬೇಕು. ಆದರೆ ಭಕ್ತ ನೋಡುವ ದೃಷ್ಟಿ ಬೇರೆ. ನಮಗಿಂತ ಹೆಚ್ಚಾಗಿ ದೇವರಿಗೆ ಗೊತ್ತಿದೆ ನಮಗೆ ಎಷ್ಟು ಕೊಡಬೇಕು, ಯಾವ ಕಾಲದಲ್ಲಿ ಕೊಡಬೇಕು, ಏನು ಕೊಡಬೇಕು ಎಂಬುದು.\break ಯಾವಾಗ ಜವಾಬ್ದಾರಿಯನ್ನೆಲ್ಲಾ ಅವನಿಗೆ ಬಿಟ್ಟಿರುವೆವೋ ನಾವು ಇನ್ನು ಮೇಲೆ ಕೊಸರಾಡುವುದಿಲ್ಲ.

ಅವನು ಯೋಗಿ ಎಂದರೆ ದೇವರೊಂದಿಗೆ ಸಂಬಂಧ ಕಲ್ಪಿಸಿಕೊಂಡಿರುವನು. ಯಾವಾಗಲೂ ಅವನನ್ನು ಇವನು ಬಿಟ್ಟಿಲ್ಲ. ಅವನು ಯತಾತ್ಮ. ತನ್ನ ದೇಹ ಇಂದ್ರಿಯ ಮನಸ್ಸು ಇವುಗಳನ್ನೆಲ್ಲ ನಿಗ್ರಹಿಸಿರುವನು. ಅವನು ದೃಢನಿಶ್ಚಯನು. ಅವನಿಗೆ ದೇವರ ವಿಷಯದಲ್ಲಿ ಲವಲೇಶವೂ ಅನು ಮಾನವಿಲ್ಲ. ಅನುಮಾನ ಸಂದೇಹಗಳನ್ನೆಲ್ಲಾ ಮೀರಿ ಹೋಗಿರುವನು ಅವನು. ಯಾವಾಗಲೂ ಭಗವಂತನ ಭಕ್ತನಾಗಿರುವನು. ಸುಖ ಕೊಡಲಿ ದುಃಖ ಕೊಡಲಿ, ಲಾಭ ಕೊಡಲಿ ನಷ್ಟ ಕೊಡಲಿ, ಜಯ ಬರಲಿ, ಅಪಜಯ ಬರಲಿ, ಸ್ತುತಿನಿಂದೆ ಯಾವುದೇ ಬರಲಿ, ಅವನಂತೂ ದೇವರನ್ನು ಬಿಡುವವನಲ್ಲ. ಅವನು ತನ್ನ ದೇಹ ಮನಸ್ಸು ಬುದ್ಧಿ ಇವುಗಳನ್ನೆಲ್ಲಾ ಭಗವಂತನಿಗೆ ಅರ್ಪಿಸಿರುವನು. ಇನ್ನುಮೇಲೆ ಅವುಗಳೆಲ್ಲಾ ಇರುವುದು ಭಗವಂತನನ್ನು ಪಡೆಯುವುದಕ್ಕಾಗಿ, ತನ್ನ ಸ್ವಾರ್ಥವನ್ನು ಹೆಚ್ಚಿಸಿಕೊಳ್ಳುವುದಕ್ಕಲ್ಲ. ಯಾವ ಭಕ್ತನಲ್ಲಿ ಅಂತಹ ಗುಣಗಳಿವೆಯೋ ಅವನನ್ನು ದೇವರು ಮೆಚ್ಚುತ್ತಾನೆ.

\begin{shloka}
ಯಸ್ಮಾನ್ನೋದ್ವಿಜತೇ ಲೋಕೋ ಲೋಕಾನ್ನೋದ್ವಿಜತೇ ಚ ಯಃ~।\\ಹರ್ಷಾಮರ್ಷಭಯೋದ್ವೇಗೈರ್ಮುಕ್ತೋ ಯಃ ಸ ಚ ಮೇ ಪ್ರಿಯಃ \hfill॥ ೧೫~॥
\end{shloka}

\begin{artha}
ಯಾರಿಂದ ಜನ ಉದ್ವೇಗವನ್ನು ಹೊಂದುವುದಿಲ್ಲವೋ, ಯಾರು ಜನರಿಂದ ಉದ್ವೇಗ ಹೊಂದುವು\-ದಿಲ್ಲವೋ ಮತ್ತು ಯಾರಿಗೆ ಹರ್ಷ, ಅಸಹನೆ, ಉದ್ವೇಗಗಳಿಲ್ಲವೋ ಅವನು ನನಗೆ ಪ್ರಿಯನು.
\end{artha}

ಭಕ್ತ ಇತರರ ಸ್ವಾರ್ಥಕ್ಕೆ ಅಡ್ಡಿ ತರುವುದಿಲ್ಲ. ಅವನ ಪಾಡಿಗೆ ಅವನು ಇರುತ್ತಾನೆ. ಈ ಪ್ರಪಂಚದಲ್ಲಿ ಕೇವಲ ವಾದದಿಂದ ಯಾರೂ ಒಬ್ಬನನ್ನು ಮೇಲಕ್ಕೆ ಎತ್ತುವುದಕ್ಕೆ ಸಾಧ್ಯವಿಲ್ಲ. ಈ ಪ್ರಪಂಚ ನಶ್ವರ, ಕ್ಷಣಿಕ, ಇದನ್ನು ನೆಚ್ಚಬೇಡ ಎಂದು ಎಷ್ಟು ಹೇಳಿದರೂ ಪ್ರಪಂಚದ ಜನ ಏನೂ ಕೇಳುವವರಲ್ಲ ಎಂಬುದನ್ನು ಭಕ್ತ ಚೆನ್ನಾಗಿ ತಿಳಿದುಕೊಂಡಿರುವನು. ಅವರಿಗೆ ಅನುಭವ ಆಗಬೇಕು. ಅವರಿಗೇ ಬಿಸಿ ತಾಗಬೇಕು; ಆಗಲೇ ಅವರು ಅದನ್ನು ಬಿಡುವರು. ಅದಕ್ಕಾಗಿಯೇ ಭಕ್ತ ತನ್ನ ಪಾಡಿಗೆ ತಾನಿರುತ್ತಾನೆ. ಇನ್ನೊಬ್ಬನಿಗೆ ತನ್ನಿಚ್ಛೆಯಂತೆ ಇರಲು ತಾನು ಅಡ್ಡಿ ಬರುವುದಿಲ್ಲ.

ಹಾಗೆಯೇ ಪ್ರಾಪಂಚಿಕ ಜನರು ಭಕ್ತರ ನೆಮ್ಮದಿಗೆ ಅಡ್ಡಿ ಬರಲಾರರು. ಬೇಕಾದರೆ ಅವರು ತಮ್ಮಲ್ಲಿರುವ ಅಧಿಕಾರ ದರ್ಪ ಐಶ್ವರ್ಯ ಇವುಗಳಿಂದ ಭಕ್ತರನ್ನು ಅಂಜಿಸಬಹುದು. ಆದರೆ ಭಕ್ತನಾದವನು ಯಾರಿಗೂ ಅಂಜುವವನಲ್ಲ. ಅವನು ದೀನರಲ್ಲಿ ದೀನನಂತೆ ಕಾಣುವನು. ಅವನಲ್ಲಿ ಯಾವ ಶಕ್ತಿಯೂ ಇಲ್ಲದಂತೆ ತೋರುವನು. ಆದರೆ ಅವನಷ್ಟು ಬಲಿಷ್ಠ ಈ ಪ್ರಪಂಚದಲ್ಲಿ ಬೇರೆ ಯಾರೂ ಇಲ್ಲ. ಅವನು ದೇವರನ್ನು ನಂಬಿದವನು, ನೆಚ್ಚಿದವನು, ಅವನ ಕೈ ಹಿಡಿದು ನಡೆಯುವನು. ಈ ಪ್ರಪಂಚದ ಹುಲು ಮನುಷ್ಯರ ಗೊಡ್ಡು ಬೆದರಿಕೆಗೆ ಅಂಜುವವನಲ್ಲ. ಏಕಾಂಗಿಯಾಗಿ ಈತ ಪ್ರಪಂಚದಲ್ಲಿ ಎದುರಿಸಬಲ್ಲ. ತನ್ನ ಶಾಂತಿಗೆ ಯಾರೂ ಭಂಗ ತರದ ರೀತಿಯಲ್ಲಿ ಅವನು ತನ್ನ ಮನಸ್ಸನ್ನು ಅಣಿಮಾಡಿಕೊಂಡಿರುವನು.

ಅವನಿಗೆ ಏನಾದರೂ ಒಳ್ಳೆಯದಾದರೆ ಹರ್ಷವಿಲ್ಲ. ಏಕೆಂದರೆ ಇದರಿಂದ ಹಿಂದೆ ಏನು ದುಃಖವಿದೆಯೋ ಅದು ಯಾವಾಗ ಬರುವುದೋ ಗೊತ್ತಿಲ್ಲ. ಈಗಿರುವ ಹರ್ಷವೂ ಶಾಶ್ವತವಾಗಿರುವುದಿಲ್ಲ. ಅದನ್ನು ನೋಡುತ್ತಿದ್ದಂತೆ ಅದು ಮಾಯವಾಗುವುದು. ಇದು ಕ್ಷಣಿಕ, ತಾತ್ಕಾಲಿಕ ಎಂಬುದನ್ನು ಚೆನ್ನಾಗಿ ಅರಿತಿರುವನು.

ಯಾರಾದರೂ ಇತರರಿಗೆ ಒಳ್ಳೆಯದಾದರೆ ಇವನು ಅಸಹನೆ ಪಡುವುದಿಲ್ಲ. ಅವನೊಂದಿಗೆ ಸಂತೋಷ ಪಡುವನು. ದೇವರು ಅವನಿಗೆ ಒಳ್ಳೆಯದನ್ನು ಕೊಟ್ಟನಲ್ಲ, ಸಂತೋಷ ಎಂದು ಆನಂದಿಸು\-ವನು.‘ಅಯ್ಯೊ, ನನಗೆ ಅದು ಬರಲಿಲ್ಲವಲ್ಲ’ ಎಂದು ಕರುಬಿದರೆ ಅದು ನನಗೆ ಬರುವುದೇ ಇಲ್ಲ. ಈಗ ಇರುವ ಸ್ವಾಸ್ಥ್ಯವೂ ಭಂಗವಾಗುವುದು.

ಅವನಿಗೆ ಭಯವೆಂಬುದು ಎಳ್ಳಷ್ಟೂ ಇಲ್ಲ. ಜೀವನದಲ್ಲಿ ಸ್ವಾರ್ಥಿ ಅಂಜಬೇಕು, ತನ್ನಲ್ಲಿರುವುದನ್ನು ಎಲ್ಲಿ ಕಳೆದುಕೊಳ್ಳುತ್ತೇನೆಯೋ ಎಂದು. ಭಗವಂತನನ್ನು ನಂಬಿದವನಿಗೆ ಈ ಅಂಜಿಕೆ ಇಲ್ಲ. ಜೀವನದಲ್ಲಿ ಎಂತಹ ಕಹಿ ಅನುಭವ ಬರಲಿ ಅವನು ಅದನ್ನು ಅನುಭವಿಸಲು ಸಿದ್ಧನಾಗಿರುವನು. ಯಾಕೆಂದರೆ ಎಲ್ಲದರ ಹಿಂದೆ ಅವನು ದೇವರನ್ನು ಕಾಣುವನು. ಯಾವ ಬಗೆಯ ಉದ್ವೇಗಕ್ಕೂ ಅವನು ಎಡೆಗೊಡುವುದಿಲ್ಲ. ದೇವರನ್ನು ಮರೆತರೆ ಅಂಜಿಕೆಯ ಪಿಶಾಚಿ ನಮ್ಮನ್ನು ಹಿಡಿಯುವುದು. ದೇವರನ್ನು ಚಿಂತಿಸುವ ವ್ಯಕ್ತಿಯನ್ನು ಯಾರೂ ಯಾವುದೂ ವ್ಯಸ್ಥಗೊಳಿಸಲಾರದು.

\begin{shloka}
ಅನಪೇಕ್ಷಃ ಶುಚಿರ್ದಕ್ಷ ಉದಾಸೀನೋ ಗತವ್ಯಥಃ~।\\ಸರ್ವಾರಂಭಪರಿತ್ಯಾಗೀ ಯೋ ಮದ್ಭಕ್ತಃ ಸ ಮೇ ಪ್ರಿಯಃ \hfill॥ ೧೬~॥
\end{shloka}

\begin{artha}
ಯಾರು ಅಪೇಕ್ಷಾಶೂನ್ಯನೋ, ಶುಚಿಯೋ, ದಕ್ಷನೋ, ಉದಾಸೀನನೋ, ವ್ಯಥೆ ಇಲ್ಲದವನೋ, ಸರ್ವಕರ್ಮ ಪರಿತ್ಯಾಗಿಯೋ ಅಂತಹ ಭಕ್ತನು ನನಗೆ ಪ್ರಿಯನು.
\end{artha}

ಭಗವಂತನು ತನಗೆ ಪ್ರಿಯನಾದ ಭಕ್ತನ ಲಕ್ಷಣಗಳನ್ನು ಇನ್ನಷ್ಟು ಕೊಡುತ್ತಾನೆ. ಅವನು ಅಪೇಕ್ಷಾಶೂನ್ಯನು. ಅವನಿಗೆ ಜೀವನದಲ್ಲಿ ಲೌಕಿಕವಾದ ಯಾವುದನ್ನೂ ಪಡೆದು ಅನುಭವಿಸ\-ಬೇಕೆಂಬ ಆಸೆ ಇಲ್ಲ. ದೇವರಲ್ಲದ ಇತರ ವಸ್ತುಗಳು ದುಃಖಕ್ಕೆ ಕೊಂಡೊಯ್ಯುವುವು ಎಂಬುದನ್ನು ಚೆನ್ನಾಗಿ ಅರಿತವನು ಅವನು.

ಅವನು ಶುಚಿಯಾಗಿರುತ್ತಾನೆ. ಅವನ ದೇಹ ಶುಚಿ, ಮನಸ್ಸು ಶುಚಿಯಾಗಿರುವುದು. ಯಾವಾ ಗಲೂ ಭಗವಂತ ಎಲ್ಲಿ ಶುಚಿ ಇದೆಯೋ ಅಲ್ಲಿ ಎಲ್ಲೆಡೆಗಿಂತ ಹೆಚ್ಚಾಗಿ ವ್ಯಕ್ತನಾಗುತ್ತಾನೆ. ದೇಹವನ್ನು ಅವನು ಶುಚಿಯಾಗಿಡುತ್ತಾನೆ. ಅದಕ್ಕೆ ಸ್ನಾನ ಮಾಡಿಸಬೇಕು. ಅದಕ್ಕೆ ಬಟ್ಟೆ ಹಾಕಬೇಕು. ಅದಕ್ಕೆ ಆಹಾರಾದಿಗಳನ್ನು ಕೊಡಬೇಕು. ಅದಕ್ಕೆ ವಿಹಾರಗಳನ್ನು ಕೊಡಬೇಕು. ದೇಹ ಈ ಸಂಸಾರವನ್ನು ದಾಟುವುದಕ್ಕೆ ಇರುವ ದೋಣಿ. ಇದು ಬಲವಾಗಿರಬೇಕು ಇತರ ಸಾಧನೆಗಳನ್ನು ಮಾಡಲು. ಇದು ಬಲವಾಗಿರಬೇಕಾದರೆ ಶುಚಿಯಾಗಿರಬೇಕು. ರೋಗಕ್ಕೆಲ್ಲ ಈ ನಿಯಮವನ್ನು ಉಲ್ಲಂಘಿಸುವುದೇ ಕಾರಣ. ಇಲ್ಲಿ ಅವನು ದೇಹವನ್ನು ಶುಚಿಯಾಗಿ ಇಟ್ಟಿರುತ್ತಾನೆ. ಹಾಗಾದರೆ ಅವನು ದೇಹಾಸಕ್ತನಾಗಿ ಇದನ್ನು ಮಾಡುತ್ತಿರುವನು ಎಂದಲ್ಲ. ಇದೆಲ್ಲ ದೇಹದ ಸಾಲ. ಈ ಸಾಲವನ್ನು ತೀರಿಸಬೇಕಾಗಿದೆ, ಎಂಬ ದೃಷ್ಟಿಯಿಂದ ಮಾಡುತ್ತಾನೆ. ನಾವು ಯಾವಾಗ ಸಾಲವನ್ನು ತೀರಿಸುವುದಿಲ್ಲವೊ ಆಗ ಅದನ್ನು ಬಡ್ಡಿ ಸಮೇತ ತೀರಿಸಬೇಕಾಗುವುದು. ಅದಕ್ಕೆ ಅನಂತರ ಔಷಧ ಪಥ್ಯಗಳನ್ನು ತೆಗೆದುಕೊಳ್ಳಬೇಕಾಗುವುದು. ದೇಹವನ್ನು ಹೊತ್ತುಬಂದ ಸಾಲಕ್ಕೆ ಬಡ್ಡಿಯನ್ನು ತೆರುವುದು. ಭಕ್ತ ಈ ಸಾಲವನ್ನು ಸಾಧ್ಯವಾದಷ್ಟು ಕಡಮೆ ಮಾಡುತ್ತಾನೆ.

ಬರಿಯ ದೇಹವನ್ನು ಮಾತ್ರ ಶುಚಿಯಾಗಿಡುವುದಲ್ಲ, ಮನಸ್ಸನ್ನು ಶುಚಿಯಾಗಿಡುವುದು ಮುಖ್ಯ. ಅದು ಕೊಳೆಯಾಗುವುದು ಕೆಟ್ಟ ಆಲೋಚನೆಗಳಿಂದ. ಅದನ್ನು ಅವನು ಮಾಡುವುದಿಲ್ಲ. ಹೇಗೆ ದೇಹವನ್ನು ಶುಚಿಯಾಗಿಟ್ಟಿರಬೇಕಾದರೆ ಸ್ನಾನಾದಿಗಳನ್ನು ಮಾಡಿಸಬೇಕೋ, ಹಾಗೆ ಮನಸ್ಸನ್ನು ಶುಚಿಯಾಗಿಡಬೇಕಾದರೆ, ಅದನ್ನು ಧ್ಯಾನ, ಪೂಜೆ, ಪ್ರಾರ್ಥನೆ ಎಂಬ ಸ್ನಾನಾದಿಗಳಿಂದ ತೊಳೆಯ ಬೇಕಾಗುವುದು. ಪಾತ್ರೆಯನ್ನು ಪ್ರತಿದಿನ ಬೆಳಗುವಂತೆ ಮನಸ್ಸನ್ನು ಬೆಳಗಬೇಕು.\break ಯಾವಾಗ ನಾವು ಬೆಳಗುವುದರಲ್ಲಿ ಉದಾಸೀನರಾಗುವೆವೊ ಆಗ ಕಿಲುಬು ಕಟ್ಟುವುದು. ಕಿಲುಬು ತುಂಬಿದ ಮನಸ್ಸಿನಿಂದ ಭಗವಂತನನ್ನು ಹೇಗೆ ಚಿಂತಿಸುವುದಕ್ಕೆ ಸಾಧ್ಯವಾಗುವುದು?

ಭಕ್ತ ಯಾವ ಕೆಲಸ ತನ್ನ ಪಾಲಿಗೆ ಬರುವುದೋ ಅದನ್ನು ದಕ್ಷತೆಯಿಂದ ಮಾಡುವನು. ಭಕ್ತನೆಂದರೆ ವ್ಯವಹಾರ ದೃಷ್ಟಿಯಿಂದಲೂ ದಕ್ಷ. ಅವನು ಮಾಡುವ ಕೆಲಸದಲ್ಲಿ ಯಾವ ಅಸಡ್ಡೆ ಮರೆವು ಇವು ಬಾರದಂತೆ ನೋಡಿಕೊಳ್ಳುವನು. ಒಬ್ಬ ಇನ್ನೊಬ್ಬನಿಂದ ಹೊಗಳಿಸಿಕೊಳ್ಳಬೇಕಾದರೆ ಎಷ್ಟೋ ಮನಸ್ಸಿಟ್ಟು ಕೆಲಸ ಮಾಡುವನು. ಆದರೆ ಭಕ್ತನಾದರೋ ತನ್ನ ಪಾಲಿಗೆ ಬರುವ ಕೆಲಸವನ್ನು ಇದು ಭಗವಂತನಿಗೆ ಮಾಡುವ ಪೂಜೆ ಎಂಬ ದೃಷ್ಟಿಯಿಂದ ಮಾಡುವನು. ಅವನು ಪಡಪೋಸಿ ಅಲ್ಲ, ಬೇಕಾಬಿಟ್ಟಿ ಮಾಡುವವನಲ್ಲ. ಯಾವ ಒಂದು ಸಣ್ಣ ಪುಟ್ಟ ಕೆಲಸವೇ ಆಗಲಿ ಅದನ್ನು ಅಚ್ಚುಕಟ್ಟಾಗಿ ಮಾಡುವನು.

ಹಾಗೆ ಅವನ ಕೆಲಸ ಮಾಡುತ್ತಿರುವಾಗ ಒಬ್ಬೊಬ್ಬರು ಅದನ್ನು ಒಂದೊಂದು ದೃಷ್ಟಿಯಿಂದ ನೋಡುವರು. ಕೆಲವರು ಅವನು ಮಾಡುವ ಕೆಲಸದಲ್ಲಿ ತಪ್ಪು ಕಂಡು ಹಿಡಿಯುವರು. ಯಾವ ಯಾವುದೋ ಲೌಕಿಕ ಉದ್ದೇಶಗಳನ್ನು ಅಲ್ಲಿ ಆರೋಪಿಸುವರು. ಇಲ್ಲವೇ ಇವನ ಸಮಾನವಿಲ್ಲ ಎಂದು ಹೊಗಳುವರು. ಆದರೆ ಇವನು ಇವುಗಳಾವುದಕ್ಕೂ ಕಿವಿಕೊಡುವುದಿಲ್ಲ. ಇವನಿಗೆ ಜನ ಏನು ಹೇಳುತ್ತಾರೆ ಎಂಬುದು ಗೌಣ. ಈ ಪ್ರಪಂಚದಲ್ಲಿ ಪ್ರತಿಯೊಂದನ್ನೂ ಟೀಕಿಸುವ ಜನ ಇದ್ದಾರೆ, ಹಾಗೆಯೇ ಪ್ರತಿಯೊಂದನ್ನೂ ಹೊಗಳುವ ಜನರಿದ್ದಾರೆ. ಇದನ್ನು ನಾವು ಕಿವಿಗೆ ಹಾಕಿಕೊಂಡರೆ ಮಂದಿ ಹೇಳಿದಂತೆ ಕುಣಿಯಬೇಕಾಗುವುದು. ಭಕ್ತ ಇದನ್ನು ಮಾಡುವುದಿಲ್ಲ. ಭಗವಂತ ಅವನ ದೇವರು. ದೇವರ ಮೇಲಿನ ಪ್ರೀತಿಗಾಗಿ ತನ್ನ ಪಾಲಿನ ಕರ್ತವ್ಯವನ್ನು ಶ್ರದ್ಧಾಭಕ್ತಿಯಿಂದ ನಿರ್ವಹಿಸುವನು. ಅಲ್ಲಿಗೆ ಅದು ಕೊನೆಗೊಳ್ಳುವುದು. ಅವನಿಗೆ ಇದಲ್ಲದೆ ಬೇರೆ ಯಾವ ಫಲವೂ ಬೇಕಾಗಿಲ್ಲ. 

ಭಕ್ತನಲ್ಲಿ ವ್ಯಥೆ ನೋಡುವುದಿಲ್ಲ. ಕೊರಗುವುದು, ಅಳುಮೋರೆ ಹಾಕುವುದು, ಭಕ್ತನ ಲಕ್ಷಣವಲ್ಲ. ಜೀವನದಲ್ಲಿ ಅವನು ಯಾವುದಕ್ಕೂ ವ್ಯಥೆ ಪಡುವುದಿಲ್ಲ. ಯಾವನು ಸಚ್ಚಿದಾನಂದಸ್ವರೂಪನಾದ ಭಗವಂತನ ಕೈಯನ್ನು ಹಿಡಿದುಕೊಂಡು ಜೀವನದಲ್ಲಿ ನಡೆಯುತ್ತಿರುವನೊ ಅವನಲ್ಲಿ ಅಳು\-ಮೋರೆಯನ್ನು ಹೇಗೆ ನೋಡಲು ಸಾಧ್ಯ? ಅವನು ತನ್ನ ಜೀವನದ ಜವಾಬ್ದಾರಿಯನ್ನು ಭಗವಂತನಿಗೆ ಅರ್ಪಿಸಿರುವನು. ಅವನು ಮಾಡಿಸಿದಂತೆ ಮಾಡುತ್ತಾನೆ. ಅವನು ನುಡಿದಂತೆ ನುಡಿಯುತ್ತಾನೆ. ನನ್ನನ್ನು ಆಡಿಸುವವನು ನುಡಿಸುವವನು ಎಂದೆಂದಿಗೂ ತಪ್ಪು ಮಾಡುವವನಲ್ಲ. ಅವನನ್ನು ಮರೆತಾಗ ನಾವು ವ್ಯಥೆಪಡಬೇಕಾಗುವುದು. ಆದರೆ ಭಕ್ತ ಎಂತಹ ಅನುಭವಗಳಾದರೂ ದೇವರನ್ನು ಎಂದಿಗೂ ಮರೆಯುವವನಲ್ಲ. ಎಲ್ಲವನ್ನೂ ಹಸನ್ಮುಖದಿಂದ ಸ್ವೀಕರಿಸುತ್ತಾನೆ.

ಅವನು ಎಲ್ಲಾ ವಿಧವಾದ ಕಾಮ್ಯಕರ್ಮಗಳನ್ನು ಬಿಟ್ಟಿದ್ದಾನೆ, ಇಹಲೋಕದಲ್ಲಾಗಲೀ, ಪರ\-ಲೋಕದಲ್ಲಾಗಲೀ ಭೋಗವನ್ನು ಆಶಿಸುವುದಿಲ್ಲ. ಅವನು ಭೋಗ ಏನು ಮಾಡಬಲ್ಲದು ಎಂಬು\-ದನ್ನು ಚೆನ್ನಾಗಿ ಮೊದಲೇ ಬಲ್ಲ. ಭೋಗದ ದಾಸ್ಯಕ್ಕೆ ಬಿದ್ದರೆ ಇಂದ್ರಿಯದ ಸೆರೆಮನೆಯಲ್ಲಿ ಕೊಳೆಯ ಬೇಕಾಗುವುದು. ಅವನು ಏನನ್ನೂ ಇಚ್ಛಿಸುವುದಿಲ್ಲ. ಯಾವುದಕ್ಕೂ ತೊಂದರೆ ಪಡುವುದಿಲ್ಲ. ಭಗವಂತ ಯಾವ ಯಾವ ಕಾಲಕ್ಕೆ ಏನನ್ನು ಕೊಡುತ್ತಾನೆಯೋ ಅದನ್ನು ಸ್ವೀಕರಿಸುವನು.

\begin{shloka}
ಯೋ ನ ಹೃಷ್ಯತಿ ನ ದ್ವೇಷ್ಟಿ ನ ಶೋಚತಿ ನ ಕಾಂಕ್ಷತಿ~।\\ಶುಭಾಶುಭಪರಿತ್ಯಾಗೀ ಭಕ್ತಿಮಾನ್ ಯಃ ಸ ಮೇ ಪ್ರಿಯಃ \hfill॥ ೧೭~॥
\end{shloka}

\begin{artha}
ಯಾರು ಹರ್ಷಿಸುವುದಿಲ್ಲವೊ, ದ್ವೇಷಿಸುವುದಿಲ್ಲವೊ, ಶೋಕಿಸುವುದಿಲ್ಲವೊ, ಇಚ್ಛಿಸುವುದಿಲ್ಲವೊ, ಶುಭ ಅಶುಭಗಳನ್ನು ತ್ಯಾಗ ಮಾಡಿರುವನೋ ಅವನು ನನಗೆ ಪ್ರಿಯ.
\end{artha}

ನಿಜವಾದ ಭಕ್ತ ಬೇಕಾದುದು ಬಂದಾಗ ಸಂತೋಷಪಡುವುದೂ ಇಲ್ಲ, ಬೇಡದುದು ಬಂದಾಗ ಶೋಕಿಸುವುದೂ ಇಲ್ಲ. ಬೇಕಾದುದೇನೊ ಈಗ ಬಂದಿರಬಹುದು. ಆದರೆ ಅದು ಯಾವಾಗಲೂ ನಮ್ಮ ಹತ್ತಿರ ಇರುವುದಿಲ್ಲ. ಅದು ಹೇಗೆ ನಮ್ಮನ್ನು ಹೇಳದೆ ಕೇಳದೆ ಬಂತೋ ಹಾಗೆಯೇ ಅದು ಹೇಳದೆ ಕೇಳದೆ ಹೋಗುವುದು. ಯಾವಾಗ ಅದರ ಬಲೆಗೆ ಬೀಳುತ್ತೇವೆಯೋ ಆಗ ಅದರ ದಾಸರಾಗುತ್ತೇವೆ. ಇದನ್ನು ಭಕ್ತ ಚೆನ್ನಾಗಿ ಬಲ್ಲ. ಅದರಂತೆಯೇ ನಮಗೆ ಬೇಡವಾದುದು ಬಂದರೆ ದ್ವೇಷಿಸುವುದೂ ಇಲ್ಲ. ನಮಗೆ ಬೇಡದ ಖಾಯಿಲೆ ಬರಬಹುದು. ನಮಗೆ ಪ್ರಿಯವಾದವರನ್ನು ಅಕಸ್ಮಾತ್ ಕೊಂಡೊಯ್ಯಲು ಮೃತ್ಯು ಬರಬಹುದು. ಅದನ್ನು ದ್ವೇಷಿಸುವುದಿಲ್ಲ. ಅದು ಬೇಕಾದರೆ ಬರಬಹುದು, ಬೇಡದೆ ಇದ್ದರೆ ಬಿಡಬಹುದು. ಅವನು ಮನೆಯ ಬಾಗಿಲನ್ನು ತೆರೆದಿರುವನು. ಬರುವವರು ಮತ್ತು ಹೋಗುವವರನ್ನು ಅವನು ಗಮನಿಸುವುದಿಲ್ಲ. ಇವುಗಳೆಲ್ಲಾ ಭಗವಂತನ ನಿಯಮಾನುಸಾರ ಆಗುತ್ತಿವೆ ಎಂಬುದನ್ನರಿತು ಇವುಗಳನ್ನು ಸಹಿಸುವುದನ್ನು ಕಲಿತಿರುವನು.

ಭಕ್ತನು ತುಂಬಾ ಪ್ರಿಯವಾದುದು ಹೋದಾಗಲೂ ಗೋಳಿಡುವುದಿಲ್ಲ. ಪ್ರಿಯವಾದುದನ್ನು ಕೊಟ್ವವನು ದೇವರು. ಅವನಿಗೆ ತೆಗೆದುಕೊಂಡು ಹೋಗುವ ಸ್ವಾತಂತ್ರ್ಯವಿದೆ. ಜೀವನದಲ್ಲಿ ಯಾವುದೂ ಅಕಸ್ಮಾತ್ತಾಗಿ ಆಗುವುದಿಲ್ಲ. ಎಲ್ಲಾ ದೈವೇಚ್ಛೆಯಂತೆ ಆಗುವುದು ಎಂದು ಯಾವಾಗಲೂ ಭಗವಂತನ ಇಚ್ಛೆಗೆ ತಲೆಬಾಗುವುದು ಅವನ ಸ್ವಭಾವವಾಗಿದೆ. ನಮಗೆ ದೊರಕ\-ದಿರುವ ವಸ್ತುಗಳು ಎಷ್ಟೋ ಇವೆ ಈ ಪ್ರಪಂಚದಲ್ಲಿ. ಭಕ್ತ ಅವುಗಳಾವುದನ್ನೂಇಚ್ಛಿಸುವುದಿಲ್ಲ. ಯಾವಾಗ ಇಚ್ಛಿಸುತ್ತೇವೆಯೋ ಆಗ ಅದನ್ನು ಪಡೆಯಲು ನಿರತರಾಗುತ್ತೇವೆ. ಆಗ ಅದಕ್ಕೆ ಅಂಟಿಕೊಳ್ಳುತ್ತೇವೆ. ದೇವರನ್ನು ಮರೆಯುತ್ತೇವೆ. ಇಂದು ನಾಳೆ ಹೋಗುವ ಕ್ಷಣಿಕ ವಸ್ತುಗಳಿಗೆ ನಾವು ಜೀವನವನ್ನು ಒತ್ತೆ ಇಟ್ಟು ನಮ್ಮ ಸ್ವಾಸ್ಥ್ಯವನ್ನು ಕೆಡಿಸಿಕೊಳ್ಳುತ್ತೇವೆ.

ನಿಜವಾದ ಭಕ್ತನಿಗೆ ಶುಭ ಅಶುಭ ಎಂಬುದಿಲ್ಲ. ಎಲ್ಲಾ ಬರುವುದು ಭಗವಂತನಿಂದ. ಅವನು ಯಾವುದನ್ನಾದರೂ ಕೊಡಲಿ ಅದನ್ನು ಸ್ವೀಕರಿಸುತ್ತಾನೆ. ಅವನು ನಮಗೆ ಯಾವಾಗಲೂ ಸಿಹಿಯನ್ನೇ ಕೊಡಬೇಕಾಗಿಲ್ಲ. ಅವನು ಕೆಲವು ವೇಳೆ ಕಹಿಯಾಗಿರುವುದನ್ನೂ ಕೊಡುವನು. ನಮಗಿಂತ ಅವನಿಗೆ ಚೆನ್ನಾಗಿ ಗೊತ್ತಾಗಿದೆ ನಮಗೆ ಯಾವುದು ಒಳ್ಳೆಯದು ಎಂಬುದು. ಇದನ್ನರಿತು\break ದೇವರು ಕೊಡುವುದನ್ನು ಸ್ವೀಕರಿಸುತ್ತಾನೆಯೇ ಹೊರತು ದೇವರಿಗೆ ಅದು ಕೊಡು ಇದು ಕೊಡು ಎಂದು ಕೇಳುವುದಿಲ್ಲ.

\begin{shloka}
ಸಮಃ ಶತ್ರೌ ಚ ಮಿತ್ರೇ ಚ ತಥಾ ಮಾನಾಪಮಾನಯೋಃ~।\\ಶೀತೋಷ್ಣಸುಖದುಃಖೇಷು ಸಮಃ ಸಂಗವಿವರ್ಜಿತಃ \hfill॥ ೧೮~॥
\end{shloka}

\begin{shloka}
ತುಲ್ಯನಿಂದಾಸ್ತುತಿರ್ಮೌನೀ ಸಂತುಷ್ಟೋ ಯೇನ ಕೇನಚಿತ್~।\\ಅನಿಕೇತಃ ಸ್ಥಿರಮತಿರ್ಭಕ್ತಿಮಾನ್ ಮೇ ಪ್ರಿಯೋ ನರಃ \hfill॥ ೧೯~॥
\end{shloka}

\begin{artha}
ಶತ್ರು ಮಿತ್ರರಲ್ಲಿ ಸಮಬುದ್ಧಿ ಉಳ್ಳವನು, ಮಾನಾಪಮಾನಗಳಲ್ಲಿಯೂ ಶೀತೋಷ್ಣ ಸುಖದುಃಖ\-ಗಳಲ್ಲಿಯೂ ಸಮನಾಗಿರುವವನು, ಆಸಕ್ತಿ ಇಲ್ಲದವನು, ಸ್ತುತಿ ನಿಂದೆಗಳಲ್ಲಿ ಸಮಭಾವನೆಯುಳ್ಳ\-ವನು, ಮೌನಿಯೂ, ಬಂದುದರಲ್ಲಿ ತೃಪ್ತನೂ, ನಿರ್ದಿಷ್ಟ ಸ್ಥಾನವಿಲ್ಲದವನೂ, ಸ್ಥಿರಬುದ್ಧಿಯುಳ್ಳವನೂ, ಭಕ್ತಿವಂತನೂ ಆದ ಮನುಷ್ಯನು ನನಗೆ ಪ್ರಿಯನು.
\end{artha}

ಭಕ್ತ ಶತ್ರುಮಿತ್ರರನ್ನು ಒಂದೇ ಸಮನಾಗಿ ನೋಡುತ್ತಾನೆ. ಅವನಿಗೆ ಇಬ್ಬರ ಹಿಂದೆಯೂ ಒಬ್ಬನೇ ಭಗವಂತ ಕೆಲಸ ಮಾಡುತ್ತಿರುವುದು ಗೋಚರವಾಗುವುದು. ಇಬ್ಬರೂ ಉತ್ಪ್ರೇಕ್ಷೆ ಮಾಡು\-ತ್ತಾರೆ ಎಂಬುದನ್ನು ಆತ ಚೆನ್ನಾಗಿ ಬಲ್ಲ. ಶತ್ರು ನಮ್ಮ ಅವಗುಣಗಳನ್ನು ಉತ್ಪ್ರೇಕ್ಷೆ ಮಾಡುತ್ತಾನೆ. ಮಿತ್ರ ನಮ್ಮ ಸದ್ಗುಣಗಳನ್ನು ಉತ್ಪ್ರೇಕ್ಷೆ ಮಾಡುತ್ತಾನೆ. ಅಂತೂ ಇಬ್ಬರೂ ಹೇಳುವುದನ್ನು ನಾವು ವಿಮರ್ಶಿಸಬೇಕಾಗಿದೆ ಎಂಬುದು ಅವನಿಗೆ ಚೆನ್ನಾಗಿ ಗೊತ್ತು.

ಅದರಂತೆಯೇ ಮಾನ ಮತ್ತು ಅಪಮಾನ. ಇವೆರಡೂ ನಮ್ಮನ್ನು ಪರೀಕ್ಷೆ ಮಾಡುವುದಕ್ಕೆ ಬರುವುವು. ಕೆಲವರು ಅವನನ್ನು ಬೇಕಾದಷ್ಟು ಹೊಗಳುವರು. ಗೌರವ ಕೊಡುವರು, ಪೂಜೆ ಮಾಡುವರು. ಆಗ ಅವನ ತಲೆ ತಿರುಗಿ ಹೋಗುವುದಿಲ್ಲ. ಹಾಗೆಯೇ ಅವನನ್ನು ಕೆಲವರು ಟೀಕಿಸುವರು; ಅಗೌರವ ತೋರುವರು; ನಿಕೃಷ್ಟವಾಗಿ ಕಾಣುವರು. ಆಗ ಕೋಪವೂ ಇಲ್ಲ, ವ್ಯಥೆಯನ್ನು ವ್ಯಕ್ತಪಡಿಸುವುದಿಲ್ಲ. ಭಕ್ತ ಇದನ್ನು ಗಮನಿಸುವುದೇ ಇಲ್ಲ. ಇದನ್ನು ಉದಾಸೀನವಾಗಿ ನೋಡುತ್ತಾನೆ. ಒಬ್ಬೊಬ್ಬ ಮನುಷ್ಯರು ತಮಗೆ ತೋಚಿದ ರೀತಿಯಲ್ಲಿ ಇನ್ನೊಬ್ಬರನ್ನು ಅಳೆಯುತ್ತಾರೆ. ಅದನ್ನೆಲ್ಲಾ ನಾವು ಅಕ್ಷರಶಃ ತೆಗೆದುಕೊಂಡರೆ ಈ ಪ್ರಪಂಚದಲ್ಲಿ ಬಾಳುವಂತೆಯೇ ಇಲ್ಲ. ನಾವು ಹುಚ್ಚರಾಗಿ ಹೋಗಬೇಕಾಗುವುದು. ಅದರಂತೆಯೇ ಶೀತೋಷ್ಣಗಳು, ಇವುಗಳೆಲ್ಲಾ ಪುತುಧರ್ಮ. ಆಯಾ ಕಾಲದಲ್ಲಿ ತನ್ನ ಧರ್ಮವನ್ನು ತೋರಲೇಬೇಕು, ಇಲ್ಲದೆ ಇದ್ದರೆ ಅದು ಸಂಸಾರವೇ ಆಗುವುದಿಲ್ಲ, ಬೇಸಿಗೆ ಬಂದಾಗ ಈ ಸಲ ಎಂತಹ ಬಿಸಿಲು, ಹಿಂದೆ ಎಂದೂ ಇರಲಿಲ್ಲ ಎಂದು ಹೇಳುವೆವು. ನಮಗೆ ಹಿಂದಿನದು ಮರೆತುಹೋಗಿದೆ. ಅದೊಂದೇ ಅಲ್ಲ, ಬಿಸಿಲು ಕಾಲದಲ್ಲಿ ಬಿಸಿಲಲ್ಲದೆ ತಂಪಾಗಿರ ಬಲ್ಲದೇ! ಈ ವಿಷಯದಲ್ಲಿ ಅವನು ಗೊಣಗಾಡದೆ ಅನುಭವಿಸುವುದನ್ನು ಕಲಿತಿರುವನು. ಅದರಂತೆಯೇ ಸುಖದುಃಖಗಳು. ಇವೆರಡೂ ನಮ್ಮ ಜೀವನಕ್ಕೆ ಅವಶ್ಯಕ. ಅದಕ್ಕಾಗಿ ದೇವರು ಕೊಡುತ್ತಾನೆ. ಅವನು ಕೊಟ್ಟದ್ದು ತಪ್ಪು, ಅಥವಾ ಅದು ಜಾಸ್ತಿ ಎಂದು ದೇವರೊಂದಿಗೆ ಜಗಳ ಕಾಯುವುದಿಲ್ಲ. ನಮಗೆಷ್ಟು ಬೇಕೋ ಅದಕ್ಕಿಂತ ಜಾಸ್ತಿ ಅವನು ಕೊಡುವುದಿಲ್ಲ ಎಂದು ಚೆನ್ನಾಗಿ ನಂಬಿದವನು ಭಕ್ತ. ಈ ದ್ವಂದ್ವ ಅನುಭವಗಳಿಲ್ಲದೆ ಇದ್ದರೆ ಇದು ಸಂಸಾರವೇ ಆಗುವುದಿಲ್ಲ. ಅವನು ಅದಕ್ಕೆ ಸಿದ್ಧನಾಗಿರುವನು.

ಅವನಿಗೆ ಯಾವ ಸ್ಥಳದ ಮೇಲೂ, ಯಾವ ವ್ಯಕ್ತಿಯ ಮೇಲೂ ಆಸಕ್ತಿ ಇಲ್ಲ. ದೇವರು ಎಲ್ಲಿರುವನೋ ಅಲ್ಲಿರಲು ಅವನು ಸದಾ ಸಿದ್ಧ. ಅವನೊಂದು ಕುಂಡದಲ್ಲಿರುವ ಗಿಡದಂತೆ. ಅದನ್ನು ಎಲ್ಲಿಗೆ ಬೇಕಾದರೂ ತೆಗೆದುಕೊಂಡು ಹೋಗಬಹುದು. ಆದರೆ ನೆಲದೊಳಗೆ ಬೇರು ಬಿಟ್ಟ ಗಿಡವನ್ನು ಸ್ವಲ್ಪ ಅಲ್ಲಾಡಿಸಿದರೆ ಆಗಲೇ ಬಾಡುವುದು. ಒಂದು ವಸ್ತುವಿನ ಮಧ್ಯ ಇರುವಾಗ ಅದರಿಂದ ಹೋಗುವುದಕ್ಕೆ ಅವನು ಅಣಿಯಾಗಿರುವನು. ನಾವೆಲ್ಲ ಜೀವನದಲ್ಲಿ ಸಂಧಿಸುತ್ತೇವೆ–ಎಂದೆಂದಿಗೂ ಒಟ್ಟಿಗೆ ಇರುವುದಕ್ಕಲ್ಲ, ಒಬ್ಬೊಬ್ಬರು ಅವರ ಕಾಲ ಬಂದಾಗ ಬೇರೆ ದಾರಿಯನ್ನು ಹಿಡಿಯುವುದಕ್ಕೆ. ಇದನ್ನು ಚೆನ್ನಾಗಿ ಅರಿತವನು ಭಕ್ತ. 

ಅವನು ಜನ ತನ್ನನ್ನು ಹೊಗುಳುವಾಗ ತೆಗಳುವಾಗ ಮೌನವಾಗಿರುತ್ತಾನೆ. ಜನ ತಪ್ಪು ಕಂಡುಹಿಡಿಯುವಾಗ ತಾನು ಮಾಡಿದ್ದು ಸರಿ ಎಂದು ಸಮರ್ಥಿಸುವುದಕ್ಕೆ ಹೋಗುವುದಿಲ್ಲ. ಅವರು ಹೊಗಳುವುದು ತನಗೆ ಸಲ್ಲುತ್ತದೆ ಎಂದು ಆನಂದಿಸುವುದೂ ಇಲ್ಲ. ಇದನ್ನು ಕೇಳಿ ಅವನು ಸುಮ್ಮನಿರುವನು. ಮನಸ್ಸು ಯಾವ ಉದ್ವೇಗಕ್ಕೂ ಬೀಳದಂತೆ ನೋಡಿಕೊಳ್ಳವನು. ಅವನಿಗೆ ಏನು ಪ್ರಾಪ್ತವಾದರೂ ಅದರಲ್ಲಿ ಅವನು ತೃಪ್ತವಾಗಿರವನು. ಅಯ್ಯೋ ಜಾಸ್ತಿ ಬರಲಿಲ್ಲವಲ್ಲ ಎಂದು ಅತೃಪ್ತಿ ಇಲ್ಲ. ದೇವರು ನನಗೆ ಜೀವನದಲ್ಲಿ ಇಷ್ಟನ್ನು ಕೊಟ್ಟನಲ್ಲ ಎಂದು ತೃಪ್ತಿ ಪಡುತ್ತಾನೆ. ದೇವರಿಗೆ ಧನ್ಯವಾದವನ್ನು ಅರ್ಪಿಸುತ್ತಾನೆ ಅದಕ್ಕಾಗಿ. ಅವನಿಗೆ ಒಂದು ನಿರ್ದಿಷ್ಟ ಸ್ಥಾನ ಇಲ್ಲ. ತರಗೆಲೆಗೆ ತಾನು ಎಂತಹ ಕಡೆ ಇರಬೇಕು ಎಂಬ ಇಚ್ಛೆ ಇಲ್ಲ. ಗಾಳಿ ಎತ್ತ ಒಯ್ದರೆ ಅತ್ತ ಚಲಿಸುವುದು. ದೇವರು ಇಟ್ಟ ಕಡೆಯಲ್ಲಿ ಆತ ತೃಪ್ತ. ತಾನು ಇಂತಹ ಕಡೆಯಲ್ಲಿ ಇರಬೇಕು, ಹೀಗೆ ಇರಬೇಕು ಎಂಬ ಆಸೆಯನ್ನು ಬಿಟ್ಟು, ದೇವರು ತನ್ನಿಚ್ಛೆಯಂತೆ ಮಾಡಲಿ ಎಂದು ತನ್ನ ಜೀವನದ ಜವಾಬ್ದಾರಿಯನ್ನು ಭಗವಂತನಿಗೆ ಅರ್ಪಿಸಿಬಿಟ್ಟುರುವನು.

ಅವನ ಬುದ್ಧಿ ಸ್ಥಿರವಾಗಿದೆ. ಆಧ್ಯಾತ್ಮಿಕ ವಿಷಯಗಳಲ್ಲಿ ಸ್ವಲ್ಪವೂ ಸಂದೇಹದ ಮೋಡ ಅವನ ಮನಸ್ಸನ್ನು ಆವರಿಸುವುದಿಲ್ಲ. ಅವನು ಸಂದೇಹವನ್ನು ಮೀರಿರುವನು. ಎಲ್ಲವನ್ನೂ ಸರಿಯಾಗಿ ನೋಡುತ್ತಿರುವನು. ಅವನ ಕಣ್ಣಿಗೆ ಭಗವಂತ ಸದಾ ಧೃವತಾರೆಯಂತೆ ಗೋಚರಿಸುತ್ತಿರುವನು. ಹೃದಯದಲ್ಲಿ ಅವನಿಗಾಗಿ ಪ್ರೇಮ ಉಕ್ಕಿ ಹರಿಯುತ್ತಿರುವುದು. ಇಂತಹ ಭಕ್ತನೇ ಭಗವಂತನಿಗೆ ತುಂಬಾ ಪ್ರಿಯನು. ಅವನು ಎಷ್ಟುಮಟ್ಟಿಗೆ ತನಗೆ ಅರ್ಪಣೆ ಮಾಡಿಕೊಂಡಿರುವನು, ಅವನ ಹೃದಯ ಸಂಸಾರದಿಂದ ಎಷ್ಟು ಖಾಲಿಯಾಗಿದೆ ಎಂದು ದೇವರು ನೋಡುತ್ತಾನೆಯೇ ಹೊರತು ಅವನ ಬುದ್ಧಿವಂತಿಕೆ, ಜಾಣತನ ಪಾಂಡಿತ್ಯ ಮುಂತಾದುವುಗಳನ್ನು ಅವನು ಗಮನಿಸುವುದಿಲ್ಲ.

\begin{shloka}
ಯೇ ತು ಧರ್ಮ್ಯಾಮೃತಮಿದಂ ಯಥೋಕ್ತಂ ಪರ್ಯುಪಾಸತೇ~।\\ಶ್ರದ್ದಧಾನಾ ಮತ್ಪರಮಾ ಭಕ್ತಾಸ್ತೇಽತೀವ ಮೇ ಪ್ರಿಯಾಃ \hfill॥ ೨ಂ~॥
\end{shloka}

\begin{artha}
ಯಾವ ಭಕ್ತರು ಶ್ರದ್ಧೆಯುಳ್ಳವರಾಗಿ ಮತ್ಪರಾಯಣರಾಗಿ ಮೇಲೆ ಹೇಳಿದ ಧರ್ಮ್ಯಾಮೃತವನ್ನು ಉಪಾಸನೆ ಮಾಡುತ್ತಾರೆಯೋ ಅವರು ನನಗೆ ಅತ್ಯಂತ ಪ್ರಿಯರು.
\end{artha}

ಭಕ್ತರಿಗೆ ಭಗವಂತನಲ್ಲಿ ಶ್ರದ್ಧೆ ಇರಬೇಕು. ಶ್ರದ್ಧೆ ಎಂಬ ಪದ ಬಹಳ ಧ್ವನಿಪೂರ್ಣವಾದುದು. ಅಲ್ಲಿ ಪ್ರೀತಿ ಭಕ್ತಿ ಉತ್ಸಾಹ ಎಲ್ಲಾ ಇದೆ ಭಗವಂತನ ಮೇಲೆ. ಭಗವಂತನಲ್ಲೇ ಸಂಪೂರ್ಣ ನೆಲೆಸಿರುವವರೇ ಅವನಲ್ಲಿ ಪರಾಯಣರಾಗಿರುವವರು. ಯಾವಾಗಲೂ ಅವನನ್ನೇ ಕುರಿತು ಚಿಂತಿಸುವವರು, ಅವನಲ್ಲೇ ಮುಳುಗಿರುವವರು, ಅವನಲ್ಲಿಯೇ ಬಾಳುತ್ತಿರುವವರು. ಇದುವರೆಗೆ ಹೇಳಿದ ಒಳ್ಳೆಯ ಗುಣಗಳನ್ನೆಲ್ಲ ತಮ್ಮ ಮನಸ್ಸೆಂಬ ಹೂದೋಟದಲ್ಲಿ ಆರೈಕೆ ಮಾಡಿರುವವರು. ಭಗವಂತ ನಮ್ಮ ಹೊಗಳಿಕೆಯನ್ನಲ್ಲ ಮೆಚ್ಚುವುದು, ನಮ್ಮ ಜೀವನದಲ್ಲಿ ಕೃಷಿಮಾಡಿದ ಸದ್ಗುಣಗಳೆಂಬ ಪುಷ್ಪಗಳನ್ನು ಅವನು ಮೆಚ್ಚುವನು. ಭಕ್ತ ದ್ವಂದ್ವಾತೀತನಾಗಿರುವನು. ಬುದ್ಧಿ ಅನುಮಾನದಿಂದ ಪಾರಾಗಿರುವುದು. ಹೃದಯ ಭಗವಂತನ ಮೇಲಿನ ಪ್ರೇಮದಿಂದ ತುಂಬಿ ತುಳುಕುತ್ತಿರುವುದು. ಭಗವಂತನ ಬಳಿಗೆ ಹೋಗಲು ಇರುವ ಆತಂಕವನ್ನೆಲ್ಲ ಕಳಚಿಕೊಂಡು ಅವನೆಡೆಗೆ ಹೋಗಲು ತವಕಪಡುತ್ತಿರುವುದು. ಇಂತಹ ಗುಣಗಳನ್ನು ತನ್ನ ಜೀವನದಲ್ಲಿ ಬೆಳಗಿಸಲು ಶ್ರಮಿಸುತ್ತಿರುವ ಭಕ್ತರನ್ನು ಕಂಡರೆ ಭಗವಂತನಿಗೆ ಪ್ರೀತಿ.

