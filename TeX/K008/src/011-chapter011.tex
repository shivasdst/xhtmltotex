
\chapter{ವಿಶ್ವರೂಪದರ್ಶನಯೋಗ}

ಅರ್ಜುನ ಶ್ರೀಕೃಷ್ಣನಿಗೆ ಹೇಳುತ್ತಾನೆ:

\begin{verse}
ಮದನುಗ್ರಹಾಯ ಪರಮಂ ಗುಹ್ಯಮಧ್ಯಾತ್ಮಸಂಜ್ಞಿತಮ್ ।\\ಯತ್ತ್ವಯೋಕ್ತಂ ವಚಸ್ತೇನ ಮೋಹೋಽಯಂ ವಿಗತೋ ಮಮ \versenum{॥ ೧ ॥}
\end{verse}

{\small ನನ್ನ ಮೇಲೆ ಕೃಪೆ ಇಟ್ಟು, ಪರಮ ಗುಹ್ಯವೂ, ಆಧ್ಯಾತ್ಮವೆಂಬ ಹೆಸರುಳ್ಳದ್ದೂ ಆದ ಯಾವ ಬೋಧನೆಯನ್ನು ಮಾಡಿದೆಯೊ ಅದರಿಂದ ನನ್ನ ಈ ಮೋಹ ನಾಶವಾಯಿತು.}

ಅರ್ಜುನ ಶ್ರೀಕೃಷ್ಣನಿಗೆ ನನ್ನ ಮೇಲೆ ಕೃಪೆ ಇಟ್ಟು ಎನ್ನುತ್ತಾನೆ. ಭಗವಂತನ ಕೃಪೆ ನಮ್ಮ ಮೇಲೆ ಬರಬೇಕಾದರೆ ನಾವು ಅವನ ಶಿಷ್ಯರಾಗಬೇಕು. ಆಗಲೆ ಅವನ ಕೃಪೆ ನಮಗೆ ಇಳಿದು ಬರುವುದು. ನಾವು ಅವನಲ್ಲಿ ಶರಣಾಗತರಾಗಿ ದೇವರೇ ನನ್ನನ್ನು ಉದ್ಧಾರ ಮಾಡು ಎಂದು ಕೇಳಿಕೊಳ್ಳಬೇಕು. ಸಾಧಾರಣಮನುಷ್ಯನನ್ನು ಮೊರೆಹೊಕ್ಕರೆ ಸಾಕು, ಅವನು ತನ್ನ ಕೈಲಾದುದನ್ನೆಲ್ಲಾ ನಮ್ಮ ಸಹಾಯಕ್ಕೆ ಮಾಡುತ್ತಾನೆ. ಯಾವಾಗ ದೇವರನ್ನು ಮೊರೆ ಹೋಗುವೆವೊ ಅವನು ನಮ್ಮ ಉದ್ಧಾರಕ್ಕೆ ಸರ್ವವನ್ನೂ ಮಾಡುತ್ತಾನೆ. ಶ್ರೀರಾಮಕೃಷ್ಣರು ಹೇಳುತ್ತಿದ್ದರು, ಭಗವಂತನ ಕೃಪೆ ಎಂಬ ತಂಗಾಳಿ ಯಾವಾಗಲೂ ಬೀಸುತ್ತಿದೆ. ಅದರಿಂದ ನಾವು ಪ್ರಯೋಜನ ಪಡೆದುಕೊಳ್ಳಬೇಕಾದರೆ ನಾವು ನಮ್ಮ ಜೀವನದ ದೋಣಿಯ ಧ್ವಜಪಟವನ್ನು ಬಿಚ್ಚಿ ಬೀಸುವ ಗಾಳಿಯ ಸಹಾಯವನ್ನು ಪಡೆದುಕೊಳ್ಳಬೇಕು. ಆಗ ಅದು ನಮ್ಮನ್ನು ಗುರಿ ಎಡೆಗೆ ತಳ್ಳುವುದು. ನಮ್ಮನ್ನು ಅವನಿಗೆ ಅರ್ಪಿಸಿಕೊಳ್ಳಬೇಕು. ಆಗ ಗೊತ್ತಾಗುವುದು ಅವನು ಹೇಗೆ ಕೆಲಸ ಮಾಡುತ್ತಿರುವನು ಎಂಬುದು.

ನೀನು ಹೇಳಿರುವುದು ಪರಮ ಗುಹ್ಯವಾಗಿರುವುದು. ಪರಮಾತ್ಮನ ರಹಸ್ಯವನ್ನು ನಾವು ಎಲ್ಲೊ ಹೊರಗೆ ಹುಡುಕಬೇಕಾಗಿಲ್ಲ. ಇದು ಹತ್ತಿರಕ್ಕೆ ಹತ್ತಿರ. ಇದು ನನ್ನ ಅಂತರಾಳದಲ್ಲೆ ಇದೆ. ಆದರೆ ನನ್ನ ಅಜ್ಞಾನದ ಪಾಚಿಯಿಂದ ಅದು ಕಾಣೆಯಾಗಿದೆ. ಯಾವಾಗ ಅಜ್ಞಾನವನ್ನು ಆಚೆಗೊಯ್ಯುವೆವೋ ಆಗ ಹಿಂದಿರುವ ಜ್ಞಾನಜಲ ಕಾಣುವುದು. ಯಾವುದೊ ಒಂದು ಬೆಲೆಬಾಳುವ ವಸ್ತುವನ್ನು ಒಂದು ಕಡೆ ಇಟ್ಟು ಬೀಗಹಾಕಿ ಬೀಗದ ಕೈಯನ್ನು ಇಟ್ಟ ಸ್ಥಳವನ್ನು ಮರೆತಿರುವೆವು. ಅದನ್ನು ನಾವೇ ಇಟ್ಟವರು. ನಮಗೇ ಮರೆತು ಹೋಗಿದೆ ಎಲ್ಲಿಟ್ಟೆವು ಎಂದು. ಅದರಂತೆಯೇ ಪರಮಾತ್ಮವಸ್ತು. ನಮ್ಮಲ್ಲಿಯೇ ಅದು ಇದೆ. ನಾನು ಎಂದು ಹೇಳುವ ಅಹಂಕಾರದ ಹಿಂದೆಯೇ ಅದು ಅವಿತಿದೆ. ಶ್ರೀಕೃಷ್ಣ ಹಿಂದಿನ ಅಧ್ಯಾಯದಲ್ಲಿ ಪಾಂಡವರಲ್ಲಿ ತಾನು ಧನಂಜಯ ಎನ್ನುತ್ತಾನೆ. ಧನಂಜಯ ಎಂದು ತೋರುತ್ತಿರುವ ವ್ಯಕ್ತಿಯ ಹಿಂದೆ ನಿಜವಾಗಿರುವುದು ಪರಮಾತ್ಮನೆ. ಆದರೆ ಅಜ್ಞಾನದ ನಾನು ಎಂಬುದು ಅವನನ್ನು ಕಾಣದಂತೆ ಮಾಡಿ ತಾನೆ ತುಂಬ ದೊಡ್ಡವನು ಎಂದು ಮೆರೆಸುತ್ತಿದೆ. ಇದು ಕಸ್ತೂರಿ ಮೃಗದ ನಾಭಿಯಿಂದ ಬರುವ ಕಸ್ತೂರಿಯಂತೆ. ಮೊದಲು ಅದು ಪರಿಮಳವನ್ನು ಹುಡುಕಿಕೊಂಡು ಎಲ್ಲೆಲ್ಲೋ ಅಲೆಯುವುದು. ಕೊನೆಗೆ ಎಲ್ಲಿಯೂ ಸಿಕ್ಕದೆ ಸಾಕಾಗಿ ಮಲಗಿದಾಗ ಅದು ತನ್ನ ನಾಭಿಯಿಂದಲೇ ಬರುತ್ತಿದೆ ಎಂಬುದು ಹೊಳೆಯುವುದು.

ಇಲ್ಲಿ ಆಧ್ಯಾತ್ಮಿಕ ವಿದ್ಯೆಯ ರಹಸ್ಯವನ್ನು ಹೇಳುತ್ತಾನೆ. ಪರಮಾತ್ಮ ನಮ್ಮ ಅಂತರಾಳದಲ್ಲಿ ಹೇಗೆ ಬೆಳಗುತ್ತಿರುವನೋ ಹಾಗೆಯೇ ಪ್ರಪಂಚದ ಸರ್ವ ವಸ್ತುಗಳಲ್ಲಿ ಇರುವನು. ಕೆಲವು ಕಡೆ ಅವನು ತನ್ನ ವಿಭೂತಿಯಿಂದ ಚೆನ್ನಾಗಿ ಕಾಣುತ್ತಿರುವನು,ಮತ್ತೆ ಕೆಲವು ಕಡೆ ಮೊಬ್ಬು ಮೊಬ್ಬಾಗಿ ಕಾಣುತ್ತಿರು ವನು. ಹೊರಗೆ ಅಷ್ಟೊಂದು ಅಜ್ಞಾನ ಮುಚ್ಚಿಕೊಂಡಿರುವುದರಿಂದ ಅವನು ಕಾಣುತ್ತಿಲ್ಲ. ಕಾಣದೆ ಇದ್ದರೂ ಅವನು ಇಲ್ಲದೆ ಇಲ್ಲ. ಅವನು ಸರ್ವಾಂತರ್ಯಾಮಿ. ವಿದ್ಯೆಗಳಲ್ಲಿ ಅಧ್ಯಾತ್ಮವಿದ್ಯೆ ಶ್ರೇಷ್ಠ ವಾದದು. ಅಂತಹ ಶ್ರೇಷ್ಠವಾದುದನ್ನು ಅರ್ಜುನನ ಮೇಲೆ ಕೃಪೆಯಿಟ್ಟು ಹೇಳಿರುವನು. ಈ ಪ್ರಪಂಚ ಅವನಿಂದಲೇ ಬಂದಿದೆ, ಅವನಲ್ಲಿಯೇ ಇದೆ. ಆದರೂ ನಮಗೆ ಅವನಷ್ಟು ಅಜ್ಞಾತವಾಗಿರುವುದು ಯಾವುದೂ ಇಲ್ಲ. ಪರಮಾತ್ಮನ ಮೇಲೆ ಪರಮಾತ್ಮನಲ್ಲದ ಕಸದಿಂದ ಅವನನ್ನು ಕಾಣದಂತೆ ಮಾಡಿಕೊಂಡಿರುವವರು ನಾವೆ. ನಮ್ಮ ಹಿಂದೆ ಅವನಿರುವನು. ಆದರೆ ನಾವು ಅವನ ಮೇಲೆ ದೇಹ ಮನೋಬುದ್ಧಿ ಅಹಂಕಾರಗಳನ್ನು ಹಾಕಿ ಕಾಣದಂತೆ ಮಾಡಿರುವೆವು. ಕೆಲವು ವೇಳೆ ಎಷ್ಟು ಬರದಂತೆ ಮಾಡಿದರೂ ಬಿಡದೆ ಕಾಣುವ ವಿಭೂತಿಯನ್ನು ಭಗವಂತನದು ಎಂದು ತಿಳಿಯದೆ ಬರೀ ಆಯಾ ವಸ್ತುಗಳಿಗೆ ಮತ್ತು ವ್ಯಕ್ತಿಗಳಿಗೆ ಮಾತ್ರ ಗಮನ ಕೊಡುತ್ತೇವೆ. ಈ ಶಕ್ತಿ ಯಾರದು, ಎಲ್ಲಿಂದ ಬರುತ್ತಿದೆ ಎಂದು ತಿಳಿದುಕೊಳ್ಳಲು ಯತ್ನಿಸಿದರೆ ಅದು ನಮ್ಮನ್ನು ಭಗವಂತನ ಮೂಲಕ್ಕೆ ಒಯ್ಯುವುದು.

ನನ್ನ ಮೋಹ ನಾಶವಾಯಿತು ಎನ್ನುತ್ತಾನೆ. ಇವನ ಮೋಹ ಯಾವುದು? ನಾನು ಅರ್ಜುನ, ಈ ಯುದ್ಧದಲ್ಲಿ ಗುರುಹಿರಿಯರನ್ನು ರಾಜ್ಯದಾಸೆಗೆ ಕೊಲ್ಲುತ್ತಿರುವೆನು, ಇದೊಂದು ಮಹಾಪಾಪ ಎಂದು ಭಾವಿಸಿದ್ದ. ಆದರೆ ಈಗ ಅವನಿಗೆ ಗೊತ್ತಾಗುತ್ತಿದೆ. ಅರ್ಜುನನ ಹಿಂದೆ ನಿಜವಾಗಿ ನಿಂತು ಕೆಲಸ ಮಾಡುತ್ತಿರುವವನು ಭಗವಂತನೇ. ಇವನೊಂದು ನಿಮಿತ್ತ. ಪಾಪಪುಣ್ಯಗಳಾವುವೂ ಇಲ್ಲ ಇವನಿಗೆ. ಭಗವಂತ ತನ್ನ ಕೆಲಸವನ್ನು ಮಾಡುವುದಕ್ಕೆ ಕೆಲವು ವ್ಯಕ್ತಿಗಳನ್ನು ಆರಿಸಿಕೊಂಡಾಗ ನಾವು ಅವನ ನಿಮಿತ್ತ ಆಗುವೆವು. ಒಂದು ರೇಡಿಯೋ ಮೂಲಕ ಚೆನ್ನಾಗಿ ಹಾಡು ಬರುತ್ತಿದೆ. ಆ ಹಾಡನ್ನು ನಾವು ಮೆಚ್ಚುತ್ತೇವೆ. ರೇಡಿಯೋವನ್ನು ಹೊಗಳುವುದಿಲ್ಲ. ಇದರ ಮೂಲಕ ಅದು ಬರುತ್ತಿದೆಯೆ ಹೊರತು, ಇದೇ ಅದನ್ನು ಹಾಡುತ್ತಿಲ್ಲ. ಹಾಗೆಯೆ ಭಗವಂತನ ಶಕ್ತಿ ಕೆಲಸ ಮಾಡುವುದು ಕೆಲವು ವ್ಯಕ್ತಿಗಳ ಮೂಲಕ. ಸಾಧಾರಣ ಜನ ಆ ವ್ಯಕ್ತಿಯನ್ನು ನೋಡುತ್ತಾರೆ, ಆ ಮನುಷ್ಯ ಮಾಡುವ ಕೆಲಸವನ್ನು ಕೊಂಡಾಡುತ್ತಾರೆ. ಆದರೆ ಸತ್ಯ ಬೇರೆ. ಹೊಗಳಿಕೆಗೆ ಇವನು ಪಾತ್ರನಲ್ಲ. ಭಗವಂತನೊಬ್ಬನೇ ಹೊಗಳಿಕೆಗೆ ಪಾತ್ರ. ಇವನು ಇಲ್ಲದೆ ಇದ್ದರೆ ಭಗವಂತ ಮತ್ತೆ ಯಾರನ್ನೊ ತೆಗೆದುಕೊಂಡು ತನ್ನ ಕೆಲಸವನ್ನು ಮಾಡುತ್ತಿದ್ದ. ಅವನ ಕೆಲಸ ಯಾವುದೊ ಒಂದು ವ್ಯಕ್ತಿಯ ಮೇಲೆ ನಿಂತಿಲ್ಲ. ಯಾವಾಗ ನಮ್ಮನ್ನು ನಿಮಿತ್ತವಾಗಿ ಆರಿಸಿಕೊಳ್ಳುವನೊ ಆಗ ಧನ್ಯರಾಗುವವರು ನಾವು. ಅವನು ನಮ್ಮ ಮೇಲಿನ ಕೃಪೆಯಿಂದ ಇದನ್ನು ಉಪಯೋಗಿಸುತ್ತಿರುವನು.

ದೇವರು ತನ್ನ ಕೆಲಸಕ್ಕೆ ವ್ಯಕ್ತಿಗಳನ್ನು ಆರಿಸಿಕೊಳ್ಳುತ್ತಾನೆ. ಎಲ್ಲಿ ಅವನ ಶಕ್ತಿ ಯಾವ ತಡೆಯೂ ಇಲ್ಲದೆ ಹರಿಯುವುದೊ ಅಂತಹ ವಸ್ತುವನ್ನು ಆರಿಸಿಕೊಳ್ಳುತ್ತಾನೆ. ಕೊಳಲೂದುವವನು ಗಟ್ಟಿಯ ಬಿದಿರನ್ನು ತೆಗೆದುಕೊಳ್ಳುವುದಿಲ್ಲ. ಟೊಳ್ಳಾದ ಬಿದಿರನ್ನು ತೆಗೆದುಕೊಂಡು ಕೊಳಲು ಮಾಡುತ್ತಾನೆ. ಅದರ ಮೂಲಕ ತನ್ನ ಗಾನವನ್ನು ಅವನು ಹೊರಸೂಸುವನು. ಕೊಳಲಿಗಲ್ಲ ಆ ಕೀರ್ತಿ. ಯಾರು ಅದನ್ನು ಊದುತ್ತಿರುವನೋ ಅವನಿಗೆ ಸಲ್ಲುವುದು.

\begin{verse}
ಭವಾಪ್ಯಯೌ ಹಿ ಭೂತಾನಾಂ ಶ್ರುತೌ ವಿಸ್ತರಶೋ ಮಯಾ ।\\ತ್ವತ್ತಃ ಕಮಲಪತ್ರಾಕ್ಷ ಮಾಹಾತ್ಮ್ಯಮಪಿ ಚಾವ್ಯಯಮ್ \versenum{॥ ೨ ॥}
\end{verse}

{\small ಶ್ರೀಕೃಷ್ಣ, ಪ್ರಾಣಿಗಳ ಉತ್ಪತ್ತಿ ಸ್ಥಿತಿ ನಾಶದ ವಿಚಾರವಾಗಿ ನಿನ್ನಿಂದ ಕೇಳಿದೆ. ಅದೇ ರೀತಿ ನಿನ್ನ ಅವಿನಾಶಿಯಾದ ಮಹಾತ್ಮ್ಯವನ್ನು ಕೂಡ ವಿಸ್ತಾರವಾಗಿ ಕೇಳಿದೆನು.}

ಅರ್ಜುನ ಶ್ರೀಕೃಷ್ಣನಿಂದ ಪ್ರಾಣಿಗಳಿಗೆಲ್ಲ ಅವನೇ ಬೀಜಸ್ವರೂಪನಾಗಿರುವನು, ಎಲ್ಲ ಅವ ನಿಂದಲೇ ಬಂದಿದೆ ಎಂಬುದನ್ನು ಕೇಳಿದ. ಅವು ಪ್ರಪಂಚಕ್ಕೆ ಬಂದ ಮೇಲೆ ಅವುಗಳನ್ನೆಲ್ಲ ರಕ್ಷಿಸುವವನು ಅವನೇ, ಮತ್ತು ಅಂತ್ಯಕಾಲದಲ್ಲಿ ಕೊಂಡೊಯ್ಯುವವನೂ ಅವನೇ ಎಂಬುದನ್ನು ಕೇಳಿದ. ಅವನ ಮಹಾತ್ಮ್ಯೆ ಅವ್ಯಯವಾದುದು. ಎಂದಿಗೂ ನಾಶವಾಗುವುದಿಲ್ಲ. ಯಾವುದರ ಮೂಲಕ ಅದು ವ್ಯಕ್ತವಾಗುವುದೋ ಅದು ನಾಶವಾಗಬಹುದು. ಆದರೆ ಅವನ ಮಹಾತ್ಮ್ಯೆಗೆ ನಾಶವಿಲ್ಲ. ಅವನು ಬೇರೊಂದು ವಸ್ತು ಮತ್ತು ವ್ಯಕ್ತಿಗಳ ಮೂಲಕ ವ್ಯಕ್ತವಾಗುವನು. ವಸ್ತು ವ್ಯಕ್ತಿಗಳೆಲ್ಲ ನಿಮಿತ್ತಗಳಷ್ಟೆ. ಒಂದು ನಿಮಿತ್ತ ಹೋದರೆ ಮತ್ತೊಂದು ನಿಮಿತ್ತವನ್ನು ತೆಗೆದುಕೊಳ್ಳುವನು. ಅವನಿಗೆ ನಿಮಿತ್ತಕ್ಕೇನು ದಾರಿದ್ರ್ಯವಿಲ್ಲ.

\begin{verse}
ಏವಮೇತದ್ಯಥಾತ್ಥ ತ್ವಮಾತ್ಮಾನಂ ಪರಮೇಶ್ವರ ।\\ದ್ರಷ್ಟುಮಿಚ್ಛಾಮಿ ತೇ ರೂಪಮೈಶ್ವರಂ ಪುರುಷೋತ್ತಮ \versenum{॥ ೩ ॥}
\end{verse}

{\small ಪರಮೇಶ್ವರ, ನೀನು ನಿನ್ನನ್ನು ಯಾವ ರೀತಿ ಹೇಳಿರುವೆಯೊ ಅದು ಹಾಗೆಯೇ ಸರಿ. ಆದರೂ ನಿನ್ನ ವಿಶ್ವರೂಪವನ್ನು ನೋಡಲು ಇಚ್ಛಿಸುತ್ತೇನೆ.}

ಅರ್ಜುನ ಶ್ರೀಕೃಷ್ಣನಿಗೆ, ನಿನ್ನನ್ನು ಯಾವ ರೀತಿ ವಿವರಿಸಿಕೊಂಡಿರುವೆಯೋ ಅದು ಹಾಗೆಯೆ ಸರಿ ಎನ್ನುತ್ತಾನೆ. ಇನ್ನು ಶ್ರೀಕೃಷ್ಣನನ್ನು ಸಾಧಾರಣ ವ್ಯಕ್ತಿಯಂತೆ ನೋಡುವುದಿಲ್ಲ. ಶ್ರೀಕೃಷ್ಣನಿಗೆ ಎರಡು ವ್ಯಕ್ತಿತ್ವಗಳಿವೆ. ಒಂದು ತಾತ್ಕಾಲಿಕವಾಗಿ ಅರ್ಜುನನಿಗೆ ಸಾರಥಿಯಾಗಿರುವುದು. ಮತ್ತೊಂದು ಅವನ ವಿಭುರೂಪ. ಆ ರೂಪದಲ್ಲಿ ಈ ಬ್ರಹ್ಮಾಂಡವನ್ನೆಲ್ಲ ಸೃಷ್ಟಿಸಿ ಪಾಲಿಸುತ್ತಿರುವನು ಮತ್ತು ಕೊನೆಗೆ ತನ್ನಲ್ಲಿಯೇ ಸೆಳೆದುಕೊಳ್ಳುವನು. ಈ ಸೃಷ್ಟಿಯಲ್ಲೆಲ್ಲಾ ಓತಪ್ರೋತನಾಗಿರುವನು. ಅವನಿಲ್ಲದ ಎಡೆಯಿಲ್ಲ. ಅವನ ಜ್ಞಾನಕ್ಕೆ, ಶಕ್ತಿಗೆ, ಪವಿತ್ರತೆಗೆ ಒಂದು ಅಂತ್ಯವಿಲ್ಲ. ಇವುಗಳನ್ನೆಲ್ಲ ಸ್ವಲ್ಪವೂ ಅನುಮಾನಿಸದೆ ಅರ್ಜುನ ಒಪ್ಪಿಕೊಳ್ಳುತ್ತಾನೆ. ಇನ್ನು ಮೇಲೆ ಅವನಿಗೆ ಶ್ರೀಕೃಷ್ಣ ಸಖನಲ್ಲ, ಸರಿಸಮಾನನಲ್ಲ. ಅವನು ಸಾಕ್ಷಾತ್ ಪರಮೇಶ್ವರ. ಇವನಾದರೋ ಅವನ ಶಿಷ್ಯ, ಅವನ ಭಕ್ತ. ಶ್ರೀಕೃಷ್ಣನ ಬೋಧನೆ ಅರ್ಜುನನ ಮನಸ್ಸಿನ ಮೇಲೆ ಅದ್ಭುತವಾದ ಪರಿಣಾಮವನ್ನು ಉಂಟು ಮಾಡಿದೆ. ಆದರೂ ನಿನ್ನ ಈಶ್ವರೀರೂಪವನ್ನು ನೋಡಬೇಕೆಂದು ಬಯಸುವೆನು ಎನ್ನುತ್ತಾನೆ. ಈಶ್ವರೀರೂಪ ಎಂದರೆ ಅವನು ಪ್ರಪಂಚವನ್ನೆಲ್ಲ ಹೇಗೆ ನಡೆಸುತ್ತಿರುವನು ಮತ್ತು ಧ್ವಂಸ ಮಾಡುತ್ತಿರುವನು ಎಂಬುದನ್ನು ನೋಡಬಯಸವನು. ಸಾಮಾನ್ಯ ರೂಪಿನಲ್ಲಿಯೂ ಪರಮಾತ್ಮ ನಿದ್ದಾನೆ. ಆದರೆ ಅವನ ಈಶ್ವರೀರೂಪು ಆಕರ್ಷಕವಾಗಿದೆ. ಮಿಲಿಟರಿ ಕಮ್ಯಾಂಡರ್ ಚೀಫ್ ಮನೆಯಲ್ಲಿ ಮಕ್ಕಳೊಡನೆ ಇರುವಾಗ ಅವರೊಡನೆ ಆಡುತ್ತಿರುವನು. ಅವನಿಗೆ ಮಿಲಿಟರಿಯ ಯಾವ ಯೂನಿಫಾರಂ ಇಲ್ಲ. ಸಾದಾ ಡ್ರೆಸ್ಸಿನಲ್ಲಿ ಇರುವನು. ಆದರೆ ಅವನೆ ಒಂದು ಮಿಲಿಟರಿ ಪೆರೇಡಿನಲ್ಲಿ ಹೋಗುವಾಗ, ತನ್ನ ಯೂನಿಫಾರಂ ಧರಿಸಿ, ತನಗೆ ಬಂದ ಮೆಡಲುಗಳನ್ನೆಲ್ಲ ಹಾಕಿಕೊಂಡು ಬಹಳ ಠೀವಿಯಿಂದ ಹೋಗುವನು. ಪ್ರತಿಯೊಬ್ಬ ಸಿಪಾಯಿಯೂ ಅವನನ್ನು ಅಷ್ಟು ಗೌರವದಿಂದ ಕಾಣು ವನು. ಅವನ ಒಂದು ಮಾತು, ಒಂದು ಸನ್ನೆ ಇತರರಿಗೆ ಆಜ್ಞೆ, ಮತ್ತು ಶಾಸನ. ಕೆಲವು ವೇಳೆ ಮಕ್ಕಳೂ ಕೂಡ ಈ ಡ್ರೆಸ್ಸಿನಲ್ಲಿರುವ ತಮ್ಮ ಅಪ್ಪನ್ನನ್ನು ನೋಡಲು ಹೋಗುವರು. ಅಲ್ಲಿ ಅವನ ಠೀವಿಯನ್ನು ನೋಡಿ ಬೆರಗಾಗುವರು. ಅರ್ಜುನ ಕೂಡ ಅದರಂತೆಯೇ ಅವನ ಐಶ್ವರ್ಯರೂಪವನ್ನು ನೋಡ ಬಯಸುವನು.

\begin{verse}
ಮನ್ಯಸೇ ಯದಿ ತಚ್ಛಕ್ಯಂ ಮಯಾ ದ್ರಷ್ಟುಮಿತಿ ಪ್ರಭೋ ।\\ಯೋಗೇಶ್ವರ ತತೋ ಮೇ ತ್ವಂ ದರ್ಶಯಾತ್ಮಾನಮವ್ಯಯಮ್ \versenum{॥ ೪ ॥}
\end{verse}

{\small ಪ್ರಭು! ಅದನ್ನು ನಾನು ನೋಡಲು ಸಾಧ್ಯ ಎಂದು ನೀನು ಭಾವಿಸಿದರೆ, ಯೋಗೇಶ್ವರ, ಅಕ್ಷಯವಾದ ನಿನ್ನ ರೂಪವನ್ನು ತೋರಿಸು.}

ಅರ್ಜುನ ಶ್ರೀಕೃಷ್ಣನನ್ನು ಕೇಳಿಕೊಳ್ಳುತ್ತಿದ್ದಾನೆ ನನಗೆ ಅದನ್ನು ನೋಡಲು ಸಾಧ್ಯವಾದರೆ ತೋರಿಸು ಎಂದು. ಇವನಿಗೇನೊ ನೋಡಬೇಕೆಂಬ ಆಸೆ ಇದೆ. ಆದರೆ ಜೀವನದಲ್ಲಿ ಒಂದು ವಸ್ತುವನ್ನು ಪಡೆಯಬೇಕಾದರೆ ಆಸೆಯೊಂದೆ ಸಾಲದು. ಅದಕ್ಕೆ ಯೋಗ್ಯತೆ ಕೂಡ ಇರಬೇಕು. ಎಷ್ಟು ಜನ ಹುಡುಗರಿಗೆ ಪರೀಕ್ಷೆಯಲ್ಲಿ ರ್ಯಾಂಕ್ ಬರಬೇಕೆಂದು ಆಸೆ ಇಲ್ಲ. ಆದರೆ ಎಲ್ಲರಿಗೂ ಆ ಯೋಗ್ಯತೆ ಇದೆಯೆ? ಆದಕಾರಣವೆ ಅರ್ಜುನನಿಗೆ ಆಸೆ ಇದ್ದರೂ ಅದಕ್ಕೆ ತಾನು ಯೋಗ್ಯನೊ ಅಲ್ಲವೊ ಎಂಬುದು ಗೊತ್ತಿಲ್ಲ. ಅದನ್ನು ಅವನು ತಿಳಿದುಕೊಳ್ಳಲಾರ. ಅದನ್ನು ಗುರು ಮಾತ್ರ ತಿಳಿದುಕೊಳ್ಳ ಬಲ್ಲ. ಶಿಷ್ಯನ ಯೋಗ್ಯತೆ ಏನು, ಇವನೆಷ್ಟನ್ನು ಅರಗಿಸಿಕೊಳ್ಳುತ್ತಾನೆ ಎಂಬುದನ್ನು ಅರಿತು ಅದಕ್ಕೆ ತಕ್ಕಂತೆ ಬೋಧನೆಯನ್ನು ಮಾಡುತ್ತಾನೆ. ಮಕ್ಕಳು ಏನೇನೊ ಬೇಕಾದಷ್ಟು ತಿನ್ನಬೇಕೆಂದು ಬಯಸು ವುವು. ಆದರೆ ಮನೆಯಲ್ಲಿರುವ ತಾಯಿಗೆ ಈ ಮಕ್ಕಳ ಜೀರ್ಣಶಕ್ತಿ ಎಷ್ಟುಮಟ್ಟಿನದು ಎಂಬುದು ಚೆನ್ನಾಗಿ ಗೊತ್ತಿದೆ. ಅದಕ್ಕೆ ತಕ್ಕಂತೆ ಅವರಿಗೆ ಬಡಿಸುವಳೆ ಹೊರತು ಅವರು ಕೇಳಿದ್ದನ್ನೆಲ್ಲ ಕೊಡುವುದಿಲ್ಲ. ಇಲ್ಲಿಯೂ ಕೂಡ ಅರ್ಜುನ ಶ್ರೀಕೃಷ್ಣನಿಗೆ ನೀನು ಈಶ್ವರೀರೂಪವನ್ನು ನನಗೆ ತೋರಿಸಲೇಬೇಕು ಎಂದು ಕೇಳುವುದಿಲ್ಲ. ನಾನು ಅದನ್ನು ನೋಡಲು ಶಕ್ತನಾಗಿದ್ದೇನೆ ಎಂದು ನೀನು ಭಾವಿಸಿದರೆ ತೋರು ಎನ್ನುವನು. ತೋರಿಸುವುದು ಬಿಡುವುದು ಅದನ್ನೆಲ್ಲಾ ದೇವರ ಪಾಲಿಗೆ ಬಿಡುವನು. ಅನೇಕರು ದೇವರನ್ನು ನೋಡಬೇಕೆಂದು ಆಶಿಸುವರು. ಅವನು ಸಾಮಾನ್ಯ ರೂಪಿನಲ್ಲಿ ಬಂದು ನಿಂತರೆ ಅವನನ್ನು ದೇವರು ಎಂದು ನಾವು ನಂಬುವುದೇ ಇಲ್ಲ. ಅವನೇನಾದರೂ ವಿಶೇಷ ರೂಪಿನಂತೆ ಪ್ರಕಟವಾದರೆ ನಾವು ಅದನ್ನು ತಾಳಲಾರದೆ ಮೂರ್ಛೆ ಬೀಳುತ್ತೇವೆ. ನರೇಂದ್ರ ಶ್ರೀರಾಮಕೃಷ್ಣರ ಬಳಿಗೆ ಬಂದು ದೇವರಿದ್ದಾನೆಯೆ, ಅವನನ್ನು ನೀವು ತೋರಿಸುವಿರಾ ಎಂದು ಕೇಳಿದಾಗ ಶ್ರೀರಾಮಕೃಷ್ಣರು ಅವನನ್ನು ಸ್ಪರ್ಶಿಸಿದರು. ಅವನ ಎದುರಿಗೆ ಇರುವ ವಸ್ತು ಮಾಯ ವಾಗತೊಡಗಿತು. ಅವನ ವ್ಯಕ್ತಿತ್ವವೇ ಕರಗತೊಡಗಿತು. ಆಗ ಅವನು ಅರಚಿಕೊಳ್ಳುತ್ತಾನೆ. ಅಯ್ಯೊ ನೀವು ಏನು ಮಾಡುತ್ತಿರುವಿರಿ, ನನಗೆ ಮನೆಯಲ್ಲಿ ಅಪ್ಪ ಇದ್ದಾರೆ, ಅಮ್ಮ ಇದ್ದಾರೆ ಎಂದು. ಶ್ರೀರಾಮಕೃಷ್ಣರು ನಗುತ್ತ ಸಾಕು ಇವತ್ತಿಗೆ ಇಷ್ಟೆ ಎನ್ನುತ್ತಾರೆ. ನರೇಂದ್ರನು ಉತ್ಸಾಹದಿಂದ ಕೇಳುತ್ತಾನೆ. ಆದರೆ ಬರುವುದನ್ನು ಸ್ವೀಕರಿಸುವುದಕ್ಕೆ ಅರ್ಹತೆಯನ್ನು ಪಡೆದುಕೊಂಡಿದ್ದನೆ? ಇಲ್ಲ. ಅವನು ಇನ್ನೂ ಅದಕ್ಕೆ ಅಣಿಯಾಗಬೇಕಾಗಿತ್ತು. ಸಹಸ್ರಾರು ಕಿಲೋವಾಟ್ ವಿದ್ಯುತ್​ಶಕ್ತಿ ಒಂದು ಸಣ್ಣ ತಂತಿಯಲ್ಲಿ ಹರಿಯತೊಡಗಿದರೆ, ಆ ತಂತಿ ಕರಗಿಹೋಗುವುದು. ಹಾಗೆಯೇ ದೇವರನ್ನು ನೋಡಬೇಕಾದರೆ ನಮ್ಮನ್ನು ಅಣಿಮಾಡಿಕೊಳ್ಳಬೇಕು. ಆಗಲೇ ಶ್ರೀಕೃಷ್ಣ ಅರ್ಜುನನಿಗೆ ಹೇಳುತ್ತಾನೆ.

\begin{verse}
ಪಶ್ಯ ಮೇ ಪಾರ್ಥ ರೂಪಾಣಿ ಶತಶೋಽಥ ಸಹಸ್ರಶಃ ।\\ನಾನಾವಿಧಾನಿ ದಿವ್ಯಾನಿ ನಾನಾವರ್ಣಾಕೃತೀನಿ ಚ \versenum{॥ ೫ ॥}
\end{verse}

{\small ನಾನಾ ವಿಧವಾದ ದಿವ್ಯವಾದ ಅನೇಕ ವರ್ಣ ಮತ್ತು ಆಕೃತಿಗಳುಳ್ಳ ನನ್ನ ಅಸಂಖ್ಯಾತ ರೂಪಗಳನ್ನು ನೋಡು.}

ಶ್ರೀಕೃಷ್ಣ ಅರ್ಜುನನಿಗೆ ಮಾರ್ಗದರ್ಶಕನಂತೆ ಮುಂಚೆಯೇ ಅವನ ಮನಸ್ಸನ್ನು ಏನೇನನ್ನು ನೋಡುವನೊ ಅದಕ್ಕೆ ಅಣಿಮಾಡುವನು. ನಾನಾ ವಿಧವಾದ ರೂಪಗಳು ಇಲ್ಲಿವೆ. ಒಂದಲ್ಲ ಬಗೆಬಗೆಯ ರೂಪಗಳು. ಯಾವ ಯಾವ ರೂಪದಲ್ಲಿ ಈ ಪ್ರಪಂಚದಲ್ಲಿ ಕೆಲಸಮಾಡುತ್ತಿರುವನೊ ಅವನ್ನೆಲ್ಲಾ ನೋಡುತ್ತಾನೆ. ಈ ರೂಪಗಳೋ ದಿವ್ಯವಾದುವು ಸುಂದರವಾದುವುಗಳು ಮತ್ತು ಭವ್ಯವಾದುವುಗಳು. ಇಂತಹ ರೂಪವನ್ನು ಅವನು ಜಾಗ್ರತಾವಸ್ಥೆಯಲ್ಲಿ ಎಂದೂ ನೋಡಿರಲಿಕ್ಕಿಲ್ಲ. ಆ ರೂಪಗಳಾದರೂ ಅನೇಕ ಬಣ್ಣಗಳಿಂದ ಕೂಡಿವೆ. ಅದೊಂದು ಬಣ್ಣಬಣ್ಣದ ಸಂತೆ. ಅನೇಕ ಆಕೃತಿಗಳುಳ್ಳ ರೂಪ ಅದು. ಅದಕ್ಕೆ ಒಂದು ರೂಪವಲ್ಲ ಇರುವುದು. ಹಲವಾರು ರೂಪಗಳಿಂದ ಕೂಡಿದೆ. ಅವು ಅಸಂಖ್ಯಾತ. ಇವುಗಳನ್ನು ನೋಡುವಾಗ ನಮ್ಮ ವ್ಯಕ್ತಿತ್ವ ತತ್ತರಿಸುವುದು. ಇಂದ್ರಿಯಾ ತೀತ ಅನುಭವವನ್ನು ಇಂದ್ರಿಯಗಳ ಭಾಷೆಯ ಮೂಲಕ ವಿವರಿಸಬೇಕಾದರೆ ಕಷ್ಟ. ಅದಕ್ಕೆ ಹಲವು ಆಕಾರ ಬಣ್ಣ ಭಾವಗಳ ಸಹಾಯ ಬೇಕು. ಇದುವರೆಗೆ ಅರ್ಜುನನಿಗೆ ಇಂದ್ರಿಯಜನ್ಯ ಅನುಭವ ಮಾತ್ರ ಪರಿಚಯವಾಗಿತ್ತು. ಈಗ ಅವನು ಎಂದಿಗೂ ನೋಡದ ಕೇಳದ ಅನುಭವಿಸದ ಅನುಭವಕ್ಕೆ ಸಿದ್ಧನಾಗಬೇಕು. ಮನೆ ಹಿಂದಿನ ಕೊಳದಲ್ಲಿ ಈಜುತ್ತಿದ್ದವರನ್ನು ಭೋರ್ಗರೆದು ಮೊರೆಯುತ್ತಿರುವ ಪರ್ವತದಂತೆ ಎದ್ದು ಬೀಳುತ್ತಿರುವ ಅಸಂಖ್ಯಾತ ಅಲೆಗಳಿಂದ ಕೂಡಿದ ಅನಂತಸಾಗರದ ಮಧ್ಯದಲ್ಲಿ ಬಿಟ್ಟಂತಿದೆ.

\begin{verse}
ಪಶ್ಯಾದಿತ್ಯಾನ್ ವಸೂನ್ ರುದ್ರಾನಶ್ವಿನೌ ಮರುತಸ್ತಥಾ ।\\ಬಹೂನ್ಯದೃಷ್ಟಪೂರ್ವಾಣಿ ಪಶ್ಯಾಶ್ಚರ್ಯಾಣಿ ಭಾರತ \versenum{॥ ೬ ॥}
\end{verse}

{\small ಅರ್ಜುನ! ಆದಿತ್ಯರನ್ನು ವಸುಗಳನ್ನು ರುದ್ರರನ್ನು ಅಶ್ವಿನಿ ದೇವತೆಗಳನ್ನು ಮರುದ್ಗಣಗಳನ್ನು ಮತ್ತು ಹಿಂದೆ ಎಂದೂ ನೋಡದಿರುವ ಅನೇಕ ಆಶ್ಚರ್ಯಗಳನ್ನು ನೋಡು.}

ಶ್ರೀಕೃಷ್ಣನ ವ್ಯಕ್ತಿತ್ವವೇ ಒಂದು ಪರದೆ ಆಯಿತು. ಹೇಗೆ ಸಿನಿಮಾ ಪರದೆಯ ಮೇಲೆ ಬೀಳುವುದೊ ಹಾಗೆಯೆ ಶ್ರೀಕೃಷ್ಣನೆಂಬ ವ್ಯಕ್ತಿಯ ಪರದೆಯ ಮೇಲೆ ವಿಶ್ವರೂಪವೆಂಬ ದೃಶ್ಯ ಬೀಳಲು ಮೊದಲಾಗು ವುದು. ಅದು ಏನೇನು ಎಂಬುದು ಶ್ರೀಕೃಷ್ಣನೇ ಹಿಂದಿನಿಂದ ನಿಂತು ಅರ್ಜುನನಿಗೆ ವಿಸ್ತರಿಸುತ್ತಿರು ವನು. ಅಲ್ಲಿ ವಿಭೂತಿಯೋಗದಲ್ಲಿ ಅರ್ಜುನನಿಗೆ ಹೇಳಿದ ವ್ಯಕ್ತಿಗಳೆಲ್ಲರೂ ಇಲ್ಲಿ ಕಾಣಿಸಿಕೊಳ್ಳುವರು. ಆದಿತ್ಯರು, ರುದ್ರರು, ವಸುಗಳು ಅಶ್ವಿನೀ ದೇವತೆಗಳು ಮತ್ತು ಮರುದ್​ಗಣಗಳೆಲ್ಲರೂ ಇರುವರು. ಶ್ರೀಕೃಷ್ಣನೆಂಬ ಪರದೆಯಮೇಲೆ ಹಿಂದೆ ಅರ್ಜುನ ಎಂದೂ ನೋಡದ ಆಶ್ಚರ್ಯವನ್ನು ನೋಡುತ್ತಾನೆ. ಶ್ರೀಕೃಷ್ಣನೆಂಬ ವ್ಯಕ್ತಿಯನ್ನು ಅರ್ಜುನ ಒಬ್ಬ ಸಾಮಾನ್ಯ ವ್ಯಕ್ತಿ ಎಂದು ಭಾವಿಸಿದ್ದ. ಅವನು ಅರ್ಜುನನಿಗೆ ಸಖನಾಗಿದ್ದ. ಈಗ ತತ್ತ್ವಜ್ಞಾನಿ ಎಂದು ಗೊತ್ತಾಯಿತು ಮತ್ತು ಅರ್ಥವಾಯಿತು. ಒಂದು ಹೆಜ್ಜೆ ಮುಂದೆ ಹೋಗಿ ಅವನು ಅರ್ಜುನನಿಗೆ ಗುರು ಮಾತ್ರವಲ್ಲ, ಲೋಕಗುರುವಾದ ಭಗವಂತನೆ ಎಂಬುದು ಗೊತ್ತಾಗುವುದು. ಆ ಭಗವಂತ ಈಗ ವಿಶ್ವರೂಪದ ಪೋಷಾಕಿನಲ್ಲಿ ವ್ಯಕ್ತವಾಗುತ್ತಿರುವನು. ಇದು ಅವನ ಐಶ್ವರೀರೂಪ. ಶ್ರೀಕೃಷ್ಣ ಸಾಮಾನ್ಯವಾದ ತನ್ನ ವ್ಯಕ್ತಿತ್ವದ ಹಿಂದೆ ಇರುವ ವಿಶ್ವರೂಪವನ್ನು ಅರ್ಜುನನಿಗೆ ತೋರುವನು.

\begin{verse}
ಇಹೈಕಸ್ಥಂ ಜಗತ್ಕೃತ್ಸ್ನಂ ಪಶ್ಯಾದ್ಯ ಸಚರಾಚರಮ್ ।\\ಮಮ ದೇಹೇ ಗುಡಾಕೇಶ ಯಚ್ಚಾನ್ಯದ್ದ್ರಷ್ಟುಮಿಚ್ಛಸಿ \versenum{॥ ೭ ॥}
\end{verse}

{\small ಅರ್ಜುನ! ಇಲ್ಲಿ ನನ್ನ ದೇಹದ ಒಂದು ಕಡೆಯಲ್ಲಿ ಚರಾಚರವಾದ ಜಗತ್ತನ್ನು ನೋಡು ಮತ್ತು ಬೇರೆ ಯಾವುದನ್ನು ನೋಡಬೇಕೆಂದು ಇಚ್ಛಿಸುವೆಯೊ ಅದನ್ನು ನೋಡು.}

ಶ್ರೀಕೃಷ್ಣನ ದೇಹದಲ್ಲೆ ಬ್ರಹ್ಮಾಂಡವೆಲ್ಲ ಇದೆ. ಈ ಬ್ರಹ್ಮಾಂಡದಲ್ಲಿರುವ ಚರಾಚರ ವಸ್ತುಗಳೆಲ್ಲ ಅಲ್ಲಿವೆ. ಭೂತ ವರ್ತಮಾನ ಭವಿಷ್ಯತ್ ಎಲ್ಲವೂ ಅಲ್ಲಿದೆ. ಈಗ ಏನಾಗುತ್ತಿದೆ ಕುರುಕ್ಷೇತ್ರದಲ್ಲಿ ಅದನ್ನು ನೋಡಬೇಕಾದರೆ ನೋಡಬಹುದು. ಮುಂದೆ ಏನಾಗುವುದು ಅದನ್ನು ಬೇಕಾದರೆ ನೋಡ ಬಹುದು. ವಿಶ್ವರೂಪದಲ್ಲಿ ಬರುವ ಭಗವಂತನ ಸ್ವರೂಪ ಸುಂದರವಾದ ಮನೋಹರವಾದ ರೂಪವಲ್ಲ. ಇದನ್ನು ನೋಡತ್ತಿದ್ದರೆ ಇನ್ನೂ ನೋಡಬೇಕು ಎನಿಸುವುದಿಲ್ಲ. ನೋಡಿದ ತತ್​ಕ್ಷಣ ಕಣ್ಣುಮುಚ್ಚಿಕೊಳ್ಳುವಷ್ಟು ಕಾಂತಿಯಿಂದ ಕೂಡಿದೆ. ಭಗವಂತನ ಸಾಧಾರಣ ರೂಪ ಅರುಣ ಸಂಧ್ಯಾರಾಗದ ಮೋಡಗಳ ಹಿಂದೆ ಇರುವ ಸೂರ್ಯನಂತೆ ನೋಡಲು ಮನೋಹರವಾಗಿರುವುದು. ಎಷ್ಟುಹೊತ್ತು ಬೇಕಾದರೂ ಅದನ್ನು ನೋಡುತ್ತಿರಬಹುದು. ಆದರೆ ಮಧ್ಯಾಹ್ನದ ದಹಿಸುತ್ತಿರುವ ಸೂರ್ಯನನ್ನು ಒಂದು ಕ್ಷಣ ನೋಡುವುದಕ್ಕೆ ಕಷ್ಟವಾಗುವುದು. ವಿಶ್ವರೂಪ ಈ ಗುಂಪಿಗೆ ಸೇರಿದ್ದು. ಅದನ್ನು ನಮ್ಮ ಸಾಧಾರಣ ಕಣ್ಣಿನಿಂದ ನೋಡಲು ಸಾಧ್ಯವಿಲ್ಲ. ಅದಕ್ಕೆ ಬೇರೆ ಒಂದು ಕನ್ನಡಕ ಬೇಕಾಗುವುದು.

\begin{verse}
ನ ತು ಮಾಂ ಶಕ್ಯಸೇ ದ್ರಷ್ಟುಮನೇನೈವ ಸ್ವಚಕ್ಷುಷಾ ।\\ದಿವ್ಯಂ ದದಾಮಿ ತೇ ಚಕ್ಷುಃ ಪಶ್ಯ ಮೇ ಯೋಗಮೈಶ್ವರಮ್ \versenum{॥ ೮ ॥}
\end{verse}

{\small ಆದರೆ ನಿನ್ನ ಕಣ್ಣಿನಿಂದಲೇ ನನ್ನನ್ನು ನೋಡಲು ಶಕ್ತನಾಗಲಾರೆ. ನಿನಗೆ ದಿವ್ಯವಾದ ಚಕ್ಷುಸ್ಸನ್ನು ಕೊಡುತ್ತೇನೆ. ನನ್ನ ಈಶ್ವರೀ ರೂಪವನ್ನು ನೋಡು.}

ನಮ್ಮ ಕಣ್ಣು ಜಡ ಕಣ್ಣು. ಕಣ್ಣು ಮುಂದಿರುವ ಸಾಮಾನ್ಯವಾದುದನ್ನು ಮಾತ್ರ ತಿಳಿದುಕೊಳ್ಳ ಬಹುದು. ಕಣ್ಣಮುಂದೆ ಇದ್ದರೂ, ಅದು ತುಂಬಾ ಸೂಕ್ಷ್ಮವಾಗಿದ್ದರೆ, ಅಥವಾ ಅದು ತುಂಬಾ ದೂರದಲ್ಲಿದ್ದರೆ ಅದನ್ನು ಅರಿಯಲಾರರು. ಅದಕ್ಕೇ ಸೂಕ್ಷ್ಮ ವಸ್ತುವನ್ನು ಒಂದು ಮೈಕ್ರಾಸ್ಕೋಪಿನ ಕೆಳಗಡೆ ಇಡುವರು. ಆಗ ನಮಗೆ ಮುಂಚೆ ಕಾಣದ ಒಂದು ಹೊಸ ಪ್ರಪಂಚವೇ ತೋರುವುದು. ಅದರಂತೆಯೇ ದೂರದಲ್ಲಿರುವುದನ್ನು ನೋಡಬೇಕಾದರೆ ಟೆಲಿಸ್ಕೋಪನ್ನು ತೆಗೆದುಕೊಳ್ಳಬೇಕು. ಆ ಟೆಲಿಸ್ಕೋಪಿನ ಮೂಲಕ ಸೂರ್ಯ, ಚಂದ್ರ ಮುಂತಾದವುಗಳು ನಮ್ಮ ಹತ್ತಿರ ಇರುವಂತೆ ಭಾಸವಾಗುವುವು. ಅಲ್ಲಿ ಹಲವು ವಿವರಗಳನ್ನು ನೋಡುತ್ತೇವೆ. ಈಗ ಅರ್ಜುನನಿಗೆ ತೋರುತ್ತಿರುವ ದೃಶ್ಯ ಭಗವಂತನ ವಿಶ್ವರೂಪ. ಇದೊಂದು ಅತೀಂದ್ರಿಯದ ಅನುಭವ. ಇದನ್ನು ನೋಡಬೇಕಾದರೆ ಅತೀಂದ್ರಿಯ ನಯನಗಳೇ ಬೇಕಾಗುವುದು. ಯಾರು ತನ್ನ ವಿಶ್ವರೂಪವನ್ನು ತೋರುವನೋ, ಅವನೇ ಅರ್ಜುನನಿಗೆ ಅದನ್ನು ನೋಡುವುದಕ್ಕೆ ದಿವ್ಯ ಚಕ್ಷುವನ್ನೂ ಕೊಡುವನು. ನಮ್ಮಲ್ಲಿರುವುದನ್ನು ಮತ್ತೊಬ್ಬನಿಗೆ ತೋರಿಸುವುದೇ ಒಂದು ದೊಡ್ಡಸ್ತಿಕೆಯಲ್ಲ. ತೋರಿಸಿದ್ದನ್ನು ಸರಿಯಾಗಿ ತಿಳಿದು ಕೊಳ್ಳುವಂತೆ ಮಾಡಬೇಕು. ಅತೀಂದ್ರಿಯದ ಅನುಭವದ ತಾಪವನ್ನು ಸಹಿಸುವುದಕ್ಕೆ ವ್ಯಕ್ತಿಯನ್ನು ಅಣಿ ಮಾಡಬೇಕು. ಅರ್ಜುನನನ್ನು ಮೊದಲು ಅಣಿಮಾಡುವನು.

ಆಗಲೇ ಕುರುಕ್ಷೇತ್ರದಿಂದ ಬಹಳ ದೂರದಲ್ಲಿ ಅರಮನೆಯಲ್ಲಿ ಕುಳಿತಿರುವ ಅಂಧ ಧೃತರಾಷ್ಟ್ರನ ಹತ್ತಿರ ಇರುವ ಸಂಜಯನಿಗೂ ಈ ಅನುಭವ ಆಗುವುದು. ಅರ್ಜುನ ದಿವ್ಯಚಕ್ಷುಸ್ಸನ್ನು ಪಡೆದುಕೊಂಡ ಮೇಲೆ ಏನನ್ನು ನೋಡುವನೋ ಅದನ್ನು ಸಂಜಯನೂ ನೋಡುವನು. ವ್ಯಾಸರ ವರದ ಮೂಲಕ ಸಮರ ಕ್ಷೇತ್ರದಲ್ಲಿ ಹೊರಗೆ ಏನಾಗುತ್ತಿದೆ, ಮತ್ತು ಯಾರ ಮನಸ್ಸಿನಲ್ಲಿ ಏನಾಗುತ್ತಿದೆ ಅದನ್ನೆಲ್ಲ ನೋಡಬಹುದಾಗಿತ್ತು. ಅರ್ಜುನ ಏನನ್ನು ನೋಡುತ್ತಾನೆಯೋ ಅದು ಸಂಜಯನಿಗೂ ಕಾಣುವುದು. ಅದನ್ನು ಧೃತರಾಷ್ಟ್ರನಿಗೆ ವಿವರಿಸುವನು.

ಸಂಜಯ ಹೇಳುತ್ತಾನೆ:

\begin{verse}
ಏವಮುಕ್ತ್ವಾ ತತೋ ರಾಜನ್ ಮಹಾಯೋಗೇಶ್ವರೋ ಹರಿಃ ।\\ದರ್ಶಯಾಮಾಸ ಪಾರ್ಥಾಯ ಪರಮಂ ರೂಪಮೈಶ್ವರಮ್ \versenum{॥ ೯ ॥}
\end{verse}

{\small ರಾಜನೆ, ಮಹಾಯೋಗೇಶ್ವರನಾದ ಹರಿಯು ಹೀಗೆ ಹೇಳಿ ಅನಂತರ ಅರ್ಜುನನಿಗೆ ತನ್ನ ವಿಶ್ವರೂಪವನ್ನು ತೋರಿದನು.}

ಶ್ರೀಕೃಷ್ಣ ಯೋಗೇಶ್ವರ. ಏನನ್ನು ಬೇಕಾದರೂ ತಾನು ಅನುಭವಿಸಬಲ್ಲ ಮತ್ತು ಅದನ್ನು ಮತ್ತೊಬ್ಬ ಅನುಭವಿಸುವಂತೆ ಮಾಡಬಲ್ಲ ದಿವ್ಯಶಕ್ತಿ ಇತ್ತು. ಅವನು ಅರ್ಜುನನಿಗೆ ತನ್ನ ವಿಶ್ವರೂಪ ವನ್ನು ತೋರುತ್ತಾನೆ ಮತ್ತು ಅದನ್ನು ನೋಡುವುದಕ್ಕೆ ದಿವ್ಯ ಕಣ್ಣುಗಳನ್ನು ಕೂಡ ಕೊಡುತ್ತಾನೆ. ಮಹಾಭಾರತದಲ್ಲಿ ಇನ್ನು ಕೆಲವರಿಗೆ ಶ್ರೀಕೃಷ್ಣ ತನ್ನ ವಿಶ್ವರೂಪವನ್ನು ತೋರುತ್ತಾನೆ. ಉದ್ಯೋಗ ಪರ್ವದಲ್ಲಿ ದುರ್ಯೋಧನನಿಗೆ ಸಂಧಿ ಮಾಡಿಕೊ ಎಂದು ಒಡಂಬಡಿಸಲು ಬಂದಾಗ, ಅವನನ್ನು ಹಿಡಿದು ಹಗ್ಗದಿಂದ ಕಟ್ಟುವಂತೆ ಆಳುಗಳಿಗೆ ಆಜ್ಞಾಪಿಸಿದನು. ಆಗ ಅಲ್ಲಿರುವವರೆಲ್ಲ ಶ್ರೀಕೃಷ್ಣನಂತೆ ಕಾಣುತ್ತಾರೆ. ಯಾರನ್ನು ಕಟ್ಟುವುದು ಎಂದು ಕಕ್ಕಾಬಿಕ್ಕಿಯಾಗುತ್ತಾರೆ. ಭೀಷ್ಮನಂತಹ ಜ್ಞಾನಿಗಳು ಭಗವಂತನ ವಿಶ್ವರೂಪಕ್ಕೆ ಮಣಿದರು. ಆದರೆ ಭಂಡನಾದ ದುರ್ಯೋಧನನಾದರೋ ಶ್ರೀಕೃಷ್ಣ ಒಬ್ಬ ಮಾಯಾವಿ, ನಮ್ಮನ್ನು ವಂಚಿಸಲು ಇದನ್ನು ಮಾಡಿರುವನು ಎಂದು ಹೇಳಿದನು. ದುರ್ಯೋಧನ ತನ್ನ ಕಣ್ಣಿನಿಂದಲೇ ವಿಶ್ವರೂಪವನ್ನು ನೋಡಿದ. ಆದರೆ ಅದನ್ನು ತಿಳಿದುಕೊಳ್ಳಲು ಯೋಗ್ಯತೆ ಅವನಿಗೆ ಇರಲಿಲ್ಲ. ಈ ಯೋಗ್ಯತೆಯನ್ನು ನಾವು ಸಾಧನೆಯ ಮೂಲಕ ಸಾಧಿಸಬೇಕು. ಇಲ್ಲದೇ ಇದ್ದರೆ ದೇವರೆ ಎದುರಿಗೆ ಬಂದು ನಿಂತುಕೊಂಡು ನಾನು ದೇವರು ಎಂದರೆ ಎಷ್ಟು ಜನ ಇದನ್ನು ತಿಳಿದುಕೊಳ್ಳ ಬಲ್ಲರು!

ಯುದ್ಧವಾದ ಮೇಲೆ ಶ್ರೀಕೃಷ್ಣ ದ್ವಾರಕೆಗೆ ಹಿಂದಿರುಗುತ್ತಿದ್ದಾಗ ಉದಂಕನೆಂಬ ಪುಷಿ ಶ್ರೀಕೃಷ್ಣ ನನ್ನು ಪ್ರಾರ್ಥಿಸಿಕೊಂಡಾಗ ವಿಶ್ವರೂಪವನ್ನು ತೋರಿಸಿದನು. ಶಾಂತಿಪರ್ವದಲ್ಲಿ ನಾರಾಯಣ ನಾರದರಿಗೆ ತನ್ನ ವಿಶ್ವರೂಪವನ್ನು ತೋರಿಸಿದ ಕೆಲವು ವರ್ಣನೆಗಳಿವೆ. ಆದರೆ ಎಲ್ಲೂ ಅರ್ಜುನನಿಗೆ ಕಂಡ ವಿಶ್ವರೂಪದಂತೆ ಇಲ್ಲ. ಇಲ್ಲಿ ಅವನ ರುದ್ರಾವತಾರವೆ ಹೆಚ್ಚಾಗಿ ಕಾಣುವುದು. ದ್ವಾಪರ ಯುಗದ ಕೊನೆ. ಎಲ್ಲವನ್ನೂ ಸಂಹಾರ ಮಾಡುವುದಕ್ಕೆ ಸಿದ್ಧನಾದ ಭಗವಂತನನ್ನು ನೋಡುವೆವು. ರೈತ ತನ್ನ ಕೈಯಲ್ಲಿ ಕುಡುಗೋಲನ್ನು ತೆಗೆದುಕೊಂಡು ಬಲಿತ ತೆನೆಗಳನ್ನೆಲ್ಲ ಹೇಗೆ ಕೀಳುವನೊ ಹಾಗೆ, ಭಗವಂತ ಎಲ್ಲರನ್ನೂ ನಿರ್ನಾಮ ಮಾಡುತ್ತಿರುವನು.

ವಿಶ್ವರೂಪದರ್ಶನವೆಂಬುದು ಒಂದು ಅಪೂರ್ವ ಅನುಭವ. ಸಾಧಾರಣವಾಗಿ ಭಗವಂತನ ಸೌಮ್ಯರೂಪವನ್ನು ಭಕ್ತರು ಬಯಸುತ್ತಾರೆ. ಆದರೆ ಇಲ್ಲಿ ಬರುವ ದೃಶ್ಯವಂತೂ ಭೀಕರವಾಗಿದೆ, ಭಯಾನಕವಾಗಿದೆ. ಇದನ್ನು ಸಹಿಸಬೇಕಾದರೆ ಗಂಡೆದೆ ಬೇಕು. ಸತ್ಯ ಸುಂದರವಾಗಿರುವಂತೆ ಕಠೋರ ವಾಗಿಯೂ ಇರುವುದು. ಅದರ ಒಂದು ಭಾಗ ಸುಂದರ, ಮತ್ತೊಂದು ಭಾಗ ಕಠೋರ. ಅವನು ವನಮಾಲಿ ಮಾತ್ರವಲ್ಲ, ರುಂಡಮಾಲಿ. ವರ ಮತ್ತು ಅಭಯಗಳನ್ನು ಕೊಡುವ ಕೈಗಳೇ ಅಲ್ಲ ಅವನಿಗಿರುವುದು. ನಿರ್ದಯವಾಗಿ ಅನಾವಶ್ಯಕವಾದುದನ್ನೆಲ್ಲ ಧ್ವಂಸಮಾಡುವುದಕ್ಕೆ ಶಸ್ತ್ರಸನ್ನದ್ಧ ನಾಗಿರುವನು ಮತ್ತೊಂದು ಕೈಯಲ್ಲಿ. ಪ್ರಪಂಚದ ಮತ್ತಾವ ಸಾಹಿತ್ಯದಲ್ಲಿಯೂ ಇಂತಹ ಭಗವ ದ್ಭಾವನೆ ದೊರಕುವುದಿಲ್ಲ.

\begin{verse}
ಅನೇಕವಕ್ತ್ರನಯನಮನೇಕಾದ್ಭುತದರ್ಶನಮ್ ।\\ಅನೇಕದಿವ್ಯಾಭರಣಂ ದಿವ್ಯಾನೇಕೋದ್ಯತಾಯುಧಮ್ \versenum{॥ ೧೦ ॥}
\end{verse}

{\small ಅನೇಕ ಮುಖಗಳು, ಕಣ್ಣುಗಳು ಇರುವ, ಅನೇಕ ಅದ್ಭುತ ದರ್ಶನಗಳು ಇರುವ, ದಿವ್ಯಾಭರಣಗಳಿಂದ ಅಲಂಕೃತವಾದ ಮತ್ತು ಎತ್ತಿ ಹಿಡಿಯಲ್ಪಟ್ಟ ಅನೇಕ ದಿವ್ಯಾಯುಧಗಳಿಂದ ಕೂಡಿದ....}

ಇಲ್ಲಿ ಅನಂತವನ್ನು ಸಾಂತದ ಭಾಷೆಯಲ್ಲಿ ವಿವರಿಸಲು ಪ್ರಯತ್ನಿಸಿದೆ. ಭಗವಂತನನ್ನು ಹಲವು ರೀತಿ ವಿವರಿಸಲು ಯತ್ನಿಸಿರುವರು. ಒಂದೇ, ನೇತಿ ನೇತಿ, ಇದಾವುದೂ ಅಲ್ಲ, ಎಂದು ಹೇಳುವುದು. ನಮಗೆ ಕಾಣುವುದೆಲ್ಲ ದೇಶಕಾಲ ನಿಮಿತ್ತದ ಒಳಗೆ ಇರುವುದು. ಅನಂತವನ್ನು ಇದರ ಮೂಲಕ ಹಿಡಿಯುವುದಕ್ಕೆ ಆಗುವುದಿಲ್ಲ. ತೂತಿನ ಬಲೆಯಲ್ಲಿ ಗಾಳಿಯನ್ನು ಹಿಡಿಯುವುದಕ್ಕೆ ಯತ್ನಿಸಿದಂತೆ ಅದು. ಎರಡನೆಯದೆ ಪರಸ್ಪರ ವಿರೋಧಗಳಿಂದ ಅವನನ್ನು ವಿವರಿಸಲು ಯತ್ನಿಸುವುದು. ಅದು ಅಣುವಿಗೆ ಅಣು, ಮಹತ್ತಿಗೆ ಮಹತ್ತು. ಅದು ಹತ್ತಿರದಲ್ಲಿದೆ ಮತ್ತು ದೂರದಲ್ಲಿದೆ. ಅದು ಎಲ್ಲ ವಸ್ತುಗಳ ಒಳಗೆ ಇದೆ, ಮತ್ತು ಎಲ್ಲ ವಸ್ತುಗಳ ಹೊರಗೆ ಇದೆ, ಅದು ಚಲಿಸುತ್ತಿದೆ, ಚಲಿಸುತ್ತಿಲ್ಲ ಇತ್ಯಾದಿ. ಉಪನಿಷತ್ತಿನಲ್ಲಿ ಈ ರೀತಿ ವಿವರಿಸುವುದಕ್ಕೆ ಆಗದುದನ್ನು ವಿವರಿಸಲು ಯತ್ನಿಸುವರು. ಮೂರನೆಯ ಮಾರ್ಗವೇ ಎಲ್ಲ ಸದ್ಗುಣಗಳನ್ನೂ, ಜ್ಞಾನವನ್ನೂ, ಶಕ್ತಿಯನ್ನೂ, ಪವಿತ್ರತೆಯನ್ನೂ ಒಂದು ವ್ಯಕ್ತಿಗೆ ಆರೋಪ ಮಾಡುವುದು. ಈ ವಿವರಣೆ ಕೂಡ ಅಪೂರ್ಣವಾಗಿದೆ. ಏಕೆಂದರೆ ಇದು ತನಗೆ ವಿರೋಧವಾದುದನ್ನು ತ್ಯಜಿಸುವುದು. ಗೀತೆಯಲ್ಲಿ ಬರುವ ಭಾವನೆಯಾದರೋ ಮೇಲೆ ಹೇಳಿದ ಒಳ್ಳೆಯದನ್ನೆಲ್ಲ ತೆಗೆದುಕೊಳ್ಳುವುದು. ಜೊತೆಗೆ ಕಾಲರುದ್ರನ ಗುಣಗಳನ್ನು ಕೂಡ ಅದಕ್ಕೆ ಬೆರಸು ವುದು. ಅದು ಏಕಕಾಲದಲ್ಲಿ ಸುಂದರವಾಗಿದೆ ಮತ್ತು ಭೀಕರವಾಗಿದೆ. ಅದನ್ನು ನೋಡಿದರೆ ಭಕ್ತಿ ಗೌರವಗಳು ನಮ್ಮ ಹೃದಯದಲ್ಲಿ ಸ್ಫುರಿಸುವುದು ಮಾತ್ರವಲ್ಲ, ಅಂಜಿಕೆ ಆಶ್ಚರ್ಯಗಳೂ ಉಂಟಾಗು ವುವು. ಇದು ಬರೀ ನಮ್ಮ ಹೃದಯಕ್ಕೆ ಆನಂದವನ್ನು ಕೊಡುವ ದೇವರಲ್ಲ, ಭೀಕರವಾದ ನಗ್ನ ಸತ್ಯವನ್ನು ಹೃದಯ ಕಂಪಿಸುವ ಭಾಷೆಯಲ್ಲಿ ಹೇಳುವುದು.

ಅವನಿಗೆ ಅನೇಕ ಮುಖಗಳು. ಎಲ್ಲಿ ನೋಡಿದರೂ ಅವನೇ ಕಾಣುತ್ತಿರುವನು. ಅವನಿಗೆ ಅನೇಕ ಕಣ್ಣುಗಳು. ಎಲ್ಲವನ್ನೂ ನೋಡುತ್ತಿವೆ ಆ ಕಣ್ಣುಗಳು. ಯಾವುದನ್ನು ಅದರ ಮುಂದೆ ಬಚ್ಚಿಡುವಂತೆ ಇಲ್ಲ. ಹೊರಗೆ ಆಗುತ್ತಿರುವುದನ್ನು ಮಾತ್ರ ನೋಡುವ ಕಣ್ಣುಗಳಲ್ಲ ಅವು, ಹೃದಯವನ್ನು ತೂರಿ ನೋಡಬಲ್ಲವು. ಅನೇಕ ಅದ್ಭುತ ದರ್ಶನಗಳನ್ನು ಕಾಣುತ್ತಿರುವನು. ಒಂದು ರೂಪವಲ್ಲ, ಒಂದು ಆಕಾರವಲ್ಲ, ಒಂದು ಬಣ್ಣವಲ್ಲ, ಮನುಷ್ಯ ಕಲ್ಪಿಸಿಕೊಳ್ಳಬಲ್ಲ ಎಲ್ಲಾ ಆಕಾರಗಳ ಹಿಂದೆಯೂ ಅವನು ಇರುವನು. ದಿವ್ಯ ಆಭರಣಗಳಿಂದ ಅವನು ಅಲಂಕೃತನಾಗಿರುವನು. ವಿರಾಟ್​ರೂಪಿಯಾದ ಭಗವಂತನಿಗೆ ವಿರಾಟ್ ಪ್ರಕೃತಿಯೇ ಅಲಂಕಾರವನ್ನು ಮಾಡಿದೆ. ಅವನಿಗೆ ಒಂದು ಜೊತೆಯಲ್ಲ ಕೈಗಳು, ಎಲ್ಲೆಲ್ಲೂ ಅವನ ಕೈಗಳೇ, ಒಂದೊಂದು ಕೈಗಳಲ್ಲಿಯೂ ಒಂದೊಂದು ಬಗೆಯ ಆಯುಧ. ಒಂದೊಂದನ್ನು ನಾಶಮಾಡುವುದಕ್ಕೆ ಒಂದೊಂದು ಬಗೆಯ ಆಯುಧಗಳನ್ನು ಉಪಯೋಗಿಸುವನು. ಆಯುಧಗಳು ಕೂಡ ಎತ್ತಿ ಹಿಡಿಯಲ್ಪಟ್ಟಿವೆ. ಉಪಯೋಗಕ್ಕೆ ಅಣಿಯಾಗಿವೆ, ಕೇವಲ ಅಲಂಕಾರಕ್ಕೆ ಅಲ್ಲ.

\begin{verse}
ದಿವ್ಯಮಾಲ್ಯಾಂಬರಧರಂ ದಿವ್ಯಗಂಧಾನುಲೇಪನಮ್ ।\\ಸರ್ವಾಶ್ಚರ್ಯಮಯಂ ದೇವಮನಂತಂ ವಿಶ್ವತೋಮುಖಮ್ \versenum{॥ ೧೧ ॥}
\end{verse}

{\small ದಿವ್ಯವಾದ ಮಾಲೆಗಳನ್ನು ವಸ್ತ್ರಗಳನ್ನು ಧರಿಸಿರುವ, ದಿವ್ಯ ಗಂಧದಿಂದ ಅನುಲಿಪ್ತನಾದ ಸರ್ವ ಆಶ್ಚರ್ಯ ಮಯನೂ ಅನಂತನೂ ವಿಶ್ವತೋಮುಖನೂ ಆದ ದೇವರನ್ನು ತೋರಿಸಿದನು.}

ಅವನು ದಿವ್ಯ ಮಾಲೆಯನ್ನು ಧರಿಸಿರುವನು. ಬಾಡುವ ಹೂಗಳಿಂದ ಮಾಡಿದ ಮರ್ತ್ಯರ ಮಾಲೆಯಲ್ಲ. ನಯನ ಮನೋಹರವಾದ ವಸ್ತ್ರಗಳಿಂದ ಅಲಂಕೃತನಾಗಿರುವನು. ಅವನು ಲೇಪಿಸಿ ಕೊಂಡಿರುವ ಗಂಧ ದಿಸೆದಿಸೆಗೆ ಪರಿಮಳವನ್ನು ಬೀರುತ್ತಿವೆ. ಅವನು ಆಶ್ಚರ್ಯದಂತೆ ಕಾಣುತ್ತಿರು ವನು. ಇಂತಹ ರೂಪ ನಮ್ಮ ಕಣ್ಣಿಗೆ ಎಂದೂ ಹೊಳೆದಿಲ್ಲ. ಅವನು ಧರಿಸಿದ ಹಾರ, ಉಟ್ಟ ಬಟ್ಟೆ, ಪೂಸಿದ ಗಂಧ ಇವುಗಳನ್ನು ಮಾನವ ಇದುವರೆಗೆ ಕಂಡು ಕೇಳಿ ಅನುಭವಿಸಿ ಅರಿಯ. ಅದಕ್ಕೆ ಅಂತಹ ಆಶ್ಟರ್ಯಪರವಶನಾದುದು.

ಅವನು ಅನಂತ. ಎಲ್ಲಿ ನೋಡಿದರೂ ಅವನೇ ಕಾಣುತ್ತಿರುವನು, ಅವನಿಗೆ ಒಂದು ಆದಿ ಅಂತ್ಯವಿಲ್ಲ. ಎಲ್ಲಾ ರೂಪಗಳ ಹಿಂದೆಯೂ ಅವನೇ ಇರುವನು.

\begin{verse}
ದಿವಿ ಸೂರ್ಯಸಹಸ್ರಸ್ಯ ಭವೇದ್ಯುಗಪದುತ್ಥಿತಾ ।\\ಯದಿ ಭಾಃ ಸದೃಶೀ ಸಾ ಸ್ಯಾದ್ಭಾಸಸ್ತಸ್ಯ ಮಹಾತ್ಮನಃ \versenum{॥ ೧೨ ॥}
\end{verse}

{\small ಸಹಸ್ರ ಸೂರ್ಯನ ಪ್ರಭೆಯು ಒಂದೇ ಸಲ ಆಕಾಶದಲ್ಲಿ ಉಂಟಾದರೆ ಅದನ್ನು ಆ ಮಹಾತ್ಮನ ತೇಜಸ್ಸಿಗೆ ಹೋಲಿಸ ಬಹುದು.}

ಸೂರ್ಯ ಯಾವಾಗಲೂ ಜ್ಞಾನಕ್ಕೆ ನಿಂತಿರುವನು. ಹೇಗೆ ಅವನು ಉದಿಸಿದರೆ ಕತ್ತಲೆ ತೊಲುಗುವುದೊ ಹಾಗೆ ಜ್ಞಾನಭಾಸ್ಕರನೆದುರಿಗೆ ಅಜ್ಞಾನ ಮಾಯವಾಗುವುದು. ಸೂರ್ಯ ಸ್ವಯಂಜ್ಯೋತಿಃಸ್ವರೂಪ. ಅವನು ಗ್ರಹಾದಿಗಳಿಗೆ ಬೆಳಕನ್ನು ದಾನ ಮಾಡುತ್ತಾನೆ. ಅದರಂತೆಯೆ ಪರಮಾತ್ಮ ಸ್ವಯಂಜ್ಯೋತಿಃ ಸ್ವರೂಪ. ಅವನು ಬೆಳಗುವುದರಿಂದ ಈ ಪ್ರಪಂಚದಲ್ಲಿ ಉಳಿದುವುಗಳೆಲ್ಲ ಬೆಳಗುವುವು. ಒಂದು ಸೂರ್ಯನ ಪ್ರಭೆಯನ್ನೇ ನಾವು ಸಹಿಸುವುದು ಕಷ್ಟವಾಗುವುದು. ಹಾಗಿರುವಾಗ ಸಹಸ್ರ ಸೂರ್ಯನ ಪ್ರಭೆ ಆಕಾಶದಲ್ಲಿದ್ದರೆ ಆಗ ನಮ್ಮ ಮನಸ್ಸಿನಲ್ಲಿ ಆಗುವ ಸ್ಥಿತಿಯನ್ನು ನಾವು ಕುರಿತು ಕಲ್ಪಿಸಿಕೊಳ್ಳಬೇಕು. ನಮ್ಮ ಜಡ ಕಣ್ಣುಗಳು ಈಗಿರುವ ಕಾಂತಿಗಿಂತ ಎರಡು ಮೂರಷ್ಟು ಆಗುತ್ತ ಬಂದರೆ ಅದನ್ನು ನೋಡುವ ಸ್ಥಿತಿಯಲ್ಲಿ ಇರುವುದಿಲ್ಲ. ನಮ್ಮ ಕಣ್ಣು ತುಂಬ ಕಡಮೆ ಪ್ರಕಾಶಮಾನವಾದುದನ್ನು ಅರಿಯಲಾರದು. ಹಾಗೆಯೇ ಅತಿಪ್ರಕಾಶವನ್ನು ಕೂಡ ಅರಿಯಲಾರದು. ಶ್ರೀಕೃಷ್ಣ ಅರ್ಜುನ ನಿಗೆ ದಿವ್ಯಚಕ್ಷುಸ್ಸನ್ನು ದಾನಮಾಡಿರುವುರಿಂದ ಬಹುಶಃ ಅದರ ಸಹಾಯದಿಂದ ಅವನು ಸ್ವಲ್ಪಹೊತ್ತು ನೋಡುವ ಸ್ಥಿತಿಯಲ್ಲಿರಬಹುದು. ಗೊತ್ತಿರುವುದರ ಸಹಾಯದಿಂದ ಗೊತ್ತಿಲ್ಲದ ವಸ್ತುವನ್ನು ವಿವರಿಸುವುದು ರೂಢಿ. ಸೂರ್ಯ ಪ್ರಭೆಯ ಪರಿಚಯ ನಮಗೆ ಇದೆ. ಅದಕ್ಕಿಂತ ಸಹಸ್ರಪಾಲು ಜಾಸ್ತಿ ಎಂದರೆ ಮನಸ್ಸಿಗೆ ಅದನ್ನು ಕಲ್ಪಿಸಿಕೊಳ್ಳುವುದಕ್ಕೆ ಆಧಾರ ಸಿಕ್ಕುವುದು.

\begin{verse}
ತತ್ರೈಕಸ್ಥಂ ಜಗತ್ಕೃತ್ಸ್ನಂ ಪ್ರವಿಭಕ್ತಮನೇಕಧಾ ।\\ಅಪಶ್ಯದ್ದೇವದೇವಸ್ಯ ಶರೀರೇ ಪಾಂಡವಸ್ತದಾ \versenum{॥ ೧೩ ॥}
\end{verse}

{\small ಆಗ ಅನೇಕ ಪ್ರಕಾರವಾಗಿ ವಿಭಾಗವಾಗಿರುವ ಜಗತ್ತೆಲ್ಲವೂ ದೇವದೇವನ ಶರೀರದಲ್ಲಿ ಒಂದು ಕಡೆ ಇರುವುದನ್ನು ಅರ್ಜುನ ನೋಡಿದನು.}

ಆಗ ಮನುಷ್ಯಲೋಕ, ಪಿತೃಲೋಕ, ಸ್ವರ್ಗಲೋಕ, ಪಾತಾಳ, ನರಕ ಮುಂತಾದ ಲೋಕಗಳನ್ನೆಲ್ಲ ಅಲ್ಲಿ ನೋಡುತ್ತಾನೆ. ಇಡೀ ಬ್ರಹ್ಮಾಂಡ ಭಗವಂತನ ಸಣ್ಣ ಅಂಶ. ಅವನಲ್ಲಿ ಇದು ಯಾವುದೋ ಒಂದು ಮೂಲೆಯಲ್ಲಿದೆ. ಖಗೋಳಶಾಸ್ತ್ರಜ್ಞರು ಸೂರ್ಯನಲ್ಲಿ ಕೆಲವು ಕಪ್ಪುಚುಕ್ಕಿಗಳಿವೆ ಎನ್ನುತ್ತಾರೆ. ಆ ಕಪ್ಪುಚುಕ್ಕಿಯಲ್ಲಿ ನಮ್ಮ ಭೂಮಿಯ ಗ್ರಹವನ್ನು ಹಾಕಬಹುದಂತೆ. ಈ ಕಪ್ಪುಚುಕ್ಕೆ ಆ ಸೂರ್ಯನಲ್ಲಿರುವ ಒಂದು ರೋಮಕೂಪದಂತೆ. ಇನ್ನು ಆ ಸೂರ್ಯ ಈ ಭೂಮಿಗಿಂತ ಎಷ್ಟು ಪಾಲು ದೊಡ್ಡವನು ಎಂಬುದನ್ನು ಊಹಿಸಿಕೊಳ್ಳಬಹುದು. ಹಾಗೆಯೆ ಆ ಸೂರ್ಯನು ಸಮುದ್ರತೀರದಲ್ಲಿರುವ ಒಂದು ಮರಳ ಕಣದಂತೆ ಈ ವಿಶ್ವದಲ್ಲಿರುವ ಇತರ ಸೂರ್ಯರೊಡನೆ ಹೋಲಿಸಿದಾಗ. ಈ ವಿಶ್ವ ಆಗ ಮತ್ತೆಷ್ಟು ಬೃಹತ್ತಾಗಿರುವುದು! ಕೆಲವು ನಕ್ಷತ್ರಗಳು ನಮ್ಮ ಭೂಮಿಗೆ ಎಷ್ಟು ದೂರದಲ್ಲಿವೆಯೆಂದರೆ, ಅವುಗಳ ಕಾಂತಿ ಇನ್ನೂ ನಮಗೆ ಬಂದಿಲ್ಲ. ಅದು ಸೆಕೆಂಡಿಕೆ ೧,೮೬,೦೦೦ ಮೈಲಿ ವೇಗದಲ್ಲಿ ಸಹಸ್ರಾರು ವರ್ಷಗಳಿಂದ ಚಲಿಸುತ್ತಿದ್ದರೂ ನಮ್ಮ ಭೂಮಿಗೆ ತಲುಪದೆ ಇದ್ದರೆ ಆ ನಕ್ಷತ್ರಗಳು ಎಷ್ಟು ದೂರದಲ್ಲಿವೆ ಎಂಬುದನ್ನು ನೀವೇ ಊಹಿಸಿಕೊಳ್ಳಬಹುದು. ಇಂತಹ ಬೃಹತ್ತಾದ ವಿಶ್ವವೆಲ್ಲ ಆಗಿರುವುದು ಭಗವಂತನ ಯಾವುದೋ ಒಂದು ಅಂಶದಿಂದ. ಯಾವಾಗ ಒಬ್ಬ ದೇವರನ್ನು ನೋಡುತ್ತಾನೆಯೊ ಆಗ ಈ ಬ್ರಹ್ಮಾಂಡ ಅವನ ಯಾವುದೋ ಒಂದು ನಗಣ್ಯವಸ್ತು ಎಂಬುದು ಗೊತ್ತಾಗುವುದು.

\begin{verse}
ತತಸ್ಸ ವಿಸ್ಮಯಾವಿಷ್ಟೋ ಹೃಷ್ಟರೋಮಾ ಧನಂಜಯಃ ।\\ಪ್ರಣಮ್ಯ ಶಿರಸಾ ದೇವಂ ಕೃತಾಂಜಲಿರಭಾಷತ \versenum{॥ ೧೪ ॥}
\end{verse}

{\small ಅನಂತರ ಆಶ್ಚರ್ಯಭರಿತನೂ ಹರ್ಷದಿಂದ ರೋಮಾಂಚವುಳ್ಳವನೂ ಆದ ಅರ್ಜುನ ಶ್ರೀಕೃಷ್ಣನಿಗೆ ಶಿರಸಾ ನಮಸ್ಕರಿಸಿ ಕೈ ಜೋಡಿಸಿಕೊಂಡು ಹೀಗೆ ಹೇಳುತ್ತಾನೆ:}

ಯಾವಾಗ ಶ್ರೀಕೃಷ್ಣನಲ್ಲಿ ವಿಶ್ವರೂಪವನ್ನು ಅರ್ಜುನ ನೋಡುತ್ತಾನೆಯೊ ಆಗ ಆಶ್ಚರ್ಯಭರಿತ ನಾಗುತ್ತಾನೆ. ಶ್ರೀಕೃಷ್ಣನನ್ನು ತನ್ನ ಸ್ನೇಹಿತ ಆಪ್ತ ಮತ್ತು ಇತ್ತೀಚೆಗೆ ಅವನೊಬ್ಬ ದೊಡ್ಡ ತತ್ತ್ವಜ್ಞಾನಿ ಎಂದು ತಿಳಿದುಕೊಂಡಿದ್ದ. ಆದರೆ ಈಗ ನೋಡಿದರೆ ಅವನೇ ಸಾಕ್ಷಾತ್ ಪರಮೇಶ್ವರನಾಗಿದ್ದಾನೆ. ಜೀವನದಲ್ಲಿ ನಾವು ಊಹಿಸದೆ ಇರುವುದು ಆದಾಗ ಆಶ್ಚರ್ಯಪಡುತ್ತೇವೆ. ನಮ್ಮ ಬುದ್ಧಿ ಅದನ್ನು ವಿವರಿಸಲಾರದು. ಮೊದಲನೆ ಪ್ರತಿಕ್ರಿಯೆಯೇ ಆಶ್ಚರ್ಯ.

ಎರಡನೆಯದೆ ಹರ್ಷ, ಆನಂದ. ಸಾಕ್ಷಾತ್ ವಿರಾಡ್ರೂಪಿಯಾದ ಭಗವಂತನೇ ಎದುರಿಗೆ ನಿಂತಿರು ವನು. ಅವನೊಂದು ವಜ್ರದ ಗಣಿ. ಇಹಲೋಕದ ಭೋಗ ಸಾಮಗ್ರಿಗಳೆಲ್ಲ ಗಾಜಿನ ಚೂರು. ಅವನನ್ನು ಪಡೆಯುವುದಕ್ಕೆ ಸರ್ವಸ್ವವನ್ನು ತ್ಯಜಿಸಿ ಹಗಲು ರಾತ್ರಿ ತಪಸ್ಸು ಮಾಡುವವರು ಯೋಗಿ ಗಳು. ನಿತ್ಯ ಅನಿತ್ಯ ವಸ್ತುವನ್ನು ವಿಭಜನೆಮಾಡುವವರು ಜ್ಞಾನಿಗಳು. ಅವನನ್ನು ಪಡೆಯುವುದಕ್ಕೆ ಪ್ರಾರ್ಥಿಸುವರು, ಕೀರ್ತಿಸುವರು, ಯಾಚಿಸುವರು ಭಕ್ತರು. ಇಂತಹ ಪರಮವ್ಯಕ್ತಿಯೇ ಅರ್ಜುನನ ಮುಂದೆ ನಿಂತಿರುವಾಗ ಹರ್ಷದಿಂದ ರೋಮಾಂಚನವಾಗುವುದರಲ್ಲಿ ಆಶ್ಚರ್ಯವಿಲ್ಲ. ಇದೊಂದು ಅಪೂರ್ವ ಅನುಭವ ಅರ್ಜುನನಿಗೆ. ಇದು ಕೂಡ ಭಗವಂತನ ಕೃಪೆಯಿಂದಲೇ ಬಂದದ್ದು.

ಅರ್ಜುನನಿಗೆ ಈ ಭೂಮಾನುಭವ ಆದಾಗ ತಾನೇ ಅಡ್ಡ ಬೀಳುತ್ತಾನೆ. ಈ ಭೂಮಕ್ಕೆ ಅಂತಹ ಒಂದು ಶಕ್ತಿ ಇದೆ ನಮ್ಮ ಮೇಲೆ. ಯಾರೂ ಅದಕ್ಕೆ ಗೌರವ ತೋರು ಎಂದು ಹೇಳಬೇಕಾಗಿಲ್ಲ. ನಾವೇ ಸ್ವಾಭಾವಿಕವಾಗಿ ಹಾಗೆ ಮಾಡುತ್ತೇವೆ. ಸೂರ್ಯ ಹುಟ್ಟಿದರೆ ಕಮಲ ಅರಳುವುದು ಹೇಗೆ ಸ್ವಾಭಾವಿಕವೊ ಹಾಗೆ ಭಗವಂತನೆದುರಿಗೆ ಬಾಗುವುದು ಭಕ್ತನ ಸ್ವಭಾವ. ನಾವು ಅಲ್ಪರು ಆ ಭೂಮದೆದುರಿಗೆ. ಅರ್ಜುನ ದಂಡಪ್ರಣಾಮ ಮಾಡಿ ಎದ್ದಾದ ಮೇಲೆ ಕೈಜೋಡಿಸಿಕೊಂಡು ಸ್ತುತಿಸುತ್ತಾನೆ. ಮುಂದೆ ಬರುವ ಕೆಲವು ಶ್ಲೋಕಗಳಂತೂ ಅತ್ಯಂತ ಶ್ರೇಷ್ಠವಾದ ಭಾವನೆಗಳಿಂದ ತುಂಬಿ ತುಳುಕಾಡುತ್ತಿರುವ ಪ್ರಾರ್ಥನೆಯ ಗೊಂಚಲು. ಸಾಕ್ಷಾತ್ ಭಗವಂತನೇ ಅರ್ಜುನನ ಹೃದಯ ನಾಡಿಯನ್ನು ಮಿಡಿದಾಗ ಹೊರಹೊಮ್ಮಿದ ಗಾನಲಹರಿ. ಇದು ಸುಮ್ಮನೆ ಅರ್ಜುನನ ಹೃದಯದಿಂದ ಉಕ್ಕಿಬರುತ್ತಿದೆ, ತಿಳಿನೀರು ಚಿಲುಮೆಯಿಂದ ಜಿನುಗಿ ಬರುವಂತೆ.

ಅರ್ಜುನನ ಪ್ರಾರ್ಥನೆ:

\begin{verse}
ಪಶ್ಯಾಮಿ ದೇವಾಂಸ್ತವ ದೇವ ದೇಹೇ\\ ಸರ್ವಾಂಸ್ತಥಾ ಭೂತವಿಶೇಷಸಂಘಾನ್ ।\\ಬ್ರಹ್ಮಾಣಮೀಶಂ ಕಮಲಾಸನಸ್ಥಂ\\ ಪುಷೀಂಶ್ಚ ಸರ್ವಾನುರಗಾಂಶ್ಚ ದಿವ್ಯಾನ್ \versenum{॥ ೧೫ ॥}
\end{verse}

{\small ದೇವ, ನಿನ್ನ ದೇಹದಲ್ಲಿ ನಾನು ದೇವತೆಗಳನ್ನೂ ನಾನಾ ಬಗೆಯ ಸಕಲ ಪ್ರಾಣಿಗಳನ್ನೂ ಕಮಲಾಸನದ ಮೇಲೆ ವಿರಾಜಮಾನನಾದ ಬ್ರಹ್ಮನನ್ನೂ ಸಕಲ ಪುಷಿಗಳನ್ನೂ ಮತ್ತು ಸರ್ಪಗಳನ್ನೂ ನೋಡುತ್ತಿದ್ದೇನೆ.}

ಶ್ರೀಕೃಷ್ಣ ವಿಭೂತಿಯೋಗದಲ್ಲಿ ತಾನು ಎಲ್ಲೆಲ್ಲಿರುತ್ತೇನೆ ಎಂಬುದನ್ನು ವಿವರಿಸುತ್ತಾನೆ. ಇಲ್ಲಿ ವಿಶ್ವರೂಪದರ್ಶನದ ಯೋಗದಲ್ಲಿ, ಅರ್ಜುನ ಅವನಲ್ಲಿ ಎಲ್ಲವನ್ನೂ ನೋಡುವನು. ಎಲ್ಲಾ ದೇವತೆಗಳೂ ಅವನಲ್ಲಿಯೇ ಇರುವರು. ಇಂದ್ರ, ವರುಣ, ಅಗ್ನಿ, ಮರುತ್, ಕುಬೇರ ಮುಂತಾದವ ರೆಲ್ಲ ಭಗವಚ್ಛಕ್ತಿಯ ಕಿಡಿಗಳು ಮಾತ್ರ. ಮತ್ತು ಅವನಲ್ಲಿ ವಿಶೇಷ ಆಕಾರಗಳುಳ್ಳ ಪ್ರಾಣಿಗಳನ್ನು ನೋಡುತ್ತಾನೆ. ನಮ್ಮ ಭೂಮಿಯಲ್ಲಿ ಒಂದು ಕಾಲದಲ್ಲಿ ಇದ್ದು ಈಗ ಮಾಯವಾಗಿ ಹೋಗಿರುವ ಪ್ರಾಣಿಗಳು ಮತ್ತು ಈ ಭೂಮಿಯಲ್ಲಿ ಹಿಂದೆ ಮತ್ತು ಈಗ ಎಂದಿಗೂ ಇಲ್ಲದೆ, ಮತ್ತಾವುದೊ ಗ್ರಹದಲ್ಲಿ ಅಲ್ಲಿಯ ವಾತಾವರಣಕ್ಕೆ ಸರಿಯಾಗಿ ಹೊಂದಿಕೊಂಡು ವಿಕಾಸವಾದ ಪ್ರಾಣಿಗಳು ಇವುಗಳನ್ನೆಲ್ಲ ನೋಡುತ್ತಾನೆ. ಈ ಪ್ರಪಂಚವನ್ನು ಸೃಷ್ಟಿಸುವ ಬ್ರಹ್ಮನನ್ನೂ ಕೂಡ ಶ್ರೀಕೃಷ್ಣನಲ್ಲಿ ನೋಡುತ್ತಾನೆ. ಬ್ರಹ್ಮನ ಮೂಲಕ ಕೆಲಸ ಮಾಡುವ ಶಕ್ತಿಯೂ ಅವನೇ. ಎಲ್ಲ ಪುಷಿಗಳು, ಹಿಂದೆ ಈ ಭೂಮಿಯಲ್ಲಿದ್ದು ಈಗ ಅಳಿದುಹೋಗಿರುವವರೆಲ್ಲ ಭಗವಂತನಲ್ಲಿ ಈಗಲೂ ಇರುವರು. ಭಗವಂತನಲ್ಲಿ ಎಲ್ಲಾ ಇದೆ. ಹಿಂದೆ ಇದ್ದದ್ದು, ಈಗ ಇರುವುದು, ಮುಂದೆ ಇರುವುದು ಎಲ್ಲವೂ ಅವನಲ್ಲಿದೆ. ಅದಕ್ಕೆ ಶ್ರೀಕೃಷ್ಣ ಅರ್ಜುನನಿಗೆ, ನೀನು ಏನು ನೋಡಬೇಕೆಂದು ಬಯಸುವೆಯೊ ಅವನ್ನೆಲ್ಲ ನೋಡು ಎನ್ನುತ್ತಾನೆ. ಆದಿಶೇಷ ಮುಂತಾದ ಹಾವುಗಳನ್ನೂ ಕೂಡ ಅವನಲ್ಲಿ ನೋಡು ತ್ತಾನೆ.

\begin{verse}
ಅನೇಕಬಾಹೂದರವಕ್ತ್ರನೇತ್ರಂ \\ ಪಶ್ಯಾಮಿ ತ್ವಾ ಸರ್ವತೋಽನಂತರೂಪಮ್ ।\\ನಾಂತಂ ನ ಮಧ್ಯಂ ನ ಪುನಸ್ತವಾದಿಂ\\ ಪಶ್ಯಾಮಿ ವಿಶ್ವೇಶ್ವರ ವಿಶ್ವರೂಪ \versenum{॥ ೧೬ ॥}
\end{verse}

{\small ವಿಶ್ವೇಶ್ವರಾ! ಅನೇಕ ಬಾಹುಗಳು, ಅನೇಕ ಹೊಟ್ಟೆಗಳು, ಮುಖಗಳು, ಕಣ್ಣುಗಳು ಇವುಗಳುಳ್ಳವನೂ, ಅನಂತರೂಪನೂ ಆದ ನಿನ್ನನ್ನು ಎಲ್ಲೆಲ್ಲಿಯೂ ನೋಡುತ್ತಿದ್ದೇನೆ. ಪುನಃ ವಿಶ್ವರೂಪನೆ, ನಿನ್ನ ಆದಿ ಮಧ್ಯ ಅಂತ್ಯಗಳನ್ನು ಕಾಣಲಾರೆನು.}

ಭಗವಂತನಲ್ಲಿ ಅನೇಕ ಬಾಹುಗಳನ್ನು ನೋಡುತ್ತಾನೆ. ಎಲ್ಲಾ ಕಡೆಯಲ್ಲಿಯೂ ಅವನು ಕೆಲಸ ಮಾಡುತ್ತಿರುವನು. ಅವನ ಕೈವಾಡವೆ ಕಿರಿಯ ಕಣಗಳ ಚಲನವಲನಗಳ ಹಿಂದೆ, ಅನಂತ ಸೂರ್ಯ ತಾರಾವಳಿಯ ಚಲನವಲನದ ಹಿಂದೆ ಕಾಣತ್ತಿದೆ. ಬಿಡುವಿಲ್ಲದೆ ಅವನ ಕೈಗಳು ಎಲ್ಲಾ ಕಡೆಯಲ್ಲಿಯೂ ಕೆಲಸ ಮಾಡುತ್ತಿರುವುದು. ಅವನಿಗೆ ಅನಂತ ಹೊಟ್ಟೆಗಳು ಇವೆ. ಎಲ್ಲವನ್ನೂ ಕಬಳಿಸುತ್ತಿರುವನು. ಹಲವು ವಸ್ತುಗಳನ್ನು ನುಂಗಿ ತನ್ನ ಹೊಟ್ಟೆಯಲ್ಲಿ ಇಟ್ಟುಕೊಳ್ಳುತ್ತಿರುವನು. ಎಲ್ಲಾ ಬಗೆಯ ಆಹಾರಗಳನ್ನು ಅವನು ಅರಗಿಸಿಕೊಳ್ಳುತ್ತಿರುವನು. ಅವನ ಮೊಗದಾವರೆ ಕಾಣದ ಸ್ಥಳವೆ ಇಲ್ಲ. ಹಿಮಮಣಿಯಲ್ಲಿ ಅವನು ಕಾಣುವನು. ಹೂವಿನಲ್ಲಿ ಕಾಣುವನು, ಹಕ್ಕಿಯಲ್ಲಿ ಕಾಣುವನು, ಪ್ರಾಣಿ ಯಲ್ಲಿ ಕಾಣುವನು, ಮನುಷ್ಯನಲ್ಲಿ ಕಾಣುವನು, ಸೂರ್ಯ, ಚಂದ್ರ, ನೀಹಾರಿಕೆಗಳೂ ಅವನ ಮುಖಬಿಂಬಗಳೇ. ಅವನಿಗೆ ಅನಂತ ಕಣ್ಣುಗಳು ಇವೆ. ಎಲ್ಲವನ್ನೂ ಅವನು ನೋಡುತ್ತಿರುವನು. ಅವನು ನೋಡದ ಘಟನೆ ಇಲ್ಲ. ನಮ್ಮ ಮನಸ್ಸಿನಲ್ಲಿ ಆಗುತ್ತಿರುವುದನ್ನು ನೋಡುತ್ತಿರುವನು. ಅವನು ಯಾವಾಗಲೂ ಕಣ್ಣನ್ನು ಮುಚ್ಚುವುದಿಲ್ಲ, ಯಾರೂ ಅವನ ಕಣ್ಣನ್ನು ಮುಚ್ಚಲೂ ಆರರು. ಅವನು ಅನಂತ ರೂಪನು ಮತ್ತು ಸರ್ವಾಂತರ್ಯಾಮಿ. ಅವನಿಲ್ಲದ ಎಡೆಯಿಲ್ಲ. ಯಾವುದನ್ನು ತುಂಬಾ ಹೇಯ ಎಂದು ಭಾವಿಸುತ್ತೇವೆಯೊ ಅದರ ಹಿಂದೆಯೂ ಅವನೆ ಇರುವನು. ಯಾವುದನ್ನು ಅತ್ಯಂತ ಸಣ್ಣದು ಎಂದು ಭಾವಿಸುತ್ತೇವೆಯೊ ಅಲ್ಲಿಯೂ ಅವನಿರುವನು. ಜಡದಲ್ಲಿ ಅವನಿರುವನು. ಅವನು ವಿಶ್ವರೂಪಿ. ಅವನ ಆದಿ ಮಧ್ಯ ಅಂತ್ಯಗಳನ್ನು ಅರ್ಜುನ ಕಾಣದವನಾಗಿದ್ದಾನೆ. ಅವನ ಸುಳಿಯಲ್ಲಿ ಸಿಕ್ಕಿ ಸುತ್ತುತ್ತಿರುವವನು ಅರ್ಜುನ. ಅವನು ಎಲ್ಲಿ ಪ್ರಾರಂಭವಾದ ಮತ್ತು ಅವನು ಎಲ್ಲಿ ಅಂತ್ಯವಾದ ಎಂದು ಹೇಳುವಂತೆ ಇಲ್ಲ. ಸಾಂತವಸ್ತುವಿಗೆ ಇದನ್ನು ಹೇಳಬಹುದು. ಅನಂತಕ್ಕೆ ನಾವು ಇದನ್ನು ಹೇಗೆ ಹೇಳುವುದಕ್ಕೆ ಆಗುವುದು? ನಾವು ಭಾವಿಸುವ ಆದಿಗೆ ಆದಿ ಅವನು, ಅಂತ್ಯಕ್ಕೆ ಅಂತ್ಯ ಅವನು. ಮಧ್ಯ ಎಂದರೆ, ಹಿಂದೆ ಆಗಿಹೋಗಿರುವುದು, ಮುಂದೆ ಆಗುವುದು, ಇವುಗಳ ಅಂತರದಲ್ಲಿರುವುದು. ಆಗಿರುವುದು, ಆಗುತ್ತಿರುವುದು, ಮುಂದೆ ಆಗುವುದು ಎಲ್ಲ ಏಕಕಾಲದಲ್ಲಿ ಅವನಲ್ಲಿದೆ. ನಮ್ಮ ಬುದ್ಧಿ ಕಕ್ಕಾಬಿಕ್ಕಿಯಾಗುವುದು. ಇದನ್ನು ಗ್ರಹಿಸುವುದಕ್ಕೆ ಕಾಲದೇಶ ನಿಮಿತ್ತದ ಮೇಲೆ ತೆವಳಿಕೊಂಡು ಹೋಗುತ್ತಿರುವ ಕೀಟ ನಾವು. ಹಿಂದಿನದನ್ನು ಬಿಟ್ಟು ವರ್ತಮಾನ ಕಾಲದಲ್ಲಿ ಮುಂದಿನ ಕಡೆಗೆ ತೆವಳಿಕೊಂಡು ಹೋಗುತ್ತಿರುವುದು. ಅದು ಆಕಾಶದಲ್ಲಿ ಸ್ವೇಚ್ಛೆಯಿಂದ ಹಾರಾಡು ತ್ತಿರುವ ಹಕ್ಕಿಯನ್ನು ನೋಡಿದಾಗ ಇದು ಹೇಗೆ ಸಾಧ್ಯ ಎಂದು ಭಾವಿಸುವುದು.

\begin{verse}
ಕಿರೀಟಿನಂ ಗದಿನಂ ಚಕ್ರಿಣಂ ಚ\\ ತೇಜೋರಾಶಿಂ ಸರ್ವತೋ ದೀಪ್ತಿಮಂತಮ್ ।\\ಪಶ್ಯಾಮಿ ತ್ವಾಂ ದುರ್ನಿರೀಕ್ಷ್ಯಂ ಸಮಂತಾ\\ ದ್ದೀಪ್ತಾನಲಾರ್ಕದ್ಯುತಿಮಪ್ರಮೇಯಮ್ \versenum{॥ ೧೭ ॥}
\end{verse}

{\small ಕಿರೀಟ, ಗದೆ, ಚಕ್ರ ಇವುಗಳನ್ನು ಧರಿಸಿದವನು, ತೇಜೋರಾಶಿಯೂ ಎಲ್ಲೆಲ್ಲಿಯೂ ಪ್ರಕಾಶವುಳ್ಳವನೂ, ಸುತ್ತಲೂ ನೋಡಲಸಾಧ್ಯನೂ, ಪ್ರಕಾಶಿಸುತ್ತಿರುವ ಅಗ್ನಿ ಸೂರ್ಯರಂತೆ ಪ್ರಭೆಯುಳ್ಳವನೂ, ಅಪ್ರಮೇಯನೂ ಆದ ನಿನ್ನನ್ನು ನೋಡುತ್ತಿದ್ದೇನೆ.}

ಹಿಂದೂಗಳಾದ ನಾವು ದೇವರಿಗೆ ಸಾಧಾರಣವಾಗಿ ಕೊಡುವ ಉಡುಪು ಮತ್ತು ಆಯುಧಗಳು ಕಿರೀಟ, ಚಕ್ರ ಇವುಗಳು. ಈ ರೂಪದಲ್ಲಿಯೂ ಅವನು ಕಾಣುತ್ತಿರುವನು. ಆದರೆ ಅವನು ಬರೀ ಆಕಾರದಲ್ಲೆ ಖಾಲಿ ಆಗಿರುವವನು ಮಾತ್ರವಲ್ಲ. ಅವನು ಆಕಾರನೂ ಆಗಿದ್ದಾನೆ, ನಿರಾಕಾರನೂ ಆಗಿದ್ದಾನೆ. ಸುಗುಣನೂ ಆಗಿದ್ದಾನೆ. ಅವನು ತೇಜೋರಾಶಿ. ಅವನಿಂದ ಸೂರ್ಯ, ಚಂದ್ರ, ಅಗ್ನಿಗಳು ಬೆಳಕನ್ನು ಎರವಲಾಗಿ ಪಡೆಯಬೇಕು. ಈ ತೇಜಸ್ಸಿನ ಮುಂದೆ ಉಳಿದ ತೇಜಸ್ಸುಗಳೆಲ್ಲ ಹಗಲಲ್ಲಿ ಹೊಳೆವ ಮಿಂಚುಹುಳುಗಳಂತೆ. ಅವನು ಎಲ್ಲೋ ಒಂದು ಕಡೆ ಇಲ್ಲ. ಎತ್ತ ಕಣ್ಣನ್ನು ಪಸರಿಸಿದರೂ ಅಲ್ಲೆಲ್ಲಾ ಅವನೇ ಇರುವನು. ಮೊಬ್ಬು ಮೊಬ್ಬಾಗಿಲ್ಲ, ಅಸ್ಪಷ್ಟವಾಗಿಲ್ಲ. ಸ್ಪಷ್ಟವಾಗಿ ಕಾಣುತ್ತಿರುವನು. ನೋಡುವುದಕ್ಕೆ ಅಸಾಧ್ಯ ಎಂದರೆ ತುಂಬ ಕಾಲ ಅವನ ಕಾಂತಿಯನ್ನು ಕಣ್ಣು ಬಿಟ್ಟುಕೊಂಡು ನೋಡುವುದಕ್ಕೆ ಆಗುವುದಿಲ್ಲ. ಕಣ್ಣನ್ನು ಕೋರೈಸುವ ಕಾಂತಿ ಅದು. ಅವನು ಸೂರ್ಯನಂತೆ ಸ್ವಯಂ ಜ್ಯೋತಿಯಿಂದ ಬೆಳಗುತ್ತಿರುವನು. ಅವನು ಯಾರಿಂದಲೂ ಕಾಂತಿಯನ್ನು ಸಾಲವಾಗಿ ತೆಗೆದುಕೊಳ್ಳುವುದಿಲ್ಲ. ಎಲ್ಲರಿಗೂ ಕಾಂತಿಯನ್ನು ಕೊಡುವನು ಅವನು. ಅವನು ಅಪ್ರಮೇಯ. ಹೀಗೆ ಎಂದು ಬಣ್ಣಿಸುವುದಕ್ಕೆ ಆಗುವುದಿಲ್ಲ. ನಮ್ಮ ವರ್ಣನೆಗೆ ನಿಲುಕದವನು ಅವನು. ನಮ್ಮ ಬುದ್ಧಿ ಸೋತೆ ಎನ್ನುವುದು ಅವನೆದುರಿಗೆ.

\begin{verse}
ತ್ವಮಕ್ಷರಂ ಪರಮಂ ವೇದಿತವ್ಯಂ ತ್ವಮಸ್ಯ ವಿಶ್ವಸ್ಯ ಪರಂ ನಿಧಾನಮ್ ।\\ತ್ವಮವ್ಯಯಃ ಶಾಶ್ವತಧರ್ಮಗೋಪ್ತಾ ಸನಾತನಸ್ತ್ವಂ ಪುರುಷೋ ಮತೋ ಮೇ \versenum{॥ ೧೮ ॥}
\end{verse}

{\small ನೀನು ಅರಿಯಲು ಯೋಗ್ಯನಾದ ಅಕ್ಷರೂಪ. ಈ ಜಗತ್ತಿನ ಶ್ರೇಷ್ಠ ಆಧಾರ. ನೀನು ಅವ್ಯಯ ಮತ್ತು ಶಾಶ್ವತಧರ್ಮವನ್ನು ಕಾಪಾಡುವವನು, ಸನಾತನನಾದ ಪುರುಷನು ನೀನು ಎಂಬುದು ನನ್ನ ಮತ.}

ನೀನು ತಿಳಿಯಬೇಕಾಗಿರುವ ವಸ್ತು. ಈ ಪ್ರಪಂಚದಲ್ಲಿ ತಿಳಿಯುವುದಕ್ಕೆ ಎಷ್ಟೋ ವಸ್ತುಗಳಿವೆ. ಆದರೆ ಒಮ್ಮೆ ಭಗವಂತನನ್ನು ತಿಳಿದುಕೊಂಡರೆ ಎಲ್ಲವನ್ನೂ ತಿಳಿದುಕೊಂಡಂತಾಗುತ್ತದೆ. ಅವನನ್ನು ತಿಳಿದರೆ ಸರ್ವಜ್ಞನಾಗುವನು. ಇತರ ವಸ್ತುಗಳನ್ನು ತಿಳಿದರೆ, ಅಷ್ಟೇ ಅವನ ವಿದ್ಯೆ, ಅದಕ್ಕಿಂತ ಹೆಚ್ಚಾಗಲಾರದು. ಆಧ್ಯಾತ್ಮವಾದರೋ ಹಾಗಲ್ಲ. ಈ ಪ್ರಪಂಚದ ಸಾರವನ್ನೇ ಅದು ವಿವರಿಸುವುದು.

ನೀನು ಶ್ರೇಷ್ಠನಾದ ಅಕ್ಷರ ಎನ್ನುತ್ತಾನೆ. ಈ ಪ್ರಪಂಚದಲ್ಲಿ ನಾಶವಾಗದೆ ಇರುವ ವಸ್ತು ಒಂದೇ ಇರುವುದು. ಅದೇ ಭಗವಂತ. ಉಳಿದವುಗಳೆಲ್ಲ ದೇಶಕಾಲನಿಮಿತ್ತ ಪ್ರಪಂಚದಲ್ಲಿ ಗುಳ್ಳೆಗಳಂತೆ ಎದ್ದು ಕೆಲವುಕಾಲ ಹರಿದುಹೋಗಿ ಅನಂತರ ನಾಶವಾಗುವುದು. ಈ ಬ್ರಹ್ಮಾಂಡವೇ ನಾಶವಾದರೂ ಅದನ್ನು ಸೃಷ್ಟಿಸಿದ ಭಗವಂತ ನಾಶವಾಗುವುದಿಲ್ಲ. ವಿಶ್ವಕ್ಕೆ ಅವನು ಆಧಾರ. ಆದರೆ ಅವನಿಗೆ ವಿಶ್ವದ ಆಧಾರ ಬೇಕಾಗಿಲ್ಲ. ಅಲೆ ಇರಬೇಕಾದರೆ ಸಾಗರದ ಆಧಾರ ಬೇಕು, ಆದರೆ ಸಾಗರ ಇರಬೇಕಾದರೆ ಅಲೆಯ ಆಧಾರ ಬೇಕಾಗಿಲ್ಲ. ಈ ಬ್ರಹ್ಮಾಂಡವೆಂಬ ಹಲವು ಅಲೆಗಳು ಏಳುವುವು, ಬೀಳುವುವು, ಅವನಲ್ಲಿ. ಆದರೆ ಅವನಾದರೊ ಎಂದಿಗೂ ಚ್ಯುತನಾಗದೆ ಇರುವನು. 

ಇವನು ಅವ್ಯಯ. ಎಂದಿಗೂ ನಾಶವಾಗದವನು. ದೇಶಕಾಲನಿಮಿತ್ತದಲ್ಲಿರುವ ವಸ್ತು ಹುಟ್ಟುವುದು, ಬದಲಾಯಿಸುವುದು, ನಾಶವಾಗುವುದು. ಆದರೆ ಭಗವಂತನಿಗಾದರೊ ಯಾವ ಬದಲಾವಣೆಯೂ ಇಲ್ಲ. ಯಾವ ನಾಶವೂ ಇಲ್ಲ. ಹುಟ್ಟಿದ ವಸ್ತುವಿಗೆ ತಾನೆ ನಾಶವಾಗುವ ಭಯ. ಯಾವಾಗ ಒಂದು ವಸ್ತು ಹುಟ್ಟಿಯೇ ಇಲ್ಲವೋ ಅದು ನಾಶವಾಗುವುದು ಹೇಗೆ?

ಇವನು ಪ್ರಪಂಚದಲ್ಲಿ ಶಾಶ್ವತ ಧರ್ಮವನ್ನು ಕಾಪಾಡುವನು. ಪ್ರತಿಯೊಂದು ಧರ್ಮವೂ ಭಗವಂತನ ಕಡೆಗೆ ಹೋಗುವುದಕ್ಕೆ ಒಂದೊಂದು ಮಾರ್ಗ. ಪ್ರತಿಯೊಂದು ಧರ್ಮದಲ್ಲಿಯೂ ಮೂಲವಾದ, ಸಾರವಾದ ತಿರುಳಿದೆ. ಅದನ್ನು ಮುಚ್ಚಿದ ಕರಟ ಸಿಪ್ಪೆಗಳು ಬೇರೆ ಇವೆ, ಇದು ಗೌಣ. ಎಲ್ಲಾ ಧರ್ಮಗಳಲ್ಲಿಯೂ ಕೆಲವು ಕಾಲದಮೇಲೆ ಕೆಲಸಕ್ಕೆ ಬಾರದ ಕಳೆಗಳು ಬೆಳೆಯುವುವು. ಹಾಗೆ ಕಳೆ ಬೆಳೆದಾಗದೇವರೇ ಅವತಾರದಂತೆ ಇಳಿದು ಅದನ್ನು ಕಿತ್ತು ಅದರಲ್ಲಿರುವ ಮೂಲಚೈತನ್ಯಕ್ಕೆ ಹೊಸ ಕಾಂತಿಯನ್ನು ಇತ್ತು ಅದನ್ನು ಬೆಳಗುವನು. ದೇವರೇ ಯಾವಾಗಲೂ ಅವತಾರದಂತೆ ಇಳಿಯಬೇಕಾಗಿಲ್ಲ. ಕೆಲವು ವೇಳೆ ಅವನು ಈ ರಕ್ಷಣೆಯ ಕೆಲಸವನ್ನು ತನ್ನ ಭಕ್ತರ ಮೂಲಕ ಮಾಡುವನು. ದೇವರ ದೃಷ್ಟಿಗೆ ಎಲ್ಲಾ ಧರ್ಮಗಳೂ ಒಂದೇ. ಅವನು ಎಲ್ಲಾ ಧರ್ಮಗಳನ್ನೂ ಚೆನ್ನಾಗಿ ಇಟ್ಟಿರಲು ಯತ್ನಿಸುವನು.

ಇವನೇ ಸನಾತನ ಪುರುಷ. ಎಂದೆಂದಿಗೂ ಇರುವ ಪರಮೇಶ್ವರ. ಅವನು ಹಿಂದೆ ಇದ್ದ, ಈಗ ಇರುವನು, ಮುಂದೆ ಇರುವನು. ಅವನು ನಮಗೆ ಗೊತ್ತಿಲ್ಲದಾಗ ಇದ್ದ. ಈಗ ಅನೇಕರಿಗೆ ಅವನು ಬೇಕಿಲ್ಲ, ಆದರೂ ಇರುವನು. ಮುಂದೆ ವಿಶ್ವವೇ ನಿರ್ನಾಮವಾದರೂ ಅವನು ಇರುವನು. ಅವನನ್ನು ಯಾವ ಹೆಸರಿನಿಂದ ಬೇಕಾದರೂ ಕರೆಯಬಹುದು, ಯಾವ ಆಕಾರದ ಮೂಲಕವಾಗಿ ಬೇಕಾದರೂ ಚಿಂತಿಸಬಹುದು. ಅವನಿಲ್ಲದ ಕಾಲವಿಲ್ಲ. ಅವನಿಲ್ಲದ ದೇಶವಿಲ್ಲ.

\begin{verse}
ಅನಾದಿಮಧ್ಯಾಂತಮನಂತವೀರ್ಯಮನಂತಬಾಹುಂ ಶಶಿಸೂರ್ಯನೇತ್ರಮ್ ।\\ಪಶ್ಯಾಮಿ ತ್ವಾಂ ದೀಪ್ತಂಹುತಾಶವಕ್ತ್ರಂ ಸ್ವತೇಜಸಾ ವಿಶ್ವಮಿದಂ ತಪಂತಮ್ \versenum{॥ ೧೯ ॥}
\end{verse}

{\small ಆದಿ ಮಧ್ಯ ಅಂತ್ಯ ರಹಿತನು, ಅನಂತ ವೀರ್ಯನು, ಅನಂತ ಬಾಹುಗಳುಳ್ಳವನು, ಸೂರ್ಯಚಂದ್ರರೇ ನೇತ್ರಗಳಾಗಿ ಉಳ್ಳವನು; ಪ್ರದೀಪ್ತವಾದ ಅಗ್ನಿಯಂತೆ ಬಾಯಿಯುಳ್ಳವನು. ತನ್ನ ತೇಜಸ್ಸಿನಿಂದ ಈ ವಿಶ್ವವನ್ನು ತಪಿಸುತ್ತಿರುವವನೂ ಆದ ನಿನ್ನನ್ನು ನೋಡುತ್ತಿದ್ದೇನೆ.}

ಅವನಿಗೆ ಆದಿ ಇಲ್ಲ. ಯಾವಾಗ ನಾವು ಒಂದು ಆದಿಯನ್ನು ಊಹಿಸುತ್ತೇವೆಯೋ ಆಗ ಆದಿಗೆ ಮುಂಚೆ ಏನಿತ್ತು ಎಂದು ಕೇಳುತ್ತೇವೆ. ಬೇಕಾದರೆ ಯುಗಕ್ಕೆ ಆದಿ ಇರಬಹುದು. ಕಲ್ಪಕ್ಕೆ ಆದಿ ಇರಬಹುದು. ಈ ಭೂಮಿಗೆ ಸೂರ್ಯನಿಗೆ ಆದಿಯನ್ನು ಕಂಡುಹಿಡಿಯಲು ಯತ್ನಿಸಬಹುದು. ಆದಿ ಅಂತ್ಯ ಎಂಬಿವುಗಳು ಕಾಲ ಬಂದಾದಮೇಲೆ ಬರುವುದು. ದೇವರಾದರೊ ಕಾಲಕ್ಕೆ ಪಿತನಾಗಿದ್ದಾನೆ. ಅವನಿಗೆ ಆದಿ ಹೇಗೆ ಸಾಧ್ಯ? ಅದರಂತೆಯೇ ಅವನಿಗೆ ಅಂತ್ಯವಿಲ್ಲ. ಅಂತ್ಯವನ್ನು ಒಪ್ಪಿಕೊಂಡರೆ ಅದಾದಮೇಲೆ ಇನ್ನು ಏನು ಬರುವುದು ಎಂಬ ಪ್ರಶ್ನೆ ನಮ್ಮನ್ನು ಕಾಡುವುದು. ಅವನಿಗೆ ಅಂತ್ಯವೇ ಇಲ್ಲ. ಅವನು ಅನಂತ. ಹಿಂದೆ ಮುಂದೆ ಇವುಗಳ ಮಧ್ಯದಲ್ಲಿ ಇರುವುದು ತಾನೆ ಈಗ ಎಂಬುದು. ಯಾವುಕ್ಕೆ ಹಿಂದೆಯೂ ಇಲ್ಲವೊ ಮುಂದೆಯೂ ಇಲ್ಲವೊ ಅದಕ್ಕೆ ಮಧ್ಯ ಹೇಗೆ ಇರಬಲ್ಲದು?

ಅವನು ಅನಂತವೀರ್ಯನು. ಅವನ ಶಕ್ತಿಗೆ ಒಂದು ಅಂತ್ಯವಿಲ್ಲ. ಒಂದು ಸರ್ಕಸ್ಸಿನಲ್ಲಿ ನಾಲ್ಕೈದು ಚೆಂಡುಗಳನ್ನು ಕೆಳಗೆ ಬೀಳಿಸದೆ ಒಂದೇ ಸಮನಾಗಿ ಕೈಯಲ್ಲಿ ಆಡಿಸುತ್ತಿರುವುದನ್ನು ನೋಡಿದಾಗ ನಾವು ಆಶ್ಚರ್ಯಚಕಿತರಾಗುತ್ತೇವೆ. ಅನಂತ ಸೂರ್ಯ ಚಂದ್ರ ತಾರಿಕೆಗಳನ್ನು ಆಕಾಶದಲ್ಲಿ ಅವನು ಹೇಗೆ ಆಡಿಸುತ್ತಿರುವನು! ಒಂದೂ ಬೀಳುವುದಿಲ್ಲ. ಒಂದು ಮತ್ತೊಂದಕ್ಕೆ ತಾಕುವುದಿಲ್ಲ. ಅವುಗಳ ಉದಯ ಅಸ್ತಮಗಳಲ್ಲಿ ಸ್ವಲ್ಪವೂ ಹೆಚ್ಚು ಕಡಮೆಯಿಲ್ಲ. ಆ ಭೀಮ ಬಾಹುಗಳು ಬ್ರಹ್ಮಾಂಡವನ್ನೇ ಹೇಗೆ ಎಸೆದಾಡುತ್ತಿವೆ. ಸಹಸ್ರಾರು ಆಟಂಬಾಂಬುಗಳನ್ನು ಮೀರಿಸುವ ಶಕ್ತಿ ಒಂದು ಭೂಕಂಪ ದಲ್ಲಿದೆ, ಒಂದು ಜ್ವಾಲಾಮುಖಿಯಲ್ಲಿದೆ. ಸಹಸ್ರಾರು ಟನ್ನುಗಳಷ್ಟು ತೂಗುವ ಹಡಗುಗಳನ್ನು ಕಡ್ಡಿಯಪೆಟ್ಟಿಗೆಯಂತೆ ಎತ್ತಿ ಬಿಸಾಡಬಲ್ಲುದು ಒಂದು ಚಂಡಮಾರುಡ. ಈ ಶಕ್ತಿಗಳ ಹಿಂದೆಲ್ಲ ಭಗವಂತನೇ ಇರುವನು.

ಸೂರ್ಯಚಂದ್ರರೇ ಅವನ ಕಣ್ಣುಗಳು. ಸೂರ್ಯ ಕಾಂತಿಯನ್ನೂ ಕಾವನ್ನೂ ಕೊಡುವನು. ಕತ್ತಲೆ ಓಡಿಹೋಗುವುದು ಅವನೆದುರಿಗೆ. ಅವನು ಜ್ಞಾನ. ಅಜ್ಞಾನ ಅವನೆದುರಿಗೆ ನಿಲ್ಲಲಾರದು. ಅವನ ಮತ್ತೊಂದು ಕಣ್ಣೇ ಚಂದ್ರ. ಆಹ್ಲಾದಕರವಾದ ಕಾಂತಿಯನ್ನು ಕೊಡುವನು. ನಮ್ಮ ಕಣ್ಣನ್ನು ಕುಕ್ಕುವ ಕಾಂತಿಯನ್ನಲ್ಲ. ಅವನ ಪ್ರೇಮ, ಕರುಣೆಯ ಚಂದ್ರಮನ ಕಾಂತಿ ಪ್ರಪಂಚದಮೇಲೆಲ್ಲ ಬೀಳುವುದು.

ಅವನ ಬಾಯಿ ಅಗ್ನಿಯಂತೆ. ನಾಲಗೆಯನ್ನು ಚಾಚಿ ಉರಿಯುತ್ತಿರುವ ಅಗ್ನಿ ತನಗೆ ಬಿದ್ದುದನ್ನೆಲ್ಲ ಹೇಗೆ ಕಬಳಿಸಿಬಿಡುವುದೊ ಹಾಗೆ ಅವನು ಸಂಹಾರ ಕಾಲದಲ್ಲಿ ಎಲ್ಲವನ್ನೂ ದಹಿಸಿಬಿಡುವನು. ಅವನು ತನ್ನ ತೇಜಸ್ಸಿನಿಂದ ಈ ಪ್ರಪಂಚವನ್ನು ಬೆಳಗಿಸುತ್ತಿರುವನು. ಬೆಳಕು ಒಂದು ವಸ್ತುವಿನ ಮೇಲೆ ಬಿದ್ದಾಗ ಮಾತ್ರ ಅದೇನು ಎಂಬುದು ನಮಗೆ ಗೊತ್ತಾಗಬೇಕಾದರೆ, ಭಗವಂತನ ಕಾಂತಿ ಈ ಸೃಷ್ಟಿಯ ಮೇಲೆ ಬಿದ್ದಾಗ ಮಾತ್ರ ಅದನ್ನು ನಾವು ತಿಳಿದುಕೊಳ್ಳಬಹುದು. ಈ ಬ್ರಹ್ಮಾಂಡವನ್ನು ಬೆಳಗುವುದೇ ಭಗವಂತನ ಕಾಂತಿ.

\begin{verse}
ದ್ಯಾವಾಪೃಥಿವ್ಯೋರಿದಮಂತರಂ ಹಿ \\ ವ್ಯಾಪ್ತಂ ತ್ವಯೈಕೇನ ದಿಶಶ್ಚ ಸರ್ವಾಃ ।\\ದೃಷ್ಟ್ವಾದ್ಭುತಂ ರೂಪಮುಗ್ರಂ ತವೇದಂ \\ ಲೋಕತ್ರಯಂ ಪ್ರವ್ಯಥಿತಂ ಮಹಾತ್ಮನ್ \versenum{॥ ೨ಂ ॥}
\end{verse}

{\small ಆಕಾಶ ಪೃಥ್ವಿಗಳ ಮಧ್ಯ ಈ ಅಂತರದಲ್ಲಿ ಸಕಲ ದಿಕ್ಕುಗಳಲ್ಲಿ ನೀನು ವ್ಯಾಪ್ತನಾಗಿದ್ದೀಯೆ. ಮಹಾತ್ಮ, ನಿನ್ನ ಅದ್ಭುತ ಉಗ್ರರೂಪವನ್ನು ನೋಡಿ ಮೂರು ಲೋಕಗಳೂ ಗಡಗಡ ನಡುಗುತ್ತಿವೆ.}

ಮೇಲೆ ಆಕಾಶ, ಕೆಳಗೆ ಪೃಥ್ವಿ ಮತ್ತು ಮಧ್ಯದಲ್ಲಿರುವ ಲೋಕಗಳಲ್ಲೆಲ್ಲ ಭಗವಂತನೇ ವ್ಯಕ್ತನಾಗಿ ದ್ದಾನೆ. ಅವನು ತುಂಬಿ ತುಳುಕುತ್ತಿದ್ದಾನೆ. ಅವನಿಲ್ಲದ ಸ್ಥಳವೇ ಇಲ್ಲ. ಈ ವಿಶ್ವರೂಪ ಅದ್ಭುತ. ಹಿಂದೆ ಎಂದೂ ಇಂತಹ ರೂಪವನ್ನು ನೋಡಿಲ್ಲ, ಕೇಳಿಲ್ಲ, ಕಲ್ಪಿಸಿಕೊಂಡಿಲ್ಲ, ಇದೊಂದು ಅತ್ಯಾ ಶ್ಚರ್ಯಕರವಾಗಿದೆ, ಉಗ್ರವಾಗಿದೆ. ಮಂದಹಾಸಪೂರಿತನಾದ ದಯಾಮಯನಾದ ಭಗವಂತನ ಮುಖವಲ್ಲ ಇಲ್ಲಿ ಕಾಣುವುದು. ಅವನಿಂದ ಬರುವ ತೇಜಸ್ಸು ಮತ್ತು ವೀರ್ಯದ ಕಿರಣಗಳು ನಮ್ಮ ಕಣ್ಣನ್ನು ಕುಕ್ಕುವಂತಿವೆ. ಬಾಯಿ ಜ್ವಾಲಾಮುಖಿಯಂತೆ ಇದೆ. ಸರ್ವಭಕ್ಷಕನಾದ ಯಮ ಕಾಣುತ್ತಿರು ವನು. ಇವನು ನಮ್ಮಲ್ಲಿ ಭಕ್ತಿಯನ್ನು ಹುಟ್ಟಿಸುವಂತಿಲ್ಲ. ಭಯವನ್ನು ಹುಟ್ಟುಸುತ್ತಿರುವನು. ಅರ್ಜುನನಿಗೆ ಕಾಣುವ ವ್ಯಕ್ತಿಗಳೆಲ್ಲ ಇವನಂತೆ ಆ ಉಗ್ರ ರೂಪವನ್ನು ನೋಡಿ ಭಯಗೊಂಡಿರುವರು.

\begin{verse}
ಅಮೀ ಹಿ ತ್ವಾಂ ಸುರಸಂಘಾ ವಿಶಂತಿ \\ ಕೇಚಿದ್ಭೀತಾಃ ಪ್ರಾಂಜಲಯೋ ಗೃಣಂತಿ ।\\ಸ್ವಸ್ತೀತ್ಯುಕ್ತ್ವಾ ಮಹರ್ಷಿಸಿದ್ಧಸಂಘಾಃ \\ ಸ್ತುವಂತಿ ತ್ವಾಂ ಸ್ತುತಿಭಿಃ ಪುಷ್ಕಲಾಭಿಃ \versenum{॥ ೨೧ ॥}
\end{verse}

{\small ಈ ದೇವತೆಗಳ ಸಂಘಗಳು ನಿನ್ನನ್ನು ಪ್ರವೇಶಿಸುತ್ತಿರುವನು. ಕೆಲವರು ಭೀತರಾಗಿ ನಿನ್ನನ್ನು ಕೈಜೋಡಿಸಿಕೊಂಡು ಸ್ತುತಿಸುತ್ತಿರುವರು. ಮಹರ್ಷಿ ಸಿದ್ಧ ಸಂಘಗಳು “ಜಗತ್ತಿಗೆ ಕಲ್ಯಾಣವಾಗಲಿ” ಎಂದು ನಿನ್ನನ್ನು ಪ್ರಚುರವಾದ ಸ್ತುತಿಗಳಿಂದ ಸ್ತೋತ್ರ ಮಾಡುತ್ತಿರುವರು.}

ಅರ್ಜುನ ತನ್ನ ದೃಶ್ಯದಲ್ಲಿಮೂರು ಬಗೆಯ ಜನರನ್ನು ನೋಡುತ್ತಾನೆ. ಒಬ್ಬರು ಭಗವಂತನಿಂದ ಬಂದ ಸತ್ ಶಕ್ತಿಗಳು. ತಮ್ಮ ಪಾಲಿನ ಕರ್ತವ್ಯಗಳನ್ನು ಮಾಡಿ ಪುನಃ ಅವನನ್ನು ಪ್ರವೇಶಿಸು ತ್ತಿರುವರು. ಮತ್ತೊಬ್ಬರು ಅಂಜಿಕೊಂಡಿರುವರು. ಓಡಿಹೋಗುವುದಕ್ಕೆ ಅವರಿಗೆ ಶಕ್ತಿ ಇಲ್ಲ. ನಿಂತ ಕಡೆಯೇ ನಿಂತುಕೊಂಡು ಭಗವಂತನನ್ನು ಹೊಗಳುತ್ತಿರುವರು. ಈ ಭಕ್ತಿಯ ಹಿಂದೆ ಪ್ರೀತಿಯಲ್ಲ ಇರುವುದು. ಅಂಜಿಕೆಯಿಂದ ತಲ್ಲಣಿಸುತ್ತಿರುವರು. ಮಹರ್ಷಿ ಮತ್ತು ಸಿದ್ಧರ ಸಂಘ ಭಗವಂತನನ್ನು ಅರಿತವರು. ಅವರು ಭಯಗೊಂಡಿಲ್ಲ. ಈ ಉಗ್ರರೂಪಿನ ಹಿಂದೆ ಇರುವ ಪರಮಾತ್ಮನನ್ನು ಸ್ತುತಿಸುತ್ತಿರುವರು. ನರಸಿಂಹ ಕಂಬದಿಂದ ಹೊರಗೆ ಬಂದಾಗ ಅಸುರರು ಓಡಿಹೋದರು. ಆದರೆ ಪ್ರಹ್ಲಾದನಾದರೋ ಆ ಉಗ್ರರೂಪಿನ ಹಿಂದೆ ಇರುವ ಪರಮಾತ್ಮನನ್ನು ಕಂಡು ಅವನನ್ನು ಸ್ತುತಿಸುವನು. ಎಲ್ಲ ಸಿದ್ಧರು ಮಹರ್ಷಿಗಳು ಪರಮಾತ್ಮನನ್ನು ಅರಿತವರು. ಅವರು ಪ್ರಪಂಚಕ್ಕೆ ಒಳ್ಳೆಯದಾಗಲಿ, ಅದು ನಾಶವಾಗದೆ ಇರಲಿ ಎಂದು ಬೇಡುತ್ತಿರುವರು.

\begin{verse}
ರುದ್ರಾದಿತ್ಯಾ ವಸವೋ ಯೇ ಚ ಸಾಧ್ಯಾ \\ ವಿಶ್ವೇಽಶ್ವಿನೌ ಮರುತಶ್ಚೋಷ್ಮಪಾಶ್ಚ ।\\ಗಂಧರ್ವಯಕ್ಷಾಸುರಸಿದ್ಧಸಂಘಾ \\ ವೀಕ್ಷಂತೇ ತ್ವಾಂ ವಿಸ್ಮಿತಾಶ್ಚೈವ ಸರ್ವೇ \versenum{॥ ೨೨ ॥}
\end{verse}

{\small ರುದ್ರರು, ಆದಿತ್ಯರು, ವಸುಗಳು ಸಾಧ್ಯರು ವಿಶ್ವೇದೇವತೆಗಳು, ಅಶ್ವಿನೀದೇವತೆಗಳು, ಮರುದ್ಗಣಗಳು, ಉಷ್ಮ ಪರು, ಗಂಧರ್ವ ಯಕ್ಷಾಸುರ ಸಿದ್ಧಸಂಘಗಳು, ಇವರೆಲ್ಲರೂ ಆಶ್ಚರ್ಯಚಕಿತರಾಗಿ ನಿನ್ನನ್ನೇ ನೋಡುತ್ತಿರುವರು.}

ಇಲ್ಲಿ ಬರುವ ದೇವದೇವತೆಗಳೆಲ್ಲ ಆಗಲೇ ವಿಭೂತಿಯೋಗದಲ್ಲಿ ಬಂದಿವೆ. ಅಲ್ಲಿ ಬರದ ಒಂದೆರಡು ಮಾತ್ರ ಇಲ್ಲಿ ವಿಶೇಷ. ಅದೇ ವಿಶ್ವದೇವಾ ಎಂಬುದು. ಬಹಳ ಹಿಂದೆ ವಿಶ್ವೇದೇವಾ ಎಂಬುದು ಸಮಗ್ರ ದೇವರುಗಳಿಗೆಲ್ಲ ಉಪಯೋಗಿಸುವ ಪದವಾಗಿತ್ತು. ಕಾಲಕ್ರಮೇಣ ಅದಕ್ಕೆ ಬೇರೆ ಅರ್ಥ ಬಂದಿತು. ಪುರಾಣದಲ್ಲಿ ದಕ್ಷನ ಮಗಳಾದ ವಿಶ್ವಾ ಎಂಬುವಳಿಗೆ ಆದ ಮಕ್ಕಳಿಗೆ ವಿಶ್ವೇದೇವಾ ಎಂಬ ಹೆಸರು ಬರುವುದು. ಅವರು ಹತ್ತು ಮಂದಿ ಎಂದು ಭಾವಿಸಿದ್ದರು. ಕೆಲವು ವೇಳೆ ಹನ್ನೆರಡು ಎಂದು ಹೇಳುವರು. ಉಷ್ಮಪ ಎಂಬುದು ಒಂದು ಬಗೆಯ ಪಿತೃಗಳಿಗೆ ಅನ್ವಯಿಸುವುದು. ಉಷ್ಮಪ ಎಂದರೆ ಬಿಸಿಯನ್ನು ಕುಡಿಯುವವರು ಎಂದು. ಎಂದರೆ ಯಾಗ ಯಜ್ಞಗಳ ಸಮಯದಲ್ಲಿ ಅವರಿಗೆ ಬಿಸಿಯಾಗಿ ಹಾಕಿದ ಹವಿಸ್ಸುಗಳನ್ನು ಭಕ್ಷಿಸುವ ಪಿತೃಗಳು ಎಂದು ಅರ್ಥ. ಸಾಧ್ಯರು ದೇವತೆಗಳಿಗೆ ಊಳಿಗ ಮಾಡುವ ಒಂದು ಬಗೆಯ ಜನರು.

\begin{verse}
ರೂಪಂ ಮಹತ್ತೇ ಬಹುವಕ್ತ್ರನೇತ್ರಂ \\ ಮಹಾಬಾಹೋ ಬಹುಬಾಹೂರುಪಾದಮ್ ।\\ಬಹೂದರಂ ಬಹುದಂಷ್ಟ್ರಾಕರಾಲಂ \\ ದೃಷ್ಟ್ವಾ ಲೋಕಾಃ ಪ್ರವ್ಯಥಿತಾಸ್ತಥಾಹಮ್ \versenum{॥ ೨೩ ॥}
\end{verse}

{\small ಮಹಾಬಾಹು, ಅನೇಕ ಮುಖ, ಕಣ್ಣು, ಕೈ, ತೊಡೆ, ಕಾಲು, ಹೊಟ್ಟೆ, ಕೋರೆದಾಡೆಗಳಿಂದ ಕರಾಳವಾಗಿ ಕಾಣುವ ನಿನ್ನ ಮಹದ್ರೂಪವನ್ನು ನೋಡಿ ಲೋಕಗಳು ಮತ್ತು ನಾನು ಭಯಗೊಂಡಿದ್ದೇವೆ.}

ಅರ್ಜುನ ಮೊದಲು ವಿಶ್ವರೂಪವನ್ನು ನೋಡಬೇಕೆಂದು ಬಯಸಿದನು. ಭಗವಂತ ಅವನ ಬಯಕೆಯನ್ನು ಈಡೇರಿಸಲು ತೋರಿಸಿದನು. ಆದರೆ ಆ ಭಯಾನಕವಾದ ಉಗ್ರರೂಪವನ್ನು ನೋಡಿ ತತ್ತರಿಸುತ್ತಾನೆ. ಹಾಗೆಯೇ ಅವನ ಕಣ್ಣಿಗೆ ಕಾಣುವ ಇತರರು ಕೂಡ ಅಂಜಿಕೆಯಿಂದ ನಡುಗುತ್ತಿರು ವಂತೆ ಕಾಣುವರು.

\begin{verse}
ನಭಃಸ್ಪೃಶಂ ದೀಪ್ತಮನೇಕವರ್ಣಂ \\ ವ್ಯಾತ್ತಾನನಂ ದೀಪ್ತವಿಶಾಲನೇತ್ರಮ್ ।\\ದೃಷ್ಟ್ವಾ ಹಿ ತ್ವಾಂ ಪ್ರವ್ಯಥಿತಾಂತರಾತ್ಮಾ \\ ಧೃತಿಂ ನ ವಿಂದಾಮಿ ಶಮಂ ಚ ವಿಷ್ಣೋ \versenum{॥ ೨೪ ॥}
\end{verse}

{\small ವಿಷ್ಣುವೆ, ಅಂತರಿಕ್ಷವನ್ನು ಮುಟ್ಟಿರುವ, ತೇಜೋಮಯನಾದ, ಅನೇಕ ವರ್ಣಗಳುಳ್ಳ, ತೆರೆಯಲ್ಪಟ್ಟ ಬಾಯಿ ಗಳುಳ್ಳ ಉಜ್ವಲ ವಿಶಾಲ ನೇತ್ರಗಳುಳ್ಳ ನಿನ್ನನ್ನು ನೋಡಿ ನನ್ನ ಮನಸ್ಸು ಭೀತಗೊಂಡಿದೆ. ಧೈರ್ಯವನ್ನು ಮತ್ತು ಶಾಂತಿಯನ್ನು ಹೊಂದದೆ ಇರುವೆನು.}

ಅರ್ಜುನ ನನ್ನ ಮನಸ್ಸು ಭಯಗೊಂಡಿದೆ ಎನ್ನುತ್ತಾನೆ. ಭಗವಂತನ ಉಗ್ರರೂಪವನ್ನು ನೋಡಿ ಸಂತೋಷಪಡುವುದಿಲ್ಲ. ಕೇಳಿದವನು ಇವನೇ ನನಗೆ ನಿನ್ನ ವಿಶ್ವರೂಪವನ್ನು ತೋರು ಎಂದು. ಆದರೆ ಅವನು ಅದನ್ನು ಮನಸ್ಸಿನಲ್ಲಿ ಬೇರೆ ಬಗೆಯಾಗಿ ಚಿತ್ರಿಸಿಕೊಂಡಿದ್ದ ಎಂದು ತೋರುವುದು. ಆದರೆ ಎದುರಿಗೆ ಕಾಣುವ ವಿರಾಟ್ ರೂಪವನ್ನು ನೋಡುವುದಕ್ಕೆ ಅವನು ತಯಾರಾಗಿರಲಿಲ್ಲ. ನಮಗೆಲ್ಲ ಸುಂದರವಾಗಿರುವ ಸತ್ಯ ಬೇಕಾಗಿದೆ. ಅದಕ್ಕೇ ರುಂಡಮಾಲೆಯ ಬದಲು ವನಮಾಲೆಯನ್ನು ಹಾಕು ತ್ತೇವೆ. ವೀರಾಧಿವೀರರು ಮಾತ್ರ ಭಗವಂತ ಹೇಗಿರುವನೋ ಹಾಗೆ ಆರಾಧಿಸಬಲ್ಲರು. ಉಳಿದವರಿಗೆ ಇದೇನಾದರೂ ಆದರೆ ಅದರಿಂದ ತತ್ತರಿಸುತ್ತಾರೆ. ಆದಕಾರಣವೆ ಮೊದಲು ಆಧ್ಯಾತ್ಮ ಜೀವನದಲ್ಲಿ ಮುಂದುವರಿದು ಅನಂತರ ಪಡೆದುಕೊಳ್ಳುವುದು ಎನ್ನುವುದು. ಶ್ರೀರಾಮಕೃಷ್ಣರ ಸೇವೆ ಮಾಡುತ್ತಿದ್ದ ಮಥುರನಾಥ ಎಂಬ ಶ್ರೀಮಂತ ಶ್ರೀರಾಮಕೃಷ್ಣರು ಭಗವದ್ಭಕ್ತಿಯಿಂದ ತಲ್ಲೀನರಾಗುತ್ತಿದ್ದುದನ್ನು ನೋಡುತ್ತಿದ್ದ. ಆಗ ಅವರಿಗೆ ಬಾಹ್ಯಪ್ರಪಂಚದ ಅರಿವೇ ಇರುತ್ತಿರಲಿಲ್ಲ. ಮಥುರನಾಥ ಈ ಸ್ಥಿತಿಯನ್ನು ನೋಡಿ ಪರವಶನಾದ. ನನಗೂ ಇದು ಬಂದರೆ ಎಷ್ಟು ಚೆನ್ನಾಗಿರುವುದು ಎಂದು ಭಾವಿಸಿ ಶ್ರೀರಾಮಕೃಷ್ಣರಿಗೆ ಆ ಅನುಭವವನ್ನು ಕರುಣಿಸಬೇಕೆಂದು ಕೇಳಿಕೊಂಡ. ಶ್ರೀರಾಮಕೃಷ್ಣರು ಎಲ್ಲಾ ಸಕಾಲದಲ್ಲಿ ಬರುವುದು; ಅದಕ್ಕೆ ಆತುರ ಪಡಬೇಡ ಎಂದರು. ಸಕಾಲ ಎಂದರೆ ಮನಸ್ಸು ಪರಿಶುದ್ಧವಾಗಿ ಅಣಿಯಾದಮೇಲೆ ಎಂದು ಅರ್ಥ. ಆದರೆ ಮಥುರನಾಥನಿಗೆ ಆತುರ. ಅದನ್ನು ಬೇಗ ಕರುಣಿಸಬೇಕೆಂದು ಕಾಡಿದ. ಶ್ರೀರಾಮಕೃಷ್ಣರು ದೇವರ ಇಚ್ಛೆ ಇದ್ದರೆ ನಿನಗೆ ಲಭಿಸಲಿ ಎಂದು ಹೇಳಿದರು. ಒಂದು ದಿನ ಮಥುರ ಬೆಳಗ್ಗೆ ಎದ್ದ. ಎಲ್ಲಾ ಭಗವಂತನಿಂದ ತುಂಬಿ ತುಳುಕಾಡುವಂತೆ ಕಂಡ. ತನ್ನ ಒಳಗೆ ಹೊರಗೆ ಎಲ್ಲ ಅವನು ಹಾಸುಹೊಕ್ಕಾಗಿರುವನು. ಮಥುರನಿಗೆ ಸಂಸಾರದ ಯಾವ ಕೆಲಸವನ್ನೂ ಮಾಡುವುದಕ್ಕೆ ಆಗಲಿಲ್ಲ. ಜಮೀನ್​ದಾರಿಗೆ ಸಂಬಂಧಪಟ್ಟ ಹಲವು ತುರ್ತು ಕೆಲಸ ಗಳಿದ್ದುವು. ಆತ ರಾಮಕೃಷ್ಣರ ಬಳಿಗೆ ಓಡಿ, “ದಯವಿಟ್ಟು ನನ್ನನ್ನು ದಿವ್ಯಭಾವದಿಂದ ಪಾರುಮಾಡಿ. ಈ ಸ್ಥಿತಿ ನನ್ನನ್ನು ಮೆಟ್ಟಿಕೊಂಡರೆ ನಾನು ಯಾವ ಕೆಲಸವನ್ನೂ ಮಾಡುವಂತಿಲ್ಲ” ಎಂದು ಬೇಡಿದ. ಮುಂಚೆ ಈ ಅವಸ್ಥೆಗೆ ಕಾಡಿದವನು ಇವನೆ. ಆದರೆ ಈಗ ಅದರಿಂದ ಪಾರುಮಾಡಬೇಕೆಂದು ಬೇಡುತ್ತಿರುವನು. ಶ್ರೀರಾಮಕೃಷ್ಣರು ನಗುತ್ತಾ ನೀನೆ ಅಲ್ಲವೆ ಕೇಳಿದ್ದು. ಅದಕ್ಕೆ ಅದು ಬಂತು ಎಂದರು. ಆಗ ಮಥುರ, “ಹೌದು, ನಾನೇ ಕೇಳಿದ್ದು. ಆದರೆ ಅದು ಹೀಗೆ ಪಿಶಾಚಿಯಂತೆ ನನ್ನನ್ನು ಮೆಟ್ಟಿಕೊಳ್ಳುವುದೆಂದು ಹೇಗೆ ಗೊತ್ತಾಗಬೇಕು” ಎಂದು ಹೇಳಿದ. ಶ್ರೀರಾಮಕೃಷ್ಣರು ಆ ಮಾತನ್ನು ಕೇಳಿ ಅವನನ್ನು ಭಾವಮುಖದಿಂದ ಪಾರುಮಾಡಿದರು. ಅರ್ಜುನ ಕೂಡ ಇದೇ ಸ್ಥಿತಿಯಲ್ಲಿರುವನು. ವಿಶ್ವರೂಪವನ್ನು ತೋರೆಂದು ಕೇಳಿಕೊಂಡವನು ಇವನು. ಭಗವಂತ ಕೊಟ್ಟ, ಆದರೆ ಅದನ್ನು ನೋಡಿದಾಗ ತರಗೆಲೆ ಬಿರುಗಾಳಿಗೆ ತತ್ತರಿಸುವಂತೆ ತತ್ತರಿಸುತ್ತಾನೆ. ಏಕೆಂದರೆ ಇನ್ನೂ ಅವನು ಆಧ್ಯಾತ್ಮಿಕ ಸಾಧನೆಮಾಡಿ ಯೋಗ್ಯತೆಯನ್ನು ಪಡೆದುಕೊಂಡಿಲ್ಲ. ಅದಕ್ಕೇ ಅರ್ಜುನನ ಧೈರ್ಯ ಮತ್ತು ಶಾಂತಿ ಕುಂದುವುದು.

\begin{verse}
ದಂಷ್ಟ್ರಾಕರಾಲಾನಿ ಚ ತೇ ಮುಖಾನಿ \\ ದೃಷ್ವೈವ ಕಾಲಾನಲಸನ್ನಿಭಾನಿ ।\\ದಿಶೋ ನ ಜಾನೇ ನ ಲಭೇ ಚ ಶರ್ಮ \\ ಪ್ರಸೀದ ದೇವೇಶ ಜಗನ್ನಿವಾಸ \versenum{॥ ೨೫ ॥}
\end{verse}

{\small ದೇವೇಶ, ಕೋರೆ ದಾಡೆಗಳಿಂದ ಕರಾಳವಾಗಿರುವ ಪ್ರಳಯಾಗ್ನಿಯಂತಿರುವ ನಿನ್ನ ಮುಖಗಳನ್ನು ನೋಡಿ ದಿಕ್ಕು ಕಾಣದೆ ಇರುವೆನು, ಶಾಂತಿ ಹೊಂದದೇ ಇರುವೆನು. ಜಗನ್ನಿವಾಸ, ಪ್ರಸನ್ನನಾಗು.}

ಭಗವಂತನ ಪ್ರಳಯಾಗ್ನಿಯಂತಿರುವ ಮುಖವನ್ನು ನೋಡಿ ದಿಕ್ಕುಕಾಣದೆ ಇರುವನು. ಅವನ ನೆಮ್ಮದಿಗೆ ಭಂಗವಾಯಿತು. ಈ ಅನುಭವ ದೊಡ್ಡದೊಂದು ವಿದ್ಯುತ್ ಶಕ್ತಿಯ ಪಾತದಂತೆ ಅವನಿಗೆ ತಾಕಿತು. ಇದಕ್ಕೆ ಅವನ ಮನಸ್ಸು ಸಿದ್ಧವಾಗಿರಲಿಲ್ಲ. ಭಗವಂತನ ಸೌಂದರ್ಯದ ಭಾಗವನ್ನು ನೋಡಿ ಆನಂದಿಸುವುದು ಸುಲಭ. ಅವನ ಉಗ್ರ ರೂಪವನ್ನು ನೋಡಬೇಕಾದರೆ ದಿಟ್ಟತನ ಬೇಕು. ನಾವು ದೇವರ ವಿಷಯದಲ್ಲಿ ಕಲ್ಪಿಸಿಕೊಂಡ ಸುಂದರ ಭಾವನೆಗಳೆಲ್ಲ ಉದುರಿ ಹೋಗುವುದು ಕರಾಳ ಸತ್ಯವನ್ನು ನೋಡಿದಾಗ. ಅದನ್ನು ತಿಳಿದುಕೊಳ್ಳಬೇಕಾದರೆ, ಅದನ್ನು ಮೆಚ್ಚಬೇಕಾದರೆ, ಅದಕ್ಕೆ ಸಿದ್ಧ ರಾಗಿರಬೇಕು. ಅರ್ಜುನ ದೇವರೆ, ಪ್ರಸನ್ನನಾಗು ಎಂದು ಬೇಡುತ್ತಿರುವನು. ನಿನ್ನ ಕೋಮಲ ಸ್ವಭಾವವನ್ನು ತೋರು. ನಿನ್ನ ಭೀಕರ ಮುಖವಾಡವನ್ನು ತೆಗೆದು ಹಾಕು ಎಂದು ಬೇಡುತ್ತಿರುವನು.

\begin{verse}
ಅಮೀ ಚ ತ್ವಾಂ ಧೃತರಾಷ್ಟ್ರಸ್ಯ ಪುತ್ರಾಃ \\ ಸರ್ವೇ ಸಹೈವಾವನಿಪಾಲಸಂಘೈಃ ।\\ಭೀಷ್ಮೋ ದ್ರೋಣಃ ಸೂತಪುತ್ರಸ್ತಥಾಸೌ \\ ಸಹಾಸ್ಮದೀಯೈರಪಿ ಯೋಧಮುಖ್ಯೈಃ \versenum{॥ ೨೬ ॥}
\end{verse}

\begin{verse}
ವಕ್ತ್ರಾಣಿ ತೇ ತ್ವರಮಾಣಾ ವಿಶಂತಿ \\ ದಂಷ್ಟ್ರಾಕರಾಲಾನಿ ಭಯನಕಾನಿ ।\\ಕೇಚಿದ್ವಿಲಗ್ನಾ ದಶನಾಂತರೇಷು \\ ಸಂದೃಶ್ಯಂತೇ ಚೂರ್ಣಿತೈರುತ್ತಮಾಂಗೈಃ \versenum{॥ ೨೭ ॥}
\end{verse}

{\small ರಾಜಸಮೂಹಗಳಿಂದ ಕೂಡಿದ ಧೃತರಾಷ್ಟ್ರನ ಸಮಸ್ತ ಪುತ್ರರು, ಭೀಷ್ಮ ದ್ರೋಣ ಮತ್ತು ಕರ್ಣ ಇವರೆಲ್ಲ ನಮ್ಮ ಮುಖ್ಯರಾದ ಯೋಧರಿಂದ ಕೂಡಿ ಕೋರೆದಾಡೆಗಳಿಂದ ಕರಾಳವೂ ಭಯಂಕರವೂ ಆದ ನಿನ್ನ ಮುಖಗಳನ್ನು ತ್ವರೆಯಿಂದ ಪ್ರವೇಶಿಸುತ್ತಿರುವರು. ಕೆಲವರು ನಿನ್ನ ಹಲ್ಲಿನ ಸಂದುಗಳಲ್ಲಿ ಸಿಕ್ಕಿಕೊಂಡು ಪುಡಿಪುಡಿಯಾದ ತಲೆಗಳಿಂದ ಕಾಣುತ್ತಿರುವರು.}

ಶ್ರೀಕೃಷ್ಣ ಅರ್ಜುನನಿಗೆ ಮುಂದೆ ಏನಾಗುವುದೋ ಅವನ್ನು ಕೂಡ ತೋರುತ್ತಿರುವನು. ಈ ಯುದ್ಧದಲ್ಲಿ ಕೊನೆಗೆ ಯಾರಾದರೊಬ್ಬರು ಸೋಲುವುದು ನಿಜವಾದರೂ ಪಾಂಡವರ ಕಡೆ ಇರುವ ಅತಿರಥ ಮಹಾರಥರುಗಳಲ್ಲಿ ಹಲವರು ಕೌರವರೊಡನೆ ಕಾಲ ರುದ್ರನ ತೆರೆದ ಬಾಯನ್ನು ತ್ವರೆ ಯಿಂದ ಪ್ರವೇಶಿಸುತ್ತಿರುವರು. ಹಲವರು ಆಗಲೆ ಅವನ ಹಲ್ಲಿನ ಸಂದಿಯಲ್ಲಿ ಸಿಕ್ಕಿಕೊಂಡು ಜಜ್ಜಿ ಪುಡಿಯಾಗುತ್ತಿರುವರು. ಭವಿಷ್ಯದ ದೃಶ್ಯ ಆಗಲೇ ಕಾಣುತ್ತಿದೆ. ಅರ್ಜುನ ತಾನು ಕೊಂದರೆ ಮಾತ್ರ ಕೌರವರು ಸಾಯುವರು ಎಂದು ಭಾವಿಸಿದ್ದ. ಇವನು ಕೊಲ್ಲಲಿ, ಕೊಲ್ಲದಿರಲಿ ನಾಶವಾಗುವುದಂತೂ ನಿಲ್ಲುವಹಾಗಿಲ್ಲ. ದೇವರು ಆಗಲೆ ಅದನ್ನು ನಿಶ್ಚಯಿಸಿರುವನು. ಅವನು ನಿಶ್ಚಯಿಸಿದಂತೆ ಆಗಿಯೇ ಆಗುವುದು. ಅವನಿಚ್ಛೆಗೆ ಯಾರೂ ಅಡ್ಡಿಯನ್ನು ತರಲಾರರು.

\begin{verse}
ಯಥಾ ನದೀನಾಂ ಬಹುವೋಂಬುವೇಗಾಃ \\ ಸಮುದ್ರಮೇವಾಭಿಮುಖಾ ದ್ರವಂತಿ ।\\ತಥಾ ತವಾಮೀ ನರಲೋಕವೀರಾ \\ ವಿಶಂತಿ ವಕ್ತ್ರಾಣ್ಯಭಿವಿಜ್ವಲಂತಿ \versenum{॥ ೨೮ ॥}
\end{verse}

{\small ಹೇಗೆ ನದಿಗಳ ಮಹಾ ಪ್ರವಾಹ ಸಮುದ್ರದ ಕಡೆಗೆ ಹರಿಯುತ್ತಿರುವುದೊ ಹಾಗೆಯೇ ಈ ನರಲೋಕವೀರರು ಉರಿಯುತ್ತಿರುವ ನಿನ್ನ ಮುಖವನ್ನು ಪ್ರವೇಶಿಸುತ್ತಿರುವರು.}

ಕೌರವ ಪಾಂಡವರ ಸೇನೆ ಹೇಗೆ ಕಾಲನ ಬಾಯನ್ನು ಪ್ರವೇಶಿಸುತ್ತಿರುವುದೊ ಅದನ್ನು ಸುಂದರ ವಾದ ಉಪಮಾನದ ಮೂಲಕ ವಿವರಿಸುವನು. ಹೇಗೆ ತುಂಬು ಪ್ರವಾಹದಿಂದ ಹರಿಯುವ ನದಿಗಳು ಸಮುದ್ರವನ್ನು ಸೇರಿ ಅದರಲ್ಲಿ ಮಾಯವಾಗುವುವೋ ಹಾಗೆ ಸೇನಾ ಸಮೂಹ ದಳ್ಳುರಿ ಏಳುತ್ತಿರುವ ಭಗವಂತನ ಬಾಯನ್ನು ಪ್ರವೇಶಿಸಿ ನಾಮಾವಶೇಷವಾಗಿ ಹೋಗುತ್ತಿರುವುದು.

\begin{verse}
ಯಥಾ ಪ್ರದೀಪ್ತಂ ಜ್ವಲನಂ ಪತಂಗಾ \\ ವಿಶಂಶಿ ನಾಶಾಯ ಸಮೃದ್ಧವೇಗಾಃ ।\\ತಥೈವ ನಾಶಾಯ ವಿಶಂತಿ ಲೋಕಾ—\\ ಸ್ತವಾಪಿ ವಕ್ತ್ರಾಣಿ ಸಮೃದ್ಧವೇಗಾಃ \versenum{॥ ೨೯ ॥}
\end{verse}

{\small ಹೇಗೆ ಪತಂಗಗಳು ಹಾರಿಬಂದು ಉರಿವ ದೀಪದಲ್ಲಿ ಬೀಳುವುವೊ ಹಾಗೆಯೇ ಈ ಜನರು ಕೂಡ ವೇಗದಿಂದ ತಮ್ಮ ನಾಶಕ್ಕಾಗಿ ನಿನ್ನ ಮುಖವನ್ನು ಪ್ರವೇಶಿಸುತ್ತಿರುವರು.}

ಇಲ್ಲಿ ನಮ್ಮ ಮನಸ್ಸಿಗೆ ನಾಟುವಂತಹ ಮತ್ತೊಂದು ಉಪಮಾನವನ್ನು ಕೊಡುವನು. ದೂರದಲ್ಲಿ ಕತ್ತಲೆಯಲ್ಲಿ ಪತಂಗ ಹಾರಾಡುತ್ತಿರುವುದು. ಉರಿವ ಜ್ವಾಲೆಯೊಂದು ಅದಕ್ಕೆ ಕಾಣುವುದು. ಆ ಜ್ವಾಲೆಯಮೇಲೆ ಪತಂಗಕ್ಕೆ ಅಂತಹ ಆಕರ್ಷಣೆ. ಅಲ್ಲಿಂದ ದೌಡಾಯಿಸಿ ಬಂದು ಜ್ವಾಲೆಯ ಸುತ್ತಲೂ ಹಾರಾಡಿ ಕೊನೆಗೆ ಅದರಲ್ಲಿ ಬಿದ್ದು ಸಾಯುವುದು. ಹಾಗೆಯೆ ಸೇನೆಗಳು ಪತಂಗಗಳಂತೆ ಮೃತ್ಯು ಶಿಖೆಗೆ ಬಂದು ಬೀಳುತ್ತಿವೆ.

\begin{verse}
ಲೇಲಿಹ್ಯಸೇ ಗ್ರಸಮಾನಃ ಸಮಂತಾ—\\ ಲ್ಲೋಕಾನ್ ಸಮಗ್ರಾನ್ ವದನೈರ್ಜ್ವಲದ್ಭಿಃ ।\\ತೇಜೋಭಿರಾಪೂರ್ಯ ಜಗತ್ಸಮಗ್ರಂ \\ ಭಾಸಸ್ತವೋಗ್ರಾಃ ಪ್ತತಪಂತಿ ವಿಷ್ಣೋ \versenum{॥ ೩೦ ॥}
\end{verse}

{\small ಎಲ್ಲಾ ಲೋಕಗಳನ್ನೂ ಗ್ರಾಸವಾಗಿ ತೆಗೆದುಕೊಳ್ಳುತ್ತ ಉರಿಯುತ್ತಿರುವ ಮುಖದಿಂದ ನೀನು ನೆಕ್ಕುತ್ತಿದ್ದೀಯೆ. ಸರ್ವವ್ಯಾಪಿಯಾದ ವಿಷ್ಣುವೆ, ನಿನ್ನ ಉಗ್ರ ಪ್ರಕಾಶ ಸಮಗ್ರ ಜಗತ್ತನ್ನು ಬೆಳಗಿ ತಪಿಸುತ್ತಿದೆ.}

ಮತ್ತೂ ಭೀಕರವಾದ ಇನ್ನೊಂದು ಚಿತ್ರವನ್ನು ಕೊಡುವನು. ಲೋಕಗಳನ್ನೆಲ್ಲ ಅನ್ನದ ತುತ್ತಿನಂತೆ ನುಂಗುತ್ತಿರುವನು. ಮುಖವೋ ಜ್ವಾಲಾಮಯವಾಗಿದೆ. ನುಂಗುವುದು ಮಾತ್ರವಲ್ಲ, ನುಂಗಿ ಚಪ್ಚರಿ ಸುತ್ತಿರುವನು. ಈ ಧ್ವಂಸಲೀಲೆಯಲ್ಲಿ ಅಂತಹ ಆಸಕ್ತಿ. ಸರ್ವವ್ಯಾಪಿಯಾಗಿರುವ ಭಗವಂತ ಪ್ರಪಂಚವನ್ನೆಲ್ಲ ಬೆಳಗುತ್ತಿರುವನು. ಆ ಬೆಳಕಿನಲ್ಲಿ ಈ ಪ್ರಪಂಚ ಕಂಪಿಸುತ್ತಿದೆ, ತಮ್ಮ ಸರದಿ ಯಾವಾಗ ಬರುವುದೊ, ಮೃತ್ಯು ಬಾಯಿಗೆ ಪ್ರವೇಶ ಮಾಡುವುದಕ್ಕೆ ಎಂದು.

\begin{verse}
ಆಖ್ಯಾಹಿ ಮೇ ಕೋ ಭವಾನುಗ್ರರೂಪೋ \\ ನಮೋಽಸ್ತು ತೇ ದೇವವರ ಪ್ರಸೀದ ।\\ವಿಜ್ಞಾತುಮಿಚ್ಛಾಮಿ ಭವಂತಮಾದ್ಯಂ \\ ನ ಹಿ ಪ್ರಜಾನಾಮಿ ತವ ಪ್ರವೃತ್ತಿಮ್ \versenum{॥ ೩೧ ॥}
\end{verse}

{\small ಉಗ್ರರೂಪಿ, ನೀನು ಯಾರು ಹೇಳು. ದೇವದೇವ, ನಿನಗೆ ನಮಸ್ಕಾರ. ಪ್ರಸನ್ನನಾಗು, ಆದಿಪುರುಷನಾದ ನಿನ್ನನ್ನು ತಿಳಿಯಲು ಇಚ್ಛಿಸುತ್ತೇನೆ. ನಿನ್ನ ಪ್ರವೃತ್ತಿಯನ್ನು ನಾನು ಅರಿಯೆನು.}

ಅರ್ಜುನನೇ ಕೇಳಿದ್ದು ಭಗವಂತನ ವಿಶ್ವರೂಪವನ್ನು ನೋಡಬೇಕೆಂದು. ಈಗ ನೀನು ಯಾರು ಎಂದು ಕೇಳುತ್ತಾನೆ. ಏಕೆಂದರೆ ಇವನು ಮನದಲ್ಲಿ ಬೇರೊಂದನ್ನು ಕಲ್ಪಿಸಿಕೊಂಡಿದ್ದನು. ಈಗ ಇವನಿಗೆ ಕಾಣುವುದು ಭಯಾನಕವಾಗಿರುವುದು. ಆದಕಾರಣವೇ ಇದನ್ನೇ ಏನು ತಾನು ನೋಡ ಬೇಕೆಂದು ಬಯಸಿದ್ದು ಎಂಬ ಸಂದೇಹದಿಂದ ಕೇಳುತ್ತಾನೆ. ನೀನು ಪ್ರಸನ್ನನಾಗು ಎಂದು ಕೇಳುತ್ತಾನೆ. ಈಗ ಕೋಪಾನಲನಂತೆ ವಿಶ್ವವನ್ನೆಲ್ಲ ನುಂಗುತ್ತಿರುವ ಭೈರವ ಮೂರ್ತಿಯಾಗಿರುವನು. ಅದರ ಹಿಂದೆ ಇರುವ ಸೌಮ್ಯ ಮೂರ್ತಿಯನ್ನು ನೋಡಬೇಕೆಂದು ಬಯಸುವನು. ಭಗವಂತ ಏತಕ್ಕೆ ಇಂತಹ ಕಾರ್ಯದಲ್ಲಿ ತೊಡಗಿದ್ದಾನೆ ಎಂಬುದನ್ನು ತಿಳಿಯಲು ಆಶಿಸುತ್ತಾನೆ. ಭಗವಂತ ತನ್ನ ಕಾಲರುದ್ರನ ರೂಪವನ್ನು ಅರ್ಜುನನಿಗೆ ತೋರಿಸಿ, ಅವನನ್ನು ಕೂಡ ಸಂಹಾರ ಕಾರ್ಯಕ್ಕೆ ನಿಮಿತ್ತವಾಗಿ ಉಪಯೋಗಿಸಬೇಕೆಂದಿರುವನು. ಅರ್ಜುನನ ಮನಸ್ಸನ್ನು ಅದಕ್ಕೆ ಅಣಿಮಾಡಬೇಕಾಗಿದೆ. ಅದನ್ನು ಭಗವಂತನ ಬಾಯಿಂದಲೇ ಕೇಳ ಬಯಸುವನು. ಆಗಲೇ ಭಗವಂತ ತನ್ನ ಸ್ವರೂಪದ ಅರ್ಥವನ್ನು ವಿವರಿಸುವನು.

ಭಗವಂತ ಹೀಗೆ ಹೇಳುತ್ತಾನೆ:

\begin{verse}
ಕಾಲೋಽಸ್ಮಿ ಲೋಕಕ್ಷಯಕೃತ್ಪ್ರವೃದ್ಧೋ \\ ಲೋಕಾನ್ ಸಮಾಹರ್ತುಮಿಹ ಪ್ರವೃತ್ತಃ\\ಪುತೇಽಪಿ ತ್ವಾ ನ ಭವಿಷ್ಯಂತಿ ಸರ್ವೇ \\ ಯೇಽವಸ್ಥಿತಾಃ ಪ್ರತ್ಯನೀಕೇಶು ಯೋಧಾಃ \versenum{॥ ೩೨ ॥}
\end{verse}

{\small ಲೋಕಗಳನ್ನು ನಾಶಮಾಡಲು ಹೊರಟಿರುವ ಕಾಲ ನಾನು, ಲೋಕಗಳನ್ನೆಲ್ಲಾ ನಾಶಗೊಳಿಸಲೆಂದೇ ಇಲ್ಲಿಗೆ ಬಂದಿದ್ದೇನೆ. ಈ ಸೇನೆಗಳೆಲ್ಲ, ಯುದ್ಧಕ್ಕಾಗಿ ಬಂದಿರುವ ಈ ಯೋಧರೆಲ್ಲರೂ, ನೀನು ಯುದ್ಧ ಮಾಡೆನೆಂದರೂ ಬದುಕುವವರಲ್ಲ. }

ಶ್ರೀಕೃಷ್ಣ ಈಗ ಅರ್ಜುನನಿಗೆ ತೋರುತ್ತಿರುವ ವಿಶ್ವರೂಪ ಸಂಹಾರಕನ ರೂಪ. ಅವನು ಹಿಂದೆ ತೋರಿದ ವಿಶ್ವರೂಪಗಳಲ್ಲಿ ಇದು ಇರಲಿಲ್ಲ. ಅವನು ತಾನು ಸರ್ವರಲ್ಲಿಯೂ ಇದ್ದೇನೆ, ಸರ್ವವೂ ತನ್ನಲ್ಲಿವೆ ಎಂಬ ಭಾವವನ್ನು ತೋರಿದ. ಅದು ನೋಡುವುದಕ್ಕೆ ಅಷ್ಟು ಭಯಾನಕವಲ್ಲ. ಸಣ್ಣ ಮಗುವಾದಾಗ ಶ್ರೀಕೃಷ್ಣ ತನ್ನ ಬಾಯನ್ನು ಯಶೋದೆಗೆ ತೋರಿದ. ಅವಳು ಅಲ್ಲಿ ಬ್ರಹ್ಮಾಂಡವನ್ನು ಕಂಡಳು. ದುರ್ಯೋಧನನಿಗೆ ಸಂಧಿಯ ಸಮಯದಲ್ಲಿ ವಿಶ್ವರೂಪವನ್ನು ತೋರಿದ. ಅಲ್ಲಿ ಎಲ್ಲರೂ ಕೃಷ್ಣನಂತೆ ಕಂಡರು. ಕೃಷ್ಣನಲ್ಲದವರು ಯಾರೂ ಇರಲಿಲ್ಲ. ಇದೊಂದು ವಿಶ್ವರೂಪ. ಆದರೆ ಇಲ್ಲಿ, ಕುರುಕ್ಷೇತ್ರದಲ್ಲಿ, ಅರ್ಜುನನಿಗೆ ತೋರುತ್ತಿರುವ ರೂಪವಾದರೊ ಸಂಹಾರಕನ ವಿಶ್ವರೂಪ. ಭಗ ವಂತ ಆಗಲೇ ಸಂಹಾರಕನ ಉಡುಪಿನಲ್ಲಿದ್ದಾನೆ. ಆಗ ಅವನನ್ನು ನೋಡುವುದಕ್ಕೆ ಬಹಳ ಧೈರ್ಯ ಬೇಕು. ಪ್ರಪಂಚ ಸಂಹಾರಕ್ಕೆ ಅಣಿಯಾಗಿದೆ. ಇವನು ಕೂಡ ಅದನ್ನು ಮಾಡಲು ಉದ್ಯುಕ್ತನಾಗಿರು ವನು. ತೆನೆ ಹೊಲದಲ್ಲಿ ಬಲಿತು ನಿಂತಿದೆ, ರೈತ ಕೂಡ ತನ್ನ ಕುಡುಗೋಲನ್ನು ತೆಗೆದುಕೊಂಡು ಬಂದಿರುವನು ತೆನೆಗಳನ್ನೆಲ್ಲ ಕತ್ತರಿಸುವುದಕ್ಕೆ.

ಶತ್ರುಸೇನೆಯಲ್ಲಿ ಯಾರಿರುವರೊ ಅವರು ನೀನಿಲ್ಲದೇ ಇದ್ದರೂ ಜೀವಿಸಲಾರರು ಎನ್ನುವನು. ಅರ್ಜುನ ಭಾವಿಸಿದ್ದು ತಾನು ಯುದ್ಧ ಮಾಡಿ ಅವರನ್ನು ಕೊಲ್ಲುತ್ತೇನೆ, ಇದರಿಂದ ತನಗೆ ಪಾಪ ಬರುವುದು ಎಂದು. ಆದರೆ ನೀನು ಯುದ್ಧ ಮಾಡದೇ ಹೊರಟು ಹೋದರೂ ಈ ಕೊಲೆಯ ಕೆಲಸವಂತೂ ನಿಲ್ಲುವಂತಿಲ್ಲ. ಭಗವಂತನಿಗೆ ತನ್ನ ಕೆಲಸವನ್ನು ಮಾಡುವುದಕ್ಕೆ ವಸ್ತುಗಳಿಗೆ, ವ್ಯಕ್ತಿಗಳಿಗೆ ಅಭಾವವಿಲ್ಲ. ಅವನು ಯಾವುದನ್ನಾದರೂ ನಿಮಿತ್ತವಾಗಿ ತೆಗೆದುಕೊಂಡು ಅದರ ಮೂಲಕ ತನ್ನ ಕೆಲಸವನ್ನು ಮಾಡುತ್ತಾನೆ. ಈ ಬಗೆಯ ವಿಶ್ವರೂಪವನ್ನು ಅರ್ಜುನನಿಗೆ ತೋರ ಬೇಕಾಗಿತ್ತು. ಆಗಲೇ ಅವನಲ್ಲಿರುವ ಪಾಪಭೀತಿ ಓಡಬೇಕಾದರೆ.

\begin{verse}
ತಸ್ಮಾತ್ ತ್ವಮುತ್ತಿಷ್ಠ ಯಶೋ ಲಭಸ್ವ\\ ಜಿತ್ವಾ ಶತ್ರೂನ್ ಭುಂಕ್ಷ್ವ ರಾಜ್ಯಂ ಸಮೃದ್ಧಮ್ ।\\ಮಯೈವೈತೇ ನಿಹತಾಃ ಪೂರ್ವಮೇವ\\ ನಿಮಿತ್ತಮಾತ್ರಂ ಭವ ಸವ್ಯಸಾಚಿನ್ \versenum{॥ ೩೩ ॥}
\end{verse}

{\small ಆದುದರಿಂದ ನೀನು ಏಳು, ಕೀರ್ತಿಯನ್ನು ಗಳಿಸು. ಶತ್ರುಗಳನ್ನು ಗೆದ್ದು ಧನ ಧಾನ್ಯ ಸಮೃದ್ಧವಾದ ರಾಜ್ಯವನ್ನು ಅನುಭವಿಸು. ಇವರನ್ನೆಲ್ಲ ನಾನು ಮೊದಲೆ ಸಂಹರಿಸಿದ್ದೇನೆ. ಸವ್ಯಸಾಚಿ, ನೀನು ಬರೀ ನಿಮಿತ್ತರೂಪನಾಗು.}

ಆದುದರಿಂದ ನೀನು ಯುದ್ಧ ಮಾಡುವುದಕ್ಕೆ ಅಣಿಯಾಗು ಎನ್ನುವನು. ಇನ್ನು ಮೇಲೆ ಶೋಕಿಸು ವುದು ಬೇಡ. ಮುಂದೆ ಏನು ಆಗಬೇಕೊ ಅದೆಲ್ಲ ನಿಶ್ಚಯವಾಗಿದೆ, ಆಡುವ ನಾಟಕದಲ್ಲಿ ಮುಂದೆ ಏನೇನು ಆಗಬೇಕು ಎಂಬುದು ಮೊದಲೆ ನಿಶ್ಚಯಿಸಿ ಆಗಿರುವ ಹಾಗೆ. ನಾಟಕದಲ್ಲಿ ಒಬ್ಬೊಬ್ಬನಿಗೆ ಒಂದೊಂದು ಪಾತ್ರವಿದೆ. ಯಾರೋ ಒಬ್ಬ ಪಾತ್ರಧಾರಿ ಖಾಯಿಲೆ ಬಿದ್ದರೆ ಆ ನಾಟಕ ನಿಂತು ಹೋಗುವುದಿಲ್ಲ. ಇನ್ನೊಬ್ಬನಿಗೆ ಆ ಪಾತ್ರವನ್ನು ಕೊಟ್ಟು ಆಡಿಸುವರು.

ನೀನು ಯಶಸ್ಸನ್ನು ಹೊಂದು ಎನ್ನುತ್ತಾನೆ. ಯುದ್ಧದಲ್ಲಿ ಭೀಷ್ಮ ದ್ರೋಣ ಕರ್ಣರಂತಹ ವ್ಯಕ್ತಿಗಳನ್ನೇ ಕೊಂದಾಗ ಪ್ರಪಂಚದಲ್ಲಿ ಬರುವ ಕೀರ್ತಿ ಆಚಂದ್ರಾರ್ಕವಾದುದು. ಅಂತಹ ಕೀರ್ತಿಯೇ ಬೇಡವೆಂದರೂ ಇವನ ಮನೆಬಾಗಿಲನ್ನು ತಟ್ಟಿ ಪ್ರವೇಶಿಸುವುದರಲ್ಲಿದೆ. ಇದು ಎಷ್ಟು ಜನಕ್ಕೆ ಒದಗುವುದು? ಎಲ್ಲೋ ಕೆಲವರ ಭಾಗ್ಯ. ಎಲ್ಲರಿಗೂ ಸಿಕ್ಕುವ ಅವಕಾಶವಲ್ಲ ಇದು. ಕೀರ್ತಿ ಒಂದೇ ಅಲ್ಲ, ಜೊತೆಗೆ ರಾಜ್ಯ ಬರುವುದು ಆಳುವುದಕ್ಕೆ. ಅದು ಎಂತಹ ರಾಜ್ಯ? ಧನ ಧಾನ್ಯಗಳಿಂದ ಸಮೃದ್ಧಿಯಾದ ರಾಜ್ಯ. ಅನಂತರ ಇವುಗಳನ್ನು ಸಂಪಾದನೆ ಮಾಡುವುದಕ್ಕೆ ಅವನು ಕಷ್ಟಪಡ ಬೇಕಾಗಿಲ್ಲ. ಇವುಗಳೆಲ್ಲ ಇವೆ. ಅದೆಲ್ಲ ಸಿಕ್ಕುವುದು. ನಿರಾತಂಕವಾಗಿ ಆಳಬಹುದು.

ಇವರೆಲ್ಲ ಮೊದಲೇ ನನ್ನಿಂದ ಹತರಾಗಿದ್ದಾರೆ. ಇವರೆಲ್ಲ ಆಗಲೇ ಕೊಲ್ಲಲ್ಪಟ್ಟಿದ್ದಾರೆ. ಅರ್ಜುನ ಸತ್ತವನನ್ನು ನೂಕಬೇಕಾಗಿದೆ ಅಷ್ಟೆ. ಇವನು ತಾನು ಕೊಂದೆ ಎಂದು ಮರುಗಬೇಕಾಗಿಲ್ಲ. ನಾಟಕದಲ್ಲಿ ಯಾರು ಬೀಳಬೇಕು ಎಂಬುದನ್ನು, ಯಾರು ಸಾಯಬೇಕು ಎಂಬುದನ್ನು ನಾಟಕ ಬರೆಯುವವನು ಆಗಲೆ ನಿರ್ಧಾರ ಮಾಡಿದ್ದಾನೆ. ರಂಗಭೂಮಿಯ ಮೇಲೆ ಅವನು ಹಾಗೆಯೇ ಬೀಳಬೇಕಾಗಿದೆ. ಅವನಿಗೆ ಬೇರೆ ವಿಧಿಯೇ ಇಲ್ಲ. ಸೂತ್ರಧಾರ ಹೇಳುವಂತೆ ನಟ ಕೇಳಬೇಕು.

ನೀನು ನಿಮಿತ್ತನಾಗು ಎನ್ನುತ್ತಾನೆ. ಭಗವಂತನ ಕೈಯಲ್ಲಿ ಒಂದು ಯಂತ್ರವಾಗಬೇಕಾಗಿದೆ. ಮಹಾಕವಿ ಹೇಳುತ್ತಾನೆ. ಅದನ್ನು ಕುಳಿತುಕೊಂಡು ಯಾರೋ ಬರೆಯುತ್ತಾರೆ. ಬರೆಯವವನು ಇಲ್ಲದೇ ಇದ್ದರೆ ಹೇಳುವವರು ನಿಲ್ಲಿಸುವುದಿಲ್ಲ. ಇನ್ನು ಯಾರನ್ನೊ ಕರೆಸಿ ತನ್ನ ಕೆಲಸ ಮಾಡಿಸುತ್ತಾನೆ. ಅರ್ಜುನ ಕೀರ್ತಿಯನ್ನು ಗಳಿಸಿದ ಎಂದು ಪ್ರಪಂಚ ನೋಡುವುದು. ಆದರೆ ಭಗವಂತ ಅವನಿಗೆ ಕೀರ್ತಿಯನ್ನು ಗಳಿಸಿಕೊಟ್ಟ. ಕೊಟ್ಟದ್ದನ್ನು ಸ್ವೀಕರಿಸಿದ ಅರ್ಜುನ, ಅಷ್ಟೆ. ಇದು ತೆರೆಯ ಮರೆಯ ಹಿಂದಿನ ದೃಶ್ಯ. ಕುಳಿತು ನೋಡುವವರಿಗೆ ಇದು ಕಾಣುವುದಿಲ್ಲ. ಭಗವಂತ ಒಂದು ಯಃಕಶ್ಚಿತ್ ವಸ್ತುವಿನ ಮೂಲಕ ಅಮೋಘವಾದುದನ್ನು ಮಾಡಬಲ್ಲ. ಜನ ಹೊರಗೆ ಇರುವ ವಸ್ತುವನ್ನು ನೋಡುತ್ತಾರೆಯೇ ಹೊರತು ಅದನ್ನು ಉಪಯೋಗಿಸುವವನು ಅವರಿಗೆ ಕಾಣುವುದಿಲ್ಲ. ಒಬ್ಬ ಭಗವಂತನ ನಿಮಿತ್ತವಾಗುವುದಕ್ಕೆ ಒಪ್ಪಿಕೊಂಡರೆ ಭಗವಂತ ಅವನನ್ನು ಉಪಯೋಗಿಸುವನು. ಅವನು ನಾನು ಒಲ್ಲೆ ಎಂದರೂ ಭಗವಂತನ ಕೆಲಸ ನಿಂತು ಹೋಗುವುದಿಲ್ಲ. ಇಲ್ಲಿ ನಷ್ಟ, ಆ ನಿಮಿತ್ತಕ್ಕೆ ಹೊರತು ದೇವರಿಗಲ್ಲ.

ಈ ಪ್ರಪಂಚ ಭಗವಂತನ ಇಚ್ಛಾನುಸಾರ ನಡೆಯುತ್ತಿದೆ. ಯಾವ ಒಂದು ವ್ಯಕ್ತಿಯ ಮೇಲೂ ನಿಂತಿಲ್ಲ. ಎಂತಹ ಮಹಾಮಹಾ ವ್ಯಕ್ತಿಗಳು ಹೋದರು. ಈ ಪ್ರಪಂಚವೇನು ನಿಂತುಹೋಯಿತೆ? ಹಿಂದಿನಂತೆಯೇ ನಡೆಯುತ್ತಿದೆ. ಮಹಾ ವ್ಯಕ್ತಿಗಳಲ್ಲ ಈ ಪ್ರಪಂಚವನ್ನು ನಡೆಸುವವರು. ದೇವರು ಮಹಾವ್ಯಕ್ತಿಗಳನ್ನು ನಡೆಸುವುದು. ಅವನ ಇಚ್ಛೆಯಂತೆ ಈ ಪ್ರಪಂಚವಾಗುವುದು. ಅದಕ್ಕೆ ನಾವು ಮಣಿದರೆ ಉದ್ಧಾರವಾಗುವವರು ನಾವು, ಮಣಿಯದೆ ಇದ್ದರೆ ಹಾಳಾಗಿ ಹೋಗುವವರು ನಾವು.

\begin{verse}
ದ್ರೋಣಂ ಚ ಭೀಷ್ಮಂ ಚ ಜಯದ್ರಥಂ ಚ \\ ಕರ್ಣಂ ತಥಾನ್ಯಾನಪಿ ಯೋಧವೀರಾನ್ ।\\ಮಯಾ ಹತಾಂಸ್ತ್ವಂ ಜಹಿ ಮಾ ವ್ಯಥಿಷ್ಠಾ \\ ಯುಧ್ಯಸ್ವ ಜೇತಾಸಿ ರಣೇ ಸಪತ್ನಾನ್ \versenum{॥ ೩೪ ॥}
\end{verse}

{\small ನನ್ನಿಂದ ಹತರಾದ ದ್ರೋಣ ಭೀಷ್ಮ ಜಯದ್ರಥ ಕರ್ಣ ಇವರನ್ನು ಮತ್ತು ಇತರ ವೀರರನ್ನು ನೀನು ಕೊಲ್ಲು, ವ್ಯಥೆಪಡಬೇಡ. ಯುದ್ಧದಲ್ಲಿ ಶತ್ರುವನ್ನು ಜಯಿಸುವೆ. ಆದುದರಿಂದ ಯುದ್ಧಮಾಡು.}

ಅರ್ಜುನ ಕೊಲ್ಲುವುದಕ್ಕೆ ಮುಂಚೆ ಶ್ರೀಕೃಷ್ಣ ಅವರೆಲ್ಲ ಕಾಲವಾಗಬೇಕೆಂದು ಸಂಕಲ್ಪ ಮಾಡಿರು ವನು. ಯಾರೂ ಅದಕ್ಕೆ ವಿರೋಧವಾಗಿ ಹೋಗುವ ಹಾಗಿಲ್ಲ. ದ್ರೋಣ ಪಾಂಡವರ ಗುರು. ಅದರಲ್ಲಿಯೂ ಅರ್ಜುನನ ಮೆಚ್ಚಿನ ಗುರು. ಭೀಷ್ಮಾಚಾರ್ಯರಾದರೋ ಅಷ್ಟವಸುಗಳಲ್ಲಿ ಒಬ್ಬರು. ಇಚ್ಛಾಮರಣಿಗಳು. ಜಯದ್ರಥ ದುರ್ಯೋಧನನ ತಂಗಿಯನ್ನು ಮದುವೆಯಾದನು. ತಪಸ್ಸುಮಾಡಿ ಅರ್ಜುನನ ವಿನಾ ಪಾಂಡವರನ್ನೆಲ್ಲ ಒಂದು ದಿವಸದ ಮಟ್ಟಿಗೆ ಸೋಲಿಸುತ್ತೇನೆ ಎಂಬ ವರವನ್ನು ಪಡೆದಿದ್ದನು. ಕರ್ಣ ಇಡೀ ಕೌರವನ ಪಕ್ಷದಲ್ಲಿ ಅರ್ಜುನನಿಗೆ ಸಮನಾಗಿ ಕಾದಬಲ್ಲ ವೀರ. ದುರ್ಯೋಧನನ ಬಲಗೈ ಇದ್ದಂತೆ ಇರುವವನು. ಇಂತಹ ಅತಿರಥ ಮಹಾರಥರೆಲ್ಲರ ಆಯುಸ್ಸು ಮುಗಿಯುತ್ತ ಬಂದಿದೆ. ಅರ್ಜುನ ನಿಮಿತ್ತ ಮಾತ್ರವಾಗಿ ಅವರೊಂದಿಗೆ ಯುದ್ಧ ಮಾಡಬೇಕಾಗಿದೆ, ಅಷ್ಟೆ. ಅವರೆಲ್ಲ ಸತ್ತುಬೀಳುವರು.

ವ್ಯಥೆಪಡಬೇಡ ಎನ್ನುತ್ತಾನೆ. ಅರ್ಜುನನು ಗುರುಗಳನ್ನು ಹಿರಿಯರನ್ನು ಕೊಲ್ಲುತ್ತಿರುವೆ ಎಂದು ತಿಳಿಯಬೇಕಾಗಿಲ್ಲ. ನಿಜವಾಗಿ ಇವರನ್ನೆಲ್ಲ ಕೊಲ್ಲುತ್ತಿರುವವನು ಭಗವಂತ. ಅರ್ಜುನನಿಗೆ ಇದರಿಂದ ಯಾವ ಪಾಪವೂ ಬರುವುದಿಲ್ಲ.

ಅರ್ಜುನನ ಮನಸ್ಸಿನಲ್ಲಿ ಅಳುಕುತ್ತಿದ್ದ ಮತ್ತೊಂದು ಪ್ರಶ್ನೆಯೇ, ಯುದ್ಧದಲ್ಲಿ ನಾವೇ ಗೆಲ್ಲು ತ್ತೇವೆಯೊ ಅವರು ಗೆಲ್ಲುತ್ತಾರೆಯೊ ಎಂಬ ಸಂದೇಹ. ಆ ಸಂದೇಹವನ್ನು ನಿವಾರಣೆ ಮಾಡುವುದ ಕ್ಕಾಗಿ, ನೀವೇ ಗೆಲ್ಲುವಿರಿ, ಅದರಲ್ಲಿ ಯಾವ ಸಂದೇಹವೂ ಇಲ್ಲ ಎಂದು ಹೇಳುತ್ತಾನೆ.

ನಂತರ ಸಂಜಯ ಹೇಳುತ್ತಾನೆ:

\begin{verse}
ಏತಚ್ಛ್ರುತ್ವಾ ವಚನಂ ಕೇಶವಸ್ಯ ಕೃತಾಂಜಲಿರ್ವೇಪಮಾನಃ ಕಿರೀಟೀ ।\\ನಮಸ್ಕೃತ್ವಾ ಭೂಯ ಏವಾಹ ಕೃಷ್ಣಂ ಸಗದ್ಗದಂ ಭೀತಭೀತಃ ಪ್ರಣಮ್ಯ \versenum{॥ ೩೫ ॥}
\end{verse}

{\small ಶ್ರೀಕೃಷ್ಣನ ಈ ಮಾತನ್ನು ಕೇಳಿ ಅರ್ಜುನನು ಕೈಗಳನ್ನು ಜೋಡಿಸಿಕೊಂಡು ನಡುಗುತ್ತ ಪುನಃ ಪುನಃ ನಮಸ್ಕರಿಸುತ್ತ, ಅತ್ಯಂತ ಭೀತನಾಗಿ ತಲೆಬಾಗಿಸಿಕೊಂಡು ಶ್ರೀಕೃಷ್ಣನನ್ನು ಕುರಿತು ಗದ್ಗದ ಕಂಠದಿಂದ ಹೀಗೆ ಹೇಳುತ್ತಾನೆ.}

ಯುದ್ಧ ಕ್ಷೇತ್ರಕ್ಕೆ ದೂರದಲ್ಲಿರುವ ಸಂಜಯನಿಗೆ ಅರ್ಜುನ ಏನು ಮಾಡುತ್ತಿದ್ದಾನೆ ಎಂಬುದು ಕಾಣುತ್ತಿದೆ. ಅವನು ಧೃತರಾಷ್ಟ್ರನಿಗೆ ಹೇಳುತ್ತಿದ್ದಾನೆ. ಅರ್ಜುನನಿಗೆ ಇನ್ನು ಯಾವ ಸಂದೇಹವೂ ಇಲ್ಲ. ಮುಂದೆ ಏನಾಗುವುದು ಎಂಬುದು ಸ್ಪಷ್ಟವಾಗಿ ಗೊತ್ತಾಗುವುದು. ಭಕ್ತಿ ಮತ್ತು ಭಯದಿಂದ ಕೈಗಳನ್ನು ಜೋಡಿಸಿಕೊಂಡು ಕೃಷ್ಣನನ್ನು ಪ್ರಾರ್ಥನೆ ಮಾಡುತ್ತಾನೆ. ಹಾಗೆ ಪ್ರಾರ್ಥಿಸುವಾಗ ಗದ್ಗದ ಕಂಠ. ಭಾವ ಒಳಗಿನಿಂದ ಉಕ್ಕುತ್ತಿರುವಾಗ ಬಾಯಿನಲ್ಲಿ ಮಾತು ಸರಾಗವಾಗಿ ಹೊರಡುವುದಿಲ್ಲ, ತಡೆ ತಡೆದು ಬರುತ್ತಿದೆ. ಅರ್ಜುನ ಕ್ಷತ್ರಿಯ, ವೀರಯೋಧ. ಅವನು ಕವಿಯಲ್ಲ. ಆದರೆ ಭಾವ ಉಕ್ಕಿ ಬಂದರೆ ಅದೊಂದು ಕಾವ್ಯದ ಹೊನಲಾಗುವುದು.

ಅರ್ಜುನ ಹೀಗೆ ಪ್ರಾರ್ಥಿಸುತ್ತಾನೆ:

\begin{verse}
ಸ್ಥಾನೇ ಹೃಷೀಕೇಶ ತವ ಪ್ರಕೀರ್ತ್ಯಾ ಜಗತ್ ಪ್ರಹೃಷ್ಯತ್ಯನುರಜ್ಯತೇ ಚ ।\\ರಕ್ಷಾಂಸಿ ಭೀತಾನಿ ದಿಶೋ ದ್ರವಂತಿ ಸರ್ವೇ ನಮಸ್ಯಂತಿ ಚ ಸಿದ್ಧಸಂಘಾಃ \versenum{॥ ೩೬ ॥}
\end{verse}

{\small ಹೃಷೀಕೇಶ, ನಿನ್ನ ಮಹಾತ್ಮ್ಯ ಕೀರ್ತನೆಯಿಂದ ಜಗತ್ತು ಸಂತೋಷಪಡುತ್ತಿದೆ. ಮತ್ತು ಅನುರಾಗವನ್ನು ಹೊಂದುತ್ತಿದೆ. ರಾಕ್ಷಸರು ಭೀತರಾಗಿ ದೆಸೆದೆಸೆಗೂ ಓಡುತ್ತಿರುವರು. ಸಿದ್ಧರ ಸಂಘ ನಮಸ್ಕರಿಸುತ್ತಿದೆ. ಇವೆಲ್ಲ ಯುಕ್ತವೇ ಸರಿ.}

ಈ ವಿಶ್ವರೂಪವನ್ನು ನೋಡುತ್ತಿದ್ದವರು, ಅಧರ್ಮನಾಶದಲ್ಲಿ ಭಗವಂತನು ಹೇಗೆ ಉದ್ಯುಕ್ತ ನಾಗಿದ್ದಾನೆ ಎಂಬುದನ್ನು ನೋಡಿ ಕೊಂಡಾಡುತ್ತಿದ್ದಾರೆ. ಅವನನ್ನೆ ನೆಚ್ಚಿದ ಧರ್ಮಾತ್ಮರನ್ನು ಅವನು ಎಂದಿಗೂ ಕೈ ಬಿಡುವುದಿಲ್ಲ ಎಂಬುದನ್ನು ನೋಡುತ್ತಿದ್ದಾರೆ: ಅದಕ್ಕಾಗಿ ಅವನನ್ನು ಪ್ರೀತಿಸುತ್ತಿದ್ದಾರೆ. ಇಲ್ಲಿ ಎರಡು ಬಗೆಯ ಜನರು ಇರುವರು. ರಾಕ್ಷಸರು ಎಂದರೆ ಅಧರ್ಮಿಗಳು. ಭಗವಂತನ ಕಾಲರುದ್ರನ ರೂಪವನ್ನು ನೋಡಿ ಅಂಜಿಕೆಯಿಂದ ಓಡಿಹೋಗುತ್ತಿರುವರು. ದುಷ್ಟರ ಸ್ವಭಾವವೇ ಹಾಗೆ. ತಮಗಿಂತ ದುರ್ಬಲರಾದವರನ್ನು ಕಂಡರೆ ಪೀಡಿಸುವರು, ತಮಗಿಂತ ಬಲಿಷ್ಠರಾದವರನ್ನು ಕಂಡರೆ ಓಡಿಹೋಗುವರು. ಈ ದೃಶ್ಯವನ್ನು ತಾಳಲಾರರು ರಾಕ್ಷಸರು. ಅಂಜಿಕೆಯಿಂದ ಪಲಾಯನ ಮಾಡುತ್ತಿರುವರು. ಮತ್ತೊಂದು ವರ್ಗದವರೆ ಸಿದ್ಧರು. ಭಗವಂತನ ನೈಜಸ್ಥಿತಿಯನ್ನು ಬಲ್ಲವರು. ಇಂತಹ ರುದ್ರರೂಪವನ್ನು ನೋಡಿದರೂ, ಅದರ ಹಿಂದೆ ಇರುವ ಪ್ರೇಮಮೂರ್ತಿ ಇವರಿಗೆ ಕಾಣದೆ ಇಲ್ಲ. ಅವನು ಅಧರ್ಮಿಗಳಿಗೆ ರುದ್ರ. ಆದರೆ ಯಾರು ಅವನ ಭಕ್ತರೊ ಅವರಿಗೆ ಅವನು ಯಾವಾಗಲೂ ಪ್ರೇಮಮೂರ್ತಿಯೇ. ನರಸಿಂಹ ರಾಕ್ಷಸನನ್ನು ಸಿಗಿದಾಗ ಪ್ರಹ್ಲಾದ ಹತ್ತಿರ ಬಂದು ನಿಂತುಕೊಂಡರೆ, ಅವನನ್ನು ಪ್ರೇಮದಿಂದ ನೆಕ್ಕುತ್ತಾನೆ. ಬೆಕ್ಕು ಇಲಿಯನ್ನು ತನ್ನ ಬಾಯಲ್ಲಿ ಹಿಡಿದು ನಿಷ್ಕರುಣೆಯಿಂದ ಅಗಿದರೂ, ತನ್ನ ಮರಿಯನ್ನು ಅದೇ ಬಾಯಿಂದ ಸ್ವಲ್ಪವೂ ನೋವಾಗದ ರೀತಿ ಪ್ರೀತಿಯಿಂದ ಹಿಡಿದುಕೊಳ್ಳುವುದು. ಅದರಂತೆ ಭಕ್ತರು ಭಗವಂತ ಎಂತಹ ರುದ್ರನ ರೂಪವನ್ನು ಧರಿಸಿದ್ದರೂ ಅದರ ಹಿಂದೆ ಇರುವ ಪ್ರೇಮಮೂರ್ತಿಯನ್ನು ನೋಡುವರು.

\begin{verse}
ಕಸ್ಮಾಚ್ಚ ತೇ ನ ನಮೇರನ್ ಮಹಾತ್ಮನ್ ಗರೀಯಸೇ ಬ್ರಹ್ಮಣೋಽಪ್ಯಾದಿಕರ್ತ್ರೇ ।\\ಅನಂತ ದೇವೇಶ ಜಗನ್ನಿವಾಸ ತ್ವಮಕ್ಷರಂ ಸದಸತ್ತತ್ಪರಂ ಯತ್ \versenum{॥ ೩೭ ॥}
\end{verse}

{\small ಮಹಾತ್ಮನೇ, ಬ್ರಹ್ಮನಿಗೂ ಆದಿಕಾರಣನಾದ ಉತ್ಕೃಷ್ಟನಾದ ನಿನಗೆ ಹೇಗೆ ನಮಸ್ಕರಿಸದೆ ಇದ್ದಾರು? ಅನಂತ, ದೇವೇಶ, ಜಗನ್ನೀವಾಸ, ನೀನೇ ಅಕ್ಷರ, ಸತ್ತು, ಅಸತ್ತು ಮತ್ತು ಪರವಾದದ್ದು.}

ವಿಷ್ಣು ಪೌರಾಣಿಕ ರೀತಿಯಲ್ಲಿ ಬ್ರಹ್ಮನಿಗೂ ಮೂಲ. ವಿಷ್ಣುವಿನ ನಾಭಿಯಿಂದ ಬ್ರಹ್ಮ ಹುಟ್ಟಿದ. ಅನಂತರ ಪ್ರಪಂಚವನ್ನು ಸೃಷ್ಟಿಸಲು ಪ್ರಾರಂಭಿಸುವನು. ಭಗವಂತ ಎಲ್ಲರಿಗಿಂತಲೂ ಉತ್ಕೃಷ್ಟ. ಈ ಪ್ರಪಂಚದಲ್ಲಿ ಎಲ್ಲಿಯಾದರೂ ಯಾವುದಾದರೂ ಒಂದು ಒಳ್ಳೆಯ ವಸ್ತುವಿದ್ದರೆ ಅದೆಲ್ಲ ಭಗವಂತನ ಪ್ರಭೆ. ಎಲ್ಲರೂ ಅವನನ್ನು ಪ್ರತಿಬಿಂಬಿಸುವವರೇ. ಅವನನ್ನು ಮೀರಿದವರಿಲ್ಲ, ಮತ್ತು ಅವನಿಗೆ ಸಮನಾದವರೂ ಇಲ್ಲ. ಇಂತಹ ಪವಿತ್ರಾತ್ಮ ಎದುರಿಗೆ ನಿಂತಿರುವಾಗ ಯಾರಿಗೆ ನಮಸ್ಕರಿಸ ದಿರಲು ಸಾಧ್ಯ. ಪವಿತ್ರತೆ ಮತ್ತು ಮಹಾತ್ಮ್ಯೆಗೆ ಅಂತಹ ಶಕ್ತಿ ಇದೆ. ನಮಗೆ ಅದು ಗೊತ್ತಾಗದೆ ಇದ್ದರೂ ಅದು ತನ್ನ ಪ್ರತಿಭೆಯನ್ನು ನಮ್ಮ ಮೇಲೆ ಬೀರುವುದು. ಹೇಗೆ ಕಬ್ಬಿಣದ ಅದುರು ಸ್ಪರ್ಶಶಿಲೆ ಕಡೆಗೆ ಧಾವಿಸುವುದೊ ಹಾಗೆ ಮಾನವ ಪರಮಾತ್ಮನ ಮಹಿಮೆಗೆ ಬಾಗುತ್ತಾನೆ. ಅವನು ಅನಂತ. ಅವನಿಗೆ ಒಂದು ಆದಿ ಅಂತ್ಯಗಳಿಲ್ಲ. ಎಲ್ಲೆಲ್ಲಿಯೂ ತುಂಬಿರುವನು. ಅವನಿಲ್ಲದ ಸ್ಥಳವಿಲ್ಲ. ಅವನು ದೇವೇಶ. ಇಂದ್ರ ವರುಣ ಅಗ್ನಿ ಮುಂತಾದ ದೇವತೆಗಳೆಲ್ಲ ಭಗವಂತನ ಆಜ್ಞೆಯನ್ನು ಶಿರಸಾಧರಿಸಿ ಪಾಲಿಸುವರು. ಅವನು ಈ ದೇವತೆಗಳ ಮೂಲಕ ಕೆಲಸಮಾಡುವುದು. ಅವನು ಜಗತ್ತಿಗೆಲ್ಲ ನಿವಾಸಸ್ಥಾನ, ಆಶ್ರಯಸ್ಥಾನ. ಅವನಿದ್ದರೆ ಜಗತ್ತು. ಸಾಗರ ಇದ್ದರೆ ಅಲೆಗಳು ಹೇಗೋ ಹಾಗೆ ಅವನಿಂದ ಜಗತ್ತಿನ ಗುಳ್ಳೆ ಏಳುವುದು. ಅವನು ಅಕ್ಷರ. ಎಂದಿಗೂ ನಾಶವಾಗದವನು. ಈ ಪ್ರಪಂಚದಲ್ಲಿ ಅವನು ವಿನಾ ಉಳಿದುವುಗಳೆಲ್ಲ ಅಲೆಯಂತೆ ಏಳುವುವು, ಬೀಳುವುವು, ಇರುವಾಗ ಬದಲಾಯಿಸುವುವು. ಏಳುವುದಕ್ಕೆ ಮುಂಚೆ ಇರುವುದಿಲ್ಲ. ಬಿದ್ದಾದ ಮೇಲೆ ಇರುವುದಿಲ್ಲ. ಆದರೆ ಭಗವಂತನಾದರೊ ತ್ರಿಕಾಲದಲ್ಲಿಯೂ ಇರುವನು. ಅವನೇ ಸತ್, ವ್ಯಕ್ತವಾಗಿರುವ ಪ್ರಕೃತಿ. ಅವ್ಯಕ್ತ, ವ್ಯಕ್ತವಾಗದೆ ಇರುವ ಪ್ರಕೃತಿ ಮತ್ತು ಪರಾ ಎಂದರೆ ಅವ್ಯಕ್ತಕ್ಕಿಂತಲೂ ಆಳವಾದ ಅಗಮ್ಯವಾದ ಭಾವ.

\begin{verse}
ತ್ವಮಾದಿದೇವಃ ಪುರುಷಃ ಪುರಾಣಸ್ತ್ವಮಸ್ಯ ವಿಶ್ವಸ್ಯ ಪರಂ ನಿದಾನಮ್ ।\\ವೇತ್ತಾಽಸಿ ವೇದ್ಯಂ ಚ ಪರಂ ಚ ಧಾಮ ತ್ವಯಾ ತತಂ ವಿಶ್ವಮನಂತರೂಪ \versenum{॥ ೩೮ ॥}
\end{verse}

{\small ಆದಿದೇವನು ನೀನು, ಪುರಾಣಪುರುಷನು ನೀನು, ವಿಶ್ವಕ್ಕೆ ಉತ್ಕೃಷ್ಟವಾದ ಆಶ್ರಯ ನೀನು. ನೀನು ಸರ್ವಜ್ಞ. ಜ್ಞೇಯವಸ್ತು ಮತ್ತು ಪರಂಧಾಮ. ಅನಂತರೂಪ. ನಿನ್ನಿಂದಲೇ ವಿಶ್ವ ವ್ಯಾಪ್ತವಾಗಿದೆ.}

ಭಗವಂತನೇ ದೇವತೆಗಳಿಗೆಲ್ಲ ಆದಿ. ಅವನಿಗಿಂತ ಮುಂಚೆ ಯಾರೂ ಇರಲಿಲ್ಲ. ಮುಂಚೆ ಅವನು, ಅನಂತರ ಉಳಿದ ದೇವತೆಗಳೆಲ್ಲ ಅವನ ಇಚ್ಛಾಮಾತ್ರದಿಂದ ಜನಿಸಿದರು. ಇವನಷ್ಟು ಹಳಬರು ಯಾರೂ ಇಲ್ಲ. ಈ ಪ್ರಪಂಚದಲ್ಲಿ ಆದಿಯಿಂದ ಅಂತ್ಯದವರೆಗೆ ನೋಡುತ್ತಿರುವವನು ಇವ ನೊಬ್ಬನೇ. ಇವನೇ ಈ ಸೃಷ್ಟಿನಾಟಕದ ಕರ್ತೃ, ನಿತ್ಯಪ್ರೇಕ್ಷಕ. ಈ ಬ್ರಹ್ಮಾಂಡಕ್ಕೆ ಇವನಷ್ಟು ಶ್ರೇಷ್ಠವಾದ ಆಶ್ರಯ ಮತ್ತೊಂದಿಲ್ಲ. ಈ ಬ್ರಹ್ಮಾಂಡವೇನು ಸಾಮಾನ್ಯವೆ? ಸಮುದ್ರ ತೀರದ ಮರಳಿನ ಕಣಗಳನ್ನು ಮೀರಿ ಸೂರ್ಯ ಚಂದ್ರ ನಿಹಾರಿಕೆಗಳಿವೆ. ಇಂತಹ ಬೃಹತ್ ವಿಶ್ವಕ್ಕೆ ತಳಪಾಯ ಇನ್ನೆಷ್ಟು ದೊಡ್ಡದಾಗಿರಬೇಕು ಮತ್ತು ಭದ್ರವಾಗಿರಬೇಕು. ಇವನು ಸರ್ವಜ್ಞ. ಈ ಪ್ರಪಂಚದಲ್ಲಿ ಎಲ್ಲವನ್ನೂ ಸೃಷ್ಟಿಸಿದವನು. ಅವನು ಏತಕ್ಕಾಗಿ ಸೃಷ್ಟಿಸಿದನೋ ಅದು ಅವನೊಬ್ಬನಿಗೇ ಗೊತ್ತು. ನಾವೆಲ್ಲ ನಮ್ಮ ಎದುರಿಗೆ ಇರುವ ವಸ್ತುವನ್ನು ಸ್ವಲ್ಪ ತಿಳಿದುಕೊಳ್ಳಬಲ್ಲೆವು ಅಷ್ಟೆ. ಆಳಕ್ಕೆ ಹೋಗಲಾರೆವು. ಅದರ ಗುಣಗಳನ್ನೆಲ್ಲ ತಿಳಿದುಕೊಳ್ಳಲಾರೆವು. ಅದಕ್ಕೂ ಮತ್ತೊಂದಕ್ಕೂ ಇರುವ ಸಂಬಂಧ ನಮಗೆ ಗೊತ್ತಿಲ್ಲ. ಭೂಮದಲ್ಲಿ ಅದಕ್ಕೆ ಇರುವ ಸ್ಥಾನ ನಮಗೆ ಗೊತ್ತಿಲ್ಲ. ಇದೆಲ್ಲ ಗೊತ್ತಿರುವುದು ಒಬ್ಬನಿಗೇ. ಅವನೇ ಪರಮಾತ್ಮ. ಈ ಪ್ರಪಂಚದಲ್ಲಿ ಜ್ಞೇಯ ವಸ್ತು ಎಂದರೆ ತಿಳಿದುಕೊಳ್ಳಬೇಕಾಗಿರುವುದು ಇದೊಂದೇ. ಉಳಿದವುಗಳೆಲ್ಲ ಅಲ್ಪ. ಅಲ್ಪವನ್ನು ತಿಳಿದುಕೊಂಡರೆ ಅಲ್ಲಿಗೆ ಅದು ಕೊನೆಗಾಣುವುದು. ಆದರೆ ಯಾವಾಗ ಪ್ರಪಂಚದಲ್ಲಿ ಭಗವಂತನನ್ನು ಅರಿಯುವೆವೊ ಆಗ ಎಲ್ಲವನ್ನೂ ಅರಿಯುವೆವು. ನಾವೆಂದರೆ ಏನೆನ್ನುವುದು ಗೊತ್ತಾಗುವುದು, ಜಗತ್ತೆಂದರೆ ಏನೆನ್ನುವುದು ಗೊತ್ತಾಗುವುದು. ಅವನೇ ಶ್ರೇಷ್ಠವಾದ ಸ್ಥಾನ, ಪರಂಧಾಮ. ಯಾವಾಗ ನಾವು ಅಲ್ಲಿಗೆ ಹೋಗುವೆವೋ ಈ ಸಂಸಾರ ಚಕ್ರದಿಂದ ತಪ್ಪಿಸಿಕೊಳ್ಳುತ್ತೇವೆ. ಕರ್ಮ ಇನ್ನು ಮೇಲೆ ನಮ್ಮನ್ನು ಅಲ್ಲಿಂದ ಎಳೆತರಲಾರದು, ಬಂಧನದ ನೊಗವನ್ನು ಕತ್ತಿನಮೇಲೆ ಇಡಲಾರದು. ನಾವು ಮುಕ್ತರಾಗು ತ್ತೇವೆ, ಬಂಧನದಿಂದ ಎಂದೆಂದಿಗೂ ಪಾರಾಗುತ್ತೇವೆ. ಎಲ್ಲಾ ಸಂಶಯಗಳೂ ನಾಶವಾಗುತ್ತವೆ, ಆಸೆಗಳು ಸೀದು ಹೋಗುತ್ತವೆ.

ಅನಂತರೂಪನು ಅವನು. ಈ ಪ್ರಪಂಚದಲ್ಲಿ ಇರುವ ಎಲ್ಲಾ ನಾಮರೂಪಗಳ ಹಿಂದೆ ಇರುವವನು ಅವನೆ. ಅವನೇ ವಿಶ್ವವನ್ನೆಲ್ಲಾ ವ್ಯಾಪಿಸಿಕೊಂಡಿರುವನು. ಆಕಾಶ ಹೇಗೆ ನಾಮರೂಪ ಗಳನ್ನೆಲ್ಲ ವ್ಯಾಪಿಸಿಕೊಂಡಿದೆಯೋ, ಅದರ ಒಳಗೆ ಮತ್ತು ಹೊರಗೆ ಇದೆಯೋ ಹಾಗೆಯೇ ಭಗವಂತ ನಮ್ಮೊಳಗೆ ಮತ್ತು ಹೊರಗೆ ಇರುವನು.

\begin{verse}
ವಾಯುರ್ಯಮೋಽಗ್ನಿರ್ವರುಣಃ ಶಶಾಂಕಃ ಪ್ರಜಾಪತಿಸ್ತ್ವಂ ಪ್ರಪಿತಾಮಹಶ್ಚ ।\\ನಮೋ ನಮಸ್ತೇಽಸ್ತು ಸಹಸ್ರಕೃತ್ವಃ ಪುನಶ್ಚ ಭೂಯೋಽಪಿ ನಮೋ ನಮಸ್ತೇ \versenum{॥ ೩೯ ॥}
\end{verse}

{\small ವಾಯು, ಯಮ, ಅಗ್ನಿ, ವರುಣ, ಚಂದ್ರ, ಪ್ರಜಾಪತಿ, ಮತ್ತು ಅವರ ಪಿತಾಮಹನೂ ಅವರೆಲ್ಲರೂ ನೀನೇ. ನಿನಗೆ ಸಾವಿರ ನಮಸ್ಕಾರಗಳು. ಪುನಃ ನಮಸ್ಕಾರ ಮತ್ತೊಮ್ಮೆ ನಮಸ್ಕಾರ.}

ನೀನೆ ಎಲ್ಲವೂ ಆಗಿರುವೆ ಎಂದು ಹೇಳಿ ಬಾರಿ ಬಾರಿಗೆ ನಮಸ್ಕರಿಸುತ್ತಾನೆ. ಯಾವಾಗ ಒಬ್ಬನಲ್ಲಿ ಉದ್ವೇಗಾತಿಶಯವಿರುವುದೊ, ಆಗ ಒಮ್ಮೆ ಅವನು ಹೇಳಿದರೆ ತೃಪ್ತಿಯಾಗುವುದಿಲ್ಲ, ಪುನಃ ಅದನ್ನು ಹೇಳುವುದು ಮಾನವಸಹಜವಾದುದು.

\begin{verse}
ನಮಃ ಪುರಸ್ತಾದಥ ಪೃಷ್ಠತಸ್ತೇ \\ ನಮೋಽಸ್ತು ತೇ ಸರ್ವತ ಏವ ಸರ್ವ ।\\ ಅನಂತವೀರ್ಯಾಮಿತವಿಕ್ರಮಸ್ತ್ವಂ \\ ಸರ್ವಂ ಸಮಾಪ್ನೋಷಿ ತತೋಽಸಿ ಸರ್ವಃ \versenum{॥ ೪೦ ॥}
\end{verse}

{\small ಪರಮಾತ್ಮನೇ, ನಿನಗೆ ಹಿಂದೆ ಮುಂದೆ, ಮತ್ತು ಎಲ್ಲಾ ಕಡೆಯಲ್ಲಿಯೂ ನಮಸ್ಕಾರ. ಅನಂತವೀರ್ಯನೆ, ನೀನು ಅಮಿತ ಪರಾಕ್ರಮನು. ನೀನು ಎಲ್ಲವನ್ನೂ ವ್ಯಾಪಿಸಿಕೊಂಡಿರುವುದರಿಂದ ಸರ್ವಸ್ವರೂಪನಾಗಿರುವೆ.}

ಪರಮಾತ್ಮನೇ ಎಲ್ಲರಿಗೂ ಆತ್ಮಸ್ವರೂಪನಾಗಿರುವನು. ಮನೆ ಮನೆಯಲ್ಲಿ ವಿದ್ಯುಚ್ಛಕ್ತಿ ಬಲ್ಬಿ ನಲ್ಲಿ ಬೆಳಗುತ್ತಿದೆ. ಆದರೆ ಅದು ಒಂದೇ ಮೂಲದಿಂದ ಬಂದುದು. ವಿವಿಧ ಬಣ್ಣಗಳ ಬಲ್ಬುಗಳ ಹಿಂದೆ ಬೆಳಗುವುದೊಂದೇ ಶಕ್ತಿ. ಅದರಂತೆಯೇ ನಮ್ಮ ಹಿಂದೆಲ್ಲ ಬೆಳಗುವವನು ಅವನೇ. ಯಾವುದನ್ನು ನಾವು ಎಂದು ಹೇಳಿಕೊಳ್ಳುತ್ತೇವೆಯೊ ಅದರೊಳಗೆ ಬೆಳಗುವವನೂ ಅವನೇ. ಅವನು ಯಾವಾಗ ಎಲ್ಲ ಕಡೆಯೂ ಇರುವನೊ, ಆಗ ಎಲ್ಲ ಕಡೆಗೂ ನಮಸ್ಕಾರ. ಅವನ ಒಂದು ಭಾಗದಷ್ಟೇ ಮತ್ತೊಂದು ಭಾಗ ಪವಿತ್ರ. ಸಕ್ಕರೆಯಿಂದ ಮಾಡಿದ ಬೊಂಬೆಯ ತಲೆಯನ್ನು ತಿಂದರೂ ಒಂದೇ, ಕಾಲನ್ನು ತಿಂದರೂ ಒಂದೇ.

ಅವನು ಅನಂತ ಪರಾಕ್ರಮಶಾಲಿ. ಈ ಬ್ರಹ್ಮಾಂಡವನ್ನು ಸೃಷ್ಟಿಸಿದವನು, ಆಳುತ್ತಿರವವನು, ಎಲ್ಲವನ್ನು ತನ್ನ ನಿಯಮವೆಂಬ ವಜ್ರಮುಷ್ಟಿಯಲ್ಲಿ ಬಿಗಿದು ಹಿಡಿದಿರುವನು. ಅತಿ ಸಣ್ಣದಾದರೂ ಆಗಲೀ, ದೊಡ್ಡದಾದರೂ ಆಗಲೀ, ಯಾವುದೂ ಅವನನ್ನು ಮೀರಿ ಹೋಗಲಾರದು. ಈ ಪ್ರಪಂಚ ವನ್ನೆಲ್ಲ ಅವನು ವ್ಯಾಪಿಸಿಕೊಂಡಿರುವನು. ಎಲ್ಲವೂ ಅವನೇ ಆಗಿರುವನು. ಹೇಗೆ ಚಿನ್ನ ಅದರಿಂದ ಆದ ಪ್ರತಿಯೊಂದು ನಗನಾಣ್ಯದ ಹಿಂದೆಯೂ ಇದೆಯೋ, ಹಾಗೆಯೇ ಈ ಪ್ರಪಂಚದ ಎಲ್ಲಾ ನಾಮರೂಪಗಳ ಹಿಂದೆಯೂ ಅವನೇ ಇರುವನು.

\begin{verse}
ಸಖೇತಿ ಮತ್ವಾ ಪ್ರಸಭಂ ಯದುಕ್ತಂ \\ ಹೇ ಕೃಷ್ಣ ಹೇ ಯಾದವ ಹೇ ಸಖೇತಿ ।\\ಅಜಾನತಾ ಮಹಿಮಾನಂ ತವೇದಂ \\ ಮಯಾ ಪ್ರಮಾದಾತ್ ಪ್ರಣಯೇನ ವಾಪಿ \versenum{॥ ೪೧ ॥}
\end{verse}

\begin{verse}
ಯಚ್ಚಾವಹಾಸಾರ್ಥಮಸತ್ಕೃತೋಽಸಿ \\ ವಿಹಾರಶಯ್ಯಾಸನಭೋಜನೇಷು ।\\ಏಕೋಽಥವಾಪ್ಯಚ್ಯುತ ತತ್ಸಮಕ್ಷಂ \\ ತತ್ಕಾ ್ಷಮಯೇ ತ್ವಾಮಹಮಪ್ರಮೇಯಮ್ \versenum{॥ ೪೨ ॥}
\end{verse}

{\small ಅಚ್ಯುತ, ನಿನ್ನ ಮಹಿಮೆಯನ್ನು ಮತ್ತು ವಿಶ್ವರೂಪವನ್ನು ಅರಿಯದೆ ನೀನು ನನ್ನ ಸ್ನೇಹಿತನೆಂದು ತಿಳಿದು, ಮರೆವಿನಿಂದ ಆಗಲಿ, ಪ್ರೀತಿಯಿಂದ ಆಗಲಿ, “ಕೃಷ್ಣ, ಯಾದವ, ಸಖ,” ಎಂದು ನಿನ್ನನ್ನು ಕರೆದು ಅವಿವೇಕವನ್ನೇ ನಾದರೂ ಮಾಡಿದ್ದರೆ, ವಿನೋದವಾಗಿ ಆಟವಾಡುವಾಗಲೊ, ಊಟಮಾಡುವಾಗಲೊ, ಮಲಗುವಾಗಲೊ, ನಿನ್ನೊಟ್ಟಿಗಿದ್ದಾಗ, ನನ್ನಿಂದ ನಿನಗೆ ಏನಾದರೂ ಅಪಮಾನವಾಗಿದ್ದರೆ, ಅದನ್ನು ಕ್ಷಮಿಸಬೇಕೆಂದು ಪ್ರಾರ್ಥಿಸು ತ್ತೇನೆ.}

ಅರ್ಜುನ ಮೊದಲು ಶ್ರೀಕೃಷ್ಣನನ್ನು ತನ್ನ ಕೇವಲ ಸಖ ಎಂದು ಮಾತ್ರ ತಿಳಿದಿದ್ದ. ಶ್ರೀಕೃಷ್ಣನ ಮಹಿಮೆ ಗೊತ್ತಾದದ್ದು ಕುರುಕ್ಷೇತ್ರದ ಯುದ್ಧಭೂಮಿಯಲ್ಲಿ. ಚಿನ್ನವನ್ನು ಒರೆಗಲ್ಲಿಗೆ ತಿಕ್ಕಿದಾಗ ತಾನೆ ಅದು ಏನು ಎಂಬುದು ಗೊತ್ತಾಗುವುದು. ಯಾವಾಗ ವಿಶ್ವರೂಪವನ್ನು ನೋಡಿದನೊ, ಆಗ ಶ್ರೀಕೃಷ್ಣನ ಮಹಿಮೋನ್ನತ ವ್ಯಕ್ತಿತ್ವ ಗೊತ್ತಾಗುವುದು. ಆಗ ಹಿಂದೆ ಶ್ರೀಕೃಷ್ಣನನ್ನು ಅಷ್ಟು ಸದರದಿಂದ ಕಂಡುದಕ್ಕೆ ಕ್ಷಮಾಪಣೆ ಬೇಡುವನು. ಇದನ್ನೆಲ್ಲ ಅಜ್ಞಾನದಲ್ಲಿದ್ದಾಗ ಮಾಡಿದ್ದು. ಆವಾಗ ನಮಗೆ ಏನೂ ಅನ್ನಿಸುವುದಿಲ್ಲ. ಆದರೆ ಜ್ಞಾನೋದಯವಾದ ಕೂಡಲೆ ಮೊದಲು ನಮಗೆ ಗೊತ್ತಾಗುವುದು ನಮ್ಮ ಅವಿವೇಕ. ಅಯ್ಯೋ ಎಂತಹ ತಪ್ಪನ್ನು ನಾನು ಮಾಡಿದೆ ಎಂದು ಆಗ ತಳಮಳಿಸುವೆವು.

\begin{verse}
ಪಿತಾಽಸಿ ಲೋಕಸ್ಯ ಚರಾಚರಸ್ಯ ತ್ವಮಸ್ಯ \\ ಪೂಜ್ಯಸ್ಯ ಗುರುರ್ಗರೀಯಾನ್ ।\\ನ ತ್ವತ್ಸಮೋಽಸ್ತ್ಯಭ್ಯಧಿಕಃ ಕುತೋಽನ್ಯೋ \\ ಲೋಕತ್ರಯೇಽಪ್ಯಪ್ರತಿಮಪ್ರಭಾವ \versenum{॥ ೪೩ ॥}
\end{verse}

{\small ಅಪ್ರತಿಮ ಪ್ರಭಾವನೆ, ಸ್ಥಾವರ ಜಂಗಮ ಲೋಕಕ್ಕೆ ತಂದೆ ನೀನು. ಅದಕ್ಕೆ ಪೂಜ್ಯನೂ ಗುರುವೂ ಶ್ರೇಷ್ಠನೂ ಆಗಿರುವೆ. ಲೋಕತ್ರಯದಲ್ಲಿ ನಿನಗೆ ಸಮಾನರಿಲ್ಲ. ಹೀಗಿರುವಾಗ ನಿನಗಿಂತ ದೊಡ್ಡವರು ಎಲ್ಲಿರುವರು?}

ಭಗವಂತನ ಪ್ರಭಾವ ಅಪ್ರತಿಮವಾದುದು. ಇದರಷ್ಟು ತನ್ನ ಪ್ರಭಾವವನ್ನು ನಮ್ಮ ಮೇಲೆ ಬೀರುವ ಮತ್ತೊಂದು ಇಲ್ಲ. ಅದು ಅಮೋಘವಾದುದು. ಎಷ್ಟು ದಿನಗಳ ಮೇಲೆ ಆದರೂ ಅದು ನಮ್ಮ ಮೇಲೆ ಪ್ರಭಾವವನ್ನು ಬೀರಿಯೆ ಬೀರುವುದು. ಎಂದಿಗೂ ಅದು ವ್ಯರ್ಥವಾಗುವುದಿಲ್ಲ. ಅವನ ಪ್ರಭಾವವೆಂಬುದು ಬೀಜದಂತೆ. ಒಂದಲ್ಲ ಒಂದು ದಿನ ಅದು ಮೊಳೆಯುವುದು. ಯಾವಾಗ ಅದಕ್ಕೆ ಸ್ವಲ್ಪ ತೇವ ಮತ್ತು ನೆಲದ ಆಸರೆ ಸಿಕ್ಕುವುದೊ ಆಗ ಅದರೊಳಗೆ ಇರುವ ಸಸಿ ಬೇರು ಬಿಡಲೆತ್ನಿಸುವುದು. ಭಗವಂತ ಕೆಲವರ ಮೇಲೆ ತನ್ನ ಪ್ರಭಾವವನ್ನು ಬೀರಲು ಸ್ವಲ್ಪಕಾಲ ತೆಗೆದು ಕೊಳ್ಳುವನು. ಅವರು ಕಲ್ಲು ನೆಲದಂತೆ, ಬಂಡೆಯಂತಿರುವರು. ಆದರೆ ಅನೇಕ ಕಡೆ ನಾವು ಬಂಡೆಯ ಸಂದಿನಿಂದ ದೊಡ್ಡ ಗಿಡ ಎದ್ದಿರುವುದನ್ನು ನೋಡುತ್ತೇವೆ. ಬೆಳವಣಿಗೆಗೆ ವಿರೋಧವಾಗಿ ವಾತಾವರಣ ಇದ್ದರೂ ಅವನ ಕರುಣೆ ನಮ್ಮ ಮೇಲೆ ಬಿದ್ದರೆ ಹೇಗೊ ಅವನ ಕಡೆ ಹೋಗುವುದಕ್ಕೆ ಅವಕಾಶವನ್ನು ಅವನು ಕಲ್ಪಿಸುವನು.

ಅವನು ಚರ ಅಚರ ವಸ್ತುಗಳಿಗೆಲ್ಲಾ ತಂದೆಯಾಗಿರುವನು. ಎಲ್ಲವನ್ನು ಸೃಷ್ಟಿಸಿದವನು ಮಾತ್ರ ವಲ್ಲ. ಎಲ್ಲರ ರಕ್ಷಣೆಯೂ ಅವನ ಜವಾಬ್ದಾರಿ. ಅವನು ಯಾವುದನ್ನೂ ತಿರಸ್ಕರಿಸುವುದಿಲ್ಲ. ಬಂದುದೆಲ್ಲ ಮುಂದುವರಿಯುವುದಕ್ಕೆ ಅವಕಾಶ ಕಲ್ಪಿಸುವನು. ಬೆಟ್ಟದ ಮೇಲೆ ಹುಟ್ಟಿರುವ ಗಿಡಕ್ಕೆ ನೀರೆರೆಯುವನು. ಪಶುಪಕ್ಷಿ ಪ್ರಾಣಿಗಳಿಗೆಲ್ಲ ಕಾಲಕಾಲಕ್ಕೆ ಆಹಾರ ಒದಗಿಸುವನು. ಅವುಗಳಿಗೆಲ್ಲ ಅಷ್ಟು ಸುಂದರವಾದ ಬಟ್ಟೆಯನ್ನು ಕೊಟ್ಟಿರುವನು. ಪಕ್ಷಿಗಳಿಗೆಲ್ಲ ಎಷ್ಟೊಂದು ಆಕಾರದ ಪುಕ್ಕಗಳು, ಅದರಲ್ಲಿ ಎಂತೆಂತಹ ಬಣ್ಣಗಳು! ಬೆಳಗ್ಗೆ ಹುಟ್ಟಿ ಸಾಯಂಕಾಲ ಸತ್ತು ಬೀಳುವ ಚಿಟ್ಟೆಯ ರೆಕ್ಕೆಯ ಮೇಲೆ ಎಂತಹ ಅದ್ಭುತವಾದ ಚಿತ್ರ ಬರೆದಿರುವನು ದೇವರು! ಮನುಷ್ಯ ಅವುಗಳನ್ನು ನೋಡಿ ತನ್ನ ಬಟ್ಟೆಗೆ ಬಣ್ಣಹಾಕಬೇಕಾಗಿದೆ. ಯಾವಾಗ ಅವನು ಲೋಕಕ್ಕೆ ತಂದೆ ಎಂಬ ಭಾವ ಬರುವುದೊ, ನಾವು ಆಗ ಎಲ್ಲವನ್ನೂ ಪ್ರೀತಿಸುತ್ತೇವೆ. ಏಕೆಂದರೆ ಇವರೆಲ್ಲರೂ ನಮ್ಮಂತೆಯೆ ಭಗವಂತನ ಮಕ್ಕಳು.

ಅವನು ಬರೀ ತಂದೆಯಲ್ಲ, ಪೂಜ್ಯ ತಂದೆ. ಪರಮ ಪವಿತ್ರ ಭಗವಂತ. ಅವನಷ್ಟು ಗೌರವಕ್ಕೆ ಅರ್ಹವಾದುದು ಈ ಪ್ರಪಂಚದಲ್ಲಿ ಇಲ್ಲ. ಅವನು ಲೋಕಗುರು. ಆ ಲೋಕಗುರುವೇ ಹಲವು ಮಾನವ ಗುರುಗಳ ಮೂಲಕ ಕೆಲಸ ಮಾಡಿ ನಮ್ಮನ್ನು ಗುರಿಯೆಡೆಗೆ ಒಯ್ಯುತ್ತಿರುವನು, ನಮ್ಮ ಆಧ್ಯಾತ್ಮಿಕ ಜೀವನ ವಿಕಾಸವಾಗಲು ಅವಕಾಶ ಕಲ್ಪಿಸುವನು. ಅದಕ್ಕೆ ಬೇಕಾದ ಗ್ರಂಥಗಳನ್ನು ಒದಗಿಸುವನು. ಆ ಜೀವನ ವಿಕಾಸವಾಗುವುದಕ್ಕೆ ವಾತಾವರಣವನ್ನು ಕಲ್ಪಿಸುವನು. ಇರುವ ಆತಂಕ ಗಳನ್ನು ನಿವಾರಿಸುವನು, ದೊರಕದ ವಸ್ತುಗಳನ್ನು ಒದಗಿಸುವನು, ಲೋಕಗುರು ಸದಾಕಾಲದಲ್ಲಿ ಕೆಲಸ ಮಾಡುತ್ತಿರುವನು. ಯಾರು ಉತ್ತಮ ಜೀವನ ನಡೆಸಲು ತವಕಪಡುತ್ತಿರುವರೋ, ಅವರಿಗೆ ಭಗವಂತ ಹೇಗೆ ಕೆಲಸ ಮಾಡುತ್ತಿರುವನು ಎಂಬುದು ವೇದ್ಯ.

ಈ ಪ್ರಪಂಚದಲ್ಲಿ, ಜ್ಞಾನದಲ್ಲಿ ಶಕ್ತಿಯಲ್ಲಿ ಪವಿತ್ರತೆಯಲ್ಲಿ ಅವನಿಗೆ ಸರಿ ಸಮಾನರಾರು ಇರಬಲ್ಲರು? ನೂರಾರು ಜನ ಗೌಣ ದೇವರುಗಳು ಇರಬಹುದು. ಆದರೆ ಅವರುಗಳ ಮೂಲಕ ಕೆಲಸ ಮಾಡುತ್ತಿರುವವನು ದೇವದೇವನಾದ ಭಗವಂತನೊಬ್ಬನೇ. ಇತರರೆಲ್ಲ ಅವನ ಆಜ್ಞಾ ಧಾರಕರು. ಅವನಿಗೆ ಸಮನೇ ಇಲ್ಲದಿರುವಾಗ ಅವನನ್ನು ಮೀರಲು ಹೇಗೆ ಸಾಧ್ಯ?

\begin{verse}
ತಸ್ಮಾತ್ ಪ್ರಣಮ್ಯ ಪ್ರಣಿಧಾಯ ಕಾಯಂ \\ ಪ್ರಸಾದಯೇ ತ್ವಾಮಹಮೀಷಮೀಡ್ಯಮ್ ।\\ಪಿತೇವ ಪುತ್ರಸ್ಯ ಸಖೇವ ಸಖ್ಯುಃ \\ ಪ್ರಿಯಃ ಪ್ರಯಾಯಾರ್ಹಸಿ ದೇವ ಸೋಢುಮ್ \versenum{॥ ೪೪ ॥}
\end{verse}

{\small ದೇವ, ಆದಕಾರಣ ನನ್ನ ಶರೀರವನ್ನು ದಂಡದಂತೆ ಕೆಳಗೆ ಉರುಳಿಸಿ, ನಮಸ್ಕರಿಸಿ ಸ್ತುತ್ಯರ್ಹನೂ ಈಶ್ವರನೂ ಆದ ನಿನ್ನನ್ನು ಪ್ರಸನ್ನನಾಗೆಂದು ಪ್ರಾರ್ಥೀಸುತ್ತೇನೆ. ತಂದೆ ಮಗನ ಅಪರಾಧವನ್ನು, ಸ್ನೇಹಿತರು ಸ್ನೇಹಿತನ ಅಪರಾಧವನ್ನು, ಪ್ರಿಯನು ಪ್ರಿಯಳ ಅಪರಾಧವನ್ನು ಸಹಿಸಿಕೊಳ್ಳುವಂತೆ ಸಹಿಸಿಕೊಳ್ಳಬೇಕು.}

ಅರ್ಜುನ ತನ್ನ ಶರೀರವನ್ನು ದಂಡದಂತೆ ಕೆಳಗೆ ಉರುಳಿಸಿ ಸಾಷ್ಟಾಂಗ ಪ್ರಣಾಮ ಮಾಡುತ್ತಾನೆ. ಈ ನಮಸ್ಕಾರದಲ್ಲಿ ಅಹಂಕಾರ ಸಂಪೂರ್ಣವಾಗಿ ಶರಣಾಗಿದೆ ಭಗವಂತನಿಗೆ. ನಾನು ಎಂಬುದನ್ನು ಭಗವಂತನಿಗೆ ಅರ್ಪಿಸಿ, ದೇವರೆ, ನಾನು ಸೊನ್ನೆ, ನೀನು ಸೊನ್ನೆಗೆ ಬೆಲೆ ಕೊಡುವ ಒಂದು ಎಂದು ಒಪ್ಪಿಕೊಂಡಿರುವನು.

ನನ್ನ ಮೇಲೆ ಪ್ರಸನ್ನನಾಗು ಎಂದು ಬೇಡುತ್ತಿರುವನು. ತಾನೇನೊ ಹಿಂದೆ ದೊಡ್ಡ ಅಪರಾಧವನ್ನು ಇವನಿಗೆ ಮಾಡಿದ್ದೆ; ಅದನ್ನು ಕ್ಷಮಿಸಬೇಕು ಎಂದು ಬೇಡುವನು. ಹಾಗೆ ಕ್ಷಮಾಪಣೆಯನ್ನು ಬೇಡುವಾಗ ಮೂರು ಉದಾಹರಣೆಗಳನ್ನು ತರುವನು. ತಂದೆ ತನ್ನ ಮಗ ಮಾಡಿದ ತಪ್ಪನ್ನು ಮನ್ನಿಸುವಂತೆ ಮನ್ನಿಸು ಎಂದು ಕೇಳಿಕೊಳ್ಳುತ್ತಾನೆ. ಇಲ್ಲಿ ತಂದೆಗೆ ಎಲ್ಲಾ ಗೊತ್ತಿದೆ. ಮಗುವಿ ಗಾದರೋ ಪ್ರಪಂಚದಲ್ಲಿ ಯಾವುದನ್ನು ಮಾಡಬೇಕು, ಯಾವುದನ್ನು ಬಿಡಬೇಕು ಎಂಬುದು ಗೊತ್ತಿಲ್ಲ. ಅನೇಕ ವೇಳೆ ಮಾಡಬಾರದುದನ್ನು ಮಾಡುತ್ತದೆ, ಮಾಡುವುದನ್ನು ಬಿಡುತ್ತದೆ. ಆದರೆ ಪ್ರೀತಿ ಇರುವ ತಂದೆಯಾದರೊ, ಪಾಪ ಮಗು, ಅದಕ್ಕೆ ಗೊತ್ತಿಲ್ಲ, ಹಾಗೆ ಮಾಡಿತು ಎಂದು ಅದನ್ನೆಲ್ಲ ಮರೆಯುತ್ತಾನೆ. ಪ್ರೀತಿ ಇದ್ದರೆ ಅದು ಎಲ್ಲವನ್ನು ಮರೆಸುವುದು. ಪ್ರೀತಿಸುವ ವ್ಯಕ್ತಿಯಲ್ಲಿ ನಾವು ತಪ್ಪನ್ನೆ ಕಾಣುವುದಿಲ್ಲ. ಒಂದು ವೇಳೆ ಕಂಡರೂ, ಇದೇನು ಮಹಾ, ಎಂದು ತಳ್ಳಿಬಿಡುವೆವು. ಒಂದು ವೇಳೆ ಅದು ದೊಡ್ಡ ತಪ್ಪೇ ಎಂದು ಗೊತ್ತಾದರೂ, ಈ ಪ್ರಪಂಚದಲ್ಲಿ ತಪ್ಪುಮಾಡದವರು ಯಾರು ಇರುವರು ಎಂದು ತೇಲಿಸಿಬಿಡುತ್ತೇವೆ. ಪ್ರೀತಿಯೊಂದಿದ್ದರೆ ಶತಶತ ಅಪರಾಧಗಳನ್ನೂ ಕ್ಷಮಿಸ ಬಲ್ಲದು. ಯಾವಾಗ ಪ್ರೀತಿ ಇಲ್ಲವೊ, ಸಣ್ಣದು ಅಕ್ಷಮ್ಯವಾದ ಮಹಾ ಅಪರಾಧವಾಗಿ ಪರಿಣಮಿಸು ವುದು. ಎಲ್ಲಿ ಪ್ರೀತಿ ಇದೆಯೊ ಅದನ್ನು ಗಮನಿಸಿಸುವುದೇ ಇಲ್ಲ. ಸ್ನೇಹಿತನನ್ನು ಮನ್ನಿಸುವಂತೆ, ಪ್ರಿಯ ಪ್ರಿಯಳ ಅಪರಾಧವನ್ನು ಮನ್ನಿಸುವಂತೆ ಮನ್ನಿಸು ಎನ್ನುತ್ತಾನೆ. ಸ್ನೇಹಿತರನ್ನು ಪ್ರಿಯ ಪ್ರಿಯಳನ್ನು ಬಂಧಿಸಿರುವುದು ಪ್ರೀತಿಯೆ. ಪ್ರೀತಿಯ ಆಧಾರದಮೇಲೆ ನಿಂತುಕೊಂಡರೆ ಮಾತ್ರ ಏನನ್ನು ಬೇಕಾದರೂ ಕ್ಷಮಿಸಬಹುದು. ಅರ್ಜುನ ಭಗವಂತನನ್ನು ಪ್ರಾರ್ಥಿಸುವುದು ಈ ದೃಷ್ಟಿ ನಾವೊಂದು ತಪ್ಪಿನ ಕಂತೆಯೇ ಆಗಿದ್ದೇವೆ. ಅವನು ನಮ್ಮ ಯೋಗ್ಯತೆಗೆ ತಕ್ಕ ವರ ಕೊಡುವಂತೆ ಆದರೆ ನಾವು ಎಂದೆಂದಿಗೂ ಉದ್ಧಾರವಾಗುವಂತೆಯೇ ಇಲ್ಲ. ಅವನು ನಮ್ಮ ಯೋಗ್ಯತೆಗಿಂತ ಹೆಚ್ಚನ್ನು ಕೊಡುತ್ತಾನೆ. ನಾವು ಮಾಡಿರುವ ಮಹಾಪರಾಧಗಳನ್ನೆಲ್ಲಾ ಮನ್ನಿಸುತ್ತಾನೆ. ಗಾಢಾಂಧಕಾರದಲ್ಲಿ ಮಿಂಚುತ್ತಿರುವ ಹುಳುಗಳಂತೆ ನಮ್ಮ ಒಳ್ಳೆಯ ಗುಣ. ಎಲ್ಲೊ ಮಿಣುಕು ಮಿಣುಕು ಎನ್ನುತ್ತಿದೆ. ದೇವರು ನಮ್ಮಲ್ಲಿರುವ ಕತ್ತಲೆಯನ್ನೆಲ್ಲ ಮರೆತು ಆ ಮಿಣುಕು ಹುಳವನ್ನು ಎತ್ತಿಕೊಂಡು ಅದನ್ನು ಮೆಚ್ಚುವನು.ದೇವರೇ ನಾನು ಮಹಾಪರಾಧಮಾಡಿದೆ, ನನ್ನನ್ನು ಉದ್ಧರಿಸು ಎನ್ನುವುದು ನಮ್ಮ ಒಂದು ಒಳ್ಳೆಯ ಗುಣ. ನಾವು ಇದನ್ನು ತೋರಿಸಿದರೆ ಸಾಕು, ದೇವರು ಮೆಚ್ಚುತ್ತಾನೆ, ಕ್ಷಮಿಸುತ್ತಾನೆ. ಅದರಲ್ಲಿ ಸಂದೇಹವೇ ಇಲ್ಲ. ಅರ್ಜುನ ಭಗವಂತನನ್ನು ಬೇಡುತ್ತಿರುವುದು ಕೂಡ ಈ ವರವನ್ನೇ.

\begin{verse}
ಅದೃಷ್ಟಪೂರ್ವಂ ಹೃಷಿತೋಽಸ್ಮಿ ದೃಷ್ಟ್ವಾ ಭಯೇನ ಚ ಪ್ರವ್ಯಥಿತಂ ಮನೋ ಮೇ ।\\ತದೇವ ಮೇ ದರ್ಶಯ ದೇವ ರೂಪಂ ಪ್ರಸೀದ ದೇವೇಶ ಜಗನ್ನಿವಾಸ \versenum{॥ ೪೫ ॥}
\end{verse}

ದೇವ, ಹಿಂದೆ ಎಂದೂ ನೋಡದ ವಿಶ್ವರೂಪವನ್ನು ನೋಡಿ ಹರ್ಷಿತವಾಗಿದ್ದೇನೆ ಮತ್ತು ಭಯದಿಂದ ನನ್ನ ಮನಸ್ಸು ಕ್ಲೇಶಕೊಂಡಿದೆ. ದೇವೇಶ, ನಿನ್ನ ಆ ರೂಪವನ್ನೇ ನನಗೆ ತೋರಿಸು. ಜಗನ್ನಿವಾಸ ದಯೆ ತೋರು.

ಇಲ್ಲಿ ಅರ್ಜುನ ಎರಡು ಭಾವಗಳಿಂದ ತಾಡಿತನಾಗಿರುವನು. ಒಂದು ಹಿಂದೆ ಎಂದೂ ನೋಡದ ಭಗವಂತನ ಐಶ್ವರ್ಯ ರೂಪವನ್ನು ನೋಡಿ ಆನಂದಪಡುತ್ತಿರುವನು. ಮತ್ತು ಅದೇ ಭಯಂಕರ ವಿಶ್ವರೂಪವನ್ನು ನೋಡಿ ತಲ್ಲಣಿಸುತ್ತಿರುವನು. ಅದನ್ನು ಹೆಚ್ಚು ಕಾಲ ನೋಡಲಾರನು. ಇವನ ಮನಸ್ಸು ಭಗವಂತನ ರುದ್ರರೂಪವನ್ನು ನೋಡುವುದಕ್ಕೆ ಇನ್ನೂ ಅಣಿಯಾಗಿಲ್ಲ. ತುಂಬಾ ಸಿಹಿಯಾಗಿರುವುದನ್ನು ನಾವು ಹೆಚ್ಚು ತಿನ್ನುವುದಕ್ಕೆ ಆಗುವುದಿಲ್ಲ. ಸ್ವಲ್ಪ ತಿನ್ನುವುದರೊಳಗೆ ವಾಂತಿ ಬರುವುದು. ಹಾಗಾಗಿದೆ ವಿಶ್ವರೂಪ ಅರ್ಜುನನಿಗೆ. ನಿನ್ನ ರೂಪವನ್ನೇ ತೋರಿಸು ಎನ್ನುತ್ತಾನೆ. ಅಂದರೆ ಶ್ರೀಕೃಷ್ಣನ ಹಳೆಯ ರೂಪ. ಯಾವ ರೂಪಿನಲ್ಲಿ ಅವನನ್ನು ಸಖ ಎಂದು ಕರೆದನೊ, ಯಾವ ರೂಪಿನಲ್ಲಿ ಈಗ ಸಾರಥ್ಯವನ್ನು ಮಾಡುತ್ತಿರುವನೊ ಅದನ್ನೆ ತೋರು ಎನ್ನುವನು. ಜೀವನದಲ್ಲಿ ಒಂದು ವಸ್ತುವನ್ನು ಅದು ಏನು ಎಂಬುದನ್ನು ಆಲೋಚಿಸದೆ ಇಚ್ಛಿಸುತ್ತೇವ. ಅದು ಬಂದಾಗ ತಲ್ಲಣಿಸುತ್ತೇವೆ. ಏಕೆಂದರೆ ನಾವು ಸತ್ಯವನ್ನು ಒಂದು ರೀತಿ ಕಲ್ಪಿಸಿಕೊಂಡಿರುವೆವು. ವಾಸ್ತವದ ಸಿಡಿಲು ಹೊಡೆಯಿತು ಎಂದರೆ ನಮ್ಮ ಸುಂದರ ಕಲ್ಪನೆಯ ಅರಮನೆ ಪುಡಿಪುಡಿಯಾಗುವುದು. ಆ ಬಡಿತವನ್ನು ಸಹಿಸುವುದಕ್ಕೆ ಸಿದ್ಧರಾಗಿಲ್ಲ ನಾವಿನ್ನು.

\begin{verse}
ಕಿರೀಟಿನಂ ಗದಿನಂ ಚಕ್ರಹಸ್ತ ಮಿಚ್ಛಾಮಿ ತ್ವಾಂ ದ್ರಷ್ಟುಮಹಂ ತಥೈವ ।\\ತೇನೈವ ರೂಪೇಣ ಚತುರ್ಭುಜೇನ ಸಹಸ್ರಬಾಹೋ ಭವ ವಿಶ್ವಮೂರ್ತೇ \versenum{॥ ೪೬ ॥}
\end{verse}

ಸಹಸ್ರ ಬಾಹುವೆ, ಮೊದಲಿನಂತೆ ಕಿರೀಟಧಾರಿಯೂ ಗದಾಧರನೂ ಚಕ್ರಪಾಣಿಯೂ ಆಗಿರುವ ನಿನ್ನನ್ನು ನೋಡಲು ಇಚ್ಛಿಸುತ್ತೇನೆ. ವಿಶ್ವಮೂರ್ತೆ, ಚತುರ್ಭುಜವುಳ್ಳ ನಿನ್ನ ಹಿಂದಿನ ರೂಪಿನಿಂದಲೇ ಇರುವವನಾಗು.

ಅರ್ಜುನನಿಗೆ ಸಹಸ್ರಬಾಹುಗಳುಳ್ಳು ವಿಶ್ವರೂಪ ಬೇಕಾಗಿಲ್ಲ. ಅದು ಬಹಳ ಉಗ್ರವಾದದು. ಅದನ್ನು ಹೆಚ್ಚುಕಾಲ ಸಹಿಸಲಾರನು. ಅವನಿಗೆ ಚತುರ್ಭುಜವೇ ಸಾಕು. ಶಂಖಚಕ್ರಗದಾಪದ್ಮವೇ ಸಾಕು. ಎಲ್ಲಾ ವಿಧವಾದ ಆಯುಧಗಳನ್ನು ಅವನು ಧರಿಸಿರಬೇಕಾಗಿಲ್ಲ. ಗದೆ ಹತ್ತಿರದಲ್ಲಿರುವ ದುಷ್ಟರನ್ನು ದಮನಮಾಡುವುದುಕ್ಕೆ ಚಕ್ರ ದೂರದಲ್ಲಿರುವ ದುಷ್ಟರನ್ನು ನಿಗ್ರಹಿಸುವುದಕ್ಕೆ. ಅವನಿಗೆ ಸರ್ವವ್ಯಾಪಿಯಾದ ವಿಶ್ವಮೂರ್ತಿ ಬೇಕಾಗಿಲ್ಲ. ವಸುದೇವನ ಪುತ್ರ ಸಾಕು. ನೋಡಲು ಸೌಮ್ಯವಾಗಿರಬೇಕು. ಸೂರ್ಯ ಹುಟ್ಟುತ್ತಿರುವಾಗ ಅಥವಾ ಮುಳುಗುತ್ತಿರುವಾಗ ನೋಡಲು ಸುಂದರವಾಗಿರುವನು. ಅವನನ್ನು ಆಗ ಎಷ್ಟು ಹೊತ್ತು ಬೇಕಾದರೂ ನೋಡಬಹುದು. ಅದೇ ಸೂರ್ಯ ನೆತ್ತಿಯಮೇಲಿರುವಾಗ ಒಂದು ಕ್ಷಣ ನೋಡುವುದರೊಳಗೆ ಕಣ್ಣಿಗೆ ಕತ್ತಲೆ ಕಟ್ಟಿದಂತಾಗುವುದು. ಆಗಲೇ ಭಗವಂತ ದಯೆಯಿಂದ ಅರ್ಜುನನಿಗೆ ಹೇಳುತ್ತಾನೆ.

\begin{verse}
ಮಯಾ ಪ್ರಸನ್ನೇನ ತವಾರ್ಜುನೇದಂ \\ ರೂಪಂ ಪರಂ ದರ್ಶಿತಮಾತ್ಮಯೋಗಾತ್ ।\\ತೇಜೋಮಯಂ ವಿಶ್ವಮನಂತಮಾದ್ಯಂ \\ ಯನ್ಮೇ ತ್ವದನ್ಯೇನ ನ ದೃಷ್ಟಪೂರ್ವಂ \versenum{॥ ೪೭ ॥}
\end{verse}

{\small ಅರ್ಜುನ, ಪ್ರಸನ್ನನಾದ ನಾನು ತೇಜೋಮಯವೂ ಪರಿಪೂರ್ಣವೂ ಅನಂತವೂ ಆದ್ಯವೂ ಆದ ಈ ವಿಶ್ವರೂಪವನ್ನು ನನ್ನ ಯೋಗದ ಸಾಮರ್ಥ್ಯದಿಂದ ತೋರಿಸಿದೆನು. ನಿನ್ನ ಹೊರತು ಬೇರೆ ಯಾರೂ ಹಿಂದೆ ಇದನ್ನು ನೋಡಿರಲಿಲ್ಲ.}

ಶ್ರೀಕೃಷ್ಣ ಅರ್ಜುನನ ಭಕ್ತಿಗೆ ಮೆಚ್ಚಿ ಅವನು ಕೇಳಿದುದನ್ನು ಕೊಟ್ಟನು. ಭಗವಂತ ಇನ್ನು ಯಾವುದಕ್ಕೂ ಒಲಿಯುವುದಿಲ್ಲ, ನಾನು ನಿನ್ನಲ್ಲಿ ಶರಣಾದೆ, ನೀನು ನನ್ನನ್ನು ಉದ್ಧಾರಮಾಡಬೇಕು, ಎಂದು ಎಳೆಯ ಮಗುವಿನಂತೆ ಯಾವಾಗ ದೇವರಲ್ಲಿ ಶರಣಾಗುತ್ತೇವೆಯೋ ಆಗಲೆ ದೇವರಿಗೆ ಭಕ್ತನ ಮೇಲೆ ಪ್ರೀತಿ ವ್ಯಕ್ತವಾಗುವುದು. ನಾವು ಅವನಲ್ಲಿ ಶರಣಾದರೆ ನಮ್ಮನ್ನು ಉದ್ಧರಿಸುವದಕ್ಕೆ ಅವನ ಕೈಗಳು ಮೇಲಕ್ಕೆ ಎತ್ತುವುವು. ಭಕ್ತ ಏನನ್ನು ಕೇಳುತ್ತಾನೆಯೊ ಅದನ್ನು ದೇವರು ಇಲ್ಲ ಎನ್ನುವುದಿಲ್ಲ. ಅರ್ಜುನ ಭಗವಂತನ ವಿಶ್ವರೂಪವನ್ನೇ ನೋಡಬೇಕು ಎಂದು ಕೋರಿಕೊಂಡ. ಅದನ್ನು ದೇವರು ತೋರಿದ. ಆದರೆ ಅದನ್ನು ತೋರಬೇಕಾದರೆ ಅದ್ಭುತ ಯೋಗಶಕ್ತಿ ಇರಬೇಕು. ಅದನ್ನು ಒಬ್ಬ ನೋಡಬೇಕಾದರೂ ನಮ್ಮ ಜಡ ಕಣ್ಣು ಸಾಲದು, ದಿವ್ಯ ಚಕ್ಷುಸ್ಸು ಬೇಕು. ಶ್ರೀಕೃಷ್ಣ ಅರ್ಜುನನಿಗೆ ದಿವ್ಯ ಚಕ್ಷುಸ್ಸನ್ನು ಅನುಗ್ರಹಿಸಿ ಅದರ ಮೂಲಕ ಭಗವಂತನ ವಿಶ್ವರೂಪವನ್ನು ನೋಡುವಂತೆ ಮಾಡಿದನು.

ತೋರಿದ ವಿಶ್ವರೂಪವು ತೇಜೋಮಯವಾಗಿದೆ, ಕೋಟಿಸೂರ್ಯ ಪ್ರಕಾಶಮಾನವಾಗಿದೆ. ಅದು ಸಂಪೂರ್ಣವಾಗಿದೆ. ದಯಾಮಯನಾದ ಸೃಷ್ಟಿಯನ್ನು ಪಾಲಿಸುತ್ತಿರುವ ವಿಷ್ಣುವಿನ ಸ್ವರೂಪಕ್ಕಿಂತ ಹೆಚ್ಚಾಗಿ, ಪ್ರಪಂಚನ್ನೆಲ್ಲ ಸಂಹರಿಸುವ ಕಾಲರುದ್ರನಂತೆ ಇದ್ದ ವಿಶ್ವರೂಪವನ್ನು ನೋಡಿದ. ಅದು ಅನಂತವಾಗಿರುವುದನ್ನು ನೋಡಿದನು. ಎಲ್ಲೆಲ್ಲಿಯೂ ಅವನ ಕೈಗಳೇ, ಕಾಲಗಳೇ, ಮುಖಗಳೇ. ಅವನಿಲ್ಲದ ಸ್ಥಳವೇ ಇರಲಿಲ್ಲ. ಭಗವಂತನ ಈ ಶಕ್ತಿ ಪರಾಕ್ರಮ ಮತ್ತು ವಿಭೂತಿಗಳು ಆದಿ ಯಿಂದಲೂ ಇದೆ. ಇದನ್ನೇನು ಅವನು ನಮ್ಮಂತೆ ಸಾಧನಬಲದಿಂದ ಹೊಸದಾಗಿ ಸಂಪಾದಿಸಿಕೊಂಡಿ ದ್ದಲ್ಲ. ಇದು ಅವನ ಸಹಜ ಸ್ಥಿತಿ. ಸೂರ್ಯ ಪ್ರಯತ್ನಪಟ್ಟು ಬೆಳಗುವುದನ್ನು ಪಡೆಯಬೇಕಾಗಿಲ್ಲ. ಹಾಲು ಪ್ರಯತ್ನಪಟ್ಟು ಬಿಳಿಯಾಗಬೇಕಾಗಿಲ್ಲ. ಅದರ ಸ್ವಭಾವವೇ ಅದು. ಅದರಂತೆಯೇ ಆದಿ ಯಿಂದಲೂ ಭಗವಂತನು ಹಾಗೆಯೇ ಇದ್ದನು.

ಇಂತಹ ವಿಶ್ವರೂಪವನ್ನು ಹಿಂದೆ ಯಾರೂ ಇದುವರೆಗೆ ನೋಡಿಲ್ಲ ಎಂದು ಶ್ರೀಕೃಷ್ಣ ಹೇಳು ತ್ತಾನೆ. ಬಹುಶಃ ಯಾರೂ ಇದನ್ನು ಹೀಗೆ ನೋಡಲು ಕಾತರರಾಗಿರಲಿಲ್ಲವೆಂದು ಕಾಣುವುದು. ಯಾರೂ ಇದನ್ನು ಕೇಳಿರಲಿಲ್ಲ, ಅದಕ್ಕಾಗಿ ತೋರಲಿಲ್ಲ. ಕೇಳಿದ್ದರೆ ಭಗವಂತ ತೋರುತ್ತಿದ್ದ ಎಂದು ಊಹಿಸಬಹುದು. ಅವನಿಗೆ ಅರ್ಜುನ ಹೇಗೆ ಭಕ್ತನೋ ಹಾಗೆ ಎಲ್ಲರೂ ಕೂಡ. ಆದರೆ ಭಕ್ತನು ಆಶಿಸುವುದು ಪ್ರೀತಿಸಬಲ್ಲ ಒಬ್ಬ ಪರಮೇಶ್ವರನನ್ನು. ನೋಡಿದರೆ ಅಂಜಿಕೊಂಡು ಓಡಿಹೋಗುವ ಭಯಂಕರವಾದ ರೂಪವನ್ನು ಅಲ್ಲ. ಇಲ್ಲಿ ಅರ್ಜುನನಿಗೆ ತೋರುವ ವಿಶ್ವರೂಪ ಹಿಂದೆ ಭಗವಂತನೇ ಕೆಲವರಿಗೆ ತೋರಿದ ವಿಶ್ವರೂಪದಂತೆ ಇಲ್ಲ. ತನ್ನ ತಾಯಿ ಯಶೋದೆಗೆ ಬಾಯಿಯೊಳಗೆ ಬ್ರಹ್ಮಾಂಡ ತೋರಿದ. ದುರ್ಯೋಧನನ ಸಭೆಯಲ್ಲಿ ಎಲ್ಲರೂ ಅನಂತಕೃಷ್ಣರಂತೆ ಕಾಣುತ್ತಾರೆ. ಯುದ್ಧವನ್ನು ಪೂರೈಸಿಕೊಂಡು ಹಿಂತಿರುಗಿ ಹೋಗುತ್ತಿದ್ದಾಗ ಉದಂಕನೆಂಬ ಪುಷಿ ಶ್ರೀಕೃಷ್ಣನನ್ನು ವಿಶ್ವರೂಪ ವನ್ನು ತೋರು ಎಂದು ಕೇಳಿಕೊಂಡಾಗ ಅವನಿಗೆ ತೋರುವನು. ಆದರೆ ಅರ್ಜುನ ನೋಡಿದ ವಿಶ್ವರೂಪು ಬೇರೆ ವಿಧದ್ದು. ಇದನ್ನು ಶ್ರೀಕೃಷ್ಣ ಅರ್ಜುನನಿಗೆ ಮಾತ್ರ ತೋರುತ್ತಾನೆ. ಏಕೆಂದರೆ ಅವನಿಗೆ ನಿಸ್ಸಂಶಯವಾಗಿ ಗೊತ್ತಾಗಬೇಕು, ಈಗ ಪ್ರಳಯಮಾಡುವುದಕ್ಕೆ ಭಗವಂತನೇ ನಿಂತಿರು ವನು; ಅರ್ಜುನ ಮಾಡುವುದಿಲ್ಲವೆಂದರೂ ಅದು ನಿಲ್ಲುವಹಾಗಿಲ್ಲ ಎಂಬುದು. ಆ ಸಂಹಾರದ ಕೆಲಸ ಆಗಲೇಬೇಕಾಗಿದೆ. ಅದು ಆಗಿಯೇ ಆಗುವುದು. ಅರ್ಜುನ ಮಾಡಲಿ ಬಿಡಲಿ ಅದೇನು ನಿಲ್ಲುವ ಹಾಗಿಲ್ಲ. ಭಗವಂತನ ಲೀಲಾನಾಟಕದಲ್ಲಿ ಅರ್ಜುನ ಒಬ್ಬ ಪಾತ್ರಧಾರಿ ಅಷ್ಟೆ, ಅವನು ಬೇಕಾದರೆ ತನ್ನ ಪಾತ್ರವನ್ನು ಅಭಿನಯಿಸಬಹುದು. ಇಲ್ಲದೇ ಇದ್ದರೆ, ದೇವರು ಇನ್ನು ಯಾರನ್ನೋ ತೆಗೆದು ಕೊಂಡು ಅವರ ಮೂಲಕ ಆ ಕೆಲಸವನ್ನು ಮಾಡಿಸುತ್ತಾನೆ. ಅವನೇನು ನಾಟಕವನ್ನು ನಿಲ್ಲಿಸುವುದಿಲ್ಲ. ಈ ಘಟನೆ ಚೆನ್ನಾಗಿ ಅರ್ಜುನನ ಮನಸ್ಸಿಗೆ ನಾಟುವಂತೆ ಹೇಳಬೇಕಾಗಿದೆ. ನಾವು ಹೇಳುವ ಮಾತನ್ನು ನಿದರ್ಶಿಸುವುದಕ್ಕೆ ಒಂದು ಚಿತ್ರವಿದ್ದರೆ ಅದನ್ನು ಹೆಚ್ಚು ಪರಿಣಾಮಕರವಾದ ರೀತಿಯಲ್ಲಿ ಬೋಧಿಸ ಬಹುದು. ಅದಕ್ಕಾಗಿ ವಿಶ್ವರೂಪದ ಮೂಲಕವಾಗಿ ಬೋಧಿಸುತ್ತಾನೆ.

\begin{verse}
ನ ವೇದಯಜ್ಞಾಧ್ಯಯನೈರ್ನದಾನೈರ್ನ ಚ ಕ್ರಿಯಾಭಿರ್ನ ತಪೋಭಿರುಗ್ರೈಃ ।\\ಏವಂರೂಪಃ ಶಕ್ಯ ಅಹಂ ನೃಲೋಕೇ ದ್ರಷ್ಟುಂ ತ್ವದನ್ಯೇನ ಕುರುಪ್ರವೀರ \versenum{॥ ೪೮ ॥}
\end{verse}

{\small ಕುರುಪ್ರವೀರ, ವೇದ ಯಜ್ಞ ಅಧ್ಯಯನ–ಇವುಗಳಿಂದಾಗಲೀ ದಾನಗಳಿಂದಲಾಗಲೀ, ಮತ್ತು ಕರ್ಮಗಳಿಂದ ಲಾಗಲೀ, ಉಗ್ರವಾದ ತಪಸ್ಸುಗಳಿಂದಲಾಗಲೀ ಈ ರೂಪವುಳ್ಳ ನನ್ನನ್ನು ಮನುಷ್ಯ ಲೋಕದಲ್ಲಿ ನಿನ್ನ ಹೊರತು ಮತ್ತೊಬ್ಬನಿಂದ ನೋಡಲು ಶಕ್ಯವಿಲ್ಲ.}

ಭಗವಂತನ ವಿಶ್ವರೂಪವನ್ನು ನೋಡುವುದಕ್ಕೆ ನಮಗೆ ಆಸೆ ಇದ್ದರೆ ಮಾತ್ರ ಸಾಲದು. ಅದನ್ನು ನೋಡಲು ಯೋಗ್ಯತೆಯೂ ಬೇಕು. ದೇವರು ನಮಗೆ ತೋರಿಸಲು ಮನಸ್ಸು ಮಾಡಬೇಕು. ನಮಗೂ ಅದನ್ನು ಸಹಿಸಲು ಶಕ್ತಿ ಇರಬೇಕು. ಬರೀ ವೇದಾಧ್ಯಯನದಿಂದ ನಮಗೆ ಈ ಶಕ್ತಿ ಅಥವಾ ಯೋಗ್ಯತೆ ಬರುವುದಿಲ್ಲ. ವೇದಗಳೆಲ್ಲ ಹೆಚ್ಚು ಎಂದರೆ ಅವನ ಕಡೆಗೆ ಹೋಗುವುದಕ್ಕೆ ದಾರಿತೋರುವ ಕೈಮರದಂತೆ ಇವೆ. ಬರೀ ಕೈ ಮರವನ್ನೇ ತಬ್ಬಿ ನಿಂತರೆ ನಾವು ಗುರಿ ಸೇರಲಾರೆವು. ಅದನ್ನು ಅತಿಕ್ರಮಿಸಿ ನಡೆಯಬೇಕು.

ಬರೀ ಯಜ್ಞಗಳ ಜ್ಞಾನದಿಂದಲೂ ನಾವು ಭಗವಂತನನ್ನು ನೋಡುವ ಯೋಗ್ಯತೆಯನ್ನು ಪಡೆದು ಕೊಳ್ಳಲಾರೆವು. ಯಾವ ಯಾವ ಯಜ್ಞಗಳನ್ನು ಹೇಗೆ ಹೇಗೆ ಮಾಡಬೇಕೆಂಬುದನ್ನು ತಿಳಿದಿದ್ದರೆ, ಕೆಲವು ವೇಳೆ ಮುಂದಿನ ಜನ್ಮದಲ್ಲಿಯೋ, ಈ ಜನ್ಮದಲ್ಲಿಯೋ ನಮ್ಮ ಬಯಕೆಗಳು ಈಡೇರಬಹುದು. ಇವುಗಳ ಹಿಂದೆಲ್ಲ ಕಾಮನೆಗಳು ಇರುವುವು. ಇಂತಹ ಕಾಮನೆಗಳಿಂದ ಕೂಡಿದ ಯಾಗಯಜ್ಞಗಳನ್ನು ಮಾಡುವುದರಿಂದ ನಾವು ಭಗವಂತನನ್ನು ಪಡೆಯಲಾರೆವು. ದಾನಾದಿಗಳಿಂದ ನಮ್ಮ ಕೆಲವು ಪಾಪಗಳು ನಾಶವಾಗಿ ಪುಣ್ಯವನ್ನು ಸಂಪಾದಿಸಬಹುದೇ ಹೊರತು ಭಗವಂತನನ್ನು ಇಂತಹ ಕರ್ಮ ಗಳಿಂದ ಪಡೆಯಲು ಸಾಧ್ಯವಿಲ್ಲ. ಅಗ್ನಿಷ್ಟೋಮ ಮುಂತಾದ ಶ್ರೌತ ಕರ್ಮಗಳಿಂದಲೂ ಅವನನ್ನು ಪಡೆಯಲು ಸಾಧ್ಯವಿಲ್ಲ.

ಉಗ್ರವಾದ ತಪಸ್ಸಿನಿಂದಲೂ ಅವನನ್ನು ಪಡೆಯಲಾರೆವು. ಅವನ ಮೇಲೆ ನಮಗೆ ಪ್ರೀತಿ ಇರಬೇಕು. ಭಗವಂತನೇ ಮನಸ್ಸು ಮಾಡಿ ತನ್ನನ್ನು ವ್ಯಕ್ತಪಡಿಸಿಕೊಳ್ಳಬೇಕು. ಜೊತೆಗೆ ಅರ್ಜುನನಿಗೆ ಕೊಟ್ಟಂತೆ ದಿವ್ಯ ಚಕ್ಷುಸ್ಸನ್ನೂ ಕೊಡಬೇಕು. ಆಗ ಮಾತ್ರ ಅವನನ್ನು ನೋಡಲು ಸಾಧ್ಯ. ಜಗದ ಸಂತೆಯಲ್ಲಿ ಅವನನ್ನು ಬೆಲೆಕೊಟ್ಟು ಕೊಂಡುಕೊಳ್ಳಲು ಸಾಧ್ಯವಿಲ್ಲ. ಅವನೇ ಮನಸ್ಸು ಮಾಡಬೇಕು. ಆಗ ಮಾತ್ರ ನಾವು ನೋಡಲು ಸಾಧ್ಯ.

\begin{verse}
ಮಾ ತೇ ವ್ಯಥಾ ಮಾ ಚ ವಿಮೂಢಭಾವೋ \\ ದೃಷ್ಟ್ವಾ ರೂಪಂ ಘೋರಮೀದೃಙ್ಮಮೇದಮ್ ।\\ವ್ಯಪೇತಭೀಃ ಪ್ರೀತಮನಾಃ ಪುನಸ್ತ್ವಂ \\ ತದೇವ ಮೇ ರೂಪಮಿದಂ ಪ್ರಪಶ್ಯ \versenum{॥ ೪೯ ॥}
\end{verse}

{\small ಇಂತಹ ಘೋರವಾದ ರೂಪವನ್ನು ನೋಡಿ ನಿನಗೆ ಭಯವೂ ಭ್ರಾಂತಿಯೂ ಆಗದಿರಲಿ. ನೀನು ಭಯವನ್ನು ತ್ಯಜಿಸಿ ಪ್ರಸನ್ನಚಿತ್ತನಾಗಿ ಪುನಃ ನನ್ನ ಅದೇ ರೂಪವನ್ನು ನೋಡು.}

ಇಂತಹ ಘೋರವಾದ ರೂಪವನ್ನು ನೋಡಿದರೆ, ಮನಸ್ಸೆಲ್ಲ ಕಲಕಿ ಹೋಗುವುದು, ಭಕ್ತಿಯಲ್ಲ, ಭಯವುಂಟಾಗುವುದು. ಜ್ಞಾನ ಬರುವ ಬದಲು ಅಂಜಿಕೆ ಆವರಿಸುವುದು. ವಿದ್ಯುಚ್ಛಕ್ತಿ ಹರಿಯುವ ತಂತಿಯನ್ನು ಮುಟ್ಟಿದಂತಾಗುವುದು. ಮೈಯೆಲ್ಲ ಜುಮ್ ಎನ್ನುವುದು. ಸಾಯಲಿಲ್ಲ, ಆದರೆ ಸಾವಿನ ದಾಡೆಯೊಳಗೆ ಹೋದಂತಿರುವುದು. ಈಗ ಈ ಭಯವನ್ನು ಬಿಟ್ಟು ಶಾಂತಿಯನ್ನು ಹೊಂದಿ ಆ ಉಗ್ರರೂಪದ ಹಿಂದೆ ಇರುವ ಸೌಮ್ಯ ಸ್ವರೂಪವನ್ನು ನೋಡು ಎಂದು ಆ ಉಗ್ರರೂಪವನ್ನು ಹಿಂತಿರುಗಿ ತೆಗೆದುಕೊಳ್ಳುತ್ತಾನೆ. ಅರ್ಜುನನಿಗೆ ಚಿರಪರಿಚಿತನಾದ ಮಂದಹಾಸದಿಂದ ಕೂಡಿದ ಚತುರ್ಭುಜನಾದ ಶ್ರೀಕೃಷ್ಣನನ್ನು ತೋರುತ್ತಾನೆ.

ಆಗ ಸಂಜಯ ಹೀಗೆ ಹೇಳುತ್ತಾನೆ:

\begin{verse}
ಇತ್ಯರ್ಜುನಂ ವಾಸುದೇವಸ್ತಥೋಕ್ತ್ವಾ \\ ಸ್ವಕಂ ರೂಪಂ ದರ್ಶಯಾಮಾಸ ಭೂಯಃ ।\\ಆಶ್ವಾಸಯಾಮಾಸ ಚ ಭೀತಮೇನಂ \\ ಭೂತ್ವಾ ಪುನಃ ಸೌಮ್ಯವಪುರ್ಮಹಾತ್ಮಾ \versenum{॥ ೫ಂ ॥}
\end{verse}

{\small ಈ ರೀತಿ ಹೇಳಿ ವಾಸುದೇವ ಮತ್ತೆ ಶಾಂತರೂಪವನ್ನು ಧಾರಣೆಮಾಡಿ ಭಯಭೀತನಾದ ಅರ್ಜುನನ್ನು ಸಮಾಧಾನಗೊಳಿಸಿದನು.}

ಅರ್ಜುನ ಭಗವಂತನ ಶಾಂತ ಸ್ವರೂಪವನ್ನು ನೋಡಿ ಸಮಾಧಾನಪಟ್ಟು ಹೇಳುತ್ತಾನೆ:

\begin{verse}
ದೃಷ್ಟ್ವೇದಂ ಮಾನುಷಂ ರೂಪಂ ತವ ಸೌಮ್ಯಂ ಜನಾರ್ದನ ।\\ಇದಾನೀಮಸ್ಮಿ ಸಂವೃತ್ತಃ ಸಚೇತಾಃ ಪ್ರಕೃತಿಂ ಗತಃ \versenum{॥ ೫೧ ॥}
\end{verse}

{\small ಜನಾರ್ದನ, ನಿನ್ನ ಸೌಮ್ಯವಾದ ಮನುಷ್ಯರೂಪವನ್ನು ನೋಡಿ, ಈಗ ನಾನು ಪ್ರಸನ್ನಚಿತ್ತವುಳ್ಳವನಾಗಿ ಸ್ವಾಸ್ಥ್ಯ ವನ್ನು ಪಡೆದಿರುತ್ತೇನೆ.}

ಅರ್ಜುನ ತನ್ನ ಸಾರಥ್ಯವನ್ನು ಮಾಡುತ್ತಿದ್ದ ಮಾನವ ಸಹಜವಾದ ಶ್ರೀಕೃಷ್ಣನನ್ನು ನೋಡಿದಾಗ ಚಿತ್ತ ಸಮಾಧಾನಗೊಳ್ಳುವುದು. ಅಂಜಿಕೆ ಬಿಟ್ಟುಹೋಗುವುದು. ಅವನು ತನ್ನ ಹಿಂದಿನ ಸ್ಥಿತಿಗೆ ಬರುವನು. ವಿಶ್ವರೂಪ ಭಗವಂತನ ವಿಶೇಷರೂಪ. ಅದೊಂದು ಭೀಕರವಾದ ಕನಸಿನಂತೆ ಅರ್ಜುನ ನಿಗೆ ಕಾಣುವುದು. ದುಃಸ್ವಪ್ನವನ್ನು ನೋಡುತ್ತಾ ಅದನ್ನು ಸಹಿಸಲಾರದೆ, ಕಿಟಾರ್ ಎಂದು ಕಿರಿಚಿ ಕೊಂಡು ಏಳುತ್ತೇವೆ. ಎದ್ದಾಗ ಆ ಸ್ವಪ್ನ ಹೋಗಿ ನಾವು ನಮ್ಮ ಸುತ್ತಲೂ ಜಾಗ್ರತಾವಸ್ಥೆಯಲ್ಲಿರುವ ಮಾಮೂಲು ವಸ್ತುಗಳನ್ನು ನೋಡಿ ಸಮಾಧಾನವಾಗುವುದು. ಹಾಗೆಯೇ ಅರ್ಜುನ ಈ ಜಾಗ್ರತಾ ವಸ್ಥೆಯ ಸಹಜ ಸ್ಥಿತಿಗೆ ಇಳಿಯುತ್ತಾನೆ. ಇವನಿಗೆ ಸಮಾಧಾನವಾದುದನ್ನು ನೋಡಿ ಶ್ರೀಕೃಷ್ಣ ಹೀಗೆ ಹೇಳುತ್ತಾನೆ:

\begin{verse}
ಸುದುರ್ದರ್ಶಮಿದಂ ರೂಪಂ ದೃಷ್ಟವಾನಸಿ ಯನ್ಮಮ ।\\ದೇವಾ ಅಪ್ಯಸ್ಯ ರೂಪಸ್ಯ ನಿತ್ಯಂ ದರ್ಶನಕಾಂಕ್ಷಿಣಃ \versenum{॥ ೫೨ ॥}
\end{verse}

{\small ದುರ್ಲಭವಾದ ಯಾವ ನನ್ನ ರೂಪವನ್ನು ನೋಡಿದೆಯೊ, ಈ ವಿಶ್ವರೂಪವನ್ನು ಯಾವಾಗಲೂ ಕಾಣಲು ದೇವತೆಗಳು ಕೂಡ ಬಯಸುತ್ತಿರುವರು.}

ಈ ವಿಶ್ವರೂಪದರ್ಶನ ದುರ್ಲಭವಾದುದು, ಎಲ್ಲರಿಗೂ ಸಿಕ್ಕುವುದಿಲ್ಲ. ಭಗವಂತ ಮನಸ್ಸುಮಾಡಿ ಅರ್ಜುನನಿಗೆ ತೋರಿದ್ದರಿಂದ ಸಾಧ್ಯವಾಯಿತು. ಅನೇಕರು ಅದನ್ನು ನೋಡಲು ತವಕಪಡುತ್ತಿ ರುವರು. ಆದರೆ ಭಗವಂತ ಅದನ್ನು ತೋರಿಲ್ಲ. ಏಕೆಂದರೆ ನಮಗಿಂತ ಹೆಚ್ಚಾಗಿ, ಅವನಿಗೆ ಗೊತ್ತಿದೆ ನಾವು ತತ್ತರಿಸಿ ಹೋಗುತ್ತೇವೆ, ಅದನ್ನು ನೋಡಿದರೆ ಎಂಬುದು.

\begin{verse}
ನಾಹಂ ವೇದೈರ್ನ ತಪಸಾ ನ ದಾನೇನ ನ ಚೇಜ್ಯಯಾ ।\\ಶಕ್ಯ ಏವಂವಿಧೋ ದ್ರಷ್ಟುಂ ದೃಷ್ಟವಾನಸಿ ಮಾಂ ಯಥಾ \versenum{॥ ೫೩ ॥}
\end{verse}

{\small ನೀನು ಯಾವ ಪ್ರಕಾರ ನನ್ನನ್ನು ನೋಡಿರುವೆಯೊ, ಈ ರೂಪವುಳ್ಳ ನನ್ನನ್ನು ವೇದಾಧ್ಯಯನ, ತಪಸ್ಸು, ದಾನ, ಮತ್ತು ಯಜ್ಞದಿಂದ ನೋಡಲಾರೆ.}

ಪುನಃ ಭಗವಂತ ತನ್ನ ಸ್ವರೂಪವನ್ನು ಅರಿಯಬೇಕಾದರೆ ಬರೀ ಅಧ್ಯಯನ, ದಾನ, ತಪಸ್ಸು, ಯಜ್ಞಗಳೇ ಸಾಲದು. ಇದನ್ನೆಲ್ಲ ನಾವು ಪಡೆದಿದ್ದರೂ ಭಗವಂತನ ಕೃಪೆ ಇರಬೇಕು. ಇದ್ದರೆ ಮಾತ್ರ ಸಾಧ್ಯ ಎಂಬುದನ್ನು ಹೇಳುವನು.

\begin{verse}
ಭಕ್ತ್ಯಾ ತ್ವನನ್ಯಯಾ ಶಕ್ಯ ಅಹಮೇವಂವಿಧೊಽಜುRನ।\\ಜ್ಞಾತುಂ ದ್ರಷ್ಟುಂ ಚ ತತ್ತ್ವೇನ ಪ್ರವೇಷ್ಟುಂ ಚ ಪರಂತಪ \versenum{॥ ೫೪ ॥}
\end{verse}

{\small ಪರಂತಪನಾದ ಅರ್ಜುನ, ಆದರೆ ಅನನ್ಯವಾದ ಭಕ್ತಿಯಿಂದ ಈ ವಿಧವಾದ ನನ್ನನ್ನು ಯಥಾರ್ಥವಾಗಿ ತಿಳಿದುಕೊಳ್ಳಲು, ಸಾಕ್ಷಾತ್ಕರಿಸಲು ಮತ್ತು ಪ್ರವೇಶಿಸಲು ಸಾಧ್ಯ.}

ಅನನ್ಯವಾದ ಭಕ್ತಿಯೊಂದೇ ಭಗವಂತನನ್ನು ತಿಳಿದುಕೊಳ್ಳುವುದಕ್ಕೆ ಇರುವ ಏಕ ಮಾತ್ರ ಉಪಾಯ. ಆ ಭಕ್ತಿ ಒಂದೇ ಸಮನಾಗಿ ಭಗವಂತನ ಕಡೆ ಹರಿಯುತ್ತಿರಬೇಕು. ಅವನಿಂದ ಲೌಕಿಕವಾದ ಏನನ್ನೂ ಬೇಡದ ಭಕ್ತಿಯಾಗಿರಬೇಕು. ಆಗ ಮಾತ್ರ ಅವನನ್ನು ಯಥಾರ್ಥವಾಗಿ, ಅವನು ನಿಜವಾಗಿ ಹೇಗಿರುವನೊ ಹಾಗೆ ತಿಳಿದುಕೊಳ್ಳಲು ಸಾಧ್ಯ. ನಾವು ಅನೇಕ ವೇಳೆ, ತಿಳಿದುಕೊಳ್ಳಬೇಕೆಂಬ ವ್ಯಕ್ತಿಯ ಪ್ರಿಯವಾದ ಭಾಗವನ್ನು ಮಾತ್ರ ನೋಡಿ, ಅಪ್ರಿಯವಾದ ಭಾಗದ ಕಡೆ ಕಣ್ಣನ್ನು ಹಾಕುವುದಿಲ್ಲ. ಇದು ಸತ್ಯದ ಪೂರ್ಣದರ್ಶನವಲ್ಲ. ಪರಮಾತ್ಮನಲ್ಲಿ ಸೌಂದರ್ಯ ಸತ್ಯ ಹೇಗೆ ಇದೆಯೋ ಭಯಂಕರವೂ ಉಗ್ರವೂ, ಆದ ಮೃತ್ಯುವೂ ಕೂಡ ಇದೆ. ಇದನ್ನು ಭಗವಂತನ ಬಾಯಿಂದಲೇ ಅರ್ಜುನ ವಿಭೂತಿಯೋಗದಲ್ಲಿ ಕೇಳಿದ. ಕೇಳಿದ್ದರಿಂದಲೆ ಅರ್ಜುನನಿಗೆ ತೃಪ್ತಿ ಬರಲಿಲ್ಲ. ಕೇಳಿದ್ದನ್ನು ನೋಡಬೇಕೆಂದು ಬಯಸಿದ. ಅದಕ್ಕಾಗಿಯೇ ವಿಶ್ವರೂಪವನ್ನು ಅರ್ಜುನನಿಗೆ ಶ್ರೀಕೃಷ್ಣ ತೋರಿದ್ದು. ಇದೇ ಸಾಕ್ಷಾತ್ಕಾರ. ಬರೀ ಇದನ್ನು ನೋಡುವುದು ಕೂಡ ಕೊನೆ ಅಲ್ಲ. ನಾವು ನೋಡಿದ್ದರಲ್ಲಿ ಒಂದಾಗಬೇಕು, ಕರಗಬೇಕು. ಹನಿ ಸಾಗರದಲ್ಲಿ ಲಯವಾಗಬೇಕು. ನದಿ ಸಮುದ್ರದಲ್ಲಿ ಲಯವಾಗ ಬೇಕು. ನಾವು ಅವನನ್ನು ಪ್ರವೇಶ ಮಾಡಿದರೆ ನಾವು ನಾವಾಗಿ ಉಳಿಯುವುದಿಲ್ಲ. ಯಾವುದಕ್ಕೆ ಪ್ರವೇಶ ಮಾಡುತ್ತೇವೆಯೋ ಅದರ ಧರ್ಮ ನಮಗೆ ಬರುವುದು. ನದಿಗೆ ಸಾಗರದ ಧರ್ಮ ಬರುವುದು. ಅಗ್ನಿಕುಂಡಕ್ಕೆ ಇಳಿದ ಇದ್ದಲು ತಾನು ಕೂಡ ಅಗ್ನಿಯೇ ಆಗುವುದು.

ಕೇವಲ ಅನನ್ಯವಾದ ಭಕ್ತಿಯಿಂದಲೇ ನಾವು ಭಗವಂತನನ್ನು ತಿಳಿದುಕೊಳ್ಳಬಹುದು, ಅವನೇನು ಎಂಬುದನ್ನು ನೋಡಬಹುದು, ಅದರೊಳಗೆ ನಾವು ಕೂಡ ಪ್ರವೇಶ ಮಾಡಬಹುದು. ಇಲ್ಲಿ ಅತ್ಯಂತ ಸುಲಭದ ಹಾದಿಯನ್ನು ಜೀವರಾಶಿಗಳಿಗೆ ತೋರುತ್ತಾನೆ. ಇದು ಸುಲಭ ಮಾತ್ರವಲ್ಲ, ಯಾವುದನ್ನು ಯಜ್ಞ, ತಪಸ್ಸು ಅಧ್ಯಯನಗಳು ಮಾಡಲಾರವೊ ಅದನ್ನು ಮಾಡಬಲ್ಲದು.

\begin{verse}
ಮತ್ಕರ್ಮಕೃನ್ಮತ್ಪರಮೋ ಮದ್ಭಕ್ತಃ ಸಂಗವರ್ಜಿತಃ ।\\ನಿರ್ವೈರಃ ಸರ್ವಭೂತೇಷು ಯಃ ಸ ಮಾಮೇತಿ ಪಾಂಡವ \versenum{॥ ೫೫ ॥}
\end{verse}

{\small ಅರ್ಜುನ, ಯಾವನು ನನಗಾಗಿಯೇ ಕರ್ಮಮಾಡುವನೊ, ನನ್ನನ್ನೇ ಪರಮಗತಿಯೆಂದು ತಿಳಿದಿರುವನೋ, ನನ್ನ ಭಕ್ತನೋ, ಸಂಗರಹಿತನೋ, ಸಮಸ್ತ ಭೂತಗಳಲ್ಲಿ ವೈರವಿಲ್ಲದವನೋ, ಅವನು ನನ್ನನ್ನೇ ಸೇರುತ್ತಾನೆ.}

ಯಾರು ಭಗವಂತನನ್ನು ಪ್ರವೇಶಿಸಬಲ್ಲರೊ ಅವರು ಹೇಗಿರುವರು ಎಂಬುದನ್ನು ವಿವರಿಸುತ್ತಾನೆ. ಅವನನ್ನು ಪ್ರವೇಶಿಸಬೇಕಾದರೆ ನಾವು ಯೋಗ್ಯತೆ ಪಡೆದುಕೊಂಡಿರಬೇಕು. ಬರೀ ಆಸೆಯೊಂದೇ ಸಾಲದು. ಅನೇಕರಿಗೆ ಎವರೆಸ್ಟ್ ಪರ್ವತವನ್ನು ಏರಬೇಕೆಂಬ ಆಸೆ ಇರಬಹುದು. ಆದರೆ ಅವರಿಗೆ ಆರೋಗ್ಯ ಇರಬೇಕು, ಹೃದಯ ಬಲವಾಗಿರಬೇಕು, ಕಷ್ಟವನ್ನು ಸಹಿಸುವ ಶಕ್ತಿ ಇರಬೇಕು. ಇಲ್ಲದೇ ಬರೀ ಉತ್ಸಾಹವೊಂದರಿಂದಲೇ ಅವರು ಏರಲಾರರು. ಹಾಗೆಯೇ ಭಗವಂತನೆಡೆಗೆ ಪ್ರವೇಶ ಮಾಡಬೇಕಾದರೆ ಕೆಲವು ಗುಣಗಳು ಇರಬೇಕು. ಅವುಗಳಲ್ಲಿ ಮುಖ್ಯವಾದ ಕೆಲವನ್ನು ಇಲ್ಲಿ ಕೊಡುತ್ತಾನೆ.

ಕರ್ಮವನ್ನು ಭಗವಂತನಿಗಾಗಿಯೇ ಮಾಡಬೇಕು. ಯಾವ ಕರ್ಮ ತನ್ನ ಪಾಲಿಗೆ ಬಂದಿದೆಯೋ ಅದನ್ನು ಭಗವದರ್ಪಿತ ಭಾವನೆಯಿಂದ ಮಾಡಬೇಕು. ಅದರಿಂದ ಬರುವ ಲಾಭಕ್ಕಾಗಲಿ ಕೀರ್ತಿಗಾಗಲಿ ಫಲಕ್ಕಾಗಲಿ ಮಾಡಕೂಡದು. ಮಾಡುವುದೆಲ್ಲ ಭಗವಂತನಿಗೆ ಅರ್ಪಿತವಾಗಲಿ ಎಂಬ ಭಾವನೆ ಇರಬೇಕು. ಅವನು ಭಗವಂತನನ್ನೇ ಪರಮಗತಿ ಎಂದು ನಂಬಿರಬೇಕು. ನಾವು ಸೇರುವ ಗುರಿಯೇ ಅವನು. ಅವನನ್ನು ಮತ್ತಾವುದಕ್ಕೂ ಉಪಯೋಗಿಸಿಕೊಳ್ಳುವುದಲ್ಲ. ಅವನಿಗಾಗಿ ಎಲ್ಲ ಎಂಬ ಭಾವನೆ ಇರಬೇಕು.

ಅವನು ಭಗವಂತನ ಭಕ್ತನಾಗಿರಬೇಕು. ಅವನನ್ನು ಪ್ರೀತಿಸಬೇಕು. ಅವನಲ್ಲಿ ಶರಣಾಗಿರಬೇಕು. ಅವನನ್ನೇ ನಂಬಿರಬೇಕು. ಹತ್ತರಲ್ಲಿ ಹನ್ನೊಂದರಂತೆ ದೇವರನ್ನು ತೆಗೆದುಕೊಂಡರೆ ಸಾಲದು. ಎಲ್ಲವನ್ನೂ ಅವನಿಗಾಗಿ ಅರ್ಪಿಸಿದವನೇ ಭಕ್ತ.

ಸಂಗರಹಿತನಾಗಿರಬೇಕು. ಹೆಂಡತಿ, ಮನೆ, ಮಕ್ಕಳು, ಅಧಿಕಾರ ಪದವಿ, ಯಶಸ್ಸು ಯಾವುದಕ್ಕೂ ಅಂಟಿಕೊಂಡಿರಬಾರದು. ಇವುಗಳ ಮಧ್ಯದಲ್ಲಿ ಅವನಿರಬಹುದು. ಆದರೆ ಇದಾವುದೂ ತನ್ನದಲ್ಲ ಎಂಬುದನ್ನು ಪ್ರತಿನಿಮಿಷವೂ ಅವನು ತಿಳಿದಿರುವನು. ಕಮಲದ ಎಲೆ ನೀರಿನಲ್ಲಿದ್ದರೂ, ನೀರು ಅದಕ್ಕೆ ಸ್ವಲ್ಪವೂ ಅಂಟಿಕೊಂಡಿರುವುದಿಲ್ಲ.

ಅವನು ಯಾರನ್ನೂ ದ್ವೇಷಿಸುವುದಿಲ್ಲ. ಅವನು ಎಲ್ಲರನ್ನೂ ಪ್ರೀತಿಸುತ್ತಾನೆ. ಆ ಪ್ರೀತಿಗೆ ಕಾರಣ ವಾಸುದೇವ. ಈ ಪ್ರಪಂಚವೆಲ್ಲ ವಾಸುದೇವನ ಕುಟುಂಬ. ಆ ದೃಷ್ಟಿಯಿಂದ ಇವನು ಎಲ್ಲವನ್ನೂ ಪ್ರೀತಿಸುತ್ತಾನೆ. ಇವನು ಪ್ರೀತಿಸಿದರೂ ಕೆಲವರು ಇವನನ್ನು ದ್ವೇಷಿಸುವವರು ಇರುವರು. ಅವರು ದ್ವೇಷಿಸಿದರೂ ಇವನಲ್ಲಿ ಮಾತ್ರ ಸದಾ ಪ್ರೇಮವೇ ಜಿನುಗುತ್ತಿರುವುದು. ಭಗವಂತನನ್ನು ಪ್ರೀತಿಸು ವುದು, ಈ ಪ್ರಪಂಚವನ್ನು ದ್ವೇಷಿಸುವುದು ಒಟ್ಟಿಗೆ ಹೋಗುವುದಿಲ್ಲ. ಭಗವಂತನನ್ನು ಪ್ರೀತಿಸಿದರೆ ಅವನು ಸರ್ವವನ್ನು ಪ್ರೀತಿಸುವನು. ಏಕೆಂದರೆ ಎಲ್ಲದರ ಹಿಂದೆಯೂ ಅವನೇ ಇರುವನು. ಅವನಿಲ್ಲದ ಸ್ಥಳವಾಗಲೀ, ವ್ಯಕ್ತಿಯಾಗಲೀ ಯಾವುದೂ ಇಲ್ಲ.

ಈ ಗುಣಗಳಿದ್ದವನು ಭಗವಂತನನ್ನೇ ಸೇರುತ್ತಾನೆ, ಸತ್ತ ಮೇಲೆ ದೇವರನ್ನು ಅವನು ಹೊಸದಾಗಿ ಸೇರುವುದಿಲ್ಲ. ಬದುಕಿರುವಾಗಲೇ ಅವನು ತನ್ನ ವ್ಯವಹಾರದ ಹಿಂದೆಲ್ಲ ಭಗವಂತನನ್ನು ತಂದಿರು ವನು, ಇರುವಾಗ ಅವನಲ್ಲಿರುವನು, ಹೋಗುವಾಗಲೂ ಅವನಲ್ಲಿರುವನು. ನೀರಿನ ಗುಳ್ಳೆ ನೀರಿ ನಿಂದಲೇ ಆಗಿ ನೀರಿನ ಮೇಲೆಯೇ ಹರಿಯುತ್ತಿದ್ದು ಒಡೆದಾಗ ನೀರಿನಲ್ಲಿಯೇ ಐಕ್ಯವಾಗುವಂತೆ ಭಕ್ತನ ಜೀವನ.

