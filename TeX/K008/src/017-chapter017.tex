
\chapter{ಶ್ರದ್ಧಾತ್ರಯವಿಭಾಗಯೋಗ}

ಅರ್ಜುನ ಶ‍್ರೀಕೃಷ್ಣನನ್ನು ಹೀಗೆ ಕೇಳುತ್ತಾನೆ.

\begin{verse}
ಯೇ ಶಾಸ್ತ್ರವಿಧಿಮುತ್ಸೃಜ್ಯ ಯಜಂತೇ ಶ್ರದ್ಧಯಾನ್ವಿತಾಃ~।\\ತೇಷಾಂ ನಿಷ್ಠಾ ತು ಕಾ ಕೃಷ್ಣ ಸತ್ತ್ವಮಾಹೋ ರಜಸ್ತಮಃ \versenum{॥ ೧~॥}
\end{verse}

{\small ಶ‍್ರೀಕೃಷ್ಣ, ಯಾರು ಶಾಸ್ತ್ರವಿಧಿಯನ್ನು ಬಿಟ್ಟು ಶ್ರದ್ಧೆಯಿಂದ ಕೂಡಿ ಪೂಜಿಸುವರೊ ಅವರ ನಿಷ್ಠೆ ಎಂತಹುದು? ಸಾತ್ತ್ವಿಕವೋ ರಾಜಸಿಕವೋ ತಾಮಸಿಕವೋ?}

ಯಾವುದನ್ನು ಮಾಡಬೇಕು ಯಾವುದನ್ನು ಬಿಡಬೇಕು ಎಂಬುದನ್ನು ನಿರ್ಣಯಿಸುವುದಕ್ಕೆ ಶಾಸ್ತ್ರವೇ ಪ್ರಮಾಣ ಎಂದು ಹೇಳಿದೆ. ಕೆಲವು ವೇಳೆ ಶಾಸ್ತ್ರವನ್ನು ಒಬ್ಬ ಅನುಸರಿಸುತ್ತಿಲ್ಲ. ಆದರೂ ಆವನು ತಾನು ಮಾಡಬೇಕಾದ ಕೆಲಸವನ್ನು ಶ್ರದ್ಧೆಯಿಂದ ಮಾಡುತ್ತಾನೆ. ಇದನ್ನು ಸಾತ್ತ್ವಿಕ, ರಾಜಸಿಕ ಅಥವಾ ತಾಮಸಿಕ ಎಂದು ಹೇಗೆ ಹೇಳುವುದು?

ಇಲ್ಲಿ ಬರೀ ಶಾಸ್ತ್ರವನ್ನು ಬಿಡುವುದೇ ಮುಖ್ಯವಲ್ಲ. ಬಿಟ್ಟು ಅವನು ಏನು ಮಾಡುತ್ತಾನೆ ಎಂಬುದನ್ನು ಗಮನಕ್ಕೆ ತೆಗೆದುಕೊಳ್ಳಬೇಕಾಗಿದೆ. ಶಾಸ್ತ್ರವನ್ನು ಗಣನೆಗೆ ತಾರದೆ, ಕೇವಲ ತನ್ನ ಸ್ವಾರ್ಥ ಮತ್ತು ಸಮಾಜಕ್ಕೆ ಕಂಟಕವಾಗುವ ರೀತಿ ಕೆಲಸ ಮಾಡಿದರೆ ಅದು ಕೆಟ್ಟದ್ದು. ಆದರೆ ಕೆಲವರು ಶಾಸ್ತ್ರವನ್ನು ನಂಬುವುದಿಲ್ಲ. ಏತಕ್ಕೆ ಎಂದರೆ ಅದು ಯಾವುದೊ ಓಬೀರಾಯನ ಕಾಲದ ಭಾಷೆಯಲ್ಲಿ ಹೇಳುವುದು. ಅಲ್ಲಿ ಗೌಣವಾದ ವಿಷಯಗಳಿವೆ, ಮುಖ್ಯವಾದ ವಿಷಯಗಳಿವೆ. ಅದರಲ್ಲಿ ಯಾವುದು ಮುಖ್ಯ ಮತ್ತು ಯಾವುದು ಅಮುಖ್ಯ ಎಂಬುದನ್ನು ಕಂಡುಹಿಡಿಯವುದು ತುಂಬಾ ಕಷ್ಚ. ಆದಕಾರಣ ಶಾಸ್ತ್ರವನ್ನು ಬಿಟ್ಟು ತಮ್ಮ ಮನಸ್ಸಿಗೆ ಯಾವುದು ಸರಿ ಎಂದು ಕಾಣಿಸುವುದೊ ಅದನ್ನು ಮಾಡುತ್ತಾರೆ. ಅದು ಅನೇಕ ವೇಳೆ ಸರಿಯಾದ ಶಾಸ್ತ್ರಕ್ಕೆ ತಾಳೆ ಆಗುವುದು. ಅಂತಹ ಮನುಷ್ಯನನ್ನು ಏನೆಂದು ಹೇಳಬೇಕು? ಒಬ್ಬ ದೇವಸ್ಥಾನಕ್ಕೆ ಹೋಗುವುದಿಲ್ಲ, ಹಲವಾರು ಮಡಿ, ಮೈಲಿಗೆ ಮುಂತಾದುವು ಗಳನ್ನು ಆಚರಿಸುವುದಿಲ್ಲ. ಆದರೆ ಆತ ಭಗವಂತನನ್ನು ಚಿಂತಿಸುತ್ತಾನೆ. ಯಾರಿಗೂ ಕೆಡಕನ್ನು ಮಾಡುವುದಿಲ್ಲ. ಕೆಲವು ಶಾಸ್ತ್ರ ಹೇಳುವ ರೀತಿ ಅವನು ಇಲ್ಲದೆ ಇರಬಹುದು. ಆದರೆ ಅವನು ಮಾಡುವುದು ಶಾಸ್ತ್ರದ ಸಾರದಂತೆ ಇದೆ.

ಇನ್ನೊಂದು ಬಗೆಯ ಜನರು ಇರುವರು. ಅವರು ಅವಿದ್ಯಾವಂತರು. ಶಾಸ್ತ್ರವನ್ನು ಓದುವುದಕ್ಕೆ ಬಾರದು. ಅಷ್ಟೊಂದು ಚೆನ್ನಾಗಿ ವಿಚಾರವನ್ನು ಮಾಡುವ ಶಕ್ತಿಯಿಲ್ಲ. ತಮ್ಮ ಕರ್ತವ್ಯ ಏನು, ಅದನ್ನು ಹೇಗೆ ಮಾಡಬೇಕು ಎಂಬುದನ್ನು ಹಿರಿಯರಿಂದ ಕೇಳಿರುವರು. ಆ ಕೆಲಸವನ್ನು ಶ್ರದ್ಧಾಭಕ್ತಿಯಿಂದ ಮಾಡುವರು. ಇಂತಹ ಮನುಷ್ಯರು ಶಾಸ್ತ್ರಕ್ಕೆ ವಿರೋಧವಾಗಿ ಹೋದಂತೆ ಆಗುವುದೆ? ಅರ್ಜುನ ಇಲ್ಲಿ ಸಾಧುವಾದ ಪ್ರಶ್ನೆಯನ್ನು ಕೇಳುತ್ತಿರುವನು.

ಶ‍್ರೀಕೃಷ್ಣ ಉತ್ತರಿಸುತ್ತಾನೆ:

\begin{verse}
ತ್ರಿವಿಧಾ ಭವತಿ ಶ್ರದ್ಧಾ ದೇಹಿನಾಂ ಸಾ ಸ್ವಭಾವಜಾ~।\\ಸಾತ್ತ್ವಿಕೀ ರಾಜಸೀ ಚೈವ ತಾಮಸೀ ಚೇತಿ ತಾಂ ಶೃಣು \versenum{॥ ೨~॥}
\end{verse}

{\small ಸ್ವಭಾವದಿಂದ ಹುಟ್ಟಿದ ಜೀವಿಗಳ ಶ್ರದ್ಧೆ, ಸಾತ್ತ್ವಿಕ, ರಾಜಸ ಮತ್ತು ತಾಮಸ ಎಂದು ಮೂರು ಬಗೆಯಾಗಿದೆ. ಅದನ್ನು ಕೇಳು.}

ಮನುಷ್ಯ ಹುಟ್ಟವಾಗ ಖಾಲಿ ಬರುವುದಿಲ್ಲ. ಅವನು ಯಾವುದೋ ಶ್ರದ್ಧೆಯಿಂದ ಕೂಡಿರುವನು. ಅದು ಅವನ ಹಿಂದಿನ ಜನ್ಮ, ಕರ್ಮಗಳ ಪರಿಣಾಮ. ಬೀಜ ಹೇಗೆ ತಾನು ಬರುವಾಗಲೇ ಏನೊ ಆಗಬೇಕೆಂದು ಬರವುದೊ ಹಾಗೆಯೇ ಜೀವಿ ತಾನು ಏನು ಆಗಬೇಕಾಗಿದೆಯೊ ಆ ಸಂಸ್ಕಾರಗಳಿಂದ ಕೂಡಿ ಬರುವನು. ಇದು ಮೂರು ವಿಧವಾಗಿದೆ. ಬರೀ ಅವನು ಶಾಸ್ತ್ರವನ್ನು ನಂಬುತ್ತಾನೆಯೆ ಇಲ್ಲವೆ ಅದರ ಆಧಾರದ ಮೇಲೆಯೇ ಅದನ್ನು ನಿರ್ಧರಿಸುವುದಕ್ಕೆ ಆಗುವುದಿಲ್ಲ. ಕೆಲವರಿಗೆ ಶಾಸ್ತ್ರ ಕರತಲಾ ಮಲಕವಾಗಿ ಗೊತ್ತಿದೆ. ಆದರೆ ಕೆಲಸ ಮಾಡಬೇಕಾಗಿ ಬಂದಾಗ ಅದರಂತೆ ಮಾಡುವುದಿಲ್ಲ. ತಮ್ಮ ಸ್ವಾರ್ಥ ಮತ್ತು ಸುಖದ ದೃಷ್ಟಿಯಿಂದ ಮಾಡುವರು. ಯಾವಾಗ ಅವನು ಶಾಸ್ತ್ರವನ್ನು ಚೆನ್ನಾಗಿ ಬಲ್ಲನೊ, ತಾನು ಏನು ಮಾಡಿದರೂ ಕೂಡಾ ಒಳ್ಳೆಯದೇ ಎಂದು ಶಾಸ್ತ್ರದ ಆಧಾರದಿಂದಲೇ ಬೇಕಾದರೆ ಸಮರ್ಥಿಸುವನು. ಶಾಸ್ತ್ರವನ್ನು ಯಾರು ಬೇಕಾದರೂ ಉದಹರಿಸಬಹುದು. ಆದರೆ ಅದಕ್ಕೆ ಕೊಡುವ ವಿವರಣೆ ಬದಲಾಯಿಸುವುದು. ಇದು ಇಂಡಿಯನ್ ಪೀನಲ್ ಕೋಡಿನಂತೆ. ಪಕ್ಷ ಪ್ರತಿಪಕ್ಷದ ಲಾಯರುಗಳು ಇಬ್ಬರೂ ಒಂದೇ ಪುಸ್ತಕದಿಂದ ಉದಹರಿಸಿ, ತಮ್ಮ ಕಕ್ಷಿಯ ಪರವಾಗಿ ಮಾತನಾಡುವರು.

ಮತ್ತೊಬ್ಬ, ಶಾಸ್ತ್ರಾದಿಗಳು ಅನಾವಶ್ಯಕ, ನಾವೇ ನಮ್ಮ ಕರ್ತವ್ಯ ಏನು ಎಂಬುದನ್ನು ನಿರ್ಧರಿಸ ಬಹುದೆಂದು ಹೇಳುತ್ತಾನೆ. ಯಾವುದನ್ನು ನಮ್ಮ ಅಂತರಾತ್ಮ ಸರಿ ಎಂದು ತೋರುವುದೊ ಅದನ್ನೇ ಮಾಡುತ್ತೇವೆ, ಅದಕ್ಕೆ ಶಾಸ್ತ್ರದ ಆಧಾರ ಸಿಕ್ಕಿದರೆ ಸಿಕ್ಕಲಿ ಇಲ್ಲದೆ ಇದ್ದರೆ ಬಿಡಲಿ ಎಂದು ಹೇಳುವಂತಹ ಜನರು ಬೇರೆ ಇರುವರು. ಇವರನ್ನು ನಾವು ತಪ್ಪಿತಸ್ಥರು ಎಂದು ಹೇಳುವುದಕ್ಕೆ ಆಗುವುದಿಲ್ಲ. ಆದಕಾರಣವೇ ಒಬ್ಬ ಮನುಷ್ಯ ಶಾಸ್ತ್ರವನ್ನು ನಂಬುತ್ತಾನೆಯೇ ಬಿಡುತ್ತಾನೆಯೆ ಅದು ಅಷ್ಟು ಮುಖ್ಯವಲ್ಲ. ಅವನ ಜೀವನ ಮತ್ತು ಯಾವ ಉದ್ದೇಶಗಳಿಂದ ಕೆಲಸ ಮಾಡುತ್ತಾನೆ ಇವುಗಳನ್ನೆಲ್ಲಾ ತಿಳಿದಲ್ಲದೆ, ಅದನ್ನು ಸಾತ್ತ್ವಿಕವೊ, ರಾಜಸಿಕವೊ, ತಾಮಸಿಕವೊ ಎಂಬುದನ್ನು ಹೇಳುವುದಕ್ಕೆ ಆಗುವು ದಿಲ್ಲ.

\begin{verse}
ಸತ್ತ್ವಾನುರೂಪಾ ಸರ್ವಸ್ಯ ಶ್ರದ್ಧಾ ಭವತಿ ಭಾರತ~।\\ಶ್ರದ್ಧಾಮಯೋಽಯಂ ಪುರುಷೋ ಯೋ ಯಚ್ಛ್ರದ್ಧಃ ಏವ ಸಃ \versenum{॥ ೩~॥}
\end{verse}

{\small ಅರ್ಜುನ, ಮನುಷ್ಯರ ಶ್ರದ್ಧೆ ಅವರವರ ಸ್ವಭಾವಕ್ಕೆ ಅನುಗುಣವಾಗಿ ಇರುತ್ತದೆ. ಮನುಷ್ಯ ಶ್ರದ್ಧಾಮಯ. ಯಾರಲ್ಲಿ ಎಂತಹ ಶ್ರದ್ಧೆ ಇದೆಯೋ ಅವನು ಅಂಥವನಾಗುತ್ತಾನೆ.}

ಪ್ರತಿಯೊಬ್ಬರಲ್ಲಿಯೂ ಒಂದು ಶ್ರದ್ಧೆ ಇದೆ. ಈ ಶ್ರದ್ಧೆಯೇ ಒಂದು ಆಧಾರ, ನಂಬಿಕೆ, ಜೀವನದಲ್ಲಿ ಇರುವ ಒಂದು ಉದ್ದೇಶ. ಒಬ್ಬರಂತೆ ಮತ್ತೊಬ್ಬರಿಲ್ಲ. ಒಬ್ಬೊಬ್ಬ ಮನುಷ್ಯ ಒಂದೊಂದು ವಿಧ. ಒಬ್ಬೊಬ್ಬರಲ್ಲಿ ಒಂದೊಂದು ಒಳ್ಳೆಯ ಗುಣ ಒಂದೊಂದು ಕೆಟ್ಟ ಗುಣ ಇದೆ. ಅವನನ್ನು ನಾವು ಪೂರ್ಣವಾಗಿ ತಿಳಿದುಕೊಳ್ಳಬೇಕಾಗಿದೆ, ಯಾವುದನ್ನು ನಿರ್ಧರಿಸಬೇಕಾದರೂ.

ಮನುಷ್ಯ ಶ್ರದ್ಧಾಮಯ. ಪ್ರತಿಯೊಬ್ಬರಲ್ಲಿಯೂ ಯಾವುದಾದರೂ ಒಂದು ಉದ್ದೇಶವಿದೆ. ಅದಿಲ್ಲದೆ ಇದ್ದರೆ ಮನುಷ್ಯನೇ ಆಗುವುದಿಲ್ಲ. ದೇವರಲ್ಲಿ ನಂಬುವವನು ತಾನು ಹೇಗೆ ಬಾಳಬೇಕು, ಏನನ್ನು ಸಾಧಿಸಬೇಕು ಎಂಬುದನ್ನು ನಿರ್ಧರಿಸಿಕೊಂಡಿರುವನು. ದೇವರನ್ನು ನಂಬದವನಿಗೆ ಏನೂ ಉದ್ದೇಶವಿಲ್ಲ ಎಂದಲ್ಲ. ಅವನಿಗೂ ಜೀವನದಲ್ಲಿ ಒಂದು ಆದರ್ಶವಿದೆ, ಉದ್ದೇಶವಿದೆ. ಹಾಗೆಯೇ ಸಂದೇಹವಾದಿಗೆ, ಜಡವಾದಿಗೆ, ಚಾರ್ವಾಕನಿಗೆ, ಯಾರಲ್ಲಿಯೂ ಶ್ರದ್ದೆ ಇಲ್ಲದೇ ಇಲ್ಲ. ಕೆಲವರು ನಾವು ಯಾವುದರಲ್ಲಿಯೂ ನಂಬುವುದಿಲ್ಲ. ಆಯಾ ಕಾಲಕ್ಕೆ ಹೇಗೆ ತೋರುವುದೋ ಹಾಗೆ ಮಾಡಿ ಬಿಡುತ್ತೇವೆ ಎಂದು ಹೇಳಬಹುದು. ಇದೂ ಒಂದು ವಿಧವಾದ ಜೀವನ ದೃಷ್ಟಿಯೇ ಆಯಿತಲ್ಲ. ಒಬ್ಬ ಮುಂಚೆಯೇ ಏನು ಮಾಡಬೇಕು ಎಂಬುದನ್ನು ನಿಶ್ಚಯಿಸುತ್ತಾನೆ. ಅವನು ಕೇವಲ ಬಾಹ್ಯ ಘಟನೆಗಳ ಆಧಾರದ ಮೇಲೆ ತೇಲುತ್ತಿರುವನು. ಒಂದು ಹುಲ್ಲು ನೀರಿನ ಮೇಲೆ ತೇಲುತ್ತಿರುವಂತೆ. ನೀರು ಮೇಲೆ ಹೋದರೆ ಇದು ಮೇಲೆ ಹೋಗುವುದು, ಕೆಳಗೆ ಬಿದ್ದರೆ ಇದೂ ಕೆಳಗೆ ಬೀಳುವುದು. ಮತ್ತೊಬ್ಬ ಅದೇ ನೀರಿನಲ್ಲಿ ದೋಣಿಯಮೇಲೆ ತನಗೆ ಬೇಕಾದ ಕಡೆ ಹೋಗುವಂತೆ ಬೇಕಾದ ಕಡೆ ಹೋಗುವನು. ಒಬ್ಬನು ಆದರ್ಶವನ್ನು ಇಟ್ಟುಕೊಂಡಿರುವನು. ಹುಲ್ಲಿನಂತೆ ತೇಲುತ್ತಿರುವವನಿಗೆ ಏನೂ ಆದರ್ಶವಿಲ್ಲದೆ ಇಲ್ಲ. ಇದೇ ಅವನ ಆದರ್ಶವಾಗುವುದು.

ಮನುಷ್ಯ ತನ್ನಲ್ಲಿ ಎಂತಹ ಶ್ರದ್ಧೆ ಇದೆಯೋ ಹಾಗೆ ಆಗುತ್ತಾನೆ. ಜೀವನದಲ್ಲಿ ಅವನ ಉದ್ದೇಶ ವೇನು? ಆದರ್ಶವೇ ರೈಲ್ವೆ ಇಂಜಿನ್ನಿನಂತೆ ಒಂದು ಗುರಿಯ ಕಡೆಗೆ ಅವನನ್ನು ಎಳೆದುಕೊಂಡು ಹೋಗುವುದು. ನಮ್ಮ ನಂಬಿಕೆಗೆ ತಕ್ಕಂತೆ, ನಮ್ಮ ಆದರ್ಶ ಮತ್ತು ಅಭಿರುಚಿಗೆ ತಕ್ಕಂತೆ, ನಾವು ಯೋಚಿಸುತ್ತೇವೆ, ಅನುಭವಿಸುತ್ತೇವೆ, ಕೆಲಸ ಮಾಡುತ್ತೇವೆ. ಇವುಗಳೆಲ್ಲ ಸೇರಿ ನಮ್ಮ ಸ್ವಭಾವ ಆಗುವುದು. ಮನುಷ್ಯ ಎಲ್ಲಿದ್ದಾನೆ ಎಂದರೆ, ಅವನು ದೇಹದಲ್ಲಿ ಇಲ್ಲ, ತನ್ನ ಆದರ್ಶದಲ್ಲಿರುವನು. ಅವನ ಪ್ರಾಣ ಅಲ್ಲಿದೆ. ಒಬ್ಬ ದೇವರ ಕಡೆಗೆ ಹೋಗಬೇಕೆಂದು ಹಟ ಹಿಡಿದು ಜೀವನದಲ್ಲಿ ಬರುವ ಎಡರುತೊಡರುಗಳನ್ನೆಲ್ಲಾ ದಾಟಿ ಹೋಗುತ್ತಾನೆ. ಅದಕ್ಕಾಗಿ ಏನು ಕಷ್ಟ ಕಾರ್ಪಣ್ಯಗಳನ್ನಾಗಲೀ ಪಡಲು ಸಿದ್ಧನಿರುವನು. ಮತ್ತೊಬ್ಬ ಯಶಸ್ಸನ್ನು ಪಡೆಯಬೇಕೆಂದು ಹಗಲು ರಾತ್ರಿ ದುಡಿಯುವನು. ಇನ್ನೊಬ್ಬ ಹೆಂಡತಿಗೋ, ಮಗನಿಗೋ, ಸ್ನೇಹಿತನಿಗೋ, ಯಾರಿಗೋ, ಯಾವುದಕ್ಕೊ, ಆಸಕ್ತನಾಗಿ ದ್ದಾನೆ. ಅದಕ್ಕಾಗಿ ಶ್ರಮಿಸುವುದಕ್ಕೆ ತನ್ನ ಬಾಳನ್ನು ಅರ್ಪಿಸುವನು. ಆದರ್ಶ ಅಥವಾ ಶ್ರದ್ಧೆ ಎಂಬುದು ಬೀಜ. ಅದಕ್ಕೆ ತಕ್ಕಂತೆ ನಾವು ಬೆಳೆಯುತ್ತೇವೆ.

\begin{verse}
ಯಜಂತೇ ಸಾತ್ತ್ವಿಕಾ ದೇವಾನ್ ಯಕ್ಷರಕ್ಷಾಂಸಿ ರಾಜಸಾಃ~।\\ಪ್ರೇತಾನ್ ಭೂತಗಣಾಂಶ್ಚಾನ್ಯೇ ಯಜಂತೇ ತಾಮಸಾ ಜನಾಃ \versenum{॥ ೪~॥}
\end{verse}

{\small ಸಾತ್ತ್ವಿಕರು ದೇವತೆಗಳನ್ನು ಪೂಜಿಸುತ್ತಾರೆ. ರಾಜಸಿಕರು ಯಕ್ಷರಾಕ್ಷಸರನ್ನು ಪೂಜಿಸುತ್ತಾರೆ. ತಾಮಸಿಕರು ಪ್ರೇತ ಮತ್ತು ಭೂತಗಳನ್ನು ಪೂಜಿಸುತ್ತಾರೆ.}

ಸಾತ್ತ್ವಿಕರು ಹಲವು ಹೆಸರುಗಳನ್ನುಳ್ಳ ದೇವರುಗಳನ್ನು ಪೂಜಿಸುತ್ತಾರೆ. ಅವರದು ಆಧ್ಯಾತ್ಮಿಕ ಪ್ರವೃತ್ತಿ. ಭಗವಂತನ ಕಡೆಗೆ ಹೋಗಲು ಬಯಸುತ್ತಾರೆ. ಭಕ್ತಿ ಮುಕ್ತಿಗಳನ್ನು ಭಗವಂತನಿಂದ ಬೇಡುತ್ತಾರೆ. ಆಧ್ಯಾತ್ಮಿಕ ಜೀವನಕ್ಕೆ ಇರುವ ಆತಂಕಗಳನ್ನು ನಿವಾರಿಸೆಂದು ಬೇಡುತ್ತಾರೆ. ಅವರು ಯಾರಿಗೂ ತೊಂದರೆಯನ್ನು ಕೊಡುವುದಿಲ್ಲ. ಲೌಕಿಕವಾಗಿರುವುದನ್ನು ಬೇಡುವುದಿಲ್ಲ. ಕೇವಲ ಭಕ್ತಿ, ಜ್ಞಾನ, ವೈರಾಗ್ಯ ಕಾಮಿಗಳಾಗಿರುತ್ತಾರೆ.

ಎರಡನೆಯವರು ರಾಜಸಿಕ ಪ್ರವೃತ್ತಿಯವರು. ಅವರು ಯಕ್ಷರು, ರಾಕ್ಷಸರು ಮುಂತಾದವರನ್ನು ಆರಾಧಿಸುತ್ತಾರೆ. ಅವರಿಗೆ ಐಶ್ವರ್ಯ ಅಧಿಕಾರ ಕೀರ್ತಿ ಮುಂತಾದುವುಗಳು ಬೇಕು. ಇಂತಹ ದೇವತೆ ಗಳನ್ನು ಪೂಜಿಸಿದರೆ ಅವನ್ನು ಕೊಡುತ್ತಾರೆ ಎಂದು ಭಾವಿಸುವರು. ಇದರಂತೆಯೇ ಕೆಲವು ಸಿದ್ಧಿಗಳನ್ನು ಗಳಿಸಿದ ಮಹಾತ್ಮರ ಹತ್ತಿರವೂ ಹೋಗಿ ಲೌಕಿಕವಾದವುಗಳನ್ನು ಬೇಡುತ್ತಾರೆ. ಅವರಿಗೆ ದೇವರು ಬೇಕಾಗಿಲ್ಲ. ದೇವರ ಉಗ್ರಾಣದಲ್ಲಿರುವ ಯಾವುದಾದರೂ ವಸ್ತುಗಳು ಬೇಕು. ಅವರಿಗೆ ದೇವರು ಇವರ ಲೌಕಿಕ ಜೀವನಕ್ಕೆ ಸಹಾಯ ಮಾಡುವಂತೆ ಇರಬೇಕು. ಅವರ ಮನಸ್ಸಿನಲ್ಲಿ ಆಸೆ ಆಕಾಂಕ್ಷೆಗಳಿವೆ. ಅವುಗಳನ್ನು ತೃಪ್ತಿ ಪಡಿಸುವುದೇ ಅವರ ಜೀವನದ ಮುಖ್ಯ ಉದ್ದೇಶ. ಅದಕ್ಕಾಗಿ ಕೆಳಮಟ್ಟದ ಯಕ್ಷ ರಾಕ್ಷಸರನ್ನು ಪೂಜಿಸುತ್ತಾರೆ.

ಮೂರನೆಯವರು ತಾಮಸಿಕರು. ಅವರಿಗೆ ಜ್ಞಾನವೂ ಬೇಡ ಶಾಂತಿಯೂ ಬೇಡ. ಈ ಪ್ರಪಂಚ ದಲ್ಲಿ ತಮ್ಮ ಇಂದ್ರಿಯಗಳಿಗೆ ಪ್ರಿಯವಾದವುಗಳು ಬೇಕು. ಇವರು ಕೇವಲ ಮೃಗಸದೃಶರು. ಇವರಿಗೆ ದ್ವೇಷ, ಕೋಪ, ಮಾತ್ಸರ್ಯ ಮುಂತಾದುವುಗಳು ಹೇರಳವಾಗಿರುತ್ತವೆ. ಪ್ರೇತಭೂತಗಳು ಇವುಗಳ ಸಹಾಯದಿಂದ ತಮಗೆ ಬಂದಿರುವ ವಿಪತ್ತು ಹೋಗಬೇಕು, ತಮಗೆ ಆಗದವರಿಗೆ ಕಷ್ಟನಷ್ಟಗಳು ಇವುಗಳೆಲ್ಲಾ ಆಗಬೇಕೆಂದು ಕೇಳಿಕೊಳ್ಳುತ್ತಾರೆ. ಮಾಟ, ಮಂತ್ರ ತಂತ್ರಗಳು ಇವುಗಳನ್ನು ಕೂಡ ತಮಗೆ ಆಗದವರಿಗೆ ತೊಂದರೆಯನ್ನು ಕೊಡುವುದಕ್ಕೆ ಉಪಯೋಗಿಸುತ್ತಾರೆ.

\begin{verse}
ಅಶಾಸ್ತ್ರವಿಹಿತಂ ಘೋರಂ ತಪ್ಯಂತೇ ಯೇ ತಪೋ ಜನಾಃ~।\\ದಂಭಾಹಂಕಾರಸಂಯುಕ್ತಾಃ ಕಾಮರಾಗಬಲಾನ್ವಿತಾಃ \versenum{॥ ೫~॥}
\end{verse}

\begin{verse}
ಕರ್ಶಯಂತಃ ಶರೀರಸ್ಥಂ ಭೂತಗ್ರಾಮಮಚೇತಸಃ~।\\ಮಾಂ ಚೈವಾಂತಃಶರೀರಸ್ಥಂ ತಾನ್ ವಿದ್ಧ್ಯಾಸುರನಿಶ್ಚಯಾನ್ \versenum{॥ ೬~॥}
\end{verse}

{\small ಯಾರು ದುರಹಂಕಾರದಿಂದಲೂ ಕಾಮರಾಗ ಬಲಗಳಿಂದಲೂ ಕೂಡಿ, ಅವಿವೇಕಿಗಳಾಗಿ, ಶರೀರದಲ್ಲಿರುವ ಇಂದ್ರಿಯ ಸಮುದಾಯವನ್ನೂ ಮತ್ತು ನನ್ನನ್ನೂ ಕ್ಲೇಶಗೊಳಿಸುತ್ತಿರುವರೋ ಮತ್ತು ಶಾಸ್ತ್ರವಿಹಿತವಲ್ಲದ ಘೋರ ತಪಸ್ಸನ್ನು ಮಾಡುತ್ತಿರುವರೊ, ಅಂತಹವರನ್ನು ಆಸುರೀ ಬುದ್ಧಿಯುಳ್ಳವರು ಎಂದು ತಿಳಿ.}

ತಮೋಗುಣಿಗಳು ಆಸುರೀ ವೃತ್ತಿಯವರು. ಢಂಬದಿಂದ ಕೂಡಿದ್ದಾರೆ. ತಮ್ಮಲ್ಲಿರುವ ಅಲ್ಪ ಧಾರ್ಮಿಕ ಪ್ರವೃತ್ತಿಯನ್ನು ಬೇಕಾದಷ್ಟು ತೋರಿಸಿಕೊಳ್ಳುತ್ತಾರೆ. ಅದರ ವಿಷಯವಾಗಿ ತಾವೇ ಇತರರಿಗೆ ಹೇಳಿಕೊಳ್ಳುತ್ತಾರೆ. ಬೇಕಾದಷ್ಟು ಬಾಹ್ಯ ಧಾರ್ಮಿಕ ಪ್ರದರ್ಶನವನ್ನು ಅವರಲ್ಲಿ ನೋಡು ತ್ತೇವೆ. ಅಹಂಕಾರದಿಂದ ಕೂಡಿದ್ದಾರೆ. ತಮ್ಮಲ್ಲಿರುವ ಅಲ್ಪ ವಿದ್ಯೆ, ಧನ, ಅಧಿಕಾರ ಇವುಗಳ ಮದದಿಂದ ಕೂಡಿದ್ದಾರೆ. ತಮ್ಮ ಸಮಾನ ಯಾರೂ ಇಲ್ಲ ಎಂದು ಮೆರೆಯುತ್ತಿರುವರು.

ಕಾಮದ ಗುಳ್ಳೆಗಳು ಅವರ ಹೃದಯದಲ್ಲಿ ಸದಾ ಏಳುತ್ತಿರುವುವು. ಹಲವಾರು ಆಸೆ, ಆಕಾಂಕ್ಷೆಗಳು ಅವರಲ್ಲಿ ಮನೆ ಮಾಡಿಕೊಂಡಿವೆ, ಒಂದಾದ ಮೇಲೊಂದು ಏಳುತ್ತಿರುವುವು. ಅವರ ಆಸೆ, ಆಕಾಂಕ್ಷೆಗಳೆಲ್ಲ ಲೌಕಿಕವಾದುವುಗಳು. ಎಂದು ಆ ವಸ್ತುವನ್ನು ಪಡೆದೇನೊ ಅನುಭವಿಸಿಯೇನೊ ಎಂದು ತವಕ ಪಡುತ್ತಿರುವರು. ರಾಗ ಎಂದರೆ ಆಗಲೆ ಇರುವ ವಸ್ತುವಿನ ಮೇಲೆ ಆಸಕ್ತಿ. ಆ ವಸ್ತುವನ್ನು ಬಿಟ್ಟು ಇರಲಾರರು. ಎಲ್ಲಿ ಹೋಗಲಿ, ಏನು ಮಾಡುತ್ತಿರಲಿ ಅದರ ಮೇಲೆ ವ್ಯಾಮೋಹ ಅವರನ್ನು ಬಿಡುವಂತೆ ಇಲ್ಲ. ಸದಾ ಅದನ್ನೆ ಚಪ್ಪರಿಸುತ್ತಿರುವರು. ತಾವು ಪ್ರೀತಿಸುವ ವಸ್ತುವಿಗೆ ಅವರು ನಿಜವಾಗಿ ದಾಸರಾಗಿರುವರು. ಆದರೆ ಅಜ್ಞಾನದಿಂದ ನಾವು ಅದಕ್ಕೆ ಒಡೆಯರು ಎಂದು ಭಾವಿಸುವರು. ಯಾವಾಗ ಒಂದು ವಸ್ತುವಿನ ಮೇಲೆ ರಾಗ ನಮಗೆ ಹುಟ್ಟುವುದೊ ಆಗ ಒಂದು ದೊಡ್ಡ ಸರಪಳಿಯನ್ನು ಕಾಲಿಗೆ ಕಟ್ಟಿಕೊಂಡಂತೆ. ಹೋದ ಕಡೆ ಅದನ್ನು ಎಳೆದುಕೊಂಡು ಹೋಗುತ್ತಿರ ಬೇಕು. ಕೆಲವು ವೇಳೆ ದನದ ಕೊರಳಿಗೆ ದೊಡ್ಡದೊಂದು ಮರದ ತುಂಡನ್ನು ಕಟ್ಟುವರು. ಅದು ಓಡಿ ಹೋಗದೆ ಇರಲಿ ಎಂದು. ಹಾಗೆಯೇ ರಾಗವೇ ಒಂದು ನಮ್ಮ ಕೊರಳಿಗೆ ಕಟ್ಟಿಕೊಂಡ ಉರುಲು. ಅವರು ಬಲದಿಂದ ಕೂಡಿದ್ದಾರೆ. ತಮಗೆ ಬೇಕಾದಷ್ಟು ಶಾರೀರಕ ಬಲ ಇದೆ, ಐಶ್ವರ್ಯ, ಅಧಿಕಾರ, ದುಷ್ಟ ಸ್ನೇಹಿತರು ಇವರುಗಳ ಬಲ ಬೇರೆ ಇದೆ; ಇವುಗಳ ಸಹಾಯದಿಂದ ತಾವು ಏನನ್ನು ಬೇಕಾದರೂ ಪಡೆಯಬಹುದೆಂದು ಭಾವಿಸುವರು.

ಇವರು ಅವಿವೇಕಿಗಳು. ವಿಚಾರ ಮಾಡುವ ಶಕ್ತಿ ಇಲ್ಲ. ಈಗ ತಿಂದಿದ್ದೇ ಬಂತು, ಉಂಡಿದ್ದೇ ಬಂತು, ಅನುಭವಿಸಿದ್ದೇ ಬಂತು ಎಂದು ಭಾವಿಸುವರು. ನಾಳೆ ಏನಾಗುವುದು ಎಂಬುದನ್ನು ಕುರಿತು ವಿಚಾರ ಮಾಡುವುದಿಲ್ಲ. ತತ್ಕಾಲದಲ್ಲಿ ತಮ್ಮ ಇಂದ್ರಿಯಗಳಿಗೆ ಹೇಗೆ ತೋರುವುದೋ ಅದೇ ಸತ್ಯ. ಅದು ಯಾವಾಗಲೂ ಹಾಗೆಯೇ ಇರುವುದು ಎಂದು ಭಾವಿಸುವರು. ಈಗ ಅಮೃತದಂತೆ ಇರುವ ಇಂದ್ರಿಯಸುಖ ನಾಳೆ ವಿಷವಾಗುವುದು ಎಂಬುದು ಅವರಿಗೆ ಗೊತ್ತೇ ಆಗುವುದಿಲ್ಲ. ಯಾರಾದರೂ ಹೇಳಿದರೆ ಅದನ್ನು ನಂಬುವುದೂ ಇಲ್ಲ. ಅಂತಹವರನ್ನು ಹಾಸ್ಯ ಮಾಡುವರು, ಟೀಕಿಸುವರು. ತಮಗೆ ಅಂತಹ ಕಹಿ ಅನುಭವಗಳೆಲ್ಲ ಆಗುವುದಿಲ್ಲ ಎಂದು ಭಾವಿಸುವರು.

ಅವರು ಶರೀರದಲ್ಲಿ ಇಂದ್ರಿಯ ಸಮುದಾಯವನ್ನು ಕ್ಲೇಶಗೊಳಿಸುವರು. ಇಂದ್ರಿಯಗಳಿಗೆ ಒಂದು ಮಿತಿ ಇದೆ. ಆ ಮಿತಿಯನ್ನು ಮೀರಿ ಅದನ್ನು ಉಪಯೋಗಿಸಿದರೆ, ಪ್ರಕೃತಿ ನಿಯಮವನ್ನು ವಿರೋಧಿಸುತ್ತೇವೆ. ಆಗ ಪಡಬಾರದ ಯಾತನೆಯನ್ನೆಲ್ಲ ಅನುಭವಿಸಬೇಕಾಗುವುದು. ಸುಮ್ಮನೆ ನಾಲಿಗೆಗೆ ರುಚಿಯಾಗಿದೆ ಎಂದು ಒಂದು ಇತಿಮಿತಿಯಿಲ್ಲದೆ ತಿನ್ನುತ್ತ ಹೋದರೆ, ಅನಂತರ ಅಜೀರ್ಣ ವಾಂತಿ ಭೇದಿ ಮುಂತಾದುವುಗಳಿಂದ ನರಳಬೇಕಾಗುವುದು. ಯಾವಾಗ ಕಿವಿ ಇದೆ, ಕಣ್ಣಿದೆ, ಮೂಗಿದೆ, ಚರ್ಮವಿದೆ ಎಂದು ಅದನ್ನು ಹೆಚ್ಚು ಉಪಯೋಗಿಸುವೆವೊ ಅವು ಬಹಳ ಬೇಗ ಜೀರ್ಣವಾಗುವುವು. ಹಲವಾರು ರೋಗ ರುಜಿನಗಳು ಧಾಳಿ ಇಡುವುವು. ಜೀವನದಲ್ಲಿ ನಮ್ಮ ಆರೋಗ್ಯದ ನಾಶಕ್ಕೆ ಪ್ರಥಮ ಕಾರಣವೆ ಅತಿಭೋಗ. ನಾವು ಬಡ್ಡಿ ಸಮೇತ ಅನಂತರ ಅನುಭವಿಸಬೇಕಾಗುವುದು. ಇಂದ್ರಿಯಗಳನ್ನು ಕ್ಲೇಶಗೊಳಿಸುವುದು ಎಂದರೆ ಇದೇ.

ಹಿಂದೆ ಇರುವ ನನ್ನನ್ನೂ ಕ್ಲೇಶಗೊಳಿಸುತ್ತಾರೆ ಎನ್ನುವನು. ಭಗವಂತ ಸರ್ವವ್ಯಾಪಿಯಾಗಿ ಎಲ್ಲರ ಅಂತರಾಳದಲ್ಲಿಯೂ ಇರುವನು. ಅವನಿಗೆ ಕ್ಲೇಶಗೊಳಿಸುವುದು ಎಂದರೆ ಅವನನ್ನು ಕಾಣದಂತೆ ಮಾಡಿಕೊಳ್ಳುವುದು, ಅಜ್ಞಾನದ ತೆರೆಯಿಂದ ಅವನನ್ನು ಮುಚ್ಚುವುದು. ಅಂತರ್​ವಾಣಿಯಂತೆ ದೇವರು ಇದನ್ನು ಮಾಡಬೇಡ, ಅದನ್ನು ಮಾಡು ಎಂದು ಹೇಳುತ್ತಿರುವನು. ಆದರೆ ಅವನು ಇವುಗಳಾವುದನ್ನೂ ಕೇಳುವುದಿಲ್ಲ. ಸದ್​ವೃತ್ತಿಗಳನ್ನೆಲ್ಲ ಅವನು ಅದುಮಿ ಹಿಡಿಯುತ್ತಾನೆ. ಅವು ಬೆಳೆಯುವುದಕ್ಕೆ ಅವಕಾಶವನ್ನೇ ಕೊಡುವುದಿಲ್ಲ. ಇದೇ ಹಿಂದೆ ಇರುವ ದೇವರಿಗೆ ವ್ಯಥೆಯನ್ನು ಉಂಟುಮಾಡುವುದು.

ಇವರು ಘೋರವಾದ ತಪಸ್ಸನ್ನು ಮಾಡುತ್ತಾರೆ. ಅನೇಕ ಅಸುರರು ಈ ಗುಂಪಿಗೆ ಸೇರಿದವರು. ಆಹಾರವೇ ಇಲ್ಲದೆ ತಪಸ್ಸು ಮಾಡುತ್ತಾರೆ. ಕೆಲವರು ಕುಳಿತು ಕೊಳ್ಳುವುದೇ ಇಲ್ಲ. ಯಾವಾಗಲೂ ನಿಂತು ಕೊಂಡಿರುತ್ತಾರೆ. ಮತ್ತೆ ಕೆಲವರು ಯಾವಾಗಲೂ ಶಿರಶಾಸನದಲ್ಲೆ ಇರುತ್ತಾರೆ. ಚಳಿಗಾಲದಲ್ಲಿ ಕತ್ತಿನವರೆಗೆ ನೀರಿನಲ್ಲಿ ಮುಳುಗಿ ಜಪಮಾಡುತ್ತಾರೆ. ಬೇಸೆಗೆ ಕಾಲದಲ್ಲಿ ಸುತ್ತಲೂ ಬೆಂಕಿ ಹಾಕಿ ಕೊಂಡು ತಪಸ್ಸು ಮಾಡುತ್ತಾರೆ. ಮುಳ್ಳಿನ ಮಂಚದ ಮೇಲೆ ಇರುತ್ತಾರೆ. ಇನ್ನೂ ಹಲವು ವಿಧ ಚಿತ್ರವಿಚಿತ್ರ ರೀತಿ ದೇಹವನ್ನು ದಂಡಿಸಿ ತಪಸ್ಸು ಮಾಡುತ್ತಾರೆ. ಇಷ್ಟೊಂದು ದೇಹವನ್ನು ದಂಡಿಸಿ ತಪಸ್ಸನ್ನು ಮಾಡುತ್ತಿರುವಾಗ ಏನಾದರೂ ದೇವರು ಪ್ರತ್ಯಕ್ಷನಾದರೆ ಅವನನ್ನು ಇವರು ಕೇಳುವು ದೇನು? ನನಗೆ ಅದರಿಂದ ಸಾವು ಬೇಡ, ಇದರಿಂದ ಸಾವು ಬೇಡ, ನನ್ನನ್ನು ಮೀರುವವರಾರೂ ಈ ಜಗತ್ತಿನಲ್ಲಿ ಇರಕೂಡದು ಎಂದು ಕೇಳುತ್ತಾರೆ. ಅನಂತರ ಅವರು ಲೋಕ ಕಂಟಕರಾಗುತ್ತಾರೆ. ಇವರು ಉಗ್ರ ತಪಸ್ಸನ್ನು ಮಾಡಿ ಏನು ಸಂಪಾದಿಸಿದರು? ಲೋಕ ಹಾನಿಯನ್ನು ಮತ್ತು ಆತ್ಮ ಹಾನಿಯನ್ನು. ಇವುಗಳೆಲ್ಲ ಶಾಸ್ತ್ರಕ್ಕೆ ವಿರೋಧವಾದ ತಪಸ್ಸು. ಯಾವಾಗಲೂ ಯೋಗ್ಯವಾದ ತಪಸ್ಸಿ ನಲ್ಲಿ ಆತ್ಮಕಲ್ಯಾಣ ಮತ್ತು ಲೋಕಕಲ್ಯಾಣ ಎರಡೂ ಬೆರೆತಿರುವುದು. ಈ ಆಸುರೀ ತಪಸ್ಸಿನಲ್ಲಿ ಅವೆರಡೂ ಇಲ್ಲ. ಇದು ತಪಸ್ಸನ್ನು ದುರಪಯೋಗಪಡಿಸಿಕೊಂಡಂತೆ. ಯಾವ ಬೆಂಕಿಯಿಂದ ಅನ್ನವನ್ನು ಬೇಯಿಸಿ ಊಟ ಮಾಡಬಹುದೊ ಅದರಿಂದ ಮನೆಗೆ ಬೆಂಕಿ ಇಟ್ಟಂತೆ.

\begin{verse}
ಆಹಾರಸ್ತ್ವಪಿ ಸರ್ವಸ್ಯ ತ್ರಿವಿಧೋ ಭವತಿ ಪ್ರಿಯಃ~।\\ಯಜ್ಞಸ್ತಪಸ್ತಥಾ ದಾನಂ ತೇಷಾಂ ಭೇದಮಿಮಂ ಶೃಣು \versenum{॥ ೭~॥}
\end{verse}

{\small ಆಹಾರವು ಮೂರು ವಿಧವಾಗಿ ಪ್ರಿಯವಾಗಿರುತ್ತದೆ. ಇದರಂತೆಯೇ ಯಜ್ಞ, ತಪಸ್ಸು, ದಾನ ಇವುಗಳ ಭೇದವನ್ನು ಕೇಳು.}

ನಮ್ಮ ಗುಣಕ್ಕೆ ತಕ್ಕಂತೆ ನಾವು ತಿನ್ನುವ ಆಹಾರ ವ್ಯತ್ಯಾಸವಾಗುವುದು. ಇಲ್ಲಿ ಆಹಾರ ಎಂದರೆ ನಾವು ಬಾಯಿಯ ಮೂಲಕ ತಿನ್ನವುದು ಎಂದು ಸ್ಥೂಲವಾದ ಅರ್ಥ. ಆದರೆ ನಮ್ಮ ಪಂಚೇಂದ್ರಿಯ ಗಳ ಮೂಲಕ ನಾವು ಒಳಗೆ ಸೆಳೆದುಕೊಳ್ಳುವ ವೇದನೆಗಳೆಲ್ಲವೂ ಆಹಾರಕ್ಕೆ ಸಂಬಂಧಪಟ್ಟಿವೆ. ನಾವು ಬಾಯಿಯ ಮೂಲಕ ತಿನ್ನುವುದು ನಮ್ಮ ದೇಹವನ್ನು ಪೋಷಿಸುವುದು. ಆದರೆ ಇತರ ಇಂದ್ರಿಯಗಳ ಮೂಲಕ ನಾವು ಸೆಳೆದುಕೊಳ್ಳುವ ಅನುಭವಗಳೆಲ್ಲ ನಮ್ಮ ವ್ಯಕ್ತಿತ್ವವನ್ನು ರೂಪಿಸುವುವು. ನಾವು ನೋಡುವುದು, ಕೇಳುವುದು, ಮೂಸುವುದು, ಮುಟ್ಟವುದು ಇವುಗಳೆಲ್ಲ ಒಬ್ಬನ ವ್ಯಕ್ತಿತ್ವವನ್ನು ರೂಪಿಸುವುವು.

ಇದರಂತೆಯೇ ನಾವು ಮಾಡುವ ಯಜ್ಞ, ತಪಸ್ಸು, ದಾನ, ಹೊರಗಿನಿಂದ ನೋಡಿದರೆ, ಯಾವ ವ್ಯತ್ಯಾಸವೂ ಕಾಣುವುದಿಲ್ಲ. ಆದರೆ ಅದನ್ನು ಮಾಡುವ ದೃಷ್ಟಿಯಲ್ಲಿ ವ್ಯತ್ಯಾಸವಿದೆ. ದೃಷ್ಟಿ ವ್ಯತ್ಯಾಸ ಎಲ್ಲಿ ಆಗುವುದೊ ಅಲ್ಲಿ ಒಂದೇ ಕೆಲಸವನ್ನು ಮಾಡಬಹುದು, ಆದರೆ ಅದರಿಂದ ಬರುವ ಪರಿಣಾಮ ಸಂಪೂರ್ಣವಾಗಿ ಬೇರೆ ಆಗುವುದು. ಒಬ್ಬನದು ತಾಮಸಿಕವೆ, ರಾಜಸಿಕವೇ, ಸಾತ್ತ್ವಿಕವೇ ಎಂಬುದನ್ನು ನಿಷ್ಕರ್ಷಿಸಬೇಕಾದರೆ, ಹೊರಗಿನ ಅವನ ಕೆಲಸವನ್ನು ಮಾತ್ರ ತೆಗೆದುಕೊಂಡರೆ ಸಾಲದು. ಒಳಗಿನ ಅವನ ದೃಷ್ಟಿಯನ್ನೂ ತೆಗೆದುಕೊಳ್ಳಬೇಕಾಗುವುದು.

\begin{verse}
ಆಯುಸ್ಸತ್ತ್ವಬಲಾರೋಗ್ಯಸುಖಪ್ರೀತಿವಿವರ್ಧನಾಃ~।\\ರಸ್ಯಾಃ ಸ್ನಿಗ್ಧಾಃ ಸ್ಥಿರಾ ಹೃದ್ಯಾ ಆಹಾರಾಃ ಸಾತ್ತ್ವಿಕಪ್ರಿಯಾಃ \versenum{॥ ೮~॥}
\end{verse}

{\small ಆಯುಸ್ಸು ಸತ್ತ್ವ ಬಲ ಆರೋಗ್ಯ ಸುಖ ಪ್ರೀತಿ ಇವುಗಳನ್ನು ವೃದ್ಧಿಗೊಳಿಸುವ, ರಸಯುಕ್ತವಾದ, ಪರಿಶುದ್ಧ ವಾದ, ಪುಷ್ಟಿಕರವಾದ ಮನಸ್ಸಿಗೆ ಪ್ರಿಯವಾದ ಆಹಾರಗಳು ಸಾತ್ತ್ವಿಕನಿಗೆ ಪ್ರಿಯವಾಗಿರುವುವು.}

ಶ‍್ರೀಕೃಷ್ಣ ಇಲ್ಲಿ ಸಸ್ಯಾಹಾರ ಮಾಂಸಾಹಾರ ಮುಂತಾದುವುಗಳನ್ನು ಹೇಳುವುದಕ್ಕೆ ಹೋಗುವು ದಿಲ್ಲ. ಒಂದು ಆಹಾರದ ಪರಿಣಾಮದಿಂದ ಅದನ್ನು ಪ್ರತಿಯೊಬ್ಬನೂ ಕಂಡುಹಿಡಿಯ ಬೇಕಾಗಿದೆ. ಒಂದೆ ಒಂದು ಆಹಾರ ಸಾತ್ತ್ವಿಕವಾಗಿಯೂ ಇರಬಹುದು, ತಾಮಸಿಕವಾಗಿಯೂ ಇರಬಹುದು, ರಾಜಸಿಕವಾಗಿಯೂ ಇರಬಹುದು. ನಾವು ಅದನ್ನು ಹೇಗೆ ಸೇವನೆ ಮಾಡುತ್ತೇವೆ ಅದರ ಮೇಲೆ ನಿಂತಿದೆ. ಹಾಲು ಬಹಳ ಸುಲಭವಾಗಿ ಜೀರ್ಣವಾಗುವ ವಸ್ತು. ಶರೀರದ ಪೋಷಣೆಗೆ ಬೇಕಾದ ವಸ್ತುಗಳೆಲ್ಲ ಅದರಲ್ಲಿವೆ. ಅದನ್ನು ಮಿತವಾಗಿ ಸೇವಿಸಿದರೆ ಸಾತ್ತ್ವಿಕ ಪ್ರಯೋಜನವನ್ನು ಪಡೆಯ ಬಹುದು. ಅದೇ ಹಾಲನ್ನು ಚೆನ್ನಾಗಿ ಇಂಗಿಸಿ, ಬಾಸುಂದಿಮಾಡಿ, ಅದನ್ನು ಹೊಟ್ಟೆತುಂಬ ತಿಂದರೆ ಜೀರ್ಣಸಿಕೊಳ್ಳುವುದಕ್ಕೆ ಕಷ್ಟವಾಗುವುದು. ವಾಂತಿಯೋ, ಬೇಧಿಯೋ ಬರುವುದು. ಅದು ರಜೋ ಗುಣದ ಪರಿಣಾಮವನ್ನು ನಮ್ಮ ಮೇಲೆ ಬಿಡುವುದು. ಅದರಂತೆಯೇ ಅದೇ ಹಾಲನ್ನು ಹೆಪ್ಪು ಹಾಕಿ ಎರಡು ಮೂರು ದಿನ ಇಟ್ಟ ಮೊಸರನ್ನು ಮಜ್ಜಿಗೆ ಮಾಡಿ ಕುಡಿದರೆ, ಒಳ್ಳೆಯ ಕಳ್ಳಿನಂತೆ ಆಗುವುದು. ನಮಗೆ ನಿದ್ರೆ, ಅಮಲು ಮುಂತಾದುವನ್ನು ತರುವುದು. ಇದರಿಂದ ತಮೋಗುಣದ ಪರಿಣಾಮ ಉಂಟಾಗುವುದು. ಆದಕಾರಣವೆ ನಾವು ಸತ್ತ್ವ ರಜಸ್ ತಮಸ್ಸು ಈ ಆಹಾರಗಳನ್ನು ಬರೀ ಆಯಾ ಆಹಾರದ ಹೆಸರಿನಿಂದ ನಿರ್ಧರಿಸುವುದಕ್ಕೆ ಆಗುವುದಿಲ್ಲ. ಅವನ್ನು ಹೇಗೆ ಮಾಡಿ ತಿನ್ನುತ್ತೇವೆ, ಅವು ನಮ್ಮ ಮೇಲೆ ಎಂತಹ ಪರಿಣಾಮವನ್ನು ಉಂಟುಮಾಡುವುವು ಆ ದೃಷ್ಟಿಯಿಂದ ನಿರ್ಧರಿಸಬೇಕಾಗಿದೆ.

ಇಲ್ಲಿ ಸಾತ್ತ್ವಿಕ ಆಹಾರ ಯಾವುದು ಎಂಬುದನ್ನು ಹೇಳುತ್ತಾನೆ. ಅದು ನಮಗೆ ದೀರ್ಘ ಆಯುಸ್ಸನ್ನು ಕೊಡುವಂತೆ ಇರಬೇಕು. ಅದರಿಂದ ರೋಗ ರುಜಿನಗಳು ಬರಕೂಡದು. ಇದು ನಮ್ಮ ಜೀರ್ಣಕೋಶಗಳಿಗೆ ಅರಗಿಸಿಕೊಳ್ಳುವುದಕ್ಕೆ ಕಷ್ಟಕರವಾಗಿರಬಾರದು. ಸುಲಭವಾಗಿರಬೇಕು. ಆಗಲೆ ದೇಹದ ಶಕ್ತಿ ಉಳಿತಾಯವಾಗುವುದು.

ಅದು ನಮ್ಮಲ್ಲಿ ಸತ್ತ್ವಗುಣಗಳನ್ನು ಬೆಳಸಬೇಕು. ನಮ್ಮಲ್ಲಿ ಸತ್ತ್ವಗುಣ ವೃದ್ಧಿಯಾಗಬೇಕಾದರೆ ನಾವು ತೆಗೆದುಕೊಳ್ಳುವ ಆಹಾರ ಶುದ್ಧವಾಗಿರಬೇಕು. ಇಲ್ಲಿ ಶುದ್ಧ ಎಂದರೆ ಕೇವಲ ನೈರ್ಮಲ್ಯದ ದೃಷ್ಟಿಯಿಂದಲ್ಲ. ನೈತಿಕ ದೃಷ್ಟಿಯಿಂದ ಶುದ್ಧವಾಗಿರಬೇಕು. ನಾನು ನ್ಯಾಯವಾಗಿ ಸಂಪಾದಿಸಿದ್ದು ಆಗಿರಬೇಕು, ಸೇವಿಸುವ ಆಹಾರ. ಅದನ್ನು ನಿರ್ಮಲವಾಗಿ ಅಡಿಗೆ ಮಾಡಿರಬೇಕು. ಭಗವಂತನಿಗೆ, ಅತಿಥಿಗಳಿಗೆ ಅದನ್ನು ಕೊಟ್ಟಿರಬೇಕು. ಅನಂತರ ಉಳಿದ ಆಹಾರವನ್ನು ನಾನು ಸೇವಿಸಬೇಕು. ಆಗ ಅದು ಶುದ್ಧವಾದ ಆಹಾರವಾಗಿ ನನ್ನ ಮನಸ್ಸಿನಲ್ಲಿ ಪವಿತ್ರವಾದ ಭಾವನೆಗಳು ಏಳುವುದಕ್ಕೆ ಸಹಾಯವಾಗುವುದು. ಅನ್ಯಾಯದಿಂದ ಸಂಪಾದನೆ ಮಾಡಿದ್ದನ್ನು ಎಷ್ಟೇ ಮಡಿಯಿಂದ ಮಾಡಿದರೂ, ಎಷ್ಟೇ ನೈರ್ಮಲ್ಯವಾಗಿದ್ದರೂ ಅದು ಶುದ್ಧವಾದ ಆಹಾರವಾಗಲಾರದು.

ನಾವು ತೆಗೆದುಕೊಳ್ಳುವ ಆಹಾರದಿಂದ ನಮಗೆ ಬಲ ಬರಬೇಕು, ಶಕ್ತಿ ಬರಬೇಕು. ಇಲ್ಲಿ ಒಬ್ಬ ಪೈಲ್ವಾನನಂತೆ ಆಗಿರಬೇಕಾಗಿಲ್ಲ. ಬಲ ಎಂದರೆ ದಿನಬೆಳಗಾದರೆ ಹೊರಗಿನ ಜಾಡ್ಯಗಳಿಗೆ ತುತ್ತಾಗದೆ ಇರುವಷ್ಟು ಬಲವಂತನಾಗಿರುತ್ತಾನೆ. ಶರೀರ ಬಲವಾಗಿದ್ದರೆ ಮಾತ್ರ ನಾವು ಯಾವ ಸಾಧನೆಯ ನ್ನಾದರೂ ಮಾಡಲು ಸಾಧ್ಯವಾಗುವುದು. ಮುಂಚೆ ಬಲವಾದ ಶರೀರ ಇರಬೇಕು. ಅದರ ಹಿಂದೆ ಬಲವಾದ ಮನಸ್ಸಿರಬೇಕು.

ಆತನ ಆರೋಗ್ಯವನ್ನು ಹೆಚ್ಚಿಸುವಂತಹ ಆಹಾರವಾಗಿರಬೇಕು. ಎಲ್ಲಿ ಅಂದರೆ ಅಲ್ಲಿ, ಯಾರು ಏನು ಕೊಟ್ಟರೆ ಅದನ್ನು ತಿನ್ನುತ್ತಿದ್ದರೆ ಆರೋಗ್ಯವನ್ನು ಕಾಪಾಡಿಕೊಳ್ಳುವುದಕ್ಕೆ ಆಗುವುದಿಲ್ಲ. ನಮ್ಮ ಆರೋಗ್ಯ ನಿಂತಿರುವುದು ನಾವು ತೆಗೆದುಕೊಳ್ಳುವ ಶುದ್ಧವಾದ ಆಹಾರದ ಮೇಲೆ. ಆಹಾರಕ್ಕೆ ಯಾವ ಕ್ರಿಮಿಕೀಟಗಳೂ ತಾಕಿರಬಾರದು. ಕೊಳೆ ಕಸ ಬಿದ್ದಿರಬಾರದು. ಕೊಳೆಯ ಕೈಗಳಿಂದ ಅದನ್ನು ಮುಟ್ಟಿರಬಾರದು. ಈ ಮುಂಜಾಗ್ರತೆಯನ್ನೆಲ್ಲ ತೆಗೆದುಕೊಂಡರೆ ಆಹಾರ ಶುದ್ಧವಾಗಿರುವುದು. ಅಂತಹ ಆಹಾರವನ್ನು ಮಿತವಾಗಿ ಸೇವಿಸಿದರೆ ಮಾತ್ರ ಆರೋಗ್ಯವಾಗಿರಬಹುದು. ಆರೋಗ್ಯ ಅತ್ಯಂತ ಮುಖ್ಯ. ದಿನ ಬೆಳಗಾದರೆ ಖಾಯಿಲೆ ಬೀಳುತ್ತಿದ್ದರೆ, ಯಾವ ಧ್ಯಾನ ಅಧ್ಯಯನವನ್ನೂ ಮಾಡುವುದಕ್ಕೆ ಆಗುವುದಿಲ್ಲ.

ನಾವು ತಿನ್ನುವ ಆಹಾರದಿಂದ ನಮಗೆ ಸುಖ, ಪ್ರೀತಿ ಆಗಬೇಕು ಎನ್ನುತ್ತಾನೆ. ಯಾವುದಾದರೂ ಒಂದು ಬಗೆಯ ಆಹಾರಕ್ಕೆ ಈ ಗುಣಗಳಿವೆಯೆ ಎಂದರೆ, ಬರೀ ಆಹಾರಕ್ಕೇ ಈ ಗುಣವಿಲ್ಲ. ಈ ಆಹಾರವನ್ನು ಸಂಪಾದನೆ ಮಾಡುವ ರೀತಿಗೂ ಇದು ಅನ್ವಯಿಸುವುದು. ನಾನು ಮತ್ತೊಬ್ಬನಿಗೆ ಸುಖ ಕೊಟ್ಟಿದ್ದರೆ, ಶಾಂತಿ ಕೊಟ್ಟಿದ್ದರೆ, ಆನಂದ ಕೊಟ್ಟಿದ್ದರೆ, ಅದರಿಂದ ಅನ್ನವನ್ನು ಸಂಪಾದನೆ ಮಾಡಿದ್ದರೆ ಆಗ ನನಗೂ ಆ ಗುಣಗಳು ಬರುವುವು. ಬರೀ ಹಾಲು, ಹಣ್ಣು, ತುಪ್ಪ ತಿಂದರೆ ಇವು ಬರುವುದಿಲ್ಲ. ಇವನ್ನು ಮತ್ತೊಬ್ಬನಿಗೆ ಒಳ್ಳೆಯದನ್ನು ಮಾಡಿ ಸಂಪಾದಿಸಿದ್ದು ಆಗಿರಬೇಕು. ಹಾಗೆಯೇ ಪ್ರೀತಿ ಕೂಡ. ನಾವು ಜೀವನವನ್ನು ಇತರರಿಗೆ ಪ್ರೀತಿಯನ್ನು ಹಂಚುವುದರಲ್ಲಿ ಕಳೆದಿದ್ದರೆ ನಮಗೆ ಪ್ರೀತಿ ಬರುವುದು. ಇದೊಂದು ಜೀವನದ ಗಾಢವಾದ ನಿಯಮ. ನಾವು ಕೊಟ್ಟಿದ್ದೇ ನಮಗೆ ಕಟ್ಟಿಟ್ಟದ್ದು. ನಾವು ಮತ್ತೊಬ್ಬರ ಅಳಲನ್ನು ಪರಿಹರಿಸಿದ್ದರೆ, ನಮ್ಮ ಅಳಲನ್ನು ಮತ್ತೊಬ್ಬನು ಪರಿಹರಿಸುವನು. ನಾನು ಮತ್ತೊಬ್ಬನಿಗೆ ಸುಖ ಕೊಟ್ಟಿದ್ದರೆ, ಆನಂದ ಕೊಟ್ಟಿದ್ದರೆ, ಶಾಂತಿ ಕೊಟ್ಟಿದ್ದರೆ, ಅದು ನನಗೆ ಬೇಡವೆಂದರೂ ಬರುವುದು. ಹಾಗೆಯೇ ಪ್ರೀತಿ. ನಾವು ಏನನ್ನು ಹೊರಗೆ ಕಳುಹಿಸುವೆವೊ ಅದೇ ಮತ್ತೂ ಬಲವಾಗಿ ನಮ್ಮನ್ನು ಉದ್ಧರಿಸುವುದಕ್ಕೆ ಬರುವುದು. ಅವು ರಸಯುಕ್ತ ವಾಗಿರಬೇಕು, ರುಚಿಕರವಾಗಿರಬೇಕು. ಆಹಾರ ನಾಲಗೆ ಮೂಲಕ ಹೊಟ್ಟೆಗೆ ಹೋಗಬೇಕಾಗಿದೆ. ಬಾಯಿಗೆ ರುಚಿಯಾಗಿದ್ದರೇ, ಲಾವಾಗ್ರಂಥಿಗಳು ಮತ್ತು ಇತರ ಗ್ರಂಥಿಗಳೆಲ್ಲ ಚೆನ್ನಾಗಿ ಕೆಲಸವನ್ನು ಮಾಡಿ ಜೀರ್ಣವಾಗುವುದಕ್ಕೆ ಸಹಾಯ ಮಾಡುವುವು. ಅದು ಪರಿಶುದ್ಧವಾಗಿರಬೇಕು. ಚೆನ್ನಾಗಿ ಮಾಡಿದ್ದು ಆಗಿರಬೇಕು. ಯಾವ ಕೊಳೆ ಕಷ್ಮಲವೂ ಬೆರೆಯದೆ ಇರಬೇಕು. ಅದು ದೇಹದಲ್ಲಿ ಸ್ವಲ್ಪಕಾಲ ಇರುವಂತಹುದು ಆಗಬೇಕು. ತಿಂದ ಸ್ವಲ್ಪಹೊತ್ತಿನಲ್ಲಿಯೇ ಹಸಿವು ಆಗುವ ಹಾಗಿದ್ದರೆ ಯಾವಾಗಲೂ ಅನ್ನಚಿಂತನೆ ಮಾಡುತ್ತಿರಬೇಕಾಗುವುದು.

ಅದು ಮನಸ್ಸಿಗೆ ಪ್ರಿಯವಾದ ಆಹಾರವಾಗಿರಬೇಕು, ಎಂದರೆ ಮನಸ್ಸಿಗೆ ಒಗ್ಗಿರಬೇಕು. ಇದು ಅತ್ಯಂತ ಮುಖ್ಯ. ಯಾವಾಗ ನಮ್ಮ ಮನಸ್ಸು ಒಂದು ಆಹಾರವನ್ನು ಸ್ವೀಕರಿಸಲು ಒಡಂಬಡುವು ದಿಲ್ಲವೋ ಅದರಲ್ಲಿ ಎಲ್ಲಾ ವಿಧವಾದ ಇತರ ಒಳ್ಳೆಯ ಗುಣಗಳೆಲ್ಲ ಇರಬಹುದು, ಆದರೂ ದೇಹ ಅದನ್ನು ತಿರಸ್ಕರಿಸುವುದು. ಆದಕಾರಣವೇ ನಮ್ಮ ಮನಸ್ಸಿಗೆ ಒಗ್ಗುವ ಆಹಾರವನ್ನೇ ಸೇವಿಸಬೇಕು.

\begin{verse}
ಕಟ್ವಮ್ಲಲವಣಾತ್ಯುಷ್ಣತೀಕ್ಷ್ಣರೂಕ್ಷವಿದಾಹಿನಃ~।\\ಆಹಾರಾ ರಾಜಸಸ್ಯೇಷ್ಟಾ ದುಃಖಶೋಕಾಮಯಪ್ರದಾಃ \versenum{॥ ೯~॥}
\end{verse}

{\small ಅತಿ ಕಹಿಯೂ, ಹುಳಿಯೂ, ಉಪ್ಪೂ, ಬಿಸಿಯೂ, ಖಾರವೂ, ಒಣಕಲೂ, ದಾಹವನ್ನುಂಟು ಮಾಡುವ ಮತ್ತು ದುಃಖ ಶೋಕ ರೋಗಗಳನ್ನುಂಟು ಮಾಡುವ ಆಹಾರಗಳು ರಾಜಸಿಕನಿಗೆ ಪ್ರಿಯವಾಗಿರುವುವು.}

ರಾಜಸಿಕನ ರುಚಿ ಇದು. ಅವನಿಗೆ ಅತಿ ಕಹಿಯನ್ನು ಕಂಡರೆ ತುಂಬಾ ಇಷ್ಟ. ಬೇವಿನ ಸೊಪ್ಪು, ಹಾಗಲಕಾಯಿ ಅವುಗಳನ್ನೆಲ್ಲ ತಿನ್ನುವನು. ತಿನ್ನುವಾಗಲೂ ಅದರ ಕಹಿಯನ್ನೇನೂ ಕಡಿಮೆ ಮಾಡಿ ತಿನ್ನುವುದಿಲ್ಲ. ಸಾಧಾರಣ ಮನುಷ್ಯನಿಗೆ ಸಿಹಿ ಮತ್ತು ಹುಳಿ ಹೇಗೆ ಬಾಯಲ್ಲಿ ನೀರು ಊರುವಂತೆ ಮಾಡುವುದೊ ಹಾಗೆಯೇ ಕಹಿ ಅವನ ಬಾಯಲ್ಲಿ ನೀರು ತರುವುದು.

ಅದರಂತೆಯೇ ಅವರಿಗೆ ತುಂಬಾ ಹುಳಿ ಬೇಕು. ಮಜ್ಜಿಗೆಯನ್ನು ಎರಡು ಮೂರು ದಿನಗಳಿಟ್ಟು ಅದು ಚೆನ್ನಾಗಿ ಹುಳಿ ಬಂದ ಮೇಲೆ ಅದರಿಂದ ಮಜ್ಜಿಗೆ ಹುಳಿಯನ್ನು ಮಾಡುವರು. ಆ ಮಜ್ಜಿಗೆ ಹುಳಿಯನ್ನು ಕೆಲವು ದಿನ ಇಟ್ಟುಕೊಂಡು ಉಪ್ಪಿನ ಕಾಯಿಯಂತೆ ಅಥವಾ ಗೊಜ್ಜಿನಂತೆ ತಿನ್ನುತ್ತಾರೆ. ಬರೀ ಹುಣಸೇ ಹಣ್ಣಿನಿಂದ ಗೊಜ್ಜನ್ನು ಮಾಡಿ ತಿನ್ನುತ್ತಾರೆ. ಅದರಂತೆಯೇ ಕೆಲವರಿಗೆ ಉಪ್ಪು ಜಾಸ್ತಿ ಬೇಕು. ಎಲೆಯಲ್ಲಿ ಪದಾರ್ಥಕ್ಕೆ ಹಾಕುವ ಉಪ್ಪಿನ ಜೊತೆಗೆ ಬೇರೆ ಉಪ್ಪನ್ನೇ ಬಡಿಸುವರು. ಜಾಸ್ತಿ ಬೇಕಾದರೆ ತೆಗೆದುಕೊಳ್ಳಲಿ ಎಂದು. ತೆಗೆದುಕೊಳ್ಳುವಾಗ ಕಾಫಿ, ಟೀ ಮುಂತಾದವು ತುಂಬಾ ಬಿಸಿಯಾಗಿರಬೇಕು. ಅದಿರುವ ಪಾತ್ರೆಯ ಬಿಸಿಯನ್ನು ಅವರ ಕೈ ತಡೆಯುವುದಿಲ್ಲ. ಒಂದು ಕೈಯಲ್ಲಿ ಬಟ್ಟೆ ಇಟ್ಟುಕೊಂಡು ಅದನ್ನು ಹಿಡಿದುಕೊಂಡು ಬಿಸಿಬಿಸಿಯಾಗಿರುವುದನ್ನು ಕುಡಿಯುತ್ತಾರೆ. ಕೆಲವ ರಿಗೆ ಟೀ ಅಥವಾ ಕಾಫಿ ಒಲೆಯ ಮೇಲೆಯೆ ಇರಬೇಕು. ಅಲ್ಲಿದ್ದರೆ ಅವರು ಕುಡಿಯುವರು. ಕೆಳಗೆ ಇಟ್ಟು ಕುಡಿದರೆ ಅದು ಆರಿ ಹೋಗುವುದು ಎಂದು. ಕೆಲವರಿಗೆ ಖಾರದ ಮೇಲೆ ಅಷ್ಟು ಪ್ರೀತಿ. ಹಸಿಮೆಣಸಿನಕಾಯಿ ಪಲ್ಯ ಮಾಡುತ್ತಾರೆ. ಅದನ್ನು ತಿನ್ನುತ್ತಿರುವಾಗ ಕಣ್ಣಿನಲ್ಲಿ ನೀರು ಸುರಿಯು ತ್ತಿದ್ದರೂ ಅದನ್ನು ಒರೆಸಿಕೊಂಡು ಆ ಖಾರವನ್ನು ತಿನ್ನುವರು. ಬರೀ ಹಸಿಮೆಣಸಿನ ಕಾಯಿಯನ್ನು ನೆಂಚಿಕೊಂಡು ಊಟ ಮಾಡುವುದು ಕೆಲವರಲ್ಲಿ ಬಹಳ ಸಾಮಾನ್ಯ.

ಹೋಗುವಾಗ ದಾರಿಯಲ್ಲಿ ತಿನ್ನುವುದಕ್ಕೆಂದು ಒಣಕಲು ರೊಟ್ಟಿ ಮುಂತಾದುವನ್ನು ತೆಗೆದುಕೊಂಡು ಹೋಗುವರು. ಅದನ್ನು ತಿಂದಾದಮೇಲೆ ನೀರನ್ನು ಪದೇ ಪದೇ ಕುಡಿಯುತ್ತಿರಬೇಕು. ಹಾಗೆ ಕುಡಿಯದೇ ಇದ್ದರೆ ಅರಗುವುದಿಲ್ಲ. ಹೊಟ್ಟೆಗೆ ಹೋದ ತಕ್ಷಣವೇ ಅದು ಎಲ್ಲಾ ರಸಗಳನ್ನೂ ಹೀರಿಕೊಳ್ಳುವುದು. ಆಗ ಯಮ ದಾಹವಾಗುವುದು. ಲೋಟಗಟ್ಟಲೆ ನೀರನ್ನು ಕುಡಿದಲ್ಲದೆ ಆ ದಾಹ ಶಮನವಾಗದು.

ಇದು ಮನಸ್ಸಿನ ಮೇಲೆ ಶೋಕ ಮತ್ತು ದುಃಖವನ್ನು ಉಂಟುಮಾಡುವುದು. ಆಹಾರ ಅರಗಿಸಿಕೊಳ್ಳುವುದಕ್ಕೆ ಕಷ್ಟ. ಜೊತೆಗೆ ಅವನು ಅನುಮಾನಾಸ್ಪದ ರೀತಿಯಲ್ಲಿ ಸಂಪಾದಿಸಿರುವನು. ಸುಳ್ಳು ಅನ್ಯಾಯ ಮುಂತಾದುವುಗಳನ್ನೆಲ್ಲ ಮಾಡಿ ಸಂಪಾದಿಸಿದ ಆಹಾರ. ಅಂತಹ ಆಹಾರವನ್ನು ತೆಗೆದುಕೊಂಡರೆ ಮನಸ್ಸಿಗೆ ನೆಮ್ಮದಿ ಹೇಗೆ ಇರಬಹುದು? ನಾವು ಮಾಡಿದ ಅನ್ಯಾಯ, ಪಾಪಗಳೆಲ್ಲ ನಮ್ಮ ಮನಸ್ಸಿನ ಮೇಲೆ ತಮ್ಮ ಪರಿಣಾಮವನ್ನು ಬೀರುತ್ತಿರುವುವು.

ಪದೇ ಪದೇ ರೋಗಕ್ಕೂ ತುತ್ತಾಗುತ್ತಾ ಬರುವನು. ಅವನು ತೆಗೆದುಕೊಳ್ಳುವ ಆಹಾರದಿಂದ ಜಠರಕೋಶ ಕೆಡುವುದು. ಹೊಟ್ಟೆಯ ಹುಣ್ಣು ಮುಂತಾದುವು ಆಗುವುದು ಅದರಿಂದಲೇ. ಅನಂತರ ದೇಹದ ಯಕೃತ್ \enginline{(liver)} ಕೆಡುವುದು. ಇದಕ್ಕಾಗಿ ಜೀವಾವಧಿ ಗುಳಿಗೆಗಳನ್ನು ನುಂಗುತ್ತಿರಬೇಕು. ಹೊಟ್ಟೆ ಜಾಡ್ಯಗಳು ಮತ್ತು ಇಪ್ಪತ್ತೆಂಟು ಶೋಕಗಳು ಮನಸ್ಸಿನಲ್ಲಿರುವುದರಿಂದ ಮಧುಮೇಹ \enginline{(diabetes)} ಮತ್ತು ಹೃದಯ ಜಾಡ್ಯಗಳಿಗೆ ತುತ್ತಾಗುತ್ತಾನೆ.

\begin{verse}
ಯಾತಯಾಮಂ ಗತರಸಂ ಪೂತಿ ಪರ್ಯುಷಿತಂ ಚ ಯತ್~।\\ಉಚ್ಛಿಷ್ಟಮಪಿ ಚಾಮೇಧ್ಯಂ ಭೋಜನಂ ತಾಮಸಪ್ರಿಯಮ್ \versenum{॥ ೧೦~॥}
\end{verse}

{\small ಆರಿರುವುದು, ರಸಹೀನವಾಗಿರುವುದು, ದುರ್ಗಂಧವಾಗಿರುವುದು, ಹಳಸಿ ಹೋಗಿರುವುದು, ಎಂಜಲು ಮತ್ತು ಅಶುದ್ಧವೂ ಆದ ಆಹಾರ ತಾಮಸಿಕನಿಗೆ ಪ್ರಿಯ.}

ತಾಮಸಿಕರಿಗೆ ಪ್ರಿಯವಾದ ಆಹಾರಗಳನ್ನು ಹೇಳುವನು. ಅದು ಆರಿಹೋಗಿದೆ. ಯಾವಾಗಲೋ ಮಾಡಿದ್ದು. ಆಹಾರ ಆರಿಹೋದರೆ ಅದರ ರುಚಿ ಒಂದು ಪಾಲು ಕೆಡುವುದು ಮತ್ತು ಕೊಳೆಯಲು ಪ್ರಾರಂಭವಾಗುವುದು. ಅದನ್ನು ಬಹಳ ಕಾಲ ಬಿಟ್ಟಿರುವುದರಿಂದ ಅದರ ಮೇಲೆ ನೊಣ ಮುಂತಾ ದುವುಗಳೆಲ್ಲ ಕುಳಿತಿರಬಹುದು. ಇಂತಹ ಆಹಾರ ಸೇವನೆಯಿಂದಲೇ ಕಾಲರ ಬೇಧಿ ಮುಂತಾದ ರೋಗಗಳು ಹರಡುವುವು. ಅವರು ತಿನ್ನುವ ಹಣ್ಣು, ಉಪಯೋಗಿಸುವ ತರಕಾರಿ ಎಲ್ಲ ಒಣಗಿ ಹೋಗಿರುವುವು. ನಾವು ಅದಕ್ಕೆ ನೀರನ್ನು ಬೆರೆಸಿ ಹಸಿ ಮಾಡಬೇಕು. ತರಕಾರಿ ಹಣ್ಣು ಮುಂತಾದುವು ಹೊಸದಾಗಿರುವಾಗ ಅದರಲ್ಲಿರುವ ಕೆಲವು ಒಳ್ಳೆಯ ವಿಟಮಿನ್​ಗಳು ಹಳೆಯದಾಗುತ್ತ ಹೊರಟು ಹೋಗುತ್ತವೆ. ಬರಿ ನಾರು ಗುಂಜೇ ಅದರಲ್ಲಿರುವುದು, ಸಾರವೆಲ್ಲ ಹೊರಟು ಹೋಗಿದೆ. ಅದರ ದುರ್ವಾಸನೆಯನ್ನು ತಡೆಯುವುದಕ್ಕೆ ಆಗುವುದಿಲ್ಲ. ಅದರ ಅಭ್ಯಾಸವಿಲ್ಲದವನಿಗೆ ಆ ವಾಸನೆಯೇ ಸಾಕು ವಾಂತಿಯನ್ನು ತರುವುದಕ್ಕೆ. ಆದರೆ ಈ ತಾಮಸಿಯ ಮೂಗು ಅದಕ್ಕೆ ಒಗ್ಗಿದೆ. ಅಡಿಗೆ ಆದಮೇಲೆ ಅದಕ್ಕೆ ಏನಾದರೂ ದುರ್ವಾಸನೆ ಬರುವುದನ್ನು ಒಗ್ಗರಣೆಯಂತೆ ಬೆರಸುತ್ತಾನೆ. ಜಾಸ್ತಿ ಇಂಗು ಹಾಕುವನು, ಬೆಳ್ಳುಳ್ಳಿ, ಈರುಳ್ಳಿಯನ್ನು ಚೆನ್ನಾಗಿ ಬೆರಸುವನು. ಅವನು ಬಿಡುವ ಉಸಿರಿನಲ್ಲೆಲ್ಲ ಈ ವಾಸನೆ ಇರುವುದು. ಅವನ ತೇಗಿನಲ್ಲಿ, ಅವನ ಬೆವರಿನಲ್ಲಿ, ಅವನ ಮಲಮೂತ್ರಗಳಲ್ಲಿಯೂ ಇದು ಇರುವುದು. ಕೆಲವು ಕಡೆ ಅಡಿಗೆ ಆದಮೇಲೆ ವಾಸನೆಗೊಂದು ಮೀನಿನ ಪುಡಿಯನ್ನು ಬೆರೆಸುವರು. ಅದೋ ದುರ್ಗಂಧಮಯ. ಆದರೆ ಅದನ್ನು ತಿನ್ನುವವರಿಗೆ ಅದರಷ್ಟು ಸುವಾಸನೆ ಬೇರೆ ಇರುವುದಿಲ್ಲ. ಆ ವಾಸನೆಗಾಗಿಯೇ ಅವರು ಅದನ್ನು ತಿನ್ನುವರು. ಸಾಸಿವೆ ಎಣ್ಣೆ ಮೂಗಿನ ಹತ್ತಿರ ಬಂದರೆ ಕೆಲವರಿಗೆ ಆಗದು. ಆದರೆ ಅದನ್ನು ಉಪಯೋಗಿಸುವವರು ಅನ್ನ ಪಲ್ಯ ಮುಂತಾದುವಕ್ಕೆ ಹಸಿಯ ಸಾಸಿವೆ ಎಣ್ಣೆಯನ್ನು ಸ್ವಲ್ಪ ಬೆರೆಸಿ ತಿನ್ನುತ್ತಾರೆ.

ತಂಗಳು ಎಂಜಲು ಅವನಿಗೆ ಇಷ್ಟ. ಒಂದು ದಿನದ ಹಿಂದೆ ಮಾಡಿದ್ದು, ಅದಾಗಲೇ ವಾಸನೆ ಬರುತ್ತಿದೆ. ಅವನಿಗೆ ಆ ವಾಸನೆಯೇ ಇಷ್ಟ. ಅದನ್ನು ಆ ವಾಸನೆಗಾಗಿಯೇ ತಿನ್ನುವನು. ಯಾರೊ ತಿಂದು ಮಿಕ್ಕಿದ್ದು ಎಂಜಲು ಅವನಿಗೆ ಸಿಕ್ಕಿದರೆ ಅದನ್ನು ತಿನ್ನುವನು. ಆಹಾರ ಕೊಳೆ ಕಶ್ಮಲದಿಂದ ಕೂಡಿದೆ. ಕೆಲವು ವೇಳೆ ಮಾರ್ಕೆಟ್ಟಿನಲ್ಲಿ ಕೊಳೆತ ಹಣ್ಣುಗಳನ್ನು ಆಚೆಗೆ ಎಸೆಯುವರು. ಅದನ್ನು ತಿನ್ನುವುದಕ್ಕೂ ಜನರು ಇರುವರು. ಅವರೇನು ಯಾವ ಕ್ರಿಮಿಗಳಿಗೂ ಅಂಜುವುದಿಲ್ಲ. ಒಬ್ಬ ಹುಡುಗ ಒಂದು ಹುಳುಕು ಸೇಬಿನ ಹಣ್ಣನ್ನು ತಿನ್ನುತ್ತಿದ್ದ. ತಾಯಿ ಹೇಳಿದಳು, “ಮಗು, ಅದರಲ್ಲಿ ಹುಳು ಇದೆ, ಜೋಪಾನವಾಗಿರು” ಎಂದು. ಅದಕ್ಕೆ ಆ ಹುಡುಗ, “ನಾನೇಕೆ ಜೋಪಾನವಾಗಿರಬೇಕು? ಆ ಹುಳು ಜೋಪಾನವಾಗಿರಲಿ” ಎಂದ.

ಸಾಧಾರಣವಾಗಿ ಇಂತಹ ಆಹಾರವನ್ನು ತೆಗೆದುಕೊಳ್ಳುವವರು ಬಹುಪಾಲು ಆರ್ಥಿಕ ಮಟ್ಟದಲ್ಲಿ ಕೆಳಗೆ ಇರುವವರು. ಯಾವಾಗ ಇಂತಹ ಆಹಾರವನ್ನು ತೆಗೆದುಕೊಳ್ಳುತ್ತಾನೆಯೋ ನಿದ್ರೆ ಆಲಸ್ಯ ಇವುಗಳಿಗೆ ತುತ್ತಾಗುತ್ತಾನೆ. ಅನೇಕ ಜಠರ ವ್ಯಾಧಿಗಳಿಗೆ ತುತ್ತಾಗುತ್ತಾನೆ. ಮರವು, ಅಸಡ್ಡೆ ಇವನಿಗೆ ಮಾಮೂಲು. ದುಃಖ, ಕಷ್ಟ, ರೋಗ ರುಜಿನಗಳಿಗೆ ಇವನು ತುತ್ತಾಗುತ್ತಾನೆ.

\begin{verse}
ಅಫಲಾಕಾಂಕ್ಷಿಭಿರ್ಯಜ್ಞೋ ವಿಧಿದೃಷ್ಟೋ ಯ ಇಜ್ಯತೇ~।\\ಯಷ್ಟವ್ಯಮೇವೇತಿ ಮನಃ ಸಮಾಧಾಯ ಸ ಸಾತ್ತ್ವಿಕಃ \versenum{॥ ೧೧~॥}
\end{verse}

{\small ಫಲಾಕಾಂಕ್ಷೆ ಇಲ್ಲದವನು, ಯಜ್ಞವನ್ನು ಮಾಡಬೇಕೆಂದು ಮನಸ್ಸನ್ನು ಸ್ಥಿರಗೊಳಿಸಿ ಶಾಸ್ತ್ರಸಮ್ಮತವಾದ ಯಜ್ಞ ವನ್ನು ಮಾಡುವುದು ಸಾತ್ತ್ವಿಕವಾದದ್ದು.}

ಆಹಾರವಾದ ಮೇಲೆ ಯಜ್ಞದಲ್ಲಿ ಮೂರು ಬಗೆಯನ್ನು ಹೇಳುತ್ತಾನೆ. ಸಾತ್ತ್ವಿಕ ದೃಷ್ಟಿಯಿಂದ ಯಜ್ಞ ಮಾಡುವವನು ಹೀಗೆ ಇರುತ್ತಾನೆ. ಅವನಲ್ಲಿ ಫಲಾಪೇಕ್ಷೆ ಇಲ್ಲ. ಅವನಿಗೆ ಯಜ್ಞವನ್ನು ಮಾಡಿ ತಾನೇ ಸಂಪಾದಿಸಬೇಕಾದ ಯಾವುದೂ ಇಲ್ಲ. ಅವನಿಗೆ ಇದರಿಂದ ಏನೂ ಬೇಕಾಗಿಲ್ಲ. ಅದು ತನ್ನ ಕರ್ತವ್ಯವೆಂದು ಭಾವಿಸುತ್ತಾನೆ. ಪ್ರಪಂಚಕ್ಕೆ ಬಂದ ಮೇಲೆ ಮನುಷ್ಯ ಪಂಚಯಜ್ಞಗಳನ್ನು ಮಾಡ ಬೇಕು. ಯಜ್ಞದ ಆಧಾರದ ಮೇಲೆ ಈ ಸಮಾಜ ನಿಂತಿರುವುದು. ನಮ್ಮ ಪಾಲಿನ ಕರ್ತವ್ಯವನ್ನು ನಿರ್ವಂಚನೆಯಿಂದ ಯಜ್ಞರೂಪದಲ್ಲಿ ಭಗವಂತನಿಗೆ ಅರ್ಪಿಸಬೇಕು. ಮನುಷ್ಯರಾಗಿ ಹುಟ್ಟಿದ್ದಕ್ಕೆ ನಾವು ತೀರಿಸಬೇಕಾದ ಪುಣ ಇದು. ಎಲ್ಲಾ ಕಡೆಯಿಂದಲೂ ನಾವು ತೆಗೆದುಕೊಂಡಿದ್ದೇವೆ. ಇದನ್ನು ನಾವು ಕೊಟ್ಟೇ ಆ ಸಾಲದಿಂದ ಪಾರಾಗುವುದು. ಅದಕ್ಕಾಗಿ ಯಜ್ಞವನ್ನು ಮಾಡುತ್ತಾನೆ..ಯಜ್ಞವನ್ನು ಅವನು ಮನಸ್ಸನ್ನು ಸ್ಥಿರಗೊಳಿಸಿ ಮಾಡುತ್ತಾನೆ. ಮಾಡಲೆ ಬಿಡಲೆ ಎಂದು ಚಿತ್ತ ಚಂಚಲವಾಗಿರುವು ದಿಲ್ಲ. ಮಾಡುವುದಕ್ಕೆ ಮುಂಚೆ ಎಲ್ಲವನ್ನೂ ಪರ್ಯಾಲೋಚಿಸಿ ಅನಂತರ ಒಂದು ನಿರ್ಧಾರಕ್ಕೆ ಬಂದ ಮೇಲೆ ಮಾಡುವನು. ಒಂದು ಒಳ್ಳೆಯ ಕೆಲಸವನ್ನು ಮಾಡಿಬಿಟ್ಟು, ಏತಕ್ಕಾದರೂ ಅದನ್ನು ಮಾಡಿದೆನೊ, ಅದಕ್ಕೆ ಅಷ್ಟೊಂದು ದುಡ್ಡು ಖರ್ಚಾಯಿತು, ಮಾಡದೆ ಇದ್ದರೆ ಚೆನ್ನಾಗಿತ್ತಲ್ಲ ಎಂದು ಪಶ್ಚಾತ್ತಾಪ ಪಡುವವನಲ್ಲ ಅವನು.

ಅವನು ಮಾಡುವುದೆಲ್ಲ ಶಾಸ್ತ್ರಸಮ್ಮತವಾದ ಯಜ್ಞಗಳು. ಶಾಸ್ತ್ರಕ್ಕೆ ವಿರೋಧವಾಗಿ ಹೋಗುವು ದಿಲ್ಲ. ಶಾಸ್ತ್ರದಲ್ಲಿ ಹೇಳಿರುವ ಯಜ್ಞಗಳನ್ನು ಮಾಡುತ್ತಾನೆ. ಅಲ್ಲಿ ಮಂತ್ರ ಲೋಪವಿಲ್ಲ, ದಕ್ಷಿಣೆ ಲೋಪವಿಲ್ಲ, ಅತಿಥಿ ಅಭ್ಯಾಗತರ ಲೋಪವಿಲ್ಲ. ಎಲ್ಲರಿಗೂ ಯೋಗ್ಯತಾನುಸಾರ ಸಂಭಾವನೆಯನ್ನು ಮಾಡಿ, ಅದರಿಂದ ಬರುವ ಪುಣ್ಯವನ್ನು ಶ‍್ರೀಕೃಷ್ಣಾರ್ಪಣಮಸ್ತು ಎಂದು ಭಗವಂತನಿಗೆ ಅರ್ಪಿಸು ತ್ತಾನೆ. ಅವನಿಗೆ ಏನೂ ಬೇಕಾಗಿಲ್ಲ. ಇದನ್ನು ಮಾಡಬೇಕಾದ್ದು ನನ್ನ ಕರ್ತವ್ಯ, ಯಜ್ಞಕ್ಕೆಲ್ಲ ಅಧಿಕಾರಿ ಭಗವಂತನೊಬ್ಬನೇ, ನಾನೇನು ಉಪಯೋಗಿಸುತ್ತೇನೆಯೋ ಅದೆಲ್ಲ ಅವನಿಂದ ಬಂದದ್ದು; ಅವನಿಗೆ ಕೊಟ್ಟು ಧನ್ಯನಾಗುತ್ತೇನೆ ಎಂದು ಭಾವಿಸುತ್ತಾನೆ. ‘ಕೆರೆಯ ನೀರನು ಕೆರೆಗೆ ಚೆಲ್ಲಿ ಧನ್ಯರಾಗಿರೊ’ ಎಂದು ದಾಸರು ಹಾಡುವಂತೆ.

\begin{verse}
ಅಭಿಸಂಧಾಯ ತು ಫಲಂ ದಂಭಾರ್ಥಮಪಿ ಚೈವ ಯತ್~।\\ಇಜ್ಯತೇ ಭರತಶ್ರೇಷ್ಠ ತಂ ಯಜ್ಞಂ ವಿದ್ಧಿ ರಾಜಸಮ್ \versenum{॥ ೧೨~॥}
\end{verse}

{\small ಆದರೆ ಅರ್ಜುನ, ಫಲಕ್ಕಾಗಿಯೂ ತೋರಿಕೆಗಾಗಿಯೂ ಯಾವ ಯಜ್ಞವನ್ನು ಮಾಡುತ್ತಾರೆಯೊ ಅದು ರಾಜಸಿಕ ಎಂದು ತಿಳಿ.}

ಎರಡನೆಯವನೇ ರಾಜಸಿಕ. ಅವನು ಯಜ್ಞವನ್ನು ಮಾಡುವುದೇ ಫಲಾಪೇಕ್ಷೆಯಿಂದ. ಅನಿಷ್ಟವನ್ನು ಕಳೆದುಕೊಳ್ಳಲೆತ್ನಿಸುವನು. ಇಷ್ಟವನ್ನು ಸಂಪಾದಿಸಲೆತ್ನಿಸುವನು. ಅನಿಷ್ಟಗಳೇ ಕಷ್ಟ, ನಷ್ಟ, ಅಪಾಯ, ರೋಗರುಜಿನಗಳ ಕಾಟ, ನಿಂದೆ ಮುಂತಾದುವು. ಇಷ್ಟವೇ ಲಾಭ, ಐಶ್ವರ್ಯ, ಕೀರ್ತಿ ಮತ್ತು ಕಾಲವಾದ ಮೇಲೆ ಉತ್ತಮ ಲೋಕಗಳನ್ನು ಪಡೆಯುವುದು.

ಇವನು ಕೇವಲ ತೋರಿಕೆಗೆ ಮಾಡುತ್ತಾನೆ. ಅವನು ಏನು ಮಾಡುತ್ತಾನೊ ಅದೆಲ್ಲರಿಗೂ ಗೊತ್ತಾಗಬೇಕು. ಅದರಿಂದ ಇವನಿಗೆ ಒಳ್ಳೆಯ ಹೆಸರು ಬರಬೇಕು. ಕೀರ್ತಿ ಬರಬೇಕು, ಜನರೆಲ್ಲ ಇದನ್ನು ಕುರಿತು ಹೊಗಳಬೇಕು. ಇವನು ಒಳಗಿನದಕ್ಕಿಂತ ಹೊರಗಿನದಕ್ಕೆ ಹೆಚ್ಚು ಪ್ರಾಮುಖ್ಯತೆಯನ್ನು ಕೊಡುವನು. ಒಳಗಿನ ಸತ್ತ್ವ ಕಡಿಮೆ. ಹೊರಗಿನ ಆಟೋಪ ಜಾಸ್ತಿ. ಒಂದು ಸ್ವಲ್ಪ ಹಾಲನ್ನು ದೊಡ್ಡ ಪಾತ್ರೆಯಲ್ಲಿಟ್ಟು ಕಾಯಿಸುತ್ತಿದ್ದಂತೆ. ಹಾಲು, ಬೆಂಕಿ ಬಲವಾದಾಗ ಪಾತ್ರೆಯ ಮೇಲಕ್ಕೆ ಉಕ್ಕುವುದು. ಏನೊ ತುಂಬಾ ಹಾಲು ಇರುವಂತೆ ಕಾಣುವುದು. ಆದರೆ ಉರಿ ನಿಂತಮೇಲೆ ನೋಡಿದರೆ ಇದ್ದ ಸ್ವಲ್ಪ ಹಾಲೂ ತಳ ಹಿಡಿದು ಹೋಗಿರುವುದು. ರಾಜಸಿಕನ ತೋರಿಕೆಯ ಭಕ್ತಿ ಇಂತಹುದು.

\begin{verse}
ವಿಧಿಹೀನಮಸೃಷ್ವಾನ್ನಂ ಮಂತ್ರಹೀನಮದಕ್ಷಿಣಮ್~।\\ಶ್ರದ್ಧಾವಿರಹಿತಂ ಯಜ್ಞಂ ತಾಮಸಂ ಪರಿಚಕ್ಷತೇ \versenum{॥ ೧೩~॥}
\end{verse}

{\small ವಿಧಿಹೀನ, ಅನ್ನದಾನಹೀನ, ಮಂತ್ರಹೀನ, ದಕ್ಷಿಣಾಶೂನ್ಯ, ಶ್ರದ್ಧಾರಹಿತವಾದ ಯಜ್ಞವನ್ನು ತಾಮಸ ಎಂದು ಹೇಳುತ್ತಾರೆ.}

ಇಲ್ಲಿ ತಾಮಸಿಯ ಯಜ್ಞವನ್ನು ವಿವರಿಸುತ್ತಾನೆ. ಇದು ವಿಧಿಹೀನವಾದುದು. ಶಾಸ್ತ್ರದಲ್ಲಿ ಹೇಗೆ ಮಾಡಬೇಕೆಂದು ಹೇಳಿದೆಯೊ ಅವುಗಳಾವುದನ್ನೂ ಇವನು ಗಮನಿಸುವುದಿಲ್ಲ. ಅದನ್ನು ಅನುಸರಿಸದೆ ಇರುವುದು ಮಾತ್ರವಲ್ಲ. ಅನೇಕ ವೇಳೆ ಅದಕ್ಕೇ ವಿರೋಧವಾಗಿ ಮಾಡುವನು.

ಯಜ್ಞದ ಸಮಯದಲ್ಲಿ ಹಲವಾರು ಜನ ನೆರೆಯುತ್ತಾರೆ. ಅವರಿಗೆಲ್ಲ ಯಜ್ಞದಲ್ಲಿ ಅರ್ಪಣೆ ಮಾಡಿದುದನ್ನು ಪ್ರಸಾದದಂತೆ ಕೊಡಬೇಕು. ಯಜ್ಞಪ್ರಸಾದವನ್ನು ಅಮೃತ ಎಂದು ಹೇಳುವರು. ಅದನ್ನು ಎಲ್ಲರಿಗೂ ಕೊಟ್ಟು ಉಳಿದುದನ್ನು ಯಜಮಾನನು ಭುಂಜಿಸಬೇಕು. ಯಾವ ದೊಡ್ಡ ಯಜ್ಞವಾಗಲಿ ಅಲ್ಲಿ ಅನ್ನದಾನವಿರುವುದು. ಬಂದ ಅತಿಥಿ ಅಭ್ಯಾಗತರಿಗೆ ಊಟಕೊಟ್ಟು ತಣಿಸಬೇಕು. ಏಕೆಂದರೆ ದೂರ ದೂರದಿಂದ ಅದನ್ನು ನೋಡಲು ಬಂದಿರುವರು. ಯಾವಾಗ ಅವರಿಗೆ ಊಟ ಕೊಡದೆ ಕಳುಹಿಸುತ್ತೇವೆಯೊ ಆಗ ನಮಗೆ ಅವರ ಶುಭೇಚ್ಛೆ ದೊರಕುವುದಿಲ್ಲ. ಆದಕಾರಣವೇ ಹಿಂದಿನ ಕಾಲದಲ್ಲಿ ದೊಡ್ಡ ದೊಡ್ಡ ಯಜ್ಞಗಳನ್ನು ಮಾಡುತ್ತಿದ್ದಾಗ ಸಹಸ್ರಾರು ಜನರಿಗೆ ಅನ್ನ ಸಂತರ್ಪಣೆಯನ್ನು ಏರ್ಪಡಿಸುತ್ತಿದ್ದರು. ಈ ತಾಮಸಿಯ ಯಜ್ಞದಲ್ಲಿ ಅದೇ ಇಲ್ಲ.

ಅದಕ್ಕೆ ಮಂತ್ರ ಹೇಳಲು ಬರುವವರು ಕೂಡ ಚೆನ್ನಾಗಿ ಅದರಲ್ಲಿ ಪರಿಣತರಾಗದೆ ಇರುವವರು. ಕಾಟಾಚಾರಕ್ಕೆ ಎಲ್ಲೋ ಸ್ವಲ್ಪವನ್ನು ಹೇಳುವರು. ಉಳಿದುದನ್ನು ಹಾರಿಸುವರು. ಆ ಸಮಯಕ್ಕೆ ಔಚಿತ್ಯವಾಗಿರುವುದೂ ಇರುವುದಿಲ್ಲ. ಯಜ್ಞದ ಸಮಯದಲ್ಲಿ ಮಂತ್ರೋಚ್ಚಾರಣೆಯನ್ನು ಸರಿಯಾಗಿ ಮಾಡಬೇಕು. ಯಾವಾಗ ಅದರಲ್ಲಿ ಲೋಪಬರುವುದೋ ಅದನ್ನು ಮಾಡಿಸಿದ ಯಜಮಾನನಿಗೆ ಆಪತ್ತು ತಗುಲುವುದು ಎಂದು ಬೇರೆ ಹೇಳುವರು. ಇದು ಮಂತ್ರದೇವತೆಗೆ ಅಪಚಾರ ಮಾಡಿದಂತೆ. ಸಾಧಾರಣ ಮನುಷ್ಯನಿಗೆ ಹಾನಿ ಮಾಡಿದರೇನೆ ಅವನಿಂದ ನಮಗೆ ಎಷ್ಟೋ ಹಾನಿಯಾಗುವುದು. ಯಾವಾಗ ಮಂತ್ರದೇವತೆಗೇ ಅಪಚಾರ ಮಾಡುವನೋ ಆಗ ಅವನಿಗೆ ಅಭಿಶಾಪ ತಟ್ಟುವುದು. ಹೀಗೆ ಮಾಡುವುದಕ್ಕಿಂತ ಮಾಡದೇ ಇರುವುದು ಮೇಲು. 

ಅವನು ಇದಕ್ಕೆ ನೇಮಿಸಿರುವ ಪುತ್ವಿಜರು ಮುಂತಾದವರಿಗೆ ಸಾಕಷ್ಟು ದಕ್ಷಿಣೆಯನ್ನು ಕೊಡುವು ದಿಲ್ಲ. ಅವರು ಅಷ್ಟು ಶ್ರಮಪಟ್ಟು ಇವನ ಕೆಲಸ ಮಾಡಲು ಬರುತ್ತಾರೆ. ಅವರ ಕಷ್ಟಕ್ಕೆ ತಕ್ಕ ಪಾರಿತೋಷಕವನ್ನು ಕೊಡಬೇಕು. ಆಗಲೇ ಆ ವಿದ್ಯೆಗೆ ಪುರಸ್ಕಾರ ಸಿಕ್ಕುವುದು. ಅದು ಮುಂದುವರಿಯು ವುದಕ್ಕೆ ಸಾಧ್ಯವಾಗುವುದು. ಇವನು ಸಾಕಷ್ಟು ದಕ್ಷಿಣೆ ಕೊಡದೆ ಹೋದರೆ, ಅವನ ಯಜ್ಞ ಪೂರ್ಣ ವಾಗುವುದಿಲ್ಲ. ಇದರಲ್ಲಿ ಭಾಗವಹಿಸಲು ಬಂದವರು ಅತೃಪ್ತಿಪಟ್ಟರೆ ಮಾಡಿಸಿದ ಯಜಮಾನನಿಗೆ ಹಾನಿಯಾಗುವುದು.

ಅವನು ಮಾಡಿಸುವ ಯಜ್ಞದಲ್ಲಿ ಯಾವ ಶ್ರದ್ಧೆಯೂ ಇಲ್ಲ. ಬರೀ ಕಾಟಾಚಾರಕ್ಕೆ ಮಾಡುವನು. ಬಂದರೆ ಬರಲಿ ಹೋದರೆ ಹೋಗಲಿ ಎಂಬುದೇ ಇವನ ಸ್ವಭಾವ. ನಾವು ಯಾವಾಗ ಕಾಟಾಚಾರದಿಂದ ಮಾಡುವೆವೊ ನಮಗೆ ವಿಪತ್ತು ಬರುವುದು. ನಾವೊಂದು ಪವಿತ್ರವಾದ ಕಾರ್ಯಕ್ಕೆ ಅವಮಾನ ಮಾಡಿದಂತೆ. ಮಾಡದೆ ಸುಮ್ಮನಿರುವುದೇ ಕಾಟಾಚಾರಕ್ಕೆ ಮಾಡುವುದಕ್ಕಿಂತಲೂ ಮೇಲು. ಆದರೆ ತಮೋಗುಣಿ ತೆಪ್ಪಗೆ ಮಲಗಿಕೊಂಡಿರುವ ಬೇಟೆನಾಯಿಯನ್ನು ತನ್ನ ಮೇಲೆ ಅಟ್ಟಿಸಿಕೊಂಡು ಬರುವಂತೆ ಮಾಡುತ್ತಾನೆ.

\begin{verse}
ದೇವದ್ವಿಜಗುರುಪ್ರಾಜ್ಞಪೂಜನಂ ಶೌಚಮಾರ್ಜವಮ್~।\\ಬ್ರಹ್ಮಚರ್ಯಮಹಿಂಸಾ ಚ ಶಾರೀರಂ ತಪ ಉಚ್ಯತೇ \versenum{॥ ೧೪~॥}
\end{verse}

{\small ದೇವ ಬ್ರಾಹ್ಮಣ ಗುರು ಜ್ಞಾನಿಗಳ ಪೂಜೆ, ಶೌಚ, ಆರ್ಜವ, ಬ್ರಹ್ಮಚರ್ಯ ಮತ್ತು ಅಹಿಂಸೆ ಇವನ್ನು ಶಾರೀರಕ ತಪಸ್ಸು ಎಂದು ಹೇಳುತ್ತಾರೆ.}

ತಪಸ್ಸಿನಲ್ಲಿ ಮೂರು ವಿಧಗಳಿವೆ. ಒಂದು ಶಾರೀರಕ ತಪಸ್ಸು, ಎರಡನೆಯದು ವಾಚಿಕ ತಪಸ್ಸು, ಮೂರನೆಯದು ಮಾನಸಿಕ ತಪಸ್ಸು. ತುಂಬಾ ಸ್ಥೂಲವಾಗಿ ಎಲ್ಲರಿಗೂ ಕಾಣುವುದೇ ಶಾರೀರಕ ತಪಸ್ಸು. ಇಲ್ಲಿ ದೇವರು ಬ್ರಾಹ್ಮಣರು, ಗುರುಗಳು ಮತ್ತು ಜ್ಞಾನಿಗಳಿಗೆ ಪೂಜೆ ಮಾಡುವುದನ್ನು ಶಾರೀರಕ ತಪಸ್ಸಿನ ಒಂದು ಭಾಗ ಎಂದು ಹೇಳುತ್ತಾನೆ. ನಮ್ಮ ದೇಹವನ್ನು ದೇವರ ಹೆಸರಿನಲ್ಲಿ ದಂಡಿಸಬೇಕು, ದೇವರ ಪೂಜೆ ಮಾಡಬೇಕು, ಅವನಿಗೆ ನಮಸ್ಕಾರ ಮಾಡಬೇಕು. ಅವನ ಪೂಜೆಗೆ ಹೂವು ಕೀಳಬೇಕು, ಹೂವಿನ ಹಾರ ಮಾಡಬೇಕು, ಅವನಿಗೆ ನೈವೇದ್ಯಾದಿಗಳನ್ನು ತಯಾರು ಮಾಡಬೇಕು. ದೇವರ ಮನೆಯನ್ನು ಸಾರಿಸಿ, ಗುಡಿಸಿ ರಂಗೋಲಿ ಇಡುವುದು, ಪೂಜೆಗೆ ಬೇಕಾದ ಪಾತ್ರೆ ಮುಂತಾದುವುಗಳನ್ನೆಲ್ಲ ತೊಳೆಯುವುದು ಅದರಲ್ಲಿ ಸೇರಿದೆ. ಇವುಗಳೆಲ್ಲ ನಮ್ಮ ಚಿತ್ತವನ್ನು ಶುದ್ಧಿಮಾಡುವುವು ಮತ್ತು ಮನಸ್ಸನ್ನು ಭಗವಂತನ ಮೇಲೆ ಏಕಾಗ್ರ ಮಾಡುವುದಕ್ಕೆ ಸಹಾಯವಾಗು ವುವು.

ಅನಂತರವೇ ಬ್ರಾಹ್ಮಣರ ಪೂಜೆ. ಇಲ್ಲಿ ಬ್ರಾಹ್ಮಣ ಎಂದರೆ ಎಲ್ಲೊ ಆ ಕುಲದಲ್ಲಿ ಹುಟ್ಟಿದವನಿಗೆ ಅನ್ವಯಿಸುವುದಿಲ್ಲ. ವೃತ್ತಿಯಲ್ಲಿ ಯಾರು ಬ್ರಹ್ಮೋಪಾಸನೆಯಲ್ಲಿ ನಿರತರಾಗಿರುವರೋ ಅವರಿಗೆ ನಾವು ಗೌರವ ತೋರಬೇಕು. ಅಂತಹ ವ್ಯಕ್ತಿ ಯಾವ ಕುಲದಿಂದಲಾದರೂ ಬಂದಿರಲಿ, ತನ್ನ ಬಾಳನ್ನು ಭಗವದುಪಾಸನೆಗೆ ಮೀಸಲಾಗಿಟ್ಟಿದ್ದರೆ ಅವನು ಗೌರವಾರ್ಹ. ಅವನಿಗೆ ನಾವು ಗೌರವ ತೋರಿದರೆ, ಅವನ ಆಶೀರ್ವಾದ, ಶುಭಾಶಯ, ಪ್ರೀತಿಯನ್ನು ಗಳಿಸಿಕೊಂಡರೆ, ಇದೊಂದು ಶ್ರೇಷ್ಠವಾದ ಲಾಭ, ಆಧ್ಯಾತ್ಮಿಕ ಜೀವನಕ್ಕೆ. ಅವನೇ ನನ್ನ ಆಧ್ಯಾತ್ಮಿಕ ಕಣ್ಣನ್ನು ತೆರೆದ ಗುರು. ಇವನು ನನ್ನನ್ನು ಭಗವಂತನೆಡೆಗೆ ಕರೆದುಕೊಂಡು ಹೋಗುವನು. ನನಗೂ ದೇವರಿಗೂ ಒಂದು ಸಂಬಂಧವನ್ನು ಕಲ್ಪಿಸುವವನು. ಇವನು ನನಗೆ ಒಂದು ಹೊಸ ಜನ್ಮವನ್ನು ಕೊಟ್ಟವನು. ಆದಕಾರಣವೆ ಭಗವಂತನ ಆವಿರ್ಭಾವನೆ ಮೊದಲು ನಮಗೆ ಕಾಣುವುದು ಗುರುವಿನಲ್ಲಿ. ಅವನಿಗೆ ಪೂಜೆ ಮಾಡಬೇಕು. ಅವನನ್ನು ನಮ್ಮ ಸೇವೆಯಿಂದ ಒಲಿಸಿಕೊಳ್ಳಬೇಕು. ಅವನು ಇಚ್ಛಿಸಿದರೆ ನಮಗೆ ಏನನ್ನು ಬೇಕಾದರೂ ಅನುಗ್ರಹಿಸಬಲ್ಲ. ಪ್ರಪಂಚದಲ್ಲಿ ಕೆಲವರು ಕೇವಲ ಗುರುಸೇವೆಯಿಂದಲೇ ಆಧ್ಯಾತ್ಮಿಕ ಜೀವನದ ಗಹನವಾದ ಅನುಭವವನ್ನು ಗಳಿಸಿರುವರು. ಮಹಾಗುರುವಿಗೆ ಮಾಡಿದ ಯಾವ ಒಂದು ಸೇವೆಯೂ ನಿರರ್ಥಕವಾಗುವುದಿಲ್ಲ. ಜ್ಞಾನ ಭಕ್ತಿ ವೈರಾಗ್ಯಗಳನ್ನೆಲ್ಲ ಅವನ ಇಚ್ಛಾಮಾತ್ರದಿಂದ ನಾವು ಗಳಿಸ ಬಹುದು, ಅವನನ್ನು ನಮ್ಮ ಹೃತ್ಪೂರ್ವಕವಾದ ಸೇವೆಯಿಂದ ಒಲಿಸಿಕೊಂಡಾಗ.

ಈ ಜೀವನದಲ್ಲಿ ಒಬ್ಬರು ನನಗೆ ಗುರುವಾಗಿಲ್ಲ. ಆದರೆ ದೊಡ್ಡ ಜ್ಞಾನಿಗಳು, ಅನುಭವಿಗಳು, ಅವರು ಯಾವ ಕುಲಗೋತ್ರಗಳಿಗೆ ಸೇರಿರಲಿ, ಅವರು ಪೂಜೆಗೆ ಅರ್ಹರು. ನಾವು ಅವರನ್ನು ಪೂಜಿಸಿದರೆ ಅವರಿಗೇನೂ ಕಿರೀಟ ಬರುವುದಿಲ್ಲ. ನಾವು ಪೂಜಿಸಿದರೂ ಒಂದೇ ತೆಗಳಿದರೂ ಒಂದೇ ಅವರಿಗೆ. ಆದರೆ ಅದರಿಂದ ಉದ್ಧಾರವಾಗುವವರು ನಾವು. ಜ್ಞಾನಿಗಳು, ಚಲಿಸುವ ತೀರ್ಥಗಳು, ಮಾತನಾಡುವ ಶಾಸ್ತ್ರಗಳು. ಅಂತಹ ಮಹಿಮರ ಪೂಜೆಯೇ ಒಂದು ತಪಸ್ಸು, ದೊಡ್ಡ ಸಾಧನೆ. 

ಅವನ ಆಚಾರ ವ್ಯವಹಾರದಲ್ಲಿ ಆರ್ಜವ ಇರಬೇಕು. ಒಳಗೊಂದು ಹೊರಗೊಂದು ಇರಬಾರದು. ಗ್ಲಾಸಿನ ಹಿಂದೆ ಇರುವ ವಸ್ತು ಹೇಗೆ ಕಾಣುವುದೊ ಹಾಗೆ ಅವನಲ್ಲಿ ಮುಚ್ಚುಮರೆಯಿಲ್ಲ. ಕಂಡುದನ್ನು ಕಂಡಂತೆ ಹೇಳುತ್ತಾನೆ. ಅದರಲ್ಲಿ ಯಾವ ಉತ್ಪ್ರೇಕ್ಷೆಯೂ ಇಲ್ಲ. ಯಾವ ಕೇಡೂ ಇಲ್ಲ. ಡೊಂಕೂ ಇಲ್ಲ. ಅವನು ಬ್ರಹ್ಮಚರ್ಯ ವ್ರತದಲ್ಲಿ ನೆಲೆಸಿರುವನು. ಕಾಮವನ್ನು ಗೆದ್ದಿರುವನು. ಇಂದ್ರಿಯಗಳನ್ನು ನಿಗ್ರಹಿಸಿ ಅದನ್ನು ಅಂತರ್ಮುಖ ಮಾಡಿರುವನು. ಅಹಿಂಸಾವ್ರತಧಾರಿ ಅವನು. ಯಾರಿಗೂ ಹಿಂಸೆಯನ್ನು ಕೊಡುವುದಿಲ್ಲ. ಹಿಂಸೆಯಿಂದ ಹಿಂಸೆ ಬೆಳೆಯುವುದು. ಅದು ಅಧಿಕವಾಗುತ್ತ ಬರು ವುದು. ಎಲ್ಲಿಯಾದರೂ ಒಂದು ಕಡೆ ಅದು ನಿಲ್ಲಬೇಕು. ಅದು ನನ್ನಿಂದ ಪ್ರಾರಂಭವಾಗಲಿ ಎನ್ನುವನು. ಜೀವನದಲ್ಲಿ ಯಾರಾದರೊಬ್ಬರು ಅನುಭವಿಸಬೇಕಾಗಿದೆ. ನಾನು ಅನುಭವಿಸುತ್ತೇನೆ ಎನ್ನುವನು. ಅವನ ಸಾನ್ನಿಧ್ಯವೇ ಅಹಿಂಸಾಭಾವನೆಯಿಂದ ಓತಪ್ರೋತವಾಗಿರುವುದು. ಯಾರು ಅವನ ಸಮೀಪದಲ್ಲಿರುವರೋ ಅವರ ಮೇಲೆಲ್ಲ ಇದು ತನ್ನ ಪ್ರಭಾವವನ್ನು ಬೀರುವುದು.

\begin{verse}
ಅನುದ್ವೇಗಕರಂ ವಾಕ್ಯಂ ಸತ್ಯಂ ಪ್ರಿಯಹಿತಂ ಚ ಯತ್~।\\ಸ್ವಾಧ್ಯಾಯಾಭ್ಯಸನಂ ಚೈವ ವಾಙ್ಮಯಂ ತಪ ಉಚ್ಯತೇ \versenum{॥ ೧೫~॥}
\end{verse}

{\small ಉದ್ವೇಗವಿಲ್ಲದೆ ಸತ್ಯವೂ ಪ್ರಿಯವೂ ಹಿತವೂ ಆದ ಮಾತು, ಮತ್ತು ತಮ್ಮ ವೇದಶಾಖೆಯ ಅಧ್ಯಯನ ಇವನ್ನು ವಾಚಿಕ ತಪಸ್ಸು ಎಂದು ಹೇಳುತ್ತಾರೆ.}

ಜೀವನದಲ್ಲಿ ಮಾತಿನಷ್ಟು ಸುಲಭ ಯಾವುದೂ ಇಲ್ಲ. ಬೆಳಗಿನಿಂದ ಸಾಯಂಕಾಲದವರೆಗೆ ಮಾತನಾಡುತ್ತಿರುವೆವು. ಬಹುಪಾಲು ಈ ಮಾತೆಲ್ಲ ವ್ಯರ್ಥವಾಗುತ್ತಿದೆ. ಅದರಿಂದ ನಾವು ಶಕ್ತಿಯನ್ನು ಸಂಗ್ರಹಿಸುತ್ತಿಲ್ಲ. ಈ ಮಾತನ್ನೇ ಒಂದು ತಪಸ್ಸಿಗೆ ಪರಿವರ್ತಿಸಬಹುದು. ಇದರಷ್ಟು ಸುಲಭವಾದ ತಪಸ್ಸು ಮತ್ತೊಂದಿಲ್ಲ. ಇದರಷ್ಟು ಖರ್ಚಿಲ್ಲದ ತಪಸ್ಸು ಮತ್ತೊಂದಿಲ್ಲ. ಇದಕ್ಕೆ ದುಡ್ಡು ಬೇಕಾಗಿಲ್ಲ. ಯಾರು ಬೇಕಾದರೂ ಮಾಡಬಹುದು. ಮಾತನ್ನು ತಪಸ್ಸಿನಂತೆ ಪರಿವರ್ತಿಸಬೇಕಾದರೆ ಏನು ಮಾಡಬೇಕು ಎಂಬುದನ್ನು ಹೇಳುತ್ತಾನೆ.

ಮಾತಿನಲ್ಲಿ ಉದ್ವೇಗ ಇರಕೂಡದು. ನಾವು ಮಾತನಾಡುವಾಗ ಅಲ್ಲಿ ಕೋಪವಿರಬಾರದು. ಮಾತ್ಸರ್ಯ ಇರಬಾರದು. ಮಾತಿನಷ್ಟು ಸುಲಭ ಮತ್ತೊಂದಿಲ್ಲ. ಆದರೆ ಎಂತಹ ಗಾಯವನ್ನು ಮಾಡುವುದು ಮಾತು? ಕತ್ತಿಯಿಂದ ಆದ ಗಾಯ ಮಾಗುವುದು. ಆದರೆ ಕೋಪದ ಮಾತಿನಿಂದ ಆದ ಗಾಯ ಬೇಗ ಮಾಗುವುದಿಲ್ಲ. ಹಾವು ಚೇಳಿನ ವಿಷಕ್ಕಿಂತ ಹೆಚ್ಚು ವಿಷವನ್ನು ಕಾರಬಹುದು, ಮಾತಿನ ಮೂಲಕ ಅಥವಾ ಅದರಿಂದ ಅಮೃತವನ್ನೇ ಹರಿಸಬಹುದು. ಮಾತನ್ನು ತಪಸ್ಸು ಮಾಡಬೇಕಾದರೆ ಅಲ್ಲಿ ಉದ್ವೇಗವಿರಕೂಡದು, ಅದು ಮೃದುವಾಗಿರಬೇಕು, ಮಧುರವಾಗಿರಬೇಕು.

ಯಾವಾಗಲೂ ಸತ್ಯವನ್ನೇ ಹೇಳಬೇಕು. ಕಲಿಯುಗದಲ್ಲಿ ಇದೇ ಒಂದು ತಪಸ್ಸು ಎಂದು ಶ‍್ರೀರಾಮಕೃಷ್ಣರು ಹೇಳುತ್ತಿದ್ದರು. ನಾವು ಹೇಳುವುದು ಬರೀ ಸತ್ಯವಾಗಿರುವುದು ಮಾತ್ರವಲ್ಲ, ಅದು ಕೇಳುವವನಿಗೆ ಪ್ರಿಯವಾಗಿರಬೇಕು. ಮಾತನ್ನು ಕೇಳಿದಾಗ ಅವನು ದುಃಖವನ್ನು ಮರೆಯಬೇಕು. ಸಂಶಯದಿಂದ ಪಾರಾಗಬೇಕು. ಮಾತು ಅವನಿಗೆ ಒಂದು ಭರವಸೆಯನ್ನು ಕೊಡಬೇಕು. ನಾವು ಹೊತ್ತಿದ್ದ ದೊಡ್ಡ ಭಾರವನ್ನು ಯಾರೋ ಇಳಿಸಿದಂತಾಗುವುದು ಪ್ರಿಯವಾಗಿರುವ ಮಾತು ಕೇಳಿ ದಾಗ. ಪ್ರಿಯವಾಗಿರುವುದನ್ನೇ ಹೇಳಬೇಕೆಂದು ಪ್ರಿಯವಾದ ಸುಳ್ಳನ್ನು ಹೇಳಬಾರದು. ಅದು ಹಿತಕಾರಿಯೂ ಆಗಿರಬೇಕು. ಕೇಳುವುದಕ್ಕೆ ಆನಂದ ಮಾತ್ರವಲ್ಲ, ಅದರಿಂದ ನನ್ನ ಜೀವನಕ್ಕೆ ಸಹಾಯವಾಗಬೇಕು. ಇದು ರುಚಿಕರವಾಗಿರುವ ಔಷಧಿಯಂತೆ, ಬಾಯಿಗೆ ರುಚಿಕರ, ಹೊಟ್ಟೆಗೆ ಹೋದ ಮೇಲೂ ಅದು ನಮ್ಮ ರೋಗವನ್ನು ಗುಣ ಮಾಡಬೇಕು. ಮಾತು ಹಾಗಿರಬೇಕು.

ನಾವು ಯಾವ ಶಾಖೆಗೆ ಸೇರಿರುವೆವೊ ಅದಕ್ಕೆ ಸಂಬಂಧಪಟ್ಟ ಶಾಸ್ತ್ರಗಳನ್ನು ಓದಬೇಕು. ಅದರಲ್ಲಿರುವುದನ್ನು ತಿಳಿದುಕೊಳ್ಳಬೇಕು. ಪಾರಾಯಾಣ ಮುಂತಾದುವುಗಳನ್ನು ಮಾಡಬೇಕು. ಬರೀ ಒಂದು ಶಾಖೆಗೆ ಹುಟ್ಟುವುದಲ್ಲ ದೊಡ್ಡದು. ಅಲ್ಲಿರುವ ಶಾಸ್ತ್ರಗಳನ್ನು ನಾವು ಓದಬೇಕು. ಮತ್ತೊಬ್ಬ ನಿಗೆ ನಾವು ಕಲಿತಿದ್ದನ್ನು ಹೇಳಿಕೊಡಬೇಕು. ಆಗಲೆ ಜ್ಞಾನ ನಂದದೆ ಮುಂದುವರಿಯಬೇಕಾದರೆ. ಹಿಂದಿನ ಕಾಲದಲ್ಲಿ ಅಕ್ಷರ ಬರುವುದಕ್ಕೆ ಮೊದಲು ಜ್ಞಾನ ಹರುಡುತ್ತಿದ್ದುದೇ ಮಾತಿನ ಮೂಲಕ. ಅದಕ್ಕೇ ತೈತ್ತಿರೀಯ ಉಪನಿಷತ್ತಿನಲ್ಲಿ ಅಧ್ಯಯನ ಮತ್ತು ಪ್ರವಚನವನ್ನು ಬಿಡಕೂಡದು ಎಂದು ಹೇಳಿದೆ.

\begin{verse}
ಮನಃಪ್ರಸಾದಃ ಸೌಮ್ಯತ್ವಂ ಮೌನಮಾತ್ಮವಿನಿಗ್ರಹಃ~।\\ಭಾವಸಂಶುದ್ಧಿರಿತ್ಯೇತತ್ತಪೋ ಮಾನಸಮುಚ್ಯತೇ \versenum{॥ ೧೬~॥}
\end{verse}

{\small ಮನಸ್ಸಿನ ಪ್ರಸನ್ನತೆ, ಸೌಮ್ಯಭಾವ, ಮೌನ, ಮನೋನಿರೋಧ, ಭಾವಶುದ್ಧಿ ಇವುಗಳನ್ನೇ ಮಾನಸಿಕ ತಪಸ್ಸು ಎಂದು ಹೇಳುತ್ತಾರೆ.}

ಇಲ್ಲಿ ಮಾನಸಿಕ ತಪಸ್ಸನ್ನು ಹೇಳುವನು. ಇದು ದೈಹಿಕ ಮತ್ತು ವಾಚಿಕಕ್ಕಿಂತ ಮುಖ್ಯವಾದುದು. ಹೊರಗಿನದನ್ನು ನಿಗ್ರಹಿಸುವುದು ಸುಲಭ. ಆದರೆ ಮನಸ್ಸನ್ನು ನಿಗ್ರಹಿಸುವುದು ಕಷ್ಟ. ಮನಸ್ಸೇ ಎಲ್ಲದಕ್ಕೂ ಮೂಲ. ಇದು ಸರಿಯಾಗಿದ್ದರೆ ನಮ್ಮ ಮಾತು, ನಡವಳಿಕೆ ಎಲ್ಲಾ ಸರಿಯಾಗಿರುವುದು. ಇದು ಮೂಲ. ಉಳಿದಿರುವುದು ಅದರಿಂದ ಬಂದ ರೆಂಬೆಗಳು. ಮೊದಲು ಮನಸ್ಸು ಪ್ರಸನ್ನವಾಗಿರ ಬೇಕು. ಮನಸ್ಸು ತಿಳಿಯಾಗಿರಬೇಕು, ಶಾಂತವಾಗಿರಬೇಕು. ಅಲೆಯಿಲ್ಲದ ಸರೋವರದಂತೆ ಇರ ಬೇಕು. ಅದು ಯಾವಾಗಲೂ ಸಂತೋಷದಿಂದ ಮತ್ತು ತೃಪ್ತಿಯಿಂದ ಅರಳಿರಬೇಕು. ಹೂವು ಅರಳಿದಾಗ ಪರಿಮಳವನ್ನು ಬೀರುವಂತೆ, ಮನಸ್ಸು ಪ್ರಸನ್ನವಾದಾಗ ಎಲ್ಲಾ ಒಳ್ಳೆಯ ಗುಣಗಳ ಪರಾಗವನ್ನು ಚೆಲ್ಲುತ್ತಿರುವುದು.

ಮನಸ್ಸು ಸೌಮ್ಯವಾಗಿರಬೆಕು. ಅಲ್ಲಿ ಅಲ್ಲೋಲಕಲ್ಲೋಲವಿರಕೂಡದು. ಯಾವ ಉದ್ವಿಗ್ನತೆಯ ಅಲೆಯೂ ಏಳಬಾರದು. ಇತರರಿಗೆ ಯಾವಾಗಲೂ ಒಳ್ಳೆಯದನ್ನೆ ಆಶಿಸುತ್ತಿರಬೇಕು. ಮನಸ್ಸು ಎಂದಿಗೂ ಇತರರಿಗೆ ಕೇಡನ್ನು ಬಯಸಕೂಡದು. ಎಲ್ಲರೂ ಉದ್ಧಾರವಾಗಲಿ, ಎಲ್ಲರೂ ಕಷ್ಟದಿಂದ ಪಾರಾಗಲಿ, ಎಲ್ಲರಿಗೂ ಶಾಂತಿ ಲಭಿಸಲಿ ಎಂದು ಮನಸ್ಸು ಸದಾ ಆಶಿಸುತ್ತಿರಬೇಕು. ಇತರರ ಮೇಲೆ ಕರುಣೆ ಇರಬೇಕು, ದಯೆ ಇರಬೇಕು. ಇವುಗಳೆಲ್ಲ ಇದ್ದರೆ ಸೌಮ್ಯ ಭಾವ ಬರುವುದು.

ಮೌನವಾಗಿರಬೇಕು ಎಂದರೆ ಮಾತನ್ನೇ ಆಡಕೂಡದು ಎಂದಲ್ಲ. ಮಾತನ್ನೇ ಆಡದೆ ಇದ್ದರೆ ವಾಚಿಕವಾದ ತಪಸ್ಸನ್ನು ಹೇಗೆ ಮಾಡವುದು? ಆದಕಾರಣವೇ ಇಲ್ಲಿ ಮೌನ ಎಂದರೆ ಸಾಧ್ಯವಾದಷ್ಟು ಕಡಿಮೆ ಅವನು ಮಾತನಾಡುತ್ತಾನೆ. ಎಷ್ಟು ಬೇಕೊ ಅಷ್ಟನ್ನು ಮಾತ್ರ ಮಾತನಾಡುವನು. ಏಕೆಂದರೆ ಅನಾವಶ್ಯಕವಾಗಿ ಮಾತನಾಡಿ ನಮ್ಮ ಮಾನಸಿಕ ಶಕ್ತಿಯನ್ನು ನಾವು ವ್ಯಯಮಾಡಿಕೊಳ್ಳುವೆವು. ಎಷ್ಟು ಮಾತನಾಡುವುದನ್ನು ಕಡಿಮೆ ಮಾಡುವೆವೊ ಅಷ್ಟು ಮನಸ್ಸಿನ ಮೇಲೆ ನಮಗೆ ಹತೋಟಿ ಬರುವುದು.

ಮನಸ್ಸನ್ನು ನಿಗ್ರಹಿಸಬೇಕು. ನಾವು ಮಾತನಾಡುವುದನ್ನು ಕಡಿಮೆ ಮಾಡಬಹುದು. ಆದರೆ ಮನಸ್ಸಿನಲ್ಲಿ ಇಪ್ಪತ್ತೆಂಟು ಆಲೋಚನೆಗಳು ಬರುತ್ತಿದ್ದರೆ, ಮಾತನ್ನು ನಿಲ್ಲಿಸಿಯೂ ಪ್ರಯೋಜನ ವಾಗುವುದಿಲ್ಲ. ಚಿತ್ತ ಚಾಂಚಲ್ಯವೇ ಮುಂದಿನ ಅನೇಕ ದುಷ್ಕರ್ಮಗಳಿಗೆ ಕಾರಣ. ನಾವು ಯಾವಾಗ ಮನಸ್ಸನ್ನು ಹರಿಬಿಡುತ್ತೇವೆಯೊ ಆಗ ಜಾರಲು ಅದು ಮೊದಲಾಗುವುದು. ಅದು ಕ್ರಮೇಣ ಜಾರುತ್ತ ಜಾರುತ್ತ ಮಾಡಬಾರದ ಕ್ರಿಯೆಗಳಲ್ಲಿ ಪರ್ಯವಸಾನವಾಗುವುದು. ಹೀನ ಕಾರ್ಯಗಳೆಲ್ಲ ಹುಟ್ಟಿದ್ದು ಮುಂಚೆ ಹೀನ ಆಲೋಚನೆಯಲ್ಲಿ. ನಾವು ಯಾವಾಗ ಹೀನ ಆಲೋಚನೆಯನ್ನು ನಿಗ್ರಹಿಸುವೆವೊ ಅನಂತರ ನಾವು ಹೀನ ಕಾರ್ಯವನ್ನು ನಿರೋಧಿಸಲೇ ಬೇಕಾಗಿಲ್ಲ. ಏಕೆಂದರೆ ಆ ಕಾರ್ಯವೇ ಆಗುವುದಿಲ್ಲ. ಕಾರ್ಯವನ್ನು ಮಾಡುವಂತೆ ಪ್ರೇರೇಪಿಸುವುದೇ ಹೀನ ಆಲೋಚನೆ. ಅದೇ ಇಲ್ಲದೇ ಇದ್ದರೆ ಇನ್ನು ಹೀನ ಕಾರ್ಯ ಹೇಗೆ ಸಾಧ್ಯ? ಬೆಂಕಿಯೇ ಇಲ್ಲದೆ ಇದ್ದರೆ ಇನ್ನು ಒಲೆ ಹೇಗೆ ಉರಿಯುವುದು?

ಭಾವ ಶುದ್ಧಿಯಾಗಬೇಕು. ನಮ್ಮ ಉದ್ದೇಶ ಪರಿಶುದ್ಧವಾಗಿರಬೇಕು ಮತ್ತೊಬ್ಬರೊಡನೆ ವ್ಯವ ಹರಿಸುವಾಗ. ಹೊರಗೆ ಒಳ್ಳೆಯ ಮಾತು ಕೆಲಸ ಮಾಡುತ್ತಿದ್ದರೂ ಮತ್ತಾವುದೋ ಒಂದು ಉದ್ದೇಶವನ್ನು ಇಟ್ಟುಕೊಂಡು ಇವುಗಳನ್ನು ಮಾಡಬಹುದು. ಆ ಉದ್ದೇಶ ಲೌಕಿಕವಾಗಿರಬಹುದು. ಇಲ್ಲಿನ ಉದ್ದೇಶಗಳು ಕೂಡ ಶುದ್ಧವಾಗಿರಬೇಕು. ನಮ್ಮಲ್ಲಿ ಯಾವ ಒಂದು ಸ್ವಾರ್ಥವೂ ಇರ ಕೂಡದು, ಅಧಿಕಾರಲಾಲಸೆಯೂ ಇರಕೂಡದು. ಬರೀ ಒಳ್ಳೆಯ ಕೆಲಸಗಳನ್ನು ಮಾತ್ರ ಮಾಡಿದ್ದರೆ ಸಾಲದು. ಅನೇಕವೇಳೆ ಸಮಾಜದಲ್ಲಿ ಹಲವು ಜನ ಹಲವು ಒಳ್ಳೆಯ ಕೆಲಸಗಳನ್ನು ಮಾಡುತ್ತಿರುವರು. ಇದು ಬರೀ ಸ್ವಾರ್ಥರಾಗಿರುವುದಕ್ಕಿಂತ ಮೇಲು. ಈ ಒಳ್ಳೆಯ ಕೆಲಸ ಮಾಡಿ ಅದರಿಂದ ಜನರಿಗೆ ಗೊತ್ತಾದ ಮೇಲೆ ಚುನಾವಣೆಗೆ ನಿಲ್ಲುವರು. ಅದರಲ್ಲಿ ಗೆದ್ದಮೇಲೆ ಯಾವುದಾದರೂ ಮಂತ್ರಿ ಪದವಿಯೋ ಮಣ್ಣು ಮಸಿಯನ್ನೋ ಆಶಿಸುವರು. ಇದು ಪ್ರಾರಂಭವಾಗಿದ್ದು ಒಳ್ಳೆಯ ಕೆಲಸದಿಂದ. ಆದರೆ ಒಳ್ಳೆಯ ಕೆಲಸದ ಹಿಂದೆ ಇದ್ದ ಉದ್ದೇಶ ಒಳ್ಳೆಯದಲ್ಲ. ಭಾವ ಸಂಶುದ್ಧಿಯಾಗಬೇಕಾದರೆ, ಆ ಉದ್ದೇಶವೂ ಶುದ್ಧಿಯಾಗಿರಬೇಕು. ಅವನು ಯಾರಿಂದ ಏನನ್ನೂ ಆಶಿಸುವುದಿಲ್ಲ. ಒಳ್ಳೆಯದನ್ನು ಮಾಡುವುದು ಅವನ ಸ್ವಭಾವ, ಅದಕ್ಕಾಗಿ ಅವನು ಮಾಡುವನು. ಮಲ್ಲಿಗೆ ಪರಿಮಳವನ್ನು ಬೀರು ವುದು ಅದರ ಸ್ವಭಾವ. ಅದಕ್ಕಾಗಿ ಬೀರುವುದು. ಜನ ಅದನ್ನು ಸ್ವೀಕರಿಸಿದರೆಷ್ಟು, ಬಿಟ್ಟರೆಷ್ಟು, ಹೊಗಳಿದರೇನು, ತೆಗಳಿದರೇನು?

\begin{verse}
ಶ್ರದ್ಧಯಾ ಪರಯಾ ತಪ್ತಂ ತಪಸ್ತತ್ತ್ರಿವಿಧಂ ನರೈಃ~।\\ಅಫಲಾಕಾಂಕ್ಷಿಭಿರ್ಯುಕೆ¦ಃ ಸಾತ್ತ್ವಿಕಂ ಪರಿಚಕ್ಷತೇ \versenum{॥ ೧೭~॥}
\end{verse}

{\small ಫಲಾಪೇಕ್ಷೆಯನ್ನು ಬಿಟ್ಟು ದೃಢತೆಯಿಂದ ಪರಮಶ್ರದ್ಧೆಯಿಂದ ಮಾಡುವ ಈ ಮೂರು ವಿಧವಾದ ತಪಸ್ಸನ್ನು ಸಾತ್ತ್ವಿಕ ಎಂದು ಹೇಳುತ್ತಾರೆ.}

ನಾವು ಮಾಡುವ ಕರ್ಮ ತಪಸ್ಸಾಗಬೇಕಾದರೆ ಯಾವ ಪ್ರತಿಫಲಾಪೇಕ್ಷೆಯನ್ನೂ ಬಯಸಬಾರದು. ಪ್ರತಿಫಲಕ್ಕೆ ಕೈಯೊಡ್ಡುವುದು ನಮ್ಮನ್ನು ಗುಲಾಮರನ್ನಾಗಿ ಮಾಡುವುದು. ಪ್ರತಿಫಲ ಹೇಳಿದಂತೆ ಕುಣಿಯುತ್ತೇವೆ. ಅದು ಕುಣಿಸಿದರೆ ಕುಣಿಯುತ್ತೇವೆ. ನಗಿಸಿದರೆ ನಗುತ್ತೇವೆ. ಅದರ ಕೈಗೊಂಬೆ ಗಳಾಗುತ್ತೇವೆ ನಾವು. ಕೆಲಸ ಮಾಡುವವನು ಯಾವಾಗ ಫಲಾಪೇಕ್ಷೆಯನ್ನು ಬಿಡುವನೊ ಆಗ ಶ್ರೇಷ್ಠಫಲವೇ ಅವನಿಗೆ ಬರುವುದು. ನಾವು ಬೇಡ ಎಂದರೆ ಅದು ಬರುವುದು ನಿಲ್ಲುವುದಿಲ್ಲ. ಅನಾಸಕ್ತಿಯಿಂದ ಮಾಡಿದ ಕಾರ್ಯಕ್ಕೆ ಎಲ್ಲಕ್ಕಿಂತ ಹೆಚ್ಚು ಪ್ರತಿಫಲ ಸಿಕ್ಕುವುದು. ಆದರೆ ಆ ಪ್ರತಿಫಲಕ್ಕೆ ದಾಸನಲ್ಲ, ಅದಕ್ಕಾಗಿ ಮಾಡುವುದಿಲ್ಲ. ಬಂದರೆ ಅದಕ್ಕೆ ತಾನು ಕೈಯೊಡ್ಡುವುದಿಲ್ಲ, ಅದನ್ನು ಭಗವದರ್ಪಣೆ ಮಾಡುತ್ತಾನೆ.

ಅವನಲ್ಲಿ ದೃಢತೆಯನ್ನು ನೋಡುತ್ತೇವೆ. ಚಿತ್ತಚಾಂಚಲ್ಯವಿಲ್ಲ. ಒಂದು ಒಳ್ಳೆಯ ಕೆಲಸವನ್ನು ಮಾಡಿದಮೇಲೆ ಪಶ್ಚಾತ್ತಾಪ ಪಡುವುದಿಲ್ಲ. ಅಯ್ಯೋ! ಅದನ್ನು ಮಾಡದೆ ಇದ್ದಿದ್ದರೆ ಇಷ್ಟೊಂದು ಖರ್ಚನ್ನು ಉಳಿಸಬಹುದಾಗಿತ್ತಲ್ಲ ಎಂದು ಪೇಚಾಡುವುದಿಲ್ಲ. ಆತುರದಲ್ಲಿ ಉದ್ವೇಗಪರವಶನಾಗಿ ಒಂದು ಒಳ್ಳೆಯ ಕೆಲಸ ಮಾಡಿ, ಅನಂತರ ನಿಧಾನವಾಗಿ ಪೇಚಾಡುವವನಲ್ಲ. ಮಾಡುವುದಕ್ಕೆ ಮುಂಚೆಯೇ ಎಲ್ಲವನ್ನೂ ಪರಿಶೀಲಿಸಿ ಅನಂತರ ಕೆಲಸಕ್ಕೆ ಕೈ ಹಾಕುವನು. ಒಂದು ಸಲ ಕೆಲಸಕ್ಕೆ ಕೈ ಹಾಕಿದ ಮೇಲೆ ಅದನ್ನು ಅರ್ಧದಲ್ಲಿ ಅವನು ಬಿಡುವವನಲ್ಲ. ಕೊನೆಯ ತನಕ ಅದನ್ನು ಮುಂದು ವರಿಸುವನು. ದೃಢಚಿತ್ತನ ಸ್ವಭಾವ ಇದು.

ಅವನು ತಾನು ಮಾಡುವುದನ್ನು ಪರಮಶ್ರದ್ಧೆಯಿಂದ ಮಾಡುವನು. ಏನೊ ಮಾಡಬೇಕಲ್ಲ ಎಂದು ಕಾಟಾಚಾರಕ್ಕೆ ಮಾಡುವುದಿಲ್ಲ. ಫಲಾಪೇಕ್ಷೆ ಬಿಡಬೇಕು ಎಂದರೆ, ಬಂದರೆ ಬರಲಿ, ಇಲ್ಲದೇ ಇದ್ದರೆ ಇರಲಿ ಎಂದು ಬೇಕಾಬಿಟ್ಟಿಯಾಗಿ ಕೆಲಸಮಾಡುವುದಲ್ಲ. ಅವನು ಮಾಡುವ ಕೆಲಸದ ಹಿಂದೆ ಸಾಧಾರಣಶ್ರದ್ಧೆಯಲ್ಲ, ಪರಮಶ್ರದ್ಧೆ ಇರುವುದು. ನಾನು ಮಾಡುತ್ತಿರುವುದು ಒಂದು ಸೇವೆ. ಅದನ್ನು ಭಗವಂತನಿಗಾಗಿ ಮಾಡುತ್ತಿದ್ದೇನೆ. ಅಲ್ಲಿ ತೋರಿಕೆ ಇರಕೂಡದು, ಅಶ್ರದ್ಧೆ ಇರಕೂಡದು. ಅದನ್ನು ಒಂದು ಪೂಜೆಯಂತೆ ಮಾಡುವನು. ಅದೊಂದು ಅತ್ಯಂತ ಪವಿತ್ರವಾದ ಕೆಲಸ ಎಂದು ಭಾವಿಸುವನು. ಇದೊಂದು ಸಾಧನ, ಇದನ್ನು ಸರಿಯಾಗಿ ಮಾಡಿದರೆ ಉದ್ಧಾರವಾಗುವವನು ನಾನು, ಅದಕ್ಕಾಗಿ ದೇವರು ಒಂದು ಅವಕಾಶವನ್ನು ಕೊಟ್ಟಿರುವನು, ಅದನ್ನು ನಾನು ವ್ಯರ್ಥಮಾಡಿಕೊಳ್ಳಬಾರದು, ಎಂಬ ಭಾವನೆಗಳೆಲ್ಲ ಬೆರೆತಿದೆ ಪರಮಶ್ರದ್ಧೆ ಎಂಬ ಪದದಲ್ಲಿ.

\begin{verse}
ಸತ್ಕಾರಮಾನಪೂಜಾರ್ಥಂ ತಪೋ ದಂಭೇನ ಚೈವ ಯತ್~।\\ಕ್ರೀಯತೇ ತದಿಹ ಪ್ರೋಕ್ತಂ ರಾಜಸಂ ಚಲಮಧ್ರುವಮ್ \versenum{॥ ೧೮~॥}
\end{verse}

{\small ಸತ್ಕಾರ ಗೌರವ ಪೂಜೆಗಳಿಗಾಗಿ ಯಾವ ತಪಸ್ಸನ್ನು ತೋರಿಕೆಗಾಗಿ ಮಾಡುವರೊ ಅದು ಅಸ್ಥಿರ ಮತ್ತು ಅನಿತ್ಯವಾದುದು. ಅದನ್ನು ರಾಜಸಿಕ ತಪಸ್ಸು ಎಂದು ಹೇಳುತ್ತಾರೆ.}

ಸಾತ್ತ್ವಿಕ ತಪಸ್ಸಿನಲ್ಲಿ ಕಾಯಕ, ವಾಚಿಕ ಮತ್ತು ಮಾನಸಿಕ ತಪಸ್ಸುಗಳು ಮೂರು ವಿಧವೆಂದು ಹೇಳಿ ಆದಮೇಲೆ, ರಾಜಸಿಕ ತಪಸ್ಸನ್ನು ಹೇಳುವನು. ರಾಜಸಿಕನು, ಕಾಯಕ ತಪಸ್ಸನ್ನೆ ಮಾಡುತ್ತಿರಲಿ, ವಾಚಿಕ ತಪಸ್ಸನ್ನೆ ಮಾಡುತ್ತಿರಲಿ, ಮಾನಸಿಕ ತಪಸ್ಸನ್ನೆ ಮಾಡುತ್ತಿರಲಿ, ಅವನಲ್ಲಿ ಮುಖ್ಯವಾಗಿರು ವುದು ಢಂಬಾಚಾರ. ಇನ್ನೊಬ್ಬನಿಗೆ ತೋರಿಸಿಕೊಳ್ಳುವುದಕ್ಕಾಗಿ ಅದನ್ನು ಮಾಡುತ್ತಿರುವನು. ಅವನ ಭಕ್ತಿ ಅವನಿಗಲ್ಲ. ಅವನು ದೊಡ್ಡ ಭಕ್ತ ಎಂದು ಜನರಿಂದ ಹೇಳಿಸಿಕೊಳ್ಳಬೇಕು. ಅದಕ್ಕಾಗಿ ಭಕ್ತಿ ಪ್ರದರ್ಶನ ಮಾಡುವನು. ಅವನು ಏನು ಮಾಡಿದರೂ ಅದರ ಹಿಂದೆ ತಾನು ಪ್ರಖ್ಯಾತನಾಗಬೇಕು, ತನಗೆ ಅದರಿಂದ ಹೆಚ್ಚು ಪ್ರಯೋಜನ ಸಿಕ್ಕಬೇಕು ಎಂಬುದೊಂದೇ ಆದರ್ಶ.

ಇತರರಿಂದ ಹೊಗಳಿಸಿಕೊಳ್ಳುವುದಕ್ಕಾಗಿ ಅವನು ದಾನ, ತಪಸ್ಸು, ಕರ್ಮಗಳನ್ನು ಮಾಡುವನು. ಅವನಿಗೆ ಇತರರಿಂದ ಗೌರವ ಬೇಕು. ಇಂತಹ ದೊಡ್ಡ ಕೆಲಸವನ್ನು ಮಾಡಿದವನು ಇವನು ಎಂದು ಎಲ್ಲರ ಬಾಯಿಂದಲೂ ಹೊಗಳಿಸಿಕೊಳ್ಳಬೇಕು. ಅವರು ಇವನನ್ನು ಅಡ್ಡಪಲ್ಲಕ್ಕಿಯಲ್ಲಿ ಕುಳ್ಳಿರಿಸಿ ಕೊಂಡು ಮೆರವಣಿಗೆ ಮಾಡಿಸಬೇಕು. ಜನರು ಅಡ್ಡ ಬೀಳಬೇಕು. ಬಿನ್ನವತ್ತಳೆಗಳನ್ನು ಅರ್ಪಿಸಬೇಕು. ಆಗಲೆ ಇವನಿಗೆ ತೃಪ್ತಿ. ತಾನು ಮಾಡಿದ್ದು ಸಾರ್ಥಕವಾಯಿತು ಎಂದು ಭಾವಿಸುತ್ತಾನೆ.

ಆದರೆ ಈ ಪರಿಣಾಮ ಅಸ್ಥಿರ. ಇಂದು ಗೌರವದಿಂದ ನೋಡುವವರು ನಾಳೆ ಉದಾಸೀನರಾಗಿ ಕಾಣುವರು. ಅವರೇನು ನಮ್ಮ ಯೋಗ್ಯತೆಯನ್ನು ನೋಡಿ ಗೌರವ ಕೊಡುತ್ತಿಲ್ಲ. ನಮ್ಮ ಹೊರಗಿನ ವೇಷಕ್ಕಾಗಿ ಇದನ್ನೆಲ್ಲ ಮಾಡುತ್ತಿರುವರು. ಸ್ವಲ್ಪ ಕಾಲದ ಮೇಲೆ ಅವರಿಗೆ ಬುದ್ಧಿ ಬರುವುದು. ಮನುಷ್ಯ ಯಾವಾಗಲೂ ಮೂಢನಾಗಿರುವುದಿಲ್ಲ. ನಮ್ಮ ಬಂಡವಾಳ ಅವನಿಗೆ ಗೊತ್ತಾಗುವುದು. ಈಗ ಅವರು ನಮ್ಮನ್ನು ಪೂಜಿಸುತ್ತಿರಬಹುದು. ನಾಳೆ ನಾವು ಅವರ ಅನಾದರದ ಕಸದ ಬುಟ್ಟಿಯಲ್ಲಿ ಬೀಳುವೆವು.

ಇದು ಅನಿತ್ಯ, ಶಾಶ್ವತವಲ್ಲ. ನಮ್ಮ ಯೋಗ್ಯತೆಯ ಮೇಲೆ ನಾವು ನಿಂತುಕೊಳ್ಳಬೇಕು. ಇಲ್ಲದೇ ಇನ್ನೊಬ್ಬರು ನಮ್ಮನ್ನು ಮೇಲೆ ಎತ್ತಿ ಹಿಡಿದರೆ, ಅವರು ಕೈ ಬಿಟ್ಟೊಡನೆಯೆ ನಾವು ಕೆಳಗೆ ಉರುಳುತ್ತೇವೆ. ಆದರೆ ರಾಜಸಿಕ ನಿಂತಿರುವುದೇ ಇತರರ ಹೊಗಳಿಕೆಯ ಮೇಲೆ. ಅದೋ ಇಂದು ಇದ್ದಂತೆ ನಾಳೆ ಇರುವುದಿಲ್ಲ. ರಾಜಸಿಕನ ಬಾಳು ಜನರ ಗೌರವದ, ಅಭಿಪ್ರಾಯದ ಗಾಳಿಗೆ ಸಿಕ್ಕಿದ ತರಗೆಲೆಯಂತಾಗುವುದು.

\begin{verse}
ಮೂಢಗ್ರಾಹೇಣಾತ್ಮನೋ ಯತ್ಪೀಡಯಾ ಕ್ರಿಯತೇ ತಪಃ~।\\ಪರಸ್ಯೋತ್ಸಾದನಾರ್ಥಂ ವಾ ತತ್ತಾಮಸಮುದಾಹೃತಮ್ \versenum{॥ ೧೯~॥}
\end{verse}

{\small ಮೂಢಬುದ್ಧಿಯಿಂದ ತನಗೆ ಪೀಡೆಯನ್ನುಂಟು ಮಾಡಿಕೊಳ್ಳುವ ಅಥವಾ ಪರರನ್ನು ನಾಶ ಗೊಳಿಸುವ ಉದ್ದೇಶದಿಂದ ಮಾಡಿದ ತಪಸ್ಸನ್ನು ತಾಮಸಿಕ ತಪಸ್ಸು ಎಂದು ಹೇಳುತ್ತಾರೆ.}

ತಾಮಸ ತಪಸ್ಸಿನವನಿಗೆ ಬುದ್ಧಿ ಜಾಗತ್ರವಾಗಿಲ್ಲ. ಯಾವುದನ್ನು ಮಾಡಬೇಕು, ಯಾವುದನ್ನು ಬೀಡಬೇಕು ಎಂಬುದು ಗೊತ್ತಿಲ್ಲ. ಅನೇಕ ವೇಳೆ ಅದರಿಂದ ತನಗೆ ಕೇಡಾಗುವುದು. ತನಗೆ ಕೇಡನ್ನು ಉಂಟುಮಾಡಿಕೊಳ್ಳಬೇಕೆಂದು ಮಾಡುವುದಿಲ್ಲ. ಪರ್ಯಾಲೋಚನೆಯಿಲ್ಲದೆ ಮಾಡುವುದರಿಂದ ತನಗೇ ಕೆಟ್ಟದ್ದಾಗುವುದು. ಯಾವುದಕ್ಕೊ ಉಪವಾಸ ಮಾಡುವುದಕ್ಕೆ ಹೋಗುವನು. ಅವನಿಗೆ ಅಭ್ಯಾಸವಿಲ್ಲ. ಸ್ವಲ್ಪ ಕಾಲದ ಮೇಲೆ ಸುಸ್ತಾಗಿ ತಿಂದು ಬಿಡುವನು ಅಥವಾ ಕಬ್ಬಿಣದ ಮೊಳೆಯ ಹಾಸಿಗೆ ಮೇಲೆ ಕುಳಿತುಕೊಳ್ಳುವನು. ಇದರಿಂದ ತನ್ನ ದೇಹಕ್ಕೆ ಯಾತನೆ. ಯಾವ ಒಂದು ಪುರುಷಾರ್ಥವೂ ಲಭಿಸಲಿಲ್ಲ. ಅಥವಾ ಇನ್ನೊಬ್ಬನಿಗೆ ಕೇಡನ್ನು ಬಯಸುವುದಕ್ಕಾಗಿ ಯಾವುದೊ ಹೋಮವನ್ನೊ ಮಾಟವನ್ನೊ ಮಾಡಿಸುವನು. ಈ ಮನುಷ್ಯನ ತಪಸ್ಸಿನ ಹಿಂದೆ ತನ್ನ ಶ್ರೇಯಸ್ಸು ಇಲ್ಲ, ಪರರ ಶ್ರೇಯಸ್ಸು ಇಲ್ಲ.

\begin{verse}
ದಾತವ್ಯಮಿತಿ ಯದ್ದಾನಂ ದೀಯತೇಽನುಪಕಾರಿಣೇ~।\\ದೇಶೇ ಕಾಲೇ ಚ ಪಾತ್ರೇ ಚ ತದ್ದಾನಂ ಸಾತ್ತ್ವಿಕಂ ಸ್ಮೃತಮ್ \versenum{॥ ೨ಂ~॥}
\end{verse}

{\small ದಾನ ಮಾಡತಕ್ಕದ್ದು ಎಂದು ಯಾವ ದಾನ ಅನುಪಕಾರಿಯಾದವನಿಗೆ ದೇಶಕಾಲಪಾತ್ರಗಳಲ್ಲಿ ಕೊಡಲ್ಪಡು ವುದೊ ಆ ದಾನವನ್ನು ಸಾತ್ತ್ವಿಕ ಎಂದು ಹೇಳುತ್ತಾರೆ.}

ಸಾತ್ತ್ವ ಕ ದಾನ ಮಾಡುವವನು, ತಾನು ದಾನ ಮಾಡಬೇಕಾಗಿದೆ, ಅದು ತನ್ನ ಕರ್ತವ್ಯವೆಂದು ಭಾವಿಸುತ್ತಾನೆ. ಒಂದು ಕಡೆ ಸಂಗ್ರಹವಾಗುವುದು ಅನಂತರ ಇತರ ಕಡೆಗಳಿಗೆ ಹಂಚಿ ಹೋಗುವು ದಕ್ಕೆ. ಇದು ಪ್ರಕೃತಿಯ ನಿಯಮ. ಯಾವಾಗ ಸಂಗ್ರಹ ಮಾಡುವನೊ, ಆದರೆ ಇತರ ಕಡೆಗಳಿಗೆ ಹರಿದು ಹೋಗುವುದಕ್ಕೆ ಬಿಡುವುದಿಲ್ಲವೊ ಆಗ ಪ್ರಕೃತಿಗೆ ವಿರೋಧವಾಗಿ ಅವನು ವರ್ತಿಸುವನು. ಆಗ ಪ್ರಕೃತಿ ತನ್ನದೇ ರೀತಿಯಲ್ಲಿ ಅವನಿಂದ ಅಪಹರಿಸುವುದು. ಕಳ್ಳಕಾಕರಂತೆ ಬಂದು ತೆಗೆದು ಕೊಂಡು ಹೋಗುವುದು. ಸರ್ಕಾರ ಸುಂಕದಂತೆ ಬಂದು ಬಹುಭಾಗವನ್ನು ವಸೂಲಿ ಮಾಡುವುದು. ರೋಗರುಜಿನಗಳು, ಕೋರ್ಟು ಕಛೇರಿ ಇವುಗಳಿಗೆ ಅಲೆದು ಹಣ ಕಳೆದುಕೊಳ್ಳುವೆವು. ಅಥವಾ ಹಣವಿಟ್ಟ ಬ್ಯಾಂಕೇ ಪಾಪರಾಗಿ ಹೋಗಬಹುದು, ಹಣವಿಟ್ಟ ಮನೆಗೇ ಬೆಂಕಿ ಬೀಳಬಹುದು. ಯಾವಾಗ ನಾವು ಪ್ರಕೃತಿಗೆ ವಿರೋಧವಾಗಿ ಹೋಗುತ್ತೇವೊ ಆಗ ಪ್ರಕೃತಿ ತನ್ನ ಅಸ್ತ್ರಗಳನ್ನು ನಮ್ಮ ಮೇಲೆ ಹಲವು ರೀತಿ ಕರೆಯವುದು. ಆದಕಾರಣವೇ ಕೊಡುವುದು ನಮ್ಮ ಧರ್ಮ, ನಮ್ಮ ಕರ್ತವ್ಯ. ಇದರಿಂದ ನಾವೇನೂ ಮಹೋಪಕಾರ ಮಾಡುತ್ತಿಲ್ಲ. ನಾವೊಂದು ನಮ್ಮ ಪಾಲಿನ ಕರ್ತವ್ಯವನ್ನು ಮಾತ್ರ ನಿರ್ವಹಿಸುತ್ತಿದ್ದೇವೆ. ದಾನ ಮಾಡಿದರೆ ಹೊಗಳಿಕೊಳ್ಳುವುದೇನೂ ಇಲ್ಲ. ಮಾಡದೆ ಇದ್ದರೆ ತಪ್ಪಿತಸ್ಥರಾಗುತ್ತೇವೆ. ಈ ಮರ್ಮವನ್ನು ಚೆನ್ನಾಗಿ ಅರಿತವನು ಸಾತ್ತ್ವ ಕ ದಾನಿ.

ಅದನ್ನು ಕೊಡುವಾಗ ಅನುಪಕಾರಿಯಾದವನಿಗೆ ಕೊಡುತ್ತಾನೆ. ಹಿಂತಿರುಗಿ ಅವನು ನನಗೆ ಒಂದಲ್ಲ ಒಂದು ವಿಧದಲ್ಲಿ ಕೊಡಲಿ ಎಂದು ಕೊಡುವುದಿಲ್ಲ. ತನ್ನ ನೆಂಟರಿಷ್ಟರಿಗೇ ದಾನ ಮಾಡುವುದಿಲ್ಲ. ಅವನು ಹಿಂತಿರುಗಿ ಕೊಡುವುದಕ್ಕೆ ಆಗದವನಿಗೂ ಕೊಡುವನು. ಅವನಿಗೆ ಕೊಟ್ಟಿದ್ದು ಹೋಯಿತು, ಹಿಂತಿರುಗಿ ಬರುವಂತಿಲ್ಲ–ಇದನ್ನು ಚೆನ್ನಾಗಿ ಅರಿತೇ ಕೊಡುವನು. ನಾನು ಅವನಿಗೆ ಕೊಟ್ಟೆ, ಅವನು ನನಗೆ ಕೊಡಲಿಲ್ಲ ಎಂದು ಅವನು ಪರಿತಾಪವನ್ನು ಅನಂತರ ಪಡುವುದಿಲ್ಲ.

ದಾನವನ್ನು ಮಾಡುವಾಗ ದೇಶ ಕಾಲ, ಪಾತ್ರವನ್ನು ನೋಡಿ ಮಾಡಬೇಕು. ಯಾವ ದೇಶದಲ್ಲಿ ನಾವು ದಾನ ಮಾಡುತ್ತೇವೆಯೋ, ಅಲ್ಲಿ ಆತನಿಗೆ ಅತ್ಯಾವಶ್ಯಕವಾಗಿ ಬೇಕಾಗಿರುವ ವಸ್ತು ಆಗಬೇಕು. ತೀರ್ಥಸ್ಥಳಗಳಲ್ಲಿ ಮಾಡಿದ ದಾನದಿಂದ ಅತ್ಯಂತ ಹೆಚ್ಚು ಫಲ ಬರುತ್ತದೆ ಎಂದು ಹೇಳುತ್ತಾರೆ. ಅಂತಹ ತೀರ್ಥಸ್ಥಳಗಳಿಗೆ ಹೋದಾಗ ಯೋಗ್ಯರಾದ ವ್ಯಕ್ತಿಗಳಿಗೆ ಮತ್ತು ಉದ್ದೇಶಗಳಿಗೆ ದಾನ ಮಾಡಬೇಕು. ಅನೇಕ ಜನ ಯಾತ್ರಿಕರು ಬಂದು ಹೋಗುವ ಕಡೆ, ಯಾತ್ರಿಕರಿಗೆ ತಂಗುವ ಸ್ಥಳಗಳು, ಅವರಿಗೆ ಊಟ ಉಪಚಾರಗಳು, ಖಾಯಿಲೆ ಬಿದ್ದರೆ ಅವರಿಗೆ ಔಷಧಿ ಪಥ್ಯಗಳು–ಇವುಗಳಿಗಾಗಿ ನಮ್ಮ ಕೈಲಾದುದನ್ನು ಮಾಡಬೇಕು.

ಕೆಲವು ಕಾಲ ದಾನಕ್ಕೆ ಬಹಳ ಪ್ರಶಸ್ತ ಎಂದು ಹೇಳುವರು. ಕೆಲವು ಪುಣ್ಯದಿನಗಳು ಬರುತ್ತವೆ. ಆ ಸಮಯದಲ್ಲಿ ಯೋಗ್ಯವ್ಯಕ್ತಿಗಳಿಗೆ ದಾನ ಮಾಡಬೇಕು. ಸಂಕ್ರಾಂತಿ, ಉತ್ತರಾಯನ, ದಕ್ಷಿಣಾಯನ ಮುಂತಾದ ಪವಿತ್ರ ದಿನಗಳು ಅಂಥವು. ಅದಲ್ಲದೇ ಮದುವೆ, ಮುಂಜಿ ಮುಂತಾದುವನ್ನು ಮಾಡುವಾಗ ಯೋಗ್ಯಸಂಸ್ಥೆ ಮತ್ತು ವ್ಯಕ್ತಿಗಳಿಗೆ ದಾನ ಮಾಡಬೇಕು. ಕೆಲವು ವೇಳೆ ನಮ್ಮ ಹತ್ತಿರದ ನೆಂಟರಿಷ್ಟರು ತೀರಿಕೊಂಡಾಗ ಅವರಿಗೆ ಸದ್ಗತಿ ಸಿಕ್ಕಲೆಂದು ದಾನಧರ್ಮಗಳನ್ನು ಮಾಡಬೇಕು. ಅದರಿಂದ ಸಂಪಾದಿಸಿದ ಪುಣ್ಯ ಆ ಜೀವಿಯ ಉದ್ಧಾರಕ್ಕೆ ಹೋಗುವುದು.

ಪಾತ್ರವನ್ನು ನೋಡಿ ದಾನ ಮಾಡಬೇಕು. ಒಬ್ಬ ಯೋಗ್ಯನಾಗಿರಬೇಕು. ಉತ್ತಮ ಜೀವನ ನಡೆಸುತ್ತಿರಬೇಕು. ಅಂತಹ ವ್ಯಕ್ತಿಗೆ ಏನನ್ನಾದರೂ ಕೊಟ್ಟರೆ ಅವನು ಅದನ್ನು ಯಾವಾಗಲೂ ಒಳ್ಳೆಯದಕ್ಕೆ ಉಪಯೋಗಿಸುವನು ಅಥವಾ ಒಬ್ಬ ಅಂಗಹೀನನಾಗಿರಬಹುದು, ಬಡವನಾಗಿರ ಬಹುದು, ರೋಗಿಯಾಗಿರಬಹುದು, ಅಂತಹವರಿಗೆ ದಾನ ಮಾಡಬೇಕು. ಆಗ ಅದು ಸದ್ವಿನಿಯೋಗ ವಾಗುವುದು. ನಾವು ಮಾಡುವ ದಾನದಿಂದ ಒಳ್ಳೆಯದಾಗಿರಬೇಕು. ಅದರಿಂದ ಮತ್ತೊಬ್ಬರಿಗೆ ಸೋಮಾರಿತನ ಕಲಿಸಕೂಡದು. ಆ ಹಣದಿಂದ ಅವನು ಪಾಪಕಾರ್ಯಗಳನ್ನು ಮಾಡಕೂಡದು. ಹಾಗೇನಾದರೂ ಅವನು ದುರುಪಯೋಗಪಡಿಸಿಕೊಂಡರೆ ಅದರ ಒಂದು ಪಾಲು ದಾನ ಮಾಡಿದವ ನಿಗೂ ಹೋಗುವುದು. ಶ‍್ರೀರಾಮಕೃಷ್ಣರು ಈ ಒಂದು ಉದಾಹರಣೆಯನ್ನು ಕೊಡುತ್ತಾರೆ. ಒಬ್ಬ ಕಟುಕನಿಗೆ ಒಂದು ಹಸುವನ್ನು ಮಾರಲು ಹೋಗಿದ್ದ. ದಾರಿಯಲ್ಲಿ ಹಸಿವಾಯಿತು. ಯಾರದೋ ಮನೆಗೆ ಹೋಗಿ ಊಟ ಮಾಡಿ ಸ್ವಲ್ಪ ಸುಧಾರಿಸಿಕೊಂಡು ಶಕ್ತಿ ಬಂದ ಮೇಲೆ ಆ ಹಸುವನ್ನು ಹೊಡೆದುಕೊಂಡು ಹೋಗಿ ಕಟುಕನಿಗೆ ಕೊಟ್ಟ. ಅವನು ಅದನ್ನು ಕೊಂದ. ಇಲ್ಲಿ ಯಾವ ಮನೆಯವರು ಅವನಿಗೆ ಊಟ ಕೊಟ್ಟರೊ ಅವರಿಗೂ ಗೋಹತ್ಯದ ಒಂದು ಪಾಲು ಬಂತು ಎನ್ನುವರು. ಆದಕಾರಣ ನಾವು ಕೊಡುವುದರಲ್ಲಿ ತುಂಬಾ ಜೋಪನವಾಗಿರಬೇಕು.

\begin{verse}
ಯತ್ತು ಪ್ರತ್ಯುಪಕಾರಾರ್ಥಂ ಫಲಮುದ್ದಿಶ್ಯ ವಾ ಪುನಃ~।\\ದೀಯತೇ ಚ ಪರಿಕ್ಲಿಷ್ಟಂ ತದ್ರಾಜಸಮುದಾಹೃತಮ್ \versenum{॥ ೨೧~॥}
\end{verse}

{\small ಆದರೆ ಯಾವುದು ಪ್ರತ್ಯುಪಕಾರಕ್ಕಾಗಿ ಅಥವಾ ಪುನಃ ಫಲವನ್ನು ಉದ್ದೇಶಿಸಿ ಪರಿಕ್ಲೇಶದಿಂದ ಕೊಡಲ್ಪಡುವುದೊ ಅದನ್ನು ರಾಜಸಿಕ ದಾನವೆಂದು ಹೇಳುತ್ತಾರೆ.}

ರಾಜಸಿಕದಾನಿ ಸುಮ್ಮನೆ ಕೊಡುವುದರಲ್ಲಿ ನಂಬುವುದಿಲ್ಲ. ಅವನಿಗೆ ಕೊಡುವುದು ಒಂದು ಬ್ಯಾಂಕಿನಲ್ಲಿಟ್ಟಂತೆ. ಅದರಿಂದ ಬಡ್ಡಿ ಬರುತ್ತಿರಬೇಕು. ಆ ಬಡ್ಡಿಯೇ ಯಾರಿಗೆ ಇವನು ಉಪಕಾರ ಮಾಡಿದನೋ ಆತ ಇವನಿಗೆ ಪುನಃ ಉಪಕಾರ ಮಾಡಬೇಕೆಂದು ಆಶಿಸುವುದು. ನಾನು ಅವನಿಗೆ ಸಮಯದಲ್ಲಿ ಕೊಟ್ಟರೆ ಅವನು ನನಗೆ ಸಮಯದಲ್ಲಿ ಕೊಡಬೇಕು. ಇಲ್ಲಿ ದಾನ ಒಂದು ವ್ಯಾಪಾರ ಆಗುವುದು. ಅಥವಾ ಕೊಡುವುದರಿಂದ ಅವನಿಗೆ ಕೀರ್ತಿ ಬರಬೇಕು ಅಥವಾ ಮತ್ತಾವುದಾದರೂ ಸೌಲಭ್ಯಗಳು ಬರಬೇಕು. ಮತ್ತೆ ಕೆಲವು ವೇಳೆ ಗೋದಾನ ಮುಂತಾದ ದಾನಗಳನ್ನು ಕೆಲವು ಶ‍್ರೀಮಂತರು ಕೆಲವು ಪರ್ವಕಾಲದಲ್ಲಿ ಮಾಡುತ್ತಾರೆ. ಅದನ್ನು ಕೊಡುವುದಕ್ಕೆ ತಮ್ಮ ಹತ್ತಿರದ ಮಗಳು, ಅಳಿಯ ಇವರೇ ಪಾತ್ರಗಳು. ಇವರಿಗೆ ಸಾವಿರಾರು ರೂಪಾಯಿಗಳ ದಾನ ಹೋಗುವುದು. ಒಂದು ಬೀಸಣಿಗೆಯನ್ನೊ, ಪಂಚಪಾತ್ರೆ ಉದ್ಧರಣೆಯನ್ನೊ ಒಬ್ಬ ಬಡವನಿಗೆ ಕೊಡುವರು. ಇಲ್ಲಿ ಕೇವಲ ದಾನದ ಅಕ್ಷರವನ್ನು ಪಾಲಿಸುತ್ತಾರೆಯೆ ಹೊರತು ಅದರ ಉದ್ದೇಶವನ್ನಲ್ಲ.

ಕೆಲವು ವೇಳೆ ಕೊಡುವಾಗ ತುಂಬಾ ವ್ಯಥೆಪಟ್ಟುಕೊಂಡು ಕೊಡುತ್ತಾರೆ. ಅದನ್ನು ಕೊಡಬೇಕಲ್ಲ ಇಲ್ಲದಿದ್ದರೆ ಬಿಡುವುದಿಲ್ಲವಲ್ಲ ಎಂದು ಕೊಡುತ್ತಾರೆ. ಅಯ್ಯೋ ಶನಿ ತೊಲಗಿ ಹೋಗಲಿ ಎಂದು ಒಂದು ಕಾಟದಿಂದ ಪಾರಾಗುವುದಕ್ಕೆ ದಾನ ಮಾಡುತ್ತಾರೆ. ಇವೆಲ್ಲ ರಾಜಸಿಕ ದಾನ.

\begin{verse}
ಅದೇಶಕಾಲೇ ಯದ್ದಾನಮಪಾತ್ರೇಭ್ಯಶ್ಚ ದೀಯತೇ~।\\ಅಸಕ್ಕೃತಮವಜ್ಞಾತಂ ತತ್ತಾಮಸಮುದಾಹೃತಮ್ \versenum{॥ ೨೨~॥}
\end{verse}

{\small ಯಾವ ದಾನವನ್ನು ದೇಶಕಾಲ ಪಾತ್ರಗಳನ್ನು ಲೆಕ್ಕಿಸದೆ ಸತ್ಕಾರವಿಲ್ಲದೆ ತಿರಸ್ಕಾರದಿಂದ ಕೊಡಲ್ಪಡುವುದೋ ಅದನ್ನು ತಾಮಸಿಕ ಎಂದು ಹೇಳುತ್ತಾರೆ.}

ತಾಮಸಿಕ ದಾನದಲ್ಲಿ ಕೊಡುವವನು ದೇಶ ಕಾಲ ಪಾತ್ರವನ್ನು ಸರಿಯಾಗಿ ವಿಚಾರಿಸುವುದಿಲ್ಲ. ಅಪಾತ್ರರಿಗೇ ಕೊಡುತ್ತಾನೆ. ಯಾವ ಸ್ಥಳದಲ್ಲಾಗಲಿ ತನಗೆ ತೋಚಿದ್ದನ್ನು ಇತರರಿಗೆ ಕೊಡುತ್ತಾನೆ. ಬೇರೆ ದೇಶದಲ್ಲಿ ಬೇರೆ ಧರ್ಮ ಅಥವಾ ಸಂಸ್ಕೃತಿಗೆ ಸೇರಿದವರಿಗೆ ಅವರಿಗೆ ಉಪಯೋಗವಿಲ್ಲದು ದನ್ನು ಕೊಡುವನು. ಅವನು ಕಾಲವನ್ನು ಪರಿಗಣಿಸುವುದಿಲ್ಲ. ಈಗ ತಾನೆ ಊಟ ಮಾಡಿಕೊಂಡು ಬಂದಿರುವನು. ಅವನಿಗೆ ಊಟ ಕೊಡುತ್ತಾನೆ. ಆ ಕಾಲಕ್ಕೆ ಪ್ರಯೋಜನವಿಲ್ಲದುದನ್ನು ಕೊಡುತ್ತಾನೆ. ಅಪಾತ್ರರು ಮತ್ತು ಅಯೋಗ್ಯರು ಈ ತಾಮಸಿಕ ದಾನದ ಮೂಲಕ ಪ್ರಯೋಜನವನ್ನು ಹೊಂದತಕ್ಕ ವರು. ಸುಳ್ಳು ಹೇಳುವವರು, ಕಳ್ಳತನ ಮಾಡುವವರು, ಸಮಾಜ ಕಂಟಕರು ಇಂತಹ ವ್ಯಕ್ತಿಗಳಿಗೇ ಅವನು ದಾನ ಮಾಡುತ್ತಾನೆ.

ಅವನಿಗೆ ದಾನ ಮಾಡುವಾಗ ಸತ್ಕಾರ ಮಾಡುವುದಿಲ್ಲ. ಹಿಂದೂ ಸಂಸ್ಕೃತಿಯಲ್ಲಿ ಕೊಡುವವನಿ ಗಿಂತ ಸ್ವೀಕರಿಸುವವನು ಉತ್ತಮ ಸ್ಥಿತಿಯಲ್ಲಿರುವನು. ಆದಕಾರಣವೇ ಅತಿಥಿಯನ್ನು ದೇವ ರೆನ್ನುತ್ತಾರೆ. ಕರೆಸಿ ಬರುವವನಿಗಿಂತ ಕರೆಸಿಕೊಳ್ಳದೇ ಬರುವವನು ಸಾಕ್ಷಾತ್ ದೇವನೆಂದೆ ಭಾವಿಸಿ ಅವನನ್ನು ಗೌರವಿಸಿ ದಾನ ಮಾಡಬೇಕು. ಆದರೆ ತಾಮಸಿ ಅವನನ್ನು ಗೌರವಿಸುವುದಿಲ್ಲ. ಗೌರವಿಸದೇ ಇದ್ದರೆ ಉದಾಸೀನನಾಗಿ ಬೇಕಾದರೂ ಇರಬಹುದಲ್ಲ ಎಂದು ನಾವು ತಿಳಿಯಬಹುದು. ಆದರೆ ಇಲ್ಲಿ ಅವನು ತೆಗೆದುಕೊಳ್ಳುವವನನ್ನು ಅಸಡ್ಡೆಯಿಂದ ಕಾಣುವನು, ಅವನನ್ನು ಚುಚ್ಚುಮಾತಿನಿಂದ ನೋಯಿಸುವನು, ಅವನ ಮಾನ ಕಳೆಯುವನು. ಆತ್ಮ ಗೌರವವಿರುವ ಯಾರೂ ಇಂತಹ ದಾನವನ್ನು ಸ್ವೀಕರಿಸಲು ಆಗುವುದಿಲ್ಲ.

\begin{verse}
ಓಂ ತತ್ಸದಿತಿ ನಿರ್ದೇಶೋ ಬ್ರಹ್ಮಣಸ್ತ್ರಿವಿಧಃ ಸ್ಮೃತಃ~।\\ಬ್ರಾಹ್ಮಣಾಸ್ತೇನ ವೇದಾಶ್ಚ ಯಜ್ಞಾಶ್ಚ ವಿಹಿತಾಃ ಪುರಾ \versenum{॥ ೨೩~॥}
\end{verse}

{\small “ಓಂ ತತ್​ಸತ್​” ಎಂದು ಬ್ರಹ್ಮಕ್ಕೆ ಮೂರು ವಿಧವಾದ ನಿದರ್ಶನಗಳನ್ನು ಹೇಳುವರು. ಅದರ ಮೂಲಕ ಹಿಂದೆ ಬ್ರಾಹ್ಮಣರು ಯಜ್ಞಗಳು ವೇದಗಳು ವಿಹಿತವಾಗಿದೆ.}

ಬ್ರಹ್ಮನನ್ನು ಸೂಚಿಸುವುದಕ್ಕೆ ‘ಓಂ ತತ್ ಸತ್ ’ ಎಂಬ ಪದವನ್ನು ಉಪಯೋಗಿಸುತ್ತಾರೆ. ಇದೊಂದು ಭವ್ಯವಾದ ಭಗವಂತನ ಭಾವನೆ. ಇಲ್ಲಿ ಯಾವ ಆಕಾರವೂ, ರೂಪವೂ ಇಲ್ಲ, ಲಿಂಗವೂ ಇಲ್ಲ. ಓಂಕಾರ ಪರಮ ಪವಿತ್ರವೆಂದು ಹಿಂದಿನಿಂದಲೂ ಪರಿಗಣಿಸಲ್ಪಟ್ಟಿದೆ. ಓಂಕಾರವೇ ಶಬ್ದ ಬ್ರಹ್ಮ. ಅದರಿಂದ ನಾವು ನೋಡುವ ನಾಮರೂಪಗಳನ್ನು ವಿವರಿಸುವ ಶಬ್ದಗಳೆಲ್ಲವೂ ಆಗಿವೆ. ಅದೇ ನಾವು ನೋಡುವ ಅನುಭವಿಸುವ ಪ್ರಪಂಚವನ್ನೆಲ್ಲಾ ಆವರಿಸಿದೆ. ‘ತತ್​’ ಎಂದರೆ ಅದು. ಅದು ಎಂದರೆ ಅವ್ಯಕ್ತ. ನಿರ್ಗುಣವಾದುದೂ, ನಿರಾಕಾರವಾದುದೂ ಆದ ಬ್ರಹ್ಮನ ಅತೀತ ಅವಸ್ಥೆ. ಬ್ರಹ್ಮನ ಯಾವುದೋ ಒಂದು ಸಣ್ಣ ಅಂಶ ನಮ್ಮ ಪಂಚೇಂದ್ರಿಯಗಳಿಗೆ ದೊರಕುವ ಈ ಪ್ರಪಂಚವಾಗಿದೆ. ಸಾಗರದ ಯಾವುದೊ ಒಂದು ಅಲ್ಪ ಭಾಗ ತೇಲುವ ಹಿಮರಾಶಿ ಆಗಿದೆ. ಬಹುಭಾಗ ದ್ರವರೂಪ ದಲ್ಲಿದೆ. ಹಾಗೆಯೇ ಬ್ರಹ್ಮನ ಯಾವುದೋ ಅಲ್ಪ ಭಾಗ ದೇಶ ಕಾಲ ನಿಮಿತ್ತಕ್ಕೆ ಸಿಕ್ಕಿ ಈ ಪ್ರಪಂಚದಂತೆ ಕಾಣುತ್ತಿದೆ. ಬಹುಭಾಗ ಅದಕ್ಕೆ ಅತೀತವಾಗಿದೆ. ಅತೀತವಾಗಿರುವುದರಿಂದ ಇದು ಬಂದಿದೆ.

ಸತ್ ಎಂದರೆ ನಿಜವಾಗಿರುವುದು. ವ್ಯಕ್ತ, ಅವ್ಯಕ್ತ ಇವುಗಳ ಹಿಂದೆಲ್ಲ ಪರಬ್ರಹ್ಮನೇ ಇರುವುದು. ಈ ಪರಬ್ರಹ್ಮನೇ ಬ್ರಾಹ್ಮಣ, ಯಜ್ಞ, ವೇದ ಇವುಗಳನ್ನೆಲ್ಲಾ ಆವರಿಸಿರುವುದು.

ಬ್ರಾಹ್ಮಣ ಎಂದರೆ ಇಲ್ಲಿ ಅದನ್ನು ಕಂಡ ವ್ಯಕ್ತಿ. ಅವನನ್ನು ಮಂತ್ರದ್ರಷ್ಟ ಎನ್ನುವರು. ಜಗತ್ತಿಗೆ ಅದನ್ನು ಮೊದಲು ಕೊಟ್ಟವನು ಅವನು. ಅದನ್ನು ತಯಾರು ಮಾಡಿದವನಲ್ಲ. ಅವನು ತನ್ನ ಇಂದ್ರಿಯ ಮನಸ್ಸುಗಳನ್ನು ನಿಗ್ರಹಿಸಿ ಈ ಅನುಭವಗಳನ್ನು ಗಳಿಸಿದ. ಹೇಗೆ ರೇಡಿಯೋವನ್ನು ಟ್ಯೂನ್ ಮಾಡಿದಾಗ ಸಂಗೀತ ಕೇಳುವುದೊ ಹಾಗೆ ಮಂತ್ರದ್ರಷ್ಟ ಇದನ್ನು ಅನುಭವಿಸಿ ವಿವರಿಸುತ್ತಾನೆ.

ಯಾಗ ಯಜ್ಞ ಪೂಜೆ ವ್ರತ ಹಲವು ಆಚಾರಗಳು ಇವುಗಳೆಲ್ಲಾ ಆ ಪರಬ್ರಹ್ಮನನ್ನು ಉಪಾಸನೆ ಮಾಡುವುದಕ್ಕಾಗಿ ಬಂದಿರುವುವು. ವೇದ ಅದನ್ನು ವಿವರಿಸುವ ಗ್ರಂಥ. ಈ ಮೂರು ಕೂಡ ಆ ಪರಬ್ರಹ್ಮನನ್ನು ತೋರುವ ಕೈಮರದಂತಿವೆ. ಬ್ರಾಹ್ಮಣ ಅದನ್ನು ಅನುಭವಿಸುವ ವ್ಯಕ್ತಿ. ಯಜ್ಞವೆಂದರೆ ಅದರ ಉಪಾಸನೆಯ ಭಾಗ, ಅದನ್ನು ಪಡೆಯುವ ವಿಧಾನ. ವೇದ ಅದನ್ನು ವಿವರಿಸುವುದು.

\begin{verse}
ತಸ್ಮಾದೋಮಿತ್ಯುದಾಹೃತ್ಯ ಯಜ್ಞದಾನತಪಃಕ್ರಿಯಾಃ~।\\ಪ್ರವರ್ತಂತೇ ವಿಧಾನೋಕ್ತಾಃ ಸತತಂ ಬ್ರಹ್ಮವಾದಿನಾಮ್ \versenum{॥ ೨೪~॥}
\end{verse}

{\small ಆದುದರಿಂದ ಬ್ರಹ್ಮವಾದಿಗಳು ಓಂ ಎಂದು ಉಚ್ಚರಿಸುತ್ತಾ ಶಾಸ್ತ್ರ ಸಮ್ಮತವಾದ ಯಜ್ಞ ದಾನ ತಪಸ್ಸು ಕ್ರಿಯೆಗಳನ್ನು ಯಾವಾಗಲೂ ಪ್ರಾರಂಭಿಸುತ್ತಾರೆ.}

ನಾವು ದೇವರಿಗೆ ಏನನ್ನು ಅರ್ಪಣೆಮಾಡಬೇಕಾದರೂ ಮುಂಚೆ ಅದರ ಮೇಲೆ ಮಂತ್ರಜಲವನ್ನು ಪ್ರೋಕ್ಷಣೆಮಾಡಿ, ಪವಿತ್ರಮಾಡಿ, ಅನಂತರ ಅರ್ಪಣೆ ಮಾಡುತ್ತೇವೆ, ಅದರಲ್ಲಿ ಯಾವ ವಿಧವಾದ ದೋಷವಿದ್ದರೂ ಅದು ನಿವಾರಣೆಯಾಗಲಿ ಎಂದು. ಅದರಂತೆ ಯಾರು ಬ್ರಹ್ಮನಲ್ಲಿ ನಂಬುವರೋ, ಅದೇ ಪರಮ ಸತ್ಯವೆಂದು ನಂಬುವರೋ ಅವರು ಆ ಬ್ರಹ್ಮಕ್ಕೆ ಚಿಹ್ನೆಯಾಗಿರುವ ಓಂಕಾರವನ್ನು ಉಚ್ಚರಿಸಿ ತಾವು ಮಾಡಬೇಕಾದ ಕ್ರಿಯೆಗಳನ್ನು ಮಾಡುತ್ತಾರೆ. ಓಂಕಾರ ನಮ್ಮ ಹಿಂದೂ ಧರ್ಮದಲ್ಲಿ ಅತ್ಯಂತ ಪವಿತ್ರವಾದ ಚಿಹ್ನೆ. ಇದನ್ನೇ ಸಂಕ್ಷಿಪ್ತ ಬ್ರಹ್ಮವೆಂದು ಹೇಳುತ್ತಾರೆ. ಇದರಲ್ಲಿ ಎಲ್ಲಾ ಶಾಸ್ತ್ರಗಳ ಸಾರವೂ ದೇವದೇವಿಯರ ತತ್ತ್ವಗಳೂ ಸಂಗಮವಾಗಿವೆ ಎಂದು ಭಾವಿಸುತ್ತಾರೆ. ದೇವದೇವಿಯರ ಹೆಸರು ಎಲ್ಲರಿಗೂ ಅನ್ವಯಿಸದೇ ಇರಬಹುದು. ಶೈವರಿಗೆ ವಿಷ್ಣು ಹಿಡಿಸುವುದಿಲ್ಲ, ವೈಷ್ಣವರಿಗೆ ಶಿವ ಹಿಡಿಸುವುದಿಲ್ಲ. ಹಾಗೆಯೇ ದೇವಿ ಉಪಾಸಕರು ಉಳಿದವರೆಲ್ಲಾ ಅವಳ ಆಜ್ಞಾಧಾರಕ ರೆಂದು ಭಾವಿಸುತ್ತಾರೆ. ಆದರೆ ಪ್ರಣವವನ್ನು ಎಲ್ಲರೂ ಒಪ್ಪುತ್ತಾರೆ. ಇದರಷ್ಟು ಸರ್ವವ್ಯಾಪಿಯಾದ ಚಿಹ್ನೆ ಮತ್ತೊಂದಿಲ್ಲ. 

ಇದನ್ನು ಮಾಡುವವರು ಬ್ರಹ್ಮವಾದಿಗಳು ಎಂದರೆ ಬ್ರಹ್ಮನನ್ನು ನಂಬುವವರು, ಅದನ್ನು ಅನುಭವಿಸಿದವರು, ಮತ್ತು ಅದನ್ನು ಜಿಜ್ಞಾಸೆ ಮಾಡುವವರು, ಇತರರಿಗೆ ಅದನ್ನು ಬೋಧಿಸುವವರು. ಅವರು ಎಂದಿಗೂ ಶಾಸ್ತ್ರಕ್ಕೆ ವಿರುದ್ಧವಾದ ಕ್ರಿಯೆಯನ್ನು ಮಾಡುವುದಿಲ್ಲ. ಶಾಸ್ತ್ರಕ್ಕೆ ಸಂಬಂಧಿಸಿದ ಕ್ರಿಯೆಯನ್ನು ಮಾಡುವಾಗಲೂ ಕೇವಲ ಅದರ ಅಕ್ಷರವನ್ನು ಪಾಲಿಸುವುದಿಲ್ಲ. ಅಕ್ಷರಕ್ಕಿಂತ ಹೆಚ್ಚಾಗಿ ಅದರಲ್ಲಿರುವ ಭಾವವನ್ನು ಅನುಸರಿಸುವರು. ಅನೇಕರು ಶಾಸ್ತ್ರದ ಹೆಸರನ್ನು ಹೇಳುತ್ತಾರೆ. ಮಾತೆತ್ತಿ ದ್ದಕ್ಕೆ ಶ್ಲೋಕಗಳನ್ನು ಉದಾಹರಿಸುವರು. ಆದರೆ ಶಾಸ್ತ್ರದ ಸಾರವನ್ನು ಬಲ್ಲವರೆಲ್ಲೊ ಕೆಲವರು. ಮುಕ್ಕಾಲುಪಾಲು ಜನ ಶಾಸ್ತ್ರದ ಕರಟವನ್ನು ಹಿಡಿದಿರುವರೆ ವಿನಃ ಶಾಸ್ತ್ರದ ತಿರುಳಲ್ಲ ಅವರ ಕೈಯಲ್ಲಿರುವುದು. ಆದರೆ ನಿಜವಾದ ಬ್ರಹ್ಮವಾದಿಗೆ ಶಾಸ್ತ್ರದ ಸಾರ ಗೊತ್ತಿದೆ. ಅದರ ಭಾವ ಗೊತ್ತಿದೆ. ಅವನೇನು ಮಾಡಿದರೂ ಶಾಸ್ತ್ರಕ್ಕೆ ವಿರೋಧವಾಗಿ ಹೋಗುವುದಿಲ್ಲ. ಅದಕ್ಕೆ ಶಾಸ್ತ್ರದ ಸಮ್ಮತವಿದೆ.

\begin{verse}
ತದಿತ್ಯನಭಿಸಂಧಾಯ ಫಲಂ ಯಜ್ಞ ತಪಃಕ್ರಿಯಾಃ~।\\ದಾನಕ್ರಿಯಾಶ್ಚ ವಿವಿಧಾಃ ಕ್ರಿಯಂತೇ ಮೋಕ್ಷಕಾಂಕ್ಷಿಭಿಃ \versenum{॥ ೨೫~॥}
\end{verse}

{\small ಮೋಕ್ಷಾಕಾಂಕ್ಷಿಗಳು ತತ್ ಎಂದು ಉಚ್ಚರಿಸಿ ಫಲವನ್ನು ಅಪೇಕ್ಷಿಸದೆ ಹಲವು ಬಗೆಯ ಯಜ್ಞ, ತಪಸ್ಸು ದಾನಕ್ರಿಯೆಗಳನ್ನು ಮಾಡುತ್ತಾರೆ.}

ಮಾಡುವ ಕರ್ಮದ ಮೂಲಕ ಮುಕ್ತಿಯನ್ನು ಪಡೆಯಬಹುದು. ಹಾಗೆ ಪಡೆಯಬೇಕಾದರೆ ತತ್ ಎಂದರೆ ಭಗನಂತನಿಗೆ ಅರ್ಪಿತವಾಗಲಿ ಎಂದು ಮಾಡಬೇಕು. ಅದರಿಂದ ಬರುವ ಫಲಗಳ ಮೇಲೆ ಯಾವ ಆಸೆಯೂ ಇರಕೂಡದು.

ನಾವು ಮಾಡುವ ಯಜ್ಞಗಳನ್ನೆಲ್ಲಾ ಅವನಿಗೆ ಮಾಡಬಹುದು. ನಾವು ಹಿಂದಿನ ಅಧ್ಯಾಯದಲ್ಲಿ ಓದಿದ್ದೇವೆ ಹಲವಾರು ಬಗೆಯ ಯಜ್ಞಗಳಿವೆ ಎಂಬುದನ್ನು. ನಾಲ್ಕನೇ ಅಧ್ಯಾಯದ ೨೮ ನೇ ಶ್ಲೋಕದಲ್ಲಿ ಹೀಗೆ ಹೇಳಿದೆ. “ಕೆಲವರು ದ್ರವ್ಯಯಜ್ಞ ಮಾಡುತ್ತಾರೆ. ಕೆಲವರು ತಪೋಯಜ್ಞ ಮಾಡುತ್ತಾರೆ. ಕೆಲವರು ಯಾಗಯಜ್ಞವನ್ನು ಮಾಡುತ್ತಾರೆ. ಕೆಲವರು ಸ್ವಾಧ್ಯಾಯಯಜ್ಞ ಮಾಡು ತ್ತಾರೆ. ಕೆಲವರು ಜ್ಞಾನಯಜ್ಞ ಮಾಡುತ್ತಾರೆ.” ನಾವು ಇವುಗಳನ್ನು ಸ್ವಾರ್ಥದ ದೃಷ್ಟಿಯಿಂದ ಬೇಕಾದರೂ ಮಾಡಬಹುದು. ನಿಸ್ವಾರ್ಥದೃಷ್ಟಿಯಿಂದ ಬೇಕಾದರೂ ಮಾಡಬಹುದು. ಸ್ವಾರ್ಥದೃಷ್ಟಿ ಯಿಂದ ಮಾಡಿದರೆ ಇದು ನಮ್ಮನ್ನು ಪ್ರಪಂಚಕ್ಕೆ ಹೆಚ್ಚು ಬಂಧಿಸುವುದು. ನಿಃಸ್ವಾರ್ಥದೃಷ್ಟಿಯಿಂದ ಇದರಿಂದ ಬರುವುದೆಲ್ಲಾ ಭಗವಂತನಿಗೇ ಅರ್ಪಿತವಾಗಲೆಂದು ಮಾಡಿದರೆ, ನಮ್ಮ ಚಿತ್ತ ಶುದ್ಧವಾಗಿ ನಾವು ಮುಕ್ತರಾಗುತ್ತೇವೆ.

ಅದರಂತೆಯೇ ನಾವು ತಪಸ್ಸನ್ನು ಕೂಡ ಸ್ವಾರ್ಥ ದೃಷ್ಟಿಯಿಂದ ಬೇಕಾದರೂ ಮಾಡಬಹುದು. ತಪಸ್ಸಿನಿಂದ ಶಕ್ತಿ ಪಡೆಯುತ್ತೇವೆ. ಅದನ್ನು ನಾವು ಯಾವುದಕ್ಕೆ ಬೇಕಾದರೂ ಉಪಯೋಗಿಸಬಹುದು. ಬರೀ ತಪಸ್ಸೇ ದೊಡ್ಡದಲ್ಲ. ಆ ತಪಸ್ಸಿನಿಂದ ಸಾಧಿಸಿದ್ದೇನು? ನಾವು ಅದನ್ನು ಕುರಿತು ಯೋಚಿಸ ಬೇಕಾಗಿದೆ. ಶ‍್ರೀರಾಮಕೃಷ್ಣರು ಹೇಳುವಂತೆ ಒಬ್ಬ ಹನ್ನೆರಡು ವರ್ಷ ತಪಸ್ಸು ಮಾಡಿ ನೀರಿನ ಮೇಲೆ ನಡೆಯುವುದನ್ನು ಪಡೆದ. ಒಬ್ಬ ಜ್ಞಾನಿ ಹೇಳಿದ: “ಇದಕ್ಕೆ ಹನ್ನೆರಡು ವರುಷ ತಪಸ್ಸು ಏಕೆ ನಷ್ಟ ಮಾಡಿಕೊಂಡೆ. ದೋಣಿಯವನಿಗೆ ಒಂದು ಆಣೆ ಕೊಟ್ಟರೆ ಆ ಕೆಲಸ ಮಾಡುತ್ತಾನೆ.” ರಾವಣ, ಇಂದ್ರಜಿತು, ಹಿರಣ್ಯಾಕ್ಷ, ಹಿರಣ್ಯಕಶಿಪುಗಳು ಉಗ್ರವಾಗಿ ತಪಸ್ಸು ಮಾಡಿದರೂ ಅದರಿಂದ ಗಳಿಸಿ ದ್ದೇನು? ಲೋಕಕಂಟರಾಗುವುದನ್ನು. ತಪಸ್ಸು ಎಂಬುದು ಬೆಂಕಿಯಂತೆ. ಅದರಿಂದ ಅಡಿಗೆ ಮಾಡಿ ಕೊಳ್ಳಬಹುದು. ಅಥವಾ ಒಂದು ಮನೆಯನ್ನು ಸುಟ್ಟುಹಾಕಬಹುದು. ಅದು ಒಳ್ಳೆಯ ತಪಸ್ಸಾದರೆ ಅದರಿಂದ ಲೋಕಕಲ್ಯಾಣ, ಆತ್ಮಕಲ್ಯಾಣ ಎರಡೂ ಆಗಬಹುದು. ಕೆಟ್ಟ ತಪಸ್ಸಾದರೆ ಲೋಕಹಾನಿ, ಆತ್ಮಹಾನಿ ಎರಡೂ ಆಗಬಹುದು. ಈ ತಪಸ್ಸು ಕಾಯಕ ವಾಚಿಕ ಮತ್ತು ಮಾನಸಿಕ ಎಂದು ಮೂರು ಬಗೆಯಿದೆ ಎಂದು ಹೇಳಿ ಅದನ್ನು ಯಾವಾಗ ಸಾತ್ತ್ವಿಕ ದೃಷ್ಟಿಯಿಂದ ಮಾಡುತ್ತಾನೋ ಆಗ ಅವನು ಉದ್ಧಾರವಾಗುತ್ತಾನೆ ಎಂದು ಶ‍್ರೀಕೃಷ್ಣ ಹಿಂದೆಯೆ ಸಾರಿದ. ಇಲ್ಲಿ ತತ್ ಎಂದರೆ ಆ ಸಾತ್ತ್ವಿಕ ದೃಷ್ಟಿಗೆ ಅನ್ವಯಿಸುವುದು.

ಅದರಂತೆಯೇ ಹಲವು ಬಗೆಯ ದಾನಗಳಿವೆ. ದೇಶ ಕಾಲ ಪಾತ್ರವನ್ನು ನೋಡಿ ದಾನ ಮಾಡುವಾಗಲೂ ತತ್ ಎಂದು ಉಪಯೋಗಿಸುತ್ತಾರೆ. ದಾನ ಮಾಡುವಾಗ ನಿಜವಾಗಿ ನಾವು ಕೊಡು ವುದು ಭಗವಂತನಿಗೆ. ಎದುರಿಗೆ ನಿಂತಿರುವ ವ್ಯಕ್ತಿಗಳು ಕೇವಲ ನಿಮಿತ್ತ ಮಾತ್ರ. ಸಾಕ್ಷಾತ್ ಭಗವಂತನಿಗೆ ಕೊಡುತ್ತೇವೆ ಎಂಬ ಭಾವನೆ ಇರಬೇಕು. ಯಾವಾಗ ನಾವು ಅವನಿಗೆ ಕೊಡುತ್ತೇವೆಯೋ ಅವನಿಂದ ಯಾವ ಪ್ರತಿಫಲವನ್ನು ಇಚ್ಛಿಸುವುದಿಲ್ಲ. ಸಾತ್ತಿ ್ವಕ ಮನುಷ್ಯ ಇದೆಲ್ಲ ಅವನದು, ಅವನದನ್ನು ಅವನಿಗೆ ಕೊಡುತ್ತೇವೆ ಎಂದು ಭಾವಿಸುವನು.

\begin{verse}
ಸದ್ಭಾವೇ ಸಾಧುಭಾವೇ ಚ ಸದಿತ್ಯೇತತ್ಪ್ರಯುಜ್ಯತೇ~।\\ಪ್ರಶಸ್ತೇ ಕರ್ಮಣಿ ತಥಾ ಸಚ್ಛಬ್ದಃ ಪಾರ್ಥ ಯುಜ್ಯತೇ \versenum{॥ ೨೬~॥}
\end{verse}

{\small ಅರ್ಜುನ, ಸದ್ ಭಾವದಲ್ಲಿಯೂ ‘ಸತ್​’ ಎಂಬುದನ್ನು ಉಪಯೋಗಿಸುತ್ತಾರೆ. ಹಾಗೆಯೆ ಶುಭ ಕಾರ್ಯವನ್ನು ಮಾಡುವಾಗಲೂ ‘ಸತ್​’ ಎಂಬುದನ್ನು ಉಪಯೋಗಿಸುತ್ತಾರೆ.}

ಸತ್ ಭಾವವನ್ನು ತಾತ್ವಿಕ ದೃಷ್ಟಿಯಿಂದ ಉಪಯೋಗಿಸುತ್ತಾರೆ. ತಾತ್ವಿಕ ದೃಷ್ಟಿಯಿಂದ ಈ ಪ್ರಪಂಚದಲ್ಲಿ ಸತ್ಯ ಯಾವುದು ಎಂಬುದನ್ನು ತಿಳಿದುಕೊಳ್ಳುವುದು. ಭಗವಂತನೊಬ್ಬನೇ ಸತ್ಯ. ಉಳಿದುವುಗಳೆಲ್ಲಾ ಅವನ ಆಧಾರದಿಂದ ಮಾತ್ರ ಸತ್ಯ. ಸಾಗರ ಒಂದೇ ಸತ್ಯ. ಅದರ ಅಲೆ ತೆರೆ ನೊರೆ ಗುಳ್ಳೆ ಇವುಗಳೆಲ್ಲಾ ಈಗ ಇದ್ದುದು ಇನ್ನೊಂದು ಸ್ವಲ್ಪ ಹೊತ್ತಿನಲ್ಲಿ ಇರುವುದಿಲ್ಲ. ಭಗವಂತನ ಆಶ್ರಯವಿದ್ದರೆ ಮಾತ್ರ ಇವುಗಳಿಗೆಲ್ಲಾ ಸತ್ಯತ್ವ ಬರುವುದು.

ಸತ್ ಎಂಬ ಪದವನ್ನು ನೈತಿಕ ದೃಷ್ಟಿಯಿಂದ ಉಪಯೋಗಿಸುತ್ತಾರೆ. ಯಾವುದು ಸಾಧು ಭಾವವೋ, ಯಾವುದು ನಮ್ಮನ್ನು ಭಗವಂತನ ಕಡೆಗೆ ಕರೆದುಕೊಂಡು ಹೋಗುವುದೋ, ಪ್ರಪಂಚದ ಗೋಜಿನಿಂದ ಬಿಡಿಸುವುದೊ, ಅಂತಹ ಉತ್ತಮ ಕ್ರಿಯೆಗಳಿಗೆ ಸತ್ ಎಂದು ಉಪಯೋಗಿಸುತ್ತಾರೆ. ಇದೊಂದೇ ಸಾರ್ಥಕವಾದ ಕ್ರಿಯೆ. ಉಳಿದುವುಗಳೆಲ್ಲಾ ವ್ಯರ್ಥ ಕ್ರಿಯೆ, ಪ್ರಪಂಚಕ್ಕೆ ನಮ್ಮ ಪುಣವನ್ನು ಹೆಚ್ಚಿಸುವ ಕ್ರಿಯೆಗಳು. ನಿಃಸ್ವಾರ್ಥತೆಯಿಂದ ಭಗವದರ್ಪಿತವಾಗಲಿ ಎಂದು ಮಾಡಿದ ಕೆಲಸ ಸತ್ ಕೆಲಸ.

ಅದರಂತೆಯೇ ಒಳ್ಳೆಯ ಕರ್ಮಗಳನ್ನು ಮಾಡುವಾಗಲೂ ಸತ್ ಎಂಬುದನ್ನು ಉಪಯೋಗಿಸು ತ್ತಾರೆ. ಮಗುವಿಗೆ ಅನ್ನಪ್ರಾಶನಮಾಡುವಾಗ ದೇವರ ಪ್ರಸಾದವನ್ನು ಅದರ ಬಾಯಲ್ಲಿ ಮೊದಲಿಡು ತ್ತೇವೆ. ಅಕ್ಷರಾಭ್ಯಾಸ ಮಾಡಿಸುವಾಗ ಅದರ ಕೈಯಿಂದ ದೇವರ ಹೆಸರನ್ನು ಮೊದಲು ಬರೆಸುತ್ತೇವೆ. ಉಪನಯನ ಸಮಯದಲ್ಲಿ ಗಾಯತ್ರಿ ಮಂತ್ರವನ್ನು ಉಪದೇಶಿಸುತ್ತೇವೆ. ಆತ ಯುವಕನಾಗಿ ಮದುವೆಯಾಗುವಾಗಲೂ ಇದೊಂದು ಪವಿತ್ರ ವ್ರತ. ಭಗವಂತನ ಕಡೆಗೆ ಹೋಗುವಾಗ ಸತಿಪತಿ ಯರು ಒಬ್ಬರು ಮತ್ತೊಬ್ಬರಿಗೆ ಅನುಕೂಲವಾಗಲಿ ಎಂದು ಬೋಧಿಸುತ್ತೇವೆ. ಗೃಹಪ್ರವೇಶಮಾಡು ವಾಗ ಮೊದಲು ದೇವರನ್ನು ತೆಗೆದುಕೊಂಡು ಹೋಗುತ್ತೇವೆ. ಯಾವ ಒಂದು ಒಳ್ಳೆಯ ಕೆಲಸವನ್ನು ಮಾಡುವಾಗಲೂ ಹಿಂದೆ ಸತ್ ಎಂಬ ಪದವನ್ನು ಉಪಯೋಗಿಸುತ್ತೇವೆ. ಸತ್ ಎಂಬ ಪದ ನಮ್ಮ ಬಾಳಿನಲ್ಲಿ ಓತಪ್ರೋತವಾಗಿ ಆವರಿಸಿದೆ.

\begin{verse}
ಯಜ್ಞೇ ತಪಸಿ ದಾನೇ ಚ ಸ್ಥಿತಿಃ ಸದಿತಿ ಚೋಚ್ಯತೇ~।\\ಕರ್ಮ ಚೈವ ತದರ್ಥೀಯಂ ಸದಿತ್ಯೇವಾಭಿಧೀಯತೇ \versenum{॥ ೨೭~॥}
\end{verse}

{\small ಯಜ್ಞ ತಪಸ್ಸು ದಾನಗಳಲ್ಲಿ ದೃಢತೆಗೂ ಸತ್ ಎನ್ನುತ್ತಾರೆ. ಅದಕ್ಕಾಗಿ ಮಾಡುವ ಕರ್ಮವೂ ಸತ್ ಎಂದೇ ಹೇಳಲ್ಪಡುತ್ತದೆ.}

ಕೆಲವು ವೇಳೆ ಮನಸ್ಸು ಉತ್ತಮ ಮಟ್ಟದಲ್ಲಿದ್ದಾಗ ಒಳ್ಳೆಯ ದೃಷ್ಟಿಯಿಂದ ಯಜ್ಞ ತಪಸ್ಸು ದಾನ ಮುಂತಾದುವುಗಳನ್ನು ಮಾಡುತ್ತಾನೆ. ಆದರೆ ಮನಸ್ಸು ದೃಢವಾಗಿಲ್ಲ. ಪುನಃ ಕೆಳಮಟ್ಟಕ್ಕೆ ಬರುವುದು. ಆಗ ಏತಕ್ಕಾದರೂ ಅದನ್ನು ಹಾಗೆ ಮಾಡಿದೆನೊ ಎಂದು ಪರಿತಪಿಸುವನು. ಆದರೆ ಇಲ್ಲಿ ಸತ್ ಎಂಬುದನ್ನು ಉಪಯೋಗಿಸುವುದು ಹಾಗೆ ಕೆಳಮಟ್ಟಕ್ಕೆ ಬರದೆ ಯಾವಾಗಲೂ ಸಾತ್ವಿಕದ ಮೇಲೆಯೇ ವಿಹರಿಸುತ್ತಿರುವ ಸ್ಥಿತಿಗೆ ಸಂಬಂಧಿಸಿದಂತೆ. ಅವನು ಸಾತ್ವಿಕದೃಷ್ಟಿಯಿಂದ ಎಂದೂ ಕೆಳಕ್ಕೆ ಬೀಳುವು ದಿಲ್ಲ. ಯಾವಾಗಲೂ ಸಾತ್ವಿಕ ಪ್ರಪಂಚದಲ್ಲಿಯೇ ವಿಹರಿಸುತ್ತಿರುವನು. ಅವನು ಮಾಡುವ ಯಜ್ಞ ದಾನ ತಪಸ್ಸು ಮುಂತಾದುವುಗಳಿಗೆ ಎಂದೂ ರಜೋಗುಣ ತಮೋಗುಣ ಪ್ರವೇಶ ಮಾಡದೆ ಇರುವ ಸ್ಥಿತಿಗೆ ಸತ್ ಎಂದು ಹೆಸರು.

ಅದಕ್ಕಾಗಿ ಮಾಡುವ ಕರ್ಮವೂ ಸತ್. ಅದಕ್ಕೆ ಎಂದರೆ ಭಗವಂತನಿಗೆ ಮಾಡಿದ್ದೆಲ್ಲಾ ಸತ್ ಆಗುವುದು. ಅದೆಲ್ಲ ಒಂದು ಪೂಜೆಯಾಗುವುದು. ನಮ್ಮನ್ನು ಉದ್ಧಾರ ಮಾಡುವ ಶಕ್ತಿಯಾಗು ವುದು.

\begin{verse}
ಅಶ್ರದ್ಧಯಾ ಹುತಂ ದತ್ತಂ ತಪಸ್ತಪ್ತಂ ಕೃತಂ ಚ ಯತ್~।\\ಅಸದಿತ್ಯುಚ್ಯತೇ ಪಾರ್ಥ ನ ಚ ತತ್ಪ್ರೇತ್ಯ ನೋ ಇಹ \versenum{॥ ೨೮~॥}
\end{verse}

{\small ಅರ್ಜುನ, ಅಶ್ರದ್ಧೆಯಿಂದ ಮಾಡಿದ ಹೋಮ ದಾನ ತಪಸ್ಸು ಕರ್ಮ ಇವುಗಳನ್ನೆಲ್ಲಾ ಅಸತ್ ಎಂದು ಹೇಳುತ್ತಾರೆ. ಅದರಿಂದ ಸತ್ತ ಮೇಲೆಯೂ ಸುಖವಿಲ್ಲ, ಇರುವಾಗಲೂ ಸುಖವಿಲ್ಲ.}

ಯಾರು ಅಶ್ರದ್ಧೆಯಿಂದ ಹೋಮ ತಪಸ್ಸು ದಾನಾದಿಗಳನ್ನು ಮಾಡುತ್ತಾರೋ ಅದೆಲ್ಲಾ ಅಸತ್, ಅದು ಪ್ರಯೋಜನಕ್ಕೆ ಬಾರದುದು. ಅವನು ಬರೀ ಹೆಸರಿಗೆ ಮಾಡುತ್ತಾನೆ. ಕಾಟಾಚಾರಕ್ಕೆ ಮಾಡು ತ್ತಾನೆ. ಅದರಲ್ಲಿ ಶಾಸ್ತ್ರ ಸಮ್ಮತವಿಲ್ಲ. ಅದರಿಂದ ಬರುವ ಪ್ರಯೋಜನ ಸಾರ್ಥಕವಾದುದಲ್ಲ. ಅದರಿಂದ ನಾವು ಇನ್ನೂ ಹೆಚ್ಚು ಗೋಜಿಗೆ ಸಿಕ್ಕಿಕೊಳ್ಳುತ್ತೇವೆಯೇ ಹೊರತು ಗೋಜಿನಿಂದ ಪಾರಾಗು ವುದಿಲ್ಲ.

ಅದರಿಂದ ಸತ್ತ ಮೇಲೆ ನಾವು ಒಳ್ಳೆಯ ಲೋಕಕ್ಕೆ ಅಥವಾ ಒಳ್ಳೆಯ ಜನ್ಮದಲ್ಲಿ ಹುಟ್ಟುವಂತಹ ಪುಣ್ಯ ಸಂಪಾದಿಸಿಕೊಂಡು ಹೋಗುವುದಿಲ್ಲ. ಇಲ್ಲಿರುವಾಗಲೂ ನಮಗೆ ಸುಖ ಸಿಗುವುದಿಲ್ಲ. ಇಲ್ಲಿ ಮಾಡುವುದೆಲ್ಲಾ ತೋರಿಕೆಗೆ. ಮೊದಲು ಜನ ತಿಳಿಯದೆ ಕೊಂಡಾಡಿದರೂ ಅನಂತರ ಅವನ ಗುಟ್ಟೆಲ್ಲ ರಟ್ಟಾಗುವುದು. ಅವನದು ಬಂಡವಾಳವಿಲ್ಲದ ಬಡಾಯಿ. ಯಾರೂ ಅವನನ್ನು ಗೌರವಿಸುವುದಿಲ್ಲ. ಎಲ್ಲರೂ ಅವನನ್ನು ಲಘುವಾಗಿ ಕಾಣುವರು.

ಅರ್ಜುನ ಈ ಅಧ್ಯಾಯದಲ್ಲಿ ಮುಂಚೆ ಕೇಳಿದ ಪ್ರಶ್ನೆ ಶಾಸ್ತ್ರವಿಧಿಯನ್ನು ಬಿಟ್ಟು ಶ್ರದ್ಧೆಯಿಂದ ಮಾಡಿದರೆ ಅಂತಹ ನಿಷ್ಠೆ ಎಂತಹುದು ಎಂದು. ಇಲ್ಲಿ ಶಾಸ್ತ್ರಕ್ಕೆ ವಿರೋಧವಾಗಿ ಹೋಗುವುದು ಬೇರೆ, ಶಾಸ್ತ್ರ ತಿಳಿಯದೇ ಇದ್ದರೂ ಅಥವಾ ತಿಳಿದಿದ್ದರೂ, ಅದಕ್ಕೆ ವಿರೋಧವಾಗಿ ಹೋಗದೆ ಅದರ ಭಾರವನ್ನು ಅನುಸರಿಸುವುದು ಬೇರೆ ಎಂಬುದನ್ನು ವಿವರಿಸುವನು. ಶ‍್ರೀಕೃಷ್ಣ ಇಲ್ಲಿ ಹೆಚ್ಚು ಬೆಲೆ ಕೊಡುವುದು, ನಾವು ಮಾಡುವ ಕೆಲಸದ ಹಿಂದೆ ಇರುವ ಶ್ರದ್ಧೆಗೆ. ಈ ಶ್ರದ್ಧೆಯೇ ಸಾರ, ಜೀವಾಳ, ತಿರುಳು. ಇದು ಇದ್ದರೆ ಮಿಕ್ಕಿರುವುದೆಲ್ಲ ಗೌಣ. ಶ್ರದ್ಧೆ ಇಲ್ಲದೆ, ಇನ್ನೆಲ್ಲ ಇದ್ದರೂ ಅದು ನಿಸ್ಸಾರ, ಅದೆಲ್ಲ ಶವದ ಶೃಂಗಾರ. ಬಾಹ್ಯವಲ್ಲ ಮುಖ್ಯ. ಅಂತರಿಕ. ಡಂಭಾಚಾರಿ ಪ್ರಪಂಚವನ್ನೇ ದಂಗುಮಾಡಿ ಬಿಡಬಹುದು. ಆದರೆ ದೇವರು ನೋಡುವುದು ವೇಷವನ್ನಲ್ಲ, ಅದರ ಹಿಂದಿರುವ ಉದ್ದೇಶವನ್ನು.

