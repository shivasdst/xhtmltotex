
\chapter{ಪುರುಷೋತ್ತಮಯೋಗ}

ಶ್ರೀಕೃಷ್ಣ ಹೇಳುತ್ತಾನೆ:

\begin{verse}
ಊರ್ಧ್ವಮೂಲಮಧಃಶಾಖಮಶ್ವತ್ಥಂ ಪ್ರಾಹುರವ್ಯಯಮ್ ।\\ಛಂದಾಂಸಿ ಯಸ್ಯ ಪರ್ಣಾನಿ ಯಸ್ತಂ ವೇದ ಸ ವೇದವಿತ್ \versenum{॥ ೧ ॥}
\end{verse}

{\small ಮೇಲೆ ಬುಡವುಳ್ಳ ಕೆಳಗೆ ಕೊಂಬೆಗಳುಳ್ಳ ಅಶ್ವತ್ಥ ಮರವು ಅವಿನಾಶಿಯಾದುದು. ಅದರ ಎಲೆಗಳು ಛಂದಸ್ಸು ಗಳು. ಅದನ್ನು ಯಾರು ತಿಳಿದವನೊ ಅವನು ವೇದವನ್ನು ತಿಳಿದವನು.}

ಶ್ರೀಕೃಷ್ಣ ಈ ಸಂಸಾರವನ್ನು ಒಂದು ಅಶ್ವತ್ಥ ವೃಕ್ಷಕ್ಕೆ ಹೋಲಿಸಿರುವನು. ಆದರೆ ಈ ಸಂಸಾರದ ವೃಕ್ಷ ಒಂದು ವಿಚಿತ್ರ ರೀತಿಯಾಗಿದೆ. ಇದರ ಬೇರು ಮೇಲೆ, ಇದರ ಕೊಂಬೆ ಮತ್ತು ಎಲೆಗಳು ಕೆಳಗೆ. ಇದರ ಬೇರು ಮೇಲಿದೆ ಎಂದರೆ ಪರಬ್ರಹ್ಮನಲ್ಲಿ ನೆಲೆಸಿದೆ. ಈ ಮರ ಚಿಗುರಿ ಬಂದಿರುವುದು ಬ್ರಹ್ಮನ ಮೂಲದಿಂದ. ಇದು ತನಗೆ ತಾನೆ ಹುಟ್ಟಿಲ್ಲ. ಭಗವಂತನಿಂದ ಬಂದಿದೆ. ಅದರ ಕೆಳಗಡೆಯೇ ಶಾಖೋಪಶಾಖೆಗಳು ಒಳಗೊಂಡ ವೃಕ್ಷ. ಈ ಬ್ರಹ್ಮಾಂಡವೆಲ್ಲ ಅದರ ರೆಂಬೆಗಳು. ದೇಶಕಾಲ ನಿಮಿತ್ತದ ಒಳಗಿರುವ ಪ್ರತಿಯೊಂದು ವಸ್ತುವೂ ಅದಕ್ಕೆ ಸೇರಿರುವುದು. ಈ ಮರವಾದರೋ ಅವಿನಾಶಿ. ಈ ಸಂಸಾರ ವೃಕ್ಷ ಎಂದಿಗೂ ನಾಶವಾಗುವುದಿಲ್ಲ. ಕೆಲವು ಜೀವಿಗಳು ಮುಕ್ತರಾಗಿಹೋಗ ಬಹುದು. ಆದರೆ ಇನ್ನೂ ಬಹು ಜೀವಿಗಳು ಇದಕ್ಕೆ ಅಂಟಿಕೊಂಡಿರುವರು. ಕೆಲವು ವೇಳೆ ಪ್ರಳಯಕಾಲ ಬಂದಾಗ ಎಲೆಗಳು ಉದುರಿ ಹೋಗಿ ಬರೀ ಮರ ನಿಂತಂತೆ ಇರುವುದು. ಆದರೆ ವೃಕ್ಷ ಸತ್ತಿಲ್ಲ. ಅದರೊಳಗೆ ಹಸಿ ಇದೆ. ವಸಂತ ಪುತು ಬಂದೊಡನೆಯೇ ಅದರ ಮೂಲೆ ಮೂಲೆಯಿಂದ ಹೊಸ ಚಿಗುರುಗಳು ಅಂಕುರಿಸುವುವು. ಈ ಸಂಸಾರ ವೃಕ್ಷವನ್ನು ತುಂಬಿರುವುದೇ ವೇದಗಳೆಂಬ ಎಲೆಗಳು. ಈ ಸಂಸಾರ ಬದುಕಿರುವುದೇ ಕರ್ಮದಿಂದ. ಒಂದು ಸಾರಿ ಯಾವಾಗ ಕರ್ಮಮಾಡುತ್ತೇವೆಯೋ ಆಗ ಅದು ಮತ್ತೊಂದು ಕರ್ಮಕ್ಕೆ ಕಾರಣವಾಗುವುದು. ಹೀಗೆಯೇ ಕರ್ಮದ ಶೃಂಖಲೆ ಜೀವಿಯನ್ನು ಕಟ್ಟುವುದು. ಕರ್ಮದಲ್ಲಿ ಒಳ್ಳೆಯ ಕರ್ಮವೂ ಇದೆ, ಕೆಟ್ಟ ಕರ್ಮವೂ ಇದೆ. ಎರಡೂ ಜೀವನನ್ನು ಬಂಧನಕ್ಕೆ ಒಳಗು ಮಾಡುವುವು. ಒಂದು ಕಬ್ಬಿಣದ ಸಂಕೋಲೆ, ಇನ್ನೊಂದು ಚಿನ್ನದ ಸಂಕೋಲೆ.

ಯಾವನು ಇದನ್ನು ಅರಿಯುವನೊ ಅವನೇ ಶಾಸ್ತ್ರವನ್ನು ಬಲ್ಲವನು ಎಂದರೆ, ಶಾಸ್ತ್ರದ ಅಂತ ರಾರ್ಥವನ್ನು ಬಲ್ಲವನು. ಶಾಸ್ತ್ರದ ಸಾರವನ್ನು ಬಲ್ಲವನು. ಶಾಸ್ತ್ರವನ್ನು ತಿಳಿಯುವುದು ಒಂದು, ಶಾಸ್ತ್ರದ ಸಾರವನ್ನು ತಿಳಿಯುವುದು ಬೇರೊಂದು. ಶಾಸ್ತ್ರ ಬಲ್ಲವರಿಗೆಲ್ಲ ಅದರ ಸಾರ ಗೊತ್ತಿರುವುದಿಲ್ಲ. ಶಾಸ್ತ್ರದಲ್ಲಿ ಗೌಣವಾದ ವಿಷಯಗಳಿರುತ್ತವೆ ಮತ್ತು ಮುಖ್ಯವಾದ ವಿಷಯಗಳಿರುತ್ತವೆ. ಕೆಲವು ವೇಳೆ ಮುಖ್ಯವಾದ ವಿಷಯಗಳನ್ನು ಗೌಣದಿಂದ ಸುತ್ತಿಟ್ಟಿರುವರು. ಹೊರಗಿನದನ್ನು ಹಿಡಿದು ಒಳಗಿನದನ್ನು ಮರೆಯುವವನಿಗೆ ಶಾಸ್ತ್ರದ ಸಾರ ಗೊತ್ತಿಲ್ಲ. ಹೊರಗಿನದು ಮತ್ತು ಒಳಗಿನದು ಎರಡಕ್ಕೂ ಒಂದೇ ಬೆಲೆ ಕೊಡುವವನಿಗೂ ಶಾಸ್ತ್ರದ ಸಾರ ಅಷ್ಟು ಚೆನ್ನಾಗಿ ಗೊತ್ತಿಲ್ಲ. ಒಳಗಿನದನ್ನು ಮಾತ್ರ ಹಿಡಿದು ಹೊರಗಿನದು ಕೇವಲ ಒಳಗಿನದನ್ನು ರಕ್ಷಿಸುವುದಕ್ಕೆ ಸುತ್ತಿಟ್ಟಿರುವರು ಎಂಬುದನ್ನು ಯಾರು ಅರಿತಿರುವರೊ ಅವರೇ ಶಾಸ್ತ್ರದ ಸಾರವನ್ನು ಬಲ್ಲವರು. ಶ್ರೀರಾಮಕೃಷ್ಣರು ಭತ್ತದ ಒಂದು ನಿದರ್ಶನವನ್ನು ಕೊಡುವರು. ಭತ್ತದ ಹೊಟ್ಟು ಅಕ್ಕಿಯನ್ನು ರಕ್ಷಿಸುವುದಕ್ಕೆ ಇದೆ. ತಿನ್ನುವಾಗ ಹೊಟ್ಟನ್ನು ಚೆಲ್ಲಿ ಅಕ್ಕಿಯನ್ನು ತೆಗೆದುಕೊಳ್ಳುವೆವು. ಇಲ್ಲಿ ಹೊಟ್ಟೇ ಗೌಣ, ಒಳಗಿರುವ ಅಕ್ಕಿಯೇ ಮುಖ್ಯ. ಯಾವನು ಇದನ್ನು ನಿಷ್ಕರ್ಷಿಸಬಲ್ಲನೊ ಅವನಿಗೆ ಮಾತ್ರ ಶಾಸ್ತ್ರದ ಸಾರ ಗೊತ್ತಿರುವುದು.

\begin{verse}
ಅಧಶ್ಚೋರ್ಧ್ವಂ ಪ್ರಸೃತಾಸ್ತಸ್ಯ ಶಾಖಾ ಗುಣಪ್ರವೃದ್ಧಾ ವಿಷಯಪ್ರವಾಲಾಃ ।\\ಅಧಶ್ಚ ಮೂಲಾನ್ಯನುಸಂತತಾನಿ ಕರ್ಮಾನುಬಂಧೀನಿ ಮನುಷ್ಯಲೋಕೇ \versenum{॥ ೨ ॥}
\end{verse}

{\small ಗುಣಗಳಿಂದ ಬೆಳೆದುಕೊಂಡಿರುವ ವಿಷಯವೆಂಬ ಚಿಗುರುಗಳುಳ್ಳ ಅದರ ಕೊಂಬೆಗಳು ಮೇಲಕ್ಕೂ ಕೆಳಕ್ಕೂ ಹಬ್ಬಿವೆ, ಮನುಷ್ಯಲೋಕದಲ್ಲಿ ಕರ್ಮದ ಸಂಬಂಧವುಳ್ಳ ಬೇರುಗಳು ಕೆಳಗಡೆ ಹರಿದುಕೊಂಡಿವೆ.}

ಈ ಸಂಸಾರ ವೃಕ್ಷದೊಳಗೆ ಇರುವ ಹಸಿಯೇ ತಮಸ್ಸು, ರಜಸ್ಸು ಮತ್ತು ಸತ್ತ್ವವೆಂಬ ತ್ರಿಗುಣಗಳು. ಈ ಗುಣ ಇವುಗಳಲ್ಲಿ ಹರಿಯುತ್ತಿರುವುದರಿಂದಲೇ ಅದು ಹಸಿಯಾಗಿರುವುದು. ಎಷ್ಟೋ ವೇಳೆ ರೆಂಬೆಗಳು ಬಿದ್ದರೂ ಹೊಸದಾಗಿ ಹುಟ್ಟುತ್ತಿರುವುವು. ಈ ವೃಕ್ಷವನ್ನೆಲ್ಲಾ ತುಂಬಿರುವುದೇ ವಿಷಯ ವಸ್ತುಗಳೆಂಬ ಚಿಗುರುಗಳು. ನಮ್ಮ ಪಂಚೇಂದ್ರಿಯಗಳ ಮೂಲಕ ಇವನ್ನು ಗ್ರಹಿಸಬಹುದು. ಅದಕ್ಕೆ ಶಬ್ದವಿದೆ ಕಿವಿಯಿಂದ ಕೇಳಬಹುದು. ಅದು ಗಟ್ಟಿಯಾಗಿ ಅಥವಾ ಮೃದುವಾಗಿರುವುದು. ತಂಪಾಗಿಯೂ ಶಾಖಯುಕ್ತವಾಗಿಯೂ ಇದೆ. ಕೈಗಳ ಮೂಲಕ ಮುಟ್ಟಬಹುದು. ಅಂದವಾದ ರೂಪುಗಳಿವೆ. ಅದನ್ನು ಕಣ್ಣಿನ ಮೂಲಕ ನೋಡಬಹುದು. ರುಚಿ ಇದೆ ನಾಲಗೆಯ ಮೂಲಕ ಗ್ರಹಿಸಬಹುದು. ಅದಕ್ಕೆ ವಾಸನೆ ಇದೆ. ಮೂಗು ಅನುಭವಿಸಬಲ್ಲುದು. ನಮ್ಮ ಇಂದ್ರಿಯಗಳು ವಿಷಯ ವಸ್ತುಗಳಿಗೆ ಸಿಕ್ಕಿಕೊಳ್ಳುವುದು ಹೀಗೆ. ಒಮ್ಮೆ ಸಿಕ್ಕಿಕೊಂಡರೆ ನೊಣ ಹಲಸಿನ ಅಂಟಿಗೆ ಅಂಟಿಕೊಂಡಂತಾಗುವುದು ನಮ್ಮ ಸ್ಥಿತಿ. ಅಲ್ಲಿಂದ ತಪ್ಪಿಸಿಕೊಂಡು ಬರುವುದು ಬಹಳ ಕಷ್ಟ.

ಸಂಸಾರ ವೃಕ್ಷದ ಕೊಂಬೆಗಳು ಕೆಳಕ್ಕೆ ಮೇಲಕ್ಕೆ ಹರಡಿವೆ. ಸಾಧಾರಣವಾಗಿ ಭೂಲೋಕ ಮತ್ತು ಇದಕ್ಕಿಂತ ಇನ್ನು ವಿಕಾಸದ ದೃಷ್ಟಿಯಿಂದ ಕೆಳಕ್ಕಿರುವ ಲೋಕಗಳೆಲ್ಲ ಕೆಳಗೆ ಇರುವುವು. ದೇವತೆಗಳು ಮುಂತಾದವರು ಇರುವ ಲೋಕಗಳೆಲ್ಲ ಮೇಲಿರುವುವು. ಅಂತೂ ಮೇಲಿರಲಿ ಕೆಳಗಿರಲಿ ಎಲ್ಲಾ ಇರುವುದು ಸಂಸಾರ ಮರದಲ್ಲೆ. ಮನುಷ್ಯ ಲೋಕದಲ್ಲಿ ಕೊಂಬೆಗಳಿಂದ ನೇತಾಡುತ್ತಿರುವ ಕರ್ಮ ಸಂಬಂಧದ ಬಿಳಲುಗಳೆಲ್ಲಾ ಕೆಳಗಡೆ ಹರಡಿಕೊಂಡಿವೆ. ಕರ್ಮವೇ ನಮ್ಮನ್ನು ಸಂಸಾರಕ್ಕೆ ಕಟ್ಟಿಹಾಕಿರು ವುದು. ಕರ್ಮ ಮಾಡುವುದಕ್ಕೆ ಒಂದು ಉದ್ದೇಶವಿದೆ. ಅದೇ ನಮ್ಮ ಬಯಕೆಯ ತೃಪ್ತಿ. ನಮ್ಮಲ್ಲಿ ಆಸೆಯಿದೆ; ಅದನ್ನು ತೃಪ್ತಿ ಪಡಿಸಬೇಕಾಗಿದೆ. ಅದಕ್ಕಾಗಿ ಕರ್ಮ ಮಾಡುತ್ತೇವೆ. ಮಾಡಿದರೆ ಅಲ್ಲಿಗೇ ಕೊನೆಗಾಣುವುದಿಲ್ಲ. ಅದಕ್ಕಾಗಿ ಪುನಃ ಪುನಃ ಮಾಡುತ್ತ ಇರುತ್ತೇವೆ. ಕೊನೆಗೆ ನಾವು ಆಸೆಯೆಂಬ ಗಾಣಕ್ಕೆ ಕಟ್ಟಿದಂತೆ ಆಗುವೆವು. ಆಸೆಯೆಂಬ ಗಾಣಕ್ಕೆ ದಿನವೂ ಆಡಿಸುವುದಕ್ಕೆ ಹೊಸ ಬೀಜವಿರುವುದು. ಅದಕ್ಕೆ ಅಂತ್ಯವೇ ಇಲ್ಲ. ಆಸೆ ಆಸೆಯನ್ನು ಮರಿಹಾಕುವುದು. ಕೊನೆಗೆ ನಾವು ಆಸೆಯ ಮರಿಗಳಿಂದ ಆವೃತರಾಗಿ ಹೋಗುತ್ತೇವೆ.

\begin{verse}
ನ ರೂಪಮಸ್ಯೇಹ ತಥೋಪಲಭ್ಯತೇ ನಾಂತೋ ನ ಚಾದಿರ್ನ ಚ ಸಂಪ್ರತಿಷ್ಠಾ ।\\ಅಶ್ವತ್ಥಮೇನಂ ಸುವಿರೂಢಮೂಲಮಸಂಗಶಸ್ತ್ರೇಣ ದೃಢೇನ ಛಿತ್ತ್ವಾ \versenum{॥ ೩ ॥}
\end{verse}

\begin{verse}
ತತಃ ಪದಂ ತತ್ಪರಿಮಾರ್ಗಿತವ್ಯಂ ಯಸ್ಮಿನ್ ಗತಾ ನ ನಿವರ್ತಂತಿ ಭೂಯಃ ।\\ತಮೇವ ಚಾದ್ಯಂ ಪುರುಷಂ ಪ್ರಪದ್ಯೇ ಯತಃ ಪ್ರವೃತ್ತಿಃ ಪ್ರಸೃತಾ ಪುರಾಣೀ \versenum{॥ ೪ ॥}
\end{verse}

{\small ಈ ಸಂಸಾರದಲ್ಲಿ ಇದರ ಸ್ವರೂಪ ಮೇಲೆ ಹೇಳಿರುವಂತೆ ಕಾಣುವುದಿಲ್ಲ. ಇದರ ಆದಿಯಾಗಲೀ ಮಧ್ಯವಾಗಲೀ ಅಂತ್ಯವಾಗಲೀ ಕಾಣುವುದಿಲ್ಲ. ಚೆನ್ನಾಗಿ ಬೇರೂರಿರುವ ಅಶ್ವತ್ಥವನ್ನು ಅಸಂಗವೆಂಬ ದೃಢವಾದ ಶಸ್ತ್ರದಿಂದ ಕತ್ತರಿಸಿ, ಅನಂತರ ಯಾವುದನ್ನು ಸೇರಿದವರು ಪುನಃ ಹಿಂತಿರುಗುವುದಿಲ್ಲವೋ ಆ ಪದವನ್ನು ಹುಡುಕ ತಕ್ಕದ್ದು. ಯಾರ ದೆಸೆಯಿಂದ ಪುರಾತನವಾದ ಈ ಸಂಸಾರ ಬಂದಿದೆಯೋ ಆ ಆದಿಪುರುಷನನ್ನು ಶರಣು ಹೋಗುತ್ತೇನೆ.}

ಈ ಸಂಸಾರ ದೇವರಿಂದ ಬಂದಿರುವಂತೆ ನಮಗೆ ಕಾಣುವುದಿಲ್ಲ. ಈ ಸಂಸಾರವನ್ನು ನೋಡುತ್ತ ಹೋದರೆ ದೇವರನ್ನೆ ಮರೆಯುತ್ತೇವೆ ನಾವು. ಇದು ತನಗೆ ತಾನೆ ಹುಟ್ಟಿ ಬೆಳೆಯುತ್ತಿರುವಂತಿದೆ. ದೇವರನ್ನು ಮರೆಸುವುದಕ್ಕೆ ಎಷ್ಟು ಬೇಕೋ ಅಷ್ಟೆಲ್ಲ ಈ ಸಂಸಾರ ವೃಕ್ಷದಲ್ಲಿದೆ. ಆದಕಾರಣವೇ ಜಡವಾದಿಗಳು ಚಾರ್ವಾಕರು ಸಂದೇಹವಾದಿಗಳು ಇವರೆಲ್ಲ ಈ ಪ್ರಪಂಚದಲ್ಲಿರುವರು. ಇದನ್ನು ಮಾಡಿದವನು ಕಣ್ಣಿಗೆ ಕಾಣುವಂತೆ ಇದ್ದರೆ, ಅವನನ್ನು ಕಂಡುಹಿಡಿಯುವುದು ಸುಲಭ. ಆದರೆ ಅವನು ನುಣುಚಿಕೊಂಡಿರುವನು.

ಈ ಸಂಸಾರ ವೃಕ್ಷ ಆದಿ ಮಧ್ಯ ಅಂತ್ಯರಹಿತವಾದುದು. ಒಂದು ಸೊನ್ನೆಯನ್ನು ತೆಗೆದುಕೊಂಡು ಇದು ಎಲ್ಲಿಂದ ಪ್ರಾರಂಭವಾಯಿತು ಎಂಬುದನ್ನು ಹೇಗೆ ಹೇಳಲಾರೆವೊ ಹಾಗೆಯೇ ಈ ಸಂಸಾರ ವೆಂಬುದು. ಬೇಕಾದರೆ ಯುಗದ ಆದಿಗೆ ಹೋಗಬಹುದು. ಆದರೆ ಕಾಲದ ಆದಿಗೆ ಹೋಗುವುದಕ್ಕಾಗು ವುದಿಲ್ಲ. ಅದರಂತೆಯೇ ಇದಕ್ಕೆ ಕೊನೆ ಯಾವುದು ಎಂಬುದನ್ನು ಹೇಳಲಸಾಧ್ಯ. ಇದಕ್ಕೆ ಕೊನೆಯೇ ಇಲ್ಲ. ಒಂದರ ಕೊನೆ ಮತ್ತೊಂದರ ಆದಿಗೆ ಪ್ರಾರಂಭ. ಸಂಸಾರಸಾಗರದಲ್ಲಿರುವ ಅಲೆಗೆ ಆದಿ ಅಂತ್ಯವೆಂಬುದು ಇಲ್ಲ. ಇದು ಅಪಾರ, ಅನಂತ. ಇದರ ಮಧ್ಯೆ ನಾವಿರುವ ಈಗಿನ ಸ್ಥಿತಿಯಾದರೂ ನಮಗೆ ಗೊತ್ತೆ ಎಂದರೆ, ಅದರ ವಿಷಯವಾಗಿ ನಮಗೆ ಗೊತ್ತಿರುವುದು ಅಷ್ಟು ಅಲ್ಪ. ಎಲ್ಲೊ ಒಂದು ಮಾರುದ್ದ ಹೋಗುವುದು ನಮ್ಮ ವಿಚಾರ, ಅದರ ಮುಂದಕ್ಕೆ ಹೋಗಲಾರದು. ಅದರ ಹಿಂದಕ್ಕೆ ಹೋಗಲಾರದು. ಇದೂ ಹೇಗೆ ಆಯಿತು ಎಂಬುದನ್ನು ಮಾತ್ರ ಕೆಲವು ಕಡೆ ವಿವರಿಸಬಹುದು. ಏತಕ್ಕಾಯಿತು ಎಂದರೆ ನಮಗೆ ಗೊತ್ತಿಲ್ಲ. ಒಬ್ಬೊಬ್ಬನು ಇದಕ್ಕೆ ಒಂದೊಂದು ವಿವರಣೆ ಕೊಡು ತ್ತಾನೆ. ಒಂದರಷ್ಟೇ ಮತ್ತೊಂದು ಸಮರ್ಪಕವಾಗಿರುವಂತೆ ತೋರುವುದು. ಇದೇ ಸರಿ ಎಂದು ಯಾವುದನ್ನೂ ನಿರ್ಧರಿಸುವಂತಿಲ್ಲ. ಈ ಅಶ್ವತ್ಥವೃಕ್ಷದ ಬಿಳಲುಗಳು ಬೇರೂರಿ ನೇತಾಡುತ್ತವೆ. ಎಷ್ಟೇ ಜಗ್ಗಿಸಿದರೂ ಅದು ಬಿದ್ದು ಹೋಗುವುದಿಲ್ಲ. ಕೆಲವು ವೇಳೆ ಅದರಿಂದ ಬರುವ ಬಹಳ ಕಹಿಯಾದ ಹಣ್ಣನ್ನು ತಿಂದಾಗ ಇನ್ನು ಮೇಲೆ ಅವನ್ನು ತಿನ್ನುವುದಿಲ್ಲ, ಎಂದು ಶಪಥಮಾಡಿದರೂ, ಸ್ವಲ್ಪ ಹೊತ್ತಾದ ಮೇಲೆ ಪುನಃ ಮನಸ್ಸು ಆಶಿಸುವುದು. ನಮಗೂ ಆ ವಿಷಯಗಳಿಗೂ ಇರುವ ಸಂಬಂಧ ಜನ್ಮ ಜನ್ಮಾಂತರದಿಂದ ಬಂದಿರುವುದು. ಒಮ್ಮೆಯೇ ಬಿಟ್ಟು ಹೋಗುವುದಿಲ್ಲ ಅದು. ಅದನ್ನು ಅಸಂಗವೆಂಬ ದೃಢವಾದ ಶಸ್ತ್ರದಿಂದ ಕತ್ತರಿಸಬೇಕು. ವಿಷಯವಸ್ತುವಿನ ಮೇಲಿರುವ ವ್ಯಾಮೋಹ ವನ್ನು ಕಟುಕನಂತೆ ಛೇದಿಸಬೇಕು. ಆಗ ಮಾತ್ರ ಸಾಧ್ಯ. ಸಂಸಾರ ವೃಕ್ಷಕ್ಕೂ ನನಗೂ ಇರುವ ಸಂಬಂಧವನ್ನು ಕಡಿದುಕೊಂಡರೆ ನಾನು ಪಾರಾಗುತ್ತೇನೆ. ಈ ಸಂಸಾರ ವೃಕ್ಷವೇನೊ ಬಿದ್ದು ಹೋಗುವುದಿಲ್ಲ. ಆದರೆ ನಾನು ಆದರಿಂದ ಪಾರಾಗುತ್ತೇನೆ. ನನ್ನ ಪಾಲಿಗೆ ಅದು ಇಲ್ಲದಂತೆ. ನಾವು ಅದರಿಂದ ತಪ್ಪಿಸಿಕೊಳ್ಳಬೇಕು ಮತ್ತು ಯಾವುದನ್ನು ಸೇರಿದರೆ ಪುನಃ ಹಿಂತಿರುಗಲಾಗುವುದಿಲ್ಲವೊ, ಆ ಪದವನ್ನು ಹುಡುಕಬೇಕು. ನಾವು ಸುಮ್ಮನೆ ಸಂಸಾರದಿಂದ ಪಾರಾಗಿ ಬಹಳ ಕಾಲ ಇರುವುದಕ್ಕೆ ಆಗುವುದಿಲ್ಲ. ನಾವು ಭಗವಂತನ ಕೈಯನ್ನು ಹಿಡಿದುಕೊಳ್ಳಬೇಕು. ಇಲ್ಲದೆ ಇದ್ದರೆ, ಪುನಃ ಸಂಸಾರದ ಆಕರ್ಷಣೆಗಳಿಗೆ ಒಳಗಾಗುತ್ತೇವೆ. ಅನೇಕ ಕ್ಷಿಪಣಿಗಳು ಭೂಮಿಯಿಂದ ತಪ್ಪಿಸಿಕೊಂಡು ಸುತ್ತಲೂ ತಿರುಗುತ್ತಿರುವುವು. ಕಾಲ ಕ್ರಮೇಣ ಅವು ಭೂಮಿಯ ಆಕರ್ಷಣೆಗೆ ಒಳಪಟ್ಟು ಹಿಂತಿರುಗಿ ಬೀಳುವುವು. ಹಾಗೆ ಬರದೇ ಇರಬೇಕಾದರೆ ಸೂರ್ಯನ ಆಕರ್ಷಣೆಗೆ ಸಿಕ್ಕಬೇಕು. ನಾವು ಕೂಡಾ ಭೂಮಿಯ ಆಕರ್ಷಣೆಯಿಂದ ಕಿತ್ತು ಹೋಗಿ, ಭಗವಂತನ ಆಕರ್ಷಣೆಗೆ ಬೀಳಬೇಕು. ನಾವು ಅವನನ್ನು ಸೇರಿದರೆ ಮಾತ್ರ ಸಂಸಾರಕ್ಕೆ ಹಿಂತಿರುಗಿ ಬರುವುದಿಲ್ಲ. ಅವನನ್ನು ಸೇರದೇ ಇದ್ದರೆ ಪುನಃ ಸಂಸಾರದ ವಾತಾವರಣಕ್ಕೆ ಸಿಕ್ಕಿ ಬೀಳುತ್ತೇವೆ.

ಈ ಸಂಸಾರ ಯಾರಿಂದ ಬಂದಿದೆಯೋ ಅವನನ್ನು ಹುಡುಕಬೇಕು. ಆಗ ಮಾತ್ರ ನಾವು ಪಾರಾಗಲು ಸಾಧ್ಯ. ಭಗವಂತನಲ್ಲಿ ಸಂಪೂರ್ಣ ಶರಣಾದರೆ ಮಾತ್ರ ಅವನು ನಮ್ಮನ್ನು ಕೆಳಗೆ ಬೀಳದಂತೆ ರಕ್ಷಿಸುತ್ತಾನೆ. ಇಲ್ಲಿ ಶರಣಾಗುವುದು ಎಂದರೆ ಸಂಪೂರ್ಣ ನಾವು ಅವನಿಗೆ ಅರ್ಪಿಸಿ ಕೊಂಡಿರಬೇಕು. ಆಗ ಅವನು ನಮ್ಮನ್ನು ಉದ್ಧರಿಸುತ್ತಾನೆ. ಸುಮ್ಮನೆ ಬಾಯಿ ಮಾತಿನಲ್ಲಿ ಹೇಳುವುದಲ್ಲ. ಕಾರ್ಯತಃ ಅವನಿಗೆ ಅರ್ಪಿತವಾಗಬೇಕು. ಈ ಸಂಸಾರ ಭಗವಂತನ ಸೃಷ್ಟಿ. ಯಾರು ಭಗವಂತನನ್ನು ಮರೆತಿರುವರೋ ಅವರನ್ನು ಕಾಡುವುದು. ಯಾರು ಭಗವಂತನನ್ನು ನೆರೆ ನಂಬಿರು ವರೋ ಅವರನ್ನು ಬಿಡುವುದು. ಈ ಸಂಸಾರ ಬೇಟೆಗಾರನ ಜೊತೆಗಿರುವ ಬೇಟೆನಾಯಿ. ಬೇಟೆಗಾರ ಸುಮ್ಮನಿರು ಬೊಗಳಬೇಡ ಎಂದಲ್ಲಿ ಅದು ಸುಮ್ಮನಾಗುವುದು. ಅವನು ಹಾಗೆನ್ನಬೇಕಾದರೆ ನಾವು ಅವನಿಗೆ ಹತ್ತಿರದವರಾಗಿರಬೇಕು. ಶರಣಾಗತಿಯಿಂದ ನಾವು ಅದನ್ನು ಸಂಪಾದನೆ ಮಾಡುತ್ತೇವೆ. ಭಗವಂತ ತನ್ನಲ್ಲಿ ಶರಣಾದವರ ತಪ್ಪನ್ನೆಲ್ಲ ಮನ್ನಿಸಿ ಮೇಲೆತ್ತುತ್ತಾನೆ.

\begin{verse}
ನಿರ್ಮಾನಮೋಹಾ ಜಿತಸಂಗದೋಷಾ \\ ಅಧ್ಯಾತ್ಮನಿತ್ಯಾ ವಿನಿವೃತ್ತಕಾಮಾಃ ।\\ದ್ವಂದ್ವೈರ್ವಿಮುಕ್ತಾಃ ಸುಖದುಃಖಸಂಜ್ಞೆ ೈ—\\ ಗRಚ್ಛಂತ್ಯಮೂಢಾಃ ಪದಮವ್ಯಯಂ ತತ್ \versenum{॥ ೫ ॥}
\end{verse}

{\small ಅಹಂಕಾರ ಮೋಹಗಳಿಲ್ಲದವರೂ, ಆಸಕ್ತಿ ಎಂಬ ದೋಷದಿಂದ ಪಾರಾದವರೂ, ಆಧ್ಯಾತ್ಮನಿತ್ಯರೂ, ಸಮಸ್ತ ಕಾಮಗಳನ್ನು ಬಿಟ್ಟಿರುವವರೂ, ಸುಖದುಃಖಗಳೆಂಬ ದ್ವಂದ್ವಗಳಿಂದ ಬಿಡಲ್ಪಟ್ಟವರೂ ಆದ ಜ್ಞಾನಿಗಳು ಅವ್ಯಯವಾದ ಪದವನ್ನು ಪಡೆಯುತ್ತಾರೆ.}

ಈ ಶ್ಲೋಕದಲ್ಲಿ ಶ್ರೀಕೃಷ್ಣ ಭಗವಂತನಲ್ಲಿ ಶರಣಾಗಬಲ್ಲವರು ಯಾರು ಎಂಬುದನ್ನು ವಿವರಿಸು ತ್ತಾನೆ. ಎಲ್ಲರಿಗೂ ಭಗವಂತನಲ್ಲಿ ಶರಣಾಗಿ ಈ ಪ್ರಪಂಚದ ಕೋಟಲೆಯಿಂದ ಪಾರಾಗಬೇಕೆಂಬ ಆಸೆಯೇನೋ ಇರಬಹುದು. ಆದರೆ ಒಂದು ವಸ್ತುವನ್ನು ಹೊಂದುವುದಕ್ಕೆ ಆಸೆಯೊಂದೇ ಸಾಲದು. ಅದಕ್ಕೆ ಯೋಗ್ಯತೆ ಇಲ್ಲದೇ ಇದ್ದರೆ ಅದು ಬರೀ ಆಸೆಯಲ್ಲೇ ಕೊನೆಗಾಣುವುದು. ಆಸೆ ಎಂಬುದು ಬೀಜದಂತೆ. ಅದನ್ನು ನೆಲಕ್ಕೆ ಹಾಕಿ ಆರೈಕೆ ಮಾಡಬೇಕು. ಆಗಲೇ ಅದರಿಂದ ಫಲ ದೊರಕಬೇಕಾದರೆ. ಆ ಯೋಗ್ಯತೆಗಳೇನೇನು ಎಂಬುದನ್ನು ವಿವರಿಸುವನು.

ಅವನಲ್ಲಿ ಅಹಂಕಾರವಿರಕೂಡದು. ತನ್ನ ಸಮಾನ ಇಲ್ಲ ಎಂದು ಮೆರೆಯುತ್ತಿರಬಾರದು. ದೇವರಲ್ಲಿ ಶರಣಾಗಬೇಕಾದರೆ, ನನ್ನ ಬಂಡವಾಳ ನನಗೆ ಚನ್ನಾಗಿ ಗೊತ್ತಾಗಿರಬೇಕು. ನಿಜವಾದ ನನ್ನ ಬಂಡವಾಳ ಏನು ಎಂಬುದನ್ನು ತಿಳಿದುಕೊಂಡಾಗ ಅದು ಸೊನ್ನೆ ಎಂದು ಗೊತ್ತಾಗುವುದು. ಈ ಸೊನ್ನೆಗೆ ಅಹಂಕಾರ ಪಡುವುದಕ್ಕೆ ಏನು ಇದೆ? ಇದಕ್ಕೆ ಬೆಲೆ ಬರುವುದು, ಇದರ ಹಿಂದೆ ಇರುವ ಒಂದರಿಂದ. ದೇವರು ನಮ್ಮ ಹಿಂದೆ ಇದ್ದು ನಮ್ಮ ದೇಹ ಮನಸ್ಸು ಬುದ್ಧಿ ಇಂದ್ರಿಯಗಳಿಗೆ ಚೇತನ ತುಂಬುತ್ತಿರುವುದರಿಂದ ಇದಕ್ಕೆ ಒಂದಿಷ್ಟು ಬೆಲೆ. ಒಂದು ವಿದ್ಯುತ್ ಬಲ್ಬ್ ನಾನು ಬೆಳಗುತ್ತಿರುವೆ ಎಂದು ಅಹಂಕಾರ ಪಡಬಹುದು. ಆದರೆ ಅದೇ ಬೆಳಗುವುದು ನಿಜವಾಗಿ? ಅದರ ಹಿಂದೆ ಇರುವ ವಿದ್ಯುತ್ ಶಕ್ತಿ. ಅದು ನಿಂತೊಡನೆಯೆ ಅದರ ಅಹಂಕಾರ ಅಡಗುವುದು. ಅದರಂತೆಯೇ ದೇವರೆಡೆಗೆ ಹೋಗಬೇಕಾದರೆ ಮೊದಲನೆ ಯೋಗ್ಯತೆಯೇ ದೈನ್ಯತೆ. ಅದಿಲ್ಲದೆ ಇದ್ದರೆ ಶರಣಾಗತಿ ಬರುವುದಿಲ್ಲ. ನನ್ನಲ್ಲಿ ಹೆಮ್ಮೆ ತಾಳುವುದಕ್ಕೆ ಏನೂ ಇಲ್ಲ. ಒಂದು ವೇಳೆ ಇರುವಂತೆ ತೋರಿದರೆ, ಅದೆಲ್ಲ ನಿನ್ನದು. ಭ್ರಮೆಯಿಂದ ಅದನ್ನು ನನ್ನದು ಎಂದು ಭಾವಿಸಿದ್ದೆ. ಬಲ್ಬು ಕಾಂತಿ ನನ್ನದು ಎಂದು ಭಾವಿಸುವಂತೆ, ನಲ್ಲಿ ತನ್ನ ಮೂಲಕ ಬರುವ ನೀರು ತನ್ನದು ಎಂದು ಭಾವಿಸುವಂತೆ. ಆದರೆ ಹಿಂದಿನಿಂದ ಬರುವುದು ನಿಲ್ಲಲಿ, ಬಲ್ಬು ಇದೆ ಕಾಂತಿಯಿಲ್ಲ, ನಲ್ಲಿ ಇದೆ ನೀರು ಇಲ್ಲ. ಎಲ್ಲದರ ಮೂಲಕ ಕೆಲಸ ಮಾಡುವುದೇ ಅವನ ಶಕ್ತಿ ಅವನ ಯುಕ್ತಿ, ಅವನ ಜ್ಞಾನ. ಅದು ಕೆಲಸ ಮಾಡುವುದಕ್ಕೆ ಇರುವ ಆತಂಕಗಳನ್ನು ತೆಗೆದಾಗ ಅದು ಚೆನ್ನಾಗಿ ಕೆಲಸ ಮಾಡುವುದು. ಕಾಲುವೆಯಲ್ಲಿ ನೀರು ಹರಿದು ಹೋಗುತ್ತಿದೆ. ನಮ್ಮ ಗದ್ದೆಗ ನೀರು ಬರುವ ಕಾಲುವೆ ಮಣ್ಣಿನಿಂದ ಮುಚ್ಚಿಹೋಗಿದೆ. ಯಾವಾಗ ಅದನ್ನು ತೆಗೆಯುವೆವೋ ಆಗ ನೀರು ಹರಿದುಕೊಂಡು ಬರುವುದು. ದೇವರನ್ನು ನನ್ನೊಳಗೆ ಬಾರದಂತೆ ಮಾಡಿರುವುದು ನಾನು ಎಂಬ ದುರಹಂಕಾರ. ಅದನ್ನು ಓಡಿಸಿದ ಒಡನೆಯೇ ದೇವರು ಬರುವನು. ಕೊಡವನ್ನು ನೀರಿನಲ್ಲಿ ಬಾಯಿಯನ್ನು ಒಳಮುಖ ಮಾಡಿ ದಬ್ಬಿದರೆ ಕೊಡ ಬಲಾತ್ಕಾರಕ್ಕೆ ನೀರಿನ ಒಳಗೆ ಹೋದರೂ ನೀರು ಅದರ ಒಳಗೆ ನುಗ್ಗಲಾರದು. ಏಕೆಂದರೆ ಆಗಲೆ ಅದರೊಳಗೆ ಗಾಳಿ ಇದೆ. ಒಳಗಿರುವ ಗಾಳಿ ತಪ್ಪಿಸಿಕೊಳ್ಳಲು ಕೊಡವನ್ನು ಸ್ವಲ್ಪ ಒಳಗೆ ತಿರಿಗಿಸಿದೊಡನೆ, ಹೊರಗಿನಿಂದ ನೀರು ಬರುವುದು. ಇಲ್ಲಿ ಅಹಂಕಾರ ನಮಗೆ ಮಾಡುವ ಕೆಲಸವೇ ಇದು. ಬರುವ ದೇವರನ್ನು ತಡೆಯುವುದು. ನಾನೇ ಇದ್ದೇನೆ, ನನಗೇನು ಕೆಲಸ ಇಲ್ಲಿ ಎನ್ನುವುದು. ಆದರೆ ಯಾವಾಗ ಅದನ್ನು ಅಟ್ಟುತ್ತೇವೆಯೋ ಆಗ ನಮ್ಮ ಮನಸ್ಸು ಕ್ಷಣಕಾಲವೂ ಖಾಲಿ ಇರಲಾರದು. ಪ್ರತಿಯೊಬ್ಬರ ಹೃದಯವನ್ನು ಪ್ರವೇಶಿಸುವುದಕ್ಕೆ ದೇವರು ಕಾಯುತ್ತಿರುವನು. ಅವನಿಗೆ ಅವಕಾಶ ಬಂದೊಡನೆಯೇ ಅವನು ಒಳಗೆ ನುಗ್ಗುವನು. ಅಹಂಕಾರಕಲ್ಲಿನಂತೆ ನೀರಿನಲ್ಲಿ ಮುಳುಗುವುದು. ಯಾವಾಗ ಅಹಂಕಾರವನ್ನು ಒಣಗಿಸುತ್ತೇವೆಯೋ ಅದು ಹಗುರವಾಗಿ ನೀರಿನ ಮೇಲೆ ತೇಲುವುದು. ಅದನ್ನೇ ಹಿಡಿದುಕೊಂಡು ಸಂಸಾರದಲ್ಲಿ ಈಜಿಕೊಂಡು ಹೋಗಬಹುದು. ನಾವು ಮೊದಲು ಅಹಂಕಾರವನ್ನು ಬಿಡುವುದಿಲ್ಲ. ಏನೋ ಬಹಳ ತೂಕವಾಗಿದೆ, ಅದರಿಂದ ಬಹಳ ಪ್ರಯೋಜನ ಇದೆ ಎಂದು ಭಾವಿಸುತ್ತೇವೆ. ಆದರೆ ಅದರ ಅನನುಕೂಲಗಳೆಲ್ಲ ಚೆನ್ನಾಗಿ ಗೊತ್ತಾದ ಮೇಲೆಯೇ, ಒಳಗಿರುವುದನ್ನೆಲ್ಲಾ ಚೆನ್ನಾಗಿ ಬಿಸಿಲಿನಲ್ಲಿ ಒಣಗಿಸಿದಾಗಲೇ ಅದು ಹಗುರವಾಗವುದು. ಇಲ್ಲಿ ಬಿಸಿಲೆಂದರೆ ಭಗವಂತನ ಮೇಲೆ ಭಕ್ತಿ, ಭಗವಂತನ ಜ್ಞಾನ.

ಅವನಲ್ಲಿ ಮೋಹವಿಲ್ಲ, ಅಹಂಕಾರ ಯಾವಾಗ ನಮ್ಮನ್ನು ಬಿಟ್ಟುಹೋಗುವುದೋ ಮೋಹ ಜೊತೆಯಲ್ಲಿಯೇ ಬಿಟ್ಟುಹೋಗುವುದು. ಅಹಂಕಾರದ ನೆರಳು ಮೋಹ. ಒಂದನ್ನು ಆಚೆಗೆ ಕಳಿಸಿದರೆ, ಮತ್ತೊಂದನ್ನು ಪ್ರತ್ಯೇಕವಾಗಿ ಕಳುಹಿಸಬೇಕಾಗಿಲ್ಲ. ಅದೂ ಕೂಡ ಅಹಂಕಾರವನ್ನು ಬೆನ್ನು ಹಿಡಿದು ಹೋಗುವುದು. ಯಾವಾಗ ಮೋಹವಿಲ್ಲವೋ ಆಗ ವಸ್ತುವಿನ ನಿಜವಾದ ಪರಿಸ್ಥಿತಿ ನಮಗೆ ಅರಿವಾಗುವುದು. ನಾವು ಇನ್ನು ಮೇಲೆ ಬಣ್ಣದ ಕನ್ನಡಕದ ಮೂಲಕ ಮತ್ತೊಂದನ್ನು ನೋಡುವುದಿಲ್ಲ. ಇನ್ನವನು ಬಾಹ್ಯವಸ್ತುವಿನ ಆಕರ್ಷಣೆಗೆ ಬೀಳುವುದಿಲ್ಲ.

ಅವನು ಆಸಕ್ತಿ ಎಂಬ ದೋಷದಿಂದ ಪಾರಾಗಿದ್ದಾನೆ. ಆಸಕ್ತಿ ಎಂಬುದು ಒಂದು ಸಂಕೋಚ ವಿಕಾಸವಾಗುವ ಹಗ್ಗ. ಯಾವಾಗ ಈ ಹಗ್ಗದಿಂದ ನಾವು ಕಟ್ಟಿಕೊಳ್ಳುವೆವೊ, ಕಟ್ಟಿಕೊಂಡ ವಸ್ತುವಿಗೆ ಗುಲಾಮರು ಆಗುತ್ತೇವೆ. ಅದು ನಮ್ಮನ್ನು ಕುಣಿಸುತ್ತದೆ. ಕಪಿ ಆಡಿಸುವವನ ಕೈಯಲ್ಲಿ ಇರುವ ಕಪಿಯಂತೆ ಆಗುತ್ತೇವೆ ನಾವು. ಮನುಷ್ಯ ಆಸಕ್ತಿಗೆ ಬದ್ಧನಾಗಿ ಎಂತಹ ನೀಚ ಕೆಲಸವನ್ನಾದರೂ ಮಾಡುತ್ತಾನೆ. ನ್ಯಾಯ ಅನ್ಯಾಯ, ಧರ್ಮ ಅಧರ್ಮ ಅವನ್ನೆಲ್ಲಾ ಗಾಳಿಗೆ ದೂಡುತ್ತಾನೆ. ಎಲ್ಲಿಯ ವರೆಗೆ ನಾವು ಆಸಕ್ತಿಗೆ ಬದ್ಧರೋ ಅಲ್ಲಿಯವರೆಗೆ ಆ ವಸ್ತುವಿನಿಂದ ನಮ್ಮ ದೇಹ ದೂರ ಇರಬಹುದು. ಆದರೂ ಮನಸ್ಸು ಅದಕ್ಕೆ ಅಂಟಿಕೊಂಡಿರುವುದು. ಆಸಕ್ತಿ ಎಂಬ ಹಗ್ಗ ದೂರದಿಂದ ಕಡಿದುಹೋಗುವುದಿಲ್ಲ. ಅದು ಅಷ್ಟೂ ಉದ್ದವಾಗುವುದು ಅಷ್ಟೇ. ಭಕ್ತನಾದವನು ಪ್ರಾಪಂಚಿಕ ವಸ್ತುವಿನ ಮೇಲಿನ ಆಸಕ್ತಿಯನ್ನು ಬಿಡುತ್ತಾನೆ. ಆದರೆ ಭಗವಂತನ ಮೇಲೆ ಅದ್ಭುತವಾದ ಆಸಕ್ತಿಯನ್ನು ರೂಢಿಸಿಕೊಳ್ಳುತ್ತಾನೆ.

ಆತ ತನ್ನ ಆಧ್ಯಾತ್ಮಿಕ ಸಾಧನೆಗಳನ್ನು ಬಿಡದೆ ಮಾಡುತ್ತಿರುವನು. ಅವನಲ್ಲಿ ಉಬ್ಬರ ಇಳಿತಗಳನ್ನು ನೋಡುವುದಿಲ್ಲ. ಒಂದೇ ಸಮನಾಗಿ ಧ್ಯಾನ, ಅಧ್ಯಯನ, ಪೂಜೆ, ಪ್ರಾರ್ಥನೆ ಮೊದಲಾದುವುಗಳನ್ನು ಬಿಡದೆ ಮಾಡಿಕೊಂಡು ಹೋಗುವನು. ಅದು ನೀರಸವಾಗಲಿ ಸರಸವಾಗಲಿ ಸಿದ್ಧಿಯಲ್ಲಿ ಪರ್ಯವಸಾನ ವಾಗುವವರೆಗೆ ಅವನು ಸಾಧನೆಯನ್ನು ಬಿಡುವುದಿಲ್ಲ. ಎಷ್ಟೇ ಆತಂಕಗಳು ಬರಲಿ, ಎಷ್ಟೇ ನೀರಸ ವಾಗಲಿ, ಎಷ್ಟು ಸಲ ಸೋಲಲಿ, ಅವನಂತು ಛಲದಿಂದ ತನ್ನ ಅಭ್ಯಾಸಗಳನ್ನು ಮುಂದುವರಿಸುವನು. ಆಧ್ಯಾತ್ಮಿಕ ಜೀವನದಲ್ಲಿ ಸಾಧನೆ ಎಲ್ಲೋ ಕೆಲವು ತಿಂಗಳಲ್ಲಿ ಕೆಲವು ವರುಷಗಳಲ್ಲಿ ಪೂರೈಸುವುದಲ್ಲ. ನಮ್ಮ ಇಡೀ ಬಾಳನ್ನೇ ಅದಕ್ಕೆ ನಿವೇದಿಸಬೇಕು. ಆ ಹೋರಾಟದಲ್ಲಿಯೇ ನಾವೊಂದು ಆನಂದವನ್ನು ಹುಟ್ಟಿಸಿಕೊಳ್ಳಬೇಕು. ಎಷ್ಟನ್ನು ನಾನು ಪಡೆದಿದ್ದೇನೆ ಎಂದು ನೋಡಿಕೊಳ್ಳುವುದಲ್ಲ. ಏಕೆಂದರೆ ಆ ದೃಷ್ಟಿಯಿಂದ ನೋಡಿದರೆ ನಮಗೆ ಸಿಕ್ಕಿರುವುದು ಬಹಳ ಅಲ್ಪ. ಎಷ್ಟು ಪ್ರಯತ್ನಮಾಡಿದ್ದೇವೆ ಎಂದು ನಾವು ನೋಡಿಕೊಳ್ಳಬೇಕು. ಬಿಡದೆ ಸತತ ದುಡಿದಿದ್ದರೆ, ಒಂದು ಸಮಾಧಾನ ಕಾಲವನ್ನು ವ್ಯರ್ಥ ಮಾಡಲಿಲ್ಲ ಎಂದು. ಹಿಂದೆ ಎಷ್ಟೊಂದು ಜನ್ಮಗಳು ನಾವು ಸಂಸಾರದಲ್ಲಿ ವಿಹರಿಸಿ ವಾಸನೆಯನ್ನು ಕಟ್ಟಿಕೊಂಡಿದ್ದೇವೆ. ಅದು ಒಂದೇ ಸಲ ನಮ್ಮನ್ನು ಬಿಟ್ಟು ಹೋಗುವುದಿಲ್ಲ. ಅದರಿಂದ ಬಿಡಿಸಿ ಕೊಳ್ಳಲು ಸತತ ಪ್ರಯತ್ನ ನಡೆಯುತ್ತಿರಬೇಕು.

ಅವನು ಸಮಸ್ತ ಕಾಮನೆಗಳನ್ನು ಬಿಟ್ಟಿರುವನು, ಈ ಲೋಕ ಅಥವಾ ಪರಲೋಕಗಳಲ್ಲಿ ಸಿಕ್ಕಬಹುದಾದ ಯಾವ ಬಾಹ್ಯಸುಖದ ಆಸೆಯೂ ಅವನಲ್ಲಿ ಇಲ್ಲ. ಆಸೆಯೇ ನಮ್ಮನ್ನು ವಿಷಯ ವಸ್ತುವಿನ ಹತ್ತಿರ ಕರೆದುಕೊಂಡು ಹೋಗುವುದು. ಎಷ್ಟು ಅನುಭವ ಕೊಟ್ಟರೂ ಈ ಆಸೆ ಏನೂ ಸಾಕು ಎನ್ನುವುದಿಲ್ಲ. ಇನ್ನೂ ಬೇಕು ಎಂತಲೇ ಕೇಳುತ್ತಿರುವುದು. ಆ ಕೇಳುವುದು ಮತ್ತೂ ಹೆಚ್ಚುವುದು ಅಷ್ಟೇ. ಯಾವ ವಿಧವಾದ ಆಸೆ ಆಗಲಿ, ಅದಕ್ಕೊಂದು ದುಃಖ ಹಿಂಬದಿಯಲ್ಲಿದೆ. ಒಂದಕ್ಕೆ ಕೈಯೊಡ್ಡಿದರೆ ಮತ್ತೊಂದನ್ನು ಸ್ವೀಕರಿಸಬೇಕಾಗಿದೆ. ದೇವರಲ್ಲಿ ಶರಣಾಗತನಾಗಲು ಹೊರಟವನಿಗೆ ಆಸೆಯೊಂದು ಆತಂಕ. ಅದು ನಮ್ಮನ್ನು ಸಂಸಾರದ ಗೂಟಕ್ಕೆ ಕಟ್ಟಿಹಾಕುವುದು. ಇದರಿಂದ ಕಿತ್ತುಕೊಂಡು ಹೋಗಬೇಕಾಗಿದೆ ದೇವರ ಕಡೆ ಹೋಗಬೇಕಾದರೆ.

ಅವನು ಸುಖದುಃಖಗಳ ದ್ವಂದ್ವದಿಂದ ಬಿಡಲ್ಪಟ್ಟವನು. ಈ ಜೀವನದಲ್ಲಿ ಇವೆರಡು ಯಾರನ್ನೂ ಬಿಟ್ಟಿಲ್ಲ. ಒಂದಾದಮೇಲೊಂದು ನಮಗೆ ಬರುವುದು. ಆದರೆ ಯಾವುದೂ ಶಾಶ್ವತವಾಗಿರುವುದಿಲ್ಲ. ಸೇತುವೆಯ ಕೆಳಗೆ ನೀರು ಒಂದು ಕಡೆಯಿಂದ ಬರುವುದು, ಮತ್ತೊಂದು ಕಡೆಯಿಂದ ಹರಿದುಕೊಂಡು ಹೋಗುವುದು. ಹೀಗೆಯೇ ಸುಖದುಃಖ ಯಾವುದೂ ಬಹಳಹೊತ್ತು ಇರುವುದಿಲ್ಲ. ಚಂದ್ರನ ಕಾಂತಿಯಂತೆ ಹೆಚ್ಚು ಕಡಿಮೆಯಾಗುತ್ತ ಹೋಗುತ್ತಿರುವುದು. ಈ ದ್ವಂದ್ವ ಅನುಭವಗಳಿಗೆ ಅವನು ಮನಸ್ಸನ್ನು ಕೊಡುವುದಿಲ್ಲ.

ಈ ಗುಣಗಳಿರುವವನೇ ಜ್ಞಾನಿ, ಬುದ್ಧಿವಂತ, ಜಾಣ. ಭಗವಂತನೆಡೆಗೆ ಹೊರಟಿರುವ ಪ್ರಯಾಣಿಕ ನಲ್ಲಿ ಇರಬೇಕಾದ ಸರಕುಗಳೇ ಇವು. ಇವುಗಳಿಲ್ಲದೆ ನಾವು ತುಂಬಾ ದೂರ ನಡೆದುಕೊಂಡು ಹೋಗುವುದಕ್ಕೆ ಆಗುವುದಿಲ್ಲ. ಈ ಬುತ್ತಿ ಸರಿಯಾಗಿದ್ದರೆ ಭಗವಂತನೆಂಬ ಅವ್ಯಯವಾದ ಯಾವ ಬದಲಾವಣೆಗೂ ಸಿಕ್ಕದ, ನಿತ್ಯವಾದ ಶಾಶ್ವತವಾದ, ಸಚ್ಚಿದಾನಂದ ಅನುಭವವನ್ನು ಪಡೆಯುತ್ತೇವೆ.

\begin{verse}
ನ ತದ್ಭಾಸಯತೇ ಸೂರ್ಯೋ ನ ಶಶಾಂಕೋ ನ ಪಾವಕಃ ।\\ಯದ್ಗತ್ವಾ ನ ನಿವರ್ತಂತೇ ತದ್ಧಾಮ ಪರಮಂ ಮಮ \versenum{॥ ೬ ॥}
\end{verse}

{\small ಅದನ್ನು ಸೂರ್ಯ ಬೆಳಗುವುದಿಲ್ಲ. ಚಂದ್ರನಾಗಲಿ ಅಗ್ನಿಯಾಗಲಿ ಬೆಳಗುವುದಿಲ್ಲ. ಯಾವುದನ್ನು ಹೊಂದಿದರೆ ಹಿಂತಿರುಗುವುದಿಲ್ಲವೋ ಅದು ನನ್ನ ಶ್ರೇಷ್ಠವಾದ ಸ್ಥಾನ.}

ಸೂರ್ಯ, ಚಂದ್ರ, ಅಗ್ನಿ ಇವುಗಳಾವುವೂ ಅಲ್ಲಿ ಬೆಳಗುವುದಿಲ್ಲ. ಇವುಗಳೆಲ್ಲ ಅಲ್ಲಿ ಮ್ಲಾನವಾಗು ವುವು. ಹಗಲುಹೊತ್ತು ನಕ್ಷತ್ರಗಳು ಹೇಗೋ ಹಾಗೆ ಆಗುವುದು. ನಕ್ಷತ್ರಗಳೇನೊ ಬೆಳಗುತ್ತಿವೆ, ಸೂರ್ಯಪ್ರಭೆಯ ಎದುರಿಗೆ ಅವುಗಳಾವುವೂ ಕಾಣುವುದಿಲ್ಲ. ಹಾಗೆಯೇ ಪರಮಾತ್ಮನ ಜ್ಯೋತಿ ಎದುರಿಗೆ ಉಳಿದ ಕಾಂತಿಗಳೆಲ್ಲ ಸಪ್ಪೆಯಾಗುವುವು. ಇದು ಬರೀ ಭೌತಿಕವಾದ ಕಾಂತಿಯಲ್ಲ. ಕೇವಲ ಉದಾಹರಣೆಗಾಗಿ ಇದನ್ನು ಕೊಟ್ಟಿದೆ. ಅವನದು ಚಿತ್​ಪ್ರಭೆ. ಅವನ ಮುಂದೆ ಉಳಿದವುಗಳಾವುವೂ ತಮ್ಮ ತಲೆಯನ್ನು ಎತ್ತಲಾರವು.

ಯಾವುದನ್ನು ಪಡೆದರೆ ನಾವು ಹಿಂತಿರುಗುವುದಿಲ್ಲವೋ ಅದೇ ಅವನ ಸ್ಥಾನ. ನಾವು ಒಂದೇ ಸಲ ಮುಕ್ತರಾಗಿ ಹೋಗುತ್ತೇವೆ. ನಮ್ಮ ಹೃದಯದ ವಾಸನೆಗಳು ಮತ್ತು ಸಂಶಯಗಳೆಲ್ಲ ಪರಮಾತ್ಮನ ಸಾಕ್ಷಾತ್ಕಾರದ ಜ್ಞಾನಾಗ್ನಿಯಲ್ಲಿ ದಗ್ಧವಾಗಿ ಹೋಗುತ್ತವೆ. ವಾಸನೆಯ ಬಾಕಿ ಇದ್ದರೆ ತಾನೆ ನಾವು ತೀರಿಸಲು ಈ ಪ್ರಪಂಚಕ್ಕೆ ಬರಬೇಕು. ಅವನು ಪುಣಮುಕ್ತನಾಗಿ ಹೋಗುತ್ತಾನೆ.

ಪರಮಾತ್ಮನ ಸಾನ್ನಿಧ್ಯ ಸ್ವರ್ಗಾದಿ ಲೋಕಗಳಿಗಿಂತಲೂ ಶ್ರೇಷ್ಠವಾದ ಸ್ಥಾನ. ಸ್ವರ್ಗಲೋಕದಲ್ಲಿ ಈ ಪ್ರಪಂಚದಲ್ಲಿ ನಾವು ಸಂಪಾದಿಸಿಕೊಂಡು ಹೋಗಿರುವ ಪುಣ್ಯ ತೀರುವವರೆಗೆ ಮಾತ್ರ ಇರಬಹುದು. ಅದು ತೀರಿತು ಎಂದರೆ ಪುನಃ ನಾವು ಈ ಭೂಮಿಗೆ ಉರುಳಬೇಕಾಗುವುದು. ಆದರೆ ಮುಕ್ತಿ ಹಾಗಲ್ಲ. ನಾವು ಎಂದೆಂದಿಗೂ ಈ ಪ್ರಪಂಚದಿಂದ ಬಿಟ್ಟುಹೋಗುವೆವು.

\begin{verse}
ಮಮೈವಾಂಶೋ ಜೀವಲೋಕೇ ಜೀವಭೂತಃ ಸನಾತನಃ ।\\ಮನಃಷಷ್ಠಾನೀಂದ್ರಿಯಾಣಿ ಪ್ರಕೃತಿಸ್ಥಾನಿ ಕರ್ಷತಿ \versenum{॥ ೭ ॥}
\end{verse}

{\small ಸನಾತನವಾದ ನನ್ನ ಅಂಶವೇ ಸಂಸಾರದಲ್ಲಿ ಜೀವನಾಗಿ ಪ್ರಕೃತಿಯಲ್ಲಿರುವ ಆರನೆಯದಾದ ಮನಸ್ಸನ್ನು ಸೆಳೆದುಕೊಳ್ಳುವುದು.}

ಇಲ್ಲಿ ಜೀವಾತ್ಮನಿಗೂ ಪರಮಾತ್ಮನಿಗೂ ಇರುವ ಸಂಬಂಧವನ್ನು ವಿವರಿಸುವನು. ಜೀವಾತ್ಮ ಪರಮಾತ್ಮನಿಂದ ಬಂದವನು. ಒಂದು ಉರಿಯುತ್ತಿರುವ ಅಗ್ನಿಕುಂಡದಿಂದ ಕಿಡಿ ಹೇಗೆ ಸಿಡಿಯು ವುದೋ ಹಾಗೆ. ಸೂರ್ಯನಿಂದ ಹಲವಾರು ಕಿರಣಗಳು ಹೇಗೆ ಬರುತ್ತಿವೆಯೊ ಹಾಗೆ ಪ್ರತಿಯೊಂದು ಕಿಡಿಯಲ್ಲಿಯೂ, ಪ್ರತಿಯೊಂದು ಕಿರಣದಲ್ಲಿಯೂ ಮೂಲದ ಕಾಂತಿ ಒಂದು ಪ್ರಮಾಣದಲ್ಲಿದೆ. ಅದೆಂದಿಗೂ ಆರದ ಕಾಂತಿ; ಆರದ ಕಾವು. ಆದರೆ ಈಗ ಅದು ಕಾಣದಂತೆ ಇದೆ. ಮೇಲೆ ಬೂದಿ ಕವಿದುಕೊಂಡಿದೆ. ನಾವು ಆಧ್ಯಾತ್ಮಿಕ ಜೀವನದಲ್ಲಿ ಮಾಡುವ ಸಾಧನೆಯೆಲ್ಲ, ಮೇಲೆ ಕವಿದ ಬೂದಿಯನ್ನು ಕೊಡಹುವುದಾಗಿದೆ. ಜೀವಾತ್ಮ ಪರಮಾತ್ಮನಿಂದ ಬೇರೆ ಇರುತ್ತಾನೆ ಎಂದು ಭಾವಿ ಸುವುದು ಭ್ರಾಂತಿಯಿಂದ. ಕಿರಣ ಸೂರ್ಯನಿಲ್ಲದೆ ಇರಲಾರದು. ಯಾವಾಗಲೂ ಕಿರಣ ಬರುತ್ತಿರುವುದು ಸೂರ್ಯನಿಂದಲೇ. ಈಗ ನಮಗೂ ಅವನಿಗೂ ಸಂಬಂಧ ಕಡಿದು ಹೋಗಿರುವಂತೆ ಕಾಣುತ್ತಿದೆ. ಆದರೆ ನಿಜವಾಗಿ ಇದು ಎಂದಿಗೂ ಕಡಿದುಹೋಗಿಲ್ಲ. ಇದು ನಮ್ಮ ಒಂದು ಭ್ರಮೆ. ಈ ಭ್ರಮೆಯಿಂದ ಪಾರಾದರೆ ನಮ್ಮ ನೈಜಸ್ಥಿತಿ ನಮಗೆ ಅರಿವಾಗುವುದು.

ನಾವೆಲ್ಲ ಸನಾತನವಾದ ಭಗವಂತನ ಅಂಶವೇ. ಹೇಗೆ ಅವನು ಸನಾತನನೊ ನಾವು ಹಾಗೆ. ಪೂರ್ಣಕ್ಕೆ ಇರುವ ಗುಣವೇ ಅಂಶಕ್ಕೂ ಇರುವುದು. ಅವನಿಂದ ಸಿಡಿದು ಬಂದ ಕಿಡಿಯೇ ಜೀವ. ಈ ಜೀವ ದೇಶಕಾಲನಿಮಿತ್ತದ ಪ್ರಪಂಚದಲ್ಲಿ ಬಿದ್ದಿದೆ. ಇದು ಪಂಚೇಂದ್ರಿಯಗಳು ಮತ್ತು ಮನಸ್ಸೆಂಬ ಆರನೆಯದನ್ನು ಪಡೆದು ಒಂದು ದೇಹವೆಂಬ ಗೂಡನ್ನು ಕಟ್ಟಿಕೊಂಡಿದೆ. ತನ್ನ ನೈಜಸ್ಥಿತಿಯನ್ನು ಮರೆತು ಆ ಗೂಡಿನ ಧರ್ಮವನ್ನು ತನ್ನ ಮೇಲೆ ಆರೋಪಮಾಡಿಕೊಂಡು ಈಗ ಬಂಧನಕ್ಕೆ ಸಿಕ್ಕಿ ನರಳುತ್ತಿದೆ. ಆ ಗೂಡು ಹಿಂದೆ ಇರಲಿಲ್ಲ. ಇನ್ನು ಕೊಂಚ ಕಾಲವಾದ ಮೇಲೆ ಇರುವುದಿಲ್ಲ. ಮಧ್ಯದಲ್ಲಿ ಇರುವಾಗಲೂ ಕಾಲಕಾಲಕ್ಕೆ ಬದಲಾಯಿಸುತ್ತಿದೆ. ಈ ವಿಕಾರಗಳೆಲ್ಲ ಆಗುತ್ತಿರುವುದು ಆ ಗೂಡಿಗೆ ಹೊರತು, ಆ ಗೂಡಿನಲ್ಲಿ ಇರುವ ಜೀವಾತ್ಮನಿಗಲ್ಲ. ಮನೆಗೂ ಮತ್ತು ಮನೆಯಲ್ಲಿ ವಾಸ ಮಾಡುವವರಿಗೂ ಇರುವ ವ್ಯತ್ಯಾಸದಂತೆ ಇದು. ಇವೆರಡೂ ಸಂಪೂರ್ಣ ಬೇರೆ. ಹಾಗೆಯೇ ಜೀವ ಮತ್ತು ದೇಹ. ಜೀವ ದೇಹವನ್ನು ಕಟ್ಟಿಕೊಳ್ಳುವುದು. ಶಂಖದ ಹುಳು ತನ್ನ ಮನೆಯನ್ನು ತಾನೇ ಕಟ್ಟಿಕೊಂಡು ಅದರೊಳಗೆ ವಾಸಮಾಡುತ್ತ, ಹೋದ ಹೋದ ಕಡೆ ಅದು ತನ್ನ ಮನೆಯನ್ನು ಎಳೆದುಕೊಂಡು ಹೋಗುವಂತೆ ನಾವು ದೇಹವನ್ನು ಎಳೆದುಕೊಂಡು ಹೋಗುತ್ತಿರುವೆವು.

\begin{verse}
ಶರೀರಂ ಯದವಾಪ್ನೋತಿ ಯಚ್ಚಾಪ್ಯುತ್ಕಾ ್ರಮತೀಶ್ವರಃ ।\\ಗೃಹೀತ್ವೈತಾನಿ ಸಂಯಾತಿ ವಾಯುರ್ಗಂಧಾನಿವಾಶಯಾತ್ \versenum{॥ ೮ ॥}
\end{verse}

{\small ಜೀವನು ತನ್ನ ಶರೀರವನ್ನು ಯಾವಾಗ ಪ್ರವೇಶಿಸುತ್ತಾನೆಯೋ ಮತ್ತು ಯಾವಾಗ ಬಿಡುತ್ತಾನೆಯೋ ಆಗ ಗಾಳಿ ವಾಸನೆಯನ್ನು ತೆಗೆದುಕೊಂಡು ಹೋಗುವ ಹಾಗೆ ಇವುಗಳನ್ನು ತೆಗೆದುಕೊಂಡು ಹೋಗುವನು.}

ಜೀವಿ ಈ ದೇಹವನ್ನು ಪ್ರವೇಶಿಸುವಾಗ ತನ್ನ ಸೂಕ್ಷ್ಮ ಇಂದ್ರಿಯ ಮತ್ತು ಮನಸ್ಸು ಇವುಗಳನ್ನು ತರುವನು. ಹೇಗೆ ನಾವು ಒಂದು ಮನೆಯನ್ನು ಬಾಡಿಗೆಗೆ ತೆಗೆದುಕೊಂಡರೆ, ನಾವು ಹಿಂದಿನ ಮನೆಯಲ್ಲಿ ಉಪಯೋಗಿಸುತ್ತಿದ್ದ ಸಾಮಾನುಗಳನ್ನು ತೆಗೆದುಕೊಂಡು ಬರುತ್ತೇವೆಯೋ ಹಾಗೆ. ಈಗಿನ ನಮ್ಮ ದೇಹದಲ್ಲಿ ಸ್ಥೂಲ ಇಂದ್ರಿಯಗಳಿವೆ. ಆದರೆ ನಿಜವಾಗಿ ಕೆಲಸ ಮಾಡುವುದು, ಅದರ ಹಿಂದೆ ಇರುವ ಸೂಕ್ಷ್ಮ ಇಂದ್ರಿಯಗಳು. ಏಕೆಂದರೆ ನಾವು ನಿದ್ರಿಸುವಾಗಲೂ ಈ ಸ್ಥೂಲೇಂದ್ರಿಯ ಗಳೆಲ್ಲ ಇರುತ್ತವೆ. ಅವುಗಳ ಹಿಂದೆ ಜೀವಿ ಇಲ್ಲದೆ ಇದ್ದರೆ ಇವುಗಳಿಂದ ಏನೂ ಪ್ರಯೋಜನವಿಲ್ಲ. ಮನುಷ್ಯ ಈ ದೇಹವನ್ನು ಬಿಟ್ಟು ಹೋಗುವಾಗಲೂ ಸ್ಥೂಲ ದೇಹ ಮತ್ತು ಇಂದ್ರಿಯಗಳನ್ನು ಇಲ್ಲೇ ಬಿಟ್ಟು ಹೋಗುವನು. ಅವನು ತೆಗೆದುಕೊಂಡು ಹೋಗುವುದು ಸೂಕ್ಷ್ಮೇಂದ್ರಿಯಗಳು ಮತ್ತು ಮನಸ್ಸು ಇವುಗಳನ್ನು. ನಾವು ಕನಸಿಗೆ ಪ್ರತಿದಿನ ಹೋಗುತ್ತೇವೆ. ಹೋಗುವಾಗ ಈ ಸ್ಥೂಲದೇಹ ಮತ್ತು ಇಂದ್ರಿಯಗಳನ್ನು ಇಲ್ಲೇ ಬಿಟ್ಟು, ಕನಸಿನಲ್ಲಿ ಬೇರೆ ನಮ್ಮ ಸೂಕ್ಷ್ಮೇಂದ್ರಿಯಗಳನ್ನು ಉಪಯೋಗಿಸಿಕೊಳ್ಳುತ್ತೇವೆ. ಹಾಗೆಯೇ ಸಾವು ಎಂಬುದೂ ಕೂಡಾ. ಕನಸಿಗೂ ಮರಣಕ್ಕೂ ಇರುವ ದೊಡ್ಡ ವ್ಯತ್ಯಾಸವೇ ಇದು. ನಾವು ಕನಸಿಗೆ ಹೋದರೆ ಅದೇ ದೇಹದಲ್ಲಿ ಏಳುತ್ತೇವೆ. ಮರಣಕ್ಕೆ ಹೋದರೆ ಬೇರೆ ದೇಹದಲ್ಲಿ ಏಳುತ್ತೇವೆ. ಇದೊಂದೇ ವ್ಯತ್ಯಾಸ. ಉಳಿದ ಕ್ರಮಗಳೆಲ್ಲ ಒಂದೇ. ಮರಣದ ಹಿಂದೆ ಅಷ್ಟೊಂದು ಭಯಾನಕವಾದ ಭಾವನೆಗಳನ್ನೆಲ್ಲ ಕಟ್ಟಿಕೊಂಡಿದ್ದೇವೆ. ಕನಸಿನ ಹಿಂದೆ ಯಾವ ಅಂಜಿಕೆಯೂ ಇಲ್ಲ, ಅಸ್ವಾಭಾವಿಕತೆಯೂ ಇಲ್ಲ. ನಿದ್ರೆಯೆನ್ನುವುದು ನಿತ್ಯ ಸಾವು. ಮರಣವೆನ್ನುವುದು ನೈಮಿತ್ತಿಕ ಸಾವು.

ಒಂದು ದೇಹವನ್ನು ಬಿಟ್ಟು ಮತ್ತೊಂದು ದೇಹಕ್ಕೆ ಹೋಗುವಾಗ ಜೀವಿ ಹಿಂದಿನ ದೇಹದಲ್ಲಿ ಮಾಡಿದ ಕರ್ಮಗಳ ಸಾರವನ್ನೆಲ್ಲ ಹೊತ್ತುಕೊಂಡು ಹೋಗುವುದು. ಗಾಳಿ ವಾಸನೆಯನ್ನು ಹೇಗೆ ಹೊತ್ತುಕೊಂಡು ಹೋಗುವುದೋ ಹಾಗೆ. ಉಪನಿಷತ್ತಿನಲ್ಲಿ ಬೇರೊಂದು ಉದಾಹರಣೆಯನ್ನು ಕೊಡುವರು. ರಾಣಿಜೇನು ಒಂದು ಗೂಡಿನಿಂದ ಮತ್ತೊಂದು ಕಡೆಗೆ ಹೋಗುವಾಗ ಅದರ ಅನುಯಾಯಿಗಳೆಲ್ಲ ಹೇಗೆ ಅದನ್ನು ಅನುಸರಿಸಿ ಹೋಗುವುವೋ ಹಾಗೆ ಜೀವನ ಹಿಂದೆ ಅವನ ಮನಸ್ಸು, ಸೂಕ್ಷ್ಮ ಇಂದ್ರಿಯ ಮತ್ತು ಅವನು ಸಂಪಾದಿಸಿದ ಸಂಸ್ಕಾರಗಳು ಎಲ್ಲವೂ ಹೋಗುವುವು. ಹಾಗೆಯೇ ರಾಣಿಜೇನು ಒಂದು ಅನುಕೂಲವಾದ ಸ್ಥಳದಲ್ಲಿ ಬೀಡು ಬಿಡುವಾಗ ಇತರ ಜೇನುನೊಣ ಗಳೆಲ್ಲ ಅದಕ್ಕೆ ಅಲ್ಲೊಂದು ಗೂಡು ಕಟ್ಟುವುವು. ಅದರಂತೆಯೇ ನಾವು ಪ್ರಪಂಚಕ್ಕೆ ಮಗುವಿನಂತೆ ಬರುವೆವು. ಅದರ ಸರಳತೆಯನ್ನು ಮುಗ್ಧ ಮುಖವನ್ನು ನೋಡಿ ಏನೂ ಅರಿಯದ ಹಸುಳೆ ಎನ್ನುವೆವು. ಆದರೆ ಅದೇನು ಬರೀ ಕೈಯಲ್ಲಿ ಬಂದಿಲ್ಲ. ಒಂದು ಕರ್ಮದ ಹೊರೆಯನ್ನು ತಂದಿದೆ. ಕಾಲ ಕಳೆದಂತೆ ಆ ಗಂಟನ್ನು ಬಿಚ್ಚುತ್ತಾ ಹೋಗುವುದು. ಯಾರೂ ಖಾಲಿ ಬರುವುದಿಲ್ಲ, ಖಾಲಿ ಹೋಗುವುದಿಲ್ಲ. ಒಂದು ದೊಡ್ಡ ಕರ್ಮದ ಲಾರಿಯೇ ನಮ್ಮೊಡನೆ ಇರುವುದು. ಇದು ನಮ್ಮ ಸ್ಥೂಲ ಕಣ್ಣಿಗೆ ಕಾಣುವುದಿಲ್ಲ ಅಷ್ಟೇ.

\begin{verse}
ಶ್ರೋತ್ರಂ ಚಕ್ಷುಃ ಸ್ಪರ್ಶನಂ ಚ ರಸನಂ ಘ್ರಾಣಮೇವ ಚ ।\\ಅಧಿಷ್ಠಾಯ ಮನಶ್ಚಾಯಂ ವಿಷಯಾನುಪಸೇವತೇ \versenum{॥ ೯ ॥}
\end{verse}

{\small ಜೀವನು ಕಿವಿ, ಕಣ್ಣು, ಚರ್ಮ, ನಾಲಗೆ, ಮೂಗು, ಮನಸ್ಸನ್ನು ಆಶ್ರಯ ಪಡೆದು ವಿಷಯಗಳನ್ನು ಅನುಭವಿಸು ತ್ತಾನೆ.}

ಮುಂಚೆ ನಿರ್ದೇಹಿಯಾದ ಜೀವಿ ದೇಹವನ್ನು ಪ್ರವೇಶಿಸಿ ಆದಮೇಲೆ ತನ್ನ ದೇಹದಲ್ಲಿರುವ ಇಂದ್ರಿಯದ ಮೂಲಕ ಹೊರಗಡೆ ಇರುವ ವಿಷಯ ವಸ್ತುಗಳನ್ನು ಅನುಭವಿಸುವುದು. ಈ ಇಂದ್ರಿಯದ ಹಿಂದೆಲ್ಲ ಮನಸ್ಸು ಇದ್ದರೇನೇ ಅದು ಹೊರಗಿನಿಂದ ಬರುವ ವೇದನೆಯನ್ನು ಜೋಡಿಸಿ ಅನುಭವಿಸಲು ಸಾಧ್ಯ. ನಾವು ಒಂದು ಮನೆಯಲ್ಲಿ ಕುಳಿತುಕೊಂಡು ಕಿಟಕಿಯ ಮೂಲಕ ಹೊರಗೆ ಏನಾಗುತ್ತಿದೆಯೋ ಅದನ್ನು ನೋಡುವಂತೆ ಇಂದ್ರಿಯಗಳ ಕಿಟಕಿ ಮತ್ತು ಗವಾಕ್ಷಿಗಳ ಮೂಲಕ ಹೊರಗಿನದನ್ನು ಅರಿಯುತ್ತೇವೆ. ಕಿವಿಗಳ ಮೂಲಕ ಕೇಳುತ್ತೇವೆ. ಕಣ್ಣಿನ ಮೂಲಕ ನೋಡುತ್ತೇವೆ. ನಾಲಗೆಯ ಮೂಲಕ ರುಚಿ ನೋಡುತ್ತೇವೆ. ಮೂಗಿನ ಮೂಲಕ ವಾಸನೆ ಕಂಡುಹಿಡಿಯುತ್ತೇವೆ. ಚರ್ಮದ ಮೂಲಕ ಶೀತೋಷ್ಣ, ಮೃದು ಗಟ್ಟಿ ಮುಂತಾದುವನ್ನು ಅರಿಯುತ್ತೇವೆ. ಈ ಅನುಭವ ಗಳೇ ದೇಹದ ಒಳಗಿರುವ ಜೀವಿಗೆ ಆಹಾರ ಇದ್ದಹಾಗೆ. ನಮ್ಮ ಇಂದ್ರಿಯಗಳೆಲ್ಲಾ ಹೊರಗಿನಿಂದ ಬರುವುದನ್ನು ಸ್ವೀಕರಿಸುವುದಕ್ಕೆ ಸರಿಯಾಗಿ ರಚಿಸಲ್ಪಟ್ಟಿವೆ.

\begin{verse}
ಉತ್ಕ್ರಾಮಂತಂ ಸ್ಥಿತಂ ವಾಪಿ ಭುಂಜಾನಂ ವಾ ಗುಣಾನ್ವಿತಮ್ ।\\ವಿಮೂಢಾ ನಾನುಪಶ್ಯಂತಿ ಪಶ್ಯಂತಿ ಜ್ಞಾನಚಕ್ಷುಷಃ \versenum{॥ ೧೦ ॥}
\end{verse}

{\small ಈ ದೇಹವನ್ನು ಬಿಡುವಾಗ, ಇದರಲ್ಲೆ ಇರುವಾಗ ಅಥವಾ ವಿಷಯಗಳನ್ನು ಅನುಭವಿಸುವಾಗ ಅಥವಾ ಗುಣಗಳಿಂದ ಕೂಡಿರುವಾಗ ಜೀವನನ್ನು ಮೂಢರು ನೋಡಲಾರರು. ಜ್ಞಾನಿಗಳು ನೋಡುತ್ತಾರೆ.}

ಈ ಜೀವ ನಮ್ಮ ಹತ್ತಿರಕ್ಕೆ ಹತ್ತಿರ. ದೇಹೇಂದ್ರಿಯಗಳೆಲ್ಲಕ್ಕಿಂತಲೂ ಹತ್ತಿರವಾಗಿರುವುದು ಅದು. ಆದರೂ ನಮಗೆ ಪತ್ತೆಯೇ ಇಲ್ಲ. ನಮ್ಮ ದೇಹದ ಮೇಲೆ ಒಂದು ಕೋಟನ್ನು ಹಾಕಿಕೊಂಡು ಯಾವಾಗಲೂ ಕೋಟನ್ನು ನೋಡಿಕೊಳ್ಳುತ್ತೇವೆಯೆ ಹೊರತು ಕೋಟಿನ ಹಿಂದೆ ಇರುವ ದೇಹವನ್ನು ನೋಡಿಕೊಳ್ಳದಂತೆ ಇದು. ಈ ದೇಹದಲ್ಲೇ ಜೀವಿ ಇರುವನು. ಅಜ್ಞಾನಿ ಈ ದೇಹವನ್ನೇ ಜೀವ ಎಂದು ಭಾವಿಸುತ್ತಾನೆ. ಜ್ಞಾನಿ ಜೀವ ಬೇರೆ, ದೇಹ ಬೇರೆ ಎಂಬುದನ್ನು ಚೆನ್ನಾಗಿ ಅರಿತಿರುವನು. ಜೀವ ಈ ದೇಹವನ್ನು ಬಿಟ್ಟು ಹೋಗುವಾಗ ಜ್ಞಾನಿ ಇದನ್ನು ಅರಿಯುವನು. ಇದು ಒರೆಯಿಂದ ಸೆಳೆದ ಕತ್ತಿಯಂತೆ ಎಂದು ಶ್ರೀರಾಮಕೃಷ್ಣರು ಹೇಳುತ್ತಿದ್ದರು. ಒರೆ ಇಲ್ಲೆ ಇರುವುದು. ಅದರ ಹಿಂದೆ ಕತ್ತಿ ಇರುವುದಿಲ್ಲ. ದೇಹ ಇಲ್ಲೇ ಇರುವುದು. ಅದರ ಹಿಂದೆ ಜೀವ ಮಾತ್ರ ಹಾರಿಹೋಗಿದೆ, ಹಕ್ಕಿ ಗೂಡನ್ನು ಹಿಂದೆ ಬಿಟ್ಟು ಹಾರಿ ಹೋದಂತೆ. ಸಾಯುತ್ತಿರುವ ಮನುಷ್ಯನನ್ನು ಎಲ್ಲರೂ ನೋಡುತ್ತಾರೆ. ಜ್ಞಾನಿ ಮಾತ್ರ ಜೀವ ದೇಹದಿಂದ ಬೇರೆ ಆಗುತ್ತಿರುವುದನ್ನು ನೋಡಬಲ್ಲ. ಅಜ್ಞಾನಿಯು ಸಾಯುವ ಮನುಷ್ಯನಲ್ಲಿ ಇಂದ್ರಿಯಗಳಾದ ಮೇಲೆ ಇಂದ್ರಿಯಗಳು ಕೆಲಸ ಮಾಡದೆ ಇರುವುದನ್ನು ನೋಡು ತ್ತಾನೆ. ಆದರೆ ಅದರ ಹಿಂದೆ ಇರುವ ಜೀವ ಇವುಗಳಿಂದ ಬೇರೆಯಾಗುವುದನ್ನು ನೋಡಲಾರ. ಜ್ಞಾನಿ ಈ ಸೂಕ್ಷ್ಮಾತಿಸೂಕ್ಷ್ಮವಾದ ಜೀವವನ್ನೂ ಅರಿಯಬಲ್ಲ.

ದೇಹದಲ್ಲಿ ಜೀವ ಇರುವಾಗ ಒಂದೊಂದು ಸಲ ಒಂದೊಂದು ಗುಣಕ್ಕೆ ವಶನಾಗಿ ಹಲವು ಇಂದ್ರಿಯಗಳ ಮೂಲಕ ಅನುಭವಿಸುವುದನ್ನು ಜ್ಞಾನಿ ಸ್ಪಷ್ಟವಾಗಿ ನೋಡಬಲ್ಲ. ತನ್ನಲ್ಲಿಯೇ, ತಾನು ಬೇರೆ, ಇಂದ್ರಿಯಗಳು ಬೇರೆ, ಎಂಬ ಭಾವ ಯಾವಾಗಲೂ ಅವನಲ್ಲಿರುತ್ತದೆ. ಅವನು ಇಂದ್ರಿಯ ಗಳನ್ನು ಉಪಯೋಗಿಸುವನು. ಆದರೆ ಅದರಲ್ಲೇ ತಾದಾತ್ಮ್ಯಭಾವವನ್ನು ಪಡೆದುಕೊಂಡಿಲ್ಲ. ನಾವು ನಡೆಯುವಾಗ ಕಾಲಿಗೆ ಎಕ್ಕಡ ಉಪಯೋಗಿಸುತ್ತೇವೆ. ಮನೆ ಒಳಗೆ ಬಂದರೆ ಅದನ್ನು ತೆಗೆದು ಇಡುತ್ತೇವೆ. ಹಾಗೆಯೇ ಛತ್ರಿ, ಕೈಯಲ್ಲಿ ಹಿಡಿದುಕೊಳ್ಳುವ ಕೋಲು ಮುಂತಾದುವು. ಜ್ಞಾನಿ ಯಾವಾಗಲೂ ಇವುಗಳಿಂದ ಬೇರೆಯಾಗಿರುವನು. ಅವನ ಆಚಾರ ವ್ಯವಹಾರವನ್ನು ನೋಡಿದರೆ ಇದು ಗೊತ್ತಾಗುವುದು. ಇವನು ನೀರಿನ ಮೇಲೆ ಎಣ್ಣೆಯನ್ನು ಹಾಕಿದಂತೆ. ನೀರಿಗೂ ಎಣ್ಣೆಗೂ ಯಾವ ಸಂಬಂಧವೂ ಇರುವುದಿಲ್ಲ. ಎಣ್ಣೆ ಮೇಲೆ ತೇಲುತ್ತಿರುವುದು, ನೀರು ಕೆಳಗಿರುವುದು. ಜ್ಞಾನಿಗಳು ಯಾವಾಗಲೂ ಜೀವನನ್ನು ದೇಹೇಂದ್ರಿಯಗಳಿಂದ ಬೇರೆ ಮಾಡಿಕೊಂಡು ಅದನ್ನು ಉಪಯೋಗಿಸು ವರು. ಅವರೆಂದಿಗೂ ಇದಕ್ಕೆ ದಾಸರಲ್ಲ. ಜ್ಞಾನಿಯೂ ತಿನ್ನುತ್ತಾನೆ. ಆದರೆ ಅವನೊಬ್ಬ ಹೊಟ್ಟೆಬಾಕ ನಂತೆ ತಿನ್ನುವುದಿಲ್ಲ. ಅವನೂ ನೋಡುತ್ತಾನೆ, ಕೇಳುತ್ತಾನೆ. ಆದರೆ ಅಲ್ಲೆ ಅಂಟಿಹೋಗಿಬಿಡುವುದಿಲ್ಲ. ಜ್ಞಾನಿ ಬೇಕಾದರೆ ಒಂದು ಇಂದ್ರಿಯವನ್ನು ಉಪಯೋಗಿಸುತ್ತಾನೆ, ಬೇಡದೆ ಇದ್ದರೆ ಆ ವಿಷಯವಸ್ತು ಹತ್ತಿರ ಇದ್ದರೂ ಆ ಇಂದ್ರಿಯವನ್ನು ಉಪಯೋಗಿಸುವುದಿಲ್ಲ. ಅಜ್ಞಾನಿ ಹಾಗಲ್ಲ. ವಿಷಯವಸ್ತು ಹತ್ತಿರ ಬಂತು ಎಂದರೆ ಇಂದ್ರಿಯ ಆಗಲೆ ಅದರ ಹತ್ತಿರ ನೆಗೆಯುವುದು, ಸಂಬಂಧವನ್ನು ಕಲ್ಪಿಸಿಕೊಳ್ಳುವುದು. ಅದೇನು ಯಾರನ್ನೂ ಹೇಳುವುದಿಲ್ಲ, ಕೇಳುವುದಿಲ್ಲ. ಜೀವ ಇಂತಹ ಒಂದು ಸದರವನ್ನು ಕೊಟ್ಟಿದೆ. ಅದನ್ನು ನಿಗ್ರಹಿಸಲಾರದು, ಎಳೆದತ್ತ ಸಾಗುವುದು. ಜ್ಞಾನಿಯೂ ಈ ದೇಹದ ಗೂಡಿನಲ್ಲಿರುವನು, ಅಜ್ಞಾನಿಯೂ ಈ ದೇಹದ ಗೂಡಿನಲ್ಲಿರುವನು. ಆದರೆ ಜ್ಞಾನಿ, ಕೊಬ್ಬರಿ ಗಿಟುಕು ತೆಂಗಿನ ಕರಟದಲ್ಲಿರುವಂತೆ ಇರುವನು. ಅವನು ಚಿಪ್ಪಿನಿಂದ ಬೇರೆಯಾಗಿರುವನು. ದೇಹಕ್ಕೂ ಜೀವಕ್ಕೂ ಇರುವ ಸಂಬಂಧ ಬಿಟ್ಟಿದೆ. ಅಜ್ಞಾನಿಯಾದರೋ ಇನ್ನೂ ಎಳೆನೀರಿನಂತೆ ಇರುವನು. ಅದರಲ್ಲಿರುವ ಸ್ವಲ್ಪ ತಿರಳು ಕೂಡ ಕರಟಕ್ಕೆ ಭದ್ರವಾಗಿ ಅಂಟಿಕೊಂಡಿದೆ. ಅದರಿಂದ ಬೇರೆ ಆಗಿಲ್ಲ ಇನ್ನೂ.

\begin{verse}
ಯತಂತೋ ಯೋಗಿನಶ್ಚೈನಂ ಪಶ್ಯಂತ್ಯಾತ್ಮನ್ಯವಸ್ಥಿತಮ್ ।\\ಯತಂತೋಽಪ್ಯಕೃತಾತ್ಮಾನೋ ನೈನಂ ಪಶ್ಯಂತ್ಯಚೇತಸಃ \versenum{॥ ೧೧ ॥}
\end{verse}

{\small ಪ್ರಯತ್ನ ಮಾಡುತ್ತಿರುವ ಯೋಗಿಗಳು, ಈ ಜೀವನು ತಮ್ಮಲ್ಲಿರುವವನೆಂದು ನೋಡುತ್ತಾರೆ. ಇಂದ್ರಿಯಗಳನ್ನು ಜಯಿಸದ ಅವಿವೇಕಿಗಳು ಪ್ರಯತ್ನ ಮಾಡಿದರೂ ಇದನ್ನು ನೋಡಲಾರರು.}

ಸಾಧನೆ ಮಾಡುತ್ತಿರುವ ಯೋಗಿಗಳಿಗೆ ಜೀವ ಬೇರೆ, ದೇಹ ಬೇರೆ ಎಂಬುದು ಸ್ಪಷ್ಟವಾಗಿ ಕಾಣುವುದು. ಅವರು ತಮ್ಮ ಇಂದ್ರಿಯಗಳನ್ನು ನಿಗ್ರಹಿಸಿರುವರು. ಅವರ ಮನಸ್ಸು ಶುದ್ಧವಾಗಿದೆ; ಬುದ್ಧಿ ಹರಿತವಾಗಿದೆ. ಬಹಳ ಸೂಕ್ಷ್ಮವಾದ ವಿಷಯಗಳನ್ನು ಅವರು ಗ್ರಹಿಸುವ ಸ್ಥಿತಿಯಲ್ಲಿರುವರು. ಒಂದು ಪಾತ್ರೆಯಲ್ಲಿ ನೀರು ಹಾಕಿ ಅದರ ಕೆಳಗೆ ಏನಾದರೂ ಇದ್ದರೆ, ನೀರು ಶಾಂತವಾಗಿರುವಾಗ ಅದು ಕಾಣುವುದು. ನೀರು ಅಲ್ಲಾಡುತ್ತಿದ್ದರೆ ಕಾಣುವುದಿಲ್ಲ. ಹಾಗೆಯೇ ಜ್ಞಾನಿಯ ಚಿತ್ತದ ಪಾತ್ರೆ ಶಾಂತವಾಗಿದೆ. ಹಿಂದೆ ಇರುವ ಜೀವ ಬೇರೆ ಎಂಬುದನ್ನು ನೋಡುತ್ತಾನೆ.

ಅಜ್ಞಾನಿಯ ಮನಸ್ಸು ಶುದ್ಧವಾಗಿಲ್ಲ. ಬಗ್ಗಡದ ನೀರಿನಂತಿದೆ. ಜೊತೆಗೆ ಕುಲುಕಾಡುತ್ತಿದೆ. ಅವನೂ ಆ ನೀರಿನ ಕೆಳಗೆ ಏನಿದೆಯೋ ನೋಡಬೇಕೆಂದು ಬಯಸುವನು, ಆದರೆ ಅವನಿಗೆ ಕಾಣಿಸುವುದಿಲ್ಲ. ಇಬ್ಬರೂ ಪ್ರಯತ್ನ ಮಾಡುವರು. ಜ್ಞಾನಿಯ ಪ್ರಯತ್ನ ಜಯಪ್ರದವಾಗುವುದು. ಅಜ್ಞಾನಿಯ ಪ್ರಯತ್ನ ವೃಥಾ ಶ್ರಮದಲ್ಲಿ ಪರ್ಯವಸಾನವಾಗುವುದು. ಅಜ್ಞಾನಿಯಲ್ಲಿಯೂ ಜೀವನು ಇರುವನು. ಆದರೆ ದೇಹ ಇಂದ್ರಿಯ ಇವುಗಳ ಬಗ್ಗಡದಲ್ಲಿ ಕಾಣುವುದಿಲ್ಲ. ಬಗ್ಗಡದ ಜೊತೆಗೆ ನೀರು ಕೂಡ ಬೇಕಾದಷ್ಟು ಕುಲಕಾಡುತ್ತಿದೆ. ನೀರನ್ನು ಶುದ್ಧಿಮಾಡಬೇಕು. ಅದನ್ನು ಶಾಂತವಾಗಿರು ವಂತೆ ಮಾಡಬೇಕು. ಆಗಲೇ ಅದರ ಹಿಂದೆ ಇರುವುದು ಕಾಣುವುದು. ಆಧ್ಯಾತ್ಮಿಕ ಜೀವನದ ಸಾಧನೆಯೆಲ್ಲಾ ಇದೇ. ಯಾವುದೋ ಇಲ್ಲದ ವಸ್ತುವನ್ನು ಹೊಸದಾಗಿ ಪಡೆದುಕೊಳ್ಳುವುದಲ್ಲ. ನಾವು ಹುಡುಕುವ ವಸ್ತು ಆಗಲೇ ನಮ್ಮಲ್ಲಿ ಇದೆ, ಕಸ್ತೂರಿಮೃಗದ ನಾಭಿಯಲ್ಲೇ ಕಸ್ತೂರಿ ಇರುವಂತೆ. ಆದರೆ ನಾವು ಚಿತ್ತವನ್ನು ಶುದ್ಧಿಮಾಡಿಕೊಂಡಿಲ್ಲ. ಅದಕ್ಕಾಗಿ ಅದು ಕಾಣುತ್ತಿಲ್ಲ. ಯಾವಾಗ ಚಿತ್ತವನ್ನು ಶುದ್ಧಿ ಮಾಡಿಕೊಳ್ಳುತ್ತೇವೆಯೊ ಅದರ ಹಿಂದೆಯೇ ಇದು ಕಾಣುವುದು. ರೇಡಿಯೋ ಕೆಟ್ಟುಹೋಗಿದೆ. ಅದಕ್ಕಾಗಿ ಸಂಗೀತ ಕೇಳಿಸುತ್ತಿಲ್ಲ. ಯಾವಾಗ ಅದನ್ನು ರಿಪೇರಿಮಾಡುತ್ತೇವೆಯೋ ತತ್​ಕ್ಷಣವೇ ಸಂಗೀತ ಕೇಳಿಸುವುದು. ಆ ಸಂಗೀತವೇನೊ ಹೊಸದಾಗಿ ತಯಾರಾಗುವುದಿಲ್ಲ. ಯಾವಾಗಲೂ ಇದ್ದದ್ದೇ, ಈಗ ಕೇಳಿಸುವುದು.

\begin{verse}
ಯದಾದಿತ್ಯಗತಂ ತೇಜೋ ಜಗದ್ಭಾಸಯತೇಽಖಿಲಮ್ ।\\ಯಚ್ಚಂದ್ರಮಸಿ ಯಚ್ಚಾಗ್ನೌ ತತ್ತೇಜೋ ವಿದ್ಧಿ ಮಾಮಕಮ್ \versenum{॥ ೧೨ ॥}
\end{verse}

{\small ಸೂರ್ಯನ ಯಾವ ತೇಜಸ್ಸು ಸಮಸ್ತ ಜಗತ್ತನ್ನು ಪ್ರಕಾಶಗೊಳಿಸುವುದೋ, ಯಾವ ತೇಜಸ್ಸು ಚಂದ್ರ ಅಗ್ನಿಗಳಲ್ಲಿದೆಯೋ, ಆ ತೇಜಸ್ಸು ನನ್ನದೇ ಎಂದು ತಿಳಿ.}

ಬೆಳಕು ಯಾವಾಗಲೂ ಜ್ಞಾನಕ್ಕೆ ಒಂದು ಚಿಹ್ನೆ. ಸೂರ್ಯನ ಕಾಂತಿ ಜಗತ್ತನ್ನೆಲ್ಲ ಬೆಳಗಿಸುತ್ತಿದೆ. ಆ ಸೂರ್ಯನ ಕಾಂತಿಯ ಹಿಂದೆ ಇರುವ ತೇಜಸ್ಸು ಪರಮಾತ್ಮನದೇ. ಅವನಿಂದ ಸೂರ್ಯ ಬೆಳಗುತ್ತಾನೆ. ಆ ಸೂರ್ಯ ನಮ್ಮ ಸೃಷ್ಟಿಗೆ ಅತ್ಯಂತ ಮುಖ್ಯವಾದ ವಸ್ತು. ಅವನ ಕಾಂತಿಯಿಂದಲೆ ನಮಗೆ ಹಗಲು. ಆ ಸೂರ್ಯನ ಕಾಂತಿ ಹಸಿರು ಮರದ ಮೇಲೆ ಬಿದ್ದಾಗಲೇ ಅದೆಲ್ಲ ಚೆನ್ನಾಗಿ ಬೆಳೆಯಬೇಕಾದರೆ ಮತ್ತು ಫಲಪುಷ್ಪಗಳನ್ನು ಕೊಡಬೇಕಾದರೆ. ಹಾಗೆಯೇ ಚಂದ್ರನ ಕಾಂತಿ ಸೂರ್ಯನಷ್ಟು ತಪಿಸುವುದಿಲ್ಲ, ಜ್ವಲಿಸುವುದಿಲ್ಲ. ಆಹ್ಲಾದಕರವಾದ ತಂಪಿನ ಬೆಳಕು ಚಂದ್ರನದು. ಔಷಧಿಗಳಿಗೆಲ್ಲ ಅವನೇ ರಾಜ ಎಂದು ಪುರಾತನ ಹಿಂದೂ ಶಾಸ್ತ್ರಗಳು ಸಾರುವುವು. ಆ ಚಂದ್ರನ ಹಿಂದೆ ಇರುವ ಕಾಂತಿಯೂ ಭಗವಂತನಿಂದ ಬಂದುದೇ. ಇನ್ನು ಅಗ್ನಿಯ ತೇಜಸ್ಸು. ಅಗ್ನಿಯಲ್ಲಿರುವ ಶಕ್ತಿ ಮನುಷ್ಯನ ನಾಗರಿಕತೆಗೆ ಅತ್ಯಂತ ಆವಶ್ಯಕವಾದ ವಸ್ತು. ನಮ್ಮ ಅಡಿಗೆಮನೆಯಿಂದ ಹಿಡಿದು ದೊಡ್ಡ ದೊಡ್ಡ ಫ್ಯಾಕ್ಟರಿಗಳಲ್ಲಿ ಕೆಲಸಗಳೆಲ್ಲ ಆಗುವುದು ಅಗ್ನಿದೇವನ ದಯೆಯಿಂದ. ಅವನ ಕಾವಿನಿಂದಲೇ ಕೈಗಾ ರಿಕೆಯ ಯಂತ್ರದ ಚಕ್ರಗಳು ಉರುಳಬೇಕಾದರೆ. ಬೆಂಕಿ ಉರಿಯುತ್ತಿದೆ ಎಂದು ಸಹಜವಾಗಿ ಹೇಳುತ್ತೇವೆ. ಹುಡುಗರಾದಾಗಿನಿಂದಲೂ ಅದನ್ನು ನೋಡುತ್ತ ಇರುತ್ತೇವೆ. ಆದರೆ ಆ ಬೆಂಕಿಗೆ ಉರಿಯುತ್ತಿರುವ ಶಕ್ತಿ ಇರುವುದು ಒಂದು ಅದ್ಭುತ. ಅದನ್ನು ಕೊಟ್ಟವನೇ ಪರಮಾತ್ಮ.

\begin{verse}
ಗಾಮಾವಿಶ್ಯ ಚ ಭೂತಾನಿ ಧಾರಯಾಮ್ಯಹಮೋಜಸಾ ।\\ಪುಷ್ಣಾಮಿ ಚೌಷಧೀಃ ಸರ್ವಾಃ ಸೋಮೋ ಭೂತ್ವಾ ರಸಾತ್ಮಕಃ \versenum{॥ ೧೩ ॥}
\end{verse}

{\small ನಾನು ಭೂಮಿಯನ್ನು ಪ್ರವೇಶ ಮಾಡಿ ಪ್ರಾಣಿಗಳನ್ನು ಓಜಸ್ಸಿನಿಂದ ಧರಿಸುತ್ತೇನೆ. ರಸಸ್ವರೂಪದ ಚಂದ್ರನಾಗಿ ಎಲ್ಲಾ ವನಸ್ಪತಿಗಳನ್ನು ಪೋಷಿಸುತ್ತೇನೆ.}

ದೇವರೆಲ್ಲೊ ಸೃಷ್ಟಿಗೆ ದೂರ ನಿಂತುಕೊಂಡು ಇದನ್ನು ನಡೆಸುತ್ತಿಲ್ಲ. ಅವನು ಈ ಸೃಷ್ಟಿಯಲ್ಲಿ ಪ್ರವೇಶಿಸಿರುವನು, ನಾವು ನೋಡುವುದರ ಹಿಂದೆಲ್ಲ ಅವನು ತುಂಬಿ ತುಳುಕಾಡುತ್ತಿರುವನು. ಈ ಭೂಮಿಯಲ್ಲಿರುವ ಪ್ರತಿಯೊಂದು ಪ್ರಾಣಿಯನ್ನೂ ಪ್ರವೇಶಮಾಡಿ ಅಲ್ಲಿ ಓಜಸ್ಸಿನಿಂದ ಇದ್ದೇನೆ ಎನ್ನುವನು. ಆ ಪರಮಾತ್ಮನ ಅಂಶವಾದ ಓಜಸ್ಸು ಇರುವುದರಿಂದಲೇ ಅದಕ್ಕೆ ಪ್ರಾಣವಿರುವುದು, ಸುತ್ತಮುತ್ತಲೂ ಇರುವ ಆತಂಕಗಳೊಡನೆ ಹೋರಾಡುವುದು, ಒಂದು ಗುರಿ ಎಡೆಗೆ ಧಾವಿಸುವುದು, ಒಂದು ಉದ್ದೇಶವನ್ನು ಸಾಧಿಸುವುದು, ಚಟುವಟಿಕೆಯಿಂದ ಕೂಡಿರುವುದು.

ಚಂದ್ರನಲ್ಲಿರುವ ಶಕ್ತಿಯೇ ಮಳೆಯ ಮೂಲಕ ಧರೆಗೆ ಬಿದ್ದು ಬೀಜಗಳನ್ನು ಪ್ರವೇಶಿಸಿ ಅಲ್ಲಿ ಹುಲುಸಾದ ಪೈರನ್ನು ಬರುವಂತೆ ಮಾಡುವುದು. ಅವನು ಧರೆಯ ಮೇಲೆ ಇರುವ ಎಲ್ಲಾ ಪ್ರಾಣಿಗಳಲ್ಲಿಯೂ ವನಸ್ಪತಿಗಳಲ್ಲಿಯೂ ಇರುವನು. ಅಂತರಿಕ್ಷದಲ್ಲಿ ಕಾಂತಿಯನ್ನು ಬೀರುತ್ತಿರುವ ಸೂರ್ಯಚಂದ್ರರ ಹಿಂದೆ ಇರುವನು.

\begin{verse}
ಅಹಂ ವೈಶ್ವಾನರೋ ಭೂತ್ವಾ ಪ್ರಾಣಿನಾಂ ದೇಹಮಾಶ್ರಿತಃ ।\\ಪ್ರಾಣಾಪಾನಸಮಾಯುಕ್ತಃ ಪಚಾಮ್ಯನ್ನಂ ಚತುರ್ವಿಧಂ \versenum{॥ ೧೪ ॥}
\end{verse}

{\small ನಾನು ಜಠರಾಗ್ನಿಯ ರೂಪದಿಂದ ಎಲ್ಲಾ ಪ್ರಾಣಿಗಳ ಶರೀರವನ್ನು ಆಶ್ರಯಿಸಿ, ಪ್ರಾಣಾಪಾನವಾಯುಗಳ ಮೂಲಕ ನಾಲ್ಕು ವಿಧವಾದ ಅನ್ನವನ್ನು ಜೀರ್ಣವಾಗುವಂತೆ ಮಾಡುತ್ತೇನೆ.}

ಆ ಪರಮಾತ್ಮನ ಶಕ್ತಿಯೇ ದೇಹವನ್ನು ಪ್ರವೇಶಿಸಿ ಪ್ರಾಣ, ಅಪಾನ, ವ್ಯಾನ, ಉದಾನ, ಸಮಾನ ವಾಯುಗಳಂತೆ ಕವಲೊಡೆದು ದೇಹದಲ್ಲಿ ಪ್ರತಿಯೊಂದು ಅಂಗಾಂಗಗಳ ಕೆಲಸಗಳನ್ನು ಮಾಡುತ್ತಿರು ವುದು. ಶ್ವಾಸಕೋಶ ಚಲಿಸುವುದು ಅದರಿಂದ, ಹೃದಯ ಬಡಿದು ರಕ್ತಚಲನೆಯಾಗುವುದು ಅದರಿಂದ, ನಾವು ತಿಂದದ್ದು ಅರಗುವುದು ಅದರಿಂದಲೇ. ರುಚಿ ರುಚಿಯಾಗಿರುವ ಆಹಾರಗಳನ್ನು ನಾವು ಬಾಯಿಯ ಮೂಲಕ ತಿನ್ನುತ್ತೇವೆ. ಹಾಗೆ ತಿನ್ನುವಾಗ ಮಾತ್ರ ನಮ್ಮದು, ನಾಲಗೆಯಿಂದ ಕೆಳಗೆ ಬಿದ್ದ ಮೇಲೆ ನಮ್ಮದಲ್ಲ. ಶ್ರೀಕೃಷ್ಣ ನಾವು ನಾಲ್ಕು ವಿಧವಾದ ಆಹಾರವನ್ನು ತಿನ್ನುತ್ತೇವೆ ಎನ್ನುವನು. ಮೊದಲನೆಯದೆ ಹಲ್ಲಿನ ಮೂಲಕ ಅಗಿಯುತ್ತೇವೆ. ಗಟ್ಟಿ ಪದಾರ್ಥಗಳೆಲ್ಲ ಮೊದಲು ಹಲ್ಲಿಗೆ ಸಿಕ್ಕಿಕೊಂಡು ಚೂರುಚೂರಾಗಿ ಒಳಗೆ ಹೋಗುವುದು. ಎರಡನೆಯದೆ ನಾವು ಕುಡಿಯುವುದು, ದ್ರವದ ರೂಪವಾದ ಆಹಾರವನ್ನೆಲ್ಲ ನಾವು ಕುಡಿಯುತ್ತೇವೆ. ಹಾಲು ನೀರು ಕಾಫಿ ಮುಂತಾದುವುಗಳೇ ಅವು. ಮೂರನೆಯದೇ ಚೀಪುವುದು, ನೆಕ್ಕಿ ತಿನ್ನುವುದು. ನಾಲ್ಕನೆಯದೆ ಸುಮ್ಮನೆ ನುಂಗುವುದು, ಹಿಟ್ಟು ಮುಂತಾದುವು. ಇವೆಲ್ಲ ಹೊಟ್ಟೆಗೆ ಹೋಗುವತನಕ ಮಾತ್ರ ಪ್ರಯತ್ನಪೂರ್ವಕ ಮಾಡುವ ಕ್ರಿಯೆ ಗಳು. ಯಾವಾಗ ಒಂದು ಸಲ ಅವು ಜಠರಕೋಶದೊಳಗೆ ಹೋಗಿ ಬೀಳುವುವೋ ಅಲ್ಲಿಂದಲೇ ಅವು ಅದ್ಭುತವಾಗಿ ಪರಿಣಾಮವಾಗುವುವು. ನಾವು ತಿಂದ ಅನ್ನ ಜೀರ್ಣವಾಗಿ, ನಮ್ಮ ರಕ್ತ ಮಾಂಸವಾಗಿ, ಎಲುಬು ಕೊಬ್ಬಾಗಿ ನಮ್ಮ ಅಂಗಾಗಗಳೆಲ್ಲ ಆಗಿ ಪರಿವರ್ತನವಾಗುವುವು. ಮನುಷ್ಯ ಹೊರಗಡೆ ಕಂಡುಹಿಡಿದ ಯಾವ ಯಂತ್ರವೂ ಇದನ್ನು ಮಾಡಲಾರದು. ಆಹಾರದಿಂದ ರಕ್ತವನ್ನು ಯಾವ ಬಾಹ್ಯ ಯಂತ್ರವೂ ಮಾಡಲಾರದು. ಆದರೆ ದೇಹದೊಳಗೆ ಇರುವ ಯಂತ್ರ ಅದನ್ನು ಮಾಡುವುದು.

ಹಾಗೆ ಮಾರ್ಪಡಿಸುವುದಕ್ಕೆ ಎಷ್ಟು ಶಕ್ತಿಯನ್ನು ಉಪಯೋಗಿಸುವುದು? ಅತ್ಯಂತ ಕಡಿಮೆ ಶಕ್ತಿಯನ್ನು. ಹೊರಗಿರುವ ಯಾವ ಫ್ಯಾಕ್ಟರಿಯೂ ಇಂತಹ ಅದ್ಭುತವನ್ನು ಮಾಡಲಾರದು. ನಾವು ದೇವರ ಮಹಿಮೆಯನ್ನು ಅರಿಯಬೇಕಾದರೆ ದೂರದಲ್ಲಿರುವ ಸೂರ್ಯ ಚಂದ್ರರ ಕಡೆ ತಿರುಗಬೇಕಾ ಗಿಲ್ಲ. ನಮ್ಮ ಒಳಗೇ ಅವನ ಒಂದು ಅದ್ಭುತವಾದ ವ್ಯಾಪಾರ ನಡೆಯುತ್ತಿದೆ. ನಾವೇನಾದರೂ ಪ್ರಯತ್ನಪೂರ್ವಕವಾಗಿ ಅರಗಿಸಿಕೊಳ್ಳುತ್ತೇವೆಯೇ? ಇಲ್ಲ. ತಿನ್ನುವುದೊಂದೇ ಮಾಡುವ ಮಹಾ ಪ್ರಯತ್ನ. ಆ ತಿಂದದ್ದನ್ನು ಅರಗಿಸುವವನು ನಮ್ಮೊಳಗಿರುವ ಮತ್ತಾರೋ. ಅವನೇ ಭಗವಂತ. ಸೂರ್ಯಚಂದ್ರರು ಹೇಗೆ ಅವನಾಜ್ಞೆಯಂತೆ ತಮ್ಮ ಪಥದಲ್ಲಿ ಪ್ರತಿದಿನ ತಿರುಗುತ್ತಿರುವರೋ ಹಾಗೆಯೇ ಅವನಾಜ್ಞೆಯಂತೆ ನಮ್ಮ ಜಠರ, ಯಕೃತ್, ಕರಳು ಮತ್ತು ಹಲವಾರು ರಸನೇಂದ್ರಿಯಗಳು ಕೆಲಸ ಮಾಡುತ್ತಿವೆ. ಅವನೇ ನಿಂತು ಇವುಗಳನ್ನೆಲ್ಲ ಮಾಡಿಸುತ್ತಿರುವನು. ತಿಂದ ಒಂದು ಹಿಡಿ ಅನ್ನ, ಕೆಲವು ತೊಟ್ಟು ರಕ್ತ ಆಗುವವರೆಗೆ ಎಷ್ಟೊಂದು ಬದಲಾವಣೆಗಳ ಮೂಲಕ ಹೋಗಬೇಕಾಗುವುದು. ಒಂದೊಂದು ಬದಲಾವಣೆಯೂ ಒಂದೊಂದು ಪವಾಡ. ಎಂತಹ ಮಂತ್ರಗಾರನೂ ಇದನ್ನು ಮಾಡಲಾರ; ನಾವೊಂದು ದೇಹವನ್ನು ಯಾವಾಗಲೂ ಹೊತ್ತು ಅಲೆಯುತ್ತಿರುವೆವು. ಈ ಒಂದು ದೇಹದಲ್ಲಿ ಎಂತಹ ಅದ್ಭುತ ನಡೆಯುತ್ತಿದೆ ಎಂಬುದನ್ನು ನಾವು ಎಂದಾದರೂ ವಿಚಾರ ಮಾಡಿರು ವೆವೆ?

\begin{verse}
ಸರ್ವಸ್ಯ ಚಾಹಂ ಹೃದಿ ಸಂನಿವಿಷ್ಟೋ ಮತ್ತಃ ಸ್ಮೃತಿರ್ಜ್ಞಾನಮಪೋಹನಂ ಚ ।\\ವೇದೈಶ್ಚ ಸರ್ವೈರಹಮೇವ ವೇದ್ಯೋ ವೇದಾಂತಕೃದ್ವೇದವಿದೇವ ಚಾಹಮ್ \versenum{॥ ೧೫ ॥}
\end{verse}

{\small ನಾನು ಎಲ್ಲರ ಹೃದಯದಲ್ಲಿಯೂ ಇದ್ದೇನೆ. ಸ್ಮೃತಿ ಜ್ಞಾನ ಮತ್ತು ಅವುಗಳ ಅಭಾವ ನನ್ನಿಂದಲೇ ಆಗುವುದು. ಎಲ್ಲಾ ವೇದಗಳಿಂದಲೂ ನಾನೇ ತಿಳಿಯತಕ್ಕವನು. ವೇದಾಂತವನ್ನು ಮಾಡಿದವನು ನಾನೆ. ವೇದವನ್ನು ತಿಳಿದವನು ನಾನೆ.}

ಭಗವಂತ ಸರ್ವಾಂತರ್ಯಾಮಿಯಾಗಿ ಎಲ್ಲರ ಹೃದಯದಲ್ಲಿಯೂ ಇದ್ದಾನೆ. ಅವನಿಲ್ಲದ ಸ್ಥಳವೇ ಇಲ್ಲ. ಹಾಗಾದರೆ ಮನುಷ್ಯ ಪಾಪವನ್ನೇಕೆ ಮಾಡುತ್ತಾನೆ? ಅವನಿದ್ದಾನೆ ನಿಜ. ಆದರೆ ಅವನ ಪ್ರಯೋಜನವನ್ನು ಪಡೆದುಕೊಳ್ಳುವುದಕ್ಕೆ ನಮ್ಮ ಸಂಸ್ಕಾರ ಇನ್ನೂ ಅಣಿಯಾಗಿಲ್ಲ. ಶ್ರೀರಾಮ ಕೃಷ್ಣರು ಇದನ್ನೇ ಒಂದು ದೀಪದ ಉದಾಹರಣೆ ಮೂಲಕ ವಿವರಿಸುತ್ತಾರೆ. ಕೋಣೆಯಲ್ಲಿ ಒಂದು ದೀಪ ಉರಿಯುತ್ತಿದೆ. ಒಬ್ಬ ಅದರಿಂದ ಒಂದು ಧರ್ಮಗ್ರಂಥ ಪಾರಾಯಣ ಮಾಡುತ್ತಾನೆ. ಮತ್ತೊಬ್ಬ ಅದೇ ಬೆಳಕಿನ ಸಹಾಯದಿಂದ ಒಂದು ಸುಳ್ಳು ಅರ್ಜಿಯನ್ನು ಬರೆಯುತ್ತಾನೆ. ಅದಕ್ಕೆ ನಾವು ದೀಪವನ್ನು ಹೊಣೆ ಮಾಡುವುದಕ್ಕೆ ಆಗುವುದಿಲ್ಲ. ಭಗವಂತ ಮಳೆಯಂತೆ ಎಲ್ಲಾ ಕಡೆಯೂ ಬೀಳುತ್ತಾನೆ. ಯಾರು ಹೊಲವನ್ನು ಉತ್ತಿರುವರೋ ಬಿತ್ತಿರುವರೋ ಅಲ್ಲಿ ಚೆನ್ನಾಗಿ ಪೈರು ಬರುವುದು. ಅದಿಲ್ಲದೆಡೆ ಏನು ಇದೆಯೋ ಅದೇ ಹುಲಸಾಗಿ ಬೆಳೆಯುವುದು. ಮುಳ್ಳು ಮುಂತಾದುವೆ ಅವು. ಅದಕ್ಕಾಗಿ ಮಳೆಯನ್ನು ದೂರುವುದಕ್ಕೆ ಆಗುವುದೇ?

ಪರಮಾತ್ಮನಿರುವುದರಿಂದಲೇ ನಮಗೆ ಸ್ಮೃತಿಶಕ್ತಿ ಬರುವುದು. ವಿಷಯಗಳನ್ನು ಚೆನ್ನಾಗಿ ನೆನಪಿ ನಲ್ಲಿಟ್ಟುಕೊಳ್ಳುತ್ತೇವೆ. ಸ್ಮೃತಿ ಅತ್ಯಂತ ಆವಶ್ಯಕ ಬುದ್ಧಿ ಶಕ್ತಿಗೆ. ಸ್ಮೃತಿ ಶಕ್ತಿ ಚೆನ್ನಾಗಿದ್ದರೇ ಅವುಗಳ ಆಧಾರದಮೇಲೆ, ಅವುಗಳನ್ನು ಹೋಲಿಸಿ ತತ್​ಕ್ಷಣವೇ ಬುದ್ಧಿ ಒಂದು ನಿರ್ಣಯಕ್ಕೆ ಬರಬೇಕಾದರೆ. ನಮ್ಮಲ್ಲಿರುವ ಜ್ಞಾನವೂ ಅವನ ವರವೇ. ಎಲ್ಲರಲ್ಲಿಯೂ ಜ್ಞಾನವಿದೆ. ಆದರೆ ಅಜ್ಞಾನದ ಪಾಚಿ ಮುಚ್ಚಿಕೊಂಡಿದೆ ಅದನ್ನು. ಯಾವಾಗ ಪಾಚಿಯನ್ನು ಆಚೆಗೆ ದಬ್ಬುತ್ತೇವೆಯೋ ಆಗ ಜ್ಞಾನ ಹೊಳೆಯುವುದು.

ಇವುಗಳ ಅಭಾವ ಕೂಡ ಅವನದೇ. ಅಂದರೆ ನಮ್ಮಲ್ಲಿರುವ ಮರೆವು, ಅಜ್ಞಾನ, ಅಸಡ್ಡೆ ಮುಂತಾದುವುಗಳು. ಇವೆಲ್ಲ ನಾವೇ ಸಂಪಾದಿಸಿದ ಕಶ್ಮಲಗಳು. ಜ್ಞಾನವನ್ನು ಕಾಣದಂತೆ ಮಾಡಿ ಕೊಂಡಿದ್ದೇವೆ. ಈ ಹೀನ ಕರ್ಮಗಳು ಕೆಲಸ ಮಾಡುವುದು ಕೂಡ ಅವನ ಆಣತಿಯಂತೆಯೇ. ತಪ್ಪು ಮಾಡಿದವನಿಗೆ ನ್ಯಾಯಾಧಿಪತಿ ಶಿಕ್ಷೆಯನ್ನು ಕೊಡುತ್ತಾನೆ. ಒಳ್ಳೆಯದನ್ನು ಮಾಡಿದವನಿಗೆ ಬಹುಮಾನ ವನ್ನು ಕೊಡುತ್ತಾನೆ. ಎರಡರ ಹಿಂದೆಯೂ ಅವನ ಶಕ್ತಿಯೇ ಇದೆ.

ಎಲ್ಲಾ ವೇದಗಳ ಹಿಂದೆಯೂ ನಾನೇ ತಿಳಿಯತಕ್ಕವನು ಎನ್ನುವನು. ಶ್ರೀಕೃಷ್ಣ ಇಲ್ಲಿ ಹಿಂದೂಗಳ ನಾಲ್ಕು ವೇದಗಳ ದೃಷ್ಟಿಯಿಂದಲೇ ಹೇಳುತ್ತಿಲ್ಲ. ಇದು ಮಾನವ ಕೋಟಿಯ ಇಡೀ ಜ್ಞಾನ ಭಂಡಾರಕ್ಕೇ ಅನ್ವಯಿಸುವುದು, ಎಲ್ಲಾ ಆಧ್ಯಾತ್ಮಿಕ ಶಾಸ್ತ್ರಕ್ಕೂ ಅನ್ವಯಿಸುವುದು. ಅವುಗಳೆಲ್ಲ ಸಾರುವುದು ಒಂದೇ ಗುರಿಯನ್ನು. ವ್ಯತ್ಯಾಸ ಅವುಗಳ ಭಾಷೆಯಲ್ಲಿ ಮಾತ್ರ. ತಿರುಳಿನಲ್ಲಿ ಯಾವ ವ್ಯತ್ಯಾಸವೂ ಇಲ್ಲ. ಸತ್ಯ ಒಂದು, ಬಲ್ಲವರು ಅದನ್ನು ಹಲವು ರೀತಿ ಹೇಳುತ್ತಾರೆ. ಎಲ್ಲಾ ಆಧ್ಯಾತ್ಮಿಕ ಸತ್ಯಗಳೂ ಪರಮಾತ್ಮನೆಂಬ ಕೇಂದ್ರದಲ್ಲಿ ಸಂಧಿಸುವುವು. ಅವನ ಕಡೆಗೆ ಹೋಗುವುದಕ್ಕಾಗಿಯೇ ಈ ಹಾದಿಗಳು ಇರುವುವು.

ವೇದಾಂತವನ್ನು ಮಾಡಿದವನು ನಾನೇ ಎನ್ನುವನು. ವೇದಾಂತ ಎಂದರೆ ತತ್ತ್ವಶಾಸ್ತ್ರದ ಸಾರ, ತತ್ತ್ವದ ಚರಮ ಅವಸ್ಥೆ. ಅದನ್ನು ಮಾಡಿದವನು ದೇವರೇ. ಪರಮಾತ್ಮನೆಡೆಗೆ ಒಯ್ಯುವ ಸೋಪಾನ ಪಂಕ್ತಿ ಈ ವೇದಾಂತ. ಈ ಮೆಟ್ಚಲನ್ನು ಮುಂಚೆ ಅವನೇ ಕಟ್ಟಿದವನು. ಅದನ್ನು ಕಂಡು ಹಿಡಿದ ಪುಷಿಗಳು ಈ ಮೆಟ್ಟಲನ್ನು ಕಟ್ಟಲಿಲ್ಲ. ಅವರನ್ನು ಮಂತ್ರದ್ರಷ್ಟಾರರು ಎನ್ನುತ್ತಾರೆ. ಆಗಲೆ ಇರುವ ಸತ್ಯವನ್ನು ಮನಗಂಡವರು ಎಂದು. ಅದನ್ನು ತಯಾರು ಮಾಡಿದವರು ಎಂದಲ್ಲ. ಮೆಟ್ಟಲುಗಳು ಆಗಲೇ ಇದ್ದುವು. ಅವು ಮರೆತು ಹೋಗಿದ್ದುವು. ಪುಷಿಗಳು ಬಂದು ಅದನ್ನು ಕಂಡುಹಿಡಿದರು ಅಷ್ಟೆ.

ವೇದಗಳನ್ನೆಲ್ಲ ತಿಳಿದವನು ನಾನೆ ಎನ್ನುವನು. ವೇದ ಎಂದರೆ ಜ್ಞಾನರಾಶಿ. ಎಲ್ಲವನ್ನು ಮೊದಲು ತಿಳಿದುಕೊಂಡವನು ಭಗವಂತ. ಅವನೊಬ್ಬನೇ ಸರ್ವಜ್ಞ. ಏಕೆಂದರೆ ಅವನೇ ಎಲ್ಲವನ್ನೂ ಮಾಡಿದ ವನು. ಮಾಡಿದವನೊಬ್ಬನಿಗೆ ಮಾಡಿರುವುದೆಲ್ಲ ಗೊತ್ತಿದೆ. ಸೃಷ್ಟಿಯನ್ನು ಮಾಡಿದವನು ಅವನು. ಅವನಿಗೆ ಸೃಷ್ಟಿಯ ಅರ್ಥವೆಲ್ಲ ಗೊತ್ತಿದೆ. ಸೃಷ್ಟಿಯ ಉದ್ದೇಶವೆಲ್ಲ ಗೊತ್ತಿದೆ.

\begin{verse}
ದ್ವಾವಿಮೌ ಪುರುಷೌ ಲೋಕೇ ಕ್ಷರಶ್ಚಾಕ್ಷರ ಏವ ಚ ।\\ಕ್ಷರಃ ಸರ್ವಾಣಿ ಭೂತಾನಿ ಕೂಟಸ್ಥೋಽಕ್ಷರ ಉಚ್ಯತೇ \versenum{॥ ೧೬ ॥}
\end{verse}

{\small ಈ ಲೋಕದಲ್ಲಿ ನಾಶವಾಗುವ ಮತ್ತು ನಾಶವಾಗದ ಎರಡು ಬಗೆಯ ಪುರುಷರು ಇದ್ದಾರೆ. ಸರ್ವ ಪ್ರಾಣಿಗಳು ನಾಶವಾಗುವುವು. ಅವುಗಳಲ್ಲಿ ನಾಶವಾಗದವನೆ ಕೂಟಸ್ಥ.}

ಈ ಪ್ರಪಂಚದಲ್ಲಿ ಎರಡು ಬಗೆಯ ಜೀವಿಗಳಿದ್ದಾರೆ. ಒಬ್ಬರೇ ನಾಶವಾಗುವವರು, ಬದ್ಧರು. ಇವರು ಹುಟ್ಟುತ್ತಾರೆ, ಬೆಳೆಯುತ್ತಾರೆ. ನಾಶವಾಗುತ್ತಾರೆ. ಈ ಪ್ರಪಂಚಕ್ಕೆ ಬರುತ್ತ ಹೋಗುತ್ತಾ ಇರುತ್ತಾರೆ. ಇವರು ಇನ್ನೂ ಜ್ಞಾನವನ್ನು ಪಡೆದುಕೊಂಡಿಲ್ಲ. ಇವರಲ್ಲಿ ಹಲವಾರು ಆಸೆ ಆಕಾಂಕ್ಷೆಗಳು ಇವೆ. ಅವುಗಳನ್ನು ತೃಪ್ತಿಪಡಿಸಿಕೊಳ್ಳುತ್ತ ಹೋದರೆ ಎಂತಹ ದುರವಸ್ಥೆಗೆ ನಾವು ಬರುತ್ತೇವೆ ಎಂಬುದು ಇನ್ನೂ ಇವರಿಗೆ ಗೊತ್ತಿಲ್ಲ. ಇವರು ಎಷ್ಟೋ ಸಲ ಈ ಪ್ರಪಂಚಕ್ಕೆ ಬಂದಿರುವರು. ಮುಂದೆ ಎಷ್ಟೋ ಸಲ ಬರಲಿರುವರು.

ಸರ್ವ ಪ್ರಾಣಿಗಳೂ ನಾಶವಾಗುತ್ತಿವೆ. ಮುಕ್ತಿಯನ್ನು ಪಡೆಯದೆ ಇರುವ ಜೀವಿಗಳು ಎಂತಹ ಲೋಕದಲ್ಲಿ ಬೇಕಾದರೂ ಇರಬಹುದು; ಎಂತಹ ಉತ್ತಮ ಪದವಿಯನ್ನು ಬೇಕಾದರೂ ಪಡೆದಿರ ಬಹುದು; ಆದರೆ ಅದು ತಾತ್ಕಾಲಿಕ. ಅವು ಜನನ ಮರಣಗಳ ಚಕ್ರದಿಂದ ಪಾರಾಗಿಲ್ಲ. ಅವರು ಈಗ ಸ್ವರ್ಗ ಮುಂತಾದ ಲೋಕಗಳಲ್ಲಿ ಸುಖವನ್ನು ಅನುಭವಿಸುತ್ತಿರುವುದು ಇಲ್ಲಿ ಯಾವುದೋ ಪುಣ್ಯ ಕರ್ಮವನ್ನು ಫಲಾಪೇಕ್ಷೆಯ ದೃಷ್ಚಿಯಿಂದ ಮಾಡಿದ್ದರಿಂದ. ಆ ಪುಣ್ಯ ಎಂದೆಂದಿಗೂ ಇರುವುದಿಲ್ಲ, ಕೂತು ಉಣ್ಣುವವನಿಗೆ ಕುಡಿಕೆ ಹಣ ಸಾಲದು ಎಂಬ ಗಾದೆಯಂತೆ. ಆ ಪುಣ್ಯ ತೀರಿದೊಡನೆಯೇ ಅವನು ಪುನಃ ಕರ್ಮ ಪ್ರಪಂಚಕ್ಕೆ ಬರಬೇಕಾಗುವುದು.

ಕೂಟಸ್ಥ ಎನ್ನುವವನೆ ಮುಕ್ತ ಜೀವಿ. ಅವನು ಎಲ್ಲಿಗೂ ಹೋಗುವುದೂ ಇಲ್ಲ, ಬರುವುದೂ ಇಲ್ಲ. ಶಾಲೆಯ ಎಲ್ಲಾ ತರಗತಿಗಳಿಂದಲೂ ಪಾಸಾದ ವಿದ್ಯಾರ್ಥಿಯಂತೆ ಅವನು. ಅವನು ಇನ್ನು ಮೇಲೆ ಯಾವ ತರಗತಿಯಲ್ಲೂ ಕುಳಿತುಕೊಂಡು ಪಾಠ ಕಲಿಯಬೇಕಾಗಿಲ್ಲ. ಅವನು ಕೂಟಸ್ಥ. ಯಾವ ಬದಲಾವಣೆಗೂ ಸಿಕ್ಕುವುದಿಲ್ಲ. ಉರುಳುತ್ತಿರುವ ಚಕ್ರದ ಮಧ್ಯದಲ್ಲಿರುವ ನಾಭಿಯಂತೆ. ಕೂಟಸ್ಥ ಅಜ್ಞಾನದಿಂದ ಪಾರಾಗಿರುವನು. ಭ್ರಾಂತಿಯಿಂದ ಪಾರಾಗಿರುವನು. ಅವನಲ್ಲಿ ಯಾವ ಸಂಸ್ಕಾರವೂ ಇಲ್ಲ. ಆದಕಾರಣ ಅದನ್ನು ಸಮೆಸುವುದಕ್ಕಾಗಿ ಅವನು ಯಾವ ಹೊಸ ದೇಹವನ್ನೂ ಧರಿಸಬೇಕಾಗಿಲ್ಲ.

\begin{verse}
ಉತ್ತಮಃ ಪುರುಷಸ್ತ್ವನ್ಯಃ ಪರಮಾತ್ಮೇತ್ಯುದಾಹೃತಃ ।\\ಯೋ ಲೋಕತ್ರಯಮಾವಿಶ್ಯ ಬಿಭರ್ತ್ಯವ್ಯಯ ಈಶ್ವರಃ \versenum{॥ ೧೭ ॥}
\end{verse}

{\small ಆದರೆ ಕ್ಷರ ಮತ್ತು ಅಕ್ಷರಗಳಿಗಿಂತ ಬೇರೆಯಾದ ಪುರುಷೋತ್ತಮನನ್ನು ಪರಮಾತ್ಮನೆಂದು ಕರೆಯುತ್ತಾರೆ. ಆ ಅವ್ಯಯನಾದ ಈಶ್ವರನು ಮೂರು ಲೋಕಗಳನ್ನು ಪ್ರವೇಶಿಸಿ ಧರಿಸಿಕೊಂಡಿರುವನು.}

ಪುರುಷೋತ್ತಮನೇ ಭಗವಂತ. ಅವನು ಕ್ಷರ ಮತ್ತು ಅಕ್ಷರಗಳಿಗಿಂತ ಬೇರೆಯಾಗಿರುವನು. ಕ್ಷರ ಎಂದರೆ ಬದ್ಧ ಜೀವಿಗಳು. ಇನ್ನೂ ಅಜ್ಞಾನದಲ್ಲಿ ನರಳುತ್ತಿರುವವರು. ಸುಖ ದುಃಖಗಳನ್ನು ಅನುಭವಿಸುತ್ತಿರುವರು. ಅಕ್ಷರ ಮುಕ್ತಜೀವಿಗಳು, ಈ ಬರುವುದು ಹೋಗುವುದರಿಂದ ಪಾರಾಗಿರು ವರು. ಅವರೇ ವಿಕಾಸದ ಏಣಿಯಲ್ಲಿ ತುತ್ತತುದಿಯನ್ನು ಮುಟ್ಟಿದವರು. ಈ ಒಂದು ಸ್ಥಿತಿಯನ್ನು ದೇವರಲ್ಲಿ ನಂಬದ ಇತರ ಧರ್ಮಗಳೂ ಒಪ್ಪುತ್ತವೆ. ಬುದ್ಧ ಇಂತಹ ಮುಕ್ತನಾದ ಜೀವಿ; ತೀರ್ಥಂಕರರು ಇಂತಹ ಮುಕ್ತರಾದ ಜೀವಿಗಳು. ಆದರೆ ಪುರುಷೋತ್ತಮ ಈ ಗುಂಪಿಗೆ ಸೇರಿಲ್ಲ. ಅವನು ಮುಕ್ತನಾಗಬೇಕಿಲ್ಲ. ಬದ್ಧನಾಗಿದ್ದರೆ ತಾನೇ ಮುಕ್ತನಾಗಬೇಕಾಗಿರುವುದು? ಅವನು ಎಂದಿಗೂ ಬದ್ಧನಲ್ಲ. ಈ ಪ್ರಪಂಚದಲ್ಲಿ ಸರ್ವಕ್ಕೂ ಒಡೆಯ ಅವನು. ಈ ಪ್ರಪಂಚವನ್ನು ತನ್ನಿಂದಲೇ ಸೃಷ್ಟಿಸಿದನು. ತಾನೇ ಪಾಲಿಸುತ್ತಿರುವನು. ತಾನೇ ಕೊನೆಗೆ ಇವುಗಳನ್ನು ತನ್ನೊಳಗೆ ಸೆಳೆದುಕೊಳ್ಳುವನು. ಅವನಲ್ಲಿ ಅನಂತ ಜ್ಞಾನ ಶಕ್ತಿ ಪವಿತ್ರತೆಗಳು ಇವೆ. ಅವನು ಇವುಗಳನ್ನೆಲ್ಲ ಸಾಧನೆ ಮಾಡಿ ಪಡೆಯಬೇಕಾಗಿಲ್ಲ. ಇವೆಲ್ಲ ಯಾವಾಗಲೂ ಅವನವೇ. ಅವನನ್ನೇ ಪರಮೇಶ್ವರ ಎನ್ನುತ್ತಾರೆ. ಈ ಭಾವನೆ ಬೌದ್ಧರಲ್ಲಿ ಜೈನರಲ್ಲಿ ಇಲ್ಲ. ಇಲ್ಲದೇ ಇದ್ದರೂ ಬುದ್ಧನನ್ನು ತೀರ್ಥಂಕರ ರನ್ನು ಅಂತಹ ಒಂದು ಪೀಠದ ಮೇಲೆ ಕೂಡಿಸಿ ಪೂಜಿಸುವರು.

ಪರಮಾತ್ಮ ಇಂತಹ ಮುಕ್ತ ಪುರುಷನಂತೆ ಅಲ್ಲ. ಮುಕ್ತ ಪುರುಷ ಪ್ರಪಂಚವನ್ನು ಸೃಷ್ಟಿ ಮಾಡಲಾರ, ಪಾಲನೆ ಮಾಡಲಾರ, ಸಂಹಾರ ಮಾಡಲಾರ. ಮುಕ್ತನಲ್ಲಿರುವ ಶಕ್ತಿ ಜ್ಞಾನ ಪವಿತ್ರತೆ ಗಳಿಗೆ ಒಂದು ಮಿತಿ ಇದೆ. ಆದರೆ ಪರಮಾತ್ಮನಲ್ಲಿ ಇವು ಅಪಾರವಾಗಿವೆ. ಅವನು ಸೃಷ್ಟಿ ಮಾಡಿರುವುದು ಮಾತ್ರವಲ್ಲ. ಮಾಡಿಯಾದ ಮೇಲೆ ಅದನ್ನು ಧರಿಸಿರುವನು. ಅದರೊಳಗೆ ಇರುವನು. ಅವನು ಸೃಷ್ಟಿಯಲ್ಲಿ ಅಂತರ್ಯಾಮಿಯಾಗಿರುವನು. ಮತ್ತು ಅವನು ಇದನ್ನು ಮೀರಿಯೂ ಇರುವನು. ನಾವು ಯಾವುದನ್ನು ಸೃಷ್ಟಿ ಎನ್ನುತ್ತೇವೆಯೋ ಅದೊಂದು ಸಾಂತ. ಪರಮಾತ್ಮನ ಯಾವುದೊ ಸಣ್ಣ ಅಂಶ ಈ ಬ್ರಹ್ಮಾಂಡವಾಗಿದೆ. ಇನ್ನು ಬಹು ಅಂಶ ಅದಕ್ಕೆ ಅತೀತವಾಗಿದೆ. ಸಮುದ್ರದಲ್ಲಿ ಯಾವುದೊ ಸ್ವಲ್ಪ ಭಾಗ ಮಂಜಿನಗಡ್ಡೆಯಾಗಿ ತೇಲುತ್ತಿದೆ. ಎಲ್ಲಾ ಮಂಜಿನಗಡ್ಡೆ ಯಾಗಿಲ್ಲ. ಸಮುದ್ರದ ನೀರು ಮಂಜಿನಗಡ್ಡೆಯಲ್ಲಿ ವ್ಯಾಪಿಸಿಕೊಂಡಿದೆ. ಆದರೆ ಅಲ್ಲಿಯೇ ಖಾಲಿ ಯಾಗಿ ಹೋಗುವುದಿಲ್ಲ. ಈ ಬ್ರಹ್ಮಾಂಡದ ಅಸ್ತಿತ್ವಕ್ಕೆ ಅವನೇ ಕಾರಣ. ಅವನಿದ್ದರೇ ಈ ಬ್ರಹ್ಮಾಂಡ. ಇಲ್ಲದೇ ಇದ್ದರೆ ಇಲ್ಲ. ಆದರೆ ಅವನ ಅಸ್ತಿತ್ವಕ್ಕೆ ಯಾವುದೂ ಬೇಕಾಗಿಲ್ಲ. ಅಲೆ ಇರಬೇಕಾದರೆ ಸಾಗರ ಇರಬೇಕು. ಆದರೆ ಸಾಗರಕ್ಕೆ ಅಲೆಯ ಹಂಗೇಕೆ?

ಇಲ್ಲಿ ಮೂರು ಲೋಕ ಎಂದು ಶ್ರೀಕೃಷ್ಣ ಹೇಳುತ್ತಾನೆ. ಇದು ಬರೀ ಮೂರು ಲೋಕಗಳೇ ಅಲ್ಲ. ಈ ಮೂರು ಬ್ರಹ್ಮಾಂಡವನ್ನೆಲ್ಲಾ ವ್ಯಾಪಿಸಿಕೊಳ್ಳುವುವು. ಮೇಲೆ, ಮಧ್ಯೆ, ಕೆಳಗೆ ಇರುವುದನ್ನೆಲ್ಲ ಅವನು ವ್ಯಾಪಿಸಿಕೊಂಡಿರುವನು. ಅವನು ವ್ಯಾಪಿಸಿಕೊಳ್ಳದೆ ಇರುವ ಯಾವ ಒಂದು ಚೂರೂ ಇಲ್ಲ. ಬದ್ಧನ ಹಿಂದೆ ಅವನು ಇದ್ದಾನೆ, ಮುಕ್ತನ ಹಿಂದೆ ಅವನು ಇದ್ದಾನೆ, ಜಡದಲ್ಲಿ ಅವನಿರುವನು, ಚೇತನದಲ್ಲಿ ಅವನು ಇರುವನು.

\begin{verse}
ಯಸ್ಮಾತ್ ಕ್ಷರಮತೀತೋಽಹಮಕ್ಷರಾದಪಿ ಚೋತ್ತಮಃ ।\\ಅತೋಽಸ್ಮಿ ಲೋಕೇ ವೇದೇ ಚ ಪ್ರಥಿತಃ ಪುರುಷೋತ್ತಮಃ \versenum{॥ ೧೮ ॥}
\end{verse}

{\small ನಾನು ಕ್ಷರವನ್ನು ಮೀರಿದವನು. ಅಕ್ಷರಕ್ಕಿಂತ ಉತ್ತಮನು. ಆದಕಾರಣ ಲೋಕದಲ್ಲಿ, ವೇದದಲ್ಲಿ ಪುರುಷೋ ತ್ತಮನೆಂದು ಪ್ರಸಿದ್ಧನಾಗಿದ್ದೇನೆ.}

ಶ್ರೀಕೃಷ್ಣನೇ ಇಲ್ಲಿ ತಾನು ಯಾರು ಎಂಬುದನ್ನು ಸಾರುತ್ತಾನೆ. ಅವನು ಕ್ಷರನೂ ಅಲ್ಲ, ಅಕ್ಷರವೂ ಅಲ್ಲ. ಇದಕ್ಕಿಂತ ಮಿಗಿಲಾದವನು. ಅವನೇ ಪರಮಾತ್ಮ. ಈಗ ವಸುದೇವನ ಪುತ್ರನಾದ ಶ್ರೀಕೃಷ್ಣ ನಂತೆ ಕಾಣುತ್ತಿರುವನು. ಇನ್ನು ಕೆಲವು ಕಾಲದ ಮೇಲೆ ದೇಹವನ್ನು ವಿಸರ್ಜಿಸುವನು. ಆದರೆ ಆ ಚಾರಿತ್ರಿಕ ವ್ಯಕ್ತಿತ್ವದ ಹಿಂದೆ, ಆ ಪಾತ್ರವನ್ನು ಅಭಿನಯಿಸುತ್ತಿರುವವನು ಸಾಕ್ಷಾತ್ ಪರಮಾತ್ಮನೇ. ಮುಕ್ತ ಜೀವಿಗಳಿಗಿಲ್ಲದ ಶಕ್ತಿ ಪರಮಾತ್ಮನಿಗೆ ಇದೆ. ಪರಮಾತ್ಮ ಸರ್ವ ಅಂತರಾತ್ಮ. ಎಲ್ಲರ ಹೃದಯದಲ್ಲಿಯೂ ಇರುವವನು, ಎಲ್ಲವನ್ನೂ ಅರಿಯುವವನು. ಬ್ರಹ್ಮಾಂಡಕ್ಕೆಲ್ಲ ಆಧಾರ ಸ್ವರೂಪ. ಇತರ ಮುಕ್ತ ಜೀವಿಗಳಲ್ಲಿ ಜ್ಞಾನಪಂಜಿನಂತೆ ಉರಿಯುತ್ತಿದ್ದರೆ ಪರಮಾತ್ಮನಲ್ಲಿ ಜ್ಞಾನಭಾಸ್ಕರನೇ ಬೆಳಗುತ್ತಿರುವನು. ಅವನು ನಿತ್ಯ ಸ್ವತಂತ್ರಿ.

ಅವನು ಲೋಕದಲ್ಲಿ ಪುರುಷೋತ್ತಮನೆಂದು ಪ್ರಖ್ಯಾತನಾಗಿದ್ದಾನೆ. ಒಬ್ಬ ದೇವರನ್ನು ನಂಬುವ ಧರ್ಮಗಳೆಲ್ಲ ಕರೆಯುವುದು ಇವನನ್ನೆ. ಯಾವ ಹೆಸರಿನಿಂದ ಕರೆದರೂ ಇವನಿಗೇ ಅನ್ವಯಿಸುವುದು. ಯಾವ ಆಕಾರದಿಂದ ಚಿಂತಿಸಿದರೂ ಅವುಗಳೆಲ್ಲದರ ಹಿಂದೆಯೂ ಇರುವವನು ಇವನೊಬ್ಬನೆ. ಭವಜೀವಿಗಳಿಗೆ ಒಂದು ಊರುಗೋಲಿನಂತೆ ಇರುವವನು ಅವನೆ. ಎಲ್ಲಾ ಜೀವರಿಗೂ ದಾರಿ ಬೆಳಕಾಗಿರುವವನೂ ಅವನೇ.

ವೇದಗಳು ಎಂದರೆ, ಭಗವಂತನಿಗೆ ಸಂಬಂಧಪಟ್ಟ ಶಾಸ್ತ್ರಗಳು. ಅವೆಲ್ಲ ಹೊರಟಿರುವುದು ಅವನನ್ನು ಸಾರುವುದಕ್ಕಾಗಿ. ಅವನೆಡೆಗೆ ದಾರಿದೋರುವ ಕೈಮರಗಳೇ ಶಾಸ್ತ್ರಗಳು. ಅವನಿಲ್ಲದ ಶಾಸ್ತ್ರವೇ ಇಲ್ಲ. ಕೆಲವು ಧರ್ಮಗಳು ಒಂದು ವ್ಯಕ್ತಿತ್ವವನ್ನು ನಂಬದೇ ಬರೀ ಒಂದು ಸ್ಥಿತಿಯನ್ನು ನಂಬಬಹುದು. ಅಂತಹ ಒಂದು ಸ್ಥಿತಿಯು ಕೂಡ ಪರಮಾತ್ಮನ ಕಿರಣವೇ ಆಗಿರುವುದು. ವ್ಯಕ್ತಿಯಂತೆ ತೋರುವವನು ಅವನೇ. ತತ್ತ್ವದಂತೆ ತೋರುವವನು ಅವನೇ. ದಾರಿಯೂ ಅವನೇ, ಗುರಿಯೂ ಅವನೇ.

\begin{verse}
ಯೋ ಮಾಮೇವಮಸಂಮೂಢೋ ಜಾನಾತಿ ಪುರುಷೋತ್ತಮಮ್ ।\\ಸ ಸರ್ವವಿದ್ಭಜತಿ ಮಾಂ ಸರ್ವಭಾವೇನ ಭಾರತ \versenum{॥ ೧೯ ॥}
\end{verse}

{\small ಅರ್ಜುನ, ಯಾರು ಮೋಹಿತನಾಗದೆ ನನ್ನನ್ನು ಹೀಗೆ ಪುರುಷೋತ್ತಮನೆಂದು ಅರಿಯುವನೋ ಆ ಸರ್ವಜ್ಞನು ನನ್ನನ್ನು ಪೂರ್ಣಭಾವದಿಂದ ಭಜಿಸುತ್ತಾನೆ.}

ಅವನನ್ನು ಪುರುಷೋತ್ತಮ ಎಂದು ಅರಿಯಬೇಕಾದರೆ ಮೋಹದಿಂದ ಪಾರಾಗಬೇಕು. ಇಲ್ಲದೇ ಇದ್ದರೆ, ಅವನು ಹಾಕಿಕೊಂಡಿರುವ ತಾತ್ಕಾಲಿಕ ವೇಷವನ್ನು ಮಾತ್ರ ನೋಡುವನು. ವೇಷದ ಹಿಂದೆ ಇರುವ ಪರಮಾತ್ಮ ಗೋಚರಿಸುವುದಿಲ್ಲ. ಅವತಾರ ಪ್ರಪಂಚಕ್ಕೆ ಬಂದಾಗ ಅದನ್ನು ಅವತಾರ ಎಂದು ತಿಳಿದುಕೊಳ್ಳುವುಕ್ಕೆ ಸಾಧ್ಯವಿಲ್ಲ. ನಾವು ಮೋಹದಿಂದ ಪಾರಾಗಿದ್ದರೆ ಮಾತ್ರ ಅವನನ್ನು ಅರಿಯ ಬಹುದು. ಇಲ್ಲದೇ ಇದ್ದರೆ ನಮ್ಮಂತೆಯೆ ಅವನೂ ಒಬ್ಬ ಎಂದು ತಿಳಿಯುವೆವು. ಒಂದು ಸ್ವಲ್ಪ ಚೌಕಾಶಿ ಮಾಡಿದರೆ, ನಮಗಿಂತ ಮೇಲೆ ಎಂದು ಒಪ್ಪಿಕೊಳ್ಳುವೆವು. ಅದಕ್ಕಿಂತ ಹೆಚ್ಚು ಹೋಗು ವವರು ಅಪರೂಪ. ನಾವು ಅವನ ಅಂಶ ಎಂಬುದನ್ನು ಚೆನ್ನಾಗಿ ಅರಿತಾಗ ಮಾತ್ರ ಅವನನ್ನು ಪುರುಷೋತ್ತಮನೆಂದು ಅರಿಯಬಹುದು. ಎಲ್ಲಿಯವರೆಗೆ ನಾವು ನಮ್ಮಲ್ಲಿರುವ ದೈವತ್ವವನ್ನು ಮರೆತಿರುವೆವೊ ಅಲ್ಲಿಯವರೆಗೆ ಭಗವಂತನ ದೈವತ್ವವನ್ನು ಅರಿಯುವುದಕ್ಕೆ ಆಗುವುದಿಲ್ಲ. ನಮಗೂ ಅವನಿಗೂ ಸಾಮಾನ್ಯವಾಗಿರುವುದು ಯಾವುದಾದರೂ ಒಂದನ್ನು ಅರಿತರೆ ಮಾತ್ರ ಅವನನ್ನು ಅರಿಯಲು ಸಾಧ್ಯ. ಯಾವಾಗ ನಾವು ನಮ್ಮಲ್ಲಿರುವ ಪರಮಾತ್ಮನ ಅಂಶವನ್ನು ಮರೆತು, ಕೆಲಸಕ್ಕೆ ಬಾರದ ದೇಹ ಇಂದ್ರಿಯ ಮನಸ್ಸು ಬುದ್ಧಿ ಎಂದು ಭಾವಿಸುತ್ತಿರುವೆವೋ ಅಲ್ಲಿಯವರೆಗೆ ನಾವು ಅವನನ್ನು ಪುರಷೋತ್ತಮ ಎಂದು ಅರಿಯುವುದಕ್ಕೆ ಆಗುವುದಿಲ್ಲ.

ಯಾವಾಗ ನಾವು ಭಗವಂತನನ್ನು ಅರಿಯುತ್ತೇವೆಯೋ ಆಗ ನಾವು ಕೂಡ ಸರ್ವಜ್ಞರಾಗುತ್ತೇವೆ. ಅವನ ಧರ್ಮವೇ ನಮಗೆ ಹರಿದು ಬರುವುದು. ನದಿ ಸಾಗರಕ್ಕೆ ಸೇರುತ್ತಿದ್ದರೆ, ಸಾಗರದ ನೀರೆಲ್ಲ ನದೀ ಮುಖಕ್ಕೆ ಹರಿದು ಬರುವಂತೆ, ನದಿಯೇ ಸಾಗರದ ಒಂದು ಭಾಗವಾಗಿರುವಂತೆ ಕಾಣುವುದು. ಇಲ್ಲಿ ಸರ್ವಜ್ಞ ಎಂದರೆ, ಎಲ್ಲವನ್ನು ಅಕ್ಷರಶಃ ತಿಳಿದುಕೊಳ್ಳುವುದಲ್ಲ. ಒಬ್ಬ ಭಗವಂತನನ್ನು ಅರಿತುಕೊಂಡು ಮುಕ್ತನಾಗಿದ್ದಾನೆ ಎಂದು ಭಾವಿಸೋಣ. ಅವನು ಸರ್ವಜ್ಞನಾದರೆ, ಅವನಿಗೆ ಟ್ರಿಗ್ನಾಮೆಟ್ರಿ, ಫಿಸಿಕ್ಸ್, ಕೆಮಿಸ್ಟ್ರಿ ಮುಂತಾದುವುಗಳಿಗೆ ಸಂಬಂಧಪಟ್ಟ ಪ್ರಶ್ನೆಗಳನ್ನು ಹಾಕಿದರೆ ಅವನು ಉತ್ತರಿಸುತ್ತಾನೆಯೆ? ಇಲ್ಲ. ಸರ್ವಜ್ಞ ಎಂದರೆ ಈ ದೃಷ್ಟಿಯಿಂದಲ್ಲ. ಭಗವಂತನ ಸಾಕ್ಷಾತ್ಕಾರವಾದ ವನಿಗೆ ಬರುವ ಸರ್ವಜ್ಞತೆ, ಎಂದರೆ ಎಲ್ಲದರ ಹಿಂದೆಯೂ ಇರುವವನು ಪರಮಾತ್ಮನೆ ಎಂಬುದನ್ನು ಅರಿಯುತ್ತಾನೆ. ಅದು ಶಕ್ತಿಯಾಗಬಹುದು, ವ್ಯಕ್ತಿಯಾಗಬಹುದು, ಜಡವಾಗಬಹುದು, ಚೇತನವಾಗ ಬಹುದು. ಬ್ರಹ್ಮಾಂಡದಲ್ಲಿ ಬಹುರೂಪವನ್ನು ಧರಿಸಿರುವ ಯಾವುದಾದರೂ ವಸ್ತುವಾಗಿರಬಹುದು. ಅವನಿಗೆ ಗೊತ್ತಿದೆ ಎಲ್ಲದರ ಹಿಂದೆಯೂ ಅವನೇ ಇರುವವನು ಎಂಬುದು. ಅವನು ಮನಸ್ಸುಮಾಡಿ ದರೆ ಚೂರುಪಾರು ವಿದ್ಯೆಗಳನ್ನು ಬೇಕಾದರೆ ಇತರರೆಲ್ಲರಿಗಿಂತ ಚೆನ್ನಾಗಿ ತಿಳಿದುಕೊಳ್ಳಬಲ್ಲ. ಆದರೆ ಅವನು ಅದಕ್ಕೆ ಆಶಿಸುವುದಿಲ್ಲ. ಭೂಮವನ್ನು ರುಚಿ ನೋಡಿದವನು ಅಲ್ಪಕ್ಕೆ ಮನಸ್ಸು ಹಾಕುವುದಿಲ್ಲ.

ಅಂತಹ ಸರ್ವಜ್ಞ ಭಗವಂತನನ್ನು ಪೂರ್ಣಭಾವದಿಂದ ಭಜಿಸುತ್ತಾನೆ. ಅವನನ್ನು ಚೆನ್ನಾಗಿ ಅರಿತಮೇಲೆಯೇ, ನಮ್ಮ ಹೃದಯದ ಪ್ರೇಮಚಿಲುಮೆ ಜಿನುಗುವುದು. ಆಗ ಅವನನ್ನು ಸ್ವಾಭಾವಿಕ ವಾಗಿ ಪ್ರೀತಿಸುತ್ತೇವೆ. ಅದಕ್ಕಿಂತ ಮುಂಚೆ, ಪ್ರೀತಿಸುವುದನ್ನು ಕಲಿಯುತ್ತಿರುತ್ತೇವೆ. ಅದೊಂದು ಯಾಂತ್ರಿಕವಾಗಿರುವ ಪ್ರೀತಿ. ಭಾವದಿಂದ ತುಂಬಿದ ಪ್ರೀತಿಯಲ್ಲ. ಯಾವಾಗ ಭಕ್ತನ ಹೃದಯ ಭಾವದಿಂದ ತುಂಬುವುದೊ, ಆಗ ನಿಜವಾಗಿ ಭಗವಂತನನ್ನು ಪ್ರೀತಿಸುತ್ತಾನೆ. ಆ ಪ್ರೀತಿಗೆ ಸಮನಾ ಗಿರುವುದು ಪ್ರಪಂಚದಲ್ಲಿ ಮತ್ತಾವುದೂ ಇಲ್ಲ. ಆ ಪ್ರೀತಿಗೆ ತನ್ನ ಸರ್ವಸ್ವವನ್ನೂ ಅರ್ಪಿಸುತ್ತಾನೆ. ಆ ಪ್ರೀತಿಯ ಸರೋವರದಲ್ಲಿ ಮೀನು ನೀರಿನಲ್ಲಿ ತೇಲುವಂತೆ ತೇಲುವನು. ಅವನನ್ನು ಪೂರ್ಣಭಾವ ದಿಂದ ಪ್ರೀತಿಸುತ್ತಾನೆ ಎಂದರೆ, ಅವನ ಹೃದಯ ಭಗವಂತನ ಪ್ರೀತಿಯಿಂದ ತುಂಬಿ ತುಳುಕಾಡು ತ್ತಿರುವುದು. ಹಸುವಿನ ಕೆಚ್ಚಲಲ್ಲಿ ಹಾಲು ತುಂಬಿಕೊಂಡಿರುವಂತೆ ಅವನ ಹೃದಯದಲ್ಲಿ ಪ್ರೀತಿ ತುಂಬಿಕೊಂಡಿದೆ.

\begin{verse}
ಇತಿ ಗುಹ್ಯತಮಂ ಶಾಸ್ತ್ರಮಿದಮುಕ್ತಂ ಮಯಾನಘ ।\\ಏತದ್ಬುದ್ಧ್ವಾ ಬುದ್ಧಿಮಾನ್ ಸ್ಯಾತ್ ಕೃತಕೃತ್ಯಶ್ಚ ಭಾರತ \versenum{॥ ೨ಂ ॥}
\end{verse}

{\small ಪಾಪರಹಿತನಾದ ಅರ್ಜುನ, ರಹಸ್ಯಾತಿರಹಸ್ಯವಾದ ಈ ಶಾಸ್ತ್ರ ನನ್ನಿಂದ ನಿನಗೆ ಹೇಳಲ್ಪಟ್ಚಿತು. ಇದನ್ನರಿತ ಮನುಷ್ಯ ಬುದ್ಧಿವಂತನೂ, ಕೃತ್ಯಕೃತ್ಯನೂ ಆಗುತ್ತಾನೆ.}

ಇಲ್ಲಿ ಶ್ರೀಕೃಷ್ಣ ಅತ್ಯಂತ ರಹಸ್ಯವಾದ ಶಾಸ್ತ್ರವನ್ನು ಅರ್ಜುನನಿಗೆ ಹೇಳುತ್ತಾನೆ. ಇದು ರಹಸ್ಯಕ್ಕೆಲ್ಲ ರಹಸ್ಯ. ಜೀವ ಜಗತ್ ಈಶ್ವರ ಇವರಿಗೆ ಸಂಬಂಧಪಟ್ಟ ವಿಷಯಗಳು ಇಲ್ಲಿ ಬರುತ್ತವೆ. ಪುರುಷೋ ತ್ತಮನೇ ಪರಮಾತ್ಮ. ಅವನ ಅಂಶವೇ ಜೀವ, ಅವನು ಸೃಷ್ಟಿಸಿದ್ದೇ ಈ ಪ್ರಕೃತಿ. ಇಲ್ಲೆಲ್ಲ ಅವನೇ ಇರುವನು. ಜೀವನದಲ್ಲಿ ಪರಮಾತ್ಮನೊಬ್ಬನೇ ಮುಖ್ಯ ಸತ್ಯ. ಮುಂಚೆ ಅವನು, ಅನಂತರ ಈ ಜಗತ್ತು. ಇದನ್ನು ರಹಸ್ಯ ಎಂದು ಹೇಳುತ್ತಾನೆ. ಇದು ಗುಪ್ತವಾಗಿರುವುದು, ಕಾಣದೇ ಇರುವುದು, ಸೂಕ್ಷ್ಮವಾಗಿರುವುದು ಎಂದೆಲ್ಲ ಬೇಕಾದರೆ ಹೇಳಬಹುದು. ಪುರುಷೋತ್ತಮನೇ ಅಂತಹ ದೊಡ್ಡ ಸತ್ಯವಾದರೂ, ಅವನು ಈಗ ನಮಗೆ ಕಾಣುತ್ತಿಲ್ಲ. ಅವನಲ್ಲದ, ಅವನ ಮೇಲೆ ಆರೋಪಿತವಾದ, ಅವನನ್ನು ಕಾಣದಂತೆ ಮಾಡುವುದೆಲ್ಲ ಕಾಣುತ್ತಿದೆ. ಆದರೆ ಅವನು ಮಾತ್ರ ಕಾಣಿಸುವುದಿಲ್ಲ. ಅವನನ್ನು ನಾವು ಹುಡುಕಬೇಕಾದರೆ ಎಲ್ಲೊ ಹೊರಗೆ ಹುಡುಕಿಕೊಂಡು ಹೋಗಬೇಕಾಗಿಲ್ಲ. ನಮಗೆ ಅತ್ಯಂತ ಸಮೀಪದಲ್ಲಿರುವನು. ನಾವೆಲ್ಲ ನಂಬುವುದು ಈ ದೇಹವನ್ನು. ಜಡವಾದಿಯಾಗಲಿ ಸಂದೇಹವಾದಿಯಾಗಲಿ, ಚಾರ್ವಾಕನಾಗಲಿ ಎಲ್ಲರೂ ದೇಹವನ್ನು ನಂಬುತ್ತಾರೆ. ದೇವರು ಈ ದೇಹದ ಒಳಗೆ ಅದ್ಭುತ ವಿಚಿತ್ರವನ್ನು ಮಾಡುತ್ತಿರುವನು. ನಾವು ತಿಂದದ್ದನ್ನು ಅರಗಿಸುತ್ತಿರುವನು. ನಮ್ಮ ಆಲೋಚನೆಯ ಹಿಂದೆ ಇರುವನು, ನಮ್ಮ ಇಂದ್ರಿಯದ ಹಿಂದೆ ಇರುವನು. ನಾನೆಂಬ ಅಹಂಕಾರದ ಹಿಂದೆ ಬೆಳಗುವವನು ಅವನು. ರಹಸ್ಯವೆಂದರೆ ಇದೇ. ನಾವೇ ಅವನಮೇಲೆ ಕುಳಿತುಕೊಂಡು ಎಲ್ಲೆಲ್ಲಿಯೋ ಹುಡುಕುತ್ತಿರುವೆವು. ಕೊನೆಗೆ ನಮಗೆ ಗೊತ್ತಾಗುವುದು, ನಾವು ಅದನ್ನು ಹೊರಗೆ ಹುಡುಕಬೇಕಾಗಿಲ್ಲ. ಅಂತರವನ್ನು ಶೋಧಿಸಿದರೆ ಸಾಕು ಎಂಬುದು. ಇಲ್ಲಿ ಇಂತಹ ರಹಸ್ಯವಾದ ಶಾಸ್ತ್ರವನ್ನು ಅರ್ಜುನನಿಗೆ ಹೇಳುತ್ತಾನೆ. ಏಕೆಂದರೆ ಅರ್ಜುನ ಶ್ರೀಕೃಷ್ಣನನ್ನು ಗುರುವೆಂದು ಸ್ವೀಕರಿಸು ತ್ತಾನೆ, ಅವನಲ್ಲಿ ಶರಣಾಗಿದ್ದಾನೆ, ಯಾವುದು ತನಗೆ ಶ್ರೇಯಸ್ಕರವೋ ಅದನ್ನು ಹೇಳು ಎಂದು ಕೇಳಿಕೊಂಡಿದ್ದಾನೆ. ಭಗವಂತ ತನ್ನ ಭಕ್ತನಿಗೆ ಏನನ್ನು ಬೇಕಾದರೂ ಮಾಡುತ್ತಾನೆ, ಏನನ್ನು ಬೇಕಾದರೂ ಕೊಡುತ್ತಾನೆ. ನಾವು ಅವನ ಭಕ್ತರಾಗಬೇಕು. ಆಗ ಇದು ಅರಿವಾಗುವುದು.

ಶ್ರೀಕೃಷ್ಣ ಈ ಅಧ್ಯಾಯದಲ್ಲಿ ಹೇಳಿರುವುದನ್ನು ಮನುಷ್ಯ ಯಾವಾಗ ತಿಳಿದುಕೊಳ್ಳುತ್ತಾನೋ ಅವನು ಬುದ್ಧಿವಂತನಾಗುತ್ತಾನೆ. ಈ ಅಧ್ಯಾಯದಲ್ಲಿ ಗೀತಾರ್ಥಸಾರವೆಲ್ಲ ಇದೆ. ಇಲ್ಲಿ ಸಕಲ ವೇದಾರ್ಥಸಾರವೂ ಇದೆ. ಯಾವಾಗ ಇದನ್ನು ತಿಳಿದುಕೊಳ್ಳುತ್ತಾನೆಯೋ ಆಗ ಕೃತಕೃತ್ಯನಾಗುತ್ತಾನೆ, ಮನುಷ್ಯನಾಗಿ ಹುಟ್ಟಿದ್ದಕ್ಕೆ ಏನನ್ನು ಮಾಡಬೇಕೋ ಅದನ್ನು ಮಾಡಿದವನಾಗುತ್ತಾನೆ.

ಮನುಷ್ಯ ಭಗವಂತನನ್ನು ತಿಳಿದುಕೊಳ್ಳಬೇಕು. ಅದೇ ಜೀವನದ ಚರಮಗುರಿ. ಯಾವಾಗ ನಾವು ಅವನನ್ನು ತಿಳಿದುಕೊಳ್ಳುತ್ತೇವೆಯೋ, ನಮ್ಮ ಯಾತ್ರೆ ಪೂರೈಸಿದಂತೆ. ಇನ್ನು ಮೇಲೆ ನಾವು ಯಾವುದಕ್ಕೂ ಬದ್ಧರಲ್ಲ. ಈ ಸಂಸಾರದಲ್ಲಿ ಯಾವ ಸಾಲವನ್ನೂ ತೀರಿಸಬೇಕಾಗಿಲ್ಲ. ಪುಣ ವಿಮುಕ್ತರಾಗುತ್ತೇವೆ.

