
\chapter{ಧ್ಯಾನಯೋಗ}

ಧ್ಯಾನಯೋಗ ಅನುಷ್ಠಾನ ಪ್ರಧಾನವಾಗಿರುವುದು. ನಾವು ಬೇಕಾದಷ್ಟು ಸಿದ್ಧಾಂತಗಳನ್ನು ತಿಳಿದು\-ಕೊಂಡಿರಬಹುದು. ಆದರೆ ಜೀವನದಲ್ಲಿ ಅದೆಲ್ಲ ನಮ್ಮದಾಗಿರುವುದಿಲ್ಲ. ಎಷ್ಟನ್ನು ನಾವು ಅರಗಿಸಿಕೊಂಡಿರುವೆವೊ ಅದು ಮಾತ್ರ ನಮ್ಮದು. ಮಿಕ್ಕಿರುವುದಾವುದೂ ಪರೀಕ್ಷಾ ಸಮಯದಲ್ಲಿ ನಮ್ಮದಾಗುವುದಿಲ್ಲ. ಒಬ್ಬ ಬ್ಯಾಂಕಿನಲ್ಲಿ ಕೆಲಸ ಮಾಡುವವನು, ಬೆಳಗಿನಿಂದ ಸಾಯಂಕಾಲದವರೆಗೆ ಕಂತೆ ಕಂತೆ ನೋಟನ್ನು ಎಣಿಸಿ ಕೊಡುತ್ತಿರಬಹುದು, ಅಥವಾ ತೆಗೆದುಕೊಳ್ಳುತ್ತಿರಬಹುದು. ಇದಾವುದೂ ಅವನದಲ್ಲ. ಎಷ್ಟನ್ನು ಅವನು ತಿಂಗಳ ಕೊನೆಯಲ್ಲಿ ಸಂಬಳರೂಪವಾಗಿ ಪಡೆಯು\-ವನೊ ಅದು ಮಾತ್ರ ಅವನದು. ಹಾಗೆಯೇ ನಾವು ಶಾಸ್ತ್ರಾದಿಗಳಿಂದ ಹಲವು ವಿಷಯಗಳನ್ನು ತಿಳಿದುಕೊಂಡಿರಬಹುದು. ಮಾತೆತ್ತಿದರೆ ಅದನ್ನು ಉದಹರಿಸುತ್ತಲೂ ಇರಬಹುದು. ಆದರೆ ಇವು ನಮ್ಮ ಜೀವನದಲ್ಲಿ ಬೆರೆತಿದ್ದರೆ ಮಾತ್ರ ಪ್ರಯೋಜನ. ಶ‍್ರೀಕೃಷ್ಣ ಈ ಅಧ್ಯಾಯದಲ್ಲಿ ಆಧ್ಯಾತ್ಮಿಕ ಜೀವನದಲ್ಲಿ ನಾವು ನಿಜವಾಗಿ ಮುಂದುವರಿಯಬೇಕಾದರೆ ಏನೇನು ಮಾಡಬೇಕು ಎಂದು ವಿವರಿಸುತ್ತಾನೆ. ಒಂದು ಸಿದ್ಧಾಂತವನ್ನು ಪ್ರತಿಪಾದಿಸುವಾಗ ಈ ವಿಷಯಗಳು ಬರುವುದಿಲ್ಲ. ಒಬ್ಬ ಒಂದು ಚೆನ್ನಾಗಿ ಕಾಣುವ ಕಟ್ಟಡದ ನಕ್ಷೆ ಬರೆಯುತ್ತಾನೆ. ಅದನ್ನು ನೋಡಿ ನಾವು ಮಾರುಹೋಗುತ್ತೇವೆ. ಆ ನಕ್ಷೆಯಲ್ಲಿರುವುದನ್ನೆಲ್ಲ ಕಟ್ಟಬೇಕಾದರೆ, ಆ ನಕ್ಷೆಯಲ್ಲಿ ಬರೆಯದ ಮತ್ತೆ ಏನೇನೊ ನಾವು ಮಾಡಬೇಕಾಗಿದೆ. ತಳಪಾಯ ತೋಡಬೇಕು, ಕಾಲಾವಧಿ ಕಟ್ಟಬೇಕು, ಹಲವು ಜನ ಕೂಲಿಗಳನ್ನು ಸಂಗ್ರಹಿಸಬೇಕು. ಹಲವು ಸಾಮಗ್ರಿಗಳನ್ನು ಸಂಗ್ರಹಿಸಬೇಕು. ಅವುಗಳನ್ನೆಲ್ಲ ಹೇಗೆ ಉಪಯೋಗಿಸಿಕೊಳ್ಳಬೇಕೆಂಬುದು ನಮಗೆ ಚೆನ್ನಾಗಿ ಗೊತ್ತಿರಬೇಕು. ಆಗಲೆ ಆ ಮನೆಯ ನಕ್ಷೆ ಒಂದು ಜೀವಂತ ಮನೆಯಾಗಬೇಕಾದರೆ.

ಸ್ಥಿತಪ್ರಜ್ಞನ ಲಕ್ಷಣಗಳನ್ನು ನಾವು ಓದಿದಾಗ, ಎಲ್ಲಾ ಕಡೆಯಲ್ಲಿಯೂ ವಾಸುದೇವನ ದರ್ಶನವನ್ನು ಮಾಡುವ ವ್ಯಕ್ತಿಯನ್ನು ನೋಡಿದಾಗ ಈ ಅನುಭವ ಒಂದು ಶಿಖರದ ಮೇಲೆ ನಿಂತು ನೋಡಿದಾಗ ಕಾಣುವ ದೃಶ್ಯದಂತೆ. ಇದನ್ನು ಪಡೆಯಬೇಕಾದರೆ ಧ್ಯಾನದ ಮೆಟ್ಟಲಿನ ಮೂಲಕ ಏರಬೇಕು. ಈ ಧ್ಯಾನಕ್ಕೆ ಸಹಾಯಕವಾಗಿ, ಶೀಲ, ಕರ್ಮ, ಜ್ಞಾನಗಳೆಲ್ಲ ಇರಬೇಕು. ಆಗಲೆ ಈ ಮನೋಹರವಾದ ದೃಶ್ಯವನ್ನು ನೋಡುವ ಶಿಖರವನ್ನು ಏರಬೇಕಾದರೆ. ಈ ಅಧ್ಯಾಯದಲ್ಲಿ ಬರುವುದೇ ಆ ಧ್ಯಾನದ ಮೆಟ್ಟಲುಗಳು. ಪ್ರಥಮದಲ್ಲಿಯೇ ಶ‍್ರೀಕೃಷ್ಣ ಹೀಗೆ ಪ್ರಾರಂಭಿಸುವನು.

\begin{shloka}
ಅನಾಶ್ರಿತಃ ಕರ್ಮಫಲಂ ಕಾರ್ಯಂ ಕರ್ಮ ಕರೋತಿ ಯಃ~।\\ಸ ಸಂನ್ಯಾಸೀ ಚ ಯೋಗೀ ಚ ನ ನಿರಗ್ನಿರ್ನ ಚಾಕ್ರಿಯಃ \hfill॥ ೧~॥
\end{shloka}

\newpage

\begin{artha}
ಕರ್ಮಫಲವನ್ನು ಇಚ್ಛಿಸದೇ, ಮಾಡಬೇಕಾದ ಕರ್ಮವನ್ನು ಯಾರು ಮಾಡುತ್ತಿರುವನೊ ಅವನೇ ಸಂನ್ಯಾಸಿ ಮತ್ತು ಯೋಗಿ. ಅಗ್ನಿ ಇಲ್ಲದವನೂ ಅಲ್ಲ, ಕರ್ಮ ಮಾಡದೇ ಇರುವವನೂ ಅಲ್ಲ.
\end{artha}

ನಾವು ಪ್ರಪಂಚದಲ್ಲಿ ಎರಡು ಬಗೆಯ ವ್ಯಕ್ತಿಗಳನ್ನು ನೋಡುತ್ತೇವೆ. ಒಂದು ಬಾಹ್ಯ ಪ್ರಕೃತಿಯ ಸ್ವಭಾವವುಳ್ಳವರು, ಮತ್ತೊಬ್ಬರು ಅಂತರ್ಮುಖರು. ಒಂದು ಬಗೆಯ ಜನ ಕರ್ಮವನ್ನು ಮಾಡುತ್ತಿರುವವರು, ಮತ್ತೊಂದು ಬಗೆಯ ಜನ ವಿಚಾರವನ್ನು ಮಾಡುತ್ತಿರುವವರು. ಒಬ್ಬರು ಗೃಹಸ್ಥರು ಮತ್ತೊಬ್ಬರು ಸಂನ್ಯಾಸಿಗಳು. ಸುಮ್ಮನೆ ಒಬ್ಬ ಹೊರಗೆ ಮಾಡುತ್ತಿರುವ ಕ್ರಿಯೆಯನ್ನು ಮಾತ್ರ ನೋಡಿ ಒಬ್ಬ ಸಂನ್ಯಾಸಿಯೇ ಅಥವಾ ಕರ್ಮಯೋಗಿಯೇ ಎಂದು ನಿಶ್ಚಯಿಸುವುದಕ್ಕೆ ಆಗುವುದಿಲ್ಲ. ಅವರು ಮಾಡುತ್ತಿರುವ ಕೆಲಸದ ಹಿಂದೆ ಇರುವ ಉದ್ದೇಶವನ್ನು ತೆಗೆದುಕೊಳ್ಳಬೇಕಾಗಿದೆ. ಶ‍್ರೀಕೃಷ್ಣ, ಇಲ್ಲಿ ಸಂನ್ಯಾಸಿಯ ಹಿಂದೆ ಇರುವುದು, ಗೃಹಸ್ಥನ ಹಿಂದೆ ಇರುವುದು, ಜ್ಞಾನಿಯ ಹಿಂದೆ ಇರುವುದು, ಕರ್ಮಿಯ ಹಿಂದೆ ಇರುವುದನ್ನು ನೋಡುತ್ತಾನೆ. ಮುಂದೆ ಹಾಕಿಕೊಂಡಿರುವ ವೇಷ ಅಲ್ಲ ಮುಖ್ಯ. ಆ ವೇಷದ ಹಿಂದೆ ಇರುವ ಮನೋಧರ್ಮ ಅತ್ಯಂತ ಮುಖ್ಯವಾದುದು. ನಿಜವಾದ ಸಂನ್ಯಾಸಿ ಯಾರು? ಕರ್ಮಫಲವನ್ನು ಇಚ್ಛಿಸದೇ ಕರ್ಮವನ್ನು ಮಾಡುತ್ತಿರುವವನು. ಇಲ್ಲಿ ಸಂನ್ಯಾಸಿ ಸಾಧಾರಣ ಗೃಹಸ್ಥನಂತೆ ಮನೆಗೆ ಸಮಾಜಕ್ಕೆ ದೇಶಕ್ಕೆ ಸೇರಿದ ಕರ್ಮವನ್ನು ಮಾಡುತ್ತಿಲ್ಲ. ಅವನು ಇವುಗಳನ್ನೆಲ್ಲ ಮೀರಿರುವನು. ಅವನ ಕರ್ಮವೇ ಜ್ಞಾನವನ್ನು ಆರ್ಜಿಸುವುದು ಮತ್ತು ಇತರರಿಗೆ ಅದನ್ನು ಹೇಳುವುದು. ಹೀಗೆ ಅವನು ಸಂಗ್ರಹಿಸುವಾಗ ಮತ್ತು ಅದನ್ನು ನೀಡುವಾಗ ಅವನು ಯಾವ ಫಲವನ್ನೂ ಆಶಿಸಬಾರದು. ಕೀರ್ತಿಗಾಗಿ ದ್ರವ್ಯಕ್ಕಾಗಿ ಅದನ್ನು ಮಾಡಬಾರದು. ಜ್ಞಾನವನ್ನು ಮತ್ತೊಬ್ಬನಿಗೆ ಕೊಡುವಾಗ ತಾನು ತುಂಬಾ ದೊಡ್ಡವನು ಎಂದು ಭಾವಿಸುವುದು, ಸ್ವೀಕರಿಸುವವನು ತನಗೆ ಕೃತಜ್ಞನಾಗಿರಬೇಕೆಂದು ಭಾವಿಸುವುದು, ತಾನು ಪ್ರಪಂಚಕ್ಕೆ ಮಹದುಪಕಾರವನ್ನು ಮಾಡಿದವನೆಂದು ಭಾವಿಸುವುದು, ಸಮಾಜ ಅಡ್ಡಬಿದ್ದು ತನ್ನನ್ನು ಪುರಸ್ಕರಿಸಬೇಕೆಂದು ಭಾವಿಸುವುದು, ಇವುಗಳೆಲ್ಲ ಅವನು ಮಾಡುವ ಕೆಲಸಕ್ಕೆ ಕಳಂಕವನ್ನು ತರುವುವು. ಈತ ಒಳ್ಳೆಯ ಕೆಲಸವನ್ನೇನೊ ಮಾಡುತ್ತಿರುವನು. ಆದರೆ ಶ‍್ರೀಕೃಷ್ಣ ಆ ಕೆಲಸದ ಹಿಂದೆ ಇರುವ ಉದ್ದೇಶ ಯಾವುದು ಎಂಬುದನ್ನು ಗಮನಿಸುತ್ತಾನೆ. ಅನೇಕ ವೇಳೆ ತುಂಬಾ ಒಳ್ಳೆಯ ಕೆಲಸವನ್ನು ಕೆಟ್ಟ ದೃಷ್ಟಿಯಿಂದ ಮಾಡುತ್ತಿರಬಹುದು. ಇದರಿಂದ ನಾವು ಹೆಚ್ಚು ಬಂಧನಕ್ಕೆ ಒಳಗಾಗುವೆವು. ಅನೇಕ ವೇಳೆ ತುಂಬಾ ನಿಕೃಷ್ಟವಾದ ಕೆಲಸವನ್ನು ಒಬ್ಬ ಒಳ್ಳೆಯ ದೃಷ್ಟಿಯಿಂದ ಮಾಡುತ್ತಿರಬಹುದು. ಊರನ್ನು ಗುಡಿಸುವವರು, ಪೋಲಿಸಿನವರು, ನ್ಯಾಯಾಲಯಗಳು, ಮಿಲಿಟರಿ ಇದಕ್ಕೆ ಸೇರಿರುವರೆಲ್ಲ ತತ್ಕಾಲಕ್ಕೆ ಅಹಿತವಾಗಿ ಕಾಣುವ ಕೆಲಸವನ್ನು ಮಾಡುತ್ತಿರುವರು. ಆದರೆ ಅದರ ಹಿಂದೆ ಇರುವ ಉದ್ದೇಶವನ್ನು ನೋಡಬೇಕು. ಫಲಾಪೇಕ್ಷೆಯಿಲ್ಲದೆ, ಇದು ನಮ್ಮ ಪಾಲಿಗೆ ಬಂದ ಕರ್ತವ್ಯ, ಯಾರು ಏನು ಅಂದುಕೊಂಡರೂ ಚಿಂತೆ ಇಲ್ಲ, ನಾವು ನಿರ್ವಂಚನೆಯಿಂದ ಈ ಕೆಲಸವನ್ನು ಮಾಡಬೇಕು, ಎಂಬ ಭಾವವಿದ್ದರೆ, ಅವರು ಮಾಡುವ ಕೆಲಸದಿಂದ ಬಾಧಿತರಾಗುವುದಿಲ್ಲ.

ಶ‍್ರೀಕೃಷ್ಣ, ಸಂನ್ಯಾಸಿ ಮತ್ತು ಕರ್ಮಯೋಗಿ ಇವರಿಬ್ಬರ ಹಿಂದೆ ಹೋಗಿ ಅಲ್ಲಿ ಒಂದು ಸಾಮರಸ್ಯ ಬರುವಂತೆ ಮಾಡುವನು. ನಿಜವಾದ ಸಂನ್ಯಾಸಿಯಾಗಲಿ, ನಿಜವಾದ ಕರ್ಮಯೋಗಿಯಾಗಲಿ, ಇಬ್ಬರ ಹಿಂದೆಯೂ ಅವರು ಮಾಡುವ ಕೆಲಸದ ಹಿಂದೆ ಫಲಾಪೇಕ್ಷೆ ಇಲ್ಲ. ಅವರು ಮಾಡುವುದಕ್ಕೆ ಅಂಟಿಕೊಂಡಿರುವುದಿಲ್ಲ. ಹಿಂದಿನ ಕಾಲದಲ್ಲಿ ಮನೆಯಲ್ಲಿ ಯಾರು ಅಗ್ನಿಯನ್ನು ರಕ್ಷಿಸುತ್ತಿದ್ದರೊ ಅವರನ್ನು ಗೃಹಸ್ಥರೆಂದು ಕರೆಯುತ್ತಿದ್ದರು. ಅದರಂತೆಯೇ ಯಾರು ಅದಕ್ಕೆ ಅತೀತರಾಗಿರುತ್ತಿದ್ದರೊ ಅವರನ್ನು ಸಂನ್ಯಾಸಿಗಳು ಎಂದು ಕರೆಯುತ್ತಿದ್ದರು. ಇವೆಲ್ಲ ತೋರಿಕೆಯ ವಿವರಣೆ. ಶ‍್ರೀಕೃಷ್ಣನಾದರೊ ತೋರಿಕೆಯನ್ನು ಭೇದಿಸಿ ಅಂತರಾಳಕ್ಕೆ ಹೋಗುವನು. ತೋರಿಕೆಗೆ ಪರಸ್ಪರ ವಿರೋಧವಾದ ಮಾರ್ಗವನ್ನು ಹಿಡಿದವರ ಹಿಂದೆ ಇರುವ ಸಾಮಾನ್ಯ ಒಳ್ಳೆಯ ಗುಣವನ್ನು ನೋಡುವನು.

\begin{shloka}
ಯಂ ಸಂನ್ಯಾಸಮಿತಿ ಪ್ರಾಹುರ್ಯೋಗಂ ತಂ ವಿದ್ಧಿ ಪಾಂಡವ~।\\ನ ಹ್ಯಸಂನ್ಯಸ್ತಸಂಕಲ್ಪೋ ಯೋಗೀ ಭವತಿ ಕಶ್ಚನ \hfill॥ ೨~॥
\end{shloka}

\begin{artha}
ಯಾವುದನ್ನು ಸಂನ್ಯಾಸವೆಂದು ಹೇಳುವರೊ ಅದನ್ನೇ ಯೋಗವೆಂದು ತಿಳಿ. ಏಕೆಂದರೆ ಸಂಕಲ್ಪವನ್ನು ತ್ಯಾಗ ಮಾಡದೆ ಯಾರೂ ಯೋಗಿಯಾಗಲಾರರು.
\end{artha}

ಸಂನ್ಯಾಸಿ ಎಂದರೆ ಪ್ರಪಂಚದಲ್ಲಿ ಲೌಕಿಕ ಕೆಲಸಗಳನ್ನೆಲ್ಲ ತ್ಯಜಿಸಿ ಕೇವಲ ಆತ್ಮಸಾಕ್ಷಾತ್ಕಾರಕ್ಕಾಗಿ ನಿರತನಾಗಿರುವವನು. ಅವನಲ್ಲಿ ಯಾವ ಕೀರ್ತಿಯ ಆಸೆಯೂ ಇರಬಾರದು. ನಾನು ಮಹಾತಪಸ್ವಿ ಅಥವಾ ಜ್ಞಾನಿ ಎಂಬ ಅಹಂಕಾರ ಇರಬಾರದು. ಲೋಕ ತನ್ನನ್ನು ಗೌರವಿಸಬೇಕು, ಪೂಜ್ಯದೃಷ್ಟಿಯಿಂದ ನೋಡಬೇಕು ಮುಂತಾದವುಗಳು ಇರಬಾರದು. ಕೇವಲ ಸತ್ಯಸಾಕ್ಷಾತ್ಕಾರ ಒಂದೇ ಅವನ ಗುರಿ ಆಗಿರಬೇಕು.

ಯೋಗಿ ಎಂದರೆ ಅವನು ಹಲವಾರು ಕೆಲಸಗಳಲ್ಲಿ ನಿರತನಾಗಿರುವನು. ಅವು ಮನೆ, ಸಮಾಜ, ದೇಶ ಇವಕ್ಕೆ ಸಂಬಂದಪಟ್ಟಿರಬಹುದು. ಈ ಕರ್ಮಗಳ ಹಿಂದೆ ಫಲಾಪೇಕ್ಷೆ ಇರಬಾರದು. ಅಧಿಕಾರ, ಐಶ್ವರ್ಯ, ಯಾವುದನ್ನೂ ಅವನು ನಿರೀಕ್ಷಿಸಬಾರದು. ಅಂದರೆ ಎರಡರ ಹಿಂದೆಯೂ ಒಂದು ಸಾಮಾನ್ಯವಾದ ಭಾವನೆ ಇದೆ. ಅದೇ ಫಲಾಪೇಕ್ಷೆ ಇಲ್ಲದೆ ಇರುವುದು. ಇಬ್ಬರೂ ಮಾಡುವ ಕೆಲಸಗಳು ವ್ಯತ್ಯಾಸವಾಗಬಹುದು. ಒಬ್ಬ ತಪಸ್ಸು ಮಾಡಿದರೆ ಮತ್ತೊಬ್ಬ ಯಾವುದೋ ಲೌಕಿಕ ಕರ್ಮವನ್ನು ಮಾಡಬಹುದು. ಆದರೆ ಅದರ ಹಿಂದೆ ಒಂದು ಸಾಮಾನ್ಯ ನಿಲುವು ಇದೆ. ಅದು ಅತ್ಯಂತ ಮುಖ್ಯ. ಇಬ್ಬರೂ ಲೋಕಕ್ಕೆ ಅನಾಸಕ್ತರಾಗಿರಬೇಕು. ಇಬ್ಬರೂ ಮತ್ತೊಂದನ್ನು ಮಾಡಬೇಕಾಗಿದೆ. ಅದೇ ಸಂಕಲ್ಪವನ್ನು ತ್ಯಜಿಸಬೇಕು. ಇಲ್ಲಿ ಸಂಕಲ್ಪ ಎಂದರೆ ಮನಸ್ಸಿನಲ್ಲಿ ಚಿತ್ರಿಸಿಕೊಳ್ಳುವುದು, ಊಹಿಸಿಕೊಳ್ಳುವುದು. ನಾನು ಹೀಗೆ ಮಾಡುತ್ತೇನೆ, ಹಾಗೆ ಮಾಡುತ್ತೇನೆ, ಇದನ್ನು ಅನುಭವಿಸುತ್ತೇನೆ, ಇಲ್ಲಿ ಸಿಕ್ಕದೆ ಇದ್ದರೆ ಬೇರೆ ಕಡೆ, ಒಂದಲ್ಲದೆ ಇದ್ದರೆ ಮತ್ತೊಂದರಿಂದ ತನಗೆ ಪ್ರಿಯವಾದುವುಗಳನ್ನು ಈಡೇರಿಸಿಕೊಳ್ಳುತ್ತೇನೆ ಎಂದು ಮನದಲ್ಲಿ ಚಿಂತಿಸುತ್ತಿರುವುದು. ಇದನ್ನು ತ್ಯಜಿಸಿದಲ್ಲದೆ ಯಾರೂ ಯೋಗಿಗಳಾಗಲಾರರು. 

\begin{shloka}
ಆರುರುಕ್ಷೋರ್ಮುನೇರ್ಯೋಗಂ ಕರ್ಮ ಕಾರಣಮುಚ್ಯತೇ~।\\ಯೋಗಾರೂಢಸ್ಯ ತಸ್ಯೈವ ಶಮಃ ಕಾರಣಮುಚ್ಯತೇ \hfill॥ ೩~॥
\end{shloka}

\begin{artha}
ಯೋಗವನ್ನು ಹೊಂದಬೇಕೆಂದು ಇಚ್ಛಿಸುವ ಮುನಿಗೆ ಕರ್ಮ ಸಾಧನವಾಗುವುದು. ಯೋಗದಲ್ಲಿ ಆರೂಢನಾದವನಿಗೆ ಶಮ ಸಾಧನವಾಗುವುದು.
\end{artha}

ಯೋಗ ಎಂದರೆ ಒಂದುಗೂಡಿಸುವುದು ಎಂದು ಅರ್ಥ- ಇಲ್ಲಿ ಜೀವಾತ್ಮ ಪರಮಾತ್ಮನೊಡನೆ ಸಂಬಂಧವನ್ನು ಬೆಳೆಸುವುದು. ಈಗ ಮನಸ್ಸು ಪ್ರಪಂಚದ ಕಡೆ ಹರಿಯುತ್ತಿದೆ. ಯೋಗಿ, ಬಯಸಿದಾಗ ಆ ಮನಸ್ಸನ್ನು ದೇವರಕಡೆಗೆ ಹೋಗುವಂತೆ ಮಾಡಬೇಕು. ಮನಸ್ಸನ್ನು ದೇವರೆಡೆಗೆ ಹರಿಸಬೇಕಾದರೆ ಮೊದಲು ಅದರಲ್ಲಿರುವ ರಜೋಪ್ರವೃತ್ತಿ ಕಡಿಮೆಯಾಗಬೇಕು. ರಜಸ್ಸು ಎಂಬ ಬೆರಿಕೆ ಅದರಲ್ಲಿ ಹುದುಕಿಕೊಂಡಿದೆ. ಅದರಲ್ಲಿ ಹಲವಾರು ವಾಸನೆಗಳಿವೆ. ಮನಸ್ಸನ್ನು ಮೊದಲು ಅದರಿಂದ ಶುದ್ಧ ಮಾಡಬೇಕು. ಹಾಗೆ ಶುದ್ಧ ಮಾಡುವುದಕ್ಕೆ ಕರ್ಮ ಸಹಕಾರಿ ಆಗುವುದು. ಇದು ಮನಸ್ಸನ್ನು ಸ್ತಿಮಿತಕ್ಕೆ ತರುವುದು. ಮನಸ್ಸಿನ ಅಂತರಾಳದಲ್ಲಿರುವ ವಾಸನೆಗಳನ್ನು ಮೇಲಕ್ಕೆ ಬರುವಂತೆ ಮಾಡುವುದು. ಮೊಸರನ್ನು ಕಡೆಗೋಲಿನಿಂದ ಕಡೆದಾಗ ಹೇಗೆ ಅದರೊಳಗೆ ಇರುವ ಬೆಣ್ಣೆ ಮೇಲೆ ಬರುವುದೊ ಹಾಗೆ ಕರ್ಮ ಮನಸ್ಸಿನ ಅಂತರಾಳದಲ್ಲಿರುವುದನ್ನೆಲ್ಲ ಮೇಲಕ್ಕೆ ತರುವುದು. ಆಗ ನಮ್ಮಲ್ಲಿ ಏನೇನು ಇದೆ ಎಂಬುದು ಗೊತ್ತಾಗುವುದು. ಅದರಿಂದ ಪಾರಾಗಬೇಕಾದರೆ, ಫಲಾಪೇಕ್ಷೆ ಇಲ್ಲದೆ ಕರ್ಮ ಮಾಡಬೇಕು. ಅಥವಾ ಇದು ಭಗವದರ್ಪಿತವಾಗಲಿ ಎಂಬ ದೃಷ್ಟಿಯಿಂದ ಅದನ್ನು ಮಾಡಬೇಕು. ಆಗ ಚಿತ್ತ ಶುದ್ಧಿಯಾಗುವುದು. ಒಬ್ಬ ಒಂದು ಕಡೆ ಕುಳಿತುಕೊಂಡು ನಿರ್ಜನ ಪ್ರದೇಶದಲ್ಲಿ ಭಗವಂತನನ್ನು ಧ್ಯಾನಿಸಬೇಕಾದರೆ, ನಿತ್ಯಅನಿತ್ಯವಸ್ತುವಿವೇಚನೆ ಮಾಡಬೇಕಾದರೆ ಈ ಪ್ರಪಂಚದಲ್ಲಿ ತನ್ನ ಪಾಲಿಗೆ ಬಂದ ಕೆಲಸವನ್ನು ಸರಿಯಾಗಿ ಮಾಡಬೇಕಾದರೆ ಚಿತ್ತಶುದ್ಧಿ ಅತ್ಯಂತ ಆವಶ್ಯಕ. ಅದಿಲ್ಲದೇ ಯಾವ ಮಾರ್ಗದಲ್ಲಿಯೂ ಒಬ್ಬ ಮುಂದುವರಿಯು\-ವಂತೆಯೇ ಇಲ್ಲ.

ಯಾವಾಗ ಒಬ್ಬ ಪ್ರಯತ್ನಮಾಡಿ ಯೋಗದಲ್ಲಿ ನೆಲಸುತ್ತಾನೆಯೊ, ಆತನಿಗೆ ಇನ್ನು ಮೇಲೆ ಜಾರಿ ಹೋಗದಂತೆ ಇರಬೇಕಾದರೆ, ಚಿತ್ತ ಚಂಚಲವಾಗದೆ ಇರಬೇಕಾದರೆ, ಚಿತ್ತ ಶಾಂತವಾಗಿರಬೇಕು. ಚಿತ್ತ ಶಾಂತವಾಗಿದ್ದರೆ ಮಾತ್ರ ಜ್ಞಾನ ಸಾಧ್ಯ, ಭಕ್ತಿ ಸಾಧ್ಯ, ಕರ್ಮಯೋಗ ಸಾಧ್ಯ. ಅಂತೂ ಯೋಗದ ಸೌಧ ನಿಂತಿರುವುದೇ ಚಿತ್ತದ ಶಾಂತಿಯ ಮೇಲೆ. ಮುಂಚೆ ಎಲ್ಲರೂ ಫಲಾಪೇಕ್ಷೆ ಇಲ್ಲದ ಗರಡೀಮನೆಯಲ್ಲಿ ತರಬೇತು ತೆಗೆದುಕೊಳ್ಳಲೇಬೇಕು. ಶಾಲೆಯಲ್ಲಿ ಓದುವಾಗ ಕೆಲವು ಐಚ್ಛಿಕ ಪಾಠಗಳಿವೆ, ಯಾವುದನ್ನು ಬೇಕಾದರೂ ತೆಗೆದುಕೊಳ್ಳಬಹುದು. ಮತ್ತೆ ಕೆಲವು ಪಾಠಗಳು ಎಲ್ಲರೂ ತೆಗೆದುಕೊಳ್ಳಲೇಬೇಕಾದ \enginline{compulsory subjects. }ಅದರಂತೆಯೇ ಅನಾಸಕ್ತಿಯ ಕರ್ಮ, ಮತ್ತು ಮನಶ್ಶಾಂತಿ ಎಲ್ಲರಲ್ಲಿಯೂ ಕೆಳಗಿನ ಘಟ್ಟದಲ್ಲಿ ಇರಲೇಬೇಕಾಗಿದೆ. ಮುಂದು ಮುಂದುವರಿದಂತೆ ಒಬ್ಬ ಒಂದನ್ನು ಬಿಟ್ಟು ಮತ್ತೊಂದನ್ನು ತಗೆದುಕೊಳ್ಳಬೇಕು. ಮರ ಬುಡದಲ್ಲಿ ಒಂದೇ, ನಂತರ ಮೇಲೆ ಹೋದಂತೆ ಕವಲೊಡೆಯುತ್ತಾ ಬರುವುದು.

\begin{shloka}
ಯದಾ ಹಿ ನೇಂದ್ರಿಯಾರ್ಥೇಷು ನ ಕರ್ಮಸ್ವನುಷಜ್ಜತೇ~।\\ಸರ್ವಸಂಕಲ್ಪಸಂನ್ಯಾಸೀ ಯೋಗಾರೂಢಸ್ತದೋಚ್ಯತೇ \hfill॥ ೪~॥
\end{shloka}

\begin{artha}
ಯಾವಾಗ ಯೋಗಿಯು ಇಂದ್ರಿಯ ವಿಷಯಗಳಲ್ಲಿ ಮತ್ತು ಅವುಗಳ ಕರ್ಮಗಳಲ್ಲಿ ಆಸಕ್ತನಾಗಿರು\-ವುದಿಲ್ಲವೊ, ಸರ್ವಸಂಕಲ್ಪಗಳನ್ನೂ ತ್ಯಾಗಮಾಡಿರುವನೊ ಆಗ ಅವನನ್ನು ಯೋಗದಲ್ಲಿ ಆರೂಢ\-ನಾಗಿರುವನು ಎನ್ನುತ್ತಾರೆ.
\end{artha}

ಇಲ್ಲಿ ಯೋಗದಲ್ಲಿ ಯಾರು ನೆಲಸಿರುವರು ಎಂಬುದನ್ನು ವಿವರಿಸುತ್ತಾನೆ. ಆರೂಢ ಎಂದರೆ ಬಲವಾಗಿ ನೆಲಸಿರುವುದು. ಇನ್ನು ಅವನನ್ನು ಜಾರುವಂತೆ ಮಾಡಲಾಗುವುದಿಲ್ಲ. ಬಂಡೆ ಭದ್ರವಾಗಿ ಬೆಟ್ಟದಮೇಲೆ ಇದ್ದರೆ, ಎಷ್ಟೇ ಬಲವಾಗಿ ಗಾಳಿ ಬೀಸಿದರೂ ಅದನ್ನು ಚಲಿಸುವಂತೆ ಮಾಡಲಾರದು. ದೋಣಿ ನದಿಯಲ್ಲಿ ಹೋಗುತ್ತಿರುವಾಗ ಕೆಲವುವೇಳೆ ಅದನ್ನು ನೀರಿನ ಮಧ್ಯದಲ್ಲಿ ನಿಲ್ಲಿಸಬೇಕಾದರೆ, ಆ ದೋಣಿಯ ಮುಂದೆ ಹಿಂದೆ ಲಂಗರನ್ನು ನೀರಿನ ಒಳಗೆ ಹಾಕುವರು. ಅದು ಕೆಳಗೆ ಹೋಗಿ ತಳವನ್ನು ಭದ್ರವಾಗಿ ಕಚ್ಚುವುದು. ಆಗ ದೋಣಿಯನ್ನು ಒಂದು ಗೂಟ ಹೊಡೆದು ಕಟ್ಟಿಹಾಕಿದಂತೆ ಆಗುವುದು. ಎಷ್ಟೇ ಗಾಳಿ ಬರಲಿ ಆ ದೋಣಿ ಚಲಿಸುವುದಿಲ್ಲ. ನೀರು ಅಲ್ಲೋಲಕಲ್ಲೋಲವಾದರೂ ದೋಣಿಯನ್ನು ಮುಳುಗಿಸಲಾರದು. ಹಾಗೆಯೆ, ಯೋಗಿ ಇಂದ್ರಿಯದ ಪ್ರಪಂಚದಲ್ಲಿದ್ದರೂ, ಅವನು ಇಂದ್ರಿಯಕ್ಕೆ ಅತೀತವಾದ ಪರಮಾತ್ಮನ ಸತ್ಯವನ್ನು ಕಚ್ಚಿರುವನು, ಬಿಗಿಯಾಗಿ ಹಿಡಿದುಕೊಂಡಿರುವನು. ಇನ್ನು ಮೇಲೆ ಈ ಪ್ರಪಂಚ ಅವನನ್ನು ಕದಲಿಸಲಾರದು. ಇಂತಹ ಒಂದು ಸ್ಥಿತಿಯನ್ನು ಪಡೆಯಬೇಕಾದರೆ ಅವನೇನು ಮಾಡಬೇಕು ಎಂದು ಹೇಳುವನು.

ಆರೂಢನಾದ ಯೋಗಿ ಇನ್ನುಮೇಲೆ ಇಂದ್ರಿಯಗಳ ವಿಷಯಗಳಲ್ಲಿ ಮತ್ತು ಅದಕ್ಕೆ ಸಂಬಂಧ ಪಟ್ಟ ಕರ್ಮಗಳಲ್ಲಿ ಆಸಕ್ತನಾಗುವುದಿಲ್ಲ. ಅವನು ಅದನ್ನು ತೊರೆದಿರುವನು. ಎಲ್ಲಾ ಅನಾಹುತಕ್ಕೆ ಕಾರಣ ಇಂದ್ರಿಯ ವಸ್ತುಗಳ ಕಡೆ ಮನಸ್ಸು ವಾಲುವುದೇ ಎಂಬುದನ್ನು ಬಹಳ ಚೆನ್ನಾಗಿ ಅವನು ಅರಿತಿರುವನು. ಅದರ ಕಡೆ ಮನಸ್ಸನ್ನು ಹರಿಬಿಡುವುದಿಲ್ಲ. ಬಿಟ್ಟರೆ ತಾನೆ ಅದರಮೇಲೆ ಆಸಕ್ತಿ ಬರುವುದು. ಆಸಕ್ತಿಯಿಂದ ಅದನ್ನು ಹೊಂದಬೇಕು, ಅದನ್ನು ಅನುಭವಿಸಬೇಕು ಎಂಬ ಇಚ್ಛೆ ಬರುವುದು. ಒಮ್ಮೆ ಇಚ್ಛೆ ಬಂದರೆ ಅದು ಬರುಬರುತ್ತ ಬಲವಾಗುವುದು. ಅದೊಂದು ಸಣ್ಣ ಅಲೆಯಂತೆ ಪ್ರಾರಂಭವಾಗಿ ಮಹಾ ಪ್ರವಾಹವಾಗುವುದು. ಆ ಮಹಾ ಪ್ರವಾಹವನ್ನು ಯಾರೂ ತಡೆಗಟ್ಟಲಾರರು. ಅದು ತನ್ನ ಎದುರಿಗೆ ಇರುವುದನ್ನೆಲ್ಲ ನೂಕಿಕೊಂಡು ಹೋಗುವುದು. ಇದೊಂದು ಅನಾಹುತದಲ್ಲಿ ಕೊನೆಗೊಳ್ಳುವುದು. ಅದಕ್ಕೇ ಆದಿಯಲ್ಲೆ ಅದನ್ನು ತಡೆಗಟ್ಟುವನು.

ಇಂದ್ರಿಯಕ್ರಿಯೆಗಳಲ್ಲಿಯೂ ಅವನು ಆಸಕ್ತನಲ್ಲ. ಇಂದ್ರಿಯ ವಿಷಯಗಳಲ್ಲಿ ಆಸಕ್ತನಾದರೆ ಮಾತ್ರ, ಇಂದ್ರಿಯ ಕ್ರಿಯೆಗಳು ಪ್ರಾರಂಭವಾಗುವುದು. ಯಾವಾಗ ಅವನು ಆದಿಯನ್ನೇ ನಿಗ್ರಹಿ ಸಿರುವನೊ ಅನಂತರ ಅದರ ರೆಂಬೆ ಕೊಂಬೆಗಳಾದ ಕರ್ಮಗಳು ಅವನಲ್ಲಿ ಹುಟ್ಟುವುದೇ ಇಲ್ಲ. ಅವನು ಮನಸ್ಸಿನಲ್ಲಿ ಆಸೆಯ ಸಂಕಲ್ಪವನ್ನು ನಾಶಮಾಡಿರುವನು. ಸಂಕಲ್ಪವೇ ಬೀಜ. ಅದರಿಂದ ಆಲೋಚನೆ, ಅದರಿಂದ ಕರ್ಮ. ಯಾವಾಗ ಬೀಜವನ್ನೇ ಹುರಿದುಹಾಕಿದೆಯೊ ಅನಂತರ ಅದರಿಂದ ಯಾವ ರೆಂಬೆ ಕೊಂಬೆಗಳೂ ಹುಟ್ಟಲಾರವು. ಅವಿಚ್ಛಿನ್ನವಾಗಿ ಯೋಗದಲ್ಲಿ ನೆಲಸಿರುವುದು ಎಂದರೆ ಇದೇ.

\begin{shloka}
ಉದ್ಧರೇದಾತ್ಮನಾತ್ಮಾನಂ ನಾತ್ಮಾನಮವಸಾದಯೇತ್~।\\ಆತ್ಮೈವ ಹ್ಯಾತ್ಮನೋ ಬಂಧುರಾತ್ಮೈವ ರಿಪುರಾತ್ಮನಃ \hfill॥ ೫~॥
\end{shloka}

\begin{artha}
ತನ್ನನ್ನು ತನ್ನಿಂದಲೇ ಉದ್ಧಾರಮಾಡಿಕೊಳ್ಳಬೇಕು. ತನ್ನನ್ನು ಕುಗ್ಗಿಸಿಕೊಳ್ಳಬಾರದು. ಏಕೆಂದರೆ ತಾನೇ ತನ್ನ ಬಂಧು, ತಾನೇ ತನ್ನ ಶತ್ರು.
\end{artha}

ಶ‍್ರೀಕೃಷ್ಣ ಗೀತೆಯಲ್ಲಿ ಬರುವ ದೊಡ್ಡದೊಂದು ಸಂದೇಶವನ್ನು ಇಲ್ಲಿ ಕೊಡುತ್ತಾನೆ. ಅದೇ ಮನುಷ್ಯ ತನ್ನನ್ನು ತನ್ನಿಂದಲೇ ಉದ್ಧಾರಮಾಡಿಕೊಳ್ಳಬೇಕು ಎಂಬುದು. ಮನುಷ್ಯ ಇದನ್ನು ಮುಂಚೆ ತಿಳಿಯುವುದಿಲ್ಲ. ದೂರದಲ್ಲಿರುವ ದೇವರಿಗೆ ಬಾಗುವನು, ಗುರುವಿಗೆ ಬಾಗುವನು, ಶಾಸ್ತ್ರಾದಿಗಳಿಗೆ ಬಾಗುವನು. ಇವುಗಳೆಲ್ಲ ನಮ್ಮನ್ನು ಉದ್ಧಾರಮಾಡುವುವು ಎಂದು ಭಾವಿಸುತ್ತೇವೆ. ನಿಜ, ಸ್ವಲ್ಪ ಮಟ್ಟಿಗೆ ಇವುಗಳೆಲ್ಲ ಉಪಕಾರವನ್ನು ಮಾಡುತ್ತವೆ. ಗುರು ಹೇಳುವ ಸಲಹೆ, ಶಾಸ್ತ್ರದಲ್ಲಿ ಬರುವ ಬುದ್ಧಿವಾದ ಇವುಗಳೆಲ್ಲ, ನಾವೆ ಅವನ್ನು ತೆಗೆದುಕೊಂಡು ಜೀರ್ಣಿಸಿಕೊಂಡಲ್ಲದೇ, ಅವೇ ತಮಗೆ ತಾವೇ ಆ ಕೆಲಸವನ್ನು ಮಾಡಲಾರವು. ವೈದ್ಯ ಕೊಡುವ ಔಷಧಿಯನ್ನು ರೋಗಿ ತಾನೆ ಬಾಯಿ ಮೂಲಕ ತೆಗೆದುಕೊಳ್ಳಬೇಕು. ಅನಂತರ ಅವನ ದೇಹವೇ ಅದನ್ನು ಅರಗಿಸಿಕೊಳ್ಳಬೇಕು. ಅವನ ದೇಹವೇ ರೋಗದೊಂದಿಗೆ ಹೋರಾಡಬೇಕು. ನಾನು ರೋಗದಿಂದ ನರಳುತ್ತಿದ್ದರೆ, ನನಗಾಗಿ ನೀವು ಔಷಧಿ ತೆಗೆದುಕೊಳ್ಳಿ ಎಂದು ವೈದ್ಯರಿಗೆ ಹೇಳಲಾಗುವುದಿಲ್ಲ.

ನಿಜವಾದ ಶಾಸ್ತ್ರ ಮತ್ತು ಗುರುಗಳು ತಾವೇ ನಾವು ಮಾಡಬೇಕಾದ ಕೆಲಸವನ್ನು ಮಾಡದೆ, ನಮ್ಮ ಕೈಯಿಂದ ಅದನ್ನು ಮಾಡಿಸಲು ಪ್ರಚೋದಿಸುತ್ತಾರೆ. ನಮ್ಮಲ್ಲಿರುವ ಸುಪ್ತವಾದ ಶಕ್ತಿಯ ಕಡೆ ಕೈ ತೋರಿಸಿ ಅದನ್ನು ಜಾಗ್ರತಗೊಳಿಸುವಂತೆ ಮಾಡುವರು. ಬುದ್ಧ ತನ್ನ ಕೊನೆಗಾಲದಲ್ಲಿ ಮೃತ್ಯು ಶಯ್ಯೆಯಮೇಲೆ ಮಲಗಿದ್ದಾಗ ತನ್ನ ಶಿಷ್ಯರಿಗೆ ನಿಮಗೆ ಏನಾದರೂ ಅನುಮಾನವಿದ್ದರೆ ಕೇಳಿ, ಆಮೇಲೆ ಪಶ್ಚಾತ್ತಾಪ ಪಡಬೇಡಿ ಎನ್ನುತ್ತಾನೆ. ದುಃಖಾಕ್ರಾಂತನಾಗಿ ಅವನ ಪರಮಶಿಷ್ಯ ಆನಂದ, ನೀವು ಹೋದಮೇಲೆ ನಮ್ಮನ್ನು ಯಾರು ನಡೆಸುತ್ತಾರೆ ಎಂದು ಕೇಳುತ್ತಾನೆ. ಆಗ ಬುದ್ಧ ನಿಮಗೆ ನೀವೇ ಬೆಳಕಾಗಬೇಕು ಎನ್ನುತ್ತಾನೆ. ನಮ್ಮನ್ನು ಉದ್ಧಾರಮಾಡಿಕೊಳ್ಳುವ ಶಕ್ತಿ ನಮ್ಮಿಂದಲೇ ಬರಬೇಕು. ಅದೇನೂ ಆಕಾಶದಿಂದ ಉದುರುವುದಿಲ್ಲ. ಇತರರು ಇದನ್ನು ತಮ್ಮ ಜೇಬಿನಿಂದ ಕೊಡಲು ಆಗುವುದಿಲ್ಲ. ಶ‍್ರೀರಾಮಕೃಷ್ಣರು ಈ ಒಂದು ಉದಾಹರಣೆ ಕೊಡುತ್ತಾರೆ. ಗುರುವೊಬ್ಬನಿಗೆ, ರಾಜನಾದ ಶಿಷ್ಯ ಇದ್ದ. ಪ್ರತಿದಿನ ರಾಜ ತನ್ನ ಗುರುಗಳಿಗೆ, ದಯವಿಟ್ಟು ನನ್ನನ್ನು ಮಾಯೆಯಿಂದ ಪಾರುಮಾಡಿ ಎಂದು ಕೇಳಿಕೊಳ್ಳುತ್ತಿದ್ದ. ಆದರೆ ಆತನಲ್ಲಿ ಇನ್ನೂ ಆಧ್ಯಾತ್ಮಿಕ ಅಭೀಪ್ಸೆ ಬಲವಾಗಿರಲಿಲ್ಲ. ಆದಕಾರಣ ಅವನು ಆಡುವ ಮಾತನ್ನು ಕಿವಿಗೆ ಹಾಕಿಕೊಳ್ಳುತ್ತಿರಲಿಲ್ಲ. ಒಂದು ದಿನ ಅವನ ಕಾಟ ಬಲವಾಯಿತು. ಅವನಿಗೆ ಬುದ್ಧಿ ಕಲಿಸಬೇಕೆಂದು ಅರಮನೆಯಲ್ಲಿರುವ ಒಂದು ಕಂಬವನ್ನು ಗುರು ಬಲವಾಗಿ ಅಪ್ಪಿಕೊಂಡು ರಾಜನನ್ನು ಬಿಡಿಸು ಎನ್ನುತ್ತಾನೆ. ಆಗ ರಾಜ, ನೀವೇ ಕಂಬವನ್ನು ಹಿಡಿದುಕೊಂಡವರು, ಅದನ್ನು ನೀವೆ ಬಿಟ್ಟರೆ ಆಯಿತಲ್ಲ, ನಾನೇಕೆ ಅದನ್ನು ಬಿಡಿಸಬೇಕು ಎಂದು ಹೇಳುತ್ತಾನೆ. ತಕ್ಷಣವೇ ಗುರುಗಳು ಹಾಗೆಯೇ ಮಾಯೆಯನ್ನು ಅಪ್ಪಿರುವವನು ನೀನು, ನೀನೇ ಬಿಟ್ಟರೆ ಆಯಿತಲ್ಲ, ನಾನೇಕೆ ಅದನ್ನು ಬಿಡಿಸಬೇಕು ಎಂದರು.

ನಾವೇ ಸಂಸಾರದ ಗೋಜಿಗೆ ಸಿಕ್ಕಿಕೊಂಡಿರುವೆವು. ಆ ಗೋಜಿನಿಂದ ಬಿಡಿಸಿಕೊಳ್ಳುವುದೂ ನಮ್ಮ ಕೈಯಲ್ಲಿದೆ. ಅನೇಕವೇಳೆ ಇದು ನಮ್ಮ ಕೈಯಲ್ಲಿದೆ ಎಂಬುದು ಗೊತ್ತಾಗುವುದಿಲ್ಲ. ನಾವು ದುರ್ಬಲರು ಅಜ್ಞಾನಿಗಳು ಪಾಪಿಗಳು; ನಮಗೆ ಅಂತಹ ಶಕ್ತಿ ಎಲ್ಲಿದೆ ಎಂದು ಹೇಳಿಕೊಳ್ಳುತ್ತೇವೆ. ಆದರೆ ಇದೆಲ್ಲ ತೋರಿಕೆಗೆ ಹಾಗೆ ಕಾಣುವುದು. ಪ್ರತಿಯೊಬ್ಬ ಜೀವಿಯಲ್ಲಿಯೂ, ದೇವರು ಆ ಶಕ್ತಿಯನ್ನು, ಜ್ಞಾನವನ್ನು ಪವಿತ್ರತೆಯನ್ನು ಇಟ್ಟಿರುವನು. ಅದನ್ನು ನಾವು ಮರೆತಿರುವೆವು. ಅದೆಲ್ಲಿ ದೆಯೋ ನಮಗೆ ಗೊತ್ತಿಲ್ಲ. ಅದಕ್ಕಾಗಿ ಆಕಾಶದ ಕಡೆ ತಿರುಗುತ್ತೇವೆ. ಇನ್ನೊಬ್ಬರ ಹತ್ತಿರ ತಿರುಪೆ ಬೇಡಲು ಹೋಗುತ್ತೇವೆ. ಶ‍್ರೀರಾಮಕೃಷ್ಣರು ಇನ್ನೊಂದು ಸುಂದರವಾದ ಉಪಮಾನವನ್ನು ಕೊಡುವರು. ಒಬ್ಬನಿಗೆ ಬೀಡಿ ಸೇದಲು ರಾತ್ರಿ ಮನಸ್ಸಾಯಿತು. ನೋಡಿದ. ದೀಪದಕಡ್ಡಿ ಇರಲಿಲ್ಲ. ಉರಿಯುತ್ತಿದ್ದ ಲಾಟೀನ್ ತೆಗೆದುಕೊಂಡು ಪಕ್ಕದ ಮನೆಗೆ ಹೋಗಿ ಅವನನ್ನು ಕೂಗಿ ಎಬ್ಬಿಸಿದ. ಏತಕ್ಕೆ ಇಷ್ಟು ಹೊತ್ತಿನಲ್ಲಿ ಬಂದೆ ಎಂದು ಕೇಳಿದಾಗ, ಬೀಡಿ ಸೇದಲು ಕಡ್ಡಿ ಕೇಳುವುದಕ್ಕೆ ಬಂದೆ ಎನ್ನುತ್ತಾನೆ. ಆ ನೆರೆಯವನು ಹೇಳುತ್ತಾನೆ, ಎಂತಹ ಮೂಢ ನೀನು, ಉರಿಯುತ್ತಿರುವ ದೀಪ ನಿನ್ನ ಕೈಯಲ್ಲಿಯೇ ಇದೆ. ಅದನ್ನು ಮರೆತು ಕಡ್ಡಿಯನ್ನು ಕೇಳಲು ಬಂದೆಯಾ? ಎನ್ನುತ್ತಾನೆ. ಹಾಗೆಯೆ ನಮ್ಮಲ್ಲಿ ಭಗವಂತನಿಟ್ಟಿರುವ ನಿಧಿಯನ್ನು ಮರೆತು ಭಿಕಾರಿಗಳಂತೆ ಇತರರಿಂದ ಬೇಡಿ ಜೀವನ ಯಾಪನೆ ಮಾಡುತ್ತಿರುವೆವು. ನಮ್ಮಲ್ಲೆ ಎರಡು ವ್ಯಕ್ತಿತ್ವ ಇದೆ. ಒಂದು ತತ್ಕಾಲಿಕವಾಗಿ ನಮ್ಮ ನೈಜಸ್ಥಿತಿಯನ್ನು ಮರೆಸಿರುವುದು. ಮತ್ತೊಂದು ಅದರ ಹಿಂದೆ ಇದೆ. ಅದೇ ಎಲ್ಲಾ ಶಕ್ತಿಯಿಂದ ಕೂಡಿಕೊಂಡು ನಮ್ಮ ಸಹಾಯಕ್ಕೆ ಅನುಗಾಲವೂ ಬರಲು ಸಿದ್ಧವಾಗಿದೆ. ಆದರೆ ಅದರ ಕಡೆ ನಮ್ಮ ಗಮನವೆ ಹೋಗುವುದಿಲ್ಲ. ಮೇಲೆದ್ದು ಕಾಣುವ ನಮ್ಮ ಕ್ಷುದ್ರ ವ್ಯಕ್ತಿತ್ವ ಒಂದೇ ನಮಗೆ ಗೋಚರವಾಗುವುದು. ನನ್ನಂತಹ ಅಯೋಗ್ಯನಲ್ಲಿ ಅಂತಹ ಶಕ್ತಿಯಾದರೊ ಇರುವುದು ಹೇಗೆ ಸಾಧ್ಯ ಎಂದು ಭಾವಿಸುತ್ತೇವೆ. ನಾವೊಂದು ಕತ್ತಲೆ ಮನೆಯಲ್ಲಿರುವೆವು. ಅಯ್ಯೊ ಕತ್ತಲೆ ಕತ್ತಲೆ ಎಂದು ಅರಚುತ್ತಿರುವೆವು. ದೀಪದ ಪೆಟ್ಟಿಗೆ ನಮ್ಮ ಜೇಬಿನಲ್ಲಿದೆ. ಅದನ್ನು ಗೀಚಿದೊಡನೆ ಕತ್ತಲೆ ಎಷ್ಟು ಕಾಲದಿಂದ ಇದ್ದರೇನು, ತಕ್ಷಣವೇ ಮಾಯವಾಗುವುದು. ಹಾಗೆಯೆ ಭಗವಂತ ನಮ್ಮಲ್ಲಿಟ್ಟಿರುವ ಶಕ್ತಿ ಆಪದ್ಧನದಂತೆ ನನ್ನ ಹೆಸರಿನಲ್ಲಿದೆ. ಅದನ್ನು ಮರೆತು ಭಿಕ್ಷೆ ಬೇಡುವುದಕ್ಕೆ ಹೋಗುತ್ತಿರುವೆವು.

ಶ‍್ರೀಕೃಷ್ಣ ಇಲ್ಲಿ ಇನ್ನೊಂದು ಅಮೃತ ಸಂದೇಶವನ್ನು ಕೊಡುತ್ತಾನೆ. ಅದೇ ತನ್ನನ್ನು ತಾನು ಕುಗ್ಗಿಸಿಕೊಳ್ಳಬಾರದು ಎಂಬುದು. ಜೀವನದಲ್ಲಿ ನನ್ನನ್ನು ಇತರರು ಎಷ್ಟು ಬೈದರೂ ಹಿಂಸಿಸಿದರೂ ಅದರಿಂದ ಅಷ್ಟೊಂದು ಪರಿಣಾಮವಾಗುವುದಿಲ್ಲ. ಅನೇಕ ವೇಳೆ ಅವರೇನಾದರೂ ಬೈದರೆ,\break ನಾವು ಹಾಗಲ್ಲ ಎಂದು ನನ್ನನ್ನು ನಾನೇ ನಿಂದಿಸಿಕೊಳ್ಳುತ್ತೇನೆಯೋ, ನನಗೆ ನಾನೇ ಶಾಪ\break ಹಾಕಿಕೊಳ್ಳುತ್ತೇ\-ನೆಯೊ, ಆಗ ನನ್ನ ಪುರೋಗಮನ ತಾತ್ಕಾಲಿಕವಾಗಿ ನಿಂತಂತೆ. ನಮ್ಮಲ್ಲಿ ಆತ್ಮಶ್ರದ್ಧೆ ಇರಬೇಕು. ನಾವು ಅಯೋಗ್ಯರಲ್ಲ, ಅಜ್ಞಾನಿಗಳಲ್ಲ, ಪಾಪಿಗಳಲ್ಲ, ಇವೆಲ್ಲ ನಮ್ಮ ಮೇಲೆ ಮುತ್ತಿರುವ ಧೂಳು, ತಾತ್ಕಾಲಿಕ ಅವಸ್ಥೆ. ಕೂಡಲೆ ಇದರಿಂದ ಪಾರಾಗಬೇಕು. ಇದರ ಹಿಂದೆ ಇರುವ ನನ್ನ ನೈಜವಾದ ಪವಿತ್ರ ಸ್ಥಿತಿಯನ್ನು ಉದ್ದೀಪನಗೊಳಿಸಬೇಕು. ಆತ್ಮಶ್ರದ್ಧೆ ಇರಬೇಕು, ಆತ್ಮನ ಅವಹೇಳನವನ್ನು ಮಾಡಿಕೊಳ್ಳಬಾರದು. ಈ ಶಕ್ತಿ ಸಂಜೀವಿನಿಯನ್ನು ಶ‍್ರೀಕೃಷ್ಣ ಅರ್ಜುನನನ್ನು ನಿಮಿತ್ತ ಮಾಡಿಕೊಂಡು ಇಡೀ ಮಾನವಕೋಟಿಗೆ ಸಾರುತ್ತಾನೆ. ಸ್ವಾಮಿ ವಿವೇಕಾನಂದರು ಆತ್ಮಶ್ರದ್ಧೆಯ ವಿಷಯವಾಗಿ ಹೀಗೆ ಹೇಳುತ್ತಾರೆ: “ಯಾರಿಗೆ ದೇವರಲ್ಲಿ ನಂಬಿಕೆ ಇಲ್ಲವೋ ಅವನನ್ನು ನಾಸ್ತಿಕ ಎಂದು ಹಿಂದಿನ ಧರ್ಮಗಳು ಸಾರುತ್ತಿದ್ದವು. ಆದರೆ ಯಾರಿಗೆ ಆತ್ಮಶ್ರದ್ಧೆ ಇಲ್ಲವೋ ಅವನನ್ನು ನಾಸ್ತಿಕ ಎನ್ನುತ್ತೇನೆ ನಾನು”. ಈಶ್ವರ ನಿಂದೆಯನ್ನು ಮಹಾ ಪಾಪ ಎನ್ನುತ್ತಾರೆ, ಆತ್ಮ ನಿಂದೆಗಿಂತ ಮಹಾಪಾಪವಿನ್ನಿಲ್ಲ ಎಂದು ನಾನು ಹೇಳುತ್ತೇನೆ ಎನ್ನುತ್ತಾರೆ. ಇದೇನು ಹೊಸದಾಗಿ ಒಂದು ಸಂದೇಶವನ್ನು ಅವರು ಹೇಳುತ್ತಿಲ್ಲ. ಈ ಸಂದೇಶಕ್ಕೆ ಮೂಲ ನಮಗೆ ಗೀತೆಯಲ್ಲಿ ದೊರಕುವುದು.

ಮನಸ್ಸೆ ನಮ್ಮ ಶತ್ರು ಮತ್ತು ನಮ್ಮ ಮಿತ್ರ. ನಮ್ಮನ್ನು ನಾಶ ಮಾಡುವ ಶಕ್ತಿ ನಮ್ಮಲ್ಲೇ ಇದೆ. ನಮ್ಮನ್ನು ಉದ್ಧಾರ ಮಾಡುವ ಶಕ್ತಿಯೂ ನಮ್ಮಲ್ಲಿದೆ. ಇಲ್ಲಿ ಪಾಷಾಣವೂ ಇದೆ. ಅಮೃತವೂ ಇದೆ. ಒಬ್ಬ ಪಾಷಾಣವನ್ನು ನೋಡಿ, ಎಂದರೆ ಮನಸ್ಸಿನಲ್ಲಿರುವ ತನ್ನ ಹೀನ ಅಭ್ಯಾಸಗಳು ಸಂಸ್ಕಾರಗಳು,ಇವುಗಳ ಕಡೆ ನೋಡಿ, ಇದೆಲ್ಲ ನನ್ನ ಮನಸ್ಸಿನಲ್ಲಿದೆ; ಸುಮ್ಮನೆ ನಾನು ಒಳ್ಳೆಯವನು ಎಂದುಕೊಂಡರೆ ಪ್ರಯೋಜನವೇನು ಎಂದುಕೊಳ್ಳುತ್ತಾನೆ. ಆದರೆ ಅದೇ ಮನಸ್ಸಿನ ಉಗ್ರಾಣದಲ್ಲಿ ಸುಪ್ತವಾಗಿರುವ ಎಷ್ಟೋ ಪವಿತ್ರತೆಯ ಶಕ್ತಿಗಳು ಹುದುಗಿವೆ. ಅವನ್ನು ನೋಡಬೇಕು, ಅವನ್ನು ನಮ್ಮ ಮೇಲೆ ಪ್ರಯೋಗಿಸಿಕೊಳ್ಳುವುದನ್ನು ಕಲಿತುಕೊಳ್ಳಬೇಕು ಆಗ ನಮ್ಮ ಮನಸ್ಸೆ ನಮ್ಮನ್ನು ಉದ್ಧಾರ ಮಾಡುವುದು. ಎರಡೂ ಇದೆ; ಯಾವುದು ಬೇಕೊ ಅದನ್ನು ನಾವು ತೆಗೆದುಕೊಳ್ಳಬಹುದು. ಒಬ್ಬ ಒಂದು ತಪ್ಪನ್ನು ಮಾಡಿದ, ಅದಕ್ಕೆ ಶಿಕ್ಷೆಯಂತೆ ಊರಿನಿಂದ ಹೊರಗಡೆ ಇರುವ ಒಂದು ಸೆರೆಮನೆಯಲ್ಲಿಟ್ಟರು. ಅದರ ಸುತ್ತಲೂ ಕಿಟಕಿಗಳು ಇದ್ಧವು. ಊರಿನ ಗಲೀಜನ್ನು ತಂದು ಒಂದು ಕಡೆ ಹಾಕುತ್ತಿದ್ದರು. ಇನ್ನು ಒಂದು ಕಡೆ ಅಸೀಮ ಅನಂತವಾದ ಸಾಗರ ತನ್ನ ಗಾಂಭೀರ್ಯದಿಂದ ಮೆರೆಯುತ್ತಿತ್ತು. ಸೆರೆಯಾಳು ಪ್ರತಿ ದಿನವೂ ಊರ ಕಸ ಬಂದು ಬೀಳುವ ಕಡೆ ನೋಡಿ, ಅಯ್ಯೊ ಬರೀ ಕಸ ಎಂದು ಕೊರಗುತ್ತಿದ್ದ. ಆದರೆ ಆತ ತನ್ನ ದೃಷ್ಟಿಯನ್ನು ಸ್ವಲ್ಪ ಬದಲಾಯಿಸಿದರೆ ಸಾಕು. ಅಲ್ಲಿ ಸುಂದರ ಮನೋಹರ ದೃಶ್ಯವಿರುವುದು ಕಾಣುತ್ತಿತ್ತು. ಎರಡೂ ಇದೆ. ನಾವು ಒಂದನ್ನು ನೋಡುತ್ತೇವೆ, ಇನ್ನೊಂದನ್ನು ಮರೆಯುತ್ತೇವೆ. ಶ‍್ರೀಕೃಷ್ಣ ಇಲ್ಲಿ ಹೇಳುವುದು, ನಾವು ಮರೆತಿರುವ ನಮ್ಮಲ್ಲಿ ಸುಪ್ತವಾಗಿರುವ ಶಕ್ತಿಯನ್ನು. ನಮ್ಮನ್ನು ಉದ್ಧಾರಮಾಡಲು ಶಕ್ತಿಗಳು ಕಾದುಕೊಂಡಿವೆ. ಅದರೊಂದಿಗೆ ಸ್ನೇಹ ಬೆಳಸಿಕೊಳ್ಳಬೇಕು. ನಮ್ಮನ್ನು ಹಾಳುಮಾಡುವುದಕ್ಕೆ ಇರುವ ಶಕ್ತಿಯೊಂದಿಗೆ ಇರುವ ಸ್ನೇಹವನ್ನು ನಾವು ಬಿಡಬೇಕು. ಒಂದು ಮನೆ ಹಿಂದುಗಡೆ ಗಲೀಜು ಇದೆ. ಅದೇ ಮನೆಯಲ್ಲಿ ಮುಂದುಗಡೆ ಪರಿಮಳ ಬೀರುತ್ತಿರುವ ಸುಂದರ ಹೂವಿನ\break ಗಿಡಗಳಿವೆ. ಏತಕ್ಕೆ ನಾವು ಬಂದು ಅದರ ಪರಿಮಳವನ್ನು ಆಘ್ರಾಣಿಸಬಾರದು?

\begin{shloka}
ಬಂಧುರಾತ್ಮಾತ್ಮನಸ್ತಸ್ಯ ಯೇನಾತ್ಮೈವಾತ್ಮನಾ ಜಿತಃ~।\\ಅನಾತ್ಮನಸ್ತು ಶತ್ರುತ್ವೇ ವರ್ತೇತಾತ್ಮೈವ ಶತ್ರುವತ್ \hfill॥ ೬~॥
\end{shloka}

\begin{artha}
ಯಾರು ತನ್ನ ಮನಸ್ಸನ್ನು ನಿಗ್ರಹಿಸುವನೊ ಅವನಿಗೆ ಅದೇ ಬಂಧುವಾಗಿರುವುದು. ಆದರೆ ಯಾರು ತನ್ನ ಮನಸ್ಸನ್ನು ಗೆದ್ದಿಲ್ಲವೊ ಅವನಿಗೆ ಅದೇ ಶತ್ರುವಿನಂತೆ ವರ್ತಿಸುವುದು.
\end{artha}

ನಿಗ್ರಹಿಸಿದ ಮನಸ್ಸಿನಷ್ಟು ದೊಡ್ಡ ಬಂಧು ನಮಗೆ ಬೇರೆ ಇಲ್ಲ. ಇದು ಯಾವಾಗಲೂ ನಮ್ಮೊಂದಿಗೆ ಇರುವುದು. ನಾವು ಮನಸ್ಸನ್ನು ನಿಗ್ರಹಿಸಿ ಯಾವ ಯಾವ ಸತ್ ಸಂಸ್ಕಾರಗಳನ್ನು ಮನಸ್ಸಿನಲ್ಲಿ ಸಂಗ್ರಹಿಸಿಕೊಂಡಿರುವೆವೊ ಅವು ಯಾವಾಗಲೂ ನಮ್ಮ ಕೈಬಿಡುವುದಿಲ್ಲ. ಕೆಲವು ವೇಳೆ ಮನಸ್ಸು ಪ್ರಲೋಭನಗೆ ಸಿಕ್ಕಿ ದುರ್ಬಲವಾಗಿ ಇನ್ನೇನು ಆ ಕೆಲಸವನ್ನು ಮಾಡುವುದಕ್ಕೆ ಹೋಗಬೇಕು ಅನ್ನುವಷ್ಟರಲ್ಲಿ, ಆ ಸತ್ ಸಂಸ್ಕಾರ ನನ್ನನ್ನು ಹೇಗೊ ಕಾಪಾಡುವುದು.

ಹಾಗೆಯ ನನ್ನ ಬದ್ಧವೈರಿ ಹೊರಗಿರುವವನಲ್ಲ. ನಾವು ಅವನಿಂದ ಹೇಗಾದರೂ ಪಾರಾಗಬಹುದು. ಆದರೆ ದುರ್ಬಲವಾದ ನಮ್ಮ ಮನಸ್ಸಿನಷ್ಟು ಪ್ರಬಲವಾದ ವೈರಿ ಮತ್ತೊಂದು ಇಲ್ಲ. ಗೆಲ್ಲದ ನಮ್ಮ ಮನಸ್ಸಿನಲ್ಲಿ ಆಸೆ ಆಕಾಂಕ್ಷೆಗಳು ಕುದಿಯುತ್ತಿರುವುವು. ನಮ್ಮ ಮನಸ್ಸಿನ ದೌರ್ಬಲ್ಯ ವನ್ನೇ ಕಾಯುತ್ತಿರುವುವು. ಮನಸ್ಸು ಕೆಳಮಟ್ಟಕ್ಕೆ ಬಂದ ಗಳಿಗೆಯಲ್ಲಿ ಹೀನ ಸಂಸ್ಕಾರಗಳು ಮೇಲೆದ್ದು ಕೈಲಿ ಮಾಡಬಾರದ್ದನ್ನು ಮಾಡಿಸುವುವು. ಈ ದೌರ್ಬಲ್ಯವನ್ನು ನಾವೇ ಒಳಗೆ ಸಾಕಿ ಸಲಹಿದ್ದು. ಮನಸ್ಸಿನಲ್ಲಿರುವ ಹೀನ ಸಂಸ್ಕಾರಗಳನ್ನು ಅಡಗಿಸದೇ ಇದ್ದರೆ, ಅವು ದುರ್ಬಲವಾಗುವುದಿಲ್ಲ. ಕ್ರಮೇಣ ಬಲವಾಗುತ್ತ ಬರುವುವು. ಸಮಯವನ್ನು ಹೊಂಚುಹಾಕಿಕೊಂಡಿದ್ದು ನಮ್ಮ ಮೇಲೆ ಬಿದ್ದು ನಮ್ಮ ಕೈಯಿಂದ ಆ ಕೆಲಸವನ್ನು ಮಾಡಿಸಿ ಅವು ತಮ್ಮ ಬಯಕೆಯನ್ನು ಈಡೇರಿಸಿಕೊಳ್ಳುವುವು. ಒಮ್ಮೆ ಅದನ್ನು ಮಾಡಿತು ಎಂದರೆ ಪುನಃ ಪುನಃ ಅದನ್ನು ಮಾಡಲು ಪ್ರಾರಂಭಿಸುವೆವು. ಇದಕ್ಕೆಲ್ಲ ಆದಿಕಾರಣ ನಮ್ಮ ಮನಸ್ಸಿನಲ್ಲಿರುವ ದೌರ್ಬಲ್ಯವನ್ನು ನಾಶಮಾಡದೆ ಬಿಟ್ಟಿದ್ದು.

\begin{shloka}
ಜಿತಾತ್ಮನಃ ಪ್ರಶಾಂತಸ್ಯ ಪರಮಾತ್ಮಾ ಸಮಾಹಿತಃ~।\\ಶೀತೋಷ್ಣಸುಖದುಃಖೇಷು ತಥಾ ಮಾನಾಪಮಾನಯೋಃ \hfill॥ ೭~॥
\end{shloka}

\begin{artha}
ಶೀತೋಷ್ಣಗಳಲ್ಲಿ, ಸುಖದುಃಖಗಳಲ್ಲಿ ಮಾನಾಪಮಾನಗಳಲ್ಲಿ ಸಮನಾಗಿರುವವನಿಗೆ, ಜಯಿಸಲ್ಪಟ್ಟ ಮನಸ್ಸು ಳ್ಳವನೂ ಪ್ರಶಾಂತನೂ ಆದವನಿಗೆ ಆತ್ಮ ವಿಶೇಷವಾಗಿ ಸಮಾಹಿತವಾಗಿರುವುದು.
\end{artha}

ಯಾವ ಬಾಧಕವೂ ಇಲ್ಲದೆ ಮನಸ್ಸು ಪರಮಾತ್ಮನ ಕಡೆ ಹರಿಯಬೇಕಾದರೆ ಇಲ್ಲಿ ಹೇಳಿರುವ ಕೆಲವು ಗುಣಗಳು ನಮ್ಮಲ್ಲಿರಬೇಕು. ಶೀತೋಷ್ಣದಲ್ಲಿ ಅವನು ಒಂದೇ ಸಮನಾಗಿರುತ್ತಾನೆ. ಹಾಗಾದರೆ ಬಿಸಿಲು ಚಳಿ ಅವನಿಗೆ ತಾಕುವುದಿಲ್ಲ ಎಂದಲ್ಲ. ಇವುಗಳೆಲ್ಲ ಅವನಿಗೆ ಗೊತ್ತಾಗುವುದು. ಆದರೆ ಅವನು ಗೊಣಗಾಡದೆ ಇವನ್ನು ಸಹಿಸುವನು. ಇವೆಲ್ಲ ನಮಗೆ ಜೀವನದಲ್ಲಿ ಅನಿವಾರ್ಯ. ಬೇಸಿಗೆಯ ಕಾಲದಲ್ಲಿ ಸೆಖೆಯಾಗುವುದು, ಛಳಿಗಾಲದಲ್ಲಿ ಛಳಿಯಾಗುವುದು,\break ಇವುಗಳೆಲ್ಲ ಪುತುಧರ್ಮಗಳು. ತುಂಬಾ ಸೆಖೆ, ತುಂಬಾ ಛಳಿ ಎಂದು ಹೇಳುತ್ತಿದ್ದರೆ ಫಲವೇನು. ನಾವೊಂದು ದೇಹ. ಅದು ಹೊರಗಿನದಕ್ಕೆ ಹೊಂದಿಕೊಂಡು ಹೋಗಬೇಕು. ಸುಮ್ಮನೆ ಗೊಣಗಾಡು\-ತ್ತಿದ್ದರೆ ಹೊರಗಿನದು ನಮ್ಮ ಇಚ್ಛೆಯಂತೆ ಹೊಂದಿಕೊಳ್ಳುವುದಿಲ್ಲ.

ಅದರಂತೆಯೆ ಸುಖದುಃಖ. ನಮ್ಮ ಮನಸ್ಸಿನ ಬಹುಪಾಲು ಶಕ್ತಿ ಇವುಗಳಕಡೆ ವ್ಯಯವಾಗಿ ಹೋಗುತ್ತದೆ. ಬರದ ಸುಖವನ್ನು ಚಪ್ಪರಿಸುತ್ತ ಇರುತ್ತೇವೆ, ಎಂದು ಬರುವುದೊ, ಎಂದು ಅನು ಭವಿಸಿಯೇನೊ ಎಂದು. ಇದೊಂದು ತಿರುಕನ ಕನಸಿನಂತೆ. ಆದರೆ ಮನುಷ್ಯ ಇದನ್ನು ಮಾಡದೆ ಇರಲಾರ. ಕುಳಿತು ಬೇಜಾರು. ಅದಕ್ಕಾಗಿ ಸುಖದ ಅರಮನೆಗಳನ್ನು ಕಲ್ಪಿಸಿಕೊಳ್ಳುತ್ತಿರುವುದು. ಅದರಂತೆಯೆ ದುಃಖ. ಬಂದ ದುಃಖದ ನೆನಪು ಮುಂದೆ ಮತ್ತೆ ಯಾವ ವಿಪತ್ತು ಕಾದುಕೊಂಡಿದೆಯೊ ನನಗೆ ಎಂದು ತಳಮಳಗೊಳಿಸುವುದು. ಅನುಭವಿಸಿದ ದುಃಖವನ್ನೇ ಮತ್ತೆ ಮತ್ತೆ ಮನಸ್ಸಿನಲ್ಲಿ ಮೆಲಕುಹಾಕುತ್ತಿರುವೆವು. ಅದು ಆಗಿ ಎಷ್ಟೊ ದಿನಗಳಾದವು. ಆದರೆ ನಮ್ಮ ಮನಸ್ಸಿನಲ್ಲಿ ಅದು ನಿತ್ಯ ವರ್ತಮಾನ ಕಾಲದಂತೆ ಇರುವುದು. ನಮಗೆ ಬೇಕಾದಾಗ ಗ್ರಾಮಫೋನ್ ರಿಕಾರ್ಡನ್ನು ಪದೇಪದೇ ಹಾಕಿ ಕೇಳುವಂತೆ ಇದು. ನಾವು ಮಾತ್ರ ಕೇಳುವುದಿಲ್ಲ. ಇತರರಿಗೂ ಕೇಳಿಸುವೆವು. ಇದು ಕಾಲಹರಣ, ವ್ಯರ್ಥ. ಮನಸ್ಸು ಈ ದುಃಖದ ಬಿಲವನ್ನು ಬಿಟ್ಟು ಬರಲೊಲ್ಲದು. ಎತ್ತು ಗಾಣದ ಸುತ್ತಲೂ ಸುತ್ತುತ್ತಿರುವಂತೆ,ಆಗಿಹೋದ ಘಟನೆಯ ಸುತ್ತಲೂ ಮನಸ್ಸು ತಿರುಗುತ್ತಿರುವುದು. ಯೋಗಿಯಾದವನು ಇದನ್ನು ಕೂಡ ನಿಗ್ರಹಿಸುತ್ತಾನೆ. ನಮಗೆ ಬರೀ ಸುಖವೆ ಬರುವುದಿಲ್ಲ ಅಥವಾ ಬರೀ ದುಃಖವೇ ಬರುವುದಿಲ್ಲ. ಒಂದಾದ ಮೇಲೆ ಒಂದು ಬಂದು ಹೋಗುತ್ತಿರುತ್ತವೆ. ಇದನ್ನು ಸ್ವೀಕರಿಸುವುದಕ್ಕೆ ಅವನು ಅಣಿಯಾಗಿರುವನು. ಇದರಿಂದ ಉದ್ವಿಗ್ನನಾಗುವುದಿಲ್ಲ. ಅವನಿಗೆ ಇದೆಲ್ಲ ಸರ್ವೆಸಾಮಾನ್ಯ.

ಯೋಗಿ ಹಾಗೆಯೇ ಮಾನ ಅವಮಾನಗಳಲ್ಲಿಯೂ ಒಂದೇಸಮನಾಗಿರುವನು. ಜನ ಇವನನ್ನು ಗೌರವಿಸಿದಾಗ, ಕೊಂಡಾಡಿದಾಗ, ಆ ಹೊಗಳಿಕೆಯ ಸುರೆಯನ್ನು ಕುಡಿದು ಮೈಮರೆಯು\-ವನಲ್ಲ. ಅದರಂತೆಯೇ ಅಜ್ಞಾನಿಗಳು, ಅಲ್ಪಬುದ್ಧಿಗಳು, ಇವನನ್ನು ತೆಗಳುವಾಗ ಹತಾಶನಾಗಿ ಅವನು ಕುಗ್ಗಿಹೋಗುವುದೂ ಇಲ್ಲ. ಈ ದ್ವಂದ್ವ ಅನುಭವಗಳ ಮೂಲಕ ವ್ಯರ್ಥವಾಗುತ್ತಿದ್ದ ಮನಸ್ಸಿನ ಶಕ್ತಿಯನ್ನೆಲ್ಲ ಸಂಗ್ರಹಿಸಿ ಪರಮಾತ್ಮನ ಕಡೆ ಕೇಂದ್ರೀಕರಿಸುತ್ತಾನೆ. ಅಂತಹ ಪ್ರಶಾಂತವಾದ ಮನಸ್ಸಿನಲ್ಲಿ ಪರಮಾತ್ಮನ ಪ್ರತಿಬಿಂಬ ಇತರ ಎಲ್ಲಾ ಕಡೆಗಿಂತಲೂ ಚೆನ್ನಾಗಿ ವ್ಯಕ್ತವಾಗುವುದು. ಇಲ್ಲಿ ಪರಮಾತ್ಮ ವಿಶೇಷವಾಗಿ ಪ್ರತ್ಯಕ್ಷನಾದಂತೆ ಕಾಣುವುದು. ದೇವರಿಗೇನು ಅಂತಹ ಮನಸ್ಸಿನಮೇಲೆ ವಿಶೇಷವಾದ ಪಕ್ಷಪಾತವಿದೆ ಎಂದಲ್ಲ. ಅವನು ಎಲ್ಲರ ಮೇಲೂ ಒಂದೇ ಸಮನಾಗಿ ಸದಾಕಾಲದಲ್ಲಿಯೂ ಬೆಳಗುತ್ತಿರುವನು. ಆದರೆ ಎಲ್ಲರಿಗೂ ಒಂದೇರೀತಿ ಅದನ್ನು ಪ್ರತಿಬಿಂಬಿಸಲು ಯೋಗ್ಯತೆ ಇಲ್ಲ. ಏಕೆಂದರೆ ಅವರ ಮನಸ್ಸು ಉದ್ವಿಗ್ನವಾಗಿ ಬೀಳುವ ಪ್ರತಿಬಿಂಬ ಅಲ್ಲೋಲಕಲ್ಲೋಲವಾಗಿರುತ್ತದೆ. ಇದು ಪರಮಾತ್ಮನ ತಪ್ಪಲ್ಲ. ಪ್ರತಿಬಿಂಬಿಸುವ ಮನಸ್ಸಿನ ತಪ್ಪು. ಯೋಗಿ ಈ ತಪ್ಪನ್ನು ಸರಿಮಾಡಿರುವನು. ಆದಕಾರಣವೇ ಇಲ್ಲಿ ವಿಶೇಷವಾಗಿ ಸಮಾಹಿತವಾಗಿದೆ.

\begin{shloka}
ಜ್ಞಾನವಿಜ್ಞಾನತೃಪ್ತಾತ್ಮಾ ಕೂಟಸ್ಥೋ ವಿಜಿತೇಂದ್ರಿಯಃ~।\\ಯುಕ್ತ ಇತ್ಯುಚ್ಯತೇ ಯೋಗೀ ಸಮಲೋಷ್ವಾಶ್ಮಕಾಂಚನಃ \hfill॥ ೮~॥
\end{shloka}

\begin{artha}
ಜ್ಞಾನ ಮತ್ತು ವಿಜ್ಞಾನ ಇವುಗಳಿಂದ ತೃಪ್ತಿಹೊಂದಿದ ಮನಸ್ಸುಳ್ಳವನು, ಕೂಟಸ್ಥನೂ ಜೀತೇಂದ್ರಿಯನೂ, ಮಣ್ಣು ಕಲ್ಲು ಚಿನ್ನ ಇವನ್ನು ಸಮನಾಗಿ ನೋಡುವವನೂ ಆದ ಯೋಗಿ ಯುಕ್ತನೆಂದು ಕರೆಯಲ್ಪಡುತ್ತಾನೆ.
\end{artha}

ಪರಮಾತ್ಮನನ್ನು ಪಡೆಯಲು ಯೋಗ್ಯತೆಯನ್ನು ಪಡೆದುಕೊಳ್ಳಬೇಕಾಗಿದೆ. ನಮಗೆ ಬರೀ ಆಸೆ ಯೊಂದಿದ್ದರೆ ಸಾಲದು. ಯೋಗ್ಯತೆಗಳು ಯಾವುವು ಎಂಬುದನ್ನು ವಿವರಿಸುತ್ತಾನೆ ಶ‍್ರೀಕೃಷ್ಣ ಇಲ್ಲಿ. ಅವನಲ್ಲಿ ಜ್ಞಾನ ಮತ್ತು ವಿಜ್ಞಾನ ಎರಡೂ ಇರಬೇಕು. ಮೊದಲನೆಯದು ಜ್ಞಾನ, ಬೌದ್ಧಿಕವಾಗಿ ಅದನ್ನು ತಿಳಿದುಕೊಳ್ಳಬೇಕು. ಶಾಸ್ತ್ರ ಮತ್ತು ದೇವರನ್ನು ಬಲ್ಲವರು ಆ ವಿಷಯದಲ್ಲಿ ಏನನ್ನು ಹೇಳುತ್ತಾರೆ ಎಂಬುದು. ಮತ್ತು ನಾವೇ ಅದನ್ನು ಕುರಿತು ವಿಚಾರಮಾಡಿ ಪರಮಾತ್ಮನೆಂಬ ವಸ್ತು ಈ ಪ್ರಪಂಚದಲ್ಲೆಲ್ಲ ಓತಪ್ರೋತವಾಗಿದೆ; ಅದೊಂದೆ ಎಲ್ಲಕ್ಕಿಂತಲೂ ಸತ್ಯ, ಎಂಬುದನ್ನು ಚೆನ್ನಾಗಿ ತಿಳಿದುಕೊಂಡಿರಬೇಕು. ಅನಂತರ ನಾವು ಏನನ್ನು ತಿಳಿದುಕೊಂಡಿರುವೆವೊ ಅದನ್ನು ಅನುಭವಕ್ಕೆ ತಂದುಕೊಳ್ಳಬೇಕು. ಅನುಭವಕ್ಕೆ ಬರಬೇಕಾದರೆ ನಾವು ಆ ಸತ್ಯವನ್ನು ಅನುಷ್ಠಾನ ಮಾಡಬೇಕು. ಬರೀ ತಿಳಿದುಕೊಂಡ ಮಾತ್ರಕ್ಕೆ ಅದರಿಂದ ಎಲ್ಲಾ ಪ್ರಯೋಜನವೂ ಬರುವುದಿಲ್ಲ. ರೋಗದಿಂದ ನಾನು ನರಳುತ್ತಿದ್ದರೆ, ಅದು ಎಂತಹ ರೋಗ, ಅದು ಹೋಗಬೇಕಾದರೆ ಏನು ಔಷಧಿ ತೆಗೆದುಕೊಳ್ಳಬೇಕು ಇವುಗಳನ್ನು ತಿಳಿದುಕೊಂಡರೆ ಮಾತ್ರ ಸಾಲದು. ಔಷಧಿ ತೆಗೆದುಕೊಳ್ಳಬೇಕು. ಹಾಲಿನಲ್ಲಿ ಬೆಣ್ಣೆ ಇದೆ ಎಂದು ಹೇಳುತ್ತಿದ್ದರೆ ಆ ಬೆಣ್ಣೆ ಮೇಲೆದ್ದು ಬರುವುದಿಲ್ಲ. ಅದನ್ನು ಹೆಪ್ಪುಹಾಕಿ ಕುಳಿತುಕೊಂಡು ಕಡೆಯಬೇಕು. ಒಬ್ಬನಿಗೆ ಎಲ್ಲಾ ಬಗೆಯ ಅಡಿಗೆಯನ್ನು ಮಾಡುವುದಕ್ಕೆ ಗೊತ್ತಿದೆ. ಆದರೆ ಉಪವಾಸ ವಿರುವಾಗ ಈ ಜ್ಞಾನದಿಂದ ಏನೂ ಪ್ರಯೋಜನವಿಲ್ಲ. ಏನಾದರೂ ಆಹಾರ ಪದಾರ್ಥಗಳನ್ನುತಯಾರುಮಾಡಿ ತಿನ್ನಬೇಕು. ಇಲ್ಲಿ ತಿಳಿದುಕೊಳ್ಳುವುದು ಅದನ್ನು ಪಡೆಯುವುದಕ್ಕೆ ಸಹಾಯ ಮಾಡಬೇಕು. ಹಾಗಿದ್ದರೆ ಮಾತ್ರ ತಿಳಿದುಕೊಂಡುದು ಸಾರ್ಥಕವಾಗುವುದು. ಸುಮ್ಮನೆ ಇದೆಲ್ಲ ನನಗೆ ಗೊತ್ತಿದೆ ಎಂದು ಹೆಮ್ಮೆ ಪಟ್ಟರೆ ಏನೂ ಪ್ರಯೋಜನವಿಲ್ಲ. ನಮಗೆ ಗೊತ್ತಿರುವುದೆಲ್ಲ ನಮ್ಮದಲ್ಲ. ನಾವು ಎಷ್ಟನ್ನು ಅರಗಿಸಿಕೊಂಡಿರುವೆವೊ ಅದು ಮಾತ್ರ ನಮ್ಮದು. ಅರಗಿಸಿಕೊಂಡು ತನ್ನದನ್ನಾಗಿ ಮಾಡಿಕೊಂಡಿರುವುದನ್ನೇ ವಿಜ್ಞಾನ ಎನ್ನುವುದು.

ಅವನು ಕೂಟಸ್ಥನಾಗಿರಬೇಕು. ಚಂಚಲಚಿತ್ತನಾಗಕೂಡದು. ಬಲವಾಗಿ ನಿಂತಿರಬೇಕು. ಈ ಪ್ರಪಂಚದಲ್ಲಿ ನಮ್ಮ ಸ್ಥಾನಕ್ಕೆ ಭಂಗಬರುವ ಹಲವು ಅನುಭವಗಳಾಗಿರುತ್ತಿರುತ್ತವೆ. ಎಂತಹ ಬಿರುಗಾಳಿ ಬೀಸಿದರೂ ಕಲ್ಲುಬಂಡೆ ಹೇಗೆ ಭದ್ರವಾಗಿರುವುದೊ ಹಾಗೆ ಈ ಪ್ರಪಂಚದಲ್ಲಿ ನಿಂತಿರು\-ವವನೇ ಕೂಟಸ್ಥ. ಹೀಗೆ ಇರಬೇಕಾದರೆ ಒಬ್ಬನಿಗೆ ಪರಮಾತ್ಮನ ಅನುಭವ ಆಗಿರಬೇಕು. ಆಗ ಮಾತ್ರ ಸಾಧ್ಯ. ಭಗವಂತನ ಕೈಯನ್ನು ಹಿಡಿದಿರಬೇಕು. ಆಗ ಮಾತ್ರ ಈ ಪ್ರಪಂಚದ ಸುಂಟರಗಾಳಿ ಬೀಸಿದರೆ ತರಗೆಲೆಯಂತೆ ಹಾರಿಹೋಗದೆ ಬಂಡೆಯಂತೆ ನಿಂತಿರಲು ಸಾಧ್ಯ. ಅವನು ಜೀತೇಂದ್ರಿಯನಾಗಿರಬೇಕು. ಬಾಹ್ಯ ಪ್ರಪಂಚದ ಆಕರ್ಷಣೆಯಿಂದ ಪಾರಾಗಿರಬೇಕು. ಅವನು ತನ್ನ ಇಂದ್ರಿಯಗಳನ್ನು ಆಯಾ ಸ್ಥಾನದಲ್ಲಿಯೇ ಹಿಡಿದು ಕಟ್ಟಿರುವನು. ವಿಷಯ ವಸ್ತುವಿನ ಕಡೆ ಮೇಯಲು ಬಿಡುವುದಿಲ್ಲ. ಹೊರಗೆ ಹೋಗುವ ಮನಸ್ಸನ್ನು ನಿಗ್ರಹಿಸಿ ಅದನ್ನು ಅಂತರ್ಮುಖ ಮಾಡಿದಲ್ಲದೆ, ಒಬ್ಬನಿಗೆ ಪರಮಾತ್ಮನ ದರ್ಶನ ಸಿಕ್ಕುವುದಿಲ್ಲ. ಇಲ್ಲಿ ಪ್ರಪಂಚವನ್ನೂ ರುಚಿ ನೋಡುತ್ತೇನೆ. ಕಣ್ಣು ಮುಚ್ಚಿಕೊಂಡು ದೇವರನ್ನೂ ರುಚಿ ನೋಡುತ್ತೇನೆ ಎಂದರೆ ಸಾಧ್ಯವಿಲ್ಲ. ದೇವರು ಬೇಕಾದರೆ ಮತ್ತೊಂದರ ಛಾಯೆಯೂ ಇರಬಾರದು ಮನಸ್ಸಿನಲ್ಲಿ. ಅಂತಹ ಚಿತ್ತದಲ್ಲಿ ಮಾತ್ತ ವ್ಯಕ್ತವಾಗುತ್ತಾನೆ ಭಗವಂತ.

ಅಂತಹ ಯೋಗಿ ಮಣ್ಣು ಕಲ್ಲು ಚಿನ್ನ ಇವನ್ನು ಒಂದೇ ಸಮನಾಗಿ ನೋಡುತ್ತಾನೆ. ಅವನ ದೃಷ್ಟಿ ಪಾರಮಾರ್ಥಿಕವಾಗುವುದು. ವ್ಯವಹಾರ ದೃಷ್ಟಿಯಿಂದ ಮಾತ್ರ ಚಿನ್ನ ಹೆಚ್ಚು, ಮಣ್ಣು ಕಡಿಮೆ. ಆದರೆ ಭಗವಂತನನ್ನು ನೋಡುವ ಇಚ್ಛೆ ಇದ್ದರೆ ಚಿನ್ನವನ್ನೂ ಮಣ್ಣಿನ ಸಮನಾಗಿ ನೋಡಬೇಕು. ಶ‍್ರೀರಾಮಕೃಷ್ಣರು ತಮ್ಮ ಸಾಧನಾ ಸಮಯದಲ್ಲಿ ಒಂದು ಕೈಯಲ್ಲಿ ಹಣವನ್ನು ಹಿಡಿದುಕೊಂಡು ಮತ್ತೊಂದು ಕೈಯ್ಯಲ್ಲಿ ಮಣ್ಣನ್ನು ಹಿಡಿದುಕೊಂಡು ಈ ಹಣ ಮಣ್ಣಿನ ಸಮ. ಭಗವಂತನನ್ನು ನೋಡಬೇಕಾದರೆ ಇದೊಂದು ಬಂಧನ. ಲೌಕಿಕ ದೃಷ್ಟಿಯಿಂದ ಹಣದಿಂದ ಬೇಕಾದಷ್ಟು ಪ್ರಯೋಜನವನ್ನು ಪಡೆಯಬಹುದು. ಆದರೆ ದೇವರ ಕಡೆಗೆ ಹೋಗುವವನಿಗೆ ಇದರಿಂದ ಏನೂ ಪ್ರಯೋಜನವಿಲ್ಲವೆಂದು ಮನಸ್ಸಿಗೆ ಹೇಳಿ ಅದನ್ನು ಗಂಗಾನದಿಗೆ ಬಿಸಾಡುತ್ತಿದ್ದರು. ಯೋಗಿಗೆ ಚಿನ್ನದ ಬೆಲೆ ಗೊತ್ತಿಲ್ಲ ಎಂದಲ್ಲ. ಅವನಿಗೆ ಗೊತ್ತಿದೆ ಚಿನ್ನಕ್ಕೆ ಜಗದ ಸಂತೆಯಲ್ಲಿ ಬೆಲೆ ಅಧಿಕ ಎಂಬುದು. ಆದರೆ ಭಗವತ್ ವಸ್ತುಗಳನ್ನು ಕೊಂಡುಕೊಳ್ಳುವುದಕ್ಕೆ ಇದರಿಂದ ಸಾಧ್ಯವಿಲ್ಲ. ಮುಂಚೆ ಒಬ್ಬ ಕಾಮಕಾಂಚನಾಸಕ್ತಿಯಿಂದ ಪಾರಾಗಬೇಕು. ಆಗ ಮಾತ್ರ ದೇವರನ್ನು ಪಡೆಯಲು ಸಾಧ್ಯ. ಇಂದ್ರಿಯ ಜಯವೇ ಕಾಮ ತ್ಯಾಗ. ಮಣ್ಣು ಕಲ್ಲು ಚಿನ್ನ ಇವನ್ನು ಒಂದೇ ಸಮನಾಗಿ ನೋಡುವುದೇ ಕಾಂಚನ ತ್ಯಾಗ.

\begin{shloka}
ಸುಹೃನ್ಮಿತ್ರಾರ್ಯುದಾಸೀನಮಧ್ಯಸ್ಥದ್ವೇಷ್ಯಬಂಧುಷು~।\\ಸಾಧುಷ್ವಪಿ ಚ ಪಾಪೇಷು ಸಮಬುದ್ಧಿರ್ವಿಶಿಷ್ಯತೇ \hfill॥ ೯~॥
\end{shloka}

\begin{artha}
ಸುಹೃತ್, ಮಿತ್ರ, ಅರಿ, ಉದಾಸೀನ, ಮಧ್ಯಸ್ಥ, ದ್ವೇಷ, ಬಂಧು—ಇವರಲ್ಲಿಯೂ ಸಾಧುಗಳಲ್ಲಿಯೂ ಪಾಪಿಗಳಲ್ಲಿಯೂ ಸಮ ಬುದ್ಧಿಯುಳ್ಳವನು ಯೋಗಿಗಳಲ್ಲಿ ಶ್ರೇಷ್ಠ.
\end{artha}

ಮನಸ್ಸನ್ನು ದೇವರ ಕಡೆ ಹರಿಸಬೇಕಾದರೆ ಮೊದಲು ಅದನ್ನು ಅಣಿ ಮಾಡಬೇಕು. ಹಾಡುವುದಕ್ಕೆ ಮುಂಚೆ ಶ್ರುತಿ ಮಾಡಿಕೊಳ್ಳಬೇಕು. ಹಾಗೆಯೇ ಮನಸ್ಸನ್ನು ಅಣಿ ಮಾಡಿಕೊಳ್ಳುವುದು. ಅದರ ಶಕ್ತಿ ಹೊರಗಡೆ ಹರಿದುಕೊಂಡು ಹೋಗಕೂಡದು. ಯಾವ ಯಾವ ಬಿಲದ ಮೂಲಕ ಅದು ಹೋಗುತ್ತದೆಯೊ ಅವನ್ನು ಮುಚ್ಚಬೇಕು. ನೀರನ್ನು ಸೇದುವುದಕ್ಕೆ ಕೊಡವನ್ನು ಬಿಡುವುದಕ್ಕೆ ಮುಂಚೆ ಅದರ ತೂತುಗಳನ್ನೆಲ್ಲ ಮುಚ್ಚಬೇಕು. ಇಲ್ಲದೇ ಇದ್ದರೆ ಕಷ್ಟಪಟ್ಟು ನೀರು ಸೇದುತ್ತೇವೆ. ಆದರೆ ಅದರಲ್ಲಿ ನೀರಿರುವುದು ಅಲ್ಪ. ನಮ್ಮ ಕಷ್ಟಕ್ಕೆ ತಕ್ಕ ಪ್ರತಿಫಲ ಸಿಕ್ಕಲಿಲ್ಲ. ಯೋಗಿಯಾಗಿರುವವನು, ಪ್ರಪಂಚದಲ್ಲಿ ವೈವಿಧ್ಯತೆಗಳನ್ನು ನೋಡಿದಾಗ, ಅದರ ಕಡೆ ಅಷ್ಟು ವಾಲುವುದಿಲ್ಲ. ಯಾವಾಗ ನಾವು ಕೆಟ್ಟದ್ದನ್ನು ಕಂಡು ದ್ವೇಷಿಸುತ್ತೇವೆಯೊ ಆಗ ಒಳ್ಳೆಯದನ್ನು ಕಂಡು ಪ್ರೀತಿಸಬೇಕಾಗುವುದು. ಒಂದನ್ನು ದ್ವೇಷಿಸಿದರೆ ಮತ್ತೊಂದನ್ನು ಪ್ರೀತಿಸಬೇಕಾಗುವುದು. ಈ ಆಕರ್ಷಣೆ ವಿಕರ್ಷಣೆಗಳಿಂದ ಮನಸ್ಸು ವಿಚಲಿತವಾಗುವುದು. ಅವನು ಎಲ್ಲಾ ಕಡೆಯೂ ದೇವರನ್ನೆ ನೋಡಲು ಇಚ್ಛೆಪಡುತ್ತಾನೆ. ಅವನನ್ನು ಕಾಣದಂತೆ ಮಾಡಿರುವುದೇ ಅದಕ್ಕೆ ತಗಲುಹಾಕಿರುವ ಉಪಾಧಿಯ ಮುಖವಾಡ. ಈ ಮುಖವಾಡಗಳಲ್ಲಿ ಒಳ್ಳೆಯವೂ ಇವೆ, ಕೆಟ್ಟದ್ದೂ ಇವೆ. ಒಳ್ಳೆಯದು ಎಷ್ಟುಮಟ್ಟಿಗೆ ನಮ್ಮನ್ನು ಕಟ್ಟಿಹಾಕುವುದೊ ಹಾಗೆಯೇ ಕೆಟ್ಟದ್ದು ನಮ್ಮನ್ನು ಕಟ್ಟಿಹಾಕುವುದು. ಇದನ್ನು ಅರಿತೇ ಯೋಗಿಯಾದವನು ಉಪಾಧಿಯ ಕಡೆ ನೋಡದೆ ಹಿಂದೆ ಇರುವ ಪರಮಾತ್ಮ ವಸ್ತುವಿನ ಕಡೆ ನೋಡುತ್ತಾನೆ.

ಸುಹೃತ್ ಎಂದರೆ ಪ್ರತ್ಯುಪಕಾರವನ್ನು ಬಯಸದೆ ಉಪಕಾರವನ್ನು ಮಾಡುವವನು. ಇಂತಹ ವ್ಯಕ್ತಿಗಳು ಬಹಳ ಅಪರೂಪ, ಈ ಸ್ವಾರ್ಥತೆಯಿಂದ ಕೂಡಿರುವ ಪ್ರಪಂಚದಲ್ಲಿ. ಈ ಸುಹೃತ್ತಿನ ಸ್ವಭಾವ ಮತ್ತೊಬ್ಬನಿಗೆ ಸಹಾಯ ಮಾಡುವುದು. ನನಗೆ ಅವನಿಂದ ಏನಾದರೂ ಬರುತ್ತದೆಯೆ ಎಂದು ಗಮನಿಸುವುದೇ ಇಲ್ಲ. ಮಲ್ಲಿಗೆಯ ಗಿಡ ತನ್ನ ಪರಿಮಳವನ್ನು ಸುತ್ತಲೂ ಚೆಲ್ಲುತ್ತಿವುದು. ಜನರು ಅದನ್ನು ಹೊಗಳಲಿ ಬಿಡಲಿ ಅದನ್ನು ಲೆಕ್ಕಿಸುವುದೇ ಇಲ್ಲ. ಈ ಗುಂಪಿಗೆ ಸೇರಿದ ವ್ಯಕ್ತಿ ಸುಹೃತ್ ಎಂಬುವನು. ಮಾನವನ ವಿಕಾಸದ ಏಣಿಯಲ್ಲಿ ತುತ್ತತುದಿಯಲ್ಲಿ ಇರುವವನು ಇವನು. ಇವನ ನಂತರ ಬರುವವನೆ ಮಿತ್ರ; ನನ್ನ ಸ್ನೇಹಿತ. ಇಲ್ಲಿ ನಾವಿಬ್ಬರೂ ಒಂದೇ ಮೆಟ್ಟಲಿನಲ್ಲಿ ಇರುತ್ತೇವೆ. ನಾನು ಅವನಿಗೆ ಸಹಾಯ ಮಾಡುತ್ತೇನೆ, ಅವನು ನನಗೆ ಸಹಾಯ ಮಾಡುತ್ತಾನೆ. ನನ್ನ ಕಷ್ಟಕಾಲದಲ್ಲಿ ಅವನು ಆಗುತ್ತಾನೆ. ಅವನ ಕಷ್ಟಕಾಲದಲ್ಲಿ ನಾನು ಆಗುತ್ತೇನೆ. ಅನಂತರವೇ ಅರಿ. ನನಗೆ ಆಗದವನು, ನನಗೆ ಕೆಟ್ಟದ್ದನ್ನು ಹಿಂದೆ ಮಾಡಿದವನು, ಮುಂದೆ ಮಾಡುವುದಕ್ಕೆ ಹೊಂಚು ಹಾಕುತ್ತಿರುವವನು. ಇವನನ್ನು ನೆನೆಸಿಕೊಂಡರೆ ಸಾಧಾರಣ ವ್ಯಕ್ತಿಗೆ ಕೋಪ ಬರುವುದು. ಆದರೆ ಯೋಗಿಯಾದವನು ಇದನ್ನು ತಡೆಯಬೇಕು. ಉದಾಸೀನ ಎಂದರೆ ಯಾರ ಪಕ್ಷವನ್ನೂ ವಹಿಸದವನು –ನಮಗೂ ಅವನಿಗೂ ಯಾವ ಸಂಬಂಧವೂ ಇಲ್ಲ. ಮಧ್ಯಸ್ಥ ಎಂದರೆ, ಇಬ್ಬರು ಕಾದಾಡುತ್ತಿದ್ದರೆ ಅವನು ಹಾಗೆಯೇ ನೋಡುತ್ತ ಹೋಗುವುದಿಲ್ಲ. ಆ ಜಗಳದ ಮಧ್ಯೆ ಬಂದು ಬಿಡಿಸುವನು. ನೀವು ಕಾದಾಡಬೇಡಿ ಎಂದು ಹೇಳುವವನು. ನನ್ನನ್ನು ಕಂಡರೆ ಆಗದ ದ್ವೇಷಿ ಒಬ್ಬ, ಇನ್ನೊಬ್ಬ ನನಗೆ ಸಂಬಂಧಿಯಾದ ನಂಟ. ಒಬ್ಬ ಒಂದು ತಪ್ಪು ಕೆಲಸವನ್ನೂ ಮಾಡದ ಸಾಧುವೃತ್ತಿಯುಳ್ಳವನು, ಮತ್ತೊಬ್ಬ ಎಲ್ಲಾ ಪಾಪಕೃತ್ಯಗಳನ್ನೂ ಮಾಡಿರುವವನು. ಇಂತಹ ವ್ಯಕ್ತಿಗಳನ್ನೆಲ್ಲ ಯೋಗಿ ಒಂದೇ ಸಮನಾಗಿ ನೋಡುತ್ತಾನೆ. ಅವನು ಆಸಕ್ತನಾಗುವುದಿಲ್ಲ. ಸಾಕ್ಷಿಯಂತೆ ನಿಂತು ನೋಡುತ್ತಾನೆ. ಈ ಪ್ರಪಂಚದಲ್ಲಿ ಬಹುಬಗೆಯ ಪುಷ್ಪಗಳಿವೆ. ಸಂಪಿಗೆಯಿಂದ ಹಿಡಿದು ದುರ್ವಾಸನೆ ಕೊಡುವ ಹೂಗಳವರೆಗೂ ಹೂ ಬಿಡುವ ವೃಕ್ಷಗಳಿವೆ. ವೈವಿಧ್ಯತೆಯೇ ಈ ಪ್ರಪಂಚದ ನಿಯಮ. ವಿಕಾಸದ ಏಣಿಯಲ್ಲಿ ಒಬ್ಬ ಮೇಲಿರುವನು, ಇನ್ನೊಬ್ಬ ಮಧ್ಯದಲ್ಲಿರುವನು, ಮತ್ತೊಬ್ಬ ಕೆಳಗಿರುವನು. ಅವನು ಒಳ್ಳೆಯ ವನನ್ನು ಹೊಗಳುವುದಕ್ಕೂ ಹೋಗುವುದಿಲ್ಲ, ಕೆಟ್ಟವನನ್ನು ದೂರುವುದಕ್ಕೂ ಹೋಗುವುದಿಲ್ಲ. ಎಲ್ಲರೂ ತಮ್ಮ ತಮ್ಮ ಧರ್ಮಗಳನ್ನು ವ್ಯಕ್ತಪಡಿಸುತ್ತಿರುವರು. ಚೇಳು ಒಂದು ಕಡೆ ಬಾಲದಲ್ಲಿ ವಿಷ ತುಂಬಿಕೊಂಡು ಹೋಗುತ್ತಿರುವುದು. ಹಸು ಒಂದು ಕಡೆ ಕೆಚ್ಚಲಿನಲ್ಲಿ ಹಾಲು ತುಂಬಿಕೊಂಡು ಹೋಗುತ್ತಿರುವುದು. ಚೇಳು ಕುಟುಕಿದಾಗ ವಿಷ ಸೋರುವುದು. ಹಸುವಿನ ಕೆಚ್ಚಲನ್ನು ಹಿಂಡಿದರೆ ಹಾಲು ತೊಟ್ಟಿಕ್ಕುವುದು. ಇದರಂತೆಯೇ ಈ ಪ್ರಪಂಚದಲ್ಲಿ ವಿವಿಧ ವ್ಯಕ್ತಿಗಳು ವಿವಿಧ ಧರ್ಮವನ್ನು ವ್ಯಕ್ತಪಡಿಸುತ್ತಿರುವರೆಂದು ತಿಳಿದು ಯಾರ ಕಡೆಯೂ ವಾಲದೆ ಎಲ್ಲದರ ಹಿಂದೆ ಇರುವ ಪರಮಾತ್ಮನ ಕಡೆ ಮನಸ್ಸಿಟ್ಟವನೇ ಯೋಗಿ.

\begin{shloka}
ಯೋಗೀ ಯುಂಜೀತ ಸತತಮಾತ್ಮಾನಂ ರಹಸಿ ಸ್ಥಿತಃ~।\\ಏಕಾಕೀ ಯತಚಿತ್ತಾತ್ಮಾ ನಿರಾಶೀರಪರಿಗ್ರಹಃ \hfill॥ ೧೦~॥
\end{shloka}

\begin{artha}
ನಿರ್ಜನ ಪ್ರದೇಶದಲ್ಲಿದ್ದುಕೊಂಡು ಏಕಾಕಿಯಾಗಿ ದೇಹ ಮತ್ತು ಮನಸ್ಸುಗಳನ್ನು ನಿಗ್ರಹಿಸಿ, ಯಾವ ಭರವಸೆಯನ್ನೂ ಇಟ್ಟುಕೊಳ್ಳದೆ, ಅಪರಿಗ್ರಹನಾಗಿ ಯಾವಾಗಲೂ ಮನಸ್ಸನ್ನು ಯೋಗದಲ್ಲಿಡಬೇಕು.
\end{artha}

ಉತ್ತಮ ಫಲವನ್ನು ಪಡೆಯಬೇಕಾದರೆ ಧ್ಯಾನ ಮಾಡುವುದಕ್ಕೆ ಮನಸ್ಸನ್ನು ಹೇಗೆ ಅನುಗೊಳಿಸಬೇಕು ಎಂಬುದನ್ನು ಹೇಳುತ್ತಾನೆ. ಯೋಗಿಯಾಗಬೇಕೆಂದು ಬಯಸುವವನು ನಿರ್ಜನಪ್ರದೇಶ\-ದಲ್ಲಿರಬೇಕು. ಜನರು ಸಂಚಾರಮಾಡುವ ಸ್ಥಳವಾಗಿರಬಾರದು. ಅದು ಊರು ಕೇರಿಗಳಿಂದ ದೂರವಿರಬೇಕು. ಹಿಂದಿನ ಕಾಲದಲ್ಲಿ ಪುಷಿಗಳು ತಪಸ್ಸನ್ನು ಮಾಡುವುದಕ್ಕೆ ಬೆಟ್ಟ, ಕಾಡು, ಗುಹೆ, ನದೀತೀರ ಮುಂತಾದ ಸ್ಥಳಗಳನ್ನು ಆರಿಸಿಕೊಳ್ಳುತ್ತಿದ್ದರು. ನಿರ್ಜನಪ್ರದೇಶದಲ್ಲಿಯೂ ಅವನು ಇತರರೊಡನೆ ಇರಬಾರದು. ಏಕಾಕಿಯಾಗಿರಬೇಕು. ಯಾವಾಗ ಒಬ್ಬರಿಗಿಂತ ಹೆಚ್ಚು ಸೇರುವರೊ, ಆಗ ಮಾತ ನಾಡಲೇಬೇಕು. ಇನ್ನೊಬ್ಬ ಯಾವುದಾದರೂ ಮಾತನ್ನು ಎತ್ತಿದರೆ ಇವನು ಕೂಡ ಅದರಲ್ಲಿ ಬೆರೆಯಬೇಕಾಗುವುದು. ಇನ್ನೊಬ್ಬ ಯಾವಾಗಲೂ ಆಧ್ಯಾತ್ಮಿಕ ವಿಷಯಗಳನ್ನೇ ಆಡದೆ ಇರಬಹುದು. ಇದರಿಂದ ಮನಸ್ಸು ಚಂಚಲವಾಗುವುದು. ನಮ್ಮ ಮನಸ್ಸನ್ನು ನಾವು ಶೋಧಿಸಿಕೊಳ್ಳಬೇಕಾದರೆ ನಾವು ಒಬ್ಬರೇ ಇರಬೇಕು. ಯಾರೂ ನಮ್ಮನ್ನು ನೋಡುತ್ತಿರಬಾರದು. ಇಂತಹ ಒಂದು ವಾತಾವರಣದಲ್ಲಿ ಮನಸ್ಸು ಅಡ್ಡಾಡುವುದು ಕಡಮೆಯಾಗುವುದು.\break ಶ‍್ರೀರಾಮಕೃಷ್ಣರು ಧ್ಯಾನಮಾಡುವುದಕ್ಕೆ ಒಂದು ನಿರ್ಜನ ಪ್ರದೇಶವನ್ನು ಆರಿಸಿಕೊ, ಇಲ್ಲದೇ ಇದ್ದರೆ, ನೀನೊಬ್ಬನೇ ಇರುವ ಕೋಣೆ, ಅದು ಸಾಧ್ಯವಿಲ್ಲದೇ ಇದ್ದರೆ ಮನೆಯಲ್ಲಿ ಒಂದು ಮೂಲೆ ಇವನ್ನು ಆರಿಸಿಕೊ ಎಂದು ಹೇಳುತ್ತಿದ್ದರು. ಜನ ನಾವು ಮಾಡುವುದನ್ನು ನೋಡಿದರೆ, ಕೆಲವರು ನಮಗೆ ಎಲ್ಲೂ ಇಲ್ಲದ ಗೌರವ ತೋರಿಸಬಹುದು. ಇದು ನಮ್ಮನ್ನು ಹಾಳುಮಾಡುವುದು. ಮತ್ತೆ ಕೆಲವರು ನಮ್ಮನ್ನು ಹಾಸ್ಯ ಮಾಡುವರು, ಟೀಕಿಸುವರು. ಇವುಗಳನ್ನೆಲ್ಲ ಸಹಿಸಬೇಕಾದರೆ ಬಹಳ ಕಷ್ಟ, ಪ್ರಾರಂಭದಲ್ಲಿ. ಅನಾವಶ್ಯಕವಾಗಿ ಈ ಟೀಕೆ ಟಿಪ್ಪಣಿಗಳಿಗೆಲ್ಲ ಏತಕ್ಕೆ ಅವಕಾಶ ಕೊಡಬೇಕು? ಸುಮ್ಮನೆ ಇತರರಿಗೆ ಕಾಣದಂತೆ ಮಾಡಿಕೊಂಡರೆ ಮೇಲು.

ಅವನು ನಿರ್ಜನ ಪ್ರದೇಶದಲ್ಲಿ ಏಕಾಕಿಯಾಗಿರುವುದೊಂದೇ ಸಾಲದು. ಅಲ್ಲಿ ಮನಸ್ಸು ವಿಷಯ ವಸ್ತುಗಳಿಂದ ದೂರದಲ್ಲಿರಬಹುದು. ಆದರೆ ಮನಸ್ಸು ಅದನ್ನು ಕುರಿತು ಚಿಂತಿಸದಂತೆ ಮಾಡಬೇಕು. ಇಲ್ಲದೇ ಇದ್ದರೆ ಹೊರಗೆ ಎಲ್ಲಾ ಪ್ರಶಾಂತವಾಗಿದೆ, ನಮ್ಮ ಮನಸ್ಸಾದರೋ ಒಂದು ಸಂತೆಯಂತೆ ಇದೆ. ಇದರಿಂದ ಏನೂ ಪ್ರಯೋಜನವಿಲ್ಲ. ನಾವು ನಮ್ಮ ಆಲೋಚನೆಯನ್ನು ನಿಗ್ರಹಿಸಬೇಕು. ವೈಷಯಿಕ ವಸ್ತುಗಳನ್ನು ಕುರಿತು ಚಿಂತಿಸುವುದನ್ನು ಬಿಟ್ಟು, ದೇವರ ಕಡೆ ಹರಿಸಬೇಕು. ಇಂದ್ರಿಯಗಳನ್ನು ನಾವು ನಿಗ್ರಹಿಸಿರಬೇಕು.

ಅವನು ನಿರಾಶಿಯಾಗಿರಬೇಕು. ಮುಂದೆ ಏನಾಗುತ್ತೇನೆ, ಏನು ಪ್ರಾಪ್ತವಾಗುತ್ತದೆ, ಎಂತಹ ಆಸೆಗಳು ಕೈಗೂಡಬಹುದು ಎಂಬ ವಿಷಯಗಳನ್ನೆಲ್ಲ ಕಲ್ಪಿಸಿಕೊಳ್ಳುವುದನ್ನು ಬಿಡಬೇಕು. ಗುರಿಯ ಕಡೆ ನಡೆಯುವುದೊಂದರಲ್ಲೇ ಆಸಕ್ತನಾಗಿರಬೇಕು. ದಾರಿಯ ಅಕ್ಕಪಕ್ಕದಲ್ಲಿ ಏನಾಗುವುದೆಂಬು\-ದನ್ನು ನೋಡುವುದಕ್ಕಾಗಲಿ, ಅಡ್ಡಹಾದಿ ಹಿಡಿದರೆ ಅಲ್ಲೇನು ಸಿಕ್ಕುವುದು ಎಂಬ ಕುತೂಹಲವನ್ನು ತೃಪ್ತಿ ಪಡಿಸುವುದಾದರೂ ಆಗಲಿ, ಯಾವುದೂ ಇರಬಾರದು. ಬಾಣವನ್ನು ಬಿಲ್ಲಿನಿಂದ ಹೆದೆಯೇರಿಸಿ ಗುರಿಕಡೆಗೆ ಬಿಟ್ಟರೆ, ಹೇಗೆ ಅದು ಗುರಿ ಕಡೆಗೆ ಒಂದು ಸರಳರೇಖೆಯಲ್ಲಿ ಹೋಗಿ ಭೇದಿಸುವುದೊ ಹಾಗೆ ಮನಸ್ಸು ಧ್ಯಾನಿಸುವ ವಸ್ತುವಿನ ಕಡೆಗೆ ಸರಿಯುತ್ತಿರಬೇಕು. ಯೋಗಿ ಅಪರಿಗ್ರಹನಾಗಿರಬೇಕು. ಯಾರಿಂದಲೂ ಯಾವ ಬಹುಮಾನವನ್ನೂ ಯಾವ ವಸ್ತುವನ್ನೂ ಸ್ವೀಕರಿಸಬಾರದು. ಯಾವಾಗ ನಾವು ಸ್ವೀಕರಿಸುತ್ತೇವೆಯೊ ಆಗ ನಮ್ಮ ಸ್ವಾತಂತ್ರ್ಯವನ್ನು ಕಳೆದುಕೊಳ್ಳುತ್ತೇವೆ. ಅವನು ಕೊಡುವ ಒಂದು ಚೂರಿಗೆ ಕೈಯೊಡ್ಡಿದರೆ, ಅವನು ಹೇಳಿದಂತೆ ಕೇಳಬೇಕಾಗುವುದು. ಏನನ್ನೂ ನಿರೀಕ್ಷಿಸದೆ ಜೀವನದಲ್ಲಿ ಇತರರಿಗೆ ಕೊಡುವುದು ಅಪರೂಪ ಈ ಪ್ರಪಂಚದಲ್ಲಿ. ಕೊಡುವವರ ಮನಸ್ಸಿನಲ್ಲಿ ಏನಾದರೂ ಫಲಾಪೇಕ್ಷೆಯ ಬುದ್ಧಿ ಇದ್ದೇ ಇರುವುದು. ಆದಕಾರಣವೇ ಯೋಗಿ ಕೈಯೊಡ್ಡಬಾರದು ಇನ್ನೊಬ್ಬರಿಗೆ.

ಇನ್ನೊಂದು ಸಲಹೆಯನ್ನು ಶ‍್ರೀಕೃಷ್ಣ ಇಲ್ಲಿ ಕೊಡುತ್ತಾನೆ. ಯೋಗಿಯಾಗಲು ಬಯಸುವವನು ಕೇವಲ ಧ್ಯಾನ ಮಾಡುವಾಗ ಮಾತ್ರ ಮನಸ್ಸನ್ನು ಏಕಾಗ್ರಗೊಳಿಸಿ ಧ್ಯೇಯವಸ್ತುವಿನ ಕಡೆ ಬಿಡುವು ದಲ್ಲ. ಇತರ ಕಾಲದಲ್ಲಿಯೂ ಮನಸ್ಸನ್ನು ಸಾಧ್ಯವಾದಷ್ಟು ಆದರ್ಶದ ಕಡೆಗೆ ಹರಿಸುತ್ತಿರಬೇಕು. ಆದರ್ಶವನ್ನು ಪದೇಪದೇ ಮೆಲಕು ಹಾಕುತ್ತಿರಬೇಕು. ಅದನ್ನು ಕುರಿತು ಮನಸ್ಸಿನಲ್ಲಿ ಚಿಂತಿಸುವುದು ನಮ್ಮ ಸ್ವಭಾವ ಆಗಿರಬೇಕು. ಆಗಲೇ ಧ್ಯಾನಕ್ಕೆ ಕುಳಿತಾಗ ಧ್ಯಾನ ಚೆನ್ನಾಗಿ ಆಗುವುದು. ಇಲ್ಲದೇ ಇದ್ದರೆ ಆಗಲೂ ಕೂಡ ಒಂದು ಹೋರಾಟ ಆಗುವುದು.

\begin{shloka}
ಶುಚೌ ದೇಶೇ ಪ್ರತಿಷ್ಠಾಪ್ಯ ಸ್ಥಿರಮಾಸನಮಾತ್ಮನಃ~।\\ನಾತ್ಯುಚ್ಛ್ರಿತಂ ನಾತಿ ನೀಚಂ ಚೈಲಾಜಿನ ಕುಶೋತ್ತರಮ್ \hfill॥ ೧೧~॥
\end{shloka}

\begin{artha}
ಶುಚಿಯಾದ ಪ್ರದೇಶದಲ್ಲಿ, ಸ್ಥಿರವಾದ, ಅತಿ ಎತ್ತರವೂ ಅತಿ ತಗ್ಗೂ ಅಲ್ಲದ, ಬಟ್ಟೆ ಚರ್ಮ ಕುಶ ಇವುಗಳನ್ನು ಒಂದರಮೇಲೊಂದನ್ನು ಹಾಕಿರುವ ಆಸನವನ್ನು ಮಾಡಿಕೊಳ್ಳಬೇಕು.
\end{artha}

ಧ್ಯಾನ ಮಾಡುವ ಸ್ಥಳ ಶುಚಿಯಾಗಿರಬೇಕು. ಅಲ್ಲಿ ಕಸ ಕೊಳೆ ಇವುಗಳು ಇರಕೂಡದು. ಸ್ಥಳ ಮಾತ್ರ ಶುಚಿಯಾಗಿರುವುದಲ್ಲ. ನಾವು ಕೂಡ ಶುಚಿಯಾಗಿರಬೇಕು. ಸ್ನಾನಾದಿಗಳನ್ನು ಮಾಡಿರಬೇಕು. ಕೈಕಾಲುಗಳನ್ನು ತೊಳೆದುಕೊಂಡಿರಬೇಕು. ಒಗೆದ ಬಟ್ಟೆಗಳನ್ನು ಉಟ್ಟುಕೊಳ್ಳಬೇಕು. ಹೊರಗೆ ಮಾತ್ರ ಶುಚಿಯಲ್ಲ, ನಮ್ಮ ದೇಹ ಮತ್ತು ನಾವು ಉಡುವ ಬಟ್ಟೆಬರೆಗಳೂ ಕೂಡ ಮಡಿಯಾಗಿರಬೇಕು.

ನಾವು ಯಾವುದರಮೇಲೆ ಕುಳಿತುಕೊಳ್ಳುವೆವೊ ಅದು ಸ್ಥಿರವಾಗಿರಬೇಕು. ಅಲ್ಲಾಡಕೂಡದು. ಯಾವಾಗ ಅದು ವಾಲುವುದೊ, ಅಲ್ಲಾಡುವುದೊ, ಆಗ ಮನಸ್ಸೆಲ್ಲ ದೇಹದ ಕಡೆ ವಾಲುವುದು. ಚಿತ್ತ ಚಂಚಲವಾಗುವುದು. ಅದು ಎತ್ತರವಾದ ಸ್ಥಳದಲ್ಲಿಯೂ ಇರಕೂಡದು, ಸುತ್ತಮುತ್ತಲ ಸ್ಥಳಕ್ಕಿಂತ ತಗ್ಗೂ ಇರಬಾರದು. ಎತ್ತರದಲ್ಲಿದ್ದರೆ ಎಲ್ಲಿ ಬೀಳುತ್ತೇವೆಯೊ ಎಂಬ ಭಯ, ತಗ್ಗಿನ ಪ್ರದೇಶದಲ್ಲಿದ್ದರೆ ಜೌಗು ಇರಬಹುದು, ಸುತ್ತಮುತ್ತ ಹರಿದಾಡುವ ಹುಳುಹುಪ್ಪಡೆಗಳೆಲ್ಲ ಅಲ್ಲಿ ಬರಬಹುದು. ಅದು ಸಾಮಾನ್ಯವಾದ ಸ್ಥಳವಾಗಿರಬೇಕು. ನಾವು ಧ್ಯಾನಮಾಡಬೇಕಾದರೆ ಒಂದು ಆಸನದ ಮೇಲೆ ಕುಳಿತುಕೊಳ್ಳಬೇಕು. ಆಸನ ಪವಿತ್ರವಾಗಿರಬೇಕು, ಅಷ್ಟು ಗಟ್ಟಿಯಾಗಿರಬಾರದು. ಅದರೊಂದಿಗೆ ಪವಿತ್ರ ಸಂಬಂಧ ಬೆಳೆದಿರಬೇಕು. ಅದನ್ನು ಧ್ಯಾನಮಾಡುವಾಗ ಮಾತ್ರ ಬಳಸಬೇಕು. ಬೇರೆ ಕೆಲಸ ಮಾಡುವಾಗ ಅದರ ಮೇಲೆ ಕುಳಿತುಕೊಳ್ಳಬಾರದು. ಕೇವಲ ಧ್ಯಾನಕ್ಕೆ ಮಾತ್ರ ಮೀಸಲಾಗಿಡಬೇಕು ಆಸನ. ಕೆಳಗೆ ಕುಶಾಸನ ಇರಬೇಕು. ಅದರ ಮೇಲೆ ಜಿಂಕೆಯೊ ಅಥವಾ ಯಾವುದಾದರೊ ಮೃಗದ ಚರ್ಮ ಇರಬೇಕು. ಅದರ ಮೇಲೆ ಬಟ್ಟೆ ಇರಬೇಕು. ಇದನ್ನು ನಾವು ಮಾತ್ರ ಉಪಯೋಗಿಸಬೇಕು. ಅದನ್ನು ಇತರರು ಉಪಯಾಗಿಸಬಾರದು ಅಥವಾ ಇತರರು ಉಪ ಯೋಗಿಸುವ ಆಸನದ ಮೇಲೆ ನಾವು ಕೂಡಬಾರದು. ಏಕೆಂದರೆ ನಾವು ಉಪಯೋಗಿಸುವ ವಸ್ತುವಿನ ಮೇಲೆಲ್ಲ ನಮ್ಮವಾಸನೆಯನ್ನು ಬಿಡುವೆವು. ಅವು ನಮ್ಮ ಮೇಲೆ ತಮ್ಮ ಪ್ರಭಾವವನ್ನು ಬೀರುವುವು. ನಾವು ಯಾವಾಗ ಇನ್ನೊಬ್ಬರ ಆಸನವನ್ನು ಉಪಯೋಗಿಸುತ್ತೇವೆಯೊ ಅವರ ವಾಸನೆಗೂ ನಮ್ಮ ಮನಸ್ಸು ಒಳಗಾಗುವುದು. ಹೇಗೆ ನಾವು ನಮ್ಮ ಬಟ್ಟೆಬರೆಗಳನ್ನು ಮಾತ್ರ ಉಪಯೋಗಿಸುತ್ತೇವೆಯೊ,ಅದಕ್ಕೆ ನಮ್ಮ ಶರೀರ ಒಗ್ಗಿರುವುದೊ, ಹಾಗೆಯೆ ಧ್ಯಾನಮಾಡುವ ಆಸನವೂ ನಮ್ಮ ಸ್ವಂತದ್ದಾಗಿರಬೇಕು.

\begin{shloka}
ತತ್ರೈಕಾಗ್ರಮ್ ಮನಃ ಕೃತ್ವಾ ಯತಚಿತ್ತೇಂದ್ರಿಯಕ್ರಿಯಃ~।\\ಉಪವಿಶ್ಯಾಸನೇ ಯುಂಜ್ಯಾದ್ಯೋಗಮಾತ್ಮವಿಶುದ್ಧಯೇ \hfill॥ ೧೨~॥
\end{shloka}

\begin{artha}
ಆಸನದ ಮೇಲೆ ಕುಳಿತುಕೊಂಡು ಮನಸ್ಸನ್ನು ಏಕಾಗ್ರ ಮಾಡಿ ಚಿತ್ತ ಇಂದ್ರಿಯಗಳನ್ನು ನಿಗ್ರಹಿಸಿ ಅಂತಃಕರಣ ಶುದ್ಧಿಗಾಗಿ ಯೋಗವನ್ನು ಮಾಡಬೇಕು.
\end{artha}

ಅವನು ಆಸನದ ಮೇಲೆ ಕುಳಿತುಕೊಂಡು ಚಿತ್ತ ಮತ್ತು ಇಂದ್ರಿಯಗಳನ್ನು ನಿಗ್ರಹಿಸಬೇಕು. ಹೊರಗಡೆಯಿಂದ ಒಳಗೆ ಸುದ್ದಿ ಸಮಾಚಾರಗಳು ಬರದಂತೆ ಬಾಗಿಲನ್ನು ಹಾಕಬೇಕು. ನಾವೊಂದು ಮ್ಯಾಟಿನಿ ಸಿನಿಮಾ ನೋಡುವುದಕ್ಕೆ ಹೋದರೆ ಸಿನಿಮಾ ಪ್ರಾರಂಭವಾಗುವುದಕ್ಕೆ ಮುಂಚೆ ಕಿಟಿಕಿ ಮತ್ತು ಬಾಗಿಲುಗಳನ್ನು ಹಾಕುತ್ತಾರೆ. ಏಕೆಂದರೆ ಹೊರಗಡೆಯಿಂದ ಬೆಳಕು ಬರುತ್ತಿದ್ದರೆ ತೆರೆಯ ಮೇಲೆ ಬೀಳುವ ಚಿತ್ರ ಚೆನ್ನಾಗಿ ಕಾಣುವುದಿಲ್ಲ. ಹಾಗೆಯೇ ಪ್ರಪಂಚದಿಂದ ಬರುವ ಕಾಂತಿಯನ್ನು ತಗ್ಗಿಸುವುದಕ್ಕೆ, ಇಂದ್ರಿಯ ಮತ್ತು ಮನಸ್ಸನ್ನು ವಿಷಯವಸ್ತುಗಳ ಕಡೆಯಿಂದ ಸೆಳೆಯಬೇಕು. ಅನಂತರ ಮನಸ್ಸನ್ನು ಏಕಾಗ್ರಮಾಡಿ ಧ್ಯೇಯವಸ್ತುವಿನ ಕಡೆ ಬಿಡಬೇಕು. ಏಕಾಗ್ರವಾದ ಮನಸ್ಸಿಗೆ ಅದ್ಭುತ ಶಕ್ತಿ ಬರುವುದು. ಸೂರ್ಯಕಿರಣ ಎಲ್ಲಾಕಡೆಯೂ ಬೀಳುತ್ತದೆ. ನಾವು ಒಂದು ಭೂತಕನ್ನಡಿಯನ್ನು ಸೂರ್ಯನ ಬೆಳಕಿಗೆ ಅಡ್ಡಲಾಗಿ ಹಿಡಿದು ಅದರ ಕೆಳಗೆ ನಮ್ಮ ಕೈಯನ್ನು ಇಟ್ಟರೆ ಆ ಬೆಳಕು ಕೇಂದ್ರೀಕೃತವಾದ ಚುಕ್ಕಿ ನಮ್ಮ ಕೈಯನ್ನು ಸುಡುವುದು. ಈ ಸುಡುವ ಶಕ್ತಿ ಎಲ್ಲಿಂದ ಬಂತು? ಸಾಧಾರಣವಾಗಿ ಸೂರ್ಯಕಿರಣಗಳು ಚದುರಿಹೋಗುವುವು. ಇಲ್ಲಿ ಒಟ್ಟುಗೂಡಿ ಒಂದು ಸ್ಥಳದಲ್ಲಿ ಕೇಂದ್ರೀಕೃತವಾಗಿದೆ. ಅಲ್ಲಿ ಹೊಸದೊಂದು ಶಕ್ತಿ ಬಂದಂತೆ ಕಾಣುವುದು. ಹಾಗೆಯೇ ನಿಗ್ರಹಿಸಿದ ಮನಸ್ಸನ್ನು ಧ್ಯೇಯವಸ್ತುವಿನ ಕಡೆ ಬಿಟ್ಟರೆ ಅಂತಹ ಒಂದು ಶಕ್ತಿಯನ್ನು ಮನಸ್ಸು ಪಡೆಯುವುದು.

ಇಲ್ಲಿ ನಾವು ಧ್ಯೇಯವಸ್ತುವನ್ನು ಕುರಿತು ಚಿಂತಿಸುವುದು ಏತಕ್ಕೆ ಎಂದರೆ ನಮ್ಮ ಮನಸ್ಸನ್ನು ಶುದ್ಧಿ ಮಾಡಿಕೊಳ್ಳುವುದಕ್ಕೆ. ಮನಸ್ಸು ಯಾವಾಗ ಶುದ್ಧವಾಗುವುದೋ ಆಗ ಅದು ಪಾರದರ್ಶಕವಾಗುವುದು, ಹಿಂದೆ ಇರುವ ಪರಮಾತ್ಮನನ್ನು ಪ್ರತಿಬಿಂಬಿಸುವುದು. ಉರಿಯುತ್ತಿರುವ ದೀಪದ ಗ್ಲಾಸನ್ನು ಶುದ್ಧಮಾಡುವಂತೆ ಇದು. ನಮ್ಮ ಮನಸ್ಸು ಶುದ್ಧಿಯಾಯಿತೆಂದರೆ ಆಧ್ಯಾತ್ಮಿಕ ಸಾಧನೆಯ ಬಹುಪಾಲು ಆದಂತೆ. ಮನಸ್ಸನ್ನು ಶುದ್ಧಮಾಡುವುದು ನಮ್ಮ ಗುರಿ ಆಗಬೇಕೆ ಹೊರತು ಯಾವ ಯಾವುದೊ ಶಕ್ತಿಗಳನ್ನು ಸಂಪಾದಿಸುವುದಾಗಿರಬಾರದು. ಮನುಷ್ಯರು ಹಲವು ದೃಷ್ಟಿಯಿಂದ ಮನ ಸ್ಸನ್ನು ಏಕಾಗ್ರಮಾಡುವರು. ಇದರಿಂದ ದೂರದಲ್ಲಿ ಏನಾಗುತ್ತಿದೆ ಎಂಬುದನ್ನು ತಿಳಿಯಬಹುದು, ಇನ್ನೊಬ್ಬರು ತಮ್ಮ ಮನಸ್ಸಿನಲ್ಲಿ ಏನು ಆಲೋಚನೆ ಮಾಡುತ್ತಿರುವರು ಎಂಬುದನ್ನು ತಿಳಿಯಬಹುದು. ಇದರಿಂದ ಕೆಲವು ವೇಳೆ ರೋಗ ಗುಣಮಾಡಬಹುದು, ಇನ್ನು ಕೆಲವು ಚಮತ್ಕಾರಗಳನ್ನು ಮಾಡಬಹುದು. ಆದರೆ ಧ್ಯಾನಯೋಗದಲ್ಲಿ ಯೋಗಿ ಮನಸ್ಸನ್ನು ಏಕಾಗ್ರಮಾಡುವುದು ಇವುಗಳಾವು ದನ್ನೂ ಪಡೆಯುವುದಕ್ಕಲ್ಲ. ಇದು ಅಸಾಧ್ಯ ಎಂದಲ್ಲ.ಇದಕ್ಕೆ ಬೆಲೆ ಇಲ್ಲ. ಇದು ನಮ್ಮ ಮನಸ್ಸನ್ನು ಪ್ರಪಂಚಕ್ಕೆ ಕಟ್ಟಿಹಾಕುವುದು. ಕೀರ್ತಿ, ಲಾಭ, ಮುಂತಾದುವುಗಳ ಕಡೆ ಹೋಗುವಂತೆ ಮಾಡುವುದು ಎಂದು ಇದನ್ನು ತಿರಸ್ಕರಿಸುವನು. ಅವನು ತನ್ನ ಚಿತ್ತವನ್ನು ಶುದ್ಧಿಮಾಡಿಕೊಳ್ಳುವುದಕ್ಕೆ ಈ ಏಕಾಗ್ರತೆಯನ್ನು ಉಪಯೋಗಿಸಿಕೊಳ್ಳುವನು.

\begin{shloka}
ಸಮಂ ಕಾಯಶಿರೋಗ್ರೀವಂ ಧಾರಯನ್ನಚಲಂ ಸ್ಥಿರಃ~।\\ಸಂಪ್ರೇಕ್ಷ್ಯ ನಾಸಿಕಾಗ್ರಂ ಸ್ವಂ ದಿಶಶ್ಚಾನವಲೋಕಯನ್ \hfill॥ ೧೩~॥
\end{shloka}

\begin{shloka}
ಪ್ರಶಾಂತಾತ್ಮಾ ವಿಗತಭೀರ್ಬ್ರಹ್ಮಚಾರಿವ್ರತೇ ಸ್ಥಿತಃ~।\\ಮನಃ ಸಂಯಮ್ಯ ಮಚ್ಚಿತ್ತೋ ಯುಕ್ತ ಆಸೀತ ಮತ್ಪರಃ \hfill॥ ೧೪~॥
\end{shloka}

\begin{artha}
ಶರೀರ ಶಿರಸ್ಸು, ಕೊರಳು ಇವುಗಳನ್ನು ಸಮವಾಗಿ ಅಚಲವಾಗಿ ಇಟ್ಟು, ಸ್ಥಿರನಾಗಿ ಮೂಗಿನ ತುದಿಯನ್ನು ನೋಡುತ್ತ, ಸುತ್ತಲೂ ನೋಡದೆ ಪ್ರಶಾಂತಾತ್ಮನೂ ನಿರ್ಭಯನೂ ಆಗಿ, ಬ್ರಹ್ಮಚಾರಿವ್ರತದಲ್ಲಿ ಸ್ಥಿರನಾಗಿ ಮನಸ್ಸನ್ನು ಸಂಯಮಮಾಡಿಕೊಂಡು ನನ್ನಲ್ಲಿಯೇ ಚಿತ್ತವುಳ್ಳವನಾಗಿ, ನಾನೇ ಪರ ಎಂದು ಭಾವಿಸಿ ಯುಕ್ತನಾಗಿ ಕುಳಿತಿರಬೇಕು.
\end{artha}

ಧ್ಯಾನಕ್ಕೆ ಕುಳಿತುಕೊಳ್ಳುವಾಗ ಶರೀರ ಶಿರಸ್ಸು ಮತ್ತು ಕೊರಳು ಒಂದು ಸರಳರೇಖೆಯಲ್ಲಿರಬೇಕು. ಬೆನ್ನುಮೂಳೆ ಬಾಗಿರಬಾರದು. ಅದು ನೇರವಾಗಿರಬೇಕು. ದೇಹ ಬಾಗಿದ್ದರೆ ತಮಸ್ಸು ಮನಸ್ಸನ್ನು ಆವರಿಸುವುದು. ಆಗ ಗಹನವಾದ ವಿಷಯಗಳನ್ನು ಚಿಂತಿಸಲು ಆಗುವುದಿಲ್ಲ. ದೇಹ ಬಾಗಿರಬಾರದು ಮತ್ತು ಕಂಬ ಗೋಡೆ ಮುಂತಾದವುಗಳನ್ನು ಒರಗಿಕೊಂಡಿರಬಾರದು. ಆಗ ನಿದ್ರೆ ಬರುವುದು. ಧ್ಯಾನಕ್ಕೆ ನಿದ್ರೆ ದೊಡ್ಡ ಶತ್ರು. ಅದು ಬರದ ರೀತಿಯಲ್ಲಿರಬೇಕು. ದೇಹ ಸ್ಥಿರವಾಗಿರಬೇಕು. ಅಲ್ಲಾಡುತ್ತಿದ್ದರೆ ಚಿತ್ತಚಾಂಚಲ್ಯವನ್ನು ತೋರುವುದು. ದೀರ್ಘಧ್ಯಾನಪರನಾದಂತೆ ಅವನೊಂದು ಶಿಲಾಪ್ರತಿಮೆಯಂತೆ ಆಗುತ್ತಾನೆ. ಆ ಸಮಯದಲ್ಲಿ ಧ್ಯಾನಕ್ಕೆ ಕುಳಿತ ಬುದ್ಧನ ಭಂಗಿಯನ್ನು ಜ್ಞಾಪಿಸಿಕೊಳ್ಳಬೇಕು. ತನ್ನ ಮೂಗಿನ ತುದಿಯನ್ನು ನೋಡುತ್ತಾ ಇರಬೇಕು. ಮನಸ್ಸು ಅಂತರ್ಮುಖವಾದಂತೆ ದೃಷ್ಟಿಯೂ ಕೂಡ ಅಂತರ್ಮುಖವಾಗುವುದು. ದೃಷ್ಟಿಯ ಅಂತರ್ಮುಖತೆಗೆ ಮೂಗಿನ ತುದಿಯನ್ನು ನೋಡುವುದು ಸಹಾಯಮಾಡುವುದು. ಕುಳಿತಾಗ ಹಿಂದೆಮುಂದೆ ತಿರುಗಿ ನೋಡ ಬಾರದು. ಹೊರಗಿನ ಪ್ರಪಂಚವನ್ನು ಮತ್ತು ತನ್ನ ದೇಹವನ್ನು ಮರೆಯಲು ಯತ್ನಿಸಬೇಕು.

ಇವುಗಳೆಲ್ಲ ದೇಹವನ್ನು ಅಣಿಮಾಡಲು ಸಂಬಂಧಿಸಿರುವುದಾಯಿತು. ಇನ್ನು ಮೇಲೆ\break ಮನಸ್ಸನ್ನು ಅಣಿಮಾಡುವುದನ್ನು ಹೇಳುತ್ತಾನೆ. ಪ್ರಶಾಂತಾತ್ಮನಾಗಿರಬೇಕು. ಮನಸ್ಸು ಯಾವ ಉದ್ವೇಗಕ್ಕೂ ಸಿಕ್ಕಿರಕೂಡದು. ಹಾಗೆ ಸಿಕ್ಕಿಕೊಳ್ಳದಂತೆ ನೋಡಿಕೊಳ್ಳಬೇಕು. ಧ್ಯಾನಕ್ಕೆ ಮುಂಚೆ ಜಗಳ ಕಾಯುವುದು, ಕೋಪಗೊಳ್ಳುವುದು ಇವುಗಳನ್ನು ಮಾಡಿದರೆ ಧ್ಯಾನಕ್ಕೆ ಕುಳಿತಾಗ ಅದೇ ಜ್ಞಾಪಕಕ್ಕೆ ಬರುತ್ತದೆ. ಮನಸ್ಸನ್ನು ಧ್ಯಾನಕ್ಕೆ ಸಿದ್ಧಮಾಡುವವನು, ಅದರ ಶಾಂತಿಗೆ ಯಾವ ಭಂಗವೂ ಬರದಂತೆ ನೋಡಿಕೊಳ್ಳಬೇಕು. ಅನಂತರ ಶ‍್ರೀಕೃಷ್ಣ ಹೇಳುವ ಗುಣವೇ ನಿರ್ಭಯತೆ. ಅವನು ಯಾರಿಗೂ ಅಂಜುವು ದಿಲ್ಲ, ಯಾವುದಕ್ಕೂ ಅಂಜುವುದಿಲ್ಲ. ಆತ ಹೊರಗಿನಿಂದ ನೋಡಲು ಸೌಮ್ಯಮೂರ್ತಿ. ಯಾರಿಗೂ ತೊಂದರೆ ಕೊಡುವುದಕ್ಕೆ ಹೋಗುವುದಿಲ್ಲ. ಆದರೆ ಇತರರೂ ಅವನನ್ನು ಅಂಜಿಸಲಾರರು. ಧ್ಯಾನ ಮಾಡುವುದಕ್ಕೆ ಹೇಡಿಗೆ ಸಾಧ್ಯವಿಲ್ಲ. ಅಂಜುಬುರುಕರು ಧ್ಯಾನಮಾಡಲಾರರು. ಯಾವುದಾದರೊ ಒಂದು ಅಂಜಿಕೆಯೇ ಸಾಕು, ಅವರ ಮನಸ್ಸಿನಲ್ಲಿ ಅಶಾಂತಿಯನ್ನು ಎಬ್ಬಿಸಲು. ಧ್ಯಾನಿ ನಿರ್ಭೀತ. ಅಂಜಿಕೆಗೆ ಅಂಜಿಕೆ ಅವನು. ಅವನು ದೇವರನ್ನು ನೆಚ್ಚಿದವನು, ಹುಲುಮನುಜರನ್ನಲ್ಲ. ಯಾವುದೂ ಅವನಿಚ್ಛೆ ಇಲ್ಲದೆ ಆಗಲಾರದು ಎಂಬುದನ್ನು ಚೆನ್ನಾಗಿ ಮನಗಂಡವನು. ದೇವರನ್ನು ನೆಚ್ಚಿ ಹೊರಟವನನ್ನು ಅವನು ಎಂದಿಗೂ ಕೈಬಿಡುವುದಿಲ್ಲ ಎಂಬುದನ್ನು ಚೆನ್ನಾಗಿ ಅರಿತವನು ಅವನು. ಇನ್ನುಮೇಲೆ ಈ ಕ್ಷುದ್ರಪ್ರಪಂಚಕ್ಕೆ ಹೇಗೆ ಅಂಜುತ್ತಾನೆ?

ಯೋಗಿ, ಬ್ರಹ್ಮಚಾರಿ ವ್ರತದಲ್ಲಿ ಸ್ಥಿರನಾಗಿರುವನು. ಅವನು ಜಿತೇಂದ್ರಿಯನಾಗಿರಬೇಕು. ದೇವರ ಕಡೆ ಹೋಗಬೇಕಾದರೆ ಮನಸ್ಸೆಲ್ಲ ಅವನ ಕಡೆ ಹರಿಸಬೇಕಾಗುವುದು. ಈ ಪ್ರಪಂಚವನ್ನು ಈಗ ರುಚಿ ನೋಡುವುದು, ಅನಂತರ ದೇವರನ್ನು ಕುರಿತು ಚಿಂತಿಸುವುದು ಒಟ್ಟಿಗೆ ಹೋಗಲಾರದು. ದೇವರು ಬೇಕಾದರೆ ಇಂದ್ರಿಯ ಚಪಲವನ್ನು ಬಿಗಿ ಹಿಡಿಯಬೇಕು. ಬ್ರಹ್ಮಚರ್ಯ ಎಂದರೆ ಯಾವುದೋ ಒಂದು ಇಂದ್ರಿಯವನ್ನು ನಿಗ್ರಹಿಸುವುದು ಮಾತ್ರವಲ್ಲ. ಒಂದು\break ಇಂದ್ರಿಯಕ್ಕೂ ಮತ್ತೊಂದು ಇಂದ್ರಿಯಕ್ಕೂ ಸಂಬಂಧವಿದೆ. ಒಂದನ್ನು ನಿಗ್ರಹಿಸಬೇಕಾದರೆ ಉಳಿದ ಇಂದ್ರಿಯಗಳ ಸೆಳೆತದಿಂದಲೂ ಅವನು ಪಾರಾಗಿರಬೇಕು. ಕೆಲವು ವೇಳೆ ಒಂದು ವಸ್ತುವನ್ನು ಮುಟ್ಟುವುದಿಲ್ಲ ಎಂದು ಹೇಳುವೆವು. ಆದರೆ ಅದನ್ನು ನೋಡುತ್ತೇವೆ, ಅದರೊಡನೆ ಮಾತಾಡುತ್ತೇವೆ, ಅದರ ಸಮೀಪದಲ್ಲಿರುತ್ತೇವೆ, ಇವುಗಳನ್ನೆಲ್ಲ ಮಾಡುತ್ತಿದ್ದರೆ ಬಿಟ್ಟ, ಮುಟ್ಟದೇ ಇರುವುದನ್ನು ಕೂಡ ಮಾಡಿಬಿಡುತ್ತೇವೆ. ಪಂಚೇಂದ್ರಿಯಗಳಲ್ಲಿ ಒಂದರಷ್ಟೇ ಮತ್ತೊಂದು ವಿಪತ್ತನ್ನು ತರುವುದು. ಒಂದು ಮತ್ತೊಂದಕ್ಕೆ ಸಹಾಯಮಾಡುವುದು ಎಂದು ಅರಿತು ಅದರ ಬಲೆಗೆ ಬೀಳದೆ ಇರಬೇಕು. ಮನಸ್ಸನ್ನು ಪ್ರಪಂಚದಿಂದ ಸೆಳೆದು ಅದನ್ನು ಭಗವಂತನ ಕಡೆ ಅದೇ ಪರ, ಎಂದರೆ, ಶ್ರೇಷ್ಠ ಎಂದು ಬಿಡಬೇಕು. ತಿಳಿದುಕೊಳ್ಳುವುದಕ್ಕೆ, ಪ್ರೀತಿಸುವುದಕ್ಕೆ, ಆನಂದಿಸುವುದಕ್ಕೆ, ನಮ್ಮದೆಂದು ಕರೆದುಕೊಳ್ಳುವುದಕ್ಕೆ ದೇವರಷ್ಟು ಪವಿತ್ರವಾದುದು ಮತ್ತೊಂದು ಇಲ್ಲ ಎಂದು ಅವನಲ್ಲೆ ಮನಸ್ಸನ್ನು ಮುಳುಗಿಸಬೇಕು. ಇದು ಒಂದು ಸ್ಪಂಜನ್ನು ನೀರಿನಲ್ಲಿ ಅದ್ದಿದಂತೆ. ಧ್ಯೇಯವಸ್ತುವೇ ಅಖಂಡ ಸಚ್ಚಿದಾನಂದ ಸ್ವರೂಪನಾದ ಭಗವಂತ. ಅವನನ್ನು ಧ್ಯಾನಿಸುವುದೆಂದರೆ ಅದರಲ್ಲಿ ಮನಸ್ಸನ್ನು ಅದ್ದುವುದು. ಆಗ ಮನಸ್ಸಿನ ಪ್ರತಿಯೊಂದು ಭಾಗವೂ ಅದನ್ನು ಹೀರುವುದು. ಧ್ಯಾನಸಮಯದಲ್ಲಿ ಮನಸ್ಸು ದೇವರಿಂದ ತುಂಬಿ ತುಳುಕಾಡುವಂತೆ ಮಾಡಬೇಕು.

\begin{shloka}
ಯುಂಜನ್ನೇವಂ ಸದಾತ್ಮಾನಂ ಯೋಗೀ ನಿಯತಮಾನಸಃ~।\\ಶಾಂತಿಂ ನಿರ್ವಾಣಪರಮಾಂ ಮತ್ಸಂಸ್ಥಾಮಧಿಗಚ್ಛತಿ \hfill॥ ೧೫~॥
\end{shloka}

\begin{artha}
ನಿಯತಮಾನಸನಾಗಿ ಹೀಗೆ ಯಾವಾಗಲೂ ಮನಸ್ಸನ್ನು ಯೋಗದಲ್ಲಿಟ್ಟುಕೊಳ್ಳುವ ಯೋಗಿ ನಿರ್ವಾಣ\-ಪರಮವಾದ ನನ್ನಲ್ಲಿರುವ ಶಾಂತಿಯನ್ನು ಪಡೆಯುವನು.
\end{artha}

\newpage

ಯೋಗಿ ಮನಸ್ಸನ್ನು ನಿಗ್ರಹಿಸಿ, ಯಾವಾಗಲೂ ಅದನ್ನು ಯೋಗದಲ್ಲಿಟ್ಟಿರಬೇಕು. ಧ್ಯಾನ\-ಮಾಡುವಾಗ ಮಾತ್ರ ದೇವರಮೇಲೆ ಮನಸ್ಸಿಟ್ಟು ಇತರ ಕಾಲದಲ್ಲಿ ಅದನ್ನು ಪ್ರಪಂಚದ ರೊಚ್ಚಿನ\-ಮೇಲೆ ಬಿಡುವುದಲ್ಲ. ಧ್ಯಾನ ಸರಿಯಾದ ಧ್ಯಾನ ಆಗಬೇಕಾದರೆ ಇತರ ಕಾಲದಲ್ಲಿ ಮನಸ್ಸು ಯಾವಾಗಲೂ ಅದಕ್ಕೆ ಅಣಿಯಾಗುತ್ತಿರಬೇಕು. ಮನಸ್ಸನ್ನು ಯಾವಾಗಲೂ ದೇವರ ಕಡೆ ಕಳುಹಿ\-ಸುತ್ತಿರಬೇಕು. ಸ್ವಲ್ಪ ಹೊತ್ತು ಕಳುಹಿಸಿ ಅನಂತರ ಪ್ರಪಂಚದ ಕಡೆ ಅಟ್ಟಿದರೆ ಅದು ಏನನ್ನು ಪಡೆದುಕೊಂಡಿರುವುದೊ ಅದನ್ನು ಹಾಳುಮಾಡಿಕೊಳ್ಳುವುದು. ಬಿಡದೇ ಅಭ್ಯಾಸ ಮಾಡುತ್ತಿರಬೇಕು. ಈ ಹೋರಾಟಕ್ಕೆ ರಜಾ ಇಲ್ಲ. ನಾವು ದೇವರೊಡನೆ ಒಂದಾಗುವವರೆಗೆ ಈ ಸಾಧನೆ ಬಿಡುವಿಲ್ಲದೆ ಆಗುತ್ತಿರಬೇಕು. ನಾವು ಯಾವುದಾದರೂ ಕಾರಣದಿಂದ ಕೆಲವು ಕಾಲ ಇದನ್ನು ಬಿಟ್ಟರೆ ಪುನಃ ಪ್ರಾರಂಭಿಸಲು ಹಿಂದಿಗಿಂತ ಹೆಚ್ಚು ಶಕ್ತಿ ಸಂಕಲ್ಪಗಳು ಬೇಕಾಗುವುದು. ಒಂದು ಹೊತ್ತಿಗೆ ಎದ್ದು ಧ್ಯಾನ ಮಾಡುತ್ತಿರುವವನು ಆಲಸ್ಯದಿಂದಲೊ ನಿದ್ರೆಯಿಂದಲೊ ಒಂದು ದಿನ ಬಿಟ್ಟ ಎಂದರೆ ಮಾರನೆ ದಿನವೂ ಆಲಸ್ಯ ತಲೆದೋರುವುದು. ಆ ತಮಸ್ಸಿನಿಂದ ಮನಸ್ಸನ್ನು ಏಳಿಸಬೇಕಾದರೆ ಪುನಃ ಕಷ್ಟಪಡಬೇಕಾಗುವುದು. ಮನಸ್ಸಿನ ಒಂದು ಭಾಗ ಯಾವಾಗಲೂ ದೇವರ ಕಡೆ ತಿರುಗಿರಬೇಕು. ನಾವು ಯಾವ ಕೆಲಸವನ್ನು ಮಾಡುತ್ತಿದ್ದರೂ ಮಧ್ಯೆಮಧ್ಯೆ ಭಗವಂತನಿಗೆ ಸಂಬಂಧಪಟ್ಟ ಭಾವನೆಗಳನ್ನು ಮೆಲ್ಲುತ್ತಿರಬೇಕು. ಶ‍್ರೀರಾಮಕೃಷ್ಣರು ಇದಕ್ಕೆ ಹಲ್ಲುನೋವಿನ ಉದಾಹರಣೆಯನ್ನು ಕೊಡುತ್ತಿದ್ದರು. ಹಲ್ಲಿನ ನೋವಿನಿಂದ ನರಳುತ್ತಿರುವವನು, ಯಾವ ಕೆಲಸಗಳನ್ನು ಮಾಡುತ್ತಿದ್ದರೂ ಮನಸ್ಸಿನ ಒಂದು ಭಾಗ ಆ ನೋವನ್ನೇ ಚಿಂತಿಸುತ್ತಿರುತ್ತದೆ. ಹಾಗೆಯೇ ಮನಸ್ಸು ಭಗವಂತನನ್ನು ಇತರ ಸಮಯದಲ್ಲಿ ಚಿಂತಿಸುತ್ತಿರಬೇಕು. ಧ್ಯಾನಕ್ಕೆ ಕುಳಿತಾಗ, ಅವನಲ್ಲೆ ತನ್ಮಯವಾಗಬೇಕು.\break ದಿಕ್​ಸೂಚಿ ಎಂದರೆ ಉತ್ತರಮುಖಿ ಯಾವಾಗಲೂ ಹೇಗೆ ಉತ್ತರದಿಕ್ಕನ್ನು ತೋರಿಸುವುದೋ ಹಾಗೆ ಮನಸ್ಸು ಯಾವಾ ಗಲೂ ದೇವರ ಕಡೆ ತಿರುಗಿರಬೇಕು. ಇತರ ಕಾಲದಲ್ಲಿ ಅವನು ಹಲವು ವ್ಯವಹಾರಗಳಲ್ಲಿ ಮಗ್ನನಾಗಿದ್ದರೂ, ಅವನ ಮನಸ್ಸಿನ ಒಂದು ಭಾಗ ಮಾತ್ರ ಯಾವಾಗಲೂ ದೇವರ ಕಡೆಗೆ ತಿರುಗಿರಬೇಕು.

ಯಾರ ಮನಸ್ಸು ಹೀಗೆ ಸದಾ ಭಗವಂತನ ಕಡೆ ತಿರುಗಿರುವುದೋ, ಧ್ಯಾನಿಸುವಾಗ, ಅವನಲ್ಲೇ ತನ್ಮಯವಾಗುವುದೊ ಅಂತಹ ಮನಸ್ಸು ಶ್ರೇಷ್ಠವಾದ ನಿರ್ವಾಣಶಾಂತಿಯನ್ನು ಪಡೆಯುವುದು. ಪುನರ್ಜನ್ಮವಿಲ್ಲದ ಭಗವಂತನನ್ನು ರುಚಿ ನೋಡಿದ ಶ್ರೇಷ್ಠವಾದ ಶಾಂತಿಯನ್ನು ಪಡೆಯುತ್ತಾನೆ. ಇಲ್ಲಿಯೇ ಅದನ್ನು ಅವನು ಅನುಭವಿಸುತ್ತಾನೆ. ಹೇಗೆ ಮೂಸೆಯಲ್ಲಿ ಒಂದು ವಸ್ತುವನ್ನು ಇಟ್ಟು ಕಾಯಿಸಿದರೆ ಅದು ಕೆಂಗೆಂಡದಂತಾಗುವುದೊ ಅದರಂತೆಯೆ ಧ್ಯಾನವೆಂದರೆ ಮನಸ್ಸನ್ನು ಭಗವಂತ ನೆಂಬ ಮೂಸೆಯಲ್ಲಿಟ್ಟು ಕಾಯಿಸುವುದು. ಧ್ಯಾನಿಸುವವನಿಗೂ ಧ್ಯೇಯವಸ್ತುವಿನ ಧರ್ಮಗಳು ಬರುತ್ತವೆ. ಧ್ಯಾನಮಾಡುವವನು ಭಗವಂತನ ಸಚ್ಚಿದಾನಂದಸ್ವರೂಪವನ್ನು ಹೀರಿಕೊಳ್ಳುತ್ತಾನೆ. ಇದರಿಂದ ಪರಮಶಾಂತಿಯನ್ನು ಪಡೆಯುತ್ತಾನೆ. ಇದು ಭಂಗವಾಗದ ಶಾಂತಿ, ಎಂದೆಂದಿಗೂ ನಮ್ಮಲ್ಲಿರುವ ಶಾಂತಿ, ನಮ್ಮ ಸಂಸ್ಕಾರಗಳನ್ನು ನಿರ್ನಾಮಮಾಡಿ ಭಗವಂತನೊಡನೆ ನಮ್ಮನ್ನು ಸೇರಿಸುವ ಶಾಂತಿ.

\begin{shloka}
ನಾತ್ಯಶ್ನತಸ್ತು ಯೋಗೋಽಸ್ತಿ ನ ಚೈಕಾಂತಮನಶ್ನತಃ~।\\ನ ಚಾತಿಸ್ವಪ್ನಶೀಲಸ್ಯ ಜಾಗ್ರತೋ ನೈವ ಚಾರ್ಜುನ \hfill॥ ೧೬~॥
\end{shloka}

\begin{artha}
ಅರ್ಜುನ, ಯೋಗ ಮಿತಿಮೀರಿ ತಿನ್ನುವವನಿಗೂ ಇಲ್ಲ, ಏನೂ ತಿನ್ನದೇ ಇರುವವನಿಗೂ ಇಲ್ಲ; ಅತಿಯಾಗಿ ನಿದ್ರೆ ಮಾಡುವವನಿಗೂ ಇಲ್ಲ, ಏನೂ ನಿದ್ರೆ ಮಾಡದವನಿಗೂ ಇಲ್ಲ.
\end{artha}

ಒಂದು ಸಮಯದಲ್ಲಿ ಮನಸ್ಸು ದೇಹಕ್ಕೆ ಅಂಟಿಕೊಂಡಿರುವಂತೆ ಕಾಣುವುದು, ಎಳನೀರಿನಲ್ಲಿ ಕಾಯಿ ಕರಟಕ್ಕೆ ಅಂಟಿಕೊಂಡಿರುವಂತೆ. ಆದರೆ ನೀರು ಇಂಗಿ ಕಾಯಿ ಕೊಬ್ಬರಿ ಆದರೆ ಅದು ಕರಟದ ಒಳಗೆ ಇದ್ದರೂ ಅದು ಬೇರೆಯಾಗಿ ಅಲ್ಲಾಡುತ್ತಿರುವುದು. ಅದರಂತೆಯೇ ಮುಕ್ತಾವಸ್ಥೆಗೆ ಬಂದ ಮನುಷ್ಯ. ಆದರೆ ಯೋಗಜೀವನ ಇನ್ನೂ ಪ್ರಾರಂಭದ ಸ್ಥಿತಿಯಲ್ಲಿರುವವನಿಗೆ ದೇಹ ಮತ್ತು ಮನಸ್ಸಿಗೆ ಒಂದು ಸಂಬಂಧ ಇರುವುದು. ಯಾವುದಾದರೂ ಒಂದರ ಮೇಲೆ ಬದಲಾವಣೆ ಆದರೆ ಅದು ಇನ್ನೊಂದರ ಮೇಲೆ ತನ್ನ ಪರಿಣಾಮವನ್ನು ಬೀರುವುದು. ಈ ಸಮಯದಲ್ಲಿ ಒಬ್ಬ ಜೋಪಾನವಾಗಿರಬೇಕು. ಸಾಧ್ಯವಾದಷ್ಟು ದೇಹ ಮನಸ್ಸಿನೊಂದಿಗೆ ಸಹಕರಿಸಬೇಕು. ದೇಹ ಮನ ಸ್ಸನ್ನೇ ತನ್ನ ಕಡೆಗೆ ಸೆಳೆಯಕೂಡದು. ಯಾರು ಮಿತಿಮೀರಿ ಊಟ ಮಾಡುವರೋ ಅವರು ತಮೋಗುಣಿಗಳಾಗುವರು. ಅದನ್ನು ಅರಗಿಸುವುದಕ್ಕಾಗಿಯೆ ದೇಹದ ರಕ್ತದ ಬಹುಭಾಗವೆಲ್ಲ ಹೋಗುವುದು. ಇನ್ನು ಆಲೋಚಿಸುವುದಕ್ಕೆ ಉಳಿಯುವುದು ಬಹಳ ಕಡಮೆ. ಮನಸ್ಸು ದೇಹದ ಮೇಲೆ ಹೆಚ್ಚು ವಾಲುತ್ತ ಬರುವುದು. ದೇಹವನ್ನು ದೃಢವಾಗಿ ಮತ್ತು ಆರೋಗ್ಯವಾಗಿ ಇಡುವುದಕ್ಕೆ ಎಷ್ಟು ಬೇಕೊ ಅಷ್ಟನ್ನು ಮಾತ್ರ ತಿನ್ನಬೇಕು. ಯೋಗಿಯಾಗುವುದು ಎಂದರೆ ದೊಡ್ಡ ವ್ಯಕ್ತಿಯಾಗುವುದಲ್ಲ. ಯೋಗಜೀವನಕ್ಕೆ ಲಘುವಾದ ದೃಢವಾದ ದೇಹ ಬೇಕು. ಅಂತಹ ದೇಹವನ್ನು ಪೋಷಿಸುವುದಕ್ಕೆ ಬೇಕಾದಷ್ಟು ಮಾತ್ರ ತಿನ್ನಬೇಕು. ತಿನ್ನುವುದು ಯಾವಾಗಲೂ ಮಿತವಾಗಿರಬೇಕು. ಇನ್ನೂ ಬೇಕೆಂದಿರುವಾಗಲೇ ಏಳಬೇಕು.

ಅದರಂತೆಯೇ ಯಾರು ಬಹಳ ಕಡಮೆ ಊಟಮಾಡುವರೋ ಅವರಿಗೂ ಯೋಗಜೀವನ ಸಾಧ್ಯವಿಲ್ಲ. ಅವರ ದೇಹ ದುರ್ಬಲವಾಗುವುದು. ದೇಹ ಆರೋಗ್ಯದಿಂದ ಬಲವಾಗಿದ್ದಾಗಲೆ ಕುಳಿತುಕೊಂಡು ಧ್ಯಾನ ಮಾಡುವುದು ಕಷ್ಟ. ಹಾಗಿರುವಾಗ ದುರ್ಬಲ ದೇಹ ಕಟ್ಟಿಕೊಂಡು ಯಾರೂ ಧ್ಯಾನಮಾಡಲಾರರು. ದೇಹ ದುರ್ಬಲವಾದರೆ ಮನಸ್ಸು ದುರ್ಬಲವಾಗುವುದು. ದುರ್ಬಲವಾದ ಮನಸ್ಸಿಗೆ ಧ್ಯಾನ ಸಾಧ್ಯವಿಲ್ಲ. ನೀರಿನ ವೇಗ ಹೆಚ್ಚಾಗಿದ್ದರೆ ಮಾತ್ರ ಅದರಿಂದ ಉತ್ಪತ್ತಿಯಾಗುವ ವಿದ್ಯುಚ್ಛಕ್ತಿ ಕೂಡ ಹೆಚ್ಚಾಗಿರುವುದು. ನೀರಿನ ವೇಗದಂತೆ ಬಲವಾದ ಆಲೋಚನೆ. ಅದು ದೇವರ ಕಡೆ ಎಷ್ಟು ರಭಸದಿಂದ ಹರಿಯುವುದೋ ಅಷ್ಟು ಹೆಚ್ಚು ಮಾನಸಿಕ ಶಕ್ತಿಯನ್ನು ಉತ್ಪತ್ತಿ ಮಾಡುವುದು. ಆದಕಾರಣ ದೇಹವನ್ನು ಉಪವಾಸ ಮತ್ತು ಅಲ್ಪಾಹಾರ ಮುಂತಾದುವುಗಳಿಂದ ದಂಡಿಸಬಾರದು. ಇದರಿಂದ ಯಾವ ಮಾನಸಿಕ ಪ್ರಯೋಜನವೂ ಆಗುವುದಿಲ್ಲ. ಹೆಚ್ಚು ತಿನ್ನುವುದು ಯೋಗಜೀವನಕ್ಕೆ ಎಷ್ಟು ಹಾನಿಯೋ ಕಡಮೆ ತಿನ್ನುವುದು ಅಷ್ಟೇ ದೊಡ್ಡ ಹಾನಿ. ಮಿತಾಹಾರಿಯಾಗಿರಬೇಕು.

ನಿದ್ರೆಯಲ್ಲಿಯೂ ಮಿತಿ ಇರಬೇಕು. ದೇಹ ಆರೋಗ್ಯವಾಗಿರಬೇಕಾದರೆ, ಅದಕ್ಕೆ ಕೆಲವು ಗಂಟೆ ನಿದ್ರೆ ಆವಶ್ಯಕ. ಆಗಲೇ ದೇಹಕ್ಕೆ ವಿರಾಮ. ದೇಹದಲ್ಲಿ ಅನೇಕ ರಿಪೇರಿ ಕೆಲಸಗಳು ಆಗಲೇ ಆಗುವುದು. ಆದರೆ ಅತೀ ನಿದ್ರೆ ಮಾಡಿದರೆ ಆ ಮನುಷ್ಯ ಸೋಮಾರಿ ಆಗುವನು, ಮನಸ್ಸು ಆಲಸಿಯಾಗುವುದು. ಧ್ಯಾನಕ್ಕೆ ಕುಳಿತುಕೊಂಡಿತು ಎಂದರೆ ನಿದ್ರಾದೇವಿ ಆಗಲೆ ಅವನ ಮೇಲೆ ಸವಾರಿ ಮಾಡುವಳು. ಧ್ಯಾನಕ್ಕೆ ಅತ್ಯಂತ ಆವಶ್ಯಕವಾಗಿರುವುದೇ ತುಂಬಾ ಜಾಗೃತವಾದ ಮನಸ್ಸು. ನಿದ್ರೆ ಬಂದರೆ ಇದು ಸಿದ್ಧಿಸುವು ದಿಲ್ಲ. ಆದಕಾರಣವೇ ಊಟದಲ್ಲಿ ಹೇಗೆ ಒಂದು ಮಿತಿ ಇರಬೇಕೊ ಹಾಗೆ ನಿದ್ರೆಯಲ್ಲಿಯೂ ಒಂದು ಮಿತಿ ಇರಬೇಕು. ಕಡಮೆ ನಿದ್ರೆ ಮಾಡುವವನಿಗೂ ಯೋಗಜೀವನ ಸಾಧ್ಯವಿಲ್ಲ. ಅವನ ಮನಸ್ಸಿಗೆ ಒಂದು ಸ್ವಾಸ್ಥ್ಯವಿರುವುದಿಲ್ಲ. ದೇಹಕ್ಕೆ ಸಾಕಷ್ಟು ವಿರಾಮ ಸಿಕ್ಕದೆ ಇದ್ದರೆ ಮನಸ್ಸು ಕೂಡ ಸರಿಯಾಗಿ ಕೆಲಸಮಾಡುವುದಿಲ್ಲ. ಕೆಲವು ವೇಳೆ ಶಿವರಾತ್ರಿ ಮುಂತಾದುವುಗಳನ್ನು ಮಾಡುವಾಗ ರಾತ್ರಿಯಲ್ಲಿ ಜಾಗರಣೆ ಮಾಡುವರು. ಮಾರನೆ ದಿನ ಕುಳಿತ ಕಡೆ ನಿಂತ ಕಡೆ ತೂಕಡಿಸುತ್ತಿರುವುದನ್ನು ನೋಡುವೆವು. ಇನ್ನು ಇಂತಹ ಸ್ಥಿತಿಯಲ್ಲಿ ಧ್ಯಾನ ಮಾಡುವುದೇನು? ಧ್ಯಾನ ಮಾಡುವುದು ಎಂದರೆ ಹಾಡುವುದಕ್ಕೆ ತಂಬೂರಿಯನ್ನು ಶ್ರುತಿ ಮಾಡಿದಂತೆ. ತಂತಿ ಅಳ್ಳಕವಾಗಿದ್ದರೆ ಚೆನ್ನಾಗಿ ಶ್ರುತಿ ಹೊರಡುವುದಿಲ್ಲ. ತುಂಬ ಬಿಗಿಯಾದರೆ ತಂತಿ ಕಿತ್ತುಹೋಗುವ ಅಪಾಯವಿದೆ. ಅದು ಮಧ್ಯದಲ್ಲಿರಬೇಕು. ಆಗ ಮಾತ್ರ ಒಳ್ಳೆಯ ಶ್ರುತಿ ಕೊಡುವುದು.

\begin{shloka}
ಯುಕ್ತಾಹಾರವಿಹಾರಸ್ಯ ಯುಕ್ತಚೇಷ್ಟಸ್ಯ ಕರ್ಮಸು~।\\ಯುಕ್ತಸ್ವಪ್ನಾವಬೋಧಸ್ಯ ಯೋಗೋ ಭವತಿ ದುಃಖಹಾ \hfill॥ ೧೭~॥
\end{shloka}

\begin{artha}
ಯುಕ್ತಾಹಾರ ವಿಹಾರಿಯಾಗಿ, ಕರ್ಮಗಳಲ್ಲಿ ಮಿತವಾಗಿ, ನಿದ್ರೆ ಎಚ್ಚರಗಳಲ್ಲಿ ಮಿತವಾಗಿ ಇರುವವನಿಗೆ ಯೋಗ ದುಃಖನಾಶಕವಾಗುವುದು.
\end{artha}

ತಬಲವನ್ನು ಬಾರಿಸುವುದಕ್ಕೆ ಮುಂಚೆ ಅದನ್ನು ಮೊದಲು ಶ್ರುತಿಗೆ ತರಬೇಕು. ಹಾಗೆಯೇ ಧ್ಯಾನಮಾಡುವುದಕ್ಕೆ ಮುಂಚೆ ದೇಹ ಇಂದ್ರಿಯ ಮನಸ್ಸು ಇವುಗಳನ್ನು ಶ್ರುತಿಗೆ ತರಬೇಕಾಗಿದೆ. ಒಂದೊಂದು ಕೆಲಸ ಮಾಡುವುದಕ್ಕೆ ಅದನ್ನು ಒಂದೊಂದು ಶ್ರುತಿಗೆ ತರಬೇಕು. ಧ್ಯಾನ ಮಾಡುವುದು ಎಂದರೆ ಅದು ಹೊರಗೆ ಪ್ರಪಂಚದಲ್ಲಿ ಕಾರಕೂನ, ಕೂಲಿಯಾಳು, ಉಪಾಧ್ಯಾಯನು ಮಾಡುವಂತಹ ಕೆಲಸವಲ್ಲ. ಇದೊಂದು ಬಹಳ ಸೂಕ್ಷ್ಮವಾದ ಮಾನಸಿಕ ಕ್ರಿಯೆ. ದೇಹ ಇಂದ್ರಿಯ ಮನಸ್ಸು ಬುದ್ಧಿ ಇವುಗಳೆಲ್ಲವು ಅದಕ್ಕೆ ಸಹಕರಿಸಬೇಕು. ಆಗ ಮಾತ್ರ ಸರಿಯಾದ ಧ್ಯಾನ ಸಾಧ್ಯ.

ಯುಕ್ತ ಎಂದರೆ ಮಿತವಾದ ಆಹಾರ ತೆಗೆದುಕೊಳ್ಳಬೇಕು. ತುಂಬ ಹೆಚ್ಚಾಗಿಯೂ ಇರಕೂಡದು, ತುಂಬ ಕಡಮೆಯಾಗಿಯೂ ಇರಕೂಡದು. ಅದರಂತೆಯೇ ವಿಹಾರ. ಮನಸ್ಸಿಗೆ ಸ್ವಲ್ಪ ವಿರಾಮ ಸಿಕ್ಕಬೇಕು. ಅದಕ್ಕಾಗಿ ಯಾವುದಾದರೂ ವಿಹಾರ ಇರಬೇಕು. ಎಂದರೆ ಚೇಷ್ಟೆ ತಮಾಷೆ ಇವುಗಳಲ್ಲಿ ಮುಳುಗಿರಬೇಕು ಎಂದಲ್ಲ. ಅನೇಕ ವೇಳೆ ಕೆಲಸದ ಬದಲಾವಣೆಯೆ ಒಂದು ವಿಹಾರವಾಗುವುದು. ಅಂತೂ ನಮ್ಮ ವಿಹಾರ, ಧ್ಯೇಯವಸ್ತುವಿನಿಂದ ತುಂಬ ದೂರಕ್ಕೆ ಕರೆದುಕೊಂಡು ಹೋಗಕೂಡದು. ಅದರ ಹತ್ತಿರವೆ ತಾರಾಡುವಂತಿರಬೇಕು ವಿಹಾರ.

ಅದರಂತೆಯೆ ಯೋಗಿಯಾದವನು ಮಿತವಾಗಿ ಕರ್ಮವನ್ನು ಮಾಡಬೇಕು. ಹೆಚ್ಚು ಕರ್ಮದಲ್ಲಿ ತೊಡಗಿದರೆ ಮನಸ್ಸು ಬಾಹ್ಯಮುಖವಾಗುವುದು. ಧ್ಯಾನಕ್ಕೆ ಕುಳಿತಾಗ ಆ ಕರ್ಮವನ್ನೆ ಕುರಿತು ಚಿಂತಿಸುತ್ತಿರುವುದು. ಹೆಚ್ಚು ಕರ್ಮಮಾಡಿ ದೇಹಕ್ಕೆ ದಣಿವಾದಾಗ ಧ್ಯಾನ ಮುಂತಾದ ಆಧ್ಯಾತ್ಮಿಕ ಸಾಧನೆಗಳನ್ನು ಮಾಡಲು ಆಗುವುದಿಲ್ಲ. ಹಾಗೆಯೇ ಕರ್ಮವನ್ನು ಬಹಳ ಕಡಮೆ ಮಾಡಿದರೂ ದೇಹ ಸೋಮಾರಿಯಾಗುವುದು, ಲವಲವಿಕೆಯಿಂದ ಇರುವುದಿಲ್ಲ. ಅದನ್ನು ಒಂದು ಮಿತದಲ್ಲಿ ಇಟ್ಟುಕೊಂಡಿರಬೇಕು. ಯಾರು ನಿದ್ರೆ ಎಚ್ಚರ ಇವುಗಳಲ್ಲಿಯೂ ಒಂದು ಮಿತವನ್ನು ಇಟ್ಟುಕೊಂಡಿರುವರೊ ಅಂತಹ ವ್ಯಕ್ತಿಗೆ ಧ್ಯಾನಯೋಗ ದುಃಖವನ್ನು ನಾಶಮಾಡುವುದು. ಇಲ್ಲಿ ದುಃಖ ಎಂದರೆ ಅಜ್ಞಾನದಲ್ಲಿ ಸಿಕ್ಕಿ ನರಳುವುದು. ವಾಸನೆಗಳನ್ನು ಅನುಸರಿಸಿ ಅವುಗಳನ್ನು ನಿಗ್ರಹಿಸದೆ ಮತ್ತು ಮಾಯೆಯ ಆಳ ಆಳಕ್ಕೆ ಹೋಗುವುದು. ಯೋಗ ಸಂಸಾರದಿಂದ ನಮ್ಮನ್ನು ಮೇಲಕ್ಕೆ ತೇಲಿಸುವುದು. ಇದೊಂದು ನೀರಿನಲ್ಲಿ ಬಿದ್ದವನಿಗೆ ಸಿಕ್ಕಿದ ಸೋರೆಬುರುಡೆಯಂತೆ, ಅವನನ್ನು ಮುಳುಗಿಸದೆ ತೇಲಿಸುವುದು. ಹೆಚ್ಚು ಸಂಸ್ಕಾರಗಳನ್ನು ಉತ್ಪತ್ತಿ ಮಾಡಿಕೊಳ್ಳುವುದನ್ನು ಮಾತ್ರ ನಿಲ್ಲಿಸುವುದಲ್ಲ, ಆಗಲೆ ಇರುವ ಸಂಸ್ಕಾರವನ್ನು ನಾಶಮಾಡುತ್ತ ಬರುವುದು.

\begin{shloka}
ಯದಾ ವಿನಿಯತಂ ಚಿತ್ತಮಾತ್ಮನ್ಯೇವಾವತಿಷ್ಠತೇ~।\\ನಿಃಸ್ಪೃಹಃ ಸರ್ವಕಾಮೇಭ್ಯೋ ಯುಕ್ತ ಇತ್ಯುಚ್ಯತೇ ತದಾ \hfill॥ ೧೮~॥
\end{shloka}

\begin{artha}
ಯಾವಾಗ ನಿಗ್ರಹಿಸಲ್ಪಟ್ಟ ಚಿತ್ತ ಆತ್ಮನಲ್ಲಿಯೇ ನಿಲ್ಲುವುದೊ, ಆಗ ಸರ್ವಕಾಮಗಳಲ್ಲಿ ಬಯಕೆ ಇಲ್ಲದವನಾಗಿ ಯೋಗಿ ಎನಿಸುವನು.
\end{artha}

ಪ್ರಪಂಚದ ಕಡೆ ಹೋಗುತ್ತಿದ್ದ ಮನಸ್ಸನ್ನು ಭಗವಂತನ ಕಡೆ ಹರಿಸಿ ಅವನ ಪಾದಪದ್ಮಗಳಲ್ಲಿ ಕಟ್ಟುವುದೇ ಧ್ಯಾನ. ಮನಸ್ಸು ಅಲ್ಲಿಯೇ ನಿಲ್ಲಬೇಕು. ಆಗಲೆ ಅದರಲ್ಲಿರುವ ಕಲ್ಮಶಗಳೆಲ್ಲ ನಾಶವಾಗಬೇಕಾದರೆ. ಹೇಗೆ ಒಂದು ಪಾತ್ರೆಯನ್ನು ಬೆಂಕಿಯಮೇಲೆ ಇಟ್ಟು ಕಾಯಿಸಿದರೆ ಅದರಲ್ಲಿ ಇರುವ ಸೂಕ್ಷ್ಮ ಕ್ರಿಮಿಗಳೆಲ್ಲ ಧ್ವಂಸವಾಗುವುವೊ ಹಾಗೆ ಮನಸ್ಸಿನಲ್ಲಿರುವ ಕಲ್ಮಶಗಳೆಲ್ಲವೂ ಧ್ಯಾನಾಗ್ನಿಯಲ್ಲಿ ನಾಶವಾಗುವುವು. ಹೆಚ್ಚುಹೆಚ್ಚು ಭಗವಂತನಲ್ಲಿ ಮನಸ್ಸು ತನ್ಮಯವಾಗುತ್ತ ಬಂದಂತೆ ಅಷ್ಟೂ ಅಷ್ಟೂ ಪರಿಶುದ್ಧವಾಗುತ್ತ ಬರುವುದು. ಪರಮಾತ್ಮನ ಕಡೆ ಮನಸ್ಸು ಹೋಗುವಾಗ ಅದು ಅಖಂಡವಾಗಿ ಹರಿದುಕೊಂಡು ಹೋಗಬೇಕು. ಒಂದು ಪಾತ್ರೆಯಿಂದ ಎಣ್ಣೆಯನ್ನು ಮತ್ತೊಂದು ಪಾತ್ರೆಗೆ ಸುರಿದರೆ ಹೇಗೆ ಒಂದು ದಾರದಂತೆ ಹರಿಯುವುದೊ ಹಾಗೆ ಮನಸ್ಸು ದೇವರ ಕಡೆಗೆ ಹೋಗಬೇಕು.

\begin{shloka}
ಯಥಾ ದೀಪೋ ನಿವಾತಸ್ಥೋ ನೇಂಗತೇ ಸೋಪಮಾ ಸ್ಮೃತಾ~।\\ಯೋಗಿನೋ ಯತಚಿತ್ತಸ್ಯ ಯುಂಜತೋ ಯೋಗಮಾತ್ಮನಃ \hfill॥ ೧೯~॥
\end{shloka}

\begin{artha}
ಗಾಳಿ ಬೀಸದ ಕಡೆ ಇರುವ ದೀಪ ಅಲುಗಾಡುವುದಿಲ್ಲ. ಆತ್ಮನ ಯೋಗವನ್ನು ಮಾಡುತ್ತಿರುವ ಯೋಗಿಯ ನಿಗ್ರಹಿಸಿದ ಚಿತ್ತಕ್ಕೆ ಇದೊಂದು ಉಪಮಾನವೆನಿಸುವುದು.
\end{artha}

ಧ್ಯಾನಮಾಡುತ್ತಿರುವ ಯೋಗಿಯ ಮನಸ್ಸಿಗೆ ಒಂದು ಉರಿಯುತ್ತಿರುವ ದೀಪದ ಉದಾಹರಣೆಯನ್ನು ಕೊಡುವರು. ಹೆಚ್ಚು ಗಾಳಿ ಬೀಸದಕಡೆ ಒಂದು ದೀಪ ಉರಿಯುತ್ತಿದ್ದರೆ ಅದರ ಕುಡಿ ಅಲುಗಾಡದೆ ಒಂದೇ ಸಮನಾಗಿರುವುದು. ಅದರಂತೆಯೆ ಪರಮಾತ್ಮನನ್ನು ಚಿಂತಿಸುತ್ತಿರುವವನ ಮನಸ್ಸು. ಅದು ಪ್ರಪಂಚದ ಆಸೆಗಳಿಂದ ಚಲಿಸುವುದಿಲ್ಲ. ಆಸೆಯ ಗಾಳಿ ಅಲ್ಲಿಗೆ ಬರದಂತೆ ಅವನು ಮಾಡಿಕೊಂಡಿರುವನು. ಎಲ್ಲಾ ಇಂದ್ರಿಯಗಳ ಬಾಗಿಲುಗಳನ್ನು ಮುಚ್ಚಿ ಮನಸ್ಸನ್ನು ಅಂತರ್ಮುಖ ಮಾಡಿಕೊಂಡಿರುವನು. ಈ ಪ್ರಪಂಚದ ಮಮತೆಗಳನ್ನು ಬಿಟ್ಟಿದ್ದರೆ ಮಾತ್ರ ಒಬ್ಬನ ಮನಸ್ಸು ಪರಮಾತ್ಮನಮೇಲೆ ಕೇಂದ್ರೀಕೃತವಾಗುವುದು. ಯೋಗಿ ಇದನ್ನು ಮಾಡಿರುವನು. ಅನೇಕವೇಳೆ ಧ್ಯಾನಮಾಡಬೇಕೆಂದು ಆಸೆ ಇದೆ. ಆದರೆ ಅದಕ್ಕೆ ಪೂರ್ವಭಾವಿಯಾಗಿ ಏನು ಮಾಡಬೇಕೊ ಅದನ್ನು ಮಾಡಿರುವುದಿಲ್ಲ. ಆದಕಾರಣವೇ ಎಲ್ಲೂ ಇಲ್ಲದ ಚಿತ್ತಚಾಂಚಲ್ಯ ಆಗ ಮೊದಲಾಗುವುದು. ಯೋಗಿ ಧ್ಯಾನಮಾಡುವ ಸಮಯದಲ್ಲಿ ಮನಸ್ಸನ್ನು ಅಣಿಮಾಡಲು ಹೋಗುವುದಿಲ್ಲ. ಅವನು ಮನಸ್ಸನ್ನು ಸದಾ ಅಣಿಮಾಡುತ್ತಿರುವನು ಧ್ಯಾನಕ್ಕೆ. ಆದಕಾರಣವೇ ಧ್ಯಾನಕ್ಕೆ ಕುಳಿತರೆ ಧ್ಯಾನ ಅವನಿಗೆ ಸಿದ್ಧಿಸುವುದು. ಅಣಿಯಾಗುವುದೆಂದರೆ ಮನಸ್ಸು ವಿಷಯವಸ್ತುಗಳನ್ನು ಕುರಿತು ಚಿಂತಿಸದೆ ಇರುವುದು. ಇದು ಬರೀ ನಿಷೇಧಾತ್ಮಕವಾಯಿತು. ವಿಷಯವಸ್ತುವನ್ನು ಕುರಿತು ಚಿಂತಿಸದೆ ಇರಬೇಕಾದರೆ, ಸುಮ್ಮನೆ ಆಲೋಚನೆಯನ್ನು ಓಡಿಸಿಬಿಡುವುದಕ್ಕಾಗುವುದಿಲ್ಲ. ಪರಮಾತ್ಮನ ಆಲೋಚನೆಯನ್ನು ತಂದು ವಿಷಯವಸ್ತುಗಳ ಚಿಂತನೆಯನ್ನು ಓಡಿಸಬೇಕು, ಪ್ರಪಂಚದ ಮೇಲಿನ ಮಮತೆಯನ್ನು ಬಿಡಿಸಬೇಕು. ಭಗವಂತನನ್ನು ಮನಸ್ಸು ಹಿಡಿಯಬೇಕು. ಅವನನ್ನು ಬಿಗಿಯಾಗಿ ಹಿಡಿದುಕೊಂಡರೆ ಮಾತ್ರ ವಿಷಯ ವಸ್ತುಗಳನ್ನು ಚಿಂತಿಸುವುದನ್ನು ನಿಗ್ರಹಿಸಬಹುದು. ಅಲ್ಲಿ ನಮಗೆ ನೆಲೆ ಸಿಕ್ಕದೆ ಸುಮ್ಮನೆ ವಿಷಯವಸ್ತುಗಳನ್ನು ಚಿಂತಿಸುವುದಿಲ್ಲ ಎಂದರೆ ಸಾಧ್ಯವಿಲ್ಲ.

\begin{shloka}
ಯತ್ರೋಪರಮತೇ ಚಿತ್ತಂ ನಿರುದ್ಧಂ ಯೋಗಸೇವಯಾ~।\\ಯತ್ರಚೈವಾತ್ಮನಾತ್ಮಾನಂ ಪಶ್ಯನ್ನಾತ್ಮನಿ ತುಷ್ಯತಿ \hfill॥ ೨೦~॥
\end{shloka}

\begin{shloka}
ಸುಖಮಾತ್ಯಂತಿಕಂ ಯತ್ತದ್ಬುದ್ಧಿಗ್ರಾಹ್ಯಮತೀಂದ್ರಿಯಮ್~।\\ವೇತ್ತಿ ಯತ್ರ ನ ಚೈವಾಯಂ ಸ್ಥಿತಶ್ಚಲತಿ ತತ್ತ್ವತಃ \hfill॥ ೨೧~॥
\end{shloka}

\begin{shloka}
ಯಂ ಲಬ್ಧ್ವಾ ಚಾಪರಂ ಲಾಭಂ ಮನ್ಯತೇ ನಾಧಿಕಂ ತತಃ~।\\ಯಸ್ಮಿಂಸ್ಥಿತೋ ನ ದುಃಖೇನ ಗುರುಣಾಪಿ ವಿಚಾಲ್ಯತೇ \hfill॥ ೨೨~॥
\end{shloka}

\begin{shloka}
ತಂ ವಿದ್ಯಾದ್ದುಃಖಸಂಯೋಗವಿಯೋಗಂ ಯೋಗಸಂಜ್ಞಿತಮ್~।\\ಸ ನಿಶ್ಚಯೇನ ಯೋಕ್ತವ್ಯೋ ಯೋಗೋಽನಿರ್ವಿಣ್ಣಚೇತಸಾ \hfill॥ ೨೩~॥
\end{shloka}

\begin{artha}
ಯಾವಾಗ ಯೋಗಾಭ್ಯಾಸದಿಂದ ನಿರೋಧಿಸಲ್ಪಟ್ಟು ಚಿತ್ತ ಶಾಂತವಾಗುವುದೊ, ಯಾವಾಗ ಶುದ್ಧವಾದ ಮನಸ್ಸಿನಿಂದ ಆತ್ಮನನ್ನು ನೋಡುತ್ತ ಇವನು ತನ್ನಲ್ಲಿಯೇ ಸಂತೋಷಪಡುವನೊ, ಮತ್ತು ಬುದ್ಧಿಗ್ರಾಹ್ಯವೂ ಇಂದ್ರಿಯಗ್ರಾಹ್ಯವೂ ನಿತ್ಯವೂ ಆದ ಯಾವ ಸುಖವಿದೆಯೊ ಅದನ್ನು ತಿಳಿಯು\-ತ್ತಾನೆಯೊ, ಯಾವುದರಲ್ಲಿ ಇದ್ದುಕೊಂಡು ತತ್ತ್ವದಿಂದ ಚಲಿಸದೆ ಇರುವನೊ, ಯಾವುದನ್ನು ಪಡೆದು ಬೇರೆ ಲಾಭವನ್ನು ಅದಕ್ಕಿಂತ ಮೇಲೆಣಿಸುವುದಿಲ್ಲವೊ, ಯಾವುದರಲ್ಲಿದ್ದುಕೊಂಡು\break ಮಹತ್ತಾದ ದುಃಖದಿಂದಲೂ ಚಲಿಸದೆ ಇರುವನೊ, ದುಃಖ ಸಂಬಂಧವನ್ನು ದೂರಮಾಡುವ ಇದನ್ನು ಯೋಗವೆಂದು ಅರಿಯಬೇಕು. ಅಂತಹ ಯೋಗವನ್ನು ನಿಶ್ಚಯವಾದ ಮತ್ತು ಉತ್ಸಾಹ\-ಪೂರಿತವಾದ ಮನಸ್ಸಿನಿಂದ ಮಾಡಬೇಕು.
\end{artha}

ಯೋಗಾಭ್ಯಾಸದಿಂದ ನಿಗ್ರಹಿಸಲ್ಪಟ್ಟ ಚಿತ್ತ ಶಾಂತವಾಗುವುದು. ಚಿತ್ತ ಒಂದು ದಿನದಲ್ಲಿ ಶಾಂತವಾಗುವುದಿಲ್ಲ. ಅದನ್ನು ಅಭ್ಯಾಸದ ಮೂಲಕ ದೇವರ ಕಡೆ ಹೋಗುವಂತೆ ಮಾಡಬೇಕು. ಯೋಗ ಎಂದರೆ ಸತತ ಅಭ್ಯಾಸ. ಸತತ ಅಭ್ಯಾಸ ಅವನ ಸ್ವಭಾವವಾಗುತ್ತ ಬರುವುದು. ಯಾವಾಗ ಅದೊಂದು ಸ್ವಭಾವವಾಗಿಹೋಗುವುದೊ ಅನಂತರ ಮನಸ್ಸನ್ನು ನಿಗ್ರಹಿಸುವುದಕ್ಕೆ ಅಷ್ಟು ಕಷ್ಟಪಡ ಬೇಕಾಗಿಲ್ಲ. ಅಭ್ಯಾಸ ಎಂದರೆ ಹೋರಾಟ ಪ್ರಥಮದಲ್ಲಿ ಇದ್ದೇ ಇರುವುದು. ಆ ಹೋರಾಟವನ್ನು ಪೂರೈಸಿದ ಮೇಲೆಯೇ ಶಾಂತಿ ಮನಸ್ಸಿಗೆ ಕ್ರಮೇಣ ಬರುವುದು. ಸಮುದ್ರಕ್ಕೆ ಮೀನು ಹಿಡಿಯಲು ಸಣ್ಣ ದೋಣಿಗಳನ್ನು ತೆಗೆದುಕೊಂಡು ಹೋಗುತ್ತಾರೆ. ಮೊದಮೊದಲು ಸಮುದ್ರದ ಅಲೆಗಳು ಯಾವಾಗಲೂ ಕರೆಗೆ ಬಂದು ಹೊಡೆಯುತ್ತಿರುವುದರಿಂದ ದೋಣಿಯನ್ನು ಸಮುದ್ರಕ್ಕೆ ತೆಗೆದುಕೊಂಡು ಹೋಗುವುದು ಕಷ್ಟವಾಗುವುದು. ಕೆಲವು ವೇಳೆ ಅಲೆಗಳು ದೋಣಿಯನ್ನು ನೂಕಿಬಿಡುವುವು. ಆದರೆ ನುರಿತ ದೋಣಿಗಾರನಿಗೆ ಅಲೆಯ ಮೇಲೆ ಹೇಗೆ ತೇಲುವುದು ಎಂಬುದು ಗೊತ್ತಿದೆ. ಅದರೊಂದಿಗೆ ಸ್ವಲ್ಪ ಹೋರಾಡಿ ಸಮುದ್ರದ ಮೇಲೆ ಸ್ವಲ್ಪ ದೂರ ಹೋದರೆ ಇನ್ನು ಅವನು ಅಲೆಯ ಉಪಟಳದಿಂದ ಪಾರಾಗುವನು. ಅನಂತರ ಅವನು ನೆಮ್ಮದಿಯಿಂದ ಮೀನನ್ನು ಹಿಡಿಯುತ್ತಿರಬಹುದು. ಇದರಂತೆಯೆ ಧ್ಯಾನ ಪ್ರಾರಂಭದಲ್ಲಿ ಬಹಳ ಕಷ್ಟ. ಅಲೆಗಳೆದ್ದು ನಮ್ಮ ಮನಸ್ಸನ್ನು ಪ್ರಪಂಚದ ಕಡೆ ನೂಕುತ್ತಿರುವುವು. ಹೋರಾಟದಿಂದ ಪಾರಾಗಿ ಸ್ವಲ್ಪ ಮುಂದುವರಿದರೆ ಅನಂತರ ಅದರ ಉಪಟಳದಿಂದ ಪಾರಾಗಿ ಮನಸ್ಸು ಭಗವಂತನಲ್ಲಿ ವಿಹರಿಸುವುದು.

ಶುದ್ಧವಾದ ಮನಸ್ಸಿನಿಂದ ಆತ್ಮನನ್ನು ನೋಡುತ್ತಾನೆ. ಭಗವಂತ ಪಕ್ಷಪಾತವಿಲ್ಲದೆ ಎಲ್ಲರ ಹೃದಯಾಂತರಾಳದಲ್ಲಿಯೂ ಇರುವನು. ಅವನಿಲ್ಲದ ಸ್ಥಳವಿಲ್ಲ. ಅವನಷ್ಟು ಸತ್ಯ ಮತ್ತಾವುದೂ ಇಲ್ಲ. ಆದರೂ ಅವನು ಈಗ ನಮಗೆ ಕಾಣುತ್ತಿಲ್ಲ. ಏಕೆಂದರೆ ನಾವು ಮನಸ್ಸನ್ನು ಶುದ್ಧ ಮಾಡಿಲ್ಲ. ಮನಸ್ಸು ಎಂಬ ಕನ್ನಡಿಯ ಹಿಂದೆ ಪಾದರಸ ಬಳಿದಿರುವುದರಿಂದ ಅದು ತನ್ನ ಹಿಂದೆ ಏನಿದೆಯೊ ಅದನ್ನು ತೋರುವುದಿಲ್ಲ. ಅದೊಂದು ಕನ್ನಡಿಯಾಗಿ ತನಗೆ ಹೊರಗೆ ಇರುವುದನ್ನು ಮಾತ್ರ ಪ್ರತಿಬಿಂಬಿಸುವುದು. ಅದರ ಹಿಂದೆ ಇರುವ ಪಾದರಸವನ್ನು ತೊಳೆದರೆ ಆ ಕನ್ನಡಿ ಪಾರದರ್ಶಕವಾಗುವುದು. ಅದು ತನ್ನ ಹಿಂದೆ ಏನಿದೆಯೊ ಅದನ್ನು ತೋರುವುದು. ಭಗವಂತ ಎಲ್ಲದರಲ್ಲಿಯೂ ಇರುವನು. ಅವನು ಕಾಣದಂತೆ ಮಾಡಿರುವುದೆ ನಮ್ಮ ಮನಸ್ಸಿನ ಕೊಳೆ. ಯಾವಾಗ ಕೊಳೆಯನ್ನು ತೊಳೆಯುತ್ತೇವೆಯೊ ಅದರ ಹಿಂದೆ ಇರುವ ಪರಮಾತ್ಮ ಕಾಣುವನು. ಕಶ್ಮಲವನ್ನು ತೊಳೆದಮೇಲೆ ದೇವರು ಹೊಸದಾಗಿ ಬರುವುದಿಲ್ಲ. ಮುಂಚೆಯೂ ಇದ್ದ. ಆದರೆ ನಮಗೆ ಕಾಣುತ್ತಿರಲಿಲ್ಲ. ಶುದ್ಧವಾದಮೇಲೆ ಅವನು ಕಾಣುತ್ತಾನೆ. ಇದೇನೂ ಹೊಸದಾಗಿ ಬರುವುದಿಲ್ಲ. ಆಗಲೆ ಇರುವುದು ಕಾಣಲು ಮೊದಲಾಗುವುದು. ನೀರು ಅಲ್ಲೋಲ ಕಲ್ಲೋಲವಾಗಿದ್ದರೆ ಅದು ತನ್ನ ಅಡಿಯಲ್ಲಿ ಇರುವುದನ್ನು ತೋರುವುದಿಲ್ಲ. ಆದರೆ ಶಾಂತವಾದೊಡನೆ ಅದರ ಕೆಳಗೆ ಇರುವುದು ಗೋಚರಿಸುವುದು.

ಯೋಗಿಯು ತನ್ನಲ್ಲಿಯೇ ಸಂತೋಷಪಡುವನು. ತನ್ನ ಹೃದಯಾಂತರಾಳದಲ್ಲಿ ಬೆಳಗುವ ಪರಮಾತ್ಮನನ್ನು ನೋಡಿ ಆನಂದಭರಿತನಾಗುವನು. ಈ ಆನಂದ ಅನ್ಯಾಶ್ರಯವಲ್ಲ. ಇರುವ ಆತಂಕ ಹೋದೊಡನೆ ಪರಮಾತ್ಮನ ಆನಂದ ನಮಗೆ ಹರಿದು ಬರುವುದು. ಲೌಕಿಕ ಆನಂದವಾದರೊ ಅನ್ಯಾಶ್ರಯವಾದದ್ದು. ಅದು ಹೊರಗಿನ ಹಂಗಿಗೆ ಒಳಗಾಗಿರುವುದು. ವಿಷಯವಸ್ತುಗಳನ್ನು ಕೊಟ್ಟರೆ ಅದನ್ನು ಅನುಭವಿಸುವಾಗ ನಮಗೆ ಆನಂದ. ಅದನ್ನು ಕೊಡದೆ ಇದ್ದರೆ ಅದರ ದಾಸರಾಗಿ ಅದನ್ನು ಬೇಡುವೆವು. ವಿಷಯವಸ್ತು ಇರುವಾಗಲೂ ಅದನ್ನು ಮಿತಿಮೀರಿ ಅನುಭವಿಸುವುದಕ್ಕೆ ಆಗುವುದಿಲ್ಲ. ಅದಕ್ಕೆ ಒಂದು ಮಿತಿ ಇದೆ. ಆ ಮಿತಿಯನ್ನು ಮೀರಿದರೆ, ಆ ವಿಷಯವಸ್ತುವಿನ ಮೇಲೆ ನಮಗೆ ಜುಗುಪ್ಸೆ ಬರುವುದು. ಅದನ್ನು ಅನುಭವಿಸಿದಮೇಲೆ ದೊಡ್ಡದೊಂದು ಅಹಿತವಾದ ಪ್ರತಿಕ್ರಿಯೆ ನಮ್ಮ ಮನಸ್ಸಿನಲ್ಲಿ ಏಳುವುದು. ಆದರೆ ಭಗವಂತನ ಚಿಂತನೆಯಿಂದ ಬರುವ ಆನಂದ ಬೇರೆ. ಅದನ್ನು ಎಷ್ಟು ಬೇಕಾದರೂ ಅನುಭವಿಸುತ್ತಿರಬಹುದು. ಇದನ್ನು ಯಾರೂ ನಮಗೆ ಹೊರಗಿನ ಪ್ರಪಂಚದಲ್ಲಿ ಕೊಡಬೇಕಾಗಿಲ್ಲ. ಮನಸ್ಸನ್ನು ಭಗವಂತನ ಕಡೆಗೆ ತಿರುಗಿಸಿದರೆ ಸಾಕು ಅಲ್ಲಿಂದ ಆನಂದ ಹರಿದು ಬರುವುದು.

ತನ್ನಲ್ಲಿಯೇ ಇರುವ ಆನಂದ ಬುದ್ಧಿಗ್ರಾಹ್ಯ. ಭಗವಂತನನ್ನು ಅರಿಯುವುದಕ್ಕೆ ಶುದ್ಧವಾದ ಸಾತ್ತ್ವಿಕ ಬುದ್ಧಿಗೆ ಮಾತ್ರ ಸಾಧ್ಯ. ಯಾವ ಬುದ್ಧಿಯಿಂದ ದೇಶ, ಕಾಲ, ನಿಮಿತ್ತದಲ್ಲಿರುವ ವಸ್ತುಗಳನ್ನು ತಿಳಿದುಕೊಳ್ಳುತ್ತೇವೆಯೊ ಅದರಿಂದ ಅಲ್ಲ. ಆ ಬುದ್ಧಿ ಸ್ಥೂಲವಾದ ವಸ್ತುಗಳನ್ನು ಮಾತ್ರ ತಿಳಿಯುತ್ತದೆ. ಅದು ಸೇರು ಪಾವು ಚಟಾಕಿನಂತೆ. ದ್ರವವನ್ನು ಮತ್ತು ದವಸಧಾನ್ಯಗಳನ್ನು ಅಳೆದು ತೂಗುವುದು. ಆದರೆ ಅದು ಗಾಳಿಯನ್ನು ಹಿಡಿಯುವುದಕ್ಕೆ ಆಗುವುದಿಲ್ಲ. ಪರಮಾತ್ಮನನ್ನು ತಿಳಿಯುವ ಬುದ್ಧಿ ಪರಿಶುದ್ಧವಾಗಿರಬೇಕು, ಏಕಾಗ್ರವಾಗಿರಬೇಕು. ಅಲ್ಲಿ ಯಾವ ಕಶ್ಮಲವೂ ಇರಬಾರದು. ಇಂತಹ ಬುದ್ಧಿಗೆ ಮಾತ್ರ ಭಗವಂತನನ್ನು ತಿಳಿಯುವುದು ಸಾಧ್ಯ. ಈ ಆನಂದ ಇಂದ್ರಿಯಾತೀತವಾದುದು. ಸಾಧಾರಣವಾಗಿ ನಮಗೆ ಆನಂದ ಬರಬೇಕಾದರೆ ನಮ್ಮ ಇಂದ್ರಿಯಕ್ಕೂ, ಹೊರಗಿನ ವಿಷಯವಸ್ತುವಿಗೂ ಸಂಬಂಧ ಏರ್ಪಡಬೇಕು. ಆಗ ಆನಂದ ಬರುವುದು. ಆದರೆ ಭಗವಂತನ ಧ್ಯಾನದಿಂದ ಬರುವ ಆನಂದವಾದರೊ ಪಂಚೇಂದ್ರಿಯಗಳಿಗೆ ಅತೀತವಾದುದು. ಅದನ್ನು ಧ್ಯಾನಿಸುವವನು ಪರಮಾತ್ಮನ ಸಾನ್ನಿಧ್ಯವನ್ನು ಅನುಭವಿಸುತ್ತಾನೆ. ಅದು ಹೇಗಿದೆ ಎಂದು ಕೇಳಿದರೆ, ಅದನ್ನು ಅವನಿಗೆ ವಿವರಿಸುವುದಕ್ಕೆ ಆಗುವುದಿಲ್ಲ. ಏಕೆಂದರೆ ನಮ್ಮ ಭಾಷೆ ಇಂದ್ರಿಯ ಅನುಭವವನ್ನು ಮಾತ್ರ ವಿವರಿಸಬಲ್ಲುದು. ನಮ್ಮ ಭಾಷೆ ಬಡಭಾಷೆ. ನಾವು ಅನುಭವಿಸುವುದೆಲ್ಲವನ್ನು ವಿವರಿಸಲಾರೆವು. ಮಾತು ಇಲ್ಲಿ ಮೂಕವಾಗಿಹೋಗುವುದು. ಇಂದ್ರಿಯಾತೀತ ಅನುಭವದ ಒಂದು ಲಕ್ಷಣ ಇದು. ನಾರದರು ಭಕ್ತಿಸೂತ್ರದಲ್ಲಿ ಇದನ್ನು ಮೂಕ ಒಂದು ರುಚಿಕರವಾದ ವಸ್ತುವನ್ನು ತಿಂದಂತೆ ಎಂದು ಹೇಳುವರು. ಅವನು ತಿನ್ನುತ್ತಾನೆ,\break ಮಾತಿಲ್ಲ ಅದನ್ನು ಬಣ್ಣಿಸುವುದಕ್ಕೆ. ಹಾಗಾಗಿದೆ ಇಂದ್ರಿಯಾತೀತ ಆನಂದವನ್ನು ಅನುಭವಿಸುವವನ ಪಾಡು. ಅದು ಸತ್ಯವಾದುದು. ಯಾವಾಗಲೂ ಇರುವಂತಹುದು. ಅದನ್ನು ಅರಿತಮೇಲೆ ಎಂದೆಂದಿಗೂ ಅದು ನಮ್ಮದಾಗುವುದು. ಅದು ನಮ್ಮನ್ನು ಬಿಟ್ಟುಹೋಗುವುದಿಲ್ಲ. ಈ ಪ್ರಪಂಚದಲ್ಲಿ ಪರಮಾತ್ಮನೇ ನಿತ್ಯ. ಅವನೊಬ್ಬನೇ ಹಿಂದೆ ಇದ್ದವನು, ಈಗ ಇರುವವನು, ಎಲ್ಲ ನಾಶವಾದರೂ ಮುಂದೆ ಇರುವವನು. ಈ ಪ್ರಪಂಚದಲ್ಲಿ ಎಲ್ಲಾ ಬದಲಾಯಿಸು ತ್ತಿದೆ. ಕೆಲವು ವೇಗವಾಗಿ, ಮತ್ತೆ ಕೆಲವು ನಿಧಾನವಾಗಿ. ಆದರೆ ಬದಲಾವಣೆಯ ಚಕ್ರಕ್ಕೆ ಸಿಕ್ಕದೆ ಬದಲಾವಣೆಗಳಿಗೆಲ್ಲ ಆಧಾರವಾಗಿರುವವನು ಅವನೊಬ್ಬನೇ ಎಂಬುದನ್ನು ಧ್ಯಾನಿ ತನ್ನಲ್ಲಿ ಅರಿಯುವನು.

ಭಗವಂತನ ಆಧಾರದ ಮೇಲೆ ನಿಂತಿದ್ದರೆ ತತ್ತ್ವದಿಂದ ಚಲಿಸುವುದಿಲ್ಲ. ಧ್ಯಾನದಲ್ಲಿರುವಾಗ ಮನಸ್ಸು ಪರಮಾತ್ಮನಲ್ಲಿ ನೆಲಸಿರುವುದು. ಯಾವಾಗ ಪರಮಾತ್ಮನಲ್ಲಿ ನೆಲಸಿರುವುದೊ ಆಗ ಈ ಪ್ರಪಂಚದ ದ್ವಂದ್ವ ಅನುಭವಗಳು ಇನ್ನುಮೇಲೆ ಅವನ ಸ್ವಾಸ್ಥ್ಯಕ್ಕೆ ಭಂಗತಾರವು. ಕಂಭವನ್ನು ಹಿಡಿದುಕೊಂಡು ಹುಡುಗ ಸುತ್ತುತ್ತಿದ್ದರೆ ಹೇಗೆ ಬೀಳುವುದಿಲ್ಲವೊ ಹಾಗೆ ಭಗವಂತನನ್ನು ಹಿಡಿದುಕೊಂಡವನು ಇನ್ನುಮೇಲೆ ಈ ಪ್ರಪಂಚಕ್ಕೆ ಬೀಳುವುದಿಲ್ಲ. ಸಾಧಾರಣ ಮನೆಗಳನ್ನು ಕ್ಷಣಕಾಲದಲ್ಲಿ ಪುಡಿಪುಡಿ ಮಾಡಬಲ್ಲ ಭೂಕಂಪವು ಹೇಗೆ ಆಕಾಶದಲ್ಲಿರುವುದನ್ನು ಏನೂ ಮಾಡಲಾರದೊ ಅದರಂತೆಯೆ ಭಗವಂತನನ್ನು ಬಿಗಿಯಾಗಿ ಹಿಡಿದವನು ಈ ಪ್ರಪಂಚದ ಘಟನೆಗಳಿಂದ ವಿಚಲಿತನಾಗುವುದಿಲ್ಲ. ಇನ್ನುಮೇಲೆ ಅವನು ಇಂದ್ರಿಯ ಪ್ರಪಂಚಕ್ಕೆ ಸೇರಿದವನಲ್ಲ. ಯೋಗಿಯ ಕೇಂದ್ರ ದೇವರಾಗುವನು.

ಅದನ್ನು ಪಡೆದರೆ ಬೇರೆ ಲಾಭವನ್ನು ಅದಕ್ಕಿಂತ ಹೆಚ್ಚು ಅನ್ನುವುದಿಲ್ಲ. ಭಗವಂತನನ್ನು ರುಚಿ ನೋಡಿದವನಿಗೆ ಈ ಪ್ರಪಂಚ ಸಪ್ಪೆಯಾಗುವುದು. ಇಂದ್ರ ಪದವಿಯನ್ನು ಲೆಕ್ಕಿಸನು ಅವನು. ಈ ಜೀವನದಲ್ಲಿ ಬುದ್ಧಿಜೀವಿಗೆ ಬಹಳ ಪ್ರಿಯವಾದುದು ಯೋಗಿಗೆ ಕಸಕ್ಕಿಂತ ಕಡೆಯಾಗುವುದು. ಶ‍್ರೀಚೈತನ್ಯನ ಶಿಷ್ಯನೊಬ್ಬ ಒಂದು ಸ್ಥಳದಲ್ಲಿ ತಪಸ್ಸು ಮಾಡುತ್ತಿದ್ದ. ಯಾರೊ ಹಲವು ಮಕ್ಕಳನ್ನು ಪಡೆದು ಹೊಟ್ಟೆಗಿಲ್ಲದೆ ಅಲೆಯುತ್ತಿದ್ದವನು ಅವನ ಬಳಿಗೆ ಬಂದು ತನಗೆ ಏನಾದರೂ ಕೊಡಬೇಕು ದಾರಿದ್ರ್ಯದಿಂದ ಪಾರಾಗುವುದಕ್ಕೆ ಎಂದು ಕೇಳಿಕೊಳ್ಳುತ್ತಾನೆ. ಚೈತನ್ಯನ ಶಿಷ್ಯ ನದೀತೀರದಲ್ಲಿದ್ದ. ಬಂದವನಿಗೆ ಸ್ವಲ್ಪ ಅಲಕ್ಷ್ಯದಿಂದ ಅಲ್ಲೊಂದು ಮೆಟ್ಟಲಿನ ಬಿರುಕು ಇದೆಯಲ್ಲ ಅಲ್ಲಿ ಹೋಗಿ ನೋಡು, ನಿನಗೆ ದೊಡ್ಡದೊಂದು ರತ್ನ ಸಿಕ್ಕುವುದು ಎಂದು ಹೇಳುತ್ತಾನೆ.\break ಆತ ಅಲ್ಲಿ ನೋಡಲು ಹೋದಾಗ ಅನರ್ಘ್ಯವಾದ ದೊಡ್ಡ ರತ್ನ ಅಲ್ಲಿತ್ತು. ಅದನ್ನು ತೆಗೆದುಕೊಂಡು ಹೋಗು ಎನ್ನಬೇಕಾದರೆ ಆ ಯೋಗಿಗೆ ಮತ್ತೇನೊ, ಇದನ್ನು ಅಷ್ಟೊಂದು ಅಸಡ್ಡೆಯಿಂದ ನೋಡುವಂತೆ ಮಾಡಿದ ಅನರ್ಘ್ಯ ವಸ್ತು ಸಿಕ್ಕಿರಬೇಕು ಎಂದು ಪುನಃ ಅವನ ಹತ್ತಿರ ಬಂದು ಕೇಳುತ್ತಾನೆ: ಈ ಅನರ್ಘ್ಯ ರತ್ನವನ್ನೆ ಕೆಲಸಕ್ಕೆ ಬಾರದ ಕಲ್ಲಿನಂತೆ ನೋಡುವಂತೆ ಯಾವುದು ನಿನ್ನನ್ನು ಮಾಡಿದೆಯೊ ಅದನ್ನು ಹೇಳು ಎನ್ನುತ್ತಾನೆ. ಅದೇ ಭಕ್ತನ ಭಗವತ್ಪ್ರೇಮ. ಯೋಗಿಗೆ ಧ್ಯಾನಾನಂದಕ್ಕಿಂತ ಮಿಗಿಲಾದ ಆನಂದವಿಲ್ಲ ಈ ಪ್ರಪಂಚದಲ್ಲಿ. ಈ ಪ್ರಪಂಚದ ಸುಖಗಳಾದರೊ ಇಂದ್ರಿಯದ ಬಿಲದ ಮೂಲಕ ಹರಿದು ಬರುವುದು. ಬರುವ ಪ್ರತಿಯೊಂದು ಸುಖವೂ ತನ್ನ ನೆರಳಿನಂತೆ ತನ್ನ ದುಃಖವನ್ನು ತರುವುದು. ಒಮ್ಮೆ ಅದರ ಬಲೆಗೆ ಬಿದ್ದರೆ ಎಂದೆಂದಿಗೂ ಅದಕ್ಕೆ ಗುಲಾಮರಾಗುವೆವು. ಆದರೆ ಧ್ಯಾನಾನಂದವಾದರೊ ನಮ್ಮನ್ನು ಈ ಪ್ರಪಂಚದ ಕ್ಷುದ್ರ ಅನುಭವದಿಂದ ಮೇಲೆತ್ತಿ ಇಂದ್ರಿಯಾತೀತ ಆನಂದದ ಸವಿಯನ್ನು ನೀಡುವುದು. ಇದು ಮರ್ತ್ಯರನ್ನು ಅಮೃತರನ್ನಾಗಿ ಮಾಡುವುದು.

ಯಾವುದರಲ್ಲಿದ್ದುಕೊಂಡು ಮಹತ್ತಾದ ದುಃಖದಿಂದಲೂ ಚಲಿತನಾಗುವುದಿಲ್ಲ: ಒಮ್ಮೆ ಧ್ಯಾನದಲ್ಲಿ ಮನಸ್ಸು ನಿರಾತಂಕವಾಗಿ ಭಗವಂತನ ಕಡೆಗೆ ಹರಿದುಹೋಗುವುದು ಅಭ್ಯಾಸವಾದರೆ ಅವನಿಗೆ ಪ್ರಪಂಚದ ದುಃಖಗಳು ಚಕ್ಕಳುಗುಳಿಯನ್ನು ಇಡಲಾರವು. ಅವನು ಈ ಪ್ರಪಂಚವನ್ನು ನೋಡುವ ದೃಷ್ಟಿಯೇ ಬದಲಾಗುವುದು. ದೇವರನ್ನು ಮರೆತವನಿಗೆ ಪ್ರಪಂಚದ ದುಃಖಗಳು ತಾಕುವುವು. ದೇವರನ್ನು ಬಿಗಿಯಾಗಿ ಹಿಡಿದವನಿಗೆ ದುಃಖ ತಾಕುವುದಿಲ್ಲ. ಅವನು ಪ್ರಪಂಚದಲ್ಲಿ ಬೇರು ಬಿಟ್ಟಿಲ್ಲ, ದೇವರಲ್ಲಿ ಬೇರು ಬಿಟ್ಟಿರುವನು. ಯಾವ ದುಃಖವಾದರೂ ಆಗಲಿ, ಯಾವ ವಿಯೋಗವಾಗಲಿ, ಅವನನ್ನು ತಾಕಲಾರವು. ದೊಡ್ಡದೊಡ್ಡ ಮನೆಗಳಿಗೆ ಸಿಡಿಲು ಬಡಿಯದಂತೆ ಅದಕ್ಕೆ ರಕ್ಷೆಯನ್ನು ಮಾಡಿರುವರು. ಏನಾದರೂ ಸಿಡಿಲು ಅದಕ್ಕೆ ಬಡಿದರೆ ಆ ರಕ್ಷೆ ಸಿಡಿಲಿನ ಶಕ್ತಿಯನ್ನು ತೆಗೆದುಕೊಂಡು ಅದನ್ನು ಸುರಕ್ಷಿತವಾಗಿ ಭೂಮಿಗೆ ಇಳಿಸುವುದು. ಕಟ್ಟಡಕ್ಕೆ ಯಾವ ಅಪಾಯವೂ ಆಗುವುದಿಲ್ಲ. ಅದರಂತೆಯೆ ಭಗವಂತನ ಕೃಪಾಹಸ್ತದಡಿಯಲ್ಲಿ ವಾಸಿಸುವವನಿಗೆ ಈ ಪ್ರಪಂಚದ ಸಿಡಿಲುಗಳು ಏನೂ ಮಾಡಲಾರವು.

ಆ ಯೋಗ ದುಃಖ ಸಂಬಂಧವನ್ನು ದೂರಮಾಡುವುದು. ದುಃಖ ನಮಗೆ ಬರುವುದಕ್ಕೆ ಕಾರಣ ನಾವು ಭಗವಂತನನ್ನು ಮರೆತಿರುವುದು; ಈ ದೇಹ ಮನಸ್ಸು ಇಂದ್ರಿಯಗಳನ್ನು ಸತ್ಯ ಎಂದು ಭಾವಿಸಿರುವುದು. ಯಾವಾಗ ಇದನ್ನು ಸತ್ಯವೆಂದು ಭಾವಿಸುವೆವೊ, ಇದಕ್ಕೆ ಪ್ರಿಯವಾದ ವಸ್ತುವನ್ನು ವಿಷಯ ಪ್ರಪಂಚದಿಂದ ತರುವೆವು. ಯಾವುದರ ಮೂಲಕ ಸುಖದ ವೇದನೆಗಳು ಬರುವುವೊ ಅದರ ಮೂಲಕವೆ ದುಃಖದ ವೇದನೆಗಳೂ ಬರುವುವು. ಒಂದು ಟೆಲಿಗ್ರಾಫಿನ ತಂತಿ ಸುಖದ ಸಮಾಚಾರವನ್ನು ಹೇಗೆ ತರುವುದೊ ಹಾಗೆಯೇ ದುಃಖ ಸಮಾಚಾರವನ್ನೂ ತರುವುದು. ಸುಖದುಃಖಗಳಲ್ಲಿ ಯಾವ ವ್ಯತ್ಯಾಸವನ್ನೂ ಮಾಡುವುದಿಲ್ಲ. ಯಾವ ದೇಹದ ಮೂಲಕ ನಾವು ಸುಖದ ವೇದನೆಯನ್ನು ಅನುಭವಿಸುವೆವೊ ಅದೇ ದೇಹದ ಮೂಲಕ ದಾರುಣವಾದ ಯಾತನೆಯ ವೇದನೆಯನ್ನೂ ನಾವು ಅನುಭವಿಸ ಬೇಕಾಗಿದೆ. ಯೋಗ ಏನು ಮಾಡುವುದು ಎಂದರೆ ಈ ದೇಹದ ಮೇಲೆ ಇರುವ ನಮ್ಮ ಮಮತೆಯನ್ನು ಖಂಡಿಸುವುದು. ಅನಂತರ ಅದಕ್ಕೆ ಆಗುವ ಸುಖ ದುಃಖಗಳು ನಮಗೆ ಏನೂ ಮಾಡಲಾರವು. ವೈದ್ಯ ಶಸ್ತ್ರಚಿಕಿತ್ಸೆಯನ್ನು ಮಾಡುವುದಕ್ಕೆ ಮುಂಚೆ, ಯಾವ ಭಾಗದಲ್ಲಿ ಶಸ್ತ್ರಚಿಕಿತ್ಸೆ ಮಾಡುವನೊ ಅಲ್ಲಿ ಇರುವ ನರಗಳಿಗೆ ನೋವು ಗೊತ್ತಾಗದಂತೆ ಸೂಜಿಯಿಂದ ಮದ್ದನ್ನು ಚುಚ್ಚಿ ಅದನ್ನು ಜಡ ಮಾಡುವನು. ಆಗ ಅವನಿಗೆ ಯಾವ ನೋವೂ ಕಾಣುವುದಿಲ್ಲ. ಅದರಂತೆಯೇ ಯೋಗ, ದೇಹ ಸಂಬಂಧದಿಂದ ನಮ್ಮನ್ನು ಪಾರುಮಾಡಿದಾಗ ಅದರಿಂದ ಬರುವ ದುಃಖಸಂಬಂಧದಿಂದಲೂ ಪಾರಾಗುತ್ತೇವೆ. ಶಸ್ತ್ರಚಿಕಿತ್ಸೆಯಲ್ಲಿ ಅದು ತಾತ್ಕಾಲಿಕ. ಅನಂತರ ನೋವು ತಾನಿರುವೆ ಎಂದು ಬರುವುದು. ಯೋಗಿಗಾದರೊ ಹೀಗೆ ಕಾಡಲಾರದು. ಇಂದ್ರಿಯಕ್ಕೆ ಅತೀತವಾದ ಭಗವಂತನೆ ಸತ್ಯ ಎಂಬುದನ್ನು ಅವನು ಮನಗಂಡಿರುವನು. ಕರಟಕ್ಕೆ ಏನಾದರೂ ಆದರೆ ಅದರೊಳಗೆ ಇರುವ ಅದರಿಂದ ಬೇರೆಯಾದ ಕೊಬ್ಬರಿ ಗಿಟಕಿಗೆ ಏನೂ ಆಗುವುದಿಲ್ಲ. ಹಾಗೆಯೆ ಯೋಗಿಯ ಮನಸ್ಸು ದೇಹದಲ್ಲಿದೆ, ಆದರೂ ಅದರಿಂದ ಬೇರೆಯಾಗಿದೆ.

ಯೋಗವನ್ನು ಮಾಡುವುದಕ್ಕೆ ಚಿತ್ತ ನಿಶ್ಚಯವಾದ ಬುದ್ಧಿಯಿಂದ ಕೂಡಿಕೊಂಡಿರಬೇಕು ಮತ್ತು ಎಂದೆಂದಿಗೂ ಬತ್ತದ ಉತ್ಸಾಹ ಇರಬೇಕು ಎನ್ನುವನು ಶ‍್ರೀಕೃಷ್ಣ. ಮುಂಚೆ ಮನಸ್ಸನ್ನು ವಿಚಾರ ಮಾಡಿ ಸಂಶಯದ ಜಳ್ಳುಗಳನ್ನೆಲ್ಲ ತೂರಿ, ಸತ್ಯವಾದ ಕಾಳನ್ನು ಮಾತ್ರ ಇಟ್ಟುಕೊಳ್ಳಬೇಕು. ಈ ಪ್ರಪಂಚದಲ್ಲಿ ದೇವರು ಸತ್ಯ, ಅವನನ್ನು ಪಡೆಯುವುದಕ್ಕೆ ಸಾಧ್ಯ ಎಂಬುದು ಅವನಿಗೆ ಹೃದ್ಗತವಾಗಿರಬೇಕು. ಅವನು ಮುಂಚೆಯೆ ಸಂಶಯಗಳೊಂದಿಗೆ ಹೋರಾಡಿ ಒಂದು ನಿರ್ಧಾರಕ್ಕೆ ಬಂದಿರಬೇಕು. ಒಮ್ಮೆ ಒಂದು ನಿರ್ಧಾರಕ್ಕೆ ಬಂದನೆಂದರೆ ಇನ್ನು ಏನಾದರೂ ಚಿಂತೆಯಿಲ್ಲ. ಕೊನೆಯವರೆಗೂ ಬಿಡದೆ ಛಲದಿಂದ ಅದನ್ನು ಸಾಧಿಸಲು ಪ್ರಯತ್ನಿಸಬೇಕು. ಯಾವಾಗ ಮನಸ್ಸು ಸಂಶಯಗ್ರಸ್ಥವಾಗುವುದೊ ಆಗ ಅವನು ತೆಗೆದುಕೊಂಡಿರುವ ಆದರ್ಶದಲ್ಲಿಯೇ ನಂಬಿಕೆ ಜಾರುವುದು. ಇನ್ನು ಅವನು ಮಾಡುವುದೇನು? ಆದಕಾರಣವೇ ಯೋಗಿಯು ಸಂಶಯಾವಸ್ಥೆಯನ್ನು ದಾಟಿರಬೇಕು. ಸಂಶಯ ಮಾಡುವುದು ಮನುಷ್ಯನ ಸ್ವಭಾವ. ಯೋಗಿಯಾದರೊ ಈ ಅವಸ್ಥೆಯನ್ನು ಮೀರಿ ಹೋಗಿರುವನು. ಇವನಿಗೆ ಇಂದ್ರಿಯಾತೀತವಾದ ಸತ್ಯದಲ್ಲಿ ದೃಢ ನಂಬಿಕೆ ಇದೆ. ಈ ನಂಬಿಕೆ ಇನ್ನುಮೇಲೆ ಜಾರುವಂತಿಲ್ಲ. ಸರ್ವಪ್ರಯತ್ನದಿಂದ ಭಗವಂತನ ಸಾಕ್ಷಾತ್ಕಾರವನ್ನು ಪಡೆಯಲು ಯತ್ನಿಸುವನು.

ಈ ಸಾಧನೆಯನ್ನು ಉತ್ಸಾಹದಿಂದ ಮಾಡಬೇಕು ಎನ್ನುವನು ಶ‍್ರೀಕೃಷ್ಣ. ಯೋಗಿಗೆ ಧ್ಯಾನ\-ವೊಂದು ಆಟ. ಅದೊಂದು ಕಾಟವಲ್ಲ. ಹುಡುಗರು ಹೇಗೆ ಸಂತೋಷದಿಂದ ಆಡುವರೊ ಹಾಗೆಯೇ ಧ್ಯಾನಮಾಡುವನು. ಧ್ಯಾನ ಅವನಿಗೆ ಒಂದು ಸ್ವಭಾವವಾಗಿದೆ. ಭಗವಂತನ ಕಡೆ ಮನಸ್ಸು ಹರಿದುಕೊಂಡು ಹೋಗುವುದೆ ಅವನಿಗೆ ಸ್ವಭಾವ ಆಗಿದೆ. ಇಲ್ಲಿ ಅದಮ್ಯ ಉತ್ಸಾಹ ಇರಬೇಕು. ಬಲಾತ್ಕಾರಕ್ಕಾಗಿ ಮಾಡಬೇಕಲ್ಲ ಎಂದು ಗೊಣಗಾಡಿ ಧ್ಯಾನಕ್ಕೆ ಕುಳಿತರೆ ನಿಮಿಷಗಳು ಗಂಟೆಗಳಾಗಿ ಪರಿಣಮಿಸುವುವು. ಎಲ್ಲೂ ಇಲ್ಲದ ಚಿತ್ತಚಾಂಚಲ್ಯದೊಂದಿಗೆ ಹೋರಾಡುತ್ತಿರ\-ಬೇಕಾಗುವುದು. ಆದರೆ ಯೋಗಿ ಈ ಪ್ರಥಮಘಟ್ಟದಿಂದ ಪಾರಾಗಿರುವನು. ಅವನಿಗೆ ಧ್ಯಾನ ಮಾಡುವುದು ಈಜು ಬಲ್ಲವನಿಗೆ ನೀರಿನಲ್ಲಿ ಈಜುವುದು ಎಷ್ಟು ಸುಲಭವೊ ಹಾಗಾಗಿದೆ. ದಾರಿಯಲ್ಲಿ ಏನೇ ಬರಲಿ ತಾನು ಗುರಿ ಮುಟ್ಟುವವರೆಗೆ ಬಿಡುವುದಿಲ್ಲ. ಹೋರಾಡುತ್ತೇನೆ, ಅದನ್ನು ಪಡೆಯುತ್ತೇನೆ ಎಂಬ ಭರವಸೆ ಇರಬೇಕು. ಆ ಹೋರಾಟದಲ್ಲಿ ಒಂದು ಆನಂದ ಉತ್ಪತ್ತಿ ಮಾಡಿಕೊಳ್ಳಬೇಕು. ಅದಮ್ಯ ಉತ್ಸಾಹ ಅತ್ಯಂತ ಆವಶ್ಯಕ, ಆಧ್ಯಾತ್ಮಿಕ ಜೀವನದ ಸಾಧನೆಯಲ್ಲಿ. ಶ‍್ರೀರಾಮಕೃಷ್ಣರು ಇದನ್ನು ವಿವರಿಸುವ ಒಂದೆರಡು ಉದಾಹರಣೆಗಳನ್ನು ಕೊಡುವರು. ಒಂದು ಸಲ ನಾರದರು ಒಂದು ಕಾಡಿನ ಮೂಲಕ ಹೋಗುತ್ತಿದ್ದರು. ಅಲ್ಲಿ ಇಬ್ಬರು ತಪಸ್ಸು ಮಾಡುತ್ತಿದ್ದರು. ಅವರು ನಾರದರನ್ನು ಕೇಳಿದರು, ಎಲ್ಲಿಗೆ ಹೊರಟಿರುವಿರಿ ಎಂದು. ಆಗ ನಾರದರು ನಾನು ವೈಕುಂಠಕ್ಕೆ ಹೋಗುತ್ತಿರುವೆ ಎಂದರು. ಅವರಲ್ಲಿ ಒಬ್ಬ ಯೋಗಿ ತುಂಬ ತಪಸ್ಸು ಮಾಡುತ್ತಿದ್ದವನು, ನೀವು ವೈಕುಂಠದಲ್ಲಿ ಶ‍್ರೀಮನ್ನಾರಾಯಣನನ್ನು ಕೇಳಿ, ನನಗೆ\break ಯಾವಾಗ ಮುಕ್ತಿ ಸಿಕ್ಕುವುದು, ನಾನು ಇನ್ನು ಎಷ್ಟು ಜನ್ಮಗಳು ತಪಸ್ಸು ಮಾಡಬೇಕಾಗಿದೆ ಎಂದು. ಇನ್ನೊಬ್ಬ ಕೂಡ ತಪಸ್ಸು ಮಾಡುತ್ತಿದ್ದ. ಆತ ಉತ್ಸಾಹಭರಿತನಾಗಿದ್ದ, ಸಂತೋಷಭರಿತನಾಗಿದ್ದ. ಅವನೂ ಕೂಡ ನಾರದರಿಗೆ ಶ‍್ರೀಮನ್ನಾರಾಯಣನಿಗೆ ತನ್ನ ಮುಕ್ತಿ ಯಾವಾಗ ಕೇಳಿಕೊಂಡು ಬನ್ನಿ ಎಂದು ಹೇಳಿ ಕಳುಹಿಸಿದ. ಸ್ವಲ್ಪ ಕಾಲದ ಮೇಲೆ ನಾರದರು ಆ ತಪಸ್ವಿಗಳ ಹತ್ತಿರ ಬಂದರು. ಮೊದಲನೆಯವನು, ನಾರದರೆ, ಕೇಳಿದಿರ ನನಗೆ ಯಾವಾಗ ಮುಕ್ತಿ ಸಿಕ್ಕುವುದು? ಎಂದನು. ನಾರದರು, ಕೇಳಿದೆ, ಅದಕ್ಕೆ ಭಗವಂತ ನೀನು ಇನ್ನೂ ಏಳು ಜನ್ಮಗಳು ತಪಸ್ಸು ಮಾಡಿದಮೇಲೆ ಮುಕ್ತಿ ಸಿಕ್ಕುವುದು ಎಂದು ಹೇಳಿದರು. ಆತನಿಗೆ ಆಗಲೆ ತಪಸ್ಸು ಮಾಡಿ ಸಾಕಾಗಿತ್ತು. ಅಯ್ಯೋ ಇನ್ನೂ ಏಳು ಜನ್ಮಗಳು ತಪಸ್ಸು ಮಾಡಬೇಕೆ ಎಂದು ಕೊರಗಿದ. ಮತ್ತೊಬ್ಬ ಹಸನ್ಮುಖಿಯಾದ ಯೋಗಿಯ ಹತ್ತಿರ ನಾರದರು ಹೋದರು. ಅವನಿಗೆ ಹೇಳಿದರು ನೋಡು ನೀನು ಯಾವ ಮರದ ಕೆಳಗೆ ಕುಳಿತುಕೊಂಡು ತಪಸ್ಸು ಮಾಡುತ್ತಿದ್ದೀಯೊ ಆ ಮರದಲ್ಲಿ ಎಷ್ಟು ಎಲೆಗಳಿವೆಯೊ ಅಷ್ಟು ಜನ್ಮಗಳು ಆದಮೇಲೆ ನಿನಗೆ ಮುಕ್ತಿ ಸಿಕ್ಕುವುದಂತೆ ಎಂದು ಹೇಳಿದರು. ಇದನ್ನು ಕೇಳಿದೊಡನೆ ಅವನು ಸಂತೋಷದಿಂದ ಕುಣಿದಾಡಲು ಪ್ರಾರಂಭಿಸಿದ. ಅಂತೂ ತನಗೆ ಮುಕ್ತಿ ಸಿಕ್ಕುವುದಲ್ಲ ಎಂದು ಆನಂದಭರಿತನಾದ. ತಕ್ಷಣವೇ ಭಗವಂತ ಅವನಿಗೆ ಮಗು ನೀನು ಈ ಕ್ಷಣ ಮುಕ್ತ ಎಂದನು. ಮೊದಲನೆಯವನು ಏಳು ಜನ್ಮಗಳು ತಪಸ್ಸು ಮಾಡಬೇಕಾಗಿದೆ ಎಂಬುದನ್ನು ಕೇಳಿದಾಗ ಕೊರಗಿಹೋದ. ಎರಡನೆಯವನು ಮರದ ಎಲೆಗಳಷ್ಟು ಜನ್ಮವೆತ್ತಿ ಬರಬೇಕೆಂದಾಗ ಅದಕ್ಕೆ ಕುಣಿದಾಡಿದ. ಏಕೆಂದರೆ ಅವನಿಗೆ ಭಗವಂತನಿಗಾಗಿ ತಪಸ್ಸುಮಾಡುವುದೊಂದು ಆನಂದ ಮತ್ತು ಕೊನೆಗಾದರೂ ತಾನು ಗುರಿ ಮುಟ್ಟುತ್ತೇನಲ್ಲ ಎಂಬ ಶ್ರದ್ಧೆ ಇದೆ. ಇಂತಹ ಮನೋಭಾವ ಯೋಗಿಗೆ ಮುಖ್ಯ. ಜೀವನದಲ್ಲಿ ಕ್ಷುದ್ರ ವಸ್ತುಗಳೆ ನಮಗೆ ಸುಲಭವಾಗಿ ಸಿಕ್ಕುವುದಿಲ್ಲ. ಅದಕ್ಕಾಗಿ ಎಷ್ಟೊಂದು ಹೋರಾಡುತ್ತೇವೆ. ಪಡಬಾರದ ಯಾತನೆ ಪಡುತ್ತೇವೆ, ಕೇಳಬಾರದ ಮಾತು ಕೇಳುತ್ತೇವೆ, ಕಣ್ಣೀರಿನ ಕೋಡಿಯಲ್ಲಿ ಈಜುತ್ತೇವೆ. ಆದರೂ ನಾವೇನೂ ಅದನ್ನು ಬೆನ್ನಟ್ಟುವುದನ್ನು ಬಿಡುವುದಿಲ್ಲ. ನಮ್ಮ ಬಾಳನ್ನೇ ಪಾವನವನ್ನಾಗಿ ಮಾಡುವ, ನಮ್ಮನ್ನು ಕೃತಕೃತ್ಯರನ್ನಾಗಿ ಮಾಡುವ, ಜನನ ಮರಣಗಳ ಚಕ್ರದಿಂದ ತಪ್ಪಿಸಿಕೊಂಡು ಹೋಗುವಂತೆ ಮಾಡುವ ಭಗವತ್ ಸಾಕ್ಷಾತ್ಕಾರವನ್ನು ಪಡೆಯಬೇಕಾದರೆ ನಾವು ಶ್ರದ್ಧೆಯನ್ನು ತೋರಬೇಕು. ಉತ್ಸಾಹ ಮತ್ತು ಶ್ರದ್ಧೆ ಇವೇ ಅತ್ಯಂತ ಮುಖ್ಯವಾಗಿ ಬೇಕಾಗಿರುವ ಬುತ್ತಿ, ಯೋಗಿಯ ಪ್ರಯಾಣದಲ್ಲಿ.

\begin{shloka}
ಸಂಕಲ್ಪಪ್ರಭವಾನ್ ಕಾಮಾಂಸ್ತ್ಯಕ್ತ್ವಾಸರ್ವಾನಶೇಷತಃ~।\\ಮನಸೈವೇಂದ್ರಿಯಗ್ರಾಮಂ ವಿನಿಯಮ್ಯ ಸಮಂತತಃ \hfill॥ ೨೪~॥
\end{shloka}

\begin{shloka}
ಶನೈಃಶನೈರುಪರಮೇದ್ಬುದ್ಧ್ಯಾ ಧೃತಿಗೃಹೀತಯಾ~।\\ಆತ್ಮಸಂಸ್ಥಂ ಮನಃ ಕೃತ್ವಾ ನ ಕಿಂಚಿದಪಿ ಚಿಂತಯೇತ್ \hfill~॥ ೨೫~॥
\end{shloka}

\begin{artha}
ಸಂಕಲ್ಪದಿಂದ ಉತ್ಪನ್ನವಾದ ಎಲ್ಲ ಕಾಮಗಳನ್ನೂ ಸಂಪೂರ್ಣವಾಗಿ ತ್ಯಜಿಸಿ, ಮನಸ್ಸಿನಿಂದಲೇ ಇಂದ್ರಿಯಗಳ ಸಮೂಹವನ್ನೂ ಎಲ್ಲ ಕಡೆಯಿಂದಲೂ ನಿಗ್ರಹಿಸಿ, ಧೈರ್ಯದಿಂದ ಕೂಡಿದ ಬುದ್ಧಿಯಿಂದ ಮೆಲ್ಲಮೆಲ್ಲಗೆ ಹಿಂತಿರುಗಬೇಕು. ಮನಸ್ಸನ್ನು ಆತ್ಮದಲ್ಲಿ ನಿಲ್ಲಿಸಿ ಏನನ್ನೂ ಚಿಂತಿಸಕೂಡದು.
\end{artha}

ಸಂಕಲ್ಪಗಳಿಂದ ಉತ್ಪನ್ನವಾದ ಎಲ್ಲ ವಿಧವಾದ ಕಾಮನೆಗಳು: ಸಂಕಲ್ಪ ಎಂಬುದು ಬೇರು, ಅದರಿಂದ ಆಸೆಗಳ ಚಿಗುರುಗಳೆಲ್ಲ ಬರುವುವು. ಅವು ಕಾರ್ಯರೂಪಕ್ಕೆ ಬರುವುದಕ್ಕೆ ಮುಂಚೆಯೇ ಅದರ ಸುಳಿವು ಗೊತ್ತಾದಾಗಲೇ ಅವುಗಳನ್ನು ನಿಗ್ರಹಿಸಬೇಕು. ಮೊದಲಲ್ಲಿ ನಿಗ್ರಹಿಸುವುದು ಸುಲಭ. ಆಗತಾನೆ ಚಿಗುರಿದ ಗಿಡವನ್ನು ಉಗುರಿನಿಂದ ಕಿತ್ತುಬಿಡಬಹುದು. ಆದರೆ ಅದು ಚೆನ್ನಾಗಿ ಬೆಳೆದರೆ, ಕೊಡಲಿಯೇ ಬೇಕಾಗುವುದು. ಅದಕ್ಕಾಗಿಯೇ ಕಾಮನೆಗಳನ್ನು ಪ್ರಾರಂಭದಲ್ಲಿಯೇ ನಿಗ್ರಹಿಸಬೇಕು ಎನ್ನುವುದು. ನಾರದರು ಭಕ್ತಿಸೂತ್ರದಲ್ಲಿ ಮೊದಲು ಅದು ಸಣ್ಣ ತರಂಗದಂತೆ ಕಾಣುವುದು, ಕೊನೆಗೆ ಅದೊಂದು ಸಮುದ್ರವಾಗುವುದು ಎನ್ನುವರು.

ಎಲ್ಲಾ ಕಾಮನೆಗಳನ್ನು ತ್ಯಜಿಸಬೇಕು ಎನ್ನುವನು ಶ‍್ರೀಕೃಷ್ಣ. ಕಾಮನೆಗಳಲ್ಲಿ ಒಳ್ಳೆಯವು ಕೆಟ್ಟವುಗಳೆಲ್ಲಾ ಇರುವುವು. ಕೆಟ್ಟ ಕಾಮನೆ ನಮ್ಮನ್ನು ಎಷ್ಟುಮಟ್ಟಿಗೆ ಪ್ರಪಂಚಕ್ಕೆ ಕಟ್ಟಿಹಾಕುವುದೊ ಅಷ್ಟೇ ಒಳ್ಳೆಯ ಕಾಮನೆಯೂ ಕಟ್ಟಿಹಾಕುವುದು. ಒಳ್ಳೆಯದು ಚಿನ್ನದ ಸರಪಳಿ, ಕೆಟ್ಟದ್ದು ಕಬ್ಬಿಣದ ಸರಪಳಿ. ಕಟ್ಟಿನ ದೃಷ್ಟಿಯಿಂದ ಎಲ್ಲ ಒಂದೇ. ಯೋಗಿಯಾಗಬಯಸುವವನು ನಿರ್ದಯನಾಗಿ ಇವುಗಳಿಂದ ಪಾರಾಗಬೇಕು.

ಸಂಪೂರ್ಣವಾಗಿ ಆಸೆಯನ್ನು ತ್ಯಜಿಸಬೇಕು. ಎಲ್ಲಾ ಜಯಿಸಿ ಎಲ್ಲೊ ಸ್ವಲ್ಪ ಬಿಟ್ಟರೂ, ಆ ಕಿಡಿ ಅನಂತರ ಕಾಡ್ಗಿಚ್ಚಾಗುವುದು. ಬಲ್ಲವರು, ರೋಗಶೇಷ, ಶತ್ರುಶೇಷ, ಬೆಂಕಿಶೇಷಗಳನ್ನು ಸಂಪೂರ್ಣ ನಿರ್ನಾಮ ಮಾಡಬೇಕು ಎನ್ನುವರು. ಯಾವಾಗ ನಾವು ಶೇಷವೆಂದು ಅಸಡ್ಡೆಯನ್ನು ತೋರುವೆವೊ ಆ ಶೇಷವೇ ಬೃಹದಾಕಾರವನ್ನು ತಾಳುವುದು. ನಮ್ಮಲ್ಲಿರುವ ಹೀನ ಸಂಸ್ಕಾರಗಳು ಬೆಂಕಿಯ ಕಿಡಿಯಂತೆ ಜಾಗೃತವಾಗಿ ಅವು ಪುನಃ ಹೊತ್ತಿಕೊಂಡು ಉರಿಯುವುದಕ್ಕೆ ಸ್ವಲ್ಪ ಒಣಗಿದ ಕಡ್ಡಿಯ ಸಂಗ ಬೇಕಾಗಿದೆ. ನಿತ್ಯಜೀವನದಲ್ಲಿ ನಾವು ಅಜಾಗರೂಕತೆಯಿಂದ ಸ್ವಲ್ಪ ವ್ಯವಹರಿಸಿದರೂ ಅವು ಭಗ್ ಎಂದು ಹೊತ್ತಿಕೊಳ್ಳುವ ಪ್ರಸಂಗಗಳಿವೆ.

ಮನಸ್ಸಿನಿಂದಲೆ ಇಂದ್ರಿಯಗಳ ಸಮೂಹವನ್ನು ಎಲ್ಲ ಕಡೆಯಿಂದಲೂ ನಿಗ್ರಹಿಸಬೇಕು: ದನಗಳು ಹೇಗೆ ಹೊರಗೆ ಮೈದಾನದಲ್ಲಿ ಒಂದೊಂದು ಒಂದೊಂದು ಕಡೆ ಮೇಯುತ್ತವೆಯೊ ಹಾಗೆ ಇಂದ್ರಿಯಗಳು ವಿಷಯವಸ್ತುವಿನಲ್ಲಿ ನಿರತವಾಗಿವೆ. ಸಂಜೆ ಮನೆಗೆ ದನ ಹೊಡೆದುಕೊಂಡು ಹೋಗುವವನು, ಇವುಗಳನ್ನೆಲ್ಲ ಒಂದು ಕಡೆ ಸೇರಿಸಿ ಊರಿಗೆ ಕರೆದುಕೊಂಡು ಹೋಗುವನು. ಅದರಂತೆಯೆ ಯೋಗಿ ಮನಸ್ಸಿನಿಂದ ಇಂದ್ರಿಯಗಳನ್ನೆಲ್ಲ ನಿಗ್ರಹಿಸಬೇಕು. ಇಂದ್ರಿಯಗಳು ವಿಷಯ ವಸ್ತುವಿನ ಕಡೆ ಚೆದುರಿಹೋಗುವುದಕ್ಕೆ ಬಿಡಕೂಡದು. ಮನಸ್ಸಿನಿಂದಲೆ ಹಿಂತಿರುಗಿಸಬೇಕು.

ಧೈರ್ಯದಿಂದ ಕೂಡಿದ ಬುದ್ಧಿಯಿಂದ ಮೆಲ್ಲಮೆಲ್ಲನೆ ಹಿಂತಿರುಗಬೇಕು. ನಾವು ಇಂದ್ರಿಯಗಳನ್ನು ವಿಷಯವಸ್ತುವಿನಿಂದ ಎಳೆಯುವಾಗ ಧೈರ್ಯವಿರಬೇಕು. ವಿಷಯವಸ್ತುಗಳು ಕೂಡ ಇಂದ್ರಿಯವನ್ನು ತನ್ನ ಕಡೆಗೆ ಸೆಳೆಯುತ್ತಿವೆ. ಅವುಗಳಿಂದ ಪಾರಾಗಬೇಕಾದರೆ ಬುದ್ಧಿ ಎಂದರೆ ತಿಳಿವಳಿಕೆ ಧೈರ್ಯವಾಗಿರಬೇಕು. ವಿಷಯವಸ್ತುವಿನ ಸಂಗ ಮನಸ್ಸನ್ನು ಅಧೋಗತಿಗೆ ತರುವುದೆಂದು ಅದಕ್ಕೆ ಒತ್ತಿ ಹೇಳಬೇಕು. ಅದನ್ನು ಮೆಲ್ಲಮೆಲ್ಲಗೆ ಬಿಡಿಸಿಕೊಳ್ಳಬೇಕು. ಅವಾಂತರದಿಂದ ಕೆಲಸ ಮಾಡಿದರೆ ಸಾಧ್ಯವಾಗುವುದಿಲ್ಲ. ಜಾಣತನದಿಂದ, ತಾಳ್ಮೆಯಿಂದ ಆ ಕೆಲಸವನ್ನು ಮಾಡಬೇಕು. ದಾರಿಯಲ್ಲಿ ಹೋಗುವಾಗ ಮುಳ್ಳಿನ ಪೊದೆಗೆ ನಮ್ಮ ಬಟ್ಟೆ ತಾಕಿದೆ ಎಂದು ಭಾವಿಸೋಣ. ಸುಮ್ಮನೆ ಬಟ್ಟೆಯನ್ನು ಎಳೆದರೆ ಮುಳ್ಳಿನಿಂದ ತಪ್ಪಿಸಿಕೊಳ್ಳುವುದಕ್ಕಾಗುವುದಿಲ್ಲ. ಅಂಟಿರುವ ಮುಳ್ಳನ್ನು ಒಂದೊಂದಾಗಿ ಬಿಡಿಸಿಕೊಳ್ಳಬೇಕು. ಅನಂತರವೆ ಎಲ್ಲಾ ಮುಳ್ಳುಗಳಿಂದ ಪಾರಾಗಬಹುದು.

ಕೊನೆಗೆ ಮನಸ್ಸನ್ನು ಪರಮಾತ್ಮನಲ್ಲಿಡಬೇಕು. ಪರಮಾತ್ಮನೆಂಬ ಗೂಟಕ್ಕೆ ಮನಸ್ಸನ್ನು ಕಟ್ಟಿಹಾಕಬೇಕು. ಯಾವ ಲೌಕಿಕವಾದವುಗಳನ್ನೂ ಚಿಂತಿಸಕೂಡದು. ದೇವರಿಗೆ ಸಂಬಂಧಪಟ್ಟ ಚಿಂತನೆ\-ಯಲ್ಲಿಯೆ ಮನಸ್ಸು ಮುಳಗಬೇಕು. ನದಿ ಹೇಗೆ ಸಮುದ್ರವನ್ನು ಸೇರುತ್ತಿರುವುದೊ ಹಾಗೆ ಪರಮಾತ್ಮನನ್ನು ಸೇರುತ್ತಿರಬೇಕು.

\begin{shloka}
ಯತೋ ಯತೋ ನಿಶ್ಚರತಿ ಮನಶ್ಚಂಚಲಮಸ್ಥಿರಮ್~।\\ತತಸ್ತತೋ ನಿಯಮ್ಯೈತದಾತ್ಮನ್ಯೇವ ವಶಂ ನಯೇತ್ \hfill॥ ೨೬~॥
\end{shloka}

\begin{artha}
ಚಂಚಲವೂ ಅಸ್ಥಿರವೂ ಆದ ಮನಸ್ಸು ಯಾವಯಾವ ವಿಷಯದಿಂದ ಹೊರಗೆ ಬರುವುದೊ ಆಯಾ ವಿಷಯದಿಂದ ಅದನ್ನು ನಿಗ್ರಹಿಸಿ ಆತ್ಮವಶವಾಗುವಂತೆ ಮಾಡಬೇಕು.
\end{artha}

ಮನಸ್ಸಿನ ಸ್ವಭಾವವೆ ಚಂಚಲ. ಯಾವಾಗಲೂ ಒಂದರಿಂದ ಮತ್ತೊಂದಕ್ಕೆ ನೆಗೆಯುತ್ತಿರುವುದು. ಅದರಲ್ಲಿಯೂ ಯಾವಾಗ ಅದನ್ನು ಧ್ಯಾನಮಾಡು ಎಂದು ಕಟ್ಟಿಹಾಕುವೆವೊ ಆಗಲೆ ಅದು ಕಿತ್ತುಕೊಂಡು ಹೋಗುವುದು. ಅದರಂತೆಯೆ ಅದು ಅಸ್ಥಿರ. ಒಮ್ಮೆ ಒಂದು ಶಪಥ ಮಾಡುವುದು. ಇನ್ನು ಸ್ವಲ್ಪ ಹೊತ್ತಿನಲ್ಲಿಯೆ ಆ ಶಪಥವನ್ನು ಮುರಿದಿರುವುದು. ನಮ್ಮ ಜೀವನವೆ ಮುರಿದ ಶಪಥಗಳ ರಾಶಿ. ಒಂದೊಂದು ಸಲ ಒಂದೊಂದು ನಿರ್ಧಾರಕ್ಕೆ ಬರುವುದು. ಅಂಕೆಯಿಂದ ತಪ್ಪಿಸಿಕೊಂಡು ಬರುವುದಕ್ಕೆ ಏನಾದರೂ ಒಂದು ನೆವ ಬೇಕು. ಒಂದು ಕಡೆ ಹೋಗಬೇಡ ಎಂದು ಬಾಗಿಲು ಹಾಕಿದರೆ ಹಿಂದಿನಿಂದ ಹೋಗುವುದು. ಅದು ಕೆಟ್ಟದ್ದು, ಅದರ ಹತ್ತಿರ ಹೋಗಬೇಡ ಎಂದರೆ, ಅದು ಒಳ್ಳೆಯದರಂತೆ ತೋರುವುದು ಎಂದು ಹಲವು ಕಾರಣಗಳನ್ನು ಕೊಡುವುದು. ವಿಷಯವಸ್ತುವಿನ ಸಹವಾಸವನ್ನು ಅದು ಬಹಳ ದಿನದಿಂದ ಮಾಡಿದೆ. ಅದನ್ನು ಒಂದೇ ಸಲ ಬಿಟ್ಟುಹೋಗು ಎಂದರೆ ಹೋಗಿಬಿಡುವುದಿಲ್ಲ. ಪದೇ ಪದೇ ಅದು ವಿಷಯವಸ್ತುವಿನ ಕಡೆ ಬರುತ್ತಿರುವುದು. ಪ್ರತಿಯೊಂದು ಸಲ ಬಂದಾಗಲೂ ಅದಕ್ಕೆ ಬುದ್ಧಿ ಹೇಳಿ ಛೀಮಾರಿ ಮಾಡಿ ಹಿಂದಕ್ಕೆ ತೆಗೆದುಕೊಂಡು ಹೋಗಿ ದೇವರ ಸನ್ನಿಧಿಯಲ್ಲಿ ಬಿಡಬೇಕು. ಅದು ಇನ್ನೂ ಭಗವಂತನ ಆನಂದ ಆಸ್ವಾದನೆಗೆ ಒಗ್ಗಿಲ್ಲ. ಅದನ್ನು ಕ್ರಮೇಣ ಒಗ್ಗಿಸಿಕೊಳ್ಳಬೇಕಾಗಿದೆ. ಅಲ್ಲಿಯವರೆಗೂ ನಾವು ಪದೇ ಪದೇ ಅದರ ಹಿಂದೆ ಓಡಿ ಹಿಡಿದುಕೊಂಡು ಬರಬೇಕಾಗುವುದು. ಅಭ್ಯಾಸಬಲದಿಂದ ಮಾತ್ರ ಏಕಾಗ್ರತೆ ನಮಗೆ ಸ್ವಾಧೀನವಾಗುವುದು. ಇದ್ದಕ್ಕಿದ್ದಂತೆ ಇದು ಯಾರಿಗೂ ಬರುವುದಿಲ್ಲ. ಬರುವುದಿಲ್ಲ ಎಂದರೆ ನಾವು ನೆಚ್ಚುಗೆಡ ಕೂಡದು, ಜುಗುಪ್ಸೆ ಪಡಬಾರದು. ಅಭ್ಯಾಸವನ್ನು ಮುಂದುವರಿಸಿಕೊಂಡು ಹೋಗಬೇಕು.

\begin{shloka}
ಪ್ರಶಾಂತಮನಸಂ ಹ್ಯೇನಂ ಯೋಗಿನಂ ಸುಖಮುತ್ತಮಮ್~।\\ಉಪೈತಿ ಶಾಂತರಜಸಂ ಬ್ರಹ್ಮಭೂತಮಕಲ್ಮಶಮ್ \hfill॥ ೨೭~॥
\end{shloka}

\begin{artha}
ಪ್ರಶಾಂತಚಿತ್ತನೂ, ರಜೋಗುಣವಿಲ್ಲದವನೂ, ಕಶ್ಮಲರಹಿತನೂ, ಬ್ರಹ್ಮಭಾವವನ್ನು ಹೊಂದಿದವನೂ ಆದ ಆ ಯೋಗಿಯನ್ನು ಉತ್ತಮ ಸುಖ ಬಂದು ಸೇರುವುದು.
\end{artha}

ಯಾವ ಯೋಗಿ ಪ್ರಶಾಂತಚಿತ್ತನಾಗಿರುವನೋ, ಎಂದರೆ ಮನಸ್ಸನ್ನು ಸ್ಥಿರಮಾಡಿಕೊಂಡಿರು ವನೋ, ಅಲ್ಲಿ ಯಾವ ವಿಧವಾದ ಉದ್ವಿಗ್ನತೆಗೂ ಅವಕಾಶವಿರುವುದಿಲ್ಲ. ಅಲ್ಲಿ ರಜೋಗುಣವಿಲ್ಲ. ಮನಸ್ಸಿನ ಬಾಹ್ಯ ಪ್ರವೃತ್ತಿ ಇರುವುದಿಲ್ಲ. ಅವನು ಕರ್ಮದ ಸುಂಟರಗಾಳಿಯಲ್ಲಿ ಸಿಕ್ಕಿಕೊಂಡಿರುವುದಿಲ್ಲ. ರಜೋಗುಣವಿದ್ದರೆ ಅವನನ್ನು ಯಾವಾಗಲೂ ತಿವಿಯುತ್ತಿರುವುದು ಏನನ್ನಾದರೂ ಮಾಡುವುದಕ್ಕೆ. ಯಾವಾಗ ಅದು ನಾಶವಾಗಿದೆಯೊ ಅವನು ಸೌಮ್ಯವಾಗಿರುವನು. ಅವನ ಮನಸ್ಸಿನಲ್ಲಿ ಯಾವ ಕಲ್ಮಶವೂ ಇಲ್ಲ. ಯಾವ ಅಸೂಯೆಯಾಗಲಿ, ಆಸೆಯಾಗಲಿ ಇಲ್ಲ. ಇವುಗಳೆಲ್ಲ ಮನಸ್ಸನ್ನು ಮುತ್ತಿರುವ ಕಿಲುಬು. ಯೋಗಿ ಇವುಗಳ ಕೊಳೆಯಿಂದ ಪಾರಾಗಿರುವನು. ಪರಿಶುದ್ಧಮಾಡಿದ ಚಿನ್ನದಲ್ಲಿ ಬೆರಕೆ ಯಾವುದು ಹೇಗೆ ಇಲ್ಲವೊ ಅದರಂತೆಯೇ ಯೋಗಿಯ ಕಿಲ್ಬಿಶರಹಿತ ಮನಸ್ಸು.

ಅವನ ಮನಸ್ಸು ಬ್ರಹ್ಮಭಾವವನ್ನು ಹೊಂದುವುದು: ಇದ್ದಿಲನ್ನು ಬೆಂಕಿಗೆ ಹಾಕಿದರೆ ಅದೂ ಕೂಡ ಬೆಂಕಿಯೇ ಆಗುವುದು. ಅದರಂತೆಯೇ ಪರಮಾತ್ಮನನ್ನು ಧ್ಯಾನಿಸುತ್ತಿರುವ ಯೋಗಿಯ ಮನಸ್ಸು. ಅವನ ಬಳಿಗೆ ಶ್ರೇಷ್ಠವಾದ ಸುಖ ತಾನಾಗಿ ಬರುವುದು. ಇವನು ಸುಖವನ್ನು\break ಹುಡುಕಿಕೊಂಡು ಹೋಗಬೇಕಾಗಿಲ್ಲ. ಸುಖ ಇವನನ್ನು ಹುಡುಕಿಕೊಂಡು ಬರುವುದು. ನಾವು ಮನಸ್ಸನ್ನು ಅಣಿಮಾಡಿಕೊಂಡರೆ ಆ ಸುಖ ಆಗಲೆ ಎಲ್ಲಾ ಜೀವರಾಶಿಗಳಲ್ಲೂ ಹರಿಯುತ್ತಿರುವುದು ಗೊತ್ತಾಗುವುದು. ನದೀಮುಖದಲ್ಲಿ ಮೇಲೆ ಮುತ್ತಿರುವ ಮರಳನ್ನು ತೆಗೆದರೆ ಹೇಗೆ ಕೆಳಗಿನಿಂದ ತಿಳಿನೀರು ಹಳ್ಳದಲ್ಲಿ ಬಂದು ಸಂಗ್ರಹಿಸುವುದೊ ಹಾಗೆ ಪರಮಸುಖ ಯೋಗಿಯ ಮನಸ್ಸನ್ನು ಪ್ರವೇಶಿಸುವುದು. ಪರಮ ಸುಖವೆಂಬ ಪರಮಾತ್ಮನ ಆನಂದಸಾಗರದಿಂದ ಪ್ರತಿಯೊಂದು ಜೀವರಾಶಿಗೂ ಒಂದು ನಾಲೆ ಇದೆ. ಆ ನಾಲೆಯಲ್ಲಿ ಅಡಚಣೆಗಳಿವೆ. ಅದಕ್ಕಾಗಿ ನೀರು ಬರುತ್ತಿಲ್ಲ. ಯಾವಾಗ ಅಡಚಣೆಯನ್ನು ತೆರೆಯುವನೊ ಆಗ ನೀರು ಪ್ರತಿಯೊಂದು ಜೀವಿಯ ಕ್ಷೇತ್ರಕ್ಕೂ ನುಗ್ಗಿ ಬರುವುದು. ಬಾಹ್ಯ ಪ್ರಪಂಚದಲ್ಲಿ ಸುಖವನ್ನು ಹುಡುಕಿಕೊಂಡು ಹೋಗುವವನಿಗೆ ದುಃಖಮಿಶ್ರ ಸುಖ, ಅದೂ ಕ್ಷಣಿಕವಾಗಿರುವುದು ಸಿಕ್ಕುವುದು. ಯೋಗಿ ಹೊರಗೆ ಹೋಗುವುದಿಲ್ಲ. ಇಂದ್ರಿಯ ಮತ್ತು ಮನಸ್ಸನ್ನು ನಿಗ್ರಹಿಸಿ ಭಗವಂತನ ಕಡೆ ಮನಸ್ಸು ತಿರುಗಿಸುವನು. ಆಗ ಶ್ರೇಷ್ಠಸುಖ ಇವನ ಬಳಿಗೆ ಹರಿದುಬರುವುದು. ಆ ಸುಖಕ್ಕೆ ದುಃಖಮಿಶ್ರವಿಲ್ಲ. ಅದು ಕ್ಷಣಿಕವಲ್ಲ. ಶಾಶ್ವತವಾದ ಪರಮಾತ್ಮಾನಂದವದು.

\begin{shloka}
ಯುಂಜನ್ನೇವಂ ಸದಾತ್ಮಾನಂ ಯೋಗೀ ವಿಗತಕಲ್ಮಷಃ~।\\ಸುಖೇನ ಬ್ರಹ್ಮಸಂಸ್ಪರ್ಶಮತ್ಯಂತಂ ಸುಖಮಶ್ನುತೇ \hfill॥ ೨೮~॥
\end{shloka}

\begin{artha}
ಹೀಗೆ ಆತ್ಮನನ್ನು ಯಾವಾಗಲೂ ಯೋಗದಲ್ಲಿಟ್ಟುಕೊಂಡು ಕಲ್ಮಶರಹಿತನಾದ ಯೋಗಿ ಆಯಾಸವಿಲ್ಲದೆ ಬ್ರಹ್ಮಸಂಸ್ಪರ್ಶವೆಂಬ ಉತ್ಕೃಷ್ಟವಾದ ಸುಖವನ್ನು ಹೊಂದುತ್ತಾನೆ.
\end{artha}

ಯೋಗಿ ಧ್ಯಾನದಲ್ಲಿರುವಾಗ ಮಾತ್ರ ಭಗವಂತನನ್ನು ಚಿಂತಿಸುತ್ತಿದ್ದು ಪ್ರಕೃತಿಸ್ಥನಾದ ಮೇಲೆ ದೇವರನ್ನು ಮರೆಯುವುದಿಲ್ಲ. ಅವನು ಧ್ಯಾನಮಾಡದೇ ಇದ್ದರೂ ಮನಸ್ಸಿನ ಒಂದು ಭಾಗ ಯಾವಾಗಲೂ ಧ್ಯಾನಾವಸ್ಥೆಯಲ್ಲಿಯೇ ಇರುವುದು. ಭಗವಂತನ ಅನುಭವದ ಹಿನ್ನೆಲೆ ಅವನನ್ನು\break ಯಾವಾಗಲೂ ಬಿಟ್ಟಿರುವುದಿಲ್ಲ. ಹೇಗೆ ಉತ್ತರಮುಖಿ ಯಾವಾಗಲೂ ಉತ್ತರ ದಿಕ್ಕನ್ನೆ ತೋರುತ್ತಿರು\-ವುದೋ ಹಾಗೆ ಯೋಗಿಯ ಮನಸ್ಸು ಭಗವಂತನ ಕಡೆ ತಿರುಗಿರುವುದು.

ಹೇಗೆ ಬೆಂಕಿಯಲ್ಲಿ ಕಾಯುತ್ತಿರುವ ವಸ್ತುವಿನಲ್ಲಿರುವ ಕ್ರಿಮಿಗಳು ನಾಶವಾಗುತ್ತವೆಯೊ ಹಾಗೆ ಪರಮಾತ್ಮನ ಅನುಭವವೆಂಬ ಮೂಸೆಯಲ್ಲಿ ಕಾಯುತ್ತಿರುವ ಯೋಗಿಯ ಮನಸ್ಸಿನಲ್ಲಿರುವ ಕಲ್ಮಶ\-ವೆಲ್ಲ ನಾಶವಾಗುವುದು. ಇಂತಹ ಯೋಗಿ ಆಯಾಸವಿಲ್ಲದೆ ಬ್ರಹ್ಮಸಂಸ್ಪರ್ಶವೆಂಬ ಯೋಗವನ್ನು ಪಡೆಯುತ್ತಾನೆ. ಅವನು ಅನಂತರ ಭಗವಂತನನ್ನು ಪಡೆಯುವುದಕ್ಕೆ ಹೆಚ್ಚು ಬಳಲಬೇಕಾಗಿಲ್ಲ. ಪರಮಾತ್ಮನ ಕಡೆ ಹೋಗುವುದು ಯೋಗಿಯ ಮನಸ್ಸಿನ ಒಂದು ಸ್ವಭಾವವಾದರೆ ಅವನು ಸುಲಭವಾಗಿ ಗುರಿ ಸೇರುವನು. ಗಾಳಿ ಬೀಸುತ್ತಿದ್ದರೆ, ದೋಣಿ ತನ್ನ ಧ್ವಜವನ್ನು ಹರಡಿದರೆ, ಆ ಬೀಸುವ ಗಾಳಿಯ ಸಹಾಯದಿಂದ ಹೇಗೆ ಗುರಿಯನ್ನು ಮುಟ್ಟುವುದೋ ಹಾಗೆ ಯೋಗಿ ಸುಲಭವಾಗಿ ಅನಂತರ ಗುರಿಯನ್ನು ಮುಟ್ಟುವನು.

ಯೋಗಿ ಬ್ರಹ್ಮಸಂಸ್ಪರ್ಶ ಎಂಬ ಉತ್ಕೃಷ್ಟವಾದ ಸುಖವನ್ನು ಪಡೆಯುತ್ತಾನೆ. ಹೇಗೆ ಸ್ಪರ್ಶ ಮಣಿಗೆ ಯಾವುದಾದರೂ ಹೀನಲೋಹ ತಾಕಿದರೆ ಅದೂ ಕೂಡ ಚಿನ್ನವಾಗುವುದೊ ಅದರಂತೆಯೇ, ಯೋಗಿ ಬ್ರಹ್ಮನನ್ನು ಪಡೆದಮೇಲೆ ಅವನಂತೆಯೇ ಆಗುತ್ತಾನೆ. ಅವನ ಧರ್ಮವೆ, ಇವನಿಗೆ ಬರುವುದು, ನದಿ ಸಾಗರವನ್ನು ಸೇರಿದಮೇಲೆ ಸಾಗರವೇ ಆಗುವಂತೆ. ಇದು ಉತ್ಕೃಷ್ಟವಾದ ಸುಖ, ಪರಮಾತ್ಮನನ್ನು ಮುಟ್ಟಿ ಬರುವ ಸುಖ. ನಾವು ಇಂದ್ರಿಯವಸ್ತುಗಳನ್ನು ಮುಟ್ಟಿಬರುವ ಸುಖದ ರುಚಿ ನೋಡಿರುವೆವು. ಅದೇನು ಎಂಬುದು ಗೊತ್ತಿದೆ. ಅದೆಷ್ಟು ಕ್ಷಣಿಕ ಎಂಬುದು ಗೊತ್ತಿದೆ. ಅದು ಎಂತಹ ದುಃಖದಲ್ಲಿ ಪರ್ಯವಸಾನವಾಗುವುದು ಎಂಬುದೂ ಗೊತ್ತಿದೆ. ಹೀಗಲ್ಲ ಪರಮಾತ್ಮನ ಸಂಗದಿಂದ ಬರುವ ಸುಖ. ಇದು ಶಾಶ್ವತವಾದುದು, ಅಮೃತಸ್ವರೂಪವಾದುದು, ದುಃಖದ ಗಂಧವೇ ಇಲ್ಲದುದು.

\begin{shloka}
ಸರ್ವಭೂತಸ್ಥಮಾತ್ಮಾನಂ ಸರ್ವಭೂತಾನಿ ಚಾತ್ಮನಿ~।\\ಈಕ್ಷತೇ ಯೋಗಯುಕ್ತಾತ್ಮಾ ಸರ್ವತ್ರ ಸಮದರ್ಶನಃ \hfill॥ ೨೯~॥
\end{shloka}

\begin{artha}
ಎಲ್ಲೆಲ್ಲಿಯೂ ಸಮದರ್ಶಿಯಾಗಿರುವ ಯೋಗದಿಂದ ಕೂಡಿರುವವನು, ಆತ್ಮನನ್ನು ಸರ್ವಭೂತ\-ಗಳಲ್ಲಿಯೂ ಸರ್ವಭೂತಗಳನ್ನು ಆತ್ಮನಲ್ಲಿಯೂ ಕಾಣುತ್ತಾನೆ.
\end{artha}

ಯೋಗದಿಂದ ಕೂಡಿರುವವನು ಎಲ್ಲೆಲ್ಲಿಯೂ ಒಂದೇ ಸಮನಾಗಿ ನೋಡುತ್ತಿರುವನು.\break ಅವನು ನೋಡುವುದು ಬಾಹ್ಯದೃಷ್ಟಿಯಿಂದಲ್ಲ. ಹೊರಗಿನದು ಕೇವಲ ನಾಮರೂಪಗಳು ಮಾತ್ರವೆ. ಯೋಗಿ ನಾಮರೂಪವನ್ನೇ ಭೇದಿಸಿ ಹೋಗುತ್ತಾನೆ. ಯಾವಾಗ ನಾಮರೂಪಗಳನ್ನು ತೂರಿಹೋಗಿ ನೋಡುತ್ತಾನೆಯೊ ಆಗ ಎಲ್ಲಾ ಕಡೆಯೂ ಇರುವುದೊಂದೆ. ಅವನು ಸಮನಾಗಿ ನೋಡುವುದಕ್ಕೆ ಇರುವ ಕಾರಣ ಅದೊಂದೆ. ಸೂರ್ಯ ಪ್ರಕಾಶಿಸುತ್ತಿರುವನು. ಹಿಮಮಣಿಗಳು ಅವನನ್ನು ತಮ್ಮ ಯೋಗ್ಯತಾನು ಸಾರ ಪ್ರತಿಬಿಂಬಿಸುವುವು. ಎಲ್ಲಾ ಪ್ರತಿಬಿಂಬಿಸುವುದೂ ಒಂದೇ ವಸ್ತುವನ್ನು. ಆದರೆ ತಮ್ಮ ಮಧ್ಯವರ್ತಿಗೆ ತಕ್ಕಂತೆ ಕೆಲವು ಸ್ಪಷ್ಟವಾಗಿ ಪ್ರತಿಬಿಂಬಿಸುವುವು ಮತ್ತೆ ಕೆಲವು ಮೊಬ್ಬುಮೊಬ್ಬಾಗಿ ಪ್ರತಿಬಿಂಬಿಸುವುವು.

ಯೋಗಿ ಆತ್ಮನನ್ನು ಸರ್ವಭೂತಗಳಲ್ಲಿಯೂ ನೋಡುತ್ತಾನೆ. ಎಲ್ಲಾ ಜೀವರಾಶಿಗಳ ಹಿಂದೆ, ಜಡವಸ್ತುವಿನ ಹಿಂದೆ ಸರ್ವಸಾಮಾನ್ಯವಾದ ಹಿನ್ನೆಲೆ ಪರಮಾತ್ಮನೊಬ್ಬನೆ. ಯೋಗಿಗಾದರೊ\break ಅವನಿಗೆ ಕಾಣುವ ಪ್ರತಿಯೊಂದು ನಾಮರೂಪಗಳು ಪರಮಾತ್ಮನನ್ನು ನೋಡುವುದಕ್ಕೆ ಇರುವ ಕಿಟಕಿಗಳಂತೆ. ಪಂಡಿತ ಪಾಮರ, ಪಾಪಿ ಪುಣ್ಯವಂತ, ಮನುಷ್ಯ ಪ್ರಾಣಿ, ವೃಕ್ಷ ತರುಲತೆಗಳು ಚರಾಚರ ವಸ್ತುಗಳ ಹಿಂದೆಲ್ಲ ಅವನಿಗೆ ಪರಮಾತ್ಮನೆ ಕಾಣುತ್ತಾನೆ. ಹೇಗೆ ಸಾಗರದ ಮೇಲಿರುವ ಅಲೆತೆರೆ ನೊರೆಗುಳ್ಳೆ ಇವುಗಳಲ್ಲೆಲ್ಲ ಸಾಗರವೇ ಕಾಣುತ್ತಿದೆಯೊ ಹಾಗೆ ಪರಮಾತ್ಮನೊಬ್ಬನೇ ಅವನಿಗೆ ಎಲ್ಲಾ ದೃಶ್ಯವಸ್ತುವಿನಲ್ಲಿಯೂ ಕಾಣುವುದು.

\newpage

ಸರ್ವಭೂತಗಳನ್ನು ಆತ್ಮನಲ್ಲಿ ಕಾಣುತ್ತಾನೆ. ಪರಮಾತ್ಮನಲ್ಲಿ ಎಲ್ಲಾ ನೆಲೆಸಿರುವುದನ್ನು ಕಾಣುತ್ತಾನೆ. ಹೇಗೆ ಮಹಾಕಾಶದಲ್ಲಿ ಅನೇಕ ಬ್ರಹ್ಮಾಂಡ ಮತ್ತು ಸಣ್ಣಪುಟ್ಟ ವಸ್ತುಗಳೆಲ್ಲ ತೇಲುತ್ತಿವೆಯೊ ಹಾಗೆ ಪ್ರಪಂಚದಲ್ಲಿರುವುದೆಲ್ಲ ಪರಮಾತ್ಮನಲ್ಲಿ ತೇಲುತ್ತಿವೆ. ಪರಮಾತ್ಮನನ್ನು ಬಿಟ್ಟು ಯಾವುದೂ ಇಲ್ಲ.

ಯೋಗಿ ಪ್ರತಿಯೊಂದು ವಸ್ತುವಿನ ಅಂತರಾಳದಲ್ಲಿಯೂ ಪರಮಾತ್ಮನನ್ನು ಕಾಣುತ್ತಾನೆ. ಪ್ರತಿಯೊಂದು ವಸ್ತುವೂ ಪರಮಾತ್ಮನಲ್ಲಿಯೇ ಇದೆ. ಅವನನ್ನು ಬಿಟ್ಟು ಬೇರೆ ಇರಲಾರದು. ಅದಕ್ಕೆ ಬೇರೆ ವ್ಯಕ್ತಿತ್ವವೇ ಇಲ್ಲ. ಅವನಲ್ಲಿದ್ದರೆ ಅದಕ್ಕೊಂದು ವ್ಯಕ್ತಿತ್ವ. ಅವನನ್ನು ಬಿಟ್ಟರೆ ಅದು ಬೇರೆ ಇರಲಾರದು. ಅಲೆ ಸಮುದ್ರದ ಮೇಲೆ ಮಾತ್ರ ಇರಬಲ್ಲುದು. ಸಮುದ್ರವನ್ನು ಬಿಟ್ಟು ಬೇರೆ ಇರಲಾರದು. ನಾವು ಒಂದು ಚೆಂಬನ್ನು ನೀರಿನಲ್ಲಿ ಮುಳುಗಿಸಿದರೆ ಚೆಂಬಿನೊಳಗೆ ನೀರು ಇರುವುದನ್ನು ನೋಡುತ್ತೇವೆ. ಆ ನೀರಿನಲ್ಲಿ ಚೆಂಬು ಇರುವುದನ್ನು ನೋಡುತ್ತೇವೆ. ಯೋಗಿ ನೋಡುವುದು ಈ ದೃಷ್ಟಿಯಿಂದ. ಅವನು ನೋಡುವ ಅನುಭವಿಸುವ ಪ್ರತಿಯೊಂದು ವಸ್ತುವನ್ನು ಪರಮಾತ್ಮನೊಡನೆ ಸಂಬಂಧಿಸುವನು. ಎಲ್ಲಾ ಜೀವರಾಶಿಗಳಲ್ಲಿಯೂ ಪರಮಾತ್ಮನು ಇರುವನು, ಆ ಪರಮಾತ್ಮನಲ್ಲಿ ಎಲ್ಲಾ ಜೀವರಾಶಿಗಳೂ ಇವೆ. ಅಜ್ಞಾನಿ ನೋಡುವಾಗ ಪರಮಾತ್ಮನನ್ನು ಮರೆಯುವನು. ಕೇವಲ ಹೊರಗೆ ಇರುವ ನಾಮರೂಪಗಳಿಂದ ಕೂಡಿರುವ ವಸ್ತುಗಳನ್ನು ಮಾತ್ರ ನೋಡುವನು. ಅದು ಚೆನ್ನಾಗಿದ್ದರೆ ಅದರಿಂದ ಆಕರ್ಷಿಸಲ್ಪಡುವನು, ಅದು ಚೆನ್ನಾಗಿಲ್ಲದೆ ಇದ್ದರೆ ಅದರಿಂದ ದೂರ ಸರಿಯುವನು. ಯೋಗಿಗಾದರೊ ಚೆನ್ನಾಗಿದೆ ಚೆನ್ನಾಗಿಲ್ಲ ಎಂಬ ಮಾತೆ ಇಲ್ಲ. ಏಕೆಂದರೆ ಅವನು ಅದರ ಒಳಗೆ ಇರುವ ಪರಮಾತ್ಮನ ದೃಷ್ಟಿಯಿಂದ ನೋಡುವನು. ಅವನಿಗೆ ಎಲ್ಲಾ ಕಡೆಯೂ ಅವೇ ಕಾಣುವುದು.

\begin{shloka}
ಯೋ ಮಾಂ ಪಶ್ಯತಿ ಸರ್ವತ್ರ ಸರ್ವಂ ಚ ಮಯಿ ಪಶ್ಯತಿ~।\\ತಸ್ಯಾಹಂ ನ ಪ್ರಣಶ್ಯಾಮಿ ಸ ಚ ಮೇ ನ ಪ್ರಣಶ್ಯತಿ \hfill॥ ೩೦~॥
\end{shloka}

\begin{artha}
ಯಾರು ನನ್ನನ್ನು ಎಲ್ಲಾ ಕಡೆಯಲ್ಲಿಯೂ ನೋಡುವನೋ ಮತ್ತು ಎಲ್ಲವನ್ನೂ ನನ್ನಲ್ಲಿ ನೋಡುವನೋ, ಅವನಿಗೆ ನಾನು ಕಾಣದೆ ಹೋಗುವುದಿಲ್ಲ. ಮತ್ತು ಅವನು ನನಗೆ ಕಾಣದೆ ಹೋಗುವುದಿಲ್ಲ.
\end{artha}

ಯೋಗಿ ಎಲ್ಲಾ ಕಡೆಯಲ್ಲಿಯೂ ಎಂದರೆ ಪವಿತ್ರ ಅಪವಿತ್ರ ಎಲ್ಲಾ ಸ್ಥಳಗಳಲ್ಲಿಯೂ ಅವನನ್ನು ನೋಡುವನು. ಲೌಕಿಕ ದೃಷ್ಟಿಯಿಂದ ಒಂದು ಪವಿತ್ರ ಮತ್ತೊಂದು ಅಪವಿತ್ರ ಸ್ಥಳ ಇರಬಹುದು. ಆದರೆ ಯೋಗಿ ಹೊರಗಿನ ತೋರಿಕೆಯ ಉಪಾಧಿಯನ್ನು ತೂರಿ ನೋಡುವುದರಿಂದ ಅವನಿಗೆ ಒಂದೇ ಪರಮಾತ್ಮನ ದರ್ಶನವಾಗುವುದು. ಅವನು ಭಗವಂತನಲ್ಲಿ ಪ್ರಪಂಚದಲ್ಲಿರುವ ಚರಾಚರ ವಸ್ತುಗಳೆಲ್ಲ ಮುಳುಗಿರುವುದನ್ನು ನೋಡುವನು. ಭಗವಂತ ಎಲ್ಲವನ್ನೂ ಹಾಸುಹೊಕ್ಕಾಗಿ ವ್ಯಾಪಿಸಿಕೊಂಡು, ಅವುಗಳನ್ನೆಲ್ಲ ಮೀರಿರುವನು. ಇಂತಹ ಯೋಗಿಗೆ ಯಾವುದನ್ನು ನೋಡಿದರೂ ಅದು ಭಗವಂತನಲ್ಲಿರುವುದು ವ್ಯಕ್ತವಾಗುವುದು. ಅವನಲ್ಲಿ ದೇವರಿರುವನು, ದೇವರಲ್ಲಿ ಅವನಿರುವನು. ದೇವರನ್ನು ಬಿಟ್ಟು ಪ್ರಪಂಚದಲ್ಲಿ ಯಾವುದೂ ಇಲ್ಲ.

ಇಂತಹ ಯೋಗಿಗೆ ನಾನು ಕಾಣದೆ ಹೋಗುವುದಿಲ್ಲ ಎನ್ನುವನು. ಎಂತಹ ಭಯಂಕರವಾದ ಅಥವಾ ಸುಂದರವಾದ ವೇಷಗಳನ್ನು ಹಾಕಿಕೊಂಡಿದ್ದರೂ ಅದರ ಹಿಂದುಗಡೆ ದೇವರನ್ನು ಸದಾ ನೋಡುತ್ತಿರುವನು. ಈ ಪ್ರಪಂಚದಲ್ಲಿ ಏನು ಆಗಬೇಕಾದರೂ ಅದರ ಹಿಂದುಗಡೆ ಭಗವಂತನ ಕೈವಾಡವನ್ನು ಕಾಣುವನು ಯೋಗಿ. ಅವನ ಇಚ್ಛೆಯಿಲ್ಲದೆ ಒಂದು ಹುಲ್ಲಿನ ಎಸಳೂ ಚಲಿಸಲಾರದು. ಜೀವನದಲ್ಲಿ ನಮಗೆ ಬರುವ ಸುಖದುಃಖಗಳ ಹಿಂದುಗಡೆ, ಲಾಭನಷ್ಟಗಳ ಹಿಂದುಗಡೆ, ಜೀವನದಲ್ಲಿ, ಮರಣದಲ್ಲಿ, ವಿಷದಲ್ಲಿ, ಅಮೃತದಲ್ಲಿ, ಲಾಭನಷ್ಟಗಳಲ್ಲಿ ಎಲ್ಲದರ ಹಿಂದುಗಡೆಯೂ ಭಗವಂತನನ್ನು ಕಾಣುತ್ತಿರುವನು.

ಅವನು ನನಗೆ ಕಾಣದೆ ಹೋಗುವುದಿಲ್ಲ ಎನ್ನುವನು. ದೇವರು ಎಂದಿಗೂ ತನ್ನ ಭಕ್ತರನ್ನು ಮರೆಯುವುದಿಲ್ಲ. ಯಾವಾಗಲೂ ಅವರನ್ನು ತನ್ನ ಗಮನದಲ್ಲಿಟ್ಟಿರುವನು. ತನ್ನ ಕಡೆಗೆ ಅವರು ಬರಬೇಕಾದರೆ, ಅವರಿಗೆ ಮಾರ್ಗದಲ್ಲಿರುವ ಆತಂಕಗಳನ್ನು ನಿವಾರಣೆ ಮಾಡುವನು. ಸದಾಕಾಲದಲ್ಲಿಯೂ ಎಲ್ಲ ವಿಪತ್ತುಗಳಿಂದಲೂ ಅವನ ಅಭಯ ಹಸ್ತ ರಕ್ಷಿಸುವುದು. ನಾವು\break ಅವನನ್ನು ಮರೆಯದಂತೆ ನೋಡಿಕೊಂಡರೆ ಅವನು ನಮ್ಮನ್ನು ಮರೆಯದಂತೆ ನೋಡಿಕೊಳ್ಳುವನು. ಭಕ್ತನಿಗೂ ಭಗವಂತನಿಗೂ ಒಂದು ಅನ್ಯೋನ್ಯ ಸಂಬಂಧವಿದೆ. ಇದನ್ನು ಅರಿತವನೆ ಯೋಗಿ.

\begin{shloka}
ಸರ್ವಭೂತಸ್ಥಿತಂ ಯೋ ಮಾಂ ಭಜತ್ಯೇಕತ್ವಮಾಸ್ಥಿತಃ~।\\ಸರ್ವಥಾ ವರ್ತಮಾನೋಽಪಿ ಸ ಯೋಗೀ ಮಯಿ ವರ್ತತೇ \hfill॥ ೩೧~॥
\end{shloka}

\begin{artha}
ಏಕತ್ವದಲ್ಲಿ ನೆಲೆಸಿರುವ ಯಾವನು ಸರ್ವಪ್ರಾಣಿಗಳಲ್ಲಿಯೂ ನನ್ನನ್ನು ಭಜಿಸುವನೋ ಆ ಯೋಗಿ ಹೇಗೆ ಇದ್ದರೂ ನನ್ನಲ್ಲಿಯೇ ಇರುವನು.
\end{artha}

ಯೋಗಿ ಏಕತ್ವದಲ್ಲಿ ನೆಲೆಸಿರುವನು. ನಾವು ಸಾಧಾರಣವಾಗಿ ಎದುರಿಗೆ ನಾನಾತ್ವವನ್ನು ನೋಡುತ್ತೇವೆ. ಒಂದು ಮತ್ತೊಂದಕ್ಕಿಂತ ಭಿನ್ನವಾಗಿದೆ. ಆದರೆ ಯೋಗಿಯಾದರೊ ಈ ನಾನಾತ್ವದ ಹಿಂದೆ ಹೋಗಿ, ನಾನಾತ್ವಕ್ಕೆ ಸರ್ವಸಾಮಾನ್ಯವಾದ ಏಕತ್ವದ ಹಿನ್ನೆಲೆಯನ್ನು ಮನಗಂಡಿರುವನು. ಆ ಏಕತ್ವವೇ ಭಗವಂತ. ಅಜ್ಞಾನಿ ನಾನಾತ್ವದಲ್ಲಿಯೇ ನಿಲ್ಲುವನು. ಯೋಗಿಯಾದರೊ ನಾನಾತ್ವದ ಹಿಂದೆ ಹೋಗುವನು. ಹಲವು ನೊರೆಗಳು, ಗುಳ್ಳೆಗಳು, ಅಲೆಗಳು ಸಾಗರದ ಮೇಲಿವೆ. ಅವುಗಳನ್ನೆಲ್ಲ ವಿಭಜನೆ ಮಾಡಿದರೆ ಹಿಂದೆ ಸಾಗರವೆ ಇದೆ. ಅದು ಅಖಂಡವಾಗಿದೆ. ಅದು ಒಡೆದು\-ಹೋದಂತೆ ಕಾಣುವುದು ತೋರಿಕೆಗೆ ಮಾತ್ರ.

ಯೋಗಿ ಎಲ್ಲಾ ಪ್ರಾಣಿಗಳ ಅಂತರಾಳದಲ್ಲಿರುವ ಪರಮಾತ್ಮನನ್ನು ಭಜಿಸುತ್ತಿರುವನು.\break ಅವನನ್ನು ಯಾವಾಗಲೂ ಪ್ರೀತಿಯಿಂದ ಭಕ್ತಿಯಿಂದ ಚಿಂತಿಸುತ್ತಿರುವನು. ಜೀವನದಲ್ಲಿ ಎಂತಹ ಕಷ್ಟದ ಕಣಿವೆಯ ಮೂಲಕ ಹೋಗಬೇಕಾದರೂ, ಭಗವಂತನನ್ನು ಅವನು ಚಿಂತಿಸುವುದನ್ನು ಮರೆಯುವುದಿಲ್ಲ. ಅವನೆಲ್ಲೊ ಮೇಲುಗಡೆ ನಮಗೆ ಕಾಣದ ವೈಕುಂಠದಲ್ಲೋ ಕೈಲಾಸದಲ್ಲೋ ಇರುವವನಲ್ಲ, ಯೋಗಿಗೆ. ಅವನು ಸರ್ವರ ಹೃದಯದಲ್ಲಿರುವನು, ತನ್ನಲ್ಲಿರುವನು, ಹಾಗೆಯೇ ಎಲ್ಲರಲ್ಲಿಯೂ ಅವನಿರುವನು. ಇದನ್ನು ಅನುಭವಿಸುತ್ತಿರುವ ಯೋಗಿ ಹೇಗಿದ್ದರೂ ದೇವರಲ್ಲಿಯೇ ಇರುವನು. ಈ ಅನುಭವ ಬಂದಮೇಲೆ ಒಬ್ಬ ತಾನು ಯಾವ ಕೆಲಸ ಮಾಡುತ್ತಿದ್ದನೋ ಅದನ್ನೆಲ್ಲ ಬಿಟ್ಟು ಸುಮ್ಮನಿರುವುದಿಲ್ಲ. ಅವನು ಜೀವನದಲ್ಲಿ ಬದುಕಿರುವ ಪರ್ಯಂತರ, ತನ್ನ ಪಾಲಿಗೆ ಬಂದ ಕರ್ತವ್ಯಗಳನ್ನು ಹೊರೆ ಹೊಣೆಗಳನ್ನು ಅನಾಸಕ್ತನಾಗಿ ನಿರ್ವಹಿಸುತ್ತಿರುವನು. ಅವನು ಎಂತಹ ಕೆಲಸವನ್ನು ಮಾಡುತ್ತಿದ್ದರೂ ದೇವರಲ್ಲಿಯೇ ಇರುವನು. ಮಹಾಭಕ್ತರು ಮತ್ತು ಯೋಗಿಗಳು ಜೀವನದಲ್ಲಿ ಹಲವು ಕಾರ್ಯಕ್ಷೇತ್ರದಲ್ಲಿ ನಮಗೆ ಸಿಕ್ಕುವರು. ಯೋಗಿಗಳೆಲ್ಲ ಕೆಲಸ ಬಿಟ್ಟು ಬರಿ ಭಗವಂತನ ಧ್ಯಾನದಲ್ಲೆ ಮುಳುಗಿರುವವರಲ್ಲ. ಹಲವರು ಹಲವು ವೃತ್ತಿಗಳನ್ನು ಮಾಡುತ್ತಿರುವರು. ಮುಂಚೆ ಯಾವ ಕೆಲಸವನ್ನು ಮಾಡುತ್ತಿದ್ದನೋ, ಅದೇ ಕೆಲಸವನ್ನೇ ಪೂರ್ಣತೆ ಪಡೆದಮೇಲೂ ಮಾಡುತ್ತಿರಬಹುದು. ಕಬೀರ್ ನೇಯ್ಗೆಯವನು, ಭಕ್ತಗೋರ ಕುಂಬಾರರವನು. ಧರ್ಮವ್ಯಾಧ ಕಟುಕರವನು. ಅದರಂತೆಯೆ ಮೋಚಿಗಳಲ್ಲಿ, ಬೆಸ್ತರಲ್ಲಿ ಯೋಗಿಗಳು, ಮಹಾಭಕ್ತರು ಇರುವರು. ಯೋಗಿಗಳಾದ ಮೇಲೆ ಅವರು ಮಾಡುವ ಕೆಲಸಗಳಿಂದ ಬಾಧಿತರಾಗುವುದಿಲ್ಲ. ಏಕೆಂದರೆ ಅವರ ಕೆಲಸದ ಹಿಂದೆ ಯಾವ ಆಸಕ್ತಿಯೂ ಇರುವುದಿಲ್ಲ. ಯಾವ ಫಲಾಕಾಂಕ್ಷೆಯೂ ಇರುವುದಿಲ್ಲ.

\begin{shloka}
ಆತ್ಮೌಪಮ್ಯೇನ ಸರ್ವತ್ರ ಸಮಂ ಪಶ್ಯತಿ ಯೋಽರ್ಜುನ~।\\ಸುಖಂ ವಾ ಯದಿ ವಾ ದುಃಖಂ ಸ ಯೋಗೀ ಪರಮೋ ಮತಃ \hfill॥ ೩೨~॥
\end{shloka}

\begin{artha}
ಅರ್ಜುನ, ಯಾರು ಎಲ್ಲಾ ಪ್ರಾಣಿಗಳಲ್ಲಿ ಸುಖದುಃಖಗಳನ್ನು ತನ್ನಂತೆ ಸಮನಾಗಿ ನೋಡುವನೊ ಆ ಯೋಗಿ ಶ್ರೇಷ್ಠನೆಂದು ನನ್ನ ಅಭಿಪ್ರಾಯ.
\end{artha}

ಯೋಗಿ ಎಲ್ಲರನ್ನೂ ತನ್ನಂತೆ ಭಾವಿಸುವನು. ಅವನ ಮನಸ್ಸು ವಿಶಾಲವಾಗುವುದು. ಎಲ್ಲರನ್ನೂ ಪ್ರೀತಿಯಿಂದ ಬಾಚಿ ತಬ್ಬುವಷ್ಟು ವಿಶಾಲಹೃದಯನು. ಅವನು ಇನ್ನೊಬ್ಬರನ್ನು ಅಳೆಯುವಾಗ ತನ್ನನ್ನು ಆ ಸ್ಥಿತಿಯಲ್ಲಿಟ್ಟು ಅಳೆಯುವನು. ತನ್ನಂತೆಯೇ ಇತರರು. ತನಗೆ ಹೇಗೆ ಅಪ್ರಿಯ ಬೇಡವೊ ಇತರರಿಗೆ ಅದು ಬೇಡ. ಅದಕ್ಕಾಗಿ ಅವನು ಅದನ್ನು ಇತರರಿಗೆ ಕೊಡುವುದಿಲ್ಲ. ತನಗೆ ಹೇಗೆ ಪ್ರಿಯವಾದುದು ಬೇಕೊ, ಅದರಂತೆಯೆ ಅವನು ಇತರರಿಗೆ ಅದನ್ನು ಕೊಡಲು ಸಿದ್ಧವಾಗಿರುವನು.

ಯಾರಲ್ಲಿ ಸಣ್ಣತನ, ಸಣ್ಣ ವ್ಯಕ್ತಿತ್ವ, ಸ್ವಾರ್ಥತೆ ಇವುಗಳು ಅಳಿಸಿ ಹೋಗಿ ಎಲ್ಲರನ್ನೂ ತನ್ನಂತೆ ಭಾವಿಸುವುದು ಬಂದಿದೆಯೊ ಅವನೇ ಶ್ರೇಷ್ಠ ಎನ್ನುವನು ಶ‍್ರೀಕೃಷ್ಣ. ಅನೇಕ ವೇಳೆ ಹಲವರು ಅದ್ಭುತವಾದ ಪವಾಡಗಳನ್ನು ಮಾಡಿ ತೋರಬಹುದು. ತಮಗೆ ಬಗೆಬಗೆಯ ಆಧ್ಯಾತ್ಮಿಕ ಅನುಭವಗಳಾಗಿವೆ ಎನ್ನಬಹುದು. ಆದರೆ ಯೋಗಿಯನ್ನು ಅಳೆಯಬೇಕಾದರೆ ಒಂದು ಒರೆಗಲ್ಲಿದೆ. ಅದೇ ಅವನ ಹೃದಯ ಎಷ್ಟು ವಿಶಾಲವಾಗಿದೆ ಎಂಬುದು. ಭಗವಂತನೆಡೆಗೆ ಸಮೀಪಿಸುವುದರ ಒಂದು ಚಿನ್ಹೆಯೇ ಅವನಿಗೆ ಎಲ್ಲರ ಮೇಲೆಯೂ ಪ್ರೀತಿ ಉದಿಸುವುದು. ಎಷ್ಟು ಹೃದಯ ವಿಶಾಲವಾಗುವುದೋ, ಮಾನವಕೋಟಿಗೆ ಎಷ್ಟು ಅನುಕಂಪ ಅವನಲ್ಲಿದೆಯೋ ಅಷ್ಟು ಅವನು ಭಗವಂತನ ಸಮೀಪದಲ್ಲಿರುವನು. ದೀಪದ ಸಮೀಪಕ್ಕೆ ಹೋದಂತೆ ಬೆಳಕು ಹೆಚ್ಚಾಗುವಂತೆ, ಬೆಂಕಿಯ ಸಮೀಪಕ್ಕೆ ಹೋದಂತೆ ಶಾಖ ಹೆಚ್ಚಾದಂತೆ, ಭಗವಂತನ ಸಮೀಪಕ್ಕೆ ಹೋದಂತೆ ಹೃದಯ ವಿಶಾಲವಾಗುತ್ತ ಬರುವುದು.

ಅನಂತರ ಅರ್ಜುನ ಶ‍್ರೀಕೃಷ್ಣನನ್ನು ಹೀಗೆ ಪ್ರಶ್ನಿಸುತ್ತಾನೆ.

\begin{shloka}
ಯೋಽಯಂ ಯೋಗಸ್ತ್ವಯಾ ಪ್ರೋಕ್ತಃ ಸಾಮ್ಯೇನ ಮಧುಸೂದನ~।\\ಏತಸ್ಯಾಹಂ ನ ಪಶ್ಯಾಮಿ ಚಂಚಲತ್ವಾತ್ ಸ್ಥಿತಿಂ ಸ್ಥಿರಾಮ್ \hfill॥ ೩೩~॥
\end{shloka}

\begin{artha}
ಮಧುಸೂದನ, ಸಮತ್ವದ ದೃಷ್ಟಿಯ ಮೂಲಕ ನೀನು ಯಾವ ಯೋಗವನ್ನು ಹೇಳಿದೆಯೋ ಮನಸ್ಸು ಚಂಚಲವಾಗಿರುವುದರಿಂದ ಇದರ ಸ್ಥಿರವಾದ ಸ್ಥಿತಿಯನ್ನು ನಾನು ನೋಡುತ್ತಿಲ್ಲ.
\end{artha}

ಅರ್ಜುನ ಇಲ್ಲಿ ಶ‍್ರೀಕೃಷ್ಣನಿಗೆ ಅವನು ಏನನ್ನು ಹೇಳಿದನೊ ಅದನ್ನು ತಿಳಿದುಕೊಳ್ಳಲು ಸಾಧ್ಯವಾಗಲಿಲ್ಲ ಎನ್ನುತ್ತಾನೆ. ಏಕೆಂದರೆ ಅವನ ಮನಸ್ಸು ಚಂಚಲವಾಗಿತ್ತು. ಸಾಕ್ಷಾತ್ ಅವತಾರ\-ಸ್ವರೂಪನಾದ ಶ‍್ರೀಕೃಷ್ಣನೆ ಅರ್ಜುನನ ಪಕ್ಕದಲ್ಲಿಯೇ ನಿಂತುಕೊಂಡು ಹೇಳುತ್ತಿದ್ದರೂ, ಅರ್ಜುನ ತನಗೆ ಗೊತ್ತಾಗಲಿಲ್ಲ ಎನ್ನಬೇಕಾದರೆ, ಬಹಳ ಕಾಲಗಳಾದಮೇಲೆ ಶ‍್ರೀಕೃಷ್ಣ ಏನನ್ನು ಹೇಳಿದನೊ ಅದನ್ನು ಓದುವವರ ಮನಸ್ಸಿನಲ್ಲಿ ಭಿನ್ನಭಿನ್ನ ಅಭಿಪ್ರಾಯಗಳು ಬಂದರೆ ಅದರಲ್ಲೇನು ಆಶ್ಚರ್ಯವಿಲ್ಲ ಎನ್ನಿಸುವುದು. ಜೀವನದಲ್ಲಿ ದೊಡ್ಡ ಗುರುವನ್ನು ಪಡೆಯುವುದೊಂದು ಭಾಗ್ಯ. ಹಾಗೆ ಅಂತಹ ಗುರು ದೊರೆತ ಮೇಲೆ, ಅವನು ಹೇಳುವುದನ್ನು ತಿಳಿದುಕೊಳ್ಳಬೇಕಾದರೆ ಅದಕ್ಕೆ ನಮಗೆ ಯೋಗ್ಯತೆಯೂ ಇರಬೇಕು. ಇಲ್ಲದೆ ಇದ್ದರೆ ತಿಳಿದುಕೊಳ್ಳುವುದಕ್ಕೆ ಆಗುವುದಿಲ್ಲ. ನಮಗೆ ಹೊಸ ಅನುಭವಗಳು ತಕ್ಷಣವೇ ಅರ್ಥವಾಗುವುದಿಲ್ಲ. ಅದಕ್ಕೆ ಮತ್ತೊಂದು ಕಾರಣವೂ ಇದೆ. ನಮ್ಮ ಮನಸ್ಸನ್ನು ನಾವು ಇನ್ನೂ ಪರಿಶುದ್ಧ ಮಾಡಿಲ್ಲ ಮತ್ತು ಕೇಂದ್ರೀಕರಿಸಿಲ್ಲ. ಇಂತಹ ಮನಸ್ಸಿನಿಂದ ಅರ್ಜುನ ಅರ್ಥಮಾಡಿ ಕೊಳ್ಳಲು ಸಾಧ್ಯವಾಗಲಿಲ್ಲ ಎಂದು ನಿರ್ವಂಚನೆಯಿಂದ ಹೇಳುತ್ತಾನೆ. ಮಹಾಗುರುವಿನೆದುರಿಗೆ ನಮ್ಮ ಅಜ್ಞಾನವನ್ನು ವ್ಯಕ್ತಪಡಿಸುವುದು ಮೇಲು. ಅವನು ಅದನ್ನು ತಿದ್ದುತ್ತಾನೆ. ನಾವು ಗೊತ್ತಾಗದೇ ಇದ್ದರೂ ಗೊತ್ತಾಯಿತೆಂದು ನಟಿಸಿದರೆ ಅದು ಅಕ್ಷಮ್ಯ. ಇಲ್ಲಿ ಅರ್ಜುನ ನನಗೆ ಇನ್ನೂ ನೀನು ಹೇಳಿದ್ದು ಅರ್ಥವಾಗಲಿಲ್ಲ ಎಂದು ಸಂಕೋಚವಿಲ್ಲದೆ ತಿಳಿಸುತ್ತಾನೆ. ಊಟಕ್ಕೆ ಕುಳಿತಾಗ ಸಂಕೋಚ ಪಟ್ಟುಕೊಂಡು ಕೇಳದೆ ಇದ್ದರೆ ಹಾಕಿದಷ್ಟನ್ನು ತಿಂದು ಎದ್ದು ಹೋಗಬೇಕಾಗುವುದು. ನಾಚಿಕೆಯಿಲ್ಲದೆ ಕೇಳಬೇಕು. ಆಗ ಹಾಕುತ್ತಾರೆ. ಅದರಂತೆಯೇ ಒಂದು ಸೂಕ್ಷ್ಮ ವಿಷಯವನ್ನು ತಿಳಿದುಕೊಳ್ಳುವಾಗಲೂ ನಾವು ಗೊತ್ತಾಗಲಿಲ್ಲ ಎಂದು ಹೇಳದೆ ಹೋದರೆ, ಹೇಳುವವನು ನಮಗೆ ಅವನು ಹೇಳಿದ್ದೆಲ್ಲ ಅರ್ಥವಾಗಿದೆ ಎಂದು ಮುಂದಕ್ಕೆ ಹೋಗುತ್ತಾನೆ. ಆದಕಾರಣ ಶ‍್ರೀಕೃಷ್ಣ ಗುರುವನ್ನು “ಪರಿಪ್ರಶ್ನೇನ” ಎಂದರೆ ಪ್ರಶ್ನೆಗಳನ್ನು ಹಾಕಿ ತಿಳಿದುಕೊಳ್ಳಬೇಕು ಎನ್ನುವನು.

ಅರ್ಜುನ ತನ್ನ ಮನಸ್ಸಿನ ಚಾಂಚಲ್ಯ ಸ್ವಭಾವದ ಮೇಲೆ ಹೀಗೆ ಹೇಳುತ್ತಾನೆ.

\begin{shloka}
ಚಂಚಲಂ ಹಿ ಮನಃ ಕೃಷ್ಣ ಪ್ರಮಾಥಿ ಬಲವದ್ದೃಢಮ್~।\\ತಸ್ಯಾಹಂ ನಿಗ್ರಹಂ ಮನ್ಯೇ ವಾಯೋರಿವ ಸುದುಷ್ಕರಮ್ \hfill॥ ೩೪~॥
\end{shloka}

\begin{artha}
ಶ‍್ರೀಕೃಷ್ಣ, ಮನಸ್ಸು ಚಂಚಲವಾಗಿದೆ, ದಂಗೆ ಎದ್ದಂತಿದೆ. ಬಹಳ ಬಲವಾಗಿ ಸ್ವಾಧೀನಕ್ಕೆ ಬರದೆ ಇದೆ. ಗಾಳಿಯನ್ನು ನಿಗ್ರಹಿಸುವಷ್ಟು ಕಷ್ಟ ಅದು ಎಂದು ನಾನು ಭಾವಿಸುತ್ತೇನೆ.
\end{artha}

ಮನಸ್ಸಿನ ಸ್ವಭಾವವೆ ಚಂಚಲ, ಅದರಲ್ಲಿಯೂ ನಿಗ್ರಹಿಸದ ಮನಸ್ಸು. ಸ್ವಾಮಿ ವಿವೇಕಾನಂದರು ಮನಸ್ಸನ್ನು ಒಂದು ಕೋತಿಗೆ ಹೋಲಿಸುವರು. ಆ ಕೋತಿ ಚಂಚಲ ಸ್ವಭಾವದ್ದು, ಅದು ಕಳ್ಳು ಕುಡಿದಿದೆ. ಆಗ ಮತ್ತೂ ಚಂಚಲವಾಗುವುದು. ಆ ಕಳ್ಳು ಕುಡಿದ ಕಪಿಗೆ ದೆವ್ವ ಮೆಟ್ಟಿಕೊಂಡಿದೆ. ಆ ದೆವ್ವ ಮೆಟ್ಟಿಕೊಂಡ ಕಪಿಗೆ ಚೇಳು ಬೇರೆ ಕಡಿದಿದೆ. ಎಂದರೆ ನಾವು ಊಹಿಸಬಹುದು ಅದರ ಮನಸ್ಸಿನ ಸ್ಥಿತಿಯನ್ನು. ಒಂದನ್ನೂ ಸ್ಥಿರವಾಗಿ ಕುಳಿತು ಚಿಂತಿಸಲಾರದು, ನಿಗ್ರಹಿಸದ ಮನಸ್ಸು. ಒಮ್ಮೆ ಒಂದರ ಮೇಲೆ, ಇನ್ನೊಂದು ಸಲ ಇನ್ನೊಂದರ ಮೇಲೆ, ನೊಣದಂತೆ ಕುಳಿತು ಹಾರಾಡುತ್ತದೆ. ಅದರಲ್ಲಿಯೂ ನಾವು ಮನಸ್ಸನ್ನು ಏಕಾಗ್ರಮಾಡಬೇಕು, ಧ್ಯಾನ ಮಾಡಬೇಕು, ಎಂದು ಪ್ರಯತ್ನಿಸಿದಾಗ ಆ ಚಂಚಲ ಸ್ವಭಾವ ಎಂದಿಗಿಂತ ಹೆಚ್ಚಾಗುವುದು.

ಅದು ದಂಗೆಯೆದ್ದಿದೆ ನಮ್ಮಮೇಲೆ. ನಾವು ಹೇಳಿದಂತೆ ಕೇಳುವುದಿಲ್ಲ. ತನ್ನ ದಾರಿಯನ್ನೇ ಹಿಡಿದು ಹೋಗುವುದು. ಅದು ತಾನು ಮತ್ತೊಬ್ಬನಿಗೆ ದಾಸ ಎಂಬುದನ್ನು ಮರೆತು, ಯಜಮಾನ\-ನಾಗಿದೆ. ನಾವು ದುರ್ಬಲರಾದ ಯಜಮಾನರಂತೆ ಆಳು ಹೇಳಿದಂತೆ ಕೇಳಿಕೊಂಡು ಹೋಗುವ ಸ್ಥಿತಿಗೆ ಬಂದಿದ್ದೇವೆ. ಆಳಿಗೆ ಅಂಜುತ್ತೇವೆ ನಾವು. ಅವನಿಗೆ ಇಚ್ಛೆಯಿಲ್ಲದ ಏನನ್ನೂ ಹೇಳುವುದಕ್ಕೆ ನಮಗೆ ಧೈರ್ಯವಿಲ್ಲ.

ಅದು ಬಲವಾಗಿದೆ. ಮನಸ್ಸು ನಾವು ಹೇಳಿದಂತೆ ಕೇಳುವುದಿಲ್ಲ. ನಮ್ಮನ್ನೇ ಅದರ ಇಚ್ಛೆಗೆ ಬಗ್ಗಿಸುವುದು. ಜೀವನದಲ್ಲಿ ಎಷ್ಟು ದುರಭ್ಯಾಸಗಳನ್ನು ನಾವು ಮಾಡಿಕೊಂಡಿರುವೆವು! ಎಷ್ಟೋ ವಾಸನೆಗಳಿವೆ ನಮ್ಮ ಮನಸ್ಸಿನಲ್ಲಿ. ನಾವು ಅದೆಲ್ಲ ಕೆಟ್ಟದ್ದು, ಇನ್ನುಮೇಲೆ ನಾವು ಹಾಗೆ ಮಾಡುವು ದಿಲ್ಲ ಎಂದು ಸಂಕಲ್ಪ ಮಾಡುತ್ತೇವೆ. ಆದರೆ ಪರೀಕ್ಷಾ ಸಮಯ ಬಂದಾಗ ನಮಗೆ ಇಚ್ಛೆಯಿಲ್ಲ ದುದನ್ನು ಮಾಡಿಹಾಕುತ್ತೇವೆ. ಇನ್ನೊಂದು ಸಲ ಮಾಡುವುದಿಲ್ಲ ಎನ್ನುತ್ತೇವೆ. ಆದರೆ ಇನ್ನೊಂದು ಸಲ ಅವಕಾಶ ಬಂದಾಗಲೂ ಅದನ್ನು ಮಾಡಿ ಮತ್ತೊಂದು ಸಲ ಮಾಡುವುದಿಲ್ಲ ಎಂದುಕೊಳ್ಳುತ್ತೇವೆ. ಈ ಇನ್ನೊಂದು ಸಲ ಮತ್ತೊಂದು ಸಲ ಎನ್ನುವುದಕ್ಕೆ ಅಂತ್ಯವೇ ಇಲ್ಲ. ಅಂತೂ ನಮ್ಮ ನಿಗ್ರಹಿಸದ ಮನಸ್ಸು ನಮ್ಮನ್ನು ಬಲವಾಗಿ ಎಳೆದುಕೊಂಡು ಹೋಗುತ್ತದೆ.

ಅದು ದೃಢವಾಗಿದೆ. ಆ ಮನಸ್ಸಿನ ಹಳೆಯ ಆಸೆ, ಚಪಲಗಳು ಬಹಳ ಆಳದವರೆಗೂ ಹೋಗಿವೆ. ಮರದ ಬುಡದ ಬೇರು ಆಳಕ್ಕೆ ಹೋಗಿದ್ದರೆ, ನಾವು ಎಷ್ಟು ವೇಳೆ ಅದರ ರೆಂಬೆಗಳನ್ನು ಕತ್ತರಿಸುತ್ತಿದ್ದರೂ ಅದು ಪುನಃ ಪುನಃ ಚಿಗುರುವುದು. ಒಂದು ವೇಳೆ ಮನಸ್ಸಿನ ಚಪಲವನ್ನು ಒಂದು ಸ್ಥಿತಿಯಲ್ಲಿ ನಿಗ್ರಹಿಸುವುದರಲ್ಲಿ ಜಯಶಾಲಿಗಳಾದೆವು ಎಂದು ಭಾವಿಸೋಣ. ಆದರೂ ಮನಸ್ಸು ಒಂದೇ ಸಲ ಸೋಲನ್ನು ಒಪ್ಪಿಕೊಳ್ಳುವುದಿಲ್ಲ. ಅದು ತನ್ನ ವೇಷವನ್ನು ಬದಲಾಯಿಸಿಕೊಂಡು ಬರುತ್ತದೆ. ನಾವು ಅಜಾಗರೂಕರಾಗಿ ಅದಕ್ಕೆ ಸ್ವಲ್ಪ ಅವಕಾಶ ಕೊಟ್ಟೆವೆಂದರೆ ಅದು ನಮ್ಮ ಹಳೆಯ ಸ್ವಭಾವವೇ ಹೊಸ ವೇಷದಲ್ಲಿ ಬರುವುದೆಂದು ವ್ಯಕ್ತವಾಗುವುದು. ನಮ್ಮ ಮನಸ್ಸಿನಲ್ಲಿರುವ ಆಸೆ ಆಕಾಂಕ್ಷೆಗಳು ಬಡಪೆಟ್ಟಿಗೆ ಸಾಯುವುದಿಲ್ಲ. ಸೋಲನ್ನು ಒಪ್ಪಿಕೊಳ್ಳುವುದಿಲ್ಲ. ಭಂಡಾಸುರನಂತೆ ಒಂದು ವೇಷದಲ್ಲಿ ಅದನ್ನು ಕೊಂದರೆ ಮತ್ತೊಂದು ವೇಷದಲ್ಲಿ ಅದು ಬರುವುದು. ಅಂತೂ ನಮ್ಮ ಬಾಳೆಲ್ಲ ಆ ವೇಷಧಾರಿಯೊಡನೆ ಹೋರಾಡಿ ಸೋಲುವುದೇ ಆಗಿದೆ.

ಮನಸ್ಸಿನ ನಿಗ್ರಹ ವಾಯುವಿನ ನಿಗ್ರಹದಂತೆ ಕಷ್ಟ ಎನ್ನುವನು ಅರ್ಜುನ. ವಾಯುವನ್ನು ನಿಗ್ರಹಿಸಬೇಕಾದರೆ ನಾವು ಬೇರೆ ಪಾತ್ರೆ ಉಪಯೋಗಿಸಬೇಕು, ಬೇರೆ ಯಂತ್ರ ಉಪಯೋಗಿಸಬೇಕು. ಮೀನು ಹಿಡಿಯುವ ಬಲೆಯಲ್ಲಿ ವಾಯುವನ್ನು ಹಿಡಿಯಲಾಗುವುದಿಲ್ಲ. ನೀರನ್ನು ತರುವ ಪಾತ್ರೆಯಲ್ಲಿ, ಗಾಳಿಯನ್ನು ಅಳೆಯುವುದಕ್ಕೆ ತೂಗುವುದಕ್ಕೆ ಆಗುವುದಿಲ್ಲ. ಗಾಳಿಯನ್ನು ಹಿಡಿಯಬೇಕಾದರೆ ಬೇರೆ ಸಾಧನವೇ ಬೇಕಾಗುವುದು. ನಾವು ಅದನ್ನು ಇನ್ನೂ ಕಲಿಯಬೇಕಾಗಿದೆ. ಅದು ಹೊಸ ಉದ್ಯಮ. ಅದನ್ನು ಕಲಿಯುವವರೆಗೆ ಅದು ದುಸ್ತರವಾಗಿ ನಮಗೆ ಕಾಣುವುದು.

ಅದಕ್ಕೆ ಶ‍್ರೀಕೃಷ್ಣ ಅರ್ಜುನನಿಗೆ ಹೀಗೆ ಹೇಳುತ್ತಾನೆ:

\begin{shloka}
ಅಸಂಶಯಂ ಮಹಾಬಾಹೋ ಮನೋ ದುರ್ನಿಗ್ರಹಂ ಚಲಮ್~।\\ಅಭ್ಯಾಸೇನ ತು ಕೌಂತೇಯ ವೈರಾಗ್ಯೇಣ ಚ ಗೃಹ್ಯತೇ \hfill॥ ೩೫~॥
\end{shloka}

\begin{artha}
ಮನಸ್ಸನ್ನು ನಿಗ್ರಹಿಸುವುದು ಕಷ್ಟ ಮತ್ತು ಅದು ಚಂಚಲ ಎಂಬುದರಲ್ಲಿ ಸಂಶಯವಿಲ್ಲ. ಅಭ್ಯಾಸ\-ದಿಂದಲೂ ಮತ್ತು ವೈರಾಗ್ಯದಿಂದಲೂ ಇದನ್ನು ನಿಗ್ರಹಿಸಬಹುದು.
\end{artha}

ಶ‍್ರೀಕೃಷ್ಣ ಅರ್ಜುನ ಹೇಳುವುದನ್ನು ಒಪ್ಪಿಕೊಳ್ಳುತ್ತಾನೆ. ಅದು ಚಂಚಲ ಮತ್ತು ನಿಗ್ರಹಿಸುವುದು ಕಷ್ಟ ಎಂಬ ವಿಷಯದಲ್ಲಿ ಎಳ್ಳಷ್ಟೂ ಸಂದೇಹವೇ ಇಲ್ಲ. ಆದರೆ ಅದನ್ನು ಬಗ್ಗಿಸುವುದಕ್ಕೆ ಒಂದು ದಾರಿ ಇದೆ ಎಂಬ ಭರವಸೆಯನ್ನು ಕೊಡುತ್ತಾನೆ. ಮನಸ್ಸಿನ ಚಂಚಲ ಸ್ವಭಾವಕ್ಕೆ ಕಾರಣವೇನು ಎಂಬುದನ್ನು ನಾವು ವಿಚಾರಿಸಬೇಕಾಗಿದೆ. ಅದರ ಚಪಲತೆಗೆ ಕಾರಣ ಅದಕ್ಕೆ ವಿಷಯ ವಸ್ತುವಿನಮೇಲೆ ಇರುವ ಆಸಕ್ತಿ. ಎಲ್ಲಿಯವರೆವಿಗೂ ಆಸಕ್ತಿ ಕುಗ್ಗಿಲ್ಲವೊ ಅಲ್ಲಿಯವರೆಗೂ ಮನಸ್ಸನ್ನು ನಿಗ್ರಹಿಸುವುದು ಕಷ್ಟ. ನಮ್ಮ ದೇಹ ಇರುವ ಕಡೆಯೆಲ್ಲ ನಮ್ಮ ಮನಸ್ಸು ಇರುವುದು. ಅದಕ್ಕೆ ಪ್ರಿಯವಾದ ವಸ್ತುವಿನ ಹತ್ತಿರ ಅದು ಇರುವುದು. ಅದಕ್ಕೆ ಯಾವುದು ಪ್ರಿಯವಾಗಿರುವುದು? ಹಿಂದಿನಿಂದ ಮೇಯುತ್ತಿದ್ದ ಇಂದ್ರಿಯ ಕಸ. ಅದರ ರುಚಿಯೊಂದೇ ಅದಕ್ಕೆ ಗೊತ್ತಿರುವುದು. ಇಂದ್ರಿಯಾತೀತ ಅನುಭವದ ರಸ ಅದಕ್ಕೆ ಇನ್ನೂ ಹತ್ತಿಲ್ಲ. ಅದನ್ನು ನೀರಸ ಎಂದು ಭಾವಿಸುವುದು. ದೇವರ ಕಡೆಗೆ ಮನಸ್ಸು ಹೋಗಬೇಕಾದರೆ, ಮೊದಲು ಇಂದ್ರಿಯ ವಸ್ತುವಿನ\break ಕಡೆ ಮನಸ್ಸು ಹೋಗದಂತೆ ನೋಡಿಕೊಳ್ಳಬೇಕು. ಇಂದ್ರಿಯ ವಸ್ತುವಿನ ಮೇಲೆ ಜುಗುಪ್ಸೆ\break ಹುಟ್ಟಿಸಿಕೊಳ್ಳಬೇಕು. ಎಷ್ಟು ಸಲ ನಾವು ಅದನ್ನು ಅನುಭವಿಸಿರುವೆವು! ಅದರಿಂದೇನು ನಮಗೆ ತೃಪ್ತಿ ಸಿಕ್ಕಿದೆಯೆ? ಇನ್ನೂ ಹಾಳಾಗಿರುವೆವೆ ಹೊರತು ಮೇಲೆ ಬರಲಿಲ್ಲ. ಸಾಕು ಇಷ್ಟು ಅನುಭವಿ\-ಸಿರುವುದೆ, ಇನ್ನು ಮೇಲಾದರೂ ಬಾಗಿಲನ್ನು ಹಾಕಿ ಹೊರಗೆ ಬರುವ ಅನುಭವಗಳನ್ನು ತಡೆದು ಮನಸ್ಸನ್ನು ಒಳ್ಳೆಯ ಭಾವನೆಗಳಿಂದ ಶುದ್ಧಿಮಾಡಿ ಕೊಳ್ಳಬೇಕು ಎನ್ನಿಸಬೇಕು. ಶುದ್ಧವಾದ ಮನಸ್ಸಿಲ್ಲದೇ ಇದ್ದರೆ ಏಕಾಗ್ರತೆ ಸಿದ್ಧಿಸಲಾರದು. ಆದ ಕಾರಣವೇ ನಮ್ಮ ಇಂದ್ರಿಯ, ದೇಹ ಮನಸ್ಸು ಭಗವಂತನಿಗೆ ಸಂಬಂಧಪಟ್ಟ ವೇದನೆಗಳನ್ನು ಸಹಿಸಲು ಅಭ್ಯಾಸವಾಗಬೇಕು. ಇಂದ್ರಿಯದ ಮೂಲಕ ನಮ್ಮನ್ನು ಪ್ರಪಂಚಕ್ಕೆ ಕಟ್ಟುವ ವೇದನೆಗಳನ್ನು ನಾವು ಇದುವರೆಗೆ ಸಂಗ್ರಹಿ\-ಸಿದ್ದೇವೆ. ಇನ್ನು ಮೇಲೆ ಅದರ ಮೂಲಕ ದೇವರಿಗೆ ಸಂಬಂಧಪಟ್ಟ ವೇದನೆಗಳನ್ನು ಸಂಗ್ರಹಿಸೋಣ ಎಂದು ಅದಕ್ಕೆ ಅಣಿಯಾಗಬೇಕು. ದೇವರಿಗೆ ಸಂಬಂಧಪಟ್ಟದ್ದನ್ನು ಕೇಳೋಣ, ನೋಡೋಣ, ಮಾಡೋಣ, ಮಾತನಾಡೋಣ, ಅದಕ್ಕೆ ಸಂಬಂಧಪಟ್ಟದ್ದನ್ನು ಓದೋಣ. ಇವುಗಳೆಲ್ಲ ಉತ್ತಮ ಸಂಸ್ಕಾರಗಳನ್ನು ನಮ್ಮ ಮನಸ್ಸಿನಲ್ಲಿ ಬಿಡುವುವು. ಈ ಉತ್ತಮ ಸಂಸ್ಕಾರಗಳು, ಹೀನ ಸಂಸ್ಕಾರಗಳನ್ನು ಅಳಿಸುತ್ತ ಬರುವುವು. ಕ್ರಮೇಣ ನಮ್ಮ ಮನಸ್ಸು ಶುದ್ಧವಾಗುತ್ತ ಬರುವುದು.

ಇದನ್ನು ಒಂದು ದಿನದಲ್ಲಿ ಮಾಡುವುದಕ್ಕೆ ಆಗುವುದಿಲ್ಲ. ಯಾವುದೊ ಮಂತ್ರ ಉಚ್ಚಾರ ಮಾಡಿದರೆ, ಯಾವುದೊ ಒಬ್ಬ ಮಹಾತ್ಮನ ಹತ್ತಿರ ಹೋದರೆ, ನಮಗೆ ಏಕಾಗ್ರತೆ ತಕ್ಷಣವೇ ಸಿದ್ಧಿಸುವುದೆಂದರೆ ಅದೆಲ್ಲ ಒಂದು ಭ್ರಮೆ. ಅದು ನಮಗೆ ಹೊರಗಡೆಯಿಂದ ಬರುವುದಿಲ್ಲ. ಹೊರಗಡೆ ಅದನ್ನು ಹೇಗೆ ಪಡೆಯಬೇಕು ಎಂಬುದನ್ನು ಹೇಳುವರು ಅಷ್ಟೆ. ಅಲ್ಲಿಗೆ ಮುಗಿಯುವುದು. ಆದರೆ ನಾವು ಅದಕ್ಕಾಗಿ ಕಷ್ಟಪಡಬೇಕು. ನಾವು ಬಲಶಾಲಿಗಳಾಗಬೇಕಾದರೆ, ಜಟ್ಟಿಗಳಾಗ\-ಬೇಕಾದರೆ, ಕಷ್ಟಪಟ್ಟು ಪ್ರತಿದಿನವೂ ಅಂಗಸಾಧನೆ ಮಾಡಬೇಕು. ಆಗಲೆ ಬಹಳ ಬಲವಾದ ಕಸರತ್ತುಗಳನ್ನು ಮಾಡಲು ಸಾಧ್ಯವಾಗುವುದು. ಯಾವ ಅಭ್ಯಾಸವೂ ಇಲ್ಲದೆ ನಮಗೆ ಏನೂ ಸಿಕ್ಕುವುದಿಲ್ಲ. ಉಪಾಯದಿಂದ ಒಬ್ಬ ಜಟ್ಟಿಯಾಗಲಾರ. ಅವನು ಅದಕ್ಕೆ ಕಷ್ಟಪಡಬೇಕು. ಅದರಂತೆಯೆ ಉಪಾಯದಿಂದ ಒಬ್ಬ ಯೋಗಿಯಾಗಲಾರ. ಯಾರ ಕೃಪಾಕಟಾಕ್ಷವೇ ಬೀಳಲಿ, ಯಾವ ಮಹಾಮಂತ್ರವೇ ಆಗಲಿ, ನಮ್ಮ ಶ್ರಮದ ಬೆವರನ್ನು ಅದಕ್ಕೆ ಎರೆಯದೇ ಇದ್ದರೆ ಅದು ಫಲಕ್ಕೆ ಬರುವುದಿಲ್ಲ. ಆಧ್ಯಾತ್ಮಿಕ ಜೀವನದಲ್ಲಿ ಸತತ ಅಭ್ಯಾಸ ಒಂದೇ ನಮ್ಮನ್ನು ದೇವರ ಕಡೆಗೆ ಕರೆದುಕೊಂಡು ಹೋಗುವುದು. ಇದನ್ನು ಶ‍್ರೀಕೃಷ್ಣ ಅಭ್ಯಾಸಯೋಗವೆಂತಲೇ ಹೇಳುತ್ತಾನೆ. ಅವತಾರವಾಗಿರುವಂತಹ ಶ‍್ರೀಕೃಷ್ಣ ಅರ್ಜುನನ ಮನಸ್ಸನ್ನು ನಿಗ್ರಹಿಸುವುದಕ್ಕೆ ಅವನಿಗೆ\break ಶಕ್ತಿಯನ್ನು ಕೊಡಬಹುದಾಗಿತ್ತಲ್ಲ ಎಂದು ನಾವು ಕೇಳಬಹುದು. ಹೌದು ಅರ್ಜುನನಿಗೆ ಅದನ್ನು ಕೊಟ್ಟಿರುವನು. ಬಿಡದೆ ಅಭ್ಯಾಸಮಾಡಿದರೆ ನಿನಗೆ ಅದು ಸಿಕ್ಕುವುದು ಎಂಬ ದಾರಿಯನ್ನು ತೋರುವನು. ನಿಜವಾದ ಗುರು ತನ್ನ ಶಿಷ್ಯನನ್ನು ತನ್ನ ಕಾಲಮೇಲೆ ನಿಂತುಕೊಳ್ಳುವಂತೆ ಮಾಡು\-ವನು. ಒಬ್ಬ ಭಿಕ್ಷುಕನಿಗೆ ನಾವು ಏನನ್ನಾದರೂ ಕೊಟ್ಟು ಕಳುಹಿಸಬಹುದು. ಅದು ಅಂದಿಗೇ ಅವತ್ತಿಗೇ ಮಾತ್ರ. ಮಾರನೆ ದಿನದಿಂದ ಅವನು ಪುನಃ ತಿರುಪೆಗೆ ಹೋಗಬೇಕಾಗುವುದು. ಆದರೆ ನಿಜವಾದ ಗುರು ಹಾಗೆ ಮಾಡುವುದಿಲ್ಲ. ನೀನೇ ಸಂಪಾದನೆ ಮಾಡಬೇಕಾದರೆ ಏನು ಮಾಡಬೇಕು ಎಂಬುದನ್ನು ಕಲಿಸುವನು. ವೈದ್ಯ ಎಷ್ಟೇ ಚೆನ್ನಾಗಿರುವ ಔಷಧಿಯನ್ನು ರೋಗಿಗೆ ಕೊಟ್ಟರೂ, ರೋಗಿ ಅದನ್ನು ಸೇವಿಸಬೇಕು. ಆಗಲೆ ರೋಗದಿಂದ ಪಾರಾಗಬೇಕಾದರೆ. ರೋಗಿಯ ಬದಲು ವೈದ್ಯನೇ ಸೇವಿಸುವುದಕ್ಕೆ ಆಗುವುದಿಲ್ಲ. ಅದರಂತೆಯೇ ಭವವೈದ್ಯನಾದ ಶ‍್ರೀಕೃಷ್ಣ ಅರ್ಜುನನಿಗೆ ಮನೋ ಏಕಾಗ್ರತೆಯನ್ನು ಪಡೆಯಬೇಕಾದರೆ ಏನು ಮಾಡಬೇಕು ಎಂಬುದನ್ನು ಹೇಳುತ್ತಾನೆಯೆ ಹೊರತು, ಅವನೇ ಅದನ್ನು ಕೊಡುವುದಿಲ್ಲ. ಅರ್ಜುನನಿಗೆ ಆ ಮನೋ ಏಕಾಗ್ರತೆಯನ್ನು ಕರುಣಿಸುವುದು ಶ‍್ರೀಕೃಷ್ಣನಿಗೆ ಅಸಾಧ್ಯವೆಂದಲ್ಲ. ಅವನು ಮಹಾಯೋಗಿ, ಸತ್ಯಸಂಕಲ್ಪ, ಇಚ್ಛಿಸಿದ್ದು ಆಗಿಯೇ ಆಗುವುದು. ಆದರೆ ಶ‍್ರೀಕೃಷ್ಣ ಅರ್ಜುನನಿಗೆ ಅದನ್ನು ಹೇಗೆ ಪಡೆಯಬೇಕೆಂಬುದನ್ನು ಹೇಳುತ್ತಾನೆ. ಆಗಲೆ ಅದು ಅವನದಾಗುವುದು. ಜೀವನದಲ್ಲಿ ಒಂದು ವಸ್ತುವಿನ ಬೆಲೆ ನಮಗೆ ಚೆನ್ನಾಗಿ ಗೊತ್ತಾಗುವುದು ನಾವೇ ಕಷ್ಟಪಟ್ಟು ಅದನ್ನು ಪಡೆದಾಗ.

ಮಹಾಗುರುವಾದ ಶ‍್ರೀರಾಮಕೃಷ್ಣರ ಬಳಿಗೆ ಶಿಷ್ಯರು ಬಂದು ದೇವರ ವಿಷಯ ಮುಂತಾದುವನ್ನು ಕೇಳಿದಾಗ, ಶ‍್ರೀರಾಮಕೃಷ್ಣರು ಒಂದು ಸನ್ನೆಯ ಮೂಲಕ, ಕೇವಲ ಇಚ್ಛೆಯ ಮೂಲಕ ಶಿಷ್ಯನಿಗೆ ಅತೀಂದ್ರಿಯ ಭಗವದಾನಂದ ಎಂತಹದು ಎಂಬುದನ್ನು ರುಚಿ ತೋರಿಸುತ್ತಿದ್ದರು. ಆದರೆ ಆ ಕ್ಷಣಿಕ ಅನುಭವವನ್ನು ತಮ್ಮದನ್ನಾಗಿ ಮಾಡಿಕೊಳ್ಳಬೇಕಾದರೆ ನಿರಂತರ ಸಾಧನೆಯನ್ನು ಅವರು ಮಾಡಬೇಕಾಗಿತ್ತು. ಹೀಗೆ ಸಾಧನೆಯನ್ನು ಮಾಡಿ ತಾತ್ಕಾಲಿಕ ಅನುಭವವನ್ನು ನಿಮ್ಮ ಸ್ವಂತದ್ದನ್ನಾಗಿ ಮಾಡಿಕೊಳ್ಳಿ ಎಂದು ಹುರಿದುಂಬಿಸುತ್ತಿದ್ದರು. ಗುರು ಕೊಟ್ಟ ಜೋಳಿಗೆ ಎಂದು ಗೂಟಕ್ಕೆ ನೇತುಹಾಕಿದರೆ ಹೊಟ್ಟೆ ತುಂಬುವುದಿಲ್ಲ. ಅದನ್ನು ತೆಗೆದುಕೊಂಡು ಮನೆಮನೆಗೆ ಅಲೆಯಬೇಕು. ಆಗಲೆ ಭಿಕ್ಷೆ ಸಿಕ್ಕಬೇಕಾದರೆ. ಅದರಂತೆಯೇ ಗುರು ಹೇಳಿದುದನ್ನು ನಾವು ಅನುಷ್ಠಾನಕ್ಕೆ\break ತರಬೇಕು. ಹಾಗೆ ಅನುಷ್ಠಾನರಂಗಕ್ಕೆ ಇಳಿದಾಗ ಹಲವು ವೇಳೆ ಜಾರುತ್ತೇವೆ, ಸೋಲುತ್ತೇವೆ. ಆದರೂ ಸ್ಫೂರ್ತಿ ತಗ್ಗದೆ ಉತ್ಸಾಹದಿಂದ ಬಿಡದೆ ನಾವು ಅಭ್ಯಾಸ ಮಾಡುತ್ತಿದ್ದರೆ ಗುರಿ ಮುಟ್ಟಬಹುದು. ಗುರಿಯನ್ನು ಮುಟ್ಟಲು ಅಭ್ಯಾಸವೇ ಏಣಿ.

\begin{shloka}
ಅಸಂಯತಾತ್ಮನಾ ಯೋಗೋ ದುಷ್ಪ್ರಾಪ ಇತಿ ಮೇ ಮತಿಃ~।\\ವಶ್ಯಾತ್ಮನಾ ತು ಯತತಾ ಶಕ್ಯೋಽವಾಪ್ತುಮುಪಾಯತಃ \hfill॥ ೩೬~॥
\end{shloka}

\begin{artha}
ಮನಸ್ಸನ್ನು ನಿಗ್ರಹಿಸದವನಿಗೆ ಯೋಗ ಅಸಾಧ್ಯ ಎಂಬುದೇ ನನ್ನ ಮತ. ಆದರೆ ಯತ್ನಶೀಲನಾದ ಜಿತೇಂದ್ರಿಯನಿಗೆ ಇದನ್ನು ಹೊಂದಲು ಸಾಧ್ಯ.
\end{artha}

ಮನಸ್ಸನ್ನು ನಿಗ್ರಹಿಸದೆ ಇದ್ದರೆ ಯೋಗ ಸಾಧ್ಯವಿಲ್ಲ. ವಿಷಯವಸ್ತುಗಳ ಕಡೆ ಹೋಗುತ್ತ ಜೊತೆಗೆ ಧ್ಯಾನಜೀವಿಯೂ ಆಗಬೇಕೆಂದು ಬಯಸಿದರೆ ಆಗ ಬಿಸಿಮಂಜಿನಂತೆ ಆಗುವುದು. ಮಂಜು ಎಂದಾದರೂ ಬಿಸಿಯಾಗಿರುವುದೆ? ಬಿಸಿಯಾಗಿದ್ದರೆ ಮಂಜಾಗುವುದೆ? ಇಂದ್ರಿಯ ಚಾಪಲ್ಯಗಳನ್ನು ಯಾವಾಗ ತೀರಿಸಿಕೊಳ್ಳುವೆವೊ ಆಗ ಅದಕ್ಕೆ ಸಂಬಂಧಪಟ್ಟ ಸಂಸ್ಕಾರಗಳನ್ನು ಅದು ನಮ್ಮ ಮನಸ್ಸಿನಲ್ಲಿ ಬಿಡುವುದು. ನಾವು ಏನನ್ನು ತಿನ್ನುತ್ತೇವೆಯೊ ಅದನ್ನೇ ತೇಗುವಂತೆ ಧ್ಯಾನಕ್ಕೆ ಕೂತಾಗಲೂ ಮನಸ್ಸಿನ ಅಂತರಾಳದಿಂದ ಈ ವಾಸನೆಯ ಗುಳ್ಳೆಗಳು ಏಳುತ್ತಿರುತ್ತವೆ. ಆಗ ಎಂತಹ ಏಕಾಗ್ರತೆ ಸಾಧ್ಯವಾಗುವುದು? ಆದಕಾರಣವೆ ಯೋಗ ಬೇರೆ, ಭೋಗ ಬೇರೆ. ಒಂದು ಬೇಕಾದರೆ ಮತ್ತೊಂದನ್ನು ಬಲಿ ಕೊಡಲು ಸಿದ್ಧವಾಗಿರಬೇಕು. ಯೋಗ ಬೇಕಾದರೆ ಭೋಗವನ್ನು ತ್ಯಾಗ ಮಾಡಬೇಕು. ಭೋಗ ನಮ್ಮನ್ನು ವಿಷಯವಸ್ತುಗಳಿಗೆ ಕಟ್ಟಿಹಾಕುವುದು. ನಮ್ಮ ಸ್ವಾತಂತ್ರ್ಯವೆಲ್ಲ ಹೋಗುವುದು. ವಿಷಯವಸ್ತುವಿನ ಸುತ್ತಲೂ ಗಾಣಕ್ಕೆ ಕಟ್ಟಿದ ಎತ್ತು ಸುತ್ತುವಂತೆ ಸುತ್ತುತ್ತಿರ ಬೇಕಾಗುವುದು. ಯೋಗಿಯಾಗಬೇಕು ಎಂದು ಬಯಸುವವನು ಇವುಗಳನ್ನೆಲ್ಲ ಪರೀಕ್ಷೆ ಮಾಡಿ ಬಿಟ್ಟಿರುವನು. ವಿಷಯವಸ್ತುಗಳ ಮೇಲಿನ ಆಸೆಯನ್ನು ಬಿಡದೆ ಯೋಗಿಯಾಗಬೇಕೆಂದು ಬಯಸಿದರೆ ನಾವು ಮಾಡುವ ಪ್ರಯತ್ನಗಳೆಲ್ಲ ನಿಷ್ಫಲವಾಗುವುದು. ಶ‍್ರೀರಾಮಕೃಷ್ಣರು ಇದನ್ನು ವಿವರಿಸುವುದಕ್ಕೆ ಒಂದು ಸಣ್ಣ ನಿದರ್ಶನ ಹೇಳುವರು: ನಾಲ್ಕು ಜನ ಚೆನ್ನಾಗಿ ಕುಡಿದು ಬಂದು ಗಂಗಾ ನದಿಯ ಮೇಲೆ ನೌಕಾ ವಿಹಾರಕ್ಕೆ ಹೊರಟರು. ದೋಣಿಯನ್ನು ತುಂಬಾ ಕಾಲ ಕೋಲಿನಿಂದ ಹುಟ್ಟುಹಾಕಿದಮೇಲೆ ನಾವು ತುಂಬಾ ದೂರ ಬಂದಿರಬೇಕೆಂದು ಸುತ್ತಲೂ ನೋಡಿದರು. ಆ ದೋಣಿಯಾದರೋ ಇದ್ದಕಡೆಯೇ ಇತ್ತು. ಏಕೆಂದರೆ ದೋಣಿಯನ್ನು ಒಂದು ಹಗ್ಗದಿಂದ ಹತ್ತಿರ ಇದ್ದ ಮರಕ್ಕೆ ಕಟ್ಟಿದ್ದರು. ಆ ಹಗ್ಗವನ್ನು ಬಿಚ್ಚದೆ ಕುಡುಕರು ನೌಕಾ ವಿಹಾರ ಮಾಡಿದ್ದೇ ಮಾಡಿದ್ದು. ಹಾಗೆ ವಿಷಯವಸ್ತುವಿಗೆ ಕಟ್ಟಿಕೊಂಡು ನಾವು ಆಧ್ಯಾತ್ಮಿಕ ಸಾಧನೆ ಮಾಡುವುದು ನಿಷ್ಫಲವಾಗುವುದು.

ಯಾರು ತನ್ನ ಮನಸ್ಸನ್ನು ನಿಗ್ರಹಿಸಲು ಪ್ರಯತ್ನಿಸುತ್ತಿರುವನೊ ಅಂತಹ ಸಾಧಕ ಯೋಗ\-ಜೀವನದಲ್ಲಿ ಮುಂದುವರಿಯುವನು. ಪ್ರಾರಂಭದಲ್ಲಿ ಸಂಪೂರ್ಣ ಜಿತೇಂದ್ರಿಯನಾಗದೆ ಇರಬಹುದು. ಆದರೆ ಅವನು ತನ್ನ ಮೇಲೆ ಸ್ವಾಮಿತ್ವವನ್ನು ಪಡೆಯುವುದಕ್ಕೆ ಪ್ರಯತ್ನ ಮಾಡುತ್ತಿರು\-ವನು. ಯಾರು ಮನಸ್ಸನ್ನು ನಿಗ್ರಹಿಸಲು ಪ್ರಯತ್ನಿಸುವನೊ, ಅವನು ಕ್ರಮೇಣ ಅದನ್ನು ಪಡೆಯುವನು. ಅನಂತರ ಯೋಗಜೀವನದಲ್ಲಿ ಮುಂದುವರಿಯುವುದು ಸಾಧ್ಯವಾಗುವುದು. ಮನಸ್ಸನ್ನು ನಿಗ್ರಹಿಸುವುದಕ್ಕೆ ನಾವು ಪ್ರಯತ್ನ ಮಾಡಬೇಕು. ಈ ಪ್ರಯತ್ನ ಅತ್ಯಂತ ಮುಖ್ಯ. ನಾವು ಅದನ್ನು ನಿಗ್ರಹಿಸಲು ಆಗುವುದಿಲ್ಲ. ನಾವು ಆಧ್ಯಾತ್ಮಿಕ ಜೀವನದಲ್ಲಿ ನಡೆಯುವುದು ನಮ್ಮ ಹಣೆಯಲ್ಲಿ ಬರೆದಿಲ್ಲ; ನೋಡಿಕೊಳ್ಳೋಣ ಮುಂದಿನ ಜನ್ಮಕ್ಕೆ ಎಂದರೆ ನಮಗೆ ಅದು ಬರುವುದು ಹೇಗೆ? ಈಗ ಏನನ್ನು ಸಂಪಾದಿಸಿಕೊಂಡಿರುವೆವೊ ಅದೇ ನಮ್ಮ ಮುಂದಿನ ಜೀವನದ ಬುತ್ತಿ. ಈಗ ದಿವಾಳಿಗಳಾಗಿ, ಏನೋ ಮುಂದಿನ ಜನ್ಮದಲ್ಲಿ ನಾವು ಪಡೆಯುತ್ತೇವೆ ಎಂದರೆ ಮಲಗಿಕೊಳ್ಳುವಾಗ ಭಿಕ್ಷುಕನಾಗಿದ್ದು ಎದ್ದಾಗ ಐಶ್ವರ್ಯವಂತನಾಗುತ್ತೇನೆ ಎಂದು ಭಾವಿಸುವಂತೆ. ಸೋಮಾರಿಗೆ ಯಾವ ಸಹಾಯವೂ ಬರುವುದಿಲ್ಲ. ದೇವರು ಸಹಾಯ ಮಾಡುವುದೂ ಕೂಡ ಯಾರು ಪ್ರಯತ್ನ ಮಾಡುತ್ತಿರುವನೊ ಅವನಿಗೆ. ಅವನಿಗೆ ಸಹಾಯ ಮಾಡಿದರೆ ಪ್ರಯೋಜನ ಉಂಟು, ಪ್ರಯತ್ನದಿಂದ ಅಸಾಧ್ಯವಾಗಿರುವುದು ಕೂಡ ಸಾಧ್ಯವಾಗುವುದು. ದೇವರ ಮೇಲೆ ಭಾರವನ್ನು ಹಾಕಿ ನಾವು ಇಂದ್ರಿಯಗಳ ಬಲೆಯಿಂದ ತಪ್ಪಿಸಿಕೊಂಡು, ಅವನ ಕಡೆ ಹೋಗಬೇಕೆಂದು ಪ್ರಯತ್ನಪಟ್ಟರೆ ಅವನ ಕೃಪಾಹಸ್ತ ನಮ್ಮ ಸಹಾಯಕ್ಕೆ ನಿಂತಿರುವುದು ಕಾಣುವುದು. ಇದನ್ನು ಶ‍್ರೀರಾಮಕೃಷ್ಣರು ತಮ್ಮ ಬೋಧನೆಯಲ್ಲಿ ಮನೋಜ್ಞವಾಗಿ ವಿವರಿಸುವರು. ಒಂದು ಹಕ್ಕಿ ಸಮುದ್ರದ ತೀರದಲ್ಲಿ ಇದ್ದ ಒಂದು ಮರದ ಮೇಲೆ ಗೂಡು ಕಟ್ಟಿಕೊಂಡಿತ್ತು. ಪ್ರತಿಯೊಂದು ಸಲ ಮೊಟ್ಟೆ ಹಾಕಿದಾಗಲೂ ಸಮುದ್ರದ ಉಬ್ಬರದ ಪ್ರವಾಹ ಬಂದು ಆ ಮೊಟ್ಟೆಗಳನ್ನೆಲ್ಲ ತೆಗೆದುಕೊಂಡು ಹೋಗಿಬಿಡುತ್ತಿತ್ತು. ಹಕ್ಕಿಗೆ ತುಂಬ ವ್ಯಥೆಯಾಯಿತು. ಈ ಸಮುದ್ರವನ್ನೇ ಬೇರೆಕಡೆಗೆ ಸಾಗಿಸಿದರೆ ತಾನು ತನ್ನ ಮೊಟ್ಟೆಗಳನ್ನೊಡೆದು ಮರಿಗಳನ್ನು ನೋಡಬಹುದೆಂದು ಬಯಸಿ ತನ್ನ ಕೊಕ್ಕಿನಿಂದ ನೀರನ್ನು ಎತ್ತಿಕೊಂಡು ದೂರಕ್ಕೆ ಹಾರಿಹೋಗಿ ಅಲ್ಲಿ ಉಗುಳಿ ಬರುತ್ತಿತ್ತು. ಒಮ್ಮೆ\break ನಾರದರು ಆ ಸಮಯದಲ್ಲಿ ಬರುತ್ತಿದ್ದರು. ಹಕ್ಕಿ ಬಿಡುವಿಲ್ಲದೆ ಸಮುದ್ರ ಖಾಲಿ ಮಾಡುತ್ತಿರುವುದನ್ನು ನೋಡಿ ‘ಏನು ಮಾಡುತ್ತಿರುವೆ ನೀನು?’ ಎಂದು ಕೇಳಿದರು. ಅದು ‘ಸಮುದ್ರವನ್ನು ಬೇರೆ ಕಡೆ ಸಾಗಿಸುತ್ತಿರುವೆನು’ ಎಂದಿತು. ನಾರದರು ಆಶ್ಚರ್ಯದಿಂದ ‘ನೀನು ಆ ಕೆಲಸ ಮಾಡುವುದಕ್ಕೆ ಎಷ್ಟು ಯುಗಗಳು ಹಿಡಿಯುವುದು?’ ಎಂದರು. ಅದಕ್ಕೆ ‘ತಾನು ಎಷ್ಟು ಯುಗಗಳಾದರೂ ಆ ಕೆಲಸ ಮಾಡುವೆ. ನಾನು ಪ್ರತಿಯೊಂದು ಸಲ ಜನ್ಮವೆತ್ತಿದರೂ ಈ ಕೆಲಸವನ್ನು ಮಾಡುವೆ. ನನಗೆ ಅನಂತ ಜನ್ಮಗಳಿವೆ. ಈ ಸಾಗರವಾದರೊ ದೊಡ್ಡದಾಗಿರುವುದು. ಆದರೆ ಅದು ಸಾಂತ’ ಎಂದಿತು. ಈ ನುಡಿಯನ್ನು ಕೇಳಿ ನಾರದರು ಆಶ್ಚರ್ಯಪಟ್ಟರು. ಇಂತಹ ಛಲ ನಮ್ಮ ಪ್ರಯತ್ನದಲ್ಲಿರಬೇಕು; ಆಗ ದೇವರು ಬಂದು ನಮಗೆ ಸಹಾಯ ಮಾಡುತ್ತಾನೆ. ಭಗವಂತನ ಕೃಪೆ ಎಂಬ ತಂಗಾಳಿ ಯಾವಾಗಲೂ ಬೀಸುತ್ತಿದೆ. ನಾವು ಪ್ರಯತ್ನಪಟ್ಟು ನಮ್ಮ ಜೀವನದೋಣಿಯ ಪಟವನ್ನು ಹರಡಿ ಕಟ್ಟಿದರೆ ಅದರಿಂದ ಪ್ರಯೋಜನವನ್ನು ಪಡೆಯಬಹುದು.

ಈ ಪ್ರಪಂಚದಲ್ಲಿ ಹುಟ್ಟು ಸಿದ್ಧಪುರುಷರು ಅಪರೂಪ. ಸಾಧನಾಬಲದಿಂದಲೇ, ಪ್ರಯತ್ನದ ಶ್ರಮದ ಮೇಲೆಯೇ ಎಲ್ಲರೂ ಸಿದ್ಧರಾಗಿರುವರು. ನಾವು ಪ್ರಯತ್ನ ಮಾಡುತ್ತಿದ್ದರೆ ಉಳಿದವುಗಳೆಲ್ಲ ನಮಗೆ ಸಹಾಯಕ್ಕೆ ಬರುತ್ತವೆ. ಶಾಸ್ತ್ರ ಸಿಕ್ಕುವುದು, ಅದನ್ನು ವಿವರಿಸುವ ಗುರು ಸಿಕ್ಕುವನು. ನಮ್ಮ ಜೀವನದಲ್ಲಿ ಭಗವಂತನ ಕಡೆ ಹೋಗುವುದಕ್ಕೆ ಇರುವ ಆತಂಕಗಳನ್ನು ಅವನು ಕಡಮೆ ಮಾಡುತ್ತ ಬರುವನು. ಆದರೆ ಪ್ರಯತ್ನ ಮುಂಚೆ ಬೇಕು. ಭಗವಂತನ ಕಡೆ ಹೋಗಬೇಕು ಎಂಬ ಅಭೀಪ್ಸೆಯ ಬೀಜ ಹೃದಯದಲ್ಲಿ ಬೀಳಬೇಕು. ಆಗ ಮಳೆ ಬರುವುದು. ಗೊಬ್ಬರ ಬೀಳುವುದು. ಕಳೆ ಕೀಳುವರು. ಅದು ಕ್ರಮೇಣ ಒಂದು ಹುಲುಸಾದ ಮರವಾಗಿ ಫಲಕ್ಕೆ ಬರುವುದು.

ಅರ್ಜುನ ಶ‍್ರೀಕೃಷ್ಣನನ್ನು ಪುನಃ ಪ್ರಶ್ನಿಸುತ್ತಾನೆ:

\begin{shloka}
ಅಯತಿಃ ಶ್ರದ್ಧಯೋಪೇತೋ ಯೋಗಾಚ್ಚಲಿತಮಾನಸಃ~।\\ಅಪ್ರಾಪ್ಯ ಯೋಗಸಂಸಿದ್ಧಿಂ ಕಾಂ ಗತಿಂ ಕೃಷ್ಣ ಗಚ್ಛತಿ \hfill॥ ೩೭~॥
\end{shloka}

\newpage

\begin{artha}
ಶ‍್ರೀಕೃಷ್ಣ, ಶ್ರದ್ಧಾಯುಕ್ತನಾದರೂ, (ಸಾಕಷ್ಟು) ಪ್ರಯತ್ನವನ್ನು ಮಾಡದೆ ಮನಸ್ಸು ಯೋಗದಿಂದ ಚ್ಯುತವಾದರೆ, ಯೋಗ ಸಂಸಿದ್ಧಿಯನ್ನು ಪಡೆಯದ ಅವನ ಪಾಡೇನು?
\end{artha}

ಇಲ್ಲಿ ನಾವುಗಳೆಲ್ಲ ಕೇಳುವ ಒಂದು ಪ್ರಶ್ನೆಯನ್ನೇ ಅರ್ಜುನ ಶ‍್ರೀಕೃಷ್ಣನಿಗೆ ಹಾಕುತ್ತಾನೆ. ಮಾಡುವ ಪ್ರಯತ್ನವನ್ನೆಲ್ಲ ಮಾಡಿ ಕೊನೆಗೆ ಗುರಿಯನ್ನು ಮುಟ್ಟದೆ ಹೋದರೆ, ಪ್ರಯತ್ನವೆಲ್ಲ ವ್ಯರ್ಥವಾಗಲಿಲ್ಲವೆ? ಅವನ ಪಾಡೇನಾಗುವುದು ಎಂದು ಕೇಳುತ್ತಾನೆ. ಹೋರಾಡುವವನಿಗೆ ಶ್ರದ್ಧೆ ಇದೆ. ಆಧ್ಯಾತ್ಮಿಕ ಜೀವನದ ವಿಷಯದಲ್ಲಿ ದೃಢವಾದ ನಂಬಿಕೆ ಇದೆ. ಅದರ ಕಡೆ ಹೋಗಬೇಕೆಂದು ಹೊರಟಿರುವನು. ಕೆಲವು ದೂರ ಹೋಗಿಯೇ ಹೋಗಿರುವನು. ಆದರೆ ಆಗ ಮನಸ್ಸು ಸೋಲುವುದು. ಹಿಂದಿನ ಆಸೆ ಆಕಾಂಕ್ಷೆಗಳು ಮೇಲಕ್ಕೆ ಏಳುವುವು. ದಾರಿಯಲ್ಲಿ ಜಾರುವನು. ಯಾವು\-ದನ್ನು ಬಿಟ್ಟಿದ್ದನೊ ಅದನ್ನು ಪೂರ್ಣ ನಿಗ್ರಹಿಸುವುದಿಲ್ಲ. ಹಿಂದಿನಿಂದ ಓಡಿಬಂದು ಇವನನ್ನು ಹಿಡಿಯುವುದು, ಅದರ ವಶವಾಗುವನು. ಅವನ ಪಾಡೇನು ಆಗುವುದು? ಪ್ರಯತ್ನ ಮಾಡಿ ಗೆಲ್ಲಬೇಕು. ಗೆಲ್ಲದೇ ಇದ್ದರೆ ಮಾಡಿದ ಪ್ರಯತ್ನವೆಲ್ಲ ವ್ಯರ್ಥವಾಗಲಿಲ್ಲವೆ ಎಂದು ಕೇಳುತ್ತಾನೆ.

\begin{shloka}
ಕಚ್ಚಿನ್ನೋಭಯವಿಭ್ರಷ್ಟಶ್ಛಿನ್ನಾಭ್ರಮಿವ ನಶ್ಯತಿ~।\\ಅಪ್ರತಿಷ್ಠೋ ಮಹಾಬಾಹೋ ವಿಮೂಢೋ ಬ್ರಹ್ಮಣಃ ಪಥಿ \hfill॥ ೩೮~॥
\end{shloka}

\begin{artha}
ಶ‍್ರೀಕೃಷ್ಣ, ಉಭಯಭ್ರಷ್ಟನಾಗಿ ಬ್ರಹ್ಮನ ದಾರಿಯಲ್ಲಿ ದಿಕ್ಕು ತೋರದೆ ಚೂರಾದ ಮೋಡದಂತೆ ಆಶ್ರಯವಿಲ್ಲದೆ ಕೆಟ್ಟುಹೋಗುವುದಿಲ್ಲವೆ?
\end{artha}

ಒಬ್ಬ ಬ್ರಹ್ಮನ ಕಡೆ ಹೋಗುತ್ತಿರುವಾಗ, ಅದನ್ನು ಸೇರುವುದಕ್ಕೆ ಮುಂಚೆ ಭ್ರಷ್ಟನಾದರೆ ಅವನ ಸ್ಥಿತಿ ತ್ರಿಶಂಕುವಿನಂತೆ ಆಗುವುದಿಲ್ಲವೆ ಎಂದು ಕೇಳುತ್ತಾನೆ ಅರ್ಜುನ. ಅವನ ಕಡೆ ಹೋಗಬೇಕೆಂದು ಇಂದ್ರಿಯಗಳ ಸಹವಾಸವನ್ನು ಬಿಟ್ಟ. ಅನಂತರ ಗುರಿ ಮುಟ್ಟಲಿಲ್ಲ. ದಾರಿಯಲ್ಲಿ ಎಡವಿ ಭ್ರಷ್ಟನಾದರೆ ಹಿಂದಕ್ಕೆ ಹೋಗುವುದಕ್ಕೆ ನಾಚಿಕೆ. ಬಿಟ್ಟದ್ದನ್ನು ತೆಗೆದುಕೊಳ್ಳುವುದು ವಾಂತಿ ಮಾಡಿದ್ದನ್ನು ಪುನಃ ಉಂಡಂತೆ ಆಗುವುದು. ಮೊದಲಿನದು ಬರಲಿಲ್ಲ, ಕೆಳಗಿನದು ಬಿಟ್ಟು ಆಯಿತು. ಇದು ಹೇಗಿದೆ ಎಂಬುದನ್ನು ಒಂದು ತುಂಡು ಮೋಡದ ಉದಾಹರಣೆಯಿಂದ ಸ್ಪಷ್ಟಪಡಿಸುವನು. ಒಂದು ತುಂಡು ಮೋಡ ತಾನಿದ್ದ ದೊಡ್ಡ ಮೋಡವನ್ನು ಬಿಟ್ಟು ಇನ್ನೊಂದು ದೊಡ್ಡ ಮೋಡವನ್ನು ಸೇರುವುದಕ್ಕೆ ಹೊರಟಿತು. ಇದ್ದ ದೊಡ್ಡ ಮೋಡವನ್ನು ಬಿಟ್ಟಿತು. ಮುಂದೆ ಸೇರಬೇಕಾದ ದೊಡ್ಡ ಮೋಡವನ್ನು ಸೇರಲಿಲ್ಲ. ಅದು ಯಾವ ಪ್ರಯೋಜನಕ್ಕೂ ಬರದೆ ನಾಶವಾಗುವುದು.

\begin{shloka}
ಏತನ್ಮೇ ಸಂಶಯಂ ಕೃಷ್ಣ ಛೇತ್ತುಮರ್ಹಸ್ಯಶೇಷತಃ~।\\ತ್ವದನ್ಯಃ ಸಂಶಯಾಸ್ಯಾಸ್ಯ ಛೇತ್ತಾ ನ ಹ್ಯುಪಪದ್ಯತೇ \hfill॥ ೩೯~॥
\end{shloka}

\begin{artha}
ಶ‍್ರೀಕೃಷ್ಣ, ನನ್ನ ಸಂಶಯವನ್ನು ನಿಶ್ಶೇಷವಾಗಿ ಪರಿಹರಿಸಲು ನೀನು ಮಾತ್ರ ಅರ್ಹನು. ನಿನ್ನನ್ನು ಬಿಟ್ಟರೆ ಬೇರೆ ಯಾರೂ ಇಲ್ಲ.
\end{artha}

\newpage

ಶ‍್ರೀಕೃಷ್ಣನಿಗೊಬ್ಬನಿಗೇ ತನ್ನ ಸಂಶಯವನ್ನು ನಿಶ್ಶೇಷವಾಗಿ ಪರಿಹರಿಸಲು ಸಾಧ್ಯ ಎಂದು ಅರ್ಜುನ ಹೇಳುತ್ತಾನೆ. ಇದು ಅಕ್ಷರಶಃ ಸತ್ಯ. ಶ‍್ರೀಕೃಷ್ಣ ಯೋಗೇಶ್ವರ. ಅವನಿಗೆ ಆಧ್ಯಾತ್ಮಿಕ ಜೀವನ ಆಪಾದಮಸ್ತಕವಾಗಿ ತಿಳಿದಿದೆ. ಇಂತಹ ವ್ಯಕ್ತಿ ಮಾತನಾಡಿದಾಗ ಆ ಮಾತಿನ ಹಿಂದೆ ಒಂದು ಅಧಿಕಾರವಾಣಿ ಇರುವುದು. ಇಂತಹ ಮಾತಿಗೆ ಮತ್ತೊಬ್ಬರ ಸಂಶಯವನ್ನು ನಿವಾರಿಸುವ ಶಕ್ತಿ ಇದೆ. ಮತ್ತೊಬ್ಬರ ಮೇಲೆ ಪರಿಣಾಮವನ್ನು ಉಂಟುಮಾಡುವುದೇ ಹಿಂದುಗಡೆ ಇರುವ ಶಕ್ತಿ. ಬರೀ ಸುಂದರವಾಗಿ ಜಾಣತನದಿಂದ ಮಾತನಾಡುವವರೆಷ್ಟೊ ಮಂದಿ ಇರುವರು.\break ಪಾಂಡಿತ್ಯಪೂರ್ಣರಾಗಿರುವವರೆಷ್ಟೊ ಮಂದಿಗಳು ಇರುವರು. ಇದರ ಹಿಂದೆ ಶಕ್ತಿ ಇಲ್ಲ, ಇದು ಚಂದ್ರನ ಚಿತ್ರಕ್ಕೂ ನಿಜವಾದ ಚಂದ್ರನಿಗೂ ಇರುವ ವ್ಯತ್ಯಾಸದಂತೆ. ಶ‍್ರೀಕೃಷ್ಣನಂತಹ ವ್ಯಕ್ತಿ ಮಾತನಾಡಿದಾಗ ಸಂಶಯ ಬೇರುಸಹಿತ ನಾಶವಾಗುತ್ತದೆ. ಮತ್ತೊಮ್ಮೆ ತಲೆ ಎತ್ತುವಂತಿಲ್ಲ. ಸಾಧಾರಣ ವ್ಯಕ್ತಿಗಳು ಮಾತನಾಡಿದಾಗ ತಾತ್ಕಾಲಿಕವಾಗಿ ಸಂಶಯ ಹೋದಂತಿರುವುದು. ಆದರೆ ಪುನಃ ಬರುವುದು. ಇಲ್ಲವೇ ಒಂದು ಸಂಶಯ ಹೋಗುವುದು, ಅದಕ್ಕಿಂತ ದೊಡ್ಡ ಸಂಶಯ ಬಂದು ಆ ಸ್ಥಳದಲ್ಲಿ ಕೂಡುವುದು.

ಬೇರು ಸಹಿತ ಸಂಶಯವನ್ನು ಕೀಳಬೇಕಾದರೆ, ಶ‍್ರೀಕೃಷ್ಣನಂತಹ ವ್ಯಕ್ತಿಗೆ ಮಾತ್ರ ಸಾಧ್ಯ. ಉರಿ ಮಾತ್ರ ಮತ್ತೊಂದಕ್ಕೆ ಉರಿಯನ್ನು ದಾನಮಾಡಬಲ್ಲುದು. ಸುಮ್ಮನೆ ಉರಿ ಉರಿ ಎಂದು ಹೇಳುತ್ತಿದ್ದರೆ ಉರಿ ಬರುವುದಿಲ್ಲ. ಯಾರ ಜೀವನದಲ್ಲಿ ಆಧ್ಯಾತ್ಮಿಕ ಜ್ಞಾನ ನಂದಾದೀಪದಂತೆ ಉರಿಯುತ್ತಿದೆಯೋ, ಅವನು ಮಾತ್ರ ಇತರ ದೀವಿಗೆಗಳನ್ನು ಹಚ್ಚಬಲ್ಲನು. ಅನ್ಯರಿಗೆ ಸಾಧ್ಯವಿಲ್ಲ. ಶ‍್ರೀಕೃಷ್ಣ ಒಂದು ಅಪೂರ್ವ ವ್ಯಕ್ತಿ. ಅದಕ್ಕಾಗಿಯೇ ಅರ್ಜುನ ಅವನನ್ನು ಬೆಳಕಿಗಾಗಿ ಯಾಚಿಸುವನು.

ಶ‍್ರೀಕೃಷ್ಣ ಹೇಳುತ್ತಾನೆ:

\begin{shloka}
ಪಾರ್ಥ ನೈವೇಹ ನಾಮುತ್ರ ವಿನಾಶಸ್ತಸ್ಯ ವಿದ್ಯತೇ~।\\ನ ಹಿ ಕಲ್ಯಾಣಕೃತ್ ಕಶ್ಚಿದ್ದುರ್ಗತಿಂ ತಾತ ಗಚ್ಛತಿ \hfill॥ ೪೦~॥
\end{shloka}

\begin{artha}
ಅರ್ಜುನ, ಯೋಗಭ್ರಷ್ಟನಿಗೆ ಈ ಲೋಕದಲ್ಲಾಗಲಿ ಪರಲೋಕದಲ್ಲಾಗಲಿ ನಾಶವಿಲ್ಲ. ಏಕೆಂದರೆ ಒಳ್ಳೆಯದನ್ನು ಮಾಡಿದವನು ಎಂದಿಗೂ ದುರ್ಗತಿಯನ್ನು ಹೊಂದುವುದಿಲ್ಲ.
\end{artha}

ಶ‍್ರೀಕೃಷ್ಣ ಇಲ್ಲಿ ಅರ್ಜುನನಿಗೆ, ಅವನ ಮೂಲಕ ಎಲ್ಲಾ ಮಾನವಕೋಟಿಗೆ ಒಂದು ದೊಡ್ಡ ಭರವಸೆಯ ಸಂದೇಶವನ್ನು ನೀಡುತ್ತಾನೆ. ಯೋಗಿಯಾಗುವುದಕ್ಕೆ ಯತ್ನಿಸುತ್ತಿರುವವನು, ಈ ಪ್ರಪಂಚದಲ್ಲಿ ಎಲ್ಲರೂ ವಿಷಯವಸ್ತುವಿನ ಕಡೆ ಧಾವಿಸುತ್ತಿರುವಾಗ, ಅದನ್ನು ಬಿಟ್ಟು ದೇವರ ಕಡೆ ಹೋಗುವನು. ಸಾಗರದ ಕಡೆ ಹರಿದುಕೊಂಡು ಹೋಗುತ್ತಿರುವ ನದಿಯ ಮೇಲೆ, ಅದಕ್ಕೆ ವಿರೋಧವಾಗಿ ಈಜಿಕೊಂಡು ಹೋಗುವಂತೆ ಇದು. ಇದೊಂದು ಕಲ್ಯಾಣಕಾರ್ಯ, ಒಳ್ಳೆಯ ಕೆಲಸ. ಇಲ್ಲಿ ಜಯಶೀಲನಾಗದೆ ಇರಬಹುದು. ಆದರೆ ಎಂದಿಗೂ ಅವನು ಕೆಡುವುದಿಲ್ಲ. ಅವನು ಇಲ್ಲಿಯೂ ಕೆಡುವುದಿಲ್ಲ, ಮುಂದೆಯೂ ಕೆಡುವುದಿಲ್ಲ ಎನ್ನುವನು. ಇಲ್ಲಿ ತನ್ನ ಮನಸ್ಸನ್ನು ನಿಗ್ರಹಿಸಿ ಭಗವಂತನ ಕಡೆಗೆ ಹೋಗಬೇಕೆಂದು ಪ್ರಯತ್ನ ಮಾಡುತ್ತಿರುವವನಲ್ಲಿ ಪ್ರತಿಯೊಂದು ಪ್ರಯತ್ನವೂ ಒಂದು ಉತ್ತಮ ಸಂಸ್ಕಾರವನ್ನು ಬಿಡುವುದು. ಇದು ಎಂದಿಗೂ ವಿಫಲವಾಗುವುದಿಲ್ಲ. ಅಷ್ಟು ಹೀನ ಸಂಸ್ಕಾರ ಹೋಯಿತು. ಅದರ ಸ್ಥಳದಲ್ಲಿ ಉತ್ತಮ ಸಂಸ್ಕಾರ ಬಂತು. ನಾನೆಷ್ಟು ಪ್ರಯತ್ನ ಮಾಡಿದೆನೊ ಅದಕ್ಕೆ ತಕ್ಕ ಪ್ರತಿಫಲ ಆಗಲೇ ನನ್ನ ಮನಸ್ಸಿನ ಮೇಲೆ ಆಗಿದೆ. ಅದೆಂದಿಗೂ ನಷ್ಟವಾಗುವುದಿಲ್ಲ. ಒಂದು ವಸ್ತುವನ್ನು ಕೊಳ್ಳಬೇಕಾದರೆ ಅದಕ್ಕೆ ನಾನು ಒಂದು ಸಾವಿರ ರೂಪಾಯಿಯನ್ನು ಕೊಡಬೇಕಾಗಿದೆ. ನಾನು ಕಷ್ಟಪಟ್ಟು ನೂರು ರೂಪಾಯಿ ಸಂಪಾದಿಸುವೆನು. ಇದರಿಂದ ಆ ವಸ್ತು ಸಿಕ್ಕುವುದಿಲ್ಲ. ಆದರೆ ನಾನು ಸಂಪಾದನೆ ಮಾಡಿರುವುದು ವ್ಯರ್ಥವಲ್ಲ. ನಾನು ಅದಕ್ಕೆ ಮತ್ತಷ್ಟು ಕೂಡಿಸಬೇಕಾಗಿದೆ. ಅದನ್ನು ಸಂಗ್ರಹಿಸುವುದಕ್ಕೆ ನಮಗೆ ಅವಕಾಶ ಬೇಕಾದಷ್ಟು ಇದೆ. ಇದೊಂದೇ ಜನ್ಮವಲ್ಲ ನಮಗೆ. ಹಿಂದೆ ಎಷ್ಟೋವೇಳೆ ಲೆಕ್ಕವಿಲ್ಲದಷ್ಟು ಜನ್ಮ ಎತ್ತಿರುವೆವು. ಮುಂದೆ ಪೂರ್ಣತೆಯನ್ನು ಪಡೆಯುವವರೆಗೆ ಜನ್ಮಗಳನ್ನು ಎತ್ತುವೆವು. ಕಾಲದ ಅನಂತತೆಯೊಡನೆ ಹೋಲಿಸಿದರೆ ಒಂದು ಜನ್ಮವೆಂಬುದು ಕಣ್ಮಿಟುಕಿನಂತೆ. ನಾವು ಯಾವಾಗ ಈ ಜನ್ಮವನ್ನು ತೊರೆಯುವೆವೊ ಆಗ ಇಲ್ಲಿ ಮಾಡಿದ ಪಾಪಪುಣ್ಯಗಳೆಲ್ಲ ನಮ್ಮೊಡನೆ ಬರುವುದು. ನಾವು ಮಾಡಿದ ಯಾವ ಸತ್ಕರ್ಮದ ಫಲವೂ ವ್ಯರ್ಥವಾಗುವುದಿಲ್ಲ. ನನ್ನ ನೆರಳಿನಂತೆ ನನ್ನನ್ನು ಹಿಂಬಾಲಿಸುವುದು. ನಾನು ಮರೆತರೂ ಅದು ನನ್ನನ್ನು ಮರೆಯದು. ಅದರಿಂದ ಮುಂದೆಯೂ ನಷ್ಟವಾಗುವುದಿಲ್ಲ.

ಒಳ್ಳೆಯದನ್ನು ಮಾಡಿದವನಿಗೆ ದುರ್ಗತಿ ಇಲ್ಲ ಎನ್ನುವನು ಶ‍್ರೀಕೃಷ್ಣ. ನಾವು ಕಡಲೇಕಾಯಿಯನ್ನು ಬಿತ್ತಿದರೆ ನಮಗೆ ನೆಗ್ಗಲು ಮುಳ್ಳು ಬರುವುದಿಲ್ಲ. ನಾವು ಬಿತ್ತಿರುವುದೇ ನಮಗೆ ಹತ್ತಾಗಿ ನೂರಾಗಿ ಬರುವುದು. ಇದು ಪ್ರಕೃತಿಯ ನಿಯಮ. ಇದೇ ಭಗವಂತನ ನಿಯಮ. ನಾವು ಒಳ್ಳೆಯ ಕೆಲಸ ಮಾಡಿ ಸೋತರೆ, ನಮ್ಮನ್ನು ಹಳ್ಳಕ್ಕೆ ಹಾಕುವ ಹಾಗಿದ್ದರೆ, ಈ ಪ್ರಪಂಚದಲ್ಲಿ ಒಂದು ಸಲವೂ ಸೋಲದೆ ಎಷ್ಟು ಜನ ಗುರಿ ಮುಟ್ಟಿರುವರು? ಸೋಲೇ ಮೆಟ್ಟಿಲು, ಗೆಲುವಿನ ಕಡೆ ಏರಲು. ಸ್ವಾಮಿ ವಿವೇಕಾನಂದರು ಒಮ್ಮೆ ಈ ವಿಷಯವನ್ನು ಕುರಿತು ಮಾತನಾಡುತ್ತಿದ್ದಾಗ ಹೀಗೆ ಹೇಳುತ್ತಾರೆ: ಒಂದು ಮಹಾ ಆದರ್ಶಕ್ಕೆ ಹೋರಾಡಿ ಸೋತವನನ್ನು ನೋಡುವುದು ಕೂಡಾ ಒಂದು ಪುಣ್ಯ. ಆ ದೃಶ್ಯವೇ ನಿನ್ನನ್ನು ಪರಿಶುದ್ಧಿ ಮಾಡುವುದು ಎನ್ನುವರು. ಜೀವನದಲ್ಲಿ ನಡೆಯುವವರೇ ಎಡಹುವುದು. ಸುಮ್ಮನೆ ಕುಳಿತವನಲ್ಲ, ಮಲಗಿದವನಲ್ಲ, ನಡೆಯುವವರನ್ನು ನೋಡಿ ಟೀಕಿಸುವವನಲ್ಲ, ಹೀಯಾಳಿಸುವವನಲ್ಲ. ಇತರರೆಲ್ಲ ಅಜ್ಞಾನದ ನಿದ್ರೆಯಲ್ಲಿ ಮಲಗಿರುವಾಗ ವಿಷಯವಸ್ತುಗಳ ಮರೀಚಿಕೆಯನ್ನು ಬೇಟೆಯಾಡುತ್ತಿರುವಾಗ ಸಾರ್ಥಕವಾದುದನ್ನು ಪಡೆಯುವುದಕ್ಕೆ ಯತ್ನಿಸಿದವನು ಯೋಗಭ್ರಷ್ಟ. ಈ ಮಹಾ ಹೋರಾಟದಲ್ಲಿ ಸೋತಿರುವನು. ಈ ಸೋಲು ತಾತ್ಕಾಲಿಕ. ಇದು ನಡೆದು ಸಾಕಾದವನು ಸುಧಾರಿಸಿಕೊಳ್ಳುವಂತೆ. ಅನಂತರ ಮೇಲೆದ್ದು ಗುರಿ ಎಡೆಗೆ ನಡೆಯುವನು. ಒಂದಲ್ಲ ಒಂದು ದಿನ ಗುರಿಯನ್ನು ಸೇರಿಯೇ ಸೇರುವನು. ದೇವರು ಮೆಚ್ಚುವುದು ಪ್ರಯತ್ನವನ್ನು. ಎಲ್ಲಿ ಪ್ರಯತ್ನವಿದೆಯೋ ಅದನ್ನು ವ್ಯರ್ಥವಾಗದಂತೆ ನೋಡಿಕೊಳ್ಳುವನು ದೇವರು. ಕೊನೆಗೆ ಇದು ಪರಮ ಜಯದಲ್ಲಿ ಪರ್ಯವಸಾನವಾಗುವುದು.

\begin{shloka}
ಪ್ರಾಪ್ಯ ಪುಣ್ಯಕೃತಾಂ ಲೋಕಾನುಷಿತ್ವಾ ಶಾಶ್ವತೀಃ ಸಮಾಃ~।\\ಶುಚೀನಾಂ ಶ‍್ರೀಮತಾಂ ಗೇಹೇ ಯೋಗಭ್ರಷ್ಟೋಽಭಿಜಾಯತೇ \hfill॥ ೪೧~॥
\end{shloka}

\begin{artha}
ಯೋಗಭ್ರಷ್ಟ ಪುಣ್ಯಶಾಲಿಗಳ ಲೋಕವನ್ನು ಹೊಂದಿ, ಅಲ್ಲಿ ಅನೇಕ ವರ್ಷಗಳವರೆಗೂ ವಾಸ\-ವಾಗಿದ್ದು, ಅನಂತರ ಸದಾಚಾರಶೀಲರೂ ಶ‍್ರೀಮಂತರೂ ಆಗಿರುವವರ ಮನೆಯಲ್ಲಿ ಹುಟ್ಟುವನು.
\end{artha}

ಯೋಗದಿಂದ ಚ್ಯುತನಾಗುವುದಕ್ಕೆ ಕಾರಣ ಅವನಲ್ಲಿರುವ ಆಸೆ ಆಕಾಂಕ್ಷೆಗಳು. ಇವು\break ಅವನನ್ನು ಅಡ್ಡಹಾದಿಗೆ ಹೋಗುವಂತೆ ಮಾಡುವುವು. ಮುಖ್ಯವಾದ ಗುರಿಯನ್ನು ಮರೆಯುತ್ತಾನೆ. ತಾತ್ಕಾಲಿಕ ಚಪಲಕ್ಕೆ ಒಳಗಾಗುತ್ತಾನೆ. ಇಂತಹ ಸ್ಥಿತಿಯಲ್ಲಿ ಅವನು ಕಾಲವಾದರೆ,\break ದೇವರು ಅವನನ್ನೇನು ನರಕಕ್ಕೆ ತಳ್ಳುವುದಿಲ್ಲ. ಎಷ್ಟೋ ಜನ ಇಂದ್ರಿಯ ಪ್ರಪಂಚದ ಸುಖವನ್ನು ಹೀರುವುದರಲ್ಲಿ ಉನ್ಮತ್ತರಾಗಿರುವರು. ಅಲ್ಲಿ ಎಷ್ಟು ಪೆಟ್ಟು ಬಿದ್ದರೂ, ಚೇಳಿನಂತೆ ಅವುಗಳ ಕೈಯಿಂದ ಕುಟುಕಿಸಿಕೊಂಡರೂ ಪುನಃ ಪುನಃ ಅವುಗಳ ಸಮೀಪಕ್ಕೆ ಹೋಗುತ್ತಿರುವರು. ಇಂಥವರ ಗುಂಪಿನಲ್ಲಿ ದೇವರ ಕಡೆ ಹೋಗುವವನು ಸಹಸ್ರಾರು ಜನರಲ್ಲಿ ಒಬ್ಬ. ದೇವರು ಇದನ್ನು ಮೆಚ್ಚುತ್ತಾನೆ. ತಾತ್ಕಾಲಿಕವಾಗಿ ಹಿಂದಿನ ಹೀನವಾಸನೆಗಳಿಗೆ ಬಲಿಯಾಗಿ ಆಧ್ಯಾತ್ಮಿಕ ಪುರೋಗಮನ ನಿಂತರೂ, ದೇವರು ಅವನನ್ನು ಮರೆಯುವುದಿಲ್ಲ. ಯಾವ ಲೋಕದಲ್ಲಿ ಅವನು ತನ್ನ ಬಯಕೆಗಳನ್ನು ಪೂರ್ಣಮಾಡಿಕೊಳ್ಳಲು ಸಾಧ್ಯವೋ ಅಂತಹ ಕಡೆ ಅವನು ಹೋಗುತ್ತಾನೆ. ಅಲ್ಲಿ ತನ್ನ ಬಯಕೆಗಳನ್ನು ಪೂರ್ಣಮಾಡಿಕೊಳ್ಳುತ್ತಾನೆ. ಇಷ್ಟೇ ಇದರ ಬಂಡವಾಳ ಎಂದು ಅದರ ಕ್ಷಣಿಕತೆ ಚೆನ್ನಾಗಿ ಮನದಟ್ಟಾದಮೇಲೆ, ಹಿಂದೆ ಆಧ್ಯಾತ್ಮಿಕ ಜೀವನದ ಹಾದಿಯಲ್ಲಿ ನಡೆಯುತ್ತಿದ್ದಾಗ ಎಲ್ಲಿ ದಾರಿ ತಪ್ಪಿದನೋ, ಅಡ್ಡಹಾದಿ ಹಿಡಿದಿದ್ದನೊ ಅಲ್ಲಿಗೆ ಬಂದು ಪುನಃ ಗುರಿಯೆಡೆಗೆ ಪ್ರಯಾಣವನ್ನು ಮುಂದುವರಿಸುವನು. ಆಧ್ಯಾತ್ಮಿಕ ಜೀವನದ ನೆನಪು ಬಂದು, ಪ್ರಚೋದನೆಯ ವ್ಯಾಮೋಹದಿಂದ ಪಾರಾದ ಮೇಲೆ, ಸದಾಚಾರಶೀಲರು ಶ‍್ರೀಮಂತರು ಆದವರ ಮನೆಯಲ್ಲಿ ಜನ್ಮವೆತ್ತುತ್ತಾನೆ. ಅಲ್ಲಿ ಆಧ್ಯಾತ್ಮಿಕ ಜೀವನ ಮನೋವೃತ್ತಿ ಇವನಿಗೆ ವಿಕಾಸವಾಗುತ್ತದೆ. ಯೋಗ್ಯ ವಾತಾವರಣ ಸಿಕ್ಕುವುದು. ಶ‍್ರೀಮಂತರ ಮನೆಯಲ್ಲಿ ಹುಟ್ಟುತ್ತಾನೆ. ಆ ಮನೆಯಲ್ಲಿ ಹೊಟ್ಟೆಗೆ ಬಟ್ಟೆಗೆ ಕಷ್ಟವಿಲ್ಲ. ಜೀವನೋಪಾಯ ಸುಲಭವಾಗಿ ಸಾಗುತ್ತಿರುವುದು. ಹೆಚ್ಚು ಕಾಲವನ್ನು ಗಹನವಾದ ಚಿಂತನೆಗೆ ಕೊಡಲು ಸಾಧ್ಯವಾಗುವುದು. ಆ ಮನೆಯವರು ಜೊತೆಗೆ ಸದಾಚಾರಶೀಲರು. ಶ‍್ರೀಮಂತರೆಲ್ಲ ಸದಾಚಾರಶೀಲರಲ್ಲ. ಅನೇಕವೇಳೆ ಶ‍್ರೀಮಂತಿಕೆ ಮತ್ತು ದುರಾಚಾರ ಒಂದನ್ನೊಂದು ಎಡೆಬಿಡದೆ ಹೋಗುವುದು. ಆದರೆ ಭ್ರಷ್ಟನಾದ ಯೋಗಿ ಸದಾಚಾರಶೀಲರ ಮನೆಯಲ್ಲಿ ಹುಟ್ಟುತ್ತಾನೆ. ಅವನಿಗೆ ಅಲ್ಲಿ ಒಂದು ಒಳ್ಳೆಯ ವಾತಾವರಣ ಸಿಕ್ಕುವುದು. ಆಧ್ಯಾತ್ಮಿಕ ಜೀವನದ ವಿಕಾಸಕ್ಕೆ ಸುತ್ತಮುತ್ತಲೂ ನಮ್ಮ ಆದರ್ಶಕ್ಕೆ ವಿರೋಧವಾಗಿರುವವರೆ ಇದ್ದರೆ, ನಮ್ಮ ಮಾನಸಿಕ ಶಕ್ತಿಯ ಬಹುಪಾಲು ಅದರೊಂದಿಗೆ ಹೋರಾಡುವುದಕ್ಕೆ ವ್ಯರ್ಥವಾಗುವುದು. ಆದರೆ ವಾತಾವರಣ ಯೋಗ್ಯವಾಗಿದ್ದರೆ, ಅದು ನನ್ನ ಆಧ್ಯಾತ್ಮಿಕ ವಿಕಾಸಕ್ಕೆ ಹೆಚ್ಚು ಸಹಾಯ ಮಾಡುವುದು. ಇಲ್ಲಿ ದೇವರು ಭ್ರಷ್ಟನಾದನಲ್ಲ ಎಂದು ಕೋಪಿಸಿಕೊಳ್ಳುವುದಿಲ್ಲ. ಈತ ಇಷ್ಟನ್ನಾದರೂ ಮಾಡಿದನಲ್ಲ ಎಂದು ಜೀವಿ ಸಾಧಿಸಿರುವುದನ್ನು ನೋಡುತ್ತಾನೆಯೆ ಹೊರತು, ಸಾಧಿಸದೆ ಇರುವುದಕ್ಕೆ ಶಿಕ್ಷೆಯನ್ನು ಕೊಡುವುದಿಲ್ಲ. ಅನೇಕ ವೇಳೆ ನಾವು ಆಧ್ಯಾತ್ಮಿಕ ಜೀವನದ ಹಾದಿಯಲ್ಲಿ ಹೋಗುವುದಕ್ಕೆ ಅನುಮಾನಿಸುವುದೇ ಈ ಕಾರಣಕ್ಕಾಗಿ. ನಮಗೆ ಗುರಿ ಸಿಕ್ಕದೆ ಇದ್ದರೆ, ಎರಡನ್ನೂ ಕಳೆದುಕೊಳ್ಳು\-ತ್ತೇವಲ್ಲ ಎಂಬ ಚೌಕಾಶಿಯ ಬುದ್ಧಿ. ಆದರೆ ಚೌಕಾಶಿಯ ಬುದ್ಧಿ ಪ್ರಪಂಚದ ಇತರ ಕಡೆಗಳಲ್ಲಿಯೂ ಇರುವುದೆ? ಸಂಸಾರದ ಕಡೆ ನುಗ್ಗುವೆವು. ಅಲ್ಲಿ ನಮಗೆ ಯಾವ ಪೆಟ್ಟುಗಳು ಕಾದಿವೆಯೊ\break ಗೊತ್ತಿಲ್ಲ. ಯಾವ ಕಣ್ಣೀರಿನ ಕೋಡಿಯಲ್ಲಿ ಈಜಬೇಕಾಗಿದೆಯೋ, ಯಾರಿಂದ ಕೇಳಬಾರದ ಮಾತನ್ನು ಕೇಳಬೇಕಾಗಿದೆಯೊ, ಯಾವುದೂ ಗೊತ್ತಿಲ್ಲ. ಎಲ್ಲಾ ನಮ್ಮ ಬಯಕೆಗಳು ಈಡೇರುತ್ತವೆ ಎಂದು ಕಣ್ಣು ಮುಚ್ಚಿಕೊಂಡು ಧುಮುಕುವೆವು. ನಮಗೆ ಎಚ್ಚರವಾಗುವುದು ಕಹಿ ನೆನಪು ಬಂದಾಗಲೆ. ಆದರೆ ಇದೇ ಮನುಷ್ಯನನ್ನು ದೇವರಿಗಾಗಿ ಅಪಾಯವನ್ನು ಎದುರಿಸು ಎಂದರೆ ಚೌಕಾಶಿಯ ಬುದ್ಧಿ ಬರುವುದು. ಪ್ರಪಂಚದ ಕಡೆ ನುಗ್ಗು ಎಂದರೆ ದೌಡಾಯಿಸಿ ಓಡುವನು. ಇಲ್ಲಿ ಅವನಿಗೆ ಯಾವ ಅನುಮಾನವೂ ಇಲ್ಲ, ಯಾವ ಅಪಾಯವೂ ಇಲ್ಲ ಎಂದು ಭಾವಿಸುತ್ತಾನೆ.

ಆಧ್ಯಾತ್ಮಿಕ ಜೀವಿಯು ಸಾಹಸಪರನೆ. ಅವನ ಸಾಹಸವೆಲ್ಲ ದೇವರ ಕಡೆ ಹೋಗುವುದರಲ್ಲಿದೆ. ಅದಕ್ಕಾಗಿ ಏನು ಅಪಾಯವನ್ನು ಬೇಕಾದರೂ ಎದುರಿಸುವನು. ಸಾಯುವ ಹಾಗಿದ್ದರೆ ಒಂದು ಮಹಾ ಆದರ್ಶಕ್ಕಾಗಿ ಹೋರಾಡುವಾಗ ಸಾಯುವುದು ಮೇಲು ಎನ್ನುತ್ತಾನೆ. ಯೋಗಭ್ರಷ್ಟನಾಗುವುದು ಎಂದರೆ ಮಹಾ ಆದರ್ಶಕ್ಕೆ ಹೋರಾಡುವಾಗ ಸೋಲುವುದು. ಆದರೆ ಈ ಸೋಲು ತಾತ್ಕಾಲಿಕ. ಸ್ವಲ್ಪ ಕಾಲದ ಮೇಲೆ ಎದ್ದು ಪುನಃ ಗುರಿಯ ಕಡೆ ಧಾವಿಸುವನು.

\begin{shloka}
ಅಥವಾ ಯೋಗಿನಾಮೇವ ಕುಲೇ ಭವತಿ ಧೀಮತಾಮ್~।\\ಏತದ್ಧಿ ದುರ್ಲಭತರಂ ಲೋಕೇ ಜನ್ಮ ಯದೀದೃಶಮ್ \hfill॥ ೪೨~॥
\end{shloka}

\begin{artha}
ಅಥವಾ ಬುದ್ಧಿವಂತರಾದ ಯೋಗಿಗಳ ಕುಲದಲ್ಲಿ ಇವನು ಜನಿಸುವನು. ಇಂತಹ ಜನ್ಮ ಲೋಕದಲ್ಲಿ ಬಹಳ ದುರ್ಲಭವಾದದ್ದು.
\end{artha}

ಒಬ್ಬನಲ್ಲಿ ಸಾಂಸಾರಿಕತೆ ಬಹಳ ಕ್ಷೀಣವಾಗಿದ್ದರೆ, ಆ ಚಪಲವನ್ನು ತೃಪ್ತಿಪಡಿಸಿಕೊಳ್ಳಲು ಬೇರೆ ಲೋಕಗಳಿಗೆ ಅವನು ಹೋಗಬೇಕಾಗಿಲ್ಲ. ಇರುವ ಆಸೆ ಕ್ರಮೇಣ ಒಣಗಿ ಹೋಗುವುದು. ಬೆಂಕಿಗೆ ಹೊಸದಾಗಿ ಸೌದೆಯನ್ನು ಹಾಕದಿದ್ದರೆ, ಅದು ಉರಿದು ಬೂದಿಯಾಗಿ ಹೋಗುವುದು. ಅದರಿಂದ ಯಾವ ಅಪಾಯವೂ ಇಲ್ಲ. ಇಂತಹ ಮನುಷ್ಯ ಬುದ್ಧಿವಂತರಾದ ಯೋಗಿಗಳ ಕುಲದಲ್ಲಿ ಹುಟ್ಟುತ್ತಾನೆ. ಆ ಕುಲದಲ್ಲಿ ಎಲ್ಲಾ ಬಲ್ಲವರು ಇರುವರು. ಅವರು ಲೌಕಿಕ ದೃಷ್ಟಿಯಲ್ಲಿ ಜಾಣರು ಮತ್ತು ಬುದ್ಧಿವಂತರು ಎಂದು ತೆಗೆದುಕೊಳ್ಳುವುದಕ್ಕಿಂತ ಹೆಚ್ಚಾಗಿ, ಈ ಪ್ರಪಂಚವೇನು, ಇಲ್ಲಿ ಜನ್ಮವೆತ್ತಿದ ಮೇಲೆ ನಾವು ಮಾಡಬೇಕಾದುದೇನು ಎಂಬುದನ್ನೆಲ್ಲ ಮುಂಚೆಯೇ ತಿಳಿದವರು. ಅವರು ಬರೀ ತಿಳಿದುಕೊಂಡ ಪಂಡಿತರು ಮಾತ್ರ ಅಲ್ಲ, ಏನನ್ನು ತಿಳಿದುಕೊಂಡಿರುವರೊ ಅದನ್ನು ಅನುಷ್ಠಾನಕ್ಕೆ ತರಲು ಯತ್ನಿಸುತ್ತಿರುವ ಯೋಗಿಗಳು. ಇಂತಹ ಕಡೆ ಜನ್ಮವೆತ್ತುವುದು ಒಂದು ಮಹಾಪುಣ್ಯ ಕರ್ಮದ ಫಲ. ಶ‍್ರೀಮಂತರ ಮನೆಯಲ್ಲಿ ಹುಟ್ಟಬಹುದು. ಅಲ್ಲಿ ಭೋಗಿಸುವುದಕ್ಕೆ ಬೇಕಾದಷ್ಟು ಇದೆ. ಇದರಿಂದ ನಮ್ಮ ದುರ್ವ್ಯಸನಗಳನ್ನೆಲ್ಲ ಇನ್ನೂ ವೃದ್ಧಿಮಾಡಿಕೊಳ್ಳಬಹುದು. ಸುಮ್ಮನೆ ಶ‍್ರೀಮಂತರ ಮನೆಯಲ್ಲಿ ಹುಟ್ಟುವುದೇ ಪುಣ್ಯವಲ್ಲ. ಶ‍್ರೀಮಂತಿಕೆಯನ್ನು ನಾವು ಏತಕ್ಕೆ ಉಪ ಯೋಗಿಸಿಕೊಳ್ಳುತ್ತೇವೆಯೊ ಅದರ ಮೇಲೆ ನಿಂತಿದೆ, ಅದು ಪುಣ್ಯವೊ, ಪಾಪವೊ, ಎಂಬುದು. ಅದನ್ನು ಒಳ್ಳೆಯದಕ್ಕೆ ಉಪಯೋಗಿಸಿಕೊಂಡರೆ ಅದು ಪುಣ್ಯ. ದುರ್ವ್ಯಸನದ ಕೆಸರಿನಲ್ಲಿ ಮುಳುಗಿಸುವುದಕ್ಕೆ ಉಪಯೋಗಿಸಿಕೊಂಡರೆ ಅದೊಂದು ಪಾಪವಾಗುವುದು. ಒಳ್ಳೆಯ ಶ‍್ರೀಮಂತರ, ಜೊತೆಗೆ ಬುದ್ಧಿ ವಂತರ, ಆಧ್ಯಾತ್ಮಿಕ ಜೀವನದಲ್ಲಿ ಭಗವಂತನ ಕಡೆಗೆ ಹೋಗುವುದಕ್ಕೆ ಸತತ ಯತ್ನಿಸುತ್ತಿರುವವರ ಮನೆಯಲ್ಲಿ ಜನ್ಮವೆತ್ತುವುದು ಸ್ವರ್ಗಲೋಕದಲ್ಲಿ ಹುಟ್ಟುವುದಕ್ಕಿಂತ ಮಿಗಿಲು. ಈ ಮನೆಯಲ್ಲಿ ಆಧ್ಯಾತ್ಮಿಕ ವಾತಾವರಣ ಸ್ಪಂದಿಸುತ್ತಿದೆ. ಆಧ್ಯಾತ್ಮಿಕ ಜೀವನದಲ್ಲಿ ಅಭೀಪ್ಸೆ ಇರುವವನಿಗೆ ಇಂತಹ ವಾತಾವರಣ ಸಿಕ್ಕಿದರೆ ಬಹಳ ಬೇಗ ಅವನ ಜೀವನ ವಿಕಾಸವಾಗುವುದು. ಇದು ಭಗವಂತ ನೀಡಿದ ದೊಡ್ಡ ವರ ಅವನಿಗೆ. 

\begin{shloka}
ತತ್ರ ತಂ ಬುದ್ಧಿಸಂಯೋಗಂ ಲಭತೇ ಪೌರ್ವದೇಹಿಕಮ್~।\\ಯತತೇ ಚ ತತೋ ಭೂಯಃ ಸಂಸಿದ್ಧೌ ಕುರುನಂದನ \hfill॥ ೪೩~॥
\end{shloka}

\begin{artha}
ಅರ್ಜುನ, ಅಲ್ಲಿ ಅವನು ಹಿಂದಿನ ಜನ್ಮದಲ್ಲಿ ಸಂಪಾದಿಸಿದ ಜ್ಞಾನವನ್ನು ಪಡೆಯುತ್ತಾನೆ. ಅನಂತರ ಪೂರ್ಣತೆಯನ್ನು ಪಡೆಯುವುದಕ್ಕೆ ಹಿಂದಿಗಿಂತ ಹೆಚ್ಚಾಗಿ ಪ್ರಯತ್ನ ಪಡುತ್ತಾನೆ.
\end{artha}

ಭ್ರಷ್ಟನಾದ ಯೋಗಿ ಯಾವಾಗ ತನ್ನ ಚಪಲವನ್ನು ಪೂರ್ಣಮಾಡಿಕೊಂಡು ಒಂದು ಉತ್ತಮ ಯೋಗಿಗಳ ಕುಲದಲ್ಲಿ ಜನ್ಮವೆತ್ತುತ್ತಾನೊ ಅಲ್ಲಿ ಅವನಿಗೆ ಮೊದಲಿನಿಂದಲೂ ಆಧ್ಯಾತ್ಮಿಕ ಪ್ರವೃತ್ತಿ ಇರುವುದು. ಹಿಂದಿನ ಜನ್ಮದಲ್ಲಿ ಪ್ರಲೋಭನೆಯ ಬಲೆಗೆ ಬೀಳುವುದಕ್ಕೆ ಮುಂಚೆ ಯಾವ ದೃಷ್ಟಿ ಯಿತ್ತೊ ಅದು ಅವನಿಗೆ ಬರುವುದು. ಜೊತೆಗೆ ಯಾವ ಆಸೆ ಅವನಲ್ಲಿತ್ತು, ಯಾವುದಕ್ಕಾಗಿ ಅವನು ಭ್ರಷ್ಟನಾದ ಎಂಬ ಜ್ಞಾನವೂ ಸೂಕ್ಷ್ಮವಾಗಿ ಅವನಲ್ಲಿರುವುದು. ಯಾವಾಗ ಈ ಹಿನ್ನೆಲೆ ಇರುವುದೊ, ಆಗ ಪುನಃ ಹಿಂದೆ ಮಾಡಿದ ತಪ್ಪನ್ನು ಮಾಡುವುದಕ್ಕೆ ಹೋಗುವುದಿಲ್ಲ. ಅದರಿಂದ ಬುದ್ಧಿ ಕಲಿತಿರುವನು. ತನ್ನನ್ನು ತಡೆದುದು ಯಾವುದು, ಅದರಿಂದ ಪಾರಾಗಬೇಕಾದರೆ ಏನು ಮಾಡಬೇಕು, ಎಂಬುದನ್ನು ಅವನು ಚೆನ್ನಾಗಿ ಅರಿತಿರುವನು. ಗುರಿಯನ್ನು ಸಾಧಿಸುವುದಕ್ಕೆ ಎಂದಿಗಿಂತ ಹೆಚ್ಚಾಗಿ ಪ್ರಯತ್ನಪಡುವನು. ಒಂದು ಸಲ ಪರೀಕ್ಷೆಯಲ್ಲಿ ಕುಳಿತು ಫೇಲಾದವನು, ಅವನಿಗೆ ನಂಬರು ಗೊತ್ತಾಗಿ ಯಾವ ಭಾಗದಲ್ಲಿ ಫೇಲಾಗಿರುವನೊ ಅದನ್ನು ಚೆನ್ನಾಗಿ ಮನಸ್ಸಿಟ್ಟು ಓದಿ ಪುನಃ ಪರೀಕ್ಷೆಯಲ್ಲಿ ಕುಳಿತುಕೊಂಡು ತೇರ್ಗಡೆಯಾಗಿ ಹೋಗುವಂತೆ ಇದು.

\begin{shloka}
ಪೂರ್ವಾಭ್ಯಾಸೇನ ತೇನೈವ ಹ್ರಿಯತೇ ಹ್ಯವಶೋಽಪಿ ಸಃ~।\\ಜಿಜ್ಞಾಸುರಪಿ ಯೋಗಸ್ಯ ಶಬ್ದಬ್ರಹ್ಮಾತಿವರ್ತತೇ \hfill॥ ೪೪~॥
\end{shloka}

\begin{artha}
ಆ ಪೂರ್ವ ಅಭ್ಯಾಸದಿಂದಲೇ ಅವನು ಅವಶ್ಯಕವಾಗಿ ಯೋಗದ ಕಡೆ ಸೆಳೆಯಲ್ಪಡುತ್ತಾನೆ. ಸಕಾಮನಾಗಿ ವೈದಿಕ ಕರ್ಮಗಳನ್ನು ಮಾಡುವವನ ಸ್ಥಿತಿಯನ್ನು ಯೋಗ ಜಿಜ್ಞಾಸು ದಾಟಿ ಹೋಗುತ್ತಾನೆ.
\end{artha}

ಹಲವು ಜನ್ಮಗಳಿಂದ ಸಂಗ್ರಹಿಸಿಕೊಂಡಿರುವ ಉತ್ತಮವಾದ ಪೂರ್ವ ಅಭ್ಯಾಸಗಳ ಪರಿಣಾಮದ ಫಲವಾಗಿ ಯೋಗಿ ಯಾವಾಗಲೂ ಬ್ರಹ್ಮನ ಕಡೆಯೇ ವಾಲುತ್ತಾನೆ. ಸುತ್ತಮುತ್ತಲು ಅದಕ್ಕೆ ವಿರೋಧವಾದ ಯಾವುದಿದ್ದರೂ ಅದರ ಕಡೆ ಅವನು ಹೋಗುವುದಿಲ್ಲ. ಆದಕಾರಣವೆ ಅನೇಕ ವೇಳೆ ಒಂದು ಹೀನ ಸಂಸ್ಕಾರಗಳಿಂದ ತುಂಬಿದ ಮನೆಯೊಂದರಲ್ಲಿ ಅಪೂರ್ವ ಯೋಗಿ ಜನಿಸಿ, ಸುತ್ತಲಿರುವ ವಾತಾವರಣದ ಬಲೆಗೆ ಬೀಳದೆ ತನ್ನದೇ ಆದ ದಾರಿಯನ್ನು ಮಾಡಿಕೊಂಡು ಹೋಗುವುದನ್ನು ನೋಡುತ್ತೇವೆ. ಕಾಗೆಗಳ ಸಮೂಹದಲ್ಲಿದ್ದ ಕೋಗಿಲೆಯಂತೆ ಇವನು. ಇವನು ಪತಿತನಾಗ ಬೇಕೆಂದು ಬಯಸಿದರೂ ಇವನಲ್ಲಿರುವ ಪೂರ್ವ ಅಭ್ಯಾಸದ ಫಲ ಬೀಳುಕೊಡುವುದಿಲ್ಲ. ಹೇಗೊ ಪರೀಕ್ಷಾ ಸಮಯದಲ್ಲಿ ಮೇಲೆತ್ತುವುದು.

ಅವನು ಯೋಗ ಜಿಜ್ಞಾಸು, ಎಂದರೆ ಯೋಗಜೀವನಕ್ಕೆ ಸಂಬಂಧಪಟ್ಟ ವಿಷಯಗಳನ್ನು ತಿಳಿಯಬೇಕೆಂದು ಆಸೆ ಇದೆ. ಆದರೂ ಅವನು ವೈದಿಕ ಕರ್ಮಗಳನ್ನು ಮೀರಿ ಹೋಗುತ್ತಾನೆ. ಅವನಿಗೆ ಬರೀ ಬಾಹ್ಯಾಚಾರ ರುಚಿಸುವುದಿಲ್ಲ. ಅವನು ಅವುಗಳನ್ನೆಲ್ಲ ಅತಿಕ್ರಮಿಸಿ ಅನುಭವದ ಕಡೆ ಹೋಗುತ್ತಾನೆ. ತತ್ತ್ವ ಎಲ್ಲಿ ಕೊನೆಗೊಳ್ಳುವುದೊ, ಧರ್ಮ ಅಲ್ಲಿಂದ ಪ್ರಾರಂಭವಾಗುವುದು. ಇವನೇ ಅದಕ್ಕೆ ಒಂದು ಉದಾಹರಣೆ.

ಇವನು ಶಬ್ದಬ್ರಹ್ಮನನ್ನು ಮೀರಿಹೋಗುತ್ತಾನೆ. ಮುಂಚೆ ಏನನ್ನು ಪಡೆದುಕೊಳ್ಳಬೇಕೊ ಅದನ್ನು ತಿಳಿದುಕೊಳ್ಳಬೇಕು. ಅದು ಶಬ್ದ. ತಿಳಿದುಕೊಂಡಿರುವುದನ್ನು ಸಂಪಾದಿಸಬೇಕು. ಶ‍್ರೀರಾಮಕೃಷ್ಣರು ಇದನ್ನು ವಿವರಿಸುವುದಕ್ಕೆ ಒಂದು ಉದಾಹರಣೆಯನ್ನು ಕೊಡುವರು. ಒಬ್ಬ ಹಳ್ಳಿಯಿಂದ ಊರಿನಲ್ಲಿ ದ್ದವನಿಗೆ ರಜಾ ಸಮಯದಲ್ಲಿ ಬರುವಾಗ ಬಟ್ಟೆ ಮತ್ತು ಕೆಲವು ಸಿಹಿತಿಂಡಿಯನ್ನು ತೆಗೆದುಕೊಂಡು ಬಾ ಎಂದು ಬರೆದಿದ್ದ. ಊರಿನಲ್ಲಿದ್ದವನು ಕಾಗದವನ್ನು ಓದಿ, ಅದನ್ನು ಬರೆದಿರುವುದನ್ನು ತರುವುದಕ್ಕೆ ಹೋದ. ಕಾಗದದಲ್ಲಿ ಬರೆದಿರುವುದನ್ನು ಓದುವುದು, ತಿಳಿದು\-ಕೊಳ್ಳುವುದು ಶಬ್ದಬ್ರಹ್ಮ. ಅದನ್ನು ತರುವುದಕ್ಕೆ ಹೋಗುವುದೇ, ಶಬ್ದಬ್ರಹ್ಮವನ್ನು ಅತಿಕ್ರಮಿಸುವುದು.

\begin{shloka}
ಪ್ರಯತ್ನಾದ್ಯತಮಾನಸ್ತು ಯೋಗೀ ಸಂಶುದ್ಧಕಿಲ್ಬಿಷಃ~।\\ಅನೇಕಜನ್ಮಸಂಸಿದ್ಧಸ್ತತೋ ಯಾತಿ ಪರಾಂ ಗತಿಮ್ \hfill॥ ೪೫~॥ 
\end{shloka}

\begin{artha}
ಪ್ರಯತ್ನದಿಂದ ಅಭ್ಯಾಸ ಮಾಡುತ್ತಿರುವವನೂ, ಕಿಲ್ಬಷರಹಿತನೂ, ಅನೇಕ ಜನ್ಮಗಳಿಂದ ಸಿದ್ಧಿಯನ್ನು ಹೊಂದಿದ ವನೂ ಆದ ಯೋಗಿ ಅನಂತರ ಪರಮಗತಿಯನ್ನು ಹೊಂದುವನು.
\end{artha}

\newpage

ಗುರಿಯನ್ನು ಸೇರುವವರೆಗೂ ಬಿಡದೆ ಅಭ್ಯಾಸ ಮಾಡುತ್ತಿರುವವನು ಯೋಗಿ. ಏನೊ ಕೆಲವು ಕಾಲ ಅಭ್ಯಾಸಮಾಡಿ ಆದಮೇಲೆ ಇನ್ನು ಸರಾಗವಾಗಿ ಗುರಿಸೇರಬಹುದು ಎಂದಲ್ಲ ಈ ಜೀವನದಲ್ಲಿ. ಇಲ್ಲಿ ದಿನನಿತ್ಯವೂ ಹೋರಾಟವಿರಬೇಕು. ಈ ಹೋರಾಟಕ್ಕೆ ರಜ ಇಲ್ಲ. ಯಾವಾಗ ಸೋಮಾರಿಯಾಗುವನೊ ಆಗ ಹಿಂದೆ ಸರಿಯುವನು. ಇಲ್ಲಿ ಮುಂದೆ ಹೋಗಬೇಕು. ಇಲ್ಲದೇ ಇದ್ದರೆ ಹಿಂದೆ ಬೀಳಬೇಕು. ನದಿ ಮೂರು ಮೈಲಿ ವೇಗದಲ್ಲಿ ಹೋಗುತ್ತಿದ್ದರೆ, ಅದಕ್ಕೆ ವಿರೋಧವಾಗಿ ಈಜುತ್ತಿರುವವನು ಇರುವ ಕಡೆ ಇರಬೇಕಾದರೆ ಮೂರು ಮೈಲಿ ವೇಗದಲ್ಲಿ ಈಜುತ್ತಿದ್ದರೆ ಮಾತ್ರ ಸಾಧ್ಯ. ಯೋಗಿ ಈ ಹೋರಾಟದಲ್ಲೆ ಒಂದು ಆನಂದವನ್ನು ಹುಟ್ಟಿಸಿಕೊಳ್ಳುವನು. ಆಡುವವನಿಗೆ ಸೋಲು ಗೆಲವುಗಳಿಗಿಂತ ಆಟದಲ್ಲಿಯೇ ಹೇಗೆ ಒಂದು ಆನಂದವಿದೆಯೋ ಹಾಗೆ. ನಿರಂತರ ಅಭ್ಯಾಸ ಮಾಡುತ್ತಿದ್ದರೆ ಮಾತ್ರ ಒಂದಲ್ಲ ಒಂದು ಸಲ ಅವನು ಗುರಿ ಸೇರುವನು.

ಆಧ್ಯಾತ್ಮಿಕ ಜೀವನದಲ್ಲಿ ಹೋರಾಟ ನಡೆಸುತ್ತಿರುವಾಗ ಉತ್ತಮ ಸಂಸ್ಕಾರಗಳು ಉತ್ಪತ್ತಿ\-ಯಾಗುತ್ತ ಬರುತ್ತವೆ. ಅವು ನಮ್ಮ ಮನಸ್ಸಿನ ಹೀನ ವಾಸನೆಗಳನ್ನು ಅಳಿಸುತ್ತಾ ಬರುವುವು. ಕೊನೆಗೆ ಹೀನಸಂಸ್ಕಾರಗಳೆಲ್ಲ ನಾಶವಾಗಿ, ಆ ಸ್ಥಳದಲ್ಲಿ ಉತ್ತಮ ಸಂಸ್ಕಾರಗಳು ಇರುತ್ತವೆ. ಆಗಲೆ ಅವನ ಚಿತ್ತ ಶುದ್ಧವಾಗುವುದು. 

ಒಂದೇ ಒಂದು ಜನ್ಮದ ಹೋರಾಟದಲ್ಲಿಯೇ ಗುರಿಯನ್ನು ಮುಟ್ಟುವವರೆಗೆ ಹೋರಾಡುವವರು ಅಪರೂಪ. ಪ್ರತಿ ಜನ್ಮವೂ ಹೋರಾಡುತ್ತಿದ್ದರೆ ನಮ್ಮ ಉತ್ತಮ ಸಂಸ್ಕಾರಗಳು ವೃದ್ಧಿಯಾಗಿ ಹಲವು ಜನ್ಮಗಳ ಸಾಧನಾಫಲದಿಂದ ಪರಿಪಾಕ ಹೊಂದಿ ಮುಕ್ತಿಯನ್ನು ಗಳಿಸುವೆವು. ಒಂದೇ ಜನ್ಮ ಸಾಲದು ಯೋಗಿ ಸಿದ್ಧನಾಗಬೇಕಾದರೆ. ಅದಕ್ಕೆ ಹಲವು ಜನ್ಮಗಳ ಹೋರಾಟ ಬೇಕಾಗಿದೆ. ಒಂದೇ ಜನ್ಮದ ಭೂಮಿಕೆಯ ಮೇಲೆ ನಿಂತು ನೋಡುವವನಿಗೆ ಎಲ್ಲವನ್ನೂ ವಿವರಿಸುವುದಕ್ಕಾಗುವುದಿಲ್ಲ. ಶ‍್ರೀರಾಮ ಕೃಷ್ಣರು ಇದನ್ನು ವಿವರಿಸಲು ಒಂದು ಸುಂದರವಾದ ನಿದರ್ಶನವನ್ನು ಕೊಡುವರು. ಒಬ್ಬ ಶಾಕ್ತ ಸಾಧಕ ಒಂದು ಹೆಣದ ಮೇಲೆ ಕುಳಿತುಕೊಂಡು ಸಾಧನೆ ಮಾಡುತ್ತಿದ್ದ. ಅದೇ ಹಾದಿಯಲ್ಲಿ ಒಬ್ಬ ಪ್ರಯಾಣಿಕ ಬಂದ. ಇವನು ನೋಡುತ್ತಿದ್ದಂತೆ ಒಂದು ಹುಲಿ ಬಂದು ಸಾಧನೆ ಮಾಡುತ್ತಿದ್ದವನ ಮೇಲೆ ಬಿದ್ದು ಅವನನ್ನು ಕೊಂದು ಹಾಕಿತು. ಹತ್ತಿರ ನಿಂತು ನೋಡುತ್ತಿದ್ದವನು ಅಪಾಯದಿಂದ ಪಾರಾಗುವುದಕ್ಕೆ ಮರವನ್ನು ಹತ್ತಿ ಕುಳಿತ. ಹುಲಿ ಹೆಣದಮೇಲೆ ಕುಳಿತು ಸಾಧನೆಮಾಡುತ್ತಿದ್ದ ಸಾಧಕನನ್ನು ಕೊಂದು ಹೊರಟುಹೋಯಿತು. ಮರದ ಮೇಲೆ ಕುಳಿತಿದ್ದವನು ಕೆಳಗಿಳಿದು ಬಂದು ನೋಡುತ್ತಾನೆ, ದೇವಿಯನ್ನು ಉಪಾಸನೆ ಮಾಡುವುದಕ್ಕೆ ಎಲ್ಲಾ ಸಿದ್ಧವಾಗಿತ್ತು. ತಾನೂ ಕೂಡ ಸ್ವಲ್ಪ ಸಾಧನೆ ಮಾಡೋಣ ಎಂದು ಕುಳಿತು ಪೂಜಾದಿಗಳನ್ನು ಮಾಡಿ ಧ್ಯಾನ ಮಾಡುತ್ತಿರುವಾಗ ತಕ್ಷಣವೇ ಅವನಿಗೆ ದೇವಿ ಪ್ರತ್ಯಕ್ಷಳಾದಳು. ಇದನ್ನು ಕಂಡ ಭಕ್ತ ಆಶ್ಚರ್ಯಚಕಿತನಾಗಿ ದೇವಿಯನ್ನು ಈ ಪ್ರಶ್ನೆ ಕೇಳುತ್ತಾನೆ: “ಯಾರು ನಿನ್ನನ್ನು ಪೂಜೆಮಾಡಬೇಕೆಂದು ಬಂದನೊ ಅವನನ್ನು ಹುಲಿ ಕೊಂದುಹಾಕಿತು. ನಾನು ಸುಮ್ಮನೆ ತಮಾಷೆ ನೋಡುತ್ತಿದ್ದವನು, ಸ್ವಲ್ಪ ಸಾಧನೆ ಮಾಡಿದೊಡನೆಯೇ ನೀನು ನನಗೆ ಪ್ರತ್ಯಕ್ಷಳಾದೆಯಲ್ಲ ಇದರ ರಹಸ್ಯವೇನು?” ಅದಕ್ಕೆ ದೇವಿ, “ನೀನು ಈಗ ಮಾಡಿದ ಸ್ವಲ್ಪ ಸಾಧನೆಯಿಂದಲೇ ಸಿದ್ಧಿ ಸಿಕ್ಕಲಿಲ್ಲ. ನೀನು ಹಿಂದಿನ ಜನ್ಮದಲ್ಲಿ ಬೇಕಾದಷ್ಟು ಸಾಧನೆ ಮಾಡಿದ್ದೆ. ಇನ್ನು ಸ್ವಲ್ಪ ಮಾತ್ರ ಉಳಿದಿತ್ತು. ಅದನ್ನು ಈಗ ಮಾಡಿದೆ. ಅದರಿಂದ ನಿನ್ನ ಸಾಧನೆ ಪೂರ್ಣವಾಯಿತು” ಎಂದು ಹೇಳಿದಳು. ಹೀಗೆಯೇ ಯೋಗಿ ಯಾವುದೊ ಜನ್ಮದಲ್ಲಿ ಸ್ವಲ್ಪ ಸಾಧನೆಮಾಡಿ ಸಿದ್ಧನಾದಂತೆ ಕಂಡರೂ ಅವನು ಹಿಂದಿನ ಜನ್ಮಗಳಲ್ಲಿ ಬೇಕಾದಷ್ಟು ಮಾಡಿರುವನು. ಇನ್ನು ಎಲ್ಲೊ ಸ್ವಲ್ಪವನ್ನು ಮಾತ್ರ ಅವನು ಈಗ ಮಾಡುತ್ತಾನೆ.

ಯೋಗಿ ಪುರ್ಣಾತ್ಮನಾಗಿ ಪರಮಗತಿಯನ್ನು ಪಡೆಯುತ್ತಾನೆ. ಎಲ್ಲಿಗೆ ಹೋದರೆ ಹಿಂದಿರುಗಿ ಬರುವುದಿಲ್ಲವೊ ಆ ಗತಿಯನ್ನು ಪಡೆಯುತ್ತಾನೆ. ಯಾವುದನ್ನು ತಿಳಿದುಕೊಂಡರೆ, ಈ ಪ್ರಪಂಚದಲ್ಲಿ ಮತ್ತಾವುದನ್ನೂ ತಿಳಿದುಕೊಳ್ಳಬೇಕಾಗಿಲ್ಲವೊ ಅದನ್ನು ತಿಳಿದುಕೊಳ್ಳುತ್ತಾನೆ.

\begin{shloka}
ತಪಸ್ವಿಭ್ಯೋಽಧಿಕೋ ಯೋಗೀ ಜ್ಞಾನಿಭ್ಯೋಽಪಿ ಮತೋಽಧಿಕಃ~।\\ಕರ್ಮಿಭ್ಯಶ್ಚಾಧಿಕೋ ಯೋಗೀ ತಸ್ಮಾದ್ಯೋಗೀ ಭವಾರ್ಜುನ \hfill॥ ೪೬~॥
\end{shloka}

\begin{artha}
ತಪಸ್ವಿಗಳು, ಜ್ಞಾನಿಗಳು ಮತ್ತು ಕರ್ಮಿಗಳಿಗಿಂತಲೂ ಯೋಗಿ ಶ್ರೇಷ್ಠನೆಂಬುದು ನನ್ನ ಅಭಿಪ್ರಾಯ. ಆದಕಾರಣ ಅರ್ಜುನ ನೀನು ಯೋಗಿಯಾಗು.
\end{artha}

ಬರೀ ತಪಸ್ಸು ಮಾಡುವವರನ್ನೆಲ್ಲ ಶ್ರೇಷ್ಠರು ಎಂದು ಹೇಳಲು ಆಗುವುದಿಲ್ಲ. ರಾವಣ ಹಿರಣ್ಯಾಕ್ಷ ಮುಂತಾದ ಅಸುರರು ಉಗ್ರವಾದ ತಪಸ್ಸನ್ನು ಮಾಡಿದರು. ದೇವರು ಪ್ರತ್ಯಕ್ಷನಾದಾಗ ಅವನನ್ನು ಬೇಡಿದ್ದು ಏನನ್ನು? ನನಗೆ ಅದರಿಂದ ಸಾವು ಬೇಡ, ಇದರಿಂದ ಸಾವು ಬೇಡ ಎಂದು ವರವನ್ನು ಪಡೆದು ಲೋಕಕಂಟಕರಾದರು. ದೇಹವನ್ನು ದಂಡಿಸಿ ಮಾಡುವ ತಪಸ್ಸೆ ಮುಖ್ಯವಲ್ಲ. ಏತಕ್ಕಾಗಿ ಒಬ್ಬ ತಪಸ್ಸು ಮಾಡುತ್ತಾನೆ ಎಂಬುದು ಮುಖ್ಯ. ಇಂತಹ ತಪಸ್ವಿಗಳಿಗಿಂತ ಧ್ಯಾನಯೋಗಿ ಮೇಲು.

ಯೋಗಿ ಜ್ಞಾನಿಗಳಿಗಿಂತ ಮೇಲು ಎನ್ನುವನು ಶ‍್ರೀಕೃಷ್ಣ. ಗೀತೆಯಲ್ಲಿ ಶ‍್ರೀಕೃಷ್ಣ ಜ್ಞಾನಕ್ಕೂ ವಿಜ್ಞಾನಕ್ಕೂ ದೊಡ್ಡದೊಂದು ವ್ಯತ್ಯಾಸವನ್ನು ತರುವನು. ಜ್ಞಾನ ಎಂದರೆ ಪಾಂಡಿತ್ಯ. ಅವನು ಹಲವು ಶಾಸ್ತ್ರಾದಿಗಳನ್ನು ಓದಿ ವಿಷಯ ಸಂಗ್ರಹ ಮಾಡಿದ್ದಾನೆ. ಯಾವುದು ಸರಿ, ಯಾವುದು ತಪ್ಪು ಎಂಬುದನ್ನು ಅರಿತಿರುವನು. ಆದರೆ ಅದಿನ್ನೂ ಅನುಷ್ಠಾನದೊಳಗೆ ಇಳಿದಿಲ್ಲ. ಜೀವನದ ಪರೀಕ್ಷಾ ಸಮಯದಲ್ಲಿ ನೆರವಾಗತಕ್ಕ ವಿದ್ಯೆಯಲ್ಲ ಅದು. ಎಲ್ಲಾ ಚೆನ್ನಾಗಿರುವಾಗ ಈ ಮನುಷ್ಯ ವೇದಾಂತಿ, ಜ್ಞಾನಿ. ಬೇಕಾದರೆ ಇನ್ನೊಬ್ಬನಿಗೆ ಬುದ್ಧಿ ಹೇಳುತ್ತಾನೆ. ಆದರೆ ಪ್ರಲೋಭನೆ ಇವನನ್ನೇ ಆವರಿಸಿದರೆ ಇವನು ಅದಕ್ಕೆ ತುತ್ತಾಗುತ್ತಾನೆ. ಏನನ್ನು ಹೇಳುವನೊ ಅದನ್ನು ಅನುಭವಿಸಿಲ್ಲ. ಬರೀ ಜ್ಞಾನ. ಶ‍್ರೀರಾಮಕೃಷ್ಣರು ಒಂದು ನಿದರ್ಶನ ಕೊಡುವರು. ಒಂದು ಮನೆಯಲ್ಲಿ ಸಾಕುತ್ತಿದ್ದ ಅರಗಿಳಿಗೆ ರಾಧಕೃಷ್ಣ ಎನ್ನುವದನ್ನು ಕಲಿಸಿದ್ದರು. ಅದು ಯಾರು ಪಂಜರದ ಹತ್ತಿರ ಬಂದು ನಿಂತುಕೊಂಡರೂ ರಾಧಕೃಷ್ಣ ಎನ್ನುತ್ತಿತ್ತು. ಆದರೆ ಬೆಕ್ಕು ಬಂದು ಅದನ್ನು ಹಿಡಿದಾಗ ರಾಧಕೃಷ್ಣ ಮರೆತುಹೋಗಿ ಅದರ ಕೀರಲು ಧ್ವನಿಯೊಂದೇ ಜ್ಞಾಪಕಕ್ಕೆ ಬರುವುದು. ಈ ರಾಧಾಕೃಷ್ಣ ಎಂಬ ಹೆಸರನ್ನು\break ಕಷ್ಟಕಾಲದಲ್ಲಿ ಮರೆಯುವುದು. ಯಾವಾಗ ಅತ್ಯಂತ ಆವಶ್ಯಕವಾಗಿ ಬೇಕೊ ಆಗಲೆ ಇಲ್ಲ. ವಿಪತ್ತಿನಿಂದ ಪಾರಾದ ಮೇಲೆ ಅದು ಜ್ಞಾಪಕಕ್ಕೆ ಬರುವುದು. ಇಂತಹ ಅನುಭವದ ಆಳಕ್ಕೆ ಇಳಿಯದ ವಿದ್ಯೆಯನ್ನೇ ಜ್ಞಾನವೆನ್ನುವುದು ಶ‍್ರೀಕೃಷ್ಣ. ಈ ಜ್ಞಾನ ಅವನ ಜೀವನದಲ್ಲಿ ವ್ಯಾಪಿಸಿದ್ದರೆ,\break ಅವನಲ್ಲಿ ಓತಪ್ರೋತವಾಗಿದ್ದರೆ, ಅದು ವಿಜ್ಞಾನ ಆಗುವುದು. ಆದರೆ ಅಂತಹ ಜ್ಞಾನಿಗಿಂತ ಯೋಗಿ ಮೇಲು ಎನ್ನುವನು. ಯೋಗಿ ತಾನು ತಿಳಿದಿರುವಷ್ಟನ್ನು ಅನುಷ್ಠಾನಕ್ಕೆ ತರಲು ಯತ್ನಿಸುವನು. ತನ್ನ ಜೀವನದಲ್ಲಿಯೇ ಅದನ್ನು ಪ್ರಯೋಗ ಮಾಡಿ ನೋಡುವವನೇ ಯೋಗಿ.

ಯೋಗಿ ಕರ್ಮಿಗಳಿಗಿಂತಲೂ ಶ್ರೇಷ್ಠ ಎನ್ನುವನು. ಪ್ರಪಂಚದಲ್ಲಿ ಹಲವು ಪ್ರಚಂಡ ಕರ್ಮಿ\-ಗಳಿದ್ದಾರೆ. ಬೇಕಾದಷ್ಟು ಕರ್ಮಗಳನ್ನು ಸುಂಟರಗಾಳಿಯಂತೆ ಮಾಡುತ್ತಿರುವರು. ಅವರಿಲ್ಲದ ಕಮಿಟಿಯೇ ಇಲ್ಲ. ಎಲ್ಲದರಲ್ಲಿಯೂ ಸೇರಿರುವರು. ಒಂದಕ್ಕೆ ಅಧ್ಯಕ್ಷ ಮತ್ತೊಂದಕ್ಕೆ ಕೋಶಾಧಿಕಾರಿ, ಇನ್ನೊಂದಕ್ಕೆ ಸೆಕ್ರೆಟರಿ ಆಗಿರುವನು. ಆದರೆ ಇಷ್ಟೊಂದು ಕರ್ಮಗಳಲ್ಲಿ ನಿರತನಾಗಿದ್ದರೂ, ಅವನು ಈ ಕರ್ಮವನ್ನೆಲ್ಲ ಮತ್ತಾವುದೊ ಒಂದಕ್ಕೆ ಉಪಯೋಗಿಸುತ್ತಾನೆ. ಇದರಿಂದ ಕೀರ್ತಿ ಪಡೆಯುವುದು, ಲಾಭ ಪಡೆಯುವುದು, ಅಧಿಕಾರ ಪಡೆಯುವುದು, ಇವುಗಳಿಗಾಗಿ ಕರ್ಮವನ್ನು ಉಪಯೋಗಿಸುತ್ತಾನೆ. ಬರೀ ಒಬ್ಬ ಮಾಡುವ ಕರ್ಮವನ್ನೇ ನಾವು ತೆಗೆದುಕೊಳ್ಳಬಾರದು. ಒಬ್ಬ ಆ ಕರ್ಮವನ್ನು ಯಾವ ಉದ್ದೇಶದಿಂದ ಮಾಡುತ್ತಾನೆ ಎಂಬುದನ್ನು ಗಮನಿಸ ಬೇಕಾಗಿದೆ. ಆದಕಾರಣವೇ ಶ‍್ರೀಕೃಷ್ಣ ಫಲಾಸಕ್ತಿಯಿಂದ ಮಾಡುವ ಕರ್ಮಿಗಳಿಗಿಂತಲೂ ಯೋಗಿ ಶ್ರೇಷ್ಠ ಎನ್ನುವನು. ಆದಕಾರಣ ನೀನು ಯೋಗಿ ಆಗು ಎನ್ನುತ್ತಾನೆ. ಯೋಗಿ ಉತ್ತರಮುಖಿ ಉತ್ತರದಿಕ್ಕಿನ ಕಡೆ ತಿರುಗುವಂತೆ ದೇವರ ಕಡೆ ತಿರುಗುವನು. ಇವನು ದೇವರಿಂದ ಏನನ್ನೂ ಯಾಚಿಸುವುದಿಲ್ಲ. ದೇವರ ಕಡೆ ಏತಕ್ಕೆ ಹೋಗುತ್ತಾನೆ ಎಂದರೆ, ಇದು ಅವನ ಧರ್ಮವಾಗಿದೆ. ಅವನ ಸ್ವಭಾವ ಆಗಿದೆ. ಅದನ್ನು ತಪ್ಪಿಸುವುದಕ್ಕೆ ಆಗುವುದಿಲ್ಲ. ನದಿ ಏತಕ್ಕೆ ಸಾಗರದ ಕಡೆ ಹರಿದುಕೊಂಡು ಹೋಗಬೇಕು? ಅದಕ್ಕೆ ಸಾಗರ ಕಂಡರೆ ಆಸೆ. ಅದಕ್ಕಾಗಿಯೇ ಹಗಲು ರಾತ್ರಿ ಹರಿಯುತ್ತಿದೆ. ದಾರಿಯಲ್ಲಿ ಬರುವ ಆತಂಕಗಳನ್ನು ಭೇದಿಸಿಕೊಂಡೊ, ಅದು ಸಾಧ್ಯವಿಲ್ಲದೇ ಇದ್ದರೆ, ಅದನ್ನು ಬಳಸಿಕೊಂಡೋ ಅಂತು ಸಾಗರವನ್ನು ಸೇರಿದಾಗಲೇ ಅದಕ್ಕೆ ತೃಪ್ತಿ. ಅದಕ್ಕೆ ಇನ್ನೇನೂ ಬೇಕಾಗಿಲ್ಲ. ಅದರಂತೆಯೇ ಯೋಗಿ. ದೇವರಿಂದ ಅವನು ಏನೂ ಪ್ರತಿಫಲ ಬೇಡುವುದಿಲ್ಲ.

\begin{shloka}
ಯೋಗಿನಾಮಪಿ ಸರ್ವೇಷಾಂ ಮದ್ಗತೇನಾಂತರಾತ್ಮನಾ~।\\ಶ್ರದ್ಧಾವಾನ್ ಭಜತೇ ಯೋ ಮಾಂ ಸ ಮೇ ಯುಕ್ತತಮೋ ಮತಃ \hfill॥ ೪೭~॥
\end{shloka}

\begin{artha}
ನನ್ನಲ್ಲಿ ಮನಸ್ಸನ್ನು ಲೀನಮಾಡಿ, ಶ್ರದ್ಧೆಯಿಂದ ನನ್ನನ್ನು ಭಜನೆ ಮಾಡುವವನೆ ಯೋಗಿಗಳಲ್ಲಿ ಶ್ರೇಷ್ಠನೆಂದು ನಾನು ತಿಳಿಯುತ್ತೇನೆ.
\end{artha}

ಭಗವಂತನಲ್ಲಿ ಮನಸ್ಸನ್ನು ಲೀನಮಾಡುವುದು ಎಂದರೆ ಅವನಲ್ಲಿ ಮನಸ್ಸನ್ನು ಅದ್ದಿರುವುದು, ಅವನಲ್ಲಿ ಮನಸ್ಸನ್ನು ಮುಳುಗಿಸಿರುವುದು. ನಾವೊಂದು ಬಟ್ಟೆಯನ್ನು ನೀರಿನಲ್ಲಿ ಅದ್ದಿದರೆ ಅದು ನೀರನ್ನು ಹೇಗೆ ಹೀರಿಕೊಳ್ಳುವುದೊ ಹಾಗೆ ಭಕ್ತನ ಮನಸ್ಸು ಭಗವಂತನನ್ನು ಹೀರಿಕೊಳ್ಳುವುದು. ಅದನ್ನು ಹಿಂಡಿದರೆ ಮೂಲೆಮೂಲೆಯಲ್ಲಿ ನೀರು ತೊಟ್ಟಿಕ್ಕುವಂತೆ ಯೋಗಿಯ ಮನಸ್ಸಿನಿಂದ ದೇವರೇ ತೊಟ್ಟಿಕ್ಕುವನು. ಚೆಂಬನ್ನು ನೀರಿನಲ್ಲಿ ಅದ್ದಿದರೆ ಅದರ ಒಳಗೆ ಮತ್ತು ಹೊರಗೆಲ್ಲ ನೀರು ಹೇಗೆ ಆವರಿಸಿಕೊಂಡಿರುವುದೊ, ಅದರಂತೆಯೆ ಮನಸ್ಸು ಭಗವಂತನಲ್ಲಿ ಮುಳುಗಿರುವುದು. 

ಅವನು ಶ್ರದ್ಧೆಯಿಂದ ಭಗವಂತನನ್ನು ಭಜಿಸುವನು. ಅವನು ಭಗವಂತನನ್ನು ಚಿಂತಿಸುವುದು ಬರೀ ಯಾಂತ್ರಿಕವಾಗಲ್ಲ, ನೀರಸವಾಗಲ್ಲ. ಭಕ್ತಿಯಿಂದ ಅವನನ್ನು ಚಿಂತಿಸುತ್ತಿರುವನು. ಪ್ರಪಂಚದಲ್ಲಿ ಅವನೇ ಸರ್ವಶ್ರೇಷ್ಠವಾದ ಸತ್ಯ ಎಂಬ ಶ್ರದ್ಧೆ ಇದೆ. ಅವನನ್ನು ಬಿಟ್ಟರೆ ಈ ಬೃಹತ್ ಜಗತ್ತಿಗಾಗಲಿ, ಜೀವರಾಶಿಗಳಿಗಾಗಲಿ ಯಾವ ಬೆಲೆಯೂ ಇಲ್ಲ. ಅದೆಲ್ಲ ಬರೀ ಸೊನ್ನೆ ಆಗುವುದು. ಆ ಸೊನ್ನೆಗೆ ಬೆಲೆ ಬರುವಂತೆ ಮಾಡುವುದು, ಅದರ ಹಿಂದೆ ಇರುವ ಒಂದು. ಅವನೇ ಪವಿತ್ರತಮ ವಸ್ತು, ಕಾರುಣ್ಯ ನಿಧಿ, ಅವನಲ್ಲಿ ಶರಣಾದವರನ್ನು ಕೈಬಿಡದೆ ಸಂರಕ್ಷಿಸುವನು ಎಂಬ ಭಾವನೆಗಳೆಲ್ಲ ಶ್ರದ್ಧೆ ಎಂಬ ಮಾತಿನಲ್ಲಿ ಹುದುಗಿದೆ. ಯಾವ ಯೋಗಿಯ ಹೃದಯದಲ್ಲಿ ಈ ಭಾವನೆಗಳೆಲ್ಲ ತುಂಬಿ ತುಳುಕುತ್ತಿದೆಯೊ, ಅವನು ಯಾವುದೊ ಲೌಕಿಕ ಕಾಮನೆಯನ್ನು ಇಟ್ಟುಕೊಂಡು ತಪಸ್ಸು ಮಾಡುವವ ನಿಗಿಂತ, ಅನುಷ್ಠಾನದಲ್ಲಿ ಬಳಸದೆ ಬರೀ ಸ್ಮೃತಿಯ ಪೆಟ್ಟಿಗೆಯಲ್ಲಿ ಜ್ಞಾನ\-ವನ್ನು ಕೂಡಿಟ್ಟುಕೊಂಡಿರುವವನಿಗಿಂತ, ಪ್ರಪಂಚದ ಆಸೆ ಆಕಾಂಕ್ಷೆಗಳಿಂದ ಪ್ರೇರೇಪಿತವಾದ ಕರ್ಮವನ್ನು ಮಾಡುವವನಿಗಿಂತ ಶ್ರೇಷ್ಠ. ಶ‍್ರೀಕೃಷ್ಣ ಅರ್ಜುನನಿಗೆ ಅಂತಹ ಯೋಗಿಯಾಗು ಎಂದು ಹೇಳುತ್ತಾನೆ. ಯಾವುದು ಶ್ರೇಷ್ಠವೋ ಅದನ್ನು ಪಡೆಯಲಿ ಅರ್ಜುನ ಎಂದು ಆಶಿಸುವನು. 

