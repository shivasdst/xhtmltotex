
\chapter{ಸಾಂಖ್ಯಯೋಗ}

ಸಂಜಯ ಹೀಗೆ ಹೇಳುತ್ತಾನೆ.

\begin{shloka}
ತಂ ತಥಾ ಕೃಪಯಾವಿಷ್ಟಮಶ್ರುಪೂರ್ಣಾಕುಲೇಕ್ಷಣಮ್~।\\ವಿಷೀದಂತಮಿದಂ ವಾಕ್ಯಮುವಾಚ ಮಧುಸೂದನಃ \hfill॥ ೧~॥
\end{shloka}

\begin{artha}
ಈ ಪ್ರಕಾರ ಮರುಕದಿಂದ ಕೂಡಿದ ಕಂಬನಿದುಂಬಿದ ಮುಂದುಗಾಣದ ಕಣ್ಣುಗಳಿಂದ ವ್ಯಥೆ ಪಡುತ್ತಿದ್ದ ಅರ್ಜುನನನ್ನು ಕುರಿತು ಮಧುಸೂದನ ಹೀಗೆ ಹೇಳಿದನು.
\end{artha}

ಭಗವದ್ಗೀತೆಯ ಸಂದೇಶ ನಿಜವಾಗಿ ಇಲ್ಲಿಂದ ಪ್ರಾರಂಭವಾಗುವುದು. ಮೊದಲನೆ ಅಧ್ಯಾಯ ಅದಕ್ಕೆ ಒಂದು ಹಿನ್ನೆಲೆಯನ್ನು ಕಲ್ಪಿಸುವುದು. ಶ‍್ರೀಕೃಷ್ಣ ದೊಡ್ಡ ತತ್ವಜ್ಞಾನಿಯಾಗಿದ್ದ, ಸ್ಥಿತಪ್ರಜ್ಞ\-ನಾಗಿದ್ದ. ಅವನ ಮೇರು ಸದೃಶ ವ್ಯಕ್ತಿತ್ವವನ್ನು ತಿಳಿದುಕೊಳ್ಳಬೇಕಾದರೆ ಅದಕ್ಕೆ ಒಂದು ಸನ್ನಿವೇಶ ಒದಗಬೇಕು. ಆಗಲೆ ಅವನ ಮಹಿಮೆ ಗೊತ್ತಾಗಬೇಕಾದರೆ. ಕಬ್ಬನ್ನು ಗಾಣದಲ್ಲಿ ಹಿಂಡಿದಾಗ ರಸ ಬರುವುದು. ಚಿನ್ನವನ್ನು ಒರೆಗಲ್ಲಿನ ಮೇಲೆ ತೀಡಿದಾಗ ಅದರ ಬಣ್ಣ ಗೊತ್ತಾಗುವುದು. ಅದರಂತೆಯೇ ಒಬ್ಬನ ಯೋಗ್ಯತೆ ಜೀವನದ ಪರೀಕ್ಷಾ ಸಮಯದ ಒರೆಗಲ್ಲಿನ ಮೇಲೆ ತಿಕ್ಕಿದಾಗಲೆ ಗೊತ್ತಾಗಬೇಕಾದರೆ. ಕುರುಕ್ಷೇತ್ರದ ಸಮರಾಂಗಣದಲ್ಲಿ ಶ‍್ರೀಕೃಷ್ಣ ಮತ್ತು ಅರ್ಜುನ ರಥದ ಮೇಲೆ ಕುಳಿತಿರುವರು. ಅರ್ಜುನನ ಬಂಧುಬಾಂಧವರು ಕುರುಕ್ಷೇತ್ರದ ಯುದ್ಧದಲ್ಲಿ ಕೌರವನ ಪಕ್ಷದಲ್ಲಿರುವರು. ಹಾಗೆಯೇ ಶ‍್ರೀಕೃಷ್ಣನ ಅಣ್ಣನಾದ ಬಲರಾಮನ ಮೆಚ್ಚಿನ ಶಿಷ್ಯನೇ ದುರ್ಯೋಧನ. ಪಾಂಡವರ ಪಕ್ಷದಲ್ಲಿಯೂ ತನ್ನ ತಂಗಿಯ ಮಕ್ಕಳುಗಳು ಇವರೆಲ್ಲ ಇರುವರು. ಈ ಯುದ್ಧ ಮಧ್ಯದಲ್ಲಿ ನಿಲ್ಲುವಂತಿಲ್ಲ. ಕೊನೆಯ ತನಕ ಸಾಗುವುದು. ಈ ಸಮರ ದೇವತೆ ಎರಡು ಪಕ್ಷದವರನ್ನೂ ಪಕ್ಷಪಾತವಿಲ್ಲದೆ ಕಬಳಿಸುವಳು ಎಂಬುದನ್ನು ಚೆನ್ನಾಗಿ ಮನಗಂಡವನು ಶ‍್ರೀಕೃಷ್ಣ. ಆದರೆ ಶ‍್ರೀಕೃಷ್ಣ ಈ ಘಟನೆಯನ್ನು ನೋಡುವ ದೃಷ್ಟಿಯೇ ಬೇರೆ. ಶ‍್ರೀಕೃಷ್ಣ ಇದನ್ನು ಭೂಮ ದೃಷ್ಟಿಯಿಂದ ನೋಡುತ್ತಾನೆ, ದೂರದೃಷ್ಟಿಯಿಂದ ನೋಡುತ್ತಾನೆ, ಧರ್ಮ ಕರ್ತವ್ಯ ಇವುಗಳ ದೃಷ್ಟಿಯಿಂದ ನೋಡುತ್ತಾನೆ. ಅರ್ಜುನನಾದರೋ ತನ್ನ ಕರ್ತವ್ಯವನ್ನು ಮರೆತು, ಮೋಹಕ್ಕೆ ವಶವಾಗಿ, ತಾತ್ಕಾಲಿಕ ದೃಷ್ಟಿಯಿಂದ ನೋಡುತ್ತಿರುವನು. ಇಲ್ಲಿಯೇ ಅವರಿಬ್ಬರಿಗೂ ಇರುವ ವ್ಯತ್ಯಾಸ ನಮಗೆ ಚೆನ್ನಾಗಿ ಗೊತ್ತಾಗುವುದು. ಈ ಅಧ್ಯಾಯದಲ್ಲಿ ನಾವು ಇದನ್ನು ಚೆನ್ನಾಗಿ ನೋಡುವೆವು.

ಅರ್ಜುನ ಇಲ್ಲಿ ಮರುಕದಿಂದ ಕೂಡಿದ್ದಾನೆ ಎಂದು ಸಂಜಯ ವರ್ಣಿಸುವನು. ಜೀವನದಲ್ಲಿ ದಯೆ ಬೇರೆ, ವ್ಯಾಮೋಹದಿಂದ ಪ್ರೇರಿತವಾದ ಮರುಕ ಬೇರೆ. ಹೊರಗಡೆಯಿಂದ ನೋಡುವುದಕ್ಕೆ ಎರಡೂ ಒಂದೆಯಾಗಿ ಕಾಣುವುದು. ಸುಣ್ಣದ ನೀರೂ ಒಂದೆ, ಹಾಲಿನ ಬಣ್ಣವೂ ಒಂದೆ. ಆದರೆ ಅದರ ಗುಣಗಳು ಬೇರೆ ಬೇರೆ. ಅದರಂತೆಯೇ ದಯೆಯೆಂಬುದು. ಇದು ನಿಷ್ಪಕ್ಷಪಾತವಾಗಿ ಎಲ್ಲರಿಗೂ ಮರುಕಪಡುವುದು. ವ್ಯಾಮೋಹದಿಂದ ಪ್ರೇರಿತವಾಗಿರುವ ಮರುಕಕ್ಕೆ ಕಾರಣ ಇವರು ನನ್ನವರು ಎಂಬುದು.

ಅರ್ಜುನ ಎಷ್ಟೊಂದು ವೇಳೆ ಇದಕ್ಕೆ ಮುಂಚೆ ಯುದ್ಧ ಮಾಡಲಿಲ್ಲ. ಎಷ್ಟೊಂದು ಜನರನ್ನು ಕೊಲ್ಲಲಿಲ್ಲ. ಆಗ ಅವನ ಹೃದಯದಲ್ಲಿ ಯಾವ ಅಳುಕೂ ಇರಲಿಲ್ಲ. ಅದೆಲ್ಲ ಈಗ ಬರುವುದು. ಇದಕ್ಕೆಲ್ಲ ಕಾರಣ ಮಮತೆ. ಮಮತೆಯ ಮಂಜು ಯಾವಾಗ ಮುಚ್ಚಿಕೊಳ್ಳುವುದೋ ಆಗ ತನ್ನ ಕರ್ತವ್ಯವನ್ನು ಮರೆಯುವನು, ಯುಕ್ತಾಯುಕ್ತ ಪರಿಜ್ಞಾನವನ್ನು ಕಳೆದುಕೊಳ್ಳುವನು, ಅಜ್ಞಾನಿಯಂತೆ ಕಣ್ಣೀರನ್ನು ಕರೆಯುವನು. ಕಣ್ಣೀರಿಗೆ ಒಂದು ಸ್ಥಾನವಿದೆ, ಕಾಲವಿದೆ. ಆಗ ಅದು ಅನರ್ಘ್ಯ ರತ್ನದಂತೆ ಬೆಳಗುವುದು. ಆದರೆ ಈ ಸಮಯದಲ್ಲಿ ಈ ಸ್ಥಾನದಲ್ಲಿ ಕಣ್ಣೀರಿಗೆ ಸ್ಥಳವಿಲ್ಲ.

ಅರ್ಜುನ ಇಲ್ಲಿ ಶೋಕದಿಂದ ಕುದಿಯುತ್ತಿರುವನು. ಬೇಯುತ್ತಿರುವ ಶೋಕಕ್ಕೆ ಉರಿಯಲು ಸೌದೆಯನ್ನು ಒಡ್ಡುತ್ತಿರುವವನೇ ಅರ್ಜುನ. ಇದನ್ನು ಅವನು ಅರಿಯನು. ಶ‍್ರೀಕೃಷ್ಣನು ಇದನ್ನು ಹೋಗಲಾಡಿಸಲು ಹೀಗೆ ಹೇಳುತ್ತಾನೆ.

\begin{shloka}
ಕುತಸ್ತ್ವಾ ಕಶ್ಮಲಮಿದಂ ವಿಷಮೇ ಸಮುಪಸ್ಥಿತಮ್~।\\ಅನಾರ್ಯಜುಷ್ಟಮಸ್ವರ್ಗ್ಯಮಕೀರ್ತಿಕರಮರ್ಜುನ \hfill॥ ೨~॥
\end{shloka}

\begin{artha}
ಅರ್ಜುನ! ಇಂತಹ ವಿಷಮ ಸಮಯದಲ್ಲಿ ನಿನಗೆ ಬುದ್ಧಿಮಾಲಿನ್ಯ ಹೇಗೆ ಬಂತು? ಇದು ಅನಾರ್ಯರಿಗೆ ಸೇರಿದ್ದು, ನರಕಕ್ಕೆ ಒಯ್ಯುವುದು. ಅಪಕೀರ್ತಿಯನ್ನು ತರುವುದು.
\end{artha}

ಶ‍್ರೀಕೃಷ್ಣ ಅರ್ಜುನನಿಗೆ ಇದನ್ನು ವಿಷಮ ಸಮಯ ಎನ್ನುತ್ತಾನೆ. ಯುದ್ಧ ಮಾಡುವುದಕ್ಕೆ ಎಲ್ಲಾ ಅಣಿಯಾಗಿದೆ. ಈಗ ಹಿಂತಿರುಗಿ ಹೋಗುವುದಕ್ಕೆ ಸಮಯ ಮೀರಿದೆ. ನಾಟಕವೊಂದು ಇಂತಹ ದಿವಸ ಆಗುವುದು ಎಂದು ಬೇಕಾದಷ್ಟು ಜಾಹಿರಾತು ಮಾಡಿ ಟಿಕೇಟನ್ನು ಮಾರಿ, ಪ್ರೇಕ್ಷಕರು ನೋಡುವುದಕ್ಕೆ ಕಿಕ್ಕಿರಿದು ನೆರೆದಿರುವಾಗ, ಎರಡು ಬೆಲ್ಲು ಆಗಿ, ತೆರೆ ಮೇಲೇರುವುದಕ್ಕೆ ಮೂರನೆ ಬೆಲ್ಲು ಕೂಡ ಆಗಿರುವಾಗ, ನಾಟಕವನ್ನು ಮುಂದೂಡಿದೆ ಎಂದ ಹಾಗಿದೆ. ಈಗಿನ ಕಾಲದಲ್ಲಿ ಕೆಲವು ವೇಳೆ ವಿದ್ಯುತ್ ಶಕ್ತಿ ಅಥವಾ ಸಿನಿಮಾ ಪ್ರೊಜೆಕ್ಟರ್ ಕೆಟ್ಟು ಹೋದರೆ ಹಾಗೆ ಮಾಡುತ್ತಾರೆ. ಆದರೆ ಯಾವ ಆತಂಕವೂ ಇಲ್ಲದೆ ಇರುವಾಗ, ಇಂದು ನಾಟಕ ಆಡುವುದನ್ನು ನಿಲ್ಲಿಸಲು ಸಂಕಲ್ಪ ಮಾಡಿದ್ದೇವೆ ಎಂದಂತೆ ಆಗುವುದು. ಅರ್ಜುನನಿಗೆ ಯಾರೊಡನೆ ಯುದ್ಧ ಮಾಡಬೇಕಾಗಿದೆ ಎಂಬುದನ್ನು ಈಗ ಪ್ರತ್ಯಕ್ಷ ನೋಡುತ್ತಿದ್ದರೂ, ಹಿಂದಿನಿಂದ ಗೊತ್ತಿತ್ತು; ಯಾರ್ಯಾರು ದುರ್ಯೋಧನನಿಗೆ ಸಹಾಯಕ್ಕೆ ಇರುವರು ಎಂಬುದು. ಇದನ್ನೆಲ್ಲ ಮುಂಚೆಯೇ ಆಲೋಚನೆ ಮಾಡಬೇಕಾಗಿತ್ತು. ಈಗ ಹೊತ್ತು ಮೀರಿ ಹೋಗಿದೆ. ಹಿಂತಿರುಗಿದರೆ ಆಭಾಸವಾಗುವುದು.

\newpage

ಅರ್ಜುನನನ್ನು ಮುಸುಕಿರುವ ಈ ಮಾಯೆಯ ಮಂಜನ್ನು, ದಯೆ, ಕೃಪೆ, ಮರುಕ ಕರುಣೆ ಎಂಬ ದೊಡ್ಡ ದೊಡ್ಡ ಹೆಸರಿನಿಂದ ತಿಳಿಯದ ಜನ ಕರೆಯಬಹುದು. ಆದರೆ ಶ‍್ರೀಕೃಷ್ಣ ಅದು ಹೇಗಿದೆಯೋ ಹಾಗೆ ಹೇಳುತ್ತಾನೆ. ಅದನ್ನು ಒಂದು ಬುದ್ಧಿಮಾಲಿನ್ಯ, ಬುದ್ಧಿಯನ್ನು ಮುಸುಕಿರುವ ಕಸ, ಕಿಲುಬು ಎನ್ನುವನು. ಈ ಕಿಲುಬು ಈ ಕಸ ಬರುವುದಕ್ಕೆ ಸಮಯವನ್ನು ಹೊಂಚು ಹಾಕುತ್ತಿರುವುದು. ಮನಸ್ಸು ದುರ್ಬಲವಾದಾಗ ಅದು ಧಾಳಿ ಇಡುವುದು. ಅನೇಕ ವೇಳೆ ಪರೀಕ್ಷಾ ಸಮಯವನ್ನು ಹೊಂಚು ಹಾಕುತ್ತಿರುವುದು ಅದು ನಮಗೆ ಮೋಸವನ್ನು ಮಾಡಬೇಕಾದರೆ. ಕಬ್ಬಿಣದ ಅದುರುಗಳನ್ನು ದೊಡ್ಡ ದೊಡ್ಡ ಮೂಸೆಯಲ್ಲಿಟ್ಟು ಕಾಸುತ್ತಿರುವಾಗ ಕೆಲಸಕ್ಕೆ ಬಾರದ ಕಶ್ಮಲ ಮೇಲೆ ತೇಲುವುದು. ಅದನ್ನು ತೆಗೆದು ಹಾಕಿದರೆ ಘಟ್ಟಿಯಾದ ಕೆಳಗಿರುವ ಲೋಹ ಸಿಕ್ಕುವುದು. ಕೆಲವು ವೇಳೆ ಕಬ್ಬಿನ ಹಾಲನ್ನು ಬೆಲ್ಲ ಮಾಡಲು ಒಲೆಯ ಮೇಲೆ ಇಟ್ಟು ಕಾಯಿಸುತ್ತಿರುವಾಗ ಅದರಲ್ಲಿರುವ ಕಶ್ಮಲವೆಲ್ಲ ಮೇಲೆ ತೇಲುವುದು. ಅದು ಬೆಲ್ಲವಲ್ಲ. ಅದನ್ನು ತೆಗೆದು ಹಾಕಿದರೇನೆ ಶುದ್ಧವಾದ ಬೆಲ್ಲ ಆಗಬೇಕಾದರೆ. ಅರ್ಜುನನು ಈಗ ಪರೀಕ್ಷಾ ಸಮಯದಲ್ಲಿರುವನು. ಅವನೊಳಗೆ ಹೊಂಚು ಹಾಕುತ್ತಿದ್ದ ಮನೋ ದೌರ್ಬಲ್ಯವೆಲ್ಲ ಹೊರಗೆ ಬರುವುದು. ಶ‍್ರೀಕೃಷ್ಣನ ತೀಕ್ಷ್ಣ ಮತಿಗೆ ಇದು ಚೆನ್ನಾಗಿ ವ್ಯಕ್ತವಾಗುವುದು. ಅವನು ಈ ಗುಣವನ್ನು ಕೊಂಡಾಡುವುದಕ್ಕೆ ಹೋಗುವುದಿಲ್ಲ. ಶ‍್ರೀಕೃಷ್ಣ ಬಚ್ಚಿಡುವುದಿಲ್ಲ. ನಾರುತ್ತಿರುವ ಹುಣ್ಣನ್ನು ಗುಣ ಮಾಡಲು ರೋಗಿಗೆ ವ್ಯಥೆಯಾದರೂ ಚಿಂತೆಯಿಲ್ಲ ಎಂದು ಹೇಳಿ ಮದ್ದನ್ನು ಪ್ರಯೋಗಿಸುವನು. ನಮ್ಮಲ್ಲಿರುವ ದೋಷ ನಮಗೆ ದೋಷ ಎಂದು ಗೊತ್ತಾಗಬೇಕು. ಆಗಲೆ ಅದರಿಂದ ನಾವು ಪಾರಾಗಲು ಪ್ರಯತ್ನಿಸಬೇಕಾದರೆ. ನಮಗೆ ಅದು ಗೊತ್ತಾಗದೆ ಇದ್ದರೆ ತಿಳಿದವರು ಅದನ್ನು ಹೇಳಬೇಕು. ಹಾಗೆ ಹೇಳದೆ ಇದ್ದರೆ ಅವರೊಂದು ಅಪರಾಧವನ್ನು ಮಾಡುತ್ತಿರುವರು. ಪ್ರಿಯವಾಗಿರುವುದನ್ನು ಯಾರು ಬೇಕಾದರೂ ಹೇಳಬಲ್ಲರು. ಅದನ್ನು ಹೇಳುವುದಕ್ಕೆ ಕಾತರಿಸುತ್ತಿರುವರು. ಆದರೆ ನಿಜವಾಗಿಯೂ ನಾವು ಮತ್ತೊಬ್ಬನ ಹಿತೈಷಿಯಾಗಿದ್ದರೆ ಅವನ ನ್ಯೂನತೆಯನ್ನು ಎತ್ತಿ ತೋರಬೇಕು. ಶ‍್ರೀಕೃಷ್ಣನಿಗೆ ಅರ್ಜುನ ಸಖ. ಅರ್ಜುನನ ಅವಗುಣವನ್ನು ಶ‍್ರೀಕೃಷ್ಣ ಅವನಿಗೆ ಚೆನ್ನಾಗಿಯೇ ತೋರುವನು. ಆದರೆ ಶ‍್ರೀಕೃಷ್ಣನ ಮಾತಿನ ಹಿಂದೆ ವಿಷವಿಲ್ಲ.

ಇದು ಅನಾರ್ಯರು ಮಾಡುವ ಕೆಲಸ ಎನ್ನುವನು. ಆರ್ಯ ಯಾವಾಗಲೂ ಒಂದು ಕೆಲಸವನ್ನು ಮಾಡುವುದಕ್ಕೆ ಮುಂಚೆ ಎಲ್ಲವನ್ನೂ ಚೆನ್ನಾಗಿ ಪರಿಶೀಲಿಸುವನು. ಉದ್ವೇಗಕ್ಕೆ ವಶನಾಗಿ ಮಾಡು ವವನೂ ಅಲ್ಲ; ಉದ್ವೇಗಕ್ಕೆ ವಶನಾಗಿ ಬಿಡುವವನೂ ಅಲ್ಲ. ಇಲ್ಲಿ ಆರ್ಯ ಎಂದರೆ ಎಲ್ಲಿಯೋ ಹೊರಗಿನಿಂದ ಬಂದ ಒಂದು ಜನಾಂಗ, ದ್ರಾವಿಡರನ್ನು ಆಳುತ್ತಿದ್ದವರು ಎಂಬ ಅರ್ಥವಲ್ಲ. ಇಲ್ಲಿ ಆರ್ಯ ಎಂಬ ಪದಕ್ಕೆ ಬೇರೆ ಧ್ವನಿ ಇದೆ. ಯೋಗ್ಯ, ಬುದ್ಧಿವಂತ, ಜಾಣ, ಕಾರ್ಯಕುಶಲಿ ಎಂಬ ಭಾವನೆಯಲ್ಲಿ ಹೇಳುವನು. ಅರ್ಜುನನಿಗೆ ಅರಿವಾಗಬೇಕು ತಾನೊಂದು ಉತ್ತಮ ದರ್ಜೆಗೆ ಸೇರಿದ ವ್ಯಕ್ತಿ, ಕೆಳಗೆ ಬೀಳಬಾರದು ಎಂದು. ತನ್ನಿಂದ ಸಮಾಜ ಶ್ರೇಷ್ಠವಾಗಿರುವುದನ್ನು ನಿರೀಕ್ಷಿಸುತ್ತಿದೆ, ಅದರ ನಿರೀಕ್ಷಣೆ ವಿಫಲವಾಗಕೂಡದು, ತನ್ನ ಸ್ಥಾನದ ಮಾನವನ್ನು ಕಾಯಬೇಕು, ಅದಕ್ಕೆ ಗೌರವವನ್ನು ತರಬೇಕು–ಎಂಬ ಅರ್ಥವೆಲ್ಲ ಬರಬೇಕು, ಉಪಯೋಗಿಸುವ ಒಂದು ಪದದಿಂದ. ಶ‍್ರೀಕೃಷ್ಣನ ಮಾತು ಧ್ವನಿಪೂರ್ಣವಾಗಿದೆ. ಅವನು ನರನ ಅಶ್ವಗಳನ್ನು ಹೊಡೆಯುತ್ತಿದ್ದ ಸಾರಥಿ ಮಾತ್ರವಲ್ಲ. ಅರ್ಜುನನೆಂಬ ಅಶ್ವವನ್ನು ಚಾವಟಿಯ ತುದಿಯ ಶಬ್ದದಿಂದಲೇ ಮುಂದೆ ಹೋಗುವಂತೆ ಮಾಡ ಬಲ್ಲವನಾಗಿದ್ದ. ತಾನಾಡುವ ಮಾತಿನಲ್ಲಿ ಅಂತಹ ಶಕ್ತಿ ಮತ್ತು ಜ್ಞಾನವನ್ನು ತುಂಬುವನು.

ನೀನು ಯುದ್ಧಮಾಡದೆ ಇದ್ದರೆ ನರಕಕ್ಕೆ ಹೋಗುವೆ ಎನ್ನುವನು. ಅರ್ಜುನ ಭಾವಿಸಿದ್ದ ಯುದ್ಧಮಾಡಿ ಗುರುಹಿರಿಯರನ್ನು ಬಂಧುಬಾಂಧವರನ್ನು ಕೊಂದರೆ ತಾನು ನರಕಕ್ಕೆ ಹೋಗುತ್ತೇನೆ ಎಂದು. ಆದರೆ ಶ‍್ರೀಕೃಷ್ಣನಾದರೋ ಅದಕ್ಕೆ ವಿರೋಧವಾಗಿ ಹೇಳುತ್ತಾನೆ. ಶ‍್ರೀಕೃಷ್ಣ ಈ ಘಟನೆಯನ್ನು ನೋಡುವ ದೃಷ್ಟಿಯೇ ಬೇರೆ. ನಮ್ಮ ಕಣ್ಣೆದುರಿಗೆ ಒಂದು ಅನ್ಯಾಯ ಆಗುತ್ತಿದ್ದರೆ, ಅದನ್ನು ತಡೆಗಟ್ಟುವುದಕ್ಕೆ ಪ್ರಯತ್ನ ಮಾಡದೆ ಇರುವುದು, ಆಗುತ್ತಿರುವ ಅನ್ಯಾಯಕ್ಕೆ ಪ್ರೋತ್ಸಾಹ ಕೊಟ್ಟಂತೆ. ಪ್ರಪಂಚದಲ್ಲಿ ಬಲವಿದ್ದವನೇ ಗೆಲ್ಲುವನು; ನ್ಯಾಯ, ಸತ್ಯ, ಧರ್ಮ ಇವುಗಳೆಲ್ಲ ಅರ್ಥವಿಲ್ಲದ ಕ್ರಂದನ ಎಂಬುದನ್ನೆ ಶ್ರುತಿಪಡಿಸಿದಂತೆ ಆಗುವುದು ಅರ್ಜುನ ಯುದ್ಧಮಾಡದೆ ಇದ್ದರೆ. ಅರ್ಜುನ ತನಗೆ ಆಗುವ ನಷ್ಟವನ್ನು ಅಷ್ಟು ಗಮನಿಸದೆ ಇರಬಹುದು. ಆದರೆ ಆ ನಷ್ಟದಿಂದ ಸಮಾಜದ ಮೇಲೆ ದೊಡ್ಡದೊಂದು ಪ್ರತಿಕ್ರಿಯೆಯಾಗುವುದು. ಜವಾಬ್ದಾರಿಯಲ್ಲಿರುವ ಮನುಷ್ಯ ಇದನ್ನು ಗಮನಿಸ ಬೇಕಾಗಿದೆ.

ಇದರಿಂದ ಅಪಕೀರ್ತಿ ಬರುವುದು ಎನ್ನುವನು. ಅರ್ಜುನ ಚೆನ್ನಾಗಿ ತಿಳಿದುಕೊಳ್ಳಬಲ್ಲ ಮಾತನ್ನು ಶ‍್ರೀಕೃಷ್ಣ ಈಗ ಹೇಳುವನು. ಪಾಪದಿಂದ ಸತ್ತ ಮೇಲೆ ನರಕ ಪ್ರಾಪ್ತಿ. ಆದರೆ ಬದುಕಿರುವಾಗಲೇ ಆ ನರಕಯಾತನೆಗಿಂತ ಮಿಗಿಲಾದ ಯಾತನೆಯನ್ನು ತರುವುದು ಅಪಕೀರ್ತಿ. ಹಣ ಹೋದರೆ ಚಿಂತೆ ಯಿಲ್ಲ. ಪದವಿ ಹೋದರೆ ಚಿಂತೆಯಿಲ್ಲ. ಹತ್ತಿರದವರು ತೀರಿಕೊಂಡರೆ ಮೊದಲು ಕುಗ್ಗಿಹೋದರೂ ಅನಂತರ ಹೇಗೋ ಚೇತರಿಸಿಕೊಳ್ಳುವನು. ಆದರೆ ಕೆಟ್ಟ ಹೆಸರು ಒಮ್ಮೆ ಬಂದರೆ ಆಯಿತು, ಬದುಕಿರುವ ತನಕ ಹೋಗುವುದಿಲ್ಲ. ಅದನ್ನು ತಾಳಲಾರದೆ ಅನೇಕರು ಆತ್ಮಹತ್ಯೆ ಮಾಡಿಕೊಳ್ಳುವರು. ಶೋಕಾಕುಲನಾಗಿರುವಾಗ ಜನರ ಮಾತನ್ನು ತಾನು ಗಣನೆಗೆ ತರುವುದಿಲ್ಲ ಎನ್ನಬಹುದು ಅರ್ಜುನ. ಆದರೆ ಕಾಲಕ್ರಮೇಣ ಶೋಕ ಹೋಗುವುದು. ಹಿಂದಿನ ಉದಾಸೀನ ಅಥವಾ ವೈರಾಗ್ಯ ಆಗ ಇರುವುದಿಲ್ಲ. ಶಸ್ತ್ರಚಿಕಿತ್ಸಕ ಕೆಲವು ವೇಳೆ ಆಪರೇಷನ್ ಮಾಡುವಾಗ ನೋವು ಕಾಣದಿರಲಿ ಎಂದು ಸೂಜಿಯಿಂದ ಮದ್ದನ್ನು ಚುಚ್ಚುವನು. ಆಗ ರೋಗಿಗೆ ನೋವು ಕಾಣುವುದಿಲ್ಲ. ಆದರೆ ಔಷಧಿಯ ಪ್ರಭಾವ ತಗ್ಗುತ್ತ ತಗ್ಗುತ್ತ ನೋವಿನ ಪ್ರಭಾವ ಹೆಚ್ಚುತ್ತಾ ಹೋಗುವುದು.

\begin{shloka}
ಕ್ಲೈಬ್ಯಂ ಮಾ ಸ್ಮ ಗಮಃ ಪಾರ್ಥ ನೈತತ್ತ್ವಯ್ಯುಪಪದ್ಯತೇ~।\\ಕ್ಷುದ್ರಂ ಹೃದಯದೌರ್ಬಲ್ಯಂ ತ್ಯಕ್ತ್ವೋತ್ತಿಷ್ಠ ಪರಂತಪ\hfill॥ ೩~॥
\end{shloka}

\begin{artha}
ಪಾರ್ಥ, ಹೇಡಿಯಾಗಬೇಡ. ಇದು ನಿನಗೆ ಯೋಗ್ಯವಲ್ಲ. ಅತಿ ಹೇಯವಾದ ಹೃದಯ ದೌರ್ಬಲ್ಯದಿಂದ ಪಾರಾಗಿ ಎದ್ದು ನಿಲ್ಲು.
\end{artha}

ಶ‍್ರೀಕೃಷ್ಣ ಆಡುವ ಮಾತು ಎಂತಹ ಶಕ್ತಿ ಸಂಜೀವಿನಿಯಂತೆ ಇದೆ! ಶ‍್ರೀಕೃಷ್ಣ ಇಲ್ಲಿ ವೈರಾಗ್ಯವನ್ನು ಹೊಗಳುವುದಿಲ್ಲ. ಅರ್ಜುನನ ಕ್ಷಾತ್ರ ಪೌರುಷವನ್ನು ಮುಚ್ಚಿರುವ ಬೂದಿ ಅವನ ಶೋಕ. ತನ್ನ ಬೋಧನೆಯ ಬಿರುಗಾಳಿಯಿಂದ ಅದನ್ನು ಊದಿ ಆಚೆಗೆ ಬಿಸುಡುವನು.

ಹೇಡಿಯಾಗಬೇಡ ಎನ್ನುವನು. ವೀರ ಸಾಯುವುದು ಒಂದು ಬಾರಿ. ಹೇಡಿ ಪದೇ ಪದೇ ಸಾಯುತ್ತಿರುವನು. ಯಾವುದನ್ನು ಅರ್ಜುನ ದೊಡ್ಡ ವೈರಾಗ್ಯದ ಮಾತು ಎಂದು ಭಾವಿಸುವನೊ ಅದನ್ನು ಹೇಡಿತನವೆಂದು ಮುಚ್ಚುಮರೆಯಿಲ್ಲದೆ ಹೇಳುವನು. ಯಾವನು ತಾನೆಲ್ಲಿ ಸೋತುಹೋಗು ವೆನೊ ಎಂದು ಭಾವಿಸುತ್ತಾನೆಯೋ ಅವನು ಸೋಲುವುದರಲ್ಲಿ ಸಂದೇಹವಿಲ್ಲ ಎನ್ನುವನು, ವೀರ ಯೋಧನಾದ ನೆಪೋಲಿಯನ್ ಎಂಬುವನು. ಹೇಡಿತನ ಕ್ಷತ್ರಿಯನಿಗೆ ಭೂಷಣವಲ್ಲ. ಅದರಲ್ಲಿಯೂ ಹಲವು ಯುದ್ಧಗಳಲ್ಲಿ ಜಯವನ್ನು ಗಳಿಸಿ ಒಳ್ಳೆಯ ಬಿಲ್ಲುಗಾರ ಪರಂತಪ ಎಂಬ ಬಿರುದನ್ನು ಗಳಿಸಿದ ಅರ್ಜುನನಿಗೆ ಇದು ಸ್ವಲ್ಪವೂ ತರವಲ್ಲ. ಒಂದು ಹೆಸರನ್ನು ಗಳಿಸುವುದು ಎಷ್ಟು ಕಷ್ಟ; ಆದರೆ ಅದನ್ನು ಎಷ್ಟು ಸುಲಭವಾಗಿ ಹಾಳುಮಾಡಿಕೊಳ್ಳುವ ಸ್ಥಿತಿಯಲ್ಲಿರುವನು!

ಇದು ನಿನಗೆ ಯೋಗ್ಯವಲ್ಲ ಎನ್ನುವನು. ನೀನು ಉತ್ತಮವಾದ ಕೆಲಸವನ್ನು ಮಾಡಿ ಎಲ್ಲರ ಬಾಯಿಯಿಂದಲೂ ಹೊಗಳಿಸಿಕೊಳ್ಳಬೇಕು. ನಿನ್ನಂತಹ ದೊಡ್ಡ ವೀರ ಮಾಡುವ ಅಲ್ಪ ಕೆಲಸವಲ್ಲ ಇದು. ಇದು ಅಲ್ಪ ಕೆಲಸವೇ ಅಲ್ಲ. ಯುದ್ಧ ಮಾಡದೇ ಹಿಂದಿರುಗುವುದು ಅಯೋಗ್ಯದ ಕೆಲಸ. ಈ ಒಂದು ಮಾತು ಅರ್ಜುನನ ಹೃದಯದಲ್ಲಿ ಸುಪ್ತವಾಗಿರುವ ಪೌರುಷ ಮತ್ತು ಶಕ್ತಿ ಎಲ್ಲವನ್ನೂ ಸ್ಪಂದಿಸುವಂತೆ ಮಾಡುವುದು.

ಯುದ್ಧಮಾಡದೆ ಹಿಂತಿರುಗುವುದು ಅತಿ ಹೇಯವಾದುದು, ತುಚ್ಛವಾದುದು. ನಾವು ಹೇಗೆ ಕದಿಯುವುದು, ಸುಳ್ಳು ಹೇಳುವುದು, ಕೊಲೆ ಮಾಡುವುದು ಮುಂತಾದುವುಗಳನ್ನು ತುಚ್ಛವಾದ ಕೆಲಸಗಳು ಎಂದು ಭಾವಿಸುವೆವೊ ಆ ಗುಂಪಿನ ಪಾಪಕ್ಕೆ ಶ‍್ರೀಕೃಷ್ಣ ಹೇಡಿತನವನ್ನೂ ಸೇರಿಸುವನು. ಮೇಲಿನ ಪಾಪಗಳಾದರೋ ಸಣ್ಣ ಪ್ರಮಾಣದಲ್ಲಿ ಸಮಾಜದಲ್ಲಿ ಅನ್ಯಾಯ ಅಧರ್ಮ ಮುಂತಾದುವು ಹರಡಲು ಅವಕಾಶವನ್ನು ಉಂಟುಮಾಡುವುವು. ಆದರೆ ಅರ್ಜುನ ಯುದ್ಧ ಮಾಡದೆ ಹೋದರೆ ಅನ್ಯಾಯ ಅಧರ್ಮಕ್ಕಾಗಿ ನಿಂತ ಹನ್ನೊಂದು ಅಕ್ಷೋಹಿಣಿಯ ಮಹಾ ಸೈನ್ಯಕ್ಕೆ ಲೂಟಿಗೆ ದರೋಡೆಗೆ ಅವಕಾಶ ಮಾಡಿಕೊಟ್ಟಂತೆ ಆಗುವುದು. ವೀರಕ್ಷತ್ರಿಯ ಇಂತಹ ಸಮಯದಲ್ಲಿ ಅದನ್ನು ತಡೆಗಟ್ಟಬೇಕು. ತಡೆಗಟ್ಟಲು ಸಾಧ್ಯವಾಗದೆ ಹೋದರೆ ಆ ಮಹಾ ಪ್ರಯತ್ನದಲ್ಲಿ ಪ್ರಾಣವನ್ನಾದರೂ ಅರ್ಪಣೆ ಮಾಡಬೇಕು.

ಇದೊಂದು ಹೃದಯದೌರ್ಬಲ್ಯದ ಚಿಹ್ನೆ. ಕೊಂಡಾಡುವ ಮಹಾ ಕರುಣೆಯ ಕೆಲಸವಲ್ಲ. ಶ‍್ರೀಕೃಷ್ಣ ವ್ಯಕ್ತಿಯನ್ನು ಅಷ್ಟು ಚೆನ್ನಾಗಿ ತಿಳಿದುಕೊಳ್ಳಬಲ್ಲ. ವೈದ್ಯ ಹೇಗೆ ಕೆಲವು ಚಿಹ್ನೆಗಳಿಂದ ರೋಗಿ ಎಂತಹ ದಾರುಣ ಖಾಯಿಲೆಯಿಂದ ನರಳುತ್ತಿರುವನು ಎಂಬುದನ್ನು ಊಹಿಸುವನೊ ಹಾಗೆ ಭವವೈದ್ಯನಾದ ಶ‍್ರೀಕೃಷ್ಣ ಅರ್ಜುನನ ಜಾಡ್ಯವನ್ನು ಹೃದಯದೌರ್ಬಲ್ಯ ಎನ್ನುವನು. ವೀರ ಹೃದಯದೌರ್ಬಲ್ಯಕ್ಕೆ ಅವಕಾಶ ಕೊಡಕೂಡದು. ಹಾಗಾದರೆ ಅವನೇನಾದರೂ ಕಟುಕನಾಗ\-ಬೇಕೆ? ಯಾವ ಮಾನವೀಯ ಭಾವನೆಯೂ ಇಲ್ಲದ ಒಂದು ಕಲ್ಲಿನ ಗೊಂಬೆಯಾಗಬೇಕೆ? ಹಾಗಲ್ಲ. ಮಹಾಪುರುಷ ಸಮಯ ಬಂದಾಗ ಬೆಣ್ಣೆಯಂತೆ ಕರಗಬೇಕು. ಆದರೆ ಅವನು ಕೆಲವು ವೇಳೆ ವಜ್ರದಂತೆ ಕಠೋರನೂ ಆಗಬೇಕು. ಅಲ್ಲಿ ಕಣ್ಣೀರಿಗೆ ಆಸ್ಪದವಿಲ್ಲ. ಎರಕದಲ್ಲಿ ನಮ್ಮ ಚೇತನ ಪುಟಗೊಳ್ಳಬೇಕಾದರೆ ನಾವು ಅದಕ್ಕೆ ಕೊಡುವ ಬೆಲೆ ಇದು. ಇಲ್ಲದೇ ಇದ್ದರೆ ಯಾರು ಬೇಕಾದರೂ ವೀರರಾಗಬಹುದಿತ್ತು. ಅಂತಹ ಮಂದಿ ಯಾವಾಗಲೂ ಕಡಮೆ. ಏಕೆಂದರೆ ಅದಕ್ಕೆ ನಾವು ಕೊಡಬೇಕಾದ ಬೆಲೆ ಅಧಿಕ. ಆದಕಾರಣವೇ ಭವಭೂತಿ ಶ‍್ರೀರಾಮಚಂದ್ರನ ಶೀಲವನ್ನು ಬಣ್ಣಿಸುವಾಗ ಅವನನ್ನು “ವಜ್ರಾದಪಿ ಕಠೋರಾಣಿ ಮೃದೂನಿ ಕುಸುಮಾದಪಿ” ಎನ್ನುವನು.

‘ಉತ್ತಿಷ್ಠ’ ಎಂದರೆ ಎದ್ದುನಿಲ್ಲು ಎನ್ನುವನು. ಕಣ್ಣೀರನ್ನು ಒರಸಿಕೊ, ಗಾಂಡೀವವನ್ನು ಧರಿಸು, ಸೊಂಟ ಕಟ್ಟು ಯುದ್ಧ ಮಾಡುವುದಕ್ಕೆ. ಈ ಒಂದು ಸಣ್ಣ ಮಾತಿನಿಂದ ಸಮಸ್ಯೆಯನ್ನು ಎದುರಿಸ ಬೇಕಾಗಿದೆ ಎನ್ನುವನು. ಇದರಿಂದ ತಪ್ಪಿಸಿಕೊಂಡು ಹೋಗುವುದಕ್ಕೆ ಆಗುವುದಿಲ್ಲ. ಕಷ್ಟದಿಂದ ಪಾರಾಗುವುದಕ್ಕೆ ಸುಲಭವಾದ ಮಾರ್ಗವೇ ಅದನ್ನು ಭೇದಿಸಿಕೊಂಡು ಹೋಗುವುದು. ನಾವು ಓಡಿಹೋದರೆ ಅದು ನಮ್ಮನ್ನು ಅಟ್ಟಿಸಿಕೊಂಡು ಬರುವುದು. ನಾವು ಅದನ್ನು ಎದುರಿಸಿ ನಿಂತರೆ, ಅದು ಓಡುವುದು, ಕೊನೆಗೆ ಸೋಲನ್ನು ಒಪ್ಪಿಕೊಳ್ಳುವುದು. ನಾವು ಓಡಬೇಕು, ಇಲ್ಲವೆ ಎದುರಿಸ ಬೇಕು. ಅರ್ಜುನ ಓಡಿಹೋಗುವವರ ಗುಂಪಿಗೆ ಸೇರಿದವನಲ್ಲ. ಇಲ್ಲಿ ಸಮಸ್ಯೆಯೊಂದಿಗೆ ಮುಷ್ಟಾ ಮುಷ್ಟಿ ಯುದ್ಧ ಮಾಡಬೇಕು. ಸ್ವಾಮಿ ವಿವೇಕಾನಂದರು ಒಮ್ಮೆ ಮಾತನಾಡುವಾಗ ಹೇಳುತ್ತಿದ್ದರು: “ನೀವು ಗೀತೆಯಲ್ಲಿ ಏನನ್ನೂ ಓದಬೇಕಾಗಿಲ್ಲ. ‘ಕುತಸ್ತ್ವಾ ಕಶ್ಮಲಮಿದಂ’ ಮತ್ತು ‘ಕ್ಲೈಬ್ಯಂ ಮಾ ಸ್ಮ ಗಮಃ ಪಾರ್ಥ’ ಈ ಎರಡು ಶ್ಲೋಕಗಳನ್ನು ಓದಿ ಪಾರಾಯಣ ಮಾಡಿ ಅದರಲ್ಲಿರುವ ಅರ್ಥವನ್ನು ಮನನ ಮಾಡಿದರೆ ಸಾಕು. ಇಡೀ ಗೀತೆಯ ಬೋಧನೆಯ ಸಾರ ಇಲ್ಲಿದೆ.” ಎಂತಹ ವೀರನ ಗಾಯತ್ರಿ ಈ ಶ್ಲೋಕಗಳು! ಇದು ಬರೀ ಅರ್ಜುನನಿಗೆ ಮಾತ್ರ ಅನ್ವಯಿಸುವ ಶ್ಲೋಕವಲ್ಲ. ಪ್ರತಿ ಜೀವಿಯೂ ಒಂದಲ್ಲ ಒಂದು ಸಮಯದಲ್ಲಿ, ಒಂದಲ್ಲ ಒಂದು ಕಡೆ ಸಮಸ್ಯೆಯನ್ನು ಎದುರಿಸಬೇಕಾಗಿದೆ. ಅನೇಕ ವೇಳೆ ನಾವು ತಪ್ಪಿಸಿಕೊಂಡು ಹೋಗುವುದಕ್ಕೆ ಪ್ರಯತ್ನಿಸುತ್ತೇವೆ. ಆದರೆ ಸಮಸ್ಯೆ ಓಡಿಬಂದು ನಮ್ಮ ಜುಟ್ಟನ್ನು ಹಿಡಿಯುವುದು. ಆದ ಕಾರಣವೇ ಜೀವನವನ್ನು ಎದುರಿಸಬೇಕು. ಈ ಹೋರಾಟವೇ ನಮ್ಮನ್ನು ಪುರುಷಸಿಂಹರನ್ನಾಗಿ ಮಾಡುವುದು. ವೀರರನ್ನು ತಯಾರು ಮಾಡುವ ಗರಡಿಯ ಮನೆಯೇ ಹೋರಾಟ. ಯಾವನು ಇದಕ್ಕೆ ಅಳುಕುವನೋ ಅವನು ನಗಣ್ಯನಾಗುವನು. ಮೇಲಿನ ಎರಡು ಶ್ಲೋಕಗಳ ಭಾವನೆ ಕುರುಕ್ಷೇತ್ರದ ಸಮರಾಂಗಣದಲ್ಲಿ ಅರ್ಜುನನಿಗೆ ಎಷ್ಟು ಆವಶ್ಯಕವಾಗಿತ್ತೋ ಅಷ್ಟೇ ಆವಶ್ಯಕ ಇಂದು ನಮ್ಮ ದೇಶಕ್ಕೆ; ಇಂದು ಒಂದು ಹೊಸ ರಾಷ್ಟ್ರವನ್ನು ನಿರ್ಮಾಣ ಮಾಡುತ್ತಿರುವ ಸಂದಿಗ್ಧ ಪರಿಸ್ಥಿತಿಯಲ್ಲಿ ಹಲವಾರು ಸಮಸ್ಯೆಗಳು ಮೇಲೆದ್ದು ನಮ್ಮ ಉತ್ಸಾಹ ಸ್ಫೂರ್ತಿಯನ್ನು ಕುಗ್ಗಿಸಲು ಬರುತ್ತಿರುವಾಗ ಇದು ಅತ್ಯಾವಶ್ಯಕ. ಅರ್ಜುನ ಪುನಃ ತನ್ನ ಶೋಚನೀಯ ಸ್ಥಿತಿಯನ್ನು ವಿವರಿಸುತ್ತಾನೆ:

\begin{shloka}
ಕಥಂ ಭೀಷ್ಮಮಹಂ ಸಂಖ್ಯೇ ದ್ರೋಣಂ ಚ ಮಧುಸೂದನ~।\\ಇಷುಭಿಃ ಪ್ರತಿಯೋತ್ಸ್ಯಾಮಿ ಪೂಜಾರ್ಹಾವರಿಸೂದನ \hfill॥ ೪~॥
\end{shloka}

\begin{artha}
ಅರಿನಾಶಕ ಮಧುಸೂದನ, ಪೂಜಾರ್ಹರಾದ ಭೀಷ್ಮ, ದ್ರೋಣ ಮುಂತಾದವರೊಡನೆ ಬಾಣದಿಂದ ಹೇಗೆ ಎದುರಿಸಿ ಯುದ್ಧ ಮಾಡಲಿ?
\end{artha}

ಅರ್ಜುನನ ಹೃದಯದಲ್ಲಿ ಮಾನವಸಹಜವಾದ ಭಾವನೆ ಏಳುವುದು. ಭೀಷ್ಮರು ಇವನ ತಾತ, ತನ್ನ ತಂದೆಯನ್ನು ಕಳೆದುಕೊಂಡು ಕಾಡಿನಲ್ಲಿ ಅನಾಥರಾಗಿದ್ದಾಗ ಕರೆತಂದು ಅವರಿಗೆ ಹಸ್ತಿನಾವತಿಯಲ್ಲಿ ಆಶ್ರಯ ಕೊಟ್ಟು ಬೆಳೆಸಿದವರು ಅವರು. ಅವರ ಆಶ್ರಯ ಮತ್ತು ಹರಕೆಯ ಬಲದಿಂದ ಬೆಳೆದವರು ಪಾಂಡವರು. ಅದರಂತೆಯೇ ಇವನ ಬಿಲ್ಲಿನ ಗುರುವಾದ ದ್ರೋಣಾ\-ಚಾರ್ಯರು. ಅವರು ತನ್ನ ಮಗ ಅಶ್ವತ್ಥಾಮನಿಗಿಂತ ಹೆಚ್ಚಾಗಿ ಅರ್ಜುನನನ್ನು ಪ್ರೀತಿಸಿದ್ದರು. ಆದರ್ಶ ಗುರುವಿಗೆ, ಪ್ರೀತಿ ತನ್ನ ಮಗನಿಗಿಂತ ಯೋಗ್ಯ ಶಿಷ್ಯನ ಮೇಲೆ ಅಧಿಕ. ಶಿಷ್ಯ ನಿಜ\-ವಾಗಿಯೂ ಮಾನಸಿಕ ಪುತ್ರ. ಗುರುವನ್ನು ಅರ್ಥ ಮಾಡಿಕೊಂಡವನು, ಅವನಿತ್ತ ವಿದ್ಯೆಯನ್ನು ಅರಗಿಸಿಕೊಂಡವನು, ಆ ವಿದ್ಯೆಯನ್ನು ಮುಂದು ವರಿಸಿಕೊಂಡವನು, ಗುರುವಿನ ಪ್ರೀತಿಯ ಸವಿಯನ್ನು ಚೆನ್ನಾಗಿ ರುಚಿ ನೋಡಿದವನು ಅರ್ಜುನ. ಇವರ ಮೇಲೆ ಬಾಣದ ಮಳೆಗರೆಯುವುದೆ, ಇವರನ್ನು ನೋಯಿಸುವುದೆ ಎಂಬ ಭಾವನೆ ಅವನನ್ನು ಬಾಧಿಸಿದುದರಲ್ಲಿ ಆಶ್ಚರ್ಯವೇನಿಲ್ಲ. ಮಾನವ ಸಹಜವಾದ ಭಾವನೆ ಇದು. ಆದರೆ ಕೆಲವು ವೇಳೆ ಕರ್ತವ್ಯಪರಾಯಣತೆಯ ದೃಷ್ಟಿಯಿಂದ ಇದನ್ನು ಮೀರಿ ಹೋಗಬೇಕು. ಹಿಂದಿನ ಮಮತೆಯಿಂದ ಕಿತ್ತುಕೊಳ್ಳುವಾಗ ತುಂಬಾ ಯಾತನೆ ಆಗುವುದು. ಆದರೆ ಮಹಾಪುರುಷ ಈ ಯಾತನೆಯ ಬೆಲೆಯನ್ನು ಕೊಡಲೇ ಬೇಕಾಗಿದೆ.

\begin{shloka}
ಗುರೂನಹತ್ವಾ ಹಿ ಮಹಾನುಭಾವಾನ್​\\ಶ್ರೇಯೋ ಭೋಕ್ತುಂ ಭೈಕ್ಷ್ಯಮಪೀಹ ಲೋಕೇ~।\\ಹತ್ವಾರ್ಥಕಾಮಾಂಸ್ತು ಗುರೂನಿಹೈವ\\ಭುಂಜೀಯ ಭೋಗಾನ್ ರುಧಿರಪ್ರದಿಗ್ಧಾನ್ \hfill॥ ೫~॥
\end{shloka}

\begin{artha}
ಮಹಾನುಭಾವರಾದ ಗುರುಗಳನ್ನು ಕೊಲ್ಲದೆ ಈ ಲೋಕದಲ್ಲಿ ಭಿಕ್ಷಾನ್ನದಿಂದಲಾದರೂ ಜೀವಿಸುವುದು ಶ್ರೇಯ ಸ್ಕರ. ಆದರೆ ಪ್ರಯೋಜನಾಪೇಕ್ಷಿಗಳಾದ ಗುರುಗಳನ್ನು ಕೊಂದ ರಕ್ತದಿಂದ ಲೇಪನವಾದ ಭೋಗಗಳನ್ನು ಈ ಲೋಕದಲ್ಲಿ ಮಾತ್ರ ಅನುಭವಿಸಬಹುದು.
\end{artha}

ಯುದ್ಧಮಾಡಿ ಗುರುಗಳನ್ನು ಕೊಂದು ಬರುವ ರಾಜ್ಯವನ್ನು ಅನುಭವಿಸುವುದಕ್ಕಿಂತ ಬೇಡಿ ಜೀವನ ತಳ್ಳುವುದು ಮೇಲೆಂದು ಭಾವಿಸುವನು. ದುಃಖ ಬಂದಾಗ ನಾವು ಏನೇನೋ ಮಾತನಾಡುತ್ತೇವೆ. ಆಡುವ ಮಾತಿನಲ್ಲಿ ಯುಕ್ತಾಯುಕ್ತ ಪರಿಜ್ಞಾನ ಇರುವುಲ್ಲ. ಬೇಡಿದರೆ ಸುಖ ಎಂದು ಅರ್ಜುನ ಭಾವಿಸುತ್ತಾನೆ. ಬೇಡಿದಾಗ ಯಾರಾರು ಎಂತೆಂತಹ ಮಾತುಗಳನ್ನು ಆಡುತ್ತಾರೆ ಎಂಬುದನ್ನು ಅರಿಯ ಅರ್ಜುನ. ಸ್ವಾಭಿಮಾನಿಯಾದ ಅರ್ಜುನ ಇವುಗಳನ್ನೆಲ್ಲ ಕಿವುಡುಗಿವಿಯಿಂದ ಕೇಳುವಷ್ಟು ಸಾತ್ತ್ವಿಕ ಶಕ್ತಿಯನ್ನು ಸಂಪಾದಿಸಿಕೊಂಡಿಲ್ಲ. ಅಂತಹ ಜೀವನಕ್ಕೆ ಎಷ್ಟೊಂದು ಅಣಿ ಯಾಗಬೇಕು. ನಿಂದಾ ಸ್ತುತಿಗಳಲ್ಲಿ ಮೌನವಾಗಿರುವುದು ನಾವಿಚ್ಛಿಸಿದರೆ ಬಂದುಬಿಡುವ ಗುಣವಲ್ಲ. ಕ್ಷತ್ರಿಯನ ಕುಲದಲ್ಲಿ ಹುಟ್ಟಿ, ಅದಕ್ಕೆ ತರಬೇತನ್ನು ತೆಗೆದುಕೊಂಡು, ಪರೀಕ್ಷಾ ಸಮಯ ಬಂದಾಗ ಕ್ಷತ್ರಿಯನ ಗುಣಗಳನ್ನೇ ಮರೆಯುವ ಸ್ಥಿತಿಯಲ್ಲಿರುವ ಅರ್ಜುನ, ಯಾವ ತರಬೇತನ್ನೂ ತೆಗೆದುಕೊಳ್ಳದೆ ಸ್ಥಿತಪ್ರಜ್ಞನ ಶಿಖರಕ್ಕೆ ನೆಗೆಯಲು ಇಚ್ಛಿಸುವನು!

ಗುರುಗಳು, ಹಿರಿಯರು ಯಾರು? ತನ್ನ ಶಿಷ್ಯ ಮತ್ತು ಮೊಮ್ಮಗನಾದ ಅರ್ಜುನನನ್ನೇನೋ ಪ್ರೀತಿಸುತ್ತಿದ್ದರು. ಆದರೆ ಪ್ರೀತಿಗಾಗಿ ಎಲ್ಲವನ್ನೂ ಅವರು ಬಲಿಕೊಡಲು ಸಿದ್ಧರಾಗಿರಲಿಲ್ಲ. ದುರ್ಯೋಧನ ಮಾಡುತ್ತಿರುವುದು ತಪ್ಪು ಎಂದು ಬುದ್ಧಿವಾದ ಹೇಳಿದರೇ ಹೊರತು ಅದನ್ನು ಬಲಾತ್ಕಾರವಾಗಿ ದುರ್ಯೋಧನನ ಮೇಲೆ ಪ್ರಯೋಗ ಮಾಡಲಿಲ್ಲ. ಹೇಗೂ ಅವನೊಂದಿಗೆ ರಾಜಿ ಮಾಡಿಕೊಂಡರು. ಭೀಷ್ಮ ದ್ರೋಣರಿಬ್ಬರೂ ತಮ್ಮ ಮಾತನ್ನು ಕೇಳದೆ ಇದ್ದರೆ, ನಿನ್ನೊಂದಿಗೆ ನಾವು ಯುದ್ಧಮಾಡುವುದಿಲ್ಲ ಎಂದು ದುರ್ಯೋಧನನಿಗೆ ಮುಂಚೆಯೇ ಹೇಳಿದ್ದರೆ, ಇಂತಹ ಸಂದಿಗ್ಧ ಪರಿಸ್ಥಿತಿ ಉದಿಸುತ್ತಿರಲೇ ಇರಲಿಲ್ಲ ಎನ್ನಬಹುದು. ಇವರಿಬ್ಬರೂ ಧಣಿ ಕೊಡುವ ಸಂಬಳಕ್ಕೆ ಕೆಲಸ ಮಾಡಿದರೇ ಹೊರತು ಅವನ ಗುಣಾವಗುಣಗಳನ್ನು ವಿಮರ್ಶಿಸುವ ಗೋಜಿಗೆ ಹೋಗಲಿಲ್ಲ. ಪಾಪವನ್ನು ಮಾಡಿದರೂ ಒಂದೇ, ಹಾಗೆ ಮಾಡುತ್ತಿರುವವನಿಗೆ ಸಹಾಯವನ್ನು ಮಾಡುವುದೂ ಒಂದೇ. ಅದರ ಪ್ರತಿಫಲವನ್ನು ಅನುಭವಿಸಲೇಬೇಕು.

ಇಂತಹವರನ್ನು ಕೊಂದರೆ ಕೇವಲ ಇಹದಲ್ಲಿ ಮಾತ್ರ ಸುಖ, ಅನಂತರ ನರಕಪ್ರಾಪ್ತಿ ಎಂದು ಅರ್ಜುನ ಭಾವಿಸುವನು. ಸಂಕುಚಿತ ದೃಷ್ಟಿಯಿಂದ ಮಾತ್ರ ಅರ್ಜುನ ಈ ಸಮಸ್ಯೆಯನ್ನು ನೋಡುವನು. ಗುರುಹಿರಿಯರ ಕೊಲೆ ಮಾಡಿದರೆ ಪಾಪ. ಆದರೆ ಅವರು ಪಾಪಕ್ಕೆ ಬೆಂಬಲಿಗ\-ರಾಗಿರುವಾಗ ಅವರನ್ನು ಕೊಂದರೆ ಪಾಪ, ಎಂದರೆ ಒಂದು ಕೃತ್ಯವನ್ನು ಮಾತ್ರ ತೆಗೆದುಕೊಳ್ಳು\-ತ್ತಾನೆಯೇ ಹೊರತು, ಅದರ ಉದ್ದೇಶವನ್ನು ಗಮನಿಸುವುದಿಲ್ಲ. ಅಲ್ಲಿ ಮಾಡುವ ಕೃತ್ಯಕ್ಕಿಂತ ಉದ್ದೇಶ ಮುಖ್ಯ. ಸಾಧಾರಣವಾಗಿ ಇನ್ನೊಬ್ಬನನ್ನು ಹಣಕ್ಕೋ ಜಿದ್ದಿಗೋ ಕೊಂದರೆ ಅವನಿಗೆ ಫಾಸಿ ವಿಧಿಸುವರು. ಯುದ್ಧದಲ್ಲಿ ಸಿಪಾಯಿ ಶತ್ರುಗಳನ್ನು ಕೊಂದರೆ ಅವನಿಗೆ ವೀರಪದಕವನ್ನು ಕೊಡುವರು. ಅದನ್ನು ಶ್ಲಾಘಿಸುವರು. ಮಮತೆ ಮತ್ತು ಶೋಕಗಳಿಂದ ಕದಡಿಹೋದ ಅರ್ಜುನನ ಬುದ್ಧಿಗೆ ಇವುಗಳಾವುವೂ ಅರ್ಥವಾಗುವಂತಿಲ್ಲ.

\begin{shloka}
ನ ಚೈತದ್ವಿದ್ಮಃ ಕತರನ್ನೋ ಗರೀಯೋ\\ಯದ್ವಾ ಜಯೇಮ ಯದಿ ವಾ ನೋ ಜಯೇಯುಃ~।\\ಯಾನೇನ ಹತ್ವಾನ ಜಿಜೀವಿಷಾಮ\\ಸ್ತೇsವಸ್ಥಿತಾಃ ಪ್ರಮುಖೇ ಧಾರ್ತರಾಷ್ಟ್ರಾಃ \hfill॥ ೬~॥
\end{shloka}

\begin{artha}
ನಾವು ಇವರನ್ನು ಜಯಿಸುತ್ತೇವೆಯೋ ಅಥವಾ ಇವರೇ ನಮ್ಮನ್ನು ಜಯಿಸುತ್ತಾರೆಯೋ ಮತ್ತು ಇವೆರಡರಲ್ಲಿ ಯಾವುದು ಮೇಲೋ ನಾವು ಅರಿಯೆವು. ಯಾರನ್ನು ಕೊಂದು ಬಾಳಬೇಕೆಂದು ಬಯಸುವುದಿಲ್ಲವೋ ಅಂತಹ ಕೌರವನೇ ನಮ್ಮೆದುರಿಗ ಇರುವನು.
\end{artha}

ಯುದ್ಧದಲ್ಲಿ ಯಾರು ಜಯಿಸುತ್ತಾರೆ ಎಂದು ಹೇಳುವಂತಿಲ್ಲ. ಕೌರವರ ಕಡೆ ಪಾಂಡವರ ಕಡೆಗಿಂತ ಹೆಚ್ಚಾಗಿ ಸೈನ್ಯವಿದೆ. ಅವರ ಸಹಾಯಕ್ಕೆ ಬಂದಿರುವ ವೀರಾಧಿವೀರರಿಗೆ ಬರಗಾಲವಿಲ್ಲ. ಬರೀ ವ್ಯಾವಹಾರಿಕ ದೃಷ್ಟಿಯಿಂದ ನೋಡಿದರೆ ಜಯ ಖಂಡಿತ ಪಾಂಡವರಿಗೇ ಸಿಕ್ಕುವುದು ಎಂದು ಹೇಳುವಂತಿಲ್ಲ. ಆದಕಾರಣವೇ ಅರ್ಜುನ ನಾವು ಗೆಲ್ಲುತ್ತೇವೆಯೋ ಅವರು ಗೆಲ್ಲುತ್ತಾರೆಯೋ ಹೇಳುವಂತಿಲ್ಲ ಎನ್ನುವನು. ಒಂದು ವೇಳೆ ನಾವು ಸೋತುಹೋದರೆ ಆಗಲೆ ಆಗಿರುವ ಅಪಕಾರಕ್ಕೆ ಸಾಲವನ್ನು ತೀರಿಸುವುದಕ್ಕೆ ಹೆಚ್ಚು ಬಡ್ಡಿ ಕೊಟ್ಟು ಮತ್ತಷ್ಟು ಸಾಲವನ್ನು ತೀರಿಸಿದಂತೆ ಆಗುವುದು. ಈ ಸೋಲಿನಿಂದ ಎಂತಹ ಒಂದು ತೇಜೋವಧೆ ಆಗುವುದು ಪಾಂಡವರಿಗೆ.

ಒಂದು ವೇಳೆ ಪಾಂಡವರೇ ಗೆದ್ದರು ಎಂದು ಭಾವಿಸಿದರೂ ಅರ್ಜುನನ ದೃಷ್ಟಿಯಲ್ಲಿ ಅದೇನೂ ಅಷ್ಟು ಮೇಲಲ್ಲ. ಏಕೆಂದರೆ ಕೌರವರು ಬಂಧು ಬಾಂಧವರು ಗುರು ಹಿರಿಯರು ಇವರನ್ನೆಲ್ಲಾ ಕೊಲ್ಲಬೇಕಾಗುವುದು. ಅವರ ಕಡೆಯವರೂ ಏನೂ ಸುಮ್ಮನೆ ಸಾಯುವುದಕ್ಕೆ ಬಂದವರಲ್ಲ. ಸಾಯುವುದಕ್ಕೆ ಮುಂಚೆ ಅವರೂ ಕೂಡ ಬೇಕಾದಷ್ಟು ಜನರನ್ನು ಆಹುತಿ ತೆಗೆದುಕೊಳ್ಳುವರು. ಕೌರವರು ಮತ್ತು ಪಾಂಡವರು ಎರಡು ಕಡೆಯೂ ಬೇಕಾದಷ್ಟು ಕೊಲೆಯಾಗುವುದು. ಕೌರವರ ಕಡೆ ದುರ್ಯೋಧನಾದಿಗಳೆಲ್ಲಾ ತೀರಿಕೊಂಡರೆ ಆ ವೃದ್ಧ ಧೃತರಾಷ್ಟ್ರ ಮತ್ತು ಗಾಂಧಾರಿಯರ ಗೋಳು ಮತ್ತು ಹಲವು ವಿಧವೆಯರ ಮತ್ತು ಅನಾಥರ ಕಂಬನಿ ಇವುಗಳನ್ನೆಲ್ಲಾ ಸಹಿಸಬೇಕಾಗುವುದು. ಆದಕಾರಣವೇ ಇವೆರಡರಲ್ಲಿ ಯಾವುದು ಮೇಲೋ ಹೇಳುವುದಕ್ಕೆ ಆಗುವುದಿಲ್ಲ ಎನ್ನುವನು. ಎತ್ತ ಕಡೆ ತಕ್ಕಡಿ ಮೇಲೆ ಹೋದರೂ ದುಃಖ ತಪ್ಪಿದ್ದಲ್ಲ. ಸೋತರೆ, ಆಗಿರುವ ಅವಮಾನಕ್ಕೆ ಮತ್ತಷ್ಟನ್ನು ಸೇರಿಸಿದಂತೆ ಆಗುವುದು. ಗೆದ್ದರೆ, ಎಲ್ಲರ ಗೋಳಿನ ಜ್ವಾಲಾಮುಖಿಯ ಮೇಲೆ ಮಾತ್ರ ಇವನು ನಿಲ್ಲಬಹುದು. ಅವೆರಡರಲ್ಲಿ ಪರಿಣಾಮದ ಉಗ್ರತೆಯ ದೃಷ್ಟಿಯಿಂದ ಕಡಮೆ ಯಾವುದು ಎಂಬುದನ್ನು ನಿಶ್ಚಯಿಸಲು ಸಾಧ್ಯವಿಲ್ಲ.

‘ಯಾರನ್ನು ಕೊಂದು ನಮಗೆ ಬದುಕಲು ಇಚ್ಛೆ ಇಲ್ಲವೊ’ ಎನ್ನುವನು. ಪಾಂಡವರಿಗೆ ಜಯ ಬರಬೇಕಾದರೆ ಭೀಷ್ಮ ದ್ರೋಣಾದಿಗಳು, ಕೌರವನ ಸಹೋದರರು, ಅವನ ಸಹಾಯಕ್ಕೆ ಬಂದ ಅನೇಕ ಜನ ನಿರ್ನಾಮವಾಗುವರು. ಅರ್ಜುನನಿಗೆ ದುರ್ಯೋಧನಾದಿಗಳು ತಮಗೆ ಅನ್ಯಾಯ ಮಾಡಿರುವರು ಎಂಬುದು ಗೊತ್ತಿದೆ. ಅವರಿಗೆ ಬುದ್ಧಿಯನ್ನು ಕಲಿಸಬೇಕು ಎಂದು ಇಚ್ಛೆಯೂ ಇದೆ. ಆದರೆ ಹೀಗೆ ಬುದ್ಧಿ ಕಲಿಸುವುದಕ್ಕಾಗಿ ವಂಶವನ್ನೇ ನಿರ್ಮೂಲ ಮಾಡಿ ಹಲವು ಜನರ ಸಾವಿಗೆ ಮತ್ತು ನೋವಿಗೆ ಕಾರಣವಾಗಬೇಕಲ್ಲ! ಅಂತಹ ಒಂದು ಪರಿಸ್ಥಿತಿಯಲ್ಲಿ ಯಾರೆದುರಿಗೆ ಜಯವನ್ನು ತೋರಿ ಸುಖವನ್ನು ಅನುಭವಿಸುವುದು? ಕೌರವರು ಸೋತು ಬದುಕಿದರೆ ತಾನೆ ಅವರು ತಾವು ಮಾಡಿದ್ದು ತಪ್ಪು ಎಂದು ಪಶ್ಚಾತ್ತಾಪಪಡುವುದು. ಆದರೆ ಆ ಛಲಗಾರ ದುರ್ಯೋಧನ ಸೋಲನ್ನು ಕೇಳುವ ವ್ಯಕ್ತಿಯಲ್ಲ. ಅವನು ಬದುಕಿರುವ ತನಕ ಸೋಲೆಂಬುದಿಲ್ಲ. ಸತ್ತಮೇಲೆ ಸೋತನೆಂದು ಕೇಳುವಂತೆಯೇ ಇಲ್ಲ.

ರಾಜ್ಯ, ಲೌಕಿಕಸಂಪತ್ತು, ನಮ್ಮ ಬಟ್ಟೆ, ಒಡವೆ, ಸೀರೆ ಮುಂತಾದುವನ್ನು ಧರಿಸಿದಾಗ ಅದನ್ನು ನೋಡುವುದಕ್ಕೆ, ಮೆಚ್ಚುವುದಕ್ಕೆ ಜನರಿರಬೇಕು. ಯಾರೂ ಇಲ್ಲದೆ ಹಾಳೂರಿಗೆ ರಾಜನಾಗಿ ವೈಭವದಿಂದ ಸಿಂಹಾಸನದ ಮೇಲೆ ಕುಳಿತರೆ ಅದನ್ನು ನೋಡುವುದಕ್ಕೆ ಯಾರಿದ್ದಾರೆ? ಅದರಿಂದ ಎಂತಹ ಆನಂದ ಬಂದೀತು?–ಎಂದು ಅರ್ಜುನ ಭಾವಿಸುವನು.

\begin{shloka}
ಕಾರ್ಪಣ್ಯದೋಷೋಪಹತಸ್ವಭಾವಃ\\ಪೃಚ್ಛಾಮಿ ತ್ವಾಂ ಧರ್ಮಸಂಮೂಢಚೇತಾಃ~।\\ಯಚ್ಛ್ರೇಯಃ ಸ್ಯಾನ್ನಿಶ್ಚಿತಂ ಬ್ರೂಹಿ\\ತನ್ಮೇ ಶಿಷ್ಯಸ್ತೇsಹಂ ಶಾಧಿ ಮಾಂ ತ್ವಾಂ ಪ್ರಪನ್ನಮ್ \hfill॥ ೭~॥
\end{shloka}

\begin{artha}
ಕನಿಕರವೆಂಬ ದೋಷದಿಂದ ನನ್ನ ಸ್ವಭಾವ ಮಲಿನವಾಗಿದೆ. ಯಾವುದು ಧರ್ಮ ಎಂಬುದನ್ನು ತಿಳಿದುಕೊಳ್ಳಲಾರದವನಾಗಿ ನಿನ್ನನ್ನು ಕೇಳುತ್ತಿದ್ದೇನೆ. ಯಾವುದು ನನಗೆ ಶ್ರೇಯಸ್ಕರವೋ ಅದನ್ನು ನಿಶ್ಚಯಿಸಿ ನೀನು ಹೇಳು. ನಾನು ನಿನ್ನ ಶಿಷ್ಯ. ನಿನ್ನಲ್ಲಿ ಶರಣಾಗಿರುವ ನನಗೆ ಬುದ್ಧಿಯನ್ನು ಹೇಳು.
\end{artha}

ಅರ್ಜುನನಿಗೆ ಈಗ ಅನ್ನಿಸುತ್ತಿದೆ, ಈ ಕನಿಕರವೆಂಬ ಸುಳಿ ಮನಸ್ಸನ್ನೆಲ್ಲಾ ಕಲಕುತ್ತಿದೆ ಎಂದು. ಅದು ಒಂದು ಒಳ್ಳೆಯ ಸ್ವಭಾವ ಎಂದು ಮುಂಚೆ ಬಗೆದಿದ್ದ. ಆದರೆ ಅದು ಮನಸ್ಸನ್ನು ಕದಡಿದಾಗ, ಅದರಿಂದ ತಿಳಿವಳಿಕೆಯನ್ನೆಲ್ಲಾ ಕಳೆದುಕೊಂಡಾಗ, ಅದರಿಂದ ಬರೀ ವ್ಯಥೆಯಾಗುವಾಗ, ಅದೊಂದು ಜಾಡ್ಯವಾಗಿರಬೇಕು ಎಂದು ಅರಿವಾಗುವುದು. ನಮ್ಮ ದೌರ್ಬಲ್ಯದಿಂದ ಪಾರಾಗಬೇಕಾದರೆ ಅದು ದೌರ್ಬಲ್ಯ ಎಂದು ಮೊದಲು ನಮಗೆ ಅರ್ಥವಾಗಬೇಕು. ಆಗಲೇ ಅದರಿಂದ ಪಾರಾಗಲು ಯತ್ನಿಸು ವೆವು. ಯಾವಾಗ ದೋಷ ಎಂದು ಗೊತ್ತಾಗುವುದಿಲ್ಲವೋ ಅದೇನೋ ಒಂದು ಎಲ್ಲರೆದುರಿಗೆ ತೋರಿಸಿಕೊಂಡು ಹೆಮ್ಮೆಪಡುವ ವಿಷಯವೆಂದು ಭಾವಿಸುತ್ತೇವೆ. ಯಾವುದಕ್ಕೆ ನಾಚಬೇಕೋ ಅದಕ್ಕೆ ಹೆಮ್ಮೆಪಡುತ್ತೇವೆ. ಅರ್ಜುನನಿಗೆ ಈಗ ಚೆನ್ನಾಗಿ ಅರ್ಥ ಆಗಿದೆ, ಕೌರವರ ಮೇಲೆ ತನಗೆ ಇರುವ ಕರುಣೆ ಅಜ್ಞಾನಜನ್ಯವಾಗಿದೆ, ತನ್ನ ನೈಜ ತಿಳಿವಳಿಕೆಗೆ ಕಿಲುಬು ಕಟ್ಟಿದಂತಾಗಿದೆ ಎಂಬುದು. ಹೇಗೆ ಒಂದು ಕೊಳೆಯ ಕನ್ನಡಿ ಚೆನ್ನಾಗಿ ಪ್ರತಿಬಿಂಬಿಸಲಾರದೋ ಹಾಗೆ ಆಗಿದೆ ಅರ್ಜುನನ ಮನಸ್ಸು –ಪರಿಸ್ಥಿತಿಯನ್ನು ಸರಿಯಾಗಿ ತಿಳಿದುಕೊಳ್ಳಲಾರ.

ಮಂಜು ಆವರಿಸಿಕೊಂಡಾಗ ನಾವು ಯಾವುದನ್ನೂ ಸರಿಯಾಗಿ ತಿಳಿದುಕೊಳ್ಳಲಾರೆವು. ತಿಳಿದು ಕೊಳ್ಳಬೇಕಾದರೆ ಆ ಮಂಜು ಹೋಗಬೇಕು. ಇಲ್ಲಿ ಮಂಜು ಎಲ್ಲೋ ಹೊರಗೆ ಇರುವ ವಸ್ತುವಲ್ಲ. ಅರ್ಜುನನ ದೃಷ್ಟಿಯಲ್ಲಿಯೇ ಮಂಜಿನ ದೋಷವಿದೆ. ಈಗ ಅವನ ಬುದ್ಧಿ ಸರಿಯಾಗಿ ಕೆಲಸ ಮಾಡುವ ಸ್ಥಿತಿಯಲ್ಲಿ ಇಲ್ಲ. ಇದು ಅರ್ಜುನನಿಗೆ ಅರ್ಥವಾಗುತ್ತಿದೆ. ನಮಗೆ ಖಾಯಿಲೆ ಇದೆ ಎಂದು ಗೊತ್ತಾದಾಗಲೆ ವೈದ್ಯನ ಹತ್ತಿರ ಹೋಗುತ್ತೇವೆ. ಅದರಂತೆ ಈಗ ಅರ್ಜುನ ಶ‍್ರೀಕೃಷ್ಣನನ್ನು ಕೇಳುತ್ತಾನೆ, ಯಾವುದು ಶ್ರೇಯಸ್ಕರವೋ ಅದನ್ನು ನೀನು ನಿಶ್ಚಯಿಸಿ ಹೇಳು ಎಂದು. ಇದುವರೆಗೆ ತಾನೆ ಸರಿ ಎಂದು ತಿಳಿದಿದ್ದ. ಈಗ ತಾನು ಸರಿಯಲ್ಲ, ತನ್ನ ದೃಷ್ಟಿಯಲ್ಲಿ ಏನೋ ಒಂದು ದೋಷವಿದೆ, ನೀನು ಸರಿಯಾಗಿ ನೋಡುತ್ತಿರುವೆ, ನೀನು ನಿಶ್ಚಯಿಸಿ ಹೇಳು ಎನ್ನುವನು. ಜೀವನದಲ್ಲಿ ಶ್ರೇಯಸ್ಸು ಬೇರೆ, ಪ್ರೇಯಸ್ಸು ಬೇರೆ. ಶ್ರೇಯಸ್ಸು ಮೊದಲು ಮಾಡಲು ಅರುಚಿ, ಅಪ್ರಿಯ, ನೀರಸವಾಗಿ ಕಾಣುವುದು. ಅನಂತರ ಅದೇ ನಮ್ಮನ್ನು ಪೂರ್ಣತೆಯೆಡೆಗೆ ಒಯ್ಯುವುದು. ಪ್ರೇಯಸ್ಸು ಮೊದ ಮೊದಲು ಬಹಳ ಪ್ರಿಯವಾಗಿರುವುದು. ಆದರೆ ಕ್ರಮೇಣ ನಮ್ಮನ್ನು ವಿಷಯ ಪ್ರಪಂಚದ ಗೋಜಿಗೆ ಸಿಕ್ಕಿಸುವುದು. ರೋಗಿ ವೈದ್ಯನಿಗೆ ತನಗೆ ರುಚಿಕರವಾದ ಔಷಧಿಯನ್ನು ಕೊಡಿ ಎಂದು ಕೇಳದೆ ಯಾವುದರಿಂದ ತಾನು ಖಾಯಿಲೆಯಿಂದ ಪಾರಾಗಬಹುದು ಅಂತಹ ಔಷಧವನ್ನು ಕೊಡಿ ಎಂದು ಕೇಳಿಕೊಳ್ಳುವನು, ಔಷಧ ಮತ್ತು ಪಥ್ಯವನ್ನು ನಿರ್ಣಯಿಸುವುದಕ್ಕೆ ವೈದ್ಯನಿಗೆ ಅವಕಾಶವನ್ನು ಕೊಡುವನು. ಇದರಂತೆ ಅರ್ಜುನ ತನಗೆ ಯಾವುದು ಶ್ರೇಯಸ್ಕರ ಎಂಬುದು ತನಗಿಂತ ಹೆಚ್ಚಾಗಿ ಶ‍್ರೀಕೃಷ್ಣನಿಗೆ ಗೊತ್ತಿದೆ ಎಂದು ನಂಬುವನು.

ನಾನು ನಿನ್ನ ಶಿಷ್ಯ ಎಂದು ಹೇಳಿಕೊಳ್ಳುವನು. ಇದುವರೆಗೆ ತಾನು ಶ‍್ರೀಕೃಷ್ಣನ ಸಖ, ಅವನಿಗೆ ಸರಿಸಮ ಎಂದು ಭಾವಿಸಿದ್ದ. ಜೀವನದಲ್ಲಿ ಪರೀಕ್ಷಾ ಸಮಯ ತಾನೆ ನಮ್ಮ ಯೋಗ್ಯತೆಯನ್ನು ಒರೆಗೆ ತಿಕ್ಕುವುದು? ಈ ಸಮರಾಂಗಣದಲ್ಲಿ ಅರ್ಜುನನಿಗೆ ತಾನೆಷ್ಟು ಕೆಳಗೆ ಇರುವೆನು, ಶ‍್ರೀಕೃಷ್ಣ ಎಂತಹ ಮಹೋನ್ನತ ವ್ಯಕ್ತಿ ಎಂಬುದು ಅರ್ಥವಾಗುವುದು. ತಾನಾಗಿ ಆ ಮಹೋನ್ನತ ವ್ಯಕ್ತಿ ಎದುರಿಗೆ ಬಾಗುವನು. ನಾನು ಶಿಷ್ಯ, ನಿನ್ನಿಂದ ಕಲಿತುಕೊಳ್ಳುವುದಕ್ಕೆ ಕಾತರನಾಗಿರುವವನು, ನನ್ನ ಕೊರತೆಯನ್ನು ನೀನು ಪೂರ್ಣ ಮಾಡಬೇಕು ಎಂಬ ಭಾವವಿದೆ ಇದರಲ್ಲಿ. ಗುರು ತನ್ನ ಹೃದಯಕ್ಕೆ ಪರಮಾಪ್ತವಾದ ಆಧ್ಯಾತ್ಮಿಕ ಸತ್ಯಗಳನ್ನು ಹೇಳಬೇಕಾದರೆ ಅವನಿಗೆ ಶಿಷ್ಯನ ಹೃದಯ ಅರಿತಿರಬೇಕು. ಅಲ್ಲಿ ಮಾತ್ರ ಅವನ ಬೋಧನೆ ಬೇರೂರುವುದು. ಗುರುವಿಗೆ ತುಂಬಾ ಪ್ರಿಯನಾದವನು ತನ್ನ ಶಿಷ್ಯ. ಹೇಗೆ ವಿದ್ಯುತ್​ಶಕ್ತಿ ಒಂದು ಕಡೆಯಿಂದ ಮತ್ತೊಂದು ಕಡೆಗೆ ಹೋಗಬೇಕಾದರೆ ಅದಕ್ಕೆ ಒಂದು ತಂತಿ ಆವಶ್ಯಕವೋ ಹಾಗೆ ಗುರುವಿನಲ್ಲಿರುವ ಜ್ಞಾನ ಪ್ರಪಂಚಕ್ಕೆ ಬರಬೇಕಾದರೆ ಒಬ್ಬ ಯೋಗ್ಯ ಶಿಷ್ಯನ ಆವಶ್ಯಕತೆಯಿರುವುದು. ಗುರು ತನ್ನಲ್ಲಿರುವ ಜ್ಞಾನವನ್ನು ಕೊಡುವುದಕ್ಕೆ ಸಿದ್ಧನಾಗಿದ್ದರೂ ಅದಕ್ಕೆ ಹದವಾದ ಶಿಷ್ಯನ ಹೃದಯ ಬೇಕು. ಶ‍್ರೀಕೃಷ್ಣಾರ್ಜುನರು ಸ್ನೇಹಿತರಾಗಿದ್ದರು. ಏನೇನೊ ವಿಷಯಗಳನ್ನು ಮಾತನಾಡುತ್ತಿದ್ದರು. ಆದರೆ ಜೀವನದ ಮಹಾ ಆಧ್ಯಾತ್ಮಿಕ ಸತ್ಯಗಳನ್ನು ಹೇಳುವುದಕ್ಕೆ ಸಮಯ ಬಂದಿರಲಿಲ್ಲ. ಸಮಯ ಬರುವುದಕ್ಕೆ ಮುಂಚೆ ಅದನ್ನು ಹೇಳಿದರೆ ಹಿಡಿಸುವುದಿಲ್ಲ. ಈಗ ಅರ್ಜುನನಿಗೆ ಆ ಸಮಯ ಬಂದಿದೆ. ತಾನು ಸ್ವೀಕರಿಸಲು ಸಿದ್ಧನಾಗಿದ್ದೇನೆ, ಎಂದು ಹೇಳಿಕೊಳ್ಳುವನು.

ನಾನು ನಿನ್ನಲ್ಲಿ ಶರಣಾಗಿದ್ದೇನೆಂದು ಹೇಳಿಕೊಳ್ಳುವನು. ಇದು ಇನ್ನೂ ಒಂದು ಹೆಜ್ಜೆ ಮುಂದೆ ಹೋದಂತೆ. ನಾನು ನಿನ್ನನ್ನು ಸಂಪೂರ್ಣ ನಂಬಿದ್ದೇನೆ, ನನಗೆ ಯಾರು ಗತಿಯಿಲ್ಲ, ನೀನೊಬ್ಬನೇ ಕಾಪಾಡಬೇಕು, ಕಾಪಾಡಿದರೂ ನೀನೇ ನನಗೆ ಗತಿ, ಕಾಪಾಡದೆ ಇದ್ದರೂ ನಾನಂತೂ ನಿನ್ನನ್ನೇ ನೆಚ್ಚಿದ್ದೇನೆ, ನಿನಗೆ ತೋರಿದುದನ್ನು ನೀನು ಮಾಡಬಹುದೆಂದು ಎಲ್ಲಾ ಸ್ವಾತಂತ್ರ್ಯವನ್ನು ಗುರುವಿಗೆ ಕೊಡುವನು. ಸಾಧಾರಣ ಮನುಷ್ಯನಲ್ಲಿ ಶರಣಾದರೇ ಅವನು ನಮ್ಮನ್ನು ಕೈಬಿಡುವುದಿಲ್ಲ. ಇನ್ನು ಜಗದ ಗುರುವಾದ ಭಗವಂತನಲ್ಲಿ ಶರಣಾದರೆ ಅವನು ಕೈಬಿಡುವನೆ? ಇಲ್ಲ. ಆದರೆ ಅವನನ್ನು ನಂಬುವ ಧೈರ್ಯ ನಮಗೆ ಇರಬೇಕು. ಆಗ ಮಾತ್ರ ನಮ್ಮನ್ನು ಉದ್ಧರಿಸುವನು. ದಾರಿ ತೋರು ಎಂದು ದೇವರನ್ನು ಬೇಡಿದರೆ ಅವನು ದಾರಿ ತೋರದೆ ಇರನು. ಆದರೆ ನಾವು ಅವನನ್ನು ಕೇಳಬೇಕು. ಆಗ ಮಾತ್ರ ಗೊತ್ತಾಗುವುದು, ನಾವು ಅವನು ಕೊಡುವುದನ್ನು ತೆಗೆದುಕೊಳ್ಳುವುದಕ್ಕೆ ಸಿದ್ಧವಾಗಿದ್ದೇವೆ ಎಂಬುದು. ಕ್ರೈಸ್ತ ಬೈಬಲ್ಲಿನಲ್ಲಿ ನೀನು ದೇವರನ್ನು ಕೇಳು, ಅವನು ಕೊಡುವನು ಎನ್ನುವನು. ಅರ್ಜುನ ಕದಡಿಹೋದ ತನ್ನ ಮನಸ್ಸಿನ ಪ್ರಯತ್ನದಿಂದ ಸರಿಯಾದ ಮಾರ್ಗವನ್ನು ನಿಷ್ಕರ್ಷಿಸಲಾರದೆ ದಾರಿತೋರೆಂದು ಭಗವಂತನಿಗೆ ಹಲುಬುವನು.

\begin{shloka}
ನ ಹಿ ಪ್ರಪಶ್ಯಾಮಿ ಮಮಾಪನುದ್ಯಾ—\\ದ್ಯಚ್ಛೋಕಮುಚ್ಛೋಷಣಮಿಂದ್ರಿಯಾಣಾಮ್~।\\ಅವಾಪ್ಯ ಭೂಮಾವಸಪತ್ನಮೃದ್ಧಂ \\ರಾಜ್ಯಂ ಸುರಾಣಾಮಪಿ ಚಾಧಿಪತ್ಯಮ್ \hfill॥ ೮~॥
\end{shloka}

\begin{artha}
ಭೂಲೋಕದಲ್ಲಿ ಶತ್ರುಗಳಿಲ್ಲದೆ ಸಂಪತ್​ಸಮೃದ್ಧವಾಗಿರುವ ರಾಜ್ಯವನ್ನು ಪಡೆದರೂ ಮತ್ತು ದೇವತೆಗಳ ಮೇಲೆ ಒಡೆತನವನ್ನು ಪಡೆದರೂ, ಇಂದ್ರಿಯಗಳನ್ನು ಸೊರಗಿಸುತ್ತಿರುವ ತನ್ನ ಈ ಶೋಕವನ್ನು ಹೋಗಲಾಡಿಸುವು ದಾವುದೋ ಅದನ್ನು ಕಾಣೆನು.
\end{artha}

ಅರ್ಜುನ ಈಗ ಮೇಲಿನ ದೃಷ್ಟಿಯ ಕಡೆ ಹೋಗಬೇಕಾಗಿದೆ. ಇದುವರೆಗೆ ಕೇವಲ ತನ್ನ ಪ್ರಯೋಜನದ ದೃಷ್ಟಿಯಿಂದ ಈ ಯುದ್ಧವನ್ನು ನೋಡುತ್ತಿದ್ದ. ಇನ್ನು ಮೇಲೆ ಇದನ್ನು ಗೌಣಮಾಡಿ ಭೂಮದೃಷ್ಟಿಯಿಂದ ಸತ್ಯ ಅಥವಾ ಧರ್ಮದ ದೃಷ್ಟಿಯಿಂದ ನೋಡಬೇಕಾಗಿದೆ. ಯಾವಾಗಲೂ ದೃಷ್ಟಿ ಬದಲಾವಣೆ ಆಗುವಾಗ, ಕೆಳಗಿನಿಂದ ಮೇಲಿನ ದೃಷ್ಟಿಯ ಕಡೆ ಹೋಗುವಾಗ, ಮನಸ್ಸಿನಲ್ಲಿ ದೊಡ್ಡದೊಂದು ಆಂದೋಳನವಾಗುವುದು. ಅವನು ಹಿಂದೆ ಇದ್ದ ಸ್ಥಿತಿಯಲ್ಲಿಯೇ ಇರುವುದಕ್ಕೂ ಆಗುವುದಿಲ್ಲ. ಮೇಲಿನ ಕಡೆ ಪ್ರಯಾಣ ಮಾಡಬೇಕಾಗಿದೆ. ಮೇಲಿನ ದೃಷ್ಟಿಗೆ ಹತ್ತಬೇಕಾದರೆ ಕೆಳಗಿನ ದೃಷ್ಟಿಗೆ ಪ್ರಿಯವಾಗಿರುವುದನ್ನೆಲ್ಲಾ ತ್ಯಜಿಸಬೇಕಾಗಿದೆ. ಹೊಸ ದೃಷ್ಟಿಗೆ ಹೊಂದಿಕೊಳ್ಳುವ ತನಕ ಚೇತನ ಬಾಡಿಹೋದಂತೆ ಇರುವುದು. ಇದುವರೆಗೆ ತೌರುಮನೆಯಲ್ಲಿ ಬೆಳೆದ ಹೆಣ್ಣು\-ಮಗಳು ಮದುವೆಯಾಗಿ ಗಂಡನ ಮನೆಗೆ ಪ್ರಥಮ ಬಾರಿ ಹೊರಟಂತಿದೆ. ಇಷ್ಟು ದಿನ ಬೆಳೆದ ಸ್ಥಳ, ಒಡಹುಟ್ಟಿದ ಸಹೋದರ ಸಹೋದರಿಯರು ಎಲ್ಲರನ್ನೂ ಬಿಟ್ಟು ಕೈಹಿಡಿದವನ ಊರಿಗೆ ಹೋಗಬೇಕು. ಸುಮ್ಮನೆ ತೌರುಮನೆಯಲ್ಲಿ ಇನ್ನು ಮೇಲೆ ಇರುವಂತೆ ಇಲ್ಲ. ಗಂಡನ ಮನೆಗೆ ಹೋಗುವಾಗ ಕಂಬನಿದುಂಬಿ ಹಿಂದೆ ನೋಡುತ್ತ ನೋಡುತ್ತ ಮುಂದೆ ಹೋಗುವಳು.

ಅರ್ಜುನನಿಗೂ ಇಂತಹ ಒಂದು ಸನ್ನಿವೇಶ ಬಂದಿದೆ. ಭೂಲೋಕದ ಒಡೆತನ ಮಾತ್ರವಲ್ಲ, ದೇವತೆಗಳ ಮೇಲೆ ಒಡೆತನ ಬಂದರೂ ಅವನಿಗೆ ತೃಪ್ತಿಯಾಗುವಂತಿಲ್ಲ. ಮಾಡುವ ಕೆಲಸವನ್ನು ಬಿಡುವಂತಿಲ್ಲ. ಆದರೆ ಅದನ್ನು ಮಾಡುವ ದೃಷ್ಟಿಯ ಬದಲಾವಣೆ ಆಗಬೇಕಾಗಿದೆ. ಕೇವಲ ರಾಜ್ಯ ಲಾಭಕ್ಕಾಗಿ ಯುದ್ಧಮಾಡುವುದಕ್ಕೆ ಅರ್ಜುನ ಹೊರಟ. ಆದರೆ ದೇವರು ಈಗ ಅವನನ್ನು ಧರ್ಮ ಸಂಸ್ಥಾಪನೆಗೆ ಒಂದು ನಿಮಿತ್ತವಾಗಿರಲು ಉಪಯೋಗಿಸಿಕೊಳ್ಳುತ್ತಿರುವನು. ಅವನ ಕೆಲಸವನ್ನು ಮಾಡುವುದಕ್ಕೆ ತಾನೊಂದು ನಿಮಿತ್ತ; ಅದರಿಂದ ಬರುವ ಫಲಾಫಲಗಳು ತನಗಲ್ಲ ಎಂಬ ಭಾವನೆ ಅವನಲ್ಲಿ ದೃಢವಾಗಬೇಕು.

ಈಗ ಇರುವ ಸ್ಥಿತಿಯಲ್ಲಿ ತೀವ್ರ ಅತೃಪ್ತಿ ಇದೆ. ಯಾವ ದೃಷ್ಟಿಯ ಮೇಲೆ ನಿಂತು ಕೆಲಸ ಮಾಡಿದರೆ ಈ ಅತೃಪ್ತಿ ಮಾಯವಾಗುವುದೋ ಅದನ್ನು ತಿಳಿದುಕೊಳ್ಳಲು ಶ‍್ರೀಕೃಷ್ಣನನ್ನು ಬೇಡು\-ತ್ತಿರುವನು. ತೀವ್ರ ಅತೃಪ್ತಿಯೇ ಅಂಕುಶ ನಮ್ಮನ್ನು ಮುಂದೆ ಹೋಗುವಂತೆ ದೂಡಬೇಕಾದರೆ. ಜೀವನದಲ್ಲಿ ಎಲ್ಲಿ ಎಂದು ಅತೃಪ್ತಿ ಇದೆಯೋ ಅದನ್ನು ತೃಪ್ತಿಪಡಿಸುವ ಚೈತನ್ಯವೂ ಕಾದಿದೆ. ಇದೊಂದು ವಿಧಿ ನಿಯಮ. ಮುಂದೆ ಹೇಗೆ ಶ‍್ರೀಕೃಷ್ಣ ಅರ್ಜುನನ ಅತೃಪ್ತಿಯನ್ನು ಹೋಗ\-ಲಾಡಿಸುವನು ಎಂಬುದನ್ನು ನೋಡುವೆವು.

ಸಂಜಯ ಧೃತರಾಷ್ಟ್ರನಿಗೆ ಹೇಳುತ್ತಾನೆ:

\begin{shloka}
ಏವಮುಕ್ತ್ವಾ ಹೃಷೀಕೇಶಂ ಗುಡಾಕೇಶಃ ಪರಂತಪಃ~।\\ನ ಯೋತ್ಸ್ಯ ಇತಿ ಗೋವಿಂದಮುಕ್ತ್ವಾ ತೂಷ್ಣೀಂ ಬಭೂವ ಹ \hfill॥ ೯~॥
\end{shloka}

\begin{artha}
ಶತ್ರುಗಳಿಗೆ ಸಂತಾಪವನ್ನು ಉಂಟುಮಾಡುವ ಅರ್ಜುನ ಈ ಪ್ರಕಾರ ಶ‍್ರೀಕೃಷ್ಣನಿಗೆ ತಿಳಿಸಿ “ತಾನು ಯುದ್ಧ ಮಾಡುವುದಿಲ್ಲ” ಎಂದು ಹೇಳಿ ಸುಮ್ಮನಾದನು.
\end{artha}

\begin{shloka}
ತಮುವಾಚ ಹೃಷೀಕೇಶಃ ಪ್ರಹಸನ್ನಿವ ಭಾರತ~।\\ಸೇನಯೋರುಭಯೋರ್ಮಧ್ಯೇ ವಿಷೀದಂತಮಿದಂ ವಚಃ \hfill॥ ೧೦~॥
\end{shloka}

\begin{artha}
ಎರಡು ಸೈನ್ಯಗಳ ಮಧ್ಯದಲ್ಲಿ ವಿಷಾದಪಡುತ್ತಿರುವ ಅರ್ಜುನನನ್ನು ಕುರಿತು ಶ‍್ರೀಕೃಷ್ಣ ನಗುವವನಂತೆ ಈ ಮಾತು ಆಡುತ್ತಾನೆ.
\end{artha}

ನಾನು ಯುದ್ಧ ಮಾಡುವುದಿಲ್ಲ ಎಂದು ಹೇಳಿ ಅರ್ಜುನ ಸುಮ್ಮನಾದ. ಆಗ ಶ‍್ರೀಕೃಷ್ಣನ ಬೋಧನೆ ಮೊದಲಾಗುವುದು. ಅದು ಹೇಗೆ ಮೊದಲಾಗುವುದು? ಒಂದು ಮಂದಹಾಸದಿಂದ, ನಗುವಿನಿಂದ. ಇದೊಂದು ತುಂಬಾ ಸುಂದರವಾದ ಚಿತ್ರ. ಅರ್ಜುನ ದುಃಖದಲ್ಲಿ ಸೊರಗಿ ನಿಂತಿರುವನು. ಆದರೆ ಶ‍್ರೀಕೃಷ್ಣನಾದರೋ ಆ ಸೊರಗನ್ನು ನೋಡಿ ನಗುವನು. ಅರ್ಜುನನಿಗೆ ಜೀವನದಲ್ಲಿ ವಜ್ರಾಯುಧದಿಂದ ಪೆಟ್ಟು ಬಿದ್ದಂತಿದೆ. ಆದರೆ ಶ‍್ರೀಕೃಷ್ಣ ಅದನ್ನು ಬಹು ಲಘುವಾಗಿ ಭಾವಿಸುವನು. ಮಗು ಹೊರಗೆ ಆಟಕ್ಕೆ ಹೋಗಿ ಚಿಣ್ಣಿಕೋಲೊ ಗೋಲಿಯನ್ನೊ ಕಳೆದುಕೊಂಡು ತನ್ನ ಸರ್ವಸ್ವವೂ ಅಪಹಾರವಾಯಿತು ಎಂದು ಬಿಕ್ಕಿಬಿಕ್ಕಿ ಹೇಳುತ್ತಿದ್ದರೆ, ಹತ್ತಿರ ಇರುವ ಅಪ್ಪನೊ ಅಮ್ಮನೊ ಅದನ್ನು ನಗುತ್ತಾ ಕೇಳುವಂತಿದೆ. ಅರ್ಜುನನ ಮೇಲೆ ಶ‍್ರೀಕೃಷ್ಣ ಕೋಪಿಸಿಕೊಳ್ಳುವುದಿಲ್ಲ. ಅದರ ಬದಲು ಅವನ ಮಾತನ್ನು ಕೇಳಿ ನಗುವನು. ಶ‍್ರೀಕೃಷ್ಣ ಅರ್ಜುನ ಹೇಳುವುದನ್ನು ಗಂಭೀರವಾಗಿ ತೆಗೆದುಕೊಂಡರೆ ಕೋಪ ಬರಬಹುದು. ಆದರೆ ಶ‍್ರೀಕೃಷ್ಣ ಅದನ್ನು ಒಂದು ಹುಡುಗಾಟಿಕೆಯಂತೆ ನೋಡುವನು. ಶ‍್ರೀಕೃಷ್ಣ ಕೋಪಗೊಳ್ಳುವುದಿಲ್ಲ. ಕೋಪಗೊಂಡರೆ ಆದರ್ಶ ಗುರುವಾಗಲಾರ. ಶಿಷ್ಯನ ಪ್ರಶ್ನೆಯನ್ನು ಸಹನೆಯಿಂದ ಕೇಳಿ, ಅವನಿರುವ ಮೆಟ್ಟಿಲಿಗೆ ಇಳಿದು ಬಂದು, ಅವನ ಸಂಶಯವನ್ನು ಪರಿಹರಿಸಬೇಕಾಗಿದೆ. ಆಕಾಶವೆಲ್ಲಾ ಮೋಡದಿಂದ ಆಚ್ಛಾದಿತವಾಗಿರುವಾಗ ಫಳಾರ್ ಎಂದು ಮಿಂಚಿ ಮುಸಲ ಧಾರೆಯಂತೆ ಮಳೆ ಕೆಲವು ವೇಳೆ ಪ್ರಾರಂಭವಾಗುವುದು. ಅದರಂತೆ ಶ‍್ರೀಕೃಷ್ಣನ ನಗುವಿನ ಮಿಂಚು. ಮುಂದಿನ ಶ್ಲೋಕದಿಂದ ಗೀತೆಯ ಬೋಧನೆಯ ಅಮೃತವರ್ಷ ಪ್ರಾರಂಭವಾಗುವುದು. ಇನ್ನು ಮೇಲೆ ಕುರುಕ್ಷೇತ್ರದ ಸಮರಾಂಗಣ, ವೀರರು ಅವರ ಶಸ್ತ್ರಾಸ್ತ್ರಗಳು ಎಲ್ಲಾ ನಮ್ಮ ಮನಸ್ಸಿನಿಂದ ಮಾಯವಾಗುವುವು. ಅರ್ಜುನನ ರಥವೇ ಒಂದು ಪುಷ್ಯಾಶ್ರಮವಾಗುವುದು. ಅಲ್ಲಿ ಆಚಾರ್ಯ ಶ‍್ರೀಕೃಷ್ಣ ತನ್ನ ಶಿಷ್ಯನಾದ ಅರ್ಜುನನಿಗೆ ಬೋಧಿಸುವನು.

\begin{shloka}
ಅಶೋಚ್ಯಾನನ್ವಶೋಚಸ್ತ್ವಂ ಪ್ರಜ್ಞಾವಾದಾಂಶ್ಚ ಭಾಷಸೇ~।\\ಗತಾಸೂನಗತಾಸೂಂಶ್ಚ ನಾನುಶೋಚಂತಿ ಪಂಡಿತಾಃ \hfill॥ ೧೧~॥
\end{shloka}

\begin{artha}
ಯಾರಿಗೆ ವ್ಯಥೆಪಡಬಾರದೊ ಅವರನ್ನು ಕುರಿತು ಶೋಕಿಸುತ್ತಿದ್ದೀಯೆ. ಜ್ಞಾನಿಗಳಂತೆ ಮಾತ\-ನಾಡುತ್ತಿರುವೆ. ಪಂಡಿತರು ಸತ್ತವರಿಗಾಗಲಿ ಬದುಕಿರುವವರಿಗಾಗಲಿ ಶೋಕಿಸುವುದಿಲ್ಲ.
\end{artha}

ಅರ್ಜುನ ಈಗ ವ್ಯಥೆಪಡುತ್ತಿರುವುದು ದುರ್ಯೋಧನಾದಿಗಳು ಭೀಷ್ಮ ದ್ರೋಣಾದಿಗಳಿಗೆ. ಇವರು ವ್ಯಥೆಪಡುವುದಕ್ಕೆ ಯೋಗ್ಯರಲ್ಲ. ಏಕೆಂದರೆ ದುರ್ಯೋಧನ ಅಧರ್ಮ, ಅನ್ಯಾಯದಿಂದ ಇವರ ರಾಜ್ಯವನ್ನು ಇನ್ನೂ ಇವರಿಗೆ ಕೊಡದೆ ತನ್ನಲ್ಲಿ ಇಟ್ಟುಕೊಂಡಿರುವನು. ಭೀಷ್ಮದ್ರೋಣಾದಿಗಳು ಹಿರಿಯರಾಗಿರಬಹುದು, ಗುರುಗಳಾಗಿರಬಹುದು. ಆದರೆ ದುರ್ಯೋಧನನಿಗೆ ಬೆಂಬಲವಾಗಿ ನಿಂತಿರುವರು. ಅನ್ಯಾಯ ಮಾಡುವುದೂ ಒಂದೇ, ಹಾಗೆ ಮಾಡುತ್ತಿರುವವರಿಗೆ ಸಹಾಯ ಮಾಡುತ್ತಿರುವುದೂ ಒಂದೇ. ನ್ಯಾಯ ಅನ್ಯಾಯಗಳನ್ನು ವಿಮರ್ಶಿಸುವಾಗ ಇದನ್ನು ಚೆನ್ನಾಗಿ ಪರಿಶೀಲಿಸಬೇಕು. ಕೇವಲ ಮರುಕದ ಆಧಾರದಿಂದಲೇ ಒಂದು ನಿರ್ಣಯಕ್ಕೆ ಬರಬಾರದು.

ಶ‍್ರೀಕೃಷ್ಣ, ಅರ್ಜುನನಿಗೆ ನೀನೊಳ್ಳೆಯ ವಿದ್ವಾಂಸನಂತೆ ಮಾತನಾಡುತ್ತಿರುವೆ ಎನ್ನುವನು. ಅರ್ಜುನ ತುಂಬಾ ಕಷ್ಟಪಟ್ಟು ಒಂದು ವಾದಜಾಲವನ್ನು ನೇಯ್ದಿದ್ದ. ಹತ್ಯದಿಂದ ಪಾಪ ಬರುವುದು, ಇದರಿಂದ ಕುಲಧರ್ಮ ನಾಶವಾಗುವುದು, ಕುಲಸ್ತ್ರೀಯರು ಕೆಡುವರು, ವರ್ಣಸಂಕರವಾಗುವುದು, ಪಿತೃಗಳಿಗೆ ಪಿಂಡ ಕೊಡುವವರೇ ಇಲ್ಲವಾಗುವರು. ಇದಕ್ಕೆಲ್ಲಾ ಕಾರಣನಾಗುವವನು ಖಂಡಿತ ನರಕಕ್ಕೆ ಹೋಗುವನು ಎಂದು ವಾದಿಸುವನು. ಶ‍್ರೀಕೃಷ್ಣ ಇದೆಲ್ಲಾ ಕೆಲಸಕ್ಕೆ ಬಾರದ ತರ್ಕ ಎಂದು ಒಂದೇ ಸಲ ಹರಿದು ಹಾಕುವನು. ಅರ್ಜುನ ಯುದ್ಧ ಮಾಡಿದರೆ ಇದೆಲ್ಲ ಆಗುವುದೆಂದು ಭಾವಿಸಿದ್ದ. ಮಾಡದೆ ಇದ್ದರೆ ಏನು ಆಗುವುದು ಎಂಬುದನ್ನು ಎಂದೂ ಊಹಿಸಿರಲಿಲ್ಲ. ಜ್ಞಾನಿ ಯಾರು? ದೂರದರ್ಶಿ. ಭವಿಷ್ಯದಲ್ಲಿ ಹೇಗೆ ಪರಿಣಾಮ ಉಂಟಾಗುವುದು ಎಂಬ ದೃಷ್ಟಿಯಿಂದ ಕೆಲಸಮಾಡುವವನು. ಯುದ್ಧ ಮಾಡದೆ ಹೋದರೆ ಕೌರವರಿಗೆ ಮತ್ತಷ್ಟು ಧೈರ್ಯ ಬರುವುದು. ಅನ್ಯಾಯದ ಕೆಲಸಗಳನ್ನು ಮಾಡುವುದಕ್ಕೆ ಪ್ರಪಂಚದಲ್ಲಿ ಯಾರಿಗಾದರೂ ಬಲವಿದ್ದರೆ, ಅವರೆಣಿಸಿದಂತೆಯೇ ಆಗುವಂತಿದ್ದರೆ, ಮತ್ತು ಯಾರೂ ಅವರನ್ನು ವಿರೋಧಿಸದಿದ್ದರೆ ಸಮಾಜದಲ್ಲಿ ಬೆಲೆ ಯಾವುದಕ್ಕೆ–ಕೇವಲ ಮೃಗೀಯ ಶಕ್ತಿಗೆ. ಸತ್ಯ ಧರ್ಮ ಇವುಗಳನ್ನೆಲ್ಲ ಗಾಳಿಗೆ ತೂರಬೇಕಾಗುವುದು. ಒಂದು ಮೃಗೀಯ ಗುಂಪಿಗೆ ಮನುಷ್ಯ ಇಳಿಯುವನು. ಆಸುರೀ ಶಕ್ತಿ ತಾಂಡವವಾಡುವುದು ಪ್ರಪಂಚದಲ್ಲಿ. ಅರ್ಜುನ ಕಾಲವಾದ ಮೇಲೆ ಬರುವ ನರಕಕ್ಕೆ ಅಂಜುತ್ತಿರುವನು. ಬದುಕಿರುವಾಗಲೇ ಈ ಭೂಮಿಯೆ ನರಕ ಸದೃಶವಾಗುವುದು. ಇದಕ್ಕೆಲ್ಲ ಕಾರಣ ಯಾವುದು? ದಂಡನೆಗೆ ಅರ್ಹರಾದವರನ್ನು ದಂಡಿಸದೆ ಬಿಡುವುದು.

ಪಂಡಿತ ಸತ್ತವರಿಗಾಗಲಿ ಬದುಕಿರುವವರಿಗಾಗಲಿ ಶೋಕಿಸುವುದಿಲ್ಲ. ಪಂಡಿತ ಎಂದರೆ ಬರೀ ಶ್ಲೋಕಗಳನ್ನು ಕಂಠಪಾಠ ಮಾಡಿಕೊಂಡವನಲ್ಲ. ಅದರ ಅರ್ಥವನ್ನು ಚೆನ್ನಾಗಿ ಗ್ರಹಿಸಿರುವವನು. ಪಾಂಡವರಿಗೆ ವಿರೋಧವಾಗಿ ನಿಂತು ಹೋರಾಡುತ್ತಿರುವವರು ಸತ್ತರೆ ಅದಕ್ಕೆ ಶೋಕಿಸಬಾರದು. ಅನ್ಯಾಯದ ಪರಿಣಾಮವೆ ಇದು. ಮೊದಲು ನಮ್ಮ ಸಮಾನ ಇಲ್ಲ ಎಂದು ಮೆರೆದರು. ಯಾವುದಕ್ಕಾಗಿ ಅಷ್ಟೊಂದು ಅನ್ಯಾಯವನ್ನು ಮಾಡಿದ್ದರೊ ಅದನ್ನು ಬಿಟ್ಟು ಹೋಗುವಾಗ ಅದು ಅವರ ಹಿಂದೆ ಬರುವುದಿಲ್ಲ. ಅವರೂ ನಾಶವಾಗುವರು. ತಮಗೆ ಸಹಾಯ ಮಾಡಿದ್ದವರನ್ನೆಲ್ಲಾ ನಾಶ ಮಾಡುವರು. ಇದು ಅನಿವಾರ್ಯ ಘಟನೆ. ಇನ್ನು ಉಳಿದವರಿಗೆ ಶೋಕಿಸಬಾರದು. ಅವರು ತೀರಿಕೊಂಡ ಮೇಲೆ, ಅವರ ಹೆಂಡತಿ ಮಕ್ಕಳು ಮತ್ತು ಅವರನ್ನು ನೆಚ್ಚಿದ್ದ ಇತರರು ಕಷ್ಟಕ್ಕೆ ಸಿಕ್ಕಿಕೊಂಡು ನರಳುವರು. ಒಬ್ಬ ಅನ್ಯಾಯ ಮಾಡುವವನನ್ನು ಕೈಹಿಡಿದರೆ, ಅವನನ್ನು ಹೆತ್ತರೆ, ಅವನ ಒಡನಾಡಿಗಳಾದರೆ, ಈ ವ್ಯಥೆಯನ್ನು ನಾವು ಸಹಿಸಲೇಬೇಕಾಗಿದೆ. ಒಬ್ಬ ಕೊಲೆಪಾತಕ ಯಾರನ್ನೋ ಕೊಂದಿದ್ದಾನೆ. ಅವನಿಗೆ ಶಿಕ್ಷೆ ಕೊಡುವಾಗ, ಅವನಿಗೆ ಹೆಂಡತಿ ಮಕ್ಕಳು ಇದ್ದಾರೆ, ಅವನನ್ನು ಶಿಕ್ಷೆಗೆ ಗುರಿ ಮಾಡಿದರೆ, ಅವರು ಕಷ್ಟಕ್ಕೆ ಬೀಳುತ್ತಾರೆ ಎಂದು ಅವನನ್ನು ಶಿಕ್ಷಿಸದೆ ಬಿಡಬಹುದೆ? ಒಂದು ಮನೆ ಹಿತವನ್ನು ನೋಡಿಕೊಂಡು ಒಬ್ಬ ಸಮಾಜಕಂಟಕನಿಗೆ ಪ್ರೋತ್ಸಾಹ ಮಾಡುವುದೆ? ಒಂದು ಮನೆ ಉಳಿಸುವುದಕ್ಕಾಗಿ ಊರಿಗೆಲ್ಲಾ ಬೆಂಕಿ ಹಾಕುವುದೆ? ಅದೂ ಉಳಿಸಲು ಪ್ರಯತ್ನಿಸುವ ಮನೆ ಒಬ್ಬ ನಿರಪರಾಧಿಯದಲ್ಲ, ಕಂಡಂತೆಯೇ ಅನ್ಯಾಯ ಮಾಡಿರುವವನದು. ಪ್ರಪಂಚದಲ್ಲಿ ಎಲ್ಲರೂ ಸುಖಪಡುವುದಕ್ಕಾಗುವು ದಿಲ್ಲ. ಕೆಲವರು ಸುಖ ಕೆಲವರು ದುಃಖಪಡಬೇಕಾಗಿದೆ. ಪ್ರತಿಯೊಬ್ಬನೂ ಅವನಿಗೆ ಬರುವುದನ್ನು ಅವನೇ ಸಂಪಾದಿಸಿದ್ದಾನೆ. ದುರ್ಯೋಧನಾದಿಗಳು ತಮ್ಮ ನಾಶಕ್ಕೆ ತಾವೇ ಕಾಲು ಕೆರೆಯುತ್ತಿರುವರು. ಅವರ ಕೈಹಿಡಿದವರು ವಿಧವೆಯರಾಗಬಹುದು, ನೆಚ್ಚಿನ ಮಕ್ಕಳು ತಬ್ಬಲಿಗಳಾಗಬಹುದು, ಇವುಗಳನ್ನೆಲ್ಲಾ ಅವರೇ ಮಾಡಿಕೊಳ್ಳುತ್ತಿರುವರು. ಒಬ್ಬ ಬಿತ್ತಿದಂತೆ ಫಲ ಅನುಭವಿಸಬೇಕಾಗಿದೆ. ಅದಕ್ಕಾಗಿ ಅರ್ಜುನ ಅತ್ತರೆ ಬಂದದ್ದೇನು ಎನ್ನುವನು ಶ‍್ರೀಕೃಷ್ಣ.

\begin{shloka}
ನ ತ್ವೇವಾಹಂ ಜಾತು ನಾಸಂ ನ ತ್ವಂ ನೇಮೇ ಜನಾಧಿಪಾಃ~।\\ನ ಚೈವ ನ ಭವಿಷ್ಯಾಮಃ ಸರ್ವೇ ವಯಮತಃ ಪರಮ್ \hfill॥ ೧೨~॥
\end{shloka}

\begin{artha}
ನಾನು ಯಾವ ಕಾಲದಲ್ಲಿ ಇರಲಿಲ್ಲ ಎಂಬುದಿಲ್ಲ. ನೀನಾಗಲೀ ಈ ರಾಜರುಗಳಾಗಲೀ ಇರ\-ಲಿಲ್ಲವೆಂಬುದಿಲ್ಲ. ನಾವು ಯಾರೂ ಮುಂದೆ ಇರುವುದಿಲ್ಲ ಎಂಬುದಿಲ್ಲ.
\end{artha}

ಹಿಂದೂಧರ್ಮದಲ್ಲಿರುವ ಒಂದು ಮುಖ್ಯವಾದ ಭಾವನೆಗೆ ಇಲ್ಲಿ ಬರುತ್ತೇವೆ. ಅದೇ ನಾವೆಲ್ಲ ಹಿಂದೆ ಇದ್ದೆವು, ಮತ್ತೆ ಮುಂದೆ ಇರುತ್ತೇವೆ ಎಂಬುದು. ನಾವು ತುಂಬಾ ಹಳಬರು ಈ ಪ್ರಪಂಚಕ್ಕೆ. ಹಿಂದೆ ಎಷ್ಟೋ ವೇಳೆ ಜನ್ಮಗಳನ್ನು ಧರಿಸಿ ಹಲವು ಸುಖ ದುಃಖಗಳನ್ನು ಅನುಭವಿಸಿದ್ದೇವೆ. ಅದರಂತೆಯೇ ನಾವೇನೂ ಒಂದೇ ಸಲ ಸತ್ತಾಗ ಹೋಗಿಬಿಡುವವರಲ್ಲ. ಇನ್ನು ಮುಂದೆ ಹಲವು ಜನ್ಮಗಳ ಪಾತ್ರ ಧರಿಸುತ್ತೇವೆ, ಹಲವು ಸುಖ ದುಃಖಗಳನ್ನು ಅನುಭವಿಸುತ್ತೇವೆ. ಭಗವಂತನಲ್ಲಿ ಒಂದಾಗುವವರೆಗೆ ನಾವು ಹಲವು ವೇಳೆ ಬರುತ್ತಾ ಹೋಗುತ್ತಾ ಇರುತ್ತೇವೆ. ಬದ್ಧ ತನ್ನ ಕರ್ಮವನ್ನು ಸಮೆಸುವುದಕ್ಕಾಗಿ ಬರುತ್ತಾನೆ. ಶ‍್ರೀಕೃಷ್ಣನಂತಹ ಪೂರ್ಣಾತ್ಮನಾದರೊ ಹಲವರಿಗೆ ದಾರಿ ತೋರಲು ಅವತಾರವೆತ್ತಿ ಬರುತ್ತಾನೆ. ಅವನು ಹಿಂದೆ ಎಷ್ಟೋ ವೇಳೆ ಬಂದಿದ್ದಾನೆ. ಈಗ ಕುರುಕ್ಷೇತ್ರದಲ್ಲಿ ಅರ್ಜುನನ ಸಾರಥಿಯಾಗಿ ಇದ್ದಾನೆ. ಮುಂದೆ ಅವನು ಎಷ್ಟೋ ಸಾರಿ ಬರುವವನೂ ಆಗಿರುವನು. ಬದ್ಧಜೀವಿ ವಿಧಿಯಿಲ್ಲದೆ ತನ್ನ ಕರ್ಮವನ್ನು ಸಮೆಸುವುದಕ್ಕಾಗಿ ಬರುತ್ತಾನೆ. ದೇವರಾದರೋ ಇತರರನ್ನು ಮೇಲೆತ್ತಲು ಕೆಳಗೆ ಇಳಿಯಬೇಕಾಗಬಹುದು.

ಆದರೆ ಹಲವರು ಈಗಿನ ಜನ್ಮವನ್ನು ಒಪ್ಪಿಕೊಳ್ಳುತ್ತಾರೆ, ಹಿಂದಿನದನ್ನು ಒಪ್ಪಿಕೊಳ್ಳುವುದಿಲ್ಲ. ನಮಗೆ ಒಂದು ಹಿಂದಿನ ಜನ್ಮವಿದ್ದರೆ ಹಿಂದೆ ನೀವು ಏನಾಗಿದ್ದಿರಿ ಹೇಳಿ ಎಂದು ಕೇಳುವರು. ಬರೀ ನೆನಪಿನ ಆಧಾರದ ಮೇಲೆಯೆ ಒಂದು ವಸ್ತು ಇರುವುದು ಸತ್ಯವಾದರೆ ನಾವು ಸಣ್ಣ ಮಕ್ಕಳಾಗಿದ್ದಾಗ ಏನೇನು ಮಾಡಿದ್ದೆವೊ ಅವೆಲ್ಲಾ ಮರೆತುಹೋಗಿದೆ. ಹಾಗಾದರೆ ನಾವು ಮಕ್ಕಳಾಗಿಯೇ ಇರಲಿಲ್ಲವೆ? ಈಗಲೂ ಕೆಲವು ವರುಷಗಳ ಹಿಂದೆ ನಾವು ಏನೇನು ಮಾಡಿದೆವೋ ಅದರಲ್ಲೆ ಕೆಲವು ವಿನ ಉಳಿದವುಗಳೆಲ್ಲಾ ಮರೆತುಹೋಗಿದೆ. ಆದರೆ ಈ ಘಟನೆಗಳ ಸಾರವಾದ ಸಂಸ್ಕಾರ ನಮ್ಮ ಮನಸ್ಸಿನಲ್ಲಿ ವಾಸನೆಯ ರೂಪದಂತೆ ಇರುವುದು. ಅದು ನಾವು ಈಗ ಮಾಡುವ ಕೆಲಸ ಆಲೋಚನೆ ಎಲ್ಲದರ ಮೇಲೆಯೂ ತನ್ನ ಪ್ರಭಾವವನ್ನು ಬೀರುತ್ತಿರುವುದು. ಯಾರಿಗಾದರೂ ಹಿಂದಿನದನ್ನು ಸ್ಮರಿಸಿಕೊಳ್ಳ ಬೇಕೆಂದು ಮನಸ್ಸಾದರೆ ತಮ್ಮ ಮನಸ್ಸಿನಲ್ಲಿರುವ ಸಂಸ್ಕಾರದ ಮೇಲೆ ಮನಸ್ಸನ್ನು ಏಕಾಗ್ರ ಮಾಡಿದರೆ ಆಗ ಅದಕ್ಕೆ ಸಂಬಂಧಪಟ್ಟ ವಿವರಗಳೆಲ್ಲ ವ್ಯಕ್ತವಾಗುವುವು. ಇವನನ್ನೆ ಜಾತಿಸ್ಮರ ಎಂದು ಯೋಗಿಗಳು ಹೇಳುವರು. ಆದರೆ ನಾವು ಪೂರ್ಣತೆಯನ್ನು ಮುಟ್ಟುವುದಕ್ಕೆ ಮುಂಚೆ ಅದನ್ನು ತಿಳಿದುಕೊಳ್ಳುವುದು ಬಹಳ ಅಪಾಯಕರ. ಅದಕ್ಕಾಗಿಯೇ ದೇವರು ವಿವರಗಳನ್ನೆಲ್ಲಾ ಅಳಿಸಿರುವನು, ಕೇವಲ ಸಂಸ್ಕಾರಗಳನ್ನು ನಮ್ಮ ಮನಸ್ಸಿನಲ್ಲಿಟ್ಟಿರುವನು. ಈ ಮರೆವನ್ನು ದೇವರು ಕೃಪೆಯಿಂದ ನಮ್ಮ ಮೇಲೆ ಬೀರಿರುವನು. ಈ ಒಂದು ಜನ್ಮದಲ್ಲಿ ಮಾಡಿರುವ ಕೆಲಸಗಳು, ಅನುಭವಿಸಿರುವ ಅನುಭವಗಳು ಇದರ ಹೊರೆಯನ್ನೆ ತಾಳಲಾರದೆ ಕುಸಿದು ಬೀಳುವ ನಮಗೆ, ಹೋದ ಜನ್ಮಗಳ ನೆನಪೆಲ್ಲಾ ಒಂದೇ ಸಲ ಬಂತು ಎಂದರೆ ನಾವು ಹುಚ್ಚರಾಗಿ ಹೋಗುವೆವು. ಆದರೆ ಸತ್ಯ ಸಾಕ್ಷಾತ್ಕಾರವನ್ನು ಮಾಡಿಕೊಂಡ ಧೀರರಿಗೆ ಇದು ಸಾಧ್ಯ. ಅವರು ಇದರಿಂದ ವ್ಯಸ್ತರಾಗದೆ ಸಾಕ್ಷಿಯಂತೆ ನಿಂತು ನೋಡಬಲ್ಲರು. ಆ ಸ್ಥಿತಿ ಪ್ರಾಪ್ತವಾಗುವುದಕ್ಕೆ ಮುಂಚೆ ನಮ್ಮ ಪೂರ್ವ ಜನ್ಮಗಳ ನೆನಪು ಒಂದು ಹೆಣಭಾರವಾಗುವುದು, ನಾವು ಮುಂದುವರಿಯಲಾರದೆ ಕುಸಿದು ಬೀಳುವೆವು. 

ಕೆಲವು ಧರ್ಮಗಳವರು ಒಂದೇ ಒಂದು ಜನ್ಮದ ಸಿದ್ಧಾಂತವನ್ನು ನಂಬುವರು. ಜೀವಿಗೆ ಒಂದು ಅವಕಾಶ ಮಾತ್ರ ಸಿಕ್ಕುವುದು. ಅವನು ಇಲ್ಲಿ ಏನಾದರೂ ಕೆಟ್ಟದ್ದು ಮಾಡಿದರೆ ನರಕ, ಒಳ್ಳೆಯದನ್ನು ಮಾಡಿದರೆ ಸ್ವರ್ಗ ಎನ್ನುವರು. ನಾವು ಇಲ್ಲಿ ಒಂದು ಜನ್ಮದಲ್ಲಿ ಮಾತ್ರ ಕೆಟ್ಟದನ್ನು ಮಾಡುವೆವು. ಅದಕ್ಕೆ ಎಂದೆಂದಿಗೂ ನರಕವಾಸ ಎಂಬುದು ಕುತರ್ಕವಾಗುವುದು. ಒಂದು ನಿಯಮಿತ ಕಾರಣಕ್ಕೆ \enginline{(limited cause)} ಅನಿಯಮಿತವಾದ \enginline{(unlimited)} ಪರಿಣಾಮ\break ಬಂದಂತೆ ಆಗುವುದು. ಇದು ಎಲ್ಲಾ ತಿಳಿದ ಕಾರಣಗಳ ಸಂಬಂಧಕ್ಕೂ ವಿರೋಧವಾದುದು. ಈ ಪ್ರಪಂಚದಲ್ಲಿಯೇ ಜೀವಾವಧಿ ಸಜ ಎಂದರೆ ಸುಮಾರು ಹದಿನಾಲ್ಕು ವರುಷಗಳು ಜೈಲಿನಲ್ಲಿರಬೇಕು. ಅದರಲ್ಲಿಯೂ ಚೆನ್ನಾಗಿ ನಡೆದುಕೊಂಡರೆ ವರುಷಕ್ಕೆ ಒಂದು ತಿಂಗಳಿನಂತೆ ಸೋಡಿ ಸಿಕ್ಕುವುದು. ಆದರೆ ದೇವರ ನ್ಯಾಯಾಸ್ಥಾನ ಇದಕ್ಕಿಂತ ಉಗ್ರವಾಗುವುದು. ಒಂದು ಸಲ ನರಕಕ್ಕೆ ಹೋದರೆ ಅದರಿಂದ ಪಾರಾಗಲು ಅವಕಾಶವೇ ಇಲ್ಲ. ಏಕೆಂದರೆ ಅವನಿಗೆ ಇನ್ನೊಂದು ಜನ್ಮದ ಅವಕಾಶವೇ ಇಲ್ಲ. ಒಂದೇ ಒಂದು ಜನ್ಮದಲ್ಲಿ ಒಬ್ಬ ಪೂರ್ಣನಾಗುವುದಕ್ಕೆ ಆಗುವುದಿಲ್ಲ. ಅವನು ಎಷ್ಟೋ ಒಳ್ಳೆಯದನ್ನು ಮಾಡಿರುತ್ತಾನೆ, ಪುಣ್ಯಾತ್ಮನಾಗುವಷ್ಟು ಮಾಡಿರುವುದಿಲ್ಲ. ಅವನು ಅಷ್ಟೊಂದು ಪಾಪವನ್ನೂ ಮಾಡಿರುವುದಿಲ್ಲ. ಅವನು ಸ್ವರ್ಗಕ್ಕೆ ಹೋಗುವುದು ಹೇಗೆ? ಅವನು ಸ್ವಲ್ಪ ಒಳ್ಳೆಯದನ್ನು ಮಾಡಿದ್ದಾನೆ ಎನ್ನುವರು. ಇಲ್ಲಿಯೂ ನಿಯಮಿತ ಕಾರಣಕ್ಕೆ ಅನಂತ ಪರಿಣಾಮದ ದೋಷವೇ ಬರುವುದು. ಒಂದು ಎಳೆಯ ಮಗು ಏನೂ ತಪ್ಪು ಮಾಡಿರುವುದಿಲ್ಲ. ಅದು ಹುಟ್ಟುತ್ತಲೇ ಸಾಯುತ್ತದೆ. ಅದಕ್ಕೇನು ಗತಿ, ಎಲ್ಲಿಗೆ ಹೋಗುತ್ತದೆ ಎಂದು ಕೇಳಿದರೆ, ಆ ಮಗು ಎಳೆಯ ಹಸುಳೆ, ಪಾಪವನ್ನೇ ಕಂಡರಿಯದು, ಅದು ಸ್ವರ್ಗಕ್ಕೆ ಹೋಗುವುದು ಎಂದು ಹೇಳುವರು. ಪಾಪವನ್ನೇ ಮಾಡಿಲ್ಲ ಎಂದರೆ ಪುಣ್ಯವನ್ನು ಮಾಡಿದಂತೆ ಆಯಿತೆ? ಅದಕ್ಕೆ ಸ್ವರ್ಗಪ್ರಾಪ್ತಿಯಾಗುವ ಹಾಗೆ ಇದ್ದರೆ ಅಂತಹ ಹಸುಳೆಗಳನ್ನು ಸಾಯಿಸುವುದೇ ಮೇಲೆಂದು ಕೆಲವರು ವಾದಿಸುವರು. ಏಕೆಂದರೆ ಬೆಳೆದು ದೊಡ್ಡದಾದರೆ ಪುಣ್ಯಕ್ಕಿಂತ ಹೆಚ್ಚಾಗಿ ಪಾಪವನ್ನು ಮಾಡುವ ಸಂಭವವೇ ಹೆಚ್ಚು. ಆದಕಾರಣ ಅದಕ್ಕೆ ಮುಂಚೆಯೇ ಸದ್ಗತಿ ಸಿಕ್ಕುವಂತೆ ಮಾಡುವುದು ಪುಣ್ಯ ಕೆಲಸವಾಗುವುದಿಲ್ಲವೆ? ಆ ಮಗುವಿಗೇನೋ ಸ್ವರ್ಗ ಬರುವುದು. ಆದರೆ ಯಾರು ಅದನ್ನು ಕೊಲ್ಲುತ್ತಾರೋ ಅವರು ನರಕಕ್ಕೆ ಹೋಗುತ್ತಾರೆ ಎಂದರೆ, ಒಬ್ಬನಿಗೆ ಸ್ವರ್ಗಸಿಕ್ಕುವಂತಹ ಕೆಲಸಕ್ಕಿಂತ ಪುಣ್ಯಕೆಲಸ ಯಾವುದಿದೆ? ಅಂತಹ ಪುಣ್ಯಕೆಲಸ ಮಾಡಿದವನಿಗೆ ನರಕವೆ? ಇದೆಂತಹ ಅನ್ಯಾಯ? ಪ್ರತಿಯೊಂದು ಕ್ರಿಯೆಯನ್ನು ಅದರ ಫಲದ ಮೂಲಕ ಅಳೆಯಬೇಕು.

\newpage

ಒಬ್ಬನಿಗೆ ಹಿಂದಿನ ಜನ್ಮವೇ ಇಲ್ಲ, ಇದೇ ಪ್ರಥಮ ಬಾರಿ ಪ್ರಪಂಚಕ್ಕೆ ಬರುವ ಹಾಗಿದ್ದರೆ ಒಬ್ಬರು ಬುದ್ಧಿವಂತರು, ಒಬ್ಬರು ದಡ್ಡರು, ಒಬ್ಬರು ಅಂಗಹೀನರು ಮತ್ತು ಒಬ್ಬ ಸರ್ವಾಂಗ ಸುಂದರನು ಇದಕ್ಕೆಲ್ಲಾ ಕಾರಣ ಯಾರು? ದೇವರು ಎಂದರೆ ಅವನು ಪಕ್ಷಪಾತಿಯಾಗುತ್ತಾನೆ. ಅವನು ಏತಕ್ಕೆ ಒಬ್ಬನಿಗೆ ಅಷ್ಟೊಂದು ಬುದ್ಧಿಯನ್ನು ಕೊಟ್ಟು ಕಳುಹಿಸಬೇಕು, ಮತ್ತೊಬ್ಬನನ್ನು ದಡ್ಡನಾಗಿ ಮಾಡಬೇಕು? ಇಂತಹ ಪಕ್ಷಪಾತವಿದ್ದರೆ ಅವನು ಸಮದರ್ಶಿಯಾದ ದೇವರು ಹೇಗೆ ಆಗುತ್ತಾನೆ? ಹಿಂದು ಇಲ್ಲಿ ಪ್ರತಿಯೊಂದು ಜೀವಿಗಳಿಗೂ ಅವರವರ ಕರ್ಮಫಲಗಳ ಅನುಸಾರ ಇವುಗಳೆಲ್ಲಾ ಪ್ರಾಪ್ತವಾಗುವುದು ಎಂದು ದೇವರನ್ನು ದೂರುವುದಕ್ಕೆ ಹೋಗುವುದಿಲ್ಲ. ಒಂದು ಜನ್ಮವನ್ನು ನಂಬುವವರಾದರೊ ದೇವರನ್ನು ಪಕ್ಷಪಾತದ ತಪ್ಪಿನಿಂದ ಪಾರುಮಾಡುವುದಕ್ಕಾಗಿ ಇನ್ನೊಂದು ವಿವರಣೆಯನ್ನು ಕೊಡುವರು. ಅದೇ ತಂದೆತಾಯಿಗಳು ಮಾಡಿದ ಪಾಪಕ್ಕೆ ತಕ್ಕ ಶಿಕ್ಷೆ; ಅಂತಹ ದಡ್ಡರು ಮತ್ತು ಅಂಗಹೀನರಾದ ಮಕ್ಕಳು ಎನ್ನುವರು. ಆದರೆ ಇಲ್ಲಿ ಅನುಭವಿಸುವವರು ತಾಯಿ ತಂದೆಗಳಲ್ಲ, ಅವರು ಬೇಗ ಕಣ್ಣು ಮುಚ್ಚಿಕೊಂಡು ಹೋಗಿಬಿಡುವರು. ಆದರೆ ಮಗು, ಇನ್ನಾರೋ ಮಾಡಿದ ತಪ್ಪಿಗೆ ತಾನು ಶಿಕ್ಷೆಯನ್ನು ಜೀವಾವಧಿ ಅನುಭವಿಸಬೇಕಾಗುವುದು. ಇದೆಂತಹ ನ್ಯಾಯ? ಯಾರೋ ಮಾಡಿದ ತಪ್ಪಿಗೆ ಯಾರಿಗೋ ಶಿಕ್ಷೆ. ಈ ಪ್ರಪಂಚದ ನ್ಯಾಯಸ್ಥಾನದಲ್ಲಿಯೇ ಅಪ್ಪ ಮಾಡಿದ ತಪ್ಪಿಗೆ ಮಗನನ್ನು ಶಿಕ್ಷಿಸುವುದಿಲ್ಲ. ಸ್ವರ್ಗದಲ್ಲಿರುವ ದೇವರ ನ್ಯಾಯ ಇದಕ್ಕಿಂತಲೂ ವಿಚಿತ್ರವಾಯಿತಲ್ಲ!

ಒಂದೊಂದು ಮಗು ಒಂದೊಂದು ರೀತಿ ಏತಕ್ಕೆ ಬೆಳೆಯುತ್ತಾ ಹೋಗುತ್ತದೆ? ಅದಕ್ಕೆಲ್ಲ ಸುತ್ತಮುತ್ತಲಿರುವ ವಾತಾವರಣ ಮತ್ತು ತರಬೇತು ಎನ್ನುವರು. ವಾತಾವರಣ ಮತ್ತು ತರಬೇತಿನಲ್ಲಿ ಸ್ವಲ್ಪ ಸತ್ಯವೇನೋ ಇದೆ. ಆದರೆ ಪೂರ್ಣ ಸತ್ಯವಲ್ಲ. ಅನೇಕ ವೇಳೆ ಒಂದೇ ಮನೆಯಲ್ಲಿ ಒಬ್ಬ ತಂದೆತಾಯಿಗಳಿಗೆ ಆದ ಮಕ್ಕಳು ಒಂದೇ ವಾತಾವರಣದಲ್ಲಿ ಬೆಳೆಯುತ್ತ ಇದ್ದರೂ ಒಬ್ಬೊಬ್ಬರು ಒಂದೊಂದು ರೀತಿ ಆಗುವರು. ಇದಕ್ಕೆ ಕಾರಣವೇನು? ಎಲ್ಲೊ ಹೊರಗೆ ಕಾರಣವನ್ನು ಹುಡುಕುವುದು ವೈಜ್ಞಾನಿಕವಲ್ಲ. ಕಾರಣ ಒಳಗಿನಿಂದಲೇ ಬರಬೇಕು. ಆದಕಾರಣವೆ ಹಿಂದುಧರ್ಮ, ಪ್ರತಿಯೊಂದು ಜೀವಿಯೂ ಹುಟ್ಟುವಾಗ ತನ್ನ ಸಂಸ್ಕಾರದ ಬೀಜವನ್ನು ತಂದಿರುವುದು; ಅದು ವಾತಾವರಣದಲ್ಲಿ ತನ್ನ ಬೆಳವಣಿಗೆಗೆ ಬೇಕಾದುದನ್ನು ಮಾತ್ರ ಹೀರುವುದು, ಉಳಿದುದನ್ನು ತ್ಯಜಿಸುವುದು ಎಂದು ಹೊರೆ ಹೊಣೆಯನ್ನೆಲ್ಲ ಪ್ರತಿಯೊಂದು ಜೀವಿಯ ಪಾಲಿಗೆ ಬಿಡುವುದು. ಪ್ರತಿಯೊಬ್ಬನೂ ತನ್ನ ಅದೃಷ್ಟಕ್ಕೆ ತಾನೆ ಹೊಣೆ ಎಂದು ತಿದ್ದಿಕೊಳ್ಳಲು ಯತ್ನಿಸುವನು. ಸುಮ್ಮನೆ ಹೆತ್ತವರನ್ನೋ ವಾತಾವರಣವನ್ನೋ ದೂರುತ್ತಿರುವುದಿಲ್ಲ. ಜೀವಿಯು ಈ ಜನ್ಮದ ಅವಕಾಶವನ್ನು ತೆಗೆದುಕೊಂಡು ಇಲ್ಲಿ ಪುಣ್ಯವೊ ಪಾಪವೊ ಯಾವುದನ್ನೋ ಮಾಡಿ, ಮತ್ತೊಂದು ಜನ್ಮದಲ್ಲಿ ಅದನ್ನು ತಿದ್ದಿಕೊಳ್ಳುತ್ತ ಹೋಗುವುದು. ಒಂದೇ ಜನ್ಮದ ಅವಕಾಶ ಸಾಲದು. ಇದರಲ್ಲಿ ನಾವು ಸ್ವರ್ಗಕ್ಕೆ ಹೋಗುವಷ್ಟು ಪುಣ್ಯವನ್ನು ಮಾಡುವುದಕ್ಕೆ ಆಗುವುದಿಲ್ಲ. ಅಥವಾ ನರಕದಿಂದ ತಪ್ಪಿಸಿಕೊಂಡು ಹೋಗುವಷ್ಟು ಪ್ರಯತ್ನವನ್ನು ಮಾಡುವುದಕ್ಕೆ ಆಗುವುದಿಲ್ಲ. ಇದೇ ಪೂರ್ಣ ಸತ್ಯ ಎಂದು ನಾವು ಹೇಳುವುದಿಲ್ಲ. ಇದಕ್ಕಿಂತ ಉತ್ತಮವಾದ ವಿವರಣೆ ಕೊಟ್ಟರೆ ಅದನ್ನು ಸ್ವೀಕರಿಸಲು ನಾವು ಸಿದ್ಧವಾಗಿದ್ದೇವೆ. ನಮ್ಮ ಪಾಪ ಪುಣ್ಯಗಳಿಗನುಸಾರ ನಾವು ಯಾವುದೊ ಒಂದು ಕಡೆ ಹುಟ್ಟುತ್ತೇವೆ. ನಮ್ಮ ಕರ್ಮಾನುಸಾರ ಈ ಜನ್ಮದಲ್ಲಿ ಅನುಭವಿಸುತ್ತೇವೆ. ಇಲ್ಲಿ ಮಾಡಿರುವುದನ್ನು ಮುಂದಿನ ಜನ್ಮದಲ್ಲಿಯೂ ಮುಂದುವರಿಸಿಕೊಂಡು ಹೋಗುತ್ತೇವೆ ಎಂಬ ಕರ್ಮಸಿದ್ಧಾಂತ ನಮ್ಮ ಅದೃಷ್ಟಕ್ಕೆ ನಮ್ಮನ್ನು ಹೊಣೆಗಾರರನ್ನಾಗಿ ಮಾಡುವುದು. ಅಧೋಗತಿಗೆ ಬಿದ್ದಿರುವುದಕ್ಕೆ ನಾನೆ ಕಾರಣ, ಊರ್ಧ್ವ ಗತಿಯ ಕಡೆ ಹೋಗಬೇಕಾ ದರೂ ನಾನೆ ಪ್ರಯತ್ನ ಮಾಡಬೇಕಾಗಿದೆ ಎಂಬುದು ಹೆಚ್ಚು ಆಶಾಜನಕವಾಗಿದೆ. ಇಲ್ಲಿ ಜೀವ ತನ್ನ ಈಗಿನ ಸ್ಥಿತಿಗೆ ತಾನೇ ಜವಾಬ್ದಾರಿಯನ್ನು ತೆಗೆದುಕೊಳ್ಳುವನು.

\begin{shloka}
ದೇಹಿನೋsಸ್ಮಿನ್ ಯಥಾ ದೇಹೇ ಕೌಮಾರಂ ಯೌವನಂ ಜರಾ~।\\ತಥಾ ದೇಹಾಂತರಪ್ರಾಪ್ತಿರ್ಧೀರಸ್ತತ್ರ ನ ಮುಹ್ಯತಿ \hfill॥ ೧೩~॥
\end{shloka}

\begin{artha}
ಜೀವನಿಗೆ ಈ ದೇಹದಲ್ಲಿ ಬಾಲ್ಯ, ಯೌವನ, ವೃದ್ಧಾಪ್ಯ ಹೇಗೋ ಹಾಗೆಯೇ ಬೇರೆ ದೇಹವೂ ಪ್ರಾಪ್ತವಾಗುವುದು. ಹೀಗಿರುವಾಗ ಧೀರ ಇಲ್ಲಿ ಮೂಢನಾಗುವುದಿಲ್ಲ.
\end{artha}

ಜೀವಿ ಯಾವಾಗ ದೇಹವನ್ನು ಧರಿಸುತ್ತಾನೆಯೊ ಆಗ ಹುಟ್ಟಿನಿಂದ ಸಾವಿನ ತನಕ ಹಲವು ಅವಸ್ಥೆಗಳ ಮೂಲಕ ಸಾಗಿಹೋಗುವನು. ಯಾವ ಒಂದು ಅವಸ್ಥೆಯಲ್ಲಿಯೆ ಬಹಳ ಕಾಲ ಇರಲಾರ. ಎಷ್ಟೇ ಅಂಟಿಕೊಂಡಿದ್ದರೂ ಅವನು ಮುಂದಿನ ಅವಸ್ಥೆಗೆ ಹೋಗಲೇಬೇಕಾಗುವುದು. ಬಾಲ್ಯ ಆಟಪಾಟಗಳಲ್ಲಿ ಕಳೆಯುತ್ತದೆ. ಯೌವನ ಪ್ರಾಪ್ತವಾದಾಗ ಹಲವಾರು ಆಸೆ ಆಕಾಂಕ್ಷೆಗಳು ಅವನನ್ನು ಮುತ್ತುತ್ತವೆ. ಅವುಗಳಲ್ಲಿ ಕೆಲವನ್ನು ಕೂಡ ಈಡೇರಿಸಿಕೊಳ್ಳುವುದಕ್ಕೆ ಕಾಲ ಸಿಗುವುದಿಲ್ಲ, ಅಷ್ಟು ಹೊತ್ತಿಗೆ ಆಗಲೆ ನಮ್ಮ ದೇಹದ ಅಂಗಾಂಗಗಳು ಸಮೆಯುತ್ತ ಬರುವುದು. ಒಂದಾದ ಮೇಲೊಂದು ಖಾಯಿಲೆಗಳು ಧಾಳಿ ಇಡುವುವು ವೃದ್ಧಾಪ್ಯದಲ್ಲಿ. ದೇಹ ನಶಿಸಿ ಸೊರಗಿ ಬೀಳುವುದು. ಇದನ್ನು ನಾವು ಕಣ್ಣಾರ ನೋಡುತ್ತಿರುವೆವು.

ಇದರಂತೆಯೇ ಈ ದೇಹವನ್ನು ತ್ಯಜಿಸಿ ಆದಮೇಲೂ ನಾವೇನೂ ಇಲ್ಲದೆ ಹೋಗುವುದಿಲ್ಲ. ಏಕೆಂದರೆ ಯಾವುದನ್ನೂ ನಾವು ನಾಶಮಾಡಲು ಆಗುವುದಿಲ್ಲ. ಅದರ ರೂಪವನ್ನು ಬೇಕಾದಂತೆ ಬದಲಾಯಿಸಬಹುದು. ನೀರು ದ್ರವರೂಪದಲ್ಲಿದೆ, ಘನೀರೂಪ ತಾಳುತ್ತದೆ, ಅಥವಾ ಆವಿಯಾಗಿ ಮೇಲೆದ್ದು ಹೋಗುವುದು. ಆವಾಗ ಅದೇನೂ ಇಲ್ಲದೆ ಇಲ್ಲ. ಆ ನೀರಿನ ಅಂಶ ಪುನಃ ಎಲ್ಲೋ ತುಂತುರು ಹನಿಯಾಗಿ ಬೀಳುವುದು. ಚೇತನವೂ ಹೀಗೆಯೆ. ಬದುಕಿರುವಾಗ ಆ ಚೈತನ್ಯ ದೇಹದ ಮೂಲಕ ಕೆಲಸ ಮಾಡುತ್ತಿರುವುದು. ಆ ದೇಹ ಇನ್ನು ಮೇಲೆ ಬೇರೆ ಉಪಯೋಗಕ್ಕೆ ಬಾರದ ರೀತಿ ಆದರೆ ಅದರ ಹಿಂದೆ ಇರುವ ಚೈತನ್ಯವೇನು ನಾಶವಾಗುವುದಿಲ್ಲ. ಮನೆಯಲ್ಲಿ ಬಲ್ಬಿನ ಮೂಲಕ ವಿದ್ಯುತ್​ಶಕ್ತಿ ಪ್ರಕಾಶಿಸುತ್ತಿರುವುದು. ಆ ಬಲ್ಬಿನ ಒಳಗೆ ಇರುವ ತಂತಿ ಕಿತ್ತುಹೋದರೆ ದೀಪ ಉರಿಯುವುದಿಲ್ಲ. ಆದರೆ ವಿದ್ಯುತ್ ಶಕ್ತಿಯೇ ನಾಶವಾಯಿತೆ? ಇನ್ನೊಂದು ಬಲ್ಬನ್ನು ಹಾಕಿದರೆ ಆಗ ಅದು ಉರಿಯುವುದು. ಅದರಂತೆಯೇ ಚೇತನದ ಕೆಲಸ ಮಾಡಲು ಈ ದೇಹ. ಇದು ನಾಶವಾದರೆ ಇನ್ನೊಂದನ್ನು ತನ್ನ ಕರ್ಮಾನುಸಾರ ಸೃಷ್ಚಿಸಿಕೊಳ್ಳುವುದು. ಜ್ಞಾನಿಗಳು ಈ ವಿಷಯದಲ್ಲಿ ಮರುಳಾಗುವುದಿಲ್ಲ. ಹುಟ್ಟುವವರೆಲ್ಲ ಸಾಯಲೇಬೇಕು. ಸತ್ತವರೆಲ್ಲ ಪೂರ್ಣತೆಯನ್ನು ಪಡೆಯುವವರೆಗೆ ಪುನಃ ಪುನಃ ಹಲವು ದೇಹಗಳನ್ನು ಧರಿಸುವುದಕ್ಕೆ ಬರಲೇಬೇಕು. ನಾವೆಲ್ಲ ಒಂದೇ ಸಲ ಹೋಗುವಷ್ಟು ಪುಣ್ಯ ಗಳಿಸಿಕೊಂಡಿಲ್ಲ. ಒಂದು ಬಾಗಿಲಿನಿಂದ ಹೋಗುವೆವು, ಮತ್ತೊಂದು ಬಾಗಿಲಿನಿಂದ ಬರು ವೆವು. ಈಜುಗಾರ ಮೇಲಿನಿಂದ ನೀರಿನ ಒಳಕ್ಕೆ ಧುಮುಕುವನು. ಅವನು ನೀರಿನಲ್ಲಿ ಬಿದ್ದೊಡನೆ ಒಳಕ್ಕೆ ಈಜಿಕೊಂಡು ಹೋಗಿ ಸ್ವಲ್ಪ ದೂರದಲ್ಲಿ ಮೇಲೇಳುವನು. ಹಾಗೆಯೆ ನಮ್ಮ ಸಾವು ಕೂಡ. ಇಲ್ಲಿ ಕಣ್ಣು ಮುಚ್ಚಿಕೊಳ್ಳುವೆವು, ಇನ್ನೆಲ್ಲಿಯೋ ಇನ್ನು ಯಾವ ದೇಹದ ಮೂಲಕವಾಗಿಯೋ ಕಣ್ಣನ್ನು ತೆಗೆಯುವೆವು.

\begin{shloka}
ಮಾತ್ರಾಸ್ಪರ್ಶಾಸ್ತು ಕೌಂತೇಯ ಶೀತೋಷ್ಣಸುಖದುಃಖದಾಃ~।\\ಆಗಮಾಪಾಯಿನೋsನಿತ್ಯಾಸ್ತಾಂಸ್ತಿತಿಕ್ಷಸ್ವ ಭಾರತ \hfill॥ ೧೪~॥
\end{shloka}

\begin{artha}
ಕೌಂತೇಯ, ಇಂದ್ರಿಯ ಮತ್ತು ಅದಕ್ಕೆ ಸಂಬಂಧಪಟ್ಟ ವಿಷಯಗಳ ಸಂಬಂಧದಿಂದ ಶೀತ ಉಷ್ಣ, ಸುಖ ದುಃಖ ಇವುಗಳು ಉಂಟಾಗುತ್ತವೆ. ಇವುಗಳ ಸ್ವಭಾವವೆ ಬರುವುದು ಹೋಗುವುದು ಮತ್ತು ಅನಿತ್ಯ. ಆದಕಾರಣ ನೀನು ಅವುಗಳನ್ನು ಸಹಿಸಿಕೊಳ್ಳಬೇಕು.
\end{artha}

ಈ ಪ್ರಪಂಚದಲ್ಲಿ ಎಲ್ಲೊ ಸುಖ ಅಥವಾ ದುಃಖದ ರಾಶಿ ಹೊರಗೆ ಬಿದ್ದಿಲ್ಲ. ಅದನ್ನೆಲ್ಲ ಮಾಡಿಕೊಳ್ಳುವವರು ನಾವು. ನಮ್ಮ ದೇಹದಲ್ಲಿ ಪಂಚೇಂದ್ರಿಯಗಳಿವೆ. ಹೊರಗೆ ಪಂಚಭೂತಗಳಿವೆ. ಆ ಪಂಚಭೂತಗಳ ಸಂಯೋಗದಿಂದ ಹಲವಾರು ನಾಮರೂಪವುಳ್ಳ ವಸ್ತುಗಳಾಗಿವೆ.\break ಯಾವಾಗ ಆ ವಸ್ತುವಿಗೂ ನಮ್ಮ ಇಂದ್ರಿಯಕ್ಕೂ ಸಂಬಂಧ ಉಂಟಾಗುವುದೊ ಆಗಲೆ ಅನುಭವದ ಕಿಡಿ ಏಳುವುದು. ಅದು ಸುಖವಾದರೂ ಇರಬಹುದು, ದುಃಖವಾದರೂ ಇರಬಹುದು. ಅದನ್ನು ನಾವು ಹೇಗೆ ಉಪಯೋಗಿಸಿಕೊಳ್ಳುವೆವೊ ಅದರ ಮೇಲಿದೆ. ಕಿವಿ ಇದೆ. ಅದು ಒಳ್ಳೆಯದನ್ನು ಕೇಳುವಂತೆ ಕೆಟ್ಟದನ್ನೂ ಕೇಳುವುದು. ಕಣ್ಣು ಸುಂದರವಾದ ದೃಶ್ಯವನ್ನು ನೋಡುವಂತೆ ಭಯಾನಕವಾದುದನ್ನೂ ನೋಡುವುದು. ಸುವಾಸನೆಯನ್ನು ಮೂಸುವಂತೆ ಮೂಗು ದುರ್ಗಂಧವನ್ನೂ ಮೂಸುವುದು. ಹಾಗೆ ಇತರ ಇಂದ್ರಿಯಗಳು. ಚಳಿಗಾಲದಲ್ಲಿ ಬೆಂಕಿಯ ಮುಂದೆ ಕುಳಿತು ಕಾಸಿಕೊಂಡಾಗ ಹಾಯಾಗಿರುವುದು. ಅದೇ ಒಂದು ಕಿಡಿ ನಮ್ಮ ದೇಹದ ಮೇಲೆ ಬಿದ್ದಾಗ ಬರೆ ಹಾಕುವುದು. ಆಗ ನೋವಾಗುವುದು. ಈ ಅನುಭವಕ್ಕೆ ಅರ್ಧ ನನ್ನಲ್ಲಿದೆ. ಅವೇ ಇಂದ್ರಿಯಗಳು. ಇನ್ನು ಅರ್ಧ ಹೊರಗೆ ಇದೆ. ಅವೇ ವಿಷಯವಸ್ತುಗಳು. ಇವೆರಡೂ ಒಟ್ಟಾಗಿ ಸೇರಿದಾಗ ಮಾತ್ರ ಅನುಭವ.

ಈ ಅನುಭವಗಳ ಸ್ವಭಾವವೆ ಬಂದು ಹೋಗುವುದು. ಯಾವುದೂ ಶಾಶ್ವತವಾಗಿರುವುದಿಲ್ಲ. ಒಂದು ವೇಳೆ ಅದು ಬಹಳಕಾಲ ನಮ್ಮಲ್ಲಿದ್ದರೆ ನಾವು ಅದಕ್ಕೆ ಹೊಂದಿಕೊಂಡು ಹೋಗಿ ಅದನ್ನು ಸುಖವೆಂತಲೂ ಹೇಳುವುದಿಲ್ಲ, ದುಃಖವೆಂತಲೂ ಹೇಳುವುದಿಲ್ಲ. ಆ ಸ್ಥಿತಿಗೆ ಬರುವುದು. ಜೀವನದಲ್ಲಿ ನಾವು ಎಷ್ಟೊಂದು ದುಃಖವನ್ನು ಅನುಭವಿಸಿರುವೆವು. ಎಷ್ಟೊಂದು ಕಣ್ಣೀರನ್ನು ಹರಿಸಿರುವೆವು. ನಮ್ಮ ಕಣ್ಣೆದುರಿಗೇ ನಮ್ಮ ಎಷ್ಟು ಪ್ರಿಯವಾದ ವಸ್ತುಗಳು ಮಾಯವಾಗಿವೆ. ಅವುಗಳ ಅಗಲಿಕೆಯಿಂದ ಎಷ್ಟು ನೊಂದಿರುವೆವು. ಈ ದೇಹ ಮನಸ್ಸು ಬುದ್ಧಿ ಇಂದ್ರಿಯದ ಮೂಲಕ ಬೇಕಾದಷ್ಟು ಸುಖವನ್ನು ಅನುಭವಿಸಿರುವೆವು. ಆದರೆ ಯಾವುದೂ ಸ್ಥಿರವಲ್ಲ. ಈಗ ಒಂದು ಇರುವುದು. ಸ್ವಲ್ಪ ಕಾಲವಾದ ಮೇಲೆ ಅದು ಹೋಗುವುದು, ಮತ್ತೊಂದು ಬರುವುದು. ಒಂದು ಮನೆಗೆ ನೆಂಟರಿಷ್ಟರು ಹೇಗೆ ಬಂದು ಹೋಗುತ್ತಿರುವರೋ ಹಾಗೆ ಸುಖ ದುಃಖಗಳೂ ಬಂದು ಹೋಗುತ್ತಿರುತ್ತವೆ.

ಇವೆಲ್ಲವೂ ಅನಿತ್ಯ. ಇವೆಲ್ಲ ಸುಳ್ಳು ಎಂದು ಬೇಕಾದರೂ ತೆಗೆದುಕೊಳ್ಳಬಹುದು ಅಥವಾ ಕ್ಷಣಿಕ ಎಂದು ಬೇಕಾದರೂ ನೋಡಬಹುದು. ಇವೆಲ್ಲ ಅನಿತ್ಯ–ಯಾವುದೊ ಒಂದು ಸ್ಥಿತಿಯಲ್ಲಿ, ಒಂದು ಅವಸ್ಥೆಯಲ್ಲಿ ಮಾತ್ರ ಸತ್ಯ. ಕನಸಿನಂತೆ. ಜಾಗ್ರತಾವಸ್ಥೆಗೆ ಕನಸು ಸುಳ್ಳು, ಕನಸಿನ ಅವಸ್ಥೆಗೆ ಜಾಗ್ರತಾವಸ್ಥೆ ಸುಳ್ಳು. ಆಯಾ ಸ್ಥಿತಿಯಲ್ಲಿ ಮಾತ್ರ ಇರುವ ಅನುಭವ ಅದು. ಜ್ಞಾನಿಯೊಬ್ಬನಿದ್ದ. ಅವನ ಒಬ್ಬನೇ ಮಗ ಕಾಲವಾದ. ಹೆಂಡತಿ ದುಃಖದಿಂದ ವ್ಯಸ್ತಳಾಗಿ ಅಳುತ್ತಿದ್ದಳು. ಗಂಡ ಮುಂದೇನು ಕೆಲಸ ಮಾಡಬೇಕೋ ಆ ಕೆಲಸವನ್ನು ಮಾಡಿದ. ಹೆಂಡತಿ ಗಂಡನನ್ನು ನಿಮ್ಮದು ಎಂತಹ ಕಲ್ಲು ಹೃದಯ, ಮಗುವಿಗೆ ಸ್ವಲ್ಪವೂ ಮರುಗುವುದಿಲ್ಲ ಎಂದಳು. ಆಗ ಗಂಡ ಹಿಂದಿನ ದಿನ ತನಗಾದ ಒಂದು ಕನಸನ್ನು ಹೇಳಿದ. ಅಲ್ಲಿ ಅವನು ರಾಜನಾಗಿದ್ದ. ನಾಲ್ಕು ಜನ ಗಂಡು ಮಕ್ಕಳು ಇದ್ದರು. ಸೌಂದರ್ಯದಲ್ಲಿ ಪರಾಕ್ರಮದಲ್ಲಿ ಒಬ್ಬರನ್ನೊಬ್ಬರು ಮೀರುವಂತೆ ಇದ್ದರು. ಕಣ್ದೆರೆದ, ಆ ಕನಸೆಲ್ಲ ಹೋಯಿತು. ಈಗ ಆ ನಾಲ್ಕು ಮಕ್ಕಳನ್ನು ಕಳೆದುಕೊಂಡಿದ್ದಕ್ಕಾಗಿ ಅಳಲೆ, ಈ ಜಾಗ್ರತಾವಸ್ಥೆಯಲ್ಲಿ ಒಬ್ಬನನ್ನು ಕಳೆದುಕೊಂಡಿದ್ದಕ್ಕಾಗಿ ಅಳಲೆ ಎಂದ. ಆದರೆ ಅದು ಕನಸು, ಇದು ನಿಜ ಎನ್ನಬಹುದು. ಆದರೆ ಜ್ಞಾನಿಗಾದರೊ ಒಂದರಷ್ಟೇ ಮತ್ತೊಂದು ಸುಳ್ಳು. ಎದ್ದ ತಕ್ಷಣ ಕನಸು ಸುಳ್ಳು. ಮಲಗಿದ ತಕ್ಷಣ ಈ ಜಾಗ್ರತಾವಸ್ಥೆ ಸುಳ್ಳು.

ಅರ್ಜುನ, ಅವುಗಳನ್ನು ಸಹಿಸಿಕೊ ಎನ್ನುವನು. ಏಕೆಂದರೆ ಬಾಹ್ಯ ಘಟನೆಗಳ ಮೇಲೆ ನಮ್ಮ ಹತೋಟಿ ಇಲ್ಲ. ಅವು ಬಂದಾಗ ಸಹಿಸಿಕೊಳ್ಳಬೇಕು. ಸುಖ ಬಂದಾಗ ನನ್ನ ಸಮಾನ ಇಲ್ಲ ಎಂದು ಮೆರೆಯಲೂ ಕೂಡದು, ದುಃಖ ಬಂದಾಗ ಹತಾಶರಾಗಿಯೂ ಹೋಗಕೂಡದು. ಪ್ರಪಂಚದ ಧರ್ಮವೇ ಇದು. ಪ್ರತಿಯೊಂದಕ್ಕೂ ಒಂದೊಂದು ಧರ್ಮವಿದೆ. ಅದನ್ನು ಗೊಣಗಾಡದೆ ಸ್ವೀಕರಿಸಬೇಕು. ಬಿಸಿಲುಕಾಲದಲ್ಲಿ ಅಯ್ಯೊ ತುಂಬಾ ಬಿಸಿಲು ಎಂದರೆ ಪ್ರಯೋಜನವೇನು? ಛಳಿಗಾಲದಲ್ಲಿ ಛಳಿ ಎಂದರೆ ಪ್ರಯೋಜನವೇನು? ಇವೆಲ್ಲಾ ಪುತುಧರ್ಮ. ಅದಕ್ಕೆ ಬಾಗಿ ಅದನ್ನು ಸ್ವೀಕರಿಸಬೇಕು. ಈ ಪ್ರಪಂಚಕ್ಕೆ ಬರುವುದೇ ಇವುಗಳನ್ನೆಲ್ಲ ಅನುಭವಿಸುವುದಕ್ಕೆ. ಇಲ್ಲಿಗೆ ಬಂದು ನಾವು ಇವುಗಳನ್ನು ಅನುಭವಿಸುವುದಿಲ್ಲ ಎಂದರೆ ಬಂದದ್ದು ಏತಕ್ಕೆ? ಬಂದಮೇಲೆ ಕಷ್ಟವನ್ನು ಅನುಭವಿಸಲೇಬೇಕಾಗಿದೆ. ಸಂತೆಗೆ ಬಂದು ಇಲ್ಲಿ ಗದ್ದಲ ಎನ್ನುವುದು, ಕಾಡಿಗೆ ಬಂದು ಇಲ್ಲಿ ದುಷ್ಟಮೃಗಗಳಿವೆ ಎನ್ನುವುದು, ಬೆಟ್ಟದ ಮೇಲೆ ಮನೆ ಕಟ್ಟಿ ಇಲ್ಲಿ ಗಾಳಿಹೊಡೆತ\break ಎನ್ನುವುದು; ಎಲ್ಲಾ ಒಂದೇ ಗುಂಪಿಗೆ ಸೇರಿದ್ದು. ನಿಜವಾದ ಜ್ಞಾನಿ ಹೊರಗಿನದನ್ನು ಬದಲಾಯಿ\-ಸುವುದಕ್ಕೆ ಹೋಗುವುದಿಲ್ಲ. ತನ್ನನ್ನು ಬದಲಾಯಿಸಿಕೊಳ್ಳುವನು. ಅವುಗಳನ್ನು ನೋಡುವ ದೃಷ್ಟಿಯನ್ನು ಬದಲಾಯಿಸಿಕೊಳ್ಳುವನು. ಇದೇ ಸಂಸಾರ ಎಂದು ಅದನ್ನು ಅನುಭವಿಸಲು ಅವನು ಸಿದ್ಧನಾಗುವನು. ಅದರಿಂದ ತಪ್ಪಿಸಿಕೊಂಡು ಹೋಗಲು ಪ್ರಯತ್ನಿಸುವುದಿಲ್ಲ. ನಾವು ಎಷ್ಟು ವೇಗದಲ್ಲಿ ಓಡಿದರೆ ಅವು ನಮಗಿಂತಲೂ ಹೆಚ್ಚು ವೇಗದಲ್ಲಿ ಬಂದು ನಮ್ಮನ್ನು ಹಿಡಿಯುವುವು. ಅಥವಾ ಸುಮ್ಮನೆ ಗೊಣಗಾಡುತ್ತಲೂ ಇರುವುದಿಲ್ಲ. ಗೊಣಗಾಡಿಕೊಳ್ಳುತ್ತಿದ್ದರೆ ಪರಿಸ್ಥಿತಿ ಸುಧಾರಿಸುವು\-ದೇನು? ಇದರಿಂದ ಸುಮ್ಮನೆ ನಮ್ಮ ಶಕ್ತಿ ವ್ಯಯವಾಗುವುದು ಅಷ್ಟೆ. ಅದಕ್ಕೆ ಶ‍್ರೀಕೃಷ್ಣ ಸುಖದುಃಖಗಳು ಇಲ್ಲದಂತೆ ಮಾಡುವುದಿಲ್ಲ, ಅದನ್ನು ಎದುರಿಸಬೇಕು ಎನ್ನುವನು. ಕೆಲವು ವೇಳೆ ಸಮುದ್ರದಲ್ಲಿ ಇಳಿಯುವೆವು. ದೊಡ್ಡದೊಂದು ಅಲೆ ನಮ್ಮನ್ನು ಅಪ್ಪಳಿಸಲು ಬರುವುದು. ಅಲೆಯನ್ನು ಬರಬೇಡ ಎನ್ನುವುದಕ್ಕೆ ಆಗುವುದಿಲ್ಲ. ಆದರೆ ಅದರಿಂದ ನಾವು ಎಷ್ಟು ಕಡಿಮೆ ಅಪಾಯಕ್ಕೆ ಒಳಗಾಗಬಹುದೋ ಹಾಗೆ ವರ್ತಿಸಬಹುದು ಅಷ್ಟೆ. ನೀರಿನಲ್ಲಿ ಮುಳುಗಬಹುದು. ಆಗ ಅಲೆ ನಮ್ಮ ತಲೆ ಮೇಲೆ ಹೊರಟು ಹೋಗುವುದು. ಈ ಸಂಸಾರಕ್ಕೆ ಬಂದಿರುವೆವು. ಇಲ್ಲಿ ನಾವು ಹೇಗೆ ಕಷ್ಟವನ್ನು ಸಹಿಸಬೇಕು ಎನ್ನುವುದನ್ನು ಶ‍್ರೀಕೃಷ್ಣ ಬೋಧಿಸುವನು.

\begin{shloka}
ಯಂ ಹಿ ನ ವ್ಯಥಯಂತ್ಯೇತೇ ಪುರುಷಂ ಪುರುಷರ್ಷಭ~।\\ಸಮದುಃಖಸುಖಂ ಧೀರಂ ಸೋsಮೃತತ್ವಾಯ ಕಲ್ಪತೇ \hfill॥ ೧೫~॥
\end{shloka}

\begin{artha}
ಪುರುಷಶ್ರೇಷ್ಠನೆ, ಸುಖದುಃಖದಲ್ಲಿ ಸಮನಾಗಿರುವ ಧೀರನಾದ ಯಾವ ಪುರುಷನನ್ನು ಇವು ವ್ಯಥೆಗೊಳಿಸುವು ದಿಲ್ಲವೋ ಅವನು ಅಮರತ್ವಕ್ಕೆ ಯೋಗ್ಯನಾಗಿರುವನು.
\end{artha}

ಈ ಜೀವನದಲ್ಲಿ ನಾವು ಏನನ್ನು ಗಳಿಸಬೇಕಾದರೂ ಒಂದು ಯೋಗ್ಯತೆ ಪಡೆದುಕೊಂಡಿರಬೇಕು. ಬರೀ ಆಸೆ ಇದ್ದರೇ ಸಾಲದು. ಲೌಕಿಕ ಪ್ರಪಂಚದಲ್ಲಿ ನಾವೇನಾದರೂ ಒಂದು ಕೆಲಸವನ್ನು ಸಂಪಾದಿಸಬೇಕಾದರೆ ನಮ್ಮ ಯೋಗ್ಯತೆ ಏನು ಎಂದು ಕೇಳುವರು. ಹಾಗಿರುವಾಗ ಅಮೃತತ್ತ್ವ ಎಂದರೆ ಮುಕ್ತಿಯನ್ನು ಪಡೆಯಬೇಕಾದರೆ ಅದಕ್ಕೆ ಕನಿಷ್ಟಪಕ್ಷದ ಯೋಗ್ಯತೆ ಒಂದು ಇದೆ. ಅದು ಇದ್ದರೆ ಮುಕ್ತಿಯ ಪಥದಲ್ಲಿ ಮುಂದುವರಿದು ಗುರಿ ಸೇರಬಹುದು. ಎವರೆಸ್ಟ್ ಶಿಖರವನ್ನು ಹತ್ತಬೇಕಾದರೆ ದೃಢಕಾಯನಾಗಿರಬೇಕು, ಶ್ರಮಸಹಿಷ್ಣು ಆಗಿರಬೇಕು, ಶ್ವಾಸಕೋಶಗಳಿಗೆ ಬಲವಾದ ಶಕ್ತಿ ಇರಬೇಕು. ಆಗ ಮಾತ್ರ ಮೇಲೆ ಹೋಗಲು ಸಾಧ್ಯ. ಯೋಗ್ಯತೆ ಇಲ್ಲದೆ ಬರೀ ಇಚ್ಛೆಯ ಆಸರೆಯಿಂದಲೇ ಗುರಿ ಮುಟ್ಟುವುದಕ್ಕಾಗುವುದಿಲ್ಲ.

ಜನನ ಮರಣಗಳ ಕೋಟಲೆಯಲ್ಲಿ ಸಿಕ್ಕಿಕೊಂಡು ಬೇಕಾದಷ್ಟು, ಪಡಬಾರದ ಯಾತನೆ\-ಯನ್ನೆಲ್ಲಾ ಅನುಭವಿಸುತ್ತಿರುವಾಗ ಮುಕ್ತಿಗೆ ಆಶಿಸುತ್ತೇವೆ. ಆದರೆ ಆ ದಾರಿಯಲ್ಲಿ ನಡೆಯಬೇಕಾದರೆ, ನಾವು ಧೀರರಾಗಿರಬೇಕು. ಸುಖದುಃಖಗಳನ್ನು ಸಮನಾಗಿ ನೋಡಬೇಕು. ಅವುಗಳ ಉಪಟಳಕ್ಕೆ ಸಿಕ್ಕಿ ನರಳಕೂಡದು. ಇದೇ ಪ್ರಪ್ರಥಮವಾದ ಯೋಗ್ಯತೆ. ಅವನು ಸಾಧಾರಣ ಮನುಷ್ಯನಲ್ಲ. ಅವನು ಧೀರಾಧಿಧೀರ. ಈ ಪ್ರಪಂಚದ ಸುಖ ದುಃಖಗಳ ಬಿರುಗಾಳಿಗೆ ಸಾಧಾರಣ ಜೀವರು ತರಗೆಲೆಯಂತೆ ಕೊಚ್ಚಿಹೋಗುವರು. ಆದರೆ ಧೀರನಾದರೊ ಕಲ್ಲುಬಂಡೆಯಂತೆ ನಿಲ್ಲುವನು. ಸುಖದ ಚಕ್ಕಳಗುಳಿಗೆ ಅವನು ಕಿಲಕಿಲ ನಗುವವನಲ್ಲ. ದುಃಖದ ಪೆಟ್ಟಿಗೆ ಅವನು ಕೊರಗುವವನೂ ಅಲ್ಲ. ಅವನು ಇದನ್ನು ಸಮಾನವಾಗಿ ನೋಡುತ್ತಾನೆ. ಒಂದು ಮೇಲಲ್ಲ ಮತ್ತೊಂದು ಕೀಳಲ್ಲ. ಈ ಜೀವನದಲ್ಲಿ ವ್ಯಕ್ತಿಯ ಶೀಲವನ್ನು ಬಿಡಿಸುವಾಗ ಸುಖದುಃಖಗಳಿಗೆಲ್ಲಾ ಸಮಾನ ಪಾತ್ರಗಳಿವೆ. ಇವೆರಡೂ ಶಿಲ್ಪಿಯ ಕೈಯಲ್ಲಿರುವ ಉಳಿಗಳಂತೆ. ಆ ಶಿಲ್ಪಿಯಾದರೋ ಒಂದು ಕಲ್ಲನ್ನು ತೆಗೆದುಕೊಂಡು ಅದರಲ್ಲಿ ಹುದುಗಿರುವ ದಿವ್ಯವಾದ ಮೂರ್ತಿಯನ್ನು ಬಿಡಿಸುವಾಗ ಅದಕ್ಕೆ ಆತಂಕವಾಗಿರುವ ಭಾಗವನ್ನೆಲ್ಲಾ ತೆಗೆದು ಹಾಕಬೇಕು. ಅನಂತರವೇ ಆ ಕಾಡಿನ ಕಲ್ಲಿನಲ್ಲಿ ಒಂದು ಸುಂದರವಾದ ವಿಗ್ರಹ ಮೂಡಿ ನಿಲ್ಲುವುದು.

ಈ ಜೀವನದಲ್ಲಿ ಹಗಲೂ ರಾತ್ರಿಗಳಂತೆ ಸುಖ ದುಃಖಗಳೂ ಒಂದಾದ ಮೇಲೊಂದು ಬರುವುವು. ಇವು ಬರುವಾಗ ಧೀರ, ಹೋಗುವುದಕ್ಕೆ ಬರುತ್ತಿವೆ ಎಂದು ಎದುರಿಸುವನು. ಈ ಸಂಸಾರದ ಕಣಿವೆಯಲ್ಲಿ ನಾವು ಹೋಗುತ್ತಿರುವಾಗ ಬರೀ ಸುಖವೇ ಅಥವಾ ದುಃಖವೇ ಒಬ್ಬನಿಗೆ ಬರುವುದಿಲ್ಲ. ಇವು ಒಂದಾದ ಮೇಲೊಂದು ಬರುವುವು. ಕೆಲವು ವೇಳೆ ಬೇಗ ಬೇಗ ಬರುತ್ತವೆ. ಕೆಲವು ವೇಳೆ ನಿಧಾನವಾಗಿ ಬರುತ್ತವೆ.

ಇವು ಬಂದಾಗ ಧೀರನನ್ನು ವ್ಯಥೆಗೊಳಿಸಲಾರವು. ಏಕೆಂದರೆ ಅವನು ಇದಕ್ಕೆ ಸಿದ್ಧನಾಗಿರುವನು. ತಾನು ಬಯಸಿದ ವಸ್ತುಗಳೇ ಬರುತ್ತವೆ ಎಂದು ನಂಬುವವನಲ್ಲ ಅವನು. ಬಯಸದೇ ಇರುವ ವಸ್ತುಗಳೂ ಬರುತ್ತವೆ ಎಂಬುದನ್ನು ಮುಂಚೆಯೇ ತಿಳಿದು ಯಾವಾಗ ಅದಕ್ಕೆ ಸನ್ನದ್ಧನಾಗಿರುವನೋ ಅವನನ್ನು ಇವು ಕೆಣಕಲಾರವು. ಪ್ಲೇಗು ಅಥವಾ ಕಾಲರ ಬರುವುದಕ್ಕೆ ಮುಂಚೆ ಇನಾಕ್ಯುಲೇಷನ್ ಹಾಕಿಸಿಕೊಂಡಿದ್ದರೆ ಆ ರೋಗ ಇವನಿಗೆ ತಾಕಿದಾಗ ಅವನು ಇದಕ್ಕೆ ತುತ್ತಾಗುವುದಿಲ್ಲ. ದೇಹ ಆಗಲೇ ಅದಕ್ಕೆ ಹೊಂದುಕೊಂಡಿರುತ್ತದೆ.

ಇಲ್ಲಿ ಇದನ್ನು ಸಹಿಸುವವನಿಗೆ ಧೀರ ಎಂಬ ಸುಂದರವಾದ ಪದವನ್ನು ಉಪಯೋಗಿಸುವನು. ಈ ಪ್ರಪಂಚದಲ್ಲಿ ಶಾರೀರಕವಾಗಿ ಧೀರರಾದ ಅನೇಕರನ್ನು ನಾವು ನೋಡುತ್ತೇವೆ. ಆನೆಯ ಕಾಲಿನಿಂದ ತುಳಿಸಿಕೊಳ್ಳುವವರೂ ಇದ್ದಾರೆ. ಫಿರಂಗಿಯ ಬಾಯಿಗೆ ಎದೆಯೊಡ್ಡುವ ಧೀರರೂ ಇದ್ದಾರೆ. ಸ್ಪುಟ್ನಿಕ್​ನಲ್ಲಿ ಕುಳಿತು ಚಂದ್ರಲೋಕ ಮತ್ತು ಇತರ ಲೋಕಗಳಿಗೆ ತಮ್ಮ ಪ್ರಾಣದ ಹಂಗನ್ನೇ ತೊರೆದು ಹೋಗಬಲ್ಲ ಧೀರರಿದ್ದಾರೆ. ಆದರೆ ಜೀವನದಲ್ಲಿ ನಮ್ಮನ್ನು ಅಳುವಂತೆ ನಗಿಸುವಂತೆ ಮಾಡುವ ಪ್ರಕೃತಿಯ ಬಲೆಗೆ ಬೀಳದೆ ಇರುವವನೇ ನಿಜವಾದ ಧೀರ. ಇಂದ್ರಿಯಗಳ ಸೆಳೆತವನ್ನು ಹಿಡಿದು ನಿಲ್ಲಿಸುವುದು ವೇಗವಾಗಿ ಹೋಗುತ್ತಿರುವ ಕಾರನ್ನು ಹಿಡಿದು ನಿಲ್ಲಿಸುವುದಕ್ಕಿಂತ ಸಾಹಸದ ಕೆಲಸ. ಅಮೃತತ್ವವನ್ನು ಪಡೆಯಬೇಕಾದರೆ ವಿಷಯವಸ್ತುಗಳ ಆಕರ್ಷಣೆಯನ್ನು ತಡೆದು ನಿಲ್ಲಿಸಬೇಕು. ಇಂತಹ ಪಥಿಕ ಮಾತ್ರ ಆ ದಾರಿಯಲ್ಲಿ ಯಾತ್ರಿಕನಾಗಬಹುದು.

\begin{shloka}
ನಾಸತೋ ವಿದ್ಯತೇ ಭಾವೋ ನಾಭಾವೋ ವಿದ್ಯತೇ ಸತಃ~।\\ಉಭಯೋರಪಿ ದೃಷ್ಟೋಽಂ ತಸ್ತ್ವನಯೋಸ್ತತ್ತ್ವದರ್ಶಿಭಿಃ \hfill॥ ೧೬~॥
\end{shloka}

\begin{artha}
ಅಸತ್ತಿನ ವಸ್ತುವಿಗೆ ಇರುವಿಕೆಯಿಲ್ಲ. ಸತ್​ವಸ್ತುವಿಗೆ ಇಲ್ಲದೆ ಇರುವಿಕೆಯಿಲ್ಲ. ಇವೆರಡರ ಸ್ಥಿತಿಯನ್ನು ತತ್ವದರ್ಶಿಗಳು ತಿಳಿದಿರುತ್ತಾರೆ.
\end{artha}

ಅಸತ್ತಿಗೆ ಅಸ್ತಿತ್ವ ಎಂದೂ ಇರಲಿಲ್ಲ. ಅದಿರುವಂತೆ ನಮಗೆ ಕಾಣುತ್ತಿರಬಹುದು. ಆದರೆ ಎಂದಿಗೂ ಅದು ಇರಲಿಲ್ಲ. ಅದು ಬರಿ ತೋರಿಕೆ. ನಾವು ಕನಸಿನಲ್ಲಿ ಏನೇನನ್ನೋ ನೋಡುತ್ತೇವೆ. ಆ ಕಾಲದಲ್ಲಿ ಆ ಸ್ಥಿತಿಯಲ್ಲಿ ಮಾತ್ರ ಅವು ಇರುವಂತೆ ಕಾಣುತ್ತವೆ. ನಾವು ಕಣ್ದೆರೆದೊಡನೆ ಅವೆಲ್ಲ ಮಂಗಮಾಯವಾಗುವುವು. ಹಾಗಾದರೆ ಜಾಗ್ರತಾವಸ್ಥೆ ಹಾಗಲ್ಲ. ಪ್ರತಿದಿನವೂ ಕನಸಾದ ಮೇಲೆ ನಾವು ಒಂದೇ ಜಗತ್ತಿಗೆ ಬರುತ್ತೇವೆ. ಒಂದೇ ಮನೆ, ಒಂದೇ ನಂಟರಿಷ್ಟರು, ಒಂದೇ ಕೆಲಸ ಇತ್ಯಾದಿಗಳು. ಸಾಪೇಕ್ಷ ದೃಷ್ಟಿಯಿಂದ ಕನಸಿಗಿಂತ ಹೆಚ್ಚು ದೀರ್ಘಕಾಲ ಇರುವಂತೆ ಕಾಣುವುದು. ಆದರೆ ಯಾವಾಗ ಒಬ್ಬ ಈ ದೇಹವನ್ನು ತ್ಯಜಿಸುವನೊ ಅವನಿಗೆ ಇಷ್ಟು ದಿವಸ ಬಾಳಿದ ಈ ದೇಹದ ಮೂಲಕ ಗಳಿಸಿದ ಅನುಭವಗಳೆಲ್ಲ ಒಂದು ಕನಸಿನಂತೆ ಕಾಣುವುದು. ಇಲ್ಲಿ ಅಸತ್ ಎಂದರೆ ಇಲ್ಲ ಎಂಬ ದೃಷ್ಟಿಯಿಂದಲೇ ತೆಗೆದುಕೊಳ್ಳಬೇಕಾಗಿಲ್ಲ. ಇವು ಕ್ಷಣಿಕ, ತಾತ್ಕಾಲಿಕವಾಗಿರುವುವು. ಇಂದಿನದು ನಾಳೆ ಇಲ್ಲ. ಈ ದೃಷ್ಟಿಯಿಂದ ಬೇಕಾದರೂ ನಾವು ನೋಡಬಹುದು.

ನಿಜವಾಗಿ ಏನಿದೆಯೋ ಅದು ಒಂದಲ್ಲ ಒಂದು ಸ್ಥಿತಿಯಲ್ಲಿ ಇದ್ದೇ ತೀರುವುದು. ಅದನ್ನು ನಾಶಮಾಡುವುದಕ್ಕೆ ಆಗುವುದಿಲ್ಲ. ಈ ದೇಹದ ಹಿಂದೆ ಇರುವ ಆತ್ಮನಾದರೋ ಅದರಂತೆಯೇ. ಅವನು ನಾಶವಾದ ಎಂದರೆ, ಹಳೆಯ ನಾಮರೂಪು ಹೋಗುವುದು, ಇನ್ನಾವುದೊ ಬೇರೊಂದು ನಾಮರೂಪನ್ನು ಧರಿಸಿ ಬೇರೊಂದು ಕಡೆ ಏಳುವನು.

ಇವೆರಡನ್ನೂ ತಿಳಿದವನೆ ತತ್ತ್ವದರ್ಶಿ. ಸುಳ್ಳು ಎಂದಿಗೂ ಇರಲಿಲ್ಲ. ಸತ್ಯ ಎಂದೆಂದಿಗೂ ಮಾಯವಾಗುವುದಿಲ್ಲ. ತತ್ತ್ವದರ್ಶಿ ವಸ್ತುವಿನ ನೈಜಸ್ಥಿತಿಯನ್ನು ಅರಿತುಕೊಳ್ಳಬಲ್ಲವನು. ಅವನು ಹೊರಗೆ ಕಾಣುವ ಕೇವಲ ನಾಮರೂಪುಗಳಿಂದಲೇ ಪರವಶನಾಗಿ ಬಿಡುವವನಲ್ಲ. ಸಾಧಾರಣ ಕಣ್ಣಿಗೆ ದೇಹ ಸುಂದರವಾಗಿ ಕಾಣುವುದು. ಅದೇ ದೇಹದ ಎಕ್ಸ್ ರೇ ಚಿತ್ರದಲ್ಲಿ ಅವನೊಂದು ಎಲುಬು ಗೂಡಾಗಿ ಕಾಣುವನು. ಎಕ್ಸ್ ರೇ ಒಳಗಿನದನ್ನು ನೋಡುವುದು. ನಮ್ಮ ಸಾಧಾರಣ ಕಣ್ಣು ಹೊರಗಿನದನ್ನು ನೋಡುವುದು. ಸುಳ್ಳು ತುಂಬಾ ಮೋಹಕವಾಗಿರುವುದು. ಆದರೆ ಅದರ ಬಲೆಗೆ ಬೀಳುವವನಲ್ಲ ತತ್ತ್ವದರ್ಶಿ. ಈ ಸುಳ್ಳು ಈಗ ನಿಜದಂತೆ ಕಾಣುತ್ತಿದೆ. ಇದು ಬರೀ ವೇಷ, ಬಣ್ಣ. ಇದನ್ನು ಚೆನ್ನಾಗಿ ಅರಿತವನೆ ತತ್ವದರ್ಶಿ. ಅದರಂತೆಯೇ ಸತ್ಯ ಎಂದಿಗೂ ಇಲ್ಲದೇ ಇರುವುದಿಲ್ಲ. ಆದರೆ ಅದು ಯಾವಾಗಲೂ ಒಂದೇ ಸ್ಥಿತಿಯಲ್ಲಿ ಇರಬೇಕಾಗಿಲ್ಲ. ಒಂದಲ್ಲ ಒಂದು ಸ್ಥಿತಿಯಲ್ಲಿ ಇರುವುದು. ನೀರು ಘನೀಭೂತವಾಗಿ ಗೆಡ್ಡೆ ಕಟ್ಟಿರಬಹುದು. ಅದೇ ನೀರು ದ್ರವರೂಪವಾಗಿ ಹರಿಯುತ್ತಿರಬಹುದು. ಆವಿಯಾಗಿ ಕಣ್ಣಿಗೆ ಕಾಣದೆ ಹೋಗಬಹುದು. ಕಣ್ಣಿಗೆ ಕಾಣದೆ ಹೋಯಿತು ಎಂದರೆ ಅದೇನು ನಾಶವಾಗಿಲ್ಲ. ಅದರ ರೂಪು ಮಾತ್ರ ಬದಲಾವಣೆ ಆಯಿತು. ಅದರಂತೆಯೇ ಜ್ಞಾನಿಗೆ ಆತ್ಮ ಧರಿಸಿರುವ ದೇಹ ತಾತ್ಕಾಲಿಕ, ಆಗಂತುಕ, ಈಗಿದ್ದು ನಾಳೆ ಹೋಗುವುದು, ಇರುವಾಗಲೇ ಕ್ಷಣಕ್ಷಣ ಬದಲಾಯಿಸುತ್ತಿರುವುದು. ಇದು ಹಿಂದೆ ಇರಲಿಲ್ಲ, ಮುಂದೆ ಇರುವಂತೆ ಇಲ್ಲ. ಈಗಲಾದರೂ ಇದೆಯಲ್ಲ, ಎಂದರೆ ಇರುವ ಕೆಲವು ಕಾಲದಲ್ಲಿಯೂ ಬದಲಾಯಿಸುತ್ತಾ ಹೋಗುತ್ತದೆ. ಇದರಲ್ಲಿ ಯಾವ ಸ್ಥಿತಿಯನ್ನು ನಮ್ಮದೆಂದು ಹೇಳುವುದು? ಇದರ ಹಿಂದೆ ಇರುವ ಆತ್ಮನು ಧರಿಸಿರುವ ದೇಹ ಮಾಯವಾದರೂ ಧರಿಸಿರುವವನು ಎಂದಿಗೂ ಮಾಯವಾಗಲಾರ. ಗಡಿಗೆ ಒಡೆದು ಹೋದರೆ ಅದರ ಹಿಂದೆ ಇರುವ ಆಕಾಶವನ್ನು ಯಾರಾದರೂ ಧ್ವಂಸಮಾಡಲು ಸಾಧ್ಯವೇ? ಚೈತನ್ಯ ಒಂದು ದೇಹವನ್ನು ಉಪಯೋಗಿಸಿಕೊಳ್ಳುವುದು. ಅದೇ ದೇಹವಲ್ಲ. ಒಂದು ಬಲ್ಬು ಒಡೆದುಹೋದರೆ ಅದರ ಹಿಂದೆ ಇರುವ ವಿದ್ಯುತ್ ಶಕ್ತಿ ನಾಶವಾಗಲಿಲ್ಲ. ಇದನ್ನು ಚೆನ್ನಾಗಿ ತಿಳಿದವನೇ ತತ್ತ್ವದರ್ಶಿ.

ಇನ್ನು ಮೇಲೆ ಶ‍್ರೀಕೃಷ್ಣ ತಾತ್ತ್ವಿಕ ದೃಷ್ಟಿಯಿಂದ ಮಾತನಾಡುವನು. ಮೊದಲನೆಯದು ವ್ಯಾವ ಹಾರಿಕ ದೃಷ್ಟಿ. ಹುಟ್ಟುವವರೆಲ್ಲಾ ಸಾಯಲೇಬೇಕು. ನಾವೇನು ಮಾಡಿದರೂ ಅವರನ್ನು ಎಂದೆಂದಿಗೂ ಇರುವಂತೆ ಮಾಡಲು ಆಗುವುದಿಲ್ಲ. ಈ ಪ್ರಪಂಚದಲ್ಲಿ ಎಲ್ಲಾ ಬಂದು ಹೋಗುತ್ತಿವೆ. ಇದೊಂದು ನಾಮರೂಪುಗಳ ಮೆರವಣಿಗೆ. ಸುಖ ದುಃಖ ಜನನ ಮರಣ ಒಂದಾದ ಮೇಲೊಂದು ಬರುತ್ತಿರುವುವು. ಅವನ್ನು ಸಹಿಸಬೇಕು ಎಂಬುದು ಮೊದಲನೆ ವಿಚಾರ. ಇನ್ನು ಮೇಲೆ ದೊಡ್ಡ ತತ್ವದರ್ಶಿಯಂತೆ ಮಾತನಾಡುವನು. ಒಂದಲ್ಲ ಮತ್ತೊಂದಾದರೂ ಅವನ ಮನಸ್ಸಿಗೆ ತಾಕಲಿ ಅಥವಾ ಇವುಗಳೆಲ್ಲದರ ಪರಿಣಾಮವಾದರೂ ಅವನ ಮೇಲೆ ಆಗಿ ಅವನನ್ನು ಕಾರ್ಯೋನ್ಮುಖ\-ನನ್ನಾಗಿ ಮಾಡಲಿ ಎಂಬುದೇ ಅವನ ಉದ್ದೇಶ. ರೋಗಿಗೆ ವೈದ್ಯ ಯಾವುದೋ ಒಂದು ಔಷಧಿಯನ್ನು ಕೊಡದೆ, ಹಲವು ಔಷಧಿಗಳನ್ನು ಪ್ರಯೋಗ ಮಾಡಿ ಒಂದಲ್ಲ ಮತ್ತೊಂದರಿಂದಲಾದರೂ ಅವನು ಚೇತರಿಸಿಕೊಳ್ಳಲಿ ಎಂದು ಮಾಡಿದಂತೆ.

\begin{shloka}
ಅವಿನಾಶಿ ತು ತದ್ವಿದ್ಧಿ ಯೇನ ಸರ್ವಮಿದಂ ತತಮ್~।\\ವಿನಾಶಮವ್ಯಯಸ್ಯಾಸ್ಯ ನ ಕಶ್ಚಿತ್ಕರ್ತುಮರ್ಹತಿ \hfill॥ ೧೭~॥
\end{shloka}

\begin{artha}
ಯಾವುದರಿಂದ ಇದೆಲ್ಲವೂ ವ್ಯಾಪಿಸಿಕೊಂಡಿರುವುದೊ ಅದು ಅವಿನಾಶಿ ಎಂದು ತಿಳಿದುಕೊ. ಅವ್ಯಯವಾಗಿರುವ ಇದರ ನಾಶವನ್ನು ಯಾರೂ ಮಾಡಲಾರರು.
\end{artha}

ಯಾವುದರಿಂದ ಇವೆಲ್ಲ ವ್ಯಾಪಿಸಿಕೊಂಡಿದೆಯೊ, ಎಂದರೆ ಚೈತನ್ಯ ಈ ವಿಶ್ವವನ್ನೆಲ್ಲಾ ಆವರಿಸಿ\-ಕೊಂಡಿದೆ. ಅದನ್ನು ಯಾರೂ ನಾಶ ಮಾಡಲಾಗುವುದಿಲ್ಲ. ಈ ಚೈತನ್ಯ ಒಂದೊಂದು ಕಡೆ ಒಂದೊಂದು ಅವಸ್ಥೆಯಲ್ಲಿದೆ. ಎಂದರೆ ಪಂಚಭೂತಗಳಿಂದ ಆದ ಜಡವಸ್ತುವಿನಂತೆ ಇರುವ ಕಲ್ಲುಮಣ್ಣು ಮುಂತಾದುವುಗಳಲ್ಲಿ ಅದು ಮಲಗಿ ನಿದ್ರಿಸುತ್ತಿದೆ. ಅಲ್ಲಿ ಅದು ಸುಪ್ತಾವಸ್ಥೆಯಲ್ಲಿದೆ. ಸಸ್ಯಗಳ ಗುಂಪಿಗೆ ಬಂದಾಗ ಅದು ಈಗತಾನೆ ಎಚ್ಚರವಾಗುತ್ತಿದೆ. ಮನುಷ್ಯನ ಗುಂಪಿಗೆ\break ಬಂದಾಗ ಚೆನ್ನಾಗಿ ಕೆಲಸ ಮಾಡುತ್ತಿದೆ. ಯಾವಾಗ ಒಬ್ಬ ತತ್ವದರ್ಶಿಯ ಮಟ್ಟಕ್ಕೆ ಏರುವನೊ ಅದು ಅಲ್ಲಿ ಚೆನ್ನಾಗಿ ಪ್ರಬುದ್ಧವಾಗಿರುವುದನ್ನು ನೋಡುತ್ತೇವೆ. ಜೀವವಿಲ್ಲದ ವಸ್ತುವೇ ಇಲ್ಲ. ಆದರೆ ಅದು ಇರುವ ಪರಿಸ್ಥಿತಿ ವ್ಯತ್ಯಾಸವಾಗುವುದು. ಲೋಹಾದಿಗಳಲ್ಲಿ ಜೀವವಿದೆ. ಆದರೆ ನಮ್ಮ ರೀತಿಯಲ್ಲಿ ಇಲ್ಲ. ಸಸ್ಯಾದಿಗಳಲ್ಲಿಯೂ ಜೀವವಿದೆ. ಅದನ್ನು ನಾವು ನಿತ್ಯವೂ ನೋಡುತ್ತೇವೆ. ಅದು ಬೆಳೆಯುವುದು, ವೃದ್ಧಿಯಾಗುವುದು, ಒಂದು ಹಲವನ್ನು ಸೃಷ್ಟಿಸುವುದು, ಯಾವ ಆತಂಕಗಳು ಅದರ ಬೆಳವಣಿಗೆಗೆ ಬಂದರೂ ಅದರೊಂದಿಗೆ ಹೋರಾಡುವುದು. ಸುತ್ತಲಿರುವ ಗಾಳಿ ಬೆಳಕು, ಗೊಬ್ಬರ ಮುಂತಾದುವುಗಳಿಂದ ತನ್ನ ಪೋಷಣೆಗೆ ಬೇಕಾದ ವಸ್ತುವನ್ನು ತಯಾರು ಮಾಡಿಕೊಳ್ಳುವುದು. ಸೂರ್ಯನ ಬೆಳಕಿನಲ್ಲಿರುವ ಒಂದು ಹಸಿರು ಎಲೆ, ಮನುಷ್ಯನು ಮಾಡಿದ ಕೆಮಿಕಲ್ ಫ್ಯಾಕ್ಟರಿಯನ್ನು ಮೀರಿಸಿ ಕೆಲಸ ಮಾಡುತ್ತಿರುವುದು. ಪ್ರಾಣಿಗಳಲ್ಲಿ ಮತ್ತು ಮನುಷ್ಯನಲ್ಲಿ ಚೈತನ್ಯವಿದೆ ಎಂಬುದನ್ನು ಎಲ್ಲರೂ ಬಲ್ಲರು.

ಈ ಚೈತನ್ಯ ವಿಶ್ವವ್ಯಾಪಿಯಾಗಿದೆ. ಒಂದೊಂದು ಕಡೆ ಅದು ಮಧ್ಯವರ್ತಿಗೆ ತಕ್ಕಂತೆ ಒಂದೊಂದು ರೀತಿ ಬೆಳಗುತ್ತಿದೆ. ವಿದ್ಯುತ್​ಶಕ್ತಿ ತಂತಿಯಲ್ಲಿ ಹೋಗುತ್ತಿರುವಾಗ ನಮಗೆ ಕಾಣುವುದಿಲ್ಲ. ಅದರಲ್ಲಿ ಶಕ್ತಿ ಇದೆ ಎಂಬುದನ್ನು ಮ್ಯಾಗ್ನೆಟಿಕ್ ಪ್ರತಿಕ್ರಿಯೆಯಿಂದ ಮಾತ್ರ ತಿಳಿದುಕೊಳ್ಳಬಹುದು. ಅದೇ ವಿದ್ಯುತ್​ಶಕ್ತಿ ರಾತ್ರಿ ಹೊತ್ತು ಉರಿಸುವ ಜೀರೋ ಕ್ಯಾಂಡಲ್ ಬಲ್ಬಿನ ಹಿಂದುಗಡೆ ಇದೆ, ಸಣ್ಣ ಬಲ್ಬಿನಲ್ಲಿದೆ, ನೂರಾರು ಕ್ಯಾಂಡಲ್ ಬಲ್ಬಿನ ಹಿಂದೆಯೂ ಇದೆ. ಮಿಂಚುಹುಳುವಿನ ಕಾಂತಿಯಿಂದ ಹಿಡಿದು ಕೋರೈಸಿ ಬೀರುವ ಕಾಂತಿಯವರೆಗೆ ವಿದ್ಯುತ್​ಶಕ್ತಿ ಹೆಚ್ಚು ಕಡಮೆ ಬೆಳಗುತ್ತಿದೆ. ಇದನ್ನು ಯಾರೂ ನಾಶ ಮಾಡುವುದಕ್ಕೆ ಎಂದರೆ ಅದನ್ನು ಇಲ್ಲದಂತೆ ಮಾಡುವುದಕ್ಕೆ ಆಗುವುದಿಲ್ಲ. ಯಾವುದರ ಮೂಲಕ ವಿದ್ಯುತ್​ಶಕ್ತಿ ಬೆಳಗುತ್ತಿದೆಯೋ ಅವುಗಳನ್ನೆಲ್ಲ ಧ್ವಂಸ ಮಾಡಿದರೂ ವಿದ್ಯುತ್​ಶಕ್ತಿಯನ್ನು ಧ್ವಂಸಮಾಡಿದಂತೆ ಆಗಲಿಲ್ಲ. ಅದರಂತೆ ಆತ್ಮ ಎಲ್ಲಾ ವಸ್ತುಗಳಲ್ಲಿ ಆಯಾ ವಸ್ತುವಿಗೆ ತಕ್ಕಂತೆ ಬೆಳಗುತ್ತಿದೆ. ನಾವು ಇದು ವ್ಯಕ್ತವಾಗುವ ಮಧ್ಯವರ್ತಿಯನ್ನು ನಾಶಮಾಡಬಹುದು. ಆದರೆ ಅದರ ಹಿಂದೆ ಇರುವ ಶಕ್ತಿಯನ್ನು ನಾಶಮಾಡುವುದಕ್ಕೆ ಆಗುವುದಿಲ್ಲ. ನಾವೊಂದು ಹೈಡ್ರೊಜನ್ ಬಾಂಬನ್ನು ಎಸೆದು ಒಂದು ಕಡೆ ಇರುವ ವಸ್ತುಗಳೆಲ್ಲಾ ಧೂಳೀಕಣಗಳಾಗಿ ಹೋಗುವಂತೆ ಮಾಡಬಹುದು. ಆದರೆ ವಸ್ತುಗಳು ಇದ್ದ ಆಕಾಶವನ್ನು ಯಾವ ಬಾಂಬು ಏನು ಮಾಡ ಬಲ್ಲದು? ಅದರಂತೆಯೇ ಆತ್ಮ.

\begin{shloka}
ಅಂತವಂತ ಇಮೇ ದೇಹಾ ನಿತ್ಯಸ್ಯೋಕ್ತಾಃ ಶರೀರಿಣಃ~।\\ಅನಾಶಿನೋಽಪ್ರಮೇಯಸ್ಯ ತಸ್ಮಾದ್ಯುಧ್ಯಸ್ವ ಭಾರತ \hfill॥ ೧೮~॥
\end{shloka}

\begin{artha}
ಸತ್ಯವಾಗಿಯೂ ಅವಿನಾಶವಾಗಿಯೂ ಅಪ್ರಮೇಯನಾಗಿಯೂ ಇರುವ ಆತ್ಮನ ದೇಹಗಳು ನಶ್ವರ. ಆದಕಾರಣ ಭಾರತ ಯುದ್ಧವನ್ನು ಮಾಡು.
\end{artha}

ಆತ್ಮ ಮತ್ತು ದೇಹ ಪರಸ್ಪರ ವಿರೋಧವುಳ್ಳವುಗಳು. ಆದರೆ ಹೇಗೊ ಅವೆರಡಕ್ಕೂ ಇಲ್ಲಿ ಗಂಟುಬಿದ್ದಿದೆ. ದೇಹ ಹುಟ್ಟಿದೆ, ಸಾಯುವುದು. ಅದು ಹಿಂದೆ ಇರಲಿಲ್ಲ, ಮುಂದೆ ಇರುವಂತೆ ಇಲ್ಲ. ಈ ಮಧ್ಯದಲ್ಲಿ ಇರುವಾಗಲೂ ಬದಲಾಯಿಸುತ್ತಿರುವುದು. ಆತ್ಮನಾದರೋ ಯಾವಾಗಲೂ ಇದ್ದ. ದೇಹ ಮನಸ್ಸು ಬುದ್ಧಿ ಇಂದ್ರಿಯಗಳು ಬದಲಾಯಿಸುತ್ತಿದ್ದರೂ, ಆತ್ಮ ಅದರ ಬದಲಾವಣೆಯನ್ನು ಸಾಕ್ಷಿಯಂತೆ ನಿಂತು ನೋಡುತ್ತಿದ್ದ. ಆತ್ಮ ಅವಿನಾಶಿ; ಅದನ್ನು ಯಾವುದೂ ಸಾಯಿಸುವುದಕ್ಕೆ ಆಗುವುದಿಲ್ಲ. ಆಕಾಶ ಹೇಗೋ ಹಾಗೆ. ಆಕಾಶ ಇರುವ ಪಾತ್ರೆಯನ್ನು ಧ್ವಂಸಮಾಡಿದರೂ ಆಕಾಶ ಹೇಗೆ ಇರುವುದೋ ಹಾಗೆ. ಆತ್ಮ ಅಪ್ರಮೇಯ ಎಂದರೆ ಪ್ರತ್ಯಕ್ಷಾದಿ ಪ್ರಮಾಣಗಳ ಮೂಲಕ ತಿಳಿಯತಕ್ಕದ್ದಲ್ಲ. ಅದು ಸ್ವತಃ ಸಿದ್ಧವಾದ ವಸ್ತು. ಅದನ್ನು ನಾವು ನಮ್ಮಿಂದ ಹೊರಗಡೆ ನೋಡುವುದಕ್ಕೆ ಆಗುವುದಿಲ್ಲ. ಅದಕ್ಕೆ ಕಾರಣವಿಲ್ಲ, ಬಣ್ಣವಿಲ್ಲ, ಅದರ ವಿಷಯವಾಗಿ ಹೀಗಿದೆ ಎಂದು ವಿವರಿಸಲಾಗದು. ಮಾತು ನಾವು ಕಂಡು ಕೇಳಿದ ವಿಷಯವನ್ನು ಮಾತ್ರ ವಿವರಿಸಬಲ್ಲುದು, ಇಂದ್ರಿಯ ಅನುಭವವನ್ನು ವಿವರಿಸಬಲ್ಲುದು. ಇಂದ್ರಿಯಾತೀತ ಅನುಭವವನ್ನು ಅದು ವಿವರಿಸಲಾರದು. ಮಾತು ಮೂಕವಾಗುವುದು ಅಲ್ಲಿ. ಇದರಂತೆಯೇ ಮನಸ್ಸು ಅದನ್ನು ಕಲ್ಪಿಸಿಕೊಳ್ಳಲಾರದು. ನಮ್ಮ ಕಲ್ಪನೆ ಯೆಲ್ಲ ನಾವು ಎಲ್ಲೊ ಏನನ್ನೊ ನೋಡಿದ ಅನುಭವವನ್ನು ಚಿತ್ರ ವಿಚಿತ್ರ ರೀತಿಯಾಗಿ ಊಹಿಸುವುದು. ಆತ್ಮವು ಇದರಂತೆ ಅಲ್ಲ. ಬೇಕಾದರೆ ಹಾಗೆ ಇರಬಹುದು, ಹೀಗೆ ಇರಬಹುದು ಎಂದು ಉಪಮಾನವನ್ನು ಕೊಡಬಹುದೆ ಹೊರತು, ಅದೇ ಇದು ಎಂದು ಹೇಳಲಾಗುವುದಿಲ್ಲ. ಅನೇಕವೇಳೆ ನಾವು ಆಕಾಶ ಸಮುದ್ರ ವಿದ್ಯುತ್ ಗಾಳಿ ಈ ಉದಾಹರಣೆಗಳನ್ನು ಕೊಡುವೆವು, ತುಂಬಾ ಸೂಕ್ಷ ್ಮವಾದುದನ್ನು ಗ್ರಹಿಸಲು ಸಾಧ್ಯವಾಗಲಿ ಎಂದು. ಆದರೆ ಈ ಉದಾಹರಣೆಗಳೆಲ್ಲಾ ಜಡ, ಚೈತನ್ಯವಲ್ಲ. ಹಾಗಾದರೆ ಕಿವಿ ಕೇಳದ, ಕೈ ಮುಟ್ಟದ, ಕಣ್ಣು ನೋಡದ, ನಾಲಿಗೆ ರುಚಿನೋಡದ, ಮೂಗು ಮೂಸಿ ನೋಡದ, ಕಲ್ಪಿಸಿಕೊಳ್ಳುವುದಕ್ಕೆ ಸಾಧ್ಯವಾಗದ ಒಂದು ವಸ್ತು ಇದೆ ಎಂದು ಏತಕ್ಕಾದರೂ ಊಹಿಸಬೇಕು? ಮನಸ್ಸಿನಿಂದ ಹಿಡಿದು ಉಳಿದವುಗಳೆಲ್ಲಾ ಕೆಲಸ ಮಾಡಬೇಕಾದರೆ ಹಿಂದೆ ಆತ್ಮನಿರಬೇಕು. ಆತ್ಮ ಇದರಿಂದ ಬೇರೆ ಆದೊಡನೆ ವಿದ್ಯುತ್​ಶಕ್ತಿ ಹರಿಯುವುದು ನಿಂತರೆ ರೇಡಿಯೊ ತೆಪ್ಪಗಾಗುವಂತೆ ಇಂದ್ರಿಯಗಳೆಲ್ಲಾ ತೆಪ್ಪಗಾಗುವುವು. ಅವು ಕೆಲಸ ಮಾಡಬೇಕಾದರೆ ಹಿಂದೆ ಆತ್ಮನಿರಬೇಕು. ಆದರೆ ಆತ್ಮನಿರಬೇಕಾದರೆ ಇವುಗಳಾವುವೂ ಇರಬೇಕಾಗಿಲ್ಲ. ಅಲೆಗಳಿರಬೇಕಾದರೆ ಸಾಗರ ಇರಬೇಕು. ಆದರೆ ಸಾಗರ ಇರಬೇಕಾದರೆ ಅಲೆಗಳ ಹಂಗೇ ಬೇಕಿಲ್ಲ. ಆತ್ಮಕ್ಕೆ ಇತರ ಪ್ರಮಾಣಗಳು ಅನಾವಶ್ಯಕ. ಅದು ಸ್ವತಃಸಿದ್ಧವಸ್ತು. ಅದರ ಆಧಾರದ ಮೇಲೆ ಇವುಗಳೆಲ್ಲಾ ಇರುವುದು. ಆತ್ಮಕ್ಕೆ ಯಾವ ಆಧಾರವೂ ಬೇಕಿಲ್ಲ.

ಅರ್ಜುನ ಗುರು ಹಿರಿಯರ ದೇಹಗಳು ನಾಶವಾದರೆ ಅದರ ಹಿಂದೆ ಇರುವವರೂ ನಾಶವಾಗು\-ತ್ತಾರೆ ಎಂದು ಭಾವಿಸಿದ್ದನಲ್ಲ ಅದನ್ನು ಬಿಡಿಸುವುದಕ್ಕೆ ಹೀಗೆ ಶ‍್ರೀಕೃಷ್ಣ ಹೇಳಿ ಇನ್ನುಮೇಲಾದರೂ ನಿಶ್ಚಿಂತೆಯಿಂದ ಯುದ್ಧವನ್ನು ಮಾಡು ಎನ್ನುವನು.

\begin{shloka}
ಯ ಏನಂ ವೇತ್ತಿ ಹಂತಾರಂ ಯಶ್ಚೈನಂ ಮನ್ಯತೇ ಹತಮ್~।\\ಉಭೌ ತೌ ನ ವಿಜಾನೀತೋ ನಾಯಂ ಹಂತಿ ಹನ್ಯತೇ \hfill॥ ೧೯~॥
\end{shloka}

\begin{artha}
ಈತನು ಕೊಲ್ಲುವವನೆಂದು ಯಾರು ತಿಳಿಯುತ್ತಾರೆಯೋ ಮತ್ತು ಈತನು ಕೊಲ್ಲಲ್ಪಡತಕ್ಕವನು ಎಂದು ಯಾರು ತಿಳಿಯುತ್ತಾರೆಯೋ ಅವರಿಬ್ಬರೂ ಅರಿಯರು. 
 ಈತನು ಕೊಲ್ಲುವುದೂ ಇಲ್ಲ; ಕೊಲ್ಲಿಸಿಕೊಳ್ಳುವುದೂ ಇಲ್ಲ.
\end{artha}

ಈತ ಕೊಲ್ಲುವವನಲ್ಲ. ಆತ್ಮ, ದೇಹ ಮನಸ್ಸು ಬುದ್ಧಿ ಇಂದ್ರಿಯಗಳ ಹಿಂದೆ ಇದೆ. ಇದು ಕೇವಲ ಸಾಕ್ಷಿ. ಮಾಡುವುದಲ್ಲ, ಮಾಡುವುದನ್ನು ನೋಡುವುದು. ಹಾಗಾದರೆ ಅಜ್ಞಾನಿಯೂ ಕೂಡ ತಾನು ಒಬ್ಬನನ್ನು ಕೊಂದು ನಿಜವಾದ ನಾನು ಮಾಡುವವನಲ್ಲ ಎನ್ನಬಹುದು. ನ್ಯಾಯಾಧಿ\-ಪತಿಯು ಕೂಡ, ನಿಜವಾದ ನಾನು ಶಿಕ್ಷೆಯನ್ನು ವಿಧಿಸುತ್ತಿಲ್ಲ; ನಿನಗೆ ಉಪಾಧಿಯಾದ ದೇಹಕ್ಕೆ ಮಾತ್ರ ಶಿಕ್ಷೆ ವಿಧಿಸುತ್ತಿರುವೆ. ನಿಜವಾದ ನೀನು ಈ ದೇಹ ಮಾಡಿದ ಕೊಲೆಯನ್ನು ಹೇಗೆ ನೋಡುತ್ತಿತ್ತೊ ಹಾಗೆಯೇ ಈ ದೇಹಕ್ಕೆ ವಿಧಿಸಿರುವ ಶಿಕ್ಷೆಯನ್ನು ನೋಡುತ್ತಿರು ಎನ್ನಬಹುದು.

ಹಾಗೆಯೆ ಅವನು ಕೊಲ್ಲಲ್ಪಟ್ಟನು ಎಂದರೆ ಚೈತನ್ಯ ಯಾವ ಮಧ್ಯವರ್ತಿ ಮೂಲಕ ವ್ಯಕ್ತವಾಗು ತ್ತಿತ್ತೊ ಆ ಮಧ್ಯವರ್ತಿ ಹಾಳಾಯಿತು ಎಂದು ಅರ್ಥವೇ ಹೊರತು ಅದರ ಹಿಂದೆ ಇರುವ ಚೈತನ್ಯವಲ್ಲ. ವಿದ್ಯುತ್​ಶಕ್ತಿ ಬಲ್ಬಿನ ಮೂಲಕ ಕಾಂತಿಯನ್ನು ವ್ಯಕ್ತಮಾಡುವುದು. ಆ ಬಲ್ಬನ್ನು ಒಡೆದುಹಾಕಿದರೆ ವಿದ್ಯುತ್​ಶಕ್ತಿಯನ್ನು ಹಾಳುಮಾಡಿದಂತೆ ಆಗುವುದಿಲ್ಲ. ಜ್ಞಾನಿ ಇದನ್ನು ಚೆನ್ನಾಗಿ ತಿಳಿದಿರುವನು. ಇದನ್ನು ತಿಳಿಯದೆ ಇದ್ದರೆ ಅವನು ಅಜ್ಞಾನಿ. ಆತ್ಮ ಮತ್ತೊಂದನ್ನು ಕೊಲ್ಲಲಾರದು. ಹಾಗೆ ಮಾಡುವುದು ಅದರ ಮೇಲೆ ಆರೋಪ ಮಾಡಿದ ಅಜ್ಞಾನದಿಂದ ಕೂಡಿದ ಉಪಾಧಿಗಳು. ಅದು ಮಾಡುವುದಾದರೂ ಏನು? ಮತ್ತೊಂದು ಜೀವವನ್ನು ಇಲ್ಲದಂತೆ ಮಾಡಲಾರದು. ಜೀವಕ್ಕೂ ದೇಹಕ್ಕೂ ಇರುವ ಸಂಬಂಧ ತಪ್ಪುವುದು ಅಷ್ಟೆ. ಕೊಲ್ಲಲ್ಪಡುವುದು ಎಂದರೆ ಸರ್ವನಾಶವಾಗಿ ಹೋಗುವುದು ಎಂದಲ್ಲ. ಅದು ಬೇರೆಯಾಗುವುದು.

\begin{shloka}
ನ ಜಾಯತೇ ಮ್ರಿಯತೇ ವಾ ಕದಾಚಿ—\\ನ್ನಾಯಂ ಭೂತ್ವಾ ಭವಿತಾ ವಾ ನ ಭೂಯಃ~।\\ಅಜೋ ನಿತ್ಯಃ ಶಾಶ್ವತೋsಯಂ ಪುರಾಣೋ\\ನ ಹನ್ಯತೇ ಹನ್ಯಮಾನೇ ಶರೀರೇ \hfill॥ ೨ಂ~॥
\end{shloka}

\begin{artha}
 ಎಂದಿಗೂ ಹುಟ್ಟುವುದೂ ಇಲ್ಲ. ಸಾಯುವುದೂ ಇಲ್ಲ. ಇವನು ಮೊದಲು ಇದ್ದು ಆಮೇಲೆ ಇಲ್ಲದೇ ಹೋಗುವುದೂ ಇಲ್ಲ. ಇವನು ಅಜ ನಿತ್ಯ ಶಾಶ್ವತ ಪುರಾಣ. ಶರೀರ ಹತವಾಗುತ್ತಿದ್ದರೂ ಇವನು ಹತನಾಗುವುದಿಲ್ಲ.
\end{artha}

ಹುಟ್ಟುವುದು ಎಂದರೆ, ಹಿಂದೆ ಇರಲಿಲ್ಲ, ಈಗ ಇದೆ ಎಂದು ಅರ್ಥ. ಮಗು ಹುಟ್ಟಿದಾಗ ಈ ದೇಹ ಹಿಂದೆ ಇರಲಿಲ್ಲ, ಈಗ ಇದೆ. ನಾವು ನಾಮಕರಣ, ಹುಟ್ಟಿದ ಹಬ್ಬ ಇವುಗಳನ್ನೆಲ್ಲ ಮಾಡುವುದು ಹೊರಗಿನದಕ್ಕೆ; ಅದರ ಒಳಗೆ ಇರುವ ಚೈತನ್ಯಕ್ಕೆ ಅಲ್ಲ. ಅದು ಇಲ್ಲಿಗೆ ಬರುವುದಕ್ಕಿಂತ ಮುಂಚೆಯೂ ಇತ್ತು.

\newpage

ಇವನು ಸಾಯುವುದೂ ಇಲ್ಲ ಎಂದರೆ ದೇಹ ಹೋದರೆ ದೇಹದ ಹಿಂದೆ ಇರುವ ಜೀವ ನಿರ್ನಾಮವಾಗಿ ಹೋಗುವುದಿಲ್ಲ. ಮಡಕೆ ಒಡೆದರೆ ಅದರ ಒಳಗೆ ಇರುವ ಆಕಾಶ ನಾಶವಾಗುವುದಿಲ್ಲ. ಅದರಂತೆಯೇ ಜೀವ.

ಅವನು ಹುಟ್ಟಿ ಇಲ್ಲದಂತೆ ಆಗುವುದಿಲ್ಲ. ಹುಟ್ಟಿರುವಾಗ ಆತ್ಮ ದೇಹ ಇಂದ್ರಿಯ ಮನಸ್ಸು ಬುದ್ಧಿ ಈ ಕರಣಗಳ ಮೂಲಕ ಕೆಲಸ ಮಾಡುತ್ತಿರುತ್ತದೆ. ಈಗ ಚೈತನ್ಯ ಇರುವುದು ಸತ್ಯವಾದರೆ ಯಾವ ಕರಣಗಳ ಮೂಲಕ ಕೆಲಸ ಮಾಡುತ್ತಿದೆಯೋ ಆ ಕರಣಗಳೆಲ್ಲ ಹಾಳಾದರೂ ಅದರ ಹಿಂದೆ ಆ ಚೈತನ್ಯ ಇರಲೇಬೇಕು. ಒಂದು ರೇಡಿಯೋ ಇಟ್ಟುಕೊಂಡು ಸಂಗೀತ ಕೇಳುತ್ತಿರುತ್ತೇವೆ. ಆ ರೇಡಿಯೋ ಕೆಟ್ಟುಹೋದರೆ ಸಂಗೀತ ಕೇಳುವುದಿಲ್ಲ. ಆದರೆ ಆಕಾಶದಲ್ಲಿ ಸಂಗೀತ ಸ್ಪಂದನವೇ ಇಲ್ಲವೆ? ಇದೆ, ಆದರೆ ಅದು ನಮಗೆ ಕೇಳುತ್ತಿಲ್ಲ. ಅದರಂತೆಯೇ ದೇಹದ ಮೂಲಕ ಕೆಲಸ ಮಾಡುತ್ತಿದ್ದಾಗ ಸ್ಥೂಲವಾಗಿತ್ತು. ದೇಹಾತೀತವಾದ ಮೇಲೆ ಸೂಕ್ಷ್ಮವಾಗುವುದು. ವೈಜ್ಞಾನಿಕವಾಗಿ ನೋಡಿದರೆ ಯಾವ ವಸ್ತುವನ್ನು ನಿರ್ನಾಮ ಅಂದರೆ ಅದಿಲ್ಲದಂತೆ, ಅದನ್ನು ಒಂದು ಶೂನ್ಯ ಸ್ಥಿತಿಗೆ ತೆಗೆದುಕೊಂಡು ಹೋಗುವುದಕ್ಕೆ ಆಗುವುದಿಲ್ಲ. ಸ್ಥೂಲ ದ್ರವ್ಯಗಳಿಗೇ ಇದನ್ನು ನಾವು ಮಾಡಲಾಗುವುದಿಲ್ಲ. ಬೇಕಾದರೆ ಅವನ್ನು ಬೇರೆ ಬೇರೆ ಅವಸ್ಥೆಗೆ ಕಳುಹಿಸಬಹುದಷ್ಟೆ. ಮಡಕೆಯನ್ನು ಒಡೆದರೆ ಯಾವ ಜೇಡಿ ಮಣ್ಣಿನಿಂದ ಇದು ಆಗಿದೆಯೋ ಆ ಜೇಡಿ ಮಣ್ಣು ನಾಶವಾಯಿತೇ? ಅದು ಹಿಂದೆ ಮಡಕೆ ಆಕಾರದಲ್ಲಿತ್ತು. ಈಗ ಒಂದು ಬೊಗಸೆ ಅದರ ಚೂರುಗಳಾಗಿವೆ. ಅದನ್ನು ಪುಡಿ ಮಾಡಿದರೆ ಧೂಳಿಕಣಗಳಾಗುವುದು. ಅದಕ್ಕೆ ಹೆಚ್ಚು ಶಾಖವನ್ನು ತಾಕಿಸಿದರೆ ಅನಿಲ ರೂಪಕ್ಕೆ ಹೋಗುವುದು. ಇನ್ನೂ ಹೆಚ್ಚು ಶಾಖವನ್ನು ಕೊಟ್ಟರೆ ರೇಡಿಯೇಷನ್​ನಂತೆ ಮತ್ತೂ ಸೂಕ್ಷ್ಮವಾಗುವುದು. ಇದು ಜಡ. ಆ ಜಡವನ್ನೇ ನಾವು ಇಲ್ಲದಂತೆ ಮಾಡಲಾಗುವುದಿಲ್ಲ. ಜೀವವಾದರೊ ಸೂಕ್ಷ್ಮಾತಿಸೂಕ್ಷ್ಮ. ಅದು ಪಂಚಭೂತಗಳ ಬಲೆಗೆ ಸಿಕ್ಕುವುದಿಲ್ಲ. ಇನ್ನು ಅದನ್ನು ಮಾಡುವುದೇನು? ಆಕಾಶದಲ್ಲಿರುವ ಹಲವು ವಸ್ತುಗಳ ನಾಮರೂಪಗಳನ್ನು ಕೆಡಿಸಿ ಬೇರೆ ನಾಮರೂಪಗಳನ್ನೆ ಅದಕ್ಕೆ ಕೊಡಬಹುದು. ಆದರೆ ಆಕಾಶಕ್ಕೆ ನಾವು ಏನು ಮಾಡಲು ಸಾಧ್ಯ?

ಈತನಿಗೆ ಹುಟ್ಟು ಸಾವು ಎಂಬುದೇ ಇಲ್ಲ. ನಾವು ಮಾಡುವ ಹುಟ್ಟಿದ ಹಬ್ಬ ಶ್ರಾದ್ಧ ಇವುಗಳೆಲ್ಲ ದೇಹಕ್ಕೆ. ಅದರ ಹಿಂದೆ ಇರುವ ಜೀವ ಇದೇ ಮೊದಲ ಬಾರಿ ಅಲ್ಲ ಬಂದಿದ್ದು. ಅವನು ಎಷ್ಟೋ ವೇಳೆ ಬಂದಿದ್ದಾನೆ. ಇದೇ ಕೊನೆಯ ಬಾರಿ ಅಲ್ಲ ಹೊರಟುಹೋಗುವುದು. ಎಲ್ಲಿಯವರೆಗೆ ಮುಕ್ತನಾಗಿಲ್ಲವೋ ಅಲ್ಲಿಯವರೆಗೆ ಅವನು ಹೋಗುತ್ತಾ ಬರುತ್ತಾ ಇರುವನು. ಪ್ರತಿಯೊಂದು ಜನ್ಮವೂ ಜೀವಿ ಹಾಕುವ ಒಂದು ವೇಷ. ಇಂತಹ ವೇಷಗಳನ್ನೆಲ್ಲ ಎಷ್ಟೋ ಹಿಂದೆ ಹಾಕಿರುವನು, ಮುಂದೆ ಹಾಕಲಿರುವನು. ಅವನು ಶಾಶ್ವತ; ಅಕ್ಕಸಾಲೆಯ ಮನೆಯಲ್ಲಿರುವ ಅಡಿಗಲ್ಲಿನಂತೆ. ಬೆಳಗಿನಿಂದ ಸಾಯಂಕಾಲದವರೆಗೆ ಅಕ್ಕಸಾಲಿಗ ಅಡಿಗಲ್ಲಿನ ಮೇಲೆ ಹಲವು ವಸ್ತುಗಳನ್ನು ಕುಟ್ಟಿ ಬಡಿದು ಹೊಸ ಹೊಸ ರೂಪನ್ನು ಕೊಡುತ್ತಿರುವನು. ಆದರೆ ಅಡಿಗಲ್ಲಾದರೊ ಒಂದೇ ಸಮನಾಗಿರುವುದು. ಅವನು ಒಬ್ಬ; ಆದರೆ ಅನಂತ ವೇಷಗಳನ್ನು ಹಾಕಿಕೊಳ್ಳುವನು. ಆ ವೇಷಗಳಿಗೆ ಒಂದು ಆದಿ\break ಅಂತ್ಯವಿದೆ. ಹಾಕುವವನಿಗೆ ಆದಿಯೂ ಇಲ್ಲ, ಅಂತ್ಯವೂ ಇಲ್ಲ.

ಅವನು ಪುರಾಣಪುರುಷ, ಬಹಳ ಹಳೆಯವನು. ಅವನಿಲ್ಲದ ಕಾಲವೇ ಇಲ್ಲ. ಕಾಲದ ಎಷ್ಟು ಹಿಂದೆ ಹೋದರೂ ಅವನಿರುವನು. ಮೊದಲು ಅವನು, ಆಮೇಲೆ ಕಾಲದೇಶಗಳು. ಮೊದಲು ಕಾಲದೇಶ, ಆಮೇಲೆ ಅವನಲ್ಲ. ಅವನು ಭೂತ ಭವಿಷ್ಯತ್ತುಗಳಿಗೆ ಸಾಕ್ಷಿ. ಶರೀರ ಹುಟ್ಟುವುದು ಬೆಳೆಯುವುದು, ಬದಲಾವಣೆ ಆಗುವುದು, ನಾಶವಾಗುವುದು. ಆದರೆ ಅದರ ಹಿಂದೆ ಇರುವ ಜೀವನಾದರೋ ಶರೀರಕ್ಕಿಂತ ಮುಂಚೆ ಇದ್ದನು. ಶರೀರವೆಂಬುದು ನೀರಿನ ಮೇಲಿರುವ ಯಾವುದೋ ಒಂದು ಗುಳ್ಳೆ. ಗುಳ್ಳೆ ಇಲ್ಲದೆ ನೀರು ಇರಬಲ್ಲುದು, ಆದರೆ ನೀರಿಲ್ಲದೆ ಗುಳ್ಳೆ ಇರಲಾರದು. ಗುಳ್ಳೆ ನಾಶವಾದರೆ ನೀರು ನಾಶವಾಗುವುದಿಲ್ಲ.

\begin{shloka}
ವೇದಾವಿನಾಶಿನಂ ನಿತ್ಯಂ ಯ ಏನಮಜಮವ್ಯಯಮ್~।\\ಕಥಂ ಸ ಪುರುಷಃ ಪಾರ್ಥ ಕಂ ಘಾತಯತಿ ಹಂತಿ ಕಮ್ \hfill॥ ೨೧~॥
\end{shloka}

\begin{artha}
ಪಾರ್ಥ, ಈತನು ಅವಿನಾಶಿ ನಿತ್ಯ ಅಜ ಅವ್ಯಯ ಎಂದು ಯಾರು ತಿಳಿದುಕೊಂಡಿರುವನೋ ಅವನು ಯಾರನ್ನು ಹೇಗೆ ಕೊಲ್ಲಿಸುತ್ತಾನೆ, ಯಾರನ್ನು ಹೇಗೆ ಕೊಲ್ಲುತ್ತಾನೆ?
\end{artha}

ಯಾರು ಆತ್ಮನ ನೈಜಸ್ಥಿತಿಯನ್ನು ಅರಿತಿರುವರೋ ಅವರು ಇನ್ನೊಬ್ಬರನ್ನು ಕೊಲ್ಲಿಸುವುದಕ್ಕೆ ಹೋಗುವುದಿಲ್ಲ. ಅಥವಾ ತಾವೆ ಕೊಲ್ಲುವುದೂ ಇಲ್ಲ. ಆತ್ಮವನ್ನು ಏನು ಮಾಡಿದರೂ ನಾಶಮಾಡುವುದಕ್ಕೆ ಆಗುವುದಿಲ್ಲ. ಹುತ್ತ ಬಡಿದರೆ ಹಾವು ಸಾಯುವುದೆ ಎಂಬ ಗಾದೆ ಇದೆ. ಈ ದೇಹ ಒಂದು ಹುತ್ತ. ಇದರಲ್ಲಿರುವ ಹಾವೇ ಜೀವ. ಮೇಲುಗಡೆ ಬೇಕಾದಷ್ಟು ಹುತ್ತವನ್ನು ಬಡಿಯಬಹುದು. ಆದರೆ ಒಳಗೆ ಇರುವ ಹಾವಿಗೆ ಏನೂ ಆಗುವುದಿಲ್ಲ. ಅವನು ನಿತ್ಯ. ಯಾವಾಗಲೂ ಇರುವವನು. ಈ ಜೀವ ನಾಮರೂಪಗಳ ವೇಷ ಹಾಕಿಕೊಳ್ಳುವುದಕ್ಕೆ ಮುಂಚೆ ಇದ್ದ. ಹಾಕಿಕೊಂಡ ವೇಷವೆಲ್ಲ ಅಳಿಸಿ ಹೋದರೂ ಅವನು ಇರುವನು. ಅವನು ಅಜ, ಹುಟ್ಟಿಯೇ ಇಲ್ಲ. ಹುಟ್ಟು ಸಾವು ಎಂಬುದು ವ್ಯಾವಹಾರಿಕ ಜಗತ್ತಿನಲ್ಲಿ ನಾವು ಉಪಯೋಗಿಸುವ ಪದ. ಸೂರ್ಯ ಹುಟ್ಟುತ್ತಾನೆ, ಮುಳುಗುತ್ತಾನೆ ಎನ್ನುತ್ತೇವಲ್ಲ ಹಾಗೆ. ಸೂರ್ಯ ಹುಟ್ಟುವುದಕ್ಕೆ ಮುಂಚೆ ಇರಲಿಲ್ಲವೆ? ಮುಳುಗಿಹೋದ ಮೇಲೆ ಇರುವುದಿಲ್ಲವೆ? ಅವನು ಹುಟ್ಟುವುದಕ್ಕೆ ಮುಂಚೆ ಮತ್ತಾವುದೋ ದೇಶದ ಮೇಲೆ ಬೆಳಗುತ್ತಿದ್ದ. ಅವನು ಮುಳುಗಿಹೋದ ಮೇಲೆ ಮತ್ತಾವುದೋ ದೇಶದ ಮೇಲೆ ಬೆಳಗುತ್ತಿರುವನು. ಆತ್ಮ ಯಾವ ವಿಕಾರಕ್ಕೂ ಸಿಕ್ಕಿಲ್ಲ. ಈ ದೇಹಕ್ಕೆ ಬಾಲ್ಯ ಯೌವನ ವೃದ್ಧಾಪ್ಯಗಳು, ಅದರ ಹಿಂದೆ ಇರುವ ಆತ್ಮನಿಗಲ್ಲ.

ಆತ್ಮನ ನೈಜಸ್ವಭಾವವನ್ನು ಯಾರು ಅರಿತಿರುವರೋ ಅವನು ಪರಿಪೂರ್ಣನಾಗಿ ಹೋಗುತ್ತಾನೆ. ಅವನು ಮತ್ತೊಂದು ವಸ್ತುಗಳ ಮತ್ತು ವ್ಯಕ್ತಿಗಳ ಸಹಾಯದಿಂದ ಇತರರನ್ನು ಕೊಲ್ಲಿಸುವುದಿಲ್ಲ ಅಥವಾ ಅವರ ಸಹಾಯವಿಲ್ಲದೆ ತಾನೇ ಆ ಕೆಲಸವನ್ನು ಮಾಡುವುದೂ ಇಲ್ಲ.

\begin{shloka}
ವಾಸಾಂಸಿ ಜೀರ್ಣಾನಿ ಯಥಾ ವಿಹಾಯ\\ನವಾನಿ ಗೃಹ್ಣಾತಿ ನರೋsಪರಾಣಿ~।\\ತಥಾ ಶರೀರಾಣಿ ವಿಹಾಯ ಜೀರ್ಣಾ—\\ನ್ಯನ್ಯಾನಿ ಸಂಯಾತಿ ನವಾನಿ ದೇಹೀ \hfill॥ ೨೨~॥
\end{shloka}

\begin{artha}
ಮನುಷ್ಯ ಹೇಗೆ ಹರಿದ ಬಟ್ಟೆಯನ್ನು ಬಿಟ್ಟು ಬೇರೆ ಹೊಸಬಟ್ಟೆಯನ್ನು ಧರಿಸುತ್ತಾನೋ ಹಾಗೆಯೇ ದೇಹಿ, ಜೀರ್ಣವಾದ ಶರೀರವನ್ನು ಬಿಟ್ಟು ಹೊಸ ದೇಹವನ್ನು ಧರಿಸುತ್ತಾನೆ.
\end{artha}

ಸಾವೆಂದರೆ ಏನೊ ಭಯಂಕರವಾದ ಚಿತ್ರವನ್ನು ನಾವು ಕಟ್ಟಿಕೊಳ್ಳುತ್ತೇವೆ. ಶ‍್ರೀಕೃಷ್ಣ ಅದನ್ನು ಎಷ್ಟು ಸ್ವಾಭಾವಿಕವಾಗಿ ಮಾಡಿರುವನು. ಅದರಲ್ಲಿರುವ ಭಯ ಕ್ರೌರ್ಯ ಎಲ್ಲವನ್ನೂ ತೆಗೆದುಹಾಕಿರುವನು. ಬಟ್ಟೆ ಹಳೆಯದಾದರೆ ಅದನ್ನು ಎಸೆದು ಹೊಸದನ್ನು ಧರಿಸುತ್ತೇವೆ. ಅದನ್ನು ಬಿಡುವಾಗ ಅಳುತ್ತೇವೆಯೆ ಇಷ್ಟೊಂದು ದಿನಗಳು ಇದನ್ನು ಹಾಕಿಕೊಂಡಿದ್ದೆನಲ್ಲ ಎಂದು? ಅದರಂತೆಯೆ ದೇಹ ಎಂಬ ಯಂತ್ರ. ನಮಗಾಗಿ ದುಡಿದು ದುಡಿದು ಅದು ಸವಕಲಾಗುವುದು. ಸಾಧ್ಯವಾದಷ್ಟು ಅದನ್ನು ರಿಪೇರಿ ಮಾಡುವೆವು. ಪ್ರಪಂಚದ ಸಂತೆಯಲ್ಲಿ ಅದಕ್ಕೆ ಬಿಡಿಭಾಗಗಳು ಸಿಕ್ಕಿದಷ್ಟನ್ನೆಲ್ಲಾ ತೆಗೆದುಕೊಳ್ಳುತ್ತೇವೆ. ಆದರೆ ಈ ದೇಹವನ್ನು ಶಾಶ್ವತ ಮಾಡಲಾಗುವುದಿಲ್ಲ. ಒಂದು ಕಡೆ ರಿಪೇರಿ ಮಾಡಿದರೆ ಮತ್ತೊಂದು ಕಡೆ ರಿಪೇರಿಗೆ ಕರೆಯುತ್ತಿರುವುದು. ಕೊನೆಗೆ ಅನ್ನಿಸುವುದು ಈ ರಿಪೇರಿ ಮಾಡುವ ಕೆಲಸಕ್ಕಿಂತ ಹೊಸ ಮನೆ ಕಟ್ಟಿಸುವುದು ಸುಲಭ ಎಂದು. ಹೊಸ ಮನೆ ಕಟ್ಟಿಸುವುದು ಎಂದರೆ ಹಳೆಯದನ್ನು ನಾಶಮಾಡಿ ಹೊಸ ಮನೆ ಕಟ್ಟಿಕೊಳ್ಳುವುದು. ಹಾವು ತನ್ನ ಹಳೆಯ ಪರೆಯನ್ನು ಬಿಡುವುದು. ಹೊಸದಾಗಲೆ ಅದರ ಹಿಂದೆ ಇದೆ. ಆದರೆ ಅದು ತುಂಬಾ ಸೂಕ್ಷ ್ಮವಾಗಿದೆ. ಅದ ರಂತೆಯೇ ಮನುಷ್ಯ ಮುಂದಿನದನ್ನು ಸೃಷ್ಟಿಸಿಕೊಂಡು ಹಿಂದಿನದನ್ನು ತೊರೆಯುವನು. ಕೀಟ ತೆವಳಿಕೊಂಡು ಹೋಗುತ್ತಿರುವಾಗ ಮುಂದಿನದನ್ನು ಹಿಡಿದುಕೊಂಡು ಹಿಂದಿನದನ್ನು ಬಿಡುವಂತೆ. ನಾವು ಏಣಿಯನ್ನು ಹತ್ತುವಾಗ ಮೇಲಿನ ಮೆಟ್ಟಿಲನ್ನು ಹಿಡಿದುಕೊಂಡು ಕೆಳಗಿನ ಮೆಟ್ಟಿಲನ್ನು ಬಿಡುವಂತೆ. ಜಾಗ್ರತಾವಸ್ಥೆಯಲ್ಲಿ ಸುತ್ತಮುತ್ತಲಿರುವುದನ್ನೆಲ್ಲ ಅಷ್ಟೊಂದು ಪ್ರೀತಿಸುತ್ತಾ ಇರುತ್ತೇವೆ. ಮಲಗಿ ಸ್ವಪ್ನಲೋಕಕ್ಕೆ ಹೋಗುವಾಗ ಅಲ್ಲಿ ಆಗುವ ಅನುಭವವೇ ಬೇರೆ. ಆದರೂ ಅದು ಸ್ವಾಭಾವಿಕವಾಗುವುದು. ಅದರಂತೆಯೇ ಕನಸಿನಲ್ಲಿ ಏನೇನೋ ನೋಡಿ ಅನುಭವಿಸುತ್ತಾ ಇರುತ್ತೇವೆ. ಅಲ್ಲಿಂದ ಎದ್ದ ತಕ್ಷಣವೇ ನಾವು ಕನಸಿಗೆ ಅಂಟಿಕೊಂಡಿರುವುದಿಲ್ಲ. ಜಾಗ್ರತಾವಸ್ಥೆಯಿಂದ ಕನಸಿಗೆ ಹೋಗುವುದು, ಕನಸಿನಿಂದ ಜಾಗ್ರತಾವಸ್ಥೆಗೆ ಬರುವುದು ನಾವು ಪ್ರತಿದಿನ ಮಾಡುತ್ತಿರುವ ಸಂಚಾರ. ಇದು ಸ್ವಾಭಾವಿಕವಾದುದು. ಮನೆಯಿಂದ ಆಫೀಸಿಗೆ ಹೋಗುವುದು, ಆಫೀಸಿನಿಂದ ಮನೆಗೆ ಬರುವಂತೆ. ಇದರಷ್ಟೇ ಸ್ವಾಭಾವಿಕವಾದುದು ಹಳೆಯ ದೇಹವನ್ನು ತ್ಯಜಿಸುವುದು, ಹೊಸ ದೇಹವನ್ನು ಧರಿಸುವುದು. ಆದರೆ ವ್ಯಾವಹಾರಿಕ ಜಗತ್ತಿನಲ್ಲಿ ಈ ಸ್ವಾಭಾವಿಕ ಘಟನೆ ಹಿಂದೆ ಎಷ್ಟೊಂದು ಅಳು ನಗು ಇವುಗಳೆಲ್ಲ ಹುದುಗಿರುವುದು! ಸತ್ತರೆ ಅಳು ಗೋಳು ಸುತ್ತಲೂ. ಹುಟ್ಟಿದ ಮನೆಯಲ್ಲಿ ಅಲ್ಲಿಯವರಿಗೆ ಆನಂದ ಅವರಿಗೆ ಒಂದು ಮಗು ಆಯಿತಲ್ಲ ಎಂದು. ಆ ಮಗುವೋ ಅಳುವುದು, ಇನ್ನೊಂದು ಸಲ ಬಂದೆನಲ್ಲ ಈ ಸಂಸಾರಕ್ಕೆ, ಇನ್ನೆಷ್ಟು ಅನುಭವಿಸುವುದು ಕಾದಿದೆಯೊ ಎಂದು. ಈ ಅಳುವನ್ನು ನೋಡಿ ಹೆತ್ತವರಿಗೆ ಹಿಗ್ಗು. ಮಗು ಬದುಕಿದೆಯೆಂದು ಈ ಅಳುವಿನಿಂದಲೇ ಅವರು ನಿಶ್ಚಯಿಸುವರು. ಪಡಬಾರದ ಯಾತನೆಯನ್ನೆಲ್ಲ ಪಟ್ಟು ಸದ್ಯಕ್ಕೆ ಈ ದೇಹದ ನರಕದಿಂದ ಬಿಡುಗಡೆ ಆಯಿತಲ್ಲ ಇಂದು ಎಂದು ಆತ್ಮ ದೇಹದಿಂದ ಬೇರೆ ಆಗುವಾಗ ನಿಟ್ಟುಸಿರುಬಿಡುವುದು. ಆದರೆ ಅವನನ್ನೇ ನೆಚ್ಚಿದ ಬಂಧು ಬಳಗದವರೆಲ್ಲ, ಅಯ್ಯೊ ನಮ್ಮನ್ನು ಅಗಲಿ ಹೋದನಲ್ಲ ಎಂದು ಹೃದಯ ಕರಗುವಂತೆ ಗೋಳಿಡುವರು. ಶ‍್ರೀಕೃಷ್ಣ ಇಲ್ಲಿ ದೊಡ್ಡ ಒಂದು ತಾತ್ತ್ವಿಕ ದೃಷ್ಟಿಯನ್ನು ನೀಡುವನು. ಇದರ ಮೇಲೆ ನಿಂತು ನೋಡಿದರೆ ಜನನ ಮರಣಗಳೊಂದು ಸ್ವಾಭಾವಿಕ ಘಟನೆಯಾಗಿ ಕಾಣುವುದು.

\begin{shloka}
ನೈನಂ ಛಿಂದಂತಿ ಶಸ್ತ್ರಾಣಿ ನೈನಂ ದಹತಿ ಪಾವಕಃ~।\\ನ ಚೈನಂ ಕ್ಲೇದಯಂತ್ಯಾಪೋ ನ ಶೋಷಯತಿ ಮಾರುತಃ \hfill॥ ೨೩~॥
\end{shloka}

\begin{artha}
ಆತ್ಮನನ್ನು ಶಸ್ತ್ರಗಳು ಕತ್ತರಿಸಲಾರವು, ಅಗ್ನಿ ಸುಡಲಾರದು, ನೀರು ನೆನೆಯಿಸಲಾರದು, ಗಾಳಿ ಒಣಗಿಸಲಾರದು.
\end{artha}

ಆತ್ಮನ ಧರ್ಮವೇ ಬೇರೆ, ದೇಹದ ಧರ್ಮವೇ ಬೇರೆ. ದೇಹ ಪಾಂಚಭೌತಿಕವಾದುದು. ಆತ್ಮವಾದರೊ ದೇಶಕಾಲಾತೀತವಾದುದು. ದೇಹವನ್ನು ಬೇಕಾದರೆ ಕತ್ತಿಯಿಂದ ಕತ್ತರಿಸಬಹುದು. ಬೆಂಕಿಯಿಂದ ಸುಡಬಹುದು, ನೀರಿನಿಂದ ನೆನೆಯಿಸಬಹುದು. ಗಾಳಿಯಿಂದ ಒಣಗಿಸಬಹುದು. ಆದರೆ ಅದರ ಹಿಂದೆ ಇರುವ ಆಕಾಶದಂತಿರುವ ಆತ್ಮವನ್ನು ಏನೂ ಮಾಡುವುದಕ್ಕೆ ಆಗುವುದಿಲ್ಲ.

\begin{shloka}
ಅಚ್ಛೇದ್ಯೋsಯಮದಹ್ಯೋsಯಮಕ್ಲೇದ್ಯೋsಶೋಷ್ಯ ಏವ ಚ~।\\ನಿತ್ಯಃ ಸರ್ವಗತಃ ಸ್ಥಾಣುರಚಲೋsಯಂ ಸನಾತನಃ \hfill॥ ೨೪~॥~॥
\end{shloka}

\begin{artha}
ಇವನನ್ನು ಕತ್ತರಿಸುವುದಕ್ಕಾಗುವುದಿಲ್ಲ. ಸುಡುವುದಕ್ಕಾಗುವುದಿಲ್ಲ. ನೆನೆಸುವುದಕ್ಕಾಗುವುದಿಲ್ಲ. ಒಣಗಿಸುವುದ ಕ್ಕಾಗುವುದಿಲ್ಲ. ಇವನು ನಿತ್ಯ ಸರ್ವಗತ ಸ್ಥಾಣು ಅಚಲ ಸನಾತನ.
\end{artha}

ಪಂಚಭೂತಗಳು ಆತ್ಮನನ್ನು ಏನೂ ಮಾಡಲಾರವು. ಆತ್ಮ ದೇಹವಲ್ಲ. ದೇಹವನ್ನು ಉಪಯೋಗಿಸುವವನು. ದೇಹಕ್ಕೆ ಏನಾದರೇನಂತೆ. ಅದರ ಹಿಂದೆ ಇರುವ ಆತ್ಮಕ್ಕೆ ಏನೂ ಬಾಧಕವಿಲ್ಲ. ನೆರಳಿಗೆ ಬೆಂಕಿ ಇಟ್ಟರೆ ಯಾರ ನೆರಳೊ ಅವನಿಗೆ ಶಾಖ ತಾಕುವುದೆ? ಅದರಂತೆಯೆ ಆತ್ಮ ಮತ್ತು ದೇಹ.

ಅವನು ನಿತ್ಯ. ಯಾವ ಬದಲಾವಣೆಗೂ ಸಿಕ್ಕದವನು. ಸಿನಿಮದಲ್ಲಿ ತೆರೆಯ ಮೇಲೆ ಏನೇನನ್ನೊ ನೋಡುವೆವು. ಆದರೆ ಆ ತೆರೆ ಮಾತ್ರ ಯಾವ ಬದಲಾವಣೆಯಿಂದಲೂ ಬಾಧಿತವಾಗುವುದಿಲ್ಲ. ತೆರೆಯ ಮೇಲೆ ದೊಡ್ಡ ಸಮುದ್ರ, ಕಾಡ್ಗಿಚ್ಚು ಮುಂತಾದುವೆಲ್ಲ ಕಂಡರೂ ಅದು ಇದರಿಂದ ಬಾಧಿತವಾಗುವುದಿಲ್ಲ. ಅವನು ಸರ್ವಗತ, ಎಲ್ಲ ಕಡೆಯೂ ಇರುವವನು, ಸರ್ವವನ್ನೂ ವ್ಯಾಪಿಸಿಕೊಂಡಿರುವವನು. ಸ್ಥಾಣು, ಅಂದರೆ ಚಲಿಸದವನು. ಈ ಪ್ರಪಂಚದಲ್ಲಿ ಅದು ವಿನಃ ಬೇರೆ ಎಲ್ಲವೂ ಚಲಿಸುತ್ತದೆ. ಹೇಗೆ ಒಂದು ಉರುಳುವ ಗಾಲಿಯ ಮಧ್ಯದಲ್ಲಿರುವ ಕೋಲು ತಾನೆ ಸುಮ್ಮನೆ ಇರುವುದೋ, ಆದರೂ ಉರುಳುವುದಕ್ಕೆಲ್ಲ ಆಸ್ಪದವಾಗಿದೆಯೋ ಹಾಗೆ. ಅದು ಮಂದರ ಪರ್ವತದಂತೆ ಅಚಲ. ಬೇಕಾದಷ್ಟು ನಾಮರೂಪದ ಮಳೆ ಮತ್ತು ಗಾಳಿ ಬೀಸಬಹುದು. ಅದೆಂದಿಗೂ ಬಾಧಿತವಾಗದೆ ಇರುವುದು. ದೇಹದಲ್ಲಿದ್ದರೆ ಒಂದು ಕಡೆಯಿಂದ ಮತ್ತೊಂದು ಕಡೆಗೆ ಚಲಿಸಬಹುದು. ಯಾವುದು ದೇಹ ಮನಸ್ಸು ಬುದ್ಧಿಗೆ ಅತೀತವೊ ಅದು ಚಲಿಸುವುದಾದರೂ ಎಲ್ಲಿಗೆ? ಅದು ಅಚಲ ಮಾತ್ರವಲ್ಲ ಚಲಿಸುವುದಕ್ಕೆಲ್ಲ ಆಧಾರ ಅದು. ಸನಾತನ, ಎಂದೆಂದಿಗೂ ಇರುವುದು. ಅದು ಮನುಷ್ಯ ಅಜ್ಞಾನದಲ್ಲಿರುವಾಗ ಇರುವುದು. ತಿಳಿದಾಗ ಏನು ಹೊಸದಾಗಿ ಬರುವುದಿಲ್ಲ, ಆಗಲೆ ಇರುವುದನ್ನು ಬುದ್ಧಿಯಿಂದ ಅರಿಯುವನು. ದೇಹಾದಿಗಳೆಲ್ಲ ಹೋದಮೇಲೆ ಅದು ಇರುವುದು. ಹಿಂದೆ ಈಗ ಮುಂದೆ ನಾಶವಾಗುವ ವಸ್ತುವಿನ ಮಧ್ಯದಲ್ಲಿ ನಾಶವಾಗದೆ ಉಳಿಯುವುದು.

\begin{shloka}
ಅವ್ಯಕ್ತೋsಯಮಚಿಂತ್ಯೋsಯಮವಿಕಾರ್ಯೋsಯಮುಚ್ಯತೇ~।\\ತಸ್ಮಾದೇವಂ ವಿದಿತ್ವೈನಂ ನಾನುಶೋಚಿತುಮರ್ಹಸಿ \hfill॥ ೨೫~॥
\end{shloka}

\begin{artha}
ಈತನು ಅವ್ಯಕ್ತ, ಅಚಿಂತ್ಯ, ಅವಿಕಾರಿ ಎನಿಸಿರುವನು. ಆದುದರಿಂದ ಇದನ್ನು ಹೀಗೆ ತಿಳಿದುಕೊಂಡು ನೀನು ಶೋಕಿಸಬಾರದು.
\end{artha}

ಈ ಶ್ಲೋಕದಲ್ಲಿ ಶ‍್ರೀಕೃಷ್ಣ ತಾತ್ತ್ವಿಕ ದೃಷ್ಟಿಯಿಂದ ಹೇಳುವ ವಾದವನ್ನು ಪೂರೈಸುವನು. ಇವನು ಅವ್ಯಕ್ತ ಎಂದರೆ ದೃಶ್ಯವಸ್ತುವಿನಂತೆ ನಮಗೆ ಕಾಣುವುದಿಲ್ಲ. ನಾವು ಒಂದು ದೇಹವನ್ನು ನೋಡಬಹುದು. ಆದರೆ ಹಿಂದೆ ಇರುವವನನ್ನು ನೋಡುವುದಕ್ಕೆ ಆಗುವುದಿಲ್ಲ. ಅವನನ್ನು ಊಹಿಸಬೇಕಾಗುವುದು. ನಾವು ಹೊರಗಿನ ದೃಶ್ಯವಸ್ತುವನ್ನು ಗ್ರಹಿಸಿಕೊಳ್ಳುವುದಕ್ಕೆ ಪಂಚೇಂದ್ರಿಯಗಳಿವೆ. ಅವು ಸ್ಥೂಲ ವಿಷಯಗಳನ್ನು ಮಾತ್ರ ಹಿಡಿಯಬಲ್ಲುವು. ಬಲೆಯನ್ನು ನೀರಿನಲ್ಲಿ ಬೀಸಿ ಮನುಷ್ಯ ಮೀನುಗಳನ್ನು ಹಿಡಿಯುತ್ತಾನೆ. ಆ ಬಲೆಯನ್ನು ಬೀಸಿ ಗಾಳಿಯನ್ನು ಹಿಡಿಯುವುದಕ್ಕೆ ಆಗುವುದೆ? ಗಾಳಿ ಒಂದು ಕಡೆಯಿಂದ ಬಂದು ಮತ್ತೊಂದು ಕಡೆ ಹೊರಟು ಹೋಗುವುದು.

ಅವನು ಅಚಿಂತ್ಯ; ಎಂದರೆ ನಮ್ಮ ಮನಸ್ಸಿನಲ್ಲಿ ಅವನನ್ನು ಹೀಗೆ ಎಂದು ಚಿಂತಿಸುವುದಕ್ಕೆ ಆಗುವುದಿಲ್ಲ. ನಾವು ಮನಸ್ಸಿನಲ್ಲಿ ಏನನ್ನು ಚಿಂತಿಸಬೇಕಾದರೂ ಹಿಂದೆ ಅದನ್ನು ಎಲ್ಲಿಯೊ ಕೇಳಿದ್ದು ನೋಡಿದ್ದು ಮೂಸಿದ್ದು ಮುಟ್ಟಿದ್ದು ರುಚಿ ನೋಡಿದ್ದು ಆಗಿರುವುದು. ಅಥವಾ ಅವು ಸೂಕ್ಷ್ಮವಾದ ವಿಷಯಗಳಾದರೆ, ಅದನ್ನು ಒಳ್ಳೆಯದು ಕೆಟ್ಟದ್ದು, ಸತ್ಯ ಅಸತ್ಯ ಮುಂತಾದ ಗುಣಗಳ ಮೂಲಕ ಕಲ್ಪಿಸಿಕೊಳ್ಳುತ್ತೇವೆ. ಇವೆಲ್ಲ ಮನಸ್ಸಿಗೆ ನಿಲುಕುವಂತಹವು. ಮನಸ್ಸು ಹಾಗೆ ಆತ್ಮನನ್ನು ಕುರಿತು ಆಲೋಚಿಸುವುದಕ್ಕೆ ಆಗುವುದಿಲ್ಲ. ಆತ್ಮ, ದೇಶ ಕಾಲ ನಿಮಿತ್ತವನ್ನು ಮೀರಿ ನಿಂತಿದೆ. ಯಾವ ಮನಸ್ಸಿನಿಂದ ಹೊರಗೆ ಇರುವ ವಸ್ತುವನ್ನು ಮುಟ್ಟುತ್ತೇವೆಯೊ ಅದನ್ನು ತೆಗೆದುಕೊಂಡು ಆತ್ಮವನ್ನು ಹಿಡಿಯುವುದು, ಇಕ್ಕಳದಿಂದ ಗಾಳಿಯನ್ನು ಹಿಡಿಯುವುದಕ್ಕೆ ಪ್ರಯತ್ನಿಸಿದಂತೆ\break ಆಗುವುದು. ವಿಕಾರ ರಹಿತ; ಅವನಲ್ಲಿ ಯಾವ ವಿಕಾರಗಳೂ ಇಲ್ಲ. ಬಾಲ್ಯ ಯೌವನ ವಾರ್ಧಕ್ಯ ಇವೆಲ್ಲ ದೇಹಧರ್ಮಗಳು, ಆತ್ಮನ ಧರ್ಮವಲ್ಲ. ಬಲವಾಗಿರುವುದು, ದುರ್ಬಲವಾಗಿರುವುದು, ಆರೋಗ್ಯವಾಗಿರುವುದು, ಅನಾರೋಗ್ಯದಲ್ಲಿರುವುದು, ತೆಳ್ಳಗೆ ದಪ್ಪಗೆ ಇರುವುದು, ಕೊನೆಗೆ ಹುಟ್ಟುವುದು ಸಾಯುವುದು ಇವುಗಳೆಲ್ಲ ಆತ್ಮ ತೊಟ್ಟುಕೊಂಡಿರುವ ದೇಹವೆಂಬ ಅಂಗಿಯ ಧರ್ಮಗಳು. ಇವು ಬದಲಾಯಿಸುತ್ತವೆ. ಹಿಂದಿರುವ ಅವನಾದರೊ ಯಾವಾಗಲೂ ಒಂದೇ ಸಮನಾಗಿರುವನು. ನಮ್ಮ ಶೋಕಕ್ಕೆಲ್ಲ ಕಾರಣ ಅಜ್ಞಾನ. ಯಾವಾಗ ಜ್ಞಾನದಿಂದ ಅಜ್ಞಾನವನ್ನು ದೂಡುವೆವೊ ಆಗ ಸಮಸ್ಯೆ ನಿವಾರಣೆ ಆಗುವುದು. ಒಂದು ವ್ಯಾವಹಾರಿಕ ದೃಷ್ಟಿ, ಮತ್ತೊಂದು ಪಾರಮಾರ್ಥಿಕ ದೃಷ್ಟಿ. ವ್ಯಾವಹಾರಿಕ ದೃಷ್ಟಿಯಲ್ಲಿ ಹುಟ್ಟುವುದು ಸಾಯುವುದು ಇರುವುದು. ಪಾರಮಾರ್ಥಿಕ ದೃಷ್ಟಿಯಲ್ಲಿ ಅದು ಹೊಸದಾಗಿ ಹುಟ್ಟಲೂ ಇಲ್ಲ, ಅದು ಇನ್ನು ಮೇಲೆ ಸಾಯುವುದೂ ಇಲ್ಲ. ಯುದ್ಧದಲ್ಲಿ ದೇಹಗಳು ನಾಶವಾಗಬಹುದು. ಅದರ ಹಿಂದೆ ಇರುವ ಆತ್ಮ ಎಂದಿಗೂ ನಾಶವಾಗುವುದಿಲ್ಲ. ಅದು ಕರ್ಮ ಸವೆಯದೆ ಇದ್ದರೆ ಒಂದನ್ನು ಬಿಟ್ಟು ಮತ್ತೊಂದಕ್ಕೆ ಹೋಗುವುದು. ಕರ್ಮ ಸವೆದಿದ್ದರೆ ಹನಿ ಸಾಗರದಲ್ಲಿ ಒಂದಾಗುವಂತೆ ಪರಬ್ರಹ್ಮನಲ್ಲಿ ಒಂದಾಗುವುದು. ಹೇಗೆ ನೋಡಿದರೂ ಶೋಕಿಸುವುದಕ್ಕೆ ಕಾರಣವಿಲ್ಲ. ವ್ಯಾವಹಾರಿಕವಾಗಿ ಆದರೂ ಕೊಲ್ಲುವುದು ಆತ್ಮನನ್ನು ಅಲ್ಲ. ಅದು ಧರಿಸಿರುವ ದೇಹವನ್ನು. ಅದನ್ನು ಏತಕ್ಕೆ ಕೊಲೆಮಾಡಬೇಕು, ಹಾಗೆ ಬಿಡಬಹುದಲ್ಲ ಎಂದು ಅರ್ಜುನ ಭಾವಿಸಬಹುದು. ಯುದ್ಧಕ್ಕೆ ಕಾಲು ಕೆರೆಯುತ್ತ ನಿಂತಿರುವವರು ಕೌರವರು, ಪಾಂಡವರಲ್ಲ. ಅವರು ತಮ್ಮ ನಾಶವನ್ನು ತಾವೇ ಮಾಡಿಕೊಳ್ಳುತ್ತಿರುವರು, ಅನ್ಯರಲ್ಲ.

\begin{shloka}
ಅಥ ಚೈನಂ ನಿತ್ಯಜಾತಂ ನಿತ್ಯಂ ವಾ ಮನ್ಯಸೇ ಮೃತಮ್~।\\ತಥಾಪಿ ತ್ವಂ ಮಹಾಬಾಹೋ ನೈವಂ ಶೋಚಿತುಮರ್ಹಸಿ \hfill॥ ೨೬~॥
\end{shloka}

\begin{artha}
ಒಂದು ವೇಳೆ ಈತ ಯಾವಾಗಲೂ ಹುಟ್ಟುತ್ತಿರುವನು, ಯಾವಾಗಲೂ ಸಾಯುತ್ತಿರುವನು ಎಂದು ಭಾವಿಸಿದರೂ ಮಹಾಬಾಹುವೆ, ಆಗಲೂ ನೀನು ಶೋಕಿಸಬಾರದು.
\end{artha}

ಶ‍್ರೀಕೃಷ್ಣ ಇಲ್ಲಿಂದ ಬೇರೆ ಒಂದು ವಾದ ಸರಣಿಯನ್ನು ತೆಗೆದುಕೊಳ್ಳುವನು. ಇದೇ ಲೋಕ ವ್ಯವಹಾರದದೃಷ್ಟಿ. ಅರ್ಜುನ ತಾತ್ತ್ವಿಕದೃಷ್ಟಿಯಿಂದ ಅರ್ಥಮಾಡಿಕೊಳ್ಳದೆ ಹೋದರೆ ದಿನನಿತ್ಯವೂ ನಾವು ನೋಡುವ ದೃಷ್ಟಿಯಿಂದಲಾದರೂ ಅರ್ಥಮಾಡಿಕೊಳ್ಳಬಹುದು. ಜೀವಿಗಳು ಹುಟ್ಟುತ್ತಿರುವರು, ಸಾಯುತ್ತಿರುವರು. ಹುಟ್ಟಿದವರಾರೂ ಶಾಶ್ವತವಾಗಿ ಈ ಪ್ರಪಂಚದಲ್ಲಿ ಇರುವುದಕ್ಕೆ ಆಗುವು ದಿಲ್ಲ. ಎಲ್ಲರೂ ಸತ್ತೇ ತೀರಬೇಕು. ಹುಟ್ಟುವುದು ಒಂದು ಸಾಲ ಮಾಡಿದಂತೆ. ಅದನ್ನು ತೀರಿಸ ಬೇಕಾದರೆ ಸಾಯಬೇಕು. ಕೆಲವರು ಬೇಗ ಹೋಗುವರು, ಮತ್ತೆ ಕೆಲವರು ಸ್ವಲ್ಪ ನಿಧಾನವಾಗಿ ಹೋಗುವರು, ಅಷ್ಟೇ ವ್ಯತ್ಯಾಸ. ಲೋಕದಲ್ಲಿ ಪ್ರಸಿದ್ಧವಾಗಿರುವ ಸತ್ಯ ಇದು. ಯಾರು ಏನು ಮಾಡಿದರೂ ಇದನ್ನು ಬದಲಾಯಿಸುವುದಕ್ಕೆ ಆಗುವುದಿಲ್ಲ. ಈ ಪರಿಸ್ಥಿತಿಯೊಂದಿಗೆ ನಾವು ರಾಜಿ ಮಾಡಿಕೊಳ್ಳಬೇಕಾಗಿದೆ. ಸುಮ್ಮನೆ ಇದಕ್ಕೆ ವಿರೋಧವಾಗಿ ಬಿದ್ದು ಒದ್ದಾಡುತ್ತಿದ್ದರೆ ಪರಿಸ್ಥಿತಿ ಏನೂ ಬದಲಾಯಿಸುವುದಿಲ್ಲ. ವ್ಯಸ್ತರಾಗುವವರು ನಾವೆ.

\begin{shloka}
ಜಾತಸ್ಯ ಹಿ ಧ್ರುವೋ ಮೃತ್ಯುರ್ಧ್ರುವಂ ಜನ್ಮ ಮೃತಸ್ಯ ಚ~।\\ತಸ್ಮಾದಪರಿಹಾರ್ಯೇsಥೆR| ನ ತ್ವಂ ಶೋಚಿತುಮರ್ಹಸಿ \hfill॥ ೨೭~॥
\end{shloka}

\begin{artha}
ಹುಟ್ಟಿದವನಿಗೆ ಮರಣ ನಿಜ. ಸತ್ತವನಿಗೆ ಜನನ ನಿಜ, ಅಲ್ಲವೇ? ಹೀಗಿರುವಾಗ ಪರಿಹರಿಸಲಾಗದ ವಿಷಯದಲ್ಲಿ ನೀನು ಶೋಕಿಸಬಾರದು.
\end{artha}

ಇದು ಘಟನೆಯನ್ನು ಕೇವಲ ವ್ಯಾವಹಾರಿಕ ದೃಷ್ಟಿಯಿಂದ ಪರಿಗಣಿಸುವುದಾಗಿದೆ. ದೇಹಧಾರಿಗಳೆಲ್ಲ ಈ ದೇಹವನ್ನು ಕೆಲವು ಕಾಲ ಆದಮೇಲೆ ತ್ಯಜಿಸಬೇಕಾಗಿದೆ. ನಾವು ಏನು ಮಾಡಿದರೂ ಈ ದೇಹವನ್ನು ಶಾಶ್ವತವಾಗಿಟ್ಟಿರುವುದಕ್ಕೆ ಆಗುವುದಿಲ್ಲ. ಶಡ್​ವಿಕಾರಾತ್ಮಕವಾದ ದೇಹ ಜನನದಿಂದ ಮೊದಲಾದುದು; ಮರಣದಲ್ಲಿ ಪರ್ಯವಸಾನವಾಗಲೇಬೇಕು. ಇದರಂತೆಯೇ ಯಾರು ಹೋಗುವರೋ ಅವರೇನು ಒಂದೇ ಸಲ ಹೊರಟುಹೋಗುವುದಕ್ಕೆ ಆಗುವುದಿಲ್ಲ. ಬರುವ ಟಿಕೇಟನ್ನು ತೆಗೆದುಕೊಂಡು ಹೋದಂತಿದೆ. ನಮ್ಮಲ್ಲಿ ಇನ್ನೂ ಹಲವಾರು ವಾಸನೆಗಳು ಇವೆ. ಅವನ್ನು ತೃಪ್ತಿಪಡಿಸಿಕೊಳ್ಳಬೇಕಾಗಿದೆ. ಅದಕ್ಕಾಗಿ ಒಂದು ಬಾಗಿಲಿನಿಂದ ಹೋಗುತ್ತೇವೆ, ಮತ್ತೊಂದು ಬಾಗಿಲಿನಿಂದ ಬರುತ್ತೇವೆ. ಮರಣ ಎಂಬುದು ಎರಡು ಜೀವನಗಳ ಮಧ್ಯೆ ಒಂದು ವಿರಾಮವಷ್ಟೆ. ದೊಡ್ಡ ಒಂದು ನಾಟಕವನ್ನು ನೋಡುತ್ತಿರುವಾಗ ಮಧ್ಯದಲ್ಲಿ ಪ್ರೇಕ್ಷಕರಿಗೆ ಸ್ವಲ್ಪ ವಿಶ್ರಾಂತಿಯನ್ನು ಕೊಡುತ್ತಾರಲ್ಲ ಹಾಗೆ! ಜೀವನದಲ್ಲಿ ಜಾಣತನವೆ ಯಾವುದನ್ನು ಬದಲಾಯಿಸುವುದಕ್ಕೆ ಆಗುವುದಿಲ್ಲವೋ ಅದನ್ನು ಒಪ್ಪಿಕೊಳ್ಳುವುದು. ಇದರಿಂದ ನಾವು ಎಷ್ಟೋ ಶಕ್ತಿಯನ್ನು ಉಳಿಸಿಕೊಳ್ಳಬಹುದು. ಸುಮ್ಮನೆ ಇದಕ್ಕೆ ವಿರೋಧವಾಗಿ ರೇಗಾಡುತ್ತಿದ್ದರೆ ನಮ್ಮ ಅಶಾಂತಿ ಮತ್ತೂ ಹೆಚ್ಚುವುದು.

\begin{shloka}
ಅವ್ಯಕ್ತಾದೀನಿ ಭೂತಾನಿ ವ್ಯಕ್ತಮಧ್ಯಾನಿ ಭಾರತ~।\\ಅವ್ಯಕ್ತನಿಧನಾನ್ಯೇವ ತತ್ರ ಕಾ ಪರಿದೇವನಾ \hfill॥ ೨೮~॥
\end{shloka}

\begin{artha}
ಭಾರತ, ಜೀವಿಗಳು ಹುಟ್ಟುವುದಕ್ಕೆ ಮುಂಚೆ ಕಾಣಿಸುವುದಿಲ್ಲ. ಮಧ್ಯದಲ್ಲಿ ಹುಟ್ಟಿದ ಮೇಲೆ ಮಾತ್ರ ಅವು ಕಾಣಿಸಿಕೊಳ್ಳುವುವು. ನಾಶವಾದಮೇಲೆ ಪುನಃ ಕಾಣಿಸಿಕೊಳ್ಳುವುದಿಲ್ಲ. ಇದಕ್ಕೆ ಶೋಕವೇಕೆ?
\end{artha}

ಮಾನವ ಜೀವನ ಎಂಬುದು ನೀರಿನ ಮೇಲೆ ಹರಿವ ಗುಳ್ಳೆಯಂತೆ. ಅದು ಏಳುವುದಕ್ಕೆ ಮುಂಚೆ ಇರಲಿಲ್ಲ. ಒಡೆದ ಮೇಲೆ ಇರುವುದಿಲ್ಲ. ಎಲ್ಲೊ ಕೆಲವು ಮಾರು ಹರಿದುಕೊಂಡು ಹೋಗುವಾಗ ಮಾತ್ರ ಇರುವುದು. ಆ ಕೆಲವು ಮಾರು ಹರಿದುಕೊಂಡು ಹೋಗುವಾಗ ಸುತ್ತಮುತ್ತಲಿರುವ ಗುಳ್ಳೆಗಳೊಂದಿಗೆ ಇವರು ನಮ್ಮ ಬಂಧು ಬಳಗ ಮಿತ್ರ ಶತ್ರು ಎಂಬ ಸಂಬಂಧವನ್ನು ಕಲ್ಪಿಸಿಕೊಳ್ಳುತ್ತೇವೆ. ಈ ಸಂಬಂಧವೇನು ಹಿಂದೆ ಇರಲಿಲ್ಲ, ಮುಂದೆ ಇರುವಂತೆ ಇಲ್ಲ. ಮನುಷ್ಯ ಹಿಂದೆ ನೋಡದೆ, ಮುಂದೆ ನೋಡದೆ ಈಗ ಇರುವ ತಾತ್ಕಾಲಿಕ ಸ್ಥಿತಿಯನ್ನು ಪರಮ ಸತ್ಯ ಎಂದು ಭಾವಿಸಿ ವ್ಯಥೆಪಡುವನು. ಬರುವುದಕ್ಕೆ ಮುಂಚಿನ ಸ್ಥಿತಿ, ಹೋದಮೇಲೆ ನಮ್ಮ ಸ್ಥಿತಿ ಇದರೊಡನೆ ಹೋಲಿಸಿದರೆ ಈಗಿನ ಸ್ಥಿತಿ ಕ್ಷಣ ಮಾತ್ರ. ಈ ಕ್ಷಣವನ್ನು ಕುರಿತು ಪರಿತಪಿಸುತ್ತೇವೆ. ಇದನ್ನು ಆವರಿಸಿರುವ ಯುಗವನ್ನು ಗಮನಿಸುವುದೇ ಇಲ್ಲ. ಜ್ಞಾನಿ ಭೂಮವನ್ನು ನೋಡುವನು, ಅಲ್ಪವನ್ನು ಮರೆಯುವನು. ಭೂಮದಲ್ಲಿ ಈ ಅಲ್ಪಗಳೆಲ್ಲವೂ ಸಂಲಗ್ನವಾಗಿವೆ. ಅಜ್ಞಾನಿ ಅಲ್ಪವನ್ನು ಎಷ್ಟೇ ಬಿಗಿಯಾಗಿ ಹಿಡಿದಿದ್ದರೂ ಅದು ಇವನಿಗೆ ಕೈಕೊಡುವುದು, ಇವನಿಂದ ನುಸುಳಿಕೊಂಡು ಹೋಗುವುದು.

\begin{shloka}
ಆಶ್ಚರ್ಯವತ್ ಪಶ್ಯತಿ ಕಶ್ಚಿದೇನಮಾಶ್ಚರ್ಯವದ್ವದತಿ ತಥೈವ ಚಾನ್ಯಃ~।\\ಆಶ್ಚರ್ಯವಚ್ಚೈನಮನ್ಯಃ ಶೃಣೋತಿ ಶ್ರುತ್ವಾಪ್ಯೇನಂ ವೇದ ನ ಚೈವ ಕಶ್ಚಿತ್ \hfill॥ ೨೯~॥
\end{shloka}

\begin{artha}
ಒಬ್ಬ ಆತ್ಮನನ್ನು ಆಶ್ಚರ್ಯದಂತೆ ಕಾಣುತ್ತಾನೆ. ಇನ್ನೊಬ್ಬ ಆಶ್ಚರ್ಯದಂತೆ ಹೇಳುತ್ತಾನೆ. ಮತ್ತೊಬ್ಬ ಆಶ್ಚರ್ಯದಂತೆ ಕೇಳುತ್ತಾನೆ. ಬೇರೊಬ್ಬ ಇದನ್ನು ಕೇಳಿದರೂ ತಿಳಿಯಲಾರನು.
\end{artha}

ಈ ಪ್ರಪಂಚದಲ್ಲಿ ಆತ್ಮನಷ್ಟು ಸತ್ಯವಾದ ವಸ್ತು ಮತ್ತೊಂದು ಇಲ್ಲ. ಆದರೂ ಇದರಷ್ಟು ಅಜ್ಞಾತವಾದುದು ಮತ್ತೊಂದಿಲ್ಲ. ನಾವೆಲ್ಲ ನಮ್ಮಿಂದ ಹೊರಗೆ ಏನು ಆಗುತ್ತಿದೆ ಎಂಬುದನ್ನು ತಿಳಿದುಕೊಳ್ಳಲು ಕುತೂಹಲಿಗಳಾಗಿರುವೆವು. ನಿಜವಾಗಿ ನಾವಾರು, ನಮ್ಮ ಸ್ವಭಾವವೇನು ಎಂಬುದನ್ನು ಕಲಿತುಕೊಳ್ಳುವುದೇ ಇಲ್ಲ. ಎಲ್ಲೊ ಕೆಲವು ಧೀರರು ಈ ಅಂತರ್ಮುಖ ಪಯಣದಲ್ಲಿ ಯಶಸ್ವಿಯಾಗಿ, ತಮ್ಮ ದೇಹ ಇಂದ್ರಿಯ ಮನಸ್ಸು ಬುದ್ಧಿ ಇವುಗಳ ಹಿಂದೆ ಹೋಗಿ ನೋಡುತ್ತಾರೆ. ಆಗ ಆಶ್ಚರ್ಯಪಡುವರು. ನಾವು ಯಾವ ಹೊಸ ಅನುಭವ ಆದರೂ ಆಗಲೆ ನಮ್ಮಲ್ಲಿರುವ ಹಳೆಯ ಅನುಭವದೊಡನೆ ಹೋಲಿಸಿ ನೋಡುತ್ತೇವೆ. ಆ ಅನುಭವಕ್ಕೆ ಮೀರಿದ್ದರೆ, ಆಗ ಆಶ್ಚರ್ಯಪಡುವೆವು. ಇದೊಂದು ಅತೀಂದ್ರಿಯ ಅನುಭವ. ಇಂಥ ಅನುಭವವನ್ನು ಇದುವರೆಗೆ ಅವನು ಅನುಭವಿಸಿದವನಲ್ಲ. ಅದರಂತೆಯೇ ಕೇಳುವವನು. ಆತ್ಮನ ವಿಷಯವನ್ನು ಕೇಳಿದಾಗ ಆಶ್ಚರ್ಯಪಡುವನು. ಅವನು ಇದುವರೆಗೆ ಇಂದ್ರಿಯ ಅನುಭವಕ್ಕೆ ಸಂಬಂಧಪಟ್ಟ ವಿಷಯಗಳನ್ನು ಕೇಳುತ್ತಿದ್ದ. ಈಗ ಆತ್ಮನ ವಿಷಯವನ್ನು, ಯಾರು ಸಾಕ್ಷಾತ್ ಕಂಡಿರುವರೊ ಅವನ ಬಾಯಿಂದ ಕೇಳಿದಾಗ ಆಶ್ಚರ್ಯಪಡುವನು. ಇಂತಹ ವಿಷಯವನ್ನು ಅವನು ಹಿಂದೆ ಎಂದೂ ಕೇಳಿರಲಿಲ್ಲ. ಹಾಗೆ ಹೇಳುವವನಿಗೂ ಆಶ್ಚರ್ಯ. ಇದು ನಮ್ಮ ಸಾಮಾನ್ಯ ಗ್ರಹಿಕೆಗೆ ನಿಲುಕುವುದಿಲ್ಲ. ಇದೊಂದು ದೊಡ್ಡ ಸೋಜಿಗವಾಗಿ ಕಾಣುವುದು. ಬೇರೆ ಗ್ರಹಕ್ಕೆ ಹೋಗಿ ಅಲ್ಲೇನಾದರೂ ನಮ್ಮ ಭೂಮಿಯಲ್ಲಿಲ್ಲದ ವಿಚಿತ್ರ ಪ್ರಾಣಿಯನ್ನು ನೋಡಿಕೊಂಡು ಬಂದವನು ಅದನ್ನು ವಿವರಿಸುವುದನ್ನು ಕೇಳುವವನು ಎಷ್ಟು ಆಶ್ಚರ್ಯಪಡುತ್ತಾನೆಯೋ ಹಾಗೆ.

ಹಲವರು ಆತ್ಮನ ವಿಷಯವನ್ನು ಕೇಳಬಹುದು. ಆದರೆ ಅದನ್ನು ತಿಳಿದುಕೊಳ್ಳುವ ಯೋಗ್ಯತೆಯನ್ನು ಪಡೆದುಕೊಂಡಿರುವುದಿಲ್ಲ. ಯೋಗ್ಯತೆಯನ್ನು ಪಡೆದುಕೊಳ್ಳುವುದು ಬಹಳ ಮುಖ್ಯ. ಯಾರ ಮನಸ್ಸು ಶುದ್ಧವಾಗಿಲ್ಲವೋ, ಕೇಂದ್ರೀಕೃತವಾಗಿಲ್ಲವೋ, ಅವರು ಬೇಕಾದಷ್ಟು ಕೇಳಬಹುದು. ಅದು ಕಿವುಡನ ಮುಂದೆ ಕಿಂದರಿ ಬಾರಿಸಿದಂತೆ ಆಗುವುದು. ಜೀವನದಲ್ಲಿ ಚೆನ್ನಾಗಿರುವ ಸಂಗೀತ ಕೇಳಿ ಆನಂದ ಪಡಬೇಕಾದರೆ, ಒಂದು ಸುಂದರವಾದ ಚಿತ್ರವನ್ನು ನೋಡಿ ಮೆಚ್ಚಬೇಕಾದರೆ, ಅದಕ್ಕೆ ತರಬೇತನ್ನು ತೆಗೆದುಕೊಂಡಿರಬೇಕು. ಅದರ ಮೇಲೆ ಆಸಕ್ತಿ ಇರಬೇಕು. ಆಗ ಮಾತ್ರ ಅದು ಸಾಧ್ಯವಾಗುವುದು. ಆದ ಕಾರಣವೇ ಹಿಂದಿನ ಕಾಲದಲ್ಲಿ ವೇದಾಂತ ವಿಷಯಗಳನ್ನು ತಿಳಿದುಕೊಳ್ಳ ಬೇಕಾದರೆ, ಮೊದಲು ಅವನು ಸಾಧನ ಚತುಷ್ಟಯದ ಗರಡಿಯಲ್ಲಿ ಚೆನ್ನಾಗಿ ಪಳಗಿರಬೇಕು ಎನ್ನುತ್ತಿದ್ದರು. ಈ ಸಾಧನೆ ಇಲ್ಲದಿದ್ದರೆ ಆತ್ಮನ ವಿಷಯ ತಿಳಿದುಕೊಳ್ಳುವುದು ಅಸಾಧ್ಯವಾಗುವುದು.

\begin{shloka}
ದೇಹೀ ನಿತ್ಯಮವಧ್ಯೋಽಯಂ ದೇಹೇ ಸರ್ವಸ್ಯ ಭಾರತ~।\\ತಸ್ಮಾತ್ ಸರ್ವಾಣಿ ಭೂತಾನಿ ನ ತ್ವಂ ಶೋಚಿತುಮರ್ಹಸಿ \hfill॥ ೩00~॥
\end{shloka}

\begin{artha}
ಭಾರತ, ಎಲ್ಲರ ದೇಹದಲ್ಲಿರುವ ಈ ಆತ್ಮನು ಎಂದಿಗೂ ನಾಶವಾಗುವುದಿಲ್ಲ. ಆದ ಕಾರಣವೇ ನೀನು ಯಾರನ್ನೂ ಕುರಿತು ಶೋಕಿಸಬಾರದು.
\end{artha}

ಎಲ್ಲರಲ್ಲಿರುವ ಆತ್ಮ, ದೇಹದಲ್ಲಿದ್ದರೂ ದೇಹಧರ್ಮದಿಂದ ಬಾಧಿತವಾಗಿಲ್ಲ. ಪಾತ್ರೆಗೆ ಏನಾ ದರೂ ಆದರೆ ಅದರ ಒಳಗಿರುವ ಆಕಾಶಕ್ಕೆ ಹೇಗೆ ಏನನ್ನೂ ಮಾಡಲಾಗುವುದಿಲ್ಲವೊ ಹಾಗೆ. ಇಲ್ಲಿಗೆ ಆತ್ಮದೃಷ್ಟಿಯಿಂದ ಮಾತನಾಡುವುದನ್ನು ನಿಲ್ಲಿಸಿ ಶ‍್ರೀಕೃಷ್ಣ ಸ್ವಧರ್ಮ ದೃಷ್ಟಿಯಿಂದ ಮಾತನಾಡ ತೊಡಗುವನು.

\begin{shloka}
ಸ್ವಧರ್ಮಮಪಿ ಚಾವೇಕ್ಷ್ಯ ನ ವಿಕಂಪಿತುಮರ್ಹಸಿ~।\\ಧರ್ಮ್ಯಾದ್ಧಿ ಯುದ್ಧಾಚ್ಛ್ರೇಯೋಽನ್ಯತ್ ಕ್ಷತ್ರಿಯಸ್ಯ ನ ವಿದ್ಯತೇ \hfill॥ ೩೧~॥
\end{shloka}

\begin{artha}
ಸ್ವಧರ್ಮದ ದೃಷ್ಟಿಯಿಂದ ನೋಡಿದರೂ ನೀನು ಅನುಮಾನಿಸಬಾರದು. ಏಕೆಂದರೆ ಧರ್ಮಯುದ್ಧಕ್ಕಿಂತ ಕ್ಷತ್ರಿಯನಿಗೆ ಶ್ರೇಯಸ್ಕರವಾಗಿರುವುದು ಬೇರೊಂದಿಲ್ಲ.
\end{artha}

ಅರ್ಜುನ ಸ್ವಧರ್ಮ ದೃಷ್ಟಿಯಿಂದ ಈ ಯುದ್ಧವನ್ನು ನೋಡಬೇಕಾಗಿದೆ. ಕ್ಷತ್ರಿಯನ ಧರ್ಮ ರಾಜ್ಯವನ್ನು ನ್ಯಾಯವಾಗಿ ಪರಿಪಾಲಿಸುವುದು, ಅಲ್ಲಿ ಸತ್ಯ ಧರ್ಮ ಮುಂತಾದವುಗಳಿಗೆ ರಕ್ಷಣೆ ಕೊಡುವುದು ಆಗಿದೆ. ಯಾವಾಗ ಇಂತಹ ಆದರ್ಶಗಳಿಗೆ ಕುಂದು ಬರುವುದೊ, ಅವನು ಅದಕ್ಕೆ ವಿರೋಧವಾಗಿ ಹೋರಾಡಬೇಕು. ಈ ಒಂದು ಕೆಲಸ ಮಾಡುವ ವರ್ಣದಲ್ಲಿ ಅವನು ಹುಟ್ಟಿರುವನು. ಈ ಕೆಲಸವನ್ನು ಮಾಡುವುದಕ್ಕಾಗಿ ತರಬೇತನ್ನು ತೆಗೆದುಕೊಂಡಿರುವನು. ಈಗ ಆ ಕೆಲಸ ಮಾಡುವುದಕ್ಕೆ ಸಮಯ ಬಂದಿದೆ. ಸ್ವಲ್ಪವೂ ಅಳುಕದೆ ಅವನು ಯುದ್ಧ ಮಾಡಬೇಕು. ಇದನ್ನು ತನ್ನ ಸ್ವಂತ ಪ್ರಯೋಜನಕ್ಕಾಗಿ ಅಲ್ಲದೆ ಇದ್ದರೂ ಕ್ಷತ್ರಿಯನ ಧರ್ಮದ ದೃಷ್ಟಿಯಿಂದ ನೋಡಬೇಕಾಗಿದೆ. ಯಾವಾಗ ಅನ್ಯಾಯ ಮಾಡುವವರನ್ನು ದಂಡಿಸುವುದಿಲ್ಲವೋ ಆಗ ಅನ್ಯಾಯಕ್ಕೆ ಪ್ರೋತ್ಸಾಹ ಕೊಟ್ಟಂತೆ ಆಗುವುದು. ಅದನ್ನು ನೋಡಿಯೂ ನೋಡದಂತೆ ಇರುವುದು ಪಾಪ. ಸಾಧಾರಣ ಯುದ್ಧಕ್ಕೆ ಅವಕಾಶ ಬಂದಾಗಲೇ ಅದಕ್ಕೆ ವಿರುದ್ಧನಾಗದೆ ಇರುವುದು ಕ್ಷತ್ರಿಯನ ಕರ್ತವ್ಯ. ಹೀಗಿರುವಾಗ ಇದು ಧರ್ಮಯುದ್ಧ. ಇಂತಹ ಅವಕಾಶಗಳು ಸಿಕ್ಕುವುದು ಬಹಳ ಅಪರೂಪ. ಕಾದಾಡುವುದಕ್ಕೆ ಅವಕಾಶ ಸಿಕ್ಕುವುದು. ಆದರೆ ಒಂದು ಧರ್ಮಕ್ಕೆ ಸತ್ಯಕ್ಕೆ ಕಾದಾಡುವ ಅವಕಾಶ ಎಲ್ಲರ ಪಾಲಿಗೂ ಯಾವಾಗಲೂ ಸಿಕ್ಕುವುದಿಲ್ಲ. ಈಗ ಅಂತಹ ಸಮಯ ಬಂದಿದೆ. ಅದನ್ನು ಕಳೆದುಕೊಂಡರೆ ಅಂತಹ ಅವಕಾಶ ಸಿಕ್ಕುವುದು ಬಹಳ ಅಪರೂಪ ಜೀವನದಲ್ಲಿ. ಈಗ ಯುದ್ಧ ಮಾಡುವುದು ಅತ್ಯಂತ ಶ್ರೇಯಸ್ಕರವಾದುದು ಎಂದು ಹೇಳುವನು.

\begin{shloka}
ಯದೃಚ್ಛಯಾ ಚೋಪಪನ್ನಂ ಸ್ವರ್ಗದ್ವಾರಮಪಾವೃತಮ್~।\\ಸುಖಿನಃ ಕ್ಷತ್ರಿಯಾಃ ಪಾರ್ಥ ಲಭಂತೇ ಯುದ್ಧಮೀದೃಶಮ್ \hfill॥ ೩೨~॥
\end{shloka}

\begin{artha}
ಅರ್ಜುನ, ತಾನಾಗಿ ಬಂದಿರುವ ತೆರೆದ ಸ್ವರ್ಗದ ಬಾಗಿಲಿನಂತೆ ಇರುವ ಇಂತಹ ಯುದ್ಧವನ್ನು ಅದೃಷ್ಟಶಾಲಿಗಳಾದ ಕ್ಷತ್ರಿಯರು ಮಾತ್ರ ಪಡೆಯುತ್ತಾರೆ.
\end{artha}

ಈ ಧರ್ಮಯುದ್ಧವನ್ನು ನಾವು ಹುಡುಕಾಡುತ್ತಿರುತ್ತೇವೆ. ಈಗ ಅದು ತಾನಾಗಿ ಬಂದಿದೆ. ಈ ಯುದ್ಧವನ್ನು ತಾವಾಗಿ ಮಾಡಬೇಕೆಂದು ಕಾಲುಕೆರೆಯುತ್ತಿರಲಿಲ್ಲ ಪಾಂಡವರು. ಹೆಜ್ಜೆಹೆಜ್ಜೆಗೆ ಕೌರವರು ಪಾಂಡವರಿಗೆ ಅನ್ಯಾಯ ಮಾಡಿ ಇಂತಹ ಪರಿಸ್ಥಿತಿಗೆ ತಂದಿರುವರು. ರಾಜ್ಯಗಳನ್ನು ಬೇಡಿ ಪಡೆಯಬೇಡಿ, ಕಾದು ಅದನ್ನು ಪಡೆಯಿರಿ ಎಂದವನು ದುರ್ಯೋಧನ. ಯಾವಾಗ ಅವನು ಹಾಕಿದ ಸವಾಲನ್ನು ಅರ್ಜುನ ಅಲ್ಲಗಳೆದನೊ ಆಗ ಅರ್ಜುನ ತಪ್ಪನ್ನು ಮಾಡುತ್ತಾನೆ. ಮುಂಚೆ ಯುದ್ಧವನ್ನೇ ಮಾಡುವುದು ಪಾಪ ಎಂದು ಭಾವಿಸಿದ್ದ. ಈಗ ಅಧರ್ಮಕ್ಕೆ ಬೆಂಬಲಿಗರಾಗಿ ನಿಂತವರು ಯಾರಾದರೂ ಆಗಿರಲಿ ಅವರನ್ನು ಕಳೆಯಂತೆ ನಿರ್ದಾಕ್ಷಿಣ್ಯದಿಂದ ಕಿತ್ತು ಹಾಕದೆ ಇರುವುದು ಪಾಪವಾಗುವುದು.

ಈ ಯುದ್ಧ ತೆರೆದ ಸ್ವರ್ಗದ ಬಾಗಿಲಿನಂತೆ ಇದೆ. ಯುದ್ಧವೆಲ್ಲ ಧರ್ಮಯುದ್ಧವಲ್ಲ. ಹಲವಾರು ಯುದ್ಧಗಳನ್ನು ಕೇವಲ ಸ್ವಾರ್ಥದ ದೃಷ್ಟಿಯಿಂದ ತಮ್ಮ ರಾಜ್ಯವನ್ನು ವಿಸ್ತರಿಸಿಕೊಳ್ಳುವುದಕ್ಕೆ, ಯಾರಮೇಲೊ ಸೇಡನ್ನು ತೀರಿಸಿಕೊಳ್ಳುವುದಕ್ಕೆ, ಬೇಟೆ ಆಡಬೇಕೆಂದು ಜನ ಹೇಗೆ ಆಡುತ್ತಾರೊ ಹಾಗೆ ಕೂತು ಸಾಕಾಗಿ ಯುದ್ಧಮಾಡಬೇಕೆಂದೂ ಕೆಲವು ವೇಳೆ ಮಾಡುವರು. ಇದು ಅಂಥವರಿಗೆ ಒಂದು ಕ್ರೀಡೆಯಾಗುವುದು. ಆದರೆ ಅರ್ಜುನ ಬೇಡ ಬೇಡವೆಂದರೂ ಈ ಯುದ್ಧ ಅವನ ಪಾಲಿಗೆ ಬಂದಿದೆ. ಇಲ್ಲಿ ಸರಿಯೇ ತಪ್ಪೇ ಎಂಬುದನ್ನು ಅನುಮಾನಿಸುವ ಪ್ರಸಂಗವೇ ಇಲ್ಲ. ದುರ್ಯೋಧನನಿಂದ ಬೇಕಾದಷ್ಟು ಘಾಸಿಪಟ್ಟಿರುವವರು ಇವರು. ಇಲ್ಲಿ ಯುದ್ಧಮಾಡಿದರೆ ಅನುಮಾನವಿಲ್ಲದೆ ಸದ್ಗತಿ ಸಿಕ್ಕುವುದು.

ಇಂತಹ ಧರ್ಮಯುದ್ಧ ಮಾಡುವ ಅವಕಾಶ ಎಲ್ಲರ ಪಾಲಿಗೂ ಬರುವುದಿಲ್ಲ. ಸಾಯುವುದು ದೊಡ್ಡದಲ್ಲ. ಯಾರಿಗಾಗಿ ಯಾವ ಉದ್ದೇಶಕ್ಕಾಗಿ ಸಾಯುತ್ತೇವೆಯೊ ಅದು ಮುಖ್ಯ. ಧರ್ಮಯುದ್ಧದಲ್ಲಿ ಅರ್ಜುನ ಗೆಲ್ಲದೆ ಇರಬಹುದು. ಆದರೂ ಒಂದು ಒಳ್ಳೆಯ ಉದ್ದೇಶಕ್ಕೆ ತನ್ನ ಪ್ರಾಣವ ನ್ನಾದರೂ ಕೊಟ್ಟೆ ಎಂಬ ಸಮಾಧಾನವಾದರೂ ಇರುವುದು.

\begin{shloka}
ಅಥ ಚೇತ್ ತ್ವಮಿಮಂ ಧರ್ಮ್ಯಂ ಸಂಗ್ರಾಮಂ ನ ಕರಿಷ್ಯಸಿ~।\\ತತಃ ಸ್ವಧರ್ಮಂ ಕೀರ್ತಿಂ ಚ ಹಿತ್ವಾ ಪಾಪಮವಾಪ್ಸ್ಯಸಿ \hfill॥ ೩೩~॥
\end{shloka}

\begin{artha}
ಹಾಗಿಲ್ಲದೆ ನೀನು ಈ ಧರ್ಮಯುದ್ಧವನ್ನು ಮಾಡದೆ ಹೋದರೆ, ಸ್ವಧರ್ಮಚ್ಯುತನಾಗಿ ಕೀರ್ತಿಯನ್ನು ಕಳೆದುಕೊಂಡು ಪಾಪವನ್ನು ಮಾಡುತ್ತೀಯೆ.
\end{artha}

ಈ ಧರ್ಮಯುದ್ಧವನ್ನು ಮಾಡದೆ ಇದ್ದರೆ ಪಾಪವನ್ನು ಮಾಡಿದಂತೆ ಆಗುವುದು. ಇದರಿಂದ ಅಧರ್ಮಕ್ಕೆ ಪ್ರೋತ್ಸಾಹ ಕೊಟ್ಟಂತೆ ಆಗುವುದು. ಕಳ್ಳತನ ಮಾಡುವುದು ಪಾಪ, ಹಾಗೆ ಮಾಡುವಂತೆ ಅವನಿಗೆ ಸಹಾಯ ಮಾಡುವುದು ಪಾಪ. ಹೀಗೆ ಕಳ್ಳತನ ಎದುರಿಗೆ ಆಗುತ್ತಿದ್ದರೂ ಅದನ್ನು ದಂಡಿಸುವುದಕ್ಕೆ ಯೋಗ್ಯತೆಯನ್ನು ಪಡೆದು ಅವನು ಸುಮ್ಮನೆ ಕೈಕಟ್ಟಿಕೊಂಡು ನಿಂತಿರುವುದು ಎಲ್ಲಕ್ಕಿಂತಲೂ ಪಾಪ. ಇದನ್ನು ತನಗಾಗುವ ಲಾಭನಷ್ಟದ ದೃಷ್ಟಿಯಿಂದಲೇ ನೋಡಬಾರದು. ಅದು ಇಲ್ಲಿ ಗೌಣ. ಇದರಿಂದ ಸಮಾಜದ ಮೇಲೆ ಎಂತಹ ಪರಿಣಾಮ ಉಂಟಾಗುವುದೋ ಆ ದೃಷ್ಟಿಯಿಂದ ನೋಡಬೇಕು. ಪಾಂಡವರು ಕೌರವರನ್ನು ದಂಡಿಸದೆ ಬಿಟ್ಟರೆ, ಅನ್ಯಾಯ ಅಧರ್ಮ ಪ್ರಪಂಚದಲ್ಲಿ ತಾನೇ ತಾನಾಗಿ ತಾಂಡವವಾಡುವುದು. ಮೃಗೀಯ ಶಕ್ತಿಗೆ ಮಾತ್ರ ಪ್ರಪಂಚದಲ್ಲಿ ಮಾನ್ಯತೆ. ಸಾಧು ಸ್ವಭಾವದವರಿಗೆ, ಸತ್ಯನಿಷ್ಠರಿಗೆ ಈ ಪ್ರಪಂಚದಲ್ಲಿ ಬಾಳುವುದಕ್ಕೆ ಆಗುವುದಿಲ್ಲ. ಇಂತಹ ಪ್ರಪಂಚ ನರಕಕ್ಕಿಂತ ಭಯಾನಕವಾಗುವುದು.

ಅರ್ಜುನ ಮಹಾ ಪರಾಕ್ರಮಶಾಲಿ ಎಂಬ ಬಿರುದನ್ನು ಗಳಿಸಿದ್ದ ಪ್ರಪಂಚದಲ್ಲಿ. ಯಾವಾಗ ಯುದ್ಧಮಾಡದೆ ಹೋಗುವನೊ ಜನರೆಲ್ಲ ಇವನು ಕೈಲಾಗದೆ ಬಿಟ್ಟ ಎನ್ನುವರು. ಅವನ ಕೀರ್ತಿಗೆ ಕಳಂಕ ಬರುವುದು. ಜೀವನದಲ್ಲಿ ಒಂದು ಒಳ್ಳೆಯ ಹೆಸರನ್ನು ಪಡೆಯುವುದಕ್ಕೆ ಎಷ್ಟು ಕಷ್ಟಪಡಬೇಕು. ಆದರೆ ಅದನ್ನು ಒಂದು ಕ್ಷಣದಲ್ಲಿ ನಮ್ಮ ಅಜಾಗರೂಕತೆಯಿಂದ, ಅಲಕ್ಷ್ಯದಿಂದ ಕಳೆದುಕೊಂಡು ಬಿಡಬಹುದು. ಶ‍್ರೀಕೃಷ್ಣ ಇನ್ನು ಮೇಲೆ ಕೀರ್ತಿಯ ದೃಷ್ಟಿಯಿಂದ ಮಾತನಾಡುವನು. ಅರ್ಜುನನ ಹೃದಯಕ್ಕೆ ಈ ಮಾತಾದರೂ ತಾಕಲಿ ಎಂದು. ಅನೇಕರಿಗೆ ತತ್ತ್ವ ತಿಳಿಯದು, ಸ್ವಧರ್ಮ ತಿಳಿಯದು. ಒಂದು ಒಳ್ಳೆಯ ಹೆಸರನ್ನು ಪಡೆಯುವುದಕ್ಕೆ, ಆಗಲೆ ಒಳ್ಳೆಯ ಹೆಸರನ್ನು ಪಡೆದಿದ್ದರೆ ಅದನ್ನು ಉಳಿಸಿಕೊಳ್ಳುವುದಕ್ಕೆ ಕೆಲಸ ಮಾಡುವರು.

\begin{shloka}
ಅಕೀರ್ತಿಂ ಚಾಪಿ ಭೂತಾನಿ ಕಥಯಿಷ್ಯಂತಿ ತೇಽವ್ಯಯಾಮ್~।\\ಸಂಭಾವಿತಸ್ಯ ಚಾಕೀತಿರ್ಮರಣಾದತಿರಿಚ್ಯತೇ \hfill॥ ೩೪~॥
\end{shloka}

\begin{artha}
ಜನ ನಿನ್ನ ಅಪಯಶಸ್ಸನ್ನು ಎಂದೆಂದಿಗೂ ಆಡಿಕೊಳ್ಳುವರು. ಗೌರವಸ್ಥನಿಗೆ ಅಪಕೀರ್ತಿ ಮರಣಕ್ಕಿಂತ ಹೆಚ್ಚು.
\end{artha}

ಶೂರ ಧೀರ ಎನ್ನಿಸಿಕೊಂಡ ಅರ್ಜುನ, ಎಲ್ಲರ ಬಾಯಿಂದಲೂ ಹೇಡಿ ಎಂದು ಕರೆಸಿಕೊಳ್ಳ ಬೇಕಾಗುವುದು. ಒಳ್ಳೆಯ ಸಂಧಿಗ್ಧ ಸಮಯದಲ್ಲಿ ಮಾಡುವ ಕೆಲಸವನ್ನು ಬಿಟ್ಟರೆ ಒಬ್ಬ ಅಪಹಾಸ್ಯಕ್ಕೆ ಈಡಾಗುವನು. ಗೌರವಸ್ಥನಿಗೆ ಮರಣಕ್ಕಿಂತ ಭಯಾನಕವಾಗಿರುವುದು ಇದು. ಮನುಷ್ಯ ಸತ್ತರೆ ವ್ಯಥೆಪಡುವವನು ಅವನಲ್ಲ. ಸುತ್ತಮುತ್ತಲಿರುವವರು, ಅವನನ್ನು ನೆಚ್ಚಿದವರು, ನಂಬಿದವರು, ಪ್ರೀತಿಸಿದವರು. ಸತ್ತವನು ಹಾಯಾಗಿ ಹೋಗುವನು. ಅವನಿಗೆ ಇದರ ಅರಿವೇ ಇರುವುದಿಲ್ಲ. ಯಾರು ಏನೆಂದರೂ ಕೇಳಿಸುವುದೂ ಇಲ್ಲ. ಆದರೆ ಅಪಕೀರ್ತಿಯನ್ನಾದರೊ ಬದುಕಿರುವವರೆಗೆ ಕೇಳುತ್ತ ಇರಬೇಕು. ಮನುಷ್ಯನ ಮನಸ್ಸು ಯಾವಾಗಲೂ ಶೋಕದಲ್ಲಿ ತಪ್ತವಾಗಿರುವುದಿಲ್ಲ. ಅದರಿಂದ ಚೇತರಿಸಿಕೊಂಡಮೇಲೆ ಅವರಾಡುವ ಮಾತುಗಳು ಬಾಣದಂತೆ ಇರಿಯುವುವು. ಆಗ ಅದರ ನೋವು ಗೊತ್ತಾಗುವುದು. ಅರ್ಜುನ ನಿಂದಾಸ್ತುತಿಗಳನ್ನು ಒಂದೇ ಸಮನಾಗಿ ನೋಡುವ ಸ್ಥಿತಪ್ರಜ್ಞನ ಹಂತಕ್ಕೆ ಏರಿದವನಲ್ಲ. ಜನ ಇವನನ್ನು ನಿಂದಿಸುವಾಗ ಕೋಪದಿಂದ ರೊಚ್ಚಿಗೆದ್ದು, ಈಗ ಏನನ್ನು ಮಾಡುವುದಿಲ್ಲ ಎಂದು ಹೇಳಿರುತ್ತಾನೆಯೊ ಅದನ್ನು ಮಾಡಲು ಯತ್ನಿಸುವನು. ಮಗು ಮನೆಯಲ್ಲಿ ಯಾವ ಕಾರಣದಿಂದಲೊ ದೊಡ್ಡವರ ಮೇಲಿನ ಕೋಪಕ್ಕೆ ನಾನು ಊಟ ಮಾಡದೆ ಪ್ರಾಣಬಿಡುತ್ತೇನೆ ಎನ್ನುವುದು. ಹೊಟ್ಟೆ ಹಸಿಯುವುದಕ್ಕೆ ಶುರುವಾದಾಗ ಮನೆಯವರು ಊಟಕ್ಕೆ ಏಳು ಎಂದರೆ ಸಾಕು ಎಂದು ಕಾದು ಕುಳಿತಿರುವುದು. ಒಂದು ವೇಳೆ ಯಾರೂ ಏಳಿಸದೆ ಇದ್ದರೂ ಅಡಿಗೆ ಮನೆಗೆ ತಾನೇ ಹೋಗಿ ಊಟಮಾಡುವುದು. ಮಗು ಬೆಳಿಗ್ಗೆ ಎಂತಹ ಭೀಷ್ಮಪ್ರತಿಜ್ಞೆ ಮಾಡಿತು! ಆದರೆ ಕ್ರಮೇಣ ಅದು ದುರ್ಬಲವಾಗಿ ಕೊನೆಗೆ ಹೇಗೆ ಹಾರಿ ಹೋಗುವುದು ಎಂಬುದು ಅದಕ್ಕೆ ಗೊತ್ತಿರುವುದಿಲ್ಲ. ಆ ಮಗು ಮುನಿಸಿನಿಂದ ಹೇಳಿತು. ಅರ್ಜುನ ದುಃಖಪರವಶನಾಗಿ ಹೇಳಿದ. ಆದರೆ ಎರಡು ಸ್ಥಿತಿಯೂ ಕ್ರಮೇಣ ಬದಲಾವಣೆ ಆಗುವುದು. ಆ ಮಗುವಿಗೆ ಹೊಟ್ಟೆ ಹಸಿವಿನ ಚುರುಕು ಹೇಗೆ ತಾಕುವುದೊ ಹಾಗೆ ಅರ್ಜುನನಿಗೆ ಜನರ ನಿಂದೆಯ ಚುರುಕು ತಾಗುವುದು.

\begin{shloka}
ಭಯಾದ್ರಣಾದುಪರತಂ ಮಂಸ್ಯಂತೇ ತ್ವಾಂ ಮಹಾರಥಾಃ~।\\ಯೇಷಾಂ ಚ ತ್ವಂ ಬಹುಮತೋ ಭೂತ್ವಾ ಯಾಸ್ಯಸಿ ಲಾಘವಮ್ \hfill॥ ೩೫~॥
\end{shloka}

\begin{artha}
ನೀನು ಭಯದಿಂದ ಯುದ್ಧದಿಂದ ಹಿಂಜರಿದೆ ಎಂದು ಮಹಾರಥಿಗಳು ಹೇಳುವರು. ಯಾವ ಶೂರರಿಗೆ ನೀನು ಬಹುಮಾನ್ಯನಾಗಿದ್ದೆಯೊ, ಅವರು ನಿನ್ನನ್ನು ತುಚ್ಛವಾಗಿ ಕಾಣುವರು.
\end{artha}

ಅರ್ಜುನನಿಗೆ ಶ‍್ರೀಕೃಷ್ಣ ಮುಂದಾಗುವುದನ್ನೆಲ್ಲ ಚೆನ್ನಾಗಿ ವಿವರಿಸುವನು. ಇವುಗಳೆಲ್ಲ ಅಷ್ಟು ಮಾನವ ಸಹಜವಾಗಿದೆ. ಶೂರರು ನಿನ್ನನ್ನು ಭಯದಿಂದ ಹಿಮ್ಮೆಟ್ಟಿದೆ ಎಂದು ಭಾವಿಸುವರು. ಶ‍್ರೀಕೃಷ್ಣ ಅರ್ಜುನನ ಸಾರಥಿಯಾದವನು. ಅವನಿಗೆ ಚೆನ್ನಾಗಿ ಗೊತ್ತಿದೆ ಅರ್ಜುನನ ಮನಸ್ಸು. ಅವನು ಕೊಲೆಗೆ ಒಪ್ಪದೆ ಹಿಂಜರಿದ ಎಂದು. ಆದರೆ ಎಷ್ಟು ಜನಕ್ಕೆ ಅರ್ಜುನ ಇದಕ್ಕಾಗಿ ಮಾಡಿದ ಎಂದು ಗೊತ್ತಾಗುವುದು? ಯುದ್ಧ ಮಾಡುವುದಕ್ಕೆ ಮುಂಚೆ ತನ್ನ ಕಡೆಗೆ ಸೈನ್ಯವನ್ನು ಸಂಗ್ರಹಿಸುತ್ತಿದ್ದಾಗ ಅರ್ಜುನನಿಗೆ ಗೊತ್ತಿರಲಿಲ್ಲವೆ ಯಾರ ಮೇಲೆ ಯುದ್ಧ ಮಾಡಬೇಕು ಎಂಬುದು? ಅದೇನು ಕುರುಕ್ಷೇತ್ರದ ಯುದ್ಧಭೂಮಿಗೆ ಬಂದಮೇಲೆ ಗೊತ್ತಾದ ವಿಷಯವಲ್ಲ. ಆದಕಾರಣ ಹೇಡಿತನದಿಂದಲೇ ಅವನು ಹೋದ ಎನ್ನುವರು. ಜನ ತಮ್ಮ ಮನಸ್ಸಿಗೆ ಬಂದುದನ್ನು ಹೇಳುತ್ತಾರೆ ಅದನ್ನು ಕಟ್ಟಿಕೊಂಡು ನನಗೇನು ಎಂದು ಅರ್ಜುನ ಹೇಳಬಹುದು. ಆದರೆ ಈ ದೃಷ್ಟಿ ಯಾವಾಗಲೂ ಇರುವುದಿಲ್ಲ. ಅವನ ಮನಸ್ಸು ಬದಲಾಗುವುದು. ಆಗ ಇದನ್ನು ಕೇಳಿದಾಗ ರೇಗಿ ಯುದ್ಧಕ್ಕೆ ಇಳಿಯುವನು. ಇಲ್ಲವೆ ತಾನು ಮಾಡಿದ ತಪ್ಪಿಗೆ ಆತ್ಮಹತ್ಯೆ ಮಾಡಿಕೊಳ್ಳುವನು. ಇವೆರಡೂ ಉಚಿತವಾದ ಕೆಲಸಗಳಲ್ಲ. 

ಅರ್ಜುನನ್ನು ಇತರ ವೀರರು ಗೌರವಿಸುತ್ತಿದ್ದರು. ತನ್ನ ಸರಿಸಮಾನರಾದ ಶೂರರಿಂದ ಅರ್ಜುನ ಹೊಗಳಿಸಿಕೊಳ್ಳುತ್ತಿದ್ದ. ಭದ್ರಮನುಷ್ಯನಿಗೆ ತನಗೆ ಸರಿಸಮಾನರಾದವರಿಂದ ಗೌರವ ಸಿಗಬೇಕು. ಇದನ್ನು ಪಡೆಯುತ್ತಿದ್ದ ಅರ್ಜುನ. ಇನ್ನುಮೇಲೆ ಅವರ ಅಲಕ್ಷ ್ಯವನ್ನು ನೋಡುವಾಗ ಆಗುವ ವ್ಯಥೆಯನ್ನು ವಿವರಿಸಲು ಸಾಧ್ಯವಿಲ್ಲ. ಅನುಭವಿಸಿದಾಗ ಮಾತ್ರ ಅದರ ದಾರುಣತೆ ಗೊತ್ತಾಗುವುದು.

\begin{shloka}
ಅವಾಚ್ಯವಾದಾಂಶ್ಚ ಬಹೂನ್ ವದಿಷ್ಯಂತಿ ತವಾಹಿತಾಃ~।\\ನಿಂದಂತಸ್ತವ ಸಾಮರ್ಥ್ಯಂ ತತೋ ದುಃಖತರಂ ನು ಕಿಮ್ \hfill॥ ೩೬~॥
\end{shloka}

\begin{artha}
ನಿನಗೆ ಆಗದವರು, ನಿನ್ನ ಸಾಮರ್ಥ್ಯವನ್ನು ನಿಂದಿಸುತ್ತ ಆಡಬಾರದ ಮಾತುಗಳನ್ನು ಆಡುವರು. ಅದಕ್ಕಿಂತ ಹೆಚ್ಚು ದುಃಖಕರವಾಗಿರುವುದು ಯಾವುದು ಇದೆ?
\end{artha}

ಅರ್ಜುನನಿಗೆ ಆಗದವರು ಇದೇ ಸಮಯಕ್ಕೆ ಕಾದು ಕುಳಿತುಕೊಂಡಿರುವರು. ಅರ್ಜುನ ಹೇಡಿ, ಷಂಡ, ಅಂಜುಬುರುಕ, ಕೈಲಾಗದವನು ಎಂದು ತಮಗೆ ತೋಚಿದ ಹೀನ ಗುಣಗಳನ್ನೆಲ್ಲ ಆರೋಪ ಮಾಡುವರು. ಇದನ್ನು ಗಮನಕ್ಕೆ ತೆಗೆದುಕೊಳ್ಳದೆ ಇರಬೇಕಾದರೆ, ಅವನು ಸ್ಥಿತಪ್ರಜ್ಞನಾಗಿರಬೇಕು. ಇಲ್ಲದೆ ಇದಾವುದನ್ನು ಅರಿಯದ ಪ್ರಾಣಿಯಾಗಿರಬೇಕು. ಅರ್ಜುನ ಇವೆರಡೂ ಅಲ್ಲ. ಇತರರ ಟೀಕೆಯನ್ನು ಕೇಳುವಾಗ ಅದನ್ನು ಮನಸ್ಸಿಗೆ ಹಾಕಿಕೊಳ್ಳದೆ ಇದ್ದರೂ, ಕ್ರಮೇಣ ಮನಸ್ಸು ದುರ್ಬಲ ಸ್ಥಿತಿಗೆ ಬಂದಾಗ ಅದಕ್ಕೆ ಪರವಶವಾಗುವುದು. ಮನಸ್ಸಿಗೆ ಏಳು ಬೀಳುಗಳಿವೆ. ಎದ್ದ ಸಮಯದಲ್ಲಿ, ಉತ್ತಮಸ್ಥಿತಿಯಲ್ಲಿರುವಾಗ ಜನರೆನ್ನುವುದನ್ನು ಗಣನೆಗೆ ತರುವುದಿಲ್ಲ. ಆದರೆ ಆ ಸ್ಥಿತಿಯಲ್ಲಿ ಮನಸ್ಸು ಬಹುಕಾಲ ಇರಬಲ್ಲ ಸಾತ್ತ್ವಿಕ ಶಕ್ತಿಯನ್ನು ಅರ್ಜುನನು ಇನ್ನೂ ಸಂಪಾದಿಸಿಕೊಂಡಿಲ್ಲ. ಅದು ಯಾವುದೊ ಕೆಲವು ಘಟನೆಗಳಿಂದ ಮೇಲಕ್ಕೆ ಹೋಗಿದೆ. ಆಡುವ ಚೆಂಡನ್ನು ಹುಡುಗರು ಒದ್ದಾಗ ಮೇಲೆ ಹೋಗುವಂತೆ. ಅಲ್ಲಿ ಅದು ಎಷ್ಟು ಕಾಲ ಇರಬಲ್ಲುದು? ಎಲ್ಲೋ ಸ್ವಲ್ಪ ಹೊತ್ತು. ಅನಂತರ ಅದು ಕೆಳಗೆ ಬರುವುದು. ಕೆಳಗೆ ಬಂದಾಗ ಅರ್ಜುನನಿಗೆ ಜನರಾಡುವ ನಿಂದೆಯ ಮಾತಿನ ಬಿಸಿ ಗೊತ್ತಾಗುವುದು. ಆಗ ಬೆಂಕಿಯ ಮೇಲೆ ಬಿದ್ದು ಒದ್ದಾಡುವ ಕೀಟಕ್ಕಿಂತ ಶೋಚನೀಯವಾಗುವುದು ಅವನ ಸ್ಥಿತಿ.

\begin{shloka}
ಹತೋ ವಾ ಪ್ರಾಪ್ಯಸಿ ಸ್ವರ್ಗಂ ಜಿತ್ವಾ ವಾ ಭೋಕ್ಷ್ಯಸೇ ಮಹೀಮ್~।\\ತಸ್ಮಾದುತ್ತಿಷ್ಠ ಕೌಂತೇಯ ಯುದ್ಧಾಯ ಕೃತನಿಶ್ಚಯಃ \hfill॥ ೩೭~॥
\end{shloka}

\begin{artha}
ಯುದ್ಧದಲ್ಲಿ ಸತ್ತರೆ ವೀರಸ್ವರ್ಗವಿದೆ. ಗೆದ್ದರೆ ಅನುಭವಿಸುವುದಕ್ಕೆ ರಾಜ್ಯವಿದೆ. ಆದ ಕಾರಣ ಯುದ್ಧಮಾಡುವುದಕ್ಕೆ ಮನಸ್ಸು ಮಾಡಿ ಏಳು.
\end{artha}

ಶ‍್ರೀಕೃಷ್ಣ ಮೊದಲು ತಾತ್ತ್ವಿಕ ದೃಷ್ಟಿಯಿಂದ ಸ್ಥಿತಿಯನ್ನು ಗಮನಿಸುವನು. ಅನಂತರ ಕರ್ತವ್ಯದ ದೃಷ್ಟಿಯನ್ನು ತರುವನು. ಇದಾದಮೇಲೆ ಕೀರ್ತಿ ಅಪಕೀರ್ತಿಗಳನ್ನು ತರುವನು. ಈಗ ಕೊನೆಗೆ ಉಪಯೋಗಿಸುವ ಬಾಣವೇ ಸ್ವಾರ್ಥದೃಷ್ಟಿ. ಇನ್ನು ಯಾವುದಕ್ಕೆ ಜಗ್ಗದೇ ಇದ್ದರೂ ಇದಕ್ಕಾಗಿಯಾದರೂ ಜಗ್ಗಲಿ ಎಂದು. ಶ‍್ರೀಕೃಷ್ಣ ಅರ್ಜುನನನ್ನು ಯುದ್ಧದಿಂದ ತಪ್ಪಿಸಿಕೊಂಡು ಹೋಗುವುದಕ್ಕೆ ಬಿಡುವುದಿಲ್ಲ. ಅವನನ್ನು ಒಂದು ನಿಮಿತ್ತವಾಗಿ ಮಾಡಿಕೊಂಡು ಧರ್ಮ ಸಂಸ್ಥಾಪನೆಯ ಕೆಲಸವನ್ನು ಮಾಡಿಯೇ ಮಾಡುವನು. ಒಳ್ಳೆಯ ಆದರ್ಶದಿಂದ ಮಾಡಿದರೆ, ದೇವರ ಕೆಲಸವೂ ಆಗುವುದು, ಅರ್ಜುನನೂ ಉದ್ಧಾರವಾಗುವನು. ಒಳ್ಳೆಯ ಆದರ್ಶವನ್ನು ಬಿಟ್ಟರೆ, ದೇವರ ಕೆಲಸವಂತೂ ನಿಲ್ಲುವುದಿಲ್ಲ. ಇನ್ನಾವುದೋ ಕೆಳದೃಷ್ಟಿಯಿಂದ ಕೆಲಸ ಮಾಡಿಹಾಕುವನು. ಇದರಿಂದ ದೇವರ ಕೆಲಸವಾಗುವುದು. ಆದರೆ ಅರ್ಜುನನ ಆತ್ಮೋದ್ಧಾರಕ್ಕೆ ತಡೆಯಾಗುವುದು. ಅದಕ್ಕಾಗಿ ಉತ್ತಮ ಮಟ್ಟದಲ್ಲಿ ನಿಂತು ಆದರ್ಶ ದೃಷ್ಟಿಯಿಂದ ಯುದ್ಧಮಾಡು ಎನ್ನುವನು. ಯಾವಾಗ ಅದಕ್ಕೆ ಸೋಲುವುದಿಲ್ಲವೊ, ಆಗ ಅತ್ಯಂತ ಹತ್ತಿರವಾದ ವ್ಯಕ್ತಿಯ ಸ್ವಾರ್ಥದೃಷ್ಟಿಯಿಂದ ಮಾತನಾಡುವನು. ಅವನು ಒಂದು ಸಲ ಯುದ್ಧವನ್ನು ಪ್ರಾರಂಭಮಾಡಿದರೆ ಮಧ್ಯದಲ್ಲಿ ಬೆನ್ನು ತೋರುವವರ ಗುಂಪಿಗೆ ಸೇರಿದವನಲ್ಲ. ಸತ್ತರೆ ಕ್ಷತ್ರಿಯನಿಗೆ ವೀರಸ್ವರ್ಗವಿದೆ ಎಂದು ಹಿಂದಿನಿಂದಲೂ ನಮ್ಮ ಶಾಸ್ತ್ರಗಳು ಸಾರುತ್ತಿವೆ. ಬೇರೆ ಯಾವುದೋ ಒಂದು ಸ್ವರ್ಗಲೋಕಕ್ಕೆ ಹೋಗಿ ಆನಂದ ಪಡುತ್ತಾರೆ ಎಂದು ಭಾವಿಸಬಹುದು. ಇಲ್ಲವೇ ಈ ಪ್ರಪಂಚದಲ್ಲಿ ಒಂದು ಉತ್ತಮವಾದ ಕುಲದಲ್ಲಿ ಹುಟ್ಟಿ ಆನಂದಪಡಬಹುದು ಎಂದು ಬೇಕಾದರೆ ಭಾವಿಸಬಹುದು. ಏಕೆಂದರೆ ಅರ್ಜುನ ಒಂದು ಧರ್ಮಯುದ್ಧವನ್ನು ಮಾಡುವಾಗ ಮಡಿಯಬಹುದು. ಅದಕ್ಕೆ ಪುಣ್ಯ ಅವನಿಗೆ ಬರುವುದರಲ್ಲಿ ಸಂದೇಹವಿಲ್ಲ. ಗೆದ್ದರೆ ಈ ಪ್ರಪಂಚವೇ ಇದೆ ಅನುಭವಿಸುವುದಕ್ಕೆ. ಇದನ್ನು ಯಾರೂ ಅನುಮಾನಿಸುವಂತಿಲ್ಲ. ಅಂತು ಹೇಗಾ ದರೂ ಆಗಲಿ, ಯುದ್ಧಮಾಡುವುದಕ್ಕೆ ಮನಸ್ಸನ್ನು ಅಣಿಮಾಡಿ ಸೊಂಟ ಕಟ್ಟಿ ನಿಲ್ಲು ಎನ್ನುವನು.

\begin{shloka}
ಸುಖದುಃಖೇ ಸಮೇ ಕೃತ್ವಾ ಲಾಭಾಲಾಭೌ ಜಯಾಜಯೌ~।\\ತತೋ ಯುದ್ಧಾಯ ಯುಜ್ಯಸ್ವ ನೈವಂ ಪಾಪಮವಾಪ್ಸ್ಯಸಿ \hfill॥ ೩೮~॥
\end{shloka}

\begin{artha}
ಸುಖ ದುಃಖ, ಲಾಭ ನಷ್ಟ, ಜಯ ಅಪಜಯ ಇವುಗಳನ್ನು ಸಮನಾಗಿ ನೋಡಿಕೊಂಡು ಅನಂತರ ಯುದ್ಧಕ್ಕೆ ಸಿದ್ಧನಾಗು. ಆಗ ನೀನು ಪಾಪವನ್ನು ಹೊಂದುವುದಿಲ್ಲ.
\end{artha}

ಶ‍್ರೀಕೃಷ್ಣ ಕೆಳಮಟ್ಟಕ್ಕೆ ಬಂದು ಪುನಃ ಮೇಲೇಳುತ್ತಾನೆ. ಹಾಡುವವನು ತುಂಬಾ ಕೆಳಗೆ ಬಂದು ಹಾಡಿದಮೇಲೆ ಬಹಳ ಮೇಲಿರುವ ಮೆಟ್ಟಿಲಿಗೆ ಹೋಗುತ್ತಾನಲ್ಲ ಹಾಗೆ ಭಗವಂತನ ಸಂದೇಶ. ಯುದ್ಧಮಾಡುವುದಕ್ಕೆ ಮುಂಚೆ ಮನಸ್ಸನ್ನು ಅಣಿಮಾಡಿಕೊಳ್ಳಬೇಕು. ಅನಂತರ ಯುದ್ಧಮಾಡಿದರೆ ಅದರ ಪರಿಣಾಮಕ್ಕೆ ಸಿಲುಕುವುದಿಲ್ಲ. ಅದನ್ನೇ ಶ‍್ರೀಕೃಷ್ಣ ಮುಂದೆ ಕರ್ಮಕೌಶಲ ಎಂದು ಕರೆದದ್ದು. ಶ‍್ರೀರಾಮಕೃಷ್ಣರು ಹಲಸಿನ ಹಣ್ಣನ್ನು ಬಿಡಿಸುವುದಕ್ಕೆ ಮುಂಚೆ ಕೈಗೆ ಎಣ್ಣೆ ಸವರಿಕೊಳ್ಳಿ ಎಂದು ಹೇಳುತ್ತಿದ್ದರು. ಕೈಗೆ ಎಣ್ಣೆ ಸವರಿಕೊಂಡರೆ ಅಂಟು ತಾಕುವುದಿಲ್ಲ. ಇಲ್ಲದೆ ಇದ್ದರೆ ಹಲಸಿನ ಹಣ್ಣಿನ ವಾಸನೆ ಹೋದರೂ ಅಂಟು ಬಿಡುವಂತಿಲ್ಲ. ಆ ಎಣ್ಣೆಯನ್ನೇ ಅನಾಸಕ್ತಿ ಎಂದು ಕರೆಯುತ್ತಿದ್ದರು.

ದ್ವಂದ್ವ ಅನುಭವಗಳನ್ನು ಒಂದೇ ದೃಷ್ಟಿಯಿಂದ ನೋಡುವುದು ಹುಟ್ಟಿದೊಡನೆ ಎಲ್ಲರಿಗೂ ಬರುವುದಿಲ್ಲ. ಎಲ್ಲಿಯೋ ಕೆಲವು ಅದೃಷ್ಟಶಾಲಿಗಳಿಗೆ ಮಾತ್ರ ಇದು ಸಾಧ್ಯ. ಉಳಿದವರು ಇದನ್ನು ಅಭ್ಯಾಸ ಮಾಡಿದರೂ ತಕ್ಷಣವೇ ಸಿದ್ಧಿಸುವಂತಹದಲ್ಲ. ಎಷ್ಟುಸಾರಿ ಆದರ್ಶದಿಂದ ಜಾರಿದರೂ ಪಟ್ಟು ಹಿಡಿದು ಬಿಡದೆ ಹೋರಾಡುತ್ತಿರಬೇಕು. ಅನಂತರವೇ ಕ್ರಮೇಣ ಇವು ನಮ್ಮ ಸ್ವಭಾವವಾಗುತ್ತ ಬರುವುವು. ಅಭ್ಯಾಸ ಮಾಡುವಾಗ ಮೊದಲು ಮನಸ್ಸಿಟ್ಟು ನಮಗೆ ನಾವೇ ಬುದ್ಧಿಹೇಳಿಕೊಂಡು ತಿದ್ದಿಕೊಳ್ಳಬೇಕು.

ಸುಖದುಃಖವನ್ನು ಸಮನಾಗಿ ನೋಡಬೇಕು. ಜೀವನದಲ್ಲಿ ಎಲ್ಲ ಜೀವರ ಪಾಲಿಗೂ ಸುಖದುಃಖ ಒಂದಾದ ಮೇಲೊಂದು ಹಗಲು ರಾತ್ರಿಗಳಂತೆ ಬರುತ್ತಿರುವುದು. ಯಾವುದೂ ಶಾಶ್ವತವಾಗಿ ಇರುವುದಿಲ್ಲ. ಒಂದಿರುವಾಗಲೇ ಮತ್ತೊಂದು ಬರುವುದಕ್ಕೆ ಸನ್ನಾಹವಾಗುತ್ತಿರುವುದು. ಸೇತುವೆಯ ಒಂದು ಕಡೆಯಿಂದ ನೀರು ಬಂದು ಹರಿದುಕೊಂಡು ಇನ್ನೊಂದು ಕಡೆಗೆ ಹೋಗುವಂತೆ ಇದೆ.

ಇದರಂತೆಯೇ ಲಾಭ ನಷ್ಟ. ಯಾವನು ವ್ಯಾಪಾರಕ್ಕೆ ಇಳಿಯುವನೊ ಅವನು ಎರಡಕ್ಕೂ ಸಿದ್ಧನಾಗಿರಬೇಕು. ಒಂದು ಸರಿ ಸಾಮಾನಿನ ಬೆಲೆ ಏರುವುದು. ಬೇಕಾದಷ್ಟು ಲಾಭ ಬರುವುದು. ಅದರಂತೆ ಅವನಿಗೆ ನಷ್ಟವೂ ಕಾದಿರುವುದು. ಬರೀ ಲಾಭಕ್ಕೆ ಬಾಯಿಬಿಟ್ಟುಕೊಂಡಿರುವವನು ವ್ಯಾಪಾರದಲ್ಲಿ ಮುಂದುವರಿಯಲಾರ. ನಷ್ಟದ ಪೆಟ್ಟನ್ನು ಸಹಿಸಲು ಸಿದ್ಧವಾದವನಿಗೆ ಮಾತ್ರ ಲಕ್ಷಿ ್ಮ ಒಲಿಯ ಬೇಕಾದರೆ.

ಜಯ ಅಪಜಯ, ಯುದ್ಧದಲ್ಲಾಗಲೀ ಆಟದಲ್ಲಾಗಲೀ ಯಾವನು ಎರಡಕ್ಕೂ ಸಿದ್ಧನಾಗಿರುವನೊ ಅವನೇ ಶ್ರೇಷ್ಠ ಆಟಗಾರ. ಗೆದ್ದಾಗ ಅವನ ತಲೆತಿರುಗಿ ಹೋಗಿಬಿಡುವುದಿಲ್ಲ. ಎಷ್ಟೊ ಸಲ ಹಲವು ಪ್ರಸಂಗಗಳಲ್ಲಿ ಸೋಲನ್ನೂ ಅನುಭವಿಸಬೇಕಾಗುವುದು ಎಂಬುದನ್ನು ಅವನು ಚೆನ್ನಾಗಿ ಮನಗಂಡಿರುವನು. ಹಾಗೆಯೇ ಅವನು ಸೋತರೂ ಹತಾಶನಾಗುವುದಿಲ್ಲ. ಅದರ ಭಾರಕ್ಕೆ ಕುಗ್ಗಿಹೋಗುವುದಿಲ್ಲ.

ಈ ರೀತಿ ಯುದ್ಧಮಾಡಿದರೆ, ಅದರ ಫಲವನ್ನು ಗಮನಿಸದೆ, ಇದೊಂದು ಕ್ಷತ್ರಿಯನ ಕರ್ತವ್ಯ ಧರ್ಮಯುದ್ಧದಲ್ಲಿ ಭಾಗವಹಿಸುವುದು, ಇದರಿಂದ ಸೋಲಾದರೂ ಬರಲಿ ಜಯವಾದರೂ ಬರಲಿ ಎಂಬ ದೃಷ್ಟಿಯಿಂದ ಮಾಡಿದರೆ, ಯಾವ ಪಾಪವೂ ಬರುವುದಿಲ್ಲ. ಇದು ಕೇವಲ ಯುದ್ಧಮಾಡುವುದಕ್ಕೆ ಮಾತ್ರ ಅನ್ವಯಿಸುವುದಿಲ್ಲ. ಎಲ್ಲಾ ಕಾರ್ಯಕ್ಷೇತ್ರಗಳಲ್ಲಿಯೂ ಸತ್ಯ ಇದು.

\begin{shloka}
ಏಷಾ ತೇಽಭಿಹಿತಾ ಸಾಂಖ್ಯೇ ಬುದ್ಧಿರ್ಯೋಗೇ ತ್ವಿಮಾಂ ಶೃಣು~।\\ಬುದ್ಧ್ಯಾ ಯುಕ್ತೋ ಯಯಾ ಪಾರ್ಥ ಕರ್ಮಬಂಧಂ ಪ್ರಹಾಸ್ಯಸಿ \hfill॥ ೩೯~॥
\end{shloka}

\begin{artha}
ಇದುವರೆಗೆ ನಿನಗೆ ಹೇಳಿದ್ದು ಸಾಂಖ್ಯದೃಷ್ಟಿಗೆ ಸೇರಿದ್ದು. ಇನ್ನು ಯೋಗದ ವಿಷಯವಾಗಿ ಕೇಳು. ಪಾರ್ಥ, ನೀನು ಈ ಬುದ್ಧಿಯಿಂದ ಕೂಡಿದವನಾದರೆ ಬಂಧನದಿಂದ ಪಾರಾಗುವೆ.
\end{artha}

ಸಾಂಖ್ಯ ಎಂದರೆ ಜ್ಞಾನದೃಷ್ಟಿ. ಪಾರಮಾರ್ಥಿಕ ದೃಷ್ಟಿ. ಆತ್ಮ ಬೇರೆ, ದೇಹ ಬೇರೆ. ದೇಹ ಧರ್ಮಗಳಿಂದ ಬದ್ಧವಾಗಿಲ್ಲ ಆತ್ಮ. ಈ ಪ್ರಪಂಚದಲ್ಲಿ ಸುಖ ದುಃಖಗಳು ಬಂದು ಹೋಗುವ ಸ್ವಭಾವವುಳ್ಳವುಗಳು. ಅವನ್ನು ಸಹಿಸಿಕೊಳ್ಳಬೇಕು ಮುಂತಾದ ವಿಷಯಗಳೆಲ್ಲ ಆಧ್ಯಾತ್ಮಿಕ ದೃಷ್ಟಿಗೆ ಸೇರಿದ್ದು. ಅನಂತರ ಕೀರ್ತಿಲಾಭ ಮುಂತಾದ ವ್ಯಾವಹಾರಿಕ ದೃಷ್ಟಿಯಿಂದ ಮಾತನಾಡುತ್ತಾನೆ. ಇವುಗಳನ್ನೆಲ್ಲ ಬೇರೆ ಬೇರೆಯವರ ದೃಷ್ಟಿಯಿಂದ ನೋಡಿ ಅರ್ಜುನ ಕರ್ಮಗಳನ್ನು ಮಾಡಬೇಕಾಗಿದೆ ಎಂಬುದನ್ನು ಸಮರ್ಥಿಸುತ್ತಾನೆ.

ಇನ್ನು ಮೇಲೆ ಯೋಗದ ವಿಷಯ ಎಂದರೆ ಕರ್ಮಯೋಗದ ದೃಷ್ಟಿಯಿಂದ ಕೆಲಸ ಮಾಡಿದರೆ ಕರ್ಮಬಂಧನದಿಂದ ಪಾರಾಗುವೆ ಎನ್ನುತ್ತಾನೆ. ಅಂದರೆ ಕರ್ಮವನ್ನು ಕರ್ಮಯೋಗವನ್ನಾಗಿ ಮಾಡಿದರೆ ಎಂದು ಅರ್ಥ. ನಾವೆಲ್ಲ ಕರ್ಮವನ್ನು ಮಾಡುತ್ತಾ ಇರುವೆವು. ಕರ್ಮವನ್ನು ಮಾಡದೆ ಇರುವವರು ಅಪರೂಪ ಪ್ರಪಂಚದಲ್ಲಿ. ಕರ್ಮವನ್ನು ಕರ್ಮಯೋಗವನ್ನಾಗಿ ಹೇಗೆ ಮಾರ್ಪಡಿಸುವುದು ಎಂಬುದನ್ನು ವಿವರಿಸುತ್ತಾನೆ. ಮನುಷ್ಯನನ್ನು ಕಟ್ಟಿಹಾಕುವುದು ಕರ್ಮವಲ್ಲ; ಕರ್ಮಫಲಾಸಕ್ತಿ. ಯಾವಾಗ ನಾವು ಆಸಕ್ತಿಯನ್ನು ತೊರೆದು ಕೆಲಸ ಮಾಡುತ್ತೇವೆಯೊ ಆಗ ಎಷ್ಟು ಕೆಲಸ ಮಾಡಿದರೂ ನಾವು ಬದ್ಧರಾಗುವುದಿಲ್ಲ. ಈ ದೃಷ್ಟಿಯೇ ಶ‍್ರೀಕೃಷ್ಣನ ಒಂದು ಹೊಸ ಕೊಡುಗೆ ಆಧ್ಯಾತ್ಮಿಕ ಜೀವನಕ್ಕೆ.

\begin{shloka}
ನೇಹಾಭಿಕ್ರಮನಾಶೋಽಸ್ತಿ ಪ್ರತ್ಯವಾಯೋ ನ ವಿದ್ಯತೇ~।\\ಸ್ವಲ್ಪಮಪ್ಯಸ್ಯ ಧರ್ಮಸ್ಯ ತ್ರಾಯತೇ ಮಹತೋ ಭಯಾತ್ \hfill॥ ೪ಂ~॥
\end{shloka}

\begin{artha}
ಇಲ್ಲಿ ಮಾಡಿದ ಪ್ರಯತ್ನ ವ್ಯರ್ಥವಾಗುವಂತಿಲ್ಲ. ಒಂದಕ್ಕೆ ಮತ್ತೊಂದು ಫಲ ಬರುವುದಿಲ್ಲ. ಈ ಧರ್ಮವನ್ನು ಸ್ವಲ್ಪ ಅನುಷ್ಠಾನ ಮಾಡಿದರೂ ಮಹಾಭಯದಿಂದ ಪಾರುಮಾಡುವುದು.
\end{artha}

ಇಲ್ಲಿ ಎಂದರೆ ನಿಷ್ಕಾಮ ಕರ್ಮಯೋಗದಲ್ಲಿ ಮಾಡಿದ ಪ್ರಯತ್ನ ಯಾವುದೂ ವ್ಯರ್ಥವಾಗುವುದಿಲ್ಲ. ಹಲವಾರು ಕರ್ಮಗಳಲ್ಲಿ ಕೊನೆಯನ್ನು ಮುಟ್ಟುವ ತನಕ ಕೆಲಸ ಮಾಡದೆ ಅರ್ಧದಲ್ಲಿ ಬಿಟ್ಟರೆ ಅದು ಬರೀ ವ್ಯರ್ಥವಾಗುವುದು. ಒಂದು ಬಾವಿ ತೋಡುತ್ತಾ ಇರುತ್ತಾರೆ. ಇನ್ನೂ ಜಲ ಸಿಕ್ಕಿರುವುದಿಲ್ಲ. ಇನ್ನು ಅಗೆಯದೆ ಬಿಟ್ಟರೆ ಮಾಡಿದ ಕೆಲಸವೆಲ್ಲ ವ್ಯರ್ಥವಾದಂತೆ. ಅದರಂತೆ ಅಲ್ಲ ಇಲ್ಲಿ. ತನ್ನ ಪಾಲಿಗೆ ಯಾವ ಕೆಲಸ ಬರುವುದೊ ಅದನ್ನು ಗುರಿ ಮುಟ್ಟುವ ತನಕವೂ ಮಾಡಬೇಕಾ ಗಿಲ್ಲ. ಎಷ್ಟು ಮಾಡಿದರೆ ಅಷ್ಟು ಪ್ರಯೋಜನ ದೊರಕುವುದು. ಕೂಲಿ ಒಂದು ದಿನ ಕೆಲಸ ಮಾಡುತ್ತಾನೆ. ಅವನಿಗೆ ಒಂದು ದಿನದ ಕೂಲಿಯನ್ನು ಕೊಡುತ್ತಾರೆ. ಆ ಕೆಲಸ ಪೂರ್ತಿ ಆಗುವ ತನಕ ಏನೂ ಅವನು ಕಾದುಕೊಂಡಿರಬೇಕಾಗಿಲ್ಲ. ಕರ್ಮಯೋಗದ ದೃಷ್ಟಿಯಿಂದ ನಾವು ಎಷ್ಟು ಕೆಲಸ ಮಾಡುತ್ತೇವೆಯೊ ಅಷ್ಟು ಚಿತ್ತ ಶುದ್ಧಿಯಾಗುವುದು.

ಇಲ್ಲಿ ಒಂದಕ್ಕೆ ಮತ್ತೊಂದು ಫಲ ಬರುವ ಅಂಜಿಕೆಯೂ ಇಲ್ಲ. ಕೆಲವು ವೇಳೆ ರೋಗಿಗೆ ಔಷಧಿಯನ್ನು ಕೊಡುವಾಗ ಒಂದು ಖಾಯಿಲೆಗೆ ಮತ್ತಾವುದೋ ಖಾಯಿಲೆಯ ಮದ್ದನ್ನು ಕೊಟ್ಟರೆ ಅವನ ರೋಗ ಮತ್ತೂ ಉಲ್ಬಣವಾಗುವುದು. ಇಲ್ಲಿ ಹಾಗಲ್ಲ. ಫಲಾಪೇಕ್ಷೆಯಿಲ್ಲದೆ ಯಾವ ಕೆಲಸವನ್ನು ಮಾಡಿದರೂ ಒಳ್ಳೆಯದೇ ಆಗುವುದು. ಆ ಒಳ್ಳೆಯದೇ ಚಿತ್ತ ಶುದ್ಧಿ. ಇಲ್ಲಿ ಕೆಲಸ ಮಾಡುವಾಗ ಬಾಹ್ಯ ದೃಷ್ಟಿಯನ್ನು ನಿರ್ಲಕ್ಷಿಸುವುದು. ಅದರಿಂದ ಯಾವಾಗಲೂ ಬಾಹ್ಯದಲ್ಲಿ ಶ್ರೇಷ್ಠ ಪರಿಣಾಮ ಆಗಿಯೇ ಆಗುವುದು. ಆದರೆ ಕರ್ಮಯೋಗಿಗೆ ಇದಲ್ಲ ಅಷ್ಟು ಮುಖ್ಯ.ಅವನು ಇದನ್ನು ತನ್ನ ಶೀಲದ ಮೇಲೆ ಎಂತಹ ಪರಿಣಾಮವನ್ನು ಬಿಟ್ಟಿದೆ ಎಂಬ ದೃಷ್ಟಿಯಿಂದ ನೋಡುವನು. ಹೆಚ್ಚು ನಿಸ್ವಾರ್ಥನಾಗುತ್ತ ಬಂದರೆ, ಹೆಚ್ಚು ಪರಿಶುದ್ಧನಾಗುತ್ತ ಬಂದರೆ, ಅಷ್ಟು ಯೋಗದ ದೃಷ್ಟಿಯಿಂದ ಚೆನ್ನಾಗಿ ಕೆಲಸ ಮಾಡುತ್ತಿರುವೆನು ಎಂದು ಅವನಿಗೇ ಸಮಾಧಾನವಾಗುವುದು. ಇಲ್ಲಿ ಯಾರೋ ಹೊರಗಡೆಯವರು ಇವನನ್ನು ಹೊಗಳುವುದಿಲ್ಲ. ಅಥವಾ ಅಲ್ಲಿಂದ ಬರುವ ಫಲವೂ ಅಲ್ಲ ಮುಖ್ಯ. ಚೆನ್ನಾಗಿ ಕೆಲಸ ಮಾಡಿದರೆ, ಅದು ಇವನ ಚಾರಿತ್ರ್ಯವನ್ನು ಶುದ್ಧಿಮಾಡುವುದು, ಭಗವಂತನ ಸಮೀಪಕ್ಕೆ ಇವನನ್ನು ಕರೆದುಕೊಂಡು ಹೋಗುವುದು.

ಇಲ್ಲಿ ಒಂದಕ್ಕೆ ಮತ್ತೊಂದು ಫಲ ಬರುವ ಅಂಜಿಕೆಯೂ ಇಲ್ಲ. ಕೆಲವು ವೇಳೆ ರೋಗಿಗೆ ಔಷಧಿಯನ್ನು ಕೊಡುವಾಗ ಒಂದು ಖಾಯಿಲೆಗೆ ಮತ್ತಾವುದೋ ಖಾಯಿಲೆಯ ಮದ್ದನ್ನು ಕೊಟ್ಟರೆ ಅವನ ರೋಗ ಮತ್ತೂ ಉಲ್ಬಣವಾಗುವುದು. ಇಲ್ಲಿ ಹಾಗಲ್ಲ. ಫಲಾಪೇಕ್ಷೆಯಿಲ್ಲದೆ ಯಾವ ಕೆಲಸವನ್ನು ಮಾಡಿದರೂ ಒಳ್ಳೆಯದೇ ಆಗುವುದು. ಆ ಒಳ್ಳೆಯದೇ ಚಿತ್ತ ಶುದ್ಧಿ. ಇಲ್ಲಿ ಕೆಲಸ ಮಾಡುವಾಗ ಬಾಹ್ಯ ದೃಷ್ಟಿಯನ್ನು ನಿರ್ಲಕ್ಷಿಸುವುದು. ಅದರಿಂದ ಯಾವಾಗಲೂ ಬಾಹ್ಯದಲ್ಲಿ ಶ್ರೇಷ್ಠ ಪರಿಣಾಮ ಆಗಿಯೇ ಆಗುವುದು. ಆದರೆ ಕರ್ಮಯೋಗಿಗೆ ಇದಲ್ಲ ಅಷ್ಟು ಮುಖ್ಯ. ಅವನು ಇದನ್ನು ತನ್ನ ಶೀಲದ ಮೇಲೆ ಎಂತಹ ಪರಿಣಾಮವನ್ನು ಬಿಟ್ಟಿದೆ ಎಂಬ ದೃಷ್ಟಿಯಿಂದ ನೋಡುವನು. ಹೆಚ್ಚು ನಿಃಸ್ವಾರ್ಥನಾಗುತ್ತ ಬಂದರೆ, ಹೆಚ್ಚು ಪರಿಶುದ್ಧನಾಗುತ್ತ ಬಂದರೆ, ಅಷ್ಟು ಯೋಗದ ದೃಷ್ಟಿಯಿಂದ ಚೆನ್ನಾಗಿ ಕೆಲಸ ಮಾಡುತ್ತಿರುವೆನು ಎಂದು ಅವನಿಗೇ ಸಮಾಧಾನವಾಗುವುದು. ಇಲ್ಲಿ ಯಾರೋ ಹೊರಗಡೆಯವರು ಇವನನ್ನು ಹೊಗಳುವುದಿಲ್ಲ, ಅಥವಾ ಅಲ್ಲಿಂದ ಬರುವ ಫಲವೂ ಅಲ್ಲ ಮುಖ್ಯ. ಚೆನ್ನಾಗಿ ಕೆಲಸ ಮಾಡಿದರೆ, ಅದು ಇವನ ಚಾರಿತ್ರ್ಯವನ್ನು ಶುದ್ಧಿಮಾಡುವುದು, ಭಗವಂತನ ಸಮೀಪಕ್ಕೆ ಇವನನ್ನು ಕರೆದುಕೊಂಡು ಹೋಗುವುದು.

ಒಂದು ಸ್ವಲ್ಪ ಈ ದೃಷ್ಟಿಯ ಅನುಷ್ಠಾನದಿಂದಲೂ ಒಬ್ಬ ಮಹಾ ಭಯದಿಂದ ಪಾರಾಗುವನು. ಒಂದು ಸ್ವಲ್ಪ ಅನಾಸಕ್ತಿಯ ದೃಷ್ಟಿಯಿಂದ ಕೆಲಸ ಮಾಡಿದರೂ ಮಹಾ ಭಯದಿಂದ ಪಾರುಮಾಡುವುದು. ಆ ಮಹಾ ಭಯ ಯಾವುದು? ಅಧಿಕಾರಕ್ಕೆ, ಕೀರ್ತಿಗೆ, ಲಾಭಕ್ಕೆ ಕೆಲಸ ಮಾಡುತ್ತ ಅದರ ಗೋಜಿನಲ್ಲಿ ನರಳುವುದೇ ಮಹಾ ಭಯ. ಏಕಾದರೂ ಈ ಕೆಲಸಕ್ಕೆ ಕೈಹಾಕಿದೆವೊ ಎನ್ನಿಸುವುದು. ಕರ್ಮಯೋಗಿ ಇಂತಹ ಜಂಜಡಕ್ಕೆ ಸಿಕ್ಕುವುದಿಲ್ಲ. ಫಲಾಪೇಕ್ಷೆಯನ್ನು ಬಿಟ್ಟು ಕೆಲಸ ಮಾಡುತ್ತ ಹೋಗುವನು. ಯಾವಾಗ ಬೇಕಾದರೂ ಎಂತಹ ಕೆಲಸದಿಂದ ಬೇಕಾದರೂ ಅವನು ತನ್ನ ಕೈಯನ್ನು ತೊಳೆದುಕೊಳ್ಳಬಲ್ಲ. ಅವನು ಆಸಕ್ತನಲ್ಲ. ಪ್ರಪಂಚವನ್ನು ಉದ್ಧರಿಸುವುದಕ್ಕೆ ಹೊರಟಿರುವೆ ಎಂಬು ದಲ್ಲ ಅವನ ದೃಷ್ಟಿ. ಭಗವಂತನ ಕೈಗಳಲ್ಲಿ ತಾನೊಂದು ನಿಮಿತ್ತವಾಗಿ ಕೆಲಸ ಮಾಡುತ್ತಿರುವನು.

\begin{shloka}
ವ್ಯವಸಾಯಾತ್ಮಿಕಾ ಬುದ್ಧಿರೇಕೇಹ ಕುರುನಂದನ~।\\ಬಹುಶಾಖಾ ಹ್ಯನಂತಾಶ್ಚ ಬುದ್ಧಯೋಽವ್ಯವಸಾಯಿನಾಮ್ \hfill॥ ೪೧~॥
\end{shloka}

\begin{artha}
ಕುರುನಂದನ, ಇಲ್ಲಿ ನಿಶ್ಚಯ ಸ್ವಭಾವವುಳ್ಳವರ ಬುದ್ಧಿ ಏಕನಿಷ್ಠವಾಗಿರುತ್ತದೆ. ಯಾರು ಅನಿಶ್ಚಿತರಾಗಿರುವರೊ ಅವರ ಬುದ್ಧಿ ಕವಲೊಡೆದು ಹಲವು ಶಾಖೆಗಳಾಗಿವೆ.
\end{artha}

ನಿಶ್ಚಯ ಸ್ವಭಾವವುಳ್ಳವರೇ ಕರ್ಮಯೋಗಿಗಳು. ಅವರು ಮುಂಚೆ ಯಾವ ಕೆಲಸವನ್ನು ಮಾಡ ಬೇಕಾಗಿದೆ, ಹೇಗೆ ಮಾಡಬೇಕಾಗಿದೆ ಎಂಬುದನ್ನೆಲ್ಲ ಮೊದಲೇ ವಿಚಾರಮಾಡಿ ನಿರ್ಣಯಿಸುತ್ತಾರೆ. ಕೆಲಸ ಮಾಡುವಾಗ ಅನುಮಾನಿಸುವುದೇ ಇಲ್ಲ. ಬಾಣ ಬಿಲ್ಲಿನಿಂದ ಹೊರಹೊಮ್ಮುವಾಗ ಅತ್ತ ಇತ್ತ ಚಲಿಸದೆ ಗುರಿ ಎಡೆಗೆ ಹೇಗೆ ಧಾವಿಸುವುದೋ ಹಾಗೆಯೇ ಕರ್ಮಯೋಗಿಯ ಮನಸ್ಸು. ಅವನು ಕೆಲಸದಲ್ಲಿ ನಿರತನಾಗಿರುವಾಗ, ಯಾರು ಏನನ್ನಾದರೂ ಹೇಳಲಿ, ಯಾವ ಆತಂಕಗಳೇ ಬರಲಿ, ಕೆಲಸವನ್ನು ಸಾಗಿಸಿಕೊಂಡೇ ಹೋಗುವನು. ಜನರ ಟೀಕೆ ನಿಂದೆಗಳ ಮೇಲೆ ಅವನ ಗಮನವೇ ಇರುವುದಿಲ್ಲ. ಹಾಗೆಯೆ ಅವರ ಹೊಗಳಿಕೆ ಮುಂತಾದುವುಗಳನ್ನು ಗಮನಿಸುವುದೇ ಇಲ್ಲ.

ಯಾರು ಅವ್ಯವಸಾಯಿಗಳೊ ಎಂದರೆ ನಿಶ್ಚಯ ಬುದ್ಧಿ ಇಲ್ಲದವರೊ ಅವನ ಮನಸ್ಸು ಇಪ್ಪತ್ತೆಂಟು ಕವಲೊಡೆದು ಹೋಗಿರುತ್ತದೆ. ಹೆಜ್ಜೆಹೆಜ್ಜೆಗೆ ಅವನು ಬದಲಾಯಿಸುತ್ತ ಇರುವನು. ಕೀರ್ತಿ ಲಾಭ ಯಶಸ್ಸು ಈ ದಾರಿಗಳಲ್ಲೆಲ್ಲ ನುಗ್ಗಿ ಹೋಗುತ್ತಿರುತ್ತದೆ. ಯಾವ ಆತಂಕ ಬಂದರೂ ಹತಾಶನಾಗುವನು. ಯಾರಾದರೂ ಹೊಗಳಿದರೆ ಕುಣಿದಾಡುವನು. ಫಲ ಬರುವಾಗ ಬಾಯಿ ನೀರೂರುವುದು. ಬರದೆ ಇದ್ದರೆ ದುಃಖಪಡುವನು. ಕಹಿ ಫಲ ಬಂದರೆ, ಏಕಾದರೂ ನಾನು ಈ ಕೆಲಸ ಮಾಡಿದೆನೊ ಎಂದು ಪೇಚಾಡುವನು.

\begin{shloka}
ಯಾಮಿಮಾಂ ಪುಷ್ಪಿತಾಂ ವಾಚಂ ಪ್ರವದಂತ್ಯವಿಪಶ್ಚಿತಃ~।\\ವೇದವಾದರತಾಃ ಪಾರ್ಥ ನಾನ್ಯದಸ್ತೀತಿ ವಾದಿನಃ \hfill॥ ೪೨~॥
\end{shloka}

\begin{shloka}
ಕಾಮಾತ್ಮಾನಃ ಸ್ವರ್ಗಪರಾ ಜನ್ಮಕರ್ಮಫಲಪ್ರದಾಮ್~।\\ಕ್ರಿಯಾವಿಶೇಷಬಹುಲಾಂ ಭೋಗೈಶ್ವರ್ಯಗತಿಂ ಪ್ರತಿ \hfill॥ ೪೩~॥
\end{shloka}

\begin{shloka}
ಭೋಗೈಶ್ವರ್ಯಪ್ರಸಕ್ತಾನಾಂ ತಯಾಪಹೃತಚೇತಸಾಮ್~।\\ವ್ಯವಸಾಯಾತ್ಮಿಕಾ ಬುದ್ಧಿಃ ಸಮಾಧೌ ನ ವಿಧೀಯತೇ \hfill॥ ೪೪~॥
\end{shloka}

\begin{artha}
ವೇದಗಳಲ್ಲಿರುವ ಅರ್ಥವಾದಗಳಲ್ಲಿಯೇ ನಿರತರಾಗಿ ಮತ್ತೇನು ಇಲ್ಲವೆಂದು ವಾದಿಸುವ, ಕಾಮಾತ್ಮರಾಗಿ ಸ್ವರ್ಗದಲ್ಲಿಯೇ ಮನಸ್ಸನ್ನು ಇಟ್ಟಿರುವ ಅಜ್ಞಾನಿಗಳು, ಜನ್ಮ ಕರ್ಮ ಫಲಪ್ರದವಾಗಿರುವ ಭೋಗೈಶ್ವರ್ಯವನ್ನು ಪಡೆಯುವುದಕ್ಕೆ ಬಗೆಬಗೆಯಾದ ಬಣ್ಣಗಳಿಂದ ಕೂಡಿದ ಬಣ್ಣನೆಯ ಮಾತುಗಳನ್ನು ಆಡುತ್ತಿರುವರು. ಆ ಮಾತಿಗೆ ಮನಸೋತು ಭೋಗೈಶ್ವರ್ಯದಲ್ಲಿ ಆಸಕ್ತರಾಗಿರುವ ಮನಸ್ಸಿರುವವರಿಗೆ ಅಂತಃಕರಣದಲ್ಲಿ ನಿಶ್ಚಯಾತ್ಮಕವಾದ ಜ್ಞಾನ ಎಂದಿಗೂ ಉಂಟಾಗುವುದಿಲ್ಲ.
\end{artha}

ಶ‍್ರೀಕೃಷ್ಣ ಇಲ್ಲಿ ಕರ್ಮಕಾಂಡಕ್ಕೂ ಕರ್ಮಯೋಗಕ್ಕೂ ವ್ಯತ್ಯಾಸವನ್ನು ತೋರಿಸುತ್ತಾನೆ. ಕರ್ಮ ಕಾಂಡಿಗಳು ಮನುಷ್ಯನಿಗೆ ಬೇಕಾದ ವಸ್ತುಗಳನ್ನು ಆಯಾ ಯಾಗ ಯಜ್ಞಾದಿಗಳಿಂದ ಪಡೆಯಬಹುದು ಎಂದು ಸಾರುತ್ತಾರೆ. ಅವರಿಗೆ ಬೇಕಾಗಿರುವುದು ಭೋಗವಸ್ತುಗಳು, ಭಗವತ್ ಪ್ರೀತಿಯೂ ಅಲ್ಲ, ಮುಕ್ತಿಯೂ ಅಲ್ಲ. ಇರುವಾಗ ಚೆನ್ನಾಗಿ ಐಶ್ವರ್ಯ ಅಧಿಕಾರ ಹೆಂಡತಿ ಮಕ್ಕಳು ಮುಂತಾದುವುಗಳೊಂದಿಗೆ ಸುಖಪಡುವುದು. ಇಲ್ಲಿಂದ ಬಿಟ್ಟು ಬೇರೆ ಲೋಕಕ್ಕೆ ಹೋದರೆ ಅಲ್ಲಿಯೂ ಸುಖವಾಗಿರುವುದನ್ನು ಅನುಭವಿಸಬೇಕೆಂದು ಆಸೆ ಇದೆ. ಆದರೆ ಇಲ್ಲಿರುವುದನ್ನು ಅಲ್ಲಿಗೆ ತೆಗೆದುಕೊಂಡು ಹೋಗಲು ಆಗುವುದಿಲ್ಲ. ಅದಕ್ಕಾಗಿ ಪುಣ್ಯ ಸಂಪಾದನೆಗೆ ಹಲವು ಯಾಗ ಯಜ್ಞಗಳನ್ನು ಮಾಡುವರು.

ಅವರು ಎಂದರೆ ಭೋಗವಾಸನೆಯಿಂದ ಕೂಡಿದ ಕರ್ಮಕಾಂಡಿಗಳು, ಇಂತಹ ಸ್ವರ್ಗವನ್ನು ಬಿಟ್ಟರೆ ಇನ್ನು ಯಾವುದನ್ನೂ ಗಣನೆಗೆ ತೆಗೆದುಕೊಳ್ಳುವುದಿಲ್ಲ. ಅವರಿಗೆ ಚಿತ್ತಶುದ್ಧಿಯಾಗಲಿ ಜ್ಞಾನವಾಗಲಿ ಭಕ್ತಿಯಾಗಲಿ ಏನೂ ಬೇಕಾಗಿಲ್ಲ. ಭೋಗದಾಸೆಯ ತೃಪ್ತಿಗೆ ಬಗೆಬಗೆಯ ಬಣ್ಣದ ಮಾತುಗಳನ್ನು ಆಡುತ್ತಿರುತ್ತಾರೆ.

ಇಂತಹ ಆಸೆಯಿಂದ ಕೂಡಿದವರಿಗೆ ಚಿತ್ತ ಏಕಮುಖವಾಗಿರುವುದಿಲ್ಲ. ಅದು ಶುದ್ಧಿಯಾಗಿಯೂ ಇರುವುದಿಲ್ಲ. ಅವರ ಬುದ್ಧಿ ಕದಡಿಹೋಗಿರುತ್ತದೆ. ತಾತ್ಕಾಲಿಕವಾಗಿ ಬರುವ ಸ್ವರ್ಗವನ್ನೇ ಅನಂತವೆಂದು ಭಾವಿಸುತ್ತಾರೆ. ಅವರ ದೃಷ್ಟಿ ಸತ್ಯಾನ್ವೇಷಣೆಯಲ್ಲ, ಭೋಗಾನ್ವೇಷಣೆ. ಎಲ್ಲಿಯವರೆಗೆ ಭೋಗವಾಸನೆಯಿಂದ ಚಿತ್ತ ಹಲವು ಮುಖವಾಗಿ ಒಡೆದುಹೋಗಿದೆಯೊ ಅಂಥವರು ಪರಮ ಜ್ಞಾನವನ್ನು ಪಡೆಯಲಾರರು. ಜ್ಞಾನಕ್ಕೆ ಮನಸ್ಸು ಏಕಮುಖವಾಗಿ ಹರಿಯಬೇಕು. ಬುದ್ಧಿ ಹರಿತವಾಗಿರಬೇಕು. ಸ್ವರ್ಗಾಕಾಂಕ್ಷಿಗಳಲ್ಲಿ ಇವು ಇಲ್ಲ.

\begin{shloka}
ತ್ರೈಗುಣ್ಯವಿಷಯಾ ವೇದಾ ನಿಸ್ತ್ರೈಗುಣ್ಯೋ ಭವಾರ್ಜುನ~।\\ನಿರ್ದ್ವಂದ್ವೋ ನಿತ್ಯಸತ್ತ್ವಸ್ಥೋ ನಿರ್ಯೋಗಕ್ಷೇಮ ಆತ್ಮವಾನ್ \hfill॥ ೪೫~॥
\end{shloka}

\begin{artha}
ಅರ್ಜುನ, ವೇದಗಳು ಮೂರು ಗುಣಗಳನ್ನು ಮಾತ್ರ ಕುರಿತು ಇರುವುವು. ನೀನು ತ್ರಿಗುಣಗಳಿಗೆ ಅತೀತನಾಗು. ನಿರ್ದ್ವಂದ್ವನೂ, ನಿತ್ಯ ಸತ್ತ್ವಸ್ಥನೂ, ನಿರ್ಯೋಗಕ್ಷೇಮನೂ ಆತ್ಮವಂತನೂ ಆಗು.
\end{artha}

ವೇದಗಳು ಸತ್ತ್ವ ರಜಸ್ಸು ಮತ್ತು ತಮೋಗುಣಗಳನ್ನು ಹೇಳುವುವು. ತಮಸ್ಸು ಅಜ್ಞಾನದಲ್ಲಿ ಕಟ್ಟಿಹಾಕುವುದು. ರಜಸ್ಸು ಕರ್ಮದಿಂದ ಬಂಧಿಸುವುದು. ಸತ್ತ್ವ ಜ್ಞಾನ ಸುಖಗಳನ್ನು ನೀಡುವುದು. ಅದು ಒಳ್ಳೆಯದಾದರೂ ಒಂದು ಬಂಧನವೇ. ಶ‍್ರೀರಾಮಕೃಷ್ಣರು ಈ ಮೂರು ಗುಣಗಳು ಮನುಷ್ಯನನ್ನು ಹೇಗೆ ಕಟ್ಟಿಹಾಕುವುವು ಎಂಬುದಕ್ಕೆ ಒಂದು ಉದಾಹರಣೆಯನ್ನು ಕೊಡುತ್ತಿದ್ದರು. ಒಬ್ಬ ಪ್ರಯಾಣಿಕ ಕಾಡಿನಲ್ಲಿ ಹೋಗುತ್ತಿದ್ದ. ಒಂದು ಕಳ್ಳರ ಗುಂಪು ಅವನ ಮೇಲೆ ಬಿದ್ದು ಅವನ ಹತ್ತಿರ ಇರುವುದನ್ನೆಲ್ಲ ಕಸಿದುಕೊಂಡಿತು. ಅವನನ್ನು ಹಾಗೆಯೇ ಬಿಟ್ಟರೆ ಅವನು ಪೋಲೀಸಿನವರಿಗೆ ಹೇಳುತ್ತಾನೆ, ಅವನನ್ನೇ ಕೊಂದುಬಿಡುವಾ ಎಂದು ಹೇಳಿದನು ಒಬ್ಬ. ಇನ್ನೊಬ್ಬ, ಅವನನ್ನು ಏತಕ್ಕೆ ಕೊಲ್ಲಬೇಕು? ಕೈಕಾಲು ಕಟ್ಟಿ ಇವನನ್ನು ಇಲ್ಲಿ ಬಿಟ್ಟರೆ ಸಾಕು ಎಂದು ಹೇಳಿದನು. ಇವರೆಲ್ಲ ಸ್ವಲ್ಪ ಮುಂದೆ ಹೋದಮೇಲೆ ಇವರಲ್ಲಿಯೇ ಇನ್ನೊಬ್ಬ, ಪ್ರಯಾಣಿಕನ ಬಳಿಗೆ ಬಂದು ಅವನ ಕಟ್ಟನ್ನು ಬಿಚ್ಚಿ ಮುಂದೆ ಹೋಗಲು ದಾರಿ ತೋರಿದನು. ಊರು ಇನ್ನೇನು ಹತ್ತಿರ ಬರುತ್ತದೆ ಎಂದಾಗ ಅವನಿಗೆ ಮನೆಗೆ ಹೋಗು ಎಂದನು. ಪ್ರಯಾಣಿಕ ಇಷ್ಟೊಂದು ಸಹಾಯ ಮಾಡಿದ ಮನುಷ್ಯನಿಗೆ ಏನಾದರೂ ಕೊಡಬೇಕೆಂದಿದ್ದನು. ತನ್ನ ಮನೆಗೆ ಬಂದು ಹೋಗು ಎಂದನು. ಆ ಒಳ್ಳೆಯ ಕಳ್ಳನಾದರೊ ನಾನು ಊರಿಗೆ ಬಂದರೆ ಪೋಲೀಸಿನವರು ಹಿಡಿಯುವರು. ನಾನು ನಿನಗೆ ಒಳ್ಳೆಯವನಾದರೂ ಕಳ್ಳರ ಗುಂಪಿಗೆ ಸೇರಿದವನು ಎಂದು ಹೇಳಿ ಹೊರಟುಹೋದನು. ಇಲ್ಲಿ ಇವನನ್ನು ಕೊಲ್ಲಿ ಎಂದವನು ತಮೋಗುಣಿ. ಇವನನ್ನು ಕಟ್ಟಿಹಾಕಿ ಎಂದವನು ರಜೋಗುಣಿ. ಇವನಿಗೆ ದಾರಿ ತೋರಲು ಬಂದವನು ಸತ್ತ್ವಗುಣಿ. ಒಂದಕ್ಕಿಂತ ಮತ್ತೊಂದು ಮೇಲಾದರೂ ಇವೆಲ್ಲಮಾಯಾಪ್ರಪಂಚದಲ್ಲಿವೆ. ಇವುಗಳಿಂದ ತಪ್ಪಿಸಿಕೊಳ್ಳಬೇಕಾದರೆ ಎಲ್ಲಾ ಗುಣಗಳಿಗೂ ಅತೀತನಾದರೆ ಮಾತ್ರ ಸಾಧ್ಯ.

ನೀನು ನಿರ್ದ್ವಂದ್ವನಾಗು ಎನ್ನುವನು ಅರ್ಜುನನಿಗೆ ಶ‍್ರೀಕೃಷ್ಣ. ಅದೇ ಸುಖದುಃಖ ಲಾಭ ನಷ್ಟ ಜಯ ಅಪಜಯ ಮುಂತಾದ ದ್ವಂದ್ವಗಳಿಗೆ ಅತೀತನಾಗಿರುವುದು. ಜೀವನದಲ್ಲಿರುವಾಗ ನಾವು ಎಷ್ಟೇ ಜೋಪಾನವಾಗಿದ್ದರೂ ಈ ದ್ವಂದ್ವಗಳು ನಮ್ಮ ಸಮೀಪಕ್ಕೆ ಬರುವುವು. ಆದರೆ ನಾವು ಆಸಕ್ತರಾಗಬಾರದು. ಅದು ಬರುವುದು ಹೋಗುವುದು. ಒಳ್ಳೆಯದಕ್ಕೆ ಬಾಯಿ ಚಪ್ಪರಿಸುವುದೂ ಇಲ್ಲ, ಕೆಟ್ಟದ್ದು ಬಂದರೆ ವ್ಯಥೆ ಪಡುವುದೂ ಇಲ್ಲ. ಜೀವನದ ಪಯಣದಲ್ಲಿ ಇವುಗಳ ಮೂಲಕ ನಾವು ಸಾಗಿ ಹೋಗಬೇಕಾಗಿದೆ. ಇವೇ ನಮ್ಮ ಗುರಿಯಲ್ಲ.

ಸದಾ ಸತ್ತ್ವಗುಣದಲ್ಲಿ ನೆಲೆಸಿರು ಎನ್ನುವನು. ತ್ರಿಗುಣಗಳಿಗೂ ಅತೀತನಾಗಿ ಹೋಗು ಎಂದು ಹೇಳಿ, ಇಲ್ಲಿ ಸದಾ ಸತ್ತ್ವಗುಣದಲ್ಲಿ ನೆಲಸು ಎಂದರೆ ಪರಸ್ಪರ ವಿರೋಧವಾಗಲಿಲ್ಲವೆ ಎಂದು ನಾವು ಭಾವಿಸಬಹುದು. ಪ್ರಪಂಚದಲ್ಲಿರಬೇಕಾದರೆ ಯಾವುದಾದರೂ ಒಂದು ಗುಣದ ಮೇಲೆ ನಿಂತಿರ ಬೇಕಾಗುವುದು. ಹಾಗೆ ನಿಲ್ಲುವುದಕ್ಕೆ ಸುರಕ್ಷಿತವಾದ ಬಂಡೆಯೇ ಸತ್ತ್ವಗುಣ. ಆದರೆ ಅದರ ಮೇಲಿರುವಾಗ ನಾವು ಅದಕ್ಕೆ ಆಸಕ್ತರಾಗಿರಬಾರದು. ಎಲ್ಲಾ ಗುಣಗಳಿಗಿಂತ ಸತ್ತ್ವಗುಣ ಪರಮಸತ್ಯಕ್ಕೆ ತುಂಬಾ ಹತ್ತಿರದಲ್ಲಿರುವುದು.

ಯೋಗಕ್ಷೇಮವಿಲ್ಲದವನಾಗು ಎನ್ನುವನು. ಮನುಷ್ಯನಲ್ಲಿ ಈ ಎರಡು ಗುಣಗಳು ಯಾವಾಗಲೂ ಇರುವುವು. ಅಲಭ್ಯ ಲಾಭ ಎಂದರೆ ಪಡೆಯದೆ ಇರುವ ವಸ್ತುವನ್ನು ಸಂಪಾದಿಸಬೇಕೆಂದಿರುವುದು. ಲಬ್ಧ ಸಂರಕ್ಷಣ, ಎಂದರೆ ಆಗಲೆ ಪಡೆದಿರುವುದನ್ನು ಬಿಟ್ಟುಹೋಗದಂತೆ ಸಂರಕ್ಷಿಸಿಕೊಂಡಿರುವುದು. ಈ ಎರಡು ಗುಣಗಳೇ ನಮ್ಮ ಬಾಳನ್ನೆಲ್ಲ ಆವರಿಸುವುದು. ನಮಗೆ ಐಶ್ವರ್ಯ ಅಧಿಕಾರ ಪದವಿ ಮುಂತಾದವುಗಳು ಬೇಕು. ಅದನ್ನು ಪಡೆಯುವುದಕ್ಕೆ ಎಲ್ಲಾ ವಿಧದಿಂದಲೂ ಪ್ರಯತ್ನಿಸುತ್ತೇವೆ. ಅದನ್ನು ಪಡೆದಾದಮೇಲೆ ಅದು ನಮ್ಮನ್ನು ಬಿಟ್ಟುಹೋಗದಂತೆ ನೋಡಿಕೊಳ್ಳುತ್ತಿರುತ್ತೇವೆ. ಸಾಧ್ಯವಾದರೆ ಅದು ವಿಸ್ತಾರವಾಗಬೇಕೆಂದು ಇಚ್ಛಿಸುತ್ತೇವೆಯೇ ಹೊರತು ಅದನ್ನು ಕಡಮೆ ಮಾಡಿಕೊಳ್ಳುವುದಕ್ಕೆ ಯಾರಿಗೂ ಇಷ್ಟವಿಲ್ಲ. ಇವೆರಡನ್ನು ಬಿಡು ಎಂದು ಶ‍್ರೀಕೃಷ್ಣ ಅರ್ಜುನನಿಗೆ ಹೇಳುತ್ತಾನಲ್ಲ, ಬಿಟ್ಟರೆ ನಾವು ಬದುಕುವುದು ಹೇಗೆ ಎಂದು ಕೇಳುತ್ತೇವೆ. ಬಿಟ್ಟರೆ ನಮಗೆ ಏನೂ ಅಭಾವ ಇರುವುದಿಲ್ಲ. ಯಾವುದು ನಮ್ಮ ಜೀವನಕ್ಕೆ ಅತ್ಯಾವಶ್ಯಕವಾಗಿ ಬೇಕೊ, ಅದನ್ನು ದೇವರೇ ಒದಗಿಸುವನು, ಮತ್ತೂ ಅವನೇ ರಕ್ಷಿಸುವನು. ಅದನ್ನು ಕುರಿತು ಚಿಂತಿಸಬೇಕಾಗಿಲ್ಲ. ಮನಸ್ಸನ್ನು ಭಗವಂತನ ಸೇವೆ ಧ್ಯಾನ ಮತ್ತು ಅವನ ಪ್ರೀತಿಯಿಂದ ತುಂಬಿದರೆ ದೇವರು ನಮ್ಮ ಇತರ ಜವಾಬ್ದಾರಿಯನ್ನು ನೋಡಿಕೊಳ್ಳುವನು.

ಆತ್ಮವಂತನಾಗು, ಯಾವಾಗಲೂ ನಿನ್ನ ಮನಸ್ಸನ್ನು ಪರಮಾತ್ಮನಿಂದ ತುಂಬಿರು. ಅದರ ಆಧಾರದ ಮೇಲೆ ನಿಂತುಕೊಂಡಿರುವುದು. ಮಗು ಒಂದು ಕಂಬವನ್ನು ಹಿಡಿದುಕೊಂಡು ಸುತ್ತುತ್ತಿರುವಂತೆ ಅದು. ದೇವರನ್ನು ನೆಚ್ಚಿ ಅದನ್ನು ಹಿಡಿದುಕೊಂಡು ನಮ್ಮ ಕೆಲಸವನ್ನೆಲ್ಲ ಮಾಡುತ್ತಿರುವುದು.

\begin{shloka}
ಯಾವಾನರ್ಥ ಉದಪಾನೇ ಸರ್ವತಃಸಂಪ್ಲುತೋದಕೇ~।\\ತಾವಾನ್ ಸರ್ವೇಷು ವೇದೇಷು ಬ್ರಾಹ್ಮಣಸ್ಯ ವಿಜಾನತಃ \hfill॥ ೪೬~॥
\end{shloka}

\begin{artha}
ಸುತ್ತಲೂ ನೀರು ಆವರಿಸಿಕೊಂಡಿರುವಾಗ ಸಣ್ಣ ಕೆರೆಯಿಂದ ಎಷ್ಟು ಪ್ರಯೋಜನವೊ, ಎಲ್ಲವನ್ನೂ ಬಲ್ಲ ತತ್ತ್ವಜ್ಞಾನಿಗೆ ವೇದಗಳಿಂದಲೂ ಅಷ್ಟೇ ಪ್ರಯೋಜನ.
\end{artha}

ಇಲ್ಲಿ ಶಾಸ್ತ್ರಗಳನ್ನು ಬಲ್ಲವನು, ಮತ್ತು ಅದರ ಅನುಭವವನ್ನು ಜೀವನದಲ್ಲಿ ಮನಗಂಡಿರುವ ತತ್ತ್ವಜ್ಞಾನಿ ಇವರಿಬ್ಬರ ಹೋಲಿಕೆ ಇದೆ. ಇನ್ನೂ ಏನೂ ಅನುಭವ ಆಗದವನಿಗೆ ಶಾಸ್ತ್ರವೇ ಆಧಾರ. ಅದೊಂದು ಸಣ್ಣ ಕೆರೆಯಂತೆ. ಅದನ್ನು ಕುಡಿಯುವುದಕ್ಕೆ, ಸ್ನಾನ ಮಾಡುವುದಕ್ಕೆ, ಬಟ್ಟೆ ಒಗೆಯುವುದಕ್ಕೆ ಉಪಯೋಗಿಸುತ್ತಾರೆ. ದಿಗಂತದವರೆಗೂ ವ್ಯಾಪಿಸಿರುವ ದೊಡ್ಡ ತಟಾಕವಾದರೊ, ಅದನ್ನೆ ಸಣ್ಣ ಕೆರೆಯ ಕೆಲಸಗಳಿಗೆಲ್ಲಕ್ಕೂ ಉಪಯೋಗಿಸಬಹುದು, ಮತ್ತು ಅದಕ್ಕಿಂತ ಹೆಚ್ಚನ್ನೂ ಮಾಡಬಹುದು. ಅದರಿಂದ ಸಾವಿರಾರು ಎಕರೆ ಜಮೀನನ್ನು ಬೇಸಾಯ ಮಾಡಬಹುದು. ದೊಡ್ಡ ತಟಾಕ ಸಣ್ಣ ಕಟ್ಟೆ ಮಾಡುವ ಕೆಲಸ ಮತ್ತು ಅದು ಮಾಡಲಾರದ ಕೆಲಸಗಳಿಗೂ ಹೇಗೆ ಸಹಾಯಕ್ಕೆ ಬರುವುದೋ ಹಾಗೆ ಆತ್ಮಜ್ಞಾನಿಗೆ ವೇದಗಳ ಸಾರವೆಲ್ಲ ಅರ್ಥವಾಗಿರುವುದು, ಮತ್ತೂ ಅದಕ್ಕೂ ನಿಲುಕದ ಅನುಭವವನ್ನು ಕೂಡ ಅವನು ಪಡೆದಿರುವನು. ಬರೀ ಶಾಸ್ತ್ರಗಳನ್ನು ಓದಿ ಪಂಡಿತನಾದವನಿಗೆ ಸ್ವಲ್ಪ ಜ್ಞಾನವಿದೆ. ಆದರೆ ಆತ್ಮಸಾಕ್ಷಾತ್ಕಾರವನ್ನು ಪಡೆದ ಜ್ಞಾನಿಯ ಅನುಭವವಾದರೋ ಅನಂತ.

\begin{shloka}
ಕರ್ಮಣ್ಯೇವಾಧಿಕಾರಸ್ತೇ ಮಾ ಫಲೇಷು ಕದಾಚನ~।\\ಮಾ ಕರ್ಮಫಲಹೇತುರ್ಭೂರ್ಮಾ ತೇ ಸಂಗೋಽಸ್ತ್ವಕರ್ಮಣಿ \hfill॥ ೪೭~॥
\end{shloka}

\begin{artha}
ಕರ್ಮ ಮಾಡುವುದಕ್ಕೆ ಮಾತ್ರ ನಿನಗೆ ಅಧಿಕಾರ. ಫಲದಲ್ಲಿ ಎಂದಿಗೂ ಆಸಕ್ತನಾಗಬೇಡ. ಕರ್ಮಫಲಕ್ಕೆ ಕಾರಣವಾಗದೆ ಇರು. ಅಕರ್ಮದಲ್ಲಿಯೂ ನಿನಗೆ ಆಸೆ ಹುಟ್ಟದಿರಲಿ.
\end{artha}

ಕರ್ಮಯೋಗದ ಒಂದು ಶ್ರೇಷ್ಠವಾದ ಭಾವನೆಯನ್ನು ಇಲ್ಲಿ ನೋಡುತ್ತೇವೆ. ಕರ್ಮ ಮಾಡುವುದಕ್ಕೆ ನಮಗೆ ಅಧಿಕಾರ. ಫಲಕ್ಕೆ ಆಸಕ್ತನಾಗಬೇಡ; ಎಂದರೆ ಆ ಫಲಕ್ಕಾಗಿ ಕರ್ಮ ಮಾಡಬೇಡ ಎನ್ನುವನು. ನಾವು ಜೀವನದಲ್ಲಿ ಕಷ್ಟಕ್ಕೆ ಸಿಕ್ಕಿಬೀಳುವುದು, ದುಃಖ ಅನುಭವಿಸುವುದು ಇವಕ್ಕೆಲ್ಲ ಕಾರಣ ಕರ್ಮವಲ್ಲ, ಕರ್ಮಫಲಕ್ಕೆ ನಾವು ಆಸಕ್ತರಾಗಿರುವುದು. ಯಾವಾಗ ನಾವು ಫಲಕ್ಕೆ ಆಸಕ್ತರಾಗುವೆವೊ ಆಗ ಫಲ ಬರದೆ ಇದ್ದರೆ ನಾವು ಮಾಡಿ ಪ್ರಯೋಜನವೇನು ಎಂದು ಭಾವಿಸುವೆವು. ಅದಕ್ಕೆ ಬದ್ಧರಾಗುವೆವು, ಅದನ್ನು ಬಿಡುವುದಕ್ಕೆ ನಮಗೆ ಮನಸ್ಸೇ ಇರುವುದಿಲ್ಲ. ಏನಾದರೂ ಮಾಡಿ ಅದನ್ನು ಉಳಿಸಿಕೊಳ್ಳಲು ಯತ್ನಿಸುವೆವು. ಕೆಲವು ವೇಳೆ ಫಲ ರುಚಿಯಾಗಿಲ್ಲದೆ ಇದ್ದರೆ ಅದನ್ನು ಅನುಭವಿಸಿ ಆದಮೇಲೆ ವ್ಯಥೆಪಡುವೆವು. ಆದಕಾರಣವೆ ನಮ್ಮ ಪಾಲಿಗೆ ಬಂದ ಕೆಲಸವನ್ನು ಮಾಡಿ ಹಾಕಿ ಬಿಡೋಣ.

ಕರ್ಮಫಲಕ್ಕೆ ನೀನು ಕಾರಣನಾಗದೆ ಇರು. ಫಲ ಬರದೇ ಇರುವ ರೀತಿ ಕೆಲಸ ಮಾಡು ಎಂದು ಅಲ್ಲ. ಅದು ಕೆಲಸವನ್ನು ಹಾಳುಮಾಡಿದಂತೆ. ನಿನ್ನಿಂದ ಆ ಫಲ ಬಂತು. ನೀನು ಆ ಫಲಕ್ಕೆ ಒಡೆಯ ಎಂದು ಭಾವಿಸಬೇಡ. ನಾವೆಲ್ಲ ಭಗವಂತನ ಕೆಲಸ ಮಾಡುವುದಕ್ಕೆ ಒಂದು ನಿಮಿತ್ತ. ಇದೆಲ್ಲ ಅವನಿಗೆ ಸೇರಿದ್ದು, ನಮಗಲ್ಲ. ಅವನು ತನ್ನ ಮನಸ್ಸಿಗೆ ತೋರಿದ ರೀತಿಯಲ್ಲಿ ಅದನ್ನು ಇತ್ಯರ್ಥ ಮಾಡಲಿ. ಶ‍್ರೀರಾಮಕೃಷ್ಣರು ಮನೆಯಲ್ಲಿ ಪರಿಚಾರಕ ಹೆಂಗಸಿನಂತೆ ಸಂಸಾರದಲ್ಲಿ ಇರಿ ಎಂದು ಹೇಳುತ್ತಿದ್ದರು. ಆಕೆ ಮನೆಯ ಯಜಮಾನನ ಮಗುವನ್ನು ಎತ್ತಿಕೊಂಡು ಮುದ್ದಾಡುವಾಗ ನನ್ನ ಹರಿ, ನನ್ನ ಕೃಷ್ಣ ಎಂದು ಕರೆಯುವಳು. ಆದರೆ ಅವಳ ಮನಸ್ಸಿನಲ್ಲಿ ಚೆನ್ನಾಗಿ ಗೊತ್ತಿದೆ, ಈ ಮಕ್ಕಳು, ಅವಳ ಹರಿಯೂ ಅಲ್ಲ, ಅವಳ ಕೃಷ್ಣನೂ ಅಲ್ಲ ಎಂಬುದು. ಇದರಂತೆ ನಮ್ಮ ನಮ್ಮ ವರ್ಣಾಶ್ರಮಗಳಿಗನುಸಾರವಾಗಿ ಕರ್ತವ್ಯಗಳು ಬರುವುವು. ಅದನ್ನು ಮಾಡುವಾಗ ಕೇವಲ ಇವೆಲ್ಲ ನನ್ನ ಕೆಲಸ ಎಂದು ಮಾಡಿ ಬರುವುದನ್ನೆಲ್ಲ ಅವನಿಗೆ ಅರ್ಪಣ ಮಾಡುವ. ಜನರ ಹೊಗಳಿಕೆ ತೆಗಳಿಕೆಗೆ ಗಮನ ಕೊಡದೆ ಇರುವ.

ಕರ್ಮವನ್ನು ಬಿಡುವ ಸ್ಥಿತಿಗೂ ಬರಬೇಡ ಎನ್ನುತ್ತಾನೆ ಶ‍್ರೀಕೃಷ್ಣ. ಕೆಲವು ವೇಳೆ ಕೆಲಸ ಮಾಡುವಾಗ ಅದರ ಗೋಜಿಗೆ ಸಿಕ್ಕಿಕೊಂಡು ನರಳುವಾಗ, ಏತಕ್ಕಾದರೂ ಈ ಕೆಲಸಕ್ಕೆ ಕೈಹಾಕಿದೆವೊ, ಇನ್ನು ಮೇಲೆ ಇವುಗಳಿಗಾವುದಕ್ಕೂ ಕೈಹಾಕುವುದಿಲ್ಲವೆಂದು ಸುಮ್ಮನೆ ಇರುವುದಕ್ಕೆ ಪ್ರಯತ್ನಿಸಬಹುದು. ಆದರೆ ಹಾಗೆ ಸುಮ್ಮನೆ ಇರುವುದು ನಮಗೆ ಕೆಟ್ಟದ್ದು. ಕೆಲಸಕ್ಕೆ ಕೈಹಾಕಿದಾಗಲೇ ನಮ್ಮಲ್ಲಿರುವ ನ್ಯೂನತೆಗಳು ಮೇಲೆದ್ದು ಬರಬೇಕಾದರೆ. ಮೊಸರನ್ನ ಕಡೆದಾಗಲೆ ಅದರಲ್ಲಿರುವ ಬೆಣ್ಣೆ ಮೇಲೆದ್ದು ಬರುವಂತೆ. ಆಗಲೆ ಆ ನ್ಯೂನತೆಗಳಿಂದ ನಾವು ಪಾರಾಗುವುದಕ್ಕೆ ಯತ್ನಿಸಬೇಕು. ತೆಪ್ಪಗೆ ಇದ್ದರೆ ಇವು ಮೇಲೆ ಬರುವಂತೆಯೇ ಇಲ್ಲ. ಮೇಲೆ ಬರದೆ ಇದ್ದರೆ ಅವು ನಮ್ಮಲ್ಲಿ ಇಲ್ಲವೆಂದಲ್ಲ. ಅವು ಕಾಣದ ರೀತಿ ಹೊಕ್ಕಿಕೊಂಡಿವೆ. ಕರ್ಮದ ಗಲಾಟೆ ಆದಾಗ ಅವೆಲ್ಲ ಪೊದೆಗಳಿಂದ ಎದ್ದು ಬರುವುವು. ಕೆಲಸ ಮಾಡದೆ ಸುಮ್ಮನೆ ಇದ್ದರೆ ನಮಗೆ ನೈಷ್ಕರ್ಮ ಸಿದ್ಧಿಸುವುದಿಲ್ಲ. ನಾವೆಲ್ಲ ತಮಸ್ಸಿನಲ್ಲಿರುವೆವು. ಸತ್ತ್ವದ ಕಡೆ ಪ್ರಯಾಣ ಮಾಡಬೇಕಾಗಿದೆ. ಆಗ ಪ್ರಚಂಡ ರಜೋಗುಣದ ಮೂಲಕ ಮಾತ್ರ ಹೋಗಬೇಕು. ನಾವು ರಜೋಗುಣವನ್ನು ತಪ್ಪಿಸಿಕೊಂಡು ತಮಸ್ಸಿನಿಂದ ಸತ್ತ್ವಕ್ಕೆ ಡಬ್ಬಲ್ ಪ್ರಮೋಷನ್ ಇಚ್ಛಿಸುವೆವು. ಇದು ಜೀವನದ ವಿಕಾಸದಲ್ಲಿ ಸಾಧ್ಯವೇ ಇಲ್ಲ. ಕರ್ಮದಿಂದ ಅಲ್ಲ ನಾವು ಗೋಜು ಮಾಡಿಕೊಳ್ಳುವುದು. ಕರ್ಮದ ರಹಸ್ಯ ನಮಗೆ ಗೊತ್ತಿಲ್ಲ. ಅದರಿಂದಲೇ ಗೋಜಿಗೇಳುವುದು.

\begin{shloka}
ಯೋಗಸ್ಥಃ ಕುರು ಕರ್ಮಾಣಿ ಸಂಗಂ ತ್ಯಕ್ತ್ವಾ ಧನಂಜಯ~।\\ಸಿದ್ಧ್ಯಸಿದ್ಧ್ಯೋಃ ಸಮೋ ಭೂತ್ವಾ ಸಮತ್ವಂ ಯೋಗ ಉಚ್ಯತೇ \hfill॥ ೪೮~॥
\end{shloka}

\begin{artha}
ಧನಂಜಯ, ಯೋಗದ ಆಧಾರದ ಮೇಲೆ ನಿಂತುಕೊಂಡು, ಸಂಗವನ್ನು ಬಿಟ್ಟು ಸಿದ್ಧಿ ಅಸಿದ್ಧಿ ಎರಡನ್ನೂ ಸಮವಾಗಿ ಭಾವಿಸಿ ಕರ್ಮವನ್ನು ಮಾಡು. ಸಮತ್ವವನ್ನೇ ಯೋಗ ಎನ್ನುವುದು.
\end{artha}

ಶ‍್ರೀಕೃಷ್ಣ ಈ ಶ್ಲೋಕದಲ್ಲಿ ಯಾವ ದೃಷ್ಟಿಯಿಂದ ಕೆಲಸ ಮಾಡಬೇಕು ಎನ್ನುತ್ತಾನೆ. ಯೋಗ ಎಂದರೆ ಒಂದುಗೂಡಿಸುವುದು ಎಂದು ಅರ್ಥ. ಇಲ್ಲಿ ನಮ್ಮನ್ನು ದೇವರೊಂದಿಗೆ ಒಂದುಗೂಡಿಸುವುದು. ಈ ಪ್ರಪಂಚಕ್ಕೆಲ್ಲ ಸರ್ವೇಶ್ವರನಾದ ಭಗವಂತನೊಬ್ಬನೇ ಒಡೆಯ. ನಾವೆಲ್ಲ ಅವನ ಕೆಳಗೆ ಊಳಿಗಮಾಡುತ್ತಿರುವವರು. ಅವನು ಒಬ್ಬೊಬ್ಬನಿಗೆ ಒಂದೊಂದು ಕೆಲಸವನ್ನು ಕೊಟ್ಟಿರುತ್ತಾನೆ. ಆ ಕೆಲಸಗಳಲ್ಲಿ ಒಂದು ಮೇಲಲ್ಲ, ಮತ್ತೊಂದು ಕೀಳಲ್ಲ. ಇದೆಲ್ಲ ಅವನ ಕೆಲಸ. ಅವನಿಗಾಗಿ ಮಾಡಿಹಾಕೋಣ.

ಹಾಗೆ ಮಾಡುವಾಗ ನಾವು ಸಂಗ ಎಂದರೆ ಆಸಕ್ತಿಯನ್ನು ಬಿಡಬೇಕು. ಇಂತಹ ಕೆಲಸವನ್ನು ಮಾತ್ರ ನಾನು ಮಾಡಬೇಕು. ಉಳಿದವುಗಳನ್ನು ಮಾಡುವುದಿಲ್ಲ ಎಂಬ ಭಾವ ಇರಕೂಡದು. ಒಂದು ಕೆಲಸವನ್ನು ಮಾಡುತ್ತಿರುವಾಗಲೂ ಕೂಡ ಯಾವ ಸಮಯದಲ್ಲಿ ಬೇಕಾದರೆ ನಾವು ಆ ಕೆಲಸವನ್ನು ಬಿಡಲು ಸಿದ್ಧರಾಗಿರಬೇಕು. ದೇವರು ತನಗೆ ಇಚ್ಛೆಬಂದವರಿಂದ ಈ ಕೆಲಸವನ್ನು ಮಾಡಿಸಲಿ, ನನ್ನಿಂದ ಎಷ್ಟು ಬೇಕೋ ಅಷ್ಟನ್ನು ಮಾಡಿಸಿದ, ಅನಂತರ ಇನ್ನೊಬ್ಬನನ್ನು ತೆಗೆದುಕೊಂಡು ಮುಂದಿನ ಕೆಲಸವನ್ನು ಮಾಡುತ್ತಾನೆ. ನಾವೆಲ್ಲ ಅವನ ನಿಮಿತ್ತಗಳು. ಅವನು ತನಗೆ ಇಚ್ಛೆ ಬಂದವರನ್ನು ಉಪಯೋಗಿಸುತ್ತಾನೆ ಎಂದು ಎಲ್ಲವನ್ನೂ ಅವನ ವಶಕ್ಕೆ ಬಿಡಬೇಕು.

ಕೆಲಸ ಮಾಡುವಾಗ ಸಿದ್ಧಿ ಅಸಿದ್ಧಿ ಎರಡನ್ನೂ ಸಮನಾಗಿ ನೋಡುವುದನ್ನು ಕಲಿತುಕೊಳ್ಳಬೇಕು. ಸಿದ್ಧಿಸಿದರೆ ಅದು ದೇವರ ಕೆಲಸ, ಸಿದ್ಧಿಸದೇ ಇದ್ದರೆ ಅದು ಅವನ ಕೆಲಸವಲ್ಲ ಎಂದಲ್ಲ. ಅವನು ತನ್ನ ಕೆಲಸವನ್ನು ಸಿದ್ಧಿಯ ಮೂಲಕವಾಗಿಯೂ ಮಾಡುತ್ತಾನೆ. ಕೆಲವು ವೇಳೆ ಅಸಿದ್ಧಿಯ ಮೂಲಕವಾಗಿಯೂ ಆಗುತ್ತದೆ. ನಮ್ಮ ಅಲ್ಪ ದೃಷ್ಟಿಯಿಂದ ಒಂದು ಅಸಿದ್ಧಿ ಎಂದು ಭಾವಿಸುತ್ತೇವೆ. ನಮಗೆ ದೂರದೃಷ್ಟಿ ಇಲ್ಲ, ಭೂಮ ದರ್ಶನವಿಲ್ಲ. ಅದಕ್ಕೆ ಹಾಗೆ ಕಾಣಬಹುದು. ಪೂರ್ಣದೃಷ್ಟಿಯಿಂದ ನೋಡುವ ಭಗವಂತನಿಗೆ ಎಲ್ಲದಕ್ಕೂ ಒಂದು ಸ್ಥಾನವಿದೆ. ವಿಜ್ಞಾನ ಪ್ರಪಂಚದಲ್ಲಿ ಎಡಿಸನ್ ಎಂಬ ವಿಜ್ಞಾನಿ ಮೊದಲು ವಿದ್ಯುತ್ ದೀಪವನ್ನು ಕಂಡುಹಿಡಿದ. ಅದರೊಳಗೆ ವಿದ್ಯುತ್ ಪ್ರಕಾಶಿಸುವ ಸಣ್ಣ ತಂತಿಯನ್ನು ಯಾವ ವಸ್ತುವಿನಿಂದ ಮಾಡಬೇಕು ಎಂದು ಪರೀಕ್ಷಿಸಲು ಸಾವಿರಾರು ವಸ್ತುಗಳ ಮೇಲೆ ಪ್ರಯೋಗ ನಡೆಸುತ್ತಿದ್ದ. ಕೊನೆಗೆ ಯಾವುದು ಸರಿಯೋ ಅದನ್ನು ಕಂಡುಹಿಡಿದ. ಒಬ್ಬ ನೀವು ಇಷ್ಟೊಂದು ಪ್ರಯೋಗ ಮಾಡಿದ್ದು ವ್ಯರ್ಥವಾಯಿತಲ್ಲ ಎಂದಾಗ, ಎಡಿಸನ್ ಅದು ವ್ಯರ್ಥವಾಗಲಿಲ್ಲ, ಅದರ ಮೂಲಕ ವಿದ್ಯುತ್ ಪ್ರಕಾಶಿಸಲಾರದು ಎಂಬುದನ್ನು ನಾನು ಕಂಡುಹಿಡಿದೆ ಎಂದ.

ಶ‍್ರೀಕೃಷ್ಣ ಇಲ್ಲಿ ಯೋಗ ಎಂದರೆ ಸಮತ್ವ ಎಂಬ ಹೊಸ ಅರ್ಥವನ್ನು ಕೊಡುತ್ತಾನೆ. ಕೆಲವು ವೇಳೆ ಒಂದು ಒಡ್ಡರ ಬಂಡಿಯಲ್ಲಿ ಕುಳಿತುಕೊಂಡು ಹೋಗುವಾಗ ಅಷ್ಟೊಂದು ಕುಲುಕಾಡುವುದು. ಏಕೆಂದರೆ ಅದರಲ್ಲಿ ಕುಲುಕಾಟವನ್ನು ತಗ್ಗಿಸುವುದಕ್ಕೆ ಸ್ಪ್ರಿಂಗ್ ಇಲ್ಲ. ಅದೇ ಒಂದು ಸ್ಪ್ರಿಂಗ್​ನಿಂದ ಕೂಡಿದ ಗಾಡಿಯಲ್ಲಿ ಪ್ರಯಾಣ ಮಾಡಿದರೆ ಕುಲುಕಾಟವೇ ಕಾಣುವುದಿಲ್ಲ. ಏಕೆಂದರೆ ಆ ಕುಲುಕಾಟ ವನ್ನೆಲ್ಲ ತಗ್ಗಿಸುವುದಕ್ಕೆ ಸ್ಪ್ರಿಂಗ್ ಇದೆ. ಅದರಂತೆಯೇ ಕರ್ಮಯೋಗಿ ಸಿದ್ಧಿ ಅಸಿದ್ಧಿ ಎಂಬ ತಗ್ಗುಏರುಗಳಲ್ಲಿ ಪ್ರಯಾಣ ಮಾಡುತ್ತಿರುವಾಗ ಕುಲುಕಾಟದಿಂದ ತಪ್ಪಿಸಿಕೊಳ್ಳುವುದಕ್ಕೆ ಒಂದು ಉಪಾಯವನ್ನು ಕಂಡುಹಿಡಿದಿರುವನು. ಅವನೇನು ತಗ್ಗು ದಿಣ್ಣೆಗಳನ್ನೆಲ್ಲ ಕಡಿದು ಒಂದು ಚೆನ್ನಾದ ಸಿಮೆಂಟ್​ರೋಡ್ ಮಾಡಿದ್ದಾನೆಯೆ? ಅವನೇನು ಹೊರಗೆ ಬದಲಾವಣೆಯನ್ನು ಮಾಡಿಕೊಂಡಿಲ್ಲ. ತನ್ನ ಮನಸ್ಸಿನ ಒಳಗೆ ಬದಲಾವಣೆಯನ್ನು ಮಾಡಿಕೊಂಡಿರುವನು. ಇದೇ ಸಮತ್ವ ಎಂಬ ದೃಷ್ಟಿ.

\begin{shloka}
ದೂರೇಣ ಹ್ಯವರಂ ಕರ್ಮ ಬುದ್ಧಿಯೋಗಾದ್ಧನಂಜಯ~।\\ಬುದ್ಧೌ ಶರಣಮನ್ವಿಚ್ಛ ಕೃಪಣಾಃ ಫಲಹೇತವಃ \hfill॥ ೪೯~॥
\end{shloka}

\begin{artha}
ಅರ್ಜುನ, ಕಾಮ್ಯ ಕರ್ಮ, ಸಮತ್ವ ಬುದ್ಧಿಯಿಂದ ಮಾಡುವ ಕರ್ಮಕ್ಕಿಂತ ಕೀಳು. ಆದಕಾರಣ ನೀನು ಸಮತ್ವ ಬುದ್ಧಿಯಲ್ಲಿ ಆಶ್ರಯವನ್ನು ಹೊಂದು. ಫಲಕಾಂಕ್ಷಿಗಳಾದ ಜನ ಅಲ್ಪರು.
\end{artha}

ಕಾಮ್ಯಕರ್ಮ ಸಮತ್ವದ ದೃಷ್ಟಿಯಿಂದ ಮಾಡುವ ಕೆಲಸಕ್ಕಿಂತ ಬಹಳ ಕೆಳಮಟ್ಟದಲ್ಲಿರುವುದು. ಆಸೆಯಿಂದ ಪ್ರೇರಿತರಾಗಿ ಕೆಲಸ ಮಾಡುವೆವು. ಆಗ ಕೆಲಸಕ್ಕೆ ಬದ್ಧರು, ಕೆಲಸ ಮಾಡಿದ ಮೇಲೆ ಬರುವ ಫಲಕ್ಕೆ ಕೈ ಒಡ್ಡುವೆವು. ಆಗ ಆ ಫಲಕ್ಕೆ ಬದ್ಧರಾಗುವೆವು. ಆ ಫಲ ಸಿಹಿಯಾಗಿದ್ದರೆ ಕುಣಿದಾಡುವೆವು. ಕಹಿಯಾಗಿದ್ದರೆ ಒದ್ದಾಡುವೆವು. ಆ ಫಲವಂತೂ ನಮಗೆ ಹಿತಕರವಾಗಿದ್ದರೆ ಅದನ್ನು ಪದೇ ಪದೇ ಪಡೆಯುವುದು ಮತ್ತು ಅದು ನಮ್ಮನ್ನು ಬಿಟ್ಟು ಹೋಗದಂತೆ ನೋಡಿಕೊಳ್ಳುವುದಕ್ಕೆ ಯತ್ನಿಸುವೆವು. ಫಲದ ಗುಲಾಮರಾಗುತ್ತೇವೆ. ಅದು ನಮ್ಮನ್ನು ಆಡಿಸುವುದು. ನಾವು ಅದರ ಕೈಗೊಂಬೆಯಾಗುತ್ತೇವೆ. ದನವನ್ನು ಹೇಗೆ ಗೂಟಕ್ಕೆ ಕಟ್ಟಿಹಾಕುತ್ತೇವೆಯೊ ಹಾಗೆ ನಾವು ಫಲವೆಂಬ ಗೂಟಕ್ಕೆ ಕಟ್ಟಿಹಾಕಿಕೊಳ್ಳುತ್ತೇವೆ.

ಸಮತ್ವಬುದ್ಧಿಯಿಂದ ಕೆಲಸ ಮಾಡಿದರೆ, ಫಲದ ಮೇಲೆ ನಮಗೆ ಆಸಕ್ತಿಯೇ ಇಲ್ಲದೆ ಇದ್ದರೆ ಬದುಕುವುದು ಹೇಗೆ? ಫಲದ ಆಸಕ್ತಿಯಿಂದ ಕೆಲಸ ಮಾಡಿದರೆ ಫಲ ಬರುವುದೇನೊ ನಿಜ. ಆದರೆ ಫಲದ ಆಸಕ್ತಿಯನ್ನು ಬಿಟ್ಟು ಕೆಲಸ ಮಾಡಿದರೆ ಅದಕ್ಕಿಂತ ಹೆಚ್ಚು ನಮಗೆ ಸಿಕ್ಕುವುದು. ಆದರೆ ನಮಗೆ ಕಾಯಲು ತಾಳ್ಮೆಯಿಲ್ಲ. ನಾವು ಬೇಡವೆಂದರೂ ಫಲ ನಮ್ಮನ್ನು ಹುಡುಕಿಕೊಂಡು ಬರುವುದು. ನಮ್ಮ ಆಳಾಗಿ ಅದು ಇರುವುದು. ನಾನು ಅದರ ಆಳಾಗುವ ಬದಲು ಅದು ನನ್ನ ಆಳಾಗುವುದು. ಇದೇ ಜೀವನದ ರಹಸ್ಯ. ಫಲಾಕಾಂಕ್ಷಿಗಳಾದವರಿಗೆ ಆ ಫಲ ವಿನ ಬೇರೇನೂ ಸಿಕ್ಕುವುದಿಲ್ಲ. ಆ ಫಲವೂ ಮಿಶ್ರವಾಗಿರುವುದು. ಅದನ್ನು ನಾವು ಕೊಡವಿಬಿಡುವುದಕ್ಕೆ ಆಗುವುದಿಲ್ಲ. ಅದು ನಮಗೆ ಅಂಟಿಕೊಳ್ಳುವುದು. ಆದರೆ ಫಲದ ಮೇಲೆ ಆಸಕ್ತಿ ಬಿಟ್ಟು ಕೆಲಸ ಮಾಡಿದರೆ ಬೇಡವೆಂದರೂ ಕೀರ್ತಿ ಬರುವುದು, ಅಧಿಕಾರ ಬರುವುದು, ಪ್ರಭಾವ ಬರುವುದು. ಇದಾವುದಕ್ಕೂ ದಾಸನಲ್ಲ. ಇದೆಲ್ಲ ಬಾಗಿಲಿನ ಹೊರಗೆ ಆಳಿನಂತೆ ಇರುವುವು.

\begin{shloka}
ಬುದ್ಧಿಯುಕ್ತೋ ಜಹಾತೀಹ ಉಭೇ ಸುಕೃತದುಷ್ಕೃತೇ~।\\ತಸ್ಮಾದ್ಯೋಗಾಯ ಯುಜ್ಯಸ್ವ ಯೋಗಃ ಕರ್ಮಸು ಕೌಶಲಮ್ \hfill॥ ೫ಂ~॥
\end{shloka}

\begin{artha}
ಬುದ್ಧಿಯಿಂದ ಕೂಡಿದವನು ಪಾಪಪುಣ್ಯಗಳೆರಡನ್ನೂ ಇಲ್ಲಿಯೆ ಬಿಡುವನು. ಆದಕಾರಣ ಯೋಗಕ್ಕಾಗಿ ಪ್ರಯತ್ನ ಪಡು. ಕರ್ಮ ಕೌಶಲವೇ ಯೋಗ.
\end{artha}

ಸಮತ್ವಬುದ್ಧಿಯಿಂದ ಕೂಡಿದವನು ಪಾಪ ಪುಣ್ಯಗಳೆರಡನ್ನೂ ಇಲ್ಲಿಯೇ ಬಿಡುತ್ತಾನೆ. ಸ್ವಾರ್ಥ ದೃಷ್ಟಿಯಿಂದ ಕೆಲಸ ಮಾಡಿದರೆ ಅದು ನಮ್ಮನ್ನು ಈ ಪ್ರಪಂಚಕ್ಕೆ ಕಟ್ಟಿಹಾಕುವುದು. ನಮ್ಮ ಉದ್ದೇಶ ಸಾಧನೆಗೆ ಹಲವರಿಗೆ ಹಿಂಸೆ ಕೊಟ್ಟು, ಕಣ್ಣೀರು ಕರೆಸಿ, ಅವರ ದುಃಖ ಮತ್ತು ಕಷ್ಟದ ಜ್ವಾಲಾ ಮುಖಿಯ ಮೇಲೆ ನಮ್ಮ ಸುಖದ ಅರಮನೆಯನ್ನು ಕಟ್ಟಿದೆವು. ಆ ಜ್ವಾಲಾಮುಖಿ ಶಾಂತವಾಗಿರುವುದಿಲ್ಲ. ಅದು ಸಿಡಿಯುವುದು. ನಮ್ಮ ಸುಖದ ಅರಮನೆ ಪುಡಿಪುಡಿಯಾಗುವುದು. ಮತ್ತೊಬ್ಬರಿಂದ ಅನ್ಯಾಯವಾಗಿ ಪಡೆಯುವುದೆಲ್ಲ ಅಷ್ಟೆ. ಒಂದು ಕೆಲಸ ಮಾಡಿ ಪೋಲೀಸಿನವರಿಗೆ ಮತ್ತು ಲಾಯರಿಗೆ ದುಡ್ಡು ಸುರಿದು ಬರುವ ಪರಿಣಾಮದಿಂದ ಪಾರಾಗಬಹುದು ಈ ಪ್ರಪಂಚದಲ್ಲಿ. ಆದರೆ ಕರ್ಮ ನಿಯಮದ ಕಣ್ಣಿಗೆ ಯಾರೂ ಮಣ್ಣನ್ನು ಎರಚಲಾಗುವುದಿಲ್ಲ. ಮಾಡಿದ್ದುಣ್ಣೋ ಮಹರಾಯ ಎಂಬ ಗಾದೆ ವೈಜ್ಞಾನಿಕ ನಿಯಮಗಳಿಗಿಂತ ಹೆಚ್ಚು ನಿಜ. ಪಾಪಕಾರ್ಯವನ್ನು ಮಾಡಿ ಅದರ ಪರಿಣಾಮದಿಂದ ನಾವು ತಪ್ಪಿಸಿಕೊಳ್ಳುವುದಕ್ಕೆ ಆಗುವುದಿಲ್ಲ. ಅದು ನಮ್ಮನ್ನು ಅದರ ಪರಿಣಾಮಕ್ಕೆ ಬಂಧಿಸುವುದು.

ಅದರಂತೆಯೇ ಫಲಾಪೇಕ್ಷೆಯಿಂದ ಮಾಡಿದ ಪುಣ್ಯಕೆಲಸಗಳೂ ಕೂಡ ನಮ್ಮನ್ನು ಬಂಧಿಸುವುವು. ಈ ಪ್ರಪಂಚದಲ್ಲಿ ಈಗ ಚೆನ್ನಾಗಿರಬೇಕು. ಬೇರೆ ಲೋಕಕ್ಕೆ ಹೋದರೆ ಅಲ್ಲಿ ಸುಖವನ್ನು ಅನುಭವಿಸ ಬೇಕೆಂದು ಹಲವಾರು ಪುಣ್ಯ ಕೆಲಸಗಳನ್ನು ಮಾಡುತ್ತೇವೆ. ಆ ಫಲ ನಮಗೆ ಬರುವುದು. ಆದರೆ ಅದು ಶಾಶ್ವತವಾಗಿ ಇರುವುದಿಲ್ಲ. ಸ್ವರ್ಗದಲ್ಲಿ ಐಶ್ವರ್ಯ ಅಧಿಕಾರ ಮುಂತಾದುವು ಇರುವ ಕಡೆ ನಾವು ಹುಟ್ಟಬಹುದು. ಆದರೆ ಅಲ್ಲಿ ಶಾಶ್ವತವಾಗಿರುವುದಕ್ಕಾಗುವುದಿಲ್ಲ. ಸ್ವರ್ಗದಿಂದ ಪುಣ್ಯದ ಬುತ್ತಿ ತೀರಿದಮೇಲೆ ಪುನಃ ಧರೆಗೆ ಉರುಳಬೇಕು. ಅದರಂತೆಯೇ ಈ ಪ್ರಪಂಚದ ಎಂತಹ ಸುಖವಾದರೂ. ಒಂದು ಸುಖವಿದ್ದರೆ ಮತ್ತೊಂದು ಅಸುಖ ಕಾಡುವುದು. ನಮಗೆ ಬೇಕಾಗುವ ಸುಖದ ವಸ್ತುವಿ ನೊಂದಿಗೆ ಬೇಡದ ದುಃಖವೂ ನೆರಳಿನಂತೆ ಬರುವುದು. ಒಂದನ್ನು ತೆಗೆದುಕೊಂಡು ಮತ್ತೊಂದನ್ನು ಕಳುಹಿಸಿ ಬಿಡುವುದಕ್ಕೆ ಆಗುವುದಿಲ್ಲ. ಸುಖ ದುಃಖ ನಾಣ್ಯದ ಎರಡು ಕಡೆಯಂತೆ. ಒಂದನ್ನು ತೆಗೆದುಕೊಂಡರೆ ಮತ್ತೊಂದನ್ನು ತೆಗೆದುಕೊಳ್ಳಲೇಬೇಕಾಗಿದೆ. ಸಮತ್ವಬುದ್ಧಿಯಿಂದ ಕೂಡಿದವನು ಇವೆರಡರಿಂದಲೂ ಪಾರಾಗುತ್ತಾನೆ. ಅವನಿಗೆ ಸುಖವೂ ಬೇಡ, ಅದಕ್ಕೆ ಸಂಬಂಧಪಟ್ಟ ದುಃಖವೂ ಬೇಡ ಎಂದು ಎರಡನ್ನೂ ನಿರಾಕರಿಸುವನು.

ಸಮತ್ವಬುದ್ಧಿಯನ್ನು ರೂಢಿಸುವುದಕ್ಕೆ ಪ್ರಯತ್ನ ಪಡಬೇಕು. ಪ್ರಯತ್ನವಿಲ್ಲದೆ ಯಾವುದೂ ಸಿದ್ಧಿಸುವುದಿಲ್ಲ. ಮುಂಚೆ ಅದು ಅಸಾಧ್ಯವಾಗಿ ಕಾಣಬಹುದು. ಅಸಾಧ್ಯವೂ ಪ್ರಯತ್ನದಿಂದ ಸಾಧ್ಯ. ಪ್ರಯತ್ನದ ಬೆಲೆಯನ್ನು ಕೊಟ್ಟರೆ ಅದು ನಮಗೆ ಸಿದ್ಧಿಸುವುದು. ನಾವೇನೂ ಪ್ರಯತ್ನಮಾಡದೆ ಸುಮ್ಮನೇ ಯಾರ ಅನುಗ್ರಹದಿಂದಲೋ ದೇವರ ದಯೆಯಿಂದಲೋ ಅದು ನಮಗೆ ಸುಲಭವಾಗಿ ಸಿಕ್ಕುವುದಿಲ್ಲ. ಈ ಪ್ರಪಂಚದಲ್ಲಿ ಎಲ್ಲದಕ್ಕೂ ನಾವು ಪ್ರಯತ್ನಪಡುತ್ತೇವೆ. ಈಜುವುದಕ್ಕೆ, ಟೈಪ್ ಮಾಡುವುದಕ್ಕೆ, ಆಫೀಸಿನಲ್ಲಿ ಒಂದು ಕೆಲಸ ಕಲಿಯುವುದಕ್ಕೆ, ಎಲ್ಲೆಲ್ಲಿಯೂ ಪ್ರಯತ್ನಮಾಡುತ್ತೇವೆ. ಸರ್ಕಸ್ಸಿನಲ್ಲಿ ಕೆಲವು ಸಾಹಸ ಕೃತ್ಯಗಳನ್ನು ಮಾಡುವುದನ್ನು ನೋಡಿ ನಮಗೆ ರೋಮಾಂಚನ ಉಂಟಾಗುವುದು. ಅವರು ಅದನ್ನೆಲ್ಲ ಹೇಗೆ ಕಲಿತರು? ಅಭ್ಯಾಸದಿಂದ. ಸಮತ್ವಬುದ್ಧಿಯನ್ನು ಈ ಜನ್ಮದಲ್ಲಿ ದೇವರು ನಮ್ಮ ಹಣೆಯಲ್ಲಿ ಬರೆದಿಲ್ಲ; ನೋಡೋಣ ಮುಂದಿನ ಜನ್ಮಕ್ಕೆ ಎಂದರೆ, ಅದೇನೊ ಮುಂದಿನ ಜನ್ಮದಲ್ಲಿ ಸುಲಭವಾಗಿ ಸಿಕ್ಕುವುದಿಲ್ಲ. ಈ ಜನ್ಮದಲ್ಲಿ ಪ್ರಯತ್ನ ಮಾಡಿದ್ದರೆ ಮುಂದಿನ ಜನ್ಮದಲ್ಲಿ ನಾವು ಆ ಪ್ರಯತ್ನವನ್ನು ಮುಂದುವರಿಸಿಕೊಂಡು ಹೋಗಬಹುದೇ ಹೊರತು, ಇದ್ದಕ್ಕೆ ಇದ್ದಂತೆಯೇ ಮುಂದಿನ ಜನ್ಮದಲ್ಲಿ ನಮಗೆ ಅದು ದಕ್ಕುವುದಿಲ್ಲ. ಈ ಜೀವನದಲ್ಲಿ ಒಂದು ಒಳ್ಳೆಯ ಗುಣವನ್ನು ಕಲಿಯುವುದಕ್ಕೆ ಹೋರಾಡುವುದರಲ್ಲಿ ಆನಂದವಿದೆ. ಅದು ನಮಗೆ ಪೂರ್ಣ ಸಿದ್ಧಿಸದೆ ಇರಬಹುದು. ಸಮತ್ವ ದೃಷ್ಟಿಯನ್ನು ರೂಢಿಸುವುದು ನಮ್ಮ ಇಹಪರ ಎರಡಕ್ಕೂ ಒಳ್ಳೆಯದು. ಇಹಲೋಕದಲ್ಲಿ ಈ ಗುಣವಿದ್ದರೆ ಅತ್ಯಂತ ಶ್ರೇಷ್ಠವಾಗಿ ಕೆಲಸವನ್ನು ಮಾಡಬಹುದು. ಈ ಲೋಕವನ್ನು ತ್ಯಜಿಸಿದರೆ ನಾವು ಯಾವುದಕ್ಕೂ ಬದ್ಧರಲ್ಲ. ನಾವು ಹೋದರೆ ಮುಕ್ತರಾಗಿ ಹೊರಟೇ ಹೋಗಿ ಬಿಡಬಹುದು. ನಾವು ಹಿಂತಿರುಗಿ ಬರಬೇಕಾಗಿಲ್ಲ.

ಕರ್ಮ ಕೌಶಲ್ಯವೇ ಯೋಗ ಎಂಬ ಸೂತ್ರದಂತಹ ಮಾತನ್ನು ಆಡುವನು ಶ‍್ರೀಕೃಷ್ಣ. ಇದೊಂದು ಆಧ್ಯಾತ್ಮಿಕ ದೃಷ್ಟಿ. ಒಂದು ಕೆಲಸವನ್ನು ನೋಡುವುದಕ್ಕೆ ಎರಡು ದೃಷ್ಟಿಗಳಿವೆ. ಒಂದು ಬಾಹ್ಯದೃಷ್ಟಿ, ಮತ್ತೊಂದು ಆಂತರಿಕ ದೃಷ್ಟಿ. ಬಾಹ್ಯದೃಷ್ಟಿಯಿಂದ ನೋಡಿದರೆ ಕೆಲಸವನ್ನು ಅಚ್ಚುಗಟ್ಟಾಗಿ ಮಾಡಿರಬೇಕು; ಕೆಲಸವನ್ನು ಹೆಚ್ಚು ಮಾಡಿರಬೇಕು. ಮೊತ್ತ ಮತ್ತು ಯೋಗ್ಯತೆ ಎರಡು ದೃಷ್ಟಿಯಿಂದಲೂ ಯಾವುದು ಹೆಚ್ಚಾಗಿದೆಯೊ ಅದು ಶ್ರೇಷ್ಠವಾದ ಕೆಲಸ. ಕರ್ಮಕುಶಲಿಯಲ್ಲಿ ನಾವು ಇದನ್ನು ನೋಡುತ್ತೇವೆ. ಅವನು ಕರ್ಮವನ್ನು ಮಾಡುವಾಗ, ಮೇಲಿರುವ ಅಧಿಕಾರಿಗಳನ್ನು ಮೆಚ್ಚಿಸುವುದಕ್ಕಾಗಲಿ, ಹೆಸರಿಗಾಗಲಿ ಮಾಡುವುದಿಲ್ಲ. ಭಗವಂತನಿಗೆ ಅರ್ಪಿತವಾಗಲಿ ಎಂಬ ಪೂಜಾ ದೃಷ್ಟಿಯಿಂದ ಮಾಡುವನು. ಆ ಕೆಲಸದಲ್ಲಿ ಹುಡುಕಿದರೂ ತಪ್ಪು ಸಿಕ್ಕುವುದಿಲ್ಲ. ಹಾಗೆ ಮಾಡಿರುವನು. ಅವನು ಎಲ್ಲರಿಗಿಂತಲೂ ಹೆಚ್ಚು ಕೆಲಸವನ್ನು ಮಾಡುವನು. ಏಕೆಂದರೆ ಅವನು ಕಾಲವನ್ನು ವ್ಯರ್ಥಮಾಡುವುದಿಲ್ಲ. ಕೆಲಸವನ್ನು ಮಾಡುವಾಗ ಕಾಲವನ್ನು ವ್ಯರ್ಥಮಾಡುವುದು ಮಹಾ ಪಾಪ, ಅಕ್ಷಮ್ಯ ಎಂದು ಅವನು ಭಾವಿಸುವನು. ಇದು ದಣಿಗೆ ದೇವರಿಗೆ ಮೋಸ ಮಾಡುವುದೇ ಅಲ್ಲ, ಕೊನೆಗೆ ನಾವು ಆತ್ಮವಂಚನೆ ಮಾಡಿಕೊಳ್ಳುತ್ತೇವೆ. ಕರ್ಮಕುಶಲಿ ಇದನ್ನು ಚೆನ್ನಾಗಿ ಬಲ್ಲವನು. ಇತರರು ಅವನಿಗೆ ಈ ವಿಷಯದಲ್ಲಿ ಜಾಗ್ರತೆಯನ್ನು ಕೊಡಬೇಕಾಗಿಲ್ಲ.

ಕೆಲಸವನ್ನು ನೋಡುವ ಮತ್ತೊಂದು ದೃಷ್ಟಿಯೇ ಆಂತರಿಕ ದೃಷ್ಟಿ. ಈ ಕೆಲಸವನ್ನು ಮಾಡಿ ನನ್ನ ಮೇಲೆ ಏನು ಪರಿಣಾಮ ಆಯಿತು? ನನ್ನ ಮನಸ್ಸನ್ನು ಇದು ಸಂಕೋಚ ಮಾಡಿತೆ, ಕೊಳೆಯಿಂದ ತುಂಬಿತೆ ಅಥವಾ ಹೃದಯ ವಿಕಾಸವಾಗಿ ಚಿತ್ತಶುದ್ಧಿಯಾಗಿದೆಯೆ ಈ ಕೆಲಸದಿಂದ ಎಂದು ನೋಡುವನು. ಕರ್ಮಕುಶಲಿ ಚೆನ್ನಾಗಿ ಕೆಲಸ ಮಾಡುತ್ತಾನೆ. ಹೊರಗೆ ನೋಡಿದರೆ ಅತ್ಯಂತ ಶ್ರೇಷ್ಠವಾದ ಕೆಲಸವನ್ನು ಅವನು ಮಾಡಿರುವನು. ಆದರೆ ಅದನ್ನು ಯಾವ ದೃಷ್ಟಿಯಿಂದ ಮಾಡಿರುತ್ತಾನೆ? ಬಂಧನಕ್ಕೆ ಬೀಳಬಾರದು ಎಂಬ ದೃಷ್ಟಿಯಿಂದ ಮಾಡಿರುತ್ತಾನೆ. ಯಾವಾಗ ಅವನಿಗೆ ಫಲದ ಮೇಲೆ ಆಸಕ್ತಿ ಇಲ್ಲವೊ ಅವನು ಬಂಧನಕ್ಕೆ ಬೀಳುವುದಿಲ್ಲ. ಕರ್ಮ ಅವನ ಚಿತ್ತಶುದ್ಧಿಗೆ ಒಂದು ಸಾಧನೆ. ಚಿತ್ತಶುದ್ಧಿಯೇ ಭಗವತ್ ತತ್ತ್ವವನ್ನು ತಿಳಿದುಕೊಳ್ಳುವುದಕ್ಕೆ ಇರುವ ಉಪಾಯ.

\begin{shloka}
ಕರ್ಮಜಂ ಬುದ್ಧಿಯುಕ್ತಾ ಹಿ ಫಲಂ ತ್ಯಕ್ತ್ವಾ ಮನೀಷಿಣಃ~।\\ಜನ್ಮಬಂಧವಿನಿರ್ಮುಕ್ತಾಃ ಪದಂ ಗಚ್ಛಂತ್ಯನಾಮಯಮ್ \hfill॥ ೫೧~॥
\end{shloka}

\begin{artha}
ಏಕೆಂದರೆ ಬುದ್ಧಿವಂತರಾದ ಜ್ಞಾನಿಗಳು ಕರ್ಮಫಲವನ್ನು ತ್ಯಜಿಸಿ, ಜನ್ಮ ಬಂಧನದಿಂದ ಪಾರಾಗಿ, ಕ್ಲೇಶವಿಲ್ಲದೆ ಪರಮಪದವನ್ನು ಹೊಂದುತ್ತಾರೆ.
\end{artha}

ಬುದ್ಧಿವಂತರಾದ ಜ್ಞಾನಿಗಳು ಯಾವುದು ಸರಿ ಯಾವುದು ತಪ್ಪು ಎಂಬುದನ್ನು ತಿಳಿದುಕೊಂಡವರು ಮಾತ್ರವಲ್ಲ, ಸರಿಯನ್ನು ಮಾಡಿದವರು, ತಪ್ಪನ್ನು ಬಿಟ್ಟವರು. ಇಲ್ಲಿ ಬುದ್ಧಿವಂತ ಎಂದರೆ ಸಮತ್ವಬುದ್ಧಿಯಿಂದ ಇರುವ ಜಾಣ, ಸುಖ, ದುಃಖ, ಲಾಭ ನಷ್ಟ, ಜಯ ಅಪಜಯ ಇವುಗಳು ಯಾವುದು ಬಂದರೂ ಒಂದೇ ಸಮನಾಗಿ ವ್ಯಸ್ತನಾಗದೆ ನೋಡುವವನು. ಈ ಜಾಣ ಏನು ಮಾಡುತ್ತಾನೆ? ಕರ್ಮಫಲವನ್ನು ಇಲ್ಲಿಯೇ ಬಿಡುತ್ತಾನೆ. ಪ್ರತಿಯೊಂದು ಕರ್ಮಕ್ಕೂ ತಕ್ಕ ಫಲ ಇದ್ದೇ ಇರುವುದು. ಸಮತ್ವಬುದ್ಧಿಯನ್ನು ಪಡೆದವನು ಕೆಟ್ಟ ಕರ್ಮವನ್ನು ಮಾಡುವಂತೆಯೇ ಇಲ್ಲ. ಆದಕಾರಣ ಅವನಿಗೆ ಕೆಟ್ಟ ಪ್ರತಿಫಲ ಬರುವುದೇ ಇಲ್ಲ. ಅವನು ಒಳ್ಳೆಯ ಕೆಲಸವನ್ನು ಮಾಡುತ್ತಾನೆ. ಆದರೆ ಅದರಿಂದ ಬರುವ ಒಳ್ಳೆಯ ಫಲಕ್ಕೆ ಕೈ ಒಡ್ಡುವುದಿಲ್ಲ. ಒಡ್ಡಿದರೆ ಬದ್ಧರಾಗುತ್ತೇವೆ. ಯಾವಾಗ ನಾವು ಕೈ ಒಡ್ಡುವುದಿಲ್ಲವೊ ಅದು ಆಳಿನಂತೆ ನಮ್ಮನ್ನು ಅನುಸರಿಸುವುದು, ನಾವು ಅದನ್ನು ಅನುಸರಿಸುವುದಿಲ್ಲ. ಈ ಪ್ರಪಂಚದಲ್ಲಿ ಕರ್ಮವಲ್ಲ ನಮ್ಮನ್ನು ಬಂಧಿಸುವುದು. ಅದರಿಂದ ಬರುವ ಫಲದ ಮೇಲಿನ ಆಸಕ್ತಿ ಎಂಬುದನ್ನು ಚೆನ್ನಾಗಿ ಅವನು ಅರಿತಿರುವನು. ಫಲದ ಮೇಲಿನ ಆಸಕ್ತಿಯೇ ಹಾವಿನ ಬಾಯಲ್ಲಿರುವ ವಿಷದ ಹಲ್ಲು. ಸಮತ್ವಬುದ್ಧಿಯುಳ್ಳವನು ಅದನ್ನು ಕಿತ್ತಿರುವನು. ಅನಂತರ ಹಾವನ್ನು ಹೇಗೆ ಬೇಕಾದರೂ ಆಡಿಸಬಲ್ಲ. ಅದು ಅವನನ್ನು ಕಚ್ಚಲಾರದು.

ಅವನು ಮಾತ್ರ ಬಂಧನದಿಂದ ಪಾರಾಗುತ್ತಾನೆ. ಬದುಕಿರುವಾಗ ಯಾವ ಕೆಲಸ ಮಾಡಿದರೂ, ಆ ಕೆಲಸದ ಫಲಕ್ಕೆ ಅಂಟಿಕೊಂಡಿರುವುದಿಲ್ಲ. ಆದ ಕಾರಣ ಅವನಿಗೆ ಏನು ಬಂದರೂ ಹಿಗ್ಗುವುದಿಲ್ಲ, ಕುಗ್ಗುವುದೂ ಇಲ್ಲ. ಈ ಫಲಗಳು ಇನ್ನು ಮೇಲೆ ಅವನಿಗೆ ಚಕ್ಕುಳಗುಳಿಯನ್ನು ಇಡಲಾರವು. ಅವನು ಬದುಕಿರುವಾಗಲೇ ಮುಕ್ತ. ಈ ಪ್ರಪಂಚವನ್ನು ಬಿಟ್ಟಾದ ಮೇಲೆ ಅವನು ಮತ್ತೊಮ್ಮೆ ಈ ಪ್ರಪಂಚಕ್ಕೆ ಬರುವುದಿಲ್ಲ. ಆಸೆ ಇದ್ದರೆ ಇನ್ನೊಮ್ಮೆ ಬರುತ್ತಾನೆ. ಇವನಿಗೆ ಯಾವ ಆಸೆಯೂ ಇಲ್ಲ. ಅವನಿಗೆ ಸ್ವರ್ಗಸುಖದ ಮೇಲೆ ಆಸೆ ಇಲ್ಲ. ಅವನು ಈ ಪ್ರಪಂಚವನ್ನು ಬಿಡುವಾಗ ಹಿಂದಿರುಗಿ ನೋಡದೆ ಹೋಗುವನು. ಅವನು ಹುರಿದ ಬೀಜದಂತೆ ಆಗುವನು. ಇನ್ನು ಮೇಲೆ ಅದನ್ನು ಹುಟ್ಟಿಹಾಕಿದರೆ ಅದು ಮೊಳೆಯುವುದಿಲ್ಲ. ಆಸೆಯ ಹಸಿಯೆಲ್ಲ ಸೀದುಹೋಗಿರುವುದು.

ಇಂತಹವನು ಹೋದಮೇಲೆ ಕ್ಲೇಶವಿಲ್ಲದ ಪರಮಪದವನ್ನು ಪಡೆಯುತ್ತಾನೆ. ಇದೇನು ಸ್ವರ್ಗ ಸುಖವಲ್ಲ. ಪ್ರಪಂಚದಲ್ಲಿ ಫಲಾಪೇಕ್ಷೆಯಿಂದ ಮಾಡಿದ ಕೆಲಸಗಳಿಗೆ ಬಹುಮಾನ ಸ್ವರ್ಗಸುಖ. ಅಲ್ಲಿರುವ ಸುಖವಾದರೋ ತೀವ್ರವಾದ ಇಂದ್ರಿಯ ಸುಖ. ಮರ್ತ್ಯರಿಗೆ ಇಂದ್ರಿಯ ಸುಖವನ್ನು ಸಾಕಷ್ಟು ಅನುಭವಿಸುವುದಕ್ಕೆ ಆಗುವುದಿಲ್ಲ. ಆಸೆ ಇದೆ. ಆದರೆ ಯಾವುದರ ಮೂಲಕ ಅನುಭವಿಸ ಬೇಕೊ ಆ ಇಂದ್ರಿಯ ಹಳೆಯದಾಗಿದೆ, ಅಥವಾ ಬಿದ್ದುಹೋಗಿದೆ. ತಿನ್ನುವುದಕ್ಕೆ ಆಸೆ ಇದೆ. ಆದರೆ ಹಲ್ಲೆಲ್ಲ ಬಿದ್ದುಹೋಗಿರುವಂತೆ. ನೋಡುವುದಕ್ಕೆ ಆಸೆ ಇದೆ. ಆದರೆ ಕಣ್ಣು ಇಂಗಿಹೋಗಿದೆ. ಸ್ವರ್ಗದಲ್ಲಿ ಆದರೋ ಇಂದ್ರಿಯಗಳು ಪಟುವಾಗಿಯೇ ಇರುತ್ತವೆ. ಅಲ್ಲಿ ಜರೆ ಇಲ್ಲ, ರೋಗ ಇಲ್ಲ ಎಂದು ಹೇಳುತ್ತಾರೆ. ಆದರೆ ಎಂದೆಂದಿಗೂ ಆ ಸುಖವನ್ನು ಮೇಯುತ್ತಿರುವುದಕ್ಕೆ ಆಗುವುದಿಲ್ಲ. ಪುಣ್ಯದ ಬುತ್ತಿ ತೀರಿದರೆ ಅವನನ್ನು ಓಡಿಸುವರು. ಇಂತಹ ಸುಖವನ್ನು ಸಮತ್ವಬುದ್ಧಿಯುಳ್ಳವನು ಆಶಿಸುವುದಿಲ್ಲ. ಅವನು ಆಶಿಸುವುದು ಈ ಪ್ರಪಂಚವನ್ನೆಲ್ಲ ಆಳುತ್ತಿರುವ ಪರಮಾತ್ಮನ ಸಾನ್ನಿಧ್ಯವನ್ನು. ಅಲ್ಲಿ ಮಾತ್ರ ಕ್ಲೇಶವಿಲ್ಲ. ಅವನಲ್ಲಿ ಒಂದಾಗುವನು. ಅಥವಾ ಅವನ ಅಂಶವಾಗುವನು, ಅಥವಾ ಸಾನ್ನಿಧ್ಯದಲ್ಲಿರುವನು. ಪ್ರತಿಯೊಬ್ಬನೂ ತನಗೆ ಯಾವ ಸ್ಥಿತಿ ಬೇಕೊ ಅದನ್ನು ತನ್ನ ಸಂಸ್ಕಾರಕ್ಕೆ ತಕ್ಕಂತೆ ಆರಿಸಿಕೊಳ್ಳಬಹುದು.

\begin{shloka}
ಯದಾ ತೇ ಮೋಹಕಲಿಲಂ ಬುದ್ಧಿರ್ವ್ಯತಿತರಿಷ್ಯತಿ~।\\ತದಾ ಗಂತಾಸಿ ನಿರ್ವೇದಂ ಶ್ರೋತವ್ಯಸ್ಯ ಶ್ರುತಸ್ಯ ಚ \hfill॥ ೫೨~॥
\end{shloka}

\begin{artha}
ಯಾವಾಗ ನಿನ್ನ ಬುದ್ಧಿ ಮೋಹದ ಕೊಳೆಯಿಂದ ಪಾರಾಗುವುದೋ ಆಗ ಮುಂದೆ ಕೇಳಬೇಕಾಗಿರುವುದು ಮತ್ತು ಹಿಂದೆ ಕೇಳಿದ್ದರ ವಿಷಯದಲ್ಲಿ ಉದಾಸೀನನಾಗುವೆ.
\end{artha}

ಮೋಹ ಎನ್ನುವುದು ಬುದ್ಧಿಯನ್ನು ಮುತ್ತುವ ಒಂದು ಕೊಳೆ. ಆಗ ವಸ್ತುವಿನ ನೈಜಸ್ಥಿತಿ ನಮಗೆ ಗೊತ್ತಾಗುವುದಿಲ್ಲ. ಆ ಕೊಳೆಯ ಬಣ್ಣ ನೋಡುವ ವಸ್ತುಗಳಿಗೆ ಬರುವುದು. ನಾವೊಂದು ಬಣ್ಣದ ಗಾಜಿನ ಮೂಲಕ ನೋಡಿದಾಗ ಹೇಗೆ ನೋಡುವ ವಸ್ತುವಿನ ಮೇಲೆಲ್ಲ ಆ ಬಣ್ಣ ಬಳಿದಂತೆ ಕಾಣುವುದೋ ಹಾಗೆ. ಮೋಹ ವಸ್ತುವಿನ ಸತ್ಯವನ್ನು ಮುಚ್ಚುವುದು, ಮಿಥ್ಯವನ್ನು ಸತ್ಯದಂತೆ ಕಾಣುವಂತೆ ಮಾಡುವುದು. ಆ ಮಿಥ್ಯಾವಸ್ತುವನ್ನು ಸಂಪಾದಿಸಲು ನಾವು ಹೋಗುತ್ತೇವೆ. ಅದರಿಂದ ಪಡಬಾರದ ಯಾತನೆಯನ್ನೆಲ್ಲ ಪಡುತ್ತೇವೆ. ಜ್ಞಾನಿ ಮುಂಚೆ ನೋಡುವ ಕನ್ನಡಿಯ ಕೊಳೆಯನ್ನು ಶುದ್ಧ ಮಾಡುವನು. ಆಗ ವಸ್ತುವಿನ ಯಥಾರ್ಥ ಸ್ಥಿತಿ ವೇದ್ಯವಾಗುವುದು.

ಬುದ್ಧಿ ಯಾವಾಗ ಮೋಹದಿಂದ ಎಂದರೆ ಅಜ್ಞಾನ ಮತ್ತು ಆಸೆಯಿಂದ ಪಾರಾಗುವುದೊ ಆಗ ಮುಂದೆ ಕೇಳಬೇಕಾಗುವ ವಿಷಯದಲ್ಲಿ ಉದಾಸೀನನಾಗುವನು. ಅದರ ಮೇಲೆ ಅವನಿಗೆ ಯಾವ ಆಸಕ್ತಿಯೂ ಇರುವುದಿಲ್ಲ. ಯಾವುದು ಹೇಗೆ ಬೇಕಾದರೂ ಆಗಲಿ ಅದನ್ನು ಚಿಂತಿಸುವುದಿಲ್ಲ. ಏಕೆಂದರೆ ಅವನು ಇನ್ನು ಮೇಲೆ ತನ್ನ ಶಾಂತಿಗೆ ಆನಂದಕ್ಕೆ ಬಾಹ್ಯವಸ್ತುವನ್ನು ಅಪೇಕ್ಷಿಸುವವನಲ್ಲ. ಅವನ ಸುಖ, ಆನಂದ ಎಲ್ಲಾ ಅವನಲ್ಲಿಯೇ ಇದೆ.

ಅದರಂತೆಯೇ ಅವನು ಹಿಂದೆ ಕೇಳಿದುದರ ವಿಷಯದಲ್ಲಿಯೂ ಅನಾಸಕ್ತಿಯನ್ನು ತೋರುವನು. ನಮ್ಮಲ್ಲಿ ಇನ್ನೂ ಅಜ್ಞಾನವಿದ್ದಾಗ, ಆಸೆ ಇದ್ದಾಗ ಕೆಲವು ಘಟನೆಗಳು ಮತ್ತು ವಸ್ತುಗಳು ನಮ್ಮ ಬಾಳನ್ನೇ ಕಲಕಿ ಒಂದು ಪ್ರಚಂಡ ಪ್ರಭಾವವನ್ನೇ ಬೀರುತ್ತವೆ. ಆದರೆ ಅವನು ತನ್ನ ಹಿಂದಿನದನ್ನು ಅಜ್ಞಾನದಿಂದ ಮತ್ತು ವ್ಯಾಮೋಹ ದೃಷ್ಟಿಯಿಂದ ನೋಡುವುದಿಲ್ಲ. ಅವೆಲ್ಲ ಒಂದು ಮಕ್ಕಳಾಟದಂತೆ ಕಾಣುವುದು. ಯಾವುದು ಹಿಂದೆ ತುಂಬಾ ದೊಡ್ಡ ಘಟನೆಯಂತೆ ಕಾಣುತ್ತಿತ್ತೊ, ಅದು ಪೊಳ್ಳು, ಕೆಲಸಕ್ಕೆ ಬಾರದುದು ಎಂದು ಈಗ ಭಾವಿಸುವನು. ಮನುಷ್ಯನ ದೃಷ್ಟಿ ವ್ಯತ್ಯಾಸವಾದಂತೆ ಅವನು ಘಟನೆಗಳಿಗೆ ಕೊಡುವ ಬೆಲೆಯೂ ವ್ಯತ್ಯಾಸವಾಗುವುದು. ಮಗು ಆಡುವಾಗ ತನ್ನ ಬುಗುರಿಯನ್ನೋ ಗೋಲಿಯನ್ನೋ ಒಡೆದು ಹಾಕಿದವನ ಮೇಲೆ ಸೇಡು ತೀರಿಸಿಕೊಳ್ಳುವುದೇ ಜೀವನದ ಮುಖ್ಯಗುರಿ ಎಂದು ಭಾವಿಸುವುದು. ಸ್ವಲ್ಪ ದೊಡ್ಡವನಾದ ಮೇಲೆ, ಎಂತಹ ಅಲ್ಪವನ್ನು ಅದು ಅಷ್ಟು ಮಹತ್ತಾದ ಘಟನೆಯೆಂದು ಭಾವಿಸಿದ್ದೆನಲ್ಲ ಎಂದು ತಾನೇ ನಾಚುವುದು. ಅದರಂತೆಯೇ ಅಜ್ಞಾನದಲ್ಲಿರುವ ಯುವಕ, ತಾನು ಸುಖಪಡುವುದಕ್ಕೆ ಸುಂದರವಾದ ಸವಿಕನಸನ್ನು ಕಲ್ಪಿಸಿಕೊಳ್ಳುತ್ತಾನೆ. ಹೆಂಡತಿ, ಮನೆ, ಮಕ್ಕಳು, ಐಶ್ವರ್ಯ, ಅಧಿಕಾರ, ಕೀರ್ತಿ ಮುಂತಾದುವುಗಳನ್ನೆಲ್ಲ ಅವನು ಸರ್ವಸ್ವ ಎಂದು ಭಾವಿಸುತ್ತಾನೆ. ಅವನಿಗೆ ಸ್ವಲ್ಪ ಜ್ಞಾನೋದಯವಾದರೆ, ಬುದ್ಧಿ ಮೋಹವನ್ನು ದಾಟಿ ಹೋದರೆ, ಎಂತಹ ತಿರುಕನ ಕನಸು ಅದು ಎಂದು ತಾನೆ ತನ್ನ ಕಲ್ಪನೆಗೆ ನಾಚುತ್ತಾನೆ. ಜೀವನ ಅವನಿಗೆ ಹಲವು ಸಾರಿ ಓದಿದ ಪುಸ್ತಕದಂತೆ ಕಾಣುವುದು. ಅಲ್ಲಿ ಯಾವ ಕುತೂಹಲವೂ ಇಲ್ಲ, ಉದ್ವಿಗ್ನತೆಯೂ ಇಲ್ಲ. ಬಣ್ಣಬಣ್ಣದ ವಿಷಯವಸ್ತುಗಳನ್ನು ಇಟ್ಟುಕೊಂಡು ಆಟದಲ್ಲಿ ಮೈಮರೆತ ಮಕ್ಕಳ ಸ್ಥಿತಿಯಿಂದ ಪಾರಾಗಿರುವನು ಅವನು. ಇನ್ನು ಮೇಲೆ ಅವನು ಆಟದ ಸಾಮಾನಿನ ಬಣ್ಣ, ರೂಪ ಇವುಗಳ ಬಲೆಗೆ ಬೀಳುವವನಲ್ಲ. ಅವನ ಆಸಕ್ತಿಯನ್ನು ಇವುಗಳಾವುವೂ ಇನ್ನು ಮೇಲೆ ಕೆರಳಿಸಲಾರವು. ಆಸೆಯ ಹಸಿ ಇದ್ದರೆ ಒಂದು ಕೊಂಬೆ ನೆಲಕ್ಕೆ ತಾಕಿ ಅಲ್ಲಿ ತೇವ ಇದ್ದರೆ ಚಿಗುರುವುದು. ಅದೆಲ್ಲ ಒಣಗಿ ಹೋಗಿದ್ದರೆ ಅಲ್ಲಿ ಕುತೂಹಲದ ಚಿಗುರುಗಳು ಹೇಗೆ ಬಂದಾವು?

\begin{shloka}
ಶ್ರುತಿವಿಪ್ರತಿಪನ್ನಾ ತೇ ಯದಾ ಸ್ಥಾಸ್ಯತಿ ನಿಶ್ಚಲಾ~।\\ಸಮಾಧಾವಚಲಾ ಬುದ್ಧಿಸ್ತದಾ ಯೋಗಮವಾಪ್ಸ್ಯಸಿ \hfill॥ ೫೩~॥
\end{shloka}

\begin{artha}
ಶ್ರುತಿಗಳ ಭಿನ್ನ ಅಭಿಪ್ರಾಯಗಳಿಂದ ಕದಡಿಹೋದ ನಿನ್ನ ಬುದ್ಧಿ ಯಾವಾಗ ಸಮಾಧಿಯಲ್ಲಿ ನಿಶ್ಚಲವಾಗಿ ಅಲುಗಾಡದೆ ನಿಂತುಕೊಳ್ಳುವುದೊ ಆಗ ಯೋಗವನ್ನು ಪಡೆಯುವೆ.
\end{artha}

ಬುದ್ಧಿ ಒಂದು ಸರೋವರದಂತೆ. ಅದು ತಿಳಿಯಾಗಿದ್ದರೆ ಕೆಳಗೆ ಇರುವುದು ಸರಿಯಾಗಿ ಕಾಣುವುದು. ಅದೇನಾದರೂ ಕಲಕಿ ಹೋಗಿದ್ದರೆ ಕೆಳಗಿರುವುದು ಡೊಂಕು ಡೊಂಕಾಗಿ ಕಾಣುವುದು. ಕುಲುಕಾಟ ತುಂಬಾ ಜಾಸ್ತಿ ಇದ್ದರೆ ಕೆಳಗಿರುವುದು ಕಾಣುವುದೇ ಇಲ್ಲ, ಮೇಲಿರುವ ಕುಲುಕಾಟವೊಂದೇ ಕಾಣುವುದು. ಈ ಕುಲುಕಾಟಕ್ಕೆ ಕಾರಣವಾದರೂ ಏನು? ಬೀಸುವ ಗಾಳಿ, ಎಸೆಯುವ ಕಲ್ಲುಗಳು ಮುಂತಾದುವು. ಅದರಂತೆಯೇ ನಮ್ಮ ಬುದ್ಧಿಯ ಮೇಲೆ ಆಸೆಯ ಗಾಳಿ ಬೀಸುತ್ತಿದೆ. ಜೊತೆಗೆ ಹಲವು ಶಾಸ್ತ್ರಗಳು ಹಲವು ವ್ಯಕ್ತಿಗಳು ತಮ್ಮ ತಮ್ಮ ನೆಲೆಯನ್ನೇ ಪರಮಸತ್ಯವೆಂದು ಬೇರೆ ಬೋಧಿಸುತ್ತಿವೆ. ಪ್ರತಿಯೊಂದನ್ನೂ ಕೇಳಿದಾಗ ಅದೇ ಸರಿ ಎಂದು ಕಾಣುವುದು. ಆದರೆ ಹಲವನ್ನು ಕೇಳಿದ ಮೇಲೆ ಇವುಗಳಲ್ಲಿ ಯಾವುದನ್ನು ಹಿಡಿಯುವುದು ಎಂದು ನಮ್ಮ ಮನಸ್ಸು ಕ್ಷೋಭೆಗೆ ಒಳಗಾಗುವುದು.

ಎಲ್ಲಿಯವರೆಗೆ ನಾವು ವಾದಮಾಡುವ ಹಂತದಲ್ಲಿರುವೆವೊ ಅಲ್ಲಿಯವರೆಗೆ ಭಿನ್ನಾಭಿಪ್ರಾಯಗಳು ಇದ್ದೇ ಇರುವುವು. ಒಂದು ಸಂದೇಹ ಹೋಗುವುದು, ಮತ್ತೊಂದು ಸಂದೇಹ ಬರುವುದು. ಅನೇಕ ವೇಳೆ ಹೋದ ಸಂದೇಹಕ್ಕಿಂತ ಬಂದ ಸಂದೇಹವೇ ದೊಡ್ಡದಾಗಿ ಕಾಣುವುದು. ನಾವು ಕುಲುಕಾಟದಿಂದ ಪಾರಾಗಬೇಕಾದರೆ ಅನುಭವದ ಕ್ಷೇತ್ರಕ್ಕೆ ಇಳಿಯಬೇಕು. ಇದೇ ಸಮಾಧಿ. ಇಲ್ಲಿ ನಮ್ಮ ಬುದ್ಧಿಗೆ ಸಮಾಧಾನ ಸಿಕ್ಕುವುದು. ಇನ್ನು ಮೇಲೆ ಯಾರೂ ಇದನ್ನು ಕದಲಿಸುವಂತೆ ಇಲ್ಲ. ಒಬ್ಬ ಊರನ್ನು ನೋಡಿ ಬಂದಮೇಲೆ ಅನಂತರ ಯಾರೂ ಅವನೊಡನೆ ಆ ಊರಿಲ್ಲ ಎಂದು ವಾದಿಸುವ ಹಾಗಿಲ್ಲ. ಒಂದು ವೇಳೆ ವಾದಿಸಿದರೆ ಇವನು ಪ್ರತಿವಾದವನ್ನು ಹೂಡದೆ ತೆಪ್ಪಗೆ ಇರುವನು. ದೋಣಿಯನ್ನು ತೀರಕ್ಕೆ ತಂದು ಅದನ್ನು ನಾಲ್ಕು ಕಡೆಯೂ ಚೆನ್ನಾಗಿ ಕಟ್ಟಿದ್ದರೆ ಎಷ್ಟು ಗಾಳಿ ಬೀಸಿದರೂ ಏನೂ ಆಗುವ ಹಾಗಿಲ್ಲ. ಅದರಂತೆಯೇ ಸಾಕ್ಷಾತ್ಕಾರದ ನೆಲೆಯ ಮೇಲೆ ನಿಂತ ಬುದ್ಧಿ ಚಂಚಲವಾಗುವುದಿಲ್ಲ. ಈ ಸ್ಥಿತಿಯನ್ನೇ ಯೋಗ ಎನ್ನುವುದು. ಈ ಸ್ಥಿತಿಯಲ್ಲಿ ಪರಮ ಸತ್ಯದೊಂದಿಗೆ ಬುದ್ಧಿ ಒಂದಾಗಿರುವುದು. ಅರ್ಜುನ ಯಾವಾಗ ಈ ಸ್ಥಿತಿಯನ್ನು ಕೇಳುತ್ತಾನೆಯೋ ಅದನ್ನು ಮತ್ತಷ್ಟು ವಿವರವಾಗಿ ತಿಳಿಯಲೆತ್ನಿಸುವನು. ಅದಕ್ಕೆ ಶ‍್ರೀಕೃಷ್ಣನನ್ನು ಹೀಗೆ ಕೇಳುವನು:

\begin{shloka}
ಸ್ಥಿತಪ್ರಜ್ಞಸ್ಯ ಕಾ ಭಾಷಾ ಸಮಾಧಿಸ್ಥಸ್ಯ ಕೇಶವ~।\\ಸ್ಥಿತಧೀಃ ಕಿಂ ಪ್ರಭಾಷೇತ ಕಿಮಾಸೀತ ವ್ರಜೇತ ಕಿಮ್ \hfill॥ ೫೪~॥
\end{shloka}

\begin{artha}
ಕೇಶವ, ಸಮಾಧಿಯಲ್ಲಿರುವ ಸ್ಥಿತಪ್ರಜ್ಞನ ಲಕ್ಷಣಗಳೇನು? ಅವನು ಹೇಗೆ ಮಾತಾಡುತ್ತಾನೆ? ಹೇಗೆ ಕುಳಿತುಕೊಳ್ಳುತ್ತಾನೆ? ಹೇಗೆ ನಡೆದುಕೊಳ್ಳುತ್ತಾನೆ?
\end{artha}

ಬದ್ಧನೂ ಸ್ಥಿತಪ್ರಜ್ಞನೂ ಒಂದೇ ಪ್ರಪಂಚದಲ್ಲಿ ಇರುತ್ತಾರೆ. ಈ ಪ್ರಪಂಚದ ಸುದ್ದಿ ಸಮಾಚಾರಗಳು ಇಬ್ಬರಿಗೂ ತಾಕುತ್ತವೆ. ಅವರು ಹೇಗೆ ಅದನ್ನು ಪ್ರತಿಭಟಿಸುತ್ತಾರೆ ಎಂಬುದನ್ನು ತಿಳಿದು ಕೊಳ್ಳಲು ಅರ್ಜುನ ಯತ್ನಿಸುವನು. ಇಬ್ಬರೂ ಒಂದೇ ಘಟನೆಗಳನ್ನು ನೋಡುತ್ತಾರೆ. ಆದರೆ ಅವರ ಪ್ರತಿಕ್ರಿಯೆ ಬೇರೆಬೇರೆ ಆಗಿರುವುದು. ಪ್ರಪಂಚವನ್ನು ನೋಡುವ ದೃಷ್ಟಿ ಬದಲಾವಣೆ ಆಗುವುದು. ಶ‍್ರೀ ಶಾರದಾದೇವಿ, ಒಬ್ಬನಿಗೆ ಆತ್ಮಸಾಕ್ಷಾತ್ಕಾರವಾಗಿದೆ ಎಂದರೆ ಅವನಿಗೇನು ಇತರರಿಗಿಲ್ಲದ ಒಂದು ಜೊತೆ ಕೊಂಬು ಬೆಳೆಯುವುದಿಲ್ಲ, ಅವನು ಈ ಪ್ರಪಂಚವನ್ನು ನೋಡುವ ದೃಷ್ಟಿ ಬದಲಾವಣೆ ಆಗಿದೆ ಅಷ್ಟೆ ಎನ್ನುತ್ತಾರೆ. ಅಂತಹ ದೃಷ್ಟಿ ಬದಲಾವಣೆ ಆಗಿರುವ ಮನುಷ್ಯನ ಮಾತು ಕತೆ ಅವನ ನಡವಳಿಕೆ ಇವು ಹೇಗಿರುತ್ತವೆ ಎಂಬುದನ್ನು ತಿಳಿದುಕೊಳ್ಳಲು ಇಚ್ಛಿಸುವನು. ಆಗ ಶ‍್ರೀಕೃಷ್ಣ ಅಂತಹ ವ್ಯಕ್ತಿ ಹೇಗಿರುತ್ತಾನೆ ಎಂಬುದನ್ನು ವಿವರಿಸುವನು. ಇಂತಹ ಸ್ಥಿತಪ್ರಜ್ಞ ಭೋರ್ಗರೆಯುತ್ತಿರುವ ಸಂಸಾರ ಸಾಗರದಿಂದ ಮೇಲೆದ್ದು ನಿಂತಿರುವ ಒಂದು ದ್ವೀಪದಂತೆ ಇರುವನು. ಅತ್ತ ಅಲ್ಲೋಲ ಕಲ್ಲೋಲ ಸಮುದ್ರ. ಇತ್ತ ದ್ವೀಪದಲ್ಲಿ ಪರಮ ಶಾಂತಿ. ಇದೊಂದು ಅತ್ಯಂತ ಶ್ರೇಷ್ಠವಾದ ಭಾವನೆಯ ಗೊಂಚಲು.

\begin{shloka}
ಪ್ರಜಹಾತಿ ಯದಾ ಕಾಮಾನ್ ಸರ್ವಾನ್ ಪಾರ್ಥ ಮನೋಗತಾನ್~।\\ಆತ್ಮನ್ಯೇವಾತ್ಮನಾ ತುಷ್ಟಃ ಸ್ಥಿತಪ್ರಜ್ಞಸ್ತದೋಚ್ಯತೇ \hfill॥ ೫೫~॥
\end{shloka}

\begin{artha}
ಪಾರ್ಥ, ಮನಸ್ಸಿನಲ್ಲಿರುವ ಕಾಮನೆಗಳನ್ನೆಲ್ಲ ಬಿಟ್ಟು, ಯಾವಾಗ ಆತ್ಮನಲ್ಲಿ ಆತ್ಮನಿಂದಲೇ ಸಂತುಷ್ಟನಾಗಿರುವನೊ ಆಗ ಸ್ಥಿತಪ್ರಜ್ಞ ಎನಿಸುತ್ತಾನೆ.
\end{artha}

ಸ್ಥಿತಪ್ರಜ್ಞ ಮನಸ್ಸಿನಲ್ಲಿರುವ ಕಾಮನೆಗಳನ್ನೆಲ್ಲ ಬಿಡುತ್ತಾನೆ. ಏಕೆಂದರೆ ಅವನಿಗೆ ಗೊತ್ತಿದೆ, ಈ ಕಾಮನೆಗಳೆ ಅನಂತರ ಇವನ ಕೈಯಲ್ಲಿ ಅದನ್ನು ತೃಪ್ತಿ ಪಡಿಸಿಕೊಳ್ಳುವುದಕ್ಕೆ ಕೆಲಸವನ್ನು ಮಾಡುವಂತೆ ಪ್ರೇರೇಪಿಸುವುದು. ಒಮ್ಮೆ ತೃಪ್ತಿಗೆ ಕೈ ಹಾಕಿತು ಎಂದರೆ ಅದಕ್ಕೆ ಒಂದು ಅಂತ್ಯವಿಲ್ಲ. ಒಮ್ಮೆ ಅದರಲ್ಲಿ ಬಿದ್ದಮೇಲೆ ಏಳುವುದಕ್ಕೆ ಆಗುವುದಿಲ್ಲ. ಪುನಃ ಪುನಃ ಅದರಲ್ಲಿಯೇ ಜಾರುತ್ತಿರಬೇಕಾಗುವುದು. ಇಲ್ಲಿ ಎಲ್ಲಾ ಕಾಮನೆಗಳನ್ನು ಎಂದು ಹೇಳುವನು. ಐದು ಇಂದ್ರಿಯಗಳಲ್ಲಿ ಕೆಲವನ್ನು ನಿಗ್ರಹಿಸಿ ಇನ್ನೊಂದರಿಂದ ಅಷ್ಟೊಂದು ಅಪಾಯ ಇಲ್ಲ ಎಂದು ಭಾವಿಸಿದರೆ ನಿಗ್ರಹಿಸದೆ ಇರುವುದರಿಂದ ಮತ್ತಷ್ಟು ಹೆಚ್ಚು ಮಾಡುವೆವು. ಒಂದು ಬಾಗಿಲನ್ನು ಹಾಕಿದರೆ ಮತ್ತೊಂದು ಬಾಗಿಲಿನ ಮೂಲಕ ಹೊಗೆ ಹೆಚ್ಚಾಗಿ ಹೋಗುವಂತೆ ಇದು. ಅನಂತರ ಯಾವುದನ್ನು ನಿಗ್ರಹಿಸಿರುವೆವೊ ಅದೂ ಕ್ರಮೇಣ ಸಡಿಲವಾಗುತ್ತ ಬರುವುದು. ಆ ಕಟ್ಟೆಯೂ ಕಿತ್ತುಹೋಗುವುದು. ಆದಕಾರಣ ಎಲ್ಲಾ ವಿಧವಾದ ಕಾಮನೆಗಳನ್ನು ನಿಗ್ರಹಿಸಿ ಯೋಗಿ ಮನಸ್ಸನ್ನು ಅಂತರ್ಮುಖ ಮಾಡಿರುವನು.

ಆತ್ಮನಲ್ಲಿ ಆತ್ಮನಿಂದಲೇ ಸಂತುಷ್ಟನಾದಾಗ ಅವನು ಸ್ಥಿತಪ್ರಜ್ಞ. ಅವನ ಆತ್ಮತೃಪ್ತಿಗೆ ಮತ್ತು ಆನಂದಕ್ಕೆ ಹೊರಗಡೆಯಿಂದ ಇನ್ನು ಮೇಲೆ ಅವನಿಗೆ ಏನೂ ಬೇಡ. ಪರಮಾತ್ಮ ತನ್ನ ಆತ್ಮನ ಹಿಂದೆಯೇ ಇರುವನು. ಆ ಪರಮಾತ್ಮನಿಂದ ತನ್ನ ಆತ್ಮನಿಗೆ ಸುಖಶಾಂತಿಗಳನ್ನು ಹೀರುತ್ತಿರುವನು. ಬಾಹ್ಯತೃಪ್ತಿಯನ್ನು ಇನ್ನು ಮೇಲೆ ಅವನು ಹುಡುಕಿಕೊಂಡು ಹೋಗುವುದಿಲ್ಲ. ಹಾಗೆ ಹುಡುಕಿಕೊಂಡು ಹೋದರೆ ದುರಂತದಲ್ಲಿ ಪರ್ಯವಸಾನವಾಗುವುದು ಎಂಬುದನ್ನು ಅವನು ಚೆನ್ನಾಗಿ ಬಲ್ಲನು. ಹೊರಗೆ ಹೋಗಿ ಅನಂತರ ವ್ಯಥೆಪಡುವವನಲ್ಲ ಅವನು. ಹೋಗುವುದಕ್ಕೆ ಮುಂಚೆಯೇ ಅಲ್ಲಿಗೆ ಹೋದರೆ ಅದರಿಂದ ಏನಾಗುವುದು ಎಂಬುದನ್ನು ತಕ್ಷಣ ಅವನು ನಿಷ್ಕರ್ಷಿಸಬಲ್ಲ. ಆದಕಾರಣವೇ ಅವನು ತನ್ನ ಜೀವನದ ಸೌಧವನ್ನು ತನ್ನ ಅಂತರಾಳದಲ್ಲಿರುವ ಪರಮಾತ್ಮನ ಆಧಾರದ ಮೇಲೆ ಕಟ್ಟಿಕೊಳ್ಳುವನು. ಈ ಜೀವನದಲ್ಲಿ ಹೊರಗೆ ಇರುವುದೆಲ್ಲ ಕುಸಿದು ಬೀಳುವುದು. ಆದರೆ ತನ್ನ ಅಂತರತಮವಾದ ಪರಮಾತ್ಮನ ತಳಪಾಯ ಮಾತ್ರ ಕುಸಿಯುವುದಿಲ್ಲ. ದೃಶ್ಯ ಪ್ರಪಂಚವೆಲ್ಲ ನಿರ್ನಾಮವಾಗಬಹುದು. ಆದರೆ ನಾನು ಎನ್ನುವ ಪ್ರತ್ಯಯಕ್ಕೆ ಆಧಾರವಾದ ನೀನು ಎಂಬ ಪರಮಾತ್ಮನಿಗೆ ನಾಶವಿಲ್ಲ.

\begin{shloka}
ದುಃಖೇಷ್ವನುದ್ವಿಗ್ನಮನಾಃ ಸುಖೇಷು ವಿಗತಸ್ಪೃಹಃ~।\\ವೀತರಾಗಭಯಕ್ರೋಧಃ ಸ್ಥಿತಧೀರ್ಮುನಿರುಚ್ಯತೇ \hfill॥ ೫೬~॥
\end{shloka}

\begin{artha}
ದುಃಖದಲ್ಲಿ ಉದ್ವೇಗಗೊಳ್ಳದವನು, ಸುಖದಲ್ಲಿ ಆಸೆ ಇಲ್ಲದವನು, ರಾಗ ಭಯ ಕ್ರೋಧಗಳನ್ನು ಗೆದ್ದವನು ಆದ ಮುನಿ ಸ್ಥಿತಪ್ರಜ್ಞ.
\end{artha}

ಸ್ಥಿತಪ್ರಜ್ಞನ ಮತ್ತೊಂದು ಲಕ್ಷಣವನ್ನು ಇಲ್ಲಿ ಹೇಳುತ್ತಾನೆ. ದುಃಖ ಬಂದರೆ ಅವನ ಮನಸ್ಸು ವ್ಯಸ್ತವಾಗುವುದಿಲ್ಲ. ಮಳೆ ಬಂಡೆಯ ಮೇಲೆ ಬಿದ್ದರೆ ಒಳಗೆ ಹೋಗದೆ ಹೇಗೆ ಮೇಲಿನಿಂದ ಹರಿದುಕೊಂಡು ಹೋಗುವುದೊ ಹಾಗೆ ಇರುವನು ಸ್ಥಿತಪ್ರಜ್ಞ. ಏಕೆಂದರೆ ಅವನು ಈ ಪ್ರಪಂಚದಿಂದ ಬರುವ ದುಃಖಕ್ಕೆ ಮೊದಲೆ ಅಣಿಯಾಗಿರುವನು. ನನಗೆಲ್ಲ ಸುಖವೇ ಸಿಕ್ಕುವುದು ಎಂಬ ಭ್ರಮೆಯಿಂದ ಕೂಡಿದ್ದರೆ ದುಃಖದ ಸಿಡಿಲು ಬಡಿದಾಗ ದಿಗ್ಭ್ರಾಂತನಾಗುವನು. ಯಾವನು ಅದಕ್ಕೆ ಸಿದ್ಧನಾಗಿರು ವನೊ ಅವನನ್ನೇನೂ ಅದು ಮಾಡಲಾರದು. ಕೆಲವು ವೇಳೆ ದೊಡ್ಡ ದೊಡ್ಡ ಕಟ್ಟಡಗಳಿಗೆ ಸಿಡಿಲು ಹೊಡೆಯದೆ ಇರಲಿ ಎಂದು ಮೇಲಿನಿಂದ ಕೆಳಗಿನವರೆಗೆ ವಿದ್ಯುತ್ ಶಕ್ತಿಯನ್ನು ಒಯ್ಯುವಂತಹ ಲೋಹದ ಪಟ್ಟಿಯನ್ನು ಜೋಡಿಸಿರುವರು. ಸಿಡಿಲು ಯಾವಾಗ ಹೊಡೆಯಬೇಕಾದರೂ ಕಟ್ಟಡದ ಅತ್ಯಂತ ಎತ್ತರದ ಸ್ಥಳಕ್ಕೆ. ಆ ಸ್ಥಳದಲ್ಲಿರುವ ಲೋಹ ಈ ಶಕ್ತಿಯನ್ನು ಕ್ಷಣಾರ್ಧದಲ್ಲಿ ಭೂಮಿಗೆ ಒಯ್ಯುವುದು, ಕಟ್ಟಡವನ್ನು ಸುರಕ್ಷಿತವಾಗಿಟ್ಟಿರುವುದು. ಸ್ಥಿತಪ್ರಜ್ಞನೂ ಕೂಡ ಪ್ರಪಂಚದಲ್ಲಿ ಬೆಳ ಗೆದ್ದರೆ ಸಿಡಿಲು ಬಡಿಯುವುದು ಎಂದು ತಿಳಿದುಕೊಂಡು ಮೊದಲೆ ಇದರಿಂದ ಪಾರಾಗುವುದಕ್ಕೆ ಅಣಿಯಾಗಿರುವನು. ಅವನಿಗೇನು ದುಃಖ ತಗಲುವುದಿಲ್ಲ ಎಂದಲ್ಲ. ಎಲ್ಲರಿಗೂ ಬರುವ ದುಃಖ ಅವನಿಗೂ ಬರುವುದು. ಕೆಲವು ವೇಳೆ ಇತರರಿಗಿಂತ ಅವನಿಗೆ ಹೆಚ್ಚು ಬರುವುದು. ಆದರೆ ಅದು ಇವನನ್ನು ಬಾಧಿಸದಂತೆ ರಕ್ಷಣೆ ಮಾಡಿಕೊಂಡಿರುವನು. ಬಿರುಗಾಳಿ ಎದ್ದಾಗ ನೆಲದ ಮೇಲಿರುವ ಕಸಕಡ್ಡಿ ಧೂಳು ಇವುಗಳನ್ನೆಲ್ಲ ಹೊಡೆದುಕೊಂಡು ಹೋಗುವುದು. ಅದೊಂದು ಕಲ್ಲುಬಂಡೆಯನ್ನು ಅಲುಗಿಸಲಾರದು. ಸ್ಥಿತಪ್ರಜ್ಞ ದುಃಖದ ಬಿರುಗಾಳಿ ಬೀಸುತ್ತಿರುವ ಈ ಸಂಸಾರದಲ್ಲಿ ಒಂದು ಭೀಮಬಂಡೆಯಂತೆ ಇರುವನು. ಯಾರು ದೇವರ ಕೈಯನ್ನು ಹಿಡಿದುಕೊಂಡಿರುವನೊ ಅವನನ್ನು ಯಾವ ದುಃಖವೂ ಓಡಿಸುವಂತೆ ಇಲ್ಲ. ದುಃಖ ಬಂದಾಗ ಅವನು ಮತ್ತೂ ಭದ್ರವಾಗಿ ದೇವರನ್ನು ಹಿಡಿದುಕೊಳ್ಳುವನು. ದೇವರಲ್ಲದ ಇತರ ವಸ್ತುಗಳನ್ನು ಹಿಡಿದುಕೊಂಡವನು, ಕೊಚ್ಚಿಕೊಂಡು ಹೋಗುತ್ತಾನೆ. ಆದರೆ ದೇವರನ್ನು ಹಿಡಿದವನನ್ನು ಯಾವುದೂ ಕೊಚ್ಚಿಕೊಂಡು ಹೋಗುವಂತೆ ಮಾಡಲಾರದು.

ಸ್ಥಿತಪ್ರಜ್ಞ ದುಃಖಕ್ಕೆ ಹೇಗೆ ಅಂಜುವುದಿಲ್ಲವೊ ಹಾಗೆಯೇ ಸುಖ ಬಂದಾಗ ಅದರ ಮೇಲೆ ಅವನಿಗೆ ಆಸೆಯೂ ಇರುವುದಿಲ್ಲ. ಇದು ತಾತ್ಕಾಲಿಕ ಎಂಬುದನ್ನು ಅವನು ತುಂಬ ಚೆನ್ನಾಗಿ ತಿಳಿದುಕೊಂಡಿರುವನು. ಯಾವಾಗ ಮುಂದಿರುವ ಸುಖವನ್ನು ಚೀಪುತ್ತಾನೆಯೊ ಹಿಂದೆ ಇರುವ ದುಃಖವನ್ನು ಅವನು ಸ್ವೀಕರಿಸಬೇಕಾಗುವುದು ಎಂಬುದನ್ನು ಅವನು ಚೆನ್ನಾಗಿ ಅರಿತಿರುವನು. ಅಜ್ಞಾನಿಗೆ ಹೊರಗೆ ಇರುವ ಒಂದು ಸುಂದರ ವ್ಯಕ್ತಿ, ಅಧಿಕಾರ, ಐಶ್ವರ್ಯ, ಕೀರ್ತಿ, ಗೌರವಗಳು ಬಾಯಲ್ಲಿ ನೀರೂರುವಂತೆ ಮಾಡುವುವು. ಆದರೆ ಜ್ಞಾನಿಯಾದರೊ ಹೊರಗಿನ ಬಣ್ಣದ ಬಲೆಗೆ ಬೀಳುವವನಲ್ಲ. ಅವನು ಅದನ್ನು ಚೆನ್ನಾಗಿ ವಿಮರ್ಶಿಸುತ್ತಾನೆ. ಆ ಬಣ್ಣದ ಹಿಂದೆ ಇರುವುದೇನು, ಈಗ ನಮಗೆ ಕಾಣುವಂತೆಯೇ ಅನಂತರವೂ ಇರುವುದೇ ಎಂಬ ಪ್ರಶ್ನೆಯನ್ನು ಹಾಕುತ್ತಾನೆ. ಈ ಪ್ರಶ್ನೆಯನ್ನು ಎದುರಿಸಿ ನಿಲ್ಲಲಾರವು ಈ ಬಣ್ಣದ ಆಟದ ಸಾಮಾನುಗಳು. ಸುಖದಲ್ಲಿ ಆಸೆ ಇಲ್ಲ ಎಂದರೆ ಅವನು ಸುಖವನ್ನೆಲ್ಲ ನಿರಾಕರಿಸುವನೆ? ಇಲ್ಲಿ ಅವನು ಇಂದ್ರಿಯ ಸುಖಕ್ಕೆ ಕೈ ಒಡ್ಡುವವನಲ್ಲ. ಅದು ತಾನಾಗಿ ಬಂದರೂ ಅದರ ಬಲೆಗೆ ಬೀಳುವವನಲ್ಲ. ಆದರೂ ಅವನು ನಿತ್ಯಸುಖಿ. ಈ ಪ್ರಪಂಚದಲ್ಲಿ ಅವನಂತಹ ಸುಖ ಪುರುಷ ಯಾರೂ ಇಲ್ಲ. ಅವನ ಸುಖಕ್ಕೆ ಆಧಾರ ಪ್ರೇಮಮಯನಾಗಿರುವ ಭಗವಂತ.

ಅವನಲ್ಲಿ ಆಸಕ್ತಿ ಇಲ್ಲ. ಆಸಕ್ತಿ ನಮ್ಮನ್ನು ವಸ್ತುವಿಗೆ ಕಟ್ಟಿಹಾಕುವುದು. ಯಾವಾಗಲೂ ನಾವು ಅದನ್ನೇ ಕುರಿತು ಚಿಂತಿಸುತ್ತಿರುವೆವು. ಎಲ್ಲಿ ಹೋದರೂ ನಮಗೆ ಅದೇ ಚಿಂತೆ. ಗೂಟಕ್ಕೆ ಕಟ್ಟಿಕೊಂಡ ದನದಂತೆ ಆಗುವೆವು ನಾವು. ಸ್ಥಿತಪ್ರಜ್ಞ ಹಾಗೆ ಯಾವ ಒಂದು ವಸ್ತುವಿಗೂ ವ್ಯಕ್ತಿಗೂ ಬದ್ಧನಲ್ಲ. ಹಾಗಾದರೆ ಅವನು ಯಾರನ್ನೂ ಯಾವುದನ್ನೂ ನೋಡಿ ಸಂತೋಷಿಸುವುದೇ ಇಲ್ಲವೆ? ಅವನಷ್ಟು ನೋಡಿ ಆನಂದಪಡುವವರು ಇನ್ನು ಯಾರೂ ಇಲ್ಲ. ಅವನು ಆನಂದ ಪಡುವಾಗ, ಇದು ನನ್ನದು, ನನಗೆ ಸೇರಿದ್ದು, ನನ್ನ ಹತ್ತಿರ ಇರಬೇಕು, ಎಂದು ಯೋಚಿಸುವುದಿಲ್ಲ. ಅವನು ಎಲ್ಲ ವ್ಯಕ್ತಿಗಳನ್ನೂ ಪ್ರೀತಿಸುತ್ತಾನೆ. ಏಕೆಂದರೆ ಎಲ್ಲರ ಹಿಂದೆಯೂ ಭಗವಂತನೇ ಇರುವನು. ಎಲ್ಲ ಸುಂದರ ವಸ್ತುಗಳನ್ನೂ ಪ್ರೀತಿಸುತ್ತಾನೆ. ಅದರ ಹಿಂದೆಲ್ಲ ಇರುವವನು ದೇವರೆ ಎಂಬುದು ಗೊತ್ತಿದೆ. ಅವನು ಯಾವುದಕ್ಕೂ ಬದ್ಧನಲ್ಲ. ಯಾವುದೂ ಅವನ ಮನಸ್ಸಿನಲ್ಲಿ ಅಂಟಿಕೊಳ್ಳುವುದಿಲ್ಲ. ಕನ್ನಡಿ ತನ್ನ ಎದುರಿಗೆ ಇರುವುದನ್ನೆಲ್ಲ ಪ್ರತಿಬಿಂಬಿಸುವುದು. ಆದರೆ ಅದು ಕ್ಯಾಮರಾದಂತೆ ಚಿತ್ರವನ್ನು ಹಿಡಿದು ನಿಲ್ಲಿಸುವುದಿಲ್ಲ. ಸ್ಥಿತಪ್ರಜ್ಞ, ಹಾಗೆ ಎಲ್ಲವನ್ನೂ ಪ್ರತಿಬಿಂಬಿಸುವನು, ಆದರೆ ಮನಸ್ಸಿನಲ್ಲಿ ಹಿಡಿದು ನಿಲ್ಲಿಸುವುದಿಲ್ಲ.

ಅವನಲ್ಲಿ ಭಯವಿಲ್ಲ. ಯಾವಾಗ ಒಂದು ವಸ್ತುವಿಗೆ ಅಂಟಿಕೊಂಡಿರುವನೋ ಆಗ ಭಯ. ಅದು ಯಾವಾಗ ನನ್ನ ಕೈಬಿಟ್ಟು ಹೋಗುವುದೊ, ಅದಕ್ಕೆ ಏನಾಗುವುದೊ ಎಂದು. ಈ ಪ್ರಪಂಚದಲ್ಲಿ ಯಾವ ಭಯಾನಕವಾದ ಘಟನೆಯೇ ಆಗಲಿ ಉಗ್ರಸ್ವರೂಪದ ಮನುಷ್ಯನೇ ಆಗಲಿ ಇವನನ್ನು ಅಂಜಿಸಲಾರದು. ಭಯಕ್ಕೆ ಭಯ, ಮೃತ್ಯುವಿಗೆ ಮೃತ್ಯುವಾದ ಭಗವಂತನನ್ನು ಪ್ರೀತಿಸುವುದು, ಈ ಪ್ರಪಂಚಕ್ಕೆ ಅಂಜುವುದು ಒಟ್ಟಿಗೆ ಹೋಗಲಾರದು. ಅವನನ್ನು ಪ್ರೀತಿಸಿದರೆ, ಆತನ ಆಧಾರದ ಮೇಲೆ ತನ್ನ ಜೀವನವನ್ನು ರೂಪಿಸಿಕೊಂಡಿದ್ದರೆ, ಅನುಗಾಲವೂ ಎಡೆಬಿಡದೆ ಅವನನ್ನು ಚಿಂತಿಸುತ್ತಿರು ವವನು, ನಿರ್ಭೀತನಾಗುವನು. ಅವನಲ್ಲಿ ಭಯ ಇಲ್ಲದೆ ಇರುವುದು ಮಾತ್ರವಲ್ಲ, ಅವನ ಸಾನ್ನಿಧ್ಯವೇ ನಿರ್ಭೀತಿಯಿಂದ ಸ್ಪಂದಿಸುತ್ತಿರುವುದು.

ಅವನಲ್ಲಿ ಕ್ರೋಧವಿಲ್ಲ. ಯಾರಲ್ಲಿ ಆಸಕ್ತಿ ಇದೆಯೋ ಅಲ್ಲಿ ಕ್ರೋಧ ಬರುವುದು. ಯಾವಾಗ ನಾವು ಒಂದು ವಸ್ತುವಿನಲ್ಲಿ ಆಸಕ್ತರಾಗಿರುವೆವೊ ಆಗ ನನಗೂ ಆ ವಸ್ತುವಿಗೂ ಮಧ್ಯೆ ಆತಂಕವಾಗಿ ಯಾವುದು ಬಂದರೂ ಅದರ ಮೇಲೆ ಕೋಪ ಬರುವುದು. ಆಸಕ್ತಿಯೇ ಇಲ್ಲದ ಮನುಷ್ಯನಿಗೆ ಇನ್ನು ಕೋಪ ಬರುವುದು ಹೇಗೆ? ಕೋಪದ ಉರಿಗೆ ಆಸಕ್ತಿಯೇ ಸೌದೆ. ಯಾವಾಗ ಆ ಸೌದೆ ಇಲ್ಲವೊ ಇನ್ನು ಅದು ಉರಿಯುವುದು ಹೇಗೆ?

ಆಸಕ್ತಿ, ಭಯ, ಕ್ರೋಧ ಇವು ಮೂರೇ ನಮ್ಮ ಮಾನಸಿಕ ಶಕ್ತಿಯನ್ನೆಲ್ಲ ವ್ಯಯ ಮಾಡುವುದು. ಇಲ್ಲಿ ವ್ಯಯವಾಗುವುದನ್ನೆಲ್ಲ ತಡೆಗಟ್ಟಿ ಅದನ್ನು ಅಂತರ್ಮುಖ ಮಾಡಿ, ಪರಮಾತ್ಮನ ಕಡೆಗೆ ಹರಿಸುತ್ತಾನೆ ಸ್ಥಿತಪ್ರಜ್ಞ.

\begin{shloka}
ಯಃ ಸರ್ವತ್ರಾನಭಿಸ್ನೇಹಸ್ತತ್ತತ್ ಪ್ರಾಪ್ಯ ಶುಭಾಶುಭಮ್~।\\ನಾಭಿನಂದತಿ ನ ದ್ವೇಷ್ಟಿ ತಸ್ಯ ಪ್ರಜ್ಞಾ ಪ್ರತಿಷ್ಠಿತಾ \hfill॥ ೫೭~॥
\end{shloka}

\begin{artha}
ಯಾರು ಯಾವುದರಲ್ಲಿಯೂ ಆಸಕ್ತಿಯಿಲ್ಲದೆ ಪ್ರಿಯ ಅಪ್ರಿಯ ವಿಷಯಗಳನ್ನು ಪಡೆದಾಗ ಸಂತೋಷಿಸುವುದೂ ಇಲ್ಲವೊ ದ್ವೇಷಿಸುವುದೂ ಇಲ್ಲವೊ ಅವನ ಪ್ರಜ್ಞೆ ಸ್ಥಿರವಾಗಿರುವುದು.
\end{artha}

ಸ್ಥಿತಪ್ರಜ್ಞನಿಗೆ ಯಾವುದರಲ್ಲಿಯೂ ಆಸಕ್ತಿ ಇಲ್ಲ. ಯಾವುದು ಬರುವುದೊ ಅದನ್ನು ಸ್ವೀಕರಿಸುವನು. ಬರದೇ ಇದ್ದರೆ ಇದು ಬರುವುದಿಲ್ಲವೆಂದು ಚಿಂತಿಸುವುದಾಗಲೀ ಬಂದರೆ ಅದಕ್ಕೆ ಆಸಕ್ತನಾಗಿರುವುದಾಗಲೀ ಇಲ್ಲ. ಹಾಗಾದರೆ ಎಲ್ಲಾ ವಸ್ತು ಮತ್ತು ವ್ಯಕ್ತಿಗಳಲ್ಲಿ ಉದಾಸೀನನಾಗಿರುವನೆ? ಹೊರಗೆ ನೋಡಿದರೆ ಹಾಗೆ ಕಾಣುವುದಿಲ್ಲ. ಎಲ್ಲರನ್ನೂ ಸ್ವಾಗತಿಸುವನು, ಬೀಳ್ಕೊಡುವನು. ಮಾನವೀಯ ಭಾವನೆಗಳೆಲ್ಲ ಇರುವುದನ್ನು ನೋಡುತ್ತೇವೆ. ಆದರೆ ಅವನು ನಿಸ್ಸಂಗಿ. ಕಮಲದ ಎಲೆಯ ಮೇಲೆ ನೀರು ಕುಳಿತಿರುವುದು. ಅದನ್ನು ಸ್ವಲ್ಪ ಅಲುಗಾಡಿಸಿದೊಡನೆಯೆ ಅಂಟಿಕೊಳ್ಳದೆ ಉರುಳಿಹೋಗುವುದು. ಹಾಗೆಯೆ ಇವನು ಪ್ರಪಂಚದ ವಸ್ತುಗಳೊಂದಿಗೆ.

ಪ್ರಿಯವಾಗಿರುವ ವಸ್ತುವನ್ನು ಪಡೆದಾಗ ಸಂತೋಷಪಡುವುದಿಲ್ಲ. ಏಕೆಂದರೆ ಅವನಿಗೆ ಗೊತ್ತಿದೆ ಈ ಪ್ರಿಯವಾದ ವಸ್ತುವಿನ ಛಾಯೆಯ ಅನಿತ್ಯತೆ. ಯಾವಾಗ ಒಂದಕ್ಕೆ ಅಂಟಿಕೊಳ್ಳುವನೊ ಮತ್ತೊಂದನ್ನು ಅವನು ಸ್ವೀಕರಿಸಬೇಕಾಗುವುದು. ಪ್ರಿಯವಾದ ವಸ್ತುವನ್ನು ಪಡೆಯುವುದೊಂದು ಸಾಲ ಮಾಡಿದಂತೆ. ಅದರ ಅಪ್ರಿಯವಾದ ಭಾಗವನ್ನು ಅನುಭವಿಸಿಯೇ ಆ ಸಾಲವನ್ನು ತೀರಿಸ ಬೇಕಾದರೆ. ಸಾಲ ಮಾಡಿದರೆ ತೀರಿಸಬೇಕಾಗುವುದು. ಆದಕಾರಣ ಅವನು ಸಾಲಮಾಡುವ ಗೋಜಿಗೆ ಹೋಗುವುದಿಲ್ಲ.

ಹಾಗೆಯೆ ಅಪ್ರಿಯವಾಗಿರುವುದು ಬಂದರೆ ಅವನು ಹತಾಶನೂ ಆಗುವುದಿಲ್ಲ. ಅದು ಬರುವುದು ಹೋಗುವುದಕ್ಕೆ, ಎಂದೆಂದಿಗೂ ಇರುವುದಕ್ಕಲ್ಲ ಎಂಬುದನ್ನು ಅರಿತಿರುವನು. ಅದಿರುವಾಗಲೂ ಅದನ್ನು ಲಕ್ಷ್ಯದಲ್ಲಿಟ್ಟರೆ ಅದು ನಮ್ಮನ್ನು ಬಾಧಿಸುವುದು. ಆದರೆ ಅದನ್ನು ಲಕ್ಷ ್ಯದಲ್ಲಿಡದೇ ಇದ್ದರೆ ಅದು ನಮ್ಮ ಮನಸ್ಸಿನ ಮೇಲೆ ಯಾವ ಪರಿಣಾಮವನ್ನೂ ಉಂಟುಮಾಡಲಾರದು. ಪ್ರಪಂಚದಲ್ಲಿರುವಾಗ ಅವನು ಎರಡನ್ನೂ ನೋಡುವುದಕ್ಕೆ ಅಣಿಯಾಗಿರುವನು. ಯಾವುದೂ ಇದ್ದಕ್ಕೆ ಇದ್ದಂತೆ ಇವನ ಮೇಲೆ ಬೀಳುವುದಕ್ಕೆ ಆಗುವುದಿಲ್ಲ. ಸಾಧಾರಣ ಮನುಷ್ಯರು ಇರುವ ಜಗತ್ತಿನಲ್ಲಿಯೇ ಇವನೂ ಇರುವನು. ಬೇಕು ಬೇಡ ಎಂಬ ಅಲೆಗಳು ಇತರರನ್ನು ಅಪ್ಪಳಿಸುವಂತೆ ಸ್ಥಿತಪ್ರಜ್ಞನ ಮೇಲೂ ಬೀಳುವುವು. ಆದರೆ ಸ್ಥಿತಪ್ರಜ್ಞ ಈ ಅಲೆಯ ಹೊಡೆತದಿಂದ ತಪ್ಪಿಸಿಕೊಳ್ಳುವುದನ್ನು ಚೆನ್ನಾಗಿ ಬಲ್ಲನು. ಅವನು ಪ್ರಪಂಚವನ್ನು ಬದಲಾಯಿಸುವುದಕ್ಕೆ ಹೋಗುವುದಿಲ್ಲ. ಇದು ಸಾಧ್ಯವಿಲ್ಲ. ಪ್ರಪಂಚ ಯಾವಾಗಲೂ ನಾಯಿ ಬಾಲದಂತೆ ಡೊಂಕು ಎನ್ನುವುದನ್ನು ಅವನು ಚೆನ್ನಾಗಿ ಅರಿತಿರುವನು. ಆದರೆ ಅವನು ತನ್ನನ್ನು ಬದಲಾವಣೆ ಮಾಡಿಕೊಂಡಿರುವನು. ತನ್ನನ್ನು ಬದಲಾವಣೆ ಮಾಡಿಕೊಂಡರೆ ಸಾಕು, ಇಡೀ ಪ್ರಪಂಚವೇ ಬದಲಾವಣೆ ಆಗುವುದು ಇವನ ಪಾಲಿಗೆ. ಪ್ರಿಯ ಅಪ್ರಿಯದ ಮುಳ್ಳು ಎಲ್ಲೆಲ್ಲಿದೆಯೊ ಅದನ್ನೆಲ್ಲ ಗುಡಿಸಿ ಆಚೆಗೆ ಎಸೆಯುವುದಕ್ಕೆ ಹೋಗುವು ದಿಲ್ಲ. ಅವನು ಕಾಲಿಗೆ ಒಂದು ಜೊತೆ ಎಕ್ಕಡ ಹಾಕಿಕೊಂಡು ಹೋಗುವನು. ಈ ಸಣ್ಣ ಬದಲಾವಣೆ ಸಾಕು, ಆ ದೊಡ್ಡ ಕಿರುಕುಳದಿಂದ ಪಾರಾಗುವುದಕ್ಕೆ.

\begin{shloka}
ಯದಾ ಸಂಹರತೇ ಚಾಯಂ ಕೂರ್ಮೋಽಂಗಾನೀವ ಸರ್ವಶಃ~।\\ಇಂದ್ರಿಯಾಣೀಂದ್ರಿಯಾರ್ಥೇಭ್ಯಸ್ತಸ್ಯ ಪ್ರಜ್ಞಾ ಪ್ರತಿಷ್ಠಿತಾ \hfill॥ ೫೮~॥
\end{shloka}

\begin{artha}
ಆಮೆ ಹೊರಗಿರುವ ತನ್ನ ಅಂಗಾಗಗಳನ್ನೆಲ್ಲ ಹೇಗೆ ಒಳಗೆ ಸೆಳೆದುಕೊಳ್ಳುವುದೊ ಹಾಗೆ ಇವನು ತನ್ನ ಇಂದ್ರಿಯಗಳನ್ನು ಅವುಗಳ ವಿಷಯ ವಸ್ತುವಿನಿಂದ ಯಾವಾಗ ಸೆಳೆದುಕೊಳ್ಳುವನೊ ಆಗ ಇವನ ಪ್ರಜ್ಞೆ ಸ್ಥಿರವಾಗುವುದು.
\end{artha}

ಆಮೆ ನೀರಿನಲ್ಲಿ ವಿಹರಿಸುತ್ತಿರುವಾಗ ಕೈಕಾಲುಗಳನ್ನು ಹೊರಗೆ ಚಾಚಿ ಅದರಲ್ಲಿರುವ ಆಹಾರಗಳನ್ನು ತೆಗೆದುಕೊಳ್ಳುತ್ತಿರುತ್ತದೆ. ಆದರೆ ಯಾವಾಗ ಅಪಾಯ ಬರುವುದೊ ಆಗ ತನ್ನ ಚಿಪ್ಪಿನೊಳಗೆ ಅದನ್ನು ಸೆಳೆದುಕೊಳ್ಳುವುದು. ಆಮೆಯ ಚಿಪ್ಪು ಅಷ್ಟು ಗಟ್ಟಿಯಾಗಿದೆ. ಅದರ ಒಳಗಿರುವುದನ್ನು ಯಾರೂ ಏನೂ ಮಾಡುವುದಕ್ಕೆ ಆಗುವುದಿಲ್ಲ.

ಸ್ಥಿತಪ್ರಜ್ಞನೂ ಹಾಗೆಯೇ ತನ್ನ ಶರೀರ ಪೋಷಣೆಗೆ ಬೇಕಾದಷ್ಟು ವ್ಯವಹಾರವನ್ನು ಇಂದ್ರಿಯಗಳ ಮೂಲಕ ನೆರವೇರಿಸುತ್ತಿರುವನು. ಆದರೆ ವಿಷಯ ವಸ್ತುಗಳ ಮೂಲಕ ಅವನನ್ನು ಪ್ರಪಂಚಕ್ಕೆ ಕಟ್ಟಿಹಾಕಲು ಯಾರಾದರೂ ಪ್ರಯತ್ನಪಟ್ಟರೆ ಅವನು ತನ್ನ ಇಂದ್ರಿಯಗಳನ್ನು ಅಂತರ್ಮುಖ ಮಾಡುವನು. ಮನಸ್ಸನ್ನು ಇಂದ್ರಿಯದ ಕಡೆ ಹರಿದು ಹೋಗದಂತೆ ನೋಡಿಕೊಳ್ಳುವನು. ಕೆಲವುವೇಳೆ ವಿಷಯ ವಸ್ತುಗಳಿಂದ ನಾವು ಓಡಿಹೋಗುವುದಕ್ಕೆ ಆಗುವುದಿಲ್ಲ. ನಮಗೆ ಬೇಡದೇ ಇದ್ದರೂ ಅವು ನಮ್ಮೆದುರಿಗೆ ಬರುತ್ತವೆ. ಆದರೆ ನಾವು ಅವುಗಳೊಡನೆ ವ್ಯವಹರಿಸಿ ಸಂಬಂಧವನ್ನು ಕಲ್ಪಿಸಿಕೊಂಡಾಗಲೇ ಅದರ ದಾಸರಾಗಬೇಕಾದರೆ. ಸ್ಥಿತಪ್ರಜ್ಞ ಇಲ್ಲಿ ತಡೆಯುವನು. ಎರಡು ಕೈಗಳು ಸೇರಿದರೆ ಚಪ್ಪಾಳೆ ಆಗುವಂತೆ ಬಾಹ್ಯವಸ್ತು ಮತ್ತು ಇಂದ್ರಿಯ ಒಟ್ಟಿಗೆ ಸೇರಿದರೆ ಒಂದು ಅನುಭವ ಆಗುವುದು. ಪ್ರಪಂಚದ ಮೇಲೆ ಸ್ಥಿತಪ್ರಜ್ಞನಿಗೆ ಹತೋಟಿ ಇಲ್ಲ. ಅದನ್ನು ಬದಲಾವಣೆ ಮಾಡುವುದಕ್ಕೆ ಹೋಗುವುದಿಲ್ಲ. ಯಾವುದರ ಮೇಲೆ ಇವನಿಗೆ ಅಧಿಕಾರವಿರುವುದೊ ಅದನ್ನು ತಡೆಗಟ್ಟುವನು. ಅನುಭವ ಆಗುವುದಕ್ಕೆ ಇವನು ಬಿಡುವುದಿಲ್ಲ. ಸ್ಥಿತಪ್ರಜ್ಞನ ಇಂದ್ರಿಯಗಳಾದರೋ ಅವನ ಆಜ್ಞಾನು ಸಾರ ವರ್ತಿಸುತ್ತಿರುತ್ತವೆ. ಒಳ್ಳೆಯ ಡ್ರೈವರು ಕಾರು ಎಷ್ಟು ವೇಗವಾಗಿ ಹೋಗುತ್ತಿದ್ದರೂ ಕ್ಷಣಾರ್ಧದಲ್ಲಿ ನಿಲ್ಲಿಸಬಲ್ಲ. ಹಾಗಿದೆ ಸ್ಥಿತಪ್ರಜ್ಞನಿಗೆ ತನ್ನ ಇಂದ್ರಿಯಗಳ ಮೇಲಿನ ಒಡೆತನ.

\begin{shloka}
ವಿಷಯಾ ವಿನಿವರ್ತಂತೇ ನಿರಾಹಾರಸ್ಯ ದೇಹಿನಃ~।\\ರಸವರ್ಜಂ ರಸೋಽಪ್ಯಸ್ಯ ಪರಂ ದೃಷ್ಟ್ವಾ ನಿವರ್ತತೇ \hfill॥ ೫೯~॥
\end{shloka}

\begin{artha}
ನಿರಾಹಾರನಾದ ದೇಹಿಯನ್ನು ಬಿಟ್ಟು ವಿಷಯಗಳು ಹಿಂದಿರುಗುತ್ತವೆ. ಆದರೆ ರಸ ಹೋಗಿರುವುದಿಲ್ಲ. ಯಾವಾಗ ಪರಮಾತ್ಮನನ್ನು ಸಾಕ್ಷಾತ್ಕಾರ ಮಾಡಿಕೊಳ್ಳುವನೊ ಆಗ ರಸವು ಕೂಡ ಹಿಂತಿರುಗುವುದು.
\end{artha}

ಇಲ್ಲಿ ಆಹಾರ ಎಂದರೆ ಪಂಚೇಂದ್ರಿಯಗಳ ಮೂಲಕ ನಾವು ತೆಗೆದುಕೊಳ್ಳುವ ಅನುಭವಕ್ಕೆಲ್ಲ ಇದು ಅನ್ವಯಿಸುವುದು. ಬರೀ ಬಾಯಿನ ಮೂಲಕ ತೆಗೆದುಕೊಳ್ಳುವುದು ಮಾತ್ರವಲ್ಲ ನಮ್ಮ ಮನಸ್ಸನ್ನು ಪೋಷಿಸುತ್ತಿರುವುದು. ಕೇಳುವುದು, ನೋಡುವುದು, ಮುಟ್ಟುವುದು, ಮೂಸಿ ನೋಡುವುದು ಇವುಗಳೆಲ್ಲ ನಮ್ಮ ಮನಸ್ಸಿನಲ್ಲಿ ಸಂಸ್ಕಾರಗಳನ್ನು ಬಿಡುತ್ತವೆ. ಇವುಗಳೇ ನಮ್ಮಲ್ಲಿ ವಾಸನೆಯಾಗಿ ನಿಲ್ಲುವುದು. ಯಾವಾಗ ಒಬ್ಬ ಸಾಧಕ ಈ ವಾಸನೆಗಳೇ ತನ್ನನ್ನು ವಿಷಯ ವಸ್ತುಗಳಿಗೆ ಕಟ್ಟಿಹಾಕಿರುವುದು, ಹೊರಗಿನಿಂದ ಸ್ವೀಕರಿಸುವುದನ್ನು ನಿಲ್ಲಿಸಬೇಕು ಎಂದು ಸಂಕಲ್ಪ ಮಾಡುವನೊ ಅವನಿಂದ ವಿಷಯವಸ್ತುಗಳು ದೂರ ಸರಿಯುವುವು. ಆದರೆ ಮನಸ್ಸಿನ ಆಳದಲ್ಲಿ ಆಸೆಯ ಕೆಂಗೆಂಡ ಇರುತ್ತದೆ. ಒಲೆಗೆ ಹೊಸದಾಗಿ ಸೌದೆ ಹಾಕದೆ ಬಿಟ್ಟರೆ ಇರುವ ಸೌದೆಯೆಲ್ಲ ಉರಿದು ಕೆಂಗೆಂಡವಾಗಿ ಕೊನೆಗೆ ಬೂದಿಯಿಂದ ಮುಚ್ಚಿಹೋಗುವುದು. ಆ ಬೂದಿಯನ್ನು ಕೆದಕಿದರೆ ಒಳಗೆ ಕೆಂಡ ಇರುವುದು ಕಾಣುವುದು. ಇದರಂತೆಯೇ ನಾವು ಇನ್ನು ಮೇಲೆ ಹೊಸದನ್ನು ಹಾಕದೆ ಹೋದರೂ ಹಳೆಯ ವಾಸನೆಗಳು ಸಂಪೂರ್ಣ ನಾಶವಾಗಿರುವುದಿಲ್ಲ. ಸ್ವಲ್ಪ ಶೇಷ ಇದ್ದೇ ಇರುವುದು. ಎಲ್ಲಿಯವರೆಗೆ ಆಸೆಯ ಶೇಷ ಇರುವುದೊ ಅಲ್ಲಿಯವರೆಗೆ ನಾವು ಜೋಪಾನವಾಗಿರಬೇಕು. ಅಜಾಗರೂಕತೆಯಿಂದ ಅಥವಾ ಹಿಂದಿನ ಅಭ್ಯಾಸದ ಬಲದಿಂದ ಅದಕ್ಕೆ ಸ್ವಲ್ಪ ಕಟ್ಟಿಗೆ ಬಿದ್ದರೂ ಕ್ರಮೇಣ ಹೊಗೆಯಾಡಿ ಹತ್ತಿಕೊಳ್ಳುವುದು.

ಆ ಸ್ವಲ್ಪ ಆಸೆಯ ಶೇಷ ಬಿಟ್ಟುಹೋಗಬೇಕಾದರೆ ಪರಮಾತ್ಮನ ಸಾಕ್ಷಾತ್ಕಾರವಾಗಬೇಕು. ಆಗಲೆ ಅದು ಸಂಪೂರ್ಣ ನಾಶವಾಗಬೇಕಾದರೆ. ಬೀಜವನ್ನು ಯಾವಾಗ ಒಲೆಯ ಮೇಲೆ ಇಟ್ಟು ಹುರಿಯುತ್ತೇವೆಯೊ ಅಥವಾ ನೀರಿನಲ್ಲಿಟ್ಟು ಬೇಯಿಸುತ್ತೇವೆಯೊ ಆಗಲೆ ಅದರಲ್ಲಿರುವ ಮೊಳಕೆ ಸಾಯ ಬೇಕಾದರೆ. ಅದರಂತೆಯೇ ಭಗವಂತನ ಅನುಭವದ ಕಾವು ತಾಕಿದರೆ ನಮ್ಮ ಉಳಿದ ಆಸೆಯ ಶೇಷ ಹಾಗೇ ನಾಶವಾಗುತ್ತದೆ.

\begin{shloka}
ಯತತೋ ಹ್ಯಪಿ ಕೌಂತೇಯ ಪುರುಷಸ್ಯ ವಿಪಶ್ಚಿತಃ~।\\ಇಂದ್ರಿಯಾಣಿ ಪ್ರಮಾಥೀನಿ ಹರಂತಿ ಪ್ರಸಭಂ ಮನಃ \hfill॥ ೬ಂ~॥
\end{shloka}

\begin{artha}
ಕೌಂತೇಯ, ಸಾಧನೆ ಮಾಡುತ್ತಿರುವ ಜ್ಞಾನಿಯ ಮನಸ್ಸನ್ನು ಪ್ರಬಲವಾದ ಇಂದ್ರಿಯಗಳು ಬಲಾತ್ಕಾರವಾಗಿ ಸೆಳೆಯುವುವು.
\end{artha}

ಇಂದ್ರಿಯಗಳು ಬಡಪೆಟ್ಟಿಗೆ ನಮ್ಮನ್ನು ಕಾಡುವುದನ್ನು ಬಿಡುವುದಿಲ್ಲ. ಒಮ್ಮೆ ಬರಬೇಡ ಎಂದು ಬಾಗಿಲು ಹಾಕಿದರೆ ಅವು ಹೊರಟು ಹೋಗಿಬಿಡುವುದಿಲ್ಲ. ಹೊರಗೆ ಹೋಗಿ ಎಲ್ಲಿಯಾದರೂ ಒಂದು ಕಡೆ ಅವಿತುಕೊಂಡಿರುವುವು. ನಮ್ಮ ಮನಸ್ಸಿಗೆ ಏಳು ಬೀಳುಗಳಿವೆ. ನಮ್ಮ ಮನಸ್ಸು ದುರ್ಬಲವಾಗಿರುವ ಸಮಯವನ್ನು ಹೊಂಚುಹಾಕಿಕೊಂಡಿದ್ದು ಕ್ರಮೇಣ ನಮ್ಮನ್ನು ಸಮೀಪಿಸುವುವು. ಆತನಿಗೆ ಅದರ ಸಹವಾಸ ಸಾಕಾಗಿದೆ. ಆದರೂ ಕೆಲವು ವೇಳೆ ಚಪಲ ಏಳುವುದು. ಅದು ಸಂಪೂರ್ಣ ನಾಶವಾಗಿಲ್ಲ. ಆಸೆಯ ಕಿಡಿ ನಮ್ಮಲ್ಲಿಇದೆ. ಅದು ಇರುವವರೆಗೆ ಜೋಪಾನವಾಗಿ ಇರಬೇಕು. ಅಜಾಗರೂಕತೆಯಿಂದ ಒಂದು ಕಟ್ಟಿಗೆ ಸಿಕ್ಕಿದರೆ ಸಾಕು. ಅದು ಪುನಃ ಧಗ್ ಎಂದು ಹತ್ತಿಕೊಳ್ಳುವುದು.

ಸಾಧನೆ ಮಾಡುತ್ತಿರುವವನು ಜ್ಞಾನಿ, ಬೌದ್ಧಿಕವಾಗಿ ಯಾವುದನ್ನು ಮಾಡಬೇಕು, ಯಾವುದನ್ನು ಬಿಡಬೇಕು, ಯಾವ ಯಾವುದರಿಂದ ನಮಗೆ ಏನೇನು ಪರಿಣಾಮಗಳು ಕಾದುಕೊಂಡಿವೆ ಎಂಬುದನ್ನು ಬಲ್ಲವನು. ಆದರೂ ಚಪಲ ನಮ್ಮನ್ನು ಒಂದೇ ಸಲ ಬಿಡುವುದಿಲ್ಲ. ಕೆಲವು ವೇಳೆ ತೀವ್ರವಾಗಿ ಮೇಲೇಳುವುದು. ಪರಿಣಾಮಗಳೆಲ್ಲ ಗೊತ್ತಿವೆ. ಆದರೂ ಸಾಯದ ಹಿಂದಿನ ಸಂಸ್ಕಾರದ ಕೈಗೆ ಸಿಕ್ಕಿ ಬಲಾತ್ಕಾರದಿಂದ ಒಂದು ಕ್ರಿಯೆಯನ್ನು ಮಾಡಿ ಹಾಕುವೆವು. ನಮ್ಮ ಮನಸ್ಸಿನ ಯಾವುದೋ ಒಂದು ಭಾಗ ಇಂದ್ರಿಯದ ಮೇಲಿನ ಆಸೆಯನ್ನು ಬಿಟ್ಟಿದೆ. ಉಳಿದ ಭಾಗಗಳು ಬೇಕೆಂದು ಕೇಳುತ್ತವೆ. ಕೆಲವು ವೇಳೆ ಹೊಸದು ಗೆಲ್ಲುವುದು, ಕೆಲವು ವೇಳೆ ಹಳೆಯದು ಗೆಲ್ಲುವುದು. ಆದಕಾರಣವೇ ಆಧ್ಯಾತ್ಮಿಕ ಜೀವನದಲ್ಲಿ ಎಷ್ಟು ಮುಂದುವರಿದರೂ ಪೂರ್ಣತೆಯನ್ನು ಮುಟ್ಟುವವರೆಗೆ ಸುರಕ್ಷಿತರು ಎಂದು ಹೇಳಲಾಗುವುದಿಲ್ಲ. ಒಂದು ಸಲ ಶಿಷ್ಯನೊಬ್ಬ ಶ‍್ರೀಶಾರದಾದೇವಿಯವರ ಹತ್ತಿರ ಮಾತನಾಡುತ್ತಿದ್ದಾಗ ಈಗ ನನ್ನ ಮನಸ್ಸು ಪರಿಶುದ್ಧವಾಗಿದೆ, ಅಪಾಯದಿಂದ ಪಾರಾದೆ ಎಂದು ಹೇಳಿದ. ತಕ್ಷಣವೇ ಶಾರದಾದೇವಿಯವರು “ಹಾಗೆ ಹೇಳಬೇಡ. ತಿಳಿ ಆಕಾಶದಲ್ಲಿ ಯಾವಾಗ ಬೇಕಾದರೂ ಮೋಡ ಕವಿದು ಮಳೆ ಬರುತ್ತದಲ್ಲ ಹಾಗೆ ಎಲ್ಲಿಂದಲೊ ಪೂರ್ವಸಂಸ್ಕಾರದ ಮೋಡಗಳು ಬಂದು ನಾವು ಪಾರಾದೆವೆಂದು ಭಾವಿಸುತ್ತಿರುವಾಗಲೆ ಬೀಳುವುದಕ್ಕೆ ಹೊಂಚುಕಾಯುತ್ತಿರಬಹುದು” ಎಂದರು.

ಪರಿಸ್ಥಿತಿ ಹೀಗಿದ್ದರೆ ನಾವು ನಿರಾಶರಾಗಬೇಕಾಗಿಲ್ಲ. ಇದೇನು ತೋರುತ್ತದೆ ಎಂದರೆ ನಾವು ಯಾವಾಗಲೂ ಜಾಗರೂಕರಾಗಿರಬೇಕು. ನಾವು ಎಷ್ಟು ಮುಂದುವರಿದರೂ ನನ್ನ ಸಮಾನ ಇಲ್ಲ ಎಂದು ಹೆಮ್ಮೆ ಪಡಕೂಡದು. ಇದುವರೆಗೆ ಬಿದ್ದಿಲ್ಲ ಎಂದರೆ ಮುಂದೆಯೂ ಬೀಳುವುದಿಲ್ಲ ಎಂದು ಹೇಳುವಂತಿಲ್ಲ. ಇದೆಲ್ಲೊ ಹೊರಗಡೆ ಇರುವ ಶತ್ರುವಿನೊಂದಿಗೆ ಹೋರಾಡುವುದಲ್ಲ. ಸದಾಕಾಲ ನಮ್ಮ ಮನಸ್ಸಿನ ಕಾಡಿನಲ್ಲಿಯೇ ಹುದುಗಿಕೊಂಡಿರುವ ದುಷ್ಟಮೃಗದೊಡನೆ ನಡೆಸುವ ಹೋರಾಟ. ಆ ಮೃಗಕ್ಕೆ ನಮಗಿಂತ ಹೆಚ್ಚಾಗಿ ನಮ್ಮ ಬಲಾಬಲಗಳು ಗೊತ್ತಿವೆ. ಯಾವ ಸಮಯದಲ್ಲಿ ನಮ್ಮ ಮೇಲೆ ಬೀಳುವುದೋ ನಮಗೆ ಗೊತ್ತಿಲ್ಲ. ಸಾಧಾರಣ ಕಾಲದಲ್ಲಿ ಎಂತಹ ದೊಡ್ಡ ವಿವೇಕಿಯಾದರೂ ಜೀವನ ಪರೀಕ್ಷಾಸಮಯಗಳು ಬಂದಾಗ ವಿವೇಕ ತಾತ್ಕಾಲಿಕವಾಗಿ ವ್ಯಕ್ತಿಯನ್ನು ಬಿಟ್ಟು ಹೋದಂತಿರುವುದು. ಜಿಂಕೆಯೊಂದು ಕಾಡಿನಲ್ಲಿ ನೀರು ಕುಡಿಯುತ್ತಿತ್ತು. ಆ ನೀರಿನಲ್ಲಿ ತನ್ನ ಪ್ರತಿಬಿಂಬವನ್ನು ನೋಡಿಕೊಂಡಿತು. ಎಂತಹ ಚೂಪಾದ ಕೊಂಬುಗಳು, ಎಷ್ಟು ಬಲವಾದ ಕೈಕಾಲುಗಳು ತನಗಿವೆ ಎಂದು ತನ್ನ ಹತ್ತಿರ ಇದ್ದ ಮರಿಗೆ ಹೇಳಿಕೊಂಡಿತು. ಸ್ವಲ್ಪ ಹೊತ್ತಿನಲ್ಲಿಯೇ ದೂರದಲ್ಲಿ ಇದನ್ನು ಹಿಡಿಯುವ ಬೇಟೆಯ ನಾಯಿಗಳ ಸಪ್ಪಳ ಕೇಳಿಬಂತು. ಅಲ್ಲಿಂದ ತನ್ನ ಪ್ರಾಣ ರಕ್ಷಣೆಗೆ ಆ ಜಿಂಕೆ ನಾಗಾಲೋಟದಲ್ಲಿ ಓಡಿದ್ದೇ ಓಡಿದ್ದು. ತುಂಬ ದೂರ ಓಡಿದ ಮೇಲೆ ಏದುಸಿರು ಬಿಡುತ್ತಾ ನಿಂತುಕೊಂಡಿತು. ಆಗ ಮರಿ ಕೇಳಿತು “ನಿನಗೆ ಅಷ್ಟೊಂದು ಚೂಪಾದ ಕೊಂಬುಗಳಿವೆ ಎಂದು ಹೇಳಿಕೊಳ್ಳುತ್ತಿದ್ದೆಯಲ್ಲ, ಏತಕ್ಕೆ ನಾಯಿಗಳನ್ನು ಅದರಿಂದ ತಿವಿಯಬಾರದಿತ್ತು?” ಆಗ ಜಿಂಕೆ, “ಮಗು ಬೇಟೆಯ ನಾಯಿಯ ಧ್ವನಿ ಕೇಳಿದರೆ ನನ್ನ ಶಕ್ತಿಯ ಅರಿವೇ ನನಗೆ ಇರುವುದಿಲ್ಲ” ಎಂದಿತು. ಹಾಗೆಯೇ ಪಂಡಿತರು, ಜ್ಞಾನಿಗಳು ತಮ್ಮ ವಿವೇಕ ವೈರಾಗ್ಯಗಳನ್ನು ಕುರಿತು ಪರೀಕ್ಷಾಸಮಯ ಬರುವುದಕ್ಕೆ ಮುಂಚೆ ಹೆಮ್ಮೆಕೊಚ್ಚಿಕೊಳ್ಳಬಹುದು. ಆದರೆ ನಿಜವಾದ ಆ ಪರೀಕ್ಷಾಸಮಯವನ್ನು ಎದುರಿಸುವಾಗ ಕೆಳಗೆ ಉರುಳುವವರೇ ಬಹುಪಾಲು. ಹಾಗೆ ಉರುಳುವವರಲ್ಲಿ ಮುಂಚೆ ಹೆಮ್ಮೆ ಕೊಚ್ಚಿಕೊಳ್ಳುತ್ತಿದ್ದವರೇ ಅಧಿಕ.

\begin{shloka}
ತಾನಿ ಸರ್ವಾಣಿ ಸಂಯಮ್ಯ ಯುಕ್ತ ಆಸೀತ ಮತ್ಪರಃ~।\\ವಶೇ ಹಿ ಯಸ್ಯೇಂದ್ರಿಯಾಣಿ ತಸ್ಯ ಪ್ರಜ್ಞಾ ಪ್ರತಿಷ್ಠಿತಾ \hfill॥ ೬೧~॥
\end{shloka}

\begin{artha}
ಆದಕಾರಣ ಇಂದ್ರಿಯಗಳನ್ನೆಲ್ಲ ನಿಗ್ರಹಿಸಿ ಮತ್ಪರಾಯಣನಾಗಿ ಸಮಾಹಿತ ಚಿತ್ತನಾಗಿರಬೇಕು. ಯಾರ ಇಂದ್ರಿಯ ವಶದಲ್ಲಿರುವುದೋ ಅವನ ಪ್ರಜ್ಞೆ ಸ್ಥಿರವಾಗಿದೆ.
\end{artha}

ಇಂದ್ರಿಯಗಳನ್ನೆಲ್ಲ ನಿಗ್ರಹಿಸಬೇಕು. ಒಂದನ್ನು ನಿಗ್ರಹಿಸಿ ಮತ್ತೊಂದನ್ನು ನಿಗ್ರಹಿಸದಿದ್ದರೆ ಏನೂ ಪ್ರಯೋಜನವಿಲ್ಲ. ಒಂದರಲ್ಲಿ ಉಳಿದಿರುವುದು ಮತ್ತೊಂದರಲ್ಲಿ ವ್ಯಯವಾಗುವುದು. ಒಂದು ಕೊಡಕ್ಕೆ ಏಳೆಂಟು ತೂತುಗಳಿದ್ದರೆ ಯಾವುದೊ ಒಂದೆರಡು ತೂತುಗಳನ್ನು ಮಾತ್ರ ಮುಚ್ಚಿದರೆ ಮುಚ್ಚದೇ ಇರುವ ತೂತಿನ ಮೂಲಕ ನೀರು ಸೋರಿಹೋಗುವುದು. ಎಲ್ಲಾ ಇಂದ್ರಿಯಗಳೂ ನಮ್ಮನ್ನು ವಿಷಯವಸ್ತುವಿಗೆ ಕಟ್ಟಿಹಾಕುವುವು. ಒಂದು ಮೇಲಲ್ಲ ಮತ್ತೊಂದು ಕೀಳಲ್ಲ. ಒಂದರ ಬಲೆಗೆ ಬಿದ್ದರೆ ಉಳಿದವುಗಳೆಲ್ಲದರ ಬಲೆಗೂ ಕ್ರಮೇಣ ಬೀಳುತ್ತೇವೆ. ಚೆಂಡು ಮೇಲಿನ ಮೆಟ್ಟಲಿನಿಂದ ಜಾರಿದರೆ ಸಾಕು, ಕೆಳಗೆ ಉರುಳಿಕೊಂಡು ಬಂದುಬಿಡುವುದು. ಹಾಗೆಯೇ ಮನಸ್ಸು ಒಂದು ಇಂದ್ರಿಯ ನಿಗ್ರಹಿಸುವುದರಲ್ಲಿ ಸ್ವಲ್ಪ ಸಡಿಲವಾದರೆ, ಉಳಿದವುಗಳ ಬಲೆಗೂ ಕ್ರಮೇಣ ಬೀಳುವುದರಲ್ಲಿ ಸಂದೇಹವಿಲ್ಲ.

ಮತ್ಪರಾಯಣನಾಗಿರಬೇಕು. ಮನಸ್ಸನ್ನೆಲ್ಲ ಇಂದ್ರಿಯ ವಸ್ತುವಿನಿಂದ ತೆಗೆದು ಭಗವಂತನೇ ಜೀವನದಲ್ಲಿ ಸರ್ವಶ್ರೇಷ್ಠ ಎಂದು ಅವನ ಕಡೆಗೆ ಹರಿಸಬೇಕು. ಸುಮ್ಮನೆ ಇಂದ್ರಿಯವನ್ನು ಜಯಿಸುವುದಲ್ಲ. ಅದರ ಮೂಲಕ ಹರಿದು ಹೋಗುತ್ತಿದ್ದ ಶಕ್ತಿಯನ್ನು ಭಗವಂತನ ಕಡೆಗೆ ಹರಿಸಬೇಕು. ಭಗವಂತನ ಮೇಲೆ ಮನಸ್ಸನ್ನು ಇಟ್ಟಾಗ ಮಾತ್ರ ನಾವು ಇಂದ್ರಿಯದ ಉಪಟಳದಿಂದ ಪಾರಾಗುತ್ತೇವೆ. ಮನಸ್ಸಿಗೆ ಬಿಗಿಯಾಗಿ ಭಗವಂತನನ್ನು ಹಿಡಿದುಕೊಳ್ಳುವುದಕ್ಕೆ ಶಕ್ತಿ ಇಲ್ಲದೇ ಇದ್ದರೆ, ಇಂದ್ರಿಯ ಪ್ರಪಂಚಕ್ಕೆ ಪುನಃ ಜಾರುವುದು. ಒಂದನ್ನು ಬಿಡಬೇಕು ಮತ್ತೊಂದನ್ನು ಹಿಡಿದು ಕೊಳ್ಳಬೇಕು. ಇಂದ್ರಿಯ ಸುಖವನ್ನು ಬಿಡಬೇಕು. ದೇವರ ಪಾದಪದ್ಮವನ್ನು ಹಿಡಿಯಬೇಕು. ಇಂದ್ರಿಯ ಸುಖವನ್ನು ಬಿಟ್ಟರೇನೆ ಭಗವಂತನ ಸುಖವನ್ನು ಅನುಭವಿಸಬೇಕಾದರೆ. ಭಗವಂತನ ಸುಖದ ಅನುಭವ ನಮಗೆ ಸಿಕ್ಕಿದ್ದರೇನೇ ನಾವು ಪುನಃ ಇಂದ್ರಿಯ ಸುಖಕ್ಕೆ ಜಾರುವುದಿಲ್ಲ.

ಅವನು ಸಮಾಹಿತ ಚಿತ್ತನಾಗಿರಬೇಕು. ಮನಸ್ಸನ್ನೆಲ್ಲ ಕೇಂದ್ರೀಕರಿಸಿ ದೇವರ ಕಡೆಗೆ ಬಿಡಬೇಕು. ಎಲ್ಲೊ ಒಂದು ಸ್ವಲ್ಪವನ್ನು ದೇವರ ಕಡೆಗೆ ಕಳುಹಿಸುವುದು, ಉಳಿದವುಗಳನ್ನು ಕೀರ್ತಿ, ಐಶ್ವರ್ಯ, ಲಾಭ ಮುಂತಾದುವುಗಳಿಗೆ ಹಂಚಿದರೆ, ನಮಗೆ ಅದರಿಂದ ಅಷ್ಟೊಂದು ಪ್ರಯೋಜನವಾಗುವುದಿಲ್ಲ. ದೇವರ ಕಡೆಗೆ ಮನಸ್ಸನ್ನು ಹಾಕದೆ ಇರುವುದಕ್ಕಿಂತ ಸ್ವಲ್ಪ ಹಾಕುವುದೇನೋ ಮೇಲು. ಆದರೆ ಸರ್ವಶ್ರೇಷ್ಠವೇ ಮನಸ್ಸನ್ನೆಲ್ಲ ಭಗವಂತನ ಕಡೆ ಹೋಗುವಂತೆ ಮಾಡುವುದು. ಆಗಲೆ ಅದ್ಭುತವಾದ ಆಂತರಿಕ ಶಕ್ತಿ ನಮ್ಮಲ್ಲಿ ಉತ್ಪತ್ತಿಯಾಗಬೇಕಾದರೆ, ನಮ್ಮನ್ನು ಬಿಗಿದ ಮಾಯಾತೆರೆಯನ್ನು ಸೀಳಿಕೊಂಡು ಹೋಗಬೇಕಾದರೆ.

ಇಂದ್ರಿಯ ನಿಗ್ರಹ ಅತ್ಯಂತ ಆವಶ್ಯಕ ಆಧ್ಯಾತ್ಮಿಕ ಜೀವನಕ್ಕೆ. ಒಬ್ಬ ಆಧ್ಯಾತ್ಮಿಕ ಜೀವನದಲ್ಲಿ ಎಷ್ಟು ಮುಂದುವರಿದು ಹೋಗಿರುವನು ಎಂಬುದಕ್ಕೆ ಇದೊಂದೇ ಪರೀಕ್ಷೆ. ಬೇಕಾದಷ್ಟು ವೇದಾಂತ ವಿಷಯಗಳನ್ನು ಮಾತಾಡುವುದು, ಜೊತೆಗೆ ಕಾಮಕಾಂಚನದ ಮೇಲೆಯೂ ಮನಸ್ಸನ್ನು ಇಟ್ಟಿರುವುದು ಒಟ್ಟಿಗೆ ಹೋಗಲಾರದು. ಶ‍್ರೀರಾಮಕೃಷ್ಣರು, ರಣಹದ್ದು ಮೇಲೇನೊ ಹಾರುವುದು, ಆದರೆ ಅದರ ದೃಷ್ಟಿಯೆಲ್ಲ ಕೆಳಗೆ, ಎಲ್ಲಿ ಕೊಳೆತಿರುವ ಹೆಣಗಳು ಬಿದ್ದಿವೆ ಎಂಬುದನ್ನು ಹುಡುಕುವುದಕ್ಕಾಗಿ ಎಂದು ಹೇಳುತ್ತಿದ್ದರು. ಆ ಕೊಳೆತ ಹೆಣಗಳೇ ಕಾಮಕಾಂಚನ. ಒಬ್ಬ ನಾನು ಅದ್ವೈತತತ್ತ್ವವನ್ನು ಸಾಕ್ಷಾತ್ಕಾರ ಮಾಡಿಕೊಂಡಿರುವೆನು ಎನ್ನಬಹುದು. ಮತ್ತೊಬ್ಬ ನಾನು ದೇವರನ್ನೇ ಕಂಡಿರುವೆನು, ಅವನೊಡನೆ ಮಾತನಾಡಿರುವೆನು ಎನ್ನಬಹುದು. ಆದರೆ ಅವನು ಇಂದ್ರಿಯಗಳನ್ನು ಜಯಿಸಿಲ್ಲದೇ ಇದ್ದರೆ ಇವೆಲ್ಲ ಠಕ್ಕು ಎಂದು ಒಂದೇ ಸಲ ನಿರ್ಣಯಕ್ಕೆ ಬರಬಹುದು. ಒಂದನ್ನು ಬಿಟ್ಟರೇನೇ ಮತ್ತೊಂದು ಸಿಕ್ಕಬೇಕಾದರೆ. ಒಮ್ಮೆ ದೇವರನ್ನು ರುಚಿ ನೋಡುವುದು, ಮತ್ತೊಮ್ಮೆ ಪ್ರಾಪಂಚಿಕ ವಸ್ತುಗಳನ್ನು ರುಚಿನೋಡುವುದು ಸಾಧ್ಯವಿಲ್ಲ. ಈಗ ತಾನೆ ಮೃಷ್ಟಾನ್ನವನ್ನು ಉಂಡವನು ಇಂದ್ರಿಯ ಸುಖ ಎಂಬ ಆರಿದ ಎಂಜಲಿಗೆ ಕೈಯೊಡ್ಡುವುದಿಲ್ಲ. ಹಾಗೆ ಒಡ್ಡಿದರೆ ಮೃಷ್ಟಾನ್ನವನ್ನು ತಿಂದಿಲ್ಲ, ತಿಂದೆ ಎಂದು ಸುಮ್ಮನೆ ಹೇಳುತ್ತಾನೆ ಎಂದರ್ಥ.

\begin{shloka}
ಧ್ಯಾಯತೋ ವಿಷಯಾನ್ ಪುಂಸಃ ಸಂಗಸ್ತೇಷೂಪಜಾಯತೇ~।\\ಸಂಗಾತ್ ಸಂಜಾಯತೇ ಕಾಮಃ ಕಾಮಾತ್ ಕ್ರೋಧೋಽಭಿಜಾಯತೇ \hfill॥ ೬೨~॥
\end{shloka}

\begin{shloka}
ಕ್ರೋಧಾದ್ಭವತಿ ಸಂಮೋಹಃ ಸಂಮೋಹಾತ್ ಸ್ಮೃತಿವಿಭ್ರಮಃ~।\\ಸ್ಮೃತಿಭ್ರಂಶಾದ್ಬುದ್ಧಿನಾಶೋ ಬುದ್ಧಿನಾಶಾತ್ ಪ್ರಣಶ್ಯತಿ \hfill॥ ೬೩~॥
\end{shloka}

\begin{artha}
ವಿಷಯವನ್ನು ಚಿಂತಿಸುತ್ತಿರುವ ಮನುಷ್ಯನಿಗೆ ಅವುಗಳಲ್ಲಿ ಆಸಕ್ತಿ ಹುಟ್ಟುತ್ತದೆ. ಆಸಕ್ತಿಯಿಂದ ಕಾಮ ಹುಟ್ಟುವುದು. ಕಾಮದಿಂದ ಕ್ರೋಧ ಹುಟ್ಟುವುದು. ಕ್ರೋಧದಿಂದ ಸಂಮೋಹವುಂಟಾಗುವುದು. ಸಂಮೋಹದಿಂದ ಸ್ಮೃತಿಭ್ರಮೆ ಉಂಟಾಗುವುದು. ಸ್ಮೃತಿಭ್ರಮೆಯಿಂದ ಬುದ್ಧಿನಾಶ. ಬುದ್ಧಿನಾಶದಿಂದ ಜೀವಿ ನಾಶವಾಗುವನು.
\end{artha}

ಈ ಎರಡು ಶ್ಲೋಕಗಳಲ್ಲಿ ಮನುಷ್ಯ ಹೇಗೆ ಜಾರುತ್ತಾನೆ ಎಂಬುದನ್ನು ವಿವರಿಸುತ್ತಾನೆ. ದೇವರೆಡೆಗೆ ಏರುವುದು ಒಂದು ದೊಡ್ಡ ಸಾಹಸ. ಅಂಗುಲ ಅಂಗುಲ ತೆವಳಿಕೊಂಡು ನಾವು ಅತ್ತ ಏರಬೇಕು. ಆದರೆ ಜಾರುವುದಾದರೋ ಬಹಳ ಸುಲಭ. ಹಲವು ವರುಷಗಳಲ್ಲಿ ಸಾಧಿಸಿದುದನ್ನು ಕೆಲವು ನಿಮಿಷಗಳಲ್ಲಿ ನಮ್ಮ ಮನೋದೌರ್ಬಲ್ಯದಿಂದ ಹಾಳುಮಾಡಿಕೊಂಡುಬಿಡಬಹುದು. ಜೀವಿ ಜಾರುವುದು ಕ್ರಮೇಣ ಒಳಗೆ ಮೊದಲಾಗುವುದು. ಮನಸ್ಸಿನಲ್ಲಿ ಇಂದ್ರಿಯಕ್ಕೆ ಸಂಬಂಧಪಟ್ಟ ವಿಷಯವನ್ನು ಕುತೂಹಲದಿಂದಲೋ ಅಥವಾ ಮತ್ತೆ ಯಾವುದರಿಂದಲೋ ಪ್ರೇರಿತರಾಗಿ ಯೋಚಿಸುವುದಕ್ಕೆ ಶುರು ಮಾಡುವನು. ಹಾಗೆ ಯೋಚಿಸುವುದು ತಪ್ಪೆಂದು ಭಾವಿಸುವುದಿಲ್ಲ. ಮೊದಲು ಮನಸ್ಸಿನಲ್ಲಿ ಮಾಡಿದ ಪಾಪ ಪಾಪವೇ ಅಲ್ಲ ಈ ಕಲಿಯುಗದಲ್ಲಿ ಎಂದು ಹೇಳಿಕೊಳ್ಳುತ್ತಾರೆ. ಇದರಿಂದ ಹೊರಗಡೆ ಯಾರಿಗೂ ಏನೂ ಆಗಿಲ್ಲ, ನನಗೂ ಏನೂ ಆಗಿಲ್ಲ ದೈಹಿಕವಾಗಿ, ಎಂದು ಭಾವಿಸುತ್ತೇವೆ. ಎಲ್ಲಾ ಘೋರ ಅತ್ಯಾಚಾರಗಳು, ಪಾಪಕೃತ್ಯಗಳು ಮೊದಲು ಜನಿಸುವುದು ಹೀಗೆ. ನಾವು ಇಲ್ಲಿ ನಿಗ್ರಹಿಸಿದ್ದರೆ ಮತ್ತೊಂದಕ್ಕೆ ಬೀಳುತ್ತಿರಲಿಲ್ಲ. ಇಲ್ಲಿ ಸಡಿಲ ಮಾಡಿದೆವು. ಅದು ಉರುಳುವುದಕ್ಕೆ ಪ್ರಾರಂಭವಾಯಿತು. ಉರುಳುತ್ತ ಉರುಳುತ್ತ ಅದಕ್ಕೆ ವೇಗ ಹೆಚ್ಚುವುದು. ಮುಂಚೆ ಅದನ್ನು ನಿಲ್ಲಿಸುವುದು ಸುಲಭವಾಗಿತ್ತು. ಈಗತಾನೆ ಚಿಗುರುತ್ತಿರುವುದನ್ನು ಉಗುರಿನಿಂದ ಕಿತ್ತುಹಾಕಬಹುದು. ಆಗ ಮಾಡದೆ ಹೋದರೆ ಅನಂತರ ಕೊಡಲಿಯನ್ನು ತೆಗೆದುಕೊಳ್ಳಬೇಕಾಗುವುದು.

ವಿಷಯ ವಸ್ತುಗಳನ್ನು ಮುಂಚೆ ಸುಮ್ಮನಿರಿಸುವುದು ಬೇಜಾರು ಎಂದು ಚಿಂತಿಸುತ್ತೇವೆ. ಹಾಗೆ ಚಿಂತಿಸಿದರೆ ಅದರ ಮೇಲೆ ನಮಗೆ ಆಸಕ್ತಿ ಹುಟ್ಟುವುದು. ಅದನ್ನೇ ಪದೇ ಪದೇ ಚಿಂತಿಸುತ್ತೇವೆ. ಅದನ್ನು ಮತ್ತೂ ಆಳವಾಗಿ ಯೋಚಿಸುತ್ತೇವೆ. ಅದೊಂದು ನಮ್ಮ ಸ್ವಭಾವ ಆಗುತ್ತ ಬರುವುದು. ಅನಂತರ ಅದರ ಮೇಲೆ ನಮಗೆ ಕಾಮ ಹುಟ್ಟುವುದು. ಅದರ ಮೇಲೆ ಆಸೆ ಹುಟ್ಟುವುದು. ಆ ವಸ್ತುವನ್ನು ಪಡೆಯಬೇಕು, ಅದನ್ನು ಅನುಭವಿಸಬೇಕು, ಎಂದು ಮನಸ್ಸಾಗುವುದು. ಸುಮ್ಮನೆ ನಾವೊಂದು ನಾಯಿಯನ್ನು ಸಾಕುತ್ತೇವೆ. ಆ ನಾಯಿಯನ್ನು ನೋಡುತ್ತ ನೋಡುತ್ತ ನಮಗೆ ಅದರ ಮೇಲೆ ಆಸೆ ಹುಟ್ಟುವುದು, ಅದನ್ನು ಎತ್ತಿ ಮುದ್ದಿಸುವೆವು. ಒಮ್ಮೆ ಮುದ್ದಿಸಿದರೆ ಸಾಕು ನಾಯಿ ಬಾಲ ಅಲ್ಲಾಡಿಸುತ್ತ ನಮ್ಮ ಹತ್ತಿರ ನಮ್ಮ ಮುದ್ದಿಗಾಗಿ ಬಂದು ನಿಲ್ಲುವುದು. ನಾವು ಮುದ್ದಿಸುವುದು ತಡ ಮಾಡಿದರೆ ಅದೇ ನಮ್ಮ ಮೇಲೆ ಬೀಳುವುದು. ಇದಕ್ಕೆಲ್ಲ ಅವಕಾಶ ಕೊಟ್ಟವರು ನಾವು. ಯಾವಾಗ ನಮಗೆ ಒಂದು ವಸ್ತುವಿನ ಮೇಲೆ ಕಾಮವಿದೆಯೊ ಆಗ ಕೋಪ ಬರುವುದು. ನಾವು ಯಾವುದನ್ನು ಕಾಮಿಸುತ್ತೇವೆಯೋ ಅದು ನನಗೆ ಸೇರಿದ್ದು, ಅದನ್ನು ಇತರರು ಅನುಭವಿಸಬಾರದು, ಎಂದು ಅದರ ಸುತ್ತಲೂ ಒಂದು ಬೇಲಿ ಕಟ್ಟಿ ರಕ್ಷಿಸುವೆವು. ನಮಗೂ ಆ ವಸ್ತುವಿಗೂ ಮಧ್ಯೆ ಯಾರು ಬಂದರೂ ನಾಯಿಯಂತೆ ಬಗುಳುವೆವು, ಅವರನ್ನು ಕಚ್ಚುವುದಕ್ಕೆ ಹೋಗುವೆವು. ಅಷ್ಟು ಮಾತ್ರವಲ್ಲ. ನಾವು ರಕ್ಷಿಸುತ್ತಿರುವ ವಸ್ತುವೇ ನಮ್ಮನ್ನು ಪ್ರೀತಿಸದೇ ಇದ್ದರೆ ಆ ವಸ್ತುವನ್ನು ನಾಶಮಾಡಲೂ ಅನುಮಾನಿಸುವುದಿಲ್ಲ.

ಕ್ರೋಧದಿಂದ ನಮಗೆ ಮೋಹ ಹುಟ್ಟುವುದು. ಒಂದು ಪರಿಸ್ಥಿತಿಯನ್ನು ಸರಿಯಾಗಿ ನೋಡುವು ದಿಲ್ಲ. ಮೋಹ ವಸ್ತುವಿನ ಯಥಾರ್ಥ ಸ್ಥಿತಿಯನ್ನು ಮುಚ್ಚುವುದು. ಮಂಜು ಆವರಿಸಿದಾಗ ಅದು ಹೇಗೆ ಎಲ್ಲವನ್ನೂ ಮುಚ್ಚುವುದೊ ಹಾಗೆ. ಮೋಹ ಹುಟ್ಟಿದಾಗ ಸ್ಮೃತಿ ನಾಶವಾಗುವುದು. ನಮ್ಮ ಮನಸ್ಸಿನಲ್ಲಿ ಶಾಸ್ತ್ರಗಳು ಮತ್ತು ಹಿರಿಯರು ಹೇಳಿದ ಅನುಭವಗಳನ್ನು ಸಂಗ್ರಹಿಸಿ ಇಟ್ಟುಕೊಂಡಿರು ವೆವು. ಅದರ ಆಧಾರದ ಮೇಲೆ ಮುಂದಿನದನ್ನು ತರ್ಕಿಸುತ್ತೇವೆ, ನಿಷ್ಕರ್ಷಿಸುತ್ತೇವೆ. ಲಾಯರ್ ತನ್ನ ಹತ್ತಿರ ಇರುವ ರೆಫರೆನ್ಸ್ ಪುಸ್ತಕವನ್ನು ನೋಡಿ ತನ್ನ ಕೇಸಿನ ಪರವಾಗಿ ವಾದ ಮಾಡುವಂತೆ ಅದು. ಯಾವಾಗ ಸ್ಮೃತಿ ನಾಶವಾಗುವುದೊ ಅನಂತರ ನಮ್ಮ ಬುದ್ಧಿ ಪ್ರಯೋಜನಕ್ಕೆ ಬರುವುದಿಲ್ಲ. ಯಾವುದು ಸರಿ ಯಾವುದು ತಪ್ಪು ಎಂಬುದನ್ನು ನಿಷ್ಕರ್ಷಿಸುವ ಸ್ಥಿತಿ ನಮಗೆ ಇರುವುದಿಲ್ಲ. ಅನಂತರ ಅವನು ಆ ಪರಿಸ್ಥಿತಿಯಲ್ಲಿ ಯಾವುದನ್ನು ಮಾಡಬಾರದೊ ಅದನ್ನು ಮಾಡುತ್ತಾನೆ. ಬುದ್ಧಿ ಅದಕ್ಕೆ ಕಾರಣವನ್ನೂ ಕೊಡುತ್ತದೆ. ಆ ಬುದ್ಧಿ ನನಗೆ ಸತ್ಯವನ್ನು ತೋರುವ ಬದಲು ನನ್ನ ಇಚ್ಛೆಯನ್ನು ತೃಪ್ತಿಪಡಿಸುವ ದಾಸನಾಗಿದೆ. ಬೀದಿಯಲ್ಲಿ ಹೋಗುವ ಮಾರಿಯನ್ನು ಮನೆಗೆ ಬಂದು ಹೋಗು ಎನ್ನುವಂತೆ ಆಯಿತು ಇದು. ಸುಮ್ಮನೆ ಕೂತಿರಲಾರದೆ ಯಾವುದೋ ಆಲೋಚನೆಯನ್ನು ಮೆಲ್ಲುತ್ತೇವೆ. ಆ ಆಲೋಚನೆ ಪ್ರಬಲವಾಗಿ ಒಂದು ಕಾಮದ ರೂಪನ್ನು ಧರಿಸಿ ನಮ್ಮನ್ನು ಮೆಟ್ಟಿಕೊಳ್ಳುವುದು. ಮನೆಗೆ ಬಂದ ಮಾರಿಗೆ ಬಲಿ ಕೊಟ್ಟಲ್ಲದೆ ಹೋಗುವುದಿಲ್ಲ. ಅದರಂತೆಯೇ ಆ ಕಾಮದ ತೃಪ್ತಿಗೆ ಕರ್ಮವನ್ನು ಮಾಡಿ ಹಾಕುತ್ತೇವೆ. ಹಲವು ದಿನ ಕಷ್ಟಪಟ್ಟು ಸಂಗ್ರಹಿಸಿರುವುದನ್ನು ಒಂದು ದಿನದಲ್ಲಿ ಖರ್ಚುಮಾಡಿ ಮಾರನೆ ದಿನದಿಂದ ದಿವಾಳಿಯಾದಂತೆ ನಮ್ಮ ಪಾಡು.

\begin{shloka}
ರಾಗದ್ವೇಷವಿಯುಕ್ತೈಸ್ತು ವಿಷಯಾನಿಂದ್ರಿಯೈಶ್ಚರನ್~।\\ಆತ್ಮವಶ್ಯೈರ್ವಿಧೇಯಾತ್ಮಾ ಪ್ರಸಾದಮಧಿಗಚ್ಛತಿ \hfill॥ ೬೪~॥
\end{shloka}

\begin{artha}
ರಾಗದ್ವೇಷಗಳಿಂದ ಬಿಡುಗಡೆಯಾಗಿ ತನ್ನ ವಶದಲ್ಲಿರುವ ಇಂದ್ರಿಯಗಳಿಂದ ವಿಷಯವನ್ನು ಅನುಭವಿಸುತ್ತಿರುವ ವಿಧೇಯಾತ್ಮನು ಪ್ರಸನ್ನತೆಯನ್ನು ಹೊಂದುವನು.
\end{artha}

ಯಾರು ರಾಗದ್ವೇಷಗಳಿಂದ ಬಿಡುಗಡೆಯನ್ನು ಹೊಂದುವರೊ ಅವರ ಮನಸ್ಸು ಪ್ರಶಾಂತವಾಗಿರುವುದು. ಮನಸ್ಸಿನ ಸ್ವಾಸ್ಥ್ಯವನ್ನು ಸದಾ ಕೆಡಿಸುತ್ತಿರುವುದು ಈ ಎರಡು ಭಾವನೆಗಳು. ಒಂದೇ ರಾಗ. ನಮಗೆ ಇಷ್ಟವಾದ ವಸ್ತುವಿನ ಕಡೆ ಮನಸ್ಸು ಪದೇ ಪದೇ ಹೋಗುತ್ತಿರುವುದು. ಆ ವಸ್ತು ಹತ್ತಿರ ಇಲ್ಲದೇ ಇರಬಹುದು. ಆದರೂ ಅದನ್ನು ಕುರಿತು ಮೆಲುಕುತ್ತಿರುವುದು. ಆ ವಸ್ತು ದೂರ ಹೋದರೆ ಅದರ ಮೇಲಿರುವ ರಾಗವೇನೂ ಕಡಿಮೆಯಾಗುವುದಿಲ್ಲ. ರಬ್ಬರಿನಂತೆ ಅದು ಉದ್ದವಾಗುವುದು. ರಬ್ಬರಿಗೂ ಒಂದು ಮಿತಿ ಇದೆ. ಜಾಸ್ತಿ ಎಳೆದರೆ ಅದು ಕಡಿದುಹೋಗುವುದು. ಆದರೆ ಮನಸ್ಸು ಹಾಗಲ್ಲ. ಜಾಸ್ತಿ ಎಳೆದರೂ ವಿಸ್ತಾರವಾಗುವುದೇ ಹೊರತು ಹರಿದು ಹೋಗುವುದಿಲ್ಲ. ಅದರಂತೆಯೇ ದ್ವೇಷ ನಮ್ಮ ಮನಸ್ಸಿನ ಬಹುಪಾಲು ಶಕ್ತಿಯನ್ನು ವ್ಯಯಮಾಡುವುದು. ರಾಗದಿಂದ ನಮ್ಮ ಮನಸ್ಸಿಗೆ ಆಹ್ಲಾದವಾಗುವುದು, ದ್ವೇಷದಿಂದ ನಮ್ಮ ಮನಸ್ಸು ಕೆಡುವುದು. ಆ ವಸ್ತುವನ್ನೇ ನೆನೆದು ಅದನ್ನು ಶಪಿಸುವೆವು. ದ್ವೇಷಿಸುವ ವಸ್ತುವನ್ನೇ ನಾವು ಅನುಗಾಲವೂ ಚಿಂತಿಸುತ್ತಿರಬೇಕು. ಪ್ರೀತಿಗಿಂತ ದ್ವೇಷ ತೀವ್ರವಾದುದು. ಹಿಂದೆ ಜಯವಿಜಯರಿಗೆ ಶಾಪ ಕೊಟ್ಟಾಗ ದೂರ್ವಾಸರು, ನೀವು ಸ್ನೇಹಿತರಾಗಿ ಏಳು ಜನ್ಮಗಳು ಹುಟ್ಟುತ್ತೀರೋ ದ್ವೇಷಿಗಳಾಗಿ ಮೂರು ಜನ್ಮಗಳು ಹುಟ್ಟುತ್ತೀರೋ ಎಂದು ಕೇಳಿದಾಗ, ಬೇಗ ಹಿಂದಿರುಗಿ ಬರಬಹುದೆಂದು ದ್ವೇಷಿಗಳಾಗಿ ಮೂರು ಜನ್ಮ ಕಳೆದರು ಎಂದಿದೆ. ರಾಗ ದ್ವೇಷಗಳನ್ನು ಯಾವಾಗ ತಡೆಗಟ್ಟುವೆವೊ ಆಗ ಬೇಕಾದಷ್ಟು ಶಕ್ತಿ ವ್ಯಯವಾಗುವುದನ್ನು ನಿಲ್ಲಿಸಬಹುದು.

ಎಂತಹ ಸ್ಥಿತಪ್ರಜ್ಞನಾದರೂ ಪ್ರಪಂಚದಲ್ಲಿರುವ ತನಕ ಅವನು ದೇಹವನ್ನು ಪೋಷಣೆ ಮಾಡ ಬೇಕಾಗಿದೆ. ಅವನೂ ಊಟ ಮಾಡಬೇಕು, ನೋಡಬೇಕು, ನುಡಿಯಬೇಕು, ಬಟ್ಟೆಬರೆ ಧರಿಸಬೇಕು. ಆದರೆ ಹೀಗೆ ಮಾಡುವಾಗ ಅವನು ಇಂದ್ರಿಯಕ್ಕೆ ದಾಸನಾಗಿ ಮಾಡುವುದಿಲ್ಲ. ಬದ್ಧ ತಿನ್ನುವುದಕ್ಕೆ ಬದುಕಿರುವವನಂತೆ ಕಾಣುವನು. ಜ್ಞಾನಿ ಬದುಕಬೇಕಾದರೆ ತಿನ್ನಬೇಕು ಎನ್ನುವನು. ಅವನು ಪಂಚೇಂದ್ರಿಯಗಳನ್ನೆಲ್ಲ ಉಪಯೋಗಿಸುತ್ತ ಇರುವಾಗ ಯಜಮಾನನಂತೆ ಇರುವನು. ಆಳಿನಂತೆ ಇರುವುದಿಲ್ಲ.

ಇಂತಹ ತನ್ನ ಇಂದ್ರಿಯವನ್ನು ಗೆದ್ದವನು ಇಂದ್ರಿಯವನ್ನು ಧ್ವಂಸಮಾಡುವುದಿಲ್ಲ. ಅದನ್ನು ಬೇರೆ ದಾರಿಗೆ ತಿರುಗಿಸುತ್ತಾನೆ. ಬದ್ಧ ಅದನ್ನು ಒಂದು ರೀತಿ ಉಪಯೋಗಿಸಿ ದುಃಖಕ್ಕೆ ಸಿಕ್ಕಿಕೊಂಡ. ಮುಕ್ತ ಅದನ್ನು ಉಪಯೋಗಿಸುತ್ತಿರುವನು, ಆದರೂ ಅದರ ಬಂಧನಕ್ಕೆ ಸಿಕ್ಕಿಬಿದ್ದಿಲ್ಲ. ಇದು ಹಾವಾಡಿಸುವವನು ಹಾವಿನ ವಿಷದ ಹಲ್ಲನ್ನು ಕಿತ್ತು ಆಡಿಸುವ ಹಾಗೆ. ಯಾರು ಇಂದ್ರಿಯವನ್ನು ತಮ್ಮ ಸ್ವಾಧೀನದಲ್ಲಿಟ್ಟುಕೊಂಡಿರುವನೊ ಅವನ ಮನಸ್ಸು ಪ್ರಶಾಂತವಾಗಿರುವುದು. ಆ ಚಿತ್ತ ಸರೋವರಕ್ಕೆ ಹೊರಗಿನಿಂದ ಯಾವ ಕಲ್ಲೂ ಬಿದ್ದು ಸ್ವಾಸ್ಥ್ಯಕ್ಕೆ ಭಂಗ ತರುವುದಿಲ್ಲ. ಅವನು ಬೀಳುವುದನ್ನೆಲ್ಲ ಹೊರಗಡೆಯೇ ತಡೆಯುತ್ತಾನೆ. ಅವನು ಒಂದು ದೊಡ್ಡ ಸಂತೆಯ ಮಧ್ಯದಲ್ಲಿ ದ್ದರೂ ಪ್ರಶಾಂತಚಿತ್ತನಾಗಿರುವನು. ಇಂದ್ರಿಯ ಜಿತನಲ್ಲದವನು ಒಂದು ಪ್ರಶಾಂತ ಸ್ಥಳದಲ್ಲಿ ದ್ದರೂ ಮನಸ್ಸು ಅಲ್ಲೋಲ ಕಲ್ಲೋಲವಾಗಿರುವುದು.

\begin{shloka}
ಪ್ರಸಾದೇ ಸರ್ವದುಃಖಾನಾಂ ಹಾನಿರಸ್ಯೋಪಜಾಯತೇ~।\\ಪ್ರಸನ್ನಚೇತಸೋ ಹ್ಯಾಶು ಬುದ್ಧಿಃ ಪರ್ಯವತಿಷ್ಠತೇ \hfill॥ ೬೫~॥
\end{shloka}

\begin{artha}
ಮನಸ್ಸು ಪ್ರಶಾಂತವಾದರೆ ಸರ್ವದುಃಖಗಳ ಹಾನಿ ಉಂಟೋಗುವುದು. ಚಿತ್ತಶುದ್ಧವಾದವನ ಬುದ್ಧಿ ಬೇಗ ಸ್ಥಿರವಾಗುವುದು.
\end{artha}

ಎಲ್ಲ ದುಃಖಕ್ಕೆ ಕಾರಣ ಹೊರಗಡೆಯಿಂದ ಬರುವ ಘಟನೆಗಳು ನಮ್ಮ ಮನಸ್ಸನ್ನು ಕಲಕುವುದರಿಂದ. ಪ್ರಪಂಚದಲ್ಲಿ ನನ್ನಿಂದ ಹೊರಗಡೆ ಆಗುವ ಘಟನೆಗಳನ್ನು ತಡೆಗಟ್ಟಲು ಅಥವಾ ನನ್ನ ಇಚ್ಛೆಯಂತೆ ಆಗುವಂತೆ ಮಾಡಲು ನಮಗೆ ಸಾಧ್ಯವಿಲ್ಲ. ಜ್ಞಾನಿಯಾದವನು ಅದನ್ನು ನೋಡುವ ದೃಷ್ಟಿಯನ್ನು ಬದಲಾಯಿಸುವನು. ಅವನು ಯಾವುದಕ್ಕೂ ಆಸಕ್ತನಲ್ಲ. ಏನಾದರೂ ಅವನಿಗೆ ಲಾಭವಿಲ್ಲ, ನಷ್ಟವೂ ಇಲ್ಲ. ಬಾಹ್ಯ ಪ್ರಪಂಚ ಎಸೆಯುವ ಕೂಳಿಗೆ ಕಾದು ನಿಂತವನು ಅದು ಹೇಳಿದಂತೆ ಕೇಳಬೇಕು. ಯಾವ ಬಾಹ್ಯಘಟನೆಯೂ ಅವನ ಮನಸ್ಸಿನ ಅಂತರಾಳವನ್ನು ಮುಟ್ಟಲಾರದು. ಸುಖದುಃಖಗಳನ್ನು ತರುವ ಮುಳ್ಳುಗಳು ಹೊರಗಡೆಯೆಲ್ಲ ಬಿದ್ದಿದೆ. ಆದರೆ ಅವನು ಕಾಲಿಗೆ ಎಕ್ಕಡವನ್ನು ಹಾಕಿಕೊಂಡು ನಡೆಯುವುದರಿಂದ ಅವುಗಳಿಂದ ಬಾಧಿತನಾಗುವುದಿಲ್ಲ. ಯಾರ ಮನಸ್ಸು ಶುದ್ಧವಾಗಿದೆಯೊ ಅವರ ಬುದ್ಧಿ ವಸ್ತುವಿನ ನೈಜಸ್ಥಿತಿಯನ್ನು ಅರಿಯುವ ಸ್ಥಿತಿಗೆ ಬರುವುದು. ವಸ್ತು ಹೇಗಿದೆಯೊ ಹಾಗೆ ತಿಳಿದುಕೊಳ್ಳುವವರು ಅಪರೂಪ. ತಮಗೆ ಹೇಗೆ ಕಾಣುತ್ತದೆಯೊ ಅದೇ ಸತ್ಯವೆಂದು ಭಾವಿಸುವರು. ಆದರೆ ಜ್ಞಾನಿಯಾದರೋ ತನ್ನ ಚಿತ್ತವನ್ನು ಶುದ್ಧಿಮಾಡಿಕೊಂಡಿರುವನು. ಅವನಿಗೆ ಬೇಕಾಗಿರುವುದು ಪ್ರಿಯವಾಗಿರುವುದೂ ಅಲ್ಲ, ಅಪ್ರಿಯವಾಗಿರುವುದೂ ಅಲ್ಲ, ಆದರೆ ಹಿಂದೆ ಇರುವ ಸತ್ಯದ ಸ್ವರೂಪ. ಇದನ್ನು ಅರಿಯಬೇಕಾದರೆ ಸ್ಥಿರವಾದ ಏಕಮುಖವಾದ ಬುದ್ಧಿಬೇಕು.

\begin{shloka}
ನಾಸ್ತಿ ಬುದ್ಧಿರಯುಕ್ತಸ್ಯ ನ ಚಾಯುಕ್ತಸ್ಯ ಭಾವನಾ~।\\ನ ಚಾಭಾವಯತಃ ಶಾಂತಿರಶಾಂತಸ್ಯ ಕುತಃ ಸುಖಮ್ \hfill॥ ೬೬~॥
\end{shloka}

\begin{artha}
ಅಯುಕ್ತನಾದವನಿಗೆ ಬುದ್ಧಿ ಇಲ್ಲ. ಅಯುಕ್ತನಾದವನಿಗೆ ಭಾವನೆಯೂ ಇಲ್ಲ. ಭಾವನೆ ಮಾಡದವನಿಗೆ ಶಾಂತಿ ಇಲ್ಲ. ಶಾಂತಿ ಇಲ್ಲದವನಿಗೆ ಸುಖವೆಲ್ಲಿ?
\end{artha}

ಯಾರು ತನ್ನ ಇಂದ್ರಿಯವನ್ನು ನಿಗ್ರಹಿಸಿಲ್ಲವೊ ಅವನ ಬುದ್ಧಿ ಸರಿಯಾಗಿರುವುದಿಲ್ಲ. ಎಲ್ಲಿಯವರೆಗೆ ಆಸೆ ಆಕಾಂಕ್ಷೆಗಳು ಬುದ್ಧಿಯ ಹಿಂದೆ ಇರುವುವೋ ಅವನ್ನು ತೃಪ್ತಿಪಡಿಸಿಕೊಳ್ಳುವುದಕ್ಕೆ ಆ ಬುದ್ಧಿ ಯತ್ನಿಸುವುದೇ ಹೊರತು ಸತ್ಯವನ್ನು ತಿಳಿದುಕೊಳ್ಳುವುದಕ್ಕೆ ಯತ್ನಿಸುವುದಿಲ್ಲ. ಯಾರು ತಮ್ಮ ಇಂದ್ರಿಯವನ್ನು ನಿಗ್ರಹಿಸಿಲ್ಲವೊ ಅವನ ಮನಸ್ಸಿನಲ್ಲಿ ಒಳ್ಳೆಯ ಭಾವನೆಗಳೂ ಇರುವುದಿಲ್ಲ. ಒಳ್ಳೆಯ ಭಾವನೆ ಎಂಬುದು ಸುಗಂಧ ಪುಷ್ಪಗಳಂತೆ. ಅದು ಸುತ್ತಲೂ ತನ್ನ ಪರಾಗವನ್ನು ಬೀರುತ್ತಿರುವುದು. ಅಂತಹ ಒಳ್ಳೆಯ ಭಾವನೆಗಳಲ್ಲಿ ಕೆಲವೇ ಇವು: ಮೈತ್ರಿ, ಕರುಣ, ಮುದಿತ, ಉಪೇಕ್ಷೆ. ಅವನ ಹೃದಯದಲ್ಲಿ ಮೈತ್ರೀ ಭಾವನೆ ನೋಡುತ್ತೇವೆ. ಎಲ್ಲರನ್ನೂ ಪ್ರೀತಿಸುವನು. ಪ್ರೀತಿಯೆ ಅವನ ಸಹಜ ಸ್ವಭಾವವಾಗಿರುವುದು. ಇವರು ನನ್ನ ಬಂಧು ಬಳಗ ಎಂದಲ್ಲ. ಎಲ್ಲರೂ ದೇವರಿಗೆ ಸೇರಿದವರಾದುದರಿಂದ ಅವರೆಲ್ಲರೂ ತನ್ನವರು ಎಂದು ನೋಡುವನು. ಯಾರು ಇಂತಹ ಪ್ರೀತಿಯ ಸ್ಪಂದನವನ್ನು ವ್ಯಕ್ತಗೊಳಿಸುತ್ತಿರುವನೊ ಅವನ ಬಳಿ ದ್ವೇಷ ಇರಲಾರದು. ಅವನು ಯಾರು ದುಃಖ ಕಷ್ಟದಲ್ಲಿ ಸಿಕ್ಕಿದ್ದಾರೆ ಅವರಿಗೆಲ್ಲ ಮರುಗುವನು. ದೇವರು ಅವನಿಗೆ ಒಳ್ಳೆಯದನ್ನು ಮಾಡಲಿ ಎಂದು ಆಶಿಸುವನು. ಯಾರು ಒಳ್ಳೆಯವರೊ ಅವರನ್ನು ಕೊಂಡಾಡುವನು. ಅವರಿಗೆ ಗೌರವ ಕೊಡುವನು, ಅವರನ್ನು ನೋಡಿ ಆನಂದಪಡುವನು. ಭಗವಂತನ ಹೂದೋಟದಲ್ಲಿ ವಿಕಸಿತವಾದ ಸುಂದರ ಪುಷ್ಪಗಳು ಇವು ಎಂದು ಭಾವಿಸುವನು. ಅದರಂತೆಯೇ ತನ್ನನ್ನು ಯಾರಾದರೂ ಟೀಕಿಸಿದರೆ, ತನಗೆ ತೊಂದರೆ ಕೊಡಲು ಬಂದರೆ, ಅವರ ಮೇಲೆ ಪ್ರತೀಕಾರವನ್ನು ತೀರಿಸಿಕೊಳ್ಳುವುದಕ್ಕೆ ಹೋಗುವುದಿಲ್ಲ. ಅದನ್ನೇ ಉದಾಸೀನದಿಂದ ನೋಡುವನು. ಈ ಪ್ರಪಂಚದಲ್ಲಿ ನೋಡಲು ಅಂದವಾದ ಪರಿಮಳವನ್ನು ಬೀರುವ ಹೂಗಳು ಹೇಗೆ ಇರುವುವೋ ಹಾಗೆಯೇ ಮುಟ್ಟಿದರೆ ಮುಳ್ಳು ಕೈಗೆ ಅಂಟಿಕೊಳ್ಳುವ ಪಾಪಾಸು ಕಳ್ಳಿಯ ಹೂಗಳೂ ಇವೆ. ಅವನ್ನು ಇವನು ದೂರುವುದಕ್ಕೆ ಹೋಗುವುದಿಲ್ಲ. ಆದರೆ ಅದನ್ನು ಲಕ್ಷ್ಯದಲ್ಲಿಡುವುದಿಲ್ಲ.

ಯಾವ ಹೃದಯದಲ್ಲಿ ಒಳ್ಳೆಯ ಭಾವನೆ ಇಲ್ಲವೊ ಅಲ್ಲಿ ಶಾಂತಿಯಿಲ್ಲ. ಇತರರನ್ನು ನೋಡಿ ಕರುಬುತ್ತಿದ್ದರೆ, ಇತರರಿಗೆ ಅನ್ಯಾಯಮಾಡಿ ನಾವು ಸುಖವನ್ನು ಅನುಭವಿಸುತ್ತಿದ್ದರೆ, ನಮಗೆ ಇತರರಿಂದ ಆದ ವ್ಯಥೆ ಮತ್ತು ಅನ್ಯಾಯವನ್ನೇ ಮೆಲಕುತ್ತಿದ್ದರೆ, ಅಂತಹ ಬಾಳಿಗೆ ಶಾಂತಿಯಾದರೂ ಹೇಗೆ ಬರಬೇಕು? ನಮ್ಮ ಮನಸ್ಸಿನಿಂದ ಯಾವ ಭಾವನೆಯನ್ನು ಕಳುಹಿಸುತ್ತೇವೆಯೋ ಅದೇ ಒಂದಾಗಿ ನೂರಾಗಿ ನಮಗೆ ಹಿಂದಿರುಗಿ ಬರುವುದು. ಎಲ್ಲರೂ ಕಷ್ಟದಿಂದ ಪಾರಾಗಲಿ, ಎಲ್ಲರೂ ಶುಭವನ್ನು ನೋಡಲಿ, ದೇವರು ಎಲ್ಲರಿಗೂ ಸದ್​ಬುದ್ಧಿಯನ್ನು ನೀಡಲಿ, ಎಲ್ಲರೂ ಎಲ್ಲಾ ಬಂಧನದಿಂದ ಪಾರಾಗಲಿ ಎಂದು ಆಶಿಸುತ್ತಿದ್ದರೆ ಅಂತಹ ಒಳ್ಳೆಯ ಭಾವನೆಗಳೇ ಇವನಿಗೆ ಬರುವುವು. ನಾವು ಕೊಟ್ಟಿದ್ದೇ ನಮಗೆ ಸಿಕ್ಕುವುದು ಜೀವನದ ಗಾಢ ನಿಯಮ. ಇತರರಿಗೆ ನಾವು ಅಶಾಂತಿಯನ್ನು ಕೊಟ್ಟಿದ್ದರೆ ನಮಗೆ ಶಾಂತಿ ಹೇಗೆ ಬರಬೇಕು?

ಜೀವನದಲ್ಲಿ ಯಾವಾಗ ಶಾಂತಿ ಇಲ್ಲವೊ ಆಗ ಎಷ್ಟೊಂದು ಪ್ರಪಂಚದ ಸೌಲಭ್ಯಗಳು ಇದ್ದರೇನು? ಮಲಗಲು ಹಂಸತೂಲಿಕಾತಲ್ಪ, ಅರಮನೆ, ಕಾರು, ಮೃಷ್ಟಾನ್ನ ಎಲ್ಲ ಇರಬಹುದು. ಆದರೆ ಇದರಿಂದ ನಮಗೆ ಶಾಂತಿ ಸಿಕ್ಕುವುದೇನು? ಅದಕ್ಕೇ ತ್ಯಾಗರಾಜರು “ಶಾಂತಮು ಲೇಕ ಸೌಖ್ಯಮು ಲೇದು” ಎಂದು ಹಾಡಿದರು.

\begin{shloka}
ಇಂದ್ರಿಯಾಣಾಂ ಹಿ ಚರತಾಂ ಯನ್ಮನೋಽನುವಿಧೀಯತೇ~।\\ತದಸ್ಯ ಹರತಿ ಪ್ರಜ್ಞಾಂ ವಾಯುರ್ನಾವಮಿವಾಂಭಸಿ \hfill॥ ೬೭~॥
\end{shloka}

\begin{artha}
ನಿಗ್ರಹಿಸದ ಇಂದ್ರಿಯದ ಹಿಂದೆ ಮನಸ್ಸನ್ನು ಬಿಡುವುದು, ಗಾಳಿ ನೀರಿನಲ್ಲಿರುವ ಹಡಗನ್ನು ಒಯ್ಯುವಂತೆ ಇವನ ಪ್ರಜ್ಞೆಯನ್ನು ನಾಶಮಾಡುವುದು.
\end{artha}

ಯಾವಾಗ ನಾವು ಇಂದ್ರಿಯವನ್ನು ನಿಗ್ರಹಿಸದೆ ಅದು ಎಳೆದತ್ತ ಹೋಗುವೆವೊ ಆಗ ನಾವು ನಾಶವಾಗುತ್ತೇವೆ. ಇಂದ್ರಿಯ ತನಗೆ ಮುಂದಾಗುವುದನ್ನು ಯೋಚಿಸುವುದಿಲ್ಲ. ತತ್ಕಾಲಕ್ಕೆ ಯಾವುದು ಪ್ರಿಯವಾಗಿರುವುದೊ ಅದನ್ನು ಮಾತ್ರ ನೋಡುತ್ತದೆ. ಮೊದಲು ಸುಖಪಟ್ಟು ಅನಂತರ ಎಂದೆಂ ದಿಗೂ ಸಹಿಸಲಾರದ ವ್ಯಾಧಿಯಿಂದ ನರಳಬೇಕಾಗುತ್ತದೆ. ಗಾಳಿ ತಳ್ಳಿದತ್ತ ದೋಣಿಯನ್ನು ಬಿಡುತ್ತಿ ದ್ದರೆ ಅದು ಯಾವುದೋ ಬಂಡೆಗೆ ಹೊಡೆದು ಮುಳುಗಿಹೋಗುವುದು. ಒಳ್ಳೆಯ ನಾವಿಕ ಬೀಸುವ ಗಾಳಿಯಿಂದ ಪ್ರಯೋಜನ ಪಡೆಯುತ್ತಾನೆಯೆ ಹೊರತು, ಅದು ಹೇಳಿದಂತೆ ಕೇಳುವುದಿಲ್ಲ. ಗಾಳಿಯನ್ನು ತನಗೆ ಹೇಗೆ ಬೇಕೊ ಹಾಗೆ ಉಪಯೋಗಿಸಿಕೊಳ್ಳುತ್ತಾನೆ. ಇಂದ್ರಿಯ ಹೇಳಿದಂತೆ ಕೇಳುವವನ ಬುದ್ಧಿ ನಾಶವಾಗುತ್ತದೆ. ಈ ದುರ್ಬುದ್ಧಿ ನಮಗೆ ಹೇಗೆ ಮಾರ್ಗದರ್ಶನ ಮಾಡ ಬಲ್ಲುದು? ಶ‍್ರೀರಾಮಕೃಷ್ಣರು ಒಂದು ಸುಂದರವಾದ ಉಪಮಾನವನ್ನು ಕೊಡುತ್ತಿದ್ದರು. ಯಾರಿ ಗಾದರೂ ದೆವ್ವ ಮೆಟ್ಟಿಕೊಂಡರೆ ಅವರ ಮೇಲೆ ಮಂತ್ರಾಕ್ಷತೆಗಳನ್ನು ಎರಚಿ ದೆವ್ವ ಬಿಡಿಸುತ್ತಾರೆ. ಆದರೆ ಮಂತ್ರಾಕ್ಷತೆ ಒಳಗೇ ದೆವ್ವ ಮೆಟ್ಟಿಕೊಂಡಿದ್ದರೆ ಅದನ್ನು ಬಿಡಿಸುವುದು ಹೇಗೆ? ಬುದ್ಧಿಯನ್ನು ಉಪಯೋಗಿಸಿ ನಾವು ಕಷ್ಟದಿಂದ ಪಾರಾಗಲು ಯತ್ನಿಸುತ್ತೇವೆ. ಆದರೆ ಆ ಬುದ್ಧಿಯೇ ಕುಲಗೆಟ್ಟಿದ್ದರೆ ಇನ್ನು ನಾವು ಪಾರಾಗುವುದು ಹೇಗೆ?\\

\begin{shloka}
ತಸ್ಮಾದ್ಯಸ್ಯ ಮಹಾಬಾಹೋ ನಿಗೃಹೀತಾನಿ ಸರ್ವಶಃ~।\\ಇಂದ್ರಿಯಾಣೀಂದ್ರಿಯಾರ್ಥೇಭ್ಯಸ್ತಸ್ಯ ಪ್ರಜ್ಞಾ ಪ್ರತಿಷ್ಠಿತಾ \hfill॥ ೬೮~॥
\end{shloka}

\begin{artha}
ಮಹಾಬಾಹುವೆ, ಯಾರು ತನ್ನ ಇಂದ್ರಿಯಗಳನ್ನು ಎಲ್ಲಾ ವಿಧದಿಂದಲೂ ವಿಷಯವಸ್ತುಗಳಿಂದ ಸೆಳೆದು ನಿಗ್ರಹಿಸುತ್ತಾನೆಯೊ ಅವನ ಪ್ರಜ್ಞೆ ಸ್ಥಿರವಾಗಿರುವುದು.
\end{artha}

ಆಧ್ಯಾತ್ಮಿಕ ಅನುಭವ ನಿಂತಿರುವುದೆ ಶುದ್ಧ ಜೀವನದ ಭದ್ರ ತಳಹದಿಯ ಮೇಲೆ. ಯಾವಾಗ ಶುದ್ಧ ಚಾರಿತ್ರವಿಲ್ಲವೊ ಅವನು ತನಗೆ ಏನೋ ಕಾಣುತ್ತಿದೆ ಎನ್ನಬಹುದು. ಅದೊಂದು ಕಲ್ಪನೆ ಇರಬಹುದು, ಸುಳ್ಳಾಗಿರಬಹುದು. ಅದನ್ನು ನಿಜ ಎಂದು ನಂಬುವುದಕ್ಕೆ ಆಗುವುದಿಲ್ಲ. ಆದ ಕಾರಣವೇ ಇಂದ್ರಿಯಗಳನ್ನು ಅದರ ವಸ್ತುಗಳಿಂದ ಎಲ್ಲಾ ವಿಧದಿಂದಲೂ ಸೆಳೆಯಬೇಕು ಎನ್ನುವುದು. ಪ್ರತಿಯೊಂದು ಇಂದ್ರಿಯವೂ ತನಗೆ ಸಂಬಂಧಪಟ್ಟ ವಿಷಯವಸ್ತುವಿನ ಕಡೆ ಹರಿದುಹೋಗಲು ಇಚ್ಛೆಪಡುವುದು. ಕಣ್ಣು ಸುಂದರವಾದ ಬಣ್ಣಗಳನ್ನು ರೂಪುಗಳನ್ನು ನೋಡಲು ತವಕಪಡುವುದು. ಕಿವಿ ಇಂಪಾದ ಧ್ವನಿಯನ್ನು ಕೇಳಲು ತವಕಪಡುವುದು. ಹಾಗೆಯೇ ಇತರ ಇಂದ್ರಿಯಗಳು. ನಮ್ಮಲ್ಲಿರುವ ಇಂದ್ರಿಯ ಅನುಭವಕ್ಕೆ ಬೇರು ಇಂದ್ರಿಯಕ್ಕೆ ಸಂಬಂಧಪಟ್ಟ ವಸ್ತುವಿನಲ್ಲಿರುವ ಹಾಗೆ ತೋರುವುದು. ಯಾವಾಗಲೂ ಅಲ್ಲಿಂದ ತನ್ನ ಪೋಷಣೆಗೆ ಸಾರವನ್ನು ಹೀರಿಕೊಳ್ಳುತ್ತಿರಬೇಕು. ಅದಕ್ಕಾಗಿ ಅದು ಪುನಃ ಪುನಃ ಆ ಇಂದ್ರಿಯಕ್ಕೆ ಸಂಬಂಧಪಟ್ಟ ಕ್ರಿಯೆಯನ್ನು ಮಾಡುವುದು. ಈ ಕ್ರಿಯೆಯಲ್ಲಿ ಎರಡು ವಿಧಗಳಿವೆ. ಒಂದು ಸ್ಥೂಲ, ಎಲ್ಲರ ಕಣ್ಣಿಗೂ ಬೀಳುವಂತೆ ಮಾಡುವುದು. ಮತ್ತೊಂದು ಸೂಕ್ಷ ್ಮ, ಅದನ್ನು ತನ್ನ ಮನಸ್ಸಿನಲ್ಲಿ ಮಾಡಿಕೊಳ್ಳುವುದು. ಇದು ಇತರರಿಗೆ ಕಾಣದೇ ಇರಬಹುದು. ಈ ಸೂಕ್ಷ ್ಮ ಕ್ರಿಯೆಯೂ ನನ್ನ ಮನಸ್ಸಿನ ಮೇಲೆ ತನ್ನ ಪ್ರಭಾವವನ್ನು ಬಿಡುವುದು. ಇದೂ ಕೂಡ ನನ್ನ ಮನಸ್ಸನ್ನು ಮೈಲಿಗೆ ಮಾಡುವುದು. ಅದೊಂದೆ ಅಲ್ಲ. ಸೂಕ್ಷ ್ಮದಲ್ಲಿ ನಾವು ಅದನ್ನು ನಿಗ್ರಹಿಸದೆ ಇದ್ದರೆ ಕ್ರಮೇಣ ಅದು ಸ್ಥೂಲರೂಪವಾಗಿಯೂ ವ್ಯಕ್ತವಾಗುವುದು. ಆದ ಕಾರಣವೇ ಸ್ಥೂಲ ಮತ್ತು ಸೂಕ್ಷ ್ಮವಾಗಿ ಎರಡು ವಿಧದಲ್ಲಿಯೂ ಆ ಕ್ರಿಯೆಯನ್ನು ನಿಲ್ಲಿಸಬೇಕು ಎನ್ನುವುದು. ಸ್ಥೂಲ ಸೂಕ್ಷ ್ಮ; ಇವೆರಡೂ ಒಂದು ನಾಣ್ಯದ ಎರಡು ಕಡೆಗಳಂತೆ ಇವೆ. ಸ್ಥೂಲವಾಗಿ ಕೆಲಸ ಮಾಡಿದರೆ, ಅದರ ಸಂಸ್ಕಾರ ನಮ್ಮಲ್ಲಿ ಬಿಡುವುದು. ಕೆಲವು ಕಾಲವಾದ ಮೇಲೆ ಪುನಃ ಪುನಃ ಆ ಸಂಸ್ಕಾರ ನಮ್ಮಲ್ಲಿರುವುದರಿಂದ ನಿರ್ವಾಹವಿಲ್ಲದೆ ಆ ಕ್ರಿಯೆಯನ್ನು ನಾವು ಮಾಡಬೇಕಾಗುವುದು. ಪ್ರತಿಯೊಂದು ಸಲ ಆ ಕ್ರಿಯೆಯನ್ನು ಮಾಡಿದಾಗಲೂ ನಮಗೆ ಅದರ ಮೇಲೆ ಜುಗುಪ್ಸೆ ಹುಟ್ಟುವು ದಿಲ್ಲ. ಪುನಃ ಇನ್ನೊಂದು ಸಲ ಹಾಗೆ ಮಾಡುವುದಕ್ಕೆ ಅಣಿಯಾಗುವುದು ಮನಸ್ಸು. ಬಯಕೆಗಳನ್ನು ತೃಪ್ತಿಪಡಿಸಿ ಯಾರೂ ಅದರ ಸೆಳೆತದಿಂದ ಪಾರಾಗಿಲ್ಲ. ಅದನ್ನು ನಿಗ್ರಹಿಸಿದರೆ ಮಾತ್ರ ಸಾಧ್ಯ. ಸೂಕ್ಷ್ಮವಾಗಿ ನಮ್ಮ ಮನಸ್ಸಿನಲ್ಲೆ ಒಂದು ವೈಷಯಿಕ ಅನುಭವವನ್ನು ಕುರಿತು ಚಿಂತಿಸುವುದು ಕೂಡ ಹಾಗೆಯೆ. ಒಂದು ಸಲ ಅದನ್ನು ಮಾಡುವುದು ಅಭ್ಯಾಸವಾದರೆ ಪದೇ ಪದೇ ಅದನ್ನು ಮನಸ್ಸು ಮಾಡುತ್ತಿರುವುದು. ಆ ಅನುಭವದ ಗೂಟಕ್ಕೆ ಮನಸ್ಸನ್ನು ಕಟ್ಟಿದಂತೆ ಆಗುವುದು. ಅದರ ಸುತ್ತಲೂ ಬರಬಹುದೆ ಹೊರತು ಅದನ್ನು ಕಿತ್ತುಕೊಂಡು ಹೋಗಲು ಸಾಧ್ಯವಿಲ್ಲ. ಸೂಕ್ಷ್ಮದ ಒತ್ತಡ ಜಾಸ್ತಿಯಾದಾಗ ಅದೇ ಸ್ಥೂಲವಾಗಿ ನಮ್ಮ ಮೂಲಕ ಆ ಕ್ರಿಯೆ ಬಲಾತ್ಕಾರವಾಗಿ ಆಗುವಂತೆ ಮಾಡುವುದು. ಆದಕಾರಣವೇ ಮನಸ್ಸನ್ನು ಎಲ್ಲಾ ವಿಧದಿಂದಲೂ ಆಯಾ ವಿಷಯವಸ್ತುಗಳಿಂದ ಸೆಳೆಯಬೇಕು ಎನ್ನುವುದು.

ಯಾರು ತನ್ನ ಮನಸ್ಸನ್ನು ಚೆನ್ನಾಗಿ ನಿಗ್ರಹಿಸುವನೊ ಅವನ ಪ್ರಜ್ಞೆ ಮಾತ್ರ ಸ್ಥಿರವಾಗಿರಬಹುದು. ಅದಕ್ಕೆ ಭದ್ರವಾದ ತಳಹದಿ ಸಿಕ್ಕಿದೆ. ಒಳ್ಳೆಯ ಕಲ್ಲು ಬಂಡೆಯ ತಳಪಾಯದ ಮೇಲೆ ಮನೆಯನ್ನು ಕಟ್ಟಿದಂತೆ ಇದು. ಯಾವಾಗ ತಳಪಾಯವೇ ಸರಿ ಇಲ್ಲವೋ ಆಗ ಸ್ವಲ್ಪ ಜೋರಾಗಿ ಮಳೆ ಬಂದರೆ ಸಾಕು; ಮನೆಯೆಲ್ಲ ಕುಸಿದು ಬೀಳುವುದು. ನಾವು ಶೀಲದ ಮೇಲೆ ಗಮನ ಕೊಡದೆ ಆಧ್ಯಾತ್ಮಿಕ ಅನುಭವಗಳನ್ನು ಪಡೆಯಬೇಕೆಂದು ನೇರವಾಗಿ ಅದರ ಕಡೆ ಧಾವಿಸುವೆವು. ಇದರಿಂದ ಕೇವಲ ಉದ್ವೇಗ ಪರವಶತೆ ಆಗುವುದೇ ಹೊರತು ಶಾಶ್ವತವಾದ ಅನುಭವ ಸಿಕ್ಕುವುದಿಲ್ಲ. ಒಬ್ಬ ಎಷ್ಟು ದೂರ ಆಧ್ಯಾತ್ಮಿಕ ಜೀವನದಲ್ಲಿ ಹೋಗಿರುವನು ಎಂಬುದನ್ನು ತಿಳಿದುಕೊಳ್ಳುವುದಕ್ಕೆ ಇರುವ ಏಕಮಾತ್ರ ಪ್ರಮಾಣವೇ ಅವನೆಷ್ಟು ಇಂದ್ರಿಯವನ್ನು ನಿಗ್ರಹಿಸಿರುವನು ಎಂಬುದು. ಪರಮಾತ್ಮನ ಆಧಾರ ಮನಸ್ಸಿಗೆ ಸಿಕ್ಕಿದರೆ ಇಂದ್ರಿಯದ ಕಡೆ ಬರುವುದನ್ನು ಬಿಡುವುದು. ಇಂದ್ರಿಯದ ಕಡೆ ಹೋಗುವ ಮನಸ್ಸನ್ನು ತಡೆಗಟ್ಟಿದರೇನೆ ಪರಮಾತ್ಮನ ಕಡೆ ಹೋಗುವುದಕ್ಕೆ ಸಾಧ್ಯವಾಗುವುದು. ಒಬ್ಬ ಇಂದ್ರಿಯಸುಖವನ್ನು ರುಚಿ ನೋಡಿ, ದೇವರನ್ನೂ ಏಕ ಕಾಲದಲ್ಲಿ ರುಚಿ ನೋಡುವುದಕ್ಕೆ ಆಗುವು ದಿಲ್ಲ. ಒಂದಕ್ಕೆ ಮತ್ತೊಂದನ್ನು ಬಲಿ ಕೊಡಬೇಕು. ದೇವರು ಬೇಕಾದರೆ ಇಂದ್ರಿಯದ ಸುಖವನ್ನು ತ್ಯಜಿಸಬೇಕು. ಇಂದ್ರಿಯ ಸುಖ ಬೇಕಾದರೆ ದೇವರನ್ನು ಮರೆಯಬೇಕು. ಕತ್ತಲೆ—ಬೆಳಕು, ರಾಮ— ಕಾಮ ಎರಡೂ ಏಕಕಾಲದಲ್ಲಿ ಇರುವುದಕ್ಕೆ ಸಾಧ್ಯವಿಲ್ಲ.

\begin{shloka}
ಯಾ ನಿಶಾ ಸರ್ವಭೂತಾನಾಂ ತಸ್ಯಾಂ ಜಾಗರ್ತಿ ಸಂಯಮೀ~।\\ಯಸ್ಯಾಂ ಜಾಗ್ರತಿ ಭೂತಾನಿ ಸಾ ನಿಶಾ ಪಶ್ಯತೋ ಮುನೇಃ \hfill॥ ೬೯~॥
\end{shloka}

\begin{artha}
ಎಲ್ಲಾ ಪ್ರಾಣಿಗಳಿಗೆ ಯಾವುದು ರಾತ್ರಿಯೊ ಅಲ್ಲಿ ಸಂಯಮಿ ಎಚ್ಚೆತ್ತಿರುವನು. ಎಲ್ಲಿ ಪ್ರಾಣಿಗಳು ಎಚ್ಚೆತ್ತಿರು ವುವೊ ಅದು ನೋಡತಕ್ಕ ಮುನಿಗೆ ರಾತ್ರಿಯಾಗಿರುವುದು.
\end{artha}

ಎಲ್ಲಾ ಪ್ರಾಣಿಗಳಿಗೆ ಯಾವುದು ರಾತ್ರಿಯೊ ಎಂದರೆ ಏನೂ ಕಾಣುವುದಿಲ್ಲವೊ, ಎಲ್ಲಿ ಅವರು ನಿದ್ರಿಸುತ್ತಿರುವರೊ, ಆ ಸಮಯದಲ್ಲಿ ಜ್ಞಾನಿ ಎಚ್ಚೆತ್ತಿರುವನು. ಜ್ಞಾನಿ ಅಜ್ಞಾನದ ಅಂಧಕಾರದಿಂದ ಪಾರಾಗಿರುವನು. ಯಾವುದು ಸಾಧಾರಣ ಮನುಷ್ಯನ ಪಾಲಿಗೆ ಅನಾವಶ್ಯವಾಗಿದೆಯೊ ಎಂದರೆ, ದೇವರು ಸತ್ಯ ಆಧ್ಯಾತ್ಮಿಕ ಜೀವನ ಇಂತಹವುಗಳ ಆಧಾರದ ಮೇಲೆ ನಿಂತಿಲ್ಲ, ಯಾವುದನ್ನು ಅವನು ಪಂಚೇಂದ್ರಿಯಗಳ ಮೂಲಕ ಹಿಡಿದುಕೊಳ್ಳಬಲ್ಲನೊ, ಅನುಭವಿಸಬಲ್ಲನೊ, ಅದನ್ನು ಮಾತ್ರ ನಂಬುತ್ತಾನೆ. ಎಲ್ಲಿ ಇವನಿಗೆ ಇದು ಹಾಗೆ ಅನುಭವಿಸುವುದಕ್ಕೆ ಸಾಧ್ಯವಿಲ್ಲವೊ ಅದು ಇವನ ಪಾಲಿಗೆ ಇಲ್ಲ. ಇವನು ಇದನ್ನು ಗಣನೆಗೇ ತೆಗೆದುಕೊಳ್ಳುವುದಿಲ್ಲ. ಇವನಿಗೆ ಇದು ರಾತ್ರಿಯಂತೆ. ಆದರೆ ಇವನಿಗೆ ಕಾಣಿಸದೇ ಇದ್ದರೂ ಜ್ಞಾನಿಗೆ ಇದು ಕಾಣಿಸುವುದು. ಪಂಚೇಂದ್ರಿಯದ ಹಿಡಿತಕ್ಕೆ ಸಿಕ್ಕದೇ ಇದ್ದರೂ ಚಿತ್ತ ಶುದ್ಧವಾದ ಕೈಯಲ್ಲಿರುವ ಸ್ಥಿರಬುದ್ಧಿಗೆ ಗೋಚರ ಅದು. ಅದನ್ನು ಪಡೆಯುವುದರಲ್ಲಿ ಅವನು ಜಾಗ್ರತನಾಗಿರುವನು. ರೇಡಿಯೋ ಸಂಗೀತ ಸುತ್ತಲೂ ಇದೆ. ಆದರೆ ಯಾರಲ್ಲಿ ರೇಡಿಯೋ ಇಲ್ಲವೊ ಅವನು ಕೇಳಲಾರ. ಜ್ಞಾನಿ ತನ್ನ ಹೃದಯವನ್ನು ಪರಮಾತ್ಮನಿಗೆ ಅಣಿಮಾಡಿರುವುದರಿಂದ ಬರುವ ಸಂಗೀತವನ್ನು ಅವನು ಕೇಳುವನು. ಅಜ್ಞಾನಿ ತನ್ನ ಹೃದಯವನ್ನು ಅದಕ್ಕೆ ಅಣಿಮಾಡದೆ ಇರುವುದರಿಂದ ಅದನ್ನು ಕೇಳಲಾರ. ಯಾವುದನ್ನು ಅವನು ಕೇಳಲಾರನೊ, ಅನುಭವಿಸಲಾರನೊ ಅದು ಅವನ ಪಾಲಿಗೆ ಇಲ್ಲ.

ಎಲ್ಲಿ ಎಲ್ಲರೂ ಎಚ್ಚತ್ತಿರುವರೊ ಅಲ್ಲಿ ಜ್ಞಾನಿ ಮಲಗಿರುವನು. ಎಲ್ಲಿ ಎಲ್ಲರೂ ಎಂದರೆ ಅಜ್ಞಾನಿಗಳು ತಮಗೆ ಅನುಭವಿಸುವುದಕ್ಕೆ ಯೋಗ್ಯವಾದ ಅನುಭವಗಳನ್ನು ಇಂದ್ರಿಯ ಕರಣಗಳ ಮೂಲಕ ಹಿಡಿಯುವುದಕ್ಕಾಗಿ ಜಾಗ್ರತರಾಗಿರುವರೊ ಅಲ್ಲಿ ಜ್ಞಾನಿ ಇದರಿಂದ ಪ್ರಯೋಜನವಿಲ್ಲವೆಂದು ಅದನ್ನು ನಿರ್ಲಕ್ಷಿಸಿರುವನು. ಅವನು ಇದನ್ನು ಗಣನೆಗೇ ತಂದುಕೊಳ್ಳುವುದಿಲ್ಲ. ಸಾಧಾರಣ ಮನುಷ್ಯರು ಹಿಡಿಯುವುದಕ್ಕೆ ಕಾತರರಾಗಿರುವುದೇ ಪಂಚೇಂದ್ರಿಯಗಳ ಬಲೆಗೆ ಬೀಳುವ ಇಂದ್ರಿಯ ಅನುಭವ, ಐಶ್ವರ್ಯ, ಕೀರ್ತಿ ಮುಂತಾದುವುಗಳು. ಆದರೆ ಜ್ಞಾನಿಗಾದರೊ ಇದರ ಕ್ಷಣಿಕತೆ ತಕ್ಷಣವೇ ವೇದ್ಯವಾಗುವುದು. ಎಲ್ಲಿ ಇಂತಹ ವಸ್ತುಗಳನ್ನು ಹಿಡಿಯುವುದಕ್ಕೆ ನೂಕುನುಗ್ಗಲೊ ಅಲ್ಲಿ ಜ್ಞಾನಿ ಹೋಗುವುದೇ ಇಲ್ಲ. ಬಿಟ್ಟಿಯಾಗಿ ಅದನ್ನು ಕೊಟ್ಟರೂ ಅವನು ಒಲ್ಲ. ಏಕೆಂದರೆ ಇವುಗಳೆಲ್ಲ ಬಣ್ಣದ ಆಟದ ಸಾಮಾನುಗಳು ಎಂಬುದು ಅವನಿಗೆ ಗೊತ್ತಿದೆ. ಅದನ್ನು ಚೀಪುವುದೇ ತಡ ಅದು ತನ್ನ ಬಣ್ಣವನ್ನು ಕಳೆದುಕೊಳ್ಳುವುದು. ಅನಂತರ ಅದನ್ನು ನಾವು ಆಚೆಗೆ ಬಿಸಾಡಲು ಆಗುವುದಿಲ್ಲ. ನಾವು ಅದನ್ನು ಬಿಸಾಡಲು ಯತ್ನಿಸಿದರೂ ಅದು ನಮ್ಮನ್ನು ಮೆಟ್ಟಿಕೊಂಡಿರುವುದು. ಅದು ಬಿಟ್ಟು ಹೋಗುವುದಿಲ್ಲ.

\begin{shloka}
ಆಪೂರ್ಯಮಾಣಮಚಲಪ್ರತಿಷ್ಠಂ\\ಸಮುದ್ರಮಾಪಃ ಪ್ರವಿಶಂತಿ ಯದ್ವತ್~।\\ತದ್ವತ್ ಕಾಮಾ ಯಂ ಪ್ರವಿಶಂತಿ ಸರ್ವೇ\\ಸ ಶಾಂತಿಮಾಪ್ನೋತಿ ನ ಕಾಮಕಾಮೀ \hfill॥ ೭ಂ~॥
\end{shloka}

\begin{artha}
ಪೂರ್ಣವಾಗಿ ಅಚಲವಾಗಿರುವ ಸಮುದ್ರವನ್ನು ನದಿ ಹೇಗೆ ಪ್ರವೇಶಿಸುವುದೊ ಹಾಗೆ ಕಾಮನೆಗಳೆಲ್ಲ ಯಾರನ್ನು ಪ್ರವೇಶಿಸುವುವೊ ಅವನು ಶಾಂತಿಯನ್ನು ಹೊಂದುತ್ತಾನೆ. ಕಾಮಗಳನ್ನು ಬಯಸುವವನು ಅಲ್ಲ.
\end{artha}

ನದಿಯ ನೀರು ಹಗಲು ರಾತ್ರಿ ಬಂದು ಸಮುದ್ರಕ್ಕೆ ಬೀಳುತ್ತಿರುವುದು. ಆದರೂ ಸಮುದ್ರ ಕಟ್ಟೆಯೊಡೆದು ಹೋಗುವುದಿಲ್ಲ. ತನ್ನ ಮೇರೆಯಲ್ಲಿಯೇ ಇರುವುದು. ಅದರಲ್ಲಿ ಸ್ವಲ್ಪವೂ ನೀರು ಹೆಚ್ಚಾಗುವುದಿಲ್ಲ. ಅದರಂತೆಯೇ ದೊಡ್ಡ ಬರಗಾಲ ಬಂದು ಮಳೆಯೇ ನಿಂತುಹೋದರೂ, ನದಿಗಳೆಲ್ಲ ಬತ್ತಿಹೋದರೂ, ನೀರು ಸ್ವಲ್ಪವೂ ಕಡಿಮೆಯಾಗುವುದಿಲ್ಲ. ಬರುವ ಅಥವಾ ಬಾರದಿರುವ ನೀರಿನಿಂದ ಸಮುದ್ರ ಉಬ್ಬುವುದೂ ಇಲ್ಲ, ತಗ್ಗುವುದೂ ಇಲ್ಲ.

ಇದರಂತೆಯೇ ಜ್ಞಾನಿ. ಹಗಲು ರಾತ್ರಿ ಈ ಪ್ರಪಂಚದ ಸುದ್ದಿ ಸಮಾಚಾರಗಳನ್ನು ಪಂಚೇಂದ್ರಿಯಗಳು ಇವನಿಗೆ ತರುತ್ತಿರುತ್ತವೆ. ಆದರೆ ಅವನು ಇದರಿಂದ ಉಬ್ಬಿಹೋಗಿಬಿಡುವುದಿಲ್ಲ. ಹಿಂದೆ ಹೇಗೆ ಇದ್ದನೋ ಅನಂತರವೂ ಹಾಗೆಯೇ ಇರುತ್ತಾನೆ. ಅವನು ಇಂದ್ರಿಯದ ಸುದ್ದಿಸಮಾಚಾರಗಳಿಗೆ ಬಾಧಿತನಾಗಿ ಹೋಗುವುದಿಲ್ಲ. ಅವೆಲ್ಲ ಇವನಲ್ಲಿಗೆ ಬರುತ್ತವೆ. ಒಳ್ಳೆಯದು ಕೆಟ್ಟದ್ದು, ಸ್ತುತಿ ನಿಂದೆ, ಸುಖ ದುಃಖ, ದ್ವಂದ್ವ ಅನುಭವಗಳೆಲ್ಲ ಬೀಳುತ್ತಿದ್ದರೂ ಅವನು ಅಸಂಗನಾಗಿರುವನು. ಅವನು ಯಾವುದನ್ನೂ ಹುಡುಕಿಕೊಂಡು ಹೋಗುವುದಿಲ್ಲ. ಅವನೆಡೆಗೆ ಕೀರ್ತಿ ಬರುವುದು, ಧನ ಬರುವುದು, ಅಧಿಕಾರ ಬರುವುದು. ಆದರೆ ಅವನು ಯಾವುದಕ್ಕೂ ಕೈಯೊಡ್ಡುವುದಿಲ್ಲ. ಅದು ಬರುವುದಕ್ಕೆ ಇವನು ತೊಂದರೆಯನ್ನು ತೆಗೆದುಕೊಳ್ಳುವುದಿಲ್ಲ. ಅದಾಗಿ ಬಂದರೆ ಅದನ್ನು ಸ್ವೀಕರಿಸುವುದೂ ಇಲ್ಲ. ಬಂದರೆ ಬರಲಿ ಹೋದರೆ ಹೋಗಲಿ ಎಂದು ನಿರ್ಲಕ್ಷ ್ಯವಾಗಿರುವನು. ಇಂತಹವನಿಗೆ ನಿಜವಾದ ಶಾಂತಿ. ಯಾರು ಇವುಗಳನ್ನು ಹುಡುಕಿಕೊಂಡು ಹೋಗುವನೊ ಅವನು ಇವುಗಳ ದಾಸನಾಗುವನು. ಸುಖದ ವಸ್ತುಗಳು ಬಂದರೆ ಸುಖಪಡುವನು. ತನ್ನ ಸಮಾನ ಇಲ್ಲವೆಂದು ಮೆರೆಯುವನು. ಹೋದರೆ ಹತಾಶನಾಗುವನು.

ಜ್ಞಾನಿಗೆ ಯಾವುದೂ ಬರುವುದಿಲ್ಲ ಎಂದಲ್ಲ. ಬಹುಶಃ ಆಸ್ತಿ, ಹೆಸರು, ಕೀರ್ತಿ ಇವುಗಳಿಗೆ ಒದ್ದಾಡುತ್ತಿರುವ ಮನುಷ್ಯನಿಗೆ ಎಷ್ಟು ಬರುವುದೊ ಅದಕ್ಕಿಂತ ಹೆಚ್ಚಾಗಿ ಅವನಿಗೆ ಇವುಗಳೆಲ್ಲ ಬರಬಹುದು. ಆದರೆ ಅವನು ಇದರಿಂದ ಬಾಧಿತನಾಗುವುದಿಲ್ಲ. ಏಕೆಂದರೆ ಅವನು ಇವುಗಳಿಗೆ ಕೈಯೊಡ್ಡುವುದಿಲ್ಲ. ಒಡ್ಡಿದರೆ ಅದು ಕುಣಿಸುವುದು.

\begin{shloka}
ವಿಹಾಯ ಕಾಮಾನ್ ಯಃ ಸರ್ವಾನ್ ಪುಮಾಂಶ್ಚರತಿ ನಿಃಸ್ಪೃಹಃ~।\\ನಿರ್ಮಮೋ ನಿರಹಂಕಾರಃ ಸ ಶಾಂತಿಮಧಿಗಚ್ಛತಿ \hfill॥ ೭೧~॥
\end{shloka}

\begin{artha}
ಯಾರು ಕಾಮನೆಗಳನ್ನೆಲ್ಲ ಬಿಟ್ಟು ಸ್ಪೃಹೆ ಇಲ್ಲದವನಾಗಿ, ನಿರ್ಮಮ ನಿರಹಂಕಾರನಾಗಿ ಚಲಿಸುತ್ತಿರುವನೊ ಅವನು ಶಾಂತಿಯನ್ನು ಹೊಂದುತ್ತಾನೆ.
\end{artha}

ಪರಮಶಾಂತಿಯನ್ನು ಪಡೆಯಬೇಕಾದರೆ ಕಾಮನೆಗಳನ್ನೆಲ್ಲ ಬಿಡಬೇಕು. ನಾವು ನೋಡಿದ ಆದರೆ ಪಡೆಯದ ವಸ್ತುವಿನ ಮೇಲೆ ಇರುವ ಆಸೆಯನ್ನು ಬಿಡಬೇಕು. ಎಲ್ಲಿಯವರೆಗೆ ಈ ಚಪಲ ನಮ್ಮಲ್ಲಿರುವುದೊ ಅಲ್ಲಿಯವರೆಗೆ ನಾವು ತೆಪ್ಪಗೆ ಇರುವುದಕ್ಕೆ ಆಗುವುದಿಲ್ಲ. ಆಸೆ ಯಾವಾಗಲೂ ನಮ್ಮ ಹೃದಯವನ್ನು ಮೀಟುತ್ತ ಇರುವುದು. ಮನಸ್ಸು ಅಲ್ಲೋಲಕಲ್ಲೋಲವಾಗುತ್ತ ಇರುವುದು. ಇದರಂತೆಯೇ ನಾವು ನೋಡಿಲ್ಲ, ಆದರೆ ಆ ತರಹ ಲೋಕವೋ ಅನುಭವವೊ ಇರುವುದು ಎಂದು ಭಾವಿಸುವೆವು. ಅಲ್ಲಿಗೆ ಹೋಗಬೇಕು, ಅಲ್ಲಿರುವುದನ್ನು ಅನುಭವಿಸಬೇಕೆಂದಿರುವುದೇ ಸ್ಪೃಹೆ. ಸ್ವರ್ಗಾದಿ ಲೋಕಗಳಿವೆ. ಅಲ್ಲಿ ಮನುಷ್ಯ ಸೂಕ್ಷ್ಮ ದೇಹೇಂದ್ರಿಯಗಳೊಡನೆ ಅನುಭವಿಸಬಹುದು ಎಂಬುದನ್ನು ಕೇಳಿರುವನು, ಅಥವಾ ಓದಿರುವನು. ಅಂತಹ ಒಂದು ಸ್ಥಳವಿದ್ದರೆ ಅಲ್ಲಿಗೆ ಹೋಗಿ ವಿಹರಿಸಬೇಕೆಂಬ ಬಯಕೆ ಇಲ್ಲದೆ ಇರುವವನೆ ಸ್ಪೃಹೆ ಇಲ್ಲದವನು. ಇಂತಹ ಒಂದು ಲೋಕವಿಲ್ಲವೆಂದು ಅದರ ಮೇಲಿನ ಆಸೆಯನ್ನು ಬಿಡುವುದಿಲ್ಲ. ಒಂದು ವೇಳೆ ಇದ್ದರೂ ಈ ಪ್ರಪಂಚಕ್ಕಿಂತ ಮೇಲಲ್ಲ. ಆ ಸುಖವೆನ್ನುವುದನ್ನು ಮೊದಲೇ ಅರಿತು ತ್ಯಜಿಸಿದವನು ಶಾಂತಿಯನ್ನು ಪಡೆಯುತ್ತಾನೆ.

ಇಂತಹ ಜ್ಞಾನಿ, ಈ ಪ್ರಪಂಚದಲ್ಲಿರುವಾಗಲೂ, ಅವನಿಗೆ ಜೀವನದ ಅತ್ಯಂತ ಆವಶ್ಯಕವಾದ ವಸ್ತುಗಳಲ್ಲಿಯೂ ಇದು ನನ್ನದು, ನನಗೆ ಸೇರಿದ್ದು ಎಂಬ ಮಮತೆ ಇರುವುದಿಲ್ಲ. ಇದರ ಸುತ್ತಮುತ್ತ ಇರುವನು. ಯಾವಾಗ ಬೇಕಾದರೂ ಇವುಗಳಿಂದ ಬಿಟ್ಟುಹೋಗಲು ಅಣಿಯಾಗಿರುವನು. ಇದರಂತೆಯೇ ಅವನು ತನ್ನಲ್ಲಿರುವ ವಿದ್ವತ್ತಿಗಾಗಲೀ, ಅಧಿಕಾರಕ್ಕಾಗಲೀ ಅಥವಾ ಆಧ್ಯಾತ್ಮಿಕ ಅನುಭವಗಳಿಗಾಗಲೀ ಅಹಂಕಾರ ಪಡುವುದಿಲ್ಲ.

ಇಂತಹ ಸ್ಥಿತಪ್ರಜ್ಞ ಶಾಂತಿ ಎಂದರೆ ಪರಮ ನಿರ್ವಾಣ ಪಡೆಯುವನು. ಈ ಪ್ರಪಂಚದಲ್ಲಿರುವಾಗ ಯಾವುದಕ್ಕೂ ಅಂಟಿಕೊಂಡಿರುವುದಿಲ್ಲ. ಹೋಗುವಾಗ ಅಯ್ಯೋ ಹೋಗಬೇಕಲ್ಲ ಎಂದು ವ್ಯಥೆಪಡು ವವನೂ ಅಲ್ಲ. ಅವನು ಹುರಿದ ಬೀಜದಂತೆ ಆಗಿರುವನು. ಅಜ್ಞಾನದ ಆಸೆಯೆಲ್ಲ ಸುಟ್ಟುಹೋಗಿದೆ. ಈ ಪ್ರಪಂಚದಿಂದ ಹೋದರೆ ಹಿಂತಿರುಗಿ ಬರದ ರೀತಿಯಲ್ಲಿ ಹೋಗುವವನಿವನೊಬ್ಬನೆ.

\begin{shloka}
ಏಷಾ ಬ್ರಾಹ್ಮೀ ಸ್ಥಿತಿಃ ಪಾರ್ಥ ನೈನಾಂ ಪ್ರಾಪ್ಯ ವಿಮುಹ್ಯತಿ~।\\ಸ್ಥಿತ್ವಾಸ್ಯಾಮಂತಕಾಲೇಽಪಿ ಬ್ರಹ್ಮನಿರ್ವಾಣಮೃಚ್ಛತಿ \hfill॥ ೭೨~॥
\end{shloka}

\begin{artha}
ಅರ್ಜುನ, ಇದು ಬ್ರಾಹ್ಮೀ ಸ್ಥಿತಿ. ಇದನ್ನು ಪಡೆದುಕೊಂಡವನು ಪುನಃ ಮೋಹಕ್ಕೆ ಬೀಳುವುದಿಲ್ಲ. ಅಂತ್ಯಕಾಲದಲ್ಲಿ ಈ ಸ್ಥಿತಿಯಲ್ಲಿದ್ದರೂ ಅವನು ಬ್ರಹ್ಮನಿರ್ವಾಣವನ್ನು ಪಡೆದುಕೊಳ್ಳುತ್ತಾನೆ.
\end{artha}

ಇದು ಬ್ರಾಹ್ಮೀ ಸ್ಥಿತಿ, ಎಂದರೆ ಬ್ರಹ್ಮನ ಆಧಾರದ ಮೇಲೆ ಇರುವುದು. ಬ್ರಹ್ಮನ ಅನುಭವದಿಂದ ಓತಪ್ರೋತನಾಗಿ ಈ ಜಗದಲ್ಲಿ ಬಾಳುವುದು. ಒಂದು ಸ್ಪಂಜನ್ನು ನೀರಿಗೆ ಅದ್ದಿದರೆ ಅದು ನೀರನ್ನು ತನ್ನೊಳಗೆಲ್ಲ ಹೇಗೆ ಹೀರಿಕೊಳ್ಳುವುದೊ, ಯಾವ ಭಾಗವನ್ನು ಹಿಂಡಿದರೂ ಅದರೊಳಗಿನ ನೀರು ಬರುವುದೊ ಅದರಂತೆಯೇ ಬ್ರಹ್ಮಾನುಭವವನ್ನು ಹೀರಿಕೊಂಡು ಬಾಳುವುದು.

ಯಾವಾಗ ಈ ಸ್ಥಿತಿಯನ್ನು ಒಮ್ಮೆ ಪಡೆದುಕೊಳ್ಳುತ್ತಾನೆಯೊ ಅವನು ಇನ್ನು ಮೋಹಕ್ಕೆ ಬೀಳುವುದಿಲ್ಲ. ಒಂದು ಹೀನ ಲೋಹವನ್ನು ಸ್ಪರ್ಶಶಿಲೆಗೆ ತಾಕಿಸಿದರೆ ಅದು ಚಿನ್ನವಾಗುವುದು ಎಂದು ಹೇಳುತ್ತಾರೆ. ಒಮ್ಮೆ ಅದು ಚಿನ್ನವಾದರೆ ಎಲ್ಲಿ ಬಿಸಾಡಿದರೂ ಅದು ಚಿನ್ನವಾಗಿಯೇ ಉಳಿದುಕೊಳ್ಳುವುದೇ ಹೊರತು ಹಿಂದಿನ ಸ್ಥಿತಿಗೆ ಹೋಗುವುದಿಲ್ಲ. ಅದನ್ನು ಸಂದೂಕಿನಲ್ಲಿ ಹಾಕಿ ಕೂಡಿಟ್ಟರೂ ಚಿನ್ನವೇ, ಎಲ್ಲೊ ಹೊರಗಡೆ ಅದನ್ನು ಬಿಸುಟರೂ ಚಿನ್ನವೇ ಆಗಿರುವುದು. ಮರಳುಕಾಡಿನಲ್ಲಿ ಹೋಗುತ್ತಿರುವಾಗ ಮರೀಚಿಕೆಯನ್ನು ಮೊದಲು ಪ್ರಯಾಣಿಕನು ಕಂಡು ಅಲ್ಲಿ ನೀರನ್ನೇ ಹುಡುಕಲು ಹೋಗುವನು. ಆದರೆ ಅದೊಂದು ಭ್ರಮೆ, ಮರೀಚಿಕೆ ಎಂದು ಗೊತ್ತಾದಮೇಲೆ ಅಲ್ಲಿಗೆ ಸ್ನಾನ ಮಾಡುವುದಕ್ಕಾಗಲಿ, ಆ ನೀರನ್ನು ಕುಡಿಯುವುದಕ್ಕಾಗಲಿ ಹೋಗುವುದಿಲ್ಲ.

ಕೊನೆ ಗಳಿಗೆಯಲ್ಲಿ ಈ ಸ್ಥಿತಿಯಲ್ಲಿದ್ದರೂ ಅವನು ಬ್ರಹ್ಮನಿರ್ವಾಣ ಪಡೆಯುತ್ತಾನೆ ಎನ್ನುತ್ತಾನೆ. ಮೊದಲಿನಿಂದ ಆ ಸ್ಥಿತಿಯಲ್ಲಿ ಇದ್ದರೆ ಮಾತ್ರ ಕೊನೆಗೆ ಅಂತಹ ಸ್ಥಿತಿ ಪ್ರಾಪ್ತವಾಗುವುದು. ಇರುವ ತನಕ ಪ್ರಾಪಂಚಿಕ ಜೀವನ ನಡೆಸಿ ಕೊನೆ ಗಳಿಗೆಯಲ್ಲಿ ಬ್ರಹ್ಮಚಿಂತನೆಯನ್ನು ಮಾಡುತ್ತ ಸಾಯುವುದಕ್ಕೆ ಆಗುವುದಿಲ್ಲ. ಕೊನೆ ಗಳಿಗೆಯಲ್ಲಿ ಅದು ನಮ್ಮ ಸ್ಥಿತಿಯಾಗಬೇಕಾದರೆ, ಅದಕ್ಕಿಂತ ಮುಂಚೆಯೇ ಅದನ್ನು ಸಾಧಿಸಿದ್ದರೆ ಮಾತ್ರ ಸಾಧ್ಯ.

ಅರ್ಜುನ ಕೇಳಿದ ಸ್ಥಿತಪ್ರಜ್ಞನ ಲಕ್ಷಣಗಳನ್ನು ಶ‍್ರೀಕೃಷ್ಣ ಇಲ್ಲಿ ಪೂರೈಸುವನು.

