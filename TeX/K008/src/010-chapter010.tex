
\chapter{ವಿಭೂತಿಯೋಗ}

ಶ‍್ರೀಕೃಷ್ಣ ಅರ್ಜುನನಿಗೆ ತನ್ನ ವಿಭೂತಿಯನ್ನು ಕುರಿತು ಹೇಳುತ್ತಾನೆ. ಭಗವಂತ ತನ್ನ ವಿಭೂತಿಯನ್ನು ತಾನೇ ಹೇಳಿಕೊಂಡರೆ ಮಾತ್ರ ಅದನ್ನು ತಿಳಿದುಕೊಳ್ಳುವುದಕ್ಕೆ ಸಾಧ್ಯ. ನಮ್ಮ ಅಲ್ಪ ಬುದ್ಧಿಗೆ ಅವನು ಅಗೋಚರ. ನಮ್ಮ ಮೇಲೆ ಕೃಪೆ ಇಟ್ಟು ಅವನು ತನ್ನ ಮಹಿಮೆಯನ್ನು ಹೇಳುತ್ತಾನೆ. ಇದು ನಮ್ಮ ಧ್ಯಾನಕ್ಕೆ ಮತ್ತು ಅವನನ್ನು ಕುರಿತು ಚಿಂತಿಸುವುದಕ್ಕೆ ಸಹಾಯಕವಾಗಲಿ ಎಂದು ಹೇಳುತ್ತಾನೆ.

\begin{shloka}
ಭೂಯ ಏವ ಮಹಾಬಾಹೋ ಶೃಣು ಮೇ ಪರಮಂ ವಚಃ~।\\ಯತ್ತೇಽಹಂ ಪ್ರೀಯಮಾಣಾಯ ವಕ್ಷ್ಯಾಮಿ ಹಿತಕಾಮ್ಯಯಾ \hfill॥ ೧~॥
\end{shloka}

\begin{artha}
ಅರ್ಜುನ, ನೀನು ಪುನಃ ನನ್ನ ಪರಮ ವಚನವನ್ನು ಕೇಳು. ಇದನ್ನು ನಾನು ಪ್ರೀತಿಯನ್ನು ಹೊಂದಿರುವ ನಿನಗೆ, ನಿನ್ನ ಹಿತಕ್ಕೋಸ್ಕರ ಹೇಳುತ್ತೇನೆ.
\end{artha}

ಶ‍್ರೀಕೃಷ್ಣ ಅರ್ಜುನನಿಗೆ ಪುನಃ ತನ್ನ ಪರಮ ವಚನವನ್ನು ಕೇಳು ಎನ್ನುತ್ತಾನೆ. ಏನೊ ಹೊಸ ವಿಷಯವನ್ನು ಹೇಳುತ್ತೇನೆ ಎನ್ನುವುದಿಲ್ಲ. ಈ ಅಧ್ಯಾಯದಲ್ಲಿ ಬರುವ ಭಾವನೆಗಳು ಆಗಲೆ ಹಿಂದಿನ ಅಧ್ಯಾಯದಲ್ಲಿ ಒಂದಲ್ಲ ಒಂದು ರೀತಿಯಲ್ಲಿ ಬಂದಿವೆ. ಯಾವಾಗ ನಾವು ಸಾಧನಾ ದೃಷ್ಟಿಯಿಂದ ನೋಡುತ್ತೇವೆಯೋ ಆಗ ಪುನಃ ಪುನಃ ನಾವು ಕೇಳಿದ್ದನ್ನೇ ಬೇರೆ ಬೇರೆ ದೃಷ್ಟಿಯಿಂದ ಕೇಳಬೇಕು, ನೋಡಿದ್ದನ್ನೇ ಬೇರೆ ಬೇರೆ ದೃಷ್ಟಿಕೋಣಗಳಿಂದ ನೋಡಬೇಕು. ಇಲ್ಲಿ ಪದೇ ಪದೇ ಹೇಳುವುದು ದೋಷವಲ್ಲ. ಒಂದು ಸಲ ಹೇಳಿದರೆ ಸಾಕು, ಅದು ನಮ್ಮ ಮನಸ್ಸಿನಲ್ಲಿ ಅಂಟಿಕೊಳ್ಳುವ ಸ್ಥಿತಿಯಲ್ಲಿ ಎಲ್ಲರೂ ಇಲ್ಲ. ಅದನ್ನು ಪದೇಪದೇ ಕೇಳುತ್ತಿದ್ದರೆ ಒಂದಲ್ಲ ಒಂದು ಸಲವಾದರೂ ನಮ್ಮ ಮನಸ್ಸಿಗೆ ಪ್ರವೇಶವಾಗುವುದು. ಅದೊಂದೇ ಅಲ್ಲ, ಮೊದಲನೆ ಬಾರಿ ಕೇಳಿದಾಗ ಸ್ವಲ್ಪ ಗೊತ್ತಾಗುವುದು, ಬೇರೆ ರೀತಿಯಲ್ಲಿ ಅದನ್ನು ಕೇಳಿದಾಗ ಇನ್ನೂ ಹೆಚ್ಚು ಅರ್ಥವಾಗುವುದು. ಕೇಳುತ್ತಾ ಕೇಳುತ್ತಾ ಅದು ಆಳ ಆಳಕ್ಕೆ ಇಳಿಯುತ್ತಾ ಹೋಗುವುದು. ಇದು ಬಹಳ ಕಾಲದಿಂದ ನರಳುವ ವ್ಯಾಧಿಗೆ ಔಷಧಿಯನ್ನು ತೆಗೆದುಕೊಳ್ಳುವಂತೆ. ಒಂದು ಸಲ ಔಷಧಿ ತೆಗೆದುಕೊಂಡರೆ ಖಾಯಿಲೆಯೆಲ್ಲ ಬಿಟ್ಟುಹೋಗುವುದಿಲ್ಲ. ಪದೇ ಪದೇ ಅದೇ ಔಷಧಿ ತೆಗೆದುಕೊಳ್ಳುತ್ತಿರಬೇಕು. ರೋಗಿ ವೈದ್ಯನಿಗೆ ದಿನ ಬೆಳಗಾದರೆ ಒಂದೇ ಔಷಧಿಯನ್ನು ಕೊಡುತ್ತೀರಿ ಎಂದು ಆಕ್ಷೇಪಣೆ ಮಾಡಬಹುದು. ಆದರೆ ಅವನಿಗೆ ಇರುವುದು ಒಂದೇ ವ್ಯಾಧಿ. ಅದು ಬಹಳ ಆಳದವರೆಗೆ ಬೇರು ಬಿಟ್ಟಿರುವುದು. ಅದನ್ನು ಬೇರು ಸಹಿತ ಕೀಳಬೇಕಾದರೆ ಒಂದೇ ಆಯುಧದಿಂದ ಆಳ ಆಳಕ್ಕೆ ಅಗೆಯುತ್ತಾ ಹೋಗಿ ಬೇರನ್ನು ನಾಶಮಾಡಬೇಕಾಗಿದೆ. ಅದಕ್ಕಾಗಿಯೇ ಶ‍್ರೀಕೃಷ್ಣ ಒಂದೇ ಬೋಧನೆಯನ್ನು ಬೇರೆ ಬೇರೆ ರೂಪಿನಲ್ಲಿ ಕೊಡುತ್ತಾನೆ ಗೀತೆಯ ಪ್ರತಿಯೊಂದು ಅಧ್ಯಾಯದಲ್ಲಿಯೂ.

\newpage

ಇದನ್ನು ನಾನು ಪ್ರೀತಿ ಇರುವ ನಿನಗೆ ಹೇಳುತ್ತೇನೆ ಎನ್ನುತ್ತಾನೆ. ಶಿಷ್ಯನಿಗೆ ಗುರುವಿನ ಮೇಲೆ ಪ್ರೀತಿ ಇರಬೇಕು. ಆಗ ಮಾತ್ರ ಬೋಧನೆ ಫಲಕಾರಿ ಆಗಬೇಕಾದರೆ. ಆಗ ಮಾತ್ರ ಗುರು ಹೇಳುವುದನ್ನು ಶಿಷ್ಯ ಕೇಳುತ್ತಾನೆ. ಯಾವಾಗ ಶಿಷ್ಯನ ಹೃದಯದಲ್ಲಿ ಗುರುವಿನ ಮೇಲೆ ಪ್ರೀತಿ ಇದೆಯೊ ಆಗ ಕಾದ ಕಬ್ಬಿಣದಂತೆ ಇರುವುದು ಅವನ ಮನಸ್ಸು. ಗುರು ಕಮ್ಮಾರನಂತೆ ಅದನ್ನು ಅಡಿಗಲ್ಲಿನ ಮೇಲೆ ಇಟ್ಟು ಕುಟ್ಟಿ ಸುಲಭವಾಗಿ ಯಾವ ರೂಪಕ್ಕೆ ಬೇಕಾದರೂ ತಿರುಗಿಸಬಹುದು. ಗುರುವಿನ ಮೇಲೆ ಪ್ರೀತಿ ಹೊಂದಿರುವುದು ನೆಲವನ್ನು ಉತ್ತು ಗೊಬ್ಬರ ಹಾಕಿ ಹದ ಮಾಡಿದ ಹೊಲದಂತೆ. ಬೀಜ ಹಾಕಿದರೆ ಅದು ವ್ಯರ್ಥವಾಗುವುದಿಲ್ಲ. ಮೊಳೆಯುವುದರಲ್ಲಿ ಸಂದೇಹವಿಲ್ಲ. ಶ‍್ರೀಕೃಷ್ಣ ಅರ್ಜುನನಲ್ಲಿ ಇದನ್ನು ಗಮನಿಸುತ್ತಾನೆ.

ಅರ್ಜುನನ ಹಿತಕ್ಕಾಗಿ ಶ‍್ರೀಕೃಷ್ಣ ಪುನಃ ಹೇಳುತ್ತಾನೆ. ಯಾವಾಗ ಅರ್ಜುನ ಶ‍್ರೀಕೃಷ್ಣನಲ್ಲಿ ಶಿಷ್ಯನಂತೆ ಶರಣಾಗುತ್ತಾನೆಯೋ ಆಗ ಅವನ ಆಧ್ಯಾತ್ಮಿಕ ಜೀವನದ ಜವಾಬ್ದಾರಿಯೆಲ್ಲವನ್ನೂ ಶ‍್ರೀಕೃಷ್ಣ ಹೊರಬೇಕಾಯಿತು. ಆದಕಾರಣವೇ ತನ್ನಲ್ಲಿ ಶರಾಣಾದವನು ಉದ್ಧಾರವಾಗಬೇಕೆಂಬ ದೃಷ್ಟಿಯಿಂದ ಹೇಳುತ್ತಾನೆ. ಗುರುವಿಗೆ ಶಿಷ್ಯನ ಮೇಲೆ ಇರುವ ವಾತ್ಸಲ್ಯ, ತಂದೆಗೆ ಮಗನ ಮೇಲೆ ಇರುವ ವಾತ್ಸಲ್ಯಕ್ಕಿಂತ ಹೆಚ್ಚು. ಮಗನಾದರೊ ತನ್ನ ರಕ್ತಮಾಂಸ ಹಂಚಿಕೊಂಡು ಹುಟ್ಟುತ್ತಾನೆ. ಹೊರಗಿನ ಆಕಾರದಲ್ಲಿ ಮತ್ತು ಬಣ್ಣದಲ್ಲಿ ತಂದೆಯನ್ನು ಹೋಲಬಹುದು. ಆದರೆ ಆದರ್ಶದಲ್ಲಿ ತಂದೆಯನ್ನು ಹೋಲುವ ಮಕ್ಕಳು ಅಪರೂಪ. ಶಿಷ್ಯನಾದರೆ ಆದರ್ಶದಲ್ಲಿ ಗುರುವನ್ನು ಹೋಲುತ್ತಾನೆ, ಗುರುವಿನ ಆದರ್ಶವನ್ನು ಪ್ರೀತಿಸುತ್ತಾನೆ. ಶಿಷ್ಯನೇ ಗುರುವಿನ ಮಾನಸಿಕ ಪುತ್ರ. ಗುರು ಕಲಿತದ್ದು ಶಿಷ್ಯನಲ್ಲಿ ಬೆಳೆಯುವುದು. ಗುರುವಿನ ವಿದ್ಯೆ ಅವ್ಯಾಹತವಾಗಿ ಮುಂದುವರಿಸಿಕೊಂಡು ಹೋಗುವಂತೆ ಮಾಡುವವನೇ ಶಿಷ್ಯ. ಆದಕಾರಣವೇ ಗುರುವಿಗೆ ಶಿಷ್ಯನನ್ನು ಕಂಡರೆ ಅಷ್ಟು ಪ್ರೀತಿ ಮತ್ತು ಅವನ ಅಭ್ಯುದಯದ ಮೇಲೆ ಅಷ್ಟೊಂದು ಆಸಕ್ತಿ.

\begin{shloka}
ನ ಮೇ ವಿದುಃ ಸುರಗಣಾಃ ಪ್ರಭವಂ ನ ಮಹರ್ಷಯಃ~।\\ಅಹಮಾದಿರ್ಹಿ ದೇವಾನಾಂ ಮಹರ್ಷೀಣಾಂ ಚ ಸರ್ವಶಃ \hfill॥ ೨~॥
\end{shloka}

\begin{artha}
ದೇವತೆಗಳಾಗಲಿ, ಮಹರ್ಷಿಗಳಾಗಲಿ, ನನ್ನ ಆದಿಯನ್ನು ಅರಿಯರು. ಏಕೆಂದರೆ ನಾನೇ ಎಲ್ಲಾ ದೇವತೆಗಳಿಗೂ ಮಹರ್ಷಿಗಳಿಗೂ ಸರ್ವ ಪ್ರಾಕಾರದಿಂದಲೂ ಆದಿಯಾಗಿರುತ್ತೇನೆ.
\end{artha}

ದೇವತೆಗಳು ಮಹರ್ಷಿಗಳು ಮಾನವರೆಲ್ಲರಿಗಿಂತಲೂ ಬುದ್ಧಿವಂತರು ಮತ್ತು ಬಹಳ ಮುಂದು ವರಿದವರು ಎಂಬ ಭಾವನೆ. ಇವರು ಕೂಡ ಭಗವಂತನ ಆದಿಯನ್ನು ಅರಿಯರು. ಏಕೆಂದರೆ ಇವರೆಲ್ಲ ಬಂದದ್ದು ಅನಂತರ. ದೇವರು ಇವರುಗಳನ್ನು ಸೃಷ್ಟಿ ಮಾಡಿದವನು. ದೇವರಿಗೆ ಇವರ ಆದಿ ಗೊತ್ತೆ ಹೊರತು ಇವರಿಗೆ ದೇವರ ಆದಿ ತಿಳಿಯಲಾರದು. ಸೃಷ್ಟಿಯಾದಮೇಲೆ ಇವರೆಲ್ಲ ಬಂದದ್ದು. ಸೃಷ್ಟಿ ಮಾಡಿದವನು ದೇವರು. ಸೃಷ್ಟಿಯಲ್ಲಿ ಇರುವವರಿಗೆ ಸೃಷ್ಟಿಕರ್ತನ ಆದಿ ಹೇಗೆ ಗೊತ್ತಾಗಬೇಕು.

\begin{shloka}
ಯೋ ಮಾಮಜಮನಾದಿಂ ಚ ವೇತ್ತಿ ಲೋಕಮಹೇಶ್ವರಮ್~।\\ಅಸಂಮೂಢಃ ಸ ಮರ್ತ್ಯೇಷು ಸರ್ವಪಾಪೈಃ ಪ್ರಮುಚ್ಯತೇ \hfill॥ ೩~॥
\end{shloka}

\begin{artha}
ಯಾರು ನನ್ನನ್ನು ಜನ್ಮರಹಿತನೆಂದು, ಆದಿಯಿಲ್ಲದವನೆಂದು, ಸರ್ವಲೋಕ ಮಹೇಶ್ವರನೆಂದು ತಿಳಿಯುತ್ತಾನೆಯೊ ಅವನು ಮನುಷ್ಯರಲ್ಲಿ ಮೋಹಶೂನ್ಯನಾಗಿ ಸರ್ವ ಪಾಪಗಳಿಂದಲೂ ಮುಕ್ತ\-ನಾಗುತ್ತಾನೆ.
\end{artha}

ಯಾವಾಗ ದೇವರು ಹೇಗಿರುವನೋ ಹಾಗೆ ತಿಳಿಯುತ್ತೇವೆಯೊ ನಾವು ಈಗಿರುವ ಅಜ್ಞಾನ ಸ್ಥಿತಿಯಿಂದ ಪಾರಾಗುತ್ತೇವೆ. ದೇವರನ್ನು ಚೆನ್ನಾಗಿ ತಿಳಿದ ಮೇಲೆ ನಾವಿನ್ನು ಹಳೆಯ ಮನುಷ್ಯರಾಗಿ ಇರುವುದಕ್ಕೆ ಆಗುವುದಿಲ್ಲ, ಸುಳ್ಳಿಗೆ ಅಂಟಿಕೊಂಡಿರುವುದಕ್ಕೆ ಆಗುವುದಿಲ್ಲ. ಹೇಗೆ ಬೆಳಕಿನ ಮುಂದೆ ಕತ್ತಲೆ ಓಡಿಹೋಗುವುದೊ ಹಾಗೆ ಭಗವಂತನನ್ನು ಅರಿತ ಮೇಲೆ ನಮ್ಮನ್ನು ಆವರಿಸಿರುವ ಅಜ್ಞಾನ ಬಿಟ್ಟುಹೋಗುವುದು. ಭಗವಂತ ಜನ್ಮರಹಿತ. ಅವನು ಯಾವ ಜನ್ಮವನ್ನೂ ಧರಿಸಿಲ್ಲ. ಧರಿಸಿರುವಂತೆ ತೋರುತ್ತಿರುವನು. ಧರ್ಮ ಕಡಮೆಯಾಗಿ, ಅಧರ್ಮ ಜಾಸ್ತಿಯಾದಾಗ ರಕ್ಷಣೆಗೆ ಭಗವಂತ ಅವತಾರ ಮಾಡಿದಾಗ ಒಂದು ವ್ಯಕ್ತಿಯಂತೆ ಗೋಚರಿಸುತ್ತಾನೆ. ಆ ವ್ಯಕ್ತಿಗೆ ಒಂದು ಆದಿ ಇದೆ. ಆದರೆ ಯಾವ ಚೈತನ್ಯ ಆ ವ್ಯಕ್ತಿ ಎಂಬ ಪಾತ್ರೆಯಲ್ಲಿರುವುದೊ ಅದಕ್ಕೆ ಆದಿ ಎಂಬುದಿಲ್ಲ. ಶ‍್ರೀಕೃಷ್ಣ ಎಂಬ ಚಾರಿತ್ರಿಕ ವ್ಯಕ್ತಿಗೆ ಒಂದು ಆದಿ ಇದೆ. ಅವನು ಕೆಲವು ಕಾಲದ ಹಿಂದೆ ದೇವಕಿ ವಸುದೇವರಿಗೆ ಹುಟ್ಟಿದ. ಅವನು ಕೆಲವು ಕಾಲವಾದ ಮೇಲೆ ಬೇಟೆಗಾರನೊಬ್ಬನ ಬಾಣದಿಂದ ಪೆಟ್ಟು ತಿಂದು ದೇಹವನ್ನು ವಿಸರ್ಜಿಸಿದನು. ಶ‍್ರೀಕೃಷ್ಣನ ದೇಹಕ್ಕೆ ಆದಿ ಅಂತ್ಯವಿದೆ. ಆದರೆ ಶ‍್ರೀಕೃಷ್ಣ ಎಂಬ ದೇಹದ ಮೂಲಕ ಯಾರು ಬೆಳಗುತ್ತಿದ್ದನೊ ಅವನು ಇದೇ ಪ್ರಥಮ ಬಾರಿಯಲ್ಲ ಪ್ರಪಂಚಕ್ಕೆ ಬಂದಿರುವುದು. ಆವನು ಯಾವಾಗಲೂ ಹಿಂದೆ ಇದ್ದ. ಶ‍್ರೀಕೃಷ್ಣನ ದೇಹ ನಾಶವಾದರೂ ಅವನು ಇರುತ್ತಾನೆ. ಆವನು ಒಂದು ಆವತಾರವನ್ನು ಧರಿಸಿ ಸಾಧಾರಣ ಮನುಷ್ಯನಂತೆ ವ್ಯವಹರಿಸುತ್ತಿರುವಾಗಲೂ ಅವನಿಗೆ ಹಿಂದಿನ ದೈವತ್ವ ಯಾವ ಕಾಲದಲ್ಲಿಯೂ ಮರೆತುಹೋಗಿರಲಿಲ್ಲ.

ಆ ಭಗವಂತನೆಂಬ ಚೈತನ್ಯಕ್ಕೆ ಒಂದು ಆದಿ ಇಲ್ಲ. ಅದು ಯಾವಾಗಲೂ ಇರುವುದು. ಆದಿ ಆಂತ್ಯವೆಂಬುದು ಕಾಲದಲ್ಲಿರುವುದು. ಆದರೆ ಯಾರು ಕಾಲಾತೀತನೊ ಅವನಿಗೆ ಹೇಗೆ ಆದಿ ಸಾಧ್ಯ? ಒಂದು ವೇಳೆ ಪ್ರಕಟವಾಗಿ ಒಂದು ಆದಿಯನ್ನು ಒಪ್ಪಿಕೊಂಡರೂ ಆ ಆದಿಗೆ ಆದಿ ಯಾವುದು ಎಂದು ನಾವು ಕೇಳಬೇಕಾಗುವುದು. ನಾವು ಒಂದು ವೃತ್ತವನ್ನು ತೆಗೆದುಕೊಂಡರೆ ಅದನ್ನು ಯಾವುದೊ ಒಂದು ಕಡೆ ಇದು ಪ್ರಾರಂಭವಾಯಿತು ಎಂದು ಹೇಳಿದರೆ ಅದರ ಹಿಂದೆ ಏಕೆ ಆಗಬಾರದು ಎಂದು ಪ್ರಶ್ನಿಸುತ್ತೇವೆ. ಪ್ರತಿಯೊಂದು ಆದಿಗೂ ಒಂದೊಂದು ಆದಿ ಇದೆ. ವೃತ್ತಕ್ಕೆ ಹೇಗೆ ಆದಿಯನ್ನು ಕಂಡುಹಿಡಿಯುವುದು ಅಸಾಧ್ಯವೋ ಹಾಗೆಯೆ ಭಗವಂತನ ಆದಿಯನ್ನು ಕಂಡುಹಿಡಿಯುವುದು ಅಸಾಧ್ಯ. ಭಗವಂತ ಆದಿಗೆ ಆದಿ. ಅವನು ಅನಾದಿ.

\newpage

ಅವನು ಸರ್ವಲೋಕ ಮಹೇಶ್ವರ. ಈ ಪ್ರಪಂಚದಲ್ಲಿ, ಭೌತಿಕ ಪ್ರಪಂಚದಲ್ಲಿ ಯಾವ ಯಾವ ನಿಯಮಗಳನ್ನು ನೋಡುತ್ತೇವೆಯೋ ಇವೆಲ್ಲ ಬಂದಿರುವುದು ಅವನಿಂದ. ಇವೆಲ್ಲ ಇರುವುದು ಅವನಿಗಾಗಿ. ಇವುಗಳೆಲ್ಲ ಮಾಡುತ್ತಿರುವುದು ಅವನ ಕೆಲಸವನ್ನು. ಸಣ್ಣ ಸಣ್ಣ ಸೂಕ್ಷ್ಮಾತಿಸೂಕ್ಷ್ಮವಾದ ಕಣಗಳಿಂದ ಹಿಡಿದು ಗ್ರಹ ಚಂದ್ರ ನೀಹಾರಿಕೆಗಳೆಲ್ಲ ಅವನ ಕಟ್ಟಪ್ಪಣೆಯನ್ನು ಪಾಲಿಸುತ್ತವೆ. ನದಿ ಹರಿಯುವುದು ಅವನಿಂದ, ಗಾಳಿ ಬೀಸುವುದು ಅವನಿಂದ, ಸಮುದ್ರ ಮೊರೆಯುವುದು ಅವನಿಂದ. ತೃಣಕಾಷ್ಟ, ಪಶುಪಕ್ಷಿ ಮನುಷ್ಯರೆಲ್ಲರೂ ಅವನ ಇಚ್ಛಾನುಸಾರ ತಮ್ಮ ಜೀವನಗತಿಯಲ್ಲಿ ಮುಂದೆ ಸಾಗುತ್ತಿರುವರು. ಆಂತರಿಕ ಪ್ರಪಂಚದಲ್ಲಿ ಇರುವ ಕರ್ಮ ನಿಯಮ ಅವನೆ. ಅವನೇ ಎಲ್ಲದರ ಹಿಂದೆ ಇರುವುದು. ಎಲ್ಲವನ್ನೂ ಸೂತ್ರದಂತೆ ಪೋಣಿಸಿರುವನು. ಯಾರೂ ಇವನ ಅಪ್ಪಣೆಯನ್ನು ಮೀರಿ ಹೋಗಲಾರರು. ಒಂದು ಹುಲ್ಲು ಎಸೆಳಾಗಲಿ, ಸೂರ್ಯ ಚಂದ್ರರಾಗಲಿ ಅವನ ಇಚ್ಛೆ ಇಲ್ಲದೆ ಚಲಿಸಲಾರರು. ಈ ಜೀವನದಲ್ಲಿ ಎಲ್ಲರಿಗೂ ಭಯವನ್ನು ಹುಟ್ಟಿಸುವವನು ಮೃತ್ಯು. ಆ ಮೃತ್ಯು ಕೂಡ ಇವನಿಗೆ ಬಾಗಿ ತನ್ನ ಕೆಲಸವನ್ನು ಮಾಡುವನು.

ಯಾರು ಭಗವಂತನನ್ನು ಹೀಗೆ ಅರಿಯುತ್ತಾರೆಯೊ ಅವರು ಇನ್ನು ಮೇಲೆ ದೇಹವೆಂಬ ಪಂಜರದಲ್ಲಿ ಪ್ರಾಣಿಯಂತೆ ಸೆರೆಯಲ್ಲಿರುವುದಿಲ್ಲ. ಭಗವಂತನ್ನು ಚೆನ್ನಾಗಿ ತಿಳಿಯುತ್ತ ಇವನು ಮೋಹದಿಂದ ಪಾರಾಗುತ್ತಾ ಬರುವನು. ದೇಹ ಮನಸ್ಸು ಬುದ್ಧಿ ಇಂದ್ರಿಯ ಅಹಂಕಾರಗಳ ಮೇಲೆ ತಾದಾತ್ಮ್ಯ ಭಾವವನ್ನು ಬಿಡುತ್ತಾ ಬರುವನು. ಅವನೆದುರಿಗೆ ನಾವಾರು ಹುಲು ಮನುಜರು? ಸ್ವಾತಂತ್ರ್ಯವೆಂಬುದೆಲ್ಲಿದೆ ನಮಗೆ? ಇದನ್ನು ಕೊಟ್ಟವನೆ ಅವನು. ಅದನ್ನು ಮರೆತು ಇದಕ್ಕೆಲ್ಲ ನಾವೇ ಒಡೆಯರು ಎಂದು ಭಾವಿಸುತ್ತೇವೆ. ಅಪ್ಪ ಕೊಟ್ಟ ಕಾಸನ್ನು ಜೇಬಿನೊಳಗೆ ಇಟ್ಟುಕೊಂಡು ಇದು ನನ್ನದು; ಇದನ್ನು ಸಂಪಾದಿಸಿದವನು ನಾನು ಎಂದು ಹೇಳಿಕೊಳ್ಳುವಂತೆ. ಇದು ಹೇಗೆ ಬಂತು, ಅದನ್ನು ಕೊಟ್ಟವನು ಯಾರು ಎಂದು ವಿಚಾರಿಸಿದರೆ ನಮ್ಮ ಅಜ್ಞಾನ ಬಿಟ್ಟುಹೋಗುವುದು. ಇದೆಲ್ಲ ಅವನದು. ಸರ್ವಲೋಕ ಮಹೇಶ್ವರ ಅವನು. ನಮ್ಮದೆಂಬುದು ಯಾವುದೂ ಇಲ್ಲ. ಈ ಮಹಾಜ್ಞಾನದೆದುರಿಗೆ ನಮ್ಮದೆಂಬುವ ಅಲ್ಪ ಅಹಂಕಾರ ಬಿಟ್ಟು ಹೋಗುವುದು.

ಇಂತಹವನು ಎಲ್ಲಾ ವಿಧವಾದ ಪಾಪಗಳಿಂದಲೂ ಪಾರಾಗುತ್ತಾನೆ. ಅಜ್ಞಾನದಲ್ಲಿ ಇದ್ದರೆ ತಾನೆ ಒಬ್ಬ ಪಾಪವನ್ನು ಮಾಡುತ್ತಾನೆ? ಯಾವಾಗ ಭಗವತ್ ಜ್ಞಾನದಲ್ಲಿಯೇ ಒಬ್ಬ ಇರುವನೋ ಅವನು ಇನ್ನು ಮೇಲೆ ಪಾಪಕಾರ್ಯವನ್ನು ಮಾಡುವುದಿಲ್ಲ. ಒಂದು ವೇಳೆ ಭಗವತ್ ಜ್ಞಾನ ನಮ್ಮನ್ನು ಬಿಟ್ಟುಹೋದ ಮೇಲೆ ಪುನಃ ಪಾಪವನ್ನು ಮಾಡುವ ಸಂಭವವಿರಬಹುದೆಂದು ನಾವು ಶಂಕಿಸಬಹುದು. ಆದರೆ ಅದು ಒಂದು ಸಲ ಬಂದರೆ ನಮ್ಮನ್ನು ಬಿಡುವುದಿಲ್ಲ. ಒಂದು ಸಲ ಭಗವತ್ ಜ್ಞಾನದಲ್ಲಿ ಬೆಂದ ಮೇಲೆ ಜೀವಿಯ ಒಳಗೆ ಇರುವ ಅಜ್ಞಾನದ ಮೊಳಕೆ ನಾಶವಾಗಿ ಹೋಗುವುದು. ಅನಂತರ ಅದು ಪುನಃ ಚಿಗುರುವಂತೆ ಇಲ್ಲ.

\begin{shloka}
ಬುದ್ಧಿರ್ಜ್ಞಾನಮಸಂಮೋಹಃ ಕ್ಷಮಾ ಸತ್ಯಂ ದಮಃ ಶಮಃ~॥\\ಸುಖಂ ದುಃಖಂ ಭವೋಽಭಾವೋ ಭಯಂ ಚಾಭಯಮೇವ ಚ \hfill॥ ೪~॥
\end{shloka}

\begin{shloka}
ಅಹಿಂಸಾ ಸಮತಾ ತುಷ್ಟಿಸ್ತಪೋ ದಾನಂ ಯಶೋಽಯಶಃ\\ಭವಂತಿ ಭಾವಾ ಭೂತಾನಾಂ ಮತ್ತ ಏವ ಪೃಥಗ್ವಿಧಾಃ \hfill॥ ೫~॥
\end{shloka}

\begin{artha}
ಬುದ್ಧಿ, ಜ್ಞಾನ, ಅಸಂಮೋಹ, ಕ್ಷಮೆ, ಸತ್ಯ, ಬಾಹ್ಯೇಂದ್ರಿಯನಿಗ್ರಹ, ಅಂತರಿಂದ್ರಿಯನಿಗ್ರಹ, ಸುಖ, ದುಃಖ, ಉತ್ಪತ್ತಿ, ವಿನಾಶ, ಭಯ, ಅಭಯ, ಅಹಿಂಸೆ, ಸಮಚಿತ್ತತೆ, ತುಷ್ಟಿ, ತಪಸ್ಸು, ದಾನ, ಯಶಸ್ಸು, ಅಯಶಸ್ಸು—ಈ ನಾನಾ ವಿಧವಾದ ಭಾವಗಳು ಪ್ರಾಣಿಗಳಿಗೆ ನನ್ನಿಂದಲೇ ಉಂಟಾಗುವುವು.
\end{artha}

ಮೇಲೆ ಹೇಳಿರುವ ನಾನಾ ಭಾವಗಳೆಲ್ಲ ಪ್ರಾಣಿಗಳಿಗೆ ಭಗವಂತನಿಂದಲೇ ಉಂಟಾಗುತ್ತವೆ. ಹಾಗಾದರೆ ಅವನು ಕೆಲವರಿಗೆ ಒಳ್ಳೆಯದನ್ನು ಏತಕ್ಕೆ ಕೊಡಬೇಕು, ಕೆಲವರಿಗೆ ಕೆಟ್ಟದ್ದನ್ನು ಏತಕ್ಕೆ ಕೊಡಬೇಕು? ಹಾಗೆ ಕೊಟ್ಟರೆ ಅವನು ಪಕ್ಷಪಾತಿ ಆಗುತ್ತಾನಲ್ಲ ಎಂದು ನಾವು ಕೇಳಬಹುದು. ಅವನು ತಾನೇ ಸ್ವೇಚ್ಛೆಯಿಂದ ಕೊಡುವುದಿಲ್ಲ. ನಮ್ಮ ಕರ್ಮಾನುಸಾರ ಆಯಾ ಫಲಗಳನ್ನು ಕೊಡುತ್ತಾನೆ. ಅವನೇ ಕೊಟ್ಟದ್ದು ನಿಜ, ಆದರೆ ಹಾಗೆ ಕೊಡುವಂತೆ ಮಾಡಿದ್ದು ಯಾವುದು? ನಮ್ಮ ಕರ್ಮಗಳು. ನ್ಯಾಯಾಧಿಪತಿ ಶಿಕ್ಷೆಯನ್ನು ವಿಧಿಸುತ್ತಾನೆ. ಅವನಿಗೆ ಅಂತಹ ಶಿಕ್ಷೆಯನ್ನು ನ್ಯಾಯಾಧಿಪತಿ ಏತಕ್ಕೆ ವಿಧಿಸುತ್ತಾನೆ ಎಂದರೆ ಅವನು ಅಂತಹ ಕರ್ಮ ಮಾಡಿರುವುದರಿಂದ ಅಂತಹ ಶಿಕ್ಷೆಯನ್ನು ವಿಧಿಸುತ್ತಾನೆ. ಹಾಗೆಯೇ ಸುಮ್ಮನೆ ದೇವರನ್ನು ದೂರಿದರೆ ಪ್ರಯೋಜವಿಲ್ಲ. ನಾವು ಮಾಡಿದಂತೆ ನಾವು ಅನುಭವಿಸಬೇಕಾಗುವುದು.

ಜೀವರಾಶಿಗಳಲ್ಲಿ ಅತ್ಯಂತ ಶ್ರೇಷ್ಠವಾದ ಮತ್ತು ಅತ್ಯಂತ ಕನಿಷ್ಠವಾದ ಗುಣಗಳನ್ನು ನೋಡುತ್ತೇವೆ. ಇದೆಲ್ಲ ಅವರು ತಮ್ಮ ಹೊಲದಲ್ಲಿ ತಾವೇ ಬಿತ್ತಿದ್ದು.

ಮನುಷ್ಯನಲ್ಲಿ ದೇವರು ಬುದ್ಧಿ ಕೊಟ್ಟು ಕಳುಹಿಸಿರುವನು. ಯಾರು ಅದನ್ನು ಚೆನ್ನಾಗಿ ಕೃಷಿ ಮಾಡಿರುವರೋ, ಅದು ವಿಕಾಸವಾಗುವುದಕ್ಕೆ ಶ್ರಮ ತೆಗೆದುಕೊಂಡಿರುವರೋ, ಅವರು ಕಣ್ಣಿಗೆ ಕಾಣದ ವಸ್ತುವಿನ ಅಂತರಾಳದಲ್ಲಿರುವ ಸೂಕ್ಷ್ಮ ವಸ್ತುಗಳನ್ನು ಮತ್ತು ಅವುಗಳನ್ನು ಆಳುವ ಜಗದ ನಿಯಮವನ್ನು ಕಂಡುಹಿಡಿಯಬಲ್ಲರು. ಸಣ್ಣ ಒಂದು ಗ್ರಹದ ಮೇಲೆ ಇರುವ ಮನುಷ್ಯ ಚಂದ್ರನಲ್ಲಿ ಏನಿದೆ, ಇತರ ಗ್ರಹಗಳಲ್ಲಿ ಏನಿದೆ ಎಂಬುದನ್ನು ಕಂಡುಹಿಡಿಯುತ್ತಾನೆ, ತನ್ನ ಬುದ್ಧಿ ಶಕ್ತಿಯ ಸಾಮರ್ಥ್ಯದಿಂದ. ಮತ್ತೊಬ್ಬ ಬುದ್ಧಿಯ ಬೀಜವನ್ನು ಚೆನ್ನಾಗಿ ಕೃಷಿ ಮಾಡಲಿಲ್ಲ. ಅವನು ದಡ್ಡ. ಯಾರು ಅದನ್ನು ಕೃಷಿ ಮಾಡಿರುವರೋ ಅವನು ಬುದ್ಧಿವಂತ. ಅಗಾಧವಾದ ಶಕ್ತಿಯನ್ನು ಅವನು ಚಲಾಯಿಸುತ್ತಾನೆ. ಹುಲಿ ಸಿಂಹಗಳಿಗಿಂತ, ಆಕಾಶದಲ್ಲಿ ಹಾರುವ ಹಕ್ಕಿಗಳಿಗಿಂತ, ನೀರಲ್ಲಿರುವ ಮೀನುಗಳಿಗಿಂತ ಅಗಾಧವಾದ ಕ್ರಿಯೆಗಳನ್ನು ಅವನು ಬುದ್ಧಿಯ ಸಹಾಯದಿಂದ ಮಾಡುತ್ತಾನೆ. ಈ ಬುದ್ಧಿಯಿಂದಲೇ ಎಲ್ಲವನ್ನೂ ತಿಳಿಯುತ್ತಾನೆ, ಆಳುತ್ತಾನೆ.

\newpage

ವೈಜ್ಞಾನಿಕ ವಿಷಯಗಳನ್ನು, ವ್ಯಾವಹಾರಿಕ ವಿಷಯಗಳನ್ನು ತಿಳಿದುಕೊಳ್ಳುವುದು ಬುದ್ಧಿ. ಇದಕ್ಕಿಂತ ಸೂಕ್ಷ್ಮವಾಗಿರುವುದೇ ಜ್ಞಾನ. ಇದು ಅಗೋಚರವಾದ ಅತೀಂದ್ರಿಯ ವಿಷಯಗಳನ್ನು ತಿಳಿದುಕೊಳ್ಳಬಲ್ಲದು. ನಮ್ಮ ಕಣ್ಣಿನ ಹೊರಗೆ ಇರುವ ವಿಷಯಗಳನ್ನು ತಿಳಿದುಕೊಳ್ಳುವುದು ಬುದ್ಧಿ. ನಮ್ಮ ಇಂದ್ರಿಯದ ಹಿಂದೆ ನಮ್ಮ ವ್ಯಕ್ತಿತ್ವದ ಹಿಂದೆ ಇರುವುದನ್ನು ಬ್ರಹ್ಮಾಂಡವನ್ನೆಲ್ಲ ಆಳುವ ಪರಾತ್ಪರ ವಸ್ತುವನ್ನು, ತಿಳಿದುಕೊಳ್ಳುವುದು ಮತ್ತು ಅದನ್ನು ಹೇಗೆ ಪಡೆಯಬೇಕೆಂಬುದನ್ನು ಅರಿಯುವುದು ಜ್ಞಾನ.

ಅಸಂಮೋಹ ಎಂದರೆ ಮೋಹವಿಲ್ಲದೆ ಇರುವುದು. ಬಹುಪಾಲು ಜನ ಮೋಹದ ಮುಲಾ\-ಮನ್ನು ಕಣ್ಣಿಗೆ ಹಾಕಿಕೊಂಡು ನೋಡುತ್ತಾರೆ. ಆಗ ಅಲ್ಲಿ ಇಲ್ಲದಿರುವುದು ಕಾಣುವುದು; ಇರುವುದು ಕಾಣುವುದಿಲ್ಲ. ಒಂದು ಇದ್ದರೆ ಮತ್ತೊಂದು ನಮಗೆ ಕಾಣುವುದು. ನಾವು ಯಾವಾಗಲೂ ಸತ್ಯವನ್ನು ತಿಳಿದುಕೊಳ್ಳಬೇಕೆಂದು ಬಯಸುವುದಿಲ್ಲ. ಏಕೆಂದರೆ ಸಮ್ಮೋಹ ಎಂಬ ಅಂಜನ ಕಣ್ಣಿಗೆ ಬಳಿದುಕೊಂಡಿರುವುದರಿಂದ ಹೇಗೆ ಇರಬೇಕೆಂದು ಬಯಸುತ್ತೇವೆಯೋ ಹಾಗೆ ಇದೆ ಎಂದು ಭಾವಿಸುತ್ತೇವೆ. ಭ್ರಮಿಸಿ ಅದರ ಬಳಿಗೆ ಹೋಗುತ್ತೇವೆ. ಮಾರೀಚನಲ್ಲಿ ಸುಂದರವಾದ ಮೃಗವನ್ನು ನೋಡುತ್ತೇವೆ. ಅಲ್ಲಿರುವ ರಾಕ್ಷಸ ಕಾಣುವುದು ಅವನು ನಮ್ಮ ಕೈಗೆ ಸಿಕ್ಕಿದಾಗಲೆ. ಈ ವ್ಯಾಮೋಹ ನಮ್ಮನ್ನು ಮೆಟ್ಟಿಕೊಂಡಿರುವುದರಿಂದಲೇ ಸಂಸಾರದ ಮರಳುಗಾಡಿನಲ್ಲಿ ಸುಂದರವಾದ ಮರೀಚಿಕೆಗಳನ್ನು ನೋಡುತ್ತೇವೆ. ಅದನ್ನು ಅರಸಿಕೊಂಡು ಹೊರಟರೆ ಯಾವ ಸರೋವರವೂ ಕಾಣುವುದಿಲ್ಲ, ಪ್ರಾಣಿಗಳೂ ಕಾಣುವುದಿಲ್ಲ. ನಮ್ಮಂತೆ ಅದನ್ನು ಹುಡುಕಿಕೊಂಡು ಹೊರಟ ಜನರ ಅಸ್ತಿಪಂಜರಗಳು ಮಾತ್ರ ಕಾಣುವುವು.

ಕ್ಷಮೆ, ಮನುಷ್ಯನಲ್ಲಿರುವ ಒಂದು ಅತ್ಯಂತ ಹಿರಿಯ ಗುಣ. ಯಾರು ನನಗೆ ಎಂತಹ ಅಪರಾಧವನ್ನು ಮಾಡಿದ್ದರೂ ಏನೊ ಅವನು ಮನೋದೌರ್ಬಲ್ಯದ ಸಮಯದಲ್ಲಿ ಅದನ್ನು ಮಾಡಿದನು; ಅವನು ನನಗೆ ಕೆಟ್ಟದ್ದನ್ನು ಮಾಡಿದರೂ ನಾನು ಅವನಿಗೆ ಒಳ್ಳೆಯದನ್ನು ಆಶಿಸುತ್ತೇನೆ ಎಂಬ ಔದಾರ್ಯ ಮನುಷ್ಯನನ್ನು ದೇವನಂತೆ ಮಾಡುವುದು. ಪೆಟ್ಟಿಗೆ ಪೆಟ್ಟು, ಗಾಯಕ್ಕೆ ಗಾಯ, ಅದು ಪ್ರಾಣಿ ಪ್ರಪಂಚದ ನೀತಿ. ದೇವನೆಡೆಗೆ ಹೋಗ ಬಯಸುವವನ ಆದರ್ಶ ಅದಲ್ಲ. ತನಗೆ ಎಂತಹ ಕೆಟ್ಟದ್ದನ್ನು ಮಾಡಿದ್ದರೂ ಅದನ್ನು ಕ್ಷಮಿಸುವ ಶಕ್ತಿ ಇದ್ದರೆ ಮಾತ್ರ ಭಗವಂತನ ಸಮೀಪಕ್ಕೆ ಹೋಗಲು ಸಾಧ್ಯ.

ಸತ್ಯವನ್ನೇ ಯಾವಾಗಲೂ ಹೇಳಬೇಕು. ಕೆಲವು ವೇಳೆ ಅದರ ಪರಿಣಾಮ ಮೊದಲಲ್ಲಿ ಕಹಿಯಾಗಿರಬಹುದು. ಆದರೆ ಮೊದಲು ಕಹಿ ತಿಂದು ಸಿಹಿಗೆ ಕಾಯುವುದು ಮೇಲು. ಜೀವನದಲ್ಲಿ ನಿಜ ಎಂದಿದ್ದರೂ ಬಯಲಾಗಲೇಬೇಕು. ನಾವು ಅದನ್ನು ಬಚ್ಚಿಡಲು ಆಗುವುದಿಲ್ಲ. ಬಚ್ಚಿಡಲು ಯತ್ನಿಸಿದರೆ ಅದಕ್ಕಾಗಿ ಎಷ್ಟೊಂದು ಶಕ್ತಿ ವ್ಯಯವಾಗುವುದು! ಯಾವ ಸಮಯದಲ್ಲಿ ಯಾರಿಂದ\break ಇದು ಬಯಲಾಗಬಹುದೊ ಎಂಬ ಕಳವಳ ಒಳಗೆ ಕುದಿಯುತ್ತಲೇ ಇರುವುದು. ಮನಸ್ಸಿನಲ್ಲಿ ದೊಡ್ಡದೊಂದು ಬಿರುಗಾಳಿ ಏಳುವುದು. ಸತ್ಯವನ್ನು ಬಚ್ಚಿಡಲು ಯತ್ನಿಸಿ ಮನುಷ್ಯ ತನ್ನ\break ಸ್ವಾಸ್ಥ್ಯವನ್ನು ಕೆಡಿಸಿಕೊಳ್ಳುತ್ತಾನೆ. ಹುಚ್ಚನಾಗುವ ಸಂಭವವೂ ಇದೆ. ಅದಕ್ಕೆ ಸತ್ಯವನ್ನು ಹೇಳಿ ಹೃದಯವನ್ನು ಹಗುರ ಮಾಡಿಕೊಳ್ಳುವುದು ಮೇಲು.

ಬಾಹ್ಯ ಇಂದ್ರಿಯವನ್ನು ಅದಕ್ಕೆ ಸಂಬಂಧಿಸಿದ ಎಲ್ಲ ವಿಷಯ ವಸ್ತುವಿನ ಸಂಪರ್ಕದಿಂದ ನಿಗ್ರಹಿಸಬೇಕು. ಆ ಸಂಪರ್ಕದಿಂದಲೇ ನಾವು ಅದಕ್ಕೆ ದಾಸರಾಗುವುದು. ಅದು ಹೇಳಿದಂತೆ ಕೇಳುವುದು. ಹಾಗೆ ಕೇಳಿ, ಹಾಗೆ ಮಾಡಿ, ಮತ್ತೊಂದು ಸಂಸ್ಕಾರದ ಸಾಲ ಮಾಡಿಕೊಂಡಿರುವೆವು ನಾವು ಆಗಲೆ. ಇದುವರೆಗೆ ಆಗಿರುವುದೇ ಸಾಕು. ಅದನ್ನು ತೀರಿಸುವುದಕ್ಕೆ ಹಲವು ಜನ್ಮಗಳು ಬೇಕಾಗುವುದು.ಇನ್ನು ಮೇಲಾದರೂ ಅದನ್ನು ನಿಲ್ಲಿಸುವ ಎಂದು ಮೊದಲೆ ಅದನ್ನು ತಡೆಗಟ್ಟಬೇಕು.

ಸುಮ್ಮನೆ ಹೊರಗಡೆ ವಿಷಯ ವಸ್ತುವಿನ ಕಡೆ ಹೋಗುವುದನ್ನು ತಡೆದರೆ ಸಾಲದು. ಮನಸ್ಸಿನಲ್ಲಿ ಅದಕ್ಕೆ ಸಂಬಂಧಪಟ್ಟ ಅನುಭವಗಳನ್ನು ಮೆಲಕುತ್ತಿರುವುದು ನಿಲ್ಲಬೇಕು. ಸ್ಥೂಲವಾಗಿ ನಿಗ್ರಹಿಸಿ, ಸೂಕ್ಷ್ಮವಾಗಿ ಮನಸ್ಸು ಅದನ್ನು ಚಪ್ಪರಿಸುತ್ತಿದ್ದರೆ, ಬಹಳ ಕಾಲ ಮನಸ್ಸು ಸುಮ್ಮನಿರಲು ಆಗುವು ದಿಲ್ಲ. ಸ್ಥೂಲವಾಗಿ ಆ ಕೆಲಸವನ್ನು ಕ್ರಮೇಣ ಮಾಡುವ ಪರಿಸ್ಥಿತಿಯೂ ಒದಗುವುದು. ಆದಕಾರಣವೆ ನಮ್ಮ ಪತನಕ್ಕೆ ಮೂಲ ಮನಸ್ಸಿನಲ್ಲಿದೆ. ಅಲ್ಲಿ ಅದನ್ನು ತಡೆಗಟ್ಟಬೇಕು. ಅದೊಂದು ಸಣ್ಣ ಅಲೆಯಂತೆ ಪ್ರಾರಂಭಿಸುವುದು. ಅದನ್ನು ನಿಗ್ರಹಿಸುವುದು ಆಗ ಸುಲಭ. ಹಾಗೆಯೆ ಬಿಟ್ಟರೆ ಅದೊಂದು ಮಹಾ ಪ್ರವಾಹ ಆಗುವುದು. ಆಗ ಎಂತಹ ಕಟ್ಟೆಯನ್ನು ಎದುರಿಗೆ ಇಟ್ಟರೂ ಕೊಚ್ಚಿಕೊಂಡು ಹೋಗುವುದು. ಪ್ರಾರಂಭದಲ್ಲಿ ಹೀನ ಆಲೋಚನೆಯ ಸುಳಿಯನ್ನು ಉಗುರಿನಲ್ಲಿ ಕಿತ್ತುಹಾಕಿಬಿಡಬಹುದು. ಯಾವಾಗ ಅದನ್ನು ಹಾಗೆಯೇ ಬಿಡುತ್ತೇವೆಯೊ, ಆಮೇಲೆ ಕೊಡಲಿ ಗರಗಸ ಮೊದಲಾದ ಮಹಾ ಆಯುಧಗಳಿಂದಲೂ ಅದನ್ನು ಕಡಿದು ಹಾಕಲು ಕಷ್ಟವಾಗುವುದು. ಇದನ್ನು ಅರಿತೆ ಜ್ಞಾನಿಯಾದವನು ಪ್ರಾರಂಭದಲ್ಲಿ ತನ್ನ ಮನಸ್ಸಿನಲ್ಲಿಯೇ ಅದನ್ನು ನಿಗ್ರಹಿಸುತ್ತಾನೆ. 

ಇಂದ್ರಿಯ ಸುಖ ಆಧ್ಯಾತ್ಮಿಕ ಸುಖ ಇವುಗಳೆಲ್ಲ ದೇವರಿಂದಲೇ ಆಗುವುದು. ಹಾಗೆಯೆ ದುಃಖ. ನಾವು ಮಾಡಬಾರದುದನ್ನು ಮಾಡಿದಾಗ ಮತ್ತು ಈ ಪ್ರಪಂಚದಲ್ಲಿ ಅನಿವಾರ್ಯವಾಗಿ ಎಲ್ಲ ಸುಖವನ್ನು ನೆರಳಂತೆ ಬೆನ್ನು ಹತ್ತಿ ಬರುವ ದುಃಖ ಕೂಡ ಅವನಿಂದಲೇ. ಇದರಂತೆಯೆ ಒಂದು ವಸ್ತು ಸೃಷ್ಟಿಯಾಗುವುದು ಕೂಡ ಅವನಿಂದಲೆ, ಮತ್ತು ಹೋಗುವುದು ಕೂಡ ಅವನಿಂದಲೆ. ಒಂದು ಕೈ ಕೊಡುವುದು, ಮತ್ತೊಂದು ಕೈ ತೆಗೆದುಕೊಳ್ಳುವುದು. ಕೊಡುವಾಗ ಕೈಯೊಡ್ಡಿ ಕುಣಿದಾಡುತ್ತಾ ತೆಗೆದುಕೊಳ್ಳುತ್ತೇವೆ. ಆದರೆ ಯಾವಾಗ ತೆಗೆದುಕೊಳ್ಳಲು ಬರುವನೊ, ಅದನ್ನು ಬಿಗಿದ ಮುಷ್ಟಿಯಲ್ಲಿ ಹಿಡಿದಿಟ್ಟುಕೊಳ್ಳುತ್ತೇವೆ. ಆದರೆ ಅವನು ತೆಗೆದುಕೊಂಡು ಹೋಗಲು\break ಬಂದಾಗ ಕಟುಕನಂತಿರುವನು. ಯಾರ ಬೇಡಿಕೆಯನ್ನೂ ಕೇಳುವುದಿಲ್ಲ, ಯಾರ ಕಣ್ಣೀರಿಗೂ ಕರಗುವುದಿಲ್ಲ. ಕೊಟ್ಟಿದ್ದನ್ನು ತನ್ನ ರೀತಿಯಲ್ಲಿಯೇ ತೆಗೆದುಕೊಂಡು ಹೋಗುವನು.

ಭಯ ಮತ್ತು ಅಭಯ ಕೂಡ ಅವನಿಂದಲೇ ಬರುವುದು. ಮುಂದೇನಾಗುವುದು ಎಂಬ ಭಯ ಇರಬಹುದು. ಅಯ್ಯೋ ನಾನು ಕೆಟ್ಟದ್ದನ್ನು ಮಾಡಬಾರದು, ಕೆಟ್ಟದ್ದನ್ನು ಮಾಡಿದರೆ ಅದು ಪುನಃ ನನ್ನ ಮೇಲೆ ಕೆಟ್ಟ ಪರಿಣಾಮದಂತೆ ಬೀಳುವುದು ಎಂಬ ಭಯ ಇರಬಹುದು. ಎಲ್ಲ\break ವಿಧವಾದ ಭಯಗಳು ಬರುವುದು ಅವನಿಂದಲೇ. ಹಾಗೆಯೇ ಅಭಯ ಕೂಡ ಅವನದೆ. ಯಾವುದಕ್ಕೂ ಜಗ್ಗದೆ, ಕುಗ್ಗದೆ ಇರುವುದು. ಜೀವನದಲ್ಲಿ ಏನನ್ನು ಬೇಕಾದರೂ ದಿಟ್ಟತನದಿಂದ ಎದುರಿಸುವ ಸಾಹಸ, ಎಂತಹ ವಿರುದ್ಧ ಅಲೆಗೂ ಕೊಚ್ಚಿಕೊಂಡು ಹೋಗದೆ ಎದೆಕೊಡುವ ಪೌರುಷ, ಇವೂ ಕೂಡ ಅವನಿಂದಲೇ ಬರುವುವು.

ಅಹಿಂಸೆ ಎಂದರೆ, ಯಾರಿಗೂ ಅವನು ಹಿಂಸೆಯನ್ನು ಕೊಡಲೆತ್ನಿಸುವುದಿಲ್ಲ. ಬೇಕಾದರೆ ತಾನು ಏನು ಬೇಕಾದರೂ ಅನುಭವಿಸುತ್ತಾನೆ. ಇನ್ನೊಬ್ಬನಿಗೆ ಸ್ವಲ್ಪವೂ ತೊಂದರೆಯನ್ನುಂಟು ಮಾಡಲು ಯತ್ನಿಸುವುದಿಲ್ಲ. ಇವನು ಪೆಟ್ಟಿಗೆ ಪೆಟ್ಟು ಎಂಬ ನೀತಿಯಿಂದ ಮೇಲೆ ಇರುವನು. ಹಿಂಸೆಗಿಂತ ಅಹಿಂಸೆ ಅದ್ಭುತ ಪರಿಣಾಮಕಾರಿಯಾದುದು. ಆದರೆ ಎಲ್ಲರಿಗೂ ಅಹಿಂಸೆಯ ಬ್ರಹ್ಮಾಸ್ತ್ರವನ್ನು ಉಪಯೋಗಿಸುವ ಯೋಗ್ಯತೆ ಇಲ್ಲ ಅಷ್ಟೆ. ಇದು ಹೇಡಿತನವಲ್ಲ. ವೀರಾಧಿವೀರನ ಧೈರ್ಯ ಮತ್ತು ಸಹಿಷ್ಣತೆ ಇದಕ್ಕೆ ಬೇಕು. ಸಮಚಿತ್ತತೆ ಎಂದರೆ ಎಲ್ಲರನ್ನೂ ಅವನು ಒಂದೇ ರೀತಿಯಲ್ಲಿ ನೋಡುತ್ತಾನೆ. ಅವನು ನಾಮ ರೂಪಗಳ ಹಿಂದೆ ತೂರಿ ಎಲ್ಲಾ ಕಡೆಯಲ್ಲಿಯೂ ಒಂದೇ ಸಮನಾಗಿರುವ ಪರಮಾತ್ಮನನ್ನು ನೋಡುತ್ತಾನೆ ಅಥವಾ ಎಲ್ಲಾ ಒಂದೇ ಭಗವಂತನ ಕುಟುಂಬಕ್ಕೆ ಸೇರಿದವರು ಎಂಬ ದೃಷ್ಟಿಯಿಂದಲಾದರೂ ನೋಡಬಹುದು. ಅಂತೂ ಸಮದೃಷ್ಟಿ ಬರಬೇಕಾದರೆ ಭಗವಂತನ ಭಾವನೆ ಬರಬೇಕು. ಆಗ ಮಾತ್ರ ಇದು ಸಾಧ್ಯ. ಎಲ್ಲರ ಹಿಂದೆಯೂ ದೇವರು ಇರುವನು ಎಂದಾದರೂ ನೋಡಬಹುದು; ಎಲ್ಲರೂ ದೇವರಿಗೆ ಸೇರಿದವರು ಎಂದಾದರೂ ನೋಡಬಹುದು.

ತೃಪ್ತಿ ಕೂಡ ಅವನಿಂದಲೇ ಬರುವುದು. ಜೀವನದಲ್ಲಿ ಅಪೂರ್ವ ವರ ಇದು. ಇರುವವರಿಗೆಲ್ಲ ತೃಪ್ತಿ ಬರುವುದಿಲ್ಲ. ಇರುವವನಿಗೆ ಈಗ ಇರುವುದು ಸಾಲದು. ಇನ್ನೂ ಬೇಕು ಎಂಬ ಆಸೆ ಯಾವಾಗಲೂ ಕಾಡುತ್ತಿರುವುದು. ಭಿಕ್ಷುಕನಿಂದ ಲಕ್ಷಾಧೀಶ್ವರನವರೆಗೆ ಇದನ್ನು ನೋಡುತ್ತೇವೆ. ಆದರೆ ಎಲ್ಲಿ ಜೀವಿ ದೇವರು ಕೊಟ್ಟಿದ್ದರಲ್ಲಿ ತೃಪ್ತನಾಗಿರುವನೊ, ಈ ಜೀವನದಲ್ಲಿ ಆಸೆಗೆ ಪಾರವಿಲ್ಲ ಎಂದರಿತಿರುವನೊ, ಅವನೊಂದು ಬಡ ಕುಟೀರದಲ್ಲಿರಬಹುದು; ಅವನಿಗೆ ತೃಪ್ತಿ ಇದೆ. ಆದರೆ ಅರಮನೆಯಲ್ಲಿದ್ದರೂ, ಬೇಕಾದಷ್ಟು ಇದ್ದರೂ ಯಾರಿಗೆ ಅತೃಪ್ತಿ ಇದೆಯೊ ಅವನು ಎಲ್ಲಿದ್ದರೂ ಭಿಕಾರಿ. ಏಕೆಂದರೆ ಅವನಿನ್ನೂ ಬೇಡುತ್ತಿರುವನು. ಯಾರಿಗೆ ಏನೂ ಬೇಕಾಗಿಲ್ಲವೊ ಇರುವುದರಲ್ಲಿ ತೃಪ್ತನಾಗಿರುವನೊ ಅವನಿಗೆ ಏನೂ ಇಲ್ಲದೆ ಇದ್ದರೂ ಅವನಿಗೆ ಕೊರತೆಗಳು ಇಲ್ಲ.

ತಪಸ್ಸು ಕೂಡ ಭಗವಂತನ ವರವೆ. ನಮ್ಮ ಮನಸ್ಸನ್ನೆಲ್ಲಾ ಏಕಾಗ್ರ ಮಾಡಿ ನಾವು ಪಡೆಯ ಬೇಕೆಂಬ ವಸ್ತುವಿನ ಕಡೆ ಹರಿಸಬೇಕು. ಅದಕ್ಕಾಗಿ ಎಲ್ಲವನ್ನೂ ಮರೆಯಬೇಕು. ಎಂತಹ ಕಷ್ಟವ ನ್ನಾದರೂ ಸಹಿಸಬೇಕು. ಜೀವನದ ಅತ್ಯಂತ ರಹಸ್ಯವಾದ ವಸ್ತುಗಳನ್ನು ಪಡೆಯುವುದಕ್ಕೆ ಇರುವು ದೊಂದೇ ಮಾರ್ಗ. ಅದೇ ತಪಸ್ಸು. ವಿಜ್ಞಾನ ಪ್ರಪಂಚದಲ್ಲಿ ಹುದುಗಿರುವ ನಿಯಮಗಳನ್ನು ಕಂಡು ಹಿಡಿಯಬೇಕಾದರೆ ವರುಷ ವರುಷಗಳು ಅದಕ್ಕಾಗಿ ಎಲ್ಲವನ್ನೂ ಅರ್ಪಣೆ ಮಾಡಿ ತದೇಕ ಧ್ಯಾನದಲ್ಲಿ ನಿರತನಾಗಬೇಕು. ಆಗಲೆ ಒಬ್ಬ ಗ್ಯಾಲಿಲಿಯೊ, ನ್ಯೂಟನ್, ಐನ್​ಸ್ಟಿನ್ ಮುಂತಾದವರು ಪ್ರಕೃತಿಯ ರಹಸ್ಯದ ಕಡಲಿನಿಂದ ಒಂದು ಮುತ್ತನ್ನು ತರಬೇಕಾದರೆ. ಆಧ್ಯಾತ್ಮಿಕ ಜೀವನ ಕೂಡ ಹೀಗೆ. ಆತ್ಮವಿಚಾರದಲ್ಲಿ ತನ್ಮಯರಾಗಬೇಕು. ಆಗಲೆ ಒಂದು ಅನುಭವ ದೊರಕ ಬೇಕಾದರೆ. ಈ ಜೀವನದಲ್ಲಿ ಅಪೂರ್ವದ್ದಾಗಿರುವುದು ಅನರ್ಘ್ಯವಾಗಿರುವುದು ಯಾವುದೂ ಬಿಟ್ಟಿ ಸಿಕ್ಕುವುದಿಲ್ಲ. ಅದಕ್ಕೆ ತಪಸ್ಸಿನ ಬೆಲೆಯನ್ನು ಕೊಡಬೇಕು. ಆಗಲೆ ಅದು ನಮಗೆ ಸಿಕ್ಕಬೇಕಾದರೆ.

ದಾನ ಮಾಡಬೇಕೆಂಬ ಆಸೆ ಒಬ್ಬನಲ್ಲಿ ಇರಬೇಕಾದರೆ ಅದು ಭಗವಂತನ ಒಂದು ಅಪೂರ್ವ ವರವೇ ಸರಿ. ಇರುವವರಿಗೆಲ್ಲ ಕೊಡಬೇಕೆಂಬ ಬುದ್ಧಿ ಬರುವುದಿಲ್ಲ. ಅವನು ಬದುಕಿರುವಾಗ ತಾನು ತಿನ್ನಬೇಕು ಅನಂತರ ತನ್ನವರು ತಿನ್ನಬೇಕು, ತನ್ನವರು ಇಲ್ಲದೇ ಇದ್ದರೆ ಯಾರನ್ನಾದರೂ ದತ್ತು ತೆಗೆದುಕೊಂಡಾದರೂ ಅವರಿಗೆ ತನ್ನ ಆಸ್ತಿಯನ್ನೆಲ್ಲ ಕಟ್ಟುವನೆ ಹೊರತು ಎಲ್ಲರಿಗೂ ಒಳ್ಳೆಯ\-ದಾಗುವ ಒಂದು ಕಾರ್ಯಕ್ಕೆ ಕೊಡಲು ಮನಸ್ಸು ಬರುವುದಿಲ್ಲ. ಈ ಜೀವನದಲ್ಲಿ ನಮ್ಮ ಹಿಂದೆ ಬರುವುದೇ ಈಗ ಕೊಟ್ಟದ್ದು. ಅದೊಂದೇ ಅಲ್ಲ. ಜೀವನದಲ್ಲಿ ಕೊಡುವುದರಲ್ಲಿ ಒಂದು ಆನಂದವಿದೆ. ನಾನೇ ಅದನ್ನು ಅನುಭವಿಸುವುದಕ್ಕಿಂತ ಹಿರಿಯ ಆನಂದ ಕೊಡುವುದರಲ್ಲಿದೆ. ಜೀವನದಲ್ಲಿ ಎಲ್ಲ ಕೊಡುವುದಕ್ಕಾಗಿಯೆ ಜನ್ಮತಾಳಿದಂತಿದೆ. ಎಲ್ಲಾ ಕಡೆಯಲ್ಲಿಯೂ ಈ ನೀತಿಯನ್ನೆ ಪ್ರಕೃತಿ ಸಾರುತ್ತಿದೆ. ಮರ ಹಣ್ಣುಕೊಡುವುದು, ಗಿರಿ ಮೋಡವನ್ನು ತಡೆದು ಮಳೆ ಕೊಡುವುದು, ನದಿ ನೀರನ್ನು ಕೊಡುವುದು. ಸಾಗರ ನದಿಗೆಲ್ಲ ನೀರು ಕೊಡುವುದು. ಈ ಪ್ರಪಂಚದಲ್ಲಿರುವುದೆಲ್ಲ ಭೂಮಕ್ಕೆ ತನ್ನನ್ನು ಅರ್ಪಣೆ ಮಾಡಿಕೊಳ್ಳುವುದನ್ನು ನೋಡುತ್ತೇವೆ. ದಾನಿಯಾದವನು ಈ ನಿಯಮವನ್ನು ಅರಿತವನು, ಈ ಆನಂದವನ್ನು ಅರಿತವನು. ಕೊಡುವ ಸದವಕಾಶ ಬಂದಾಗ ಅದನ್ನು ಕಳೆದುಕೊಳ್ಳುವುದಿಲ್ಲ. ಕೊಡುವವರೆಲ್ಲ ದಾನಿಗಳಲ್ಲ. ಎಷ್ಟೊ ಜನ ಹೆಸರಿಗೆ ಕೊಡುವರು, ಕೀರ್ತಿಗೆ ಕೊಡುವರು. ಈಗ ಕೊಟ್ಟರೆ ಬಡ್ಡಿ ಸಮೇತ ನಮಗೆ ಹಿಂದಿರುಗಿ ಬರುವುದು ಎಂದು ಕೊಡುವರು. ಈ ಜೀವನದಲ್ಲಿ ಕೊಡುವುದೊಂದು ದೊಡ್ಡ ಕಲೆ. ಅವನು ಯಾರಿಗೂ ತಿಳಿಯದೆ ಕೊಡುವನು. ಲಜ್ಜೆಯಿಂದ ಕೊಡುವನು, ತನ್ನ ಗೌರವಕ್ಕೆ ತಕ್ಕಂತೆ ಕೊಡುವನು. ಅವನು ಕೊಡುವಾಗ ಯಾವ ಷರತ್ತನ್ನೂ ಹಾಕುವುದಿಲ್ಲ. ಮಲ್ಲಿಗೆ ಕಂಪನ್ನು ದಿಸೆದಿಸೆಗೂ ಎಸೆಯುವಂತೆ ಅವನು ಕೊಡುವನು. ಯಾರಿಂದ ಅದಕ್ಕೆ ಏನೂ ಬೇಕಾಗಿಲ್ಲ. ಕೊಡುವುದು ಅದರ ಧರ್ಮ, ಅದಕ್ಕಾಗಿ ಕೊಡುವುದು. ಅಲ್ಲಿಗೆ ಅದು ಕೊನೆಗೊಳ್ಳುವುದು.

ಯಶಸ್ಸು ಬರುವುದು ಅವನಿಂದಲೆ. ಜೀವನದಲ್ಲಿ ಒಂದು ಒಳ್ಳೆಯ ಕೆಲಸವನ್ನು ಮಾಡಿದಾಗ ಕೀರ್ತಿ ಇವನನ್ನು ಹುಡುಕಿಕೊಂಡು ಬರುವುದು. ಈ ಪ್ರಪಂಚದಲ್ಲಿ ಎರಡು ಬಗೆಯ ಮನುಷ್ಯರು ಇರುವರು. ಮೊದಲನೆಯ ಮನುಷ್ಯ ಕೀರ್ತಿಯನ್ನು ಹುಡುಕಿಕೊಂಡು ಹೋಗುವನು. ಇಂಥವರೆ ಬಹಳಮಂದಿ. ತುಂಬ ಶ್ರಮಪಟ್ಟು ಕೀರ್ತಿ ಗಳಿಸುವುದಕ್ಕಾಗಿ ಹಗಲು ರಾತ್ರಿ ಏನನ್ನಾದರೂ ಮಾಡುತ್ತಿರುವರು. ಮತ್ತೊಂದು ಬಗೆಯ ಜನರಿರುವರು. ಅವರು ಕೀರ್ತಿಯನ್ನು ಗಳಿಸಬೇಕೆಂದು ಮನಸ್ಸನ್ನು ಮಾಡುವುದಿಲ್ಲ. ಯಾವುದೊ ಒಂದು ಅಸಾಧ್ಯವಾದುದನ್ನು ಸ್ಥಾಪಿಸುವರು. ಮಾನವ ಕೋಟಿ ಅವರಿಗೆ ಕೃತಜ್ಞತೆಯನ್ನು ಅರ್ಪಿಸುವುದು. ಆಚಂದ್ರಾರ್ಕವಾಗಿ ಅವರನ್ನು\break ಹೊಗಳುವುದು. ಕೀರ್ತಿ ಇಂತಹ ವ್ಯಕ್ತಿಗಳನ್ನು ಹುಡುಕಿಕೊಂಡು ಬೇಡವೆಂದರೂ ತನ್ನ ಕಿರೀಟವನ್ನು ಅವರ ತಲೆಯಮೇಲೆ ಇಡುವುದು. ಈ ಗುಂಪಿಗೆ ಸೇರಿದವರು ವಿಭೂತಿಪುರುಷರು. ಇಂತಹವರ ಸಂಖ್ಯೆ ಕಡಮೆ. ಆದರೆ ಇವರೇ ಜಗದ ಜೀವಾಳ. ಸೃಷ್ಟಿಯ ವೃಕ್ಷದಲ್ಲಿ ಬಿಟ್ಟಿರುವ ಅನರ್ಘ್ಯವಾದ ಫಲಪುಷ್ಪಗಳು ಇವರು.

ಯಶಸ್ಸಿನಂತೆಯೇ ಜೀವನದಲ್ಲಿ ಅಪಯಶಸ್ಸು ಕೂಡ ಆಚಂದ್ರರ್ಕವಾಗಿ ನಿಲ್ಲುವುದು. ಒಬ್ಬ ದುರ್ಯೋಧನ, ರಾವಣ, ಧರ್ಮರಾಯನಷ್ಟೇ ಶ‍್ರೀರಾಮನಷ್ಟೇ ಅಮರತ್ವವನ್ನು ಗಳಿಸಿರುವರು. ಈ ವ್ಯಕ್ತಿಗಳ ಮೂಲಕವಾಗಿಯೂ ದೇವರು ಒಂದು ದೊಡ್ಡ ನೀತಿಯನ್ನು ಎಲ್ಲರಿಗೂ ಕಲಿಸುತ್ತಿರುವನು. ದುರ್ಯೋಧನನ ದಾರಿ ಹಿಡಿದರೆ, ರಾವಣನ ಹಾದಿಯನ್ನು ಹಿಡಿದರೆ, ನಮಗೂ ಅದೇ ಗತಿ ಎಂಬುದನ್ನು ಅಪಕೀರ್ತಿಯ ಮೂಲಕವೂ ದೇವರು ಸಾರುತ್ತಿರುವನು.

ಎಲ್ಲ ಬಗೆಯ ಸುಗುಣ ದುರ್ಗುಣಗಳ ಬೀಜವೂ ಪ್ರಕೃತಿಯಲ್ಲಿವೆ. ಅವೆಲ್ಲ ಮೊಳೆಯ\-ಬೇಕಾದರೆ, ಫಲಕೊಡಬೇಕಾದರೆ, ಭಗವಂತ ಮಳೆಯಂತೆ ಬಿದ್ದರೆ ಮಾತ್ರ ಸಾಧ್ಯ. ಈ ದೃಷ್ಟಿಯಿಂದ ಎಲ್ಲ ಬರುವುದು ಅವನಿಂದ. ಜೀವರಾಶಿಗಳ ಕರ್ಮಾನುಸಾರ ಬೆಳೆ ಬರುವುದಕ್ಕೆ ಅವನು ಮಳೆಗರೆಯುವನು.

\begin{shloka}
ಮಹರ್ಷಯಃ ಸಪ್ತ ಪೂರ್ವೇ ಚತ್ವಾರೋ ಮನವಸ್ತಥಾ~।\\ಮದ್ಭಾವಾ ಮಾನಸಾ ಜಾತಾ ಯೇಷಾಂ ಲೋಕ ಇಮಾಃ ಪ್ರಜಾಃ \hfill॥ ೬~॥
\end{shloka}

\begin{artha}
ಪೂರ್ವಕಾಲದ ಸಪ್ತಮಹರ್ಷಿಗಳು ಮತ್ತು ನಾಲ್ಕು ಜನ ಮನುಗಳು ನನ್ನ ಭಾವವುಳ್ಳವರಾಗಿ ನನ್ನ ಸಂಕಲ್ಪದಿಂದಲೇ ಜನಿಸಿದರು. ಲೋಕದಲ್ಲಿ ಈ ಪ್ರಜೆಗಳೆಲ್ಲ ಅವರಿಂದಲೇ ಉತ್ಪನ್ನರಾದರು.
\end{artha}

ಕಲ್ಪದ ಆದಿಯಲ್ಲಿ ಸಪ್ತಮಹರ್ಷಿಗಳು ಇದ್ದರು. ಅವರು ಮಹಾಜ್ಞಾನಿಗಳು. ಭಗವಂತನ ಇಚ್ಛೆಯಿಂದಲೇ ಅವರು ಅವತರಿಸಿದರು. ಅವರು ಅವನ ಭಾವದಂತೆಯೇ ಇದ್ದರು. ಅಗ್ನಿಕುಂಡದಿಂದ ಸಿಡಿದ ಕಿಡಿ ಹೇಗೋ ಹಾಗೆ ಪರಬ್ರಹ್ಮನ ಭಾವವೇ ಅವರಲ್ಲಿತ್ತು. ಆ ಸಪ್ತಮಹರ್ಷಿಗಳು ಭಗವಂತನಿಂದ ಬಂದವರು, ಅವನನ್ನು ತಿಳಿದುಕೊಂಡವರು. ನಾಲ್ಕು ಜನ ಮನುಗಳು, ಒಂದೊಂದು ಯುಗಕ್ಕೆ ಒಂದೊಂದು ಧರ್ಮವನ್ನು ಕೊಟ್ಟವರು ಎಂದು ಬೇಕಾದರೆ ತಿಳಿದುಕೊಳ್ಳಬಹುದು. ಒಂದು ಸಮಾಜ ಭದ್ರವಾಗಿರಬೇಕಾದರೆ ಅದಕ್ಕೆ ಕಾಯಿದೆ ಕಾನೂನುಗಳು ಇರಬೇಕು, ವಿಧಿ ನಿಷೇಧಗಳಿರಬೇಕು. ಅದನ್ನು ಕೊಡಬೇಕಾದರೆ ಸಮಾಜವನ್ನೆಲ್ಲ ಚೆನ್ನಾಗಿ ಬಲ್ಲವನಾಗಿರಬೇಕು. ಅದರ ಗುರಿ ಏನು ಎಂಬುದನ್ನು ಚೆನ್ನಾಗಿ ಅರಿತು ಅದಕ್ಕೆ ತಕ್ಕಂತೆ ಕರ್ತವ್ಯಗಳನ್ನು ವಿಧಿಸುವವರೆ ಧರ್ಮಶಾಸ್ತ್ರ ಕರ್ತೃಗಳು. ಸಪ್ತಪುಷಿಗಳು ಕೇವಲ ನಿವೃತ್ತಿ ಮಾರ್ಗಾವಲಂಬಿಗಳು. ಭಗವಂತನ ಕಡೆಗೆ ನೇರವಾಗಿ ಹೋಗಬೇಕು ಎಂಬ ದಾರಿಯನ್ನು ತೋರುತ್ತಾರೆ. ಆದರೆ ಈ ದಾರಿಯಲ್ಲಿ ನಡೆಯವವರು ಬಹಳ ಅಲ್ಪ ಮಂದಿ. ಮಹಾಧೀರರಿಗೆ ಮಾತ್ರ ಸಾಧ್ಯ ಇದು. ಮತ್ತೊಂದು ದಾರಿ ಇದೆ. ಇದೇ ಪ್ರವೃತ್ತಿ ಮಾರ್ಗ. ಧರ್ಮ ಅರ್ಥ ಕಾಮದ ಮೂಲಕ ಮೋಕ್ಷಕ್ಕೆ ಹೋಗುವುದು. ಇಲ್ಲಿ ಯಾವುದನ್ನೂ ನಿರಾಕರಿಸದೆ ಅನುಭವಗಳನ್ನು ಪಡೆದುಕೊಂಡು ಮೋಕ್ಷವೆಂಬ ಗುರಿಯ ಕಡೆಗೆ ಸೇರುವುದು. ಈ ಮಾರ್ಗದ ನಿಯಮಾವಳಿಗಳನ್ನು ಮನುಶಾಸ್ತ್ರದಲ್ಲಿ ನೋಡುತ್ತೇವೆ. ಒಂದು ಎಲ್ಲೂ ನಿಲ್ಲದೆ ಮೈಸೂರಿನಿಂದ ಬೆಂಗಳೂರಿಗೆ ಹೋಗುವ ಬಸ್ಸಿನಂತೆ. ಇದು ‘ನಾನ್​\-ಸ್ಟಾಪ್​’ ಬಸ್ಸು. ಇದೇ ಸಪ್ತಪುಷಿಗಳ ಮಾರ್ಗ. ಮತ್ತೊಂದು ಕೈ ತೋರಿದ ಎಡೆಯಲ್ಲೆಲ್ಲಾ ನಿಂತು ನಿಂತು ಬೆಂಗಳೂರನ್ನು ತಲಪುವ ‘ಶಟಲ್​’ ಬಸ್ಸಿನಂತೆ. ಅದೂ ಬೆಂಗಳೂರನ್ನು ತಲಪುವುದು. ಆದರೆ ಸ್ವಲ್ಪ ನಿಧಾನವಾಗಿ. ಇದೇ ಪ್ರವೃತ್ತಿ ಮಾರ್ಗ. ಈ ಎರಡು ದಾರಿಗಳನ್ನು ತೋರಿದವರೂ ಭಗವಂತನ ಇಚ್ಛೆಯಿಂದಲೇ ಜನಿಸಿದರು. ಮಾನವಕೋಟಿಯನ್ನೇ ದೇವರೆಡೆಗೆ ಕರೆದುಕೊಂಡು ಹೋಗುವ ಮಾರ್ಗದರ್ಶಕರಂತೆ ಇರುವರು ಇವರು. ಮುಂದಿನ ಪ್ರಜೆಗಳೆಲ್ಲ ಇವರು ತೋರಿದ ದಾರಿಯಲ್ಲೇ ನಡೆದವರು. ಇವರಿಂದ ಉತ್ಪನ್ನರಾದವರು.

\begin{shloka}
ಏತಾಂ ವಿಭೂತಿಂ ಯೋಗಂ ಚ ಮಮ ಯೋ ವೇತ್ತಿ ತತ್ತ್ವತಃ~।\\ಸೋಽವಿಕಂಪೇನ ಯೋಗೇನ ಯುಜ್ಯತೇ ನಾತ್ರ ಸಂಶಯಃ \hfill॥ ೭~॥
\end{shloka}

\begin{artha}
ಯಾರು ಈ ನನ್ನ ವಿಭೂತಿಯನ್ನು ಮತ್ತು ಯೋಗವನ್ನು ಯಥಾರ್ಥವಾಗಿ ತಿಳಿದುಕೊಳ್ಳುವನೊ ಅವನು ಅಚಲವಾದ ಯೋಗದಿಂದ ಯುಕ್ತನಾಗುತ್ತಾನೆ. ಇದರಲ್ಲಿ ಸಂಶಯವಿಲ್ಲ. 
\end{artha}

ಭಗವಂತನ ವಿಭೂತಿ ಮತ್ತು ಯೋಗವನ್ನು ತಿಳಿದುಕೊಳ್ಳವುದು ಒಂದು ದೊಡ್ಡ ಸಾಧನೆ. ಯಾರು ಈ ಸಾಧನೆಯನ್ನು ಮಾಡುತ್ತಾರೊ ಅವರು ಕೂಡ ಭಗವಂತನನ್ನು ಮುಟ್ಟುವರು. ವಿಭೂತಿ ಅಂದರೆ ಅವನ ಮಹಿಮೆ, ಅವನ ಶಕ್ತಿ. ಅವನು ಈ ಬ್ರಹ್ಮಾಂಡವನ್ನು ಸೃಷ್ಟಿಸಿದವನು, ಇಲ್ಲಿ ಜೀವರಾಶಿಗಳು ತನ್ನೆಡೆಗೆ ಬರುವುದಕ್ಕೆ ದಾರಿ ತೋರಿದವನು. ಅವನು ಸರ್ವೇಶ್ವರ, ಎಲ್ಲದಕ್ಕೂ ಒಡೆಯ. ಅವನಾಜ್ಞೆಯನ್ನು ಯಾರೂ ಮೀರಲಾರರು. ಅವನು ಸೃಷ್ಟಿಸಿದವನು. ಅವನೊಬ್ಬನಿಗೆ ಮಾತ್ರ ಗೊತ್ತು ಈ ಪ್ರಪಂಚದಲ್ಲಿ ಪ್ರತಿಯೊಂದರ ಕೆಲಸವೇನು, ಒಟ್ಟಿನಲ್ಲಿ ಅದರ ಸ್ಥಾನ ಯಾವುದು ಎಂಬುದು. ಈ ಬ್ರಹ್ಮಾಂಡವನ್ನು ಅಷ್ಟೊಂದು ಅಂದವಾಗಿ ಜೋಡಿಸಿರುವನು. ಅವನು ಇದನ್ನು ಸೃಷ್ಟಿಸಿ ಎಲ್ಲೂ ಹೊರಗೆ ವಿರಾಮವಾಗಿಲ್ಲ. ಈ ಬ್ರಹ್ಮಾಂಡದಲ್ಲಿ ಓತಪ್ರೋತನಾಗಿ ಸರ್ವವ್ಯಾಪಿಯಾಗಿರುವನು. ಈ ಪ್ರಪಂಚದಲ್ಲಿಯೇ ಅವನು ಖರ್ಚಾಗಿಯೂ ಹೋಗಿಲ್ಲ. ದೇಶಕಾಲನಿಮಿತ್ತದಲ್ಲಿರುವ ಈ ಬ್ರಹ್ಮಾಂಡ ಭಗವಂತನ ಯಾವುದೋ ಸಣ್ಣ ಅಂಶ. ಅವನು ಇದನ್ನು ಮೀರಿರುವನು. ಅವನು ಇಲ್ಲಿರುವನು ಮತ್ತು ಇದಕ್ಕೆ ಅತೀತನೂ ಆಗಿರುವನು.

ಇನ್ನು ಅವನ ಯೋಗೈಶ್ವರ್ಯ. ಈ ಪ್ರಪಂಚದಲ್ಲಿ ಅವನು ಹೇಗಿರುವನು, ಎಲ್ಲೆಲ್ಲಿ ಇತರ ಕಡೆಗಿಂತ ಹೆಚ್ಚು ಕಾಣುತ್ತಿರುವನು ಎಂಬುದು. ಹೇಗೆ ಕಾವ್ಯದ ಹಿಂದೆ ಒಬ್ಬ ಕವಿಯ ಪ್ರತಿಭೆಯನ್ನು ನೋಡುತ್ತೇವೊ, ಚಿತ್ರದ ಹಿಂದೆ ಚಿತ್ರಕನ ಜಾಣತವನ್ನು ನೋಡುತ್ತೇವೆಯೊ ಹಾಗೆಯೆ ಈ ಬ್ರಹ್ಮಾಂಡದಲ್ಲಿ ಈ ಮಹಿಮೆಯನ್ನು ನೊಡುತ್ತೇವೆ. ಈ ಸೃಷ್ಟಿ ಎಂತಹ ಒಂದು ಮಹಾಕಾವ್ಯ! ಇಲ್ಲಿ ಯಾವ ರಸಕ್ಕೂ ಬರಗಾಲವಿಲ್ಲ. ಇದನ್ನು ನೋಡಿ ಸ್ಫೂರ್ತಿಗೊಂಡು ವ್ಯಾಸ ವಾಲ್ಮೀಕಿಗಳಂತಹ ಋಷಿಗಳು ಕಾವ್ಯ ಬರೆದರು. ಅದೇ ಜನರ ಅನುರಾಗಕ್ಕೆ ಅಷ್ಟು ಪಾತ್ರವಾಗಿದೆ. ಭಗವಂತ ಪುರಾಣ ಕವಿ. ಈ ಬ್ರಹ್ಮಾಂಡ ಅವನ ಮಹಾಕಾವ್ಯ. ಇದೆಷ್ಟು ಅದ್ಭುತವಾಗಿಬೇಕು! ತಾಜಮಹಲ್, ಶಾಜಹಾನ್ ತನ್ನ ಹೆಂಡತಿಗೆ ಕಟ್ಟಿದ ಪ್ರೇಮಸ್ಮಾರಕವೆನ್ನುತ್ತೇವೆ. ಈ ಬ್ರಹ್ಮಾಂಡವೇ ಭಗವಂತನ ಒಂದು ಸ್ಮಾರಕ. ಆದರೆ ಅದನ್ನು ಕಟ್ಟಿಕೊಂಡವನು ಸೃಷ್ಟಿಗೊಡೆಯನೆ. ಇಲ್ಲಿರುವುದೆಲ್ಲ ಅವನ ಮಹಿಮೆಯನ್ನು ಸಾರುತ್ತಿವೆ. ಈ ಪ್ರಪಂಚವನ್ನು ತಿಳಿದುಕೊಳ್ಳುತ್ತಾ ಹೋದಂತೆ, ಆಳ ಆಳಕ್ಕೆ ಪ್ರಕೃತಿಯೊಳಗೆ ಇಳಿದಂತೆ, ಇಲ್ಲಿರುವ ನಿಯಮಗಳಲ್ಲಿ ಒಂದಕ್ಕೂ ಮತ್ತೊಂದಕ್ಕೂ ಇರುವ ಸಂಬಂಧ, ಯಾವ ಜಾಣ್ಮೆಯಿಂದ ಇದನ್ನು ಜೋಡಿಸಿದ್ದಾನೆ ಎಂಬುದನ್ನು ಅರಿಯುತ್ತಾ ಹೋದಂತೆ, ಮನುಷ್ಯ ಪರವಶನಾಗುತ್ತ ಬರುವನು, ಅವನ ಜಾಣ್ಮೆಗೆ ತಲೆದೂಗುತ್ತಾ ಬರುವನು.

ಅವನನ್ನು ಮೀರಿಸುವ ಚಿತ್ರಕಾರನಾರು? ಎಂತಹ ಬಣ್ಣದ ಸಂತೆ ಈ ಪ್ರಪಂಚದಲ್ಲಿ ತಾಂಡವಾಡು ತ್ತಿದೆ! ಎಂತೆಂತಹ ನಾಮರೂಪಗಳ ಅದ್ಭುತ ಕಲಾಪ್ರದರ್ಶನ ಈ ವಿಶ್ವ! ಮೂಡಲಂಚಿನಲ್ಲಿ ಪ್ರತಿದಿನ ಮೂಡುವ ರವಿ ಮುಗಿಲ ಪಂಕ್ತಿಯಲ್ಲಿ ಎಂತಹ ಸೌಂದರ್ಯವನ್ನು ರಚಿಸುವನು. ಈ ಪ್ರಪಂಚವನ್ನು ಬಿಡುವಾಗ ಎಂತಹ ಸೌಂದರ್ಯವನ್ನು ಎರಚಿ ಮಾಯವಾಗುವನು. ಒಂದು ದಿನ ಇದ್ದಂತೆ ಮತ್ತೊಂದು ದಿನ ಇಲ್ಲ. ಪ್ರತಿ ದಿನವೂ ಒಂದು ನೂತನತೆ. ಹಕ್ಕಿಗಳಲ್ಲಿ ಎಷ್ಟು ವಿಧ. ಅದರ ಪುಕ್ಕಗಳು ಎಷ್ಟು ಅಂದ. ಅದರ ಗಾನ ಎಷ್ಟೊಂದು ವೈವಿಧ್ಯತೆಯಿಂದ ಕೂಡಿದೆ. ಬೆಳಗ್ಗೆ ಹುಟ್ಟಿ ಸಂಜೆ ಸಾಯುವ ಚಿಟ್ಟೆಯನ್ನು ನೋಡಿ. ಅದರ ರೆಕ್ಕೆಯನ್ನು ಎಷ್ಟು ಅಂದವಾಗಿ ಚಿತ್ರಿಸಿರುವನು. ಎಂತಹ ಸುಂದರವಾದ ಬಣ್ಣವನ್ನು ಆಕಾರವನ್ನು ಜೋಡಿಸಿರುವನು. ಯಾವ ರಾಜ ಮಹಾರಾಜನ ಬೆಡಗಿನ ಉಡುಗೆಯೂ ಇದರಷ್ಟು ಸುಂದರವಾಗಿಲ್ಲ. ಇನ್ನು ಅರಳುವ ಹೂಗಳು ಸೃಷ್ಟಿಯ ಒಂದು ಅದ್ಭುತ. ಇವುಗಳನ್ನೆಲ್ಲ ಭಗವಂತ ತಾನೇ ಧರಿಸಿದಂತಿದೆ. ಒಂದೊಂದು ಹೂವು ಒಂದೊಂದು ತರಹ. ಅದರ ಆಕಾರ ಬೇರೆ, ಪರಿಮಳ ಬೇರೆ, ಅಂದ ಬೇರೆ. ಸಮುದ್ರ ತೀರದಲ್ಲಿ ಆ ಸಾಗರ ಬಿಸುಟಿರುವ ಕಪ್ಪೆಯ ಚಿಪ್ಪುಗಳನ್ನು ತೆಗೆದುಕೊಂಡು ನೋಡುವ. ಯಃಕಶ್ಚಿತ್ ವಸ್ತುವಿನ ಮೇಲೆ ಎಂತಹ ಲಲಿತ ಕಲೆಯನ್ನು ನೋಡುತ್ತೇವೆ! ಸಾವಿರಾರು ಬಗೆಯ ಕಪ್ಪೆಯ ಚಿಪ್ಪುಗಳಿವೆ. ನಿನಗಿಂತ ನಾನು ಚೆನ್ನ, ನಾನು ಚೆನ್ನವೆಂದು ಅವು ಪೈಪೋಟಿ ಮಾಡುತ್ತಿವೆ. ಇವುಗಳನ್ನೆಲ್ಲಾ ನೋಡುತ್ತಾ ನೋಡುತ್ತಾ ನಾವು ಭಗವಂತನನ್ನು ಮರೆಯುವುದು ಹೇಗೆ? ನಾವು ಮರೆತರೂ ಇವುಗಳ ಮೂಲಕ ಅವನು ಬಂದು ನಮ್ಮ ಹೃದಯವನ್ನು ಸ್ಪರ್ಶಿಸುವನು.

ಭಗವಂತನ ವಿಭೂತಿ ಮತ್ತು ಅವನ ಐಶ್ವರ್ಯವನ್ನು ತಿಳಿಯುತ್ತಾ ಹೋದಂತೆ ಅವನ ಯಥಾರ್ಥ ದರ್ಶನ ನಮಗೆ ಆಗುವುದು. ಅವನು ಕಾಣಿಸದಾಗ, ಅವನಿಂದ ದೂರವಿರುವಾಗ, ನಾವು ಅವನ ವಿಷಯದಲ್ಲಿ ಅವನು ಹಾಗಿರಬಹುದು, ಹೀಗಿರಬಹುದು ಎಂದು ಕಲ್ಪಿಸಿಕೊಳ್ಳುವೆವು. ಸಾಕ್ಷಾತ್ತಾಗಿ ಸೃಷ್ಟಿಯಲ್ಲಿ ಅವನನ್ನೇ ಅನುಭವಿಸುತ್ತ ಹೋದಂತೆ ಅವನ ಯಥಾರ್ಥ ಸ್ಥಿತಿ ನಮಗೆ ಅರಿವಾಗುವುದು. ಆಗ ಅವನು ಅಲ್ಲಿಂದ ಉರುಳುವುದಿಲ್ಲ. ಅವನಿಗೆ ಭಗವಂತನ ಆಶ್ರಯ ಸಿಕ್ಕಿರುವುದು. ಈ ಸೃಷ್ಟಿಯಲ್ಲಿ ಅವನನ್ನು ನೋಡುವುದೇ ಮಹಾ ಪೂಜೆ, ಮಹಾ ಯಾಗ, ಮಹಾ ಸಾಧನೆ. ಕಣ್ಣು ಮುಚ್ಚಿಕೊಂಡು ನಿರ್ವಿಕಲ್ಪ ಸಮಾಧಿಗೆ ಹೋಗುವುದರಲ್ಲಿ ಒಂದು ಸಾಹಸ ಇದೆ. ಆದರೆ ಕಣ್ತೆರೆದುಕೊಂಡು ಎದುರಿಗೆ ಇರುವ ಪ್ರತಿಯೊಂದು ವಸ್ತುವಿನ ಹಿಂದೆಯೂ ಪ್ರತಿ ಕ್ಷಣವೂ ಅವನನ್ನು ಅನುಭವಿಸುವುದೇನು ಅದಕ್ಕಿಂತ ಕಡಮೆ ಸಾಹಸವೇನೂ ಅಲ್ಲ. ಒಮ್ಮೆ ಯೋಗದಿಂದ ಅವನ ಕೈಯನ್ನು ಹಿಡಿದುಕೊಂಡ ಮೇಲೆ ಇನ್ನವನು ನಮ್ಮನ್ನು ಕೆಡಹುವುದಿಲ್ಲ. ಇದರ ವಿಷಯದಲ್ಲಿ ಸಂಶಯವಿಲ್ಲದಿರಲಿ ಎಂದು ಶ‍್ರೀಕೃಷ್ಣ ಹೇಳುತ್ತಾನೆ. ಅನುಮಾನಗ್ರಸ್ತರು ನಾವು. ನಮಗೆ ನಿಜ ಬೇಕಾಗಿದೆ, ಭರವಸೆ ಬೇಕಾಗಿದೆ. ಅದನ್ನು ಶ‍್ರೀಕೃಷ್ಣ ಪದೇ ಪದೇ ಗೀತೆಯಲ್ಲಿ ಅನೇಕ ವೇಳೆ ಕೊಡುವನು.

\begin{shloka}
ಅಹಂ ಸರ್ವಸ್ಯ ಪ್ರಭಾವೋ ಮತ್ತಃ ಸರ್ವಂ ಪ್ರವರ್ತತೇ~।\\ಇತಿ ಮತ್ವಾ ಭಜಂತೇ ಮಾಂ ಬುಧಾ ಭಾವಸಮನ್ವಿತಾಃ \hfill॥ ೮~॥
\end{shloka}

\begin{artha}
ನಾನು ಸಮಸ್ತ ಜಗತ್ತಿಗೆ ಉತ್ಪತ್ತಿ ಸ್ಥಾನ. ನನ್ನಿಂದಲೇ ಎಲ್ಲವೂ ನಡೆಯುತ್ತಿದೆ. ಹೀಗೆಂದು ತಿಳಿದು ಜ್ಞಾನಿಗಳು ಭಾವಪೂರ್ವಕ ನನ್ನನ್ನು ಭಜಿಸುತ್ತಾರೆ.
\end{artha}

ಜ್ಞಾನಿಗಳಿಗೆ, ಭಗವಂತನಿಂದಲೇ ಈ ಪ್ರಪಂಚ ಬಂದಿದೆ, ಅವನಲ್ಲಿದೆ, ಕೊನೆಗೆ ಅವನಲ್ಲಿಯೇ ಐಕ್ಯವಾಗುತ್ತದೆ ಎಂಬುದು ಹೃದ್ಗತವಾದ ಸತ್ಯ. ಅವನೇ ಎಲ್ಲವನ್ನೂ ನಡೆಸುತ್ತಿರುವನು. ಈ ಭೂಮಿ ಸೂರ್ಯನ ಸುತ್ತಲೂ ಸುತ್ತುತ್ತಿರುವುದು ಅವನಿಂದ. ಎಲ್ಲದರ ಹಿಂದೆಯೂ ಒಂದು ಉದ್ದೇಶವನ್ನು ನೋಡುತ್ತೇವೆ. ಎಲ್ಲವೂ ಒಂದು ನಿಯಮಾನುಸಾರ ಹೋಗುವುದನ್ನು ನೋಡು\-ವೆವು. ಎಲ್ಲಿ ನಮಗೆ ನಿಯಮ ಗೊತ್ತಿಲ್ಲವೋ ಅಲ್ಲಿ ನಿಯಮ ಇಲ್ಲವೆಂದಲ್ಲ. ಇನ್ನೂ ನಾವು ಅವನನ್ನು ಅರ್ಥ ಮಾಡಿಕೊಂಡಿಲ್ಲ ಅಷ್ಟೆ. ಬಾಹ್ಯ ಪ್ರಪಂಚದಲ್ಲಿ ಆಗುವ ಚಲನವಲನಗಳೆಲ್ಲಾ ಭೌತಿಕ ನಿಯಮವನ್ನು ಅನುಸರಿಸುವುವು. ಆಧ್ಯಾತ್ಮಿಕ ಜೀವನದಲ್ಲಿ ಆಗುವ ಚಲನವಲನಗಳೆಲ್ಲ ಕರ್ಮ ನಿಯಮವನ್ನು ಅನುಸರಿಸುವುವು. ಈ ನಿಯಮಜಾಲವನ್ನು ನೆಯ್ದಿರುವವನೇ ಭಗವಂತ. ಹೂವಾಡಿಗ ಬಗೆಬಗೆಯ ಹೂವುಗಳನ್ನು ತನ್ನ ದಾರದಿಂದ ಕಟ್ಟಿ ಸುಂದರವಾದ ಹಾರವನ್ನು ಮಾಡಿರುವನು. ಕಣದ ಅಂತರಾಳದಲ್ಲಿ, ಜೀವಾಣುವಿನಲ್ಲಿ, ಬ್ರಹ್ಮಾಂಡದಲ್ಲಿ, ಅಣುವಿನಲ್ಲಿ, ಮಹತ್ತಿನಲ್ಲಿ, ಜಡದಲ್ಲಿ, ಚೇತನದಲ್ಲಿ ಎಲ್ಲ ಕಡೆಯೂ ಜ್ಞಾನಿಗೆ ಭಗವಂತನ ಸೂತ್ರ ಪರಿಚಯವಾಗುವುದು.

ಯಾವಾಗ ಭಗವಂತನ ಮಹಿಮೆಯನ್ನು ಕಣ್ಣಾರೆ ನೋಡುವರೊ ಅವರಿನ್ನು ನೀರಸವಾಗಿರುವುದಕ್ಕೆ ಆಗುವುದಿಲ್ಲ. ಅವರ ಹೃದಯದಲ್ಲಿ ಭಕ್ತಿ ಜಿನುಗುವುದು. ಸೃಷ್ಟಿಕರ್ತನನ್ನು ಭಕ್ತಿಯಿಂದ ಕೊಂಡಾಡುವರು. ತಂತಿಗೆ ಬೆರಳು ತಾಕಿದರೆ ಗಾನ ಹೊಮ್ಮುವಂತೆ ಜ್ಞಾನಿಯ ಹೃದಯನಾಡಿಯನ್ನು ಭಗವಂತ ಮಿಡಿಯುವನು. ಆಗಲೇ ಅಪೂರ್ವ ಜ್ಞಾನಮಾಧುರ್ಯ ಹೊಮ್ಮುವುದು ಅಲ್ಲಿಂದ. ಅವರಾಡುವ ಮಾತು ಜ್ಞಾನದಲ್ಲಿ ಹುರಿದಿದೆ, ಭಕ್ತಿ ಎಂಬ ಸಕ್ಕರೆಯ ಪಾಕದಲ್ಲಿ ಅದ್ದಿದೆ. ಕೇಳಿದವರಿಗೆಲ್ಲ ಇದು ವೇದ್ಯವಾಗುವುದು.

\begin{shloka}
ಮಚ್ಚಿತ್ತಾ ಮದ್ಗತಪ್ರಾಣಾ ಬೋಧಯಂತಃ ಪರಸ್ಪರಮ್​~।\\ಕಥಯಂತಶ್ಚ ಮಾಂ ನಿತ್ಯಂ ತುಷ್ಯಂತಿ ಚ ರಮಂತಿ ಚ \hfill॥ ೯~॥
\end{shloka}

\begin{artha}
ನನ್ನಲ್ಲಿಯೆ ನೆಲೆಗೊಳಿಸಿದ ಮನಸ್ಸು, ನನ್ನಲ್ಲಿಯೇ ಪ್ರತಿಷ್ಠಿತವಾದ ಪ್ರಾಣವುಳ್ಳವರಾಗಿ, ನನ್ನನ್ನೇ ಪರಸ್ಪರ ಬೋಧಿಸುತ್ತ, ಯಾವಾಗಲೂ ನನ್ನ ವಿಷಯವನ್ನೇ ಹೇಳುತ್ತ ತೃಪ್ತರಾಗಿ ಸಂತೋಷಿಸುತ್ತಿರುವರು.
\end{artha}

ಅವರ ಮನಸ್ಸು ಭಗವಂತನಲ್ಲಿ ಕೇಂದ್ರೀಕೃತವಾಗಿದೆ. ಭಗವಂತನಿಗೆ ಸಂಬಂಧಪಟ್ಟ ಆಲೋ\-ಚನೆಗಳೊಂದೇ ಅಲ್ಲಿ ಏಳುವುದು. ಅವನಲ್ಲದೆ ಮತ್ತಾವುದನ್ನು ಕುರಿತು ಚಿಂತಿಸುವುದಕ್ಕೆ ಆಗುವುದಿಲ್ಲ. ಅವರ ಪ್ರಾಣ ನಿಂತಿರುವುದು ಅವನಲ್ಲೆ. ಹೇಗೆ ನೀರಿಲ್ಲದೆ ಮೀನು ಬದುಕಲಾರದೊ, ಗಾಳಿಯಿಲ್ಲದೆ ನಾವು ಬದುಕಲಾರೆವೊ, ಹಾಗೆ ಭಗವಂತನೇ ಅವರ ಪ್ರಾಣವಾಯುವಾಗಿರುವನು. ಅವನನ್ನೆ ಅವರು ಉಚ್ಛ್ವಾಸ ನಿಶ್ವಾಸದಂತೆ ಉಸಿರಾಡುತ್ತಿರುವರು. ಒಬ್ಬೊಬ್ಬನಿಗೆ ಒಂದೊಂದರಲ್ಲಿ ಪ್ರಾಣವಿದೆ. ಮಗುವಿಗೆ ಆಟದಲ್ಲಿ, ಲೋಭಿಗೆ ಹಣದಲ್ಲಿ, ಕಾಮುಕನಿಗೆ ಹೆಂಗಸಿನ ಮೇಲೆ, ಭೋಗಿಗೆ ವಿಷಯ ವಸ್ತುವಿನ ಮೇಲೆ. ಅಂತೂ ಒಬ್ಬೊಬ್ಬನಿಗೆ ಒಂದೊಂದರಲ್ಲಿ ಪ್ರಾಣ. ಹಾಗೆಯೇ ಭಕ್ತರಿಗೆ ದೇವರಲ್ಲಿ ಪ್ರಾಣ. ಅವರು ಪರಸ್ಪರ ಬೋಧಿಸುವುದೇ ಭಗವಂತನ ವಿಷಯವನ್ನು. ಅವನ ವಿಷಯವನ್ನು ಮಾತಾಡುವುದರಲ್ಲಿ ಆನಂದವಿದೆ. ಅವನ ವಿಷಯವನ್ನು ಕೇಳುವುದರಲ್ಲಿ ಆನಂದವಿದೆ. ಎಂದಿಗೂ ಹಳೆಯದಾಗದ ವಿಷಯ ಅದು. ಎಷ್ಟು ಕೇಳಿದರೂ ಹೊಸದೆ. ಅವನ ವಿಷಯವನ್ನು ಮಾತನಾಡ ಬೇಕಾದರೆ ಸ್ಫೂರ್ತಿ ತಾನಾಗಿ ಉಕ್ಕಿ ಬರುವುದು. ಎಷ್ಟು ಅವನ ವಿಷಯವನ್ನು ಕೇಳಿದರೂ ಬೇಸರವಾಗುವುದಿಲ್ಲ. ಇನ್ನೂ ಕೇಳುತ್ತಿರಬೇಕು ಎನ್ನಿಸುವುದು. ಕಾಲ ಸಾಗುವುದೇ ಅರಿಯುವುದಿಲ್ಲ. ಗಂಟೆಗಳು ನಿಮಿಷದಂತೆ ಉರುಳಿ ಹೋಗುವುವು.

ಯಾವಾಗಲೂ ಭಕ್ತರು ಭಗವಂತನ ವಿಷಯವನ್ನೇ ಕುರಿತು ಮಾತನಾಡುತ್ತಾ ಇರುತ್ತಾರೆ. ಜೀವನದಲ್ಲಿ ಸುಖವಾಗಿರಲಿ ಕಷ್ಟವಾಗಿರಲಿ, ರೋಗದಲ್ಲಿ ನರಳುತ್ತಿರಲಿ, ದೇವರು ಎಷ್ಟು ಕಷ್ಟಕ್ಕೆ ಇವರನ್ನು ಗುರಿಮಾಡಲಿ, ಅವನ ಮಾತನ್ನು ಮಾತ್ರ ಆಡುವುದನ್ನು ಬಿಡುವುದಿಲ್ಲ. ಶ‍್ರೀರಾಮಕೃಷ್ಣ\-ರಿಗೆ ಭಕ್ತರೊಡನೆ ಮಾತನಾಡಿ ಮಾತನಾಡಿ ಗಂಟಿಲಿನಲ್ಲಿ ವ್ರಣ \enginline{(cancer)} ಬಂತು. ಆಗ ವೈದ್ಯರು ಮಾತನಾಡಬೇಡಿ ಎಂದು ಕಟ್ಟಪ್ಪಣೆ ಕೊಟ್ಟರು. ಆಗ ಮಾತನಾಡಿದರೆ ನರಕಯಾತನೆ ಆಗುತ್ತಿತ್ತು. ಆ ಸಮಯದಲ್ಲಿಯೂ ಯಾರಾದರೂ ಭಕ್ತರು ಬಂದು ದೇವರ ವಿಷಯವನ್ನು ಕೇಳಿದರೆ ಸಾಕು ಅವರು ಮಾತುಕತೆಯಲ್ಲಿ ತನ್ಮಯರಾಗುತ್ತಿದ್ದರು. ದೇಹ ಯಾತನೆಯನ್ನು ಗಮನಿಸಲಿ, ಮನವೇ ನೀನು ದೇವರನ್ನು ಚಿಂತಿಸು ಎಂದು ಹೇಳುತ್ತಿದ್ದರು.

ದೇವರು ಅವರನ್ನು ಯಾವ ಸ್ಥಿತಿಯಲ್ಲಿಟ್ಟರೂ ತೃಪ್ತಿಪಡುತ್ತಾರೆ. ಜೀವನದಲ್ಲಿ ಗೊಣಗಾಡುವುದು ಅವರ ಸ್ವಭಾವವಲ್ಲ. ತಪ್ಪು ಕಂಡುಹಿಡಿಯುವುದು ಅವರ ಪ್ರಕೃತಿಗೆ ವಿರೋಧ. ಕೊಟ್ಟಿದ್ದರಲ್ಲಿ ತೃಪ್ತರಾಗಿ, ಇಟ್ಟ ಸ್ಥಿತಿಯಲ್ಲಿ ನೆಮ್ಮದಿಯಿಂದ ಇರುತ್ತಾರೆ. ಅವರು ಸದಾ ಆನಂದದಲ್ಲಿ ಇರುವರು. ಅವರ ಆನಂದ ಬಾಹ್ಯ ವಸ್ತುವಿನ ಸಂಘಾತದಿಂದ ಬಂದುದಲ್ಲ. ಹೃದಯದಲ್ಲಿ ಭಗವಂತ ಮಂಡಿಸಿರುವುದನ್ನು ನೋಡುವರು. ತಮ್ಮ ಹೊರಗೆ ಎಲ್ಲದರಲ್ಲಿಯೂ ಭಗವಂತನೇ ಇರುವುದನ್ನು ಅನುಭವಿಸುವರು. ಇಂತಹವರಿಗೆ ನಿರಾನಂದ ಹೇಗಿರಬಹುದು? ನಿಜವಾದ ಆನಂದವನ್ನು ಅನು ಭವಿಸುವವರನ್ನು ನೋಡಬೇಕಾದರೆ ಇಂತಹ ಭಕ್ತರಲ್ಲಿ ನೋಡಬೇಕು, ವಿಷಯ ವಸ್ತುಗಳಲ್ಲಿ ಉನ್ಮತ್ತರಾಗಿರುವವರಲ್ಲಿ ಅಲ್ಲ. ಭಕ್ತರು ತಮ್ಮ ಹೃದಯವೆಂಬ ರೇಡಿಯೋ ಯಂತ್ರವನ್ನು ಭಗವಂತನಕಡೆಗೆ ತಿರುಗಿಸಿರುವರು. ಅಲ್ಲಿ ಯಾವಾಗಲೂ ಅವನಿಗೆ ಸಂಬಂಧಪಟ್ಟ ಗಾನವೇ ಹೊರಹೊಮ್ಮುತ್ತಿರುವುದು.

\begin{shloka}
ತೇಷಾಂ ಸತತಯುಕ್ತಾನಾಂ ಭಜತಾಂ ಪ್ರೀತಿಪೂರ್ವಕಮ್~।\\ದದಾಮಿ ಬುದ್ಧಿಯೋಗಂ ತಂ ಯೇನ ಮಾಮುಪಯಾಂತಿ ತೇ \hfill॥ ೧ಂ~॥
\end{shloka}

\begin{artha}
ನಿತ್ಯಯುಕ್ತರೂ ಪ್ರೀತಿಪೂರ್ವಕವಾಗಿ ಭಜಿಸುತ್ತಿರುವವರೂ ಆದ ಅವರಿಗೆ ಬುದ್ಧಿಯೋಗವನ್ನು ಕೊಡುತ್ತೇನೆ. ಅದರಿಂದ ಅವರು ನನ್ನನ್ನೇ ಹೊಂದುವರು.
\end{artha}

ನಿತ್ಯಯುಕ್ತರು ಅಂದರೆ ಯಾವಾಗಲೂ ಭಗವಂತನನ್ನು ಕುರಿತು ಚಿಂತಿಸುತ್ತಿರುವವರು. ನದಿ ಹೇಗೆ ಒಂದೇ ಸಮನೆ ಸಾಗರದ ಕಡೆ ಹರಿದುಕೊಂಡು ಹೋಗುತ್ತಿರುವುದೋ ಹಾಗೆ ಭಕ್ತ ಅನು\-ಗಾಲವೂ ದೇವರನ್ನೇ ಕುರಿತು ಚಿಂತಿಸುತ್ತಿರುವನು. ಅವನು ಇತರ ಯಾವ ಕೆಲಸ ಕಾರ್ಯಗಳಲ್ಲಿ ನಿರತನಾಗಿದ್ದರೂ ಭಗವತ್ ಚಿಂತನೆ ಹಿನ್ನೆಲೆಯ ಸಂಗೀತದಂತೆ ವ್ಯಾಪಿಸಿಕೊಂಡಿರುವುದು. ಹಾಗೆ ಅವನನ್ನು ಕುರಿತು ಚಿಂತಿಸುವಾಗ ಬರೀ ಯಾಂತ್ರಿಕವಾಗಿ ಮಾಡುವುದಿಲ್ಲ, ಅದು ನೀರಸವಾಗಿರುವು ದಿಲ್ಲ. ಚಿಂತನೆಯ ಗಾಲಿಗಳು ಶಬ್ದಮಾಡದಂತೆ, ಘರ್ಷಣೆಯಾಗದಂತೆ ಪ್ರೀತಿಯ ಎಣ್ಣೆಯನ್ನು ಸವರಿರುವನು. ಅವನ ಹೃದಯದಲ್ಲಿ ಭಗವಂತನ ಮೇಲೆ ಪ್ರೀತಿ ವಿಶ್ವಾಸಗಳು ಸದಾ ಜಿನುಗು\-ತ್ತಿರುವುವು. ಅವನನ್ನು ಕುರಿತು ಚಿಂತಿಸಿದರೇನೇ ಅವನ ಹೃದಯ ಉಬ್ಬುವುದು, ಚಂದ್ರನನ್ನು ಕಂಡು ಸಾಗರ ಉಬ್ಬುವಂತೆ.

ಯಾವಾಗ ಎಡೆಬಿಡದೆ ಭಕ್ತ ಹೀಗೆ ಭಗವಂತನನ್ನು ಕುರಿತು ಚಿಂತಿಸುತ್ತಿರುವನೋ ಆಗ ಭಗವಂತ ಭಕ್ತನಿಗೆ ಜ್ಞಾನವನ್ನು ಕೊಡುತ್ತಾನೆ. ಇದರಿಂದ ಅವನಿಗೆ ತಾನು ಯಾರು ಎಂಬುದು ಗೊತ್ತಾಗುವುದು, ಭಗವಂತನ ಮಹಿಮೆ ಏನು ಎಂಬುವುದು ಗೊತ್ತಾಗುವುದು. ಅವನು ಶಾಸ್ತ್ರಗಳನ್ನು ಓದಿ ಈ ವಿಷಯಗಳನ್ನು ತಿಳಿದುಕೊಳ್ಳುವುದಿಲ್ಲ. ಭಗವಂತನೆ ಇವುಗಳನ್ನೆಲ್ಲ ಅವನಿಗೆ ಕೊಡುವನು. ಭಕ್ತ ಭಗವಂತನಿಂದ ಜ್ಞಾನವನ್ನು ಕೇಳುವುದಿಲ್ಲ. ಅವನಿಗೆ ಮುಕ್ತಿಯೂ ಬೇಕಿಲ್ಲ. ಸುಮ್ಮನೇ ಭಗವಂತನನ್ನು ಪ್ರೀತಿಸುವನು. ಆದರೆ ಕನಿಕರದಿಂದ ತನ್ನ ಭಕ್ತ ಸರ್ವತೋಮುಖವಾಗಿ ಬೆಳೆಯಬೇಕೆಂದು ಜ್ಞಾನವನ್ನು ಭಗವಂತ ಕರುಣಿಸುವನು. ಭಕ್ತಿ ಎಲ್ಲಿ ಇದೆಯೋ ಅಲ್ಲಿ ಜ್ಞಾನದ ಅಭಾವ ಇರುವುದಿಲ್ಲ. ಜ್ಞಾನ ಕೂಡ ಇರುವುದು. ಭಕ್ತಿ ಮುಂದೆ ಹೋದರೆ, ಜ್ಞಾನ ಹಿಂದಿನಿಂದ ಬರುವುದು. ಮುಂಚೆ ಭಗವಂತನನ್ನು ಪ್ರೀತಿಸುವನು. ಅನಂತರ ಅವನಾರು ಎಂಬುದನ್ನು ಇನ್ನು ಬೇರೆ ಯಾರೂ ಹೇಳಬೇಕಾಗಿಲ್ಲ. ಅವನೇ ತಾನಾರು ಎಂಬುದನ್ನೆಲ್ಲ ಹೇಳುವನು. ಯಾವಾಗ ಅವನನ್ನು ಚೆನ್ನಾಗಿ ತಿಳಿಯುವನೊ ಆಗ ಮುಕ್ತಿಯನ್ನು ಪಡೆಯುವನು.

\begin{shloka}
ತೇಷಾಮೇವಾನುಕಂಪಾರ್ಥಮಹಮಜ್ಞಾನಜಂ ತಮಃ~।\\ನಾಶಯಾಮ್ಯಾತ್ಮಭಾವಸ್ಥೋ ಜ್ಞಾನದೀಪೇನ ಭಾಸ್ವತಾ \hfill॥ ೧೧~॥
\end{shloka}

\begin{artha}
ಅವನಿಗೆ ಅನುಕಂಪೆಯನ್ನು ತೋರಿಸಬೇಕೆಂದು ನಾನು ಅವನ ಬುದ್ಧಿಯಲ್ಲಿ ನೆಲೆಸಿ ಪ್ರಕಾಶಿಸುತ್ತಿರುವ ಜ್ಞಾನದಿಂದ ಅಜ್ಞಾನದಿಂದ ಹುಟ್ಟಿದ ತಮಸ್ಸನ್ನು ನಾಶಮಾಡುತ್ತೇನೆ.
\end{artha}

ಯಾವಾಗ ಭಕ್ತ ಭಗವಂತನಲ್ಲಿ ಶರಣಾಗುತ್ತಾನೆಯೋ, ಆಗ ದೇವರು ಅವನಿಗೆ ಅನುಕಂಪೆಯನ್ನು ತೋರಿಸುತ್ತಾನೆ. ಪಾಪ ಇವನು ನನ್ನನ್ನೇ ನೆಚ್ಚಿದ್ದಾನೆ, ಇವನ ಉದ್ಧಾರಕ್ಕೆ ನಾನು ಎಲ್ಲವನ್ನೂ ಮಾಡಬೇಕೆಂದು ದೇವರು ಭಾವಿಸುತ್ತಾನೆ. ಇಲ್ಲದಿದ್ದರೆ ಭಕ್ತನಿಗೆ ಅಲ್ಲ ಅಗೌರವ, ದೇವರಿಗೆ ಅಗೌರವ. ಮಗು ಹರಕಲು ಅಂಗಿ ಹಾಕಿಕೊಂಡು ಅಲೆಯುತ್ತಿದ್ದರೆ ಮಗುವಿಗೇನೂ ಲಜ್ಜೆಯಿಲ್ಲ, ಅದನ್ನು ಹೆತ್ತವರಿಗೆ ಅವಮಾನ! ಅದಕ್ಕಾಗಿ ಮುಂಚೆ ಅದಕ್ಕೆ ಚೆನ್ನಾಗಿರುವ ಬಟ್ಟೆ ಬರೆ ಹೊಲಿಸುತ್ತಾರೆ. ಆ ಮಗು ಏನು ತನಗೆ ಕೊಡು ಎಂದು ಕೇಳುವುದಿಲ್ಲ. ಆದರೆ ದೇವರೆ ಇವನಿಗೆ ಕೊಡದೆ ಇದ್ದರೆ ನನ್ನ ಮಾನ ಉಳಿಯುವಂತಿಲ್ಲ ಎಂದು ಮುಂಚೆ ಅವನ ಬುದ್ಧಿಯಲ್ಲಿ ನೆಲೆಸುತ್ತಾನೆ. ಯಾವಾಗ\break ಭಗವಂತನೇ ಅವನ ಬುದ್ಧಿಯಲ್ಲಿ ನೆಲಸುತ್ತಾನೋ ಅದು ಪವಿತ್ರವಾದ ಬುದ್ಧಿಯಾಗುವುದು, ಬಹು ಹರಿತವಾದ ಬುದ್ಧಿಯಾಗುವುದು. ಅನಾತ್ಮ ವಸ್ತುಗಳೆಲ್ಲ ತಕ್ಷಣ ಗ್ರಹಿಸುವ ಸ್ಥಿತಿಗೆ ಬರುವುವು. ಅವನು ಶಾಸ್ತ್ರಾದಿಗಳನ್ನು ಅಧ್ಯಯನ ಮಾಡಿಲ್ಲದೆ ಇರಬಹುದು. ಯಾವ ತರ್ಕದ ಗರಡಿ\-ಯಲ್ಲಿಯೂ ಅವನು ಕಸರತ್ತನ್ನು ಮಾಡಿಲ್ಲ. ಆದರೆ ಇವನೆದುರಿಗೆ ಯಾವ ಶಾಸ್ತ್ರ ಬಲ್ಲವನೂ ನಿಲ್ಲಲಾರ, ಯಾವ ವಾದಿಯೂ ಇವನನ್ನು ಸೋಲಿಸಲಾರ. ಏಕೆಂದರೆ ಇವನ ಹಿಂದೆ ಭಗವಂತನೇ ಇವನಿಗೆ ಬೇಕಾದುದನ್ನೆಲ್ಲಾ ಒದಗಿಸುತ್ತಿರುವನು. ಪುಸ್ತಕದಿಂದ ಕಲಿತ ವಿದ್ಯೆಗೆ ಅಂತ್ಯವಿದೆ. ಯಾರ ಹಿಂದೆ ದೇವರೇ ನಿಂತಿರುವನೋ ಅವನ ಜ್ಞಾನಕ್ಕೆ ಇನ್ನು ಬರಗಾಲ ಇರುವುದಿಲ್ಲ. ಅವನ ಮಾತು ವೇದ ಉಪನಿಷತ್ತುಗಳಿಗೆ ಸಮನಾಗಿರುತ್ತದೆ.

ಭಕ್ತನಲ್ಲಿ ಬೆಳಗುತ್ತಿರುವ ಜ್ಞಾನದ ಕಾಂತಿಯಿಂದ ಮೊದಲು ದೇವರು ಅವನ ಅಜ್ಞಾನವನ್ನು ನಾಶಮಾಡುತ್ತಾನೆ. ಭಗವಂತ ಭಕ್ತನಲ್ಲಿ ಮುಂಚೆ ಮಾಡುವ ಕೆಲಸವೇ ಇದು. ಮೊದಲು ಅವನನ್ನು ನೇರ ಮಾಡುವನು, ಪವಿತ್ರ ಮಾಡುವನು, ಪರಿಶುದ್ಧನನ್ನಾಗಿ ಮಾಡುವನು. ಮೊದಲು ಭಗವಂತ ತನ್ನನ್ನು ನೆಚ್ಚಿರುವ ಭಕ್ತನನ್ನು ಮುತ್ತಿರುವ ಅಜ್ಞಾನದ ಕೆಸರನ್ನು ತೊಳೆಯುತ್ತಾನೆ. ಅನಂತರ ಅವನ ಮೂಲಕ ಇತರರ ಅಜ್ಞಾನವನ್ನು ತೊಳೆಯುವ ಕೆಲಸವನ್ನು ಮಾಡುತ್ತಾನೆ.

ಅರ್ಜುನ ಆಗ ಭಗವಂತನಿಗೆ ಹೀಗೆ ಹೇಳುತ್ತಾನೆ:

\begin{shloka}
ಪರಂ ಬ್ರಹ್ಮ ಪರಂ ಧಾಮ ಪವಿತ್ರಂ ಪರಮಂ ಭವಾನ್~।\\ಪುರುಷಂ ಶಾಶ್ವತಂ ದಿವ್ಯಮಾದಿದೇವಮಜಂ ವಿಭುಮ್ \hfill॥ ೧೨~॥
\end{shloka}

\begin{shloka}
ಆಹುಸ್ತ್ವಾಮೃಷಯಃ ಸರ್ವೇ ದೇವರ್ಷಿರ್ನಾರದಸ್ತಥಾ~।\\ಅಸಿತೋ ದೇವಲೋ ವ್ಯಾಸಃ ಸ್ವಯಂ ಚೈವ ಬ್ರವೀಷಿ ಮೇ \hfill॥ ೧೩~॥
\end{shloka}

\begin{artha}
ನೀನು ಪರಬ್ರಹ್ಮ, ಪರಮಧಾಮ, ಮತ್ತು ಪರಮ ಪವಿತ್ರ. ಎಲ್ಲಾ ಪುಷಿಗಳು ದೇವರ್ಷಿಯಾದ ನಾರದನು ಅಸಿತ, ದೇವಲ, ವ್ಯಾಸ ಮುಂತಾದವರು ನಿನ್ನನ್ನು ಶಾಶ್ವತನೆಂದೂ ದಿವ್ಯನೆಂದೂ ಪುರುಷನೆಂದೂ ಆದಿದೇವ ನೆಂದೂ ಅಜನೆಂದೂ ವಿಭುವೆಂದೂ ಹೇಳುತ್ತಾರೆ. ನೀನು ಕೂಡ ನನಗೆ ಹಾಗೆಯೇ ಹೇಳುತ್ತಿರುವೆ. 
\end{artha}

ಅರ್ಜುನ ಮುಂಚೆ ಸಖ ಎಂದು ಕರೆಯುತ್ತಿದ್ದ. ಈಗ ಶ‍್ರೀಕೃಷ್ಣನಲ್ಲಿ ಸಾಕ್ಷಾತ್ ಪರಬ್ರಹ್ಮನನ್ನು ನೋಡುತ್ತಾನೆ. ನನ್ನ ಭಕ್ತನ ಹೃದಯದಲ್ಲಿದ್ದು ನಾನು ಅವನ ಬುದ್ಧಿಯನ್ನು ಬೆಳುಗುತ್ತೇನೆ ಎಂದು ಶ‍್ರೀಕೃಷ್ಣ ಹಿಂದಿನ ಶ್ಲೋಕದಲ್ಲಿ ಹೇಳಿದ. ಅದರಂತೆಯೇ ಅವನ ಭಕ್ತನಾದ ಅರ್ಜುನನ ಬುದ್ಧಿಯಲ್ಲಿಯೂ ಬೆಳಗುತ್ತಾನೆ. ಆಗ ಅರ್ಜುನನಿಗೆ ಶ‍್ರೀಕೃಷ್ಣನ ದೈವತ್ವ ಹೊಳೆಯುವುದು. ಅದನ್ನು ತಾನೇ ಅನುಭವಿಸಿ ಹೇಳುತ್ತಾನೆ. ಅರ್ಜುನ ಶಾಸ್ತ್ರಾದಿಗಳನ್ನು ಓದಿ ಈ ಜ್ಞಾನವನ್ನು ಸಂಗ್ರಹಿಸಲಿಲ್ಲ. ಶ‍್ರೀಕೃಷ್ಣನ ಸ್ಪರ್ಶದಿಂದ ಅರ್ಜುನನ ಹೃದಯ ಚಿಲುಮೆ ಬಾಯಿತೆರೆದು ಜಿನುಗತೊಡಗಿತು.

ನೀನು ಪರಬ್ರಹ್ಮ ಎನ್ನುತ್ತಾನೆ. ದೇಶ ಕಾಲ ನಿಮಿತ್ತಕ್ಕೆ ಅತೀತವಾದ ಪರಬ್ರಹ್ಮನೇ ಶ‍್ರೀಕೃಷ್ಣ. ಎಲ್ಲ ನಾಮಗಳನ್ನು ಆಕಾರಗಳನ್ನು ಮೀರಿರುವನೆ ಅವನು. ಗುಣಾತೀತ ರೂಪಾತೀತ ಸರ್ವವ್ಯಾಪಿ ಅವನು. ನೀನೇ ಪರಂಧಾಮ ಎನ್ನುತ್ತಾನೆ. ನಿನ್ನನ್ನು ಮೀರಿದ ಲೋಕವಿಲ್ಲ. ಯಾರು ನಿನ್ನನ್ನು ಮುಟ್ಟುತ್ತಾರೋ ಅವರು ಮಾಯೆಯ ಆಟದಿಂದ ಪಾರಾದವರು. ಮಾಯೆ ಇವನಿರುವೆಡೆ ಕಾಡಲು ಬಾರದು. ಒಮ್ಮೆ ಇವನನ್ನು ಸೇರಿದರೆ ಇನ್ನು ಮೇಲೆ ಜನನ ಮರಣಗಳ ಸಂಸಾರ ಚಕ್ರಕ್ಕೆ ಸಿಕ್ಕುವುದಿಲ್ಲ. ಇವನಷ್ಟು ಪವಿತ್ರ ಮತ್ತಾವುದೂ ಇಲ್ಲ. ಇತರ ವಸ್ತುಗಳು ತಾತ್ಕಾಲಿಕವಾಗಿ ಮನುಷ್ಯನನ್ನು ಪವಿತ್ರ ಮಾಡುವುದು. ಎಲ್ಲ ಹೊರೆಗೆ ಇರುವ ಕಸವನ್ನು ಮಾತ್ರ ಗುಡಿಸುವುದು. ಆದರೆ ಭಗವಂತನಾದರೊ ಒಮ್ಮೆ ಜೀವಿಯನ್ನು ಸ್ಪರ್ಶಮಾಡಿದ ಎಂದರೆ ಜನ್ಮ ಜನ್ಮಾಂತರದಿಂದ ಬಂದ ಅಜ್ಞಾನ ಹೃದಯದ ಅಂತರಾಳದಿಂದ ತೊಲಗಿ ಹೋಗುವುದು, ಮತ್ತೊಮ್ಮೆ ಬರದ ರೀತಿಯಲ್ಲಿ ಹೋಗುವುದು. ಭಗವಂತನ ಪವಿತ್ರತೆ ತೊಳೆಯಲಾರದ ಕಸವಿಲ್ಲ, ಕೊಳೆಯಿಲ್ಲ.

ಸಾಧಾರಣ ಮನುಷ್ಯರು ಕೊಂಡಾಡುವುದಿರಲಿ, ಅವರಿಗೆ ದೇವರ ವಿಷಯ ಗೊತ್ತಿರುವುದು ಬಹಳ ಅಲ್ಪ. ಯಾರನ್ನು ನಾವು ದೊಡ್ಡ ಜ್ಞಾನಿಗಳೆಂದು ಭಾವಿಸುವೆವೊ ಅಂತಹ ನಾರದ, ದೇವಲ, ವ್ಯಾಸ ಮುಂತಾದವರು ಭಗವಂತನನ್ನು ಬಣ್ಣಿಸುವರು. ವಜ್ರದ ವ್ಯಾಪಾರಿ ಮಾತ್ರ ವಜ್ರಕ್ಕೆ ಬೆಲೆಕಟ್ಟುವಂತೆ ಇಂತಹ ಮಹಾವ್ಯಕ್ತಿಗಳು ಮಾತ್ರ ಭಗವಂತನನ್ನು ಸ್ಪಲ್ಪ ಅರಿಯಬಲ್ಲರು. ಅವರು ಹೀಗೆ ಹೇಳುತ್ತಾರೆ:

ಅವನು ಶಾಶ್ವತ ಪುರುಷ. ಈ ಪ್ರಪಂಚದಲ್ಲಿ ಅವನು ವಿನಃ ಉಳಿದುದೆಲ್ಲ ಬದಲಾವಣೆಯ ಚಕ್ರಕ್ಕೆ ಸಿಕ್ಕಿ ರೂಪಾಂತರ ಹೊಂದಿ ನಾಶವಾಗಿ ಹೋಗುವುವು. ಅವನೊಬ್ಬನೇ ಸೃಷ್ಟಿ ಸ್ಥಿತಿ ಪ್ರಳಯದ ಹಿಂದೆ ಯಾವ ಬದಲಾವಣೆಗೂ ಸಿಲುಕದೆ ಶಾಶ್ವತನಾಗಿರುವುದು. ಅವನು ದಿವ್ಯ, ಸ್ವಯಂಜ್ಯೋತಿ ಸ್ವರೂಪ. ಅವನೇ ಎಲ್ಲವನ್ನೂ ಬೆಳಗುವವನು, ಅವನನ್ನು ಬೆಳಗುವವರಾರೂ ಇಲ್ಲ. ಅವನೇ ಆದಿ ದೇವ. ಎಲ್ಲಾ ದೇವತೆಗಳಿಗೂ ಆದಿ ಪುರುಷ. ಇವನು ಮುಂಚೆ, ಅನಂತರ ಇತರ ವಸ್ತುಗಳೆಲ್ಲ. ಮುಂಚೆ ಬೇರು ಅನಂತರ ಮರದ ರೆಂಬೆ ಕೊಂಬೆಗಳಂತೆ. ಅವನು ಅಜ ಎಂದಿಗೂ ಹುಟ್ಟಿಲ್ಲ. ಯಾವುದು ಹುಟ್ಟುವುದೊ ಅದು ನಾಶವಾಗಲೇಬೇಕು. ಹುಟ್ಟು, ನಾಶವೆಂಬುದು ಕಾಲದೊಳಗೆ ಇರುವುದು. ಕಾಲದಲ್ಲಿ ಇರುವುದಕ್ಕೆ ಮಾತ್ರ ಇದು ಅನ್ವಯಿಸುವುದು. ಅವನು ಕಾಲಾತೀತ, ಆದಕಾರಣವೇ ದೇಶಕಾಲನಿಮಿತ್ತದ ಪ್ರಪಂಚದೊಳಗೆ ಒಂದು ಗುಳ್ಳೆಯಂತೆ ಎದ್ದಿಲ್ಲ. ಇವನು ವಿಭು, ಸರ್ವವ್ಯಾಪಿ, ಅವನಿಲ್ಲದ ಸ್ಥಳವೇ ಇಲ್ಲ. ಅಣುರೇಣು ತೃಣಕಾಷ್ಟದಿಂದ ಕೂಡಿ ಬ್ರಹ್ಮಾಂಡವನ್ನೆಲ್ಲಾ ವ್ಯಾಪಿಸಿಕೊಂಡಿರುವನು. ನಾವು ಯಾವುದನ್ನು ಮಹಾಬ್ರಹ್ಮಾಂಡ ಎನ್ನುತ್ತೇವೆಯೊ ಅದು ಅವನ ಒಂದು ಸಣ್ಣ ಅಂಶ–ನಮ್ಮ ದೇಹದ ಒಂದು ರೋಮದಂತೆ.

ಈ ವಿಷಯವನ್ನು ಮಹಾಪುಷಿಗಳು ಮಾತ್ರ ಹೇಳುವುದಿಲ್ಲ. ನಿನ್ನ ವಿಷಯ ನಿನಗಿಂತ ಹೆಚ್ಚಾಗಿ ಯಾರಿಗೂ ಗೊತ್ತಿಲ್ಲ. ನೀನೇ ಮನಸ್ಸು ಮಾಡಿ ಹೇಳಿದರೆ ಮಾತ್ರ ನಾವು ಸ್ಪಲ್ಪ ತಿಳಿದುಕೊಳ್ಳಬಲ್ಲೆವು. ಪೋಲೀಸಿನವನು ಕೈಯಲ್ಲಿರುವ ದೀಪದಿಂದ ಎಲ್ಲರ ಮುಖಕ್ಕೆ ಹಿಡಿದು ಅವರು ಯಾರು ಎಂಬು\-ದನ್ನು ನೋಡುತ್ತಾನೆ. ಆದರೆ ಅವನು ನಮಗೆ ಕಾಣುವುದಿಲ್ಲ. ನಾವೆ ಅವನನ್ನು ಕೇಳಿಕೊಳ್ಳಬೇಕು. ನಿನ್ನ ಮುಖವನ್ನು ನೋಡಬೇಕೆಂದಿರುವೆವು, ದಯವಿಟ್ಟು ನಿನ್ನ ಕೈಯಲ್ಲಿರುವ ಬೆಳಕನ್ನು ನಿನ್ನ ಮುಖದ ಕಡೆಗೆ ಸ್ವಲ್ಪ ಹಿಡಿದುಕೊ ಎಂದು. ಅವನು ಕೃಪೆ ಮಾಡಿದರೆ ಮಾತ್ರ ಅವನನ್ನು ತಿಳಿದುಕೊಳ್ಳಬಹುದು ಎಂದು ಶ‍್ರೀರಾಮಕೃಷ್ಣರು ಹೇಳುತ್ತಿದ್ದರು.

\begin{shloka}
ಸರ್ವಮೇತದೃತಂ ಮನ್ಯೇ ಯನ್ಮಾಂ ವದಸಿ ಕೇಶವ~।\\ನ ಹಿ ತೇ ಭಗವನ್ ವ್ಯಕ್ತಿಂ ವಿದುರ್ದೇವಾ ನ ದಾನವಾಃ \hfill॥ ೧೪~॥
\end{shloka}

\begin{artha}
ಕೇಶವಾ, ನೀನು ಯಾವುದನ್ನು ನನಗೆ ಹೇಳುತ್ತಿರುವೆಯೋ ಇವೆಲ್ಲವನ್ನೂ ಸತ್ಯವೆಂದು ತಿಳಿಯುತ್ತೇನೆ. ಭಗವಂತ, ದೇವತೆಗಳು, ದಾನವರು ಕೂಡ ನಿನ್ನ ವ್ಯಕ್ತಿತ್ವವನ್ನು ಅರಿಯರು.
\end{artha}

ಶ‍್ರೀಕೃಷ್ಣನ ಬೋಧನೆಯನ್ನು ಕೇಳಿ ಆದಮೇಲೆ ಇವುಗಳನ್ನೆಲ್ಲ ಸತ್ಯ ಎಂದು ತಿಳಿಯುತ್ತೇನೆ, ಎನ್ನುತ್ತಾನೆ ಅರ್ಜುನ. ಇನ್ನು ಮೇಲೆ ಅವನ ವಾಣಿಯನ್ನು ಅನುಮಾನಿಸುವುದಿಲ್ಲ. ಸತ್ಯವನ್ನು ಅರಿತವನು ಮಾತನಾಡಿದರೆ ಅವನ ಮಾತಿನಲ್ಲಿ ಒಂದು ಅಪೂರ್ವ ಶಕ್ತಿ ಇರುವುದು. ಅದು ನಮ್ಮ ಮೇಲೆ ತನ್ನ ಪ್ರಭಾವವನ್ನು ಬಿಡುವುದು. ನಮ್ಮ ಮನಸ್ಸು ಅದಕ್ಕೆ ಎಷ್ಟು ಅಣಿಯಾಗುವುದೊ ಅಷ್ಟು ಬೇಗ ಅದು ಪರಿಣಾಮಕಾರಿಯಾರುವುದು. ಇನ್ನು ಮೇಲೆ ಶ‍್ರೀಕೃಷ್ಣ ಪರಮಾತ್ಮನೇ ಎಂಬುದನ್ನು ಸಂಶಯ ಪಡುವುದಿಲ್ಲ ಅರ್ಜುನ. ಅವನ ದಯದಿಂದ ಇವನ ಒಳಗಣ್ಣು ತೆರೆದಿದೆ. ಅದರ ಮೂಲಕ ನೋಡಿದಾಗ ಮಾತ್ರ ಶ‍್ರೀಕೃಷ್ಣನ ಮಹಿಮೆ ಗೊತ್ತಾಗುವುದು. ಒಳಗಿನ ಕಣ್ಣು ತೆರೆಯದೆ ಇದ್ದರೆ ನಾವು ಒಬ್ಬ ಅವತಾರದ ಹತ್ತಿರವೇ ದಿನಬೆಳಗಾದರೆ ಇರಬಹುದು, ಅದರ ಮಹಿಮೆ ನಮಗೆ ಗೊತ್ತಾಗುವುದಿಲ್ಲ. ದೇವರೇ ನಮ್ಮ ಎದುರಿಗೆ ಬಂದು ನಾನು ದೇವರು ಎಂದರೆ ಎಷ್ಟು ಜನ ನಂಬುತ್ತಾರೆ? ಶ‍್ರೀರಾಮ ಕಾಡಿನಲ್ಲಿ ಸಂಚಾರಮಾಡುತ್ತಿದ್ದಾಗ, ಕೆಲವು ಪುಷಿಗಳು ಅವನನ್ನು ಅವತಾರವೆಂದು ಗುರುತಿಸಿ ಕೊಂಡಾಡಿದರು. ಇನ್ನು ಹಲವರು ರಾಮನ ಎದುರಿಗೇನೆ ಸಂಕೋಚವಿಲ್ಲದೆ, ಅವರೆಲ್ಲ ನಿನ್ನನ್ನು ಅವತಾರ ಎಂದು ಕೊಂಡಾಡಬಹುದು; ಆದರೆ ನಾವು, ನೀನು ದಶರಥನ ಮಗ ಎಂದು ಮಾತ್ರ ನೋಡುತ್ತೇವೆ ಎಂದರು. ಶ‍್ರೀರಾಮ ಇದನ್ನು ಕೇಳಿ ನಸುನಗುತ್ತ ಮುಂದೆ ಹೊರಟ. ಭಗವಂತ ನಮ್ಮ ಎದುರಿಗೆ ಬಂದರೆ ಸಾಲದು. ಅವನು ನಮಗೆ ಅವನನ್ನು ತಿಳಿದುಕೊಳ್ಳುವುದಕ್ಕೆ ಯೋಗ್ಯತೆಯನ್ನು ಕೊಡಬೇಕು. ಆಗಲೆ ಅವನನ್ನು ತಿಳಿದುಕೊಳ್ಳಲು ಸಾಧ್ಯ. ಅವನು ಜ್ಞಾನಭಾಸ್ಕರನಂತೆ ಹೃದಯದಲ್ಲಿ ನೆಲಿಸಿದಾಗ ಮಾತ್ರ ಸಾಧ್ಯ.

ದೇವತೆಗಳು ಮತ್ತು ದಾನವರು ಕೂಡ ಭಗವಂತನ ವ್ಯಕ್ತಿತ್ವವೇನು, ಅವನು ಈ ಸೃಷ್ಟಿಯಲ್ಲಿ ಹೇಗೆ ಇದ್ದಾನೆ ಎಂಬುದನ್ನು ಅರಿಯಲಾರರು. ದೇವದಾನವರಲ್ಲಿ ಕಠೋರ ತಪಸ್ಸನ್ನು ಮಾಡಿರುವವರೂ ಇದ್ದಾರೆ. ಆದರೆ ದೇವರು ಅವರ ತಪಸ್ಸಿನಿಂದ ಸುಪ್ರೀತನಾಗಿ ಪ್ರತ್ಯಕ್ಷ\-ನಾದಾಗ ಅವನನ್ನು ತಿಳಿದುಕೊಳ್ಳಬೇಕೆಂದು ಬಯಸಿದವರು ಅಪರೂಪ. ಕೇವಲ ಪ್ರಪಂಚದಲ್ಲಿ ಲೂಟಿಮಾಡುವುದಕ್ಕೆ, ಯಾವ ಅಪಾಯವೂ ಇಲ್ಲದೆ ಸುರಕ್ಷಿತವಾಗಿ ಬಾಳುವುದಕ್ಕೆ ಕೆಲವು ವರಗಳನ್ನು ಬೇಡಿದರೆ ಹೊರತು ಜ್ಞಾನ ಕೇಳಲಿಲ್ಲ. ಭಕ್ತಿ ಕೇಳಲಿಲ್ಲ. ಇಂತಹವರಿಗೆ ದೇವರ ಒಂದು ಅಂಶಮಾತ್ರ ಅರ್ಥವಾಗುವುದು. ಅವನು ದೊಡ್ಡ ವ್ಯಕ್ತಿ, ಅವನ ಉಗ್ರಾಣದಲ್ಲಿ ಬೇಕಾದಷ್ಟು ಸರಕುಗಳಿವೆ. ನಾವು ಅದನ್ನು ಕೇಳಿದರೆ ಕೊಡುತ್ತಾನೆ ಎಂಬುದು ಮಾತ್ರ ಅವರ ಅರಿವಿಗೆ ನಿಲುಕುವುದು. ಇಂತಹವರು ಭಗವಂತನ ವಿಭೂತಿಯನ್ನು ಅರಿಯುವುದು ಹೇಗೆ?

\begin{shloka}
ಸ್ವಯಮೇವಾತ್ಮನಾತ್ಮಾನಂ ವೇತ್ಥ ತ್ವಂ ಪುರುಷೋತ್ತಮ~।\\ಭೂತಭಾವನ ಭೂತೇಶ ದೇವದೇವ ಜಗತ್ಪತೇ \hfill॥ ೧೫~॥
\end{shloka}

\begin{artha}
ಪುರುಷೋತ್ತಮ, ಜೀವಗಳ ಪಿತ, ಜೀವೇಶ್ವರ, ದೇವದೇವ, ಜಗದೀಶ್ವರ ನೀನು. ನೀನೆ, ನಿನ್ನನ್ನು ನಿನ್ನಿಂದಲೇ ತಿಳಿದುಕೊಂಡಿದ್ದೀಯೆ.
\end{artha}

ಪುರುಷೋತ್ತಮ ಎಂದರೆ ಪುರುಷರಲ್ಲಿ ಶ್ರೇಷ್ಠನಾದವನು. ಅವನಲ್ಲಿ ಎಲ್ಲ ಕಡೆಗಿಂತ ಹೆಚ್ಚಾಗಿ ದಿವ್ಯಗುಣಗಳು ಪ್ರಕಾಶಿಸುತ್ತಿವೆ. ಈ ಪ್ರಪಂಚದಲ್ಲಿ ಒಂದು ವ್ಯಕ್ತಿಯಲ್ಲಿ ಅಷ್ಟೊಂದು ಜ್ಞಾನ ಪ್ರೇಮ ಕರುಣೆ ಸಹಾನುಭೂತಿ ಐಶ್ವರ್ಯ ಅಧಿಕಾರ ಮುಂತಾದುವು ಕಾಣುವಂತೆ ಬೇರೆ ಇನ್ನೆಲ್ಲಿಯೂ ಕಾಣುವುದಿಲ್ಲ. ನದಿಗಳೆಲ್ಲ ಹೇಗೆ ಸಾಗರಕ್ಕೆ ಸೇರಿವೆಯೊ ಹಾಗೆ ಎಲ್ಲಾ ಒಳ್ಳೆಯ ಗುಣಗಳೂ ಪರಮಾತ್ಮನೆಂಬ ಮಹಾಸಾಗರಕ್ಕೆ ಸೇರಿವೆ. ಎಲ್ಲಾ ದಿವ್ಯಗುಣಗಳು ತಮ್ಮ ಪರಾಕಾಷ್ಠೆಯನ್ನು ಮುಟ್ಟುವುದು ಅವನಲ್ಲಿ. ಸಾಧಾರಣ ಮನುಷ್ಯರಲ್ಲಿ ಸುಗುಣಗಳು ಮಿಂಚಿನ ಹುಳುವಿನಂತೆ ಕಂಗೊಳಿಸುತ್ತಿದ್ದರೆ, ಸೂರ್ಯನಂತೆ ಪ್ರಕಾಶಿಸುತ್ತಿವೆ ಆ ಪುರುಷೋತ್ತಮನಲ್ಲಿ. ಮಾನವ ವಿಕಾಸದ ತುತ್ತತುದಿ ಅದು. ಅದನ್ನು ಮೀರಿದ ವಿಕಾಸವನ್ನು ನಾವು ಊಹಿಸಲಾರೆವು. ದೇವರು\break ಮಾನವನಿಗೆ ದಾರಿಯನ್ನು ತೋರಲು ಇಳಿದು ಬಂದ ಎಂದು ಹೇಳಿದಾಗ ಅವನು ಅವತಾರವಾಗುತ್ತಾನೆ. ಪ್ರತಿಯೊಂದು ಜೀವಿಯಲ್ಲೂ ಭಗವದಂಶ ಸುಪ್ತವಾಗಿದೆ. ಜನ್ಮ ಜನ್ಮಗಳು ಹೋರಾಡಿ ಸಾಧನಾ ಬಲದಿಂದ ಇಂತಹ ಶಿಖರಕ್ಕೆ ಏರಿರುವನು ಎಂದು ಅವತಾರವನ್ನು ನಂಬದವರು ಹೇಳುತ್ತಾರೆ. ಸಾಧನಾ ಬಲದಿಂದ ಒಬ್ಬ ಪೂರ್ಣಾತ್ಮನಾಗಬಹುದು. ಆದರೆ ಪುರುಷೋತ್ತಮ ಆಗಲಾರ. ಈ ಪಾತ್ರ ದೇವರೊಬ್ಬನಿಗೇ ಮೀಸಲು. ಇತರರು ಅದನ್ನು ಮಾಡಲಾರರು.

ಅವನು ಜೀವಿಗಳ ಪಿತ. ಜೀವಿಗಳನ್ನೆಲ್ಲ ತನ್ನ ಅಂಶದಿಂದಲೇ ಸೃಷ್ಟಿಸಿದವನು. ಹೇಗೆ ಕಿಡಿಗಳು ಅಗ್ನಿಕುಂಡದಿಂದ ಏಳುತ್ತವೆಯೊ, ಸೂರ್ಯನಿಂದ ಕಿರಣಗಳು ಬರುತ್ತವೆಯೊ, ಹಾಗೆ ಜೀವರಾಶಿಗಳೆಲ್ಲಾ ಅವನಿಂದ ಬಂದವರು. ಅವು ತಾವಾಗಿ ಬರಲಿಲ್ಲ. ಭಗವಂತ ಇಚ್ಛಿಸಿದುದರಿಂದ ಬಂದವು. ಅವನು ಎಲ್ಲರಿಗೂ ತಂದೆ–ಒಳ್ಳೆಯವನಿಗೆ, ಕೆಟ್ಟವನಿಗೆ, ಬುದ್ಧಿವಂತನಿಗೆ ದಡ್ಡನಿಗೆ. ಯಾರೂ ಅವನ ಪ್ರೀತಿಗೆ ಬಾಹಿರರಲ್ಲ. ಎಲ್ಲರನ್ನೂ ಒಂದೇ ಸಮನಾಗಿ ಪ್ರೀತಿಸುವನು ಅವನು. ಎಲ್ಲರಿಗೂ ಅವರ ಆತ್ಮವಿಕಾಸಕ್ಕೆ ಒಂದೇ ಸಮನಾಗಿ ಅವಕಾಶಗಳನ್ನು ಕೊಡುವನು ಅವನು.

ಅವನು ಜೀವೇಶ್ವರ. ಎಲ್ಲರಿಗೂ ಒಡೆಯ. ಯಾರೂ ಅವನ ಆಜ್ಞೆಯನ್ನು ಮೀರಲಾರರು. ನಮ್ಮ ದೇಹದಿಂದ ಹುಟ್ಟುವ ಮಕ್ಕಳು ಎಳೆಯವರಾಗಿರುವಾಗ ಮಾತ್ರ ನಮ್ಮ ಮಾತನ್ನು ಕೇಳುತ್ತಾರೆ. ವಯಸ್ಸು ಬರಬರುತ್ತ ತಂದೆ ತಾಯಿಗಳು ಒಂದು ಇಚ್ಛಿಸಿದರೆ, ಅವರು ಬೇರೊಂದು ರೀತಿ ಬೆಳೆಯುತ್ತಾರೆ. ಕೊನೆಯತನಕ ಅಪ್ಪನ ಮಾತನ್ನೇ ಕೇಳುವ ಮಕ್ಕಳು ಅಪರೂಪ ಈ ಪ್ರಪಂಚದಲ್ಲಿ. ಆದರೆ ದೇವರೆಂಬ ತಂದೆ ಹಾಗಲ್ಲ. ಅವನೆಲ್ಲರನ್ನೂ ಆಳುತ್ತಿರುವನು. ಯಾರೂ ಅವನಿಗೆ ವಿರೋಧವಾಗಿ ಹೋಗಲಾರರು. ಅವನು ರಚಿಸಿದ ನಿಯಮಗಳ ಕಣ್ಣಿಗೆ ಮಣ್ಣನ್ನು ಎರಚುವರಾರು! ಏನಾದರೂ ತಪ್ಪುಮಾಡಿದರೆ, ಲಾಯರನ್ನು ಇಟ್ಟು ಬೇಕಾದಷ್ಟು ಲಂಚ ಸುರಿದು ಪ್ರಪಂಚದಲ್ಲಿ ಬರುವ ಶಿಕ್ಷೆಯಿಂದ ತಪ್ಪಿಸಿಕೊಳ್ಳಬಹುದು. ಆದರೆ ಭಗವಂತನ ನಿಯಮವನ್ನು ಪಾಲಿಸುವ ಕರ್ಮದೂತರು ಯಾರ ಬಲೆಗೂ ಬೀಳುವುದಿಲ್ಲ. ಸೃಷ್ಟಿಸುವುದು ಸುಲಭ. ಸೃಷ್ಟಿಸಿದ್ದಕ್ಕೆಲ್ಲ ಕೊನೆಯತನಕ ಒಡೆಯನಾಗ ಬೇಕಾದರೆ ದೇವರೊಬ್ಬನಿಗೆ ಸಾಧ್ಯ.

ಅವನು ದೇವದೇವ. ಈ ಪ್ರಪಂಚದಲ್ಲಿ ಹಲವಾರು ದೇವದೇವತೆಗಳು ಇರುವರು. ಒಂದೊಂದು ದೇಶದವರು, ಒಂದೊಂದು ಧರ್ಮದವರು ದೇವರನ್ನು ತಮ್ಮ ತಮ್ಮ ರೀತಿಯಲ್ಲಿ ಚಿತ್ರಿಸಿಕೊಂಡು ಉಪಾಸನೆ ಮಾಡುತ್ತಿರುವರು. ಇವುಗಳ ಹಿಂದೆ ಇರುವುದೆಲ್ಲ ಒಬ್ಬ ಪರಮಾತ್ಮನೇ. ಇವುಗಳಲ್ಲೆಲ್ಲ ಒಬ್ಬ ಪರಮಾತ್ಮನ ಶಕ್ತಿಯೇ ವ್ಯಕ್ತವಾಗುತ್ತಿದೆ, ಒಂದೇ ವಿದ್ಯುತ್ ಶಕ್ತಿ ಹಲವು ಮನೆಗಳಲ್ಲಿ ಬಲ್ಬಿನ ಮೂಲಕ ಬೆಳಗುವಂತೆ.

ಅವನು ಜೀವರಾಶಿಗೆ ಮಾತ್ರ ಒಡೆಯನಲ್ಲ. ಈ ಜಗತ್ತಿಗೆಲ್ಲ ಒಡೆಯ. ಅವನ ಇಚ್ಛೆ ಇಲ್ಲದೆ ಒಂದು ಹುಲ್ಲಿನ ಎಸಳು ಕೂಡ ಚಲಿಸಲಾರದು, ಒಂದು ಕಣ ಸ್ಪಂದಿಸಲಾರದು. ಅವನಿಚ್ಛೆಯಿಂದ ಭೂಮಿ ಉರುಳುವುದು, ಚಂದ್ರ ಬೆಳಗುವುದು, ಸೂರ್ಯ ತಪಿಸುವುದು, ಮಳೆ ಬರುವುದು, ನದಿ ಹರಿಯುವುದು. ಈ ಪ್ರಪಂಚದಲ್ಲಿ ಎಲ್ಲ ಭೌತಿಕ ಘಟನೆಗಳೂ ಅವನ ಆಜ್ಞಾನುಸಾರ ನಡೆಯುವುವು.

ಅವನೊಬ್ಬನೇ ತಾನು ಯಾರು ಎಂಬುದನ್ನು ಸಂಪೂರ್ಣ ತಿಳಿದವನು. ಪರಮಾತ್ಮ ಮಾತ್ರ ಪರಮಾತ್ಮನನ್ನು ತಿಳಿದುಕೊಳ್ಳಬಲ್ಲ. ಇತರರು ತಮ್ಮ ಯೋಗ್ಯತಾನುಸಾರ ಭಗವಂತನನ್ನು ತಮ್ಮ ಪಾತ್ರೆಯಲ್ಲಿ ಹಿಡಿದಿಟ್ಟಿರುವರು. ಅದೆಲ್ಲಾ ಒಂದು ಸಣ್ಣ ಅಂಶ ಮಾತ್ರ. ನಾವು ಸಾಗರಕ್ಕೆ ಹೋಗಿ ನಮ್ಮ ಪಾತ್ರೆಯಲ್ಲಿ ಅದರಿಂದ ಸ್ವಲ್ಪ ನೀರನ್ನು ತಂದು ಮನೆಯಲ್ಲಿಡಬಹುದು. ಇದನ್ನು ಸಾಗರ ಎಂದು ಕರೆಯಬಹುದು. ಸಾಗರದ ನೀರದು ನಿಜ. ಆದರೆ ಆ ಮಹಾಸಾಗರದಲ್ಲಿರುವ ವಸ್ತುಗಳೆಲ್ಲ ಇದರಲ್ಲಿ ಇದೆಯೆ? ಸಾಗರ ವಿನಃ ಬೇರೆ ಯಾವುದೂ ಅವನ್ನು ಒಳಗೊಳ್ಳಲಾರದು. ಅದರಂತೆಯೇ ಯಾರನ್ನು ತುಂಬ ದೊಡ್ಡ ಜ್ಞಾನಿಗಳು ಎನ್ನುತ್ತೇವೆಯೊ ಅವರು ತಮ್ಮ ಪಾತ್ರೆಯಲ್ಲಿ ಸ್ವಲ್ಪ ದೇವರನ್ನು ಹಿಡಿದಿಟ್ಟಿರುವರು. ಆದರೆ ದೇವರೆಲ್ಲವನ್ನೂ ಹಿಡಿಯುವ ಪಾತ್ರೆ ನಮ್ಮಲ್ಲಿ ಎಲ್ಲಿದೆ? ನಾವು ಸಾಂತ ಮನುಷ್ಯರು. ನಮ್ಮ ದೇಹ, ಮನಸ್ಸು, ಬುದ್ಧಿ, ಇಂದ್ರಿಯ ಅಹಂಕಾರಗಳಿಗೆಲ್ಲ ಒಂದು ಮಿತಿ ಇದೆ. ಇವುಗಳೆಲ್ಲ ಸೇರು ಪಾವು ಚಟಾಕಿನಂತೆ. ಇವುಗಳಿಂದ ಅನಂತ ಸಾಗರವನ್ನೆ ಅಳೆಯುವುದಕ್ಕೆ ಆದೀತೆ? ಅಳೆದಷ್ಟೂ ಇನ್ನೂ ಇರುವುದು. ತಿಳಿದಷ್ಟೂ ಇನ್ನೂ ತಿಳಿಯುವುದು ಇರುವುದು. ಅದಕ್ಕೇ ಭಗವಂತನಿಗೊಬ್ಬನಿಗೇ ಗೊತ್ತು ಅವನು ಯಾರು ಎಂಬುದು. ಉಳಿದವರಲ್ಲಿ ಇರುವುದೆಲ್ಲ ಅವನ ಒಂದು ‘ಸ್ಯಾಂಪಲ್​’ ಅಷ್ಟೆ. ಅದಕ್ಕಿಂತ ಹೆಚ್ಚಿಲ್ಲ.

\begin{shloka}
ವಕ್ತುಮರ್ಹಸ್ಯಶೇಷೇಣ ದಿವ್ಯಾ ಹ್ಯಾತ್ಮವಿಭೂತಯಃ~।\\ಯಾಭಿರ್ವಿಭೂತಿಭಿರ್ಲೋಕಾನಿಮಾಂಸ್ತ್ವಂ ವ್ಯಾಪ್ಯ ತಿಷ್ಠಸಿ \hfill॥ ೧೬~॥
\end{shloka}

\begin{artha}
ಯಾವ ಯಾವ ವಿಭೂತಿಯಿಂದ ನೀನು ಈ ಲೋಕವನ್ನು ವ್ಯಾಪಿಸಿಕೊಂಡಿರುವೆಯೊ ಆ ದಿವ್ಯವಾದ ನಿನ್ನ ವಿಭೂತಿಗಳನ್ನು ನಿಶ್ಶೇಷವಾಗಿ ಹೇಳು.
\end{artha}

ಪುರುಷಸೂಕ್ತದ ಪ್ರಕಾರ ಭಗವಂತನಿಂದಲೇ ಈ ಪ್ರಪಂಚ ಆದುದು. ಅಷ್ಟು ಮಾತ್ರವೇ ಅಲ್ಲ. ತಾನೇ ಈ ಬ್ರಹ್ಮಾಂಡವನ್ನು ಸೃಷ್ಟಿಸಿ ಆದ ಮೇಲೆ ಅದನ್ನು ವ್ಯಾಪಿಸಿಕೊಂಡಿರುವನು. ಅವನು ಹೇಗೆ ವ್ಯಾಪಿಸಿಗೊಂಡಿರುವನು ಎಂಬುದು ಎಲ್ಲರಿಗಿಂತ ಹೆಚ್ಚಾಗಿ ಭಗವಂತನಿಗೆ ಗೊತ್ತಿದೆ. ಅದಕ್ಕಾಗಿಯೇ ಅರ್ಜುನ ಭಗವಂತನ ಬಾಯಿಯಿಂದಲೇ ಅದನ್ನು ಕೇಳಬೇಕೆಂದು ಇಚ್ಛಿಸುವನು. ನಾವು ಭಗವಂತ ಹೇಗಿರುವನೊ ಹಾಗೆ ಚಿಂತಿಸುವ ಸ್ಥಿತಿಯಲ್ಲಿ ಇಲ್ಲ. ನಮ್ಮ ಮನಸ್ಸು ಇನ್ನೂ ಅಷ್ಟು ಪರಿಶುದ್ಧವಾಗಿಲ್ಲ. ಅದು ಸ್ಥೂಲವನ್ನು ಮಾತ್ರ ಗ್ರಹಿಸಬಲ್ಲುದು. ವಿಭೂತಿಗಳ ಮೂಲಕ ಯಾರಿಗೆ ಈ ವಿಭೂತಿಗಳು ಇದೆಯೊ ಅಲ್ಲಿಗೆ ಹೋಗಬೇಕಾಗಿದೆ. ಇದು ಸೂರ್ಯನ ಕಿರಣಗಳ ಮೂಲಕ ಸೂರ್ಯನ ಕಡೆಗೆ ಹೊರಟಂತೆ. ಸೂರ್ಯ ಕೋಟ್ಯಂತರ ಮೈಲಿಗಳ ದೂರದಲ್ಲಿದ್ದಾನೆ. ಆದರೆ ಅವನ ಕಿರಣಗಳಾದರೊ ನಮ್ಮ ಸುತ್ತಮುತ್ತ ನಮ್ಮ ಮೇಲೆಲ್ಲ ಬೀಳುತ್ತಿವೆ. ಆದರೆ ಎಲ್ಲವೂ ಅವನನ್ನು ಒಂದೇ ಸಮನಾಗಿ ಪ್ರತಿಬಿಂಬಿಸುತ್ತಿಲ್ಲ. ಕನ್ನಡಿ ಚೆನ್ನಾಗಿ ಪ್ರತಿಬಿಂಬಿಸುವುದು. ನೀರು ಅದಕ್ಕಿಂತ ಕಡಮೆ, ನೆಲ ಮತ್ತು ಇತರ ವಸ್ತುಗಳು ಅವಕ್ಕಿಂತ ಕಡಮೆ. ನಾವು ಸೂರ್ಯನನ್ನು ನೋಡಬೇಕಾಗಿದೆ. ಕನ್ನಡಿ ಮೂಲಕ ಮಾತ್ರ ಅವನ ಪ್ರತಿಬಿಂಬ ದರ್ಶನ ಮೊದಲು ಮಾಡಿ ಅನಂತರ ಸೂರ್ಯನ ಕಡೆಗೆ ತಿರುಗಬೇಕಾಗುವುದು. ಪ್ರತಿಬಿಂಬ ನಮ್ಮ ಹತ್ತಿರದಲ್ಲೆ ಇದೆ. ಅದನ್ನು ನೋಡುವುದು ಮತ್ತು ಅದನ್ನು ತಿಳಿದುಕೊಳ್ಳುವುದು ಸುಲಭ. ಮೂಲವಸ್ತುವಿನಿಂದಲೇ ಬಂದಿರುವುದು ಪ್ರತಿಬಿಂಬ. ಪ್ರತಿಬಿಂಬದಲ್ಲಿ ಬೆಳಗುತ್ತಿರುವುದು ಮೂಲವಸ್ತುವೆ.

ಆ ವಿಭೂತಿಗಳು ಎಂದರೆ ಎಲ್ಲೆಲ್ಲಿ ಪ್ರತಿಬಿಂಬ ಚೆನ್ನಾಗಿ ಕಾಣುವುದೊ ಅವೆಲ್ಲ ದಿವ್ಯವಾಗಿವೆ. ಪ್ರತಿಬಿಂಬಿಸಬೇಕಾದರೂ ಒಂದು ಯೋಗ್ಯತೆ ಇರಬೇಕು. ಎಲ್ಲವೂ ಸೂರ್ಯನನ್ನು ಪ್ರತಿಬಿಂಬಿಸ\-ಲಾರವು. ಅವನು ನಿಷ್ಪಕ್ಷಪಾತಿಯಂತೆ ಎಲ್ಲದರ ಮೇಲೆ ಬೀಳುತ್ತಿದ್ದರೂ ಎಲ್ಲವೂ ಒಂದೇ ಸಮನಾಗಿ ಅದನ್ನು ಪ್ರತಿಬಿಂಬಿಸಲಾರವು. ಹಾಗೆ ಪ್ರತಿಬಿಂಬಿಸಬೇಕಾದರೆ ಆ ವಸ್ತುವಿನಲ್ಲಿ ಯೋಗ್ಯತೆ ಇದ್ದರೆ ಮಾತ್ರ ಸಾಧ್ಯ. ಆ ವಿಭೂತಿಗಳನ್ನು ಚಿಂತಿಸುವುದೇ ಒಂದು ವಿಧವಾದ ಧ್ಯಾನವಾಗುವುದು. ಈ ವಿಭೂತಿ ಚಿಂತನೆಯೇ ನಮ್ಮನ್ನು ಭಗವಂತನ ಕಡೆಗೆ ಕರೆದುಕೊಂಡು ಹೋಗುವುದು.

ಅದನ್ನು ನಿಶ್ಶೇಷವಾಗಿ ಹೇಳು ಎನ್ನುವನು. ಅವನ ವಿಭೂತಿಗೆ ಒಂದು ಅಂತ್ಯವಿಲ್ಲ. ಈ ಸೃಷ್ಟಿಯಲ್ಲಿ ಹಲವು ಕಡೆ ಹಲವು ರೀತಿ ಅವನು ವ್ಯಾಪಿಸಿಕೊಂಡಿರುವನು. ಅವನು ಎಷ್ಟೆಷ್ಟು ರೂಪಿನಲ್ಲಿ ವ್ಯಾಪಿಸಿಕೊಂಡಿರುವನೊ ಅದನ್ನೆಲ್ಲ ಕೇಳಬೇಕೆಂದು ಆಶಿಸುವನು. ಭಗವಂತ ಅನಂತ, ಅವನ ವಿಭೂತಿಗಳೂ ಅನಂತವೇ. ಅದನ್ನು ಯಾರಿಗಾದರೂ ಹೇಳಿ ಪೂರೈಸಲು ಸಾಧ್ಯವೆ? ಶಿವಮಹಿಮ್ನಾ ಸ್ತೋತ್ರದಲ್ಲಿ ಸುರತರುವನ್ನೇ ಲೇಖನಿಯನ್ನಾಗಿ ಮಾಡಿಕೊಂಡು, ನೀಲಗಿರಿಯನ್ನೇ ಮಸಿಯನ್ನಾಗಿ ಮಾಡಿಕೊಂಡು ಭೂಮಿಯ ಕಾಗದದ ಮೇಲೆ ಸಾಕ್ಷಾತ್ ಸರಸ್ವತಿಯೆ ಬರೆಯುತ್ತಾ ಹೋದರೂ ಅದನ್ನು ಪೂರೈಸುವುದಕ್ಕೆ ಆಗುವುದಿಲ್ಲ ಅನ್ನುವನು. ಆಗಿನ ಕಾಲದಲ್ಲಿ ಈ ಭೂಮಿ ಎಂಬ ಗೋಳ ಒಂದೇ ಅವರಿಗೆ ಚೆನ್ನಾಗಿ ಗೊತ್ತಿತ್ತು. ಈಗ ಹಲವು ಗ್ರಹಗಳಿವೆ. ಅನಂತ ತಾರಾವಳಿಗಳಿವೆ. ಬಹುಪಾಲು ಭೂಮಿಗಿಂತ ದೊಡ್ಡದು. ಈ ಗೋಳಗಳ ಕಾಗದದ ಮೇಲೆಲ್ಲ ಶಾರದೆ ಬರೆಯುತ್ತಾ ಹೋದರೂ ಇನ್ನು ಮುಗಿಯುತು ಎಂಬ ಕಾಲ ಕೇಳುವಂತಿಲ್ಲ. ಆದರೂ ಭಕ್ತನಿಗೆ ಆಸೆ ಮೊದಲು ದೇವರ ಎಲ್ಲವನ್ನೂ ತಿಳಿದುಕೊಳ್ಳಬೇಕೆಂದು. ಅವನಿಗೆ ಇನ್ನೂ ತನ್ನ ಮಿತಿ ಗೊತ್ತಿಲ್ಲ. ಬರೀ ಉದ್ವೇಗ ಒಂದೇ ಅವನಲ್ಲಿರುವುದು. ಇದನ್ನು ವಿವರಿಸುವುದಕ್ಕೆ ಶ‍್ರೀರಾಮಕೃಷ್ಣರು, ಟೈಫಾಯಿಡ್ ರೋಗಿಯ ಉದಾಹರಣೆ ಕೊಡುತ್ತಾರೆ. ಜ್ವರದಲ್ಲಿ ನರಳುವವನು ಹಂಡೆ ಹಂಡೆಗಳಷ್ಟು ನೀರು ಕುಡಿಯಬೇಕು, ರಾಶಿರಾಶಿ ಅನ್ನ ತಿನ್ನಬೇಕು ಎನ್ನುವನು. ಅನಂತರ ಅವನಿಗೆ ಗೊತ್ತಾಗುವುದು. ಇವತ್ತೆಲ್ಲ ಒಂದು ಪಾವು ಅಕ್ಕಿ ಅನ್ನ ತಿನ್ನವುದರಲ್ಲಿ ಸುಸ್ತಾಗಿ ಹೋಗತ್ತೇನೆ, ಒಂದೆರಡು ಲೋಟ ನೀರು ಕುಡಿಯುವುದರಲ್ಲಿ ತೃಪ್ತನಾಗಿ ಹೋಗುತ್ತೇನೆ ಎಂಬುದು. ಆಗ ಅನ್ನಿಸುವುದು ಭೂಮಿಯಲ್ಲಿ ಎಷ್ಟು ನೀರಿದೆ ಎಂಬುದು ನನಗೇಕೆ ಬೇಕು, ಅದರಿಂದ ಒಂದೆರಡು ಲೋಟ ನೀರು ಕುಡಿದರೆ ಸಾಕು, ನನ್ನ ಬಾಯಾರಿಕೆ ಹಿಂಗುವುದು. ಹೆಂಡದ ಅಂಗಡಿಯಲ್ಲಿರುವುದನ್ನೆಲ್ಲ ನಾನು ಕಟ್ಟಿಕೊಂಡು ಏನು ಮಾಡುವುದು. ನನಗೆ ಒಂದೆರಡು ಬುಡ್ಡಿ ಕುಡಿದರೆ ಸಾಕು, ಜ್ಞಾನ ತಪ್ಪಿ ಬೀಳುತ್ತೇನೆ. ಸಾಕು ಅಷ್ಟೇ ನನಗೆ ಎಂದು ಅನಂತರ ಅನ್ನಿಸುವುದು. ಹಾಗೆಯೇ ಅರ್ಜುನನ ವಿಪರೀತ ಆಸೆ ಭಗವಂತನ ವಿಭೂತಿಗಳನ್ನೆಲ್ಲ ತಿಳಿದುಕೊಳ್ಳಬೇಕೆಂದು. ಶ‍್ರೀಕೃಷ್ಣ ತನ್ನ ವಿಭೂತಿಗಳನ್ನೆಲ್ಲ ಹೇಳುತ್ತ ಹೋಗಬಹುದು. ಆದರೆ ಅರ್ಜುನನಿಗೆ ಅದನ್ನು ತಿಳಿದುಕೊಳ್ಳುವ ಯೋಗ್ಯತೆ ಆದರೂ ಎಷ್ಟಿರುವುದು? ಸುಮ್ಮನೆ ಸುರಿದರೆ ಜಾಸ್ತಿಹಿಡಿಯುವುದೆ? ನಲ್ಲಿಯ ಕೆಳಗೆ ಪಾತ್ರೆಯನ್ನು ಇಟ್ಟು ನೀರನ್ನು ತುಂಬಬಹುದು. ನೀರು ತುಂಬಿ ಆದಮೇಲೆ ನೀರು ಇನ್ನೂ ಅದಕ್ಕೆ ಬೀಳುತ್ತಿದ್ದರೆ ಬಿದ್ದದ್ದೆಲ್ಲಾ ಹರಿದುಕೊಂಡು ಹೋಗುವುದು, ಅದು ಪಾತ್ರೆಯಲ್ಲಿ ಇರುವುದಕ್ಕೆ ಆಗುವುದಿಲ್ಲ. ಆ ಪಾತ್ರೆಗೆ ಒಂದು ಮಿತಿ ಇದೆ. ಅರ್ಜುನ ಅತಿ ಆಸೆಯಿಂದ ಎಲ್ಲವನ್ನೂ ಹೇಳು ಎಂದರೂ ಶ‍್ರೀಕೃಷ್ಣನಿಗೆ ಅರ್ಜುನನ ಶಕ್ತಿಯ ಯೋಗ್ಯತೆ, ಅರ್ಜುನನಿಗಿಂತ ಚೆನ್ನಾಗಿ ಗೊತ್ತಿದೆ. ಅದಕ್ಕೆ ಅವನು ಒಂದು ಸ್ವಲ್ಪವನ್ನು ಅವನಿಗೆ ಹೇಳುತ್ತಾನೆ.

\begin{shloka}
ಕಥಂ ವಿದ್ಯಾಮಹಂ ಯೋಗಿಂಸ್ತ್ವಾಂ ಸದಾ ಪರಿಚಿಂತಯನ್~।\\ಕೇಷು ಕೇಷು ಚ ಭಾವೇಷು ಚಿಂತ್ಯೋಽ ಭಗವನ್ಮಯಾ \hfill॥ ೧೭~॥
\end{shloka}

\begin{artha}
ಯೋಗೇಶ್ವರನೆ, ನಿನ್ನನ್ನು ಯಾವಾಗಲೂ ಧ್ಯಾನಿಸುತ್ತ ಹೇಗೆ ನಾನು ತಿಳಿದುಕೊಳ್ಳಲಿ? ಭಗವಾನ್, ಯಾವ ಯಾವ ಭಾವವಗಳಲ್ಲಿ ನಾನು ನಿನ್ನನ್ನು ಧ್ಯಾನಿಸಲಿ?
\end{artha}

ಇಲ್ಲಿ ಅರ್ಜುನ ಶ‍್ರೀಕೃಷ್ಣನನ್ನು ಯೋಗೇಶ್ವರ ಎಂದು ಕರೆಯುತ್ತಾನೆ. ಯೋಗ ಅಂದರೆ ಒಂದುಗೂಡಿಸುವುದು. ಜೀವಾತ್ಮನನ್ನು ಪರಮಾತ್ಮನೊಂದಿಗೆ ಒಂದುಗೂಡಿಸುವುದು. ಶ‍್ರೀಕೃಷ್ಣ ಇದರಲ್ಲಿ ಪಾರಗಂತನಾದವನು. ಅವನ ಕಡೆಗೆ ಹೋಗಬೇಕಾದರೆ ಮೊದಲು ಬಹಿರ್ಮುಖವಾಗಿ ಹೋಗುತ್ತಿರುವ ಇಂದ್ರಿಯಗಳನ್ನೆಲ್ಲ ಅಂತರ್ಮುಖ ಮಾಡಬೇಕು. ಅನಂತರ ಅದನ್ನು ಪರಮಾತ್ಮನ ಕಡೆ ತಿರುಗಿಸಬೇಕು. ನಾವು ಕೂಡ ಯೋಗೇಶ್ವರನನ್ನು ಅರಿಯಬೇಕಾದರೆ, ಯೋಗಶಕ್ತಿಯನ್ನು ಸಂಪಾದಿಸಿರಬೇಕು. ಯೋಗಶಕ್ತಿ ಇಲ್ಲದೆ ನಾವು ಯೋಗೇಶ್ವರನನ್ನು ಅರಿಯಲಾಗುವುದಿಲ್ಲ. ನಾವು ಕೂಡ ಇಂದ್ರಿಯಗಳನ್ನು ನಿಗ್ರಹಿಸಿ ಮನಸ್ಸನ್ನು ಏಕಾಗ್ರಮಾಡಿ ಭಗವಂತನ ಕಡೆ ಹರಿಸಿದರೆ ಮಾತ್ರ ಅವನನ್ನು ಅರಿಯಬಹುದು.

ಭಗವಂತನನ್ನು ಯಾವಾಗಲೂ ಧ್ಯಾನಿಸುತ್ತಿರಬೇಕು. ಆಗ ಮಾತ್ರ ಮನಸ್ಸಿಗೆ ಅವನನ್ನು ತಿಳಿದುಕೊಳ್ಳುವ ಶಕ್ತಿ ಬರುವುದು. ಎಲ್ಲೋ ಕೆಲವು ಸಮಯದಲ್ಲಿ ಮಾತ್ರ ಅವನನ್ನು ಕುರಿತು ಚಿಂತಿಸಿ ಇತರ ಸಮಯದಲ್ಲಿ ಬೇರೆ ವಸ್ತುವಿನ ಕಡೆ ಹರಿಯ ಬಿಟ್ಟರೆ ಅವನನ್ನು ಚೆನ್ನಾಗಿ ತಿಳಿದುಕೊಳ್ಳಲಾಗುವುದಿಲ್ಲ. ಅವನನ್ನು ಕುರಿತು ಧ್ಯಾನಿಸಬೇಕಾದರೆ ಮನಸ್ಸಿಗೆ ಯಾವುದಾದರೂ ಆಧಾರ ಇರಬೇಕು. ಬಳ್ಳಿ ಮೇಲೆ ಹಬ್ಬಿ ಹೋಗಬೇಕಾದರೆ ಅದಕ್ಕೊಂದು ಚಪ್ಪರವೊ ಬೇಲಿಯೋ ಯಾವುದಾದರೂ ಆಧಾರ ಬೇಕು. ಅದರಂತೆಯೇ ನಮ್ಮ ಮನಸ್ಸಿನ ಬಳ್ಳಿ ಧ್ಯಾನ. ಅದಕ್ಕೆ ಕೆಲವು ಗುಣಗಳ ಸಹಾಯ ಬೇಕು ಹಿಡಿದುಕೊಳ್ಳುವುದಕ್ಕೆ. ಗುಣಗಳು, ರೂಪ, ಇವು ಹಿಡಿದುಕೊಳ್ಳುವುದಕ್ಕೆ ಇದ್ದರೆ ಮನಸ್ಸು ಜಾರುವ ಸಂಭವ ಇಲ್ಲ. ಆದಕಾರಣವೇ ಯಾವ ಯಾವ ಗುಣಗಳ ಮೂಲಕ ನಿನ್ನನ್ನು ಧ್ಯಾನಿಸಲಿ ಎಂದು ಕೇಳಿಕೊಳ್ಳುತ್ತಾನೆ. ಭಗವಂತನ ವಿಭೂತಿಯೆಲ್ಲಾ ಅವನ ಒಂದೊಂದು ಗುಣ. ಆ\break ಗುಣವನ್ನು ಮನನ ಮಾಡುತ್ತ ಹೋದರೆ, ಆ ಗುಣಿಯ ಪರಿಚಯ ನಮಗೆ ಆಗುವುದು. ಇದಕ್ಕೆ ಮುಂಚೆ ನಿನ್ನನ್ನು ಹೇಗೆ ತಿಳಿದುಕೊಳ್ಳಲಿ ಎಂದು ಕೇಳುತ್ತಾನೆ. ಬುದ್ಧಿಯು ಅವನನ್ನು ಹಿಡಿಯುವುದು. ನಾವು ಇಕ್ಕಳದಿಂದ ಸಾಮಾನನ್ನು ಬೀಳದಂತೆ ಹಿಡಿದುಕೊಳ್ಳುತ್ತೇವೆ. ಅದರಂತೆಯೇ ಬುದ್ಧಿ ದೇವರನ್ನು ಗ್ರಹಿಸಬೇಕಾಗುವುದು. ತನಗೆ ತಿಳಿದಿರುವುದರಿಂದ ತಿಳಿಯದೆ ಇರುವುದನ್ನು ಹಿಡಿಯಬೇಕಾಗಿದೆ. ನಮ್ಮ ಮನಸ್ಸು ಹಲವಾರು ಸೂಕ್ಷ್ಮ ಗುಣಗಳನ್ನು ಗ್ರಹಿಸುವುದು. ಹಾಗೆಯೇ ದೇವರಿಗೆ ಸಂಬಂಧಪಟ್ಟ, ಅವನ ಸರ್ವವ್ಯಾಪಿತ್ವ, ಅವನ ಶಕ್ತಿ, ಅವನ ಜ್ಞಾನ ಇವುಗಳನ್ನು ತನ್ನ ಯೋಗ್ಯತಾನುಸಾರ ತಿಳಿದುಕೊಳ್ಳುವ ಬಗೆ ಹೇಗೆ ಎಂದು ಕೇಳುತ್ತಾನೆ.

\begin{shloka}
ವಿಸ್ತರೇಣಾತ್ಮನೋ ಯೋಗಂ ವಿಭೂತಿಂ ಚ ಜನಾರ್ದನ~।\\ಭೂಯಃ ಕಥಯ ತೃಪ್ತಿರ್ಹಿ ಶೃಣ್ವತೋ ನಾಸ್ತಿ ಮೇಽಮೃತಮ್ \hfill॥ ೧೮~॥
\end{shloka}

\begin{artha}
ಜನಾರ್ದನ, ನಿನ್ನ ಯೋಗ ಮತ್ತು ವಿಭೂತಿಯನ್ನು ವಿಸ್ತಾರವಾಗಿ ಪುನಃ ಹೇಳು. ನಿನ್ನ ಅಮೃತಮಯವಾದ ವಾಣಿಯನ್ನು ಕೇಳುತ್ತಿರುವ ನನಗೆ ತೃಪ್ತಿಯಿಲ್ಲ.
\end{artha}

ಅರ್ಜುನ ಶ‍್ರೀಕೃಷ್ಣನ ಯೋಗೈಶ್ವರ್ಯ ಮತ್ತು ವಿಭೂತಿಯನ್ನು ವಿಸ್ತಾರವಾಗಿ ಕೇಳಬಯಸುತ್ತಾನೆ. ಈ ಪ್ರಪಂಚವನ್ನು ಸೃಷ್ಟಿ ಮಾಡಿರುವುದೇ ಅವನ ಯೋಗೈಶ್ವರ್ಯ. ಅವನು ಸೃಷ್ಟಿ ಮಾಡಿದ ಪ್ರಪಂಚದಲ್ಲಿ ಹೇಗೆ ಹಬ್ಬಿರುವನೋ ಅದೇ ವಿಭೂತಿ. ಈ ಬ್ರಹ್ಮಾಂಡವನ್ನು ಅವನು ತನ್ನ ಸಂಕಲ್ಪದಿಂದ ಸೃಷ್ಟಿಸಿರುವುದು ಮಾತ್ರವಲ್ಲ, ಆ ಸೃಷ್ಟಿಯ ಮಧ್ಯೆ ಅವನು ವ್ಯಾಪಿಸಿರುವನು. ಕಲ್ಲು ಭೂಮಿಯಲ್ಲೆಲ್ಲಾ ಸ್ವಲ್ಪ ಒಳಗೆ ಹೋದರೆ ಸಿಕ್ಕುವುದು. ಇದೇ ಕಲ್ಲು ನೆಲದಿಂದ ದೊಡ್ಡ ಪರ್ವತದಂತೆ ಕೆಲವು ಕಡೆ ಏಳುವುದು. ಹಾಗೆಯೇ ದೇವರು ಈ ಪ್ರಪಂಚದಲ್ಲೆಲ್ಲಾ ಹಿಂದೆ ಇರುವುದು ಮಾತ್ರವಲ್ಲ, ಕೆಲವು ಕಡೆ ಅವನು ಮೇಲೆದ್ದು ಕಾಣುವನು. ಮೇಲೆ ಇರುವುದನ್ನು ಎಲ್ಲರೂ ನೋಡಬಲ್ಲರು. ಇದರಿಂದ ಕೆಳಗಡೆಯೂ ಇದೇ ಕಲ್ಲು ಇರುವುದು ಎಂದು ಊಹಿಸುತ್ತೇವೆ. ಸ್ಥೂಲದಿಂದ ನಮ್ಮ ಮನಸ್ಸು ಸೂಕ್ಷ್ಮದ ಕಡೆ ಹೋಗುವುದು. ವ್ಯಕ್ತದಿಂದ ನಮ್ಮ ಮನಸ್ಸು ಅವ್ಯಕ್ತದ ಕಡೆಗೆ ಹೋಗುವುದು.

ವಿಸ್ತಾರವಾಗಿ ಪುನಃ ಹೇಳು ಎನ್ನುತ್ತಾನೆ. ಇದುವರೆಗೆ ಸಂಗ್ರಹವಾಗಿ ಹೇಳಿರುತ್ತಾನೆ. ಇದನ್ನು ಅರ್ಜುನ ರುಚಿ ನೋಡಿಯೂ ಇದ್ದಾನೆ. ಅದನ್ನೇ ಪುನಃ ರುಚಿ ನೋಡಬಯಸುವನು. ಊಟಕ್ಕೆ ಕುಳಿತರೆ ಮೊದಲು ಮಾಡಿರುವುದೆಲ್ಲ ಸ್ವಲ್ಪ ಸ್ವಲ್ಪವಾಗಿ ಎಲೆಗೆ ಬಡಿಸುವರು. ಅದನ್ನು ಮುಂಚೆ ರುಚಿ ನೋಡಿದ ಮೇಲೆ ನಮಗೆ ಯಾವುದು ಬೇಕೋ ಅದನ್ನು ಇನ್ನೊಮ್ಮೆ ಕೇಳಿ ಹಾಕಿಸಿಕೊಳ್ಳುವೆವು. ಇನ್ನೊಮ್ಮೆ ಹಾಕಿಸಿಕೊಳ್ಳುವಾಗ ಜಾಸ್ತಿ ಹಾಕಿಸಿಕೊಳ್ಳುವೆವು. ಅದರಂತೆಯೇ ಭಗವಂತನ ವಿಭೂತಿಯ ಮೇಲೆ ಹೆಚ್ಚು ಕಾಲ ಮನನ ಮಾಡಬೇಕೆಂದು ಅರ್ಜುನ ಬಯಸುವನು. ಅದನ್ನು ಪುನಃ ಕೇಳಬೇಕೆಂದು ಬಯಸುವನು. ಒಂದು ಸಲ ಕೇಳಿದರೆ ಸಪ್ಪೆಯಾಗುವುದಿಲ್ಲ ಇದು. ಇದೇ ಇಂದ್ರಿಯದ ಸುಖಕ್ಕೂ ಭಗವಂತನ ಆನಂದಕ್ಕೂ ಇರುವ ವ್ಯತ್ಯಾಸ. ಇಂದ್ರಿಯಸುಖವಾದರೂ ಒಮ್ಮೆ ಅನುಭವಿಸಿದರೆ ತಕ್ಷಣವೇ ಅದನ್ನು ಅನುಭವಿಸಲು ಸಾಧ್ಯವಿಲ್ಲ. ಅನುಭವಿಸುವಾಗಲೂ ಅದಕ್ಕೊಂದು ಮಿತಿ ಇದೆ. ಅದನ್ನು ಮೀರಿ ಹೋಗಲಾಗದು. ಆದರೆ ಭಗವಂತನಿಗೆ ಸಂಬಂಧಪಟ್ಟ ವಿಷಯವಾದರೋ ಭಕ್ತನಿಗೆ ಎಂದಿಗೂ ಹಳೆಯದಾಗುವುದಿಲ್ಲ. ಎಷ್ಟು ಕೇಳಿದರೂ ಕೇಳಬೇಕು ಎನ್ನಿಸುವುದು. ಎಷ್ಟು ಅನುಭವಿ ಸಿದರೂ ಅನುಭವಿಸಬೇಕು ಎನ್ನಿಸುವುದು. ಇನ್ನಾರೋ ಅವನ ವಿಷಯವನ್ನು ಹೇಳುತ್ತಿದ್ದರೆ ಕೇಳುವುದಕ್ಕೆ ಭಕ್ತನಿಗೆ ಆನಂದವಾಗುವುದು. ಆದರೆ ಸ್ವಯಂ ಭಗವಂತನೇ ತನ್ನ ವಿಷಯವನ್ನು ಹೇಳುವಾಗ ಅದರಲ್ಲಿ ಒಂದು ಅದ್ಭುತವಾದ ಆಕರ್ಷಣೆ ಇದೆ. ಆದಕಾರಣವೇ ಅರ್ಜುನ ನಿನ್ನ ವಾಣಿ ಅಮೃತ ದಂತಿದೆ ಎಂದು ಹೇಳುವುದು. ಅಮೃತವನ್ನು ಎಷ್ಟು ಕುಡಿದರೂ ಇನ್ನೂ ಬೇಕೆನಿಸುವುದೇ ಹೊರತು ಸಾಕೆನಿಸುವುದಿಲ್ಲ. ಸಾಕೆಂದರೆ ಅದು ಅಮೃತವಲ್ಲ. ಭಗವಂತನ ಬಾಯಿಂದಲೇ ಅವನ ವಿಷಯವನ್ನು ಕೇಳುವ ಭಾಗ್ಯ ಎಲ್ಲರ ಪಾಲಿಗೂ ಬರುವುದಿಲ್ಲ. ದುರ್ಲಭವಾದ ವಸ್ತು ಅರ್ಜುನನಿಗೆ ಲಭಿಸಿದೆ.

ಆಗಲೇ ಭಗವಂತ ತನ್ನ ವಿಭೂತಿಯನ್ನು ಮತ್ತೊಮ್ಮೆ ಅರ್ಜುನನಿಗೆ ಹೇಳುತ್ತಾನೆ:

\begin{shloka}
ಹಂತ ತೇ ಕಥಯಿಷ್ಯಾಮಿ ದಿವ್ಯಾ ಹ್ಯಾತ್ಮಾವಿಭೂತಯಃ।\\ಪ್ರಾಧ್ಯಾನ್ಯತಃ ಕುರುಶ್ರೇಷ್ಠ ನಾಸ್ತ್ಯಂತೋ ವಿಸ್ತರಸ್ಯ ಮೇ \hfill॥ ೧೯~॥
\end{shloka}

\begin{artha}
ಅರ್ಜುನ, ನನ್ನ ದಿವ್ಯವಾದ ವಿಭೂತಿಗಳಲ್ಲಿ ಮುಖ್ಯವಾಗಿರುವುದನ್ನು ನಿನಗೆ ಹೇಳುತ್ತೇನೆ. ಏಕೆಂದರೆ ನನ್ನ ವಿಭೂತಿಯ ವಿಸ್ತಾರಕ್ಕೆ ಕೊನೆಯಿಲ್ಲ.
\end{artha}

ಈ ಸೃಷ್ಟಿಯೇ ಅವನ ವಿಭೂತಿ. ಎದುರುಗಿರುವ ಪ್ರಪಂಚವನ್ನು ಎಷ್ಟು ಎಷ್ಟು ಆಳಕ್ಕೆ ಹೋಗಿ ತಿಳಿದುಕೊಳ್ಳುತ್ತ ಬರುವೆವೊ ಅಷ್ಟು ಅಷ್ಟು ಅವನ ಸಮೀಪಕ್ಕೆ ನಾವು ಹೋಗುತ್ತೇವೆ. ಸಣ್ಣದು ದೊಡ್ಡದು, ಜಡ ಚೇತನ ಎಲ್ಲದರ ಹಿಂದೆಯೂ ಅವನ ಮಹಿಮೆಯೇ ಇದೆ. ಆದರೆ ನಾವು ಎಲ್ಲವನ್ನೂ ಕುರಿತು ಚಿಂತಿಸುವುದಕ್ಕೆ ಸಾಧ್ಯವಿಲ್ಲ. ಅದಕ್ಕಾಗಿ ಕೆಲವನ್ನು ಆರಿಸಿಕೊಳ್ಳುತ್ತೇವೆ. ಭಗವಂತನಿಗೆ ಅನಂತ ಗುಣಗಳಿವೆ. ಆದರೆ ಅವನನ್ನು ಪೂಜೆಮಾಡುವಾಗ ಅಷ್ಟೋತ್ತರ ಶತ ನಾಮಾರ್ಚನೆಯೋ ಸಹಸ್ರನಾಮಾರ್ಚನೆಯೋ ಯಾವುದನ್ನಾದರೂ ಮಾಡುತ್ತೇವೆ. ಸಹಸ್ರಕ್ಕಿಂತ ಹೆಚ್ಚಾಗಿ ಅವನಿಗೆ ಗುಣಗಳಿಲ್ಲವೆಂದಲ್ಲ. ಅಷ್ಟನ್ನು ತಿಳಿಯುವುದರೊಳಗೇ ನಮಗೆ ಅವನ ಮೇಲೆ ಅಮಲೇರಿ, ಇನ್ನು ಹೆಚ್ಚಾಗಿ ತಿಳಿದುಕೊಳ್ಳಬೇಕೆಂದು ಬಯಸದೆ ಅವನಲ್ಲಿ ಪರವಶರಾಗುತ್ತೇವೆ. ಅವನ ಭಾವ ನಮ್ಮ ಮೇಲೆ ಬರುವುದಕ್ಕೆ ಮುಂಚೆ ಕೆಲವು ಗುಣಗಳನ್ನು ನಾವು ಚಿಂತಿಸುತ್ತೇವೆ. ಅನಂತರ ಆ ಭಾವವೇ ನಮ್ಮನ್ನು ಮೇಲಕ್ಕೆ ಒಯ್ಯುವುದು. ವಿಮಾನ ಆಕಾಶದಲ್ಲಿ ಸಂಚರಿಸುವುದು, ಆದರೆ ಮೊದಲು ಅದು ನೆಲದ ಮೇಲೆ ಇರುವುದು. ಏರೋಡ್ರೋಮಿನ ಮೇಲೆ ಕೆಲವು ಕಾಲ ವೇಗವಾಗಿ ಹೋಗಿ ಮೇಲೆದ್ದ ಮೇಲೆ ಇನ್ನು ನೆಲದ ಆವಶ್ಯಕತೆ ಇಲ್ಲ. ಅದರಂತೆಯೇ ಮೊದಲು ಮನಸ್ಸು ಭಗವಂತನ ವಿಭೂತಿಯ ಮೇಲೆ ನಿಂತು ಅದನ್ನು ತಿಳಿದುಕೊಳ್ಳುತ್ತ ಜಪಿಸಿ ಅನಂತರ ಭಗವಂತನ ಮೇಲೆ ಬರುವ ಪ್ರೇಮದಿಂದ ಮೇಲೆದ್ದ ಮೇಲೆ ಇನ್ನು ಯಾರ ಆಸರೆಯೂ ಇಲ್ಲದೆ ಇರಬಲ್ಲದು.

ಎಲ್ಲ ಕಡೆಯಲ್ಲಿಯೂ ಅವನ ವಿಭೂತಿಯು ಪಸರಿಸಿದೆ. ಯಾವ ಸಣ್ಣದನ್ನು ನೋಡಿದರೂ ಅದರ ಹಿಂದೆ ಮಹತ್ತು ಹುದುಗಿದೆ. ಯಾವ ಯಃಕಶ್ಚಿತ್ ವಸ್ತುವೂ ಭಗವಂತನ ಸೃಷ್ಟಿಯಲ್ಲಿ ಕೆಲಸಕ್ಕೆ ಬಾರದುದಲ್ಲ. ಭೂಮದಲ್ಲಿ ಅದಕ್ಕೊಂದು ಸ್ಥಾನವಿದೆ. ವಿಭೂತಿಗಳನ್ನು ಹೇಳುವಾಗ\break ಯಾವುದನ್ನು ಸುಲಭವಾಗಿ ಎಲ್ಲರೂ ಗ್ರಹಿಸಬಹುದೋ ಎಲ್ಲಿ ಅವನ ಶಕ್ತಿ ಎಲ್ಲಾ ಕಡೆಗಿಂತ ಚೆನ್ನಾಗಿ ವ್ಯಕ್ತವಾಗುವುದೊ, ಅಂತಹ ನಿದರ್ಶನಗಳನ್ನು ಕೊಡುತ್ತಾನೆ. ಏಕೆಂದರೆ ಶ‍್ರೀಕೃಷ್ಣ ಗೀತೆಯನ್ನು ಹೇಳುವುದಕ್ಕೆ ಅರ್ಜುನ ಒಂದು ನಿಮಿತ್ತ. ಭಗವಂತನ ಮುಂದೆ ಇಡೀ ಮಾನವ ಕುಲವೇ ಇದೆ. ಅರ್ಜುನ ಅದರ ಪ್ರತಿನಿಧಿ ಮಾತ್ರ. ಶ‍್ರೀಕೃಷ್ಣ ಜನರಿಗೆ ಅರ್ಥವಾಗುವುದನ್ನು ಹೇಳುತ್ತಾನೆ. ನಾವು ತಿಳಿದುಕೊಳ್ಳಲಾರ ದುದನ್ನೆಲ್ಲ ನಮ್ಮ ಮುಂದೆ ಚೆಲ್ಲಿ ನಮ್ಮನ್ನು ಕಕ್ಕಾಬಿಕ್ಕಿ ಮಾಡಲೆತ್ನಿಸುವುದಿಲ್ಲ. ತಾಯಿಗೆ ತನ್ನ ಅಡಿಗೆಮನೆಯಲ್ಲಿ ಭಕ್ಷ್ಯಭೋಜ್ಯಗಳನ್ನು ತಯಾರು ಮಾಡಲು ಎಷ್ಟೋ ವಸ್ತುಗಳಿವೆ. ಆದರೆ ಅವಳು ತನ್ನ ಎಳೆಯ ಮಗುವಿಗೆ ಕೊಡುವುದು ಅದು ಅರಗಿಸಿಕೊಳ್ಳುವ ಸ್ವಲ್ಪ ಹಾಲನ್ನು ಮಾತ್ರ. ಹಾಗೆಯೇ ಭಗವಂತ ತನ್ನ ಉಗ್ರಾಣದಲ್ಲಿರುವ ವಿಭೂತಿಗಳನ್ನೆಲ್ಲ ಹೇಳಲು ಹೋಗುವುದಿಲ್ಲ. ಯಾವುದನ್ನು ಅರ್ಜುನ ಅರ್ಥಮಾಡಿಕೊಳ್ಳಬಲ್ಲನೊ, ಅರ್ಜುನನಂತಿರುವ ಮುಕ್ಕಾಲುಪಾಲು ಮಾನವ ಕೋಟಿ ಅರ್ಥ ಮಾಡಿಕೊಳ್ಳಬಲ್ಲರೋ ಅದನ್ನು ಬಡಿಸುತ್ತಾನೆ. 

\begin{shloka}
ಅಹಮಾತ್ಮಾ ಗುಡಾಕೇಶ ಸರ್ವಭೂತಾಶಯಸ್ಥಿತಃ~।\\ಅಹಮಾದಿಶ್ಚ ಮಧ್ಯಂ ಚ ಭೂತನಾಮಂತ ಏವ ಚ \hfill॥ ೨೦~॥
\end{shloka}

\begin{artha}
ಅರ್ಜುನ, ನಾನು ಸಮಸ್ತ ಪ್ರಾಣಿಗಳ ಹೃದಯದಲ್ಲಿರುವ ಆತ್ಮ. ನಾನೇ ಸಮಸ್ತ ಪ್ರಾಣಿಗಳ ಆದಿ, ಮಧ್ಯ ಮತ್ತು ಅಂತ್ಯ.
\end{artha}

ಭಗವಂತನೇ ಎಲ್ಲಾ ಜೀವಾತ್ಮರ ಹಿಂದೆಯೂ ಇರುವನು. ಹೇಗೆ ಎಲ್ಲ ಅಲೆಗಳ ಹಿಂದೆ ಅನಂತ ಸಾಗರ ಇದೆಯೋ ಹಾಗೆಯೇ ಎಲ್ಲಾ ಜೀವ ಜೀವಿಗಳ ಹಿಂದೆ ಪರಮಾತ್ಮನೆಂಬ ಸಾಗರ ಇರುವುದು. ಅವನು ಇಲ್ಲಿ ಮಾನವನ ಆತ್ಮದ ಹಿಂದೆ ಮಾತ್ರ ಇರುವವನಲ್ಲ. ಈ ಪ್ರಪಂಚದಲ್ಲಿ ಸರ್ವ ಪ್ರಾಣಿಗಳ ಆತ್ಮದ ಹಿಂದೆಯೂ ಅವನೇ ಇರುವುದು. ಎಲ್ಲಾ ಪ್ರಾಣಿರಾಶಿಗಳು–ಅವರು ಮನುಷ್ಯರಾಗಲಿ, ಪ್ರಾಣಿಗಳಾಗಲಿ, ತರುಲತೆಗಳಾಗಲಿ, ಪಕ್ಷಿಕೀಟಗಳಾಗಲಿ, ಎಲ್ಲವೂ ಒಂದು ದೇವಾಲಯದಂತೆ. ಕೆಲವು ಅಚಲದೇವಾಲಯ ಮತ್ತೆ ಕೆಲವು ಸಚಲದೇವಾಲಯ. ಎಲ್ಲಾ ಜೀವರಾಶಿಗಳ ಗರ್ಭಗುಡಿಯಲ್ಲಿ ಬೆಳುಗುವವನೇ ಭಗವಂತ.

ಭಗವಂತನೇ ಸರ್ವ ಪ್ರಾಣಿಗಳ ಆದಿ. ಅವನೇ ಎಲ್ಲವನ್ನೂ ಸೃಷ್ಟಿಸಿರುವನು. ಮನುಷ್ಯರಲ್ಲಿ ಎಷ್ಟೋ ವಿಧ, ಪ್ರಾಣಿಗಳಲ್ಲಿ ಎಷ್ಟೋ ವಿಧ, ಪಶುಪಕ್ಷಿಗಳಲ್ಲಿ ಎಷ್ಟೋ ವಿಧ, ತರುಲತೆ ಪುಷ್ಪಗಳಲ್ಲಿ ಎಷ್ಟೋ ವಿಧ, ಲಕ್ಷಾಂತರ ಜಾತಿಗಳಿವೆ. ಪ್ರತಿಯೊಂದು ಜಾತಿಯಲ್ಲೂ ಒಂದರಂತೆ ಮತ್ತೊಂದು ಇಲ್ಲ. ಒಬ್ಬ ಮನುಷ್ಯನಂತೆ ಮತ್ತೊಬ್ಬನಿಲ್ಲ. ಒಬ್ಬ ಮನುಷ್ಯನಲ್ಲಿಯೇ, ಒಂದು ಬೆರಳಿನಂತೆ ಮತ್ತೊಂದು ಬೆರಳಿಲ್ಲ. ಒಂದು ಕೂದಲಿನಂತೆ ಮತ್ತೊಂದು ಇಲ್ಲ. ಸೃಷ್ಟಿಕರ್ತನಿಗೆ ಎರಡು ಒಂದೇ ಆಗಿದ್ದರೆ ಬೇಜರಾಗಿ ಕಾಣುವುದು. ಮರಗಳನ್ನೇ ತೆಗೆದುಕೊಳ್ಳೋಣ. ಒಂದೊಂದು ಮರವೂ ತನ್ನ ಕೊಂಬೆಗಳನ್ನು ಹರಡುವ ರೀತೆಯೇ ಬೇರೆ. ಪ್ರತಿಯೊಂದಕ್ಕೂ ಒಂದೊಂದು ವೈಶಿಷ್ಟ್ಯವಿರುವಂತೆ ಕಾಣುವುದು. ಲಕ್ಷೋಪಲಕ್ಷ ಜಾತಿಗಳನ್ನು ಸೃಷ್ಟಿಸಿದವ ಅವನೆ. ಆದರೆ ಇವುಗಳಾವುವೂ ತಮಗೆ ತಾವೆ ಇಲ್ಲ. ಒಂದಕ್ಕೂ ಮತ್ತೊಂದಕ್ಕೂ ಒಂದು ಸಂಬಂಧವಿದೆ. ದಾರದಿಂದ ವಿಧವಿಧದ ಹೂವುಗಳನ್ನು ತೆಗೆದುಕೊಂಡು ಹಾರ ಮಾಡುವಂತೆ ವೈವಿಧ್ಯತೆಗಳನ್ನೆಲ್ಲಾ ಒಂದು ಸಂಬಂಧಸೂತ್ರದಿಂದ ಬಿಗಿದಿರುವನು. ಕೋಟ್ಯಂತರ ಜ್ಯೋತಿ ವತ್ಸರಗಳಾಚೆಯಿಂದ ಬರುತ್ತಿರುವ ಬೆಳಕಿಗೂ ನನಗೂ ಸಂಬಂಧವಿದೆ. ನನ್ನ ಸುತ್ತಮುತ್ತಲಿರುವ ಪ್ರತಿಯೊಂದಕ್ಕೂ ನನ್ನ ಸಂಬಂಧವಿದೆ, ಅದರ ಸಂಬಂಧ ನನಗೆ ಇದೆ. ನನ್ನ ಪ್ರಭಾವವನ್ನು ಇತರ ವಸ್ತುಗಳ ಮೇಲೆ ಬೀರುತ್ತಿರುವುದು ಮಾತ್ರವಲ್ಲ. ಇತರ ವಸ್ತುಗಳ ಪ್ರಭಾವಕ್ಕೆ ನಾನು ಒಳಗಾಗುತ್ತಿರುವೆನು. ಈ ಬ್ರಹ್ಮಾಂಡವೆಲ್ಲ ಅನ್ಯೋನ್ಯ ಸಂಬಂಧ ಚೌಕಟ್ಟಿನೊಳಗೆ ಇದೆ. ಈ ಚೌಕಟ್ಟನ್ನು ನಿರ್ಮಿಸಿದವನು ಅವನೆ. ಅದರಲ್ಲಿರುವ ಚಿತ್ರವನ್ನು ಬರೆದವನು ಅವನೇ.

ಅವನೇ ಪಾಲನಾಶಕ್ತಿಯಂತೆ ಇರುವನು. ಜೀವಿಗಳ ಪರಿಪುಷ್ಟಿಗೆ ಎಲ್ಲಾ ಕಡೆಯಲ್ಲಿಯೂ ಒಂದು ವಾತಾವರಣವನ್ನು ನಿರ್ಮಿಸಿರುವನು. ಮಳೆಯಂತೆ ಬೀಳುವುದು, ಬೆಳೆಯಂತೆ ಬರುವುದು, ಹೂವು, ಹಣ್ಣುಗಳನ್ನು ಬಿಡುವುದು, ತಾಯಿತಂದೆಗಳಲ್ಲಿ ಮಕ್ಕಳ ಬೆಳವಣಿಗಾಗಿ ಇಟ್ಟಿರುವ ತ್ಯಾಗ ಮತ್ತು ಪ್ರೇಮ ಇವುಗಳೆಲ್ಲ ಅವನಿಂದಲೇ ಬಂದವು. ಈ ಪಾಲನ ಶಕ್ತಿ ಎಲ್ಲ ಜೀವರಾಶಿ\-ಗಳಲ್ಲಿಯೂ ತಾಂಡವವಾಡುವುದನ್ನು ನೋಡುತ್ತೇವೆ. ಈ ಪಾಲನ ಶಕ್ತಿಯೇ ಭಗವಂತನ\break ವಿಭೂತಿಯ ಒಂದು ಅಂಶ.

ಇವನೇ ರುದ್ರಶಕ್ತಿ, ಸಂಹಾರಕಶಕ್ತಿ, ಯಾವ ರೈತನ ಕೈ ಬೀಜಗಳನ್ನು ಬಿತ್ತುವುದೊ, ಕಳೆಗಳನ್ನು ಕೀಳುವುದೊ, ಅದೇ ಕೈ ಕುಡುಗೋಲನ್ನು ಹಿಡಿದು ಪೈರನ್ನು ಕತ್ತರಿಸುವುದು. ಹಾಗೆಯೆ ಭಗವಂತ ಒಂದು ಕಡೆಯಿಂದ ಸೃಷ್ಟಿಮಾಡುತ್ತಿದ್ದರೆ, ಮತ್ತೊಂದು ಕಡೆಯಿಂದ ಸಂಹಾರ ಮಾಡುತ್ತಿರುವನು. ಜ್ವಾಲಾಮುಖಿ ಅವನೆ, ಸಂಕ್ರಾಮಿಕ ಜಾಡ್ಯಗಳು, ಘೋರ ಯುದ್ಧಗಳು ಅವನೆ. ಪ್ರವಾಹದ ಹಿಂದೆ, ಬರಗಾಲದ ಹಿಂದೆ ಇರುವವನೂ ಅವನೆ. ಸ್ವಲ್ಪ ಸ್ವಲ್ಪವಾಗಿ ಕ್ಷಯಿಸಿ ದೇಹದಾರ್ಢ್ಯವನ್ನೆಲ್ಲ ಕಳೆದುಕೊಂಡು ದುರ್ಬಲನಾಗಿ, ನಿಸ್ಸಹಾಯನಾಗಿ ಸಾವಿನಲ್ಲಿ ಕೊನೆಗಾಣುವ ಘಟನೆಯಲ್ಲಿಯೂ ಅವನೇ ಇರುವನು. ಮಗುವಿನ ಎಳೆನಗೆ, ತಾರುಣ್ಯದ ಉತ್ಸಾಹ, ವೃದ್ಧಾಪ್ಯದ ವ್ಯಾಧಿ ಎಲ್ಲದರ ಹಿಂದೆಯೂ ಕೆಲಸಮಾಡುತ್ತಿರುವುದೇ ಭಗವಂತನ ಶಕ್ತಿ. ಸೃಷ್ಟಿಯಲ್ಲಿ, ಪಾಲನದಲ್ಲಿ ದೇವರನ್ನು ನೋಡುವುದು ಸುಲಭ. ಆದರೆ ಸಂಹಾರದ ಹಿಂದೆ ನೋಡುವ ಧೀರರು ಬಹಳ ಅಪರೂಪ. ಅಂಜಿಕೆಯಿಂದ ಕಣ್ಣುಮುಚ್ಚಿಕೊಳ್ಳುವವರೇ ಜಾಸ್ತಿ. ಆದರೆ ನಿಜವಾದ ಜ್ಞಾನಿಗಳು, ನಿಜವಾದ ಭಕ್ತರು, ಎಲ್ಲದರ ಹಿಂದೆಯೂ ಅವನ ಕೈವಾಡವನ್ನೇ ಕಾಣುತ್ತಾರೆ.

\begin{shloka}
ಆದಿತ್ಯಾನಾಮಹಂ ವಿಷ್ಣುರ್ಜ್ಯೋತಿಷಾಂ ರವಿರಂಶುಮಾನ್~।\\ಮರೀಚಿರ್ಮರುತಾಮಸ್ಮಿ ನಕ್ಷತ್ರಾಣಾಮಹಂ ಶಶೀ \hfill॥ ೨೧~॥
\end{shloka}

\begin{artha}
ಆದಿತ್ಯರಲ್ಲಿ ವಿಷ್ಣು ನಾನು, ಜ್ಯೋತಿಗಳಲ್ಲಿ ಪ್ರಕಾಶಮಾನವಾದ ಸೂರ್ಯ ನಾನು, ವಾಯುಗಳಲ್ಲಿ ಮರೀಚಿ, ನಕ್ಷತ್ರಗಳಲ್ಲಿ ಚಂದ್ರ ನಾನು.
\end{artha}

ಅದಿತಿಗೆ ಹನ್ನೆರಡು ಜನ ಮಕ್ಕಳಿದ್ದರು. ಅವರಲ್ಲಿ ವಿಷ್ಣು ಬಹಳ ಪ್ರಖ್ಯಾತನಾದವನು. ಎಲ್ಲರಲ್ಲಿ ಅವನಿದ್ದರೂ ವಿಷ್ಣುವಿನಲ್ಲಿ ಅವನು ಹೆಚ್ಚು ವ್ಯಕ್ತವಾಗುತ್ತಾನೆ. ಆದಕಾರಣ ತಾನು ವಿಷ್ಣು ಎನ್ನುವನು. ಇದು ಪೌರಾಣಿಕ ಭಾವನೆ. ಪ್ರಖ್ಯಾತವ್ಯಕ್ತಿಯ ಹಿಂದುಗಡೆ ಇರುವ ಪ್ರಖ್ಯಾತಿಯೂ ತಾನೆ ಎನ್ನುವನು.

ನಮಗೆ ಸೂರ್ಯನನ್ನು ಮೀರಿದ ಜ್ಯೋತಿ ಮತ್ತಾವುದೂ ಇಲ್ಲ. ಇದರಷ್ಟು ಪ್ರಯೋಜನವಾಗಿರುವ ಜ್ಯೋತಿ ಮತ್ತಾವುದೂ ಇಲ್ಲ. ನಮ್ಮ ಜೀವನವೇ ಸೂರ್ಯನ ಜ್ಯೋತಿಯ ಮೇಲೆ ನಿಂತಿದೆ. ಆದಕಾರಣವೇ ನಾವು ಸೂರ್ಯನನ್ನು ಬರೀ ಸೂರ್ಯ ಎಂದು ಕರೆಯುವುದಿಲ್ಲ. ಅವನನ್ನು ಸೂರ್ಯನಾರಾಯಣ ಎಂದು ಕರೆಯುತ್ತೇವೆ. ನಾರಾಯಣನೇ ಸೂರ್ಯನಂತೆ ಬೆಳಗುತ್ತಿರುವನು.

ಮರುತ್ ಗಣಗಳು ಎಂದರೆ ಪ್ರಪಂಚವನ್ನೆಲ್ಲ ವ್ಯಾಪಿಸಿಕೊಂಡಿರುವ ಗಾಳಿ. ಅವು ಬಗೆ ಬಗೆಯಾಗಿರುತ್ತವೆ. ಕೆಲವು ಭೂಮಿಗೆ ಸಮೀಪದಲ್ಲಿವೆ. ಅದು ತಂಗಾಳಿಯಂತೆ ಬೀಸುವುದು, ಚಂಡಮಾರುತದಂತೆ ಬೀಸುವುದು, ಸುಂಟರಗಾಳಿಯಂತೆ ಬೀಸುವುದು. ಗಾಳಿ ಒಂದೇ ಆದರೂ, ಅದು ವಿವಿಧ ಆಕಾರಗಳನ್ನು ಮತ್ತು ಹೆಸರುಗಳನ್ನು ತಾಳುವುದು. ಅದರಲ್ಲಿ ಮರೀಚಿ ಎಂಬ ಗಾಳಿಯ ಪ್ರವಾಹ ನಮಗೆ ಬಹಳ ಹತ್ತಿರವಾಗಿರುವುದು. ಅದರಿಂದಲೇ ನಾವು ಬಾಳಿ ಬದುಕ\-ಬೇಕಾದರೆ. ಎಲ್ಲಾ ಗಾಳಿಯಲ್ಲಿ ಅವನಿದ್ದರೂ, ನಮ್ಮ ದೃಷ್ಟಿಯಿಂದ ಅವನು ಇದರಲ್ಲಿ ಹೆಚ್ಚಾಗಿ ಇರುವಂತೆ ತೋರುತ್ತಾನೆ.

ನಕ್ಷತ್ರಗಳ ಮಧ್ಯದಲ್ಲಿ ಚಂದ್ರ ಎನ್ನುತ್ತಾನೆ. ಸೂರ್ಯನ ವಿನಃ ಬೇರೆ ಪ್ರಕಾಶಮಾನವಾದ ವಸ್ತುಗಳು ನಕ್ಷತ್ರಗಳು. ಬಹುಶಃ ಆಗಿನ ಕಾಲದಲ್ಲಿ ಗ್ರಹಗಳು ಮತ್ತು ನಕ್ಷತ್ರಗಳು ಇವುಗಳಿಗೆ ವ್ಯತ್ಯಾಸವನ್ನು ಅಷ್ಟು ಮಾಡಿರಲಿಲ್ಲ ಎಂದು ತೋರುವುದು. ಮೇಲೆ ಬೆಳಗುವ ವಸ್ತುಗಳಲ್ಲಿ ನಮಗೆ ಬಹಳ ಸಮೀಪವಾಗಿರುವವನು, ಸೂರ್ಯನನ್ನು ಬಿಟ್ಟರೆ ಅತ್ಯಂತ ಕಾಂತಿ ಕೊಡುವವನು ಎಂದರೆ ಚಂದ್ರ. ಆ ಪ್ರಕಾಶದ ಹಿಂದೆಯೂ ಅವನೇ ಇರುವನು. ಶಾಖವನ್ನು ಕೊಡುವ ಕೋರೈಸುತ್ತಿರುವ ಸೂರ್ಯನ ಹಿಂದೆ ಹೇಗೆ ಅವನು ಇರುವನೊ, ಹಾಗೆಯೇ ಆಹ್ಲಾದಕರವಾಗಿ ತಂಪಿನ ಕಾಂತಿಯನ್ನು ಕೊಡುವ ಚಂದ್ರನ ಹಿಂದೆಯೂ ಅವನೇ ಇರುವನು.

\begin{shloka}
ವೇದಾನಾಂ ಸಾಮವೇದೋಽಸ್ಮಿ ದೇವಾನಾಮಸ್ಮಿ ವಾಸವಃ~।\\ಇಂದ್ರಿಯಾಣಾಂ ಮನಶ್ಚಾಸ್ಮಿ ಭೂತಾನಾಮಸ್ಮಿ ಚೇತನಾ \hfill॥ ೨೨~॥
\end{shloka}

\begin{artha}
ನಾನು ವೇದಗಳಲ್ಲಿ ಸಾಮವೇದ. ದೇವತೆಗಳಲ್ಲಿ ಇಂದ್ರ, ಇಂದ್ರಿಯಗಳಲ್ಲಿ ಮನಸ್ಸು ಮತ್ತು ಭೂತಗಳಲ್ಲಿ ಚೈತನ್ಯ.
\end{artha}

ನಾನು ವೇದಗಳಲ್ಲಿ ಸಾಮವೇದ ಎನ್ನುವನು. ಸಾಮವೇದವನ್ನು ರಾಗವಾಗಿ ಹೇಳಲೂಬಹುದು ಮತ್ತು ಅದರಲ್ಲಿ ಬಹಳ ಉದಾತ್ತವಾದ ಅರ್ಥವೂ ಇದೆ. ಇಂದ್ರಿಯಗಳಲ್ಲಿ ಮನಸ್ಸು ಎನ್ನುತ್ತಾನೆ. ಐದು ಕರ್ಮೇಂದ್ರಿಯಗಳಿವೆ ಮತ್ತು ಐದು ಜ್ಞಾನೇಂದ್ರಿಯಗಳಿವೆ. ಅದರ ಹಿಂದೆ ಮನಸ್ಸಿದೆ. ಇದೆಲ್ಲ ಸೇರಿ ಇಂದ್ರಿಯವಾಗುವುದು. ಕರ್ಮೇಂದಿಯಗಳೇ ಕೈ, ಕಾಲು, ವಾಕ್ಕು, ಪಾಯು ಮತ್ತು ಉಪಸ್ಥ. ಜ್ಞಾನೇಂದ್ರಿಯಗಳೇ ಕಣ್ಣು, ಕಿವಿ, ಮೂಗು, ಬಾಯಿ, ಚರ್ಮ. ಇವುಗಳೆಲ್ಲ ಕೆಲಸಮಾಡಬೇಕಾದರೆ ಅದರ ಹಿಂದೆ ಮನಸ್ಸಿದ್ದರೆ ಮಾತ್ರ ಸಾಧ್ಯ. ಮನಸ್ಸಿಲ್ಲದಿದ್ದರೆ ಕಣ್ಣು ನೋಡುತ್ತಿದ್ದರೂ ನೋಡಲಾರದು, ಕಿವಿ ಕೇಳುತ್ತಿದ್ದರೂ ಕೇಳಲಾರದು. ತಿನ್ನುತ್ತಿದ್ದರೂ ರುಚಿ ತಿಳಿದುಕೊಳ್ಳಲಾರದು. ನಾವು ಯಾವುದೋ ಒಂದು ಗಾಢ ಆಲೋಚನೆಯಲ್ಲಿ ತತ್ಪರರಾಗಿರುವಾಗ ನಮ್ಮ ಕಣ್ಣುಮುಂದೆಯೇ ಏನೇನೋ ಆಗಬಹುದು. ಆದರೂ ಅದನ್ನು ನಾವು ಗಮನಿಸುವುದಿಲ್ಲ. ಏಕೆಂದರೆ ಮನಸ್ಸು ಆ ಸಮಯದಲ್ಲಿ ಮತ್ತಾವುದರಲ್ಲೋ ತನ್ಮಯವಾಗಿರುವುದು. ಇತರ ಇಂದ್ರಿಯಗಳನ್ನು ಉಜ್ಜೀವನ ಗೊಳಿಸುವುದೇ ಅದರ ಹಿಂದೆ ಇರುವ ಮನಸ್ಸು.

ದೇವತೆಗಳೆಲ್ಲೆಲ್ಲ ಸರ್ವಶ್ರೇಷ್ಠನಾದವನೆ ಇಂದ್ರ. ಅವನ ಆಜ್ಞಾನುಸಾರ ಇತರ ದೇವತೆಗಳು ಕೆಲಸ ಮಾಡುವರು. ಭೂತಗಳಲ್ಲಿ ಚೈತನ್ಯ ಎನ್ನುತ್ತಾನೆ. ಜೀವರಾಶಿಗಳಲ್ಲಿ ಅತ್ಯಂತ ಮುಖ್ಯವಾಗಿರುವುದೇ ಚೈತನ್ಯ ಎಂದರೆ ಜೀವದ ಅಂಶ. ಅದಿದ್ದರೇ ಬದುಕಿರಬೇಕಾದರೆ. ಎಂತಹ ಪ್ರಾಣಿಯಾಗಲಿ, ಎಂತಹ ದೊಡ್ಡ ಮೇಧಾವಿ ಆಗಲಿ ಮುಂಚೆ ಅವನು ಬದುಕಿದ್ದರೆ ಮಾತ್ರ ಬೆಲೆ. ಇಲ್ಲದೆ ಇದರೆ ಇಲ್ಲ.

\begin{shloka}
ರುದ್ರಾಣಾಂ ಶಂಕರಶ್ಚಾಸ್ಮಿ ವಿತ್ತೇಶೋ ಯಕ್ಷರಕ್ಷಸಾಮ್~।\\ವಸೂನಾಂ ಪಾವಕಶ್ಚಾಸ್ಮಿ ಮೇರುಃ ಶಿಖರಿಣಾಮಹಮ್ \hfill॥ ೨೩~॥
\end{shloka}

\begin{artha}
ನಾನು ಏಕಾದಶ ರುದ್ರರಲ್ಲಿ ಶಂಕರ. ಯಕ್ಷ ರಾಕ್ಷಸರಲ್ಲಿ ಕುಬೇರ, ಅಷ್ಟವಸುಗಳಲ್ಲಿ ಅಗ್ನಿ, ಶಿಖರಗಳಲ್ಲಿ ಮೇರು.
\end{artha}

ಹನ್ನೊಂದು ಜನ ರುದ್ರರಲ್ಲಿ ತುಂಬಾ ಪ್ರಖ್ಯಾತನಾದವನೆ ಶಂಕರ. ಆದಕಾರಣವೇ ನಾನೆ ಅವನು ಎನ್ನುತ್ತಾನೆ. ಯಕ್ಷರಾಕ್ಷಸರ ತಂಡಕ್ಕೆ ಕುಬೇರ ಅಧಿಪತಿ. ಅವನು ಐಶ್ವರ್ಯಕ್ಕೆ ಅಧಿಪತಿ ಎಂದು ಹೆಸರಾಂತ ವ್ಯಕ್ತಿಯಾಗಿರುವನು. ಅವನ ಹಿಂದೆ ಒಂದು ರಾಕ್ಷಸರ ತಂಡವಿತ್ತೆಂದು ಪುರಾಣಗಳು ಹೇಳುವುವು. ರಾವಣ ಕೂಡ ಕುಬೇರನ ತಮ್ಮನಾಗಿದ್ದ. ಆರು ಜನ ವಸುಗಳಲ್ಲಿ ಅಗ್ನಿ ತುಂಬಾ ಮುಖ್ಯ. ಹಿಂದಿನ ಕಾಲದಲ್ಲಿ ನಾವು ಕೊಡುತ್ತಿದ್ದ ಆಹುತಿಗಳನ್ನು ಅಗ್ನಿಯೇ ಆಯಾ ದೇವದೇವತೆಗಳಿಗೆ ಅರ್ಪಿಸುತ್ತಿದ್ದನು. ಪೌರಾಣಿಕ ದೇವತೆಗಳಲ್ಲಿ ಇವನು ತುಂಬಾ ಪ್ರಸಿದ್ಧ. ಈಗಿನ ಕಾಲ ದಲ್ಲಂತೂ ಬೆಂಕಿಯ ಮೇಲೆ ನಮ್ಮ ನಾಗರಿಕತೆ ನಿಂತಿರುವುದು. ನಮ್ಮ ದೇಹ ಕೆಲಸಮಾಡಲು ಆಹಾರವನ್ನು ಕೊಡುವುದಕ್ಕೆ ಮುಂಚೆ ಅದು ಅಗ್ನಿಯಿಂದ ಬೆಂದಿರಬೇಕು. ಫ್ಯಾಕ್ಟರಿಯ ಯಂತ್ರಗಳು ಕೆಲಸ ಮಾಡಬೇಕಾದರೆ ಒಂದಲ್ಲ ಒಂದು ರೂಪಿನಲ್ಲಿ ಅಗ್ನಿ ಇರಬೇಕು. ನಮಗೆ ದೀಪ ಬೇಕಾದರೆ, ಕಾವು ಬೇಕಾದರೆ ಅಗ್ನಿ ಬೇಕು. ಯಾಗ ಯಜ್ಞಗಳನ್ನು ಮಾಡಬೇಕಾದರೂ ಅಗ್ನಿ ಬೇಕು. ಶಿಖರಗಳಲ್ಲಿ ಮೇರು ಎನ್ನುವರು. ಮೇರು ಪರ್ವತ ಎಂಬುದು ಬಹಳ ಎತ್ತರವಾದ ಪರ್ವತ. ಇದು ಪೌರಾಣಿಕ ಪರ್ವತ. ದೇವತೆಗಳು ಇದರ ಮೇಲೆ ವಾಸ ಮಾಡುತ್ತಿದ್ದರೆಂದೂ ಸೂರ್ಯ ಚಂದ್ರರು ಇದರ ಸುತ್ತ ಸುತ್ತುತ್ತಿದ್ದರೆಂದೂ ಹೇಳುವರು. ಇದರಲ್ಲಿ ಬಹಳ ಸುಂದರವಾದ ಶಿಖರಗಳಿವೆ. ಆ ಶಿಖರಗಳಲ್ಲಿ ಮೇರು ಶಿಖರ ನಾನೇ ಎನ್ನುವನು. ಇಲ್ಲಿ ಆ ಹಳೆಯ ಪರ್ವತ ವಾಸ್ತವಿಕವಾಗಿ ಇರಲೇಬೇಕೆಂದು ಅಲ್ಲ. ಯಾವ ಪ್ರಪಂಚದಲ್ಲಾದರೂ ಇರಲಿ, ಕೇವಲ ಕಲ್ಪನೆಯೇ ಆಗಲಿ, ಪುರಾಣಗಳಲ್ಲಿ ಬರುವುದೇ ಆಗಲಿ ಅಥವಾ ಸುಂದರವಾದ ಶಿಖರಗಳುಳ್ಳ ಮತ್ತೆ ಬೇರಾವುದಾದರೂ ಆಗಲಿ; ಅಂತೂ ಆ ಸೌಂದರ್ಯಕ್ಕೆ ಇದು ಅನ್ವಯಿಸುವುದು.

\begin{shloka}
ಪುರೋಧಸಾಂ ಚ ಮುಖ್ಯಂ ಮಾಂ ವಿದ್ಧಿ ಪಾರ್ಥ ಬೃಹಸ್ಪತಿಮ್~।\\ಸೇನಾನೀನಾಮಹಂ ಸ್ಕಂದಃ ಸರಸಾಮಸ್ಮಿ ಸಾಗರಃ \hfill॥ ೨೪~॥
\end{shloka}

\begin{artha}
ಅರ್ಜುನ, ರಾಜಪುರೋಹಿತರಲ್ಲಿ ಮುಖ್ಯನಾದ ಬೃಹಸ್ಪತಿ ನಾನು ಎಂದು ತಿಳಿದುಕೊ. ನಾನು ಸೇನಾಪತಿಗಳಲ್ಲಿ ಸ್ಕಂದ. ಸರಸ್ಸುಗಳಲ್ಲಿ ಸಾಗರ.
\end{artha}

ಬೃಹಸ್ಪತಿ ಎಂಬುವನು ದೇವತೆಗಳ ಪುರೋಹಿತ. ಆತ ಬಹಳ ಬುದ್ಧಿವಂತ. ಈಗಲೂ ಕೂಡ ಯಾರಾದರೂ ಬಹಳ ಬುದ್ಧಿವಂತರಾಗಿದ್ದರೆ ನಾವು ಅವರನ್ನು ಬೃಹಸ್ಪತಿ ಎಂದು ಕರೆಯುತ್ತೇವೆ. ಆಗಿನ ಕಾಲದ ಪುರೋಹಿತನನ್ನು ಈಗಿನ ಕಾಲದಲ್ಲಿ ಶೋಚನೀಯ ಸ್ಥಿತಿಯಲ್ಲಿರುವ ಪುರೋಹಿತ ರೊಂದಿಕೆ ಹೋಲಿಸಬಾರದು. ಆಗಿನ ಕಾಲದಲ್ಲಿ ಅವನು ಶ್ರೇಷ್ಠ ಮಂತ್ರಿಯಂತೆ. ಅತ್ಯಂತ ಶ್ರೇಷ್ಠವಾದ ಸಲಹೆಗಳನ್ನು ರಾಜನಿಗೆ ಕೊಡುವ ವಿದ್ವತ್ ಇತ್ತು, ಯಾವುದು ಸರಿ, ಯಾವುದು ತಪ್ಪು ಎಂಬುದನ್ನು ನಿರ್ಣಯಿಸುವ ಬುದ್ಧಿಶಕ್ತಿ ಇತ್ತು ಅವನಿಗೆ. ಆ ವಿದ್ವತ್ ಮತ್ತು ಬುದ್ಧಿಶಕ್ತಿಯ ಹಿಂದೆ ಇದ್ದುದು ಭಗವತ್ ಶಕ್ತಿ.

ನಾನು ಸೇನಾಪತಿಗಳಲ್ಲಿ ಸ್ಕಂದ ಎನ್ನುವನು. ತಾರಕಾಸುರನ ಸಂಹಾರಕ್ಕೆ ಶಿವನಿಂದ ಉದಯಿಸಿ\-ದವನು ಸ್ಕಂದ. ಹಲವು ಪುರಾಣಗಳು ಇದನ್ನು ವಿವರಿಸುತ್ತವೆ. ಕಾಳಿದಾಸನ ‘ಕುಮಾರಸಂಭವ’ದ ಕಥಾವಸ್ತುವಾದರೋ ಈ ಸ್ಕಂದನ ಜನನ ವಿಷಯವಾಗಿಯೇ ಇದೆ. ಸ್ಕಂದನೇ ಶಿವನೊಡನೆ ಸೇನಾನಿಯಾಗಿ ಹೋಗಿ ತಾರಕಾಸುರನನ್ನು ಗೆಲ್ಲುವನು.

ಸರಸ್ಸುಗಳಲ್ಲಿ ಅಂದರೆ, ನೀರು ನಿಲ್ಲುವ ಸ್ಥಳಗಳಲ್ಲಿ ಸಾಗರ ಎನ್ನುವರು. ಸಾಗರದಷ್ಟು ವಿಶಾಲವಾಗಿ ಪ್ರಪಂಚವನ್ನೆಲ್ಲ ಹಬ್ಬಿರುವುದು ಮತ್ತಾವುದೂ ಇಲ್ಲ. ಇರುವುದು ಒಂದೇ ಅನಂತ ಸಾಗರ. ಅದು ಬೇರೆಬೇರೆ ಕಡೆಗಳಲ್ಲಿ ಬೇರೆ ಬೇರೆ ಹೆಸರುಗಳಿಂದ ಕರೆಯಲ್ಪಪಡುತ್ತಿದೆ. ಅನಂತ ಸಾಗರದ ತಡಿಯಲ್ಲಿ ನಿಂತರೆ ನಮಗೆ ಭೂಮದ ದರ್ಶನವಾಗುವುದು. ಹೇಗೆ ಹಿಂದೂಗಳಿಗೆ ಗಂಗಾಸ್ನಾನ ಪವಿತ್ರವೋ ಹಾಗೇ ಸಾಗರ ಸ್ನಾನವೂ ಪವಿತ್ರ. ಆದಕಾರಣವೆ ಸಾಗರದ ಸಮೀಪದಲ್ಲಿರುವವರು ಪುಣ್ಯ ದಿನಗಳಲ್ಲಿ ಸ್ನಾನ ಮಾಡುವುದಕ್ಕೆ ಲಕ್ಷೋಪಲಕ್ಷ ಸಂಖ್ಯೆಯಲ್ಲಿ ನೆರೆಯುವರು.

\begin{shloka}
ಮಹರ್ಷೀಣಾಂ ಭೃಗುರಹಂ ಗಿರಾಮಸ್ಮ್ಯೇಕಮಕ್ಷರಮ್~।\\ಯಜ್ಞಾನಾಂ ಜಪಯಜ್ಞೋಽಸ್ಮಿ ಸ್ಥಾವರಾಣಾಂ ಹಿಮಾಲಯಃ \hfill॥ ೨೫~॥
\end{shloka}

\begin{artha}
ನಾನು ಮಹಾ ಪುಷಿಗಳಲ್ಲಿ ಭೃಗು. ಪದಗಳಲ್ಲಿ ಏಕಾಕ್ಷರವಾದ ಪ್ರಣವ, ಯಜ್ಞಗಳಲ್ಲಿ ಜಪಯಜ್ಞ, ಸ್ಥಾವರಗಳಲ್ಲಿ ಹಿಮಾಲಯ.
\end{artha}

ಭೃಗು ಸಪ್ತಮಹರ್ಷಿಗಳಲ್ಲಿ ಶ್ರೇಷ್ಠನಾದವನು. ತಪಸ್ಸಿನಲ್ಲಿ ಜ್ಞಾನದಲ್ಲಿ ಅವನನ್ನು ಮೀರು\-ವವರಿಲ್ಲ. ಪದಗಳಲ್ಲೆಲ್ಲಾ ಓಂಕಾರದಷ್ಟು ಸಂಕ್ಷೇಪವಾದ ಪದ ಮತ್ತೊಂದು ಇಲ್ಲ. ಜೊತೆಗೆ ಇದರಷ್ಟು ಪವಿತ್ರವಾದ ಮತ್ತಾವ ಪದವೂ ಇಲ್ಲ. ಓಂಕಾರ ಎಲ್ಲಾ ಮಂತ್ರಗಳ ತಾಯಿ, ಎಲ್ಲಾ ಶಬ್ದಗಳ ಮೂಲ. ಸಗುಣ ಮತ್ತು ನಿರ್ಗುಣಗಳೆರಡಕ್ಕೂ ಸಂಕೇತವಾಗಿರುವುದು ಇದು. ಪರಬ್ರಹ್ಮನ ಸಂಕೇತರೂಪವಾಗಿ ನಾವು ಓಂಕಾರವನ್ನು ಉಪಯೋಗಿಸುತ್ತೇವೆ.

ಯಜ್ಞಗಳಲ್ಲಿ ಜಪಯಜ್ಞವೇ ನಾನು ಎನ್ನುತ್ತಾನೆ ಶ‍್ರೀಕೃಷ್ಣ. ಜಪದಷ್ಟು ಸುಲಭವಾಗಿರುವುದು ಸರಳವಾಗಿರುವುದು ಮತ್ತಾವುದೂ ಇಲ್ಲ. ಇದು ಸುಲಭ ಮತ್ತು ಸರಳ ಮಾತ್ರ ಅಲ್ಲ. ಇದರಷ್ಟು ಪ್ರಭಾವಕಾರಿಯಾಗಿರುವುದು ಮತ್ತೊಂದು ಇಲ್ಲ. ಇದನ್ನು ಮಾಡುವುದು ಸುಲಭ. ನಮಗೆ ಪ್ರಿಯವಾದ ಯಾವುದಾದರೂ ದೇವರ ಹೆಸರನ್ನು ಮನದಲ್ಲೇ ಜಪಿಸಬಹುದು. ದೊಡ್ಡ ಮಂತ್ರವನ್ನು ಜ್ಞಾಪಕದಲ್ಲಿಡಬೇಕಾದರೆ ಕಷ್ಟ. ಸಣ್ಣ ದೇವರ ಹೆಸರನ್ನು ಜ್ಞಾಪಕದಲ್ಲಿಡುವುದು ಸುಲಭ. ಬಾಹ್ಯ ಪೂಜಾದಿಗಳನ್ನು ಮಾಡಬೇಕಾದರೆ ಅದಕ್ಕೆ ಸ್ಥಳ ಬೇಕು, ಇಪ್ಪತ್ತೆಂಟು ವಸ್ತುಗಳು ಬೇಕು. ಇವುಗಳಾವುವು ಬೇಕಾಗಿಲ್ಲ ಭಗವಂತನ ಹೆಸರನ್ನು ಜಪ ಮಾಡುವುದಕ್ಕೆ. ಇದನ್ನು ಮಾಡುವುದಕ್ಕೆ ಸ್ತ್ರೀ ಪುರುಷ ಭೇದವಿಲ್ಲ, ಬ್ರಾಹ್ಮಣ ಬ್ರಾಹ್ಮಣೇತರ ಭೇದವಿಲ್ಲ, ಮಡಿ ಮೈಲಿಗೆಯ ಕಟ್ಟುನಿಟ್ಟುಗಳೂ ಬೇಕಾಗಿಲ್ಲ. ಯಾರು ಬೇಕಾದರೂ ಯಾವ ಸ್ಥಿತಿಯಲ್ಲಿ ಬೇಕಾದರೂ ಇದನ್ನು ಮಾಡಬಹುದು.

ಜಪವೆಂದರೆ ಭಗವಂತನ ನಾಮವನ್ನು ಮನದಲ್ಲಿ ಚಿಂತಿಸುವುದು. ದೇವರ ಹೆಸರೇ ಅವನಿಗೆ ಕಟ್ಟಿರುವ ದಾರ. ಪಟ ದೂರದಲ್ಲಿ ಆಡುತ್ತಿದ್ದರೂ ಅದರ ದಾರ ನಮ್ಮ ಕೈಯ್ಯಲ್ಲಿ ಇದ್ದರೆ, ಅದನ್ನು ನಮ್ಮ ಹತ್ತಿರ ಸೆಳೆಯಬಹುದು. ದೇವರ ಹೆಸರಿನಲ್ಲಿ ಅವನ ಶಕ್ತಿಯೆಲ್ಲ ಸ್ಪಂದಿಸುತ್ತಿದೆ. ಸಾಸುವೆ ಕಾಳಿಗಿಂತ ಸಣ್ಣದಾದ ಬೀಜದಲ್ಲಿ ದೊಡ್ಡ ಆಲದಮರ ಇರುವ ಹಾಗೆ ಭಗವಂತನ ಸಣ್ಣ ನಾಮದಲ್ಲಿ ದೊಡ್ಡ ಶಕ್ತಿ ಅವಿತಿರುವುದು. ಆ ನಾಮವನ್ನು ಶ್ರದ್ಧೆಯಿಂದ ಕೃಷಿ ಮಾಡಿದರೆ ಎಕರೆಗಳಷ್ಟು ವಿಸ್ತಾರವಾಗಿ ಹಬ್ಬುವ ಆಲದಮರದಂತೆ ಬೆಳಸಬಹುದು. ದೇವರು ಹೇಗಿದ್ದಾನೆಯೋ ನಮಗೆ ಅದು ಗೊತ್ತಿಲ್ಲ. ಆದರೆ ಅವನ ನಾಮ ನಮಗೆ ಗೊತ್ತಿದೆ. ಅದನ್ನು ಜಪಿಸುತ್ತಾ ಹೋದರೆ ಸಾಕು ಅದರಲ್ಲಿರುವ ಶಕ್ತಿಯೆಲ್ಲ ಜಾಗೃತವಾಗಿ ನಮ್ಮ ಜೀವನದ ಮೇಲೆ ತನ್ನ ಪ್ರಭಾವವನ್ನು ಬೀರುತ್ತದೆ. ಶ‍್ರೀರಾಮಕೃಷ್ಣರು ನಾಮ ಮತ್ತು ನಾಮಿ ಎರಡೂ ಒಂದೇ ನಾಣ್ಯದ ಎರಡು ಕಡೆಗಳಂತೆ ಎಂದು ಹೇಳುತ್ತಿದ್ದರು. ಒಂದು ಕಡೆ ಅವನ ಹೆಸರಿದೆ, ಮತ್ತೊಂದು ಕಡೆ ಅವನ ಜ್ಞಾನ ಪವಿತ್ರತೆ ಶಕ್ತಿಗಳೆಲ್ಲಾ ಇವೆ. ಒಂದಿದ್ದರೆ ಮತ್ತೊಂದು ಅದರ ಹಿಂದೆ ಇದ್ದೇ ಇರುವುದು. ಭಗವಂತನಿಗಿಂತ ಅವನ ಹೆಸರಿನಲ್ಲಿ ಹೆಚ್ಚು ಶಕ್ತಿ ಇದೆ ಎಂದು ನಮ್ಮ ದಾಸರು ಹಾಡುತ್ತಾರೆ. “ನೀನ್ಯಾಕೊ ನಿನ್ನ ಹಂಗ್ಯಾಕೊ, ನಿನ್ನ ನಾಮದ ಬಲವೊಂದಿದ್ದರೆ ಸಾಕೊ” ಎಂದು ಕೇವಲ ಅವನ ನಾಮದಲ್ಲೆ ಶರಣಾಗುವವರು ಇದ್ದಾರೆ. ಶ‍್ರೀರಾಮಕೃಷ್ಣರು ಒಮ್ಮೆ ಹಾಸ್ಯವಾಗಿ ಹೇಳುತ್ತಾರೆ, ಸಾಕ್ಷಾತ್ ರಾಮನೆ ಲಂಕೆಗೆ ಹೋಗಲು ಸೇತುವೆ ಕಟ್ಟಬೇಕಾಯಿತು. ಆದರೆ ಹನುಮಂತನಾದರೋ ರಾಮನ ಹೆಸರಿನ ಬಲದಿಂದಲೇ ಸಾಗರವನ್ನು ದಾಟಿದ! ಭಗವಂತನ ನಾಮ ಜೇಬಿನಲ್ಲಿಟ್ಟುರುವ ರೇಡಿಯೋ ಸೆಟ್ಟಿನಂತೆ. ಕೂತ ಕಡೆ ನಿಂತ ಕಡೆ ಅವನ ನಾಮವನ್ನು ಉಚ್ಚರಿಸುತ್ತ ಅವನಿಗೆ ಸಂಬಂಧಪಟ್ಟ ಭಾವಗಳನ್ನು ಮೆಲ್ಲಬಹುದು. ಅವನನ್ನು ಜಪಿಸುತ್ತ ಹೋದರೆ ಮನಸ್ಸು ಶುದ್ಧವಾಗುವುದು, ಏಕಾಗ್ರವಾಗುವುದು. ಇವುಗಳನ್ನೆಲ್ಲಾ ಮತ್ತಾವ ಸಾಧನೆಯಲ್ಲಿಯೂ ಇಷ್ಟು ಸುಲಭವಾಗಿ ಪಡೆಯಲಾರೆವು. ಆದಕಾರಣವೇ ಭಗವಂತ ತಾನು ಈ ಸಾಧನೆಯ ಹಿಂದೆ ಎಲ್ಲಾ ಸಾಧನೆಗಳಿಗಿಂತಲೂ ಹೆಚ್ಚಾಗಿ ಇದ್ದೇನೆ ಎಂದು ಹೇಳುತ್ತಾನೆ.

ಪರ್ವತಗಳಲ್ಲಿ ನಾನು ಹಿಮಾಲಯ ಎನ್ನುತ್ತಾನೆ. ಇದರಷ್ಟು ಭವ್ಯವಾಗಿರುವುದು, ವಿಸ್ತಾರ\-ವಾಗಿರುವುದು, ಎತ್ತರವಾಗಿರುವುದು ಮತ್ತಾವುದೂ ಇಲ್ಲ. ಸುಮಾರು ಸಾವಿರದ ಐದುನೂರು ಮೈಲಿಗಳಷ್ಟು ಉದ್ದವಾಗಿದೆ, ಐನೂರು ಮೈಲಿಗಳಷ್ಟು ವಿಸ್ತಾರವಾಗಿದೆ. ಆಕಾಶವನ್ನು ಭೇದಿಸಿಕೊಂಡು ಹೋಗುವಷ್ಟು ಶಿಖರಗಳೆಷ್ಟೊಂದು ಇವೆ. ತುಷಾರಹಾರವನ್ನು ಧರಿಸಿ ಧ್ಯಾನಾವಸ್ಥೆ\-ಯಲ್ಲಿರುವ ಪರಮಶಿವನ ನೆನಪನ್ನು ತರುವುವು ಈ ಶಿಖರಗಳು. ಇದರಡಿಯಲ್ಲಿ ನಮ್ಮ ಹಿಂದೂ ಸಂಸ್ಕೃತಿ ಅರಳಿತು. ಇದರಿಂದ ಉದಿಸಿ ಬಂದ ನದಿ ತೀರದಲ್ಲಿ ನಮ್ಮ ಪುಷಿಗಳು ತಪಸ್ಸು ಮಾಡಿದರು. ನಾವೆಲ್ಲ ಎಲ್ಲಿ ಇರಲಿ ಹಿಮಾಲಯದ ಮಕ್ಕಳು. ಅದರ ಪದತಳದಲ್ಲಿ ನಾವು ಬಾಳಿ ಬದುಕಿದೆವು, ಬೆಳಕು ಕಂಡೆವು. ಈ ಹಿಮಾಲಯವೇ ನಾನು ಎಂದರೆ ಅದೊಂದು ಅತಿಶಯೋಕ್ತಿಯಲ್ಲ. ಈ ಬ್ರಹ್ಮಾಂಡದಲ್ಲಿ ಪ್ರಕೃತಿಯೇ ಭಗವಂತನಿಗೆ ನಿರ್ಮಿಸಿದ ಬೃಹತ್ ದೇವಾಲಯ ಹಿಮಾಲಯ. ಅದರಲ್ಲಿರುವ ಆರಾಧ್ಯಮೂರ್ತಿಯೇ ಪರಮಾತ್ಮ. ನಾವು ಯಾವ ಹೆಸರಿನಿಂದ ಕರೆಯಲಿ ಎಲ್ಲಾ ಸಲ್ಲುವುದು ಅವನಿಗೆ.

\begin{shloka}
ಅಶ್ವತ್ಥಃ ಸರ್ವವೃಕ್ಷಾಣಾಂ ದೇವರ್ಷೀಣಾಂ ಚ ನಾರದಃ~।\\ಗಂಧರ್ವಾಣಾಂ ಚಿತ್ರರಥಃ ಸಿದ್ಧಾನಾಂ ಕಪಿಲೋ ಮುನಿಃ \hfill॥ ೨೬~॥ 
\end{shloka}

\begin{artha}
ನಾನು ಸಮಸ್ತ ವೃಕ್ಷಗಳಲ್ಲಿ ಅಶ್ವತ್ಥ, ದೇವಪುಷಿಗಳಲ್ಲಿ ನಾರದ, ಗಂಧರ್ವರಲ್ಲಿ ಚಿತ್ರರಥ, ಸಿದ್ಧರಲ್ಲಿ ಕಪಿಲಮುನಿ.
\end{artha}

ವೃಕ್ಷಗಳಲ್ಲಿ ಅಶ್ವತ್ಥ ಪವಿತ್ರವೆಂಬ ಭಾವನೆ ಹಿಂದಿನಿಂದ ಹಿಂದೂಗಳಲ್ಲಿ ಬಂದಿದೆ. ಯಾಗಯಜ್ಞಗಳ ಕಾಲದಲ್ಲಿ ಬೆಂಕಿಯನ್ನು ಉತ್ಪತ್ತಿಮಾಡುವಾಗ ಇದರ ಕಡ್ಡಿಯನ್ನು ಉಪಯೋಗಿಸುವರು. ತ್ರಿಮೂರ್ತಿಗಳ ವಾಸಸ್ಥಾನವೆಂದು ಈ ವೃಕ್ಷ ಪ್ರಖ್ಯಾತವಾಗಿದೆ. ವಿಷ್ಣು ಪ್ರಳಯಕಾಲದಲ್ಲಿ ಅರಳಿಯ ಎಲೆಯ ಮೇಲೆ ಪ್ರಳಯಜಲದಲ್ಲಿ ಮಗುವಿನ ರೂಪದಲ್ಲಿ ತೇಲುತ್ತಿರುವನು ಎಂಬ ಭಾವನೆ ಕೂಡ ಪುರಾಣಗಳಲ್ಲಿ ಬರುವುದು.

ನಾರದರು ದೇವಪುಷಿಗಳಲ್ಲಿ ಬಹಳ ಪ್ರಖ್ಯಾತರಾದವರು. ಅವರು ಭಗವಂತನ ಪರಮ ಭಕ್ತರು ಎಂದು ಪ್ರಖ್ಯಾತರಾಗಿದ್ದಾರೆ. ರಾಮಾಯಣ ಮಹಾಕಾವ್ಯವನ್ನು ವಾಲ್ಮೀಕಿ ಮಹರ್ಷಿಗಳಿಗೆ ಬರೆಯುವುದಕ್ಕೆ ಸ್ಫೂರ್ತಿಯನ್ನು ಕೊಟ್ಟವರು ನಾರದರು. ವ್ಯಾಸರಿಗೆ ಭಾಗವತವನ್ನು ಬರೆಯಬೇಕೆಂದು ಹೇಳಿದವರು ನಾರದರು. ಈ ಹೆಸರು ಹಿಂದೂಗಳಲ್ಲಿ ಚಿರಪರಿಚಿತವಾಗಿದೆ. ಯಾವ ಕಾಲದಲ್ಲಾಗಲೀ ದೇಶದಲ್ಲಾಗಲೀ, ನಾರದರು ಇರುವರು. ಅವರೊಬ್ಬ ಚಿರಂಜೀವಿಗಳು ಎಂದು ಪುರಾಣ ಸಾರುವುದು.

ಗಂಧರ್ವರು ಎಂಬ ಜನಾಂಗ ಶ‍್ರೀಕೃಷ್ಣ ಇದ್ದ ಕಾಲದಲ್ಲಿ ಇದ್ದಿರಬಹುದು. ಅವರು ಹಾಡುವು ದರಲ್ಲಿ ಪ್ರಖ್ಯಾತಿಯನ್ನು ಪಡೆದವರು. ಚಿತ್ರರಥ ಎಂಬುವನು ಅವರಲ್ಲಿ ಪ್ರಖ್ಯಾತನಾದ ರಾಜ. ಬಹುಶಃ ಆತ ಶ‍್ರೀಕೃಷ್ಣನ ಸಮಕಾಲೀನ ಇರಬಹುದು. ಯೋಗಾಭ್ಯಾಸ ಮಾಡಿ ಸಿದ್ಧಿಯನ್ನು ಪಡೆದವರಲ್ಲಿ ಕಪಿಲ ನಾನು ಎಂದು ಶ‍್ರೀಕೃಷ್ಣ ಹೇಳುತ್ತಾನೆ. ಇವರೇ ಸಾಂಖ್ಯ ಸಿದ್ಧಾಂತದ ಮೂಲ ಪುರುಷರು. ಕಾಲಕ್ರಮೇಣ ವೇದಾಂತ ಸಾಂಖ್ಯ ಸಿದ್ಧಾಂತದ ಮುಖ್ಯವಾದ ತತ್ತ್ವಭಾಗವನ್ನೆಲ್ಲ ಹೀರಿಕೊಂಡು ತನ್ನಲ್ಲಿ ಅದನ್ನು ಅಳವಡಿಸಿಕೊಂಡಿತು. ಶ‍್ರೀಕೃಷ್ಣನ ಕಾಲಕ್ಕೆ ಕಪಿಲರು ದೊಡ್ಡ ಜ್ಞಾನಿಗಳು ಮತ್ತು ಸಿದ್ಧಪುರುಷರು ಎಂದು ಖ್ಯಾತಿವಂತರಾಗಿದ್ದರು.

\begin{shloka}
ಉಚ್ಚೈಃಶ್ರವಸಮಶ್ವಾನಾಂ ವಿದ್ಧಿ ಮಾಮಮೃತೋದ್ಭವಮ್~।\\ಐರಾವತಂ ಗಜೇಂದ್ರಾಣಾಂ ನರಾಣಾಂ ಚ ನರಾಧಿಪಮ್ \hfill॥೨೭॥
\end{shloka}

\begin{artha}
ಕುದರೆಗಳಲ್ಲಿ ಅಮೃತದಿಂದ ಹುಟ್ಟಿದ ಉಚ್ಚೈಶ್ರವ ನಾನೆಂದು ತಿಳಿ. ಆನೆಗಳಲ್ಲಿ ಐರಾವತ, ನರರಲ್ಲಿ ನರಪತಿ ನಾನು.
\end{artha}

ಉಚ್ಚೈಶ್ರವಸ್ಸು ಎಂಬ ಕುದುರೆ, ಐರಾವತ ಎಂಬ ಆನೆ ಎರಡೂ ಕೂಡ ಕ್ಷೀರ ಸಮುದ್ರವನ್ನು ದೇವದಾನವರು ಅಮೃತಕ್ಕಾಗಿ ಕಡೆದಾಗ ಹುಟ್ಟಿದ್ದು ಎಂದು ಪುರಾಣದಲ್ಲಿ ಬರುವುದು. ಕುದುರೆಗಳಲ್ಲಿ ಇದು ಸರ್ವಶ್ರೇಷ್ಠವಾದುದೆಂದು, ಆನೆಗಳಲ್ಲಿ ಐರಾವತ ಸರ್ವಶ್ರೇಷ್ಠವಾದುದೆಂದು ಪ್ರತೀತಿ.

ಮನುಷ್ಯರಲ್ಲಿ ಅವನು ರಾಜ. ಹಿಂದಿನ ಕಾಲದಲ್ಲಿ ರಾಜನೆ ಅತ್ಯಂತ ಮುಖ್ಯ. ಅವನ ಮಾತು ಶಾಸನ; ಯಾರೂ ಅವನನ್ನು ಮೀರುವವರಿಲ್ಲ ಅಧಿಕಾರದಲ್ಲಿ. ಅವನು ಎಲ್ಲರನ್ನೂ ಆಳುವವನು, ಎಲ್ಲರನ್ನೂ ರಕ್ಷಿಸುವವನು. ಭಗವಂತನ ಶಕ್ತಿ ಯಾವುದಾದರೂ ಒಂದು ವ್ಯಕ್ತಿಯನ್ನು ತೆಗೆದುಕೊಂಡು ಅದರ ಮೂಲಕ ಕೆಲಸ ಮಾಡಬೇಕಾಗಿದೆ. ರಾಜನನ್ನು ದೇವರು ಆರಿಸಿಕೊಳ್ಳುವನು. ಏಕೆಂದರೆ ಅವನ ಮೂಲಕ ಹೆಚ್ಚು ಕೆಲಸ ಮಾಡಬಹುದು. ಆಗಿನ ಕಾಲದ ರಾಜರ ಸ್ಥಾನವನ್ನು ಈಗಿನ ಕಾಲದಲ್ಲಿರುವ ರಾಜರ ಸ್ಥಾನದೊಂದಿಗೆ ಹೋಲಿಸಲಾಗುವುದಿಲ್ಲ. ಈಗಿರುವ ಅನೇಕ ರಾಜರು ಕೇವಲ ಹೆಸರಿಗೆ ಮಾತ್ರ ರಾಜರು. ಅವರಿಗೆ ಅಧಿಕಾರವಾಗಲಿ ಸಾಮರ್ಥ್ಯವಾಗಲಿ ಇಲ್ಲ. ರಾಜನ ಶಕ್ತಿ ಮತ್ತು ಪ್ರಭಾವವನ್ನು ನೋಡಬೇಕಾದರೆ ನಾವು ಆಗಿನ ಕಾಲದ ರಾಜರನ್ನು ತೆಗೆದುಕೊಳ್ಳಬೇಕು.

\begin{shloka}
ಆಯುಧಾನಾಮಹಂ ವಜ್ರಂ ಧೇನೂನಾಮಸ್ಮಿ ಕಾಮಧುಕ್~।\\ಪ್ರಜನಶ್ಚಾಸ್ಮಿ ಕಂದರ್ಪಃ ಸರ್ಪಾಣಾಮಸ್ಮಿ ವಾಸುಕಿಃ \hfill॥ ೨೮~॥
\end{shloka}

\begin{artha}
ನಾನು ಆಯುಧಗಳಲ್ಲಿ ವಜ್ರಾಯುಧ, ಗೋವುಗಳಲ್ಲಿ ಕಾಮಧೇನು, ಪ್ರಜೋತ್ಪತ್ತಿಗೆ ಕಾರಣನಾದ ಕಾಮದೇವ, ಸರ್ಪಗಳಲ್ಲಿ ವಾಸುಕಿ.
\end{artha}

ದೇವತೆಗಳ ಒಡೆಯನಾದ ಇಂದ್ರ ವೃತ್ರಾಸುರನನ್ನು ಕೊಂದದ್ದು ವಜ್ರಾಯುಧದಿಂದ. ಈ ಆಯುಧ ಆದದ್ದು ದಧೀಚಿ ತನ್ನ ದೇಹದ ಮೂಳೆಯನ್ನೆ ಕೊಟ್ಟದ್ದರಿಂದ. ಏಕೆಂದರೆ ಆ ದಾನವ ತನಗೆ ಮತ್ತೆ ಯಾವ ಆಯುಧಗಳಿಂದಲೂ ಮರಣ ಬಾರದಿರಲಿ ಎಂದು ಶಿವನನ್ನು ಬೇಡಿಕೊಂಡಿದ್ದ.

ಗೋವುಗಳಲ್ಲಿ ಕಾಮಧೇನು ಎನ್ನುವನು. ಕಾಮಧೇನು ಬರೀ ಹಾಲನ್ನು ಮಾತ್ರ ಬೇಕಾದಷ್ಟು ಕೊಡುತ್ತಿದ್ದ ಹಸುವಲ್ಲ. ನಮಗೆ ಬೇಕಾದುದನ್ನೆಲ್ಲಾ ಅದಕ್ಕೆ ಕೊಡುವುದಕ್ಕೆ ಶಕ್ತಿ ಇತ್ತು ಎಂದು ಪುರಾಣಗಳು ಹೇಳುವುವು. ಕಲ್ಪವೃಕ್ಷ ಕಾಮಧೇನು ಎರಡೂ ಒಟ್ಟಿಗೆ ಹೋಗುವುವು, ಮನುಷ್ಯನ ಬಯಕೆಗಳನ್ನು ಪೂರ್ಣಗೊಳಿಸುವುದಕ್ಕೆ.

ಪ್ರಜೋತ್ಪತ್ತಿಗೆ ಕಾರಣನಾದ ಕಂದರ್ಪ ಎನ್ನುವನು. ಇಲ್ಲಿ ಬರೀ ತಮ್ಮ ಕಾಮವನ್ನು ತೃಪ್ತಿಪಡಿಸಿಕೊಳ್ಳುವುದಕ್ಕಾಗಿ ಮಾಡಿದ ಕ್ರಿಯೆಯಲ್ಲ. ಮುಂದಿನ ತಮ್ಮ ಪೀಳಿಗೆಯವರು ಬೇಕೆಂಬ ಆಸೆಯಿಂದ ಪ್ರೇರಿತವಾಗಿ ಮಾಡುವ ಕಾರ್ಯ. ಇದು ಇರುವುದರಿಂದಲೇ ಸೃಷ್ಟಿ ಮುಂದುವರಿ\-ಯುತ್ತಿದೆ. ಪ್ರತಿಯೊಂದೂ ಸಾಯುವುದಕ್ಕೆ ಮುಂಚೆ ತನ್ನ ಪ್ರತಿನಿಧಿಯನ್ನು ಬಿಟ್ಟು ಸಾಯುತ್ತದೆ.

ಸರ್ಪಗಳಲ್ಲಿ ವಾಸುಕಿ ಎನ್ನುವನು. ಮುಂದಿನ ಶ್ಲೋಕಗಳಲ್ಲಿ ನಾಗಗಳಲ್ಲಿ ಅನಂತ ಎನ್ನುವನು. ಆದಕಾರಣ ಇವೆರಡರಲ್ಲಿಯೂ ಒಂದು ವ್ಯತ್ಯಾಸವನ್ನು ಮಾಡಬೇಕಾಗಿದೆ. ಇಲ್ಲಿ ಸರ್ಪಗಳು ಎಂದರೆ ವಿಷವಿರುವ ಸರ್ಪಗಳು ಎಂದು ತೆಗೆದುಕೊಳ್ಳಬಹುದು. ಹಾಗೆ ವಿಷಪೂರಿತ ಸರ್ಪಗಳಲ್ಲಿ ವಾಸುಕಿ. ವಾಸುಕಿಯಲ್ಲಿರುವಷ್ಟು ಭಯಂಕರವಾದ ವಿಷ ಮತ್ತೊಂದು ಸರ್ಪದಲ್ಲಿ ಇಲ್ಲ. ಹೇಗೆ ಸೃಷ್ಟಿಗೆ ಸಹಾಯಕನಾದ ಕಂದರ್ಪನ ಹಿಂದೆ ಅವನು ಇರುವನೊ, ಹಾಗೆಯೆ ಕಚ್ಚಿದ ತಕ್ಷಣ ಸಾಯಿಸುವ ವಿಷದ ಹಿಂದೆಯೂ ಅವನೇ ಇದ್ದಾನೆ.

\begin{shloka}
ಅನಂತಶ್ಚಾಸ್ಮಿ ನಾಗಾನಾಂ ವರುಣೋ ಯಾದಸಾಮಹಮ್~।\\ಪಿತೄಣಾಮರ್ಯಮಾ ಚಾಸ್ಮಿ ಯಮಃ ಸಂಯಮತಾಮಹಮ್ \hfill॥ ೨೯~॥
\end{shloka}

\begin{artha}
ನಾನು ನಾಗಗಳಲ್ಲಿ ಅನಂತ, ಜಲಚರಗಳಲ್ಲಿ ವರುಣ, ಪಿತೃಗಳಲ್ಲಿ ಅರ್ಯಮ, ಶಾಸಿಸುವವರಲ್ಲಿ ಯಮ.
\end{artha}

ನಾಗ ಎಂದರೆ ವಿಷವಿಲ್ಲದ ಸರ್ಪಗಳು ಎಂದು ಬೇಕಾದರೆ ನಾವು ತೆಗೆದುಕೊಳ್ಳಬಹುದು. ಇದರ ಮೇಲೆಯೆ ವಿಷ್ಣು ಮಲಗುವುದು ಎಂದು ಪುರಾಣ ಸಾರುವುದು. ಜಲದಲ್ಲಿ ವಾಸಮಾಡುವ ಪ್ರಾಣಿಗಳಲ್ಲಿ ವರುಣ ನಾನು ಎನ್ನುವನು. ವರುಣ ಸಮುದ್ರ ರಾಜ. ಅವನು ಸಮುದ್ರದಲ್ಲಿ ಯಾವಾಗಲೂ ವಾಸಮಾಡುತ್ತಿರುವನು ಎಂಬುದು ಪೌರಾಣಿಕ ಭಾವನೆ. ಪಿತೃಗಳು ಎಂದರೆ ಗತಿಸಿ ಹೋಗಿ ಈಗ ಪಿತೃಲೋಕದಲ್ಲಿ ಇರುವವರಲ್ಲಿ ಅರ್ಯಮ ಬಹಳ ಹಿಂದೆ ಹೋದವನು, ಮತ್ತು ಪ್ರಖ್ಯಾತನಾದವನು. ಇನ್ನು ಶಾಸನ ಮಾಡುವವರಲ್ಲಿ ತಾನು ಯಮ ಎಂದು ಹೇಳುತ್ತಾನೆ. ಯಮ ನಾವು ಕಾಲವಾದ ಮೇಲೆ ತನ್ನ ಲೋಕದಲ್ಲಿ ನಮ್ಮ ನ್ಯಾಯ ಅನ್ಯಾಯಗಳನ್ನು ವಿಚಾರಿಸುತ್ತಾನೆ. ಅವನು ಜೀವಿಯ ಕ್ರಿಯೆಗಳನ್ನೆಲ್ಲ ನೋಡಿ ಅದಕ್ಕೆ ತಕ್ಕ ಶಿಕ್ಷೆಯನ್ನು ಕೊಡುವನು. ಅವನ ಕಣ್ಣಿಗೆ ಯಾರೂ ಮಣ್ಣೆರೆಚಲು ಆಗುವುದಿಲ್ಲ. ಮರ್ತ್ಯಲೋಕದ ಲಾಯರುಗಳನ್ನು ಕರೆದುಕೊಂಡು ಹೋಗಿ ಅಲ್ಲಿ ಅವನನ್ನು ನಮ್ಮ ಪರವಾಗಿ ವಕಾಲತ್ತಿಗೆ ವಹಿಸುವುದಕ್ಕೆ ಆಗುವುದಿಲ್ಲ. ಯಮನ ಪ್ರತಿನಿಧಿಗಳೇ ಆದ ಚಿತ್ರಗುಪ್ತರು ನಮ್ಮ ಹೃದಯದಲ್ಲಿ ಹಗಲು ರಾತ್ರಿ ನಾವು ಮಾಡಿದ್ದನ್ನೆಲ್ಲ ಬರೆದುಕೊಳ್ಳುವರು ಎಂಬ ನಂಬಿಕೆ ಇದೆ. ಯಮಲೋಕದಲ್ಲಿ ನಮ್ಮನ್ನು ವಿಚಾರಣೆಗೆ ಒಳಪಡಿಸಿದಾಗ ಚಿತ್ರಗುಪ್ತರ ದಪ್ತರನ್ನು ನೋಡಿ ಯಮ, ಶಿಕ್ಷೆಯನ್ನೋ ಬಹುಮಾನವನ್ನೋ ಕೊಡುತ್ತಾನೆ. ಇಲ್ಲಿ ಯಮ ಬರೀ ಸಾವಿನ ದೇವತೆ ಅಲ್ಲ. ಸತ್ತಾದಮೇಲೆ ನಮ್ಮಕರ್ಮಗಳಿಗೆ ತಕ್ಕಂತೆ ಬೇರೆ ಬೇರೆ ಜನ್ಮಗಳನ್ನು ಕೊಡುವ ನ್ಯಾಯಾಧಿಪತಿ. ಆದಕಾರಣವೇ ಯಮನನ್ನು ಯಮಧರ್ಮರಾಯ ಎಂದು ಕರೆಯುತ್ತಾರೆ.

\begin{shloka}
ಪ್ರಹ್ಲಾದಶ್ಚಾಸ್ಮಿ ದೈತ್ಯಾನಾಂ ಕಾಲಃ ಕಲಯತಾಮಹಮ್~।\\ಮೃಗಾಣಾಂ ಚ ಮೃಗೇಂದ್ರೋಽಹಂ ವೈನತೇಯಶ್ಚ ಪಕ್ಷಿಣಾಮ್ \hfill॥ ೩೦~॥
\end{shloka}

\begin{artha}
ನಾನು ದೈತ್ಯರಲ್ಲಿ ಪ್ರಹ್ಲಾದ, ಎಣಿಸುವವರಲ್ಲಿ ಕಾಲ, ಪಶುಗಳಲ್ಲಿ ಸಿಂಹ, ಪಕ್ಷಿಗಳಲ್ಲಿ ಗರುಡ.
\end{artha}

ದೈತ್ಯರಲ್ಲಿ ನಾನು ಪ್ರಹ್ಲಾದ ಎನ್ನುತ್ತಾನೆ. ಹಿರಣ್ಯಕಷ್ಯಪನೆಂಬ ರಾಕ್ಷಸನಿಗೆ ಹುಟ್ಟಿದ ಮಗು ಪ್ರಹ್ಲಾದ. ಬಾಲ್ಯದಿಂದಲೂ ಪರಮ ದೈವಭಕ್ತ. ಅವನನ್ನು ರಕ್ಷಿಸುವುದಕ್ಕಾಗಿಯೆ ಪರಮಾತ್ಮ ನರಸಿಂಹನಂತೆ ಅವತಾರವೆತ್ತಿದ. ಆಗ ಪ್ರಹ್ಲಾದನ ಭಕ್ತಿಯನ್ನು ನೋಡಿ, ನಿನಗೆ ಬೇಕಾದುದನ್ನು ಕೇಳು ಎನ್ನುತ್ತಾನೆ. ಪ್ರಹ್ಲಾದ ಪ್ರಾಪಂಚಿಕರು ಎಂತಹ ಒಂದು ನಿಕಟತೆಯಿಂದ ವಿಷಯವಸ್ತುಗಳನ್ನು ಚಿಂತಿಸುತ್ತಾರೆಯೊ ಅಂತಹ ತೀವ್ರತೆಯಿಂದ ನಾನು ನಿನ್ನ ಪಾದಪದ್ಮಗಳನ್ನು ಚಿಂತಿಸುವಂತೆ ಆಶೀರ್ವದಿಸು ಎಂದು ಬೇಡಿಕೊಳ್ಳುತ್ತಾನೆ. ಪ್ರಹ್ಲಾದ ಹುಟ್ಟಿದ್ದು ರಾಕ್ಷಸಕುಲದಲ್ಲಿ. ಅತ್ಯಾಚಾರ ಅನಾಚಾರಗಳಿಗೆ ಪ್ರಖ್ಯಾತವಾದ ಕುಲ ಅದು. ಆದರೆ ಅಲ್ಲಿಯೇ ಕೆಸರಿನಲ್ಲಿ ಹುಟ್ಟಿದ ಕಮಲದಂತೆ ಪ್ರಹ್ಲಾದ ಜನಿಸುವನು. ಅಂತಹ ದುಷ್ಟ ಕುಲದಲ್ಲಿ ಇಂತಹ ಸುಪುತ್ರ. ಇಂತಹ ಸತ್​ಪುತ್ರನ ಹಿಂದೆ ಇರುವವನೇ ಭಗವಂತ.

\newpage

ಎಣಿಸುವವರಲ್ಲಿ ಕಾಲ ನಾನು ಎನ್ನುತ್ತಾನೆ. ಈ ಸೃಷ್ಟಿಸ್ಥಿತಿ ಪ್ರಳಯಗಳೆಲ್ಲ ಆಗುವುದು ಆ ಮಹಾಕಾಲದಲ್ಲಿ. ಈ ಬ್ರಹ್ಮಾಂಡ, ಕಾಲದ ಪ್ರವಾಹದಲ್ಲಿ ನೀರಿನ ಗುಳ್ಳೆಗಳಂತೆ ಏಳುವುದು, ಒಡೆದುಹೋಗುವುದು. ಆದಿಯಿಂದ ಇದುವರೆಗೆ ಆದುದೆಲ್ಲ ಕಾಲದಲ್ಲಿದೆ. ಮುಂದೆ ಆಗುವುದೆಲ್ಲ ಅದರಲ್ಲಿರುವುದು.

ಸಿಂಹ ಪ್ರಾಣಿಗಳಲ್ಲೆಲ್ಲ ಬಹಳ ಬಲಿಷ್ಠವಾದುದು. ಅದನ್ನು ಮೃಗರಾಜ ಎನ್ನುತ್ತಾರೆ. ಆ ಪರಾಕ್ರಮದ ಹಿಂದೆ ಇರುವುದು ಭಗವಂತನೇ. ಪಕ್ಷಿಗಳು ಗಗನದಲ್ಲಿ ಹಾರುವುವು. ಅವುಗಳಲ್ಲೆಲ್ಲಾ ಗರುಡ ನಾನು ಎನ್ನುತ್ತಾನೆ. ಇದರ ಬೆನ್ನಿನಮೇಲೆಯೇ ಶ‍್ರೀಕೃಷ್ಣ ಕುಳಿತುಕೊಳ್ಳುವುದು. ಇದೇ ಪಕ್ಷಿರಾಜ.

\begin{shloka}
ಪವನಃ ಪವತಾಮಸ್ಮಿ ರಾಮಃ ಶಸ್ತ್ರಭೃತಾಮಹಮ್~।\\ಝಷಾಣಾಂ ಮಕರಶ್ಚಾಸ್ಮಿ ಸ್ರೋತಸಾಮಸ್ಮಿ ಜಾಹ್ನವೀ \hfill॥ ೩೧~॥
\end{shloka}

\begin{artha}
ನಾನು ಪಾವನ ಮಾಡುವವರಲ್ಲಿ ಗಾಳಿ, ಶಸ್ತ್ರಧಾರಿಗಳಲ್ಲಿ ಶ‍್ರೀರಾಮ, ಮೀನುಗಳಲ್ಲಿ ಮೊಸಳೆ, ನದಿಗಳಲ್ಲಿ ಗಂಗೆ.
\end{artha}

ಎಂತಹ ವಿಷಗಾಳಿಯಾದರೂ ಹೊರಗೆ ಬೀಸುವ ಗಾಳಿಯೊಂದಿಗೆ ಸೇರಿಕೊಂಡರೆ ಸ್ವಲ್ಪಹೊತ್ತಿನಲ್ಲಿ ಅದು ತನ್ನ ಗುಣವನ್ನು ಕಳೆದುಕೊಂಡು ಪರಿಶುದ್ಧವಾದ ಗಾಳಿಯಾಗುವುದು. ದೊಡ್ಡ ದೊಡ್ಡ ಕಾರ್ಖಾನೆಗಳು ಇರುವ ಕಡೆ ಅದರ ಕೊಳವೆಗಳಿಂದ ಎಷ್ಟೊಂದು ದುರ್ಗಂಧ ಹೊರಗಿನ ಗಾಳಿ ಯೊಂದಿಗೆ ಬೆರೆಯುತ್ತಿರುವುದು. ಆದರೂ ಸ್ವಲ್ಪದೂರ ಹರಿದು ಹೋಗುವುದರೊಳಗೆ ಪರಿಶುದ್ಧವಾಗಿ ಹೋಗುವುದು.

ಶ‍್ರೀರಾಮ ಮತ್ತು ಅವನ ಬಾಣ ಅಮೋಘವಾದುದೆಂದು ಹಿಂದಿನಿಂದ ಬಂದಿತ್ತು. ನೀರಿನಲ್ಲಿ ಹಲವು ಬಗೆಯ ಪ್ರಾಣಿಗಳಿವೆ. ಅವುಗಳಲ್ಲೆಲ್ಲ ಮೊಸಳೆ ಬಹಳ ಬಲವಾದುದು. ಹಿಂದೂಗಳಿಗೆ ಗಂಗಾನದಿಯಷ್ಟು ಪವಿತ್ರವಾದುದು ಮತ್ತೊಂದು ಇಲ್ಲ. ಶಿವನ ಜಟಾಜೂಟದಿಂದ ಉರುಳಿಬಂದು ಈ ಭೂಮಿಯಮೇಲೆ ಪಾಪಿಗಳನ್ನು ಉದ್ಧಾರಮಾಡುವುದಕ್ಕಾಗಿ ಹರಿಯುತ್ತಿರು\-ವಳು. ಈ ನದಿಯ ತೀರದಲ್ಲೆ ನಮ್ಮ ನಾಗರಿಕತೆ ಅರಳಿತು. ಇದರ ತೀರದಲ್ಲಿ ಮಹಾಮುನಿಗಳು ತಪಸ್ಸು ಮಾಡಿದರು. ಸಾವಿರಾರು ಮೈಲಿಗಳಿಂದ ಬಂದು, ಈ ನದಿಯಲ್ಲಿ ಮಿಂದು, ಸ್ವಲ್ಪ ಗಂಗಾವಾರಿಯನ್ನು ತೆಗೆದುಕೊಂಡುಹೋಗಿ ಮನೆಯಲ್ಲಿಟ್ಟು ಪೂಜಿಸುವರು. ಅಂತ್ಯಕಾಲದಲ್ಲಿ ಉದ್ಧರಣೆ ಗಂಗಾ ತೀರ್ಥವನ್ನು ಕುಡಿದು ಪ್ರಾಣಬಿಡಬೇಕೆಂದು ಹಿಂದು ಆಶಿಸುವನು.

\begin{shloka}
ಸರ್ಗಾಣಾಮಾದಿರಂತಶ್ಚ ಮಧ್ಯಂ ಚೈವಾಹಮರ್ಜುನ~।\\ಅಧ್ಯಾತ್ಮವಿದ್ಯಾ ವಿದ್ಯಾನಾಂ ವಾದಃ ಪ್ರವದತಾಮಹಮ್ \hfill॥ ೩೨~॥
\end{shloka}

\begin{artha}
ಸೃಷ್ಟಿಯ ಆರಂಭ ಮಧ್ಯ ಮತ್ತು ಅಂತ್ಯ ನಾನು. ವಿದ್ಯೆಗಳಲ್ಲಿ ಅಧ್ಯಾತ್ಮವಿದ್ಯೆ ನಾನು. ಚರ್ಚೆ ಮಾಡುವವರಲ್ಲಿ ವಾದ ನಾನು.
\end{artha}

ಈ ಬ್ರಹ್ಮಾಂಡ ಬಂದಿರುವುದು ಅವನಿಂದ. ಇದನ್ನು ಪಾಲಿಸುತ್ತಿರುವವನು ಅವನು. ಕೊನೆಗೆ ಇದನ್ನೆಲ್ಲಾ ಸಂಹಾರ ಮಾಡುವವನೂ ಅವನೆ. ಈ ಮೂರು ಕೆಲಸವನ್ನು ಮಾಡುವುದಕ್ಕೆ ಪುರಾಣದಲ್ಲಿ ಬ್ರಹ್ಮ ವಿಷ್ಣು ರುದ್ರರನ್ನು ನಿಯಮಿಸುತ್ತಾರೆ. ಈ ಮೂರು ದೇವತೆಗಳ ಮೂಲಕ ಕೆಲಸ ಮಾಡುತ್ತಿರುವವನು ನಿಜವಾಗಿ ದೇವರೆ. ಈ ಮೂರರ ಹಿಂದೆಯೂ ಅವನು ಒಂದೇ ಸಮನಾಗಿರುವನು. ಒಂದು ಕಡೆ ಹೆಚ್ಚು ಮತ್ತೊಂದು ಕಡೆ ಕಡಿಮೆಯಲ್ಲ. ಯಾವ ಕೈಗಳು ಪಾಲಿಸುವುವೋ ಅದೇ ಕೈಗಳೇ ಜೀವಹರಣ ಮಾಡುವುವು.

ವಿದ್ಯೆಗಳಲ್ಲಿ ಅಧ್ಯಾತ್ಮವಿದ್ಯೆ ನಾನು ಎನ್ನುವನು. ಇತರ ವಿದ್ಯೆಗಳಿಂದ ದೃಶ್ಯ ಪ್ರಪಂಚದ ಸ್ಥಿತಿಯನ್ನು ಅರಿತುಕೊಳ್ಳುತ್ತೇವೆ. ಆದರೆ ಪರಮಾತ್ಮನನ್ನು ಇವುಗಳ ಮೂಲಕ ಅರಿಯುವುದಕ್ಕೆ ಆಗುವುದಿಲ್ಲ. ಹೊರಗಿನ ವಸ್ತುಗಳು ಹೇಗೆ ಇವೆ ಎಂಬುದನ್ನು ನಾವು ಅವುಗಳ ಮೂಲಕ ತಿಳಿದುಕೊಳ್ಳಬಹುದು. ಆದರೆ ಏತಕ್ಕೆ ಎಂಬುದಕ್ಕೆ ಅವು ಉತ್ತರವನ್ನು ಕೊಡಲಾರವು. ನಮಗೆ ಆ ಉತ್ತರ ಬೇಕಾದರೆ ಅಧ್ಯಾತ್ಮವಿದ್ಯೆಯ ಕಡೆ ತಿರುಗಬೇಕು. ಶರೀರಶಾಸ್ತ್ರ ನಮ್ಮ ದೇಹವನ್ನು ಕುರಿತು ಹೇಳುತ್ತದೆ. ಮಾನಸಿಕ ಶಾಸ್ತ್ರವಾದರೋ ಮನಸ್ಸಿನ ವಿಷಯವನ್ನು ಸ್ವಲ್ಪ ವಿವರಿಸುವುದು. ಆದರೆ ಅದರ ಹಿಂದೆ ಇರುವ ಆತ್ಮನ ವಿಷಯವನ್ನು ತಿಳಿದುಕೊಳ್ಳಬೇಕಾದರೆ, ಈ ಬ್ರಹ್ಮಾಂಡದ ಹಿಂದೆ ಮತ್ತು ನನ್ನ ಹಿಂದೆ ಇರುವ ಪರಮಾತ್ಮವಸ್ತುವನ್ನು ತಿಳಿದುಕೊಳ್ಳಬೇಕಾದರೆ ಇತರ ವಿದ್ಯೆಗಳಾವುವೂ ಸಹಾಯಕ್ಕೆ ಬಾರವು. ಅಧ್ಯಾತ್ಮವಿದ್ಯೆಯೊಂದೇ ಅದನ್ನು ತಿಳಿಯಬಲ್ಲದು. ಜೀವಿಯಲ್ಲಿರುವ ಭಗವತ್ ದಾಹವನ್ನು ಪೂರ್ಣಗೊಳಿಸುವುದು ಈ ಒಂದು ವಿದ್ಯೆಯೇ.

ಚರ್ಚೆಮಾಡುವವರಲ್ಲಿ ವಾದ ನಾನು ಎಂದು ಈಗ ನಾವು ಮಾತನಾಡುವ ಭಾಷೆಯೊಳಗೂ ಪ್ರವೇಶಿಸಿ ಹೇಳುತ್ತಾನೆ. ಚರ್ಚೆ ಮಾಡುವುದರಲ್ಲಿ ಮೂರು ವಿಧಗಳಿವೆ. ಒಂದು ಜಲ್ಪ, ಮತ್ತೊಂದು ವಿತಂಡ, ಮೂರನೆಯದೆ ವಾದ. ಜಲ್ಪವೆಂದರೆ ಚರ್ಚಿಸುವವನು ಸತ್ಯವನ್ನು ತಿಳಿದುಕೊಳ್ಳಬೇಕೆಂದು ಇಚ್ಛಿಸುವುದಿಲ್ಲ. ತಾನು ಚರ್ಚೆಯಲ್ಲಿ ಗೆಲ್ಲಬೇಕೆಂಬುದೇ ಅವನ ಮುಖ್ಯ ಉದ್ದೇಶ. ಅದಕ್ಕಾಗಿ ಅವನು ಯಾವ ಲಂಗು ಲಗಾಮೂ ಇಲ್ಲದೆ ಚರ್ಚೆಮಾಡುವನು, ಆಟಗಾರ ಯಾವ ನಿಯಮವನ್ನೂ ಪಾಲಿಸದೆ ಮನಬಂದಂತೆ ಆಡುವಂತೆ. ಎರಡನೆಯದೆ ವಿತಂಡ. ಇದರಲ್ಲಿ ಸೋಲಿಸುವುದೊಂದೇ ಅವನ ಗುರಿ. ಅವನು ಸತ್ಯವನ್ನೇ ಹೇಳುತ್ತಿದ್ದರೂ ಇವನು ಸುಳ್ಳು ಎನ್ನುವನು. ಅವನು ಏನು ಹೇಳಿದರೂ ಅದನ್ನು ವಿರೋಧಿಸುವುದೇ ಉದ್ದೇಶ. ನಿನ್ನ ನಿಲುವು ಏನೆಂದು ಕೇಳಿದರೆ, ನಿನ್ನನ್ನು ಸೋಲಿಸುವುದೇ ನನ್ನ ನಿಲುವು ಎನ್ನುತ್ತಾನೆ. ಸತ್ಯವನ್ನು ಅರಿಯುವುದಲ್ಲ ಇವರ ಗುರಿ. ಕುಸ್ತಿಯಲ್ಲಿ ಇನ್ನೊಬ್ಬನನ್ನು ಸೋಲಿಸುವಂತೆ, ಸೋಲಿಸುವುದೇ ಅವನ ಉದ್ದೇಶ. ವಾದ ಎಂದರೆ ಚರ್ಚೆ. ಆದರೆ ಆ ಚರ್ಚೆಗೆ ಒಂದು ಉದ್ದೇಶವಿದೆ. ಅದೇ ಸತ್ಯವನ್ನು ತಿಳಿದುಕೊಳ್ಳಬೇಕೆಂಬುದು. ಕೆಂಡವನ್ನೆಲ್ಲಾ ಬೂದಿ ಆವರಿಸಿದೆ. ಆ ಬೂದಿಯನ್ನು ಕೆದಕದೆ ಇದ್ದರೆ ಅದರ ಹಿಂದೆ ಇರುವ ಕೆಂಡ ಕಾಣುವುದಿಲ್ಲ. ಅದಕ್ಕಾಗಿ ಅವನು ಕೆದಕುತ್ತಾನೆ, ಕೆಣಕುತ್ತಾನೆ. ಸುಮ್ಮನೆ ಕೆದಕುವುದು, ಕೆಣಕುವುದೇ ಅವನ ಗುರಿಯಲ್ಲ. ಇಲ್ಲಿ ತಾನು ಗೆಲ್ಲಬೇಕು ಎಂಬ\break ಉದ್ದೇಶವಿಲ್ಲ. ತಾನು ಸತ್ಯವನ್ನು ತಿಳಿದುಕೊಳ್ಳ ಬೇಕೆಂಬುದೇ ಅವನ ಉದ್ದೇಶ. ಸೋಲುಗೆಲವುಗಳು ಗೌಣ. ಜ್ಞಾನಾರ್ಜನೆಯೇ ಗುರಿ. ಇಂತಹ ವಾದದ ಹಿಂದೆ ತಾನೆ ಇರುವೆ ಎನ್ನುವನು ಶ‍್ರೀಕೃಷ್ಣ.

\begin{shloka}
ಅಕ್ಷರಾಣಾಮಕಾರೋಽಸ್ಮಿ ದ್ವಂದ್ವಃ ಸಾಮಾಸಿಕಸ್ಯ ಚ~।\\ಅಹಮೇವಾಕ್ಷಯಃ ಕಾಲೋ ಧಾತಾಽಹಂ ವಿಶ್ವತೋಮುಖಃ \hfill॥ ೩೩~॥ 
\end{shloka}

\begin{artha}
ನಾನು ಅಕ್ಷರಗಳಲ್ಲಿ ಅಕಾರ, ಸಮಾಸಗಳಲ್ಲಿ ದ್ವಂದ್ವಸಮಾಸ. ನಾನೇ ಅಕ್ಷಯವಾದ ಕಾಲ, ಸರ್ವತೋಮುಖನಾದ ಧಾತೃ.
\end{artha}

ಅಕ್ಷರಗಳಲ್ಲೆಲ್ಲ ಅಕಾರದಷ್ಟು ಹೆಚ್ಚಾಗಿ ಉಪಯೋಗಿಸುವ ಅಕ್ಷರ ಮತ್ತೊಂದಿಲ್ಲ. ಹಲವು ಪದಗಳು ಇದರ ಆಧಾರದ ಮೇಲೆ ಆಗುವುವು. ಮುದ್ರಣಾಲಯಗಳಲ್ಲಿ ‘ಅ’ ಎಂಬ ಅಕ್ಷರಕ್ಕೆ ಮಾಡಿರುವಷ್ಟು ದೊಡ್ಡ ಸ್ಥಾನ ಮತ್ತಾವುದಕ್ಕೂ ಇಲ್ಲ.

ಸಂಸ್ಕೃತಭಾಷೆ ಹಲವು ಸಂಯುಕ್ತ ಪದಗಳಿಂದ ಕೂಡಿದೆ. ಅವುಗಳಲ್ಲಿ ಮುಖ್ಯವಾಗಿ ನಾಲ್ಕು ಸಮಾಸಗಳಿವೆ. ಅವ್ಯಯ ಸಮಾಸಗಳಲ್ಲಿ ಮೊದಲನೆಯ ಪದ ಮುಖ್ಯ ಎರಡನೆಯದು ಗೌಣ. ತತ್ಪರುಷ ಸಮಾಸದಲ್ಲಿ ಎರಡನೆಯ ಪದ ಮುಖ್ಯ. ಬಹುವ್ರೀಹಿಯಲ್ಲಿ ಎರಡು ಪದಗಳೂ ಸೇರಿ ಮತ್ತಾವುದೊ ಒಂದನ್ನು ಸೂಚಿಸುವುದು. ಆದರೆ ದ್ವಂದ್ವ ಸಮಾಸದಲ್ಲಿ ಮೊದಲನೆಯದರಷ್ಟೆ ಎರಡನೆಯದೂ ಮುಖ್ಯ.

ನಾನು ಅಕ್ಷಯವಾದ ಕಾಲ ಎನ್ನುತ್ತಾನೆ. ಎಂದರೆ ಅನಂತಕಾಲ. ಈ ಅನಂತಕಾಲಕ್ಕೆ ಆದಿ ಇಲ್ಲ, ಅಂತ್ಯವಿಲ್ಲ. ಆದಿಅಂತ್ಯವಿಲ್ಲದ ಕಾಲದ ಭಾವನೆ ಪರಮಾತ್ಮನೊಬ್ಬನಲ್ಲಿ ಮಾತ್ರ ಇರುವುದಕ್ಕೆ ಸಾಧ್ಯ. ಏಕೆಂದರೆ ಅವನು ಕಾಲಕ್ಕೆ ಕಾಲ.

ಸರ್ವತೋಮುಖನಾದ ಧಾತೃ. ಎಲ್ಲಾ ಕಡೆಯಲ್ಲಿಯೂ ಜೀವರಾಶಿಗಳಿಗೆ ಅವರವರ ಕರ್ಮಗಳಿಗೆ ತಕ್ಕ ಫಲವನ್ನು ಅವನು ಕೊಡುತ್ತಿರುವನು. ಅವನಿಲ್ಲದ ಸ್ಥಳವೇ ಇಲ್ಲ. ಅವನು ಸರ್ವವ್ಯಾಪಿಯಾಗಿ ಸರ್ವವನ್ನೂ ನಿಯಮಿಸುತ್ತಿರುವನು.

\begin{shloka}
ಮೃತ್ಯುಃ ಸರ್ವಹರಶ್ಚಾಹಮುದ್ಭವಶ್ಚ ಭವಿಷ್ಯತಾಮ್~।\\ಕೀರ್ತಿಃ ಶ‍್ರೀರ್ವಾಕ್ ಚ ನಾರೀಣಾಂ ಸ್ಮೃತಿರ್ಮೇಧಾ ಧೃತಿಃ ಕ್ಷಮಾ \hfill॥ ೩೪~॥
\end{shloka}

\begin{artha}
ಸಕಲವನ್ನೂ ಸಂಹರಿಸುವ ಮೃತ್ಯು ನಾನು, ಭವಿಷ್ಯದಲ್ಲಿ ಉತ್ಪತ್ತಿಯಾಗುವುದಕ್ಕೆ ಕಾರಣವೂ ನಾನೆ. ಸ್ತ್ರೀಯರಲ್ಲಿ ಕೀರ್ತಿ, ಶ‍್ರೀ, ವಾಕ್, ಸ್ಮೃತಿ, ಮೇಧಾ ಧೃತಿ, ಕ್ಷಮೆ ಇವು ನಾನು.
\end{artha}

ನಾನೇ ಸರ್ವಹರನಾದ ಮೃತ್ಯು ಎನ್ನುತ್ತಾನೆ. ಈ ಬ್ರಹ್ಮಾಂಡದಲ್ಲಿ ಎಲ್ಲವೂ ನಾಶವಾಗುವುದು. ಕೆಲವು ಬೇಗ ನಾಶವಾಗುವುವು. ಮತ್ತೆ ಕೆಲವು ನಿಧಾನವಾಗಿ ನಾಶವಾಗುವುವು. ಅಂತೂ ನಾಶದ ಕೈಯಿಂದ ಯಾವುದೂ ತಪ್ಪಿಸಿಕೊಳ್ಳುವಂತಿಲ್ಲ. ನಾಶವೆಂದರೆ ಎಂದೆಂದಿಗೂ ನಾಶವಾಗಿಯೇ ಹೋಗುವುದಿಲ್ಲ. ಅದು ಮುಂದೆ ಮತ್ತೆಲ್ಲೊ ಪ್ರಾರಂಭಿಸುವುದು. ಹಾಗೆ ಪ್ರಾರಂಭವಾಗಬೇಕಾದರೆ ಅದಕ್ಕೊಂದು ಬೀಜ ಇರಬೇಕು. ಆ ಪ್ರಾರಂಭದ ಬೀಜವೂ ಅವನೇ.

ಇನ್ನುಮೇಲೆ ಮನೆಯ ಒಳಗೆ ಪ್ರವೇಶಿಸಿ ಹೇಳುವನು. ನಾರಿಯರಲ್ಲಿ ಕೀರ್ತಿ ನಾನು ಎನ್ನುವನು. ಕೀರ್ತಿ ಎಂದರೆ ಒಳ್ಳೆಯ ಹೆಸರು. ಒಳ್ಳೆಯ ಹೆಸರನ್ನು ಗಳಿಸಿದ ಸ್ತ್ರೀಯ ಗುಣ ಪರಿಮಳದಂತೆ ತಾನೇ ಹಬ್ಬುವುದು. ಅದನ್ನು ಯಾರೂ ಜಾಹಿರಾತು ಮಾಡಬೇಕಾಗಿಲ್ಲ. ಮಲ್ಲಿಗೆಯೊಂದು ಅರಳುವುದು. ನಾನಿಲ್ಲಿರುವೆ ಎಂದು ಗಲಾಟೆ ಮಾಡದೆ ತನ್ನ ಸೌರಭವನ್ನು ಹರಡುವುದು. ಪುಣ್ಯವತಿಯರ ಚಾರಿತ್ರ್ಯವೂ ಹಾಗೆಯೇ. ನಾರಿಯರಲ್ಲಿರುವ ಶ‍್ರೀ ಎಂದರೆ ಮಂಗಳ, ಸಂಪತ್ತು, ಒಳ್ಳೆಯ ಗುಣಗಳು. ಇವುಗಳೆಲ್ಲ ನಾನೇ ಎನ್ನುವನು. ಮನು, “ನ ಗೃಹಂ ಗೃಹ ಉಚ್ಯತೇ, ಗೃಹಿಣೀ ಗೃಹ ಉಚ್ಯತೇ” ಎನ್ನುವನು. ಬರೀ ಮನೆಯನ್ನು ಮನೆ ಎನ್ನುವುದಕ್ಕೆ ಆಗುವುದಿಲ್ಲ, ಎಲ್ಲಿ ಗೃಹಿಣಿ ಇರುವಳೊ ಅದೇ ಮನೆ. ಗೃಹಿಣಿ ಯಾರು? ಸಂಪತ್ತಿನಿಂದ ಕೂಡಿರುವವಳು, ಮಂಗಳದಿಂದ ಕೂಡಿರುವವಳು, ಶುಭದಿಂದ ಕೂಡಿರುವವಳು. ಇವಳೇ ಮನೆಯ ದೀವಿಗೆ, ಮನೆಯನ್ನೆಲ್ಲಾ ತುಂಬಿರುವವಳು, ಮನೆಯ ಲ್ಲೆಲ್ಲಾ ಆನಂದವನ್ನು ಹರಿಸುತ್ತಿರುವವಳು.

ನಾರಿಯರಲ್ಲಿರುವ ವಾಕ್ ನಾನೇ ಎನ್ನುವನು. ವಾಕ್ ಎಂದರೆ ಬರೀ ಮಾತಲ್ಲ. ಎಲ್ಲರೂ ಬೇಕಾದಷ್ಟು ಮಾತನಾಡುವರು. ಆದರೆ ಮಾತೊಂದು ಕಲೆ. ಆಡುವವರಿಗೆಲ್ಲ ಅದು ಗೊತ್ತಿಲ್ಲ. ಮಾತನ್ನು ಕೂರಲಗಿನಂತೆ ಉಪಯೋಗಿಸುವವರು ಹಲವರು. ಅದರಿಂದಾದ ಗಾಯ ಕತ್ತಿಯ ಗಾಯಕ್ಕಿಂತ ಆಳವಾಗಿ ಹೋಗುವುದು. ಆದರೆ ಯಾರು ಮಾತಿನಿಂದ ಶಾಂತಿ ಕೊಡುವರೊ, ಸಮಾಧಾನ ಕೊಡುವರೊ, ದುಃಖವನ್ನು ತಗ್ಗಿಸುವರೊ, ಕೇಳಿದವರಿಗೆ ಭರವಸೆಯನ್ನು ಕೊಡುವರೊ, ಈ ಮಾತಿನ ಹಿಂದೆ ಭಗವಂತನೇ ಇರುವನು. ಹಣ ಮಾಡದುದನ್ನು, ಔಷಧಿ ಮಾಡದುದನ್ನು, ಶಕ್ತಿ ಮಾಡದುದನ್ನು ಮಾತೆಂಬ ಮಂತ್ರ ಮಾಡುವುದು.

ಸ್ಮೃತಿಯೇ ನಾನು ಎನ್ನುವನು. ಇಲ್ಲಿ ಸ್ಮೃತಿ ಎಂದರೆ ಎಂತಹ ಸ್ಮೃತಿ? ನನಗೆ ಮಾಡಿದ ಅನ್ಯಾಯವನ್ನೇ ಮೆಲುಕುತ್ತಾ, ಯಾವಾಗ ಅದನ್ನು ಬಡ್ಡಿ ಸಹಿತ ಕೊಡುವೆನೊ ಎಂದು ಮರೆಯದಂತೆ ಕಾಪಾಡಿಕೊಂಡಿರುವುದಲ್ಲ. ಇನ್ನೊಬ್ಬರು ನಮಗೆ ಮಾಡಿದ ಒಳ್ಳೆಯದನ್ನು ಎಂದೆಂದಿಗೂ ಮರೆಯದೆ ನಮ್ಮ ಸ್ಮೃತಿ ಭಂಡಾರದಲ್ಲಿ ಅದನ್ನು ಒಂದು ಅನರ್ಘ್ಯ ರತ್ನದಂತೆ ಸಂರಕ್ಷಿಸುವುದು. ನನ್ನ ಹೃದಯ ಯಾವಾಗಲೂ ಅವರಿಗೆ ಕೃತಜ್ಞತೆಯ ಹೊನಲನ್ನು ಹರಿಸುತ್ತಿರಬೇಕು. ಇದೆಲ್ಲ ಅವರಿಗೆ ಬೇಕೆಂತಲ್ಲ. ಆದರೆ ನಮ್ಮ ಶ್ರೇಯಸ್ಸಿಗೆ ದಾರಿ ಅದು.

ಮೇಧಸ್ಸು ಎಂದರೆ ಬುದ್ಧಿಶಕ್ತಿ. ಒಂದು ಮನೆಯನ್ನು ಚೆನ್ನಾಗಿ ನಡೆಸಿಕೊಂಡು ಹೋಗ\-ಬೇಕಾದರೆ ದೊಡ್ಡ ಜಾಣ್ಮೆ ಬೇಕಾಗುವುದು. ಮನೆ ಎಂದರೆ ನೆಂಟರು ಇಷ್ಟರು, ಮಕ್ಕಳು ಸ್ನೇಹಿತರು, ನೆರೆಹೊರೆಯವರು, ಇವರಿಗೆ ಸಂಬಂಧಪಟ್ಟ ಹಲವಾರು ಸಮಸ್ಯೆಗಳು ಉದ್ಭವಿಸುವುವು. ಅವು\-ಗಳನ್ನೆಲ್ಲಾ ಎದುರಿಸಿ ಸಾವಧಾನದಿಂದ ಸಂಸಾರವನ್ನು ನಡೆಸುವುದಕ್ಕೆ ಬೇಕಾದಷ್ಟು ಬುದ್ಧಿವಂತಿಕೆ ಬೇಕು.

\newpage

ಧೃತಿ ಎಂದರೆ ಧೈರ್ಯ. ಗಂಡಸು ಸಮರಾಂಗಣದಲ್ಲಿ ತನ್ನ ಧೈರ್ಯ ತೋರಿಸುತ್ತಾನೆ. ಆದರೆ ಹೆಂಗಸಿನ ಕುರುಕ್ಷೇತ್ರವೇ ಮನೆ. ಎಷ್ಟೊಂದು ಕಷ್ಟ ನಷ್ಟಗಳು ಇವಳ ಸತ್ವವನ್ನು ಪರೀಕ್ಷಿಸುವುದಕ್ಕೆ ಕಾದು ನಿಂತಿವೆ! ಬೆಳಗಾದರೆ ಅವನ್ನು ಎದುರಿಸುವುದೂ ಒಂದು ದೊಡ್ಡ ಧೈರ್ಯವೇ. ಸಮರಾಂಗಣದಲ್ಲಿ ವೀರ ತೋರುವ ಧೈರ್ಯ ಎಷ್ಟು ಮುಖ್ಯವೋ ನಮ್ಮ ದೇಶಕ್ಕೆ, ಅಷ್ಟೆ ಮುಖ್ಯ ನಾರಿ ತೋರುವ ಧೈರ್ಯ ಮನೆಯ ಶ್ರೇಯಸ್ಸಿಗೆ. 

ಕ್ಷಮೆ ನಾನು ಎನ್ನುವನು. ಕ್ಷಮೆಯಂತಹ ಗುಣ ಮತ್ತೊಂದಿಲ್ಲ. ಯಾರೇನು ಅನ್ಯಾಯ ಮಾಡಿದರೂ ಎಲ್ಲವನ್ನೂ ಮರೆತು, ಎಲ್ಲರಿಗೂ ಒಳ್ಳೆಯದನ್ನು ಆಶಿಸಬೇಕಾದರೆ ಹೃದಯ ಎಷ್ಟು ಉದಾರವಾಗಿರಬೇಕು. ರಾಮಾಯಣದಲ್ಲಿ ಸೀತೆಯ ಶೀಲದಲ್ಲಿ ಈ ಗುಣವನ್ನು ಸುಂದರವಾಗಿ ಚಿತ್ರಿಸಿದ್ದಾನೆ ಕವಿ ವಾಲ್ಮೀಕಿ. ರಾವಣನ ವಧೆ ಆಯಿತು ಎಂಬ ಸಮಾಚಾರವನ್ನು ಹನುಮಂತ ಬಂದು ಸೀತೆಗೆ ಕೊಡುತ್ತಾನೆ. ಇದನ್ನು ಕೇಳಿದಾಗ ಅವಳಿಗೆ ಪರಮಾನಂದವಾಗುವುದು. ಅವಳು ಆಗ, ಮಗು ನನಗೆ ಇಂತಹ ಪ್ರಿಯವಾದ ವಾರ್ತೆಯನ್ನು ತಂದಿರುವೆ. ನಿನಗೆ ಕೊಡುವುದಕ್ಕೆ ನನ್ನಲ್ಲಿ ಏನೂ ಇಲ್ಲವಲ್ಲ ಎಂದು ವ್ಯಥೆಪಡುತ್ತಾಳೆ. ಆಗ ಹನುಮಂತ ನಿಮಗೆ ಒಂದು ಸಾಧ್ಯ, ದಯವಿಟ್ಟು ಅದನ್ನು ಕೊಡಿ ಎನ್ನುತ್ತಾನೆ. ಅದೇನು ಎಂದು ಕೇಳಿದಾಗ ನಿಮ್ಮನ್ನು ಹಿಂಸಿಸುತ್ತಿದ್ದ ರಾಕ್ಷಸಿಯರಿಗೆ ಸ್ವಲ್ಪ ಬುದ್ಧಿ ಕಲಿಸಬೇಕೆಂದಿರುವೆ. ಅದಕ್ಕೆ ಅವಕಾಶ ಕೊಡಿ ಎಂದು ಕೇಳುತ್ತಾನೆ ಹನುಮಂತ. ಆಗ ಸೀತೆ, ರಾಕ್ಷಸಿಯರು ಪರಾಧೀನರು, ಅವರ ಸ್ವಾಮಿ ಹೇಳಿದಂತೆ ಮಾಡಿದರು. ಅದಕ್ಕಾಗಿ ಅವರನ್ನು ದೂರ ಕೂಡದು. ಪ್ರಪಂಚದಲ್ಲಿ ಎಂತಹ ತಪ್ಪನ್ನು ಮಾಡಿದ್ದರೂ ಅದನ್ನು ಕ್ಷಮಿಸಬೇಕು. ತಪ್ಪನ್ನು ಮಾಡದವರು ಯಾರಿದ್ದಾರೆ? ಎನ್ನುತ್ತಾಳೆ ಜಾನಕಿ.

ಭಗವಂತ ಹೇಗೆ ಬಾಹ್ಯ ಪ್ರಪಂಚದಲ್ಲಿ ನಮ್ಮನ್ನು ಆಕರ್ಷಿಸುವ ಅದ್ಭುತ ಕಾರ್ಯಗಳಲ್ಲಿ ಇದ್ದಾನೆಯೋ ಹಾಗೆಯೆ ಮನೆಮನೆಯಲ್ಲಿ ಯಾರ ಕಣ್ಣಿಗೂ ಬೀಳದೆ ದೈನಂದಿನ ಕೆಲಸ ಕಾರ್ಯಗಳನ್ನು ಸುಸೂತ್ರವಾಗಿ ಮಾಡಿಕೊಂಡು ಹೋಗುತ್ತ, ಶಾಂತಿ ಸಮಾಧಾನಗಳನ್ನು ಚೆಲ್ಲುತ್ತಿರುವ ನಾರಿಯರ ಶೀಲದಲ್ಲಿಯೂ ತಾನಿರುವೆ ಎನ್ನುತ್ತಾನೆ.

\begin{shloka}
ಬೃಹತ್ಸಾಮ ತಥಾ ಸಾಮ್ನಾಂ ಗಾಯತ್ರೀ ಛಂದಸಾಮಹಮ್~।\\ಮಾಸಾನಾಂ ಮಾರ್ಗಶೀರ್ಷೋಽಹಮೃತೂನಾಂ ಕುಸುಮಾಕರಃ \hfill॥ ೩೫~॥
\end{shloka}

\begin{artha}
ನಾನು ಸಾಮಗಳಲ್ಲಿ ಬೃಹತ್ಸಾಮ, ಛಂದಸ್ಸುಗಳಲ್ಲಿ ಗಾಯತ್ರಿ, ಮಾಸಗಳಲ್ಲಿ ಮಾರ್ಗಶೀರ್ಷ ಮತ್ತು ಪುತುಗಳಲ್ಲಿ ವಸಂತ.
\end{artha}

ಹಿಂದೆಯೆ ವೇದಗಳಲ್ಲಿ ನಾನು ಸಾಮವೇದವೆಂದು ಶ‍್ರೀಕೃಷ್ಣನು ಹೇಳಿರುವನು. ಇಲ್ಲಿ ಆ ಸಾಮವೇದದಲ್ಲಿ ಬರುವ ಬೃಹತ್ಸಾಮ ಎಂಬ ಭಾಗ ಮುಖ್ಯ. ಕಾಯಿಯೊಳಗೆ ತಿರುಳು ಇದ್ದಂತೆ. ವೇದಮಂತ್ರಗಳನ್ನು ಹಲವು ಛಂದಸ್ಸುಗಳಲ್ಲಿ ಬರೆದಿರುವರು. ಅವುಗಳಲ್ಲಿ ಗಾಯತ್ರಿ ಎಂಬುದು ಒಂದು ಛಂದಸ್ಸು. ಈ ಛಂದಸ್ಸಿನಲ್ಲಿಯೇ ಗಾಯತ್ರೀ ಮಂತ್ರ ಕೂಡ ಇರುವುದು. ಇಂದಿಗೂ ಕೂಡ ಪ್ರತಿಯೊಬ್ಬ ಹಿಂದೂವಿಗೂ ಗೊತ್ತಿರುವ ಪವಿತ್ರವಾದ ಮಂತ್ರ ಇದು. ಮೂರು ಲೋಕಗಳನ್ನೂ ಸೃಷ್ಟಿಸಿರುವ ಆ ಚೈತನ್ಯ, ನನ್ನ ಬುದ್ಧಿಯನ್ನು ಪ್ರಚೋದಿಸಲಿ ಎಂದು ಹೇಳುವುದು ಆ ಮಂತ್ರ. ಮಾಸಗಳಲ್ಲಿ ಮಾರ್ಗಶೀರ್ಷ ಎನ್ನುವನು. ಈ ಮಾಸದಲ್ಲಿಯೇ ವ್ರತಗಳು ಜಾಸ್ತಿ. ವೈದಿಕ ಕಾಲದಲ್ಲಿ ಈ ಮಾಸದಿಂದಲೇ ವರುಷ ಪ್ರಾರಂಭ ಆಗುತ್ತಿತ್ತು. ಚಂದ್ರ ಈ ಮಾಸದಲ್ಲಿ ಪೂರ್ಣಿಮೆಯ ದಿನ ಮಾರ್ಗಶಿರ ನಕ್ಷತ್ರ ರಾಶಿಯಲ್ಲಿ ಇರುವನು. ಪುತುಗಳಲ್ಲೆಲ್ಲ ವಸಂತ ಬಹಳ ಆಹ್ಲಾದಕರವಾಗಿರುವುದು. ಅತಿ ಛಳಿಯೂ ಇಲ್ಲ, ಅತಿ ಬಿಸಿಲೂ ಇಲ್ಲ. ಹೊರಗೆ ಪ್ರಕೃತಿಯೆಲ್ಲ ಚಿಗುರಿನಿಂದ ಕೂಡಿ ನೋಡಲು ಅಂದವಾಗಿ ಕಾಣುವುದು.

\begin{shloka}
ದ್ಯೂತಂ ಛಲಯತಾಮಸ್ಮಿ ತೇಜಸ್ತೇಜಸ್ವಿನಾಮಹಮ್~।\\ಜಯೋಽಸ್ಮಿ ವ್ಯವಸಾಯೋಽಸ್ಮಿ ಸತ್ತ್ವಂ ಸತ್ತ್ವವತಾಮಹಮ್ \hfill॥ ೩೬~॥
\end{shloka}

\begin{artha}
ಮೋಸ ಮಾಡುವುದರಲ್ಲಿ ದ್ಯೂತವೂ, ತೇಜಸ್ವಿಗಳ ತೇಜಸ್ಸೂ ನಾನೇ ಆಗಿರುವೆನು. ನಾನು ಜಯಶಾಲಿಗಳ ಜಯ, ವ್ಯವಸಾಯಿಗಳ ವ್ಯವಸಾಯ ಮತ್ತು ಸಾತ್ವಿಕರ ಸತ್ತ್ವ.
\end{artha}

ನಾನು ಮೋಸ ಮಾಡುವವರಲ್ಲಿ ದ್ಯೂತ ಎನ್ನುತ್ತಾನೆ. ಆ ಮೋಸದ ಹಿಂದೆ ಇರುವ ಜಾಣ್ಮೆಯೂ ಇವನದೆ. ಆಟದಲ್ಲಿ ಇನ್ನೊಬ್ಬನನ್ನು ಸೋಲಿಸಬೇಕಾದರೆ ಆಡುವವನು ಬೀಳುವ ಗರದ ಮೇಲೆ ತನ್ನ ಆಟವನ್ನು ಆಡಬೇಕಾಗಿದೆ. ತನಗೆ ಬೇಕಾದ ಗರ ಬಿದ್ದರೆ ಯಾರು ಬೇಕಾದರೂ ಗೆಲ್ಲುತ್ತಾರೆ, ಬೇಡದ ಗರ ಬಿದ್ದರೂ ಗೆಲ್ಲುವವನೇ ನಿಜವಾದ ಆಟಗಾರ. ಈ ಆಟವನ್ನು ಅನೇಕ ವೇಳೆ ಇನ್ನೊಬ್ಬನಲ್ಲಿರುವ ದುಡ್ಡನ್ನು ಸುಲಿಯುವುದಕ್ಕೆ ಆಡುತ್ತಾರೆ. ಕೌರವರು ಪಾಂಡವರನ್ನು ಸೋಲಿಸಿದ್ದು ಹೀಗೆ. ಇಲ್ಲಿ ರಹಸ್ಯವಾಗಿ ಗುಟ್ಟಾಗಿ ಏನನ್ನೂ ಮಾಡುವುದಿಲ್ಲ. ಕಳ್ಳತನ, ದರೋಡೆ, ಕೊಲೆ ಹಾಗಲ್ಲ. ಈ ಆಟದಲ್ಲಿ ಇಬ್ಬರೂ ಅದಕ್ಕೆ ಸಿದ್ಧರಾಗಿರುತ್ತಾರೆ. ಸುತ್ತಲೂ ಬೇಕಾದಷ್ಟು ಜನ ಇದನ್ನು ನೋಡಲು ನೆರೆದಿರುವರು. ಇಲ್ಲಿ ತನ್ನ ಬುದ್ಧಿವಂತಿಕಯಿಂದ ಒಬ್ಬ ಗೆಲ್ಲುವನು. ಇನ್ನೊಬ್ಬನ ಹಣವನ್ನು ಅಪಹರಿಸುವುದಕ್ಕೆ ಹಲವು ರೀತಿಗಳಿವೆ. ಅವುಗಳಲ್ಲೆಲ್ಲ ನಾನು ದ್ಯೂತ ಎನ್ನುತ್ತಾನೆ. 

ತೇಜಸ್ ಎಂಬ ಪದ ಧ್ವನಿವೂರ್ಣವಾದುದು. ಒಬ್ಬ ತಾನು ಯಾರು ಎಂಬುದನ್ನು ಹೇಳಿ\-ಕೊಳ್ಳಲೇಬೇಕಾಗಿಲ್ಲ. ಬ್ರಹ್ಮತೇಜಸ್ಸಾಗಲೀ ಕ್ಷಾತ್ರತೆಜಸ್ಸಾಗಲಿ ಅವನಲ್ಲಿ ಉಕ್ಕಿ ಬರುತ್ತಿರುವುದು! ನೋಡಿದವರು ಅದರಿಂದ ಆಕರ್ಷಿತರಾಗುವರು. ಒಬ್ಬನ ವ್ಯಕ್ತಿತ್ವದ ಸಾರವೇ ತೇಜಸ್ಸು ಎಂಬ ಪದದಲ್ಲಿದೆ. 

ಜಯಶಾಲಿಗಳಲ್ಲಿರುವ ಫಲವೇ ಜಯ. ಇವನು ಜಯಶಾಲಿಗಳ ಫಲಸ್ವರೂಪನಾಗಿದ್ದಾನೆ. ವ್ಯವಸಾಯಿಗಳಲ್ಲಿ ಎಂದರೆ ಕಷ್ಟಪಟ್ಟು ಕೆಲಸಮಾಡುವವರಲ್ಲಿ, ಅದು ಬೇಸಾಯ ಇರಬಹುದು, ಫ್ಯಾಕ್ಟರಿಗಳಲ್ಲಿ ಮಾಡುವ ಕೆಲಸ ಇರಬಹುದು, ಹಲವು ಕುಶಲಕಲೆಗಳು ಇರಬಹುದು, ಎಲ್ಲಿಯಾದರೂ ಆಗಲಿ ಒಬ್ಬ ಕಷ್ಟಪಟ್ಟು ಒಂದು ವಸ್ತುವನ್ನು ಉತ್ಪತ್ತಿಮಾಡುವುದರಲ್ಲಿ ನಿರತನಾಗಿರುವನೊ ಆ ಶ್ರಮದ ಹಿಂದೆ ಇವನೇ ಇರುವನು. ಅಂದರೆ ನಮ್ಮ ಜೀವನದಲ್ಲೆಲ್ಲಾ ಅವನು ಹಾಸು ಹೊಕ್ಕಾಗಿರುವನು. ತೇಜಸ್ವಿಗಳಲ್ಲಿ ಹೆಚ್ಚು ಕಾಣುವನು. ಇತರರಲ್ಲಿ ಇಲ್ಲ ಎಂದಲ್ಲ. ಎಲ್ಲರನ್ನೂ ಆಕರ್ಷಿಸುವಷ್ಟು ಇಲ್ಲ ಅಷ್ಟೆ. ಹಾಗೆಯೆ ಶ್ರಮಜೀವಿಗಳ ಶ್ರಮದ ಹಿಂದೆಯೂ ಅವನೇ ಇರುವನು.

\begin{shloka}
ವೃಷ್ಣೀನಾಂ ವಾಸುದೇವೋಽಸ್ಮಿ ಪಾಂಡವಾನಾಂ ಧನಂಜಯಃ~।\\ಮುನೀನಾಮಪ್ಯಹಂ ವ್ಯಾಸಃ ಕವೀನಾಮುಶನಾ ಕವಿಃ \hfill॥ ೩೭~॥
\end{shloka}

\begin{artha}
ನಾನು ಯಾದವರಲ್ಲಿ ವಾಸುದೇವ, ಪಾಂಡವರಲ್ಲಿ ಧನಂಜಯ, ಮುನಿಗಳಲ್ಲಿ ವೇದವ್ಯಾಸ ಮತ್ತು ಕವಿಗಳಲ್ಲಿ ಉಶನಾ ಕವಿ.
\end{artha}

ಶ‍್ರೀಕೃಷ್ಣ ಯಾದವರಲ್ಲಿ ವಸುದೇವನಿಗೆ ಮಗನಾಗಿ ಹುಟ್ಟಿದನು. ಈತ ಯಾದವಕುಲದಲ್ಲೆಲ್ಲಾ ಪ್ರಖ್ಯಾತನಾದವನು ಮಾತ್ರವಲ್ಲ, ಈಗ ಭಗವತ್ ಗೀತೆಯನ್ನು ಹೇಳುತ್ತಿರುವವನೂ ಇವನೇ ಆಗಿರುವನು. ಪಾಂಡವರಲ್ಲಿ ಪ್ರಖ್ಯಾತನಾದ ಬಿಲ್ಲುಗಾರನಾದ ಅರ್ಜುನನ ಸಾಮರ್ಥ್ಯದ ಹಿಂದೆ ಇರುವುದು ಕೂಡ ಭಗವಂತನ ವಿಭೂತಿಯೆ. ಮುನಿಗಳು ಎಂದರೆ ಎಲ್ಲವನ್ನೂ ಬಲ್ಲವರು ಎಂದು ಅರ್ಥ. ಅವರಲ್ಲಿ ವ್ಯಾಸನೇ ಇವನು. ವ್ಯಾಸರು ಮಹಾಭಾರತ, ಬ್ರಹ್ಮಸೂತ್ರ, ಭಾಗವತವನ್ನು ಬರೆದವರು ಎಂದು ಖ್ಯಾತಿ ಗಳಿಸಿರುವರು. ಕವಿಗಳು ಎಂದರೆ ಹಿಂದಿನ ಕಾಲದಲ್ಲಿ ದ್ರಷ್ಟಾರ, ಎಲ್ಲವನ್ನೂ ತಿಳಿಯ ಬಲ್ಲವನು ಎಂದು ಅರ್ಥವಿತ್ತು. ಅದಕ್ಕೇ, ರವಿ ಕಾಣದ್ದನ್ನು ಕವಿ ಕಾಣುತ್ತಾನೆ ಎಂದು ಹೇಳುತ್ತಾರೆ. ಅಂತಹ ಕವಿಗಳಲ್ಲಿ ಶುಕ್ರಾಚಾರ್ಯರು ಪ್ರಖ್ಯಾತರು. ಇವರು ಅಸುರರ ಗುರುಗಳು. ಈತನಿಗೆ ಸತ್ತವರನ್ನೆ ಬದುಕಿಸುವ ಶಕ್ತಿ ಇತ್ತು. ಬೃಹಸ್ಪತಿ ತನ್ನ ಮಗನಾದ ಕಚನನ್ನು ಶುಕ್ರಾಚಾರ್ಯನ ಬಳಿಗೆ ಆ ವಿದ್ಯೆಯನ್ನು ಕಲಿತುಕೊಂಡು ಬರುವಂತೆ ಹೇಳಿಕಳುಹಿಸುತ್ತಾನೆ.

\begin{shloka}
ದಂಡೋ ದಮಯತಾಮಸ್ಮಿ ನೀತಿರಸ್ಮಿ ಜಿಗೀಷತಾಮ್~।\\ಮೌನಂ ಚೈವಾಸ್ಮಿ ಗುಹ್ಯಾನಾಂ ಜ್ಞಾನಂ ಜ್ಞಾನವತಾಮಹಮ್ \hfill॥ ೩೮~॥
\end{shloka}

\begin{artha}
ನಾನು ದಂಡಿಸುವವರ ಶಿಕ್ಷೆ, ಜಯಾಭಿಲಾಷೆಗಳ ನೀತಿ, ಗುಹ್ಯಗಳಲ್ಲಿ ಮೌನ, ಮತ್ತು ಜ್ಞಾನಿಗಳಲ್ಲಿ ಜ್ಞಾನ.
\end{artha}

ಸಮಾಜ ಭದ್ರವಾಗಿರಬೇಕಾದರೆ ಅಲ್ಲಿ ಕೆಲವು ನೀತಿನಿಯಮಗಳನ್ನು ಪಾಲಿಸುವುದು ಅತ್ಯಂತ ಆವಶ್ಯಕ. ಯಾರು ಅದನ್ನು ಅತಿಕ್ರಮಿಸಿ ಹೋಗುವರೊ, ಸಮಾಜಘಾತುಕರಂತೆ ವರ್ತಿಸುವರೊ, ಅವರಿಗೆ ದಂಡನೆಯನ್ನು ಕೊಡಬೇಕು. ಅದಕ್ಕಾಗಿ ಕೋರ್ಟು, ಕಚೇರಿ, ಪೋಲೀಸು,\break ನ್ಯಾಯಾಧಿ\-ಪತಿ, ಜೈಲು ಮುಂತಾದುವುಗಳೆಲ್ಲಾ ಇರುವುದು. ಈ ದಂಡನೆಯ ಹಿಂದೆಯೂ ಭಗವಂತನೇ ಕೆಲಸಮಾಡುತ್ತಿರುವನು.

ಜಯವನ್ನು ಪಡೆಯಬೇಕಾದರೆ ಕೆಲವು ನೀತಿಗಳನ್ನು ಅನುಸರಿಸಬೇಕಾಗಿದೆ. ಒಂದು ಯಾವಾ\-ಗಲೂ ಫಲಕಾರಿಯಾಗುವುದಿಲ್ಲ. ಅದಕ್ಕಾಗಿಯೆ ಸಾಮ ದಾನ ಭೇದ ದಂಡ ಎಂದು ಚತುರೋ\-ಪಾಯಗಳನ್ನು ಉಪಯೋಗಿಸುವರು. ಈ ಚತುರೋಪಾಯಗಳ ಹಿಂದೆಲ್ಲ ಕೆಲಸಮಾಡುವುದೇ ಭಗವಂತನ ಶಕ್ತಿ.

ಗುಹ್ಯಗಳಲ್ಲಿ ಎಂದರೆ ಅತ್ಯಂತ ರಹಸ್ಯವಾದ ವಿಷಯವನ್ನು ಬಚ್ಚಿಡಬೇಕಾದರೆ ಅದನ್ನು ಮೌನದ ಗುಹೆಯಲ್ಲಿ ಮಾತ್ರ ಇಡಬೇಕು. ಮಾತಿನಲ್ಲಿ ನಾವು ಯಾವುದನ್ನು ಹೇಳಬಾರದು ಎಂದು ಇರುತ್ತೇ ವೆಯೋ ಅದು ಹೇಗೊ ನಮ್ಮ ಅರಿವಿಲ್ಲದೆ ವ್ಯಕ್ತವಾಗಿ ಬಿಡುವುದು. ಆದಕಾರಣ ಬಚ್ಚಿಡಬೇಕಾದರೆ ಸುಮ್ಮನಿರಬೇಕು.

ಜ್ಞಾನಿಗಳು ಎಂದರೆ ಲೌಕಿಕ ಮತ್ತು ಆಧ್ಯಾತ್ಮಿಕ ವಿಷಯಗಳನ್ನು ತಿಳಿದವರು. ಎದುರಿಗೆ ಕಾಣುವುದನ್ನು ತಿಳಿದುಕೊಳ್ಳುವುದು ಸುಲಭ. ಆದರೆ ಅದರ ಅಂತರಾಳಕ್ಕೆ ಹೋಗಿ ಅದರ ನೈಜ ಸ್ವರೂಪವನ್ನು ಅರಿಯಬೇಕಾದರೆ ಅದಕ್ಕೊಂದು ದೊಡ್ಡ ತಪಸ್ಸನ್ನೇ ಮಾಡಬೇಕು.ಇದೇನು ಸುಮ್ಮನೇ ಸಿಕ್ಕುವುದಿಲ್ಲ. ವಿಜ್ಞಾನ ಪ್ರಪಂಚದಲ್ಲಿ ಖ್ಯಾತಿಯನ್ನು ಗಳಿಸುವ ವ್ಯಕ್ತಿಗಳು ನಮ್ಮ ಪುಷಿಗಳಿಗಿಂತ ಏನೂ ಕಡಮೆಯಲ್ಲ. ಆ ಸಾಮರ್ಥ್ಯದ ಹಿಂದೆ ಮತ್ತು ಆಧ್ಯಾತ್ಮಿಕ ಸಾಮರ್ಥ್ಯದ ಹಿಂದೆ ಇರುವುದು ಕೂಡ ಭಗವತ್ ವಿಭೂತಿಯೆ.

\begin{shloka}
ಯಚ್ಚಾಪಿ ಸರ್ವಭೂತಾನಾಂ ಬೀಜಂ ತದಹಮರ್ಜುನ~।\\ನ ತದಸ್ತಿ ವಿನಾ ಯತ್ ಸ್ಯಾನ್ಮಯಾ ಭೂತಂ ಚರಾಚರಮ್ \hfill॥ ೩೯~॥
\end{shloka}

\begin{artha}
ಅರ್ಜುನ, ಎಲ್ಲ ಭೂತಗಳಿಗೆ ಯಾವುದು ಕಾರಣವೋ ಅದು ಕೂಡ ನಾನೆ. ಸ್ಥಾವರ ಜಂಗಮಗಳಲ್ಲಿ ನನ್ನ ಹೊರತು ಯಾವುದೂ ಇಲ್ಲ.
\end{artha}

ಎಲ್ಲ ಭೂತಗಳ ಬೀಜವೂ ಪರಮಾತ್ಮನಿಂದಲೇ ಬಂದಿದೆ. ಅವನೇ ಉಪಾದಾನ ಕಾರಣ ಮತ್ತು ನಿಮಿತ್ತ ಕಾರಣ ಆಗಿರುವನು. ಅವನಿಂದಲೇ ಎಲ್ಲವೂ ಬಂದಿದೆ. ಎಲ್ಲದರ ಹಿಂದೆಯೂ ಅವನೇ ಇರುವನು. ಅಲೆ ಸಾಗರದಿಂದ ಬಂದಿದೆ. ಸಾಗರ ಅಲೆಯ ಹಿಂದೆ ಇದೆ. ಹಾಗೆಯೆ ಸರ್ವವಸ್ತುಗಳ ಹಿನ್ನೆಲೆ ದೇವರೇ ಆಗಿದ್ದಾನೆ.

ಸ್ಥಾವರ ಜಂಗಮಗಳೆಲ್ಲ ನಾಮರೂಪದ ದೃಷ್ಟಿಯಿಂದ ನೋಡಿದಾಗ ಬೇರೆಬೇರೆಯಾಗಿ ಕಾಣುವುವು. ಆದರೆ ಅದರ ಅಂತರಾಳಕ್ಕೆ ಹೋದರೆ ಭಿನ್ನತೆಗಳೆಲ್ಲ ಮಾಯವಾಗುವುವು. ಪರ\-ಮಾತ್ಮನ ಮೇಲೆ ಅವನು ಕಾಣದಂತೆ ಹಲವಾರು ನಾಮರೂಪಗಳ ಬಣ್ಣವನ್ನು ಬಳೆದಿದ್ದೇವೆ. ಅದರ ಹಿಂದೆಲ್ಲ ಅವನೇ ಇರುವನು. ಸಕ್ಕರೆಯಿಂದ ಹಲವು ವಿಧದ ಗೊಂಬೆಗಳನ್ನು ಮಾಡುತ್ತೇವೆ. ಒಂದೊಂದನ್ನು ಒಂದೊಂದು ಹೆಸರಿನಿಂದ ಕರೆಯುತ್ತೇವೆ. ಆದರೆ ಅದು ಯಾವುದರಿಂದ ಆಗಿದೆ ಎನ್ನುವಾಗ, ಸಕ್ಕರೆಯಲ್ಲದೆ ಅಲ್ಲಿ ಬೇರಾವುದೂ ಇಲ್ಲ ಎಂದು ಹೇಳಬೇಕಾಗಿದೆ. ಪರಮಾತ್ಮನೆ ಎಲ್ಲಾ ವಸ್ತುಗಳ ಹಿಂದೆ ಹುದುಗಿಕೊಂಡಿರುವನು. ಎಲ್ಲಿಯವರೆಗೆ ನಾವು ನಾಮರೂಪಕ್ಕೆ ವಶರಾಗಿ ಅದರ ಮೋಹಕ್ಕೆ ಬಿದ್ದಿರುವೆವೋ ಅಲ್ಲಿಯವರೆಗೆ ಅದನ್ನು ದಾಟಿಹೋಗಲು ನಮಗೆ ಸಾಧ್ಯವಿಲ್ಲ.\break ಯಾವಾಗ ನಾವು ಅದನ್ನು ದಾಟಿಹೋಗುತ್ತೇವೆಯೊ ಆಗ ಎಲ್ಲಾ ಕಡೆಯಲ್ಲಿಯೂ ಪರಮಾರ್ಥ ತತ್ವವೊಂದೇ ಗೋಚರಿಸುವುದು.

\begin{shloka}
ನಾಂತೋಽಸ್ತಿ ಮಮ ದಿವ್ಯಾನಾಂ ವಿಭೂತೀನಾಂ ಪರಂತಪ~।\\ಏಷತೂದ್ದೇಶತಃ ಪ್ರೋಕ್ತೋ ವಿಭೂತೇರ್ವಿಸ್ತರೋ ಮಯಾ \hfill॥ ೪ಂ~॥
\end{shloka}

\begin{artha}
ಅರ್ಜುನ, ನನ್ನ ದಿವ್ಯ ವಿಭೂತಿಗಳಿಗೆ ಅಂತ್ಯವಿಲ್ಲ. ವಿಭೂತಿಗಳ ವಿಸ್ತಾರವನ್ನು ನಾನು ಕೇವಲ ಸಂಕ್ಷೇಪದಿಂದ ಹೇಳಿದ್ದೇನೆ.
\end{artha}

ಭಗವಂತನ ದಿವ್ಯ ವಿಭೂತಿಗಳಿಗೆ ಅಂತ್ಯವಿಲ್ಲ. ಅವನು ಅನಂತ ಸ್ವರೂಪ. ಅವನ ವಿಭೂತಿಯೂ ಅನಂತ ಸ್ವರೂಪವೆ. ಅದನ್ನು ವರ್ಣಿಸುವುದಕ್ಕೆ ಆಗುವುದಿಲ್ಲ ಎಂದು ಮುಂಚೆಯೇ ಹೇಳದೆ, ಕೆಲವನ್ನು ವಿವರಿಸಿ, ಹೀಗೆಯೆ ಅನಂತವಾಗಿದೆ ಎನ್ನಬೇಕಾಗಿದೆ. ಇಲ್ಲಿ ಸಂಕ್ಷೇಪದಿಂದ ಹೇಳಿದ್ದೇನೆ ಎನ್ನುತ್ತಾನೆ. ಅರ್ಜುನನಿಗೆ ಅವನ ವಿಭೂತಿಯ ರುಚಿಯನ್ನು ಮಾತ್ರ ತೋರಿಸಿದ್ದಾನೆ. ಇದರಿಂದ ಅವನಿಗೆ ತೃಪ್ತಿಕೊಡುವುದಕ್ಕಲ್ಲ. ಅವನ ಅತೃಪ್ತಿಯನ್ನು ಹೆಚ್ಚಿಸಿ ಭಗವಂತ ಹೇಗಿರುವನೊ ಹಾಗೆ ತಿಳಿದುಕೊಳ್ಳುವುದಕ್ಕೆ ಅವನನ್ನು ಪ್ರೋತ್ಸಾಹಿಸುವುದಕ್ಕೆ ಇದು. ಭಗವಂತ ಹೇಳಿರುವ ಸಂಕ್ಷೇಪದಿಂದಲೇ ನಾವು ಅರಿಯಬಹುದು, ಅವನಿಲ್ಲದ ಸ್ಥಳವೇ ಇಲ್ಲ, ಅವನಿಲ್ಲದ ವಸ್ತುವೇ ಇಲ್ಲ ಎಂಬುದನ್ನು. ಅವನು ವಿಶ್ವದಲ್ಲೆಲ್ಲಾ ಪರಿವ್ಯಾಪ್ತನಾಗಿದ್ದಾನೆ.

\begin{shloka}
ಯದ್ಯದ್ವಿಭೂತಿಮತ್ ಸತ್ತ್ವಂ ಶ‍್ರೀಮದೂರ್ಜಿತಮೇವ ವಾ~।\\ತತ್ತದೇವಾವಗಚ್ಛ ತ್ವಂ ಮಮ ತೇಜೋಽಂಶಸಂಭವಮ್ \hfill॥ ೪೧~॥
\end{shloka}

\begin{artha}
ಯಾವ ಯಾವ ವಸ್ತು ವಿಭೂತಿಯುಳ್ಳದ್ದೊ, ಶ‍್ರೀಯುಕ್ತವೋ, ಅಥವಾ ಊರ್ಜಿತವಾಗಿದೆಯೊ ಅವೆಲ್ಲ ನನ್ನ ತೇಜಸ್ಸಿನ ಅಂಶವೆಂದು ತಿಳಿ.
\end{artha}

ಎಲ್ಲಿ ವಿಭೂತಿ ಇರಲಿ, ಅದು ಯಾವ ಕಾರ್ಯಕ್ಷೇತ್ರಕ್ಕೆ ಬೇಕಾದರೂ ಸೇರಿರಲಿ–\-ಅದು ತಪಸ್ಸಾಗಿರಬಹುದು, ಶೌರ್ಯವಾಗಿರಬಹುದು, ಬುದ್ಧಿಯಾಗಿರಬಹುದು, ಜಾಣ್ಮೆಯಾಗಿರ\break ಬಹುದು, ಶ್ರಮವಾಗಿರಬಹುದು, ರಣರಂಗದಲ್ಲಾಗಲೀ ಮನೆಯಲ್ಲಾಗಲೀ ತೋರುವ ಸಾಮರ್ಥ್ಯ ಆಗಿರಬಹುದು, ಕಲೆಯಾಗಿರಬಹುದು, ಕವಿತ್ವ ಆಗಿರಬಹುದು, ಸಂಗೀತವಾಗಿರಬಹುದು, ಚಿತ್ರ\-ಕಲೆ ಆಗಿರಬಹುದು, ಕೊನೆಗೆ ಜೂಜು ಆಗಿರಬಹುದು, ಅದರ ಹಿಂದೆ ಇರುವುದೆಲ್ಲ ಭಗವಂತನ ಅಂಶವೇ. ಆಕಾಶದಲ್ಲಿ ನಕ್ಷತ್ರದಲ್ಲಿ ಅವನಿರುವನು, ಸೂರ್ಯ ಚಂದ್ರರಲ್ಲಿ ಇರುವನು, ಹಾರುವ ಹಕ್ಕಿಯಲ್ಲಿ ಇರುವನು, ಬೀಸುವ ಗಾಳಿಯಲ್ಲಿರುವನು, ತೊನೆಯುವ ಮರದಲ್ಲಿರುವನು, ಕಾಲ\-ದಲ್ಲಿರುವನು, ಜನನದಲ್ಲಿರುವನು, ಮರಣದಲ್ಲಿ ಅವನಿರುವನು. ನಾವು ಕಲ್ಲಿನಿಂದ ಕಟ್ಟಿದ ದೇವಸ್ಥಾನದಲ್ಲಿ ಅವನನ್ನು ನೋಡಲು ಹೋಗುತ್ತೇವೆ. ಆದರೆ ಗೀತೆಯಲ್ಲಿ ಬರುವ ವಿಭೂತಿ\-ಯೋಗದ ಪ್ರಕಾರ ಈ ಭೂಮಂಡಲವೇ ಅವನ ಒಂದು ವಿರಾಟ್ ದೇವಾಲಯ. ಎಲ್ಲದರ ಹಿಂದೆ ಅವನು ಸ್ಪಂದಿಸುತ್ತಿರುವನು. ಎಲ್ಲವೂ ಅವನ ಗರ್ಭಗುಡಿಯೆ, ಎಲ್ಲವೂ ಅವನ ಮಹಿಮೆಯನ್ನು ಸಾರುತ್ತಲೇ ಇರುವುವು. ನಮ್ಮ ದೇಹದಲ್ಲಿರುವ ಸೂಕ್ಷ್ಮಾತಿಸೂಕ್ಷ್ಮವಾದ ಜೀವಾಣುವಾಗಲೀ, ಹೊರಗೆ ಅನಂತ ಸೂರ್ಯರನ್ನೊಳಗೊಂಡ ನೀಹಾರಿಕೆಯಾಗಲೀ ಎಲ್ಲ ಕಡೆಯಲ್ಲಿಯೂ ಅವನು ಮಿಡಿಯುತ್ತಿರುವನು. ಭಗವಂತನ ಇಂತಹ ಭೂಮದರ್ಶನವನ್ನು ನಮಗೆ ಈ ಅಧ್ಯಾಯ ಕೊಡುವುದು. ವಿಭೂತಿ ಇರುವ ಕಡೆ ನಮ್ಮನ್ನು ಆಕರ್ಷಿಸಿ ಬೆಳಗುವನು. ಎಲ್ಲಿ ಅದು ಕಡಮೆ ಇದೆಯೊ ಅಲ್ಲಿಯೂ ಅವನಿರುವನು. ಆದರೆ ಅದು ನಮ್ಮ ಜಡ ಬುದ್ಧಿಯನ್ನು ಆಕರ್ಷಿಸಲಾರದು ಅಷ್ಟೆ.

ಎಲ್ಲಿ ಶ‍್ರೀ ಇದೆಯೊ ಎಂದರೆ ಸಂಪತ್ತು ಮಂಗಳ ಶುಭ ಇವುಗಳೆಲ್ಲವು ಇವೆಯೊ ಅಲ್ಲೆಲ್ಲ ಭಗವಂತ ತುಂಬಿ ತುಳುಕುತ್ತಿರುವನು.

ಊರ್ಜಿತವಾಗಿ ಯಾವ ವಸ್ತು ಇರುವುದೊ ಅದರ ಹಿಂದೆ ಭಗವತ್ ಶಕ್ತಿ ಇದೆ. ಕಾಲ ಜಳ್ಳನ್ನು ತೂರಿ ಗಟ್ಟಿಯನ್ನು ಮಾತ್ರ ಇಟ್ಟುಕೊಳ್ಳುವುದು. ಎಷ್ಟೋ ಜನ ಕಾವ್ಯಗಳನ್ನು ಬರೆದರು, ಎಲ್ಲೊ ಕೆಲವು ಮಾತ್ರ ಉಳಿದಿವೆ. ಉಳಿದವುಗಳೆಲ್ಲ ಕಾಲನ ಕಸದ ಬುಟ್ಟಿಯಲ್ಲಿ ನಾಮಾವಶೇಷವಾಗಿ ಹೋಗಿವೆ. ಹಾಗೆ ಉಳಿಯಬೇಕಾದರೆ ಅದರಲ್ಲಿ ಯಾವುದೊ ಒಂದು ಶಕ್ತಿ ಇದೆ. ಅದರಿಂದ ಮಾನವಕೋಟಿ ಪ್ರಯೋಜನವನ್ನು ಪಡೆಯಬೇಕಾಗಿದೆ. ಅದಕ್ಕಾಗಿಯೇ ಅವು ಉಳಿದಿವೆ. ಅವಕ್ಕೆ ಇನ್ನೂ ಪಾತ್ರವಿದೆ ಸೃಷ್ಟಿಯಲ್ಲಿ. ಊರ್ಜಿತವಾಗಿ ನಿಂತಿರುವ ತೀರ್ಥವಾಗಬಹುದು, ಬೆಟ್ಟವಾಗಬಹುದು, ನದಿಯಾಗಬಹುದು, ಕಾವ್ಯವಾಗಬಹುದು, ಮಹಾಮಹಿಮರ ವ್ಯಕ್ತಿತ್ವಗಳಾಗಬಹುದು, ಇವುಗಳ ಹಿಂದೆಲ್ಲ ಭಗವಂತನ ವಿಭೂತಿ ಇದೆ. ಇದೆಲ್ಲ ಭಗವಂತನ ಅನಂತ ವಿಭೂತಿಯಲ್ಲಿ ಒಂದೆರಡು ಅಂಶಗಳನ್ನು ತೆಗೆದುಕೊಂಡು ಬೆಳಗುತ್ತಿವೆ. ಈ ಅಂಶ ಅವನ ಒಂದು ಕಿರಣ. ಅದೇ ಹೀಗಿದ್ದರೆ, ಎಲ್ಲಿಂದ ಈ ಕಿರಣಗಳೆಲ್ಲ ಬರುವುದೋ ಅದು ಇನ್ನೆಷ್ಟು ಭವ್ಯವಾಗಿರಬೇಕು! ಆ ಸ್ಥಿತಿ ವಾಕ್ ಮತ್ತು ವರ್ಣನೆಗೆ ಅತೀತ. ಅದನ್ನು ಕಂಡು ಅನುಭವಿಸಬೇಕೆ ಹೊರತು ಕೇಳಿ ತೃಪ್ತರಾಗುವ ಹಾಗಿಲ್ಲ.

\begin{shloka}
ಅಥವಾ ಬಹುನೈತೇನ ಕಿಂ ಜ್ಞಾತೇನ ತವಾರ್ಜುನ~।\\ವಿಷ್ಟಭ್ಯಾಹಮಿದಂ ಕೃತ್ಸ್ನಮೇಕಾಂಶೇನ ಸ್ಥಿತೋ ಜಗತ್ \hfill॥ ೪೨~॥
\end{shloka}

\begin{artha}
ಅಥವಾ ಅರ್ಜುನ, ಈ ವಿಭೂತಿಯನ್ನು ಅನೇಕ ವಿಧವಾಗಿ ಅರಿತುಕೊಂಡು ಪ್ರಯೋಜನವೇನು? ನಾನು ಈ ಜಗತ್ತನ್ನು ಒಂದೇ ಅಂಶದಿಂದ ಧರಿಸಿದ್ದೇನೆ.
\end{artha}

ಈ ವಿಭೂತಿಯನ್ನು ತಿಳಿದುಕೊಳ್ಳುತ್ತ ಹೋದರೆ ಇದಕ್ಕೆ ಒಂದು ಅಂತ್ಯವಿಲ್ಲ. ಏಕೆಂದರೆ ಅವನು ಅನಂತಗುಣನು. ಯಾವ ಒಂದನ್ನು ತಿಳಿದುಕೊಂಡರೆ ಈ ಪ್ರಪಂಚದಲ್ಲಿ ಎಲ್ಲವನ್ನೂ ತಿಳಿದುಕೊಳ್ಳ ಬಹುದೊ ಅದನ್ನು ತಿಳಿಯಬೇಕು. ಸುಮ್ಮನೆ ಅಂಶವನ್ನು ತಿಳಿದುಕೊಂಡರೆ ಪ್ರಯೋಜನವಿಲ್ಲ. ನಾವು ಮೂಲಕ್ಕೆ ಹೋಗಬೇಕು. ಆಗ ಇವೆಲ್ಲವೂ ನಮಗೆ ಗೊತ್ತಾಗುವುದು. ಈ ಬ್ರಹ್ಮಾಂಡವನ್ನು ಸೃಷ್ಟಿಸಿ ಭಗವಂತ ಖಾಲಿಯಾಗಿ ಹೋಗಿಲ್ಲ. ಇದು ಒಂದು ಪಾತ್ರೆಯ ಹಾಲನ್ನು ಮೊಸರು ಮಾಡಿದರೆ ಹೇಗೆ ಹಿಂದಿನ ಹಾಲು ಹೋಗಿ ಬರೀ ಮೊಸರು ಮಾತ್ರ ಉಳಿಯುವುದೊ ಹಾಗಲ್ಲ. ದೇವರ ಯಾವುದೊ ಸಣ್ಣ ಅಂಶ ಈ ಬ್ರಹ್ಮಾಂಡವಾಗಿದೆ. ಅನಂತ ಸಾಗರದಲ್ಲಿ ಯಾವುದೊ ಒಂದು ಅಲ್ಪ ಭಾಗ ಮಾತ್ರ ಉತ್ತರ ದಕ್ಷಿಣ ಧ್ರುವಗಳಲ್ಲಿ ಘನೀಭೂತವಾದ ಮಂಜಿನಗಡ್ಡೆಯಂತೆ ತೇಲುತ್ತಿದೆ. ಉಳಿದ ಜಲ ಜಲರೂಪದಲ್ಲಿಯೇ ಇದೆ. ಹಾಗೆಯೇ ಪರಿಣಾಮದ ದೃಷ್ಟಿಯಿಂದ ನೋಡಿದರೂ, ಪರಿಣಾಮಕ್ಕೆ ಸಿಕ್ಕಿರುವುದು ಸ್ವಲ್ಪ, ಸಿಕ್ಕದೆ ಇರುವುದು ಬಹಳ. ವಿಭೂತಿಗೆ ಅತೀತವಾಗಿರುವುದೇ ಭಗವಂತನ ಬಹುಭಾಗ. ಯಾರಿಂದ ಈ ವಿಭೂತಿ ಬರುವುದೊ ಆ ಕಡೆ ನಾವು ಹೋಗಬೇಕಾಗಿದೆ. ಇದೊಂದು ಕೈಮರ ಅಷ್ಟೆ. ಇದು ತೋರುವೆಡೆ ನಡೆಯಬೇಕು. ಸುಮ್ಮನೆ ಕೈಮರವನ್ನು ತಬ್ಬಿ ಕುಳಿತುಕೊಂಡರೆ ಗುರಿ ಸೇರಲಾರೆವು.

