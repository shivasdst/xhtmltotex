
\chapter{ಜ್ಞಾನಯೋಗ}

ಶ‍್ರೀಕೃಷ್ಣ ಅರ್ಜುನನಿಗೆ ಹೇಳುತ್ತಾನೆ:

\begin{shloka}
ಇಮಂ ವಿವಸ್ವತೇ ಯೋಗಂ ಪ್ರೋಕ್ತವಾನಹಮವ್ಯಯಮ್~।\\ವಿವಸ್ವಾನ್ ಮನವೇ ಪ್ರಾಹ ಮನುರಿಕ್ಷ್ವಾಕವೇಽಬ್ರವೀತ್ \hfill॥ ೧~॥
\end{shloka}

\begin{artha}
ಈ ಅವ್ಯಯವಾದ ಯೋಗವನ್ನು ನಾನು ಸೂರ್ಯನಿಗೆ ಹೇಳಿದೆನು. ಸೂರ್ಯ ಮನುವಿಗೆ ಹೇಳಿದನು. ಮನು ಇಕ್ಷ್ವಾಕುವಿಗೆ ಹೇಳಿದನು.
\end{artha}

ಶ‍್ರೀಕೃಷ್ಣ ಈ ಅವ್ಯಯವಾದ ಯೋಗವನ್ನು ಹಿಂದಿನ ಯುಗಗಳಲ್ಲಿ ಸೂರ್ಯನಿಗೆ ಹೇಳಿದೆ ಎನ್ನುತ್ತಾನೆ. ಅನಾಸಕ್ತಿಯಿಂದ ಕರ್ಮ ಮಾಡುವುದು, ಸಮತ್ವವೇ ಯೋಗವೆನ್ನುವುದು, ಕರ್ಮ ಮಾಡಿಯೂ ಅದರ ಫಲಕ್ಕೆ ಬೀಳದಿರುವ ಕರ್ಮ ಕೌಶಲವನ್ನು ಇದೇ ಪ್ರಥಮ ಬಾರಿಯಲ್ಲ ಹೇಳುತ್ತಿರುವುದು. ಹಿಂದೆ ಎಷ್ಟೋ ಸಲ ಹೇಳಿ ಆಗಿದೆ. ಮುಂದೆ ಎಷ್ಟೋ ಸಲ ಹೇಳಬೇಕಾಗಿದೆ. ಇದು ಅವ್ಯಯ. ಎಂದಿಗೂ ನಾಶವಾಗುವುದಿಲ್ಲ. ಈ ಪ್ರಪಂಚದಲ್ಲಿ ಹಲವು ಭೌತಿಕ ತತ್ತ್ವಗಳು ಇವೆ. ಆದರೆ ಮನುಷ್ಯ ಎಲ್ಲವನ್ನೂ ಕಂಡುಹಿಡಿದಿಲ್ಲ. ಅವನು ಕ್ರಮೇಣ ಕಂಡುಹಿಡಿಯುತ್ತಾ ಹೋಗುವನು. ಆ ನಿಯಮವೇನೊ ಇವನು ಕಂಡುಹಿಡಿದ ಮೇಲೆ ಜಾರಿಗೆ ಬಂದಂತೆ ತೋರಿದರೂ ಅದು ಯಾವಾಗಲೂ ಪೂರ್ವದಿಂದಲೂ ಇತ್ತು. ನ್ಯೂಟನ್ ಎಂಬುವನು ಆಕರ್ಷಣ ಸಿದ್ಧಾಂತವನ್ನು ಕಂಡುಹಿಡಿದನು. ಆದರೆ ಅದಕ್ಕೆ ಮುಂಚೆ ಆ ನಿಯಮ ಜಾರಿಯಲ್ಲಿರಲಿಲ್ಲವೆ? ಇತ್ತು. ಆದರೆ ಅದು ನಮಗೆ ಗೊತ್ತಿರಲಿಲ್ಲ. ಅದರಂತೆಯೇ ಆಧ್ಯಾತ್ಮಿಕ ಪ್ರಪಂಚದ ಸೂಕ್ಷ್ಮನಿಯಮಗಳು ಈ ಪ್ರಪಂಚದಲ್ಲಿವೆ. ಸಾಧನೆ ಮಾಡಿ ಜೀವನದಲ್ಲಿ ಮುಂದುವರಿದು ಹೋದ ವ್ಯಕ್ತಿಗಳು ಅದನ್ನು ಕಂಡುಹಿಡಿಯುವರು. ಇತರರಿಗೆ ಅದನ್ನು ಹೇಳುವರು. ಈ ನಿಯಮ ಯಾರಿಗೆ ಗೊತ್ತಿಲ್ಲದೇ ಇದ್ದರೂ, ಅದೇನು ನಾಶವಾಗುವುದಿಲ್ಲ. ಅದು ಸದಾ ಸ್ಪಂದಿಸುತ್ತಿರುತ್ತದೆ. ಆಕಾಶದಲ್ಲಿ ರೇಡಿಯೋ ಸಂಗೀತ ವಿದ್ಯುತ್ ಅಲೆಗಳಂತೆ ತೇಲುತ್ತಿದೆ. ನಮ್ಮ ಹತ್ತಿರ ರೇಡಿಯೋ ಇದ್ದರೆ ಅದನ್ನು ಟ್ಯೂನ್ ಮಾಡಿದರೆ ಕೇಳುತ್ತೇವೆ. ಆದರೆ ಟ್ಯೂನ್ ಮಾಡದೆ ಇದ್ದರೆ ಇಲ್ಲವೆಂದಲ್ಲ. ಅದಂತೂ ಚಿರ ಜಾಗ್ರತವಾಗಿ ಸ್ಪಂದಿಸುತ್ತಿರುವುದು.

ಮೊದಲಲ್ಲಿ ಸೂರ್ಯ, ಮನು, ಇಕ್ಷ್ವಾಕು ಮುಂತಾದ ಕ್ಷತ್ರಿಯರಿಗೆ ಇದು ಗೊತ್ತಾಗುವುದು ಎನ್ನುವನು ಶ‍್ರೀಕೃಷ್ಣ. ಏಕೆಂದರೆ ಕ್ಷತ್ರಿಯನು ಆಗಿನ ಕಾಲದಲ್ಲಿ ಸರ್ವಾಧಿಕಾರಿಗಳಾಗಿ ಎಲ್ಲವನ್ನೂ ಮಾಡಬೇಕಾಗಿತ್ತು. ಕರ್ಮಕ್ಷೇತ್ರದಲ್ಲಿ ಅವನು ನಿರಂಕುಶ ಪ್ರಭುವಾಗಿದ್ದ. ಅವನು ರಾಜ, ಸೇನಾನಿ, ನ್ಯಾಯಾಧಿಪತಿ ಮುಂತಾದ ಎಲ್ಲಾ ಕೆಲಸಗಳನ್ನೂ ಒಟ್ಟಿಗೆ ಮಾಡಬೇಕಾಗಿತ್ತು. ಬಹಳ ಜವಾಬ್ದಾರಿಯಾದ ಕೆಲಸ ಇದು. ಸಮಾಜದಲ್ಲಿ ಅದ್ಭುತ ಪರಿಣಾಮವನ್ನು ಅವನು ಬೀರುತ್ತಿದ್ದನು.\break ಅವನಿಗೆ ಎಲ್ಲರಿಗಿಂತ ಹೆಚ್ಚಾಗಿ ಕರ್ಮಯೋಗ ತಿಳಿದಿರಬೇಕಾಗಿತ್ತು. ಈ ಯೋಗದ ರಹಸ್ಯ ಗೊತ್ತಿಲ್ಲದೆ ಇದ್ದರೆ ಅವನು ತಾನು ಮಾಡುವ ಕರ್ಮದ ಪ್ರವಾಹಕ್ಕೆ ಸಿಕ್ಕಿ ಕೊಚ್ಚಿಕೊಂಡು ಹೋಗಿಬಿಡು\-ವನು. ಆದ ಕಾರಣವೇ ಭಗವಂತ ಮೊದಲು ಇದನ್ನು ಕ್ಷತ್ರಿಯರಿಗೆ ಕೊಡುವನು. ಅನೇಕವೇಳೆ ನಾವು ಆಧ್ಯಾತ್ಮಿಕ ಜೀವನ ಪ್ರಪಂಚದ ಹವ್ಯಾಸದಿಂದ ದೂರವಿರುವವರಿಗೆ ಮಾತ್ರ ಸಾಧ್ಯ ಎಂದು ಭಾವಿಸುತ್ತೇವೆ. ನಿರ್ಜನಾರಣ್ಯಗಳು, ಗಿರಿ, ಗುಹೆ ಇವೇ ಅವುಗಳನ್ನು ಕಲಿತುಕೊಳ್ಳುವುದಕ್ಕೆ ಇರುವ ಸ್ಥಾನಗಳು ಎಂದು ಭಾವಿಸುವೆವು. ತತ್ತ್ವಕ್ಕೆ ಅದೊಂದೇ ಅಲ್ಲ ಸ್ಥಳ. ಅದಕ್ಕೆ ಒಂದು ಅವಕಾಶ ಬರಬೇಕು. ಬಂದಾಗ ಯಾರಿಗಾದರೂ ಆಗಲಿ, ಎಲ್ಲಿಯಾದರೂ ಆಗಲಿ, ಅದು ಗೊತ್ತಾಗುವುದು. ಇಲ್ಲಿಯೇ ಭಗವದ್ಗೀತೆಯನ್ನು ಕುರುಕ್ಷೇತ್ರದ ಯುದ್ಧಭೂಮಿಯಲ್ಲಿ ಹೇಳುತ್ತಾನೆ. ಅಂದರೆ ಮನುಷ್ಯ ಪ್ರಚಂಡ ಕರ್ಮದಲ್ಲಿ ಮುಳುಗಿರುವಾಗಲೂ ಯಾವಾಗಲೂ ತನ್ನನ್ನು ಫಲದ ಕಡೆ ಹರಿದುಹೋಗದಂತೆ ನೋಡಿಕೊಳ್ಳು ವನೋ, ಕೇವಲ ಕರ್ಮ ಮಾಡುವುದಕ್ಕೆ ಮಾತ್ರ ತನಗೆ ಅಧಿಕಾರ, ಅದರಿಂದ ಬರುವ ಫಲಕ್ಕೆ ಅಲ್ಲ ಎಂಬುದನ್ನು ತಿಳಿದುಕೊಳ್ಳುವನೋ, ಆಗ ಅವನು ಉದ್ವೇಗವಶನಾಗದೆ ಎಂತಹ ಜಟಿಲ ಸಮಸ್ಯೆ ಯನ್ನಾದರೂ ಸೂಕ್ಷ್ಮ ತಾತ್ತ್ವಿಕ ವಿಷಯಗಳನ್ನಾದರೂ ತಿಳಿದುಕೊಳ್ಳುವ ಸ್ಥಿತಿಗೆ ಬರುವನು. ಆದ ಕಾರಣವೆ ಶ‍್ರೀಕೃಷ್ಣ ಮೊದಲು ಅದನ್ನು ಕ್ಷತ್ರಿಯರ ಮೂಲಕ ಜಗತ್ತಿಗೆ ಗೊತ್ತಾಗುವಂತೆ ಮಾಡುವನು.

\begin{shloka}
ಏವಂ ಪರಂಪರಾಪ್ರಾಪ್ತಮಿಮಂ ರಾಜರ್ಷಯೋ ವಿದುಃ~।\\ಸ ಕಾಲೇನೇಹ ಮಹತಾ ಯೋಗೋ ನಷ್ಟಃ ಪರಂತಪ \hfill॥ ೨~॥
\end{shloka}

\begin{artha}
ಹೀಗೆ ಪರಂಪರೆಯಿಂದ ಬಂದ ಇದನ್ನು ರಾಜರ್ಷಿಗಳು ತಿಳಿದಿದ್ದರು. ಆ ಯೋಗ ದೀರ್ಘ\-ಕಾಲವಾದುದರಿಂದ ಇಲ್ಲಿ ನಷ್ಟವಾಯಿತು.
\end{artha}

ಈ ಯೋಗ ಹಿಂದಿನಿಂದ ಪರಂಪರಾಗತವಾಗಿ ಬಂದುದು. ಜ್ಞಾನ ಒಬ್ಬನಿಂದ ಮತ್ತೊಬ್ಬನಿಗೆ ಹರಿದು ಬರುವುದು. ಒಂದು ದೀಪ ಆರಿಹೋಗುವುದಕ್ಕೆ ಮುಂಚೆ ಅದರಿಂದ ಮತ್ತೊಂದನ್ನು ಹತ್ತಿಸುವಂತೆ ಇದು. ಒಬ್ಬ ಮೊದಲು ಕಷ್ಟಪಟ್ಟು ಸಾಧನೆ ಮಾಡಿ ಒಂದು ನಿಯಮವನ್ನು ಕಂಡುಹಿಡಿದಮೇಲೆ, ಅನಂತರ ಬರುವವನು ಅದನ್ನು ತಾನು ತಿಳಿದುಕೊಂಡು ಇತರರಿಗೆ ಹೇಳಿ\-ಕೊಡುತ್ತಾನೆ.

ಈ ಕರ್ಮಯೋಗ ರಹಸ್ಯವನ್ನು ರಾಜರ್ಷಿಗಳು ತಿಳಿದಿದ್ದರು ಎನ್ನುವರು. ಇವರು ಬರೀ ಪುಷಿಗಳಲ್ಲ, ರಾಜಪುಷಿಗಳು. ಬರೀ ಪುಷಿಗಳಿಗೆ ಏನೂ ಕೆಲಸವಿಲ್ಲ. ಧ್ಯಾನ, ಜ್ಞಾನ, ಅಧ್ಯಯನ ಮುಂತಾದುವೇ ಅವರು ಮಾಡಬೇಕಾದ ಕೆಲಸ. ಆದರೆ ರಾಜಪುಷಿಯಾದರೋ ತನ್ನ ಪಾಲಿಗೆ ಸಂಬಂಧಪಟ್ಟ ಲೌಕಿಕ ಕೆಲಸಗಳನ್ನೆಲ್ಲಾ ಮಾಡಬೇಕಾಗಿತ್ತು. ಸಿಂಹಾಸನದ ಮೇಲೆ ಕುಳಿತು ಆಳಬೇಕು, ಯುದ್ಧರಂಗದಲ್ಲಿ ಕಾದಾಡಬೇಕು, ಜನರ ದೂರುಗಳನ್ನು ಕೇಳಬೇಕು. ಆದರೂ ಅವನು ನಿಸ್ಸಂಗನಾಗಿರಬೇಕು. ಎಲ್ಲಾ ಕೆಲಸಗಳನ್ನೂ ಮಾಡಬೇಕು. ಬೆಳಗಿನಿಂದ ಸಾಯಂಕಾಲದವರೆಗೆ ದುಡಿಯಬೇಕು. ಆದರೆ ಬರುವ ಫಲಕ್ಕಾಗಿ ಕೈಯೊಡ್ಡಕೂಡದು. ಜೀವನದಲ್ಲಿ ಏನಾದರೂ ತಳಮಳಿಸಕೂಡದು. ಕೆಲಸವನ್ನು ಮಾಡಬೇಕು. ಸಾಕ್ಷಿಯ ಮನೋಭಾವವನ್ನು ರೂಢಿಸಬೇಕು. ಇವರೇ ರಾಜರ್ಷಿಗಳು. ಕಾಲಕ್ರಮೇಣ ಈ ಯೋಗ ಇಲ್ಲಿ ನಷ್ಟವಾಯಿತು ಎನ್ನುತ್ತಾನೆ. ಈ ಜ್ಞಾನ ನಷ್ಟವಾಗಬೇಕಾದರೆ ಅದನ್ನು ತಿಳಿದುಕೊಳ್ಳುವವರ ಅಭಾವ ಇರಬೇಕು ಅಥವಾ ಅದನ್ನು ತಪ್ಪಾಗಿ ತಿಳಿದುಕೊಂಡು ದುರುಪಯೋಗ ಪಡಿಸಿ ಆ ತತ್ತ್ವಕ್ಕೇ ಒಂದು ಕೆಟ್ಟ ಹೆಸರನ್ನು ತಂದಿರಬೇಕು. ಇವೆರಡರಿಂದ ಇದು ನಷ್ಟವಾಗುವುದು. ಎಂದರೆ ಸಂಪೂರ್ಣ ನಾಶವಾಗುವುದು ಎಂದಲ್ಲ. ಮಿಶ್ರವಾಗಿ ಹೋಗುವುದು. ಗಂಗಾನದಿ ಹಿಮಾ ಲಯದಲ್ಲಿ ಹರಿಯುವಾಗ ಎಂತಹ ತಿಳಿನೀರು ಇರುವುದು ಅಲ್ಲಿ! ಅದೇ ನದಿ ತಗ್ಗುಭೂಮಿಗೆ ಬಿದ್ದು ಬೈಲು ಪ್ರದೇಶದಲ್ಲಿ ಹರಿದುಕೊಂಡು ಹೋಗುವಾಗ ಜನ ಅದರ ನೀರನ್ನು ಉಪಯೋಗಿಸಿಕೊಂಡು ಊರಿನ ಕಶ್ಮಲಗಳನ್ನೆಲ್ಲಾ ಅದಕ್ಕೆ ಬಿಡುವರು. ಹಾಗೆಯೇ ಹಲವು ಊರುಗಳ ಮೂಲಕ ಹರಿದು ಬರುವಾಗ ಮೂಲ ನೀರಿನಲ್ಲಿ ಈ ಕಶ್ಮಲಗಳೆಲ್ಲಾ ಸೇರಿಕೊಂಡು, ಮೂಲ ನೀರಿನ ಗುಣವನ್ನೇ ಹಾಳುಮಾಡಿದಂತೆ ತೋರುವುದು. ಅದರಂತೆ ಒಬ್ಬ ಯಾವಾಗ ತತ್ತ್ವವನ್ನು ತಪ್ಪಾಗಿ ತಿಳಿದುಕೊಳ್ಳುತ್ತಾನೋ, ಅದರಿಂದ ಅವನು ಕೆಡುವುದು ಮಾತ್ರವಲ್ಲ, ಎಲ್ಲರನ್ನೂ ಕೆಡಿಸುತ್ತಾನೆ. ತತ್ತ್ವಕ್ಕೆ ಒಂದು ಕೆಟ್ಟ ಹೆಸರನ್ನು ತರುತ್ತಾನೆ. ಅನೇಕ ವೇಳೆ ಗಹನವಾದ ತತ್ತ್ವಗಳು ಹಾಳಾಗಿರುವುದು ಹೀಗೆ. ಅದ್ವೈತ ಎಲ್ಲಾ ಬ್ರಹ್ಮಮಯ ಎಂದು ಸಾರುವುದು. ಆ ಅನುಭವದ ಶಿಖರಕ್ಕೆ ಹೋಗಬೇಕಾದರೆ ಎಂತಹ ಸಂಯಮ ಇರಬೇಕು, ಎಂತಹ ಶಿಸ್ತು ಇರಬೇಕು! ಆದರೆ ಸಾಧನ ಚತುಷ್ಟಯ ಯಾವುದೂ ಇಲ್ಲದೆ ಚಿತ್ತವನ್ನು ಶುದ್ಧಿಮಾಡದೆ, ಆಸೆಯನ್ನು ನಿಗ್ರಹಿಸದೆ, ಅದನ್ನು ತಿಳಿದುಕೊಳ್ಳುವುದಕ್ಕೆ ಪ್ರಯತ್ನಿಸಿದರೆ ನಮ್ಮ ಸ್ವಾರ್ಥ ಮತ್ತು ಪ್ರಾಪಂಚಿಕತೆಯನ್ನೇ ಆ ಗಹನ ತತ್ತ್ವದ ಮೂಲಕ ತೃಪ್ತಿಪಡಿಸಿಕೊಳ್ಳುತ್ತೇವೆ. ಆಗ ಆ ತತ್ತ್ವಕ್ಕೆ ಒಂದು ಕೆಟ್ಟ ಹೆಸರು. ಶ‍್ರೀರಾಮಕೃಷ್ಣರು ದಕ್ಷಿಣೇಶ್ವರದಲ್ಲಿದ್ದಾಗ ಅಲ್ಲಿಗೆ ಒಬ್ಬ ದೊಡ್ಡ ಅದ್ವೈತ ಸಾಧು ಬಂದ. ಅವನು ಎಲ್ಲಾ ಮಿಥ್ಯ ಎಂದು ಸಾರುವ ಗುಂಪಿಗೆ ಸೇರಿದವನು. ಆದರೆ ಅವನ ಶೀಲ ಚೆನ್ನಾಗಿರಲಿಲ್ಲ. ಒಂದು ದಿನ ಶ‍್ರೀರಾಮಕೃಷ್ಣರು, “ಜನ ನಿನ್ನ ಶೀಲದ ವಿಚಾರವಾಗಿ ಏನೇನೋ ಹೇಳುತ್ತಿರುವರಲ್ಲ, ಅದು ನಿಜವೆ?”ಎಂದು ಕೇಳುತ್ತಾರೆ. ಆಗ ಆ ವೇದಾಂತಿ ಈ ಪ್ರಪಂಚವೇ ಒಂದು ಭ್ರಾಂತಿ, ಇದರಲ್ಲಿ ನಾನು ಮಾಡುವ ಕೆಲಸ ಮಾತ್ರ ಹೇಗೆ ನಿಜವಾಗುವುದು? ಎನ್ನುತ್ತಾನೆ. ಆಗ ಶ‍್ರೀರಾಮಕೃಷ್ಣರು ಇಂತಹ ವೇದಾಂತಕ್ಕೆ ನಾನು ಉಗುಳುತ್ತೇನೆ, ಎನ್ನುತ್ತಾರೆ. ಎಲ್ಲಾ ಮಿಥ್ಯ ಎನ್ನುವವನು ಕಾಮಕಾಂಚನದ ಕೆಸರಿಗೆ ಇಳಿದು ಅದರಲ್ಲಿ ವಿಹರಿಸುವುದಕ್ಕೆ ಹೇಗೆ ಸಾಧ್ಯ? ಒಂದಿದ್ದರೆ ಮತ್ತೊಂದಿರಲಾರದು. ಒಂದು ತತ್ತ್ವಕ್ಕೆ ಕೆಟ್ಟ ಹೆಸರನ್ನು ತರುವವರೇ ಅದನ್ನು ತಪ್ಪಾಗಿ ತಿಳಿದುಕೊಂಡವರು.

ಜೀವನದಲ್ಲಿ ಅದ್ವೈತ ಒಂದೇ ಅಲ್ಲ, ಭಕ್ತಿಯಾಗಲಿ, ಕರ್ಮವಾಗಲಿ, ಧ್ಯಾನವಾಗಲಿ ಎಲ್ಲವನ್ನೂ ಮನುಷ್ಯ ತಪ್ಪಾಗಿ ತಿಳಿದುಕೊಂಡು ತಾನು ಹಾಳಾಗುವುದು ಮಾತ್ರವಲ್ಲ, ತನ್ನ ಜೊತೆಗೆ ಇತರರನ್ನೂ ಸೆಳೆದುಕೊಳ್ಳುವನು.

ಯಾವಾಗ ತತ್ತ್ವ ಹಾಳಾಗುವುದೋ ಆಗ, ಅಂತಹ ತತ್ತ್ವವನ್ನು ಪುನಃ ಪರಿಶುದ್ಧವಾದ ರೂಪಿನಲ್ಲಿ ಭಗವಂತ ಜನರ ಮುಂದೆ ಇಡಬೇಕಾಗಿದೆ. ಆ ತತ್ತ್ವಕ್ಕೆ ತನ್ನ ತಪಃಶಕ್ತಿಯನ್ನು ಧಾರೆಯೆರೆಯಬೇಕು. ಪುನಃ ಅದಕ್ಕೆ ಪ್ರಾಣಪ್ರತಿಷ್ಠೆ ಮಾಡಿ ಸ್ಪಂದಿಸುವಂತೆ ಮಾಡಬೇಕು. ಇದು ಸೃಷ್ಟಿಕರ್ತನ ಒಂದು ಜವಾಬ್ದಾರಿ. ಅದನ್ನು ಅವನು ಹಲವು ರೀತಿ ಮಾಡುವನು. ತಾನೇ ಧರೆಗೆ ಇಳಿದು ಬಂದು, ಅದರಂತೆ ಬಾಳಿ ಬದುಕಿ ಒಂದು ಆದರ್ಶವನ್ನು ತೋರುತ್ತಾನೆ. ಅಥವಾ ಜೀವನದಲ್ಲಿ ಮುಂದುವರಿದ ಕೆಲವು ವ್ಯಕ್ತಿಗಳನ್ನು ನಿಮಿತ್ತ ಮಾಡಿಕೊಂಡು ಪ್ರಪಂಚಕ್ಕೆ ಆ ಸಂದೇಶವನ್ನು ಸಾರುತ್ತಾನೆ.

\begin{shloka}
ಸ ಏವಾಯಂ ಮಯಾ ತೇಽದ್ಯ ಯೋಗಃ ಪ್ರೋಕ್ತಃ ಪುರಾತನಃ~।\\ಭಕ್ತೋಽಸಿ ಮೇ ಸಖಾ ಚೇತಿ ರಹಸ್ಯಂ ಹ್ಯೇತದುತ್ತಮಮ್ \hfill॥ ೩~॥
\end{shloka}

\begin{artha}
ನೀನು ನನಗೆ ಭಕ್ತನೂ ಮತ್ತು ಸ್ನೇಹಿತನೂ ಆಗಿರುವುದರಿಂದ ಈ ಪುರಾತನವಾದ ಯೋಗವನ್ನೇ ನಿನಗೆ ಹೇಳಿದೆ. ಏಕೆಂದರೆ ಇದು ಉತ್ತಮವಾದ ರಹಸ್ಯ.
\end{artha}

ಶ‍್ರೀಕೃಷ್ಣ ಈ ಪುರಾತನವಾದ ಯೋಗವನ್ನೇ ನಿನಗೆ ಪುನಃ ಹೇಳಿದೆ ಎನ್ನುವನು. ಇದೇನೂ ಈಗತಾನೇ ಕಂಡುಹಿಡಿದುದು ಅಲ್ಲ. ಅನಾಸಕ್ತಿ ಕರ್ಮದ ಮೂಲಕ ಭಗವಂತನಲ್ಲಿ ಒಂದಾಗುವ ಪಥ ಬಹಳ ಪುರಾತನವಾದುದು. ಹಿಂದೆ ಎಷ್ಟೋ ಜನ ಈ ಮಾರ್ಗದಲ್ಲಿ ನಡೆದು ಉದ್ಧಾರವಾಗಿರುವರು. ಮತ್ತು ಇದರಿಂದ ಜಗತ್ತಿಗೆ ಕಲ್ಯಾಣವೂ ಆಗಿರುವುದು. ಕೆಲವು ವೇಳೆ ಸರಿಯಾಗಿ ಅನುಷ್ಠಾನ ಮಾಡುವವರಿಲ್ಲದೆ ಆ ಜ್ಞಾನ ಪ್ರಪಂಚಕ್ಕೆ ನಷ್ಟವಾಗಿ ಹೋಗುವುದು. ಆಗ ಪುನಃ ಅದನ್ನು ಬಳಸುವುದು ಹೇಗೆ ಎಂಬುದನ್ನು ಮತ್ತೊಮ್ಮೆ ಜನರಿಗೆ ತಮ್ಮ ಸಾಧನೆ ಮತ್ತು ಪ್ರಯೋಗದ ಮೂಲಕ ತೋರಿದಾಗ ಮಾತ್ರ ಅದು ಜನರಿಗೆ ಅರ್ಥವಾಗುವುದು. ಇದನ್ನೇ ಶ‍್ರೀಕೃಷ್ಣ ಮಾಡುವುದು. ಇಂಗ್ಲೀಷಿನಲ್ಲಿ \enginline{putting the old wine in the new bottle} ಎಂದು ಹೇಳುತ್ತಾರೆ. ಇಲ್ಲಿ ಮಾಡುತ್ತಿರುವುದಾದರೂ ಅದನ್ನೆ. ನಮ್ಮ ಭರತಖಂಡದಲ್ಲಿ ಇರುವ ಜನರ ಮನೋಧರ್ಮ, ಅಧ್ಯಾತ್ಮ ಮತ್ತು ಧಾರ್ಮಿಕ ಜೀವನದಲ್ಲಿ ಏನು ಹೊಸದಾಗಿ ಬಂದರೂ ಅದನ್ನು ತುಂಬಾ ಅನುಮಾನದಿಂದ ನೋಡುವುದು. ಹೊಸದಾಗಿ ಏನು ಬಂದರೂ ಅದಕ್ಕೆ ಸೋಲುವ ಮನೋಧರ್ಮದವರು ಒಂದು ಬಗೆಯ ಜನ. ಅದರಂತೆಯೇ ಹೊಸದಾಗಿ ಏನು ಬಂದರೂ ಅದನ್ನು ಸ್ವಲ್ಪ ಅಂಜಿಕೆಯಿಂದ ನೋಡುವವರು ಕೆಲವರು. ಅದರಲ್ಲಿಯೂ ಧಾರ್ಮಿಕ ಜೀವನದಲ್ಲಂತೂ ನಮ್ಮವರ ಅಭಿಪ್ರಾಯ ಪೂರ್ವದಿಂದ ಬಂದಿದ್ದರೆ ಸ್ವೀಕರಿಸಬಹುದು ಎಂಬುದು. ಕಾಲ ಅದನ್ನು ತೂರಿ ಗಟ್ಟಿ ಎಂದು ನಿರ್ಧರಿಸಿದೆ. ಈಗತಾನೆ ಬಂದಿರುವುದರಲ್ಲಿ ಜೊಳ್ಳು ಎಷ್ಟು, ಕಾಳು ಎಷ್ಟು ಎಂದು ಇನ್ನೂ ನಿರ್ಧರಿಸಿಲ್ಲ. ಅದನ್ನು ಸುಮ್ಮನೆ ಅನುಸರಿಸಿದರೆ, ಹಿಂದಿನ ಗಟ್ಟಿಯನ್ನು ಬಿಟ್ಟು ಈಗಿನ ಜೊಳ್ಳನ್ನು ಆಯ್ದುಕೊಂಡಂತೆ ಆಗಬಹುದು ಎಂದು ಅಂಜುವರು. ಆದಕಾರಣವೇ ಯಾರು ದೊಡ್ಡ ಮೇಧಾವಿಗಳೋ ಅವರು ಯಾವುದಾದರೂ ಹೊಸ ವಿಷಯವನ್ನು\break ಸಾರುವಾಗ ಜನರಿಗೆ ನಾನು ಹೇಳುವುದರಲ್ಲಿ ಯಾವುದೂ ಹೊಸದು ಇಲ್ಲ, ಆಗಲೇ ಇದೆಲ್ಲಾ ಹಿಂದೆ ವೇದದಲ್ಲಿದೆ, ಉಪನಿಷತ್ತಿನಲ್ಲಿದೆ, ಹಿಂದೆ ಜೀವನದಲ್ಲಿ ಇದನ್ನು ಬಳಸುತ್ತಿದ್ದರು ಎಂದರೆ ಧೈರ್ಯದಿಂದ ಇದನ್ನು ಸ್ವೀಕರಿಸಲು ಮುಂದೆ ಬರುವರು. ಶ‍್ರೀಕೃಷ್ಣ ಕೂಡ ಈ ಮರ್ಮವನ್ನು ಚೆನ್ನಾಗಿ ಅರಿತವನು. ಆದಕಾರಣವೇ ಇದನ್ನು ಬಹಳ ಪುರಾತನವಾದ ಧರ್ಮ ಎಂದು ಸಾರುವನು.

ಶ‍್ರೀಕೃಷ್ಣ ಅರ್ಜುನನಿಗೆ ಹೇಳುವುದಕ್ಕೆ ಕಾರಣವನ್ನು ಕೊಡುವನು. ಅವನು ಭಕ್ತನಾಗಿದ್ದಾನೆ ಮತ್ತು ಸ್ನೇಹಿತನಾಗಿದ್ದಾನೆ. ಇದೇ ಆ ಕಾರಣ. ಇದಕ್ಕೆ ಮುಂಚೆ ಶ‍್ರೀಕೃಷ್ಣ ಅರ್ಜುನನಿಗೆ ಈ ವಿಷಯವನ್ನು ಹೇಳಿರಲಿಲ್ಲ. ಏಕೆಂದರೆ ಅದಕ್ಕೆ ಸಮಯ ಬಂದಿರಲಿಲ್ಲ. ಜೀವನದಲ್ಲಿ ಎಂತಹ ಆಧ್ಯಾತ್ಮಿಕ ತತ್ತ್ವವನ್ನು ತಿಳಿದುಕೊಳ್ಳಬೇಕಾದರೂ ಅದಕ್ಕೆ ಸರಿಯಾದ ಸಮಯ ಒದಗಿ ಬರಬೇಕು. ಅದಕ್ಕೆ ಮುಂಚೆ ಹೇಳಿದರೆ ಅವನು ಅದನ್ನು ಲಕ್ಷಿಸುವುದಿಲ್ಲ. ಏಕೆಂದರೆ ಅವನಿಗೆ ಅದು ಬೇಕಾಗಿರುವುದಿಲ್ಲ. ಯಾವಾಗ ಹಸಿವಾಗಿರುವುದೋ ಆಗ ಆಹಾರವನ್ನು ಕೊಡಬೇಕಾಗಿದೆ. ಆ ಸಮಯ ಈಗ ಅರ್ಜುನನಿಗೆ ಪ್ರಾಪ್ತವಾಗಿದೆ. ಅರ್ಜುನ ಈಗ ಹೊಸದಾಗಿ ಶ‍್ರೀಕೃಷ್ಣನಿಗೆ ಶಿಷ್ಯನಾಗಿದ್ದಾನೆ. ಯಾವಾಗ ಒಬ್ಬ ನಾನು ನಿನ್ನ ಭಕ್ತ, ನಿನ್ನ ಶಿಷ್ಯ, ನಿನ್ನನ್ನೇ ನಂಬಿದವನು, ನೀನು ಉದ್ಧಾರ ಮಾಡಬೇಕು ಎಂದು ಸಂಪೂರ್ಣ ನಂಬುವನೋ ಆಗ ಅವನನ್ನು ಉದ್ಧರಿಸುವುದಕ್ಕೆ ಭಗವಂತನ ಕರಗಳು ಮೇಲಕ್ಕೆ ಏಳುವುವು. ಅವನಿಗೆ ಗೊತ್ತಾಗಬೇಕು, ನಾವು ಅವನನ್ನು ನೆಚ್ಚಿದವರು, ಅವನಿಗಾಗಿ ಎಲ್ಲವನ್ನೂ ಬಿಟ್ಟು ನಿಂತಿರುವವರು ಎಂದು. ಆಗ ಆ ಭಕ್ತನಿಗೆ ಗೊತ್ತಾಗುವುದು ಭಗವಂತ ತನ್ನ ಶಿಷ್ಯನ ಉದ್ಧಾರಕ್ಕೆ ಏನನ್ನು ಬೇಕಾದರೂ ಮಾಡಲು ಸಿದ್ಧನಾಗಿರುವನು ಎಂಬುದು. ಇಲ್ಲದೇ ಇದ್ದರೆ ಇದು ಕಿವುಡನ ಮುಂದೆ ಕಿಂದರಿಯನ್ನು ಬಾರಿಸಿದಂತೆ. ಶ‍್ರೀಕೃಷ್ಣ ಇನ್ನೊಂದು ಗುಣವಾಚಕವನ್ನು ಹೇಳುವನು. ನೀನು ನನ್ನ ಸ್ನೇಹಿತ ಬೇರೆ ಆಗಿರುವೆ ಎಂದು. ಯಾವಾಗ ನಾನು ಒಬ್ಬನಿಗೆ ಪರಮಾಪ್ತ ಸ್ನೇಹಿತನಾಗುವೆನೋ ಆಗ ಅವನನ್ನು ತಪ್ಪುದಾರಿಗೆ ಎಳೆಯಲಾರೆ–ಅವನಿಗೆ ಶ್ರೇಯಸ್ಸನ್ನುಂಟುಮಾಡಲು ಯತ್ನಿಸುವೆನು. ನಂಬಿದ ಸ್ನೇಹಿತನಿಗೆ ದ್ರೋಹವನ್ನು ಬಗೆದರೆ ಅದು ಸ್ನೇಹವೇ ಅಲ್ಲ. ಒಬ್ಬ ಸ್ನೇಹಿತ ಮತ್ತೊಬ್ಬನಿಗೆ ಸರ್ವೋತ್ಕೃಷ್ಟವಾದುದನ್ನು ಬಯಸುವನು. ಅದಕ್ಕಾಗಿಯೇ ಇದನ್ನು ಅರ್ಜುನನಿಗೆ ಹೇಳುವನು.

ಗುರುವಾದವನು ಜ್ಞಾನವನ್ನು ರಕ್ಷಣೆ ಮಾಡಬೇಕು ಮತ್ತು ಅದನ್ನು ಪ್ರಚಾರ ಮಾಡಬೇಕು. ರಕ್ಷಣೆ ಮಾಡಬೇಕಾದರೆ, ಅದನ್ನು ತಾನು ಮುಂಚೆ ಚೆನ್ನಾಗಿ ತಿಳಿದುಕೊಂಡಿರಬೇಕು. ಅದನ್ನು ಅನುಭವಿ ಸಿರಬೇಕು. ಅದನ್ನು ಅಯೋಗ್ಯರಿಗೆ ನಿಲುಕದಂತೆ ಮಾಡಬೇಕು. ಯಾರು ಸರಿಯಾಗಿ ಅವನು ಹೇಳುವುದನ್ನು ಅರ್ಥ ಮಾಡಿಕೊಳ್ಳಬಲ್ಲರು ಎಂದು ಗೊತ್ತಾಗುವುದೋ ಅಂತಹವರಿಗೆ ಮಾತ್ರ ಹೇಳಬೇಕು. ಏಕೆಂದರೆ ಇದೊಂದು ದೊಡ್ಡ ಜವಾಬ್ದಾರಿ. ಯಾರಿಗೆ ಒಂದು ಹರಿತವಾದ ಶಸ್ತ್ರವನ್ನು ಹೇಗೆ ಉಪಯೋಗಿಸಬೇಕು ಎಂದು ಗೊತ್ತಿಲ್ಲವೋ ಅವರಿಗೆ ಕೊಟ್ಟರೆ ಬಹಳ ಅಪಾಯವಿದೆ. ಅವನು ತನ್ನನ್ನೇ ಹಾಳುಮಾಡಿಕೊಳ್ಳುವನು ಆ ಶಸ್ತ್ರದಿಂದ. ಆಗ ಯಾರು ಇದನ್ನು ಕೊಟ್ಟನೋ ಅವನಿಗೂ ಆ ಪಾಪ ತಗಲುವುದು. ಬಹಳ ಸೂಕ್ತವಾದ ಆಧ್ಯಾತ್ಮಿಕತತ್ತ್ವಗಳು\break ಹರಿತವಾದ ಶಸ್ತ್ರಕ್ಕಿಂತ ಅಪಾಯಕಾರಿ. ನಾವು ಯಾವಾಗ ಅದನ್ನು ಸರಿಯಾಗಿ ಅರ್ಥಮಾಡಿ\-ಕೊಳ್ಳುವು\-ದಿಲ್ಲವೋ, ತಪ್ಪು ಭಾವಿಸುವೆವೋ ಅದರಿಂದ ನಾವು ಹಾಳಾಗುವುದಲ್ಲದೆ ಅದನ್ನು ಇತರರಿಗೆ ಬೋಧಿಸಿ ಅವರನ್ನೂ ಹಾಳುಮಾಡುವೆವು. ಇದು ಕುರುಡ ಇನ್ನೊಬ್ಬ ಕುರುಡನಿಗೆ ದಾರಿ ತೋರಿದಂತೆ ಆಗುವುದು.

ಆದರೆ ಬರೀ ಘಟಸರ್ಪ ನಿಧಿಯನ್ನು ಕಾಯುವಂತೆ ತತ್ತ್ವವನ್ನು ಯಾರಿಗೂ ಕೊಡದೆ ರಕ್ಷಿಸಿದರೂ ಅದರಿಂದ ಏನೂ ಪ್ರಯೋಜನವಾಗುವುದಿಲ್ಲ. ಅದು ಜನರಲ್ಲಿ ಬಳಕೆಗೆ ಬರಬೇಕು. ಆಗಲೆ ಅದು ಉದ್ಧಾರದ ಶಕ್ತಿಯಾಗಬೇಕಾದರೆ. ಹಾಗೆ ಬಳಕೆಗೆ ಬರಬೇಕಾದರೆ ಯೋಗ್ಯ ವ್ಯಕ್ತಿಗಳನ್ನು ಆರಿಸಿಕೊಂಡು ಅವರಿಗೆ ನಮಗೆ ತಿಳಿದಿರುವುದನ್ನು ಹೇಳಬೇಕು. ನಾನು ಕಣ್ಣು ಮುಚ್ಚಿಕೊಳ್ಳುವುದಕ್ಕೆ ಮುಂಚೆ ನನ್ನ ಅನುಭವ ಕೆಲವು ಜೀವಿಗಳ ಎದೆಯಲ್ಲಿ ಬೇರು ಬಿಟ್ಟಿರಬೇಕು. ಆಗ ಅದು ಒಬ್ಬರಿಂದ ಮತ್ತೊಬ್ಬರಿಗೆ ಹಬ್ಬುತ್ತಾ ಹೋಗುವುದು. ಶ‍್ರೀಕೃಷ್ಣ ತನ್ನ ಬೋಧನೆಯನ್ನು ಜಗತ್ತಿಗೆ ಹರಡುವುದಕ್ಕೆ ಅರ್ಜುನನನ್ನು ಆರಿಸಿಕೊಳ್ಳುವನು. ತನಗೆ ಆ ವಿಷಯದಲ್ಲಿ ಏನು ತಿಳಿದಿರುವುದೋ ಅದನ್ನೆಲ್ಲಾ ಹೇಳುವನು. ಏಕೆಂದರೆ ಅವನಿಗೆ ಗೊತ್ತಿದೆ, ಅವನ ಮೂಲಕ ಅದು ವಿಶ್ವಕ್ಕೆ ಪ್ರಚಾರವಾಗುವುದು ಎಂದು.

ಇದನ್ನು ಒಂದು ಶ್ರೇಷ್ಠವಾದ ರಹಸ್ಯ ಎನ್ನುವನು. ನಮ್ಮಲ್ಲಿ ಅನೇಕ ವೇಳೆ ರಹಸ್ಯ ಎನ್ನುವುದಕ್ಕೆ ಬೇರೆ ಅರ್ಥ ಬಂದುಹೋಗಿದೆ. ರಹಸ್ಯ ಎಂದರೆ ಅದನ್ನು ಬೇರೆ ಯಾರಿಗೂ ಹೇಳಕೂಡದು, ಹೇಳಿದರೆ ಅದರ ಪ್ರಯೋಜನವೆಲ್ಲ ನಾಶವಾಗುವುದು ಎಂಬ ತಪ್ಪು ಅರ್ಥ ಬಂದಿದೆ. ಆದರೆ ಇದಲ್ಲ ಅದರ ಧ್ವನಿ. ಇದು ಬಹಳ ಪವಿತ್ರ. ಪವಿತ್ರವಾಗಿರುವುದನ್ನು ನಾವು ಎಲ್ಲರೆದುರಿಗೆ ಯಾವಾಗ ಎಂದರೆ ಆಗ ತೋರಿಸುವುದಿಲ್ಲ. ಯಾರು ನಿಜವಾಗಿ ಇದನ್ನು ಅರ್ಥ ಮಾಡಿಕೊಂಡು ಪ್ರಯೋಜನ ಪಡೆಯ ಬಲ್ಲರೋ, ಅದಕ್ಕಾಗಿ ತೊಂದರೆ ತೆಗೆದುಕೊಳ್ಳಲು ಸಿದ್ಧರಾಗಿರುವರೋ ಅವರಿಗೆ ಮಾತ್ರ ಇದನ್ನು ಹೇಳಬೇಕು. ಇದು ನಮ್ಮ ದೇವರಮನೆಯಲ್ಲಿ ಪೂಜೆಮಾಡುವ ದೇವರಂತೆ. ಅದು ಡ್ರಾಯಿಂಗ್ ರೂಮಿನಲ್ಲಿ ಎಲ್ಲರ ಕಣ್ಣಿಗೆ ಬೀಳುವಂತಹ ಗೋಡೆಗೆ ತಗುಲಿಹಾಕಿರುವ ಪಟಗಳಂತೆ ಅಲ್ಲ.

ಅರ್ಜುನ ಶ‍್ರೀಕೃಷ್ಣನನ್ನು ಹೀಗೆ ಕೇಳುತ್ತಾನೆ:

\begin{shloka}
ಅಪರಂ ಭವತೋ ಜನ್ಮ ಪರಂ ಜನ್ಮ ವಿವಸ್ವತಃ~।\\ಕಥಮೇತದ್ವಿಜಾನೀಯಾಂ ತ್ವಮಾದೌ ಪ್ರೋಕ್ತವಾನಿತಿ \hfill॥ ೪~॥
\end{shloka}

\begin{artha}
ನಿನ್ನ ಜನ್ಮ ಈಚಿನದು. ಸೂರ್ಯನ ಜನ್ಮ ಹಿಂದಿನದು. ನೀನು ಇದನ್ನು ಮೊದಲೇ ಉಪದೇಶಮಾಡಿದೆ ಎಂಬುದನ್ನು ಹೇಗೆ ತಿಳಿದುಕೊಳ್ಳಲಿ?
\end{artha}

ಅರ್ಜುನ ಇಲ್ಲಿ ಕೇವಲ ಮಿತಿಯಿಂದ ಕೂಡಿದ ಮಾನವನ ದೃಷ್ಟಿಯಿಂದ ಮಾತನಾಡುತ್ತಿರು\-ವನು. ಸಾಧಾರಣ ಮನುಷ್ಯನಿಗೆ ಹುಟ್ಟುವುದಕ್ಕೆ ಹಿಂದೆ ಗೊತ್ತಿಲ್ಲ. ಜನನ ಮರಣಗಳ ಮಧ್ಯದಲ್ಲಿ\break ಇರುವ ಕೆಲವು ಕಾಲದ ಬಾಳುವೆಯೆ ಸರ್ವಸ್ವ ಎಂದು ಅಜ್ಞಾನಿ ಭಾವಿಸುವನು. ಒಂದೇ ಒಂದು ಜನ್ಮದ ಹೊಸಲಿನ ಮೇಲೆ ನಿಂತಿರುವ ಅರ್ಜುನನಿಗೆ ಶ‍್ರೀಕೃಷ್ಣ ತಾನು ಬಹಳ ಹಿಂದೆ ಹಲವು ಜನ್ಮಗಳ ಹಿಂದೆ, ಸೂರ್ಯ, ಇಕ್ಷ್ವಾಕು, ಮನು ಮುಂತಾದವರಿಗೆ ತಿಳಿಸಿದೆ ಎಂದು ಹೇಳಿದಾಗ ದಿಗ್ಭ್ರಾಂತಿ ಹಿಡಿಯುವುದು. ಇದು ಮಗು ಅಪ್ಪನಿಗೆ ನಾಮಕರಣ ಮಾಡಿದಾಗ ನಾನಿದ್ದೆ ಎಂದರೆ ಎಷ್ಟು ಹಾಸ್ಯಾಸ್ಪದವಾಗುವುದೋ ಹಾಗಿದೆ. ಆದಕಾರಣವೇ ಅರ್ಜುನನ ಕಣ್ಣೆದುರಿಗೆ ಶ‍್ರೀಕೃಷ್ಣ ಇರುವನು. ಇವತ್ತೆಲ್ಲ ವಯಸ್ಸಿನಲ್ಲಿ ಕೆಲವು ವರ್ಷಗಳು ಅರ್ಜುನನಿಗಿಂತ ಹಿರಿಯನಿರಬಹುದು ಅಷ್ಟೆ. ಪೂರ್ವಯುಗದಲ್ಲಿದ್ದವರಿಗೆ ಹೇಗೆ ಗುರುವಾಗಬಹುದು ಎಂದುಕೊಳ್ಳುತ್ತಾನೆ. ಆಗಲೆ ಶ‍್ರೀಕೃಷ್ಣ ಪುನರ್ಜನ್ಮದ ತತ್ತ್ವವನ್ನು ಅರ್ಜುನನಿಗೆ ಹೇಳುತ್ತಾನೆ.

\begin{shloka}
ಬಹೂನಿ ಮೇ ವ್ಯತೀತಾನಿ ಜನ್ಮಾನಿ ತವ ಚಾರ್ಜುನ~।\\ತಾನ್ಯಹಂ ವೇದ ಸರ್ವಾಣಿ ನ ತ್ವಂ ವೇತ್ಥ ಪರಂತಪ \hfill॥ ೫~॥
\end{shloka}

\begin{artha}
ಅರ್ಜುನ, ನನಗೂ ನಿನಗೂ ಹಲವು ಜನ್ಮಗಳು ಆಗಿವೆ. ಅವುಗಳೆಲ್ಲವನ್ನೂ ನಾನು ಬಲ್ಲೆ, ಆದರೆ ಪರಂತಪ, ನೀನು ತಿಳಿಯಲಾರೆ.
\end{artha}

ಹಲವು ವೇಳೆ ಶ‍್ರೀಕೃಷ್ಣ ಸಾಕ್ಷಾತ್ ಭಗವಂತನಾದರೂ ಈ ಪ್ರಪಂಚಕ್ಕೆ, ಜನಗಳಿಗೆ ಜ್ಞಾನವನ್ನು ಬೋಧಿಸಲು ಇಳಿದುಬಂದಿರುವನು. ಹಾಗೆ ಬಂದಾಗ ಯಾವುದೋ ದೇಶದಲ್ಲಿ, ಯಾವುದೋ ಹೆಸರಿನಿಂದ ಆ ಕೆಲಸವನ್ನು ಮಾಡಿ ಹೋಗಿರುವನು. ಆದರೆ ಅರ್ಜುನನಾದರೋ ಇನ್ನೂ ಬದ್ಧಜೀವಿ. ಅವನು ತನ್ನ ಸಂಸ್ಕಾರಗಳ ಬಲಾತ್ಕಾರದಿಂದ ಈ ಪ್ರಪಂಚಕ್ಕೆ ಬಂದು ಹಲವು ಕರ್ಮಗಳನ್ನು ಮಾಡಿ ಪಾಪ ಪುಣ್ಯಗಳನ್ನು ಅನುಭವಿಸಿ ಹೋಗಿರುವನು. ಅಜ್ಞಾನಿಯಂತೆ ಅರ್ಜುನ ವಿಧಿವಶದಿಂದ ಇಳಿದು ಬಂದಿರುವನು. ಆದರೆ ಶ‍್ರೀಕೃಷ್ಣನಾದರೋ ಸುಜ್ಞಾನಿಯಂತೆ ಬಂದಿರುವನು.

ಶ‍್ರೀಕೃಷ್ಣನಿಗೆ ತನ್ನ ಹಿಂದಿನ ಜನ್ಮಗಳೆಲ್ಲ ಗೊತ್ತಿದೆ. ಆದರೆ ಅರ್ಜುನನಿಗೆ ಗೊತ್ತಿಲ್ಲ. ಶ‍್ರೀಕೃಷ್ಣ ಅವತಾರ, ಅಜ್ಞಾನ ಲವಲೇಶವೂ ಅವನಲ್ಲಿಲ್ಲ. ಭೂತಭವಿಷ್ಯತ್ತುಗಳನ್ನೆಲ್ಲಾ ನೋಡಬಲ್ಲ ಶಕ್ತಿ ಶ‍್ರೀಕೃಷ್ಣನಿಗೆ ಇದೆ. ಪತಂಜಲಿ ರಾಜಯೋಗಸೂತ್ರದಲ್ಲಿ ಸಾಧಾರಣ ಸಿದ್ಧಪುರುಷನಿಗೆ\break ಯಾವಾಗ ತನ್ನ ಸಂಸ್ಕಾರದ ಮೇಲೆ ಮನಸ್ಸನ್ನು ಕೇಂದ್ರೀಕರಿಸುವುದು ಸಾಧ್ಯವಾಗುವುದೋ ಆಗ ಹಿಂದಿನ ಜನ್ಮಗಳ ನೆನಪು ಅವನಿಗೆ ಬರುವುದು, ಅಂತಹವರನ್ನು ಜಾತಿಸ್ಮರ ಎನ್ನುತ್ತಾರೆ ಎನ್ನುವನು. ಹೀಗಿರುವಾಗ, ಶ‍್ರೀಕೃಷ್ಣನಂತಹ ಅವತಾರ ವ್ಯಕ್ತಿಗೆ ಇದನ್ನು ತಿಳಿದುಕೊಳ್ಳುವುದಕ್ಕೆ ಅಸಾಧ್ಯವಲ್ಲ. ಅರ್ಜುನನಾದರೋ ಅಜ್ಞಾನದಲ್ಲಿ ತೆವಳುತ್ತಿರುವನು. ಅವನಿಗೆ ತನ್ನ ಕಣ್ಣೆದುರಿಗಿರುವ ಈಗ ಮತ್ತು ಈ ಜನ್ಮದಲ್ಲಿ ನಡೆದುಹೋದ ಕೆಲವು ಮುಖ್ಯವಾದ ಘಟನೆಗಳು ವಿನ ಮತ್ತಾವುದೂ ತೋರದು. ನಮಗೆ ಜ್ಞಾಪಕವಿಲ್ಲ ಎಂದರೆ ಆ ಸತ್ಯ ಇಲ್ಲವೇ ಇಲ್ಲ ಎಂದಲ್ಲ. ದೇವರೇ ಈ ಮರೆವನ್ನು ಬದ್ಧರಾಗಿರುವ ನಮಗೆ ಒಂದು ವರದಂತೆ ಕರುಣಿಸಿರುವನು. ಏನಿಲ್ಲ, ಒಂದು ಜನ್ಮದಲ್ಲಿ ಆಗಿರುವುದನ್ನೆಲ್ಲಾ ನಾವು ಚಾರುಚೂರು ಬಿಡದಂತೆ ಜ್ಞಾಪಿಸಿಕೊಳ್ಳುತ್ತಿದ್ದರೆ ಅದರಿಂದ ಹುಚ್ಚರಾಗಿ ಹೋಗುವ ಸಂಭವ ಇದೆ. ಹೀಗಿರುವಾಗ ಹಿಂದಿನ ಜನ್ಮಗಳ ನೆನಪೆಲ್ಲಾ ಬಂದರೆ ನಾವು ಮುಂದುವರಿ ಯುವಂತಿಲ್ಲ. ಅದರ ಭಾರಕ್ಕೆ ಸೊರಗಿ ಬೀಳುವೆವು. ಅದನ್ನೆಲ್ಲಾ ಎದುರಿಸುವ ಶಕ್ತಿ ನಮ್ಮಲ್ಲಿರುವುದಿಲ್ಲ. ಆದಕಾರಣವೇ ದೇವರು ನಮ್ಮ ಮೇಲೆ ಕೃಪೆಯಿಟ್ಟು ಒಂದು ಜನ್ಮ ತೀರಿತು ಎಂದರೆ, ಆ ವಿವರಗಳೆಲ್ಲ ಮರೆತುಹೋಗುವಂತೆ ಮಾಡುವನು. ಆದರೆ ಅಲ್ಲಿ ಕರ್ಮದ ಮೂಲಕ ನಾವು ಸಂಪಾದನೆ ಮಾಡಿದ ಸಂಸ್ಕಾರಗಳ ಬೀಜದಲ್ಲಿ ಅವುಗಳೆಲ್ಲ ಸುಪ್ತಾವಸ್ಥೆಯಲ್ಲಿ ಇರುವುವು. ಯಾರಿಗಾದರೂ ಅದನ್ನು ಜಾಗ್ರತಗೊಳಿಸಬೇಕೆಂದು ಇಚ್ಛೆಯಿದ್ದರೆ ಆ ಸಂಸ್ಕಾರಗಳ ಮೇಲೆ ಮನಸ್ಸನ್ನು ಏಕಾಗ್ರ ಮಾಡಬೇಕು. ಆಗ ಆ ಬೀಜ ಹಿಂದೆ ಎಂತಹ ವೃಕ್ಷವಾಗಿತ್ತು ಎಂಬುದು ಗೊತ್ತಾಗುವುದು. ಆದರೆ ಅದು ಹಾಗೆ ಎದ್ದಾಗ ಅಂಜದೆ, ಸಾಕ್ಷಿಯಂತೆ ನಿಂತು ನೋಡುವ ಶಕ್ತಿಯನ್ನು ಪಡೆದುಕೊಂಡಿರಬೇಕು. ಇಲ್ಲದೇ ಇದ್ದರೆ ಅದರಿಂದ ಅಪಾಯವಿದೆ, ಸಾಧಾರಣ ಮನುಷ್ಯನಿಗೆ. ಆದರೆ ಶ‍್ರೀಕೃಷ್ಣನಾದರೋ ಈಗ ನಡೆಯುತ್ತಿರುವುದನ್ನು ಸಾಕ್ಷಿಯಂತೆ ನಿಂತು ನೋಡಬಲ್ಲ ಧೈರ್ಯ ಇರುವವನು. ಈ ಸಂಸ್ಕಾರಗಳ ಬಿರುಗಾಳಿ ಮಂದರ ಪರ್ವತವನ್ನು ಅಡಗಿಸಲಾರದು. ಆದರೆ ಸಾಧಾರಣ ಜೀವಿಗಳ ತರಗೆಲೆಯನ್ನು ಕೊಚ್ಚಿಕೊಂಡು ಹೋಗುವುದು.

\begin{shloka}
ಅಜೋಽಪಿ ಸನ್ನವ್ಯಯಾತ್ಮಾ ಭೂತಾನಾಮೀಶ್ವರೋಽಪಿ ಸನ್​~।\\ಪ್ರಕೃತಿಂ ಸ್ವಾಮಧಿಷ್ಠಾಯ ಸಂಭವಾಮ್ಯಾತ್ಮಮಾಯಯಾ \hfill॥ ೬~॥
\end{shloka}

\begin{artha}
ನಾನು ಅಜನು, ಅವ್ಯಯನು ಆಗಿದ್ದರೂ, ಪ್ರಾಣಿಗಳಿಗೆ ಒಡೆಯನಾಗಿದ್ದರೂ ನನ್ನ ಪ್ರಕೃತಿಯನ್ನು ಹಿಡಿತದಲ್ಲಿ ಇಟ್ಟುಕೊಂಡು ನನ್ನ ಮಾಯೆಯ ಮೂಲಕ ಅವತಾರ ಮಾಡುತ್ತೇನೆ.
\end{artha}

ಭಗವಂತ ಅವತಾರದಂತೆ ಇಳಿದು ಬರುವುದಕ್ಕೂ ಅಜ್ಞಾನಿ ಬದ್ಧಜೀವಿ ಪ್ರಪಂಚಕ್ಕೆ ಬರುವು\-ದಕ್ಕೂ ದೊಡ್ಡ ಒಂದು ವ್ಯತ್ಯಾಸವಿದೆ. ಅವನು ಅಜ, ಎಂದರೆ ಎಂದಿಗೂ ಹುಟ್ಟದವನು. ಹುಟ್ಟು ಎಂದರೆ, ಹುಟ್ಟುವುದಕ್ಕೆ ಮುಂಚೆ ಇರಲಿಲ್ಲ ಎಂದು ಭಾಸವಾಗುವುದು. ಶ‍್ರೀಕೃಷ್ಣ ಹಾಗಲ್ಲ, ಅವನು ದೇವಕೀ ವಸುದೇವರ ಪುತ್ರನಾಗಿ ಹುಟ್ಟುವುದಕ್ಕೆ ಮುಂಚೆಯೂ ಇದ್ದವನು. ಶ‍್ರೀಕೃಷ್ಣದೇಹ ಎಂಬುದು ಅವನು ಈಗ ಹಾಕಿಕೊಂಡಿರುವ ವೇಷ ಅಷ್ಟೆ.

ಅವ್ಯಯ ಎಂದರೆ ಯಾವ ಬದಲಾವಣೆಗೂ ತುತ್ತಾಗದವನು. ದೇಹ, ಮನಸ್ಸು, ಬುದ್ಧಿ ಇಂದ್ರಿಯಗಳಿಗೆ ಬದಲಾವಣೆಗಳಿವೆ. ಆದರೆ ಯಾರು ಅದರ ಹಿಂದೆ ಇವುಗಳಿಗೆ ನಿಯಾಮಕನಾಗಿರು ವನೋ ಅವನಿಗೆ ಯಾವ ಬದಲಾವಣೆಯೂ ಇಲ್ಲ. ಅವನು ಹುಟ್ಟುವಾಗಲೆ ಜ್ಞಾನಿಯಾಗಿರುವನು, ಸಿದ್ಧನಾಗಿರುವನು, ಪೂರ್ಣಕಾಮನಾಗಿರುವನು. ಅವನು ಅನಂತರ ಪ್ರಯತ್ನಪಟ್ಟು ಇವುಗಳನ್ನೆಲ್ಲಾ ಪಡೆಯಬೇಕಾಗಿಲ್ಲ.

ಭಗವಂತ ಪ್ರಾಣಿಗಳಿಗೆ ಒಡೆಯ. ಎಲ್ಲಾ ಜೀವರಾಶಿಗಳಿಗೂ ಸ್ವಾಮಿ. ಯಾರೂ ಅವನ ಆಜ್ಞೆಯನ್ನು ಮೀರಿ ಹೋಗಲಾರರು. ಜಡ ಚೇತನ ನಿಯಮಗಳೆಲ್ಲಾ ಅವನ ಇಚ್ಛಾನುಸಾರ ತಮ್ಮ ತಮ್ಮ ಗುಣಧರ್ಮಗಳನ್ನು ಪಡೆದಿವೆ. ಬೆಂಕಿ ತಪಿಸುವುದು ಅವನಿಂದ, ಗಾಳಿ ಬೀಸುವುದು ಅವನಿಂದ, ದಿನ ರಾತ್ರಿಗಳಾಗುವುದು ಅವನಿಂದ. ಎಲ್ಲರಿಗೂ ಕಂಪನಕಾರಿಯಾದ ಮೃತ್ಯು ಕೂಡ ಅವನಿಗೆ ಅಂಜಿ ತನ್ನ ಕೆಲಸವನ್ನು ಮಾಡುವುದು.

ಅವನು ಈ ಪ್ರಪಂಚಕ್ಕೆ ಒಡೆಯ. ಅವನು ಹೇಗೆ ಇಳಿದು ಬರುತ್ತಾನೆ ಎಂಬುದನ್ನು ಹೇಳುತ್ತಾನೆ. ನಾವು ಮೇಲಿನಿಂದ ಕೆಳಗೆ ಹೇಗೆ ಒಂದು ಏಣಿಯ ಸಹಾಯದಿಂದ ಇಳಿದು ಬರುವೆವೋ ಹಾಗೆ ಭಗವಂತ ತನ್ನದೇ ಆದ ತ್ರಿಗುಣಗಳ ಮಾಯೆಯ ಮೂಲಕ ಇಳಿದು ಬರುವನು. ಸಾಧಾರಣ ಜೀವಿ ಮಾಯೆಗೆ ಬದ್ಧ, ಭಗವಂತ ಮಾಯೆಗೆ ಒಡೆಯ. ಸಾಧಾರಣ ಮನುಷ್ಯನನ್ನು ಗುಣಗಳು ಎಳೆದುಕೊಂಡು ಬರುವುವು. ಭಗವಂತನಾದರೊ ಏಣಿಯ ಮೂಲಕ ತಾನೇ ಇಳಿದು ಬರುವನು. ಅವನು ಕರ್ಮಪಾಶದಿಂದ ಬದ್ಧನಾಗಿ ಇಳಿದು ಬರುವುದಿಲ್ಲ. ಅವನು ತಾನೇತಾನಾಗಿ ಇಳಿದು ಬರುವನು. ಹಾಗೆ ಇಳಿದು ಬರುವುದಕ್ಕೆ ಕಾರಣವಾದರೋ ವಿಶ್ವಾನುಕಂಪೆ. ತನಗೆ ಏನೂ ಬೇಕಾಗಿಲ್ಲ. ಇತರರಿಗೆ ಇವನ ಸಹಾಯ ಬೇಕಾದಾಗ ಇಳಿದು ಬರುವನು.

\begin{shloka}
ಯದಾ ಯದಾ ಹಿ ಧರ್ಮಸ್ಯ ಗ್ಲಾನಿರ್ಭವತಿ ಭಾರತ~।\\ಅಭ್ಯುತ್ಥಾನಮಧರ್ಮಸ್ಯ ತದಾತ್ಮಾನಂ ಸೃಜಾಮ್ಯಹಮ್ \hfill॥ ೭~॥
\end{shloka}

\begin{shloka}
ಪರಿತ್ರಾಣಾಯ ಸಾಧೂನಾಂ ವಿನಾಶಾಯ ಚ ದುಷ್ಕೃತಾಮ್~।\\ಧರ್ಮಸಂಸ್ಥಾಪನಾರ್ಥಾಯ ಸಂಭವಾಮಿ ಯುಗೇ ಯುಗೇ \hfill॥ ೮~॥
\end{shloka}

\begin{artha}
ಅರ್ಜುನ, ಯಾವಾಗ ಧರ್ಮಕ್ಕೆ ಗ್ಲಾನಿಯಾಗುವುದೋ, ಅಧರ್ಮ ವೃದ್ಧಿಯಾಗುವುದೋ ಆಗ ನಾನು ಅವತಾರ ಮಾಡುತ್ತೇನೆ. ಸಾಧುಗಳನ್ನು ರಕ್ಷಣೆಮಾಡುವುದಕ್ಕೆ, ದುಷ್ಟರನ್ನು ನಾಶಮಾಡುವುದಕ್ಕೆ, ಧರ್ಮವನ್ನು ಈ ಪ್ರಪಂಚದಲ್ಲಿ ಸ್ಥಾಪನೆ ಮಾಡುವುದಕ್ಕೆ ಪ್ರತಿ ಯುಗದಲ್ಲಿಯೂ ಅವತರಿಸುತ್ತೇನೆ.
\end{artha}

ಇಲ್ಲಿ ಶ‍್ರೀಕೃಷ್ಣ ಅವತಾರತತ್ತ್ವದ ರಹಸ್ಯವನ್ನು ವಿವರಿಸುವನು. ಇದು ಹಿಂದೂಗಳಲ್ಲಿ ಪ್ರಚಾರವಾಗಿರುವ ಒಂದು ಗಾಢವಾದ ನಂಬಿಕೆ. ದೇವರು ಸೃಷ್ಟಿಯನ್ನು ಮಾಡಿಯಾದಮೇಲೆ ತನ್ನ ಪಾಡಿಗೆ ತಾನು ಇಲ್ಲ. ಸದಾ ಈ ಸೃಷ್ಟಿ ಹೇಗೆ ನಡೆಯುತ್ತದೆ ಎಂದು ಮೇಲ್ವಿಚಾರಣೆ ನೋಡಿಕೊಳ್ಳುತ್ತಿರುವನು. ಯಾವಾಗ ಇಲ್ಲಿ ಅಧರ್ಮ ಹೆಚ್ಚಾಗುವುದೋ, ಆಗ ದುಷ್ಟರನ್ನು ನಾಶಮಾಡಿ ಧರ್ಮವನ್ನು ಪುನಃ ಸ್ಥಾಪಿಸುವುದಕ್ಕಾಗಿ ದೇವರೇ ಪುನಃ ಪುನಃ ಅವತಾರವೆತ್ತುವನು. ಕ್ರೈಸ್ತ ಮತ್ತು ಮಹಮ್ಮದೀಯ ಧರ್ಮಗಳ ಪ್ರಕಾರ ದೇವರೇ ಇಳಿದು ಬರುವುದಿಲ್ಲ. ಅವನು ಕೆಲವು ವೇಳೆ ತನ್ನ ಮಗನನ್ನು ಕಳುಹಿಸುವನು. ಕ್ರಿಸ್ತನೇ ದೇವರ ಮಗ. ಅವನು ಧರೆಗೆ ಬಂದು ಜನರಿಗೆ ಧರ್ಮದ ಹಾದಿಯನ್ನು ತೋರುವನು. ಅದರಂತೆಯೇ ಮಹಮ್ಮದೀಯ ಧರ್ಮದಲ್ಲಿ ದೇವರು ತನ್ನ ಒಬ್ಬ ದೂತನನ್ನು ಧರ್ಮಸಂಸ್ಥಾಪನೆಗೆ ಕಳುಹಿಸುವನು. ಅವನೇ ಮಹಮ್ಮದ್. ಮಗ, ದೂತ, ಎಂದು ಬೇರೆ ಬೇರೆ ಹೆಸರಿನಿಂದ ಕರೆದರೂ ಅವರೆಲ್ಲ ಭಗವದಂಶ. ಮಾನವನಂತೆ ಅವತಾರಮಾಡಿ ಜನರಿಗೆ ದಾರಿ ತೋರುವರು. ಹಿಂದೂಗಳಾದರೋ, ಮಹಮ್ಮದ್ ಮತ್ತು ಕ್ರೈಸ್ತಧರ್ಮಗಳಲ್ಲಿರುವಂತೆ, ಅಂಶಾವತಾರಗಳು ಮಾತ್ರವಲ್ಲ ಕೆಲವು ವೇಳೆ ಸ್ವಯಂ ಭಗವಂತನೆ ಹಾಗೆ ಇಳಿದು ಬರುವನು ಎಂದು ಹೇಳುವರು. ಶ‍್ರೀಕೃಷ್ಣ ಅಂತಹ ಒಂದು ಅವತಾರ.

ಭವಜೀವಿಗಳಲ್ಲಿ ಯಾರು ಬೇಕಾದರೂ ಸಾಧನೆ ಮಾಡಿ ಸಿದ್ಧರಾಗಬಹುದು. ಆದರೆ ಅವತಾರವಾದರೋ ಹಾಗೆ ಸಾಧನೆಯಿಂದ ಸಿದ್ಧರಾಗುವುದಲ್ಲ. ಅವರು ಹುಟ್ಟುವಾಗಲೇ ಸಿದ್ಧ\-ಪುರುಷರಾಗಿ ಬರುವರು. ಅವರು ಮಾಡುವ ಸಾಧನೆಗಳೆಲ್ಲಾ ಜನರಿಗೆ ಮೇಲ್ಪಂಕ್ತಿಯನ್ನು ತೋರುವುದಕ್ಕೆ ಮಾತ್ರ. ಅವೆಲ್ಲ ಶ‍್ರೀರಾಮಕೃಷ್ಣರು ಹೇಳುತ್ತಿದ್ದಂತೆ ಸೋರೆಕಾಯಿ, ಪಡವಲಕಾಯಿ ಹೂವುಗಳಂತೆ–ಹುಟ್ಟುವಾಗಲೇ ಕಾಯಿಸಹಿತ ಹುಟ್ಟುತ್ತವೆ.

ಸಾಧಾರಣ ಸಿದ್ಧಪುರುಷನಾದರೋ ಸಾಧನ ಬಲದಿಂದ ತಾನು ಮುಕ್ತನಾಗುವನು. ಸುತ್ತಮುತ್ತ ತನ್ನ ಹತ್ತಿರ ನೆರೆದವರಿಗೆ ಸ್ವಲ್ಪ ದಾರಿ ಬೆಳಕಾಗುವನು. ಒಂದು ಸಾಧಾರಣ ಹುಲ್ಲು ನೀರಿನ ಮೇಲೆ ತೇಲಿಕೊಂಡು ಹೋಗುತ್ತಿರುವುದು. ಅದು ಕೂಡ ಸಮುದ್ರವನ್ನು ಸೇರುವುದು. ಆದರೆ ದೊಡ್ಡ ದೊಂದು ದೋಣಿಯೂ ತೇಲುತ್ತಿರುವುದು. ಅದು ತಾನು ಸೇರುವುದಲ್ಲದೆ ನೂರಾರು ಜನರನ್ನು ತನ್ನ ವಕ್ಷದ ಮೇಲೆ ಹೊತ್ತುಕೊಂಡು ಹೋಗುವುದು. ಅವತಾರ ಎಂದರೆ ಹೀಗೆ.

ಅವತಾರವಾದರೋ ಮೇಲಿನಿಂದ ಇಳಿದು ಬರುವನು. ಸಿದ್ಧಪುರುಷನಾದರೋ ಕೆಳಗಿನಿಂದ ಏರಿ ಹೋಗುವನು.

ಅವತಾರ ಬರುವಾಗ ಒಂದು ಹೊಸ ದೃಷ್ಟಿಯನ್ನು ಆಧ್ಯಾತ್ಮಿಕ ಜೀವನದಲ್ಲಿ ತರುವನು. ಅದು ಆ ಕಾಲ ದೇಶಕ್ಕೆ ಅನ್ವಯಿಸುವುದು ಮಾತ್ರವಲ್ಲ, ಸನಾತನವಾದ ಸತ್ಯಗಳು ಅದರಲ್ಲಿರುತ್ತವೆ. ಕರ್ಮ, ಕೃಷ್ಣನಿಗೆ ಮುಂಚೆ, ಬರೀ ಚಿತ್ತಶುದ್ಧಿಯಾಗುವುದಕ್ಕೆ ಒಂದು ಮಾರ್ಗವಾಗಿ ಮಾತ್ರ ಇತ್ತು. ಅನಂತರ ಅವರು ಜ್ಞಾನವನ್ನೋ ಭಕ್ತಿಯನ್ನೋ ಧ್ಯಾನವನ್ನೋ ಹಿಡಿದು ಮುಂದುವರಿಯ\-ಬೇಕಾಗಿತ್ತು. ಆದರೆ ಶ‍್ರೀಕೃಷ್ಣನಾದರೋ ಕರ್ಮ ಚಿತ್ತಶುದ್ಧಿಯನ್ನು ಮಾಡುವುದು ಮಾತ್ರವಲ್ಲ, ಅದರಲ್ಲಿಯೆ ಗುರಿಯನ್ನೂ ಸೇರಬಹುದು ಎನ್ನುತ್ತಾನೆ. ಫಲಾಪೇಕ್ಷೆ ಇಲ್ಲದೆ ಕರ್ಮಮಾಡಿದರೆ, ಯಜ್ಞದೃಷ್ಟಿಯಿಂದ ಕರ್ಮ ಮಾಡಿದರೆ, ಕರ್ಮವನ್ನೆಲ್ಲಾ ಭಗವದರ್ಪಿತ ಭಾವದಿಂದ ಮಾಡಿದರೆ, ಅವರಿಗೂ ಮುಕ್ತಿ ಸಿಕ್ಕುವುದು ಎಂದು ಕರ್ಮಕ್ಕೆ ಕರ್ಮಯೋಗದ ಸ್ಥಾನವನ್ನು ಕೊಡುವನು.

ಅವತಾರ ಬರುವುದೇ ದೇಶದಲ್ಲಿ ಒಂದು ಆಧ್ಯಾತ್ಮಿಕ ಪ್ರವಾಹವನ್ನು ಹರಿಸುವುದಕ್ಕೆ. ಅವನು ಪ್ರಾರಂಭ ಮಾಡಿದ ತತ್ತ್ವ ಸಂದೇಶ ಕಾಲಾನುಕಾಲದವರೆಗೆ ಹರಿದುಕೊಂಡು ಹೋಗುತ್ತದೆ. ಅವನಿರುವಾಗ ಅದು ಎಷ್ಟು ಬಳಕೆಯಲ್ಲಿತ್ತೊ, ಅವನು ಕಾಲವಾದಮೇಲೆ ಅದಕ್ಕಿಂತ ಹೆಚ್ಚು ವಿಸ್ತಾರವಾಗುತ್ತ ಬರುವುದು. ಇಂತಹ ಒಂದು ಆಧ್ಯಾತ್ಮಿಕ ತರಂಗವನ್ನು ದೇಶದ ಮೇಲೆ ಹರಡ\-ಬೇಕಾದರೆ, ಮೊದಲು ತಾನು ಅದನ್ನು ಅನುಭವಿಸಿರಬೇಕು. ಅನಂತರ ಅದನ್ನು ಪ್ರಪಂಚದ ಮೇಲೆ ಹರಡುವುದಕ್ಕೆ ಅವನಿಗೆ ಶಕ್ತಿ ಇರಬೇಕು. ಯಾವುದು ಅದನ್ನು ವಿರೋಧಿಸುವುದೋ ಅದನ್ನು ನಿರ್ಮೂಲ ಮಾಡುವ ಶಕ್ತಿ ಅವನಲ್ಲಿರಬೇಕು. ಬರೀ ಧಾರ್ಮಿಕ ಜೀವನವನ್ನು ನಡೆಸಬೇಕು ಎಂದು ಹೇಳಿದರೆ ಸಾಲದು. ಹಾಗೆ ನಡೆಸದವರಿಗೆ ಏನಾಗುವುದು ಎಂಬುದನ್ನು ತೋರುತ್ತಾನೆ. ಆತಂಕಗಳನ್ನು ನಿವಾರಣೆ ಮಾಡುವ ಶಕ್ತಿ ಅವನಲ್ಲಿರುತ್ತದೆ. ರೈತ ಬೀಜವನ್ನು ನೆಡುತ್ತಾನೆ. ಅದು ಬೆಳೆಯುತ್ತಿರುವಾಗ ಕೆಲಸಕ್ಕೆ ಬಾರದ ಕಳೆಗಳು ಸುತ್ತಲೂ ಬೆಳೆಯುತ್ತಿದ್ದರೆ ಅದನ್ನು ನಿರ್ದಾಕ್ಷಿಣ್ಯವಾಗಿ ಕಿತ್ತುಹಾಕುತ್ತಾನೆ. ಅದರಂತೆಯೇ ಅವತಾರ ಧಾರ್ಮಿಕ ಜೀವನದ ಬೀಜವನ್ನು ನೆಡುವುದಲ್ಲದೆ, ನಿರ್ದಾಕ್ಷಿಣ್ಯದಿಂದ ಅಧರ್ಮದ ಕಳೆಗಳನ್ನು ಕಿತ್ತುಹಾಕುತ್ತಾನೆ.

ಸಾಧು ಸ್ವಭಾವದ ಮನುಷ್ಯನನ್ನು ಉದ್ಧಾರ ಮಾಡುವುದಕ್ಕೆ, ದುಷ್ಟರನ್ನು ಶಿಕ್ಷಿಸುವುದಕ್ಕೆ ಪರಮಾತ್ಮ ಬರುತ್ತಾನೆ ಎಂಬ ನಂಬಿಕೆ ಹಿಂದೂಗಳಲ್ಲಿ ಅಷ್ಟು ಆಳವಾಗಿ ಬೇರೂರಿದೆ. ಎಂತಹ ಹೀನಸ್ಥಿತಿಗೆ ಹೋದರೂ ಭರವಸೆ ಕೆಡದೆ ನಿರಾಶರಾಗದೆ ಹಿಡಿದ ಧಾರ್ಮಿಕ ಪಥವನ್ನು ಬಿಡದೆ ಮುಂದುವರಿಸಿಕೊಂಡು ಹೋಗುವುದಕ್ಕೆ ಶಕ್ತಿಯನ್ನು ನೀಡಿದೆ ಈ ನಂಬಿಕೆ. ಇದು ಬರೀ ಒಂದು ನಂಬಿಕೆ ಮಾತ್ರವಲ್ಲ. ಭರತಖಂಡದ ಐತಿಹಾಸಿಕ ಜೀವನದಲ್ಲಿ ಇದನ್ನು ನೋಡಿರುವೆವು. ಎಲ್ಲಾ ಸಂದಿಗ್ಧ ಪರಿಸ್ಥಿತಿಯಲ್ಲಿಯೂ ಭಗವಂತನೊ, ಅವನ ಅಂಶದವರೋ ಇಳಿದುಬಂದು ಅಧರ್ಮದ ಕಳೆಯನ್ನು ಕಿತ್ತುಹಾಕಿರುವರು. ಧರ್ಮಸಸಿಗೆ ಜೀವವನ್ನು ದಾನಮಾಡಿರುವರು, ತಮ್ಮ ಆಧ್ಯಾತ್ಮಿಕ ಶಕ್ತಿಯಿಂದ.

\begin{shloka}
ಜನ್ಮ ಕರ್ಮ ಚ ಮೇ ದಿವ್ಯಮೇವಂ ಯೋ ವೇತ್ತಿ ತತ್ತ್ವತಃ~।\\ತ್ಯಕ್ತ್ವಾ ದೇಹಂ ಪುನರ್ಜನ್ಮ ನೈತಿ ಮಾಮೇತಿ ಸೋಽಜುRನ \hfill॥ ೯~॥
\end{shloka}

\begin{artha}
ಅರ್ಜುನ, ಯಾರು ನನ್ನ ದಿವ್ಯವಾದ ಅವತಾರವನ್ನು ಮತ್ತು ಕರ್ಮವನ್ನು ಯಥಾರ್ಥವಾಗಿ ತಿಳಿದುಕೊಳ್ಳುತ್ತಾರೋ ಅವನು ದೇಹವನ್ನು ತ್ಯಾಗಮಾಡಿದ ನಂತರವೂ ಪುನರ್ಜನ್ಮವನ್ನು ಹೊಂದುವುದಿಲ್ಲ. ನನ್ನನ್ನೇ ಹೊಂದುತ್ತಾನೆ.
\end{artha}

ದೇವರು ಅವತಾರವಾಗಿ ಇಳಿದು ಬಂದಾಗ ಅವತಾರವೇ ಒಂದು ಹೊಸ ಮಾರ್ಗ ಆಗುವುದು ಭಗವಂತನ ಕಡೆ ಹೋಗುವುದಕ್ಕೆ. ಗಂಗಾನದಿಗೆ ಹೋಗಿ ಸ್ನಾನ ಮಾಡುವುದಕ್ಕೆ ಎಷ್ಟೋ ಘಾಟುಗಳಿವೆ. ಅವತಾರವಾದರೋ ಜನರೆಲ್ಲ ಬಂದು ಮಿಂದು ಹೋಗುವುದಕ್ಕೆ ಒಂದು ಹೊಸ ಘಾಟಿನಂತೆ ಆಗುವುದು.

ಅವತಾರದ ವ್ಯಕ್ತಿಯಲ್ಲಾದರೋ ವೇದವೇದಾಂತಗಳ ಸಾರ ಸ್ಫುರಿಸುತ್ತಿರುವುದು. ಅವನ ಜನ್ಮ ದಿವ್ಯವಾದುದು; ಕರ್ಮ ದಿವ್ಯವಾದುದು. ಭಕ್ತ ಮತ್ತಾವುದನ್ನೂ ಮಾಡಬೇಕಾಗಿಲ್ಲ. ಈ ಅವತಾರದ ವ್ಯಕ್ತಿಯನ್ನು ಆದರ್ಶವಾಗಿಟ್ಟುಕೊಂಡು ಅದನ್ನು ಯಥಾರ್ಥವಾಗಿ ತಿಳಿದುಕೊಳ್ಳುತ್ತಾ ಬಂದರೆ ಸಾಕು. ಅವನು ಮುಕ್ತನಾಗುತ್ತಾನೆ ಎನ್ನುವನು. ಅವನ ಜನ್ಮವೇ ಪವಿತ್ರವಾದುದು. ಅವನು ಹೇಳುವುದೇ ಜಗದ ಉದ್ಧಾರಕ್ಕೆ. ಅವನು ಮಾಡುವ ಕರ್ಮವೆಲ್ಲ ಲೋಕಸಂಗ್ರಹಕ್ಕೆ.\break ತನಗಾಗಿ ಏನೂ ಇಲ್ಲ. ಲೋಕಾನುಕಂಪೆಯಿಂದ ಮಾತ್ರ ಅವನು ಬರುವನು. ಅವನನ್ನು ನಾವು ಸರಿಯಾಗಿ ತಿಳಿದುಕೊಳ್ಳಬೇಕು. ಅದೇ ಒಂದು ದೊಡ್ಡ ಸಾಧನೆ. ಶ‍್ರೀಕೃಷ್ಣನ ಜೀವನದ\break ಮಹತ್ವವನ್ನು, ಅವನ ಬೋಧನೆಯ ಮಹತ್ವವನ್ನು ಮನನ ಮಾಡುತ್ತಿದ್ದರೆ ನಮ್ಮ ಜೀವನದ ಮೇಲೆ ಅದ್ಭುತ ಪರಿಣಾಮವನ್ನು ಬೀರುವುದು. ಕೀಟ ಯಾವಾಗಲೂ ಒಂದು ಭ್ರಮರವನ್ನು ಚಿಂತಿಸುತ್ತಿದ್ದರೆ ಅದೂ ಕೂಡ ತನ್ನ ಕೀಟತ್ವವನ್ನು ಕಳೆದುಕೊಂಡು ತಾನೂ ಭ್ರಮರವಾಗುವಂತೆ, ಶ‍್ರೀಕೃಷ್ಣನಂತಹ ಅವತಾರವ್ಯಕ್ತಿಯನ್ನು ಚಿಂತಿಸುತ್ತಿದ್ದರೆ ಆ ವ್ಯಕ್ತಿಯಲ್ಲಿರುವ ಶಕ್ತಿಗಳು ನಮ್ಮನ್ನು ಪ್ರವೇಶಿಸಿ ಅದು ನಮ್ಮ ಜೀವನವನ್ನು ಸಂಪೂರ್ಣ ಬದಲಾಯಿಸಿ ನಮ್ಮನ್ನು ಕೂಡ ಮುಕ್ತರನ್ನಾಗಿ ಮಾಡುವುದು. ನಾವು ಔಷಧವನ್ನು ಸೇವನೆ ಮಾಡಿದರೆ ಅದು ನಮ್ಮ ದೇಹಕ್ಕೆ ಹೋಗಿ ಅಲ್ಲಿ ನಮ್ಮ ರೋಗವನ್ನೆಲ್ಲಾ ಹೇಗೆ ನಿರ್ಮೂಲ ಮಾಡುವುದೋ ಹಾಗೆ.

ಇಂತಹ ಅವತಾರ ವ್ಯಕ್ತಿಯನ್ನು ಉಪಾಸನೆ ಮಾಡುತ್ತಿರುವ ವ್ಯಕ್ತಿ ಈ ದೇಹವನ್ನು ತ್ಯಜಿಸಿದ ಮೇಲೆ ಪುನಃ ಸಂಸಾರಕ್ಕೆ ಬರುವುದಿಲ್ಲ. ಏಕೆಂದರೆ ಅವನ ಅಜ್ಞಾನ ಸೀದುಹೋಗಿದೆ. ಹೃದಯ ನಿರ್ಮಲವಾಗಿದೆ. ಆಸೆಗಳೆಲ್ಲಾ ಬೂದಿಯಾಗಿವೆ. ಇನ್ನು ಅಂತಹ ಜೀವಿ ಬರುವುದಕ್ಕಾದರೂ ಕಾರಣವೇನಿದೆ? ಲೌಕಿಕ ಆಸೆ ಇದ್ದರೆ ಅದರಿಂದ ಆಕರ್ಷಿತನಾಗಿ ಬಂದು ಜನ್ಮವನ್ನು ಧರಿಸಬೇಕಾಗುವುದು. ಇರುವಾಗಲೆ ಆಸೆಯೆಲ್ಲ ದಗ್ಧವಾಗಿ ಹೋಗಿ, ಭಗವಂತನೊಬ್ಬನೇ ಅವನ ಹೃದಯದಲ್ಲಿರುವಾಗ, ಅವನು ಹೋದರೆ ಹೊರಟೇ ಹೋಗುವನು. ಪುನಃ ಬರುವುದಿಲ್ಲ. ಕಲ್ಲನ್ನು ಮೇಲಕ್ಕೆ ಎಸೆದರೆ ಕೆಳಗೆ ಬೀಳುವುದು. ಆದರೆ ಸುಮಾರು ಹದಿನೆಂಟು ಸಾವಿರ ಮೈಲಿ ವೇಗದಲ್ಲಿ ಒಂದು ಕ್ಷಿಪಣಿಯನ್ನು ಎಸೆದರೆ ಅದು ಪುನಃ ಧರೆಗೆ ಹಿಂತಿರುಗಿ ಬರುವುದಿಲ್ಲ. ಆಕರ್ಷಣದಿಂದ ಒಂದೇ ಸಲ ಕಿತ್ತುಕೊಂಡು ಹೋಗುವುದು. ಅದರಂತೆಯೆ ಮುಕ್ತಾತ್ಮ. ಇರುವಾಗ ದೇವರಲ್ಲಿರುವನು. ಅಳಿವಾಗಲೂ ಅವನಲ್ಲಿರುವನು. ಅವನು ಭಗವಂತನನ್ನೇ ಹೊಂದುತ್ತಾನೆ.

\begin{shloka}
ವೀತರಾಗಭಯಕ್ರೋಧಾ ಮನ್ಮಯಾ ಮಾಮುಪಾಶ್ರಿತಾಃ~।\\ಬಹವೋ ಜ್ಞಾನತಪಸಾ ಪೂತಾ ಮದ್ಭಾವಮಾಗತಾಃ \hfill॥ ೧೦~॥
\end{shloka}

\begin{artha}
ಆಸಕ್ತಿ, ಭಯ, ಕ್ರೋಧಗಳಿಂದ ಪಾರಾಗಿ, ನನ್ನಲ್ಲಿ ಮನಸ್ಸುಳ್ಳವರಾಗಿ, ನನ್ನನ್ನೇ ಆಶ್ರಯಿಸಿ ಅನೇಕರು ಜ್ಞಾನರೂಪವಾದ ತಪಸ್ಸಿನಿಂದ ಪವಿತ್ರರಾಗಿ ನನ್ನ ಭಾವವನ್ನು ಹೊಂದುತ್ತಾರೆ.
\end{artha}

ನಮ್ಮನ್ನು ಪ್ರಪಂಚಕ್ಕೆ ಕಟ್ಟಿಹಾಕಿರುವ ಗೂಟಗಳೇ ಆಸಕ್ತಿ, ಭಯ ಮತ್ತು ಕ್ರೋಧ ಇವುಗಳು. ಯಾವಾಗ ಒಬ್ಬ ಭಗವಂತನ ಕಡೆ ಮನಸ್ಸನ್ನು ಹರಿಸುತ್ತಾನೋ ಅವನು ಇವುಗಳಿಂದ ಪಾರಾಗುತ್ತಾನೆ. ಒಂದನ್ನು ಪಡೆಯಬೇಕಾದರೆ ನಾವು ಮತ್ತೊಂದನ್ನು ತ್ಯಜಿಸಬೇಕು. ಎರಡನ್ನು ಏಕಕಾಲದಲ್ಲಿ ಇಟ್ಟುಕೊಂಡಿರುವುದಕ್ಕಾಗುವುದಿಲ್ಲ. ಪ್ರಪಂಚದ ಮೇಲೆ ಆಸಕ್ತರಾಗಿದ್ದರೆ ದೇವರ ಕಡೆಗೆ ಹೋಗಲು ಆಗುವುದಿಲ್ಲ. ದೇವರ ಮೇಲೆ ಆಸಕ್ತರಾಗಿದ್ದರೆ ಪ್ರಪಂಚದ ಮೇಲೆ ಆಸಕ್ತರಾಗಿರುವುದಕ್ಕಾಗುವುದಿಲ್ಲ. ಇದು ಶ‍್ರೀರಾಮಕೃಷ್ಣರು ಹೇಳಿದ ಕುಡುಕರು ನೌಕಾವಿಹಾರಕ್ಕೆ ಹೋದ ಕಥೆಯನ್ನು ನೆನಪಿಗೆ ತರುವುದು. ನಾಲ್ಕೈದು ಜನ ಕುಡುಕರು ಚೆನ್ನಾಗಿ ಕುಡಿದು ಅಂದು ಪೂರ್ಣಿಮೆಯಾಗಿ\-ದ್ದುದರಿಂದ ದೋಣಿಯ ಮೇಲೆ ನೌಕಾ ವಿಹಾರಕ್ಕೆ ಹೊರಟರು. ದೋಣಿಯೊಳಗೆ ಕುಳಿತು ಬೆವರು ಸುರಿಯುವತನಕ ಕೋಲಿನಿಂದ ಅದನ್ನು ನಡೆಸುತ್ತಿದ್ದರು. ಬಹಳ ಹೊತ್ತಾದ ಮೇಲೆ ನೋಡುತ್ತಾರೆ, ದೋಣಿ ಇದ್ದಕಡೆಯೇ ಇದೆ. ಸ್ವಲ್ಪವೂ ಮುಂದಕ್ಕೆ ಹೋಗಿರಲಿಲ್ಲ. ಇದು ಏತಕ್ಕೋ ಹೀಗೆ ಎಂದು ನೋಡಲಾಗಿ, ದೋಣಿಯನ್ನು ತೀರದ ಗೂಟಕ್ಕೆ ಕಟ್ಟಿದ್ದ ಹಗ್ಗವನ್ನು ಬಿಚ್ಚದೆ ಇವರು ದೋಣಿ ನಡೆಸುತ್ತಿದ್ದುದೇ ಕಾರಣ ಎಂದು ಗೊತ್ತಾಯಿತು. ಮೊದಲು ಪ್ರಪಂಚದ ಮೇಲೆ ಆಸಕ್ತಿಯನ್ನು ತ್ಯಜಿಸಬೇಕು. ಆಗಲೇ ದೇವರ ಕಡೆ ನಾವು ಹೋಗಲು ಸಾಧ್ಯ.

ಅನಂತರ ನಮ್ಮನ್ನು ಸಂಸಾರಕ್ಕೆ ಬಿಗಿದಿರುವ ಗೂಟವೇ ಭಯ. ಯಾವ ಸಮಯದಲ್ಲಿ ಏನಾಗುವುದೋ ಎಂಬ ಕಳವಳ. ಯಾವಾಗ ನಾವು ಪ್ರಿಯವಾಗಿರುವ ವಸ್ತುವಿಗೆ ಅಂಟಿಕೊಂಡಿರು ವೆವೋ ಆಗ ಪ್ರಿಯವಾದ ವಸ್ತುವಿಗೆ ಏನಾದರೂ ಆದರೆ ಏನು ಗತಿ ಎಂದು ಆಗುವುದಕ್ಕೆ ಮುಂಚಿನಿಂದಲೇ ಅವನ್ನು ಮನನ ಮಾಡುತ್ತಿರುವೆವು. ಅಪ್ರಿಯವಾಗಿರುವುದು ನಮಗೆ ಬಂದರೆ ಏನು ಗತಿ, ಯಾವುದೋ ಗುಣಪಡಿಸಲಾಗದ ಖಾಯಿಲೆ, ಅಪಕೀರ್ತಿ, ಕಷ್ಟ ಇವುಗಳು ಬಂದರೆ ನಮ್ಮ ಗತಿ ಏನು ಎಂದು ಅನವರತ ಚಿಂತಿಸುತ್ತಿರುವೆವು. ಪ್ರಿಯವಾಗಿರುವುದು, ನಾನು ನಿನ್ನನ್ನು ಎಂದು ಬಿಟ್ಟುಹೋಗುವೆನೋ ಎಂದು ಅಂಜಿಸುತ್ತಿರುವುದು. ಅಪ್ರಿಯವಾಗಿರುವುದು, ನೋಡು, ನಾನು ನಿನ್ನ ಮೇಲೆ ಎಂದು ಬೀಳುವೆನೋ ಎಂದು ಅಂಜಿಸುತ್ತಿರುವುದು. ಅಂತೂ ನಾವು ಎರಡು ಅಂಜಿಕೆಗಳ ಮಧ್ಯದಲ್ಲಿರುವೆವು. ಈ ಭಯ ಯಾರನ್ನೂ ಬಿಟ್ಟಿಲ್ಲ. ಶ‍್ರೀಮಂತನಾಗಿರಲಿ, ಬಡವನಾಗಿರಲಿ, ಪಂಡಿತನಾಗಿರಲಿ, ಪಾಮರನಾಗಿರಲಿ, ಆಳಾಗಿರಲಿ, ಅರಸಾಗಿರಲಿ, ಈ ಭಯ ಕಾಡುತ್ತಿರುವುದು. ಇದಕ್ಕೆ ಅಂಜುವವನು ದೇವರ ಕಡೆ ಹೋಗಲಾರ. ದೇವರಕಡೆಗೆ ಹೊರಟ ಯಾತ್ರಿಕನಲ್ಲಿ ನೋಡುವ ಒಂದು ಗುಣವೆ ನಿರ್ಭಯ. ಅವನು ಯಾರಿಗೂ ಅಂಜುವುದಿಲ್ಲ, ಯಾವುದನ್ನೂ ಲಕ್ಷಿಸುವುದಿಲ್ಲ. ಅಂಜಿಕೆಗೆ ಅಂಜಿಕೆಯನ್ನು ತರುವ ಭಗವಂತನನ್ನು ಅವನು ಕೈಹಿಡಿದಿರುವನು. ಈ ಸಂಸಾರದ ನಾಯಿನರಿಗಳ ಕೂಗಿಗೆ ಅಂಜುವವನಲ್ಲ ಅವನು.

ಅದರಂತೆಯೇ ಅವನು ಕ್ರೋಧದಿಂದ ಪಾರಾಗಿರುವನು. ಯಾವಾಗ ಕ್ರೋಧ ಬರುವುದೋ ಆಗ ಕಿಂಕರ್ತವ್ಯ ಮೂಢರಾಗುತ್ತೇವೆ. ಮಾಡಬಾರದ ಕೆಲಸವನ್ನು ಮಾಡುತ್ತೇವೆ. ಆಡಬಾರದ ಮಾತನ್ನು ಆಡುತ್ತೇವೆ. ನಮ್ಮ ಕೆಲಸದಿಂದ ಮುಂದೆ ನಮಗೇನಾಗುವುದು ಎಂಬುದನ್ನು ಕೂಡ ನಾವು ಚಿಂತಿಸುವುದಿಲ್ಲ. ನನ್ನಲ್ಲಿ ಲೌಕಿಕ ಬಯಕೆಗಳಿದ್ದರೆ ನನಗೂ ಆ ಬಯಕೆಯನ್ನು ಈಡೇರಿಸಿ\-ಕೊಳ್ಳುವುದಕ್ಕೂ ಮಧ್ಯದಲ್ಲಿ ಆತಂಕಗಳು ಬಂದರೆ ನನಗೆ ಅವುಗಳ ಮೇಲೆ ಕೋಪ ಬರುವುದು. ಆದರೆ ಭಗವತ್ ಭಕ್ತನಿಗಾದರೋ ಲೋಕದ ಬಯಕೆಗಳೇ ಇಲ್ಲ. ಯಾವುದನ್ನು ಕೊಡಲಿ,\break ಯಾವುದನ್ನು ಕಿತ್ತುಕೊಂಡು ಹೋಗಲಿ ಅವುಗಳ ಹಿಂದೆಲ್ಲ ದೇವರ ವ್ಯಾಪಾರವನ್ನೇ ನೋಡುವನು. ಕೊಟ್ಟರೆ ಕುಣಿದಾಡುವುದಿಲ್ಲ. ಕಿತ್ತುಕೊಂಡರೆ ಒದ್ದಾಡುವುದಿಲ್ಲ. ಧನ್ಯವಾಗಲಿ ಅವನ ನಾಮ ಎನ್ನುವನು, ಎರಡು ಸಮಯಗಳಲ್ಲಿಯೂ.

ಭಕ್ತ ಭಗವಂತನಲ್ಲಿಯೇ ತನ್ನ ಮನಸ್ಸನ್ನೆಲ್ಲಾ ಇಟ್ಟಿರುವನು. ಒಂದು ನೀರಿಗೆ ಬಟ್ಟೆಯನ್ನು ಅದ್ದಿದರೆ ಅದು ಎಲ್ಲಾ ರಂಧ್ರಗಳಿಂದಲೂ ನೀರನ್ನು ಹೇಗೆ ಹೀರಿಕೊಳ್ಳುವುದೋ ಹಾಗೆ\break ಭಗವಂತನನ್ನು ಹೀರಿಕೊಂಡು ತುಂಬಿಕೊಂಡಿರುವನು. ಇವನ ಮನಸ್ಸೆಲ್ಲಾ ಯಾವಾಗಲೂ ದೇವರ ಕಡೆಯೇ ಹರಿಯುತ್ತಿರುವುದು. ನದಿ ಎಡೆಬಿಡದೆ ಹಗಲು ರಾತ್ರಿ ಸಾಗರದ ಕಡೆ ಹೇಗೆ ಹರಿಯುತ್ತಿರುವುದೋ ಹಾಗೆ ಇವನ ಮನಸ್ಸು ಸದಾ ದೇವರ ಕಡೆ ಹರಿಯುತ್ತಿರುವುದು. ಉತ್ತರಮುಖಿ ಯಾವಾಗಲೂ ಉತ್ತರ ದಿಕ್ಕನ್ನೇ ತೋರುತ್ತಿರುತ್ತದೆ. ಅದನ್ನು ಎಷ್ಟೇ ತಲೆಕೆಳಗು ಮಾಡಲಿ, ಯಾವ ದಿಕ್ಕಿಗೆ ಹಿಡಿದರೂ, ಅದು ಮಾತ್ರ ಉತ್ತರವನ್ನೇ ತೋರುತ್ತದೆ. ಅದರಂತೆಯೇ ಭಕ್ತನ ಮನಸ್ಸು. ಅವನಿಗೆ ಯಾವ ಬಡತನ ಬರಲಿ, ಕಷ್ಟ ಬರಲಿ, ರೋಗರುಜಿನ ಬರಲಿ, ಇದರಿಂದ ಪಾರುಮಾಡು ಎಂದು ಕೇಳುವುದಿಲ್ಲ. ಇವುಗಳೆಲ್ಲಾ ನಮ್ಮ ಕರ್ಮಾನುಸಾರ ಪ್ರಾಪ್ತವಾಗುವುವು. ಆದರೆ ನನ್ನ ಮನಸ್ಸು ಮಾತ್ರ ದೇವರ ಕಡೆಗೆ ಅನುಗಾಲವೂ ಹರಿಯಲಿ ಎಂದು ಬೇಡುವನು.

ಭಕ್ತ ಭಗವಂತನನ್ನೇ ಆಶ್ರಯಿಸಿರುವನು. ಈ ಪ್ರಪಂಚದಲ್ಲಿ ಎಲ್ಲರಿಗೂ ಒಂದು ಆಶ್ರಯಸ್ಥಾನವಿದೆ. ಯಾರಾದರೂ ನಮ್ಮನ್ನು ಅಟ್ಟಿಸಿಕೊಂಡು ಬಂದರೆ, ನಾವು ಆಶ್ರಯದ ಬಿಲದ ಕಡೆಗೆ ಹೋಗುವೆವು. ಕಷ್ಟಕಾಲಕ್ಕಾಗಲಿ ಎಂದು ದುಡ್ಡು ಕೂಡಿಡುವೆವು, ನೆಂಟರಿಷ್ಟರನ್ನು ಮಾಡಿಕೊಂಡಿರು ವೆವು. ಆದರೆ ಭಕ್ತನ ಏಕಮಾತ್ರ ಆಶ್ರಯವೇ ಭಗವಂತನ ಪಾದಪದ್ಮಗಳು. ಸಂಸಾರದ ಉರಿಬಿಸಿಲಿನ ತಾಪದಿಂದ ಅವನಿಗೆ ತಣ್ನೆರಳ ಆಶ್ರಯವನ್ನು ನೀಡುವುದೇ ಭಗವಂತನ ವಟವೃಕ್ಷ. ಅವನಿಗೆ ಇದಲ್ಲದೆ ಇನ್ನಾವುದೂ ಬೇಕಾಗುವುದಿಲ್ಲ. ಉಳಿದ ಆಶ್ರಯಗಳೆಲ್ಲ ನಮಗೆ ಕೈಕೊಡುವುವು. ಆದರೆ ಎಲ್ಲಾ ಕೈಬಿಟ್ಟರೂ ಕೈಬಿಡದಿರುವುದೊಂದಿದೆ. ಅದೇ ಭಗವಂತನ ಕೃಪಾಹಸ್ತ. ಭಕ್ತ ನೆಚ್ಚಿ ಹೊರಟಿರುವುದು ಇದನ್ನು. ಇದೊಂದೇ ಅವನ ಏಕಮಾತ್ರ ಆಶ್ರಯ.

ಯಾವಾಗ ಭಕ್ತ ಭಗವಂತನನ್ನು ಆಶ್ರಯಿಸುವನೋ, ಆಗ ಅವನಿಗೆ ದೇವರು, ತಾನು ಯಾರು, ಈ ಜಗತ್ತೇನು, ಜೀವವೆಂದರೇನು, ಈ ಬಂಧನಮೋಕ್ಷಗಳು ಹೇಗೆ ಆಗಿವೆ,ಇವುಗಳನ್ನೆಲ್ಲಾ ಹೇಳಿ ಕೊಡುವನು. ಇದೇ ಜ್ಞಾನರೂಪವಾದ ತಪಸ್ಸು. ಇದು ಅವನಿಗೆ ಪ್ರಾಪ್ತವಾಗುವುದು. ಇದನ್ನು ಕೇಳಿ ಅವನು ಪಡೆಯುವುದಿಲ್ಲ. ಕೇಳದೆ ಇದ್ದರೂ ಅದು ಬರುವುದು. ಒಲೆಯ ಸಮೀಪದಲ್ಲಿದ್ದರೆ ಬೇಡವೆಂದರೂ ಶಾಖ ತಾಗುವುದು. ಬೆಂಕಿ ಎಲ್ಲೂ ಹತ್ತಿರ ಇಲ್ಲದೆ ಇದ್ದರೆ, ನಾವು ಎಷ್ಟೋ ಅಗ್ನಿದೇವನ ಜಪ ಮಾಡಿದರೂ ಚಳಿಯಿಂದ ಪಾರಾಗುವುದಿಲ್ಲ. ಈ ಜ್ಞಾನದ ತಪಸ್ಸು ನಮ್ಮ ಅಜ್ಞಾನದ ಕೊಳೆಯನ್ನೆಲ್ಲಾ ಹರಿಸುವುದು.

ಅಂತಹ ಭಕ್ತನಿಗೆ ನನ್ನ ಭಾವವೇ ಬರುವುದು ಎನ್ನುವನು ಶ‍್ರೀಕೃಷ್ಣ. ನಾವು ಯಾರನ್ನು ಚಿಂತಿಸುತ್ತೇವೆಯೋ ಅವರಂತಾಗುವೆವು. ಇದೊಂದು ಮಾನಸಿಕ ನಿಯಮ. ಕೀಟ ಒಂದು ಗೂಡಿನಲ್ಲಿ ಭ್ರಮರವನ್ನು ಚಿಂತಿಸುತ್ತ ಚಿಂತಿಸುತ್ತ ಕೊನೆಗೆ ಕೀಟತ್ವವನ್ನು ತ್ಯಜಿಸಿ ಭ್ರಮರವಾಗಿ ಹೊರಬರುವುದು. ಒಂದು ದೊಡ್ಡ ಅಗ್ನಿಕುಂಡದ ಮಧ್ಯದಲ್ಲಿ ಒಂದು ಕಪ್ಪುಇದ್ದಿಲನ್ನು ಎಸೆದರೆ ಕ್ರಮೇಣ ಅದೂ ಕೂಡ ಧಗಧಗಿಸುತ್ತಿರುವ ಕೆಂಡವಾಗುವುದು. ಅದರಂತೆಯೇ ಭಕ್ತ ಅವನನ್ನು ಚಿಂತಿಸುತ್ತ ಅವನ ಭಾವವನ್ನು ಪಡೆಯುತ್ತಾನೆ. ನದಿಯೊಂದು ಸಾಗರಕ್ಕೆ ಸೇರಿದರೆ, ತನ್ನ ನಾಮರೂಪಗಳನ್ನು ಕಳೆದುಕೊಂಡು ಸಾಗರದಲ್ಲಿ ಒಂದಾಗಿ, ಸಾಗರದ ಧರ್ಮವನ್ನು ಪಡೆಯುವುದು. ಭಕ್ತನ ಜೀವ ನದಿಯೂ ಹೀಗೆಯೇ. ಸಂಸಾರದ ಮಾರ್ಗದಲ್ಲಿ ಹರಿದು ದೇವರನ್ನು ಸೇರಿ ಅವನಲ್ಲಿ ಒಂದಾಗುವನು.

\begin{shloka}
ಯೇ ಯಥಾ ಮಾಂ ಪ್ರಪದ್ಯಂತೇ ತಾಂಸ್ತಥೈವ ಭಜಾಮ್ಯಹಮ್~।\\ಮಮ ವರ್ತ್ಮಾನುವರ್ತಂತೇ ಮನುಷ್ಯಾಃ ಪಾರ್ಥ ಸರ್ವಶಃ \hfill॥ ೧೧~॥
\end{shloka}

\begin{artha}
ಹೇಗೆ ನನ್ನನ್ನು ಉಪಾಸನೆ ಮಾಡುತ್ತಾರೆಯೋ ಹಾಗೆ ನಾನು ಅವರಿಗೆ ಅನುಗ್ರಹಿಸುತ್ತೇನೆ. ಅರ್ಜುನ, ಎಲ್ಲಾ ವಿಧಗಳಿಂದಲೂ ಮನುಷ್ಯರು ನನ್ನ ಮಾರ್ಗವನ್ನೇ ಅನುಸರಿಸುತ್ತಾರೆ.
\end{artha}

ಶ‍್ರೀಕೃಷ್ಣನ ವಾಣಿಯ ಔದಾರ್ಯತೆಯನ್ನು ಮೇಲಿನ ಶ್ಲೋಕದಲ್ಲಿ ನೋಡುತ್ತೇವೆ. ಯಾರು ಹೇಗೆ ಅವನ ಬಳಿಗೆ ಬಂದರೆ ಅವನು ಹಾಗೆ ಓ ಕೊಡುವನು. ಅವನು ಹೀಗೆಯೇ ಬರಬೇಕು, ಅದು ಮಾತ್ರ ಶ್ರೇಷ್ಠ, ಉಳಿದುವೆಲ್ಲಾ ಗೌಣ ಎನ್ನುವುದಿಲ್ಲ. ಯಾರು ಹೇಗೆ ಬಂದರೆ ಎಂಬ ವಾಕ್ಯ ಧ್ವನಿಪೂರ್ಣವಾದ ಪದ. ಮನುಷ್ಯರೆಲ್ಲ ಒಂದು ವೃತ್ತದ ಸುತ್ತಲೂ ಇರುವ ಚುಕ್ಕೆಗಳಂತೆ. ಆ ವೃತ್ತದ ಕೇಂದ್ರವೇ ದೇವರು. ಆ ವೃತ್ತದ ಯಾವ ಕಡೆಯಿಂದ ಹೊರಟರೂ ಕೇಂದ್ರವನ್ನು ಸೇರುತ್ತೇವೆ. ಅದರಂತೆಯೇ ಪ್ರತಿಯೊಬ್ಬನೂ ತನ್ನ ಸಂಸ್ಕಾರಕ್ಕೆ ತಕ್ಕ ಜ್ಞಾನ, ಭಾವ, ಯೋಗದೊಡನೆ ಹುಟ್ಟುವನು. ಒಬ್ಬ ವಿಚಾರಮತಿ ಆಗಿರಬಹುದು. ಅವನ ದಾರಿ ಜ್ಞಾನದ ಮೂಲಕ. ಅವನು ವಸ್ತುವನ್ನು ತಿಳಿದುಕೊಳ್ಳ ಬಯಸುವನು. ಅದರ ಮೂಲಕ್ಕೆ ಹೋಗಲೆಣಿಸುವನು. ಅದು ನಮ್ಮ ಇಂದ್ರಿಯಗಳಿಗೆ ಕೊಡುವ ತೃಪ್ತಿಗೆ ಆಕರ್ಷಿತನಾಗುವುದಿಲ್ಲ. ದೃಶ್ಯ ವಸ್ತುವನ್ನೆಲ್ಲ ವಿಚಾರದ ಒರೆಗಲ್ಲಿನ ಮೇಲೆ ತೀಡಿ ನೋಡುವನು. ಆ ದೃಶ್ಯವಸ್ತುವಾದರೋ ನಾನೆಂಬ ಅಹಂಕಾರದಿಂದ ಹಿಡಿದು ನಮ್ಮ ಕಣ್ಣಿನ ಹೊರಗೆ ಇರುವ ಪಂಚಭೂತಗಳಿಂದ ಆದ ವಸ್ತುಗಳೆಲ್ಲ ಸೇರಿ ಆದುದು. ಒಂದು ವಸ್ತು ಸತ್ಯವಾಗಬೇಕಾದರೆ ಅದು ಎಂದೆಂದಿಗೂ ಇರಬೇಕು. ಅದು ಬದಲಾಯಿಸದೆ ಇರಬೇಕು. ಅವನು ಈ ದೃಷ್ಟಿಯಿಂದ ಪರೀಕ್ಷಿಸುವನು. ಹೀಗೆ ಪರೀಕ್ಷಿಸುತ್ತ ಹೋದರೆ ದೃಶ್ಯವಸ್ತುವಿನ ಜೊಳ್ಳೆಲ್ಲಾ ಹಾರಿ ಹೋಗುವುದು. ದೃಗ್ ಒಂದೇ ಕೊನೆಗೆ ಉಳಿಯುವುದು. ಅದನ್ನೇ ಪರಮ ಸತ್ಯವೆಂದು ಕರೆಯುವನು.

ಮತ್ತೊಬ್ಬನದೇ ಕರ್ಮದ ಹಾದಿ. ಅವನು ಪ್ರಕೃತಿಯನ್ನು ವಿಚಾರ ಮಾಡುವುದಕ್ಕೆ ಅಷ್ಟು ತವಕ ಪಡುವುದಿಲ್ಲ. ಕರ್ಮ ಮಾಡುವನು. ತನ್ನ ಸುತ್ತಮುತ್ತ ಇರುವವರಿಗೆ ಸಹಾಯ ಮಾಡುವನು. ಸಾಧ್ಯವಾದಷ್ಟು ತನ್ನಲ್ಲಿ ದೇವರು ಏನು ಒಳ್ಳೆಯದನ್ನು ಕೊಟ್ಟಿರುವನೋ ಅದನ್ನು ಎಲ್ಲರಿಗೂ ಕೊಟ್ಟು ಅನಂತರ ತಾನು ಅನುಭವಿಸುವನು. ಹಾಗೆ ಕೊಡುವುದರಲ್ಲಿ ಒಂದು ಆನಂದವಿದೆ. ಇನ್ನೊಬ್ಬರಿಗೆ ಯಾವ ಪ್ರತಿಫಲಾಪೇಕ್ಷೆಯೂ ಇಲ್ಲದೆ ಕೊಡುವನು. ಇರುವತನಕ ಪರರಿಗೆ ದುಡಿಯುವನು. ಆ ದುಡಿತದ ಹಿಂದೆ ಹಿಂಸೆ ಇಲ್ಲ, ಆಕಾಂಕ್ಷೆ ಇಲ್ಲ, ಕೀರ್ತಿಯ ಲಾಭದ ಮೋಹವಿಲ್ಲ. ಈ ಜೀವನಕ್ಕೆ ಬಂದಿರುವುದೇ ಸಾಯುವುದಕ್ಕೆ. ಸ್ವಾರ್ಥಕ್ಕೆ ಸಾಯುವುದಕ್ಕಿಂತ ಇನ್ನೊಬ್ಬರಿಗಾಗಿ ದುಡಿಯುತ್ತಾ ಸಾಯುವೆ ಎನ್ನುವನು ಅವನು. ಇದರಿಂದ ಅವನ ಹೃದಯ ಶುದ್ಧವಾಗಿ ತನ್ನೊಳಗೆ ಪ್ರಪಂಚದಲ್ಲೆಲ್ಲಾ ಓತಪ್ರೋತನಾದ ಪರಮಾತ್ಮನ ತತ್ತ್ವವನ್ನು ಅವನು ಅನುಭವಿಸುವನು. ಭಗವಂತನ ಕೈಯಲ್ಲಿ ಒಂದು ನಿಮಿತ್ತವಾಗುವನು.

ಮತ್ತೊಬ್ಬ ಇರುವನು. ಅವನು ಭಾವಜೀವಿ. ಉದ್ವೇಗಪರ ಅವನು. ಅವನಿಗೆ ಪ್ರೀತಿಸುವುದಕ್ಕೆ ಒಂದು ವಸ್ತು ಬೇಕು. ಈ ಪ್ರಪಂಚದ ಅಲ್ಪವಸ್ತುಗಳು ಅವನಿಗೆ ತೃಪ್ತಿಯನ್ನು ತಾರವು.\break ಅವನಿಗೆ ಭೂಮ ಬೇಕು ಪ್ರೀತಿಸುವುದಕ್ಕೆ. ಅದಕ್ಕಾಗಿ ದೇವರ ಕಡೆಗೆ ಬರುವನು. ಈ ಪ್ರಪಂಚದಲ್ಲಿ ಪ್ರೀತಿಸುವುದಕ್ಕೆ ಯೋಗ್ಯವಾದ ವಸ್ತು ಅದೊಂದೆ ಎಂಬುದನ್ನು ಅರಿತಿರುವನು. ಅವನನ್ನು ಪ್ರೀತಿಸುವನು. ಹಾಗೆ ಪ್ರೀತಿಸುವಾಗ ಎಲ್ಲಾ ಭಾವಗಳ ಮೂಲಕ ಪ್ರೀತಿಸಲೆತ್ನಿಸುವನು. ತಂದೆಯಂತೆ ಪ್ರೀತಿಸುವನು, ಪತಿಯಂತೆ ಪ್ರೀತಿಸುವನು, ದಾಸನಂತೆ ಪ್ರೀತಿಸುವನು. ಅವನನ್ನು ತನಗೆ ಪ್ರಿಯವಾದ ಯಾವ ಹೆಸರಿನಿಂದ ಬೇಕಾದರೂ ಕರೆಯುವನು. ಆ ಪ್ರೀತಿಗೆ ತನ್ನ ಸರ್ವಸ್ವವನ್ನೂ ಅರ್ಪಿಸುವನು. ಆ ಪ್ರೀತಿಶಿಖೆಯಲ್ಲಿ ತಾನು ಪತಂಗದಂತೆ ಬಿದ್ದು ಸಾಯಲಿಚ್ಛಿಸುವನು. ಒಬ್ಬ ನಾನು ಬೇರೆ, ದೇವರು ಬೇರೆ, ಜಗತ್ತು ಬೇರೆ ಎಂದು ದ್ವೈತದೃಷ್ಟಿಯಿಂದ ಹೊರಡುವನು. ಮತ್ತೊಬ್ಬ ನಾನು ಅಂಶ, ಅವನು ಪೂರ್ಣ ಎಂಬ ವಿಶಿಷ್ಟಾದ್ವೈತದ ದೃಷ್ಟಿಯಿಂದ ಹೊರಡಬಹುದು. ಇನ್ನೊಬ್ಬ ಇಲ್ಲಿ ಯಾವುದು ನಾನೆಂದು ಕಾಣುತ್ತಿರುವುದೊ ಅದಕ್ಕೆ ಬೇರೆ ವ್ಯಕ್ತಿತ್ವವಿಲ್ಲ. ಅದು ಪರಮಾತ್ಮನ ಪ್ರತಿಬಿಂಬ. ಒಬ್ಬ ಸೂರ್ಯ ನೂರಾರು ಹಿಮಮಣಿಗಳಲ್ಲಿಯೂ ಪುಟ್ಟಪುಟ್ಟ ಸೂರ್ಯರಂತೆ ಪ್ರತಿಬಿಂಬಿಸುತ್ತಿರಬಹುದು. ಆದರೆ ಅದಕ್ಕೆಲ್ಲ ಮೂಲ ಮೇಲಿರುವ ಒಂದು ಸೂರ್ಯ. ಈ ಪ್ರಪಂಚದಲ್ಲಿ ಅದೊಂದೇ ಸತ್ಯ, ಆ ಸತ್ಯದ ಛಾಯೆ, ಅದರ ಪ್ರತಿಬಿಂಬ ಉಳಿದವುಗಳೆಲ್ಲ ಎಂಬ ಅದ್ವೈತದ ದೃಷ್ಟಿಯಿಂದ ನೋಡಬಹುದು.

ಒಬ್ಬ ಹಿಂದೂವಿನಂತೆ ಆ ಪರಮ ಸತ್ಯದ ಕೇಂದ್ರದ ಕಡೆಗೆ ಬರಬಹುದು, ಮತ್ತೊಬ್ಬ ಮಹಮ್ಮದೀಯ ಅಥವಾ ಕ್ರೈಸ್ತನಂತೆ ಬರಬಹುದು. ಒಬ್ಬ ಆ ಸತ್ಯವನ್ನು ಒಂದು ಸ್ಥಿತಿ, ಅದು ಒಂದು ವ್ಯಕ್ತಿಯಲ್ಲ ಎನ್ನಬಹುದು. ಅಂತೂ ಯಾವ ಹೆಸರಿನಿಂದಲಾದರೂ ಬರಲಿ, ಹೇಗಾದರೂ ಬರಲಿ, ಇವರೆಲ್ಲ ಭಗವಂತನೆಡೆಗೆ ಹೊರಟಿರುವವರು ಎನ್ನುವನು ಶ‍್ರೀಕೃಷ್ಣ. ಹೊರಟ ಸ್ಥಳಗಳು ಬೇರೆ ಬೇರೆ, ಆದರೆ ತಲಪುವ ಸ್ಥಳಗಳೆಲ್ಲ ಒಂದೇ. ನದಿಗಳು ಹಲವು ದೇಶಗಳಲ್ಲಿ ಹುಟ್ಟಿ, ಹಲವು ನಾಮಗಳನ್ನು ತಾಳಿ, ನೇರವಾಗಿಯೋ ವಕ್ರವಾಗಿಯೋ ಹರಿದುಕೊಂಡು, ಪ್ರಪಂಚವನ್ನೆಲ್ಲಾ ವ್ಯಾಪಿಸಿ ರುವ ಸಾಗರಕ್ಕೆ ಯಾವುದೋ ಮೂಲೆಯಿಂದ ಸೇರುವುವು. ಕೊನೆಗೆ ಆ ಸಾಗರದಲ್ಲಿ ಒಂದಾಗುವುವು. ಯಾರು ಎಲ್ಲಿಂದಲಾದರೂ ಹೊರಡಲಿ, ಆದರೆ ಮಧ್ಯ ಎಲ್ಲಿಯೂ ನಿಲ್ಲದಿರಲಿ, ಆಗ ಎಲ್ಲರೂ ಒಂದೇ ಗುರಿ ಸೇರುವರು ಎನ್ನುವನು ಶ‍್ರೀಕೃಷ್ಣ. ಇಲ್ಲಿ ನಾವು ಯಾವ ದಾರಿಯನ್ನು ಆರಿಸಿಕೊಂಡಿರು ವೆವೋ ಅದರಲ್ಲಿ ಒಂದು ನಿಷ್ಠೆ ಇರಬೇಕು. ಆದರೆ ಅದು ಮತಭ್ರಾಂತಿ ಆಗಬಾರದು ಅಷ್ಟೇ. ನಾನು ಸರಿ ಉಳಿದವರೆಲ್ಲ ತಪ್ಪು, ಅಥವಾ ಅವರಲ್ಲಿ ಸ್ವಲ್ಪ ಸತ್ಯವಿದೆ, ನನ್ನಲ್ಲಿ ಹೆಚ್ಚು ಸತ್ಯವಿದೆ ಎಂದು ಭಾವಿಸಬಾರದು. ಭಗವಂತ ಯಾವ ಧರ್ಮಕ್ಕೂ ಪಕ್ಷಪಾತಿಯಲ್ಲ. ಪಕ್ಷಪಾತಿ ಆದರೆ ಅವನು ದೇವರೇ ಆಗಲಾರ. ಭಗವಂತನ ಈ ಶ್ಲೋಕದ ವಾಣಿಯಲ್ಲಿ ನಮ್ಮ ಮಾರ್ಗವನ್ನು ನಾವು ಹಿಡಿಯುವ, ಇತರರಿಗೂ ದೇವರು ಒಳ್ಳೆಯದನ್ನು ಮಾಡಲಿ, ಅವರನ್ನು ಮುಂದೆ ನೂಕುತ್ತಿರುವ ಶಕ್ತಿಯೇ ನನ್ನೊಳಗೆ ನನ್ನನ್ನು ಮುಂದೆ ನೂಕುತ್ತಿರುವ ಶಕ್ತಿ ಎಂಬ ಒಂದು ಉದಾರ ಸಂದೇಶವನ್ನು ನೋಡುವೆವು. ಇದು ಹಿಂದೂಗಳಿಗೆ ಎಷ್ಟು ಅನ್ವಯಿಸುವುದೋ ಅಷ್ಟೇ ಅನ್ವಯಿಸುವುದು ಇತರರಿ ಗೆಲ್ಲ. ಇದು ಎಲ್ಲಾ ಕಾಲಕ್ಕೆ, ದೇಶಕ್ಕೆ, ಎಲ್ಲಾ ಬಗೆಯ ಜನಕ್ಕೆ ಅನ್ವಯಿಸಬಲ್ಲ ಸತ್ಯ.

\begin{shloka}
ಕಾಂಕ್ಷಂತಃ ಕರ್ಮಣಾಂ ಸಿದ್ಧಿಂ ಯಜಂತ ಇಹ ದೇವತಾಃ~।\\ಕ್ಷಿಪ್ರಂ ಹಿ ಮಾನುಷೇ ಲೋಕೇ ಸಿದ್ಧಿರ್ಭವತಿ ಕರ್ಮಜಾ \hfill॥ ೧೨~॥
\end{shloka}

\begin{artha}
ಈ ಲೋಕದಲ್ಲಿ ಜನ ಕರ್ಮಫಲವನ್ನು ಇಚ್ಛಿಸುತ್ತ ಇತರ ದೇವತೆಗಳನ್ನು ಪೂಜಿಸುತ್ತಾರೆ. ಏಕೆಂದರೆ ಈ ಪ್ರಪಂಚದಲ್ಲಿ ಕರ್ಮದ ಮೂಲಕ ಸಿದ್ಧಿ ಬೇಗ ಆಗುವುದು.
\end{artha}

ಬಹುಪಾಲು ಜನ ಈ ಪ್ರಪಂಚದಲ್ಲಿ ಏನಾದರೂ ಕೆಲಸ ಮಾಡುತ್ತಿರುವರು. ಅವರ ಕೆಲಸಕ್ಕೆ ಒಂದು ಉದ್ದೇಶವಿದೆ. ಅದನ್ನು ಪೂರ್ಣಗೊಳಿಸುವುದಕ್ಕಾಗಿ ಕಾರ್ಯ ಮಾಡುವರು. ದೇವರನ್ನು ಬೇಡಿದರೆ ಅವನು ಕೊಡುವನು ಎಂದು ಗೊತ್ತಿದೆ. ಅದಕ್ಕಾಗಿ ಅವನ ಬಳಿಗೆ ಹೋಗುವರು. ಅವರು ಕೆಲಸ ಮಾಡುವರು, ಜಯ ಸಿಕ್ಕಲಿ ಎಂದು ದೇವರನ್ನು ಬೇಡುವರು. ಒಬ್ಬೊಬ್ಬರು ಒಂದೊಂದು ದೇವರ ಹತ್ತಿರ ಹೋಗುವರು. ಅವರೆಲ್ಲ ಬೇರೆ ಬೇರೆ ದೇವರ ಬಳಿಗೆ ಹೋದರೂ ಆ ವಿವಿಧ ದೇವರುಗಳ ಮೂಲಕ ಕೆಲಸ ಮಾಡುವುದು ಒಂದೇ ಶಕ್ತಿ. ನಾವು ನಮ್ಮ ಮನೆಗೆ ಹತ್ತಿರ ಇರುವ ಪೋಸ್ಟಿನ ಪೆಟ್ಟಿಗೆಗೆ ಕಾಗದ ಹಾಕುವೆವು. ಸರ್ಕಾರದ ಅಂಚೆಯವನು ಅದನ್ನು ತೆಗೆದುಕೊಂಡು ಎಲ್ಲಿಗೆ ಕಳುಹಿಸಬೇಕೋ ಅಲ್ಲಿಗೆ ಕಳುಹಿಸುವನು. ಇದರಂತೆಯೇ ಬೇರೆ ಬೇರೆ ದೇವತೆಗಳೆಲ್ಲ ಒಂದೇ ಭಗವಂತನ ಬೇರೆ ಬೇರೆ ಪೋಸ್ಟ್ ಪೆಟ್ಟಿಗೆಗಳು ಇರುವಂತೆ.

ಈ ಮನುಷ್ಯಲೋಕದಲ್ಲಿ ಕರ್ಮದ ಹಾದಿಯಲ್ಲಿ ಫಲ ಬೇಗ ಸಿಕ್ಕುವುದು. ಏಕೆಂದರೆ ಇದು ಸ್ಥೂಲವಾಗಿ ಕಾಣುವುದು. ಇಲ್ಲಿ ಸಾಧಿಸುವುದು ಸುಲಭ. ಆದರೆ ಬೇರೆ ಬೇರೆ ಕ್ಷೇತ್ರಗಳಲ್ಲಿಯಾದರೋ ಸಿದ್ಧಿ ಪ್ರಾಪ್ತವಾಗಬೇಕಾದರೆ ಬಹಳ ದಿನಗಳಾಗುವುವು. ಕೆಲವು ಗಂಟೆಗಳಲ್ಲಿ ಒಂದು ಕೆಲಸವನ್ನು ಮಾಡಬಹುದು. ಹೊಲದಲ್ಲಿ ಬಿತ್ತಿ ಕೆಲವು ತಿಂಗಳಲ್ಲಿ ಬೆಳೆ ತೆಗೆಯಬಹುದು. ಇಲ್ಲಿ ಬರುವ ಪರಿಣಾಮಗಳೆಲ್ಲ ಸ್ಥೂಲವಾಗಿ ನಮ್ಮ ಎಣಿಕೆಗೆ ಸಿಕ್ಕುವುದು. ಆದರೆ ಆಧ್ಯಾತ್ಮಿಕ ಜೀವನದಲ್ಲಿ ಆದರೋ ಪ್ರಯಾಣ ಬಹಳ ನಿಧಾನ. ಜೀವಾವಧಿ ಹೋರಾಡುತ್ತಿದ್ದರೂ ನಾವು ಒಂದೆರಡು ವಿಷಯಗಳಲ್ಲಿಯೂ ಪಾರಂಗತರಾಗಿರುವುದಿಲ್ಲ. ಅದರಲ್ಲಿಯೂ ಎಷ್ಟು ಪಡೆದಿರುವೆವು ಎಂಬುದನ್ನು ತೂಗುವುದಕ್ಕೆ, ಅಳೆಯುವುದಕ್ಕೆ, ಎಣಿಸುವುದಕ್ಕೆ ಆಗುವುದಿಲ್ಲ.

ಮಾನವಕೋಟಿಯಲ್ಲಿ ಬಹುಪಾಲು ಕರ್ಮದ ಹಾದಿಯಲ್ಲಿ ಹೋಗುವವರೆ. ಏಕೆಂದರೆ ನಮ್ಮ ಪ್ರವೃತ್ತಿ ಯಾವಾಗಲೂ ಬಾಹ್ಯಮುಖವಾಗಿದೆ. ಏನಾದರೂ ಹೊರಗೆ ಕೆಲಸ ಮಾಡುವುದು ಸುಲಭ. ಸುಮ್ಮನೆ ಕುಳಿತುಕೊಂಡು ಚಿಂತಿಸುವುದು ಕಷ್ಟ. ಚಿಂತಿಸುವುದರಲ್ಲಿಯೂ ಆಧ್ಯಾತ್ಮಿಕ ವಸ್ತುಗಳನ್ನು ಕುರಿತು ಚಿಂತಿಸುವುದು ಮತ್ತೂ ಕಷ್ಟದ ಕೆಲಸ. ಆದಕಾರಣವೆ ಯಾವದಾರಿ ಸುಲಭವಾಗಿದೆಯೋ, ಎಲ್ಲಿ ಏನಾದರೂ ಮಾಡಿದರೆ ತಕ್ಷಣವೇ ಪ್ರತಿಫಲ ಕಾಣುವುದೋ, ಅಂತಹ ರಸ್ತೆಯಲ್ಲಿ ಬಹುಜನರ ಜಾತ್ರೆಯನ್ನು ನೋಡುತ್ತೇವೆ. ಇತರ ಕಡೆ ಒಂಟಿ ವರೆ.

\begin{shloka}
ಚಾತುರ್ವರ್ಣ್ಯಂ ಮಯಾ ಸೃಷ್ಟಂ ಗುಣಕರ್ಮವಿಭಾಗಶಃ~।\\ತಸ್ಯ ಕರ್ತಾರಮಪಿ ಮಾಂ ವಿದ್ಧ್ಯಕರ್ತಾರಮವ್ಯಯಮ್ \hfill॥ ೧೩~॥
\end{shloka}

\begin{artha}
ಗುಣ ಮತ್ತು ಕರ್ಮಗಳ ವಿಭಾಗದಂತೆ ಚಾತುರ್ವರ್ಣ್ಯ ನನ್ನಿಂದ ಸೃಷ್ಟಿಸಲ್ಪಟ್ಟಿದೆ. ನಾನು ಅದರ ಕರ್ತೃವಾಗಿ ದ್ದರೂ ನನ್ನನ್ನು ಅವ್ಯಯ ಮತ್ತು ಅಕರ್ತೃ ಎಂದು ತಿಳಿ.
\end{artha}

ಶ‍್ರೀಕೃಷ್ಣ ಇಲ್ಲಿ ನಾಲ್ಕು ವರ್ಣಗಳು, ಗುಣ ಮತ್ತು ಕರ್ಮಕ್ಕೆ ತಕ್ಕಂತೆ ತನ್ನಿಂದಲೇ ಸೃಷ್ಟಿಯಾಯಿತು ಎನ್ನುವನು. ಶ‍್ರೀಕೃಷ್ಣ ಇಲ್ಲಿ ಮಾತನಾಡುವಾಗ ಎಲ್ಲೋ ಇಂಡಿಯಾ ದೇಶದಲ್ಲಿ ಬಳಕೆಯಲ್ಲಿರುವ, ಕೇವಲ ಜನ್ಮದಿಂದಲೇ ಆದ ನಾಲ್ಕು ಜಾತಿಗಳನ್ನು ಗಮನದಲ್ಲಿಟ್ಟುಕೊಂಡಿಲ್ಲ. ಅವನ ಮುಂದೆ ಇಡೀ ವಿಶ್ವವಿದೆ. ಮಾನವಕೋಟಿಯಲ್ಲಿ ಅವರವರ ಕರ್ಮಕ್ಕೆ ತಕ್ಕಂತೆ ವಿಭಾಗಮಾಡುವುದು ಸಹಜವೇ ಆಗಿದೆ. ಈ ವಿಭಾಗದ ಮೇಲೆಯೇ ಸಮಾಜದ ಸುವ್ಯವಸ್ಥೆ ನಿಂತಿದೆ. ಮೂರು ಗುಣಗಳು, ತಮಸ್ಸು, ರಜಸ್ಸು ಮತ್ತು ಸತ್ತ್ವ. ನಾಲ್ಕು ವರ್ಣಗಳೇ ಬ್ರಾಹ್ಮಣ, ಕ್ಷತ್ರಿಯ, ವೈಶ್ಯ ಮತ್ತು ಶೂದ್ರ. ಯಾವ ಸಮಾಜದಲ್ಲಿ ಆದರೂ ಈ ನಾಲ್ಕು ವರ್ಣಗಳು ಇರಲೇ ಬೇಕಾಗಿದೆ. ಮೊದಲನೆಯವನೆ ಬ್ರಾಹ್ಮಣ. ಆಲೋಚನಾ ಜೀವಿ ಅವನು, ಮೇಧಾವಿ. ಸಮಾಜದ ಉದ್ದೇಶಗಳೇನು, ಅವುಗಳನ್ನು ಸಾಧಿಸುವುದು ಹೇಗೆ ಎಂಬುದನ್ನು ಕೂಲಂಕಷವಾಗಿ ವಿಚಾರಿಸಿ ಅದಕ್ಕೆ ದಾರಿಯನ್ನು ತೋರುವನು. ಇವನು ಮಾನವಕೋಟಿಯ ಸಮಷ್ಟಿಯ ಜೀವನಕ್ಕೆ ಜ್ಞಾನವನ್ನು ಧಾರೆಯೆರೆಯುವನು. ಎರಡನೆಯವನೆ ಕ್ಷತ್ರಿಯ. ಸಮಾಜವನ್ನು ಆಳುವವನು. ಅದನ್ನು ತನ್ನ ಶಕ್ತಿಯ ಬೇಲಿಯಿಂದ ರಕ್ಷಿಸುವನು. ಹೊರಗಿನಿಂದ ಶತ್ರುಗಳು ಬಂದು ಸಮಾಜದ ಮೇಲೆ ಬೀಳದಂತೆ ನೋಡಿಕೊಳ್ಳುವನು. ಮತ್ತು ಸಮಾಜದೊಳಗೆ ಸಮಾಜಘಾತುಕ ಶಕ್ತಿಗಳು ಹಾವಳಿ ಮಾಡದಂತೆ ನೋಡಿಕೊಳ್ಳುವನು. ಇದಕ್ಕಾಗಿ ಸೈನ್ಯ, ಕೋರ್ಟು, ಕಚೇರಿ, ಪೋಲೀಸು ಮುಂತಾದುವುಗಳೆಲ್ಲ ಇರುವುವು. ಮೂರನೆಯವನೆ ವೈಶ್ಯ. ಐಶ್ವರ್ಯವನ್ನು, ಉತ್ಪತ್ತಿಯಾದ ವಸ್ತುವನ್ನು, ಒಂದು ಕಡೆಯಿಂದ ಮತ್ತೊಂದು ಕಡೆಗೆ ಹಂಚುವವನು. ಇರುವ ಕಡೆಯಿಂದ ಇಲ್ಲದ ಕಡೆಗೆ ಹಂಚುವವನು. ಇದರಿಂದಲೇ ಸಮಾಜದಲ್ಲಿ ಎಲ್ಲರಿಗೂ ತಮತಮಗೆ ಬೇಕಾದ ವಸ್ತುಗಳು ಸಿಕ್ಕುತ್ತವೆ. ನಾಲ್ಕನೆಯವರೇ ಶ್ರಮಜೀವಿಗಳು. ಐಶ್ವರ್ಯವನ್ನು, ಆಹಾರ, ಬಟ್ಟೆಬರೆಗಳನ್ನು ಉತ್ಪತ್ತಿಮಾಡುವ ಕಾರ್ಮಿಕ ತಂಡದವರು. ನಾವು ಯಾವ ದೇಶವನ್ನು ನೋಡಿದರೂ ಈ ವಿಭಾಗವನ್ನು ನೋಡುತ್ತೇವೆ. ಆದರೆ ಇಂಡಿಯಾ ದೇಶದಲ್ಲಿ ಮೊದಮೊದಲು ಶ‍್ರೀಕೃಷ್ಣ ಹೇಳುವಂತೆ ಅವರವರು ಮಾಡುವ ಕರ್ಮ ಮತ್ತು ಗುಣದಿಂದ ಅವರು ಎಂತೆಂತಹ ವರ್ಣಕ್ಕೆ ಸೇರಿದವರೆಂದು ನಿಷ್ಕರ್ಷಿಸಲ್ಪಟ್ಟಿದ್ದರೂ, ಕಾಲಕ್ರಮೇಣ ಅದು ಜನ್ಮದಿಂದಲೇ ನಿಷ್ಕರ್ಷಿಸಲ್ಪಡುತ್ತಾ ಹೋಯಿತು. ಬ್ರಾಹ್ಮಣ ಕುಲದಲ್ಲಿ ಹುಟ್ಟಿದರೆ, ಅವನು ಆ ಕರ್ಮವನ್ನು ಮಾಡದೇ ಇದ್ದರೂ ಆ ಗುಣ ಇಲ್ಲದೇ ಇದ್ದರೂ ಬ್ರಾಹ್ಮಣ ಎಂದು ಕರೆದರು. ಅದರಂತೆಯೇ ಕ್ಷತ್ರಿಯ, ವೈಶ್ಯ, ಶೂದ್ರರು. ಆ ಜನ್ಮದಲ್ಲಿ ಹುಟ್ಟಿ ಆ ವೃತ್ತಿಯನ್ನು ಮಾಡದೆ, ಆ ವೃತ್ತಿಗೆ ಸಂಬಂಧಪಟ್ಟ ಹಕ್ಕುಬಾಧ್ಯತೆಗಳನ್ನು ಕೇಳಿದರೆ ದೊಡ್ಡ ಅಪಾಯ ತಟ್ಟುವುದು ಸಮಾಜಕ್ಕೆ. ಯಾವಾಗ ಭಗವಂತನೇ ಈ ವರ್ಣಗಳನ್ನು ಸೃಷ್ಟಿ ಮಾಡಿದ್ದಾನೆಯೋ ಅವರು ಮಾಡುವ ಕೆಲಸದಲ್ಲಿ ವ್ಯತ್ಯಾಸವಿದೆ. ಆದರೆ ಒಂದನ್ನು ಮೇಲು ಮತ್ತೊಂದನ್ನು ಕೀಳು ಎಂದು ಶ‍್ರೀಕೃಷ್ಣ ಎಲ್ಲಿಯೂ ಹೇಳಿಲ್ಲ. ಒಂದು ಗಡಿಯಾರದಲ್ಲಿ ಹೊರಗೆ ನಮಗೆಲ್ಲಾ ಕಾಣುವುದು ಗಂಟೆ ನಿಮಿಷವನ್ನು ತೋರುವ ಕೈಗಳು. ಅವು ಸರಿಯಾಗಿ ನಡೆಯಬೇಕಾದರೆ ಅದರ ಹಿಂದೆ ಗಡಿಯಾರದಲ್ಲಿರುವ ಚಕ್ರಗಳೆಲ್ಲಾ ತಮ್ಮ ತಮ್ಮ ಕೆಲಸವನ್ನು ಮಾಡಬೇಕು. ಅದರಲ್ಲಿ ಒಂದು ಮೇಲು ಮತ್ತೊಂದು ಕೀಳು ಎಂದು ಹೇಳಲಾಗುವುದಿಲ್ಲ. ಒಂದು ಭಾಗಕ್ಕೆ ಬೆಲೆ ಜಾಸ್ತಿ ಮತ್ತೊಂದು ಭಾಗಕ್ಕೆ ಬೆಲೆ ಕಡಿಮೆ ಇರಬಹುದು. ಆದರೆ ಗಡಿಯಾರ ನಡೆಯುವ ದೃಷ್ಟಿಯಿಂದ ಒಂದರಷ್ಟೇ ಮತ್ತೊಂದು ಮುಖ್ಯ. ಅದರಂತೆಯೇ ನಮ್ಮ ಸಮಾಜ ಎಂಬುದು ದೊಡ್ಡ ಯಂತ್ರ. ಇದು ಸುಸೂತ್ರವಾಗಿ ಉರುಳಿಕೊಂಡು ಹೋಗಬೇಕಾದರೆ ಇದರಲ್ಲಿರುವ ಪ್ರತಿಯೊಬ್ಬನೂ ತನ್ನ ಕೆಲಸವನ್ನು ಸರಿಯಾಗಿ ಮಾಡಿದಾಗ ಮಾತ್ರ. ಆಳುವ ಅರಸ ತನ್ನ ತನ್ನ ಸಂಸ್ಕಾರಕ್ಕೆ ತಕ್ಕಂತೆ ತನ್ನ ಪಾಲಿಗೆ ಬರುವ ಕರ್ಮಗಳನ್ನು ಮಾಡಿಕೊಂಡು ಹೋದರೆ ಸಮಾಜ ಸುಸೂತ್ರವಾಗಿ ನಡೆದುಕೊಂಡು ಹೋಗುವುದು. ಬ್ರಹ್ಮನ ಬಾಯಿಂದ ಒಬ್ಬರು ಬಂದರು,\break ಭುಜದಿಂದ ಒಬ್ಬರು ಬಂದರು, ತೊಡೆಯಿಂದ ಒಬ್ಬರು ಬಂದರು, ಪಾದಗಳಿಂದ ಒಬ್ಬರು ಬಂದರು ಎಂದು ಪುರಾಣಗಳಲ್ಲಿ ಹೇಳುತ್ತಾರೆ. ಆ ಬ್ರಹ್ಮ ಎಲ್ಲಾ ಕಡೆಯಲ್ಲಿಯೂ ಪರಿಪೂರ್ಣನಾಗಿಯೇ ಇರುವನು. ಅವನ ಒಂದು ಭಾಗ ಮತ್ತೊಂದು ಭಾಗಕ್ಕಿಂತ ಹೇಗೆ ಹೆಚ್ಚು ಪವಿತ್ರವಾಗುವುದು? ಸಕ್ಕರೆಯಿಂದ ಮಾಡಿದ ಗೊಂಬೆಯ ಮುಖದಲ್ಲಿ ಸಕ್ಕರೆ ಇರುವಂತೆ ಕಾಲಿನಲ್ಲಿಯೂ ಸಕ್ಕರೆ ಇದೆ. ಯಾವ ಭಾಗವನ್ನು ತೆಗೆದುಕೊಂಡರೂ ಒಂದೇ ವಸ್ತು ನಮಗೆ ಕಾಣುವುದು. ಆದರೆ ಒಂದು ಮೇಲು ಮತ್ತೊಂದು ಕೀಳು ಎಂಬ ಭಾವವನ್ನು ತಂದೊಡನೆಯೇ ಮನಸ್ತಾಪಕ್ಕೆ ಕಾರಣವಾಗುವುದು. ಇದರಿಂದ ಪಾರಾಗಬೇಕಾದರೆ ಒಂದೇ ದೃಷ್ಟಿ ಇರಬೇಕು. ಅದನ್ನೇ ಸ್ವಾಮಿ ವಿವೇಕಾ\-ನಂದರು ಹೀಗೆ ಹೇಳುತ್ತಿದ್ದರು: ವರ್ಣಗಳು ಇರುವುವು, ಹಕ್ಕು ಬಾಧ್ಯತೆಗಳು ಹೋಗುವುವು.

ಶ‍್ರೀಕೃಷ್ಣ ತಾನೇ ಅವುಗಳನ್ನೆಲ್ಲಾ ಮಾಡಿದರೂ ತಾನು ಅಕರ್ತೃ ಎಂದು ಹೇಳಿಕೊಳ್ಳುತ್ತಾನೆ. ಯಾವಾಗ ಒಬ್ಬ ತಾನು ಫಲಾಪೇಕ್ಷೆಯಿಂದ ಮಾಡುವನೋ ಅವನಿಗೆ ಕರ್ತೃತ್ವ ಬರುವುದು. ಆದರೆ ದೇವರಿಗೆ ಇವುಗಳ ಮೇಲೆ ಯಾವ ಆಸಕ್ತಿಯೂ ಇಲ್ಲ. ಅವನು ಕರ್ಮ ಮಾಡಿದರೂ ಅದರ ಫಲಕ್ಕೆ ಒಳಗಾಗಿಲ್ಲದೇ ಇರುವುದರಿಂದ ಮಾಡಿಲ್ಲದಂತೆಯೆ. ಇದರಂತೆಯೇ ಅವನು ಅವ್ಯಯ, ಯಾವ ಬದಲಾವಣೆಗೂ ಸಿಕ್ಕಿಲ್ಲದೆ ಇರುವನು. ಗುಣಕ್ಕೆ ತಕ್ಕಂತೆ ಮನುಷ್ಯ ಕರ್ಮವನ್ನು ಆರಿಸಿಕೊಳ್ಳುತ್ತಾನೆ. ಕರ್ಮಕ್ಕೆ ತಕ್ಕಂತೆ ಅವನಿಗೆ ಫಲಾಫಲಗಳು ಬರುತ್ತವೆ. ಆ ಫಲಕ್ಕೆ ಕೈ ಒಡ್ಡಿದರೆ ಮನುಷ್ಯ ಬದ್ಧನಾಗುತ್ತಾನೆ. ಕೇವಲ ಸಂಸ್ಕಾರ ನಾಶವಾಗಲಿ ಎಂದು ನಿಸ್ಸಂಗನಾಗಿ ಫಲಾಪೇಕ್ಷೆಯನ್ನು ಬಿಟ್ಟು ಕರ್ಮ ಮಾಡಿದರೆ ಅವನು ಬಂಧಮುಕ್ತನಾಗುತ್ತಾನೆ. ಇದಕ್ಕೆಲ್ಲ ಒಂದು ಅವಕಾಶವನ್ನು ಕಲ್ಪಿಸುವವನು ಭಗವಂತ. ಇದಕ್ಕಿಂತ ಬೇರೆ ಯಾವ ವಿಧವಾದ ಆಸಕ್ತಿಯೂ ಅವನಲ್ಲಿ ಇಲ್ಲ.

\begin{shloka}
ನ ಮಾಂ ಕರ್ಮಾಣಿ ಲಿಂಪಂತಿ ನ ಮೇ ಕರ್ಮಫಲೇ ಸ್ಪೃಹಾ~।\\ಇತಿ ಮಾಂ ಯೋಽಭಿಜಾನಾತಿ ಕರ್ಮಭಿರ್ನ ಸ ಬಧ್ಯತೇ \hfill॥ ೧೪~॥
\end{shloka}

\begin{artha}
ನನಗೆ ಕರ್ಮಗಳು ಅಂಟುವುದಿಲ್ಲ. ನನಗೆ ಕರ್ಮಫಲದಲ್ಲಿ ಆಸೆಯಿಲ್ಲ. ಹೀಗೆಂದು ಯಾವನು ನನ್ನನ್ನು ಅರಿತುಕೊಳ್ಳುವನೋ ಅವನು ಕರ್ಮಗಳಿಂದ ಬಾಧಿತನಾಗುವುದಿಲ್ಲ.
\end{artha}

ಶ‍್ರೀಕೃಷ್ಣ ಕರ್ಮವನ್ನು ಮಾಡುತ್ತಾನೆ. ಆದರೆ ಕರ್ಮ ಅವನಿಗೆ ಅಂಟುವುದಿಲ್ಲ. ಅವನು ಅನಾಸಕ್ತನಾಗಿರುವನು. ನೀರು ತಾವರೆಯ ಎಲೆಯ ಮೇಲಿದೆ. ಆದರೆ ನೀರು ಎಲೆಗೆ ಅಂಟಿಕೊಂಡಿಲ್ಲ. ಶ‍್ರೀರಾಮಕೃಷ್ಣರು ಕೆಸರಿನ ಮೀನಿನಂತೆ ಇರಿ ಎಂದು ಹೇಳುತ್ತಿದ್ದರು. ಮೀನು ಕೆಸರಿನಲ್ಲಿದೆ. ಆದರೆ ಕೆಸರು ಅದಕ್ಕೆ ಅಂಟಿಕೊಳ್ಳುವುದಿಲ್ಲ. ಆ ಮೀನು ಅಷ್ಟು ನುಣುಪಾಗಿದೆ. ಅದರಿಂದ ಎಲ್ಲಾ ಜಾರಿಹೋಗುವುದು. ಬಾತಿಗೆ ನೀರು ಸುರಿದರೆ ಹೇಗೋ ಹಾಗೆ. ಅದು ಪುಕ್ಕವನ್ನು ಒಂದು ಸಲ ಕೊಡವಿದರೆ ಸಾಕು. ನೀರೆಲ್ಲ ಒದರಿ ಹೋಗುವುದು. ಒಂದು ತೊಟ್ಟೂ ಅದರ ದೇಹದ ಮೇಲೆ ಇರುವುದಿಲ್ಲ. ಶ‍್ರೀಕೃಷ್ಣ ಬೇಕಾದಷ್ಟು ಕರ್ಮಗಳನ್ನು ಮಾಡಿದ. ಆದರೆ ಯಾವುದರ ಮೇಲೆಯೂ ಆಸಕ್ತನಾಗಿರಲಿಲ್ಲ.

ಕರ್ಮಫಲದಲ್ಲಿ ನನಗೆ ಆಸೆ ಇಲ್ಲ ಎನ್ನುವನು ಶ‍್ರೀಕೃಷ್ಣ. ಕರ್ಮ ಮಾಡುವುದಕ್ಕೆ ಮಾತ್ರ ನಿನಗೆ ಅಧಿಕಾರ, ಅದರಿಂದ ಬರುವ ಫಲಗಳಿಗಲ್ಲ ಎಂದು ಸಾರುವನು. ಆ ಬೋಧನೆಗೆ ಒಂದು ಉದಾಹರಣೆ ಬೇಕಾದರೆ ಅದು ಶ‍್ರೀಕೃಷ್ಣ ಜೀವನದಲ್ಲಿಯೇ ಇದೆ. ಅವನಷ್ಟು ಕರ್ಮ ಮಾಡಿದವನು ಅಪರೂಪ. ಎಷ್ಟೋ ಯುದ್ಧ ಮಾಡಿದ. ಹಲವರನ್ನು ಕೊಂದ, ಸೋಲಿಸಿದ. ಆದರೆ ಯಾವಾಗಲೂ ಅದರಿಂದ ಬರುವ ಫಲಕ್ಕೆ ಅವನು ಕೈಯೊಡ್ಡಲಿಲ್ಲ. ಕಂಸನನ್ನು ಕೊಂದಾದಮೇಲೆ ತಾನು ಸಿಂಹಾಸನವನ್ನು ಏರಲಿಲ್ಲ. ಕಂಸನ ತಂದೆಯಾದ ಉಗ್ರಸೇನನ್ನು ಸಿಂಹಾಸನದ ಮೇಲೆ ಕುಳ್ಳಿರಿಸಿದ. ಶಿಶುಪಾಲ, ಜರಾಸಂಧ ಮುಂತಾದವರನ್ನು ಕೊಂದ. ಅವರ ರಾಜ್ಯವನ್ನು ತಾನು ತೆಗೆದುಕೊಳ್ಳಲಿಲ್ಲ. ಅದಕ್ಕೆ ಉತ್ತರಾಧಿಕಾರಿಗಳನ್ನು ವಿಚಾರಿಸಿ ಅವರವರ ರಾಜ್ಯಗಳನ್ನು ಅವರವರಿಗೆ ಕೊಡಿಸಿದ. ಫಲದ ಕಡೆ ಶ‍್ರೀಕೃಷ್ಣ ಗಮನವನ್ನೇ ಕೊಡುವುದಿಲ್ಲ. ಒಂದು ಕೆಲಸ ಮಾಡಬೇಕಾಗಿದೆ. ಏಕೆಂದರೆ ಅದು ಕ್ಷತ್ರಿಯನ ಕರ್ತವ್ಯ. ಆದರೆ ಕೆಲಸ ಮಾಡಿದರೆ ಫಲಕ್ಕೂ ನನಗೆ ಅಧಿಕಾರವಿದೆ ಎಂದು ಭಾವಿಸುವುದಿಲ್ಲ. ಫಲಾಸಕ್ತಿಯೇ ಮಾಡುವ ಕರ್ಮದಲ್ಲಿರುವ ವಿಷ. ಹಾವಿನ ಬಾಯೊಳಗೆ ವಿಷದ ಹಲ್ಲು ಇರುವಂತೆ. ಆ ವಿಷದ ಹಲ್ಲನ್ನು ಕಿತ್ತಾದ ಮೇಲೆ ಆ ಹಾವಿನೊಡನೆ ಅವನು ಎಷ್ಟು ಬೇಕಾದರೂ ಸದರದಿಂದ ಇರಬಹುದು. ಇನ್ನು ಮೇಲೆ ಅದರಿಂದ ಅಪಾಯವಿಲ್ಲ.

ಶ‍್ರೀಕೃಷ್ಣ ಇಲ್ಲಿ ಮತ್ತೊಂದು ಸಂದೇಶವನ್ನು ತರುವನು. ಈ ಪ್ರಕಾರ ಯಾರು ನನ್ನನ್ನು ತಿಳಿದುಕೊಳ್ಳುವರೋ ಅವರು ಕೂಡ ಕರ್ಮಬಂಧನದಿಂದ ಬಿಡುಗಡೆ ಹೊಂದುವರು ಎನ್ನುವನು. ಶ‍್ರೀಕೃಷ್ಣ ಫಲಾಸಕ್ತಿಯಿಲ್ಲದೆ ಕರ್ಮ ಮಾಡುತ್ತಿದ್ದ ಎಂದು ನಾವು ತಿಳಿದರೆ, ನಮಗೆ ಬಿಡುಗಡೆ ಹೇಗೆ ಬರಬಲ್ಲುದು ಎಂದು ನಾವು ಆಶ್ಚರ್ಯಪಡಬಹುದು. ಆದರೆ ಅದನ್ನು ಸುಮ್ಮನೆ ಬೌದ್ಧಿಕವಾಗಿ ತಿಳಿದುಕೊಳ್ಳುವುದಲ್ಲ. ಶ‍್ರೀಕೃಷ್ಣನ ಜೀವನವನ್ನು ನಾವು ಒಂದು ಉದಾಹರಣೆಯಾಗಿ ಇಟ್ಟುಕೊಂಡು ಅವನಿಗೆ ಕರ್ಮದ ಮೇಲೆ ಇರುವ ಅನಾಸಕ್ತಿ, ಮತ್ತು ಅದರಿಂದ ಬರುವ ಫಲಗಳಿಗೆ ಗಮನ ಕೊಡದೆ ಇರುವುದು, ಇವನ್ನು ಚಿಂತಿಸುತ್ತಿದ್ದರೆ, ಮನನ ಮಾಡುತ್ತಿದ್ದರೆ ಆ ಭಾವ, ಆ ಗುಣ ನಮಗೂ ಬರುವುದು. ನಾವು ಹೇಗೆ ಆಲೋಚಿಸುತ್ತೇವೆಯೋ ಹಾಗೆ ಆಗುವೆವು. ನಾವು ಮನನ ಮಾಡುವ ವಸ್ತುವಿನ ಗುಣ ನಮಗೆ ಬರುವುದು. ಇದೊಂದು ಮನಃಶಾಸ್ತ್ರದ ರಹಸ್ಯ. ಪ್ರತ್ಯಕ್ಷವಾಗಿ ನಾವೇ ಆ ಗುಣಗಳನ್ನು ನಮ್ಮ ಜೀವನದಲ್ಲಿ ರೂಢಿಸಿಕೊಳ್ಳುವುದಕ್ಕೆ ಕಷ್ಟವಾಗಬಹುದು. ಆದರೆ ಯಾರ ಜೀವನದಲ್ಲಿ ಇದು ಜ್ವಲಿಸುತ್ತಿದೆಯೋ ಅವನನ್ನು ಕುರಿತು ಚಿಂತಿಸುತ್ತಿದ್ದರೂ ನಮ್ಮ ಅರಿವಿಲ್ಲದೆ ನಮ್ಮ ಜೀವನದ ಮೇಲೆ ಅದು ತನ್ನ ಪ್ರಭಾವವನ್ನು ಬೀರುವುದು. ಗಂಧದ ಎಣ್ಣೆ ಕಾರ್ಖಾನೆಗೆ ನಾವು ಹೋಗಿಬಂದರೆ ನಾವು ಬೇಡವೆಂದರೂ ಸ್ವಲ್ಪ ಆ ಪರಿಮಳ ನಮಗೆ ಅಂಟಿಕೊಳ್ಳುವುದು. ಅದರಂತೆಯೇ ಶ‍್ರೀಕೃಷ್ಣನಂತಹ ಮಹದ್​ವ್ಯಕ್ತಿಗಳು ಯಾವ ಗುಣಕ್ಕೆ ಪ್ರಖ್ಯಾತರಾಗಿರುವರೋ ಅವರ ಜೀವನವನ್ನು ಚಿಂತಿಸುತ್ತಿದ್ದರೆ ಆ ಗುಣಗಳು ನಮಗೂ ಬರುವುವು.

\begin{shloka}
ಏವಂ ಜ್ಞಾತ್ವಾ ಕೃತಂ ಕರ್ಮ ಪೂರ್ವೈರಪಿ ಮುಮುಕ್ಷುಭಿಃ~।\\ಕುರು ಕರ್ಮೈವ ತಸ್ಮಾತ್ ತ್ವಂ ಪೂರ್ವೈಃ ಪೂರ್ವತರಂ ಕೃತಮ್ \hfill॥ ೧೫~॥
\end{shloka}

\begin{artha}
ಹೀಗೆ ತಿಳಿದುಕೊಂಡು ಹಿಂದಿನ ಕಾಲದ ಮುಮುಕ್ಷುಗಳು ಕರ್ಮವನ್ನು ಮಾಡುತ್ತಿದ್ದರು. ನೀನೂ ಕೂಡ ಹಿಂದಿನವರು ಮಾಡಿದಂತೆ ಕರ್ಮವನ್ನು ಮಾಡು.
\end{artha}

ಕರ್ಮ ಅಲ್ಲ ನಮ್ಮನ್ನು ಕಟ್ಟಿಹಾಕುವುದು, ಅದರ ಹಿಂದೆ ಇರುವ ಫಲಾಸಕ್ತಿ. ಎರಡೂ ಬೆರೆತುಹೋಗಿದೆ ಸಾಧಾರಣ ಮನುಷ್ಯನ ಜೀವನದಲ್ಲಿ. ಏನಾದರೂ ವ್ಯಥೆ ಬಂದರೆ ಕರ್ಮದಿಂದಲೇ ಹಾಗಾಯಿತು ಎನ್ನುವನು. ಆತ ಕರ್ಮ\-ದಿಂದಲ್ಲ, ಅದರ ಕೊನೆಯಲ್ಲಿ ಬಂದ ಫಲಾಸಕ್ತಿಯಿಂದ ಎಂಬುದನ್ನು ಅರಿಯಲಾರ. ಶ‍್ರೀಕೃಷ್ಣ ಇಲ್ಲಿ ಇದನ್ನು ವಿಭಜನೆ ಮಾಡುವನು. ಕರ್ಮದಲ್ಲಿ\break ಫಲಾಸಕ್ತಿ ಬೆರೆಯದಂತೆ ನೋಡಿಕೊಳ್ಳುವನು. ಆಗ ಕರ್ಮದಿಂದ ಅವನು ಬಾಧಿತನಾಗುವುದಿಲ್ಲ. ನಾವು ಅನಾಸಕ್ತಿಯನ್ನು ಎರಡು ದೃಷ್ಟಿಯಿಂದ ರೂಢಿಸಬಹುದು. ಒಂದು ಜ್ಞಾನದೃಷ್ಟಿ, ಮತ್ತೊಂದು ಭಕ್ತಿಯ ದೃಷ್ಟಿ. ಜ್ಞಾನದ ದೃಷ್ಟಿಯಲ್ಲಾದರೊ ಗುಣಗಳು ಹೊರಗೆ ಇರುವ ವಿಷಯಗಳೊಂದಿಗೆ ಸೇರಿ ಕರ್ಮವಾಗುತ್ತವೆ, ನಾನು ಕೇವಲ ಸಾಕ್ಷಿ, ಅದರಿಂದ ಬಾಧಿತನಾಗದವನು ಎಂದು ಅರಿಯುತ್ತಾನೆ. ಭಕ್ತನಾದರೊ, ನಾನೊಂದು ನಿಮಿತ್ತ, ಭಗವಂತನೇ ಎಲ್ಲವನ್ನೂ ಮಾಡುವವನು, ಅವನ ಕೆಲಸವನ್ನು ನನ್ನ ಮೂಲಕ ಮಾಡುತ್ತಾನೆ, ಅದರಿಂದ ಬರುವ ಫಲಾಫಲಗಳಿಗೆ ನಾನಲ್ಲ ಹಕ್ಕುದಾರ ಎಂಬುದನ್ನು ಚೆನ್ನಾಗಿ ಅರಿತವನು. ಇದನ್ನು ಅರಿಯದ ಮೂಢರು ಈ ಕೆಲಸವನ್ನು ನೋಡಿ ಮಾಡುವವನನ್ನು ಹೊಗಳಬಹುದು. ಆದರೆ ಭಗವಂತನ ಕೈಯಲ್ಲಿ ನಿಮಿತ್ತವಾಗಿರುವ ಭಕ್ತ\-ನಾದರೋ, ಜನ ಕೊಡುವ ಕೀರ್ತಿಯನ್ನು ಯಾರಿಗೆ ಸಲ್ಲಬೇಕೋ ಅವರಿಗೆ ಕೊಟ್ಟು ಧನ್ಯನಾಗುವನು. ಕೆರೆಯ ನೀರನು ಕೆರೆಗೆ ಚೆಲ್ಲಿ ಧನ್ಯರಾಗಿರೊ ಎಂದು ದಾಸರು ಹಾಡಿದಂತೆ.

ಅನಾಸಕ್ತನಾಗಿ ಕರ್ಮ ಮಾಡುತ್ತಿದ್ದುದಕ್ಕೆ ಶ‍್ರೀಕೃಷ್ಣನೇ ಒಂದು ಉದಾಹರಣೆ. ಆದರೆ ಅವನೇ ಮೊದಲಿಗನಲ್ಲ. ಶ‍್ರೀಕೃಷ್ಣನು ಈ ಪರಂಪರೆಗೆ ಸೇರಿದವನು. ಹಿಂದೆ ಎಷ್ಟೋ ಜನ ಹಾಗೆ ಕರ್ಮ ಮಾಡಿದವರು ಇದ್ದಾರೆ ಎಂದು ಹೇಳುವನು. ಜನಕ, ವಿವಸ್ವಂತ, ಮನು ಮುಂತಾದವರೆ ಅವರು. ಶ‍್ರೀಕೃಷ್ಣ ಎಷ್ಟೇ ಯುಕ್ತಿಯನ್ನು ಬಳಸಿದರೂ ಹಿಂದಿನವರ ಉದಾಹರಣೆಯನ್ನು ಮರೆಯುವುದಿಲ್ಲ. ಬರೀ ಯುಕ್ತಿಯನ್ನು ನಂಬಿಹೋದರೆ, ಬಹುಶಃ ನಾವೆ ಮೊದಲಿಗರಿರಬಹುದು, ಇದು ಸರಿಯೋ ಅಲ್ಲವೋ ಎಂಬ ಸಂದೇಹಗಳು ಬರುವುದು ಸಹಜ. ಆದರೆ ಆ ದಾರಿಯಲ್ಲಿ ಜನ ಹೋಗಿದ್ದರು ಎಂಬುದನ್ನು ನೋಡಿದಾಗ ನಮ್ಮ ಸಂದೇಹ ಪರಿಹಾರವಾಗುವುದು. ಶ‍್ರೀಕೃಷ್ಣ ಹಿಂದಿನವರ ಅನು ಭವಕ್ಕೆ ಸ್ಥಾನವನ್ನು ಕೊಡುವನು. ಯುಕ್ತಿ ಆಪ್ತವಾಕ್ಯವಾಗುವುದು ಇತರರಿಗೆ. ಯುಕ್ತಿಗಿಂತ ಹೆಚ್ಚಾಗಿ ಒಂದು ವಿಷಯದಲ್ಲಿ ನಂಬಿಕೆ ಹುಟ್ಟಿಸುವುದು ಇದು.

ಅನಂತರ ಅರ್ಜುನನಿಗೆ ನೀನೂ ಕೂಡ ಹಿಂದಿನ ಪ್ರಖ್ಯಾತ ಮುಮುಕ್ಷುಗಳು ಹೇಗೆ ಕರ್ಮವನ್ನು ಮಾಡಿದರೋ ಹಾಗೆಯೇ ಮಾಡು ಎನ್ನುತ್ತಾನೆ. ಇಲ್ಲಿ ಹಾಗೆ ಮಾಡುವುದಕ್ಕೆ ಅಂಜಿಕೆ ಇರುವುದಿಲ್ಲ. ಅನುಮಾನ ಇರುವುದಿಲ್ಲ. ಇತರರೂ ಹಾಗೆ ಮಾಡಿದ್ದಾರೆ ಎಂಬ ಧೈರ್ಯ ಬರುವುದು. ಒಬ್ಬ ಇನ್ನೂ ಸಾಧಕನಾಗಿದ್ದರೆ ಕರ್ಮವನ್ನು ತನ್ನ ಚಿತ್ತಶುದ್ಧಿಗಾಗಿ ಫಲಾಪೇಕ್ಷೆ ಇಲ್ಲದೆ ಮಾಡಲಿ. ಆತ ಮುಕ್ತಾತ್ಮನಾಗಿದ್ದರೆ ಲೋಕಸಂಗ್ರಹದ ದೃಷ್ಟಿಯಿಂದಲಾದರೂ ಮಾಡಲಿ. ಇವನನ್ನು ನೋಡಿ ಜನ ಹೇಗೆ ಫಲಾಪೇಕ್ಷೆ ಇಲ್ಲದೆ ಕರ್ಮವನ್ನು ಮಾಡುವುದು ಎಂಬುದನ್ನು ಕಲಿಯಲಿ. ಅಂತೂ ಒಬ್ಬ ಕರ್ಮವನ್ನು ಮಾಡುತ್ತಲೇ ಇರಬೇಕು. ಅದರಿಂದ ತಪ್ಪಿಸಿಕೊಂಡು ಹೋಗುವುದಕ್ಕೆ ಆಗುವುದಿಲ್ಲ. ಒಬ್ಬ ತಾನು ಮಾಡಬೇಕು. ಇತರರ ಕೈಯಿಂದಲೂ ಮಾಡಿಸಬೇಕು. ಇತರರಿಗೆ ಕರ್ಮವನ್ನು ಮಾಡಿ ಎಂದರೆ ಅವರು ಕೇಳುವುದಿಲ್ಲ. ಅವರು ಕಣ್ಣಾರೆ ನೋಡಬೇಕು. ಆಗಲೇ ಅವರು ಕಾರ್ಯೋನ್ಮುಖರಾಗ ಬೇಕಾದರೆ.

\begin{shloka}
ಕಿಂ ಕರ್ಮ ಕಿಮಕರ್ಮೇತಿ ಕವಯೋಽಪ್ಯತ್ರ ಮೋಹಿತಾಃ~।\\ತತ್ತೇ ಕರ್ಮ ಪ್ರವಕ್ಷ್ಯಾಮಿ ಯಜ್ಜ್ಞಾತ್ವಾ ಮೋಕ್ಷ್ಯಸೇಽಶುಭಾತ್ \hfill॥ ೧೬~॥
\end{shloka}

\begin{artha}
ಕರ್ಮ ಯಾವುದು, ಅಕರ್ಮ ಯಾವುದು ಎಂಬ ವಿಷಯದಲ್ಲಿ ವಿದ್ವಾಂಸರೇ ಭ್ರಾಂತರಾಗಿದ್ದಾರೆ. ಆದುದರಿಂದ ನಿನಗೆ ಕರ್ಮವನ್ನು ಹೇಳುತ್ತೇನೆ. ಅದನ್ನು ತಿಳಿದುಕೊಂಡರೆ ಅಶುಭದಿಂದ ಪಾರಾಗುವೆ.
\end{artha}

ಈ ಪ್ರಪಂಚದಲ್ಲಿ ಒಬ್ಬ ಬಹಳ ವಿದ್ಯಾವಂತನಾಗಿರಬಹುದು. ಆದರೆ ಯಾವ ಕೆಲಸವನ್ನು ಮಾಡಬೇಕು, ಯಾವುದನ್ನು ಬಿಡಬೇಕು ಎಂಬ ವಿಚಾರ ಅನೇಕ ವೇಳೆ ಅವನಿಗೆ ಗೊತ್ತಿರುವುದಿಲ್ಲ. ಒಳ್ಳೆಯ ಕೆಲಸವನ್ನು ಮಾಡುವನು. ಅದರಿಂದ ಬರುವ ಫಲಾಫಲದ ಗೋಜಿಗೆ ಸಿಕ್ಕಿಕೊಂಡು ನರಳುವನು. ಆಗ ಕರ್ಮ ಮಾಡಿದ್ದರಿಂದಲೇ ಇದೆಲ್ಲಾ ಆಯಿತು ಎಂದು ಮತ್ತೊಮ್ಮೆ ಆ ಕರ್ಮವನ್ನು ಮಾಡಲು ಹೋಗುವುದಿಲ್ಲ. ಒಂದು ಸಲ ತುಂಬಾ ಬಿಸಿಹಾಲನ್ನು ಕುಡಿದು ಬಾಯಿಯನ್ನು ಸುಟ್ಟುಕೊಂಡ ಬೆಕ್ಕು ಆರಿದ ಹಾಲಿಗೂ ಅಂಜುವಂತೆ ಅದರ ಹತ್ತಿರವೇ ಬರುವುದಿಲ್ಲ. ಇಲ್ಲಿ ಹಾಲಿನದಲ್ಲ ತಪ್ಪು. ಅದರಲ್ಲಿರುವ ಬಿಸಿ ನಮ್ಮ ಬಾಯನ್ನು ಸುಟ್ಟಿದ್ದು. ಕರ್ಮ ಮಾಡುವುದು ದೊಡ್ಡದಲ್ಲ. ಅದನ್ನು ಸರಿಯಾದ ದೃಷ್ಟಿಯಿಂದ ಮಾಡಬೇಕು. ಪ್ರತಿಫಲಾಪೇಕ್ಷೆ ಇಲ್ಲದೆ ಇರಬೇಕು. ಅನಾಸಕ್ತನಾಗಿರಬೇಕು. ಇವೆರಡೂ ಮಾಡುವ ಕೆಲಸದ ಹಿಂದೆ ಇದ್ದರೆ ನಾವು ಕೆಲಸವನ್ನು ಮಾಡಿದರೂ ಅವುಗಳ ಪರಿಣಾಮಕ್ಕೆ ಬೀಳುವುದಿಲ್ಲ.

ಕೆಲಸ ಮಾಡದೇ ಇರುವುದೇ ಅಕರ್ಮ ಎಂದು ನಾವು ಭಾವಿಸುತ್ತೇವೆ. ಶ‍್ರೀಕೃಷ್ಣನೇ ಹೇಳುತ್ತಾನೆ ಯಾರೂ ಒಂದು ಕ್ಷಣವೂ ಕರ್ಮವಿಲ್ಲದೆ ಇರುವುದಕ್ಕೆ ಆಗುವುದಿಲ್ಲ ಎಂದು. ಬೇಕಾದರೆ ಅವನು ತಾನು ಮಾಡಬೇಕಾದ ಕರ್ಮವನ್ನು ಮಾಡದೆ ಅನಾವಶ್ಯಕವಾದ ಕರ್ಮಗಳನ್ನು ಮಾಡುತ್ತಿರಬಹುದು. ಹೊರಗೆ ಕರ್ಮ ಮಾಡದವನಂತೆ ನಟಿಸುತ್ತ ಮನಸ್ಸಿನೊಳಗೆ ಅದಕ್ಕೆ ಸಂಬಂಧಪಟ್ಟ ಆಲೋಚನೆಗಳನ್ನೆಲ್ಲಾ ಮಾಡುತ್ತಿರಬಹುದು. ಇದು ಮಿಥ್ಯಾಚಾರ ಎಂದು ಶ‍್ರೀಕೃಷ್ಣ ಹೇಳುತ್ತಾನೆ. ಹೊರಗಿನಿಂದ ನೋಡಿದರೆ ಸುಮ್ಮನೆ ಕುಳಿತಿರುವನು. ಆದರೆ ಮನಸ್ಸು ಆಸೆಗಳಿಂದ ಅಲ್ಲೋಲ ಕಲ್ಲೋಲವಾಗಿದೆ. ಕೆಲಸವನ್ನು ಮಾಡುವುದಕ್ಕೆ ಮುಂದೆ ಬರುತ್ತಿಲ್ಲ. ಅದಕ್ಕೆ ಹಲವು ಕಾರಣಗಳಿವೆ. ಅದನ್ನು ಸಮಾಜದ ಎದುರಿಗೆ ಮಾಡುವುದಕ್ಕೆ ಅವನಿಗೆ ನಾಚಿಕೆ. ಎರಡನೆಯದೇ ಅದನ್ನು ಮಾಡಬೇಕಾದರೆ ತುಂಬಾ ಕಷ್ಟಪಡಬೇಕಾಗಿದೆ. ಸುಲಭವಾಗಿ ಬಿಟ್ಟಿ ಆ ವಸ್ತು ಪ್ರಪಂಚದಲ್ಲಿ ಸಿಕ್ಕುವುದಿಲ್ಲ. ಇವುಗಳ ಕಾರಣದಿಂದ ಅವನು ಕೆಲಸವನ್ನು ಮಾಡದೆ ಸುಮ್ಮನೆ ಇರಬಹುದು. ಇಲ್ಲಿ ಒಬ್ಬ ಕೆಲಸವನ್ನು ಮಾಡದೆ ಇರುವುದನ್ನು ಮಾತ್ರ ತೆಗೆದುಕೊಂಡರೆ ಸಾಲದು. ಅವನು ಯಾವ ಉದ್ದೇಶದಿಂದ ಕೆಲಸ ಮಾಡುವು ದಿಲ್ಲವೋ ಅದನ್ನು ತೆಗೆದುಕೊಳ್ಳಬೇಕಾಗಿದೆ.

ಶ‍್ರೀಕೃಷ್ಣ ಯಾವುದನ್ನು ತಿಳಿದುಕೊಂಡು ಅಶುಭದಿಂದ ಪಾರಾಗುವೆಯೋ ಅದನ್ನು ಹೇಳುತ್ತೇನೆ ಎನ್ನುತ್ತಾನೆ. ಇಲ್ಲಿ ಅಶುಭ ಎಂದರೆ ಪುನಃ ಕರ್ಮದ ಉಪಟಳದ ಕೈಯಲ್ಲಿ ಸಿಕ್ಕಿಕೊಂಡು ನರಳುವುದು. ಕರ್ಮ ಮಾಡಿಯೂ ಅದರ ಪರಿಣಾಮಕ್ಕೆ ಬೀಳದಂತೆ ಮಾಡುವುದು. ಇದೇ ರಹಸ್ಯ. ಒಂದು ರೀತಿ ಕರ್ಮ ಮಾಡಿದರೆ ನಾವು ಬಂಧನಕ್ಕೆ ಬೀಳುವೆವು. ಮತ್ತೊಂದು ರೀತಿ ಕರ್ಮ ಮಾಡಿದರೆ ಬಂಧನದಿಂದ ಪಾರಾಗುವೆವು. ರೇಷ್ಮೆಹುಳು ತನ್ನ ಬಾಯಿನಿಂದಲೇ ಗೂಡನ್ನು ನೇಯ್ದು ಅದರಲ್ಲಿ ಬಂದಿಯಾಗುವುದು. ಅನಂತರ ತನ್ನ ಬಾಯಿನಿಂದಲೇ ಗೂಡನ್ನು ಕೊರೆದುಕೊಂಡು ಹೊರಬರುವುದು. ಎರಡನ್ನೂ ತಾನೇ ಮಾಡಿಕೊಂಡಿತು. ಬಂಧನ ಮೋಕ್ಷ ಎರಡೂ ಕರ್ಮಕ್ಕೆ ಸಾಧ್ಯ. ಇದು ಒಂದು ಸ್ಕ್ರೂಡ್ರೈವರಿನಿಂದ ಒಂದು ರೀತಿ ಮೊಳೆಯನ್ನು ತಿರುಗಿಸಿದರೆ ಮೊಳೆ ಒಳಗೆ ಹೋಗುವುದು. ಮತ್ತೊಂದು ರೀತಿ ತಿರುಗಿಸಿದರೆ ಅದು ಮೇಲೆ ಬರುವುದು. ಆಗುವ ಕ್ರಿಯೆ ಅದನ್ನು ತಿರುಗಿಸುವ ವಿಧಾನದಲ್ಲಿದೆ.

\begin{shloka}
ಕರ್ಮಣೋ ಹ್ಯಪಿ ಬೋದ್ಧವ್ಯಂ ಬೋದ್ಧವ್ಯಂ ಚ ವಿಕರ್ಮಣಃ~।\\ಅಕರ್ಮಣಶ್ಚ ಬೋದ್ಧವ್ಯಂ ಗಹನಾ ಕರ್ಮಣೋ ಗತಿಃ \hfill॥ ೧೭~॥
\end{shloka}

\begin{artha}
ಕರ್ಮ, ವಿಕರ್ಮ, ಅಕರ್ಮ ಇವುಗಳ ವಿಷಯದಲ್ಲಿ ತಿಳಿದುಕೊಳ್ಳಬೇಕಾಗಿದೆ. ಏಕೆಂದರೆ ಕರ್ಮದ ಗತಿ ತುಂಬಾ ಗಹನವಾಗಿದೆ.
\end{artha}

ಇಲ್ಲಿ ಶ‍್ರೀಕೃಷ್ಣ ಕೆಲವು ಪದಗಳನ್ನು ಬಳಸುತ್ತಾನೆ. ಇವು ನೋಡುವುದಕ್ಕೆ ಬಹಳ ಸರಳವಾಗಿ ಕಾಣುತ್ತವೆ. ಆದರೆ ಅರ್ಥ ಬಹಳ ಸೂಕ್ಷ್ಮವಾಗಿರುವುದು. ಕರ್ಮ ಎಂದರೆ ಕೆಲಸ ಮಾಡುವುದು. ಮನುಷ್ಯನಲ್ಲಿ ಗುಣಗಳಿವೆ. ಅವು ಇರುವ ತನಕ ಅವನು ಸುಮ್ಮನೆ ಇರುವುದಕ್ಕಾಗುವುದಿಲ್ಲ. ಏನಾದರೂ ಮಾಡುತ್ತಲೇ ಇರುತ್ತಾನೆ. ಲಾಭಕ್ಕೆ, ಕೀರ್ತಿಗೆ, ಅಧಿಕಾರಕ್ಕೆ ಅವನು ಕೆಲಸಗಳನ್ನು ಮಾಡುತ್ತಾ ಇರುತ್ತಾನೆ. ಇದರಿಂದ ಬರುವ ಫಲಾಫಲಗಳಿಗೆ ತುತ್ತಾಗುತ್ತಾನೆ. ಇವನು ಬಯಸುವುದು ಬರುವುದು. ಜೊತೆಗೆ ಇವನು ಬಯಸದೆ ಇರುವುದೂ ಅದರ ನೆರಳಿನಂತೆ ಇವನು ಬೇಡವೆಂದರೂ ಬರುವುದು. ಕೀರ್ತಿಯ ಜೊತೆಯಲ್ಲಿಯೇ ಅಪಕೀರ್ತಿಯ ಭಯ. ಅಧಿಕಾರದ ಜೊತೆಯಲ್ಲಿ ಅದನ್ನು ಕಳೆದುಕೊಳ್ಳುವ ಭಯ. ಲಾಭ ಬಂದರೆ ಮುಂದೆ ಎಲ್ಲಿ ನಷ್ಟ ಬರುವುದೋ ಎಂಬ ಕಳವಳ. ಅಂತೂ ಒಂದು ಬಂದರೆ ಅದರ ವಿರೋಧವೂ ಬೇಡವೆಂದರೂ ಬರುವುದು.\break ಒಂದಕ್ಕೆ ಸಂತೋಷಪಟ್ಟರೆ ಮತ್ತೊಂದಕ್ಕೆ ದುಃಖವನ್ನೂ ಪಡಬೇಕಾಗುವುದು.

ಎರಡನೆಯದೆ ವಿಕರ್ಮ. ಎಂದರೆ ನಾವು ಯಾವ ಕೆಲಸವನ್ನು ಮಾಡಬೇಕೋ ಅದನ್ನು ಮಾಡುವು ದಿಲ್ಲ. ಮತ್ತೊಂದು ಕರ್ಮವನ್ನು ಮಾಡುವೆವು. ಕ್ಷತ್ರಿಯ ತಾನು ಯುದ್ಧ ಮಾಡುವುದಕ್ಕೆ ಮತ್ತು ಆಳುವುದಕ್ಕೆ ತರಬೇತನ್ನು ತೆಗೆದುಕೊಂಡಿರುವನು. ಇಲ್ಲಿ ಅದನ್ನು ಬಿಡುವನು. ಮತ್ತೊಬ್ಬನ ಕರ್ಮವನ್ನು ಮಾಡಲು ಹೋಗುವನು. ಧ್ಯಾನ ತಪಸ್ಸು ಅಧ್ಯಯನ ಇವುಗಳನ್ನು ಮಾಡಲು ಹೋಗುವನು. ಸಮಾಜಕ್ಕೆ ವಿರೋಧವಾದ ಕಳ್ಳತನ ದರೋಡೆ, ಸುಳ್ಳು ಹೇಳುವುದು ಮುಂತಾದ ಶಾಸ್ತ್ರಕ್ಕೆ ವಿರೋಧವಾದ ಕರ್ಮಗಳೂ ಕೂಡ ಈ ಗುಂಪಿಗೆ ಬರುವುವು. ನಾವು ಮಾಡುವ ಕೆಲಸವನ್ನು ಬಿಟ್ಟು ಇನ್ನೊಬ್ಬನ ಕೆಲಸಕ್ಕೆ ಕೈಹಾಕುವುದು ಮತ್ತು ಸಮಾಜ ಕಂಟಕಗಳಾದ ಕೆಲಸಗಳನ್ನು ಮಾಡುವುದು ಇವೆರಡೂ ಒಂದೇ ದರ್ಜೆಗೆ ಸೇರಿದ ವಿಕರ್ಮಗಳು.

ಮೂರನೆಯದೆ ಅಕರ್ಮ ಎಂಬುದೊಂದು ಇದೆ ಎನ್ನುವನು ಶ‍್ರೀಕೃಷ್ಣ. ಅಕರ್ಮ ಎಂದರೆ ಸುಮ್ಮನಿರುವುದು ಎಂದಲ್ಲ. ಕರ್ಮದಲ್ಲಿ ನಿರತನಾಗಿರುವಾಗಲೂ ಅದರ ಪಾಶಕ್ಕೆ ಸಿಕ್ಕದೆ ಕರ್ಮ ಮಾಡುವುದು. ಇದೇನೆ ಶ‍್ರೀಕೃಷ್ಣ ಭಗವದ್ಗೀತೆಯಲ್ಲೆಲ್ಲ ಸಾರುವ ಒಂದು ಆದರ್ಶ. ಶ‍್ರೀಕೃಷ್ಣನಷ್ಟು ಕರ್ಮವನ್ನು ಇತರರು ಯಾರೂ ಆಗಿನ ಕಾಲದಲ್ಲಿ ಮಾಡಿರಲಿಲ್ಲ. ಆದರೂ ಕರ್ಮ ಅವನ ಮನಸ್ಸಿನ ಮೇಲೆ ಯಾವ ಪರಿಣಾಮವನ್ನೂ ಉಂಟುಮಾಡಿ ಅವನನ್ನು ಬಂಧನಕ್ಕೆ ಗುರಿ ಮಾಡಲಿಲ್ಲ. ನೀರಿನ ಮೇಲೆ ಬರೆದ ಅಕ್ಷರದಂತೆ ಇತ್ತು. ಬರೆಯುವಾಗಲೇ ಅದು ಅಳಿಸಿಹೋಗುತ್ತಿತ್ತು. ಅಷ್ಟು ನಿರ್ಲಿಪ್ತವಾಗಿತ್ತು ಮನಸ್ಸು.

\newpage

ಈ ಕರ್ಮದ ಗತಿಯನ್ನು ತಿಳಿದುಕೊಳ್ಳಬೇಕಾದರೆ ತುಂಬಾ ಕಷ್ಟ ಎನ್ನುವನು. ಕಾಣುತ್ತಿರುವುದು ಒಂದು, ಅದರೊಳಗೆ ಇರುವುದು ಬೇರೊಂದು. ಒಬ್ಬ ಮನುಷ್ಯ ಸಮಾಜ ಕಲ್ಯಾಣಕ್ಕೆ ಕೆಲಸ ಮಾಡುತ್ತಿರುವನು. ಹಗಲು ರಾತ್ರಿ ಕಷ್ಟಪಡುವನು. ಕ್ರಮೇಣ ಅದರಿಂದ ಬರುವ ಹೆಸರಿಗೆ, ಅಧಿಕಾರಕ್ಕೆ ಸಿಕ್ಕಿಕೊಂಡು ನೀರಿನ ಸುಳಿಯಲ್ಲಿ ಸಿಕ್ಕಿದ ತರಗೆಲೆಯಂತೆ ಸುತ್ತಲು ಉಪಕ್ರಮಿಸುವನು. ಒಳ್ಳೆಯ ಕೆಲಸವನ್ನೇನೋ ಮಾಡಲು ಹೊರಟ. ಮಾಡುವಾಗ ಸಾಕಷ್ಟು ಜಾಗರೂಕನಾಗಿರಲಿಲ್ಲ. ಫಲಾಪೇಕ್ಷೆಯ ಕ್ರಿಮಿಗಳು ಇವನನ್ನು ಮೆಟ್ಟಿ ಪಡಬಾರದ ಕಷ್ಟಕ್ಕೆ ಸಿಕ್ಕಿಸುವುದು. ಪ್ರಾರಂಭ ಮಾಡಿದ್ದೇನೋ ಒಳ್ಳೆಯ ಕೆಲಸದಿಂದ. ಕೊನೆಗೆ ಬಿದ್ದದ್ದು ಫಲಾಪೇಕ್ಷೆಯ ಹಳ್ಳಕ್ಕೆ. ಪ್ರಾರಂಭವಾದದ್ದು ಎಲ್ಲಿ, ಕೊನೆ ಗೊಂಡಿದ್ದು ಎಲ್ಲಿ?

ಒಬ್ಬ ತನ್ನ ಕರ್ತವ್ಯವನ್ನು ಮಾಡುವಾಗ ಇನ್ನೊಬ್ಬನಿಗೆ ವ್ಯಥೆ ಕೊಡುತ್ತಾನೆ. ಕೆಲವು ವೇಳೆ ಕೊಲ್ಲುತ್ತಾನೆ. ಆದರೂ ಅವನು ಮಾಡುವುದು ಒಳ್ಳೆಯ ಕೆಲಸ ಎನ್ನುತ್ತಾರೆ. ಯೋಧ ಯುದ್ಧದಲ್ಲಿ ಶತ್ರುವನ್ನು ಕೊಲ್ಲುತ್ತಾನೆ. ನ್ಯಾಯಾಧಿಪತಿ ಅನ್ಯಾಯ ಮಾಡಿದವರನ್ನು ಶಿಕ್ಷಿಸುತ್ತಾನೆ. ವೈದ್ಯ ರೋಗಿಗೆ ಚಿಕಿತ್ಸೆ ಮಾಡುವಾಗ ನೋವನ್ನು ಕೊಡುತ್ತಾನೆ. ಹೊರಗಿನಿಂದ ನೋಡಿದರೆ ಇವೆಲ್ಲ ಕ್ರೌರ್ಯಕೃತ್ಯಗಳು. ಆದರೆ ಒಳಗೆ ಹೊಕ್ಕು ನೋಡಿದರೆ ಇವಾವುದನ್ನೂ ಅವರು ತಮಗಾಗಿ ಮಾಡಿಕೊಳ್ಳುತ್ತಿಲ್ಲ. ಇಲ್ಲಿ ಮಾಡುವ ಕರ್ಮಯಾತನಾಮಯವಾಗಿದ್ದರೂ ಕಲ್ಯಾಣಪ್ರದವಾಗಿರುವುದು.

ಕೆಲವು ವೇಳೆ ತಮ್ಮ ಬಾಲ್ಯದಲ್ಲಿ ಯಾವ ಕೆಟ್ಟ ಕೆಲಸವನ್ನೂ ಮಾಡದ ಜನರಿರುವರು. ಕೆಲವು ಕಾಲದ ಮೇಲೆ ಘೋರ ಸಮಾಜ ಘಾತುಕರಾಗುವರು. ಅವರನ್ನು ನಾವು ಸಾಧು ಸ್ವಭಾವದವರು ಎಂದು ಹೇಳಿದಾಗ ಅವರ ಹೊರಗಿನದನ್ನು ಮಾತ್ರ ತೆಗೆದುಕೊಂಡೆವು. ಅವರ ಮನಸ್ಸಿನಲ್ಲಿದ್ದ ಆಸೆಯ ಬೀಜ ಇನ್ನೂ ಮೊಳೆಯುತ್ತಿತ್ತು, ಫಲಕ್ಕೆ ಬಂದಿರಲಿಲ್ಲ. ಆಗ ಒಂದು ನಿರ್ಣಯಕ್ಕೆ ಬರುತ್ತೇವೆ. ಆದರೆ ಆಸೆಯ ಬೀಜಗಳು ಬಲವಾಗಿ ನೆಲದಲ್ಲಿ ಬೇರೂರಿ ಫಲಿಸಿದಾಗ ಅದನ್ನು ನೋಡಿ ಇನ್ನೊಂದು ನಿರ್ಣಯಕ್ಕೆ ಬರುತ್ತೇವೆ. ಆದಕಾರಣವೆ, ಯಾವುದು ಕರ್ಮ, ವಿಕರ್ಮ ಮತ್ತು ಅಕರ್ಮ ಎಂಬುದನ್ನು ನಿರ್ಧರಿಸುವುದು ಬಹಳ ಕಷ್ಟ. ಅದನ್ನು ಬಹಳ ಆಳಕ್ಕೆ ಹೋಗಿ ನಿರ್ಧರಿಸಬೇಕಾಗಿದೆ.

\begin{shloka}
ಕರ್ಮಣ್ಯಕರ್ಮ ಯಃ ಪಶ್ಯೇದಕರ್ಮಣಿ ಚ ಕರ್ಮ ಯಃ~।\\ಸ ಬುದ್ಧಿಮಾನ್ ಮನುಷ್ಯೇಷು ಸ ಯುಕ್ತಃ ಕೃತ್ಸ್ನ ಕರ್ಮಕೃತ್ \hfill॥ ೧೮~॥
\end{shloka}

\begin{artha}
ಯಾರು ಕರ್ಮದಲ್ಲಿ ಅಕರ್ಮವನ್ನು, ಅಕರ್ಮದಲ್ಲಿ ಕರ್ಮವನ್ನು ನೋಡುತ್ತಾನೆಯೋ ಅವನು ಮನುಷ್ಯರಲ್ಲಿ ಬುದ್ಧಿವಂತ ಮತ್ತು ಯೋಗಿ ಮತ್ತು ಕರ್ಮಗಳನ್ನೆಲ್ಲಾ ಮಾಡಿರುವನು.
\end{artha}

ಶ‍್ರೀಕೃಷ್ಣ ಇಲ್ಲಿ ಮನುಷ್ಯರಲ್ಲಿ ಯಾರು ಬುದ್ಧಿವಂತ ಎಂಬುದನ್ನು ವಿವರಿಸುತ್ತಾನೆ. ಸಾಧಾರಣ ಮನುಷ್ಯ ಹೊರಗಿನದನ್ನು ಮಾತ್ರ ನೋಡುತ್ತಾನೆ. ಬುದ್ಧಿವಂತ ತೋರಿಕೆಯನ್ನು ಭೇದಿಸಿಕೊಂಡು ಹೋಗುತ್ತಾನೆ. ಅದರೊಳಗೆ ಇರುವ ಉದ್ದೇಶಗಳನ್ನು ನೋಡುತ್ತಾನೆ. ಒಬ್ಬ ಪ್ರಚಂಡ ಕರ್ಮದಲ್ಲಿ ನಿರತನಾಗಿರುತ್ತಾನೆ. ಆದರೆ ಮನಸ್ಸಿನಲ್ಲಿ ಸ್ವಲ್ಪವೂ ಉದ್ವೇಗವಿಲ್ಲ, ಕಳವಳವಿಲ್ಲ. ಫಲಾಪೇಕ್ಷೆಯ ಮೇಲೆ ದೃಷ್ಟಿ ಇಲ್ಲ. ಇವನು ಇಷ್ಟೊಂದು ಕೆಲಸದಲ್ಲಿ ನಿರತನಾಗಿದ್ದರೂ, ಒಳಗೆ ಪ್ರಶಾಂತ ಚಿತ್ತನಾಗಿರುವನು. ಕರ್ಮವನ್ನೇ ಮಾಡದಂತೆ ತೋರುವನು. ಮತ್ತೊಬ್ಬ ತೋರಿಕೆಗೆ ಕರ್ಮ ಮಾಡುತ್ತಿಲ್ಲ. ಆದರೆ ಮನಸ್ಸಿನೊಳಗೆ ಎಲ್ಲಾ ವಾಸನೆಗಳೂ ಕುದಿಯುತ್ತಿವೆ. ಯಾವ ಸಮಯದಲ್ಲಿ ಮೇಲೇಳುತ್ತವೋ ಗೊತ್ತಾಗುವುದಿಲ್ಲ. ದೊಡ್ಡ ಕರ್ಮಕ್ಕೆ ಸನ್ನಾಹ ಆಗುತ್ತಿದೆ. ಸಾಧಾರಣ ಮನುಷ್ಯ ಇಂತಹ ವ್ಯಕ್ತಿಯನ್ನು ನೋಡಿದಾಗ ಎಂತಹ ಸಾಧು ಸ್ವಭಾವದವರು, ನಿರ್ಲಿಪ್ತರು ಎಂದು ನಿರ್ಣಯಿಸಬಹುದು. ಆದರೆ ಜ್ಞಾನಿಯಾದರೋ ಈ ತೆರೆಯನ್ನು ಭೇದಿಸಿ ಹೋಗುವನು. ಒಳಗಿರುವ ಗುಟ್ಟನ್ನು ಕಂಡುಹಿಡಿಯುವನು. ಶ‍್ರೀರಾಮಕೃಷ್ಣರು ಒಮ್ಮೆ ಕಲ್ಕತ್ತೆಗೆ ಒಂದು ಪ್ರಾರ್ಥನಾಲಯಕ್ಕೆ ಹೋದರು. ಅಲ್ಲಿ ಪ್ರಚಾರಕ ಬಂದವರಿಗೆಲ್ಲಾ, ಈಗ ಧ್ಯಾನಕ್ಕೆ ಕುಳಿತುಕೊಳ್ಳಿ ಎಂದ. ಎಲ್ಲರೂ ಕಣ್ಣು ಮುಚ್ಚಿಕೊಂಡು ಧ್ಯಾನ ಮಾಡತೊಡಗಿದರು. ಶ‍್ರೀರಾಮಕೃಷ್ಣರು ಇದನ್ನು ನೋಡಿದರು. ಅನಂತರ ಹೇಳುತ್ತಿದ್ದರು, ಇವರು ಮಾಡುವ ಧ್ಯಾನ ಹೇಗಿತ್ತು ಎಂದರೆ\break ಕಪಿಗಳು ಒಂದು ತೋಟಕ್ಕೆ ಲೂಟಿಗೆ ಹೋಗುವ ಮುನ್ನ ಸುಮ್ಮನೆ ಕುಳಿತ ಸ್ಥಿತಿ ಜ್ಞಾಪಕಕ್ಕೆ ತರುತ್ತಿತ್ತು ಎನ್ನುವರು. ಜ್ಞಾನಿ ತೋರಿಕೆಯ ಅಕರ್ಮದಲ್ಲಿ ದೊಡ್ಡ ಕರ್ಮಕ್ಕೆ ಸನ್ನಾಹ ಆಗುತ್ತಿರುವುದನ್ನು ನೋಡುತ್ತಾನೆ.

ಇಂತಹ ಜ್ಞಾನಿಯೇ ಯೋಗಿ, ಕರ್ಮಕುಶಲಿ. ಯಾರಿಗೆ ತನ್ನ ಮನಸ್ಸನ್ನು ಚೆನ್ನಾಗಿ ತಿಳಿದು\-ಕೊಳ್ಳುವುದಕ್ಕೆ ಸಾಧ್ಯವೋ, ಅವನು ಇತರರನ್ನೂ ಚೆನ್ನಾಗಿ ತಿಳಿದುಕೊಳ್ಳಬಲ್ಲ. ಅವನು ಎಲ್ಲಾ ಕರ್ಮವನ್ನೂ ಮಾಡಿದವನು ಎಂದರೆ ಗುರಿಮುಟ್ಟಿದವನು. ಅವನಿಗೆ ಕರ್ಮದಿಂದ ಇನ್ನುಮೇಲೆ ಆಗ\-ಬೇಕಾಗಿರುವುದು ಏನೂ ಇಲ್ಲ. ಬಿಟ್ಟರೆ ನಷ್ಟವಿಲ್ಲ. ಮಾಡಿದರೆ ಅವನಿಗೇನೂ ಲಾಭವಿಲ್ಲ.

\begin{shloka}
ಯಸ್ಯ ಸರ್ವೇ ಸಮಾರಂಭಾಃ ಕಾಮಸಂಕಲ್ಪವರ್ಜಿತಾಃ~।\\ಜ್ಞಾನಾಗ್ನಿದಗ್ಧಕರ್ಮಾಣಂ ತಮಾಹುಃ ಪಂಡಿತಂ ಬುಧಾಃ \hfill॥ ೧೯~॥
\end{shloka}

\begin{artha}
ಯಾರ ಕರ್ಮಗಳು ಕಾಮ ಮತ್ತು ಸಂಕಲ್ಪಗಳಿಂದ ಪಾರಾಗಿವೆಯೋ ಯಾರ ಕರ್ಮಗಳು ಜ್ಞಾನಾಗ್ನಿಯಿಂದ ಸುಡಲ್ಪಟ್ಟಿವೆಯೋ ಅವರನ್ನು ಜ್ಞಾನಿಗಳು ಪಂಡಿತರು ಎಂದು ಕರೆಯುತ್ತಾರೆ.
\end{artha}

ನಮ್ಮ ನಿತ್ಯ ವ್ಯವಹಾರದಲ್ಲಿ ಪಂಡಿತ ಎಂಬುದಕ್ಕೆ ಒಂದು ಅರ್ಥವಿದೆ. ಅದೇ ವಿಷಯಗಳನ್ನು ತಿಳಿದುಕೊಂಡಿರುವವನು, ಬೇಕಾದಷ್ಟು ಶ್ಲೋಕಗಳನ್ನು ಪದೇ ಪದೇ ಉದಹರಿಸುತ್ತಿರುವವನು ಎಂದು. ಆದರೆ ಶ‍್ರೀಕೃಷ್ಣ ಇಲ್ಲಿ ಕೊಡುವ ಅರ್ಥವೇ ಬೇರೆ ರೀತಿಯಾಗಿದೆ. ಯಾರ ಜೀವನದಲ್ಲಿ ಶಾಸ್ತ್ರಸಾರವೇ ಉಸಿರಾಗಿದೆಯೋ ಅವನು. ಅವನು ಹೇಳುವುದಕ್ಕೆ ಹೋಗುವುದಿಲ್ಲ. ಜೀವನದಲ್ಲಿ ತನ್ನ ಬಾಳಿನಲ್ಲಿಯೇ ಅದನ್ನು ತೋರುತ್ತಿರುವನು. ದೀಪ ನಾನು ಇಲ್ಲಿದ್ದೇನೆ ಎಂದು ಕಿರಿಚಿಕೊಳ್ಳ ಬೇಕಾಗಿಲ್ಲ. ಅದರ ಕಾಂತಿಯೇ ಅದನ್ನು ಸಾರುವುದು. ಪಂಡಿತ ಈ ಗುಂಪಿಗೆ ಸೇರಿದವನು. ಅವನ ಕರ್ಮಗಳೆಲ್ಲ ಆಸೆ ಮತ್ತು ಅಹಂಕಾರದಿಂದ ಪಾರಾಗಿವೆ. ಯಾವ ಕರ್ಮವನ್ನು ಮಾಡಿದರೂ ಅದರಿಂದ ಬರುವ ಫಲಗಳಿಗೆ ಆಸಕ್ತನಲ್ಲ ಅವನು. ನಾನು ಇದನ್ನು ಮಾಡುತ್ತಿರುವವನು, ಇದನ್ನು ನಾನು ಸಾಧಿಸಿದೆ ಎಂಬ ಅಹಂಕಾರವಿಲ್ಲ. ಇಂತಹ ವ್ಯಕ್ತಿ ಸ್ವಾರ್ಥವಾಗಿರುವುದನ್ನು ಮಾಡಲೇ ಆರ. ಏಕೆಂದರೆ ಇವನಲ್ಲಿ ಈಗ ಮಿಕ್ಕಿರುವುದು ಉತ್ತಮ ಸಂಸ್ಕಾರಗಳು ಮಾತ್ರ. ಅದರಿಂದ ಲೋಕ ಕಲ್ಯಾಣದ ಕೆಲಸಗಳು ಮಾತ್ರ ಆಗುತ್ತಿರುತ್ತವೆ. ಹೀನವಾಸನೆಗಳೇ ಇವನಲ್ಲಿರುವುದಿಲ್ಲ. ಇದ್ದರೆ ಇವನು ಆ ಮಟ್ಟಕ್ಕೆ ಏರುತ್ತಲೇ ಇರಲಿಲ್ಲ. ಇವನು ಬಿಡುವಿಲ್ಲದೆ ಕೆಲಸವನ್ನು ಮಾಡುತ್ತಿದ್ದರೂ ಒಳಗೆ ಆರಾಮ. ವಿಶ್ರಾಂತಿ ತಾಂಡವಾಡುತ್ತಿದೆ. ಇವನ ಕರ್ಮವನ್ನು ಯಾರು ಎಷ್ಟು ಹೊಗಳಿದರೂ ಇವನೇನೂ ಅದಕ್ಕೆ ವಾಲುವವನಲ್ಲ. ಎಷ್ಟೇ ಟೀಕಿಸಿದರೂ ಅದರ ಭಾರಕ್ಕೆ ಕುಗ್ಗಿಹೋಗುವವನಲ್ಲ.

ಇವನ ಕರ್ಮಗಳೆಲ್ಲ ಜ್ಞಾನಾಗ್ನಿಯಲ್ಲಿ ಬೆಂದಿವೆ. ಬೀಜದಲ್ಲಿರುವ ಮೊಳಕೆ ಸೀದುಹೋಗಿದೆ. ಅವನನ್ನು ಅದು ಕಟ್ಟಿಹಾಕಲಾರದು. ಅವನು ಕರ್ಮಫಲದಿಂದ ನಿವೃತ್ತನಾಗಿರುವನು. ಎಲ್ಲಿಯವರೆಗೂ ಜ್ಞಾನಾಗ್ನಿಯಲ್ಲಿ ಬೆಂದಿಲ್ಲವೋ ಅಲ್ಲಿಯವರೆಗೂ ಒಬ್ಬನು ಲೋಕಸಂಗ್ರಹಕ್ಕಾಗಿ ಕೆಲಸ ಮಾಡುತ್ತಿರಬಹುದು. ಆದರೂ ಬದ್ಧನಾಗುತ್ತಾನೆ. ಲೋಕಸಂಗ್ರಹದ ಕೆಲಸದಿಂದ ಬರುವ ಕೀರ್ತಿ, ಅಧಿಕಾರದ ಬಲೆಗೆ ಬೀಳುತ್ತಾನೆ. ಆದರೆ ಕರ್ಮಕುಶಲಿ, ಅದರ ರಹಸ್ಯವನ್ನು ಅರಿತವನು ನಮ್ಮನ್ನು ಕಟ್ಟಿಹಾಕುವುದನ್ನು ಕಿತ್ತುಹಾಕುತ್ತಾನೆ. ಅನಂತರ ಕರ್ಮ ಮಾಡುತ್ತಾನೆ. ಶಸ್ತ್ರಚಿಕಿತ್ಸೆಯನ್ನು ಮಾಡುವುದಕ್ಕೆ ಮುಂಚೆ ಸರ್ಜನ್ ತನ್ನ ಕೆಲಸಕ್ಕೆ ಉಪಯೋಗಿಸುವ ಉಪಕರಣಗಳನ್ನೆಲ್ಲಾ ಚೆನ್ನಾಗಿ ಕುದಿಯುವ ನೀರಿನಲ್ಲಿ ಬೇಯಿಸುತ್ತಾನೆ. ಅದಕ್ಕೆ ಅಂಟಿಕೊಂಡಿರುವ ಕ್ರಿಮಿಗಳು ನಾಶವಾಗಲಿ ಎಂದು. ಅನಂತರ ಶಸ್ತ್ರಕ್ರಿಯೆಗೆ ಪ್ರಾರಂಭಿಸುತ್ತಾನೆ. ಅದರಂತೆಯೆ ಜ್ಞಾನಿ ತಾನು ಮಾಡುವ ಕ್ರಿಯೆಗೆ ಮೆತ್ತಿಕೊಂಡಿರುವ ಆಸಕ್ತಿ ಮತ್ತು ಫಲಾಪೇಕ್ಷೆ ಇವನ್ನು ಜ್ಞಾನದ ಬೆಂಕಿಯ ಮೇಲೆ ಇಟ್ಟು ನಾಶಮಾಡುತ್ತಾನೆ.

\begin{shloka}
ತ್ಯಕ್ತ್ವಾ ಕರ್ಮಫಲಾಸಂಗಂ ನಿತ್ಯತೃಪ್ತೋ ನಿರಾಶ್ರಯಃ~।\\ಕರ್ಮಣ್ಯಭಿಪ್ರವೃತ್ತೋಽಪಿ ನೈನ ಕಿಂಚಿತ್ ಕರೋತಿ ಸಃ \hfill॥ ೨ಂ~॥
\end{shloka}

\begin{artha}
ಕರ್ಮಫಲಸಂಗವನ್ನು ಬಿಟ್ಟು ನಿತ್ಯತೃಪ್ತನಾಗಿ ನಿರಾಶ್ರಯನಾಗಿರುವ ಅವನು ಕರ್ಮದಲ್ಲಿ ತೊಡ\-ಗಿದ್ದರೂ ನಿಜವಾಗಿ ಏನನ್ನೂ ಮಾಡುತ್ತಿಲ್ಲ.
\end{artha}

ಕರ್ಮಫಲವನ್ನು ಯಾವಾಗ ಒಬ್ಬ ಬಿಡುತ್ತಾನೆಯೋ ಆಗ ಅವನು ಎಂತಹ ಕರ್ಮವನ್ನು ಮಾಡುತ್ತಿದ್ದರೂ ಬಾಧಕವಿಲ್ಲ. ಅದರಿಂದ ಅವನು ಬದ್ಧನಾಗುವುದಿಲ್ಲ. ನಮ್ಮನ್ನು ಸಂಸಾರದ ಗಾಣಕ್ಕೆ ಕಟ್ಟಿಹಾಕಿರುವುದೇ ಫಲಾಪೇಕ್ಷೆ. ಫಲಾಪೇಕ್ಷೆ ಬಿಟ್ಟವನು ಶುದ್ಧ ಸೋಮಾರಿಯಾಗಿ ಕುಳಿತುಕೊಂಡು ಕಾಲ ಕಳೆಯುವುದಿಲ್ಲ. ಅವನು ಯಾವಾಗಲೂ ಯಾವುದಾದರೂ ಲೋಕಸಂಗ್ರಹದ ಕೆಲಸದಲ್ಲಿ ನಿರತನಾಗಿರುವನು. ಫಲಾಪೇಕ್ಷೆಯನ್ನು ಬಿಟ್ಟುಬಿಟ್ಟರೆ ಇಷ್ಟು ಕರ್ಮ ಮಾಡುವುದಕ್ಕೆ ನಮ್ಮನ್ನು ಪ್ರೇರೇಪಿಸುವುದು ಯಾವುದು ಎಂದು ಸಾಧಾರಣ ಮನುಷ್ಯ ಆಲೋಚಿಸಬಹುದು. ಮನುಷ್ಯನ ಶ್ರೇಷ್ಠ ಕರ್ಮಗಳು ಆಗುವುದೇ ಆಗ. ಆಚಂದ್ರಾರ್ಕವಾಗಿ ಸಮಾಜದ ಮೇಲೆ ಪರಿಣಾಮವನ್ನು ಬಿಡುವಂತಹ ಕೆಲಸ ಆಗುವುದೇ ಆಗ. ಯಾವಾಗ ಇದನ್ನು ನಾನು ಮಾಡುತ್ತಿರುವುದು, ಇದರಿಂದ ಬರುವ ಫಲವನ್ನು ನಾನುಣ್ಣಬೇಕು ಎಂಬುದನ್ನು ಬಿಡುವನೊ, ಆಗ ಭೂಮ ಇವನನ್ನು ಆರಿಸಿಕೊಳ್ಳುವುದು. ಇವನ ಮೂಲಕ ಅದ್ಭುತ ಕೆಲಸವನ್ನು ಮಾಡುವುದು. ಪ್ರಪಂಚವೇ ಅಚ್ಚರಿ ಪಡಬೇಕು ಇವನು ಮಾಡುತ್ತಿರುವ ಕೆಲಸಕ್ಕೆ. ಆ ಭೂಮನಾದ ಭಗವಂತನೇ ಹಿಂದಿನಿಂದ ನಿಂತುಕೊಂಡು ಇವನಿಗೆ ಕೆಲಸಮಾಡುವುದಕ್ಕೆ ಏನೇನು ಬೇಕೊ ಅದನ್ನೆಲ್ಲಾ ಒದಗಿಸುವನು. ಬೇಕಾದ ಯುಕ್ತಿಯನ್ನು ಕೊಡುವನು, ಶಕ್ತಿಯನ್ನು ಕೊಡುವನು, ಉತ್ಸಾಹವನ್ನು ಕೊಡುವನು.

ಅವನು ಕೆಲಸ ಮಾಡುವಾಗ ನಿತ್ಯವೂ ತೃಪ್ತನಾಗಿರುವನು. ಬೇಕಾದಷ್ಟು ಮನ್ನಣೆ ಬರಬಹುದು, ಅಥವಾ ಜನ ಟೀಕಿಸಬಹುದು. ಉದ್ದೇಶ ಪೂರ್ಣವಾಗಬಹುದು, ಅಥವಾ ಮಧ್ಯ\-ದಲ್ಲಿಯೇ ಅವನು ಅದನ್ನು ಬಿಡಬಹುದು, ಅವನು ಯಾವುದಕ್ಕೂ ಕುಣಿದಾಡುವುದೂ ಇಲ್ಲ, ಗೊಣಗಾಡುವುದೂ ಇಲ್ಲ. ತನ್ನ ಪಾಡಿಗೆ ತಾನು ಕೆಲಸವನ್ನು ಮಾಡಿಕೊಂಡು ಹೋಗುವನು.

ಕೆಲಸ ಮಾಡುವಾಗ ಅವನು ಯಾರನ್ನೂ ಅದಕ್ಕಾಗಿ ಆಶ್ರಯಿಸುವುದಿಲ್ಲ. ಜನ, ಹಣ ಇವುಗಳನ್ನು ಹುಡುಕಿಕೊಂಡು ಹೋಗುವುದಿಲ್ಲ. ಎಲ್ಲಾ ಅವನ ಬಳಿಗೇ ಬರುವುದು. ಅವನು ಹಿಡಿದ ಕೆಲಸ ಲೋಕಸಂಗ್ರಹಕ್ಕಾಗಿ. ಅದನ್ನು ಪೂರೈಸಲು ಬೇಕಾದುದೆಲ್ಲ ಅವನನ್ನು ಹುಡುಕಿಕೊಂಡು ಬರುವುದು. ಅವನಿಗೆ ಯಾವುದೂ ಬೇಕಾಗಿಲ್ಲ, ಯಾರೂ ಬೇಕಾಗಿಲ್ಲ. ಅವನಿಗೆ ಯಾರ ಆಶ್ರಯದ ಹಂಗೂ ಇರುವುದಿಲ್ಲ. ಆದರೆ ಇತರರಿಗೆ ಅವನು ಆಶ್ರಯನಾಗುವನು. ಇತರರಿಗೆ ನಾನು ಆಶ್ರಯನಾಗಿದ್ದೇನೆ ಎಂಬ ಭಾವವೂ ಅವನಲ್ಲಿ ಇರುವುದಿಲ್ಲ. ದೇವರು ಇತರರಿಗೆ ತನ್ನ ಮೂಲಕ ಇಂತಹ ಆಶ್ರಯವನ್ನು ಕೊಡಿಸಿರುವನು ಎಂದು ಮಾತ್ರ ಭಾವಿಸುವನು.

ಅವನೇನೋ ಕರ್ಮದಲ್ಲಿ ನಿರತನಾಗಿರಬಹುದು. ಆದರೂ ಅವನು ಏನನ್ನೂ ಮಾಡುವವ\-ನಂತೆ ತೋರುವುದಿಲ್ಲ. ಏಕೆಂದರೆ ಅವನಲ್ಲಿ ಕರ್ತೃತ್ವದ ಭಾವನೆ ಇಲ್ಲ. ಫಲಾಪೇಕ್ಷೆ ಇಲ್ಲ. ಅವನು ಮಾಡುವ ಕರ್ಮ ಯಾವ ಸಂಸ್ಕಾರಗಳನ್ನೂ ಅವನ ಮೇಲೆ ಬಿಡುವುದಿಲ್ಲ. ಸಂಸ್ಕಾರ ಬರಬೇಕಾದರೆ ಮಾಡುವ ಕೆಲಸದ ಮೇಲೆ ಆಸಕ್ತಿ ಇರಬೇಕು. ಅಹಂ ಕರ್ತೃಭಾವ ಇರಬೇಕು. ಇವೆರಡೂ\break ಯಾವಾಗ ಇಲ್ಲವೋ ಇವೆಲ್ಲ ಮರಿಯಾಗದೆ ಇರುವ ಮೊಟ್ಟೆಗಳಂತೆ. ಅಲ್ಲಲ್ಲಿಗೆ ಕೊನೆಗೊಳ್ಳುವುವು. ಅವನನ್ನು ಒಂದು ಕ್ರಿಯೆಯ ಸರಪಳಿಗೆ ಕಟ್ಟಿಹಾಕುವುದಿಲ್ಲ. ಅವನಲ್ಲಿ ಬೇಕಾದಷ್ಟು ಕರ್ಮವನ್ನು ಹೊರಗೆ ನೋಡುತ್ತೇವೆ. ಒಳಹೊಕ್ಕು ನೋಡಿದರೆ ಅಲ್ಲೆಲ್ಲ ಪ್ರಶಾಂತಿ, ಯಾವ ಗಡಿಬಿಡಿಯೂ ಇಲ್ಲ, ಉದ್ವೇಗವೂ ಇಲ್ಲ. ಅಲ್ಲಿ ಅಕರ್ಮ. ಸುಮ್ಮನೆ ಕೆಲಸಮಾಡದೆ ಇರುವ ಅಜ್ಞಾನಿಯಲ್ಲಾದರೂ ಹೊರಗಡೆಯಿಂದ ಅಕರ್ಮ; ಒಳಹೊಕ್ಕು ನೋಡಿದರೆ ದೊಡ್ಡದೊಂದು ಆಸೆಯ ಫ್ಯಾಕ್ಟರಿಯೇ ಕಿವಿ ಕಿವುಡಾಗುವಂತೆ ಕೆಲಸ ಮಾಡುತ್ತಿರುವುದು.

\begin{shloka}
ನಿರಾಶೀರ್ಯತಚಿತ್ತಾತ್ಮಾ ತ್ಯಕ್ತಸರ್ವಪರಿಗ್ರಹಃ~।\\ಶಾರೀರಂ ಕೇವಲಂ ಕರ್ಮ ಕುರ್ವನ್ನಾಪ್ನೋತಿ ಕಿಲ್ಬಿಷಮ್ \hfill॥ ೨೧~॥
\end{shloka}

\begin{artha}
ಆಶೆ ಇಲ್ಲದೆ ಚಿತ್ತ, ಆತ್ಮಗಳನ್ನು ಬಿಗಿಹಿಡಿದುಕೊಂಡು, ಸರ್ವಪರಿಗ್ರಹವನ್ನೂ ಬಿಟ್ಟು, ಕೇವಲ ಶರೀರ ಕರ್ಮವನ್ನು ಮಾಡುತ್ತಿರುವವನಿಗೆ ಯಾವ ಪಾಪವೂ ಇಲ್ಲ.
\end{artha}

ಅನಾಸಕ್ತಿಯಿಂದ ಕೆಲಸ ಮಾಡುವವನ ಮನಸ್ಸಿನ ಪರಿಚಯವನ್ನು ಇಲ್ಲಿ ಮತ್ತೂ ವಿಶಾಲವಾಗಿ ಕೊಟ್ಟಿದೆ. ಅವನು ನಿರಾಶೀ ಎಂದರೆ ಯಾವ ಆಸೆಯನ್ನೂ ಇಟ್ಟುಕೊಂಡಿಲ್ಲ. ಅಂದರೆ ಉದ್ದೇಶವೇ ಇಲ್ಲದೆ ಕೆಲಸ ಮಾಡುತ್ತಾನೆ ಎಂದಲ್ಲ. ಆಗ ಅವನು ಹುಚ್ಚನಾಗುತ್ತಾನೆ, ಜ್ಞಾನಿಯಾಗುವುದಿಲ್ಲ. ಅದರ ಹಿಂದೆ ತಾನು ಆ ಫಲವನ್ನು ಅನುಭವಿಸಬೇಕೆಂಬ ಹಂಬಲ ಇರುವುದಿಲ್ಲ. ತನ್ನದೇ ಕೆಲಸ ಆದರೂ ಇನ್ನೊಬ್ಬರ ಕೆಲಸದಂತೆ ಮಾಡುತ್ತಿರುವನು. ಅನೇಕ ವೇಳೆ ಇನ್ನೊಬ್ಬರ ಕೆಲಸ ಎಂದರೆ ತಾತ್ಸಾರ, ಅಸಡ್ಡೆ. ಕಾಟಾಚಾರಕ್ಕೆ ಮಾಡುವುದು, ಈ ಭಾವನೆಗಳೆಲ್ಲ ಬರುವುದು. ಕರ್ಮಯೋಗಿ ಹಾಗಲ್ಲ. ತನ್ನ ಕೆಲಸವನ್ನು ಅಚ್ಚುಕಟ್ಟಾಗಿ ಮಾಡುವನು. ಆದರೆ ಅದರ ಹಿಂದೆ ಇದು ನನ್ನ ಕೆಲಸ, ನಾನು ಮಾಡುತ್ತಿರುವುದು ಎಂಬ ಭಾವನೆ ಇರುವುದಿಲ್ಲ. ಇದು ದೇವರ ಕೆಲಸ, ನನ್ನ ಮೂಲಕ ಮಾಡುತ್ತಿರುವನು. ಕೆಲವು ವೇಳೆ ಮಕ್ಕಳ ಕೈಯನ್ನು ಭದ್ರವಾಗಿ ಹಿಡಿದುಕೊಂಡು ದೊಡ್ಡವರು ಏನನ್ನಾದರೂ ಬರೆಸುತ್ತಾರೆ. ಯಾರೋ ಇವನ ಮೂಲಕ ಬರೆಸುತ್ತಿರುವರು. ಇವನು ಒಂದು ವೀಣೆ. ಅದನ್ನು ನುಡಿಸುತ್ತಿರುವವನು ದೇವರು. ಆ ಭಗವಂತನ ನುರಿತ ಕೈಗಳು ತಂತಿಗೆ ತಾಕಿದೊಡನೆ ಅಮರಗಾನ ಹೊರಹೊಮ್ಮುವುದು. ಜನ ಕೊಂಡಾಡುವರು. ಆದರೆ ಇದರಿಂದ ವೀಣೆಗೆ ಕೀರ್ತಿಯೆ! ಈ ಕೀರ್ತಿಯೆಲ್ಲ ನುಡಿಸುವವನಿಗೆ ಸೇರಿದ್ದು. ಇಂತಹ ದೃಷ್ಟಿ ಕರ್ಮಯೋಗಿಯಲ್ಲಿರುವುದು.

ಅವನು ಕೆಲಸಮಾಡುವಾಗ ಇಂದ್ರಿಯ, ಮನಸ್ಸನ್ನು ನಿಗ್ರಹಿಸುತ್ತಾನೆ. ಅವು ಕೇಳಿದ ಕಡೆ ಹರಿದು ಹೋಗುತ್ತಿದ್ದರೆ ಇವನು ಯಾವ ಭಗವಂತನ ಕೆಲಸವನ್ನೂ ಮಾಡಲಾರ. ಭಗವಂತ ತನ್ನ ನಿಮಿತ್ತ ವನ್ನಾಗಿ ಆರಿಸಿಕೊಳ್ಳಬೇಕಾದರೆ ಇಂದ್ರಿಯ ಗುಲಾಮನನ್ನು ಆರಿಸಿಕೊಳ್ಳಲಾಗುವುದಿಲ್ಲ. ಅದೊಂದು ತೂತಿನ ಪಾತ್ರೆ. ಇಟ್ಟಿದ್ದೆಲ್ಲ ವ್ಯರ್ಥವಾಗುವುದು. ಯಾರು ವಿಷಯವಸ್ತುವಿನ ಕಡೆ ಹರಿದುಹೋಗುವ ಇಂದ್ರಿಯವನ್ನು ತಡೆಗಟ್ಟಿ ಅದನ್ನು ಅಂತರ್ಮುಖ ಮಾಡಿರುವನೊ, ಅದು ಚೆನ್ನಾದ ಪಾತ್ರೆ. ದೇವರು ತನ್ನ ಕೆಲಸಕ್ಕೆ ಆರಿಸಿಕೊಳ್ಳುವುದು ಇಂಥದನ್ನು.

ಅವನು ಎಲ್ಲ ಪರಿಗ್ರಹವನ್ನು ಬಿಟ್ಟಿರುವನು. ಅವನು ಏನನ್ನೂ ತೆಗೆದುಕೊಳ್ಳುವುದಿಲ್ಲ. ಜನರ ಹೊಗಳಿಕೆಗೆ ಗಮನ ಕೊಡನು. ತೆಗಳಿಕೆಗೂ ಗಮನ ಕೊಡನು. ಲಾಭಕ್ಕೂ ಗಮನ ಕೊಡುವುದಿಲ್ಲ. ನಷ್ಟಕ್ಕೂ ಗಮನ ಕೊಡುವುದಿಲ್ಲ. ಆ ಕೆಲಸದಿಂದ ಬರುವ ಒಳ್ಳೆಯ ಹೆಸರು ಬೇಕಾಗಿಲ್ಲ.\break ಯಾವಾಗ ತೆಗೆದುಕೊಳ್ಳುತ್ತಾನೆಯೋ ಆಗ ಅಜೀರ್ಣದಿಂದ ನರಳುತ್ತಾನೆ. ಇದು ಅವನದಲ್ಲ. ಅದಕ್ಕಾಗಿ ಅವನು ಕೈ ಒಡ್ಡುವುದಿಲ್ಲ. ಕೈ ಒಡ್ಡಿದರೆ ದಾಸರಾಗುತ್ತೇವೆ, ಅದು ನಮ್ಮನ್ನು ಕುಣಿಸುವುದು. ಯಾವಾಗ ಕೈ ಒಡ್ಡುವುದಿಲ್ಲವೋ ಆಗ ಅವು ದಾಸರಾಗುತ್ತವೆ. ಎಲ್ಲವನ್ನೂ ಬಿಟ್ಟವನಿಗೆ ಏನೂ ಇಲ್ಲ ಎಂದಲ್ಲ. ಎಲ್ಲವನ್ನೂ ಬಿಟ್ಟವನಿಗೇ ಎಲ್ಲವೂ ಇರುವುದು. ಯಾವನು ಕಟ್ಟಿಕೊಂಡಿರುವನೋ ಅಷ್ಟು ಮಾತ್ರ ಅವನದು. ಅದೂ ಯಾವಾಗ ಇವನ ಕೈಯಿಂದ ನುಸುಳಿ ಹೋಗುವುದೋ ಗೊತ್ತಿಲ್ಲ.

ಇವನು ಕೇವಲ ಶಾರೀರಕ ಕರ್ಮವನ್ನು ಮಾಡುತ್ತಿರುವನು ಎಂದರೆ ದೇಹ ಉಳಿಸಿಕೊಳ್ಳು\-ವುದಕ್ಕೆ ಬೇಕಾದ ಬರೀ ತಿನ್ನುವುದು, ಕುಡಿಯುವುದು, ನಿದ್ರೆ ಇವುಗಳನ್ನು ಮಾತ್ರ ಮಾಡುತ್ತಿರುವನು. ಆದರೆ ಅದರ ಹಿಂದೆ ಆಸಕ್ತಿಯಿಲ್ಲ. ದೇಹಕ್ಕೆ ಊಟ ಕೊಡುವನು. ಅವನು ರುಚಿಗೆ ದಾಸನಲ್ಲ. ಬಟ್ಟೆ ಹಾಕಿಕೊಳ್ಳುವನು, ತಾನು ಸೊಗಸುಗಾರನಂತೆ ಕಾಣಬೇಕೆಂದಲ್ಲ. ಶರೀರಕ್ಕೆ ಸಂಬಂಧಪಟ್ಟ ಕ್ರಿಯೆಗಳನ್ನು ಆಸಕ್ತಿಯಿಲ್ಲದೆ ಮಾಡುವನು.

ಆದರೆ ಅವನು ಇಷ್ಟೇ ಮಾಡುತ್ತಾನೆ ಎಂದಲ್ಲ. ಬರೀ ಅವನು ಇಷ್ಟನ್ನೇ ಮಾಡಿಕೊಂಡು ಹೋದರೆ ಇದರಿಂದ ಲೋಕಸಂಗ್ರಹದ ಕೆಲಸ ಯಾವುದೂ ಆದಂತೆ ಕಾಣುವುದಿಲ್ಲ. ಅವನು ಲೋಕಸಂಗ್ರಹದ ಕೆಲಸ ಮಾಡುತ್ತಿರುವಾಗಲೂ ಕೇವಲ ದೇಹ ಇದನ್ನೆಲ್ಲಾ ಮಾಡುವಂತೆ ತೋರುವುದು. ದೊಡ್ಡ ಯಂತ್ರ ಕೆಲಸ ಮಾಡುವುದು. ಅದರಲ್ಲಿ ತಾನು ಮಾಡುತ್ತೇನೆ ಎಂಬ ಭಾವವಿಲ್ಲ. ಇದರಿಂದ ಬರುವ ಲಾಭ ತನಗೆ ಬರಬೇಕೆಂದಿಲ್ಲ ಅದಕ್ಕೆ. ಸುಮ್ಮನೆ ಕೆಲಸ ಮಾಡುವುದಕ್ಕಾಗಿ ಅದನ್ನು ಕಂಡುಹಿಡಿದಿರುವರು. ಅದು ಕೆಲಸವನ್ನು ಮಾಡಿಹಾಕುತ್ತದೆ. ಯೋಗಿಯ ಬಾಹ್ಯ ಮಾತ್ರ ಕೆಲಸದಲ್ಲಿ ನಿರತವಾಗಿರುವಂತೆ ಕಾಣುತ್ತಿದೆ. ಒಳಗಿನ ಮನಸ್ಸಿನಲ್ಲಿ ಯಾವ ಆಶೆಯೂ ಇಲ್ಲ. ಯಾವಾಗ ಈ ದೃಷ್ಟಿಯಿಂದ ಕೆಲಸ ಮಾಡುವನೋ ಆಗ ಅವನು ಯಾವ ಪಾಪಕ್ಕೂ ಒಳಗಾಗುವುದಿಲ್ಲ. ಇಲ್ಲಿ ಪಾಪ ಎಂದರೆ ಕೊಳೆ, ಕಶ್ಮಲ, ಮಾಡುವ ಒಳ್ಳೆ ಕೆಲಸ ಹಾಳಾಗುವುದಕ್ಕೆ ಕಾದು ಕುಳಿತಿರುವ ಕ್ರಿಮಿಗಳು. ಉಪ್ಪಿನ ಕಾಯನ್ನು ಹಾಕಿಡುತ್ತೇವೆ. ಸ್ವಲ್ಪ ಅಜಾಗರೂಕರಾಗಿದ್ದರೆ ಅದರಲ್ಲಿ ಹುಳು ಬೀಳುವುದು. ಪಾಪ ಎಂದರೆ ಇದೇ. ನಾನು ಇದನ್ನು ಮಾಡುತ್ತಿರುವವನು ಎಂಬ ಅಹಂಕಾರ. ಇದನ್ನೆಲ್ಲಾ ಸಾಧಿಸಿದ್ದೇ ನಾನು, ಅದರ ಮೇಲೆ ಹಕ್ಕಿದೆ ನನಗೆ ಎಂಬ ಮೌಢ್ಯ. ಯೋಗಿ, ಈ ವಿಷಕ್ರಿಮಿ ತನ್ನ ಕೆಲಸಕ್ಕೆ ಧಾಳಿ ಇಡದಂತೆ ನೋಡಿಕೊಳ್ಳುವನು. ಜ್ಞಾನಾಗ್ನಿಯಲ್ಲಿ ಯಾವಾಗಲೂ ಅದನ್ನು ಕಾಯಿಸುತ್ತಿರುವನು.

\begin{shloka}
ಯದೃಚ್ಛಾಲಾಭಸಂತುಷ್ಟೋ ದ್ವಂದ್ವಾತೀತೋ ವಿಮತ್ಸರಃ~।\\ಸಮಃ ಸಿದ್ಧಾವಸಿದ್ಧೌ ಚ ಕೃತ್ವಾಪಿ ನ ನಿಬಧ್ಯತೇ \hfill॥ ೨೨~॥
\end{shloka}

\begin{artha}
ತಾನಾಗಿ ಬಂದುದರಲ್ಲಿ ತೃಪ್ತನಾಗಿ ದ್ವಂದ್ವವನ್ನು ಮೀರಿ ಮಾತ್ಸರ್ಯವಿಲ್ಲದೆ ಸಿದ್ಧಿ ಅಸಿದ್ಧಿಗಳಲ್ಲಿ ಸಮಬುದ್ಧಿ ಯುಳ್ಳವನಾಗಿ ಇರುವ ಯೋಗಿ ಕರ್ಮವನ್ನು ಮಾಡಿದರೂ ಬದ್ಧನಾಗುವುದಿಲ್ಲ.
\end{artha}

ಯೋಗಿ ತನಗೆ ಬಂದುದರಲ್ಲಿ ತೃಪ್ತನಾಗಿರುವನು. ಅಯ್ಯೋ ಇದು ಬಂತಲ್ಲ ಎಂದು ಪೇಚಾಡುವುದೂ ಇಲ್ಲ, ಇಷ್ಟೇ ಬಂತಲ್ಲ, ಇನ್ನು ಸ್ವಲ್ಪ ಜಾಸ್ತಿ ಬರಬೇಕಾಗಿತ್ತಲ್ಲ ಎಂತಲೂ ವ್ಯಥೆಪಡುವುದಿಲ್ಲ. ಎಂತಹ ಬಟ್ಟೆಯಾಗಲಿ ಹೊದೆಯುವನು. ಏನು ಸಿಕ್ಕಿದರೂ ತಿನ್ನುವನು. ಯಾವ ಸ್ಥಳದಲ್ಲಾದರೂ ವಾಸವಾಗಿರುವನು. ಒಂದು ಗಿಡಕ್ಕೆ ನಾವು ಯಾವಾಗ ಏನು ಹಾಕಿದರೂ ಅದು ತೆಗೆದುಕೊಳ್ಳುವುದು. ಇದೇ ಕೊಡು, ಅದೇ ಕೊಡು ಎಂದು ಕೇಳುವುದಕ್ಕೆ ಹೋಗುವುದಿಲ್ಲ. ಜೀವನದಲ್ಲಿ ನಮಗೆ ಇಚ್ಛಿಸುವುದೆಲ್ಲಾ ದೊರಕುವುದಿಲ್ಲ. ಏನು ದೊರಕುವುದೊ ಅದು ಭಗವಂತನ ನಿಯಾಮಕವೆಂದು ಗೊಣಗಾಡದೆ ಸ್ವೀಕರಿಸುವನು.

ದ್ವಂದ್ವವನ್ನು ಮೀರಿ ನಿಂತಿರುವವನು ಅವನು. ಜೀವನದಲ್ಲಿ ಸುಖ, ದುಃಖ, ಲಾಭ ನಷ್ಟ, ಹೊಗಳಿಕೆ ತೆಗಳಿಕೆ ಇವುಗಳಲ್ಲಿ ಯಾವಾಗ ಒಂದಕ್ಕೆ ಮನಸ್ಸನ್ನು ಕೊಡುತ್ತೇವೆಯೋ ಆಗ ಇನ್ನೊಂದಕ್ಕೆ ಮನಸ್ಸನ್ನು ಕೊಡಬೇಕಾಗುವುದು. ಅನಿಷ್ಟದಿಂದ ಪಾರಾಗಬೇಕಾದರೆ ಇಷ್ಟದಿಂದಲೂ ಪಾರಾಗಬೇಕಾಗಿದೆ. ಈ ದ್ವಂದ್ವದ ಅನುಭವಗಳು ಯಾವುದೂ ಶಾಶ್ವತವಾಗಿ ಬಂದು ನಿಲ್ಲುವುದಿಲ್ಲ ನಮ್ಮ ಹತ್ತಿರ. ಒಂದಾದ ಮೇಲೊಂದು ಬಂದು ಹೋಗುತ್ತಿರುತ್ತವೆ ಹಗಲು ರಾತ್ರಿಗಳಂತೆ. ಇದಕ್ಕೆ ಯಾವುದಕ್ಕೂ ಅವನು ಅಂಟಿಕೊಂಡಿರುವುದಿಲ್ಲ. ಇದೇ ಅವನ ಗುರಿಯಲ್ಲ. ದ್ವಂದ್ವದ ಕಣಿವೆಯ ಮೂಲಕ ಜೀವ ಪ್ರಯಾಣ ಮಾಡಲೇಬೇಕಾಗಿದೆ. ಈ ಅನುಭವಗಳು ಎಡಬಲದಲ್ಲಿ ಸಿಕ್ಕುವ ರೈಲ್ವೆ ನಿಲ್ದಾಣಗಳು. ನಾವು ಮುಟ್ಟಬೇಕಾದ ಗುರಿಯಲ್ಲ. ಸಾಕ್ಷಿಯಂತೆ ರೈಲ್ವೆ ನಿಲ್ದಾಣಗಳನ್ನು ನಿಂತು ನೋಡಿ ಮುಂದೆ ಸಾಗುವನು. ಇವನು ಯಾವುದಕ್ಕೂ ಅಂಟಿಕೊಂಡಿಲ್ಲ.

ಇವನಲ್ಲಿ ಮಾತ್ಸರ್ಯವಿಲ್ಲ. ತನಗೆ ಒಳ್ಳೆಯದೆ ಬರದೆ ಇನ್ನೊಬ್ಬರಿಗೆ ಒಳ್ಳೆಯದು ಬಂದರೆ, ಅಯ್ಯೋ ಅದು ನನಗೆ ಬರಬೇಕಾಗಿತ್ತು. ಅವನಿಗೆ ಬಂತಲ್ಲ ಎಂದು ಕರುಬುವುದಿಲ್ಲ. ತನಗೇನಾ\-ದರೂ ಅಹಿತವಾಗಿರುವುದಾದರೆ ಅದನ್ನು ಗೊಣಗಾಡದೆ ಸಹಿಸುವನು. ಅವನಲ್ಲಿ ತಿತಿಕ್ಷೆಯನ್ನು ನೋಡುವೆವು. ಇನ್ನೊಬ್ಬನಿಗೆ ಒಳ್ಳೆಯದಾದರೆ, ಅವನ ಸಂತೋಷದಲ್ಲಿ ಭಾಗಿಯಾಗುವನು. ಜೀವನದಲ್ಲಿ ಒಳ್ಳೆಯದು ಎಲ್ಲೋ ಅಲ್ಪ ಪ್ರಮಾಣದಲ್ಲಿರುವುದು. ಅದನ್ನು ಎಲ್ಲರಿಗೂ ಹಂಚುವುದಕ್ಕೆ ಆಗುವುದಿಲ್ಲ. ಯಾರಿಗಾದರೂ ಕೆಲವರಿಗೆ ಮಾತ್ರ ಸಿಕ್ಕಬೇಕು. ಪರವಾ ಇಲ್ಲ ಅವನಿಗೆ ಬಂತಲ್ಲ ಎನ್ನುವುದು ಹೃದಯದ ಹಿರಿಮೆಯನ್ನು ತೋರುವುದು. ಇಂತಹ ಉದಾರ ಹೃದಯನು ಎಲ್ಲಿದ್ದರೂ ಸಂತೋಷಪಡುವನು. ಅಸೂಯೆ ಪಡುವವನಿಗೆ ಎಷ್ಟು ಬಂದರೂ ಅಸೂಯೆಯೆ. ಯಾರಿಗೆ ಇವನಿಗಿಂತ ಹೆಚ್ಚು ಬಂದಿದೆಯೋ ಅವರೊಂದಿಗೆ ಹೋಲಿಸಿಕೊಂಡು ವ್ಯಥೆ ಪಡುವನು. ಅಸೂಯೆಪಡದವನ ಸ್ವಭಾವವಾದರೋ ಹೋಲಿಸಿಕೊಳ್ಳುವುದಕ್ಕೆ ಹೋಗುವುದಿಲ್ಲ. ಅವನು ಎಲ್ಲವನ್ನೂ ದೇವರ ಇಚ್ಛೆಗೆ ಬಿಡುವನು.

ಕೆಲಸ ಮಾಡುತ್ತಿರುವಾಗ ಯಾವುದೋ ಒಂದರಲ್ಲಿ ಸಿದ್ಧಿಸುತ್ತದೆ. ಅಲ್ಲಿ ಗುರಿಯನ್ನು ಮುಟ್ಟುತ್ತಾನೆ. ಸಾಧಿಸಬೇಕಾಗಿರುವುದನ್ನು ಸಾಧಿಸುತ್ತಾನೆ. ಜಯ ಲಭಿಸುವುದು. ಆಗ ಅವನು ಸಂತೋಷ ಪಡುವುದೂ ಇಲ್ಲ. ಹಾಗೆಯೇ ಮತ್ತೊಂದರಲ್ಲಿ ಗುರಿ ಮುಟ್ಟಲಿಲ್ಲ ಎಂದು ಭಾವಿಸೋಣ. ಆಗ, ಮಾಡಿದ್ದು ನಿಷ್ಪ್ರಯೋಜನವಾಯಿತು ಎಂದು ಭಾವಿಸುವುದಿಲ್ಲ. ಅದರ ಹಿಂದೆಯೂ ದೇವರ ಕೆಲಸವೇ ಆಗಿದೆ. ದೇವರ ಮನಸ್ಸಿನಲ್ಲಿ ಏನಿತ್ತೋ ಅದು ನಮಗೆ ಗೊತ್ತಿಲ್ಲ. ಕೆಲವು ವೇಳೆ ಅವನ ಕೆಲಸ ನಾವು ಅರ್ಧ ಮಾಡುತ್ತಿರುವಾಗಲೇ ಆಗುವುದು. ಆಗ ಅದನ್ನು ಅಷ್ಟಕ್ಕೆ ನಿಲ್ಲಿಸೋಣ. ಇದನ್ನು ಅರಿಯದ ಮನುಷ್ಯನಿಗೆ ಪೂರ್ಣವಾಗಿ ಆಗಿದ್ದರೆ ಸಾರ್ಥಕವಾಗುತ್ತಿತ್ತು, ಈಗ ಇದು ವ್ಯರ್ಥ ಎಂಬ ಭಾವನೆ ಬರುವುದು. ನಮ್ಮ ಅಲ್ಪದೃಷ್ಟಿಗೆ ವ್ಯರ್ಥವಾಗಿರುವುದು, ಭೂಮದೃಷ್ಟಿಗೆ ಸಾರ್ಥಕವೇ ಆಗಿರಬಹುದು. ಕೆಲವು ವೇಳೆ ನಮ್ಮಲ್ಲಿರುವ ಚಟಾಕು ಬುದ್ಧಿಯಿಂದ ಪ್ರಪಂಚದಲ್ಲಿ ಎಲ್ಲವನ್ನೂ ತಿಳಿದುಕೊಳ್ಳುವುದಕ್ಕೆ ಹೋಗುತ್ತೇವೆ. ಆದರೆ ಯೋಗಿಗೆ ತನ್ನ ಬುದ್ಧಿಯ ಮಿತಿ ಚೆನ್ನಾಗಿ ಗೊತ್ತಿದೆ. ಅವನು ದೇವರು ಮಾಡಿಸಿದಂತೆ ಆಗಲಿ ಎನ್ನುವನು. ಇಂತಹ ಯೋಗಿ ಕರ್ಮ ಮಾಡುವಾಗ ಅವನಲ್ಲಿ ಸ್ವಾರ್ಥ ಲವಲೇಶವೂ ಇರುವುದಿಲ್ಲ. ಅವನು ತಾನು ಮಾಡುತ್ತಿದ್ದೇನೆ ಎನ್ನುವುದಿಲ್ಲ; ದೇವರು ಮಾಡುತ್ತಿರುವನು ಎನ್ನುವನು. ಇವನು ಏನು ಮಾಡಿದರೂ ಬದ್ಧನಾಗುವುದಿಲ್ಲ. ಇರುವಾಗಲೆ ಮುಕ್ತ. ಮುಕ್ತಿಯನ್ನು ಕಾಲವಾದಮೇಲೆ ಪಡೆಯಬೇಕಾಗಿಲ್ಲ.

\begin{shloka}
ಗತಸಂಗಸ್ಯ ಮುಕ್ತಸ್ಯ ಜ್ಞಾನಾವಸ್ಥಿತಚೇತಸಃ~।\\ಯಜ್ಞಾಯಾಚರತಃ ಕರ್ಮ ಸಮಗ್ರಂ ಪ್ರವಿಲೀಯತೇ \hfill॥ ೨೩~॥
\end{shloka}

\begin{artha}
ಸಂಗ ಹೋಗಿ ಮುಕ್ತನಾಗಿ, ಜ್ಞಾನದಲ್ಲಿ ನಿಂತುಕೊಂಡಿರುವ ಚಿತ್ತವುಳ್ಳವನಾಗಿ ಯಜ್ಞಕ್ಕೋಸುಗ ಕರ್ಮವನ್ನು ಮಾಡುತ್ತಿರುವವನಿಗೆ ಕರ್ಮ ಸಂಪೂರ್ಣವಾಗಿ ನಾಶವಾಗುತ್ತದೆ.
\end{artha}

ಯೋಗಿಯಲ್ಲಿ ಸಂಗ ನಾಶವಾಗಿದೆ. ಸಂಗ ಎಂಬುದೇ ಆಸೆಯ ಹಸಿ. ಒಂದು ಕೊಂಬೆಯನ್ನು ಮರದಿಂದ ಕಡಿದು ಇನ್ನೊಂದು ಕಡೆ ನೆಟ್ಟರೆ ಅದು ಪುನಃ ಚಿಗುರುವುದು. ಏಕೆಂದರೆ ಅದರಲ್ಲಿ ಇನ್ನೂ ಹಸಿ ಇದೆ. ಅದಿನ್ನೂ ಪೂರ್ತಿ ಸತ್ತಿಲ್ಲ. ಆದರೆ ಆ ಕೊಂಬೆ ಚೆನ್ನಾಗಿ ಒಣಗಿ ಹೋಗಿದ್ದರೆ ಅದನ್ನು ಹುಟ್ಟಿಹಾಕಿದರೆ ಚಿಗುರುವುದಿಲ್ಲ. ಅದರಂತೆಯೇ ಯೋಗಿಯ ಮನಸ್ಸಿನಲ್ಲಿ ಆಸೆಯ ಹಸಿ ಒಣಗಿಹೋಗಿದೆ. ಅದು ಮತ್ತೊಮ್ಮೆ ಚಿಗುರಲಾರದು. ಅವನು ಮುಕ್ತ. ಅವನಿಗೆ ಪ್ರಪಂಚದಿಂದ ಬೇಕಾಗಿರುವುದು ಏನೂ ಇಲ್ಲ. ಅವನು ತುಂಬಿದ ಪಾತ್ರೆ. ಏನು ಹೊಸದಾಗಿ ಬಿದ್ದರೂ ಕೆಳಗೆ ಹರಿದು ಹೋಗುವುದೇ ಹೊರತು ಅಲ್ಲಿ ನಿಲ್ಲುವುದಿಲ್ಲ. ಅವನು ಮಾಯೆಯ ಕನಸಿನಿಂದ ಎದ್ದಿರುವನು. ಕನಸಿನ ಆಕರ್ಷಣೆಗೆ ಇನ್ನುಮೇಲೆ ಬೀಳುವವನಲ್ಲ. ಮರೀಚಿಕೆಯನ್ನು ಮರೀಚಿಕೆಯಂತೆ ನೋಡುತ್ತಿರುವನು. ಇನ್ನುಮೇಲೆ ಅದನ್ನು ನೋಡಿ ಭ್ರಾಂತಿಪಡುವುದಿಲ್ಲ. ಮುಕ್ತನೆಂದರೆ ಭ್ರಾಂತಿಯಿಂದ ಎದ್ದವನು. ಇನ್ನುಮೇಲೆ ಅವನು ತನ್ನ ಹಿಂದಿನ ಸ್ಥಿತಿಗೆ ಹೋಗಲಾರ.

ಅವನ ಮನಸ್ಸಾದರೂ ಯಾವಾಗಲೂ ಸಮತ್ವದ ಜ್ಞಾನದಲ್ಲಿ ನೆಲಸಿದೆ. ವೈವಿಧ್ಯತೆಗಳ ಹಿಂದೆ ಇರುವ ಏಕಮಾತ್ರ ವಸ್ತುವನ್ನು ನೋಡುತ್ತಿರುವನು. ಅವನ ನಿಲುವೇ ಅದು. ಯಾವ ಸಮಯದಲ್ಲಿಯೂ ಅವನು ಅದನ್ನು ಮರೆಯುವುದಿಲ್ಲ. ಅದು ಅವನ ಸ್ವಭಾವವೇ ಆಗಿಹೋಗಿದೆ. ವೇಷವನ್ನು ಬೇಕಾದಾಗ ಹಾಕಿಕೊಳ್ಳಬಹುದು ಇಲ್ಲವೆ ಕಿತ್ತು ಇಡಬಹುದು. ಆದರೆ ನಮ್ಮ ಸಹಜ ಸ್ವಭಾವವನ್ನು ಕಳಚಿಡುವುದು ಹೇಗೆ?

ಅವನು ಬದುಕಿರುವ ಪರ್ಯಂತರ ಯಜ್ಞಕ್ಕೋಸ್ಕರ ಕರ್ಮವನ್ನು ಮಾಡುತ್ತಿರುವನು. ಯಜ್ಞವೆಂದರೆ ಪರಮಾತ್ಮನಿಗೆ ಅರ್ಪಿತವಾಗಲೆಂದು ತನ್ನ ಪಾಲಿಗೆ ಬಂದ ಕರ್ತವ್ಯಗಳನ್ನು ಮಾಡು\-ತ್ತಿರುವುದು. ಹೀಗೆ ದೇವರನ್ನು ಎದುರಿಗೆ ಇಟ್ಟುಕೊಂಡರೆ ನಾವು ಮಾಡುವ ಕರ್ಮವೆಲ್ಲ\break ಪೂಜೆಯಾಗುವುದು. ಅದು ಲೌಕಿಕವಾದ ಕರ್ಮವಾಗಿರಬಹುದು ಅಥವಾ ಭಗವತ್ಪೂಜೆಯಾಗಿರಬಹುದು. ಎಲ್ಲವೂ ದೇವರಿಗೆ ಅರ್ಪಿತ. ನರರೂಪಿ ನಾರಾಯಣನಿಗೆ ಮಾಡುತ್ತಿರುವ ಸೇವೆ. ಈ ಸೇವೆಯೇ ಯಜ್ಞ. ಈ ಯಜ್ಞವಾದರೂ ಬಹುರೂಪವನ್ನು ಧರಿಸಬಹುದು. ಅನ್ನ ಕೊಡಬಹುದು, ಬಟ್ಟೆಬರೆ ಕೊಡಬಹುದು, ರೋಗಿಗೆ ಔಷಧಿ ಕೊಡಬಹುದು, ವಿದ್ಯೆ ಕಲಿಸಬಹುದು. ಪರಮಾತ್ಮನ ವಿಷಯವನ್ನು ಹೇಳಬಹುದು, ಸಮಾಜದ ಯಾವ ಕಾರ್ಯವನ್ನಾದರೂ ಪೂಜಾಪೀಠಕ್ಕೆ ಏರಿಸಬಹುದು. ಭಗವದರ್ಪಣಭಾವದಿಂದ ಮಾಡಿದ ಕರ್ಮವೆಲ್ಲಾ ಪೂಜೆಯಾಗುವುದು. ಆ ಭಾವವನ್ನು ಮರೆತರೆ ಪೂಜೆಯೂ ಒಂದು ಬಂಧನಕಾರಿ. ತೋರಿಕೆಗೆ ಅಟ್ಟಹಾಸಕ್ಕಾಗಿ ಮಾಡಬಹುದು. ದೇವರಿಂದ ತನ್ನ ಸ್ವಾರ್ಥವನ್ನು ತೃಪ್ತಿ ಪಡಿಸಿಕೊಳ್ಳಲು ಏನನ್ನೋ ವಸೂಲಿ ಮಾಡುವುದಕ್ಕಾಗಿ ಪೂಜೆ ಮಾಡಬಹುದು.

ಇಂತಹ ಯೋಗಿ ಮಾಡಿದ ಕರ್ಮ ಬೇರುಸಹಿತ ನಾಶವಾಗಿದೆ. ಇನ್ನೊಮ್ಮೆ ಅದು ಚಿಗುರಲಾರದು. ಬೇರೊಂದು ಕರ್ಮವನ್ನು ಹುಟ್ಟಿಸಬಲ್ಲ ಬೀಜವನ್ನು ಕೊಡಲಾರದು. ಬೇರೊಂದು ಕರ್ಮ ಅವನಿಗೆ ಅಂಟಿಕೊಳ್ಳಲಾರದು. ಅವನು ಅದರಿಂದ ಕೈ ತೊಳೆದುಕೊಂಡಿರುವನು. ಅವನು ಮಾಡಿದ ಕರ್ಮವೆಲ್ಲ ಹುರಿದ ಬೀಜದಂತೆ ಆಗಿದೆ.

\begin{shloka}
ಬ್ರಹ್ಮಾರ್ಪಣಂ ಬ್ರಹ್ಮ ಹವಿರ್ಬ್ರಹ್ಮಾಗ್ನೌ ಬ್ರಹ್ಮಣಾ ಹುತಮ್~।\\ಬ್ರಹ್ಮೈವ ತೇನ ಗಂತವ್ಯಂ ಬ್ರಹ್ಮಕರ್ಮಸಮಾಧಿನಾ \hfill॥ ೨೪~॥
\end{shloka}

\begin{artha}
ಅರ್ಪಣ ಬ್ರಹ್ಮ, ಹವಿಸ್ಸು ಬ್ರಹ್ಮ, ಬ್ರಹ್ಮಾಗ್ನಿಯಲ್ಲಿ ಬ್ರಹ್ಮದಿಂದ ಹೋಮ ಮಾಡಲ್ಪಟ್ಟಿದೆ. ಬ್ರಹ್ಮಕರ್ಮ ಸಮಾಧಿಯಿಂದ ಆತನು ಪಡೆಯಬೇಕಾದದ್ದು ಬ್ರಹ್ಮವೇ.
\end{artha}

ಶ‍್ರೀಕೃಷ್ಣ ಇಲ್ಲಿ ಹಿಂದಿನಿಂದ ಬಂದ ಯಜ್ಞದ ಭಾವನೆಯನ್ನು ತೆಗೆದುಕೊಂಡು ಅದಕ್ಕೆ ಒಂದು ಹೊಸ ಅರ್ಥವನ್ನು ಕೊಡುತ್ತಾನೆ. ಇದೇ ನಮ್ಮಲ್ಲಿ ಸುಧಾರಣೆಯ ರೀತಿ. ಹಳೆಯದನ್ನು ಕಿತ್ತುಹಾಕಿ ಒಂದು ಹೊಸದನ್ನು ತರುವುದಿಲ್ಲ. ಯಾವುದಾದರೂ ಒಂದು ಚಿಹ್ನೆಯೋ, ನಿಮಿತ್ತವೋ ಆವಶ್ಯಕ. ಯಾವುದು ಜನಾಂಗದಲ್ಲಿ ಬಹಳಕಾಲದಿಂದ ಬಂದಿದೆಯೊ, ಯಾವುದು ಎಲ್ಲರಿಗೂ ಪರಿಚಯವಾಗಿದೆಯೊ ಆ ಮಧ್ಯವರ್ತಿಯನ್ನು ತೆಗೆದುಕೊಂಡು ಒಂದು ಅದ್ಭುತವಾದ ಹೊಸ ಅರ್ಥವನ್ನು ಕೊಡುವನು.

ಯಜ್ಞಕ್ಕಾಗಿ ಮಾಡುವುದೆಲ್ಲ ಪವಿತ್ರವೇ ಆಗುವುದು. ಯಜ್ಞವನ್ನು ಮಾಡಿ ಆದಮೇಲೆ ಏನು ಸಿಕ್ಕುವುದೋ ಅದೇ ಮುಖ್ಯವಲ್ಲ. ಆ ಫಲ ಸಿಕ್ಕುವುದಕ್ಕೆ ಏನೇನು ಮಾಡಿದೆಯೋ ಅದೆಲ್ಲ ಫಲಕ್ಕೆ ಸಮನಾಗುವುದು. ಒಂದು ದಾರಿಯನ್ನು ನಡೆದು ಅನಂತರ ಗುರಿಯನ್ನು ಸೇರುತ್ತೇವೆ. ಇಲ್ಲಿ ಸೇರುವ ಗುರಿ ಎಷ್ಟು ಮುಖ್ಯವೋ ನಡೆಯುವ ದಾರಿಯೂ ಅಷ್ಟೇ ಮುಖ್ಯ. ಶ‍್ರೀರಾಮಕೃಷ್ಣರು ಒಂದು ಉದಾಹರಣೆಯನ್ನು ಕೊಡುವರು. ಮಹಡಿ ಮೇಲಕ್ಕೆ ಹತ್ತಿಹೋಗುವಾಗ, ಇದಲ್ಲ, ಇದಲ್ಲ ಎಂದು ಮೆಟ್ಟಲನ್ನು ಬಿಟ್ಟು ಮೇಲುಮೇಲಕ್ಕೆ ಏರುವನು. ಅಲ್ಲಿಗೆ ಹೋಗಿ ಆದಮೇಲೆ ಅದು ಯಾವ ವಸ್ತುವಿನಿಂದ ಆಗಿದೆಯೋ ಅದೇ ವಸ್ತುವಿನಿಂದ ಎಲ್ಲವೂ ಆಗಿದೆ ಎಂಬುದು ಗೊತ್ತಾಗುವುದು. ಅಲ್ಲಿಗೆ ಹತ್ತಿಬಂದ ಮೆಟ್ಟಲು ಉಪ್ಪರಿಗೆ, ನೆಲದ ಅಂಗಳ, ಗೋಡೆ–ಎಲ್ಲ ಅದೇ ವಸ್ತುವಿನಿಂದ ಆಗಿದೆ. ಕೇವಲ ಫಲಾಪೇಕ್ಷೆಯ ದೃಷ್ಟಿಯಿಂದ ನೋಡಿದಾಗ ಉಳಿದವನ್ನೆಲ್ಲಾ ಫಲಕ್ಕೆ ಗೌಣ ಮಾಡುವೆವು. ಆದರೆ ಇವುಗಳೆಲ್ಲಾ ಯಾವುದರಿಂದ ಆಗಿದೆ, ಇವುಗಳೆಲ್ಲದರ ಸತ್ಯವೇನು ಎಂಬ ದೃಷ್ಟಿಯಿಂದ ನೋಡಿದರೆ ಒಂದರಷ್ಟೇ ಮತ್ತೊಂದು ಮುಖ್ಯವಾಗಿ ಕಾಣುವುದು.

ಅರ್ಪಣ ಬ್ರಹ್ಮ, ಯಜ್ಞಕ್ಕೆ ಹಲವು ಸಾಮಗ್ರಿಗಳನ್ನು ಉಪಯೋಗಿಸುತ್ತೇವೆ. ಅದೆಲ್ಲಾ ಬ್ರಹ್ಮ ಮಯ. ಹವಿಸ್ಸು ಬ್ರಹ್ಮ, ಯಜ್ಞಕ್ಕೆ ನಾವು ನೈವೇದ್ಯದಂತೆ ತುಪ್ಪ ಮುಂತಾದುವನ್ನು ಉಪಯೋಗಿಸುತ್ತೇವೆ. ಅದು ಬ್ರಹ್ಮ. ಯಾವ ಅಗ್ನಿಗೆ ಈ ಹವಿಸ್ಸನ್ನು ಹಾಕುವೆವೋ ಅದು ಬ್ರಹ್ಮ. ಇದನ್ನು ಮಾಡುವವನು ಬ್ರಹ್ಮ. ಮಾಡುವ ಕರ್ಮವು ಬ್ರಹ್ಮವೇ. ಕೊನೆಗೆ ಫಲರೂಪವಾಗಿ ಸಿಕ್ಕುವುದೂ ಬ್ರಹ್ಮವೇ.

ಅನೇಕ ವೇಳೆ \enginline{end justifies the means} ಎಂದು ಕೇಳಿರುವೆವು. ಗುರಿ ಸಾಧುವಾಗಿದ್ದರೆ ಸಾಕು, ಅದನ್ನು ಹೇಗೆ ಬೇಕಾದರೂ ಸಾಧಿಸಬಹುದು ಎಂದು. ಆದರೆ ಇಲ್ಲಿ ಸಾಧನೆ ಬೇರೆ ಅಲ್ಲ ಸಿದ್ಧಿ ಬೇರೆ ಅಲ್ಲ. ಸಾಧನೆ ಪರಿಶುದ್ಧವಾಗಿದ್ದರೆ ಸಿದ್ಧಿಯೂ ಪರಿಶುದ್ಧವಾಗಿರುವುದು. ಒಂದನ್ನು ಬಿಟ್ಟು ಮತ್ತೊಂ ದಿಲ್ಲ. ಒಂದರಲ್ಲಿರುವುದೇ ಮತ್ತೊಂದರಲ್ಲಿ ಇರುವುದು. ಮನುಷ್ಯ ತನ್ನ ಪಾಲಿಗೆ ಯಾವ ಕೆಲಸ ಬರಲಿ ಅದನ್ನು ಸಮಷ್ಟಿ ಎಂಬ ಪರಮಾತ್ಮನಿಗೆ ಮಾಡುತ್ತಿರುವ ಯಜ್ಞವೆಂದು ಭಾವಿಸಬೇಕು. ನಮ್ಮ ದೇಹದಲ್ಲೇ ಹಲವು ಭಾಗಗಳು ಒಟ್ಟು ದೇಹಕ್ಕೆ ದುಡಿಯುತ್ತಿವೆ. ಅವು ತಮಗಾಗಿ ಕೆಲಸ ಮಾಡುತ್ತಿಲ್ಲ. ಒಟ್ಟು ದೇಹರಕ್ಷಣೆಗೆ ಕೆಲಸ ಮಾಡುತ್ತಿವೆ. ಮೂಳೆಯ ಒಳಗೆ ರಕ್ತದ ಸೆಲ್​ಗಳು ಉತ್ಪತ್ತಿಯಾಗುತ್ತವೆ. ಎಲ್ಲಿಯೋ ಹರಿದುಕೊಂಡು ಹೋಗುವುದು, ಹುಟ್ಟುವುದು ಇನ್ನೆಲ್ಲೊ. ಹಗಲು ರಾತ್ರಿ ಹೃದಯ ಸಂಕೋಚ ವಿಕಾಸವಾಗುತ್ತ ರಕ್ತವನ್ನು ಬೇರೆ ಕಡೆಗೆ ಕಳುಹಿಸುತ್ತದೆ. ತನಗಾಗಿ ಅಲ್ಲ ಆ ಕ್ರಿಯೆ, ದೇಹಪೋಷಣೆಗಾಗಿ. ಅದರಂತೆಯೇ ಶ್ವಾಸಕೋಶ ಉಸಿರಾಡುತ್ತದೆ. ದೇಹದಲ್ಲಿ ಹಲವು ಕಡೆ ಇರುವ ಗ್ರಂಥಿಗಳು \enginline{(glands)} ತಾವು ತಯಾರುಮಾಡುವ ದ್ರವಗಳನ್ನು ದೇಹದ ಪೋಷಣೆಗೆ ಕೊಡುತ್ತಿವೆ. ಒಟ್ಟಿಗಾಗಿ ಎಲ್ಲವೂ. ಎಲ್ಲಿಯವರೆಗೆ ಈ ನಿಯಮ ಸರಿಯಾಗಿ ಪಾಲಿತವಾಗುತ್ತಾ ಇರುವುದೋ ಅಲ್ಲಿಯವರೆಗೆ ದೇಹ ಆರೋಗ್ಯದಲ್ಲಿರುವುದು. ಬೇರೆಬೇರೆ ಅಂಗಗಳು ತಾವು ಮಾಡುವ ಕೆಲಸವನ್ನು ಸರಿಯಾಗಿ ಮಾಡದೇ ಇದ್ದರೆ ಅನಾರೋಗ್ಯ. ಅದರಂತೆಯೇ ವಿರಾಟ್ ಪುರುಷ. ಇಲ್ಲಿ ಪ್ರತಿಯೊಂದೂ ಅವನ ದೇಹದಲ್ಲಿರುವ ಸಣ್ಣಸಣ್ಣ ವಸ್ತುಗಳು. ಪ್ರತಿಯೊಂದೂ ತಮ್ಮ ಪಾಲಿನದನ್ನು ವಿರಾಟ್ ಪುರುಷನಿಗೆ ಯಜ್ಞರೂಪದಿಂದ ಕೊಡುತ್ತಿದ್ದರೆ ಅವನು ಸುಪ್ರೀತನಾಗುತ್ತಾನೆ. ಆಗ ನಮಗೆ ಬೇಕಾದುದನ್ನೆಲ್ಲ ಕರುಣಿಸುವನು. ಕೊಡುವುದನ್ನು ಕೊಡದೆ ಇದ್ದರೆ ನಮಗೆ ಬೇಕಾಗಿರುವುದು ಬರುವುದಿಲ್ಲ. ಯಜ್ಞದ ಹಿಂದಿರುವ ಭಾವವೇ ಇದು. ಕೊಡುವುದಕ್ಕೆ ಹಲವಾರು ವಸ್ತುಗಳಿವೆ. ಇದರಲ್ಲಿ ಒಂದು ಮೇಲಲ್ಲ, ಮತ್ತೊಂದು ಕೀಳಲ್ಲ. ಒಂದರಷ್ಟೇ ಮತ್ತೊಂದು ಮುಖ್ಯ. ಇಲ್ಲಿ ಮುಖ್ಯವಾಗಿ ನಾವು ಗಮನಿಸಬೇಕಾಗಿರುವುದು, ಮಾಡುವ ಕೆಲಸದ ಹಿಂದೆ ಇರುವ ದೃಷ್ಟಿ. ಯಾವಾಗ ಅವನು ಯಜ್ಞ ದೃಷ್ಟಿಯಿಂದ ಮಾಡುತ್ತಾನೋ ಆಗ ಏನು ಮಾಡಿದರೂ ಬ್ರಹ್ಮಸಮಾಧಿಯೆ. ಇದೊಂದು ಗೀತೆಯಲ್ಲಿ ಬರುವ ಮಹೋನ್ನತ ಭಾವನೆ. ಮಾನವಕೋಟಿಯ ವಿಚಾರಪ್ರಣಾಳಿಗೆ ಒಂದು ಮಹೋನ್ನತ ಕೊಡುಗೆ ಈ ಯಜ್ಞ ಭಾವನೆ.

\begin{shloka}
ದೈವಮೇವಾಪರೇ ಯಜ್ಞಂ ಯೋಗಿನಃ ಪರ್ಯುಪಾಸತೇ~।\\ಬ್ರಹ್ಮಾಗ್ನಾವಪರೇ ಯಜ್ಞಂ ಯಜ್ಞೇನೈವೋಪಜುಹ್ವತಿ \hfill॥ ೨೫~॥
\end{shloka}

\begin{artha}
ಕೆಲವು ಯೋಗಿಗಳು ದೈವಯಜ್ಞವನ್ನೇ ಉಪಾಸನೆ ಮಾಡುವರು. ಇನ್ನು ಕೆಲವರು ಬ್ರಹ್ಮವೆಂಬ ಅಗ್ನಿಯಲ್ಲಿ ಯಜ್ಞವೆಂಬ ಯಜ್ಞದಿಂದಲೇ ಹೋಮಮಾಡುತ್ತಾರೆ.
\end{artha}

ಕೆಲವರು ಹಿಂದಿನಿಂದ ಬಂದ ದೇವತೆಗಳ ಹೆಸರಿನಲ್ಲಿ ಯಜ್ಞವನ್ನು ಮಾಡುತ್ತಾರೆ. ಇಲ್ಲಿ ಬದಲಾವಣೆಯನ್ನು ನೋಡುವುದಿಲ್ಲ. ಪೂರ್ವಾಚಾರವನ್ನು ಮುಂದುವರಿಸುತ್ತಿರುವರು. ಇದಕ್ಕೆ ಒಂದು ಹೊಸ ಅರ್ಥವನ್ನು ಕೊಡುವುದಕ್ಕಾಗಲಿ, ಹೊಸ ದೃಷ್ಟಿಯಿಂದ ಮಾಡುವುದಕ್ಕಾಗಲಿ ಪ್ರಯತ್ನಿಸುತ್ತಿಲ್ಲ. ಅಂತೂ ಇದಕ್ಕೆ ಒಂದು ಸ್ಥಾನವಿದೆ. ಸುಮ್ಮನೆ ಇರುವುದಕ್ಕಿಂತ ಇದನ್ನು ಮಾಡುವುದೆ ಮೇಲು.

ಅನಂತರವೇ ಬೇರೊಂದು ಬಗೆಯ ದೃಷ್ಟಿ ಬರುವುದು. ಅದೇ ಕೆಲವರು ಬ್ರಹ್ಮಾಗ್ನಿಯಲ್ಲಿ ಯಜ್ಞ ಮಾಡುತ್ತಾರೆ. ಇಲ್ಲಿ ಉರಿಯುತ್ತಿರುವ ಅಗ್ನಿ ಸಾಧಾರಣ ಕಟ್ಟಿಗೆಯಿಂದ ಆದದ್ದಲ್ಲ. ಸತ್ಯ ಜ್ಞಾನ ಅನಂತಸ್ವರೂಪನಾಗಿರುವ ಬ್ರಹ್ಮವೇ ಅಗ್ನಿ. ಯಜ್ಞದಿಂದಲೇ ಯಜ್ಞವನ್ನು ಹೋಮ\-ಮಾಡುತ್ತಾರೆ. ಆ ಯಜ್ಞಕ್ಕೆ ನಮ್ಮದಲ್ಲದಿರುವುದನ್ನೆಲ್ಲಾ ಹಾಕಿ ಪರಿಶುದ್ಧರಾಗುತ್ತಾರೆ. ಆ ಬ್ರಹ್ಮಾಗ್ನಿ ಉಪಾಧಿಗಳನ್ನೆಲ್ಲ ಭಸ್ಮೀಭೂತಮಾಡುವುದು. ಅಸತ್ಯವನ್ನೆಲ್ಲ ನಾಶಮಾಡುವುದು. ಭಕ್ತ ತನ್ನಲ್ಲಿ ಏನಿದೆಯೋ ಅದನ್ನೆಲ್ಲ ಭಗವಂತನೆಂಬ ಅಗ್ನಿಗೆ ಆಹುತಿಯಾಗಿ ಕೊಡುವನು. ಸಾಧಾರಣ ಮನುಷ್ಯನಿಗೆ ಪ್ರಿಯವಾದ ತನ್ನತನವನ್ನೇ ಭಗವಂತನಿಗೆ ಅರ್ಪಿಸುವನು. ಹೀಗೆ ಕೊಡುವುದರಲ್ಲಿ ಒಂದು ಆನಂದವಿದೆ. ನಾವು ಕ್ಷಣಿಕ ವಸ್ತುಗಳಿಗೆ, ವ್ಯಕ್ತಿಗಳಿಗೆ ನಮ್ಮನ್ನು ತೆರುತ್ತಿರುವೆವು. ಒತ್ತೆಯಿಡುತ್ತಿರುವೆವು. ನಮಗೆ ಅದರಿಂದ ಬಂದದ್ದು ಏನು? ತಾತ್ಕಾಲಿಕ ತೃಪ್ತಿ, ಅನಂತ ಕಷ್ಟ ಆಮೇಲೆ. ಆದರೆ ಭಗವಂತನಿಗೆ ನಮ್ಮನ್ನು ಅರ್ಪಣೆ ಮಾಡಿಕೊಳ್ಳುವಾಗಲೂ ಆನಂದವೇ, ಅನಂತರವೂ ಆನಂದವೇ. ನಾವು ಅವನ ವಸ್ತುವಾಗುತ್ತೇವೆ, ಅವನ ಪ್ರಸಾದವಾಗುತ್ತೇವೆ. ಪ್ರಪಂಚಕ್ಕೆ ಮಾರಿಕೊಂಡವನ ಬಾಳು ಎಂಜಲಾಗುವುದು, ಕೊಳೆತು ನಾರುವುದು.

ದೇವರನ್ನು ಬಲಿಪ್ರಿಯ ಎಂದು ಕರೆಯುತ್ತೇವೆ. ಅದಕ್ಕಾಗಿ ನಾವು ತೆಂಗಿನ ಕಾಯಿ, ಬಾಳೆ ಹಣ್ಣು ಮುಂತಾದುವನ್ನು ಕೊಡುತ್ತೇವೆ. ಆದರೆ ದೇವರಿಗೆ ತುಂಬಾ ಪ್ರಿಯವಾಗಿರುವುದು ನಮ್ಮ ವ್ಯಕ್ತಿತ್ವ, ನಮ್ಮತನ. ನಮ್ಮನ್ನೇ ಅವನಿಗೆ ಅರ್ಪಿಸಿಕೊಂಡರೆ ಅವನು ತನ್ನನ್ನು ನಮಗೆ ನೀಡುವನು. ಕೊರಡೊಂದು ಉರಿಯುತ್ತಿರುವ ಬೆಂಕಿಗೆ ಬೀಳುವುದು. ಆ ಬೆಂಕಿಯಾದರೋ ತನ್ನ ಧರ್ಮವನ್ನು ಆ ಸೌದೆಗೆ ಕೊಡುವುದು. ಅದನ್ನೂ ಬೆಂಕಿಯನ್ನಾಗಿ ಮಾಡುವುದು. ದೇವರಿಗೆ ಕೊಟ್ಟರೆ ಏನು ಗತಿ ಎಂದು ನಾವು ಒದ್ದಾಡುತ್ತೇವೆ. ನಾವು ಕೊಟ್ಟಿದ್ದೆಲ್ಲಾ ಪವಿತ್ರವಾಗುವುದು. ನಾವು ಯಾವುದನ್ನೂ ಕೊಡದೆ ರಕ್ಷಿಸಿಕೊಂಡಿದ್ದೇವೋ ಅದರಿಂದಲೇ ನಷ್ಟವಾಗುವುದು. ಕೊಡಿ ನಿಮ್ಮಲ್ಲಿರುವುದನ್ನೆಲ್ಲ ಭಗವಂತನಿಗೆ; ಅರ್ಪಣಭಾವದಿಂದ ಕೊಡಿ; ಎಂದು ಬೋಧಿಸುವುದು ಯಜ್ಞತತ್ತ್ವ. ಹಾಗೆ ಕೊಡುವುದೇ ಆನಂದ. ಆನಂದಗಳಲ್ಲಿ ಹಿರಿಯಾನಂದ ಅದು. ಅದನ್ನು ಮೀರಿದ ಆನಂದ ಇಲ್ಲ. ನಾವು ಕೊಟ್ಟಿದ್ದು ಭಗವಂತನ ಹತ್ತಿರ ಸುರಕ್ಷಿತವಾಗಿರುವಷ್ಟು ಇನ್ನೆಲ್ಲಿಯೂ ಇರುವುದಿಲ್ಲ. ನಾವು ಕೊಟ್ಟದ್ದನ್ನು ದೇವರು ರಕ್ಷಿಸುವುದು ಮಾತ್ರವಲ್ಲ. ಅದಕ್ಕೆ ಹೆಚ್ಚಾದ ಬಡ್ಡಿಯನ್ನು ಸೇರಿಸಿಡುವನು. ನಮ್ಮ ಒಂದು ಹತ್ತಾಗುವುದು, ಹತ್ತು ನೂರಾಗುವುದು, ನೂರು ಸಾವಿರವಾಗುವುದು.

\begin{shloka}
ಶ್ರೋತ್ರಾದೀನೀಂದ್ರಿಯಾಣ್ಯನ್ಯೇ ಸಂಯಮಾಗ್ನಿಷು ಜುಹ್ವತಿ~।\\ಶಬ್ದಾದೀನ್ ವಿಷಯಾನನ್ಯ ಇಂದ್ರಿಯಾಗ್ನಿಷು ಜುಹ್ವತಿ \hfill॥ ೨೬~॥
\end{shloka}

\begin{artha}
ಕೆಲವರು ಶ್ರೋತ್ರವೇ ಮೊದಲಾದ ಇಂದ್ರಿಯಗಳನ್ನು ಸಂಯಮಾಗ್ನಿಯಲ್ಲಿ ಹೋಮಮಾಡುತ್ತಾರೆ. ಮತ್ತೆ ಕೆಲವರು ಶಬ್ದವೇ ಮುಂತಾದ ವಿಷಯಗಳನ್ನು ಇಂದ್ರಿಯವೆಂಬ ಅಗ್ನಿಯಲ್ಲಿ ಹೋಮ\-ಮಾಡುತ್ತಾರೆ.
\end{artha}

ಇಂದ್ರಿಯಗಳು ಆಯಾ ವಸ್ತುವಿನ ಕಡೆ ಹರಿದುಹೋಗುವುದನ್ನು ತಡೆಗಟ್ಟುವರು. ಸಾಧಾರಣವಾಗಿ ಇಂದ್ರಿಯಗಳು ಅದಕ್ಕೆ ಸಂಬಂಧಪಟ್ಟ ವಸ್ತುವಿನ ಕಡೆ ಯಾವಾಗಲೂ ಜಾರಿಕೊಂಡು ಹೋಗುತ್ತಿರುವುವು. ಯೋಗಿಯಾದವನು ಪ್ರಾರಂಭದಲ್ಲಿಯೇ ಅವನ್ನು ನಿಗ್ರಹಿಸುವನು.\break ಇಂದ್ರಿಯ ಮತ್ತು ವಿಷಯವಸ್ತು ಎರಡೂ ಸೇರಿದಾಗ ಒಂದು ಅನುಭವ ಆಗುವುದು. ಆಗ ಮನಸ್ಸಿನ ಸಂಸ್ಕಾರ ಉಂಟಾಗುವುದು. ವಿಷಯವಸ್ತುವಿನ ಮೇಲೆ ನಮಗೆ ಹತೋಟಿ ಇಲ್ಲ. ಎಲ್ಲಾ ಕಡೆಯಲ್ಲಿಯೂ ಕೇಳುವುದು, ನೋಡುವುದು, ಮುಟ್ಟುವುದು, ಮೂಸಿನೋಡುವುದು, ರುಚಿ ನೋಡುವುದು ಬಿದ್ದಿರುವುದು. ಅವುಗಳಿಂದ ನಾವು ತಪ್ಪಿಸಿಕೊಂಡು ಹೋಗುವುದಕ್ಕೆ ಆಗುವುದಿಲ್ಲ. ಆದರೆ ನನ್ನ ಇಂದ್ರಿಯ ನನಗೆ ಸ್ವಾಧೀನವಾಗಿದ್ದರೆ ಎರಡೂ ಸೇರದಂತೆ ನೋಡಿಕೊಳ್ಳಬಹುದು. ಎರಡೂ ಕೈಸೇರಿದರೆ ಚಪ್ಪಾಳೆ. ಮತ್ತೊಂದು ಕೈಯನ್ನು ಸೇರಿಸದೆ ಇದ್ದರೆ ಆ ಸದ್ದೇ ಆಗುವುದಿಲ್ಲ. ಯೋಗಿ ಮಾಡುವುದೇ ಇದನ್ನು. ಸಂಯಮ ಎಂಬ ಅಗ್ನಿಯಲ್ಲಿ ಇಂದ್ರಿಯಗಳನ್ನು ಅರ್ಪಿಸುವನು. ಎಂದರೆ ಅವನ್ನು ನಿಗ್ರಹಿಸುವನು.

ಮತ್ತೆ ಕೆಲವರು ಇಂದ್ರಿಯದ ಅಗ್ನಿಯಲ್ಲಿ ಶಬ್ದವೇ ಮುಂತಾದ ವಿಷಯಗಳನ್ನು ಅರ್ಪಿಸುವರು. ಇನ್ನು ಕೆಲವರು ಇರುವರು. ಅವರು ಇಂದ್ರಿಯವನ್ನು ಸಂಪೂರ್ಣವಾಗಿ ನಿಗ್ರಹಿಸುವುದಿಲ್ಲ. ಅದಕ್ಕೆ ಧರ್ಮಕ್ಕೆ ವಿರೋಧವಲ್ಲದ ವಿಷಯವಸ್ತುಗಳನ್ನು ಅರ್ಪಿಸುವರು. ವಿಷಯವಸ್ತುವಿಗೆ ದಾಸರಾಗದೆ ವಿಷಯವಸ್ತುವಿನಲ್ಲಿ ಉತ್ತಮವಾಗಿರುವುದನ್ನು ಸ್ವೀಕರಿಸುವರು. ದೇಹವೆಂಬ ಯಂತ್ರವಿದೆ. ಅದು ಕೆಲಸಮಾಡಬೇಕಾದರೆ, ಅದಕ್ಕೆ ಶಕ್ತಿಯನ್ನು ಕೊಡುವ ಆಹಾರವನ್ನು ಕೊಡಬೇಕು. ದೇಹಪೋಷಣೆಗೆ ಅವನು ಆಹಾರವನ್ನು ತೆಗೆದುಕೊಳ್ಳುವನು. ರುಚಿಯಾಸೆಗೆ ತೆಗೆದುಕೊಳ್ಳುವುದಿಲ್ಲ. ದೇಹವನ್ನು ಶೀತೋಷ್ಣದಿಂದ ರಕ್ಷಿಸಬೇಕಾಗಿದೆ. ಅದಕ್ಕಾಗಿ ಬಟ್ಟೆಯನ್ನು ಹಾಕಿಕೊಳ್ಳುವನು. ಅಲಂಕಾರಕ್ಕಲ್ಲ. ಉಪಯೋಗಕ್ಕೆ ಮಾತ್ರ. ಅದರಂತೆಯೇ ನೋಡುವನು. ಪವಿತ್ರವಾದ ಮಂಗಳವಾದ ರೂಪಗಳನ್ನು ನೋಡುವನು. ಅದರಿಂದ ಉತ್ತಮ ಸಂಸ್ಕಾರ ಮನಸ್ಸಿನಲ್ಲಿ ಉಂಟಾಗುವಂತೆ ಮಾಡಿಕೊಳ್ಳುವನು. ಕೇಳುವನು, ಕಾಡುಹರಟೆಯನ್ನಲ್ಲ, ಒಳ್ಳೆಯ ಮಾತುಕತೆಗಳನ್ನು. ಹೀಗೆ ಯೋಗಿ ಇಂದ್ರಿಯಗಳ ಮೂಲಕ ತನ್ನ ಉತ್ತಮ ಸಂಸ್ಕಾರಗಳು ಉಂಟಾಗುವಂತೆ ಮಾಡಿಕೊಳ್ಳುವನು. ಯಾರನ್ನೂ ಒಳಗೆ ಬಿಡದೆ ಸಂಪೂರ್ಣವಾಗಿ ಇಂದ್ರಿಯಗಳ ಬಾಗಿಲನ್ನು ಹಾಕುವುದಿಲ್ಲ. ಚೆನ್ನಾಗಿ ವಿಮರ್ಶಿಸಿ ಯೋಗ್ಯರಾದವರನ್ನು ಬಿಡುತ್ತಾನೆ, ಅಯೋಗ್ಯರಾದವರನ್ನು ತಡೆಗಟ್ಟುತ್ತಾನೆ.

\begin{shloka}
ಸರ್ವಾಣೀಂದ್ರಿಯಕರ್ಮಾಣಿ ಪ್ರಾಣಕರ್ಮಾಣಿ ಚಾಪರೇ~।\\ಆತ್ಮಸಂಯಮಯೋಗಾಗ್ನೌ ಜುಹ್ವತಿ ಜ್ಞಾನದೀಪಿತೇ \hfill॥ ೨೭~॥
\end{shloka}

\begin{artha}
ಇನ್ನು ಕೆಲವರು ಎಲ್ಲಾ ಇಂದ್ರಿಯ ಕರ್ಮಗಳನ್ನು ಮತ್ತೂ ಪ್ರಾಣಕರ್ಮಗಳನ್ನು ಜ್ಞಾನದಿಂದ ಹೊತ್ತಿಸಿರುವ ಆತ್ಮಸಂಯಮ ಯೋಗಾಗ್ನಿಯಲ್ಲಿ ಹೋಮ ಮಾಡುತ್ತಾರೆ.
\end{artha}

ಮತ್ತೆ ಕೆಲವರು ತಾವು ಯಾವ ಕರ್ಮವನ್ನು ಮಾಡುತ್ತಿದ್ದರೂ, ಅದು ಕರ್ಮೇಂದ್ರಿಯ, ಜ್ಞಾನೇಂದ್ರಿಯಕ್ಕೆ ಸೇರಿರಬಹುದು ಅಥವಾ ಪ್ರಾಣ, ಅಪಾನ, ವ್ಯಾನ, ಉದಾನ, ಸಮಾನ ಇವಕ್ಕೆ ಸಂಬಂಧಪಟ್ಟ ಕರ್ಮಗಳಾಗಿರಬಹುದು, ಅದಕ್ಕೆಲ್ಲ ಒಂದು ಗುರಿ ಇದೆ. ಅದೇ ವಿವೇಕದಿಂದ ಪ್ರಜ್ವಲಿಸುತ್ತಿರುವ ಜ್ಞಾನಾಗ್ನಿಗೆ ಆಹುತಿಯಂತೆ ಅರ್ಪಿಸುವುದು. ಅವರು ಮಾಡುವ ಕ್ರಿಯೆಯ ಮುಂದೆಲ್ಲ ಭಗವದರ್ಪಿತ ಭಾವವನ್ನು ಇಟ್ಟುಕೊಂಡಿರುವರು. ಏನನ್ನು ಮಾಡುತ್ತಿದ್ದರೂ ಅದು ಭಗವಂತನಿಗೆ ಅರ್ಪಿತವಾಗಲಿ ಎಂದು ಯೋಚಿಸುವರು. ಆಗ ಮಾಡುವ ಕೆಲಸ ಯಾವುದೂ ನಷ್ಟವಾಗುವುದಿಲ್ಲ. ಅವು ಉತ್ತಮವಾದ ಆಧ್ಯಾತ್ಮಿಕ ಸಂಸ್ಕಾರಗಳನ್ನು ನಮ್ಮಲ್ಲಿ ಉಂಟುಮಾಡುವುವು. ಯಾವಾಗ ಅರ್ಪಿತ ಭಾವವನ್ನು ನಾವು ಮನಸ್ಸಿನ ಮುಂದೆ ಇಟ್ಟುಕೊಂಡಿರುವೆವೊ ಆಗ ಕೆಟ್ಟದ್ದನ್ನು ಮಾಡುವುದಕ್ಕೆ ಆಗುವುದಿಲ್ಲ. ಏಕೆಂದರೆ ದೇವರನ್ನು ಕುರಿತು ಚಿಂತಿಸಿದೊಡನೆ ಕೆಟ್ಟದ್ದು ಕಾಲಿಗೆ ಬೀಳುವುದು. ಅನೇಕವೇಳೆ ನಾವು ಕೆಟ್ಟದ್ದನ್ನು ಮಾಡುವುದಕ್ಕೆ ಕಾರಣ ದೇವರನ್ನು ಮರೆತಿರುವುದು. ದೇವರನ್ನು ಚಿಂತಿಸಲು ಪ್ರಾರಂಭಿಸಿದೊಡನೆ ಕೆಟ್ಟದ್ದು ಮಾಯವಾಗುವುದು. ನಾವು ಮಾಡುವ ಕರ್ಮವೆಲ್ಲ ಭಗವಂತನಿಗೆ ಅರ್ಪಿಸುವುದಕ್ಕೆ ಎಂಬ ಆದರ್ಶದ ಧ್ರುವತಾರೆ ನಮ್ಮೆದುರಿಗೆ ಇದ್ದರೆ, ಕೆಟ್ಟದ್ದನ್ನು ಮಾಡುವುದಕ್ಕೇ ಮನಸ್ಸು ಬರುವುದಿಲ್ಲ.

\begin{shloka}
ದ್ರವ್ಯಯಜ್ಞಾಸ್ತಪೋಯಜ್ಞಾ ಯೋಗಯಜ್ಞಾಸ್ತಥಾಪರೇ~।\\ಸ್ವಾಧ್ಯಾಯಜ್ಞಾನಯಜ್ಞಾಶ್ಚ ಯತಯಃ ಸಂಶಿತ ವ್ರತಾಃ \hfill॥ ೨೮~॥
\end{shloka}

\newpage

\begin{artha}
ಕೆಲವರು ದ್ರವ್ಯಯಜ್ಞವನ್ನು ಮಾಡುತ್ತಾರೆ. ಮತ್ತೆ ಕೆಲವರು ತಪೋಯಜ್ಞವನ್ನು ಮಾಡುತ್ತಾರೆ. ಕೆಲವರು ಯೋಗಯಜ್ಞವನ್ನು ಮಾಡುತ್ತಾರೆ, ಕೆಲವರು ಸ್ವಾಧ್ಯಾಯ ಜ್ಞಾನಯಜ್ಞಗಳನ್ನು ಮಾಡುತ್ತಾರೆ. ಇವರೆಲ್ಲ ದೃಢವ್ರತರಾದ ಯತಿಗಳೇ.
\end{artha}

ಕೆಲವರು ಸ್ಥೂಲವಾದ ತಮ್ಮ ಹತ್ತಿರ ಇರುವ ಹಣವನ್ನು ಯಜ್ಞರೂಪದಂತೆ ಪಾತ್ರರಾದವರಿಗೆ ದಾನ ಮಾಡುವರು. ಇದೂ ಒಂದು ಯಜ್ಞವೇ. ಜೀವನದಲ್ಲಿ ಕೊಡುವುದು ಒಂದು ಕಲೆ. ಆ ಕಲೆ ಕೊಡುವವರಿಗೆಲ್ಲ ಗೊತ್ತಿಲ್ಲ. ಹಲವಾರು ಉದ್ದೇಶಗಳಿಂದ ಹಣವನ್ನು ಕೊಡಬಹುದು. ಯಶಸ್ಸನ್ನು ಗಳಿಸುವುದಕ್ಕೆ, ತಾವು ಮಾಡಿರುವ ಪಾಪದಿಂದ ಪಾರಾಗುವುದಕ್ಕೆ, ಒಬ್ಬನ ಕಾಟದಿಂದ ಪಾರಾಗುವುದಕ್ಕೆ, ಏನನ್ನಾದರೂ ಕೊಟ್ಟು ಕಳುಹಿಸುವನು. ಇವುಗಳೆಲ್ಲ ಯಜ್ಞವಾಗುವುದಿಲ್ಲ. ತೈತ್ತಿರೀಯ ಉಪನಿಷತ್ತಿನಲ್ಲಿ ಕೊಡುವ ವಿಷಯದಲ್ಲಿ ಆ ಮಾತನ್ನು ಹೇಳುತ್ತಾರೆ: ಶ್ರದ್ಧೆಯಿಂದ ಕೊಡು, ಐಶ್ವರ್ಯಕ್ಕೆ ತಕ್ಕಂತೆ ಕೊಡು, ವಿನಯದಿಂದ ಕೊಡು, ಮಿತ್ರಭಾವದಿಂದ ಕೊಡು. ಯಜ್ಞವಾಗಬೇಕಾದರೆ ಇವುಗಳು ಇರಬೇಕು. ಶ್ರದ್ಧೆಯಿಂದ ಕೊಡಬೇಕು. ಏನೋ ಕಾಟಾಚಾರಕ್ಕೆ ಕೊಡುವುದಲ್ಲ. ಕೊಡುವುದು ಒಂದು ಪೂಜೆ ಮಾಡಿದಂತೆ. ಅದರಿಂದ ನಾನು ಉದ್ಧಾರವಾಗುತ್ತೇನೆ, ಇದು ನನಗೆ ಒದಗಿರುವ ಒಂದು ಅವಕಾಶ ಎಂದು ಭಾವಿಸಬೇಕು. ನಾನು ಕೊಡದೇ ಇದ್ದರೆ ಸ್ವೀಕರಿಸುವವರಿಗೆ ಮತ್ತಾರೂ ಕೊಡುವುದಿಲ್ಲ ಎಂದು ಭಾವಿಸಬಾರದು. ನನ್ನಿಂದ ಅಲ್ಲ ಅವನ ಉದ್ಧಾರವಾಗುವುದು. ಎಲ್ಲರನ್ನೂ ಉದ್ಧರಿಸುವವನು ಅವನನ್ನು ಉದ್ಧರಿಸುವನು. ನಾನೊಂದು ನಿಮಿತ್ತ ಎಂದು ಭಾವಿಸಬೇಕು. ಐಶ್ವರ್ಯಕ್ಕೆ ತಕ್ಕಂತೆ ಕೊಡಬೇಕು. ತನ್ನ ಯೋಗ್ಯತೆಗೆ ತಕ್ಕಂತೆ ಕೊಡಬೇಕು. ಯಾವಾಗ ತುಂಬ ಕಡಮೆ ಕೊಡುತ್ತೇನೋ ಅದೊಂದು ಅವಮಾನ. ಇತರರು ನಾನು ಕೊಟ್ಟ ಅಲ್ಪವನ್ನು ನೋಡಿ ಹೆಚ್ಚು ಕೊಡಬೇಕೆಂದಿದ್ದರೂ ಕಡಮೆ ಕೊಡುವರು. ನನ್ನ ಜಿಪುಣತನ ಉಳಿದೆಲ್ಲರಮೇಲೂ ತನ್ನ ಪ್ರಭಾವವನ್ನು ಬೀರುವುದು. ಕೊಡುವಾಗ ಔದಾರ್ಯಕ್ಕೆ ಮೇಲ್ಪಂಕ್ತಿಯಾಗಬೇಕು. ವಿನಯದಿಂದ ಕೊಡು ಎಂದು ಹೇಳುತ್ತಾರೆ. ಕೊಡುವಾಗ ಅಹಂಕಾರ ಇರಕೂಡದು. ಅಟ್ಟಹಾಸ ಇರಕೂಡದು. ನನ್ನ ಸಮಾನ ಇಲ್ಲ ಎಂದು ಮೆರೆಯಬಾರದು. ನನಗೆ ಇನ್ನೂ ಜಾಸ್ತಿ ಕೊಡುವುದಕ್ಕೆ ಸಾಧ್ಯವಾಗಲಿಲ್ಲವಲ್ಲ ಎಂದು ಪರಿತಪಿಸುವಂತಿರಬೇಕು. ಮಿತ್ರಭಾವದಿಂದ ಕೊಡಬೇಕು. ತುಂಬಾ ಮೇಲಿನಮಟ್ಟದಲ್ಲಿ ನಿಂತುಕೊಂಡು ಅತಿ ಕೆಳಗಿನವನಿಗೆ ಎಸೆಯು\-ವಂತಿರಬಾರದು. ಕೊಡುವವನು ಸ್ವೀಕರಿಸುವವನಿಗಿಂತ ಮೇಲೆ ಇದ್ದಾನೆ ಎಂದು ಭಾವಿಸಬಾರದು. ಎಲ್ಲರೂ ಒಂದೇ. ಅಕಸ್ಮಾತ್ ನಾನು ಕೊಡುವ ಸ್ಥಿತಿಯಲ್ಲಿದ್ದೇನೆ. ಸ್ವೀಕರಿಸುವವನನ್ನು ಸ್ನೇಹಿತನಂತೆ ನೋಡಬೇಕು. ಎಲ್ಲರೂ ಭಗವಂತನ ಮಕ್ಕಳು. ಎಲ್ಲರೂ ಸಹೋದರ ಸಹೋದರಿಯರು. ಯಾರೂ ಮೇಲಲ್ಲ, ಯಾರೂ ಕೀಳಲ್ಲ. ಈ ಭಾವನೆಗಳಿದ್ದರೆ ಕೊಡುವುದೊಂದು ಕಲೆಯಾಗುವುದು, ಅದೊಂದು ಯಜ್ಞವಾಗುವುದು. ಜೀವನದಲ್ಲಿ ಐಶ್ವರ್ಯವಂತರಿ ಗೆಲ್ಲ ಈ ಕಲೆ ಗೊತ್ತಿಲ್ಲ, ಕೊಡುವವರಿಗೆಲ್ಲ ಈ ಯಜ್ಞದ ಭಾವನೆ ಗೊತ್ತಿಲ್ಲ.

\newpage

ಕೆಲವರು ತಪಸ್ಸನ್ನು ಯಜ್ಞದಂತೆ ಮಾಡುವರು. ಭಗವಂತನನ್ನು ಚಿಂತಿಸುವುದು ಒಂದು ತಪಸ್ಸು. ಒಬ್ಬೊಬ್ಬರು ಒಂದೊಂದು ರೀತಿ ಆ ತಪಸ್ಸನ್ನು ಆಚರಿಸುತ್ತಾರೆ. ಕುಳಿತು ಧ್ಯಾನಿಸುವವರು ಕೆಲವರು. ಶಿರದ ಮೇಲೆ ನಿಂತು ತಪಸ್ಸುಮಾಡುವವರು ಕೆಲವರು, ನಿಂತು ಧ್ಯಾನಿಸುವವರು ಕೆಲವರು, ನೀರಿನಲ್ಲಿ ನಿಂತು ಧ್ಯಾನಿಸುವವರು ಕೆಲವರು, ಸುತ್ತಲೂ ಬೆಂಕಿ, ಮೇಲೆ ಕೋರೈಸುತ್ತಿರುವ ಸೂರ್ಯ ಇವುಗಳ ಮಧ್ಯದಲ್ಲಿ ತಪಸ್ಸು ಮಾಡುವವರು ಕೆಲವರು. ಎಷ್ಟೋ ವಿಧವಾದ ತಪಸ್ಸುಗಳಿವೆ. ಯಾವಾಗ ಇವುಗಳನ್ನು ಭಗವದ್​ದರ್ಶನಕ್ಕಾಗಿ ಮಾಡುತ್ತೇವೆಯೋ ಇದೊಂದು ಯಜ್ಞವಾಗುವುದು. ಇವುಗಳಿಂದ ಯಾವುದಾದರೂ ಸಿದ್ಧಿಯನ್ನು ಪಡೆಯಬೇಕೆಂದು ಮಾಡಿದರೆ ಯಜ್ಞಭಾವನೆ ಹೋಗುವುದು.

ಯೋಗ ಎಂದರೆ ಚಿತ್ತವೃತ್ತಿಯನ್ನು ನಿರೋಧಿಸುವುದು ಎಂದು ಪತಂಜಲಿ ಹೇಳುತ್ತಾರೆ. ಚಿತ್ತದ ಸ್ವಭಾವ ಯಾವಾಗಲೂ ಅಲೆಗಳಿಂದ ಕೂಡಿರುವುದು. ಹೊರಗಿನಿಂದ ಕಲ್ಲಿನಂತೆ ಸುದ್ದಿಸಮಾಚಾರಗಳು ಮನಸ್ಸಿನ ಸರೋವರಕ್ಕೆ ಬೀಳುತ್ತಿರುತ್ತವೆ. ಇಂದ್ರಿಯಗಳ ಬಾಗಿಲನ್ನು ಮುಚ್ಚಿ ಹೊರಗಿನಿಂದ ಕಲ್ಲುಗಳು ಚಿತ್ತಸರೋವರಕ್ಕೆ ಬೀಳದಂತೆ ನೋಡಿಕೊಳ್ಳುವನು, ಯೋಗಿ. ಮನಸ್ಸನ್ನೆಲ್ಲ ಏಕಾಗ್ರ ಮಾಡಿ ಭಗವಂತನ ಕಡೆ ಹರಿಯುವಂತೆ ಮಾಡುವನು. ಇದೊಂದು ಬಗೆಯ ಯಜ್ಞ.

ಹಲವು ಶಾಸ್ತ್ರಾದಿಗಳನ್ನು ವಿಧಿಪ್ರಕಾರ ಕಲಿಯುವರು. ಕಲಿಯುವುದೂ ಒಂದು ಯಜ್ಞ, ಇದೊಂದು ತಮಾಶೆ ಅಲ್ಲ. ಆಮೇಲೆ ಅದನ್ನು ಬಿಕರಿ ಮಾಡುವ ವ್ಯಾಪಾರವಲ್ಲ. ಜೀವನದಲ್ಲಿ ಶಾಸ್ತ್ರಾದಿಗಳನ್ನು ಓದುವುದು ಒಂದು ಪವಿತ್ರವಾದ ಕೆಲಸ. ಅದನ್ನು ಓದಬೇಕು, ಅದರಲ್ಲಿರುವುದನ್ನು ತಿಳಿದುಕೊಳ್ಳಬೇಕು, ತಿಳಿದುಕೊಂಡುದನ್ನು ಮನನ ಮಾಡುತ್ತಿರಬೇಕು. ಅನಂತರ ಜ್ಞಾನಯಜ್ಞವಿದೆ. ನಾನೇನು ತಿಳಿದುಕೊಂಡಿರುವೆನೋ ಅದನ್ನು ಇತರರಿಗೆ ಯಜ್ಞದಂತೆ ಕೊಡುತ್ತೇನೆ. ಭಗವದರ್ಪಿತವಾಗಲಿ ಎಂದು ಕಲಿತದ್ದನ್ನು ಭಗವಂತನ ಮಕ್ಕಳಿಗೆ ನೀಡುತ್ತೇನೆ. ಉಪನಿಷತ್ತಿನಲ್ಲಿ, ಅಧ್ಯಯನ, ಪ್ರವಚನ ಇವನ್ನು ಮರೆಯಬೇಡಿ ಎಂದು ಹೇಳುವರು. ನಾವು ಕಲಿಯಬೇಕು, ಕಲಿತುದನ್ನು ಮತ್ತೊಬ್ಬನಿಗೆ ಹೇಳಿಕೊಡಬೇಕು. ಕಲಿತದ್ದು ಪೂರ್ಣವಾಗಬೇಕಾದರೆ ಅದನ್ನು ಮತ್ತೊಬ್ಬನಿಗೆ ಹೇಳಿಕೊಟ್ಟಾಗಲೇ.

ಇವರೆಲ್ಲ ದೃಢವ್ರತರಾದ ಸಾಧಕರೆ. ಈ ಯಜ್ಞಗಳ ಗುರಿಯೆಲ್ಲ ಒಂದೇ. ಅದೇ ಭಗವಂತನಿಗೆ ಅರ್ಪಣೆ ಮಾಡುವುದು. ಯಾರಿಗೆ ಯಾವುದನ್ನು ಕೊಡಲು ಸಾಧ್ಯವೋ ಅದನ್ನು ಯಜ್ಞದೃಷ್ಟಿಯಿಂದ ಕೊಡಲಿ. ಆಗ ಕೊಡುವವನು, ಸ್ವೀಕರಿಸುವವನು ಇಬ್ಬರೂ ಉದ್ಧಾರವಾಗುತ್ತಾರೆ. ಒಬ್ಬ ಕೊಟ್ಟು ಉದ್ಧಾರವಾಗುತ್ತಾನೆ, ಮತ್ತೊಬ್ಬ ಕೊಡುವವನಿಗೆ ಒಂದು ಅವಕಾಶವನ್ನು ಕಲ್ಪಿಸಿ ಉದ್ಧಾರವಾಗುತ್ತಾನೆ.

\begin{shloka}
ಅಪಾನೇ ಜುಹ್ವತಿ ಪ್ರಾಣಂ ಪ್ರಾಣೇಽಪಾನಂ ತಥಾಪರೇ~।\\ಪ್ರಾಣಾಪಾನಗತೀ ರುದ್ಧ್ವಾ ಪ್ರಾಣಾಯಾಮಪರಾಯಣಾಃ \hfill॥ ೨೯~॥
\end{shloka}

\begin{artha}
ಕೆಲವರು ಅಪಾನದಲ್ಲಿ ಪ್ರಾಣವನ್ನೂ ಪ್ರಾಣದಲ್ಲಿ ಅಪಾನವನ್ನೂ ಹೋಮಮಾಡುತ್ತಾರೆ. ಮತ್ತೆ ಕೆಲವರು ಪ್ರಾಣಾಪಾನಗಳೆರಡನ್ನೂ ನಿರೋಧಿಸಿ ಕುಂಭಕವೆಂಬ ಪ್ರಾಣಾಯಾಮವನ್ನು ಮಾಡುತ್ತಾರೆ.
\end{artha}

ಪತಂಜಲಿಯ ಅಷ್ಟಾಂಗಯೋಗದಲ್ಲಿ ಪ್ರಾಣಾಯಾಮ, ಯಮ ನಿಯಮಗಳು ಆದಮೇಲೆ ಬರುವುದು. ಇಲ್ಲಿ ಪ್ರಾಣಾಯಾಮದ ನಂತರ ಪ್ರತ್ಯಾಹಾರ, ಧ್ಯಾನ, ಧಾರಣ, ಸಮಾಧಿಗಳು ಬರುವುವು. ಇದನ್ನು ಅಭ್ಯಾಸ ಮಾಡುವವರು ಒಂದು ಬಗೆಯ ಯೋಗಿಗಳು. ನಾವು ಉಸಿರಾಡುವುದಕ್ಕೂ ನಮ್ಮ ಮನಸ್ಸಿಗೂ ಒಂದು ಸಂಬಂಧವನ್ನು ಯೋಗಿಗಳು ಕಂಡುಹಿಡಿದಿರುವರು. ನಮ್ಮ ಮನಸ್ಸನ್ನು ನಿಗ್ರಹಿಸುವುದು ಕಷ್ಟ. ಆದರೆ ಉಸಿರಾಟವನ್ನು ಒಂದು ನಿಯಮಕ್ಕೆ ತರುವುದು ಸುಲಭ. ನಾವು ಶಾಂತಚಿತ್ತರಾಗಿದ್ದಾಗ ಉಸಿರಾಡುವುದು ಬಹಳ ನಿಧಾನವಾಗುವುದು. ನಾವು ಉದ್ವೇಗವಶರಾದಾಗ ಉಸಿರಾಡುವುದು ಜಾಸ್ತಿಯಾಗಿರುವುದು. ಅದರಂತೆಯೇ ಉಸಿರಾಟವನ್ನು ನಿಧಾನ ಮಾಡಿದರೆ ಉದ್ವೇಗವೂ ಕಡಿಮೆಯಾಗುವುದನ್ನು ನೋಡುವೆವು. ಈ ಉಸಿರಾಟವನ್ನು ಮೂರು ಭಾಗ ಮಾಡಿರುವರು. ರೇಚಕ, ಪೂರಕ, ಕುಂಭಕ ಎಂದು–ಉಸಿರನ್ನು ಒಳಗೆ ಸೆಳೆದುಕೊಳ್ಳುವುದು, ಉಸಿರನ್ನು ಹೊರಗೆ ಬಿಡುವುದು, ಉಸಿರನ್ನು ಒಳಗೇ ನಿಲ್ಲಿಸುವುದು. ಇದನ್ನು ಪ್ರತಿಯೊಬ್ಬರೂ ಪ್ರತಿನಿತ್ಯವೂ ಮಾಡುತ್ತಿರುವೆವು. ಯೋಗಿಯಾದವನು ಇದರಿಂದ ಮಾನಸಿಕ ಪ್ರಯೋಜನವನ್ನು ಪಡೆಯಲು ಯತ್ನಿಸುವನು. ಉಸಿರನ್ನು ಸೆಳೆದುಕೊಳ್ಳುವಾಗ ಪವಿತ್ರತೆಯನ್ನೆಲ್ಲ ಉಸಿರಿನಿಂದ ಒಳಗಡೆ ಸೆಳೆದುಕೊಳ್ಳುತ್ತಿರುವನು ಎಂದು ಭಾವಿಸುವನು. ಉಸಿರನ್ನು ಬಿಡುವಾಗ ನಮ್ಮಲ್ಲಿರುವ ಅಪೂರ್ಣತೆಗಳೆಲ್ಲ ಹೊರಗೆ ಹೋಗುತ್ತಿದೆ ಎಂದು ಭಾವಿಸುವನು. ಕುಂಭಕದಲ್ಲಿ ಉಸಿರನ್ನು ತಡೆಗಟ್ಟುವಾಗ ಉಸಿರನ್ನು ಬಿಡುವುದು, ತೆಗೆದುಕೊಳ್ಳುವುದು ಎರಡನ್ನೂ ಅರ್ಪಿಸುವನು. ನಾವು ಇವೆರಡು ಕ್ರಿಯೆಗಳನ್ನೂ ಮಾಡುತ್ತಿರುವುದು ಕುಂಭಕದಲ್ಲಿ ಭಗವಂತನೊಡನೆ ಒಂದಾಗುವುದಕ್ಕೆ ಎಂದು ಭಾವಿಸುವನು. ಸುಮ್ಮನೆ ದಿನವೆಲ್ಲಾ ನಾವು ಮಾಡುತ್ತಿರುವ ಉಸಿರಾಟವನ್ನು ಒಂದು ಮಾನಸಿಕ ಸಾಧನೆಯನ್ನಾಗಿ ಪರಿವರ್ತಿಸಿ ಅದರಿಂದ ಪ್ರಯೋಜನವನ್ನು ಪಡೆಯುವನು ಯೋಗಿ. ನೀರೊಂದು ಸುಮ್ಮನೆ ಹರಿದು ಹೋಗುತ್ತಿದೆ. ಅದರ ಮಧ್ಯದಲ್ಲಿ ಒಂದು ಚಕ್ರವನ್ನು ಇಟ್ಟು ಅದನ್ನು ತಿರುಗುವಂತೆ ಮಾಡಿ ಅದರಿಂದ ಸ್ವಲ್ಪ ವಿದ್ಯುಚ್ಛಕ್ತಿಯನ್ನೋ ಅಥವಾ ಒಂದು ಬೀಸುವ ಯಂತ್ರವನ್ನೋ ತಿರುಗುವಂತೆ ಮಾಡುವಂತೆ ಇದು. ಶಕ್ತಿ ಸುಮ್ಮನೆ ವ್ಯಯವಾಗುತ್ತಿದೆ. ಅದರ ಉಪಯೋಗ ಪಡೆದುಕೊಳ್ಳುವವನು ಯೋಗಿ.

\begin{shloka}
ಅಪರೇ ನಿಯತಾಹಾರಾಃ ಪ್ರಾಣಾನ್ ಪ್ರಾಣೇಶು ಜುಹ್ವತಿ~।\\ಸರ್ವೇಽಪ್ಯೇತೇ ಯಜ್ಞವಿದೋ ಯಜ್ಞಕ್ಷಪಿತ ಕಲ್ಮಶಾಃ \hfill॥ ೩ಂ~॥
\end{shloka}

\begin{artha}
ಇನ್ನು ಕೆಲವರು ನಿಯತಾಹಾರಿಗಳಾಗಿ ಪ್ರಾಣಗಳನ್ನು ಪ್ರಾಣಗಳಲ್ಲಿ ಹೋಮಮಾಡುತ್ತಾರೆ.\break ಇವರೆಲ್ಲರೂ ಯಜ್ಞವನ್ನು ಬಲ್ಲವರು. ಯಜ್ಞಗಳಿಂದ ಕಲ್ಮಶಗಳನ್ನು ತೊಳೆದುಕೊಂಡಿರುವರು.
\end{artha}

\newpage

ನಿಯತಾಹಾರ ಯೋಗಿಗಳಿಗೆ ಅತ್ಯಂತ ಮುಖ್ಯವಾಗಿರುವುದು. ಯಾವ ಯೋಗಿಗಳು ಹೆಚ್ಚು ಊಟ ಮಾಡುವರೋ, ಅಥವಾ ಕಡಿಮೆ ಊಟ ಮಾಡುವರೋ ಅವರಿಗೆ ಯೋಗ ಸಾಧ್ಯವಾಗುವುದಿಲ್ಲ. ಊಟ ಹೆಚ್ಚು ಮಾಡಿದರೆ ಮನಸ್ಸು ತಮೋಗುಣಕ್ಕೆ ಇಳಿಯುತ್ತದೆ. ಹೆಚ್ಚು ನಿದ್ರೆ ಮುಂತಾದುವುಗಳನ್ನು ಮಾಡಬೇಕಾಗುವುದು. ಮನಸ್ಸನ್ನು ಏಕಾಗ್ರಗೊಳಿಸಲು ಆಗುವುದಿಲ್ಲ. ಕಡಿಮೆ ಊಟ ಮಾಡಿದರೂ ಮನಸ್ಸು ದುರ್ಬಲವಾಗುವುದು. ಅದು ಯಾವ ಕೆಲಸವನ್ನೂ ಮಾಡುವ ಸ್ಥಿತಿಯಲ್ಲಿರುವುದಿಲ್ಲ. ಆದಕಾರಣವೇ ಯೋಗಿಯಾದವನು ಮಿತಾಹಾರಿಯಾಗಿರಬೇಕು.

ಇವರು ಪ್ರಾಣಗಳನ್ನು ಪ್ರಾಣಗಳಲ್ಲಿ ಹೋಮಮಾಡುತ್ತಾರೆ. ನಿಗ್ರಹಿಸಿದ ಪ್ರಾಣದಲ್ಲಿ ನಿಗ್ರಹಿಸದ ಪ್ರಾಣವನ್ನು ಹೋಮ ಮಾಡುತ್ತಾರೆ. ಯೋಗಿಯ ದೃಷ್ಟಿಯಲ್ಲಿ ಪ್ರಾಣ ಅತಿ ಮುಖ್ಯವಾದ ಕೆಲಸವನ್ನು ಮಾಡುತ್ತದೆ. ಪ್ರಾಣ, ಅಪಾನ, ವ್ಯಾನ, ಉದಾನ, ಸಮಾನ ಇವುಗಳೆಲ್ಲ ಪ್ರಾಣದ ಕ್ರಿಯೆಗಳು. ನಮ್ಮ ಜ್ಞಾನೇಂದ್ರಿಯ ಕರ್ಮೇಂದ್ರಿಯ ಕೆಲಸಗಳೆಲ್ಲವೂ ಪ್ರಾಣದ ಆವರಣಕ್ಕೆ ಒಳಪಟ್ಟ ಕ್ರಿಯೆಗಳು. ದೇಹದ ಪ್ರತಿಯೊಂದು ಅಂಗಾಂಗದ ಮೇಲೂ ಕೂಡ ಯೋಗಿ ತನ್ನ ಸ್ವಾಮಿತ್ವವನ್ನು ಸ್ಥಾಪಿಸಬಲ್ಲ. ಹೃದಯಕ್ರಿಯೆಯನ್ನು ಕೂಡ ತನ್ನ ಇಚ್ಛಾನುಸಾರ ಮಾಡಬಲ್ಲನು. ಇದನ್ನು ಅನೇಕರು ಪರೀಕ್ಷಿಸಿದ್ದಾರೆ. ನಿಗ್ರಹಿಸಿದ ಪ್ರಾಣದ ಮೂಲಕ ನಿಗ್ರಹಿಸದೆ ಇರುವ ಪ್ರಾಣಗಳನ್ನು ನಿಗ್ರಹಿಸುವುದಕ್ಕೆ ಯತ್ನಿಸುವನು. ಇದು ಪಳಗಿಸಿದ ಆನೆಯ ಸಹಾಯದಿಂದ ಕಾಡಾನೆಯನ್ನು ಕಟ್ಟಿಹಾಕುವಂತೆ. ನಾವು ಇದನ್ನು ಆನೆ ಹಿಡಿಯುವವರಲ್ಲಿ ನೋಡುತ್ತೇವೆ. ಇದೇ ಉಪಾಯವನ್ನು ಯೋಗಿ ತನ್ನ ಮನಸ್ಸಿನಲ್ಲಿರುವ ಕಾಡಾನೆಯನ್ನು ಹಿಡಿಯುವುದಕ್ಕೆ ಉಪಯೋಗಿಸುವನು.

ಇವರೆಲ್ಲರೂ ಯಜ್ಞಗಳನ್ನು ತಿಳಿದವರು ಎಂದು ಶ‍್ರೀಕೃಷ್ಣ ಹೇಳುತ್ತಾನೆ. ದ್ರವ್ಯ ಯಜ್ಞವಾಗಲಿ, ತಪೋಯಜ್ಞವಾಗಲಿ, ಸ್ವಾಧ್ಯಾಯ ಯಜ್ಞವಾಗಲಿ, ಜ್ಞಾನಯಜ್ಞವಾಗಲಿ ಎಲ್ಲವನ್ನೂ ಪರಮಾತ್ಮನಿಗೆ ಅರ್ಪಿತ ದೃಷ್ಟಿಯಿಂದ ಯಾರು ಮಾಡುವರೋ ಅವರಿಗೆಲ್ಲಾ ಯೋಗದ ಅರ್ಥ ಗೊತ್ತಿದೆ. ಯೋಗ ಎಂದರೆ ಒಂದುಗೂಡಿಸುವುದು. ಈ ಜೀವಿಯನ್ನು ಪರಮಾತ್ಮನ ಕಡೆ ಹೋಗುವಂತೆ ಮಾಡಿ ಅಲ್ಲಿ ಅದನ್ನು ಕಟ್ಟಿಹಾಕುವವನೇ ಯೋಗಿ. ಹಾಗೆ ದೇವರಡಿಗೆ ಓಡಿಸುವುದಕ್ಕೆ ಒಬ್ಬೊಬ್ಬರು ಒಂದೊಂದು ಉಪಾಯವನ್ನೂ ದಾರಿಯನ್ನೂ ಹಿಡಿದಿರುವರು. ಎಲ್ಲರ ಗುರಿ ದೇವರೆಂಬ ಕೇಂದ್ರವೇ. ವೃತ್ತದ ಲ್ಲಿರುವ ಬೇರೆ ಬೇರೆ ಕಡೆಗಳಿಂದ ಕೇಂದ್ರದ ಕಡೆಗೆ ಎಲ್ಲರೂ ಹೋಗುತ್ತಿರುವರು. ಶ‍್ರೀಕೃಷ್ಣ ಇಲ್ಲಿ, ಇಂತಹವರು ಹೆಚ್ಚು, ಇಂತಹವರು ಕಡಿಮೆ ಎನ್ನುವುದಕ್ಕೆ ಹೋಗುವುದಿಲ್ಲ. ಯಾವಾಗ ಗುರಿಯನ್ನು ಮರೆತು ಅಡ್ಡಹಾದಿಯಲ್ಲಿ ಸಿಕ್ಕುವ ಯಾವುದಾದರೂ ಆಕರ್ಷಣೆಯಲ್ಲಿ ನಿರತರಾದರೆ ಮುಂದುವರಿ ಯುವುದು ನಿಲ್ಲುವುದು. ಇದು ಯೋಗದ ಎಲ್ಲಾ ದಾರಿಗಳಿಗೂ ಅನ್ವಯಿಸುವುದು. ದ್ರವ್ಯವನ್ನು ದಾನಮಾಡುವಾಗ ಕೀರ್ತಿ ಆಸೆಗೆ, ಅದರಿಂದ ಬರುವ ಅಧಿಕಾರದಾಸೆಗೆ ಇವನ್ನು ಮಾಡಿದರೆ ಇವನು ಗುರಿ ಮರೆಯುವನು. ಜ್ಞಾನವನ್ನು ಎಲ್ಲರಿಗೂ ಧಾರಾಳವಾಗಿ ಕೊಡುವ ಬದಲು ಅದನ್ನು ಮಾರಲು ಹೊರಟರೆ ಅದೊಂದು ವ್ಯಾಪಾರವಾಗುವುದು, ಸಾಧನೆ ಆಗುವುದಿಲ್ಲ. ಅದರಂತೆಯೇ ಪ್ರಾಣಾಯಾಮ ಮುಂತಾದ ಕ್ರಿಯೆಗಳನ್ನು ಅಭ್ಯಾಸ ಮಾಡುವಾಗ ಅದರಿಂದ ಬರುವ ಯಾವುದಾದರೂ ಪವಾಡಗಳಿಗೆ ಆಕರ್ಷಿತನಾದರೆ ಅವನು ಅದನ್ನು ಜನರಿಗೆ ತೋರಿ, ದೊಂಬರಾಟವನ್ನು ಮಾಡಿ, ಹಣವನ್ನೊ ಕೀರ್ತಿಯನ್ನೊ ಗಳಿಸಿ ಸೇರಬೇಕಾದ ಗುರಿಯನ್ನು ಮರೆಯುವನು. ಎಲ್ಲಾ ಮಾರ್ಗಗಳಲ್ಲಿಯೂ ಅಡ್ಡಹಾದಿಗಳಿವೆ. ಅದರಲ್ಲಿ ಪ್ರಯಾಣ ಮಾಡುತ್ತಿರುವ ಯಾತ್ರಿಕ ಯಾವಾಗಲೂ ಯಜ್ಞ ಭಾವನೆಯನ್ನು ಮರೆಯಕೂಡದು.

ಮೇಲೆ ಹೇಳಿದ ಯಾವ ಸಾಧನೆಯನ್ನಾದರೂ ಆಗಲಿ ಯಜ್ಞಭಾವದಿಂದ ಮಾಡಿದರೆ ಅವರೆಲ್ಲ ತಮ್ಮ ಪಾಪದಿಂದ ಪಾರಾಗುವರು. ಚಿತ್ತಶುದ್ಧಿಯಾಗಿ ಎಲ್ಲರಲ್ಲಿಯೂ ಸದಾ ಉರಿಯುತ್ತಿರುವ ಸ್ವಯಂ ಜ್ಯೋತಿಸ್ವರೂಪನಾದ ಭಗವಂತ ಬೆಳಗುವುದನ್ನು ನೋಡುತ್ತೇವೆ. ಯೋಗದ ಒಂದು ಮುಖ್ಯ ಕ್ರಿಯೆಯು ಚಿತ್ತಶುದ್ಧಿ. ಯಾವಾಗ ಶುದ್ಧ ಚಿತ್ತವನ್ನು ದೇವರ ಕಡೆಗೆ ಹರಿಸುವೆವೋ ದೇವರಿಗೆ ಸಂಬಂಧಪಟ್ಟ ವಿಷಯಗಳೆಲ್ಲಾ ಗೋಚರವಾಗುವುವು. ಏಕಾಗ್ರವಾದ ಶುದ್ಧಚಿತ್ತಕ್ಕೆ ಒಂದು ಅದ್ಭುತ ಶಕ್ತಿ ಬರುವುದು. ಅದನ್ನು ಯಾವ ವಸ್ತುವಿನ ಕಡೆ ತಿರುಗಿಸಿದರೆ ಆ ವಸ್ತುವಿನ ರಹಸ್ಯವೆಲ್ಲಾ ಗೊತ್ತಾಗುವುದು. ಸೂರ್ಯನ ಕಿರಣ ಎಲ್ಲಾ ಕಡೆಯೂ ಬೀಳುತ್ತಿದೆ. ಅದನ್ನು ಒಂದು ಬೂದುಗನ್ನಡಿಯ ಮೇಲೆ ಬೀಳುವಂತೆ ಮಾಡಿ ಅದರ ಕೆಳಗೆ ಫೋಕಲ್ ಪಾಯಿಂಟಿನ ಕೆಳಗೆ ನಮ್ಮ ಕೈಯನ್ನು ಇಟ್ಟರೆ ಅದು ತಕ್ಷಣವೇ ಬರೆ ಹಾಕುವುದು. ಹಾಗೆಯೇ ಶುದ್ಧವಾದ ಚಿತ್ತ. ಅದಕ್ಕೊಂದು ಅದ್ಭುತ ಶಕ್ತಿ ಪ್ರಾಪ್ತವಾಗುವುದು. ನಮ್ಮ ಜಡ ಕಣ್ಣಿಗೆ ಕಾಣದುದು ಅದಕ್ಕೆ ಕಾಣುವುದು. ಒಂದು ಮೈಕ್ರಾಸ್ಕೋಪಿನ ಕೆಳಗಡೆ ಒಂದು ವಸ್ತುವನ್ನು ಇಟ್ಟರೆ ಅದರಲ್ಲಿರುವ ಮೊದಲು ಕಣ್ಣಿಗೆ ಕಾಣದ ಸೂಕ್ಷ್ಮವಸ್ತುಗಳೆಲ್ಲ ಬೃಹದಾಕಾರವನ್ನು ತಾಳಿ ವ್ಯಕ್ತವಾಗುವುವು. ಹಾಗೆ ಯೋಗಿಯ ಮನಸ್ಸಾಗುವುದು. ಯೋಗಿಯ ಮನಸ್ಸಿನ ಮುಂದೆ ಯಾವ ರಹಸ್ಯವನ್ನೂ ಬಚ್ಚಿಡುವುದಕ್ಕಾಗುವುದಿಲ್ಲ.

\begin{shloka}
ಯಜ್ಞ ಶಿಷ್ಟಾಮೃತಭುಜೋ ಯಾಂತಿ ಬ್ರಹ್ಮ ಸನಾತನಮ್~।\\ನಾಯಂ ಲೋಕೋಽಸ್ತ್ಯಯಜ್ಞಸ್ಯ ಕುತೋಽನ್ಯಃ ಕುರುಸತ್ತಮ \hfill॥ ೩೧~॥
\end{shloka}

\begin{artha}
ಯಜ್ಞಶಿಷ್ಟವಾದ ಅಮೃತವನ್ನು ಭೋಗಿಸುವವರು ಸನಾತನವಾದ ಬ್ರಹ್ಮವನ್ನು ಪಡೆಯುತ್ತಾರೆ. ಅರ್ಜುನ, ಯಜ್ಞ ಮಾಡದವನಿಗೆ ಈ ಲೋಕವೇ ಇಲ್ಲ. ಮತ್ತೊಂದು ಎಲ್ಲಿಂದ ಬರುವುದು?
\end{artha}

ಯಜ್ಞವನ್ನು ಮಾಡಿದ ಮೇಲೆ ಭಗವಂತನಿಗೆ ಅರ್ಪಿಸಿಯಾದಮೇಲೆ, ಉಳಿಯುವುದೇ ಅಮೃತ. ದೇವರು ನಮಗೆ ಏನನ್ನು ಕೊಟ್ಟಿರುವನೋ, ಅದು ಐಶ್ವರ್ಯವಾಗಿರಬಹುದು, ವಿದ್ಯೆಯಾಗಿರಬಹುದು, ತಪಸ್ಸು, ಧ್ಯಾನ ಏನು ಬೇಕಾದರೂ ಆಗಿರಬಹುದು. ನಾವು ಅದನ್ನು ಪರಿಶುದ್ಧ ಮಾಡಬೇಕಾದರೆ ಭಗವಂತನಿಗೆ ಅರ್ಪಿಸಬೇಕು. ಅರ್ಪಿಸಿದ ಮೇಲೆ ಅದು ಪ್ರಸಾದವಾಗುವುದು, ಪರಿಶುದ್ಧವಾಗುವುದು. ಅವನಿಗೆ ಅರ್ಪಣೆ ಮಾಡಿದ ಮೇಲೆ ನಮಗೇನು\break ಉಳಿಯುವು\-ದೆಂದು ನಾವು ಅಂಜಬೇಕಾಗಿಲ್ಲ. ಅವನು ನಾವು ಕೊಟ್ಟದ್ದನ್ನೆಲ್ಲಾ ತೆಗೆದುಕೊಂಡು\-ಬಿಡುವುದಿಲ್ಲ. ಅದನ್ನು ಪರಿಶುದ್ಧ ಮಾಡಿಕೊಡುವನು. ಅದು ನಾವು ಮುಂಚೆ ಇಟ್ಟುಕೊಂಡಿರು\-ವುದಕ್ಕಿಂತ ಶ್ರೇಷ್ಠವಾಗುವುದು. ವಿದ್ಯೆಯನ್ನು ನಾನು ಸಂಪಾದಿಸಿರು ವೆನು. ಕೆಲವು ಪುಸ್ತಕಗಳನ್ನು ಓದಿ ಹೇಗೋ ಪಾಸಾಗಿರುವೆನು. ನಾನು ಅದನ್ನು ಇತರರಿಗೆ ಹೇಳಬೇಕಾದರೆ, ನಾನು ಪಾಸು ಮಾಡುವುದಕ್ಕೆ ಮಾತ್ರ ಓದಿರುವುದು ಸಾಲದು. ವಿಷಯವನ್ನು ಮತ್ತೂ ಆಳವಾಗಿ ತಿಳಿದುಕೊಂಡಿರಬೇಕು. ಯಾವ ಪ್ರತಿಫಲಾಪೇಕ್ಷೆಯೂ ಇಲ್ಲದೆ ಅದನ್ನು ಮತ್ತೊಬ್ಬನಿಗೆ ನೀಡಬೇಕು. ಅನಂತರ ನಮ್ಮಲ್ಲಿ ಉಳಿಯುವ ಜ್ಞಾನವೇ ತಿಳಿಯಾದುದು, ಪವಿತ್ರವಾದುದು. ಇದರಂತೆಯೇ ಜೀವನದಲ್ಲಿ ಭಗವಂತ ಪ್ರತಿ ಜೀವಿಗೆ ಇತ್ತಿರುವುದೆಲ್ಲ. ಮೊದಲು ಕೊಡಬೇಕು, ಅನಂತರ ಕೊಟ್ಟು ಮಿಕ್ಕಿರುವುದನ್ನು ಅನುಭವಿಸಬೇಕು. ಅಲ್ಲೇ ನಿಜವಾದ ಆನಂದವಿರುವುದು. ಕೇವಲ ನಾನೇ ಅದನ್ನು ಅನುಭವಿಸಿದರೆ ಸ್ವಾರ್ಥವಾಗುವುದು. ಮನು, ಅಡಿಗೆ ಮಾಡುವಾಗ ಇದು ದೇವರ ನೈವೇದ್ಯಕ್ಕೆ ಮತ್ತು ಅತಿಥಿ ಪೂಜೆಗೆ ಎಂಬುದನ್ನು ಮರೆತರೆ, ಕೇವಲ ನಮ್ಮ ಹಸಿವನ್ನು ಮಾತ್ರ ತೃಪ್ತಿಪಡಿಸಿಕೊಳ್ಳುವುದಕ್ಕೆ ಅಡಿಗೆ ಮಾಡಿದರೆ ಅದು ಪೈಶಾಚಿಕ ಭೋಜನವಾಗುವುದು ಎನ್ನುವನು.

ಯಾರು ಯಜ್ಞರೂಪದಿಂದ ನಮಗೆ ದೇವರು ಕೊಟ್ಟಿರುವುದನ್ನು ಪುನಃ ಹಿಂತಿರುಗಿ ದೇವರಿಗೆ ಕೊಡುತ್ತಾರೋ ಅವರು ಸನಾತನವಾದ ಬ್ರಹ್ಮವನ್ನು ಪಡೆಯುತ್ತಾರೆ ಎನ್ನುವನು. ಯಾವಾಗ ನಾವು ದೇವರಿಗೆ ಫಲಾಪೇಕ್ಷೆಯಿಲ್ಲದೆ ಕೊಡುತ್ತೇವೆಯೋ ಅದು ನಮ್ಮ ಚಿತ್ತವನ್ನು ಶುದ್ಧಮಾಡುವುದು. ನಮ್ಮಲ್ಲಿರುವ ಸಂಸ್ಕಾರಗಳನ್ನೆಲ್ಲಾ ದಹಿಸುವುದು. ಅಂತಹವನು ಸನಾತನವಾದ ಬ್ರಹ್ಮವನ್ನು ಪಡೆಯುತ್ತಾನೆ. ಈ ಸಂಸಾರಚಕ್ರಕ್ಕೆ ಮತ್ತೊಮ್ಮೆ ಸಿಕ್ಕುವುದಿಲ್ಲ. ಜನನ, ಜರ, ಮರಣದಿಂದ ಪಾರಾಗುವನು.

ಯಾವನು ಯಜ್ಞವನ್ನು ಮಾಡುವುದಿಲ್ಲವೋ ಅವನಿಗೆ ಈ ಲೋಕವೇ ಇಲ್ಲ, ಎಂದರೆ ಈ ಲೋಕದಲ್ಲಿ ಆ ಮನುಷ್ಯ ಸುಖವಾಗಿರಲಾರ. ಕೇವಲ ಸ್ವಾರ್ಥಿ ತನಗೆ ಮತ್ತು ತಾನು–ಎರಡನ್ನೇ ಪರಿಗಣಿಸುವನು. ಇಂತಹವನು ಹೇಗೆ ಸುಖಿಯಾಗಿದ್ದಾನು? ಜೀವನದ ನಿಯಮವೇ ನಾವು ಏನು ಕೊಟ್ಟಿರುವೆವೋ ನಮಗೆ ಅದು ಬರುವುದು. ಕೊಟ್ಟಿದ್ದು ಹತ್ತಾಗಿ, ನೂರಾಗಿ, ಸಾವಿರವಾಗಿ ಬೇಡವೆಂದರೂ ದಮ್ಮಯ್ಯಗುಡ್ಡೆ ಇಟ್ಟುಕೊಂಡು ನನ್ನ ಹತ್ತಿರ ಬರುವುದು. ಇನ್ನೊಬ್ಬನಿಗೆ ಸುಖ ಕೊಟ್ಟಿದ್ದರೆ ನನಗೆ ಸುಖ ಬರುವುದು. ಇನ್ನೊಬ್ಬನಿಗೆ ಶಾಂತಿ ಕೊಟ್ಟಿದ್ದರೆ ನನಗೆ ಶಾಂತಿ ಬರುವುದು. ಇನ್ನೊಬ್ಬನಿಗೆ ವಿದ್ಯೆ ಕೊಟ್ಟಿದ್ದರೆ ನನಗೆ ಹೆಚ್ಚು ವಿದ್ಯೆ ಹತ್ತುವುದು. ಒಳ್ಳೆಯ ಭಾವನೆಗಳನ್ನು ಇತರರಿಗೆ ಕಳಿಸಿದ್ದರೆ ಇತರರಿಂದ ಆ ಭಾವನೆಗಳು ನನ್ನೆಡೆಗೆ ಹರಿದು ಬರುವುವು. ಯಾವಾಗ ನನ್ನ ಸುಖ ಆನಂದಕ್ಕಾಗಿ ಇತರರನ್ನು ಗೋಳೈಸಿರುವೆನೋ, ಅವರ ದುಃಖ ಸಂಕಟದ ಮೇಲೆ ನನ್ನ ಸುಖದ ಮನೆಯನ್ನು ಕಟ್ಟಿಕೊಂಡಿರುವೆನೋ, ಅದು ಜ್ವಾಲಾಮುಖಿಯ ನೆತ್ತಿಯಮೇಲೆ ಕಟ್ಟಿದ ಮನೆಯಂತೆ. ಎಂದೋ ಭುಗಿಲೆಂದು ಕಾರುವುದು. ನಮ್ಮ ಸುಖದ ಮನೆ ಬೂದಿಪಾಲಾಗುವುದು. ಜೀವನದ ಗಾಢ ನಿಯಮ ಇದು. ನಾವು ಸುಖಿಯಾಗಿರಬೇಕಾದರೆ ಇತರರಿಗೆ ಸುಖವನ್ನು ಕೊಟ್ಟಿರಬೇಕು, ನಾವು ಸಂತೋಷದಿಂದ ಇರಬೇಕಾದರೆ ಇತರರಿಗೆ ಸಂತೋಷವನ್ನು ಕೊಟ್ಟಿರಬೇಕು. ಇದೇ ಯಜ್ಞ ಮಾಡುವುದು ಎಂದರೆ ಅರ್ಥ.

ಯಾರು ಯಜ್ಞ ಮಾಡಿಲ್ಲವೋ, ಕೇವಲ ಸ್ವಾರ್ಥನೋ, ಅವನಿಗೆ ಇಲ್ಲಿ ಸುಖವಿಲ್ಲ, ಶಾಂತಿ\-ಯಿಲ್ಲ. ಇನ್ನು ಮುಂದೆ ಹೇಗೆ ಇದೆಲ್ಲಾ ಬರುವುದು? ಅಷ್ಟೈಶ್ವರ್ಯಗಳಿವೆ, ಆದರೆ ಮನಸ್ಸಿಗೆ ನೆಮ್ಮದಿ ಇಲ್ಲ. ಮಲಗುವುದಕ್ಕೆ ಹಂಸತೂಲಿಕಾತಲ್ಪವಿದೆ, ಆದರೆ ನಿದ್ದೆ ಬರಲೊಲ್ಲದು. ತಿನ್ನುವುದಕ್ಕೆ ಎಲ್ಲಾ ಇದೆ, ಆದರೆ ಯಾವುದನ್ನೂ ತೃಪ್ತಿಯಾಗುವಂತೆ ತಿನ್ನಲಾಗುವುದಿಲ್ಲ. ಒಂದು ತಿಂದರೆ ಹೊಟ್ಟೆಗೆ ಕೆಟ್ಟದ್ದು, ಮತ್ತೊಂದು ಹೃದಯ ರೋಗಕ್ಕೆ ಕೆಟ್ಟದ್ದು, ಮಗದೊಂದು ರಕ್ತದ ಒತ್ತಡಕ್ಕೆ ಕೆಟ್ಟದ್ದು. ಎಲ್ಲಾ ಖಾಯಿಲೆಗಳೂ ಮನೋರೋಗಗಳೂ ಸ್ವಾರ್ಥಿಯಲ್ಲಿ ಮನೆಮಾಡಿಕೊಂಡಿರುವುವು. ಇಲ್ಲೇ ಶಾಂತಿ ಇಲ್ಲ, ಇನ್ನು ಕಣ್ಣುಮುಚ್ಚಿಕೊಂಡು ಹೋದರೆ ಇಲ್ಲಿ ಇಲ್ಲದೇ ಇರುವುದು ಅಲ್ಲಿ ಹೇಗೆ ಬರುವುದು? ಇಲ್ಲಿ ಅನುಭವಿಸಬೇಕಾದರೆ, ಸತ್ತಮೇಲೆ ನಮ್ಮ ಹಿಂದೆ ಬರಬೇಕಾದರೆ ಕೊಟ್ಟಿರಬೇಕು. ಇದೇ ಮುಂದಿನ ಬೆಳೆಗೆ ಬೀಜ ಬಿತ್ತುವುದು. ಹೊಲವನ್ನು ಉಳದೆ, ಬೀಜವನ್ನು ಬಿತ್ತದೆ, ಬೆಳೆ ಕೊಯ್ಯಲು ಕುಡುಗೋಲನ್ನು ಹಿಡಿದುಕೊಂಡು ಹೋದರೆ ನಮಗೇನು ಸಿಕ್ಕುವುದು? ಅಲ್ಲಿ ತಾನೇ ತಾನಾಗಿ ಬೆಳೆದುಕೊಂಡಿರುವ ಮುಳ್ಳಿನ ಗಿಡಗಳು ಮತ್ತು ಕಳೆಗಳು ಮಾತ್ರ ಇರುತ್ತವೆ.

\begin{shloka}
ಏವಂ ಬಹುವಿಧಾ ಯಜ್ಞಾ ವಿತತಾ ಬ್ರಹ್ಮಣೋ ಮುಖೇ~।\\ಕರ್ಮಜಾನ್ ವಿದ್ಧಿ ತಾನ್ ಸರ್ವಾನೇವಂ ಜ್ಞಾತ್ವಾ ವಿಮೋಕ್ಷ್ಯಸೇ \hfill॥ ೩೨~॥
\end{shloka}

\begin{artha}
ಹೀಗೆ ಬಹುವಿಧವಾದ ಯಜ್ಞಗಳು ಬ್ರಹ್ಮದ ಮುಖದಲ್ಲಿ ಹೇಳಲ್ಪಟ್ಟಿವೆ. ಅವುಗಳೆಲ್ಲ ಕರ್ಮದಿಂದ ಹುಟ್ಟಿದವು ಎಂದು ತಿಳಿ. ಹೀಗೆಂದು ತಿಳಿದರೆ ಮುಕ್ತನಾಗುವೆ.
\end{artha}

ದ್ರವ್ಯಯಜ್ಞ, ತಪೋಯಜ್ಞ, ಯೋಗಯಜ್ಞ, ಜ್ಞಾನಯಜ್ಞ ಮುಂತಾದುವುಗಳೆಲ್ಲ ಬ್ರಹ್ಮದ ಮುಖ ಎಂದರೆ ವೇದದಲ್ಲಿ ಹೇಳಲ್ಪಟ್ಟಿವೆ. ಹಿಂದೂಗಳಿಗೆ ವೇದ ಧರ್ಮಶಾಸ್ತ್ರಗಳಾಗಿರಬಹುದು. ಅದರಂತೆಯೇ ಇತರ ಧರ್ಮದವರಿಗೆ, ಅವರವರ ಪವಿತ್ರ ಗ್ರಂಥಗಳನ್ನು ಅವರು ಭಗವಂತನ ಉಸಿರು ಎಂತಲೇ ಭಾವಿಸುತ್ತಾರೆ. ಯಾವ ಧರ್ಮವನ್ನಾದರೂ ತೆಗೆದುಕೊಳ್ಳಿ. ಎಲ್ಲಾ ನಿಂತಿರುವುದು ತ್ಯಾಗದ ನೀತಿಯ ಅಡಿಗಲ್ಲಿನ ಮೇಲೆ. ಯಾವುದೂ ಸ್ವಾರ್ಥವನ್ನು ಸಾಧಿಸುವುದಿಲ್ಲ. ಎಲ್ಲಾ ಧರ್ಮಗಳೂ ಸ್ವಾರ್ಥವನ್ನು ಪಾಪವೆನ್ನುವುವು.

ಬಗೆಬಗೆಯ ಯಜ್ಞಗಳೆಲ್ಲ ಕರ್ಮದಿಂದ ಆದುವು. ಆದರೆ ಎಲ್ಲಾ ಕರ್ಮವೂ ಒಂದೇ ರೀತಿ ಮಾಡಿದುದಲ್ಲ. ಕೆಲವನ್ನು ಕೈಯಿಂದ, ಕೆಲವನ್ನು ಬಾಯಿನಿಂದ, ಕೆಲವನ್ನು ಮನಸ್ಸಿನಿಂದ ಮಾಡಿರುವರು. ದಾನಧರ್ಮಗಳನ್ನು ಕೈಯಿಂದ ಮಾಡುತ್ತೇವೆ. ನಮ್ಮಲ್ಲಿರುವ ಜ್ಞಾನವನ್ನು ಇನ್ನೊಬ್ಬನಿಗೆ ಮಾತಿನ ಮೂಲಕ ಹೇಳುತ್ತೇವೆ. ಅಥವಾ ಬರವಣಿಗೆಯ ಮೂಲಕ ಬರೆದಿಡುತ್ತೇವೆ. ತಪಸ್ಸು ಮಾಡುವಾಗ ಮನಸ್ಸನ್ನೆಲ್ಲ ಕೇಂದ್ರೀಕರಿಸಿ ಪರಮಾತ್ಮನ ಕಡೆ ಹರಿಸುತ್ತೇವೆ. ಇದು ತುಂಬಾ ಸೂಕ್ಷ್ಮವಾದದ್ದು. ಹೊರಗಿನಿಂದ ಯಾರೂ ಇದನ್ನು ನೋಡಲಾರರು. ಆದರೂ ಇದು ಒಂದು ಮನಸ್ಸಿನ ಕರ್ಮವೇ. ಅಂತೂ ನಾವು ಮಾಡುವ ಕರ್ಮವೆಲ್ಲ ದೇಹ, ಇಂದ್ರಿಯ, ಬುದ್ಧಿ,\break ಮನಸ್ಸುಗಳಿಗೆ ಸಂಬಂಧಪಟ್ಟ ಕ್ರಿಯೆಯಾಯಿತು.

ಹೀಗೆಂದು ತಿಳಿದರೆ ಮುಕ್ತನಾಗುತ್ತೀಯೆ ನೀನು ಎನ್ನುವನು ಶ‍್ರೀಕೃಷ್ಣ. ಹೀಗೆ ತಿಳಿಯುವುದಕ್ಕೂ ಮುಕ್ತನಾಗುವುದಕ್ಕೂ ಏನು ಸಂಬಂಧ ಎಂದು ನಾವು ಆಶ್ಚರ್ಯಪಡಬಹುದು. ಇದರ ಹಿಂದೆ ಒಂದು ದೊಡ್ಡ ಸತ್ಯವಿದೆ. ಇದೆಲ್ಲ ಕರ್ಮವೇ ನಿಜ. ಆದರೆ ಒಂದು ರೀತಿ ಕರ್ಮ ಮಾಡಿದರೆ ನಾವು ಹೆಚ್ಚು ಗೋಜಿಗೆ ಸಿಕ್ಕಿಹಾಕಿಕೊಳ್ಳುತ್ತೇವೆ. ಮತ್ತೊಂದು ರೀತಿ ಕರ್ಮ ಮಾಡಿದರೆ ಗೋಜಿನಿಂದ ಪಾರಾಗುತ್ತೇವೆ. ಯಜ್ಞದೃಷ್ಟಿಯನ್ನು ಮನಸ್ಸಿನಲ್ಲಿ ಇಟ್ಟುಕೊಂಡು ಕರ್ಮ ಮಾಡಿದರೆ, ಕರ್ಮದ ಕೋಟಲೆಯಿಂದ ಹೊರಬರುತ್ತೇವೆ, ಮುಕ್ತರಾಗುತ್ತೇವೆ. ಯಾವಾಗ ಯಜ್ಞದೃಷ್ಟಿಯನ್ನು ಮರೆಯುವೆವೋ, ಕೇವಲ ಸ್ವಾರ್ಥಕ್ಕಾಗಿ ಕರ್ಮವನ್ನು ಮಾಡುವೆವೋ ಅದು ನಮ್ಮನ್ನು ಮತ್ತೂ ಪಾಶಗಳಿಂದ ಬಿಗಿಯುವುದು.

ಮೇಲಿನ ಸಿದ್ಧಾಂತವನ್ನು ಬೌದ್ಧಿಕವಾಗಿ ತಿಳಿದುಕೊಳ್ಳುವುದು ಮಾತ್ರವಲ್ಲ, ಅದನ್ನು ಅನುಷ್ಠಾನದಲ್ಲಿ ಬೆರೆಸಿರಬೇಕು ಎಂಬುದು ಅದರಲ್ಲಿ ಧ್ವನಿತವಾಗಿದೆ.

\begin{shloka}
ಶ್ರೇಯಾನ್ ದ್ರವ್ಯಮಯಾದ್ಯಜ್ಞಾಜ್ಜ್ಞಾನಯಜ್ಞಃ ಪರಂತಪ~।\\ಸರ್ವಂ ಕರ್ಮಾಖಿಲಂ ಪಾರ್ಥ ಜ್ಞಾನೇ ಪರಿಸಮಾಪ್ಯತೇ \hfill॥ ೩೩~॥
\end{shloka}

\begin{artha}
ಅರ್ಜುನ, ದ್ರವ್ಯಯಜ್ಞಕ್ಕಿಂತ ಜ್ಞಾನಯಜ್ಞ ಶ್ರೇಯಸ್ಕರ. ಎಲ್ಲ ಕರ್ಮಗಳೂ ಜ್ಞಾನದಲ್ಲಿ ಪರ್ಯವಸಾನ\-ವಾಗುವುವು.
\end{artha}

ದ್ರವ್ಯಯಜ್ಞ ಎಂದರೆ \enginline{Material} ಯಜ್ಞ ಎಂದು ನೋಡಬೇಕು. ಬರೀ ಹಣ ಮಾತ್ರವಲ್ಲ. ಸ್ಥೂಲವಾಗಿ ನಾವು ಇತರರಿಗೆ ಭಗವದರ್ಪಣ ಭಾವದಿಂದ ಕೊಡುವುದೆಲ್ಲ ಯಜ್ಞವೇ. ಹಣ ಕೊಡಬಹುದು, ವಸ್ತ್ರ ಕೊಡಬಹುದು, ಅನ್ನ ಹಾಕಬಹುದು, ಔಷಧಿ ಕೊಡಬಹುದು. ಆಸ್ಪತ್ರೆ, ಶಾಲೆ, ಛತ್ರ, ಸೇತುವೆ, ರಸ್ತೆಮುಂತಾದುವುಗಳನ್ನು ಕಟ್ಟಿಸಬಹುದು. ಇವುಗಳೆಲ್ಲ ಯಜ್ಞವೇ. ಆದರೆ ಇವುಗಳಿಂದ ಜನರಿಗೆ ಆಗುವ ಪ್ರಯೋಜನ ತಾತ್ಕಾಲಿಕ. ನಾವು ಕೊಡುವ ಹಣ ಎಷ್ಟುಕಾಲ ಬಡವನಲ್ಲಿ ಇರುವುದು? ಕೊಟ್ಟ ಬಟ್ಟೆಯನ್ನು ಎಷ್ಟುಕಾಲ ಧರಿಸುವನು? ಒಂದು ಸಲ ಔಷಧ ಕೊಟ್ಟು ವ್ಯಾಧಿಯಿಂದ ಒಬ್ಬನನ್ನು ಪಾರುಮಾಡಿದರೆ ಎಂದೆಂದಿಗೂ ರೋಗ ಬರುವುದಿಲ್ಲವೇ? ಆಸ್ಪತ್ರೆ, ಶಾಲೆ, ಛತ್ರ ಇವುಗಳನ್ನು ಕಟ್ಟಿಸಿದರೆ ಮನುಷ್ಯನನ್ನು ಅಜ್ಞಾನದಿಂದ ಪಾರುಮಾಡಿದೆವೆ? ಇಲ್ಲ. ಅದಕ್ಕಾಗಿಯೇ ಜ್ಞಾನಯಜ್ಞ ಎಲ್ಲಕ್ಕಿಂತಲೂ ಶ್ರೇಯಸ್ಕರ. ಇಲ್ಲಿ ಜ್ಞಾನವೆಂದರೆ ಕೇವಲ ಲೌಕಿಕ ವಿದ್ಯೆಯ ಜ್ಞಾನವನ್ನು ಮಾತ್ರ ಅಲ್ಲ ಹೇಳಿರುವುದು. ಆತ್ಮ ಅನಾತ್ಮ ವಸ್ತುಗಳ ವಿವೇಕ, ಅಜ್ಞಾನದಿಂದ ಪಾರಾಗುವುದು ಹೇಗೆ ಎಂದು ತಿಳಿದುಕೊಳ್ಳುವುದು, ಪರಮಾತ್ಮನ ಸಾನ್ನಿಧ್ಯವನ್ನು ಪಡೆಯುವುದು ಹೇಗೆ ಎಂದು ಅರಿಯುವುದು–ಇವುಗಳೆಲ್ಲ ಜ್ಞಾನದಲ್ಲಿ ಸೇರಿವೆ. ಈ ಜ್ಞಾನವೇ ಮನುಷ್ಯನನ್ನು ಜನನ ಮರಣಗಳ ದುಃಖದಿಂದ ಪಾರುಮಾಡುವುದು. ವಿದ್ಯೆಯಲ್ಲಿ ಆಧ್ಯಾತ್ಮಿಕ ವಿದ್ಯೆಯೇ ಶ್ರೇಷ್ಠ ಎಂಬುದನ್ನು ಶ‍್ರೀಕೃಷ್ಣ ಮುಂದಿನ ಅಧ್ಯಾಯಗಳಲ್ಲಿ ಹೇಳುತ್ತಾನೆ.

ನಾವು ಮಾಡುವ ಕರ್ಮವೆಲ್ಲ, ಅದು ಲೌಕಿಕವಾಗಿರಲಿ, ಪಾರಮಾರ್ಥಿಕವಾಗಿರಲಿ, ಕೊಡುವವನಿಗೆ ನಿಜವಾದ ಜ್ಞಾನವನ್ನು ತರುವುದು. ಯಾವ ದೃಷ್ಟಿಯಿಂದ ಕೊಡುತ್ತೇವೆಯೋ ಅದು ಮುಖ್ಯ. ಯಜ್ಞದ ದೃಷ್ಟಿಯಿಂದ ಮತ್ತೊಬ್ಬರಿಗೆ ನೀಡಿದ್ದರೆ ಎಲ್ಲರಿಗೂ ಶ್ರೇಷ್ಠವಾದ ಜ್ಞಾನವೇ ಕೊನೆಯಲ್ಲಿ ಬರುವುದು ಎನ್ನುತ್ತಾನೆ. ಒಬ್ಬ ಜ್ಞಾನವನ್ನು ಮತ್ತೊಬ್ಬನಿಗೆ ಕೊಡುವುದಕ್ಕೆ ಆಗುವುದಿಲ್ಲ. ಅವನಿಗೆ ಇನ್ನೊಬ್ಬನಿಗೆ ಒಂದು ತುತ್ತು ಅನ್ನ ಕೊಡುವುದು ಮಾತ್ರ ಸಾಧ್ಯ. ಅದನ್ನು ಇದು ಪರಮಾತ್ಮನಿಗೆ ಅರ್ಪಿತವಾಗಲಿ ಎಂದು ಮಾಡಿದರೆ ಶ್ರೇಷ್ಠವಾದ ಜ್ಞಾನವನ್ನು ಮತ್ತೊಬ್ಬರಿಗೆ ಕೊಟ್ಟರೆ ಯಾವ ಪರಿಣಾಮವಾಗುವುದೋ ಅದೇ ಆಗುವುದು. ಸ್ಪರ್ಶಶಿಲೆಯನ್ನು ತಾಕುವುದಕ್ಕೆ ಮುಂಚೆ, ಒಂದು ಹಿತ್ತಾಳೆ, ಮತ್ತೊಂದು ಕಬ್ಬಿಣ, ಇನ್ನೊಂದು ತಾಮ್ರ. ಅದಕ್ಕೆಲ್ಲ ಪೇಟೆಯ ಧಾರಣೆಯಲ್ಲಿ ವ್ಯತ್ಯಾಸವಿದೆ. ಆದರೆ ಸ್ಪರ್ಶಶಿಲೆಗೆ ತಾಕಿ ಎಲ್ಲವೂ ಚಿನ್ನವಾದಮೇಲೆ ಎಲ್ಲಕ್ಕೂ ಒಂದೇ ಬೆಲೆ. ಮುಂಚೆ ಕಬ್ಬಿಣವಾಗಿ ಈಗ ಚಿನ್ನವಾಗಿದ್ದಕ್ಕೆ ಕಡಿಮೆ ಬೆಲೆ, ಮುಂಚೆ ತಾಮ್ರವಾಗಿ ಈಗ ಚಿನ್ನವಾಗಿದ್ದಕ್ಕೆ ಹೆಚ್ಚು ಬೆಲೆ ಇಲ್ಲ. ಏಕೆಂದರೆ ಇಲ್ಲಿ ಹಿಂದಿನ ಅವಶೇಷ ಯಾವುದೂ ಇಲ್ಲ.

\begin{shloka}
ತದ್ವಿದ್ಧಿ ಪ್ರಣಿಪಾತೇನ ಪರಿಪ್ರಶ್ನೇನ ಸೇವಯಾ~।\\ಉಪದೇಕ್ಷ್ಯಂತಿ ತೇ ಜ್ಞಾನಂ ಜ್ಞಾನಿನಸ್ತತ್ತ್ವದರ್ಶಿನಃ \hfill॥ ೩೪~॥
\end{shloka}

\begin{artha}
ಅದನ್ನು ದಂಡನಮಸ್ಕಾರದಿಂದ, ಪರಿಪ್ರಶ್ನೆಯಿಂದ, ಗುರುಶುಶ್ರೂಷೆಯಿಂದ ತಿಳಿದುಕೊ. ತತ್ತ್ವವನ್ನು ಅರಿತ ಜ್ಞಾನಿಗಳು ನಿನಗೆ ಜ್ಞಾನವನ್ನು ಉಪದೇಶ ಮಾಡುವರು.
\end{artha}

ಇದನ್ನು ತಿಳಿದುಕೊಳ್ಳಬೇಕಾದರೆ ಯಾರು ಗುರುಗಳಾಗಿರುವರೋ ಅವರ ಬಳಿಗೆ ಹೋಗಬೇಕು. ಅವರಿಗೆ ದಂಡನಮಸ್ಕಾರವನ್ನು ಮಾಡಬೇಕು. ನಮ್ಮ ಅಂಗಾಂಗವೆಲ್ಲವೂ ಅವರಡಿ ಉರುಳಬೇಕು. ಇದು ದೈನ್ಯತೆಯನ್ನು ತೋರುವುದು. ಗುರುವಿನಿಂದ ಕಲಿಯಬೇಕಾದರೆ ಈ ದೈನ್ಯತೆ ಇಲ್ಲದೇ ಇದ್ದರೆ ಸಾಧ್ಯವಿಲ್ಲ. ನಾನು ಕಲಿತುಕೊಳ್ಳುವುದಕ್ಕೆ ಸಿದ್ಧನಾಗಿದ್ದೇನೆ, ನನಗೇನೂ ಗೊತ್ತಿಲ್ಲ, ನೀವು ಅದನ್ನು ನೀಡಬೇಕು ಎಂದು ಕೇಳಿಕೊಳ್ಳಬೇಕು. ಫೀಜುಕೊಟ್ಟು ಕಲಿಯುವ ವಿದ್ಯೆಯಲ್ಲ ಇದು. ಮಾರಾಟದ ಸರಕಲ್ಲ. ಗುರು ತಾನೇ ಮನಸ್ಸುಮಾಡಿ ಕೊಡಬೇಕು. ಆಗಮಾತ್ರ ಅದನ್ನು ನಾವು ಸ್ವೀಕರಿಸಬೇಕು. ಅವನು ಸಿಕ್ಕಿಸಿಕ್ಕಿದವರಿಗೆಲ್ಲಾ ಕೊಡುವುದಿಲ್ಲ. ತನ್ನ ಶಿಷ್ಯರಿಗೆ ಮಾತ್ರ ಅವನು ಹೇಳಿಕೊಡುತ್ತಾನೆ. ಇಲ್ಲಿ ಶಿಷ್ಯ ಎಂಬ ಪದ ತುಂಬಾ ಮುಖ್ಯವಾದುದು. ಗುರುವಿನ ಮಾನಸಿಕ ಪುತ್ರನೇ ಶಿಷ್ಯ. ಗುರು ತಾನೇ ಸಾಧನೆ ಮಾಡಿ ಏನನ್ನು ಗಳಿಸಿಕೊಂಡಿರುವನೋ ಅದಕ್ಕೆ ಹಕ್ಕುದಾರನಾಗುವನು ಶಿಷ್ಯ. ಗುರುವಿನ ಮಗನೇ ತಂದೆಯಿಂದ ಆಧ್ಯಾತ್ಮಿಕ ವಿಷಯಗಳನ್ನು ಕಲಿಯಬೇಕಾದರೆ ಶಿಷ್ಯನಾಗಬೇಕು. ಶಿಷ್ಯನಲ್ಲದವನಿಗೆ ಆಧ್ಯಾತ್ಮಿಕ ಗಹನ ರಹಸ್ಯಗಳನ್ನು ಕಲಿಸುವುದಿಲ್ಲ. ಮಗನಿಗಿಂತ ಹೆಚ್ಚು ಪ್ರೀತಿಗೆ ಪಾತ್ರನಾದವನು ಶಿಷ್ಯ. ಅದಕ್ಕೆ ಗೀತೆಯ ಪ್ರಾರಂಭದಲ್ಲಿ ಅರ್ಜುನ ನಾನು ನಿನ್ನ ಶಿಷ್ಯ, ನಿನ್ನಲ್ಲಿ ಶರಣಾಗಿದ್ದೇನೆ. ನನಗೆ ಶ್ರೇಯಸ್ಕರವಾಗಿರುವುದನ್ನು ಬೋಧಿಸು ಎಂದು ಕೇಳುತ್ತಾನೆ. ಅದಕ್ಕೆ ಮುಂಚೆ ಅರ್ಜುನನಿಗೆ ಶ‍್ರೀಕೃಷ್ಣ ಸ್ನೇಹಿತ ಮಾತ್ರನಾಗಿದ್ದ. ವಿಧೇಯತೆ, ದೈನ್ಯತೆ ಇವು ಶಿಷ್ಯನಲ್ಲಿರಬೇಕಾದ ಮೊದಲನೆ ಲಕ್ಷಣ.

ಇದನ್ನು ಪರಿಪ್ರಶ್ನೆಯಿಂದಲೂ ತಿಳಿದುಕೊಳ್ಳಬೇಕಾಗಿದೆ. ದೈನ್ಯತೆ ಶಿಷ್ಯನಲ್ಲಿರಬೇಕು ಎಂಬು\-ದನ್ನು ಹೇಳಿದಮೇಲೆ ಗುರು ಏನು ಹೇಳುತ್ತಾನೋ ಅದನ್ನೆಲ್ಲ ಸ್ವಲ್ಪವೂ ವಿಚಾರಮಾಡದೆ ಮರು\-ಮಾತಿಲ್ಲದೆ ಒಪ್ಪಿಕೊಳ್ಳಬೇಕು ಎನ್ನುವುದಿಲ್ಲ ಶ‍್ರೀಕೃಷ್ಣ. ಏನನ್ನು ಕೇಳಿದನೋ ಅದನ್ನು ಶಿಷ್ಯ ವಿಚಾರಮಾಡಬೇಕು. ಅದು ಹೇಗೆ ಎಂದು ತನಗೆ ತೋರಿದ ಸಂದೇಹವನ್ನು ಗುರುವಿನೆದುರಿಗೆ ಇಡಬೇಕು. ಗುರುವಿಗೆ ಶಿಷ್ಯ ಪ್ರಶ್ನೆ ಹಾಕಿದಾಗ ಮಾತ್ರವೆ, ಶಿಷ್ಯ ಅದನ್ನು ತಿಳಿದುಕೊಂಡಿರುವನೇ ಇಲ್ಲವೇ, ಅವನು ಯಾವ ಮೆಟ್ಟಿಲಲ್ಲಿರುವನು, ಅವನಿಗೆ ಹೇಗೆ ಹೇಳಿದರೆ ಚೆನ್ನಾಗಿ ಅರ್ಥವಾಗುವುದು ಇವುಗಳೆಲ್ಲ ಗೊತ್ತಾಗುವುದು. ವೈದ್ಯ ರೋಗಿಯನ್ನು ಪರೀಕ್ಷಿಸುವಾಗ ಕೆಲವು ಪ್ರಶ್ನೆಗಳನ್ನು ಹಾಕುತ್ತಾನೆ. ಅದರ ಆಧಾರದ ಮೇಲೆ ಇವನ ಖಾಯಿಲೆ ಏನಿರಬಹುದು ಎಂಬುದನ್ನು ನಿರ್ಧರಿಸುವನು. ಅದರಂತೆಯೇ ಪರಿಪ್ರಶ್ನೆಗಳು. ಗುರುವಿಗೆ ಪ್ರಶ್ನೆ ಹಾಕುವ ಶಿಷ್ಯನನ್ನು ಕಂಡರೆ ಕೋಪಬರುವುದಿಲ್ಲ. ಅವನಿಗೆ ಇನ್ನೂ ಪ್ರೀತಿ ಹೆಚ್ಚುವುದು. ತಾನು ಹೇಳುವುದನ್ನು ಚೆನ್ನಾಗಿ ವಿಚಾರಮಾಡಿ ಸ್ವೀಕರಿಸಿ ಎನ್ನುವನು ಗುರು. ಶ‍್ರೀ ರಾಮಕೃಷ್ಣರು ಶಿಷ್ಯರಿಗೆ ಬೋಧನೆ ಮಾಡುತ್ತಿದ್ದಾಗ ನಾನು ಹೇಳುವುದನ್ನು, ವರ್ತಕ ಸಾಮಾನಿಗೆ ಗಿರಾಕಿಗಳು ಕೊಡುವ ದುಡ್ಡನ್ನು ಚೆನ್ನಾಗಿ ಪರೀಕ್ಷೆ ಮಾಡಿ ತೆಗೆದುಕೊಳ್ಳುವಂತೆ, ಪರೀಕ್ಷೆ ಮಾಡಿ ಎನ್ನುತ್ತಿದ್ದರು.

ಒಂದು ವಿಷಯವನ್ನು ನಾವು ಚೆನ್ನಾಗಿ ತಿಳಿದುಕೊಳ್ಳಬೇಕಾದರೆ, ಅದನ್ನು ಪ್ರಶ್ನೆಯಿಂದ ಕೆದಕಬೇಕು. ಒಂದು ವಸ್ತುವನ್ನು ಒಲೆಯ ಮೇಲೆ ಇಟ್ಟು ಬೇಯಿಸುತ್ತಿದ್ದರೆ ಮೊಗಚುವ ಸೌಟಿನಿಂದ ತಿರುವಿ ತಿರುವಿ ಹಾಕುತ್ತಾರೆ. ಅದು ಎಲ್ಲಾಕಡೆಯೂ ಚೆನ್ನಾಗಿ ಬೇಯಲಿ ಎಂದು. ಅದರಂತೆಯೇ ಒಂದು ವಿಷಯವನ್ನು ನಾನಾ ಭಾಗಗಳಿಂದ ತೆಗೆದುಕೊಂಡು ಅದನ್ನು ನಮ್ಮದನ್ನಾಗಿ ಮಾಡಿಕೊಳ್ಳ ಬೇಕಾದರೆ ಅದಕ್ಕೆ ಸಂಬಂಧಪಟ್ಟ ಪ್ರಶ್ನೆಗಳನ್ನು ಹಾಕಬೇಕು. ಶ‍್ರೀಕೃಷ್ಣ ಇದು ಅತ್ಯಂತ ಆವಶ್ಯಕ ಎನ್ನುತ್ತಾನೆ.

ಸೇವೆ ಎಂಬ ಮತ್ತೊಂದು ಪದವನ್ನು ಶ‍್ರೀಕೃಷ್ಣ ಹೇಳುವನು. ಶಿಷ್ಯ ಗುರುವಿನ ಸೇವೆ ಮಾಡಬೇಕು. ಈ ಸೇವೆ ಎಂಬುದು ಒಂದು ಅಮೋಘವಾದ ಸಾಧನೆ. ಶಿಷ್ಯ ಗುರುವಿನ ಹೃದಯವನ್ನು ಒಲಿಸಿಕೊಳ್ಳಬೇಕು. ಅವನ ಹೃದಯವನ್ನು ಒಲಿಸಿಕೊಳ್ಳಬೇಕಾದರೆ, ಹಣದಿಂದಲ್ಲ, ಹೊಗಳಿಕೆಯಿಂದಲ್ಲ, ಅವನಿಗೆ ಮಾಡುವ ಸೇವೆಯಿಂದ. ಯಾವಾಗ ಸೇವೆಯಿಂದ ಸುಪ್ರೀತನಾದ ಗುರು ಶ್ರೇಷ್ಠವಾಗಿರುವ ಜ್ಞಾನ ಲಭಿಸಲಿ ಎಂದು ಇಚ್ಛಿಸಿದರೆ, ಶಿಷ್ಯನಿಗೆ ಅದು ಸುಲಭವಾಗಿ ಸಿದ್ಧಿಸುವುದು. ಸೇವೆಯೆಂಬುದೇ ಗುರುಭಕ್ತಿಗೆ ಬೇರೊಂದು ಪದ. ಗುರುವಿನ ಮೇಲೆ ಅದ್ಭುತವಾದ ಶ್ರದ್ಧೆಯನ್ನು, ಭಕ್ತಿಯನ್ನು ಇಡಬೇಕು. ಅವನಿಗಾಗಿ ಏನನ್ನು ಬೇಕಾದರೂ ಮಾಡಲು ಸಿದ್ಧನಾಗಿರಬೇಕು. ಯಾವ ವೇದವೇದಾಂತಗಳನ್ನೂ ಒಬ್ಬ ಓದಬೇಕಾಗಿಲ್ಲ. ಕೇವಲ ಗುರುವಿನ ಸೇವೆಯನ್ನು ಮಾಡಿಯೇ, ಅವನ ಮಾತಿನಂತೆ ನಡೆದುಕೊಂಡೇ, ಜ್ಞಾನಸಂಪಾದನೆ ಮಾಡಿಕೊಂಡ ಕೆಲವು ನಿದರ್ಶನಗಳು ನಮಗೆ ಉಪನಿಷತ್ತಿನಲ್ಲಿ ದೊರಕುತ್ತವೆ. ಸತ್ಯಕಾಮ ಎಂಬ ವಿದ್ಯಾರ್ಥಿ ಗುರುವಿನ ಬಳಿಗೆ ಹೋಗುತ್ತಾನೆ ಬ್ರಹ್ಮವಿದ್ಯೆಯನ್ನು ಕಲಿಯುವುದಕ್ಕೆ. ಆ ಗುರು ಬ್ರಹ್ಮವಿಷಯವನ್ನು ಹೇಳುವುದಿಲ್ಲ. ಕೆಲವು ದನಗಳನ್ನು ಕೊಟ್ಟು ಇದನ್ನು ಕಾಡಿಗೆ ತೆಗೆದುಕೊಂಡು ಹೋಗಿ ಮೇಯಿಸು, ಅದರ ಸಂಖ್ಯೆ ಎರಡರಷ್ಟಾದಮೇಲೆ ಬಾ ಎಂದು ಹೇಳುತ್ತಾನೆ. ಆ ದನಗಳು ಎರಡರಷ್ಟಾದಮೇಲೆ ಆ ದನಗಳೇ ಇವನಿಗೆ ನಾವು ಎರಡರಷ್ಟಾಗಿದ್ದೇವೆ, ಈಗ ನಮ್ಮನ್ನು ಗುರುವಿನ ಬಳಿಗೆ ಕರೆದುಕೊಂಡು ಹೋಗು ಎಂದು ಹೇಳಿದುವಂತೆ. ಅವನು ದಾರಿಯಲ್ಲಿ ಬರುವಾಗ ಬೆಂಕಿ ಜಿಂಕೆ ಹಕ್ಕಿ ಮುಂತಾದುವು ಬ್ರಹ್ಮನಿಗೆ ಸಂಬಂಧಪಟ್ಟ ವಿಷಯಗಳನ್ನು ಇವನಿಗೆ ಹೇಳಿದುವು. ಗುರುವಿನ ಮನೆಗೆ ಬಂದಾಗ ಸತ್ಯಕಾಮನ ತೇಜಸ್ಸಿನ ಮುಖವನ್ನು ನೋಡಿ, ಸತ್ಯಕಾಮ ನಿನಗೆ ಯಾರು ಬ್ರಹ್ಮನ ವಿಷಯವನ್ನು ಹೇಳಿದರು, ನೀನು ಬ್ರಹ್ಮವಿದನಂತೆ ಕಾಣುತ್ತಿರುವೆಯಲ್ಲ? ಎಂದು ಕೇಳಿದನು. ಸತ್ಯಕಾಮ, ಮನುಷ್ಯರಲ್ಲದ ಇತರ ವ್ಯಕ್ತಿಗಳು ಎನ್ನುತ್ತಾನೆ. ಅಂದರೆ ಗುರು ಪ್ರತ್ಯಕ್ಷವಾಗಿಯೆ ಹೇಳಬೇಕಾಗಿಲ್ಲ. ಅವನ ಸೇವೆಯನ್ನು ಮಾಡುವಾಗಲೆ ಅವುಗಳೆಲ್ಲ ನಮ್ಮಲ್ಲಿ ಸ್ಫುರಿಸುವುವು. ಜ್ಞಾನ ಯಾವುದೂ ಹೊರಗಿನಿಂದ ಬರುವುದಿಲ್ಲ. ಎಲ್ಲರಲ್ಲಿಯೂ ಅದು ಸುಪ್ತಾವಸ್ಥೆಯಲ್ಲಿದೆ. ಗುರುವಿಗೆ ಮನಸ್ಸಿಟ್ಟು ಮಾಡುವ ಸೇವೆಯಿಂದ ಆ ಸುಪ್ತಾವಸ್ಥೆಯಲ್ಲಿರುವುದು ಜಾಗ್ರತವಾಗುತ್ತಾ ಬರುವುದು. ಆದಕಾರಣವೇ ಹಿಂದೂ ಧರ್ಮದಲ್ಲಿ ಆಧ್ಯಾತ್ಮಿಕ ವಿಷಯಗಳನ್ನು ಗುರುವಿನಿಂದ ಕಲಿಯಬೇಕಾದರೆ, ಅವನಿಗೆ ಮಾಡುವ ಸೇವೆ ನಮ್ಮ ಸಾಧನೆಗಳಲ್ಲಿ ಅತ್ಯಂತ ಮುಖ್ಯ. ಕೆಲವು ಶಿಷ್ಯರು ಅಷ್ಟೊಂದು ಬುದ್ಧಿವಂತರಲ್ಲ, ಆದರೆ ಗುರುವಿಗೆ ಮಾಡುವ ಸೇವೆಯಲ್ಲಿ ಅಗ್ರಗಣ್ಯರು. ಇಂತಹ ಶಿಷ್ಯರಿಗೆ ಬಹಳ ಮೇಧಾವಿಯಾದವನಿಗೆ ಯಾವುದು ಅನೇಕವೇಳೆ ತಿಳಿದುಕೊಳ್ಳಲು ಕಷ್ಟವೋ ಅದನ್ನು ತಕ್ಷಣ ಗ್ರಹಿಸುವ ಶಕ್ತಿ ಬರುವುದು. ಇಂತಹ ಹಲವು ನಿದರ್ಶನಗಳನ್ನು ಶ‍್ರೀರಾಮಕೃಷ್ಣರ ಶಿಷ್ಯರಲ್ಲಿ ನೋಡುತ್ತೇವೆ. ಅವರ ಪ್ರಖ್ಯಾತ ಮೇಧಾವಿ ಶಿಷ್ಯ ನರೇಂದ್ರ ಒಂದು ಕಡೆ, ಕುಶಾಗ್ರಬುದ್ಧಿ ಯವನು, ಬೇಕಾದಷ್ಟು ಓದಿದವನು, ವಿಚಾರಿಸುವವನು. ಅವರ ಮತ್ತೊಬ್ಬ ಶಿಷ್ಯನೇ ಲಟು ಎಂಬ ಪರಿಚಾರಕ ಶಿಷ್ಯ. ಆ ಮನುಷ್ಯನಿಗೆ ಓದುವುದಕ್ಕೆ ಬಾರದು. ಆದರೂ ಕೇವಲ ಗುರುವಿಗೆ ಮಾಡಿದ ಸೇವೆಯ ಬಲದಿಂದ ಆಧ್ಯಾತ್ಮಿಕ ಅನುಭವದ ಗಿರಿಶೃಂಗವನ್ನೇರುತ್ತಾನೆ. ಇದನ್ನು ನೋಡಿದ ವಿವೇಕಾ ನಂದರೇ ಇದೊಂದು ಅದ್ಭುತ ಎಂದು ಅಚ್ಚರಿಪಟ್ಟು ಅವನಿಗೆ ಅದ್ಭುತಾನಂದ ಎಂಬ ಹೆಸರನ್ನು ಕೊಡುತ್ತಾರೆ.

ದೈನ್ಯತೆ, ಪರಿಪ್ರಶ್ನೆ, ಸೇವೆ ಇವುಗಳೆಲ್ಲ ಇದ್ದರೆ ತಿಳಿದುಕೊಂಡ ಗುರುಗಳು ಶಿಷ್ಯನಿಗೆ ಬೋಧನೆ ಮಾಡುತ್ತಾರೆ. ಆಧ್ಯಾತ್ಮಿಕ ಜೀವನದಲ್ಲಿ ಅನುಭಾವಿಗಳು ತಾವು ಅನುಭವಿಸುತ್ತಿರುವುದನ್ನು ಇತರರಿಗೆ ಹೇಳಲು ಸದಾ ಸಿದ್ಧರಾಗಿರುವರು. ಹಾಗೆ ಹೇಳುವುದರಲ್ಲಿ ಒಂದು ಆನಂದವಿದೆ. ನಮ್ಮಲ್ಲಿರುವ ಚೂರುಪಾರು ವಸ್ತುಗಳನ್ನು ಮತ್ತೊಬ್ಬನಿಗೆ ಕೊಡುವುದರಲ್ಲಿ ಒಂದು ಆನಂದವಿದೆ. ತಾಯಿಗೆ ಮಗುವಿಗೆ ತನ್ನ ಎದೆಹಾಲನ್ನು ಕೊಡುವುದರಲ್ಲಿ ಒಂದು ಸಂತೋಷವಿದೆ. ಇದಕ್ಕಿಂತ ಮಿಗಿಲಾದುದು ಆಧ್ಯಾತ್ಮಿಕ ಜೀವನದಲ್ಲಿ ಗುರು ಸಾಧನಾ ಬಲದಿಂದ ಸಂಗ್ರಹಿಸಿಕೊಂಡ ಅನುಭವಾಮೃತವನ್ನು ಶಿಷ್ಯನಿಗೆ ಧಾರೆಯೆರೆಯುವುದು. ಆದರೆ ಶಿಷ್ಯ ಯೋಗ್ಯನೇ ಅಲ್ಲವೇ ಎಂಬುದನ್ನು ಗುರು ಮೊದಲು ನೋಡುವನು. ಆಮೇಲೆ ಕೊಡುವನು. ಯೋಗ್ಯನಲ್ಲದೇ ಇದ್ದರೆ ಇವನು ಕೊಟ್ಟದ್ದು ಹಾಳಾಗುವುದು. ಹೊಲವನ್ನು ಚೆನ್ನಾಗಿ ಉತ್ತಿದ್ದರೆ, ಬೀಜ ಚೆಲ್ಲಿದರೆ ಪ್ರಯೋಜನ. ಬಂಡೆಯ ಮೇಲೆ, ಕಲ್ಲುನೆಲದಲ್ಲಿ, ಹದಮಾಡದ ಭೂಮಿಗೆ ಬೀಜ ಚೆಲ್ಲಿದರೆ ಪ್ರಯೋಜನವೇನು? ಶಿಷ್ಯ ಅಣಿಯಾದರೆ ಅವನಿಗೆ ಬೋಧನೆ ಮಾಡುವ ಗುರುವೂ ಸಿಕ್ಕಿಯೇ ಸಿಕ್ಕುತ್ತಾನೆ. ಇದೊಂದು ಆಧ್ಯಾತ್ಮಿಕ ಜೀವನದ ನಿಯಮ. ಒಂದು ಅಣಿಮಾಡಿದರೆ ಮತ್ತೊಂದು ಸಿಕ್ಕಿಯೇ ಸಿಕ್ಕುವುದು.

\begin{shloka}
ಯಜ್ಜ್ಞಾತ್ವಾ ನ ಪುನರ್ಮೋಹಮೇವಂ ಯಾಸ್ಯಸಿ ಪಾಂಡವ~।\\ಯೇನ ಭೂತಾನ್ಯಶೇಷೇಣ ದ್ರಕ್ಷ್ಯಸ್ಯಾತ್ಮನ್ಯಥೋ ಮಯಿ \hfill॥ ೩೫~॥
\end{shloka}

\begin{artha}
ಅರ್ಜುನ, ಈ ಜ್ಞಾನವನ್ನು ಪಡೆದುಕೊಂಡರೆ, ಪುನಃ ಮೋಹವನ್ನು ಪಡೆಯುವುದಿಲ್ಲ. ಅದರಿಂದ ಭೂತಗಳೆಲ್ಲವನ್ನು ಆತ್ಮನಲ್ಲಿಯೂ ಮತ್ತು ನನ್ನಲ್ಲಿಯೂ ನೋಡುವೆ.
\end{artha}

ಎಲ್ಲಾ ವಿಧವಾಗಿರುವ ಕರ್ಮಗಳು ಕೊನೆಗೆ ಜ್ಞಾನದಲ್ಲಿ ಪರಿಸಮಾಪ್ತಿಯಾಗುವುವು, ಅದನ್ನು ಯಜ್ಞದೃಷ್ಟಿಯಿಂದ ಮಾಡುವುದನ್ನು ಕಲಿತರೆ. ಒಂದು ಸಲ ಆ ಜ್ಞಾನ ನನಗೆ ಬಂದರೆ, ಪುನಃ ಆ ಜ್ಞಾನ ಸರಿಯುವುದಿಲ್ಲ. ಜ್ಞಾನ ಎಂದೆಂದಿಗೂ ನಮ್ಮದಾಗುವುದು. ಒಮ್ಮೆ ಮರಳುಕಾಡಿನಲ್ಲಿ ನಡೆಯುತ್ತಿರುವಾಗ ಮರೀಚಿಕೆಯನ್ನು ತಿಳಿದುಕೊಂಡರೆ, ಅನಂತರ ಅದನ್ನು ಎಷ್ಟುಸಾರಿ ನೋಡಿದರೂ ಅದು ಸುಳ್ಳು ಎಂಬ ಜ್ಞಾನ ಜೊತೆಯಲ್ಲಿಯೇ ಇರುವುದು. ಅದು ಎಂದಿಗೂ ನಮ್ಮನ್ನು ಬಿಟ್ಟುಹೋಗುವುದಿಲ್ಲ. ಒಮ್ಮೆ ಬೀಜವನ್ನು ಬಾಂಡಲೆಯಲ್ಲಿ ಚೆನ್ನಾಗಿ ಹುರಿದರೆ ಸಾಕು. ಅದನ್ನು ಎಷ್ಟುಸಲ ಹುಟ್ಟಿಹಾಕಿದರೂ ಅದು ಪುನಃ ಚಿಗುರುವುದಿಲ್ಲ. ಅದರಂತೆಯೇ ನಮ್ಮ ಅಜ್ಞಾನ ಜ್ಞಾನಾಗ್ನಿಯಲ್ಲಿ ಒಮ್ಮೆ ಚೆನ್ನಾಗಿ ಬೆಂದರೆ, ಅದು ಇನ್ನು ಬದುಕಿರಲಾರದು. ಈ ವಿಷಯವನ್ನು ಬೌದ್ಧಿಕವಾಗಿ ತಿಳಿದುಕೊಳ್ಳುವುದಲ್ಲ, ಈ ಜ್ಞಾನವನ್ನು ಅನುಭವಕ್ಕೆ ತರಬೇಕು. ಆಗ ಇದು ಸಿದ್ಧಿಸುವುದು. ಶ‍್ರೀರಾಮಕೃಷ್ಣರು ಸುಮ್ಮನೆ ಭಾಂಗ್, ಭಾಂಗ್ ಅನ್ನುತ್ತಿದ್ದರೆ ಮತ್ತು ಬರುವುದಿಲ್ಲ. ಅದನ್ನು ತಂದು ಅರೆದು ಕುಡಿಯಬೇಕು. ಆಗ ಮಾತ್ರ ಮತ್ತು ಬರಬೇಕಾದರೆ ಎಂದು ಹೇಳುತ್ತಿದ್ದರು. ಅದರಂತೆಯೇ ಈ ಜ್ಞಾನವನ್ನು ಅರೆದು ಕುಡಿದಾಗಲೆ, ನಮ್ಮ ಅನುಷ್ಠಾನದಲ್ಲಿ ಪ್ರವಹಿಸಿದಾಗಲೆ, ಇನ್ನುಮೇಲೆ ನಾವು ಮೋಹಗೊಳ್ಳುವುದಿಲ್ಲ.

ಈ ಜ್ಞಾನದಿಂದ ಎಲ್ಲಾ ಭೂತಗಳನ್ನೂ ಆತ್ಮನಲ್ಲಿ ನೋಡುವೆ ಎನ್ನುತ್ತಾನೆ. ಆತ್ಮನಲ್ಲಿ ಎಂದರೆ ನನ್ನಲ್ಲಿ ಎಂದು ಬೇಕಾದರೆ ತಿಳಿದುಕೊಳ್ಳಬಹುದು. ಅಜ್ಞಾನ ಯಾವಾಗ ಸರಿಯುವುದೋ, ದೇಶ ಕಾಲ ನಿಮಿತ್ತದ ಮೂಲಕ ನೋಡುವಾಗ ಯಾವುದನ್ನು ನಾವು ಆರೋಪ ಮಾಡಿರುವೆವೋ ಅದನ್ನು ತೆಗೆದಾಗ ಪ್ರಪಂಚವೆಲ್ಲ ಒಂದು ಅಖಂಡ ವ್ಯಕ್ತಿತ್ವವಾಗುವುದು. ನಾನೇ ಅದರಂತೆ ಕಾಣುವುದು. ನನ್ನಲ್ಲಿ ನಾನು ನೋಡುವುದು, ಅನುಭವಿಸುವುದು, ಎಲ್ಲ ಇರುವಂತೆ ಕಾಣುವುದು. ನಾನು ಒಂದು ಕನಸು ಕಾಣುತ್ತಿರುವೆನು. ಆ ಕನಸಿನಲ್ಲಿ ನನ್ನಿಂದ ಹೊರಗೆ ಹಲವು ವಸ್ತುಗಳನ್ನು ನೋಡುತ್ತಿರುತ್ತೇನೆ. ಆದರೆ ಆ ಕನಸೆಲ್ಲ ಯಾರಲ್ಲಿ ಆಗುತ್ತಿದೆ? ನನ್ನಲ್ಲಿ. ಎದ್ದಾದ ಮೇಲೆ ನಾನು ಏನೇನು ಹೊರಗಡೆ ಕನಸಿನಲ್ಲಿ ನೋಡುತ್ತಿದ್ದೆನೋ ಅದೆಲ್ಲ ನನ್ನಲ್ಲಿಯೇ ಇದ್ದಿತು ಎಂಬುದು ಅರಿವಾಗುವುದು. ಕನಸಿನಲ್ಲಿರುವಾಗ ನನ್ನಿಂದ ಹೊರಗಡೆ ಅದು ಇರುವಂತೆ ಕಾಣುತ್ತಿತ್ತು. ಇದು ಅಜ್ಞಾನ. ಕನಸಿನಿಂದ ಎದ್ದಾದಮೇಲೆ ಆ ಕನಸೆಲ್ಲಾ ನನ್ನಲ್ಲಿಯೇ ಇದ್ದಿತು ಎಂಬುದು ಅರಿವಾಗುವುದು. ಇದು ಜ್ಞಾನ. ಕನ್ನಡಿಯೊಂದು ರೂಮಿನಲ್ಲಿದೆ. ಅದು ಹೊರಗಡೆ ಇರುವುದನ್ನೆಲ್ಲಾ ತನ್ನಲ್ಲಿ ತೋರಿಸುತ್ತಿದೆ. ಪ್ರಪಂಚಕ್ಕೆ ನಾನು ಕನ್ನಡಿ. ನನ್ನಲ್ಲಿ ಆ ಪ್ರಪಂಚವೆಲ್ಲ ಇರುವಂತೆ ತೋರುವುದು.

ಮತ್ತು ನನ್ನಲ್ಲಿಯೂ ನೋಡುತ್ತೀಯೆ ಎನ್ನುವನು ಶ‍್ರೀಕೃಷ್ಣ. ಹೇಗೆ ನನ್ನಲ್ಲಿ ನನ್ನಿಂದ ಹೊರಗಡೆ ಇರುವುದನ್ನೆಲ್ಲಾ ನಾನು ಎಂಬ ಕನ್ನಡಿ ತೋರುವುದೋ ಹಾಗೆ ಪರಮಾತ್ಮನಲ್ಲಿ, ಎಲ್ಲವನ್ನೂ ನೋಡುತ್ತೇವೆ. ನಾನು, ಜೊತೆಗೆ ಈಗ ನನ್ನಲ್ಲಿರುವುದು ಎಲ್ಲವನ್ನೂ ಭಗವಂತನಲ್ಲಿ ನೋಡುತ್ತೇವೆ. ಭಗವಂತ ಇವುಗಳನ್ನೆಲ್ಲ ಒಳಗೊಂಡಿರುವನು. ಅವನಿಲ್ಲದ ಸ್ಥಳವೇ ಇಲ್ಲ. ಅವನಿಲ್ಲದ ಹೃದಯವೇ ಇಲ್ಲ ನಾವು ನೋಡುವ ವೈವಿಧ್ಯತೆಯಿಂದ ತುಂಬಿ ತುಳುಕಾಡುವ ವಸ್ತುಗಳೆಲ್ಲ ಭಗವಂತನೆಂಬ ತೆರೆಯ ಮೇಲೆ ಬರೆದ ಚಿತ್ರದಂತಿದೆ. ಎಲ್ಲಾ ಚಿತ್ರದ ಹಿಂದೆಯೂ ಆ ತೆರೆಯೇ ಇದೆ. ಎಲ್ಲಾ ನಾಮದ ಹಿಂದೆ, ರೂಪದ ಹಿಂದೆ, ವ್ಟಕ್ತಿಯ ಹಿಂದೆ ಇರುವುದು ಭಗವಂತನೊಬ್ಬನೆ. ಯಾರ ಮುಖವನ್ನು ನೋಡಿದರೂ ಅವನ ನೆನಪೇ ಬರುವುದು. ಯಾವ ದೃಶ್ಯವನ್ನೇ ನೋಡಿದರೂ ಅವನ ನೆನಪೇ ಬರುವುದು. ಯಾವ ಘಟನೆಯನ್ನು ತೆಗೆದುಕೊಂಡರೂ ಅವನ ಕೈವಾಡವನ್ನೇ ನೋಡುವನು. ಇನನ್ನು ಅವನು ಯಾರನ್ನೂ ಧಿಃಕರಿಸುವುದಿಲ್ಲ, ಯಾರನ್ನೂ ನಿರಾಕರಿಸುವುದಿಲ್ಲ. ಆ ಪವಿತ್ರಾತ್ಮನ ಕಣ್ಣುಗಳು ಇನ್ನುಮೇಲೆ ಯಾವ ಹೊಲಸನ್ನೂ ನೋಡಲಾರವು. ಆ ಹೊಸನಿನಲ್ಲಿಯೂ ಪರಮಾತ್ಮನೇ ಅವನಿಗೆ ಕಾಣುವುದು.

\begin{shloka}
ಅಪಿ ಚೇದಸಿ ಪಾಪೇಭ್ಯಃ ಸರ್ವೇಭ್ಯಃ ಪಾಪಕೃತ್ತಮಃ~।\\ಸರ್ವಂ ಜ್ಞಾನಪ್ಲವೇನೈವ ವೃಜಿನಂ ಸಂತರಿಷ್ಯಸಿ \hfill॥ ೩೬~॥
\end{shloka}

\begin{artha}
ನೀನು ಎಲ್ಲಾ ಪಾಪಗಳನ್ನು ಮೀರಿದ ಪಾಪವನ್ನು ಮಾಡಿದ್ದರೂ ಅವೆಲ್ಲವನ್ನೂ ಜ್ಞಾನವೆಂಬ ದೋಣಿ\-ಯಿಂದಲೇ ದಾಟುವೆ.
\end{artha}

ನಾವು ಅಜ್ಞಾನದಲ್ಲಿದ್ದಾಗ ಹಲವಾರು ಪಾಪಕೃತ್ಯಗಳನ್ನು ಮಾಡಿರಬಹುದು. ಆಗ ಅದರಿಂದ ನಮಗೆ ಅಷ್ಟು ವ್ಯಥೆ ಆಗುವುದಿಲ್ಲ. ಆದರೆ ಅದರಿಂದ ಮೇಲೇಳಲು ಪ್ರಯತ್ನಪಟ್ಟಾಗಲೇ ಮಾಡಿದ ಪಾಪಕೃತ್ಯಗಳ ಭಾರ ನಮಗೆ ಅರಿವಾಗುವುದು. ಒಂದು ಕೊಡ ನೀರಿನಲ್ಲಿಯೇ ಮುಳುಗಿರುವಾಗ ಅದನ್ನು ಎಳೆದರೆ ಭಾರ ಗೊತ್ತಾಗುವುದಿಲ್ಲ. ಆದರೆ ನೀರಿನಿಂದ ಮೇಲಕ್ಕೆ ಬಂದಾಗಲೆ ಅದರ ಭಾರ ಗೊತ್ತಾಗುವುದು. ಅದರಂತೆಯೇ ನಾವು ಪಾಪದಲ್ಲಿ ಮುಳುಗಿರುವಾಗ ಮಾಡಿದ ಪಾಪಕೃತ್ಯಗಳು ಎಷ್ಟು ಭಯಂಕರವಾದುವುಗಳು ಎಂಬುದು ಅರ್ಥವಾಗುವುದಿಲ್ಲ. ಆದರೆ ಅದರಿಂದ ಸ್ವಲ್ಪ ಮೇಲಕ್ಕೆ ಎದ್ದಾಗ ಅದರ ಭಾರ ಗೊತ್ತಾಗುವುದು. ಅಯ್ಯೋ, ಇಷ್ಟೊಂದು ಪಾಪಕೃತ್ಯವನ್ನು ಮಾಡಿದ್ದೇನೆ, ನಾನು ನಿಜವಾಗಿಯೂ ದೇವರ ಕಡೆ ಹೋಗುವುದಕ್ಕೆ ಯೋಗ್ಯನೇ ಎಂದು ನನ್ನನ್ನು ನಾನೇ ಹಳಿದುಕೊಳ್ಳುತ್ತೇನೆ, ನನ್ನನ್ನು ನಾನೇ ಶಪಿಸಿಕೊಳ್ಳುತ್ತೇನೆ. ಇನ್ನಾರೂ ಹೊರಗಡೆಯವರು ನಮ್ಮನ್ನು ಛೀಮಾರಿ ಮಾಡಬೇಕಾಗಿಲ್ಲ. ಅನೇಕ ವೇಳೆ ಮಾಡಿದ ಹೀನಕೃತ್ಯಗಳ ನೆನಪೆಲ್ಲ\break ಬಂದಾಗ ಅದನ್ನು ಮೀರಿ ಹೋಗುವುದನ್ನು ಬಿಟ್ಟು ಪುನಃ ಕೆಸರಿನಲ್ಲಿಯೇ ಮುಳುಗುತ್ತೇವೆ. ಅನೇಕ ಆಧ್ಯಾತ್ಮಿಕ ಯಾತ್ರಿಕರಲ್ಲಿ ನಾವು ಇದನ್ನು ನೋಡುತ್ತೇವೆ. ಆಗಲೇ ನಮಗೆ ಈ ಶ್ಲೋಕ ಒಂದು ಭರವಸೆಯ ವಾಣಿಯಂತೆ ಬರುತ್ತದೆ. ಶ‍್ರೀಕೃಷ್ಣ ಎಲ್ಲಾ ಭವಜೀವಿಗಳಿಗೂ, ಅವರು ಎಂತಹ ಪಾತಕ ಮಾಡಿದ್ದರೂ ಅದರಿಂದ ಪಾರಾಗುವುದಕ್ಕೆ ಸಾಧ್ಯ ಎಂದು ಸಾರುತ್ತಾನೆ. ಯಾರು ಈ ಕೃತ್ಯಗಳನ್ನು ಮಾಡಿಲ್ಲ? ಎಲ್ಲರೂ ಅಜ್ಞಾನದಲ್ಲಿದ್ದಾಗ ಒಂದಲ್ಲ ಮತ್ತೊಂದು ವಿವಿಧ ಅಜ್ಞಾನ ಕಾರ್ಯವನ್ನು ಮಾಡಿಯೇ ಇರುತ್ತಾರೆ. ಕೆಲವರು ಒಪ್ಪಿಕೊಳ್ಳುತ್ತಾರೆ, ಮತ್ತೆ ಕೆಲವರು ಒಪ್ಪಿಕೊಳ್ಳುವುದಿಲ್ಲ. ಕೆಲವರು ಹಲವು ಕೃತ್ಯಗಳನ್ನು ಮಾಡಿರುತ್ತಾರೆ, ಕೆಲವರು ಬಹಳ ಅಲ್ಪ ಮಾಡಿರುತ್ತಾರೆ. ಅಂತೂ ಎಲ್ಲಾ ಮಾಡಿದವರ ಗುಂಪಿಗೇ ಸೇರಿದವರು. ಅವರು ಏತಕ್ಕೆ ಮಾಡಿದರು? ಅಜ್ಞಾನದಲ್ಲಿದ್ದಾಗ, ಅದೇ ಸತ್ಯವೆಂದು ಆಗ ಅವರಿಗೆ ಅನ್ನಿಸಿದ್ದರಿಂದ ಅದನ್ನು ಮಾಡಿದರು. ಇದಕ್ಕೆಲ್ಲ ಅಜ್ಞಾನ ಕಾರಣ. ಆದರೆ ಈಗ ನನಗೆ ಅದು ಗೊತ್ತಾಗಿದೆ, ಹಾಗೆ ಮಾಡುವುದು ತಪ್ಪು ಎಂದು. ಇನ್ನು ಮೇಲೆ ಮಾಡದೆ ಇರುವುದಕ್ಕೆ ಪ್ರಯತ್ನಿಸುತ್ತೇನೆ ಎಂದು ಸಂಕಲ್ಪ ಮಾಡಿಕೊಳ್ಳಬೇಕು. ಅನೇಕ ವೇಳೆ ಈಗ ಏನು ಮಾಡಬೇಕೋ ಅದನ್ನು ಮರೆತು ಆಗಲೇ ಆಗಿಹೋದುದನ್ನು ಕುರಿತು ಚಿಂತಿಸುವುದರಲ್ಲಿ ಕಾಲಕಳೆಯುವೆವು. ಇದರಿಂದ ಎಳ್ಳಷ್ಟೂ ಪ್ರಯೋಜನವಿಲ್ಲ. ಇದೊಂದು ತಮೋಗುಣದ ಚಿಹ್ನೆ. ಆಗಿದ್ದು ಆಗಿಹೋಯಿತು ಎಂದು ಮುಂದೆ ಒಳ್ಳೆಯವರಾಗುವುದಕ್ಕೆ ಸೊಂಟ ಕಟ್ಟಬೇಕು.

ಪಾಪಸಾಗರ ಅಗಾಧವಾಗಿರಬಹುದು. ಅದರಲ್ಲಿ ಬೇಕಾದಷ್ಟು ಮೊಸಳೆ ತಿಮಿಂಗಿಲಗಳಿರಬಹುದು. ದೊಡ್ಡದೊಡ್ಡ ಅಲೆಗಳೇಳುತ್ತಿರಬಹುದು. ಆದರೆ ಜ್ಞಾನವೆಂಬ ನೌಕೆ ಸಣ್ಣದಾಗಿರಬಹುದು. ಅದರ ಸಹಾಯದಿಂದ ನಾವು ಸಾಗರವನ್ನು ದಾಟುತ್ತೇವೆ. ಭಗವಂತನನ್ನು ನೆಚ್ಚಿ,\break ಅವನನ್ನು ನಮ್ಮ ಧ್ರುವತಾರೆಯನ್ನಾಗಿ ಮಾಡಿಕೊಂಡು ಹೊರಟರೆ ಗುರಿ ಸೇರುವುದರಲ್ಲಿ ಸಂದೇಹವಿಲ್ಲ. ನೀನು ಹಿಂದೆ ಏನೇನು ಮಾಡಿದ್ದಿ ಎಂದು ಕೇಳುವವನಲ್ಲ ದೇವರು. ಈಗ ನೀನು ಏನಾಗಬೇಕೆಂದು ಇರುವೆ, ಈಗಿನದನ್ನು ಸಾಧಿಸುವುದಕ್ಕೆ ಹಟ ತೊಟ್ಟಿರುವೆಯಾ–ಇದನ್ನು ಮಾತ್ರ ದೇವರು ಕೇಳುವುದು.

\vskip 2pt

\begin{shloka}
ಯಥೈಧಾಂಸಿ ಸಮಿದ್ಧೋಽಗ್ನಿರ್ಭಸ್ಮಸಾತ್ ಕುರುತೇಽರ್ಜುನ~।\\ಜ್ಞಾನಾಗ್ನಿಃ ಸರ್ವಕರ್ಮಾಣಿ ಭಸ್ಮಸಾತ್ ಕುರುತೇ ತಥಾ \hfill॥ ೩೭~॥
\end{shloka}

\vskip 2pt

\begin{artha}
ಅರ್ಜುನ, ಚೆನ್ನಾಗಿ ಹತ್ತಿಕೊಂಡು ಉರಿಯುತ್ತಿರುವ ಬೆಂಕಿ ಕಟ್ಟಿಗೆಗಳನ್ನು ಹೇಗೆ ಸುಟ್ಟು ಬೂದಿ ಮಾಡುವುದೋ ಹಾಗೆಯೇ ಜ್ಞಾನಾಗ್ನಿ ಕರ್ಮಗಳನ್ನೆಲ್ಲಾ ಭಸ್ಮಮಾಡಿಬಿಡುವುದು.
\end{artha}

\newpage

ಉರಿಯುತ್ತಿರುವ ಬೆಂಕಿಯೊಳಗೆ ನಾವು ಕಟ್ಟಿಗೆಯನ್ನು ಹಾಕಿದರೆ ಕೆಲವು ಕಾಲದ ಮೇಲೆ ಅದೂ ಕೂಡ ಉರಿದು ಹೋಗುವುದು. ಅದನ್ನು ನಾವು ಅನಂತರ ನೆಟ್ಟು ಒಂದು ಗಿಡ ಮಾಡುವುದಕ್ಕೆ ಆಗುವುದಿಲ್ಲ. ಅಗ್ನಿ ಕಟ್ಟಿಗೆಯಲ್ಲಿರುವ ಹಸಿಯನ್ನೆಲ್ಲ ಹೀರಿ, ಅದನ್ನು ಒಣಗಿಸಿ ಉರಿಸಿಬಿಡುವುದು. ಅದರಂತೆಯೇ ಯಾವಾಗ ಜ್ಞಾನಾಗ್ನಿಯಿಂದ ಒಬ್ಬ ಕೂಡಿಕೊಂಡು ಕೆಲಸ ಮಾಡುವನೋ ಆಗ ಅವನು ಮಾಡಿದ ಕೆಲಸ ಹೊಸ ಸಂಸ್ಕಾರಗಳಾವುದನ್ನೂ ಬಿಡದೆ ದಹಿಸುವುದು. ಇನ್ನು ಮೇಲೆ ಅವನು ಯಾವ ಕರ್ಮವನ್ನು ಮಾಡಿದರೂ ಬದ್ಧನಾಗುವುದಿಲ್ಲ. ಹಾಗಾದರೆ ಅವನು ಎಂತಹ ಅಶ್ಲೀಲವಾದ ಕೆಲಸ ವನ್ನಾದರೂ ಮಾಡುವುದಕ್ಕೆ ಇನ್ನು ಮೇಲೆ ಯಾವ ಅಭ್ಯಂತರವೂ ಇಲ್ಲ ಎನ್ನಬಹುದು. ಅವನಲ್ಲಿ ಹೀನ ಸಂಸ್ಕಾರಗಳು ಇದ್ದರೆ ತಾನೆ ಅಶ್ಲೀಲವಾದ ಕೆಲಸವನ್ನು ಮಾಡುವನು? ಹೀನ ಸಂಸ್ಕಾರಗಳು ಇದ್ದರೆ ಅವನಿಗೆ ಜ್ಞಾನವಾದರೂ ಹೇಗೆ ಬರುವುದು? ಒಮ್ಮೆ ಜ್ಞಾನಾಗ್ನಿಯಿಂದ ಹೀನ ಸಂಸ್ಕಾರಗಳು ನಾಶವಾದ ಮೇಲೆ ಮತ್ತೊಮ್ಮೆ ಅವು ತಲೆಯೆತ್ತಲಾರವು.

\begin{shloka}
ನ ಹಿ ಜ್ಞಾನೇನ ಸದೃಶಂ ಪವಿತ್ರಮಿಹ ವಿದ್ಯತೇ~।\\ತತ್ ಸ್ವಯಂ ಯೋಗಸಂಸಿದ್ಧಃ ಕಾಲೇನಾತ್ಮನಿ ವಿಂದತಿ \hfill॥ ೩೮~॥
\end{shloka}

\begin{artha}
ಈ ಪ್ರಪಂಚದಲ್ಲಿ ಜ್ಞಾನಕ್ಕೆ ಸಮನಾದ ವಸ್ತು ಬೇರೊಂದು ಇಲ್ಲ. ಕಾಲಕ್ರಮೇಣ ಯೋಗದಲ್ಲಿ ಸಂಸಿದ್ಧನಾದವನು ಜ್ಞಾನವನ್ನು ತನ್ನಲ್ಲಿಯೇ ಪಡೆಯುತ್ತಾನೆ.
\end{artha}

ನಾವು ದೇಹವನ್ನು ಎಷ್ಟು ಸಲ ಜಲದಿಂದ ತೊಳೆದರೂ ಪುನಃ ಬೆವರು ಕೊಳೆ ಕಾದುಕೊಂಡಿರುತ್ತದೆ, ಇದನ್ನು ಮುತ್ತುವುದಕ್ಕೆ. ಪಾತ್ರೆಯನ್ನು ಎಷ್ಟು ಸಲ ಬೆಳಗಿದರೂ ಮಾರನೆ ದಿನ ಕಿಲುಬು ಕಟ್ಟುವುದು. ಶ‍್ರೀರಾಮಕೃಷ್ಣರಿಗೆ ಅದ್ವೈತವನ್ನು ಬೋಧಿಸಿದ ಗುರು ತೋತಾಪುರಿ ಪ್ರತಿದಿನ ಧೂನಿ ಹಚ್ಚಿ ಹತ್ತಿರ ಧ್ಯಾನಕ್ಕೆ ಕುಳಿತುಕೊಳ್ಳುತ್ತಿದ್ದ. ಒಂದು ಸಲ ಶ‍್ರೀರಾಮಕೃಷ್ಣರು ಅವನನ್ನು “ನೀನೇಕೆ ಪ್ರತಿದಿನ ಧ್ಯಾನ ಮಾಡಬೇಕು?” ಎಂದು ಕೇಳುತ್ತಾರೆ. ಆತ ತನ್ನ ಹತ್ತಿರ ಇದ್ದ ತಾಮ್ರದ ಪಾತ್ರೆಯನ್ನು ತೋರಿ, “ನೋಡು, ಇದನ್ನು ಪ್ರತಿದಿನವೂ ಬೆಳಗದೆ ಇದ್ದರೆ ಕಿಲುಬು ಕಟ್ಟುವುದು ಅಲ್ಲವೆ? ಹಾಗೆಯೇ ಮನಸ್ಸು. ಅದನ್ನು ಪ್ರತಿದಿನ ಧ್ಯಾನದಿಂದ ಬೆಳಗಬೇಕು” ಎಂದನು. ಶ‍್ರೀರಾಮಕೃಷ್ಣರು ಆ ಪಾತ್ರೆ ಚಿನ್ನದ್ದಾಗಿದ್ದರೆ ಏತಕ್ಕೆ ಬೆಳಗಬೇಕು? ಎನ್ನುತ್ತಾರೆ. ತೋತಾಪುರಿ ಆಗ ನಿರುತ್ತರನಾದ. ಜ್ಞಾನ ಪಡೆಯುವುದು ಎಂದರೆ ಚಿನ್ನವಾಗುವುದು. ಒಮ್ಮೆ ಅದು ಸ್ಪರ್ಶಶಿಲೆಗೆ ತಾಕಿ ಚಿನ್ನವಾಯಿತು ಎಂದರೆ ಅದನ್ನು ಯಾರೂ ತೊಳೆಯಬೇಕಾಗಿಲ್ಲ, ಬೆಳಗಬೇಕಾಗಿಲ್ಲ. ಜ್ಞಾನವೆಂಬುದು ಇಂತಹುದು.

ಜ್ಞಾನ ಇದ್ದಕ್ಕೆ ಇದ್ದಂತೆ ಬರುವುದಿಲ್ಲ. ಕ್ರಮೇಣ ನಿಧಾನವಾಗಿ ಸ್ವಲ್ಪ ಸ್ವಲ್ಪವಾಗಿ ನಮ್ಮ ದೃಷ್ಟಿಕೋಣವನ್ನು ಬದಲಾಯಿಸುವುದು. ಅಜ್ಞಾನದಿಂದ ಸುಜ್ಞಾನಕ್ಕೆ ಕಣ್ಣುಮಿಟುಕಿಸುವುದರಲ್ಲಿ ಪ್ರಯಾಣ ಮಾಡುವುದಕ್ಕೆ ಆಗುವುದಿಲ್ಲ. ಪೂರ್ಣತೆಗೆ ಪ್ರಯಾಣ ಯಾವಾಗಲೂ ನಿಧಾನ. ಅಲ್ಲಿ ಪ್ರತಿಯೊಂದು ಅಂಗುಲವನ್ನು ತೆವಳಿಕೊಂಡು ಹತ್ತಿಕೊಂಡು ಹೋಗಬೇಕಾಗಿದೆ. ಮುಂಚೆ ಯೋಗದಲ್ಲಿ ಎಂದರೆ ಜ್ಞಾನವೋ, ಭಕ್ತಿಯೋ, ಕರ್ಮವೋ, ಧ್ಯಾನವೋ ಯಾವುದಾದರೂ ಒಂದರಲ್ಲಿಯೋ ಅಥವಾ ಅವುಗಳನ್ನೆಲ್ಲಾ ಸೇರಿಸಿಯೋ ಆಧ್ಯಾತ್ಮಿಕ ಜೀವನದಲ್ಲಿ ಮುಂದುವರಿಯಬೇಕು.

ಅನಂತರ ಆ ಜ್ಞಾನವನ್ನು ತನ್ನಲ್ಲಿಯೇ ಪಡೆದುಕೊಳ್ಳುತ್ತಾನೆ. ಇದೆಲ್ಲೊ ಹೊರಗಿನಿಂದ ನನಗೆ ಬರುವ ವಸ್ತುವಲ್ಲ. ಅದು ಎಲ್ಲಾ ಜೀವಿಗಳ ಅಂತರಂಗದಲ್ಲಿದೆ. ಅದನ್ನು ಕಾಣದಂತೆ ಮಾಡಿರುವುದು ಅಜ್ಞಾನದ ತೆರೆಗಳು. ನಾವು ಶಾಸ್ತ್ರಾದಿಗಳನ್ನು ಓದುವುದರಿಂದ, ಗುರು ಮುಖೇನ ಕೇಳುವುದರಿಂದ, ಫಲಾಪೇಕ್ಷೆ ಇಲ್ಲದೆ ಕರ್ಮವನ್ನು ಮಾಡುವುದರಿಂದ, ನಮ್ಮ ಚಿತ್ತ ಶುದ್ಧವಾಗುತ್ತ ಬಂದು ಕ್ರಮೇಣ ಆತ್ಮನ ಸ್ವಯಂಜ್ಯೋತಿಯನ್ನು ಮುಚ್ಚಿದ್ದ ತೆರೆಗಳು ತಮಗೆ ತಾವೇ ಬೀಳುತ್ತಾ ಹೋಗುವುವು. ಕೊನೆಗೆ ಆವರಣಗಳೆಲ್ಲಾ ಹೋದಮೇಲೆ ಅದು ನನ್ನೊಳಗೇ ಇರುವುದು ಕಾಣುವುದು. ಇದು ಕಸ್ತೂರಿ ಮೃಗ ತನ್ನ ನಾಭಿಯಿಂದಲೇ ಬರುವ ಕಸ್ತೂರಿಯನ್ನು ಭ್ರಾಂತಿಯಿಂದ ಹೊರಗೆ ಹುಡುಕುವಂತೆ. ಹುಡುಕಿ ಹುಡುಕಿ ಅಲೆದಲೆದು ಸಾಕಾಗಿ ಒಂದು ಕಡೆ ಕುಳಿತುಕೊಂಡಾಗ ಆ ಪರಿಮಳ ತನ್ನ ನಾಭಿಯಿಂದಲೇ ಬರುವುದು ಗೊತ್ತಾಗುವುದು.

\begin{shloka}
ಶ್ರದ್ಧಾವಾಂಲ್ಲಭತೇ ಜ್ಞಾನಂ ತತ್ಪರಃ ಸಂಯತೇಂದ್ರಿಯಃ~।\\ಜ್ಞಾನಂ ಲಬ್ಧ್ವಾ ಪರಾಂ ಶಾಂತಿಮಚಿರೇಣಾಧಿಗಚ್ಛತಿ \hfill॥ ೩೯~॥
\end{shloka}

\begin{artha}
ಶ್ರದ್ಧಾವಂತನೂ, ತತ್ಪರನೂ, ಜಿತೇಂದ್ರಿಯನೂ ಆದವನು ಜ್ಞಾನವನ್ನು ಹೊಂದುತ್ತಾನೆ. ಜ್ಞಾನವನ್ನು ಪಡೆದು, ಬೇಗಲೆ ಪರಮ ಶಾಂತಿಯನ್ನು ಪಡೆಯುವನು.
\end{artha}

ಜ್ಞಾನವನ್ನು ಯಾರು ಪಡೆಯುವರು, ಅವರಿಗೆ ಏನೇನು ಗುಣಗಳಿರಬೇಕು ಎನ್ನುವುದನ್ನು ಹೇಳುತ್ತಾನೆ. ಮೊದಲನೆಯದೆ ಆತನಲ್ಲಿ ಶ್ರದ್ಧೆ ಇರಬೇಕು, ಆಸಕ್ತಿ ಇರಬೇಕು ಅದನ್ನು ಪಡೆಯುವುದಕ್ಕೆ. ಒಂದು ವಸ್ತುವನ್ನು ಪಡೆಯುವುದಕ್ಕೆ ನಮ್ಮಲ್ಲಿ ಅತ್ಯಂತ ಮುಖ್ಯವಾಗಿ ಇರಬೇಕಾದ ಗುಣ ಅದು. ಶ್ರದ್ಧೆ ಎಂಬ ಪದ ಮಹತ್ತಾದ ಅರ್ಥದಿಂದ ಕೂಡಿದೆ. ಜೀವನದಲ್ಲಿ ಜ್ಞಾನ ಅತ್ಯಂತ ಉತ್ತಮವಾದುದು, ಪವಿತ್ರವಾದುದು, ಅದನ್ನು ನಾನು ಪಡೆಯಲೇಬೇಕು, ಅದಕ್ಕೆ ಏನು ಕಷ್ಟ ವನ್ನಾದರೂ ಪಡುವುದಕ್ಕೆ ಸಿದ್ಧನಾಗಿರುವೆನು, ಮತ್ತು ಅದನ್ನು ಪೂಜ್ಯ ದೃಷ್ಟಿಯಿಂದ ನೋಡುವೆನು– ಈ ಭಾವಗಳೆಲ್ಲ ಆ ಒಂದು ಪದದಲ್ಲಿ ಅಂತರ್ಗತವಾಗಿದೆ. ಯಾರಿಗೆ ಶ್ರದ್ಧೆ ಎಷ್ಟು ಇರುವುದೋ ಅಷ್ಟೇ ಪ್ರಮಾಣದಲ್ಲಿ ಜ್ಞಾನ ದೊರಕುವುದು. ಶ್ರದ್ಧೆಯೊಂದಿದ್ದರೆ ಮಿಕ್ಕ ಗುಣಗಳೆಲ್ಲ ತಾವೇ ಅದರ ಹಿಂಬದಿಯಾಗಿ ಬರುವುವು.

ಅದನ್ನು ಪಡೆಯಬೇಕೆಂಬ ಶ್ರದ್ಧೆ ಇದ್ದರೆ, ಅದಕ್ಕೆ ಸಂಬಂಧಪಟ್ಟ ಶಾಸ್ತ್ರ ಸಿಕ್ಕುವುದು. ಅದನ್ನು ವಿವರಿಸುವ ಗುರು ಸಿಕ್ಕುವನು. ಅದನ್ನು ಅನುಷ್ಠಾನಕ್ಕೆ ತರಲು ಇರುವ ಆತಂಕಗಳು ಕ್ರಮೇಣ ಮಾಯವಾಗುವುವು. ಜೀವನದಲ್ಲಿ ಅಸಾಧ್ಯವಾದುದನ್ನು ಸಾಧ್ಯವನ್ನಾಗಿ ಮಾಡುವುದು ಶ್ರದ್ಧೆ. ದುರ್ಬಲನನ್ನು ಸಬಲನನ್ನಾಗಿ ಮಾಡುವುದು ಶ್ರದ್ಧೆ. ಶ್ರದ್ಧೆ ಇದ್ದರೆ ಅಯೋಗ್ಯನೂ\break ಯೋಗ್ಯನಾಗುತ್ತಾನೆ. ಶ್ರದ್ಧೆಯೇ ಅಭಾವವಾಗಿರುವಾಗ ಇನ್ನು ಇತರ ಗುಣಗಳು ಎಷ್ಟು ಇದ್ದರೂ ನಮ್ಮ ಸಹಾಯಕ್ಕೆ ಬರಲಾರವು. ಶ್ರದ್ಧೆಯೇ ಜ್ಞಾನ ಭಂಡಾರವನ್ನು ತೆಗೆಯುವುದಕ್ಕೆ ಇರುವ ಬೀಗದ ಕೈ. ಶ್ರದ್ಧೆಯೊಂದಿದ್ದರೆ ಆತ್ಮಸಾಕ್ಷಾತ್ಕಾರ ನಮ್ಮ ಕೈ ಮೇಲಿರುವ ನೆಲ್ಲಿಕಾಯಂತಾಗುವುದು. ವಿವರಿಸಲಾರದ ಆತಂಕಗಳ ಮೂಲಕವೂ ಒಂದು ಸುರಂಗವನ್ನು ಕೊರೆದುಕೊಂಡು ಹೋಗುವುದು ಶ್ರದ್ಧೆ. ಶ್ರದ್ಧೆ ಇದ್ದರೆ ಅವನು ಅಪಾಯವನ್ನು ತೃಣೀಕರಿಸುವನು. ಕಷ್ಟವನ್ನು ಗಮನಕ್ಕೆ ತರುವುದಿಲ್ಲ.

ತತ್ಪರನಾಗಿರುವನು ಎಂದರೆ ಅದರಲ್ಲೇ ಮುಳುಗಿರುವನು. ಏನನ್ನು ಪಡೆಯಬೇಕೆಂದಿರು\-ವನೋ ಅದಕ್ಕೆ ಸಂಬಂಧಪಟ್ಟ ಸಾಧನೆಗಳನ್ನು ಮಾಡುವುದರಲ್ಲಿಯೇ ಮಗ್ನನಾಗಿರುವನು. ಅದಕ್ಕೆ ಸಂಬಂಧಪಟ್ಟ ಶಾಸ್ತ್ರಗಳನ್ನು ಓದುತ್ತಾನೆ, ಅದಕ್ಕೆ ಸಂಬಂಧಪಟ್ಟ ಮಾತುಕತೆ ಆಡುತ್ತಾನೆ. ಅದರ ವಿಷಯವನ್ನು ಕೇಳುವ ಸ್ಥಳಗಳಿಗೆ ಹೋಗುತ್ತಾನೆ. ಅದಕ್ಕೆ ಸಂಬಂಧಪಟ್ಟ ಯೋಚನೆ ಸದಾ ಅವನನ್ನು ಆವರಿಸಿರುವುದು. ಕೂತರೆ, ನಿಂತರೆ, ನಡೆದರೆ ಯಾವಾಗಲೂ ಅವನಲ್ಲಿರುವ ಯೋಚನೆ ಇದೊಂದೆ. ಇದನ್ನು ಒಂದು ವಿಧವಾದ ಖಯಾಲಿ ಎಂದು ಬೇಕಾದರೆ ಹೇಳಬಹುದು. ಜೀವನದಲ್ಲಿ ಯಾವುದನ್ನು ಪಡೆಯಬೇಕಾದರೂ ಇಂತಹ ತತ್ಪರತೆ ಇರಬೇಕು. ಇಲ್ಲದೇ ಇದ್ದರೆ ಜೀವನದಲ್ಲಿ ಶ್ರೇಷ್ಠವಾದುದನ್ನು ಗಳಿಸಲಾರ. ಒಂದು ದೃಷ್ಟಿಯಿಂದ ನಾವೆಲ್ಲ ಖಯಾಲಿಗಳೆ ಎಂದು ಶ‍್ರೀರಾಮಕೃಷ್ಣರು ಹೇಳುತ್ತಿದ್ದರು. ಈ ಪ್ರಪಂಚವೇ ಒಂದು ಖಯಾಲಿಗಳ ಹುಚ್ಚು ಆಸ್ಪತ್ರೆ. ಆದರೆ ಪ್ರತಿಯೊಬ್ಬನೂ ತನ್ನ ಖಯಾಲಿ ಸ್ವಾಭಾವಿಕವಾದುದು, ಮತ್ತೊಬ್ಬನದು ಅಸ್ವಾಭಾವಿಕವಾದುದು ಎಂದು ಭಾವಿಸುತ್ತಾನೆ. ಈ ಖಯಾಲಿಗಳಲ್ಲಿ ದೇವರಿಗೆ, ಆಧ್ಯಾತ್ಮಿಕ ಜೀವನಕ್ಕೆ ಖಯಾಲಿ ಆಗಿರುವವನು ಬೇಗ ಹುಚ್ಚರ ಆಸ್ಪತ್ರೆಯಿಂದ ಪಾರಾಗುವನು. ಆದರೆ ಇತರ ಖಯಾಲಿಗಳಾದರೊ ಹಲವು ಜನ್ಮಗಳನ್ನು ಸಮೆಸಬೇಕಾಗಿದೆ ಇಲ್ಲಿ.

ಜಿತೇಂದ್ರಿಯನಾಗಿರಬೇಕೆಂಬುದೇ ಮೂರನೆಯ ನಿಯಮ. ನಾವು ದೇವರನ್ನು ಪಡೆಯ\-ಬೇಕಾದರೆ ಯಾವುದು ಅವನ ಅನುಭವಕ್ಕೆ ವಿರೋಧವಾಗಿರುವುದೋ ಅದನ್ನು ಬಿಡಬೇಕು. ಭಗವತ್ ಅನುಭವ ಇಂದ್ರಿಯ ಅನುಭವಕ್ಕೆ ವಿರೋಧ. ಒಂದು ಇದ್ದರೆ ಮತ್ತೊಂದು ಸಿಕ್ಕುವುದಿಲ್ಲ. ಯಾವಾಗ ಮನಸ್ಸು ವಿಷಯ ವಸ್ತುಗಳ ಕಡೆ ಹೋಗುವುದೋ, ಆಗ ಅದರೊಂದಿಗೆ ಒಂದು ಸಂಬಂಧವನ್ನು ಕಲ್ಪಿಸಿಕೊಳ್ಳುವುದು, ಅದರ ದಾಸನಾಗುವುದು. ಅದರ ಬಳಿಗೆ ಪದೇ ಪದೇ ಹೋಗುತ್ತಿರುವುದು. ಸ್ಥೂಲವಾಗಿ ಹೋಗುವುದು, ಅದಕ್ಕೆ ಅವಕಾಶವಿಲ್ಲದೆ ಇದ್ದರೆ ಸೂಕ್ಷ್ಮವಾಗಿ ಆ ಗೂಟದ ಸುತ್ತಲೂ ಸುತ್ತುತ್ತಿರುವುದು. ಭಗವದನುಭವ ಇಂದ್ರಿಯಾತೀತವಾದ ಅನುಭವ. ಅದನ್ನು ಪಡೆಯಬೇಕಾದರೆ ಮೊದಲು ನಾವು ಮಾಡಬೇಕಾಗಿರುವುದು ಇಂದ್ರಿಯಗಳನ್ನು ನಿಗ್ರಹಿಸುವುದು. ರಾಮ, ಕಾಮ, ಕತ್ತಲೆ, ಬೆಳಕು, ಒಟ್ಟಿಗೆ ಹೋಗುವುದಿಲ್ಲ. ಒಂದು ಬೇಕಾದರೆ ಮತ್ತೊಂದನ್ನು ಬಿಡಬೇಕು.

ಶ್ರದ್ಧೆ, ತತ್ಪರತೆ, ಇಂದ್ರಿಯ ನಿಗ್ರಹ ಇವುಗಳಿದ್ದರೆ ಜ್ಞಾನವನ್ನು ಪಡೆದುಕೊಳ್ಳುತ್ತಾನೆ. ಕಾರಣ ವಿದ್ದರೆ ಪರಿಣಾಮ ಬರಬೇಕು. ಬೇಕಾಗಿರುವ ಕಾರಣಗಳನ್ನೆಲ್ಲಾ ಒಟ್ಟುಗೂಡಿಸಿದರೆ ಪರಿಣಾಮ ದೂರದಲ್ಲಿ ಇಲ್ಲ. ಕಾರಣಗಳೇ ಕ್ರಮೇಣ ವಿಕಾಸವಾಗಿ ಬೆಳೆದು ಪರಿಣಾಮದ ಫಲವನ್ನು ಕೊಡುತ್ತವೆ. ಕಾರಣವೇ ಬೀಜ. ಪರಿಣಾಮವೇ ಫಲ. ಬೀಜವನ್ನು ಉತ್ತಿದ ನೆಲದಲ್ಲಿ ಬಿತ್ತಿ, ಗೊಬ್ಬರ ಹಾಕಿ, ಶ್ರದ್ಧೆಯಿಂದ ಬೆಳೆಸಿದರೆ ಕೆಲವು ಕಾಲದ ಮೇಲೆ ನಮಗೆ ಫಲ ಬರುವುದರಲ್ಲಿ ಸಂದೇಹವಿಲ್ಲ. ಕಾರಣವೇ ಇಲ್ಲದೆ ಫಲ ಸುಮ್ಮನೆ ನಮ್ಮ ಕೈಗೆ ಬೀಳಲಾರದು.

ಯಾವಾಗ ಜ್ಞಾನವನ್ನು ಪಡೆಯುವನೋ ಆಗ ಪರಮಶಾಂತಿಯನ್ನು ಈ ಜೀವನದಲ್ಲಿ ಪಡೆ\-ಯುತ್ತಾನೆ. ಇದು ಸಾಧಾರಣವಾದ ಶಾಂತಿ ಅಲ್ಲ, ಪರಮಶಾಂತಿ, ಎಂದಿಗೂ ಹೋಗದ ಶಾಂತಿ. ಬಂದರೆ ನಮ್ಮನ್ನು ಬಿಡದ ಶಾಂತಿ, ನಮ್ಮಲ್ಲಿ ಒಂದಾಗುವ ಶಾಂತಿ. ಪರಮಶಾಂತಿಯನ್ನು ಪಡೆದುಕೊಳ್ಳಬೇಕಾದರೆ ಪರಮಾತ್ಮನನ್ನು ಮುಟ್ಟಿರಬೇಕು. ಆಗ ಅವನ ವಿಷಯವೆಲ್ಲ ಗೊತ್ತಾಗುವುದು. ಆಗ ಅವನನ್ನು ಎಲ್ಲಾ ಕಡೆಯಲ್ಲಿಯೂ ನೋಡುವೆವು. ಜೀವನದ ಎಂತಹ ಆಘಾತದ ಸಿಡಿಲುಗಳು ಬಡಿದರೂ ನಾವು ಭದ್ರವಾಗಿರುವೆವು. ಏಕೆಂದರೆ ನಮ್ಮ ಮನೆಯ ಸುತ್ತಲೂ ಬಡಿದ ಸಿಡಿಲನ್ನು ತಕ್ಷಣವೇ ಸ್ವೀಕರಿಸಿ ಭೂಮಿಗೆ ಒಯ್ಯುವ ಶಕ್ತಿಯಿಂದ ಆವೃತರಾಗಿರುವೆವು. ಭೋರ್ಗರೆದು ಮೊರೆಯುತ್ತಿರುವ ಸಂಸಾರ ಸಾಗರದ ಮಧ್ಯದಲ್ಲಿದ್ದರೂ ಅವನು ಸುರಕ್ಷಿತ. ಅವನು ತೋರಿಕೆಗೆ ಅಂಜುವವನಲ್ಲ ಇನ್ನು ಮೇಲೆ. ಎಂತಹ ಭಯಾನಕವಾದ ತೋರಿಕೆಯ ಹಿಂದೆಯೂ ಭಗವಂತನ ಮಂಗಳ ಸ್ವರೂಪವನ್ನೇ ನೋಡುವನು. ಮಾಯೆ ಇನ್ನು ಮೇಲೆ ಅವನನ್ನು ಕಾಡಲಾರಳು. ಏಕೆಂದರೆ ಮಾಯಾಧಿಪತಿಯನ್ನು ಇವನು ಕಂಡಿರುವನು.

ಜ್ಞಾನ ಪ್ರಾಪ್ತವಾದ ಕೂಡಲೆ ಪರಮಶಾಂತಿ ಲಭಿಸುವುದು. ಇನ್ನು ಹೆಚ್ಚು ಕಾಯಬೇಕಾಗಿಲ್ಲ. ದೀಪವನ್ನು ತಂದೊಡನೆಯೇ ಕತ್ತಲು ಹೋಗುವುದು. ಅದು ಒಂದೇ ಕ್ಷಣದಲ್ಲಿ ಹೋಗುವುದು, ನಿಧಾನವಾಗಿ ಹೋಗುವುದಿಲ್ಲ. ಒಂದಿದ್ದರೆ ಮತ್ತೊಂದಕ್ಕೆ ಕಾಯಬೇಕಾಗಿಲ್ಲ. ಏನೋ, ಬರುವುದೋ ಇಲ್ಲವೋ ಎಂದು ಸಂದೇಹ ಪಡಬೇಕಾಗಿಲ್ಲ. ನಾನಿದ್ದೆಡೆ ನನ್ನ ನೆರಳು ಹೇಗೆ ಬೆಂಬಿಡದೆ ಇರುವುದೋ ಹಾಗೇ ಪರಮಜ್ಞಾನ ಇದ್ದರೆ ಪರಮ ಶಾಂತಿ ಬೆಂಬಿಡದೆ ಇರುವುದು. ಅವನಲ್ಲಿರುವ ಜ್ಞಾನವನ್ನು ತಿಳಿದುಕೊಳ್ಳುವುದಕ್ಕೆ ನಮಗಾಗುವುದಿಲ್ಲ. ಆದರೆ ಆ ಮನುಷ್ಯನ ಪರಮಶಾಂತಿಯನ್ನು ನೋಡಬಹುದು. ಅವನ ನಡೆ ನುಡಿ, ಈ ಪ್ರಪಂಚವನ್ನು ನೋಡುವ ರೀತಿ ಇವುಗಳಲ್ಲೆಲ್ಲಾ ಆ ಶಾಂತಿ ಸ್ಪಂದಿಸುತ್ತಿರುವುದು. ಇನ್ನವನು ಈ ಪ್ರಪಂಚದಲ್ಲಿ ಯಾವುದಕ್ಕೂ ಮರುಗನು, ಕೊರಗನು.

\begin{shloka}
ಅಜ್ಞಶ್ಚಾಶ್ರದ್ದಧಾನಶ್ಚ ಸಂಶಯಾತ್ಮಾ ವಿನಶ್ಯತಿ~।\\ನಾಯಂ ಲೋಕೋಽಸ್ತಿ ನ ಪರೋ ನ ಸುಖಂ ಸಂಶಯಾತ್ಮನಃ \hfill॥ ೪೦~॥
\end{shloka}

\begin{artha}
ಜ್ಞಾನವಿಲ್ಲದವನು, ಶ್ರದ್ಧೆ ಇಲ್ಲದವನು, ಸಂಶಯಾತ್ಮನು ನಾಶವಾಗುತ್ತಾನೆ. ಸಂಶಯಾತ್ಮನಿಗೆ ಈ ಲೋಕವೂ ಇಲ್ಲ, ಪರಲೋಕವೂ ಇಲ್ಲ, ಸುಖವೂ ಇಲ್ಲ.
\end{artha}

\newpage

ಜ್ಞಾನವಿಲ್ಲದವನು, ಶ್ರದ್ಧೆಯಿಲ್ಲದವನು, ಸಂಶಯಾತ್ಮ ಈ ಮೂರು ಬಗೆಯ ಜನರೂ ನಾಶ\-ವಾಗುತ್ತಾರೆ ಎನ್ನುವನು ಶ‍್ರೀಕೃಷ್ಣ. ನಾಶವಾಗುತ್ತಾರೆ ಎಂದರೆ, ಎಂದೆಂದಿಗೂ ನಾಶವಾಗುತ್ತಾರೆ ಎಂದು ನಾವು ಭಾವಿಸಬಾರದು. ಅದನ್ನು ಪಡೆಯದೆ ಇರುವವರೆಗೆ ನಾಶವಾದಂತೆ ಇರುತ್ತಾರೆ. ಯಾವಾಗ ಅದನ್ನು ಪಡೆಯುತ್ತಾರೋ ಆಗ ಅವರು ಉದ್ಧಾರವಾಗುತ್ತಾರೆ. ಈಗ ಇವುಗಳು ಇಲ್ಲ ಎಂದರೆ, ಎಂದೆಂದಿಗೂ ಇಲ್ಲ ಎಂದು ನಾವು ಭಾವಿಸಬಾರದು. ಜೀವನದಲ್ಲಿ ಹೊಸ ಭವ್ಯ ಅಧ್ಯಾಯ ಯಾವಾಗ ಪ್ರಾರಂಭವಾಗುವುದಕ್ಕೆ ಕಾದುಕೊಂಡಿದೆಯೋ ನಮಗೆ ಗೊತ್ತಿಲ್ಲ. ನಮ್ಮ ಜೀವನದಲ್ಲೆ ನೋಡುತ್ತೇವೆ, ಒಂದು ಕಾಲದಲ್ಲಿ ಜಡವಾದಿಗಳು ಆಗಿದ್ದವರು, ಸಂಶಯವಾದಿಗಳು, ಚಾರ್ವಾಕರಂತಿದ್ದವರು, ದೊಡ್ಡ ಭಕ್ತರಾಗಿರುವುದನ್ನು.

ಅಂತೂ ಅಜ್ಞಾನದಲ್ಲಿರುವವರೆಗೆ ನಮಗೆ ಕಾಣುವಂತೆ ನಮ್ಮ ಮುಂದೆ ಯಾವ ಭವ್ಯ ಜೀವನದ ಆದರ್ಶವೂ ಇರುವುದಿಲ್ಲ. ಅಜ್ಞಾನಿಗೆ ಈಗ ಯಾವುದು ಕಾಣುತ್ತದೆಯೋ, ಯಾವುದನ್ನು ಅನುಭವಿಸುತ್ತಾನೆಯೋ ಅದೊಂದೇ ಸತ್ಯ. ಅದನ್ನು ಮೀರಿದ ಸತ್ಯ ಅವನಿಗೆ ಅರಿಯದು. ಇಂದ್ರಿಯ ಮತ್ತು ಅದಕ್ಕೆ ಸುಖವನ್ನು ಕೊಡುವ ವಸ್ತುಗಳು ಇವುಗಳು ಮಾತ್ರ ಸತ್ಯ ಅವನಿಗೆ. ಜಡ ಕಣ್ಣಿಗೆ ಅವನ ಆತ್ಮನಾಗಲಿ, ಎಲ್ಲಾ ವಸ್ತುಗಳ ಹಿಂದೆ ಇರುವ ಪರಮಾತ್ಮನಾಗಲಿ ಕಾಣುವುದಿಲ್ಲ. ಆದಕಾರಣ ಅವನು ಅದನ್ನು ನಂಬುವುದಿಲ್ಲ. ಅದರಂತೆಯೇ ಅವನ ಹಿಂದಿನ ಜನ್ಮದ ನೆನಪಿಲ್ಲ, ಮುಂದೆ ಸತ್ತ ಮೇಲೆ ಏನಾಗುವುದೋ ಅದು ಗೊತ್ತಿಲ್ಲ. ಅವನು ಈಗಿನದನ್ನು ಮಾತ್ರ ನಂಬುತ್ತಾನೆ. ದೇಹವನ್ನು ಮಾತ್ರ ನಂಬುತ್ತಾನೆ. ದೇಹದ ಹಿಂದಿರುವ ಸನಾತನವಾದ ಆತ್ಮನನ್ನು ನಂಬುವುದಿಲ್ಲ. ಇಲ್ಲಿ ಅಜ್ಞಾನಿ ಎಂದರೆ ಕ್ಷಣಿಕವನ್ನು ಮಾತ್ರ ನಂಬುವವನು. ಕ್ಷಣಿಕದ ಹಿಂದೆ ಇರುವ ನಿತ್ಯ ವಸ್ತುವನ್ನು ಅವನು ನಂಬುವುದಿಲ್ಲ. ಯಾವಾಗ ನಶ್ವರವನ್ನು ಮಾತ್ರ ನಂಬುತ್ತಾನೋ, ತಾತ್ಕಾಲಿಕ ಸುಖದ ವೇದನೆಗಳನ್ನು ಕೊಡುವ ಅನುಭವಗಳನ್ನು ಮಾತ್ರ ನಂಬುತ್ತಾನೋ ಅವನು ಸಂಸಾರದ ಅಲೆಯ ಉಪಟಳಕ್ಕೆ ಸಿಕ್ಕಿ ನಾಶವಾಗುವನು.

ಎರಡನೆಯವನೆ ಶ್ರದ್ಧೆ ಇಲ್ಲದವನು. ಅವನಿಗೆ ಹಿಂದಿನವರು ನಮ್ಮ ಉದ್ಧಾರಕ್ಕಾಗಿ ಬರೆ\-ದಿಟ್ಟಿರುವ ಶಾಸ್ತ್ರದಲ್ಲಿ ಶ್ರದ್ಧೆಯಿಲ್ಲ. ಅವರೆಲ್ಲೊ ಕಲ್ಪನೆಯ ಭ್ರಾಂತಿಯಿಂದ ಅದನ್ನೆಲ್ಲಾ ಬರೆದಿಟ್ಟಿರುವರು ಎಂದು ತಾತ್ಸಾರದಿಂದ ನೋಡುವನು. ಯಾರಾದರೂ ಆ ಮಾರ್ಗದಲ್ಲಿ ಹೋದವರು ಏನಾದರೂ ವಿಷಯಗಳನ್ನು ಹೇಳಿದರೆ ಅದಕ್ಕೆ ಕಿವಿಗೊಡನು. ಇದ್ದರೆ ತನಗೆ ಏತಕ್ಕೆ ಕಾಣುವುದಿಲ್ಲ ಎನ್ನುವನು. ನೀನು ಅದಕ್ಕೆ ಅಣಿಯಾದರೆ ಕಾಣುವುದು ಎಂದರೆ, ಒಂದು ಸತ್ಯ ನಿಜವಾಗಿದ್ದರೆ, ಅದು ನಾನು ಅಣಿಯಾಗಲಿ ಬಿಡಲಿ ಅದು ನನ್ನ ಮೇಲೆ ತನ್ನ ಪ್ರಭಾವವನ್ನು ಬೀರಬೇಕು ಎನ್ನುವನು. ಅವನಿಗೆ ಇನ್ನೂ ಆಧ್ಯಾತ್ಮಿಕ ಹಸಿವು ಉಂಟಾಗಿಲ್ಲ. ಆದಕಾರಣ ಎಲ್ಲವನ್ನೂ ಕಾಟಾಚಾರದಿಂದ ನೋಡುತ್ತಾನೆ. ತನ್ನನ್ನು ಮೀರಿದ ಚೈತನ್ಯವೊಂದು ಈ ಪ್ರಪಂಚವನ್ನು ಆಳುತ್ತಿದೆ ಎಂಬ ಅನುಭವ ಅವನಿಗಿನ್ನೂ ಆಗಿಲ್ಲ. ಇದು ಬೆಳವಣಿಗೆಯ ಒಂದು ಸ್ಥಿತಿ, ಕೆಳಗಿನ ಮೆಟ್ಟಿಲು. ಅಲ್ಲೇ ಎಂದೆಂದಿಗೂ ನಿಂತಿರುವುದಕ್ಕೆ ಆಗುವುದಿಲ್ಲ ಎಂಬುದು ಅವನಿಗೆ ಇನ್ನೂ ಅರ್ಥವಾಗಿಲ್ಲ. ಮತ್ತಾವುದೋ ಒಂದು ಘಟನೆ ಜೀವನದಲ್ಲಿ ಆಗಿ ಅವನನ್ನು ಮುಂದಕ್ಕೆ ತಳ್ಳುವುದು. ಆಗ, ಈಗ ನೋಡದುದನ್ನು ನೋಡುವನು. ಆದರೆ ಅದಾಗುವವರೆಗೆ ಅವನು ಭಂಡ, ಈಗಿರುವ ಸ್ಥಿತಿಯನ್ನು ಬಿಟ್ಟು ಮುಂದುವರಿಯುವುದಿಲ್ಲ.

ಮೂರನೆಯವನೆ ಸಂಶಯಾತ್ಮ. ಹಿಂದಿನ ಮೂರು ಜನರಿಗಿಂತ ಸ್ವಲ್ಪ ಮುಂದುವರಿದವನು. ಅವನು ತನಗೆ ಈಗ ಏನು ಕಾಣುವುದೋ ಅದೊಂದೇ ಸತ್ಯ ಎಂದೂ ಹೇಳುವುದಿಲ್ಲ. ಏಕೆಂದರೆ ಅದನ್ನು ಮೀರಿದ ವಸ್ತುಗಳು ಘಟನೆಗಳು ಕೆಲವು ವೇಳೆ ಅವನ ಅನುಭವಕ್ಕೆ ಬಂದಿವೆ. ಆದರೆ ಅದರಲ್ಲಿ ಸಂಪೂರ್ಣ ನಂಬಿಕೆ ಬಂದಿಲ್ಲ. ಏಕೆಂದರೆ ಮನಸ್ಸು ಬಹಳ ಕಾಲ ಮೇಲಿನ ಮಟ್ಟದಲ್ಲಿ ಇರುವುದಕ್ಕೆ ಆಗುವುದಿಲ್ಲ. ಕೆಲವು ವೇಳೆ ಹಲವಾರು ಕಾರಣಗಳಿಂದ ಮನಸ್ಸು ಸಾಧಾರಣವಾಗಿ ಕೆಲಸ ಮಾಡುವ ಕ್ಷೇತ್ರವನ್ನು ಮೀರಿ ಮೇಲಕ್ಕೆ ಹೋಗುವುದು. ಆದರೆ ಬಹಳ ಕಾಲ ಅದು ಅಲ್ಲಿರುವುದಿಲ್ಲ. ಚೆಂಡನ್ನು ಕಾಲಿನಿಂದ ಒದೆದರೆ ಅದು ಮೇಲಕ್ಕೆ ಹೋಗುವುದು. ಆದರೆ ಮೇಲೆ ಎಷ್ಟು ಹೊತ್ತು ಇರುವುದು? ಎಲ್ಲೋ ಸ್ವಲ್ಪ ಕಾಲ. ಪುನಃ ಭೂಮಿಗೆ ಬೀಳುವುದು. ಅವನ ಮನಸ್ಸು ಮನೆಯ ಹೊಸಲಿನ ಮೇಲೆ ಇರುವಂತೆ. ಸಂಪೂರ್ಣ ಒಳಗೂ ಇಲ್ಲ, ಹೊರಗೂ ಇಲ್ಲ. ಹಿಂದಿನ ಇಬ್ಬರು ಇಂದ್ರಿಯದ ಬಿಲದ ಒಳಗೇ ಇರುವರು. ಅವರು ಅದರಿಂದ ಹೊರಗೆ ಬಂದೇ ಇಲ್ಲ. ಅಲ್ಲಿರುವುದು ಮಾತ್ರ ಸತ್ಯ ಎಂದು ಹೇಳುತ್ತಾರೆ. ಈ ಮೂರನೆಯವನು ಸ್ವಲ್ಪ ಹೊರಗೆ ಬಂದು ನೋಡಿರುವನು. ಇಂದ್ರಿಯವನ್ನು ಮೀರಿದ ಸತ್ಯದ ಕ್ಷಣಿಕ ನೋಟ ಅವನ ಪಾಲಿಗೆ ಸಿಕ್ಕಿದೆ. ಆದರೆ ಅಲ್ಲೇ ಮನಸ್ಸು ಇರಲಾರದು. ಅವನ ಸಂಸ್ಕಾರಗಳು ಪುನಃ ಬಿಲದ ಒಳಕ್ಕೆ ಎಳೆಯುವುವು. ಇಂತಹ ಸ್ಥಿತಿಯಲ್ಲಿರುವವರ ಮನಸ್ಸು ನಿಜವಾಗಿಯೂ ಒಂದು ದೊಡ್ಡ ವ್ಯಾಕುಲದಲ್ಲಿರುವುದು. ಇಂದ್ರಿಯದಲ್ಲೇ ಉನ್ಮತ್ತನಾಗಿ ಅನುಭವಿಸುವುದಕ್ಕೂ ಆಗುವುದಿಲ್ಲ. ಇದೆಲ್ಲ ಸುಳ್ಳಾಗಿದ್ದರೆ ನನ್ನ ಗತಿ ಏನಾಗುವುದು ಎಂದು ಸುಖವನ್ನು ಅನುಭವಿಸುತ್ತಿರುವಾಗಲೂ ಇದು ಬಾಧಿಸುತ್ತಿರುವುದು. ಇಂದ್ರಿಯಾತೀತ ಅನುಭವಕ್ಕಾಗಿ ಇಂದ್ರಿಯ ಸುಖಗಳನ್ನೆಲ್ಲ ಬಲಿ ಕೊಡುವಷ್ಟು ಸಾಮರ್ಥ್ಯವೂ ಇವನಲ್ಲಿ ಇಲ್ಲ. ಏಕೆಂದರೆ ಆ ಅನುಭವಗಳೆಲ್ಲ ನಮ್ಮ ಭ್ರಮೆಯಾಗಿದ್ದರೆ ಕೈಯಲ್ಲಿರುವ ನಿಜವಾದ ವಸ್ತುವನ್ನು ಬಿಟ್ಟು ಸಂಶಯಾತ್ಮಕವಾದ ಸುಖವನ್ನು ಅರಸಿಕೊಂಡು ಹೋಗಬೇಕಾಗುವುದು. ಯಾರಿಗೆ ಗೊತ್ತು, ಅದು ಒಂದು ಮರೀಚಿಕೆ ಇರಬಹುದು.

ಸಂಶಯಾತ್ಮನಿಗೆ ಈ ಲೋಕವಿಲ್ಲ. ಏಕೆಂದರೆ ಎದುರಿಗೆ ಇರುವುದನ್ನು ಅನುಮಾನಿಸುವನು. ಯಾರಿಗೆ ಗೊತ್ತು ಇವೆಲ್ಲ ಒಂದು ತೋರಿಕೆ ಇರಬಹುದು ಎಂದು. ಇಲ್ಲಿ ಯಾವುದನ್ನೂ ಸರಿಯಾಗಿ ದೃಢವಾಗಿ ನಂಬದೆ ಇರುವುದರಿಂದ ಅದನ್ನು ಸಾಧಿಸುವ ಛಲ ಇರುವುದಿಲ್ಲ. ಅದರಂತೆಯೇ ಪರಲೋಕವನ್ನು ಅವನು ನಂಬುವವನಲ್ಲ. ಪುನರ್ಜನ್ಮವನ್ನಾಗಲಿ, ಹಿಂದಿನ ಜನ್ಮವನ್ನಾಗಲಿ ನಂಬುವುದಿಲ್ಲ. ನಂಬಿಕೆಯೇ ಈ ಜೀವನದ ಜಾರುಭೂಮಿಯ ಮೇಲೆ ನಡೆಯುತ್ತಿರುವಾಗ ಬೀಳದಂತೆ ನಮ್ಮನ್ನು ತಡೆದು ಹಿಡಿದಿರುವ ಊರುಗೋಲು. ಯಾವಾಗ ಈ ಊರುಗೋಲು ಹೋಗುವುದೋ ಆಗ ಮುಂದಡಿ ಇಡಲಾರದೆ ಕುಸಿದು ಹೋಗುವನು. ಇಂತಹವನಿಗೆ ಸುಖವೂ\break ಇಲ್ಲ ಎನ್ನುತ್ತಾನೆ. ಇಂದ್ರಿಯ ಸುಖವನ್ನು ನಾವು ಅನುಭವಿಸಬೇಕಾದರೆ ಮುಂದೇನಾಗುವುದು ಎಂಬುದನ್ನು ಸಂಪೂರ್ಣವಾಗಿ ಮರೆತು, ಈಗಿನದರಲ್ಲಿ ಮಾತ್ರ ಸಂಪೂರ್ಣ ತನ್ಮಯರಾಗಬೇಕು. ಪ್ರಾಣಿಗಳು ಹೀಗೆ ಮಾಡುವುವು. ಎದುರಿಗೆ ಸಿಕ್ಕಿದ ಸುಖ ಒಂದೇ ಅವಕ್ಕೆ ಸತ್ಯ. ಅನಂತರ ಏನಾದರೂ ಆಗಲಿ ಅದನ್ನು ಗಮನಿಸುವುದೇ ಇಲ್ಲ. ಅದರಂತೆಯೇ ಪ್ರಾಣಿಸಮಾನವಾದ ಮನುಷ್ಯರು ಕೂಡ. ಜಡವಾದಿಗಳು ಚಾರ್ವಾಕರು ಇವರೆಲ್ಲ ಈ ಗುಂಪಿಗೆ ಸೇರಿದವರು. ಆದರೆ ಸಂಶಯಾತ್ಮನೋ ಅವರಿಗಿಂತ ಮೇಲೆದ್ದಿರುವನು. ಈಗ ಇರುವ ಪರಿಸ್ಥಿತಿಯಲ್ಲಿ ಅತೃಪ್ತಿ ಏನೋ ಇದೆ. ಆದರೆ ಅದನ್ನು ಮೀರಿದ ಅನುಭವವನ್ನು ಪಡೆಯುವ ಛಲ ಇಲ್ಲ, ದಿಟ್ಟತನವಿಲ್ಲ. ಅದಕ್ಕಾಗಿ ಸರ್ವಸ್ವವನ್ನೂ ಅರ್ಪಣೆ ಮಾಡಿ ಹೋರಾಡುವ ಸಾಹಸ ಇಲ್ಲ. ನಿಜವಾಗಿಯೂ ಇದೊಂದು ಶೋಚನೀಯ ಸ್ಥಿತಿ, ಆದರೆ ಇದೇ ಗುರಿಯಲ್ಲ, ಎಂದೆಂದಿಗೂ ಇರುವ ಸ್ಥಿತಿಯಲ್ಲ. ಇದೊಂದು ಮಾನವ ಚೇತನ ವಿಕಾಸವಾಗುತ್ತಿರುವಾಗ ಬರುವ ಮಾನಸಿಕ ಪರಿಸ್ಥಿತಿ. ಹೋರಾಡಿದರೆ ಇದರಿಂದ ಮೇಲೆ ಬರಲು ಸಾಧ್ಯ. ಇಲ್ಲದೇ ಇದ್ದರೆ ಈ ಸಂಶಯದ ಅಲೆಗಳ ಉಪಟಳಕ್ಕೆ ಸಿಕ್ಕಿ ನಾವು ನರಳಬೇಕಾಗುವುದು.

\begin{shloka}
ಯೋಗಸಂನ್ಯಸ್ತಕರ್ಮಾಣಂ ಜ್ಞಾನಸಂಛಿನ್ನಸಂಶಯಮ್~।\\ಆತ್ಮವಂತಂ ನ ಕರ್ಮಾಣಿ ನಿಬಧ್ನಂತಿ ಧನಂಜಯ \hfill॥ ೪೧~॥
\end{shloka}

\begin{artha}
ಅರ್ಜುನ, ಯೋಗದಿಂದ ಕರ್ಮಗಳನ್ನು ನಾಶಮಾಡಿಕೊಂಡವನೂ ಜ್ಞಾನದಿಂದ ಸಂಶಯಗಳನ್ನು ನಾಶಮಾಡಿಕೊಂಡವನೂ, ಆತ್ಮವಂತನೂ ಆಗಿರುವವನನ್ನು ಕರ್ಮಗಳು ಬಂಧಿಸಲಾರವು.
\end{artha}

ನಮ್ಮನ್ನು ಸಂಸಾರಕ್ಕೆ ಪುನಃಪುನಃ ಸೆಳೆಯುವುದೇ ಕರ್ಮಶೇಷ. ಎಲ್ಲಿಯವರೆಗೆ ನಾವು ಅದರಿಂದ ಪಾರಾಗಿಲ್ಲವೋ ಅಲ್ಲಿಯವರೆಗೆ ನಮಗೆ ಬಿಡುಗಡೆ ಇಲ್ಲ. ನಾವು ಪುನಃ ಬಂದು ಆ ಸಾಲವನ್ನು ತೀರಿಸಬೇಕಾಗುವುದು. ಅನೇಕ ವೇಳೆ ಆ ಸಾಲವನ್ನು ತೀರಿಸುವುದಕ್ಕೆ ಬೇರೊಂದು ಸಾಲವನ್ನು ಮಾಡುವೆವು. ಆದರೆ ಯಾವಾಗ ಕರ್ಮ ರಹಸ್ಯವನ್ನು ತಿಳಿದುಕೊಂಡು ಕರ್ಮ ಮಾಡುವೆವೊ ಆಗ ಅದರಿಂದ ಪಾರಾಗುವೆವು. ಆ ರಹಸ್ಯವೇ ಯೋಗ, ಎಂದರೆ ಫಲಾಪೇಕ್ಷೆ ಇಲ್ಲದೆ ಕರ್ಮ ಮಾಡುವುದು, ಭಗವಂತನಿಗೆ ಅರ್ಪಿತ ಭಾವದಿಂದ ಕರ್ಮಗಳನ್ನು ಮಾಡುವುದು, ಯಜ್ಞದೃಷ್ಟಿಯಂತೆ ಕರ್ಮ ಮಾಡುವುದು.

ಯಾವಾಗ ನಾವು ಕರ್ಮವನ್ನು ಮೇಲೆ ಹೇಳಿದ ದೃಷ್ಟಿಯಿಂದ ಮಾಡುವೆವೋ ಆಗ ಜ್ಞಾನ ನಮಗೆ ಪ್ರಾಪ್ತವಾಗುವುದು. ಪರಿಸ್ಥಿತಿಯನ್ನು ಯಥಾರ್ಥವಾಗಿ ತಿಳಿದುಕೊಳ್ಳುವೆವು. ನಮ್ಮ ಆಸೆ ಆಕಾಂಕ್ಷೆಗಳು ಯಾವ ಆರೋಪವನ್ನು ಮಾಡದೆ, ಸತ್ಯ ಹೇಗಿದೆಯೋ ಹಾಗೆ ನಾವು ಅದನ್ನು ತಿಳಿದುಕೊಳ್ಳುವ ಸ್ಥಿತಿಗೆ ಬರುತ್ತೇವೆ. ಸಂದೇಹಕ್ಕೆಲ್ಲ ಕಾರಣ ಚಿತ್ತಶುದ್ಧಿಯಾಗದೆ ಒಂದು ವಸ್ತುವನ್ನು ತಿಳಿಯಲೆತ್ನಿಸುವುದು. ಚಿತ್ತ ಶುದ್ಧವಿಲ್ಲದೆ ಇದ್ದರೆ, ಒಂದಿದ್ದರೆ ಬೇರೊಂದು ಕಾಣುವುದು. ಶುದ್ಧವಲ್ಲದ ಚಿತ್ತವನ್ನು ನಾವು ನಿಗ್ರಹಿಸಲೂ ಆರೆವು. ಅದು ಆ ಕೊಳೆ ಕಶ್ಮಲದ ಕಡೆಗೆ ಯಾವಾಗಲೂ ಓಡಿ ಓಡಿ ಹೋಗುತ್ತಿರುವುದು. ಯಾವಾಗ ಚಿತ್ತಶುದ್ಧಿಯಾಗಿ ಸರಿಯಾದ ಬುದ್ಧಿಯಿಂದ ನೋಡುವೆವೋ ಆಗ ಸಂಶಯ ಪಿಶಾಚಿಗಳು ಹೋಗುವುವು. ಕತ್ತಲಲ್ಲಿ ತಾನೇ ಮೋಟು ಮರಗಳು ಮತ್ತು ಇತರ ವಸ್ತುಗಳು ದೆವ್ವಗಳಂತೆ ಕಾಣುವುದು. ಯಾವಾಗ ಸೂರ್ಯನ ಬೆಳಕು ಬೀಳುವುದೋ ಇನ್ನು ಮೇಲೆ ಯಾವ ಕತ್ತಲೆಯಿಂದ ಉಂಟಾದ ತಪ್ಪು ಭಾವನೆಗಳೂ ಇರುವುದಿಲ್ಲ.

ಯಾರು ಆತ್ಮವಂತನಾಗಿರುವನೋ ಎಂದರೆ ಇಂದ್ರಿಯಗಳನ್ನು ನಿಗ್ರಹಿಸುವನೋ, ತಾನು ಹೇಳಿದಂತೆ ಕೇಳುವಂತೆ ಮಾಡಿಕೊಂಡಿರುವನೋ ಅವನನ್ನು ಕರ್ಮಗಳು ಬಂಧಿಸಲಾರವು. ವಸ್ತುವನ್ನು ನೈಜಸ್ಥಿತಿಯಲ್ಲಿ ಬೌದ್ಧಿಕವಾಗಿ ತಿಳಿದುಕೊಳ್ಳಬಹುದು. ಅಷ್ಟೇ ಸಾಲದು. ಸುಳ್ಳನ್ನು ಅದು ಎಷ್ಟೇ ಪ್ರಿಯವಾಗಿರಲಿ ಬಿಡಬೇಕಾದರೆ ನಮ್ಮ ಇಂದ್ರಿಯಗಳು ನಮ್ಮೊಂದಿಗೆ ಸಹಕರಿಸಬೇಕು. ಇಲ್ಲದೇ ಇದ್ದರೆ, ಒಂದೊಂದು ಒಂದೊಂದು ದಿಕ್ಕಿಗೆ ಎಳೆಯುವುವು. ಅದಕ್ಕೇ ಶ‍್ರೀಕೃಷ್ಣ ಹೇಳುವುದು, ಯಾರು ಇಂದ್ರಿಯಾಕರ್ಷಣೆಯಿಂದ ಪಾರಾದವನೊ ಅವನನ್ನು ಕರ್ಮಗಳು ಬಂಧಿಸುವುದಿಲ್ಲ ಎಂದು. ಕೇವಲ ಲೋಕಸಂಗ್ರಹದ ದೃಷ್ಟಿಯಿಂದ ಮಾತ್ರ ಮಾಡುತ್ತಾನೆ. ಹಾಗೆ ಮಾಡುತ್ತಿರುವಾಗಲೂ ಇದನ್ನು ನಾನು ಮಾಡುತ್ತಿರುವೆ, ಎಂಬ ಅಹಂಕಾರವಿರುವುದಿಲ್ಲ. ತಾನೊಂದು ಯಂತ್ರ. ಅದರ ಮೂಲಕ ದೇವರು ಮಾಡಿಸುತ್ತಿರುವನು ಎಂದು ಭಾವಿಸುವನು.

\begin{shloka}
ತಸ್ಮಾದಜ್ಞಾನಸಂಭೂತಂ ಹೃತ್ಸ್ಥಂ ಜ್ಞಾನಾಸಿನಾತ್ಮನಃ~।\\ಛಿತ್ತ್ವೈನಂ ಸಂಶಯಂ ಯೋಗಮಾತಿಷ್ಠೋತ್ತಿಷ್ಠ ಭಾರತ \hfill॥ ೪೨~॥
\end{shloka}

\begin{artha}
ಆದುದರಿಂದ ಅಜ್ಞಾನಜನ್ಯವಾದ, ನಿನ್ನ ಹೃದಯದಲ್ಲಿರುವ ಆತ್ಮನಿಗೆ ಸಂಬಂಧಪಟ್ಟ ಸಂಶಯಗಳನ್ನು ಜ್ಞಾನ ಖಡ್ಗದ ಮೂಲಕ ಛೇದಿಸಿ ಯೋಗದಲ್ಲಿ ನೆಲೆಸು. ಅರ್ಜುನ, ಕಾರ್ಯೋನ್ಮುಖನಾಗು.
\end{artha}

ಅರ್ಜುನನ ಮನಸ್ಸಿನಲ್ಲಿ ಆತ್ಮನ ವಿಷಯದಲ್ಲಿ ಹಲವು ಸಂಶಯಗಳು ಬಾಧಿಸುತ್ತಿದ್ದುದನ್ನು ಮೊದಲೇ ನೋಡಿದೆವು ನಾವು. ಆತ್ಮ ಇರುವನೆ ಇಲ್ಲವೆ, ಸಾವೆ ಅವನ ಅಂತ್ಯವೇ ಅಥವಾ ಅವನಿಗೊಂದು ಭವಿಷ್ಯ ಇದೆಯೋ, ದೇಹಕ್ಕೂ ಆತ್ಮನಿಗೂ ಇರುವ ಸಂಬಂಧಗಳೇನು, ತಾನು ಈ ಘೋರವಾದ ಯುದ್ಧವನ್ನು ಮಾಡಿದರೆ, ಈ ಯುದ್ಧದಲ್ಲಿ ಗುರುಹಿರಿಯರನ್ನು, ಬಂಧು ಬಾಂಧವ ರನ್ನು ಕೊಂದರೆ ಪಾಪ ಬರುವುದಿಲ್ಲವೆ ಎಂದೆಲ್ಲ ಸಂಶಯಗಳ ತುಮುಲ ಭೂಮಿಯಾಗಿತ್ತು ಅವನ ಮನಸ್ಸು. ಶ‍್ರೀಕೃಷ್ಣನ ಬೋಧನೆಯಿಂದ ಇವುಗಳನ್ನೆಲ್ಲಾ ಸರಿಯಾಗಿ ತಿಳಿದುಕೊಳ್ಳುವುದಕ್ಕೆ ಸಾಧ್ಯವಾಯಿತು. ತನ್ನ ಜ್ಞಾನದಿಂದಲೆ, ತಿಳಿವಳಿಕೆಯಿಂದಲೇ ತನ್ನಲ್ಲಿ ಹುದುಗಿರುವ ಸಂಶಯದ ಕಳೆಗಳನ್ನು ಕಿತ್ತುಹಾಕಬೇಕು. ಇನ್ನೊಬ್ಬರು ಎಷ್ಟೇ ಹೇಳಿದರೂ ಅದನ್ನು ನಾವು ಚೆನ್ನಾಗಿ ತಿಳಿದು\-ಕೊಂಡಾಗಲೇ ಅದು ನಮ್ಮ ಜ್ಞಾನವಾಗಬೇಕಾದರೆ. ಅದರಿಂದ ನಮ್ಮಲ್ಲಿರುವ ಅಜ್ಞಾನವನ್ನು ನಾಶ ಮಾಡಿಕೊಳ್ಳಬೇಕು. ರೋಗಿಗೆ ವೈದ್ಯ ಔಷಧಿಯನ್ನು ಕೊಡಬಹುದು. ಆದರೆ ರೋಗಿ ಅದನ್ನು ಸೇವಿಸಬೇಕು. ಅದು ಅವನ ಹೊಟ್ಟೆಗೆ ಹೋಗಿ ಅಲ್ಲಿ, ದೇಹವು ರೋಗಕ್ಕೆ ಕಾರಣವಾದ ಕ್ರಿಮಿಗಳೊಡನೆ ಹೋರಾಡುವುದಕ್ಕೆ ಸಹಾಯ ಮಾಡುವುದು ಅಷ್ಟೆ. ಆಗ ಮಾತ್ರ ರೋಗದಿಂದ ಪಾರಾಗುವೆವು. ಅಂತೂ ಇದನ್ನು ಕೊನೆಗೆ ಗುಣ ಮಾಡಿಕೊಳ್ಳುವವರು ನಾವೇ. ಇನ್ನೊಬ್ಬರು ಸ್ವಲ್ಪ ಸಹಾಯ ಮಾಡಬಹುದು ಅಷ್ಟೆ. ಅದೇ ಪರಮಾವಧಿಯಲ್ಲ.

ಶ‍್ರೀಕೃಷ್ಣ ಇಷ್ಟಕ್ಕೇ ಬಿಡುವವನಲ್ಲ ಅರ್ಜುನನನ್ನು. ಯೋಗದಲ್ಲಿ ನೆಲೆಸಿ ಯುದ್ಧವನ್ನು ಮಾಡುವುದಕ್ಕೆ ಸಿದ್ಧನಾಗು ಎನ್ನುವನು. ಎಂತಹ ಗಹನವಾದ ತತ್ತ್ವವನ್ನು ಹೇಳುತ್ತಿದ್ದರೂ ಎದುರಿಗೆ ಇರುವ ಮಾಡಬೇಕಾದ ಕರ್ತವ್ಯವನ್ನು ಮರೆತವನಲ್ಲ ಶ‍್ರೀಕೃಷ್ಣ. ಆ ಕೆಲಸವನ್ನು ಮಾಡಲೇ ಬೇಕಾಗಿದೆ. ಆದರೆ ಬೇರೆ ದೃಷ್ಟಿಯಿಂದ ಈಗ ಮಾಡಬಹುದು. ಅದೇ ಕರ್ಮಯೋಗದ ದೃಷ್ಟಿಯಿಂದ, ಯಜ್ಞದ ದೃಷ್ಟಿಯಿಂದ. ಅದನ್ನು ಮಾತ್ರ ಬಿಡುವುದಕ್ಕೆ ಆಗುವುದಿಲ್ಲ.

