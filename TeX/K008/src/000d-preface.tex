
\chapter{ಗೀತಾದರ್ಶನ}

ಸರ್ವ ಹಿಂದೂಗಳಿಗೂ ಪರಮ ಪವಿತ್ರವಾದ ಧರ್ಮಗ್ರಂಥ ಭಗವದ್ಗೀತಾ. ಇಲ್ಲಿ ನಮಗೆ ದೊರೆಯುವುದು ಸತ್ಯದ ಪೂರ್ಣದೃಷ್ಚಿ. ದ್ವೈತ ವಿಶಿಷ್ಟಾದ್ವೈತ ಅದ್ವೈತವೆಂಬ ತತ್ತ್ವಗಳು, ಜ್ಞಾನ ಯೋಗ, ಕರ್ಮಯೋಗ, ರಾಜಯೋಗಗಳೆಂಬ ಯೋಗಮಹಾನದಿಗಳು ಇಲ್ಲಿ ಸಂಗಮವಾಗಿವೆ. ದೇವರನ್ನು ನಿರಾಕಾರವಾಗಿ ಉಪಾಸನೆ ಮಾಡುವವರಿಗೆ ಸ್ಥಳವಿದೆ ಇಲ್ಲಿ. ಅವನನ್ನು ಸಾಕಾರವಾಗಿ ಉಪಾಸನೆ ಮಾಡುವವರಿಗೆ ಸ್ಥಳವಿದೆ ಇಲ್ಲಿ. ಅವನನ್ನು ಯಾವುದೊ ಒಂದು ವ್ಯಕ್ತಿಯಲ್ಲಿ ಒಂದು ಸ್ಥಳದಲ್ಲಿ ನೋಡುವುದರಿಂದ ಹಿಡಿದು, ಸರ್ವಾಂತರ್ಯಾಮಿ ಅವನು, ಅವನಿಲ್ಲದ ಸ್ಥಳವಿಲ್ಲ, ಅವನಿ ಲ್ಲದ ಕಾಲವಿಲ್ಲ ಎಂಬ ಭಾವನೆಯವರೆಗೆ ಎಲ್ಲದಕ್ಕೂ ಇಲ್ಲಿ ಸ್ಥಳವಿದೆ. ಇಲ್ಲಿ ಬರುವ ತತ್ತ್ವ ಆಗಸದಷ್ಚು ವಿಶಾಲವಾಗಿದೆ, ಕಡಲಿನಷ್ಟು ಆಳವಾಗಿದೆ. ಇದು ಯಾರನ್ನೂ ನಿರಾಕರಿಸುವುದಿಲ್ಲ. ಎಲ್ಲರನ್ನೂ ಸ್ವೀಕರಿಸುವುದು. ಯಾರ ಭಾವಕ್ಕೂ ನಷ್ಟವನ್ನುಂಟುಮಾಡದೆ, ಅವರವರು ಇರುವ ಸ್ಥಳದಿಂದ ಮುಂದುವರಿದು ವಿಕಾಸವಾಗಿ, ಸರ್ವಾಂತರ್ಯಾಮಿಯಾದ ಭಗವಂತನೆಡೆಗೆ ಕರೆದುಕೊಂಡು ಹೋಗುವುದಕ್ಕೆ ಇರುವ ಮಾರ್ಗದರ್ಶಕನೇ ಗೀತಾ. ಹಾಗೆ ಮುಂದುವರಿಯುವಾಗ ಇನ್ನೊಬ್ಬರನ್ನು ನಿಕೃಷ್ಟವಾಗಿ ಕಾಣುವುದಾಗಲಿ, ಅವರನ್ನು ಭಂಗಿಸುವುದಾಗಲಿ, ಹಂಗಿಸುವುದಾಗಲಿ ಯಾವುದೂ ಇಲ್ಲ. ಯಾವಾಗಲೂ ಔದಾರ್ಯತೆಯ ಛಾಯೆಯಲ್ಲಿಯೇ ಇಲ್ಲಿ ಪಥಿಕ ಪ್ರಯಾಣ ಮಾಡುತ್ತಿರುವನು.

\section*{ಪ್ರಸ್ಥಾನತ್ರಯ}

ಹಿಂದೂಗಳ ತತ್ತ್ವಸೌಧಕ್ಕೆ ಶಾಸ್ತ್ರದ ಆಧಾರಸ್ತಂಭವಿದೆ. ಆ ಆಧಾರಸ್ತಂಭವೇ ಉಪನಿಷತ್ತು, ಬ್ರಹ್ಮಸೂತ್ರ ಮತ್ತು ಭಗವದ್ಗೀತಾ ಎಂಬ ಪ್ರಸ್ಥಾನತ್ರಯ. ಏನನ್ನು ಹೇಳಬೇಕಾದರೂ ಅವರು ಈ ಪ್ರಸ್ಥಾನತ್ರಯದ ಮೂಲಕ ಹೇಳಬೇಕು. ತಾವು ಹೇಳುವುದಕ್ಕೆ ಇದರಲ್ಲಿ ಪ್ರಮಾಣ ಸಿಕ್ಕಬೇಕು. ಇಲ್ಲಿ ಇಲ್ಲದೇ ಇದ್ದರೆ, ಅವರು ಹೇಳುವುದನ್ನು ಯಾರೂ ಗೌರವಿಸುವುದಿಲ್ಲ, ಪುರಸ್ಕರಿಸುವುದಿಲ್ಲ. ಆದಕಾರಣವೇ, ಈ ಮೂರು ಗ್ರಂಥಗಳ ಮೇಲೆ ಅಷ್ಟೊಂದು ಭಾಷ್ಯಗಳು, ಭಾಷ್ಯಗಳಿಗೆ ಟೀಕೆಗಳು, ಟೀಕೆಗಳಿಗೆ ತಾತ್ಪರ್ಯಗಳು ಹುಟ್ಟಿಕೊಂಡಿವೆ.

ಈ ಪ್ರಸ್ಥಾನತ್ರಯದಲ್ಲಿ ಉಪನಿಷತ್ತುಗಳು ಅತ್ಯಂತ ಪ್ರಾಚೀನವಾದುವುಗಳು. ಇವುಗಳು ವೇದಗಳ ಜ್ಞಾನಕಾಂಡದಲ್ಲಿ ಬರುವುವು. ಇದನ್ನೇ ವೇದಾಂತ ಎಂತಲೂ ಕರೆಯುತ್ತಾರೆ. ಈ ಉಪನಿಷತ್ತುಗಳಲ್ಲಿ ಹತ್ತು ಬಹಳ ಪ್ರಮುಖವಾದವುಗಳು. ಅವೇ ಈಶ, ಕೇನ, ಕಠ, ಪ್ರಶ್ನ, ಮುಂಡಕ, ಮಾಂಡೂಕ್ಯ, ತೈತ್ತಿರೀಯ, ಐತರೇಯ, ಛಾಂದೋಗ್ಯ ಮತ್ತು ಬೃಹದಾರಣ್ಯಕ ಎಂಬುವುಗಳು. ಹಿಂದಿನ ಕಾಲದಲ್ಲಿ ಯಾವುದೋ ಕಾರಣದಿಂದ ಉಪನಿಷತ್ತುಗಳನ್ನು ಎಲ್ಲರೂ ಓದಕೂಡದು, ಕೇವಲ ದ್ವಿಜರು ಮಾತ್ರ ಓದಬಹುದು ಎಂಬ ಭಾವನೆ ಬಂದು ಹೋಗಿತ್ತು. ಇವು ಬಹಳ ಕಷ್ಟ. ಇವನ್ನು ತಿಳಿದುಕೊಳ್ಳಬೇಕಾದರೆ ಸೂಕ್ಷ್ಮ ಬುದ್ಧಿ ಇರಬೇಕು, ಚಿತ್ತ ಶುದ್ಧವಾಗಿರಬೇಕು. ಇಲ್ಲದೇ ಇದ್ದರೆ ಒಂದು ಬರೆದಿದ್ದರೆ ಓದುವವನು ಮತ್ತೊಂದನ್ನು ಗ್ರಹಿಸುತ್ತಾನೆ. ಅದು ತಪ್ಪಿರಬಹುದು, ಅದನ್ನು ಮತ್ತೊಬ್ಬನಿಗೆ ಬೋಧಿಸುತ್ತಾನೆ. ಇದರಿಂದ ಇವನಿಗೆ ಮಾತ್ರ ಹಾನಿಯಲ್ಲ, ಕೇಳುವವರಿಗೆಲ್ಲ ಹಾನಿಯಾದೀತು. ಈ ಕಾರಣದಿಂದಲೇ ಎಲ್ಲರೂ ಓದಕೂಡದೆಂದು ನಿಯಮಿಸಿರಬಹುದೆಂದು ನಾವು ಊಹಿಸಬಹುದು. ಅಂತೂ ಉಪನಿಷತ್ತಿನ ಜ್ಞಾನ ಬಹು ಕಾಲದವರೆಗೆ ಯಾವುದೊ ಒಂದು ಕೋಮಿನವರಿಗೆ ಮಾತ್ರ ಮೀಸಲಾಗಿತ್ತು.

ಅನಂತರ ಬರುವ ಪ್ರಸ್ಥಾನವೇ ಬ್ರಹ್ಮಸೂತ್ರ. ಇದನ್ನು ವ್ಯಾಸರೇ ಕ್ರೋಢೀಕರಿಸಿದರು ಎಂದು ಹೇಳುತ್ತಾರೆ. ಇಲ್ಲಿ ಸೂತ್ರರೂಪದಲ್ಲಿ ಬರುವುದೇ ಉಪನಿಷತ್ತಿನಿಂದ ಆಯ್ದ ಸಂಕ್ಷೇಪವಾದ ವಾಕ್ಯಗಳು. ಇವುಗಳನ್ನು ತಾತ್ತ್ವಿಕ ದೃಷ್ಟಿಯಿಂದ ಪೋಣಿಸಿರುವರು. ಜೀವ, ಜಗತ್ತು, ಈಶ್ವರ, ಕರ್ಮ ಮುಂತಾದ ವಿಷಯಗಳೆಲ್ಲ ಇಲ್ಲಿ ಬರುವುವು. ಆದರೆ ಇದನ್ನು ತಿಳಿದುಕೊಳ್ಳಬೇಕಾದರೆ ಉಪನಿಷತ್ತಿಗಿಂತ ಕಷ್ಟ. ಏಕೆಂದರೆ ಬರುವ ವಾಕ್ಯಗಳು ಬಹಳ ಸಣ್ಣವು. ಆ ವಾಕ್ಯಗಳಿಗೆ ಹಿಂದೆ ಮುಂದೆ ಯಾವುದೂ ಮೊದಲನೇ ಬಾರಿ ಓದುವವನಿಗೆ ಗೊತ್ತಾಗುವುದಿಲ್ಲ. ಇದನ್ನು ತಿಳಿದುಕೊಳ್ಳಬೇಕಾದರೆ ಯಾರಾದರೂ ಭಾಷ್ಯಕಾರರ ಸಹಾಯವಿಲ್ಲದೇ ಇದ್ದರೆ ದುಸ್ಸಾಧ್ಯ. ವಿದ್ವಾಂಸರಿಗೇ ಇದನ್ನು ತಿಳಿದುಕೊಳ್ಳುವುದಕ್ಕೆ ಕಷ್ಟ ಎಂದಮೇಲೆ ಸಾಧಾರಣ ಮನುಷ್ಯರಿಗೆ ಇದು ದಕ್ಕುವ ವಸ್ತುವೇ ಅಲ್ಲ.

ಮೂರನೆಯ ಪ್ರಸ್ಥಾನವೇ ಭಗವದ್ಗೀತಾ. ಇದು ಮಹಾಭಾರತದ ಭೀಷ್ಮಪರ್ವದಲ್ಲಿ ಬರುವುದು. ಹದಿನೆಂಟು ಅಧ್ಯಾಯಗಳನ್ನೊಳಗೊಂಡಿದೆ. ಏಳುನೂರು ಶ್ಲೋಕಗಳಿವೆ. ಮಹಾಭಾರತವನ್ನು ಓದುವುದಕ್ಕೆ ಯಾವ ಅಡ್ಡಿ ಆತಂಕವೂ ಇಲ್ಲ. ಯಾರು ಬೇಕಾದರೂ ಓದಬಹುದು. ಅದನ್ನು ತಿಳಿದುಕೊಳ್ಳುವುದು ಕೂಡ ಅಷ್ಟು ಕಷ್ಟವಲ್ಲ. ಮಹಾಭಾರತವನ್ನು ಪಂಚಮವೇದ ಎನ್ನುವರು. ವೇದಗಳಲ್ಲಿ ಹೇಗೆ ವ್ಯಕ್ತಿ ಸಮಾಜ ದೇಶ ಇವುಗಳಿಗೆ ಬೇಕಾದ ಮುಖ್ಯ ವಿಷಯಗಳಿವೆಯೋ ಹಾಗೆಯೇ ಮಹಾಭಾರತದಲ್ಲಿ ಎಲ್ಲ ಜೀವರಾಶಿಗಳಿಗೂ ಬೇಕಾದ ತತ್ತ್ವವಿದೆ. ಈ ತತ್ತ್ವವನ್ನು ನಿದರ್ಶಿಸುವ ಕಥೆ ಇದೆ. ಈ ತತ್ತ್ವವನ್ನು ಹೇಳುವ ವಿಧಾನ ಕೂಡ ಬಹಳ ಸುಲಭವಾದುದು.

ಭಗವದ್ಗೀತೆಯಲ್ಲಿ ವಿಷಯಗಳನ್ನು ಶ‍್ರೀಕೃಷ್ಣ ಬಹಳ ಸುಲಭವಾಗಿ ಹೇಳುವನು. ಉಪನಿಷತ್ತುಗಳ ಸಾರವನ್ನೇ ಮಾನವಕೋಟಿಯ ಮೇಲೆ ಕರುಣೆ ತಾಳಿ ಎಲ್ಲರೂ ತಿಳಿದುಕೊಳ್ಳುವಂತಹ ತಿಳಿಭಾಷೆಯಲ್ಲಿ ಹೇಳುವನು. ಆದಕಾರಣವೇ ಗೀತೆಯ ಧ್ಯಾನ ಶ್ಲೋಕಗಳಲ್ಲಿ, “ಸರ್ವ ಉಪನಿಷತ್ತುಗಳೇ ಗೋವುಗಳು, ಅವನ್ನು ಕರೆದವನು ಶ‍್ರೀಕೃಷ್ಣ, ಆ ಹಾಲೇ ಗೀತೆ” ಎಂದಿದೆ. ಸ್ತ್ರೀ ವೈಶ್ಯ ಚಂಡಾಲ ರೆಲ್ಲರೂ ಗೀತೆಯ ಅಮೃತವನ್ನು ಪಾನಮಾಡಿ ಧನ್ಯರಾಗಲಿ ಎಂದೇ ಶ‍್ರೀಕೃಷ್ಣ ಇದನ್ನು ಅರ್ಜುನನನ್ನು ನಿಮಿತ್ತವಾಗಿ ಮಾಡಿಕೊಂಡು ಜಗತ್ತಿಗೆ ಸಾರಿರುವನು. ಗೀತೆಯನ್ನು ಓದಿ ತಿಳಿದುಕೊಳ್ಳಬೇಕಾದರೆ ಬುದ್ಧಿಯ ಕಸರತ್ತು ಬೇಕಿಲ್ಲ. ಪ್ರತಿಯೊಬ್ಬನೂ ತನ್ನ ಯೋಗ್ಯತೆಗೆ ತಕ್ಕಂತೆ ಇಲ್ಲಿ ತಿಳಿದುಕೊಳ್ಳಬಲ್ಲ. ದೊಡ್ಡ ಪಾತ್ರೆಯನ್ನು ತೆಗೆದುಕೊಂಡು ಹೋದರೆ ಹೆಚ್ಚು ಸಿಕ್ಕುವುದು. ಸಣ್ಣ ಪಾತ್ರೆ ತೆಗೆದುಕೊಂಡು ಹೋದರೆ ಕಡಿಮೆ ಸಿಕ್ಕುವುದು. ಯಾವ ಪಾತ್ರೆಯನ್ನು ತೆಗೆದುಕೊಳ್ಳದೇ ಹೋದರೂ ಕೈಯಿಂದ ನೀರನ್ನು ತೆಗೆದುಕೊಂಡು ತಾತ್ಕಾಲಿಕವಾಗಿ ಆದರೂ ಬಾಯಾರಿಕೆಯಿಂದ ಪಾರಾಗಬಹುದು. ತನ್ನೆಡೆಗೆ ಬಂದ ಯಾರನ್ನೂ ಗೀತೆ ಹಾಗೇ ಕಳುಹಿಸುವುದಿಲ್ಲ. ಅವರವರಿಗೆ ಏನನ್ನು ತಿಳಿದುಕೊಳ್ಳಲು ಸಾಧ್ಯವೋ, ಎಷ್ಟನ್ನು ಅರಗಿಸಿಕೊಳ್ಳಲು ಸಾಧ್ಯವೋ ಅಷ್ಟನ್ನು ಕೊಟ್ಟು ಕಳುಹಿಸುವುದು.

ಶ‍್ರೀ ಶಂಕರಾಚಾರ್ಯರು ಗೀತೆಗೆ ಭಾಷ್ಯವನ್ನು ಬರೆಯುವಾಗ ಮೊದಲಲ್ಲಿ “ಸಮಸ್ತ ವೇದಾರ್ಥ ಸಾರ ಸಂಗ್ರಹಭೂತಂ” “ತದರ್ಥವಿಜ್ಞಾನೇ ಸಮಸ್ತಪುರುಷಾರ್ಥಸಿದ್ಧಿಃ” ಎಂದು ಹೇಳುವರು. ಭಗವ ದ್ಗೀತೆಯಲ್ಲಿ ಸಕಲ ವೇದಗಳ ಅರ್ಥದ ಸಾರಸಂಗ್ರಹವಿದೆ. ಇದನ್ನು ತಿಳಿದುಕೊಂಡವನಿಗೆ ಧರ್ಮ, ಅರ್ಥ, ಕಾಮ, ಮೋಕ್ಷಗಳೆಲ್ಲ ಲಭಿಸುವುವು. ಇಹಪರ ಎರಡಕ್ಕೂ ಒಳ್ಳೆಯದನ್ನು ಮಾಡಬಲ್ಲ ಗ್ರಂಥ ಇದು.

ಗೀತೆಯ ಪ್ರತಿಯೊಂದು ಅಧ್ಯಾಯವನ್ನು ಒಂದು ಉಪನಿಷತ್ತಿಗೆ ಸಮ ಎಂದು ಭಾವಿಸಿದರು. ಅದಕ್ಕಾಗಿಯೇ ಪ್ರತಿ ಅಧ್ಯಾಯವೂ “ಇತಿ ಶ‍್ರೀಮದ್ಭಗವದ್ಗೀತಾಸು ಉಪನಿಷತ್ಸು, ಬ್ರಹ್ಮ ವಿದ್ಯಾಯಾಂ, ಯೋಗಶಾಸ್ತ್ರೇ, ಶ‍್ರೀಕೃಷ್ಣಾರ್ಜುನ ಸಂವಾದೇ” ಎಂದು ಕೊನೆಗೊಳ್ಳುವುದು. ಇದು ಭಗವದ್ಗೀತಾ, ಸಾಕ್ಷಾತ್ ಭಗವಂತನೇ ಅರ್ಜುನನನ್ನು ನಿಮಿತ್ತಮಾಡಿಕೊಂಡು ಮಾನವ ಕೋಟಿಗೆ ಹೇಳಿದ್ದು. ಅವನು ಹೇಳುವ ಮಾತಿನಲ್ಲಿ ಒಂದು ಅಪೂರ್ವ ಶಕ್ತಿ ಇರುವುದು. ಅವನ ಸಾಕ್ಷಾತ್ಕಾರವೆಲ್ಲ ಸ್ಪಂದಿಸುತ್ತಿರುವುದು ಅವನಾಡುವ ಮಾತಿನಲ್ಲಿ, ಅಗ್ನಿಕುಂಡದಿಂದ ಸಿಡಿಯುವ ಕಿಡಿಯಂತೆ. ಸತ್ಯದ ಪೂರ್ಣ ಪರಿಚಯ ಮಾಡಿಕೊಂಡು ಶ‍್ರೀಕೃಷ್ಣ ಇಲ್ಲಿ ಹೇಳುತ್ತಿರುವನು. ಅದನ್ನು ಹೇಳುವಾಗಲೂ ಅಷ್ಟು ಸರಳವಾಗಿ ಹೇಳುವನು. ಸ್ವಲ್ಪವೂ ಕಷ್ವವಿಲ್ಲ. ಹೇಳುವುದನ್ನು ಸ್ಪಷ್ಟವಾಗಿ ಹೇಳುವನು. ಅನುಮಾನಕ್ಕೆ ಸ್ವಲ್ಪವೂ ಅವಕಾಶವಿಲ್ಲ. ಸ್ಪಷ್ಟತೆ, ಸರಳತೆಗಳೇ ಮೂಡಿವೆ ಅಲ್ಲಿ. ಅವನಾಡುವ ಮಾತಿನಲ್ಲಿ ಒಂದು ಅಪೂರ್ವ ಶಕ್ತಿ ಇರುವುದರಿಂದ ನಮ್ಮ ಹೃದಯಕ್ಕೆ ನೇರವಾಗಿ ತಾಕುವುದು.

ಇದನ್ನು ಉಪನಿಷತ್ತು ಎನ್ನುವನು. ಹೇಗೆ ವೇದಗಳಲ್ಲಿ ಉಪನಿಷತ್ತುಗಳು ಅದರ ಶಿರವೋ ಎಲ್ಲಾ ತಾತ್ತ್ವಿಕ ಭಾಗಗಳನ್ನೂ ಅದು ಒಳಗೊಂಡಿದೆಯೋ ಹಾಗೆ ತತ್ತ್ವದ ಸಾರವನ್ನೇ ಒಳಗೊಂಡಿದೆ ಭಗವದ್ಗೀತಾ. ಆದರೆ ಅದನ್ನು ಎಲ್ಲರೂ ತಿಳಿದುಕೊಳ್ಳುವ ಶೈಲಿಯಲ್ಲಿ ಹೇಳುವನು. ಇದನ್ನು ಯಾವ ಕೋಮಿಗೂ ಮೀಸಲಾಗಿಟ್ಟಿಲ್ಲ. ಯಾರು ಬೇಕಾದರೂ ಓದಿ ಉದ್ಧಾರವಾಗಬಹುದು. ಭಗವತ್ ಸಾಕ್ಷಾತ್ಕಾರದ ಬಾಗಿಲನ್ನು ಎಲ್ಲರಿಗೂ ತೆರೆದಿರುವನು ಇಲ್ಲಿ.

ಇಲ್ಲಿರುವುದು ಬ್ರಹ್ಮವಿದ್ಯೆ–ಜೀವಜಗತ್ತಿನ ಸ್ವರೂಪವೇನು ಎಂಬ ತಾತ್ತ್ವಿಕ ವಿಷಯಗಳನ್ನೆಲ್ಲ ಹೇಳುವನು. ಜೀವನದಲ್ಲಿ ಅಧ್ಯಾತ್ಮವಿದ್ಯೆಯೇ ಎಲ್ಲಾ ವಿದ್ಯೆಗಳಿಗಿಂತ ಶ್ರೇಷ್ಠ. ಅದನ್ನು ಪ್ರತಿ ಅಧ್ಯಾಯದಲ್ಲಿಯೂ ಸಾರುತ್ತಿರುವನು.

ಇದೊಂದು ಯೋಗಶಾಸ್ತ್ರ. ಬರೀ ತತ್ತ್ವವೇ ಅಲ್ಲ. ಯೋಗ ಎಂದರೆ ಒಂದುಗೂಡಿಸುವುದು. ಜೀವಾತ್ಮನನ್ನು ಪರಮಾತ್ಮನೆಡೆಗೆ ಒಯ್ಯುವುದಕ್ಕೆ ಯೋಗ ಎಂದು ಹೇಳುತ್ತೇವೆ. ಇದು ಅನುಷ್ಠಾನ ಪ್ರಧಾನವಾದುದು. ನಾವು ತಿಳಿದುಕೊಂಡಿರುವುದನ್ನು ಅನುಷ್ಠಾನಕ್ಕೆ ಹೇಗೆ ತರುವುದು ಎಂಬುದನ್ನು ಹೇಳುವುದು. ಇಲ್ಲಿ ಸಿದ್ಧಾಂತವಿದೆ. ಆದರೆ ಅಷ್ಟೇ ಪ್ರಾಮುಖ್ಯತೆ ಸಾಧನೆಗೆ ಇದೆ. ನಾವು ಏನನ್ನು ತಿಳಿದುಕೊಂಡಿರುವೆವೊ ಅದಲ್ಲ ನಮ್ಮನ್ನು ಉದ್ಧರಿಸುವುದು. ನಾವು ಏನನ್ನು ಅನುಷ್ಠಾನಕ್ಕೆ ತಂದಿರು ವೆವೊ ಅದು ಮಾತ್ರ ನಮ್ಮದು. ಶ‍್ರೀರಾಮಕೃಷ್ಣರು ದೋಣಿಯಲ್ಲಿ ಹೋಗುತ್ತಿದ್ದ ಒಬ್ಬ ಪಂಡಿತನ ವಿಷಯವನ್ನು ಉದಹರಿಸುವರು. ಆ ಪಂಡಿತ ದೋಣಿಯಲ್ಲಿ ಕುಳಿತ ನಾವಿಕನಿಗೆ, ನಿನಗೆ ತರ್ಕ ಗೊತ್ತೆ, ನ್ಯಾಯ ಗೊತ್ತೆ, ವೇದಾಂತ ಗೊತ್ತೆ ಎಂದು ಕೇಳುತ್ತಾನೆ. ಪಾಪ ಅವನು ಯಾವುದನ್ನೂ ಓದಿದವನಲ್ಲ. ತನಗೆ ಏನೂ ಗೊತ್ತಿಲ್ಲ ಎನ್ನುತ್ತಾನೆ. ಆಗ ಪಂಡಿತ ನಿನ್ನ ಜೀವನವನ್ನೆಲ್ಲ ಹಾಳುಮಾಡಿಕೊಂಡೆಯಲ್ಲ ಎಂದು ಹೇಳುತ್ತಾನೆ. ಆಗ ಒಂದು ಬಿರುಗಾಳಿ ಎದ್ದಿತು. ದೊಡ್ಡ ದೊಡ್ಡ ಅಲೆಗಳು ಏಳಲು ಮೊದಲಾದುವು. ದೋಣಿ ಅಲ್ಲೋಲಕಲ್ಲೋಲ\-ವಾಯಿತು. ಇನ್ನೇನು ಮುಳುಗುವ ಸ್ಥಿತಿಗೆ ಬಂದಿತು. ದೋಣಿಯವನು ಪಂಡಿತನಿಗೆ, ನಿಮಗೆ ಈಜು ಬರುವುದೇ ಎಂದು ಕೇಳಿದ. ಪಂಡಿತ ಇಲ್ಲ ಎಂದ. ಆಗ ದೋಣಿಯವನು, ನನಗೆ ನ್ಯಾಯ ತರ್ಕ ವೇದಾಂತ ಗೊತ್ತಿಲ್ಲ ಆದರೆ ಈಜು ಬರುವುದು ಎಂದ. ಹಾಗೆ ಜೀವನದಲ್ಲಿ ನಮ್ಮನ್ನು ಭವಸಾಗರದಿಂದ ಪಾರು ಮಾಡುವುದು, ಎಷ್ಟನ್ನು ನಾವು ಅನುಷ್ಠಾನಕ್ಕೆ ತಂದಿರುವೆವೋ ಅದು ಮಾತ್ರ. ಆದಕಾರಣವೇ ಗೀತಾ ಎಲ್ಲಕ್ಕಿಂತ ಹೆಚ್ಚಾಗಿ ಯೋಗಶಾಸ್ತ್ರ, ಅನುಷ್ಠಾನಪ್ರಧಾನವಾದ ಶಾಸ್ತ್ರ.

ಇಲ್ಲಿ ವಿಷಯವನ್ನು ಹೇಳುವ ರೀತಿ ಸಂವಾದರೂಪವಾಗಿರುವುದು. ಶಿಷ್ಯ ಗುರುವಿನ ಮುಂದೆ ತನ್ನ ಸಂದೇಹವನ್ನು ಇಡುವನು. ಗುರು ಅದನ್ನು ಬಿಡಿಸುವನು. ಆದಕಾರಣ ಸಂಬಂಧ ನಿಕಟವಾಗಿದೆ. ಇಲ್ಲಿ ಬಹಿರಂಗ ಸಭೆಯಲ್ಲಿ ಕೊಡುವ ಉಪನ್ಯಾಸದಂತೆ ಇಲ್ಲ. ಒಂದು ಸ್ಥಳದಲ್ಲಿ ಕುಳಿತುಕೊಂಡು ಶಿಷ್ಯ ಗುರುವನ್ನು ಪ್ರಶ್ನಿಸುತ್ತಿರುವನು, ಗುರು ಸಮಾಧಾನದಿಂದ ಅದಕ್ಕೆ ಉತ್ತರಗಳನ್ನು ಕೊಡು ತ್ತಿರುವನು. ಇದು ಸಹಜವಾದ ಆತ್ಮೀಯವಾದ ಸಂಬಂಧ. ಶಿಷ್ಯ ಗುರುವಿನ ಬಳಿ ತಾನಿರುವಾಗ ಯಾವ ಪ್ರಶ್ನೆಯನ್ನಾದರೂ ಕೇಳುವುದಕ್ಕೆ ಹಿಂದೆಗೆಯುವುದಿಲ್ಲ. ಗುರುವಾದರೂ ಶಿಷ್ಯನ ಸಂದೇಹವನ್ನು ಯಾವ ಕಾಲಾವಸರವೂ ಇಲ್ಲದೆ, ಆಮೂಲಾಗ್ರವಾಗಿ ಹೋಗಲಾಡಿಸಲು ಯತ್ನಿಸುವನು. ಇಲ್ಲಿ ಅರ್ಜುನ ಕೇಳುವ ಪ್ರಶ್ನೆಗಳು ಅರ್ಜುನನಿಗೆ ಮಾತ್ರ ಸಂಬಂಧಪಟ್ಟದ್ದಲ್ಲ. ಆತ ನಮ್ಮೆಲ್ಲರ ಪ್ರತಿನಿಧಿಯಂತೆ ಇರುವನು. ನಮ್ಮ ಮನಸ್ಸಿನಲ್ಲಿ ಯಾವ ಪ್ರಶ್ನೆಗಳು ಉದಿಸಬಹುದೋ ಅವುಗಳ ನ್ನೆಲ್ಲ ನಮಗಾಗಿ ಅವನು ಕೇಳುವಂತಿದೆ. ಇಂತಹ ಒಂದು ಸಹಜವಾದ ವಾತಾವರಣದಲ್ಲಿದೆ ಆಧ್ಯಾತ್ಮಿಕ ಗ್ರಂಥ ಗೀತಾ.


\section*{ಕೆಲವು ಆಕ್ಷೇಪಣೆಗಳು}

ಹಿಂದಿನಿಂದ ಭಗವದ್ಗೀತೆ ಮಹಾಭಾರತದಲ್ಲಿ ಅಂತರ್ಗತವಾಗಿ ಬಂದಿದೆ. ಆದರೂ ಕೆಲವು ವಿಮರ್ಶಕರು ಗೀತೆಯನ್ನು ಯಾರೋ ಅನಂತರ ಬರೆದು ಮಹಾಭಾರತದಲ್ಲಿ ಸೇರಿಸಿರುವರು ಎನ್ನವರು. ಏಕೆಂದರೆ ನಮ್ಮಲ್ಲಿ ಇತಿಹಾಸ ಪುರಾಣಗಳು ಮರದಂತೆ ಕಾಲ ಕಾಲಕ್ಕೆ ಬೆಳೆಯುತ್ತಾ ಹೋಗುವುವು. ಮೊದಲು ಜಯ ಎಂಬ ಕಾವ್ಯ ಸುಮಾರು ಹತ್ತುಸಾವಿರ ಶ್ಲೋಕಗಳನ್ನೊಳ ಗೊಂಡಿತ್ತು. ಅನಂತರ ಇಪ್ಪತ್ತುನಾಲ್ಕುಸಾವಿರ ಶ್ಲೋಕದ ಭಾರತವಾಯಿತು. ಕೊನೆಗೆ ಒಂದು ಲಕ್ಷ ಶ್ಲೋಕಗಳ ಮಹಾಭಾರತವಾಯಿತು. ಸಣ್ಣ ಜಯ ಎಂಬ ಗ್ರಂಥ ಕ್ರಮೇಣ ಇತರ ಕಥೆಗಳನ್ನೆಲ್ಲ ಒಳಗೊಂಡು ಮಹಾಭಾರತವೆಂಬ ದೊಡ್ಡ ಗ್ರಂಥವಾಯಿತು. ಹೇಗೆ ನದಿ ಮೂಲದಲ್ಲಿ ಸಣ್ಣಾದಾಗಿ ದ್ದರೂ ಹರಿಯುತ್ತಿರುವಾಗ ಇತರ ಉಪನದಿಗಳನ್ನು ಸೇರಿಸಿಕೊಂಡು ದೊಡ್ಡ ನದಿಯಾಗುವುದೋ ಹಾಗೆಯೇ ಜಯ ಎಂಬ ಮೂಲಗ್ರಂಥ ಹಲವು ಕಥೆಗಳು ಮತ್ತು ತತ್ತ್ವಗಳು ಇವುಗಳನ್ನೆಲ್ಲ ಸೇರಿಸಿಕೊಂಡು ಈಗ ನಮಗೆ ಕಾಣುವ ಮಹಾಭಾರತವಾಗಿದೆ. ಈಗಿರುವುದೆಲ್ಲ ಒಂದೇ ಕಾಲದಲ್ಲಿ ಹುಟ್ಟಿದ್ದಲ್ಲ. ಅನಂತರ ಸೇರಿದ್ದು ಇರಬಹುದು. ಹಾಗೆ ಅನಂತರ ಮಹಾಭಾರತವೆಂಬ ನದಿಗೆ ಸಂಗಮವಾದ ಉಪನದಿ ಭಗವದ್ಗೀತೆ ಎಂಬುದು ಇರಬಹುದು. ಅವರು ಅದನ್ನು ಸಮರ್ಥನೆ ಮಾಡುವುದಕ್ಕೆ ಕೆಲವು ಕಾರಣಗಳನ್ನು ಕೊಡುವರು. ಅದೇ ಯುದ್ಧಕ್ಕೆ ಸಿದ್ಧನಾಗಿ ಬಂದಿರುವ ಅರ್ಜುನನಿಗೆ ಆಗಲೇ ಗೊತ್ತಿತ್ತು, ಯಾರ ಮೇಲೆ ಯುದ್ಧಮಾಡಬೇಕಾಗಿದೆ ಎಂಬುದು. ಅವರನ್ನು ಯುದ್ಧರಂಗದಲ್ಲಿ ನೋಡಿದಾಗ ವೈರಾಗ್ಯ ಬರುವುದು ಸೋಜಿಗವಾಗಿ ಕಾಣುವುದು. ಅದಕ್ಕಿಂತ ಸೋಜಿಗವಾಗಿ ಕಾಣುವುದೇ ಶ‍್ರೀಕೃಷ್ಣ ನಿಧಾನವಾಗಿ ಏಳುನೂರು ಶ್ಲೋಕಗಳಲ್ಲಿ ಗೀತೆಯ ಬೋಧನೆಯನ್ನು ಮಾಡುತ್ತಿರುವುದು. ಇದು ಅನೌಚಿತ್ಯವಾಗಿ ಕಾಣುವುದು. ಇನ್ನು ಸ್ವಲ್ಪಹೊತ್ತಿಗೆ ಯುದ್ಧ ಶುರುವಾಗಬೇಕಾಗಿದೆ. ಅಂತಹ ಸಮಯದಲ್ಲಿ ಇಂತಹ ದೀರ್ಘವಾದ ಉಪನ್ಯಾಸವನ್ನು ಕೊಡುವುದ ಕ್ಕಾದರೂ ಸಮಯವೆಲ್ಲಿದೆ?

ಇದಕ್ಕೆ ಪರಿಹಾರ ನಾವು ಈ ರೀತಿ ಕೊಡಬಹುದು. ಅರ್ಜುನನಿಗೆ ಮೊದಲೇ ಗೊತ್ತಿತ್ತು ಯಾರಮೇಲೆ ಯುದ್ಧಮಾಡಬೇಕಾಗಿದೆ, ಯಾರುಯಾರನ್ನು ಕೊಲ್ಲಬೇಕಾಗಿದೆ ಎಂಬುದು.\break ಮುಂಚೆ ಗೊತ್ತಿರುವುದು ಬೇರೆ. ಆ ಕಾರ್ಯವನ್ನು ಮಾಡಲು ಹೋಗುವಾಗ ನಮ್ಮ ಮನಸ್ಸಿನಲ್ಲಿ ಏಳುವ ಭಾವನೆಗಳೇ ಬೇರೆ. ಎಷ್ಟೋ ಜನ ಜೀವನದಲ್ಲಿ ರೋಸಿದಾಗ ಆತ್ಮಹತ್ಯವನ್ನು ಮಾಡಿಕೊಳ್ಳ ಬೇಕೆಂದು ಕೆರೆಗೊ ಬಾವಿಗೊ ಬೀಳುವುದಕ್ಕೆ ಹೋಗುವರು. ಹಾಗೆ ನೀರಿಗೆ ಇಳಿಯುತ್ತಿರುವಾಗ, ಬೀಳುತ್ತಿರುವಾಗ ಮನಸ್ಸು ಬದಲಾಯಿಸಿ ಮನೆಗೆ ಹಿಂತಿರುಗಿ ಬಂದಿರುವರು. ಆತ್ಮಹತ್ಯೆ ಮಾಡಿಕೊಂಡವರಿಗಿಂತ ಅದನ್ನು ಮಾಡಿಕೊಳ್ಳಲಾರದೆ ಹಿಂತಿರುಗಿ ಬಂದವರೇ ಜಾಸ್ತಿ. ಅನೇಕರು ಆ ಗುಂಪಿಗೆ ಸೇರಿದವರು. ಒಂದು ಶಪಥವನ್ನು ಮಾಡುತ್ತೇವೆ. ಆದರೆ ನಾವು ಕಾರ್ಯಗತಮಾಡುವುದರಲ್ಲಿ ಉದ್ಯುಕ್ತರಾಗಿರುವಾಗ ನಾವು ಮುಂಚೆ ಊಹಿಸುವುದಕ್ಕೂ ಆಗದ ಭಾವನೆಗಳು ಮೇಲೆದ್ದು ನಮ್ಮನ್ನು ಆ ಕಾರ್ಯದಿಂದ ವಿಮುಖರನ್ನಾಗಿ ಮಾಡುವುವು. ಅದರಂತೆಯೇ ಅರ್ಜುನ ಸೇಡನ್ನು ತೀರಿಸಿ ಕೊಳ್ಳಲು ಬಂದಿರುವನು. ಆಗ ಅರ್ಜುನನ ಮನಸ್ಸಿನಲ್ಲಿ ಅನುಕಂಪವೇಳುವುದು. ತಾನು ಕೊಲ್ಲ ಬೇಕಾದ ಗುರುಹಿರಿಯರನ್ನು ಕಣ್ಣಾರೆ ನೋಡಿದಾಗ ಅವರಮೇಲೆ ಅನುಕಂಪ ಬರುವುದು ಸೋಜಿಗವಲ್ಲ. ಇದು ಮಾನವಸಹಜವಾಗಿರುವ ಭಾವನೆ.

ವೈರಾಗ್ಯ ಇಂತಹ ಸಮಯದಲ್ಲೆ ಬರಬೇಕು, ಇಂತಹ ಸಮಯದಲ್ಲಿ ಬರಬಾರದು ಎಂಬ ನಿಯಮವಿಲ್ಲ. ಒಬ್ಬೊಬ್ಬ ಮನುಷ್ಯನಿಗೆ ಜೀವನದಲ್ಲಿ ಯಾವು ಯಾವುದೋ ದೃಶ್ಯವನ್ನೊ, ಮಾತನ್ನೊ, ಸನ್ನಿವೇಶವನ್ನೊ ನೋಡಿದಾಗ ವೈರಾಗ್ಯ ಬರುವುದು. ಅವನ ಸುಪ್ತ ಸಂಸ್ಕಾರವನ್ನು ಹೊಡೆದು ಮೇಲೆಬ್ಬಿಸಲು ಯಾವುದೊ ಒಂದು ಕಿಡಿ ಸಾಕು. ಒಬ್ಬೊಬ್ಬ ಮನುಷ್ಯನ ಜೀವನದಲ್ಲಿಯೂ ಆ ಕಿಡಿ ಒಂದೊಂದು ವಿಧವಾಗಿರುವುದು. ಸಾಧಾರಣವಾಗಿ ವೈರಾಗ್ಯ ಬರುವುದು ಈ ಪ್ರಪಂಚದ ಕ್ಷಣಿಕತೆ ಗೊತ್ತಾದಾಗ. ಮೃತ್ಯು, ಗೋಳು, ದುಃಖ ಇವುಗಳನ್ನು ನೋಡಿದಾಗ. ಅಶೋಕ ಕಳಿಂಗ ರಾಜ್ಯದ ಮೇಲೆ ಯುದ್ಧಮಾಡಿದ ಮೇಲೆ ಅಲ್ಲಿ ಮಡಿದ, ಗಾಯಗೊಂಡ ಲಕ್ಷಾಂತರ ಜನರನ್ನು ನೋಡಿದಾಗ ಇನ್ನು ಮೇಲೆ ಯುದ್ಧವನ್ನು ಮಾಡುವುದಿಲ್ಲವೆಂದು ಪ್ರತಿಜ್ಞೆಯನ್ನು ಮಾಡಿದನು. ಅರ್ಜುನ ಈಗ ಇಂತಹ ಒಂದು ಸನ್ನಿವೇಶದ ಎದುರಿಗೆ ಇರುವನು. ಗುರುಹಿರಿಯರನ್ನು ಕೊಲ್ಲಬೇಕು. ಹಾಗೆ ಕೊಲ್ಲುವಾಗ ಇವನ ಕಡೆಯವರೂ ಬೇಕಾದಷ್ಟು ಮಂದಿ ಸಾಯುವರು. ಕೊನೆಗೆ ಕುರುಕ್ಷೇತ್ರ ಹೆಣಗಳ ರಾಶಿಯಿಂದ ತುಂಬುವುದು. ದೇಶವೆಲ್ಲ ತಬ್ಬಲಿಯರ, ವಿಧವೆಯರ, ಮಕ್ಕಳನ್ನು ಕಳೆದುಕೊಂಡ ತಾಯಿತಂದೆಯರ ಗೋಳಿನಿಂದ ತುಂಬಿ ತುಳುಕಾಡುವುದು. ದಾರಿಯಲ್ಲಿ ಹೋಗುತ್ತಿರುವಾಗ ಒಂದು ಹೆಣವನ್ನು ನೋಡಿದರೇ ನಮಗೆ ಅನೇಕ ವೇಳೆ ನಮ್ಮ ಬಾಳೆ ಹೀಗೆ ಪರ್ಯವಸಾನವಾಗುವುದು ಎಂಬ ವೈರಾಗ್ಯದ ಉದ್ಗಾರ ಬರುವುದು. ಹೀಗಿರುವಾಗ ಯುದ್ಧವಾದ ಮೇಲೆ ಕುರುಕ್ಷೇತ್ರ ಸ್ಮಶಾನಭೂಮಿಯಾಗುವುದನ್ನು ಊಹಿಸಿಕೊಂಡರೆ ವೈರಾಗ್ಯ ಬರುವುದು ಸ್ವಾಭಾವಿಕ. ಹಾಗೆ ಬರದೆ ಇರುವುದು ಅಸ್ವಾಭಾವಿಕ ಎಂದು ಭಾವಿಸಬೇಕಾಗುವುದು.

ಶ‍್ರೀಕೃಷ್ಣ ಅಷ್ಟು ನಿಧಾನವಾಗಿ ಉತ್ತರ ಹೇಳುವುದಕ್ಕೆ ಸಮಯವೆಲ್ಲಿತ್ತು ಎಂಬುದಕ್ಕೆ ಸಮಾಧಾನ ಇದು. ಶ‍್ರೀಕೃಷ್ಣ ಅರ್ಜುನನಿಗೆ ಗೀತೆಯಲ್ಲಿ ಬರುವ ಶ್ಲೋಕಗಳ ಮೂಲಕವೇ ಉತ್ತರ ಕೊಟ್ಟನು ಎಂದು ಭಾವಿಸಬೇಕಾಗಿಲ್ಲ. ಶ‍್ರೀಕೃಷ್ಣ ಕೆಲವು ಮಾತಿನಲ್ಲಿ ಅರ್ಜುನನ ಸಂದೇಹಗಳನ್ನು ನಿವಾರಣೆ ಮಾಡಿರಬಹುದು. ವ್ಯಾಸರು ನಂತರ ಅದನ್ನು ಬರೆಯುವಾಗ ಮಾನವರಿಗೆಲ್ಲ ಇದು ಚೆನ್ನಾಗಿ ಅರ್ಥವಾಗಲಿ ಎಂದು ವಿವರವಾಗಿ ಬರೆದಿರುವರು. ಈ ವಿಷಯದಲ್ಲಿ ಗೀತೆಗೆ ಟೀಕೆಯನ್ನು ಬರೆದ ಶ‍್ರೀಧರಸ್ವಾಮಿಗಳು ತಮ್ಮ ಪ್ರಸ್ತಾವನೆಯಲ್ಲಿ ತುಂಬಾ ಚೆನ್ನಾಗಿ ಹೀಗೆ ಹೇಳುವರು: “ಭಗವಂತನು ಉಪದೇಶಿಸಿದ ವಿಷಯವನ್ನೇ ಶ‍್ರೀಕೃಷ್ಣದ್ವೈಪಾಯನರು (ವ್ಯಾಸರು)ಏಳುನೂರು ಶ್ಲೋಕಗಳಲ್ಲಿ ಬರೆದರು... ಸಂಬಂಧವನ್ನು ಸೂಚಿಸುವುದಕ್ಕೋಸ್ಕರ ಅಲ್ಲಲ್ಲಿ ತಮ್ಮ ಶ್ಲೋಕಗಳನ್ನು ಸೇರಿಸಿದರು.”

ಶೈಲಿಯ ರೀತಿಯಿಂದ ನೋಡಿದರೆ ಮಹಾಭಾರತದಲ್ಲಿ ಮತ್ತು ಗೀತೆಯಲ್ಲಿ ಒಂದೇ ಕವಿಯ ಕೈವಾಡವನ್ನು ನೋಡಬಹುದು ಎಂಬುದು ಬಾಲಗಂಗಾಧರ ತಿಲಕರ ಅಭಿಪ್ರಾಯ. ಗೀತೆಯನ್ನು ಬಿಟ್ಟರೆ ಮಹಾಭಾರತಕ್ಕೆ ಯಾವ ತಾತ್ತ್ವಿಕ ಬೆಲೆಯೂ ಇಲ್ಲ. ಅದೊಂದು ಬರೀ ಕಥೆಯಾಗುವುದು. ಗೀತೆಗಾಗಿ ಮಹಾಭಾರತ ಇರುವುದು. ಮಹಾಭಾರತದಲ್ಲಿ ಬರುವ ಕಥೆಗಳೆಲ್ಲ ಗೀತೆಯನ್ನು ಉಲ್ಲೇಖಿಸುವುದಕ್ಕೆ ಇರುವುವು, ಬರೀ ಕಥೆಗಾಗಿ ಅಲ್ಲ, ಗೀತೆಯಲ್ಲಿ ಬರುವ ಯಾವುದೋ ತತ್ತ್ವವನ್ನು ಉದಹರಿಸುವುದಕ್ಕಾಗಿವೆ. ಆದಕಾರಣವೇ ಗೀತೆಯ ಧ್ಯಾನಶ್ಲೋಕಗಳಲ್ಲಿ “ಯೇನ ತ್ವಯಾ ಭಾರತ– ತೈಲಪೂರ್ಣಃ ಪ್ರಜ್ವಾಲಿತೋ ಜ್ಞಾನಮಯಪ್ರದೀಪಃ” ಎಂದು ಇದೆ. ಇಡೀ ಮಹಾಭಾರತ ಗೀತೆಯಲ್ಲಿ ಬರುವ ತತ್ತ್ವಜ್ಯೋತಿಯನ್ನು ಪೋಷಿಸುವುದಕ್ಕೆ ಇರುವ ತೈಲದಂತೆ ಇದೆ.


\section*{ಶ‍್ರೀಕೃಷ್ಣನ ಚಾರಿತ್ರಿಕತೆ}

ಮಹಾಭಾರತ ಈಗ ನಮಗೆ ದೊರಕುವ ಸ್ಥಿತಿಯನ್ನು ಮುಟ್ಟುವುದಕ್ಕೆ ಸುಮಾರು ಹಲವು ಶತಮಾನಗಳು ಹಿಡಿದಿರಬಹುದು. ಇದನ್ನು ಕೃಷ್ಣದ್ವೈಪಾಯನ ವ್ಯಾಸರು ಸಂಗ್ರಹಿಸಿ ಒಂದು ಗ್ರಂಥರೂಪಕ್ಕೆ ತಂದರು. ಅವರು ಹಾಗೆ ಮಾಡುವುದಕ್ಕೆ ಮುಂಚೆ ಅದರಲ್ಲಿ ಬರುವ ಕಥೆ, ಉಪಕಥೆಗಳೆಲ್ಲ ಜನರ ಬಾಯಲ್ಲಿ ಇದ್ದುವು. ಈ ರೂಪಕ್ಕೆ ಬಂದ ಮಹಾಭಾರತ ಸುಮಾರು ಕ್ರಿಸ್ತಪೂರ್ವ ಐದನೇ ಶತಮಾನಕ್ಕೆ ಮುಂಚೆ ಇದ್ದಿರಬಹುದು. ಇದನ್ನು ಒಂದು ಇತಿಹಾಸವೆಂದು ಹಿಂದಿನವರು ಹೇಳುತ್ತಿದ್ದರು. ಇತಿಹಾಸ ಎಂದರೆ ಒಂದು ಚರಿತ್ರೆ, ನಿಜವಾಗಿ ನಡೆದ ಘಟನೆಗಳಿಂದ ಕೂಡಿದ್ದು. ಮಹಾಭಾರತವೆಲ್ಲಾ ಚರಿತ್ರೆ ಎನ್ನುವುದಕ್ಕೆ ಆಗುವುದಿಲ್ಲ. ಚಾರಿತ್ರಿಕ ದೃಷ್ಟಿಗೆ ವಿರೋಧವಾದ ಹಲವು ಅಂಶಗಳು ನಮಗೆ ಅಲ್ಲಿ ದೊರಕುವುವು. ಚರಿತ್ರೆಗಿಂತ ಹೆಚ್ಚಾಗಿ ಪೌರಾಣಿಕ ಅಂಶಗಳೇ ಜಾಸ್ತಿ. ಆದರೂ ಚರಿತ್ರೆಯ ಒಂದು ಅಸ್ಥಿಪಂಜರ ಅದರ ಹಿಂದೆ ಇದ್ದಿರಬೇಕು. ಕ್ರಮೇಣ ಪೌರಾಣಿಕ ಘಟನಾವಳಿಗಳು ಅದರ ಸುತ್ತಲೂ ಕವಿದಿವೆ.

ಅನೇಕ ವೇಳೆ ಶ‍್ರೀಕೃಷ್ಣ ಎಂಬ ವ್ಯಕ್ತಿ ನಿಜವಾಗಿಯೂ ಚಾರಿತ್ರಿಕ ವ್ಯಕ್ತಿಯೆ ಎಂಬ ಸಂದೇಹ ಬರುವುದು. ಒಂದು ವೇಳೆ ಶ‍್ರೀಕೃಷ್ಣ ಎಂಬುವನು ಇದ್ದರೆ, ಒಬ್ಬ ಶ‍್ರೀಕೃಷ್ಣನು ಇದ್ದನೊ, ಹಲವು ಶ‍್ರೀಕೃಷ್ಣರು ಇದ್ದರೊ ಎಂಬ ಸಮಸ್ಯೆ ಬರುವುದು. ನಮಗೆ ಶ‍್ರೀಕೃಷ್ಣನ ವಿಷಯ ಹರಿವಂಶ, ಮಹಾಭಾರತ ಮತ್ತು ಭಾಗವತಗಳಲ್ಲಿ ದೊರೆಯುವುದು. ಇವೆಲ್ಲವೂ ಒಬ್ಬನೇ ಕೃಷ್ಣನನ್ನು ಕುರಿತದ್ದೇ ಅಥವಾ ಬೇರೆ ಬೇರೆ ಕೃಷ್ಣರನ್ನು ಕುರಿತದ್ದೆ ಎಂಬ ಪ್ರಶ್ನೆ ಬೇರೆ ಏಳುವುದು.

ಶ‍್ರೀಕೃಷ್ಣ ಎಂಬ ವ್ಯಕ್ತಿ ಚಾರಿತ್ರಿಕವಾಗಿ ಹಿಂದಿನ ಕಾಲದಲ್ಲಿ ಇದ್ದಿರಬೇಕು. ಸಾಹಿತ್ಯ ಮತ್ತು ಶಾಸನಗಳ ಮೂಲಕ ಅಂತಹ ವ್ಯಕ್ತಿ ಹಿಂದಿನಿಂದ ಇದ್ದ ಎಂಬುದಕ್ಕೆ ಸಾಕಷ್ಟು ಪ್ರಮಾಣಗಳು ದೊರೆಯುತ್ತವೆ ಮತ್ತು ಅವೆಲ್ಲ ಕ್ರಿಸ್ತಪೂರ್ವಕ್ಕೆ ಸಂಬಂಧಿಸಿದ ಕಾಲ. ಛಾಂದೋಗ್ಯ ಉಪನಿಷತ್ತು\-ಗಳಲ್ಲೆಲ್ಲ ಬಹಳ ಹಳೆಯದು. ಇದು ಬುದ್ಧನಿಗಿಂತ ಮುಂಚೆ ಎಂದು ಹೇಳಬಹುದು. ಬುದ್ಧನ ಕಾಲ ಕ್ರಿಸ್ತಪೂರ್ವ ಆರನೆ ಶತಮಾನ. ಆ ಸಮಯದಲ್ಲೆ ಶ‍್ರೀಕೃಷ್ಣ ಎಂಬ ವ್ಯಕ್ತಿಯ ವಿಚಾರವಾಗಿ ನಾವು ಛಾಂದೋಗ್ಯ ಉಪನಿಷತ್ತಿನಲ್ಲಿ ನೋಡುತ್ತೇವೆ. ಛಾಂದೋಗ್ಯ ಉಪನಿಷತ್ತಿನಲ್ಲಿ \enginline{(III,} ೧೭, ೬) ದೇವಕಿ ಪುತ್ರನಾದ ಕೃಷ್ಣ ಎಂಬುವನು ಗೋರ ಅಂಗೀರಸನ ಶಿಷ್ಯನಾಗಿದ್ದನು ಎಂದಿದೆ. ಪಾಣಿನಿ ತನ್ನ ಸೂತ್ರದಲ್ಲಿ ಶ‍್ರೀಕೃಷ್ಣ ಅರ್ಜುನರು ಪೂಜೆಗೆ ಯೋಗ್ಯರಾದ ವ್ಯಕ್ತಿಗಳು ಎನ್ನುವನು. ಪಾಣಿನಿಯ ಕಾಲವನ್ನು ಕೂಡ ಬುದ್ಧನ ಕಾಲಕ್ಕೆ ಮುಂಚೆ ಎಂದು ವಿದ್ವಾಂಸರು ಊಹಿಸುವರು. ಕ್ರಿಸ್ತಪೂರ್ವ ನಾಲ್ಕನೇ ಶತಮಾನಕ್ಕೆ ಸೇರಿದ ಬೌದ್ಧಗ್ರಂಥವಾದ ‘ನಿದ್ದೀಶ’ ಎಂಬ ಪಾಲೀ ಗ್ರಂಥ ವಾಸುದೇವ ಮತ್ತು ಬಲದೇವನ ಭಕ್ತರ ವಿಷಯವನ್ನು ಹೇಳುವುದು. ಇಂಡಿಯ ದೇಶಕ್ಕೆ ಬಂದ ಗ್ರೀಕರ ಮೆಗಸ್ಥನೀಸ್ ಎಂಬುವನು ಕ್ರಿಸ್ತಪೂರ್ವ ೩೨ಂರಲ್ಲಿ ಶೌರಸೇನಿಯರು ಕೃಷ್ಣನನ್ನು ಪೂಜಿಸುತ್ತಿದ್ದರು ಎನ್ನುವನು. ಕ್ರಿಸ್ತಪೂರ್ವ ೧೮ಂಕ್ಕೆ ಸೇರಿದ್ದ ಬೆಸ್ ನಗರದ ಶಾಸನದಲ್ಲಿ ಹಲಿಯದೋರ ಎಂಬ ಗ್ರೀಕ್ ಭಾಗವತನು ವಾಸುದೇವನನ್ನು ದೇವದೇವ ಎಂದು ಕರೆಯುತ್ತಾನೆ. ಕ್ರಿಸ್ತಪೂರ್ವ ಒಂದನೆಯ ಶತಮಾನಕ್ಕೆ ಸೇರಿದ ನಾನಾ ಗಾಟಿನ ಶಾಸನದಲ್ಲಿ ದೇವರ ಹೆಸರುಗಳನ್ನು ಹೇಳುವ ಮೊದಲನೆ ಶ್ಲೋಕದಲ್ಲಿ ವಾಸುದೇವನ ಹೆಸರು ಬರುವುದು. ಪತಂಜಲಿ ತನ್ನ ಮಹಾಭಾಷ್ಯದಲ್ಲಿ ಪಾಣಿನಿಯ \enginline{(iv,} ೩, ೯೮) ಸೂತ್ರವನ್ನು ವಿವರಿಸುವಾಗ ಭಾಗವತನಾದ ವಾಸುದೇವನ ಹೆಸರು ಬರುವುದು.

ಅಂತೂ ಇವುಗಳನ್ನೆಲ್ಲ ನೋಡಿದರೆ ಶ‍್ರೀಕೃಷ್ಣ ಎಂಬ ಚಾರಿತ್ರಿಕ ವ್ಯಕ್ತಿ ಹಿಂದೆ ಇದ್ದಿರಬೇಕು. ಅವನ ಸುತ್ತಲೂ ಬೇಕಾದಷ್ಟು ಕಲ್ಪನೆ, ಉತ್ಪ್ರೇಕ್ಷೆ ಇವುಗಳೆಲ್ಲ ವ್ಯಾಪಿಸಿಕೊಂಡು ಅವನನ್ನು ಒಬ್ಬ ಅತಿಮಾನವ ವ್ಯಕ್ತಿಯನ್ನಾಗಿ ಮಾಡಿ, ಈಗ ನಮಗೆ ಅದನ್ನು ನಂಬುವುದಕ್ಕೆ ಆಗುವುದಿಲ್ಲ, ಆ ಸ್ಥಿತಿಗೆ ತಂದಿರಬಹುದು. ಆರೋಪ ಮಾಡಿರುವುದನ್ನೆಲ್ಲ ತೆಗೆದರೂ ತೆಗೆದು ಹಾಕುವುದಕ್ಕೆ ಆಗಲಾರದ ಒಂದು ವ್ಯಕ್ತಿ ನಿಲ್ಲುವುದು. ಅವನೇ ಗೀತೆಯನ್ನು ಅರ್ಜುನನಿಗೆ ಬೋಧಿಸಿದ್ದು. ಇಲ್ಲಿ ಕರ್ತವ್ಯಕ್ಕಾಗಿ ಕರ್ತವ್ಯ,\enginline{(ii} ೪೭) ಎಂಬ ಭಾವನೆ ಅತಿನೂತನವಾದುದು. ಇದು ಯಾವುದೋ ಒಂದು ವ್ಯಕ್ತಿ ಮೂಲಕ ಬಂದಿರಬೇಕು. ಅದನ್ನೇ ಇಲ್ಲಿ ಕೃಷ್ಣ ಎಂದು ಕರೆಯುತ್ತಾರೆ.

ಶ‍್ರೀಕೃಷ್ಣನ ವ್ಯಕ್ತಿತ್ವವನ್ನು ನಾವು ಎರಡನೆಯ ದೃಷ್ಟಿಯಿಂದ ನೋಡಬಹುದು. ಅದೇ ಸಾಹಿತ್ಯ ದೃಷ್ಟಿ. ಶ‍್ರೀಕೃಷ್ಣನ ಜೀವನವನ್ನು ರೂಪಿಸುವ ಎರಡು ಶ್ರೇಷ್ಠ ಗ್ರಂಥಗಳಿವೆ. ಅದರಲ್ಲಿ ಮೊದಲನೆ ಯದು ಭಾಗವತ. ಅದು ಶ‍್ರೀಕೃಷ್ಣನ ಬಾಲ್ಯ ಜೀವನ ಮುಂತಾದುವುಗಳನ್ನು ಚೆನ್ನಾಗಿ ರೂಪಿಸುವುದು. ಅನಂತರವೇ ನಮಗೆ ಮಹಾಭಾರತದಲ್ಲಿ ಅವನ ನಂತರದ ಜೀವನ ದೊರಕುವುದು. ಬರೀ ಒಂದು ಸಾಹಿತ್ಯದ ದೃಷ್ಟಿಯಿಂದ ನೋಡಿದರೇನೇ ಅದಕ್ಕೆ ಚಾರಿತ್ರಿಕತೆ ದೊರುಕುವುದಿಲ್ಲ. ಹೇಗೆ ಸಾಹಿತ್ಯ ಒಬ್ಬನ ಕಲ್ಪನೆಯೋ ಅದರಂತೆಯೇ ಶ‍್ರೀಕೃಷ್ಣನೂ ಒಬ್ಬ ಕವಿಯ ಕಲ್ಪನೆ ಇರಬಹುದು ಎನ್ನಬಹುದು. ಆದರೆ ಕವಿ ಆಗಲೇ ಪ್ರಚಾರದಲ್ಲಿ ಇರುವ ಒಂದು ವ್ಯಕ್ತಿಯನ್ನು ತೆಗೆದುಕೊಂಡು ಅದರ ಮೇಲೆ ಇರುವುದು ಇಲ್ಲದಿರುವುದು ಎಲ್ಲವನ್ನು ಸೇರಿಸಿ ಒಂದು ಕಾವ್ಯವನ್ನು ನೇಯುವನು. ಅಂತಹ ಪ್ರಚಾರದಲ್ಲಿದ್ದ ಶ‍್ರೀಕೃಷ್ಣ ಎಂಬ ವ್ಯಕ್ತಿ ಹಿಂದೆ ಇದ್ದಿರಬೇಕು ಎನ್ನುವುದನ್ನು ಊಹಿಸಬೇಕಾಗಿದೆ. ಶಿಲ್ಪಿ ಒಂದು ವಿಗ್ರಹ ಕೊರೆಯುತ್ತಾನೆ. ಆ ವಿಗ್ರಹ ಕಲ್ಪನೆ ಆಗಿರಬಹುದು. ಆದರೆ ಯಾವ ಕಲ್ಲಿನ ಮೇಲೆ ತನ್ನ ಕಲ್ಪನೆಯನ್ನು ನೇತು ಹಾಕುವನೊ ಅದು ಕಾಲ್ಪನಿಕವಾಗಲಾರದು. ಅಷ್ಟನ್ನಾದರೂ ನಾವು ಸಾಹಿತ್ಯ ದೃಷ್ಟಿಯಿಂದ ಒಪ್ಪಿಕೊಳ್ಳಬೇಕಾಗುವುದು.

ಮೂರನೆಯ ದೃಷ್ಟಿಯೇ ಆಧ್ಯಾತ್ಮಿಕ ದೃಷ್ಟಿ. ಇಲ್ಲಿ ಶ‍್ರೀಕೃಷ್ಣ ಚಾರಿತ್ರಕವಾಗಿದ್ದನೆ ಇಲ್ಲವೇ ಎಂಬುದು ಬೇಕಾಗಿಲ್ಲ. ಕಾವ್ಯದೃಷ್ಟಿಯಿಂದ ಅವನು ಒಬ್ಬ ಕವಿಯ ಕಲ್ಪನೆ ಆಗಿರಬಹುದು. ಅವನು ಏನನ್ನು ಹೇಳುತ್ತಾನೊ, ಅದನ್ನೆಲ್ಲ ಒಬ್ಬ ಕವಿಯೇ ಅವನ ಮೂಲಕ ಹೇಳಬಹುದು ಎಂದು ಭಾವಿಸಬಹುದು. ಆದರೆ ಅಂತಹ ಒಂದು ವ್ಯಕ್ತಿಯನ್ನೇ ಆದರ್ಶವಾಗಿ ಮಾಡಿಕೊಂಡು ಅನೇಕ ಜನ ಸಾಧು-ಸಂತರು, ಭಕ್ತರು ಉದ್ಧಾರವಾಗಿದ್ದಾರೆ; ಈ ಸಂಸಾರದಿಂದ ಪಾರಾಗಿದ್ದಾರೆ, ಈಗಲೂ ಅವನ ಜೀವನ ಮತ್ತು ಸಂದೇಶ ಇತರರಿಗೆ ಉದ್ಧಾರವಾಗಲು ಒಂದು ಸೇತುವೆಯಂತೆ ನಿಂತಿದೆ. ಶ‍್ರೀಕೃಷ್ಣ ಎನ್ನುವಂತಹ ವ್ಯಕ್ತಿಯನ್ನು ಚಿಂತನೆ ಮಾಡಿ, ಧ್ಯಾನ ಮಾಡಿ, ಅವನು ಹೇಳುವ ಸಂದೇಶವನ್ನು ತಮ್ಮ ಜೀವನದಲ್ಲಿ ಅಳವಡಿಸಿಕೊಂಡು ಈ ಮರ್ತ್ಯ ಅಮೃತನಾಗಿದ್ದಾನೆ, ಅಜ್ಞಾನಿ ಜ್ಞಾನಿಯಾಗಿದ್ದಾನೆ, ಪಾಪಿ ಪುಣ್ಯವಂತನಾಗಿದ್ದಾನೆ. ಇದನ್ನಾದರೂ ನಾವು ಒಪ್ಪಲೇಬೇಕು. ಒಂದು ವೈಜ್ಞಾನಿಕ ಪುಸ್ತಕವನ್ನು ಯಾರು ಬರೆದರೋ ಅವನ ಚಾರಿತ್ರಿಕತೆ ನಮಗೆ ಮುಖ್ಯವಲ್ಲ. ಅಲ್ಲಿ ಬರೆದಿರುವುದು ಸತ್ಯವೇ, ಸುಳ್ಳೇ ಎಂಬುದನ್ನು ನಾವು ಪ್ರಯೋಗಶಾಲೆಗೆ ಹೋಗಿ ಪ್ರಯೋಗ ಮಾಡಿದಾಗ ನಮಗೆ ಗೊತ್ತಾಗುವುದು. ಅದು ಸರಿಯಾಗಿದ್ದರೆ, ಆ ಪುಸ್ತಕವನ್ನು ಯಾರಾದರೂ ಬರೆದಿರಲಿ ಚಿಂತೆ ಇಲ್ಲ, ಅದು ಸ್ವೀಕಾರಕ್ಕೆ ಯೋಗ್ಯ, ಆಧ್ಯಾತ್ಮಿಕ ಜೀವನಕ್ಕೆ ತಮ್ಮ ಬಾಳನ್ನು ತೆತ್ತ ಸಾಧುಸಂತರು ಭಕ್ತರು ಅಂತಹ ಪ್ರಯೋಗ ಶಾಲೆ. ಅವರು ಶ‍್ರೀಕೃಷ್ಣನ ವ್ಯಕ್ತಿತ್ವದ ಮೇಲೆ ಧ್ಯಾನಮಾಡಿ, ಜಪಮಾಡಿ, ಅವನನ್ನು ಪ್ರಾರ್ಥಿಸಿ ಅವನನ್ನು ಪ್ರೀತಿಸಿ ಉದ್ಧಾರವಾಗಿ ಹೋಗಿದ್ದಾರೆ. ಹಾಗೆಯೇ ಅವನ ಗೀತಾ ಸಂದೇಶ ಆಧ್ಯಾತ್ಮಿಕ ಜೀವನದಲ್ಲಿ ಒಬ್ಬ ಯಾವ ಮೆಟ್ಟಲಿನಲ್ಲಿ ಇರಲಿ ಅವನನ್ನು ಅಲ್ಲಿಂದ ಪರಮ ಸತ್ಯದೆಡೆಗೆ ಮೆಟ್ಟಲು ಮೆಟ್ಟಲಾಗಿ ಕರೆದುಕೊಂಡು ಹೋಗಿ ಪೂರ್ಣಸತ್ಯದ ಪರಿಚಯ ಮಾಡಿಸಿದೆ. ಈ ವಿಷಯವನ್ನು ಯಾರೂ ಅನುಮಾನಿಸಲಾರರು. ಇದರ ಎದುರಿಗೆ ಚಾರಿತ್ರಿಕ ದೃಷ್ಟಿ, ಸಾಹಿತ್ಯದ ದೃಷ್ಟಿ ಇವುಗಳೆಲ್ಲ ಗೌಣ. ಇಲ್ಲಿ ಬರುವ ಆಧ್ಯಾತ್ಮಿಕ ವಿಷಯ ವೈಜ್ಞಾನಿಕ ನಿಯಮದಷ್ಟೇ ಸತ್ಯ. ಅದನ್ನು ಯಾರಾದರೂ ಹೇಳಿರಲಿ ಚಿಂತೆಯಿಲ್ಲ. ಶ‍್ರೀಕೃಷ್ಣನೆಂಬ ಚಾರಿತ್ರಿಕ ವ್ಯಕ್ತಿಯೇ ಹೇಳಿರಲಿ ಅಥವಾ ಯಾವನೋ ಒಬ್ಬ ಮಹಾ ದ್ರಷ್ಟಾರನಾದ ಕವಿ ಶ‍್ರೀಕೃಷ್ಣ ಎಂಬ ವ್ಯಕ್ತಿಯ ಮೂಲಕ ಹೇಳಿರಲಿ, ಚಿಂತೆಯಿಲ್ಲ. ಇದೊಂದು ಆಧ್ಯಾತ್ಮಿಕ ಸತ್ಯ. ಜೀವನದಲ್ಲಿ ಅನುಷ್ಠಾನ ಮಾಡಲು ಸಾಧ್ಯ. ಹಿಂದೆ ಎಷ್ಟೋ ಜನ ಹೀಗೆ ಮಾಡಿದ್ದಾರೆ, ಈಗ ಮಾಡುತ್ತಿರುವರು, ಮುಂದೆ ಮಾಡುವರು. ಅದನ್ನು ಯಾರು ಹೇಳುತ್ತಾರೆ ಎಂಬ ದೃಷ್ಟಿಯಿಂದ ನೋಡಬೇಕಾಗಿಲ್ಲ. ಹೇಳಿರುವುದು ಸತ್ಯವಾದರೆ, ಅದನ್ನು ಶ‍್ರೀಕೃಷ್ಣನೇ ಹೇಳಿರಲಿ, ಅಥವಾ ಇನ್ನು ಯಾರೊ ಅವನ ಹೆಸರಿನಲ್ಲಿ ಹೇಳಿರಲಿ ಚಿಂತೆ ಇಲ್ಲ. ಅದು ಸ್ವೀಕಾರಕ್ಕೆ ಯೋಗ್ಯ. ಇದು ವೈಜ್ಞಾನಿಕವಾದ ದೃಷ್ಟಿ. ನಾವು ಭಗವದ್ಗೀತೆಯನ್ನು ಈ ದೃಷ್ಟಿಯಿಂದಲೂ ನೋಡಬಹುದು.


\section*{ನಾಟಕೀಯ ಹಿನ್ನೆಲೆ}

ಪ್ರಪಂಚದಲ್ಲಿ ಮತ್ತಾವ ಆಧ್ಯಾತ್ಮಿಕ ಗ್ರಂಥವೂ ಇಷ್ಟು ನಾಟಕೀಯವಾಗಿ ಪ್ರಾರಂಭವಾಗುವುದಿಲ್ಲ. ಗೀತಾ ಸಂದೇಶ ಪ್ರಾರಂಭವಾಗುವುದಕ್ಕಾಗಿಯೇ ಕುರುಕ್ಷೇತ್ರದ ಯುದ್ಧದ ಹಿನ್ನೆಲೆ ರೂಪಿತವಾದಂತೆ ಇದೆ. ಇದೊಂದು ಗುರು ಶಿಷ್ಯರು ಇರುವ ಪುಷ್ಯಾಶ್ರಮವಲ್ಲ, ಗಿರಿಗುಹೆಯಲ್ಲ. ಒಟ್ಟು ಹದಿನೆಂಟು ಅಕ್ಷೋಹಿಣಿ ಸೈನ್ಯದ ವೀರರಿಂದ ಮೊಳಗುತ್ತಿರುವ ಕುರುಕ್ಷೇತ್ರ.

ಪಾಂಡವರು ಬಾಲ್ಯದಿಂದಲೂ ಕೌರವರಿಂದ ಪಡಬಾರದ ಯಾತನೆಯನ್ನು ಪಟ್ಟು, ಜೂಜಾಡಿ ತಮ್ಮ ರಾಜ್ಯವನ್ನು ಕಳೆದುಕೊಂಡು, ಹದಿನಾಲ್ಕು ವರುಷಗಳು ವನವಾಸ ಮಾಡಿ, ಒಂದು ವರುಷ ಅಜ್ಞಾತವಾಸದಲ್ಲಿ ಕಳೆದು, ತಮಗೆ ನ್ಯಾಯವಾಗಿ ಬರಬೇಕಾದ ಅರ್ಧ ರಾಜ್ಯವನ್ನು ಪಡೆಯುವುದಕ್ಕೆ ಶ‍್ರೀಕೃಷ್ಣನನ್ನು ಕಳುಹಿಸಿದರು. ಆದರೆ ಆ ಕೌರವ ನಾಯಕನಾದ ದುರ್ಯೋಧನ ಅರ್ಧ ರಾಜ್ಯವಲ್ಲ, ಐದು ಹಳ್ಳಿಯನ್ನೂ ಕೊಡುವುದಿಲ್ಲ, ಐದು ಹೆಜ್ಜೆಯನ್ನೂ ಕೊಡುವುದಿಲ್ಲ. ಬೇಕಾದರೆ ಯುದ್ಧಭೂಮಿಯಲ್ಲಿ ನಮ್ಮನ್ನು ಸೋಲಿಸಿ ತೆಗೆದುಕೊಳ್ಳಲಿ ಎಂದು ಹೇಳಿ ಕಳುಹಿಸಿದ. ಬೇರೆ ಮಾರ್ಗವಿಲ್ಲದೆ ಪಾಂಡವರು ಯುದ್ಧಕ್ಕೆ ಅಣಿಯಾಗಬೇಕಾಯಿತು. ಅವರು ಏಳು ಅಕ್ಷೋಹಿಣಿ ಸೈನ್ಯವನ್ನು ಸಂಗ್ರಹಿಸಿಕೊಂಡು ಯುದ್ಧಮಾಡುವುದಕ್ಕಾಗಿ ಕುರುಕ್ಷೇತ್ರದಲ್ಲಿ ನೆರೆದಿರುವರು.\break ಶ‍್ರೀಕೃಷ್ಣ ಅರ್ಜುನನ ಸಾರಥಿಯಾಗಿರುವನು. ಅವನು ಕೈದುಗಳನ್ನು ಹಿಡಿಯಲಿಲ್ಲ. ಯುದ್ಧ ಮಾಡುವುದಕ್ಕೆ ಮುಂಚೆ ದುರ್ಯೋಧನನಿಗೆ ಯಾರ್ಯಾರು ಸಹಾಯಕ್ಕೆ ಬಂದಿರುವರೋ ಅವರ\-ನ್ನೆಲ್ಲಾ ನೋಡೋಣ, ರಥವನ್ನು ಎರಡು ಸೇನೆಗಳ ಮಧ್ಯದಲ್ಲಿ ನಿಲ್ಲಿಸು ಎಂದು ಅರ್ಜುನ ಶ‍್ರೀಕೃಷ್ಣನನ್ನು ಕೇಳಿಕೊಳ್ಳುವನು. ಶ‍್ರೀಕೃಷ್ಣ ಹಾಗೆ ಮಾಡಿದ. ಅರ್ಜುನ ನೋಡಿದ ಯಾರ್ಯಾರು ಇರುವರು ಕೌರವರ ಕಡೆ ಎಂಬುದನ್ನು. ಅಜ್ಜ ಭೀಷ್ಮ–ಅವನೇ ಸೇನಾನಿ. ಅವನ ನೇತೃತ್ವದಲ್ಲಿ ಕೌರವರ ಸೇನೆಯೆಲ್ಲಾ ನೆರೆದಿದೆ. ಪಾಂಡುರಾಜ ಕಾಲವಾದ ಮೇಲೆ ಕಾಡಿನಲ್ಲಿ ಅನಾಥರಾಗಿದ್ದ ಪಾಂಡವರನ್ನು ಹಸ್ತಿನಾವತಿಗೆ ಕರೆದುಕೊಂಡು ಬಂದು ಪ್ರೀತಿಯಿಂದ ನೋಡಿಕೊಂಡವನು ಭೀಷ್ಮ. ಕೌರವರನ್ನು ಪ್ರೀತಿಸುತ್ತಿದ್ದಂತೆ ಪಾಂಡವರನ್ನೂ ಪ್ರೀತಿಸುತ್ತಿದ್ದವನು ಅವನು. ಅವನ ತೊಡೆಯ ಮೇಲೆ ಬಾಲ್ಯದಲ್ಲಿ ಆಡಿದವರು, ಅವನ ಹಿತವಚನಗಳನ್ನು ಕೇಳಿದವರು, ಅವನ ಹರಕೆಯ ಬಲದಿಂದ ವೀರಾಧಿವೀರರಾದವರು ಪಾಂಡವರು. ಅಜ್ಜನ ಪ್ರೀತಿಯ ಸವಿಯನ್ನು ನೋಡಿದವನು ಅರ್ಜುನ. ಅವನೇ ಈಗ ಎದುರಿಗೆ ನಿಂತಿರುವನು. ಮುಂಚೆ ಅವನನ್ನು ಕೊಲ್ಲಬೇಕಾಗಿದೆ. ರಣದೇವತೆಗೆ ಪ್ರಥಮ ಬಲಿಯೇ ತಮ್ಮ ಪ್ರೀತಿಯ ಅಜ್ಜ. ಅವನಿಗೆ ಸ್ವಲ್ಪ ದೂರದಲ್ಲಿ ದ್ರೋಣಾಚಾರ್ಯರು ಇರುವರು. ಪಾಂಡವರ ಶಸ್ತ್ರವಿದ್ಯಾ ಗುರುಗಳು ಅವರು. ದ್ರೋಣಾಚಾರ್ಯರಿಗೆ ಅರ್ಜುನನನ್ನು ಕಂಡರೆ ಪ್ರಾಣ. ಅಂತಹ ಪ್ರೀತಿ ಅವನ ಮೇಲೆ. ಶಿಷ್ಯನ ಶ್ರೇಯಸ್ಸಿಗಾಗಿ ತಮ್ಮ ವಿದ್ಯೆಯ ಸರ್ವಸ್ವವನ್ನು ಬೋಧಿಸಿದರು. ಅವನು ಅನುಪಮ ಬಿಲ್ಲುಗಾರನಾದಾಗ, “ಭಲೆ! ಭೇಷ್​!” “ನನ್ನ ಶಿಷ್ಯ ಇವನು” ಎಂದು ತಲೆದೂಗಿದ ಗುರು. ತಮ್ಮ ಮಗ ಅಶ್ವತ್ಥಾಮನನ್ನು ಕೂಡ ಅರ್ಜುನನನ್ನು ಪ್ರೀತಿಸುತ್ತಿದ್ದಷ್ಟು ಪ್ರೀತಿಸುತ್ತಿರಲಿಲ್ಲ. ಅಶ್ವತ್ಥಾಮ ತಮ್ಮ ದೇಹದಿಂದ ಹುಟ್ಟಿದವನು. ಆದರೆ ಅರ್ಜುನನಾದರೋ ದ್ರೋಣಾಚಾರ್ಯರ ಮಾನಸ ಪುತ್ರ. ಈ ಗುರು ಪ್ರೀತಿಯನ್ನು ಸವಿದವನು ಅರ್ಜುನ. ಇಂತಹ ಮಹಾನ್ ಗುರುವಿನ ಅಚ್ಚುಮೆಚ್ಚಿನ ಶಿಷ್ಯನಾಗುವುದು ಎಲ್ಲರ ಪಾಲಿಗೂ ಬರುವ ಅದೃಷ್ಟವಲ್ಲ. ಇದೊಂದು ಅಲಭ್ಯ ಲಾಭ. ಇಂತಹ ಗುರುವನ್ನು ಕೊಲ್ಲಬೇಕು ಅರ್ಜುನನಿಗೆ ಜಯ ಬರಬೇಕಾದರೆ. ಜೊತೆಗೆ ಗುರುಪುತ್ರರನ್ನು\break ಗುರುಗಳ ಇತರ ನೆಂಟರಿಷ್ಟರನ್ನು ಕೊಲ್ಲಬೇಕು. ಕರ್ಣ ಮುಂತಾದ ದ್ವಾಪರ ಯುಗದ ಶ್ರೇಷ್ಠತಮ ಸಾಹಸಿಗಳಲ್ಲಿ ಕೆಲವರು ಕೌರವರ ಕಡೆ ಇರುವರು. ರಣದೇವಿಗೆ ಇವರನ್ನೆಲ್ಲಾ ಬಲಿ ಕೊಡಬೇಕು. ಇಲ್ಲದೇ ಇದ್ದರೆ ಜಯ ಸಿಕ್ಕುವುದಿಲ್ಲ. ಸಮರ ಭೂಮಿಯಲ್ಲಿ ನೆರೆದಿರುವವರಾರೂ ಬೆನ್ನನ್ನು ತೋರಿಸಿ ಓಡಿಹೋಗುವವರಲ್ಲ. ಅವರೆಲ್ಲಾ ಬಂದಿರುವುದು ಮಡಿಯುವುದಕ್ಕೆ, ಇಲ್ಲವೇ ಜಯವನ್ನು ಗಳಿಸುವುದಕ್ಕೆ. ಇಂತಹ ಘೋರ ಯುದ್ಧವನ್ನು ಮಾಡಿಯಾದ ಮೇಲೆ ಉಳಿಯುವುದೇ ಕೋಟ್ಯಂತರ ಪ್ರೇತರಾಶಿಗಳು, ಎರಡೂ ಕಡೆ. ಅಂಗ ಹೀನರ ಆರ್ತನಾದ, ಇವರನ್ನೇ ನಂಬಿದ ವಿಧವೆಯರು, ತಬ್ಬಲಿ ಮಕ್ಕಳು ವೃದ್ಧ ತಂದೆ ತಾಯಿಗಳ ದುಃಖದ ಹಾಹಾಕಾರ ದೇಶವನ್ನೆಲ್ಲಾ ವ್ಯಾಪಿಸುವುದು. ಇದರಿಂದ ಇವರಿಗೆ ದೊರಕುವುದೇನು? ಜಯ, ರಾಜ್ಯಲಾಭ. ಜಯದ ಮಾರಿಗೆ ಇಷ್ಟೊಂದು ಬಲಿ. ಬರುವ ರಾಜ್ಯ ಇಷ್ಟೊಂದು ಜನರ ಗೋಳಿನ ಜ್ವಾಲಾಮುಖಿಯ ಮೇಲೆ ಕಟ್ಟಿದ ನಗರ. ಇನ್ನವರಿಗೆ ಯಾವ ಶಾಂತಿ ಬಂದೀತು? ಸುಖ ಬಂದೀತು?! ಈ ಭಾವನೆಗಳು, ಅರ್ಜುನನ ಹೃದಯವನ್ನೆಲ್ಲಾ ಕಲಕುವುವು. ಕೌರವರನ್ನು ನೋಡುತ್ತಿದ್ದಂತೆ ಮರುಕ ಬರುವುದು, ಕಣ್ಣು ಹನಿಗೂಡುವುದು, ಗಾಂಡೀವ ಜಾರುವುದು, ಮೈ ಸುಡುವುದು, ಬಾಯಿ ಒಣಗುವುದು. ಸುಂಟರ ಗಾಳಿಗೆ ಸಿಕ್ಕಿದ ತರಗೆಲೆಯಂತೆ ಆಗುವುದು ಅರ್ಜುನನ ಮನಸ್ಸು. ಕಿಂಕರ್ತವ್ಯಮೂಢನಾಗುವನು. ಇಂತಹ ಒಂದು ಕೊಲೆ ಮಾಡುವುದು ಯುದ್ಧ ಮಾಡುವುದು ಧರ್ಮವೇ ಎಂದು ಪ್ರಶ್ನಿಸುತ್ತಾನೆ. ಆಗಲೇ ಶ‍್ರೀಕೃಷ್ಣ “ಅರ್ಜುನ, ಆರ್ಯರಿಗೆ ಅಯೋಗ್ಯವೂ ಸ್ವರ್ಗಗತಿಗೆ ವಿರೋಧವೂ ಅಪಕೀರ್ತಿಯನ್ನು ಉಂಟು ಮಾಡುವುದೂ ಆದ ಈ ಮೋಹ ಇಂತಹ ವಿಷಮ ಸಮಯದಲ್ಲಿ ನಿನಗೆ ಹೇಗೆ ಬಂತು?” ಎಂದು ಕೇಳುತ್ತಾನೆ.

“ಅರ್ಜುನಾ, ಷಂಡತನವನ್ನು ಹೊಂದಬೇಡ. ಇದು ನಿನಗೆ ಯೋಗ್ಯವಲ್ಲ. ಎಲೈ ಶತ್ರು\-ತಾಪನನೇ, ತುಚ್ಛವಾದ ಮನಸ್ಸಿನ ದೌರ್ಬಲ್ಯವನ್ನು ಬಿಟ್ಟು ಏಳು.”

ಎಂದು ಗುಡುಗಿನಂತೆ ಮೊಳಗುವ ಗಂಭೀರವಾಣಿಯಿಂದ ಶ‍್ರೀಕೃಷ್ಣನ ಗೀತಾ ಸಂದೇಶ ಮೊದಲಾಗುವುದು. ಜ್ಞಾನ ಭಕ್ತಿ ಕರ್ಮದ ಬಿರುಗಾಳಿಯಿಂದ ತಾತ್ಕಾಲಿಕವಾಗಿ ಆವರಿಸಿದ್ದ ಅಜ್ಞಾನದ ಬೂದಿ ಹಾರಿಹೋಗುವುದು. ಸ್ವರ್ಗ ತಾನಾಗಿ ತೆರೆದ ಬಾಗಿಲಿನಂತಿದೆ ಧರ್ಮಯುದ್ಧ. ಗೆದ್ದರೆ ಇಹಲೋಕದ ಸುಖ, ಮಡಿದರೆ ಪರಲೋಕದ ಸುಖ. ಸ್ವಧರ್ಮಕ್ಕೋಸ್ಕರ ಮಡಿದರೂ ಶ್ರೇಷ್ಠವೇ, ಮುಂತಾದ ವಾಕ್ಯಗಳಿಂದ ಅರ್ಜುನನ ಪೌರುಷದ ಅಗ್ನಿಕುಂಡವನ್ನು ಕಲಕಿ ಪುನಃ ಕಿಡಿಯಾಗುವಂತೆ ಮಾಡುವನು. ಅರ್ಜುನನನ್ನು ನಿಮಿತ್ತ ಮಾತ್ರವಾಗಿ ಮಾಡಿಕೊಂಡು,\break ಅಧರ್ಮದ ಕಳೆಯನ್ನು ಬೇರುಸಹಿತ ನಿರ್ದಾಕ್ಷಿಣ್ಯವಾಗಿ ಕಿತ್ತೊಗೆದು ಶಾಶ್ವತ ಧರ್ಮವನ್ನು ಜಗತ್ತಿಗೆ ಬೋಧಿಸಿದನು. ಇದೇ ಗೀತೆ.


\section*{ವಿಷಾದಯೋಗ}

\vskip -8pt

ಭಗವದ್ಗೀತೆ ಪ್ರಾರಂಭವಾಗುವುದು ಅರ್ಜುನನ ವಿಷಾದದಿಂದ. ಮನುಷ್ಯನ ಜೀವನದಲ್ಲಿ ದೊಡ್ಡ ಒಂದು ವ್ಯಾಕುಲ ಪ್ರಾರಂಭವಾದಾಗಲೇ ಹೊಸ ಅಧ್ಯಾಯ ಪ್ರಾರಂಭವಾಗುವುದು. ಮನಸ್ಸಿನ ನೋವನ್ನೇ ಬೆಲೆಯಾಗಿ ಕೊಡಬೇಕು ಒಂದು ಹೊಸ ಭಾವನೆಯ ಉದಯವನ್ನು ನೋಡಬೇಕಾದರೆ. ಯಾತನೆಯಿಲ್ಲದೆ ವ್ಯಾಕುಲವಿಲ್ಲದೆ ಯಾವ ಒಂದು ಹೊಸ ಬಾಳಾಗಲೀ ಹೊಸ ದೃಷ್ಟಿಯಾಗಲೀ ಪ್ರಾರಂಭವಾಗುವುದಿಲ್ಲ. ವಾಲ್ಮೀಕಿಯ ವಿಷಾದದಿಂದ ರಾಮಾಯಣ ಹುಟ್ಟಿತು. ಬುದ್ಧನ ವಿಷಾದದಿಂದ ಅವನಿಗೆ ನಿರ್ವಾಣ ಸಿಕ್ಕಿತು. ಅರ್ಜುನನ ವಿಷಾದದಿಂದ ಲೋಕಕ್ಕೆ ಭಗವಂತನ ಗೀತೆ ಸಿಕ್ಕಿತು.

ಅರ್ಜುನ ಕುರುಕ್ಷೇತ್ರದಲ್ಲಿ ತಾನು ಮಾಡಬೇಕಾದ ಯುದ್ಧವನ್ನು ಕೇವಲ ತನ್ನ ಲಾಭನಷ್ಟದ ಅಲ್ಪ ದೃಷ್ಟಿಯಿಂದ ನೋಡುತ್ತಿರುವನು. ಅವನು ಅದನ್ನು ಬಿಟ್ಚು ಭೂಮ ದೃಷ್ಟಿಯ ಕಡೆ ಹೋಗಬೇಕಾಗಿದೆ. ಹಾಗೆ ಹೋಗುವಾಗ ಅಲ್ಪ ದೃಷ್ಟಿಯಲ್ಲಿ ಬಿಟ್ಟ ತನ್ನ ಬೇರನ್ನೆಲ್ಲಾ ಕಿತ್ತುಕೊಂಡು ಏಳಬೇಕಾಗುವುದು, ಹೊಸ ದೃಷ್ಟಿಗೆ ಹೊಂದಿಕೊಳ್ಳಬೇಕಾಗುವುದು. ಒಂದು ಸಸಿ ಒಂದು ಕುಂಡದಲ್ಲಿ ಬೆಳೆದಿರುವುದು. ಅಲ್ಲೇ ಅದು ಬೇರು ಬಿಟ್ಟು ನೆಲದಿಂದ ಸಾರವನ್ನು ಹೀರುತ್ತಿದೆ. ಅದನ್ನು ಮತ್ತೂ ವಿಶಾಲವಾಗಿ ಬೆಳೆಸುವುದಕ್ಕೆ ಬೇರೊಂದು ಸ್ಥಳದಲ್ಲಿ ನೆಟ್ಟಾಗ, ಸಸಿ ಮೊದಲು ಬಾಡಿಹೋಗುವುದು. ಎಲೆಗಳೆಲ್ಲಾ ಉದುರಿ ಹೋಗುವುವು. ಕೆಲವುಕಾಲದ ಮೇಲೆಯೇ, ಹೊಸ ನೆಲದಲ್ಲಿ ಬೇರೂರಿ ಆದಮೇಲೆಯೇ ಪುನಃ ಚಿಗುರಲು ಪ್ರಾರಂಭಿಸುವುದು. ಅದರಂತೆಯೇ ಅರ್ಜುನ ಇದುವರೆಗೆ ತನ್ನ ವ್ಯಕ್ತಿಯ ದೃಷ್ಟಿಯಿಂದ ಎಲ್ಲವನ್ನು ಅಳೆಯುತ್ತಿದ್ದ, ತೂಗುತ್ತಿದ್ದ. ಇನ್ನು ಮೇಲೆ ಭೂಮದೃಷ್ಟಿಯಿಂದ ಅದನ್ನು ನೋಡಬೇಕಾಗುವುದು. ಅಲ್ಪವನ್ನು ಭೂಮಕ್ಕೆ ಬಲಿಕೊಡಬೇಕಾಗುವುದು. ಆಗಲೇ ನಾವು ಮೇಲೇಳಬೇಕಾದರೆ. ಅರ್ಜುನ ತನ್ನ ಗುರುಹಿರಿಯರನ್ನು ಕೊಲ್ಲುವುದು, ಅದರಿಂದಾಗುವ ದುಃಖ ಸಂಕಟಗಳು, ಅದರಿಂದ ಬರುವ ಪಾಪ ಪುಣ್ಯಗಳು, ಇವುಗಳನ್ನೆಲ್ಲ ಬಿಟ್ಟು ಧರ್ಮಸಂಸ್ಥಾಪನೆಗೆ ಕ್ಷತ್ರಿಯ ಏನನ್ನು ಮಾಡಬೇಕಾಗಿದೆಯೊ ಆ ದೃಷ್ಟಿಯಿಂದ ನೋಡಬೇಕಾಗಿದೆ. ಈ ಪ್ರಪಂಚದಲ್ಲಿ ಸತ್ಯ ಮತ್ತು ಧರ್ಮ ನಿಲ್ಲಬೇಕು. ಯಾರ್ಯಾರು ಅದರ ಬೆಳವಣಿಗೆಗೆ ಆತಂಕಪ್ರಾಯರಾಗಿರುವರೊ ಅವರು ಗುರುಗಳಾಗಿರಲೀ ಹಿರಿಯರಾಗಿರಲೀ ಬಂಧುಬಾಂಧವರಾಗಿರಲೀ ಎಲ್ಲರನ್ನೂ ಕಳೆಯಂತೆ ಕಿತ್ತುಹಾಕಬೇಕು. ವ್ಯಕ್ತಿ ಇದನ್ನು ತನ್ನ ಲಾಭ ನಷ್ಟದ ದೃಷ್ಟಿಯಿಂದ ನೋಡಕೂಡದು. ಸಮಷ್ಟಿಯ ಹಿತದ ದೃಷ್ಟಿಯಿಂದ ನೋಡಬೇಕು. ಹಿಂದಿನದು ಬಡಪೆಟ್ಟಿಗೆ ನಮ್ಮನ್ನು ಮೇಲೆ ಹೋಗಗೊಡುವು ದಿಲ್ಲ. ಅದು ಹಲವಾರು ಕಾರಣಗಳನ್ನು ತಂದೊಡ್ಡುವುದು. ಅದರಂತೆಯೇ ಅರ್ಜುನ, ಈ ಯುದ್ಧವನ್ನು ಮಾಡಿದರೆ ಪಾಪ ಬರುವುದು, ಇದರಿಂದ ಕುಲಕ್ಷಯವಾಗುವುದು, ಕುಲಧರ್ಮ ನಾಶವಾಗುವುದು, ಕುಲಸ್ತ್ರೀಯರು ಕೆಡುವರು, ವರ್ಣಸಂಕರ ಬರುವುದು, ಪಿತೃಗಳು ದುರ್ಗತಿಗೆ ಬರುವರು, “ಇದಕ್ಕೆಲ್ಲ ಕಾರಣಕರ್ತರಾದ ನಮಗೆ ನರಕವಲ್ಲದೆ ಬೇರಿಲ್ಲ. ಇವರನ್ನು ಕೊಲ್ಲುವುದಕ್ಕಿಂತ ತಾನೇ ಯುದ್ಧರಂಗದಲ್ಲಿ ಮಡಿಯುವುದು ಮೇಲು” ಎನ್ನುತ್ತಾನೆ.

ಆದರೆ ಶ‍್ರೀಕೃಷ್ಣ ಇದನ್ನು ನೋಡುವ ದೃಷ್ಟಿಯೇ ಬೇರೆ. ಈ ಯುದ್ಧದಲ್ಲಿ ಅವನ ಬಂಧು ಬಾಂಧವರೂ ಕೂಡ ಕೌರವರ ಮತ್ತು ಪಾಂಡವರ ಪಕ್ಷದಲ್ಲಿದ್ದಾರೆ. ಅವರೂ ಕೂಡಾ ಈ ಯುದ್ಧಮಾರಿಗೆ ಆಹುತಿಯಾಗುವವರೆ. ಆದರೆ ಅವನು ವ್ಯಕ್ತಿಯ ಸುಖದುಃಖ ದೃಷ್ಟಿಯಿಂದ ನೋಡುವುದಿಲ್ಲ. ಅದು ಧರ್ಮಕ್ಕೂ ಅಧರ್ಮಕ್ಕೂ ಇರುವ ಪ್ರಶ್ನೆಯಾಗಿದೆ. ಇಂತಹ ಸಂದಿಗ್ಧ ಪರಿಸ್ಥಿತಿಯಲ್ಲಿ ಕ್ಷತ್ರಿಯನ ಕರ್ತವ್ಯವೇನು ಎಂಬುದನ್ನು ಶ‍್ರೀಕೃಷ್ಣ ಹೇಳುತ್ತಾನೆ. ಅರ್ಜುನ ಯುದ್ಧವನ್ನು ಮಾಡಲೇಬೇಕಾಗಿದೆ. ತನಗೆ ರಾಜ್ಯ ಬೇಡದೇ ಇದ್ದರೂ ಅವನು ಯುದ್ಧವನ್ನು ಬಿಡುವುದಕ್ಕೆ ಆಗುವುದಿಲ್ಲ. ಬೇಕಾದರೆ ಅವನು ಕರ್ಮದ ಫಲವನ್ನು ತ್ಯಜಿಸಬಹುದೇ ಹೊರತು ಕರ್ಮವನ್ನು ತ್ಯಜಿಸುವುದಕ್ಕೆ ಅವನಿಗೆ ಅಧಿಕಾರವಿಲ್ಲ. ಹಾಗೆ ಅವನ ಕೈಯಲ್ಲಿ ಯುದ್ಧವನ್ನು ಮಾಡಿಸುವುದಕ್ಕೆ ಹಲವಾರು ದೃಷ್ಟಿಕೋಣಗಳಿಂದ ಮಾತನಾಡುತ್ತಾನೆ. ತಾತ್ತ್ವಿಕ ದೃಷ್ಟಿಯಿಂದ ಮಾತನಾಡುತ್ತಾನೆ, ವ್ಯಾವಹಾರಿಕ ದೃಷ್ಟಿಯಿಂದ ಮಾತನಾಡುತ್ತಾನೆ. ಸ್ವಧರ್ಮದ ದೃಷ್ಟಿಯಿಂದ ಮಾತನಾಡುತ್ತಾನೆ. ಕೀರ್ತಿ ದೃಷ್ಟಿಯಿಂದ ಮಾತನಾಡುತ್ತಾನೆ. ಲಾಭದ ದೃಷ್ಟಿಯಿಂದ ಮಾತನಾಡುತ್ತಾನೆ. ಕೊನೆಗೆ ಕರ್ಮಯೋಗದ ದೃಷ್ಟಿಯಿಂದ ಮಾತನಾಡುತ್ತಾನೆ. ಅಂತೂ ಹೇಗಾದರೂ ಅವನು ಕರ್ಮವನ್ನು ಮಾಡಬೇಕಾಗಿದೆ ಎಂಬುದನ್ನು ವಿವರಿಸುತ್ತಾನೆ.

ಅರ್ಜುನನಿಗೆ ವಿಷಾದ ಒಂದು ಪರೀಕ್ಷೆಯಾಯಿತು. ಇದುವರೆಗೆ ಅರ್ಜುನ ಶ‍್ರೀಕೃಷ್ಣನನ್ನು ಸಖ ಎಂದು ಕರೆಯುತ್ತಿದ್ದ. ತನ್ನ ಸಾರಥಿಯನ್ನಾಗಿ ಮಾಡಿಕೊಂಡಿದ್ದ. ಅವನನ್ನು ತನ್ನ ಪರಮ ಆಪ್ತನಂತೆ ಕಾಣುತ್ತಿದ್ದ. ಆದರೆ ಶ‍್ರೀಕೃಷ್ಣನ ಮೇರು ಸದೃಶ ವ್ಯಕ್ತಿತ್ವವೇ ಇವನಿಗೆ ಪರಿಚಯವಿರಲಿಲ್ಲ. ಈ ಕುರುಕ್ಷೇತ್ರದ ಯುದ್ಧರಂಗದಲ್ಲಿ ತಾನೆಷ್ಟು ಕೆಳಗೆ ಇರುವೆನು, ಶ‍್ರೀಕೃಷ್ಣ ಎಷ್ಟು ಮಹಿಮೋನ್ನತ ವ್ಯಕ್ತಿ ಎಂಬುದು ಅರ್ಥವಾಯಿತು. ಜೀವನದಲ್ಲಿ ನಮ್ಮ ಯೋಗ್ಯತೆಯನ್ನು ಅರ್ಥಮಾಡಿಕೊಳ್ಳಬೇಕಾದರೆ ಒಂದು ಅವಕಾಶಬೇಕು. ಒರಗಲ್ಲಿನಮೇಲೆ ತಿಕ್ಕಿ ನೋಡಿದಾಗ ತಾನೇ ಅರ್ಥವಾಗುವುದು ಲೋಹದಲ್ಲಿ ಎಷ್ಟು ಚಿನ್ನವಿದೆ ಎಂಬುದು. ಗಾಳಿಗೆ ತೂರಿದಾಗ ತಾನೇ ಜೊಳ್ಳು ಹೋಗುವುದು, ಕಾಳು ಉಳಿಯುವುದು. ಅರ್ಜುನನ ಜೀವನ ವಿಷಾದದ ಬಿರುಗಾಳಿಗೆ ಸಿಕ್ಕಿ ಹಾರಿಹೋಗುತ್ತಿದೆ. ಶ‍್ರೀಕೃಷ್ಣನಾದರೋ ಆ ಸಮಯದಲ್ಲಿ ಮಂದರ ಪರ್ವತದಂತೆ ನಿಂತಿರುವನು. ಇದನ್ನು ನೋಡಿದಾಗಲೇ ಹೀಗೆ ಹೇಳುತ್ತಾನೆ: “ದೈನ್ಯವೆಂಬ ದೋಷದಿಂದ ಕುಂದಿದ ಸ್ವಭಾವವುಳ್ಳವನಾಗಿ, ಧರ್ಮ ವಿಷಯದಲ್ಲಿ ಮೂಢವಾದ ಮನಸ್ಸುಳ್ಳವನಾಗಿ, ನಿನ್ನನ್ನು ಪ್ರಶ್ನೆ ಮಾಡುತ್ತಿದ್ದೇನೆ. ಯಾವುದು ನನಗೆ ಶ್ರೇಯ ಸ್ಕರವೋ ಅದನ್ನು ನಿಶ್ಚಯಪೂರ್ವಕ ಹೇಳು. ನಾನು ನಿನ್ನ ಶಿಷ್ಯ; ನಿನ್ನನ್ನೇ ಶರಣುಹೊಂದಿದ ನನಗೆ ಬೋಧಿಸು.\enginline{” (ii}, ೭)

ಅರ್ಜುನ ಇನ್ನು ಮೇಲೆ ಶ‍್ರೀಕೃಷ್ಣನನ್ನು ತನ್ನ ಸಖ ಎಂದು ಕರೆಯುವುದಿಲ್ಲ, ‘ನಾನು ನಿನ್ನ ಶಿಷ್ಯ’ ಎಂದು ಹೇಳುತ್ತಾನೆ. ‘ನಾನು ನಿನ್ನಲ್ಲಿ ಶರಣಾಗಿದ್ದೇನೆ’ ಎನ್ನುತ್ತಾನೆ. ‘ನನಗೆ ಶ್ರೇಯಸ್ಸಿನ ದಾರಿಯನ್ನು ತೋರು’ ಎಂದು ಬೇಡುತ್ತಾನೆ. ಯಾವಾಗ ಒಬ್ಬ ತನ್ನ ಶಿಷ್ಯನಾಗುತ್ತಾನೆಯೋ ಆಗಲೇ ಗುರು ಶಿಷ್ಯನ ಜವಾಬ್ದಾರಿಯನ್ನು ತೆಗೆದುಕೊಳ್ಳುವುದು. ಅದಕ್ಕೆ ಮುಂಚೆ ಅಲ್ಲ. ಸಖನಾದ ಅರ್ಜುನನನ್ನು ಶಿಷ್ಯನ ಸ್ಥಾನದಲ್ಲಿ ಕೂರಿಸುವುದು ವಿಷಾದ. ಆಗಲೇ ಭಗವಂತನ ಬೋಧನೆ ಮೊದಲಾಗಬೇಕಾದರೆ. ಅದು ಹೇಗೆ ಪ್ರಾರಂಭವಾಗವುದು? ಸಂಜಯ ಹೇಳುವಂತೆ, “ನಗುತ್ತಿರುವವನಂತೆ ಈ ಮಾತನ್ನು ಹೇಳುತ್ತಾನೆ.” ಮೊದಲು ಕತ್ತಲೆಯನ್ನೆಲ್ಲಾ ಸೀಳುವಂತೆ ಮಿಂಚಿ, ಅಮೃತಮಯವಾದ ಮುಸಲಧಾರೆ ಯಂತಹ ಮಳೆಯನ್ನು ಗೀತೋಪದೇಶದಲ್ಲಿ ಹದಿನೇಳು ಅಧ್ಯಾಯಗಳಲ್ಲಿ ಹರಿಸುವನು.

ಜೀವನದಲ್ಲಿ ಗಹನವಾದ ಸತ್ಯವನ್ನು ತಿಳಿದುಕೊಳ್ಳುವುದಕ್ಕೆ ನಮ್ಮಲ್ಲಿ ವ್ಯಾಕುಲತೆ ಇದ್ದರೆ ಅದನ್ನು ತಿಳಿಸುವವನು ಸಿಕ್ಕುತ್ತಾನೆ. ವ್ಯಾಕುಲತೆ ಹುಟ್ಟಿದಾಗ ಅದನ್ನು ತೃಪ್ತಿಪಡಿಸುವ ಚೈತನ್ಯವೂ ಇವನನ್ನು ಹುಡುಕಿಕೊಂಡು ಬರುವುದು. ಇದೊಂದು ಜೀವನದ ಗಾಢನಿಯಮ. ಏಸುಕ್ರಿಸ್ತ, “ಕೇಳಿ ನಿಮಗದು ದೊರಕುವುದು” ಎನ್ನುತ್ತಾನೆ. ಯಾವಾಗ ಅರ್ಜುನ ಶ‍್ರೀಕೃಷ್ಣನನ್ನು ಕೇಳುತ್ತಾನೋ ತನಗೆ ಜೀವನದಲ್ಲಿ ದಾರಿ ತೋರೆಂದು, ಶ‍್ರೀಕೃಷ್ಣ ಅವನಿಗೆ ಮಾರ್ಗದರ್ಶಕನಾಗುತ್ತಾನೆ. ಇದು ಬರೀ ಅರ್ಜುನನಿಗೆ ಮಾತ್ರ ಸತ್ಯವಲ್ಲ. ಗೀತೆಯಲ್ಲಿ ಬರುವುದು ನಿತ್ಯಸತ್ಯದ ಮಾತು. ಎಲ್ಲಾ ಜೀವರ ಪಾಲಿಗೂ ಇದು ಸತ್ಯ. ನಾವು ಭಗವಂತನಲ್ಲಿ ಶರಣಾದರೆ, ದಾರಿ ತೋರೆಂದು ವ್ಯಾಕುಲತೆಯಿಂದ ಪ್ರಾರ್ಥಿಸಿದರೆ, ಭಗ ವಂತನ ಶಕ್ತಿ ಹತ್ತಿರವೇ ಕಾದು ಕುಳಿತಿದೆ ನಮ್ಮನ್ನು ಮೇಲೆತ್ತಲು.


\section*{ಯಜ್ಞಭಾವನೆ}

ಗೀತೆಯಲ್ಲಿ ಬರುವ ಒಂದು ಅತ್ಯಂತ ಶ್ರೇಷ್ಠಭಾವನೆ ಯಜ್ಞ. ಶ‍್ರೀಕೃಷ್ಣ ಅರ್ಜುನನಿಗೆ ಯಜ್ಞದೃಷ್ಟಿಯಿಂದ ಕರ್ಮವನ್ನು ಮಾಡು, ಆಗ ಅದು ನಿನ್ನನ್ನು ಬಂಧಿಸುವುದಿಲ್ಲ \enginline{(iii}, ೯) ಎಂದು ಹೇಳುತ್ತಾನೆ. ಅದರ ಭಾವನೆಯನ್ನು ವಿಸ್ತರಿಸಿ ಇಡೀ ಬ್ರಹ್ಮಾಂಡವೇ ಹೇಗೆ ಯಜ್ಞದ ಭಾವನೆಯ ಮೇಲೆ ನಿಂತಿದೆ ಎಂಬುದನ್ನು ವಿವರಿಸುವನು.

“ಪ್ರಜಾಪತಿ ಯಜ್ಞಗಳೊಂದಿಗೆ ಪ್ರಜೆಗಳನ್ನು ಸೃಷ್ಟಿಸಿ, ಪ್ರಜೆಗಳಿಗೆ ನೀವು ಯಜ್ಞದಿಂದ ವೃದ್ಧಿಯಾಗಿ. ಅದು ನಿಮ್ಮ ಬಯಕೆಗಳನ್ನು ಈಡೇರಿಸುವ ಕಾಮಧೇನುವಾಗಲಿ\enginline{” (iii,} ೧೦) ಎನ್ನುವನು. ಸೃಷ್ಟಿಯ ಆದಿಯಲ್ಲಿ ಭಗವಂತನು ಜೀವರನ್ನು ಸೃಷ್ಟಿಸುವಾಗಲೇ ಯಜ್ಞವನ್ನು ಸೃಷ್ಟಿಸಿದನು, ಎಂದು ಹೇಳಿದೆ. ನಾವು ಬೆಳೆಯಬೇಕು, ವೃದ್ಧಿಯಾಗಬೇಕು. ಹಾಗೆ ಆಗಬೇಕಾದರೆ ಒಂದು ಮಾರ್ಗವನ್ನು ತೋರುತ್ತಾನೆ. ಆ ಭಾವನೆಯೇ ಯಜ್ಞದ ಭಾವನೆಯನ್ನು ನಮ್ಮ ಮನಸ್ಸಿನಲ್ಲಿಟ್ಟುಕೊಂಡು ನಮ್ಮ ಪಾಲಿನ ಕರ್ತವ್ಯಗಳನ್ನು ಮಾಡುವುದು. ಆಗ ಯಾವ ವ್ಯಕ್ತಿಯಾಗಲೀ, ವ್ಯಕ್ತಿಗಳ ಸಮುದಾಯದ ಸಮಾಜವಾಗಲೀ ಅಭಿವೃದ್ಧಿಯಾಗುವುದು. ಅದನ್ನು ಪ್ರಪಂಚದಲ್ಲಿ ಯಾವುದೂ ನಿರ್ನಾಮ ಮಾಡುವುದಕ್ಕೆ ಆಗುವುದಿಲ್ಲ. ಒಂದು ಜನಾಂಗದ ಶಕ್ತಿ ಕೇಂದ್ರವೇ ಈ ಯಜ್ಞದ ಭಾವನೆಯಲ್ಲಿದೆ. ಎಲ್ಲಿ ಇದು ಮಾಯವಾಗಿದೆಯೋ ಅಲ್ಲಿ ಆ ಜನಾಂಗವನ್ನು ರಕ್ಷಿಸಲು ಬೇಕಾದಷ್ಟು ಸೇನೆ ಇದ್ದರೂ ಅದು ಬಹು ಬೇಗ ನಾಶವಾಗುವುದು. ನಾವು ಯಾವುದನ್ನು ಸನಾತನ ಧರ್ಮ ಎನ್ನುವೆವೊ ಅದರಲ್ಲಿ ಈ ಶಕ್ತಿ ಸುಪ್ತವಾಗಿದೆ. ಅದು ನಮ್ಮ ಆಸೆ ಆಕಾಂಕ್ಷೆಗಳನ್ನು ಈಡೇರಿಸುವ ಕಾಮಧೇನುವಾಗುವುದು.

ಯಜ್ಞದೃಷ್ಟಿ ನಮಗೆ ಧರ್ಮ, ಅರ್ಥ, ಕಾಮ, ಮೋಕ್ಷಗಳನ್ನೆಲ್ಲಾ ಕೊಡಬಲ್ಲದು. ಪ್ರಪಂಚವನ್ನು ಅನುಭವಿಸಬೇಕೆಂಬ ಆಸೆ ಇರುವವನಿಗೆ ಅದನ್ನು ಅನುಭವಿಸಬೇಡ ಎನ್ನುವುದಿಲ್ಲ. ಯಾವ ರೀತಿ ಅನುಭವಿಸಬೇಕೊ ಅದರ ಮರ್ಮವನ್ನು ತಿಳಿದುಕೊಂಡು ಅನುಭವಿಸು ಎನ್ನುವುದು. ಮಾನವನಲ್ಲಿ ಕಾಮಾಸಕ್ತಿ ಮತ್ತು ಅಧಿಕಾರಾಸಕ್ತಿ ಎಂಬು ಎರಡು ಪ್ರಬಲ ವಾಸನೆಗಳಿವೆ. ಅದನ್ನು ನಿಗ್ರಹಿಸು ಎನ್ನುವ ಬದಲು ಅನುಭವಿಸುವ ದೃಷ್ಟಿಯನ್ನು ಬದಲಾಯಿಸು ಎನ್ನುವುದು. ನಮ್ಮ ಕಾಮಾಸಕ್ತಿಯನ್ನು ತೃಪ್ತಿಪಡಿಸಿಕೊಳ್ಳಲು ಯಾರು ಯಾರನ್ನೋ ಕೂಡಿಕೊಂಡು ಮೃಗೀಯ ತೃಷ್ಣೆಗಳನ್ನು ತೃಪ್ತಿಪಡಿಸಿಕೊಂಡು ಹಲವು ರೋಗರುಜಿನಗಳಿಗೆ ತುತ್ತಾಗಿ ಸಮಾಜ ಘಾತುಕನಾಗುವ ಬದಲು ಅಗ್ನಿಸಾಕ್ಷಿಯಾಗಿ ಪುರುಷ ಒಬ್ಬ ಸ್ತ್ರೀಯನ್ನು ವರಿಸಿ ಗೌರವದಿಂದ ಸಂಸಾರವನ್ನು ಸಾಗಿಸುವುದು ಉದ್ಧಾರವಾಗುವ ಹಾದಿ. ಇದರಿಂದ ನಮ್ಮ ಬಯಕೆಗಳು ತೃಪ್ತಿಯಾಗಿ ಕ್ರಮೇಣ ನಮ್ಮ ಸ್ವಾರ್ಥ ಕೂಡ ಕಡಿಮೆಯಾಗಲು ಸಹಾಯವಾಗುವುದು. ಇದರಂತೆಯೇ ಅಧಿಕಾರಲಾಲಸೆ. ಯಾರಲ್ಲಿ ಈ ಬಯಕೆ ಇದೆಯೋ ಅವರು ಬೇಕಾದರೆ ಅದನ್ನು ತೃಪ್ತಿ ಪಡಿಸಿಕೊಳ್ಳಲಿ. ಅಧಿಕಾರವನ್ನು ನ್ಯಾಯವಾಗಿ ಸಂಪಾದಿಸಲಿ. ಅದರಿಂದ ಇತರರಿಗೆ ಒಳ್ಳೆಯದನ್ನು ಮಾಡಲಿ. ಅಧಿಕಾರವನ್ನು ಪಡೆಯುವುದು ಭೋಗಕ್ಕಲ್ಲ. ಅದೊಂದು ದೊಡ್ಡ ಜವಾಬ್ದಾರಿಯೆಂದು ಅರಿತು ಕೆಲಸ ಮಾಡಲಿ. ಆಗ ಅವರು ಮಾಡುವ ಕೆಲಸ ಒಂದು ಯಜ್ಞವಾಗುವುದು. ಅದರಂತೆಯೇ ದ್ರವ್ಯವನ್ನು ಆರ್ಜಿಸಬೇಕೆಂಬ ಆಸೆ ಹಲವರಲ್ಲಿದೆ. ಅದನ್ನು ಯೋಗ್ಯವಾದ ರೀತಿಯಲ್ಲಿ ಸಂಪಾದಿಸಲಿ; ಯೋಗ್ಯವಾದ ರೀತಿಯಲ್ಲಿ ವೆಚ್ಚಮಾಡಲಿ. ಆಗ ಅದೊಂದು ದ್ರವ್ಯಯಜ್ಞವಾಗುವುದು. ನಮ್ಮ ಪಾಲಿಗೆ ಬಂದ ಕರ್ತವ್ಯವನ್ನು ಯಜ್ಞ ದೃಷ್ಟಿಯಿಂದ ಮಾಡಿದರೆ ಸಮಾಜ ಭದ್ರವಾಗುವುದು. ಎಲ್ಲಾ ಧರ್ಮಗಳು ಅಲ್ಲಿ ಹುಲಸಾಗಿ ಬೆಳೆಯಲು ಸಾಧ್ಯವಾಗುವುದು. ಯಜ್ಞದೃಷ್ಟಿಯಿಂದ ಕೆಲಸ ಮಾಡಿದರೆ ಜ್ಞಾನ ಬರುವುದು, ಕೊನೆಗೆ ಅದರಿಂದ ಮುಕ್ತಿಯೂ ಲಭಿಸುವುದು. ಮಾನವನಿಗೆ ಬೇಕಾದುದೆಲ್ಲಾ ಇದರಿಂದ ಲಭಿಸುವುದು.

“ನೀವು ಯಜ್ಞದ ಮೂಲಕ ದೇವರನ್ನು ತೃಪ್ತಿಪಡಿಸಿದರೆ, ಆ ದೇವತೆಗಳು ನಿಮ್ಮನ್ನು ಪೋಷಿಸುವರು. ಪರಸ್ಪರ ಭಾವನೆ ಇದ್ದರೆ ಪರಮ ಶ್ರೇಯಸ್ಸು ಲಭಿಸುವುದು. ಯಜ್ಞದಿಂದ ತೃಪ್ತರಾದ ದೇವತೆಗಳು ನಿಮಗೆ ಇಷ್ಟವಾದ ಭೋಗಗಳನ್ನು ಕರುಣಿಸುವರು. ಅವರಿಂದ ಅನುಗ್ರಹಿಸಲ್ಪಟ್ಟ ಭೋಗವನ್ನು ಅವರಿಗೆ ಕೊಡದೆ ಯಾರು ತಾವೇ ಅನುಭವಿಸುವರೋ ಅವರು ಕಳ್ಳರು.\enginline{” (iii}, ೧೧-, ೧೨)

ನಾವು ದೇವರನ್ನು ಏನು ಕೇಳುತ್ತೇವೆಯೋ ಅವನು ಅದನ್ನು ನಮಗೆ ಕೊಡುವನು. ಆದರೆ ಅವನನ್ನು ಯಜ್ಞದ ಮೂಲಕ ಕೇಳಬೇಕು. ಅವನಿಗೆ ಒಂದು ರೀತಿಯಿಂದ ಕೊಟ್ಟು ಕೇಳಬೇಕು. ಹಾಗೆಯೇ ಕೇಳುವುದಲ್ಲ. ನಮ್ಮಂತಹ ಬಡ ಮಾನವರು ಭಗವಂತನಿಗೆ ಕೊಡುವುದಾದರೂ ಏನಿದೆ ಎಂದು ಭಾವಿಸಬಹುದು. ಆದರೆ ನಮ್ಮಲ್ಲಿರುವ ಯಾವ ಅಲ್ಪವನ್ನು ಕೊಡುವೆವೋ ಅವನು ಅದನ್ನು ಸ್ವೀಕರಿಸುವನು. ಬಡ ವಿದುರ ಕುಡಿಕೆ ಹಾಲನ್ನು ಕೊಡಲಿಲ್ಲವೆ? ಕುಚೇಲನು ಕೊಟ್ಟ ಒಂದು ಹಿಡಿ ಒಣ ಅವಲಕ್ಕಿಯನ್ನು ಅವನು ಸ್ವೀಕರಸಿಲಿಲ್ಲವೆ? ನಾವು ಕೊಡುವಾಗ ನಮ್ಮಲ್ಲಿರುವ ಅಲ್ಪಕ್ಕಾಗಿ ನಾವು ನಾಚಬೇಕಾಗಿಲ್ಲ. ನಾವು ಏನನ್ನು ಕೊಟ್ಟರೂ ಅವನು ಸ್ವೀಕರಿಸುವನು. ಸ್ವೀಕರಿಸಿ ಆದಮೇಲೆ ಅದನ್ನು ಮತ್ತೂ ಹೆಚ್ಚು ಮಾಡಿಕೊಡುವನು. ಹಾಗೆಯೇ ಅವನು ನಮಗೆ ಕೊಡುವಾಗ ಅದನ್ನು ಪುನಃ ಅವನಿಗೆ ಅರ್ಪಿಸಿ ನಾವು ತೆಗೆದುಕೊಳ್ಳಬೇಕು. ಇಲ್ಲದೇ ಇದ್ದರೆ ನಾವು ಕಳ್ಳರಾಗುತ್ತೇವೆ, ಎನ್ನವನು ಶ‍್ರೀಕೃಷ್ಣ. \enginline{(iii,}೧೩) ಯಜ್ಞದ ಮೂಲಕ ಕೇಳಿ, ಯಜ್ಞದ ಮೂಲಕ ಅನುಭವಿಸಿ, ಎನ್ನುವನು.

ಸೃಷ್ಟಿಚಕ್ರ ನಿಂತಿರುವುದೇ ಯಜ್ಞದ ಆಧಾರದ ಮೇಲೆ ಎಂಬುದನ್ನು ಶ‍್ರೀಕೃಷ್ಣ ಸುಂದರವಾಗಿ ವಿವರಿಸುವನು. “ಅನ್ನದಿಂದ ಪ್ರಾಣಿಗಳು ಉತ್ಪನ್ನವಾಗುತ್ತವೆ. ಮಳೆಯಿಂದ ಅನ್ನ ಉಂಟಾಗುವುದು. ಯಜ್ಞದಿಂದ ಮಳೆಯಾಗುವುದು. ಕರ್ಮದಿಂದ ಯಜ್ಞವಾಗುವುದು. ಕರ್ಮ ಬ್ರಹ್ಮದಿಂದಾಗುವುದು. ಬ್ರಹ್ಮ ಅಕ್ಷರದಿಂದ ಉದ್ಭವಿಸುವುದು. ಆದಕಾರಣ ಸರ್ವವ್ಯಾಪಿಯಾದ ಬ್ರಹ್ಮ ಸದಾ ಯಜ್ಞದಲ್ಲಿ ಪ್ರತಿಷ್ಠಿತವಾಗಿರುವುದು. ಈ ರೀತಿ ಪ್ರವರ್ತಿಸಲ್ಪಟ್ಟ ಜಗಚ್ಚಕ್ರವನ್ನು ಯಾರು ಅನುಸರಿಸುವುದಿಲ್ಲವೋ ಅವರು ಪಾಪಾಯುಗಳೂ ಇಂದ್ರಿಯಾರಾಮರೂ ಆಗಿ ಅವರ ಬಾಳು ವ್ಯರ್ಥವಾಗುವುದು. \enginline{(iii,} ೧೪, ೧೫, ೧೬)

ನಾವು ಎಂದು ಭೂಮಿಗೆ ಬರುತ್ತೇವೆಯೋ ಅಂದಿನಿಂದ ಇಡೀ ಬ್ರಹ್ಮಾಂಡ ಒಂದಲ್ಲ ಒಂದು ರೀತಿಯಲ್ಲಿ ಕೊಡುತ್ತಿದೆ. ಅದು ಕೊಟ್ಟದ್ದನ್ನು ತಿಂದು ನಾವು ಬೆಳೆಯುತ್ತಿರುವೆವು. ಪಂಚಭೂತಗಳೂ, ಎಲ್ಲಾ ಬಗೆಯ ಜೀವಜಂತುಗಳೂ ಸಹಾಯ ಮಾಡುತ್ತಿವೆ. ನಾವು ಅವರಿಂದ ತೆಗೆದುಕೊಳ್ಳುವುದು ಒಂದು ಸಾಲವನ್ನು ಮಾಡಿದಂತೆ. ಆ ಸಾಲವನ್ನು ನಾವು ಸಾಧ್ಯವಾದಷ್ಟು ತೀರಿಸಬೇಕು. ಆದಕಾರಣವೇ ನಮ್ಮ ಪೂರ್ವಿಕರೂ, ಪ್ರತಿಯೊಬ್ಬ ಹಿಂದುವೂ, ಪಂಚಯಜ್ಞಗಳನ್ನು ಮಾಡಬೇಕು ಎನ್ನುತ್ತಿದ್ದರು. ಅದೇ ದೇವಯಜ್ಞ, ಪುಷಿಯಜ್ಞ, ಪಿತೃಯಜ್ಞ, ನೃಯಜ್ಞ, ಭೂತಯಜ್ಞ. ಎಂದು ನಾವು ಯಜ್ಞದೃಷ್ಟಿಯನ್ನು ಮರೆಯುವೆವೊ, ಕೇವಲ ಸ್ವಾರ್ಥಕ್ಕಾಗಿ ಬಾಳುವೆವೊ ಆ ಬಾಳು ಶೋಚನೀಯ, ವ್ಯರ್ಥ.

ಈ ಸೃಷ್ಟಿ ನಿಂತಿರುವುದೇ ಯಜ್ಞದ ಆಧಾರದ ಮೇಲೆ. ಈ ವಿರಾಟ್ ಯಜ್ಞಕ್ಕೆ ಎಲ್ಲರೂ ಅರ್ಪಣೆ ಮಾಡಬೇಕಾಗಿದೆ. ಯಾರಲ್ಲಿ ಯಾವ ವಿಧವಾದ ಒಳ್ಳೆಯದು ಇದೆಯೋ ಅದನ್ನು ಇತರರ ಸೇವೆಗೆ ನೀಡಬೇಕು. ಹಲವು ವಿಧದ ಯಜ್ಞಗಳಿವೆ ಎಂದು ಶ‍್ರೀಕೃಷ್ಣ ಹೇಳುತ್ತಾನೆ. “ಕೆಲವರು ದ್ರವ್ಯ ದಾನ ರೂಪದ ಯಜ್ಞವನ್ನು ಮಾಡುತ್ತಾರೆ. ತಪೋರೂಪವಾದ ಯಜ್ಞವನ್ನು ಕೆಲವರು ಮಾಡುತ್ತಾರೆ. ಸ್ವಾಧ್ಯಾಯ ಮತ್ತು ಜ್ಞಾನರೂಪದ ಯಜ್ಞವನ್ನು ಕೆಲವರು ಮಾಡುತ್ತಾರೆ. ಇವರೆಲ್ಲರೂ ದೃಢವ್ರತರಾದ ಯತಿಗಳು\enginline{” (iv.} ೨೮).

\newpage

ನಾವು ಏನನ್ನು ಕೊಡುತ್ತೇವೆಯೋ ಅದಲ್ಲ ಮುಖ್ಯ. ಅದನ್ನು ಯಾವ ದೃಷ್ಟಿಯಿಂದ ಕೊಡುತ್ತೇವೆ ಎಂಬುದನ್ನು ನೋಡಬೇಕು. ನಾವು ಯಜ್ಞದೃಷ್ಟಿಯಿಂದ ಕೇವಲ ಲೌಕಿಕವಸ್ತುವನ್ನೇ ಇತರರಿಗೆ ನೀಡಬಹುದು. ಆದರೂ ಅದರಿಂದ ಪರಮಜ್ಞಾನ ಪ್ರಾಪ್ತವಾಗುವುದು. ಶ‍್ರೀಕೃಷ್ಣ ಗೀತೆಯಲ್ಲಿ “ಎಲ್ಲಾ ಕರ್ಮಗಳೂ ಜ್ಞಾನದಲ್ಲಿ ಪರಿಸಮಾಪ್ತವಾಗುತ್ತವೆ\enginline{” (iv,} ೩೩)ಎಂದು ಹೇಳುತ್ತಾನೆ. ಶ್ರೇಷ್ಠವಾದ ಆಧ್ಯಾತ್ಮಿಕ ವಸ್ತುವನ್ನು ಕೊಟ್ಟರೆ ನಮ್ಮ ಜೀವನದ ಮೇಲೆ ಯಾವ ಪರಿಣಾಮ ಉಂಟಾಗುವುದೋ ಅದೇ ಪರಿಣಾಮ ಯಜ್ಞದೃಷ್ಟಿಯಿಂದ ರೋಗಿಗೆ ಔಷಧ ಕೊಡುವುದರಿಂದ, ನಿರ್ಗತಿಕನಿಗೆ ಅನ್ನ ಕೊಡುವುದರಿಂದ, ನಿರಕ್ಷರಸ್ಥನಿಗೆ ವಿದ್ಯೆ ಕೊಡುವುದರಿಂದ ಪ್ರಾಪ್ತವಾಗುವುದು.

“ಯಜ್ಞಕ್ಕೋಸ್ಕರ ಮಾಡದೇ ಇರುವ ಕರ್ಮಗಳು ನಮ್ಮನ್ನು ಬಂಧಿಸುವುವು\enginline{” (iii,} ೫). ಯಾವಾಗ ಯಜ್ಞಭಾವನೆಯನ್ನು ಮರೆಯುತ್ತೇವೆಯೋ ಆಗ ನಾವು ಪ್ರಪಂಚದಲ್ಲಿ ಒಳ್ಳೆಯ ಕೆಲಸವನ್ನೇ ಮಾಡುತ್ತಿರಬಹುದು, ವೇದಾಂತ ವಿಷಯಗಳನ್ನೇ ಇತರರಿಗೆ ಬೋಧಿಸುತ್ತಿರಬಹುದು, ಆದರೂ ಕೀರ್ತಿ ಮುಂತಾದ ಪಿಶಾಚಿಗಳು ನಮ್ಮನ್ನು ಮೆಟ್ಟುವುವು. ಆದಕಾರಣ ಒಳ್ಳೆಯ ಕೆಲಸ ಮಾಡುವುದು, ಒಳ್ಳೆಯ ವಿಷಯಗಳನ್ನು ಇತರರಿಗೆ ಹೇಳುವುದು ಇವೇ ದೊಡ್ಡದಲ್ಲ. ಯಾವ ದೃಷ್ಟಿಯಿಂದ ಅದನ್ನು ಮಾಡುತ್ತಿರುವೆವು ಎಂಬುದನ್ನು ಗಮನಿಸಬೇಕು.

“ಅಮೃತವೆಂಬ ಹೆಸರುಳ್ಳ ಯಜ್ಞಶೇಷವನ್ನು ಯಾರು ಸೇವಿಸುವರೋ ಅವರು ಸನಾತನವಾದ ಬ್ರಹ್ಮವನ್ನು ಹೊಂದುತ್ತಾರೆ\enginline{” (iv,} ೩೧). ನಮ್ಮ ಪಾಲಿಗೆ ಯಾವ ಕರ್ತವ್ಯವೆ ಬರಲಿ, ಬರುವ ಫಲ ಭಗವಂತನಿಗೆ ಅರ್ಪಿತವಾಗಲಿ ಎಂದು ಮಾಡಿದರೆ ಅದೊಂದು ಯಜ್ಞವಾಗುವುದು. ಕೊನೆಗೆ ಏನು ಉಳಿಯುವುದೋ ಅದನ್ನು ಒಂದು ಪ್ರಸಾದದಂತೆ ಸ್ವೀಕರಿಸಿದರೆ ಅದೇ ಅಮೃತವಾಗಿ ನಮ್ಮನ್ನು ಸನಾತನ ಬ್ರಹ್ಮಸ್ಥಿತಿಗೆ ಒಯ್ಯುವುದು.

“ಯಜ್ಞ ಮಾಡದವನಿಗೆ ಇಹದಲ್ಲೇ ಸುಖವಿಲ್ಲ. ಇನ್ನು ಪರದಲ್ಲಿ ಹೇಗೆ ಸುಖ ಪ್ರಾಪ್ತ\-ವಾದೀತು\enginline{” (iv,}೩೧). ಈ ಪ್ರಪಂಚದಲ್ಲಿ ನಾವು ಸುಖವಾಗಿರಬೇಕಾದರೆ ಕೊಟ್ಟು ತೆಗೆದುಕೊಳ್ಳಬೇಕು. ಸಾಧ್ಯವಾದಷ್ಟು ಹೆಚ್ಚನ್ನು ಕೊಟ್ಟು ಸಾಧ್ಯವಾದಷ್ಟು ಕಡಿಮೆಯನ್ನು ಸ್ವೀಕರಿಸವುದೇ ಆನಂದದ ಮೂಲ. ಎಲ್ಲಿ ನಾವು ಕೊಡುವಾಗ ಗೊಣಗಾಡುವೆವೋ, ವಿಧಿಯಿಲ್ಲದೆ\break ಬಂದಾಗ ಸಾಧ್ಯವಾದಷ್ಟು ಕಡಮೆ ಕೊಡುವೆವೋ ಅದಕ್ಕಿಂತ ಹೆಚ್ಚನ್ನು ನಿರೀಕ್ಷಿಸುವೆವೋ ಆಗ ನಮಗೆ ದುಃಖ ತಪ್ಪಿದ್ದಲ್ಲ. “ನೀನು ಏನು ಕೆಲಸ ಮಾಡುತ್ತೀಯೊ, ಏನು ಊಟ ಮಾಡುತ್ತೀಯೊ, ಯಾವ ಯಜ್ಞವನ್ನು ಮಾಡುತ್ತೀಯೊ, ಯಾವ ದಾನವನ್ನು ಮಾಡುತ್ತೀಯೊ, ಯಾವ ತಪಸ್ಸನ್ನು ಮಾಡುತ್ತೀಯೊ ಅದನ್ನೆಲ್ಲಾ ನನಗೆ ಅರ್ಪಣೆ ಮಾಡು\enginline{” (ix,} ೨೭) ಎನ್ನುವನು ಶ‍್ರೀಕೃಷ್ಣ. ಈ ಅರ್ಪಿತ ಭಾವವನ್ನೇ ಯಜ್ಞವೆಂದು ಹೇಳುವುದು. ಸೃಷ್ಟಿ ನಿಂತಿರುವುದೇ ಈ ಯಜ್ಞದ ಆಧಾರದ ಮೇಲೆ. ವ್ಯಷ್ಟಿ ಸಮಷ್ಟಿಗೆ ತನ್ನನ್ನು ತಿಳಿದೋ ತಿಳಿಯದೆಯೋ ಪ್ರಯತ್ನಪೂರ್ವಕವಾಗಿಯೋ, ಬಲಾತ್ಕಾರವಾಗಿಯೋ ಧಾರೆ ಎರೆದುಕೊಳ್ಳುತ್ತಿದೆ. ಸಮಷ್ಟಿ ವ್ಯಷ್ಟಿಯ ಬೆಳವಣಿಗೆಗೆ ಏನು ಬೇಕೊ ಅದನ್ನು ನೀಡುತ್ತದೆ. ಕೇಳಿಸಿಕೊಂಡು ಸಮಷ್ಟಿ ವ್ಯಷ್ಟಿಗೆ ಅದನ್ನು ಕೊಡುವುದಿಲ್ಲ. ಕೇಳಿಸಿಕೊಳ್ಳದೆ ಅದನ್ನು ಕೊಡುವುದು. ನಾನು ಕೊಡಬೇಕಾಗಿರುವುದನ್ನು ಕೊಟ್ಟರೆ ನನಗೆ ಬರಬೇಕಾಗಿರುವುದು ಬಂದೇ ಬರುವುದು. ಇದು ಜೀವನದ ಗಾಢ ನಿಯಮ.

ಕೊಡುವುದರ ಮೇಲೆ ಜಗತ್ತು ನಿಂತಿದೆ. ಕೆಲವರು ತಿಳಿದು ಕೊಡುತ್ತಿರುವರು. ಮತ್ತೆ ಕೆಲವರು ತಿಳಿಯದೆ ಕೊಡುತ್ತಿರುವರು. ಕೆಲವರು ಸಂತೋಷದಿಂದ ಕೊಡುತ್ತಿರುವರು. ಕೆಲವರು ಅಯ್ಯೋ! ಕೊಡಬೇಕಲ್ಲಾ ಎಂದು ಗೊಣಗಾಡಿಕೊಂಡು ಕೊಡುತ್ತಿರುವರು. ಅಂತೂ ನಾವು ಕೊಡಬೇಕಾಗಿರುವುದನ್ನು ದೇವರು ವಸೂಲಿ ಮಾಡಿಯೇ ಮಾಡುವನು. ಅವನಿಗೆ ವಸೂಲಿ ಮಾಡುವುದಕ್ಕೆ ಹಲವು ಮಾರ್ಗಗಳಿವೆ. ದಾನಕ್ಕೆ ಕೊಡದೇ ಇದ್ದರೆ ದಂಡಕ್ಕೆ ಕೊಡಬೇಕಾಗುವುದು. ರೋಗ, ನಷ್ಟ, ಅತ್ಯಾಸೆ, ಕೋರ್ಟು, ಕಚೇರಿ, ಕಳ್ಳಕಾಕರು ಇವುಗಳ ಮೂಲಕ ತನಗೆ ಬರಬೇಕಾಗಿರುವುದನ್ನು ವಸೂಲಿ ಮಾಡಿಯೇ ಮಾಡುವನು. ನಗುತ್ತಾ ಕೊಡು, ಕೊಟ್ಟು ಉದ್ಧಾರವಾಗು, ಕೊಟ್ಟು ಶಾಂತಿ ಹೊಂದು ಎನ್ನುವುದು ಯಜ್ಞದೃಷ್ಟಿ.

ಬ್ರಹ್ಮಾಂಡವನ್ನು ನೋಡಲಿ, ಪಿಂಡಾಂಡವನ್ನು ನೋಡಲಿ, ಎಲ್ಲಾ ಮತ್ತೊಂದಕ್ಕೆ ದುಡಿಯುವುದು ಕಾಣುವುದು. ಸೂರ್ಯ ತನಗಾಗಿ ಕಾಂತಿಯನ್ನು ಶಾಖವನ್ನು ಚೆಲ್ಲುತ್ತಿಲ್ಲ. ಭೂಮಿಗಾಗಿ, ಇತರ ಗ್ರಹಗಳಿಗಾಗಿ ಅದನ್ನು ಕೊಡುತ್ತಿರುವನು. ಚಂದ್ರ ತನಗಾಗಿ ಸುತ್ತುತ್ತಿಲ್ಲ, ಭೂಮಿಗಾಗಿ ಸುತ್ತುತ್ತಿರು ವನು. ಭೂಮಿ ಕೂಡ ತನಗಾಗಿ ಸೂರ್ಯನ ಸುತ್ತ ಸುತ್ತುತ್ತಿಲ್ಲ. ಅಲ್ಲಿರುವ ಪ್ರಾಣಿವರ್ಗಕ್ಕಾಗಿ ಹಗಲು ರಾತ್ರಿಗಳಾಗಲೆಂದು ಸೂರ್ಯನ ಸುತ್ತಲೂ ಪ್ರದಕ್ಷಿಣೆ ಮಾಡುತ್ತಿದೆ. ಅನೇಕ ಖನಿಜ ಸಂಪತ್ತುಗಳು ಎಣ್ಣೆ ಮಂತಾದುವುಗಳನ್ನು ಭೂಮಿ ತನಗಾಗಿ ತನ್ನ ಗರ್ಭದಲ್ಲಿ ಹುದುಗಿಸಿಟ್ಟುಕೊಂಡಿಲ್ಲ. ಮುಂದೆ ಬರುವ ಮಾನವರಿಗಾಗಿ ಇಟ್ಟಿದೆ. ಭೂಮಿಯ ಮೇಲಿರುವ ಬೆಟ್ಟ ತನಗಾಗಿ ಇಲ್ಲ. ನದಿ ತನಗಾಗಿ ಹರಿಯುತ್ತಿಲ್ಲ. ಗಿಡಮರಗಳು ಕೇವಲ ತಮಗಾಗಿ ಹೂ ಹಣ್ಣನ್ನು ಬಿಡುತ್ತಿಲ್ಲ. ಎಲ್ಲ ಇತರರ ಸೌಕರ್ಯಕ್ಕಾಗಿ ಬಾಳುತ್ತಿವೆ.

ನಮ್ಮ ದೇಹದಲ್ಲಿಯೂ ಒಂದು ಮತ್ತೊಂದಕ್ಕೆ ವ್ಯಷ್ಟಿ ಸಮಷ್ಟಿಗೆ ಕೆಲಸ ಮಾಡುವುದನ್ನು ನೋಡುತ್ತೇವೆ. ಹೃದಯ ತನ್ನೆಡೆಗೆ ಬರುವ ರಕ್ತವನ್ನು ಒತ್ತಿ ನಮ್ಮ ದೇಹದ ಭಾಗಗಳಿಗೆಲ್ಲ ಕಳುಹಿಸುವುದು. ಮೂಳೆಯ ಒಳಗಡೆ ರಕ್ತದ ಕಣಗಳು ಹುಟ್ಟಿ ರಕ್ತ ಪ್ರವಾಹವನ್ನು ಸೇರುವುವು. ದೇಹದ ಹಲವು ಕಡೆಗಳಲ್ಲಿರುವ ಹಲವಾರು ಗ್ರಂಥಿಗಳು ತಮ್ಮ ರಸವನ್ನು ಸಮಷ್ಟಿ ದೇಹದ ರಕ್ಷಣೆಗೆ ಸುರಿಸುತ್ತಿವೆ. ಎಲ್ಲಾ ಅಂಗಾಂಗಗಳು ಒಟ್ಟು ದೇಹಕ್ಕೆ ಕೆಲಸ ಮಾಡುವುವು. ಒಟ್ಟು ದೇಹ ಪ್ರತಿಯೊಂದು ಅಂಗಾಂಗವನ್ನು ಗಮನದಲ್ಲಿಟ್ಟಿರುವುದು.

ಸಮಾಜದಲ್ಲಿಯೂ ಇದೇ ನಿಯಮ ಜಾರಿಯಲ್ಲಿರುವುದನ್ನು ನೋಡುತ್ತೇವೆ. ಪಾಂಡಿತ್ಯ ಅಧಿಕಾರ ಐಶ್ವರ್ಯ ಎಲ್ಲರಿಗೂ ಹಂಚಿ ಹೋಗುವುದಕ್ಕಿಂತ ಮುಂಚೆ ಕೆಲವು ವ್ಯಕ್ತಿಗಳಲ್ಲಿ ಸಂಗ್ರಹವಾಗುವುದು. ಅಲ್ಲಿಂದ ಇತರ ಕಡೆಗೆ ಹರಿದುಹೋಗುವುದು. ಮೇಲಿರುವ ಟ್ಯಾಂಕಿನಿಂದ ಕೆಳಗಿರುವ ಊರಿನ ನಲ್ಲಿಗಳಿಗೆಲ್ಲಾ ನೀರು ಬರುವುದು. ಕೆಳಗೆ ನೀರನ್ನು ಕಳುಹಿಸುವುದಕ್ಕಾಗಿ ಮೇಲೆ ಟ್ಯಾಂಕನ್ನು ಮಾಡಿದ್ದಾರೆ. ಆ ಟ್ಯಾಂಕು ಎಲ್ಲಾ ನೀರು ತನ್ನಲ್ಲೇ ಇರಬೇಕು ಎಂದರೆ\break ನಾಶವಾಗುವುದು. ಇನ್ನೊಬ್ಬರಿಗೆ ಹಂಚುವುದಕ್ಕಾಗಿ ಇನ್ನೊಬ್ಬರಿಗೆ ಒಳ್ಳೆಯದನ್ನು ಮಾಡುವುದಕ್ಕಾಗಿ ದೇವರು ಕೆಲವು ವ್ಯಕ್ತಿಗಳಲ್ಲಿ ಪಾಂಡಿತ್ಯ, ಐಶ್ವರ್ಯ, ಅಧಿಕಾರ ಮುಂತಾದುವನ್ನು ಶೇಖರಿಸಿಡುವನು. ಯಾವಾಗ ವ್ಯಕ್ತಿ ಬರೀ ಶೇಖರಿಸುತ್ತಾ ಹೋಗುವನೊ ಹಂಚುವುದಿಲ್ಲವೊ ಅವನು ಆಗ ಹಾಳಾಗುತ್ತಾನೆ. ಇತರರಿಗೆ ನೀಡುವುದಕ್ಕಾಗಿ ದೇವರು ನಮ್ಮಲ್ಲೆ ಇಟ್ಟಿದ್ದಾನೆ ಎಂದರಿತು ಅರ್ಪಿತ ಭಾವದಿಂದ ಯಾವಾಗ ಇನ್ನೊಬ್ಬರಿಗೆ ಕೊಡುತ್ತೇವೆಯೋ ಅದೇ ಯಜ್ಞ, ಇದೇ ವ್ಯಕ್ತಿಗೆ ಮತ್ತು ಜನಾಂಗಕ್ಕೆ ಕಲ್ಯಾಣಕಾರಿ. ಗೀತೆಯಲ್ಲಿ ಬರುವ ಒಂದು ಅಮೋಘವಾದ ಭಾವನೆ ಇದು. ಶ‍್ರೀಕೃಷ್ಣ ಗೀತೆಯಲ್ಲಿ ಇನ್ನು ಏನನ್ನೂ ಹೇಳದೆ ಈ ಯಜ್ಞಭಾವನೆಯೊಂದನ್ನು ಮಾತ್ರ ಕೊಟ್ಟಿದ್ದರೂ ಸಾರ್ಥಕ\-ವಾಗುತ್ತಿತ್ತು ಗೀತೆ.


\section*{ಚತುರ್ವರ್ಣ}

ಶ‍್ರೀಕೃಷ್ಣ “ಗುಣ ಮತ್ತು ಕರ್ಮಗಳಿಗೆ ತಕ್ಕಂತೆ ಚತುರ್ವರ್ಣ ನನ್ನಿಂದ ಸೃಷ್ಟಿಸಲ್ಪಟ್ಟಿತು\enginline{” (iv,} ೧೩)ಎನ್ನುತ್ತಾನೆ. ಬ್ರಾಹ್ಮಣ, ಕ್ಷತ್ರಿಯ, ವೈಶ್ಯ ಮತ್ತು ಶೂದ್ರ ಎಂಬ ವರ್ಣಗಳನ್ನು ದೇವರೇ ಮಾಡಿದನು ಎಂದು ಹೇಳಿದೆ. ಗುಣ ಮತ್ತು ಕರ್ಮಗಳಿಗೆ ಅನುಸಾರವಾಗಿ ಅವರನ್ನು ವಿಭಾಗ ಮಾಡಿರುವೆನು ಎನ್ನುತ್ತಾನೆ. ಶ‍್ರೀಕೃಷ್ಣ ನಾಲ್ಕು ವರ್ಣಗಳನ್ನು ಕುರಿತು ಹೇಳುತ್ತಿರುವಾಗ ಕೇವಲ ಭರತಖಂಡ ಒಂದನ್ನೇ ತನ್ನ ಮನಸ್ಸಿನಲ್ಲಿಟ್ಟುಕೊಂಡಿಲ್ಲ. ಇಡೀ ವಿಶ್ವದ ದೃಷ್ಟಿಯಿಂದ ಹೇಳುತ್ತಿರು ವನು. ಒಂದು ದೇಶ ಚೆನ್ನಾಗಿರಬೇಕಾದರೆ ಅಲ್ಲಿ ನಾಲ್ಕು ಬಗೆಯ ಜನರು ಇರಬೇಕು. ಎಲ್ಲಿ ಪ್ರತಿಯೊಬ್ಬನೂ ತನ್ನ ಪಾಲಿಗೆ ಬಂದ ಕರ್ತವ್ಯವನ್ನು ಮಾಡುವುದಿಲ್ಲವೋ, ಇನ್ನೊಬ್ಬರ ಕರ್ತವ್ಯವನ್ನು ಮಾಡುತ್ತಾನೋ ಆಗಲೆ ವರ್ಣಸಂಕರ ಉಂಟಾಗುವುದು, ದೇಶದಲ್ಲಿ ಅನಾಯಕತೆ ತಲೆದೋರುವುದು. ವರ್ಣವನ್ನು ಗುಣ ಮತ್ತು ಸ್ವಭಾವದ ಮೂಲಕ ನಿರ್ಧರಿಸಬೇಕಾಗಿದೆ ಎಂದಿದ್ದರೂ, ಆಯಾ ವರ್ಣದಲ್ಲಿ ಹುಟ್ಟಿದ ಮಾತ್ರಕ್ಕೆ ಅವನಿಗೆ ಆಯಾ ಸ್ವಭಾವ ಇರಲಿ ಇಲ್ಲದಿರಲಿ ಆ ವರ್ಣಕ್ಕೆ ಸೇರಿದವನೆಂಬ ತಪ್ಪು ಭಾವನೆ ನಮ್ಮಲ್ಲಿ ಬಂದುಹೋಗಿದೆ.

ಭರತಖಂಡದಲ್ಲಿ ಈಗಿರುವ ಯಾವ ಜಾತಿಯನ್ನು ತೆಗೆದುಕೊಂಡರೂ ಅದರಲ್ಲಿ ನಾಲ್ಕು ವರ್ಣದವರು ಇರುವುದನ್ನು ನೋಡುವೆವು. ಒಂದು ಬ್ರಾಹ್ಮಣ ಕುಟುಂಬವನ್ನು ತೆಗೆದುಕೊಂಡರೆ, ಅಲ್ಲಿ ಒಬ್ಬ ಬ್ರಾಹ್ಮಣನ ವೃತ್ತಿಯನ್ನು ಅನುಸರಿಸಿ ವೇದಾಧ್ಯಯನ ಮುಂತಾದುವನ್ನು ಮಾಡುವನು, ಮತ್ತೊಬ್ಬ ಮಿಲಿಟರಿಗೆ ಸೇರುವನು, ಮೂರನೆಯವನು ಒಂದು ಅಂಗಡಿ ತೆಗೆಯುವನು, ನಾಲ್ಕನೆ ಯವನು ಯಾವುದೊ ಫ್ಯಾಕ್ಟರಿಯಲ್ಲಿ ಕೆಲಸಕ್ಕೆ ಸೇರುವನು. ನಾವು ಯಾವ ವರ್ಣದಲ್ಲಿ ಹುಟ್ಟಿರು ವೆವೋ ಅದಲ್ಲ ಮುಖ್ಯ. ನಾವು ಯಾವುದನ್ನು ನಮ್ಮ ಅಭಿರುಚಿ ಮತ್ತು ಯೋಗ್ಯತೆಗೆ ತಕ್ಕಂತೆ ಆರಿಸಿಕೊಳ್ಳುವೆವೋ ಅದು ಮುಖ್ಯ. ಪ್ರತಿಯೊಂದು ವ್ಯಕ್ತಿಗೂ ಒಂದು ಧರ್ಮವಿದೆ. ಯಾವಾಗ\break ನಾವು ಅದನ್ನು ಸ್ವೀಕರಿಸುತ್ತೇವೆಯೋ ಅದನ್ನು ಅನುಸರಿಸಬೇಕು. ಯಾವ ವೃತ್ತಿಯ ಮೇಲೆ ನನಗೆ ಅಭಿರುಚಿ ಇರುವುದೊ, ಮತ್ತು ಯಾವುದಕ್ಕಾಗಿ ನಾನು ಹಲವು ವರುಷಗಳು ತರಬೇತನ್ನು ತೆಗೆದುಕೊಂಡಿರುವೆನೊ, ಅದನ್ನು ಬಿಡದೆ, ಆ ಮಾರ್ಗದಲ್ಲಿ ಮುಂದುವರಿಯಬೇಕು. ಹಾಗೆ ಮಾಡದೆ ಹೋದರೆ ತಪ್ಪಿತಸ್ಥರಾಗುವೆವು. ಅರ್ಜುನ ಇಲ್ಲಿ ಮಾಡುತ್ತಿರುವ ತಪ್ಪು ಅದೇ. ಅವನು ಕ್ಷತ್ರಿಯನಾಗಿ ಹುಟ್ಟಿದ್ದು ಮಾತ್ರವಲ್ಲ. ಅದು ಗೌಣ. ಅನಂತರ ಅವನು ಕ್ಷತ್ರಿಯನ ವೃತ್ತಿಯಲ್ಲಿ ತರಬೇತನ್ನು ತೆಗೆದುಕೊಂಡಿರುವನು. ದೇಶ ಅವನಿಂದ ಕ್ಷತ್ರಿಯನ ಕೆಲಸವನ್ನು ಒಂದು ಸನ್ನಿವೇಶದಲ್ಲಿ ನಿರೀಕ್ಷಿಸು ತ್ತಿದೆ. ಯಾವಾಗ ಅವನು ಆ ಕೆಲಸವನ್ನು ಮಾಡುವುದಿಲ್ಲವೋ ಆಗ ಅವನು ತಪ್ಪಿತಸ್ಥನಾಗುತ್ತಾನೆ.

ಭರತಖಂಡದಲ್ಲಿ ನಾವು ಈಗ ಯಾವುದನ್ನೂ ಅವನ ಹುಟ್ಟಿದ ಜಾತಿಯ ದೃಷ್ಚಿಯಿಂದ ಅಳೆಯಬೇಕಾಗಿಲ್ಲ. ಏಕೆಂದರೆ ಯಾವ ಜಾತಿಯೂ ಶುದ್ಧವಾಗಿಲ್ಲ. ಎಲ್ಲರೂ ಬೇರೆ ಬೇರೆ ವೃತ್ತಿಗಳಿಗೆ ಕೈ ಹಾಕಿರುವರು. ಆದರೆ ನಾವು ಒಂದನ್ನು ಅನುಸರಿಸಬೇಕು, ಯಾವ ವೃತ್ತಿಯನ್ನು ನಾವು ತೆಗೆದುಕೊಂಡಿರುವೆವೊ ಅದಕ್ಕೆ ಒಂದು ಧರ್ಮವಿದೆ. ಆ ಧರ್ಮವನ್ನು ಅನುಸರಿಸಬೇಕು. ಹಾಗೆ ಅನುಸರಿಸುವಾಗ ಒಂದು ವೃತ್ತಿ ಮೇಲಲ್ಲ ಮತ್ತೊಂದು ವೃತ್ತಿ ಕೀಳಲ್ಲ. ಒಂದರಷ್ಟೇ ಮತ್ತೊಂದು ಮುಖ್ಯ ಸಮಷ್ಟಿಯ ದೃಷ್ಟಿಯಿಂದ ನೋಡಿದಾಗ. ಒಂದರಷ್ಟೇ ಮತ್ತೊಂದು ಪವಿತ್ರ ಬ್ರಹ್ಮದೃಷ್ಟಿಯಿಂದ ನೋಡಿದಾಗ. ಬ್ರಹ್ಮನ ಬಾಯಿಂದ ಬ್ರಾಹ್ಮಣರು, ಭುಜದಿಂದ ಕ್ಷತ್ರಿಯರು, ತೊಡೆಯಿಂದ ವೈಶ್ಯರು, ಪಾದಗಳಿಂದ ಶೂದ್ರರೂ ಆಗಿರುವರು ಎಂದು ಪುರುಷಸೂಕ್ತದಲ್ಲಿ ಬರುವುದು. ಆ ಬ್ರಹ್ಮನಲ್ಲಿ ಒಂದು ಕೀಳು, ಮತ್ತೊಂದು ಮೇಲು ಎಂಬ ಭಾವನೆ ಹೇಗೆ ಇದ್ದೀತು? ಒಂದರಷ್ಟೇ ಮತ್ತೊಂದು ಪವಿತ್ರ. ಸಕ್ಕರೆಯಿಂದ ಮಾಡಿದ ಗೊಂಬೆಯಲ್ಲಿ ತಲೆ ತಿನ್ನಲಿ, ತೋಳು ತಿನ್ನಲಿ, ಕಾಲು ತಿನ್ನಲಿ, ಎಲ್ಲಾ ಸಿಹಿಯೇ. ಒಂದು ಹೆಚ್ಚು ಅಲ್ಲ, ಮತ್ತೊಂದು ಕಡಿಮೆ ಅಲ್ಲ. ವರ್ಣ ಮಾನವ ಜೀವನಕ್ಕೆ ಆವಶ್ಯಕ. ಹಳೆಯ ವರ್ಣವನ್ನು ಕಿತ್ತು ಹಾಕಿದರೆ ಹೊಸ ವರ್ಣ ಆ ಸ್ಥಳದಲ್ಲಿ ಹುಟ್ಟಿಕೊಳ್ಳುವುದು. ಒಂದು ವರ್ಣ ಮೇಲು, ಮತ್ತೊಂದು ಕೀಳು ಎಂಬ ಭಾವನೆ ಹೋಗಬೇಕು. ಆಯಾ ವರ್ಣದವರು ಮುಕ್ತರಾಗುವಾಗ ಅಲ್ಲಿಂದಲೇ ನೇರವಾಗಿ ಮುಕ್ತರಾಗುತ್ತಾರೆ. ಅವರು ಬೇರೆ ಬೇರೆ ವರ್ಣಕ್ಕೆ ಪ್ರಮೋಷನ್ ತೆಗೆದುಕೊಂಡು ಮುಕ್ತರಾಗುವುದಿಲ್ಲ.


\section*{ಗೀತೆಯಲ್ಲಿ ಬರುವ ಯೋಗಗಳು}

ಗೀತೆಯ ಹದಿನೆಂಟು ಅಧ್ಯಾಯಗಳ ಕೊನೆಯಲ್ಲಿಯೂ “ಯೋಗ” ಎಂದು ಹಾಕಿದೆ. ಪ್ರತಿ ಯೊಂದು ಅಧ್ಯಾಯಕ್ಕೂ ಒಂದೊಂದು ಯೋಗವೆಂದು ಹೆಸರು ಕೊಟ್ಟಿದೆ. ಯೋಗ ಎಂದರೆ ಒಂದುಗೂಡಿಸುವುದು ಎಂದು ಅರ್ಥ. ಯಾವುದು ಜೀವಾತ್ಮನನ್ನು ಪರಮಾತ್ಮನೊಡನೆ ಸೇರಿಸು\-ವುದೋ ಅದು ಯೋಗ. ಈ ದೃಷ್ಟಿಯಿಂದ ಗೀತೆಯಲ್ಲಿ ನಾಲ್ಕು ಮುಖ್ಯ ಯೋಗಗಳಿವೆ. ಅವೇ ಜ್ಞಾನಯೋಗ, ಕರ್ಮಯೋಗ, ಭಕ್ತಿಯೋಗ ಮತ್ತು ಧ್ಯಾನಯೋಗಗಳು. ಇಲ್ಲಿ ಯಾವೊಂದು ಯೋಗವೂ ಪ್ರತ್ಯೇಕವಾಗಿ ಇತರ ಯೋಗಗಳೊಂದಿಗೆ ಸಂಬಂಧವಿಲ್ಲದೆ ಇಲ್ಲ. ಪ್ರತಿಯೊಂದು ಯೋಗದಲ್ಲಿಯೂ ಇತರ ಯೋಗಗಳ ಅಂಶಗಳು ಬೆರೆತಿವೆ. ಆದರೆ ಆಯಾ ಯೋಗದ ಭಾಗ ಜಾಸ್ತಿ ಇರುವುದರಿಂದ ಅದಕ್ಕೆ ಆ ಹೆಸರು ಬಂದಿದೆ.

\textbf{ಜ್ಞಾನ:} ಇದು ಅತಿ ಕಷ್ಟದ ದಾರಿ. ವಿಚಾರ ಪ್ರಧಾನವಾದ ದಾರಿ. ವ್ಯಕ್ತಿ ತನ್ನ ಎದುರಿಗೆ ಇರುವ ದೃಶ್ಯವನ್ನೆಲ್ಲ ವಿಭಜನೆ ಮಾಡಿ ಯಾವುದು ಆತ್ಮವಸ್ತುವೊ ಅದನ್ನು ಮಾತ್ರ ಹಿಡಿದುಕೊಂಡು ಅನಾತ್ಮವಾದ ವಸ್ತುವನ್ನು ತ್ಯಜಿಸಬೇಕಾಗುವುದು. ಇಲ್ಲಿ ದೃಶ್ಯ ಪ್ರಪಂಚವೆಂದರೆ ದೇಹದಿಂದ ಹೊರಗೆ ಇರುವ ವಸ್ತು ಮಾತ್ರವಲ್ಲ. ದೇಹ, ಇಂದ್ರಿಯ, ಮನಸ್ಸು, ಬುದ್ಧಿ, ಅಹಂಕಾರ ಮುಂತಾ ದುವುಗಳೆಲ್ಲ ಪ್ರಕೃತಿಯೊಳಗೇ ಸೇರಿವೆ. ಆತ್ಮನ ಮೇಲೆ ಈ ಉಪಾಧಿಗಳೆಲ್ಲ ಆರೋಪವಾಗಿವೆ. ನಾವು ಅವುಗಳನ್ನು ಬೇರ್ಪಡಿಸಬೇಕು. ಇದೊಂದು ದೊಡ್ಡ ಸಾಹಸ. ಪ್ರಕೃತಿಯನ್ನು ಚಾಪೆಯಂತೆ ಸುತ್ತಿ ಅತ್ತ ಸರಿಸಬೇಕು. ಇದಕ್ಕೆ ನಮ್ಮ ಚಿತ್ತ ಶುದ್ಧವಾಗಿರಬೇಕು. ಇಂದ್ರಿಯಗಳನ್ನು ಜಯಿಸಬೇಕು, ಬುದ್ಧಿ ಹರಿತವಾಗಿರಬೇಕು. ಆಗ ಮಾತ್ರ ಈ ಹಾದಿಯಲ್ಲಿ ಮುಂದುವರಿದು ಗುರಿಯನ್ನು ಸೇರಬಹುದು. ಇಲ್ಲಿ ತಾನು ಸತ್ಯವನ್ನು ಕಾಣಬೇಕು, ಅದರಲ್ಲಿ ಒಂದಾಗಬೇಕು ಎಂಬುದೊಂದೇ ಮುಖ್ಯ ಉದ್ದೇಶವಾಗಿರಬೇಕು. ಇದಕ್ಕಾಗಿ ಅವನು ಪೂರ್ವಕಲ್ಪಿತ ಅಭಿಪ್ರಾಯಗಳನ್ನೆಲ್ಲ ಬಲಿ ಕೊಡಬೇಕು. ಸತ್ಯವೆಲ್ಲಿಗೆ ಕೊಂಡೊಯ್ದರೆ ಅಲ್ಲಿಗೆ ಹೋಗಲು ಸಿದ್ಧನಾಗಿರಬೇಕು. ಈ ಮಾರ್ಗ ಕಷ್ಟವಾದರೂ, ಇಲ್ಲಿ ಹೋಗ ಬಯಸುವವನಿಗೆ ಸಾಕಷ್ಚು ಪ್ರೋತ್ಸಾಹವಿದೆ ಗೀತೆಯಲ್ಲಿ.

\textbf{ಕರ್ಮ:} ಮುಂಚೆ ನಾವು ಕರ್ಮಕ್ಕೂ, ಕರ್ಮಯೋಗಕ್ಕೂ ವ್ಯತ್ಯಾಸವನ್ನು ತಿಳಿದುಕೊಳ್ಳಬೇಕು. ಕರ್ಮ ಎಂದರೆ ಕೆಲಸ ಮಾಡುವುದು. ಈ ಜೀವನದಲ್ಲಿ ಎಲ್ಲರೂ ಕೆಲಸ ಮಾಡುತ್ತಿರುವರು. ಕೆಲಸ ಮಾಡದೆ ಯಾರೂ ಇಲ್ಲ. ಕೆಲಸ ಮಾಡುವುದು ಮಾಡದೆ ಇರುವುದಕ್ಕಿಂತ ಸುಲಭ. ಅದಕ್ಕಾಗಿ ಅವನು ಕೆಲಸ ಮಾಡುವನು. ಆದರೆ ಆ ಕರ್ಮದಿಂದ ಅನೇಕ ವೇಳೆ ವ್ಯಥೆ ಅನುಭವಿಸುವನು. ಹೇಗೆ ಮಾಡಿದರೆ ಕರ್ಮ ಬಂಧನದಿಂದ ಪಾರಾಗುತ್ತೇವೆಯೋ ಅದೇ ಕರ್ಮಯೋಗ. ಕರ್ಮಕ್ಕೆ ಎರಡು ಶಕ್ತಿ ಇದೆ. ಒಂದು ರೀತಿ ಮಾಡಿದರೆ ಅದು ನಮ್ಮನ್ನು ಪ್ರಪಂಚಕ್ಕೆ ಕಟ್ಟಿಹಾಕುವುದು. ಮತ್ತೊಂದು ರೀತಿ ಮಾಡಿದರೆ ಅದು ನಮ್ಮನ್ನು ಬಂಧನದಿಂದ ಪಾರು ಮಾಡುವುದು. ರೇಶ್ಮೆಹುಳು ಸುತ್ತಲೂ ಗೂಡನ್ನು ನೆಯ್ದು ಅದರ ಬಲೆಯೊಳಗೆ ಬೀಳುವುದು. ಅನಂತರ ತಾನೇ ಆ ಗೂಡನ್ನು ಸೀಳಿಕೊಂಡು ಬರುವುದು. ಒಂದು ರೀತಿ ಕರ್ಮಮಾಡಿದರೆ ಬಂಧನಕ್ಕೆ ಬೀಳುತ್ತೇವೆ. ಮತ್ತೊಂದು ರೀತಿ ಕರ್ಮ ಮಾಡಿದರೆ ಬಂಧನದಿಂದ ಪಾರಾಗುತ್ತೇವೆ. ಇದನ್ನೇ ಶ‍್ರೀಕೃಷ್ಣ "ಯೋಗಃ ಕರ್ಮಸು ಕೌಶಲಂ" ಎನ್ನುವನು. ಕರ್ಮ ಕುಶಲಿ ಕರ್ಮಮಾಡಿ ಅದರ ಬಂಧನಕ್ಕೆ ಸಿಕ್ಕದಂತೆ ನೋಡಿಕೊಳ್ಳುತ್ತಾನೆ.

ಶ‍್ರೀಕೃಷ್ಣ, “ಕರ್ಮಮಾಡುವುದಕ್ಕೆ ಮಾತ್ರ ನಿನಗೆ ಅಧಿಕಾರ, ಅದರಿಂದ ಬರುವ ಫಲಗಳಿಗಲ್ಲ\enginline{” (ii,} ೪೭) ಎನ್ನುತ್ತಾನೆ. ನಾವು ಬಂಧನಕ್ಕೆ ಬೀಳುವುದಕ್ಕೆಲ್ಲ ಕಾರಣ ಅದರ ಫಲಕ್ಕೆ ಆಸಕ್ತರಾಗಿರುವುದರಿಂದ. ಕರ್ಮಮಾಡುತ್ತೇವೆ, ಅದರಿಂದ ಬರುವ ಹೆಸರು ಲಾಭ ಕೀರ್ತಿ ಇವುಗಳಿಗೆ ಕೈಯೊಡ್ಡುತ್ತೇವೆ. ಅದನ್ನು ತಿಂದಾದಮೇಲೆ ಬಿಟ್ಟುಹೋಗುವುದಕ್ಕೆ ಆಗುವುದಿಲ್ಲ. ನೊಣ ಹಲಸಿನ ಅಂಟಿನ ಮೇಲೆ ಕುಳಿತಂತಾಗಿ ಅಲ್ಲೆ ಸಾಯುವೆವು ನಾವು. ಹಾಗಾದರೆ ಕರ್ಮವನ್ನು ಹೇಗೆ\break ಮಾಡಬೇಕು ಎಂದರೆ ಅನಾಸಕ್ತನಾಗಿ ಕರ್ಮಮಾಡು ಎನ್ನುತ್ತಾನೆ. ಯಜ್ಞದೃಷ್ಟಿಯಿಂದ ಕರ್ಮ\-ಮಾಡು ಎನ್ನುತ್ತಾನೆ. ಬರುವ ಫಲಗಳನ್ನೆಲ್ಲ ಭಗವಂತನಿಗೆ ಅರ್ಪಿಸಿ ನಿಶ್ಚಿಂತನಾಗಿರು ಎನ್ನುತ್ತಾನೆ. ನಾನು ಕೆಲಸಮಾಡುವವನು ಎಂದು ಭಾವಿಸ ಬೇಡ, “ನನ್ನ ಕೈಯಲ್ಲಿ ಒಂದು ನಿಮಿತ್ತವಾಗು\enginline{” (xi,} ೩೩) ಎನ್ನುತ್ತಾನೆ ಶ‍್ರೀಕೃಷ್ಣ. ನಿಜವಾಗಿ ಕೆಲಸಮಾಡುವವನು ದೇವರು. ನಾವು ಅವನ ಕೈಯಲ್ಲಿ ಒಂದು ಯಂತ್ರ. ಯಾವಾಗ ಈ ದೃಷ್ಟಿ ಅವನಲ್ಲಿ ಬರುವುದೊ ಅವನು ಎಂದಿಗೂ ಕೆಟ್ಟ ಕೆಲಸವನ್ನು ಮಾಡಲಾರ.

ನಾವು ಮೂರು ದೃಷ್ಟಿಯಿಂದ ಕೆಲಸ ಮಾಡಬಹುದು. ಒಂದು ತುಂಬಾ ಕೆಳಮಟ್ಟದ್ದು. ಅಲ್ಲಿ ಲಾಭ, ಹೆಸರು, ಕೀರ್ತಿಗಾಗಿ ಕೆಲಸ ಮಾಡುತ್ತಾನೆ. ಇಲ್ಲಿ ಸ್ವಲ್ಪ ಸುಖ ಪಡುತ್ತಾನೆ. ಅನಂತರ ಅದರಿಂದ ದುಃಖವನ್ನೂ ಪಡುತ್ತಾನೆ. ಅದಕ್ಕಿಂತ ಉತ್ತಮ ಸ್ಥಿತಿಯೇ, ಕರ್ಮಮಾಡುವುದರಿಂದ ನನ್ನ ಚಿತ್ತ ಶುದ್ಧಿಯಾಗುವುದು, ಅದಕ್ಕಾಗಿ ಮಾಡುತ್ತೇನೆ ಎಂದು ನೋಡುವುದು. ಮೂರನೆಯದೆ ಒಬ್ಬನ ಚಿತ್ತಶುದ್ಧವಾಗಿದ್ದರೆ, ಅವನು ಲೋಕ ಸಂಗ್ರಹ ದೃಷ್ಟಿಯಿಂದ ಕೆಲಸಮಾಡುತ್ತಾನೆ, ಲೋಕಕಲ್ಯಾಣ\-ವಾಗಲಿ ಎಂದು ಕೆಲಸಮಾಡುತ್ತಾನೆ. ಇವನಿಗೆ ಕೆಲಸ ಮಾಡಿದರೆ ಇದರಿಂದ ಯಾವ ಲಾಭವೂ ಇಲ್ಲ, ಮಾಡದೆ ಇದ್ದರೆ ನಷ್ಟವೂ ಇಲ್ಲ. ಆದರೂ ಸುತ್ತಲೂ ಜನ ಇವನು ಹೇಗೆ ಮಾಡುತ್ತಾನೆಯೋ ಅದನ್ನು ಅನುಸರಿಸುತ್ತಾರೆ. ಶ್ರೇಷ್ಠ ಯಾವುದನ್ನು ಮಾಡುತ್ತಾನೆಯೋ ಸಾಧಾರಣರು ಆ ಕೆಲಸವನ್ನು ಮಾಡುತ್ತಾರೆ. ಶ್ರೇಷ್ಠ ಸುಮ್ಮನೆ ಇದ್ದರೆ ಸಾಧಾರಣ ಜನರೂ ಸುಮ್ಮನಿರುತ್ತಾರೆ. ಆದಕಾರಣ ಶ್ರೇಷ್ಠನಾದವನು ತನಗೆ ಅದರಿಂದ ಯಾವ ಲಾಭವಿಲ್ಲದೇ ಇದ್ದರೂ ಇತರರು ತನ್ನನ್ನು ಅನುಕರಿಸಿ ಕಷ್ಟಕ್ಕೆ ಬೀಳುತ್ತಾರೆ ಎಂದು ಯೋಚಿಸಿ ತಾನು ಕರ್ಮಮಾಡಿ ಇತರರ ಕೈಯಿಂದಲೂ ಕರ್ಮಮಾಡಿಸುತ್ತಾನೆ. ಶ‍್ರೀಕೃಷ್ಣನೇ ಇದಕ್ಕೆ ಒಂದು ಉದಾಹರಣೆ. ಅವನಿಗೆ ಯಾವ ಅಭಾವವೂ ಇಲ್ಲ. ಆದರೂ ಅವನು ಕರ್ಮದಲ್ಲಿ ನಿರತನಾಗಿರುವನು. ಸಾಧಾರಣ ಕರ್ಮವನ್ನು ಕರ್ಮಯೋಗವನ್ನಾಗಿ ಹೇಗೆ ಮಾರ್ಪಡಿಸುವುದು ಎಂಬುದನ್ನು ಶ‍್ರೀಕೃಷ್ಣ ಕರ್ಮಯೋಗದಲ್ಲಿ ಬಹಳ ಚೆನ್ನಾಗಿ ವಿವರಿಸುವನು.

\textbf{ಧ್ಯಾನ:} ಧ್ಯಾನ ಎಂದರೆ ಭಗವಂತನಮೇಲೆ ಮನಸ್ಸನ್ನು ಏಕಾಗ್ರಮಾಡುವುದು. ಮನಸ್ಸಿನ ಏಕಾಗ್ರತೆಗೆ ಅದ್ಭುತವಾದ ಶಕ್ತಿ ಇದೆ. ಅದನ್ನು ಬಾಹ್ಯಪ್ರಪಂಚದ ಮೇಲೆ ಬೀರಿದಾಗ ಅದರ ರಹಸ್ಯವನ್ನು ತಿಳಿದುಕೊಳ್ಳುತ್ತೇವೆ. ವಿಜ್ಞಾನಿ ಭೌತಿಕಪ್ರಪಂಚದಮೇಲೆ ಏಕಾಗ್ರಮಾಡಿ ಅದರಲ್ಲಿರುವ ನಿಯಮಗಳನ್ನು ಕಂಡುಹಿಡಿದಿರುವನು. ಧ್ಯಾನಿ, ಅದನ್ನು ಅಂತರ್ಮುಖಮಾಡಿ ಭಗವಂತನ ಕಡೆಗೆ ಹರಿಸುವನು. ನದಿ ಹೇಗೆ ಸಾಗರವನ್ನು ಸೇರುವುದೋ ಹಾಗೆ ಅವನಲ್ಲಿ ಒಂದಾಗಲು ಯತ್ನಿಸುವನು.

ಧ್ಯಾನವೆಂದರೆ ಸುಮ್ಮನೆ ಕಣ್ಣುಮುಚ್ಚಿಕೊಂಡು ಕುಳಿತುಕೊಳ್ಳುವುದಲ್ಲ. ಧ್ಯಾನ ಜೀವನಕ್ಕೆ ಒಂದು ಭದ್ರವಾದ ಶುದ್ಧಚಾರಿತ್ರ್ಯದ ತಳಹದಿ ಇರಬೇಕು. ಆಗ ಮಾತ್ರ ಧ್ಯಾನ ಫಲಕಾರಿ ಆಗುವುದು. ಇಲ್ಲದೇ ಇದ್ದರೆ ಇಲ್ಲ. ಹಲವಾರು ಆಸೆ ಆಕಾಂಕ್ಷೆಗಳ ಬುದ್ಬುದಗಳು ಮನಸ್ಸಿನಲ್ಲಿ ಏಳುತ್ತಿರುವಾಗ ಅವನು ಭಗವಂತನ ಕಡೆಗೆ ಮನಸ್ಸನ್ನು ಹೇಗೆ ಏಕಾಗ್ರ ಮಾಡಬಲ್ಲ? ಆದಕಾರಣವೇ ಮೊದಲು ಅವನು ದೇಹ ಮನಸ್ಸು ಬುದ್ಧಿ, ಇಂದ್ರಿಯಗಳನ್ನು ನಿಗ್ರಹಿಸಬೇಕು. ಶುದ್ಧಚಾರಿತ್ರ್ಯವೇ ಅಸ್ತಿಭಾರ ಧ್ಯಾನದ ಕಟ್ಟಡವನ್ನು ಕಟ್ಟುವುದಕ್ಕೆ. ಅದಿಲ್ಲದೆ ಪ್ರಯತ್ನಿಸಿದರೆ ಮರಳಮೇಲೆ ಕಟ್ಟಿದ ಮನೆಯಂತಾಗುವುದು.

ಧ್ಯಾನಮಾಡುವುದಕ್ಕೆ ನಿರ್ಜನಪ್ರದೇಶ ಆವಶ್ಯಕ. ಇಲ್ಲದೆ ಇದ್ದರೆ ಬಾಹ್ಯಪ್ರಪಂಚದ ಗದ್ದಲ ಮತ್ತು ನೋಟಗಳು ಯೋಗಿಯ ಮನಸ್ಸನ್ನು ಪ್ರಪಂಚದ ಕಡೆಗೆ ಎಳೆಯುವುವು. ಅವನು ಏಕಾಕಿಯಾಗಿರಬೇಕು. ಅವನು ಚಿತ್ತ ಮತ್ತು ಇಂದ್ರಿಯಗಳನ್ನು ನಿಗ್ರಹಿಸಿರಬೇಕು, ಆಶಾರಹಿತ\-ನಾಗಿರಬೇಕು, ಅಪರಿಗ್ರಹಿಯಾಗಿರಬೇಕು. ಯಾವಾಗಲೂ ಎಂದರೆ, ಅವನು ಕುಳಿತುಕೊಂಡು ಧ್ಯಾನ ಮಾಡದೇ ಇರುವಾಗಲೂ ಮನಸ್ಸಿನ ಒಂದು ಭಾಗ ಭಗವಂತನನ್ನು ಕುರಿತು ಚಿಂತಿಸುತ್ತಿರಬೇಕು. ಅವನು ಧ್ಯಾನಕ್ಕೆ ಕುಳಿತುಕೊಳ್ಳುವ ಸ್ಥಳ ಶುಚಿಯಾಗಿರಬೇಕು. ಕುಳಿತುಕೊಳ್ಳವ ಜಾಗ ಸ್ಥಿರವಾಗಿರಬೇಕು. ಅದು ಅತಿ ಮೇಲೂ ಇರಕೂಡದು ಅಥವಾ ಅತಿ ಕೆಳಗೂ ಇರಕೂಡದು. ಕುಶ, ಚರ್ಮ, ಬಟ್ಟೆಯನ್ನು ಹಾಸಿ ಅದರ ಮೇಲೆ ಧ್ಯಾನಕ್ಕೆ ಕುಳಿತುಕೊಳ್ಳಬೇಕು.

ಕುಳಿತುಕೊಳ್ಳುವಾಗ ಶರೀರ ಶಿರಸ್ಸು ಮತ್ತು ಕತ್ತುಗಳು ನೇರವಾಗಿರಬೇಕು. ಬೆನ್ನಿನ ಮೂಳೆ ಅಥವಾ ತಲೆ ಬಾಗಿದ್ದರೆ ಕ್ರಮೇಣ ನಿದ್ರೆಗೆ ಹೋಗಬೇಕಾಗುವುದು. ಮೂಗಿನ ತುದಿಯನ್ನೇ ನೋಡುತ್ತಾ ದಿಕ್ಕುಗಳನ್ನು ನೋಡದೆ ಪ್ರಶಾಂತಾತ್ಮನೂ ನಿರ್ಭಯನೂ ಆಗಿ ಬ್ರಹ್ಮಚಾರಿವ್ರತದಲ್ಲಿ ನಿರತನಾಗಿ ಮನಸ್ಸನ್ನು ಸಂಯಮಮಾಡಿ ಪರಮಾತ್ಮನಲ್ಲೇ ಚಿತ್ತವುಳ್ಳವನಾಗಿರಬೇಕು.

ಯೋಗಿಯ ಜೀವನ, “ಮಿತಿಮೀರಿ ತಿನ್ನುವವನಿಗೂ ಅಲ್ಲ. ತಿನ್ನದೆ ಇರುವವನಿಗೂ ಅಲ್ಲ. ಆಹಾರ ವಿಹಾರದಲ್ಲಿ ಮಿತವಾಗಿರಬೇಕು. ಆಗ ಯೋಗ ದುಃಖ ನಾಶಕವಾಗುವುದು.\enginline{” (vi,}೧೭)

ಶ‍್ರೀಕೃಷ್ಣ ಒಂದು ಮಧ್ಯ ಮಾರ್ಗವನ್ನು ಹೇಳುವನು. ಒಬ್ಬ ದೇಹಾಸಕ್ತನಾಗಿರಬಾರದು ಮತ್ತು ದೇಹವನ್ನು ನಿರ್ಲಕ್ಷಿಸಲೂ ಕೂಡದು. ದೋಣಿ ಬಲವಾಗಿದ್ದರೆ ನದಿಯನ್ನು ದಾಟಬಹುದು. ಅದು ದುರ್ಬಲವಾಗಿದ್ದರೆ ನದೀಪಾಲಾಗುವೆವು. ಆದರೆ ದೋಣಿಯನ್ನು ಚೆನ್ನಾಗಿಡುವುದೇ ಗುರಿಯಲ್ಲ. ಅದನ್ನು ಮತ್ತಾವುದೋ ಒಂದು ಉದ್ದೇಶಕ್ಕಾಗಿ ಇಡಬೇಕಾಗಿದೆ.

ಈ ಸಮಯದಲ್ಲಿ ಅರ್ಜುನ ನಮ್ಮ ಪರವಾಗಿಯೋ ಎಂಬಂತೆ ಕೆಲವು ಪ್ರಶ್ನೆಗಳನ್ನು ಹಾಕುತ್ತಾನೆ. “ಮನಸ್ಸು ಚಂಚಲವಾದದ್ದು, ಕ್ಷೋಭೆಯನ್ನುಂಟುಮಾಡುವುದು. ಬಲವುಳ್ಳದ್ದು ಮತ್ತು ದೃಢವಾದದ್ದು. ಅದರ ನಿಗ್ರಹ ವಾಯುವಿನಂತೆ ಅತ್ಯಂತ ಕಷ್ಟವೆಂದು ನಾನು ಭಾವಿಸುತ್ತೇನೆ.\enginline{” (iv,}೩೪)

ಅನೇಕ ವೇಳೆ ಧ್ಯಾನಮಾಡುವುದಕ್ಕೆ ಪ್ರಯತ್ನಿಸಿದಾಗಲೆ ಅದು ಎಂದಿಗಿಂತ ಹೆಚ್ಚು ದಂಗೆ ಏಳುವುದು. ಎಷ್ಟು ಅದನ್ನು ನಿಗ್ರಹಿಸುವುದಕ್ಕೆ ಪ್ರಯತ್ನಪಟ್ಟರೂ ಅದು ನಮ್ಮ ಕೈಗೆ ಸಿಕ್ಕುವುದಿಲ್ಲ. ಪ್ರಯತ್ನ\-ಮಾಡಿ ಸೋತುಹೋಗುತ್ತೇವೆ. ಅಂತಹ ಸಮಯದಲ್ಲಿ ಏನು ಮಾಡಬೇಕು ಎಂದು ಅರ್ಜುನ ಕೇಳಿದಾಗ ಶ‍್ರೀಕೃಷ್ಣ ಹೀಗೆ ಹೇಳುತ್ತಾನೆ.

\newpage

“ಮನಸ್ಸನ್ನು ನಿಗ್ರಹಿಸುವುದು ಕಷ್ಟ ಮತ್ತು ಅದು ಚಂಚಲವಾದದ್ದು ಎಂಬುದು ನಿಸ್ಸಂಶಯ, ಅಭ್ಯಾಸ ಮತ್ತು ವೈರಾಗ್ಯದಿಂದ ಇದನ್ನು ನಿಗ್ರಹಿಸಬಹುದು.\enginline{” (iv,} ೩೬)

ಶ‍್ರೀಕೃಷ್ಣ ಅಭ್ಯಾಸವನ್ನು ಒತ್ತಿ ಹೇಳುತ್ತಾನೆ. ಇದನ್ನೇ ಅಭ್ಯಾಸಯೋಗವೆಂದೂ ಹಲವು ಕಡೆ ಹೇಳಿದ್ದಾನೆ. ಅಭ್ಯಾಸವೆಂದರೆ ಮನಸ್ಸನ್ನು ನಿಗ್ರಹಿಸುವುದಕ್ಕೆ ಪದೇ ಪದೇ ಪ್ರಯತ್ನಮಾಡುವುದು. ಒಂದೇ ಸಲ ಅದು ಸಿದ್ಧಿಸುವುದಿಲ್ಲ. ಕೇಳಿ ಕೇಳಿದ್ದನ್ನೆಲ್ಲಾ ಮನಸ್ಸಿಗೆ ಇದುವರೆಗೆ ಕೊಟ್ಟು ಅದನ್ನು ಮುದ್ದಿಸಿ ಕೂಡಿಸಿರುವೆವು. ಈಗ ಅದು ಕೇಳಿದ್ದನ್ನು ಕೊಡುವುದಿಲ್ಲವೆಂದರೆ ಮೊದಮೊದಲು ಹಟಮಾಡಿಯೇ ಮಾಡುವುದು. ಆದರೆ ಅದಕ್ಕೆ ನಾವು ಬಗ್ಗಕೂಡದು. ಭಂಡರಾಗಿ ಅದಕ್ಕೆ ವಿರೋಧವಾಗಿ ಹೋಗಬೇಕು. ಮೊದಮೊದಲು ಅದು ಗೊಣಗಾಡುವುದು. ಆದರೆ ಯಾವಾಗ\break ನಾವು ಭಂಡರು ಎನ್ನುವುದು ಗೊತ್ತಾಗುವುದೊ ಆಗ ಅದು ಕ್ರಮೇಣ ಸುಮ್ಮನಾಗುವುದು. ಹೊಸದಾಗಿ ಕುದುರೆಯ ಮೇಲೆ ಕುಳಿತುಕೊಂಡರೆ, ಮೇಲೆ ಕುಳಿತುಕೊಂಡವನನ್ನು ಕುದುರೆ ಕೆಳಗೆ ಉರುಳಿಸಲು ಯತ್ನಿಸುವುದು. ಆದರೆ ಮೇಲೆ ಕುಳಿತವನು ಇಳಿಯದೇ ಇದ್ದರೆ ಅನಂತರ ಅವನು ಹೇಳಿದಂತೆ ಕೇಳುವುದು. ಮನಸ್ಸು ಕೂಡ ಹೀಗೆಯೇ.

ಅರ್ಜುನ ಮತ್ತೊಂದು ಪ್ರಶ್ನೆಯನ್ನು ಕೇಳುತ್ತಾನೆ.

“ಶ್ರದ್ಧಾಯುಕ್ತನಾಗಿದ್ದರೂ (ಸಾಕಷ್ಟು) ಪ್ರಯತ್ನವನ್ನು ಮಾಡದೇ ಇದ್ದರೆ ಯೋಗದಿಂದ ಚಲಿಸಲ್ಪಟ್ಟ ಮನಸ್ಸುಳ್ಳವನಾಗಿ ಯೋಗದಲ್ಲಿ ಸಿದ್ಧಿಯನ್ನು ಪಡೆಯದೆ ಯಾವ ಗತಿಯನ್ನು ಹೊಂದುತ್ತಾನೆ?”

“ಬ್ರಹ್ಮಪ್ರಾಪ್ತಿಯ ಮಾರ್ಗದಲ್ಲಿ ಮುಗ್ಧನೂ ಉಭಯಭ್ರಷ್ಠನೂ ಆಶ್ರಯಹೀನನೂ ತುಂಡಾದ ಮೇಘದಂತೆ ನಾಶವಾಗುವುದಿಲ್ಲವೇ\enginline{?” (vi,} ೩೮, ೩೯)

ಇಲ್ಲಿ ಅರ್ಜುನ ಕೇಳುವ ಪ್ರಶ್ನೆ ನಮ್ಮಲ್ಲಿ ಏಳುವ ಭಾವನೆಯ ಮರುಧ್ವನಿಯಂತೆ ಇದೆ. ದೇವರ ಕಡೆ ಹೋಗಬೇಕೆಂದು ಇಹಪರಸೌಖ್ಯಗಳನ್ನು ಬಿಟ್ಟು ಸಾಧನೆ ಮಾಡುತ್ತಿರುವೆವು. ಆದರೆ ಹೋರಾಟದಲ್ಲಿರುವಾಗಲೇ ಮೃತ್ಯು ಬರುವುದು. ನಮ್ಮ ಗತಿ ಆಗ ಏನಾಗುವುದು? ತ್ರಿಶಂಕುವಿನ ಸ್ಥಿತಿಯಾಗುವುದಿಲ್ಲವೆ? ಇಹವನ್ನು ಬಿಟ್ಟಾಯಿತು, ಪರ ಸಿಗಲಿಲ್ಲ. ಆಗ ನಾವು ಮಾಡಿದ ಪ್ರಯತ್ನ ಏನಾಯಿತು ಎಂದು ಕೇಳುವೆವು. ಆಗಲೇ ಶ‍್ರೀಕೃಷ್ಣ ನಮಗೆಲ್ಲಾ ಭರವಸೆಯನ್ನು ನೀಡುವ ಉತ್ತರವನ್ನು ಕೊಡುವನು:

“ಅರ್ಜುನ, ಯೋಗಭ್ರಷ್ಠನಿಗೆ ಈ ಲೋಕದಲ್ಲಿ ನಾಶವಿಲ್ಲ. ಪರಲೋಕದಲ್ಲಿಯೂ ನಾಶವಿಲ್ಲ. ಕಲ್ಯಾಣ ಕಾರ್ಯಗಳನ್ನು ಮಾಡುವ ಯಾವನೂ ದುರ್ಗತಿ ಹೊಂದುವುದಿಲ್ಲ. ಯೋಗಭ್ರಷ್ಠನು ಪುಣ್ಯ ಶಾಲಿಗಳ ಲೋಕವನ್ನು ಹೊಂದಿ ಅಲ್ಲಿ ಅನೇಕ ವರ್ಷಗಳವರೆಗೆ ವಾಸವಾಗಿದ್ದು ಅನಂತರ ಸದಾಚಾರ ಶೀಲರೂ ಶ‍್ರೀಮಂತರೂ ಆಗಿರುವವರ ಮನೆಯಲ್ಲಿ ಹುಟ್ಟುವನು. ಅಥವಾ ಧೀಮಂತ\-ರಾದ ಯೋಗಿಗಳ ಕುಲದಲ್ಲಿ ಹುಟ್ಟುವನು. ಇಂತಹ ಜನ್ಮ ಲೋಕದಲ್ಲಿ ದುರ್ಲಭತರ\-ವಾದದ್ದು. ಅಲ್ಲಿ ಪೂರ್ವಜನ್ಮಗಳ ಸಂಬಂಧವಾದ ಬುದ್ಧಿಯೋಗವನ್ನು ಹೊಂದಿ ಸಂಸಿದ್ಧಿಗಾಗಿ ಪ್ರಯತ್ನಪಡುವನು. ಪೂರ್ವಾಭ್ಯಾಸದಿಂದ ಅಸ್ವತಂತ್ರನಾಗಿ ಅವನು ಎಳೆಯಲ್ಪಡುತ್ತಾನೆ.

“ಪ್ರಯತ್ನದಿಂದ ಅಭ್ಯಾಸ ಮಾಡುತ್ತಿರುವವನೂ ಕಲ್ಮಷರಹಿತನೂ ಅನೇಕ ಜನ್ಮಗಳಿಂದ ಸಿದ್ಧಿಯನ್ನು ಹೊಂದಿದವನೂ ಆದ ಯೋಗಿ, ಅನಂತರ ಪರಮಗತಿಯನ್ನು ಹೊಂದುತ್ತಾನೆ.\enginline{” (vi,}೪ಂ\-ರಿಂದ ೪೫)

ಆಧ್ಯಾತ್ಮಿಕ ಜೀವನ ಒಂದು ಮಹಾ ಸಾಹಸಯಾತ್ರೆ. ಇದು ಮನಸ್ಸಿನ ಗೌರೀಶಂಕರ ಶಿಖರವನ್ನು ಏರುವ ಸಾಹಸ. ಇದು ಕೆಲವು ವರ್ಷಗಳಲ್ಲಿ ಅಥವಾ ಒಂದು ಜನ್ಮದಲ್ಲೇ ಮುಗಿಯುವ ಸಾಹಸವಲ್ಲ. ಇದಕ್ಕಾಗಿ ಜನ್ಮ ಜನ್ಮಗಳನ್ನು ಮುಡುಪಾಗಿ ಇಡಬೇಕು. ನಾವು ಹಿಂದೆ ಪ್ರಪಂಚವನ್ನು ಅನುಭವಿಸುವು ದಕ್ಕಾಗಿ ಮರೀಚಿಕಾಮಯವಾದ ಸುಖವನ್ನು ಬೆನ್ನಟ್ಟಿಕೊಂಡು ಹೋಗುವುದರಲ್ಲಿ ಲೆಕ್ಕವಿಲ್ಲದಷ್ಟು ಜನ್ಮಗಳನ್ನೇ ಕಳೆದಿರುವೆವು. ಅದರಿಂದ ಬಂದ ಹೀನ ಸಂಸ್ಕಾರಗಳ ರಾಶಿಯೇ ಈಗಿನ ನಮ್ಮ ಮನಸ್ಸಿನ ಉಗ್ರಾಣದಲ್ಲಿದೆ. ಈಗ ಅವುಗಳಿಗೆಲ್ಲಾ ವಿರೋಧವಾಗಿ ಹೋರಾಡಬೇಕಾಗಿದೆ. ತತ್ ಕ್ಷಣವೇ ನಮಗೆ ಜಯ ಸಿಕ್ಕುವುದಿಲ್ಲ. ಸಿಕ್ಕದೇ ಇದ್ದರೆ ಏನು ಪ್ರಯೋಜನ ಎಂದು ಕೇಳುತ್ತೇವೆ. ಸಂಪೂರ್ಣ ನಮ್ಮ ಮನಸ್ಸಿನ ನಿಗ್ರಹ ನಮಗೆ ದೊರಕದೆ ಇರಬಹುದು. ಅದಕ್ಕಾಗಿ ಹೋರಾಡುವಾಗ ನಾವು ಉತ್ತಮ ಸಂಸ್ಕಾರಗಳನ್ನು ಸಂಗ್ರಹಿಸುತ್ತಾ ಹೋಗುವೆವು. ಕೆಟ್ಟ ಸಂಸ್ಕಾರ ಕಡಿಮೆಯಾಗುತ್ತಾ ಬರುವುದು; ಕ್ರಮೇಣ ಮನಸ್ಸು ಪರಿಶುದ್ಧವಾಗಿ ಏಕಾಗ್ರವಾಗಿ ಭಗವಂತನಲ್ಲಿ ಸಂಲಗ್ನವಾಗುವ ಸಮಯ ಬರುವುದು,

ಇಲ್ಲಿ ನಷ್ಟವೆಂಬುದಿಲ್ಲ. ನಾವು ಒಬ್ಬ ಮಾನವನ ಹತ್ತಿರ ಊಳಿಗ ಮಾಡಿದರೆ ಅವನು ಎಷ್ಟೇ ಜಿಪುಣನಾದರೂ ನಮಗೆ ಏನನ್ನಾದರೂ ಕೊಡದೆ ಕಳುಹಿಸುವುದಿಲ್ಲ. ದೇವರು ದಯಾಮಯ. ನಮ್ಮ ದೌರ್ಬಲ್ಯ ಅವನಿಗೆ ಗೊತ್ತು. ಅವನ ಹೆಸರಿನಲ್ಲಿ ನಾವು ಸಾಧನೆ ಮಾಡಲು ಪ್ರಯತ್ನಿಸಿ ಗುರಿಯನ್ನು ಮುಟ್ಟದೇ ಇದ್ದರೆ ಅವನು ನಮ್ಮನ್ನು ಮರೆಯುವುದಿಲ್ಲ, ಗುರಿಯೆಡೆಗೆ ನಡೆಸಿಕೊಂಡು ಹೋಗು ವನು. ‘ನನ್ನ ಭಕ್ತ ಎಂದಿಗೂ ನಾಶವಾಗುವುದಿಲ್ಲ’ ಎಂಬ ಭರವಸೆಯನ್ನು ಕೊಡುತ್ತಾನೆ.

\textbf{ಭಕ್ತಿ:} ಮುಕ್ಕಾಲುಪಾಲು ಜನ ಮಾನವರು ಹೋಗುವ ದಾರಿಯೇ ಭಕ್ತಿ. ಏಕೆಂದರೆ ಭಕ್ತಿ ಮಾನವರಲ್ಲಿ ಸಹಜವಾಗಿರುವುದು. ದೇವರನ್ನು ನಾವು ಯಾವುದಾದರೂ ಆಕಾರದ ಮೂಲಕ ಪ್ರೀತಿಸಬಹುದು. ಯಾವ ಹೆಸರಿನ ಮೂಲಕ ಬೇಕಾದರೂ ಕರೆಯಬಹುದು. ನಾವು ಯಾವ ಆಕಾರದ ಮೂಲಕ ಪ್ರೀತಿಸಿದರೆ ಅವನು ಆ ಆಕಾರದ ಮೂಲಕ ಒಲಿಯುತ್ತಾನೆ. ಪ್ರತಿಯೊಬ್ಬರಿಗೂ ಆಯಾ ಆಕಾರದಲ್ಲಿರುವ ಭಕ್ತಿಯನ್ನು ದೇವರು ದೃಢಮಾಡುವನು. ಶ‍್ರೀಕೃಷ್ಣ ಇಂತಹ ಆಕಾರ ಮೇಲು ಇಂತಹ ಆಕಾರ ಕೀಳು ಎನ್ನುವ ತಂಟೆಗೇ ಹೋಗುವುದಿಲ್ಲ. ನಮ್ಮ ಶ್ರದ್ಧೆ ಭಕ್ತಿಗೆ ತಕ್ಕಂತೆ ಪ್ರತಿಫಲ ದೊರಕುವುದು.

ಯಾವಾಗ ನಾವು ಅವನನ್ನು ಸಾಕಾರವೆಂದು ಭಾವಿಸುವೆವೋ ಆಗ ಅವನನ್ನು ಮಾನವ ಸಹಜವಾದ ರೀತಿಯಲ್ಲಿ ಪ್ರೀತಿಸಬೇಕಾಗಿದೆ. ಅವನನ್ನು ತಂದೆ ಎಂದು ಕರೆಯಬಹುದು. ತಾಯಿ ಎಂದು ಕರೆಯಬಹುದು, ಸ್ವಾಮಿ ಎಂದೂ ಮಗುವೆಂದೂ ನೋಡಬಹುದು. ಯಾವ ಮಾನವೀಯ ಭಾವದ ಮೂಲಕ ಬೇಕಾದರೂ ಅವನೆಡೆಗೆ ಹೋಗಬಹುದು. ಅವನಿಗೆ ನಾವು ಏನನ್ನು ಬೇಕಾದರೂ ಕೊಡಬಹುದು. “ಯಾವನು ನನಗೆ ಎಲೆಯನ್ನಾಗಲಿ, ಹಣ್ಣನ್ನಾಗಲಿ, ನೀರನ್ನಾಗಲಿ ಭಕ್ತಿಯಿಂದ ಕೊಡುತ್ತಾನೆಯೊ ಅಂತಹ ಶುದ್ಧ ಚಿತ್ತವುಳ್ಳವನು ಭಕ್ತಿಯಿಂದ ಅರ್ಪಿಸಿದ ಅದನ್ನು ನಾನು ಸ್ವೀಕರಿಸುತ್ತೇನೆ.\enginline{” (ix,}೨೬) ನಾವು ಅವನಿಗೆ ಏನು ಕೊಡುತ್ತೇವೆಯೊ ಅದನ್ನಲ್ಲ ಗಮನಿಸುವುದು; ಶ್ರದ್ಧೆ ಭಕ್ತಿಯಿಂದ ಕೊಡುತ್ತೇವೆಯೆ ಎಂಬುದನ್ನು ಗಮನಿಸುತ್ತಾನೆ.

ಭಕ್ತಿಯ ಮಾರ್ಗದಲ್ಲಿ ಹೊರಟವರಲ್ಲಿ ಎಲ್ಲರೂ ಒಂದೇ ಮೆಟ್ಟಲಲ್ಲಿ ಇಲ್ಲ. ಒಬ್ಬ ಮೇಲಿರು ವನು. ಮತ್ತೊಬ್ಬ ಕೆಳಗಿರುವನು. ಆದರೆ ಶ‍್ರೀಕೃಷ್ಣ ಯಾರನ್ನೂ ನಿಕೃಷ್ಟವಾಗಿ ಕಾಣುವುದಿಲ್ಲ. ಅವರೆಲ್ಲ ತನ್ನೆಡೆಗೆ ಬಂದರಲ್ಲ ಅದಕ್ಕಾಗಿ ಅವರನ್ನು ಉದಾರಿಗಳೇ ಎನ್ನುತ್ತಾನೆ. “ಆರ್ತ, ಜಿಜ್ಞಾಸು, ಅರ್ಥಾರ್ಥಿ, ಜ್ಞಾನಿ ಎಂಬ ನಾಲ್ಕು ಬಗೆಯ ಜನರು ನನ್ನನ್ನು ಭಜಿಸುತ್ತಾರೆ. ಅವರಲ್ಲಿ ನಿತ್ಯಭಕ್ತನೂ ಏಕಭಕ್ತಿ ಯುಳ್ಳವನೂ ಆದ ಜ್ಞಾನಿ ಶ್ರೇಷ್ಠ. ನಾನು ಜ್ಞಾನಿಗೆ ಅತ್ಯಂತ ಪ್ರಿಯ, ಜ್ಞಾನಿ ನನಗೆ ಅತ್ಯಂತ ಪ್ರಿಯ. ಆದರೂ ಇವರೆಲ್ಲರೂ ಉದಾರಿಗಳೇ ಸರಿ.\enginline{” (vii,} ೧೬, ೧೭ )

ಮೊದಲನೆಯವನೆ ಆರ್ತ. ಜೀವನದಲ್ಲಿ ಸಂಕಟದ ಸುಳಿಗೆ ಸಿಲುಕಿರುವವನು ಅವನು. “ಸಂಕಟ ಬಂದಾಗ ವೆಂಕಟರಮಣ” ಎಂಬ ಗಾದೆ ಅಕ್ಷರಶಃ ನಿಜ. ಮತ್ತೆ ಯಾವ ಕಾಲದಲ್ಲಿ ದೇವರೆಡೆಗೆ ಹೋಗದೆ ಇದ್ದರೂ ಸಂಕಟದ ಬೇಟೆ ನಾಯಿಗಳು ಇವನನ್ನು ಅಟ್ಟಿಸಿಕೊಂಡು ಬಂದಾಗ ದೇವರ ಸಮೀಪಕ್ಕೆ ಓಡುವನು. ದೇವರು ಇವನ ಸಂಕಟವನ್ನು ಕೇಳಿ ಸಮಾಧಾನ ಮಾಡುವನು. ಏನನ್ನು ಕೇಳಿದರೂ ಅವನು ಇಲ್ಲ ಎನ್ನುವುದಿಲ್ಲ. ಅವನೊಂದು ಕಲ್ಪತರು. ಅದರ ಕೆಳಗೆ ನಿಂತು ನಾವು ಏನು ಪ್ರಾರ್ಥಿಸಿದರೂ ಅವನು ಇಲ್ಲವೆನ್ನುವುದಿಲ್ಲ. ಭಕ್ತರಲ್ಲಿ ಈ ಗುಂಪಿಗೆ ಸೇರಿದವರೇ ಬಹುಮಂದಿ.

ಎರಡನೆಯ ಗುಂಪಿಗೆ ಸೇರಿದವನೆ ಧನಾಕಾಂಕ್ಷಿ. ಅವನಿಗೆ ಧನಬೇಕು. ದೇವರು ಕಲ್ಪತರು ಎಂದು ಕೇಳಿರುವನು. ಅದಕ್ಕೇ ಅವನ ಬಳಿಗೆ ಬರುವನು. ವ್ಯಾಪಾರಾದಿ ಉದ್ಯಮಗಳಲ್ಲಿ ಮುಂದೆ ಬಂದರೆ ಹಲವು ಉತ್ಸವಗಳನ್ನು ಮಾಡಿಸುವೆನು ಎನ್ನುತ್ತಾನೆ. ಇವನು ದೇವರನ್ನು ಪ್ರಾರ್ಥಿಸುತ್ತಾನೆ. ಅವನೇನಾದರೂ ಬಂದರೆ ಹಣ ನೀಡೆಂದು ತನ್ನ ಭಿಕ್ಷಾಪಾತ್ರೆಯನ್ನು ಒಡ್ಡುತ್ತಾನೆ.

ಮೂರನೆ ಗುಂಪಿಗೆ ಸೇರಿದವನೆ ಜಿಜ್ಞಾಸು. ಇವನದು ವಿಚಾರ ಪ್ರಧಾನವಾಗಿರುವ ಪ್ರಕೃತಿ. ಈ ಪ್ರಪಂಚದಲ್ಲಿ ದೇವರು ಎಂಬ ಚೈತನ್ಯವೇನಾದರೂ ಇದೆಯೆ, ಇದ್ದರೆ ಅದು ಹೇಗಿದೆ ಎಂಬುದನ್ನು ತಿಳಿದುಕೊಳ್ಳಬಯಸುವನು. ಅವನಿಗೆ ಸಂಕಟವಿಲ್ಲ, ಧನದ ಆಕಾಂಕ್ಷೆ ಇಲ್ಲ. ಮನಸ್ಸಿನಲ್ಲಿ ಎದ್ದಿರುವ ಸಂದೇಹವನ್ನು ಪರಿಹರಿಸಿಕೊಳ್ಳಲು ಬಂದಿರುವನು.

ನಾಲ್ಕನೆಯವನೆ ಜ್ಞಾನಿ. ಅವನು ದೇವರೆಡೆಗೆ ಬರುತ್ತಾನೆ. ಅವನು ದೇವರ ಮೇಲಿನ ಪ್ರೇಮಕ್ಕಾಗಿ ಬರುತ್ತಾನೆ. ಅವನು ದೇವರಿಂದ ಲೌಕಿಕವಾಗಿರುವುದೇನನ್ನೂ ವಸೂಲಿ ಮಾಡುವುದಕ್ಕೆ ಬರುವುದಿಲ್ಲ. ಈ ಪ್ರಪಂಚದಲ್ಲಿ ದೇವರೊಬ್ಬನೆ ಸತ್ಯ ಎಂಬುದನ್ನು ಅರಿತಿರುವನು. ಅವನೊಬ್ಬ ಸಿಕ್ಕಿದರೆ ಉಳಿದೆಲ್ಲವೂ ಸಿಕ್ಕಿದಂತೆ. ದೇವರಿಗೆ ಇಂತಹ ಭಕ್ತರನ್ನು ಕಂಡರೆ ಪ್ರೀತಿ.

ಈ ನಾಲ್ಕು ಜನರಲ್ಲಿ ಕೊನೆಯ ಗುಂಪಿಗೆ ಸೇರಿದವನೆ ಶ್ರೇಷ್ಠ. ಆದರೆ ಉಳಿದವರನ್ನು ನಿಕೃಷ್ಟವಾಗಿ ನೋಡುವುದಿಲ್ಲ. ಅವರೆಲ್ಲರೂ ಉದಾರಿಗಳೇ ಎನ್ನುವನು. ಏಕೆಂದರೆ ಅವನೆದುರಿಗೆ ಇಡೀ ಮಾನವ ಕೋಟಿಯ ವಿಕಾಸದ ಏಣಿ ಇದೆ. ಈಗ ಸಂಕಟದಿಂದ ಪಾರಾಗಲು ದೇವರೆಡೆಗೆ ಬಂದವನೇ ಇನ್ನು ಕೊಂಚ ಕಾಲದ ಮೇಲೆ ಅವನನ್ನು ತಿಳಿಯಲು ಇಚ್ಛಿಸುತ್ತಾನೆ. ಅವನೆ ಇನ್ನು ಸ್ವಲ್ಪ ಕಾಲದ ಮೇಲೆ ಭಗವಂತನನ್ನು ಪ್ರೀತಿಗಾಗಿ ಪ್ರೀತಿಸುವ ಪರಾಭಕ್ತನಾಗುತ್ತಾನೆ. ಇವರು ತಮ್ಮ ಇಚ್ಛೆಯನ್ನು ಪೂರೈಸಿಕೊಳ್ಳುವುದಕ್ಕೆ ಬೇರೆ ಬೇರೆ ಕಡೆಗೆ ಹೋಗುವುದನ್ನು ಬಿಟ್ಟು ನನ್ನ ಸಮೀಪಕ್ಕೆ ಬಂದರಲ್ಲ ಎಂದು ಅದಕ್ಕೇ ಇವರನ್ನೆಲ್ಲ ಉದಾರಿಗಳು ಎನ್ನುತ್ತಾನೆ. ಅಂತೂ ಹೇಗೋ ಯಾವುದರಿಂದಲಾದರೂ ಪ್ರೇರಿತರಾಗಿ ಬರಲಿ. ಅವನನ್ನು ಕ್ರಮೇಣ ಮೇಲಕ್ಕೆ ಎತ್ತಲು ಕಾದು ಕುಳಿತಿರುವನು ಪರಮಾತ್ಮ.


\section*{ಅವತಾರ}

ಭಕ್ತನಾದ ಸಾಧಕನಿಗೆ ಒಂದು ಅವತಾರದ ಊರಗೋಲು ಬೇಕು. ಅದಿಲ್ಲದೆ ಇದ್ದರೆ ಆಧ್ಯಾತ್ಮಿಕ ಜೀವನದಲ್ಲಿ ನಡೆದುಕೊಂಡು ಹೋಗುವುದು ಬಹಳ ಕಷ್ಟವಾಗುವುದು. ಅದಕ್ಕಾಗಿಯೇ ಭಕ್ತರಿಗೆ ಅವನು ಅವತಾರವನ್ನು ಒದಗಿಸುವನು. ಅವನು ಬರುವುದಕ್ಕೆ ಕಾರಣವನ್ನು ಹೇಳುತ್ತಾನೆ.

“ಯಾವಾಗ ಧರ್ಮದ ಅವನತಿ ಆಗುವುದೊ, ಅಧರ್ಮದ ಉನ್ನತಿ ಆಗುವುದೊ, ಆಗ ನಾನು ಅವತಾರ ಮಾಡುತ್ತೇನೆ. ಸಾಧುಗಳ ರಕ್ಷಣೆ ಮತ್ತು ದುಷ್ಟರ ನಾಶಕ್ಕೆ ಮತ್ತು ಧರ್ಮ ಸಂಸ್ಥಾಪನೆ ಮಾಡುವುದಕ್ಕೆ ನಾನು ಯುಗ ಯುಗಗಳಲ್ಲಿ ಅವತಾರ ಮಾಡುತ್ತೇನೆ.\enginline{” (iv}, ೭, ೮)

ಯಾವಾಗ ದುರಿತಗಳಲ್ಲಿ ಸಿಲುಕಿ ಮಾನವ ದೇವರಿಗೆ ಪ್ರಾರ್ಥನೆ ಮಾಡುತ್ತಾನೆಯೋ ಆಗ ಅವನು ಮೇಲಿನಿಂದ ಕೆಳಗೆ ಇಳಿದುಬರುವನು. ಹಾಗೆ ಇಳಿದುಬರುವಾಗ ಯಾವುದಾದರೂ ಚಾರಿತ್ರಿಕ ವ್ಯಕ್ತಿಯಂತೆ ಅವತರಿಸುವನು.

ಸರ್ವವ್ಯಾಪಿಯಾದ ಸರ್ವಶಕ್ತನಾಗಿರುವ ದೇವರು ಹೇಗೆ ಒಂದು ಅವತಾರದಂತೆ ಇಳಿದು ಬರಲು ಸಾಧ್ಯ, ಆಗ ಅನಂತ ಸಾಂತವಾಗುವುದಿಲ್ಲವೇ ಎಂದು ನಾವು ತರ್ಕಿಸಬಹುದು. ಆದರೆ ಅವನು ಅವತಾರವಾಗಿ ಇಳಿದುಬರುತ್ತಾನೆ ಎಂದರೆ ಅವನು ತನ್ನ ಸರ್ವವ್ಯಾಪಿತ್ವ ಮತ್ತು ಸರ್ವಜ್ಞತ್ವ ಮುಂತಾದುವುಗಳನ್ನು ಕಳೆದುಕೊಳ್ಳುವುದಿಲ್ಲ. ಅವನ ಯಾವುದೋ ಒಂದು ಸಣ್ಣ ಅಂಶ ಅವತಾರವಾಗಿ ಇಳಿದುಬರುವುದು. ಬಹು ಅಂಶ ಇದಕ್ಕೆ ಅತೀತವಾಗಿರುವುದು. ಸಾಗರದ ನೀರಿನಲ್ಲಿ\break ಯಾವುದೋ ಕಡೆ ಸ್ವಲ್ಪ ಗಟ್ಟಿಯಾಗಿ ಮಂಜುಗೆಡ್ಡೆಯಂತೆ ಮೇಲೆದ್ದರೆ, ಸಾಗರವೆಲ್ಲಾ ಹಾಗೆ ಆಗಿದೆಯೇ? ಇಲ್ಲ. ಅಂತೆಯೇ ಅವತಾರ. ಎಲ್ಲಿಯವರೆಗೆ ಮಾನವ ಮಾನವನಾಗಿರುವನೋ ಅಲ್ಲಿಯ\-ವರೆಗೆ ಹಲವರು ಭಗವಂತನನ್ನು ಕುರಿತು ಯೋಚಿಸಬೇಕಾದರೆ ಒಂದು ಮಾನವಾಕೃತಿ ಮೂಲಕ ಮಾತ್ರ ಸಾಧ್ಯ. ಯಾವುದಾದರೂ ಒಂದು ಆಕಾರದ ಮೂಲಕ ಅವನೆಡೆಗೆ\break ಹೋಗಬೇಕಾಗಿದೆ.

“ಅವನ ಅವತಾರ ದಿವ್ಯವಾದುದು. ಅವನು ಮಾಡುವ ಕರ್ಮ ದಿವ್ಯವಾದುದು. ಅಂತಹ ಅವತಾರವನ್ನು ಯಾರು ಯಥಾರ್ಥವಾಗಿ ತಿಳಿಯುವರೊ ಅವರು ದೇಹವನ್ನು ತ್ಯಾಗಮಾಡಿದ\-ಮೇಲೆ ಪುನರ್ಜನ್ಮವನ್ನು ಹೊಂದುವುದಿಲ್ಲ. ಅವರು ನನ್ನನ್ನು ಹೊಂದುವರು.\enginline{” (iv,}೯) ಯಾವುದಾದರೂ ಒಂದು ಅವತಾರದಮೇಲೆ ಶ್ರದ್ಧೆ, ಭಕ್ತಿ ಇಟ್ಟು ಸಾಧನೆ ಮಾಡಿದರೆ ಸಾಕು. ಅವನು ನಮ್ಮನ್ನು ತನ್ನೆಡೆಗೆ ಸೆಳೆದುಕೊಳ್ಳುವನು. ಗಂಗಾ ನದಿಗೆ ಇರುವ ಹಲವು ಘಾಟುಗಳಂತೆ ಅವತಾರ. ಯಾವ ಘಾಟಿನಲ್ಲಿ ಸ್ನಾನಮಾಡಿದರೂ, ಗಂಗೆಯಲ್ಲಿಯೇ ಸ್ನಾನ ಮಾಡಿದಂತೆ ಆಗುವುದು. ಭಕ್ತನಿಗೆ ಯಾವಯಾವ ಭಗವಂತನ ಆಕಾರದ ಮೇಲೆ ಶ್ರದ್ಧೆ ಇದೆಯೋ ಅದನ್ನು ಮತ್ತೂ ಊರ್ಜಿತ ಮಾಡುವನೇ ಹೊರತು, ಎಂದಿಗೂ ಅದನ್ನು ಕಿತ್ತುಹಾಕುವುದಿಲ್ಲ. “ಯಾರು ನನ್ನನ್ನು ಹೇಗೆ ಆಶ್ರಯಿಸುತ್ತಾರೆಯೋ ಅವರನ್ನು ಹಾಗೆಯೇ ನಾನು ಅನುಗ್ರಹಿಸುತ್ತೇನೆ. ಅರ್ಜುನ, ಮನುಷ್ಯರು ಎಲ್ಲಾ ಪ್ರಕಾರಗಳಿಂದಲೂ ನನ್ನ ಪಥವನ್ನೇ ಅನುಸರಿಸುತ್ತಾರೆ.\enginline{” (iv,}೧೧) ದೇವರು ಕೇವಲ ಅವತಾರಗಳಲ್ಲಿ ಮಾತ್ರ ವ್ಯಕ್ತವಾಗುವುದಿಲ್ಲ. ಈ ಪ್ರಪಂಚದಲ್ಲಿ ಯಾವ ಯಾವ ಶ್ರೇಷ್ಠವಾದ ವ್ಯಕ್ತಿಗಳಿರುವರೊ ಅವರ ಹಿಂದೆ ಇರುವುದೆಲ್ಲ ಒಂದೇ ಭಗವಂತನ ಪ್ರಭೆಯ ಅಂಶ. “ಯಾವ ಯಾವ ವಸ್ತು ವಿಭೂತಿಯುಳ್ಳದ್ದೊ, ಶ‍್ರೀಯುಕ್ತವೊ ಅಥವಾ ಊರ್ಜಿತವಾಗಿದೆಯೋ ಅವೆಲ್ಲ ನನ್ನ ತೇಜಸ್ಸಿನ ಒಂದು ಅಂಶದಿಂದ ಉಂಟಾಯಿತು ಎಂದು ತಿಳಿ.\enginline{” (x,}೧ಂ) ಇಲ್ಲಿ ಶ‍್ರೀಕೃಷ್ಣ ಹೇಳುವುದು ಕೇವಲ ಹಿಂದುಗಳಿಗೆ ಮಾತ್ರ ಅನ್ವಯಿಸುವುದಿಲ್ಲ. ಇಡೀ ಮಾನವ ಕುಲಕ್ಕೆಲ್ಲ ಅನ್ವಯಿಸುವುದು. ಇಂತಹ ಉದಾರ ವಾಣಿಯನ್ನು ನಾವು ಕೇಳುವುದು ಬಹಳ ಅಪರೂಪ ಧಾರ್ಮಿಕ ಜಗತ್ತಿನಲ್ಲಿ.

ಅವತಾರವಾದವನ್ನು ಎಲ್ಲರೂ ನಂಬಬೇಕೆಂಬ ಬಲಾತ್ಕಾರವಿಲ್ಲ. ಅದೊಂದು ದುರ್ಬಲ ಮನಸ್ಸಿ ನವರಿಗೆ ದೇವರು ಕೊಟ್ಟಿರುವ ಊರುಗೋಲು. ಅದರ ಸಹಾಯವನ್ನು ಬೇಕಾದರೆ ಒಬ್ಬ ತೆಗೆದುಕೊಳ್ಳಬಹುದು. ನನಗೆ ಯಾವ ಊರುಗೋಲು ಬೇಕಾಗಿಲ್ಲ ಎನ್ನುವ ಧೀರಮತಿಗೆ ಬೇಕಾದಷ್ಟು ಅವಕಾಶವಿದೆ ಗೀತೆಯ ಬೋಧನೆಯಲ್ಲಿ. ಯಾರು ಹೇಗೆ ಬೇಕಾದರೂ ಅವನೆಡೆಗೆ ಹೋಗಬಹುದು. ದಾರಿಗಳು ಬೇರೆ ಬೇರೆ ಆಗಿರಬಹುದು. ಆದರೆ ಸೇರುವ ಗುರಿಯೆಲ್ಲ ಒಂದೇ.


\section*{ತತ್ತ್ವ}

ಹಿಂದೂಗಳಲ್ಲಿ ಮೂರು ಮುಖ್ಯ ತತ್ತ್ವದೃಷ್ಟಿಗಳಿವೆ. ಅವೇ ದ್ವೈತ, ವಿಶಿಷ್ಟಾದ್ವೈತ ಮತ್ತು ಅದ್ವೈತ ಎಂಬುವು. ದ್ವೈತ ಎಂದರೆ ಜೀವ ಜಗತ್ ಈಶ್ವರ ಬೇರೆ ಬೇರೆ ಎಂದು ಭಾವಿಸುವುದು. ವಿಶಿಷ್ಟಾದ್ವೈತ ಎಂದರೆ ಇರುವುದು ಒಂದೇ, ಅದರ ಯಾವುದೋ ಸಣ್ಣ ಅಂಶದಿಂದ ಜೀವ ಜಗತ್ ಆಗಿವೆ ಎಂದು ಒಪ್ಪಿಕೊಳ್ಳುವುದು. ಅವನು ಪೂರ್ಣ, ಜೀವಗಳೆಲ್ಲ ಅಂಶಗಳು, ಸಾಗರ ಮತ್ತು ಅದರ ಅಲೆಗಳಂತೆ. ಅದ್ವೈತ ಎಂದರೆ ಇರುವುದು ಒಂದೇ. ಅದೇ ಹಲವರಂತೆ ಕಾಣುವುದು. ಈ ಜೀವನ ನಿಜವಾದ ಸ್ವಭಾವವೆ ಬ್ರಹ್ಮ. ಈಗ ಅಜ್ಞಾನದಲ್ಲಿರುವುದರಿಂದ ಬೇರೆ ಎಂದು ಭಾವಿಸುವನು. ಜ್ಞಾನ ಬಂದಾದ ಮೇಲೆ ತಾನು ಯಾವಾಗಲೂ ಅದೇ ಆಗಿದ್ದೆನು ಎಂದು ತಿಳಿಯುವನು.

ಮೇಲಿನ ಮೂರು ತತ್ತ್ವಗಳನ್ನು ಕ್ರಮವಾಗಿ ಮಧ್ವಾಚಾರ್ಯರು, ರಾಮಾನುಜಾಚಾರ್ಯರು ಹಾಗೂ ಶಂಕರಾಚಾರ್ಯರು ಪ್ರತಿಪಾದಿಸಿರುವರು. ತಾವು ಪ್ರತಿಪಾದಿಸುವ ತತ್ತ್ವ ಆಗಲೆ ಇದೆ ಎಂಬುದನ್ನು ತೋರುವುದಕ್ಕೆ ಪ್ರಸ್ಥಾನತ್ರಯಗಳಿಗೆ ಭಾಷ್ಯವನ್ನು ಬರೆದಿರುವರು. ಪರಸ್ಪರ ವಿರೋಧವಾಗಿರುವ ತತ್ತ್ವಗಳೆಲ್ಲ ಒಂದೇ ಶಾಸ್ತ್ರದಲ್ಲಿ ಹೇಗೆ ಇರಲು ಸಾಧ್ಯ? ಇದರಲ್ಲಿ ಯಾವುದೊ ಒಂದು ಸರಿ ಇರಬೇಕು, ಅಥವಾ ನಾವು ತಪ್ಪುತಿಳಿದುಕೊಂಡಿರಬೇಕು ಎಂದು ಭಾವಿಸುವುದು ರೂಢಿ. ಆದರೆ ಗೀತೆಯಲ್ಲಿ ಯಾವುದೋ ಒಂದು ತತ್ತ್ವವನ್ನು ಹೇಳುವುದಕ್ಕೆ ಶ‍್ರೀಕೃಷ್ಣ ಹೊರಟಿಲ್ಲ. ಅಲ್ಲಿ ಹಲವು ತತ್ತ್ವಗಳಿವೆ. ಒಂದು ಮತ್ತೊಂದಕ್ಕೆ ವಿರೋಧವಾಗಿಲ್ಲ. ಒಂದು\break ಮತ್ತೊಂದಕ್ಕೆ ಪೂರಕವಾಗಿದೆ. ಯಾರು ತಮಗೆ ಯಾವುದು ಬೇಕೋ ಅದನ್ನು ಆರಿಸಿಕೊಳ್ಳಬಹುದು. ಒಂದು ನಿಜವಾದರೆ ಉಳಿದವು ಸುಳ್ಳಾಗಿರಬೇಕೆಂದು ಅಲ್ಲ. ಒಂದೇ ಸತ್ಯ ನಾವು ನೋಡುವ ದೃಷ್ಟಿಭೇದಕ್ಕೆ ತಕ್ಕಂತೆ ವಿಧವಿಧವಾಗಿ ಕಾಣುವುದು, ಒಂದೇ ಚಾಮುಂಡಿ ಬೆಟ್ಟ\break ಒಂದೊಂದು ದಿಕ್ಕಿನಿಂದ ಒಂದೊಂದು ರೀತಿಯಲ್ಲಿ ಕಾಣುವಂತೆ.

ಒಂದು ನದಿ ದೂರದ ಬೆಟ್ಟದಲ್ಲಿ ಹುಟ್ಟಿ ಒಂದು ಹೆಸರಿನಿಂದ ಹರಿದುಕೊಂಡು ಹೋಗುತ್ತಿದೆ. ಅದು ತನ್ನ ಪ್ರಯಾಣವನ್ನೆಲ್ಲ ಪೂರೈಸಿ ಸಾಗರವನ್ನು ಸೇರುತ್ತಿದೆ. ಸ್ವಲ್ಪ ದೂರ ಹೋದಮೇಲೆ ಸಾಗರದಲ್ಲಿ ಒಂದಾಗುವುದು. ಇಲ್ಲಿ ನದಿಯ ಒಂದು ಗತಿ ಮತ್ತೊಂದಕ್ಕೆ ಪೋಷಕವಾಗಿದೆಯೇ ಹೊರತು, ಎಂದಿಗೂ ವಿರೋಧವಾಗಿಲ್ಲ. ನದಿ ದೂರದಲ್ಲಿರುವಾಗ ನದಿ ಬೇರೆ, ಸಾಗರ ಬೇರೆ, ಅದು ಹರಿಯುವ ಪಾತ್ರ ಬೇರೆ. ಅದು ಸ್ವತಂತ್ರವಾಗಿ ಹರಿದುಕೊಂಡು ಹೋಗುತ್ತಿದೆ. ಅದಕ್ಕೂ ಸಾಗರಕ್ಕೂ ಯಾವ ಸಂಬಂಧವೂ ಇನ್ನೂ ಆಗಿಲ್ಲ. ಅದರೆ ಅದೇ ನದಿ ಸಾಗರವನ್ನು ಸೇರುತ್ತಿರುವಾಗ, ಒಂದರಲ್ಲಿ ಮತ್ತೊಂದು ಇರುವುದನ್ನು ನೋಡುತ್ತೇವೆ. ನದಿಯಲ್ಲಿ ಸಾಗರವಿದೆ, ಸಾಗರದಲ್ಲಿ ನದಿ ಇದೆ. ಆದರೆ ಕೆಲವು ಗಜಗಳು ಮುಂದೆ ಹೋದರೆ ನದಿ ಮಾಯವಾಗಿ ಸಾಗರದಲ್ಲಿ ಒಂದಾಗುವುದು. ನಾವು ನದಿಯ ಯಾವುದೋ ಒಂದು ಅವಸ್ಥೆಯನ್ನು ಮಾತ್ರ ತೆಗೆದುಕೊಂಡು, ಅದೇ ಸತ್ಯ ಎಂದರೆ ಪೂರ್ಣ ಸತ್ಯವಾಗುವುದಿಲ್ಲ. ಯಾವುದು ಅಂಶ ಸತ್ಯಗಳನ್ನು ನಿರಾಕರಿಸದೇ ಅದಕ್ಕೆ ಒಂದು ಸ್ಥಾನವನ್ನು ಕೊಟ್ಟು ವಿವರಿಸುವುದೊ ಅದೇ ಪೂರ್ಣ ಸತ್ಯ. ಅದು ಯಾವುದನ್ನೂ ಧಿಕ್ಕರಿಸುವುದಿಲ್ಲ. ಎಲ್ಲವನ್ನೂ ಸ್ವೀಕರಿಸುವುದು.

ಗೀತೆಯಲ್ಲಿ ದ್ವೈತ ಇದೆ, ವಿಷ್ಟಾದ್ವೈತ ಇದೆ, ಅದ್ವೈತ ಇದೆ ಮತ್ತು ನಮಗೆ ಗೊತ್ತಿಲ್ಲದ ಇನ್ನೂ ಎಷ್ಟೋ ದೃಷ್ಟಿಕೋಣಗಳಿವೆ. ಗೀತೆಯಲ್ಲಿ ಎಲ್ಲಾಬಗೆಯ ತತ್ತ್ವಗಳನ್ನೂ ನೋಡುತ್ತೇವೆ. ಎಲ್ಲಾ ತತ್ತ್ವಗಳ ಸಂಗಮ ಸ್ಥಾನ ಅದು. ಯಾವುದೋ ಒಂದನ್ನು ಸರಿ ಮತ್ತೊಂದು ತಪ್ಪು ಎನ್ನುವ ಗೋಜಿಗೆ ಹೋಗದೆ, ಯಾರು ಯಾವ ಸಂಪ್ರದಾಯದಲ್ಲಿ ಹುಟ್ಟಿರುವರೊ ಅವನು ಅದನ್ನು ಅನುಷ್ಠಾನ ಮಾಡಲಿ, ಉಳಿದವರನ್ನು ದೂರದೆ ಇರಲಿ ಎಂಬುದನ್ನು ಗೀತೆಯಲ್ಲಿ ಕಲಿಯುತ್ತೇವೆ. ಅನೇಕ ವೇಳೆ ನಾವು ನಮ್ಮ ತತ್ತ್ವಕ್ಕೆ ಗೌರವ ತೋರುವುದು ಇತರರ ತತ್ತ್ವಗಳನ್ನು ದೂರವುದರಿಂದ. ನಮ್ಮದನ್ನು ನಾವು ಭಕ್ತಿ ಗೌರವಗಳಿಂದ ಅನುಷ್ಠಾನಮಾಡೋಣ, ಅದರಂತೆಯೇ ಇತರರಿಗೂ ಅವರವರ ತತ್ತ್ವಕ್ಕೆ ಹಾಗೆ ಮಾಡಲು ಅವಕಾಶವನ್ನು ಕೊಡೋಣ. ಸತ್ಯಕ್ಕೆ ಗೊತ್ತಿದೆ ಎಲ್ಲರೂ ನನ್ನನ್ನೇ ಕರೆಯುತ್ತಿರುವರು, ವಿವಿಧಮಾರ್ಗಗಳಿಂದ ತನ್ನೆಡೆಗೇ ಬರುತ್ತಿರುವರು ಎಂಬುದು. “ನನಗಿಂತ ಶ್ರೇಷ್ಠವಾದ ವಸ್ತು ಬೇರೊಂದಿಲ್ಲ. ದಾರದಲ್ಲಿ ಮಣಿಗಳು ಪೋಣಿಸಲ್ಪಟ್ಟಿರುವ ಹಾಗೆ ಇದೆಲ್ಲವೂ ನನ್ನಲ್ಲಿ ಪೋಣಿಸಲ್ಪಟ್ಟಿವೆ.\enginline{” (vii,}೭)


\section*{ಅಧಿಕಾರ, ವಿಚಾರ, ಅನುಭವ}

ಶ‍್ರೀಕೃಷ್ಣನು ಅರ್ಜುನನಿಗೆ ಅಧಿಕಾರವಾಣಿಯಿಂದ ಹೇಳುವನು. ಯಾವ ವಿಷಯಗಳನ್ನು ಅವನಿಗೆ ಹೇಳುತ್ತಿರುವನೊ ಅದರಲ್ಲಿ ಸಂಶಯಕ್ಕೆ ಆಸ್ಪದವೇ ಇಲ್ಲ. ನಾವು ಕೂಡ ಮುಂಚೆ ವಿಷಯಗಳನ್ನು ಅಧಿಕಾರಿಗಳಾದವರಿಂದ ಕೇಳಬೇಕು. ಅಂತಹ ಮಹಾತ್ಮರು ಬರೆದಿರುವ ಗ್ರಂಥಗಳನ್ನು ಓದಬೇಕು. ಯಾರು ತಾತ್ತ್ವಿಕ ಜೀವನದಲ್ಲಿ ನಮಗಿಂತ ಮುಂದೆ ಹೋಗಿರುವರೊ ಅವರಿಂದ ವಿಷಯಗಳನ್ನು ತಿಳಿದುಕೊಳ್ಳಬೇಕು. ಅಂತಹ ಆಪ್ತವಾಕ್ಯ ದಾರಿಯಲ್ಲಿ ಮಿನುಗುತ್ತಿರುವ ದೀವಿಗೆಯಂತೆ.

ಆದರೆ ಅದನ್ನು ಕುರಿತು ವಿಚಾರವನ್ನೇ ಮಾಡಕೂಡದು ಎಂದು ಶ‍್ರೀಕೃಷ್ಣ ಹೇಳುವುದಿಲ್ಲ. ಯಾವುದನ್ನು ನಾವು ಓದುತ್ತೇವೆಯೊ, ಕೇಳುತ್ತೇವೆಯೊ ಅದನ್ನು ಕುರಿತು ವಿಚಾರಮಾಡಬೇಕು. ಅದನ್ನು ಚೆನ್ನಾಗಿ ತಿಳಿದುಕೊಳ್ಳುವುದಕ್ಕೆ ಪ್ರಯತ್ನಪಡಬೇಕು. ಆಗಲೆ ಸಮಸ್ಯೆಯನ್ನು ಎಲ್ಲಾ ದೃಷ್ಟಿ ಕೋಣಗಳಿಂದಲೂ ನೋಡಲು ಸಾಧ್ಯವಾಗುವುದು. ಎಂತಹ ಮಹಾನುಭಾವ ಹೇಳಲಿ, ಅವನ ಮೇಲೆ ನಮಗೆ ಎಷ್ಟೇ ಗೌರವವಿದ್ದರೂ ಅವನು ಹೇಳುವುದನ್ನು ನಾವು ಚೆನ್ನಾಗಿ ವಿಚಾರ ಮಾಡಿ ತೆಗೆದುಕೊಳ್ಳಬೇಕು. ಆಗಲೇ ಆ ಜ್ಞಾನ ನಮ್ಮದಾಗುವುದು. ನಮ್ಮ ಯುಕ್ತಿಗೆ ಸಮರ್ಪಕವಾಗಿಲ್ಲದೆ ಇದ್ದರೆ, ನಂಬಿದ ಜ್ಞಾನ ಪರೀಕ್ಷಾ ಸಮಯದಲ್ಲಿ ನಮ್ಮದಾಗುವುದಿಲ್ಲ. ಶ‍್ರೀಕೃಷ್ಣ ಜಗದ್ಗುರು, ಅನುಭವಗಳನ್ನೆಲ್ಲ ಅರ್ಜುನನಿಗೆ ಕೊಟ್ಟಾದಮೇಲೂ, “ಈಗ ನಿನ್ನ ಮನಸ್ಸಿಗೆ ಹೇಗೆ ತೋರುವುದೊ ಹಾಗೆ ಮಾಡು\enginline{” (xviii,}೭೩) ಎನ್ನುವನೇ ಹೊರತು ಬಲಾತ್ಕಾರ ಮಾಡುವುದಿಲ್ಲ.

ಅನಂತರವೇ ಅನುಭವ. ವಿಚಾರದ ಮೂಲಕ ನಾವು ಒಂದು ವಿಷಯ ಸಾಧ್ಯ ಎಂದು ತಿಳಿಯಬಹುದು, ಆದರೆ ಅದು ಅಲ್ಲೇ ನಿಲ್ಲಕೂಡದು. ಅದನ್ನು ನಮ್ಮ ಅನುಭವವನ್ನಾಗಿ ಮಾಡಿಕೊಳ್ಳಬೇಕು. ಇದನ್ನೇ ಶ‍್ರೀಕೃಷ್ಣ ಜ್ಞಾನ ಮತ್ತು ವಿಜ್ಞಾನವೆಂದು ಬೇರೆ ಬೇರೆ ಹೆಸರಿನಿಂದ ಕರೆಯುವುದು. ಜ್ಞಾನವೆಂದರೆ ಯುಕ್ತಿಯ ಆಧಾರದಮೇಲೆ ನಿಂತು ಇದು ನಿಜ ಎಂದು ಅರಿಯುವುದು. ಆದರೆ ಅನುಭವ ಇವನಿಗೆ ಇನ್ನೂ ಬಂದಿಲ್ಲ. ಶ‍್ರೀರಾಮಕೃಷ್ಣರು ಹೀಗೆ ಹೇಳುತ್ತಾರೆ. ಕೆಲವರು ಹಾಲನ್ನು ನೋಡಿದ್ದಾರೆ. ಕೆಲವರು ಅದನ್ನು ಕುಡಿದಿದ್ದಾರೆ. ನೋಡಿರುವವನು ವಿಚಾರವಾದಿ. ಅವನಿಗೆ ಅದು ಏನು ಎಂಬುದು ಗೊತ್ತು. ಅದನ್ನು ಕುಡಿದಿರುವವನು ವಿಜ್ಞಾನಿ. ಅದರಿಂದ ಪ್ರಯೋಜನ ಪಡೆದಿರುವನು. ಯುಕ್ತಿ ನಮ್ಮನ್ನು ಅನುಭವದ ಕಡೆಗೆ ಒಯ್ಯಬೇಕು.


\section*{ಸ್ವಪ್ರಯತ್ನ, ಶರಣಾಗತಿ}

\vskip -7.5pt

ಸ್ವಪ್ರಯತ್ನ ಮತ್ತು ಶರಣಾಗತಿ ಎಂಬವು ಪರಸ್ಪರ ವಿರೋಧವಾಗಿರುವ ಭಾವನೆಗಳು. ಒಂದು ಇದ್ದರೆ ಮತ್ತೊಂದು ಹೇಗೆ ಅದರ ಸಮೀಪದಲ್ಲೇ ಇರಬಹುದು ಎಂದು ನಮಗೆ ಆಶ್ಚರ್ಯವಾಗಬಹುದು. ಹೇಗೆ ಶ‍್ರೀಕೃಷ್ಣ ದ್ವೈತ ಅದ್ವೈತಗಳೆಂಬ ತತ್ತ್ವವನ್ನು ಗೀತೆಯಲ್ಲಿ ಹೊಂದಿಸಿರುವನೋ ಹಾಗೆಯೇ ಸ್ವಪ್ರಯತ್ನ ಮತ್ತು ಶರಣಾಗತಿಯನ್ನು ಹೊಂದಿಸಿರುವನು. ಎರಡೂ ಬೇರೆ ಬೇರೆಯಲ್ಲ. ಒಂದಾದಮೇಲೆ ಮತ್ತೊಂದು ಬರುವುದು. ಮೊದಲು ಹೂವು ಅನಂತರ ಕಾಯಿ ಬರುವಂತೆ. ಒಂದು ಮತ್ತೊಂದಕ್ಕೆ ವಿರೋಧವಾಗಿಲ್ಲ. ಒಂದು ಮತ್ತೊಂದಕ್ಕೆ ಪೂರಕವಾಗಿದೆ. ಹದ್ದು ಮೇಲೆ ಹೋಗುವುದಕ್ಕೆ ಮುಂಚೆ ರೆಕ್ಕೆಯನ್ನು ಬಡಿದುಕೊಂಡು ಏರುವುದು. ಮೇಲೆ ಹೋದಮೇಲೆ ಸುಮ್ಮನೆ ತನ್ನ ರೆಕ್ಕೆಗಳನ್ನು ಹರಡಿ ಬೀಸುವ ಗಾಳಿಯಿಂದ ಪ್ರಯೋಜನ ಪಡೆಯುವುದು. ಮುಂಚೆಯೇ ಅದನ್ನು ಮಾಡುವುದಕ್ಕೆ ಆಗುವುದಿಲ್ಲ. ನಮ್ಮಲ್ಲಿ ಒಂದು ಅಹಂಕಾರವಿರುವಾಗ ಅದರ ಮಿತಿಯನ್ನು ತಿಳಿದು ಕೊಳ್ಳಬೇಕಾದರೆ ಅದನ್ನು ಶ್ರಮದಿಂದ ಸಮೆಸಬೇಕು. ಆಗ ತನ್ನ ಮಿತಿ ಇದರಿಂದ ಚೆನ್ನಾಗಿ ಗೊತ್ತಾಗುವುದು. ಅನಂತರ ಶರಣಾಗತಿ ಬರುವುದು. ಒಂದು ಹಕ್ಕಿ ಹಡಗಿನ ಧ್ವಜಸ್ತಂಭದ ಮೇಲೆ ಕುಳಿತುಕೊಂಡು ಸಮುದ್ರದಮೇಲೆ ಹೋಗುತ್ತಿತ್ತು. ಅದಕ್ಕೆ ಕುಳಿತು ಬೇಜಾರಾಗಿ ತಾನು ಮುಂಚೆಯೇ ಹಾರಿಹೋಗಬೇಕೆಂದು ತೀರವನ್ನು ಹುಡುಕುವುದಕ್ಕೆ ನಾಲ್ಕು ದಿಕ್ಕುಗಳಿಗೂ ಹಾರಿಹೋಗಿ ಬಂತು. ಎಲ್ಲಿಯೂ ಸಮೀಪದಲ್ಲಿ ತೀರ ಕಾಣಲಿಲ್ಲ. ಅನಂತರ ಹಡಗು ಹೇಗೂ ಹೋಗುತ್ತಿದೆ. ಒಂದಲ್ಲ ಒಂದು ದಿನ ತೀರ ಸೇರಬಹುದು ಎಂದು ತೆಪ್ಪಗೆ ಕುಳಿತುಕೊಂಡಿತು. ಹಾಗೆಯೇ ಮುಂಚಿನ ಸ್ವಭಾವವನ್ನು ಶ‍್ರೀಕೃಷ್ಣ ಧ್ಯಾನಯೋಗದಲ್ಲಿ ಹೀಗೆ ವಿವರಿಸುವನು.

“ಆತ್ಮವನ್ನು ಆತ್ಮನಿಂದಲೇ ಉದ್ಧಾರ ಮಾಡಿಕೊಳ್ಳಬೇಕು. ಎಂದಿಗೂ ಆತ್ಮನನ್ನು ಕುಗ್ಗಿಸಿಕೊಳ್ಳ ಬಾರದು. ಆತ್ಮನೇ ತನ್ನ ಬಂಧು, ಆತ್ಮನೇ ತನ್ನ ಶತ್ರು.”

“ಯಾರು ಆತ್ಮನಿಂದ ಆತ್ಮನನ್ನು ಗೆದ್ದಿರುವನೊ ಅವನಿಗೆ ಆತ್ಮನೇ ಬಂಧು. ಯಾರು ಆತ್ಮನನ್ನು ಗೆದ್ದಿಲ್ಲವೊ ಅವನಿಗೆ ಆತ್ಮನೇ ಶತ್ರು.\enginline{” (vi,} ೫, ೬)

ಇಲ್ಲಿ ಶ‍್ರೀಕೃಷ್ಣ ಸ್ವಪ್ರಯತ್ನವನ್ನು ಒತ್ತಿ ಹೇಳುತ್ತಾನೆ. ಜೀವನು ಮುಂಚೆ ಪ್ರಚಂಡ ಪ್ರಯತ್ನ ಮಾಡಬೇಕು. ಸುಮ್ಮನೆ ಕುಳಿತುಕೊಳ್ಳಕೂಡದು. ಅಹಂಕಾರವನ್ನು ಸ್ವಪ್ರಯತ್ನದಲ್ಲಿ ಚೆನ್ನಾಗಿ ಪಳಗಿಸಬೇಕು. ಆಗಲೇ ಅದು ಶುದ್ಧವಾಗುವುದು. ಅನಂತರವೇ ಭಗವಂತನ ಕೃಪೆ ಸಹಾಯಕ್ಕೆ ಬರಬೇಕಾದರೆ. ಶುದ್ಧ ಸೋಮಾರಿಗೆ ಭಗವಂತ ಸಹಾಯಕ್ಕೆ ಬರುವುದಿಲ್ಲ.

ಶ‍್ರೀಕೃಷ್ಣ ಗೀತಾಸಂದೇಶವನ್ನು ಮುಗಿಸುವುದಕ್ಕೆ ಮುಂಚೆ ಶರಣಾಗತಿಯ ಚರಮ ಮಂತ್ರವನ್ನು ಬೋಧಿಸುತ್ತಾನೆ. “ಸರ್ವ ಧರ್ಮಗಳನ್ನು ಪರಿತ್ಯಾಗ ಮಾಡಿ ನನ್ನೊಬ್ಬನಲ್ಲಿ ಶರಣು ಹೊಂದು. ನಾನು ನಿನ್ನನ್ನು ಸರ್ವ ಪಾಪಗಳಿಂದಲೂ ಪಾರು ಮಾಡುತ್ತೇನೆ. ಶೋಕಿಸಬೇಡ.\enginline{” (xviii,} ೬೭)

ರೆಕ್ಕೆ ದಣಿದಮೇಲೆ ಹಕ್ಕಿ ಕುಳಿತುಕೊಂಡು ದೋಣಿಯಲ್ಲಿ ಶರಣಾಗುವುದು. ಹಾಗೆಯೇ ಸ್ವಪ್ರಯತ್ನದಿಂದ ಶ್ರಮಪಟ್ಟವನಿಗೆ ಶರಣಾಗತಿಯ ಅಮೃತ ರುಚಿಸಬಲ್ಲುದು. ಇಲ್ಲಿ ಒಂದು ಮತ್ತೊಂದಕ್ಕೆ ವಿರೋಧವಾಗಿಲ್ಲ. ಒಂದು ಮತ್ತೊಂದಕ್ಕೆ ಒಯ್ಯುವುದು. ಆಯಾ ಅಂತಸ್ತಿನಲ್ಲಿ ಅದೇ ನಿಜ.


\section*{ಭರವಸೆಯ ಧ್ರುವತಾರೆ}

“ಅತ್ಯಂತ ದುರಾಚಾರಿಯಾಗಿದ್ದರೂ ಅನನ್ಯ ಭಕ್ತಿಯಿಂದ ನನ್ನನ್ನು ಭಜಿಸಿದರೆ ಅವನು ಸಾಧು ಎಂದೇ ತಿಳಿಯತಕ್ಕದ್ದು. ಏಕೆಂದರೆ ಅವನು ನಿಶ್ಚಯಬುದ್ಧಿಯುಳ್ಳವನಾಗಿರುತ್ತಾನೆ.

“ಅರ್ಜುನ, ಅವನು ಬೇಗ ಧರ್ಮಾತ್ಮನಾಗುತ್ತಾನೆ ಮತ್ತು ಶಾಶ್ವತವಾದ ಶಾಂತಿಯನ್ನು ಹೊಂದುತ್ತಾನೆ. ನನ್ನ ಭಕ್ತ ನಾಶವಾಗುವುದಿಲ್ಲ, ತಿಳಿದುಕೊ.\enginline{” (ix,} ೩ಂ, ೩೧)

ಭಗವಂತ ಇಲ್ಲಿ ಮುಕ್ತಿಯ ಹೆಬ್ಬಾಗಿಲನ್ನು ಎಲ್ಲರಿಗೂ ತೆಗೆಯುತ್ತಾನೆ. ಒಬ್ಬ ಅತ್ಯಂತ ದುರಾಚಾರಿಯಾಗಿರಬಹುದು, ಭಗವಂತನಲ್ಲಿ ಶರಣಾದರೆ, ಕ್ಷಮೆಯನ್ನು ಯಾಚಿಸಿದರೆ, ಭಗವಂತ ಅವನನ್ನು ತ್ಯಜಿಸುವುದಿಲ್ಲ. ನಾವೆಲ್ಲ ಭಗವಂತನೆದುರಿಗೆ ತಪ್ಪಿತಸ್ಥರೆ. ಕೆಲವರು ಅದನ್ನು ಒಪ್ಪಿಕೊಂಡಿರುವರು, ಇನ್ನು ಕೆಲವರು ಒಪ್ಪಿಕೊಂಡಿಲ್ಲ. ಒಬ್ಬರ ಗುಟ್ಟು ರಟ್ಟಾಗಿದೆ, ಇನ್ನೊಬ್ಬರದು ಇನ್ನೂ ಆಗಿಲ್ಲ. ಒಂದು ಕಾಲದಲ್ಲಿ ಅಜ್ಞಾನದಲ್ಲಿದ್ದಾಗ ನಾವೆಲ್ಲ ತಪ್ಪನ್ನು ಮಾಡಿದವರೆ. ಮಾನವನ ಯಾತ್ರೆ ಅಜ್ಞಾನದಿಂದ ಜ್ಞಾನದ ಕಡೆ ಸಾಗುತ್ತಿದೆ. ಅಜ್ಞಾನದಲ್ಲಿರುವಾಗ ನಾವೆಲ್ಲಾ ಮಾಡಬಾರ\-ದುದನ್ನು ಮಾಡಿರುವೆವು. ಆದರೆ ಅದಕ್ಕಾಗಿ ಸುಮ್ಮನೆ ಕುಳಿತುಕೊಂಡು ಅಳಬೇಕಾಗಿಲ್ಲ. ಪಾಪವನ್ನು ತೊಳೆಯುವ ಗಂಗೆ ಹತ್ತಿರದಲ್ಲೇ ಹರಿಯುತ್ತಿರುವಳು. ಅವಳಲ್ಲಿ ಮಿಂದರೆ ಸಾಕು, ನಮ್ಮ ಪಾಪವನ್ನೆಲ್ಲಾ ತೊಳೆಯುವಳು. ನಮ್ಮನ್ನೆಲ್ಲಾ ಉದ್ಧಾರ ಮಾಡುವಳು. ನೀವು ಮಡಿಯಾಗಿ ನನ್ನಲ್ಲಿಗೆ ಬನ್ನಿ ಎಂದು ಹೇಳುವುದಿಲ್ಲ ಗಂಗೆ. ನೀವು ಹೇಗೆ ಇದ್ದೀರೊ ಹಾಗೆ ಬಂದರೆ ಸಾಕು, ನಾನು ನಿಮ್ಮನ್ನು ಉದ್ಧಾರ ಮಾಡುತ್ತೇನೆ ಎನ್ನುವಳು. ಭಗವಂತನ ಕೃಪೆಯ ಗಂಗಾನದಿ ಈಗಲೂ ಹರಿಯುತ್ತಿದೆ. ಅದು ಎಲ್ಲಾ ಪಾಪಿಗಳಿಗೂ ಅಜ್ಞರಿಗೂ ಒಂದು ಭರವಸೆಯನ್ನು ಕೊಡುತ್ತಿದೆ. “ನನ್ನ ಭಕ್ತ ಎಂದಿಗೂ ನಾಶವಾಗುವುದಿಲ್ಲ” ಎಂದು ಭಗವದ್ಗೀತೆಯನ್ನು ಹೇಳುವ ಅವತಾರ ಪುರುಷನಾದ ಶ‍್ರೀಕೃಷ್ಣನು ಅರ್ಜುನನನ್ನು ನಿಮಿತ್ತ ಮಾಡಿಕೊಂಡು ಮಾನವಕೋಟಿಗೆ ಸಾರುವನು. ರಾಮಾವತಾರದಲ್ಲಿ ವಿಭೀಷಣನನ್ನು ನಿಮಿತ್ತ ಮಾಡಿಕೊಂಡು “ಯಾವಾಗ ಒಬ್ಬ ನನ್ನಲ್ಲಿ ಶರಣಾಗುತ್ತಾನೆಯೋ ಅವನಿಗೆ ಎಲ್ಲಾ ಭೂತಗಳಿಂದಲೂ ಅಭಯವನ್ನು ಕೊಡುತ್ತೇನೆ, ಇದು ನನ್ನ ವ್ರತ” ಎಂದು ಸಾರುತ್ತಾನೆ. ಅವನ ಕೃಪೆ ಸದಾ ಕಾದಿರುವುದು ನಮ್ಮನ್ನು ಉದ್ಧರಿಸುವುದಕ್ಕೆ. ಅದನ್ನು ಹೇಗೆ ಪಡೆಯಬೇಕೆಂದು ಸಾರುವುದೇ ಗೀತೆ.

