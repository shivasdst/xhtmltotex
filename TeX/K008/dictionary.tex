\sethyphenation{kannada}{
ಂ
ಂಂಂ
ಂಕ್ಕೆ
ಅ
ಅಂಕು-ರಿ-ಸು-ವುವು
ಅಂಕುಶ
ಅಂಕು-ಶ-ದಂತೆ
ಅಂಕು-ಶ-ದಿಂದ
ಅಂಕೆ-ಯಿಂದ
ಅಂಗ
ಅಂಗ-ಗ-ಳಾ-ಗು-ತ್ತೇವೆ
ಅಂಗ-ಗ-ಳಿವೆ
ಅಂಗ-ಗಳು
ಅಂಗ-ಗ-ಳೆಲ್ಲ
ಅಂಗ-ಗ-ಳೆಲ್ಲಾ
ಅಂಗಡಿ
ಅಂಗ-ಡಿ-ಯಲ್ಲಿ
ಅಂಗ-ಡಿ-ಯ-ಲ್ಲಿ-ರು-ವು-ದ-ನ್ನೆಲ್ಲ
ಅಂಗ-ಡಿ-ಯ-ವನು
ಅಂಗ-ಡಿ-ವ-ನಿಗೇ
ಅಂಗಳ
ಅಂಗ-ವಾ-ಗಿ-ರು-ವರೋ
ಅಂಗ-ಸಾ-ಧನೆ
ಅಂಗ-ಹೀ-ನ-ನಾ-ಗಿ-ರ-ಬ-ಹುದು
ಅಂಗ-ಹೀ-ನ-ರಾದ
ಅಂಗ-ಹೀ-ನರು
ಅಂಗಾಂ-ಗ-ಗಳ
ಅಂಗಾಂ-ಗ-ಗಳು
ಅಂಗಾಂ-ಗ-ಗ-ಳೆಲ್ಲಾ
ಅಂಗಾಂ-ಗ-ಗಳೇ
ಅಂಗಾಂ-ಗದ
ಅಂಗಾಂ-ಗ-ವನ್ನು
ಅಂಗಾಂ-ಗ-ವನ್ನೂ
ಅಂಗಾಂ-ಗವು
ಅಂಗಾಂ-ಗ-ವೆ-ಲ್ಲವೂ
ಅಂಗಾ-ಗ-ಗಳನ್ನೆಲ್ಲ
ಅಂಗಾ-ಗ-ಗಳು
ಅಂಗಾ-ಗ-ಗ-ಳೆಲ್ಲ
ಅಂಗಿ
ಅಂಗಿಯ
ಅಂಗೀ-ರ-ಸನ
ಅಂಗುಲ
ಅಂಗು-ಲ-ವನ್ನು
ಅಂಗೋ-ಪಾಂಗ
ಅಂಗೋ-ಪಾಂ-ಗ-ಗ-ಳಿಗೆ
ಅಂಚಿನ
ಅಂಚೆ
ಅಂಚೆಯ
ಅಂಚೆ-ಯ-ವನು
ಅಂಜ-ದ-ವ-ನಾ-ಗಿ-ರ-ಬೇಕು
ಅಂಜ-ದ-ವನು
ಅಂಜದೆ
ಅಂಜನ
ಅಂಜ-ನ-ವನ್ನು
ಅಂಜ-ಬ-ಹುದು
ಅಂಜ-ಬೇ-ಕಾ-ಗಿಲ್ಲ
ಅಂಜ-ಬೇಕು
ಅಂಜಿ
ಅಂಜಿಕೆ
ಅಂಜಿ-ಕೆ-ಗಳ
ಅಂಜಿ-ಕೆ-ಗಳು
ಅಂಜಿ-ಕೆಗೆ
ಅಂಜಿ-ಕೆಗೇ
ಅಂಜಿ-ಕೆಯ
ಅಂಜಿ-ಕೆ-ಯನ್ನು
ಅಂಜಿ-ಕೆ-ಯಿಂದ
ಅಂಜಿ-ಕೆಯೂ
ಅಂಜಿ-ಕೆಯೇ
ಅಂಜಿ-ಕೊಂ-ಡಿ-ರು-ವರು
ಅಂಜಿ-ಕೊಂಡು
ಅಂಜಿ-ದರೆ
ಅಂಜಿ-ದಾ-ಗ-ಲೆಲ್ಲ
ಅಂಜಿ-ಸ-ಬ-ಹುದು
ಅಂಜಿ-ಸ-ಲಾ-ರದು
ಅಂಜಿ-ಸ-ಲಾ-ರರು
ಅಂಜಿ-ಸ-ಲಾ-ರವು
ಅಂಜಿ-ಸು-ತ್ತಿ-ರು-ವುದು
ಅಂಜು-ತ್ತಾನೆ
ಅಂಜು-ತ್ತಿ-ರು-ವನು
ಅಂಜು-ತ್ತೇವೆ
ಅಂಜು-ಬು-ರುಕ
ಅಂಜು-ಬು-ರು-ಕರು
ಅಂಜು-ವಂತೆ
ಅಂಜು-ವನು
ಅಂಜು-ವರು
ಅಂಜು-ವ-ವ-ನಲ್ಲ
ಅಂಜು-ವ-ವನು
ಅಂಜುವು
ಅಂಜು-ವು-ದಿಲ್ಲ
ಅಂಜು-ವು-ದಿ-ಲ್ಲವೊ
ಅಂಜು-ವುದು
ಅಂಟಿ
ಅಂಟಿ-ಕೊಂ-ಡಂ-ತಾ-ಗು-ವುದು
ಅಂಟಿ-ಕೊಂ-ಡಿದೆ
ಅಂಟಿ-ಕೊಂ-ಡಿ-ದ್ದರೂ
ಅಂಟಿ-ಕೊಂ-ಡಿ-ದ್ದೇವೆ
ಅಂಟಿ-ಕೊಂ-ಡಿ-ರ-ಬಾ-ರದು
ಅಂಟಿ-ಕೊಂ-ಡಿ-ರ-ವನೊ
ಅಂಟಿ-ಕೊಂ-ಡಿರು
ಅಂಟಿ-ಕೊಂ-ಡಿ-ರು-ತ್ತವೆ
ಅಂಟಿ-ಕೊಂ-ಡಿ-ರುವ
ಅಂಟಿ-ಕೊಂ-ಡಿ-ರು-ವಂತೆ
ಅಂಟಿ-ಕೊಂ-ಡಿ-ರು-ವನು
ಅಂಟಿ-ಕೊಂ-ಡಿ-ರು-ವನೋ
ಅಂಟಿ-ಕೊಂ-ಡಿ-ರು-ವರು
ಅಂಟಿ-ಕೊಂ-ಡಿ-ರು-ವರೋ
ಅಂಟಿ-ಕೊಂ-ಡಿ-ರು-ವು-ದಕ್ಕೆ
ಅಂಟಿ-ಕೊಂ-ಡಿ-ರು-ವು-ದಿಲ್ಲ
ಅಂಟಿ-ಕೊಂ-ಡಿ-ರು-ವು-ದಿ-ಲ್ಲವೋ
ಅಂಟಿ-ಕೊಂ-ಡಿ-ರು-ವುದು
ಅಂಟಿ-ಕೊಂ-ಡಿ-ರು-ವೆವು
ಅಂಟಿ-ಕೊಂ-ಡಿ-ರು-ವೆವೋ
ಅಂಟಿ-ಕೊಂ-ಡಿಲ್ಲ
ಅಂಟಿ-ಕೊಂಡು
ಅಂಟಿ-ಕೊ-ಳ್ಳದೆ
ಅಂಟಿ-ಕೊ-ಳ್ಳದೇ
ಅಂಟಿ-ಕೊ-ಳ್ಳ-ಲಾ-ರದು
ಅಂಟಿ-ಕೊ-ಳ್ಳು-ತ್ತವೆ
ಅಂಟಿ-ಕೊ-ಳ್ಳು-ತ್ತೇವೆ
ಅಂಟಿ-ಕೊ-ಳ್ಳುವ
ಅಂಟಿ-ಕೊ-ಳ್ಳು-ವನೊ
ಅಂಟಿ-ಕೊ-ಳ್ಳು-ವರು
ಅಂಟಿ-ಕೊ-ಳ್ಳು-ವು-ದಿಲ್ಲ
ಅಂಟಿ-ಕೊ-ಳ್ಳು-ವುದು
ಅಂಟಿಗೆ
ಅಂಟಿ-ನ-ಮೇಲೆ
ಅಂಟಿ-ರುವ
ಅಂಟಿಸಿ
ಅಂಟಿ-ಹೋ-ಗಿ-ಬಿ-ಡು-ವು-ದಿಲ್ಲ
ಅಂಟಿ-ಹೋ-ಗು-ವು-ದಿಲ್ಲ
ಅಂಟು
ಅಂಟು-ವು-ದಿಲ್ಲ
ಅಂಟೂ
ಅಂತ
ಅಂತಃ-ಕ-ರಣ
ಅಂತಃ-ಕ-ರ-ಣ-ದಲ್ಲಿ
ಅಂತಃ-ಪು-ರ-ದಲ್ಲಿ
ಅಂತಃ-ಸು-ಖಕ್ಕೆ
ಅಂತ-ಕಾಲೇ
ಅಂತಯೇ
ಅಂತ-ರಂ-ಗ-ದ-ಲ್ಲಿದೆ
ಅಂತ-ರ-ತ-ಮ-ವಾದ
ಅಂತ-ರ-ದಲ್ಲಿ
ಅಂತ-ರ-ದ-ಲ್ಲಿ-ರು-ವುದು
ಅಂತ-ರ-ರಾ-ಮ-ನಾ-ಗಿ-ರು-ವನೊ
ಅಂತ-ರ-ವನ್ನು
ಅಂತ-ರಾತ್ಮ
ಅಂತ-ರಾ-ರಾಮ
ಅಂತ-ರಾಳ
ಅಂತ-ರಾ-ಳಕ್ಕೆ
ಅಂತ-ರಾ-ಳ-ದಲ್ಲಿ
ಅಂತ-ರಾ-ಳ-ದ-ಲ್ಲಿಯೂ
ಅಂತ-ರಾ-ಳ-ದ-ಲ್ಲಿ-ರುವ
ಅಂತ-ರಾ-ಳ-ದ-ಲ್ಲಿ-ರು-ವು-ದ-ನ್ನೆಲ್ಲ
ಅಂತ-ರಾ-ಳ-ದಲ್ಲೆ
ಅಂತ-ರಾ-ಳ-ದಲ್ಲೆಲ್ಲ
ಅಂತ-ರಾ-ಳ-ದಲ್ಲೆಲ್ಲಾ
ಅಂತ-ರಾ-ಳ-ದಿಂದ
ಅಂತ-ರಾ-ಳ-ವನ್ನು
ಅಂತ-ರಿಂ-ದ್ರಿ-ಯ-ನಿ-ಗ್ರಹ
ಅಂತ-ರಿಕ
ಅಂತ-ರಿ-ಕ್ಷ-ದಲ್ಲಿ
ಅಂತ-ರಿ-ಕ್ಷ-ವನ್ನು
ಅಂತರ್
ಅಂತ-ರ್ಗ-ತ-ವಾಗಿ
ಅಂತ-ರ್ಗ-ತ-ವಾ-ಗಿದೆ
ಅಂತ-ರ್ಜ್ಯೋತಿ
ಅಂತ-ರ್ಜ್ಯೋ-ತಿಗೆ
ಅಂತ-ರ್ಜ್ಯೋ-ತಿ-ಯನ್ನು
ಅಂತ-ರ್ಜ್ಯೋ-ತಿ-ಯಾ-ಗಿ-ರು-ವನೊ
ಅಂತ-ರ್ಮುಖ
ಅಂತ-ರ್ಮು-ಖ-ತೆಗೆ
ಅಂತ-ರ್ಮು-ಖದ
ಅಂತ-ರ್ಮು-ಖ-ನಾಗಿ
ಅಂತ-ರ್ಮು-ಖ-ಮಾಡಿ
ಅಂತ-ರ್ಮು-ಖ-ರಾಗಿ
ಅಂತ-ರ್ಮು-ಖರು
ಅಂತ-ರ್ಮು-ಖ-ವಾಗಿ
ಅಂತ-ರ್ಮು-ಖ-ವಾ-ಗು-ವುದು
ಅಂತ-ರ್ಮು-ಖ-ವಾ-ಗು-ವುದೊ
ಅಂತ-ರ್ಮುಖಿ
ಅಂತ-ರ್ಯಾಮಿ
ಅಂತ-ರ್ಯಾ-ಮಿ-ಯಾ-ಗಿ-ರು-ವನು
ಅಂತ-ರ್ವಾ-ಣಿಯ
ಅಂತ-ರ್ವಾ-ಣಿ-ಯಂತೆ
ಅಂತ-ವಂತ
ಅಂತ-ವತ್ತು
ಅಂತ-ಸ್ತಿ-ನಲ್ಲಿ
ಅಂತ-ಸ್ಸುಖ
ಅಂತ-ಸ್ಸು-ಖಿ-ಯಾ-ಗಿ-ರು-ವನೊ
ಅಂತಹ
ಅಂತ-ಹ-ವ-ನಿಗೆ
ಅಂತ-ಹ-ವನು
ಅಂತ-ಹ-ವ-ರನ್ನು
ಅಂತ-ಹ-ವ-ರಿಂದ
ಅಂತ-ಹ-ವ-ರಿಗೆ
ಅಂತ-ಹ-ವರು
ಅಂತ-ಹ-ವರೂ
ಅಂತು
ಅಂತೂ
ಅಂತೆ
ಅಂತೆಯೇ
ಅಂತ್ಯ
ಅಂತ್ಯ-ಕಾ-ಲಕ್ಕೆ
ಅಂತ್ಯ-ಕಾ-ಲ-ದಲ್ಲಿ
ಅಂತ್ಯ-ಕಾ-ಲ-ದ-ಲ್ಲಿಯೂ
ಅಂತ್ಯಕ್ಕೆ
ಅಂತ್ಯ-ಗಳನ್ನು
ಅಂತ್ಯ-ಗ-ಳಿಲ್ಲ
ಅಂತ್ಯ-ದ-ವ-ರೆಗೆ
ಅಂತ್ಯ-ರ-ಹಿ-ತ-ವಾ-ದುದು
ಅಂತ್ಯ-ವನ್ನು
ಅಂತ್ಯ-ವಲ್ಲ
ಅಂತ್ಯ-ವಾ-ಗಲೀ
ಅಂತ್ಯ-ವಾದ
ಅಂತ್ಯ-ವಿದೆ
ಅಂತ್ಯ-ವಿಲ್ಲ
ಅಂತ್ಯವೂ
ಅಂತ್ಯ-ವೆಂ-ಬುದು
ಅಂತ್ಯವೇ
ಅಂಥ
ಅಂಥವ-ನನ್ನು
ಅಂಥವ-ನಾ-ಗು-ತ್ತಾನೆ
ಅಂಥವನು
ಅಂಥವನೇ
ಅಂಥವ-ರಿಗೆ
ಅಂಥವರು
ಅಂಥವರೂ
ಅಂಥವು
ಅಂದ
ಅಂದಂ-ದಿನ
ಅಂದಂದು
ಅಂದಕ್ಕೆ
ಅಂದರೆ
ಅಂದ-ವಾಗಿ
ಅಂದ-ವಾದ
ಅಂದಿಗೆ
ಅಂದಿಗೇ
ಅಂದಿ-ನಿಂದ
ಅಂದು
ಅಂದು-ಕೊಂ-ಡರೂ
ಅಂಧ
ಅಂಧ-ಕಾ-ರ-ದ-ಲ್ಲಿ-ರ-ಲಾರ
ಅಂಧ-ಕಾ-ರ-ದಿಂದ
ಅಂಧ-ವೃದ್ಧ
ಅಂಬ
ಅಂಬಿ-ಗನೇ
ಅಂಶ
ಅಂಶಕ್ಕೂ
ಅಂಶ-ಗಳನ್ನು
ಅಂಶ-ಗ-ಳಾ-ವುವು
ಅಂಶ-ಗಳು
ಅಂಶ-ಗಳೇ
ಅಂಶ-ದ-ವರೋ
ಅಂಶ-ದಿಂದ
ಅಂಶ-ದಿಂ-ದಲೇ
ಅಂಶ-ದೃಷ್ಟಿ
ಅಂಶ-ನಮ್ಮ
ಅಂಶ-ಮಾತ್ರ
ಅಂಶ-ವನ್ನು
ಅಂಶ-ವಾ-ಗಿ-ರ-ಬೇಕು
ಅಂಶ-ವಾ-ಗು-ವನು
ಅಂಶ-ವಾದ
ಅಂಶ-ವೆಂದು
ಅಂಶವೇ
ಅಂಶಾ-ವ-ತಾ-ರ-ಗ-ಳಿವೆ
ಅಂಶಾ-ವ-ತಾ-ರ-ಗಳು
ಅಕರ್ತೃ
ಅಕರ್ಮ
ಅಕ-ರ್ಮ-ಕ್ಕಿಂತ
ಅಕ-ರ್ಮಕ್ಕೆ
ಅಕ-ರ್ಮ-ಣಶ್ಚ
ಅಕ-ರ್ಮ-ದಲ್ಲಿ
ಅಕ-ರ್ಮ-ದ-ಲ್ಲಿಯೂ
ಅಕ-ರ್ಮ-ವನ್ನು
ಅಕ-ರ್ಮಿ-ಯಾ-ಗಿ-ದ್ದರೆ
ಅಕ-ಸ್ಮಾತ್
ಅಕ-ಸ್ಮಾ-ತ್ತಾಗಿ
ಅಕಾರ
ಅಕಾ-ರ-ದಷ್ಟು
ಅಕಾರ್ಯ
ಅಕಾ-ರ್ಯ-ಗಳ
ಅಕಾ-ರ್ಯ-ಗಳನ್ನು
ಅಕಾ-ರ್ಯ-ವೆಂ-ಬುದು
ಅಕೀ-ರ್ತಿಂ
ಅಕು-ಶಲ
ಅಕೌಂ-ಟಿ-ನ-ಲ್ಲಿ-ರು-ವಂತೆ
ಅಕೌಂ-ಟೆಂಟ್ಗೆ
ಅಕ್ಕ-ಪಕ್ಕ-ದಲ್ಲಿ
ಅಕ್ಕ-ಸಾ-ಲಿಗ
ಅಕ್ಕ-ಸಾ-ಲಿ-ಗನ
ಅಕ್ಕ-ಸಾ-ಲೆಯ
ಅಕ್ಕಿ
ಅಕ್ಕಿ-ಯಂತೆ
ಅಕ್ಕಿ-ಯನ್ನು
ಅಕ್ಕಿ-ಯನ್ನೇ
ಅಕ್ಕಿಯೇ
ಅಕ್ಬ-ರನ
ಅಕ್ಬರ್
ಅಕ್ರೋಧ
ಅಕ್ಷಮ್ಯ
ಅಕ್ಷ-ಮ್ಯ-ವಾದ
ಅಕ್ಷ-ಯ-ಪಾ-ತ್ರೆ-ಯಂ-ತಿ-ರುವ
ಅಕ್ಷ-ಯ-ವಾದ
ಅಕ್ಷ-ಯ-ವಾ-ದುದು
ಅಕ್ಷರ
ಅಕ್ಷರಂ
ಅಕ್ಷ-ರ-ಕ್ಕಿಂತ
ಅಕ್ಷ-ರಕ್ಕೆ
ಅಕ್ಷ-ರ-ಗಳಲ್ಲಿ
ಅಕ್ಷ-ರ-ಗ-ಳ-ಲ್ಲೆಲ್ಲ
ಅಕ್ಷ-ರ-ಗಳಿಂದ
ಅಕ್ಷ-ರ-ಗ-ಳಿ-ಗಿಂತ
ಅಕ್ಷ-ರ-ಗಳು
ಅಕ್ಷ-ರ-ದಂತೆ
ಅಕ್ಷ-ರ-ದಲ್ಲಿ
ಅಕ್ಷ-ರ-ದಿಂದ
ಅಕ್ಷ-ರ-ಬ್ರ-ಹ್ಮ-ಯೋಗ
ಅಕ್ಷ-ರ-ವನ್ನು
ಅಕ್ಷ-ರವೂ
ಅಕ್ಷ-ರಶಃ
ಅಕ್ಷ-ರಾ-ಣಾ-ಮ-ಕಾ-ರೋಽಸ್ಮಿ
ಅಕ್ಷ-ರಾ-ಭ್ಯಾಸ
ಅಕ್ಷ-ರಾ-ಭ್ಯಾ-ಸ-ದಿಂದ
ಅಕ್ಷ-ರೂಪ
ಅಕ್ಷೋ-ಹಿಣಿ
ಅಕ್ಷೋ-ಹಿ-ಣಿಯ
ಅಖಂಡ
ಅಖಂ-ಡ-ಬ್ರಹ್ಮ
ಅಖಂ-ಡ-ವಾಗಿ
ಅಖಂ-ಡ-ವಾ-ಗಿದೆ
ಅಖಂ-ಡ-ವಾದ
ಅಖಂ-ಡ-ವಾ-ದುದು
ಅಗ-ಣಿತ
ಅಗ-ಮ್ಯ-ವಾದ
ಅಗ-ಲ-ಬೇ-ಕಾ-ಗಿದೆ
ಅಗಲಿ
ಅಗ-ಲಿಕೆ
ಅಗಳು
ಅಗಾಂ-ಗ-ಗ-ಳಂತೆ
ಅಗಾ-ಧ-ವಾ-ಗಿ-ರ-ಬ-ಹುದು
ಅಗಾ-ಧ-ವಾದ
ಅಗಿ-ದರೂ
ಅಗಿದು
ಅಗಿ-ಯ-ಬ-ಹುದು
ಅಗಿ-ಯು-ತ್ತೇವೆ
ಅಗೆ-ಯದೆ
ಅಗೆ-ಯುತ್ತಾ
ಅಗೋ-ಚರ
ಅಗೋ-ಚ-ರ-ವಾಗಿ
ಅಗೋ-ಚ-ರ-ವಾದ
ಅಗೌ-ರವ
ಅಗೌ-ರ-ವ-ವನ್ನು
ಅಗ್ನಿ
ಅಗ್ನಿ-ಕುಂಡ
ಅಗ್ನಿ-ಕುಂ-ಡಕ್ಕೆ
ಅಗ್ನಿ-ಕುಂ-ಡದ
ಅಗ್ನಿ-ಕುಂ-ಡ-ದ-ಲ್ಲಿ-ರು-ವುದು
ಅಗ್ನಿ-ಕುಂ-ಡ-ದಿಂದ
ಅಗ್ನಿ-ಕುಂ-ಡ-ವನ್ನು
ಅಗ್ನಿ-ಗ-ಳ-ಲ್ಲಿ-ದೆಯೋ
ಅಗ್ನಿ-ಗಳು
ಅಗ್ನಿಗೆ
ಅಗ್ನಿ-ದೇವ
ಅಗ್ನಿ-ದೇ-ವನ
ಅಗ್ನಿ-ದೇ-ವ-ನಿಂದ
ಅಗ್ನಿಯ
ಅಗ್ನಿ-ಯಂತೆ
ಅಗ್ನಿ-ಯನ್ನು
ಅಗ್ನಿ-ಯಲ್ಲಿ
ಅಗ್ನಿ-ಯ-ಲ್ಲಿ-ರುವ
ಅಗ್ನಿ-ಯಷ್ಟು
ಅಗ್ನಿ-ಯಾ-ಗಲಿ
ಅಗ್ನಿ-ಯಿಂದ
ಅಗ್ನಿಯೂ
ಅಗ್ನಿಯೇ
ಅಗ್ನಿ-ಯೊಂ-ದಿಗೆ
ಅಗ್ನಿ-ರ್ಜ್ಯೋ-ತಿ-ರಹಃ
ಅಗ್ನಿ-ಷ್ಟೋಮ
ಅಗ್ನಿ-ಸಾ-ಕ್ಷಿ-ಯಾಗಿ
ಅಗ್ರ
ಅಗ್ರ-ಗ-ಣ್ಯರು
ಅಗ್ರ-ಪೂಜೆ
ಅಗ್ರ-ಪೂ-ಜೆ-ಯನ್ನು
ಅಘ-ಟಿ-ತ-ವಾದ
ಅಘಾ-ಯು-ರಿಂ-ದ್ರಿ-ಯಾ-ರಾಮೋ
ಅಚರ
ಅಚ-ರ-ವಾ-ಗಿ-ರುವ
ಅಚಲ
ಅಚ-ಲ-ದೇ-ವಾ-ಲಯ
ಅಚ-ಲ-ವಾಗಿ
ಅಚ-ಲ-ವಾ-ಗಿ-ರುವ
ಅಚ-ಲ-ವಾದ
ಅಚಾ-ತುರ್ಯ
ಅಚಾ-ತು-ರ್ಯ-ಗಳು
ಅಚಾ-ತು-ರ್ಯ-ವನ್ನು
ಅಚಿಂತ್ಯ
ಅಚಿಂ-ತ್ಯ-ವನ್ನು
ಅಚ್ಚರಿ
ಅಚ್ಚರಿ-ಗೊಂಡು
ಅಚ್ಚರಿ-ಗೊ-ಳ್ಳದೆ
ಅಚ್ಚರಿ-ಪಟ್ಟು
ಅಚ್ಚರಿ-ಯಾ-ಗ-ಬ-ಹುದು
ಅಚ್ಚು-ಕ-ಟ್ಟಾಗಿ
ಅಚ್ಚು-ಗ-ಟ್ಟಾಗಿ
ಅಚ್ಚು-ಗಳನ್ನು
ಅಚ್ಚು-ಮೆ-ಚ್ಚಿನ
ಅಚ್ಛೇ-ದ್ಯೋ-ಯ-ಮ-ದ-ಹ್ಯೋ-ಯ-ಮ-ಕ್ಲೇ-ದ್ಯೋ-ಶೋಷ್ಯ
ಅಚ್ಯುತ
ಅಜ
ಅಜನು
ಅಜನೂ
ಅಜ-ನೆಂದೂ
ಅಜವೂ
ಅಜಾ-ಗ-ರೂ-ಕತೆ
ಅಜಾ-ಗ-ರೂ-ಕ-ತೆ-ಯನ್ನು
ಅಜಾ-ಗ-ರೂ-ಕ-ತೆ-ಯಿಂದ
ಅಜಾ-ಗ-ರೂ-ಕ-ನಾಗಿ
ಅಜಾ-ಗ-ರೂ-ಕ-ರಾಗಿ
ಅಜಾ-ಗ-ರೂ-ಕ-ರಾ-ಗಿ-ದ್ದರೆ
ಅಜಾ-ನತಾ
ಅಜೀರ್ಣ
ಅಜೀ-ರ್ಣ-ದಿಂದ
ಅಜೋ
ಅಜೋಽಪಿ
ಅಜ್ಜ
ಅಜ್ಜನ
ಅಜ್ಜಿ
ಅಜ್ಞ-ರಿಗೂ
ಅಜ್ಞರು
ಅಜ್ಞ-ಶ್ಚಾ-ಶ್ರ-ದ್ದ-ಧಾ-ನಶ್ಚ
ಅಜ್ಞಾ-ತ-ನಾದ
ಅಜ್ಞಾ-ತ-ವಾ-ಗಿ-ರು-ವುದು
ಅಜ್ಞಾ-ತ-ವಾ-ದುದು
ಅಜ್ಞಾ-ತ-ವಾ-ಸ-ದಲ್ಲಿ
ಅಜ್ಞಾನ
ಅಜ್ಞಾನಂ
ಅಜ್ಞಾನ-ಕ್ಕಿಂತ
ಅಜ್ಞಾನಕ್ಕೂ
ಅಜ್ಞಾನಕ್ಕೆ
ಅಜ್ಞಾನ-ಗಳನ್ನು
ಅಜ್ಞಾನ-ಗಳಿಂದ
ಅಜ್ಞಾನ-ಜ-ನ್ಯ-ವಾ-ಗಿದೆ
ಅಜ್ಞಾನ-ಜ-ನ್ಯ-ವಾದ
ಅಜ್ಞಾನದ
ಅಜ್ಞಾನ-ದಲ್ಲಿ
ಅಜ್ಞಾನ-ದ-ಲ್ಲಿ-ದ್ದರೆ
ಅಜ್ಞಾನ-ದ-ಲ್ಲಿ-ದ್ದಾಗ
ಅಜ್ಞಾನ-ದ-ಲ್ಲಿ-ದ್ದಾರೆ
ಅಜ್ಞಾನ-ದ-ಲ್ಲಿ-ರುವ
ಅಜ್ಞಾನ-ದ-ಲ್ಲಿ-ರು-ವ-ವನು
ಅಜ್ಞಾನ-ದ-ಲ್ಲಿ-ರು-ವ-ವ-ರಿಗೆ
ಅಜ್ಞಾನ-ದ-ಲ್ಲಿ-ರು-ವ-ವ-ರೆಗೆ
ಅಜ್ಞಾನ-ದ-ಲ್ಲಿ-ರು-ವಾಗ
ಅಜ್ಞಾನ-ದ-ಲ್ಲಿ-ರು-ವು-ದ-ರಿಂದ
ಅಜ್ಞಾನ-ದಿಂದ
ಅಜ್ಞಾನ-ಮೋ-ಚನೆ
ಅಜ್ಞಾನ-ರೂ-ಪದ
ಅಜ್ಞಾನ-ವನ್ನು
ಅಜ್ಞಾನ-ವಿ-ದ್ದಾಗ
ಅಜ್ಞಾನವೂ
ಅಜ್ಞಾನ-ವೆಲ್ಲ
ಅಜ್ಞಾನಿ
ಅಜ್ಞಾನಿ-ಗಳ
ಅಜ್ಞಾನಿ-ಗ-ಳಲ್ಲ
ಅಜ್ಞಾನಿ-ಗಳಲ್ಲಿ
ಅಜ್ಞಾನಿ-ಗ-ಳಾಗಿ
ಅಜ್ಞಾನಿ-ಗ-ಳಿಗೆ
ಅಜ್ಞಾನಿ-ಗಳು
ಅಜ್ಞಾನಿ-ಗ-ಳು-ಎ-ನ್ನು-ತ್ತಾನೆ
ಅಜ್ಞಾನಿಗೆ
ಅಜ್ಞಾನಿಯ
ಅಜ್ಞಾನಿ-ಯಂತೆ
ಅಜ್ಞಾನಿ-ಯ-ಲ್ಲಾ-ದರೂ
ಅಜ್ಞಾನಿ-ಯಲ್ಲಿ
ಅಜ್ಞಾನಿ-ಯ-ಲ್ಲಿಯೂ
ಅಜ್ಞಾನಿ-ಯಾ-ಗಿರು
ಅಜ್ಞಾನಿ-ಯಾ-ದರೋ
ಅಜ್ಞಾನಿಯು
ಅಜ್ಞಾನಿಯೂ
ಅಜ್ಞಾನೇ-ನಾ-ವೃತಂ
ಅಟ್ಟ-ಬೇಕು
ಅಟ್ಟ-ಹಾಸ
ಅಟ್ಟ-ಹಾ-ಸ-ಕ್ಕಾಗಿ
ಅಟ್ಟ-ಹಾ-ಸವೂ
ಅಟ್ಟಿ-ದರೆ
ಅಟ್ಟಿ-ಸಿ-ಕೊಂಡು
ಅಟ್ಟು-ತ್ತಿ-ರ-ಬೇಕು
ಅಟ್ಟು-ತ್ತೇವೆ
ಅಟ್ಟು-ತ್ತೇ-ವೆಯೋ
ಅಟ್ಟು-ವರು
ಅಟ್ಟು-ವುದು
ಅಡಗಿದೆ
ಅಡಗಿ-ರ-ಬೇಕು
ಅಡಗಿ-ರುವ
ಅಡಗಿ-ರು-ವುದನ್ನು
ಅಡಗಿ-ಸದೇ
ಅಡಗಿ-ಸ-ಬೇಕು
ಅಡಗಿ-ಸ-ಲಾ-ರದು
ಅಡಗಿ-ಸಲು
ಅಡಗಿಸಿ
ಅಡ-ಗು-ವುದು
ಅಡ-ಗೂ-ಲ-ಜ್ಜಿ-ಯನ್ನು
ಅಡ-ಚಣೆ
ಅಡ-ಚ-ಣೆ-ಗ-ಳಿವೆ
ಅಡ-ಚ-ಣೆ-ಗಳು
ಅಡ-ಚ-ಣೆ-ಯನ್ನು
ಅಡ-ಚ-ಣೆ-ಯುಂ-ಟಾ-ದಾಗ
ಅಡಿ-ಗ-ಲ್ಲಾ-ದರೊ
ಅಡಿ-ಗ-ಲ್ಲಾ-ದರೋ
ಅಡಿ-ಗ-ಲ್ಲಿನ
ಅಡಿ-ಗ-ಲ್ಲಿ-ನಂತೆ
ಅಡಿ-ಗಲ್ಲು
ಅಡಿಗೆ
ಅಡಿ-ಗೆ-ಕೋ-ಣೆಗೆ
ಅಡಿ-ಗೆ-ಮ-ನೆ-ಯಲ್ಲಿ
ಅಡಿ-ಗೆ-ಮ-ನೆ-ಯಿಂದ
ಅಡಿ-ಗೆ-ಯ-ನ್ನಾ-ದರೂ
ಅಡಿ-ಗೆ-ಯನ್ನು
ಅಡಿ-ಗೆ-ಯ-ವನ
ಅಡಿ-ಗೆ-ಯ-ವ-ನನ್ನು
ಅಡಿ-ಗೆ-ಯ-ವ-ನಿಗೆ
ಅಡಿ-ಗೆ-ಯ-ವನು
ಅಡಿ-ದಾ-ವರೆ-ಗಳನ್ನು
ಅಡಿ-ಯಲ್ಲಿ
ಅಡಿ-ಯಷ್ಟು
ಅಡಿ-ಯಾಳು
ಅಡು-ಗೆ-ಯನ್ನು
ಅಡ್ಡ
ಅಡ್ಡ-ಪ-ಲ್ಲ-ಕ್ಕಿ-ಯಲ್ಲಿ
ಅಡ್ಡ-ಬಿದ್ದು
ಅಡ್ಡ-ಲಾಗಿ
ಅಡ್ಡ-ಹಾದಿ
ಅಡ್ಡ-ಹಾ-ದಿ-ಗ-ಳನ್ನೇ
ಅಡ್ಡ-ಹಾ-ದಿ-ಗ-ಳಿವೆ
ಅಡ್ಡ-ಹಾ-ದಿ-ಗಳೂ
ಅಡ್ಡ-ಹಾ-ದಿಗೆ
ಅಡ್ಡ-ಹಾ-ದಿಯ
ಅಡ್ಡ-ಹಾ-ದಿ-ಯನ್ನು
ಅಡ್ಡ-ಹಾ-ದಿ-ಯಲ್ಲಿ
ಅಡ್ಡಾ-ಡು-ವುದು
ಅಡ್ಡಿ
ಅಡ್ಡಿ-ಯನ್ನು
ಅಡ್ಡಿ-ಯಾಗಿ
ಅಡ್ಡಿ-ಯಾ-ಗಿದೆ
ಅಡ್ಡಿ-ಯಾ-ಗಿ-ರು-ವರೋ
ಅಣ-ಕಿ-ಸು-ವನು
ಅಣಿ
ಅಣಿ-ಮಾ-ಡದೆ
ಅಣಿ-ಮಾ-ಡ-ಬೇ-ಕಾ-ಗಿದೆ
ಅಣಿ-ಮಾ-ಡಲು
ಅಣಿ-ಮಾಡಿ
ಅಣಿ-ಮಾ-ಡಿ-ಕೊಂ-ಡ-ವನು
ಅಣಿ-ಮಾ-ಡಿ-ಕೊಂ-ಡಿ-ರು-ವನು
ಅಣಿ-ಮಾ-ಡಿ-ಕೊ-ಳ್ಳದೆ
ಅಣಿ-ಮಾ-ಡಿ-ಕೊ-ಳ್ಳ-ಬೇಕು
ಅಣಿ-ಮಾ-ಡಿ-ದರೆ
ಅಣಿ-ಮಾ-ಡಿರು
ಅಣಿ-ಮಾ-ಡಿ-ರು-ವನು
ಅಣಿ-ಮಾ-ಡು-ತ್ತಿ-ರು-ವನು
ಅಣಿ-ಮಾ-ಡು-ವನು
ಅಣಿ-ಮಾ-ಡು-ವುದನ್ನು
ಅಣಿ-ಮಾ-ಡು-ವುದು
ಅಣಿ-ಯಾ-ಗ-ಬೇ-ಕಾ-ಗಿತ್ತು
ಅಣಿ-ಯಾ-ಗ-ಬೇ-ಕಾ-ಯಿತು
ಅಣಿ-ಯಾ-ಗ-ಬೇಕು
ಅಣಿ-ಯಾ-ಗಲಿ
ಅಣಿ-ಯಾ-ಗಿದೆ
ಅಣಿ-ಯಾ-ಗಿ-ದ್ದರೆ
ಅಣಿ-ಯಾ-ಗಿ-ದ್ದೇವೆ
ಅಣಿ-ಯಾ-ಗಿ-ರ-ಬೇಕು
ಅಣಿ-ಯಾ-ಗಿರು
ಅಣಿ-ಯಾ-ಗಿ-ರು-ವನು
ಅಣಿ-ಯಾ-ಗಿ-ರು-ವುದು
ಅಣಿ-ಯಾ-ಗಿ-ರು-ವೆನು
ಅಣಿ-ಯಾ-ಗಿಲ್ಲ
ಅಣಿ-ಯಾ-ಗಿವೆ
ಅಣಿ-ಯಾಗು
ಅಣಿ-ಯಾ-ಗು-ತ್ತದೆ
ಅಣಿ-ಯಾ-ಗು-ತ್ತಾನೊ
ಅಣಿ-ಯಾ-ಗು-ತ್ತಿ-ರ-ಬೇಕು
ಅಣಿ-ಯಾ-ಗು-ವು-ದಕ್ಕೆ
ಅಣಿ-ಯಾ-ಗು-ವುದು
ಅಣಿ-ಯಾ-ಗು-ವು-ದೆಂ-ದರೆ
ಅಣಿ-ಯಾ-ಗು-ವುದೊ
ಅಣಿ-ಯಾ-ದ-ಮೇಲೆ
ಅಣಿ-ಯಾ-ದರೆ
ಅಣು
ಅಣು-ರೇಣು
ಅಣು-ವಿ-ಗಿಂತ
ಅಣು-ವಿಗೆ
ಅಣು-ವಿ-ನಲ್ಲಿ
ಅಣ್ಣ-ತ-ಮ್ಮಂ-ದಿರು
ಅಣ್ಣ-ನಾದ
ಅತ
ಅತ-ತ್ತ್ವಾ-ರ್ಥ-ವ-ದಲ್ಪಂ
ಅತಿ
ಅತಿ-ಕ್ರ-ಮಿ-ಸಲು
ಅತಿ-ಕ್ರ-ಮಿಸಿ
ಅತಿ-ಕ್ರ-ಮಿ-ಸಿದೆ
ಅತಿ-ಕ್ರ-ಮಿ-ಸಿಯೂ
ಅತಿ-ಕ್ರ-ಮಿ-ಸಿ-ರು-ತ್ತಾನೆ
ಅತಿ-ಕ್ರ-ಮಿ-ಸಿ-ರು-ವನು
ಅತಿ-ಕ್ರ-ಮಿ-ಸಿ-ರು-ವ-ವನು
ಅತಿ-ಕ್ರ-ಮಿ-ಸು-ವನು
ಅತಿ-ಕ್ರ-ಮಿ-ಸು-ವುದು
ಅತಿ-ಕ್ರ-ಮಿ-ಸು-ವೆವೊ
ಅತಿ-ಕ್ರ-ಮಿ-ಸು-ವೆವೋ
ಅತಿಗೆ
ಅತಿಥಿ
ಅತಿ-ಥಿ-ಗಳನ್ನು
ಅತಿ-ಥಿ-ಗ-ಳಿಗೆ
ಅತಿ-ಥಿ-ಯನ್ನು
ಅತಿ-ನೂ-ತ-ನ-ವಾ-ದುದು
ಅತಿ-ಪ್ರ-ಕಾ-ಶ-ವನ್ನು
ಅತಿ-ಭೋಗ
ಅತಿ-ಮಾನವ
ಅತಿ-ಮಾ-ನುಷ
ಅತಿಯ
ಅತಿ-ಯಾಗಿ
ಅತಿ-ಯಿಂದ
ಅತಿ-ರಥ
ಅತಿ-ವಿ-ರಳ
ಅತಿ-ವೃಷ್ಟಿ
ಅತಿ-ಶ-ಯನು
ಅತಿ-ಶ-ಯ-ವಾದ
ಅತಿ-ಶ-ಯೋ-ಕ್ತಿ-ಯಲ್ಲ
ಅತೀ
ಅತೀಂ-ದ್ರಿಯ
ಅತೀಂ-ದ್ರಿ-ಯದ
ಅತೀಂ-ದ್ರಿ-ಯ-ವಾದ
ಅತೀತ
ಅತೀ-ತ-ನಾಗಿ
ಅತೀ-ತ-ನಾ-ಗಿ-ರು-ವುದು
ಅತೀ-ತ-ನಾಗು
ಅತೀ-ತ-ನಾ-ಗು-ವನು
ಅತೀ-ತ-ನಾ-ಗು-ವನೊ
ಅತೀ-ತ-ನಾ-ದರೆ
ಅತೀ-ತ-ನಾ-ದ-ವನು
ಅತೀ-ತನೂ
ಅತೀ-ತ-ರಾಗಿ
ಅತೀ-ತ-ರಾ-ಗಿ-ರು-ತ್ತಿ-ದ್ದರೊ
ಅತೀ-ತ-ವಾಗಿ
ಅತೀ-ತ-ವಾ-ಗಿದೆ
ಅತೀ-ತ-ವಾ-ಗಿ-ದೆಯೊ
ಅತೀ-ತ-ವಾ-ಗಿ-ರು-ವುದನ್ನು
ಅತೀ-ತ-ವಾ-ಗಿ-ರು-ವು-ದ-ರಿಂದ
ಅತೀ-ತ-ವಾ-ಗಿ-ರು-ವುದು
ಅತೀ-ತ-ವಾ-ಗಿ-ರು-ವುದೇ
ಅತೀ-ತ-ವಾದ
ಅತೀ-ತ-ವಾ-ದುದು
ಅತೀ-ತವೊ
ಅತೃಪ್ತಿ
ಅತೃ-ಪ್ತಿ-ಪ-ಟ್ಟರೆ
ಅತೃ-ಪ್ತಿ-ಯನ್ನು
ಅತೃ-ಪ್ತಿಯೇ
ಅತೋಽಸ್ಮಿ
ಅತ್ತ
ಅತ್ತರೆ
ಅತ್ಯಂತ
ಅತ್ಯ-ಮೋ-ಘ-ವಾದ
ಅತ್ಯಲ್ಪ
ಅತ್ಯ-ವ-ಶ್ಯಕ
ಅತ್ಯಾ
ಅತ್ಯಾ-ಚಾರ
ಅತ್ಯಾ-ಚಾ-ರ-ಗಳು
ಅತ್ಯಾ-ವ-ಶ್ಯಕ
ಅತ್ಯಾ-ವ-ಶ್ಯ-ಕ-ವಾಗಿ
ಅತ್ಯಾ-ಶ್ರಮಿ
ಅತ್ಯಾಸೆ
ಅತ್ಯು-ನ್ನತ
ಅತ್ಯೇತಿ
ಅತ್ರ
ಅಥ
ಅಥವಾ
ಅಥೆಯೊ
ಅಥೈ-ತ-ದ-ಪ್ಯ-ಶ-ಕ್ತೋಽಸಿ
ಅದ
ಅದಂತೂ
ಅದ-ಅನ್ನು
ಅದ-ಕ್ಕ-ನು-ಸಾ-ರ-ವಾಗಿ
ಅದ-ಕ್ಕಾಗಿ
ಅದ-ಕ್ಕಾ-ಗಿಯೆ
ಅದ-ಕ್ಕಾ-ಗಿಯೇ
ಅದ-ಕ್ಕಿಂತ
ಅದ-ಕ್ಕಿಂ-ತಲೂ
ಅದಕ್ಕೂ
ಅದಕ್ಕೆ
ಅದ-ಕ್ಕೆಂದೇ
ಅದ-ಕ್ಕೆಲ್ಲ
ಅದಕ್ಕೇ
ಅದ-ಕ್ಕೇನು
ಅದ-ಕ್ಕೇನೂ
ಅದ-ಕ್ಕೊಂದು
ಅದ-ನ್ನಲ್ಲ
ಅದನ್ನು
ಅದ-ನ್ನು-ಅ-ಭ್ಯಾಸ
ಅದನ್ನೂ
ಅದನ್ನೆ
ಅದ-ನ್ನೆಲ್ಲ
ಅದ-ನ್ನೆಲ್ಲಾ
ಅದನ್ನೇ
ಅದ-ನ್ನೇನು
ಅದಮ್ಯ
ಅದರ
ಅದ-ರಂ-ತಾ-ಗದೆ
ಅದ-ರಂ-ತಾ-ಗು-ವುದು
ಅದ-ರಂತೆ
ಅದ-ರಂ-ತೆಯೆ
ಅದ-ರಂ-ತೆಯೇ
ಅದ-ರ-ದರ
ಅದ-ರ-ಮೇಲೆ
ಅದ-ರಲ್ಲಿ
ಅದ-ರ-ಲ್ಲಿ-ಆ-ಸ-ಕ್ತ-ವಾ-ಗಿದೆ
ಅದ-ರ-ಲ್ಲಿಯೂ
ಅದ-ರ-ಲ್ಲಿಯೆ
ಅದ-ರ-ಲ್ಲಿಯೇ
ಅದ-ರ-ಲ್ಲಿ-ರು-ತ್ತವೆ
ಅದ-ರ-ಲ್ಲಿ-ರುವ
ಅದ-ರ-ಲ್ಲಿ-ರು-ವುದನ್ನು
ಅದ-ರ-ಲ್ಲಿ-ರು-ವುದು
ಅದ-ರ-ಲ್ಲಿವೆ
ಅದ-ರಲ್ಲೂ
ಅದ-ರಲ್ಲೆ
ಅದ-ರ-ಲ್ಲೆಲ್ಲ
ಅದ-ರ-ಲ್ಲೆಲ್ಲಾ
ಅದ-ರಲ್ಲೇ
ಅದ-ರ-ಲ್ಲೇನು
ಅದ-ರಷ್ಟು
ಅದ-ರಷ್ಟೇ
ಅದ-ರಿಂದ
ಅದ-ರಿಂ-ದಲೂ
ಅದ-ರಿಂ-ದಲೆ
ಅದ-ರಿಂ-ದಲೇ
ಅದ-ರಿಂ-ದಾ-ಗುವ
ಅದ-ರಿಂ-ದಾದ
ಅದ-ರಿಂ-ದೇನು
ಅದರೆ
ಅದ-ರೆ-ಡೆಗೆ
ಅದ-ರೆ-ದು-ರಿಗೆ
ಅದ-ರೊಂ-ದಿಗೆ
ಅದ-ರೊ-ಡನೆ
ಅದ-ರೊ-ಳ-ಗಿಂದ
ಅದ-ರೊ-ಳ-ಗಿನ
ಅದ-ರೊ-ಳಗೆ
ಅದಲ್ಲ
ಅದ-ಲ್ಲದೆ
ಅದ-ಲ್ಲದೇ
ಅದಷ್ಟೇ
ಅದಾ-ಗ-ಬೇಕು
ಅದಾ-ಗಲೇ
ಅದಾಗಿ
ಅದಾ-ಗಿ-ರು-ವುದು
ಅದಾ-ಗು-ವ-ವ-ರೆಗೆ
ಅದಾ-ದ-ಮೇಲೆ
ಅದಿ-ತಿಗೆ
ಅದಿ-ದ್ದರೂ
ಅದಿ-ದ್ದರೆ
ಅದಿ-ದ್ದರೇ
ಅದಿನ್ನೂ
ಅದಿ-ರನ್ನು
ಅದಿ-ರಿ-ನ-ಲ್ಲಿ-ರುವ
ಅದಿ-ರುವ
ಅದಿ-ರು-ವಂತೆ
ಅದಿ-ರು-ವಾ-ಗಲೂ
ಅದಿ-ರು-ವು-ದ-ರಿಂದ
ಅದಿ-ಲ್ಲ-ದಂತೆ
ಅದಿ-ಲ್ಲದೆ
ಅದಿ-ಲ್ಲ-ದೆಡೆ
ಅದಿ-ಲ್ಲದೇ
ಅದು
ಅದುಮಿ
ಅದು-ರಿನ
ಅದು-ರಿ-ನಲ್ಲಿ
ಅದುರು
ಅದು-ರು-ಗಳನ್ನು
ಅದೂ
ಅದೃ-ಶ್ಯ-ವಾ-ಯಿತು
ಅದೃ-ಷ್ಟಕ್ಕೆ
ಅದೃ-ಷ್ಟ-ಪೂರ್ವಂ
ಅದೃ-ಷ್ಟ-ವನ್ನು
ಅದೃ-ಷ್ಟ-ವನ್ನೋ
ಅದೃ-ಷ್ಟ-ವಲ್ಲ
ಅದೃ-ಷ್ಟ-ಶಾಲಿ
ಅದೃ-ಷ್ಟ-ಶಾ-ಲಿ-ಗ-ಳಿಗೆ
ಅದೆ
ಅದೆಂ-ದಿಗೂ
ಅದೆಲ್ಲ
ಅದೆ-ಲ್ಲ-ರಿಗೂ
ಅದೆ-ಲ್ಲ-ವನ್ನೂ
ಅದೆ-ಲ್ಲವೂ
ಅದೆಲ್ಲಾ
ಅದೆಲ್ಲಿ
ಅದೆಲ್ಲೊ
ಅದೆಷ್ಟು
ಅದೇ
ಅದೇ-ನಾ-ದರೂ
ಅದೇನು
ಅದೇನೂ
ಅದೇನೊ
ಅದೇನೋ
ಅದೇ-ಶ-ಕಾಲೇ
ಅದೊಂ-ದನ್ನೇ
ಅದೊಂದು
ಅದೊಂದೆ
ಅದೊಂದೇ
ಅದೋ
ಅದ್ದ-ಬೇಕು
ಅದ್ದಿ-ಡು-ವೆವು
ಅದ್ದಿ-ದಂತೆ
ಅದ್ದಿ-ದರೆ
ಅದ್ದಿದೆ
ಅದ್ದಿ-ರ-ಬೇಕು
ಅದ್ದಿ-ರು-ವುದು
ಅದ್ದು-ವುದು
ಅದ್ಭ-ತ-ವಾದ
ಅದ್ಭುತ
ಅದ್ಭು-ತ-ವನ್ನು
ಅದ್ಭು-ತ-ವಾಗಿ
ಅದ್ಭು-ತ-ವಾ-ಗಿ-ದ್ದರೂ
ಅದ್ಭು-ತ-ವಾ-ಗಿ-ಬೇಕು
ಅದ್ಭು-ತ-ವಾದ
ಅದ್ಭು-ತ-ವಾ-ದು-ದನ್ನು
ಅದ್ಭು-ತ-ವಾ-ದುದು
ಅದ್ಭು-ತಾ-ನಂದ
ಅದ್ವೇಷ್ಟಾ
ಅದ್ವೈತ
ಅದ್ವೈ-ತ-ಗ-ಳೆಂಬ
ಅದ್ವೈ-ತ-ತ-ತ್ತ್ವ-ವನ್ನು
ಅದ್ವೈ-ತದ
ಅದ್ವೈ-ತ-ವನ್ನು
ಅದ್ವೈ-ತ-ವೆಂಬ
ಅದ್ವೈ-ತವೋ
ಅದ್ವೈ-ತಾ-ಮೃ-ತ-ವ-ರ್ಷಿ-ಣೀಂ
ಅದ್ವೈ-ತಿ-ಗಳು
ಅಧಮ
ಅಧ-ಮ-ನಾಗಿ
ಅಧ-ಮಾ-ಧ-ಮನು
ಅಧರ್ಮ
ಅಧರ್ಮಂ
ಅಧ-ರ್ಮ-ಕ್ಕಾಗಿ
ಅಧ-ರ್ಮಕ್ಕೂ
ಅಧ-ರ್ಮಕ್ಕೆ
ಅಧ-ರ್ಮ-ಗಳನ್ನು
ಅಧ-ರ್ಮ-ಗ-ಳಾ-ಗ-ದಂತೆ
ಅಧ-ರ್ಮ-ಗಳು
ಅಧ-ರ್ಮದ
ಅಧ-ರ್ಮ-ದಿಂದ
ಅಧ-ರ್ಮದ್ದೆ
ಅಧ-ರ್ಮ-ನಾ-ಶ-ದಲ್ಲಿ
ಅಧ-ರ್ಮ-ವನ್ನು
ಅಧ-ರ್ಮ-ವಾದ
ಅಧ-ರ್ಮವೇ
ಅಧ-ರ್ಮಾ-ಭಿ-ಭ-ವಾತ್
ಅಧ-ರ್ಮಿ-ಗಳನ್ನು
ಅಧ-ರ್ಮಿ-ಗ-ಳಿಗೆ
ಅಧ-ರ್ಮಿ-ಗಳು
ಅಧಶ್ಚ
ಅಧ-ಶ್ಚೋರ್ಧ್ವಂ
ಅಧಿಕ
ಅಧಿ-ಕ-ವಾ-ಗಿ-ರು-ವಾಗ
ಅಧಿ-ಕ-ವಾ-ಗಿ-ರು-ವುದೋ
ಅಧಿ-ಕ-ವಾ-ಗುತ್ತ
ಅಧಿ-ಕಾರ
ಅಧಿ-ಕಾ-ರ-ಕ್ಕಾ-ಗಲೀ
ಅಧಿ-ಕಾ-ರಕ್ಕೆ
ಅಧಿ-ಕಾ-ರದ
ಅಧಿ-ಕಾ-ರ-ದಲ್ಲಿ
ಅಧಿ-ಕಾ-ರ-ದ-ಲ್ಲಿ-ದ್ದಾಗ
ಅಧಿ-ಕಾ-ರ-ದಾ-ಸೆಗೆ
ಅಧಿ-ಕಾ-ರ-ಲಾ-ಲಸೆ
ಅಧಿ-ಕಾ-ರ-ಲಾ-ಲ-ಸೆಯೂ
ಅಧಿ-ಕಾ-ರ-ವ-ನ್ನಲ್ಲ
ಅಧಿ-ಕಾ-ರ-ವನ್ನು
ಅಧಿ-ಕಾ-ರ-ವಾ-ಗ-ಬ-ಹುದು
ಅಧಿ-ಕಾ-ರ-ವಾ-ಗಲಿ
ಅಧಿ-ಕಾ-ರ-ವಾಣಿ
ಅಧಿ-ಕಾ-ರ-ವಾ-ಣಿ-ಯಿಂದ
ಅಧಿ-ಕಾ-ರ-ವಿದೆ
ಅಧಿ-ಕಾ-ರ-ವಿ-ರಲಿ
ಅಧಿ-ಕಾ-ರ-ವಿ-ರು-ವಾಗ
ಅಧಿ-ಕಾ-ರ-ವಿ-ರು-ವುದೊ
ಅಧಿ-ಕಾ-ರ-ವಿಲ್ಲ
ಅಧಿ-ಕಾ-ರ-ವಿ-ಲ್ಲ-ದ-ವನ
ಅಧಿ-ಕಾ-ರಾ-ಸಕ್ತಿ
ಅಧಿ-ಕಾರಿ
ಅಧಿ-ಕಾ-ರಿ-ಗಳನ್ನು
ಅಧಿ-ಕಾ-ರಿ-ಗ-ಳಾ-ದ-ವ-ರಿಂದ
ಅಧಿ-ಕಾ-ರಿ-ಯಲ್ಲ
ಅಧಿ-ದೇ-ವತೆ
ಅಧಿ-ದೈವ
ಅಧಿ-ಪತಿ
ಅಧಿ-ಭೂತ
ಅಧಿ-ಭೂತಂ
ಅಧಿ-ಭೂ-ತಈ
ಅಧಿ-ಯಜ್ಞ
ಅಧಿ-ಯಜ್ಞಃ
ಅಧಿ-ಯ-ಜ್ಞ-ಎಂ-ದರೆ
ಅಧಿ-ಯ-ಜ್ಞ-ದಂತೆ
ಅಧಿ-ಯ-ಜ್ಞ-ನಂತೆ
ಅಧಿ-ಯ-ಜ್ಞ-ರೂ-ಪ-ದಲ್ಲಿ
ಅಧಿ-ಯ-ಜ್ಞೋ-ಽಹ-ಮೇ-ವಾತ್ರ
ಅಧಿ-ಷ್ಠಾನ
ಅಧಿ-ಷ್ಠಾನಂ
ಅಧಿ-ಷ್ಠಾಯ
ಅಧೀನ
ಅಧೀ-ನ-ದ-ಲ್ಲಿ-ರುವ
ಅಧೀ-ನ-ವಾ-ಗಿವೆ
ಅಧೈ-ರ್ಯ-ವನ್ನು
ಅಧೋ
ಅಧೋ-ಗ-ತಿಗೆ
ಅಧ್ಯಕ್ಷ
ಅಧ್ಯ-ಕ್ಷ-ನಾದ
ಅಧ್ಯ-ಕ್ಷರು
ಅಧ್ಯ-ಯನ
ಅಧ್ಯ-ಯ-ನ-ಇ-ವು-ಗ-ಳಿಂ-ದಾ-ಗಲೀ
ಅಧ್ಯ-ಯ-ನ-ಗಳು
ಅಧ್ಯ-ಯ-ನ-ವನ್ನೂ
ಅಧ್ಯಾತ್ಮ
ಅಧ್ಯಾ-ತ್ಮ-ಜ್ಞಾ-ನ-ದಲ್ಲಿ
ಅಧ್ಯಾ-ತ್ಮ-ಜ್ಞಾ-ನ-ನಿ-ತ್ಯತ್ವಂ
ಅಧ್ಯಾ-ತ್ಮ-ನಿತ್ಯಾ
ಅಧ್ಯಾ-ತ್ಮ-ವನ್ನು
ಅಧ್ಯಾ-ತ್ಮ-ವನ್ನೂ
ಅಧ್ಯಾ-ತ್ಮ-ವಾ-ದರೊ
ಅಧ್ಯಾ-ತ್ಮ-ವಿದ್ಯಾ
ಅಧ್ಯಾ-ತ್ಮ-ವಿದ್ಯೆ
ಅಧ್ಯಾ-ತ್ಮ-ವಿ-ದ್ಯೆಯ
ಅಧ್ಯಾ-ತ್ಮ-ವಿ-ದ್ಯೆಯೇ
ಅಧ್ಯಾ-ತ್ಮ-ವಿ-ದ್ಯೆ-ಯೊಂದೇ
ಅಧ್ಯಾಯ
ಅಧ್ಯಾ-ಯಕ್ಕೂ
ಅಧ್ಯಾ-ಯಕ್ಕೆ
ಅಧ್ಯಾ-ಯ-ಗಳ
ಅಧ್ಯಾ-ಯ-ಗ-ಳ-ನ್ನೊ-ಳ-ಗೊಂ-ಡಿದೆ
ಅಧ್ಯಾ-ಯ-ಗಳಲ್ಲಿ
ಅಧ್ಯಾ-ಯ-ಗ-ಳಿಗೆ
ಅಧ್ಯಾ-ಯ-ಗಳು
ಅಧ್ಯಾ-ಯ-ಗ-ಳುಳ್ಳ
ಅಧ್ಯಾ-ಯದ
ಅಧ್ಯಾ-ಯ-ದಲ್ಲಿ
ಅಧ್ಯಾ-ಯ-ದ-ಲ್ಲಿಯೂ
ಅಧ್ಯಾ-ಯ-ವನ್ನು
ಅಧ್ಯಾ-ಯವೂ
ಅಧ್ಯೇ-ಷ್ಯತೇ
ಅನಂತ
ಅನಂ-ತ-ಕಾಲ
ಅನಂ-ತ-ಕಾ-ಲಕ್ಕೆ
ಅನಂ-ತ-ಕೃ-ಷ್ಣ-ರಂತೆ
ಅನಂ-ತಕ್ಕೆ
ಅನಂ-ತ-ಗಳು
ಅನಂ-ತ-ಗು-ಣನು
ಅನಂ-ತ-ತೆ-ಯೊ-ಡನೆ
ಅನಂ-ತ-ತ್ವಕ್ಕೆ
ಅನಂ-ತದ
ಅನಂ-ತ-ದಲ್ಲಿ
ಅನಂ-ತನೂ
ಅನಂ-ತರ
ಅನಂ-ತ-ರದ
ಅನಂ-ತ-ರವೂ
ಅನಂ-ತ-ರವೆ
ಅನಂ-ತ-ರವೇ
ಅನಂ-ತ-ರೂಪ
ಅನಂ-ತ-ರೂ-ಪನು
ಅನಂ-ತ-ರೂ-ಪನೂ
ಅನಂ-ತ-ವನ್ನು
ಅನಂ-ತ-ವಾ-ಗಿದೆ
ಅನಂ-ತ-ವಾ-ಗಿ-ರು-ವನು
ಅನಂ-ತ-ವಾ-ಗಿ-ರು-ವುದನ್ನು
ಅನಂ-ತ-ವಾದ
ಅನಂ-ತ-ವಾ-ದುದು
ಅನಂ-ತ-ವಿ-ಜಯಂ
ಅನಂ-ತ-ವೀ-ರ್ಯನು
ಅನಂ-ತ-ವೀ-ರ್ಯನೆ
ಅನಂ-ತ-ವೀ-ರ್ಯಾ-ಮಿ-ತ-ವಿ-ಕ್ರ-ಮಸ್ತ್ವಂ
ಅನಂ-ತವೂ
ಅನಂ-ತವೇ
ಅನಂ-ತ-ಶ್ಚಾಸ್ಮಿ
ಅನಂ-ತ-ಸಾ-ಗ-ರದ
ಅನಂ-ತ-ಸ್ವ-ರೂ-ಪ-ನಾ-ಗಿ-ರುವ
ಅನಂ-ತಾ-ತ್ಮ-ನನ್ನು
ಅನಂ-ತಾ-ತ್ಮ-ನಲ್ಲಿ
ಅನಂ-ತಾ-ತ್ಮ-ನಾದ
ಅನ-ನು-ಕೂ-ಲ-ಗ-ಳೆಲ್ಲ
ಅನನ್ಯ
ಅನ-ನ್ಯ-ಚೇ-ತಾಃ
ಅನ-ನ್ಯ-ಭಕ್ತಿ
ಅನ-ನ್ಯ-ಭಾ-ವ-ದಿಂದ
ಅನ-ನ್ಯ-ಮ-ನ-ಸ್ಕ-ರಾಗಿ
ಅನ-ನ್ಯ-ಯೋ-ಗದ
ಅನ-ನ್ಯ-ಯೋ-ಗ-ದಲ್ಲಿ
ಅನ-ನ್ಯ-ವಾ-ಗಿ-ರ-ಬೇಕು
ಅನ-ನ್ಯ-ವಾ-ಗು-ವುದು
ಅನ-ನ್ಯ-ವಾದ
ಅನ-ನ್ಯಾ-ಶ್ಚಿಂ-ತ-ಯಂತೋ
ಅನ-ನ್ಯೇ-ನೈವ
ಅನ-ಪೇಕ್ಷಃ
ಅನರ್ಘ್ಯ
ಅನ-ರ್ಘ್ಯ-ವಾ-ಗಿ-ರು-ವುದು
ಅನ-ರ್ಘ್ಯ-ವಾದ
ಅನ-ವ-ರತ
ಅನ-ಹಂ-ಕಾ-ರಿಯೂ
ಅನಾ-ಗ-ರಿ-ಕನ
ಅನಾ-ಚಾ-ರಕ್ಕೂ
ಅನಾ-ಚಾ-ರ-ಗ-ಳಿಗೆ
ಅನಾ-ಟಮಿ
ಅನಾತ್ಮ
ಅನಾ-ತ್ಮ-ಗಳನ್ನು
ಅನಾ-ತ್ಮ-ದಿಂದ
ಅನಾ-ತ್ಮ-ದೊಂ-ದಿಗೆ
ಅನಾ-ತ್ಮ-ನಸ್ತು
ಅನಾ-ತ್ಮ-ನೊಂ-ದಿಗೆ
ಅನಾ-ತ್ಮ-ವಾದ
ಅನಾ-ಥರ
ಅನಾ-ಥ-ರಾ-ಗಿದ್ದ
ಅನಾ-ಥ-ರಾ-ಗಿ-ದ್ದಾಗ
ಅನಾ-ದರ
ಅನಾ-ದ-ರದ
ಅನಾ-ದ-ರ-ವನ್ನು
ಅನಾದಿ
ಅನಾ-ದಿ-ಗಳು
ಅನಾ-ದಿ-ತ್ವಾ-ನ್ನಿ-ರ್ಗು-ಣ-ತ್ವಾ-ತ್ಪ-ರ-ಮಾ-ತ್ಮಾ-ಯ-ಮ-ವ್ಯಯಃ
ಅನಾ-ದಿ-ಮ-ತ್ಪರಂ
ಅನಾ-ದಿ-ಮ-ಧ್ಯಾಂ-ತ-ಮ-ನಂ-ತ-ವೀ-ರ್ಯ-ಮ-ನಂ-ತ-ಬಾ-ಹುಂ
ಅನಾ-ದಿ-ಯಾ-ಗಿ-ರು-ವು-ದ-ರಿಂ-ದಲೂ
ಅನಾ-ಮ-ಯ-ವಾ-ಗಿದೆ
ಅನಾ-ಯ-ಕತೆ
ಅನಾ-ಯ-ಕ-ತೆ-ಯಿಂದ
ಅನಾ-ಯ-ಕ-ವಾ-ಗು-ವುದು
ಅನಾ-ಯ-ಕ-ವಾ-ಗು-ವುದೋ
ಅನಾ-ರೋಗ್ಯ
ಅನಾ-ರೋ-ಗ್ಯ-ದ-ಲ್ಲಿ-ರು-ವುದು
ಅನಾ-ರ್ಯ-ಜು-ಷ್ಟ-ಮ-ಸ್ವ-ರ್ಗ್ಯ-ಮ-ಕೀ-ರ್ತಿ-ಕ-ರ-ಮ-ರ್ಜುನ
ಅನಾ-ರ್ಯ-ರಿಗೆ
ಅನಾ-ರ್ಯರು
ಅನಾವ
ಅನಾ-ವ-ಶ್ಯಕ
ಅನಾ-ವ-ಶ್ಯ-ಕ-ವಾಗಿ
ಅನಾ-ವ-ಶ್ಯ-ಕ-ವಾ-ಗಿ-ರು-ವುದನ್ನು
ಅನಾ-ವ-ಶ್ಯ-ಕ-ವಾ-ಗಿ-ರು-ವು-ದೆಲ್ಲ
ಅನಾ-ವ-ಶ್ಯ-ಕ-ವಾ-ಗು-ವುದು
ಅನಾ-ವ-ಶ್ಯ-ಕ-ವಾದ
ಅನಾ-ವ-ಶ್ಯ-ಕ-ವಾ-ದು-ದ-ನ್ನೆಲ್ಲ
ಅನಾ-ವ-ಶ್ಯ-ವಾ-ಗಿ-ದೆಯೊ
ಅನಾ-ವೃ-ಷ್ಟಿ-ಗಳ
ಅನಾ-ವೃ-ಷ್ಟಿ-ಗಳು
ಅನಾ-ಶಿ-ನೋ-ಽಪ್ರ-ಮೇ-ಯಸ್ಯ
ಅನಾ-ಶ್ರಿತಃ
ಅನಾ-ಸಕ್ತ
ಅನಾ-ಸ-ಕ್ತ-ನಾಗಿ
ಅನಾ-ಸ-ಕ್ತ-ನಾ-ಗಿ-ರ-ಬೇಕು
ಅನಾ-ಸ-ಕ್ತ-ನಾ-ಗಿ-ರು-ವನು
ಅನಾ-ಸ-ಕ್ತ-ನಾ-ಗಿ-ರು-ವ-ವ-ನಿಗೆ
ಅನಾ-ಸ-ಕ್ತ-ನಾ-ಗಿ-ರು-ವುದು
ಅನಾ-ಸ-ಕ್ತ-ನಾ-ಗು-ವನು
ಅನಾ-ಸ-ಕ್ತ-ನಾದ
ಅನಾ-ಸ-ಕ್ತ-ರಾಗಿ
ಅನಾ-ಸ-ಕ್ತ-ರಾ-ಗಿ-ರ-ಬೇಕು
ಅನಾ-ಸ-ಕ್ತ-ರಾ-ಗು-ತ್ತೇ-ವೆಯೋ
ಅನಾ-ಸಕ್ತಿ
ಅನಾ-ಸ-ಕ್ತಿಯ
ಅನಾ-ಸ-ಕ್ತಿ-ಯನ್ನು
ಅನಾ-ಸ-ಕ್ತಿ-ಯಿಂದ
ಅನಾ-ಸ-ಕ್ತಿ-ಯೋಗ
ಅನಾ-ಹತ
ಅನಾ-ಹುತ
ಅನಾ-ಹು-ತಕ್ಕೆ
ಅನಾ-ಹು-ತ-ಗಳು
ಅನಾ-ಹು-ತ-ದಲ್ಲಿ
ಅನಾ-ಹು-ತ-ವನ್ನು
ಅನಿ-ಕೇತಃ
ಅನಿ-ಚ್ಛ-ನ್ನಪಿ
ಅನಿತ್ಯ
ಅನಿ-ತ್ಯತೆ
ಅನಿ-ತ್ಯ-ಮ-ಸುಖಂ
ಅನಿ-ತ್ಯ-ಯಾ-ವುದೊ
ಅನಿ-ತ್ಯ-ವನ್ನು
ಅನಿ-ತ್ಯ-ವಾ-ಗಿಯೇ
ಅನಿ-ತ್ಯ-ವಾ-ದುದು
ಅನಿ-ತ್ಯವೂ
ಅನಿ-ಯತ
ಅನಿ-ಯ-ಮಿ-ತ-ವಾದ
ಅನಿಲ
ಅನಿ-ಲ-ರೂಪ
ಅನಿ-ವಾರ್ಯ
ಅನಿ-ವಾ-ರ್ಯ-ವಾಗಿ
ಅನಿ-ಶ್ಚಿ-ತ-ರಾ-ಗಿ-ರು-ವರೊ
ಅನಿ-ಶ್ಚಿ-ತ-ವಾ-ಗಿ-ರು-ವುದು
ಅನಿಷ್ಟ
ಅನಿ-ಷ್ಟ-ಗಳು
ಅನಿ-ಷ್ಟ-ಗಳೇ
ಅನಿ-ಷ್ಟ-ದಂತೆ
ಅನಿ-ಷ್ಟ-ದಿಂದ
ಅನಿ-ಷ್ಟ-ಮಿಷ್ಟಂ
ಅನಿ-ಷ್ಟ-ವನ್ನು
ಅನಿ-ಷ್ಟವೇ
ಅನು
ಅನು-ಕಂಪ
ಅನು-ಕಂ-ಪ-ವೇ-ಳು-ವುದು
ಅನು-ಕಂ-ಪೆ-ಯನ್ನು
ಅನು-ಕಂ-ಪೆ-ಯಿಂದ
ಅನು-ಕರಿ
ಅನು-ಕ-ರಿ-ಸಲು
ಅನು-ಕ-ರಿಸಿ
ಅನು-ಕ-ರಿ-ಸಿ-ದಾಗ
ಅನು-ಕ-ರಿ-ಸು-ತ್ತಿ-ದ್ದರು
ಅನು-ಕ-ರಿ-ಸುವ
ಅನು-ಕ-ರಿ-ಸು-ವರು
ಅನು-ಕ-ರಿ-ಸು-ವು-ದಿ-ಲ್ಲವೋ
ಅನು-ಕ-ರಿ-ಸು-ವುದು
ಅನು-ಕೂ-ಲ-ವಾ-ಗಲಿ
ಅನು-ಕೂ-ಲ-ವಾದ
ಅನು-ಗಾಲ
ಅನು-ಗಾ-ಲವೂ
ಅನು-ಗು-ಣ-ವಾಗಿ
ಅನು-ಗು-ಣ-ವಾ-ಗಿಯೂ
ಅನು-ಗು-ಣ-ವಾದ
ಅನು-ಗೊ-ಳಿಸ
ಅನು-ಗ್ರ-ಹ-ದಿಂದ
ಅನು-ಗ್ರ-ಹ-ದಿಂ-ದಲೋ
ಅನು-ಗ್ರ-ಹಿ-ಸ-ಬಲ್ಲ
ಅನು-ಗ್ರ-ಹಿ-ಸ-ಲ್ಪಟ್ಟ
ಅನು-ಗ್ರ-ಹಿಸಿ
ಅನು-ಗ್ರ-ಹಿಸು
ಅನು-ಗ್ರ-ಹಿ-ಸು-ತ್ತಾನೆ
ಅನು-ಗ್ರ-ಹಿ-ಸು-ತ್ತೇನೆ
ಅನು-ಗ್ರ-ಹಿ-ಸು-ವನು
ಅನು-ಗ್ರ-ಹಿ-ಸು-ವರು
ಅನು-ದ್ವೇ-ಗ-ಕರಂ
ಅನು-ಪ-ಕಾ-ರಿ-ಯಾ-ದ-ವ-ನಿಗೆ
ಅನು-ಪಮ
ಅನು-ಪ-ಮವೂ
ಅನು-ಬಂಧಂ
ಅನು-ಭವ
ಅನು-ಭ-ವ-ಕ್ಕಾಗಿ
ಅನು-ಭ-ವಕ್ಕೆ
ಅನು-ಭ-ವ-ಕ್ಕೆಲ್ಲ
ಅನು-ಭ-ವ-ಗಳ
ಅನು-ಭ-ವ-ಗಳನ್ನು
ಅನು-ಭ-ವ-ಗ-ಳ-ನ್ನುಂಡು
ಅನು-ಭ-ವ-ಗಳನ್ನೂ
ಅನು-ಭ-ವ-ಗಳನ್ನೆಲ್ಲ
ಅನು-ಭ-ವ-ಗಳನ್ನೆಲ್ಲಾ
ಅನು-ಭ-ವ-ಗಳಲ್ಲಿ
ಅನು-ಭ-ವ-ಗ-ಳಾ-ಗಿ-ರು-ತ್ತಿ-ರು-ತ್ತವೆ
ಅನು-ಭ-ವ-ಗ-ಳಾ-ದರೂ
ಅನು-ಭ-ವ-ಗ-ಳಿ-ಗಾ-ಗಲೀ
ಅನು-ಭ-ವ-ಗ-ಳಿಗೆ
ಅನು-ಭ-ವ-ಗ-ಳಿ-ಲ್ಲದೆ
ಅನು-ಭ-ವ-ಗಳು
ಅನು-ಭ-ವ-ಗಳೂ
ಅನು-ಭ-ವ-ಗ-ಳೆಲ್ಲ
ಅನು-ಭ-ವದ
ಅನು-ಭ-ವ-ದಿಂದ
ಅನು-ಭ-ವ-ದೂ-ರ-ವಾದ
ಅನು-ಭ-ವ-ದೆ-ಡೆಗೆ
ಅನು-ಭ-ವ-ದೊ-ಡನೆ
ಅನು-ಭ-ವ-ವ-ನ್ನಾಗಿ
ಅನು-ಭ-ವ-ವನ್ನು
ಅನು-ಭ-ವ-ವನ್ನೇ
ಅನು-ಭ-ವ-ವಾ-ದರೊ
ಅನು-ಭ-ವ-ವಾ-ದರೋ
ಅನು-ಭ-ವವೂ
ಅನು-ಭ-ವ-ವೆಂಬ
ಅನು-ಭ-ವವೇ
ಅನು-ಭ-ವವೊ
ಅನು-ಭ-ವ-ವೊಂದೇ
ಅನು-ಭ-ವಾ-ಮೃ-ತ-ದಲ್ಲಿ
ಅನು-ಭ-ವಾ-ಮೃ-ತ-ವನ್ನು
ಅನು-ಭವಿ
ಅನು-ಭ-ವಿ-ಗಳು
ಅನು-ಭ-ವಿಸ
ಅನು-ಭ-ವಿ-ಸ-ಕೂ-ಡದು
ಅನು-ಭ-ವಿ-ಸದ
ಅನು-ಭ-ವಿ-ಸ-ಬಲ್ಲ
ಅನು-ಭ-ವಿ-ಸ-ಬ-ಲ್ಲನೊ
ಅನು-ಭ-ವಿ-ಸ-ಬ-ಲ್ಲುದು
ಅನು-ಭ-ವಿ-ಸ-ಬ-ಹುದು
ಅನು-ಭ-ವಿ-ಸ-ಬ-ಹುದೋ
ಅನು-ಭ-ವಿ-ಸ-ಬಾ-ರದು
ಅನು-ಭ-ವಿ-ಸ-ಬೇ-ಕಾ-ಗಿದೆ
ಅನು-ಭ-ವಿ-ಸ-ಬೇ-ಕಾ-ಗಿ-ರು-ವುದು
ಅನು-ಭ-ವಿ-ಸ-ಬೇ-ಕಾ-ಗು-ವುದು
ಅನು-ಭ-ವಿ-ಸ-ಬೇ-ಕಾದ
ಅನು-ಭ-ವಿ-ಸ-ಬೇ-ಕಾ-ದರೂ
ಅನು-ಭ-ವಿ-ಸ-ಬೇ-ಕಾ-ದರೆ
ಅನು-ಭ-ವಿ-ಸ-ಬೇಕು
ಅನು-ಭ-ವಿ-ಸ-ಬೇಕೆ
ಅನು-ಭ-ವಿ-ಸ-ಬೇ-ಕೆಂ-ದರೆ
ಅನು-ಭ-ವಿ-ಸ-ಬೇ-ಕೆಂ-ದಲ್ಲ
ಅನು-ಭ-ವಿ-ಸ-ಬೇ-ಕೆಂ-ದಿ-ರು-ವುದೇ
ಅನು-ಭ-ವಿ-ಸ-ಬೇ-ಕೆಂದು
ಅನು-ಭ-ವಿ-ಸ-ಬೇ-ಕೆಂಬ
ಅನು-ಭ-ವಿ-ಸ-ಬೇ-ಕೆಂ-ಬುದು
ಅನು-ಭ-ವಿ-ಸ-ಬೇಕೊ
ಅನು-ಭ-ವಿ-ಸ-ಬೇಡ
ಅನು-ಭ-ವಿ-ಸ-ಲಾ-ಗು-ವು-ದಿಲ್ಲ
ಅನು-ಭ-ವಿ-ಸಲಿ
ಅನು-ಭ-ವಿ-ಸಲು
ಅನು-ಭ-ವಿ-ಸಲೇ
ಅನು-ಭ-ವಿ-ಸ-ಲೇ-ಬೇ-ಕಾ-ಗಿದೆ
ಅನು-ಭ-ವಿ-ಸ-ಲೇ-ಬೇ-ಕಾ-ಗು-ವುದು
ಅನು-ಭ-ವಿ-ಸ-ಲೇ-ಬೇಕು
ಅನು-ಭ-ವಿ-ಸ-ವನೆ
ಅನು-ಭ-ವಿಸಿ
ಅನು-ಭ-ವಿ-ಸಿದ
ಅನು-ಭ-ವಿ-ಸಿ-ದ-ಮೇಲೆ
ಅನು-ಭ-ವಿ-ಸಿ-ದರೂ
ಅನು-ಭ-ವಿ-ಸಿ-ದರೆ
ಅನು-ಭ-ವಿ-ಸಿ-ದ-ವ-ನಿಗೆ
ಅನು-ಭ-ವಿ-ಸಿ-ದ-ವನು
ಅನು-ಭ-ವಿ-ಸಿ-ದ-ವರು
ಅನು-ಭ-ವಿ-ಸಿ-ದಾಗ
ಅನು-ಭ-ವಿ-ಸಿ-ದಾ-ಗಲೂ
ಅನು-ಭ-ವಿ-ಸಿದೆ
ಅನು-ಭ-ವಿ-ಸಿ-ದೆನೊ
ಅನು-ಭ-ವಿ-ಸಿ-ದೆವು
ಅನು-ಭ-ವಿ-ಸಿ-ದ್ದನ್ನು
ಅನು-ಭ-ವಿ-ಸಿ-ದ್ದರೆ
ಅನು-ಭ-ವಿ-ಸಿದ್ದು
ಅನು-ಭ-ವಿ-ಸಿದ್ದೇ
ಅನು-ಭ-ವಿ-ಸಿ-ದ್ದೇವೆ
ಅನು-ಭ-ವಿ-ಸಿಯೆ
ಅನು-ಭ-ವಿ-ಸಿಯೇ
ಅನು-ಭ-ವಿ-ಸಿ-ಯೇನೊ
ಅನು-ಭ-ವಿ-ಸಿ-ರ-ಬೇಕು
ಅನು-ಭ-ವಿ-ಸಿ-ರುವ
ಅನು-ಭ-ವಿ-ಸಿ-ರು-ವನು
ಅನು-ಭ-ವಿ-ಸಿ-ರು-ವರು
ಅನು-ಭ-ವಿ-ಸಿ-ರು-ವುದೆ
ಅನು-ಭ-ವಿ-ಸಿ-ರು-ವೆವು
ಅನು-ಭ-ವಿ-ಸಿಲ್ಲ
ಅನು-ಭ-ವಿಸು
ಅನು-ಭ-ವಿ-ಸುತ್ತ
ಅನು-ಭ-ವಿ-ಸು-ತ್ತದೆ
ಅನು-ಭ-ವಿ-ಸುತ್ತಾ
ಅನು-ಭ-ವಿ-ಸು-ತ್ತಾನೆ
ಅನು-ಭ-ವಿ-ಸು-ತ್ತಾ-ನೆಯೊ
ಅನು-ಭ-ವಿ-ಸು-ತ್ತಾನೋ
ಅನು-ಭ-ವಿ-ಸು-ತ್ತಾರೆ
ಅನು-ಭ-ವಿ-ಸು-ತ್ತಾ-ರೆಂದು
ಅನು-ಭ-ವಿ-ಸು-ತ್ತಿ-ದ್ದರೆ
ಅನು-ಭ-ವಿ-ಸು-ತ್ತಿ-ದ್ದೇವೆ
ಅನು-ಭ-ವಿ-ಸು-ತ್ತಿ-ರ-ಬ-ಹು-ದಲ್ಲ
ಅನು-ಭ-ವಿ-ಸು-ತ್ತಿ-ರ-ಬ-ಹುದು
ಅನು-ಭ-ವಿ-ಸು-ತ್ತಿ-ರ-ಬೇಕು
ಅನು-ಭ-ವಿ-ಸು-ತ್ತಿ-ರುವ
ಅನು-ಭ-ವಿ-ಸು-ತ್ತಿ-ರು-ವನು
ಅನು-ಭ-ವಿ-ಸು-ತ್ತಿ-ರು-ವನೊ
ಅನು-ಭ-ವಿ-ಸು-ತ್ತಿ-ರು-ವರು
ಅನು-ಭ-ವಿ-ಸು-ತ್ತಿ-ರು-ವಾಗ
ಅನು-ಭ-ವಿ-ಸು-ತ್ತಿ-ರು-ವಾ-ಗಲೂ
ಅನು-ಭ-ವಿ-ಸು-ತ್ತಿ-ರು-ವು-ದಕ್ಕೆ
ಅನು-ಭ-ವಿ-ಸು-ತ್ತಿ-ರು-ವುದನ್ನು
ಅನು-ಭ-ವಿ-ಸು-ತ್ತಿ-ರು-ವುದು
ಅನು-ಭ-ವಿ-ಸು-ತ್ತಿ-ರು-ವೆನು
ಅನು-ಭ-ವಿ-ಸು-ತ್ತೇನೆ
ಅನು-ಭ-ವಿ-ಸು-ತ್ತೇವೆ
ಅನು-ಭ-ವಿ-ಸು-ತ್ತೇ-ವೆಯೋ
ಅನು-ಭ-ವಿ-ಸುವ
ಅನು-ಭ-ವಿ-ಸು-ವಂತೆ
ಅನು-ಭ-ವಿ-ಸು-ವನು
ಅನು-ಭ-ವಿ-ಸು-ವನೋ
ಅನು-ಭ-ವಿ-ಸು-ವರು
ಅನು-ಭ-ವಿ-ಸು-ವರೋ
ಅನು-ಭ-ವಿ-ಸು-ವ-ವನ
ಅನು-ಭ-ವಿ-ಸು-ವ-ವ-ನಿಗೆ
ಅನು-ಭ-ವಿ-ಸು-ವ-ವನು
ಅನು-ಭ-ವಿ-ಸು-ವ-ವರು
ಅನು-ಭ-ವಿ-ಸು-ವ-ವರೇ
ಅನು-ಭ-ವಿ-ಸು-ವಾಗ
ಅನು-ಭ-ವಿ-ಸು-ವಾ-ಗ-ಲಾ-ದರೋ
ಅನು-ಭ-ವಿ-ಸು-ವಾ-ಗಲೂ
ಅನು-ಭ-ವಿ-ಸುವು
ಅನು-ಭ-ವಿ-ಸು-ವು-ದ-ಕ್ಕಾಗಿ
ಅನು-ಭ-ವಿ-ಸು-ವು-ದ-ಕ್ಕಿಂತ
ಅನು-ಭ-ವಿ-ಸು-ವು-ದಕ್ಕೂ
ಅನು-ಭ-ವಿ-ಸು-ವು-ದಕ್ಕೆ
ಅನು-ಭ-ವಿ-ಸು-ವುದನ್ನು
ಅನು-ಭ-ವಿ-ಸು-ವು-ದಿಲ್ಲ
ಅನು-ಭ-ವಿ-ಸು-ವುದು
ಅನು-ಭ-ವಿ-ಸು-ವು-ದೆ-ಲ್ಲ-ವನ್ನು
ಅನು-ಭ-ವಿ-ಸು-ವುದೇ
ಅನು-ಭ-ವಿ-ಸು-ವು-ದೇನು
ಅನು-ಭ-ವಿ-ಸು-ವೆವೊ
ಅನು-ಭ-ವಿ-ಸು-ವೆವೋ
ಅನು-ಭಾ-ವಿ-ಗಳು
ಅನು-ಭಾ-ವಿಗೆ
ಅನು-ಭೂತಿ
ಅನು-ಮಾನ
ಅನು-ಮಾ-ನಕ್ಕೆ
ಅನು-ಮಾ-ನ-ಗಳನ್ನು
ಅನು-ಮಾ-ನ-ಗ್ರ-ಸ್ತರು
ಅನು-ಮಾ-ನದ
ಅನು-ಮಾ-ನ-ದಲ್ಲಿ
ಅನು-ಮಾ-ನ-ದಿಂದ
ಅನು-ಮಾ-ನ-ವಿ-ದ್ದರೆ
ಅನು-ಮಾ-ನ-ವಿ-ಲ್ಲದೆ
ಅನು-ಮಾನವೂ
ಅನು-ಮಾ-ನಾ-ಸ್ಪದ
ಅನು-ಮಾ-ನಿ-ಸದೆ
ಅನು-ಮಾ-ನಿ-ಸ-ಬಾ-ರದು
ಅನು-ಮಾ-ನಿ-ಸ-ಲಾ-ರರು
ಅನು-ಮಾ-ನಿ-ಸುವ
ಅನು-ಮಾ-ನಿ-ಸು-ವಂ-ತಿಲ್ಲ
ಅನು-ಮಾ-ನಿ-ಸು-ವನು
ಅನು-ಮಾ-ನಿ-ಸು-ವ-ವರು
ಅನು-ಮಾ-ನಿ-ಸು-ವುದನ್ನು
ಅನು-ಮಾ-ನಿ-ಸು-ವು-ದಿಲ್ಲ
ಅನು-ಮಾ-ನಿ-ಸು-ವುದು
ಅನು-ಮಾ-ನಿ-ಸು-ವುದೇ
ಅನು-ಮೋ-ದಿ-ಸು-ವ-ವನು
ಅನು-ಯಾ-ಯಿ-ಗ-ಳೆಲ್ಲ
ಅನು-ರ-ಣಿ-ತ-ವಾಗಿ
ಅನು-ರಾ-ಗಕ್ಕೆ
ಅನು-ರಾ-ಗ-ವನ್ನು
ಅನು-ರೂಪ
ಅನು-ಲಿ-ಪ್ತ-ನಾದ
ಅನು-ವಾಗಿ
ಅನು-ಷ್ಠಾನ
ಅನು-ಷ್ಠಾ-ನಕ್ಕೆ
ಅನು-ಷ್ಠಾ-ನಕ್ಕೇ
ಅನು-ಷ್ಠಾ-ನದ
ಅನು-ಷ್ಠಾ-ನ-ದಲ್ಲಿ
ಅನು-ಷ್ಠಾ-ನ-ದಿಂ-ದಲೂ
ಅನು-ಷ್ಠಾ-ನ-ದೊ-ಳಗೆ
ಅನು-ಷ್ಠಾ-ನ-ಪ್ರ-ಧಾ-ನ-ವಾದ
ಅನು-ಷ್ಠಾ-ನ-ಮಾ-ಡೋಣ
ಅನು-ಷ್ಠಾ-ನ-ರಂ-ಗಕ್ಕೆ
ಅನು-ಸ-ರಿ-ಸ-ಕೂ-ಡದು
ಅನು-ಸ-ರಿ-ಸದೆ
ಅನು-ಸ-ರಿ-ಸ-ಬ-ಹುದು
ಅನು-ಸ-ರಿ-ಸ-ಬಾ-ರದು
ಅನು-ಸ-ರಿ-ಸ-ಬೇ-ಕಾ-ಗಿದೆ
ಅನು-ಸ-ರಿ-ಸ-ಬೇ-ಕಾ-ದರೆ
ಅನು-ಸ-ರಿ-ಸ-ಬೇಕು
ಅನು-ಸ-ರಿಸಿ
ಅನು-ಸ-ರಿ-ಸಿ-ಕೊಂಡು
ಅನು-ಸ-ರಿ-ಸಿ-ದರೆ
ಅನು-ಸ-ರಿ-ಸು-ತ್ತಾರೆ
ಅನು-ಸ-ರಿ-ಸು-ತ್ತಿಲ್ಲ
ಅನು-ಸ-ರಿ-ಸು-ವರು
ಅನು-ಸ-ರಿ-ಸು-ವರೊ
ಅನು-ಸ-ರಿ-ಸು-ವಾಗ
ಅನು-ಸ-ರಿ-ಸು-ವು-ದಿಲ್ಲ
ಅನು-ಸ-ರಿ-ಸು-ವು-ದಿ-ಲ್ಲವೋ
ಅನು-ಸ-ರಿ-ಸು-ವುದು
ಅನು-ಸ-ರಿ-ಸು-ವುವು
ಅನು-ಸಾರ
ಅನು-ಸಾ-ರ-ವಾಗಿ
ಅನು-ಸಾ-ರ-ವಾ-ಗಿ-ರುವ
ಅನೇಕ
ಅನೇ-ಕ-ಚಿ-ತ್ತ-ವಿ-ಭ್ರಾಂತಾ
ಅನೇ-ಕ-ಜ-ನ್ಮ-ಸಂ-ಸಿ-ದ್ಧ-ಸ್ತತೋ
ಅನೇ-ಕ-ದಿ-ವ್ಯಾ-ಭ-ರಣಂ
ಅನೇ-ಕ-ಬಾ-ಹೂ-ದ-ರ-ವ-ಕ್ತ್ರ-ನೇತ್ರಂ
ಅನೇ-ಕ-ರನ್ನು
ಅನೇ-ಕ-ರಿಗೆ
ಅನೇ-ಕರು
ಅನೇ-ಕ-ವ-ಕ್ತ್ರ-ನ-ಯ-ನ-ಮ-ನೇ-ಕಾ-ದ್ಭು-ತ-ದ-ರ್ಶ-ನಮ್
ಅನೇ-ಕ-ವೇಳೆ
ಅನೇನ
ಅನೈ-ಚ್ಛಿ-ಕ-ವಾ-ಗಿಯೂ
ಅನೈ-ಚ್ಛಿ-ಕ-ವಾ-ಗು-ವುವು
ಅನೈ-ತಿ-ಕ-ವಾ-ಗಿ-ದ್ದರೆ
ಅನೌ-ಚಿ-ತ್ಯ-ವಾಗಿ
ಅನ್ನ
ಅನ್ನಕ್ಕೆ
ಅನ್ನ-ಚಿಂ-ತನೆ
ಅನ್ನದ
ಅನ್ನ-ದಾನ
ಅನ್ನ-ದಾ-ನ-ವಿ-ರು-ವುದು
ಅನ್ನ-ದಾ-ನ-ಹೀನ
ಅನ್ನ-ದಿಂದ
ಅನ್ನ-ಪ್ರಾ-ಶ-ನ-ಮಾ-ಡು-ವಾಗ
ಅನ್ನ-ಬ-ಟ್ಟೆ-ಗ-ಳಿ-ರ-ಬೇಕು
ಅನ್ನ-ವನ್ನು
ಅನ್ನ-ವಾ-ಗು-ವುದು
ಅನ್ನಾ-ದ್ಭ-ವಂತಿ
ಅನ್ನಿ-ಸಿ-ದ್ದ-ರಿಂದ
ಅನ್ನಿ-ಸು-ತ್ತಿದೆ
ಅನ್ನಿ-ಸು-ವು-ದಿಲ್ಲ
ಅನ್ನಿ-ಸು-ವುದು
ಅನ್ನು-ತ್ತಾನೆ
ಅನ್ನು-ತ್ತಿ-ದ್ದರೆ
ಅನ್ನು-ವನು
ಅನ್ನು-ವ-ಷ್ಟ-ರಲ್ಲಿ
ಅನ್ನು-ವು-ದಿಲ್ಲ
ಅನ್ಯ
ಅನ್ಯಥಾ
ಅನ್ಯರ
ಅನ್ಯ-ರದು
ಅನ್ಯ-ರಲ್ಲ
ಅನ್ಯ-ರಿಗೆ
ಅನ್ಯ-ವ-ಸ್ತು-ಗಳನ್ನು
ಅನ್ಯ-ವ-ಸ್ತು-ಗ-ಳೆಲ್ಲ
ಅನ್ಯಾ
ಅನ್ಯಾಯ
ಅನ್ಯಾ-ಯಕ್ಕೆ
ಅನ್ಯಾ-ಯ-ಗಳನ್ನು
ಅನ್ಯಾ-ಯದ
ಅನ್ಯಾ-ಯ-ದಿಂದ
ಅನ್ಯಾ-ಯ-ಮಾಡಿ
ಅನ್ಯಾ-ಯ-ವನ್ನು
ಅನ್ಯಾ-ಯ-ವನ್ನೇ
ಅನ್ಯಾ-ಯ-ವಾಗಿ
ಅನ್ಯಾ-ಯ-ವಾ-ಗಿಯೋ
ಅನ್ಯಾ-ಯವೊ
ಅನ್ಯಾ-ಶ್ರ-ಯದ
ಅನ್ಯಾ-ಶ್ರ-ಯದ್ದು
ಅನ್ಯಾ-ಶ್ರ-ಯ-ವಲ್ಲ
ಅನ್ಯಾ-ಶ್ರ-ಯ-ವಾ-ದದ್ದು
ಅನ್ಯೇ
ಅನ್ಯೋನ್ಯ
ಅನ್ವ-ಯಿ-ಸದೇ
ಅನ್ವ-ಯಿ-ಸ-ಬಲ್ಲ
ಅನ್ವ-ಯಿ-ಸ-ಬ-ಹುದು
ಅನ್ವ-ಯಿಸು
ಅನ್ವ-ಯಿ-ಸು-ತ್ತದೆ
ಅನ್ವ-ಯಿ-ಸುವ
ಅನ್ವ-ಯಿ-ಸು-ವು-ದಿಲ್ಲ
ಅನ್ವ-ಯಿ-ಸು-ವುದು
ಅನ್ವ-ಯಿ-ಸು-ವುದೋ
ಅನ್ವೇ-ಷ-ಣೆ-ಯಿಂ-ದಲೇ
ಅಪ-ಕಾ-ರಕ್ಕೆ
ಅಪ-ಕೀರ್ತಿ
ಅಪ-ಕೀ-ರ್ತಿ-ಗಳನ್ನು
ಅಪ-ಕೀ-ರ್ತಿಯ
ಅಪ-ಕೀ-ರ್ತಿ-ಯ-ನ್ನಾ-ದರೊ
ಅಪ-ಕೀ-ರ್ತಿ-ಯನ್ನು
ಅಪ-ಚಾರ
ಅಪ-ಚಾ-ರ-ವನ್ನು
ಅಪ-ಜಯ
ಅಪ-ಜ-ಯವೋ
ಅಪ-ನಂ-ಬಿಕೆ
ಅಪ-ಮಾನ
ಅಪ-ಮಾ-ನ-ವಾ-ಗಿ-ದ್ದರೆ
ಅಪ-ಯ-ಶ-ಸ್ಸನ್ನು
ಅಪ-ಯ-ಶಸ್ಸು
ಅಪರ
ಅಪರಂ
ಅಪ-ರಂಜಿ
ಅಪ-ರಂ-ಜಿಯ
ಅಪ-ರ-ವಿದ್ಯೆ
ಅಪ-ರ-ಸ್ಪ-ರ-ಸಂ-ಭೂತಂ
ಅಪರಾ
ಅಪ-ರಾ-ಜಿ-ತ-ನಾದ
ಅಪ-ರಾ-ಧ-ಗಳನ್ನೂ
ಅಪ-ರಾ-ಧ-ಗ-ಳಿಗೆ
ಅಪ-ರಾ-ಧ-ವನ್ನು
ಅಪ-ರಾ-ಧ-ವಾಗಿ
ಅಪ-ರಾ-ಪ್ರ-ಕೃತಿ
ಅಪ-ರಿ-ಗ್ರ-ಹ-ನಾಗಿ
ಅಪ-ರಿ-ಗ್ರ-ಹ-ನಾ-ಗಿ-ರ-ಬೇಕು
ಅಪ-ರಿ-ಗ್ರ-ಹಿ-ಯಾ-ಗಿ-ರ-ಬೇಕು
ಅಪ-ರಿ-ಚಿ-ತನ
ಅಪ-ರಿ-ಚಿ-ತ-ನಂತೆ
ಅಪ-ರಿ-ಮಿತ
ಅಪ-ರಿ-ಮಿ-ತ-ವಾ-ಗಿದೆ
ಅಪ-ರೂಪ
ಅಪರೇ
ಅಪ-ರೇ-ಯ-ಮಿ-ತ-ಸ್ತ್ವ-ನ್ಯಾಂ
ಅಪ-ರ್ಯಾಪ್ತಂ
ಅಪ-ವಿತ್ರ
ಅಪ-ಶ್ಯ-ದ್ದೇ-ವ-ದೇ-ವಸ್ಯ
ಅಪ-ಹ-ರಿ-ಸ-ಲ್ಪ-ಟ್ಟಿದೆ
ಅಪ-ಹ-ರಿ-ಸಿ-ದರು
ಅಪ-ಹ-ರಿ-ಸಿ-ರು-ವ-ವರು
ಅಪ-ಹ-ರಿ-ಸು-ತ್ತಾರೊ
ಅಪ-ಹ-ರಿ-ಸು-ವ-ವ-ನಲ್ಲ
ಅಪ-ಹ-ರಿ-ಸು-ವು-ದಕ್ಕೆ
ಅಪ-ಹ-ರಿ-ಸು-ವುದು
ಅಪ-ಹಾ-ರ-ವಾ-ಯಿತು
ಅಪ-ಹಾ-ಸ್ಯಕ್ಕೆ
ಅಪಾ-ತ್ರ-ರಿಗೇ
ಅಪಾ-ತ್ರರು
ಅಪಾನ
ಅಪಾ-ನ-ದಲ್ಲಿ
ಅಪಾ-ನ-ವನ್ನೂ
ಅಪಾನೇ
ಅಪಾಯ
ಅಪಾ-ಯ-ಕರ
ಅಪಾ-ಯ-ಕಾರಿ
ಅಪಾ-ಯಕ್ಕೆ
ಅಪಾ-ಯ-ಗ-ಳಿಂ-ದಲೂ
ಅಪಾ-ಯ-ಗ-ಳಿಗೂ
ಅಪಾ-ಯ-ದಿಂದ
ಅಪಾ-ಯ-ವನ್ನು
ಅಪಾ-ಯ-ವಾ-ದರೆ
ಅಪಾ-ಯ-ವಿದೆ
ಅಪಾ-ಯ-ವಿಲ್ಲ
ಅಪಾ-ಯವೂ
ಅಪಾ-ಯಾ-ಕಾರಿ
ಅಪಾ-ಯಾ-ಕಾ-ರಿಯೋ
ಅಪಾರ
ಅಪಾ-ರ-ವಾ-ಗಿ-ರ-ಬೇಕು
ಅಪಾ-ರ-ವಾ-ಗಿವೆ
ಅಪಾರ್ಥ
ಅಪಿ
ಅಪೂ-ರ್ಣ-ತೆ-ಗ-ಳೆಲ್ಲ
ಅಪೂ-ರ್ಣ-ತೆ-ಗ-ಳೆಲ್ಲಾ
ಅಪೂ-ರ್ಣ-ತೆ-ಯಿಂದ
ಅಪೂ-ರ್ಣ-ನಾ-ಗು-ತ್ತಾನೆ
ಅಪೂ-ರ್ಣ-ನಾ-ಗು-ವನು
ಅಪೂ-ರ್ಣರು
ಅಪೂ-ರ್ಣ-ವಾ-ಗಿದೆ
ಅಪೂರ್ವ
ಅಪೂ-ರ್ವ-ದ್ದಾ-ಗಿ-ರು-ವುದು
ಅಪೂ-ರ್ವ-ವಾದ
ಅಪೇ-ಕ್ಷಾ-ಶೂ-ನ್ಯನು
ಅಪೇ-ಕ್ಷಾ-ಶೂ-ನ್ಯನೋ
ಅಪೇ-ಕ್ಷಿ-ಸದೆ
ಅಪೇ-ಕ್ಷಿ-ಸು-ವ-ವ-ನಲ್ಲ
ಅಪೇಕ್ಷೆ
ಅಪೇ-ಕ್ಷೆ-ಯನ್ನು
ಅಪ್ಪ
ಅಪ್ಪ-ಣೆ-ಯನ್ನು
ಅಪ್ಪನ
ಅಪ್ಪ-ನಿಗೆ
ಅಪ್ಪನೊ
ಅಪ್ಪ-ನ್ನನ್ನು
ಅಪ್ಪ-ಳಿ-ಸಲು
ಅಪ್ಪ-ಳಿಸಿ
ಅಪ್ಪ-ಳಿ-ಸುವ
ಅಪ್ಪ-ಳಿ-ಸು-ವಂತೆ
ಅಪ್ಪಿ-ಕೊಂಡು
ಅಪ್ಪಿ-ರು-ವ-ವನು
ಅಪ್ಯಸ್ಯ
ಅಪ್ರ-ಕಾಶ
ಅಪ್ರ-ಕಾ-ಶೋ-ಽಪ್ರ-ವೃ-ತ್ತಿಶ್ಚ
ಅಪ್ರ-ತಿಮ
ಅಪ್ರ-ತಿ-ಮ-ವಾ-ದುದು
ಅಪ್ರ-ತಿಷ್ಠೋ
ಅಪ್ರ-ಮೇಯ
ಅಪ್ರ-ಮೇ-ಯ-ನಾ-ಗಿಯೂ
ಅಪ್ರ-ಮೇ-ಯನೂ
ಅಪ್ರ-ಯೋ-ಜಕ
ಅಪ್ರ-ವೃತ್ತಿ
ಅಪ್ರಾಪ್ಯ
ಅಪ್ರಿಯ
ಅಪ್ರಿ-ಯ-ಗಳನ್ನು
ಅಪ್ರಿ-ಯ-ಗಳಲ್ಲಿ
ಅಪ್ರಿ-ಯ-ಗ-ಳೆ-ರಡೂ
ಅಪ್ರಿ-ಯದ
ಅಪ್ರಿ-ಯ-ವಾ-ಗ-ಬ-ಹುದು
ಅಪ್ರಿ-ಯ-ವಾಗಿ
ಅಪ್ರಿ-ಯ-ವಾ-ಗಿದೆ
ಅಪ್ರಿ-ಯ-ವಾ-ಗಿ-ರ-ಬ-ಹುದು
ಅಪ್ರಿ-ಯ-ವಾ-ಗಿರು
ಅಪ್ರಿ-ಯ-ವಾ-ಗಿ-ರು-ತ್ತವೆ
ಅಪ್ರಿ-ಯ-ವಾ-ಗಿ-ರುವ
ಅಪ್ರಿ-ಯ-ವಾ-ಗಿ-ರು-ವುದು
ಅಪ್ರಿ-ಯ-ವಾ-ಗಿ-ರು-ವುದೂ
ಅಪ್ರಿ-ಯ-ವಾ-ಗಿ-ರು-ವು-ದೆಲ್ಲಿ
ಅಪ್ರಿ-ಯ-ವಾ-ಗಿ-ರು-ವುದೇ
ಅಪ್ರಿ-ಯ-ವಾ-ಗು-ವುದು
ಅಪ್ರಿ-ಯ-ವಾದ
ಅಪ್ರಿ-ಯ-ವಾ-ದರೂ
ಅಪ್ರಿ-ಯ-ವಾ-ದು-ದನ್ನು
ಅಪ್ರಿ-ಯ-ವಾ-ದುದು
ಅಫ-ಲ-ಪ್ರೇ-ಪ್ಸುನಾ
ಅಫ-ಲಾ-ಕಾಂ-ಕ್ಷಿ-ಭಿ-ರ್ಯಜ್ಞೋ
ಅಫ-ಲಾ-ಕಾಂ-ಕ್ಷಿ-ಭಿ-ರ್ಯು-ಕೆಃ
ಅಫೀ-ಮನ್ನು
ಅಬ್ಬ
ಅಭಯ
ಅಭಯಂ
ಅಭ-ಯ-ಗಳನ್ನು
ಅಭ-ಯ-ವನ್ನು
ಅಭಾವ
ಅಭಾ-ವ-ವಾ-ಗಿ-ರು-ವಾಗ
ಅಭಾ-ವ-ವಿಲ್ಲ
ಅಭಾ-ವವೂ
ಅಭಿತೋ
ಅಭಿ-ನ-ಯಿ-ಸ-ಬ-ಹುದು
ಅಭಿ-ನ-ಯಿ-ಸ-ಬೇಕಾ
ಅಭಿ-ನ-ಯಿ-ಸು-ತ್ತಾರೆ
ಅಭಿ-ನ-ಯಿ-ಸು-ತ್ತಿ-ದ್ದರೂ
ಅಭಿ-ನ-ಯಿ-ಸು-ತ್ತಿ-ರುವ
ಅಭಿ-ನ-ಯಿ-ಸು-ತ್ತಿ-ರು-ವ-ವನು
ಅಭಿ-ಪ್ರಾಯ
ಅಭಿ-ಪ್ರಾ-ಯ-ಗಳ
ಅಭಿ-ಪ್ರಾ-ಯ-ಗಳನ್ನು
ಅಭಿ-ಪ್ರಾ-ಯ-ಗಳನ್ನೆಲ್ಲ
ಅಭಿ-ಪ್ರಾ-ಯ-ಗಳಿಂದ
ಅಭಿ-ಪ್ರಾ-ಯ-ಗ-ಳಿ-ಗೆಲ್ಲ
ಅಭಿ-ಪ್ರಾ-ಯ-ಗಳು
ಅಭಿ-ಪ್ರಾ-ಯದ
ಅಭಿ-ಪ್ರಾ-ಯ-ವನ್ನು
ಅಭಿ-ಪ್ರಾ-ಯ-ವಾ-ದರೆ
ಅಭಿ-ಮನ್ಯು
ಅಭಿ-ಮಾನ
ಅಭಿ-ಮಾ-ನ-ದಿಂದ
ಅಭಿ-ಮಾ-ನ-ವಿಲ್ಲ
ಅಭಿ-ಮಾ-ನ-ವಿ-ಲ್ಲದೆ
ಅಭಿ-ಮಾನಿ
ಅಭಿ-ರುಚಿ
ಅಭಿ-ರು-ಚಿಗೆ
ಅಭಿ-ಲಾಷೆ
ಅಭಿ-ವೃ-ದ್ಧಿ-ಪ-ಡಿ-ಸಿ-ಕೊ-ಳ್ಳು-ವೆವು
ಅಭಿ-ವೃ-ದ್ಧಿ-ಯನ್ನು
ಅಭಿ-ವೃ-ದ್ಧಿ-ಯಾ-ಗು-ವುದು
ಅಭಿ-ಶಾಪ
ಅಭಿ-ಷೇ-ಕ-ಮಾ-ಡು-ವು-ದಕ್ಕೆ
ಅಭಿ-ಸಂ-ಧಾಯ
ಅಭೀಪ್ಸೆ
ಅಭೀ-ಪ್ಸೆಯ
ಅಭೂತ
ಅಭ್ಯಂ-ತ-ರವೂ
ಅಭ್ಯಾ-ಗ-ತರ
ಅಭ್ಯಾ-ಗ-ತ-ರಿಗೆ
ಅಭ್ಯಾಸ
ಅಭ್ಯಾ-ಸ-ಕ್ಕಿಂತ
ಅಭ್ಯಾ-ಸ-ಗಳ
ಅಭ್ಯಾ-ಸ-ಗಳನ್ನು
ಅಭ್ಯಾ-ಸ-ಗಳು
ಅಭ್ಯಾ-ಸ-ಗ-ಳೆಲ್ಲ
ಅಭ್ಯಾ-ಸದ
ಅಭ್ಯಾ-ಸ-ದ-ಲ್ಲಿಯೂ
ಅಭ್ಯಾ-ಸ-ದಷ್ಟು
ಅಭ್ಯಾ-ಸ-ದಿಂದ
ಅಭ್ಯಾ-ಸ-ದಿಂ-ದಲೂ
ಅಭ್ಯಾ-ಸ-ದಿಂ-ದಲೇ
ಅಭ್ಯಾ-ಸ-ಬ-ಲ-ದಿಂದ
ಅಭ್ಯಾ-ಸ-ಮಾಡಿ
ಅಭ್ಯಾ-ಸ-ಮಾ-ಡಿ-ದರೆ
ಅಭ್ಯಾ-ಸ-ಮಾ-ಡು-ತ್ತಿ-ರು-ವಾಗ
ಅಭ್ಯಾ-ಸ-ಯೋಗ
ಅಭ್ಯಾ-ಸ-ಯೋ-ಗದ
ಅಭ್ಯಾ-ಸ-ಯೋ-ಗ-ಯು-ಕ್ತೇನ
ಅಭ್ಯಾ-ಸ-ಯೋ-ಗ-ವೆಂ-ತಲೇ
ಅಭ್ಯಾ-ಸ-ಯೋ-ಗ-ವೆಂದೂ
ಅಭ್ಯಾ-ಸ-ಯೋ-ಗೇನ
ಅಭ್ಯಾ-ಸ-ವ-ನ್ನಾಗಿ
ಅಭ್ಯಾ-ಸ-ವನ್ನು
ಅಭ್ಯಾ-ಸ-ವಾ-ಗ-ಬೇಕು
ಅಭ್ಯಾ-ಸ-ವಾಗಿ
ಅಭ್ಯಾ-ಸ-ವಾ-ಗಿರು
ಅಭ್ಯಾ-ಸ-ವಾ-ಗು-ವುದು
ಅಭ್ಯಾ-ಸ-ವಾ-ದರೆ
ಅಭ್ಯಾ-ಸ-ವಿ-ರು-ವ-ವ-ನಿಗೆ
ಅಭ್ಯಾ-ಸ-ವಿಲ್ಲ
ಅಭ್ಯಾ-ಸ-ವಿ-ಲ್ಲ-ದ-ವ-ನಿಗೆ
ಅಭ್ಯಾ-ಸ-ವಿ-ಲ್ಲದೆ
ಅಭ್ಯಾ-ಸವೂ
ಅಭ್ಯಾ-ಸ-ವೆಂದರೆ
ಅಭ್ಯಾ-ಸವೇ
ಅಭ್ಯಾ-ಸಾ-ದ್ರ-ಮತೇ
ಅಭ್ಯಾ-ಸೇನ
ಅಭ್ಯಾ-ಸೇ-ಽಪ್ಯ-ಸ-ಮ-ರ್ಥೋಽಸಿ
ಅಭ್ಯು-ತ್ಥಾ-ನ-ಮ-ಧ-ರ್ಮಸ್ಯ
ಅಭ್ಯು-ದ-ಯದ
ಅಮರ
ಅಮ-ರ-ಕಾವ್ಯ
ಅಮ-ರ-ಗಾನ
ಅಮ-ರತ್ವ
ಅಮ-ರ-ತ್ವಕ್ಕೆ
ಅಮ-ರ-ತ್ವ-ವನ್ನು
ಅಮಲು
ಅಮ-ಲೇರಿ
ಅಮ-ಲೇ-ರಿ-ರು-ವನು
ಅಮಾನಿ
ಅಮಾ-ನಿತ್ವ
ಅಮಾ-ನಿ-ತ್ವ-ಮಂ-ದ-ಭಿ-ತ್ವ-ಮ-ಹಿಂಸಾ
ಅಮಾ-ನ್ವಿತ
ಅಮಾ-ವಾಸ್ಯೆ
ಅಮಾ-ವಾ-ಸ್ಯೆಯ
ಅಮಾ-ವಾ-ಸ್ಯೆ-ಯನ್ನು
ಅಮಿತ
ಅಮೀ
ಅಮುಖ್ಯ
ಅಮೃತ
ಅಮೃತಂ
ಅಮೃ-ತ-ಕ್ಕಾಗಿ
ಅಮೃ-ತತ್ತ್ವ
ಅಮೃ-ತ-ತ್ತ್ವ-ವನ್ನು
ಅಮೃ-ತ-ತ್ವಕ್ಕೆ
ಅಮೃ-ತ-ತ್ವ-ವನ್ನು
ಅಮೃ-ತ-ದಂತೆ
ಅಮೃ-ತ-ದಲ್ಲಿ
ಅಮೃ-ತ-ದಿಂದ
ಅಮೃ-ತ-ನಾ-ಗಿ-ದ್ದಾನೆ
ಅಮೃ-ತ-ನಾ-ಗು-ತ್ತಾನೆ
ಅಮೃ-ತ-ನಾ-ಗುವ
ಅಮೃ-ತ-ಮಯ
ಅಮೃ-ತ-ಮ-ಯ-ವಾದ
ಅಮೃ-ತ-ರ-ನ್ನಾಗಿ
ಅಮೃ-ತ-ವನ್ನು
ಅಮೃ-ತ-ವನ್ನೇ
ಅಮೃ-ತ-ವರ್ಷ
ಅಮೃ-ತ-ವಲ್ಲ
ಅಮೃ-ತ-ವಾಗಿ
ಅಮೃ-ತ-ವಾ-ಗಿಯೂ
ಅಮೃ-ತ-ವಾ-ಗಿ-ರುವ
ಅಮೃ-ತ-ವಿದೆ
ಅಮೃ-ತವೂ
ಅಮೃ-ತ-ವೆಂಬ
ಅಮೃ-ತ-ಸ್ವ-ರೂ-ಪ-ವಾ-ದುದು
ಅಮೃ-ತಾ-ತ್ಮ-ನಾ-ಗು-ವನು
ಅಮೆ-ರಿಕಾ
ಅಮೋಘ
ಅಮೋ-ಘ-ವಾದ
ಅಮೋ-ಘ-ವಾ-ದು-ದನ್ನು
ಅಮೋ-ಘ-ವಾ-ದುದು
ಅಮೋ-ಘ-ವಾ-ದು-ದೆಂದು
ಅಮ್ಮ
ಅಮ್ಮಂ-ದಿರ
ಅಮ್ಮನೊ
ಅಯ-ತಿಃ
ಅಯ-ಥಾ-ವತ್
ಅಯ-ನೇಷು
ಅಯ-ಶ-ಸ್ಸುಈ
ಅಯುಕ್ತ
ಅಯುಕ್ತಃ
ಅಯು-ಕ್ತ-ನಾ-ದ-ವ-ನಿಗೆ
ಅಯೋ-ಗ್ಯದ
ಅಯೋ-ಗ್ಯ-ನಲ್ಲಿ
ಅಯೋ-ಗ್ಯನೂ
ಅಯೋ-ಗ್ಯ-ರಲ್ಲ
ಅಯೋ-ಗ್ಯ-ರಾ-ದ-ವ-ರನ್ನು
ಅಯೋ-ಗ್ಯ-ರಿಗೆ
ಅಯೋ-ಗ್ಯರು
ಅಯೋ-ಗ್ಯ-ವಾ-ದೆಡೆ
ಅಯೋ-ಗ್ಯವೂ
ಅಯೋ-ಧ್ಯಾ-ಕಾಂ-ಡ-ದಲ್ಲಿ
ಅಯ್ಯೊ
ಅಯ್ಯೋ
ಅರ-ಗಿ-ಳಿಗೆ
ಅರ-ಗಿಸಿ
ಅರ-ಗಿ-ಸಿ-ಕೊಂ-ಡ-ವನು
ಅರ-ಗಿ-ಸಿ-ಕೊಂ-ಡಿ-ರು-ವುದು
ಅರ-ಗಿ-ಸಿ-ಕೊಂ-ಡಿ-ರು-ವೆವೊ
ಅರ-ಗಿ-ಸಿ-ಕೊಂ-ಡಿ-ರು-ವೆವೋ
ಅರ-ಗಿ-ಸಿ-ಕೊಂಡು
ಅರ-ಗಿ-ಸಿ-ಕೊ-ಳ್ಳ-ಬೇ-ಕಾ-ದರೆ
ಅರ-ಗಿ-ಸಿ-ಕೊ-ಳ್ಳ-ಬೇಕು
ಅರ-ಗಿ-ಸಿ-ಕೊ-ಳ್ಳಲು
ಅರ-ಗಿ-ಸಿ-ಕೊ-ಳ್ಳು-ತ್ತಾನೆ
ಅರ-ಗಿ-ಸಿ-ಕೊ-ಳ್ಳು-ತ್ತಿ-ರು-ವನು
ಅರ-ಗಿ-ಸಿ-ಕೊ-ಳ್ಳು-ತ್ತೇ-ವೆಯೇ
ಅರ-ಗಿ-ಸಿ-ಕೊ-ಳ್ಳುವ
ಅರ-ಗಿ-ಸಿ-ಕೊ-ಳ್ಳು-ವು-ದಕ್ಕೆ
ಅರ-ಗಿ-ಸಿ-ಕೊ-ಳ್ಳು-ವುದು
ಅರ-ಗಿ-ಸಿ-ಕೊ-ಳ್ಳು-ವುದೇ
ಅರ-ಗಿ-ಸು-ತ್ತಿ-ರು-ವನು
ಅರ-ಗಿ-ಸು-ವ-ವನು
ಅರ-ಗಿ-ಸು-ವು-ದ-ಕ್ಕಾ-ಗಿಯೆ
ಅರ-ಗಿ-ಸು-ವು-ದಕ್ಕೆ
ಅರ-ಗು-ವು-ದಿಲ್ಲ
ಅರ-ಗು-ವುದು
ಅರ-ಚಿ-ಕೊಂಡ
ಅರ-ಚಿ-ಕೊ-ಳ್ಳ-ಬೇ-ಕಾ-ಗಿಲ್ಲ
ಅರ-ಚಿ-ಕೊ-ಳ್ಳು-ತ್ತಾನೆ
ಅರಚು
ಅರ-ಣ್ಯ-ದ-ಲ್ಲಿಯೋ
ಅರ-ಮನೆ
ಅರ-ಮ-ನೆ-ಗಳನ್ನು
ಅರ-ಮ-ನೆಯ
ಅರ-ಮ-ನೆ-ಯನ್ನು
ಅರ-ಮ-ನೆ-ಯಲ್ಲಿ
ಅರ-ಮ-ನೆ-ಯ-ಲ್ಲಿ-ದ್ದರೂ
ಅರ-ಮ-ನೆ-ಯ-ಲ್ಲಿ-ರುವ
ಅರ-ಳ-ಬೇ-ಕಾ-ದರೆ
ಅರ-ಳಿತು
ಅರ-ಳಿ-ದಾಗ
ಅರ-ಳಿದೆ
ಅರ-ಳಿಯ
ಅರ-ಳಿ-ರ-ಬೇಕು
ಅರ-ಳುವ
ಅರ-ಳು-ವುದು
ಅರ-ವತ್ತು
ಅರ-ವಿಂದ
ಅರಸ
ಅರ-ಸನ
ಅರ-ಸಾ-ಗಿ-ರಲಿ
ಅರಸಿ
ಅರ-ಸಿ-ಕೊಂಡು
ಅರಿ
ಅರಿತ
ಅರಿ-ತ-ಮೇಲೆ
ಅರಿ-ತ-ಮೇ-ಲೆಯೇ
ಅರಿ-ತರೆ
ಅರಿ-ತ-ವನ
ಅರಿ-ತ-ವನು
ಅರಿ-ತ-ವನೆ
ಅರಿ-ತ-ವರು
ಅರಿ-ತಾಗ
ಅರಿ-ತಿ-ದ್ದರೆ
ಅರಿ-ತಿ-ರ-ಬ-ಹುದು
ಅರಿ-ತಿ-ರ-ಬೇಕು
ಅರಿ-ತಿ-ರು-ತ್ತಾನೆ
ಅರಿ-ತಿ-ರುವ
ಅರಿ-ತಿ-ರು-ವನು
ಅರಿ-ತಿ-ರು-ವನೊ
ಅರಿ-ತಿ-ರು-ವರೊ
ಅರಿ-ತಿ-ರು-ವರೋ
ಅರಿ-ತಿಲ್ಲ
ಅರಿತು
ಅರಿ-ತು-ಕೊಂಡ
ಅರಿ-ತು-ಕೊಂಡು
ಅರಿ-ತು-ಕೊ-ಳ್ಳ-ಬ-ಲ್ಲ-ವನು
ಅರಿ-ತು-ಕೊ-ಳ್ಳ-ಬೇ-ಕಾ-ದರೂ
ಅರಿ-ತು-ಕೊ-ಳ್ಳ-ಬೇಕು
ಅರಿ-ತು-ಕೊ-ಳ್ಳು-ತ್ತಾರೆ
ಅರಿ-ತು-ಕೊ-ಳ್ಳು-ತ್ತಿ-ರು-ವನು
ಅರಿ-ತು-ಕೊ-ಳ್ಳು-ತ್ತೇವೆ
ಅರಿ-ತು-ಕೊ-ಳ್ಳುವ
ಅರಿ-ತು-ಕೊ-ಳ್ಳು-ವನೊ
ಅರಿ-ತು-ಕೊ-ಳ್ಳು-ವನೋ
ಅರಿತೆ
ಅರಿತೇ
ಅರಿ-ನಾ-ಶಕ
ಅರಿಯ
ಅರಿ-ಯದ
ಅರಿ-ಯ-ದ-ವ-ನಂತೆ
ಅರಿ-ಯದು
ಅರಿ-ಯದೆ
ಅರಿ-ಯನು
ಅರಿ-ಯ-ಬ-ಯ-ಸು-ತ್ತಾನೆ
ಅರಿ-ಯ-ಬಲ್ಲ
ಅರಿ-ಯ-ಬ-ಲ್ಲರು
ಅರಿ-ಯ-ಬ-ಲ್ಲೆಯೊ
ಅರಿ-ಯ-ಬ-ಹುದು
ಅರಿ-ಯ-ಬೇ-ಕಾ-ಗಿದೆ
ಅರಿ-ಯ-ಬೇ-ಕಾ-ಗಿಲ್ಲ
ಅರಿ-ಯ-ಬೇ-ಕಾ-ದರೆ
ಅರಿ-ಯ-ಬೇಕು
ಅರಿ-ಯರು
ಅರಿ-ಯ-ಲಾ-ಗು-ವು-ದಿಲ್ಲ
ಅರಿ-ಯ-ಲಾರ
ಅರಿ-ಯ-ಲಾ-ರದು
ಅರಿ-ಯ-ಲಾ-ರರು
ಅರಿ-ಯ-ಲಾ-ರವು
ಅರಿ-ಯ-ಲಾ-ರೆವು
ಅರಿ-ಯಲು
ಅರಿ-ಯ-ಲೆ-ತ್ನಿ-ಸು-ವುದು
ಅರಿ-ಯುತ್ತಾ
ಅರಿ-ಯು-ತ್ತಾನೆ
ಅರಿ-ಯು-ತ್ತಾ-ನೆಯೋ
ಅರಿ-ಯು-ತ್ತಾ-ರೆಯೊ
ಅರಿ-ಯು-ತ್ತಿ-ರು-ವನು
ಅರಿ-ಯು-ತ್ತೇವೆ
ಅರಿ-ಯು-ತ್ತೇ-ವೆಯೋ
ಅರಿ-ಯುವ
ಅರಿ-ಯು-ವಂತೆ
ಅರಿ-ಯು-ವನು
ಅರಿ-ಯು-ವನೊ
ಅರಿ-ಯು-ವನೋ
ಅರಿ-ಯು-ವ-ವನು
ಅರಿ-ಯು-ವ-ವರು
ಅರಿ-ಯು-ವು-ದಕ್ಕೆ
ಅರಿ-ಯು-ವು-ದಲ್ಲ
ಅರಿ-ಯು-ವು-ದಿಲ್ಲ
ಅರಿ-ಯು-ವುದು
ಅರಿ-ಯು-ವು-ದು-ಇದು
ಅರಿ-ಯು-ವು-ದು-ಇ-ವು-ಗ-ಳೆಲ್ಲ
ಅರಿ-ಯು-ವು-ದೊಂದೇ
ಅರಿ-ಯು-ವೆವು
ಅರಿ-ಯು-ವೆವೊ
ಅರಿ-ಯೆನು
ಅರಿ-ಯೆವು
ಅರಿ-ವಾ-ಗದೆ
ಅರಿ-ವಾ-ಗ-ಬೇ-ಕಾ-ದರೆ
ಅರಿ-ವಾ-ಗ-ಬೇಕು
ಅರಿ-ವಾ-ಗುತ್ತಾ
ಅರಿ-ವಾ-ಗು-ವು-ದಿಲ್ಲ
ಅರಿ-ವಾ-ಗು-ವುದು
ಅರಿ-ವಿಗೆ
ಅರಿ-ವಿ-ರು-ವುದು
ಅರಿ-ವಿಲ್ಲ
ಅರಿ-ವಿ-ಲ್ಲದೆ
ಅರಿವು
ಅರಿವೇ
ಅರಿ-ಷ-ಡ್ವ-ರ್ಗ-ಗಳ
ಅರುಚಿ
ಅರುಣ
ಅರೆದು
ಅರೆ-ಮ-ನ-ಸ್ಸಿ-ನಿಂದ
ಅರೇ-ಬಿ-ಯನ್
ಅರ್ಘ್ಯ-ಪಾ-ದ್ಯ-ಗಳನ್ನು
ಅರ್ಚನೆ
ಅರ್ಚ-ನೆಯೇ
ಅರ್ಜಿ
ಅರ್ಜಿ-ಯ-ನ್ನಾ-ದರೂ
ಅರ್ಜಿ-ಯನ್ನು
ಅರ್ಜುನ
ಅರ್ಜು-ನನ
ಅರ್ಜು-ನ-ನಂ-ತ-ಹ-ವ-ನಿಗೆ
ಅರ್ಜು-ನ-ನಂ-ತಿ-ರುವ
ಅರ್ಜು-ನ-ನಂತೆ
ಅರ್ಜು-ನ-ನನ್ನು
ಅರ್ಜು-ನ-ನ-ನ್ನೇನೋ
ಅರ್ಜು-ನ-ನಲ್ಲಿ
ಅರ್ಜು-ನ-ನ-ಲ್ಲಿಯೂ
ಅರ್ಜು-ನ-ನಾ-ದರೋ
ಅರ್ಜು-ನ-ನಿ-ಗಿಂತ
ಅರ್ಜು-ನ-ನಿಗೂ
ಅರ್ಜು-ನ-ನಿಗೆ
ಅರ್ಜು-ನ-ನಿಗೇ
ಅರ್ಜು-ನನು
ಅರ್ಜು-ನನೂ
ಅರ್ಜು-ನ-ನೆಂಬ
ಅರ್ಜು-ನನೇ
ಅರ್ಜು-ನ-ನೇ-ನಾ-ದರೂ
ಅರ್ಜು-ನನ್ನು
ಅರ್ಜು-ನರು
ಅರ್ಜುನಾ
ಅರ್ಥ
ಅರ್ಥ
ಅರ್ಥಕ್ಕೂ
ಅರ್ಥ-ಗಳನ್ನು
ಅರ್ಥ-ಗಳನ್ನೂ
ಅರ್ಥದ
ಅರ್ಥ-ದ-ಮೇಲೆ
ಅರ್ಥ-ದಲ್ಲಿ
ಅರ್ಥ-ದಿಂದ
ಅರ್ಥ-ಪೂ-ರಿ-ತ-ವಾ-ದುದು
ಅರ್ಥ-ಮಾಡಿ
ಅರ್ಥ-ಮಾ-ಡಿ-ಕೊಳ್ಳ
ಅರ್ಥ-ಮಾ-ಡಿ-ಕೊ-ಳ್ಳದೆ
ಅರ್ಥ-ಮಾ-ಡಿ-ಕೊ-ಳ್ಳ-ಬ-ಲ್ಲನೊ
ಅರ್ಥ-ಮಾ-ಡಿ-ಕೊ-ಳ್ಳ-ಬ-ಹುದು
ಅರ್ಥ-ಮಾ-ಡಿ-ಕೊ-ಳ್ಳ-ಬೇ-ಕಾ-ದರೆ
ಅರ್ಥ-ಮಾ-ಡಿ-ಕೊ-ಳ್ಳ-ಲಾ-ರೆವು
ಅರ್ಥ-ಮಾ-ಡಿ-ಕೊ-ಳ್ಳುವ
ಅರ್ಥ-ಮಾ-ಡಿ-ಕೊ-ಳ್ಳು-ವಂತೆ
ಅರ್ಥ-ಮಾ-ಡಿ-ಕೊ-ಳ್ಳು-ವು-ದಿ-ಲ್ಲವೋ
ಅರ್ಥ-ವನ್ನು
ಅರ್ಥ-ವಲ್ಲ
ಅರ್ಥ-ವಾ-ಗ-ತೊ-ಡ-ಗು-ವುದು
ಅರ್ಥ-ವಾ-ಗದ
ಅರ್ಥ-ವಾ-ಗ-ದುದು
ಅರ್ಥ-ವಾ-ಗ-ಬೇ-ಕಿಲ್ಲ
ಅರ್ಥ-ವಾ-ಗ-ಬೇಕು
ಅರ್ಥ-ವಾ-ಗಲಿ
ಅರ್ಥ-ವಾ-ಗ-ಲಿಲ್ಲ
ಅರ್ಥ-ವಾಗಿ
ಅರ್ಥ-ವಾ-ಗಿದೆ
ಅರ್ಥ-ವಾ-ಗಿ-ದೆಯೆ
ಅರ್ಥ-ವಾ-ಗಿ-ರು-ವುದು
ಅರ್ಥ-ವಾ-ಗಿಲ್ಲ
ಅರ್ಥ-ವಾ-ಗುತ್ತ
ಅರ್ಥ-ವಾ-ಗು-ತ್ತವೆ
ಅರ್ಥ-ವಾ-ಗುತ್ತಾ
ಅರ್ಥ-ವಾ-ಗು-ತ್ತಿದೆ
ಅರ್ಥ-ವಾ-ಗು-ತ್ತಿಲ್ಲ
ಅರ್ಥ-ವಾ-ಗು-ವಂ-ತಿಲ್ಲ
ಅರ್ಥ-ವಾ-ಗು-ವಂತೆ
ಅರ್ಥ-ವಾ-ಗು-ವುದನ್ನು
ಅರ್ಥ-ವಾ-ಗು-ವು-ದಿಲ್ಲ
ಅರ್ಥ-ವಾ-ಗು-ವುದು
ಅರ್ಥ-ವಾ-ಗು-ವು-ದೆಂದು
ಅರ್ಥ-ವಾ-ದ-ಗ-ಳ-ಲ್ಲಿಯೇ
ಅರ್ಥ-ವಾ-ಯಿತು
ಅರ್ಥ-ವಿತ್ತು
ಅರ್ಥ-ವಿದೆ
ಅರ್ಥ-ವಿಲ್ಲ
ಅರ್ಥ-ವಿ-ಲ್ಲದ
ಅರ್ಥ-ವುಳ್ಳ
ಅರ್ಥವೂ
ಅರ್ಥ-ವೆಲ್ಲ
ಅರ್ಥವೇ
ಅರ್ಥಾರ್ಥಿ
ಅರ್ಧ
ಅರ್ಧ-ದಲ್ಲಿ
ಅರ್ಪಣ
ಅರ್ಪ-ಣ-ಭಾ-ವ-ದಿಂದ
ಅರ್ಪಣೆ
ಅರ್ಪ-ಣೆ-ಮಾ-ಡ-ಬೇ-ಕಾ-ದರೂ
ಅರ್ಪ-ಣೆ-ಮಾ-ಡ-ಬೇಕು
ಅರ್ಪ-ಣೆ-ಮಾಡಿ
ಅರ್ಪಿತ
ಅರ್ಪಿ-ತ-ನಾ-ದ-ವನು
ಅರ್ಪಿ-ತ-ಭಾ-ವ-ದಿಂದ
ಅರ್ಪಿ-ತ-ರಾದ
ಅರ್ಪಿ-ತ-ವಾ-ಗ-ಬೇಕು
ಅರ್ಪಿ-ತ-ವಾ-ಗಲಿ
ಅರ್ಪಿ-ತ-ವಾ-ಗ-ಲೆಂದು
ಅರ್ಪಿ-ತ-ವಾ-ಗು-ವುದು
ಅರ್ಪಿಸ
ಅರ್ಪಿ-ಸ-ಬೇ-ಕಾ-ಗಿದೆ
ಅರ್ಪಿ-ಸ-ಬೇಕು
ಅರ್ಪಿ-ಸಲು
ಅರ್ಪಿ-ಸ-ಲ್ಪಟ್ಟ
ಅರ್ಪಿಸಿ
ಅರ್ಪಿ-ಸಿ-ಕೊಂಡ
ಅರ್ಪಿ-ಸಿ-ಕೊಂ-ಡರೆ
ಅರ್ಪಿ-ಸಿ-ಕೊ-ಳ್ಳ-ಬೇಕು
ಅರ್ಪಿ-ಸಿ-ಕೊ-ಳ್ಳು-ತ್ತಿ-ರು-ವುದೊ
ಅರ್ಪಿ-ಸಿದ
ಅರ್ಪಿ-ಸಿ-ದರೆ
ಅರ್ಪಿ-ಸಿ-ದ-ವನೇ
ಅರ್ಪಿ-ಸಿ-ದು-ದ-ನ್ನೆಲ್ಲಾ
ಅರ್ಪಿ-ಸಿ-ದ್ದನ್ನು
ಅರ್ಪಿ-ಸಿ-ಬಿ-ಟ್ಟು-ರು-ವನು
ಅರ್ಪಿ-ಸಿ-ಯಾ-ದ-ಮೇಲೆ
ಅರ್ಪಿ-ಸಿ-ರು-ವನು
ಅರ್ಪಿಸು
ಅರ್ಪಿ-ಸು-ತ್ತಾನೆ
ಅರ್ಪಿ-ಸು-ತ್ತಿ-ದ್ದನು
ಅರ್ಪಿ-ಸು-ತ್ತಿ-ದ್ದೇನೆ
ಅರ್ಪಿ-ಸು-ತ್ತೇವೆ
ಅರ್ಪಿ-ಸು-ತ್ತೇ-ವೆಯೋ
ಅರ್ಪಿ-ಸುವ
ಅರ್ಪಿ-ಸು-ವನು
ಅರ್ಪಿ-ಸು-ವರು
ಅರ್ಪಿ-ಸು-ವರೋ
ಅರ್ಪಿ-ಸು-ವು-ದಕ್ಕೆ
ಅರ್ಪಿ-ಸು-ವು-ದಿ-ಲ್ಲವೇ
ಅರ್ಪಿ-ಸು-ವುದು
ಅರ್ಯಮ
ಅರ್ಹ-ತೆ-ಯನ್ನು
ಅರ್ಹ-ನಾಗು
ಅರ್ಹ-ನಾ-ಗು-ತ್ತಾನೆ
ಅರ್ಹ-ನಾ-ಗು-ತ್ತೀಯೆ
ಅರ್ಹನು
ಅರ್ಹ-ರಾ-ದ-ವ-ರನ್ನು
ಅರ್ಹರು
ಅರ್ಹ-ವಾ-ದುದು
ಅಲಂ-ಕ-ರಿ-ಸಿ-ಕೊಂಡು
ಅಲಂ-ಕ-ರಿ-ಸಿ-ರುವ
ಅಲಂ-ಕಾ-ರ-ಕ್ಕಲ್ಲ
ಅಲಂ-ಕಾ-ರಕ್ಕೆ
ಅಲಂ-ಕಾ-ರ-ವನ್ನು
ಅಲಂ-ಕಾ-ರಿಕ
ಅಲಂ-ಕೃ-ತ-ನಾ-ಗಿ-ರು-ವನು
ಅಲಂ-ಕೃ-ತ-ವಾದ
ಅಲಕ್ಷ
ಅಲ-ಕ್ಷ್ಯ-ದಿಂದ
ಅಲಭ್ಯ
ಅಲಾರಂ
ಅಲು-ಗಾ-ಡದೆ
ಅಲು-ಗಾ-ಡಿ-ಸ-ಬಾ-ರದು
ಅಲು-ಗಾ-ಡಿ-ಸಿ-ದೊ-ಡ-ನೆಯೆ
ಅಲು-ಗಾ-ಡು-ವು-ದಿಲ್ಲ
ಅಲು-ಗಿ-ಸ-ಲಾ-ರದು
ಅಲೆ
ಅಲೆ-ಗಳ
ಅಲೆ-ಗ-ಳಂತೆ
ಅಲೆ-ಗಳಿಂದ
ಅಲೆ-ಗ-ಳಿಗೂ
ಅಲೆ-ಗ-ಳಿ-ಗೆಲ್ಲಾ
ಅಲೆ-ಗ-ಳಿ-ರ-ಬೇ-ಕಾ-ದರೆ
ಅಲೆ-ಗ-ಳಿವೆ
ಅಲೆ-ಗಳು
ಅಲೆ-ಗ-ಳೆದ್ದು
ಅಲೆ-ಗ-ಳೆಲ್ಲಾ
ಅಲೆ-ಗ-ಳೇ-ಳು-ತ್ತಿ-ರ-ಬ-ಹುದು
ಅಲೆ-ಗಿಂತ
ಅಲೆಗೂ
ಅಲೆಗೆ
ಅಲೆ-ತೆರೆ
ಅಲೆ-ದ-ಲೆದು
ಅಲೆ-ದಾ-ಡು-ತ್ತಿ-ದ್ದನು
ಅಲೆ-ದಾ-ಡು-ತ್ತಿ-ರು-ವಂತೆ
ಅಲೆ-ದಾ-ಡು-ವು-ದಕ್ಕೆ
ಅಲೆ-ದಾ-ಡು-ವು-ದಿಲ್ಲ
ಅಲೆ-ದಿ-ರು-ವರು
ಅಲೆದು
ಅಲೆಯ
ಅಲೆ-ಯಂತೆ
ಅಲೆ-ಯನ್ನು
ಅಲೆ-ಯ-ಬೇಕು
ಅಲೆ-ಯಲ್ಲಿ
ಅಲೆ-ಯಿ-ಲ್ಲದ
ಅಲೆ-ಯು-ತ್ತಿ-ದ್ದರೆ
ಅಲೆ-ಯು-ತ್ತಿ-ದ್ದ-ವನು
ಅಲೆ-ಯು-ತ್ತಿ-ರು-ವೆವು
ಅಲೆ-ಯು-ವ-ವ-ನನ್ನು
ಅಲೆ-ಯು-ವುದು
ಅಲೆಯೂ
ಅಲೆಯೇ
ಅಲ್ಪ
ಅಲ್ಪಕ್ಕಾಗಿ
ಅಲ್ಪಕ್ಕೆ
ಅಲ್ಪ-ಗ-ಳೆ-ಲ್ಲವೂ
ಅಲ್ಪ-ಜ್ಞ-ರಾದ
ಅಲ್ಪ-ತನ
ಅಲ್ಪ-ದೃಷ್ಟಿ
ಅಲ್ಪ-ದೃ-ಷ್ಟಿಗೆ
ಅಲ್ಪ-ಬು-ದ್ಧಿ-ಗಳು
ಅಲ್ಪ-ಬು-ದ್ಧಿ-ಗಳೂ
ಅಲ್ಪ-ಭಾ-ಗ-ವಾ-ಗಿ-ರುವ
ಅಲ್ಪ-ಮತಿ
ಅಲ್ಪ-ಮ-ತಿಗೆ
ಅಲ್ಪರು
ಅಲ್ಪ-ವನ್ನು
ಅಲ್ಪ-ವ-ಸ್ತು-ಗಳು
ಅಲ್ಪ-ವಾ-ಗಿ-ದ್ದರೂ
ಅಲ್ಪ-ವಾದ
ಅಲ್ಪ-ವಾ-ದದ್ದು
ಅಲ್ಪ-ವಿದ್ಯ
ಅಲ್ಪ-ವಿ-ದ್ಯೆ-ಯಿಂದ
ಅಲ್ಪ-ವ್ಯ-ಕ್ತಿ-ತ್ವ-ವಲ್ಲ
ಅಲ್ಪ-ಸ್ವ-ಲ್ಪ-ವ-ನ್ನೆಲ್ಲ
ಅಲ್ಪಾ-ಹಾರ
ಅಲ್ಲ
ಅಲ್ಲ-ಗ-ಳೆ-ದನೊ
ಅಲ್ಲ-ಗ-ಳೆ-ದರೆ
ಅಲ್ಲ-ಗ-ಳೆ-ದಾಗ
ಅಲ್ಲ-ಗ-ಳೆ-ಯು-ವನು
ಅಲ್ಲ-ಗ-ಳೆ-ಯು-ವರೊ
ಅಲ್ಲ-ಗ-ಳೆ-ಯು-ವು-ದಕ್ಕೆ
ಅಲ್ಲ-ಗ-ಳೆ-ಯು-ವು-ದಿಲ್ಲ
ಅಲ್ಲದ
ಅಲ್ಲದೆ
ಅಲ್ಲಲ್ಲಿ
ಅಲ್ಲ-ಲ್ಲಿಗೆ
ಅಲ್ಲ-ಲ್ಲಿಯೇ
ಅಲ್ಲವೆ
ಅಲ್ಲವೇ
ಅಲ್ಲವೊ
ಅಲ್ಲವೋ
ಅಲ್ಲಾ-ಡ-ಕೂ-ಡದು
ಅಲ್ಲಾಡಿ
ಅಲ್ಲಾ-ಡಿ-ಸ-ಲಾ-ರದು
ಅಲ್ಲಾ-ಡಿ-ಸಿ-ದರೆ
ಅಲ್ಲಾ-ಡಿ-ಸುತ್ತ
ಅಲ್ಲಾ-ಡು-ತ್ತಿ-ದ್ದರೆ
ಅಲ್ಲಾ-ಡು-ತ್ತಿ-ರು-ವುದು
ಅಲ್ಲಾ-ಡು-ವುದೊ
ಅಲ್ಲಿ
ಅಲ್ಲಿಂದ
ಅಲ್ಲಿಂ-ದಲೇ
ಅಲ್ಲಿಗೆ
ಅಲ್ಲಿಗೇ
ಅಲ್ಲಿತ್ತು
ಅಲ್ಲಿದೆ
ಅಲ್ಲಿ-ದ್ದರೆ
ಅಲ್ಲಿಯ
ಅಲ್ಲಿ-ಯ-ವ-ರಿಗೆ
ಅಲ್ಲಿ-ಯ-ವರೆ-ಗಾ-ದರೂ
ಅಲ್ಲಿ-ಯ-ವ-ರೆಗೂ
ಅಲ್ಲಿ-ಯ-ವ-ರೆಗೆ
ಅಲ್ಲಿಯೂ
ಅಲ್ಲಿಯೇ
ಅಲ್ಲಿ-ರಲು
ಅಲ್ಲಿ-ರುವ
ಅಲ್ಲಿ-ರು-ವನು
ಅಲ್ಲಿ-ರು-ವ-ವ-ರೆಗೆ
ಅಲ್ಲಿ-ರು-ವ-ವರೆಲ್ಲ
ಅಲ್ಲಿ-ರು-ವುದನ್ನು
ಅಲ್ಲಿ-ರು-ವು-ದಿಲ್ಲ
ಅಲ್ಲಿ-ರು-ವುದು
ಅಲ್ಲಿ-ರು-ವುದೂ
ಅಲ್ಲಿವೆ
ಅಲ್ಲೆ
ಅಲ್ಲೆಲ್ಲ
ಅಲ್ಲೆಲ್ಲಾ
ಅಲ್ಲೇ
ಅಲ್ಲೇ-ನಾ-ದರೂ
ಅಲ್ಲೇನು
ಅಲ್ಲೊಂದು
ಅಲ್ಲೋಲ
ಅಲ್ಲೋ-ಲ-ಕ-ಲ್ಲೋ-ಲ-ಗಳು
ಅಲ್ಲೋ-ಲ-ಕ-ಲ್ಲೋ-ಲ-ವಾ-ಗಿರು
ಅಲ್ಲೋ-ಲ-ಕ-ಲ್ಲೋ-ಲ-ವಾ-ಗುತ್ತ
ಅಲ್ಲೋ-ಲ-ಕ-ಲ್ಲೋ-ಲ-ವಾ-ಗು-ವುದು
ಅಲ್ಲೋ-ಲ-ಕ-ಲ್ಲೋ-ಲ-ವಾ-ದರೂ
ಅಲ್ಲೋ-ಲ-ಕ-ಲ್ಲೋ-ಲ-ವಾ-ಯಿತು
ಅಲ್ಲೋ-ಲ-ಕ-ಲ್ಲೋ-ಲ-ವಿ-ರ-ಕೂ-ಡದು
ಅಳ-ಬೇ-ಕಾ-ಗಿಲ್ಲ
ಅಳ-ಲನ್ನು
ಅಳಲೆ
ಅಳ-ಲೆ-ಕಾಯಿ
ಅಳ-ವ-ಡಿ-ಸಿ-ಕೊಂ-ಡಿತು
ಅಳ-ವ-ಡಿ-ಸಿ-ಕೊಂಡು
ಅಳಿ-ದು-ಹೋ-ಗಿ-ರು-ವ-ವರೆಲ್ಲ
ಅಳಿಯ
ಅಳಿ-ವಾ-ಗಲೂ
ಅಳಿವು
ಅಳಿಸಿ
ಅಳಿ-ಸಿ-ರು-ವನು
ಅಳಿ-ಸಿ-ಹೋ-ಗು-ತ್ತಿತ್ತು
ಅಳಿ-ಸುತ್ತ
ಅಳಿ-ಸುತ್ತಾ
ಅಳು
ಅಳು-ಕದೆ
ಅಳು-ಕು-ತ್ತಿದ್ದ
ಅಳು-ಕು-ವನೋ
ಅಳು-ಕು-ವು-ದಿಲ್ಲ
ಅಳುಕೂ
ಅಳು-ತ್ತಾ-ರ-ಲ್ಲ-ಎಂದು
ಅಳು-ತ್ತಿ-ದ್ದಳು
ಅಳು-ತ್ತಿ-ರು-ವೆವು
ಅಳು-ತ್ತೇ-ವೆಯೆ
ಅಳು-ಮೋರೆ
ಅಳು-ವಂತೆ
ಅಳು-ವನ್ನು
ಅಳು-ವಿ-ನಿಂ-ದಲೇ
ಅಳು-ವುದು
ಅಳು-ವುದೂ
ಅಳೆ-ದಷ್ಟೂ
ಅಳೆದು
ಅಳೆಯ
ಅಳೆ-ಯ-ಬೇ-ಕಾ-ಗಿಲ್ಲ
ಅಳೆ-ಯ-ಬೇ-ಕಾ-ದರೆ
ಅಳೆ-ಯ-ಬೇಕು
ಅಳೆ-ಯ-ಲಾ-ಗು-ವುದೆ
ಅಳೆ-ಯ-ಲಾ-ರದು
ಅಳೆಯು
ಅಳೆ-ಯು-ತ್ತಾರೆ
ಅಳೆ-ಯು-ತ್ತಿದ್ದ
ಅಳೆ-ಯು-ತ್ತಿ-ರು-ವನು
ಅಳೆ-ಯು-ವನು
ಅಳೆ-ಯು-ವಾಗ
ಅಳೆ-ಯು-ವು-ದಕ್ಕೆ
ಅಳೆ-ಯು-ವುದು
ಅಳೆ-ಯು-ವೆವು
ಅಳ್ಳ-ಕ-ವಾ-ಗಿ-ದ್ದರೆ
ಅಳ್ಳಾ-ಡು-ವು-ದಕ್ಕೆ
ಅವ
ಅವ-ಕಾಶ
ಅವ-ಕಾ-ಶ-ಕೊಡಿ
ಅವ-ಕಾ-ಶ-ಗಳನ್ನು
ಅವ-ಕಾ-ಶ-ಗ-ಳಿವೆ
ಅವ-ಕಾ-ಶ-ಗಳು
ಅವ-ಕಾ-ಶ-ವನ್ನು
ಅವ-ಕಾ-ಶ-ವನ್ನೇ
ಅವ-ಕಾ-ಶ-ವ-ನ್ನೇನೋ
ಅವ-ಕಾ-ಶ-ವಲ್ಲ
ಅವ-ಕಾ-ಶ-ವಾ-ದರೆ
ಅವ-ಕಾ-ಶ-ವಿದೆ
ಅವ-ಕಾ-ಶ-ವಿರು
ಅವ-ಕಾ-ಶ-ವಿ-ರು-ತ್ತಿ-ರ-ಲಿಲ್ಲ
ಅವ-ಕಾ-ಶ-ವಿ-ರು-ವು-ದಿಲ್ಲ
ಅವ-ಕಾ-ಶ-ವಿಲ್ಲ
ಅವ-ಕಾ-ಶ-ವಿ-ಲ್ಲದೆ
ಅವ-ಕಾ-ಶವೇ
ಅವ-ಕ್ಕಿಂತ
ಅವಕ್ಕೆ
ಅವ-ಕ್ಕೇನೂ
ಅವ-ಗು-ಣ-ಗಳನ್ನು
ಅವ-ಗು-ಣ-ವನ್ನು
ಅವ-ಜಾ-ನಂತಿ
ಅವ-ತ-ರಿ-ಸಿ-ದರು
ಅವ-ತ-ರಿ-ಸು-ತ್ತೇನೆ
ಅವ-ತ-ರಿ-ಸು-ವನು
ಅವ-ತ-ರಿ-ಸು-ವುದೇ
ಅವ-ತಾರ
ಅವ-ತಾ-ರ-ಗಳಲ್ಲಿ
ಅವ-ತಾ-ರ-ತ-ತ್ತ್ವದ
ಅವ-ತಾ-ರದ
ಅವ-ತಾ-ರ-ದಂತೆ
ಅವ-ತಾ-ರ-ದ-ಮೇಲೆ
ಅವ-ತಾ-ರ-ಮಾಡಿ
ಅವ-ತಾ-ರ-ವನ್ನು
ಅವ-ತಾ-ರ-ವನ್ನೂ
ಅವ-ತಾ-ರ-ವಾಗಿ
ಅವ-ತಾ-ರ-ವಾ-ಗಿ-ರು-ವಂ-ತಹ
ಅವ-ತಾ-ರ-ವಾ-ಗು-ತ್ತಾನೆ
ಅವ-ತಾ-ರ-ವಾ-ದರೋ
ಅವ-ತಾ-ರ-ವಾ-ದ-ವನ್ನು
ಅವ-ತಾ-ರ-ವೆಂದು
ಅವ-ತಾ-ರ-ವೆತ್ತಿ
ಅವ-ತಾ-ರ-ವೆ-ತ್ತಿದ
ಅವ-ತಾ-ರ-ವೆ-ತ್ತಿದ್ದೇ
ಅವ-ತಾ-ರ-ವೆ-ತ್ತು-ವನು
ಅವ-ತಾ-ರವೇ
ಅವ-ತಾ-ರ-ವ್ಯ-ಕ್ತಿ-ಯನ್ನು
ಅವ-ತಾ-ರ-ಸ್ವ-ರೂ-ಪ-ನಾದ
ಅವ-ತ್ತಿಗೇ
ಅವತ್ತು
ಅವನ
ಅವ-ನಂ-ತಹ
ಅವ-ನಂ-ತಾ-ಗು-ವುದು
ಅವ-ನಂತು
ಅವ-ನಂತೂ
ಅವ-ನಂತೆ
ಅವ-ನಂ-ತೆಯೇ
ಅವ-ನ-ಗಿಂ-ತಲೂ
ಅವ-ನತಿ
ಅವ-ನ-ದನ್ನು
ಅವ-ನ-ದಲ್ಲ
ಅವ-ನ-ದಾ-ಗಿ-ರ-ಬೇಕು
ಅವ-ನ-ದಾ-ಗಿ-ರು-ವು-ದ-ರಿಂದ
ಅವ-ನ-ದಾ-ಗು-ವುದು
ಅವ-ನದು
ಅವ-ನದೆ
ಅವ-ನದೇ
ಅವ-ನನ್ನು
ಅವ-ನನ್ನೂ
ಅವ-ನನ್ನೆ
ಅವ-ನನ್ನೇ
ಅವ-ನ-ನ್ನೇನು
ಅವ-ನ-ನ್ನೇನೂ
ಅವ-ನ-ಮೇಲೆ
ಅವ-ನಲ್ಲ
ಅವ-ನ-ಲ್ಲದ
ಅವ-ನ-ಲ್ಲದೆ
ಅವ-ನಲ್ಲಿ
ಅವ-ನ-ಲ್ಲಿ-ಅ-ಜ್ಞಾ-ನ-ವಿಲ್ಲ
ಅವ-ನ-ಲ್ಲಿಗೇ
ಅವ-ನ-ಲ್ಲಿತ್ತು
ಅವ-ನ-ಲ್ಲಿದೆ
ಅವ-ನ-ಲ್ಲಿ-ದೆಯೋ
ಅವ-ನ-ಲ್ಲಿ-ದ್ದರೆ
ಅವ-ನ-ಲ್ಲಿಯೇ
ಅವ-ನ-ಲ್ಲಿ-ರ-ಬೇಕು
ಅವ-ನ-ಲ್ಲಿ-ರು-ತ್ತದೆ
ಅವ-ನ-ಲ್ಲಿ-ರುವ
ಅವ-ನ-ಲ್ಲಿ-ರು-ವನು
ಅವ-ನ-ಲ್ಲಿ-ರು-ವಷ್ಟು
ಅವ-ನ-ಲ್ಲಿ-ರು-ವುದನ್ನು
ಅವ-ನ-ಲ್ಲಿ-ರು-ವುದು
ಅವ-ನ-ಲ್ಲಿ-ರು-ವು-ದೆಲ್ಲ
ಅವ-ನ-ಲ್ಲಿ-ರು-ವುವು
ಅವ-ನ-ಲ್ಲಿಲ್ಲ
ಅವ-ನ-ಲ್ಲಿವೆ
ಅವ-ನಲ್ಲೆ
ಅವ-ನಲ್ಲೇ
ಅವ-ನ-ವನು
ಅವ-ನವೇ
ಅವ-ನಷ್ಟು
ಅವ-ನಷ್ಟೇ
ಅವ-ನಾಗಿ
ಅವ-ನಾಜ್ಞೆ
ಅವ-ನಾ-ಜ್ಞೆ-ಯಂತೆ
ಅವ-ನಾಟ
ಅವ-ನಾ-ಡುವ
ಅವ-ನಾ-ಡು-ವನು
ಅವ-ನಾ-ಣ-ತಿ-ಯಂತೆ
ಅವ-ನಾ-ದರೂ
ಅವ-ನಾ-ದರೊ
ಅವ-ನಾ-ದರೋ
ಅವ-ನಾರು
ಅವ-ನಾ-ವಾ-ಗಲೂ
ಅವನಿ
ಅವ-ನಿಂದ
ಅವ-ನಿಂ-ದಲೆ
ಅವ-ನಿಂ-ದಲೇ
ಅವ-ನಿ-ಗ-ರಿ-ಯ-ದಂತೆ
ಅವ-ನಿ-ಗಲ್ಲ
ಅವ-ನಿ-ಗಾಗಿ
ಅವ-ನಿ-ಗಾ-ಗು-ವು-ದಿಲ್ಲ
ಅವ-ನಿ-ಗಾದ
ಅವ-ನಿ-ಗಾ-ದರೂ
ಅವ-ನಿ-ಗಿಂತ
ಅವ-ನಿ-ಗಿದೆ
ಅವ-ನಿ-ಗಿನ್ನೂ
ಅವ-ನಿ-ಗಿ-ರುವ
ಅವ-ನಿ-ಗಿ-ರು-ವುದು
ಅವ-ನಿಗೂ
ಅವ-ನಿಗೆ
ಅವ-ನಿಗೇ
ಅವ-ನಿ-ಗೇ-ನಾ-ದರೂ
ಅವ-ನಿ-ಗೇನು
ಅವ-ನಿ-ಗೇನೂ
ಅವ-ನಿ-ಗೊಂದು
ಅವ-ನಿಚ್ಛೆ
ಅವ-ನಿ-ಚ್ಛೆಗೆ
ಅವ-ನಿ-ಚ್ಛೆ-ಯಂತೆ
ಅವ-ನಿ-ಚ್ಛೆ-ಯಿಂದ
ಅವ-ನಿತ್ತ
ಅವ-ನಿ-ದದ
ಅವ-ನಿ-ದ್ದರೂ
ಅವ-ನಿ-ದ್ದರೆ
ಅವ-ನಿ-ದ್ದರೇ
ಅವ-ನಿ-ದ್ದಾನೆ
ಅವ-ನಿನ್ನು
ಅವ-ನಿ-ನ್ನು-ಮೇಲೆ
ಅವ-ನಿನ್ನೂ
ಅವ-ನಿ-ರ-ಬ-ಹುದು
ಅವ-ನಿರು
ಅವ-ನಿ-ರುವ
ಅವ-ನಿ-ರು-ವನು
ಅವ-ನಿ-ರು-ವ-ವ-ನೊ-ಬ್ಬನೇ
ಅವ-ನಿ-ರು-ವುದು
ಅವ-ನಿ-ಲ್ಲದ
ಅವ-ನಿ-ಲ್ಲ-ದಾ-ಗಲೂ
ಅವ-ನಿ-ಲ್ಲದೆ
ಅವನು
ಅವನೂ
ಅವನೆ
ಅವ-ನೆಂ-ದಿಗೂ
ಅವ-ನೆ-ಡೆಗೆ
ಅವ-ನೆ-ಡೆಗೇ
ಅವ-ನೆ-ದು-ರಿಗೆ
ಅವ-ನೆ-ಲ್ಲ-ರನ್ನೂ
ಅವ-ನೆ-ಲ್ಲಿಗೂ
ಅವ-ನೆಲ್ಲೊ
ಅವ-ನೆಷ್ಟು
ಅವನೇ
ಅವ-ನೇ-ನಾ-ದರೂ
ಅವ-ನೇನು
ಅವ-ನೇನೂ
ಅವ-ನೇ-ನೆಂದು
ಅವ-ನೇನೊ
ಅವ-ನೇನೋ
ಅವ-ನೊಂ-ದಿಗೆ
ಅವ-ನೊಂದು
ಅವ-ನೊ-ಡನೆ
ಅವ-ನೊ-ಡೆಯ
ಅವ-ನೊಬ್ಬ
ಅವ-ನೊ-ಬ್ಬ-ನಿಗೆ
ಅವ-ನೊ-ಬ್ಬ-ನಿಗೇ
ಅವ-ನೊ-ಬ್ಬನೆ
ಅವ-ನೊ-ಬ್ಬನೇ
ಅವ-ನೊ-ಳಗೆ
ಅವನ್ನು
ಅವ-ನ್ನೆಲ್ಲ
ಅವ-ನ್ನೆಲ್ಲಾ
ಅವನ್ನೇ
ಅವ-ಮಾನ
ಅವ-ಮಾ-ನಕ್ಕೆ
ಅವ-ಮಾ-ನ-ಗಳನ್ನು
ಅವ-ಮಾ-ನ-ಗ-ಳ-ಲ್ಲಿಯೂ
ಅವ-ಮಾ-ನ-ವ-ನ್ನಾ-ದರೂ
ಅವ-ಮಾ-ನಿ-ತ-ನಾಗಿ
ಅವರ
ಅವ-ರಂ-ತಾ-ಗು-ವೆವು
ಅವ-ರಂತೆ
ಅವ-ರಂ-ತೆಯೇ
ಅವ-ರಡಿ
ಅವ-ರದು
ಅವ-ರನ್ನು
ಅವ-ರನ್ನೂ
ಅವ-ರ-ನ್ನೆಲ್ಲ
ಅವ-ರ-ನ್ನೆಲ್ಲಾ
ಅವ-ರನ್ನೇ
ಅವ-ರ-ಮೇಲೆ
ಅವ-ರಲ್ಲಿ
ಅವ-ರ-ಲ್ಲಿತ್ತು
ಅವ-ರ-ಲ್ಲಿ-ರುವ
ಅವ-ರ-ವರ
ಅವ-ರ-ವ-ರಿಗೆ
ಅವ-ರ-ವ-ರಿಗೇ
ಅವ-ರ-ವರು
ಅವ-ರಾ-ಡುವ
ಅವ-ರಿಂದ
ಅವ-ರಿಂ-ದಲೇ
ಅವ-ರಿ-ಗಿಂತ
ಅವ-ರಿಗೂ
ಅವ-ರಿಗೆ
ಅವ-ರಿ-ಗೆಲ್ಲ
ಅವ-ರಿ-ಗೆಲ್ಲಾ
ಅವ-ರಿಗೇ
ಅವ-ರಿ-ಗೇನೂ
ಅವ-ರಿನ್ನು
ಅವ-ರಿ-ಬ್ಬ-ರಲ್ಲೂ
ಅವ-ರಿ-ಬ್ಬ-ರಿಗೂ
ಅವ-ರಿ-ಬ್ಬರೂ
ಅವ-ರಿ-ಲ್ಲದ
ಅವರು
ಅವ-ರು-ಗಳ
ಅವರೂ
ಅವ-ರೆಂ-ದಿಗೂ
ಅವರೆ-ಣಿ-ಸಿ-ದಂ-ತೆಯೇ
ಅವರೆ-ದ-ರು-ರಿಗೆ
ಅವರೆಲ್ಲ
ಅವರೆ-ಲ್ಲರೂ
ಅವರೆಲ್ಲಾ
ಅವರೆಲ್ಲೊ
ಅವರೇ
ಅವ-ರೇ-ನಾ-ದರೂ
ಅವ-ರೇನು
ಅವ-ರೊಂ-ದಿಗೆ
ಅವ-ರೊಂದು
ಅವ-ರೊ-ಡನೆ
ಅವ-ರೊಬ್ಬ
ಅವ-ರೊ-ಳ-ಗೆಲ್ಲ
ಅವ-ರ್ಣ-ನೀಯ
ಅವ-ಲ-ಕ್ಕಿ-ಯನ್ನು
ಅವಳ
ಅವ-ಳನ್ನು
ಅವ-ಳಲ್ಲಿ
ಅವ-ಳಿಗೆ
ಅವಳು
ಅವ-ಳೊ-ಳಗೆ
ಅವ-ಶೇಷ
ಅವ-ಶೇ-ಷ-ದಿಂದ
ಅವ-ಶ್ಯಕ
ಅವ-ಶ್ಯ-ಕ-ವಾಗಿ
ಅವ-ಸಾನ
ಅವಸ್ಥೆ
ಅವ-ಸ್ಥೆ-ಗಳ
ಅವ-ಸ್ಥೆ-ಗ-ಳ-ಲ್ಲಿಯೂ
ಅವ-ಸ್ಥೆಗೆ
ಅವ-ಸ್ಥೆಯ
ಅವ-ಸ್ಥೆ-ಯಂತೆ
ಅವ-ಸ್ಥೆ-ಯನ್ನು
ಅವ-ಸ್ಥೆ-ಯಲ್ಲ
ಅವ-ಸ್ಥೆ-ಯಲ್ಲಿ
ಅವ-ಸ್ಥೆ-ಯ-ಲ್ಲಿತ್ತು
ಅವ-ಸ್ಥೆ-ಯ-ಲ್ಲಿದೆ
ಅವ-ಸ್ಥೆ-ಯ-ಲ್ಲಿಯೆ
ಅವ-ಹೇ-ಳ-ನ-ವನ್ನು
ಅವಾಂ-ತ-ರ-ದಿಂದ
ಅವಾ-ಚ್ಯ-ವಾ-ದಾಂಶ್ಚ
ಅವಾಪ್ಯ
ಅವಿ-ಕಾರಿ
ಅವಿ-ಕಾ-ರಿ-ಯಾದ
ಅವಿ-ಚ್ಛಿ-ನ್ನ-ವಾಗಿ
ಅವಿ-ತಿದೆ
ಅವಿ-ತಿ-ರುವ
ಅವಿ-ತಿ-ರು-ವುದು
ಅವಿ-ತು-ಕೊಂ-ಡಿ-ರು-ವುವು
ಅವಿ-ದ್ಯಾ-ವಂ-ತರು
ಅವಿದ್ಯೆ
ಅವಿ-ಧಿ-ಪೂ-ರ್ವಕ
ಅವಿ-ನಾ-ಶ-ವಾ-ಗಿಯೂ
ಅವಿ-ನಾಶಿ
ಅವಿ-ನಾ-ಶಿಯಾ
ಅವಿ-ನಾ-ಶಿ-ಯಾ-ಗಿ-ರು-ವ-ವನೂ
ಅವಿ-ನಾ-ಶಿ-ಯಾದ
ಅವಿ-ನಾ-ಶಿ-ಯಾ-ದದ್ದು
ಅವಿ-ನಾ-ಶಿ-ಯಾ-ದುದು
ಅವಿ-ನಾ-ಶಿಯೊ
ಅವಿ-ಭಕ್ತಂ
ಅವಿ-ಭಾ-ಜ್ಯ-ವಾದ
ಅವಿ-ವೇಕ
ಅವಿ-ವೇ-ಕ-ವನ್ನೇ
ಅವಿ-ವೇಕಿ
ಅವಿ-ವೇ-ಕಿ-ಗ-ಳಾಗಿ
ಅವಿ-ವೇ-ಕಿ-ಗ-ಳಾದ
ಅವಿ-ವೇ-ಕಿ-ಗಳು
ಅವು
ಅವು-ಗಳ
ಅವು-ಗಳನ್ನು
ಅವು-ಗಳನ್ನೆಲ್ಲ
ಅವು-ಗಳನ್ನೆಲ್ಲಾ
ಅವು-ಗಳಲ್ಲಿ
ಅವು-ಗ-ಳ-ಲ್ಲಿದೆ
ಅವು-ಗ-ಳ-ಲ್ಲಿಲ್ಲ
ಅವು-ಗ-ಳ-ಲ್ಲೆಲ್ಲ
ಅವು-ಗ-ಳ-ಲ್ಲೆಲ್ಲಾ
ಅವು-ಗ-ಳಾ-ವು-ದನ್ನೂ
ಅವು-ಗ-ಳಾ-ವು-ದ-ನ್ನೂ-ಇ-ಚ್ಛಿ-ಸು-ವು-ದಿಲ್ಲ
ಅವು-ಗ-ಳಾ-ವುವೂ
ಅವು-ಗಳಿಂದ
ಅವು-ಗ-ಳಿ-ಗಿಂ-ತಲೂ
ಅವು-ಗ-ಳಿಗೆ
ಅವು-ಗ-ಳಿ-ಗೆಲ್ಲ
ಅವು-ಗ-ಳಿ-ಗೆಲ್ಲಾ
ಅವು-ಗಳು
ಅವು-ಗ-ಳೆ-ನ್ನೆಲ್ಲಾ
ಅವು-ಗ-ಳೆಲ್ಲ
ಅವು-ಗ-ಳೆ-ಲ್ಲ-ದರ
ಅವು-ಗ-ಳೆ-ಲ್ಲ-ವನ್ನೂ
ಅವು-ಗ-ಳೆಲ್ಲಾ
ಅವು-ಗ-ಳೊಂ-ದಿಗೆ
ಅವು-ಗ-ಳೊ-ಡನೆ
ಅವು-ಗೊಳ
ಅವೆ-ರ-ಡಕ್ಕೂ
ಅವೆ-ರ-ಡನ್ನೂ
ಅವೆ-ರ-ಡ-ರಲ್ಲಿ
ಅವೆ-ರಡೂ
ಅವೆಲ್ಲ
ಅವೆ-ಲ್ಲ-ವನ್ನೂ
ಅವೆಲ್ಲಾ
ಅವೇ
ಅವೇನು
ಅವ್ಯಕ್ತ
ಅವ್ಯಕ್ತಂ
ಅವ್ಯ-ಕ್ತ-ಕ್ಕಿಂ-ತಲೂ
ಅವ್ಯ-ಕ್ತಕ್ಕೆ
ಅವ್ಯ-ಕ್ತ-ಗಳನ್ನು
ಅವ್ಯ-ಕ್ತದ
ಅವ್ಯ-ಕ್ತ-ದಲ್ಲಿ
ಅವ್ಯ-ಕ್ತ-ದಿಂದ
ಅವ್ಯ-ಕ್ತ-ನಾ-ಗಿದ್ದ
ಅವ್ಯ-ಕ್ತ-ನಾಗು
ಅವ್ಯ-ಕ್ತ-ನಿ-ಧ-ನಾ-ನ್ಯೇವ
ಅವ್ಯ-ಕ್ತ-ಭಾವ
ಅವ್ಯ-ಕ್ತ-ಭಾ-ವ-ವಿದೆ
ಅವ್ಯ-ಕ್ತ-ಮೂ-ರ್ತಿ-ಯಾದ
ಅವ್ಯ-ಕ್ತ-ರೂ-ಪ-ದಿಂದ
ಅವ್ಯ-ಕ್ತ-ವಾ-ಗಿದೆ
ಅವ್ಯ-ಕ್ತ-ವಾ-ಗಿದ್ದು
ಅವ್ಯ-ಕ್ತ-ವಾದ
ಅವ್ಯ-ಕ್ತವೂ
ಅವ್ಯ-ಕ್ತ-ವೆಂ-ಬು-ದಿದೆ
ಅವ್ಯ-ಕ್ತವೇ
ಅವ್ಯಕ್ತಾ
ಅವ್ಯ-ಕ್ತಾ-ದೀನಿ
ಅವ್ಯ-ಕ್ತಾ-ದ್ವ್ಯ-ಕ್ತಯಃ
ಅವ್ಯ-ಕ್ತಾ-ವ-ಸ್ಥೆ-ಯ-ಲ್ಲಿ-ರು-ವವೋ
ಅವ್ಯ-ಕ್ತೋ-ಯ-ಮ-ಚಿಂ-ತ್ಯೋ-ಯ-ಮ-ವಿ-ಕಾ-ರ್ಯೋ-ಯ-ಮು-ಚ್ಯತೇ
ಅವ್ಯ-ಕ್ತೋ-ಽಕ್ಷರ
ಅವ್ಯ-ಭಿ-ಚಾ-ರ-ವಾದ
ಅವ್ಯ-ಭಿ-ಚಾ-ರ-ವಾ-ದುದು
ಅವ್ಯ-ಭಿ-ಚಾ-ರಿಣೀ
ಅವ್ಯಯ
ಅವ್ಯ-ಯ-ನಾದ
ಅವ್ಯ-ಯನು
ಅವ್ಯ-ಯನೂ
ಅವ್ಯ-ಯ-ವಾ-ಗಿಯೂ
ಅವ್ಯ-ಯ-ವಾ-ಗಿ-ರುವ
ಅವ್ಯ-ಯ-ವಾದ
ಅವ್ಯ-ಯ-ವಾ-ದದ್ದು
ಅವ್ಯ-ಯ-ವಾ-ದುದು
ಅವ್ಯ-ಯವೂ
ಅವ್ಯ-ವ-ಸಾ-ಯಿ-ಗಳೊ
ಅವ್ಯಾ-ಹ-ತ-ವಾಗಿ
ಅಶ-ಕ್ಯವೂ
ಅಶಮ
ಅಶಾಂತಿ
ಅಶಾಂ-ತಿಯ
ಅಶಾಂ-ತಿ-ಯನ್ನು
ಅಶಾ-ಶ್ವತ
ಅಶಾ-ಶ್ವ-ತ-ವಾ-ದುದು
ಅಶಾ-ಶ್ವ-ತವೂ
ಅಶಾ-ಸ್ತ್ರ-ವಿ-ಹಿತಂ
ಅಶಿಸ್ತು
ಅಶು-ಚಿಯ
ಅಶು-ಚಿ-ಯಾದ
ಅಶು-ಚಿ-ಯಾ-ದುದು
ಅಶು-ಚಿಯೂ
ಅಶು-ದ್ಧವೂ
ಅಶುಭ
ಅಶು-ಭ-ದಿಂದ
ಅಶೋಕ
ಅಶೋ-ಚ್ಯಾ-ನ-ನ್ವ-ಶೋ-ಚಸ್ತ್ವಂ
ಅಶ್ರ-ದ್ಧ-ಧಾ-ನಾಃ
ಅಶ್ರ-ದ್ಧಯಾ
ಅಶ್ರದ್ಧೆ
ಅಶ್ರ-ದ್ಧೆಯ
ಅಶ್ರ-ದ್ಧೆ-ಯಿಂದ
ಅಶ್ಲೀ-ಲ-ವನ್ನು
ಅಶ್ಲೀ-ಲ-ವಾ-ಗಿರು
ಅಶ್ಲೀ-ಲ-ವಾದ
ಅಶ್ವ-ಗಳನ್ನು
ಅಶ್ವತ್ಥ
ಅಶ್ವತ್ಥಃ
ಅಶ್ವ-ತ್ಥ-ಮೇನಂ
ಅಶ್ವ-ತ್ಥ-ವನ್ನು
ಅಶ್ವ-ತ್ಥ-ವೃ-ಕ್ಷದ
ಅಶ್ವ-ತ್ಥಾಮ
ಅಶ್ವ-ತ್ಥಾ-ಮನ
ಅಶ್ವ-ತ್ಥಾ-ಮ-ನನ್ನು
ಅಶ್ವ-ತ್ಥಾ-ಮ-ನಿಂದ
ಅಶ್ವ-ತ್ಥಾ-ಮ-ನಿ-ಗಿಂತ
ಅಶ್ವ-ತ್ಥಾ-ಮ-ರು-ಗಳೇ
ಅಶ್ವ-ತ್ಥಾ-ಮ-ವಿ-ಕ-ರ್ಣ-ಘೋ-ರ-ಮ-ಕರಾ
ಅಶ್ವ-ತ್ಥಾಮಾ
ಅಶ್ವ-ಪತಿ
ಅಶ್ವ-ಮೇಧ
ಅಶ್ವ-ವನ್ನು
ಅಶ್ವಿನಿ
ಅಶ್ವಿನೀ
ಅಶ್ವಿ-ನೀ-ದೇ-ವ-ತೆ-ಗಳು
ಅಷ್ಚಕ್ಕೆ
ಅಷ್ಟಕ್ಕೆ
ಅಷ್ಟ-ನ್ನಾ-ದರೂ
ಅಷ್ಟನ್ನು
ಅಷ್ಟ-ವ-ಸು-ಗಳಲ್ಲಿ
ಅಷ್ಟಾಂ-ಗ-ಯೋ-ಗ-ದಲ್ಲಿ
ಅಷ್ಟಾ-ದ-ಶಾ-ಧ್ಯಾ-ಯಿ-ನೀಂ
ಅಷ್ಟು
ಅಷ್ಟೂ
ಅಷ್ಟೆ
ಅಷ್ಟೆಲ್ಲ
ಅಷ್ಟೇ
ಅಷ್ಟೈ-ಶ್ವರ್ಯ
ಅಷ್ಟೈ-ಶ್ವ-ರ್ಯ-ಗಳ
ಅಷ್ಟೈ-ಶ್ವ-ರ್ಯ-ಗ-ಳಿವೆ
ಅಷ್ಟೈ-ಶ್ವ-ರ್ಯ-ಗಳೂ
ಅಷ್ಟೈ-ಶ್ವ-ರ್ಯ-ಗಳೇ
ಅಷ್ಟೊಂದು
ಅಷ್ಟೋ-ತ್ತರ
ಅಸಂ-ಖ್ಯಾತ
ಅಸಂ-ಗ-ನಾಗಿ
ಅಸಂ-ಗ-ನಾ-ಗಿ-ರು-ವನು
ಅಸಂ-ಗ-ವೆಂಬ
ಅಸಂ-ಮೂಢಃ
ಅಸಂ-ಮೋಹ
ಅಸಂ-ಯ-ತಾ-ತ್ಮನಾ
ಅಸಂ-ಶಯಂ
ಅಸಂ-ಸ್ಕಾರಿ
ಅಸಂ-ಸ್ಕಾ-ರಿ-ಯಲ್ಲಿ
ಅಸ-ಕ್ಕೃ-ತ-ಮ-ವ-ಜ್ಞಾತಂ
ಅಸಕ್ತಂ
ಅಸ-ಕ್ತ-ಬು-ದ್ಧಿಃ
ಅಸ-ಕ್ತಿ-ರ-ನ-ಭಿ-ಷ್ವಂಗಃ
ಅಸಕ್ತೋ
ಅಸಡ್ಡೆ
ಅಸ-ಡ್ಡೆ-ಯಿಂದ
ಅಸತ್
ಅಸ-ತ್ತಿಗೆ
ಅಸ-ತ್ತಿನ
ಅಸತ್ತು
ಅಸತ್ಯ
ಅಸ-ತ್ಯ-ಮ-ಪ್ರ-ತಿಷ್ಠಂ
ಅಸ-ತ್ಯ-ವ-ನ್ನೆಲ್ಲ
ಅಸ-ತ್ಯ-ವಲ್ಲ
ಅಸ-ದಿ-ತ್ಯು-ಚ್ಯತೇ
ಅಸ-ದೃ-ಶ-ವಾದ
ಅಸ-ಮಾ-ಧಾ-ನ-ವನ್ನು
ಅಸಲು
ಅಸ-ಹನೆ
ಅಸ-ಹಾ-ಯಕ
ಅಸಾ-ಧಾ-ರಣ
ಅಸಾಧ್ಯ
ಅಸಾ-ಧ್ಯ-ವನ್ನು
ಅಸಾ-ಧ್ಯ-ವಲ್ಲ
ಅಸಾ-ಧ್ಯ-ವಾಗಿ
ಅಸಾ-ಧ್ಯ-ವಾ-ಗಿ-ರು-ವುದು
ಅಸಾ-ಧ್ಯ-ವಾ-ಗಿ-ರು-ವು-ದೆಲ್ಲ
ಅಸಾ-ಧ್ಯ-ವಾ-ಗು-ವುದು
ಅಸಾ-ಧ್ಯ-ವಾದ
ಅಸಾ-ಧ್ಯ-ವಾ-ದು-ದನ್ನು
ಅಸಾ-ಧ್ಯ-ವಾ-ದುದು
ಅಸಾ-ಧ್ಯವೂ
ಅಸಾ-ಧ್ಯ-ವೆಂ-ದಲ್ಲ
ಅಸಾ-ಧ್ಯವೋ
ಅಸಿತ
ಅಸಿತೋ
ಅಸಿದ್ಧಿ
ಅಸಿ-ದ್ಧಿ-ಗಳಲ್ಲಿ
ಅಸಿ-ದ್ಧಿಯ
ಅಸೀಮ
ಅಸುಖ
ಅಸು-ಖದ
ಅಸು-ಖ-ದಲ್ಲಿ
ಅಸು-ಖವೂ
ಅಸು-ರ-ನನ್ನು
ಅಸು-ರರ
ಅಸು-ರರು
ಅಸೂಯಾ
ಅಸೂ-ಯಾ-ಪ-ರ-ನಾದ
ಅಸೂ-ಯಾ-ರ-ಹಿ-ತ-ನಾದ
ಅಸೂಯೆ
ಅಸೂ-ಯೆ-ಪ-ಡ-ದ-ವನ
ಅಸೂ-ಯೆಯ
ಅಸೂ-ಯೆ-ಯನ್ನು
ಅಸೂ-ಯೆ-ಯಾ-ಗಲಿ
ಅಸೂ-ಯೆ-ಯಿಂದ
ಅಸೂ-ಯೆ-ಯು-ಳ್ಳ-ವ-ರಾ-ಗಿ-ರು-ತ್ತಾರೆ
ಅಸೂ-ಯೆ-ಯು-ಳ್ಳ-ವರೋ
ಅಸೂ-ಯೆಯೆ
ಅಸೌ
ಅಸ್ಟ್ರಾ-ನಮಿ
ಅಸ್ತ-ಮ-ಗಳಲ್ಲಿ
ಅಸ್ತ-ಮ-ಯ-ವನ್ನು
ಅಸ್ತಿತ್ವ
ಅಸ್ತಿ-ತ್ವಕ್ಕೆ
ಅಸ್ತಿ-ಪಂ-ಜ-ರ-ಗಳು
ಅಸ್ತಿ-ಭಾರ
ಅಸ್ತ್ರ-ಗಳನ್ನು
ಅಸ್ತ್ರ-ವನ್ನು
ಅಸ್ತ್ರವೇ
ಅಸ್ಥಿ-ಪಂ-ಜರ
ಅಸ್ಥಿರ
ಅಸ್ಥಿ-ರವೂ
ಅಸ್ಪಷ್ಟ
ಅಸ್ಪ-ಷ್ಟ-ಭಾಗ
ಅಸ್ಪ-ಷ್ಟ-ವಾ-ಗಿಲ್ಲ
ಅಸ್ಮಾಕಂ
ಅಸ್ವ-ತಂ-ತ್ರ-ನಾಗಿ
ಅಸ್ವ-ತಂ-ತ್ರರು
ಅಸ್ವ-ತಂ-ತ್ರ-ವಾದ
ಅಸ್ವ-ಸ್ಥ-ರಾ-ಗು-ವೆವು
ಅಸ್ವಾ-ಭಾ-ವಿಕ
ಅಸ್ವಾ-ಭಾ-ವಿ-ಕ-ತೆಯೂ
ಅಸ್ವಾ-ಭಾ-ವಿ-ಕ-ವಾದ
ಅಸ್ವಾ-ಭಾ-ವಿ-ಕ-ವಾ-ದುದು
ಅಹಂ
ಅಹಂ-ಕಾರ
ಅಹಂ-ಕಾರಂ
ಅಹಂ-ಕಾ-ರ-ಕ-ಲ್ಲಿ-ನಂತೆ
ಅಹಂ-ಕಾ-ರಕ್ಕೆ
ಅಹಂ-ಕಾ-ರ-ಗಳ
ಅಹಂ-ಕಾ-ರ-ಗಳನ್ನು
ಅಹಂ-ಕಾ-ರ-ಗ-ಳಿ-ಗೆಲ್ಲ
ಅಹಂ-ಕಾ-ರ-ಗ-ಳೆಲ್ಲ
ಅಹಂ-ಕಾ-ರ-ಗ-ಳೆಲ್ಲಾ
ಅಹಂ-ಕಾ-ರದ
ಅಹಂ-ಕಾ-ರ-ದಿಂದ
ಅಹಂ-ಕಾ-ರ-ಪ-ಡು-ತ್ತೇವೆ
ಅಹಂ-ಕಾ-ರ-ವನ್ನು
ಅಹಂ-ಕಾ-ರ-ವ-ನ್ನೆಲ್ಲ
ಅಹಂ-ಕಾ-ರ-ವಿ-ಮೂ-ಢಾತ್ಮಾ
ಅಹಂ-ಕಾ-ರ-ವಿ-ರ-ಕೂ-ಡದು
ಅಹಂ-ಕಾ-ರ-ವಿ-ರು-ವಾಗ
ಅಹಂ-ಕಾ-ರ-ವಿ-ರು-ವು-ದಿಲ್ಲ
ಅಹಂ-ಕಾ-ರ-ವಿಲ್ಲ
ಅಹಂ-ಕಾ-ರ-ವಿ-ಲ್ಲ-ದಿ-ರು-ವುದು
ಅಹಂ-ಕಾ-ರ-ವೆಲ್ಲ
ಅಹಂ-ಕಾ-ರವೇ
ಅಹಂ-ಕಾರಿ
ಅಹಂ-ಕಾ-ರಿಯು
ಅಹ-ಮಾತ್ಮಾ
ಅಹ-ಮಾ-ದಿರ್ಹಿ
ಅಹ-ಮಾ-ದಿಶ್ಚ
ಅಹ-ಮೇ-ವಂ-ವಿ-ಧೊ-ಽಜುನ
ಅಹ-ಮೇ-ವಾ-ಕ್ಷಯಃ
ಅಹಿಂಸಾ
ಅಹಿಂ-ಸಾ-ಭಾ-ವ-ನೆ-ಯಿಂದ
ಅಹಿಂ-ಸಾ-ವ್ರ-ತ-ಧಾರಿ
ಅಹಿಂ-ಸಾ-ವ್ರತಿ
ಅಹಿಂಸೆ
ಅಹಿಂ-ಸೆಯ
ಅಹಿತ
ಅಹಿ-ತ-ವನ್ನು
ಅಹಿ-ತ-ವಾಗಿ
ಅಹಿ-ತ-ವಾ-ಗಿ-ರು-ವು-ದಾ-ದರೆ
ಅಹಿ-ತ-ವಾದ
ಅಹೋ
ಆ
ಆಂತ-ರಿಕ
ಆಂತ-ರ್ಯ-ದಲ್ಲಿ
ಆಂತ್ಯ-ವೆಂ-ಬುದು
ಆಂದೋ-ಳ-ನ-ಗ-ಳಾ-ವುವೂ
ಆಂದೋ-ಳ-ನ-ವಾ-ಗು-ವುದು
ಆಕ-ರ್ಷ-ಕ-ವಾ-ಗಿದೆ
ಆಕ-ರ್ಷಣ
ಆಕ-ರ್ಷ-ಣದ
ಆಕ-ರ್ಷ-ಣ-ದಿಂದ
ಆಕ-ರ್ಷ-ಣ-ವನ್ನು
ಆಕ-ರ್ಷ-ಣೀ-ಯ-ವಾಗಿ
ಆಕ-ರ್ಷಣೆ
ಆಕ-ರ್ಷ-ಣೆ-ಗಳಿಂದ
ಆಕ-ರ್ಷ-ಣೆ-ಗ-ಳಿಗೆ
ಆಕ-ರ್ಷ-ಣೆಗೂ
ಆಕ-ರ್ಷ-ಣೆಗೆ
ಆಕ-ರ್ಷ-ಣೆಯ
ಆಕ-ರ್ಷ-ಣೆ-ಯಂತೆ
ಆಕ-ರ್ಷ-ಣೆ-ಯನ್ನು
ಆಕ-ರ್ಷ-ಣೆ-ಯಲ್ಲಿ
ಆಕ-ರ್ಷ-ಣೆ-ಯಿಂದ
ಆಕ-ರ್ಷ-ಣೆಯೂ
ಆಕ-ರ್ಷಿ-ತ-ನಾಗಿ
ಆಕ-ರ್ಷಿ-ತ-ನಾ-ಗು-ತ್ತಾನೆ
ಆಕ-ರ್ಷಿ-ತ-ನಾ-ಗು-ವು-ದಿಲ್ಲ
ಆಕ-ರ್ಷಿ-ತ-ನಾ-ದರೆ
ಆಕ-ರ್ಷಿ-ತ-ರಾ-ಗು-ವರು
ಆಕ-ರ್ಷಿ-ಸ-ಲಾ-ರದು
ಆಕ-ರ್ಷಿ-ಸ-ಲ್ಪ-ಡು-ವನು
ಆಕ-ರ್ಷಿಸಿ
ಆಕ-ರ್ಷಿ-ಸುವ
ಆಕ-ರ್ಷಿ-ಸು-ವಷ್ಟು
ಆಕ-ರ್ಷಿ-ಸು-ವು-ದಕ್ಕೆ
ಆಕ-ರ್ಷಿ-ಸು-ವುದು
ಆಕ-ಸ್ಮಿಕ
ಆಕ-ಸ್ಮಿ-ಕ-ದಂತೆ
ಆಕಾಂಕ್ಷೆ
ಆಕಾಂ-ಕ್ಷೆ-ಗಳ
ಆಕಾಂ-ಕ್ಷೆ-ಗಳನ್ನು
ಆಕಾಂ-ಕ್ಷೆ-ಗಳನ್ನೂ
ಆಕಾಂ-ಕ್ಷೆ-ಗಳಿಂದ
ಆಕಾಂ-ಕ್ಷೆ-ಗ-ಳಿಗೆ
ಆಕಾಂ-ಕ್ಷೆ-ಗ-ಳಿವೆ
ಆಕಾಂ-ಕ್ಷೆ-ಗಳು
ಆಕಾಂ-ಕ್ಷೆ-ಗಳೂ
ಆಕಾಂ-ಕ್ಷೆ-ಗ-ಳೆಲ್ಲ
ಆಕಾಂ-ಕ್ಷೆಯೂ
ಆಕಾರ
ಆಕಾ-ರ-ಗಳ
ಆಕಾ-ರ-ಗಳನ್ನು
ಆಕಾ-ರ-ಗಳನ್ನೆಲ್ಲಾ
ಆಕಾ-ರ-ಗ-ಳಾ-ಗಿವೆ
ಆಕಾ-ರ-ಗ-ಳಿವೆ
ಆಕಾ-ರ-ಗ-ಳುಳ್ಳ
ಆಕಾ-ರ-ಗಳೂ
ಆಕಾ-ರದ
ಆಕಾ-ರ-ದಲ್ಲಿ
ಆಕಾ-ರ-ದ-ಲ್ಲಿತ್ತು
ಆಕಾ-ರ-ದ-ಲ್ಲಿ-ರುವ
ಆಕಾ-ರ-ದಲ್ಲೆ
ಆಕಾ-ರ-ದಿಂದ
ಆಕಾ-ರನೂ
ಆಕಾ-ರ-ವನ್ನು
ಆಕಾ-ರ-ವನ್ನೂ
ಆಕಾ-ರ-ವಲ್ಲ
ಆಕಾ-ರ-ವ-ಲ್ಲವೆ
ಆಕಾ-ರ-ವಾ-ಗಲಿ
ಆಕಾ-ರ-ವಿ-ಲ್ಲದ
ಆಕಾ-ರವೂ
ಆಕಾ-ರ-ವೆಲ್ಲ
ಆಕಾ-ರವೇ
ಆಕಾಶ
ಆಕಾ-ಶ-ಕ್ಕಿಂತ
ಆಕಾ-ಶಕ್ಕೆ
ಆಕಾ-ಶ-ಗಳಲ್ಲಿ
ಆಕಾ-ಶದ
ಆಕಾ-ಶ-ದಂ-ತಿ-ರುವ
ಆಕಾ-ಶ-ದಂತೆ
ಆಕಾ-ಶ-ದಲ್ಲಿ
ಆಕಾ-ಶ-ದ-ಲ್ಲಿದೆ
ಆಕಾ-ಶ-ದ-ಲ್ಲಿ-ದೆಯೊ
ಆಕಾ-ಶ-ದ-ಲ್ಲಿ-ದೆಯೋ
ಆಕಾ-ಶ-ದ-ಲ್ಲಿ-ದ್ದರೆ
ಆಕಾ-ಶ-ದ-ಲ್ಲಿ-ರುವ
ಆಕಾ-ಶ-ದ-ಲ್ಲಿ-ರು-ವುದನ್ನು
ಆಕಾ-ಶ-ದಿಂದ
ಆಕಾ-ಶ-ದೊ-ಳಗೆ
ಆಕಾ-ಶ-ವನ್ನು
ಆಕಾ-ಶ-ವಾಗಿ
ಆಕಾ-ಶ-ವಾ-ಗಿತ್ತು
ಆಕಾ-ಶ-ವಿದೆ
ಆಕಾ-ಶ-ವಿ-ಮಾನ
ಆಕಾ-ಶವೂ
ಆಕಾ-ಶ-ವೆಂಬ
ಆಕಾ-ಶ-ವೆಲ್ಲಾ
ಆಕಾ-ಶವೇ
ಆಕೃ-ತಿ-ಗ-ಳುಳ್ಳ
ಆಕೆ
ಆಕ್ರ-ಮಿ-ಸಿ-ಕೊಳ್ಳು
ಆಕ್ಷೇ-ಪಣೆ
ಆಕ್ಷೇ-ಪ-ಣೆ-ಗಳು
ಆಖ್ಯಾಹಿ
ಆಗ
ಆಗಂ-ತುಕ
ಆಗ-ತಕ್ಕ
ಆಗ-ತಾನೆ
ಆಗದ
ಆಗ-ದ-ವ-ನಿಗೂ
ಆಗ-ದ-ವ-ನಿಗೆ
ಆಗ-ದ-ವನು
ಆಗ-ದ-ವ-ರಿಗೆ
ಆಗ-ದ-ವರು
ಆಗ-ದಿ-ರಲಿ
ಆಗದು
ಆಗ-ದು-ದನ್ನು
ಆಗದೆ
ಆಗದೇ
ಆಗ-ಬ-ಲ್ಲದು
ಆಗ-ಬ-ಲ್ಲುದು
ಆಗ-ಬ-ಹುದು
ಆಗ-ಬಾ-ರ-ದಾ-ಗಿತ್ತು
ಆಗ-ಬಾ-ರದು
ಆಗ-ಬಾ-ರ-ದುದೇ
ಆಗ-ಬೇಕಾ
ಆಗ-ಬೇ-ಕಾ-ಗಿತ್ತು
ಆಗ-ಬೇ-ಕಾ-ಗಿದೆ
ಆಗ-ಬೇ-ಕಾ-ಗಿ-ದೆಯೊ
ಆಗ-ಬೇ-ಕಾ-ಗಿ-ದೆಯೋ
ಆಗ-ಬೇ-ಕಾ-ಗಿಲ್ಲ
ಆಗ-ಬೇ-ಕಾ-ಗು-ವುದು
ಆಗ-ಬೇ-ಕಾದ
ಆಗ-ಬೇ-ಕಾ-ದರೂ
ಆಗ-ಬೇ-ಕಾ-ದರೆ
ಆಗ-ಬೇಕು
ಆಗ-ಬೇಕೆ
ಆಗ-ಬೇ-ಕೆಂದು
ಆಗ-ಬೇಕೊ
ಆಗ-ಮಾತ್ರ
ಆಗ-ಮಾ-ಪಾ-ಯಿ-ನೋ-ನಿ-ತ್ಯಾ-ಸ್ತಾಂ-ಸ್ತಿ-ತಿ-ಕ್ಷಸ್ವ
ಆಗ-ಲಾ-ದರು
ಆಗ-ಲಾರ
ಆಗ-ಲಾ-ರದ
ಆಗ-ಲಾ-ರದು
ಆಗಲಿ
ಆಗ-ಲಿಲ್ಲ
ಆಗ-ಲಿ-ಲ್ಲವೆ
ಆಗ-ಲಿವೆ
ಆಗಲೀ
ಆಗಲೂ
ಆಗಲೆ
ಆಗಲೇ
ಆಗ-ಲೇ-ಬೇ-ಕಾ-ಗಿದೆ
ಆಗ-ಲೇ-ಬೇಕು
ಆಗ-ಸ-ದಷ್ಚು
ಆಗಾ-ರ-ದಿಂ-ದಲೇ
ಆಗಿ
ಆಗಿದೆ
ಆಗಿ-ದೆಯೆ
ಆಗಿ-ದೆಯೇ
ಆಗಿ-ದೆಯೊ
ಆಗಿ-ದೆಯೋ
ಆಗಿದ್ದ
ಆಗಿ-ದ್ದರೂ
ಆಗಿ-ದ್ದರೆ
ಆಗಿ-ದ್ದ-ವರು
ಆಗಿ-ದ್ದಾನೆ
ಆಗಿ-ದ್ದಾರೆ
ಆಗಿದ್ದು
ಆಗಿ-ದ್ದೆನು
ಆಗಿ-ದ್ದೆವು
ಆಗಿ-ದ್ದೇನೆ
ಆಗಿ-ದ್ದೇವೆ
ಆಗಿನ
ಆಗಿ-ಬಿ-ಡು-ವನೆ
ಆಗಿಯೇ
ಆಗಿರ
ಆಗಿ-ರ-ಬ-ಹುದು
ಆಗಿ-ರ-ಬೇ-ಕಾ-ಗಿಲ್ಲ
ಆಗಿ-ರ-ಬೇಕು
ಆಗಿ-ರಲಿ
ಆಗಿ-ರ-ಲಿಲ್ಲ
ಆಗಿರು
ಆಗಿ-ರು-ತ್ತವೆ
ಆಗಿ-ರುವ
ಆಗಿ-ರು-ವನು
ಆಗಿ-ರು-ವನೊ
ಆಗಿ-ರು-ವನೋ
ಆಗಿ-ರು-ವರು
ಆಗಿ-ರು-ವ-ವ-ನನ್ನು
ಆಗಿ-ರು-ವ-ವನು
ಆಗಿ-ರು-ವ-ವರ
ಆಗಿ-ರು-ವಾಗ
ಆಗಿ-ರು-ವು-ದ-ನ್ನೆಲ್ಲಾ
ಆಗಿ-ರು-ವು-ದ-ರಿಂದ
ಆಗಿ-ರು-ವುದು
ಆಗಿ-ರು-ವು-ದೆಲ್ಲಾ
ಆಗಿ-ರು-ವುದೇ
ಆಗಿ-ರು-ವುದೋ
ಆಗಿ-ರುವೆ
ಆಗಿ-ರು-ವೆನು
ಆಗಿಲ್ಲ
ಆಗಿ-ಲ್ಲವೋ
ಆಗಿವೆ
ಆಗಿ-ವೆ-ಇ-ವು-ಗಳನ್ನೆಲ್ಲಾ
ಆಗಿ-ಹೋ-ಗಿದೆ
ಆಗಿ-ಹೋ-ಗಿದ್ದು
ಆಗಿ-ಹೋ-ಗಿ-ರು-ವುದು
ಆಗಿ-ಹೋ-ಗಿವೆ
ಆಗಿ-ಹೋ-ಗು-ತ್ತವೆ
ಆಗಿ-ಹೋ-ಗು-ವುದು
ಆಗಿ-ಹೋದ
ಆಗಿ-ಹೋ-ದು-ದನ್ನು
ಆಗಿ-ಹೋ-ಯಿತು
ಆಗು
ಆಗುತ್ತ
ಆಗು-ತ್ತದೆ
ಆಗು-ತ್ತಲೇ
ಆಗು-ತ್ತವೆ
ಆಗು-ತ್ತ-ವೆಯೋ
ಆಗು-ತ್ತಾ-ನಲ್ಲ
ಆಗು-ತ್ತಾನೆ
ಆಗು-ತ್ತಾರೆ
ಆಗು-ತ್ತಿತ್ತು
ಆಗು-ತ್ತಿದೆ
ಆಗು-ತ್ತಿ-ದೆಯೊ
ಆಗು-ತ್ತಿದ್ದ
ಆಗು-ತ್ತಿ-ದ್ದರೂ
ಆಗು-ತ್ತಿ-ದ್ದರೆ
ಆಗು-ತ್ತಿ-ದ್ದೆವು
ಆಗು-ತ್ತಿ-ರ-ಬೇಕು
ಆಗು-ತ್ತಿ-ರ-ಲಿಲ್ಲ
ಆಗು-ತ್ತಿರು
ಆಗು-ತ್ತಿ-ರು-ತ್ತವೆ
ಆಗು-ತ್ತಿ-ರುವ
ಆಗು-ತ್ತಿ-ರುವು
ಆಗು-ತ್ತಿ-ರು-ವುದನ್ನು
ಆಗು-ತ್ತಿ-ರು-ವುದು
ಆಗು-ತ್ತಿ-ರು-ವುದೊ
ಆಗು-ತ್ತಿ-ರು-ವು-ದೊಂದೇ
ಆಗು-ತ್ತಿವೆ
ಆಗು-ತ್ತೇನೆ
ಆಗು-ತ್ತೇ-ನೆಯೋ
ಆಗು-ತ್ತೇವೆ
ಆಗುವ
ಆಗು-ವಂ-ತಿ-ದ್ದರೆ
ಆಗು-ವಂ-ತಿಲ್ಲ
ಆಗು-ವಂತೆ
ಆಗು-ವನು
ಆಗು-ವರು
ಆಗು-ವ-ವನೂ
ಆಗು-ವ-ವ-ರೆಗೆ
ಆಗು-ವಷ್ಟು
ಆಗು-ವಾಗ
ಆಗುವು
ಆಗು-ವು-ದಕ್ಕೆ
ಆಗು-ವುದನ್ನು
ಆಗು-ವು-ದ-ರ-ಲ್ಲಿವೆ
ಆಗು-ವು-ದಲ್ಲ
ಆಗು-ವು-ದಿದೆ
ಆಗು-ವು-ದಿಲ್ಲ
ಆಗು-ವು-ದಿ-ಲ್ಲ-ವಲ್ಲ
ಆಗು-ವು-ದಿ-ಲ್ಲವೆ
ಆಗು-ವು-ದಿ-ಲ್ಲ-ವೆಂದು
ಆಗು-ವು-ದಿ-ಲ್ಲವೊ
ಆಗು-ವು-ದಿ-ಲ್ಲವೋ
ಆಗು-ವುದು
ಆಗು-ವುದೆ
ಆಗು-ವು-ದೆಂದು
ಆಗು-ವು-ದೆಲ್ಲ
ಆಗು-ವುದೇ
ಆಗು-ವುದೊ
ಆಗು-ವುದೋ
ಆಗು-ವುವು
ಆಗು-ವೆವು
ಆಗೊಂದು
ಆಗೋಣ
ಆಘಾ-ತದ
ಆಘ್ರಾ-ಣಿ-ಸ-ಬಾ-ರದು
ಆಚಂ-ದ್ರ-ರ್ಕ-ವಾಗಿ
ಆಚಂ-ದ್ರಾ-ರ್ಕ-ವಾಗಿ
ಆಚಂ-ದ್ರಾ-ರ್ಕ-ವಾದ
ಆಚಂ-ದ್ರಾ-ರ್ಕ-ವಾ-ದುದು
ಆಚ-ರಣೆ
ಆಚ-ರ-ತ್ಯಾ-ತ್ಮನಃ
ಆಚ-ರಿ-ಸು-ತ್ತಾರೆ
ಆಚ-ರಿ-ಸು-ತ್ತಾ-ರೆಯೋ
ಆಚ-ರಿ-ಸು-ವರು
ಆಚ-ರಿ-ಸು-ವು-ದಕ್ಕೆ
ಆಚ-ರಿ-ಸು-ವು-ದಿಲ್ಲ
ಆಚಾರ
ಆಚಾ-ರ-ಗಳು
ಆಚಾ-ರ-ಗಳೂ
ಆಚಾ-ರದ
ಆಚಾ-ರ-ವಾ-ಗಿತ್ತು
ಆಚಾ-ರ-ವಿತ್ತು
ಆಚಾ-ರ-ವಿಲ್ಲ
ಆಚಾ-ರವೂ
ಆಚಾರ್ಯ
ಆಚಾ-ರ್ಯನೇ
ಆಚಾ-ರ್ಯ-ಮು-ಪ-ಸಂ-ಗಮ್ಯ
ಆಚಾ-ರ್ಯಾಃ
ಆಚಾ-ರ್ಯಾ-ನ್ಮಾ-ತು-ಲಾನ್
ಆಚಾರ್ಯೋ
ಆಚಾ-ರ್ಯೋ-ಪಾ-ಸನಂ
ಆಚಾ-ರ್ಯೋ-ಪಾ-ಸನೆ
ಆಚಾ-ರ್ಯೋ-ಪಾ-ಸ-ನೆ-ಯನ್ನು
ಆಚೆ
ಆಚೆಗೆ
ಆಚೆ-ಗೊ-ಯ್ಯು-ವೆವೋ
ಆಚ್ಛಾ-ದಿ-ತ-ವಾ-ಗಿ-ರು-ವಾಗ
ಆಚ್ಛಾ-ದಿ-ತ-ವಾ-ಗಿ-ರು-ವುದು
ಆಜ್ಞಾ
ಆಜ್ಞಾ-ಧಾ-ರಕ
ಆಜ್ಞಾನು
ಆಜ್ಞಾ-ನು-ಸಾರ
ಆಜ್ಞಾ-ನು-ಸಾ-ರ-ವಾಗಿ
ಆಜ್ಞಾ-ಪಿ-ಸಿ-ದನು
ಆಜ್ಞಾ-ಪಿ-ಸಿ-ದಾಗ
ಆಜ್ಞಾ-ಪಿ-ಸು-ತ್ತಿ-ರು-ವನೋ
ಆಜ್ಞೆ
ಆಜ್ಞೆಗೆ
ಆಜ್ಞೆ-ಯನ್ನು
ಆಜ್ಯ
ಆಟ
ಆಟಂ-ಬಾಂ-ಬು-ಗಳನ್ನು
ಆಟಕ್ಕೆ
ಆಟ-ಗಳು
ಆಟ-ಗಾರ
ಆಟದ
ಆಟ-ದ-ಲ್ಲಾ-ಗಲೀ
ಆಟ-ದಲ್ಲಿ
ಆಟ-ದ-ಲ್ಲಿಯೇ
ಆಟ-ದಿಂದ
ಆಟ-ಪಾ-ಟ-ಗಳಲ್ಲಿ
ಆಟ-ವನ್ನು
ಆಟ-ವಾ-ಡದೆ
ಆಟ-ವಾ-ಡು-ವಾ-ಗಲೊ
ಆಟವೇ
ಆಟೋಪ
ಆಡಂ-ಬ-ರ-ಗಳೂ
ಆಡ-ಕೂ-ಡದು
ಆಡದೆ
ಆಡ-ಬ-ಹುದು
ಆಡ-ಬಾ-ರದ
ಆಡ-ಬೇ-ಕಾ-ಗಿದೆ
ಆಡ-ಬೇ-ಕಾ-ಗಿಲ್ಲ
ಆಡ-ಬೇ-ಕೆಂದು
ಆಡಿ
ಆಡಿ-ಕೊ-ಳ್ಳು-ತ್ತಾರೆ
ಆಡಿ-ಕೊ-ಳ್ಳು-ವರು
ಆಡಿ-ಕೊ-ಳ್ಳು-ವಾಗ
ಆಡಿ-ಕೊ-ಳ್ಳು-ವು-ದಕ್ಕೆ
ಆಡಿ-ಕೊ-ಳ್ಳು-ವುದೂ
ಆಡಿ-ದ-ವರು
ಆಡಿ-ದ್ದಲ್ಲ
ಆಡಿ-ರು-ವನು
ಆಡಿ-ಸ-ಬಲ್ಲ
ಆಡಿ-ಸಿ-ರು-ವನು
ಆಡಿ-ಸು-ತ್ತಿ-ರು-ವನು
ಆಡಿ-ಸು-ತ್ತಿ-ರು-ವುದನ್ನು
ಆಡಿ-ಸುವ
ಆಡಿ-ಸು-ವನು
ಆಡಿ-ಸು-ವರು
ಆಡಿ-ಸು-ವ-ವನ
ಆಡಿ-ಸು-ವ-ವ-ನಿಗೆ
ಆಡಿ-ಸು-ವ-ವನು
ಆಡಿ-ಸು-ವು-ದಕ್ಕೆ
ಆಡಿ-ಸು-ವುದು
ಆಡು-ಗಳ
ಆಡು-ತ್ತಾನೆ
ಆಡು-ತ್ತಾರೆ
ಆಡು-ತ್ತಾರೊ
ಆಡು-ತ್ತಿ-ದ್ದರೂ
ಆಡು-ತ್ತಿ-ರು-ತ್ತಾರೆ
ಆಡು-ತ್ತಿ-ರುವ
ಆಡು-ತ್ತಿ-ರು-ವನು
ಆಡು-ತ್ತಿ-ರು-ವರು
ಆಡು-ತ್ತಿ-ರು-ವಾಗ
ಆಡು-ತ್ತಿ-ರು-ವೆವು
ಆಡು-ತ್ತೇವೆ
ಆಡುವ
ಆಡು-ವಂತೆ
ಆಡು-ವನು
ಆಡು-ವರು
ಆಡು-ವರೊ
ಆಡು-ವ-ವ-ನಿಗೆ
ಆಡು-ವ-ವನು
ಆಡು-ವ-ವ-ರಿ-ಗೆಲ್ಲ
ಆಡು-ವಾಗ
ಆಡು-ವು-ದಕ್ಕೆ
ಆಡು-ವುದನ್ನು
ಆಡು-ವು-ದ-ರಲ್ಲಿ
ಆಡು-ವು-ದಿಲ್ಲ
ಆಡು-ವುದು
ಆಢ್ಯೋ-ಽಭಿ-ಜ-ನ-ವಾ-ನಸ್ಮಿ
ಆಣ-ತಿ-ಯಂ-ತೆಯೇ
ಆಣ-ತಿ-ಯನ್ನು
ಆಣೆ
ಆತ
ಆತಂಕ
ಆತಂ-ಕ-ಗಳ
ಆತಂ-ಕ-ಗಳನ್ನು
ಆತಂ-ಕ-ಗಳನ್ನೆಲ್ಲ
ಆತಂ-ಕ-ಗ-ಳಿ-ರ-ಬ-ಹುದು
ಆತಂ-ಕ-ಗಳು
ಆತಂ-ಕ-ಗ-ಳೆಲ್ಲ
ಆತಂ-ಕ-ಗಳೇ
ಆತಂ-ಕ-ಗ-ಳೊಂ-ದಿಗೆ
ಆತಂ-ಕ-ಗ-ಳೊ-ಡನೆ
ಆತಂ-ಕ-ದಿಂದ
ಆತಂ-ಕ-ದೊ-ಡನೆ
ಆತಂ-ಕ-ಪ್ರಾ-ಯ-ರಾ-ಗಿ-ರು-ವರೊ
ಆತಂ-ಕ-ವನ್ನು
ಆತಂ-ಕ-ವನ್ನೂ
ಆತಂ-ಕ-ವ-ನ್ನೆಲ್ಲ
ಆತಂ-ಕ-ವಾಗಿ
ಆತಂ-ಕ-ವಾ-ಗಿ-ರುವ
ಆತಂ-ಕ-ವಾ-ಗಿ-ರು-ವು-ದ-ನ್ನೆಲ್ಲಾ
ಆತಂ-ಕ-ವಾ-ದ-ವು-ಗಳ
ಆತಂ-ಕವೂ
ಆತಂ-ಕ-ವೆಲ್ಲ
ಆತ-ತಾಯಿ
ಆತ-ತಾ-ಯಿ-ಗಳನ್ನು
ಆತ-ತಾ-ಯಿ-ಗಳು
ಆತನ
ಆತ-ನನ್ನು
ಆತ-ನಲ್ಲಿ
ಆತ-ನಿಗೆ
ಆತನು
ಆತನೆ
ಆತುರ
ಆತು-ರ-ದಲ್ಲಿ
ಆತ್ಮ
ಆತ್ಮ-ಕ-ಲ್ಯಾಣ
ಆತ್ಮ-ಕ-ಲ್ಯಾ-ಣಕ್ಕೆ
ಆತ್ಮಕ್ಕೆ
ಆತ್ಮ-ಗಳನ್ನು
ಆತ್ಮ-ಗ-ಳನ್ನೇ
ಆತ್ಮ-ಜ್ಞನು
ಆತ್ಮ-ಜ್ಞಾನ
ಆತ್ಮ-ಜ್ಞಾ-ನದ
ಆತ್ಮ-ಜ್ಞಾ-ನ-ದಲ್ಲಿ
ಆತ್ಮ-ಜ್ಞಾ-ನವು
ಆತ್ಮ-ಜ್ಞಾ-ನಿಗೆ
ಆತ್ಮ-ತೃಪ್ತ
ಆತ್ಮ-ತೃ-ಪ್ತ-ನಾಗಿ
ಆತ್ಮ-ತೃ-ಪ್ತ-ನಾ-ಗಿ-ರು-ವನೋ
ಆತ್ಮ-ತೃ-ಪ್ತ-ನಾ-ಗಿ-ರು-ವ-ವ-ನಿಗೆ
ಆತ್ಮ-ತೃ-ಪ್ತ-ನಿಗೆ
ಆತ್ಮ-ತೃ-ಪ್ತಿ-ಗಲ್ಲ
ಆತ್ಮ-ತೃ-ಪ್ತಿಗೆ
ಆತ್ಮದ
ಆತ್ಮ-ದ-ರ್ಶನ
ಆತ್ಮ-ದಲ್ಲಿ
ಆತ್ಮ-ದ-ಲ್ಲಿಯೇ
ಆತ್ಮ-ದೃ-ಷ್ಟಿ-ಯಿಂದ
ಆತ್ಮನ
ಆತ್ಮ-ನನ್ನು
ಆತ್ಮ-ನಲ್ಲಿ
ಆತ್ಮ-ನ-ಲ್ಲಿಯೂ
ಆತ್ಮ-ನ-ಲ್ಲಿಯೇ
ಆತ್ಮ-ನ-ಲ್ಲಿ-ರು-ವು-ದೆಂದು
ಆತ್ಮ-ನಷ್ಟು
ಆತ್ಮ-ನಾ-ಗಲಿ
ಆತ್ಮ-ನಾ-ದರೊ
ಆತ್ಮ-ನಾ-ದರೋ
ಆತ್ಮ-ನಾ-ಶಕ್ಕೆ
ಆತ್ಮ-ನಿಂದ
ಆತ್ಮ-ನಿಂ-ದಲೇ
ಆತ್ಮ-ನಿ-ಗಲ್ಲ
ಆತ್ಮ-ನಿಗೂ
ಆತ್ಮ-ನಿಗೆ
ಆತ್ಮ-ನಿ-ಗ್ರಹ
ಆತ್ಮ-ನಿ-ರ-ಬೇ-ಕಾ-ದರೆ
ಆತ್ಮ-ನಿ-ರ-ಬೇಕು
ಆತ್ಮ-ನಿ-ರು-ವನು
ಆತ್ಮನು
ಆತ್ಮನೇ
ಆತ್ಮ-ನ್ಯೇವ
ಆತ್ಮ-ನ್ಯೇ-ವಾ-ತ್ಮನಾ
ಆತ್ಮ-ಪ್ರ-ಭೆ-ಯನ್ನು
ಆತ್ಮ-ಬು-ದ್ಧಿಯ
ಆತ್ಮ-ರತಿ
ಆತ್ಮ-ರ-ತಿ-ಯಾಗಿ
ಆತ್ಮ-ರ-ತಿ-ಯಾ-ಗಿ-ರು-ವನೋ
ಆತ್ಮ-ವಂ-ಚನೆ
ಆತ್ಮ-ವಂತಂ
ಆತ್ಮ-ವಂ-ತ-ನಾ-ಗಿ-ರು-ವನೋ
ಆತ್ಮ-ವಂ-ತ-ನಾಗು
ಆತ್ಮ-ವಂ-ತನೂ
ಆತ್ಮ-ವನ್ನು
ಆತ್ಮ-ವ-ಶ-ವಾ-ಗು-ವಂತೆ
ಆತ್ಮ-ವ-ಶ್ಯೈ-ರ್ವಿ-ಧೇ-ಯಾತ್ಮಾ
ಆತ್ಮ-ವ-ಸ್ತು-ವನ್ನು
ಆತ್ಮ-ವ-ಸ್ತು-ವಿನ
ಆತ್ಮ-ವ-ಸ್ತುವೊ
ಆತ್ಮ-ವಾ-ದರೊ
ಆತ್ಮ-ವಾನ್
ಆತ್ಮ-ವಿ-ಕಾ-ಸಕ್ಕೆ
ಆತ್ಮ-ವಿ-ಚಾ-ರ-ದಲ್ಲಿ
ಆತ್ಮ-ವಿ-ದ್ದರೂ
ಆತ್ಮವು
ಆತ್ಮ-ವೆಂದು
ಆತ್ಮವೇ
ಆತ್ಮ-ಶು-ದ್ಧಿ-ಗಾಗಿ
ಆತ್ಮ-ಶ್ರದ್ಧೆ
ಆತ್ಮ-ಶ್ರ-ದ್ಧೆಯ
ಆತ್ಮ-ಶ್ಲಾ-ಘನೆ
ಆತ್ಮ-ಸಂ-ಭಾ-ವಿ-ತಾಃ
ಆತ್ಮ-ಸಂ-ಯಮ
ಆತ್ಮ-ಸಂ-ಯ-ಮ-ಯೋ-ಗಾಗ್ನೌ
ಆತ್ಮ-ಸಂಸ್ಥಂ
ಆತ್ಮ-ಸಾ-ಕ್ಷಾ-ತ್ಕಾರ
ಆತ್ಮ-ಸಾ-ಕ್ಷಾ-ತ್ಕಾ-ರ-ಕ್ಕಾಗಿ
ಆತ್ಮ-ಸಾ-ಕ್ಷಾ-ತ್ಕಾ-ರ-ವನ್ನು
ಆತ್ಮ-ಸಾ-ಕ್ಷಾ-ತ್ಕಾ-ರ-ವಾ-ಗಿದೆ
ಆತ್ಮ-ಸ್ವ-ರೂ-ಪ-ನಾ-ಗಿ-ರು-ವನು
ಆತ್ಮ-ಹ-ತ್ಯ-ವನ್ನು
ಆತ್ಮ-ಹತ್ಯೆ
ಆತ್ಮ-ಹಾನಿ
ಆತ್ಮಾ-ರಾಮ
ಆತ್ಮೀ-ಯ-ವಾದ
ಆತ್ಮೈವ
ಆತ್ಮೋ-ದ್ಧಾ-ರ-ಕ್ಕಾಗಿ
ಆತ್ಮೋ-ದ್ಧಾ-ರಕ್ಕೆ
ಆತ್ಮೌ-ಪ-ಮ್ಯೇನ
ಆದ
ಆದಂತೆ
ಆದ-ಕಾ-ರಣ
ಆದ-ಕಾರಣವೆ
ಆದ-ಕಾರಣವೇ
ಆದ-ದ್ದಲ್ಲ
ಆದದ್ದು
ಆದದ್ದೇ
ಆದ-ಮೇಲೂ
ಆದ-ಮೇಲೆ
ಆದ-ಮೇ-ಲೆಯೇ
ಆದ-ರಿಂದ
ಆದರೂ
ಆದರೆ
ಆದ-ರೇ-ನಂತೆ
ಆದ-ರೇನು
ಆದರೊ
ಆದರೋ
ಆದರ್ಶ
ಆದ-ರ್ಶ-ಕರ್ಮ
ಆದ-ರ್ಶ-ಕ್ಕಾಗಿ
ಆದ-ರ್ಶಕ್ಕೆ
ಆದ-ರ್ಶ-ಗ-ಳಿಗೆ
ಆದ-ರ್ಶದ
ಆದ-ರ್ಶ-ದಂತೆ
ಆದ-ರ್ಶ-ದಲ್ಲಿ
ಆದ-ರ್ಶ-ದ-ಲ್ಲಿಯೇ
ಆದ-ರ್ಶ-ದ-ಲ್ಲಿ-ರು-ವನು
ಆದ-ರ್ಶ-ದಿಂದ
ಆದ-ರ್ಶ-ವನ್ನು
ಆದ-ರ್ಶ-ವಾಗಿ
ಆದ-ರ್ಶ-ವಾ-ಗಿ-ಟ್ಟು-ಕೊಂ-ಡರೆ
ಆದ-ರ್ಶ-ವಾ-ಗಿ-ಟ್ಟು-ಕೊಂಡು
ಆದ-ರ್ಶ-ವಾ-ಗು-ವುದು
ಆದ-ರ್ಶ-ವಿದೆ
ಆದ-ರ್ಶ-ವಿ-ಲ್ಲದೆ
ಆದ-ರ್ಶವೂ
ಆದ-ರ್ಶ-ವೆಂದು
ಆದ-ರ್ಶವೇ
ಆದ-ವ-ನನ್ನು
ಆದ-ವ-ನಲ್ಲ
ಆದ-ವ-ನಿಗೆ
ಆದ-ವನು
ಆದ-ವರ
ಆದ-ವ-ರಿಲ್ಲ
ಆದಾಗ
ಆದಾಯ
ಆದಿ
ಆದಿ-ಅಂ-ತ್ಯ-ವಿ-ಲ್ಲದ
ಆದಿ-ಕಾ-ರಣ
ಆದಿ-ಕಾರಣ-ನಾದ
ಆದಿಗೂ
ಆದಿಗೆ
ಆದಿ-ತ್ಯ-ನಂತೆ
ಆದಿ-ತ್ಯ-ರನ್ನು
ಆದಿ-ತ್ಯ-ರಲ್ಲಿ
ಆದಿ-ತ್ಯರು
ಆದಿ-ತ್ಯಾ-ನಾ-ಮಹಂ
ಆದಿ-ದೇವ
ಆದಿ-ದೇ-ವನು
ಆದಿ-ಪು-ರುಷ
ಆದಿ-ಪು-ರು-ಷ-ನನ್ನು
ಆದಿ-ಪು-ರು-ಷ-ನಾದ
ಆದಿ-ಯನ್ನು
ಆದಿ-ಯನ್ನೇ
ಆದಿ-ಯಲ್ಲ
ಆದಿ-ಯಲ್ಲಿ
ಆದಿ-ಯಲ್ಲೆ
ಆದಿ-ಯಲ್ಲೇ
ಆದಿ-ಯಾ-ಗಲೀ
ಆದಿ-ಯಾ-ಗಿ-ರು-ತ್ತೇನೆ
ಆದಿ-ಯಿಂದ
ಆದಿ-ಯಿಲ್ಲ
ಆದಿ-ಯಿ-ಲ್ಲ-ದ-ವ-ನೆಂದು
ಆದಿಯೂ
ಆದಿ-ಶೇಷ
ಆದೀತೆ
ಆದು-ದಕ್ಕೆ
ಆದು-ದ-ರಿಂದ
ಆದು-ದ-ರಿಂ-ದಲೇ
ಆದುದು
ಆದು-ದೆಲ್ಲ
ಆದುವು
ಆದು-ವು-ಗಳು
ಆದೆವು
ಆದೊ-ಡನೆ
ಆದೊ-ಡ-ನೆಯೆ
ಆದ್ಯಂ-ತ-ವಂತಃ
ಆದ್ಯವೂ
ಆಧತ್ಸ್ವ
ಆಧಾರ
ಆಧಾ-ರದ
ಆಧಾ-ರ-ದ-ಮೇಲೆ
ಆಧಾ-ರ-ದಿಂದ
ಆಧಾ-ರ-ದಿಂ-ದಲೇ
ಆಧಾ-ರ-ವನ್ನು
ಆಧಾ-ರ-ವಾ-ಗಿ-ಟ್ಟು-ಕೊಂಡು
ಆಧಾ-ರ-ವಾ-ಗಿ-ಟ್ಟು-ಕೊ-ಳ್ಳ-ಬೇ-ಕಾ-ಗಿಲ್ಲ
ಆಧಾ-ರ-ವಾ-ಗಿ-ದ್ದೇನೆ
ಆಧಾ-ರ-ವಾ-ಗಿ-ರು-ವನು
ಆಧಾ-ರ-ವಾ-ಗಿ-ರು-ವ-ವನು
ಆಧಾ-ರ-ವಾ-ಗಿ-ರು-ವುದು
ಆಧಾ-ರ-ವಾದ
ಆಧಾ-ರ-ವಿಲ್ಲ
ಆಧಾ-ರವೂ
ಆಧಾ-ರ-ಸ್ತಂಭ
ಆಧಾ-ರ-ಸ್ತಂ-ಭ-ವಿದೆ
ಆಧಾ-ರ-ಸ್ತಂ-ಭವೇ
ಆಧು-ನಿಕ
ಆಧ್ಯಾತ್ಮ
ಆಧ್ಯಾ-ತ್ಮ-ನಿ-ತ್ಯರೂ
ಆಧ್ಯಾ-ತ್ಮ-ವಾ-ದರೋ
ಆಧ್ಯಾ-ತ್ಮ-ವೆಂಬ
ಆಧ್ಯಾ-ತ್ಮಿಕ
ಆಧ್ಯಾ-ತ್ಮಿ-ಕಅ
ಆಧ್ಯಾ-ತ್ಮಿ-ಕ-ತ-ತ್ತ್ವ-ಗಳು
ಆನಂದ
ಆನಂ-ದ-ಕ್ಕಾಗಿ
ಆನಂ-ದ-ಕ್ಕಾ-ಗಿಯೇ
ಆನಂ-ದ-ಕ್ಕಿಂತ
ಆನಂ-ದಕ್ಕೂ
ಆನಂ-ದಕ್ಕೆ
ಆನಂ-ದ-ಗಳನ್ನು
ಆನಂ-ದ-ಗಳಲ್ಲಿ
ಆನಂ-ದ-ಗಳು
ಆನಂ-ದದ
ಆನಂ-ದ-ದಲ್ಲಿ
ಆನಂ-ದ-ದ-ಲ್ಲಿದೆ
ಆನಂ-ದ-ದ-ಲ್ಲಿಯೂ
ಆನಂ-ದ-ದಿಂದ
ಆನಂ-ದ-ಪ-ಟ್ಟೆವು
ಆನಂ-ದ-ಪಡ
ಆನಂ-ದ-ಪ-ಡು-ತ್ತಿ-ರು-ವನು
ಆನಂ-ದ-ಪ-ಡು-ವನು
ಆನಂ-ದ-ಪ-ಡು-ವ-ವರ
ಆನಂ-ದ-ಪ-ಡು-ವ-ವ-ರನ್ನು
ಆನಂ-ದ-ಪ-ಡು-ವ-ವರು
ಆನಂ-ದ-ಪ-ಡು-ವುದು
ಆನಂ-ದ-ಭ-ರಿ-ತ-ನಾ-ಗು-ವನು
ಆನಂ-ದ-ಭ-ರಿ-ತ-ನಾದ
ಆನಂ-ದ-ವನ್ನು
ಆನಂ-ದ-ವಲ್ಲ
ಆನಂ-ದ-ವಾ-ಗು-ವುದು
ಆನಂ-ದ-ವಾ-ದರೊ
ಆನಂ-ದ-ವಿದೆ
ಆನಂ-ದ-ವಿ-ದೆಯೋ
ಆನಂ-ದ-ವಿ-ರು-ವುದು
ಆನಂ-ದ-ವಿಲ್ಲ
ಆನಂ-ದವೂ
ಆನಂ-ದವೇ
ಆನಂ-ದ-ವೇನೂ
ಆನಂ-ದ-ಸಾ-ಗ-ರ-ದಿಂದ
ಆನಂ-ದಿ-ಸ-ಬ-ಹುದು
ಆನಂ-ದಿ-ಸ-ಬೇಕು
ಆನಂ-ದಿಸು
ಆನಂ-ದಿ-ಸು-ತ್ತಾ-ನೆಯೋ
ಆನಂ-ದಿ-ಸು-ವ-ನು-ಅಯ್ಯೊ
ಆನಂ-ದಿ-ಸು-ವು-ದಕ್ಕೆ
ಆನಂ-ದಿ-ಸು-ವುದನ್ನು
ಆನಂ-ದಿ-ಸು-ವುದು
ಆನಂ-ದಿ-ಸು-ವುದೂ
ಆನು-ಭ-ವಿ-ಸಿ-ರು-ವನು
ಆನೆ
ಆನೆ-ಗಳ
ಆನೆ-ಗಳನ್ನು
ಆನೆ-ಗಳಲ್ಲಿ
ಆನೆ-ಗಳು
ಆನೆಯ
ಆನೆ-ಯನ್ನು
ಆನೆ-ಯಲ್ಲಿ
ಆನೆ-ಯಾ-ದರೊ
ಆನೆ-ಯೊಂದು
ಆಪತ್ತು
ಆಪ-ದ್ಧನ
ಆಪ-ರೇ-ಷನ್
ಆಪಾದ
ಆಪಾ-ದ-ಮ-ಸ್ತಕ
ಆಪಾ-ದ-ಮ-ಸ್ತ-ಕ-ವಾಗಿ
ಆಪೂ-ರ್ಯ-ಮಾ-ಣ-ಮ-ಚ-ಲ-ಪ್ರ-ತಿಷ್ಠಂ
ಆಪೋ-ಶನ
ಆಪ್ತ
ಆಪ್ತ-ನಂತೆ
ಆಪ್ತ-ವಾಕ್ಯ
ಆಪ್ತ-ವಾ-ಕ್ಯ-ವಾ-ಗು-ವುದು
ಆಪ್ತ-ವಾ-ಗಿ-ರು-ವುದು
ಆಫೀ-ಸಿ-ಗಿಂತ
ಆಫೀ-ಸಿಗೆ
ಆಫೀ-ಸಿನ
ಆಫೀ-ಸಿ-ನಲ್ಲಿ
ಆಫೀ-ಸಿ-ನಿಂದ
ಆಬಾಲ
ಆಬ್ರ-ಹ್ಮ-ಭು-ವ-ನಾ-ಲ್ಲೋ-ಕಾಃ
ಆಭ-ರ-ಣ-ಗಳಿಂದ
ಆಭ-ರ-ಣ-ದಂತೆ
ಆಭಾ-ಸ-ವಾ-ಗು-ವುದು
ಆಮ-ಶಂಕೆ
ಆಮೂ-ಲಾ-ಗ್ರ-ವಾಗಿ
ಆಮೆ
ಆಮೆಯ
ಆಮೇಲೆ
ಆಯ
ಆಯಾ
ಆಯಾ-ಕಾ-ಲಕ್ಕೆ
ಆಯಾ-ಸ-ದಿಂದ
ಆಯಾ-ಸ-ವನ್ನು
ಆಯಾ-ಸ-ವಿಲ್ಲ
ಆಯಾ-ಸ-ವಿ-ಲ್ಲದೆ
ಆಯಾ-ಸವೂ
ಆಯಿ-ತಲ್ಲ
ಆಯಿತು
ಆಯಿತೆ
ಆಯುಧ
ಆಯು-ಧ-ಗಳನ್ನು
ಆಯು-ಧ-ಗಳಲ್ಲಿ
ಆಯು-ಧ-ಗ-ಳಿಂ-ದಲೂ
ಆಯು-ಧ-ಗಳು
ಆಯು-ಧ-ದಿಂದ
ಆಯು-ಧಾ-ನಾ-ಮಹಂ
ಆಯು-ವು-ದಕ್ಕೆ
ಆಯು-ಸ್ಸತ್ತ್ವ-ಬ-ಲಾ-ರೋ-ಗ್ಯ-ಸು-ಖ-ಪ್ರೀ-ತಿ-ವಿ-ವ-ರ್ಧ-ನಾಃ
ಆಯು-ಸ್ಸನ್ನು
ಆಯು-ಸ್ಸಿನ
ಆಯು-ಸ್ಸಿ-ನೊಂ-ದಿಗೆ
ಆಯು-ಸ್ಸಿ-ನೊ-ಡನೆ
ಆಯುಸ್ಸು
ಆಯ್ದ
ಆಯ್ದು
ಆಯ್ದು-ಕೊಂ-ಡಂತೆ
ಆಯ್ದು-ಕೊಂಡು
ಆರ
ಆರಂಭ
ಆರಂ-ಭ-ವಾ-ಗು-ವುದು
ಆರಂ-ಭ-ವಾ-ಯಿತೋ
ಆರಂ-ಭಿ-ಸದೇ
ಆರದ
ಆರದು
ಆರನೆ
ಆರ-ನೆ-ಯ-ದನ್ನು
ಆರ-ನೆ-ಯ-ದಾದ
ಆರರು
ಆರವು
ಆರಾ-ಧನೆ
ಆರಾ-ಧ-ನೆಯ
ಆರಾ-ಧಿ-ಸ-ಬ-ಲ್ಲರು
ಆರಾ-ಧಿ-ಸಲಿ
ಆರಾ-ಧಿಸಿ
ಆರಾ-ಧಿಸು
ಆರಾ-ಧಿ-ಸು-ತ್ತಾರೆ
ಆರಾ-ಧಿ-ಸು-ವ-ವ-ನಾಗು
ಆರಾ-ಧಿ-ಸು-ವ-ವರು
ಆರಾ-ಧ್ಯ-ಮೂ-ರ್ತಿ-ಯಾದ
ಆರಾ-ಧ್ಯ-ಮೂ-ರ್ತಿಯೇ
ಆರಾಮ
ಆರಾ-ಮ-ವಾಗಿ
ಆರಿ
ಆರಿದ
ಆರಿ-ರು-ವುದು
ಆರಿ-ಸ-ಲೆ-ತ್ನಿ-ಸಿ-ದಂತೆ
ಆರಿಸಿ
ಆರಿ-ಸಿಕೊ
ಆರಿ-ಸಿ-ಕೊಂ-ಡಾಗ
ಆರಿ-ಸಿ-ಕೊಂ-ಡಿರು
ಆರಿ-ಸಿ-ಕೊಂ-ಡಿ-ರುವ
ಆರಿ-ಸಿ-ಕೊಂ-ಡಿ-ರು-ವರು
ಆರಿ-ಸಿ-ಕೊಂಡು
ಆರಿ-ಸಿ-ಕೊ-ಡ-ಲಾ-ಗು-ವು-ದಿಲ್ಲ
ಆರಿ-ಸಿ-ಕೊ-ಳ್ಳ-ಬ-ಹುದು
ಆರಿ-ಸಿ-ಕೊ-ಳ್ಳ-ಬೇ-ಕಾ-ಗು-ವುದು
ಆರಿ-ಸಿ-ಕೊ-ಳ್ಳ-ಬೇ-ಕಾ-ದರೆ
ಆರಿ-ಸಿ-ಕೊ-ಳ್ಳ-ಬೇಕು
ಆರಿ-ಸಿ-ಕೊ-ಳ್ಳ-ಲಾ-ಗು-ವು-ದಿಲ್ಲ
ಆರಿ-ಸಿ-ಕೊ-ಳ್ಳ-ವು-ದಿಲ್ಲ
ಆರಿ-ಸಿ-ಕೊ-ಳ್ಳು-ತ್ತಾನೆ
ಆರಿ-ಸಿ-ಕೊ-ಳ್ಳು-ತ್ತಿದ್ದ
ಆರಿ-ಸಿ-ಕೊ-ಳ್ಳು-ತ್ತಿ-ದ್ದರು
ಆರಿ-ಸಿ-ಕೊ-ಳ್ಳು-ತ್ತೇವೆ
ಆರಿ-ಸಿ-ಕೊ-ಳ್ಳು-ತ್ತೇ-ವೆಯೊ
ಆರಿ-ಸಿ-ಕೊ-ಳ್ಳು-ವನು
ಆರಿ-ಸಿ-ಕೊ-ಳ್ಳು-ವನೊ
ಆರಿ-ಸಿ-ಕೊ-ಳ್ಳು-ವರು
ಆರಿ-ಸಿ-ಕೊ-ಳ್ಳು-ವು-ದಿಲ್ಲ
ಆರಿ-ಸಿ-ಕೊ-ಳ್ಳು-ವುದು
ಆರಿ-ಸಿ-ಕೊ-ಳ್ಳು-ವೆವೋ
ಆರಿ-ಸಿ-ದೊ-ಡನೆ
ಆರಿ-ಸುವ
ಆರಿ-ಸು-ವ-ವರು
ಆರಿ-ಹೋ-ಗ-ಬ-ಹುದು
ಆರಿ-ಹೋ-ಗಿದೆ
ಆರಿ-ಹೋ-ಗುವ
ಆರಿ-ಹೋ-ಗು-ವು-ದಕ್ಕೆ
ಆರಿ-ಹೋ-ದರೆ
ಆರು
ಆರು-ರು-ಕ್ಷೋ-ರ್ಮು-ನೇ-ರ್ಯೋಗಂ
ಆರೂಢ
ಆರೂ-ಢ-ನಾ-ಗಿ-ರು-ವನು
ಆರೂ-ಢ-ನಾದ
ಆರೆವು
ಆರೈಕೆ
ಆರೈ-ಕೆ-ಮಾ-ಡುತ್ತ
ಆರೋಗ್ಯ
ಆರೋ-ಗ್ಯದ
ಆರೋ-ಗ್ಯ-ದ-ಲ್ಲಿ-ರು-ವುದು
ಆರೋ-ಗ್ಯ-ದಿಂದ
ಆರೋ-ಗ್ಯ-ವನ್ನು
ಆರೋ-ಗ್ಯ-ವಾಗಿ
ಆರೋ-ಗ್ಯ-ವಾ-ಗಿ-ರ-ಬ-ಹುದು
ಆರೋ-ಗ್ಯ-ವಾ-ಗಿ-ರ-ಬೇ-ಕಾ-ದರೆ
ಆರೋ-ಗ್ಯ-ವಾ-ಗಿ-ರು-ವುದು
ಆರೋ-ಗ್ಯವೋ
ಆರೋಪ
ಆರೋ-ಪ-ಣ-ಮಾ-ಡಿ-ಕೊಂಡು
ಆರೋ-ಪದ
ಆರೋ-ಪ-ಮಾಡಿ
ಆರೋ-ಪ-ಮಾ-ಡಿ-ಕೊಂಡು
ಆರೋ-ಪ-ಮಾ-ಡಿದ
ಆರೋ-ಪ-ಮಾ-ಡಿ-ರುವು
ಆರೋ-ಪ-ಮಾ-ಡಿ-ರು-ವೆವೊ
ಆರೋ-ಪ-ಮಾ-ಡು-ವನು
ಆರೋ-ಪ-ವನ್ನು
ಆರೋ-ಪ-ವಾ-ಗಿವೆ
ಆರೋ-ಪಿತ
ಆರೋ-ಪಿ-ತ-ವಾ-ಗಿದೆ
ಆರೋ-ಪಿ-ತ-ವಾ-ಗಿ-ರುವ
ಆರೋ-ಪಿ-ತ-ವಾದ
ಆರೋ-ಪಿ-ಸು-ವರು
ಆರ್ಕಿ-ಮಿ-ಡೀಸ್
ಆರ್ಜವ
ಆರ್ಜ-ವದ
ಆರ್ಜ-ವಮ್
ಆರ್ಜಿ-ಸ-ಬೇ-ಕೆಂಬ
ಆರ್ಜಿ-ಸು-ವುದು
ಆರ್ತ
ಆರ್ತ-ನಾದ
ಆರ್ತೋ
ಆರ್ಥಿಕ
ಆರ್ಯ
ಆರ್ಯ-ರಿಗೆ
ಆಲ-ದ-ಮರ
ಆಲ-ದ-ಮ-ರ-ದಂತೆ
ಆಲಯ
ಆಲಸಿ
ಆಲ-ಸಿ-ಯಾ-ಗು-ವುದು
ಆಲಸ್ಯ
ಆಲ-ಸ್ಯಕ್ಕೆ
ಆಲ-ಸ್ಯ-ದಿಂ-ದಲೊ
ಆಲೋ-ಚನಾ
ಆಲೋ-ಚನೆ
ಆಲೋ-ಚ-ನೆ-ಗಳ
ಆಲೋ-ಚ-ನೆ-ಗಳನ್ನು
ಆಲೋ-ಚ-ನೆ-ಗಳಲ್ಲಿ
ಆಲೋ-ಚ-ನೆ-ಗ-ಳಾ-ಗು-ವುವು
ಆಲೋ-ಚ-ನೆ-ಗಳಿಂದ
ಆಲೋ-ಚ-ನೆ-ಗಳು
ಆಲೋ-ಚ-ನೆ-ಗ-ಳೆಲ್ಲ
ಆಲೋ-ಚ-ನೆ-ಗಳೇ
ಆಲೋ-ಚ-ನೆಗೂ
ಆಲೋ-ಚ-ನೆ-ಮಾ-ಡು-ವನು
ಆಲೋ-ಚ-ನೆಯ
ಆಲೋ-ಚ-ನೆ-ಯನ್ನು
ಆಲೋ-ಚ-ನೆ-ಯನ್ನೂ
ಆಲೋ-ಚ-ನೆ-ಯಲ್ಲಿ
ಆಲೋ-ಚ-ನೆ-ಯಾಗಿ
ಆಲೋ-ಚ-ನೆಯೂ
ಆಲೋ-ಚ-ನೆ-ಯೊಂದೇ
ಆಲೋಚಿ
ಆಲೋ-ಚಿಸ
ಆಲೋ-ಚಿ-ಸದೆ
ಆಲೋ-ಚಿ-ಸ-ಬ-ಹುದು
ಆಲೋ-ಚಿ-ಸಿ-ಕೊ-ಳ್ಳು-ವುದು
ಆಲೋ-ಚಿ-ಸಿ-ರು-ವನು
ಆಲೋ-ಚಿ-ಸು-ತ್ತಾನೆ
ಆಲೋ-ಚಿ-ಸು-ತ್ತಿದ್ದ
ಆಲೋ-ಚಿ-ಸು-ತ್ತಿ-ದ್ದರೆ
ಆಲೋ-ಚಿ-ಸು-ತ್ತೇ-ವೆಯೋ
ಆಲೋ-ಚಿ-ಸು-ವನು
ಆಲೋ-ಚಿ-ಸು-ವ-ವನು
ಆಲೋ-ಚಿ-ಸು-ವು-ದಕ್ಕೆ
ಆಲೋ-ಚಿ-ಸು-ವು-ದ-ರಿಂ-ದಲೇ
ಆಲೋ-ಚಿ-ಸು-ವು-ದಿಲ್ಲ
ಆಲೋ-ಚಿ-ಸು-ವುದು
ಆಲ್ಮೈರಾ
ಆಳ
ಆಳಕ್ಕೆ
ಆಳ-ದಲ್ಲಿ
ಆಳ-ದ-ವ-ರೆಗೂ
ಆಳ-ದ-ವ-ರೆಗೆ
ಆಳ-ದಿಂದ
ಆಳ-ಬ-ಹುದು
ಆಳ-ಬೇಕು
ಆಳ-ವನ್ನು
ಆಳ-ವಾ-ಗ-ಬೇಕು
ಆಳ-ವಾಗಿ
ಆಳ-ವಾ-ಗಿದೆ
ಆಳ-ವಾ-ಗಿ-ರು-ವುದು
ಆಳ-ವಾದ
ಆಳ-ವಾ-ದುದು
ಆಳಾಗಿ
ಆಳಾ-ಗಿ-ರಲಿ
ಆಳಾ-ಗುವ
ಆಳಾ-ಗು-ವುದು
ಆಳಿಗೆ
ಆಳಿನ
ಆಳಿ-ನಂತೆ
ಆಳು
ಆಳು-ಗ-ಳಲ್ಲ
ಆಳು-ಗ-ಳಿಗೆ
ಆಳು-ಗ-ಳೇನು
ಆಳು-ತ್ತಾನೆ
ಆಳು-ತ್ತಿದೆ
ಆಳು-ತ್ತಿ-ದ್ದ-ವರು
ಆಳು-ತ್ತಿ-ರ-ವ-ವನು
ಆಳು-ತ್ತಿ-ರುವ
ಆಳು-ತ್ತಿ-ರು-ವಂತೆ
ಆಳು-ತ್ತಿ-ರು-ವನು
ಆಳುವ
ಆಳು-ವನು
ಆಳು-ವ-ವನ
ಆಳು-ವ-ವನು
ಆಳು-ವ-ವರು
ಆಳು-ವು-ದಕ್ಕೆ
ಆವ-ಕಾಶ
ಆವ-ತಾ-ರ-ವನ್ನು
ಆವನು
ಆವರ
ಆವ-ರಣ
ಆವ-ರ-ಣಕ್ಕೆ
ಆವ-ರ-ಣ-ಗ-ಳೆಲ್ಲಾ
ಆವ-ರ-ಣ-ದಂತೆ
ಆವ-ರ-ಣ-ದಲ್ಲಿ
ಆವ-ರ-ಣ-ದೊ-ಳಗೆ
ಆವ-ರಿ-ಸ-ಲ್ಪ-ಟ್ಟ-ವ-ರಾಗಿ
ಆವ-ರಿ-ಸ-ಲ್ಪ-ಟ್ಟಿ-ರು-ವರು
ಆವ-ರಿಸಿ
ಆವ-ರಿ-ಸಿ-ಕೊಂ-ಡಾಗ
ಆವ-ರಿ-ಸಿ-ಕೊಂ-ಡಿದೆ
ಆವ-ರಿ-ಸಿ-ಕೊಂ-ಡಿ-ದೆಯೊ
ಆವ-ರಿ-ಸಿ-ಕೊಂ-ಡಿ-ರು-ವಾಗ
ಆವ-ರಿ-ಸಿ-ಕೊಂ-ಡಿ-ರು-ವುದೊ
ಆವ-ರಿ-ಸಿ-ಕೊಂ-ಡಿವೆ
ಆವ-ರಿ-ಸಿ-ಕೊಂಡು
ಆವ-ರಿ-ಸಿ-ಕೊ-ಳ್ಳು-ವುದು
ಆವ-ರಿ-ಸಿ-ದರೆ
ಆವ-ರಿ-ಸಿ-ದಾಗ
ಆವ-ರಿ-ಸಿದೆ
ಆವ-ರಿ-ಸಿದ್ದ
ಆವ-ರಿ-ಸಿರು
ಆವ-ರಿ-ಸಿ-ರುವ
ಆವ-ರಿ-ಸಿ-ರು-ವುದು
ಆವ-ರಿ-ಸು-ವು-ದಿಲ್ಲ
ಆವ-ರಿ-ಸು-ವುದು
ಆವ-ರಿ-ಸು-ವುದೊ
ಆವ-ರಿ-ಸು-ವುವು
ಆವ-ಶ್ಯಕ
ಆವ-ಶ್ಯ-ಕತೆ
ಆವ-ಶ್ಯ-ಕ-ತೆ-ಯಾ-ದರೂ
ಆವ-ಶ್ಯ-ಕ-ತೆ-ಯಿ-ರು-ವುದು
ಆವ-ಶ್ಯ-ಕ-ತೆ-ಯಿಲ್ಲ
ಆವ-ಶ್ಯ-ಕ-ವಾಗಿ
ಆವ-ಶ್ಯ-ಕ-ವಾ-ಗಿತ್ತೋ
ಆವ-ಶ್ಯ-ಕ-ವಾ-ಗಿ-ದೆಯೋ
ಆವ-ಶ್ಯ-ಕ-ವಾ-ಗಿ-ರು-ವುದನ್ನು
ಆವ-ಶ್ಯ-ಕ-ವಾ-ಗಿ-ರು-ವುದು
ಆವ-ಶ್ಯ-ಕ-ವಾ-ಗಿ-ರು-ವುದೇ
ಆವ-ಶ್ಯ-ಕ-ವಾ-ಗಿ-ರು-ವುದೊ
ಆವ-ಶ್ಯ-ಕ-ವಾದ
ಆವ-ಶ್ಯ-ಕ-ವಾ-ದು-ದನ್ನು
ಆವ-ಶ್ಯ-ಕ-ವಿದೆ
ಆವ-ಶ್ಯ-ಕ-ವಿಲ್ಲ
ಆವ-ಶ್ಯ-ಕವೊ
ಆವ-ಶ್ಯ-ಕವೋ
ಆವಾಗ
ಆವಿ
ಆವಿಗೆ
ಆವಿ-ಗೆಯ
ಆವಿ-ಗೆ-ಯಲ್ಲಿ
ಆವಿ-ಯಂತೆ
ಆವಿ-ಯಾಗಿ
ಆವಿ-ಯಾ-ದರೆ
ಆವಿ-ಯಾ-ದಾಗ
ಆವಿ-ರ್ಭಾವ
ಆವಿ-ರ್ಭಾ-ವದ
ಆವಿ-ರ್ಭಾ-ವನೆ
ಆವಿ-ರ್ಭಾ-ವ-ನೆ-ಯನ್ನೇ
ಆವಿ-ರ್ಭಾ-ವವೇ
ಆವೃತಂ
ಆವೃ-ತ-ನಾಗಿ
ಆವೃ-ತ-ನಾ-ಗಿ-ರು-ವಂತೆ
ಆವೃ-ತ-ನಾದ
ಆವೃ-ತ-ರಾಗಿ
ಆವೃ-ತ-ರಾ-ಗಿ-ರು-ವೆವು
ಆವೃ-ತ-ವಾಗಿ
ಆಶಾ-ಜ-ನ-ಕ-ವಾ-ಗಿದೆ
ಆಶಾ-ಪಾ-ಶ-ಶ-ತೈ-ರ್ಬ-ದ್ಧಾಃ
ಆಶಾ-ರ-ಹಿ-ತ-ನಾ-ಗಿ-ರ-ಬೇಕು
ಆಶಿ-ಸನು
ಆಶಿ-ಸ-ಬಾ-ರದು
ಆಶಿ-ಸ-ಬೇ-ಕಾ-ದರೆ
ಆಶಿ-ಸ-ಲಿಲ್ಲ
ಆಶಿ-ಸಿ-ದರೆ
ಆಶಿ-ಸು-ತ್ತಾನೆ
ಆಶಿ-ಸು-ತ್ತಾರೆ
ಆಶಿ-ಸು-ತ್ತಿ-ದ್ದರೆ
ಆಶಿ-ಸು-ತ್ತಿ-ರ-ಬೇಕು
ಆಶಿ-ಸು-ತ್ತೇನೆ
ಆಶಿ-ಸು-ತ್ತೇವೆ
ಆಶಿ-ಸುವ
ಆಶಿ-ಸು-ವನು
ಆಶಿ-ಸು-ವರು
ಆಶಿ-ಸು-ವರೊ
ಆಶಿ-ಸು-ವ-ವನು
ಆಶಿ-ಸು-ವು-ದಿಲ್ಲ
ಆಶಿ-ಸು-ವುದು
ಆಶೀ-ರ್ವ-ದಿಸು
ಆಶೀ-ರ್ವಾದ
ಆಶೆ
ಆಶೆಯೂ
ಆಶ್ಚರ್ಯ
ಆಶ್ಚ-ರ್ಯ-ಗಳನ್ನು
ಆಶ್ಚ-ರ್ಯ-ಗಳೂ
ಆಶ್ಚ-ರ್ಯ-ಚ-ಕಿತ
ಆಶ್ಚ-ರ್ಯ-ಚ-ಕಿ-ತ-ನಾಗಿ
ಆಶ್ಚ-ರ್ಯ-ಚ-ಕಿ-ತ-ರಾಗಿ
ಆಶ್ಚ-ರ್ಯ-ಚ-ಕಿ-ತ-ರಾ-ಗು-ತ್ತೇವೆ
ಆಶ್ಚ-ರ್ಯ-ದಂತೆ
ಆಶ್ಚ-ರ್ಯ-ದಿಂದ
ಆಶ್ಚ-ರ್ಯ-ಪ-ಟ್ಟರು
ಆಶ್ಚ-ರ್ಯ-ಪ-ಡ-ಬ-ಹುದು
ಆಶ್ಚ-ರ್ಯ-ಪ-ಡು-ತ್ತಾ-ನೆಯೋ
ಆಶ್ಚ-ರ್ಯ-ಪ-ಡು-ತ್ತೇವೆ
ಆಶ್ಚ-ರ್ಯ-ಪ-ಡು-ವನು
ಆಶ್ಚ-ರ್ಯ-ಪ-ಡು-ವರು
ಆಶ್ಚ-ರ್ಯ-ಪ-ಡು-ವೆವು
ಆಶ್ಚ-ರ್ಯ-ಭ-ರಿತ
ಆಶ್ಚ-ರ್ಯ-ಭ-ರಿ-ತನೂ
ಆಶ್ಚ-ರ್ಯ-ವ-ಚ್ಚೈ-ನ-ಮನ್ಯಃ
ಆಶ್ಚ-ರ್ಯ-ವತ್
ಆಶ್ಚ-ರ್ಯ-ವನ್ನು
ಆಶ್ಚ-ರ್ಯ-ವಾಗ
ಆಶ್ಚ-ರ್ಯ-ವಾ-ಗ-ಬ-ಹುದು
ಆಶ್ಚ-ರ್ಯ-ವಾ-ಗು-ತ್ತಿದೆ
ಆಶ್ಚ-ರ್ಯ-ವಿಲ್ಲ
ಆಶ್ಚ-ರ್ಯ-ವೇ-ನಿದೆ
ಆಶ್ಚ-ರ್ಯ-ವೇ-ನಿಲ್ಲ
ಆಶ್ಟ-ರ್ಯ-ಪ-ರ-ವ-ಶ-ನಾ-ದುದು
ಆಶ್ರಮ
ಆಶ್ರ-ಮ-ಕ್ಕಾ-ಗಲಿ
ಆಶ್ರ-ಮಕ್ಕೆ
ಆಶ್ರ-ಮಕ್ಕೊ
ಆಶ್ರ-ಮ-ದ-ಲ್ಲಾ-ದರೂ
ಆಶ್ರ-ಮ-ದ-ವರು
ಆಶ್ರಯ
ಆಶ್ರ-ಯ-ಗಳನ್ನೆಲ್ಲ
ಆಶ್ರ-ಯ-ಗ-ಳೆಲ್ಲ
ಆಶ್ರ-ಯದ
ಆಶ್ರ-ಯ-ನಾ-ಗಿ-ದ್ದೇನೆ
ಆಶ್ರ-ಯ-ನಾ-ಗು-ವನು
ಆಶ್ರ-ಯ-ವನ್ನು
ಆಶ್ರ-ಯ-ವನ್ನೇ
ಆಶ್ರ-ಯ-ವಾ-ದಂ-ತಾ-ಗು-ವುದು
ಆಶ್ರ-ಯ-ವಿ-ದ್ದರೆ
ಆಶ್ರ-ಯ-ವಿ-ಲ್ಲದೆ
ಆಶ್ರ-ಯವೇ
ಆಶ್ರ-ಯ-ಸ್ಥಾನ
ಆಶ್ರ-ಯ-ಸ್ಥಾ-ನ-ಗಳು
ಆಶ್ರ-ಯ-ಹೀ-ನನೂ
ಆಶ್ರ-ಯಿ-ಸ-ಬಾ-ರದು
ಆಶ್ರ-ಯಿಸಿ
ಆಶ್ರ-ಯಿ-ಸಿ-ಕೊಂಡು
ಆಶ್ರ-ಯಿ-ಸಿದ
ಆಶ್ರ-ಯಿ-ಸಿ-ದ-ವರು
ಆಶ್ರ-ಯಿ-ಸಿದೆ
ಆಶ್ರ-ಯಿ-ಸಿ-ರು-ವನು
ಆಶ್ರ-ಯಿ-ಸಿ-ರು-ವರು
ಆಶ್ರ-ಯಿ-ಸಿ-ರು-ವ-ವ-ನಿಗೆ
ಆಶ್ರ-ಯಿ-ಸು-ತ್ತಾರೆ
ಆಶ್ರ-ಯಿ-ಸು-ತ್ತಾ-ರೆಯೋ
ಆಶ್ರ-ಯಿ-ಸುವ
ಆಶ್ರ-ಯಿ-ಸು-ವನು
ಆಶ್ರ-ಯಿ-ಸು-ವನೋ
ಆಶ್ರ-ಯಿ-ಸು-ವರು
ಆಶ್ರ-ಯಿ-ಸು-ವ-ವನು
ಆಶ್ರ-ಯಿ-ಸು-ವ-ವರು
ಆಶ್ರ-ಯಿ-ಸು-ವು-ದಿಲ್ಲ
ಆಶ್ರ-ಯಿ-ಸು-ವುದು
ಆಶ್ರ-ಯಿ-ಸು-ವುದೇ
ಆಶ್ವಾ-ಸ-ಯಾ-ಮಾಸ
ಆಸ-ಕ್ತ-ನಲ್ಲ
ಆಸ-ಕ್ತ-ನಾ-ಗ-ಬಾ-ರದು
ಆಸ-ಕ್ತ-ನಾ-ಗ-ಬೇಡ
ಆಸ-ಕ್ತ-ನಾಗಿ
ಆಸ-ಕ್ತ-ನಾ-ಗಿ-ರ-ಬೇಕು
ಆಸ-ಕ್ತ-ನಾ-ಗಿ-ರ-ಲಿಲ್ಲ
ಆಸ-ಕ್ತ-ನಾ-ಗಿ-ರು-ವು-ದಾ-ಗಲೀ
ಆಸ-ಕ್ತ-ನಾ-ಗಿ-ರು-ವು-ದಿಲ್ಲ
ಆಸ-ಕ್ತ-ನಾ-ಗಿ-ರು-ವು-ದಿ-ಲ್ಲವೊ
ಆಸ-ಕ್ತ-ನಾ-ಗು-ವು-ದಿಲ್ಲ
ಆಸ-ಕ್ತ-ನಾ-ದರೆ
ಆಸ-ಕ್ತನೂ
ಆಸ-ಕ್ತ-ರಲ್ಲ
ಆಸ-ಕ್ತ-ರಾ-ಗ-ಬಾ-ರದು
ಆಸ-ಕ್ತ-ರಾಗಿ
ಆಸ-ಕ್ತ-ರಾ-ಗಿ-ದ್ದರೆ
ಆಸ-ಕ್ತ-ರಾ-ಗಿ-ರ-ಬಾ-ರದು
ಆಸ-ಕ್ತ-ರಾ-ಗಿ-ರುವ
ಆಸ-ಕ್ತ-ರಾ-ಗಿ-ರು-ವುದ
ಆಸ-ಕ್ತ-ರಾ-ಗಿ-ರು-ವು-ದ-ಕ್ಕಾ-ಗು-ವು-ದಿಲ್ಲ
ಆಸ-ಕ್ತ-ರಾ-ಗಿ-ರು-ವುದು
ಆಸ-ಕ್ತ-ರಾ-ಗಿ-ರು-ವೆವೊ
ಆಸ-ಕ್ತ-ರಾಗು
ಆಸ-ಕ್ತ-ರಾ-ಗುತ್ತ
ಆಸ-ಕ್ತ-ರಾ-ಗು-ತ್ತಾರೆ
ಆಸ-ಕ್ತ-ರಾದ
ಆಸ-ಕ್ತ-ರಾ-ದ-ವ-ರಿಗೆ
ಆಸ-ಕ್ತ-ರಾ-ದಾಗ
ಆಸ-ಕ್ತರು
ಆಸ-ಕ್ತ-ವಾ-ಗಿದೆ
ಆಸ-ಕ್ತ-ವಾ-ಗಿ-ರು-ವುದೋ
ಆಸ-ಕ್ತ-ವಾ-ಗಿಲ್ಲ
ಆಸಕ್ತಿ
ಆಸ-ಕ್ತಿಗೆ
ಆಸ-ಕ್ತಿಯ
ಆಸ-ಕ್ತಿ-ಯ-ನ್ನಿ-ಡದೆ
ಆಸ-ಕ್ತಿ-ಯನ್ನು
ಆಸ-ಕ್ತಿ-ಯಿಂದ
ಆಸ-ಕ್ತಿ-ಯಿಲ್ಲ
ಆಸ-ಕ್ತಿ-ಯಿ-ಲ್ಲದೆ
ಆಸ-ಕ್ತಿಯೂ
ಆಸ-ಕ್ತಿ-ಯೆಲ್ಲ
ಆಸ-ಕ್ತಿಯೇ
ಆಸನ
ಆಸ-ನದ
ಆಸ-ನ-ದಲ್ಲಿ
ಆಸ-ನ-ವನ್ನು
ಆಸ-ನವೂ
ಆಸರೆ
ಆಸ-ರೆಯ
ಆಸ-ರೆ-ಯಾಗಿ
ಆಸ-ರೆ-ಯಿಂ-ದಲೇ
ಆಸ-ರೆಯೂ
ಆಸೀತ
ಆಸುರ
ಆಸುರಂ
ಆಸುರೀ
ಆಸು-ರೀಂ
ಆಸು-ರೀ-ಸಂ-ಪ-ತ್ತನ್ನು
ಆಸು-ರೀ-ಸ್ವ-ಭಾವ
ಆಸು-ರೀ-ಸ್ವ-ಭಾ-ವ-ವನ್ನು
ಆಸೆ
ಆಸೆ-ಗಳ
ಆಸೆ-ಗಳನ್ನು
ಆಸೆ-ಗಳಿಂದ
ಆಸೆ-ಗ-ಳಿ-ರ-ಬ-ಹುದು
ಆಸೆ-ಗ-ಳಿಲ್ಲ
ಆಸೆ-ಗ-ಳಿವೆ
ಆಸೆ-ಗಳು
ಆಸೆ-ಗಳೂ
ಆಸೆ-ಗ-ಳೆಲ್ಲ
ಆಸೆ-ಗ-ಳೆಲ್ಲಾ
ಆಸೆ-ಗಳೇ
ಆಸೆಗೆ
ಆಸೆಯ
ಆಸೆ-ಯಂ-ತೆಯೇ
ಆಸೆ-ಯನ್ನು
ಆಸೆ-ಯನ್ನೂ
ಆಸೆ-ಯಲ್ಲ
ಆಸೆ-ಯಲ್ಲೇ
ಆಸೆ-ಯಾ-ಗಲಿ
ಆಸೆ-ಯಿಂದ
ಆಸೆ-ಯಿದೆ
ಆಸೆ-ಯಿ-ದೆಯೊ
ಆಸೆ-ಯಿಲ್ಲ
ಆಸೆ-ಯಿ-ಲ್ಲ-ದ-ವನು
ಆಸೆಯೂ
ಆಸೆಯೆ
ಆಸೆ-ಯೆಂಬ
ಆಸೆ-ಯೆಲ್ಲ
ಆಸೆಯೇ
ಆಸೆ-ಯೇನೋ
ಆಸೆ-ಯೊಂದು
ಆಸೆ-ಯೊಂದೆ
ಆಸೆ-ಯೊಂದೇ
ಆಸೆಯೋ
ಆಸ್ತಿ
ಆಸ್ತಿಕ
ಆಸ್ತಿ-ಯನ್ನು
ಆಸ್ತಿ-ಯನ್ನೂ
ಆಸ್ತಿ-ಯ-ನ್ನೆಲ್ಲ
ಆಸ್ತೇ
ಆಸ್ಥಾ-ನ-ದಲ್ಲಿ
ಆಸ್ಥಾ-ನ-ದಿಂದ
ಆಸ್ಥಾ-ನ-ವನ್ನು
ಆಸ್ಥಿತಃ
ಆಸ್ಪತ್ರೆ
ಆಸ್ಪ-ತ್ರೆ-ಯನ್ನೋ
ಆಸ್ಪ-ತ್ರೆ-ಯಿಂದ
ಆಸ್ಪದ
ಆಸ್ಪ-ದ-ವಾ-ಗಿ-ದೆಯೋ
ಆಸ್ಪ-ದ-ವಿ-ರು-ವು-ದಿಲ್ಲ
ಆಸ್ಪ-ದ-ವಿಲ್ಲ
ಆಸ್ಪ-ದವೇ
ಆಸ್ವಾ-ದ-ನೆಗೆ
ಆಹಾರ
ಆಹಾ-ರಕ್ಕೆ
ಆಹಾ-ರಕ್ಕೇ
ಆಹಾ-ರ-ಗಳನ್ನು
ಆಹಾ-ರ-ಗಳು
ಆಹಾ-ರದ
ಆಹಾ-ರ-ದ-ಲ್ಲಿ-ರುವ
ಆಹಾ-ರ-ದಿಂದ
ಆಹಾ-ರ-ವನ್ನು
ಆಹಾ-ರ-ವ-ನ್ನೆಲ್ಲ
ಆಹಾ-ರ-ವನ್ನೇ
ಆಹಾ-ರ-ವಾ-ಗ-ಲಾ-ರದು
ಆಹಾ-ರ-ವಾಗಿ
ಆಹಾ-ರ-ವಾ-ಗಿ-ರ-ಬೇಕು
ಆಹಾ-ರ-ವಾ-ಗು-ವು-ದಿಲ್ಲ
ಆಹಾ-ರ-ವಾ-ಗು-ವುದು
ಆಹಾ-ರ-ವಾದ
ಆಹಾ-ರ-ವಿ-ರು-ವುದೋ
ಆಹಾ-ರ-ವಿ-ಲ್ಲದೇ
ಆಹಾ-ರವು
ಆಹಾ-ರವೇ
ಆಹಾ-ರ-ಸ್ತ್ವಪಿ
ಆಹಾರಾ
ಆಹಾ-ರಾಃ
ಆಹಾ-ರಾ-ದಿ-ಗಳನ್ನು
ಆಹಾ-ರಾ-ದಿ-ಗಳು
ಆಹುತಿ
ಆಹು-ತಿ-ಗಳನ್ನು
ಆಹು-ತಿ-ಯಂತೆ
ಆಹು-ತಿ-ಯಾಗಿ
ಆಹು-ತಿ-ಯಾ-ಗು-ತ್ತಾನೆ
ಆಹು-ತಿ-ಯಾ-ಗು-ವ-ವರೆ
ಆಹು-ಸ್ತ್ವಾ-ಮೃ-ಷಯಃ
ಆಹೇ-ತುಕ
ಆಹ್ಲಾ-ದ-ಕ-ರ-ವಾಗಿ
ಆಹ್ಲಾ-ದ-ಕ-ರ-ವಾ-ಗಿರು
ಆಹ್ಲಾ-ದ-ಕ-ರ-ವಾದ
ಆಹ್ಲಾ-ದ-ವಾ-ಗು-ವುದು
ಇಂಎಂ-ತಹ
ಇಂಗ-ಬೇ-ಕಾ-ಗಿದೆ
ಇಂಗಿ
ಇಂಗಿ-ಸಲು
ಇಂಗಿಸಿ
ಇಂಗಿ-ಹೋ-ಗಿದೆ
ಇಂಗು
ಇಂಗು-ವ-ವ-ರೆಗೆ
ಇಂಗ್ಲಿ-ಷ್ನಲ್ಲಿ
ಇಂಗ್ಲೀ-ಷಿ-ನಲ್ಲಿ
ಇಂಚು-ಗಳ
ಇಂಜ-ನ್ನಿ-ನಂತೆ
ಇಂಜಿನ್
ಇಂಜಿ-ನ್ನಿ-ನಂತೆ
ಇಂಜೆ-ಕ್ಷನ್
ಇಂಡಿಯ
ಇಂಡಿ-ಯನ್
ಇಂಡಿಯಾ
ಇಂತಹ
ಇಂತ-ಹ-ವ-ನಿಗೆ
ಇಂತ-ಹ-ವನು
ಇಂತ-ಹ-ವರ
ಇಂತ-ಹ-ವ-ರನ್ನು
ಇಂತ-ಹ-ವ-ರಿಗೆ
ಇಂತ-ಹ-ವರು
ಇಂತ-ಹ-ವರೆ
ಇಂತ-ಹ-ವರೇ
ಇಂತ-ಹ-ವು-ಗಳ
ಇಂತ-ಹ-ವು-ಗ-ಳೆಲ್ಲ
ಇಂತ-ಹುದು
ಇಂಥ
ಇಂಥ-ದನ್ನು
ಇಂಥ-ದಲ್ಲ
ಇಂಥ-ವನು
ಇಂಥ-ವರ
ಇಂಥ-ವರೆ
ಇಂದಿಗೂ
ಇಂದಿನ
ಇಂದಿ-ನದು
ಇಂದಿ-ನ-ವ-ರೆಗೆ
ಇಂದಿ-ನಿಂದ
ಇಂದು
ಇಂದ್ರ
ಇಂದ್ರ-ಜಿತು
ಇಂದ್ರ-ನಂತೆ
ಇಂದ್ರ-ಭೋ-ಗ-ವನ್ನು
ಇಂದ್ರಿಯ
ಇಂದ್ರಿ-ಯ-ಕ್ಕಿಂತ
ಇಂದ್ರಿ-ಯಕ್ಕೂ
ಇಂದ್ರಿ-ಯಕ್ಕೆ
ಇಂದ್ರಿ-ಯ-ಕ್ರಿ-ಯೆ-ಗ-ಳ-ಲ್ಲಿಯೂ
ಇಂದ್ರಿ-ಯ-ಗಳ
ಇಂದ್ರಿ-ಯ-ಗ-ಳಂತೆ
ಇಂದ್ರಿ-ಯ-ಗಳನ್ನು
ಇಂದ್ರಿ-ಯ-ಗಳನ್ನೂ
ಇಂದ್ರಿ-ಯ-ಗಳನ್ನೆಲ್ಲ
ಇಂದ್ರಿ-ಯ-ಗಳನ್ನೆಲ್ಲಾ
ಇಂದ್ರಿ-ಯ-ಗಳಲ್ಲಿ
ಇಂದ್ರಿ-ಯ-ಗ-ಳಾದ
ಇಂದ್ರಿ-ಯ-ಗ-ಳಾ-ದರೋ
ಇಂದ್ರಿ-ಯ-ಗಳಿಂದ
ಇಂದ್ರಿ-ಯ-ಗ-ಳಿಂ-ದಲೂ
ಇಂದ್ರಿ-ಯ-ಗ-ಳಿ-ಗಿಂತ
ಇಂದ್ರಿ-ಯ-ಗ-ಳಿಗೂ
ಇಂದ್ರಿ-ಯ-ಗ-ಳಿಗೆ
ಇಂದ್ರಿ-ಯ-ಗ-ಳಿವೆ
ಇಂದ್ರಿ-ಯ-ಗಳು
ಇಂದ್ರಿ-ಯ-ಗಳೂ
ಇಂದ್ರಿ-ಯ-ಗ-ಳೆಲ್ಲ
ಇಂದ್ರಿ-ಯ-ಗ-ಳೆ-ಲ್ಲವೂ
ಇಂದ್ರಿ-ಯ-ಗ-ಳೆಲ್ಲಾ
ಇಂದ್ರಿ-ಯ-ಗಳೇ
ಇಂದ್ರಿ-ಯ-ಗ್ರಾ-ಹ್ಯವೂ
ಇಂದ್ರಿ-ಯ-ಜನ್ಯ
ಇಂದ್ರಿ-ಯದ
ಇಂದ್ರಿ-ಯ-ದಲ್ಲಿ
ಇಂದ್ರಿ-ಯ-ದ-ಲ್ಲಿದೆ
ಇಂದ್ರಿ-ಯ-ದಲ್ಲೇ
ಇಂದ್ರಿ-ಯ-ದೃ-ಷ್ಟಿ-ಯಿಂದ
ಇಂದ್ರಿ-ಯ-ನಿ-ಗ್ರಹ
ಇಂದ್ರಿ-ಯ-ವನ್ನು
ಇಂದ್ರಿ-ಯ-ವ-ಸ್ತು-ಗಳನ್ನು
ಇಂದ್ರಿ-ಯ-ವಾ-ಗು-ವುದು
ಇಂದ್ರಿ-ಯವೂ
ಇಂದ್ರಿ-ಯ-ವೆಂಬ
ಇಂದ್ರಿ-ಯ-ಸುಖ
ಇಂದ್ರಿ-ಯ-ಸು-ಖ-ವನ್ನು
ಇಂದ್ರಿ-ಯ-ಸು-ಖ-ವಾ-ದರೂ
ಇಂದ್ರಿ-ಯ-ಸ್ಯೇಂ-ದ್ರಿ-ಯ-ಸ್ಯಾರ್ಥೇ
ಇಂದ್ರಿಯಾ
ಇಂದ್ರಿ-ಯಾ-ಕ-ರ್ಷ-ಣೆ-ಯಿಂದ
ಇಂದ್ರಿ-ಯಾ-ಗ್ನಿಷು
ಇಂದ್ರಿ-ಯಾ-ಣಾಂ
ಇಂದ್ರಿ-ಯಾಣಿ
ಇಂದ್ರಿ-ಯಾ-ಣೀಂ-ದ್ರಿ-ಯಾ-ರ್ಥೇ-ಭ್ಯ-ಸ್ತಸ್ಯ
ಇಂದ್ರಿ-ಯಾ-ಣೀಂ-ದ್ರಿ-ಯಾ-ರ್ಥೇಷು
ಇಂದ್ರಿ-ಯಾ-ತೀತ
ಇಂದ್ರಿ-ಯಾ-ತೀ-ತ-ವಾ-ಗಿ-ರು-ವುದನ್ನು
ಇಂದ್ರಿ-ಯಾ-ತೀ-ತ-ವಾದ
ಇಂದ್ರಿ-ಯಾ-ತೀ-ತ-ವಾ-ದುದು
ಇಂದ್ರಿ-ಯಾ-ರಾ-ಮ-ನಾ-ಗು-ತ್ತಾನೆ
ಇಂದ್ರಿ-ಯಾ-ರಾ-ಮರೂ
ಇಂದ್ರಿ-ಯಾ-ರ್ಥಾನ್
ಇಂದ್ರಿ-ಯಾ-ರ್ಥೇಷು
ಇಂದ್ರೀ-ಯಾ-ತೀತ
ಇಂಪಾದ
ಇಕ್ಕ-ಳ-ದಿಂದ
ಇಕ್ಕೆ-ಲ-ಗಳಲ್ಲಿ
ಇಕ್ಷಾ
ಇಕ್ಷ್ವಾಕು
ಇಕ್ಷ್ವಾ-ಕು-ವಿಗೆ
ಇಚ್ಛಾ
ಇಚ್ಛಾ-ದ್ವೇ-ಷ-ಸ-ಮು-ತ್ಥೇನ
ಇಚ್ಛಾ-ನು-ಸಾರ
ಇಚ್ಛಾ-ಮ-ರ-ಣಿ-ಗಳು
ಇಚ್ಛಾ-ಮಾ-ತ್ರ-ದಿಂದ
ಇಚ್ಛಾ-ಶಕ್ತಿ
ಇಚ್ಛಾ-ಶ-ಕ್ತಿಗೆ
ಇಚ್ಛಾ-ಶ-ಕ್ತಿ-ಯಿಂ-ದಲೇ
ಇಚ್ಛಿ-ಸದೆ
ಇಚ್ಛಿ-ಸದೇ
ಇಚ್ಛಿ-ಸ-ಬೇಕೇ
ಇಚ್ಛಿ-ಸಿದ
ಇಚ್ಛಿ-ಸಿ-ದಂತೆ
ಇಚ್ಛಿ-ಸಿ-ದಂ-ತೆಯೇ
ಇಚ್ಛಿ-ಸಿ-ದನು
ಇಚ್ಛಿ-ಸಿ-ದರೂ
ಇಚ್ಛಿ-ಸಿ-ದರೆ
ಇಚ್ಛಿ-ಸಿ-ದಾಗ
ಇಚ್ಛಿ-ಸಿ-ದು-ದ-ರಿಂದ
ಇಚ್ಛಿ-ಸಿದ್ದು
ಇಚ್ಛಿ-ಸುತ್ತ
ಇಚ್ಛಿ-ಸು-ತ್ತಾನೆ
ಇಚ್ಛಿ-ಸು-ತ್ತಾ-ನೆಯೋ
ಇಚ್ಛಿ-ಸು-ತ್ತೇನೆ
ಇಚ್ಛಿ-ಸು-ತ್ತೇವ
ಇಚ್ಛಿ-ಸು-ತ್ತೇ-ವೆಯೇ
ಇಚ್ಛಿ-ಸು-ತ್ತೇ-ವೆಯೋ
ಇಚ್ಛಿ-ಸುವ
ಇಚ್ಛಿ-ಸು-ವನು
ಇಚ್ಛಿ-ಸು-ವನೊ
ಇಚ್ಛಿ-ಸು-ವು-ದಿಲ್ಲ
ಇಚ್ಛಿ-ಸು-ವು-ದಿ-ಲ್ಲವೊ
ಇಚ್ಛಿ-ಸು-ವು-ದಿ-ಲ್ಲವೋ
ಇಚ್ಛಿ-ಸು-ವುದು
ಇಚ್ಛಿ-ಸು-ವು-ದೆಲ್ಲಾ
ಇಚ್ಛಿ-ಸು-ವೆಯೊ
ಇಚ್ಛಿ-ಸು-ವೆವು
ಇಚ್ಛೆ
ಇಚ್ಛೆ-ಗಳನ್ನು
ಇಚ್ಛೆಗೆ
ಇಚ್ಛೆ-ಪಟ್ಟು
ಇಚ್ಛೆ-ಪ-ಡದೆ
ಇಚ್ಛೆ-ಪ-ಡು-ತ್ತಾನೆ
ಇಚ್ಛೆ-ಪ-ಡು-ವು-ದಿಲ್ಲ
ಇಚ್ಛೆ-ಪ-ಡು-ವು-ದಿ-ಲ್ಲವೋ
ಇಚ್ಛೆ-ಪ-ಡು-ವುದು
ಇಚ್ಛೆ-ಬಂ-ದ-ವ-ರಿಂದ
ಇಚ್ಛೆಯ
ಇಚ್ಛೆ-ಯಂತೆ
ಇಚ್ಛೆ-ಯನ್ನು
ಇಚ್ಛೆ-ಯಿಂ-ದಲೇ
ಇಚ್ಛೆ-ಯಿದೆ
ಇಚ್ಛೆ-ಯಿ-ದ್ದರೆ
ಇಚ್ಛೆ-ಯಿಲ್ಲ
ಇಚ್ಛೆ-ಯಿ-ಲ್ಲದ
ಇಚ್ಛೆ-ಯಿ-ಲ್ಲದೆ
ಇಚ್ಛೆ-ಯಿ-ಲ್ಲದೇ
ಇಚ್ಛೆಯೂ
ಇಚ್ಛೆ-ಯೇನೋ
ಇಚ್ಛೆಯೋ
ಇಜ್ಯತೇ
ಇಟ್ಟ
ಇಟ್ಟಂತೆ
ಇಟ್ಟರೂ
ಇಟ್ಟರೆ
ಇಟ್ಟ-ವರು
ಇಟ್ಟಾಗ
ಇಟ್ಟಿಗೆ
ಇಟ್ಟಿತು
ಇಟ್ಟಿದೆ
ಇಟ್ಟಿ-ದ್ದರೆ
ಇಟ್ಟಿ-ದ್ದಳು
ಇಟ್ಟಿ-ದ್ದಾನೆ
ಇಟ್ಟಿದ್ದು
ಇಟ್ಟಿ-ದ್ದೆಲ್ಲ
ಇಟ್ಟಿ-ರ-ಬೇಕು
ಇಟ್ಟಿ-ರಲಿ
ಇಟ್ಟಿ-ರಲು
ಇಟ್ಟಿ-ರು-ತ್ತಾನೆ
ಇಟ್ಟಿ-ರುವ
ಇಟ್ಟಿ-ರು-ವನು
ಇಟ್ಟಿ-ರು-ವನೋ
ಇಟ್ಟಿ-ರು-ವರು
ಇಟ್ಟಿ-ರು-ವಾಗ
ಇಟ್ಟಿ-ರು-ವು-ದಕ್ಕೆ
ಇಟ್ಟಿ-ರು-ವುದು
ಇಟ್ಟಿ-ರು-ವು-ದೆಲ್ಲಾ
ಇಟ್ಟು
ಇಟ್ಟು-ಕೊಂ-ಡರೆ
ಇಟ್ಟು-ಕೊಂಡಿ
ಇಟ್ಟು-ಕೊಂ-ಡಿ-ದ್ದರು
ಇಟ್ಟು-ಕೊಂ-ಡಿ-ದ್ದರೆ
ಇಟ್ಟು-ಕೊಂ-ಡಿ-ರ-ಬೇಕು
ಇಟ್ಟು-ಕೊಂ-ಡಿರು
ಇಟ್ಟು-ಕೊಂ-ಡಿ-ರುವ
ಇಟ್ಟು-ಕೊಂ-ಡಿ-ರು-ವನು
ಇಟ್ಟು-ಕೊಂ-ಡಿ-ರು-ವರು
ಇಟ್ಟು-ಕೊಂ-ಡಿ-ರು-ವರೊ
ಇಟ್ಟು-ಕೊಂ-ಡಿ-ರು-ವಾ-ಗಲೂ
ಇಟ್ಟು-ಕೊಂ-ಡಿ-ರು-ವು-ದ-ಕ್ಕಾ-ಗು-ವು-ದಿಲ್ಲ
ಇಟ್ಟು-ಕೊಂ-ಡಿ-ರು-ವು-ದ-ಕ್ಕಿಂತ
ಇಟ್ಟು-ಕೊಂ-ಡಿ-ರು-ವೆವೊ
ಇಟ್ಟು-ಕೊಂ-ಡಿ-ರು-ವೆವೋ
ಇಟ್ಟು-ಕೊಂ-ಡಿಲ್ಲ
ಇಟ್ಟು-ಕೊಂಡು
ಇಟ್ಟು-ಕೊಂ-ಡೆವೊ
ಇಟ್ಟು-ಕೊ-ಳ್ಳದೆ
ಇಟ್ಟು-ಕೊ-ಳ್ಳ-ಬೇಕು
ಇಟ್ಟು-ಕೊ-ಳ್ಳ-ಬೇಡ
ಇಟ್ಟು-ಕೊ-ಳ್ಳ-ಬೇಡಿ
ಇಟ್ಟು-ಕೊಳ್ಳಿ
ಇಟ್ಟು-ಕೊ-ಳ್ಳು-ತ್ತಿ-ರು-ವನು
ಇಟ್ಟು-ಕೊ-ಳ್ಳು-ವುದು
ಇಟ್ಟು-ಕೊ-ಳ್ಳು-ವೆವೊ
ಇಟ್ಟು-ಕೊ-ಳ್ಳೋಣ
ಇಟ್ಟುಕೋ
ಇಟ್ಟೆವು
ಇಡ-ದಂತೆ
ಇಡ-ದ-ವನು
ಇಡದೆ
ಇಡ-ಬ-ಹುದು
ಇಡ-ಬೇ-ಕಾ-ಗಿದೆ
ಇಡ-ಬೇಕು
ಇಡ-ಲಾ-ಗು-ವು-ದಿಲ್ಲ
ಇಡ-ಲಾ-ರದು
ಇಡ-ಲಾ-ರದೆ
ಇಡ-ಲಾ-ರವು
ಇಡ-ಲಿಲ್ಲ
ಇಡಲು
ಇಡಿ
ಇಡೀ
ಇಡು
ಇಡು-ತ್ತವೆ
ಇಡು-ತ್ತಾನೆ
ಇಡು-ತ್ತಿದ್ದ
ಇಡು-ತ್ತಿ-ದ್ದರು
ಇಡು-ತ್ತೇನೆ
ಇಡು-ತ್ತೇವೆ
ಇಡು-ತ್ತೇವೋ
ಇಡುವ
ಇಡು-ವನು
ಇಡು-ವರು
ಇಡು-ವಳು
ಇಡು-ವು-ದ-ಕ್ಕಾಗು
ಇಡು-ವು-ದ-ಕ್ಕಾ-ಗು-ವು-ದಿಲ್ಲ
ಇಡು-ವು-ದಕ್ಕೆ
ಇಡು-ವು-ದಿಲ್ಲ
ಇಡು-ವುದು
ಇಡು-ವುವು
ಇಡು-ವೆವು
ಇತರ
ಇತ-ರರ
ಇತ-ರ-ರಂತೆ
ಇತ-ರ-ರನ್ನು
ಇತ-ರ-ರನ್ನೂ
ಇತ-ರ-ರ-ರಿಗೆ
ಇತ-ರ-ರಲ್ಲಿ
ಇತ-ರ-ರ-ಲ್ಲಿ-ರುವ
ಇತ-ರ-ರ-ಲ್ಲಿ-ರು-ವುದು
ಇತ-ರರಿ
ಇತ-ರ-ರಿಂದ
ಇತ-ರ-ರಿ-ಗಾದ
ಇತ-ರ-ರಿ-ಗಿಂತ
ಇತ-ರ-ರಿ-ಗಿಂ-ತಲೂ
ಇತ-ರ-ರಿ-ಗಿ-ಲ್ಲದ
ಇತ-ರ-ರಿಗೂ
ಇತ-ರ-ರಿಗೆ
ಇತ-ರ-ರಿ-ಗೊಂದು
ಇತ-ರರು
ಇತ-ರರೂ
ಇತ-ರ-ರೆಲ್ಲ
ಇತ-ರ-ರೆ-ಲ್ಲ-ರಿ-ಗಿಂತ
ಇತ-ರ-ರೆ-ಲ್ಲರೂ
ಇತ-ರ-ರೊಂ-ದಿಗೆ
ಇತ-ರ-ರೊ-ಡನೆ
ಇತಿ
ಇತಿ-ಮಿ-ತಿ-ಯಿ-ಲ್ಲದೆ
ಇತಿ-ಹಾಸ
ಇತಿ-ಹಾ-ಸ-ವೆಂದು
ಇತೀಯಂ
ಇತ್ತ
ಇತ್ತ-ಕಡೆ
ಇತ್ತಿ-ದ್ದಾನೆ
ಇತ್ತಿ-ರು-ವು-ದೆಲ್ಲ
ಇತ್ತೀ-ಚೆಗೆ
ಇತ್ತು
ಇತ್ಯ-ಜ್ಞಾ-ನ-ವಿ-ಮೋ-ಹಿ-ತಾಃ
ಇತ್ಯ-ರ್ಜುನಂ
ಇತ್ಯರ್ಥ
ಇತ್ಯಹಂ
ಇತ್ಯಾದಿ
ಇತ್ಯಾ-ದಿ-ಗಳು
ಇತ್ಯಾ-ದಿ-ಯಾ-ಗೆಲ್ಲ
ಇತ್ಯು-ಕ್ತ-ಸ್ತ-ಮಾ-ಹುಃ
ಇತ್ಯು-ಚ್ಯತೇ
ಇತ್ಯೇವ
ಇದ
ಇದಂ
ಇದ-ಕ್ಕಾಗಿ
ಇದ-ಕ್ಕಿಂತ
ಇದ-ಕ್ಕಿಂ-ತಲೂ
ಇದಕ್ಕೂ
ಇದಕ್ಕೆ
ಇದ-ಕ್ಕೆಲ್ಲ
ಇದ-ಕ್ಕೆಲ್ಲಾ
ಇದ-ನ್ನ-ರಿತ
ಇದ-ನ್ನ-ರಿತು
ಇದ-ನ್ನ-ರಿತೇ
ಇದ-ನ್ನಾ-ದರೂ
ಇದನ್ನು
ಇದನ್ನೂ
ಇದನ್ನೆ
ಇದ-ನ್ನೆಲ್ಲ
ಇದ-ನ್ನೆಲ್ಲಾ
ಇದನ್ನೇ
ಇದ-ನ್ನೇನು
ಇದ-ಮದ್ಯ
ಇದ-ಮ-ಸ್ತೀ-ದ-ಮಪಿ
ಇದಯೆ
ಇದರ
ಇದ-ರಂತೆ
ಇದ-ರಂ-ತೆಯೆ
ಇದ-ರಂ-ತೆಯೇ
ಇದ-ರ-ಡಿ-ಯಲ್ಲಿ
ಇದ-ರ-ಮೇಲೆ
ಇದ-ರಲ್ಲಿ
ಇದ-ರ-ಲ್ಲಿನ
ಇದ-ರ-ಲ್ಲಿಯೂ
ಇದ-ರ-ಲ್ಲಿಯೇ
ಇದ-ರ-ಲ್ಲಿ-ರುವ
ಇದ-ರಲ್ಲೆ
ಇದ-ರಲ್ಲೇ
ಇದ-ರ-ಲ್ಲೇನೂ
ಇದ-ರಷ್ಟು
ಇದ-ರಷ್ಟೇ
ಇದ-ರಿಂದ
ಇದ-ರಿಂ-ದಲೂ
ಇದ-ರಿಂ-ದಲೇ
ಇದರೆ
ಇದ-ರೊಂ-ದಿಗೆ
ಇದ-ರೊ-ಡನೆ
ಇದಲ್ಲ
ಇದ-ಲ್ಲದೆ
ಇದಾ-ಗ-ಬೇಕು
ಇದಾ-ದ-ಮೇಲೆ
ಇದಾ-ನೀ-ಮಸ್ಮಿ
ಇದಾ-ವು-ದಕ್ಕೂ
ಇದಾ-ವುದನ್ನು
ಇದಾ-ವು-ದನ್ನೂ
ಇದಾ-ವು-ದಾ-ದರೂ
ಇದಾ-ವುದು
ಇದಾ-ವುದೂ
ಇದಿ-ದ್ದರೆ
ಇದಿ-ಲ್ಲ-ದ-ವನೇ
ಇದಿ-ಲ್ಲದೆ
ಇದಿ-ಲ್ಲವೋ
ಇದು
ಇದು-ವ-ರೆಗೂ
ಇದು-ವ-ರೆಗೆ
ಇದೂ
ಇದೆ
ಇದೆಂ-ತಹ
ಇದೆ-ಯಲ್ಲ
ಇದೆ-ಯಲ್ಲಾ
ಇದೆಯೆ
ಇದೆ-ಯೆ-ಎಂದು
ಇದೆಯೇ
ಇದೆಯೊ
ಇದೆಯೋ
ಇದೆಲ್ಲ
ಇದೆ-ಲ್ಲವೂ
ಇದೆಲ್ಲಾ
ಇದೆಲ್ಲೊ
ಇದೆಷ್ಟು
ಇದೇ
ಇದೇಕೆ
ಇದೇ-ನಾ-ದರೂ
ಇದೇನು
ಇದೇನೂ
ಇದೇನೆ
ಇದೇನೋ
ಇದೊಂದು
ಇದೊಂದೆ
ಇದೊಂದೇ
ಇದ್ದ
ಇದ್ದಂ-ತಹ
ಇದ್ದಂತೆ
ಇದ್ದಂ-ತೆಯೇ
ಇದ್ದ-ಕ-ಡೆಯೇ
ಇದ್ದ-ಕ್ಕಿ-ದ್ದಂತೆ
ಇದ್ದಕ್ಕೆ
ಇದ್ದದ್ದು
ಇದ್ದದ್ದೇ
ಇದ್ದನು
ಇದ್ದನೊ
ಇದ್ದನೋ
ಇದ್ದರು
ಇದ್ದರೂ
ಇದ್ದರೆ
ಇದ್ದ-ರೆಷ್ಟು
ಇದ್ದರೇ
ಇದ್ದ-ರೇನು
ಇದ್ದ-ರೇನೆ
ಇದ್ದ-ರೇನೇ
ಇದ್ದರೊ
ಇದ್ದಲು
ಇದ್ದ-ಲೊಂದು
ಇದ್ದ-ವ-ನಿಗೆ
ಇದ್ದ-ವನು
ಇದ್ದ-ವರೆಲ್ಲ
ಇದ್ದ-ವರೇ
ಇದ್ದವು
ಇದ್ದ-ಹಾಗೆ
ಇದ್ದಾಗ
ಇದ್ದಾ-ಗಲೂ
ಇದ್ದಾನೆ
ಇದ್ದಾ-ನೆಂ-ಬುದು
ಇದ್ದಾ-ನೆಯೊ
ಇದ್ದಾ-ನೆಯೋ
ಇದ್ದಾರು
ಇದ್ದಾರೆ
ಇದ್ದಾರೊ
ಇದ್ದಿತು
ಇದ್ದಿ-ದ್ದರೆ
ಇದ್ದಿ-ರ-ಬ-ಹುದು
ಇದ್ದಿ-ರ-ಬೇಕು
ಇದ್ದಿ-ಲನ್ನು
ಇದ್ದೀತು
ಇದ್ದೀರೊ
ಇದ್ದು
ಇದ್ದು-ಕೊಂಡು
ಇದ್ದು-ದ-ನ್ನೆಲ್ಲ
ಇದ್ದುದು
ಇದ್ದುವು
ಇದ್ದೆವು
ಇದ್ದೇ
ಇದ್ದೇನೆ
ಇದ್ಧವು
ಇನನ್ನು
ಇನಾ-ಕ್ಯು-ಲೇ-ಷನ್
ಇನ್ನದು
ಇನ್ನ-ವನು
ಇನ್ನ-ವ-ರಿಗೆ
ಇನ್ನಷ್ಟು
ಇನ್ನಾ-ರನ್ನೊ
ಇನ್ನಾ-ರನ್ನೋ
ಇನ್ನಾ-ರ-ಲ್ಲಿಯೂ
ಇನ್ನಾ-ರಿಗೊ
ಇನ್ನಾರು
ಇನ್ನಾರೂ
ಇನ್ನಾರೊ
ಇನ್ನಾರೋ
ಇನ್ನಾವ
ಇನ್ನಾ-ವು-ದನ್ನೂ
ಇನ್ನಾ-ವು-ದ-ರಿಂ-ದಲೂ
ಇನ್ನಾ-ವುದೂ
ಇನ್ನಾ-ವುದೊ
ಇನ್ನಾ-ವುದೋ
ಇನ್ನಿಲ್ಲ
ಇನ್ನು
ಇನ್ನು-ಮೇ-ಲಾ-ದರೂ
ಇನ್ನು-ಮೇಲೆ
ಇನ್ನು-ಳಿ-ದ-ವರು
ಇನ್ನೂ
ಇನ್ನೆ-ರಡು
ಇನ್ನೆಲ್ಲ
ಇನ್ನೆಲ್ಲಿ
ಇನ್ನೆ-ಲ್ಲಿಯೂ
ಇನ್ನೆ-ಲ್ಲಿಯೋ
ಇನ್ನೆಲ್ಲೊ
ಇನ್ನೆಷ್ಟು
ಇನ್ನೇ-ನನ್ನೂ
ಇನ್ನೇನು
ಇನ್ನೇನೂ
ಇನ್ನೊಂ-ದಕ್ಕೂ
ಇನ್ನೊಂ-ದಕ್ಕೆ
ಇನ್ನೊಂ-ದನ್ನು
ಇನ್ನೊಂ-ದರ
ಇನ್ನೊಂ-ದ-ರಿಂದ
ಇನ್ನೊಂದು
ಇನ್ನೊಂ-ದೆ-ಡೆಗೆ
ಇನ್ನೊಬ್ಬ
ಇನ್ನೊ-ಬ್ಬನ
ಇನ್ನೊ-ಬ್ಬ-ನದು
ಇನ್ನೊ-ಬ್ಬ-ನನ್ನು
ಇನ್ನೊ-ಬ್ಬ-ನ-ಲ್ಲಿ-ರುವ
ಇನ್ನೊ-ಬ್ಬ-ನಿಂದ
ಇನ್ನೊ-ಬ್ಬ-ನಿಗೆ
ಇನ್ನೊ-ಬ್ಬನು
ಇನ್ನೊ-ಬ್ಬ-ನೊಂ-ದಿಗೆ
ಇನ್ನೊ-ಬ್ಬರ
ಇನ್ನೊ-ಬ್ಬ-ರದು
ಇನ್ನೊ-ಬ್ಬ-ರನ್ನು
ಇನ್ನೊ-ಬ್ಬ-ರನ್ನೂ
ಇನ್ನೊ-ಬ್ಬ-ರಿಂದ
ಇನ್ನೊ-ಬ್ಬ-ರಿ-ಗಾಗಿ
ಇನ್ನೊ-ಬ್ಬ-ರಿಗೆ
ಇನ್ನೊ-ಬ್ಬರು
ಇನ್ನೊಮ್ಮೆ
ಇನ್ಷು-ರೆ-ನ್ಸ್
ಇಪ್ಪ
ಇಪ್ಪ-ತ್ತ-ನೆಯ
ಇಪ್ಪತ್ತು
ಇಪ್ಪ-ತ್ತು-ನಾಲ್ಕು
ಇಪ್ಪ-ತ್ತು-ನಾ-ಲ್ಕು-ಸಾ-ವಿರ
ಇಪ್ಪ-ತ್ತೆಂಟು
ಇಬ್ಬರ
ಇಬ್ಬ-ರನ್ನು
ಇಬ್ಬ-ರಿಗೂ
ಇಬ್ಬರು
ಇಬ್ಬರೂ
ಇಬ್ಬರೊ
ಇಮಂ
ಇಮಾಂ-ಲ್ಲೋ-ಕಾನ್ನ
ಇಮಾಃ
ಇಮೇ
ಇಮೇ-ವ-ಸ್ಥಿತಾ
ಇರ
ಇರ-ಕೂ-ಡದು
ಇರ-ಗೊ-ಡದು
ಇರ-ಗೊ-ಡಿ-ಸದು
ಇರ-ಗೊ-ಡಿ-ಸು-ವು-ದಿಲ್ಲ
ಇರ-ದಂತೆ
ಇರದು
ಇರದೆ
ಇರದೇ
ಇರನು
ಇರ-ಬಲ್ಲ
ಇರ-ಬ-ಲ್ಲದು
ಇರ-ಬ-ಲ್ಲರು
ಇರ-ಬ-ಲ್ಲವು
ಇರ-ಬ-ಲ್ಲುದು
ಇರ-ಬ-ಲ್ಲುವು
ಇರ-ಬ-ಹು-ದಲ್ಲ
ಇರ-ಬ-ಹುದು
ಇರ-ಬಾ-ರದು
ಇರ-ಬೇ-ಕಾ-ಗಿದೆ
ಇರ-ಬೇ-ಕಾ-ಗಿಲ್ಲ
ಇರ-ಬೇ-ಕಾ-ಗು-ವುದು
ಇರ-ಬೇ-ಕಾದ
ಇರ-ಬೇ-ಕಾ-ದರೂ
ಇರ-ಬೇ-ಕಾ-ದರೆ
ಇರ-ಬೇಕು
ಇರ-ಬೇ-ಕೆಂ-ದರೆ
ಇರ-ಬೇ-ಕೆಂದು
ಇರ-ಬೇಕೊ
ಇರ-ಲಾರ
ಇರ-ಲಾ-ರದು
ಇರ-ಲಾ-ರದೊ
ಇರ-ಲಾ-ರನು
ಇರ-ಲಾ-ರರು
ಇರ-ಲಾ-ರವು
ಇರಲಿ
ಇರ-ಲಿಕ್ಕೆ
ಇರ-ಲಿಲ್ಲ
ಇರ-ಲಿ-ಲ್ಲವೆ
ಇರ-ಲಿ-ಲ್ಲ-ವೆಂಬ
ಇರ-ಲಿ-ಲ್ಲ-ವೆಂ-ಬು-ದಿಲ್ಲ
ಇರ-ಲಿ-ಲ್ಲವೊ
ಇರಲು
ಇರ-ಲೆಂದು
ಇರಲೇ
ಇರ-ಲೇ-ಬೇ-ಕಾ-ಗಿದೆ
ಇರ-ಲೇ-ಬೇಕು
ಇರ-ಲೇ-ಬೇ-ಕೆಂದು
ಇರ-ವನು
ಇರ-ವು-ದಿಲ್ಲ
ಇರ-ವುದೊ
ಇರ-ವುವು
ಇರಿ
ಇರಿ-ಯು-ವುವು
ಇರಿ-ಸಿ-ದ್ದೇನೆ
ಇರಿ-ಸು-ವು-ದ-ಕ್ಕಾಗಿ
ಇರು
ಇರು-ತ್ತದೆ
ಇರು-ತ್ತವೆ
ಇರು-ತ್ತಾನೆ
ಇರು-ತ್ತಾರೆ
ಇರು-ತ್ತಾ-ರೆಯೆ
ಇರು-ತ್ತಾ-ರೆಯೇ
ಇರು-ತ್ತಿತ್ತು
ಇರು-ತ್ತಿ-ರ-ಲಿಲ್ಲ
ಇರುತ್ತೇ
ಇರು-ತ್ತೇವೆ
ಇರು-ತ್ತೇ-ವೆಯೊ
ಇರು-ಳಲ್ಲಿ
ಇರುವ
ಇರು-ವಂ-ತ-ಹುದು
ಇರು-ವಂತೆ
ಇರು-ವ-ತ-ನಕ
ಇರು-ವನು
ಇರು-ವನೆ
ಇರು-ವನೊ
ಇರು-ವನೋ
ಇರು-ವರು
ಇರು-ವರೊ
ಇರು-ವರೋ
ಇರು-ವಳೊ
ಇರು-ವವ
ಇರು-ವ-ವ-ನಂತೆ
ಇರು-ವ-ವ-ನನ್ನು
ಇರು-ವ-ವ-ನಲ್ಲ
ಇರು-ವ-ವ-ನಲ್ಲಿ
ಇರು-ವ-ವ-ನಾಗು
ಇರು-ವ-ವ-ನಿ-ಗಂ-ತಲೂ
ಇರು-ವ-ವ-ನಿ-ಗಿಂತ
ಇರು-ವ-ವ-ನಿಗೂ
ಇರು-ವ-ವ-ನಿಗೆ
ಇರು-ವ-ವನು
ಇರು-ವ-ವನೂ
ಇರು-ವ-ವನೆ
ಇರು-ವ-ವನೇ
ಇರು-ವ-ವ-ನೊಂ-ದಿಗೆ
ಇರು-ವ-ವರ
ಇರು-ವ-ವ-ರನ್ನು
ಇರು-ವ-ವ-ರ-ನ್ನೆಲ್ಲ
ಇರು-ವ-ವ-ರಲ್ಲಿ
ಇರು-ವ-ವ-ರಿಗೆ
ಇರು-ವ-ವ-ರಿ-ಗೆಲ್ಲ
ಇರು-ವ-ವರು
ಇರು-ವ-ವರೂ
ಇರು-ವ-ವ-ರೆಗೆ
ಇರು-ವ-ವರೆಲ್ಲ
ಇರು-ವ-ವ-ರೊ-ಡನೆ
ಇರು-ವಷ್ಟು
ಇರು-ವಾಗ
ಇರು-ವಾ-ಗ-ಲಾ-ದರೂ
ಇರು-ವಾ-ಗಲೂ
ಇರು-ವಾ-ಗಲೆ
ಇರು-ವಾ-ಗಲೇ
ಇರು-ವಿ-ಕೆ-ಯಲ್ಲಿ
ಇರು-ವಿ-ಕೆ-ಯಿಲ್ಲ
ಇರುವು
ಇರು-ವುದ
ಇರು-ವು-ದ-ಕ್ಕಲ್ಲ
ಇರು-ವು-ದ-ಕ್ಕಾ-ಗು-ವು-ದಿಲ್ಲ
ಇರು-ವು-ದ-ಕ್ಕಿಂತ
ಇರು-ವು-ದಕ್ಕೂ
ಇರು-ವು-ದಕ್ಕೆ
ಇರು-ವು-ದ-ನ್ನಾ-ದರೂ
ಇರು-ವುದನ್ನು
ಇರು-ವು-ದನ್ನೆ
ಇರು-ವು-ದ-ನ್ನೆಲ್ಲ
ಇರು-ವು-ದ-ನ್ನೆಲ್ಲಾ
ಇರು-ವು-ದರ
ಇರು-ವು-ದ-ರಲ್ಲಿ
ಇರು-ವು-ದ-ರಿಂದ
ಇರು-ವು-ದ-ರಿಂ-ದಲೇ
ಇರು-ವು-ದಲ್ಲ
ಇರು-ವು-ದಾ-ವುದೂ
ಇರು-ವು-ದಿಲ್ಲ
ಇರು-ವು-ದಿ-ಲ್ಲವೆ
ಇರು-ವು-ದಿ-ಲ್ಲವೋ
ಇರು-ವುದು
ಇರು-ವು-ದು-ಇವು
ಇರು-ವುದೂ
ಇರು-ವುದೆ
ಇರು-ವು-ದೆಲ್ಲ
ಇರು-ವು-ದೆಲ್ಲಾ
ಇರು-ವು-ದೆಲ್ಲಿ
ಇರು-ವುದೇ
ಇರು-ವು-ದೇನು
ಇರು-ವುದೊ
ಇರು-ವು-ದೊ-ಅಂ-ತಹ
ಇರು-ವು-ದೊಂದೆ
ಇರು-ವು-ದೊಂದೇ
ಇರು-ವುದೋ
ಇರು-ವುಲ್ಲ
ಇರು-ವುವು
ಇರು-ವುವೋ
ಇರುವೆ
ಇರು-ವೆನು
ಇರು-ವೆಯ
ಇರು-ವೆ-ಯ-ಮುಂದೆ
ಇರು-ವೆ-ಯಾ-ದರೋ
ಇರು-ವೆಯೋ
ಇರು-ವೆವು
ಇರು-ವೆವೊ
ಇರು-ವೆವೋ
ಇಲಿ
ಇಲಿ-ಯನ್ನು
ಇಲ್ಲ
ಇಲ್ಲ-ಎಂದು
ಇಲ್ಲದ
ಇಲ್ಲ-ದಂ-ತಾ-ಗು-ವುದು
ಇಲ್ಲ-ದಂತೆ
ಇಲ್ಲ-ದಂ-ತೆಯೆ
ಇಲ್ಲ-ದ-ವ-ನಾಗಿ
ಇಲ್ಲ-ದ-ವ-ನಿಂದ
ಇಲ್ಲ-ದ-ವ-ನಿಗೆ
ಇಲ್ಲ-ದ-ವನು
ಇಲ್ಲ-ದ-ವನೂ
ಇಲ್ಲ-ದ-ವನೋ
ಇಲ್ಲ-ದ-ವರು
ಇಲ್ಲ-ದ-ವ-ರೆಂದೂ
ಇಲ್ಲ-ದ-ವರೊ
ಇಲ್ಲ-ದಾಗ
ಇಲ್ಲ-ದಿ-ದ್ದರೂ
ಇಲ್ಲ-ದಿ-ದ್ದರೆ
ಇಲ್ಲ-ದಿ-ದ್ದು-ದ-ರಿಂದ
ಇಲ್ಲ-ದಿ-ರಲಿ
ಇಲ್ಲ-ದಿ-ರುವ
ಇಲ್ಲ-ದಿ-ರು-ವಾಗ
ಇಲ್ಲ-ದಿ-ರು-ವುದು
ಇಲ್ಲ-ದು-ದನ್ನು
ಇಲ್ಲ-ದುದು
ಇಲ್ಲದೆ
ಇಲ್ಲದೇ
ಇಲ್ಲದ್ದು
ಇಲ್ಲ-ವಲ್ಲ
ಇಲ್ಲ-ವಾ-ಗು-ವರು
ಇಲ್ಲವೆ
ಇಲ್ಲ-ವೆಂ-ತಲೂ
ಇಲ್ಲ-ವೆಂ-ದಲ್ಲ
ಇಲ್ಲ-ವೆಂದು
ಇಲ್ಲ-ವೆ-ನ್ನು-ವ-ವ-ನಲ್ಲ
ಇಲ್ಲ-ವೆ-ನ್ನು-ವು-ದಿಲ್ಲ
ಇಲ್ಲವೇ
ಇಲ್ಲವೊ
ಇಲ್ಲವೋ
ಇಲ್ಲಾ
ಇಲ್ಲಿ
ಇಲ್ಲಿಂದ
ಇಲ್ಲಿಗೆ
ಇಲ್ಲಿದೆ
ಇಲ್ಲಿ-ದ್ದೇನೆ
ಇಲ್ಲಿನ
ಇಲ್ಲಿಯೂ
ಇಲ್ಲಿಯೆ
ಇಲ್ಲಿಯೇ
ಇಲ್ಲಿ-ರ-ಬೇಕು
ಇಲ್ಲಿ-ರುವ
ಇಲ್ಲಿ-ರು-ವನು
ಇಲ್ಲಿ-ರು-ವಾ-ಗಲೂ
ಇಲ್ಲಿ-ರು-ವುದನ್ನು
ಇಲ್ಲಿ-ರು-ವುದು
ಇಲ್ಲಿ-ರು-ವು-ದೆಲ್ಲ
ಇಲ್ಲಿವೆ
ಇಲ್ಲೆ
ಇಲ್ಲೆಲ್ಲ
ಇಲ್ಲೇ
ಇಳಿ
ಇಳಿ-ತ-ಗಳನ್ನು
ಇಳಿದ
ಇಳಿ-ದಂತೆ
ಇಳಿ-ದರೆ
ಇಳಿ-ದಾಗ
ಇಳಿ-ದಿ-ದ್ದರು
ಇಳಿ-ದಿಲ್ಲ
ಇಳಿದು
ಇಳಿ-ದು-ಬಂದ
ಇಳಿ-ದು-ಬಂ-ದಿ-ರು-ವನು
ಇಳಿ-ದು-ಬಂದು
ಇಳಿ-ದು-ಬ-ರು-ತ್ತವೆ
ಇಳಿ-ದು-ಬ-ರು-ತ್ತಾನೆ
ಇಳಿ-ದು-ಬ-ರು-ವನು
ಇಳಿ-ದು-ಬ-ರು-ವಾಗ
ಇಳಿ-ದು-ಬ-ರು-ವುದು
ಇಳಿ-ಯದ
ಇಳಿ-ಯದೇ
ಇಳಿ-ಯ-ಬ-ಹುದು
ಇಳಿ-ಯ-ಬೇ-ಕಾ-ಗ-ಬ-ಹುದು
ಇಳಿ-ಯ-ಬೇ-ಕಾ-ಗಿಲ್ಲ
ಇಳಿ-ಯ-ಬೇ-ಕಾದ
ಇಳಿ-ಯ-ಬೇ-ಕಾ-ದರೆ
ಇಳಿ-ಯ-ಬೇಕು
ಇಳಿಯು
ಇಳಿ-ಯುತ್ತ
ಇಳಿ-ಯು-ತ್ತದೆ
ಇಳಿ-ಯು-ತ್ತವೆ
ಇಳಿ-ಯುತ್ತಾ
ಇಳಿ-ಯು-ತ್ತಾನೆ
ಇಳಿ-ಯು-ತ್ತಿ-ರು-ವಾಗ
ಇಳಿ-ಯು-ತ್ತೇವೆ
ಇಳಿ-ಯುವ
ಇಳಿ-ಯು-ವನು
ಇಳಿ-ಯು-ವನೊ
ಇಳಿ-ಯು-ವು-ದಿಲ್ಲ
ಇಳಿ-ಯು-ವುದು
ಇಳಿ-ಯು-ವೆವು
ಇಳಿ-ಸ-ಬೇಕು
ಇಳಿ-ಸಿ-ದಂ-ತಾ-ಗು-ವುದು
ಇಳಿ-ಸು-ವನು
ಇಳಿ-ಸು-ವುದು
ಇಳಿ-ಸೇನು
ಇವ
ಇವಕ್ಕೆ
ಇವ-ಕ್ಕೆಲ್ಲ
ಇವ-ತ್ತಿಗೆ
ಇವ-ತ್ತಿ-ನ-ವ-ರೆಗೆ
ಇವತ್ತು
ಇವ-ತ್ತೆಲ್ಲ
ಇವನ
ಇವ-ನಂತು
ಇವ-ನಂತೂ
ಇವ-ನಂತೆ
ಇವ-ನ-ದಲ್ಲ
ಇವ-ನದು
ಇವ-ನದೆ
ಇವ-ನ-ದೆಲ್ಲ
ಇವ-ನನ್ನು
ಇವ-ನನ್ನೆ
ಇವ-ನನ್ನೇ
ಇವ-ನಲ್ಲಿ
ಇವ-ನ-ಲ್ಲಿಗೆ
ಇವ-ನ-ಲ್ಲಿ-ರುವ
ಇವ-ನ-ಲ್ಲಿ-ರು-ವುದನ್ನು
ಇವ-ನ-ಲ್ಲಿ-ರು-ವು-ದಿಲ್ಲ
ಇವ-ನಷ್ಟು
ಇವ-ನಾ-ಗಲೇ
ಇವ-ನಾ-ದರೊ
ಇವ-ನಾ-ದರೋ
ಇವ-ನಾರು
ಇವ-ನಾರೊ
ಇವ-ನಿಂದ
ಇವ-ನಿ-ಗಿಂತ
ಇವ-ನಿಗೂ
ಇವ-ನಿಗೆ
ಇವ-ನಿಗೇ
ಇವ-ನಿ-ಗೇನೂ
ಇವ-ನಿ-ಗೇನೊ
ಇವ-ನಿ-ಗೊಂದು
ಇವ-ನಿನ್ನೂ
ಇವ-ನಿರು
ಇವ-ನಿ-ರು-ವೆಡೆ
ಇವನು
ಇವ-ನು-ವಿ-ವ-ರಿ-ಸ-ಲಾರ
ಇವನೂ
ಇವನೆ
ಇವ-ನೆ-ಡೆಗೆ
ಇವ-ನೆ-ದು-ರಿಗೆ
ಇವ-ನೆ-ಷ್ಟನ್ನು
ಇವನೇ
ಇವ-ನೇ-ನಾ-ದರೂ
ಇವ-ನೇನೂ
ಇವ-ನೊಂದು
ಇವ-ನೊಬ್ಬ
ಇವ-ನೊ-ಬ್ಬನೆ
ಇವ-ನೊ-ಬ್ಬನೇ
ಇವನ್ನು
ಇವ-ನ್ನೆಲ್ಲ
ಇವ-ನ್ನೆಲ್ಲಾ
ಇವನ್ನೇ
ಇವರ
ಇವ-ರನ್ನು
ಇವ-ರನ್ನೂ
ಇವ-ರ-ನ್ನೆಲ್ಲ
ಇವ-ರ-ನ್ನೆಲ್ಲಾ
ಇವ-ರನ್ನೇ
ಇವ-ರಲ್ಲಿ
ಇವ-ರ-ಲ್ಲಿಯೇ
ಇವ-ರಾಗಿ
ಇವ-ರಾರೂ
ಇವ-ರಿಂದ
ಇವ-ರಿಗೆ
ಇವ-ರಿ-ಬ್ಬರ
ಇವ-ರಿ-ಬ್ಬರೂ
ಇವರು
ಇವ-ರು-ಗಳ
ಇವ-ರು-ಗಳನ್ನು
ಇವ-ರು-ಗ-ಳಿಗೆ
ಇವ-ರು-ಗಳು
ಇವ-ರು-ಗ-ಳೆಲ್ಲ
ಇವರೆಲ್ಲ
ಇವರೆ-ಲ್ಲ-ರಿ-ಗಿಂತ
ಇವರೆ-ಲ್ಲರೂ
ಇವರೇ
ಇವಳ
ಇವಳು
ಇವಳೇ
ಇವಾ-ವು-ದನ್ನೂ
ಇವಿ-ಲ್ಲದೇ
ಇವು
ಇವು-ಗಳ
ಇವು-ಗ-ಳ-ಕಡೆ
ಇವು-ಗಳನ್ನು
ಇವು-ಗಳನ್ನೆಲ್ಲ
ಇವು-ಗಳನ್ನೆಲ್ಲಾ
ಇವು-ಗ-ಳನ್ನೇ
ಇವು-ಗ-ಳ-ಲ್ಲದೆ
ಇವು-ಗಳಲ್ಲಿ
ಇವು-ಗ-ಳ-ಲ್ಲಿಯೂ
ಇವು-ಗ-ಳ-ಲ್ಲೆಲ್ಲ
ಇವು-ಗ-ಳ-ಲ್ಲೆಲ್ಲಾ
ಇವು-ಗ-ಳಲ್ಸಿ
ಇವು-ಗ-ಳಾವು
ಇವು-ಗ-ಳಾ-ವು-ದಕ್ಕೂ
ಇವು-ಗ-ಳಾ-ವು-ದನ್ನೂ
ಇವು-ಗ-ಳಾ-ವು-ದರ
ಇವು-ಗ-ಳಾ-ವು-ದ-ರಿಂ-ದಲೂ
ಇವು-ಗ-ಳಾ-ವುವೂ
ಇವು-ಗಳಿಂದ
ಇವು-ಗ-ಳಿಂ-ದಲೇ
ಇವು-ಗ-ಳಿ-ಗಾಗಿ
ಇವು-ಗ-ಳಿ-ಗಾ-ವು-ದಕ್ಕೂ
ಇವು-ಗ-ಳಿ-ಗಿಂತ
ಇವು-ಗ-ಳಿಗೆ
ಇವು-ಗ-ಳಿ-ಗೆಲ್ಲ
ಇವು-ಗ-ಳಿ-ಗೆಲ್ಲಾ
ಇವು-ಗ-ಳಿ-ದ್ದರೆ
ಇವು-ಗ-ಳಿ-ರು-ವುದು
ಇವು-ಗ-ಳಿ-ಲ್ಲದೆ
ಇವು-ಗ-ಳಿ-ಲ್ಲದೇ
ಇವು-ಗಳು
ಇವು-ಗ-ಳು-ಳ್ಳ-ವನೂ
ಇವು-ಗಳೂ
ಇವು-ಗ-ಳೆ-ರ-ಡ-ರಲ್ಲಿ
ಇವು-ಗ-ಳೆ-ರಡೂ
ಇವು-ಗ-ಳೆಲ್ಲ
ಇವು-ಗ-ಳೆ-ಲ್ಲ-ದರ
ಇವು-ಗ-ಳೆ-ಲ್ಲ-ವನ್ನೂ
ಇವು-ಗ-ಳೆ-ಲ್ಲವು
ಇವು-ಗ-ಳೆ-ಲ್ಲವೂ
ಇವು-ಗ-ಳೆಲ್ಲಾ
ಇವು-ಗ-ಳೆಲ್ಲೆ
ಇವು-ಗ-ಳೆವೂ
ಇವು-ಗಳೇ
ಇವು-ಗ-ಳೊಂ-ದಿಗೆ
ಇವೂ
ಇವೆ
ಇವೆಯೊ
ಇವೆಯೋ
ಇವೆ-ರ-ಡಕ್ಕೂ
ಇವೆ-ರ-ಡನ್ನು
ಇವೆ-ರ-ಡನ್ನೂ
ಇವೆ-ರ-ಡರ
ಇವೆ-ರ-ಡ-ರಲ್ಲಿ
ಇವೆ-ರ-ಡ-ರ-ಲ್ಲಿಯೂ
ಇವೆ-ರ-ಡ-ರಲ್ಲೂ
ಇವೆ-ರ-ಡ-ರಿಂದ
ಇವೆ-ರ-ಡ-ರಿಂ-ದಲೂ
ಇವೆ-ರಡು
ಇವೆ-ರಡೂ
ಇವೆಲ್ಲ
ಇವೆ-ಲ್ಲ-ಮಾ-ಯಾ-ಪ್ರ-ಪಂ-ಚ-ದ-ಲ್ಲಿವೆ
ಇವೆ-ಲ್ಲ-ವನ್ನೂ
ಇವೆ-ಲ್ಲವೂ
ಇವೆಲ್ಲಾ
ಇವೇ
ಇಷು-ಭಿಃ
ಇಷ್ಟ
ಇಷ್ಟಕ್ಕೆ
ಇಷ್ಟಕ್ಕೇ
ಇಷ್ಟ-ದಲ್ಲಿ
ಇಷ್ಟ-ದಿಂ-ದಲೂ
ಇಷ್ಟ-ದೇ-ವ-ರಾದ
ಇಷ್ಟ-ದೈವ
ಇಷ್ಟ-ನ್ನಾ-ದರೂ
ಇಷ್ಟನ್ನು
ಇಷ್ಟನ್ನೇ
ಇಷ್ಟರು
ಇಷ್ಟ-ವನ್ನು
ಇಷ್ಟ-ವಾದ
ಇಷ್ಟ-ವಿಲ್ಲ
ಇಷ್ಟವೇ
ಇಷ್ಟವೋ
ಇಷ್ಟಾ-ದರೂ
ಇಷ್ಟಾನ್
ಇಷ್ಟು
ಇಷ್ಟೆ
ಇಷ್ಟೇ
ಇಷ್ಟೊಂ-ದನ್ನು
ಇಷ್ಟೊಂದು
ಇಷ್ಟೋಽಸಿ
ಇಹ
ಇಹ-ದಲ್ಲಿ
ಇಹ-ದಲ್ಲೇ
ಇಹ-ಪರ
ಇಹ-ಪ-ರ-ಸೌ-ಖ್ಯ-ಗಳನ್ನು
ಇಹ-ಲೋ-ಕದ
ಇಹ-ಲೋ-ಕ-ದ-ಲ್ಲಾ-ಗಲೀ
ಇಹ-ಲೋ-ಕ-ದಲ್ಲಿ
ಇಹ-ವನ್ನು
ಇಹೈ-ಕಸ್ಥಂ
ಇಹೈವ
ಈ
ಈಕ್ಷತೇ
ಈಗ
ಈಗ-ತಾನೆ
ಈಗ-ತಾನೇ
ಈಗ-ಲಾ-ದರೂ
ಈಗಲೂ
ಈಗಲೇ
ಈಗ-ಲೇನೋ
ಈಗಿದ್ದು
ಈಗಿನ
ಈಗಿ-ನಂ-ತೆಯೇ
ಈಗಿ-ನ-ದನ್ನು
ಈಗಿ-ನ-ದ-ರಲ್ಲಿ
ಈಗಿ-ನಿಂದ
ಈಗಿ-ನಿಂ-ದಲೂ
ಈಗಿ-ನಿಂ-ದಲೆ
ಈಗಿ-ರುವ
ಈಗಿ-ರು-ವುದು
ಈಗಿ-ರು-ವು-ದೆಲ್ಲ
ಈಗೊಂದು
ಈಚಿ-ನದು
ಈಜನ್ನೇ
ಈಜ-ಬೇ-ಕಾ-ಗಿ-ದೆಯೋ
ಈಜಲು
ಈಜಾಡಿ
ಈಜಿ-ಕೊಂಡು
ಈಜು
ಈಜು-ಗಾರ
ಈಜು-ತ್ತಿ-ದ್ದರೆ
ಈಜು-ತ್ತಿ-ದ್ದ-ವ-ರನ್ನು
ಈಜು-ತ್ತಿರು
ಈಜು-ತ್ತೇವೆ
ಈಜು-ವು-ದಕ್ಕೆ
ಈಜು-ವುದು
ಈಡಾ-ಗು-ವನು
ಈಡೇ-ರ-ಬ-ಹುದು
ಈಡೇ-ರ-ಬೇ-ಕಾ-ದರೆ
ಈಡೇ-ರ-ಬೇಕು
ಈಡೇ-ರಿ-ಸಲು
ಈಡೇ-ರಿಸಿ
ಈಡೇ-ರಿ-ಸಿ-ಕೊ-ಳ್ಳ-ಬೇ-ಕಾ-ದರೂ
ಈಡೇ-ರಿ-ಸಿ-ಕೊಳ್ಳಿ
ಈಡೇ-ರಿ-ಸಿ-ಕೊಳ್ಳು
ಈಡೇ-ರಿ-ಸಿ-ಕೊ-ಳ್ಳು-ತ್ತೇನೆ
ಈಡೇ-ರಿ-ಸಿ-ಕೊ-ಳ್ಳು-ವು-ದಕ್ಕೆ
ಈಡೇ-ರಿ-ಸಿ-ಕೊ-ಳ್ಳು-ವುವು
ಈಡೇ-ರಿ-ಸು-ತ್ತಾನೆ
ಈಡೇ-ರಿ-ಸುವ
ಈಡೇ-ರಿ-ಸು-ವನು
ಈಡೇ-ರಿ-ಸು-ವು-ದಕ್ಕೆ
ಈಡೇ-ರಿ-ಸು-ವು-ದ-ರಲ್ಲಿ
ಈಡೇ-ರಿ-ಸು-ವುದು
ಈಡೇ-ರು-ತ್ತವೆ
ಈಡೇ-ರು-ವುದೂ
ಈತ
ಈತ-ನಲ್ಲಿ
ಈತ-ನಿಗೆ
ಈತನು
ಈರು-ಳ್ಳಿ-ಯನ್ನು
ಈವಾಗ
ಈಶ
ಈಶಾ-ವಾಸ್ಯ
ಈಶ್ವರ
ಈಶ್ವರಃ
ಈಶ್ವ-ರ-ತತ್ತ್ವ
ಈಶ್ವ-ರನ
ಈಶ್ವ-ರ-ನಂತೆ
ಈಶ್ವ-ರ-ನನ್ನು
ಈಶ್ವ-ರನು
ಈಶ್ವ-ರನೂ
ಈಶ್ವ-ರ-ಭಾ-ವ-ಇವು
ಈಶ್ವ-ರ-ಸಂ-ಬಂ-ಧ-ವಾದ
ಈಶ್ವರೀ
ಈಶ್ವ-ರೀ-ರೂಪ
ಈಶ್ವ-ರೀ-ರೂ-ಪ-ವನ್ನು
ಈಶ್ವ-ರೀ-ರೂಪು
ಈಶ್ವ-ರೋ-ಽಹ-ಮಹಂ
ಈಹಂತೇ
ಉಂಟಾ-ಗಿಲ್ಲ
ಉಂಟಾಗು
ಉಂಟಾ-ಗು-ತ್ತವೆ
ಉಂಟಾ-ಗುವ
ಉಂಟಾ-ಗು-ವಂತೆ
ಉಂಟಾ-ಗು-ವು-ದಿಲ್ಲ
ಉಂಟಾ-ಗು-ವುದು
ಉಂಟಾ-ಗು-ವುದೊ
ಉಂಟಾ-ಗು-ವುದೋ
ಉಂಟಾ-ಗು-ವುವು
ಉಂಟಾದ
ಉಂಟಾ-ದರೆ
ಉಂಟಾ-ದುದೊ
ಉಂಟಾ-ಯಿತು
ಉಂಟು
ಉಂಟು-ಮಾ-ಡ-ಬಾ-ರದು
ಉಂಟು-ಮಾ-ಡ-ಬೇ-ಕೆಂದು
ಉಂಟು-ಮಾ-ಡ-ಲಾ-ರದು
ಉಂಟು-ಮಾಡಿ
ಉಂಟು-ಮಾ-ಡಿ-ಕೊ-ಳ್ಳ-ಬೇ-ಕೆಂದು
ಉಂಟು-ಮಾ-ಡಿ-ದ-ವನು
ಉಂಟು-ಮಾ-ಡಿದೆ
ಉಂಟು-ಮಾಡು
ಉಂಟು-ಮಾ-ಡುವ
ಉಂಟು-ಮಾ-ಡು-ವು-ದಿಲ್ಲ
ಉಂಟು-ಮಾ-ಡು-ವುದು
ಉಂಟು-ಮಾ-ಡು-ವುದೇ
ಉಂಟು-ಮಾ-ಡು-ವುವು
ಉಂಟೋ
ಉಂಟೋ-ಗು-ವುದು
ಉಂಡಂತೆ
ಉಂಡ-ವನು
ಉಂಡಿದ್ದೇ
ಉಂಡೆಗೆ
ಉಂಡೆಯ
ಉಂಡೆ-ಯನ್ನು
ಉಂಡೆ-ಯಾಗಿ
ಉಕ್ಕಿ
ಉಕ್ಕಿ-ಗಿಂತ
ಉಕ್ಕಿ-ಬ-ರು-ತ್ತಿದೆ
ಉಕ್ಕು-ತ್ತಿ-ರು-ವಾಗ
ಉಕ್ಕು-ವುದು
ಉಗಮ
ಉಗು-ರಿ-ನಲ್ಲಿ
ಉಗು-ರಿ-ನಿಂದ
ಉಗುಳಿ
ಉಗುಳು
ಉಗು-ಳು-ತ್ತೇನೆ
ಉಗು-ಳು-ವು-ದಕ್ಕೆ
ಉಗುಳೇ
ಉಗ್ರ
ಉಗ್ರ-ತೆಯ
ಉಗ್ರ-ರೂ-ಪದ
ಉಗ್ರ-ರೂ-ಪ-ವನ್ನು
ಉಗ್ರ-ರೂಪಿ
ಉಗ್ರ-ರೂ-ಪಿನ
ಉಗ್ರ-ವಾಗಿ
ಉಗ್ರ-ವಾ-ಗಿದೆ
ಉಗ್ರ-ವಾ-ಗು-ವುದು
ಉಗ್ರ-ವಾದ
ಉಗ್ರ-ವಾ-ದದು
ಉಗ್ರವೂ
ಉಗ್ರ-ಸೇನ
ಉಗ್ರ-ಸೇ-ನನ್ನು
ಉಗ್ರ-ಸ್ವ-ರೂ-ಪದ
ಉಗ್ರಾ-ಣದ
ಉಗ್ರಾ-ಣ-ದಲ್ಲಿ
ಉಗ್ರಾ-ಣ-ದ-ಲ್ಲಿದೆ
ಉಗ್ರಾ-ಣ-ದ-ಲ್ಲಿ-ರು-ತ್ತವೆ
ಉಗ್ರಾ-ಣ-ದ-ಲ್ಲಿ-ರುವ
ಉಗ್ರಾ-ಣ-ದ-ಲ್ಲಿ-ರು-ವುದು
ಉಗ್ರಾ-ಣ-ದ-ಲ್ಲಿ-ರು-ವುವು
ಉಗ್ರಾ-ಣ-ದ-ಲ್ಲಿವೆ
ಉಗ್ರಾ-ಣ-ದಿಂದ
ಉಚಿತ
ಉಚಿ-ತ-ವಾಗಿ
ಉಚಿ-ತ-ವಾದ
ಉಚ್ಚ
ಉಚ್ಚ-ರಿಸಿ
ಉಚ್ಚ-ರಿ-ಸುತ್ತ
ಉಚ್ಚ-ರಿ-ಸುತ್ತಾ
ಉಚ್ಚ-ರಿ-ಸು-ತ್ತಿ-ರು-ವನು
ಉಚ್ಚ-ರಿ-ಸುವ
ಉಚ್ಚಾರ
ಉಚ್ಚಾ-ರಣೆ
ಉಚ್ಚೈಃ-ಶ್ರ-ವ-ಸ-ಮ-ಶ್ವಾ-ನಾಂ
ಉಚ್ಚೈ-ಶ್ರವ
ಉಚ್ಚೈ-ಶ್ರ-ವಸ್ಸು
ಉಚ್ಛಿ-ಷ್ಟ-ಮಪಿ
ಉಚ್ಛ್ವಾಸ
ಉಚ್ಯತೇ
ಉಜ್ಜೀ-ವನ
ಉಜ್ವಲ
ಉಜ್ವ-ಲ-ವಾ-ದುದು
ಉಟ್ಟ
ಉಟ್ಟು-ಕೊ-ಳ್ಳ-ಬೇಕು
ಉಡು-ಗಿದೆ
ಉಡು-ಗೆಯೂ
ಉಡು-ಪಿ-ನ-ಲ್ಲಿ-ದ್ದಾನೆ
ಉಡುಪು
ಉಡುವ
ಉಡು-ವುದು
ಉಣಿಸು
ಉಣ್ಣ-ಬೇಕು
ಉಣ್ಣು-ತ್ತಾರೆ
ಉಣ್ಣು-ವ-ವ-ನಿಗೆ
ಉಣ್ಣು-ವು-ದಕ್ಕೆ
ಉತ್ಕ-ಟ-ವಾಗಿ
ಉತ್ಕ-ರ್ಷೆ-ಯನ್ನು
ಉತ್ಕೃಷ್ಟ
ಉತ್ಕೃ-ಷ್ಟ-ನಾದ
ಉತ್ಕೃ-ಷ್ಟ-ವಾದ
ಉತ್ಕ್ರಾ-ಮಂತಂ
ಉತ್ತ
ಉತ್ತ-ಬೇಕು
ಉತ್ತಮ
ಉತ್ತಮಃ
ಉತ್ತ-ಮ-ಗ-ತಿಯೂ
ಉತ್ತ-ಮ-ನಾ-ಗಲು
ಉತ್ತ-ಮನು
ಉತ್ತ-ಮರೇ
ಉತ್ತ-ಮ-ವಾ-ಗ-ಬ-ಹುದು
ಉತ್ತ-ಮ-ವಾ-ಗಿ-ರು-ವುದನ್ನು
ಉತ್ತ-ಮ-ವಾದ
ಉತ್ತ-ಮ-ವಾ-ದುದು
ಉತ್ತ-ಮವೂ
ಉತ್ತ-ಮವೊ
ಉತ್ತ-ಮ-ಸ್ಥಿ-ತಿ-ಯ-ಲ್ಲಿ-ರು-ವಾಗ
ಉತ್ತ-ಮೌ-ಜಸ್
ಉತ್ತ-ಮೌ-ಜ-ಸ್ಸು-ಇ-ವರು
ಉತ್ತ-ಮೌ-ಜಾಶ್ಚ
ಉತ್ತರ
ಉತ್ತ-ರ-ಕಾಶಿ
ಉತ್ತ-ರಕ್ಕೆ
ಉತ್ತ-ರ-ಗಳನ್ನು
ಉತ್ತ-ರ-ದಿ-ಕ್ಕನ್ನು
ಉತ್ತ-ರ-ದಿ-ಕ್ಕನ್ನೇ
ಉತ್ತ-ರ-ದಿ-ಕ್ಕಿನ
ಉತ್ತ-ರ-ಮುಖಿ
ಉತ್ತ-ರ-ಮು-ಖಿ-ಯನ್ನು
ಉತ್ತ-ರ-ವನ್ನು
ಉತ್ತ-ರ-ವನ್ನೇ
ಉತ್ತ-ರ-ವಾ-ಗಿಯೇ
ಉತ್ತ-ರವೂ
ಉತ್ತ-ರ-ವೆಲ್ಲಾ
ಉತ್ತ-ರಾ-ಧಿ-ಕಾ-ರಿ-ಗಳನ್ನು
ಉತ್ತ-ರಾ-ಯ-ಣದ
ಉತ್ತ-ರಾ-ಯ-ಣಮ್
ಉತ್ತ-ರಾ-ಯನ
ಉತ್ತ-ರಿ-ಸು-ತ್ತಾನೆ
ಉತ್ತ-ರಿ-ಸು-ತ್ತಾ-ನೆಯೆ
ಉತ್ತಿ
ಉತ್ತಿದ
ಉತ್ತಿದೆ
ಉತ್ತಿ-ದ್ದರೆ
ಉತ್ತಿ-ರು-ವನು
ಉತ್ತಿ-ರು-ವರೋ
ಉತ್ತಿ-ಲ್ಲವೋ
ಉತ್ತಿಷ್ಠ
ಉತ್ತು
ಉತ್ತು-ವುದು
ಉತ್ಪತ್ತಿ
ಉತ್ಪ-ತ್ತಿ-ಸ್ಥಿ-ತಿಗೆ
ಉತ್ಪ-ತ್ತಿಗೆ
ಉತ್ಪ-ತ್ತಿ-ಮಾ-ಡಿ-ಕೊಂ-ಡರೆ
ಉತ್ಪ-ತ್ತಿ-ಮಾ-ಡುವ
ಉತ್ಪ-ತ್ತಿ-ಮಾ-ಡು-ವಾಗ
ಉತ್ಪ-ತ್ತಿ-ಮಾ-ಡು-ವು-ದ-ರಲ್ಲಿ
ಉತ್ಪ-ತ್ತಿ-ಯಾಗ
ಉತ್ಪ-ತ್ತಿ-ಯಾ-ಗ-ಬೇ-ಕಾ-ದರೆ
ಉತ್ಪ-ತ್ತಿ-ಯಾ-ಗು-ತ್ತವೆ
ಉತ್ಪ-ತ್ತಿ-ಯಾ-ಗುವ
ಉತ್ಪ-ತ್ತಿ-ಯಾ-ಗು-ವು-ದಕ್ಕೆ
ಉತ್ಪ-ತ್ತಿ-ಯಾ-ಗು-ವು-ದಿಲ್ಲ
ಉತ್ಪ-ತ್ತಿ-ಯಾ-ಗು-ವುದು
ಉತ್ಪ-ತ್ತಿ-ಯಾ-ಗು-ವುವು
ಉತ್ಪ-ತ್ತಿ-ಯಾದ
ಉತ್ಪ-ತ್ತಿಯೂ
ಉತ್ಪ-ನ್ನ-ರಾ-ದರು
ಉತ್ಪ-ನ್ನ-ರಾ-ದ-ವರು
ಉತ್ಪ-ನ್ನ-ವಾ-ಗು-ತ್ತವೆ
ಉತ್ಪ-ನ್ನ-ವಾ-ಗು-ವುದೊ
ಉತ್ಪ-ನ್ನ-ವಾದ
ಉತ್ಪಾ-ದನೆ
ಉತ್ಪಾ-ದ-ನೆ-ಯನ್ನು
ಉತ್ಪ್ರೇಕ್ಷೆ
ಉತ್ಪ್ರೇ-ಕ್ಷೆಯೂ
ಉತ್ಸನ್ನ
ಉತ್ಸವ
ಉತ್ಸ-ವ-ಗಳನ್ನು
ಉತ್ಸಾ-ದ್ಯಂತೇ
ಉತ್ಸಾಹ
ಉತ್ಸಾ-ಹ-ಗಳಿಂದ
ಉತ್ಸಾ-ಹ-ದಿಂದ
ಉತ್ಸಾ-ಹ-ಪೂ-ರಿ-ತ-ವಾದ
ಉತ್ಸಾ-ಹ-ಭ-ರಿ-ತ-ನಾ-ಗಿದ್ದ
ಉತ್ಸಾ-ಹ-ವನ್ನು
ಉತ್ಸಾ-ಹ-ವಿಲ್ಲ
ಉತ್ಸಾ-ಹ-ವೊಂ-ದ-ರಿಂ-ದಲೇ
ಉತ್ಸೀ-ದೇ-ಯು-ರಿಮೇ
ಉದಂ-ಕ-ನೆಂಬ
ಉದ-ಪಾನೇ
ಉದಯ
ಉದ-ಯ-ವನ್ನು
ಉದ-ಯಿಸಿ
ಉದ-ಹ-ರಿ-ಸ-ಬ-ಹುದು
ಉದ-ಹ-ರಿಸಿ
ಉದ-ಹ-ರಿ-ಸಿ-ದ್ದೇನೆ
ಉದ-ಹ-ರಿ-ಸು-ತ್ತಲೂ
ಉದ-ಹ-ರಿ-ಸು-ತ್ತಿ-ರು-ವ-ವನು
ಉದ-ಹ-ರಿ-ಸು-ವರು
ಉದ-ಹ-ರಿ-ಸು-ವು-ದ-ಕ್ಕಾ-ಗಿವೆ
ಉದಾ-ತ್ತ-ವಾದ
ಉದಾನ
ಉದಾರ
ಉದಾ-ರ-ವಾಗಿ
ಉದಾ-ರ-ವಾ-ಗಿ-ರ-ಬೇಕು
ಉದಾ-ರಾಃ
ಉದಾ-ರಿ-ಗಳು
ಉದಾ-ರಿ-ಗಳೇ
ಉದಾ-ಸೀನ
ಉದಾ-ಸೀ-ನ-ತೆಗೆ
ಉದಾ-ಸೀ-ನ-ತೆ-ಯಲ್ಲಿ
ಉದಾ-ಸೀ-ನ-ದಿಂದ
ಉದಾ-ಸೀ-ನ-ನಂತೆ
ಉದಾ-ಸೀ-ನ-ನಾಗಿ
ಉದಾ-ಸೀ-ನ-ನಾ-ಗಿ-ರ-ಬ-ಹು-ದಲ್ಲ
ಉದಾ-ಸೀ-ನ-ನಾ-ಗಿ-ರು-ತ್ತಾನೆ
ಉದಾ-ಸೀ-ನ-ನಾ-ಗು-ತ್ತಾನೆ
ಉದಾ-ಸೀ-ನ-ನಾ-ಗು-ವನು
ಉದಾ-ಸೀ-ನ-ನಾ-ಗುವೆ
ಉದಾ-ಸೀ-ನನೋ
ಉದಾ-ಸೀ-ನ-ಭಾ-ವ-ದಿಂದ
ಉದಾ-ಸೀ-ನ-ರಾಗಿ
ಉದಾ-ಸೀ-ನ-ರಾ-ಗಿ-ರ-ಬೇಕು
ಉದಾ-ಸೀ-ನ-ರಾ-ಗು-ವೆವೊ
ಉದಾ-ಸೀ-ನ-ವ-ದಾ-ಸೀ-ನ-ಮ-ಸಕ್ತಂ
ಉದಾ-ಸೀ-ನ-ವ-ದಾ-ಸೀನೋ
ಉದಾ-ಸೀ-ನ-ವಾಗಿ
ಉದಾ-ಸೀನೋ
ಉದಾ-ಹ-ರಣೆ
ಉದಾ-ಹ-ರ-ಣೆ-ಗಳ
ಉದಾ-ಹ-ರ-ಣೆ-ಗಳನ್ನು
ಉದಾ-ಹ-ರ-ಣೆ-ಗ-ಳೆಲ್ಲಾ
ಉದಾ-ಹ-ರ-ಣೆ-ಗಾಗಿ
ಉದಾ-ಹ-ರ-ಣೆಯ
ಉದಾ-ಹ-ರ-ಣೆ-ಯನ್ನು
ಉದಾ-ಹ-ರ-ಣೆ-ಯಾಗಿ
ಉದಾ-ಹ-ರ-ಣೆ-ಯಾ-ಗಿ-ರ-ಬೇಕು
ಉದಾ-ಹ-ರ-ಣೆ-ಯಿಂದ
ಉದಾ-ಹ-ರಿ-ಸದೆ
ಉದಾ-ಹ-ರಿ-ಸು-ವರು
ಉದಿ-ಸದೇ
ಉದಿ-ಸ-ಬ-ಹುದೋ
ಉದಿಸಿ
ಉದಿ-ಸಿ-ದರೆ
ಉದಿ-ಸು-ತ್ತಿ-ರಲೇ
ಉದಿ-ಸು-ವು-ದಿಲ್ಲ
ಉದಿ-ಸು-ವುದು
ಉದುರಿ
ಉದು-ರಿ-ಹೋಗಿ
ಉದು-ರು-ವು-ದಿಲ್ಲ
ಉದು-ರು-ವುದು
ಉದ್ಗಾರ
ಉದ್ದಕ್ಕೂ
ಉದ್ದ-ವಾ-ಗಿದೆ
ಉದ್ದ-ವಾ-ಗಿ-ದೆಯೊ
ಉದ್ದ-ವಾಗು
ಉದ್ದ-ವಾ-ಗು-ವುದು
ಉದ್ದ-ವಿ-ರುವ
ಉದ್ದೀ-ಪ-ನ-ಗೊ-ಳಿ-ಸ-ಬೇಕು
ಉದ್ದೀ-ಪನೆ
ಉದ್ದೇಶ
ಉದ್ದೇ-ಶ-ಕ್ಕಾಗಿ
ಉದ್ದೇ-ಶಕ್ಕೆ
ಉದ್ದೇ-ಶ-ಗಳನ್ನು
ಉದ್ದೇ-ಶ-ಗಳಿಂದ
ಉದ್ದೇ-ಶ-ಗ-ಳಿಗೆ
ಉದ್ದೇ-ಶ-ಗಳು
ಉದ್ದೇ-ಶ-ಗ-ಳೆಲ್ಲ
ಉದ್ದೇ-ಶ-ಗ-ಳೇನು
ಉದ್ದೇ-ಶ-ದಿಂದ
ಉದ್ದೇ-ಶ-ದಿಂ-ದಲೆ
ಉದ್ದೇ-ಶ-ವ-ನ್ನಲ್ಲ
ಉದ್ದೇ-ಶ-ವ-ನ್ನಿ-ಟ್ಟು-ಕೊಂ-ಡ-ವ-ನೆಂ-ದರೆ
ಉದ್ದೇ-ಶ-ವನ್ನು
ಉದ್ದೇ-ಶ-ವಾ-ಗಿ-ಟ್ಟು-ಕೊಂಡು
ಉದ್ದೇ-ಶ-ವಾ-ದರೆ
ಉದ್ದೇ-ಶ-ವಿದೆ
ಉದ್ದೇ-ಶ-ವಿಲ್ಲ
ಉದ್ದೇ-ಶವೂ
ಉದ್ದೇ-ಶ-ವೆಲ್ಲ
ಉದ್ದೇ-ಶವೇ
ಉದ್ದೇ-ಶಿಸಿ
ಉದ್ಧ-ರಣೆ
ಉದ್ಧ-ರ-ಣೆ-ಯನ್ನೊ
ಉದ್ಧ-ರಿ-ಸಲಿ
ಉದ್ಧ-ರಿಸು
ಉದ್ಧ-ರಿ-ಸು-ತ್ತಾನೆ
ಉದ್ಧ-ರಿ-ಸುವ
ಉದ್ಧ-ರಿ-ಸು-ವ-ದಕ್ಕೆ
ಉದ್ಧ-ರಿ-ಸು-ವನು
ಉದ್ಧ-ರಿ-ಸು-ವ-ವನು
ಉದ್ಧ-ರಿ-ಸು-ವು-ದಕ್ಕೆ
ಉದ್ಧ-ರಿ-ಸು-ವುದು
ಉದ್ಧ-ರೇ-ದಾ-ತ್ಮ-ನಾ-ತ್ಮಾನಂ
ಉದ್ಧಾರ
ಉದ್ಧಾ-ರ-ಕ್ಕಾಗಿ
ಉದ್ಧಾ-ರಕ್ಕೆ
ಉದ್ಧಾ-ರದ
ಉದ್ಧಾ-ರ-ಮಾ-ಡ-ಬೇಕು
ಉದ್ಧಾ-ರ-ಮಾ-ಡ-ಲಾ-ರದು
ಉದ್ಧಾ-ರ-ಮಾ-ಡಲು
ಉದ್ಧಾ-ರ-ಮಾ-ಡಿ-ಕೊ-ಳ್ಳ-ಬೇಕು
ಉದ್ಧಾ-ರ-ಮಾ-ಡಿ-ಕೊ-ಳ್ಳುವ
ಉದ್ಧಾ-ರ-ಮಾ-ಡು-ತ್ತೇನೆ
ಉದ್ಧಾ-ರ-ಮಾ-ಡು-ವು-ದ-ಕ್ಕಾಗಿ
ಉದ್ಧಾ-ರ-ಮಾ-ಡು-ವುದು
ಉದ್ಧಾ-ರ-ಮಾ-ಡು-ವುವು
ಉದ್ಧಾ-ರ-ವಾ-ಗ-ಬ-ಹುದು
ಉದ್ಧಾ-ರ-ವಾ-ಗ-ಬೇ-ಕೆಂಬ
ಉದ್ಧಾ-ರ-ವಾ-ಗಲಿ
ಉದ್ಧಾ-ರ-ವಾ-ಗಲು
ಉದ್ಧಾ-ರ-ವಾಗಿ
ಉದ್ಧಾ-ರ-ವಾ-ಗಿ-ದ್ದಾರೆ
ಉದ್ಧಾ-ರ-ವಾ-ಗಿರು
ಉದ್ಧಾ-ರ-ವಾಗು
ಉದ್ಧಾ-ರ-ವಾ-ಗು-ತ್ತಾನೆ
ಉದ್ಧಾ-ರ-ವಾ-ಗು-ತ್ತಾರೆ
ಉದ್ಧಾ-ರ-ವಾ-ಗು-ತ್ತಾರೊ
ಉದ್ಧಾ-ರ-ವಾ-ಗು-ತ್ತೇನೆ
ಉದ್ಧಾ-ರ-ವಾ-ಗು-ತ್ತೇವೆ
ಉದ್ಧಾ-ರ-ವಾ-ಗುವ
ಉದ್ಧಾ-ರ-ವಾ-ಗು-ವಂ-ತೆಯೇ
ಉದ್ಧಾ-ರ-ವಾ-ಗು-ವನು
ಉದ್ಧಾ-ರ-ವಾ-ಗು-ವರು
ಉದ್ಧಾ-ರ-ವಾ-ಗು-ವರೆ
ಉದ್ಧಾ-ರ-ವಾ-ಗು-ವ-ವನು
ಉದ್ಧಾ-ರ-ವಾ-ಗು-ವ-ವರು
ಉದ್ಧಾ-ರ-ವಾ-ಗು-ವು-ದ-ಕ್ಕಲ್ಲ
ಉದ್ಧಾ-ರ-ವಾ-ಗು-ವು-ದಕ್ಕೆ
ಉದ್ಧಾ-ರ-ವಾ-ಗು-ವು-ದ-ರಲ್ಲಿ
ಉದ್ಧಾ-ರ-ವಾ-ಗು-ವುದು
ಉದ್ಧಾ-ರ-ವಾ-ದೆವು
ಉದ್ಭ-ವಿ-ಸಿ-ದಾಗ
ಉದ್ಭ-ವಿ-ಸು-ವುದು
ಉದ್ಭ-ವಿ-ಸು-ವುವು
ಉದ್ಯಮ
ಉದ್ಯ-ಮಕ್ಕೆ
ಉದ್ಯ-ಮ-ಗಳಲ್ಲಿ
ಉದ್ಯ-ಮ-ಗ-ಳಿಗೆ
ಉದ್ಯುಕ್ತ
ಉದ್ಯು-ಕ್ತ-ನಾ-ಗಿರು
ಉದ್ಯು-ಕ್ತ-ನಾ-ಗಿ-ರು-ವನು
ಉದ್ಯು-ಕ್ತ-ರಾ-ಗಿ-ರು-ವಾಗ
ಉದ್ಯೋಗ
ಉದ್ರೇ-ಕಿ-ಸು-ವಂ-ತಹ
ಉದ್ವಿ-ಗ್ನತೆ
ಉದ್ವಿ-ಗ್ನ-ತೆಗೂ
ಉದ್ವಿ-ಗ್ನ-ತೆಯ
ಉದ್ವಿ-ಗ್ನ-ತೆಯೂ
ಉದ್ವಿ-ಗ್ನ-ನಾಗು
ಉದ್ವಿ-ಗ್ನ-ನಾ-ಗು-ವು-ದಿಲ್ಲ
ಉದ್ವಿ-ಗ್ನ-ವಾಗಿ
ಉದ್ವೇಗ
ಉದ್ವೇ-ಗ-ಕ್ಕಾ-ದರೂ
ಉದ್ವೇ-ಗಕ್ಕೂ
ಉದ್ವೇ-ಗಕ್ಕೆ
ಉದ್ವೇ-ಗ-ಕ್ಕೆಲ್ಲಾ
ಉದ್ವೇ-ಗ-ಗ-ಳಿ-ಲ್ಲವೋ
ಉದ್ವೇ-ಗ-ಗೊ-ಳ್ಳ-ದ-ವನು
ಉದ್ವೇ-ಗ-ದಿಂದ
ಉದ್ವೇ-ಗ-ಪರ
ಉದ್ವೇ-ಗ-ಪ-ರ-ವ-ಶ-ನಾಗಿ
ಉದ್ವೇ-ಗ-ವನ್ನು
ಉದ್ವೇ-ಗ-ವನ್ನೂ
ಉದ್ವೇ-ಗ-ವ-ಶ-ನಾ-ಗದೆ
ಉದ್ವೇ-ಗ-ವ-ಶ-ರಾ-ದಾಗ
ಉದ್ವೇ-ಗ-ವಾ-ಗಲಿ
ಉದ್ವೇ-ಗ-ವಿ-ರ-ಕೂ-ಡದು
ಉದ್ವೇ-ಗ-ವಿ-ರು-ವು-ದಿಲ್ಲ
ಉದ್ವೇ-ಗ-ವಿಲ್ಲ
ಉದ್ವೇ-ಗ-ವಿ-ಲ್ಲದೆ
ಉದ್ವೇ-ಗವೂ
ಉದ್ವೇ-ಗಾ-ತಿ-ಶ-ಯ-ವಿ-ರು-ವುದೊ
ಉನ್ನತ
ಉನ್ನತಿ
ಉನ್ಮ-ತ್ತ-ನಾಗಿ
ಉನ್ಮ-ತ್ತ-ರಾ-ಗಿರು
ಉನ್ಮ-ತ್ತ-ರಾ-ಗಿ-ರು-ವ-ವ-ರಲ್ಲಿ
ಉನ್ಮ-ತ್ತಾ-ರಾ-ಗಿ-ರು-ತ್ತೇವೆ
ಉಪ
ಉಪ-ಕ-ಥೆ-ಗಳು
ಉಪ-ಕ-ಥೆ-ಗ-ಳೆಲ್ಲ
ಉಪ-ಕ-ರ-ಣ-ಗಳನ್ನೆಲ್ಲಾ
ಉಪ-ಕಾರ
ಉಪ-ಕಾ-ರ-ವನ್ನು
ಉಪ-ಕಾ-ರಾ-ತ್ಮ-ಕ-ವಾ-ಗಿ-ರುವ
ಉಪ-ಕ್ರ-ಮಿ-ಸಿ-ದರೋ
ಉಪ-ಕ್ರ-ಮಿ-ಸು-ವನು
ಉಪ-ಗ್ರ-ಹ-ದಂತೆ
ಉಪ-ಚಾರ
ಉಪ-ಚಾ-ರ-ಗಳು
ಉಪ-ಚಾ-ರದ
ಉಪ-ಜಾ-ಯತೇ
ಉಪ-ಟ-ಳಕ್ಕೆ
ಉಪ-ಟ-ಳದ
ಉಪ-ಟ-ಳ-ದಿಂದ
ಉಪ-ದೇ-ಕ್ಷ್ಯಂತಿ
ಉಪ-ದೇಶ
ಉಪ-ದೇ-ಶ-ಗಳನ್ನು
ಉಪ-ದೇ-ಶ-ದಂತೆ
ಉಪ-ದೇ-ಶ-ಮಾ-ಡಿದೆ
ಉಪ-ದೇ-ಶವೇ
ಉಪ-ದೇ-ಶಿ-ಸಿದ
ಉಪ-ದೇ-ಶಿ-ಸು-ತ್ತೇವೆ
ಉಪ-ದ್ರ-ವ-ಗ-ಳಿವೆ
ಉಪ-ದ್ರ-ಷ್ಟಾ-ಽನು-ಮಂತಾ
ಉಪ-ನದಿ
ಉಪ-ನ-ದಿ-ಗಳನ್ನು
ಉಪ-ನ-ಯನ
ಉಪ-ನಿ-ಷ-ತ್ಗಳು
ಉಪ-ನಿ-ಷತ್ತಿ
ಉಪ-ನಿ-ಷ-ತ್ತಿಗೆ
ಉಪ-ನಿ-ಷ-ತ್ತಿನ
ಉಪ-ನಿ-ಷ-ತ್ತಿ-ನಲ್ಲಿ
ಉಪ-ನಿ-ಷ-ತ್ತಿ-ನ-ಲ್ಲಿದೆ
ಉಪ-ನಿ-ಷ-ತ್ತಿ-ನಿಂದ
ಉಪ-ನಿ-ಷತ್ತು
ಉಪ-ನಿ-ಷ-ತ್ತು-ಗಳ
ಉಪ-ನಿ-ಷ-ತ್ತು-ಗಳನ್ನು
ಉಪ-ನಿ-ಷ-ತ್ತು-ಗಳಲ್ಲಿ
ಉಪ-ನಿ-ಷ-ತ್ತು-ಗಳಿಂದ
ಉಪ-ನಿ-ಷ-ತ್ತು-ಗಳು
ಉಪ-ನಿ-ಷ-ತ್ತು-ಗಳೇ
ಉಪ-ನಿ-ಷ-ತ್ತೆಂಬ
ಉಪ-ನಿ-ಷತ್ಸು
ಉಪ-ನ್ಯಾಸ
ಉಪ-ನ್ಯಾ-ಸ-ದಂತೆ
ಉಪ-ನ್ಯಾ-ಸ-ವನ್ನು
ಉಪ-ಪಾಂ-ಡ-ವರು
ಉಪ-ಮಾನ
ಉಪ-ಮಾ-ನ-ಗಳು
ಉಪ-ಮಾ-ನದ
ಉಪ-ಮಾ-ನ-ವನ್ನು
ಉಪ-ಮಾ-ನ-ವೆ-ನಿ-ಸು-ವುದು
ಉಪ-ಯಾ-ಗಿ-ಸ-ಬಾ-ರದು
ಉಪ-ಯು-ಕ್ತ-ವಾದ
ಉಪ-ಯೋಗ
ಉಪ-ಯೋ-ಗಕ್ಕೆ
ಉಪ-ಯೋ-ಗ-ಪ-ಡಿ-ಸಿ-ಕೊ-ಳ್ಳು-ವುದು
ಉಪ-ಯೋ-ಗ-ವನ್ನು
ಉಪ-ಯೋ-ಗ-ವಾ-ಗು-ವುದು
ಉಪ-ಯೋ-ಗ-ವಿ-ಲ್ಲದು
ಉಪ-ಯೋಗಿ
ಉಪ-ಯೋ-ಗಿಸ
ಉಪ-ಯೋ-ಗಿ-ಸದೆ
ಉಪ-ಯೋ-ಗಿ-ಸ-ಬ-ಹುದು
ಉಪ-ಯೋ-ಗಿ-ಸ-ಬೇಕು
ಉಪ-ಯೋ-ಗಿ-ಸ-ಬೇ-ಕೆಂ-ದಿ-ರು-ವನು
ಉಪ-ಯೋ-ಗಿ-ಸಲಿ
ಉಪ-ಯೋ-ಗಿ-ಸಲು
ಉಪ-ಯೋ-ಗಿಸಿ
ಉಪ-ಯೋ-ಗಿ-ಸಿ-ಕೊಂ-ಡರೆ
ಉಪ-ಯೋ-ಗಿ-ಸಿ-ಕೊಂ-ಡಿರು
ಉಪ-ಯೋ-ಗಿ-ಸಿ-ಕೊಂಡು
ಉಪ-ಯೋ-ಗಿ-ಸಿ-ಕೊ-ಳ್ಳ-ಬ-ಹುದು
ಉಪ-ಯೋ-ಗಿ-ಸಿ-ಕೊ-ಳ್ಳ-ಬೇ-ಕೆಂ-ಬುದು
ಉಪ-ಯೋ-ಗಿ-ಸಿ-ಕೊಳ್ಳು
ಉಪ-ಯೋ-ಗಿ-ಸಿ-ಕೊ-ಳ್ಳು-ತ್ತಾನೆ
ಉಪ-ಯೋ-ಗಿ-ಸಿ-ಕೊ-ಳ್ಳು-ತ್ತಿದೆ
ಉಪ-ಯೋ-ಗಿ-ಸಿ-ಕೊ-ಳ್ಳು-ತ್ತಿ-ರು-ವನು
ಉಪ-ಯೋ-ಗಿ-ಸಿ-ಕೊ-ಳ್ಳು-ತ್ತೇವೆ
ಉಪ-ಯೋ-ಗಿ-ಸಿ-ಕೊ-ಳ್ಳು-ವನು
ಉಪ-ಯೋ-ಗಿ-ಸಿ-ಕೊ-ಳ್ಳು-ವರು
ಉಪ-ಯೋ-ಗಿ-ಸಿ-ಕೊ-ಳ್ಳು-ವ-ವರು
ಉಪ-ಯೋ-ಗಿ-ಸಿ-ಕೊ-ಳ್ಳು-ವು-ದಲ್ಲ
ಉಪ-ಯೋ-ಗಿ-ಸಿ-ಕೊ-ಳ್ಳು-ವುದು
ಉಪ-ಯೋ-ಗಿ-ಸಿ-ಕೊ-ಳ್ಳು-ವೆವೊ
ಉಪ-ಯೋ-ಗಿ-ಸಿದ
ಉಪ-ಯೋ-ಗಿ-ಸಿ-ದರೆ
ಉಪ-ಯೋ-ಗಿ-ಸಿ-ರು-ವನು
ಉಪ-ಯೋ-ಗಿ-ಸಿ-ರು-ವೆವು
ಉಪ-ಯೋ-ಗಿಸು
ಉಪ-ಯೋ-ಗಿ-ಸುತ್ತ
ಉಪ-ಯೋ-ಗಿ-ಸುತ್ತಾ
ಉಪ-ಯೋ-ಗಿ-ಸು-ತ್ತಾನೆ
ಉಪ-ಯೋ-ಗಿ-ಸು-ತ್ತಾರೆ
ಉಪ-ಯೋ-ಗಿ-ಸು-ತ್ತಿದ್ದ
ಉಪ-ಯೋ-ಗಿ-ಸು-ತ್ತಿ-ರು-ವನು
ಉಪ-ಯೋ-ಗಿ-ಸು-ತ್ತೇ-ನೆಯೋ
ಉಪ-ಯೋ-ಗಿ-ಸು-ತ್ತೇವೆ
ಉಪ-ಯೋ-ಗಿ-ಸು-ತ್ತೇ-ವೆಯೊ
ಉಪ-ಯೋ-ಗಿ-ಸುವ
ಉಪ-ಯೋ-ಗಿ-ಸು-ವನು
ಉಪ-ಯೋ-ಗಿ-ಸು-ವನೇ
ಉಪ-ಯೋ-ಗಿ-ಸು-ವರು
ಉಪ-ಯೋ-ಗಿ-ಸು-ವ-ವನ
ಉಪ-ಯೋ-ಗಿ-ಸು-ವ-ವನು
ಉಪ-ಯೋ-ಗಿ-ಸು-ವ-ವನೇ
ಉಪ-ಯೋ-ಗಿ-ಸು-ವ-ವರು
ಉಪ-ಯೋ-ಗಿ-ಸು-ವಾಗ
ಉಪ-ಯೋ-ಗಿ-ಸು-ವು-ದಿಲ್ಲ
ಉಪ-ಯೋ-ಗಿ-ಸು-ವು-ದಿ-ಲ್ಲವೊ
ಉಪ-ಯೋ-ಗಿ-ಸು-ವುದು
ಉಪ-ಯೋ-ಗಿ-ಸು-ವೆವೊ
ಉಪ-ವಾಸ
ಉಪ-ವಾ-ಸ-ದಿಂದ
ಉಪ-ವಾ-ಸ-ವಿ-ರು-ತ್ತಾನೆ
ಉಪ-ವಿ-ಶ್ಯಾ-ಸನೇ
ಉಪ-ಸಂ-ಹಾರ
ಉಪಸ್ಥ
ಉಪಾ-ದಾನ
ಉಪಾಧಿ
ಉಪಾ-ಧಿ-ಗಳ
ಉಪಾ-ಧಿ-ಗಳನ್ನು
ಉಪಾ-ಧಿ-ಗಳನ್ನೆಲ್ಲ
ಉಪಾ-ಧಿ-ಗಳಲ್ಲಿ
ಉಪಾ-ಧಿ-ಗಳಿಂದ
ಉಪಾ-ಧಿ-ಗಳು
ಉಪಾ-ಧಿ-ಗ-ಳೆಲ್ಲ
ಉಪಾ-ಧಿಗೆ
ಉಪಾ-ಧಿಯ
ಉಪಾ-ಧಿ-ಯನ್ನು
ಉಪಾ-ಧಿ-ಯಲ್ಲಿ
ಉಪಾ-ಧಿ-ಯಾದ
ಉಪಾ-ಧಿ-ಯೆಲ್ಲ
ಉಪಾ-ಧ್ಯಾಯ
ಉಪಾ-ಧ್ಯಾ-ಯನ
ಉಪಾ-ಧ್ಯಾ-ಯ-ನಾ-ಗದೆ
ಉಪಾ-ಧ್ಯಾ-ಯ-ನಾಗಿ
ಉಪಾ-ಧ್ಯಾ-ಯ-ನಾ-ಗು-ತ್ತಾನೆ
ಉಪಾ-ಧ್ಯಾ-ಯ-ನಾ-ಗುವ
ಉಪಾ-ಧ್ಯಾ-ಯನು
ಉಪಾ-ಧ್ಯಾ-ಯ-ರಿಗೆ
ಉಪಾಯ
ಉಪಾ-ಯ-ಗಳನ್ನು
ಉಪಾ-ಯ-ಗ-ಳಿ-ವೆಯೆ
ಉಪಾ-ಯ-ಗಳು
ಉಪಾ-ಯದ
ಉಪಾ-ಯ-ದಿಂದ
ಉಪಾ-ಯ-ದಿಂ-ದಲೇ
ಉಪಾ-ಯ-ವನ್ನು
ಉಪಾ-ಯ-ವನ್ನೂ
ಉಪಾ-ವಿ-ಶತ್
ಉಪಾ-ಸ-ಕನೂ
ಉಪಾ-ಸ-ಕರು
ಉಪಾ-ಸತೇ
ಉಪಾ-ಸನೆ
ಉಪಾ-ಸ-ನೆ-ಗ-ಳೆ-ರ-ಡಕ್ಕೂ
ಉಪಾ-ಸ-ನೆಗೆ
ಉಪಾ-ಸ-ನೆಯ
ಉಪಾ-ಸ-ನೆ-ಯನ್ನು
ಉಪಾ-ಸ-ನೆ-ಯಲ್ಲಿ
ಉಪಾ-ಸ-ನೆಯೂ
ಉಪಾ-ಸ-ನೆಯೇ
ಉಪೇಕ್ಷೆ
ಉಪೈತಿ
ಉಪ್ಪನ್ನೇ
ಉಪ್ಪ-ರಿಗೆ
ಉಪ್ಪಿನ
ಉಪ್ಪಿ-ನ-ಕಾ-ಯಿ-ಯನ್ನು
ಉಪ್ಪಿ-ನ-ಗೊಂಬೆ
ಉಪ್ಪು
ಉಪ್ಪೂ
ಉಬ್ಬರ
ಉಬ್ಬ-ರದ
ಉಬ್ಬಿ-ಹೋ-ಗಿ-ಬಿ-ಡು-ವು-ದಿಲ್ಲ
ಉಬ್ಬಿ-ಹೋ-ಗು-ವನು
ಉಬ್ಬು-ವಂತೆ
ಉಬ್ಬು-ವುದು
ಉಬ್ಬು-ವುದೂ
ಉಭ-ಯ-ಭ್ರ-ಷ್ಟ-ನಾಗಿ
ಉಭ-ಯ-ಭ್ರ-ಷ್ಠನೂ
ಉಭ-ಯೋ-ರಪಿ
ಉಭಾ-ವಪಿ
ಉಭೇ
ಉಭೌ
ಉರಿ
ಉರಿಗೆ
ಉರಿದು
ಉರಿ-ದು-ಹೋ-ಗು-ವುದು
ಉರಿ-ಬಿ-ಸಿ-ಲಿನ
ಉರಿಯ
ಉರಿ-ಯದೆ
ಉರಿ-ಯನ್ನು
ಉರಿ-ಯ-ಲಾ-ರದು
ಉರಿ-ಯಲು
ಉರಿ-ಯಲ್ಲ
ಉರಿಯು
ಉರಿ-ಯುತ್ತ
ಉರಿ-ಯು-ತ್ತದೆ
ಉರಿ-ಯು-ತ್ತಿದೆ
ಉರಿ-ಯು-ತ್ತಿ-ದೆಯೋ
ಉರಿ-ಯು-ತ್ತಿದ್ದ
ಉರಿ-ಯು-ತ್ತಿ-ದ್ದರೆ
ಉರಿ-ಯು-ತ್ತಿ-ರುವ
ಉರಿ-ಯು-ತ್ತಿ-ರು-ವುದು
ಉರಿ-ಯು-ತ್ತಿವೆ
ಉರಿ-ಯುವ
ಉರಿ-ಯುವು
ಉರಿ-ಯು-ವು-ದಕ್ಕೆ
ಉರಿ-ಯು-ವುದನ್ನು
ಉರಿ-ಯು-ವು-ದ-ರಿಂದ
ಉರಿ-ಯು-ವು-ದಿಲ್ಲ
ಉರಿ-ಯು-ವುದು
ಉರಿವ
ಉರಿಸಿ
ಉರಿ-ಸಿ-ದಾಗ
ಉರಿ-ಸಿ-ಬಿ-ಡು-ವುದು
ಉರಿ-ಸುವ
ಉರುಲು
ಉರು-ಳ-ಬೇ-ಕಾ-ಗಿದೆ
ಉರು-ಳ-ಬೇ-ಕಾ-ಗು-ವುದು
ಉರು-ಳ-ಬೇ-ಕಾ-ದರೆ
ಉರು-ಳ-ಬೇಕು
ಉರುಳಿ
ಉರು-ಳಿ-ಕೊಂಡು
ಉರು-ಳಿತು
ಉರು-ಳಿ-ಬಂದು
ಉರು-ಳಿ-ಸಲು
ಉರು-ಳಿಸಿ
ಉರು-ಳಿ-ಸುತ್ತ
ಉರು-ಳಿ-ಸು-ದಕ್ಕೆ
ಉರು-ಳಿ-ಹೋ-ಗು-ವುದು
ಉರು-ಳುತ್ತ
ಉರು-ಳುತ್ತಾ
ಉರು-ಳು-ತ್ತಾರೆ
ಉರು-ಳು-ತ್ತಿದೆ
ಉರು-ಳು-ತ್ತಿ-ರುವ
ಉರು-ಳು-ತ್ತಿವೆ
ಉರು-ಳು-ತ್ತಿ-ವೆಯೊ
ಉರು-ಳು-ತ್ತೇವೆ
ಉರು-ಳುವ
ಉರು-ಳು-ವಂತೆ
ಉರು-ಳು-ವ-ವ-ರಲ್ಲಿ
ಉರು-ಳು-ವ-ವರೇ
ಉರು-ಳು-ವು-ದಕ್ಕೆ
ಉರು-ಳು-ವು-ದ-ಕ್ಕೆಲ್ಲ
ಉರು-ಳು-ವು-ದಿಲ್ಲ
ಉರು-ಳು-ವುದು
ಉಲ್ಬ-ಣ-ವಾ-ಗು-ವುದು
ಉಲ್ಲಂ-ಘಿ-ಸು-ವನು
ಉಲ್ಲಂ-ಘಿ-ಸು-ವುದೇ
ಉಲ್ಲೇಖಿ
ಉಳದೆ
ಉಳ-ಬೇ-ಕಾ-ಗಿಲ್ಲ
ಉಳಿ
ಉಳಿ-ಗ-ಳಂತೆ
ಉಳಿ-ತಾ-ಯ-ವಾ-ಗು-ವುದು
ಉಳಿದ
ಉಳಿ-ದ-ವ-ನ್ನೆಲ್ಲಾ
ಉಳಿ-ದ-ವ-ರನ್ನು
ಉಳಿ-ದ-ವ-ರಲ್ಲಿ
ಉಳಿ-ದ-ವ-ರಿಗೆ
ಉಳಿ-ದ-ವರು
ಉಳಿ-ದ-ವರೆಲ್ಲ
ಉಳಿ-ದ-ವರೆಲ್ಲಾ
ಉಳಿ-ದವು
ಉಳಿ-ದ-ವು-ಗಳ
ಉಳಿ-ದ-ವು-ಗಳನ್ನು
ಉಳಿ-ದ-ವು-ಗ-ಳಾ-ದರೊ
ಉಳಿ-ದ-ವು-ಗ-ಳಾ-ವುವೂ
ಉಳಿ-ದ-ವು-ಗ-ಳೆಲ್ಲ
ಉಳಿ-ದ-ವು-ಗ-ಳೆ-ಲ್ಲ-ದರ
ಉಳಿ-ದ-ವು-ಗ-ಳೆ-ಲ್ಲವೂ
ಉಳಿ-ದ-ವು-ಗ-ಳೆಲ್ಲಾ
ಉಳಿ-ದ-ವೆ-ಲ್ಲವೂ
ಉಳಿ-ದಿತ್ತು
ಉಳಿ-ದಿದೆ
ಉಳಿ-ದಿ-ರುವು
ಉಳಿ-ದಿ-ರು-ವುದು
ಉಳಿ-ದಿ-ರು-ವು-ದೆಲ್ಲ
ಉಳಿ-ದಿವೆ
ಉಳಿದು
ಉಳಿ-ದು-ಕೊಂ-ಡಿ-ರು-ತ್ತವೆ
ಉಳಿ-ದು-ದನ್ನು
ಉಳಿ-ದು-ದ-ನ್ನೆಲ್ಲ
ಉಳಿ-ದು-ದೆಲ್ಲ
ಉಳಿ-ದು-ದೆಲ್ಲಾ
ಉಳಿ-ದು-ವು-ಗ-ಳೆಲ್ಲ
ಉಳಿ-ದು-ವು-ಗ-ಳೆಲ್ಲಾ
ಉಳಿ-ದು-ವೆಲ್ಲಾ
ಉಳಿ-ದೆ-ಲ್ಲ-ರ-ಮೇಲೂ
ಉಳಿ-ದೆ-ಲ್ಲವೂ
ಉಳಿ-ಯ-ಬೇ-ಕಾ-ದರೆ
ಉಳಿ-ಯುವ
ಉಳಿ-ಯು-ವಂ-ತಿಲ್ಲ
ಉಳಿ-ಯು-ವಂತೆ
ಉಳಿ-ಯು-ವು-ದಿಲ್ಲ
ಉಳಿ-ಯು-ವುದು
ಉಳಿ-ಯು-ವು-ದೆಂದು
ಉಳಿ-ಯು-ವುದೇ
ಉಳಿ-ಯು-ವುದೊ
ಉಳಿ-ಯು-ವುದೋ
ಉಳಿ-ಯು-ವೆವು
ಉಳಿವು
ಉಳಿ-ಸ-ಬ-ಹು-ದಾ-ಗಿ-ತ್ತಲ್ಲ
ಉಳಿ-ಸಲು
ಉಳಿ-ಸಿ-ಕೊ-ಳ್ಳ-ಬ-ಹುದು
ಉಳಿ-ಸಿ-ಕೊ-ಳ್ಳಲು
ಉಳಿ-ಸಿ-ಕೊ-ಳ್ಳು-ವು-ದಕ್ಕೆ
ಉಳಿ-ಸು-ತ್ತಾ-ನೆಯೋ
ಉಳಿ-ಸು-ವು-ದ-ಕ್ಕಾಗಿ
ಉಳು-ವು-ದಿಲ್ಲ
ಉಳ್ಳ-ವನು
ಉವಾಚ
ಉಶನಾ
ಉಷ್ಣ
ಉಷ್ಣ-ವಾ-ಗಿ-ದೆಯೆ
ಉಷ್ಮ
ಉಷ್ಮಪ
ಉಸಿ-ರನ್ನು
ಉಸಿ-ರಾ-ಗಿ-ದೆಯೋ
ಉಸಿ-ರಾಟ
ಉಸಿ-ರಾ-ಟ-ವನ್ನು
ಉಸಿ-ರಾಡ
ಉಸಿ-ರಾ-ಡು-ತ್ತದೆ
ಉಸಿ-ರಾ-ಡು-ತ್ತಿ-ರು-ವರು
ಉಸಿ-ರಾ-ಡುವ
ಉಸಿ-ರಾ-ಡು-ವ-ವ-ರೆಗೆ
ಉಸಿ-ರಾ-ಡು-ವು-ದಕ್ಕೂ
ಉಸಿ-ರಾ-ಡು-ವುದನ್ನು
ಉಸಿ-ರಾ-ಡು-ವುದು
ಉಸಿ-ರಾ-ಡು-ವುವು
ಉಸಿ-ರಾ-ಡು-ವೆನು
ಉಸಿ-ರಿ-ನ-ಲ್ಲೆಲ್ಲ
ಉಸಿ-ರಿ-ನಿಂದ
ಉಸಿ-ರಿ-ನಿಂ-ದಲೆ
ಉಸಿರು
ಊಟ
ಊಟ-ಕೊಟ್ಟು
ಊಟಕ್ಕೆ
ಊಟದ
ಊಟ-ದಲ್ಲಿ
ಊಟ-ಮಾ-ಡಿ-ದರೆ
ಊಟ-ಮಾ-ಡು-ವರೋ
ಊಟ-ಮಾ-ಡು-ವಾ-ಗಲೊ
ಊಟ-ಮಾ-ಡು-ವುದು
ಊಟ-ವನ್ನೇ
ಊದಿ
ಊದಿ-ದನು
ಊದಿ-ದರು
ಊದು-ತ್ತಿ-ರು-ವನೋ
ಊದು-ತ್ತೇವೆ
ಊರ
ಊರ-ಗೋಲು
ಊರನ್ನು
ಊರ-ನ್ನೆಲ್ಲ
ಊರಿಗೆ
ಊರಿ-ಗೆಲ್ಲಾ
ಊರಿನ
ಊರಿ-ನಲ್ಲಿ
ಊರಿ-ನ-ಲ್ಲಿ-ದ್ದ-ವನು
ಊರಿ-ನ-ಲ್ಲೆಲ್ಲ
ಊರಿ-ನಿಂದ
ಊರಿಲ್ಲ
ಊರು
ಊರು-ಗಳ
ಊರು-ಗಾಯಿ
ಊರು-ಗೋ-ಲನ್ನು
ಊರು-ಗೋ-ಲಿ-ನಂತೆ
ಊರು-ಗೋ-ಲಿಲ್ಲ
ಊರು-ಗೋಲು
ಊರು-ವಂತೆ
ಊರ್ಜಿತ
ಊರ್ಜಿ-ತ-ವಾಗಿ
ಊರ್ಜಿ-ತ-ವಾ-ಗಿ-ದೆಯೊ
ಊರ್ಜಿ-ತ-ವಾ-ಗಿ-ದೆಯೋ
ಊರ್ಧ್ವ
ಊರ್ಧ್ವಂ
ಊರ್ಧ್ವ-ಮೂ-ಲ-ಮ-ಧಃ-ಶಾ-ಖ-ಮ-ಶ್ವತ್ಥಂ
ಊಳಿಗ
ಊಳಿ-ಗ-ದ-ವರು
ಊಳಿ-ಗ-ಮಾ-ಡು-ತ್ತಿ-ರು-ವ-ವರು
ಊಹಿ-ಸದೆ
ಊಹಿ-ಸ-ಬಲ್ಲ
ಊಹಿ-ಸ-ಬ-ಹುದು
ಊಹಿ-ಸ-ಬೇ-ಕಾ-ಗಿದೆ
ಊಹಿ-ಸ-ಬೇ-ಕಾಗು
ಊಹಿ-ಸ-ಬೇಕು
ಊಹಿ-ಸ-ಲಾ-ರೆವು
ಊಹಿಸಿ
ಊಹಿ-ಸಿ-ಕೊಂ-ಡರೆ
ಊಹಿ-ಸಿ-ಕೊ-ಳ್ಳ-ಬ-ಹುದು
ಊಹಿ-ಸಿ-ಕೊ-ಳ್ಳು-ವು-ದಕ್ಕೆ
ಊಹಿ-ಸಿದ್ದ
ಊಹಿ-ಸಿ-ರ-ಲಿಲ್ಲ
ಊಹಿ-ಸು-ತ್ತೇವೆ
ಊಹಿ-ಸು-ತ್ತೇ-ವೆಯೋ
ಊಹಿ-ಸು-ವನೊ
ಊಹಿ-ಸು-ವರು
ಊಹಿ-ಸು-ವು-ದಕ್ಕೂ
ಊಹಿ-ಸು-ವು-ದಕ್ಕೆ
ಊಹಿ-ಸು-ವುದು
ಊಹೆ
ಊಹೆಗೆ
ಊಹೆ-ಯಂತೆ
ಎಂಜ-ಲನ್ನು
ಎಂಜ-ಲಾ-ಗು-ವುದು
ಎಂಜ-ಲಿಗೆ
ಎಂಜಲು
ಎಂಜಿನ್
ಎಂಟು
ಎಂತಲೂ
ಎಂತಲೇ
ಎಂತಹ
ಎಂತ-ಹದು
ಎಂತ-ಹುದು
ಎಂತೆಂ-ತಹ
ಎಂತೋ
ಎಂಥ
ಎಂಥದ್ದು
ಎಂದ
ಎಂದಂ-ತಾ-ಗು-ವುದು
ಎಂದಂ-ತಾ-ಯಿತು
ಎಂದಂತೆ
ಎಂದ-ನಲ್ಲ
ಎಂದನು
ಎಂದ-ಮೇಲೆ
ಎಂದ-ರಿ-ತಿ-ರು-ವನೊ
ಎಂದ-ರಿತು
ಎಂದರು
ಎಂದರೂ
ಎಂದರೆ
ಎಂದ-ರೆ-ಅ-ವನು
ಎಂದ-ರೇನು
ಎಂದರ್ಥ
ಎಂದಲ್ಲ
ಎಂದ-ಲ್ಲ-ಇ-ದಕ್ಕೆ
ಎಂದಲ್ಲಿ
ಎಂದಳು
ಎಂದ-ವನು
ಎಂದಾಗ
ಎಂದಾ-ದರೂ
ಎಂದಾ-ದರೆ
ಎಂದಿ-ಗಿಂತ
ಎಂದಿಗೂ
ಎಂದಿತು
ಎಂದಿದೆ
ಎಂದಿ-ದ್ದರೂ
ಎಂದಿ-ನಂತೆ
ಎಂದಿ-ನಂ-ತೆಯೇ
ಎಂದು
ಎಂದು-ಉ-ಸಿ-ರನ್ನು
ಎಂದು-ಕೊಂ-ಡರೆ
ಎಂದು-ಕೊ-ಳ್ಳ-ಬ-ಹುದು
ಎಂದು-ಕೊಳ್ಳು
ಎಂದು-ಕೊ-ಳ್ಳು-ತ್ತಾನೆ
ಎಂದು-ಬಿ-ಟ್ಟರೆ
ಎಂದೂ
ಎಂದೆಂ
ಎಂದೆಂ-ದಿಗೂ
ಎಂದೆಂದೂ
ಎಂದೆಲ್ಲ
ಎಂದೇ
ಎಂದೊ-ಡನೆ
ಎಂದೋ
ಎಂಬ
ಎಂಬಂತೆ
ಎಂಬ-ದನ್ನು
ಎಂಬವು
ಎಂಬ-ವು-ಗಳು
ಎಂಬಿ-ವು-ಗಳು
ಎಂಬು
ಎಂಬುದ
ಎಂಬು-ದಕ್ಕೆ
ಎಂಬು-ದ-ನ್ನ-ರಿತು
ಎಂಬು-ದ-ನ್ನ-ರಿ-ಯು-ವನು
ಎಂಬು-ದನ್ನು
ಎಂಬು-ದನ್ನೆ
ಎಂಬು-ದ-ನ್ನೆಲ್ಲ
ಎಂಬು-ದ-ನ್ನೆಲ್ಲಾ
ಎಂಬು-ದರ
ಎಂಬು-ದ-ರಲ್ಲಿ
ಎಂಬು-ದ-ರಲ್ಲೇ
ಎಂಬು-ದಲ್ಲ
ಎಂಬು-ದಿಲ್ಲ
ಎಂಬುದು
ಎಂಬುದೂ
ಎಂಬು-ದೆಲ್ಲ
ಎಂಬು-ದೆ-ಲ್ಲವೂ
ಎಂಬುದೇ
ಎಂಬು-ದೊಂದು
ಎಂಬು-ದೊಂದೆ
ಎಂಬು-ದೊಂದೇ
ಎಂಬುವ
ಎಂಬು-ವನ
ಎಂಬು-ವನು
ಎಂಬು-ವರು
ಎಂಬು-ವ-ಳಿಗೆ
ಎಂಬುವು
ಎಂಬು-ವು-ಗಳು
ಎಂಬು-ವುದು
ಎಕರೆ
ಎಕ-ರೆ-ಗ-ಳಷ್ಟು
ಎಕ್ಕಡ
ಎಕ್ಕ-ಡ-ವನ್ನು
ಎಕ್ಸ್
ಎಕ್ಸ್ರೇ
ಎಕ್ಸ್ರೇ-ಗಳು
ಎಚ್ಚ-ತ್ತಿ-ರು-ವರೊ
ಎಚ್ಚರ
ಎಚ್ಚ-ರ-ಗಳಲ್ಲಿ
ಎಚ್ಚ-ರ-ವಾ-ಗು-ತ್ತಿದೆ
ಎಚ್ಚ-ರಿಕೆ
ಎಚ್ಚೆ-ತ್ತ-ಕೂ-ಡಲೆ
ಎಚ್ಚೆ-ತ್ತಿದೆ
ಎಚ್ಚೆ-ತ್ತಿರು
ಎಚ್ಚೆ-ತ್ತಿ-ರು-ತ್ತಿದ್ದ
ಎಚ್ಚೆ-ತ್ತಿ-ರು-ವನು
ಎಟು-ಕದ
ಎಟು-ಕ-ದಾಗ
ಎಟು-ಕು-ವಂ-ತಿ-ದ್ದರೆ
ಎಟು-ಕು-ವು-ದಿಲ್ಲ
ಎಡ-ಬ-ಲ-ದಲ್ಲಿ
ಎಡ-ಬಿ-ಡದೆ
ಎಡರು
ಎಡ-ರು-ತೊ-ಡ-ರು-ಗಳನ್ನೆಲ್ಲಾ
ಎಡ-ವ-ಟ್ಟಾ-ಗು-ತ್ತಿದೆ
ಎಡ-ವ-ಟ್ಟಾ-ದರೆ
ಎಡ-ವದೆ
ಎಡ-ವ-ಬ-ಹುದು
ಎಡವಿ
ಎಡ-ವಿ-ದರೂ
ಎಡ-ವು-ತ್ತಾನೆ
ಎಡ-ವು-ತ್ತಾರೆ
ಎಡ-ವು-ತ್ತೇವೆ
ಎಡ-ಹು-ವು-ದಕ್ಕೆ
ಎಡ-ಹು-ವುದು
ಎಡ-ಹು-ವೆವು
ಎಡಿ-ಸನ್
ಎಡೆಗೆ
ಎಡೆ-ಗೊ-ಡು-ವು-ದಿಲ್ಲ
ಎಡೆ-ಬಿ-ಡದೆ
ಎಡೆ-ಯ-ಲ್ಲೆಲ್ಲಾ
ಎಡೆ-ಯಿಲ್ಲ
ಎಣಿ-ಕೆಗೆ
ಎಣಿಸಿ
ಎಣಿ-ಸು-ತ್ತಾರೆ
ಎಣಿ-ಸುವ
ಎಣಿ-ಸು-ವ-ವ-ರಲ್ಲಿ
ಎಣಿ-ಸು-ವು-ದಕ್ಕೆ
ಎಣೆ-ಯಿಲ್ಲ
ಎಣ್ಣೆ
ಎಣ್ಣೆಗೂ
ಎಣ್ಣೆಗೆ
ಎಣ್ಣೆ-ಯಂತೆ
ಎಣ್ಣೆ-ಯನ್ನು
ಎಣ್ಣೆ-ಯನ್ನೇ
ಎಣ್ಣೆ-ಯಿ-ದ್ದರೂ
ಎಣ್ಣೆ-ಯಿ-ಲ್ಲದೆ
ಎತ್ತ
ಎತ್ತನ್ನು
ಎತ್ತ-ಬೇ-ಕಾ-ಗಿಲ್ಲ
ಎತ್ತರ
ಎತ್ತ-ರಕ್ಕೆ
ಎತ್ತ-ರದ
ಎತ್ತ-ರ-ದಲ್ಲಿ
ಎತ್ತ-ರ-ದ-ಲ್ಲಿ-ದ್ದರೆ
ಎತ್ತ-ರ-ವನ್ನು
ಎತ್ತ-ರ-ವಾ-ಗಿ-ರು-ವುದು
ಎತ್ತ-ರ-ವಾದ
ಎತ್ತ-ರ-ವಿ-ರು-ವ-ವ-ನಾ-ದರೂ
ಎತ್ತ-ರವೂ
ಎತ್ತ-ಲಾ-ರದು
ಎತ್ತ-ಲಾ-ರವು
ಎತ್ತಲು
ಎತ್ತಲೂ
ಎತ್ತಿ
ಎತ್ತಿ-ಕೊಂಡು
ಎತ್ತಿ-ಕೊ-ಳ್ಳುವು
ಎತ್ತಿ-ತೋ-ರು-ವಾಗ
ಎತ್ತಿ-ದರೆ
ಎತ್ತಿ-ನಂತೆ
ಎತ್ತಿ-ರು-ವೆವು
ಎತ್ತು
ಎತ್ತು-ಗ-ಳಂತೆ
ಎತ್ತುತ್ತ
ಎತ್ತು-ತ್ತಿ-ದ್ದಂತೆ
ಎತ್ತು-ವಂ-ತಿಲ್ಲ
ಎತ್ತು-ವು-ದಕ್ಕೆ
ಎತ್ತು-ವುವು
ಎತ್ತು-ವೆವು
ಎದು-ಗಿ-ರುವ
ಎದು-ರಾಗಿ
ಎದು-ರಿ-ಗಿ-ರುವ
ಎದು-ರಿಗೆ
ಎದು-ರಿ-ಗೇನೆ
ಎದು-ರಿಸ
ಎದು-ರಿ-ಸದೆ
ಎದು-ರಿ-ಸ-ಬಲ್ಲ
ಎದು-ರಿ-ಸ-ಬೇ-ಕಾ-ಗಿದೆ
ಎದು-ರಿ-ಸ-ಬೇಕು
ಎದು-ರಿಸಿ
ಎದು-ರಿ-ಸಿ-ದಂತೆ
ಎದು-ರಿ-ಸಿ-ದರೂ
ಎದು-ರಿಸು
ಎದು-ರಿ-ಸುವ
ಎದು-ರಿ-ಸು-ವನು
ಎದು-ರಿ-ಸು-ವಷ್ಟು
ಎದು-ರಿ-ಸು-ವಾಗ
ಎದು-ರಿ-ಸು-ವು-ದಕ್ಕೆ
ಎದು-ರಿ-ಸು-ವುದನ್ನು
ಎದು-ರಿ-ಸು-ವುದೂ
ಎದುರು
ಎದು-ರು-ಗಿ-ರುವ
ಎದೆ-ಕೊ-ಡುವ
ಎದೆ-ಗುಂ-ದ-ಬೇ-ಕಾ-ಗಿಲ್ಲ
ಎದೆಗೆ
ಎದೆ-ಗೆ-ಡು-ವು-ದಿಲ್ಲ
ಎದೆ-ಯನ್ನು
ಎದೆ-ಯಲ್ಲಿ
ಎದೆ-ಯಿಲ್ಲ
ಎದೆ-ಯೊ-ಡ್ಡುವ
ಎದೆ-ಯೊ-ಡ್ಡು-ವ-ವರು
ಎದೆ-ಹಾ-ಲನ್ನು
ಎದೆ-ಹಾಲು
ಎದ್ದ
ಎದ್ದಂ-ತಿದೆ
ಎದ್ದಂ-ತೆಯೇ
ಎದ್ದರೆ
ಎದ್ದ-ವನು
ಎದ್ದಾಗ
ಎದ್ದಾದ
ಎದ್ದಾ-ದ-ಮೇಲೆ
ಎದ್ದಿತು
ಎದ್ದಿದೆ
ಎದ್ದಿ-ರುವ
ಎದ್ದಿ-ರು-ವನು
ಎದ್ದಿ-ರು-ವುದನ್ನು
ಎದ್ದಿಲ್ಲ
ಎದ್ದು
ಎದ್ದು-ನಿಲ್ಲು
ಎದ್ದೊ-ಡನೆ
ಎನನ್ನು
ಎನಿ-ಸಿ-ಕೊಂ-ಡಾಗ
ಎನಿ-ಸಿ-ಕೊ-ಳ್ಳ-ಬೇಕು
ಎನಿ-ಸಿ-ಕೊ-ಳ್ಳು-ತ್ತಾನೆ
ಎನಿ-ಸಿ-ರು-ವನು
ಎನಿ-ಸು-ತ್ತಾನೆ
ಎನಿ-ಸು-ವನು
ಎನಿ-ಸು-ವು-ದಿಲ್ಲ
ಎನ್ನದೆ
ಎನ್ನ-ಬ-ಹುದು
ಎನ್ನ-ಬೇ-ಕಾ-ಗಿದೆ
ಎನ್ನ-ಬೇ-ಕಾ-ದರೆ
ಎನ್ನ-ಬೇಕು
ಎನ್ನ-ಲಾಗು
ಎನ್ನ-ಲಾ-ಗು-ವು-ದಿಲ್ಲ
ಎನ್ನ-ಲಿಲ್ಲ
ಎನ್ನ-ವನು
ಎನ್ನ-ವರು
ಎನ್ನ-ವುದು
ಎನ್ನಿ-ಸ-ಬೇಕು
ಎನ್ನಿ-ಸಿ-ಕೊಂಡ
ಎನ್ನಿ-ಸಿ-ಬಿ-ಡು-ತ್ತದೆ
ಎನ್ನಿ-ಸು-ವುದು
ಎನ್ನು
ಎನ್ನು-ತ್ತಾನೆ
ಎನ್ನು-ತ್ತಾ-ನೆಯೆ
ಎನ್ನು-ತ್ತಾರೆ
ಎನ್ನು-ತ್ತಾ-ರೆಯೊ
ಎನ್ನು-ತ್ತಾಳೆ
ಎನ್ನು-ತ್ತಿತ್ತು
ಎನ್ನು-ತ್ತಿದೆ
ಎನ್ನು-ತ್ತಿ-ದ್ದರು
ಎನ್ನು-ತ್ತಿ-ರು-ವುದೇ
ಎನ್ನು-ತ್ತೇನೆ
ಎನ್ನು-ತ್ತೇ-ವಲ್ಲ
ಎನ್ನು-ತ್ತೇವೆ
ಎನ್ನು-ತ್ತೇ-ವೆಯೊ
ಎನ್ನು-ತ್ತೇ-ವೆಯೋ
ಎನ್ನುವ
ಎನ್ನು-ವಂ-ತಹ
ಎನ್ನು-ವಂತೆ
ಎನ್ನು-ವ-ದನ್ನು
ಎನ್ನು-ವನು
ಎನ್ನು-ವನೇ
ಎನ್ನು-ವರು
ಎನ್ನು-ವಳು
ಎನ್ನು-ವ-ವ-ನಲ್ಲ
ಎನ್ನು-ವ-ವನು
ಎನ್ನು-ವ-ವನೆ
ಎನ್ನು-ವ-ವನೇ
ಎನ್ನು-ವ-ವರು
ಎನ್ನು-ವಷ್ಟು
ಎನ್ನು-ವಾಗ
ಎನ್ನು-ವು-ದಕ್ಕೆ
ಎನ್ನು-ವುದನ್ನು
ಎನ್ನು-ವು-ದ-ರಲ್ಲಿ
ಎನ್ನು-ವು-ದಲ್ಲ
ಎನ್ನು-ವು-ದಾ-ಗಿತ್ತು
ಎನ್ನು-ವು-ದಿಲ್ಲ
ಎನ್ನು-ವುದು
ಎನ್ನು-ವುದೆ
ಎನ್ನು-ವೆವು
ಎನ್ನು-ವೆವೊ
ಎನ್ನು-ವೆವೋ
ಎಬ್ಬಿ-ಸಲು
ಎಬ್ಬಿ-ಸಿದ
ಎಬ್ಬಿ-ಸು-ತ್ತಿ-ರು-ವುದು
ಎರ-ಕ-ದಲ್ಲಿ
ಎರ-ಗು-ವುದು
ಎರ-ಚ-ಲಾ-ಗು-ವು-ದಿಲ್ಲ
ಎರಚಿ
ಎರ-ಚು-ವ-ರಾರು
ಎರ-ಡಕ್ಕೂ
ಎರ-ಡನೆ
ಎರ-ಡ-ನೆಯ
ಎರ-ಡ-ನೆ-ಯ-ದಾಗಿ
ಎರ-ಡ-ನೆ-ಯದು
ಎರ-ಡ-ನೆ-ಯದೂ
ಎರ-ಡ-ನೆ-ಯದೆ
ಎರ-ಡ-ನೆ-ಯದೇ
ಎರ-ಡ-ನೆ-ಯ-ವನು
ಎರ-ಡ-ನೆ-ಯ-ವನೆ
ಎರ-ಡ-ನೆ-ಯ-ವನೇ
ಎರ-ಡ-ನೆ-ಯ-ವರು
ಎರ-ಡ-ನೆ-ಯ-ವರೆ
ಎರ-ಡನೇ
ಎರ-ಡನ್ನು
ಎರ-ಡನ್ನೂ
ಎರ-ಡರ
ಎರ-ಡ-ರಲ್ಲೂ
ಎರ-ಡ-ರ-ಷ್ಟಾ-ಗಿ-ದ್ದೇವೆ
ಎರ-ಡ-ರ-ಷ್ಟಾ-ದ-ಮೇಲೆ
ಎರ-ಡ-ರಿಂ-ದಲೂ
ಎರ-ಡಿಲ್ಲ
ಎರಡು
ಎರಡೂ
ಎರ-ವ-ಲಾಗಿ
ಎರೆದು
ಎರೆ-ದು-ಕೊ-ಳ್ಳು-ತ್ತಿದೆ
ಎರೆ-ಯದೇ
ಎರೆ-ಯ-ಬೇ-ಕಾ-ಗಿದೆ
ಎಲು-ಬಿನ
ಎಲುಬು
ಎಲೆ
ಎಲೆ-ಗ-ಳಷ್ಟು
ಎಲೆ-ಗ-ಳಿ-ವೆಯೊ
ಎಲೆ-ಗಳು
ಎಲೆ-ಗ-ಳೆಲ್ಲ
ಎಲೆ-ಗ-ಳೆಲ್ಲಾ
ಎಲೆಗೆ
ಎಲೆಯ
ಎಲೆ-ಯ-ನ್ನಾ-ಗಲಿ
ಎಲೆ-ಯಲ್ಲಿ
ಎಲೈ
ಎಲ್ಲ
ಎಲ್ಲ-ಕ-ಡೆ-ಯಲ್ಲೂ
ಎಲ್ಲ-ಕ್ಕಿಂತ
ಎಲ್ಲ-ಕ್ಕಿಂ-ತಲೂ
ಎಲ್ಲಕ್ಕೂ
ಎಲ್ಲ-ಡೆ-ಯಲ್ಲೂ
ಎಲ್ಲದ
ಎಲ್ಲ-ದ-ಕ್ಕಿಂತ
ಎಲ್ಲ-ದಕ್ಕೂ
ಎಲ್ಲ-ದಕ್ಕೆ
ಎಲ್ಲ-ದರ
ಎಲ್ಲ-ದ-ರ-ಲ್ಲಿಯೂ
ಎಲ್ಲ-ದ-ರಿಂದ
ಎಲ್ಲ-ದ-ರಿಂ-ದಲೂ
ಎಲ್ಲರ
ಎಲ್ಲ-ರನ್ನು
ಎಲ್ಲ-ರನ್ನೂ
ಎಲ್ಲ-ರಲ್ಲಿ
ಎಲ್ಲ-ರ-ಲ್ಲಿಯೂ
ಎಲ್ಲ-ರ-ಲ್ಲಿ-ರುವ
ಎಲ್ಲ-ರಿಂದ
ಎಲ್ಲ-ರಿಂ-ದಲೂ
ಎಲ್ಲ-ರಿ-ಗಿಂತ
ಎಲ್ಲ-ರಿ-ಗಿಂ-ತಲೂ
ಎಲ್ಲ-ರಿಗೂ
ಎಲ್ಲರೂ
ಎಲ್ಲ-ರೆದು
ಎಲ್ಲ-ರೆ-ದು-ರಿಗೆ
ಎಲ್ಲ-ವನ್ನು
ಎಲ್ಲ-ವನ್ನೂ
ಎಲ್ಲವೂ
ಎಲ್ಲಾ
ಎಲ್ಲಾ-ಕ-ಡೆಯೂ
ಎಲ್ಲಾ-ಬ-ಗೆಯ
ಎಲ್ಲಾ-ಸೇರಿ
ಎಲ್ಲಿ
ಎಲ್ಲಿಂದ
ಎಲ್ಲಿಂ-ದ-ಲಾ-ದರೂ
ಎಲ್ಲಿಂ-ದಲೊ
ಎಲ್ಲಿಂ-ದಲೋ
ಎಲ್ಲಿಗೂ
ಎಲ್ಲಿಗೆ
ಎಲ್ಲಿ-ಟ್ಟೆವು
ಎಲ್ಲಿತ್ತು
ಎಲ್ಲಿದೆ
ಎಲ್ಲಿ-ದ್ದರೂ
ಎಲ್ಲಿ-ದ್ದಾನೆ
ಎಲ್ಲಿಯ
ಎಲ್ಲಿ-ಯ-ವ-ರೆಗೆ
ಎಲ್ಲಿ-ಯ-ವರೆ-ವಿಗೂ
ಎಲ್ಲಿ-ಯಾ-ದರೂ
ಎಲ್ಲಿಯೂ
ಎಲ್ಲಿಯೊ
ಎಲ್ಲಿಯೋ
ಎಲ್ಲಿ-ರು-ವನೊ
ಎಲ್ಲಿ-ರು-ವನೋ
ಎಲ್ಲಿ-ರು-ವರು
ಎಲ್ಲಿ-ರು-ವೆವು
ಎಲ್ಲೂ
ಎಲ್ಲೆ-ಡೆ-ಗಿಂತ
ಎಲ್ಲೆ-ಯ-ಲ್ಲಿಲ್ಲ
ಎಲ್ಲೆಲ್ಲಿ
ಎಲ್ಲೆ-ಲ್ಲಿ-ದೆಯೊ
ಎಲ್ಲೆ-ಲ್ಲಿಯೂ
ಎಲ್ಲೆ-ಲ್ಲಿಯೋ
ಎಲ್ಲೆ-ಲ್ಲಿ-ರು-ತ್ತೇನೆ
ಎಲ್ಲೆಲ್ಲೂ
ಎಲ್ಲೆಲ್ಲೊ
ಎಲ್ಲೆಲ್ಲೋ
ಎಲ್ಲೊ
ಎಲ್ಲೋ
ಎಳ-ನೀ-ರಿ-ನಲ್ಲಿ
ಎಳೆ
ಎಳೆ-ತ-ರ-ಲಾ-ರದು
ಎಳೆ-ದಂತೆ
ಎಳೆ-ದತ್ತ
ಎಳೆ-ದರೂ
ಎಳೆ-ದರೆ
ಎಳೆ-ದ-ಲ್ಲದೆ
ಎಳೆದು
ಎಳೆ-ದು-ಕೊಂ-ಡರೆ
ಎಳೆ-ದು-ಕೊಂಡು
ಎಳೆ-ನಗೆ
ಎಳೆ-ನೀ-ರಿ-ನಂತೆ
ಎಳೆಯ
ಎಳೆ-ಯ-ದಾ-ಗಿ-ರು-ವಾಗ
ಎಳೆ-ಯ-ಬೇ-ಕಾ-ಗು-ವುದು
ಎಳೆ-ಯ-ಲಾ-ರದು
ಎಳೆ-ಯ-ಲಾ-ರೆ-ಅ-ವ-ನಿಗೆ
ಎಳೆ-ಯಲೇ
ಎಳೆ-ಯ-ಲ್ಪ-ಡು-ತ್ತಾನೆ
ಎಳೆ-ಯ-ವ-ರಾ-ಗಿ-ರು-ವಾಗ
ಎಳೆ-ಯು-ತ್ತಿ-ರು-ವೆವು
ಎಳೆ-ಯುವ
ಎಳೆ-ಯು-ವಾಗ
ಎಳೆ-ಯು-ವಾ-ಗಲೂ
ಎಳೆ-ಯು-ವು-ದಕ್ಕೆ
ಎಳೆ-ಯು-ವುವು
ಎಳ್ಳಷ್ಟು
ಎಳ್ಳಷ್ಟೂ
ಎಳ್ಳಿನ
ಎವರೆ-ಸ್ಟ್
ಎಷ್ಚು
ಎಷ್ಚೇ
ಎಷ್ಚೊ
ಎಷ್ಟಕ್ಕೆ
ಎಷ್ಟ-ನ್ನಾ-ದರೂ
ಎಷ್ಟನ್ನು
ಎಷ್ಟಾ-ದರೂ
ಎಷ್ಟಿ-ದ್ದರೂ
ಎಷ್ಟಿ-ದ್ದ-ರೇನು
ಎಷ್ಟಿ-ರು-ವುದು
ಎಷ್ಟು
ಎಷ್ಟು-ಕಾಲ
ಎಷ್ಟು-ಮ-ಟ್ಟಿಗೆ
ಎಷ್ಟು-ಮ-ಟ್ಟಿ-ನದು
ಎಷ್ಟು-ಸಲ
ಎಷ್ಟು-ಸಾರಿ
ಎಷ್ಟು-ಹೊತ್ತು
ಎಷ್ಟೆಷ್ಟು
ಎಷ್ಟೇ
ಎಷ್ಟೊ
ಎಷ್ಟೊಂದು
ಎಷ್ಟೋ
ಎಷ್ಟೋ-ವೇಳೆ
ಎಷ್ಟೋ-ಸಲ
ಎಸಳು
ಎಸಳೂ
ಎಸೆ-ದರೆ
ಎಸೆ-ದಾ-ಡು-ತ್ತಿವೆ
ಎಸೆ-ದಿ-ರು-ವನು
ಎಸೆ-ದಿ-ರು-ವೆನೊ
ಎಸೆದು
ಎಸೆ-ಯ-ಬ-ಹುದು
ಎಸೆ-ಯ-ಬೇಕು
ಎಸೆ-ಯು-ತ್ತದೆ
ಎಸೆ-ಯು-ತ್ತಾನೆ
ಎಸೆ-ಯುವ
ಎಸೆ-ಯು-ವಂ-ತಿ-ರ-ಬಾ-ರದು
ಎಸೆ-ಯು-ವಂತೆ
ಎಸೆ-ಯು-ವನು
ಎಸೆ-ಯು-ವರು
ಎಸೆ-ಯು-ವು-ದಕ್ಕೆ
ಎಸೆ-ಯು-ವು-ದಿಲ್ಲ
ಎಸೆ-ಯು-ವುದು
ಎಸೆ-ಳಾ-ಗಲಿ
ಏಕ
ಏಕಂ
ಏಕ-ಕಾ-ಲ-ದಲ್ಲಿ
ಏಕ-ತ್ವದ
ಏಕ-ತ್ವ-ದಲ್ಲಿ
ಏಕ-ತ್ವ-ವನ್ನು
ಏಕ-ತ್ವವೇ
ಏಕ-ತ್ವೇನ
ಏಕ-ನಿ-ಷ್ಠ-ವಾ-ಗಿ-ರು-ತ್ತದೆ
ಏಕ-ನಿ-ಷ್ಠ-ವಾದ
ಏಕ-ನಿ-ಷ್ಠೆ-ಯಾದ
ಏಕ-ನಿ-ಷ್ಠೆ-ಯಿಂದ
ಏಕ-ಭಕ್ತಿ
ಏಕ-ಭ-ಕ್ತಿ-ರ್ವಿ-ಶಿ-ಷ್ಯತೇ
ಏಕ-ಮ-ಪ್ಯಾ-ಸ್ಥಿತಃ
ಏಕ-ಮಾತ್ರ
ಏಕ-ಮುಖ
ಏಕ-ಮು-ಖ-ವಾಗಿ
ಏಕ-ಮು-ಖ-ವಾ-ಗಿ-ರು-ವು-ದಿಲ್ಲ
ಏಕ-ಮು-ಖ-ವಾ-ಗಿ-ರು-ವುದು
ಏಕ-ಮು-ಖ-ವಾದ
ಏಕಯಾ
ಏಕ-ಲವ್ಯ
ಏಕಾಂ-ಗಿ-ಯಲ್ಲ
ಏಕಾಂ-ಗಿ-ಯಾಗಿ
ಏಕಾಂ-ಗಿ-ಯಾ-ಗಿ-ರಲು
ಏಕಾಂತ
ಏಕಾಕಿ
ಏಕಾ-ಕಿ-ಯಾಗಿ
ಏಕಾ-ಕಿ-ಯಾ-ಗಿ-ರ-ಬೇಕು
ಏಕಾ-ಕಿ-ಯಾ-ಗಿ-ರು-ವು-ದೊಂದೇ
ಏಕಾಕೀ
ಏಕಾ-ಕ್ಷರ
ಏಕಾ-ಕ್ಷ-ರ-ವಾದ
ಏಕಾಗ್ರ
ಏಕಾ-ಗ್ರ-ಗೊ-ಳಿ-ಸಲು
ಏಕಾ-ಗ್ರ-ಗೊ-ಳಿಸಿ
ಏಕಾ-ಗ್ರ-ಗೊ-ಳಿ-ಸು-ವುದು
ಏಕಾ-ಗ್ರ-ಚಿ-ತ್ತ-ದಿಂದ
ಏಕಾ-ಗ್ರತೆ
ಏಕಾ-ಗ್ರ-ತೆಗೆ
ಏಕಾ-ಗ್ರ-ತೆಗೇ
ಏಕಾ-ಗ್ರ-ತೆ-ಯನ್ನು
ಏಕಾ-ಗ್ರ-ತೆ-ಯಿಂದ
ಏಕಾ-ಗ್ರ-ನಾ-ಗು-ವ-ವನು
ಏಕಾ-ಗ್ರ-ಮಾ-ಡ-ಬೇಕು
ಏಕಾ-ಗ್ರ-ಮಾಡಿ
ಏಕಾ-ಗ್ರ-ಮಾ-ಡಿ-ರು-ವನು
ಏಕಾ-ಗ್ರ-ಮಾ-ಡು-ವರು
ಏಕಾ-ಗ್ರ-ಮಾ-ಡು-ವುದು
ಏಕಾ-ಗ್ರ-ವಾ-ಗದೆ
ಏಕಾ-ಗ್ರ-ವಾಗಿ
ಏಕಾ-ಗ್ರ-ವಾ-ಗಿದೆ
ಏಕಾ-ಗ್ರ-ವಾ-ಗಿ-ರ-ಬೇಕು
ಏಕಾ-ಗ್ರ-ವಾ-ಗುತ್ತ
ಏಕಾ-ಗ್ರ-ವಾ-ಗು-ವುದು
ಏಕಾ-ಗ್ರ-ವಾ-ಗು-ವುದೊ
ಏಕಾ-ಗ್ರ-ವಾದ
ಏಕಾ-ದರೂ
ಏಕಾ-ದಶ
ಏಕೆ
ಏಕೆಂ-ದರೆ
ಏಕೋ-ಽಥ-ವಾ-ಪ್ಯ-ಚ್ಯುತ
ಏಟನ್ನು
ಏಟನ್ನೂ
ಏಟಿನ
ಏಟಿ-ನಿಂದ
ಏಟು
ಏಣಿ
ಏಣಿಯ
ಏಣಿ-ಯನ್ನು
ಏಣಿ-ಯಲ್ಲಿ
ಏತ-ಕ್ಕಾಗಿ
ಏತ-ಕ್ಕಾ-ದರೂ
ಏತ-ಕ್ಕಾ-ಯಿತು
ಏತಕ್ಕೆ
ಏತ-ಕ್ಕೆಂ-ದರೆ
ಏತಕ್ಕೋ
ಏತ-ಚ್ಛ್ರುತ್ವಾ
ಏತ-ಜ್ಜ್ಞಾ-ನ-ಮಿತಿ
ಏತ-ತ್ಕ್ಷೇತ್ರಂ
ಏತದ್ಧಿ
ಏತ-ದ್ಬುದ್ಧ್ವಾ
ಏತದ್ಯೋ
ಏತ-ದ್ಯೋ-ನೀನಿ
ಏತ-ದ್ವಿ-ಜ್ಞಾಯ
ಏತನ್ಮೇ
ಏತ-ಯೋ-ರೇಕಂ
ಏತ-ರಿಂದ
ಏತ-ಸ್ಯಾಹಂ
ಏತಾಂ
ಏತಾನ್ನ
ಏತಾ-ನ್ಯಪಿ
ಏತಾ-ವ-ದಿತಿ
ಏತೇತ್ರ
ಏತೈ-ರ್ವಿ-ಮುಕ್ತಃ
ಏತೈ-ರ್ವಿ-ಮೋ-ಹ-ಯ-ತ್ಯೇಷ
ಏದು-ಸಿರು
ಏನಂ
ಏನ-ನ್ನಾ-ದರು
ಏನ-ನ್ನಾ-ದರೂ
ಏನನ್ನು
ಏನನ್ನೂ
ಏನನ್ನೊ
ಏನನ್ನೋ
ಏನ-ಮ-ಜ-ಮ-ವ್ಯ-ಯಮ್
ಏನಾ
ಏನಾ-ಗ-ಬ-ಹುದು
ಏನಾ-ಗ-ಬೇ-ಕೆಂದು
ಏನಾ-ಗಿತ್ತು
ಏನಾ-ಗಿದೆ
ಏನಾ-ಗಿ-ದೆಯೊ
ಏನಾ-ಗಿ-ದ್ದಿರಿ
ಏನಾ-ಗಿದ್ದೆ
ಏನಾ-ಗು-ತ್ತದೆ
ಏನಾ-ಗು-ತ್ತಿದೆ
ಏನಾ-ಗು-ತ್ತಿ-ದೆಯೋ
ಏನಾ-ಗು-ತ್ತೇನೆ
ಏನಾ-ಗು-ವುದು
ಏನಾ-ಗು-ವು-ದೆಂ-ಬು-ದನ್ನು
ಏನಾ-ಗು-ವುದೊ
ಏನಾ-ಗು-ವುದೋ
ಏನಾ-ದರೂ
ಏನಾ-ದ-ರೇ-ನಂತೆ
ಏನಾ-ಯಿತು
ಏನಿ
ಏನಿತ್ತು
ಏನಿತ್ತೋ
ಏನಿದೆ
ಏನಿ-ದೆಯೊ
ಏನಿ-ದೆಯೋ
ಏನಿ-ದ್ದರೂ
ಏನಿ-ದ್ದರೆ
ಏನಿ-ರ-ಬ-ಹುದು
ಏನಿಲ್ಲ
ಏನಿ-ಲ್ಲ-ನಾವು
ಏನು
ಏನು-ಬೇ-ಕಾ-ದರೂ
ಏನು-ಬೇಕೊ
ಏನು-ಮಾ-ಡಿ-ದರು
ಏನು-ಮಾ-ಡಿ-ದರೂ
ಏನೂ
ಏನೆಂ-ದರೂ
ಏನೆಂದು
ಏನೆಂ-ಬುದು
ಏನೆ-ನ್ನು-ವುದು
ಏನೇ
ಏನೇ-ನನ್ನು
ಏನೇ-ನನ್ನೊ
ಏನೇ-ನನ್ನೋ
ಏನೇ-ನಿದೆ
ಏನೇನು
ಏನೇನೊ
ಏನೇನೋ
ಏನೊ
ಏನೋ
ಏನೋನೋ
ಏರ-ಬೇ-ಕಾದ
ಏರ-ಬೇ-ಕಾ-ದರೆ
ಏರ-ಬೇಕು
ಏರ-ಬೇ-ಕೆಂಬ
ಏರ-ಲಾ-ರರು
ಏರ-ಲಿಲ್ಲ
ಏರಲು
ಏರಿ
ಏರಿ-ದ-ವ-ನಲ್ಲ
ಏರಿ-ದಾಗ
ಏರಿದೆ
ಏರಿ-ರು-ವನು
ಏರಿ-ರುವೆ
ಏರಿಲ್ಲ
ಏರಿ-ಳಿ-ತ-ಗಳು
ಏರಿ-ಸ-ಬ-ಹುದು
ಏರಿ-ಸಲು
ಏರಿಸಿ
ಏರಿ-ಸು-ವನು
ಏರು-ತ್ತಲೇ
ಏರುವ
ಏರು-ವಂತೆ
ಏರು-ವನು
ಏರು-ವನೊ
ಏರು-ವುದು
ಏರೋ-ಡ್ರೋ-ಮಿನ
ಏರ್ಪ-ಡ-ಬೇಕು
ಏರ್ಪ-ಡಿ-ಸು-ತ್ತಿ-ದ್ದರು
ಏರ್ಪ-ಡಿ-ಸು-ವುದು
ಏರ್ಪಾಡು
ಏಳ-ನೆಯ
ಏಳ-ಬಾ-ರದು
ಏಳ-ಬೇ-ಕಾ-ಗು-ವುದು
ಏಳ-ಬೇ-ಕಾ-ದರೆ
ಏಳ-ಬೇಕು
ಏಳ-ಬೇ-ಕೆಂದು
ಏಳ-ಲಾ-ರೆವು
ಏಳಲು
ಏಳಿ-ಸದೆ
ಏಳಿ-ಸ-ಬೇ-ಕಾ-ದರೆ
ಏಳು
ಏಳು-ತ್ತವೆ
ಏಳು-ತ್ತ-ವೆಯೊ
ಏಳು-ತ್ತ-ವೆಯೋ
ಏಳು-ತ್ತಾನೆ
ಏಳು-ತ್ತಾರೆ
ಏಳು-ತ್ತಿದೆ
ಏಳು-ತ್ತಿ-ರ-ಬ-ಹುದು
ಏಳು-ತ್ತಿ-ರು-ತ್ತವೆ
ಏಳು-ತ್ತಿ-ರುವ
ಏಳು-ತ್ತಿ-ರು-ವಾಗ
ಏಳು-ತ್ತಿ-ರು-ವುದು
ಏಳು-ತ್ತಿ-ರು-ವುವು
ಏಳು-ತ್ತಿವೆ
ಏಳು-ತ್ತೇವೆ
ಏಳು-ನೂರು
ಏಳುವ
ಏಳು-ವನು
ಏಳು-ವು-ದಕ್ಕೆ
ಏಳು-ವು-ದಿದೆ
ಏಳು-ವು-ದಿಲ್ಲ
ಏಳು-ವುದು
ಏಳು-ವುವು
ಏಳು-ವುವೋ
ಏಳೆಂಟು
ಏಳೊ
ಏವ
ಏವಂ
ಏವಂ-ರೂಪಃ
ಏವಂ-ವಿಧೋ
ಏವ-ಮುಕ್ತೋ
ಏವ-ಮುಕ್ತ್ವಾ
ಏವ-ಮು-ಕ್ತ್ವಾ-ರ್ಜುನಃ
ಏವ-ಮೇ-ತ-ದ್ಯ-ಥಾತ್ಥ
ಏವಾಯಂ
ಏವಾಹ
ಏವೇತಿ
ಏವೈತೇ
ಏಷ
ಏಷ-ತೂ-ದ್ದೇ-ಶತಃ
ಏಷಾ
ಏಸು-ಕ್ರಿಸ್ತ
ಐಎ-ಎಸ್
ಐಕಾಂ-ತಿಕ
ಐಕಾಂ-ತಿ-ಕ-ವಾದ
ಐಕ್ಯ-ವಾ-ಗು-ತ್ತದೆ
ಐಕ್ಯ-ವಾ-ಗು-ವಂತೆ
ಐಕ್ಯ-ವಾ-ಗು-ವುವು
ಐಚ್ಛಿಕ
ಐತ-ರೇಯ
ಐತಿ-ಹಾ-ಸಿಕ
ಐದ-ನೆಯ
ಐದ-ನೆ-ಯ-ದಾದ
ಐದ-ನೆ-ಯದು
ಐದ-ನೆ-ಯದೆ
ಐದನೇ
ಐದ-ರಲ್ಲಿ
ಐದಾರು
ಐದು
ಐದು-ನೂರು
ಐದೂ
ಐನೂರು
ಐನ್ಸ್ಟಿನ್
ಐನ್ಸ್ಟೀನ್
ಐರಾ-ವತ
ಐರಾ-ವತಂ
ಐವತ್ತು
ಐಶ್ವ-ರೀ-ರೂಪ
ಐಶ್ವರ್ಯ
ಐಶ್ವ-ರ್ಯಕ್ಕೆ
ಐಶ್ವ-ರ್ಯದ
ಐಶ್ವ-ರ್ಯ-ದಲ್ಲಿ
ಐಶ್ವ-ರ್ಯ-ರಾ-ಶಿ-ಯಲ್ಲಿ
ಐಶ್ವ-ರ್ಯ-ರೂ-ಪ-ವನ್ನು
ಐಶ್ವ-ರ್ಯ-ವಂತ
ಐಶ್ವ-ರ್ಯ-ವಂ-ತ-ನಲ್ಲ
ಐಶ್ವ-ರ್ಯ-ವಂ-ತ-ನಾ-ಗು-ತ್ತೇನೆ
ಐಶ್ವ-ರ್ಯ-ವಂ-ತ-ನೆಂದು
ಐಶ್ವ-ರ್ಯ-ವಂ-ತರಿ
ಐಶ್ವ-ರ್ಯ-ವ-ನ್ನಲ್ಲ
ಐಶ್ವ-ರ್ಯ-ವನ್ನು
ಐಶ್ವ-ರ್ಯ-ವಾ-ಗಿ-ರ-ಬ-ಹುದು
ಐಶ್ವ-ರ್ಯ-ವಿ-ರಲಿ
ಐಶ್ವ-ರ್ಯವೂ
ಒಂಟಿ
ಒಂಟಿ-ವರೆ
ಒಂದ-ಕ್ಕಿಂತ
ಒಂದಕ್ಕೂ
ಒಂದಕ್ಕೆ
ಒಂದಕ್ಕೇ
ಒಂದ-ನೆಯ
ಒಂದನ್ನು
ಒಂದನ್ನೂ
ಒಂದನ್ನೆ
ಒಂದನ್ನೇ
ಒಂದ-ನ್ನೊಂ-ದನ್ನು
ಒಂದ-ನ್ನೊಂದು
ಒಂದರ
ಒಂದ-ರಂತೆ
ಒಂದ-ರ-ಮೇ-ಲೊಂ-ದನ್ನು
ಒಂದ-ರಲ್ಲಿ
ಒಂದ-ರ-ಲ್ಲಿಯೋ
ಒಂದ-ರ-ಲ್ಲಿ-ರು-ವುದೇ
ಒಂದ-ರಲ್ಲೇ
ಒಂದ-ರಷ್ಟೇ
ಒಂದರಿ
ಒಂದ-ರಿಂದ
ಒಂದ-ರಿಂ-ದಲೇ
ಒಂದ-ರಿ-ಯಿಂದ
ಒಂದಲ್ಲ
ಒಂದ-ಲ್ಲ-ದಿ-ದ್ದರೆ
ಒಂದ-ಲ್ಲದೆ
ಒಂದ-ಲ್ಲದೇ
ಒಂದಾ-ಗ-ಬಲ್ಲ
ಒಂದಾ-ಗ-ಬೇ-ಕಾಗಿ
ಒಂದಾ-ಗ-ಬೇಕು
ಒಂದಾ-ಗಲು
ಒಂದಾಗಿ
ಒಂದಾ-ಗಿ-ದ್ದರೆ
ಒಂದಾ-ಗಿ-ಬಿ-ಡು-ವನು
ಒಂದಾ-ಗಿ-ರು-ವುದು
ಒಂದಾ-ಗಿ-ಹೋ-ಗು-ವುದು
ಒಂದಾ-ಗು-ತ್ತ-ದೆಯೋ
ಒಂದಾ-ಗು-ತ್ತಾನೆ
ಒಂದಾ-ಗು-ತ್ತೇವೆ
ಒಂದಾ-ಗುವ
ಒಂದಾ-ಗು-ವಂತೆ
ಒಂದಾ-ಗು-ವನು
ಒಂದಾ-ಗು-ವ-ವ-ರೆಗೆ
ಒಂದಾ-ಗು-ವು-ದಕ್ಕೆ
ಒಂದಾ-ಗು-ವುದು
ಒಂದಾ-ಗು-ವುದೊ
ಒಂದಾ-ಗು-ವುವು
ಒಂದಾ-ಗು-ವೆವು
ಒಂದಾದ
ಒಂದಾ-ದ-ಮೇಲೆ
ಒಂದಾ-ದ-ಮೇ-ಲೊಂದು
ಒಂದಿ
ಒಂದಿ-ದ್ದರೆ
ಒಂದಿ-ರುವ
ಒಂದಿ-ರು-ವ-ವ-ನಿಗೆ
ಒಂದಿ-ರು-ವಾಗ
ಒಂದಿ-ರು-ವಾ-ಗಲೇ
ಒಂದಿ-ಲ್ಲ-ದಿ-ದ್ದರೆ
ಒಂದಿ-ಲ್ಲದೆ
ಒಂದಿಷ್ಟು
ಒಂದು
ಒಂದು-ಆ
ಒಂದು-ಕಡೆ
ಒಂದು-ಗೂಡಿ
ಒಂದು-ಗೂ-ಡಿಸು
ಒಂದು-ಗೂ-ಡಿ-ಸು-ವು-ದಕ್ಕೆ
ಒಂದು-ಗೂ-ಡಿ-ಸು-ವುದು
ಒಂದು-ವೇಳೆ
ಒಂದು-ಸಲ
ಒಂದೂ
ಒಂದೂ-ರಿ-ನಲ್ಲಿ
ಒಂದೆ
ಒಂದೆ-ಡೆ-ಯಿಂದ
ಒಂದೆ-ಯಾಗಿ
ಒಂದೆ-ರಡು
ಒಂದೆ-ರೆಡು
ಒಂದೇ
ಒಂದೇ-ರೀತಿ
ಒಂದೇ-ಸ-ಮ-ನಾಗಿ
ಒಂದೇ-ಸ-ಮ-ನಾ-ಗಿ-ರು-ವನು
ಒಂದೇ-ಸಾರಿ
ಒಂದೊಂ-ದಕ್ಕೂ
ಒಂದೊಂ-ದ-ನ್ನಾಗಿ
ಒಂದೊಂ-ದನ್ನು
ಒಂದೊಂ-ದ-ರಲ್ಲಿ
ಒಂದೊಂ-ದಾಗಿ
ಒಂದೊಂದು
ಒಂದೊಂದೂ
ಒಂಬತ್ತು
ಒಗರು
ಒಗೆದ
ಒಗೆ-ಯ-ವರು
ಒಗೆಯು
ಒಗ್ಗ-ರ-ಣೆ-ಯಂತೆ
ಒಗ್ಗಿದೆ
ಒಗ್ಗಿ-ರ-ಬೇಕು
ಒಗ್ಗಿ-ರು-ವುದೊ
ಒಗ್ಗಿಲ್ಲ
ಒಗ್ಗಿ-ಸಿ-ಕೊ-ಳ್ಳ-ಬೇ-ಕಾ-ಗಿದೆ
ಒಗ್ಗಿ-ಹೋ-ಗಿದೆ
ಒಗ್ಗಿ-ಹೋ-ಗು-ವೆವು
ಒಗ್ಗುವ
ಒಗ್ಗು-ವು-ದಿಲ್ಲ
ಒಗ್ಗು-ವುದು
ಒಟ-ಗು-ಟ್ಟು-ವು-ದ-ರೊ-ಳಗೆ
ಒಟ್ಟನ್ನು
ಒಟ್ಟಾಗಿ
ಒಟ್ಟಿ-ಗಾಗಿ
ಒಟ್ಟಿಗೆ
ಒಟ್ಟಿನ
ಒಟ್ಟಿ-ನಲ್ಲಿ
ಒಟ್ಟು
ಒಟ್ಟು-ಗೂಡಿ
ಒಟ್ಟು-ಗೂ-ಡಿ-ಸಿ-ದರೆ
ಒಟ್ಟು-ತ್ತಿ-ರು-ವುದೇ
ಒಡಂ-ಬ-ಡಿ-ಸಲು
ಒಡಂ-ಬ-ಡುವು
ಒಡಂ-ಬ-ಡು-ವು-ದಿಲ್ಲ
ಒಡ-ನಾ-ಡಿ-ಗ-ಳಾ-ದರೆ
ಒಡ-ನೆಯೆ
ಒಡ-ನೆಯೇ
ಒಡವೆ
ಒಡ-ವೆಗೆ
ಒಡ-ವೆ-ಯನ್ನು
ಒಡ-ಹು-ಟ್ಟಿದ
ಒಡೆ-ತನ
ಒಡೆ-ತ-ನ-ವನ್ನು
ಒಡೆದ
ಒಡೆ-ದರೆ
ಒಡೆ-ದಾಗ
ಒಡೆ-ದಿ-ರ-ಬ-ಹುದು
ಒಡೆದು
ಒಡೆ-ದು-ಹಾ-ಕ-ಲಾ-ರೆವೊ
ಒಡೆ-ದು-ಹಾಕಿ
ಒಡೆ-ದು-ಹಾ-ಕಿ-ದರೆ
ಒಡೆ-ದು-ಹೋಗಿ
ಒಡೆ-ದು-ಹೋ-ಗಿ-ದೆಯೊ
ಒಡೆ-ದು-ಹೋ-ಗು-ವುದು
ಒಡೆ-ದು-ಹೋದ
ಒಡೆ-ದು-ಹೋ-ದಂತೆ
ಒಡೆ-ದು-ಹೋ-ದರೆ
ಒಡೆಯ
ಒಡೆ-ಯ-ನಲ್ಲ
ಒಡೆ-ಯ-ನ-ಲ್ಲವೊ
ಒಡೆ-ಯ-ನ-ವನು
ಒಡೆ-ಯ-ನಾಗ
ಒಡೆ-ಯ-ನಾಗಿ
ಒಡೆ-ಯ-ನಾ-ಗಿ-ದ್ದರೂ
ಒಡೆ-ಯ-ನಾ-ಗಿ-ರು-ವ-ವನೇ
ಒಡೆ-ಯ-ನಾದ
ಒಡೆ-ಯ-ನಾ-ದರೆ
ಒಡೆ-ಯ-ನಾ-ದ-ವನು
ಒಡೆ-ಯ-ನಿಗೆ
ಒಡೆ-ಯನು
ಒಡೆ-ಯನೆ
ಒಡೆ-ಯ-ನೆಂಬ
ಒಡೆ-ಯನೇ
ಒಡೆ-ಯರು
ಒಡೆ-ಯು-ತ್ತೇ-ವೆಯೊ
ಒಡೆ-ಯು-ವ-ವರ
ಒಡೆ-ಯು-ವುದು
ಒಡ್ಡ-ಬಾ-ರ-ದಾ-ಗಿತ್ತು
ಒಡ್ಡರ
ಒಡ್ಡಿ-ದರೆ
ಒಡ್ಡಿ-ರು-ತ್ತೇವೆ
ಒಡ್ಡು-ತ್ತಾನೆ
ಒಡ್ಡು-ತ್ತಿ-ದ್ದರು
ಒಡ್ಡು-ತ್ತಿ-ರು-ವ-ವನೇ
ಒಡ್ಡು-ತ್ತೇ-ವೆಯೋ
ಒಡ್ಡು-ವ-ವ-ನಲ್ಲ
ಒಡ್ಡು-ವು-ದಿಲ್ಲ
ಒಡ್ಡು-ವು-ದಿ-ಲ್ಲವೊ
ಒಡ್ಡು-ವು-ದಿ-ಲ್ಲವೋ
ಒಡ್ಡು-ವೆವು
ಒಣ
ಒಣ-ಕಲು
ಒಣ-ಕಲೂ
ಒಣ-ಗಲು
ಒಣಗಿ
ಒಣ-ಗಿದ
ಒಣ-ಗಿ-ಸ-ಬ-ಹುದು
ಒಣ-ಗಿ-ಸ-ಬೇಕು
ಒಣ-ಗಿ-ಸ-ಲಾ-ರದು
ಒಣ-ಗಿಸಿ
ಒಣ-ಗಿ-ಸಿ-ದಾ-ಗಲೇ
ಒಣ-ಗಿ-ಸು-ತ್ತೇ-ವೆಯೋ
ಒಣ-ಗಿ-ಸು-ವುದ
ಒಣ-ಗಿ-ಹೋ-ಗಿದೆ
ಒಣ-ಗಿ-ಹೋ-ಗುತ್ತ
ಒಣ-ಗಿ-ಹೋ-ಗು-ವುದು
ಒಣ-ಗಿ-ಹೋದ
ಒಣ-ಗಿ-ಹೋ-ದ-ಮೇಲೆ
ಒಣ-ಗು-ತ್ತಿದೆ
ಒಣ-ಗು-ವುದು
ಒತ್ತಡ
ಒತ್ತ-ಡಕ್ಕೆ
ಒತ್ತಿ
ಒತ್ತು-ತ್ತೇ-ವೆಯೊ
ಒತ್ತು-ವುದು
ಒತ್ತೆ
ಒತ್ತೆ-ಯಿ-ಡು-ತ್ತಿ-ರು-ವೆವು
ಒದ-ಗ-ಬೇಕು
ಒದಗಿ
ಒದ-ಗಿ-ಬ-ರು-ವುವು
ಒದ-ಗಿ-ರುವ
ಒದ-ಗಿ-ಸ-ಬೇ-ಕಾ-ಗಿದೆ
ಒದ-ಗಿ-ಸಲು
ಒದ-ಗಿಸಿ
ಒದ-ಗಿಸು
ಒದ-ಗಿ-ಸು-ತ್ತಾನೆ
ಒದ-ಗಿ-ಸು-ತ್ತಿದೆ
ಒದ-ಗಿ-ಸು-ತ್ತಿ-ರು-ವನು
ಒದ-ಗಿ-ಸು-ವನು
ಒದ-ಗಿ-ಸು-ವು-ದ-ಕ್ಕಾಗಿ
ಒದ-ಗಿ-ಸು-ವು-ದಕ್ಕೆ
ಒದ-ಗಿ-ಸು-ವುದು
ಒದ-ಗಿ-ಸು-ವುದೇ
ಒದ-ಗಿ-ಸು-ವುವು
ಒದ-ಗೀ-ಸು-ವು-ದಕ್ಕೇ
ಒದ-ಗು-ತ್ತಿ-ದ್ದರೂ
ಒದ-ಗು-ವುದು
ಒದ-ರ-ಬೇಕು
ಒದರಿ
ಒದ-ರಿ-ದರೂ
ಒದೆ-ದರೆ
ಒದೆ-ಯು-ತ್ತದೆ
ಒದೆ-ಯು-ವನು
ಒದೆ-ಸಿ-ಕೊಂ-ಡರೂ
ಒದೆ-ಸಿ-ಕೊ-ಳ್ಳಲು
ಒದ್ದಾಗ
ಒದ್ದಾ-ಡ-ಬೇ-ಕಾ-ಗು-ವುದು
ಒದ್ದಾ-ಡು-ತ್ತಿ-ದ್ದರೆ
ಒದ್ದಾ-ಡು-ತ್ತಿ-ರುವ
ಒದ್ದಾ-ಡು-ತ್ತೇವೆ
ಒದ್ದಾ-ಡುವ
ಒದ್ದಾ-ಡು-ವನು
ಒದ್ದಾ-ಡು-ವು-ದಿಲ್ಲ
ಒದ್ದಾ-ಡು-ವೆವು
ಒದ್ದು
ಒದ್ದೆ-ಯಾ-ಗ-ದಿ-ರಲಿ
ಒದ್ದೆ-ಯಾ-ಗು-ವು-ದಿಲ್ಲ
ಒಪ್ಪದೆ
ಒಪ್ಪದೇ
ಒಪ್ಪ-ಲೇ-ಬೇಕು
ಒಪ್ಪಿ
ಒಪ್ಪಿಕೊ
ಒಪ್ಪಿ-ಕೊಂಡ
ಒಪ್ಪಿ-ಕೊಂ-ಡನೇ
ಒಪ್ಪಿ-ಕೊಂ-ಡರೂ
ಒಪ್ಪಿ-ಕೊಂ-ಡರೆ
ಒಪ್ಪಿ-ಕೊಂ-ಡಿ-ರು-ವನು
ಒಪ್ಪಿ-ಕೊಂ-ಡಿಲ್ಲ
ಒಪ್ಪಿ-ಕೊ-ಳ್ಳದೆ
ಒಪ್ಪಿ-ಕೊ-ಳ್ಳದೇ
ಒಪ್ಪಿ-ಕೊ-ಳ್ಳ-ಬೇ-ಕಾ-ಗು-ವುದು
ಒಪ್ಪಿ-ಕೊ-ಳ್ಳ-ಬೇಕು
ಒಪ್ಪಿ-ಕೊ-ಳ್ಳು-ತ್ತಾನೆ
ಒಪ್ಪಿ-ಕೊ-ಳ್ಳು-ತ್ತಾರೆ
ಒಪ್ಪಿ-ಕೊ-ಳ್ಳು-ತ್ತೇವೆ
ಒಪ್ಪಿ-ಕೊ-ಳ್ಳು-ತ್ತೇ-ವೆಯೊ
ಒಪ್ಪಿ-ಕೊ-ಳ್ಳು-ವ-ವ-ನಲ್ಲ
ಒಪ್ಪಿ-ಕೊ-ಳ್ಳು-ವು-ದಕ್ಕೆ
ಒಪ್ಪಿ-ಕೊ-ಳ್ಳು-ವು-ದಿಲ್ಲ
ಒಪ್ಪಿ-ಕೊ-ಳ್ಳು-ವುದು
ಒಪ್ಪಿ-ಕೊ-ಳ್ಳು-ವೆವು
ಒಪ್ಪಿಗೆ
ಒಪ್ಪಿ-ಗೆ-ಯನ್ನು
ಒಪ್ಪಿ-ದಂತೆ
ಒಪ್ಪಿ-ಸಲು
ಒಪ್ಪು-ತ್ತವೆ
ಒಪ್ಪು-ತ್ತಾರೆ
ಒಪ್ಪು-ತ್ತೇ-ವೆಯೊ
ಒಪ್ಪುವ
ಒಪ್ಪು-ವ-ವರು
ಒಪ್ಪು-ವು-ದಿಲ್ಲ
ಒಪ್ಪೊತ್ತು
ಒಬ್ಬ
ಒಬ್ಬನ
ಒಬ್ಬ-ನಂತೆ
ಒಬ್ಬ-ನದು
ಒಬ್ಬ-ನನ್ನು
ಒಬ್ಬ-ನಲ್ಲಿ
ಒಬ್ಬ-ನ-ಲ್ಲಿ-ದ್ದರೆ
ಒಬ್ಬ-ನಷ್ಟೇ
ಒಬ್ಬ-ನಿಂದ
ಒಬ್ಬ-ನಿ-ಗಿಂತ
ಒಬ್ಬ-ನಿಗೆ
ಒಬ್ಬ-ನಿಗೇ
ಒಬ್ಬ-ನಿ-ರು-ವನು
ಒಬ್ಬನು
ಒಬ್ಬ-ನು-ಮು-ಕ್ತ-ನಾಗಿ
ಒಬ್ಬನೆ
ಒಬ್ಬನೇ
ಒಬ್ಬನೊ
ಒಬ್ಬರ
ಒಬ್ಬ-ರಂತೆ
ಒಬ್ಬ-ರ-ನ್ನೊ-ಬ್ಬರು
ಒಬ್ಬ-ರಲ್ಲಿ
ಒಬ್ಬ-ರಿಂದ
ಒಬ್ಬ-ರಿ-ಗಿಂತ
ಒಬ್ಬರು
ಒಬ್ಬರೆ
ಒಬ್ಬರೇ
ಒಬ್ಬೊ
ಒಬ್ಬೊಬ್ಬ
ಒಬ್ಬೊ-ಬ್ಬ-ನಲ್ಲಿ
ಒಬ್ಬೊ-ಬ್ಬ-ನಿಂದ
ಒಬ್ಬೊ-ಬ್ಬ-ನಿಗೆ
ಒಬ್ಬೊ-ಬ್ಬನು
ಒಬ್ಬೊ-ಬ್ಬನೂ
ಒಬ್ಬೊ-ಬ್ಬ-ರನ್ನು
ಒಬ್ಬೊ-ಬ್ಬ-ರಲ್ಲಿ
ಒಬ್ಬೊ-ಬ್ಬ-ರಿಗೆ
ಒಬ್ಬೊ-ಬ್ಬರು
ಒಮ್ಮೆ
ಒಮ್ಮೆಯೂ
ಒಮ್ಮೆಯೇ
ಒಯ್ದರೆ
ಒಯ್ದಿದೆ
ಒಯ್ಯ-ಬೇಕು
ಒಯ್ಯು-ತ್ತಿ-ರು-ವನು
ಒಯ್ಯುವ
ಒಯ್ಯು-ವಂ-ತಹ
ಒಯ್ಯು-ವಂತೆ
ಒಯ್ಯು-ವನು
ಒಯ್ಯು-ವು-ದಕ್ಕೆ
ಒಯ್ಯು-ವು-ದಿಲ್ಲ
ಒಯ್ಯು-ವುದು
ಒಯ್ಯು-ವುದೇ
ಒಯ್ಯು-ವುದೋ
ಒಯ್ಯು-ವುವು
ಒರ-ಗ-ಲ್ಲಿ-ನ-ಮೇಲೆ
ಒರ-ಗಿ-ಕೊಂ-ಡಿ-ರ-ಬಾ-ರದು
ಒರ-ಟಾ-ಗಿ-ದೆಯೆ
ಒರ-ಟಾ-ಗಿ-ರ-ಬ-ಹುದು
ಒರ-ಟಾದ
ಒರಟು
ಒರ-ಲು-ವ-ವ-ರಂತೆ
ಒರ-ಲು-ವುದೆ
ಒರ-ಳು-ಕ-ಲ್ಲಿಗೆ
ಒರಸಿ
ಒರ-ಸಿಕೊ
ಒರ-ಸಿ-ದ-ರೇನೇ
ಒರೆ
ಒರೆ-ಗ-ಲ್ಲಿಗೆ
ಒರೆ-ಗ-ಲ್ಲಿದೆ
ಒರೆ-ಗ-ಲ್ಲಿನ
ಒರೆಗೆ
ಒರೆ-ಯಿಂದ
ಒರೆ-ಸಿ-ಕೊಂಡು
ಒಲಿದು
ಒಲಿಯ
ಒಲಿ-ಯು-ತ್ತಾನೆ
ಒಲಿ-ಯು-ವು-ದಿಲ್ಲ
ಒಲಿ-ಸಿ-ಕೊಂ-ಡಾಗ
ಒಲಿ-ಸಿ-ಕೊಳ್ಳ
ಒಲಿ-ಸಿ-ಕೊ-ಳ್ಳ-ಬೇ-ಕಾ-ದರೆ
ಒಲಿ-ಸಿ-ಕೊ-ಳ್ಳ-ಬೇಕು
ಒಲಿ-ಸಿ-ಕೊಳ್ಳು
ಒಲೆ
ಒಲೆಗೆ
ಒಲೆಯ
ಒಲೆ-ಯೊ-ಳಗೆ
ಒಲ್ಲ
ಒಲ್ಲೆ
ಒಳ-ಕೊ-ಳ್ಳು-ವಷ್ಟು
ಒಳಕ್ಕೆ
ಒಳ-ಗಡೆ
ಒಳ-ಗಣ್ಣು
ಒಳ-ಗಾ-ಗ-ಬ-ಹುದೋ
ಒಳ-ಗಾಗಿ
ಒಳ-ಗಾ-ಗಿ-ರು-ವುದು
ಒಳ-ಗಾ-ಗಿ-ಲ್ಲದೇ
ಒಳ-ಗಾಗು
ಒಳ-ಗಾ-ಗು-ತ್ತಾನೆ
ಒಳ-ಗಾ-ಗು-ತ್ತಿ-ರು-ವೆನು
ಒಳ-ಗಾ-ಗು-ತ್ತೇವೆ
ಒಳ-ಗಾ-ಗು-ವನು
ಒಳ-ಗಾ-ಗು-ವು-ದಿಲ್ಲ
ಒಳ-ಗಾ-ಗು-ವುದು
ಒಳ-ಗಾ-ಗು-ವೆವು
ಒಳ-ಗಾದ
ಒಳ-ಗಿನ
ಒಳ-ಗಿ-ನ-ದ-ಕ್ಕಿಂತ
ಒಳ-ಗಿ-ನ-ದನ್ನು
ಒಳ-ಗಿ-ನದು
ಒಳ-ಗಿ-ನಿಂದ
ಒಳ-ಗಿ-ನಿಂ-ದಲೇ
ಒಳ-ಗಿ-ರುವ
ಒಳ-ಗಿ-ರು-ವುದನ್ನು
ಒಳ-ಗಿ-ರು-ವು-ದ-ನ್ನೆಲ್ಲಾ
ಒಳ-ಗಿ-ರು-ವುದು
ಒಳ-ಗಿ-ರು-ವು-ದೆಲ್ಲ
ಒಳಗು
ಒಳ-ಗು-ಮಾ-ಡು-ತ್ತವೆ
ಒಳ-ಗು-ಮಾ-ಡುವ
ಒಳ-ಗು-ಮಾ-ಡು-ವು-ದಿಲ್ಲ
ಒಳಗೂ
ಒಳಗೆ
ಒಳ-ಗೆಯೂ
ಒಳ-ಗೆಲ್ಲಾ
ಒಳಗೇ
ಒಳ-ಗೊಂಡ
ಒಳ-ಗೊಂ-ಡಿದೆ
ಒಳ-ಗೊಂ-ಡಿ-ದೆಯೋ
ಒಳ-ಗೊಂ-ಡಿ-ರು-ವನು
ಒಳ-ಗೊಂಡು
ಒಳ-ಗೊಂದು
ಒಳ-ಗೊ-ಳ್ಳ-ಲಾ-ರದು
ಒಳ-ಗೊ-ಳ್ಳು-ವುದು
ಒಳ-ಪಟ್ಟ
ಒಳ-ಪ-ಟ್ಟಿ-ಲ್ಲದ
ಒಳ-ಪಟ್ಟು
ಒಳ-ಪ-ಡಿ-ಸಿ-ದಾಗ
ಒಳ-ಪ-ಡು-ವುದು
ಒಳ-ಮುಖ
ಒಳ-ಹೊಕ್ಕು
ಒಳ್ಳೆ
ಒಳ್ಳೆಯ
ಒಳ್ಳೆ-ಯದ
ಒಳ್ಳೆ-ಯ-ದಕ್ಕೆ
ಒಳ್ಳೆ-ಯ-ದನ್ನು
ಒಳ್ಳೆ-ಯ-ದನ್ನೆ
ಒಳ್ಳೆ-ಯ-ದ-ನ್ನೆಲ್ಲ
ಒಳ್ಳೆ-ಯ-ದ-ರಂತೆ
ಒಳ್ಳೆ-ಯ-ದ-ರಲ್ಲಿ
ಒಳ್ಳೆ-ಯ-ದ-ರಿಂದ
ಒಳ್ಳೆ-ಯ-ದಲ್ಲ
ಒಳ್ಳೆ-ಯ-ದ-ಲ್ಲದೆ
ಒಳ್ಳೆ-ಯ-ದಾ-ಗಲಿ
ಒಳ್ಳೆ-ಯ-ದಾ-ಗಿ-ರ-ಬೇಕು
ಒಳ್ಳೆ-ಯ-ದಾಗು
ಒಳ್ಳೆ-ಯ-ದಾ-ಗು-ತ್ತದೆ
ಒಳ್ಳೆ-ಯ-ದಾ-ಗು-ವು-ದಿಲ್ಲ
ಒಳ್ಳೆ-ಯ-ದಾ-ಗು-ವುದು
ಒಳ್ಳೆ-ಯ-ದಾ-ಗು-ವುದೋ
ಒಳ್ಳೆ-ಯ-ದಾ-ದರೂ
ಒಳ್ಳೆ-ಯ-ದಾ-ದರೆ
ಒಳ್ಳೆ-ಯ-ದಿ-ರ-ಬ-ಹುದು
ಒಳ್ಳೆ-ಯದು
ಒಳ್ಳೆ-ಯದೆ
ಒಳ್ಳೆ-ಯ-ದೆಂದು
ಒಳ್ಳೆ-ಯದೇ
ಒಳ್ಳೆ-ಯ-ದೇನೂ
ಒಳ್ಳೆ-ಯವ
ಒಳ್ಳೆ-ಯ-ವ-ನಾ-ಗು-ತ್ತೇನೆ
ಒಳ್ಳೆ-ಯ-ವ-ನಿ-ರ-ಬ-ಹುದು
ಒಳ್ಳೆ-ಯ-ವನು
ಒಳ್ಳೆ-ಯ-ವ-ರಾ-ಗು-ವು-ದಕ್ಕೆ
ಒಳ್ಳೆ-ಯ-ವರೊ
ಒಳ್ಳೆ-ಯವು
ಒಳ್ಳೆ-ಯವೂ
ಒಳ್ಳೊ-ಳ್ಳೆಯ
ಓ
ಓಂ
ಓಂಕಾರ
ಓಂಕಾ-ರ-ದಷ್ಟು
ಓಂಕಾ-ರ-ವನ್ನು
ಓಂಕಾ-ರ-ವೆಂ-ಬುದು
ಓಂಕಾ-ರವೇ
ಓಜ-ಸ್ಸಿ-ನಿಂದ
ಓಜಸ್ಸು
ಓಟಿಗೆ
ಓಟೆ
ಓಟೆ-ಯನ್ನು
ಓಡ-ಬೇ-ಕಾ-ದರೆ
ಓಡ-ಬೇಕು
ಓಡಾ-ಡ-ಬೇಕು
ಓಡಾ-ಡು-ವುದು
ಓಡಿ
ಓಡಿದ
ಓಡಿ-ದರೆ
ಓಡಿದ್ದು
ಓಡಿದ್ದೇ
ಓಡಿ-ಬಂದು
ಓಡಿ-ಸದೆ
ಓಡಿ-ಸ-ಬ-ಲ್ಲನೆ
ಓಡಿ-ಸ-ಬೇ-ಕಾ-ಗಿದೆ
ಓಡಿ-ಸ-ಬೇಕು
ಓಡಿಸಿ
ಓಡಿ-ಸಿದ
ಓಡಿ-ಸಿ-ಬಿ-ಡು-ವು-ದ-ಕ್ಕಾ-ಗು-ವು-ದಿಲ್ಲ
ಓಡಿ-ಸಿ-ರು-ವನು
ಓಡಿ-ಸು-ವಂತೆ
ಓಡಿ-ಸು-ವರು
ಓಡಿ-ಸು-ವು-ದಕ್ಕೆ
ಓಡಿ-ಹೋ-ಗಲು
ಓಡಿ-ಹೋಗಿ
ಓಡಿ-ಹೋ-ಗು-ತ್ತಿ-ರು-ವರು
ಓಡಿ-ಹೋ-ಗುವ
ಓಡಿ-ಹೋ-ಗು-ವರು
ಓಡಿ-ಹೋ-ಗು-ವ-ವರ
ಓಡಿ-ಹೋ-ಗು-ವ-ವ-ರಲ್ಲ
ಓಡಿ-ಹೋ-ಗು-ವು-ದಕ್ಕೆ
ಓಡಿ-ಹೋ-ಗು-ವುದನ್ನು
ಓಡಿ-ಹೋ-ಗು-ವುದು
ಓಡಿ-ಹೋ-ಗು-ವುದೊ
ಓಡಿ-ಹೋ-ದರು
ಓಡಿ-ಹೋ-ದರೆ
ಓಡು-ತ್ತಾನೆ
ಓಡು-ತ್ತಿ-ರು-ವರು
ಓಡುವ
ಓಡು-ವನು
ಓಡು-ವುದು
ಓತ
ಓತ-ಪ್ರೋತ
ಓತ-ಪ್ರೋ-ತ-ನಾಗಿ
ಓತ-ಪ್ರೋ-ತ-ನಾ-ಗಿ-ದ್ದಾನೆ
ಓತ-ಪ್ರೋ-ತ-ನಾ-ಗಿ-ರು-ವನು
ಓತ-ಪ್ರೋ-ತ-ನಾದ
ಓತ-ಪ್ರೋ-ತ-ವಾಗಿ
ಓತ-ಪ್ರೋ-ತ-ವಾ-ಗಿದೆ
ಓತ-ಪ್ರೋ-ತ-ವಾ-ಗಿರು
ಓತ-ಪ್ರೋ-ತ-ವಾ-ಗಿ-ರು-ವ-ವನೆ
ಓತ-ಪ್ರೋ-ತ-ವಾ-ಗಿ-ರು-ವುದನ್ನು
ಓತ-ಪ್ರೋ-ತ-ವಾ-ಗಿ-ರು-ವುದು
ಓದ-ಕೂ-ಡದು
ಓದ-ಕೂ-ಡ-ದೆಂದು
ಓದದೆ
ಓದ-ಬ-ಹುದು
ಓದ-ಬೇ-ಕಾ-ಗಿಲ್ಲ
ಓದ-ಬೇ-ಕಾ-ಗು-ವಂತೆ
ಓದ-ಬೇಕು
ಓದಲು
ಓದಿ
ಓದಿದ
ಓದಿ-ದಂತೆ
ಓದಿ-ದ-ವ-ನಲ್ಲ
ಓದಿ-ದ-ವನು
ಓದಿ-ದಷ್ಟು
ಓದಿ-ದಾಗ
ಓದಿ-ದಾ-ಗಲೂ
ಓದಿ-ದೊ-ಡ-ನೆಯೇ
ಓದಿದ್ದ
ಓದಿದ್ದು
ಓದಿ-ದ್ದೇವೆ
ಓದಿ-ಬಿ-ಟ್ಟಕೆ
ಓದಿ-ರ-ಬ-ಹುದು
ಓದಿ-ರ-ಬೇಕು
ಓದಿ-ರು-ವನು
ಓದಿ-ರು-ವುದನ್ನು
ಓದಿ-ರು-ವುದು
ಓದಿ-ಲ್ಲದೆ
ಓದು
ಓದು-ಗರು
ಓದು-ತ್ತಾನೆ
ಓದು-ತ್ತಾರೊ
ಓದು-ತ್ತಿ-ದ್ದರೂ
ಓದು-ತ್ತಿ-ರುವ
ಓದು-ತ್ತಿ-ರು-ವನು
ಓದು-ತ್ತಿ-ರು-ವರು
ಓದು-ತ್ತೇನೆ
ಓದು-ತ್ತೇವೆ
ಓದು-ತ್ತೇ-ವೆಯೊ
ಓದುವ
ಓದು-ವನು
ಓದು-ವ-ವ-ನಿಗೆ
ಓದು-ವ-ವನು
ಓದು-ವ-ವನೂ
ಓದು-ವ-ವರ
ಓದು-ವ-ವರೂ
ಓದು-ವಾಗ
ಓದುವು
ಓದು-ವು-ದಕ್ಕೆ
ಓದು-ವು-ದ-ರಿಂದ
ಓದು-ವು-ದಲ್ಲ
ಓದು-ವು-ದಿಲ್ಲ
ಓದು-ವುದು
ಓದೋಣ
ಓಬೀ-ರಾ-ಯನ
ಓಮಿ-ತ್ಯೇ-ಕಾ-ಕ್ಷರಂ
ಓರ-ಗೆ-ಯ-ವ-ರಿಗೆ
ಓಲಾ-ಡು-ವರು
ಔಚಿ-ತ್ಯ-ವಾ-ಗಿ-ರು-ವುದೂ
ಔದಾರ್ಯ
ಔದಾ-ರ್ಯಕ್ಕೆ
ಔದಾ-ರ್ಯತೆ
ಔದಾ-ರ್ಯ-ತೆಯ
ಔದಾ-ರ್ಯ-ತೆ-ಯನ್ನು
ಔದಾ-ಸೀ-ನ್ಯ-ಭಾ-ವ-ವನ್ನು
ಔನ್ಸ್
ಔಷಧ
ಔಷ-ಧ-ದಿಂದ
ಔಷ-ಧ-ವನ್ನು
ಔಷಧಿ
ಔಷ-ಧಿ-ಗಳನ್ನು
ಔಷ-ಧಿ-ಗ-ಳಿ-ಗೆಲ್ಲ
ಔಷ-ಧಿಯ
ಔಷ-ಧಿ-ಯಂತೆ
ಔಷ-ಧಿ-ಯನ್ನು
ಔಷ-ಧಿ-ಯ-ನ್ನೇನೊ
ಕ
ಕಂ
ಕಂಗೊ-ಳಿಸು
ಕಂಗೊ-ಳಿ-ಸು-ತ್ತಿ-ದ್ದರೆ
ಕಂಟ-ಕ-ಗ-ಳಾದ
ಕಂಟ-ಕ-ನಾದ
ಕಂಟ-ಕ-ಪ್ರಾ-ಯ-ರಾ-ಗಿ-ದ್ದ-ವರು
ಕಂಟ-ಕ-ಮಯ
ಕಂಟ-ಕ-ರಾ-ಗು-ತ್ತಾರೆ
ಕಂಟ-ಕರು
ಕಂಟ-ಕ-ವಾ-ಗುವ
ಕಂಟ-ಕವೂ
ಕಂಠ
ಕಂಠ-ದಿಂದ
ಕಂಠ-ಪಾಠ
ಕಂಠ-ಶೋ-ಷ-ಣೆ-ಯಾ-ಗು-ವುದು
ಕಂಡ
ಕಂಡಂತೆ
ಕಂಡಂ-ತೆಯೇ
ಕಂಡ-ರಂತೂ
ಕಂಡ-ರಿ-ಯದು
ಕಂಡರು
ಕಂಡರೂ
ಕಂಡರೆ
ಕಂಡಳು
ಕಂಡ-ವರ
ಕಂಡಾಗ
ಕಂಡಿಯ
ಕಂಡಿ-ರು-ವನು
ಕಂಡಿ-ರು-ವರೊ
ಕಂಡಿ-ರು-ವರೋ
ಕಂಡಿ-ರು-ವಿರಾ
ಕಂಡಿ-ರುವೆ
ಕಂಡಿ-ರು-ವೆನು
ಕಂಡಿಲ್ಲ
ಕಂಡಿ-ಹಿ-ಡಿ-ಯ-ಬೇ-ಕಾ-ಗಿದೆ
ಕಂಡು
ಕಂಡು-ದಕ್ಕೆ
ಕಂಡು-ದನ್ನು
ಕಂಡು-ಹಿ-ಡಿದ
ಕಂಡು-ಹಿ-ಡಿ-ದನು
ಕಂಡು-ಹಿ-ಡಿ-ದನೊ
ಕಂಡು-ಹಿ-ಡಿ-ದ-ಮೇಲೆ
ಕಂಡು-ಹಿ-ಡಿ-ದರು
ಕಂಡು-ಹಿ-ಡಿ-ದ-ವನು
ಕಂಡು-ಹಿ-ಡಿ-ದಿರು
ಕಂಡು-ಹಿ-ಡಿ-ದಿ-ರು-ವನು
ಕಂಡು-ಹಿ-ಡಿ-ದಿ-ರು-ವರು
ಕಂಡು-ಹಿ-ಡಿ-ದಿಲ್ಲ
ಕಂಡು-ಹಿ-ಡಿದು
ಕಂಡು-ಹಿ-ಡಿ-ದುದು
ಕಂಡು-ಹಿ-ಡಿದೆ
ಕಂಡು-ಹಿ-ಡಿಯ
ಕಂಡು-ಹಿ-ಡಿ-ಯ-ಬಲ್ಲ
ಕಂಡು-ಹಿ-ಡಿ-ಯ-ಬ-ಲ್ಲರು
ಕಂಡು-ಹಿ-ಡಿ-ಯ-ಬ-ಲ್ಲುದು
ಕಂಡು-ಹಿ-ಡಿ-ಯ-ಬ-ಹುದು
ಕಂಡು-ಹಿ-ಡಿ-ಯ-ಬೇ-ಕಾ-ಗಿಲ್ಲ
ಕಂಡು-ಹಿ-ಡಿ-ಯ-ಬೇ-ಕಾ-ದರೆ
ಕಂಡು-ಹಿ-ಡಿ-ಯ-ಬೇಕು
ಕಂಡು-ಹಿ-ಡಿ-ಯಲು
ಕಂಡು-ಹಿ-ಡಿ-ಯ-ವುದು
ಕಂಡು-ಹಿ-ಡಿಯು
ಕಂಡು-ಹಿ-ಡಿ-ಯು-ತ್ತಲೇ
ಕಂಡು-ಹಿ-ಡಿ-ಯುತ್ತಾ
ಕಂಡು-ಹಿ-ಡಿ-ಯು-ತ್ತಾನೆ
ಕಂಡು-ಹಿ-ಡಿ-ಯು-ತ್ತೇವೆ
ಕಂಡು-ಹಿ-ಡಿ-ಯು-ವನು
ಕಂಡು-ಹಿ-ಡಿ-ಯು-ವಾಗ
ಕಂಡು-ಹಿ-ಡಿ-ಯು-ವು-ದಕ್ಕೆ
ಕಂಡು-ಹಿ-ಡಿ-ಯು-ವು-ದಿಲ್ಲ
ಕಂಡು-ಹಿ-ಡಿ-ಯು-ವುದು
ಕಂಡೆವು
ಕಂಡೇ
ಕಂತೆ
ಕಂತೆ-ಯಂತೆ
ಕಂತೆ-ಯಲ್ಲ
ಕಂತೆ-ಯ-ಲ್ಲದೆ
ಕಂತೆಯೇ
ಕಂದರ್ಪ
ಕಂದರ್ಪಃ
ಕಂದ-ರ್ಪನ
ಕಂದಾಯ
ಕಂಪ-ನ-ಕಾರಿ
ಕಂಪ-ನ-ಕಾ-ರಿ-ಯಾದ
ಕಂಪನ್ನು
ಕಂಪಿದೆ
ಕಂಪಿ-ಸು-ತ್ತಿದೆ
ಕಂಪಿ-ಸುವ
ಕಂಪಿ-ಸು-ವುದು
ಕಂಬ
ಕಂಬ-ದಿಂದ
ಕಂಬನಿ
ಕಂಬ-ನಿ-ದುಂಬಿ
ಕಂಬ-ನಿ-ದುಂ-ಬಿದ
ಕಂಬ-ವನ್ನು
ಕಂಬಿ
ಕಂಬಿಗೆ
ಕಂಭ-ವನ್ನು
ಕಂಸ
ಕಂಸ-ಚಾ-ಣೂ-ರ-ಮ-ರ್ದ-ನಮ್
ಕಂಸನ
ಕಂಸ-ನನ್ನು
ಕಕ್ಕಾ-ಬಿಕ್ಕಿ
ಕಕ್ಕಾ-ಬಿ-ಕ್ಕಿ-ಯಾ-ಗ-ಬೇಕು
ಕಕ್ಕಾ-ಬಿ-ಕ್ಕಿ-ಯಾ-ಗು-ತ್ತಾರೆ
ಕಕ್ಕಾ-ಬಿ-ಕ್ಕಿ-ಯಾ-ಗು-ವುದು
ಕಕ್ಷಿ
ಕಕ್ಷಿಯ
ಕಚ-ಗುಳಿ
ಕಚ-ನನ್ನು
ಕಚೇರಿ
ಕಚೇ-ರಿಯ
ಕಚೇ-ರಿ-ಯಿಂದ
ಕಚ್ಚ
ಕಚ್ಚ-ಲಾ-ರದು
ಕಚ್ಚಲು
ಕಚ್ಚಾ-ಡು-ವೆವು
ಕಚ್ಚಾ-ಸಾ-ಮಾ-ನಿ-ನಂತೆ
ಕಚ್ಚಿದ
ಕಚ್ಚಿ-ದ-ಜ್ಞಾ-ನ-ಸಂ-ಮೋಹಃ
ಕಚ್ಚಿ-ದರೆ
ಕಚ್ಚಿ-ದೇ-ತ-ಚ್ಛ್ರುತಂ
ಕಚ್ಚಿ-ನ್ನೋ-ಭ-ಯ-ವಿ-ಭ್ರ-ಷ್ಟ-ಶ್ಛಿ-ನ್ನಾ-ಭ್ರ-ಮಿವ
ಕಚ್ಚಿ-ರು-ವನು
ಕಚ್ಚಿ-ಹಾ-ಕು-ವುದು
ಕಚ್ಚು-ತ್ತದೆ
ಕಚ್ಚು-ವು-ದಕ್ಕೆ
ಕಚ್ಚು-ವುದು
ಕಛೇರಿ
ಕಛೇ-ರಿಗೆ
ಕಟು-ಕ-ನಂ-ತಿ-ರು-ವನು
ಕಟು-ಕ-ನಂತೆ
ಕಟು-ಕ-ನಾ-ಗ-ಬೇಕೆ
ಕಟು-ಕ-ನಿಗೆ
ಕಟು-ಕ-ರ-ವನು
ಕಟು-ವಾಗಿ
ಕಟ್ಟಡ
ಕಟ್ಟ-ಡಕ್ಕೆ
ಕಟ್ಟ-ಡ-ಗ-ಳಿಗೆ
ಕಟ್ಟ-ಡದ
ಕಟ್ಟ-ಡ-ವನ್ನು
ಕಟ್ಟನ್ನು
ಕಟ್ಟ-ಪ್ಪಣೆ
ಕಟ್ಟ-ಪ್ಪ-ಣೆ-ಯನ್ನು
ಕಟ್ಟ-ಬ-ಹು-ದಾ-ಗಿತ್ತು
ಕಟ್ಟ-ಬೇ-ಕಾ-ದರೆ
ಕಟ್ಟ-ಬೇ-ಕಾ-ಯಿತು
ಕಟ್ಟ-ಬೇಕು
ಕಟ್ಟ-ಲಿಲ್ಲ
ಕಟ್ಟ-ಲ್ಪ-ಟ್ಟ-ವ-ನಾಗಿ
ಕಟ್ಟಿ
ಕಟ್ಟಿ-ಕೊಂಡ
ಕಟ್ಟಿ-ಕೊಂ-ಡಂತೆ
ಕಟ್ಟಿ-ಕೊಂ-ಡರೆ
ಕಟ್ಟಿ-ಕೊಂ-ಡ-ವನು
ಕಟ್ಟಿ-ಕೊಂ-ಡಿತ್ತು
ಕಟ್ಟಿ-ಕೊಂ-ಡಿದೆ
ಕಟ್ಟಿ-ಕೊಂ-ಡಿ-ದ್ದೇವೆ
ಕಟ್ಟಿ-ಕೊಂ-ಡಿರು
ಕಟ್ಟಿ-ಕೊಂ-ಡಿ-ರುವ
ಕಟ್ಟಿ-ಕೊಂ-ಡಿ-ರು-ವನು
ಕಟ್ಟಿ-ಕೊಂ-ಡಿ-ರು-ವನೋ
ಕಟ್ಟಿ-ಕೊಂ-ಡಿ-ರು-ವೆನೋ
ಕಟ್ಟಿ-ಕೊಂ-ಡಿಲ್ಲ
ಕಟ್ಟಿ-ಕೊಂಡು
ಕಟ್ಟಿ-ಕೊ-ಳ್ಳ-ಬೇ-ಕಾ-ಗಿಲ್ಲ
ಕಟ್ಟಿ-ಕೊ-ಳ್ಳ-ಬೇ-ಕಾ-ಗು-ವುದು
ಕಟ್ಟಿ-ಕೊ-ಳ್ಳ-ಬೇಕು
ಕಟ್ಟಿ-ಕೊ-ಳ್ಳ-ಲಿ-ಲ್ಲ-ವಲ್ಲ
ಕಟ್ಟಿ-ಕೊ-ಳ್ಳು-ತ್ತಾರೆ
ಕಟ್ಟಿ-ಕೊ-ಳ್ಳು-ತ್ತೇವೆ
ಕಟ್ಟಿ-ಕೊ-ಳ್ಳು-ವುದು
ಕಟ್ಟಿ-ಕೊ-ಳ್ಳು-ವೆವೊ
ಕಟ್ಟಿಗೆ
ಕಟ್ಟಿ-ಗೆ-ಗಳನ್ನು
ಕಟ್ಟಿ-ಗೆಯ
ಕಟ್ಟಿ-ಗೆ-ಯನ್ನು
ಕಟ್ಟಿ-ಗೆ-ಯಲ್ಲಿ
ಕಟ್ಟಿ-ಗೆ-ಯ-ಲ್ಲಿ-ರುವ
ಕಟ್ಟಿ-ಗೆ-ಯಿಂದ
ಕಟ್ಟಿ-ಟ್ಟದ್ದು
ಕಟ್ಟಿದ
ಕಟ್ಟಿ-ದಂ-ತಾ-ಗಿದೆ
ಕಟ್ಟಿ-ದಂ-ತಾ-ಗು-ವುದು
ಕಟ್ಟಿ-ದಂ-ತಿದೆ
ಕಟ್ಟಿ-ದಂತೆ
ಕಟ್ಟಿ-ದರೆ
ಕಟ್ಟಿ-ದ-ವನು
ಕಟ್ಟಿ-ದೆವು
ಕಟ್ಟಿದ್ದ
ಕಟ್ಟಿ-ದ್ದರು
ಕಟ್ಟಿ-ದ್ದರೆ
ಕಟ್ಟಿ-ದ್ದಾರೆ
ಕಟ್ಟಿನ
ಕಟ್ಟಿ-ರ-ಬ-ಹುದು
ಕಟ್ಟಿ-ರುವ
ಕಟ್ಟಿ-ರು-ವನು
ಕಟ್ಟಿ-ಸ-ಬ-ಹುದು
ಕಟ್ಟಿಸಿ
ಕಟ್ಟಿ-ಸಿ-ಕೊಂ-ಡ-ವ-ರಂತೆ
ಕಟ್ಟಿ-ಸಿ-ದರೆ
ಕಟ್ಟಿ-ಸು-ತ್ತಾನೆ
ಕಟ್ಟಿ-ಸು-ವ-ವನು
ಕಟ್ಟಿ-ಸು-ವುದು
ಕಟ್ಟಿ-ಹಾಕ
ಕಟ್ಟಿ-ಹಾ-ಕ-ಬೇಕು
ಕಟ್ಟಿ-ಹಾ-ಕ-ಲಾ-ರದು
ಕಟ್ಟಿ-ಹಾ-ಕ-ಲಾ-ರವು
ಕಟ್ಟಿ-ಹಾ-ಕಲು
ಕಟ್ಟಿ-ಹಾಕಿ
ಕಟ್ಟಿ-ಹಾ-ಕಿ-ಕೊಂ-ಡಿ-ರು-ವೆನೋ
ಕಟ್ಟಿ-ಹಾ-ಕಿ-ಕೊಂಡು
ಕಟ್ಟಿ-ಹಾ-ಕಿ-ಕೊ-ಳ್ಳು-ತ್ತೇವೆ
ಕಟ್ಟಿ-ಹಾ-ಕಿ-ಕೊ-ಳ್ಳು-ವುದೂ
ಕಟ್ಟಿ-ಹಾ-ಕಿ-ದಂತೆ
ಕಟ್ಟಿ-ಹಾ-ಕಿ-ದರೆ
ಕಟ್ಟಿ-ಹಾ-ಕಿ-ದ್ದರೆ
ಕಟ್ಟಿ-ಹಾ-ಕಿರು
ಕಟ್ಟಿ-ಹಾ-ಕಿ-ರುವ
ಕಟ್ಟಿ-ಹಾ-ಕಿ-ರು-ವುದು
ಕಟ್ಟಿ-ಹಾ-ಕಿ-ರು-ವುದೇ
ಕಟ್ಟಿ-ಹಾಕು
ಕಟ್ಟಿ-ಹಾ-ಕು-ತ್ತೇವೆ
ಕಟ್ಟಿ-ಹಾ-ಕು-ತ್ತೇ-ವೆಯೊ
ಕಟ್ಟಿ-ಹಾ-ಕುವ
ಕಟ್ಟಿ-ಹಾ-ಕು-ವಂ-ತಹ
ಕಟ್ಟಿ-ಹಾ-ಕು-ವಂತೆ
ಕಟ್ಟಿ-ಹಾ-ಕು-ವನು
ಕಟ್ಟಿ-ಹಾ-ಕು-ವ-ವನೇ
ಕಟ್ಟಿ-ಹಾ-ಕು-ವುದನ್ನು
ಕಟ್ಟಿ-ಹಾ-ಕು-ವು-ದಿಲ್ಲ
ಕಟ್ಟಿ-ಹಾ-ಕು-ವುದು
ಕಟ್ಟಿ-ಹಾ-ಕು-ವುದೇ
ಕಟ್ಟಿ-ಹಾ-ಕು-ವುದೊ
ಕಟ್ಟಿ-ಹಾ-ಕು-ವುವು
ಕಟ್ಟಿ-ಹಾ-ಕು-ವೆವೊ
ಕಟ್ಟು
ಕಟ್ಟು-ತ್ತವೆ
ಕಟ್ಟು-ತ್ತಾನೆ
ಕಟ್ಟು-ತ್ತೇವೆ
ಕಟ್ಟು-ನಿ-ಟ್ಟು-ಗಳೂ
ಕಟ್ಟು-ಬೀ-ಳು-ತ್ತಾನೆ
ಕಟ್ಟುವ
ಕಟ್ಟು-ವಂತೆ
ಕಟ್ಟು-ವನು
ಕಟ್ಟು-ವನೆ
ಕಟ್ಟು-ವರು
ಕಟ್ಟು-ವ-ವರು
ಕಟ್ಟು-ವು-ದಕ್ಕೆ
ಕಟ್ಟು-ವು-ದಿಲ್ಲ
ಕಟ್ಟು-ವುದು
ಕಟ್ಟು-ವುದೂ
ಕಟ್ಟು-ವುದೇ
ಕಟ್ಟು-ವುವು
ಕಟ್ಟೆ
ಕಟ್ಟೆಗೆ
ಕಟ್ಟೆ-ಯನ್ನು
ಕಟ್ಟೆ-ಯ-ಲ್ಲಿ-ರುವ
ಕಟ್ಟೆಯೂ
ಕಟ್ಟೆ-ಯೊ-ಡೆದು
ಕಟ್ಟೆ-ಹಾಕಿ
ಕಟ್ವ-ಮ್ಲ-ಲ-ವ-ಣಾ-ತ್ಯು-ಷ್ಣ-ತೀ-ಕ್ಷ್ಣ-ರೂ-ಕ್ಷ-ವಿ-ದಾ-ಹಿನಃ
ಕಠ
ಕಠೋರ
ಕಠೋ-ರ-ನಾ-ಗ-ಬೇಕು
ಕಠೋ-ರನೂ
ಕಠೋ-ರಾಣಿ
ಕಡಮೆ
ಕಡ-ಮೆ-ಯನ್ನು
ಕಡ-ಮೆ-ಯಲ್ಲ
ಕಡ-ಮೆ-ಯಾ-ಗದೆ
ಕಡ-ಮೆ-ಯಾ-ಗ-ಬ-ಹುದು
ಕಡ-ಮೆ-ಯಾಗಿ
ಕಡ-ಮೆ-ಯಾಗಿಯೂ
ಕಡ-ಮೆ-ಯಾ-ಗುತ್ತ
ಕಡ-ಮೆ-ಯಾ-ಗು-ವುದು
ಕಡ-ಮೆ-ಯಾ-ದರೂ
ಕಡ-ಮೆ-ಯಿಲ್ಲ
ಕಡ-ಮೆಯೆ
ಕಡ-ಲ-ಡಿ-ಯಲ್ಲಿ
ಕಡ-ಲಾ-ಳ-ದಲ್ಲಿ
ಕಡ-ಲಿ-ನಷ್ಟು
ಕಡ-ಲಿ-ನಾ-ಳ-ದಲ್ಲಿ
ಕಡ-ಲಿ-ನಿಂದ
ಕಡಲು
ಕಡ-ಲೇ-ಕಾ-ಯಿ-ಯನ್ನು
ಕಡಿ-ದಲ್ಲ
ಕಡಿ-ದ-ಲ್ಲದೆ
ಕಡಿ-ದಿದೆ
ಕಡಿದು
ಕಡಿ-ದು-ಕೊಂ-ಡರೆ
ಕಡಿ-ದು-ಹಾ-ಕು-ತ್ತಿ-ದ-ದರೂ
ಕಡಿ-ದು-ಹೋ-ಗಿಲ್ಲ
ಕಡಿ-ದು-ಹೋ-ಗು-ವು-ದಿಲ್ಲ
ಕಡಿ-ದು-ಹೋ-ಗು-ವುದು
ಕಡಿ-ದು-ಹೋ-ಯಿತೊ
ಕಡಿ-ದೊ-ಡ-ನೆಯೆ
ಕಡಿಮೆ
ಕಡಿ-ಮೆ-ಮಾಡಿ
ಕಡಿ-ಮೆ-ಯನ್ನು
ಕಡಿ-ಮೆ-ಯನ್ನೂ
ಕಡಿ-ಮೆ-ಯಲ್ಲ
ಕಡಿ-ಮೆ-ಯಾ-ಗ-ಬೇಕು
ಕಡಿ-ಮೆ-ಯಾ-ಗಲು
ಕಡಿ-ಮೆ-ಯಾಗಿ
ಕಡಿ-ಮೆ-ಯಾ-ಗುತ್ತ
ಕಡಿ-ಮೆ-ಯಾ-ಗುತ್ತಾ
ಕಡಿ-ಮೆ-ಯಾ-ಗು-ವುದನ್ನು
ಕಡಿ-ಮೆ-ಯಾ-ಗು-ವು-ದಿಲ್ಲ
ಕಡಿ-ಮೆ-ಯಾ-ಗು-ವುದು
ಕಡಿ-ಮೆ-ಯಾ-ಗು-ವುದೂ
ಕಡಿ-ಮೆ-ಯಾ-ಗು-ವುವು
ಕಡಿ-ಮೆ-ಯಾ-ದರೂ
ಕಡಿ-ಮೆ-ಯಾ-ಯಿತು
ಕಡಿ-ಮೆ-ಯಿಲ್ಲ
ಕಡಿ-ಯ-ಬೇ-ಕಾ-ಯಿತು
ಕಡು
ಕಡು-ಬ-ಡ-ತ-ನ-ದಲ್ಲಿ
ಕಡು-ವೈ-ರಿ-ಯಾ-ಗಿ-ದ್ದರೂ
ಕಡೆ
ಕಡೆ-ಗ-ಳಂತೆ
ಕಡೆ-ಗಳಲ್ಲಿ
ಕಡೆ-ಗ-ಳ-ಲ್ಲಿಯೂ
ಕಡೆ-ಗ-ಳ-ಲ್ಲಿ-ರುವ
ಕಡೆ-ಗಳಿಂದ
ಕಡೆ-ಗ-ಳಿ-ಗಿಂತ
ಕಡೆ-ಗ-ಳಿಗೆ
ಕಡೆ-ಗಿಂತ
ಕಡೆ-ಗಿಂ-ತಲೂ
ಕಡೆಗೂ
ಕಡೆಗೆ
ಕಡೆಗೇ
ಕಡೆ-ಗೋ-ಲಿ-ನಿಂದ
ಕಡೆ-ದಾಗ
ಕಡೆ-ದಾ-ಗಲೆ
ಕಡೆಯ
ಕಡೆ-ಯಂತೆ
ಕಡೆ-ಯ-ಬೇಕು
ಕಡೆ-ಯಲ್ಲಿ
ಕಡೆ-ಯ-ಲ್ಲಿಯೂ
ಕಡೆ-ಯಲ್ಲೂ
ಕಡೆ-ಯ-ಲ್ಲೆಲ್ಲ
ಕಡೆ-ಯ-ವರ
ಕಡೆ-ಯ-ವ-ರನ್ನು
ಕಡೆ-ಯ-ವ-ರಿ-ಗೆಲ್ಲ
ಕಡೆ-ಯ-ವರು
ಕಡೆ-ಯ-ವರೂ
ಕಡೆ-ಯಾ-ಗು-ವುದು
ಕಡೆ-ಯಿಂದ
ಕಡೆ-ಯಿಂ-ದಲೂ
ಕಡೆ-ಯುವ
ಕಡೆ-ಯು-ವು-ದನ್ನೇ
ಕಡೆಯೂ
ಕಡೆಯೆ
ಕಡೆ-ಯೆಲ್ಲ
ಕಡೆಯೇ
ಕಡೆ-ಯೇನೂ
ಕಡೆ-ಸಂ-ಗ್ರ-ಹಿಸಿ
ಕಡೆ-ಹಾ-ಯಿ-ಸು-ವನು
ಕಡ್ಡಿ
ಕಡ್ಡಿಯ
ಕಡ್ಡಿ-ಯನ್ನು
ಕಡ್ಡಿ-ಯ-ಪೆ-ಟ್ಟಿ-ಗೆ-ಯಂತೆ
ಕಣ
ಕಣ-ಗಳ
ಕಣ-ಗಳನ್ನು
ಕಣ-ಗಳಿಂದ
ಕಣ-ಗಳು
ಕಣ-ಜ-ದಲ್ಲಿ
ಕಣದ
ಕಣ-ದಂತೆ
ಕಣ-ದ-ಲ್ಲಿ-ರು-ವುದನ್ನು
ಕಣ-ದಿಂದ
ಕಣ-ವನ್ನು
ಕಣಿವೆ
ಕಣಿ-ವೆಯ
ಕಣಿ-ವೆ-ಯಲ್ಲಿ
ಕಣ್
ಕಣ್ಣ
ಕಣ್ಣನ್ನು
ಕಣ್ಣನ್ನೂ
ಕಣ್ಣ-ಮುಂದೆ
ಕಣ್ಣಾ-ಮು-ಚ್ಚಾಲೆ
ಕಣ್ಣಾರ
ಕಣ್ಣಾರೆ
ಕಣ್ಣಿಂದ
ಕಣ್ಣಿಗೂ
ಕಣ್ಣಿಗೆ
ಕಣ್ಣಿ-ಟ್ಟು-ಕೊಂಡೇ
ಕಣ್ಣಿದೆ
ಕಣ್ಣಿನ
ಕಣ್ಣಿ-ನಲ್ಲಿ
ಕಣ್ಣಿ-ನಿಂದ
ಕಣ್ಣಿ-ನಿಂ-ದಲೇ
ಕಣ್ಣಿ-ರನ್ನು
ಕಣ್ಣಿಲ್ಲ
ಕಣ್ಣೀ-ರನ್ನು
ಕಣ್ಣೀ-ರಿಗೂ
ಕಣ್ಣೀ-ರಿಗೆ
ಕಣ್ಣೀ-ರಿನ
ಕಣ್ಣೀರು
ಕಣ್ಣು
ಕಣ್ಣು-ಗಳನ್ನು
ಕಣ್ಣು-ಗ-ಳಲ್ಲ
ಕಣ್ಣು-ಗಳಿಂದ
ಕಣ್ಣು-ಗ-ಳಿವೆ
ಕಣ್ಣು-ಗಳು
ಕಣ್ಣು-ಗ-ಳುಳ್ಳ
ಕಣ್ಣು-ಮಿ-ಟು-ಕಿ-ಸು-ವು-ದ-ರಲ್ಲಿ
ಕಣ್ಣು-ಮುಂ-ದೆಯೇ
ಕಣ್ಣು-ಮುಚ್ಚಿ
ಕಣ್ಣು-ಮು-ಚ್ಚಿ-ಕೊಂಡು
ಕಣ್ಣು-ಮು-ಚ್ಚಿ-ಕೊ-ಳ್ಳ-ಬ-ಹುದು
ಕಣ್ಣು-ಮು-ಚ್ಚಿ-ಕೊ-ಳ್ಳಲು
ಕಣ್ಣು-ಮು-ಚ್ಚಿ-ಕೊ-ಳ್ಳು-ವ-ವರೇ
ಕಣ್ಣು-ಮು-ಚ್ಚಿ-ಕೊ-ಳ್ಳು-ವಷ್ಟು
ಕಣ್ಣೆ-ದು-ರಿ-ಗಿ-ರುವ
ಕಣ್ಣೆ-ದು-ರಿಗೆ
ಕಣ್ಣೆ-ದು-ರಿಗೇ
ಕಣ್ಣೇ
ಕಣ್ತೆ-ರೆದ
ಕಣ್ತೆ-ರೆ-ದು-ಕೊಂಡು
ಕಣ್ದೆ-ರೆದ
ಕಣ್ದೆ-ರೆ-ದೊ-ಡನೆ
ಕಣ್ಮ-ನ-ಗಳನ್ನು
ಕಣ್ಮ-ರೆ-ಯಾಗು
ಕಣ್ಮಿ-ಟು-ಕಿ-ನಂತೆ
ಕಣ್ಮಿ-ಟು-ಕಿ-ನಲ್ಲಿ
ಕಣ್ಮು-ಚ್ಚಿ-ಕೊ-ಳ್ಳು-ತ್ತೇನೆ
ಕತ-ರನ್ನೋ
ಕತೆ
ಕತ್ತನ್ನು
ಕತ್ತ-ರಿ-ಸ-ಬ-ಹುದು
ಕತ್ತ-ರಿ-ಸ-ಬೇಕು
ಕತ್ತ-ರಿ-ಸ-ಲಾ-ರವು
ಕತ್ತ-ರಿ-ಸಲು
ಕತ್ತ-ರಿಸಿ
ಕತ್ತ-ರಿ-ಸಿ-ದನು
ಕತ್ತ-ರಿಸು
ಕತ್ತ-ರಿ-ಸು-ತ್ತಾನೆ
ಕತ್ತ-ರಿ-ಸು-ತ್ತಿವೆ
ಕತ್ತ-ರಿ-ಸು-ವು-ದ-ಕ್ಕಾ-ಗು-ವು-ದಿಲ್ಲ
ಕತ್ತ-ರಿ-ಸು-ವು-ದಕ್ಕೆ
ಕತ್ತ-ರಿ-ಸು-ವುದು
ಕತ್ತಲ
ಕತ್ತ-ಲಲ್ಲಿ
ಕತ್ತ-ಲಿಗೆ
ಕತ್ತಲು
ಕತ್ತಲೆ
ಕತ್ತ-ಲೆ-ಬೆ-ಳಕು
ಕತ್ತ-ಲೆ-ಯ-ನ್ನೆಲ್ಲ
ಕತ್ತ-ಲೆ-ಯ-ನ್ನೆಲ್ಲಾ
ಕತ್ತ-ಲೆ-ಯಲ್ಲಿ
ಕತ್ತ-ಲೆ-ಯಿಂದ
ಕತ್ತಿ
ಕತ್ತಿಗೆ
ಕತ್ತಿ-ನ-ಮೇಲೆ
ಕತ್ತಿ-ನ-ವ-ರೆಗೆ
ಕತ್ತಿಯ
ಕತ್ತಿ-ಯಂತೆ
ಕತ್ತಿ-ಯಿಂದ
ಕತ್ತು
ಕತ್ತು-ಗಳು
ಕತ್ತೆ
ಕಥಂ
ಕಥ-ಮೇ-ತ-ದ್ವಿ-ಜಾ-ನೀ-ಯಾಂ
ಕಥಯ
ಕಥ-ಯಂ-ತಶ್ಚ
ಕಥ-ಯತಃ
ಕಥ-ಯಿ-ಷ್ಯಂತಿ
ಕಥ-ಯಿ-ಷ್ಯಾಮಿ
ಕಥಾ-ವ-ಸ್ತು-ವಾ-ದರೋ
ಕಥೆ
ಕಥೆ-ಗಳ
ಕಥೆ-ಗಳನ್ನು
ಕಥೆ-ಗಳನ್ನೆಲ್ಲ
ಕಥೆ-ಗಳಲ್ಲಿ
ಕಥೆ-ಗಳು
ಕಥೆ-ಗ-ಳೆಂಬ
ಕಥೆ-ಗ-ಳೆಲ್ಲ
ಕಥೆ-ಗಾಗಿ
ಕಥೆಯ
ಕಥೆ-ಯಂತೆ
ಕಥೆ-ಯನ್ನು
ಕಥೆ-ಯ-ನ್ನೆಲ್ಲ
ಕಥೆ-ಯಲ್ಲ
ಕಥೆ-ಯಲ್ಲಿ
ಕಥೆ-ಯಾ-ಗು-ವುದು
ಕಥೆ-ಯಿಂ-ದಲೆ
ಕಥೆ-ಯಿಂ-ದಲೇ
ಕಥೆಯೂ
ಕಥೆಯೇ
ಕಥೆಯೊ
ಕಥೋ-ಪ-ಕ-ಥೆ-ಗಳು
ಕದಡಿ
ಕದ-ಡಿ-ದಾಗ
ಕದ-ಡಿ-ಹೋ-ಗಿ-ರು-ತ್ತದೆ
ಕದ-ಡಿ-ಹೋ-ಗು-ವುದು
ಕದ-ಡಿ-ಹೋದ
ಕದ-ಲಿ-ಸ-ಲಾ-ರದು
ಕದ-ಲಿ-ಸು-ವಂತೆ
ಕದಾ-ಚನ
ಕದಾಚಿ
ಕದಿ-ಯು-ತ್ತಾನೆ
ಕದಿ-ಯು-ವನು
ಕದಿ-ಯು-ವುದು
ಕನ-ಸನ್ನು
ಕನ-ಸಾದ
ಕನ-ಸಿ-ಗಿಂತ
ಕನ-ಸಿಗೂ
ಕನ-ಸಿಗೆ
ಕನ-ಸಿಗೊ
ಕನ-ಸಿನ
ಕನ-ಸಿ-ನಂತೆ
ಕನ-ಸಿ-ನಲ್ಲಿ
ಕನ-ಸಿ-ನ-ಲ್ಲಿ-ರು-ವ-ವ-ರೆಗೆ
ಕನ-ಸಿ-ನ-ಲ್ಲಿ-ರು-ವಾಗ
ಕನ-ಸಿ-ನ-ಲ್ಲೆ-ಲ್ಲೆಲ್ಲೂ
ಕನ-ಸಿ-ನಿಂದ
ಕನಸು
ಕನ-ಸು-ಕಾ-ಣು-ವಂತೆ
ಕನ-ಸೆಲ್ಲ
ಕನ-ಸೆಲ್ಲಾ
ಕನಸೇ
ಕನಿ-ಕ-ರ-ದಿಂದ
ಕನಿ-ಕ-ರ-ವೆಂಬ
ಕನಿ-ಷ್ಟ-ಪ-ಕ್ಷದ
ಕನಿ-ಷ್ಠ-ವಾದ
ಕನಿ-ಷ್ಠಾಂ-ಶ-ವ-ನ್ನಾ-ದರೂ
ಕನ್ನ-ಡಕ
ಕನ್ನ-ಡ-ಕದ
ಕನ್ನ-ಡ-ಕ-ವನ್ನು
ಕನ್ನಡಿ
ಕನ್ನ-ಡಿಯ
ಕನ್ನ-ಡಿ-ಯಂತೆ
ಕನ್ನ-ಡಿ-ಯನ್ನು
ಕನ್ನ-ಡಿ-ಯಲ್ಲಿ
ಕನ್ನ-ಡಿ-ಯಾಗಿ
ಕನ್ನ-ಡಿ-ಯೊಂದು
ಕಪ-ಟವೂ
ಕಪಿ
ಕಪಿ-ಗಳು
ಕಪಿಗೆ
ಕಪಿ-ಧ್ವಜಃ
ಕಪಿ-ಧ್ವ-ಜ-ನಾದ
ಕಪಿ-ಮು-ಷ್ಟಿ-ಯಿಂದ
ಕಪಿ-ಮು-ಷ್ಠಿ-ಯಿಂದ
ಕಪಿ-ಯಂತೆ
ಕಪಿಲ
ಕಪಿ-ಲ-ಮುನಿ
ಕಪಿ-ಲರು
ಕಪಿಲೋ
ಕಪ್ಪು
ಕಪ್ಪು-ಇ-ದ್ದಿ-ಲನ್ನು
ಕಪ್ಪು-ಚು-ಕ್ಕಿ-ಗ-ಳಿವೆ
ಕಪ್ಪು-ಚು-ಕ್ಕಿ-ಯಲ್ಲಿ
ಕಪ್ಪು-ಚುಕ್ಕೆ
ಕಪ್ಪೆ
ಕಪ್ಪೆ-ಚಿ-ಪ್ಪು-ಗಳನ್ನು
ಕಪ್ಪೆ-ಚಿ-ಪ್ಪು-ಗಳಿ
ಕಪ್ಪೆ-ಚಿ-ಪ್ಪು-ಗ-ಳಿಗೆ
ಕಪ್ಪೆ-ಚಿ-ಪ್ಪು-ಗಳು
ಕಪ್ಪೆಯ
ಕಪ್ಪೆ-ಯನ್ನು
ಕಪ್ಪೆ-ಯೊಂದು
ಕಬ-ಳಿಸಿ
ಕಬ-ಳಿ-ಸಿ-ಬಿ-ಡು-ವುದು
ಕಬ-ಳಿ-ಸಿ-ಬಿ-ಡು-ವುದೊ
ಕಬ-ಳಿ-ಸು-ತ್ತಾ-ನೆಯೋ
ಕಬ-ಳಿ-ಸು-ತ್ತಿ-ದ್ದರೆ
ಕಬ-ಳಿ-ಸು-ತ್ತಿ-ರು-ವನು
ಕಬ-ಳಿ-ಸು-ವಳು
ಕಬ-ಳಿ-ಸು-ವುದು
ಕಬೀರ
ಕಬೀರ್
ಕಬ್ಬನ್ನು
ಕಬ್ಬಿಣ
ಕಬ್ಬಿ-ಣದ
ಕಬ್ಬಿ-ಣ-ದಂತೆ
ಕಬ್ಬಿ-ಣ-ವನ್ನು
ಕಬ್ಬಿ-ಣ-ವಾಗಿ
ಕಬ್ಬಿನ
ಕಬ್ಬು
ಕಮಂ-ಡಲು
ಕಮಲ
ಕಮ-ಲಕ್ಕೆ
ಕಮ-ಲದ
ಕಮ-ಲ-ದಂತೆ
ಕಮ-ಲ-ಪ-ತ್ರಾಕ್ಷ
ಕಮ-ಲವೇ
ಕಮ-ಲಾ-ಸ-ನದ
ಕಮ-ಲಾ-ಸ-ನಸ್ಥಂ
ಕಮಿ-ಟಿಯೇ
ಕಮ್
ಕಮ್ಮಾರ
ಕಮ್ಮಾ-ರ-ನಂತೆ
ಕಮ್ಯಾಂ-ಡರ್
ಕರ-ಗ-ತ-ವಾ-ಗು-ವುದು
ಕರ-ಗ-ತೊ-ಡ-ಗಿತು
ಕರ-ಗ-ಬೇಕು
ಕರ-ಗಳು
ಕರಗಿ
ಕರ-ಗಿ-ದರೆ
ಕರ-ಗಿ-ಸದೆ
ಕರ-ಗಿಸಿ
ಕರ-ಗಿ-ಹೋಗ
ಕರ-ಗಿ-ಹೋಗಿ
ಕರ-ಗಿ-ಹೋ-ಗು-ವನು
ಕರ-ಗಿ-ಹೋ-ಗು-ವುದು
ಕರ-ಗಿ-ಹೋ-ದಂತೆ
ಕರ-ಗು-ವಂತೆ
ಕರ-ಗು-ವುದನ್ನು
ಕರ-ಗು-ವು-ದಿಲ್ಲ
ಕರ-ಗು-ವುದು
ಕರಟ
ಕರ-ಟಕ್ಕೆ
ಕರ-ಟದ
ಕರ-ಟ-ದ-ಲ್ಲಿ-ರು-ವಂತೆ
ಕರ-ಟ-ದಿಂದ
ಕರ-ಟ-ವನ್ನು
ಕರಣ
ಕರಣಂ
ಕರ-ಣ-ಗಳ
ಕರ-ಣ-ಗಳನ್ನೆಲ್ಲಾ
ಕರ-ಣ-ಗಳು
ಕರ-ಣ-ಗ-ಳೆಲ್ಲ
ಕರ-ತಲಾ
ಕರ-ತಾ-ಡ-ನ-ವನ್ನು
ಕರಳು
ಕರಾಳ
ಕರಾ-ಳ-ವಾಗಿ
ಕರಾ-ಳ-ವಾ-ಗಿ-ರುವ
ಕರಾ-ಳವೂ
ಕರಿದು
ಕರಿಯ
ಕರಿ-ಯಲು
ಕರಿ-ಷ್ಯತಿ
ಕರಿ-ಷ್ಯಸಿ
ಕರಿ-ಷ್ಯ-ಸ್ಯ-ವ-ಶೋಽಪಿ
ಕರಿಷ್ಯೇ
ಕರಿ-ಸಿ-ರು-ವನೊ
ಕರು
ಕರುಣ
ಕರು-ಣಿ-ಸ-ಬೇ-ಕೆಂದು
ಕರು-ಣಿ-ಸಿ-ರು-ವನು
ಕರು-ಣಿ-ಸು-ವನು
ಕರು-ಣಿ-ಸು-ವರು
ಕರು-ಣಿ-ಸು-ವುದು
ಕರುಣೆ
ಕರು-ಣೆಯ
ಕರು-ಣೆ-ಯನ್ನು
ಕರು-ಣೆ-ಯಿಂದ
ಕರು-ಣೆ-ಯುಳ್ಳ
ಕರು-ಣೆ-ಯು-ಳ್ಳ-ವನು
ಕರು-ಬಿ-ದರೆ
ಕರುಬು
ಕರು-ಬು-ತ್ತಿ-ದ್ದರೆ
ಕರು-ಬು-ವು-ದಿಲ್ಲ
ಕರುವೇ
ಕರೆ
ಕರೆಗೆ
ಕರೆ-ತಂದು
ಕರೆ-ತ-ರುವ
ಕರೆ-ದತ್ತ
ಕರೆ-ದದ್ದು
ಕರೆ-ದನೊ
ಕರೆ-ದರು
ಕರೆ-ದರೂ
ಕರೆ-ದರೆ
ಕರೆ-ದ-ವನು
ಕರೆ-ದಾಗ
ಕರೆ-ದಿ-ರು-ವರು
ಕರೆದು
ಕರೆ-ದು-ಕೊಂಡು
ಕರೆ-ದು-ಕೊಂ-ಡು-ಹೋಗಿ
ಕರೆ-ದು-ಕೊ-ಳ್ಳು-ವು-ದಕ್ಕೆ
ಕರೆ-ದೊ-ಯ್ಯ-ಬ-ಹುದು
ಕರೆ-ದೊ-ಯ್ಯುವ
ಕರೆ-ದೊ-ಯ್ಯು-ವುದು
ಕರೆಯ
ಕರೆ-ಯದೆ
ಕರೆ-ಯನ್ನು
ಕರೆ-ಯ-ಬ-ಹುದು
ಕರೆ-ಯ-ಬೇ-ಕಾ-ಗಿಲ್ಲ
ಕರೆ-ಯಲಿ
ಕರೆ-ಯ-ಲ್ಪ-ಡು-ತ್ತಾನೆ
ಕರೆ-ಯ-ಲ್ಪ-ಪ-ಡು-ತ್ತಿದೆ
ಕರೆ-ಯ-ವುದು
ಕರೆ-ಯು-ತ್ತಾನೆ
ಕರೆ-ಯು-ತ್ತಾರೆ
ಕರೆ-ಯು-ತ್ತಿದೆ
ಕರೆ-ಯು-ತ್ತಿದ್ದ
ಕರೆ-ಯು-ತ್ತಿ-ದ್ದರು
ಕರೆ-ಯು-ತ್ತಿ-ರು-ವನು
ಕರೆ-ಯು-ತ್ತಿ-ರು-ವರು
ಕರೆ-ಯು-ತ್ತಿ-ರು-ವುದು
ಕರೆ-ಯು-ತ್ತಿ-ರು-ವೆನೊ
ಕರೆ-ಯು-ತ್ತೇವೆ
ಕರೆ-ಯುವ
ಕರೆ-ಯು-ವಂತೆ
ಕರೆ-ಯು-ವನು
ಕರೆ-ಯು-ವರು
ಕರೆ-ಯು-ವಳು
ಕರೆ-ಯು-ವ-ವನು
ಕರೆ-ಯು-ವ-ವನೇ
ಕರೆ-ಯು-ವು-ದಕ್ಕೆ
ಕರೆ-ಯು-ವು-ದಿಲ್ಲ
ಕರೆ-ಯು-ವುದು
ಕರೆ-ಯು-ವೆವು
ಕರೆ-ಯು-ವೆವೋ
ಕರೆಸಿ
ಕರೆ-ಸಿ-ಕೊಳ್ಳ
ಕರೆ-ಸಿ-ಕೊ-ಳ್ಳದೇ
ಕರೋತಿ
ಕರೋ-ಮೀತಿ
ಕರ್ಣ
ಕರ್ಣಂ
ಕರ್ಣ-ನಿಗೂ
ಕರ್ಣನೇ
ಕರ್ಣ-ರಂ-ತಹ
ಕರ್ಣಶ್ಚ
ಕರ್ಣೇನ
ಕರ್ತವ್ಯ
ಕರ್ತವ್ಯಂ
ಕರ್ತ-ವ್ಯ-ಕ್ಕಾಗಿ
ಕರ್ತ-ವ್ಯಕ್ಕೆ
ಕರ್ತ-ವ್ಯ-ಗಳನ್ನು
ಕರ್ತ-ವ್ಯ-ಗ-ಳಿಗೆ
ಕರ್ತ-ವ್ಯ-ಗಳು
ಕರ್ತ-ವ್ಯ-ಗ-ಳೆಲ್ಲ
ಕರ್ತ-ವ್ಯ-ತ-ತ್ಪ-ರ-ನಾ-ಗು-ವು-ದಿಲ್ಲ
ಕರ್ತ-ವ್ಯದ
ಕರ್ತ-ವ್ಯ-ದಲ್ಲಿ
ಕರ್ತ-ವ್ಯ-ನಿ-ಷ್ಠೆಯೇ
ಕರ್ತ-ವ್ಯ-ಪ-ರಾ-ಯ-ಣ-ತೆಯ
ಕರ್ತ-ವ್ಯ-ವನ್ನು
ಕರ್ತ-ವ್ಯ-ವ-ನ್ನೇನೋ
ಕರ್ತ-ವ್ಯ-ವಲ್ಲ
ಕರ್ತ-ವ್ಯ-ವಿಲ್ಲ
ಕರ್ತ-ವ್ಯವೂ
ಕರ್ತ-ವ್ಯವೆ
ಕರ್ತ-ವ್ಯ-ವೆಂದರೆ
ಕರ್ತ-ವ್ಯ-ವೆಂದು
ಕರ್ತ-ವ್ಯ-ವೇನು
ಕರ್ತ-ವ್ಯಾ-ನೀತಿ
ಕರ್ತಾ
ಕರ್ತಾರಂ
ಕರ್ತಾ-ರ-ಮಪಿ
ಕರ್ತಾ-ರ-ಮಾ-ತ್ಮಾನಂ
ಕರ್ತಾ-ಹ-ಮಿತಿ
ಕರ್ತುಂ
ಕರ್ತು-ಮ-ರ್ಹಸಿ
ಕರ್ತು-ಮ-ವ್ಯ-ಯಮ್
ಕರ್ತು-ಮಿ-ಹಾ-ರ್ಹಸಿ
ಕರ್ತೃ
ಕರ್ತೃ-ಗಳಲ್ಲಿ
ಕರ್ತೃ-ಗಳು
ಕರ್ತೃತ್ವ
ಕರ್ತೃತ್ವಂ
ಕರ್ತೃ-ತ್ವದ
ಕರ್ತೃ-ತ್ವ-ವ-ನ್ನಾ-ಗಲಿ
ಕರ್ತೃ-ಭಾವ
ಕರ್ತೃ-ವನ್ನು
ಕರ್ತೃ-ವಾಗಿ
ಕರ್ತೃ-ವಿನ
ಕರ್ತೃ-ವಿ-ನಲ್ಲಿ
ಕರ್ತೃವು
ಕರ್ತೃ-ವೆಂದು
ಕರ್ತೇತಿ
ಕರ್ಮ
ಕರ್ಮ-ಎಂ-ದರೆ
ಕರ್ಮ-ಒಂದೇ
ಕರ್ಮ-ಕಾಂಡ
ಕರ್ಮ-ಕಾಂ-ಡಕ್ಕೂ
ಕರ್ಮ-ಕಾಂ-ಡ-ವನ್ನು
ಕರ್ಮ-ಕಾಂ-ಡಿ-ಗಳು
ಕರ್ಮ-ಕು-ಶಲಿ
ಕರ್ಮ-ಕು-ಶ-ಲಿ-ಯಲ್ಲಿ
ಕರ್ಮ-ಕೃತ್
ಕರ್ಮ-ಕೌ-ಶಲ
ಕರ್ಮ-ಕ್ಕಿಂತ
ಕರ್ಮಕ್ಕೂ
ಕರ್ಮಕ್ಕೆ
ಕರ್ಮ-ಕ್ಷೇ-ತ್ರ-ದಲ್ಲಿ
ಕರ್ಮ-ಗಳ
ಕರ್ಮ-ಗಳನ್ನು
ಕರ್ಮ-ಗಳನ್ನೂ
ಕರ್ಮ-ಗಳನ್ನೆಲ್ಲ
ಕರ್ಮ-ಗಳನ್ನೆಲ್ಲಾ
ಕರ್ಮ-ಗಳಲ್ಲಿ
ಕರ್ಮ-ಗ-ಳ-ಲ್ಲಿಯೂ
ಕರ್ಮ-ಗ-ಳಾ-ಗಿ-ರ-ಬ-ಹುದು
ಕರ್ಮ-ಗಳಿಂದ
ಕರ್ಮ-ಗ-ಳಿಂ-ದಲೂ
ಕರ್ಮ-ಗ-ಳಿಗೆ
ಕರ್ಮ-ಗ-ಳಿವೆ
ಕರ್ಮ-ಗಳು
ಕರ್ಮ-ಗಳೂ
ಕರ್ಮ-ಗ-ಳೆಲ್ಲ
ಕರ್ಮ-ಗ-ಳೆಲ್ಲಾ
ಕರ್ಮ-ಗಳೇ
ಕರ್ಮ-ಗ-ಳೊಂ-ದಿಗೆ
ಕರ್ಮ-ಗಿಡ
ಕರ್ಮ-ಚೋ-ದನಾ
ಕರ್ಮಜಂ
ಕರ್ಮಜಾ
ಕರ್ಮ-ಜಾನ್
ಕರ್ಮ-ಜಾ-ಲ-ದಲ್ಲಿ
ಕರ್ಮಣಃ
ಕರ್ಮ-ಣಸ್ತೇ
ಕರ್ಮಣಾ
ಕರ್ಮ-ಣಾಂ
ಕರ್ಮ-ಣಾ-ಮ-ನಾ-ರಂ-ಭಾ-ನ್ನೈ-ಷ್ಕರ್ಮ್ಯಂ
ಕರ್ಮ-ಣಾ-ಮ-ಶಮಃ
ಕರ್ಮಣಿ
ಕರ್ಮ-ಣೈವ
ಕರ್ಮಣೋ
ಕರ್ಮ-ಣೋ-ಽನ್ಯತ್ರ
ಕರ್ಮ-ಣ್ಯ-ಕರ್ಮ
ಕರ್ಮ-ಣ್ಯ-ತಂ-ದ್ರಿತಃ
ಕರ್ಮ-ಣ್ಯ-ಭಿ-ಪ್ರ-ವೃ-ತ್ತೋಽಪಿ
ಕರ್ಮ-ಣ್ಯ-ಭಿ-ರತಃ
ಕರ್ಮ-ಣ್ಯ-ವಿ-ದ್ವಾಂಸೋ
ಕರ್ಮ-ಣ್ಯೇ-ವಾ-ಧಿ-ಕಾ-ರಸ್ತೇ
ಕರ್ಮ-ತ್ಯಾಗ
ಕರ್ಮ-ತ್ಯಾ-ಗ-ಕ್ಕಿಂತ
ಕರ್ಮ-ತ್ಯಾಗಿ
ಕರ್ಮದ
ಕರ್ಮ-ದಲ್ಲಿ
ಕರ್ಮ-ದ-ಲ್ಲಿಯೂ
ಕರ್ಮ-ದ-ಲ್ಲಿ-ರುವ
ಕರ್ಮ-ದಷ್ಟು
ಕರ್ಮ-ದಿಂದ
ಕರ್ಮ-ದಿಂ-ದಲೇ
ಕರ್ಮ-ದಿಂ-ದಲ್ಲ
ಕರ್ಮ-ದೂ-ತರು
ಕರ್ಮ-ದೇ-ಹ-ವನ್ನು
ಕರ್ಮ-ನ-ದಿಯ
ಕರ್ಮ-ನಿ-ಯಮ
ಕರ್ಮ-ನಿಷ್ಠೆ
ಕರ್ಮ-ನಿ-ಷ್ಠೆ-ಯಲ್ಲಿ
ಕರ್ಮ-ಪ-ಟು-ವಾದ
ಕರ್ಮ-ಪಾ-ಶ-ದಿಂದ
ಕರ್ಮ-ಪಾ-ಶ-ವನ್ನು
ಕರ್ಮ-ಪ್ರ-ಪಂ-ಚ-ದಲ್ಲಿ
ಕರ್ಮ-ಫಲ
ಕರ್ಮ-ಫಲಂ
ಕರ್ಮ-ಫ-ಲಕ್ಕೆ
ಕರ್ಮ-ಫ-ಲ-ಗಳ
ಕರ್ಮ-ಫ-ಲ-ತ್ಯಾಗ
ಕರ್ಮ-ಫ-ಲ-ತ್ಯಾ-ಗ-ಸ್ತ್ಯಾ-ಗಾ-ಚ್ಛಾಂ-ತಿ-ರ-ನಂ-ತ-ರಮ್
ಕರ್ಮ-ಫ-ಲ-ತ್ಯಾಗೀ
ಕರ್ಮ-ಫ-ಲ-ದಲ್ಲಿ
ಕರ್ಮ-ಫ-ಲ-ದಾತ
ಕರ್ಮ-ಫ-ಲ-ದಿಂದ
ಕರ್ಮ-ಫ-ಲ-ಪ್ರೇ-ಪ್ಸು-ರ್ಲುಬ್ಧೋ
ಕರ್ಮ-ಫ-ಲ-ವನ್ನು
ಕರ್ಮ-ಫ-ಲ-ಸಂ-ಗ-ವನ್ನು
ಕರ್ಮ-ಫ-ಲ-ಸಂ-ಯೋಗಂ
ಕರ್ಮ-ಫ-ಲ-ಸಂ-ಯೋ-ಗ-ವನ್ನೂ
ಕರ್ಮ-ಫ-ಲ-ಹೇ-ತು-ರ್ಭೂರ್ಮಾ
ಕರ್ಮ-ಫ-ಲಾ-ಸಂಗಂ
ಕರ್ಮ-ಫ-ಲಾ-ಸಕ್ತಿ
ಕರ್ಮ-ಫಲೇ
ಕರ್ಮ-ಬಂಧಂ
ಕರ್ಮ-ಬಂ-ಧನಃ
ಕರ್ಮ-ಬಂ-ಧ-ನ-ದಿಂದ
ಕರ್ಮ-ಬಂ-ಧ-ನೈಃ
ಕರ್ಮ-ಬ್ರ-ಹ್ಮೋ-ದ್ಭವಂ
ಕರ್ಮ-ಭಿಃ
ಕರ್ಮ-ಭಿರ್ನ
ಕರ್ಮ-ಮಾ-ಡದೆ
ಕರ್ಮ-ಮಾಡಿ
ಕರ್ಮ-ಮಾ-ಡಿ-ದರೆ
ಕರ್ಮ-ಮಾ-ಡಿಸು
ಕರ್ಮ-ಮಾಡು
ಕರ್ಮ-ಮಾ-ಡು-ತ್ತೇವೆ
ಕರ್ಮ-ಮಾ-ಡು-ತ್ತೇ-ವೆಯೋ
ಕರ್ಮ-ಮಾ-ಡುವ
ಕರ್ಮ-ಮಾ-ಡು-ವನು
ಕರ್ಮ-ಮಾ-ಡು-ವನೊ
ಕರ್ಮ-ಮಾ-ಡು-ವು-ದಕ್ಕೆ
ಕರ್ಮ-ಮಾ-ಡು-ವು-ದ-ರಿಂದ
ಕರ್ಮ-ಮಾ-ಡು-ವು-ದಿಲ್ಲ
ಕರ್ಮ-ಮಾಯೆ
ಕರ್ಮ-ಯೋಗ
ಕರ್ಮ-ಯೋ-ಗಕ್ಕೂ
ಕರ್ಮ-ಯೋ-ಗದ
ಕರ್ಮ-ಯೋ-ಗ-ದಲ್ಲಿ
ಕರ್ಮ-ಯೋ-ಗ-ದಿಂದ
ಕರ್ಮ-ಯೋ-ಗ-ದಿಂ-ದಲೂ
ಕರ್ಮ-ಯೋ-ಗ-ಮ-ಸಕ್ತಃ
ಕರ್ಮ-ಯೋ-ಗ-ವ-ನ್ನಾಗಿ
ಕರ್ಮ-ಯೋ-ಗ-ವನ್ನು
ಕರ್ಮ-ಯೋ-ಗ-ವಾ-ಗ-ಬೇಕು
ಕರ್ಮ-ಯೋ-ಗ-ವಾ-ಗಿ-ರು-ವುದು
ಕರ್ಮ-ಯೋ-ಗ-ವಿ-ಲ್ಲದೆ
ಕರ್ಮ-ಯೋ-ಗವೇ
ಕರ್ಮ-ಯೋ-ಗಶ್ಚ
ಕರ್ಮ-ಯೋಗಿ
ಕರ್ಮ-ಯೋ-ಗಿ-ಗಳಲ್ಲಿ
ಕರ್ಮ-ಯೋ-ಗಿ-ಗಳು
ಕರ್ಮ-ಯೋ-ಗಿಗೆ
ಕರ್ಮ-ಯೋ-ಗಿಯ
ಕರ್ಮ-ಯೋ-ಗಿ-ಯ-ಲ್ಲಿ-ರು-ವುದು
ಕರ್ಮ-ಯೋ-ಗಿ-ಯಾ-ಗಲಿ
ಕರ್ಮ-ಯೋ-ಗಿಯೇ
ಕರ್ಮ-ಯೋ-ಗೇನ
ಕರ್ಮ-ಯೋಗೋ
ಕರ್ಮ-ರಂ-ಗ-ದಲ್ಲಿ
ಕರ್ಮ-ಲೋ-ಕ-ದ-ಲ್ಲಿಯೇ
ಕರ್ಮ-ವ-ನ್ನಾ-ಗಲಿ
ಕರ್ಮ-ವನ್ನು
ಕರ್ಮ-ವನ್ನೂ
ಕರ್ಮ-ವನ್ನೆ
ಕರ್ಮ-ವ-ನ್ನೆಲ್ಲ
ಕರ್ಮ-ವ-ನ್ನೆಲ್ಲಾ
ಕರ್ಮ-ವನ್ನೇ
ಕರ್ಮ-ವಲ್ಲ
ಕರ್ಮ-ವಾ-ಗಲಿ
ಕರ್ಮ-ವಾ-ಗಿ-ರ-ಬ-ಹುದು
ಕರ್ಮ-ವಾ-ಗಿ-ರ-ಬೇಕು
ಕರ್ಮ-ವಾ-ಗು-ತ್ತವೆ
ಕರ್ಮ-ವಾ-ಗು-ವುದು
ಕರ್ಮ-ವಾ-ದರೆ
ಕರ್ಮ-ವಾದಿ
ಕರ್ಮ-ವಾ-ಯಿತು
ಕರ್ಮ-ವಿ-ಲ್ಲದೆ
ಕರ್ಮ-ವಿ-ಲ್ಲದೇ
ಕರ್ಮವು
ಕರ್ಮವೂ
ಕರ್ಮವೆ
ಕರ್ಮ-ವೆಲ್ಲ
ಕರ್ಮ-ವೆಲ್ಲಾ
ಕರ್ಮವೇ
ಕರ್ಮವೊ
ಕರ್ಮವೋ
ಕರ್ಮ-ಶೇಷ
ಕರ್ಮ-ಸಂ-ಗ-ದಿಂದ
ಕರ್ಮ-ಸಂ-ಗಿ-ಗಳ
ಕರ್ಮ-ಸಂ-ಗಿ-ಗಳಲ್ಲಿ
ಕರ್ಮ-ಸಂ-ಗಿ-ನಾಮ್
ಕರ್ಮ-ಸಂ-ಗಿಷು
ಕರ್ಮ-ಸಂ-ಗೇನ
ಕರ್ಮ-ಸಂ-ಗ್ರಹಃ
ಕರ್ಮ-ಸಂ-ಜ್ಞಿತಃ
ಕರ್ಮ-ಸಂ-ನ್ಯಾಸ
ಕರ್ಮ-ಸಂ-ನ್ಯಾ-ಸ-ವನ್ನು
ಕರ್ಮ-ಸಂ-ನ್ಯಾ-ಸಾತ್
ಕರ್ಮ-ಸ-ಮು-ದ್ಭವಃ
ಕರ್ಮ-ಸಿ-ದ್ಧಾಂತ
ಕರ್ಮ-ಸಿ-ದ್ಧಾಂ-ತದ
ಕರ್ಮಸು
ಕರ್ಮ-ಸ್ವ-ನು-ಷ-ಜ್ಜತೇ
ಕರ್ಮಾ-ಖಿಲಂ
ಕರ್ಮಾಣಿ
ಕರ್ಮಾ-ಣ್ಯ-ಶೇ-ಷತಃ
ಕರ್ಮಾ-ದಿ-ಗಳನ್ನು
ಕರ್ಮಾನು
ಕರ್ಮಾ-ನು-ಬಂ-ಧೀನಿ
ಕರ್ಮಾ-ನು-ಷ್ಠಾ-ನ-ದಿಂ-ದಲೇ
ಕರ್ಮಾ-ನು-ಸಾರ
ಕರ್ಮಿ
ಕರ್ಮಿ-ಗ-ಳಿ-ಗಿಂ-ತಲೂ
ಕರ್ಮಿ-ಗ-ಳಿಗೆ
ಕರ್ಮಿ-ಗಳು
ಕರ್ಮಿ-ಗಳೂ
ಕರ್ಮಿಗೆ
ಕರ್ಮಿ-ಭ್ಯ-ಶ್ಚಾ-ಧಿಕೋ
ಕರ್ಮಿಯ
ಕರ್ಮೇಂ
ಕರ್ಮೇಂ-ದಿ-ಯ-ಗಳೇ
ಕರ್ಮೇಂ-ದ್ರಿಯ
ಕರ್ಮೇಂ-ದ್ರಿ-ಯ-ಗಳ
ಕರ್ಮೇಂ-ದ್ರಿ-ಯ-ಗಳನ್ನು
ಕರ್ಮೇಂ-ದ್ರಿ-ಯ-ಗಳಿಂದ
ಕರ್ಮೇಂ-ದ್ರಿ-ಯ-ಗ-ಳಿವೆ
ಕರ್ಮೇಂ-ದ್ರಿ-ಯ-ಗಳು
ಕರ್ಮೇಂ-ದ್ರಿ-ಯ-ಗ-ಳೆಲ್ಲ
ಕರ್ಮೇಂ-ದ್ರಿ-ಯಾಣಿ
ಕರ್ಮೇಂ-ದ್ರಿ-ಯೈಃ
ಕರ್ಮೈವ
ಕರ್ಶ-ಯಂತಃ
ಕರ್ಷತಿ
ಕಲಕಿ
ಕಲ-ಕು-ತ್ತಿದೆ
ಕಲ-ಕು-ವುದ
ಕಲ-ಕು-ವುವು
ಕಲ-ಯ-ತಾ-ಮ-ಹಮ್
ಕಲಶ
ಕಲಸು
ಕಲ-ಸು-ಮೇ-ಲೋ-ಗ-ರ-ವಾಗಿ
ಕಲಾ-ಪ್ರ-ದ-ರ್ಶನ
ಕಲಾ-ಭಿ-ರುಚಿ
ಕಲಿ
ಕಲಿತ
ಕಲಿ-ತ-ದ್ದನ್ನು
ಕಲಿ-ತದ್ದು
ಕಲಿ-ತರು
ಕಲಿ-ತರೂ
ಕಲಿ-ತರೆ
ಕಲಿ-ತ-ವರು
ಕಲಿ-ತಿ-ದ್ದನೋ
ಕಲಿ-ತಿ-ದ್ದನ್ನು
ಕಲಿ-ತಿ-ದ್ದಾನೆ
ಕಲಿ-ತಿದ್ದು
ಕಲಿ-ತಿ-ರು-ವನು
ಕಲಿ-ತಿ-ರು-ವರು
ಕಲಿ-ತಿಲ್ಲ
ಕಲಿತು
ಕಲಿ-ತು-ಕೊಂಡು
ಕಲಿ-ತು-ಕೊ-ಳ್ಳ-ಬೇಕು
ಕಲಿ-ತು-ಕೊ-ಳ್ಳು-ವ-ವ-ನಿಗೆ
ಕಲಿ-ತು-ಕೊ-ಳ್ಳುವು
ಕಲಿ-ತು-ಕೊ-ಳ್ಳು-ವು-ದಕ್ಕೆ
ಕಲಿ-ತು-ಕೊ-ಳ್ಳು-ವುದೇ
ಕಲಿ-ತು-ದನ್ನು
ಕಲಿ-ತೆನೋ
ಕಲಿ-ಮ-ಲ-ಪ್ರ-ಧ್ವಂಸಿ
ಕಲಿ-ಯದೆ
ಕಲಿ-ಯ-ಬಲ್ಲ
ಕಲಿ-ಯ-ಬೇ-ಕಾ-ಗಿದೆ
ಕಲಿ-ಯ-ಬೇ-ಕಾ-ಗಿಲ್ಲ
ಕಲಿ-ಯ-ಬೇ-ಕಾ-ಗು-ವುದು
ಕಲಿ-ಯ-ಬೇ-ಕಾ-ದರೆ
ಕಲಿ-ಯ-ಬೇಕು
ಕಲಿ-ಯ-ಲಾರ
ಕಲಿ-ಯಲಿ
ಕಲಿ-ಯಲು
ಕಲಿ-ಯಿರಿ
ಕಲಿ-ಯು-ಗ-ದಲ್ಲಿ
ಕಲಿ-ಯುತ್ತ
ಕಲಿ-ಯು-ತ್ತಾನೆ
ಕಲಿ-ಯು-ತ್ತಿ-ರು-ತ್ತೇವೆ
ಕಲಿ-ಯು-ತ್ತೇವೆ
ಕಲಿ-ಯುವ
ಕಲಿ-ಯು-ವನು
ಕಲಿ-ಯು-ವರು
ಕಲಿ-ಯು-ವ-ವ-ರಲ್ಲ
ಕಲಿ-ಯು-ವ-ವ-ರೆಗೆ
ಕಲಿ-ಯು-ವಾಗ
ಕಲಿ-ಯು-ವು-ದಕ್ಕೆ
ಕಲಿ-ಯು-ವು-ದಿಲ್ಲ
ಕಲಿ-ಯು-ವುದು
ಕಲಿ-ಯು-ವುದೂ
ಕಲಿ-ಯು-ವು-ದೇ-ನಿಲ್ಲ
ಕಲಿ-ಯು-ವು-ದೇನೂ
ಕಲಿ-ಯೋಣ
ಕಲಿ-ಸ-ಕೂ-ಡದು
ಕಲಿ-ಸ-ಬ-ಹುದು
ಕಲಿ-ಸ-ಬ-ಹು-ದೇನೊ
ಕಲಿ-ಸ-ಬೇ-ಕಾ-ಗಿಲ್ಲ
ಕಲಿ-ಸ-ಬೇಕು
ಕಲಿ-ಸ-ಬೇ-ಕೆಂ-ದಿ-ರುವೆ
ಕಲಿ-ಸ-ಬೇ-ಕೆಂದು
ಕಲಿಸಿ
ಕಲಿ-ಸಿ-ಕೊಟ್ಟು
ಕಲಿ-ಸಿ-ದ್ದರು
ಕಲಿ-ಸು-ತ್ತಿ-ರು-ವನು
ಕಲಿ-ಸು-ವನು
ಕಲಿ-ಸು-ವು-ದ-ಕ್ಕಾಗಿ
ಕಲಿ-ಸು-ವು-ದಿಲ್ಲ
ಕಲಿ-ಸು-ವುದು
ಕಲೆ
ಕಲೆ-ಯ-ನ್ನಾ-ದರೂ
ಕಲೆ-ಯನ್ನು
ಕಲೆ-ಯಾ-ಗಿ-ರ-ಬ-ಹುದು
ಕಲೆ-ಯಾ-ಗು-ವುದು
ಕಲೇ-ವ-ರಮ್
ಕಲ್ಕ-ತ್ತೆಗೆ
ಕಲ್ಪಕ್ಕೆ
ಕಲ್ಪ-ಕ್ಷಯೇ
ಕಲ್ಪ-ತರು
ಕಲ್ಪ-ತ-ರು-ವಾ-ಗು-ವನು
ಕಲ್ಪತೇ
ಕಲ್ಪದ
ಕಲ್ಪನೆ
ಕಲ್ಪ-ನೆ-ಗಳನ್ನು
ಕಲ್ಪ-ನೆಗೆ
ಕಲ್ಪ-ನೆಯ
ಕಲ್ಪ-ನೆ-ಯನ್ನು
ಕಲ್ಪ-ನೆ-ಯಲ್ಲ
ಕಲ್ಪ-ನೆಯೇ
ಕಲ್ಪ-ನೆಯೋ
ಕಲ್ಪ-ವೃಕ್ಷ
ಕಲ್ಪ-ವೃ-ಕ್ಷ-ದಂತೆ
ಕಲ್ಪಾದೌ
ಕಲ್ಪಿಸಿ
ಕಲ್ಪಿ-ಸಿ-ಕೊಂಡ
ಕಲ್ಪಿ-ಸಿ-ಕೊಂ-ಡ-ವರು
ಕಲ್ಪಿ-ಸಿ-ಕೊಂ-ಡಿ-ದ್ದನು
ಕಲ್ಪಿ-ಸಿ-ಕೊಂ-ಡಿ-ರು-ವನು
ಕಲ್ಪಿ-ಸಿ-ಕೊಂ-ಡಿ-ರು-ವೆವು
ಕಲ್ಪಿ-ಸಿ-ಕೊಂ-ಡಿಲ್ಲ
ಕಲ್ಪಿ-ಸಿ-ಕೊಂಡು
ಕಲ್ಪಿ-ಸಿ-ಕೊ-ಳ್ಳದೇ
ಕಲ್ಪಿ-ಸಿ-ಕೊ-ಳ್ಳ-ಬಲ್ಲ
ಕಲ್ಪಿ-ಸಿ-ಕೊ-ಳ್ಳ-ಬ-ಹುದು
ಕಲ್ಪಿ-ಸಿ-ಕೊ-ಳ್ಳ-ಬೇಕು
ಕಲ್ಪಿ-ಸಿ-ಕೊ-ಳ್ಳ-ಲಾ-ರದು
ಕಲ್ಪಿ-ಸಿ-ಕೊಳ್ಳು
ಕಲ್ಪಿ-ಸಿ-ಕೊ-ಳ್ಳು-ತ್ತಾನೆ
ಕಲ್ಪಿ-ಸಿ-ಕೊ-ಳ್ಳು-ತ್ತಿ-ದ್ದೆವು
ಕಲ್ಪಿ-ಸಿ-ಕೊ-ಳ್ಳು-ತ್ತಿ-ರು-ವುದು
ಕಲ್ಪಿ-ಸಿ-ಕೊ-ಳ್ಳು-ತ್ತೇವೆ
ಕಲ್ಪಿ-ಸಿ-ಕೊ-ಳ್ಳುವ
ಕಲ್ಪಿ-ಸಿ-ಕೊ-ಳ್ಳು-ವು-ದಕ್ಕೆ
ಕಲ್ಪಿ-ಸಿ-ಕೊ-ಳ್ಳು-ವುದನ್ನು
ಕಲ್ಪಿ-ಸಿ-ಕೊ-ಳ್ಳು-ವು-ದಿಲ್ಲ
ಕಲ್ಪಿ-ಸಿ-ಕೊ-ಳ್ಳು-ವುದು
ಕಲ್ಪಿ-ಸಿ-ಕೊ-ಳ್ಳು-ವುದೇ
ಕಲ್ಪಿ-ಸಿ-ಕೊ-ಳ್ಳು-ವೆವು
ಕಲ್ಪಿ-ಸಿ-ದರು
ಕಲ್ಪಿ-ಸಿ-ದ್ದಾನೆ
ಕಲ್ಪಿಸು
ಕಲ್ಪಿ-ಸು-ವನು
ಕಲ್ಪಿ-ಸು-ವ-ವನು
ಕಲ್ಪಿ-ಸು-ವುದು
ಕಲ್ಮ-ಶ-ಗಳನ್ನು
ಕಲ್ಮ-ಶ-ಗ-ಳೆಲ್ಲ
ಕಲ್ಮ-ಶ-ಗ-ಳೆ-ಲ್ಲವೂ
ಕಲ್ಮ-ಶ-ರ-ಹಿ-ತ-ನಾದ
ಕಲ್ಮ-ಶವೂ
ಕಲ್ಮ-ಶ-ವೆಲ್ಲ
ಕಲ್ಮ-ಶಾಃ
ಕಲ್ಮ-ಷ-ಗಳನ್ನೆಲ್ಲ
ಕಲ್ಮ-ಷ-ದಿಂದ
ಕಲ್ಮ-ಷ-ರ-ಹಿ-ತನೂ
ಕಲ್ಮ-ಷ-ವನ್ನು
ಕಲ್ಮ-ಷ-ವಿ-ಲ್ಲ-ದ-ವರು
ಕಲ್ಮ-ಷವೂ
ಕಲ್ಯಾಣ
ಕಲ್ಯಾ-ಣ-ಕಾರಿ
ಕಲ್ಯಾ-ಣ-ಕಾ-ರಿ-ಯಾ-ಗು-ತ್ತದೆ
ಕಲ್ಯಾ-ಣ-ಕಾ-ರಿ-ಯಾದ
ಕಲ್ಯಾ-ಣ-ಕಾರ್ಯ
ಕಲ್ಯಾ-ಣ-ಕೃತ್
ಕಲ್ಯಾ-ಣಕ್ಕೆ
ಕಲ್ಯಾ-ಣ-ಗು-ಣ-ಗ-ಳೆಲ್ಲ
ಕಲ್ಯಾ-ಣದ
ಕಲ್ಯಾ-ಣ-ಪ್ರ-ದ-ವಾ-ಗಿ-ರು-ವುದು
ಕಲ್ಯಾ-ಣ-ವಾ-ಗಲಿ
ಕಲ್ಯಾ-ಣವೂ
ಕಲ್ಲನ್ನು
ಕಲ್ಲಾ-ಗು-ವು-ದಿಲ್ಲ
ಕಲ್ಲಿ-ದ್ದ-ಲಿ-ನಲ್ಲಿ
ಕಲ್ಲಿ-ದ್ದಲು
ಕಲ್ಲಿನ
ಕಲ್ಲಿ-ನಂ-ತಾ-ಗುವು
ಕಲ್ಲಿ-ನಂತೆ
ಕಲ್ಲಿ-ನಲ್ಲಿ
ಕಲ್ಲಿ-ನಿಂದ
ಕಲ್ಲಿ-ನೊ-ಳಗೆ
ಕಲ್ಲಿ-ರುವ
ಕಲ್ಲು
ಕಲ್ಲು-ಗಳನ್ನು
ಕಲ್ಲು-ಗಳು
ಕಲ್ಲು-ಚ-ಪ್ಪ-ಡಿ-ಯಂತೆ
ಕಲ್ಲು-ನೆ-ಲ-ದಲ್ಲಿ
ಕಲ್ಲು-ಬಂಡೆ
ಕಲ್ಲು-ಬಂ-ಡೆ-ಯಂತೆ
ಕಲ್ಲು-ಬಂ-ಡೆ-ಯನ್ನು
ಕಲ್ಲು-ಮಣ್ಣು
ಕಲ್ಲೂ
ಕಲ್ಲೋಲ
ಕಲ್ಲೋ-ಲ-ವಾ-ಗಿದೆ
ಕಲ್ಲೋ-ಲ-ವಾ-ಗಿ-ದ್ದರೆ
ಕಲ್ಲೋ-ಲ-ವಾ-ಗಿ-ರು-ವುದು
ಕಳಂಕ
ಕಳಂ-ಕ-ವನ್ನು
ಕಳಚಿ
ಕಳ-ಚಿ-ಕೊಂ-ಡಿರು
ಕಳ-ಚಿ-ಕೊಂಡು
ಕಳ-ಚಿ-ಟ್ಟಂ-ತೆಯೆ
ಕಳ-ಚಿ-ಟ್ಟರೆ
ಕಳ-ಚಿ-ಡು-ವುದು
ಕಳ-ಚಿ-ಬಿ-ಟ್ಟರೆ
ಕಳ-ಚಿ-ಬೀ-ಳು-ವುವು
ಕಳ-ಚಿಯೊ
ಕಳಪೆ
ಕಳ-ಪೆಯ
ಕಳ-ವಳ
ಕಳ-ವ-ಳ-ಗಳು
ಕಳ-ವ-ಳ-ಪ-ಡುತ್ತ
ಕಳ-ವ-ಳ-ವಿಲ್ಲ
ಕಳ-ವ-ಳ-ವೇ-ಳು-ವುದು
ಕಳಿಂಗ
ಕಳಿ-ಸ-ಬೇ-ಕಾ-ದರೆ
ಕಳಿ-ಸಿ-ದರು
ಕಳಿ-ಸಿ-ದರೆ
ಕಳಿ-ಸಿ-ದ್ದರೆ
ಕಳಿಸು
ಕಳು-ಹಿ-ಸ-ಬ-ಹು-ದಷ್ಟೆ
ಕಳು-ಹಿ-ಸ-ಬ-ಹುದು
ಕಳು-ಹಿ-ಸ-ಬೇ-ಕಾ-ಗಿದೆ
ಕಳು-ಹಿ-ಸ-ಬೇ-ಕಾ-ಗಿಲ್ಲ
ಕಳು-ಹಿ-ಸ-ಬೇ-ಕಾ-ದರೆ
ಕಳು-ಹಿ-ಸ-ಬೇಕು
ಕಳು-ಹಿ-ಸ-ಬೇಕೋ
ಕಳು-ಹಿ-ಸ-ಲಿಲ್ಲ
ಕಳು-ಹಿಸಿ
ಕಳು-ಹಿ-ಸಿದ
ಕಳು-ಹಿ-ಸಿ-ದರು
ಕಳು-ಹಿ-ಸಿರು
ಕಳು-ಹಿ-ಸಿ-ರು-ವನು
ಕಳು-ಹಿ-ಸು-ತ್ತದೆ
ಕಳು-ಹಿ-ಸು-ತ್ತಾನೆ
ಕಳು-ಹಿ-ಸು-ತ್ತಿ-ರ-ಬೇಕು
ಕಳು-ಹಿ-ಸು-ತ್ತಿ-ರು-ವುದು
ಕಳು-ಹಿ-ಸು-ತ್ತೇ-ವೆಯೊ
ಕಳು-ಹಿ-ಸು-ತ್ತೇ-ವೆಯೋ
ಕಳು-ಹಿ-ಸು-ವನು
ಕಳು-ಹಿ-ಸು-ವು-ದ-ಕ್ಕಾಗಿ
ಕಳು-ಹಿ-ಸು-ವು-ದಿಲ್ಲ
ಕಳು-ಹಿ-ಸು-ವುದು
ಕಳು-ಹಿ-ಸು-ವೆವೊ
ಕಳೆ
ಕಳೆ-ಗಳನ್ನು
ಕಳೆ-ಗಳು
ಕಳೆ-ದಂತೆ
ಕಳೆ-ದರು
ಕಳೆ-ದರೆ
ಕಳೆ-ದಿ-ದ್ದರೆ
ಕಳೆ-ದಿ-ರು-ವೆವು
ಕಳೆದು
ಕಳೆ-ದು-ಕೊಂ-ಡಂ-ತಾ-ಗಿ-ರು-ವನು
ಕಳೆ-ದು-ಕೊಂ-ಡಂತೆ
ಕಳೆ-ದು-ಕೊಂ-ಡರೂ
ಕಳೆ-ದು-ಕೊಂ-ಡರೆ
ಕಳೆ-ದು-ಕೊಂ-ಡಾಗ
ಕಳೆ-ದು-ಕೊಂ-ಡಿ-ದ್ದ-ಕ್ಕಾಗಿ
ಕಳೆ-ದು-ಕೊಂ-ಡಿ-ರ-ಬ-ಹುದು
ಕಳೆ-ದು-ಕೊಂಡು
ಕಳೆ-ದು-ಕೊಳ್ಳ
ಕಳೆ-ದು-ಕೊ-ಳ್ಳ-ಬೇ-ಕಾ-ದರೆ
ಕಳೆ-ದು-ಕೊ-ಳ್ಳ-ಲೆ-ತ್ನಿ-ಸು-ವನು
ಕಳೆ-ದು-ಕೊಳ್ಳು
ಕಳೆ-ದು-ಕೊ-ಳ್ಳು-ತ್ತಾನೆ
ಕಳೆ-ದು-ಕೊ-ಳ್ಳು-ತ್ತಾರೆ
ಕಳೆ-ದು-ಕೊ-ಳ್ಳು-ತ್ತೇ-ನೆಯೋ
ಕಳೆ-ದು-ಕೊ-ಳ್ಳು-ತ್ತೇ-ವಲ್ಲ
ಕಳೆ-ದು-ಕೊ-ಳ್ಳು-ತ್ತೇವೆ
ಕಳೆ-ದು-ಕೊ-ಳ್ಳುವ
ಕಳೆ-ದು-ಕೊ-ಳ್ಳು-ವನು
ಕಳೆ-ದು-ಕೊ-ಳ್ಳು-ವು-ದಕ್ಕೆ
ಕಳೆ-ದು-ಕೊ-ಳ್ಳು-ವು-ದಿಲ್ಲ
ಕಳೆ-ದು-ಕೊ-ಳ್ಳು-ವುದು
ಕಳೆ-ದು-ಕೊ-ಳ್ಳು-ವುವು
ಕಳೆ-ದು-ಕೊ-ಳ್ಳು-ವೆವು
ಕಳೆ-ಯಂತೆ
ಕಳೆ-ಯನ್ನು
ಕಳೆ-ಯು-ತ್ತದೆ
ಕಳೆ-ಯು-ತ್ತಾ-ನೆಯೆ
ಕಳೆ-ಯು-ತ್ತಾರೆ
ಕಳೆ-ಯು-ತ್ತಾ-ರೆ-ಮ-ನೆಯ
ಕಳೆ-ಯು-ತ್ತಿ-ರು-ವನು
ಕಳೆ-ಯು-ತ್ತಿಲ್ಲ
ಕಳೆ-ಯು-ವನು
ಕಳೆ-ಯು-ವು-ದಕ್ಕೆ
ಕಳೆ-ಯು-ವು-ದಿಲ್ಲ
ಕಳೆ-ಯು-ವುದು
ಕಳ್ಳ
ಕಳ್ಳ-ಕಾ-ಕರ
ಕಳ್ಳ-ಕಾ-ಕ-ರಂತೆ
ಕಳ್ಳ-ಕಾ-ಕರು
ಕಳ್ಳ-ತನ
ಕಳ್ಳ-ತ-ನವೊ
ಕಳ್ಳ-ತ-ನವೋ
ಕಳ್ಳನ
ಕಳ್ಳ-ನಲ್ಲಿ
ಕಳ್ಳ-ನಾ-ಗಿ-ರ-ಬ-ಹುದು
ಕಳ್ಳ-ನಾ-ದರೊ
ಕಳ್ಳ-ನಿಗೆ
ಕಳ್ಳನೂ
ಕಳ್ಳನೊ
ಕಳ್ಳರ
ಕಳ್ಳ-ರಂತೆ
ಕಳ್ಳ-ರಾ-ಗು-ತ್ತೇವೆ
ಕಳ್ಳ-ರಾ-ದರೆ
ಕಳ್ಳ-ರಿಗೆ
ಕಳ್ಳರು
ಕಳ್ಳಿ-ನಂತೆ
ಕಳ್ಳಿಯ
ಕಳ್ಳು
ಕಳ್ಳು-ಕು-ಡಿದು
ಕವಚ
ಕವಯೋ
ಕವ-ಯೋ-ಽಪ್ಯತ್ರ
ಕವ-ಲೊ-ಡೆ-ದಿಲ್ಲ
ಕವ-ಲೊ-ಡೆದು
ಕವ-ಲೊ-ಡೆ-ಯುತ್ತಾ
ಕವ-ಲೊ-ಡೆ-ಯು-ವಂತೆ
ಕವಿ
ಕವಿಂ
ಕವಿಃ
ಕವಿ-ಗಳಲ್ಲಿ
ಕವಿ-ಗಳು
ಕವಿತ್ವ
ಕವಿದ
ಕವಿ-ದಿ-ರ-ಬ-ಹುದು
ಕವಿ-ದಿ-ರು-ವಾಗ
ಕವಿ-ದಿವೆ
ಕವಿದು
ಕವಿ-ದು-ಕೊಂ-ಡಿದೆ
ಕವಿಯ
ಕವಿ-ಯದೇ
ಕವಿ-ಯಲ್ಲ
ಕವಿ-ಯಾಗಿ
ಕವಿಯೇ
ಕವೀ-ನಾ-ಮು-ಶನಾ
ಕಶ್ಚನ
ಕಶ್ಚಿತ್
ಕಶ್ಚಿ-ತ್ಕ-ರ್ತು-ಮ-ರ್ಹತಿ
ಕಶ್ಚಿ-ದ-ರ್ಥ-ವ್ಯ-ಪಾ-ಶ್ರಯಃ
ಕಶ್ಚಿ-ದೇ-ನ-ಮಾ-ಶ್ಚ-ರ್ಯ-ವ-ದ್ವ-ದತಿ
ಕಶ್ಚಿ-ದ್ದು-ರ್ಗ-ತಿಂ
ಕಶ್ಚಿ-ದ್ಯ-ತತಿ
ಕಶ್ಚಿ-ನ್ಮಾಂ
ಕಶ್ಚಿನ್ಮೇ
ಕಶ್ಮಲ
ಕಶ್ಮ-ಲ-ಗಳನ್ನೆಲ್ಲಾ
ಕಶ್ಮ-ಲ-ಗಳು
ಕಶ್ಮ-ಲ-ಗ-ಳೆಲ್ಲಾ
ಕಶ್ಮ-ಲದ
ಕಶ್ಮ-ಲ-ದಿಂದ
ಕಶ್ಮ-ಲ-ಮಿದಂ
ಕಶ್ಮ-ಲ-ರ-ಹಿ-ತನೂ
ಕಶ್ಮ-ಲ-ವನ್ನು
ಕಶ್ಮ-ಲ-ವ-ನ್ನೆಲ್ಲ
ಕಶ್ಮ-ಲವೂ
ಕಶ್ಮ-ಲ-ವೆಲ್ಲ
ಕಷಾ-ಯ-ದಂತೆ
ಕಷ್ಚ
ಕಷ್ಚ-ವಾ-ಗಿ-ರು-ವುದು
ಕಷ್ಟ
ಕಷ್ಟ-ಕ-ರ-ವಾ-ಗಿ-ರ-ಬಾ-ರದು
ಕಷ್ಟ-ಕಾ-ರ್ಪ-ಣ್ಯ-ಗಳ
ಕಷ್ಟ-ಕಾ-ರ್ಪ-ಣ್ಯ-ಗ-ಳಿಂ-ದಲೂ
ಕಷ್ಟ-ಕಾ-ಲ-ಕ್ಕಾ-ಗಲಿ
ಕಷ್ಟ-ಕಾ-ಲ-ದಲ್ಲಿ
ಕಷ್ಟಕ್ಕೆ
ಕಷ್ಟ-ಗಳನ್ನು
ಕಷ್ಟ-ಗಳಿಂದ
ಕಷ್ಟ-ಗ-ಳಿಗೂ
ಕಷ್ಟ-ಗಳು
ಕಷ್ಟದ
ಕಷ್ಟ-ದಲ್ಲಿ
ಕಷ್ಟ-ದ-ಲ್ಲಿ-ರು-ವ-ವ-ರನ್ನು
ಕಷ್ಟ-ದಿಂದ
ಕಷ್ಟ-ನ-ಷ್ಟ-ಗ-ಳ-ನ್ನಾ-ದರೂ
ಕಷ್ಟ-ನ-ಷ್ಟ-ಗಳನ್ನು
ಕಷ್ಟ-ನ-ಷ್ಟ-ಗಳು
ಕಷ್ಟ-ಪ-ಟ್ಟರೂ
ಕಷ್ಟ-ಪಟ್ಟು
ಕಷ್ಟ-ಪಡ
ಕಷ್ಟ-ಪ-ಡದೆ
ಕಷ್ಟ-ಪ-ಡ-ಬೇ-ಕಾಗಿ
ಕಷ್ಟ-ಪ-ಡ-ಬೇ-ಕಾ-ಗಿದೆ
ಕಷ್ಟ-ಪ-ಡ-ಬೇ-ಕಾ-ಗಿಲ್ಲ
ಕಷ್ಟ-ಪ-ಡ-ಬೇ-ಕಾ-ಗು-ವುದು
ಕಷ್ಟ-ಪ-ಡ-ಬೇ-ಕಾ-ದರೂ
ಕಷ್ಟ-ಪ-ಡ-ಬೇಕು
ಕಷ್ಟ-ಪ-ಡು-ತ್ತೇವೆ
ಕಷ್ಟ-ಪ-ಡು-ವನು
ಕಷ್ಟ-ಪ-ಡುವು
ಕಷ್ಟ-ಪ-ಡು-ವುದು
ಕಷ್ಟವ
ಕಷ್ಟ-ವನ್ನು
ಕಷ್ಟ-ವನ್ನೂ
ಕಷ್ಟ-ವನ್ನೇ
ಕಷ್ಟ-ವಲ್ಲ
ಕಷ್ಟ-ವಾ-ಗ-ಬ-ಹುದು
ಕಷ್ಟ-ವಾ-ಗಲಿ
ಕಷ್ಟ-ವಾ-ಗಿದೆ
ಕಷ್ಟ-ವಾ-ಗಿ-ರಲಿ
ಕಷ್ಟ-ವಾ-ಗಿ-ರು-ವುದು
ಕಷ್ಟ-ವಾ-ಗು-ವುದು
ಕಷ್ಟ-ವಾದ
ಕಷ್ಟ-ವಾ-ದರೂ
ಕಷ್ಟ-ವಿಲ್ಲ
ಕಷ್ಟ-ವೆಂದು
ಕಷ್ಟವೇ
ಕಷ್ಟ-ವೇನೂ
ಕಷ್ಟ-ವೇನೊ
ಕಷ್ಟವೋ
ಕಷ್ಮ-ಲವೂ
ಕಷ್ವ-ವಿಲ್ಲ
ಕಸ
ಕಸ-ಕಡ್ಡಿ
ಕಸ-ಕ್ಕಿಂತ
ಕಸಕ್ಕೆ
ಕಸ-ಗಳ
ಕಸ-ಗು-ಡಿ-ಸು-ವ-ವರೆಲ್ಲ
ಕಸದ
ಕಸ-ದಿಂದ
ಕಸ-ಬನ್ನು
ಕಸ-ರ-ತ್ತನ್ನು
ಕಸ-ರ-ತ್ತಾ-ಗಿದೆ
ಕಸ-ರ-ತ್ತಿ-ನಿಂ-ದಲೇ
ಕಸ-ರತ್ತು
ಕಸ-ರ-ತ್ತು-ಗಳನ್ನು
ಕಸ-ವನ್ನು
ಕಸ-ವಿಲ್ಲ
ಕಸಿ-ದು-ಕೊಂ-ಡಿತು
ಕಸಿ-ದು-ಕೊ-ಳ್ಳು-ತ್ತಾನೆ
ಕಸಿ-ದು-ಕೊ-ಳ್ಳುವ
ಕಸಿ-ಮಾ-ಡುತ್ತಾ
ಕಸಿ-ಯ-ಬೇ-ಕಾ-ದರೆ
ಕಸು-ಬನ್ನು
ಕಸು-ಬು-ಗಳನ್ನು
ಕಸ್ತೂರಿ
ಕಸ್ತೂ-ರಿ-ಮೃ-ಗದ
ಕಸ್ತೂ-ರಿ-ಯಂತೆ
ಕಸ್ತೂ-ರಿ-ಯನ್ನು
ಕಸ್ಮಾಚ್ಚ
ಕಸ್ಯ-ಚಿತ್
ಕಹಿ
ಕಹಿ-ಯನ್ನು
ಕಹಿ-ಯ-ನ್ನೇನೂ
ಕಹಿ-ಯಲ್ಲ
ಕಹಿ-ಯಾಗಿ
ಕಹಿ-ಯಾ-ಗಿ-ದ್ದರೆ
ಕಹಿ-ಯಾ-ಗಿರ
ಕಹಿ-ಯಾ-ಗಿ-ರು-ವಂ-ತಹ
ಕಹಿ-ಯಾ-ಗಿ-ರು-ವು-ದನ್ನೂ
ಕಹಿ-ಯಾ-ಗಿ-ರು-ವುದು
ಕಹಿ-ಯಾ-ಗು-ವುದು
ಕಹಿ-ಯಾದ
ಕಹಿಯೂ
ಕಹಿಯೆ
ಕಹಿಯೇ
ಕಾ
ಕಾಂ
ಕಾಂಕ್ಷಂತಃ
ಕಾಂಕ್ಷತಿ
ಕಾಂಕ್ಷಿತಂ
ಕಾಂಕ್ಷೇ
ಕಾಂಚನ
ಕಾಂಚನಃ
ಕಾಂಡಿ-ಗಳು
ಕಾಂತಿ
ಕಾಂತಿ-ಗ-ಳೆಲ್ಲ
ಕಾಂತಿ-ಗಿಂತ
ಕಾಂತಿಯ
ಕಾಂತಿ-ಯಂತೆ
ಕಾಂತಿ-ಯದು
ಕಾಂತಿ-ಯ-ನ್ನಲ್ಲ
ಕಾಂತಿ-ಯನ್ನು
ಕಾಂತಿ-ಯನ್ನೂ
ಕಾಂತಿ-ಯಲ್ಲ
ಕಾಂತಿ-ಯಲ್ಲಿ
ಕಾಂತಿ-ಯ-ವ-ರೆಗೆ
ಕಾಂತಿ-ಯಿಂದ
ಕಾಂತಿ-ಯಿಂ-ದಲೆ
ಕಾಂತಿ-ಯಿಲ್ಲ
ಕಾಂತಿಯೂ
ಕಾಂತಿಯೇ
ಕಾಕ-ಪಿ-ಷ್ಟ-ದಂತೆ
ಕಾಗದ
ಕಾಗ-ದದ
ಕಾಗ-ದ-ದಲ್ಲಿ
ಕಾಗ-ದ-ವನ್ನು
ಕಾಗೆ-ಗಳ
ಕಾಟ
ಕಾಟ-ದಿಂದ
ಕಾಟ-ವಲ್ಲ
ಕಾಟಾ-ಚಾ-ರ-ಕ್ಕಾಗಿ
ಕಾಟಾ-ಚಾ-ರಕ್ಕೆ
ಕಾಟಾ-ಚಾ-ರ-ದಿಂದ
ಕಾಟಾ-ಚಾ-ರ-ವಾ-ಗಲಿ
ಕಾಡನ್ನು
ಕಾಡ-ಬ-ಹುದು
ಕಾಡ-ಬೇಡ
ಕಾಡ-ಲಾ-ರದು
ಕಾಡ-ಲಾ-ರಳು
ಕಾಡಲು
ಕಾಡಾ-ನೆ-ಯನ್ನು
ಕಾಡಿಗೆ
ಕಾಡಿದ
ಕಾಡಿ-ದ-ವನು
ಕಾಡಿನ
ಕಾಡಿ-ನಲ್ಲಿ
ಕಾಡಿ-ನ-ಲ್ಲಿಯೇ
ಕಾಡು
ಕಾಡು-ಗಿ-ಚ್ಚಿಗೆ
ಕಾಡು-ಗಿಚ್ಚು
ಕಾಡುತ್ತ
ಕಾಡು-ತ್ತಲೇ
ಕಾಡು-ತ್ತವೆ
ಕಾಡು-ತ್ತಿದೆ
ಕಾಡು-ತ್ತಿ-ರು-ವು-ದಕ್ಕೆ
ಕಾಡು-ತ್ತಿ-ರು-ವುದು
ಕಾಡು-ದಾರಿ-ಯಲ್ಲಿ
ಕಾಡುವ
ಕಾಡು-ವನು
ಕಾಡು-ವಾಗ
ಕಾಡು-ವು-ದಕ್ಕೆ
ಕಾಡು-ವುದನ್ನು
ಕಾಡು-ವು-ದಿಲ್ಲ
ಕಾಡು-ವುದು
ಕಾಡು-ವುವು
ಕಾಡ್ಗಿ-ಚ್ಚಾ-ಗು-ವುದು
ಕಾಡ್ಗಿಚ್ಚು
ಕಾಣ-ತ್ತಿದೆ
ಕಾಣದ
ಕಾಣ-ದಂತೆ
ಕಾಣ-ದ-ವ-ನಾ-ಗಿ-ದ್ದಾನೆ
ಕಾಣ-ದಾದ
ಕಾಣ-ದಿ-ರಲಿ
ಕಾಣದು
ಕಾಣ-ದುದು
ಕಾಣದೆ
ಕಾಣದೇ
ಕಾಣದೋ
ಕಾಣ-ದ್ದನ್ನು
ಕಾಣ-ಬ-ಹುದು
ಕಾಣ-ಬೇ-ಕಲ್ಲ
ಕಾಣ-ಬೇಕು
ಕಾಣ-ಬೇ-ಕೆಂ-ದಲ್ಲ
ಕಾಣರು
ಕಾಣ-ಲಾ-ರೆನು
ಕಾಣ-ಲಿಲ್ಲ
ಕಾಣಲು
ಕಾಣ-ವು-ದೆಲ್ಲಾ
ಕಾಣಿ-ಸ-ದಂತೆ
ಕಾಣಿ-ಸ-ದಾಗ
ಕಾಣಿ-ಸದೇ
ಕಾಣಿ-ಸ-ಲಾರ
ಕಾಣಿಸಿ
ಕಾಣಿ-ಸಿ-ಕೊಂಡು
ಕಾಣಿ-ಸಿ-ಕೊ-ಳ್ಳು-ತ್ತಿದೆ
ಕಾಣಿ-ಸಿ-ಕೊ-ಳ್ಳು-ವರು
ಕಾಣಿ-ಸಿ-ಕೊ-ಳ್ಳು-ವು-ದಿಲ್ಲ
ಕಾಣಿ-ಸಿ-ಕೊ-ಳ್ಳು-ವುವು
ಕಾಣಿ-ಸು-ತ್ತದೆ
ಕಾಣಿ-ಸು-ವಂತೆ
ಕಾಣಿ-ಸು-ವು-ದಿಲ್ಲ
ಕಾಣಿ-ಸು-ವುದು
ಕಾಣಿ-ಸು-ವುದೊ
ಕಾಣು
ಕಾಣು-ತ್ತದೆ
ಕಾಣು-ತ್ತ-ದೆಯೊ
ಕಾಣು-ತ್ತ-ದೆಯೋ
ಕಾಣು-ತ್ತವೆ
ಕಾಣು-ತ್ತಾನೆ
ಕಾಣು-ತ್ತಾರೆ
ಕಾಣುತ್ತಿ
ಕಾಣು-ತ್ತಿತ್ತು
ಕಾಣು-ತ್ತಿತ್ತೊ
ಕಾಣು-ತ್ತಿದೆ
ಕಾಣು-ತ್ತಿ-ದೆಯೊ
ಕಾಣು-ತ್ತಿದ್ದ
ಕಾಣು-ತ್ತಿ-ದ್ದರೆ
ಕಾಣು-ತ್ತಿ-ರ-ಬ-ಹುದು
ಕಾಣು-ತ್ತಿ-ರ-ಲಿಲ್ಲ
ಕಾಣು-ತ್ತಿರು
ಕಾಣು-ತ್ತಿ-ರು-ತ್ತವೆ
ಕಾಣು-ತ್ತಿ-ರುವ
ಕಾಣು-ತ್ತಿ-ರು-ವಂತೆ
ಕಾಣು-ತ್ತಿ-ರು-ವನು
ಕಾಣು-ತ್ತಿ-ರು-ವ-ನು-ಮತ್ತೆ
ಕಾಣು-ತ್ತಿ-ರು-ವರು
ಕಾಣು-ತ್ತಿ-ರು-ವಾ-ಗಲೂ
ಕಾಣು-ತ್ತಿ-ರು-ವುದು
ಕಾಣು-ತ್ತಿ-ರು-ವುದೂ
ಕಾಣು-ತ್ತಿ-ರು-ವುದೊ
ಕಾಣು-ತ್ತಿ-ರು-ವುವು
ಕಾಣು-ತ್ತಿ-ರುವೆ
ಕಾಣು-ತ್ತಿ-ರು-ವೆನು
ಕಾಣು-ತ್ತಿಲ್ಲ
ಕಾಣು-ತ್ತಿ-ವೆಯೊ
ಕಾಣು-ತ್ತೇವೆ
ಕಾಣುವ
ಕಾಣು-ವಂತೆ
ಕಾಣು-ವಂ-ತೆಯೇ
ಕಾಣು-ವನು
ಕಾಣು-ವನೋ
ಕಾಣು-ವರು
ಕಾಣು-ವ-ವ-ರೆಗೂ
ಕಾಣು-ವಾಗ
ಕಾಣುವು
ಕಾಣು-ವು-ದಕ್ಕೆ
ಕಾಣು-ವುದನ್ನು
ಕಾಣು-ವು-ದಾ-ಗಲಿ
ಕಾಣು-ವು-ದಾ-ವುದೂ
ಕಾಣು-ವು-ದಿಲ್ಲ
ಕಾಣು-ವು-ದಿ-ಲ್ಲವೊ
ಕಾಣು-ವುದು
ಕಾಣು-ವುದೂ
ಕಾಣು-ವು-ದೆಲ್ಲ
ಕಾಣು-ವುದೇ
ಕಾಣು-ವುದೊ
ಕಾಣು-ವು-ದೊಂದೇ
ಕಾಣು-ವುದೋ
ಕಾಣು-ವುವು
ಕಾಣೆ
ಕಾಣೆನು
ಕಾಣೆ-ಯಾ-ಗಿದೆ
ಕಾತ-ರ-ದಿಂದ
ಕಾತ-ರ-ನಾ-ಗಿ-ರು-ವ-ವನು
ಕಾತ-ರ-ನಾ-ಗಿ-ರು-ವು-ದಿಲ್ಲ
ಕಾತ-ರನೂ
ಕಾತ-ರ-ರಾ-ಗಿ-ರ-ಲಿ-ಲ್ಲ-ವೆಂದು
ಕಾತ-ರ-ರಾ-ಗಿ-ರು-ವುದೇ
ಕಾತ-ರಿಸು
ಕಾದ
ಕಾದ-ಬಲ್ಲ
ಕಾದ-ಲ್ಲದೆ
ಕಾದಾಟ
ಕಾದಾ-ಟಕ್ಕೆ
ಕಾದಾ-ಡ-ಬೇಕು
ಕಾದಾ-ಡ-ಬೇಡಿ
ಕಾದಾ-ಡು-ತ್ತಿ-ದ್ದರೆ
ಕಾದಾ-ಡು-ತ್ತಿ-ರು-ವರು
ಕಾದಾ-ಡುವ
ಕಾದಾ-ಡು-ವು-ದಕ್ಕೆ
ಕಾದಿದೆ
ಕಾದಿ-ದೆ-ಅ-ವ-ನಿಗೆ
ಕಾದಿ-ದೆಯೊ
ಕಾದಿದ್ದು
ಕಾದಿ-ರಿ-ಸಿ-ಕೊಂ-ಡಿ-ರು-ವನು
ಕಾದಿ-ರು-ತ್ತಿತ್ತು
ಕಾದಿ-ರು-ತ್ತಿ-ದ್ದುದು
ಕಾದಿ-ರು-ವನು
ಕಾದಿ-ರು-ವುದು
ಕಾದಿ-ವೆಯೊ
ಕಾದು
ಕಾದು-ಕೊಂ-ಡಿದೆ
ಕಾದು-ಕೊಂ-ಡಿ-ದೆಯೊ
ಕಾದು-ಕೊಂ-ಡಿ-ದೆಯೋ
ಕಾದು-ಕೊಂ-ಡಿ-ರ-ಬೇ-ಕಾ-ಗಿಲ್ಲ
ಕಾದು-ಕೊಂ-ಡಿರು
ಕಾದು-ಕೊಂ-ಡಿವೆ
ಕಾನ-ನ-ದಲ್ಲಿ
ಕಾನೂ-ನು-ಗಳು
ಕಾಪಾ-ಡದೆ
ಕಾಪಾ-ಡ-ಬೇಕು
ಕಾಪಾ-ಡಿ-ಕೊಂ-ಡಿ-ರು-ವು-ದಲ್ಲ
ಕಾಪಾ-ಡಿ-ಕೊ-ಳ್ಳು-ವು-ದಕ್ಕೆ
ಕಾಪಾ-ಡಿ-ದರೂ
ಕಾಪಾ-ಡಿಯೇ
ಕಾಪಾ-ಡು-ತ್ತಾನೆ
ಕಾಪಾ-ಡು-ವನು
ಕಾಪಾ-ಡು-ವ-ವನು
ಕಾಪಾ-ಡು-ವುದು
ಕಾಫಿ
ಕಾಫಿಗೆ
ಕಾಮ
ಕಾಮಂ
ಕಾಮಃ
ಕಾಮ-ಕಾಂ-ಚನ
ಕಾಮ-ಕಾಂ-ಚ-ನ-ಗಳು
ಕಾಮ-ಕಾಂ-ಚ-ನದ
ಕಾಮ-ಕಾಂ-ಚ-ನಾ-ಸ-ಕ್ತಿ-ಯಿಂದ
ಕಾಮ-ಕಾಮಾ
ಕಾಮ-ಕಾಮೀ
ಕಾಮ-ಕಾ-ರತಃ
ಕಾಮ-ಕಾ-ರೇಣ
ಕಾಮಕ್ಕೆ
ಕಾಮ-ಕ್ರೋಧ
ಕಾಮ-ಕ್ರೋ-ಧ-ಗಳಿಂದ
ಕಾಮ-ಕ್ರೋ-ಧ-ಗಳು
ಕಾಮ-ಕ್ರೋ-ಧದ
ಕಾಮ-ಕ್ರೋ-ಧ-ಪ-ರಾ-ಯ-ಣಾಃ
ಕಾಮ-ಕ್ರೋ-ಧ-ವಿ-ಯು-ಕ್ತಾ-ನಾಂ
ಕಾಮ-ಕ್ರೋ-ಧವೆ
ಕಾಮ-ಕ್ರೋ-ಧೋ-ದ್ಭವಂ
ಕಾಮ-ಗಳ
ಕಾಮ-ಗಳನ್ನು
ಕಾಮ-ಗಳನ್ನೂ
ಕಾಮ-ಗಳು
ಕಾಮದ
ಕಾಮ-ದಿಂದ
ಕಾಮ-ದೇವ
ಕಾಮ-ಧುಕ್
ಕಾಮ-ಧೇನು
ಕಾಮ-ಧೇ-ನು-ವಾ-ಗಲಿ
ಕಾಮ-ಧೇ-ನು-ವಾ-ಗು-ವುದು
ಕಾಮನೆ
ಕಾಮ-ನೆ-ಗಳನ್ನು
ಕಾಮ-ನೆ-ಗಳನ್ನೆಲ್ಲ
ಕಾಮ-ನೆ-ಗಳಲ್ಲಿ
ಕಾಮ-ನೆ-ಗ-ಳಾ-ವುವೂ
ಕಾಮ-ನೆ-ಗಳಿಂದ
ಕಾಮ-ನೆ-ಗ-ಳಿವೆ
ಕಾಮ-ನೆ-ಗಳು
ಕಾಮ-ನೆ-ಗಳೂ
ಕಾಮ-ನೆ-ಗಳೆ
ಕಾಮ-ನೆ-ಗ-ಳೆಲ್ಲ
ಕಾಮ-ನೆ-ಗಾಗಿ
ಕಾಮ-ನೆ-ಯನ್ನು
ಕಾಮ-ನೆಯೂ
ಕಾಮ-ಪ್ರೇ-ರ-ಣೆ-ಯಿಂದ
ಕಾಮ-ಫ-ಲ-ಗಳನ್ನು
ಕಾಮ-ಭೋ-ಗ-ಗಳಲ್ಲಿ
ಕಾಮ-ಭೋ-ಗಾ-ರ್ಥ-ಮ-ನ್ಯಾ-ಯೇ-ನಾರ್ಥ
ಕಾಮ-ಭೋ-ಗೇಷು
ಕಾಮ-ಮಾ-ಶ್ರಿತ್ಯ
ಕಾಮ-ರಾಗ
ಕಾಮ-ರಾ-ಗ-ಗ-ಳಿ-ಲ್ಲದ
ಕಾಮ-ರಾ-ಗ-ಗ-ಳಿ-ಲ್ಲವೊ
ಕಾಮ-ರಾ-ಗ-ಬ-ಲಾ-ನ್ವಿ-ತಾಃ
ಕಾಮ-ರಾ-ಗ-ವಿ-ವ-ರ್ಜಿ-ತಮ್
ಕಾಮ-ರೂಪಂ
ಕಾಮ-ರೂ-ಪವೂ
ಕಾಮ-ರೂಪಿ
ಕಾಮ-ರೂ-ಪಿ-ಯಾದ
ಕಾಮ-ರೂ-ಪೇಣ
ಕಾಮ-ವನ್ನು
ಕಾಮ-ವಿದೆ
ಕಾಮ-ವಿ-ದೆಯೊ
ಕಾಮ-ವಿ-ದ್ದರೇ
ಕಾಮ-ವಿ-ರು-ವುದೋ
ಕಾಮವೂ
ಕಾಮ-ವೆಂ-ಬುದು
ಕಾಮವೇ
ಕಾಮ-ವೊಂದೇ
ಕಾಮ-ಸಂ-ಕ-ಲ್ಪ-ವ-ರ್ಜಿ-ತಾಃ
ಕಾಮ-ಹೈ-ತು-ಕಮ್
ಕಾಮಾ
ಕಾಮಾಂ-ಸ್ತ್ಯ-ಕ್ತ್ವಾ-ಸ-ರ್ವಾ-ನ-ಶೇ-ಷತಃ
ಕಾಮಾತ್
ಕಾಮಾ-ತ್ಮ-ರಾಗಿ
ಕಾಮಾ-ತ್ಮಾನಃ
ಕಾಮಾ-ದಿ-ಗಳನ್ನು
ಕಾಮಾನ್
ಕಾಮಾ-ಸಕ್ತಿ
ಕಾಮಾ-ಸ-ಕ್ತಿ-ಯನ್ನು
ಕಾಮಿ
ಕಾಮಿ-ಗ-ಳಾ-ಗಿ-ರು-ತ್ತಾರೆ
ಕಾಮಿಸಿ
ಕಾಮಿ-ಸು-ತ್ತೇ-ವೆಯೋ
ಕಾಮಿ-ಸುವ
ಕಾಮುಕ
ಕಾಮು-ಕ-ನಿಗೆ
ಕಾಮೇ-ಪ್ಸುನಾ
ಕಾಮೈ-ಸೆ-ಸೆ-ರ್ಹೃ-ತ-ಜ್ಞಾ-ನಾಃ
ಕಾಮೋ-ಪ-ಭೋ-ಗ-ಪ-ರಮಾ
ಕಾಮೋಽಸ್ಮಿ
ಕಾಮ್ಯ
ಕಾಮ್ಯ-ಕರ್ಮ
ಕಾಮ್ಯ-ಕ-ರ್ಮ-ಗಳನ್ನು
ಕಾಮ್ಯ-ಕ-ರ್ಮ-ವನ್ನು
ಕಾಮ್ಯಾ-ನಾಂ
ಕಾಯಂ
ಕಾಯಕ
ಕಾಯ-ಕ-ಲ್ಪ-ಚಿ-ಕಿ-ತ್ಸೆ-ಯನ್ನು
ಕಾಯ-ಕ್ಲೇ-ಶ-ಭ-ಯಾ-ತ್ತ್ಯ-ಜೇತ್
ಕಾಯ-ತೊ-ಡ-ಗಿತು
ಕಾಯದ
ಕಾಯನ್ನು
ಕಾಯ-ಬಲ್ಲ
ಕಾಯ-ಬೇ-ಕಾಗಿ
ಕಾಯ-ಬೇ-ಕಾ-ಗಿದೆ
ಕಾಯ-ಬೇ-ಕಾ-ಗಿಲ್ಲ
ಕಾಯ-ಬೇ-ಕಾ-ಯಿತು
ಕಾಯ-ಬೇಕು
ಕಾಯಲು
ಕಾಯ-ಶಿ-ರೋ-ಗ್ರೀವಂ
ಕಾಯಾಗು
ಕಾಯಿ
ಕಾಯಿಕ
ಕಾಯಿದೆ
ಕಾಯಿ-ದೆ-ಗ-ಳಿಗೆ
ಕಾಯಿಯ
ಕಾಯಿ-ಯಂತೆ
ಕಾಯಿ-ಯನ್ನು
ಕಾಯಿ-ಯೊ-ಳಗೆ
ಕಾಯಿ-ಸ-ಬೇಕು
ಕಾಯಿ-ಸ-ಹಿತ
ಕಾಯಿಸಿ
ಕಾಯಿ-ಸಿ-ಕೊ-ಳ್ಳು-ವು-ದಕ್ಕೆ
ಕಾಯಿ-ಸಿ-ದರೆ
ಕಾಯಿ-ಸು-ತ್ತಿ-ದ್ದಂತೆ
ಕಾಯಿ-ಸು-ತ್ತಿ-ರು-ವನು
ಕಾಯಿ-ಸು-ತ್ತಿ-ರು-ವಾಗ
ಕಾಯಿ-ಸು-ವುದು
ಕಾಯುತ್ತ
ಕಾಯು-ತ್ತಿದ್ದ
ಕಾಯು-ತ್ತಿದ್ದೆ
ಕಾಯು-ತ್ತಿ-ರುವ
ಕಾಯು-ತ್ತಿ-ರು-ವನು
ಕಾಯು-ತ್ತಿ-ರು-ವ-ವನು
ಕಾಯು-ತ್ತಿ-ರು-ವುವು
ಕಾಯು-ವಂತೆ
ಕಾಯು-ವ-ವರ
ಕಾಯು-ವು-ದಕ್ಕೆ
ಕಾಯು-ವು-ದಿಲ್ಲ
ಕಾಯು-ವುದು
ಕಾಯೇನ
ಕಾಯೊ-ಳಗೆ
ಕಾರ-ಕೂನ
ಕಾರಣ
ಕಾರಣಂ
ಕಾರ-ಣ-ಕ-ರ್ತ-ನಾ-ದ-ವ-ನನ್ನು
ಕಾರ-ಣ-ಕ-ರ್ತ-ರಾದ
ಕಾರ-ಣ-ಕ-ರ್ತರು
ಕಾರ-ಣ-ಕ್ಕಾಗಿ
ಕಾರ-ಣಕ್ಕೆ
ಕಾರ-ಣ-ಗಳ
ಕಾರ-ಣ-ಗಳನ್ನು
ಕಾರ-ಣ-ಗಳನ್ನೆಲ್ಲಾ
ಕಾರ-ಣ-ಗ-ಳ-ಲ್ಲಿ-ರುವ
ಕಾರ-ಣ-ಗಳಿಂದ
ಕಾರ-ಣ-ಗ-ಳಿವೆ
ಕಾರ-ಣ-ಗಳು
ಕಾರ-ಣ-ಗಳೂ
ಕಾರ-ಣ-ಗ-ಳೆಲ್ಲ
ಕಾರ-ಣ-ಗಳೇ
ಕಾರ-ಣ-ದ-ಲ್ಲಿ-ತ-ನ-ಗಾಗಿ
ಕಾರ-ಣ-ದಲ್ಲೆಲ್ಲ
ಕಾರ-ಣ-ದಿಂದ
ಕಾರ-ಣ-ದಿಂ-ದಲೇ
ಕಾರ-ಣ-ದಿಂ-ದಲೊ
ಕಾರ-ಣ-ನಾ-ಗದೆ
ಕಾರ-ಣ-ನಾ-ಗು-ತ್ತಾನೆ
ಕಾರ-ಣ-ನಾ-ಗು-ವ-ವನು
ಕಾರ-ಣ-ನಾದ
ಕಾರ-ಣ-ನಾ-ದರೆ
ಕಾರ-ಣ-ನಾ-ದ-ವನೇ
ಕಾರ-ಣನೂ
ಕಾರ-ಣ-ಭೂ-ತ-ರಾ-ಗು-ತ್ತೇವೆ
ಕಾರ-ಣ-ಮು-ಚ್ಯತೇ
ಕಾರ-ಣ-ರಾ-ಗು-ತ್ತಾರೆ
ಕಾರ-ಣ-ರಾ-ಗು-ತ್ತೇವೆ
ಕಾರ-ಣ-ವನ್ನು
ಕಾರ-ಣ-ವನ್ನೂ
ಕಾರ-ಣ-ವಲ್ಲ
ಕಾರ-ಣ-ವಾ-ಗದೆ
ಕಾರ-ಣ-ವಾ-ಗ-ಬೇ-ಕಲ್ಲ
ಕಾರ-ಣ-ವಾ-ಗಿ-ರುವ
ಕಾರ-ಣ-ವಾ-ಗಿ-ರು-ವು-ದನ್ನೇ
ಕಾರ-ಣ-ವಾ-ಗಿ-ರು-ವು-ದ-ರಲ್ಲಿ
ಕಾರ-ಣ-ವಾ-ಗು-ವುದು
ಕಾರ-ಣ-ವಾದ
ಕಾರ-ಣ-ವಾ-ದರೂ
ಕಾರ-ಣ-ವಾ-ದರೋ
ಕಾರ-ಣ-ವಾ-ದುದು
ಕಾರ-ಣ-ವಿದೆ
ಕಾರ-ಣ-ವಿ-ರು-ವು-ದಿಲ್ಲ
ಕಾರ-ಣ-ವಿಲ್ಲ
ಕಾರ-ಣವೂ
ಕಾರ-ಣವೆ
ಕಾರ-ಣ-ವೆಲ್ಲ
ಕಾರ-ಣವೇ
ಕಾರ-ಣ-ವೇ-ನಿದೆ
ಕಾರ-ಣ-ವೇನು
ಕಾರ-ಣವೋ
ಕಾರ-ಣಾನಿ
ಕಾರ-ಣಾ-ವ-ಸ್ಥೆ-ಯಲ್ಲಿ
ಕಾರನ್ನು
ಕಾರನ್ನೊ
ಕಾರ-ಬ-ಹುದು
ಕಾರ-ಯನ್
ಕಾರಿಗೆ
ಕಾರಿ-ನಂತೆ
ಕಾರಿ-ನಲ್ಲಿ
ಕಾರಿ-ನ-ಲ್ಲಿ-ರುವ
ಕಾರು
ಕಾರುಣ್ಯ
ಕಾರು-ಣ್ಯ-ದಿಂ-ದಲೇ
ಕಾರು-ವುದು
ಕಾರ್ಖಾ-ನೆ-ಗಳು
ಕಾರ್ಖಾ-ನೆಗೆ
ಕಾರ್ಖಾ-ನೆ-ಯ-ಲ್ಲಿ-ರುವ
ಕಾರ್ಖಾ-ನೆ-ಯಿಂದ
ಕಾರ್ಪ-ಣ್ಯ-ಗ-ಳ-ನ್ನಾ-ಗಲೀ
ಕಾರ್ಪ-ಣ್ಯ-ದಿಂದ
ಕಾರ್ಪ-ಣ್ಯ-ದೋ-ಷ-ದಿಂದ
ಕಾರ್ಪ-ಣ್ಯ-ದೋ-ಷೋ-ಪ-ಹ-ತ-ಸ್ವ-ಭಾವಃ
ಕಾರ್ಮಿಕ
ಕಾರ್ಯ
ಕಾರ್ಯಂ
ಕಾರ್ಯ-ಕ-ರ-ಣ-ಕ-ರ್ತೃತ್ವೇ
ಕಾರ್ಯ-ಕಾ-ರಣ
ಕಾರ್ಯ-ಕು-ಶಲಿ
ಕಾರ್ಯಕ್ಕೆ
ಕಾರ್ಯ-ಕ್ರ-ಮ-ಗಳನ್ನು
ಕಾರ್ಯ-ಕ್ಷೇ-ತ್ರಕ್ಕೂ
ಕಾರ್ಯ-ಕ್ಷೇ-ತ್ರಕ್ಕೆ
ಕಾರ್ಯ-ಕ್ಷೇ-ತ್ರ-ಗ-ಳ-ಲ್ಲಾ-ದರೆ
ಕಾರ್ಯ-ಕ್ಷೇ-ತ್ರ-ಗ-ಳ-ಲ್ಲಿಯೂ
ಕಾರ್ಯ-ಕ್ಷೇ-ತ್ರ-ಗಳಿಂದ
ಕಾರ್ಯ-ಕ್ಷೇ-ತ್ರ-ದಲ್ಲಿ
ಕಾರ್ಯ-ಕ್ಷೇ-ತ್ರ-ದ-ಲ್ಲಿಯೂ
ಕಾರ್ಯ-ಕ್ಷೇ-ತ್ರ-ದ-ಲ್ಲಿ-ರಲಿ
ಕಾರ್ಯ-ಗ-ತ-ಮಾ-ಡು-ವು-ದ-ರಲ್ಲಿ
ಕಾರ್ಯ-ಗಳನ್ನು
ಕಾರ್ಯ-ಗಳನ್ನೆಲ್ಲಾ
ಕಾರ್ಯ-ಗ-ಳನ್ನೇ
ಕಾರ್ಯ-ಗಳಲ್ಲಿ
ಕಾರ್ಯ-ಗ-ಳಿ-ಗೆಲ್ಲಾ
ಕಾರ್ಯ-ಗಳು
ಕಾರ್ಯ-ಗ-ಳೆಲ್ಲ
ಕಾರ್ಯತಃ
ಕಾರ್ಯತೇ
ಕಾರ್ಯ-ದಕ್ಷ
ಕಾರ್ಯ-ದ-ಕ್ಷತೆ
ಕಾರ್ಯ-ದಲ್ಲಿ
ಕಾರ್ಯ-ದಿಂದ
ಕಾರ್ಯ-ದೃ-ಷ್ಟಿ-ಯಿಂದ
ಕಾರ್ಯ-ಮಿ-ತ್ಯೇವ
ಕಾರ್ಯ-ಮೇವ
ಕಾರ್ಯ-ರೂ-ಪಕ್ಕೆ
ಕಾರ್ಯ-ರೂ-ಪ-ದಲ್ಲಿ
ಕಾರ್ಯ-ವ-ನ್ನಾ-ದರೂ
ಕಾರ್ಯ-ವನ್ನು
ಕಾರ್ಯ-ವನ್ನೂ
ಕಾರ್ಯವೇ
ಕಾರ್ಯಾ-ಕಾ-ರ್ಯ-ವ್ಯ-ವ-ಸ್ಥಿತೌ
ಕಾರ್ಯಾ-ಕಾರ್ಯೇ
ಕಾರ್ಯೇ
ಕಾರ್ಯೋ-ನ್ಮು-ಖ-ನ-ನ್ನಾಗಿ
ಕಾರ್ಯೋ-ನ್ಮು-ಖ-ನಾಗು
ಕಾರ್ಯೋ-ನ್ಮು-ಖ-ರಾಗ
ಕಾಲ
ಕಾಲಂ
ಕಾಲಃ
ಕಾಲ-ಕ-ಳೆ-ದರೆ
ಕಾಲ-ಕ-ಳೆ-ಯು-ವನು
ಕಾಲ-ಕ-ಳೆ-ಯು-ವೆವು
ಕಾಲ-ಕಾ-ಲಕ್ಕೆ
ಕಾಲ-ಕ್ಕಿಂತ
ಕಾಲಕ್ಕೆ
ಕಾಲ-ಕ್ರ-ಮೇಣ
ಕಾಲ-ಗ-ತಿ-ಯಲ್ಲಿ
ಕಾಲ-ಗ-ಳಾ-ದ-ಮೇಲೆ
ಕಾಲ-ಗಳೇ
ಕಾಲದ
ಕಾಲ-ದಂತೆ
ಕಾಲ-ದ-ಮೇಲೆ
ಕಾಲ-ದ-ಲ್ಲಾ-ಗಲೀ
ಕಾಲ-ದ-ಲ್ಲಾ-ದರೂ
ಕಾಲ-ದಲ್ಲಿ
ಕಾಲ-ದ-ಲ್ಲಿದೆ
ಕಾಲ-ದ-ಲ್ಲಿಯೂ
ಕಾಲ-ದ-ಲ್ಲಿ-ರುವ
ಕಾಲ-ದ-ಲ್ಲಿ-ರು-ವನು
ಕಾಲ-ದ-ಲ್ಲಿ-ರು-ವ-ವರು
ಕಾಲ-ದ-ಲ್ಲಿ-ರು-ವುದು
ಕಾಲ-ದಲ್ಲೆ
ಕಾಲ-ದ-ವ-ರೆಗೆ
ಕಾಲ-ದಿಂದ
ಕಾಲ-ದಿಂ-ದಲೂ
ಕಾಲ-ದೇಶ
ಕಾಲ-ದೇ-ಶ-ಗಳು
ಕಾಲ-ದೇ-ಶ-ನಿ-ಮಿ-ತ್ತದ
ಕಾಲ-ದೊ-ಳಗೆ
ಕಾಲನ
ಕಾಲನ್ನು
ಕಾಲ-ಮೇಲೆ
ಕಾಲರ
ಕಾಲರಾ
ಕಾಲ-ರು-ದ್ರನ
ಕಾಲ-ರು-ದ್ರ-ನಂತೆ
ಕಾಲ-ವನ್ನು
ಕಾಲ-ವ-ನ್ನೆಲ್ಲಾ
ಕಾಲ-ವಾ-ಗ-ಬ-ಹುದು
ಕಾಲ-ವಾ-ಗ-ಬೇ-ಕೆಂದು
ಕಾಲ-ವಾ-ಗಿ-ರು-ವಾಗ
ಕಾಲ-ವಾದ
ಕಾಲ-ವಾ-ದ-ನಂತೆ
ಕಾಲ-ವಾ-ದ-ಮೇಲೂ
ಕಾಲ-ವಾ-ದ-ಮೇಲೆ
ಕಾಲ-ವಾ-ದರು
ಕಾಲ-ವಾ-ದರೆ
ಕಾಲ-ವಿದೆ
ಕಾಲ-ವಿದ್ದು
ಕಾಲ-ವಿಲ್ಲ
ಕಾಲ-ವಿ-ಳಂಬ
ಕಾಲವೂ
ಕಾಲ-ವೆಂಬ
ಕಾಲವೇ
ಕಾಲ-ಹ-ರಣ
ಕಾಲಾ-ತೀತ
ಕಾಲಾ-ತೀ-ತನೊ
ಕಾಲಾ-ನ-ಲ-ಸ-ನ್ನಿ-ಭಾನಿ
ಕಾಲಾ-ನು-ಕಾ-ಲ-ದ-ವ-ರೆಗೆ
ಕಾಲಾ-ವ-ಕಾ-ಶ-ವಾ-ಗ-ಬೇಕು
ಕಾಲಾ-ವಧಿ
ಕಾಲಾ-ವ-ಸ-ರವೂ
ಕಾಲಿಗೆ
ಕಾಲಿನ
ಕಾಲಿ-ನ-ಲ್ಲಿಯೂ
ಕಾಲಿ-ನಿಂದ
ಕಾಲು
ಕಾಲು-ಕೆ-ರೆ-ಯು-ತ್ತಿ-ರ-ಲಿಲ್ಲ
ಕಾಲು-ಗಳನ್ನು
ಕಾಲು-ಗಳು
ಕಾಲು-ಗಳೇ
ಕಾಲು-ಮು-ರಿದು
ಕಾಲುವೆ
ಕಾಲು-ವೆ-ಯನ್ನು
ಕಾಲು-ವೆ-ಯಲ್ಲಿ
ಕಾಲೇ
ಕಾಲೇ-ಜಿಗೆ
ಕಾಲೇ-ಜಿ-ನಿಂದ
ಕಾಲೇ-ನಾ-ತ್ಮನಿ
ಕಾಲೇ-ನೇಹ
ಕಾಲೇಷು
ಕಾಲೋ
ಕಾಲೋಽಸ್ಮಿ
ಕಾಲ್ಪ-ನಿ-ಕ-ವಾ-ಗ-ಲಾ-ರದು
ಕಾಳನ್ನು
ಕಾಳಿ-ಕಾ-ಮಾತೆ
ಕಾಳಿ-ಗಿಂತ
ಕಾಳಿಗೆ
ಕಾಳಿ-ದಾ-ಸನ
ಕಾಳು
ಕಾಳು-ಗ-ಳಿ-ರುವ
ಕಾವನ್ನೂ
ಕಾವಿ-ನಿಂದ
ಕಾವಿ-ನಿಂ-ದಲೇ
ಕಾವು
ಕಾವೇ-ರಿಯ
ಕಾವೇ-ರಿ-ಯಿಂ-ದಲೇ
ಕಾವ್ಯ
ಕಾವ್ಯಕ್ಕೆ
ಕಾವ್ಯ-ಗಳನ್ನು
ಕಾವ್ಯದ
ಕಾವ್ಯ-ದೃ-ಷ್ಟಿ-ಯಿಂದ
ಕಾವ್ಯ-ಮ-ಯ-ವಾಗಿ
ಕಾವ್ಯ-ವನ್ನು
ಕಾವ್ಯ-ವನ್ನೇ
ಕಾವ್ಯ-ವಾ-ಗ-ಬ-ಹುದು
ಕಾವ್ಯ-ವೇ-ಷ-ದಲ್ಲಿ
ಕಾಶಿ-ರಾ-ಜನು
ಕಾಶೀ-ರಾಜ
ಕಾಶೀ-ರಾ-ಜಶ್ಚ
ಕಾಶ್ಯಶ್ಚ
ಕಾಸನ್ನು
ಕಾಸಿ-ಕೊಂ-ಡಾಗ
ಕಾಸು-ತ್ತಿ-ರು-ವಾಗ
ಕಾಸೇ
ಕಾಸೋ
ಕಿಂ
ಕಿಂಕ-ರ್ತವ್ಯ
ಕಿಂಕ-ರ್ತ-ವ್ಯ-ಮೂ-ಢ-ನಾ-ಗು-ವನು
ಕಿಂಚನ
ಕಿಂಚಿತ್
ಕಿಂಚಿ-ದಪಿ
ಕಿಂಚಿ-ದಸ್ತಿ
ಕಿಂದರಿ
ಕಿಂದ-ರಿ-ಯನ್ನು
ಕಿಕ್ಕಿ-ರಿದು
ಕಿಟಕಿ
ಕಿಟ-ಕಿ-ಗ-ಳಂತೆ
ಕಿಟ-ಕಿ-ಗಳು
ಕಿಟ-ಕಿಯ
ಕಿಟಾರ್
ಕಿಟಿಕಿ
ಕಿಟಿ-ಕಿ-ಯಿಂದ
ಕಿಡಿ
ಕಿಡಿ-ಗ-ಳಂತೆ
ಕಿಡಿ-ಗಳು
ಕಿಡಿ-ಯಂತೆ
ಕಿಡಿ-ಯನ್ನು
ಕಿಡಿ-ಯ-ಲ್ಲಿಯೂ
ಕಿಡಿ-ಯಾ-ಗು-ವಂತೆ
ಕಿಡಿಯೇ
ಕಿಡಿ-ಯೊಂದು
ಕಿತ್ತರೆ
ಕಿತ್ತಾ-ಟ-ವಾ-ಗು-ವುದು
ಕಿತ್ತಾದ
ಕಿತ್ತಿ-ರು-ವನು
ಕಿತ್ತು
ಕಿತ್ತು-ಕೊಂ-ಡರೆ
ಕಿತ್ತು-ಕೊಂಡು
ಕಿತ್ತು-ಕೊ-ಳ್ಳ-ಕೂ-ಡದು
ಕಿತ್ತು-ಕೊ-ಳ್ಳು-ವನು
ಕಿತ್ತು-ಕೊ-ಳ್ಳು-ವಾಗ
ಕಿತ್ತು-ಕೊ-ಳ್ಳು-ವು-ದ-ಕ್ಕಿಂತ
ಕಿತ್ತು-ಕೊ-ಳ್ಳು-ವು-ದಕ್ಕೆ
ಕಿತ್ತು-ಬಿ-ಡ-ಬ-ಹುದು
ಕಿತ್ತು-ಹಾಕ
ಕಿತ್ತು-ಹಾ-ಕ-ಬೇಕು
ಕಿತ್ತು-ಹಾಕಿ
ಕಿತ್ತು-ಹಾ-ಕಿ-ದರೆ
ಕಿತ್ತು-ಹಾ-ಕಿ-ಬಿ-ಡ-ಬ-ಹುದು
ಕಿತ್ತು-ಹಾ-ಕಿ-ರು-ವರು
ಕಿತ್ತು-ಹಾ-ಕು-ತ್ತಾನೆ
ಕಿತ್ತು-ಹಾ-ಕು-ವು-ದಕ್ಕೆ
ಕಿತ್ತು-ಹಾ-ಕು-ವು-ದಿಲ್ಲ
ಕಿತ್ತು-ಹೋ-ಗ-ಬೇ-ಕಾ-ದರೆ
ಕಿತ್ತು-ಹೋ-ಗುವ
ಕಿತ್ತು-ಹೋ-ಗು-ವು-ದಿಲ್ಲ
ಕಿತ್ತು-ಹೋ-ಗು-ವುದು
ಕಿತ್ತು-ಹೋ-ದರೆ
ಕಿತ್ತೊ-ಗೆದು
ಕಿಮ-ಕ-ರ್ಮೇತಿ
ಕಿಮ-ಕು-ರ್ವತ
ಕಿಮ-ಧ್ಯಾತ್ಮಂ
ಕಿಮ-ನ್ಯತ್
ಕಿಮಾ-ಚಾರಃ
ಕಿಮಾ-ಸೀತ
ಕಿಮು-ಚ್ಯತೇ
ಕಿಮ್
ಕಿರಣ
ಕಿರ-ಣ-ಗಳ
ಕಿರ-ಣ-ಗ-ಳಾ-ದರೊ
ಕಿರ-ಣ-ಗಳು
ಕಿರ-ಣ-ಗ-ಳೆಲ್ಲ
ಕಿರ-ಣ-ದ-ಲ್ಲಿಯೂ
ಕಿರ-ಣ-ದೆ-ದು-ರಿಗೆ
ಕಿರ-ಣವೇ
ಕಿರಿಚಿ
ಕಿರಿ-ಚಿ-ಕೊಳ್ಳ
ಕಿರಿ-ದಾ-ಗುತ್ತಾ
ಕಿರಿಯ
ಕಿರೀಟ
ಕಿರೀ-ಟ-ಧಾ-ರಿಯೂ
ಕಿರೀ-ಟ-ವನ್ನು
ಕಿರೀ-ಟಿನಂ
ಕಿರೀಟೀ
ಕಿರು-ಕುಳ
ಕಿರು-ಕು-ಳ-ದಿಂದ
ಕಿರು-ಚಿ-ಕೊ-ಳ್ಳುತ್ತಾ
ಕಿಲ-ಕಿಲ
ಕಿಲು-ಬನ್ನು
ಕಿಲುಬು
ಕಿಲೋ-ವಾಟ್
ಕಿಲ್ಬ-ಷ-ರ-ಹಿ-ತನೂ
ಕಿಲ್ಬಿ-ಶ-ರ-ಹಿತ
ಕಿಲ್ಬಿ-ಷಮ್
ಕಿವಿ
ಕಿವಿ-ಕೊ-ಡು-ವು-ದಿಲ್ಲ
ಕಿವಿ-ಗಳ
ಕಿವಿ-ಗ-ಳು-ಳ್ಳದ್ಧು
ಕಿವಿಗೆ
ಕಿವಿ-ಗೊ-ಡನು
ಕಿವಿ-ಗೊ-ಡ-ಲಿಲ್ಲ
ಕಿವಿ-ಗೊ-ಡುವು
ಕಿವಿ-ಗೊ-ಡು-ವು-ದಿಲ್ಲ
ಕಿವಿಯ
ಕಿವಿ-ಯಾರೆ
ಕಿವಿ-ಯಿಂದ
ಕಿವು-ಡನ
ಕಿವು-ಡಾ-ಗು-ವಂತೆ
ಕಿವು-ಡು-ಗಿ-ವಿ-ಯಿಂದ
ಕೀಟ
ಕೀಟ-ಕ್ಕಿಂತ
ಕೀಟ-ತ್ವ-ವನ್ನು
ಕೀಟ-ದಿಂದ
ಕೀಟ-ವನ್ನು
ಕೀಟ-ವಾ-ಗ-ಬ-ಹುದು
ಕೀಟ-ವೊಂದು
ಕೀಟಾ-ವಸ್ಥೆ
ಕೀರಲು
ಕೀರ್ತನೆ
ಕೀರ್ತ-ನೆ-ಯನ್ನು
ಕೀರ್ತ-ನೆ-ಯಿಂದ
ಕೀರ್ತ-ಯಂತೋ
ಕೀರ್ತಿ
ಕೀರ್ತಿಂ
ಕೀರ್ತಿಃ
ಕೀರ್ತಿ-ಗ-ಳಿ-ಸ-ಬೇಕು
ಕೀರ್ತಿ-ಗ-ಳಿ-ಸು-ವ-ವನು
ಕೀರ್ತಿ-ಗಾ-ಗಲಿ
ಕೀರ್ತಿ-ಗಾಗಿ
ಕೀರ್ತಿಗೆ
ಕೀರ್ತಿಗೋ
ಕೀರ್ತಿಯ
ಕೀರ್ತಿ-ಯ-ನ್ನಲ್ಲ
ಕೀರ್ತಿ-ಯನ್ನು
ಕೀರ್ತಿ-ಯ-ನ್ನೆಲ್ಲ
ಕೀರ್ತಿ-ಯನ್ನೊ
ಕೀರ್ತಿಯೆ
ಕೀರ್ತಿ-ಯೆಲ್ಲ
ಕೀರ್ತಿಯೇ
ಕೀರ್ತಿ-ಲಾಭ
ಕೀರ್ತಿ-ಸು-ವನು
ಕೀರ್ತಿ-ಸು-ವರು
ಕೀಲು
ಕೀಲು-ಕೊ-ಡಲು
ಕೀಳ-ಬೇ-ಕಾ-ಗಿಲ್ಲ
ಕೀಳ-ಬೇ-ಕಾ-ದರೆ
ಕೀಳ-ಬೇಕು
ಕೀಳಲ್ಲ
ಕೀಳಾಗಿ
ಕೀಳಾ-ಗು-ವೆವು
ಕೀಳಾದ
ಕೀಳು
ಕೀಳು-ಮ-ಟ್ಟ-ದ-ಲ್ಲಿ-ರು-ವ-ವನು
ಕೀಳು-ವನೊ
ಕೀಳು-ವರು
ಕೀಳು-ವ-ವ-ರಲ್ಲ
ಕೀಳು-ವು-ದಕ್ಕೆ
ಕೀಳು-ವುದೊ
ಕೀವು
ಕುಂಟ-ನನ್ನು
ಕುಂಡ-ದಲ್ಲಿ
ಕುಂಡ-ದ-ಲ್ಲಿ-ರುವ
ಕುಂತಿ-ಭೋಜ
ಕುಂತಿ-ಭೋ-ಜಶ್ಚ
ಕುಂತಿಯ
ಕುಂತೀ-ಪು-ತ್ರ-ನಾದ
ಕುಂತೀ-ಪುತ್ರೋ
ಕುಂದಿದ
ಕುಂದು
ಕುಂದು-ವುದು
ಕುಂಬಾರ
ಕುಂಬಾ-ರ-ಕೇ-ರಿ-ಯಲ್ಲಿ
ಕುಂಬಾ-ರನ
ಕುಂಬಾ-ರ-ರ-ವನು
ಕುಂಭಕ
ಕುಂಭ-ಕ-ದಲ್ಲಿ
ಕುಂಭ-ಕರ್ಣ
ಕುಂಭ-ಕ-ವೆಂಬ
ಕುಕ್ಕುವ
ಕುಕ್ಕು-ವಂ-ತಿವೆ
ಕುಗ್ಗ-ಕೂ-ಡದು
ಕುಗ್ಗದೆ
ಕುಗ್ಗ-ಬ-ಹುದು
ಕುಗ್ಗ-ಬೇ-ಕಾ-ದರೆ
ಕುಗ್ಗಿ
ಕುಗ್ಗಿ-ಬಿ-ಡು-ವು-ದಿಲ್ಲ
ಕುಗ್ಗಿ-ಲ್ಲವೊ
ಕುಗ್ಗಿ-ಸಲು
ಕುಗ್ಗಿ-ಸಿ-ಕೊಳ್ಳ
ಕುಗ್ಗಿ-ಸಿ-ಕೊ-ಳ್ಳ-ಬಾ-ರದು
ಕುಗ್ಗಿ-ಹೋ-ಗು-ತ್ತೇವೆ
ಕುಗ್ಗಿ-ಹೋ-ಗು-ವ-ವ-ನಲ್ಲ
ಕುಗ್ಗಿ-ಹೋ-ಗು-ವು-ದಿಲ್ಲ
ಕುಗ್ಗಿ-ಹೋ-ಗು-ವುದೂ
ಕುಗ್ಗಿ-ಹೋ-ದರೂ
ಕುಗ್ಗು-ವು-ದಿಲ್ಲ
ಕುಗ್ಗು-ವುದೂ
ಕುಗ್ಗು-ವುವು
ಕುಗ್ರಾ-ಮ-ಗಳನ್ನು
ಕುಚೇಲ
ಕುಚೇ-ಲ-ನಿಂದ
ಕುಚೇ-ಲನು
ಕುಟೀ-ರ-ದ-ಲ್ಲಿ-ರ-ಬ-ಹುದು
ಕುಟುಂಬ
ಕುಟುಂ-ಬಕ್ಕೆ
ಕುಟುಂ-ಬ-ವನ್ನು
ಕುಟು-ಕಿ-ದಾಗ
ಕುಟು-ಕಿ-ಸಿ-ಕೊಂ-ಡಂತೆ
ಕುಟು-ಕಿ-ಸಿ-ಕೊಂ-ಡರೂ
ಕುಟು-ಕಿ-ಸಿ-ಕೊ-ಳ್ಳ-ಬೇಕು
ಕುಟು-ಕಿ-ಸಿ-ಕೊ-ಳ್ಳಲು
ಕುಟು-ಕು-ತ್ತಾನೆ
ಕುಟ್ಟಿ
ಕುಟ್ಟು-ತ್ತಾರೆ
ಕುಟ್ಟು-ವುದು
ಕುಟ್ಟೆ
ಕುಡಿ
ಕುಡಿಕೆ
ಕುಡಿ-ಕೆ-ಗಳನ್ನು
ಕುಡಿ-ಕೆ-ಗಳಲ್ಲಿ
ಕುಡಿ-ಕೆ-ಗ-ಳೆಲ್ಲ
ಕುಡಿ-ಕೆಗೆ
ಕುಡಿ-ಕೆ-ಮಾಡಿ
ಕುಡಿ-ಕೆ-ಯನ್ನು
ಕುಡಿ-ಕೆ-ಯಲ್ಲಿ
ಕುಡಿ-ಕೆ-ಯಾಗಿ
ಕುಡಿದ
ಕುಡಿ-ದರೂ
ಕುಡಿ-ದರೆ
ಕುಡಿ-ದ-ಲ್ಲದೆ
ಕುಡಿ-ದ-ವ-ರಂತೆ
ಕುಡಿ-ದ-ವರೂ
ಕುಡಿ-ದಾ-ಗಲೆ
ಕುಡಿ-ದಿದೆ
ಕುಡಿ-ದಿ-ದ್ದಾರೆ
ಕುಡಿ-ದಿ-ರು-ವ-ವನು
ಕುಡಿದು
ಕುಡಿ-ಯದೇ
ಕುಡಿ-ಯನ್ನು
ಕುಡಿ-ಯ-ಬೇಕು
ಕುಡಿಯು
ಕುಡಿ-ಯು-ತ್ತದೆ
ಕುಡಿ-ಯು-ತ್ತಾನೆ
ಕುಡಿ-ಯು-ತ್ತಾರೆ
ಕುಡಿ-ಯು-ತ್ತಿತ್ತು
ಕುಡಿ-ಯು-ತ್ತಿ-ರ-ಬೇಕು
ಕುಡಿ-ಯು-ತ್ತೇವೆ
ಕುಡಿ-ಯುವ
ಕುಡಿ-ಯು-ವನು
ಕುಡಿ-ಯು-ವನೊ
ಕುಡಿ-ಯು-ವರು
ಕುಡಿ-ಯು-ವ-ವರು
ಕುಡಿ-ಯು-ವ-ವ-ರೆಗೆ
ಕುಡಿ-ಯು-ವು-ದ-ಕ್ಕಾ-ಗಲಿ
ಕುಡಿ-ಯು-ವು-ದ-ಕ್ಕಾಗಿ
ಕುಡಿ-ಯು-ವು-ದಕ್ಕೆ
ಕುಡಿ-ಯು-ವು-ದ-ರಲ್ಲಿ
ಕುಡಿ-ಯು-ವುದು
ಕುಡು-ಕರು
ಕುಡು-ಗೋ-ಲನ್ನು
ಕುಣಿ-ದಾ-ಡಲು
ಕುಣಿ-ದಾ-ಡಿದ
ಕುಣಿ-ದಾ-ಡುತ್ತಾ
ಕುಣಿ-ದಾ-ಡು-ವನು
ಕುಣಿ-ದಾ-ಡು-ವನೆ
ಕುಣಿ-ದಾ-ಡು-ವರು
ಕುಣಿ-ದಾ-ಡು-ವು-ದಿಲ್ಲ
ಕುಣಿ-ದಾ-ಡು-ವುದೂ
ಕುಣಿ-ದಾ-ಡು-ವೆವು
ಕುಣಿ-ಯ-ಬೇ-ಕಾ-ಗು-ವುದು
ಕುಣಿ-ಯು-ತ್ತವೆ
ಕುಣಿ-ಯು-ತ್ತಿ-ರು-ವೆವು
ಕುಣಿ-ಯು-ತ್ತೇವೆ
ಕುಣಿ-ಯು-ವು-ದೇನೂ
ಕುಣಿ-ಸಿ-ದರೆ
ಕುಣಿ-ಸು-ತ್ತದೆ
ಕುಣಿ-ಸು-ತ್ತಿ-ರು-ವ-ವ-ನಿಗೆ
ಕುಣಿ-ಸುವ
ಕುಣಿ-ಸು-ವುದು
ಕುತಃ
ಕುತ-ರ್ಕ-ವಾ-ಗು-ವುದು
ಕುತಸ್ತ್ವಾ
ಕುತೂ-ಹಲ
ಕುತೂ-ಹ-ಲದ
ಕುತೂ-ಹ-ಲ-ದಿಂ-ದಲೋ
ಕುತೂ-ಹ-ಲ-ವನ್ನು
ಕುತೂ-ಹ-ಲ-ವಾ-ಯಿತು
ಕುತೂ-ಹ-ಲವೂ
ಕುತೂ-ಹ-ಲಿ-ಗ-ಳಾ-ಗಿ-ರು-ವೆವು
ಕುತೋಽನ್ಯಃ
ಕುತೋಽನ್ಯೋ
ಕುತ್ತಿ-ಗೆಗೆ
ಕುದ-ರೆ-ಗಳಲ್ಲಿ
ಕುದಿ-ಯ-ತೊ-ಡ-ಗು-ವುದು
ಕುದಿ-ಯದೆ
ಕುದಿಯು
ಕುದಿ-ಯು-ತ್ತಲೇ
ಕುದಿ-ಯು-ತ್ತಿ-ರು-ವನು
ಕುದಿ-ಯು-ತ್ತಿ-ರು-ವುದೋ
ಕುದಿ-ಯು-ತ್ತಿ-ರು-ವುವು
ಕುದಿ-ಯು-ತ್ತಿವೆ
ಕುದಿ-ಯುವ
ಕುದಿ-ಯು-ವು-ದಕ್ಕೆ
ಕುದಿ-ಸು-ವರು
ಕುದುರೆ
ಕುದು-ರೆ-ಗಳಲ್ಲಿ
ಕುದು-ರೆ-ಗಳಿಂದ
ಕುದು-ರೆಯ
ಕುದು-ರೆ-ಯನ್ನು
ಕುದುಸಿ
ಕುಬೇರ
ಕುಬೇ-ರನ
ಕುಬೇ-ರರು
ಕುಬ್ಜ-ವಾ-ಗು-ವುದು
ಕುಮಾ-ರ-ಸಂ-ಭ-ವದ
ಕುಯುಕ್ತಿ
ಕುಯ್ಯ-ಬೇ-ಕಾ-ಗಿದೆ
ಕುರ-ಕ್ಷೇ-ತ್ರದ
ಕುರಿ
ಕುರಿ-ತ-ದ್ದನ್ನು
ಕುರಿ-ತ-ದ್ದಲ್ಲ
ಕುರಿ-ತದ್ದು
ಕುರಿ-ತದ್ದೆ
ಕುರಿ-ತದ್ದೇ
ಕುರಿತು
ಕುರಿಯ
ಕುರಿ-ಯನ್ನು
ಕುರು
ಕುರು-ಕು-ಲಕ್ಕೆ
ಕುರು-ಕ್ಷೇತ್ರ
ಕುರು-ಕ್ಷೇ-ತ್ರಕ್ಕೆ
ಕುರು-ಕ್ಷೇ-ತ್ರದ
ಕುರು-ಕ್ಷೇ-ತ್ರ-ದಲ್ಲಿ
ಕುರು-ಕ್ಷೇ-ತ್ರ-ದಿಂದ
ಕುರು-ಕ್ಷೇ-ತ್ರವೇ
ಕುರು-ಕ್ಷೇತ್ರೇ
ಕುರುಡ
ಕುರು-ಡ-ನಿಗೆ
ಕುರುತೇ
ಕುರು-ತೇ-ಽಜುನ
ಕುರು-ನಂ-ದನ
ಕುರು-ಪಿ-ತಾ-ಮ-ಹ-ನಾದ
ಕುರು-ಪ್ರ-ವೀರ
ಕುರು-ಬ-ನಿಗೇ
ಕುರು-ವೃದ್ಧಃ
ಕುರು-ಶ್ರೇಷ್ಠ
ಕುರು-ಸ-ತ್ತಮ
ಕುರೂ-ನಿತಿ
ಕುರೂ-ಪ-ವಾ-ಗಿದೆ
ಕುರ್ಯಾಂ
ಕುರ್ಯಾ-ದ್ವಿ-ದ್ವಾಂ-ಸ್ತ-ಥಾ-ಸ-ಕ್ತ-ಶ್ಚಿ-ಕೀ-ರ್ಷು-ರ್ಲೋ-ಕ-ಸಂ-ಗ್ರ-ಹಮ್
ಕುರ್ವಂತಿ
ಕುರ್ವನ್
ಕುರ್ವ-ನ್ನಪಿ
ಕುರ್ವ-ನ್ನಾ-ಪ್ನೋತಿ
ಕುರ್ವಾಣೋ
ಕುಲ
ಕುಲಂ
ಕುಲ-ಕಾ-ಡು-ತ್ತಿದೆ
ಕುಲಕ್ಕೆ
ಕುಲ-ಕ್ಕೆಲ್ಲ
ಕುಲ-ಕ್ಷ-ಯ-ಕೃತಂ
ಕುಲ-ಕ್ಷ-ಯ-ದಿಂ-ದಾ-ಗುವ
ಕುಲ-ಕ್ಷ-ಯ-ವಾ-ಗು-ವುದು
ಕುಲ-ಕ್ಷ-ಯ-ವಾ-ದರೆ
ಕುಲ-ಕ್ಷಯೇ
ಕುಲ-ಗೆ-ಟ್ಟಿ-ದ್ದರೆ
ಕುಲ-ಗೋ-ತ್ರ-ಗ-ಳಿಗೆ
ಕುಲ-ಘ್ನಾ-ನಾಂ
ಕುಲ-ದಲ್ಲಿ
ಕುಲ-ದ-ವ-ರಿಗೆ
ಕುಲ-ದಿಂ-ದ-ಲಾ-ದರೂ
ಕುಲ-ಧರ್ಮ
ಕುಲ-ಧ-ರ್ಮ-ಗಳನ್ನು
ಕುಲ-ಧ-ರ್ಮ-ಗಳು
ಕುಲ-ಧ-ರ್ಮ-ವನ್ನು
ಕುಲ-ಧರ್ಮಾ
ಕುಲ-ಧ-ರ್ಮಾ-ಣಾಂ
ಕುಲ-ಧ-ರ್ಮಾಶ್ಚ
ಕುಲ-ನಾಶ
ಕುಲ-ನಾ-ಶ-ಕರ
ಕುಲ-ನಾ-ಶ-ದಿಂ-ದಾ-ಗುವ
ಕುಲ-ಪು-ರೋ-ಹಿ-ತ-ನಿಗೆ
ಕುಲ-ಪು-ರೋ-ಹಿ-ತ-ರಿಗೆ
ಕುಲ-ವ-ನ್ನೆಲ್ಲ
ಕುಲವೇ
ಕುಲ-ಸ್ತ್ರಿಯಃ
ಕುಲ-ಸ್ತ್ರೀ-ಯರು
ಕುಲ-ಸ್ತ್ರೀ-ಯ-ರೆಲ್ಲ
ಕುಲಸ್ಯ
ಕುಲೀನ
ಕುಲು-ಕಾಟ
ಕುಲು-ಕಾ-ಟಕ್ಕೆ
ಕುಲು-ಕಾ-ಟ-ದಿಂದ
ಕುಲು-ಕಾ-ಟ-ವನ್ನು
ಕುಲು-ಕಾ-ಟವೇ
ಕುಲು-ಕಾ-ಡು-ತ್ತಿದೆ
ಕುಲು-ಕಾ-ಡು-ವುದು
ಕುಲೇ
ಕುಲೋ-ದ್ಭ-ವನೆ
ಕುಳಿತ
ಕುಳಿ-ತಂ-ತಾಗಿ
ಕುಳಿ-ತ-ಕೊಳ್ಳ
ಕುಳಿ-ತರೆ
ಕುಳಿ-ತ-ವ-ನಲ್ಲ
ಕುಳಿ-ತ-ವನು
ಕುಳಿ-ತಾಗ
ಕುಳಿ-ತಾ-ಗಲೂ
ಕುಳಿ-ತಾ-ಗಲೆ
ಕುಳಿ-ತಿದೆ
ಕುಳಿ-ತಿ-ದ್ದರೂ
ಕುಳಿ-ತಿ-ದ್ದ-ವನು
ಕುಳಿ-ತಿ-ರ-ಬ-ಹುದು
ಕುಳಿ-ತಿ-ರ-ಬೇಕು
ಕುಳಿ-ತಿ-ರ-ಲಾ-ರದೆ
ಕುಳಿ-ತಿ-ರಲಿ
ಕುಳಿ-ತಿ-ರಲು
ಕುಳಿ-ತಿರು
ಕುಳಿ-ತಿ-ರು-ತ್ತದೆ
ಕುಳಿ-ತಿ-ರುವ
ಕುಳಿ-ತಿ-ರು-ವನು
ಕುಳಿ-ತಿ-ರು-ವು-ದಕ್ಕೆ
ಕುಳಿ-ತಿ-ರು-ವುದು
ಕುಳಿ-ತಿಲ್ಲ
ಕುಳಿತು
ಕುಳಿ-ತು-ಕೊಂಡ
ಕುಳಿ-ತು-ಕೊಂ-ಡನು
ಕುಳಿ-ತು-ಕೊಂ-ಡರೂ
ಕುಳಿ-ತು-ಕೊಂ-ಡರೆ
ಕುಳಿ-ತು-ಕೊಂ-ಡ-ವ-ನನ್ನು
ಕುಳಿ-ತು-ಕೊಂ-ಡಾಗ
ಕುಳಿ-ತು-ಕೊಂ-ಡಿತು
ಕುಳಿ-ತು-ಕೊಂ-ಡಿ-ರು-ವರು
ಕುಳಿ-ತು-ಕೊಂ-ಡಿ-ರು-ವು-ದಕ್ಕೆ
ಕುಳಿ-ತು-ಕೊಂ-ಡಿ-ರು-ವುದನ್ನು
ಕುಳಿ-ತು-ಕೊಂ-ಡಿ-ರು-ವುದು
ಕುಳಿ-ತು-ಕೊಂಡು
ಕುಳಿ-ತು-ಕೊ-ಳ್ಳ-ಕೂ-ಡದು
ಕುಳಿ-ತು-ಕೊ-ಳ್ಳ-ಬಾ-ರದು
ಕುಳಿ-ತು-ಕೊ-ಳ್ಳ-ಬೇ-ಕಾ-ಗಿದೆ
ಕುಳಿ-ತು-ಕೊ-ಳ್ಳ-ಬೇ-ಕಾ-ಗಿ-ರು-ವಂತೆ
ಕುಳಿ-ತು-ಕೊ-ಳ್ಳ-ಬೇಕು
ಕುಳಿ-ತು-ಕೊ-ಳ್ಳಲು
ಕುಳಿ-ತು-ಕೊ-ಳ್ಳವ
ಕುಳಿ-ತು-ಕೊಳ್ಳಿ
ಕುಳಿ-ತು-ಕೊ-ಳ್ಳು-ತ್ತಿದ್ದ
ಕುಳಿ-ತು-ಕೊ-ಳ್ಳುವ
ಕುಳಿ-ತು-ಕೊ-ಳ್ಳು-ವನು
ಕುಳಿ-ತು-ಕೊ-ಳ್ಳು-ವ-ವನ
ಕುಳಿ-ತು-ಕೊ-ಳ್ಳು-ವ-ವ-ನಲ್ಲ
ಕುಳಿ-ತು-ಕೊ-ಳ್ಳು-ವಾಗ
ಕುಳಿ-ತು-ಕೊ-ಳ್ಳು-ವು-ದ-ಕ್ಕಿಂತ
ಕುಳಿ-ತು-ಕೊ-ಳ್ಳು-ವು-ದಕ್ಕೆ
ಕುಳಿ-ತು-ಕೊ-ಳ್ಳು-ವು-ದಲ್ಲ
ಕುಳಿ-ತು-ಕೊ-ಳ್ಳು-ವು-ದಿಲ್ಲ
ಕುಳಿ-ತು-ಕೊ-ಳ್ಳು-ವುದು
ಕುಳಿ-ತು-ಕೊ-ಳ್ಳು-ವೆವೊ
ಕುಳಿ-ತು-ವನು
ಕುಳ್ಳಿ-ರಿಸಿ
ಕುಳ್ಳಿ-ರಿ-ಸಿದ
ಕುಳ್ಳಿ-ರಿ-ಸಿ-ದನು
ಕುಳ್ಳಿ-ರಿ-ಸು-ವನು
ಕುಳ್ಳಿ-ರಿ-ಸು-ವೆನು
ಕುಶ
ಕುಶಲ
ಕುಶ-ಲ-ಕ-ಲೆ-ಗಳು
ಕುಶಲಿ
ಕುಶ-ಲಿ-ಗಳಲ್ಲಿ
ಕುಶ-ಲಿ-ಯಾದ
ಕುಶಲೇ
ಕುಶಾ-ಗ್ರ-ಬುದ್ಧಿ
ಕುಶಾ-ಸನ
ಕುಶೋ-ತ್ತ-ರಮ್
ಕುಸ-ಮ-ದಂತೆ
ಕುಸಿದು
ಕುಸಿ-ಯು-ವು-ದಿಲ್ಲ
ಕುಸಿ-ಯು-ವುದು
ಕುಸುಮ
ಕುಸು-ಮ-ಕ್ಕಿಂತ
ಕುಸು-ಮ-ದಂತೆ
ಕುಸು-ಮಾ-ಕರಃ
ಕುಸು-ಮಾ-ದಪಿ
ಕುಸ್ತಿ-ಯಲ್ಲಿ
ಕುಹುಕ
ಕೂಗಿ
ಕೂಗಿಗೆ
ಕೂಟಸ್ಥ
ಕೂಟ-ಸ್ಥ-ನಾ-ಗಿ-ರ-ಬೇಕು
ಕೂಟ-ಸ್ಥನೂ
ಕೂಟ-ಸ್ಥ-ಮ-ಚಲಂ
ಕೂಟ-ಸ್ಥವೂ
ಕೂಟಸ್ಥೋ
ಕೂಟ-ಸ್ಥೋ-ಽಕ್ಷರ
ಕೂಡ
ಕೂಡದು
ಕೂಡ-ಬಾ-ರದು
ಕೂಡಲೆ
ಕೂಡಾ
ಕೂಡಿ
ಕೂಡಿ-ಕೊಂ-ಡಿ-ರ-ಬೇಕು
ಕೂಡಿ-ಕೊಂ-ಡಿ-ರುವ
ಕೂಡಿ-ಕೊಂಡು
ಕೂಡಿ-ಟ್ಟರೂ
ಕೂಡಿ-ಟ್ಟು-ಕೊಂ-ಡಿರು
ಕೂಡಿ-ಡು-ವನು
ಕೂಡಿ-ಡು-ವುದು
ಕೂಡಿ-ಡು-ವೆವು
ಕೂಡಿದ
ಕೂಡಿ-ದರೂ
ಕೂಡಿ-ದ-ವ-ನಾಗಿ
ಕೂಡಿ-ದ-ವ-ನಾ-ದರೆ
ಕೂಡಿ-ದ-ವನು
ಕೂಡಿ-ದ-ವನೂ
ಕೂಡಿ-ದ-ವ-ರಾಗಿ
ಕೂಡಿ-ದ-ವ-ರಿಗೆ
ಕೂಡಿದೆ
ಕೂಡಿ-ದ್ದರೂ
ಕೂಡಿ-ದ್ದರೆ
ಕೂಡಿ-ದ್ದಾನೆ
ಕೂಡಿ-ದ್ದಾರೆ
ಕೂಡಿದ್ದು
ಕೂಡಿರ
ಕೂಡಿ-ರ-ಬೇಕು
ಕೂಡಿ-ರಿ-ಸದು
ಕೂಡಿರು
ಕೂಡಿ-ರು-ತ್ತಾರೆ
ಕೂಡಿ-ರುವ
ಕೂಡಿ-ರು-ವಂತೆ
ಕೂಡಿ-ರು-ವನು
ಕೂಡಿ-ರು-ವನೋ
ಕೂಡಿ-ರು-ವ-ವನು
ಕೂಡಿ-ರು-ವ-ವನೊ
ಕೂಡಿ-ರು-ವ-ವಳು
ಕೂಡಿ-ರು-ವಾಗ
ಕೂಡಿ-ರು-ವುದು
ಕೂಡಿಲ್ಲ
ಕೂಡಿವೆ
ಕೂಡಿ-ಸ-ಬೇ-ಕಾ-ಗಿದೆ
ಕೂಡಿಸಿ
ಕೂಡಿ-ಸಿ-ಕೊ-ಳ್ಳು-ವೆನು
ಕೂಡಿ-ಸಿ-ದ್ದನು
ಕೂಡಿ-ಸಿ-ರು-ವೆವು
ಕೂಡಿ-ಹಾಕಿ
ಕೂಡಿ-ಹಾ-ಕಿ-ಕೊಂ-ಡಿ-ರ-ಬೇಕು
ಕೂಡಿ-ಹಾ-ಕಿ-ಕೊಂ-ಡಿ-ರು-ವನು
ಕೂಡಿ-ಹಾ-ಕಿ-ಕೊಂಡು
ಕೂಡಿ-ಹಾ-ಕಿ-ಕೊ-ಳ್ಳುತ್ತಾ
ಕೂಡಿ-ಹಾ-ಕಿ-ಕೊ-ಳ್ಳು-ತ್ತಾರೆ
ಕೂಡಿ-ಹಾ-ಕಿ-ಕೊ-ಳ್ಳು-ತ್ತೇವೆ
ಕೂಡಿ-ಹಾ-ಕು-ವು-ದಿಲ್ಲ
ಕೂಡಿ-ಹಾ-ಕು-ವು-ದೊಂದೇ
ಕೂಡು-ವುದು
ಕೂತ
ಕೂತರೆ
ಕೂತ-ಲ್ಲಿಯೋ
ಕೂತಾ-ಗಲೂ
ಕೂತಿ-ರ-ಲಾ-ರದೆ
ಕೂತು
ಕೂದ-ಲಿ-ನಂತೆ
ಕೂರ-ಲ-ಗಿ-ನಂತೆ
ಕೂರಿ-ಸು-ವುದು
ಕೂರ್ಮೋ-ಽಂಗಾ-ನೀವ
ಕೂಲಂ-ಕ-ಷ-ವಾಗಿ
ಕೂಲಿ
ಕೂಲಿ-ಗಳನ್ನು
ಕೂಲಿಯ
ಕೂಲಿ-ಯನ್ನು
ಕೂಲಿ-ಯಾಳು
ಕೂಳಿಗೆ
ಕೂಳಿ-ನಂತೆ
ಕೃತಂ
ಕೃತಕ
ಕೃತ-ಕ-ವಾಗಿ
ಕೃತ-ಕೃ-ತ್ಯ-ನಾ-ಗಿ-ದ್ದಾನೆ
ಕೃತ-ಕೃ-ತ್ಯ-ನಾ-ಗಿ-ರು-ವನು
ಕೃತ-ಕೃ-ತ್ಯ-ನಾ-ಗು-ತ್ತಾನೆ
ಕೃತ-ಕೃ-ತ್ಯ-ರ-ನ್ನಾಗಿ
ಕೃತ-ಕೃ-ತ್ಯ-ರಾಗಿ
ಕೃತ-ಕೃ-ತ್ಯಶ್ಚ
ಕೃತ-ಜ್ಞ-ತೆಯ
ಕೃತ-ಜ್ಞ-ತೆ-ಯನ್ನು
ಕೃತ-ಜ್ಞ-ನಾ-ಗಿ-ರ-ಬೇ-ಕೆಂದು
ಕೃತ-ನಿ-ಶ್ಚಯಃ
ಕೃತಮ್
ಕೃತ-ವ-ರ್ಮನ
ಕೃತ-ಸಂ-ಕ-ಲ್ಪ-ರಾಗಿ
ಕೃತಾಂ-ಜ-ಲಿ-ರ-ಭಾ-ಷತ
ಕೃತಾಂ-ಜ-ಲಿ-ರ್ವೇ-ಪ-ಮಾನಃ
ಕೃತಾಂತೇ
ಕೃತಾರ್ಥ
ಕೃತಾ-ರ್ಥ-ನಾ-ಗು-ತ್ತೇನೆ
ಕೃತಾ-ರ್ಥ-ನಾದೆ
ಕೃತಾ-ರ್ಥ-ರಾಗಿ
ಕೃತೇ-ನಾರ್ಥೋ
ಕೃತ್ಯ-ಕೃ-ತ್ಯನೂ
ಕೃತ್ಯ-ಕ್ಕಿಂತ
ಕೃತ್ಯ-ಗಳನ್ನು
ಕೃತ್ಯದ
ಕೃತ್ಯ-ವನ್ನು
ಕೃತ್ವಾ
ಕೃತ್ವಾಪಿ
ಕೃತ್ಸ್ನ
ಕೃತ್ಸ್ನಂ
ಕೃತ್ಸ್ನ-ಮ-ಧ-ರ್ಮೋ-ಭಿ-ಭ-ವ-ತ್ಯುತ
ಕೃತ್ಸ್ನ-ಮ-ಧ್ಯಾತ್ಮಂ
ಕೃತ್ಸ್ನ-ಮ-ವಶಂ
ಕೃತ್ಸ್ನ-ಮೇ-ಕಾಂ-ಶೇನ
ಕೃತ್ಸ್ನ-ವ-ದೇ-ಕ-ಸ್ಮಿನ್
ಕೃತ್ಸ್ನ-ವಿನ್ನ
ಕೃತ್ಸ್ನಸ್ಯ
ಕೃಪ
ಕೃಪ-ಣರು
ಕೃಪ-ಣಾಃ
ಕೃಪನೇ
ಕೃಪಯಾ
ಕೃಪ-ಯಾ-ವಿ-ಷ್ಟ-ಮ-ಶ್ರು-ಪೂ-ರ್ಣಾ-ಕು-ಲೇ-ಕ್ಷ-ಣಮ್
ಕೃಪಶ್ಚ
ಕೃಪಾ-ಕ-ಟಾ-ಕ್ಷವೇ
ಕೃಪಾ-ಚಾ-ರ್ಯರು
ಕೃಪಾ-ಸಿಂಧು
ಕೃಪಾ-ಹಸ್ತ
ಕೃಪಾ-ಹ-ಸ್ತ-ದ-ಡಿ-ಯಲ್ಲಿ
ಕೃಪೆ
ಕೃಪೆಗೆ
ಕೃಪೆಯ
ಕೃಪೆ-ಯಂತೆ
ಕೃಪೆ-ಯಿಂದ
ಕೃಪೆ-ಯಿಂ-ದಲೇ
ಕೃಪೆ-ಯಿಟ್ಟು
ಕೃಪೇಣ
ಕೃಷಿ
ಕೃಷಿ-ಗೌ-ರ-ಕ್ಷ್ಯ-ವಾ-ಣಿಜ್ಯಂ
ಕೃಷಿ-ಮಾ-ಡಿದ
ಕೃಷ್ಣ
ಕೃಷ್ಣಂ
ಕೃಷ್ಣಃ
ಕೃಷ್ಣ-ದ್ವೈ-ಪಾ-ಯನ
ಕೃಷ್ಣ-ನಂತೆ
ಕೃಷ್ಣ-ನನ್ನು
ಕೃಷ್ಣ-ನ-ಲ್ಲ-ದ-ವರು
ಕೃಷ್ಣ-ನಿಗೆ
ಕೃಷ್ಣನೂ
ಕೃಷ್ಣ-ಪ-ಕ್ಷ-ಗ-ಳೆಂಬ
ಕೃಷ್ಣ-ಪ-ಕ್ಷದ
ಕೃಷ್ಣರ
ಕೃಷ್ಣ-ರನ್ನು
ಕೃಷ್ಣರು
ಕೃಷ್ಣಾ-ಜಿ-ನದ
ಕೃಷ್ಣಾ-ಜಿ-ನ-ವನ್ನು
ಕೃಷ್ಣಾತ್
ಕೃಷ್ಣಾಯ
ಕೃಷ್ಣಾ-ರ್ಜು-ನರು
ಕೃಷ್ಣಾ-ರ್ಪಿತ
ಕೃಷ್ಣೋ
ಕೆಂಗೆಂಡ
ಕೆಂಗೆಂ-ಡ-ದಂ-ತಾ-ಗು-ವುದೊ
ಕೆಂಗೆಂ-ಡ-ವಾಗಿ
ಕೆಂಡ
ಕೆಂಡ-ವ-ನ್ನೆಲ್ಲಾ
ಕೆಂಡ-ವಾ-ಗು-ವುದು
ಕೆಂಡವೇ
ಕೆಂದಾ-ವರೆ-ಯಂ-ತಿ-ರುವ
ಕೆಂದಾ-ವರೆ-ಯಂತೆ
ಕೆಚ್ಚ-ಲನ್ನು
ಕೆಚ್ಚ-ಲಲ್ಲ
ಕೆಚ್ಚ-ಲಲ್ಲಿ
ಕೆಚ್ಚ-ಲಿ-ನಲ್ಲಿ
ಕೆಚ್ಚಲು
ಕೆಟ್ಟ
ಕೆಟ್ಟ-ಕೆ-ಲಸ
ಕೆಟ್ಟ-ತ-ನ-ವನ್ನು
ಕೆಟ್ಟ-ದನ್ನು
ಕೆಟ್ಟ-ದನ್ನೂ
ಕೆಟ್ಟ-ದ-ರಲ್ಲಿ
ಕೆಟ್ಟ-ದಾ-ಗ-ಲಾ-ರದು
ಕೆಟ್ಟ-ದ್ದಕ್ಕೆ
ಕೆಟ್ಟ-ದ್ದನ್ನು
ಕೆಟ್ಟ-ದ್ದನ್ನೇ
ಕೆಟ್ಟ-ದ್ದ-ರಿಂದ
ಕೆಟ್ಟ-ದ್ದಲ್ಲ
ಕೆಟ್ಟ-ದ್ದಾ-ಗ-ಬೇ-ಕಲ್ಲ
ಕೆಟ್ಟ-ದ್ದಾ-ಗಿ-ರಲಿ
ಕೆಟ್ಟ-ದ್ದಾ-ಗು-ವುದು
ಕೆಟ್ಟ-ದ್ದಾ-ದರೆ
ಕೆಟ್ಟದ್ದು
ಕೆಟ್ಟ-ದ್ದು-ದರ
ಕೆಟ್ಟದ್ದೂ
ಕೆಟ್ಟದ್ದೊ
ಕೆಟ್ಟರೆ
ಕೆಟ್ಟ-ವ-ನನ್ನು
ಕೆಟ್ಟ-ವ-ನಿಗೆ
ಕೆಟ್ಟ-ವನು
ಕೆಟ್ಟ-ವ-ರಿಲ್ಲ
ಕೆಟ್ಟವು
ಕೆಟ್ಟು
ಕೆಟ್ಟು-ಹೋಗಿ
ಕೆಟ್ಟು-ಹೋ-ಗಿದೆ
ಕೆಟ್ಟು-ಹೋ-ಗು-ವು-ದಿ-ಲ್ಲವೆ
ಕೆಟ್ಟು-ಹೋ-ದರೆ
ಕೆಟ್ಟು-ಹೋ-ಯಿತು
ಕೆಡ-ಕನ್ನು
ಕೆಡದೆ
ಕೆಡ-ವಲು
ಕೆಡವಿ
ಕೆಡ-ಹು-ವು-ದಿಲ್ಲ
ಕೆಡ-ಹು-ವುದು
ಕೆಡ-ಹು-ವುದೂ
ಕೆಡಿ-ಸ-ಲಾ-ರವು
ಕೆಡಿಸಿ
ಕೆಡಿ-ಸಿ-ಕೊ-ಳ್ಳು-ತ್ತೇವೆ
ಕೆಡಿ-ಸು-ತ್ತಾನೆ
ಕೆಡಿ-ಸು-ತ್ತಿ-ರು-ವುದು
ಕೆಡಿ-ಸು-ವೆವು
ಕೆಡು-ವರು
ಕೆಡು-ವಳೋ
ಕೆಡು-ವು-ದಿಲ್ಲ
ಕೆಡು-ವುದು
ಕೆಡು-ಹುವ
ಕೆಣ-ಕ-ಲಾ-ರವು
ಕೆಣಕಿ
ಕೆಣ-ಕು-ತ್ತಾನೆ
ಕೆಣ-ಕು-ವುದೇ
ಕೆದಕ
ಕೆದ-ಕದೆ
ಕೆದ-ಕಿ-ದರೆ
ಕೆದ-ಕು-ತ್ತಾನೆ
ಕೆದ-ಕು-ವುದು
ಕೆಮಿ-ಕಲ್
ಕೆಮಿಸ್ಟ್ರಿ
ಕೆರ-ಳಿ-ಸ-ಲಾ-ರವು
ಕೆರೆಗೆ
ಕೆರೆಗೊ
ಕೆರೆದು
ಕೆರೆಯ
ಕೆರೆ-ಯಂತೆ
ಕೆರೆ-ಯಲ್ಲಿ
ಕೆರೆ-ಯಿಂದ
ಕೆರೆ-ಯುತ್ತ
ಕೆರೆ-ಯು-ತ್ತಿ-ರು-ವರು
ಕೆರೆ-ಯು-ವುದನ್ನು
ಕೆರೆ-ಯೆಲ್ಲ
ಕೆಲ-ಕಾ-ಲದ
ಕೆಲವ
ಕೆಲ-ವಂತೂ
ಕೆಲ-ವ-ದ-ರ-ಲ್ಲಾ-ದರೂ
ಕೆಲ-ವನ್ನು
ಕೆಲ-ವನ್ನೂ
ಕೆಲ-ವರ
ಕೆಲ-ವ-ರದು
ಕೆಲ-ವ-ರನ್ನು
ಕೆಲ-ವ-ರಲ್ಲಿ
ಕೆಲ-ವ-ರಿಗೆ
ಕೆಲ-ವರು
ಕೆಲವು
ಕೆಲ-ವು-ಕಾಲ
ಕೆಲ-ವು-ಕಾ-ಲದ
ಕೆಲ-ವು-ವೇಳೆ
ಕೆಲವೇ
ಕೆಲಸ
ಕೆಲ-ಸ-ಕಾ-ರ್ಯ-ಗಳನ್ನು
ಕೆಲ-ಸ-ಕ್ಕಿಂತ
ಕೆಲ-ಸಕ್ಕೂ
ಕೆಲ-ಸಕ್ಕೆ
ಕೆಲ-ಸ-ಕ್ಕೇನೊ
ಕೆಲ-ಸ-ಗಳ
ಕೆಲ-ಸ-ಗಳನ್ನು
ಕೆಲ-ಸ-ಗಳನ್ನೂ
ಕೆಲ-ಸ-ಗಳನ್ನೆಲ್ಲ
ಕೆಲ-ಸ-ಗಳನ್ನೆಲ್ಲಾ
ಕೆಲ-ಸ-ಗ-ಳಲ್ಲ
ಕೆಲ-ಸ-ಗಳಲ್ಲಿ
ಕೆಲ-ಸ-ಗಳಿಂದ
ಕೆಲ-ಸ-ಗ-ಳಿಗೂ
ಕೆಲ-ಸ-ಗ-ಳಿಗೆ
ಕೆಲ-ಸ-ಗ-ಳಿ-ಗೆ-ಲ್ಲಕ್ಕೂ
ಕೆಲ-ಸ-ಗಳು
ಕೆಲ-ಸ-ಗಳೂ
ಕೆಲ-ಸ-ಗ-ಳೆ-ನ್ನೆಲ್ಲ
ಕೆಲ-ಸ-ಗ-ಳೆಲ್ಲ
ಕೆಲ-ಸ-ಗ-ಳೆ-ಲ್ಲವೂ
ಕೆಲ-ಸ-ಗಳೇ
ಕೆಲ-ಸದ
ಕೆಲ-ಸ-ದಂತೆ
ಕೆಲ-ಸ-ದಲ್ಲಿ
ಕೆಲ-ಸ-ದ-ಲ್ಲಿ-ರು-ವುದು
ಕೆಲ-ಸ-ದಿಂದ
ಕೆಲ-ಸ-ದೊಂ-ದಿಗೆ
ಕೆಲ-ಸ-ಮಾ-ಡದೆ
ಕೆಲ-ಸ-ಮಾ-ಡ-ಬೇ-ಕಾ-ದರೆ
ಕೆಲ-ಸ-ಮಾ-ಡಲು
ಕೆಲ-ಸ-ಮಾಡಿ
ಕೆಲ-ಸ-ಮಾಡು
ಕೆಲ-ಸ-ಮಾ-ಡು-ತ್ತಾನೆ
ಕೆಲ-ಸ-ಮಾ-ಡು-ತ್ತಿದೆ
ಕೆಲ-ಸ-ಮಾ-ಡು-ತ್ತಿ-ರು-ವನು
ಕೆಲ-ಸ-ಮಾ-ಡು-ತ್ತಿ-ರು-ವನೊ
ಕೆಲ-ಸ-ಮಾ-ಡು-ತ್ತಿ-ರು-ವುದೇ
ಕೆಲ-ಸ-ಮಾ-ಡು-ವವ
ಕೆಲ-ಸ-ಮಾ-ಡು-ವ-ವನು
ಕೆಲ-ಸ-ಮಾ-ಡು-ವ-ವ-ರಲ್ಲಿ
ಕೆಲ-ಸ-ಮಾ-ಡು-ವಾಗ
ಕೆಲ-ಸ-ಮಾ-ಡು-ವು-ದಕ್ಕೆ
ಕೆಲ-ಸ-ಮಾ-ಡು-ವು-ದಲ್ಲ
ಕೆಲ-ಸ-ಮಾ-ಡು-ವು-ದಿಲ್ಲ
ಕೆಲ-ಸ-ಮಾ-ಡು-ವುದು
ಕೆಲ-ಸ-ಮಾ-ಡು-ವುದೇ
ಕೆಲ-ಸ-ವಂತೂ
ಕೆಲ-ಸ-ವ-ನ್ನಾ-ಗಲಿ
ಕೆಲ-ಸ-ವ-ನ್ನಾ-ದರೂ
ಕೆಲ-ಸ-ವನ್ನು
ಕೆಲ-ಸ-ವನ್ನೂ
ಕೆಲ-ಸ-ವ-ನ್ನೆಲ್ಲ
ಕೆಲ-ಸ-ವ-ನ್ನೆಲ್ಲಾ
ಕೆಲ-ಸ-ವನ್ನೇ
ಕೆಲ-ಸ-ವ-ನ್ನೇನೊ
ಕೆಲ-ಸ-ವ-ನ್ನೇನೋ
ಕೆಲ-ಸ-ವನ್ನೋ
ಕೆಲ-ಸ-ವಲ್ಲ
ಕೆಲ-ಸ-ವಾ-ಗ-ಬೇ-ಕಾ-ದರೆ
ಕೆಲ-ಸ-ವಾ-ಗಿದೆ
ಕೆಲ-ಸ-ವಾಗು
ಕೆಲ-ಸ-ವಾ-ಗು-ವು-ದಿ-ಲ್ಲವೆ
ಕೆಲ-ಸ-ವಾ-ಗು-ವುದು
ಕೆಲ-ಸ-ವಾ-ದ-ಮೇಲೆ
ಕೆಲ-ಸ-ವಿಲ್ಲ
ಕೆಲ-ಸ-ವಿ-ಲ್ಲದ
ಕೆಲ-ಸವೂ
ಕೆಲ-ಸ-ವೆಲ್ಲ
ಕೆಲ-ಸವೇ
ಕೆಲ-ಸ-ವೇನು
ಕೆಳ
ಕೆಳ-ಕೆ-ಳಕ್ಕೆ
ಕೆಳ-ಕ್ಕಿ-ರುವ
ಕೆಳಕ್ಕೂ
ಕೆಳಕ್ಕೆ
ಕೆಳ-ಗಡೆ
ಕೆಳ-ಗ-ಡೆಯೂ
ಕೆಳ-ಗ-ಡೆಯೇ
ಕೆಳ-ಗಿನ
ಕೆಳ-ಗಿ-ನ-ದನ್ನು
ಕೆಳ-ಗಿ-ನದು
ಕೆಳ-ಗಿ-ನ-ವ-ನಿಗೆ
ಕೆಳ-ಗಿ-ನ-ವರು
ಕೆಳ-ಗಿ-ನ-ವ-ರೆಗೆ
ಕೆಳ-ಗಿ-ನಿಂದ
ಕೆಳ-ಗಿ-ರಲಿ
ಕೆಳ-ಗಿ-ರುವ
ಕೆಳ-ಗಿ-ರು-ವನು
ಕೆಳ-ಗಿ-ರು-ವ-ವ-ರನ್ನು
ಕೆಳ-ಗಿ-ರು-ವುದನ್ನು
ಕೆಳ-ಗಿ-ರು-ವುದು
ಕೆಳ-ಗಿ-ಳಿದು
ಕೆಳಗೂ
ಕೆಳಗೆ
ಕೆಳಗೋ
ಕೆಳ-ದೃ-ಷ್ಟಿ-ಯಿಂದ
ಕೆಳ-ಭಾಗ
ಕೆಳ-ಮ-ಟ್ಟಕ್ಕೆ
ಕೆಳ-ಮ-ಟ್ಟದ
ಕೆಳ-ಮ-ಟ್ಟ-ದ-ಲ್ಲಿ-ರ-ಬ-ಹುದು
ಕೆಳ-ಮ-ಟ್ಟ-ದ-ಲ್ಲಿ-ರುವ
ಕೆಳ-ಮ-ಟ್ಟ-ದ-ಲ್ಲಿ-ರು-ವ-ವ-ರಿಗೇ
ಕೆಳ-ಮ-ಟ್ಟ-ದ-ಲ್ಲಿ-ರು-ವುದು
ಕೆಳ-ಮ-ಟ್ಟ-ದಿಂದ
ಕೆಳ-ಮ-ಟ್ಟದ್ದು
ಕೆಸ-ರನ್ನು
ಕೆಸ-ರಿಗೆ
ಕೆಸ-ರಿನ
ಕೆಸ-ರಿ-ನಲ್ಲಿ
ಕೆಸ-ರಿ-ನ-ಲ್ಲಿದೆ
ಕೆಸ-ರಿ-ನ-ಲ್ಲಿಯೇ
ಕೆಸರು
ಕೆಸರೂ
ಕೆಸ-ರೆಲ್ಲ
ಕೇ
ಕೇಂದ್ರ
ಕೇಂದ್ರ-ಗಳಲ್ಲಿ
ಕೇಂದ್ರದ
ಕೇಂದ್ರ-ದಲ್ಲಿ
ಕೇಂದ್ರ-ವನ್ನು
ಕೇಂದ್ರವೇ
ಕೇಂದ್ರೀ
ಕೇಂದ್ರೀ-ಕ-ರಿ-ಸ-ಬ-ಲ್ಲನು
ಕೇಂದ್ರೀ-ಕ-ರಿಸಿ
ಕೇಂದ್ರೀ-ಕ-ರಿ-ಸಿದ
ಕೇಂದ್ರೀ-ಕ-ರಿ-ಸಿಲ್ಲ
ಕೇಂದ್ರೀ-ಕ-ರಿ-ಸು-ತ್ತಾನೆ
ಕೇಂದ್ರೀ-ಕ-ರಿ-ಸು-ತ್ತಾ-ನೆಯೊ
ಕೇಂದ್ರೀ-ಕ-ರಿ-ಸು-ವನು
ಕೇಂದ್ರೀ-ಕ-ರಿ-ಸು-ವುದು
ಕೇಂದ್ರೀ-ಕೃತ
ಕೇಂದ್ರೀ-ಕೃ-ತ-ವಾ-ಗಿದೆ
ಕೇಂದ್ರೀ-ಕೃ-ತ-ವಾ-ಗಿ-ಲ್ಲವೋ
ಕೇಂದ್ರೀ-ಕೃ-ತ-ವಾ-ಗು-ವುದು
ಕೇಂದ್ರೀ-ಕೃ-ತ-ವಾದ
ಕೇಚಿ-ದಾ-ತ್ಮಾ-ನ-ಮಾ-ತ್ಮನಾ
ಕೇಚಿ-ದ್ಭೀ-ತಾಃ
ಕೇಚಿ-ದ್ವಿ-ಲಗ್ನಾ
ಕೇಡನ್ನು
ಕೇಡಾ-ಗು-ವುದು
ಕೇಡು
ಕೇಡೂ
ಕೇನ
ಕೇನ-ಚಿತ್
ಕೇರಿ-ಗಳಿಂದ
ಕೇಳ
ಕೇಳದ
ಕೇಳದೆ
ಕೇಳದೇ
ಕೇಳ-ಬ-ಯ-ಸು-ತ್ತಾನೆ
ಕೇಳ-ಬ-ಹುದು
ಕೇಳ-ಬಾ-ರದ
ಕೇಳ-ಬೇ-ಕಾಗಿ
ಕೇಳ-ಬೇ-ಕಾ-ಗಿ-ದೆಯೊ
ಕೇಳ-ಬೇ-ಕಾ-ಗಿ-ರು-ವುದು
ಕೇಳ-ಬೇ-ಕಾ-ಗುವ
ಕೇಳ-ಬೇ-ಕಾ-ಗು-ವುದು
ಕೇಳ-ಬೇ-ಕಾ-ದರೆ
ಕೇಳ-ಬೇ-ಕಾ-ದಾಗ
ಕೇಳ-ಬೇಕು
ಕೇಳ-ಬೇ-ಕೆಂದು
ಕೇಳ-ಲಾರ
ಕೇಳ-ಲಾ-ರದು
ಕೇಳ-ಲಾ-ರನೊ
ಕೇಳಲಿ
ಕೇಳ-ಲಿಲ್ಲ
ಕೇಳಲು
ಕೇಳಿ
ಕೇಳಿ-ಕೊಂಡ
ಕೇಳಿ-ಕೊಂ-ಡರೆ
ಕೇಳಿ-ಕೊಂ-ಡ-ವನು
ಕೇಳಿ-ಕೊಂ-ಡಾಗ
ಕೇಳಿ-ಕೊಂ-ಡಿ-ದ್ದಾನೆ
ಕೇಳಿ-ಕೊಂ-ಡಿ-ರು-ವೆವೊ
ಕೇಳಿ-ಕೊಂಡು
ಕೇಳಿ-ಕೊ-ಳ್ಳ-ಬೇಕು
ಕೇಳಿ-ಕೊ-ಳ್ಳು-ತ್ತಾನೆ
ಕೇಳಿ-ಕೊ-ಳ್ಳು-ತ್ತಾರೆ
ಕೇಳಿ-ಕೊ-ಳ್ಳು-ತ್ತಿದ್ದ
ಕೇಳಿ-ಕೊ-ಳ್ಳು-ತ್ತಿ-ದ್ದಾನೆ
ಕೇಳಿ-ಕೊ-ಳ್ಳು-ತ್ತೇನೆ
ಕೇಳಿ-ಕೊ-ಳ್ಳು-ವನು
ಕೇಳಿತು
ಕೇಳಿದ
ಕೇಳಿ-ದಂತೆ
ಕೇಳಿ-ದನು
ಕೇಳಿ-ದನೋ
ಕೇಳಿ-ದರು
ಕೇಳಿ-ದರೂ
ಕೇಳಿ-ದರೆ
ಕೇಳಿ-ದಳು
ಕೇಳಿ-ದ-ವನು
ಕೇಳಿ-ದ-ವ-ರಿಗೆ
ಕೇಳಿ-ದ-ವ-ರಿ-ಗೆಲ್ಲ
ಕೇಳಿ-ದ-ವರು
ಕೇಳಿ-ದ-ವರೂ
ಕೇಳಿ-ದ-ವರೆ-ದೆಗೆ
ಕೇಳಿ-ದಾಗ
ಕೇಳಿ-ದಿರ
ಕೇಳಿ-ದು-ದನ್ನು
ಕೇಳಿ-ದು-ದ-ನ್ನೆಲ್ಲ
ಕೇಳಿ-ದು-ದರ
ಕೇಳಿದೆ
ಕೇಳಿ-ದೆನು
ಕೇಳಿ-ದೆಯಾ
ಕೇಳಿ-ದೊ-ಡನೆ
ಕೇಳಿದ್ದ
ಕೇಳಿ-ದ್ದನ್ನು
ಕೇಳಿ-ದ್ದ-ನ್ನೆಲ್ಲ
ಕೇಳಿ-ದ್ದ-ನ್ನೆಲ್ಲಾ
ಕೇಳಿ-ದ್ದನ್ನೇ
ಕೇಳಿ-ದ್ದರ
ಕೇಳಿ-ದ್ದ-ರಿಂ-ದಲೆ
ಕೇಳಿ-ದ್ದರೆ
ಕೇಳಿದ್ದು
ಕೇಳಿ-ದ್ದೆವೊ
ಕೇಳಿ-ಬಂತು
ಕೇಳಿ-ರ-ಲಿಲ್ಲ
ಕೇಳಿರು
ಕೇಳಿ-ರು-ತ್ತೇವೆ
ಕೇಳಿ-ರು-ವನು
ಕೇಳಿ-ರು-ವರು
ಕೇಳಿ-ರು-ವೆವು
ಕೇಳಿಲ್ಲ
ಕೇಳಿ-ಸಿ-ಕೊಂಡು
ಕೇಳಿ-ಸಿ-ಕೊ-ಳ್ಳದೆ
ಕೇಳಿ-ಸಿ-ಕೊ-ಳ್ಳ-ದೆಯೇ
ಕೇಳಿ-ಸು-ತ್ತಿಲ್ಲ
ಕೇಳಿ-ಸು-ವುದು
ಕೇಳಿ-ಸು-ವುದೂ
ಕೇಳಿ-ಸು-ವೆವು
ಕೇಳು
ಕೇಳುತ್ತ
ಕೇಳು-ತ್ತ-ದಲ್ಲ
ಕೇಳು-ತ್ತಲೇ
ಕೇಳು-ತ್ತವೆ
ಕೇಳುತ್ತಾ
ಕೇಳು-ತ್ತಾನೆ
ಕೇಳು-ತ್ತಾ-ನೆಯೊ
ಕೇಳು-ತ್ತಾ-ನೆಯೋ
ಕೇಳು-ತ್ತಾನೊ
ಕೇಳು-ತ್ತಾನೋ
ಕೇಳು-ತ್ತಾರೆ
ಕೇಳು-ತ್ತಾರೊ
ಕೇಳು-ತ್ತಾಳೆ
ಕೇಳು-ತ್ತಿದ್ದ
ಕೇಳು-ತ್ತಿ-ದ್ದರೂ
ಕೇಳು-ತ್ತಿ-ದ್ದರೆ
ಕೇಳು-ತ್ತಿ-ದ್ದ-ವರು
ಕೇಳು-ತ್ತಿ-ದ್ದೇನೆ
ಕೇಳು-ತ್ತಿ-ರ-ಬೇಕು
ಕೇಳು-ತ್ತಿ-ರು-ತ್ತೇವೆ
ಕೇಳು-ತ್ತಿ-ರುವ
ಕೇಳು-ತ್ತಿ-ರು-ವಂತೆ
ಕೇಳು-ತ್ತಿ-ರು-ವನು
ಕೇಳು-ತ್ತಿ-ರು-ವಾಗ
ಕೇಳು-ತ್ತಿ-ರು-ವುದು
ಕೇಳು-ತ್ತಿಲ್ಲ
ಕೇಳು-ತ್ತೇವೆ
ಕೇಳು-ತ್ತೇ-ವೆಯೊ
ಕೇಳು-ತ್ತೇ-ವೆಯೋ
ಕೇಳುವ
ಕೇಳು-ವಂ-ತಿದೆ
ಕೇಳು-ವಂ-ತಿಲ್ಲ
ಕೇಳು-ವಂತೆ
ಕೇಳು-ವಂ-ತೆಯೇ
ಕೇಳು-ವನು
ಕೇಳು-ವನೊ
ಕೇಳು-ವರು
ಕೇಳು-ವ-ವನ
ಕೇಳು-ವ-ವ-ನಲ್ಲ
ಕೇಳು-ವ-ವ-ನಿಗೆ
ಕೇಳು-ವ-ವನು
ಕೇಳು-ವ-ವ-ರಲ್ಲ
ಕೇಳು-ವ-ವ-ರಿಗೂ
ಕೇಳು-ವ-ವ-ರಿ-ಗೆಲ್ಲ
ಕೇಳು-ವ-ವ-ರಿಲ್ಲ
ಕೇಳು-ವ-ವರು
ಕೇಳು-ವಷ್ಟು
ಕೇಳು-ವಾಗ
ಕೇಳು-ವಾ-ಗಲೂ
ಕೇಳುವು
ಕೇಳು-ವು-ದಕ್ಕೆ
ಕೇಳು-ವುದನ್ನು
ಕೇಳು-ವು-ದ-ನ್ನೆಲ್ಲ
ಕೇಳು-ವು-ದ-ರಲ್ಲಿ
ಕೇಳು-ವು-ದ-ರಿಂದ
ಕೇಳು-ವು-ದಲ್ಲ
ಕೇಳು-ವು-ದಿಲ್ಲ
ಕೇಳು-ವು-ದಿ-ಲ್ಲವೊ
ಕೇಳು-ವುದು
ಕೇಳು-ವುದೇ
ಕೇಳು-ವು-ದೇನು
ಕೇಳು-ವುದೊ
ಕೇಳು-ವುದೋ
ಕೇಳು-ವೆವು
ಕೇಳೋಣ
ಕೇವಲ
ಕೇವಲಂ
ಕೇವ-ಲೈ-ರಿಂ-ದ್ರಿ-ಯೈ-ರಪಿ
ಕೇಶವ
ಕೇಶವಃ
ಕೇಶ-ವಸ್ಯ
ಕೇಶವಾ
ಕೇಶ-ವಾ-ರ್ಜು-ನ-ಯೋಃ
ಕೇಶ-ವಾ-ರ್ಜು-ನರ
ಕೇಷು
ಕೇಸನ್ನು
ಕೇಸ-ರ-ಗಳಿಂದ
ಕೇಸ-ರ-ಗಳೇ
ಕೇಸ-ರಿ-ಯಂತೆ
ಕೇಸಿನ
ಕೈ
ಕೈಂಕರ್ಯ
ಕೈಂಕ-ರ್ಯ-ದಂತೆ
ಕೈಒ-ಡ್ಡು-ವುದೇ
ಕೈಕಟ್ಟಿ
ಕೈಕ-ಟ್ಟಿ-ಕೊಂಡು
ಕೈಕಾಲು
ಕೈಕಾ-ಲು-ಗಳ
ಕೈಕಾ-ಲು-ಗಳನ್ನು
ಕೈಕಾ-ಲು-ಗಳು
ಕೈಕಾ-ಲು-ಗ-ಳು-ಳ್ಳದ್ದು
ಕೈಕೇಯಿ
ಕೈಕೊಂ-ಡಿ-ದ್ದರೆ
ಕೈಕೊ-ಡು-ವುದು
ಕೈಕೊ-ಡು-ವುವು
ಕೈಗಳ
ಕೈಗಳನ್ನು
ಕೈಗ-ಳನ್ನೇ
ಕೈಗಳಲ್ಲಿ
ಕೈಗ-ಳ-ಲ್ಲಿಯೂ
ಕೈಗಳಿಂದ
ಕೈಗಳು
ಕೈಗಳೇ
ಕೈಗಾ
ಕೈಗೂ-ಡದೆ
ಕೈಗೂ-ಡ-ಬ-ಹುದು
ಕೈಗೂ-ಡಿ-ದರೆ
ಕೈಗೂ-ಡು-ವು-ದಿಲ್ಲ
ಕೈಗೆ
ಕೈಗೊಂಬೆ
ಕೈಗೊಂ-ಬೆ-ಯಾ-ಗು-ತ್ತೇವೆ
ಕೈಗೊ-ಳ್ಳು-ವನು
ಕೈಜೋ-ಡಿ-ಸಿ-ಕೊಂಡು
ಕೈತು-ತ್ತಿ-ನಂತೆ
ಕೈತೊ-ಳೆದು
ಕೈತೊ-ಳೆ-ದು-ಕೊಂ-ಡಿಲ್ಲ
ಕೈತೊ-ಳೆ-ದು-ಕೊ-ಳ್ಳ-ಬ-ಲ್ಲರು
ಕೈದೀ-ವಿಗೆ
ಕೈದು-ಗಳನ್ನು
ಕೈಬಿ-ಟ್ಟರೂ
ಕೈಬಿ-ಟ್ಟರೆ
ಕೈಬಿ-ಟ್ಟಿ-ರು-ವಾಗ
ಕೈಬಿಟ್ಟು
ಕೈಬಿ-ಡ-ದಿ-ರು-ವು-ದೊಂ-ದಿದೆ
ಕೈಬಿ-ಡದೆ
ಕೈಬಿ-ಡು-ವನು
ಕೈಬಿ-ಡು-ವನೆ
ಕೈಬಿ-ಡು-ವು-ದಿಲ್ಲ
ಕೈಬಿ-ಡು-ವುದು
ಕೈಬಿ-ಡು-ವುವು
ಕೈಬಿ-ಡು-ವೆವು
ಕೈಬೆ-ರಳ
ಕೈಮರ
ಕೈಮ-ರ-ಗಳೇ
ಕೈಮ-ರದ
ಕೈಮ-ರ-ದಂ-ತಿವೆ
ಕೈಮ-ರ-ದಂತೆ
ಕೈಮ-ರ-ವನ್ನು
ಕೈಮ-ರ-ವಾಗಿ
ಕೈಯ
ಕೈಯನ್ನು
ಕೈಯಲ್ಲಿ
ಕೈಯ-ಲ್ಲಿ-ಟ್ಟು-ಕೊಂಡು
ಕೈಯ-ಲ್ಲಿದೆ
ಕೈಯ-ಲ್ಲಿಯೂ
ಕೈಯ-ಲ್ಲಿಯೇ
ಕೈಯ-ಲ್ಲಿ-ರುವ
ಕೈಯ-ಲ್ಲಿ-ರು-ವಂತೆ
ಕೈಯ-ಲ್ಲಿ-ರು-ವುದು
ಕೈಯ-ಲ್ಲಿವೆ
ಕೈಯಲ್ಲೂ
ಕೈಯಿಂದ
ಕೈಯಿಂ-ದಲೂ
ಕೈಯಿಂ-ದಲೇ
ಕೈಯೆತ್ತಿ
ಕೈಯೊಂದು
ಕೈಯೊ-ಡ್ಡ-ಕೂ-ಡದು
ಕೈಯೊ-ಡ್ಡದೆ
ಕೈಯೊ-ಡ್ಡನು
ಕೈಯೊ-ಡ್ಡ-ಬಾ-ರದು
ಕೈಯೊ-ಡ್ಡ-ಬೇಕು
ಕೈಯೊ-ಡ್ಡ-ಲಿಲ್ಲ
ಕೈಯೊಡ್ಡಿ
ಕೈಯೊ-ಡ್ಡಿ-ದರೆ
ಕೈಯೊ-ಡ್ಡಿಲ್ಲ
ಕೈಯೊ-ಡ್ಡು-ತ್ತಾ-ನೆಯೊ
ಕೈಯೊ-ಡ್ಡು-ತ್ತಾ-ನೆಯೋ
ಕೈಯೊ-ಡ್ಡು-ತ್ತೇವೆ
ಕೈಯೊ-ಡ್ಡು-ವಂ-ತಿದೆ
ಕೈಯೊ-ಡ್ಡು-ವರು
ಕೈಯೊ-ಡ್ಡು-ವು-ದಿಲ್ಲ
ಕೈಯೊ-ಡ್ಡು-ವುದು
ಕೈಯ್ಯಲ್ಲಿ
ಕೈರ್ಮಯಾ
ಕೈರ್ಲಿಂ-ಗೈ-ಸ್ತ್ರೀನ್
ಕೈಲಾ-ಗ-ದ-ವನು
ಕೈಲಾ-ಗದೆ
ಕೈಲಾ-ದು-ದನ್ನು
ಕೈಲಾ-ದು-ದ-ನ್ನೆಲ್ಲಾ
ಕೈಲಾ-ಸ-ದಲ್ಲೋ
ಕೈಲಿ
ಕೈವ-ರ್ತಕಃ
ಕೈವಾ-ಡ-ವನ್ನು
ಕೈವಾ-ಡ-ವನ್ನೇ
ಕೈವಾ-ಡವೆ
ಕೈಸೇ-ರಿ-ದರೆ
ಕೈಹಾಕ
ಕೈಹಾ-ಕ-ಬೇ-ಕಾ-ಗಿಲ್ಲ
ಕೈಹಾ-ಕ-ಬೇ-ಕಾ-ಗು-ವುದು
ಕೈಹಾ-ಕ-ಬೇಕು
ಕೈಹಾ-ಕಿ-ದರೆ
ಕೈಹಾ-ಕಿ-ದಾ-ಗಲೇ
ಕೈಹಾ-ಕಿ-ದೆವೊ
ಕೈಹಾ-ಕು-ತ್ತಾನೆ
ಕೈಹಾ-ಕು-ತ್ತೇವೆ
ಕೈಹಾ-ಕು-ವರು
ಕೈಹಾ-ಕು-ವು-ದಕ್ಕೆ
ಕೈಹಾ-ಕು-ವು-ದಿ-ಲ್ಲ-ವೆಂದು
ಕೈಹಾ-ಕು-ವುದು
ಕೈಹಿ-ಡಿ-ದರೆ
ಕೈಹಿ-ಡಿ-ದ-ವನ
ಕೈಹಿ-ಡಿ-ದ-ವರು
ಕೈಹಿ-ಡಿ-ದಿ-ರು-ವನು
ಕೈಹಿ-ಡಿದು
ಕೈಹಿ-ಡಿಸಿ
ಕೊಂಕು
ಕೊಂಚ
ಕೊಂಡ
ಕೊಂಡರೆ
ಕೊಂಡ-ವನು
ಕೊಂಡ-ವನೂ
ಕೊಂಡ-ವ-ರಿ-ಗಿಂತ
ಕೊಂಡಾ-ಗಲೇ
ಕೊಂಡಾಡ
ಕೊಂಡಾ-ಡ-ಬ-ಹುದು
ಕೊಂಡಾ-ಡ-ಬೇಕು
ಕೊಂಡಾ-ಡಿ-ದರು
ಕೊಂಡಾ-ಡಿ-ದರೂ
ಕೊಂಡಾ-ಡಿ-ದರೆ
ಕೊಂಡಾ-ಡಿ-ದಾಗ
ಕೊಂಡಾ-ಡಿ-ರು-ವರು
ಕೊಂಡಾಡು
ಕೊಂಡಾ-ಡು-ತ್ತಾನೆ
ಕೊಂಡಾ-ಡು-ತ್ತಾರೆ
ಕೊಂಡಾ-ಡು-ತ್ತಿ-ದ್ದಾರೆ
ಕೊಂಡಾ-ಡು-ತ್ತಿ-ರು-ವರು
ಕೊಂಡಾ-ಡುವ
ಕೊಂಡಾ-ಡು-ವನು
ಕೊಂಡಾ-ಡು-ವರು
ಕೊಂಡಾ-ಡು-ವು-ದಕ್ಕೆ
ಕೊಂಡಾ-ಡು-ವು-ದಿ-ರಲಿ
ಕೊಂಡಾ-ಡು-ವುದು
ಕೊಂಡಿದೆ
ಕೊಂಡಿ-ದ್ದರೆ
ಕೊಂಡಿ-ದ್ದೇವೆ
ಕೊಂಡಿ-ಯಂತೆ
ಕೊಂಡಿ-ರ-ಬ-ಹುದು
ಕೊಂಡಿ-ರ-ಬೇಕು
ಕೊಂಡಿರು
ಕೊಂಡಿ-ರು-ತ್ತಾರೆ
ಕೊಂಡಿ-ರುವ
ಕೊಂಡಿ-ರು-ವನು
ಕೊಂಡಿ-ರು-ವರು
ಕೊಂಡಿ-ರು-ವರೋ
ಕೊಂಡಿ-ರು-ವ-ವನು
ಕೊಂಡಿ-ರು-ವು-ದ-ರಿಂದ
ಕೊಂಡಿ-ರು-ವು-ದಿಲ್ಲ
ಕೊಂಡಿ-ರು-ವುದು
ಕೊಂಡಿ-ರು-ವುದೋ
ಕೊಂಡಿ-ರು-ವೆನೊ
ಕೊಂಡಿ-ರು-ವೆವೊ
ಕೊಂಡಿಲ್ಲ
ಕೊಂಡು
ಕೊಂಡು-ಕೊ-ಳ್ಳ-ಬೇಕು
ಕೊಂಡು-ಕೊ-ಳ್ಳಲಿ
ಕೊಂಡು-ಕೊ-ಳ್ಳಲು
ಕೊಂಡು-ಕೊ-ಳ್ಳು-ವು-ದಕ್ಕೆ
ಕೊಂಡು-ಹೋಗಿ
ಕೊಂಡು-ಹೋಗು
ಕೊಂಡು-ಹೋ-ದರು
ಕೊಂಡೊ
ಕೊಂಡೊ-ಯ್ದರೆ
ಕೊಂಡೊ-ಯ್ಯಲು
ಕೊಂಡೊ-ಯ್ಯು-ವ-ವನೂ
ಕೊಂಡೊ-ಯ್ಯು-ವುದು
ಕೊಂಡೊ-ಯ್ಯು-ವುವು
ಕೊಂದ
ಕೊಂದದ್ದು
ಕೊಂದನು
ಕೊಂದರು
ಕೊಂದರೂ
ಕೊಂದರೆ
ಕೊಂದ-ವನು
ಕೊಂದ-ವನೂ
ಕೊಂದಾಗ
ಕೊಂದಾದ
ಕೊಂದಾ-ದ-ಮೇಲೆ
ಕೊಂದಿ-ದ್ದಕ್ಕೆ
ಕೊಂದಿ-ದ್ದಾನೆ
ಕೊಂದು
ಕೊಂದು-ಬಿ-ಡುವಾ
ಕೊಂದು-ಬಿ-ಡೋಣ
ಕೊಂದು-ಹಾ-ಕಿತು
ಕೊಂದೆ
ಕೊಂಬು
ಕೊಂಬು-ಗ-ಳಿವೆ
ಕೊಂಬು-ಗಳು
ಕೊಂಬೆ
ಕೊಂಬೆ-ಗ-ಳಂತೆ
ಕೊಂಬೆ-ಗಳನ್ನು
ಕೊಂಬೆ-ಗ-ಳಾದ
ಕೊಂಬೆ-ಗಳಿಂದ
ಕೊಂಬೆ-ಗಳು
ಕೊಂಬೆ-ಗ-ಳುಳ್ಳ
ಕೊಂಬೆ-ಗಳೂ
ಕೊಂಬೆ-ಯನ್ನು
ಕೊಕ್ಕಿ-ನಿಂದ
ಕೊಚ್ಚಿ
ಕೊಚ್ಚಿ-ಕೊಂ-ಡರೂ
ಕೊಚ್ಚಿ-ಕೊಂಡು
ಕೊಚ್ಚಿ-ಕೊ-ಳ್ಳು-ತ್ತಾನೆ
ಕೊಚ್ಚಿ-ಕೊ-ಳ್ಳು-ತ್ತಿದ್ದ
ಕೊಚ್ಚಿ-ಕೊ-ಳ್ಳು-ತ್ತಿ-ದ್ದ-ವರೇ
ಕೊಚ್ಚಿ-ಕೊ-ಳ್ಳು-ವಂ-ತ-ಹುದು
ಕೊಚ್ಚಿ-ಕೊ-ಳ್ಳು-ವ-ವರ
ಕೊಚ್ಚಿ-ಕೊ-ಳ್ಳು-ವು-ದಕ್ಕೆ
ಕೊಚ್ಚಿ-ಕೊ-ಳ್ಳು-ವು-ದ-ನ್ನೆಲ್ಲ
ಕೊಚ್ಚಿ-ಕೊ-ಳ್ಳು-ವುದು
ಕೊಚ್ಚಿ-ಕೊ-ಳ್ಳು-ವೆವು
ಕೊಚ್ಚಿ-ಹೋ-ಗು-ವರು
ಕೊಟ್ಚಿ-ರುವ
ಕೊಟ್ಟ
ಕೊಟ್ಟಂ-ತಾ-ಗು-ವುದು
ಕೊಟ್ಟಂತೆ
ಕೊಟ್ಟ-ದ್ದನ್ನು
ಕೊಟ್ಟ-ದ್ದ-ನ್ನೆಲ್ಲಾ
ಕೊಟ್ಟ-ದ್ದ-ರಿಂದ
ಕೊಟ್ಟದ್ದು
ಕೊಟ್ಟ-ನಲ್ಲ
ಕೊಟ್ಟನು
ಕೊಟ್ಟನೋ
ಕೊಟ್ಟ-ರಾ-ದರೂ
ಕೊಟ್ಟರು
ಕೊಟ್ಟ-ರುವ
ಕೊಟ್ಟರೂ
ಕೊಟ್ಟರೆ
ಕೊಟ್ಟರೊ
ಕೊಟ್ಟ-ಲ್ಲದೆ
ಕೊಟ್ಟಳು
ಕೊಟ್ಟ-ವ-ನಲ್ಲ
ಕೊಟ್ಟ-ವನು
ಕೊಟ್ಟ-ವನೆ
ಕೊಟ್ಟ-ವನೇ
ಕೊಟ್ಟ-ವರು
ಕೊಟ್ಟಾಗ
ಕೊಟ್ಟಾ-ಗಲೇ
ಕೊಟ್ಟಾದ
ಕೊಟ್ಟಾ-ದ-ಮೇಲೂ
ಕೊಟ್ಟಾ-ದರೂ
ಕೊಟ್ಟಿದೆ
ಕೊಟ್ಟಿ-ದ್ದ-ಕ್ಕಿಂತ
ಕೊಟ್ಟಿ-ದ್ದನ್ನು
ಕೊಟ್ಟಿ-ದ್ದ-ರಲ್ಲಿ
ಕೊಟ್ಟಿ-ದ್ದರೂ
ಕೊಟ್ಟಿ-ದ್ದರೆ
ಕೊಟ್ಟಿ-ದ್ದಾನೆ
ಕೊಟ್ಟಿ-ದ್ದಾರೆ
ಕೊಟ್ಟಿ-ದ್ದಾ-ವುದೂ
ಕೊಟ್ಟಿದ್ದು
ಕೊಟ್ಟಿ-ದ್ದೆಲ್ಲಾ
ಕೊಟ್ಟಿದ್ದೇ
ಕೊಟ್ಟಿ-ರ-ಬ-ಹುದು
ಕೊಟ್ಟಿ-ರ-ಬೇಕು
ಕೊಟ್ಟಿ-ರಲಿ
ಕೊಟ್ಟಿ-ರು-ತ್ತಾನೆ
ಕೊಟ್ಟಿ-ರು-ತ್ತೇವೆ
ಕೊಟ್ಟಿ-ರುವ
ಕೊಟ್ಟಿ-ರು-ವನು
ಕೊಟ್ಟಿ-ರು-ವನೋ
ಕೊಟ್ಟಿ-ರು-ವರು
ಕೊಟ್ಟಿ-ರು-ವುದನ್ನು
ಕೊಟ್ಟಿ-ರು-ವುದು
ಕೊಟ್ಟಿ-ರು-ವೆವೆ
ಕೊಟ್ಟಿ-ರು-ವೆವೋ
ಕೊಟ್ಟು
ಕೊಟ್ಟು-ದನ್ನು
ಕೊಟ್ಟು-ದು-ದ-ನ್ನೆಲ್ಲಾ
ಕೊಟ್ಟು-ದು-ದಲ್ಲ
ಕೊಟ್ಟು-ಬಿ-ಡೋಣ
ಕೊಟ್ಟೆ
ಕೊಟ್ಟೆ-ವೆಂದರೆ
ಕೊಟ್ಟೇ
ಕೊಟ್ವ-ವನು
ಕೊಡ
ಕೊಡ-ಕೂ-ಡದು
ಕೊಡಕ್ಕೆ
ಕೊಡ-ದಿ-ದ್ದ-ರೂ-ಇ-ರುವ
ಕೊಡದೆ
ಕೊಡದೇ
ಕೊಡನು
ಕೊಡ-ಬ-ಯ-ಸು-ವು-ದಿಲ್ಲ
ಕೊಡ-ಬಲ್ಲ
ಕೊಡ-ಬ-ಲ್ಲದು
ಕೊಡ-ಬ-ಲ್ಲುದು
ಕೊಡ-ಬ-ಹು-ದಾ-ಗಿ-ತ್ತಲ್ಲ
ಕೊಡ-ಬ-ಹುದು
ಕೊಡ-ಬ-ಹುದೆ
ಕೊಡ-ಬಾ-ರದು
ಕೊಡ-ಬೇ-ಕಲ್ಲ
ಕೊಡ-ಬೇ-ಕಲ್ಲಾ
ಕೊಡ-ಬೇ-ಕಾ-ಗಿದೆ
ಕೊಡ-ಬೇ-ಕಾ-ಗಿರು
ಕೊಡ-ಬೇ-ಕಾ-ಗಿ-ರು-ವುದನ್ನು
ಕೊಡ-ಬೇ-ಕಾ-ಗಿ-ರು-ವುದು
ಕೊಡ-ಬೇ-ಕಾ-ಗಿಲ್ಲ
ಕೊಡ-ಬೇ-ಕಾ-ಗು-ವುದು
ಕೊಡ-ಬೇ-ಕಾದ
ಕೊಡ-ಬೇ-ಕಾ-ದರೆ
ಕೊಡ-ಬೇ-ಕಾ-ದಾಗ
ಕೊಡ-ಬೇಕು
ಕೊಡ-ಬೇ-ಕೆಂ-ದಿ-ದ್ದನು
ಕೊಡ-ಬೇ-ಕೆಂ-ದಿ-ದ್ದರೂ
ಕೊಡ-ಬೇ-ಕೆಂಬ
ಕೊಡ-ಲಾ-ರ-ದಿ-ರು-ವುದನ್ನು
ಕೊಡ-ಲಾ-ರದು
ಕೊಡ-ಲಾ-ರವು
ಕೊಡಲಿ
ಕೊಡ-ಲಿ-ಯನ್ನು
ಕೊಡ-ಲಿಯೇ
ಕೊಡ-ಲಿಲ್ಲ
ಕೊಡ-ಲಿ-ಲ್ಲವೆ
ಕೊಡಲು
ಕೊಡಲೂ
ಕೊಡ-ಲೆ-ತ್ನಿ-ಸು-ವು-ದಿಲ್ಲ
ಕೊಡಲೇ
ಕೊಡ-ಲ್ಪಡು
ಕೊಡ-ಲ್ಪ-ಡು-ವುದೊ
ಕೊಡ-ಲ್ಪ-ಡು-ವುದೋ
ಕೊಡ-ವನ್ನು
ಕೊಡ-ವಿ-ಕೊಳ್ಳ
ಕೊಡ-ವಿ-ದರೆ
ಕೊಡ-ವಿ-ಬಿ-ಡು-ವು-ದಕ್ಕೆ
ಕೊಡ-ಹು-ವು-ದಾ-ಗಿದೆ
ಕೊಡಿ
ಕೊಡಿ-ಸಿದ
ಕೊಡಿ-ಸಿ-ರು-ವನು
ಕೊಡು
ಕೊಡುಗೆ
ಕೊಡು-ಗೆ-ಯನ್ನು
ಕೊಡುತ್ತ
ಕೊಡು-ತ್ತದೆ
ಕೊಡು-ತ್ತಲೇ
ಕೊಡು-ತ್ತವೆ
ಕೊಡು-ತ್ತಾನೆ
ಕೊಡು-ತ್ತಾ-ನೆ-ಅದೇ
ಕೊಡು-ತ್ತಾ-ನೆಯೊ
ಕೊಡು-ತ್ತಾ-ನೆಯೋ
ಕೊಡು-ತ್ತಾ-ನೆ-ಸ-ಮು-ದ್ರದ
ಕೊಡು-ತ್ತಾ-ರಲ್ಲ
ಕೊಡು-ತ್ತಾರೆ
ಕೊಡು-ತ್ತಾರೋ
ಕೊಡು-ತ್ತಿದೆ
ಕೊಡು-ತ್ತಿದ್ದ
ಕೊಡು-ತ್ತಿ-ದ್ದರು
ಕೊಡು-ತ್ತಿ-ದ್ದರೆ
ಕೊಡು-ತ್ತಿ-ದ್ದೆನೊ
ಕೊಡು-ತ್ತಿ-ರ-ಬ-ಹುದು
ಕೊಡು-ತ್ತಿ-ರ-ಬೇ-ಕಾ-ಗು-ವುದು
ಕೊಡು-ತ್ತಿ-ರುವ
ಕೊಡು-ತ್ತಿ-ರು-ವನು
ಕೊಡು-ತ್ತಿ-ರು-ವರು
ಕೊಡು-ತ್ತಿ-ರು-ವುದು
ಕೊಡು-ತ್ತಿ-ರುವೆ
ಕೊಡು-ತ್ತಿಲ್ಲ
ಕೊಡು-ತ್ತಿವೆ
ಕೊಡು-ತ್ತೀರಿ
ಕೊಡು-ತ್ತೇನೆ
ಕೊಡು-ತ್ತೇನೋ
ಕೊಡು-ತ್ತೇವೆ
ಕೊಡು-ತ್ತೇ-ವೆಯೆ
ಕೊಡು-ತ್ತೇ-ವೆಯೊ
ಕೊಡು-ತ್ತೇ-ವೆಯೋ
ಕೊಡುವ
ಕೊಡು-ವಂ-ತಿದೆ
ಕೊಡು-ವಂತೆ
ಕೊಡು-ವನು
ಕೊಡು-ವನೊ
ಕೊಡು-ವರು
ಕೊಡು-ವರೊ
ಕೊಡು-ವರೋ
ಕೊಡು-ವಳು
ಕೊಡು-ವ-ವ-ನಲ್ಲ
ಕೊಡು-ವ-ವನಿ
ಕೊಡು-ವ-ವ-ನಿಗೂ
ಕೊಡು-ವ-ವ-ನಿಗೆ
ಕೊಡು-ವ-ವನು
ಕೊಡು-ವ-ವನೂ
ಕೊಡು-ವ-ವರ
ಕೊಡು-ವ-ವ-ರಿ-ಗೆಲ್ಲ
ಕೊಡು-ವ-ವರು
ಕೊಡು-ವ-ವರೆಲ್ಲ
ಕೊಡು-ವ-ವರೇ
ಕೊಡು-ವಷ್ಟು
ಕೊಡು-ವಾಗ
ಕೊಡುವು
ಕೊಡು-ವುದ
ಕೊಡು-ವು-ದ-ಕ್ಕಾ-ಗಲಿ
ಕೊಡು-ವು-ದ-ಕ್ಕಾಗಿ
ಕೊಡು-ವು-ದ-ಕ್ಕಾ-ಗಿಯೆ
ಕೊಡು-ವು-ದಕ್ಕೆ
ಕೊಡು-ವು-ದಕ್ಕೊ
ಕೊಡು-ವುದನ್ನು
ಕೊಡು-ವು-ದ-ನ್ನೆಲ್ಲಾ
ಕೊಡು-ವು-ದರ
ಕೊಡು-ವು-ದ-ರಲ್ಲಿ
ಕೊಡು-ವು-ದ-ರ-ಲ್ಲಿದೆ
ಕೊಡು-ವು-ದ-ರಲ್ಲೆ
ಕೊಡು-ವು-ದ-ರಿಂದ
ಕೊಡು-ವು-ದಲ್ಲ
ಕೊಡು-ವು-ದಾ-ದರೂ
ಕೊಡು-ವು-ದಾ-ವುದೂ
ಕೊಡು-ವು-ದಿಲ್ಲ
ಕೊಡು-ವು-ದಿ-ಲ್ಲ-ವೆಂದರೆ
ಕೊಡು-ವು-ದಿ-ಲ್ಲವೊ
ಕೊಡು-ವು-ದಿ-ಲ್ಲವೋ
ಕೊಡು-ವುದು
ಕೊಡು-ವು-ದೆಲ್ಲ
ಕೊಡು-ವು-ದೆಲ್ಲಾ
ಕೊಡು-ವುದೇ
ಕೊಡು-ವುದೊ
ಕೊಡು-ವು-ದೊಂದು
ಕೊಡು-ವುವು
ಕೊಡು-ವೆನೊ
ಕೊಡು-ವೆವು
ಕೊಡು-ವೆವೋ
ಕೊಡೋಣ
ಕೊನೆ
ಕೊನೆ-ಗ-ಳಿ-ಗೆ-ಯಲ್ಲಿ
ಕೊನೆ-ಗ-ಳೆಷ್ಟೋ
ಕೊನೆ-ಗಾ-ಣ-ಲಿಲ್ಲ
ಕೊನೆ-ಗಾ-ಣಿ-ಸು-ವುದು
ಕೊನೆ-ಗಾ-ಣು-ತ್ತಿದೆ
ಕೊನೆ-ಗಾ-ಣುವ
ಕೊನೆ-ಗಾ-ಣು-ವು-ದಿಲ್ಲ
ಕೊನೆ-ಗಾ-ಣು-ವುದು
ಕೊನೆ-ಗಾ-ಣು-ವುವು
ಕೊನೆ-ಗಾ-ದರೂ
ಕೊನೆ-ಗಾ-ಲ-ದಲ್ಲಿ
ಕೊನೆ-ಗಾ-ಲ-ದ-ಲ್ಲಿಯೂ
ಕೊನೆಗೂ
ಕೊನೆಗೆ
ಕೊನೆ-ಗೊಂ-ಡುದು
ಕೊನೆ-ಗೊಳ್ಳ
ಕೊನೆ-ಗೊಳ್ಳು
ಕೊನೆ-ಗೊ-ಳ್ಳು-ತ್ತದೆ
ಕೊನೆ-ಗೊ-ಳ್ಳು-ವು-ದಿಲ್ಲ
ಕೊನೆ-ಗೊ-ಳ್ಳು-ವುದು
ಕೊನೆ-ಗೊ-ಳ್ಳು-ವುದೊ
ಕೊನೆ-ಗೊ-ಳ್ಳು-ವುವು
ಕೊನೆಯ
ಕೊನೆ-ಯ-ತ-ನಕ
ಕೊನೆ-ಯನ್ನು
ಕೊನೆ-ಯಲ್ಲಿ
ಕೊನೆ-ಯ-ಲ್ಲಿಯೂ
ಕೊನೆ-ಯ-ಲ್ಲಿ-ರುವ
ಕೊನೆ-ಯಿಲ್ಲ
ಕೊನೆ-ಯೆಂದು
ಕೊನೆಯೇ
ಕೊಪ್ಪ-ರಿ-ಗೆಗೆ
ಕೊಪ್ಪ-ರಿ-ಗೆ-ಯಂತೆ
ಕೊಬ್ಬರಿ
ಕೊಬ್ಬಾಗಿ
ಕೊಬ್ಬಿ
ಕೊಬ್ಬಿ-ದಷ್ಟೂ
ಕೊಬ್ಬು
ಕೊಬ್ಬು-ತ್ತಿ-ರು-ವನು
ಕೊಬ್ಬು-ವುದು
ಕೊಯ್ದಿದೆ
ಕೊಯ್ಯ-ಬೇಕು
ಕೊಯ್ಯಲು
ಕೊಯ್ಯುವ
ಕೊಯ್ಯು-ವುದು
ಕೊಯ್ಲಿಗೆ
ಕೊರ-ಕಲು
ಕೊರ-ಕ-ಲು-ಗಳು
ಕೊರ-ಗನು
ಕೊರ-ಗಿದ
ಕೊರ-ಗಿ-ಹೋದ
ಕೊರ-ಗು-ತ್ತಾರೆ
ಕೊರ-ಗು-ತ್ತಿದ್ದ
ಕೊರ-ಗು-ತ್ತಿ-ದ್ದರೆ
ಕೊರ-ಗು-ತ್ತಿ-ರು-ವನು
ಕೊರ-ಗು-ವ-ವನೂ
ಕೊರ-ಗು-ವುದು
ಕೊರ-ಡೊಂದು
ಕೊರತೆ
ಕೊರ-ತೆ-ಗಳು
ಕೊರ-ತೆ-ಯನ್ನು
ಕೊರ-ತೆಯೂ
ಕೊರ-ಳ-ಮೇಲೆ
ಕೊರ-ಳಲ್ಲಿ
ಕೊರ-ಳಿಗೆ
ಕೊರ-ಳಿ-ನಲ್ಲಿ
ಕೊರ-ಳಿ-ನ-ಲ್ಲಿ-ರುವ
ಕೊರಳು
ಕೊರೆ-ದು-ಕೊಂಡು
ಕೊರೆ-ಯು-ತ್ತಾನೆ
ಕೊರೆ-ಯು-ತ್ತಿ-ರು-ವುದು
ಕೊರೆ-ಯು-ವುದು
ಕೊಲೆ
ಕೊಲೆ-ಗಿಂತ
ಕೊಲೆಗೆ
ಕೊಲೆ-ಪಾ-ತಕ
ಕೊಲೆ-ಮಾ-ಡ-ಬೇಕು
ಕೊಲೆಯ
ಕೊಲೆ-ಯನ್ನು
ಕೊಲೆ-ಯಾ-ಗು-ವುದು
ಕೊಲೆಯೂ
ಕೊಲೆಯೋ
ಕೊಲ್ಲ
ಕೊಲ್ಲ-ದಿ-ರಲಿ
ಕೊಲ್ಲದೆ
ಕೊಲ್ಲ-ಬಲ್ಲ
ಕೊಲ್ಲ-ಬ-ಹು-ದಷ್ಟೆ
ಕೊಲ್ಲ-ಬೇ-ಕಾ-ಗಿದೆ
ಕೊಲ್ಲ-ಬೇ-ಕಾ-ಗು-ವು-ದಲ್ಲ
ಕೊಲ್ಲ-ಬೇ-ಕಾ-ಗು-ವುದು
ಕೊಲ್ಲ-ಬೇಕು
ಕೊಲ್ಲ-ಲಾ-ರದು
ಕೊಲ್ಲ-ಲಾ-ರರು
ಕೊಲ್ಲಲಿ
ಕೊಲ್ಲ-ಲಿಲ್ಲ
ಕೊಲ್ಲಲು
ಕೊಲ್ಲ-ಲ್ಪ-ಟ್ಟನು
ಕೊಲ್ಲ-ಲ್ಪ-ಟ್ಟಿ-ದ್ದಾರೆ
ಕೊಲ್ಲ-ಲ್ಪ-ಡ-ತ-ಕ್ಕ-ವನು
ಕೊಲ್ಲ-ಲ್ಪ-ಡು-ವುದು
ಕೊಲ್ಲಿ
ಕೊಲ್ಲಿ-ಸಿ-ಕೊಂಡೆ
ಕೊಲ್ಲಿ-ಸಿ-ಕೊಳ್ಳು
ಕೊಲ್ಲಿ-ಸಿ-ಕೊ-ಳ್ಳು-ವುದೂ
ಕೊಲ್ಲಿ-ಸು-ತ್ತಾನೆ
ಕೊಲ್ಲಿ-ಸು-ವು-ದಕ್ಕೆ
ಕೊಲ್ಲಿ-ಸು-ವು-ದಿಲ್ಲ
ಕೊಲ್ಲಿ-ಸು-ವುದೂ
ಕೊಲ್ಲು
ಕೊಲ್ಲು-ತ್ತಾನೆ
ಕೊಲ್ಲು-ತ್ತಾರೋ
ಕೊಲ್ಲು-ತ್ತಿ-ರು-ವ-ವನು
ಕೊಲ್ಲು-ತ್ತಿ-ರುವೆ
ಕೊಲ್ಲು-ತ್ತಿ-ರು-ವೆನು
ಕೊಲ್ಲು-ತ್ತೇನೆ
ಕೊಲ್ಲು-ತ್ತೇ-ವೆಯೇ
ಕೊಲ್ಲು-ವಂ-ತಹ
ಕೊಲ್ಲು-ವಂತೆ
ಕೊಲ್ಲು-ವನು
ಕೊಲ್ಲು-ವ-ವ-ನಲ್ಲ
ಕೊಲ್ಲು-ವ-ವ-ನೆಂದು
ಕೊಲ್ಲು-ವ-ವ-ರಿಲ್ಲ
ಕೊಲ್ಲು-ವಾಗ
ಕೊಲ್ಲು-ವು-ದ-ಕ್ಕಿಂತ
ಕೊಲ್ಲು-ವು-ದಕ್ಕೆ
ಕೊಲ್ಲು-ವುದನ್ನು
ಕೊಲ್ಲು-ವು-ದಿಲ್ಲ
ಕೊಲ್ಲು-ವುದು
ಕೊಲ್ಲು-ವುದೂ
ಕೊಲ್ಲು-ವೆನು
ಕೊಳಕು
ಕೊಳದ
ಕೊಳ-ದಲ್ಲಿ
ಕೊಳ-ಲಿ-ಗಲ್ಲ
ಕೊಳಲು
ಕೊಳ-ಲೂ-ದು-ವ-ವನು
ಕೊಳ-ವಿ-ಯನ್ನು
ಕೊಳ-ವಿ-ಯಲ್ಲಿ
ಕೊಳ-ವೆ-ಗಳಿಂದ
ಕೊಳೆ
ಕೊಳೆತ
ಕೊಳೆ-ತಿ-ರುವ
ಕೊಳೆತು
ಕೊಳೆ-ಬ-ಟ್ಟೆ-ಯನ್ನು
ಕೊಳೆ-ಮ-ನಸ್ಸು
ಕೊಳೆಯ
ಕೊಳೆ-ಯನ್ನು
ಕೊಳೆ-ಯ-ನ್ನೆಲ್ಲ
ಕೊಳೆ-ಯ-ನ್ನೆಲ್ಲಾ
ಕೊಳೆ-ಯಲು
ಕೊಳೆ-ಯಾ-ಗಿ-ರು-ವು-ದಿಲ್ಲ
ಕೊಳೆ-ಯಾ-ಗು-ವುದು
ಕೊಳೆ-ಯಿಂದ
ಕೊಳೆ-ಯಿಲ್ಲ
ಕೊಳ್ಳ-ಬ-ಹುದು
ಕೊಳ್ಳ-ಬೇ-ಕಾ-ದರೆ
ಕೊಳ್ಳ-ಬೇಕು
ಕೊಳ್ಳ-ಬೇ-ಕೆಂಬ
ಕೊಳ್ಳ-ಲಾ-ರೆವು
ಕೊಳ್ಳಲಿ
ಕೊಳ್ಳಲು
ಕೊಳ್ಳಿ-ಗಳು
ಕೊಳ್ಳಿ-ಯಾ-ಗು-ವಂತೆ
ಕೊಳ್ಳು-ತ್ತಾನೆ
ಕೊಳ್ಳು-ತ್ತಾನೋ
ಕೊಳ್ಳು-ತ್ತಿ-ದ್ದರೆ
ಕೊಳ್ಳು-ತ್ತಿ-ರು-ವನು
ಕೊಳ್ಳು-ತ್ತೇವೆ
ಕೊಳ್ಳುವ
ಕೊಳ್ಳು-ವಂತೆ
ಕೊಳ್ಳು-ವನು
ಕೊಳ್ಳು-ವರು
ಕೊಳ್ಳು-ವಾಗ
ಕೊಳ್ಳು-ವು-ದ-ಕ್ಕಾಗಿ
ಕೊಳ್ಳು-ವು-ದಕ್ಕೆ
ಕೊಳ್ಳು-ವುದನ್ನು
ಕೊಳ್ಳು-ವು-ದಷ್ಟೆ
ಕೊಳ್ಳು-ವು-ದಿಲ್ಲ
ಕೊಳ್ಳು-ವುದು
ಕೊಳ್ಳು-ವುದೇ
ಕೊಳ್ಳು-ವುದೋ
ಕೊಳ್ಳು-ವುವು
ಕೊಸ-ರಾ-ಡು-ವು-ದಿಲ್ಲ
ಕೊಸ-ರು-ವು-ದಿಲ್ಲ
ಕೋ
ಕೋಗಿ-ಲೆ-ಯಂತೆ
ಕೋಟನ್ನು
ಕೋಟಲೆ
ಕೋಟ-ಲೆ-ಗಳಿಂದ
ಕೋಟ-ಲೆಗೆ
ಕೋಟ-ಲೆ-ಯಲ್ಲಿ
ಕೋಟ-ಲೆ-ಯಿಂದ
ಕೋಟಿ
ಕೋಟಿಗೆ
ಕೋಟಿನ
ಕೋಟಿ-ನಲ್ಲಿ
ಕೋಟಿಯ
ಕೋಟಿ-ಯಲ್ಲಿ
ಕೋಟಿ-ಸೂರ್ಯ
ಕೋಟು
ಕೋಟ್ಯಂ-ತರ
ಕೋಟ್ಯಾ-ಧೀ-ಶ್ವ-ರನೂ
ಕೋಡನ್ನು
ಕೋಡಿಗೆ
ಕೋಡಿ-ನಂತೆ
ಕೋಡಿ-ಯಲ್ಲಿ
ಕೋಣ
ಕೋಣ-ಗ-ಳಿಂ-ದಲೂ
ಕೋಣ-ವೆಂದು
ಕೋಣೆ
ಕೋಣೆ-ಗ-ಳಲ್ಲೂ
ಕೋಣೆ-ಗ-ಳಿ-ರುವ
ಕೋಣೆಗೆ
ಕೋಣೆ-ಯಲ್ಲಿ
ಕೋಣೆ-ಯಲ್ಲೆ
ಕೋಣೆ-ಯೊ-ಳಗೆ
ಕೋತಿ
ಕೋತಿಗೆ
ಕೋತಿ-ಚೇಷ್ಟೆ
ಕೋಪ
ಕೋಪಕ್ಕೆ
ಕೋಪ-ಗೊಂ-ಡರೆ
ಕೋಪ-ಗೊ-ಳ್ಳ-ಲಿಲ್ಲ
ಕೋಪ-ಗೊ-ಳ್ಳು-ವನೋ
ಕೋಪ-ಗೊ-ಳ್ಳು-ವು-ದಿಲ್ಲ
ಕೋಪ-ಗೊ-ಳ್ಳು-ವುದು
ಕೋಪದ
ಕೋಪ-ದಷ್ಟು
ಕೋಪ-ದಿಂದ
ಕೋಪ-ಬ-ರು-ವು-ದಿಲ್ಲ
ಕೋಪ-ವನ್ನು
ಕೋಪ-ವಿ-ರ-ಬಾ-ರದು
ಕೋಪ-ವಿಲ್ಲ
ಕೋಪವೂ
ಕೋಪವೇ
ಕೋಪಾ-ನ-ಲ-ನಂತೆ
ಕೋಪಿ-ಸಿ-ಕೊ-ಳ್ಳು-ವು-ದಿಲ್ಲ
ಕೋಮಲ
ಕೋಮಿಗೂ
ಕೋಮಿನ
ಕೋಮಿ-ನ-ವ-ರಿಗೆ
ಕೋರಿ-ಕೆ-ಗಳನ್ನು
ಕೋರಿ-ಕೆ-ಯಂತೆ
ಕೋರಿ-ಕೊಂಡ
ಕೋರಿ-ಕೊ-ಳ್ಳು-ವುದನ್ನು
ಕೋರುತ್ತ
ಕೋರು-ವನು
ಕೋರೆ
ಕೋರೆ-ದಾ-ಡೆ-ಗಳಿಂದ
ಕೋರೈಸಿ
ಕೋರೈ-ಸು-ತ್ತಿ-ರುವ
ಕೋರೈ-ಸುವ
ಕೋರ್ಟಿ-ನಲ್ಲಿ
ಕೋರ್ಟು
ಕೋಲನ್ನು
ಕೋಲಿನ
ಕೋಲಿ-ನಿಂದ
ಕೋಲು
ಕೋಲ್ಡ್
ಕೋಶ-ಗಳನ್ನು
ಕೋಶಾ-ಧಿ-ಕಾರಿ
ಕೋಽತ್ರ
ಕೋಽನ್ಯೋಽಸ್ತಿ
ಕೌಂತೇಯ
ಕೌಂತೇಯಃ
ಕೌಮಾರಂ
ಕೌರವ
ಕೌರ-ವನ
ಕೌರ-ವ-ನಿಗೆ
ಕೌರ-ವನೇ
ಕೌರ-ವರ
ಕೌರ-ವ-ರಂತೂ
ಕೌರ-ವ-ರನ್ನು
ಕೌರ-ವ-ರಿಂದ
ಕೌರ-ವ-ರಿಗೆ
ಕೌರ-ವರು
ಕೌರ-ವ-ರೇನೊ
ಕೌರ-ವ-ರೊ-ಡನೆ
ಕೌಶಲಂ
ಕೌಶ-ಲಮ್
ಕೌಶ-ಲ-ವನ್ನು
ಕೌಶ-ಲವೇ
ಕೌಶ-ಲ್ಯವೇ
ಕ್ಕಾಗಲಿ
ಕ್ಕಾಗಿ
ಕ್ಕಾಗು-ವು-ದಿಲ್ಲ
ಕ್ಕಾದರೂ
ಕ್ಕಿಂತ
ಕ್ಯಾಂಡಲ್
ಕ್ಯಾನ್ಸರ್
ಕ್ಯಾಮರಾ
ಕ್ಯಾಮ-ರಾ-ದಂತೆ
ಕ್ಯೂ
ಕ್ರಂದನ
ಕ್ರತು
ಕ್ರತು-ರಹಂ
ಕ್ರಮ-ಗ-ಳೆಲ್ಲ
ಕ್ರಮ-ವಾಗಿ
ಕ್ರಮೇಣ
ಕ್ರಾಸಿ-ನಂತೆ
ಕ್ರಿಕೆಟ್
ಕ್ರಿಮಿ
ಕ್ರಿಮಿ-ಕೀಟ
ಕ್ರಿಮಿ-ಕೀ-ಟ-ಗ-ಳಿವೆ
ಕ್ರಿಮಿ-ಕೀ-ಟ-ಗಳೂ
ಕ್ರಿಮಿ-ಕೀ-ಟ-ದಿಂದ
ಕ್ರಿಮಿ-ಗಳನ್ನೆಲ್ಲಾ
ಕ್ರಿಮಿ-ಗ-ಳಿಗೂ
ಕ್ರಿಮಿ-ಗಳು
ಕ್ರಿಮಿ-ಗ-ಳೆಲ್ಲ
ಕ್ರಿಮಿ-ಗ-ಳೆಲ್ಲಾ
ಕ್ರಿಮಿ-ಗ-ಳೊ-ಡನೆ
ಕ್ರಿಯಂತೇ
ಕ್ರಿಯತೇ
ಕ್ರಿಯ-ತೇ-ಽಜುನ
ಕ್ರಿಯ-ಮಾ-ಣಾನಿ
ಕ್ರಿಯಾ-ಭಿರ್ನ
ಕ್ರಿಯಾ-ವಿ-ಶೇ-ಷ-ಬ-ಹು-ಲಾಂ
ಕ್ರಿಯೆ
ಕ್ರಿಯೆ-ಗಳ
ಕ್ರಿಯೆ-ಗಳನ್ನು
ಕ್ರಿಯೆ-ಗಳನ್ನೂ
ಕ್ರಿಯೆ-ಗಳನ್ನೆಲ್ಲ
ಕ್ರಿಯೆ-ಗಳಲ್ಲಿ
ಕ್ರಿಯೆ-ಗ-ಳಾ-ಗು-ತ್ತಿ-ರು-ವುವು
ಕ್ರಿಯೆ-ಗ-ಳಿಗೆ
ಕ್ರಿಯೆ-ಗಳು
ಕ್ರಿಯೆ-ಗಳೂ
ಕ್ರಿಯೆ-ಗ-ಳೆಲ್ಲ
ಕ್ರಿಯೆಗೆ
ಕ್ರಿಯೆಯ
ಕ್ರಿಯೆ-ಯನ್ನು
ಕ್ರಿಯೆ-ಯನ್ನೂ
ಕ್ರಿಯೆ-ಯಲ್ಲ
ಕ್ರಿಯೆ-ಯಲ್ಲಿ
ಕ್ರಿಯೆ-ಯಾ-ಯಿತು
ಕ್ರಿಯೆ-ಯಿಂದ
ಕ್ರಿಯೆಯು
ಕ್ರಿಯೆಯೂ
ಕ್ರಿಯೆಯೇ
ಕ್ರಿಸ್ತ-ನಂತೆ
ಕ್ರಿಸ್ತನೇ
ಕ್ರಿಸ್ತ-ಪೂರ್ವ
ಕ್ರಿಸ್ತ-ಪೂ-ರ್ವಕ್ಕೆ
ಕ್ರೀಡೆ-ಯಂತೆ
ಕ್ರೀಡೆ-ಯಾ-ಗು-ವುದು
ಕ್ರೀಯತೇ
ಕ್ರೂರ
ಕ್ರೂರ-ಕ-ರ್ಮಿ-ಗಳು
ಕ್ರೂರರೂ
ಕ್ರೂರಾನ್
ಕ್ರೂರಿ-ಗಳು
ಕ್ರೆ
ಕ್ರೈಸ್ತ
ಕ್ರೈಸ್ತ-ಧ-ರ್ಮ-ಗ-ಳ-ಲ್ಲಿ-ರು-ವಂತೆ
ಕ್ರೈಸ್ತ-ನಂತೆ
ಕ್ರೈಸ್ತ-ನಿಗೂ
ಕ್ರೋಢೀ-ಕ-ರಿ-ಸಿ-ದರು
ಕ್ರೋಧ
ಕ್ರೋಧ-ಇ-ವು-ಗಳನ್ನು
ಕ್ರೋಧಂ
ಕ್ರೋಧಃ
ಕ್ರೋಧ-ಒಂದು
ಕ್ರೋಧ-ಗಳನ್ನು
ಕ್ರೋಧ-ಗಳಿಂದ
ಕ್ರೋಧ-ಗಳು
ಕ್ರೋಧ-ಗಳೇ
ಕ್ರೋಧ-ದಿಂದ
ಕ್ರೋಧ-ವನ್ನು
ಕ್ರೋಧ-ವಿಲ್ಲ
ಕ್ರೋಧವೂ
ಕ್ರೋಧವೇ
ಕ್ರೋಧ-ಸ್ತಥಾ
ಕ್ರೋಧಾ-ದ್ಭ-ವತಿ
ಕ್ರೋಧೋ-ಽಭಿ-ಜಾ-ಯತೇ
ಕ್ರೌರ್ಯ
ಕ್ರೌರ್ಯ-ಕೃ-ತ್ಯ-ಗಳು
ಕ್ಲಾಸಿಗೂ
ಕ್ಲಾಸಿಗೆ
ಕ್ಲಾಸಿದೆ
ಕ್ಲಾಸಿ-ನಲ್ಲಿ
ಕ್ಲಾಸಿ-ನಿಂದ
ಕ್ಲಿಷ್ಟ-ವಾದ
ಕ್ಲೇದ-ಯಂ-ತ್ಯಾಪೋ
ಕ್ಲೇಶ
ಕ್ಲೇಶ-ಕೊಂ-ಡಿದೆ
ಕ್ಲೇಶ-ಗೊ-ಳಿ-ಸು-ತ್ತಾರೆ
ಕ್ಲೇಶ-ಗೊ-ಳಿ-ಸು-ತ್ತಿ-ರು-ವರೋ
ಕ್ಲೇಶ-ಗೊ-ಳಿ-ಸು-ವರು
ಕ್ಲೇಶ-ಗೊ-ಳಿ-ಸು-ವುದು
ಕ್ಲೇಶ-ವಿಲ್ಲ
ಕ್ಲೇಶ-ವಿ-ಲ್ಲದ
ಕ್ಲೇಶ-ವಿ-ಲ್ಲದೆ
ಕ್ಲೇಶವೂ
ಕ್ಲೇಶೋ-ಽಧಿ-ಕ-ತ-ರ-ಸ್ತೇ-ಷಾ-ಮ-ವ್ಯ-ಕ್ತಾ-ಸ-ಕ್ತ-ಚೇ-ತ-ಸಾಮ್
ಕ್ಲೈಬ್ಯಂ
ಕ್ವಚಿತ್
ಕ್ಷಣ
ಕ್ಷಣ-ಕಾ-ಲ-ದಲ್ಲಿ
ಕ್ಷಣ-ಕಾ-ಲವೂ
ಕ್ಷಣ-ಕ್ಷಣ
ಕ್ಷಣ-ಕ್ಷ-ಣಕ್ಕೂ
ಕ್ಷಣ-ದಲ್ಲಿ
ಕ್ಷಣ-ದ-ಲ್ಲಿಯೇ
ಕ್ಷಣ-ಮಪಿ
ಕ್ಷಣ-ವನ್ನು
ಕ್ಷಣವೂ
ಕ್ಷಣವೇ
ಕ್ಷಣಾ-ರ್ಧ-ದಲ್ಲಿ
ಕ್ಷಣಿಕ
ಕ್ಷಣಿ-ಕತೆ
ಕ್ಷಣಿ-ಕ-ತೆ-ಯನ್ನು
ಕ್ಷಣಿ-ಕದ
ಕ್ಷಣಿ-ಕ-ವನ್ನು
ಕ್ಷಣಿ-ಕ-ವಲ್ಲ
ಕ್ಷತ್ರಿಯ
ಕ್ಷತ್ರಿ-ಯನ
ಕ್ಷತ್ರಿ-ಯ-ನಾಗಿ
ಕ್ಷತ್ರಿ-ಯ-ನಾ-ಗಿ-ದ್ದಾನೆ
ಕ್ಷತ್ರಿ-ಯ-ನಾದ
ಕ್ಷತ್ರಿ-ಯ-ನಿಗೆ
ಕ್ಷತ್ರಿ-ಯನು
ಕ್ಷತ್ರಿ-ಯರ
ಕ್ಷತ್ರಿ-ಯ-ರಿಗೆ
ಕ್ಷತ್ರಿ-ಯರು
ಕ್ಷತ್ರಿ-ಯಸ್ಯ
ಕ್ಷತ್ರಿ-ಯಾಃ
ಕ್ಷಮಾ
ಕ್ಷಮಾ-ಗುಣ
ಕ್ಷಮಾ-ಪಣೆ
ಕ್ಷಮಾ-ಪ-ಣೆ-ಯನ್ನು
ಕ್ಷಮಾ-ರ್ಹ-ನಲ್ಲ
ಕ್ಷಮಾ-ಶೀಲ
ಕ್ಷಮಾ-ಶೀ-ಲನೊ
ಕ್ಷಮಿಸ
ಕ್ಷಮಿ-ಸದ
ಕ್ಷಮಿ-ಸ-ಬ-ಹುದು
ಕ್ಷಮಿ-ಸ-ಬೇಕು
ಕ್ಷಮಿ-ಸ-ಬೇ-ಕೆಂದು
ಕ್ಷಮಿ-ಸಲು
ಕ್ಷಮಿ-ಸಿ-ಬಿ-ಟ್ಟರೆ
ಕ್ಷಮಿ-ಸು-ತ್ತಾನೆ
ಕ್ಷಮಿ-ಸು-ತ್ತಾ-ನೆಯೋ
ಕ್ಷಮಿ-ಸು-ತ್ತಾಳೆ
ಕ್ಷಮಿ-ಸುವ
ಕ್ಷಮಿ-ಸು-ವನು
ಕ್ಷಮಿ-ಸು-ವುದು
ಕ್ಷಮೀ
ಕ್ಷಮೆ
ಕ್ಷಮೆ-ಯಂ-ತಹ
ಕ್ಷಮೆ-ಯನ್ನು
ಕ್ಷಯ
ಕ್ಷಯಂ
ಕ್ಷಯಕ್ಕೂ
ಕ್ಷಯ-ಮಾ-ಡಿ-ಕೊ-ಳ್ಳ-ಬೇ-ಕಾ-ಗಿದೆ
ಕ್ಷಯ-ವಾ-ಗದ
ಕ್ಷಯ-ವಾ-ಗು-ವು-ದಕ್ಕೆ
ಕ್ಷಯ-ವಾ-ಗು-ವುದು
ಕ್ಷಯ-ವಾ-ದ-ಲ್ಲದೆ
ಕ್ಷಯಾಯ
ಕ್ಷಯಿಸಿ
ಕ್ಷಯಿ-ಸುತ್ತಾ
ಕ್ಷಯಿ-ಸು-ವು-ದಿಲ್ಲ
ಕ್ಷರ
ಕ್ಷರಃ
ಕ್ಷರನೂ
ಕ್ಷರ-ಮ-ತೀ-ತೋ-ಽಹ-ಮ-ಕ್ಷ-ರಾ-ದಪಿ
ಕ್ಷರ-ವನ್ನು
ಕ್ಷರ-ಶ್ಚಾ-ಕ್ಷರ
ಕ್ಷರೋ
ಕ್ಷಾಂತಿ
ಕ್ಷಾಂತಿ-ರಾ-ರ್ಜ-ವ-ಮೇವ
ಕ್ಷಾಂತಿ-ರಾ-ರ್ಜ-ವಮ್
ಕ್ಷಾತ್ರ
ಕ್ಷಾತ್ರಂ
ಕ್ಷಾತ್ರ-ತೆ-ಜ-ಸ್ಸಾ-ಗಲಿ
ಕ್ಷಿಪಣಿ
ಕ್ಷಿಪ-ಣಿ-ಗಳು
ಕ್ಷಿಪ-ಣಿ-ಯನ್ನು
ಕ್ಷಿಪಾ-ಮ್ಯ-ಜ-ಸ್ರ-ಮ-ಶು-ಭಾ-ನಾ-ಸು-ರೀ-ಷ್ವೇವ
ಕ್ಷಿಪ್ರಂ
ಕ್ಷೀಣ-ಕ-ಲ್ಮ-ಶ-ರಾಗಿ
ಕ್ಷೀಣ-ಕ-ಲ್ಮ-ಷಾಃ
ಕ್ಷೀಣ-ಮಾ-ಡಿ-ಕೊ-ಳ್ಳ-ಬೇಕು
ಕ್ಷೀಣ-ವಾ-ಗಿ-ದ್ದರೆ
ಕ್ಷೀಣ-ವಾ-ಗಿ-ರು-ವುದೋ
ಕ್ಷೀಣ-ವಾ-ಗು-ವು-ದಿಲ್ಲ
ಕ್ಷೀಣಿ-ಸು-ವುದೋ
ಕ್ಷೀಣಿ-ಸು-ವುವು
ಕ್ಷೀಣೇ
ಕ್ಷೀರ
ಕ್ಷೀರ-ಸಾ-ಗ-ರವೇ
ಕ್ಷುದ್ರ
ಕ್ಷುದ್ರಂ
ಕ್ಷುದ್ರ-ಗ್ರ-ಹ-ದಲ್ಲಿ
ಕ್ಷುದ್ರ-ದೇ-ವ-ತೆ-ಗಳ
ಕ್ಷುದ್ರ-ಪ್ರ-ಪಂ-ಚಕ್ಕೆ
ಕ್ಷೇತ್ರ
ಕ್ಷೇತ್ರಂ
ಕ್ಷೇತ್ರಕ್ಕೂ
ಕ್ಷೇತ್ರಕ್ಕೆ
ಕ್ಷೇತ್ರ-ಕ್ಷೆ-ತ್ರ-ಜ್ಞರ
ಕ್ಷೇತ್ರ-ಕ್ಷೇ-ತ್ರಜ್ಞ
ಕ್ಷೇತ್ರ-ಕ್ಷೇ-ತ್ರ-ಜ್ಞ-ಯೋಗ
ಕ್ಷೇತ್ರ-ಕ್ಷೇ-ತ್ರ-ಜ್ಞ-ಯೋ-ರೇ-ವ-ಮಂ-ತರಂ
ಕ್ಷೇತ್ರ-ಕ್ಷೇ-ತ್ರ-ಜ್ಞ-ಯೋ-ರ್ಜ್ಞಾನಂ
ಕ್ಷೇತ್ರ-ಕ್ಷೇ-ತ್ರ-ಜ್ಞರ
ಕ್ಷೇತ್ರ-ಗಳಲ್ಲಿ
ಕ್ಷೇತ್ರ-ಗ-ಳ-ಲ್ಲಿ-ಯಾ-ದರೋ
ಕ್ಷೇತ್ರ-ಗ-ಳ-ಲ್ಲಿಯೂ
ಕ್ಷೇತ್ರಜ್ಞ
ಕ್ಷೇತ್ರಜ್ಞಂ
ಕ್ಷೇತ್ರ-ಜ್ಞಕ್ಕೆ
ಕ್ಷೇತ್ರ-ಜ್ಞನ
ಕ್ಷೇತ್ರ-ಜ್ಞ-ನನ್ನು
ಕ್ಷೇತ್ರ-ಜ್ಞನು
ಕ್ಷೇತ್ರ-ಜ್ಞನೇ
ಕ್ಷೇತ್ರ-ಜ್ಞರ
ಕ್ಷೇತ್ರ-ಜ್ಞ-ರನ್ನು
ಕ್ಷೇತ್ರ-ಜ್ಞರು
ಕ್ಷೇತ್ರದ
ಕ್ಷೇತ್ರ-ದಲ್ಲಿ
ಕ್ಷೇತ್ರ-ದ-ಲ್ಲಿಯೇ
ಕ್ಷೇತ್ರ-ದಿಂದ
ಕ್ಷೇತ್ರ-ಮಿ-ತ್ಯ-ಭಿ-ಧೀ-ಯತೇ
ಕ್ಷೇತ್ರ-ವನ್ನು
ಕ್ಷೇತ್ರ-ವನ್ನೂ
ಕ್ಷೇತ್ರ-ವೆಂದು
ಕ್ಷೇತ್ರೀ
ಕ್ಷೇಮ
ಕ್ಷೇಮ-ಕ-ರ-ವಾ-ಗು-ವುದು
ಕ್ಷೇಮ-ತರಂ
ಕ್ಷೋತ್ರಂ
ಕ್ಷೋಭೆಗೆ
ಕ್ಷೋಭೆ-ಗೊಂ-ಡಿ-ರು-ವುದು
ಕ್ಷೋಭೆ-ಯ-ನ್ನುಂ-ಟು-ಮಾ-ಡು-ವುದು
ಖಂ
ಖಂಡಿತ
ಖಂಡಿ-ತ-ವಾದಿ
ಖಂಡಿ-ಸು-ವು-ದಕ್ಕೆ
ಖಂಡಿ-ಸು-ವು-ದಿಲ್ಲ
ಖಂಡಿ-ಸು-ವುದು
ಖಗ-ವಾಗಿ
ಖಗೋಳ
ಖಗೋ-ಳ-ಶಾ-ಸ್ತ್ರ-ಜ್ಞರು
ಖಚಿತ
ಖಜಾ-ನಿಯ
ಖಜಾ-ನಿ-ಯೊ-ಳಗೆ
ಖಡ್ಗದ
ಖನಿಜ
ಖಯಾಲಿ
ಖಯಾ-ಲಿ-ಗಳ
ಖಯಾ-ಲಿ-ಗ-ಳಾ-ದರೊ
ಖಯಾ-ಲಿ-ಗಳೆ
ಖಯಾ-ಲಿ-ಯಾಗಿ
ಖರ್ಚನ್ನು
ಖರ್ಚಾಗಿ
ಖರ್ಚಾ-ಗಿಯೂ
ಖರ್ಚಾ-ಗು-ವ-ವ-ರೆಗೆ
ಖರ್ಚಾ-ದರೆ
ಖರ್ಚಾ-ಯಿತು
ಖರ್ಚಿ-ಲ್ಲದ
ಖರ್ಚು
ಖರ್ಚು-ಮಾ-ಡಲು
ಖರ್ಚು-ಮಾಡಿ
ಖಲು
ಖಲ್ವಿದಂ
ಖಾಯಿಲೆ
ಖಾಯಿ-ಲೆ-ಗಳನ್ನು
ಖಾಯಿ-ಲೆ-ಗಳು
ಖಾಯಿ-ಲೆ-ಗಳೂ
ಖಾಯಿ-ಲೆಗೆ
ಖಾಯಿ-ಲೆಯ
ಖಾಯಿ-ಲೆ-ಯನ್ನು
ಖಾಯಿ-ಲೆ-ಯಿಂದ
ಖಾಯಿ-ಲೆ-ಯೆಲ್ಲ
ಖಾರ
ಖಾರದ
ಖಾರ-ವನ್ನು
ಖಾರ-ವಾದ
ಖಾರವೂ
ಖಾರ-ವೆ-ನಿ-ಸು-ವುದು
ಖಾಲಿ
ಖಾಲಿ-ಮಾಡಿ
ಖಾಲಿ-ಯನ್ನು
ಖಾಲಿ-ಯಾಗಿ
ಖಾಲಿ-ಯಾ-ಗಿದೆ
ಖಾಲಿ-ಯಾ-ಗಿದ್ದು
ಖಾಲಿ-ಯಾ-ಗಿ-ರು-ವಾಗ
ಖಾಲಿ-ಯಾ-ಗಿ-ಹೋ-ಗಿಲ್ಲ
ಖುಷಿ
ಖೇ
ಖೋತಾ
ಖ್ಯಾತಿ
ಖ್ಯಾತಿ-ಯನ್ನು
ಖ್ಯಾತಿ-ವಂ-ತ-ರಾ-ಗಿ-ದ್ದರು
ಗಂಗಾ
ಗಂಗಾ-ನದಿ
ಗಂಗಾ-ನ-ದಿಗೆ
ಗಂಗಾ-ನ-ದಿ-ಯಷ್ಟು
ಗಂಗಾ-ವಾ-ರಿ-ಯನ್ನು
ಗಂಗಾ-ಸ್ನಾನ
ಗಂಗೆ
ಗಂಗೆ-ಯನ್ನು
ಗಂಗೆ-ಯಲ್ಲಿ
ಗಂಗೆ-ಯ-ಲ್ಲಿಯೇ
ಗಂಟನ್ನು
ಗಂಟ-ನ್ನೆಲ್ಲ
ಗಂಟ-ಲಲ್ಲಿ
ಗಂಟ-ಲಿನ
ಗಂಟಲು
ಗಂಟಿ-ಲಿ-ನಲ್ಲಿ
ಗಂಟು
ಗಂಟು-ಬಿ-ದ್ದಿದೆ
ಗಂಟೆ
ಗಂಟೆ-ಗ-ಟ್ಟಲೆ
ಗಂಟೆ-ಗಳಲ್ಲಿ
ಗಂಟೆ-ಗ-ಳಾಗಿ
ಗಂಟೆ-ಗಳು
ಗಂಟೆಗೆ
ಗಂಟೆ-ಯಲ್ಲಿ
ಗಂಡ
ಗಂಡನ
ಗಂಡ-ನನ್ನು
ಗಂಡಸು
ಗಂಡಸೊ
ಗಂಡಾಂ-ತರ
ಗಂಡು
ಗಂಡೆದೆ
ಗಂತವ್ಯ
ಗಂತವ್ಯಂ
ಗಂತಾಸಿ
ಗಂಧ
ಗಂಧಃ
ಗಂಧ-ಗ-ಳೆಂ-ಬುದು
ಗಂಧದ
ಗಂಧ-ದಿಂದ
ಗಂಧರ್ವ
ಗಂಧ-ರ್ವ-ಯ-ಕ್ಷಾ-ಸು-ರ-ಸಿ-ದ್ಧ-ಸಂಘಾ
ಗಂಧ-ರ್ವ-ರಲ್ಲಿ
ಗಂಧ-ರ್ವರು
ಗಂಧ-ರ್ವಾ-ಣಾಂ
ಗಂಧವೂ
ಗಂಧವೇ
ಗಂಭೀ-ರ-ವಾಗಿ
ಗಂಭೀ-ರ-ವಾ-ಣಿ-ಯಿಂದ
ಗಗ-ನ-ದಲ್ಲಿ
ಗಚ್ಛ
ಗಚ್ಛಂತಿ
ಗಚ್ಛಂ-ತ್ಯ-ನಾ-ಮ-ಯಮ್
ಗಚ್ಛಂ-ತ್ಯ-ಪು-ನ-ರಾ-ವೃ-ತ್ತಿಂ
ಗಚ್ಛಂ-ತ್ಯ-ಮೂ-ಢಾಃ
ಗಚ್ಛತಿ
ಗಚ್ಛನ್
ಗಜ-ಕ-ಡ್ಡಿ-ಯಿಂದ
ಗಜ-ಗಳು
ಗಜೇಂ-ದ್ರಾ-ಣಾಂ
ಗಟ್ಚಿ
ಗಟ್ಟಿ
ಗಟ್ಟಿ-ಗ-ಟ್ಟಿ-ಯಾಗಿ
ಗಟ್ಟಿ-ಮು-ಟ್ಟಾ-ಗಿ-ರ-ಬೇಕು
ಗಟ್ಟಿ-ಮು-ಟ್ಟಾದ
ಗಟ್ಟಿಯ
ಗಟ್ಟಿ-ಯನ್ನು
ಗಟ್ಟಿ-ಯ-ನ್ನೆಲ್ಲಾ
ಗಟ್ಟಿ-ಯಾಗಿ
ಗಟ್ಟಿ-ಯಾ-ಗಿದೆ
ಗಟ್ಟಿ-ಯಾ-ಗಿ-ದೆಯೆ
ಗಟ್ಟಿ-ಯಾ-ಗಿ-ರ-ಬಾ-ರದು
ಗಟ್ಟಿ-ಯಾದ
ಗಡ-ಗಡ
ಗಡಿಗೆ
ಗಡಿ-ಗೆ-ಯೊ-ಡೆ-ಯ-ಬ-ಹುದು
ಗಡಿ-ಬಿ-ಡಿ-ಯಾ-ಗಲಿ
ಗಡಿ-ಬಿ-ಡಿ-ಯಾ-ಗಿ-ರುವ
ಗಡಿ-ಬಿ-ಡಿಯೂ
ಗಡಿ-ಯಾರ
ಗಡಿ-ಯಾ-ರಕ್ಕೆ
ಗಡಿ-ಯಾ-ರ-ದಲ್ಲಿ
ಗಡಿ-ಯಾ-ರ-ದ-ಲ್ಲಿ-ರುವ
ಗಡ್ಡೆ-ಯಂತೆ
ಗಣ-ಗಳು
ಗಣ-ನೆಗೆ
ಗಣ-ನೆಗೇ
ಗಣ-ಪತಿ
ಗಣ-ಪ-ತಿ-ಯನ್ನೊ
ಗಣಿ
ಗಣಿಗೆ
ಗಣಿಯ
ಗಣಿ-ಯಿಂದ
ಗತಃ
ಗತ-ಕಾ-ಲದ
ಗತ-ರಸಂ
ಗತ-ವ್ಯಥಃ
ಗತ-ಸಂ-ಗಸ್ಯ
ಗತ-ಸಂ-ದೇಹಃ
ಗತಾ
ಗತಾಃ
ಗತಾ-ಗತಂ
ಗತಾ-ಸೂ-ನ-ಗ-ತಾ-ಸೂಂಶ್ಚ
ಗತಿ
ಗತಿಂ
ಗತಿಃ
ಗತಿ-ಗಳು
ಗತಿಮ್
ಗತಿಯ
ಗತಿ-ಯನ್ನು
ಗತಿ-ಯಷ್ಟು
ಗತಿ-ಯಿಲ್ಲ
ಗತಿಯೂ
ಗತಿ-ಯೆಲ್ಲ
ಗತಿ-ಯೇನು
ಗತಿ-ಯೊಂ-ದಿದೆ
ಗತಿ-ರ್ದುಖಂ
ಗತಿ-ರ್ಭರ್ತಾ
ಗತಿಸಿ
ಗತಿ-ಸಿ-ಹೋದ
ಗತೀ
ಗತ್ವಾ
ಗದಾ-ಧ-ರನೂ
ಗದಿನಂ
ಗದೆ
ಗದೇ
ಗದ್ಗದ
ಗದ್ದಲ
ಗದ್ದ-ಲ-ವಿ-ಲ್ಲದೆ
ಗದ್ದೆ
ಗದ್ದೆಗ
ಗದ್ದೆ-ಯಲ್ಲಿ
ಗಬ-ಗಬ
ಗಮಃ
ಗಮನ
ಗಮ-ನಕ್ಕೆ
ಗಮ-ನ-ದ-ಲ್ಲಿ-ಟ್ಟಿ-ರು-ವನು
ಗಮ-ನ-ದ-ಲ್ಲಿ-ಟ್ಟಿ-ರು-ವುದು
ಗಮ-ನ-ದ-ಲ್ಲಿ-ಟ್ಟು-ಕೊಂ-ಡಿಲ್ಲ
ಗಮ-ನ-ವನ್ನು
ಗಮ-ನ-ವನ್ನೇ
ಗಮ-ನವೆ
ಗಮ-ನವೇ
ಗಮ-ನಿಸ
ಗಮ-ನಿ-ಸದೆ
ಗಮ-ನಿ-ಸದೇ
ಗಮ-ನಿ-ಸ-ಬೇ-ಕಾ-ಗಿದೆ
ಗಮ-ನಿ-ಸ-ಬೇ-ಕಾ-ಗಿ-ರು-ವುದು
ಗಮ-ನಿ-ಸ-ಬೇ-ಕಾ-ಗು-ವುದು
ಗಮ-ನಿ-ಸ-ಬೇಕು
ಗಮ-ನಿ-ಸಲಿ
ಗಮ-ನಿ-ಸ-ಲಿಲ್ಲ
ಗಮ-ನಿ-ಸಿ-ಸು-ವುದೇ
ಗಮ-ನಿ-ಸು-ತ್ತಾನೆ
ಗಮ-ನಿ-ಸು-ತ್ತಿದೆ
ಗಮ-ನಿ-ಸು-ತ್ತಿ-ರು-ವನು
ಗಮ-ನಿ-ಸು-ತ್ತೇವೆ
ಗಮ-ನಿ-ಸುವ
ಗಮ-ನಿ-ಸು-ವನು
ಗಮ-ನಿ-ಸು-ವು-ದಿಲ್ಲ
ಗಮ-ನಿ-ಸು-ವು-ದಿ-ಲ್ಲವೊ
ಗಮ-ನಿ-ಸು-ವುದು
ಗಮ-ನಿ-ಸು-ವುದೇ
ಗಮ್ಯತೇ
ಗಮ್ಯ-ವಸ್ತು
ಗರ
ಗರ-ಗಸ
ಗರ-ಗ-ಸ-ದಲ್ಲಿ
ಗರ-ಡಿಯ
ಗರ-ಡಿ-ಯಲ್ಲಿ
ಗರ-ಡಿ-ಯ-ಲ್ಲಿಯೂ
ಗರ-ಡೀ-ಮ-ನೆ-ಯಲ್ಲಿ
ಗರದ
ಗರಿ-ಗಳು
ಗರಿ-ಗ-ಳೆಲ್ಲ
ಗರೀ-ಬರು
ಗರೀ-ಯಸೇ
ಗರೀಯೋ
ಗರುಡ
ಗರ್ಭ
ಗರ್ಭಂ
ಗರ್ಭ-ಗು-ಡಿಗೆ
ಗರ್ಭ-ಗು-ಡಿ-ಯಲ್ಲಿ
ಗರ್ಭ-ಗು-ಡಿಯೆ
ಗರ್ಭ-ದಲ್ಲಿ
ಗರ್ಭ-ದಿಂದ
ಗರ್ಭ-ವನ್ನು
ಗರ್ಭ-ಸ್ತಥಾ
ಗರ್ಭಿ-ಣಿ-ಯಾದ
ಗಲಾಟೆ
ಗಲಾ-ಟೆ-ಗೆಲ್ಲ
ಗಲಾ-ಟೆ-ಯಿಲ್ಲ
ಗಲಾ-ಟೆಯೂ
ಗಲಾ-ಟೆ-ಯೆಲ್ಲ
ಗಲಿ-ಬಿಲಿ
ಗಲೀ-ಜನ್ನು
ಗಲೀ-ಜಾ-ಗಿದೆ
ಗಲೀ-ಜಾ-ಗು-ವುದು
ಗಲೀಜು
ಗಲೀ-ಜು-ಗಳನ್ನೂ
ಗಲೂ
ಗಲ್ಲಿಗೆ
ಗಳ
ಗಳಂತೆ
ಗಳನ್ನು
ಗಳನ್ನೂ
ಗಳ-ನ್ನೆಲ್ಲ
ಗಳ-ನ್ನೆಲ್ಲಾ
ಗಳನ್ನೇ
ಗಳಲ್ಲ
ಗಳಲ್ಲಿ
ಗಳ-ಲ್ಲಿಯೂ
ಗಳ-ಲ್ಲೆಲ್ಲ
ಗಳಾ-ಗಿವೆ
ಗಳಾ-ಗು-ತ್ತೇವೆ
ಗಳಾದ
ಗಳಾ-ವು-ದಕ್ಕೂ
ಗಳಾ-ವುವು
ಗಳಿಂದ
ಗಳಿಂ-ದಲೂ
ಗಳಿ-ಗಾ-ದರೋ
ಗಳಿಗೂ
ಗಳಿಗೆ
ಗಳಿ-ಗೆ-ಯಲ್ಲಿ
ಗಳಿ-ಗೆಲ್ಲ
ಗಳಿ-ಗೆಲ್ಲಾ
ಗಳಿ-ದ್ದಾರೆ
ಗಳಿ-ದ್ದುವು
ಗಳಿ-ರು-ತ್ತವೆ
ಗಳಿವೆ
ಗಳಿಸ
ಗಳಿ-ಸ-ಬೇ-ಕಾ-ದರೂ
ಗಳಿ-ಸ-ಬೇ-ಕಾ-ದರೆ
ಗಳಿ-ಸ-ಬೇಕು
ಗಳಿ-ಸ-ಬೇ-ಕೆಂದು
ಗಳಿ-ಸ-ಲಾರ
ಗಳಿ-ಸ-ಲಾ-ರವು
ಗಳಿ-ಸಲು
ಗಳಿಸಿ
ಗಳಿ-ಸಿ-ಕೊಂ-ಡರೆ
ಗಳಿ-ಸಿ-ಕೊಂ-ಡಿ-ರು-ವನೋ
ಗಳಿ-ಸಿ-ಕೊಂ-ಡಿಲ್ಲ
ಗಳಿ-ಸಿ-ಕೊಂಡು
ಗಳಿ-ಸಿ-ಕೊಟ್ಟ
ಗಳಿ-ಸಿದ
ಗಳಿ-ಸಿ-ದು-ದನ್ನು
ಗಳಿ-ಸಿದ್ದ
ಗಳಿ-ಸಿ-ರು-ವರು
ಗಳಿ-ಸಿ-ರು-ವೆನು
ಗಳಿಸು
ಗಳಿ-ಸು-ತ್ತಿ-ರು-ವುದನ್ನು
ಗಳಿ-ಸು-ತ್ತೇನೆ
ಗಳಿ-ಸುವ
ಗಳಿ-ಸು-ವನು
ಗಳಿ-ಸು-ವುದ
ಗಳಿ-ಸು-ವು-ದ-ಕ್ಕಾಗಿ
ಗಳಿ-ಸು-ವು-ದಕ್ಕೆ
ಗಳಿ-ಸು-ವುದು
ಗಳಿ-ಸು-ವೆವು
ಗಳು
ಗಳು-ಇ-ವು-ಗಳ
ಗಳುಳ್ಳ
ಗಳೂ
ಗಳೆಂಬ
ಗಳೆ-ರಡು
ಗಳೆಲ್ಲ
ಗಳೆ-ಲ್ಲ-ವನ್ನು
ಗಳೆ-ಲ್ಲವೂ
ಗಳೆಲ್ಲಾ
ಗಳೇ
ಗಳೊ
ಗಳೊಂ-ದಿಗೆ
ಗಳೊಂದೇ
ಗಳೋ
ಗವಾ-ಕ್ಷ-ಗಳ
ಗವಾ-ಕ್ಷ-ಗಳನ್ನು
ಗವಾ-ಕ್ಷದ
ಗವಾ-ಕ್ಷಿ-ಗಳ
ಗವಿ
ಗಹನ
ಗಹ-ನ-ವಾ-ಗಿದೆ
ಗಹ-ನ-ವಾ-ಗಿ-ರು-ವುದು
ಗಹ-ನ-ವಾದ
ಗಹ-ನ-ವಾ-ದದ್ದು
ಗಹನಾ
ಗಾಂಡೀವ
ಗಾಂಡೀವಂ
ಗಾಂಡೀ-ವ-ವನ್ನು
ಗಾಂಢಾಂ-ಧ-ಕಾ-ರ-ದಲ್ಲಿ
ಗಾಂಢೀವ
ಗಾಂಧಾರ
ಗಾಂಧಾ-ರ-ನೀ-ಲೋ-ತ್ಪಲಾ
ಗಾಂಧಾ-ರ-ರಾಜ
ಗಾಂಧಾ-ರಿ-ಯರ
ಗಾಂಧೀಜಿ
ಗಾಂಭೀ-ರ್ಯ-ದಿಂದ
ಗಾಂಭೀ-ರ್ಯ-ವಿದೆ
ಗಾಜಿನ
ಗಾಜು
ಗಾಟಿನ
ಗಾಡಾಂ-ಧ-ಕಾ-ರ-ದಲ್ಲಿ
ಗಾಡಿ
ಗಾಡಿ-ಗಳು
ಗಾಡಿ-ಗಳೂ
ಗಾಡಿ-ಗ-ಳೆಲ್ಲ
ಗಾಡಿಗೆ
ಗಾಡಿಯ
ಗಾಡಿ-ಯನ್ನು
ಗಾಡಿ-ಯನ್ನೇ
ಗಾಡಿ-ಯಲ್ಲಿ
ಗಾಢ
ಗಾಢ-ನಿ-ಯಮ
ಗಾಢ-ವಾದ
ಗಾಢಾಂ-ಧ-ಕಾರ
ಗಾಢಾಂ-ಧ-ಕಾ-ರ-ದಲ್ಲಿ
ಗಾಢಾಂ-ಧ-ಕಾ-ರ-ದಲ್ಲೂ
ಗಾಣಕ್ಕೆ
ಗಾಣದ
ಗಾಣ-ದಲ್ಲಿ
ಗಾಣ-ವನ್ನು
ಗಾಣು-ತ್ತಿದ್ದ
ಗಾತ್ರಕ್ಕೆ
ಗಾತ್ರಾಣಿ
ಗಾದರೂ
ಗಾದರೋ
ಗಾದೆ
ಗಾದೆ-ಯಂತೆ
ಗಾನ
ಗಾನ-ದಂತೆ
ಗಾನ-ದಲ್ಲಿ
ಗಾನ-ಲ-ಹರಿ
ಗಾನ-ವನ್ನು
ಗಾನವೇ
ಗಾನ-ಸೌಂ-ದರ್ಯ
ಗಾನ-ಸ್ಪಂ-ದ-ನ-ವನ್ನು
ಗಾಮಾ-ಗ-ಳಂ-ಥ-ವರು
ಗಾಮಾ-ವಿಶ್ಯ
ಗಾಯ
ಗಾಯ-ಕ್ಕಿಂತ
ಗಾಯಕ್ಕೆ
ಗಾಯ-ಗೊಂಡ
ಗಾಯತ್ರಿ
ಗಾಯತ್ರೀ
ಗಾಯದ
ಗಾಯ-ವನ್ನು
ಗಾಯ-ವಾಗಿ
ಗಾರೆಗೆ
ಗಾಲಿ-ಗಳು
ಗಾಲಿಯ
ಗಾಲಿ-ಯನ್ನು
ಗಾಳ
ಗಾಳದ
ಗಾಳಿ
ಗಾಳಿ-ಗಳು
ಗಾಳಿಗೆ
ಗಾಳಿ-ದೋ-ಣಿ-ಯನ್ನು
ಗಾಳಿಯ
ಗಾಳಿ-ಯಂತೆ
ಗಾಳಿ-ಯನ್ನು
ಗಾಳಿ-ಯಲ್ಲಿ
ಗಾಳಿ-ಯ-ಲ್ಲಿ-ರು-ವನು
ಗಾಳಿ-ಯಾ-ಗು-ವುದು
ಗಾಳಿ-ಯಿಂದ
ಗಾಳಿ-ಯಿ-ಲ್ಲದೆ
ಗಾಳಿಯೂ
ಗಾಳಿ-ಯೊಂ-ದಿಗೆ
ಗಾಳಿ-ಸೇ-ವ-ನೆಗೆ
ಗಾಳಿ-ಹೊ-ಡೆತ
ಗಾವೋ
ಗಿಂತ
ಗಿಟ-ಕಿಗೆ
ಗಿಟ-ಕಿ-ನ-ಲ್ಲಿ-ರುವ
ಗಿಟುಕು
ಗಿಡ
ಗಿಡಕ್ಕೆ
ಗಿಡ-ಗಳನ್ನು
ಗಿಡ-ಗ-ಳಿವೆ
ಗಿಡ-ಗಳು
ಗಿಡ-ದಂತೆ
ಗಿಡ-ದಲ್ಲಿ
ಗಿಡ-ದ-ಲ್ಲಿ-ರುವ
ಗಿಡ-ದಿಂದ
ಗಿಡ-ಮ-ನೆ-ಗಳನ್ನು
ಗಿಡ-ಮರ
ಗಿಡ-ಮ-ರ-ಗಳು
ಗಿಡ-ವನ್ನು
ಗಿಡ-ವಾ-ಗ-ಬೇಕು
ಗಿತ್ತು
ಗಿದೆ
ಗಿರಾಕಿ
ಗಿರಾ-ಕಿ-ಗಳು
ಗಿರಾ-ಮ-ಸ್ಮ್ಯೇ-ಕ-ಮ-ಕ್ಷ-ರಮ್
ಗಿರಿ
ಗಿರಿ-ಗು-ಹೆ-ಯಲ್ಲ
ಗಿರಿಮ್
ಗಿರಿ-ಶೃಂ-ಗ-ವ-ನ್ನೇ-ರು-ತ್ತಾನೆ
ಗಿರೀ-ಶ-ಚಂದ್ರ
ಗಿರುವ
ಗಿರು-ವು-ದಕ್ಕೆ
ಗಿರು-ವುದನ್ನು
ಗಿರು-ವುದು
ಗಿಲ್ಲ
ಗೀಚಿ-ದೊ-ಡನೆ
ಗೀತಂ
ಗೀತಾ
ಗೀತಾ-ದ-ರ್ಶನ
ಗೀತಾ-ಪಾ-ರಾ-ಯಣ
ಗೀತಾ-ಬೋ-ಧ-ನೆ-ಯಲ್ಲಿ
ಗೀತಾ-ಭ-ಗ-ವತಿ
ಗೀತಾ-ಭಾ-ವ-ಧಾರೆ
ಗೀತಾ-ಮಾತೆ
ಗೀತಾ-ಮಾ-ತೆ-ಯನ್ನು
ಗೀತಾ-ಮೃತಂ
ಗೀತಾ-ಮೃ-ತ-ದುಹೇ
ಗೀತಾ-ಮೃ-ತ-ವನ್ನು
ಗೀತಾ-ಮೃ-ತವೇ
ಗೀತಾರ್ಥ
ಗೀತಾ-ರ್ಥ-ಗಂ-ಧೋ-ತ್ಕಟಂ
ಗೀತಾ-ರ್ಥದ
ಗೀತಾ-ರ್ಥ-ಸಾ-ರ-ವೆಲ್ಲ
ಗೀತಾ-ಸಂ-ದೇ-ಶ-ವನ್ನು
ಗೀತೆ
ಗೀತೆ-ಗಾಗಿ
ಗೀತೆಗೆ
ಗೀತೆಯ
ಗೀತೆ-ಯಂ-ತಹ
ಗೀತೆ-ಯಂತೆ
ಗೀತೆ-ಯ-ನ್ನಾ-ದರೂ
ಗೀತೆ-ಯನ್ನು
ಗೀತೆ-ಯನ್ನೆ
ಗೀತೆ-ಯಲ್ಲಿ
ಗೀತೆ-ಯಿಂದ
ಗೀತೆ-ಯೆಂಬ
ಗೀತೋ-ಪ-ದೇಶ
ಗೀತೋ-ಪ-ದೇ-ಶ-ದಲ್ಲಿ
ಗುಂಜೇ
ಗುಂಡಿ-ಗಳು
ಗುಂಡಿಗೆ
ಗುಂಡಿ-ನೇ-ಟಿಗೆ
ಗುಂಡಿಯ
ಗುಂಡಿ-ಯಲ್ಲಿ
ಗುಂಪಿಗೆ
ಗುಂಪಿಗೇ
ಗುಂಪಿನ
ಗುಂಪಿ-ನಲ್ಲಿ
ಗುಂಪಿ-ನ-ವನೆ
ಗುಂಪಿ-ನ-ವರು
ಗುಂಪಿ-ನಿಂದ
ಗುಂಪು
ಗುಜ-ರಿ-ಯಲ್ಲಿ
ಗುಟುಕು
ಗುಟ್ಟನ್ನು
ಗುಟ್ಟಾಗಿ
ಗುಟ್ಟು
ಗುಟ್ಟೆಲ್ಲ
ಗುಡಾ-ಕೇಶ
ಗುಡಾ-ಕೇಶಃ
ಗುಡಾ-ಕೇ-ಶೇನ
ಗುಡಾ-ಣ-ದಲ್ಲಿ
ಗುಡಿಯ
ಗುಡಿ-ಯನ್ನು
ಗುಡಿ-ಸ-ಲ-ಮೇಲೆ
ಗುಡಿ-ಸಲು
ಗುಡಿಸಿ
ಗುಡಿ-ಸಿ-ಲಿನ
ಗುಡಿ-ಸು-ವ-ವನು
ಗುಡಿ-ಸು-ವ-ವರು
ಗುಡಿ-ಸು-ವುದು
ಗುಡು-ಗಿ-ನಂತೆ
ಗುಣ
ಗುಣ-ಕರ್ಮ
ಗುಣ-ಕ-ರ್ಮ-ವಿ-ಭಾ-ಗಶಃ
ಗುಣ-ಕ-ರ್ಮಸು
ಗುಣ-ಕ್ಕಿಂತ
ಗುಣಕ್ಕೂ
ಗುಣಕ್ಕೆ
ಗುಣ-ಗ-ಣ-ನೆ-ಯಲ್ಲಿ
ಗುಣ-ಗಳ
ಗುಣ-ಗ-ಳಂತೆ
ಗುಣ-ಗಳನ್ನು
ಗುಣ-ಗಳನ್ನೂ
ಗುಣ-ಗಳನ್ನೆಲ್ಲ
ಗುಣ-ಗಳನ್ನೆಲ್ಲಾ
ಗುಣ-ಗ-ಳನ್ನೇ
ಗುಣ-ಗಳಲ್ಲಿ
ಗುಣ-ಗ-ಳ-ಲ್ಲೆಲ್ಲ
ಗುಣ-ಗ-ಳ-ಲ್ಲೆಲ್ಲಾ
ಗುಣ-ಗ-ಳಾಚೆ
ಗುಣ-ಗ-ಳಾ-ದರೊ
ಗುಣ-ಗಳಿಂದ
ಗುಣ-ಗ-ಳಿಂ-ದಲೇ
ಗುಣ-ಗ-ಳಿ-ಗಿಂತ
ಗುಣ-ಗ-ಳಿಗೂ
ಗುಣ-ಗ-ಳಿಗೆ
ಗುಣ-ಗ-ಳಿ-ದ್ದರೆ
ಗುಣ-ಗ-ಳಿ-ದ್ದ-ವನು
ಗುಣ-ಗ-ಳಿ-ರ-ಬೇಕು
ಗುಣ-ಗ-ಳಿ-ರು-ವ-ವನೇ
ಗುಣ-ಗ-ಳಿ-ಲ್ಲದ
ಗುಣ-ಗ-ಳಿ-ಲ್ಲದೆ
ಗುಣ-ಗ-ಳಿ-ಲ್ಲದೇ
ಗುಣ-ಗ-ಳಿ-ಲ್ಲ-ವೆಂ-ದಲ್ಲ
ಗುಣ-ಗ-ಳಿವೆ
ಗುಣ-ಗ-ಳಿ-ವೆಯೆ
ಗುಣ-ಗ-ಳಿ-ವೆಯೋ
ಗುಣ-ಗಳು
ಗುಣ-ಗ-ಳುಳ್ಳ
ಗುಣ-ಗ-ಳು-ಳ್ಳ-ವ-ನನ್ನು
ಗುಣ-ಗಳೂ
ಗುಣ-ಗ-ಳೆಂಬ
ಗುಣ-ಗ-ಳೆಲ್ಲ
ಗುಣ-ಗ-ಳೆಲ್ಲಾ
ಗುಣ-ಗಳೇ
ಗುಣ-ಗ-ಳೇನು
ಗುಣ-ಗ-ಳೇನೊ
ಗುಣ-ಗ-ಳೊಂ-ದಿಗೆ
ಗುಣ-ತ-ಸ್ತ್ರಿ-ವಿಧಂ
ಗುಣ-ತ್ರ-ಯ-ವಿ-ಭಾ-ಗ-ಯೋಗ
ಗುಣದ
ಗುಣ-ದಲ್ಲಿ
ಗುಣ-ದ-ಲ್ಲಿ-ರು-ವ-ವರು
ಗುಣ-ದ-ಲ್ಲಿ-ರು-ವು-ದೆಲ್ಲ
ಗುಣ-ದಿಂದ
ಗುಣ-ದೊಂ-ದಿಗೆ
ಗುಣ-ದೊ-ಳಗೆ
ಗುಣ-ದೋ-ಷ-ಗಳು
ಗುಣ-ಧ-ರ್ಮ-ಗಳನ್ನು
ಗುಣ-ಧ-ರ್ಮ-ಗಳೆ
ಗುಣ-ಪ-ಡಿ-ಸ-ಲಾ-ಗದ
ಗುಣ-ಪ್ರ-ವೃದ್ಧಾ
ಗುಣ-ಭೇ-ದತಃ
ಗುಣ-ಭೇ-ದಾ-ನು-ಸಾ-ರ-ವಾಗಿ
ಗುಣ-ಭೋಕ್ತೃ
ಗುಣ-ಮ-ಯ-ವಾದ
ಗುಣ-ಮಯೀ
ಗುಣ-ಮಾ-ಡ-ಬ-ಹುದು
ಗುಣ-ಮಾ-ಡಿ-ಕೊ-ಳ್ಳ-ಬೇ-ಕಾ-ದರೆ
ಗುಣ-ಮಾ-ಡಿ-ಕೊ-ಳ್ಳು-ತ್ತೇವೆ
ಗುಣ-ಮಾಡು
ಗುಣ-ಮಾ-ಡು-ತ್ತೇನೆ
ಗುಣ-ಮಾ-ಡು-ವುದೋ
ಗುಣ-ವನ್ನು
ಗುಣ-ವನ್ನೇ
ಗುಣ-ವಲ್ಲ
ಗುಣ-ವಾ-ಗಲೀ
ಗುಣ-ವಾ-ಗು-ವ-ವ-ರೆಗೆ
ಗುಣ-ವಾ-ಗು-ವುದು
ಗುಣ-ವಾ-ಚ-ಕ-ಗಳ
ಗುಣ-ವಾ-ಚ-ಕ-ಗಳನ್ನು
ಗುಣ-ವಾ-ಚ-ಕ-ಗಳು
ಗುಣ-ವಾ-ಚ-ಕ-ಗಳೂ
ಗುಣ-ವಾ-ಚ-ಕ-ದಿಂದ
ಗುಣ-ವಾ-ಚ-ಕ-ವನ್ನು
ಗುಣ-ವಿ-ದ್ದರೆ
ಗುಣ-ವಿಲ್ಲ
ಗುಣವೂ
ಗುಣವೆ
ಗುಣವೇ
ಗುಣ-ಸಂ-ಖ್ಯಾನೇ
ಗುಣ-ಸಂ-ಗವೇ
ಗುಣ-ಸಂ-ಗೋಽಸ್ಯ
ಗುಣಾ
ಗುಣಾಂ-ಶ್ಚೈವ
ಗುಣಾಃ
ಗುಣಾ-ತೀತ
ಗುಣಾ-ತೀತಃ
ಗುಣಾ-ತೀ-ತನ
ಗುಣಾ-ತೀ-ತ-ನನ್ನು
ಗುಣಾ-ತೀ-ತ-ನಲ್ಲಿ
ಗುಣಾ-ತೀ-ತ-ನಾ-ಗು-ವುದೇ
ಗುಣಾ-ತೀ-ತ-ನಿಗೂ
ಗುಣಾ-ತೀ-ತನೂ
ಗುಣಾ-ತೀ-ತನೇ
ಗುಣಾ-ತೀ-ತ-ರಿಂದ
ಗುಣಾ-ನ-ತಿ-ವ-ರ್ತತೇ
ಗುಣಾ-ನು-ಸಾರ
ಗುಣಾ-ನೇ-ತಾ-ನ-ತೀತೋ
ಗುಣಾ-ನೇ-ತಾ-ನ-ತೀತ್ಯ
ಗುಣಾನ್
ಗುಣಾ-ನ್ವಿ-ತಮ್
ಗುಣಾ-ವ-ಗು-ಣ-ಗಳನ್ನು
ಗುಣಿಯ
ಗುಣೇಭ್ಯಃ
ಗುಣೇ-ಭ್ಯಶ್ಚ
ಗುಣೇಶು
ಗುಣೈಃ
ಗುಣೈರ್ಯೋ
ಗುದ್ದಾ-ಡ-ಬೇ-ಕಾ-ಗಿಲ್ಲ
ಗುಪ್ತ-ಗಾ-ಮಿ-ನಿ-ಯಂತೆ
ಗುಪ್ತ-ವಾ-ಗಿ-ರು-ವುದು
ಗುಮಾಸ್ತ
ಗುರಿ
ಗುರಿ-ಕ-ಡೆಗೆ
ಗುರಿಗೆ
ಗುರಿ-ಗೇನೋ
ಗುರಿ-ಮಾ-ಡಲಿ
ಗುರಿ-ಮು-ಟ್ಟಿ-ದ-ವನು
ಗುರಿಯ
ಗುರಿ-ಯ-ನ್ನಾಗಿ
ಗುರಿ-ಯನ್ನು
ಗುರಿ-ಯಲ್ಲ
ಗುರಿ-ಯಲ್ಲಿ
ಗುರಿ-ಯಾಗಿ
ಗುರಿ-ಯಾ-ಗಿ-ದೆಯೋ
ಗುರಿ-ಯಾ-ಗು-ವುದು
ಗುರಿಯೂ
ಗುರಿ-ಯೆ-ಡೆಗೆ
ಗುರಿ-ಯೆಲ್ಲ
ಗುರಿಯೇ
ಗುರಿ-ಸೇ-ರ-ಬ-ಹುದು
ಗುರು
ಗುರು-ಗಳ
ಗುರು-ಗಳನ್ನು
ಗುರು-ಗ-ಳಾ-ಗಿ-ರ-ಬ-ಹುದು
ಗುರು-ಗ-ಳಾ-ಗಿ-ರಲೀ
ಗುರು-ಗ-ಳಾ-ಗಿ-ರು-ವರೋ
ಗುರು-ಗ-ಳಿಗೆ
ಗುರು-ಗಳು
ಗುರು-ಣಾಪಿ
ಗುರು-ತಿಸಿ
ಗುರುತು
ಗುರು-ದ-ಕ್ಷಿಣೆ
ಗುರು-ಪು-ತ್ರ-ರನ್ನು
ಗುರು-ಭ-ಕ್ತಿಗೆ
ಗುರು-ಭಾ-ಯಿ-ಗಳು
ಗುರು-ರ್ಗ-ರೀ-ಯಾನ್
ಗುರು-ವನ್ನು
ಗುರು-ವಾ-ಗ-ಬ-ಹುದು
ಗುರು-ವಾ-ಗ-ಲಾರ
ಗುರು-ವಾ-ಗಿಲ್ಲ
ಗುರು-ವಾದ
ಗುರು-ವಾ-ದರೂ
ಗುರು-ವಾ-ದ-ವನು
ಗುರುವಿ
ಗುರು-ವಿಗೆ
ಗುರು-ವಿನ
ಗುರು-ವಿ-ನಂತೆ
ಗುರು-ವಿ-ನಲ್ಲಿ
ಗುರು-ವಿ-ನ-ಲ್ಲಿ-ರುವ
ಗುರು-ವಿ-ನಿಂದ
ಗುರು-ವಿ-ನೆ-ದು-ರಿಗೆ
ಗುರುವೂ
ಗುರು-ವೆಂದು
ಗುರು-ವೇನೊ
ಗುರು-ವೊ-ಬ್ಬ-ನಿಗೆ
ಗುರು-ಶು-ಶ್ರೂ-ಷೆ-ಯಿಂದ
ಗುರು-ಸೇ-ವೆ-ಯಿಂದ
ಗುರು-ಸೇ-ವೆ-ಯಿಂ-ದಲೇ
ಗುರು-ಹಿ-ರಿ-ಯರ
ಗುರು-ಹಿ-ರಿ-ಯ-ರನ್ನು
ಗುರು-ಹಿ-ರಿ-ಯ-ರ-ನ್ನೆಲ್ಲಾ
ಗುರು-ಹಿ-ರಿ-ಯರು
ಗುರೂ-ನ-ಹತ್ವಾ
ಗುರೂ-ನಿ-ಹೈವ
ಗುಲಾಬಿ
ಗುಲಾ-ಬಿಯ
ಗುಲಾ-ಬಿ-ಯನ್ನು
ಗುಲಾ-ಬಿ-ಯಲ್ಲಿ
ಗುಲಾ-ಮ-ನ-ನ್ನಾಗಿ
ಗುಲಾ-ಮ-ನನ್ನು
ಗುಲಾ-ಮ-ನಾಗಿ
ಗುಲಾ-ಮ-ರ-ನ್ನಾಗಿ
ಗುಲಾ-ಮ-ರಾ-ಗಿಯೇ
ಗುಲಾ-ಮ-ರಾಗು
ಗುಲಾ-ಮ-ರಾ-ಗು-ತ್ತೇವೆ
ಗುಲಾ-ಮ-ರಾ-ಗು-ತ್ತೇ-ವೆಯೊ
ಗುಲಾ-ಮ-ರಾ-ಗು-ವೆವು
ಗುಲಾ-ಮರು
ಗುಳಿ-ಗೆ-ಗಳನ್ನು
ಗುಳ್ಳೆ
ಗುಳ್ಳೆ-ಗ-ಳಂತೆ
ಗುಳ್ಳೆ-ಗಳು
ಗುಳ್ಳೆ-ಗಳೇ
ಗುಳ್ಳೆ-ಗ-ಳೊಂ-ದಿಗೆ
ಗುಳ್ಳೆ-ಯಂತೆ
ಗುಳ್ಳೆ-ಯನ್ನು
ಗುಳ್ಳೆ-ಯ-ಲ್ಲೆಲ್ಲ
ಗುವು-ದಕ್ಕೆ
ಗುಹೆ
ಗುಹೆಗೆ
ಗುಹೆ-ಯಲ್ಲಿ
ಗುಹೆ-ಯ-ಲ್ಲಿಯೋ
ಗುಹೆ-ಯ-ಲ್ಲಿ-ರಲಿ
ಗುಹೆ-ಯಿಂದ
ಗುಹ್ಯ
ಗುಹ್ಯಂ
ಗುಹ್ಯ-ಕ್ಕಿಂ-ತಲೂ
ಗುಹ್ಯ-ಗಳಲ್ಲಿ
ಗುಹ್ಯ-ತಮಂ
ಗುಹ್ಯ-ತ-ಮ-ವಾದ
ಗುಹ್ಯ-ಮ-ಧ್ಯಾ-ತ್ಮ-ಸಂ-ಜ್ಞಿ-ತಮ್
ಗುಹ್ಯ-ಮಹಂ
ಗುಹ್ಯ-ವಾ-ಗಿ-ರು-ವುದು
ಗುಹ್ಯ-ವಾದ
ಗುಹ್ಯವೂ
ಗುಹ್ಯಾ-ದ್ಗು-ಹ್ಯ-ತರಂ
ಗುಹ್ಯಾ-ನಾಂ
ಗುಹ್ಯೇಂ-ದ್ರಿ-ಯ-ಗಳು
ಗೂಟ
ಗೂಟಕ್ಕೆ
ಗೂಟಕ್ಕೇ
ಗೂಟ-ಗಳು
ಗೂಟ-ಗಳೇ
ಗೂಟದ
ಗೂಟ-ದಂತೆ
ಗೂಟ-ದಿಂದ
ಗೂಟ-ವನ್ನು
ಗೂಟ-ವಾ-ದರೊ
ಗೂಟವೇ
ಗೂಡನ್ನು
ಗೂಡಾಗಿ
ಗೂಡಿಗೆ
ಗೂಡಿನ
ಗೂಡಿ-ನಲ್ಲಿ
ಗೂಡಿ-ನ-ಲ್ಲಿ-ರು-ವಂತೆ
ಗೂಡಿ-ನ-ಲ್ಲಿ-ರು-ವನು
ಗೂಡಿ-ನಿಂದ
ಗೂಡು
ಗೂಡು-ಗಳಲ್ಲಿ
ಗೂಡೋ
ಗೂಢ-ವಾ-ಗಿ-ರು-ವು-ದ-ಕ್ಕೆಲ್ಲ
ಗೂಬೆ
ಗೃಣಂತಿ
ಗೃಹ
ಗೃಹಂ
ಗೃಹಕ್ಕೆ
ಗೃಹ-ದ-ಲ್ಲಿ-ದ್ದಳು
ಗೃಹ-ಪ್ರ-ವೇ-ಶ-ಮಾಡು
ಗೃಹ-ಸ್ಥನ
ಗೃಹ-ಸ್ಥ-ನಂತೆ
ಗೃಹ-ಸ್ಥರ
ಗೃಹ-ಸ್ಥರು
ಗೃಹ-ಸ್ಥ-ರೆಂದು
ಗೃಹಿಣಿ
ಗೃಹಿಣೀ
ಗೃಹೀ-ತ್ವೈ-ತಾನಿ
ಗೃಹ್ಣ-ನ್ನು-ನ್ಮಿ-ಷ-ನ್ನಿ-ಮಿ-ಷ-ನ್ನಪಿ
ಗೃಹ್ಣಾತಿ
ಗೃಹ್ಯತೇ
ಗೆಂದು
ಗೆಡು-ವು-ದಿಲ್ಲ
ಗೆಡ್ಡೆ
ಗೆದ್ದ-ಮೇಲೆ
ಗೆದ್ದ-ಮೇ-ಲೆಯೂ
ಗೆದ್ದರು
ಗೆದ್ದರೂ
ಗೆದ್ದರೆ
ಗೆದ್ದ-ವನು
ಗೆದ್ದ-ವ-ರಿಲ್ಲ
ಗೆದ್ದಾಗ
ಗೆದ್ದಿ-ರು-ವನು
ಗೆದ್ದಿ-ರು-ವನೊ
ಗೆದ್ದಿ-ರು-ವೆವು
ಗೆದ್ದಿ-ಲ್ಲವೊ
ಗೆದ್ದು
ಗೆಲ-ವು-ಗ-ಳಿ-ಗಿಂತ
ಗೆಲು-ವಿನ
ಗೆಲ್ಲ
ಗೆಲ್ಲದ
ಗೆಲ್ಲದೆ
ಗೆಲ್ಲದೇ
ಗೆಲ್ಲ-ಬೇ-ಕಾ-ಗಿದೆ
ಗೆಲ್ಲ-ಬೇಕು
ಗೆಲ್ಲ-ಬೇ-ಕೆಂ-ಬುದೇ
ಗೆಲ್ಲು
ಗೆಲ್ಲು-ತ್ತಾನೆ
ಗೆಲ್ಲು-ತ್ತಾರೆ
ಗೆಲ್ಲು-ತ್ತಾ-ರೆಯೊ
ಗೆಲ್ಲು-ತ್ತಾ-ರೆಯೋ
ಗೆಲ್ಲು-ತ್ತೇನೆ
ಗೆಲ್ಲು-ತ್ತೇ-ವೆಯೋ
ಗೆಲ್ಲು-ವನು
ಗೆಲ್ಲು-ವ-ವನೇ
ಗೆಲ್ಲು-ವಿರಿ
ಗೆಲ್ಲು-ವು-ದಕ್ಕೆ
ಗೆಲ್ಲು-ವು-ದ-ರಲ್ಲಿ
ಗೆಲ್ಲು-ವುದು
ಗೆಲ್ಲು-ವುದೇ
ಗೆಳೆಯ
ಗೆಳೆ-ಯರು
ಗೇಣುದ್ದ
ಗೇಣು-ದ್ದವೇ
ಗೇಹೇ
ಗೊಂಚ-ಲಿದೆ
ಗೊಂಚಲು
ಗೊಂಡಂ-ತೆಯೆ
ಗೊಂಡ-ವ-ನೆಂದು
ಗೊಂಡಿತ್ತು
ಗೊಂಡಿದ್ದು
ಗೊಂದು
ಗೊಂಬೆ
ಗೊಂಬೆ-ಆ-ಟ-ದಲ್ಲಿ
ಗೊಂಬೆ-ಗಲ್ಲ
ಗೊಂಬೆ-ಗಳ
ಗೊಂಬೆ-ಗಳನ್ನು
ಗೊಂಬೆ-ಗಳಿಂದ
ಗೊಂಬೆ-ಗಳು
ಗೊಂಬೆಯ
ಗೊಂಬೆ-ಯಂತೆ
ಗೊಂಬೆ-ಯಲ್ಲಿ
ಗೊಂಬೆ-ಯಾ-ಗ-ಬೇಕೆ
ಗೊಂಬೆಯೇ
ಗೊಜ್ಜನ್ನು
ಗೊಜ್ಜಿ-ನಂತೆ
ಗೊಡ್ಡು
ಗೊಣ-ಗಾ-ಡದೆ
ಗೊಣ-ಗಾಡಿ
ಗೊಣ-ಗಾ-ಡಿ-ಕೊಂಡು
ಗೊಣ-ಗಾಡು
ಗೊಣ-ಗಾ-ಡುತ್ತ
ಗೊಣ-ಗಾ-ಡು-ತ್ತಲೂ
ಗೊಣ-ಗಾ-ಡುತ್ತಾ
ಗೊಣ-ಗಾ-ಡು-ತ್ತಿ-ದ್ದರೆ
ಗೊಣ-ಗಾ-ಡು-ವನು
ಗೊಣ-ಗಾ-ಡು-ವರು
ಗೊಣ-ಗಾ-ಡು-ವು-ದಿಲ್ಲ
ಗೊಣ-ಗಾ-ಡು-ವುದು
ಗೊಣ-ಗಾ-ಡು-ವುದೂ
ಗೊಣ-ಗಾ-ಡು-ವೆವೋ
ಗೊಣ-ಗು-ತ್ತಿ-ರು-ವನು
ಗೊಣ-ಗು-ವು-ದಿಲ್ಲ
ಗೊತ್ತಾ
ಗೊತ್ತಾಗ
ಗೊತ್ತಾ-ಗ-ದಂತೆ
ಗೊತ್ತಾ-ಗದೆ
ಗೊತ್ತಾ-ಗದೇ
ಗೊತ್ತಾ-ಗ-ಬೇ-ಕಾ-ದರೆ
ಗೊತ್ತಾ-ಗ-ಬೇಕು
ಗೊತ್ತಾ-ಗ-ಲಿಲ್ಲ
ಗೊತ್ತಾ-ಗ-ಲಿ-ಲ್ಲ-ವೆಂದು
ಗೊತ್ತಾಗಿ
ಗೊತ್ತಾ-ಗಿದೆ
ಗೊತ್ತಾ-ಗಿ-ರ-ಬೇಕು
ಗೊತ್ತಾ-ಗಿಲ್ಲ
ಗೊತ್ತಾಗು
ಗೊತ್ತಾ-ಗು-ತ್ತಿದೆ
ಗೊತ್ತಾ-ಗುವ
ಗೊತ್ತಾ-ಗು-ವಂತೆ
ಗೊತ್ತಾ-ಗುವು
ಗೊತ್ತಾ-ಗು-ವು-ದಿಲ್ಲ
ಗೊತ್ತಾ-ಗು-ವು-ದಿ-ಲ್ಲವೋ
ಗೊತ್ತಾ-ಗು-ವುದು
ಗೊತ್ತಾ-ಗು-ವುದೆ
ಗೊತ್ತಾ-ಗು-ವುದೊ
ಗೊತ್ತಾ-ಗು-ವುದೋ
ಗೊತ್ತಾ-ಗು-ವುವು
ಗೊತ್ತಾದ
ಗೊತ್ತಾ-ದದ್ದು
ಗೊತ್ತಾ-ದ-ಮೇಲೆ
ಗೊತ್ತಾ-ದರೂ
ಗೊತ್ತಾ-ದರೆ
ಗೊತ್ತಾ-ದಾಗ
ಗೊತ್ತಾ-ದಾ-ಗಲೆ
ಗೊತ್ತಾ-ದಾ-ಗಲೇ
ಗೊತ್ತಾ-ಯಿತು
ಗೊತ್ತಾ-ಯಿ-ತೆಂದು
ಗೊತ್ತಾ-ವುದು
ಗೊತ್ತಿ
ಗೊತ್ತಿತ್ತು
ಗೊತ್ತಿದೆ
ಗೊತ್ತಿ-ದೆಯೆ
ಗೊತ್ತಿ-ದ್ದರೂ
ಗೊತ್ತಿ-ರ-ಬ-ಹುದು
ಗೊತ್ತಿ-ರ-ಬೇಕು
ಗೊತ್ತಿ-ರ-ಲಿಲ್ಲ
ಗೊತ್ತಿ-ರ-ಲಿ-ಲ್ಲವೆ
ಗೊತ್ತಿರು
ಗೊತ್ತಿ-ರುವ
ಗೊತ್ತಿ-ರುವು
ಗೊತ್ತಿ-ರು-ವುದನ್ನು
ಗೊತ್ತಿ-ರು-ವು-ದ-ನ್ನೆಲ್ಲ
ಗೊತ್ತಿ-ರು-ವು-ದ-ನ್ನೆಲ್ಲಾ
ಗೊತ್ತಿ-ರು-ವು-ದರ
ಗೊತ್ತಿ-ರು-ವು-ದಿಲ್ಲ
ಗೊತ್ತಿ-ರು-ವುದು
ಗೊತ್ತಿ-ರು-ವು-ದೆಲ್ಲ
ಗೊತ್ತಿ-ರು-ವುದೇ
ಗೊತ್ತಿ-ರು-ವು-ದೊಂದೆ
ಗೊತ್ತಿಲ
ಗೊತ್ತಿಲ್ಲ
ಗೊತ್ತಿ-ಲ್ಲದ
ಗೊತ್ತಿ-ಲ್ಲ-ದಾಗ
ಗೊತ್ತಿ-ಲ್ಲದೆ
ಗೊತ್ತಿ-ಲ್ಲದೇ
ಗೊತ್ತಿ-ಲ್ಲವೆ
ಗೊತ್ತಿ-ಲ್ಲ-ವೆಂದು
ಗೊತ್ತಿ-ಲ್ಲವೋ
ಗೊತ್ತಿವೆ
ಗೊತ್ತು
ಗೊತ್ತೆ
ಗೊತ್ತೇ
ಗೊತ್ತೊ
ಗೊಬ್ಬರ
ಗೊಬ್ಬ-ರದ
ಗೊಬ್ಬಳಿ
ಗೊಳಿ-ಸುವ
ಗೊಳಿ-ಸು-ವುದೇ
ಗೋಕುಲ
ಗೋಗ-ರೆ-ಯ-ಬೇಕು
ಗೋಚರ
ಗೋಚ-ರ-ವಾ-ಗು-ವುದು
ಗೋಚ-ರ-ವಾ-ಗು-ವುವು
ಗೋಚ-ರಿ-ಸದ
ಗೋಚ-ರಿ-ಸ-ಬೇ-ಕಾ-ದರೆ
ಗೋಚ-ರಿಸು
ಗೋಚ-ರಿ-ಸು-ತ್ತಾನೆ
ಗೋಚ-ರಿ-ಸು-ತ್ತಿ-ರು-ವನು
ಗೋಚ-ರಿ-ಸು-ವನು
ಗೋಚ-ರಿ-ಸು-ವು-ದಿಲ್ಲ
ಗೋಚ-ರಿ-ಸು-ವುದು
ಗೋಜಿಗೆ
ಗೋಜಿ-ಗೇ-ಳು-ವುದು
ಗೋಜಿ-ನಲ್ಲಿ
ಗೋಜಿ-ನಿಂದ
ಗೋಜು
ಗೋಡೆ
ಗೋಡೆ-ಎಲ್ಲ
ಗೋಡೆಗೆ
ಗೋತ್ರ-ಗ-ಳಿಗೆ
ಗೋದಾನ
ಗೋಪಾ-ಲ-ನಂ-ದನ
ಗೋಪಾ-ಲ-ನಂ-ದನಃ
ಗೋಪಿ-ಯ-ರಿಗೆ
ಗೋಪ್ಯ-ವಾ-ಗಿ-ಡು-ವನು
ಗೋಮು-ಖಾದಿ
ಗೋರ
ಗೋರ-ಕುಂ-ಬಾರ
ಗೋರ-ಕ್ಷಣೆ
ಗೋಲನ್ನು
ಗೋಲಿ-ಯನ್ನೊ
ಗೋಲಿ-ಯನ್ನೋ
ಗೋಲು
ಗೋಳ
ಗೋಳ-ಗಳ
ಗೋಳ-ಗಳನ್ನು
ಗೋಳಾ-ಗಲಿ
ಗೋಳಿ-ಡು-ವರು
ಗೋಳಿ-ಡು-ವು-ದಲ್ಲ
ಗೋಳಿ-ಡು-ವು-ದಿಲ್ಲ
ಗೋಳಿನ
ಗೋಳಿ-ನಿಂದ
ಗೋಳು
ಗೋಳೈ-ಸಿ-ರು-ವೆನೋ
ಗೋವಿಂದ
ಗೋವಿಂ-ದ-ಮುಕ್ತ್ವಾ
ಗೋವಿ-ನಲ್ಲಿ
ಗೋವು
ಗೋವು-ಗಳಲ್ಲಿ
ಗೋವು-ಗಳು
ಗೋಹ-ತ್ಯದ
ಗೌಣ
ಗೌಣಕ್ಕೆ
ಗೌಣ-ದಿಂದ
ಗೌಣ-ಮಾಡಿ
ಗೌಣ-ವ-ಸ್ತು-ಗಳು
ಗೌಣ-ವಾದ
ಗೌಣ-ವಾ-ದುದು
ಗೌಣವೋ
ಗೌರವ
ಗೌರ-ವಕ್ಕೆ
ಗೌರ-ವ-ಗಳನ್ನು
ಗೌರ-ವ-ಗಳಿಂದ
ಗೌರ-ವ-ಗಳು
ಗೌರ-ವದ
ಗೌರ-ವ-ದಿಂದ
ಗೌರ-ವ-ಬುದ್ಧಿ
ಗೌರ-ವ-ವನ್ನು
ಗೌರ-ವ-ವಿ-ದ್ದರೂ
ಗೌರ-ವ-ವಿ-ರುವ
ಗೌರ-ವ-ವೆಲ್ಲ
ಗೌರ-ವಸ್ಥ
ಗೌರ-ವ-ಸ್ಥ-ನಿಗೆ
ಗೌರ-ವ-ಸ್ಥ-ರಿಲ್ಲ
ಗೌರ-ವಾರ್ಹ
ಗೌರ-ವಿ-ಸದೇ
ಗೌರ-ವಿ-ಸ-ಬೇಕು
ಗೌರ-ವಿಸಿ
ಗೌರ-ವಿ-ಸಿ-ದಾಗ
ಗೌರ-ವಿಸು
ಗೌರ-ವಿ-ಸು-ತ್ತಿ-ದ್ದರು
ಗೌರ-ವಿ-ಸು-ತ್ತಿ-ದ್ದರೆ
ಗೌರ-ವಿ-ಸು-ವು-ದಕ್ಕೆ
ಗೌರ-ವಿ-ಸು-ವು-ದಿಲ್ಲ
ಗೌರ-ವಿ-ಸು-ವುದು
ಗೌರಿ-ಶಂ-ಕರ
ಗೌರೀ-ಶಂ-ಕರ
ಗ್ಯಾಲನ್ನು
ಗ್ಯಾಲಿ-ಲಿಯೊ
ಗ್ಯಾಸು-ಗಳನ್ನು
ಗ್ರಂಥ
ಗ್ರಂಥ-ಗಳ
ಗ್ರಂಥ-ಗಳನ್ನು
ಗ್ರಂಥ-ಗ-ಳಿವೆ
ಗ್ರಂಥದ
ಗ್ರಂಥ-ರೂ-ಪಕ್ಕೆ
ಗ್ರಂಥ-ವಾ-ಯಿತು
ಗ್ರಂಥವೂ
ಗ್ರಂಥಿ-ಗಳು
ಗ್ರಂಥಿ-ಗ-ಳೆಲ್ಲ
ಗ್ರಥಿ-ತಾಂ
ಗ್ರಸ-ಮಾನಃ
ಗ್ರಸಿಷ್ಣು
ಗ್ರಹ
ಗ್ರಹಕ್ಕೆ
ಗ್ರಹ-ಗಳಲ್ಲಿ
ಗ್ರಹ-ಗ-ಳಿ-ಗಾಗಿ
ಗ್ರಹ-ಗ-ಳಿ-ರು-ವುವು
ಗ್ರಹ-ಗ-ಳಿವೆ
ಗ್ರಹ-ಗಳು
ಗ್ರಹ-ಗ-ಳೆಲ್ಲಾ
ಗ್ರಹದ
ಗ್ರಹ-ದಲ್ಲಿ
ಗ್ರಹ-ಬಲ
ಗ್ರಹ-ವನ್ನು
ಗ್ರಹ-ವಾಗ
ಗ್ರಹ-ವಾಗಿ
ಗ್ರಹ-ವಾ-ಗಿ-ಬಿ-ಡು-ವುದು
ಗ್ರಹಾ-ದಿ-ಗ-ಳಿಗೆ
ಗ್ರಹಿಕೆ
ಗ್ರಹಿ-ಕೆಗೆ
ಗ್ರಹಿ-ಸದು
ಗ್ರಹಿ-ಸ-ಬಲ್ಲ
ಗ್ರಹಿ-ಸ-ಬ-ಲ್ಲುದು
ಗ್ರಹಿ-ಸ-ಬ-ಹುದು
ಗ್ರಹಿ-ಸ-ಬ-ಹುದೋ
ಗ್ರಹಿ-ಸ-ಬೇ-ಕಾ-ಗು-ವುದು
ಗ್ರಹಿ-ಸ-ಲಾರ
ಗ್ರಹಿ-ಸ-ಲಾ-ರೆವು
ಗ್ರಹಿ-ಸ-ಲಿಲ್ಲ
ಗ್ರಹಿ-ಸಲು
ಗ್ರಹಿ-ಸಿ-ಕೊ-ಳ್ಳು-ವು-ದಕ್ಕೆ
ಗ್ರಹಿ-ಸಿ-ರು-ವ-ವನು
ಗ್ರಹಿ-ಸು-ತ್ತಾನೆ
ಗ್ರಹಿ-ಸುವ
ಗ್ರಹಿ-ಸು-ವಂತೆ
ಗ್ರಹಿ-ಸು-ವನು
ಗ್ರಹಿ-ಸು-ವು-ದಕ್ಕೆ
ಗ್ರಹಿ-ಸು-ವುದು
ಗ್ರಾಮ-ಫೋನ್
ಗ್ರಾಸ-ವಾಗಿ
ಗ್ರೀಕರ
ಗ್ರೀಕ್
ಗ್ರೀಸಿ-ಲ್ಲದೆ
ಗ್ಲಾನಿ-ಯಾ-ಗು-ವುದೋ
ಗ್ಲಾನಿ-ರ್ಭ-ವತಿ
ಗ್ಲಾಸನ್ನು
ಗ್ಲಾಸಿನ
ಘಟ-ನಾ-ವ-ಳಿ-ಗಳ
ಘಟ-ನಾ-ವ-ಳಿ-ಗಳನ್ನೆಲ್ಲಾ
ಘಟ-ನಾ-ವ-ಳಿ-ಗಳು
ಘಟನೆ
ಘಟ-ನೆ-ಗಳ
ಘಟ-ನೆ-ಗಳನ್ನು
ಘಟ-ನೆ-ಗಳಿಂದ
ಘಟ-ನೆ-ಗ-ಳಿಗೆ
ಘಟ-ನೆ-ಗಳು
ಘಟ-ನೆ-ಗಳೂ
ಘಟ-ನೆಯ
ಘಟ-ನೆ-ಯಂತೆ
ಘಟ-ನೆ-ಯಂ-ತೆಯೋ
ಘಟ-ನೆ-ಯನ್ನು
ಘಟ-ನೆ-ಯ-ನ್ನು-ತೆ-ಗೆ-ದು-ಕೊಂ-ಡರೂ
ಘಟ-ನೆ-ಯ-ಲ್ಲಿಯೂ
ಘಟ-ನೆ-ಯಾಗಿ
ಘಟ-ನೆಯೂ
ಘಟ-ನೆ-ಯೆಂದು
ಘಟ-ನೆಯೇ
ಘಟ-ಸರ್ಪ
ಘಟ-ಸ-ರ್ಪ-ಗ-ಳಾ-ಗು-ತ್ತವೆ
ಘಟಾ-ಕಾ-ಶ-ವಾ-ಗು-ವುದು
ಘಟಾ-ನು-ಘಟಿ
ಘಟ್ಟ-ದಲ್ಲಿ
ಘಟ್ಟ-ದ-ಲ್ಲಿ-ರು-ವನು
ಘಟ್ಟ-ದಿಂದ
ಘಟ್ಟಿ-ಯಾ-ಗಿದೆ
ಘಟ್ಟಿ-ಯಾದ
ಘನ
ಘನ-ರೂ-ಪ-ದಲ್ಲಿ
ಘನೀ-ಭೂ-ತ-ವಾಗಿ
ಘನೀ-ಭೂ-ತ-ವಾದ
ಘನೀ-ರೂಪ
ಘಮ-ಘ-ಮಿ-ಸು-ತ್ತಿ-ರು-ವುವೋ
ಘರ್ಷಣೆ
ಘರ್ಷ-ಣೆ-ಯಾ-ಗ-ದಂತೆ
ಘರ್ಷ-ಣೆಯೂ
ಘಾಟಿ-ನಂತೆ
ಘಾಟಿ-ನಲ್ಲಿ
ಘಾಟು
ಘಾಟು-ಗ-ಳಂತೆ
ಘಾತ-ಯತಿ
ಘಾತು-ಕ-ಗಳು
ಘಾತು-ಕ-ನಾ-ಗುವ
ಘಾತು-ಕ-ರಾ-ಗು-ವರು
ಘಾಸಿ-ಪ-ಟ್ಟಿ-ರು-ವ-ವರು
ಘೋರ
ಘೋರಂ
ಘೋರ-ಮೀ-ದೃ-ಙ್ಮ-ಮೇ-ದಮ್
ಘೋರ-ವಾ-ಗಿರು
ಘೋರ-ವಾದ
ಘೋರೇ
ಘೋಷರು
ಘೋಷೋ
ಘೋಷ್
ಘ್ನತೋಪಿ
ಘ್ರಾಣ-ಮೇವ
ಚ
ಚಂಚಲ
ಚಂಚಲಂ
ಚಂಚ-ಲ-ಗೊ-ಳಿ-ಸು-ವು-ದಿಲ್ಲ
ಚಂಚ-ಲ-ಚಿ-ತ್ತ-ನಾ-ಗ-ಕೂ-ಡದು
ಚಂಚ-ಲತೆ
ಚಂಚ-ಲ-ತೆಯೇ
ಚಂಚ-ಲ-ತ್ವಾತ್
ಚಂಚ-ಲ-ವಾ-ಗದೆ
ಚಂಚ-ಲ-ವಾ-ಗಿತ್ತು
ಚಂಚ-ಲ-ವಾ-ಗಿದೆ
ಚಂಚ-ಲ-ವಾ-ಗಿ-ರುವು
ಚಂಚ-ಲ-ವಾ-ಗಿ-ರು-ವು-ದ-ರಿಂದ
ಚಂಚ-ಲ-ವಾ-ಗು-ವು-ದಿಲ್ಲ
ಚಂಚ-ಲ-ವಾ-ಗು-ವುದು
ಚಂಚ-ಲ-ವಾ-ದದ್ದು
ಚಂಚ-ಲ-ವಾ-ದರೂ
ಚಂಚ-ಲವೂ
ಚಂಡನ್ನು
ಚಂಡ-ಮಾ-ರುಡ
ಚಂಡ-ಮಾ-ರುತ
ಚಂಡ-ಮಾ-ರು-ತ-ಗ-ಳೇಳು
ಚಂಡ-ಮಾ-ರು-ತ-ದಂತೆ
ಚಂಡಾಲ
ಚಂಡಾ-ಲನ
ಚಂದ-ನದ
ಚಂದ್ರ
ಚಂದ್ರನ
ಚಂದ್ರ-ನದು
ಚಂದ್ರ-ನನ್ನು
ಚಂದ್ರ-ನಲ್ಲಿ
ಚಂದ್ರ-ನ-ಲ್ಲಿ-ರುವ
ಚಂದ್ರ-ನಾ-ಗಲಿ
ಚಂದ್ರ-ನಾಗಿ
ಚಂದ್ರ-ನಿ-ಗಿ-ರು-ವಂತೆ
ಚಂದ್ರ-ನಿಗೂ
ಚಂದ್ರ-ನಿಗೆ
ಚಂದ್ರ-ಮನ
ಚಂದ್ರರ
ಚಂದ್ರ-ರಲ್ಲಿ
ಚಂದ್ರ-ರ-ಲ್ಲಿ-ರುವ
ಚಂದ್ರ-ರಾ-ಗಲಿ
ಚಂದ್ರರು
ಚಂದ್ರ-ಲೋಕ
ಚಂದ್ರ-ಲೋ-ಕಕ್ಕೆ
ಚಂಪಕ
ಚಕಿ-ತ-ನಾ-ಗು-ವು-ದಿಲ್ಲ
ಚಕ್ಕ-ಳ-ಗು-ಳಿಗೆ
ಚಕ್ಕ-ಳ-ಗು-ಳಿ-ಯನ್ನು
ಚಕ್ಕ-ಳು-ಗು-ಳಿ-ಯನ್ನು
ಚಕ್ಕು-ಳ-ಗು-ಳಿ-ಯನ್ನು
ಚಕ್ರ
ಚಕ್ರಂ
ಚಕ್ರಕ್ಕೆ
ಚಕ್ರ-ಗ-ಳಂತೆ
ಚಕ್ರ-ಗ-ಳಿಗೆ
ಚಕ್ರ-ಗಳು
ಚಕ್ರ-ಗ-ಳೆಲ್ಲ
ಚಕ್ರ-ಗ-ಳೆಲ್ಲಾ
ಚಕ್ರದ
ಚಕ್ರ-ದಂತೆ
ಚಕ್ರ-ದಲ್ಲಿ
ಚಕ್ರ-ದಿಂದ
ಚಕ್ರ-ಪಾ-ಣಿಯೂ
ಚಕ್ರ-ಬಡ್ಡಿ
ಚಕ್ರ-ರ-ಕ್ಷ-ಕ-ರಾ-ಗಿ-ದ್ದರು
ಚಕ್ರ-ವನ್ನು
ಚಕ್ರ-ವೆಲ್ಲ
ಚಕ್ರ-ವ್ಯೂಹ
ಚಕ್ರ-ಹಸ್ತ
ಚಕ್ರಾಧಿ
ಚಕ್ರಿಣಂ
ಚಕ್ಷುಃ
ಚಕ್ಷು-ವನ್ನೂ
ಚಕ್ಷು-ಸ್ಸನ್ನು
ಚಕ್ಷು-ಸ್ಸನ್ನೂ
ಚಕ್ಷುಸ್ಸು
ಚಚ್ಚಿ-ಕೊ-ಳ್ಳು-ತ್ತೇವೆ
ಚಟಾ-ಕಿನ
ಚಟಾ-ಕಿ-ನಂತೆ
ಚಟಾಕು
ಚಟು-ವ-ಟಿ-ಕೆ-ಗಳನ್ನು
ಚಟು-ವ-ಟಿ-ಕೆಗೆ
ಚಟು-ವ-ಟಿ-ಕೆಯ
ಚಟು-ವ-ಟಿ-ಕೆ-ಯಿಂದ
ಚತು-ರೋ-ಪಾ-ಯ-ಗಳ
ಚತು-ರೋ-ಪಾ-ಯ-ಗಳನ್ನು
ಚತು-ರೋ-ಪಾ-ಯ-ಗಳನ್ನೆಲ್ಲ
ಚತು-ರೋ-ಪಾ-ಯ-ಗಳಲ್ಲಿ
ಚತು-ರ್ಭು-ಜ-ನಾದ
ಚತು-ರ್ಭು-ಜ-ವುಳ್ಳ
ಚತು-ರ್ಭು-ಜವೇ
ಚತು-ರ್ಭು-ಜೇನ
ಚತು-ರ್ವರ್ಣ
ಚತು-ರ್ವಿಧಂ
ಚತು-ರ್ವಿಧಾ
ಚತು-ಷ್ಟಯ
ಚತು-ಷ್ಟ-ಯದ
ಚತು-ಷ್ಪಾ-ದಿ-ಗಳನ್ನು
ಚತ್ವಾರೋ
ಚದು-ರ-ದಂತೆ
ಚದುರಿ
ಚದು-ರಿ-ರು-ವು-ದ-ನ್ನೆಲ್ಲಾ
ಚದು-ರಿ-ಹೋ-ಗು-ವುವು
ಚನ್ನಾಗಿ
ಚಪಲ
ಚಪ-ಲಕ್ಕೆ
ಚಪ-ಲ-ಗಳು
ಚಪ-ಲ-ತೆಗೆ
ಚಪ-ಲ-ವನ್ನು
ಚಪ-ಲ-ವಿಲ್ಲ
ಚಪ-ಲ-ವಿ-ಲ್ಲದೆ
ಚಪ್ಚರಿ
ಚಪ್ಪ-ರವೊ
ಚಪ್ಪ-ರಿಸಿ
ಚಪ್ಪ-ರಿ-ಸುತ್ತ
ಚಪ್ಪ-ರಿ-ಸು-ತ್ತಿ-ದ್ದರೆ
ಚಪ್ಪ-ರಿ-ಸು-ತ್ತಿ-ರು-ವರು
ಚಪ್ಪ-ರಿ-ಸು-ತ್ತಿ-ರು-ವುದು
ಚಪ್ಪ-ರಿ-ಸು-ವನು
ಚಪ್ಪ-ರಿ-ಸು-ವಾಗ
ಚಪ್ಪ-ರಿ-ಸು-ವುದೂ
ಚಪ್ಪಾಳೆ
ಚಮ-ತ್ಕಾ-ರ-ಗಳನ್ನು
ಚಮ-ತ್ಕಾ-ರ-ವಾದ
ಚಮೂಮ್
ಚರ
ಚರಂ-ಡಿಯ
ಚರಂ-ಡಿ-ಯ-ಮೇಲೆ
ಚರಂ-ಡಿ-ಯಲ್ಲಿ
ಚರಂ-ಡಿ-ಯ-ಲ್ಲಿ-ರುವ
ಚರಂತಿ
ಚರ-ತಾಂ
ಚರತಿ
ಚರಮ
ಚರ-ಮ-ಗುರಿ
ಚರ-ಮೇವ
ಚರಾ-ಚರ
ಚರಾ-ಚ-ರಮ್
ಚರಾ-ಚ-ರ-ವಾದ
ಚರಾ-ಚ-ರಸ್ಯ
ಚರಾ-ಚ-ರಾ-ತ್ಮ-ಕ-ವಾದ
ಚರಿತ್ರೆ
ಚರಿ-ತ್ರೆ-ಗಿಂತ
ಚರಿ-ತ್ರೆಯ
ಚರ್ಚಿ-ನಂತೆ
ಚರ್ಚಿ-ಸು-ವ-ವನು
ಚರ್ಚೆ
ಚರ್ಚೆಗೆ
ಚರ್ಚೆ-ಮಾ-ಡು-ವನು
ಚರ್ಚೆ-ಮಾ-ಡು-ವ-ವ-ರಲ್ಲಿ
ಚರ್ಚೆ-ಯಲ್ಲಿ
ಚರ್ಮ
ಚರ್ಮದ
ಚರ್ಮ-ವಿದೆ
ಚಲ
ಚಲ-ನ-ವ-ಲ-ನ-ಗಳ
ಚಲ-ನ-ವ-ಲ-ನ-ಗಳನ್ನು
ಚಲ-ನ-ವ-ಲ-ನ-ಗ-ಳೆಲ್ಲ
ಚಲ-ನ-ವ-ಲ-ನ-ಗ-ಳೆಲ್ಲಾ
ಚಲ-ನ-ವ-ಲ-ನದ
ಚಲನೆ
ಚಲ-ಮ-ಧ್ರು-ವಮ್
ಚಲಮ್
ಚಲಾ-ಯಿಸಿ
ಚಲಾ-ಯಿ-ಸು-ತ್ತಾನೆ
ಚಲಾ-ಯಿ-ಸು-ತ್ತಿ-ರು-ವನು
ಚಲಾ-ವ-ಣೆ-ಯ-ಲ್ಲಿ-ವೆಯೋ
ಚಲಿ-ತ-ನಾ-ಗು-ವು-ದಿಲ್ಲ
ಚಲಿಸ
ಚಲಿ-ಸ-ದ-ವನು
ಚಲಿ-ಸ-ದಿ-ರುವ
ಚಲಿ-ಸದೆ
ಚಲಿ-ಸ-ಲಾ-ರದು
ಚಲಿ-ಸ-ಲಾ-ರರು
ಚಲಿ-ಸಿ-ಕೊಂಡು
ಚಲಿ-ಸಿ-ದರೇ
ಚಲಿಸು
ಚಲಿ-ಸುತ್ತ
ಚಲಿ-ಸು-ತ್ತದೆ
ಚಲಿ-ಸು-ತ್ತಿದೆ
ಚಲಿ-ಸು-ತ್ತಿ-ದ್ದರೂ
ಚಲಿ-ಸು-ತ್ತಿ-ರು-ವನೊ
ಚಲಿ-ಸು-ತ್ತಿ-ರು-ವುದು
ಚಲಿ-ಸು-ತ್ತಿಲ್ಲ
ಚಲಿ-ಸು-ತ್ತಿವೆ
ಚಲಿ-ಸುವ
ಚಲಿ-ಸು-ವಂತೆ
ಚಲಿ-ಸು-ವಾಗ
ಚಲಿ-ಸು-ವು-ದ-ಕ್ಕೆಲ್ಲ
ಚಲಿ-ಸು-ವುದನ್ನು
ಚಲಿ-ಸು-ವು-ದಾ-ದರೂ
ಚಲಿ-ಸು-ವು-ದಿಲ್ಲ
ಚಲಿ-ಸು-ವುದು
ಚಲ್ಲಾ-ಪಿ-ಲ್ಲಿಯೊ
ಚಳಿ
ಚಳಿ-ಗಾಲ
ಚಳಿ-ಗಾ-ಲ-ದಲ್ಲಿ
ಚಳಿ-ಗಾ-ಲ-ವೆ-ನ್ನದೆ
ಚಳಿ-ಯಿಂದ
ಚಾಂಚಲ್ಯ
ಚಾಂಚ-ಲ್ಯಕ್ಕೆ
ಚಾಂಚ-ಲ್ಯ-ವಿಲ್ಲ
ಚಾಂಚ-ಲ್ಯವೇ
ಚಾಂತಿಕೇ
ಚಾಂದ್ರ-ಮಸಂ
ಚಾಕಾ-ರ್ಯ-ಮೇವ
ಚಾಕೀ-ತಿ-ರ್ಮ-ರ-ಣಾ-ದ-ತಿ-ರಿ-ಚ್ಯತೇ
ಚಾಕ್ರಿಯಃ
ಚಾಖಿಲಂ
ಚಾಚಾರೋ
ಚಾಚಿ
ಚಾಚು-ವುದು
ಚಾಡಿ
ಚಾಡಿ-ಯನ್ನು
ಚಾಣೂ-ರ-ರನ್ನು
ಚಾಣೂ-ರರು
ಚಾತಿ-ತ-ರಂ-ತ್ಯೇವ
ಚಾತಿ-ಸ್ವ-ಪ್ನ-ಶೀ-ಲಸ್ಯ
ಚಾತು-ರ್ಯ-ವನ್ನು
ಚಾತು-ರ್ವರ್ಣ್ಯ
ಚಾತು-ರ್ವರ್ಣ್ಯಂ
ಚಾತ್ಮನಿ
ಚಾದಿರ್ನ
ಚಾದ್ಯಂ
ಚಾದ್ಯಮ್
ಚಾಧಿ-ಪ-ತ್ಯಮ್
ಚಾನಘ
ಚಾನ-ನ್ಯ-ಯೋ-ಗೇನ
ಚಾನು-ಬಂಧೇ
ಚಾನ್ಯಃ
ಚಾಪಂ
ಚಾಪರಂ
ಚಾಪ-ರಾ-ನಪಿ
ಚಾಪರೇ
ಚಾಪ-ಲ್ಯ-ಗಳನ್ನು
ಚಾಪಿ
ಚಾಪೆ
ಚಾಪೆಯ
ಚಾಪೆ-ಯಂತೆ
ಚಾಪೆ-ಯನ್ನು
ಚಾಪ್ಯ-ಕ್ಷ-ರ-ಮ-ವ್ಯಕ್ತಂ
ಚಾಪ್ಯನ್ಯೇ
ಚಾಪ್ಯ-ಪ-ಲಾ-ಯ-ನಮ್
ಚಾಪ್ಯ-ಹಮ್
ಚಾಪ್ಯುಕ್ತೋ
ಚಾಪ್ರಿಯಂ
ಚಾಭ-ಯ-ಮೇವ
ಚಾಭಾ-ವ-ಯತಃ
ಚಾಭಿ-ಜಾ-ತಸ್ಯ
ಚಾಮರ
ಚಾಮುಂಡಿ
ಚಾಮುಂ-ಡಿ-ಬೆ-ಟ್ಟದ
ಚಾಮೇಧ್ಯಂ
ಚಾಯಂ
ಚಾಯು-ಕ್ತಸ್ಯ
ಚಾರಿ
ಚಾರಿ-ತ್ರ-ಕ-ವಾ-ಗಿ-ದ್ದನೆ
ಚಾರಿ-ತ್ರದ
ಚಾರಿ-ತ್ರ-ವಿ-ಲ್ಲವೊ
ಚಾರಿ-ತ್ರಿಕ
ಚಾರಿ-ತ್ರಿ-ಕತೆ
ಚಾರಿ-ತ್ರಿ-ಕ-ವಾಗಿ
ಚಾರಿ-ತ್ರ್ಯ-ವನ್ನು
ಚಾರಿ-ತ್ರ್ಯವೂ
ಚಾರಿ-ಯಾಗು
ಚಾರು-ಚೂರು
ಚಾರ್ಜುನ
ಚಾರ್ವಾಕ
ಚಾರ್ವಾ-ಕ-ನಾ-ಗಲಿ
ಚಾರ್ವಾ-ಕ-ನಾ-ಗಿ-ರ-ಬ-ಹುದು
ಚಾರ್ವಾ-ಕ-ನಿಗೆ
ಚಾರ್ವಾ-ಕ-ರಲ್ಲ
ಚಾರ್ವಾ-ಕರು
ಚಾವಟಿ
ಚಾವ-ಟಿಯ
ಚಾವುಟಿ
ಚಾವೇಕ್ಷ್ಯ
ಚಾವ್ಯ-ಯಮ್
ಚಾಶು-ಶ್ರೂ-ಷವೇ
ಚಾಸ್ಮಿ
ಚಾಸ್ಯ
ಚಾಹಂ
ಚಾಹಮ್
ಚಿಂತನೆ
ಚಿಂತ-ನೆಗೆ
ಚಿಂತ-ನೆಯ
ಚಿಂತ-ನೆ-ಯನ್ನು
ಚಿಂತ-ನೆ-ಯಿಂದ
ಚಿಂತ-ನೆಯೇ
ಚಿಂತ-ಯೇತ್
ಚಿಂತಾ-ಮ-ಪ-ರಿ-ಮೇ-ಯಾಂ
ಚಿಂತಿಸ
ಚಿಂತಿ-ಸ-ಕೂ-ಡದು
ಚಿಂತಿ-ಸ-ತೊ-ಡಗಿ
ಚಿಂತಿ-ಸ-ದಂತೆ
ಚಿಂತಿ-ಸದೆ
ಚಿಂತಿ-ಸದೇ
ಚಿಂತಿ-ಸ-ಬ-ಹುದು
ಚಿಂತಿ-ಸ-ಬೇ-ಕಾ-ಗಿಲ್ಲ
ಚಿಂತಿ-ಸ-ಬೇ-ಕಾ-ಗು-ವುದು
ಚಿಂತಿ-ಸ-ಬೇ-ಕಾ-ದರೂ
ಚಿಂತಿ-ಸ-ಬೇ-ಕಾ-ದರೆ
ಚಿಂತಿ-ಸ-ಬೇಕು
ಚಿಂತಿ-ಸ-ಲಾ-ರದು
ಚಿಂತಿ-ಸ-ಲಾ-ರೆವು
ಚಿಂತಿ-ಸಲು
ಚಿಂತಿಸಿ
ಚಿಂತಿ-ಸಿ-ಕೊ-ಳ್ಳ-ಬೇ-ಕೆ-ನ್ನು-ವುದು
ಚಿಂತಿ-ಸಿ-ದರೂ
ಚಿಂತಿ-ಸಿ-ದರೆ
ಚಿಂತಿ-ಸಿ-ದ-ರೇನೇ
ಚಿಂತಿ-ಸಿ-ದೊ-ಡನೆ
ಚಿಂತಿಸು
ಚಿಂತಿ-ಸುತ್ತ
ಚಿಂತಿ-ಸುತ್ತಾ
ಚಿಂತಿ-ಸು-ತ್ತಾನೆ
ಚಿಂತಿ-ಸು-ತ್ತಾ-ರೆಯೊ
ಚಿಂತಿ-ಸು-ತ್ತಿ-ದ್ದರೂ
ಚಿಂತಿ-ಸು-ತ್ತಿ-ದ್ದರೆ
ಚಿಂತಿ-ಸು-ತ್ತಿದ್ದು
ಚಿಂತಿ-ಸು-ತ್ತಿ-ರ-ಬೇಕು
ಚಿಂತಿ-ಸು-ತ್ತಿರು
ಚಿಂತಿ-ಸು-ತ್ತಿ-ರು-ತ್ತದೆ
ಚಿಂತಿ-ಸು-ತ್ತಿ-ರು-ತ್ತೇವೆ
ಚಿಂತಿ-ಸು-ತ್ತಿ-ರುವ
ಚಿಂತಿ-ಸು-ತ್ತಿ-ರು-ವನು
ಚಿಂತಿ-ಸು-ತ್ತಿ-ರು-ವನೆ
ಚಿಂತಿ-ಸು-ತ್ತಿ-ರು-ವನೊ
ಚಿಂತಿ-ಸು-ತ್ತಿ-ರು-ವನೋ
ಚಿಂತಿ-ಸು-ತ್ತಿ-ರು-ವ-ವನ
ಚಿಂತಿ-ಸು-ತ್ತಿ-ರು-ವ-ವರು
ಚಿಂತಿ-ಸು-ತ್ತಿ-ರು-ವಾಗ
ಚಿಂತಿ-ಸು-ತ್ತಿ-ರು-ವುದು
ಚಿಂತಿ-ಸು-ತ್ತಿ-ರು-ವೆವು
ಚಿಂತಿ-ಸು-ತ್ತಿ-ರು-ವೆವೋ
ಚಿಂತಿ-ಸು-ತ್ತೇವೆ
ಚಿಂತಿ-ಸು-ತ್ತೇ-ವೆಯೋ
ಚಿಂತಿ-ಸುವ
ಚಿಂತಿ-ಸು-ವಂತೆ
ಚಿಂತಿ-ಸು-ವನು
ಚಿಂತಿ-ಸು-ವ-ವನು
ಚಿಂತಿ-ಸು-ವಾಗ
ಚಿಂತಿ-ಸು-ವು-ದ-ಕ್ಕಾ-ಗು-ವು-ದಿಲ್ಲ
ಚಿಂತಿ-ಸು-ವು-ದಕ್ಕೆ
ಚಿಂತಿ-ಸು-ವುದನ್ನು
ಚಿಂತಿ-ಸು-ವು-ದ-ರಲ್ಲಿ
ಚಿಂತಿ-ಸು-ವು-ದ-ರ-ಲ್ಲಿಯೂ
ಚಿಂತಿ-ಸು-ವು-ದಾ-ಗಲೀ
ಚಿಂತಿ-ಸು-ವು-ದಿಲ್ಲ
ಚಿಂತಿ-ಸು-ವು-ದಿ-ಲ್ಲವೊ
ಚಿಂತಿ-ಸು-ವುದು
ಚಿಂತಿ-ಸು-ವುದೂ
ಚಿಂತಿ-ಸು-ವುದೇ
ಚಿಂತೆ
ಚಿಂತೆ-ಅದು
ಚಿಂತೆ-ಗಳಿಂದ
ಚಿಂತೆ-ಗಳು
ಚಿಂತೆಗೆ
ಚಿಂತೆ-ಯನ್ನು
ಚಿಂತೆ-ಯ-ಲ್ಲಿಯೇ
ಚಿಂತೆ-ಯಿಲ್ಲ
ಚಿಂತೆ-ಯಿ-ಲ್ಲ-ಮೊ-ದಲು
ಚಿಂತ್ಯೋಽ
ಚಿಕಿತ್ಸೆ
ಚಿಕಿ-ತ್ಸೆಯ
ಚಿಕಿ-ತ್ಸೆ-ಯನ್ನು
ಚಿಕ್ಕ
ಚಿಕ್ಕಪ್ಪ
ಚಿಗು-ರ-ಲಾ-ರದು
ಚಿಗು-ರಲು
ಚಿಗುರಿ
ಚಿಗು-ರಿದ
ಚಿಗು-ರಿ-ನಿಂದ
ಚಿಗು-ರು-ಗಳು
ಚಿಗು-ರು-ಗ-ಳುಳ್ಳ
ಚಿಗು-ರು-ಗ-ಳೆಲ್ಲ
ಚಿಗು-ರು-ತ್ತ-ದೆಯೋ
ಚಿಗು-ರು-ತ್ತಿ-ರು-ವುದನ್ನು
ಚಿಗು-ರುವ
ಚಿಗು-ರು-ವಂತೆ
ಚಿಗು-ರು-ವು-ದಿಲ್ಲ
ಚಿಗು-ರು-ವು-ದಿ-ಲ್ಲವೋ
ಚಿಗು-ರು-ವುದು
ಚಿಗು-ರು-ವುವು
ಚಿಟ್ಟೆ
ಚಿಟ್ಟೆಯ
ಚಿಟ್ಟೆ-ಯನ್ನು
ಚಿಟ್ಟೆಯೊ
ಚಿಣ್ಣಿ-ಕೋಲೊ
ಚಿತಾ-ವಣೆ
ಚಿತ್ತ
ಚಿತ್ತಂ
ಚಿತ್ತಕ್ಕೆ
ಚಿತ್ತ-ಚಾಂ-ಚಲ್ಯ
ಚಿತ್ತ-ಚಾಂ-ಚ-ಲ್ಯ-ದೊಂ-ದಿಗೆ
ಚಿತ್ತ-ಚಾಂ-ಚ-ಲ್ಯ-ವನ್ನು
ಚಿತ್ತ-ಚಾಂ-ಚ-ಲ್ಯ-ವಿಲ್ಲ
ಚಿತ್ತದ
ಚಿತ್ತ-ದಲ್ಲಿ
ಚಿತ್ತ-ದ-ಲ್ಲಿದೆ
ಚಿತ್ತ-ದಿಂದ
ಚಿತ್ತ-ನಾಗಿ
ಚಿತ್ತ-ನಾ-ಗಿ-ರ-ಬೇಕು
ಚಿತ್ತ-ನಾ-ಗಿ-ರು-ವನು
ಚಿತ್ತ-ನಿಗೆ
ಚಿತ್ತ-ಮಾ-ತ್ಮ-ನ್ಯೇ-ವಾ-ವ-ತಿ-ಷ್ಠತೇ
ಚಿತ್ತ-ವನ್ನು
ಚಿತ್ತವಾ
ಚಿತ್ತ-ವಿ-ಲ್ಲದೆ
ಚಿತ್ತ-ವು-ಳ್ಳ-ವ-ನಾಗಿ
ಚಿತ್ತ-ವು-ಳ್ಳ-ವ-ನಾ-ಗಿ-ರ-ಬೇಕು
ಚಿತ್ತ-ವು-ಳ್ಳ-ವ-ನಾ-ಗಿರು
ಚಿತ್ತ-ವು-ಳ್ಳ-ವನು
ಚಿತ್ತ-ವು-ಳ್ಳ-ವರು
ಚಿತ್ತ-ವೃ-ತ್ತಿ-ಯನ್ನು
ಚಿತ್ತ-ವೃ-ತ್ತಿ-ಯಿಂ-ದಲೇ
ಚಿತ್ತ-ಶು-ದ್ಧ-ವಾ-ಗಿ-ದ್ದರೆ
ಚಿತ್ತ-ಶು-ದ್ಧ-ವಾ-ಗು-ವುದೋ
ಚಿತ್ತ-ಶು-ದ್ಧ-ವಾ-ದ-ವನ
ಚಿತ್ತ-ಶು-ದ್ಧ-ವಿ-ದ್ದರೆ
ಚಿತ್ತ-ಶುದ್ಧಿ
ಚಿತ್ತ-ಶು-ದ್ಧಿ-ಗಾಗಿ
ಚಿತ್ತ-ಶು-ದ್ಧಿಗೆ
ಚಿತ್ತ-ಶು-ದ್ಧಿಯ
ಚಿತ್ತ-ಶು-ದ್ಧಿ-ಯನ್ನು
ಚಿತ್ತ-ಶು-ದ್ಧಿ-ಯನ್ನೇ
ಚಿತ್ತ-ಶು-ದ್ಧಿ-ಯಾ-ಗದೆ
ಚಿತ್ತ-ಶು-ದ್ಧಿ-ಯಾ-ಗಲಿ
ಚಿತ್ತ-ಶು-ದ್ಧಿ-ಯಾಗಿ
ಚಿತ್ತ-ಶು-ದ್ಧಿ-ಯಾ-ಗಿ-ದೆಯೆ
ಚಿತ್ತ-ಶು-ದ್ಧಿ-ಯಾ-ಗಿ-ರ-ಬೇಕು
ಚಿತ್ತ-ಶು-ದ್ಧಿ-ಯಾ-ಗು-ವು-ದಕ್ಕೆ
ಚಿತ್ತ-ಶು-ದ್ಧಿ-ಯಾ-ಗು-ವು-ದಿಲ್ಲ
ಚಿತ್ತ-ಶು-ದ್ಧಿ-ಯಾ-ಗು-ವುದು
ಚಿತ್ತ-ಶು-ದ್ಧಿ-ಯಾದ
ಚಿತ್ತ-ಶು-ದ್ಧಿ-ಯಾ-ದರೆ
ಚಿತ್ತ-ಶು-ದ್ಧಿಯೇ
ಚಿತ್ತ-ಸ-ರೋ-ವ-ರಕ್ಕೆ
ಚಿತ್ತ-ಸ್ವಾ-ಧೀನ
ಚಿತ್ತಾ-ಕ-ರ್ಷ-ಕ-ವಾ-ಗಿದೆ
ಚಿತ್ಪ್ರಭೆ
ಚಿತ್ರ
ಚಿತ್ರ-ಕನ
ಚಿತ್ರ-ಕಲೆ
ಚಿತ್ರ-ಕಾರ
ಚಿತ್ರ-ಕಾ-ರ-ನಾರು
ಚಿತ್ರ-ಕಾ-ರರು
ಚಿತ್ರಕ್ಕೂ
ಚಿತ್ರ-ಗಳನ್ನು
ಚಿತ್ರ-ಗಾ-ರ-ನಾ-ಗಿರ
ಚಿತ್ರ-ಗು-ಪ್ತರ
ಚಿತ್ರ-ಗು-ಪ್ತರು
ಚಿತ್ರದ
ಚಿತ್ರ-ದಂ-ತಿದೆ
ಚಿತ್ರ-ದಲ್ಲಿ
ಚಿತ್ರ-ರಥ
ಚಿತ್ರ-ರಥಃ
ಚಿತ್ರ-ವನ್ನು
ಚಿತ್ರ-ವಾ-ಗು-ವುದು
ಚಿತ್ರ-ವಿ-ಚಿತ್ರ
ಚಿತ್ರ-ವಿ-ದ್ದರೆ
ಚಿತ್ರಿ-ಸ-ಲಾಗಿದೆ
ಚಿತ್ರಿ-ಸ-ಲಾ-ಯಿತು
ಚಿತ್ರಿ-ಸಿ-ಕೊಂ-ಡಿದ್ದ
ಚಿತ್ರಿ-ಸಿ-ಕೊಂಡು
ಚಿತ್ರಿ-ಸಿ-ಕೊಳ್ಳು
ಚಿತ್ರಿ-ಸಿ-ಕೊ-ಳ್ಳು-ತ್ತಿ-ದ್ದೆವು
ಚಿತ್ರಿ-ಸಿ-ಕೊ-ಳ್ಳು-ವುದು
ಚಿತ್ರಿ-ಸಿ-ದ್ದಾನೆ
ಚಿತ್ರಿ-ಸಿ-ರು-ವನು
ಚಿತ್ರಿ-ಸಿ-ರು-ವರು
ಚಿತ್ರಿ-ಸು-ತ್ತಾನೆ
ಚಿತ್ರಿ-ಸು-ತ್ತೇ-ವೆಯೆ
ಚಿತ್ರಿ-ಸು-ವರು
ಚಿತ್ರಿ-ಸು-ವರೋ
ಚಿತ್ರೂ-ಪ-ವಾ-ಗಿದೆ
ಚಿನ್ನ
ಚಿನ್ನಕ್ಕೆ
ಚಿನ್ನದ
ಚಿನ್ನ-ದಲ್ಲಿ
ಚಿನ್ನ-ದ್ದಾ-ಗಿ-ದ್ದರೆ
ಚಿನ್ನದ್ದೇ
ಚಿನ್ನ-ವನ್ನು
ಚಿನ್ನ-ವನ್ನೂ
ಚಿನ್ನ-ವ-ಲ್ಲದ
ಚಿನ್ನ-ವ-ಲ್ಲದೆ
ಚಿನ್ನ-ವಾ-ಗಿ-ದ್ದಕ್ಕೆ
ಚಿನ್ನ-ವಾ-ಗಿಯೇ
ಚಿನ್ನ-ವಾ-ಗು-ವುದು
ಚಿನ್ನ-ವಾ-ಗು-ವುದೊ
ಚಿನ್ನ-ವಾ-ದ-ಮೇಲೆ
ಚಿನ್ನ-ವಾ-ದರೆ
ಚಿನ್ನ-ವಾ-ಯಿತು
ಚಿನ್ನ-ವಿದೆ
ಚಿನ್ನವೇ
ಚಿನ್ಹೆಯೇ
ಚಿಪ್ಪಿ-ನಿಂದ
ಚಿಪ್ಪಿ-ನೊ-ಳಗೆ
ಚಿಪ್ಪು
ಚಿಪ್ಪು-ಗಳನ್ನು
ಚಿಪ್ಪು-ಗ-ಳಿವೆ
ಚಿಮಣಿ
ಚಿಮ-ಣಿಯ
ಚಿಮ-ಣಿ-ಯನ್ನು
ಚಿಮಿ-ಣಿ-ಯಂತೆ
ಚಿರ
ಚಿರಂ-ಜೀ-ವಿ-ಗಳು
ಚಿರ-ಜಾ-ಗ್ರ-ತ-ವಾಗಿ
ಚಿರ-ಪ-ರಿ-ಚಿ-ತ-ನಾದ
ಚಿರ-ಪ-ರಿ-ಚಿ-ತ-ವಾ-ಗಿದೆ
ಚಿರ-ಮು-ದ್ರಿ-ತ-ವಾ-ಗಿ-ರು-ವುದನ್ನು
ಚಿರಾತ್
ಚಿರೇ-ಣಾ-ಧಿ-ಗ-ಚ್ಛತಿ
ಚಿಲುಮೆ
ಚಿಲು-ಮೆ-ಗಳಲ್ಲಿ
ಚಿಲು-ಮೆ-ಯಂತೆ
ಚಿಲು-ಮೆ-ಯನ್ನು
ಚಿಲು-ಮೆ-ಯಿಂದ
ಚಿಲ್ಲರೆ
ಚಿಹ್ನೆ
ಚಿಹ್ನೆ-ಗಳಿಂದ
ಚಿಹ್ನೆ-ಗಳು
ಚಿಹ್ನೆಯ
ಚಿಹ್ನೆ-ಯಂ-ತಿವೆ
ಚಿಹ್ನೆ-ಯಾ-ಗಿತ್ತು
ಚಿಹ್ನೆ-ಯಾ-ಗಿದೆ
ಚಿಹ್ನೆ-ಯಾ-ಗಿ-ರುವ
ಚಿಹ್ನೆಯೋ
ಚೀಟಿ-ಯೊಂದು
ಚೀಪಲು
ಚೀಪಿ
ಚೀಪಿದ
ಚೀಪು-ತ್ತಾ-ನೆಯೊ
ಚೀಪು-ತ್ತಿರು
ಚೀಪು-ತ್ತಿ-ರು-ವಾ-ಗಲೇ
ಚೀಪು-ವುದು
ಚೀಪು-ವುದೇ
ಚೀಫ್
ಚುಕ್ಕಿ
ಚುಕ್ಕೆ-ಗ-ಳಂತೆ
ಚುಚ್ಚಿ
ಚುಚ್ಚು
ಚುಚ್ಚು-ಮಾ-ತಿ-ನಿಂದ
ಚುಚ್ಚು-ವನು
ಚುಚ್ಚು-ವುದು
ಚುಟಿಕೆ
ಚುನಾ-ವ-ಣೆಗೆ
ಚುರು-ಕಾಗಿ
ಚುರು-ಕಾ-ಯಿತು
ಚುರುಕು
ಚೂಪಾದ
ಚೂಪಿ-ನಲ್ಲಿ
ಚೂರನ್ನು
ಚೂರಾದ
ಚೂರಿಗೂ
ಚೂರಿಗೆ
ಚೂರು
ಚೂರು-ಗ-ಳಾ-ಗಿವೆ
ಚೂರು-ಚೂ-ರಾಗಿ
ಚೂರು-ಪಾರು
ಚೂರು-ಪಾ-ರು-ಗಳನ್ನು
ಚೂರು-ಪಾ-ರು-ಗ-ಳಲ್ಲ
ಚೂರು-ರೊಟ್ಟಿ
ಚೂರೂ
ಚೂರ್ಣಿ-ತೈ-ರು-ತ್ತ-ಮಾಂ-ಗೈಃ
ಚೆಂಡನ್ನು
ಚೆಂಡಾ-ಟ-ದಲ್ಲಿ
ಚೆಂಡು
ಚೆಂಡು-ಗಳನ್ನು
ಚೆಂಡೊಂದು
ಚೆಂಬನ್ನು
ಚೆಂಬಿ-ನಲ್ಲಿ
ಚೆಂಬಿ-ನೊ-ಳಗೆ
ಚೆಂಬು
ಚೆಕ್ಕನ್ನು
ಚೆದು-ರಿ-ಹೋ-ಗು-ವು-ದಕ್ಕೆ
ಚೆನ್ನ
ಚೆನ್ನ-ವೆಂದು
ಚೆನ್ನಾಗಿ
ಚೆನ್ನಾ-ಗಿ-ಡು-ವುದೇ
ಚೆನ್ನಾ-ಗಿ-ತ್ತಲ್ಲ
ಚೆನ್ನಾ-ಗಿತ್ತು
ಚೆನ್ನಾ-ಗಿದೆ
ಚೆನ್ನಾ-ಗಿ-ದ್ದರೆ
ಚೆನ್ನಾ-ಗಿ-ದ್ದರೇ
ಚೆನ್ನಾ-ಗಿ-ದ್ದೀ-ತೇನೊ
ಚೆನ್ನಾ-ಗಿಯೇ
ಚೆನ್ನಾ-ಗಿ-ರ-ಬೇ-ಕಾ-ದರೆ
ಚೆನ್ನಾ-ಗಿ-ರ-ಬೇಕು
ಚೆನ್ನಾ-ಗಿ-ರ-ಲಿಲ್ಲ
ಚೆನ್ನಾ-ಗಿ-ರುವ
ಚೆನ್ನಾ-ಗಿ-ರು-ವಾಗ
ಚೆನ್ನಾ-ಗಿ-ರು-ವು-ದಕ್ಕೆ
ಚೆನ್ನಾ-ಗಿ-ರು-ವುದು
ಚೆನ್ನಾ-ಗಿಲ್ಲ
ಚೆನ್ನಾ-ಗಿ-ಲ್ಲದೆ
ಚೆನ್ನಾದ
ಚೆಲ್ಲಾ-ಪಿ-ಲ್ಲಿ-ಯಾ-ಗು-ವುದು
ಚೆಲ್ಲಿ
ಚೆಲ್ಲಿ-ದಂ-ತಾ-ಗು-ವುದು
ಚೆಲ್ಲಿ-ದರೆ
ಚೆಲ್ಲು-ತ್ತಿ-ರುವ
ಚೆಲ್ಲು-ತ್ತಿ-ರು-ವುದು
ಚೆಲ್ಲು-ತ್ತಿಲ್ಲ
ಚೆಲ್ಲು-ತ್ತಿ-ವುದು
ಚೆಲ್ಲು-ವರು
ಚೆಲ್ಲು-ವುದು
ಚೇಂದ್ರಿ-ಯ-ಗೋ-ಚ-ರಾಃ
ಚೇಕಿ-ತಾನ
ಚೇಜ್ಯಯಾ
ಚೇತನ
ಚೇತ-ನಕ್ಕೆ
ಚೇತ-ನದ
ಚೇತ-ನ-ದಂತೆ
ಚೇತ-ನ-ದಲ್ಲಿ
ಚೇತ-ನ-ವಾಗ
ಚೇತ-ನ-ವಾ-ಗಲಿ
ಚೇತ-ನವೂ
ಚೇತ-ನ-ವೆಲ್ಲ
ಚೇತನಾ
ಚೇತ-ನಾ-ತ್ಮ-ನಂತೆ
ಚೇತ-ರಿ-ಸಿ-ಕೊಂ-ಡ-ಮೇಲೆ
ಚೇತ-ರಿ-ಸಿ-ಕೊ-ಳ್ಳಲಿ
ಚೇತ-ರಿ-ಸಿ-ಕೊ-ಳ್ಳು-ವನು
ಚೇತ-ರಿ-ಸಿ-ಕೊ-ಳ್ಳು-ವುದು
ಚೇತಸಾ
ಚೇತಿ
ಚೇತ್
ಚೇತ್ತ್ವ-ಮ-ಹಂ-ಕಾ-ರಾನ್ನ
ಚೇತ್ಸು-ದು-ರಾ-ಚಾರೋ
ಚೇದಸಿ
ಚೇದ-ಹಮ್
ಚೇಳಿನ
ಚೇಳಿ-ನಂತೆ
ಚೇಳಿ-ನಿಂದ
ಚೇಳು
ಚೇಳು-ಗ-ಳಿವೆ
ಚೇಷ್ಟತೇ
ಚೇಷ್ಟೆ
ಚೈಕಾಂ-ತ-ಮ-ನ-ಶ್ನತಃ
ಚೈತ-ದ್ವಿದ್ಮಃ
ಚೈತನ್ಯ
ಚೈತ-ನ್ಯಕ್ಕೆ
ಚೈತ-ನ್ಯ-ಗಳನ್ನು
ಚೈತ-ನ್ಯದ
ಚೈತ-ನ್ಯನ
ಚೈತ-ನ್ಯ-ವನ್ನು
ಚೈತ-ನ್ಯ-ವನ್ನೇ
ಚೈತ-ನ್ಯ-ವಲ್ಲ
ಚೈತ-ನ್ಯ-ವಿದೆ
ಚೈತ-ನ್ಯ-ವಿ-ಲ್ಲದ
ಚೈತ-ನ್ಯವೂ
ಚೈತ-ನ್ಯ-ವೇ-ನಾ-ದರೂ
ಚೈತ-ನ್ಯ-ವೇನು
ಚೈತ-ನ್ಯ-ವೊಂದು
ಚೈತ-ನ್ಯಾಂಶ
ಚೈತ-ನ್ಯಾಂ-ಶ-ದಿಂದ
ಚೈತ-ನ್ಯಾ-ತ್ಮ-ಕ-ವಾ-ಗಿ-ರು-ವು-ದ-ರಿಂದ
ಚೈತಾಂ-ಸ್ತ್ರೀನ್
ಚೈನಂ
ಚೈಲಾ-ಜಿನ
ಚೈವ
ಚೈವಾಂ-ತಃ-ಶ-ರೀ-ರಸ್ಥಂ
ಚೈವಾತ್ರ
ಚೈವಾಯಂ
ಚೈವಾಸ್ಮಿ
ಚೈವಾ-ಹ-ಮ-ರ್ಜುನ
ಚೊಕ್ಕ-ಟ-ವಾಗಿ
ಚೋಕ್ತಂ
ಚೋಚ್ಯತೇ
ಚೋತ್ತಮಃ
ಚೋಪ-ಪನ್ನಂ
ಚೋರನ
ಚೋರ-ನಾ-ಗಿ-ರು-ವಾಗ
ಚೌಕ-ಟ್ಟನ್ನು
ಚೌಕ-ಟ್ಟಿ-ನೊ-ಳಗೆ
ಚೌಕಾಶಿ
ಚೌಕಾ-ಶಿಯ
ಚೌಕಾಸಿ
ಚೌಷ-ಧೀಃ
ಚ್ಯುತ-ನಾ-ಗದೆ
ಚ್ಯುತ-ನಾ-ಗುವ
ಚ್ಯುತ-ನಾ-ಗು-ವು-ದಕ್ಕೆ
ಚ್ಯುತ-ರಾ-ಗು-ತ್ತಾರೆ
ಚ್ಯುತ-ವಾ-ದರೆ
ಚ್ಯುತಿ
ಛಂದ-ಸಾ-ಮ-ಹಮ್
ಛಂದ-ಸ್ಸಿನ
ಛಂದ-ಸ್ಸಿ-ನ-ಲ್ಲಿಯೇ
ಛಂದಸ್ಸು
ಛಂದ-ಸ್ಸು-ಗಳಲ್ಲಿ
ಛಂದ-ಸ್ಸು-ಗ-ಳಿವೆ
ಛಂದ-ಸ್ಸು-ಗಳು
ಛಂದಾಂಸಿ
ಛಂದೋ-ಭಿ-ರ್ವಿ-ವಿ-ಧೈಃ
ಛತ್ರ
ಛತ್ರದ
ಛತ್ರ-ದಲ್ಲಿ
ಛತ್ರ-ವನ್ನೋ
ಛತ್ರಿ
ಛಲ
ಛಲ-ಗಾರ
ಛಲ-ದಿಂದ
ಛಲ-ಯ-ತಾ-ಮಸ್ಮಿ
ಛಲ-ವನ್ನು
ಛಳಿ
ಛಳಿ-ಗಾ-ಲ-ದಲ್ಲಿ
ಛಳಿ-ಯಾ-ಗು-ವುದು
ಛಳಿಯೂ
ಛಾಂದೋಗ್ಯ
ಛಾಯೆ
ಛಾಯೆಯ
ಛಾಯೆ-ಯಂ-ತಿ-ರು-ವುದು
ಛಾಯೆ-ಯಂತೆ
ಛಾಯೆ-ಯ-ಲ್ಲಿಯೇ
ಛಾಯೆಯೂ
ಛಿಂದಂತಿ
ಛಿತ್ತ್ವಾ
ಛಿತ್ತ್ವೈನಂ
ಛಿನ್ನ-ದ್ವ್ಯೆಧಾ
ಛಿನ್ನ-ಸಂ-ಶಯಃ
ಛೀಮಾರಿ
ಛೇತ್ತಾ
ಛೇತ್ತು-ಮ-ರ್ಹ-ಸ್ಯ-ಶೇ-ಷತಃ
ಛೇದಿ-ಸ-ಬೇಕು
ಛೇದಿಸಿ
ಛೇದಿ-ಸಿ-ದಾ-ಗಲೇ
ಛೇದಿಸು
ಜಂಗಮ
ಜಂಗ-ಮ-ಗಳಲ್ಲಿ
ಜಂಗ-ಮ-ಗ-ಳೆಲ್ಲ
ಜಂಜ-ಡಕ್ಕೆ
ಜಂತವಃ
ಜಂತು-ಗಳು
ಜಂತು-ವನ್ನು
ಜಂಭ
ಜಂಭ-ವಿ-ಲ್ಲದೆ
ಜಗ-ಚ್ಚ-ಕ್ರ-ವನ್ನು
ಜಗತಃ
ಜಗತೋ
ಜಗ-ತೋ-ಽಹಿ-ತಾಃ
ಜಗತ್
ಜಗ-ತ್ಕೃತ್ಸ್ನಂ
ಜಗ-ತ್ತನ್ನು
ಜಗ-ತ್ತ-ನ್ನೆಲ್ಲ
ಜಗ-ತ್ತನ್ನೇ
ಜಗತ್ತಿ
ಜಗ-ತ್ತಿ-ಗಾ-ಗಲಿ
ಜಗ-ತ್ತಿಗೆ
ಜಗ-ತ್ತಿ-ಗೆಲ್ಲ
ಜಗ-ತ್ತಿನ
ಜಗ-ತ್ತಿ-ನಂತೆ
ಜಗ-ತ್ತಿ-ನಲ್ಲಿ
ಜಗತ್ತು
ಜಗ-ತ್ತು-ಗ-ಳಿಗೆ
ಜಗ-ತ್ತೆಂ-ದರೆ
ಜಗ-ತ್ತೆಂದು
ಜಗ-ತ್ತೆಲ್ಲ
ಜಗ-ತ್ತೆ-ಲ್ಲವೂ
ಜಗ-ತ್ತೇನು
ಜಗ-ತ್ಪತೇ
ಜಗ-ತ್ಸ-ಮಗ್ರಂ
ಜಗದ
ಜಗ-ದಲ್ಲಿ
ಜಗ-ದ-ವ್ಯ-ಕ್ತ-ಮೂ-ರ್ತಿನಾ
ಜಗ-ದಾ-ಹು-ರ-ನೀ-ಶ್ವ-ರಮ್
ಜಗ-ದೀ-ಶ್ವರ
ಜಗ-ದ್ಗುರು
ಜಗ-ದ್ಗು-ರುಮ್
ಜಗ-ದ್ಗು-ರುವೂ
ಜಗ-ದ್ಭಾ-ಸ-ಯ-ತೇ-ಽಖಿ-ಲಮ್
ಜಗ-ದ್ವಿ-ಪ-ರಿ-ವ-ರ್ತತೇ
ಜಗ-ನ್ನಿ-ವಾಸ
ಜಗ-ನ್ನೀ-ವಾಸ
ಜಗ-ನ್ಮಯಿ
ಜಗ-ಲಿಯ
ಜಗಳ
ಜಗ-ಳದ
ಜಗ-ವ-ನ್ನೆಲ್ಲ
ಜಗ್ಗ-ದ-ವನು
ಜಗ್ಗದೆ
ಜಗ್ಗದೇ
ಜಗ್ಗಲಿ
ಜಗ್ಗಿಸಿ
ಜಗ್ಗಿ-ಸಿ-ದರೂ
ಜಗ್ಗು-ವ-ವ-ನಲ್ಲ
ಜಗ್ಗು-ವು-ದಿಲ್ಲ
ಜಘ-ನ್ಯ-ಗು-ಣ-ವೃ-ತ್ತಿಸ್ಥಾ
ಜಜ್ಜಿ
ಜಟಕಾ
ಜಟಾ-ಜೂ-ಟ-ದಿಂದ
ಜಟಿಲ
ಜಟಿ-ಲ-ವಾದ
ಜಟ್ಟಿ-ಗ-ಳಾ-ಗ-ಬೇ-ಕಾ-ದರೆ
ಜಟ್ಟಿ-ಯಾ-ಗ-ಲಾರ
ಜಠರ
ಜಠ-ರ-ಕೋಶ
ಜಠ-ರ-ಕೋ-ಶ-ದೊ-ಳಗೆ
ಜಠ-ರಾ-ಗ್ನಿಯ
ಜಡ
ಜಡ-ಚೇ-ತನ
ಜಡ-ಗ-ಣ್ಣಿಗೆ
ಜಡ-ಚೇ-ತ-ನದ
ಜಡತ್ವ
ಜಡದ
ಜಡ-ದಲ್ಲಿ
ಜಡ-ಮಾ-ಡು-ವುದು
ಜಡ-ವನ್ನು
ಜಡ-ವನ್ನೇ
ಜಡ-ವ-ಸ್ತು-ವಿನ
ಜಡ-ವ-ಸ್ತು-ವಿ-ನಂತೆ
ಜಡ-ವಾ-ಗ-ಬ-ಹುದು
ಜಡ-ವಾ-ಗಲಿ
ಜಡ-ವಾಗಿ
ಜಡ-ವಾ-ಗಿ-ರು-ವುದು
ಜಡ-ವಾ-ಗು-ವುದು
ಜಡ-ವಾದ
ಜಡ-ವಾದಿ
ಜಡ-ವಾ-ದಿ-ಗಳ
ಜಡ-ವಾ-ದಿ-ಗ-ಳಲ್ಲ
ಜಡ-ವಾ-ದಿ-ಗಳು
ಜಡ-ವಾ-ದಿಗೆ
ಜಡ-ವಾ-ದಿ-ಯಾ-ಗಲಿ
ಜಡ-ವೊಂದೇ
ಜಡ-ಸ್ವ-ರೂ-ಪ-ವನ್ನು
ಜಡಾ-ವ-ಸ್ಥೆ-ಯ-ಲ್ಲಿ-ರುವ
ಜಡ್ಜಿ-ಯಾ-ಗಲಿ
ಜತೆ-ಗಾ-ರ-ರನ್ನು
ಜನ
ಜನಃ
ಜನಕ
ಜನ-ಕ-ನಿಗೆ
ಜನ-ಕನೇ
ಜನ-ಕ-ರಾ-ಜನ
ಜನ-ಕಾ-ದಯಃ
ಜನಕ್ಕೆ
ಜನ-ಗ-ಳಿಗೆ
ಜನ-ಜಂ-ಗು-ಳಿ-ಯನ್ನು
ಜನನ
ಜನ-ನ-ದ-ಲ್ಲಿ-ರು-ವನು
ಜನ-ನ-ದಿಂದ
ಜನ-ನ-ಮ-ರ-ಣ-ಗಳಲ್ಲಿ
ಜನ-ನೇಂ-ದ್ರಿ-ಯ-ಗಳು
ಜನ-ಯೇ-ದ-ಜ್ಞಾ-ನಾಂ
ಜನರ
ಜನ-ರಂ-ಜ-ನೆಯ
ಜನ-ರನ್ನು
ಜನ-ರಲ್ಲ
ಜನ-ರಲ್ಲಿ
ಜನ-ರ-ಲ್ಲೆಲ್ಲ
ಜನ-ರಾ-ಡುವ
ಜನ-ರಿಂದ
ಜನ-ರಿ-ಗಿಂತ
ಜನ-ರಿಗೆ
ಜನ-ರಿ-ದ್ದಾರೆ
ಜನ-ರಿ-ರ-ಬೇಕು
ಜನ-ರಿರು
ಜನ-ರಿ-ರು-ವರು
ಜನರು
ಜನರೂ
ಜನ-ರೆ-ನ್ನು-ವುದನ್ನು
ಜನ-ರೆಲ್ಲ
ಜನ-ರೆ-ಲ್ಲರ
ಜನ-ರೆಲ್ಲಾ
ಜನರೇ
ಜನ-ಸಂ-ಗ-ವ-ನ್ನೆಲ್ಲಾ
ಜನ-ಸಂ-ದಣಿ
ಜನ-ಸ-ಮು-ದಾ-ಯದ
ಜನ-ಸ-ಮೂ-ಹ-ದಲ್ಲಿ
ಜನ-ಸಾ-ಧಾ-ರ-ಣರು
ಜನ-ಸಾ-ಧಾ-ರ-ಣ-ರೆಲ್ಲ
ಜನ-ಸಾ-ಮಾ-ನ್ಯ-ರಿಗೆ
ಜನಾ
ಜನಾಂಗ
ಜನಾಂ-ಗಕ್ಕೂ
ಜನಾಂ-ಗಕ್ಕೆ
ಜನಾಂ-ಗದ
ಜನಾಂ-ಗ-ದಲ್ಲಿ
ಜನಾಂ-ಗ-ವನ್ನು
ಜನಾಃ
ಜನಾ-ಧಿ-ಪಾಃ
ಜನಾ-ನಾಂ
ಜನಾ-ರ್ದನ
ಜನಿಸಿ
ಜನಿ-ಸಿ-ದರು
ಜನಿ-ಸು-ವಂತೆ
ಜನಿ-ಸು-ವನು
ಜನಿ-ಸು-ವುದು
ಜನಿ-ಸು-ವೆವು
ಜನ್ಮ
ಜನ್ಮ-ಕ-ರ್ಮ-ಫ-ಲ-ಪ್ರ-ದಾಮ್
ಜನ್ಮಕ್ಕೆ
ಜನ್ಮ-ಗಳ
ಜನ್ಮ-ಗಳನ್ನು
ಜನ್ಮ-ಗ-ಳನ್ನೇ
ಜನ್ಮ-ಗಳಲ್ಲಿ
ಜನ್ಮ-ಗ-ಳಾದ
ಜನ್ಮ-ಗಳಿಂದ
ಜನ್ಮ-ಗ-ಳಿಂ-ದಲೂ
ಜನ್ಮ-ಗ-ಳಿವೆ
ಜನ್ಮ-ಗಳು
ಜನ್ಮ-ಗ-ಳೆಲ್ಲ
ಜನ್ಮ-ಜ-ನ್ಮ-ಗಳು
ಜನ್ಮ-ಜ-ನ್ಮಾಂ-ತ-ರ-ಗಳಿಂದ
ಜನ್ಮ-ಜ-ನ್ಮಾಂ-ತ-ರ-ದಿಂದ
ಜನ್ಮ-ತಾ-ಳಿ-ದಂ-ತಿದೆ
ಜನ್ಮದ
ಜನ್ಮ-ದಲ್ಲಿ
ಜನ್ಮ-ದ-ಲ್ಲಿಯೂ
ಜನ್ಮ-ದ-ಲ್ಲಿಯೇ
ಜನ್ಮ-ದ-ಲ್ಲಿಯೋ
ಜನ್ಮ-ದಲ್ಲೇ
ಜನ್ಮ-ದಿಂ-ದಲೇ
ಜನ್ಮ-ಧಾ-ರಣೆ
ಜನ್ಮ-ನಾ-ಮಂತೇ
ಜನ್ಮನಿ
ಜನ್ಮ-ಬಂ-ಧ-ವಿ-ನಿ-ರ್ಮು-ಕ್ತಾಃ
ಜನ್ಮ-ಮೃ-ತ್ಯು-ಜ-ರಾ-ದುಃ-ಖೈ-ರ್ವಿ-ಮು-ಕ್ತೋ-ಽಮೃ-ತ-ಮ-ಶ್ನುತೇ
ಜನ್ಮ-ಮೃ-ತ್ಯು-ಜ-ರಾ-ವ್ಯಾ-ಧಿ-ದುಃ-ಖ-ದೋ-ಷಾ-ನು-ದ-ರ್ಶ-ನಮ್
ಜನ್ಮ-ರ-ಹಿತ
ಜನ್ಮ-ರ-ಹಿ-ತ-ನೆಂದು
ಜನ್ಮ-ವ-ನ್ನಾ-ಗಲಿ
ಜನ್ಮ-ವನ್ನು
ಜನ್ಮ-ವನ್ನೂ
ಜನ್ಮ-ವಲ್ಲ
ಜನ್ಮ-ವಿ-ದ್ದರೆ
ಜನ್ಮವೂ
ಜನ್ಮ-ವೆಂ-ಬುದು
ಜನ್ಮ-ವೆ-ತ್ತ-ಬೇ-ಕಾ-ಗಿಲ್ಲ
ಜನ್ಮ-ವೆ-ತ್ತ-ಬೇಕು
ಜನ್ಮ-ವೆ-ತ್ತ-ಲೇ-ಬೇಕಾ
ಜನ್ಮ-ವೆತ್ತಿ
ಜನ್ಮ-ವೆ-ತ್ತಿದ
ಜನ್ಮ-ವೆ-ತ್ತಿ-ದಂತೆ
ಜನ್ಮ-ವೆ-ತ್ತಿ-ದರೂ
ಜನ್ಮ-ವೆ-ತ್ತಿ-ರುವೆ
ಜನ್ಮ-ವೆ-ತ್ತು-ತ್ತಾನೆ
ಜನ್ಮ-ವೆ-ತ್ತು-ತ್ತಾನೊ
ಜನ್ಮ-ವೆ-ತ್ತು-ವಾಗ
ಜನ್ಮ-ವೆ-ತ್ತು-ವುದು
ಜನ್ಮವೇ
ಜನ್ಮಾಂ-ತ-ರ-ಗ-ಳಿಂ-ದಲೂ
ಜನ್ಮಾಂ-ತ-ರ-ದಿಂದ
ಜನ್ಮಾಂ-ಧ-ನಾದ
ಜನ್ಮಾನಿ
ಜಪ
ಜಪ-ತ-ಪಾ-ದಿ-ಗಳನ್ನು
ಜಪ-ದಷ್ಟು
ಜಪ-ಮಾಡಿ
ಜಪ-ಮಾ-ಡು-ತ್ತಾರೆ
ಜಪ-ಯಜ್ಞ
ಜಪ-ಯ-ಜ್ಞವೇ
ಜಪ-ಯ-ಜ್ಞೋಽಸ್ಮಿ
ಜಪ-ವೆಂದರೆ
ಜಪಿ-ಸ-ಬ-ಹುದು
ಜಪಿಸಿ
ಜಪಿ-ಸುತ್ತ
ಜಪಿ-ಸುತ್ತಾ
ಜಮೀ-ನನ್ನು
ಜಮೀ-ನಿ-ನಲ್ಲಿ
ಜಮೀನು
ಜಮೀ-ನ್ದಾರ
ಜಮೀ-ನ್ದಾ-ರಿಗೆ
ಜಮೀ-ನ್ದಾ-ರಿ-ಯನ್ನೇ
ಜಯ
ಜಯ-ವಿ-ಜ-ಯ-ರಿಗೆ
ಜಯ-ಕಾ-ರಿ-ಯಾ-ಗ-ಬೇ-ಕಾ-ದರೆ
ಜಯಕ್ಕೆ
ಜಯದ
ಜಯ-ದಲ್ಲಿ
ಜಯ-ದಿಂ-ದೇನು
ಜಯ-ದೊ-ಡನೆ
ಜಯ-ದ್ರಥ
ಜಯ-ದ್ರಥಂ
ಜಯ-ದ್ರ-ಥ-ಜಲಾ
ಜಯ-ಪ್ರ-ದ-ವಾ-ಗು-ವುದು
ಜಯ-ವನ್ನು
ಜಯ-ವಾ-ದರೂ
ಜಯ-ವಿ-ಜ-ಯ-ರಿಗೆ
ಜಯವೇ
ಜಯವೋ
ಜಯ-ಶಾ-ಲಿ-ಗಳ
ಜಯ-ಶಾ-ಲಿ-ಗ-ಳ-ಲ್ಲಿ-ರುವ
ಜಯ-ಶಾ-ಲಿ-ಗ-ಳಾ-ಗಿಲ್ಲ
ಜಯ-ಶಾ-ಲಿ-ಗ-ಳಾ-ದೆವು
ಜಯ-ಶೀ-ಲ-ನಾ-ಗದೆ
ಜಯ-ಶೀ-ಲ-ನಾ-ದರೆ
ಜಯ-ಶೀ-ಲ-ರಾ-ಗ-ಲಾ-ರರು
ಜಯ-ಶೀ-ಲ-ರಾ-ಗಿ-ರು-ವರೋ
ಜಯಾ-ಜಯೌ
ಜಯಾ-ಭಿ-ಲಾ-ಷೆ-ಗಳ
ಜಯಿಗೆ
ಜಯಿ-ಸದ
ಜಯಿ-ಸ-ಬೇಕು
ಜಯಿ-ಸ-ಲ್ಪಟ್ಟ
ಜಯಿಸಿ
ಜಯಿ-ಸಿ-ದ-ವನು
ಜಯಿ-ಸಿ-ಲ್ಲದೇ
ಜಯಿ-ಸು-ತ್ತಾರೆ
ಜಯಿ-ಸು-ತ್ತಾ-ರೆಯೋ
ಜಯಿ-ಸು-ತ್ತೇ-ವೆಯೋ
ಜಯಿ-ಸು-ವನು
ಜಯಿ-ಸು-ವು-ದಕ್ಕೆ
ಜಯಿ-ಸು-ವು-ದಕ್ಕೇ
ಜಯಿ-ಸು-ವು-ದಲ್ಲ
ಜಯಿ-ಸು-ವುದು
ಜಯಿ-ಸುವೆ
ಜಯೇಮ
ಜಯೇ-ಯುಃ
ಜಯೋಽಸ್ಮಿ
ಜರ
ಜರಡಿ
ಜರ-ತಾರಿ
ಜರಾ
ಜರಾ-ಮ-ರ-ಣ-ಗಳಿಂದ
ಜರಾ-ಮ-ರ-ಣ-ಮೋ-ಕ್ಷಾಯ
ಜರಾ-ಯು-ವೆಂಬ
ಜರಾ-ಸಂಧ
ಜರೆ
ಜರೆ-ಯು-ವರು
ಜರ್ಝ-ರಿ-ತ-ರಾ-ಗಿ-ರು-ವೆವು
ಜಲ
ಜಲ-ಚ-ರ-ಗಳಲ್ಲಿ
ಜಲ-ದಲ್ಲಿ
ಜಲ-ದಿಂದ
ಜಲ-ರೂ-ಪ-ದ-ಲ್ಲಿಯೇ
ಜಲ್ಪ
ಜಲ್ಪ-ವೆಂದರೆ
ಜಲ್ಲೆ-ಯನ್ನು
ಜಳ್ಳನ್ನು
ಜಳ್ಳು-ಗಳನ್ನೆಲ್ಲ
ಜವಾ-ಬ್ದಾ-ರ-ನಾ-ಗ-ಬೇ-ಕಾಗು
ಜವಾ-ಬ್ದಾ-ರ-ನಾ-ಗು-ತ್ತಾನೆ
ಜವಾ-ಬ್ದಾರಿ
ಜವಾ-ಬ್ದಾ-ರಿಯ
ಜವಾ-ಬ್ದಾ-ರಿ-ಯನ್ನು
ಜವಾ-ಬ್ದಾ-ರಿ-ಯನ್ನೂ
ಜವಾ-ಬ್ದಾ-ರಿ-ಯ-ನ್ನೆಲ್ಲ
ಜವಾ-ಬ್ದಾ-ರಿ-ಯ-ನ್ನೆಲ್ಲಾ
ಜವಾ-ಬ್ದಾ-ರಿ-ಯ-ಲ್ಲಿ-ರುವ
ಜವಾ-ಬ್ದಾ-ರಿ-ಯಲ್ಲೇ
ಜವಾ-ಬ್ದಾ-ರಿ-ಯೆಂದು
ಜವಾ-ಬ್ದಾ-ರಿ-ಯೆ-ಲ್ಲ-ವನ್ನೂ
ಜವೆ
ಜಹಾ-ತೀಹ
ಜಹಿ
ಜಾಗ
ಜಾಗ-ತ್ರ-ವಾ-ಗಿಲ್ಲ
ಜಾಗ-ರಣೆ
ಜಾಗ-ರೂ-ಕ-ನಾ-ಗಿ-ರ-ಲಿಲ್ಲ
ಜಾಗ-ರೂ-ಕ-ರಾ-ಗಿ-ರ-ಬೇಕು
ಜಾಗರ್ತಿ
ಜಾಗೃ-ತ-ವಾಗಿ
ಜಾಗೃ-ತ-ವಾ-ಗಿದೆ
ಜಾಗೃ-ತ-ವಾ-ಗಿ-ದ್ದರೆ
ಜಾಗೃ-ತ-ವಾ-ಗಿ-ರುವ
ಜಾಗೃ-ತ-ವಾದ
ಜಾಗ್ರತ
ಜಾಗ್ರ-ತ-ಗೊ-ಳಿ-ಸ-ಬೇ-ಕೆಂದು
ಜಾಗ್ರ-ತ-ಗೊ-ಳಿ-ಸು-ವಂತೆ
ಜಾಗ್ರ-ತ-ನಾ-ಗಲಿ
ಜಾಗ್ರ-ತ-ನಾ-ಗಿ-ರು-ವನು
ಜಾಗ್ರ-ತ-ನಾ-ಗು-ತ್ತಿದ್ದ
ಜಾಗ್ರ-ತ-ರಾ-ಗಿ-ರು-ವರೊ
ಜಾಗ್ರ-ತ-ವಾಗಿ
ಜಾಗ್ರ-ತ-ವಾ-ಗಿದೆ
ಜಾಗ್ರ-ತ-ವಾ-ಗಿ-ದ್ದರೆ
ಜಾಗ್ರ-ತ-ವಾ-ಗಿ-ರ-ಬೇಕು
ಜಾಗ್ರ-ತ-ವಾ-ಗುತ್ತ
ಜಾಗ್ರ-ತ-ವಾ-ಗು-ವುದು
ಜಾಗ್ರತಾ
ಜಾಗ್ರ-ತಾ-ವಸ್ಥೆ
ಜಾಗ್ರ-ತಾ-ವ-ಸ್ಥೆಗೆ
ಜಾಗ್ರ-ತಾ-ವ-ಸ್ಥೆಗೊ
ಜಾಗ್ರ-ತಾ-ವ-ಸ್ಥೆಯ
ಜಾಗ್ರ-ತಾ-ವ-ಸ್ಥೆ-ಯಲ್ಲಿ
ಜಾಗ್ರ-ತಾ-ವ-ಸ್ಥೆ-ಯ-ಲ್ಲಿ-ರುವ
ಜಾಗ್ರ-ತಾ-ವ-ಸ್ಥೆ-ಯ-ಲ್ಲಿ-ರು-ವಾಗ
ಜಾಗ್ರ-ತಾ-ವ-ಸ್ಥೆ-ಯಿಂದ
ಜಾಗ್ರತಿ
ಜಾಗ್ರ-ತೆ-ಯನ್ನು
ಜಾಗ್ರತೋ
ಜಾಜಿ
ಜಾಡ-ಮಾ-ಲಿ-ಗಳು
ಜಾಡ್ಯ
ಜಾಡ್ಯ-ಗ-ಳಿಗೆ
ಜಾಡ್ಯ-ಗ-ಳಿವೆ
ಜಾಡ್ಯ-ಗಳು
ಜಾಡ್ಯ-ದಂತೆ
ಜಾಡ್ಯ-ವನ್ನು
ಜಾಡ್ಯ-ವಾ-ಗಿ-ರ-ಬೇಕು
ಜಾಣ
ಜಾಣ-ತನ
ಜಾಣ-ತ-ನದ
ಜಾಣ-ತ-ನ-ದಿಂದ
ಜಾಣ-ತ-ನ-ವ-ನ್ನೆಲ್ಲಾ
ಜಾಣ-ತ-ನವೆ
ಜಾಣ-ತ-ವನ್ನು
ಜಾಣರು
ಜಾಣ್ಮೆ
ಜಾಣ್ಮೆಗೆ
ಜಾಣ್ಮೆ-ಯಾ-ಗಲೀ
ಜಾಣ್ಮೆ-ಯಾ-ಗಿ-ರ-ಬ-ಹುದು
ಜಾಣ್ಮೆ-ಯಿಂದ
ಜಾಣ್ಮೆಯೂ
ಜಾತಸ್ಯ
ಜಾತಾ
ಜಾತಿ
ಜಾತಿ-ಗಳನ್ನು
ಜಾತಿ-ಗ-ಳಿವೆ
ಜಾತಿಗೊ
ಜಾತಿ-ಧ-ರ್ಮ-ಗಳು
ಜಾತಿ-ಧ-ರ್ಮಾಃ
ಜಾತಿಯ
ಜಾತಿ-ಯನ್ನು
ಜಾತಿ-ಯಲ್ಲೂ
ಜಾತಿ-ಯಿಂ-ದಲೇ
ಜಾತಿಯೂ
ಜಾತಿ-ಸ್ಮರ
ಜಾತಿ-ಸ್ಮ-ರಣೆ
ಜಾತೀ
ಜಾತು
ಜಾತ್ರೆ-ಯನ್ನು
ಜಾನಕಿ
ಜಾನನ್
ಜಾನಾತಿ
ಜಾನೀಹಿ
ಜಾನೇ
ಜಾಯಂತೇ
ಜಾಯತೇ
ಜಾರ-ಬ-ಹುದು
ಜಾರಲು
ಜಾರಿ
ಜಾರಿ-ಕೊಂಡು
ಜಾರಿಗೆ
ಜಾರಿತು
ಜಾರಿ-ದರೂ
ಜಾರಿ-ದರೆ
ಜಾರಿ-ಮಾ-ಡ-ಲಾ-ರವು
ಜಾರಿ-ಮಾ-ಡು-ವು-ದಕ್ಕೆ
ಜಾರಿ-ಯಲ್ಲಿ
ಜಾರಿ-ಯ-ಲ್ಲಿ-ರ-ಲಿ-ಲ್ಲವೆ
ಜಾರಿ-ಯ-ಲ್ಲಿ-ರು-ವುದನ್ನು
ಜಾರಿ-ಹೋ-ಗು-ತ್ತೇವೆ
ಜಾರಿ-ಹೋ-ಗು-ವುದು
ಜಾರು-ಗು-ಪ್ಪೆಯ
ಜಾರು-ಗು-ಪ್ಪೆ-ಯಲ್ಲಿ
ಜಾರುತ್ತ
ಜಾರು-ತ್ತಾನೆ
ಜಾರು-ತ್ತಿ-ರ-ಬೇ-ಕಾಗು
ಜಾರು-ತ್ತಿ-ರುವ
ಜಾರು-ತ್ತೇವೆ
ಜಾರು-ಭೂ-ಮಿಯ
ಜಾರುವ
ಜಾರು-ವಂ-ತಿಲ್ಲ
ಜಾರು-ವಂತೆ
ಜಾರು-ವನು
ಜಾರು-ವು-ದಕ್ಕೆ
ಜಾರು-ವು-ದಾ-ದರೋ
ಜಾರು-ವು-ದಿಲ್ಲ
ಜಾರು-ವುದು
ಜಾರು-ವೆವು
ಜಾಲ-ದಲ್ಲಿ
ಜಾಲ-ವನ್ನು
ಜಾಸ್ತಿ
ಜಾಸ್ತಿ-ಯನ್ನೂ
ಜಾಸ್ತಿ-ಯಾಗಿ
ಜಾಸ್ತಿ-ಯಾ-ಗಿ-ರು-ವುದು
ಜಾಸ್ತಿ-ಯಾ-ಗು-ವು-ದಿ-ಲ್ಲ-ಮ-ತ್ತೊ-ಬ್ಬ-ನಿಗೆ
ಜಾಸ್ತಿ-ಯಾ-ಗು-ವುದು
ಜಾಸ್ತಿ-ಯಾ-ದಾಗ
ಜಾಸ್ತಿ-ಹಿ-ಡಿ-ಯು-ವುದೆ
ಜಾಹಿ-ರಾ-ತಿಗೆ
ಜಾಹಿ-ರಾತು
ಜಾಹ್ನವೀ
ಜಿಂಕೆ
ಜಿಂಕೆ-ಯಂತೆ
ಜಿಂಕೆಯೊ
ಜಿಂಕೆ-ಯೊಂದು
ಜಿಗ-ಣೆ-ಯಂತೆ
ಜಿಗೀ-ಷ-ತಾಮ್
ಜಿಘ್ರ-ನ್ನ-ಶ್ನನ್
ಜಿಜೀ-ವಿ-ಷಾಮ
ಜಿಜ್ಞಾಸು
ಜಿಜ್ಞಾ-ಸು-ರಪಿ
ಜಿಜ್ಞಾ-ಸು-ರ-ರ್ಥಾರ್ಥೀ
ಜಿಜ್ಞಾಸೆ
ಜಿತಃ
ಜಿತ-ನ-ಲ್ಲ-ದ-ವನು
ಜಿತ-ಸಂ-ಗ-ದೋಷಾ
ಜಿತಾ-ತ್ಮನಃ
ಜಿತಾತ್ಮಾ
ಜಿತೇಂ-ದ್ರಿಯ
ಜಿತೇಂ-ದ್ರಿಯಃ
ಜಿತೇಂ-ದ್ರಿ-ಯ-ನಾ-ಗದೆ
ಜಿತೇಂ-ದ್ರಿ-ಯ-ನಾಗಿ
ಜಿತೇಂ-ದ್ರಿ-ಯ-ನಾ-ಗಿ-ರ-ಬೇಕು
ಜಿತೇಂ-ದ್ರಿ-ಯ-ನಾ-ಗಿ-ರ-ಬೇ-ಕೆಂ-ಬುದೇ
ಜಿತೇಂ-ದ್ರಿ-ಯ-ನಾ-ಗಿ-ರು-ವನು
ಜಿತೇಂ-ದ್ರಿ-ಯನೂ
ಜಿತೇಂ-ದ್ರಿ-ಯರು
ಜಿತೇಂ-ದ್ರಿ-ಯರೂ
ಜಿತ್ವಾ
ಜಿದ್ದಿಗೋ
ಜಿದ್ದು-ಳ್ಳ-ವನು
ಜಿನು-ಗ-ತೊ-ಡ-ಗಿತು
ಜಿನುಗಿ
ಜಿನು-ಗು-ತ್ತಿರು
ಜಿನು-ಗು-ತ್ತಿ-ರು-ವುದು
ಜಿನು-ಗು-ವುದು
ಜಿಪು-ಣ-ತನ
ಜಿಪು-ಣ-ನಾ-ದರೂ
ಜಿಲೇ-ಬಿ-ಯನ್ನು
ಜೀತ
ಜೀತೇಂ-ದ್ರಿ-ಯ-ನಾ-ಗಿ-ರ-ಬೇಕು
ಜೀತೇಂ-ದ್ರಿ-ಯನೂ
ಜೀರೊ
ಜೀರೋ
ಜೀರ್ಣ-ಕೋ-ಶ-ಗ-ಳಿಗೆ
ಜೀರ್ಣ-ವಾಗಿ
ಜೀರ್ಣ-ವಾ-ಗುವ
ಜೀರ್ಣ-ವಾ-ಗು-ವಂತೆ
ಜೀರ್ಣ-ವಾ-ಗು-ವು-ದಕ್ಕೆ
ಜೀರ್ಣ-ವಾ-ಗು-ವುದು
ಜೀರ್ಣ-ವಾ-ಗು-ವುವು
ಜೀರ್ಣ-ವಾದ
ಜೀರ್ಣ-ಶಕ್ತಿ
ಜೀರ್ಣ-ಸಿ-ಕೊ-ಳ್ಳು-ವು-ದಕ್ಕೆ
ಜೀರ್ಣಾ
ಜೀರ್ಣಾನಿ
ಜೀರ್ಣಿ-ಸಿ-ಕೊಂ-ಡ-ಲ್ಲದೇ
ಜೀರ್ಣಿ-ಸಿ-ಕೊ-ಳ್ಳಲು
ಜೀರ್ಣಿ-ಸಿ-ಕೊ-ಳ್ಳುತ್ತ
ಜೀವ
ಜೀವಂತ
ಜೀವಂ-ತ-ವಾ-ಗಿ-ರ-ಲಾ-ರದು
ಜೀವಕ್ಕೂ
ಜೀವಕ್ಕೆ
ಜೀವ-ಗ-ತಿ-ಯನ್ನು
ಜೀವ-ಗಳ
ಜೀವ-ಗ-ಳಿಗೂ
ಜೀವ-ಗ-ಳೆ-ರ-ಡಕ್ಕೂ
ಜೀವ-ಗ-ಳೆಲ್ಲ
ಜೀವ-ಜಂ-ತು-ಗಳನ್ನು
ಜೀವ-ಜಂ-ತು-ಗ-ಳಿ-ಲ್ಲವೆ
ಜೀವ-ಜಂ-ತು-ಗಳು
ಜೀವ-ಜಂ-ತು-ಗಳೂ
ಜೀವತಿ
ಜೀವದ
ಜೀವ-ದಾ-ನ-ವಾ-ಗು-ವುದು
ಜೀವ-ಧಾರಿ
ಜೀವನ
ಜೀವನಂ
ಜೀವ-ನಕ್ಕೆ
ಜೀವ-ನ-ಗ-ತಿ-ಯಲ್ಲಿ
ಜೀವ-ನ-ಗಳ
ಜೀವ-ನದ
ಜೀವ-ನ-ದ-ಮೇಲೆ
ಜೀವ-ನ-ದ-ಲ್ಲಂತೂ
ಜೀವ-ನ-ದ-ಲ್ಲಾ-ದರೊ
ಜೀವ-ನ-ದಲ್ಲಿ
ಜೀವ-ನ-ದ-ಲ್ಲಿಯೂ
ಜೀವ-ನ-ದ-ಲ್ಲಿಯೇ
ಜೀವ-ನ-ದ-ಲ್ಲಿ-ರು-ವಾಗ
ಜೀವ-ನ-ದಲ್ಲೆ
ಜೀವ-ನ-ದಲ್ಲೆಲ್ಲಾ
ಜೀವ-ನ-ದಿಂದ
ಜೀವ-ನ-ದೋ-ಣಿಯ
ಜೀವ-ನನ್ನು
ಜೀವ-ನ-ಯಾತ್ರೆ
ಜೀವ-ನಲ್ಲ
ಜೀವ-ನ-ವನ್ನು
ಜೀವ-ನ-ವ-ನ್ನೆಲ್ಲ
ಜೀವ-ನ-ವನ್ನೇ
ಜೀವ-ನವೂ
ಜೀವ-ನವೆ
ಜೀವ-ನ-ವೆಲ್ಲ
ಜೀವ-ನವೇ
ಜೀವ-ನಾಗಿ
ಜೀವ-ನಾ-ದರೋ
ಜೀವ-ನಾ-ಧಾರ
ಜೀವ-ನಿಗೆ
ಜೀವನು
ಜೀವನೂ
ಜೀವನೋ
ಜೀವ-ನೋ-ಪಾಯ
ಜೀವ-ನೋ-ಪಾ-ಯ-ಕ್ಕಾಗಿ
ಜೀವ-ನೋ-ಪಾ-ಯಕ್ಕೆ
ಜೀವ-ಭೂತಃ
ಜೀವ-ಭೂ-ತಾಂ
ಜೀವರ
ಜೀವ-ರನ್ನು
ಜೀವ-ರಾಶಿ
ಜೀವ-ರಾ-ಶಿ-ಗಳ
ಜೀವ-ರಾ-ಶಿ-ಗಳನ್ನು
ಜೀವ-ರಾ-ಶಿ-ಗಳಲ್ಲಿ
ಜೀವ-ರಾ-ಶಿ-ಗ-ಳ-ಲ್ಲಿಯೂ
ಜೀವ-ರಾ-ಶಿ-ಗ-ಳ-ಲ್ಲಿ-ರುವ
ಜೀವ-ರಾ-ಶಿ-ಗ-ಳಲ್ಲೂ
ಜೀವ-ರಾ-ಶಿ-ಗ-ಳಾಗಿ
ಜೀವ-ರಾ-ಶಿ-ಗ-ಳಿ-ಗಾ-ಗಲಿ
ಜೀವ-ರಾ-ಶಿ-ಗ-ಳಿ-ಗಾಗಿ
ಜೀವ-ರಾ-ಶಿ-ಗ-ಳಿ-ಗಾ-ದರೊ
ಜೀವ-ರಾ-ಶಿ-ಗ-ಳಿಗೂ
ಜೀವ-ರಾ-ಶಿ-ಗ-ಳಿಗೆ
ಜೀವ-ರಾ-ಶಿ-ಗ-ಳಿ-ಗೆಲ್ಲ
ಜೀವ-ರಾ-ಶಿ-ಗ-ಳಿ-ಗೆಲ್ಲಾ
ಜೀವ-ರಾ-ಶಿ-ಗ-ಳಿವೆ
ಜೀವ-ರಾ-ಶಿ-ಗಳು
ಜೀವ-ರಾ-ಶಿ-ಗಳೂ
ಜೀವ-ರಾ-ಶಿ-ಗ-ಳೆಲ್ಲ
ಜೀವ-ರಾ-ಶಿ-ಗ-ಳೆಲ್ಲಾ
ಜೀವ-ರಾ-ಶಿಗೂ
ಜೀವ-ರಾ-ಶಿಗೆ
ಜೀವ-ರಾ-ಶಿಯ
ಜೀವ-ರಾ-ಶಿ-ಯ-ಲ್ಲಿಯೂ
ಜೀವ-ರಾ-ಶಿಯೂ
ಜೀವ-ರಿಗೂ
ಜೀವ-ರಿಗೆ
ಜೀವ-ರಿ-ಗೆಲ್ಲ
ಜೀವರು
ಜೀವರೂ
ಜೀವ-ರೂ-ಪವೂ
ಜೀವ-ಲೋಕೇ
ಜೀವ-ವನ್ನು
ಜೀವ-ವನ್ನೂ
ಜೀವ-ವಾ-ದರೊ
ಜೀವ-ವಿದೆ
ಜೀವ-ವಿಮೆ
ಜೀವ-ವಿ-ರು-ವ-ವ-ರೆಗೆ
ಜೀವ-ವಿ-ಲ್ಲದ
ಜೀವವೂ
ಜೀವ-ವೆಂ-ದ-ರೇನು
ಜೀವವೇ
ಜೀವ-ಸ-ಹಿತ
ಜೀವ-ಸ್ವ-ರೂ-ಪನೇ
ಜೀವ-ಹ-ರಣ
ಜೀವಾ-ಣು-ವಾ-ಗಲೀ
ಜೀವಾ-ಣು-ವಿ-ನಲ್ಲಿ
ಜೀವಾತ್ಮ
ಜೀವಾ-ತ್ಮನ
ಜೀವಾ-ತ್ಮ-ನನ್ನು
ಜೀವಾ-ತ್ಮ-ನಿ-ಗಲ್ಲ
ಜೀವಾ-ತ್ಮ-ನಿಗೂ
ಜೀವಾ-ತ್ಮ-ನಿಗೆ
ಜೀವಾ-ತ್ಮರ
ಜೀವಾ-ತ್ಮ-ರು-ಗ-ಳೆಲ್ಲ
ಜೀವಾ-ತ್ಮ-ರೆ-ಲ್ಲರೂ
ಜೀವಾಳ
ಜೀವಾ-ಳ-ವಾ-ಗಿ-ರು-ವ-ವನೇ
ಜೀವಾ-ವಧಿ
ಜೀವಿ
ಜೀವಿ-ಗಳ
ಜೀವಿ-ಗಳನ್ನು
ಜೀವಿ-ಗಳನ್ನೆಲ್ಲ
ಜೀವಿ-ಗಳಲ್ಲಿ
ಜೀವಿ-ಗ-ಳ-ಲ್ಲಿ-ರುವ
ಜೀವಿ-ಗ-ಳಿ-ಗಾ-ದರೊ
ಜೀವಿ-ಗ-ಳಿ-ಗಿ-ಲ್ಲದ
ಜೀವಿ-ಗ-ಳಿಗೂ
ಜೀವಿ-ಗ-ಳಿಗೆ
ಜೀವಿ-ಗ-ಳಿ-ಗೆಲ್ಲ
ಜೀವಿ-ಗ-ಳಿ-ದ್ದಾರೆ
ಜೀವಿ-ಗಳು
ಜೀವಿ-ಗಳೂ
ಜೀವಿ-ಗ-ಳೆಂಬ
ಜೀವಿ-ಗ-ಳೆ-ದೆ-ಯಲ್ಲಿ
ಜೀವಿಗೂ
ಜೀವಿಗೆ
ಜೀವಿಯ
ಜೀವಿ-ಯನ್ನು
ಜೀವಿ-ಯಲ್ಲಿ
ಜೀವಿ-ಯ-ಲ್ಲಿಯೂ
ಜೀವಿ-ಯ-ಲ್ಲಿ-ರುವ
ಜೀವಿ-ಯಲ್ಲೂ
ಜೀವಿ-ಯಾ-ದರೂ
ಜೀವಿಯು
ಜೀವಿಯೂ
ಜೀವಿಯೇ
ಜೀವಿ-ಸ-ಬ-ಹುದು
ಜೀವಿ-ಸ-ಬೇ-ಕಾ-ಗಿದೆ
ಜೀವಿ-ಸ-ಲಾ-ರರು
ಜೀವಿ-ಸುವ
ಜೀವಿ-ಸು-ವು-ದ-ರಿಂ-ದಲೂ
ಜೀವಿ-ಸು-ವುದು
ಜೀವೇ-ಶ್ವರ
ಜುಗುಪ್ಸೆ
ಜುಗು-ಪ್ಸೆ-ಯಾ-ಗು-ವು-ದಿಲ್ಲ
ಜುಗು-ಪ್ಸೆ-ಯಿಲ್ಲ
ಜುಗು-ಪ್ಸೆಯೂ
ಜುಟ್ಟನ್ನು
ಜುಮ್
ಜುಹ್ವತಿ
ಜೂಜಾಡಿ
ಜೂಜು
ಜೆ
ಜೇಡ
ಜೇಡನ
ಜೇಡರ
ಜೇಡಿ
ಜೇಡಿ-ಮ-ಣ್ಣಲ್ಲಿ
ಜೇಡಿ-ಮ-ಣ್ಣಿನ
ಜೇಡಿ-ಮ-ಣ್ಣಿ-ನಿಂದ
ಜೇಡಿ-ಮಣ್ಣು
ಜೇತಾಸಿ
ಜೇನಿ-ನಿಂದ
ಜೇನು-ಗೂ-ಡನ್ನು
ಜೇನು-ಗೂ-ಡೊಂ-ದ-ರಿಂದ
ಜೇನು-ತುಪ್ಪ
ಜೇನು-ನೊಣ
ಜೇನು-ನೊ-ಣ-ಗಳೇ
ಜೇಬಿಗೆ
ಜೇಬಿನ
ಜೇಬಿ-ನ-ಲ್ಲಿ-ಟ್ಟು-ರುವ
ಜೇಬಿ-ನ-ಲ್ಲಿದೆ
ಜೇಬಿ-ನ-ಲ್ಲಿ-ರುವ
ಜೇಬಿ-ನಿಂದ
ಜೇಬಿ-ನೊ-ಳಗೆ
ಜೈನ
ಜೈನ-ರಲ್ಲಿ
ಜೈಲಿಗೆ
ಜೈಲಿ-ನಲ್ಲಿ
ಜೈಲಿ-ನ-ಲ್ಲಿಯೂ
ಜೈಲಿ-ನ-ಲ್ಲಿ-ರ-ಬೇಕು
ಜೈಲಿ-ನಿಂದ
ಜೈಲು
ಜೈಲು-ವಾ-ಸ-ವನ್ನು
ಜೊತೆ
ಜೊತೆ-ಗಿ-ರುವ
ಜೊತೆಗೆ
ಜೊತೆ-ಜೊ-ತೆ-ಯ-ಲ್ಲಿಯೇ
ಜೊತೆ-ಯಲ್ಲ
ಜೊತೆ-ಯಲ್ಲಿ
ಜೊತೆ-ಯ-ಲ್ಲಿಯೇ
ಜೊತೆ-ಯಾ-ದರೂ
ಜೊಳ್ಳನ್ನು
ಜೊಳ್ಳು
ಜೊಳ್ಳೆಲ್ಲಾ
ಜೋಕೆ
ಜೋಗ-ದ-ಜ-ಲ-ಪಾ-ತ-ದಲ್ಲಿ
ಜೋಡಿಸಿ
ಜೋಡಿ-ಸಿ-ಕೊಂಡು
ಜೋಡಿ-ಸಿದ
ಜೋಡಿ-ಸಿದೆ
ಜೋಡಿ-ಸಿ-ದ್ದಾನೆ
ಜೋಡಿ-ಸಿದ್ದೇ
ಜೋಡಿ-ಸಿ-ರು-ವನು
ಜೋಡಿ-ಸಿ-ರು-ವರು
ಜೋಡಿ-ಸು-ತ್ತಾನೆ
ಜೋಡಿ-ಸು-ವುದು
ಜೋಪ-ಡಿ-ಯಂತೆ
ಜೋಪ-ನ-ವಾ-ಗಿ-ರ-ಬೇಕು
ಜೋಪಾ-ನ-ವಾಗಿ
ಜೋಪಾ-ನ-ವಾ-ಗಿ-ದ್ದರೂ
ಜೋಪಾ-ನ-ವಾ-ಗಿರ
ಜೋಪಾ-ನ-ವಾ-ಗಿ-ರ-ಬೇಕು
ಜೋಪಾ-ನ-ವಾ-ಗಿ-ರಲಿ
ಜೋಪಾ-ನ-ವಾ-ಗಿರು
ಜೋರಾಗಿ
ಜೋರಿಗೆ
ಜೋಲಾ-ಡು-ತ್ತಿ-ರು-ವನು
ಜೋಳಿಗೆ
ಜೋಷ-ಯೇತ್
ಜೌಗು
ಜ್ಞಾತುಂ
ಜ್ಞಾತೃ
ಜ್ಞಾತೃವೂ
ಜ್ಞಾತೇನ
ಜ್ಞಾತ್ವಾ
ಜ್ಞಾನ
ಜ್ಞಾನ-ವಿ-ಜ್ಞಾ-ನ-ಯೋಗ
ಜ್ಞಾನಂ
ಜ್ಞಾನ-ಕಾಂ-ಡ-ಗ-ಳೆಂದು
ಜ್ಞಾನ-ಕಾಂ-ಡ-ದಲ್ಲಿ
ಜ್ಞಾನ-ಕ್ಕಿಂತ
ಜ್ಞಾನಕ್ಕೂ
ಜ್ಞಾನಕ್ಕೆ
ಜ್ಞಾನ-ಖ-ಡ್ಗ-ದಿಂದ
ಜ್ಞಾನ-ಗಮ್ಯ
ಜ್ಞಾನ-ಗಮ್ಯಂ
ಜ್ಞಾನ-ಗಳ
ಜ್ಞಾನ-ಗಳನ್ನೆಲ್ಲಾ
ಜ್ಞಾನ-ಗಳಲ್ಲಿ
ಜ್ಞಾನ-ಗ-ಳಾ-ದರೊ
ಜ್ಞಾನ-ಗ-ಳಿವೆ
ಜ್ಞಾನ-ಗಳೂ
ಜ್ಞಾನ-ಗ-ಳೆಲ್ಲ
ಜ್ಞಾನ-ಚ-ಕ್ಷುಷಃ
ಜ್ಞಾನ-ಚ-ಕ್ಷುಷಾ
ಜ್ಞಾನ-ಜಲ
ಜ್ಞಾನ-ಜ್ಯೋತಿ
ಜ್ಞಾನ-ತ-ಪಸಾ
ಜ್ಞಾನ-ತ-ಪಸ್ಸು
ಜ್ಞಾನದ
ಜ್ಞಾನ-ದಂತೆ
ಜ್ಞಾನ-ದ-ಲ್ಲಾ-ದರೋ
ಜ್ಞಾನ-ದಲ್ಲಿ
ಜ್ಞಾನ-ದ-ಲ್ಲಿಯೇ
ಜ್ಞಾನ-ದ-ಲ್ಲಿ-ರು-ವಾಗ
ಜ್ಞಾನ-ದಿಂದ
ಜ್ಞಾನ-ದಿಂ-ದಲೂ
ಜ್ಞಾನ-ದಿಂ-ದಲೆ
ಜ್ಞಾನ-ದೀ-ಪಿತೇ
ಜ್ಞಾನ-ದೀ-ಪೇನ
ಜ್ಞಾನ-ದೃಷ್ಟಿ
ಜ್ಞಾನ-ದೃ-ಷ್ಟಿ-ಯಿಂದ
ಜ್ಞಾನ-ನಿ-ರ್ಧೂ-ತ-ಕ-ಲ್ಮ-ಷಾಃ
ಜ್ಞಾನ-ನಿ-ಷ್ಠರು
ಜ್ಞಾನ-ನಿಷ್ಠೆ
ಜ್ಞಾನ-ನಿ-ಷ್ಠೆ-ಯಲ್ಲಿ
ಜ್ಞಾನ-ಪಂ-ಜಿ-ನಂತೆ
ಜ್ಞಾನ-ಪ್ಲ-ವೇ-ನೈವ
ಜ್ಞಾನ-ಭಾ-ಸ್ಕ-ರ-ನಂತೆ
ಜ್ಞಾನ-ಭಾ-ಸ್ಕ-ರ-ನೆ-ದು-ರಿಗೆ
ಜ್ಞಾನ-ಭಾ-ಸ್ಕ-ರನೇ
ಜ್ಞಾನ-ಮ-ಭ್ಯಾ-ಸಾ-ಜ್ಜ್ಞಾ-ನಾ-ದ್ಧ್ಯಾನಂ
ಜ್ಞಾನ-ಮ-ಯ-ಪ್ರ-ದೀಪಃ
ಜ್ಞಾನ-ಮ-ಯ-ವಾದ
ಜ್ಞಾನ-ಮಾ-ಖ್ಯಾತಂ
ಜ್ಞಾನ-ಮಾ-ಧುರ್ಯ
ಜ್ಞಾನ-ಮಾರ್ಗ
ಜ್ಞಾನ-ಮಾ-ರ್ಗ-ದಲ್ಲಿ
ಜ್ಞಾನ-ಮಾ-ವೃತ್ಯ
ಜ್ಞಾನ-ಮು-ತ್ತ-ಮಮ್
ಜ್ಞಾನ-ಮು-ದ್ರಾಯ
ಜ್ಞಾನ-ಮು-ದ್ರೆ-ಯನ್ನು
ಜ್ಞಾನ-ಮು-ಪಾ-ಶ್ರಿತ್ಯ
ಜ್ಞಾನ-ಮೇ-ತೇನ
ಜ್ಞಾನ-ಯಜ್ಞ
ಜ್ಞಾನ-ಯ-ಜ್ಞ-ಗಳನ್ನು
ಜ್ಞಾನ-ಯ-ಜ್ಞದ
ಜ್ಞಾನ-ಯ-ಜ್ಞ-ವನ್ನು
ಜ್ಞಾನ-ಯ-ಜ್ಞ-ವಾ-ಗಲಿ
ಜ್ಞಾನ-ಯ-ಜ್ಞ-ವಿದೆ
ಜ್ಞಾನ-ಯ-ಜ್ಞೇನ
ಜ್ಞಾನ-ಯೋಗ
ಜ್ಞಾನ-ಯೋ-ಗ-ದಿಂದ
ಜ್ಞಾನ-ಯೋ-ಗ-ದಿಂ-ದಲೂ
ಜ್ಞಾನ-ಯೋ-ಗ-ವನ್ನು
ಜ್ಞಾನ-ಯೋ-ಗ-ವ್ಯ-ವ-ಸ್ಥಿತಿ
ಜ್ಞಾನ-ಯೋ-ಗಿಯ
ಜ್ಞಾನ-ಯೋ-ಗೇನ
ಜ್ಞಾನ-ರಾಶಿ
ಜ್ಞಾನ-ರೂ-ಪದ
ಜ್ಞಾನ-ರೂ-ಪ-ವಾದ
ಜ್ಞಾನ-ವ-ತಾ-ಮ-ಹಮ್
ಜ್ಞಾನ-ವನ್ನು
ಜ್ಞಾನ-ವನ್ನೂ
ಜ್ಞಾನ-ವನ್ನೋ
ಜ್ಞಾನ-ವಲ್ಲ
ಜ್ಞಾನ-ವಾ-ಗ-ಬೇ-ಕಾ-ದರೆ
ಜ್ಞಾನ-ವಾ-ಗಲಿ
ಜ್ಞಾನ-ವಾ-ಗಲೀ
ಜ್ಞಾನ-ವಾ-ಗು-ವು-ದಿಲ್ಲ
ಜ್ಞಾನ-ವಾ-ದರೂ
ಜ್ಞಾನ-ವಾನ್
ಜ್ಞಾನ-ವಿ-ಜ್ಞಾ-ನ-ತೃ-ಪ್ತಾತ್ಮಾ
ಜ್ಞಾನ-ವಿ-ಜ್ಞಾ-ನ-ನಾ-ಶ-ನಮ್
ಜ್ಞಾನ-ವಿದೆ
ಜ್ಞಾನ-ವಿ-ದೆಯೊ
ಜ್ಞಾನ-ವಿ-ದೆಯೋ
ಜ್ಞಾನ-ವಿನ್ನೂ
ಜ್ಞಾನ-ವಿ-ರ-ಬೇ-ಕಾ-ದರೆ
ಜ್ಞಾನ-ವಿಲ್ಲ
ಜ್ಞಾನ-ವಿ-ಲ್ಲ-ದ-ವನು
ಜ್ಞಾನ-ವಿ-ಲ್ಲದೆ
ಜ್ಞಾನ-ವಿ-ಲ್ಲದೇ
ಜ್ಞಾನವು
ಜ್ಞಾನವೂ
ಜ್ಞಾನವೆ
ಜ್ಞಾನ-ವೆಂದರೆ
ಜ್ಞಾನ-ವೆಂದು
ಜ್ಞಾನ-ವೆಂಬ
ಜ್ಞಾನ-ವೆಂ-ಬುದು
ಜ್ಞಾನ-ವೆ-ನ್ನು-ವುದು
ಜ್ಞಾನ-ವೆಲ್ಲ
ಜ್ಞಾನವೇ
ಜ್ಞಾನವೊ
ಜ್ಞಾನವೋ
ಜ್ಞಾನ-ಸಂ-ಗ-ದಿಂ-ದಲೂ
ಜ್ಞಾನ-ಸಂ-ಗೇನ
ಜ್ಞಾನ-ಸಂ-ಛಿ-ನ್ನ-ಸಂ-ಶ-ಯಮ್
ಜ್ಞಾನ-ಸಂ-ಪಾ-ದನೆ
ಜ್ಞಾನಸ್ಯ
ಜ್ಞಾನಾಗ್ನಿ
ಜ್ಞಾನಾ-ಗ್ನಿಃ
ಜ್ಞಾನಾ-ಗ್ನಿಗೆ
ಜ್ಞಾನಾ-ಗ್ನಿ-ದ-ಗ್ಧ-ಕ-ರ್ಮಾಣಂ
ಜ್ಞಾನಾ-ಗ್ನಿ-ಯಲ್ಲಿ
ಜ್ಞಾನಾ-ಗ್ನಿ-ಯಿಂದ
ಜ್ಞಾನಾ-ನಾಂ
ಜ್ಞಾನಾ-ರ್ಜ-ನೆ-ಗ-ಳಿಗೆ
ಜ್ಞಾನಾ-ರ್ಜ-ನೆಗೆ
ಜ್ಞಾನಾ-ರ್ಜ-ನೆಯೇ
ಜ್ಞಾನಾ-ವ-ಸ್ಥಿ-ತ-ಚೇ-ತಸಃ
ಜ್ಞಾನಾ-ಸಿ-ನಾ-ತ್ಮನಃ
ಜ್ಞಾನಿ
ಜ್ಞಾನಿ-ಗಳ
ಜ್ಞಾನಿ-ಗ-ಳಂತೆ
ಜ್ಞಾನಿ-ಗಳಲ್ಲಿ
ಜ್ಞಾನಿ-ಗ-ಳಿ-ಗಿಂತ
ಜ್ಞಾನಿ-ಗ-ಳಿಗೆ
ಜ್ಞಾನಿ-ಗ-ಳಿಗೇ
ಜ್ಞಾನಿ-ಗಳು
ಜ್ಞಾನಿ-ಗ-ಳೆಂದು
ಜ್ಞಾನಿ-ಗಾ-ದರೊ
ಜ್ಞಾನಿ-ಗಿಂತ
ಜ್ಞಾನಿಗೆ
ಜ್ಞಾನಿ-ನ-ಸ್ತ-ತ್ತ್ವ-ದ-ರ್ಶಿನಃ
ಜ್ಞಾನಿನೋ
ಜ್ಞಾನಿ-ನೋ-ಽತ್ಯ-ರ್ಥ-ಮಹಂ
ಜ್ಞಾನಿ-ಭ್ಯೋಽಪಿ
ಜ್ಞಾನಿಯ
ಜ್ಞಾನಿ-ಯಂತೆ
ಜ್ಞಾನಿ-ಯದು
ಜ್ಞಾನಿ-ಯಲ್ಲಿ
ಜ್ಞಾನಿ-ಯಾಗಿ
ಜ್ಞಾನಿ-ಯಾ-ಗಿ-ದ್ದರೆ
ಜ್ಞಾನಿ-ಯಾಗಿದ್ದಾನೆ
ಜ್ಞಾನಿ-ಯಾ-ಗಿ-ರು-ವನು
ಜ್ಞಾನಿ-ಯಾ-ಗು-ವು-ದಿಲ್ಲ
ಜ್ಞಾನಿ-ಯಾದ
ಜ್ಞಾನಿ-ಯಾ-ದರೂ
ಜ್ಞಾನಿ-ಯಾ-ದರೊ
ಜ್ಞಾನಿ-ಯಾ-ದರೋ
ಜ್ಞಾನಿ-ಯಾ-ದ-ವ-ನಿಗೆ
ಜ್ಞಾನಿ-ಯಾ-ದ-ವನು
ಜ್ಞಾನಿಯು
ಜ್ಞಾನಿಯೂ
ಜ್ಞಾನಿಯೇ
ಜ್ಞಾನಿ-ಯೊ-ಬ್ಬ-ನಿದ್ದ
ಜ್ಞಾನೀ
ಜ್ಞಾನೇ
ಜ್ಞಾನೇಂ-ದ್ರಿಯ
ಜ್ಞಾನೇಂ-ದ್ರಿ-ಯಕ್ಕೆ
ಜ್ಞಾನೇಂ-ದ್ರಿ-ಯ-ಗ-ಳಿವೆ
ಜ್ಞಾನೇಂ-ದ್ರಿ-ಯ-ಗಳು
ಜ್ಞಾನೇಂ-ದ್ರಿ-ಯ-ಗ-ಳೆಲ್ಲಾ
ಜ್ಞಾನೇಂ-ದ್ರಿ-ಯ-ಗಳೇ
ಜ್ಞಾನೇಂ-ದ್ರಿ-ಯದ
ಜ್ಞಾನೇಂ-ದ್ರಿ-ಯ-ದಿಂದ
ಜ್ಞಾನೇನ
ಜ್ಞಾನೋ-ದ-ಯ-ವಾ-ಗಿದೆ
ಜ್ಞಾನೋ-ದ-ಯ-ವಾದ
ಜ್ಞಾನೋ-ದ-ಯ-ವಾ-ದರೆ
ಜ್ಞಾನೋ-ದ-ಯ-ವಾ-ಯಿತು
ಜ್ಞಾಪಕ
ಜ್ಞಾಪ-ಕಕ್ಕೆ
ಜ್ಞಾಪ-ಕ-ಕ್ಕೋ-ಸ್ಕರ
ಜ್ಞಾಪ-ಕ-ದಲ್ಲಿ
ಜ್ಞಾಪ-ಕ-ದ-ಲ್ಲಿ-ಡ-ಬೇ-ಕಾ-ದರೆ
ಜ್ಞಾಪ-ಕ-ದ-ಲ್ಲಿ-ಡು-ವುದು
ಜ್ಞಾಪ-ಕ-ವಿ-ರು-ವುದು
ಜ್ಞಾಪ-ಕ-ವಿಲ್ಲ
ಜ್ಞಾಪ-ಕವೇ
ಜ್ಞಾಪಿ-ಸ-ಬೇ-ಕಾ-ಗಿದೆ
ಜ್ಞಾಪಿ-ಸಿ-ಕೊ-ಳ್ಳ-ಬೇಕು
ಜ್ಞಾಪಿ-ಸಿ-ಕೊ-ಳ್ಳು-ತ್ತಿ-ದ್ದರೆ
ಜ್ಞಾಪಿ-ಸಿ-ಕೊ-ಳ್ಳು-ತ್ತೇವೆ
ಜ್ಞಾಪಿ-ಸಿ-ಕೊ-ಳ್ಳು-ವು-ದಕ್ಕೆ
ಜ್ಞಾಪಿ-ಸಿ-ಕೊ-ಳ್ಳು-ವುದು
ಜ್ಞಾಪಿ-ಸು-ವು-ದ-ಕ್ಕಾಗಿ
ಜ್ಞಾಸ್ಯಸಿ
ಜ್ಞೇಯ
ಜ್ಞೇಯಂ
ಜ್ಞೇಯಃ
ಜ್ಞೇಯದ
ಜ್ಞೇಯ-ಮ-ಸ್ಮಾ-ಭಿಃ
ಜ್ಞೇಯ-ವಸ್ತು
ಜ್ಞೇಯ-ವ-ಸ್ತು-ವನ್ನು
ಜ್ಞೇಯವು
ಜ್ಞೇಯ-ವೆಂದು
ಜ್ಞೇಯವೋ
ಜ್ಞೇಯೋಽಸಿ
ಜ್ಯಾಯಸೀ
ಜ್ಯಾಯೋ
ಜ್ಯೋತಿ
ಜ್ಯೋತಿಃ
ಜ್ಯೋತಿ-ಗಳಲ್ಲಿ
ಜ್ಯೋತಿಗೆ
ಜ್ಯೋತಿಯ
ಜ್ಯೋತಿ-ಯನ್ನು
ಜ್ಯೋತಿ-ಯಿಂದ
ಜ್ಯೋತಿ-ವ-ತ್ಸ-ರ-ಗ-ಳಾಚೆ
ಜ್ಯೋತಿ-ವ-ತ್ಸ-ರದ
ಜ್ಯೋತಿ-ಷಾ-ಮಪಿ
ಜ್ಯೋತಿ-ಸ್ವ-ರೂ-ಪ-ನಾದ
ಜ್ವರ-ದಲ್ಲಿ
ಜ್ವಲನಂ
ಜ್ವಲಿ-ಸು-ತ್ತಿ-ದೆಯೋ
ಜ್ವಲಿ-ಸು-ವು-ದಿಲ್ಲ
ಜ್ವಾಲಾ
ಜ್ವಾಲಾ-ಮ-ಯ-ವಾ-ಗಿದೆ
ಜ್ವಾಲಾ-ಮುಖಿ
ಜ್ವಾಲಾ-ಮು-ಖಿ-ಗ-ಳಿವೆ
ಜ್ವಾಲಾ-ಮು-ಖಿಯ
ಜ್ವಾಲಾ-ಮು-ಖಿ-ಯಂತೆ
ಜ್ವಾಲಾ-ಮು-ಖಿ-ಯ-ಲ್ಲಿದೆ
ಜ್ವಾಲೆಗೆ
ಜ್ವಾಲೆಯ
ಜ್ವಾಲೆ-ಯಂತೆ
ಜ್ವಾಲೆ-ಯ-ಮೇಲೆ
ಜ್ವಾಲೆ-ಯೊಂದು
ಝಳ-ದಿಂದ
ಝಷಾ-ಣಾಂ
ಟನ್ನು
ಟನ್ನು-ಗ-ಳಷ್ಟು
ಟಾರ್ಚ್ಲೈ-ಟಿನ
ಟಾರ್ಚ್ಲೈಟ್
ಟಿಕೀ-ಟನ್ನು
ಟಿಕೀ-ಟಿ-ಲ್ಲದೆ
ಟಿಕೀಟು
ಟಿಕೇ-ಟನ್ನು
ಟಿಕೇಟು
ಟಿಪ್ಪ-ಣಿ-ಗ-ಳಿ-ಗೆಲ್ಲ
ಟಿವಿ
ಟೀ
ಟೀಕಿ-ಸ-ಬ-ಹುದು
ಟೀಕಿ-ಸಿ-ದರೂ
ಟೀಕಿ-ಸಿ-ದರೆ
ಟೀಕಿಸು
ಟೀಕಿ-ಸು-ತ್ತಾನೆ
ಟೀಕಿ-ಸುವ
ಟೀಕಿ-ಸು-ವರು
ಟೀಕಿ-ಸು-ವ-ವ-ನಲ್ಲ
ಟೀಕಿ-ಸು-ವ-ವ-ರಿಗೆ
ಟೀಕಿ-ಸು-ವುದು
ಟೀಕೆ
ಟೀಕೆ-ಗ-ಳಿಗೆ
ಟೀಕೆ-ಗಳು
ಟೀಕೆ-ಯನ್ನು
ಟೀಕೆಯೂ
ಟೆಲಿ-ಗ್ರಾ-ಫಿನ
ಟೆಲಿ-ಸ್ಕೋ-ಪನ್ನು
ಟೆಲಿ-ಸ್ಕೋ-ಪಿನ
ಟೇಬಲ್
ಟೈಪ್
ಟೈಫಾ-ಯಿಡ್
ಟೊಳ್ಳಾದ
ಟೊಳ್ಳು
ಟ್ಟಿರುವ
ಟ್ಯಾಂಕನ್ನು
ಟ್ಯಾಂಕಿ-ನಿಂದ
ಟ್ಯಾಂಕು
ಟ್ಯಾಕ್ಸ್ಗಳು
ಟ್ಯೂನ್
ಟ್ರಸ್ಟಿ-ಗಳು
ಟ್ರಿಗ್ನಾ-ಮೆಟ್ರಿ
ಠಕ್ಕು
ಠೀವಿ-ಯನ್ನು
ಠೀವಿ-ಯಿಂದ
ಡಂಭಾ-ಚಾ-ರ-ವಿ-ಲ್ಲ-ದಿ-ರು-ವುದು
ಡಂಭಾ-ಚಾರಿ
ಡಕಾ-ಯಿತ
ಡಬ್ಬಲ್
ಡಾಕ್ಟ-ರೇಟ್
ಡಿಗ್ರಿ
ಡಿಗ್ರಿ-ಯನ್ನು
ಡೈರೆ-ಕ್ಟ-ರಿಗೆ
ಡೈರೆ-ಕ್ಟರ್
ಡೊಂಕಾಗಿ
ಡೊಂಕಾಗೇ
ಡೊಂಕು
ಡೊಂಕೂ
ಡೊಂಕೆ
ಡೋಲು
ಡ್ರಾಯಿಂಗ್
ಡ್ರೆಸ್ಸಿ-ನಲ್ಲಿ
ಡ್ರೆಸ್ಸಿ-ನ-ಲ್ಲಿ-ರುವ
ಡ್ರೈವರು
ಢಂಬ-ದಿಂದ
ಢಂಬಾ
ಢಂಬಾ-ಚಾರ
ಢಂಬಾ-ಚಾ-ರಕ್ಕೆ
ಢಂಬಾ-ಚಾ-ರ-ವಿಲ್ಲ
ಢಂಬಾ-ಚಾ-ರ-ವಿ-ಲ್ಲ-ದ-ವನು
ಢಂಬಾ-ಚಾ-ರ-ವಿ-ಲ್ಲದೆ
ಣಿಜ್ಞ್ಜ-ಕಿ-ಖಿಗಿ
ತ
ತಂ
ತಂಗ-ಬ-ಹುದು
ತಂಗಲು
ತಂಗ-ಳಿನ
ತಂಗಳು
ತಂಗಾಳಿ
ತಂಗಾ-ಳಿ-ಯಂತೆ
ತಂಗಿಯ
ತಂಗಿ-ಯನ್ನು
ತಂಗುವ
ತಂಗು-ವು-ದಕ್ಕೆ
ತಂಟೆಗೇ
ತಂಡಕ್ಕೆ
ತಂಡ-ದ-ವರು
ತಂಡ-ವಿ-ತ್ತೆಂದು
ತಂತಿ
ತಂತಿಗೆ
ತಂತಿಯ
ತಂತಿ-ಯನ್ನು
ತಂತಿ-ಯಲ್ಲಿ
ತಂತಿ-ಯಿಂದ
ತಂತು-ಗಳನ್ನು
ತಂತು-ವನ್ನು
ತಂತು-ವಿ-ನಿಂದ
ತಂತ್ರ-ಗಳು
ತಂದದ್ದು
ತಂದನೋ
ತಂದರು
ತಂದರೆ
ತಂದಾಗ
ತಂದಾ-ಗಲೇ
ತಂದಿದೆ
ತಂದಿ-ರ-ಬ-ಹುದು
ತಂದಿ-ರ-ಬೇಕು
ತಂದಿರು
ತಂದಿ-ರು-ತ್ತಾ-ನೆಯೊ
ತಂದಿ-ರು-ವನು
ತಂದಿ-ರು-ವನೊ
ತಂದಿ-ರು-ವರು
ತಂದಿ-ರು-ವುದು
ತಂದಿ-ರುವೆ
ತಂದಿ-ರು-ವೆವೋ
ತಂದು
ತಂದು-ಕೊಡು
ತಂದು-ಕೊ-ಳ್ಳ-ಬೇಕು
ತಂದು-ಕೊ-ಳ್ಳು-ವು-ದಿಲ್ಲ
ತಂದೆ
ತಂದೆ-ಒ-ಳ್ಳೆ-ಯ-ವ-ನಿಗೆ
ತಂದೆಗೆ
ತಂದೆ-ತಾ-ಯಿ-ಗ-ಳ-ನ್ನಾ-ಗಲಿ
ತಂದೆ-ತಾ-ಯಿ-ಗಳಿಂದ
ತಂದೆ-ತಾ-ಯಿ-ಗ-ಳಿಗೆ
ತಂದೆ-ತಾ-ಯಿ-ಗಳು
ತಂದೆ-ತಾ-ಯಿ-ಯ-ರಿಗೆ
ತಂದೆಯ
ತಂದೆ-ಯಂತೆ
ತಂದೆ-ಯನ್ನು
ತಂದೆ-ಯಲ್ಲ
ತಂದೆ-ಯಾ-ಗಿ-ರು-ವನು
ತಂದೆ-ಯಾ-ಗು-ತ್ತಾ-ನೆಯೋ
ತಂದೆ-ಯಾದ
ತಂದೆ-ಯಾ-ದರೊ
ತಂದೆ-ಯಿಂದ
ತಂದೆಯು
ತಂದೆಯೇ
ತಂದೊ-ಡ-ನೆಯೇ
ತಂದೊಡ್ಡ
ತಂದೊ-ಡ್ಡ-ಲಿಲ್ಲ
ತಂದೊ-ಡ್ಡಿ-ದರೂ
ತಂದೊ-ಡ್ಡು-ವು-ದಿಲ್ಲ
ತಂದೊ-ಡ್ಡು-ವುದು
ತಂದೊ-ಡ್ಡು-ವೆವು
ತಂಪಾಗಿ
ತಂಪಾ-ಗಿ-ದೆಯೆ
ತಂಪಾ-ಗಿ-ದ್ದಂತೆ
ತಂಪಾ-ಗಿಯೂ
ತಂಪಾ-ಗಿರ
ತಂಪಿನ
ತಂಬೂ-ರಿ-ಯನ್ನು
ತಕಥೈ
ತಕ್ಕ
ತಕ್ಕಂ-ತಹ
ತಕ್ಕಂತೆ
ತಕ್ಕಡಿ
ತಕ್ಕ-ದ್ದನ್ನು
ತಕ್ಕದ್ದು
ತಕ್ಕೈ-ಸಿ-ದರೂ
ತಕ್ಷಣ
ತಕ್ಷ-ಣವೆ
ತಕ್ಷ-ಣವೇ
ತಗ-ಲಿ-ಹಾ-ಕಿ-ಕೊ-ಳ್ಳು-ವುವು
ತಗ-ಲಿ-ಹಾ-ಕು-ವನು
ತಗ-ಲಿ-ಹಾ-ಕು-ವೆವು
ತಗ-ಲು-ವು-ದಿಲ್ಲ
ತಗ-ಲು-ವುದು
ತಗ-ಲು-ಹಾ-ಕಿ-ರುವ
ತಗುಲಿ
ತಗು-ಲಿ-ಸ-ಬೇಕು
ತಗು-ಲಿ-ಹಾ-ಕಿ-ಕೊಂ-ಡಿ-ರುವ
ತಗು-ಲಿ-ಹಾ-ಕಿದ
ತಗು-ಲಿ-ಹಾ-ಕಿ-ರುವ
ತಗು-ಲು-ವುದು
ತಗೆ-ದು-ಕೊ-ಳ್ಳ-ಬೇಕು
ತಗ್ಗದೆ
ತಗ್ಗ-ಬೇಕು
ತಗ್ಗಲಿ
ತಗ್ಗ-ಲಿಲ್ಲ
ತಗ್ಗಿ-ದಾಗ
ತಗ್ಗಿನ
ತಗ್ಗಿ-ಸ-ಬೇ-ಕಾ-ಗು-ವುದು
ತಗ್ಗಿ-ಸ-ಲಾ-ರದು
ತಗ್ಗಿ-ಸು-ತ್ತೇವೆ
ತಗ್ಗಿ-ಸು-ವರೊ
ತಗ್ಗಿ-ಸು-ವು-ದಕ್ಕೆ
ತಗ್ಗು
ತಗ್ಗು-ಏ-ರು-ಗಳಲ್ಲಿ
ತಗ್ಗುತ್ತ
ತಗ್ಗು-ಭೂ-ಮಿಗೆ
ತಗ್ಗು-ವುದು
ತಗ್ಗು-ವುದೂ
ತಗ್ಗೂ
ತಚ್ಚ
ತಚ್ಛಕ್ಯಂ
ತಚ್ಛೃಣು
ತಜ್ಜ್ಞಾನಂ
ತಜ್ಜ್ಞೇಯಂ
ತಜ್ಜ್ಯೋ-ತಿ-ಸ್ತ-ಮಸಃ
ತಟ-ವಟ
ತಟ-ಸ್ಥ-ವಾ-ಗಿ-ರು-ವಂ-ತಿದೆ
ತಟಾಕ
ತಟಾ-ಕ-ವಾ-ದರೊ
ತಟ್ಟಿ
ತಟ್ಟಿದ
ತಟ್ಟಿ-ದಂತೆ
ತಟ್ಟಿದ್ದು
ತಟ್ಟು-ವುದು
ತಡ
ತಡ-ಕ-ಬೇ-ಕಾ-ದರೆ
ತಡಿ-ಯಲ್ಲಿ
ತಡಿ-ಯ-ಲ್ಲಿತ್ತು
ತಡೆ
ತಡೆ-ಗಟ್ಟ
ತಡೆ-ಗ-ಟ್ಟ-ಬೇಕು
ತಡೆ-ಗ-ಟ್ಟ-ಲಾ-ರರು
ತಡೆ-ಗ-ಟ್ಟಲು
ತಡೆ-ಗಟ್ಟಿ
ತಡೆ-ಗ-ಟ್ಟಿ-ದರೆ
ತಡೆ-ಗ-ಟ್ಟಿ-ದ-ರೇನೆ
ತಡೆ-ಗಟ್ಟು
ತಡೆ-ಗ-ಟ್ಟು-ತ್ತಾನೆ
ತಡೆ-ಗ-ಟ್ಟು-ವನು
ತಡೆ-ಗ-ಟ್ಟು-ವರು
ತಡೆ-ಗ-ಟ್ಟು-ವಾಗ
ತಡೆ-ಗ-ಟ್ಟು-ವು-ದಕ್ಕೆ
ತಡೆ-ಗ-ಟ್ಟು-ವುದು
ತಡೆ-ಗ-ಟ್ಟು-ವೆವೊ
ತಡೆ-ದರೆ
ತಡೆ-ದಾನು
ತಡೆದು
ತಡೆ-ದು-ಕೊ-ಳ್ಳಲು
ತಡೆ-ದುದು
ತಡೆ-ಯ-ಬೇಕು
ತಡೆ-ಯಲು
ತಡೆ-ಯಾ-ಗು-ವುದು
ತಡೆ-ಯು-ತ್ತಾನೆ
ತಡೆ-ಯು-ತ್ತೇನೆ
ತಡೆ-ಯು-ವನು
ತಡೆ-ಯು-ವು-ದಕ್ಕೂ
ತಡೆ-ಯು-ವು-ದಕ್ಕೆ
ತಡೆ-ಯು-ವು-ದರ
ತಡೆ-ಯು-ವು-ದಿಲ್ಲ
ತಡೆ-ಯು-ವುದು
ತಡೆಯೂ
ತಣಿ-ಸ-ಬೇಕು
ತಣಿ-ಸು-ವು-ದ-ಕ್ಕಾಗಿ
ತಣಿ-ಸು-ವು-ದಕ್ಕೆ
ತಣ್ಣ-ಗಾ-ಗು-ವುದು
ತಣ್ನೆ-ರಳ
ತತ
ತತಂ
ತತಃ
ತತ-ಮಿದಂ
ತತಮ್
ತತ-ಸ್ತತೋ
ತತಸ್ಸ
ತತೋ
ತತೋಽಸಿ
ತತ್
ತತ್ಕಾ
ತತ್ಕಾ-ಲಕ್ಕೆ
ತತ್ಕಾ-ಲ-ದಲ್ಲಿ
ತತ್ಕಾ-ಲಿ-ಕ-ವಾಗಿ
ತತ್ಕಿಂ
ತತ್ಕು-ರುಷ್ವ
ತತ್ಕ್ಷಣ
ತತ್ಕ್ಷ-ಣವೇ
ತತ್ತ
ತತ್ತ-ದೇ-ವಾ-ವ-ಗಚ್ಛ
ತತ್ತ-ರಿಸಿ
ತತ್ತ-ರಿ-ಸು-ತ್ತಾನೆ
ತತ್ತ-ರಿ-ಸು-ತ್ತಾರೆ
ತತ್ತ-ರಿ-ಸು-ವಂತೆ
ತತ್ತ-ರಿ-ಸು-ವುದು
ತತ್ತಾ-ಮ-ಸ-ಮು-ದಾ-ಹೃ-ತಮ್
ತತ್ತಾ-ಮ-ಸು-ದಾ-ಹೃ-ತಮ್
ತತ್ತೇ
ತತ್ತೇಜೋ
ತತ್ತ್ವ
ತತ್ತ್ವಕ್ಕೆ
ತತ್ತ್ವಕ್ಕೇ
ತತ್ತ್ವ-ಗಳ
ತತ್ತ್ವ-ಗಳನ್ನು
ತತ್ತ್ವ-ಗಳನ್ನೂ
ತತ್ತ್ವ-ಗ-ಳಿವೆ
ತತ್ತ್ವ-ಗಳು
ತತ್ತ್ವ-ಗಳೂ
ತತ್ತ್ವ-ಗ-ಳೆಲ್ಲ
ತತ್ತ್ವ-ಜ್ಞಾ-ನದ
ತತ್ತ್ವ-ಜ್ಞಾ-ನಾ-ರ್ಥ-ದ-ರ್ಶಮ್
ತತ್ತ್ವ-ಜ್ಞಾನಿ
ತತ್ತ್ವ-ಜ್ಞಾ-ನಿ-ಗಳು
ತತ್ತ್ವ-ಜ್ಞಾ-ನಿಗೆ
ತತ್ತ್ವ-ಜ್ಯೋ-ತಿ-ಯನ್ನು
ತತ್ತ್ವತಃ
ತತ್ತ್ವತೋ
ತತ್ತ್ವದ
ತತ್ತ್ವ-ದಂತೆ
ತತ್ತ್ವ-ದರ್ಶಿ
ತತ್ತ್ವ-ದಿಂದ
ತತ್ತ್ವ-ದೃ-ಷ್ಟಿ-ಕೋ-ನ-ಗಳ
ತತ್ತ್ವ-ದೃ-ಷ್ಟಿ-ಗ-ಳಿವೆ
ತತ್ತ್ವ-ಪ್ರ-ಪಂ-ಚ-ದಲ್ಲಿ
ತತ್ತ್ವ-ಬೋ-ಧನೆ
ತತ್ತ್ವ-ಭಾ-ಗ-ವ-ನ್ನೆಲ್ಲ
ತತ್ತ್ವ-ಮಿ-ಚ್ಛಾಮಿ
ತತ್ತ್ವ-ವ-ನ್ನಾ-ದರೂ
ತತ್ತ್ವ-ವನ್ನು
ತತ್ತ್ವ-ವಿತ್
ತತ್ತ್ವ-ವಿತ್ತು
ತತ್ತ್ವ-ವಿದೆ
ತತ್ತ್ವವೇ
ತತ್ತ್ವ-ಶಾಸ್ತ್ರ
ತತ್ತ್ವ-ಶಾ-ಸ್ತ್ರ-ಗಳು
ತತ್ತ್ವ-ಶಾ-ಸ್ತ್ರದ
ತತ್ತ್ವ-ಸೌ-ಧಕ್ಕೆ
ತತ್ತ್ವೇನ
ತತ್ತ್ವೇ-ನಾ-ತ-ಶ್ಚ್ಯ-ವಂತಿ
ತತ್ಪರಃ
ತತ್ಪ-ರತೆ
ತತ್ಪ-ರ-ನಾ-ಗಿ-ರು-ವನು
ತತ್ಪ-ರನೂ
ತತ್ಪ-ರಮ್
ತತ್ಪ-ರ-ರಾ-ಗಿ-ರು-ವಾಗ
ತತ್ಪ-ರ-ವಾ-ಗಿ-ರು-ವುದು
ತತ್ಪ-ರಾ-ಯ-ಣ-ರಾಗಿ
ತತ್ಪ-ರಿ-ಮಾ-ರ್ಗಿ-ತವ್ಯಂ
ತತ್ಪ-ರುಷ
ತತ್ಪ್ರ-ಸಾ-ದಾ-ತ್ಪ-ರಾಂ
ತತ್ಪ್ರೇತ್ಯ
ತತ್ರ
ತತ್ರಾ-ಪ-ಶ್ಯತ್
ತತ್ರೈ-ಕಸ್ಥಂ
ತತ್ರೈ-ಕಾ-ಗ್ರಮ್
ತತ್ರೈವಂ
ತತ್ರೈ-ವಾ-ವ್ಯ-ಕ್ತ-ಸಂ-ಜ್ಞಕೇ
ತತ್ವ
ತತ್ವ-ಗಳ
ತತ್ವ-ಜ್ಞಾನಿ
ತತ್ವ-ಜ್ಞಾ-ನಿ-ಯಾ-ಗಿದ್ದ
ತತ್ವ-ದರ್ಶಿ
ತತ್ವ-ದ-ರ್ಶಿ-ಗಳು
ತತ್ವ-ದ-ರ್ಶಿಯ
ತತ್ವ-ದ-ರ್ಶಿ-ಯಂತೆ
ತತ್ವ-ವನ್ನು
ತತ್ವ-ವನ್ನೇ
ತತ್ವವೂ
ತತ್ವ-ವೊಂದೇ
ತತ್ಸತ್
ತತ್ಸ-ದಿತಿ
ತತ್ಸ-ಮಕ್ಷಂ
ತತ್ಸ-ಮಾ-ಸೇನ
ತತ್ಸ-ರ್ವ-ತೋ-ಽಕ್ಷಿ-ಶಿ-ರೋ-ಮು-ಖಮ್
ತತ್ಸ-ರ್ವ-ಮಿದಂ
ತತ್ಸುಖಂ
ತಥಾ
ತಥಾತ್ಮಾ
ತಥಾ-ತ್ಮಾ-ನ-ಮ-ಕ-ರ್ತಾರಂ
ತಥಾ-ನ್ಯಾ-ನಪಿ
ತಥಾ-ಪರೇ
ತಥಾಪಿ
ತಥಾ-ಪ್ನೋತಿ
ತಥೈವ
ತಥೋ-ಪ-ಲ-ಭ್ಯತೇ
ತದ-ಜ್ಞಾನಂ
ತದ-ನಂ-ತ-ರಮ್
ತದರ್ಥಂ
ತದ-ರ್ಥ-ವಿ-ಜ್ಞಾನೇ
ತದ-ರ್ಥೀಯಂ
ತದ-ವಿ-ಜ್ಞೇಯಂ
ತದಸ್ತಿ
ತದ-ಸ್ಮಾಕಂ
ತದಸ್ಯ
ತದಹಂ
ತದ-ಹ-ಮ-ರ್ಜುನ
ತದಾ
ತದಾ-ತ್ಮ-ರಾಗಿ
ತದಾ-ತ್ಮಾನಂ
ತದಾತ್ಮ್ಯ
ತದಿ-ತ್ಯ-ನ-ಭಿ-ಸಂ-ಧಾಯ
ತದಿಹ
ತದೇಕ
ತದೇಕಂ
ತದೇವ
ತದೋ-ತ್ತ-ಮ-ವಿ-ದಾಂ
ತದ್ದಾನಂ
ತದ್ಧಾಮ
ತದ್ಬು-ದ್ಧ-ಯ-ಸ್ತ-ದಾ-ತ್ಮಾ-ನ-ಸ್ತ-ನ್ನಿ-ಷ್ಠಾ-ಸ್ತ-ತ್ಪ-ರಾ-ಯ-ಣಾಃ
ತದ್ಬು-ದ್ಧಿ-ಯು-ಳ್ಳ-ವ-ರಾಗಿ
ತದ್ಬ್ರಹ್ಮ
ತದ್ಭ-ವ-ತ್ಯ-ಲ್ಪ-ಮೇ-ಧ-ಸಾಮ್
ತದ್ಭಾ-ವ-ಭಾ-ವಿತಃ
ತದ್ಭಾ-ಸ-ಯತೇ
ತದ್ಯೋ-ಗೈ-ರಪಿ
ತದ್ರಾ-ಜ-ಸ-ಮು-ದಾ-ಹೃ-ತಮ್
ತದ್ವತ್
ತದ್ವಿದಃ
ತದ್ವಿ-ದುಃ
ತದ್ವಿದ್ಧಿ
ತನಕ
ತನ-ಕ-ವಾ-ದರೂ
ತನ-ಕವೂ
ತನ-ಗಲ್ಲ
ತನ-ಗಾಗಿ
ತನ-ಗಾ-ಗುವ
ತನ-ಗಾದ
ತನ-ಗಿಂತ
ತನ-ಗಿವೆ
ತನಗೂ
ತನಗೆ
ತನ-ಗೆಲ್ಲ
ತನಗೇ
ತನ-ಗೇ-ನಾ-ದರೂ
ತನ-ಗೇನೂ
ತನುಂ
ತನು-ಮಾ-ಶ್ರಿ-ತಮ್
ತನು-ವೇ-ರಿ-ದರೆ
ತನ್ನ
ತನ್ನಂ-ತಹ
ತನ್ನಂತೆ
ತನ್ನಂ-ತೆಯೇ
ತನ್ನ-ಡೆಗೆ
ತನ್ನ-ತ-ನ-ವನ್ನು
ತನ್ನ-ತ-ನ-ವನ್ನೇ
ತನ್ನ-ದ-ನ್ನಾಗಿ
ತನ್ನ-ದಲ್ಲ
ತನ್ನ-ದ-ಲ್ಲದ
ತನ್ನದು
ತನ್ನದೇ
ತನ್ನನ್ನು
ತನ್ನನ್ನೇ
ತನ್ನ-ಮೇಲೆ
ತನ್ನ-ಲ್ಲಾ-ಗುವ
ತನ್ನಲ್ಲಿ
ತನ್ನ-ಲ್ಲಿಗೆ
ತನ್ನ-ಲ್ಲಿದೆ
ತನ್ನ-ಲ್ಲಿಯೇ
ತನ್ನ-ಲ್ಲಿರು
ತನ್ನ-ಲ್ಲಿ-ರುವ
ತನ್ನ-ಲ್ಲಿ-ರು-ವನು
ತನ್ನ-ಲ್ಲಿ-ರುವು
ತನ್ನ-ಲ್ಲಿ-ರು-ವುದನ್ನು
ತನ್ನ-ಲ್ಲಿ-ರು-ವು-ದ-ನ್ನೆಲ್ಲಾ
ತನ್ನ-ಲ್ಲಿ-ರು-ವುದೇ
ತನ್ನ-ಲ್ಲಿವೆ
ತನ್ನಲ್ಲೇ
ತನ್ನ-ವನು
ತನ್ನ-ವರು
ತನ್ನವು
ತನ್ನಾತ್ಮ
ತನ್ನಿಂದ
ತನ್ನಿಂ-ದಲೇ
ತನ್ನಿ-ಚ್ಛೆ-ಯಂತೆ
ತನ್ನಿ-ಬ-ಧ್ನಾತಿ
ತನ್ನಿ-ಷ್ಠ-ರಾಗಿ
ತನ್ನೆ-ಡೆಗೆ
ತನ್ನೆ-ಡೆಗೇ
ತನ್ನೆ-ದು-ರಿ-ಗಿ-ರುವ
ತನ್ನೆ-ದು-ರಿಗೆ
ತನ್ನೊಂ-ದಿಗೆ
ತನ್ನೊ-ಳಗೆ
ತನ್ನೊ-ಳ-ಗೆಲ್ಲ
ತನ್ನೊ-ಳಗೇ
ತನ್ಮ-ಯತೆ
ತನ್ಮ-ಯ-ನಾಗಿ
ತನ್ಮ-ಯ-ನಾ-ಗು-ತ್ತಾನೆ
ತನ್ಮ-ಯ-ನಾ-ಗು-ವನು
ತನ್ಮ-ಯ-ರಾ-ಗ-ಬೇಕು
ತನ್ಮ-ಯ-ರಾ-ಗಿರು
ತನ್ಮ-ಯ-ರಾ-ಗುತ್ತಾ
ತನ್ಮ-ಯ-ರಾ-ಗು-ತ್ತಿ-ದ್ದರು
ತನ್ಮ-ಯ-ವಾ-ಗ-ಬೇಕು
ತನ್ಮ-ಯ-ವಾ-ಗಿ-ರು-ವುದು
ತನ್ಮ-ಯ-ವಾ-ಗುತ್ತ
ತನ್ಮ-ಯ-ವಾ-ಗು-ವುದೊ
ತನ್ಮಾತ್ರ
ತನ್ಮೇ
ತಪ
ತಪಂ-ತಮ್
ತಪಃ
ತಪಃ-ಕ್ರಿ-ಯಾಃ
ತಪಃ-ಕ್ಷೇತ್ರ
ತಪಃ-ಕ್ಷೇ-ತ್ರ-ವಾ-ದು-ದ-ರಿಂದ
ತಪಃ-ಶ-ಕ್ತಿ-ಯನ್ನು
ತಪಃಸು
ತಪ-ಶ್ಚಾಸ್ಮಿ
ತಪ-ಶ್ಚೈವ
ತಪಸಾ
ತಪ-ಸಾಂ
ತಪಸಿ
ತಪ-ಸ್ತ-ತ್ತ್ರಿ-ವಿಧಂ
ತಪ-ಸ್ತಪ್ತಂ
ತಪ-ಸ್ವಿ-ಗಳ
ತಪ-ಸ್ವಿ-ಗ-ಳ-ಲ್ಲಿ-ರುವ
ತಪ-ಸ್ವಿ-ಗ-ಳಿ-ಗಿಂತ
ತಪ-ಸ್ವಿ-ಗಳು
ತಪ-ಸ್ವಿ-ಭ್ಯೋ-ಽಧಿಕೋ
ತಪ-ಸ್ವಿಷು
ತಪ-ಸ್ಸನ್ನು
ತಪ-ಸ್ಸನ್ನೆ
ತಪ-ಸ್ಸನ್ನೇ
ತಪ-ಸ್ಸಾ-ಗ-ಬೇ-ಕಾ-ದರೆ
ತಪ-ಸ್ಸಾ-ಗಲಿ
ತಪ-ಸ್ಸಾ-ಗಿ-ರ-ಬ-ಹುದು
ತಪ-ಸ್ಸಾ-ದರೆ
ತಪಸ್ಸಿ
ತಪ-ಸ್ಸಿಗೆ
ತಪ-ಸ್ಸಿನ
ತಪ-ಸ್ಸಿ-ನಂತೆ
ತಪ-ಸ್ಸಿ-ನಲ್ಲಿ
ತಪ-ಸ್ಸಿ-ನ-ವ-ನಿಗೆ
ತಪ-ಸ್ಸಿ-ನಿಂದ
ತಪ-ಸ್ಸಿ-ನಿಂ-ದಲೂ
ತಪ-ಸ್ಸಿ-ನಿಂ-ದಲೆ
ತಪಸ್ಸು
ತಪ-ಸ್ಸು-ಗಳ
ತಪ-ಸ್ಸು-ಗ-ಳಿಂ-ದ-ಲಾ-ಗಲೀ
ತಪ-ಸ್ಸು-ಗ-ಳಿವೆ
ತಪ-ಸ್ಸು-ಗಳು
ತಪ-ಸ್ಸು-ಮಾಡಿ
ತಪ-ಸ್ಸು-ಮಾ-ಡು-ವ-ವರು
ತಪ-ಸ್ಸು-ಮಾ-ಡು-ವು-ದೊಂದು
ತಪಸ್ಸೂ
ತಪಸ್ಸೆ
ತಪಸ್ಸೇ
ತಪಾ-ಮ್ಯ-ಹ-ಮಹಂ
ತಪಿ-ಸಿ-ದಾ-ಗಲೆ
ತಪಿಸು
ತಪಿ-ಸು-ತ್ತಿದೆ
ತಪಿ-ಸು-ತ್ತಿ-ರು-ವ-ವನು
ತಪಿ-ಸು-ತ್ತಿ-ರು-ವ-ವನೂ
ತಪಿ-ಸು-ತ್ತೇನೆ
ತಪಿ-ಸು-ವು-ದಿಲ್ಲ
ತಪಿ-ಸು-ವುದು
ತಪಿ-ಸು-ವುದೂ
ತಪೋ
ತಪೋ-ಭಿ-ರು-ಗ್ರೈಃ
ತಪೋ-ಯಜ್ಞ
ತಪೋ-ಯ-ಜ್ಞ-ವನ್ನು
ತಪೋ-ಯ-ಜ್ಞ-ವಾ-ಗಲಿ
ತಪೋ-ರ-ಹಿ-ತ-ನಿಗೆ
ತಪೋ-ರೂ-ಪ-ವಾದ
ತಪೋ-ಶ-ಕ್ತಿಯೂ
ತಪ್ತಂ
ತಪ್ತ-ವಾ-ಗಿ-ರು-ವು-ದಿಲ್ಲ
ತಪ್ಪ-ದಂತೆ
ತಪ್ಪದೆ
ತಪ್ಪದೇ
ತಪ್ಪನ್ನು
ತಪ್ಪನ್ನೆ
ತಪ್ಪ-ನ್ನೆಲ್ಲ
ತಪ್ಪ-ನ್ನೆಲ್ಲಾ
ತಪ್ಪನ್ನೇ
ತಪ್ಪ-ಬ-ಹುದು
ತಪ್ಪಲ್ಲ
ತಪ್ಪಾಗಿ
ತಪ್ಪಾ-ಗು-ವುದು
ತಪ್ಪಾ-ದರೆ
ತಪ್ಪಿ
ತಪ್ಪಿಗೆ
ತಪ್ಪಿ-ತ-ಸ್ಥ-ನಾ-ಗು-ತ್ತಾನೆ
ತಪ್ಪಿ-ತ-ಸ್ಥನು
ತಪ್ಪಿ-ತ-ಸ್ಥ-ರನ್ನು
ತಪ್ಪಿ-ತ-ಸ್ಥ-ರಾ-ಗು-ತ್ತೇವೆ
ತಪ್ಪಿ-ತ-ಸ್ಥ-ರಾ-ಗು-ವೆವು
ತಪ್ಪಿ-ತ-ಸ್ಥರು
ತಪ್ಪಿ-ತ-ಸ್ಥರೆ
ತಪ್ಪಿ-ತ-ಸ್ಥರೇ
ತಪ್ಪಿ-ದನೋ
ತಪ್ಪಿ-ದರೂ
ತಪ್ಪಿ-ದರೆ
ತಪ್ಪಿ-ದ್ದಲ್ಲ
ತಪ್ಪಿನ
ತಪ್ಪಿ-ನಿಂದ
ತಪ್ಪಿ-ರ-ಬ-ಹುದು
ತಪ್ಪಿ-ರು-ವರು
ತಪ್ಪಿಸಿ
ತಪ್ಪಿ-ಸಿ-ಕೊಂ-ಡರೆ
ತಪ್ಪಿ-ಸಿ-ಕೊಂಡು
ತಪ್ಪಿ-ಸಿ-ಕೊ-ಳ್ಳ-ಬ-ಹುದು
ತಪ್ಪಿ-ಸಿ-ಕೊ-ಳ್ಳ-ಬೇ-ಕಾ-ದರೆ
ತಪ್ಪಿ-ಸಿ-ಕೊ-ಳ್ಳ-ಬೇಕು
ತಪ್ಪಿ-ಸಿ-ಕೊ-ಳ್ಳ-ಬೇ-ಕೆಂದು
ತಪ್ಪಿ-ಸಿ-ಕೊ-ಳ್ಳಲು
ತಪ್ಪಿ-ಸಿ-ಕೊಳ್ಳು
ತಪ್ಪಿ-ಸಿ-ಕೊ-ಳ್ಳು-ತ್ತೇವೆ
ತಪ್ಪಿ-ಸಿ-ಕೊ-ಳ್ಳು-ವಂ-ತಿ-ರ-ಲಿಲ್ಲ
ತಪ್ಪಿ-ಸಿ-ಕೊ-ಳ್ಳು-ವಂ-ತಿಲ್ಲ
ತಪ್ಪಿ-ಸಿ-ಕೊ-ಳ್ಳು-ವು-ದ-ಕ್ಕಾಗಿ
ತಪ್ಪಿ-ಸಿ-ಕೊ-ಳ್ಳು-ವು-ದ-ಕ್ಕಾ-ಗು-ವು-ದಿಲ್ಲ
ತಪ್ಪಿ-ಸಿ-ಕೊ-ಳ್ಳು-ವು-ದಕ್ಕೆ
ತಪ್ಪಿ-ಸಿ-ಕೊ-ಳ್ಳು-ವುದು
ತಪ್ಪಿ-ಸಿ-ಕೊ-ಳ್ಳು-ವುದೇ
ತಪ್ಪಿ-ಸು-ವು-ದಕ್ಕೆ
ತಪ್ಪು
ತಪ್ಪುಈ
ತಪ್ಪು-ಗಳನ್ನು
ತಪ್ಪು-ತ-ಪ್ಪಾಗಿ
ತಪ್ಪು-ತಿ-ಳಿ-ದು-ಕೊಂ-ಡಿ-ರ-ಬೇಕು
ತಪ್ಪು-ದಾರಿಗೆ
ತಪ್ಪು-ಮಾ-ಡ-ದ-ವರು
ತಪ್ಪು-ಮಾ-ಡಿ-ದರೆ
ತಪ್ಪು-ಮಾ-ಡು-ವನು
ತಪ್ಪು-ವು-ದಿಲ್ಲ
ತಪ್ಪು-ವುದು
ತಪ್ಪೆ
ತಪ್ಪೆಂದು
ತಪ್ಪೆಲ್ಲಾ
ತಪ್ಪೇ
ತಪ್ಪೋ
ತಪ್ಯಂತೇ
ತಬ-ಲ-ವನ್ನು
ತಬ್ಬಲಿ
ತಬ್ಬ-ಲಿ-ಗ-ಳಾ-ಗ-ಬ-ಹುದು
ತಬ್ಬ-ಲಿ-ಯರ
ತಬ್ಬಿ
ತಬ್ಬಿ-ಕೊ-ಳ್ಳು-ವಂ-ತಹ
ತಬ್ಬು-ವಂ-ತಹ
ತಬ್ಬು-ವಷ್ಟು
ತಮಃ
ತಮ-ಗಾಗಿ
ತಮ-ಗಿಂತ
ತಮಗೆ
ತಮಗೇ
ತಮ-ತ-ಮಗೆ
ತಮ-ಭ್ಯರ್ಚ್ಯ
ತಮ-ವಾದ
ತಮ-ಶ್ಚೈವ
ತಮಸಃ
ತಮ-ಸಾ-ವೃತಾ
ತಮಸೋ
ತಮಸ್
ತಮ-ಸ್ತ್ವ-ಜ್ಞಾ-ನಜಂ
ತಮ-ಸ್ಯೇ-ತಾನಿ
ತಮ-ಸ್ಸನ್ನು
ತಮ-ಸ್ಸಿ-ಗಿಂತ
ತಮ-ಸ್ಸಿಗೆ
ತಮ-ಸ್ಸಿನ
ತಮ-ಸ್ಸಿ-ನ-ಲ್ಲಿ-ರು-ವೆವು
ತಮ-ಸ್ಸಿ-ನಿಂದ
ತಮಸ್ಸು
ತಮಹಂ
ತಮಾಶೆ
ತಮಾಷೆ
ತಮಾ-ಹುಃ
ತಮು-ವಾಚ
ತಮೇವ
ತಮೇ-ವೈತಿ
ತಮೋ
ತಮೋ-ಗುಣ
ತಮೋ-ಗು-ಣಕ್ಕೆ
ತಮೋ-ಗು-ಣಕ್ಕೋ
ತಮೋ-ಗು-ಣ-ಗಳ
ತಮೋ-ಗು-ಣ-ಗಳನ್ನು
ತಮೋ-ಗು-ಣ-ಗಳು
ತಮೋ-ಗು-ಣದ
ತಮೋ-ಗು-ಣ-ದ-ಲ್ಲಿ-ರು-ವ-ವನು
ತಮೋ-ಗು-ಣ-ದಿಂದ
ತಮೋ-ಗು-ಣ-ವಾ-ದರೊ
ತಮೋ-ಗುಣಿ
ತಮೋ-ಗು-ಣಿ-ಗ-ಳಾ-ಗು-ವರು
ತಮೋ-ಗು-ಣಿ-ಗಳು
ತಮೋ-ಗು-ಣಿಗೂ
ತಮೋ-ಗು-ಣಿ-ಯಲ್ಲಿ
ತಮೋ-ಗು-ಣಿಯೂ
ತಮೋ-ದ್ವಾ-ರ-ಗಳಿಂದ
ತಮೋ-ದ್ವಾ-ರೈ-ಸ್ತ್ರಿ-ಭಿ-ರ್ನರಃ
ತಮೋ-ಧೈ-ರ್ಯ-ದಲ್ಲಿ
ತಮೋ-ರೀತಿ
ತಮ್ಮ
ತಮ್ಮಂ-ತೆಯೆ
ತಮ್ಮಂ-ತೆಯೇ
ತಮ್ಮ-ಇಂ-ದ್ರಿ-ಯ-ಗಳನ್ನು
ತಮ್ಮ-ದ-ನ್ನಾಗಿ
ತಮ್ಮದೇ
ತಮ್ಮ-ನಾ-ಗಿದ್ದ
ತಮ್ಮನ್ನು
ತಮ್ಮಲ್ಲಿ
ತಮ್ಮ-ಲ್ಲಿ-ರುವ
ತಮ್ಮ-ಲ್ಲಿ-ರು-ವ-ವ-ನೆಂದು
ತಯಾ
ತಯಾ-ಪ-ಹೃ-ತ-ಚೇ-ತ-ಸಾಮ್
ತಯಾ-ರಾ-ಗ-ಬೇ-ಕಾ-ದರೆ
ತಯಾ-ರಾಗಿ
ತಯಾ-ರಾ-ಗಿ-ರ-ಲಿಲ್ಲ
ತಯಾ-ರಾ-ಗಿ-ರು-ವುದು
ತಯಾ-ರಾಗು
ತಯಾ-ರಾ-ಗು-ತ್ತಿ-ದೆಯೊ
ತಯಾ-ರಾ-ಗು-ವು-ದಿಲ್ಲ
ತಯಾ-ರಾ-ಗು-ವುದು
ತಯಾ-ರಾ-ಗು-ವುದೊ
ತಯಾ-ರಾದ
ತಯಾ-ರಾ-ದು-ದನ್ನು
ತಯಾ-ರಾ-ದುದು
ತಯಾ-ರಿ-ಸು-ತ್ತಿ-ರು-ವನು
ತಯಾರು
ತಯಾ-ರು-ಮಾಡಿ
ತಯಾ-ರು-ಮಾ-ಡಿ-ಕೊಂ-ಡ-ವರು
ತಯಾ-ರು-ಮಾ-ಡು-ತ್ತೇವೆ
ತಯಾ-ರು-ಮಾ-ಡುವ
ತಯೋರ್ನ
ತಯೋಸ್ತು
ತರಂ-ಗ-ದಂತೆ
ತರಂ-ಗ-ವನ್ನು
ತರಂತಿ
ತರ-ಕಾರಿ
ತರ-ಗ-ತಿ-ಗ-ಳಿಂ-ದಲೂ
ತರ-ಗ-ತಿ-ಗ-ಳಿವೆ
ತರ-ಗ-ತಿಗೆ
ತರ-ಗ-ತಿಯ
ತರ-ಗ-ತಿ-ಯಲ್ಲಿ
ತರ-ಗ-ತಿ-ಯಲ್ಲೂ
ತರ-ಗೆಲೆ
ತರ-ಗೆ-ಲೆ-ಗಳು
ತರ-ಗೆ-ಲೆಗೆ
ತರ-ಗೆ-ಲೆ-ಯಂ-ತಾ-ಗು-ವುದು
ತರ-ಗೆ-ಲೆ-ಯಂತೆ
ತರ-ಗೆ-ಲೆ-ಯನ್ನು
ತರ-ತ-ಮ-ದಂತೆ
ತರ-ತ-ಮ-ದಿಂದ
ತರ-ತ-ರ-ಗು-ಟ್ಟು-ವಂತೆ
ತರದ
ತರ-ಬ-ಲ್ಲುದು
ತರ-ಬ-ಹುದು
ತರ-ಬಾ-ರದು
ತರ-ಬೇ-ಕಾ-ಗಿದೆ
ತರ-ಬೇ-ಕಾ-ದರೆ
ತರ-ಬೇಕು
ತರ-ಬೇ-ಕು-ಎಂಬ
ತರ-ಬೇ-ಕೆಂ-ದಾ-ಗಲಿ
ತರ-ಬೇ-ತನ್ನು
ತರ-ಬೇ-ತನ್ನೂ
ತರ-ಬೇ-ತಾ-ಗಿದೆ
ತರ-ಬೇತಿ
ತರ-ಬೇ-ತಿ-ನಲ್ಲಿ
ತರ-ಬೇತು
ತರ-ಬೇತೂ
ತರ-ಲಾಗು
ತರ-ಲಾ-ರದು
ತರ-ಲಾ-ರರು
ತರ-ಲಾ-ರವು
ತರಲು
ತರಲೆ
ತರ-ವಲ್ಲ
ತರಹ
ತರು-ತ್ತಾನೆ
ತರು-ತ್ತಿತ್ತು
ತರು-ತ್ತಿ-ರು-ತ್ತವೆ
ತರು-ತ್ತೇವೆ
ತರು-ಲತೆ
ತರು-ಲ-ತೆ-ಗ-ಳಾ-ಗಲಿ
ತರು-ಲ-ತೆ-ಗಳು
ತರು-ಲ-ತೆ-ಗಳೂ
ತರುವ
ತರು-ವನು
ತರು-ವರೊ
ತರು-ವ-ವ-ರನ್ನೇ
ತರು-ವ-ವರೇ
ತರು-ವಿ-ನಲ್ಲಿ
ತರುವು
ತರು-ವು-ದಕ್ಕೆ
ತರು-ವು-ದಿಲ್ಲ
ತರು-ವು-ದಿ-ಲ್ಲವೊ
ತರು-ವುದು
ತರು-ವು-ದೆಂದು
ತರು-ವುದೇ
ತರು-ವುದೊ
ತರು-ವುದೋ
ತರು-ವುವು
ತರು-ವೆವು
ತರ್ಕ
ತರ್ಕಕ್ಕೂ
ತರ್ಕದ
ತರ್ಕಿ-ಸ-ಬ-ಹುದು
ತರ್ಕಿ-ಸು-ತ್ತಾನೆ
ತರ್ಕಿ-ಸು-ತ್ತೇವೆ
ತರ್ಕಿ-ಸು-ವನು
ತಲ-ಪುವ
ತಲ-ಪು-ವುದು
ತಲು-ಪದೆ
ತಲು-ಪಿ-ಸಲು
ತಲೆ
ತಲೆ-ಕೆ-ಳಗು
ತಲೆ-ಗಳಿಂದ
ತಲೆಗೆ
ತಲೆ-ತ-ಲಾಂ-ತ-ರ-ಗ-ಳಿಂ-ದಲೂ
ತಲೆ-ತಿ-ರುಗಿ
ತಲೆ-ದೂ-ಗ-ಬೇಕು
ತಲೆ-ದೂ-ಗಿದ
ತಲೆ-ದೂ-ಗುತ್ತಾ
ತಲೆ-ದೂ-ಗು-ವನು
ತಲೆ-ದೋ-ರಿತು
ತಲೆ-ದೋ-ರಿದೆ
ತಲೆ-ದೋ-ರಿ-ರ-ಬ-ಹುದು
ತಲೆ-ದೋರು
ತಲೆ-ದೋ-ರು-ವುದು
ತಲೆ-ಬಾ-ಗಿ-ಸಿ-ಕೊಂಡು
ತಲೆ-ಬಾ-ಗು-ವುದು
ತಲೆ-ಬಾ-ಗು-ವುವು
ತಲೆ-ಯನ್ನು
ತಲೆ-ಯ-ಮೇಲೆ
ತಲೆ-ಯಲ್ಲಿ
ತಲೆ-ಯೆ-ತ್ತ-ಲಾ-ರವು
ತಲ್ಲ-ಣಿ-ಸು-ತ್ತಿ-ರು-ವನು
ತಲ್ಲ-ಣಿ-ಸು-ತ್ತಿ-ರು-ವರು
ತಲ್ಲ-ಣಿ-ಸು-ತ್ತೇವೆ
ತಲ್ಲ-ಣಿ-ಸು-ವು-ದಿಲ್ಲ
ತಲ್ಲೀ-ನ-ನಾಗಿ
ತಲ್ಲೀ-ನ-ನಾ-ಗಿ-ರು-ತ್ತಿದ್ದ
ತಲ್ಲೀ-ನ-ನಾ-ಗು-ವನು
ತಲ್ಲೀ-ನ-ನಾ-ಗು-ವು-ದಿಲ್ಲ
ತಲ್ಲೀ-ನ-ರಾ-ಗಿ-ರು-ವೆವು
ತಲ್ಲೀ-ನ-ರಾ-ಗು-ತ್ತಿ-ದ್ದು-ದನ್ನು
ತಲ್ಲೀ-ನ-ರಾ-ಗು-ತ್ತೇವೆ
ತಲ್ಲೀ-ನ-ವಾ-ಗಿ-ರು-ವಂತೆ
ತಳ
ತಳ-ಪಾಯ
ತಳ-ಪಾ-ಯದ
ತಳ-ಪಾ-ಯ-ವಾಗಿ
ತಳ-ಪಾ-ಯ-ವಾ-ಗಿ-ರ-ಬೇಕು
ತಳ-ಪಾ-ಯವೇ
ತಳ-ಮ-ಳ-ಗೊ-ಳಿ-ಸು-ವುದು
ತಳ-ಮ-ಳಿ-ಸ-ಕೂ-ಡದು
ತಳ-ಮ-ಳಿ-ಸು-ವೆವು
ತಳ-ವನ್ನು
ತಳ-ಹದಿ
ತಳ-ಹ-ದಿಯ
ತಳ-ಹ-ದಿ-ಯನ್ನು
ತಳೆ-ದಿ-ರು-ವುದು
ತಳ್ಳ-ಬ-ಲ್ಲನೆ
ತಳ್ಳಿ
ತಳ್ಳಿ-ದತ್ತ
ತಳ್ಳಿದ್ದು
ತಳ್ಳಿ-ಬಿ-ಡು-ವೆವು
ತಳ್ಳು-ತ್ತಾನೆ
ತಳ್ಳು-ವರು
ತಳ್ಳು-ವು-ದಿಲ್ಲ
ತಳ್ಳು-ವುದು
ತವ
ತವಕ
ತವ-ಕ-ಪ-ಡುತ್ತಿ
ತವ-ಕ-ಪ-ಡು-ತ್ತಿ-ರು-ವರೋ
ತವ-ಕ-ಪ-ಡು-ತ್ತಿ-ರು-ವುದು
ತವ-ಕ-ಪ-ಡು-ವರು
ತವ-ಕ-ಪ-ಡು-ವುದು
ತವಾಮೀ
ತವಾ-ರ್ಜುನ
ತವಾ-ರ್ಜು-ನೇದಂ
ತವಾ-ಹಿ-ತಾಃ
ತವೇದಂ
ತಸ್ಮಾ-ಚ್ಛಾಸ್ತ್ರಂ
ತಸ್ಮಾತ್
ತಸ್ಮಾ-ತ್ತ್ವ-ಮಿಂ-ದ್ರಿ-ಯಾ-ಣ್ಯಾದೌ
ತಸ್ಮಾ-ದ-ಜ್ಞಾ-ನ-ಸಂ-ಭೂತಂ
ತಸ್ಮಾ-ದನ್ಯಃ
ತಸ್ಮಾ-ದ-ಪ-ರಿ-ಹಾ-ರ್ಯೇಥೆ
ತಸ್ಮಾ-ದ-ಸಕ್ತಃ
ತಸ್ಮಾ-ದು-ತ್ತಿಷ್ಠ
ತಸ್ಮಾ-ದೇವಂ
ತಸ್ಮಾ-ದೋ-ಮಿ-ತ್ಯು-ದಾ-ಹೃತ್ಯ
ತಸ್ಮಾ-ದ್ಬ್ರ-ಹ್ಮಣಿ
ತಸ್ಮಾ-ದ್ಯಸ್ಯ
ತಸ್ಮಾ-ದ್ಯು-ಧ್ಯಸ್ವ
ತಸ್ಮಾ-ದ್ಯೋ-ಗಾಯ
ತಸ್ಮಾ-ದ್ಯೋಗೀ
ತಸ್ಮಾ-ನ್ನಾರ್ಹಾ
ತಸ್ಮಾ-ನ್ಮ-ನು-ಷ್ಯೇಷು
ತಸ್ಮಿನ್
ತಸ್ಮೈ
ತಸ್ಯ
ತಸ್ಯಾಂ
ತಸ್ಯಾ-ಚ-ಲಾಂ
ತಸ್ಯಾಹಂ
ತಸ್ಯೈವ
ತಾಂ
ತಾಂಡವ
ತಾಂಡ-ವ-ವಾ-ಡು-ವುದನ್ನು
ತಾಂಡ-ವ-ವಾ-ಡು-ವುದು
ತಾಂಡ-ವಾಡು
ತಾಂಡ-ವಾ-ಡು-ತ್ತಿದೆ
ತಾಂಸ್ತ-ಥೈವ
ತಾಕ-ಬೇಕು
ತಾಕ-ಲಾ-ರವು
ತಾಕಲಿ
ತಾಕಿ
ತಾಕಿತು
ತಾಕಿ-ದರೆ
ತಾಕಿ-ದಾಗ
ತಾಕಿ-ದಾ-ಗಲೆ
ತಾಕಿದೆ
ತಾಕಿ-ದೊ-ಡನೆ
ತಾಕಿ-ದ್ದಲ್ಲ
ತಾಕಿ-ರ-ಬಾ-ರದು
ತಾಕಿಲ್ಲ
ತಾಕಿ-ಸಿ-ಕೊಂ-ಡಿ-ರು-ವನು
ತಾಕಿ-ಸಿ-ದರೆ
ತಾಕಿ-ಸಿಲ್ಲ
ತಾಕಿ-ಸು-ವನು
ತಾಕು
ತಾಕು-ತ್ತವೆ
ತಾಕು-ವಂತೆ
ತಾಕು-ವು-ದಕ್ಕೆ
ತಾಕು-ವು-ದಿಲ್ಲ
ತಾಕು-ವು-ದಿ-ಲ್ಲವೋ
ತಾಕು-ವುದು
ತಾಕು-ವುದೆ
ತಾಕು-ವುದೊ
ತಾಕು-ವುವು
ತಾಗದೆ
ತಾಗ-ಬೇಕು
ತಾಗಿಸಿ
ತಾಗು-ವುದು
ತಾಜ-ಮ-ಹಲ್
ತಾಡಿ-ತ-ನಾ-ಗಿ-ರು-ವನು
ತಾಡಿ-ತ-ನಾ-ಗು-ತ್ತಿ-ರು-ವನು
ತಾತ
ತಾತ್ಕಾ-ಲಿಕ
ತಾತ್ಕಾ-ಲಿ-ಕ-ದಲ್ಲಿ
ತಾತ್ಕಾ-ಲಿ-ಕ-ವಾಗಿ
ತಾತ್ಕಾ-ಲಿ-ಕ-ವಾ-ಗಿ-ರು-ವುವು
ತಾತ್ಕಾ-ಲಿ-ಕ-ವಾದ
ತಾತ್ಕಾ-ಲಿ-ಕ-ವಾ-ದು-ದಲ್ಲ
ತಾತ್ತಿ
ತಾತ್ತ್ವಿಕ
ತಾತ್ತ್ವಿ-ಕ-ದೃಷ್ಟಿ
ತಾತ್ತ್ವಿ-ಕ-ದೃ-ಷ್ಟಿ-ಯಿಂದ
ತಾತ್ತ್ವಿ-ಕ-ವಾಗಿ
ತಾತ್ಪ-ರ್ಯ-ಗಳು
ತಾತ್ವಿಕ
ತಾತ್ಸಾರ
ತಾತ್ಸಾ-ರ-ದಿಂದ
ತಾತ್ಸಾ-ರ-ವಾಗಿ
ತಾತ್ಸಾ-ರ-ವಿಲ್ಲ
ತಾದಾತ್ಮ
ತಾದಾತ್ಮ್ಯ
ತಾದಾ-ತ್ಮ್ಯ-ಭಾವ
ತಾದಾ-ತ್ಮ್ಯ-ಭಾ-ವ-ವನ್ನು
ತಾದಾ-ತ್ಮ್ಯ-ಭಾ-ವ-ವಿಲ್ಲ
ತಾನ-ಕೃ-ತ್ಸ್ನ-ವಿದೋ
ತಾನಹಂ
ತಾನಾಗಿ
ತಾನಾ-ಡುವ
ತಾನಾರು
ತಾನಿ
ತಾನಿದ್ದ
ತಾನಿ-ರು-ತ್ತಾನೆ
ತಾನಿ-ರುವ
ತಾನಿ-ರು-ವಂತೆ
ತಾನಿ-ರು-ವನು
ತಾನಿ-ರು-ವಾಗ
ತಾನಿ-ರುವೆ
ತಾನಿ-ರು-ವೆನು
ತಾನು
ತಾನು-ಎ-ರ-ಡನ್ನೇ
ತಾನೂ
ತಾನೆ
ತಾನೆಲ್ಲಿ
ತಾನೆಷ್ಟು
ತಾನೇ
ತಾನೇ-ತಾ-ನಾಗಿ
ತಾನೇನೊ
ತಾನೊಂದು
ತಾನೊಬ್ಬ
ತಾನೊ-ಬ್ಬನೆ
ತಾನ್
ತಾನ್ನಿ-ಬೋಧ
ತಾನ್ಯಪಿ
ತಾನ್ಯಹಂ
ತಾನ್ಸೇ-ನ-ನೆಂಬ
ತಾನ್ಸೇನ್
ತಾಪ-ಗಳನ್ನು
ತಾಪ-ತ್ರ-ಯಕ್ಕೆ
ತಾಪ-ತ್ರ-ಯದ
ತಾಪ-ತ್ರ-ಯ-ವ-ನ್ನೆಲ್ಲಾ
ತಾಪ-ತ್ರ-ಯ-ವೆಲ್ಲ
ತಾಪ-ದಲ್ಲಿ
ತಾಪ-ದಿಂದ
ತಾಪ-ವನ್ನು
ತಾಪ-ವಾ-ಗಲಿ
ತಾಮಸ
ತಾಮಸಂ
ತಾಮ-ಸ-ಪ್ರಿ-ಯಮ್
ತಾಮ-ಸ-ವೆಂದು
ತಾಮಸಾ
ತಾಮ-ಸಾಃ
ತಾಮಸಿ
ತಾಮ-ಸಿಕ
ತಾಮ-ಸಿ-ಕ-ನಿಗೆ
ತಾಮ-ಸಿ-ಕ-ರಿಗೆ
ತಾಮ-ಸಿ-ಕರು
ತಾಮ-ಸಿ-ಕ-ವಾಗಿ
ತಾಮ-ಸಿ-ಕ-ವಾ-ಗಿಯೂ
ತಾಮ-ಸಿ-ಕವೆ
ತಾಮ-ಸಿ-ಕವೊ
ತಾಮ-ಸಿ-ಕವೋ
ತಾಮ-ಸಿಯ
ತಾಮಸೀ
ತಾಮೇವ
ತಾಮ್ರ
ತಾಮ್ರದ
ತಾಮ್ರ-ವಾಗಿ
ತಾಯಂ-ದಿರು
ತಾಯಿ
ತಾಯಿ-ಗಳ
ತಾಯಿ-ಗ-ಳಾ-ಗಿ-ರ-ಬ-ಹುದು
ತಾಯಿ-ಗ-ಳಿಗೆ
ತಾಯಿ-ಗಳು
ತಾಯಿಗೆ
ತಾಯಿ-ತಂ-ದೆ-ಗಳಲ್ಲಿ
ತಾಯಿ-ತಂ-ದೆ-ಗ-ಳಾ-ಗಿ-ರ-ಬ-ಹುದು
ತಾಯಿ-ತಂ-ದೆ-ಗಳು
ತಾಯಿ-ತಂ-ದೆ-ಯರ
ತಾಯಿಯ
ತಾಯಿ-ಯನ್ನು
ತಾಯಿ-ಯ-ಲ್ಲದೇ
ತಾಯಿ-ಯಷ್ಟು
ತಾಯಿಯೂ
ತಾಯಿ-ಯೆಂದು
ತಾಯಿಯೇ
ತಾಯೆ-ಡೆಗೆ
ತಾರ-ಕಾ-ಸು-ರನ
ತಾರ-ಕಾ-ಸು-ರ-ನನ್ನು
ತಾರ-ತಮ್ಯ
ತಾರದೆ
ತಾರವು
ತಾರಾ
ತಾರಾ-ಡು-ತ್ತಿ-ದ್ದಾಗ
ತಾರಾ-ಡು-ವಂ-ತಿ-ರ-ಬೇಕು
ತಾರಾ-ವ-ಳಿ-ಗ-ಳಿವೆ
ತಾರಾ-ವ-ಳಿಯ
ತಾರಿ-ಕೆ-ಗಳನ್ನು
ತಾರು-ಣ್ಯದ
ತಾರೆ-ಯನ್ನು
ತಾಳ
ತಾಳ-ಬಾ-ರದು
ತಾಳ-ಬೇ-ಕಾ-ಗಿಲ್ಲ
ತಾಳ-ಲಾ-ರದೆ
ತಾಳ-ಲಾ-ರರು
ತಾಳಿ
ತಾಳಿ-ದಂತೆ
ತಾಳಿದೆ
ತಾಳಿ-ದ್ದರೆ
ತಾಳಿ-ಯ-ನ್ನಾ-ದರೂ
ತಾಳು-ತ್ತದೆ
ತಾಳು-ತ್ತಾನೆ
ತಾಳು-ತ್ತಿದ್ದ
ತಾಳು-ವನು
ತಾಳು-ವ-ವ-ನಲ್ಲ
ತಾಳು-ವು-ದಕ್ಕೆ
ತಾಳು-ವು-ದಿಲ್ಲ
ತಾಳು-ವುದು
ತಾಳು-ವುವು
ತಾಳೆ
ತಾಳೆ-ಯಾ-ಗು-ವು-ದಿಲ್ಲ
ತಾಳ್ಮೆ-ಯಿಂದ
ತಾಳ್ಮೆ-ಯಿಲ್ಲ
ತಾವ-ರೆಯ
ತಾವಾಗಿ
ತಾವಾನ್
ತಾವು
ತಾವೆ
ತಾವೇ
ತಾಸಾಂ
ತಿಂಗಳ
ತಿಂಗಳಲ್ಲಿ
ತಿಂಗ-ಳಿ-ನಂತೆ
ತಿಂಗಳು
ತಿಂಗ-ಳೆಲ್ಲ
ತಿಂಡಿ
ತಿಂಡಿಗೆ
ತಿಂಡಿ-ಯನ್ನು
ತಿಂದ
ತಿಂದಂತೆ
ತಿಂದ-ದ್ದಕ್ಕೆ
ತಿಂದ-ದ್ದನ್ನು
ತಿಂದದ್ದು
ತಿಂದರೂ
ತಿಂದರೆ
ತಿಂದಾಗ
ತಿಂದಾದ
ತಿಂದಾ-ದ-ಮೇಲೆ
ತಿಂದಿದ್ದೇ
ತಿಂದಿಲ್ಲ
ತಿಂದು
ತಿಂದು-ಕೊಂಡು
ತಿಂದುದು
ತಿಂದು-ಬಿ-ಡು-ವಳೆ
ತಿಂದು-ಬಿ-ಡು-ವು-ದಿಲ್ಲ
ತಿಂದು-ಹಾ-ಕು-ವುದು
ತಿಂದು-ಹಾ-ಕು-ವುವೋ
ತಿಂದೆ
ತಿಂದೆವು
ತಿಂದೇನು
ತಿಂದೊ-ಡ-ನೆಯೇ
ತಿಕ್ಕಿ
ತಿಕ್ಕಿ-ದಾಗ
ತಿಕ್ಕಿ-ದಾ-ಗಲೆ
ತಿಕ್ಕಿ-ದೊ-ಡನೆ
ತಿಕ್ಕು-ವುದು
ತಿತಿ-ಕ್ಷೆ-ಯನ್ನು
ತಿದಿ-ಯಂತೆ
ತಿದ್ದ-ಬೇಕು
ತಿದ್ದಿ-ಕೊಂ-ಡರೆ
ತಿದ್ದಿ-ಕೊ-ಳ್ಳ-ಬೇಕು
ತಿದ್ದಿ-ಕೊ-ಳ್ಳಲು
ತಿದ್ದಿ-ಕೊ-ಳ್ಳುತ್ತ
ತಿದ್ದಿ-ಕೊ-ಳ್ಳು-ವನು
ತಿದ್ದು-ಕೊಂ-ಡಿ-ರು-ತ್ತಾನೆ
ತಿದ್ದು-ಕೊ-ಳ್ಳು-ವ-ವ-ನಲ್ಲ
ತಿದ್ದು-ತ್ತಾನೆ
ತಿನ್ನ
ತಿನ್ನದೆ
ತಿನ್ನದೇ
ತಿನ್ನ-ಬ-ಹುದು
ತಿನ್ನ-ಬಾ-ರ-ದಾ-ಗಿತ್ತು
ತಿನ್ನ-ಬಾ-ರದು
ತಿನ್ನ-ಬೇಕು
ತಿನ್ನ-ಬೇ-ಕೆಂದು
ತಿನ್ನ-ಲಾ-ಗು-ವು-ದಿಲ್ಲ
ತಿನ್ನಲಿ
ತಿನ್ನ-ಲಿಲ್ಲ
ತಿನ್ನಲು
ತಿನ್ನಲೂ
ತಿನ್ನಲೇ
ತಿನ್ನ-ಲೇ-ಬೇ-ಕಾ-ಗಿದೆ
ತಿನ್ನ-ವನು
ತಿನ್ನ-ವು-ದ-ರಲ್ಲಿ
ತಿನ್ನ-ವುದು
ತಿನ್ನು
ತಿನ್ನುತ್ತ
ತಿನ್ನು-ತ್ತದೆ
ತಿನ್ನು-ತ್ತಾನೆ
ತಿನ್ನು-ತ್ತಾರೆ
ತಿನ್ನು-ತ್ತಿದ್ದ
ತಿನ್ನು-ತ್ತಿ-ದ್ದರೂ
ತಿನ್ನು-ತ್ತಿ-ದ್ದರೆ
ತಿನ್ನು-ತ್ತಿ-ರು-ವಾಗ
ತಿನ್ನು-ತ್ತಿ-ರು-ವು-ದರ
ತಿನ್ನು-ತ್ತಿ-ರುವೆ
ತಿನ್ನು-ತ್ತೇವೆ
ತಿನ್ನು-ತ್ತೇ-ವೆಯೊ
ತಿನ್ನುವ
ತಿನ್ನು-ವಂತೆ
ತಿನ್ನು-ವನು
ತಿನ್ನು-ವರು
ತಿನ್ನು-ವ-ವ-ನಿಗೂ
ತಿನ್ನು-ವ-ವ-ರಿಗೆ
ತಿನ್ನು-ವಾಗ
ತಿನ್ನು-ವಾ-ಗಲೂ
ತಿನ್ನು-ವು-ದಕ್ಕೂ
ತಿನ್ನು-ವು-ದಕ್ಕೆ
ತಿನ್ನು-ವು-ದ-ಕ್ಕೆಂದು
ತಿನ್ನು-ವುದನ್ನು
ತಿನ್ನು-ವು-ದ-ರೊ-ಳಗೆ
ತಿನ್ನು-ವು-ದಿಲ್ಲ
ತಿನ್ನು-ವುದು
ತಿನ್ನು-ವು-ದೊಂದೇ
ತಿನ್ನು-ವುವು
ತಿನ್ನು-ವೆವು
ತಿಪ್ಪೆ
ತಿಪ್ಪೆ-ಗುಂ-ಡಿ-ಯಂತೆ
ತಿಪ್ಪೆ-ಯಲ್ಲಿ
ತಿಮಿಂ-ಗಿ-ಲ-ಗ-ಳಿ-ರ-ಬ-ಹುದು
ತಿರಳು
ತಿರ-ಸ್ಕ-ರಿ-ಸಿ-ರು-ವರು
ತಿರ-ಸ್ಕ-ರಿ-ಸು-ತ್ತಾರೆ
ತಿರ-ಸ್ಕ-ರಿ-ಸು-ತ್ತಿ-ದ್ದರು
ತಿರ-ಸ್ಕ-ರಿ-ಸು-ತ್ತೇವೆ
ತಿರ-ಸ್ಕ-ರಿ-ಸು-ವನು
ತಿರ-ಸ್ಕ-ರಿ-ಸು-ವು-ದಿಲ್ಲ
ತಿರ-ಸ್ಕ-ರಿ-ಸು-ವುದು
ತಿರ-ಸ್ಕಾರ
ತಿರ-ಸ್ಕಾ-ರ-ದಿಂದ
ತಿರಿ-ಗಿ-ಸಿ-ದೊ-ಡನೆ
ತಿರು-ಕನ
ತಿರು-ಗ-ಬೇಕಾ
ತಿರು-ಗ-ಬೇ-ಕಾ-ಗು-ವುದು
ತಿರು-ಗ-ಬೇ-ಕಾ-ದರೆ
ತಿರು-ಗ-ಬೇಕು
ತಿರು-ಗಲು
ತಿರು-ಗಲೂ
ತಿರುಗಿ
ತಿರು-ಗಿ-ದ-ವ-ರ-ಲ್ಲಿಯೂ
ತಿರು-ಗಿದೆ
ತಿರು-ಗಿ-ದೆಯೊ
ತಿರು-ಗಿ-ರ-ಬೇಕು
ತಿರು-ಗಿರು
ತಿರು-ಗಿ-ರು-ವಂತೆ
ತಿರು-ಗಿ-ರು-ವುದು
ತಿರು-ಗಿ-ರು-ವುದೋ
ತಿರು-ಗಿ-ಸ-ಬ-ಹುದು
ತಿರು-ಗಿ-ಸ-ಬೇ-ಕಾ-ಗಿದೆ
ತಿರು-ಗಿ-ಸ-ಬೇಕು
ತಿರು-ಗಿ-ಸ-ಬೇಕೇ
ತಿರು-ಗಿಸಿ
ತಿರು-ಗಿ-ಸಿ-ದರೂ
ತಿರು-ಗಿ-ಸಿ-ದರೆ
ತಿರು-ಗಿ-ಸಿಯೇ
ತಿರು-ಗಿ-ಸಿರು
ತಿರು-ಗಿ-ಸಿ-ರು-ವನು
ತಿರು-ಗಿ-ಸಿ-ರು-ವನೊ
ತಿರು-ಗಿ-ಸಿ-ರು-ವರು
ತಿರು-ಗಿ-ಸುತ್ತ
ತಿರು-ಗಿ-ಸು-ತ್ತಾನೆ
ತಿರು-ಗಿ-ಸು-ತ್ತೇನೆ
ತಿರು-ಗಿ-ಸುವ
ತಿರು-ಗಿ-ಸು-ವನು
ತಿರು-ಗಿ-ಸು-ವರು
ತಿರು-ಗು-ತ್ತದೆ
ತಿರು-ಗು-ತ್ತಾನೆ
ತಿರು-ಗು-ತ್ತಿದೆ
ತಿರು-ಗು-ತ್ತಿ-ರುವ
ತಿರು-ಗು-ತ್ತಿ-ರು-ವರೋ
ತಿರು-ಗು-ತ್ತಿ-ರು-ವುದು
ತಿರು-ಗು-ತ್ತಿ-ರು-ವುವು
ತಿರು-ಗು-ತ್ತಿ-ರು-ವೆವು
ತಿರು-ಗು-ತ್ತೇವೆ
ತಿರು-ಗು-ವಂತೆ
ತಿರು-ಗು-ವನು
ತಿರು-ಗು-ವರು
ತಿರು-ಗು-ವು-ದಿಲ್ಲ
ತಿರು-ಗು-ವುದು
ತಿರು-ಗು-ವುವೋ
ತಿರುಪೆ
ತಿರು-ಪೆಗೆ
ತಿರು-ಪೆ-ಯ-ವ-ನನ್ನು
ತಿರು-ಮಂತ್ರ
ತಿರು-ಳಲ್ಲ
ತಿರು-ಳಾ-ದರೂ
ತಿರು-ಳಿದೆ
ತಿರು-ಳಿ-ನಲ್ಲಿ
ತಿರುಳು
ತಿರುವಿ
ತಿರೆ
ತಿಲ-ಕರ
ತಿಲ-ಕರು
ತಿಳಿ
ತಿಳಿ-ಕೊ-ಳ್ಳು-ವರು
ತಿಳಿದ
ತಿಳಿ-ದಂತೆ
ತಿಳಿ-ದರೆ
ತಿಳಿ-ದ-ಲ್ಲದೆ
ತಿಳಿ-ದ-ವನ
ತಿಳಿ-ದ-ವ-ನಾ-ದರೂ
ತಿಳಿ-ದ-ವ-ನಿಗೆ
ತಿಳಿ-ದ-ವನು
ತಿಳಿ-ದ-ವನೆ
ತಿಳಿ-ದ-ವನೇ
ತಿಳಿ-ದ-ವನೊ
ತಿಳಿ-ದ-ವರು
ತಿಳಿ-ದಷ್ಟೂ
ತಿಳಿ-ದಾಗ
ತಿಳಿ-ದಾ-ದ-ಮೇಲೆ
ತಿಳಿ-ದಿದೆ
ತಿಳಿ-ದಿದ್ದ
ತಿಳಿ-ದಿ-ದ್ದರು
ತಿಳಿ-ದಿ-ದ್ದರೂ
ತಿಳಿ-ದಿ-ದ್ದರೆ
ತಿಳಿ-ದಿ-ರ-ಬ-ಹುದು
ತಿಳಿ-ದಿ-ರ-ಬೇ-ಕಾ-ಗಿತ್ತು
ತಿಳಿ-ದಿ-ರ-ಬೇಕು
ತಿಳಿ-ದಿರು
ತಿಳಿ-ದಿ-ರು-ತ್ತಾರೆ
ತಿಳಿ-ದಿ-ರುವ
ತಿಳಿ-ದಿ-ರು-ವನು
ತಿಳಿ-ದಿ-ರು-ವನೊ
ತಿಳಿ-ದಿ-ರು-ವನೋ
ತಿಳಿ-ದಿ-ರು-ವ-ವನ
ತಿಳಿ-ದಿ-ರು-ವ-ಷ್ಟನ್ನು
ತಿಳಿ-ದಿ-ರು-ವುದನ್ನು
ತಿಳಿ-ದಿ-ರು-ವು-ದ-ನ್ನೆಲ್ಲ
ತಿಳಿ-ದಿ-ರು-ವು-ದ-ರಿಂದ
ತಿಳಿ-ದಿ-ರು-ವುದು
ತಿಳಿ-ದಿ-ರು-ವು-ದೆಲ್ಲ
ತಿಳಿ-ದಿ-ರು-ವುದೋ
ತಿಳಿ-ದಿಲ್ಲ
ತಿಳಿದು
ತಿಳಿ-ದುಕೊ
ತಿಳಿ-ದು-ಕೊಂಡ
ತಿಳಿ-ದು-ಕೊಂ-ಡಂ-ತಾ-ಗು-ತ್ತದೆ
ತಿಳಿ-ದು-ಕೊಂ-ಡಂತೆ
ತಿಳಿ-ದು-ಕೊಂ-ಡ-ದ್ದನ್ನು
ತಿಳಿ-ದು-ಕೊಂ-ಡದ್ದು
ತಿಳಿ-ದು-ಕೊಂ-ಡ-ಮೇಲೆ
ತಿಳಿ-ದು-ಕೊಂ-ಡರೆ
ತಿಳಿ-ದು-ಕೊಂ-ಡ-ವ-ನಿಗೆ
ತಿಳಿ-ದು-ಕೊಂ-ಡ-ವನು
ತಿಳಿ-ದು-ಕೊಂ-ಡ-ವರು
ತಿಳಿ-ದು-ಕೊಂ-ಡಾಗ
ತಿಳಿ-ದು-ಕೊಂ-ಡಾ-ಗಲೇ
ತಿಳಿ-ದು-ಕೊಂ-ಡಾದ
ತಿಳಿ-ದು-ಕೊಂ-ಡಿದ್ದ
ತಿಳಿ-ದು-ಕೊಂ-ಡಿ-ದ್ದರೂ
ತಿಳಿ-ದು-ಕೊಂ-ಡಿ-ದ್ದೀಯೆ
ತಿಳಿ-ದು-ಕೊಂ-ಡಿ-ದ್ದೇನೆ
ತಿಳಿ-ದು-ಕೊಂ-ಡಿ-ದ್ದೇವೆ
ತಿಳಿ-ದು-ಕೊಂ-ಡಿರ
ತಿಳಿ-ದು-ಕೊಂ-ಡಿ-ರ-ಬ-ಹುದು
ತಿಳಿ-ದು-ಕೊಂ-ಡಿ-ರ-ಬೇಕು
ತಿಳಿ-ದು-ಕೊಂ-ಡಿರು
ತಿಳಿ-ದು-ಕೊಂ-ಡಿ-ರುವ
ತಿಳಿ-ದು-ಕೊಂ-ಡಿ-ರು-ವನು
ತಿಳಿ-ದು-ಕೊಂ-ಡಿ-ರು-ವನೇ
ತಿಳಿ-ದು-ಕೊಂ-ಡಿ-ರು-ವನೊ
ತಿಳಿ-ದು-ಕೊಂ-ಡಿ-ರು-ವನೋ
ತಿಳಿ-ದು-ಕೊಂ-ಡಿ-ರು-ವರು
ತಿಳಿ-ದು-ಕೊಂ-ಡಿ-ರು-ವರೊ
ತಿಳಿ-ದು-ಕೊಂ-ಡಿ-ರು-ವ-ವನು
ತಿಳಿ-ದು-ಕೊಂ-ಡಿ-ರು-ವು-ದ-ಕ್ಕಿಂತ
ತಿಳಿ-ದು-ಕೊಂ-ಡಿ-ರು-ವುದನ್ನು
ತಿಳಿ-ದು-ಕೊಂ-ಡಿ-ರು-ವು-ದಲ್ಲ
ತಿಳಿ-ದು-ಕೊಂ-ಡಿ-ರು-ವುದು
ತಿಳಿ-ದು-ಕೊಂ-ಡಿ-ರು-ವುದೂ
ತಿಳಿ-ದು-ಕೊಂ-ಡಿ-ರು-ವೆನೋ
ತಿಳಿ-ದು-ಕೊಂ-ಡಿ-ರು-ವೆವೊ
ತಿಳಿ-ದು-ಕೊಂ-ಡಿ-ರು-ವೆವೋ
ತಿಳಿ-ದು-ಕೊಂ-ಡಿಲ್ಲ
ತಿಳಿ-ದು-ಕೊಂಡು
ತಿಳಿ-ದು-ಕೊಂ-ಡು-ದನ್ನು
ತಿಳಿ-ದು-ಕೊಂ-ಡುದು
ತಿಳಿ-ದು-ಕೊಳ್ಳ
ತಿಳಿ-ದು-ಕೊ-ಳ್ಳ-ತ-ಕ್ಕ-ವನು
ತಿಳಿ-ದು-ಕೊ-ಳ್ಳ-ತ್ತಾ-ನೆಯೋ
ತಿಳಿ-ದು-ಕೊ-ಳ್ಳದೆ
ತಿಳಿ-ದು-ಕೊ-ಳ್ಳದೇ
ತಿಳಿ-ದು-ಕೊ-ಳ್ಳ-ಬ-ಯ-ಸು-ವನು
ತಿಳಿ-ದು-ಕೊ-ಳ್ಳ-ಬ-ಯ-ಸು-ವ-ವನು
ತಿಳಿ-ದು-ಕೊ-ಳ್ಳ-ಬಲ್ಲ
ತಿಳಿ-ದು-ಕೊ-ಳ್ಳ-ಬ-ಲ್ಲಂ-ತಹ
ತಿಳಿ-ದು-ಕೊ-ಳ್ಳ-ಬ-ಲ್ಲದು
ತಿಳಿ-ದು-ಕೊ-ಳ್ಳ-ಬ-ಲ್ಲನೊ
ತಿಳಿ-ದು-ಕೊ-ಳ್ಳ-ಬ-ಲ್ಲರು
ತಿಳಿ-ದು-ಕೊ-ಳ್ಳ-ಬ-ಲ್ಲೆಯೊ
ತಿಳಿ-ದು-ಕೊ-ಳ್ಳ-ಬ-ಲ್ಲೆವು
ತಿಳಿ-ದು-ಕೊ-ಳ್ಳ-ಬ-ಹುದು
ತಿಳಿ-ದು-ಕೊ-ಳ್ಳ-ಬೇ-ಕಾ-ಗಿದೆ
ತಿಳಿ-ದು-ಕೊ-ಳ್ಳ-ಬೇ-ಕಾ-ಗಿ-ರು-ವುದು
ತಿಳಿ-ದು-ಕೊ-ಳ್ಳ-ಬೇ-ಕಾ-ಗಿಲ್ಲ
ತಿಳಿ-ದು-ಕೊ-ಳ್ಳ-ಬೇ-ಕಾ-ಗಿ-ಲ್ಲವೊ
ತಿಳಿ-ದು-ಕೊ-ಳ್ಳ-ಬೇ-ಕಾ-ಗು-ವುದು
ತಿಳಿ-ದು-ಕೊ-ಳ್ಳ-ಬೇ-ಕಾ-ದರೂ
ತಿಳಿ-ದು-ಕೊ-ಳ್ಳ-ಬೇ-ಕಾ-ದರೆ
ತಿಳಿ-ದು-ಕೊ-ಳ್ಳ-ಬೇಕು
ತಿಳಿ-ದು-ಕೊ-ಳ್ಳ-ಬೇ-ಕೆಂದು
ತಿಳಿ-ದು-ಕೊ-ಳ್ಳ-ಬೇ-ಕೆಂಬ
ತಿಳಿ-ದು-ಕೊ-ಳ್ಳ-ಬೇ-ಕೆಂ-ಬುದು
ತಿಳಿ-ದು-ಕೊ-ಳ್ಳ-ಬೇಕೋ
ತಿಳಿ-ದು-ಕೊ-ಳ್ಳ-ಲಾ-ಗು-ವು-ದಿಲ್ಲ
ತಿಳಿ-ದು-ಕೊ-ಳ್ಳ-ಲಾರ
ತಿಳಿ-ದು-ಕೊ-ಳ್ಳ-ಲಾ-ರದು
ತಿಳಿ-ದು-ಕೊ-ಳ್ಳ-ಲಾ-ರರು
ತಿಳಿ-ದು-ಕೊ-ಳ್ಳ-ಲಾ-ರೆವು
ತಿಳಿ-ದು-ಕೊ-ಳ್ಳಲಿ
ತಿಳಿ-ದು-ಕೊ-ಳ್ಳಲು
ತಿಳಿ-ದು-ಕೊ-ಳ್ಳ-ವುದು
ತಿಳಿ-ದು-ಕೊಳ್ಳಿ
ತಿಳಿ-ದು-ಕೊಳ್ಳು
ತಿಳಿ-ದು-ಕೊ-ಳ್ಳುತ್ತ
ತಿಳಿ-ದು-ಕೊ-ಳ್ಳುತ್ತಾ
ತಿಳಿ-ದು-ಕೊ-ಳ್ಳು-ತ್ತಾನೆ
ತಿಳಿ-ದು-ಕೊ-ಳ್ಳು-ತ್ತಾ-ನೆಯೋ
ತಿಳಿ-ದು-ಕೊ-ಳ್ಳು-ತ್ತಾನೋ
ತಿಳಿ-ದು-ಕೊ-ಳ್ಳು-ತ್ತಾರೆ
ತಿಳಿ-ದು-ಕೊ-ಳ್ಳು-ತ್ತಾ-ರೆಯೊ
ತಿಳಿ-ದು-ಕೊ-ಳ್ಳು-ತ್ತೇನೆ
ತಿಳಿ-ದು-ಕೊ-ಳ್ಳು-ತ್ತೇವೆ
ತಿಳಿ-ದು-ಕೊ-ಳ್ಳು-ತ್ತೇ-ವೆಯೊ
ತಿಳಿ-ದು-ಕೊ-ಳ್ಳು-ತ್ತೇ-ವೆಯೋ
ತಿಳಿ-ದು-ಕೊ-ಳ್ಳುವ
ತಿಳಿ-ದು-ಕೊ-ಳ್ಳು-ವಂ-ತಹ
ತಿಳಿ-ದು-ಕೊ-ಳ್ಳು-ವಂತೆ
ತಿಳಿ-ದು-ಕೊ-ಳ್ಳು-ವನು
ತಿಳಿ-ದು-ಕೊ-ಳ್ಳು-ವನೊ
ತಿಳಿ-ದು-ಕೊ-ಳ್ಳು-ವನೋ
ತಿಳಿ-ದು-ಕೊ-ಳ್ಳು-ವರು
ತಿಳಿ-ದು-ಕೊ-ಳ್ಳು-ವರೋ
ತಿಳಿ-ದು-ಕೊ-ಳ್ಳು-ವ-ವನೇ
ತಿಳಿ-ದು-ಕೊ-ಳ್ಳು-ವ-ವರ
ತಿಳಿ-ದು-ಕೊ-ಳ್ಳು-ವ-ವರು
ತಿಳಿ-ದು-ಕೊ-ಳ್ಳು-ವಾ-ಗಲೂ
ತಿಳಿ-ದು-ಕೊ-ಳ್ಳು-ವಾ-ಗಲೇ
ತಿಳಿ-ದು-ಕೊ-ಳ್ಳುವು
ತಿಳಿ-ದು-ಕೊ-ಳ್ಳು-ವುಕ್ಕೆ
ತಿಳಿ-ದು-ಕೊ-ಳ್ಳು-ವು-ದಕ್ಕೆ
ತಿಳಿ-ದು-ಕೊ-ಳ್ಳು-ವುದನ್ನು
ತಿಳಿ-ದು-ಕೊ-ಳ್ಳು-ವು-ದಲ್ಲ
ತಿಳಿ-ದು-ಕೊ-ಳ್ಳು-ವು-ದಿಲ್ಲ
ತಿಳಿ-ದು-ಕೊ-ಳ್ಳು-ವುದು
ತಿಳಿ-ದು-ಕೊ-ಳ್ಳು-ವುದೊ
ತಿಳಿ-ದು-ಕೊ-ಳ್ಳು-ವು-ದೊಂದೇ
ತಿಳಿ-ದು-ಕೊ-ಳ್ಳು-ವೆವು
ತಿಳಿ-ದು-ಕೊ-ಳ್ಳೋಣ
ತಿಳಿ-ದು-ದರ
ತಿಳಿ-ದೊ-ಕಂ-ಡಿ-ರುವೆ
ತಿಳಿದೋ
ತಿಳಿ-ನೀರು
ತಿಳಿ-ಭಾಷೆ
ತಿಳಿಯ
ತಿಳಿ-ಯ-ತ-ಕ್ಕ-ದ್ದಲ್ಲ
ತಿಳಿ-ಯ-ತ-ಕ್ಕದ್ದು
ತಿಳಿ-ಯ-ತ-ಕ್ಕ-ವನು
ತಿಳಿ-ಯದ
ತಿಳಿ-ಯದು
ತಿಳಿ-ಯದೆ
ತಿಳಿ-ಯ-ದೆಯೋ
ತಿಳಿ-ಯದೇ
ತಿಳಿ-ಯ-ಬ-ಯ-ಸು-ತ್ತೇನೆ
ತಿಳಿ-ಯ-ಬ-ಯ-ಸು-ವನು
ತಿಳಿ-ಯ-ಬಲ್ಲ
ತಿಳಿ-ಯ-ಬ-ಲ್ಲದು
ತಿಳಿ-ಯ-ಬ-ಹುದು
ತಿಳಿ-ಯ-ಬೇ-ಕಾ-ಗಿ-ರುವ
ತಿಳಿ-ಯ-ಬೇ-ಕಾ-ಗಿಲ್ಲ
ತಿಳಿ-ಯ-ಬೇ-ಕಾ-ದರೂ
ತಿಳಿ-ಯ-ಬೇ-ಕಾ-ದರೆ
ತಿಳಿ-ಯ-ಬೇಕು
ತಿಳಿ-ಯ-ಬೇ-ಕೆಂದು
ತಿಳಿ-ಯರು
ತಿಳಿ-ಯ-ಲಾಗು
ತಿಳಿ-ಯ-ಲಾ-ರದು
ತಿಳಿ-ಯ-ಲಾ-ರನು
ತಿಳಿ-ಯ-ಲಾ-ರರು
ತಿಳಿ-ಯ-ಲಾ-ರವೊ
ತಿಳಿ-ಯ-ಲಾರೆ
ತಿಳಿ-ಯಲು
ತಿಳಿ-ಯ-ಲೆ-ತ್ನಿಸು
ತಿಳಿ-ಯ-ಲೆ-ತ್ನಿ-ಸು-ವನು
ತಿಳಿ-ಯಾ-ಗಿ-ದ್ದರೆ
ತಿಳಿ-ಯಾ-ಗಿ-ರ-ಬೇಕು
ತಿಳಿ-ಯಾ-ಗುತ್ತ
ತಿಳಿ-ಯಾ-ಗು-ವುದು
ತಿಳಿ-ಯಾ-ದಾಗ
ತಿಳಿ-ಯಾ-ದುದು
ತಿಳಿಯು
ತಿಳಿ-ಯುತ್ತ
ತಿಳಿ-ಯು-ತ್ತದೆ
ತಿಳಿ-ಯು-ತ್ತ-ದೆಯೋ
ತಿಳಿ-ಯುತ್ತಾ
ತಿಳಿ-ಯು-ತ್ತಾನೆ
ತಿಳಿ-ಯು-ತ್ತಾ-ನೆಯೊ
ತಿಳಿ-ಯು-ತ್ತಾರೆ
ತಿಳಿ-ಯು-ತ್ತಾ-ರೆಯೋ
ತಿಳಿ-ಯು-ತ್ತೇನೆ
ತಿಳಿ-ಯು-ತ್ತೇ-ವೆಯೊ
ತಿಳಿ-ಯುವ
ತಿಳಿ-ಯು-ವನು
ತಿಳಿ-ಯು-ವನೊ
ತಿಳಿ-ಯು-ವನೋ
ತಿಳಿ-ಯು-ವರು
ತಿಳಿ-ಯು-ವರೊ
ತಿಳಿ-ಯುವು
ತಿಳಿ-ಯು-ವು-ದಕ್ಕೂ
ತಿಳಿ-ಯು-ವು-ದಕ್ಕೆ
ತಿಳಿ-ಯು-ವು-ದ-ರೊ-ಳ-ಗಾಗಿ
ತಿಳಿ-ಯು-ವು-ದ-ರೊ-ಳಗೇ
ತಿಳಿ-ಯು-ವು-ದಿಲ್ಲ
ತಿಳಿ-ಯು-ವುದು
ತಿಳಿ-ಯು-ವುದೇ
ತಿಳಿ-ಯು-ವುದೊ
ತಿಳಿ-ಯು-ವೆವು
ತಿಳಿ-ವ-ಳಿಕೆ
ತಿಳಿ-ವ-ಳಿ-ಕೆಗೆ
ತಿಳಿ-ವ-ಳಿ-ಕೆ-ಯ-ನ್ನೆಲ್ಲಾ
ತಿಳಿ-ವ-ಳಿ-ಕೆ-ಯಾ-ದರೊ
ತಿಳಿ-ವ-ಳಿ-ಕೆ-ಯಿಂದ
ತಿಳಿ-ವ-ಳಿ-ಕೆ-ಯಿಂ-ದಲೇ
ತಿಳಿ-ವ-ಳಿ-ಕೆಯೇ
ತಿಳಿ-ಸ-ಬಾ-ರದು
ತಿಳಿ-ಸ-ಲಾ-ರದು
ತಿಳಿಸಿ
ತಿಳಿ-ಸಿ-ದನು
ತಿಳಿ-ಸಿ-ದರೆ
ತಿಳಿ-ಸಿದೆ
ತಿಳಿ-ಸು-ತ್ತಾನೆ
ತಿಳಿ-ಸು-ವ-ವನು
ತಿಳಿ-ಸು-ವು-ದ-ಕ್ಕಾಗಿ
ತಿಳಿ-ಸು-ವುದು
ತಿವಿದ
ತಿವಿ-ದತ್ತ
ತಿವಿದು
ತಿವಿ-ಯ-ಬಾ-ರ-ದಿತ್ತು
ತಿವಿ-ಯು-ತ್ತಿ-ದ್ದರೆ
ತಿವಿ-ಯು-ತ್ತಿರು
ತಿವಿ-ಯು-ತ್ತಿ-ರು-ವನೋ
ತಿವಿ-ಯು-ತ್ತಿ-ರು-ವುದು
ತಿವಿ-ಯು-ತ್ತಿವೆ
ತಿವಿ-ಯು-ವನು
ತಿಷ್ಠಂತಂ
ತಿಷ್ಠಂತಿ
ತಿಷ್ಠತಿ
ತಿಷ್ಠ-ತ್ಯ-ಕ-ರ್ಮ-ಕೃತ್
ತಿಷ್ಠಸಿ
ತೀಕ್ಷ
ತೀಕ್ಷ್ಣ
ತೀಡಿ
ತೀಡಿ-ದಾಗ
ತೀತ
ತೀತನ
ತೀರ
ತೀರಕ್ಕೆ
ತೀರದ
ತೀರ-ದಲ್ಲಿ
ತೀರ-ದಲ್ಲೆ
ತೀರ-ಬೇಕು
ತೀರ-ವನ್ನು
ತೀರಿ-ಕೊಂಡ
ತೀರಿ-ಕೊಂ-ಡರೆ
ತೀರಿ-ಕೊಂ-ಡಾಗ
ತೀರಿತು
ತೀರಿದ
ತೀರಿ-ದ-ಮೇಲೆ
ತೀರಿ-ದರೆ
ತೀರಿ-ದೊ-ಡನೆ
ತೀರಿ-ದೊ-ಡ-ನೆಯೇ
ತೀರಿಲ್ಲ
ತೀರಿಸ
ತೀರಿ-ಸ-ಬೇ-ಕಾ-ಗಿದೆ
ತೀರಿ-ಸ-ಬೇ-ಕಾ-ಗಿಲ್ಲ
ತೀರಿ-ಸ-ಬೇ-ಕಾ-ಗು-ವುದು
ತೀರಿ-ಸ-ಬೇ-ಕಾದ
ತೀರಿ-ಸ-ಬೇಕು
ತೀರಿ-ಸಲು
ತೀರಿಸಿ
ತೀರಿ-ಸಿ-ಕೊ-ಳ್ಳಲು
ತೀರಿ-ಸಿ-ಕೊಳ್ಳು
ತೀರಿ-ಸಿ-ಕೊ-ಳ್ಳುಲು
ತೀರಿ-ಸಿ-ಕೊ-ಳ್ಳು-ವನು
ತೀರಿ-ಸಿ-ಕೊ-ಳ್ಳು-ವು-ದಕ್ಕೆ
ತೀರಿ-ಸಿ-ಕೊ-ಳ್ಳು-ವು-ದ-ರಲ್ಲೆ
ತೀರಿ-ಸಿ-ಕೊ-ಳ್ಳು-ವುದೇ
ತೀರಿ-ಸಿ-ಕೊ-ಳ್ಳು-ವೆವೊ
ತೀರಿ-ಸಿ-ದಂತೆ
ತೀರಿ-ಸುವ
ತೀರಿ-ಸು-ವು-ದಕ್ಕೆ
ತೀರಿ-ಸು-ವು-ದಿ-ರಲಿ
ತೀರಿ-ಸು-ವು-ದಿ-ಲ್ಲವೊ
ತೀರಿ-ಸು-ವುದು
ತೀರಿ-ಹೋ-ಗು-ತ್ತಾನೆ
ತೀರಿ-ಹೋ-ಗು-ವ-ವ-ರಿ-ಗೆಲ್ಲ
ತೀರಿ-ಹೋ-ದರು
ತೀರಿ-ಹೋ-ದ-ವ-ರಿ-ಗೆಲ್ಲ
ತೀರಿ-ಹೋ-ಯಿತು
ತೀರೀತೆ
ತೀರು-ತ್ತ-ದಲ್ಲ
ತೀರು-ವ-ವ-ರೆಗೆ
ತೀರು-ವುದು
ತೀರ್ಥಂ-ಕರ
ತೀರ್ಥಂ-ಕ-ರರ
ತೀರ್ಥಂ-ಕ-ರ-ರಂ-ತಹ
ತೀರ್ಥಂ-ಕ-ರರು
ತೀರ್ಥ-ಗಳು
ತೀರ್ಥ-ದಲ್ಲಿ
ತೀರ್ಥ-ವನ್ನು
ತೀರ್ಥ-ವಾ-ಗ-ಬ-ಹುದು
ತೀರ್ಥ-ಸ್ಥ-ಳ-ಗಳಲ್ಲಿ
ತೀರ್ಥ-ಸ್ಥ-ಳ-ಗ-ಳಿಗೆ
ತೀವ್ರ
ತೀವ್ರ-ತೆ-ಯಿಂದ
ತೀವ್ರ-ವಾಗಿ
ತೀವ್ರ-ವಾ-ಗಿ-ರ-ಬೇಕು
ತೀವ್ರ-ವಾದ
ತೀವ್ರ-ವಾ-ದುದು
ತು
ತುಂಡನ್ನು
ತುಂಡ-ನ್ನೆಲ್ಲ
ತುಂಡಾದ
ತುಂಡು
ತುಂತುರು
ತುಂಬ
ತುಂಬ-ಬ-ಹುದು
ತುಂಬ-ಬೇ-ಕಾ-ಗಿದೆ
ತುಂಬ-ಬೇಕು
ತುಂಬಲು
ತುಂಬಾ
ತುಂಬಿ
ತುಂಬಿ-ಕೊಂ-ಡಿದೆ
ತುಂಬಿ-ಕೊಂ-ಡಿ-ರು-ವಂತೆ
ತುಂಬಿ-ಕೊಂ-ಡಿ-ರು-ವನು
ತುಂಬಿ-ಕೊಂಡು
ತುಂಬಿ-ಕೊ-ಳ್ಳ-ಬೇಕು
ತುಂಬಿ-ಕೊಳ್ಳು
ತುಂಬಿ-ಕೊ-ಳ್ಳುತ್ತಾ
ತುಂಬಿತೆ
ತುಂಬಿದ
ತುಂಬಿ-ದರೆ
ತುಂಬಿ-ದಾಗ
ತುಂಬಿದೆ
ತುಂಬಿ-ದೆಯೋ
ತುಂಬಿ-ದ್ದರೆ
ತುಂಬಿ-ರ-ಬೇಕು
ತುಂಬಿರು
ತುಂಬಿ-ರು-ತ್ತವೆ
ತುಂಬಿ-ರು-ವನು
ತುಂಬಿ-ರು-ವ-ವನು
ತುಂಬಿ-ರು-ವ-ವಳು
ತುಂಬಿ-ರು-ವಾಗ
ತುಂಬಿ-ರು-ವುದು
ತುಂಬಿ-ರು-ವುದೇ
ತುಂಬಿ-ಸಿ-ರ-ಬೇಕು
ತುಂಬು
ತುಂಬುತ್ತಾ
ತುಂಬು-ತ್ತಿ-ರು-ವು-ದ-ರಿಂದ
ತುಂಬು-ವಂತೆ
ತುಂಬು-ವನು
ತುಂಬು-ವು-ದಿಲ್ಲ
ತುಂಬು-ವುದು
ತುಂಬು-ವುದೊ
ತುಂಬು-ವೆವೊ
ತುಕ್ಕು
ತುಚ್ಛ-ವಾಗಿ
ತುಚ್ಛ-ವಾ-ಗಿಯೇ
ತುಚ್ಛ-ವಾದ
ತುಚ್ಛ-ವಾ-ದುದು
ತುಚ್ಛವೋ
ತುತ್ತ
ತುತ್ತ-ತುದಿ
ತುತ್ತ-ತು-ದಿಗೆ
ತುತ್ತ-ತು-ದಿ-ಯನ್ನು
ತುತ್ತ-ತು-ದಿ-ಯಲ್ಲಿ
ತುತ್ತಾ-ಗ-ದ-ವನು
ತುತ್ತಾ-ಗದೆ
ತುತ್ತಾ-ಗದೇ
ತುತ್ತಾಗಿ
ತುತ್ತಾ-ಗುತ್ತಾ
ತುತ್ತಾ-ಗು-ತ್ತಾನೆ
ತುತ್ತಾ-ಗು-ತ್ತಿ-ರು-ವನು
ತುತ್ತಾ-ಗು-ತ್ತೇವೆ
ತುತ್ತಾ-ಗು-ವನು
ತುತ್ತಾ-ಗು-ವು-ದಿಲ್ಲ
ತುತ್ತಾ-ಗು-ವುದು
ತುತ್ತಾ-ಗು-ವೆವೋ
ತುತ್ತಾ-ದರು
ತುತ್ತಾ-ದ-ವನ
ತುತ್ತಿ-ನಂತೆ
ತುತ್ತು
ತುದಿ
ತುದಿ-ಮೊ-ದ-ಲಿ-ಲ್ಲದ
ತುದಿಯ
ತುದಿ-ಯನ್ನು
ತುದಿ-ಯನ್ನೇ
ತುದಿ-ಯಲ್ಲಿ
ತುಪ್ಪ
ತುಪ್ಪಕ್ಕೂ
ತುಪ್ಪಕ್ಕೆ
ತುಪ್ಪ-ದಲ್ಲಿ
ತುಪ್ಪ-ವನ್ನು
ತುಮುಲ
ತುಮುಲೋ
ತುರಾಯಿ
ತುರು-ಕು-ವುದು
ತುರ್ತು
ತುಲ್ಯ-ನಿಂ-ದಾ-ಸ್ತು-ತಿ-ರ್ಮೌನೀ
ತುಲ್ಯ-ಪ್ರಿ-ಯಾ-ಪ್ರಿಯೋ
ತುಳಸಿ
ತುಳ-ಸಿ-ಮಣಿ
ತುಳ-ಸಿಯೋ
ತುಳ-ಸೀ-ಕಟ್ಟೆ
ತುಳ-ಸೀ-ಗಿಡ
ತುಳ-ಸೀ-ಗಿ-ಡ-ದಂತೆ
ತುಳಿ-ಸಿ-ಕೊ-ಳ್ಳು-ವ-ವರೂ
ತುಳು-ಕಾಡು
ತುಳು-ಕಾ-ಡುತ್ತ
ತುಳು-ಕಾ-ಡು-ತ್ತಿದೆ
ತುಳು-ಕಾ-ಡು-ತ್ತಿ-ದೆಯೋ
ತುಳು-ಕಾ-ಡು-ತ್ತಿರು
ತುಳು-ಕಾ-ಡು-ತ್ತಿ-ರುವ
ತುಳು-ಕಾ-ಡು-ತ್ತಿ-ರು-ವನು
ತುಳು-ಕಾ-ಡು-ತ್ತಿ-ರು-ವುದು
ತುಳು-ಕಾ-ಡು-ತ್ತಿವೆ
ತುಳು-ಕಾ-ಡುವ
ತುಳು-ಕಾ-ಡು-ವಂತೆ
ತುಳು-ಕಾ-ಡು-ವುದು
ತುಳು-ಕು-ತ್ತಿದೆ
ತುಳು-ಕು-ತ್ತಿ-ದೆಯೊ
ತುಳು-ಕು-ತ್ತಿ-ದ್ದಾನೆ
ತುಳು-ಕು-ತ್ತಿ-ರ-ಬೇಕು
ತುಳು-ಕು-ತ್ತಿ-ರು-ವನು
ತುಳು-ಕು-ತ್ತಿ-ರು-ವುದು
ತುಷಾ-ರ-ಹಾ-ರ-ವನ್ನು
ತುಷ್ಟಃ
ತುಷ್ಟಿ
ತುಷ್ಟಿ-ಸ್ತಪೋ
ತುಷ್ಯಂತಿ
ತುಷ್ಯತಿ
ತೂಕ-ಡಿ-ಸು-ತ್ತಿ-ರು-ವುದನ್ನು
ತೂಕದ
ತೂಕ-ವಾ-ಗಿದೆ
ತೂಗ-ಬೇಕು
ತೂಗು-ತ್ತಿದ್ದ
ತೂಗುವ
ತೂಗು-ವು-ದಕ್ಕೆ
ತೂಗು-ವುದು
ತೂತಾ-ಗು-ವುದೊ
ತೂತಿನ
ತೂತಿ-ನಿಂದ
ತೂತು
ತೂತು-ಗಳನ್ನು
ತೂತು-ಗಳನ್ನೆಲ್ಲ
ತೂತು-ಗ-ಳಿ-ದ್ದರೆ
ತೂಬನ್ನು
ತೂರ-ಬೇ-ಕಾ-ಗು-ವುದು
ತೂರಿ
ತೂರಿ-ಕೊಂಡು
ತೂರಿ-ದಾಗ
ತೂರಿ-ಹೋಗಿ
ತೂರಿ-ಹೋ-ಗು-ವನು
ತೂಷ್ಣೀಂ
ತೃಣ-ಕಾಷ್ಟ
ತೃಣ-ಕಾ-ಷ್ಟ-ದಿಂದ
ತೃಣ-ಕಾ-ಷ್ಠ-ಗಳ
ತೃಣ-ಕೀ-ಟ-ಗಳು
ತೃಣೀ-ಕ-ರಿ-ಸು-ವನು
ತೃಪ್ತ
ತೃಪ್ತ-ನಾಗಿ
ತೃಪ್ತ-ನಾ-ಗಿ-ದ್ದಾನೆ
ತೃಪ್ತ-ನಾ-ಗಿ-ರು-ವನು
ತೃಪ್ತ-ನಾ-ಗಿ-ರು-ವನೊ
ತೃಪ್ತ-ನಾದ
ತೃಪ್ತನೂ
ತೃಪ್ತ-ರಾಗಿ
ತೃಪ್ತ-ರಾ-ಗುವ
ತೃಪ್ತ-ರಾದ
ತೃಪ್ತ-ರಾ-ದ-ವರು
ತೃಪ್ತ-ವಾ-ಗಿ-ರ-ವನು
ತೃಪ್ತಿ
ತೃಪ್ತಿ-ಕೊ-ಡು-ವು-ದ-ಕ್ಕಲ್ಲ
ತೃಪ್ತಿಗೆ
ತೃಪ್ತಿ-ಪ-ಡಿ-ಲಾ-ಗದ
ತೃಪ್ತಿ-ಪ-ಡಿ-ಸ-ಬ-ಹುದು
ತೃಪ್ತಿ-ಪ-ಡಿ-ಸ-ಬೇ-ಕಾ-ಗಿ-ರು-ವು-ದ-ರಿಂದ
ತೃಪ್ತಿ-ಪ-ಡಿ-ಸ-ಬೇಕು
ತೃಪ್ತಿ-ಪ-ಡಿ-ಸಲು
ತೃಪ್ತಿ-ಪ-ಡಿ-ಸ-ಲೆ-ತ್ನಿ-ಸು-ವರು
ತೃಪ್ತಿ-ಪ-ಡಿಸಿ
ತೃಪ್ತಿ-ಪ-ಡಿ-ಸಿ-ಕೊಂಡು
ತೃಪ್ತಿ-ಪ-ಡಿ-ಸಿ-ಕೊ-ಳ್ಳ-ಬ-ಹುದು
ತೃಪ್ತಿ-ಪ-ಡಿ-ಸಿ-ಕೊ-ಳ್ಳ-ಬೇ-ಕಾ-ಗಿದೆ
ತೃಪ್ತಿ-ಪ-ಡಿ-ಸಿ-ಕೊ-ಳ್ಳ-ಬೇ-ಕಾ-ದರೆ
ತೃಪ್ತಿ-ಪ-ಡಿ-ಸಿ-ಕೊ-ಳ್ಳ-ಬೇ-ಕೆಂಬ
ತೃಪ್ತಿ-ಪ-ಡಿ-ಸಿ-ಕೊ-ಳ್ಳಲಿ
ತೃಪ್ತಿ-ಪ-ಡಿ-ಸಿ-ಕೊ-ಳ್ಳಲು
ತೃಪ್ತಿ-ಪ-ಡಿ-ಸಿ-ಕೊ-ಳ್ಳ-ವು-ದ-ಕ್ಕಾಗಿ
ತೃಪ್ತಿ-ಪ-ಡಿ-ಸಿ-ಕೊ-ಳ್ಳುತ್ತ
ತೃಪ್ತಿ-ಪ-ಡಿ-ಸಿ-ಕೊ-ಳ್ಳು-ತ್ತಾನೆ
ತೃಪ್ತಿ-ಪ-ಡಿ-ಸಿ-ಕೊ-ಳ್ಳು-ತ್ತೇವೆ
ತೃಪ್ತಿ-ಪ-ಡಿ-ಸಿ-ಕೊ-ಳ್ಳುವು
ತೃಪ್ತಿ-ಪ-ಡಿ-ಸಿ-ಕೊ-ಳ್ಳು-ವು-ದ-ಕ್ಕಾಗಿ
ತೃಪ್ತಿ-ಪ-ಡಿ-ಸಿ-ಕೊ-ಳ್ಳು-ವು-ದಕ್ಕೆ
ತೃಪ್ತಿ-ಪ-ಡಿ-ಸಿದ
ತೃಪ್ತಿ-ಪ-ಡಿ-ಸಿ-ದರೆ
ತೃಪ್ತಿ-ಪ-ಡಿ-ಸಿ-ದಷ್ಟೂ
ತೃಪ್ತಿ-ಪ-ಡಿ-ಸಿ-ರುವೆ
ತೃಪ್ತಿ-ಪ-ಡಿಸು
ತೃಪ್ತಿ-ಪ-ಡಿ-ಸುತ್ತ
ತೃಪ್ತಿ-ಪ-ಡಿ-ಸು-ತ್ತಾರೆ
ತೃಪ್ತಿ-ಪ-ಡಿ-ಸು-ತ್ತೇನೆ
ತೃಪ್ತಿ-ಪ-ಡಿ-ಸುವ
ತೃಪ್ತಿ-ಪ-ಡಿ-ಸು-ವ-ವನ
ತೃಪ್ತಿ-ಪ-ಡಿ-ಸು-ವುದ
ತೃಪ್ತಿ-ಪ-ಡಿ-ಸು-ವು-ದಕ್ಕೆ
ತೃಪ್ತಿ-ಪ-ಡಿ-ಸು-ವು-ದ-ರಿಂದ
ತೃಪ್ತಿ-ಪ-ಡಿ-ಸು-ವುದು
ತೃಪ್ತಿ-ಪ-ಡಿ-ಸು-ವೆವು
ತೃಪ್ತಿ-ಪ-ಡು-ತ್ತಾರೆ
ತೃಪ್ತಿ-ಮಾ-ಡಿ-ಕೊ-ಳ್ಳದೆ
ತೃಪ್ತಿಯ
ತೃಪ್ತಿ-ಯನ್ನು
ತೃಪ್ತಿ-ಯಾಗಿ
ತೃಪ್ತಿ-ಯಾ-ಗು-ವಂ-ತಿಲ್ಲ
ತೃಪ್ತಿ-ಯಾ-ಗು-ವಂತೆ
ತೃಪ್ತಿ-ಯಾ-ಗು-ವು-ದಿಲ್ಲ
ತೃಪ್ತಿ-ಯಿಂದ
ತೃಪ್ತಿ-ಯಿಲ್ಲ
ತೃಪ್ತಿರ್ಹಿ
ತೃಪ್ತಿ-ಹೊಂ-ದಿದ
ತೃಷ್ಣಾ-ಸಂ-ಗ-ಸ-ಮು-ದ್ಭ-ವಮ್
ತೃಷ್ಣೆ
ತೃಷ್ಣೆ-ಗಳನ್ನು
ತೃಷ್ಣೆಯ
ತೃಷ್ಣೆ-ಯನ್ನು
ತೃಷ್ಣೆ-ಯಿಂದ
ತೆಂಗಿನ
ತೆಗ-ಳ-ಬ-ಹುದು
ತೆಗ-ಳಲಿ
ತೆಗ-ಳಿಕೆ
ತೆಗ-ಳಿ-ಕೆಗೂ
ತೆಗ-ಳಿ-ಕೆಗೆ
ತೆಗ-ಳಿ-ದರೂ
ತೆಗ-ಳಿ-ದರೆ
ತೆಗ-ಳಿ-ದ-ರೇನು
ತೆಗ-ಳು-ವಾಗ
ತೆಗ-ಳು-ವು-ದಕ್ಕೂ
ತೆಗೆದ
ತೆಗೆ-ದರೂ
ತೆಗೆ-ದರೆ
ತೆಗೆ-ದಾಗ
ತೆಗೆ-ದಿದೆ
ತೆಗೆದು
ತೆಗೆ-ದುಕೊ
ತೆಗೆ-ದು-ಕೊಂಡ
ತೆಗೆ-ದು-ಕೊಂ-ಡಂತೆ
ತೆಗೆ-ದು-ಕೊಂ-ಡ-ದ್ದಲ್ಲ
ತೆಗೆ-ದು-ಕೊಂ-ಡ-ರಂತೆ
ತೆಗೆ-ದು-ಕೊಂ-ಡರೂ
ತೆಗೆ-ದು-ಕೊಂ-ಡರೆ
ತೆಗೆ-ದು-ಕೊಂ-ಡಾ-ದರೂ
ತೆಗೆ-ದು-ಕೊಂ-ಡಿ-ದ್ದಕ್ಕೆ
ತೆಗೆ-ದು-ಕೊಂ-ಡಿ-ದ್ದಾನೆ
ತೆಗೆ-ದು-ಕೊಂ-ಡಿ-ದ್ದೇವೆ
ತೆಗೆ-ದು-ಕೊಂ-ಡಿ-ರ-ಬೇಕು
ತೆಗೆ-ದು-ಕೊಂ-ಡಿ-ರಲಿ
ತೆಗೆ-ದು-ಕೊಂ-ಡಿ-ರುವ
ತೆಗೆ-ದು-ಕೊಂ-ಡಿ-ರು-ವಂತೆ
ತೆಗೆ-ದು-ಕೊಂ-ಡಿ-ರು-ವನು
ತೆಗೆ-ದು-ಕೊಂ-ಡಿ-ರು-ವನೋ
ತೆಗೆ-ದು-ಕೊಂ-ಡಿ-ರು-ವು-ದ-ರಿಂದ
ತೆಗೆ-ದು-ಕೊಂ-ಡಿ-ರುವೆ
ತೆಗೆ-ದು-ಕೊಂ-ಡಿ-ರು-ವೆವೊ
ತೆಗೆ-ದು-ಕೊಂ-ಡಿಲ್ಲ
ತೆಗೆ-ದು-ಕೊಂ-ಡಿ-ಲ್ಲವೋ
ತೆಗೆ-ದು-ಕೊಂಡು
ತೆಗೆ-ದು-ಕೊಂ-ಡುದು
ತೆಗೆ-ದು-ಕೊಂ-ಡು-ಬಿ-ಡು-ವು-ದಿಲ್ಲ
ತೆಗೆ-ದು-ಕೊಂ-ಡೆವು
ತೆಗೆ-ದು-ಕೊಂಡೇ
ತೆಗೆ-ದು-ಕೊಳ್ಳ
ತೆಗೆ-ದು-ಕೊ-ಳ್ಳ-ಕೂ-ಡದು
ತೆಗೆ-ದು-ಕೊ-ಳ್ಳದೆ
ತೆಗೆ-ದು-ಕೊ-ಳ್ಳದೇ
ತೆಗೆ-ದು-ಕೊ-ಳ್ಳ-ಬ-ಹುದು
ತೆಗೆ-ದು-ಕೊ-ಳ್ಳ-ಬ-ಹು-ದೇನೋ
ತೆಗೆ-ದು-ಕೊ-ಳ್ಳ-ಬಾ-ರದು
ತೆಗೆ-ದು-ಕೊ-ಳ್ಳ-ಬೇ-ಕಷ್ಟೆ
ತೆಗೆ-ದು-ಕೊ-ಳ್ಳ-ಬೇ-ಕಾ-ಗಿದೆ
ತೆಗೆ-ದು-ಕೊ-ಳ್ಳ-ಬೇ-ಕಾ-ಗಿಲ್ಲ
ತೆಗೆ-ದು-ಕೊ-ಳ್ಳ-ಬೇ-ಕಾಗು
ತೆಗೆ-ದು-ಕೊ-ಳ್ಳ-ಬೇ-ಕಾ-ಗು-ವುದು
ತೆಗೆ-ದು-ಕೊ-ಳ್ಳ-ಬೇಕು
ತೆಗೆ-ದು-ಕೊ-ಳ್ಳ-ಲಾ-ರದು
ತೆಗೆ-ದು-ಕೊ-ಳ್ಳ-ಲಾ-ರರು
ತೆಗೆ-ದು-ಕೊ-ಳ್ಳಲಿ
ತೆಗೆ-ದು-ಕೊ-ಳ್ಳ-ಲಿಲ್ಲ
ತೆಗೆ-ದು-ಕೊ-ಳ್ಳಲು
ತೆಗೆ-ದು-ಕೊ-ಳ್ಳಲೂ
ತೆಗೆ-ದು-ಕೊ-ಳ್ಳಲೇ
ತೆಗೆ-ದು-ಕೊ-ಳ್ಳ-ಲೇ-ಬೇ-ಕಾ-ಗಿದೆ
ತೆಗೆ-ದು-ಕೊ-ಳ್ಳ-ಲೇ-ಬೇ-ಕಾದ
ತೆಗೆ-ದು-ಕೊ-ಳ್ಳ-ವುದನ್ನು
ತೆಗೆ-ದು-ಕೊಳ್ಳಿ
ತೆಗೆ-ದು-ಕೊಳ್ಳು
ತೆಗೆ-ದು-ಕೊ-ಳ್ಳುತ್ತ
ತೆಗೆ-ದು-ಕೊ-ಳ್ಳು-ತ್ತಲೇ
ತೆಗೆ-ದು-ಕೊ-ಳ್ಳು-ತ್ತಾನೆ
ತೆಗೆ-ದು-ಕೊ-ಳ್ಳು-ತ್ತಾ-ನೆಯೆ
ತೆಗೆ-ದು-ಕೊ-ಳ್ಳು-ತ್ತಾ-ನೆಯೇ
ತೆಗೆ-ದು-ಕೊ-ಳ್ಳು-ತ್ತಾ-ನೆಯೋ
ತೆಗೆ-ದು-ಕೊ-ಳ್ಳು-ತ್ತಾರೆ
ತೆಗೆ-ದು-ಕೊ-ಳ್ಳು-ತ್ತಿ-ರ-ಬ-ಹುದು
ತೆಗೆ-ದು-ಕೊ-ಳ್ಳು-ತ್ತಿ-ರ-ಬೇಕು
ತೆಗೆ-ದು-ಕೊ-ಳ್ಳು-ತ್ತಿ-ರು-ತ್ತದೆ
ತೆಗೆ-ದು-ಕೊ-ಳ್ಳು-ತ್ತಿ-ರು-ವುದೇ
ತೆಗೆ-ದು-ಕೊ-ಳ್ಳು-ತ್ತೇವೆ
ತೆಗೆ-ದು-ಕೊ-ಳ್ಳು-ತ್ತೇ-ವೆಯೊ
ತೆಗೆ-ದು-ಕೊ-ಳ್ಳು-ತ್ತೇ-ವೆಯೋ
ತೆಗೆ-ದು-ಕೊ-ಳ್ಳುವ
ತೆಗೆ-ದು-ಕೊ-ಳ್ಳು-ವಂತೆ
ತೆಗೆ-ದು-ಕೊ-ಳ್ಳು-ವನು
ತೆಗೆ-ದು-ಕೊ-ಳ್ಳು-ವರು
ತೆಗೆ-ದು-ಕೊ-ಳ್ಳು-ವರೊ
ತೆಗೆ-ದು-ಕೊ-ಳ್ಳು-ವ-ವ-ನನ್ನು
ತೆಗೆ-ದು-ಕೊ-ಳ್ಳು-ವ-ವನು
ತೆಗೆ-ದು-ಕೊ-ಳ್ಳು-ವ-ವರು
ತೆಗೆ-ದು-ಕೊ-ಳ್ಳು-ವಾಗ
ತೆಗೆ-ದು-ಕೊ-ಳ್ಳು-ವಾ-ಗಲೂ
ತೆಗೆ-ದು-ಕೊ-ಳ್ಳು-ವು-ದ-ಕ್ಕಿಂತ
ತೆಗೆ-ದು-ಕೊ-ಳ್ಳು-ವು-ದಕ್ಕೆ
ತೆಗೆ-ದು-ಕೊ-ಳ್ಳು-ವು-ದ-ರಲ್ಲಿ
ತೆಗೆ-ದು-ಕೊ-ಳ್ಳು-ವು-ದಿಲ್ಲ
ತೆಗೆ-ದು-ಕೊ-ಳ್ಳು-ವುದು
ತೆಗೆ-ದು-ಕೊ-ಳ್ಳು-ವುದೂ
ತೆಗೆ-ದು-ಕೊ-ಳ್ಳು-ವುದೇ
ತೆಗೆ-ದು-ಕೊ-ಳ್ಳು-ವೆವು
ತೆಗೆ-ದು-ಕೊ-ಳ್ಳೋಣ
ತೆಗೆ-ದು-ಹಾ-ಕ-ಬ-ಹುದು
ತೆಗೆ-ದು-ಹಾ-ಕಲೂ
ತೆಗೆ-ದು-ಹಾಕಿ
ತೆಗೆ-ದು-ಹಾ-ಕಿ-ದರೆ
ತೆಗೆ-ದು-ಹಾ-ಕಿರು
ತೆಗೆ-ದು-ಹಾ-ಕಿ-ರು-ವನು
ತೆಗೆ-ದು-ಹಾ-ಕು-ವಂತೆ
ತೆಗೆಯ
ತೆಗೆ-ಯ-ಬ-ಹುದು
ತೆಗೆ-ಯ-ಬೇ-ಕಾ-ದರೆ
ತೆಗೆ-ಯ-ಬೇಕು
ತೆಗೆಯು
ತೆಗೆ-ಯು-ತ್ತಾನೆ
ತೆಗೆ-ಯು-ತ್ತಾ-ನೆಯೊ
ತೆಗೆ-ಯುವ
ತೆಗೆ-ಯು-ವನು
ತೆಗೆ-ಯು-ವನೋ
ತೆಗೆ-ಯು-ವು-ದಕ್ಕೆ
ತೆಗೆ-ಯು-ವುದು
ತೆಗೆ-ಯು-ವೆವು
ತೆಗೆ-ಯು-ವೆವೋ
ತೆತ್ತ
ತೆನೆ
ತೆನೆ-ಗಳನ್ನೆಲ್ಲ
ತೆನೆ-ಯನ್ನು
ತೆಪ್ಪ-ಗಾಗಿ
ತೆಪ್ಪ-ಗಾ-ಗು-ವಂತೆ
ತೆಪ್ಪ-ಗಾ-ಗು-ವು-ದಿಲ್ಲ
ತೆಪ್ಪ-ಗಾ-ಗು-ವುದು
ತೆಪ್ಪ-ಗಾ-ಗು-ವುವು
ತೆಪ್ಪ-ಗಿ-ರ-ಲಾ-ರದು
ತೆಪ್ಪ-ಗಿ-ರಿ-ಸುವ
ತೆಪ್ಪ-ಗಿ-ರು-ವನು
ತೆಪ್ಪ-ಗಿ-ರು-ವು-ದಿಲ್ಲ
ತೆಪ್ಪ-ಗಿ-ರು-ವುದು
ತೆಪ್ಪಗೆ
ತೆಪ್ಪ-ದಲ್ಲಿ
ತೆಪ್ಪನೆ
ತೆರಲು
ತೆರ-ವನ್ನು
ತೆರಿ-ಗೆ-ಯನ್ನು
ತೆರು-ತ್ತಿ-ರು-ವೆವು
ತೆರು-ವುದು
ತೆರೆ
ತೆರೆ-ಗಳಿಂದ
ತೆರೆ-ಗಳು
ತೆರೆಗೆ
ತೆರೆದ
ತೆರೆ-ದರೆ
ತೆರೆ-ದಿದೆ
ತೆರೆ-ದಿ-ರು-ವನು
ತೆರೆ-ದಿ-ರು-ವುದು
ತೆರೆ-ದು-ಕೊಂ-ಡಿದೆ
ತೆರೆಯ
ತೆರೆ-ಯದೆ
ತೆರೆ-ಯನ್ನು
ತೆರೆ-ಯ-ಮೇಲೆ
ತೆರೆ-ಯ-ಲ್ಪಟ್ಟ
ತೆರೆ-ಯಿಂದ
ತೆರೆಯು
ತೆರೆ-ಯು-ವನೊ
ತೆರೆ-ಯು-ವಾ-ಗಲೂ
ತೆರೆ-ಯು-ವುವು
ತೆರೆಯೇ
ತೆಳ್ಳ-ಗಿ-ದ್ದರೆ
ತೆಳ್ಳಗೆ
ತೆಳ್ಳ-ನೆಯ
ತೆವ-ಳಿ-ಕೊಂಡು
ತೆವ-ಳುತ್ತಾ
ತೆವ-ಳು-ತ್ತಿ-ರುವ
ತೆವ-ಳು-ತ್ತಿ-ರು-ವನು
ತೇ
ತೇಗಿ-ನಲ್ಲಿ
ತೇಗು-ವಂತೆ
ತೇಜಃ
ತೇಜ-ಶ್ಚಾಸ್ಮಿ
ತೇಜಸ್
ತೇಜ-ಸ್ತೇ-ಜ-ಸ್ವಿ-ನಾ-ಮ-ಹಮ್
ತೇಜಸ್ವಿ
ತೇಜ-ಸ್ವಿ-ಗಳ
ತೇಜ-ಸ್ವಿ-ಗಳಲ್ಲಿ
ತೇಜ-ಸ್ಸನ್ನು
ತೇಜ-ಸ್ಸಿಗೆ
ತೇಜ-ಸ್ಸಿನ
ತೇಜ-ಸ್ಸಿ-ನಿಂದ
ತೇಜಸ್ಸು
ತೇಜ-ಸ್ಸು-ಗ-ಳೆಲ್ಲ
ತೇಜಸ್ಸೂ
ತೇಜಸ್ಸೆ
ತೇಜೋ
ತೇಜೋ-ಭಿ-ರಾ-ಪೂರ್ಯ
ತೇಜೋ-ಮಯಂ
ತೇಜೋ-ಮ-ಯ-ನಾದ
ತೇಜೋ-ಮ-ಯ-ವಾ-ಗಿದೆ
ತೇಜೋ-ಮ-ಯವೂ
ತೇಜೋ-ರಾಶಿ
ತೇಜೋ-ರಾ-ಶಿಂ
ತೇಜೋ-ರಾ-ಶಿಯೂ
ತೇಜೋ-ವಧೆ
ತೇಜೋ-ಽಂಶ-ಸಂ-ಭ-ವಮ್
ತೇನ
ತೇನಾ-ಹ-ಮಿಷ್ಟಃ
ತೇನೇ-ದ-ಮಾ-ವೃ-ತಮ್
ತೇನೈವ
ತೇರ್ಗ-ಡೆ-ಯಾಗಿ
ತೇರ್ಗ-ಡೆ-ಯಾ-ಗು-ತ್ತೇ-ನೆಯೋ
ತೇಲಿ-ಕೊಂಡು
ತೇಲಿ-ಸಿ-ಬಿ-ಡು-ತ್ತೇವೆ
ತೇಲಿ-ಸು-ವುದು
ತೇಲು-ತ್ತಿದೆ
ತೇಲು-ತ್ತಿ-ರುವ
ತೇಲು-ತ್ತಿ-ರು-ವಂತೆ
ತೇಲು-ತ್ತಿ-ರು-ವನು
ತೇಲು-ತ್ತಿ-ರು-ವ-ವ-ನಿಗೆ
ತೇಲು-ತ್ತಿ-ರು-ವ-ವನು
ತೇಲು-ತ್ತಿ-ರು-ವುದು
ತೇಲು-ತ್ತಿವೆ
ತೇಲು-ತ್ತಿ-ವೆಯೊ
ತೇಲುವ
ತೇಲು-ವಂ-ತಿದೆ
ತೇಲು-ವಂತೆ
ತೇಲು-ವನು
ತೇಲು-ವರು
ತೇಲು-ವುದು
ತೇವ
ತೇಷಾಂ
ತೇಷಾ-ಮಹಂ
ತೇಷಾ-ಮಾ-ದಿ-ತ್ಯ-ವ-ಜ್ಜ್ಞಾನಂ
ತೇಷಾ-ಮೇ-ವಾ-ನು-ಕಂ-ಪಾ-ರ್ಥ-ಮ-ಹ-ಮ-ಜ್ಞಾ-ನಜಂ
ತೇಷು
ತೇಷ್ವ-ವ-ಸ್ಥಿತಃ
ತೇಽದ್ಯ
ತೇಽಪಿ
ತೇಽಭಿ-ಹಿತಾ
ತೇಽವ್ಯ-ಯಾಮ್
ತೇಽಹಂ
ತೇಽಹೋ-ರಾ-ತ್ರ-ವಿದೋ
ತೈತ್ತಿ-ರೀಯ
ತೈರ್ಜಿತಃ
ತೈರ್ದ-ತ್ತಾ-ನ-ಪ್ರ-ದಾ-ಯೈಭ್ಯೋ
ತೈಲ-ದಂತೆ
ತೈಲ-ದಿಂದ
ತೈಲ-ಪೂರ್ಣಃ
ತೈಲ-ವನ್ನು
ತೊಂದರೆ
ತೊಂದ-ರೆ-ಗಳು
ತೊಂದ-ರೆ-ಯ-ನ್ನಾ-ದರೂ
ತೊಂದ-ರೆ-ಯನ್ನು
ತೊಂದ-ರೆ-ಯ-ನ್ನುಂಟು
ತೊಂಬ-ತ್ತ-ರಷ್ಟು
ತೊಗ-ಲು-ಗೊಂ-ಬೆ-ಗಳು
ತೊಟ್ಟಿ
ತೊಟ್ಟಿ-ಕ್ಕು-ತ್ತಿ-ರು-ವುದು
ತೊಟ್ಟಿ-ಕ್ಕು-ವಂತೆ
ತೊಟ್ಟಿ-ಕ್ಕು-ವನು
ತೊಟ್ಟಿ-ಕ್ಕು-ವು-ದಿಲ್ಲ
ತೊಟ್ಟಿ-ಕ್ಕು-ವುದು
ತೊಟ್ಟಿ-ಯನ್ನು
ತೊಟ್ಟಿ-ರುವ
ತೊಟ್ಟಿ-ರು-ವನೊ
ತೊಟ್ಟಿ-ರು-ವೆ-ಯಾ-ಇ-ದನ್ನು
ತೊಟ್ಟು
ತೊಟ್ಟು-ಕೊಂ-ಡಿ-ರುವ
ತೊಟ್ಟೂ
ತೊಡ-ಕಾದ
ತೊಡಕು
ತೊಡ-ಗಿ-ದರೆ
ತೊಡ-ಗಿ-ದ್ದರೂ
ತೊಡ-ಗಿ-ದ್ದಾನೆ
ತೊಡ-ಗಿ-ರು-ವು-ದಕ್ಕೂ
ತೊಡ-ಗಿ-ಸು-ವುದು
ತೊಡ-ಗು-ವನು
ತೊಡ-ಗು-ವುದು
ತೊಡ-ರು-ಗಳು
ತೊಡಿಸಿ
ತೊಡೆ
ತೊಡೆಯ
ತೊಡೆ-ಯಿಂದ
ತೊನೆ-ಯುವ
ತೊರೆ
ತೊರೆ-ದರೆ
ತೊರೆ-ದಿ-ರು-ವನು
ತೊರೆದು
ತೊರೆಯು
ತೊರೆ-ಯು-ವನು
ತೊರೆ-ಯು-ವು-ದಿಲ್ಲ
ತೊರೆ-ಯು-ವೆವೊ
ತೊಲ
ತೊಲಗಿ
ತೊಲು-ಗು-ವುದೊ
ತೊಲೆಯ
ತೊಳಲಿ
ತೊಳ-ಲು-ತ್ತಿ-ರು-ವನು
ತೊಳ-ಲು-ತ್ತಿಲ್ಲ
ತೊಳೆ-ದ-ಮೇಲೆ
ತೊಳೆ-ದರೂ
ತೊಳೆ-ದರೆ
ತೊಳೆ-ದಿ-ರು-ವನು
ತೊಳೆದು
ತೊಳೆ-ದು-ಕೊಂ-ಡ-ವನು
ತೊಳೆ-ದು-ಕೊಂ-ಡಿ-ರ-ಬೇಕು
ತೊಳೆ-ದು-ಕೊಂ-ಡಿ-ರು-ವನು
ತೊಳೆ-ದು-ಕೊಂ-ಡಿ-ರು-ವರು
ತೊಳೆ-ದು-ಕೊಂಡು
ತೊಳೆ-ದು-ಕೊ-ಳ್ಳ-ಬಲ್ಲ
ತೊಳೆ-ದು-ಕೊ-ಳ್ಳ-ಬೇ-ಕಾ-ಗಿದೆ
ತೊಳೆ-ದು-ಕೊ-ಳ್ಳು-ತ್ತೇ-ವೆಯೊ
ತೊಳೆ-ದು-ಕೊ-ಳ್ಳು-ವು-ದಕ್ಕೆ
ತೊಳೆ-ದು-ಹೋದ
ತೊಳೆಯ
ತೊಳೆ-ಯ-ಬೇ-ಕಾ-ಗಿಲ್ಲ
ತೊಳೆ-ಯ-ಬೇಕು
ತೊಳೆ-ಯ-ಲಾ-ರದ
ತೊಳೆಯು
ತೊಳೆ-ಯು-ತ್ತಾನೆ
ತೊಳೆ-ಯು-ತ್ತಾಳೆ
ತೊಳೆ-ಯು-ತ್ತಿ-ರು-ವುದೊ
ತೊಳೆ-ಯು-ತ್ತೇ-ವೆಯೊ
ತೊಳೆ-ಯುವ
ತೊಳೆ-ಯು-ವಳು
ತೊಳೆ-ಯು-ವುದು
ತೋಚಿದ
ತೋಚಿ-ದು-ದನ್ನು
ತೋಚಿ-ದ್ದನ್ನು
ತೋಟಕ್ಕೆ
ತೋಟ-ಗಾರ
ತೋಟ-ಗಾ-ರನ
ತೋಟ-ಗಾ-ರ-ನನ್ನು
ತೋಡ-ಬೇಕು
ತೋಡಲು
ತೋಡಿ
ತೋಡಿ-ರುವ
ತೋಡುತ್ತಾ
ತೋಡು-ವನು
ತೋತಾ
ತೋತಾ-ಪುರಿ
ತೋತ್ರ-ವೇ-ತ್ರೈ-ಕ-ಪಾ-ಣಯೇ
ತೋಯಂ
ತೋರ
ತೋರಣ
ತೋರ-ಣ-ವನ್ನು
ತೋರದು
ತೋರದೆ
ತೋರ-ಬ-ಹುದು
ತೋರ-ಬ-ಹು-ದು-ಅ-ದ-ರ-ರಿಂದ
ತೋರ-ಬೇ-ಕಾ-ದರೆ
ತೋರ-ಬೇಕು
ತೋರ-ಲಿಲ್ಲ
ತೋರಲು
ತೋರ-ಲೇ-ಬೇಕು
ತೋರಿ
ತೋರಿಕೆ
ತೋರಿ-ಕೆ-ಗಾಗಿ
ತೋರಿ-ಕೆ-ಗಾ-ಗಿಯೂ
ತೋರಿ-ಕೆಗೆ
ತೋರಿ-ಕೆಯ
ತೋರಿ-ಕೆ-ಯದು
ತೋರಿ-ಕೆ-ಯನ್ನು
ತೋರಿ-ಕೆಯೂ
ತೋರಿದ
ತೋರಿ-ದಂತೆ
ತೋರಿ-ದನು
ತೋರಿ-ದರು
ತೋರಿ-ದರೂ
ತೋರಿ-ದರೆ
ತೋರಿ-ದ-ವನು
ತೋರಿ-ದ-ವರೂ
ತೋರಿ-ದಾಗ
ತೋರಿ-ದು-ದನ್ನು
ತೋರಿ-ದ್ದ-ರಿಂದ
ತೋರಿದ್ದು
ತೋರಿ-ರು-ವನು
ತೋರಿಲ್ಲ
ತೋರಿಸ
ತೋರಿ-ಸ-ಕೂ-ಡದು
ತೋರಿ-ಸ-ಬೇ-ಕೆಂದು
ತೋರಿ-ಸಲು
ತೋರಿ-ಸ-ಲೇ-ಬೇಕು
ತೋರಿಸಿ
ತೋರಿ-ಸಿ-ಕೊಂಡು
ತೋರಿ-ಸಿ-ಕೊ-ಳ್ಳು-ತ್ತಾರೆ
ತೋರಿ-ಸಿ-ಕೊ-ಳ್ಳು-ವನು
ತೋರಿ-ಸಿ-ಕೊ-ಳ್ಳು-ವು-ದ-ಕ್ಕಲ್ಲ
ತೋರಿ-ಸಿ-ಕೊ-ಳ್ಳು-ವು-ದ-ಕ್ಕಾಗಿ
ತೋರಿ-ಸಿ-ಕೊ-ಳ್ಳು-ವು-ದರ
ತೋರಿ-ಸಿ-ಕೊ-ಳ್ಳು-ವುದು
ತೋರಿ-ಸಿದ
ತೋರಿ-ಸಿ-ದನು
ತೋರಿ-ಸಿ-ದರೆ
ತೋರಿ-ಸಿ-ದಾಗ
ತೋರಿ-ಸಿ-ದೆನು
ತೋರಿ-ಸಿ-ದ್ದನ್ನು
ತೋರಿ-ಸಿ-ದ್ದಾನೆ
ತೋರಿಸು
ತೋರಿ-ಸು-ಕೊ-ಳ್ಳು-ವನು
ತೋರಿ-ಸು-ತ್ತಾನೆ
ತೋರಿ-ಸು-ತ್ತಿದೆ
ತೋರಿ-ಸು-ತ್ತಿ-ದ್ದರು
ತೋರಿ-ಸು-ತ್ತಿ-ದ್ದುವು
ತೋರಿ-ಸು-ತ್ತಿ-ರು-ವುದೋ
ತೋರಿ-ಸು-ವನು
ತೋರಿ-ಸು-ವ-ವರು
ತೋರಿ-ಸು-ವಿರಾ
ತೋರಿ-ಸು-ವು-ದಿಲ್ಲ
ತೋರಿ-ಸು-ವುದು
ತೋರಿ-ಸು-ವುದೇ
ತೋರಿ-ಸು-ವುದೋ
ತೋರು
ತೋರು-ತ್ತದೆ
ತೋರು-ತ್ತ-ದೆಯೋ
ತೋರು-ತ್ತಾನೆ
ತೋರು-ತ್ತಾರೆ
ತೋರು-ತ್ತಿದ್ದ
ತೋರು-ತ್ತಿರು
ತೋರು-ತ್ತಿ-ರು-ತ್ತದೆ
ತೋರು-ತ್ತಿ-ರುವ
ತೋರು-ತ್ತಿ-ರು-ವನು
ತೋರು-ತ್ತಿ-ರು-ವುದು
ತೋರು-ತ್ತಿ-ರು-ವುದೋ
ತೋರು-ತ್ತೇವೆ
ತೋರು-ತ್ತೇ-ವೆ-ಅಂ-ತಹ
ತೋರುವ
ತೋರು-ವನು
ತೋರು-ವನೋ
ತೋರು-ವರು
ತೋರು-ವ-ವ-ನಲ್ಲ
ತೋರು-ವ-ವ-ನಿ-ಗಾಗಿ
ತೋರು-ವ-ವನು
ತೋರು-ವ-ವರ
ತೋರು-ವ-ವ-ರಲ್ಲ
ತೋರು-ವು-ದ-ಕ್ಕಾಗಿ
ತೋರು-ವು-ದಕ್ಕೆ
ತೋರು-ವು-ದಿಲ್ಲ
ತೋರು-ವು-ದಿ-ಲ್ಲವೊ
ತೋರು-ವುದು
ತೋರು-ವು-ದುಕ್ಕೇ
ತೋರು-ವುದೊ
ತೋರು-ವುದೋ
ತೋರು-ವುವು
ತೋರು-ವೆಡೆ
ತೋರು-ವೆವೊ
ತೋರೆಂದು
ತೋರ್ಪ-ಡಿ-ಸಿ-ಕೊ-ಳ್ಳು-ವು-ದಕ್ಕೆ
ತೋಳು
ತೌ
ತೌರು-ಮನೆ
ತೌರು-ಮ-ನೆ-ಯಲ್ಲಿ
ತೌರೂ-ರಾದ
ತೌರೂ-ರಿಗೆ
ತ್ತದೆ
ತ್ತಮ-ನೆಂದು
ತ್ತಮನೇ
ತ್ತವೆ
ತ್ತವೆಯೋ
ತ್ತಾನೆ
ತ್ತಾನೆಯೆ
ತ್ತಾನೆಯೊ
ತ್ತಾನೆಯೋ
ತ್ತಾನೊ
ತ್ತಾರೆ
ತ್ತಾರೆಯೋ
ತ್ತಾರೋ
ತ್ತಿತ್ತೊ
ತ್ತಿದೆ
ತ್ತಿದ್ದರು
ತ್ತಿದ್ದರೂ
ತ್ತಿದ್ದರೆ
ತ್ತಿದ್ದಳು
ತ್ತಿದ್ದಾಗ
ತ್ತಿದ್ದಿರೋ
ತ್ತಿರ-ಬ-ಹುದು
ತ್ತಿರ-ಬೇಕು
ತ್ತಿರಲಿ
ತ್ತಿರ-ಲಿಲ್ಲ
ತ್ತಿರುವ
ತ್ತಿರು-ವನು
ತ್ತಿರು-ವರು
ತ್ತಿರು-ವಳು
ತ್ತಿರು-ವ-ವನು
ತ್ತಿರು-ವ-ವರು
ತ್ತಿರು-ವಾಗ
ತ್ತಿರು-ವುದು
ತ್ತಿರು-ವುದೇ
ತ್ತಿರುವೆ
ತ್ತಿರು-ವೆವು
ತ್ತಿರು-ವೆವೊ
ತ್ತೆಂಟು
ತ್ತೇನೆ
ತ್ತೇನೆಯೊ
ತ್ತೇವೆ
ತ್ತೇವೆಯೊ
ತ್ತೇವೆ-ಯೊ-ಅ-ದಕ್ಕೆ
ತ್ತೇವೆಯೋ
ತ್ಯಕ್ತ-ಜೀ-ವಿ-ತಾಃ
ತ್ಯಕ್ತ-ಸ-ರ್ವ-ಪ-ರಿ-ಗ್ರಹಃ
ತ್ಯಕ್ತುಂ
ತ್ಯಕ್ತ್ವಾ
ತ್ಯಕ್ತ್ವಾ-ತ್ಮ-ಶು-ದ್ಧಯೇ
ತ್ಯಕ್ತ್ವೋ-ತ್ತಿಷ್ಠ
ತ್ಯಜ-ತ್ಯಂತೇ
ತ್ಯಜನ್
ತ್ಯಜಿ-ಸ-ತ-ಕ್ಕ-ವ-ರಲ್ಲ
ತ್ಯಜಿ-ಸ-ಬ-ಹುದು
ತ್ಯಜಿ-ಸ-ಬ-ಹುದೇ
ತ್ಯಜಿ-ಸ-ಬೇ-ಕಾ-ಗಿದೆ
ತ್ಯಜಿ-ಸ-ಬೇ-ಕಾ-ಗು-ವುದು
ತ್ಯಜಿ-ಸ-ಬೇ-ಕಾ-ದರೆ
ತ್ಯಜಿ-ಸ-ಬೇಕು
ತ್ಯಜಿ-ಸಲು
ತ್ಯಜಿಸಿ
ತ್ಯಜಿ-ಸಿದ
ತ್ಯಜಿ-ಸಿ-ದರೆ
ತ್ಯಜಿ-ಸಿ-ದ-ಲ್ಲದೆ
ತ್ಯಜಿ-ಸಿ-ದ-ವನು
ತ್ಯಜಿ-ಸಿ-ದಾಗ
ತ್ಯಜಿ-ಸು-ತ್ತಾನೆ
ತ್ಯಜಿ-ಸು-ವನು
ತ್ಯಜಿ-ಸು-ವನೊ
ತ್ಯಜಿ-ಸು-ವು-ದಕ್ಕೆ
ತ್ಯಜಿ-ಸು-ವು-ದಿಲ್ಲ
ತ್ಯಜಿ-ಸು-ವುದು
ತ್ಯಜೇತ್
ತ್ಯಾಗ
ತ್ಯಾಗಂ
ತ್ಯಾಗಃ
ತ್ಯಾಗದ
ತ್ಯಾಗ-ದಿಂದ
ತ್ಯಾಗ-ದಿಂ-ದಲೇ
ತ್ಯಾಗ-ಫಲಂ
ತ್ಯಾಗ-ಮಾ-ಡಿದ
ತ್ಯಾಗ-ಮಾ-ಡಿ-ದ-ಮೇಲೆ
ತ್ಯಾಗ-ಮಾ-ಡಿ-ದ-ವನು
ತ್ಯಾಗ-ಮಾ-ಡಿ-ರು-ವನು
ತ್ಯಾಗ-ಮಾ-ಡಿ-ರು-ವನೊ
ತ್ಯಾಗ-ಮಾಡು
ತ್ಯಾಗ-ರಾ-ಜರು
ತ್ಯಾಗ-ವನ್ನು
ತ್ಯಾಗಸ್ಯ
ತ್ಯಾಗಿ
ತ್ಯಾಗಿಗೆ
ತ್ಯಾಗಿ-ಯಲ್ಲಿ
ತ್ಯಾಗೀ
ತ್ಯಾಗೀ-ತ್ಯ-ಭಿ-ಧೀ-ಯತೇ
ತ್ಯಾಗೇ
ತ್ಯಾಗೋ
ತ್ಯಾಜ್ಯಂ
ತ್ಯಾಜ್ಯ-ಮಿತಿ
ತ್ರಯೀ-ಧ-ರ್ಮ-ಮ-ನು-ಪ್ರ-ಪನ್ನಾ
ತ್ರಾಣ-ವಿಲ್ಲ
ತ್ರಾಯತೇ
ತ್ರಿಕಾ-ಲಕ್ಕೂ
ತ್ರಿಕಾ-ಲ-ದ-ಲ್ಲಿಯೂ
ತ್ರಿಕೋ-ಣದ
ತ್ರಿಕೋ-ಣ-ವನ್ನು
ತ್ರಿಗು-ಣ-ಗಳ
ತ್ರಿಗು-ಣ-ಗಳನ್ನೆಲ್ಲ
ತ್ರಿಗು-ಣ-ಗಳಲ್ಲಿ
ತ್ರಿಗು-ಣ-ಗಳಿಂದ
ತ್ರಿಗು-ಣ-ಗ-ಳಿಗೂ
ತ್ರಿಗು-ಣ-ಗ-ಳಿಗೆ
ತ್ರಿಗು-ಣ-ಗಳು
ತ್ರಿಗು-ಣ-ಮ-ಯ-ವಾದ
ತ್ರಿಧೈವ
ತ್ರಿಭಿ-ರ್ಗು-ಣ-ಮ-ಯೈ-ರ್ಭಾ-ವೈ-ರೇ-ಭಿಃ
ತ್ರಿಮೂ-ರ್ತಿ-ಗಳ
ತ್ರಿವಿಧಂ
ತ್ರಿವಿಧಃ
ತ್ರಿವಿಧಾ
ತ್ರಿವಿಧೋ
ತ್ರಿಶಂ-ಕು-ವಿನ
ತ್ರಿಶಂ-ಕು-ವಿ-ನಂತೆ
ತ್ರಿಷು
ತ್ರೀನ್
ತ್ರೈಗು-ಣ್ಯ-ವಿ-ಷಯಾ
ತ್ರೈಲೋ-ಕ್ಯ-ರಾ-ಜ್ಯಸ್ಯ
ತ್ರೈವಿದ್ಯಾ
ತ್ವಂ
ತ್ವಂತ-ಗತಂ
ತ್ವಕ್ಚೈವ
ತ್ವಕ್ಷ-ರ-ಮ-ನಿ-ರ್ದೇ-ಶ್ಯ-ಮ-ವ್ಯಕ್ತಂ
ತ್ವಘಂ
ತ್ವತ್ತಃ
ತ್ವತ್ಪ್ರ-ಸಾ-ದಾ-ನ್ಮ-ಯಾ-ಚ್ಯುತ
ತ್ವತ್ಸ-ಮೋ-ಽಸ್ತ್ಯ-ಭ್ಯ-ಧಿಕಃ
ತ್ವದನ್ಯಃ
ತ್ವದ-ನ್ಯೇನ
ತ್ವನ-ನ್ಯಯಾ
ತ್ವನಾ-ವೃ-ತ್ತಿ-ಮಾ-ವೃ-ತ್ತಿಂ
ತ್ವಮ-ಕ್ಷರಂ
ತ್ವಮ-ವ್ಯಯಃ
ತ್ವಮಸ್ಯ
ತ್ವಮಾ-ತ್ಮಾನಂ
ತ್ವಮಾ-ದಿ-ದೇವಃ
ತ್ವಮಾದೌ
ತ್ವಮಿಮಂ
ತ್ವಮು-ತ್ತಿಷ್ಠ
ತ್ವಯಾ
ತ್ವಯೈ-ಕಾ-ಗ್ರೇಣ
ತ್ವಯೈ-ಕೇನ
ತ್ವರ-ಮಾಣಾ
ತ್ವರೆ
ತ್ವರೆ-ಯಿಂದ
ತ್ವಹಂ
ತ್ವಾ
ತ್ವಾಂ
ತ್ವಾತ್ಮೈವ
ತ್ವಾಮ-ನು-ಸಂ-ದ-ಧಾಮಿ
ತ್ವಾಮ-ಹ-ಮ-ಪ್ರ-ಮೇ-ಯಮ್
ತ್ವಾಮ-ಹ-ಮೀ-ಷ-ಮೀ-ಡ್ಯಮ್
ತ್ವಿದ-ಮೇ-ತೇ-ಷಾಂ
ತ್ವಿದಾ-ನೀಂ
ತ್ವಿಮಾಂ
ತ್ವೇತ-ದ-ಭ್ಯ-ಸೂ-ಯಂತೋ
ತ್ವೇವ-ಮ-ಜಾ-ನಂತಃ
ತ್ವೇವಾಹಂ
ಥಳು-ಕಿಲ್ಲ
ದಂಗು-ಮಾಡಿ
ದಂಗೆ
ದಂಗೆ-ಯೆ-ದ್ದಿದೆ
ದಂಡ
ದಂಡಕ್ಕೆ
ದಂಡ-ದಂತೆ
ದಂಡ-ನ-ಮ-ಸ್ಕಾ-ರ-ದಿಂದ
ದಂಡ-ನ-ಮ-ಸ್ಕಾ-ರ-ವನ್ನು
ದಂಡ-ನೆಗೆ
ದಂಡ-ನೆಯ
ದಂಡ-ನೆ-ಯನ್ನು
ದಂಡ-ಪ್ರ-ಣಾಮ
ದಂಡಿ-ಸದೆ
ದಂಡಿ-ಸದೇ
ದಂಡಿ-ಸ-ಬಾ-ರದು
ದಂಡಿ-ಸ-ಬೇಕು
ದಂಡಿಸಿ
ದಂಡಿ-ಸಿ-ದರೆ
ದಂಡಿ-ಸು-ವ-ವರ
ದಂಡಿ-ಸು-ವು-ದಕ್ಕೆ
ದಂಡಿ-ಸು-ವು-ದಿ-ಲ್ಲವೋ
ದಂಡೋ
ದಂತಾ-ಗು-ವುದು
ದಂತಿದೆ
ದಂತೆ
ದಂಪ-ತಿ-ಗ-ಳಿ-ದ್ದರು
ದಂಭ
ದಂಭ-ಮಾ-ನ-ಮ-ದಾ-ನ್ವಿ-ತಾಃ
ದಂಭಾ-ರ್ಥ-ಮಪಿ
ದಂಭಾ-ಹಂ-ಕಾ-ರ-ಸಂ-ಯು-ಕ್ತಾಃ
ದಂಭೇನ
ದಂಭೇ-ನಾ-ವಿ-ಧಿ-ಪೂ-ರ್ವ-ಕಮ್
ದಂಭೋ
ದಂಷ್ಟ್ರ
ದಂಷ್ಟ್ರಾ-ಕ-ರಾ-ಲಾನಿ
ದಕ್ಕಾಗಿ
ದಕ್ಕುವ
ದಕ್ಕು-ವು-ದಿಲ್ಲ
ದಕ್ಕೆ
ದಕ್ಕೆಲ್ಲಾ
ದಕ್ಷ
ದಕ್ಷ-ತೆ-ಯಿಂದ
ದಕ್ಷನ
ದಕ್ಷನೋ
ದಕ್ಷರು
ದಕ್ಷಿಣ
ದಕ್ಷಿ-ಣಾ-ಯನ
ದಕ್ಷಿ-ಣಾ-ಯ-ನದ
ದಕ್ಷಿ-ಣಾ-ಯ-ನಮ್
ದಕ್ಷಿ-ಣಾ-ಶೂನ್ಯ
ದಕ್ಷಿಣೆ
ದಕ್ಷಿ-ಣೆ-ಯನ್ನು
ದಕ್ಷಿ-ಣೇ-ಶ್ವರ
ದಗ್ಧ-ವಾಗಿ
ದಡ-ಗಳ
ದಡ-ಗಳು
ದಡ-ಗಳೇ
ದಡದ
ದಡವೇ
ದಡ್ಡ
ದಡ್ಡ-ನಾಗಿ
ದಡ್ಡ-ನಿಗೆ
ದಡ್ಡ-ರ-ನ್ನಾಗಿ
ದಡ್ಡ-ರಿ-ರು-ವರು
ದಡ್ಡರು
ದಣಿಗೆ
ದಣಿ-ದ-ಮೇಲೆ
ದಣಿ-ವಾ-ದಾಗ
ದತ್ತಂ
ದತ್ತು
ದದಾಮಿ
ದದಾ-ಮ್ಯ-ಹಮ್
ದದಾಸಿ
ದಧೀಚಿ
ದಧ್ಮುಃ
ದಧ್ಮೌ
ದನ
ದನಕ್ಕೆ
ದನ-ಗಳನ್ನು
ದನ-ಗಳು
ದನ-ಗಳೇ
ದನದ
ದನ-ದಂತೆ
ದನ-ವನ್ನು
ದನವೊ
ದನು
ದನ್ನು
ದನ್ನೂ
ದನ್ನೆಲ್ಲ
ದಪ್ತ-ರನ್ನು
ದಪ್ಪಗೆ
ದಪ್ಪ-ವಾಗಿ
ದಪ್ಪ-ವಾ-ಗಿ-ದ್ದೇನೆ
ದಪ್ಪ-ವಾದ
ದಬ್ಬಿ
ದಬ್ಬಿ-ದರೆ
ದಬ್ಬು-ತ್ತೇ-ವೆಯೋ
ದಮ
ದಮ-ಇಂ-ದ್ರಿ-ಯ-ಗಳ
ದಮಃ
ದಮನ
ದಮ-ನ-ಮಾ-ಡು-ವು-ದುಕ್ಕೆ
ದಮ-ಯ-ತಾ-ಮಸ್ಮಿ
ದಮಶ್ಚ
ದಮ-ಸ್ತಪಃ
ದಮ್ಮ-ಯ್ಯ-ಗುಡ್ಡೆ
ದಯ-ದಿಂದ
ದಯ-ಪಾ-ಲಿ-ಸಿ-ದ-ನಲ್ಲ
ದಯ-ವಿಟ್ಟು
ದಯ-ವಿಲ್ಲ
ದಯಾ
ದಯಾ-ದಾ-ಕ್ಷಿಣ್ಯ
ದಯಾ-ದಾ-ಕ್ಷಿ-ಣ್ಯ-ವನ್ನೇ
ದಯಾ-ಮಯ
ದಯಾ-ಮ-ಯ-ನಾದ
ದಯಾ-ಮ-ಯಿ-ಯಾದ
ದಯೆ
ದಯೆ-ಯಂತೆ
ದಯೆ-ಯಿಂದ
ದಯೆ-ಯಿಂ-ದಲೋ
ದಯೆ-ಯೆಂ-ಬುದು
ದರಲ್ಲಿ
ದರು
ದರೂ
ದರೆ
ದರೋಡೆ
ದರೋ-ಡೆಗೆ
ದರ್ಜೆಗೆ
ದರ್ಪ
ದರ್ಪಂ
ದರ್ಪೋ-ಽತಿ-ಮಾ-ನಶ್ಚ
ದರ್ಬಾ-ರನ್ನು
ದರ್ಶ-ಕರು
ದರ್ಶನ
ದರ್ಶ-ನ-ಕಾಂ-ಕ್ಷಿಣಃ
ದರ್ಶ-ನ-ಗಳನ್ನು
ದರ್ಶ-ನ-ಗಳು
ದರ್ಶ-ನದ
ದರ್ಶ-ನ-ದಿಂದ
ದರ್ಶ-ನ-ವಾ-ಗು-ವುದು
ದರ್ಶ-ನ-ವಿಲ್ಲ
ದರ್ಶಯ
ದರ್ಶ-ಯಾ-ತ್ಮಾ-ನ-ಮ-ವ್ಯ-ಯಮ್
ದರ್ಶ-ಯಾ-ಮಾಸ
ದರ್ಶಿ-ತ-ಮಾ-ತ್ಮ-ಯೋ-ಗಾತ್
ದಲೂ
ದಲ್ಲ
ದಲ್ಲಂತೂ
ದಲ್ಲಾ-ಗಲಿ
ದಲ್ಲಾ-ಗಲೀ
ದಲ್ಲಿ
ದಲ್ಲಿ-ಇಲ್ಲ
ದಲ್ಲಿಟ್ಟು
ದಲ್ಲಿದೆ
ದಲ್ಲಿ-ದ್ದರೆ
ದಲ್ಲಿ-ದ್ದಾಗ
ದಲ್ಲಿಯೂ
ದಲ್ಲಿ-ರು-ತ್ತೇವೆ
ದಲ್ಲಿ-ರುವ
ದಲ್ಲಿ-ರು-ವನು
ದಲ್ಲಿ-ರು-ವಾಗ
ದಲ್ಲಿ-ರು-ವಾ-ಗಲೇ
ದಲ್ಲಿ-ರು-ವುದೇ
ದಲ್ಲಿವೆ
ದಲ್ಲೆ
ದಲ್ಲೆಲ್ಲಾ
ದಳ
ದಳ್ಳುರಿ
ದಳ್ಳು-ರಿ-ಯ-ನ್ನಾಗಿ
ದವ-ಡೆಗೆ
ದವ-ನಲ್ಲ
ದವ-ನಾಗಿ
ದವನು
ದವನೂ
ದವರು
ದವ-ಸ-ಧಾ-ನ್ಯ-ಗಳನ್ನು
ದಶ-ನಾಂ-ತ-ರೇಷು
ದಶ-ರಥ
ದಶ-ರ-ಥನ
ದಶೈಕಂ
ದಹತಿ
ದಹಿ-ಸಿ-ಕೊಂ-ಡಿ-ರು-ವರೊ
ದಹಿ-ಸಿ-ಬಿ-ಡು-ವನು
ದಹಿ-ಸು-ತ್ತಿ-ರುವ
ದಹಿ-ಸು-ವು-ದಕ್ಕೆ
ದಹಿ-ಸು-ವುದು
ದಾಂಧಲೆ
ದಾಂಧ-ಲೆಯೂ
ದಾಗ
ದಾಗಿ
ದಾಗಿ-ರಲಿ
ದಾಗುವ
ದಾಗು-ವು-ದಿಲ್ಲ
ದಾಟ-ಬ-ಹುದು
ದಾಟ-ಬೇ-ಕಾ-ಗಿದೆ
ದಾಟ-ಲಾ-ರೆಯ
ದಾಟಲು
ದಾಟಿ
ದಾಟಿದ
ದಾಟಿ-ದನೆ
ದಾಟಿ-ದರು
ದಾಟಿ-ದ-ವನು
ದಾಟಿದೆ
ದಾಟಿ-ರ-ಬೇಕು
ದಾಟಿ-ಸಿದ
ದಾಟಿ-ಹೋ-ಗಲು
ದಾಟಿ-ಹೋ-ಗಿ-ರು-ವನು
ದಾಟಿ-ಹೋ-ಗು-ತ್ತೇ-ವೆಯೊ
ದಾಟು-ತ್ತಾರೆ
ದಾಟು-ತ್ತೇವೆ
ದಾಟು-ವರು
ದಾಟು-ವಾಗ
ದಾಟು-ವು-ದಕ್ಕೆ
ದಾಟುವೆ
ದಾಡೆ-ಗಳಿಂದ
ದಾಡೆ-ಗ-ಳಿಗೆ
ದಾಡೆ-ಯೊ-ಳಗೆ
ದಾತ-ವ್ಯ-ಮಿತಿ
ದಾದರೂ
ದಾದ-ರೊಂದು
ದಾನ
ದಾನಂ
ದಾನಕ್ಕೆ
ದಾನ-ಕ್ರಿ-ಯಾಶ್ಚ
ದಾನ-ಕ್ರಿ-ಯೆ-ಗಳನ್ನು
ದಾನ-ಗಳನ್ನು
ದಾನ-ಗಳಲ್ಲಿ
ದಾನ-ಗ-ಳಿಂ-ದ-ಲಾ-ಗಲೀ
ದಾನ-ಗ-ಳಿವೆ
ದಾನದ
ದಾನ-ದಲ್ಲಿ
ದಾನ-ದಿಂದ
ದಾನ-ಧ-ರ್ಮ-ಗಳನ್ನು
ದಾನ-ಧ-ರ್ಮ-ಗಳನ್ನೂ
ದಾನ-ಮಾ-ಡ-ಬ-ಲ್ಲುದು
ದಾನ-ಮಾ-ಡಿ-ದ-ವನು
ದಾನ-ಮಾ-ಡಿ-ದ್ದರು
ದಾನ-ಮಾ-ಡಿ-ರು-ವರು
ದಾನ-ಮಾ-ಡಿ-ರು-ವು-ರಿಂದ
ದಾನ-ಮಾ-ಡು-ವಾಗ
ದಾನ-ಮೀ-ಶ್ವ-ರ-ಭಾ-ವಶ್ಚ
ದಾನವ
ದಾನ-ವನ್ನು
ದಾನ-ವರು
ದಾನ-ವಾಃ
ದಾನ-ವಾಗಿ
ದಾನ-ವೆಂದು
ದಾನಾ-ದಿ-ಗಳನ್ನು
ದಾನಾ-ದಿ-ಗಳಿಂದ
ದಾನಿ
ದಾನಿ-ಗ-ಳಲ್ಲ
ದಾನಿ-ಯಾ-ದ-ವನು
ದಾನೇ
ದಾನೇನ
ದಾನೇಷು
ದಾರ
ದಾರ-ಗಳನ್ನು
ದಾರದ
ದಾರ-ದಂತೆ
ದಾರ-ದಲ್ಲಿ
ದಾರ-ದಿಂದ
ದಾರ-ವನ್ನು
ದಾರಿ
ದಾರಿ-ಕೊ-ಟ್ಟಳು
ದಾರಿ-ಗಳನ್ನು
ದಾರಿ-ಗ-ಳ-ಲ್ಲೆಲ್ಲ
ದಾರಿ-ಗ-ಳಿಗೂ
ದಾರಿ-ಗಳು
ದಾರಿ-ಗಾ-ಣದೆ
ದಾರಿಗೆ
ದಾರಿ-ತೋ-ರುವ
ದಾರಿ-ತೋ-ರೆಂದು
ದಾರಿ-ದೋ-ರುವ
ದಾರಿದ್ರ್ಯ
ದಾರಿ-ದ್ರ್ಯ-ದಿಂದ
ದಾರಿ-ದ್ರ್ಯ-ವಿಲ್ಲ
ದಾರಿಯ
ದಾರಿ-ಯನ್ನು
ದಾರಿ-ಯನ್ನೂ
ದಾರಿ-ಯನ್ನೇ
ದಾರಿ-ಯಲ್ಲಿ
ದಾರಿ-ಯ-ಲ್ಲಿ-ರುವ
ದಾರಿ-ಯಲ್ಲೇ
ದಾರಿ-ಯಿದೆ
ದಾರಿಯೂ
ದಾರಿಯೇ
ದಾರುಣ
ದಾರು-ಣತೆ
ದಾರು-ಣ-ವಾದ
ದಾವುದು
ದಾವುದೋ
ದಾಸ
ದಾಸ-ನಂ-ತಿದೆ
ದಾಸ-ನಂತೆ
ದಾಸ-ನ-ನ್ನಾಗಿ
ದಾಸ-ನಲ್ಲ
ದಾಸ-ನಾಗಿ
ದಾಸ-ನಾ-ಗಿದೆ
ದಾಸ-ನಾ-ಗಿ-ರು-ವ-ವ-ನಿಗೆ
ದಾಸ-ನಾ-ಗು-ತ್ತಾನೆ
ದಾಸ-ನಾ-ಗು-ವನು
ದಾಸ-ನಾ-ಗು-ವುದು
ದಾಸ-ರ-ನ್ನಾಗಿ
ದಾಸ-ರಲ್ಲ
ದಾಸ-ರಾ-ಗದ
ದಾಸ-ರಾ-ಗದೆ
ದಾಸ-ರಾ-ಗ-ಬೇ-ಕಾ-ದರೆ
ದಾಸ-ರಾಗಿ
ದಾಸ-ರಾ-ಗಿ-ರು-ವರು
ದಾಸ-ರಾಗು
ದಾಸ-ರಾ-ಗು-ತ್ತವೆ
ದಾಸ-ರಾ-ಗು-ತ್ತೇವೆ
ದಾಸ-ರಾ-ಗು-ವು-ದಿಲ್ಲ
ದಾಸ-ರಾ-ಗು-ವುದು
ದಾಸ-ರಾ-ದ-ವರ
ದಾಸರು
ದಾಸ್ಯಂತೇ
ದಾಸ್ಯಕ್ಕೆ
ದಾಸ್ಯಾಮಿ
ದಾಹ
ದಾಹ-ವನ್ನು
ದಾಹ-ವ-ನ್ನುಂಟು
ದಾಹ-ವಾ-ಗು-ವುದು
ದಿಂದ
ದಿಂದಲೂ
ದಿಂದಲೇ
ದಿಕ್ಕನ್ನು
ದಿಕ್ಕನ್ನೆ
ದಿಕ್ಕನ್ನೇ
ದಿಕ್ಕಿಗೆ
ದಿಕ್ಕಿ-ನಲ್ಲಿ
ದಿಕ್ಕಿ-ನಿಂದ
ದಿಕ್ಕು
ದಿಕ್ಕು-ಕಾ-ಣದೆ
ದಿಕ್ಕು-ಗಳನ್ನು
ದಿಕ್ಕು-ಗಳಲ್ಲಿ
ದಿಕ್ಕು-ಗ-ಳಿಗೂ
ದಿಕ್ಕು-ದಿ-ಕ್ಕಿಗೆ
ದಿಕ್ಕೆ-ಟ್ಟ-ವರು
ದಿಕ್ಸೂಚಿ
ದಿಗಂ-ತ-ದ-ವ-ರೆಗೂ
ದಿಗಂ-ಬರ
ದಿಗೂ
ದಿಗ್ಭ್ರಾಂತ
ದಿಗ್ಭ್ರಾಂ-ತ-ನಾ-ಗು-ವನು
ದಿಗ್ಭ್ರಾಂತಿ
ದಿಟ್ಟ-ತನ
ದಿಟ್ಟ-ತ-ನ-ದಿಂದ
ದಿಟ್ಟ-ತ-ನ-ವಿಲ್ಲ
ದಿಣ್ಣೆ-ಗಳನ್ನೆಲ್ಲ
ದಿನ
ದಿನ-ಕ-ಳೆ-ದಂತೆ
ದಿನಕ್ಕೆ
ದಿನ-ಗಳ
ದಿನ-ಗಳನ್ನು
ದಿನ-ಗಳಲ್ಲಿ
ದಿನ-ಗ-ಳಾ-ಗು-ವುವು
ದಿನ-ಗ-ಳಾ-ದ-ಮೇಲೆ
ದಿನ-ಗ-ಳಾ-ದವು
ದಿನ-ಗಳಿಂದ
ದಿನ-ಗ-ಳಿಟ್ಟು
ದಿನ-ಗಳು
ದಿನದ
ದಿನ-ದಲ್ಲಿ
ದಿನ-ದಿಂದ
ದಿನ-ದಿ-ನಕ್ಕೆ
ದಿನ-ದಿ-ನವೂ
ದಿನದ್ದು
ದಿನ-ನಿ-ತ್ಯವೂ
ದಿನ-ಬೆ-ಳ-ಗಾ-ದರೆ
ದಿನವೂ
ದಿನ-ವೆಲ್ಲಾ
ದಿನಸಿ
ದಿರಲು
ದಿಲ್ಲ
ದಿಲ್ಲ-ಇ-ವನು
ದಿಲ್ಲವೆ
ದಿಲ್ಲ-ವೆಂದು
ದಿಲ್ಲವೊ
ದಿಲ್ಲವೋ
ದಿವಸ
ದಿವ-ಸದ
ದಿವಾ-ನ-ಖಾ-ನೆಗೆ
ದಿವಾ-ಳಿ-ಗ-ಳಾಗಿ
ದಿವಾ-ಳಿ-ಯಾ-ದಂತೆ
ದಿವಿ
ದಿವಿ-ದೇ-ವ-ಭೋ-ಗಾನ್
ದಿವ್ಯ
ದಿವ್ಯಂ
ದಿವ್ಯ-ಗಂ-ಧಾ-ನು-ಲೇ-ಪ-ನಮ್
ದಿವ್ಯ-ಗು-ಣ-ಗಳು
ದಿವ್ಯ-ಚ-ಕ್ಷು-ಸ್ಸನ್ನು
ದಿವ್ಯ-ಜ್ಯೋತಿ
ದಿವ್ಯ-ನಾದ
ದಿವ್ಯ-ನೆಂದೂ
ದಿವ್ಯ-ಭಾ-ವ-ದಿಂದ
ದಿವ್ಯ-ಮಾ-ದಿ-ದೇ-ವ-ಮಜಂ
ದಿವ್ಯ-ಮಾ-ಲ್ಯಾಂ-ಬ-ರ-ಧರಂ
ದಿವ್ಯ-ಮೇವಂ
ದಿವ್ಯಮ್
ದಿವ್ಯ-ವಾ-ಗಿವೆ
ದಿವ್ಯ-ವಾ-ಣಿ-ಯನ್ನು
ದಿವ್ಯ-ವಾದ
ದಿವ್ಯ-ವಾ-ದುದು
ದಿವ್ಯ-ವಾ-ದುವು
ದಿವ್ಯ-ಶಕ್ತಿ
ದಿವ್ಯ-ಸ್ತ-ವ-ಗಳಿಂದ
ದಿವ್ಯಾ
ದಿವ್ಯಾ-ನಾಂ
ದಿವ್ಯಾನಿ
ದಿವ್ಯಾ-ನೇ-ಕೋ-ದ್ಯ-ತಾ-ಯು-ಧಮ್
ದಿವ್ಯಾನ್
ದಿವ್ಯಾ-ಭ-ರ-ಣ-ಗಳಿಂದ
ದಿವ್ಯಾ-ಯು-ಧ-ಗಳಿಂದ
ದಿವ್ಯೈಽಃ
ದಿವ್ಯೌ
ದಿಶಶ್ಚ
ದಿಶ-ಶ್ಚಾ-ನ-ವ-ಲೋ-ಕ-ಯನ್
ದಿಶೋ
ದಿಸೆ-ದಿ-ಸೆಗೂ
ದಿಸೆ-ದಿ-ಸೆಗೆ
ದೀನ-ನಂತೆ
ದೀನ-ರಲ್ಲಿ
ದೀನ-ರಾ-ಗ-ಬ-ಹುದು
ದೀಪ
ದೀಪಕ್ಕೆ
ದೀಪ-ಗ-ಳಿ-ವರು
ದೀಪದ
ದೀಪ-ದಂ-ತಿ-ದ್ದಾರೆ
ದೀಪ-ದ-ಕಡ್ಡಿ
ದೀಪ-ದಲ್ಲಿ
ದೀಪ-ದಿಂದ
ದೀಪ-ದೊಂ-ದಿಗೆ
ದೀಪ-ವನ್ನು
ದೀಪ-ವಿಲ್ಲ
ದೀಪಾ-ವಳಿ
ದೀಪೋ
ದೀಪ್ತಂ-ಹು-ತಾ-ಶ-ವಕ್ತ್ರಂ
ದೀಪ್ತ-ಮ-ನೇ-ಕ-ವರ್ಣಂ
ದೀಪ್ತ-ವಿ-ಶಾ-ಲ-ನೇ-ತ್ರಮ್
ದೀಪ್ತಿ-ಮಂ-ತಮ್
ದೀಯತೇ
ದೀಯ-ತೇ-ಽನು-ಪ-ಕಾ-ರಿಣೇ
ದೀರ್ಘ
ದೀರ್ಘ-ಕಾಲ
ದೀರ್ಘ-ಕಾ-ಲದ
ದೀರ್ಘ-ಕಾ-ಲ-ವಾ-ದು-ದ-ರಿಂದ
ದೀರ್ಘ-ಧ್ಯಾ-ನ-ಪ-ರ-ನಾ-ದಂತೆ
ದೀರ್ಘ-ವಾಗಿ
ದೀರ್ಘ-ವಾದ
ದೀರ್ಘ-ಸೂತ್ರಿ
ದೀರ್ಘ-ಸೂತ್ರೀ
ದೀರ್ಘಾ-ಭ್ಯಾ-ಸದ
ದೀರ್ಘಾ-ಲೋ-ಚನೆ
ದೀವಿಗೆ
ದೀವಿ-ಗೆ-ಗಳನ್ನು
ದೀವಿ-ಗೆ-ಯಂತೆ
ದೀವಿ-ಗೆ-ಯಿಂದ
ದುಂಡಾ-ವ-ರ್ತಿ-ಯಿಂ-ದಲೇ
ದುಂಬಿ-ಗಳೇ
ದುಃಖ
ದುಃಖಂ
ದುಃಖ-ಕ-ರ-ವಾ-ಗಿ-ರು-ವುದು
ದುಃಖ-ಕ-ರ-ವಾದ
ದುಃಖ-ಕ-ಷ್ಟ-ಗಳನ್ನೆಲ್ಲಾ
ದುಃಖಕ್ಕೆ
ದುಃಖ-ಕ್ಕೆಲ್ಲ
ದುಃಖ-ಗಳ
ದುಃಖ-ಗಳನ್ನು
ದುಃಖ-ಗಳಲ್ಲಿ
ದುಃಖ-ಗ-ಳ-ಲ್ಲಿಯೂ
ದುಃಖ-ಗಳು
ದುಃಖ-ಗಳೂ
ದುಃಖ-ಗ-ಳೆಂಬ
ದುಃಖ-ಗ-ಳೆ-ನ್ನೆಲ್ಲಾ
ದುಃಖ-ತರಂ
ದುಃಖ-ತಾ-ಡಿತ
ದುಃಖದ
ದುಃಖ-ದಲ್ಲಿ
ದುಃಖ-ದ-ಲ್ಲಿ-ರು-ವ-ವ-ರನ್ನು
ದುಃಖ-ದಾ-ಯಕ
ದುಃಖ-ದಿಂದ
ದುಃಖ-ದಿಂ-ದಲೂ
ದುಃಖ-ನಾ-ಶ-ಕ-ವಾ-ಗು-ವುದು
ದುಃಖ-ಪ-ಟ್ಟ-ವರು
ದುಃಖ-ಪ-ಟ್ಟಿ-ರು-ವನು
ದುಃಖ-ಪ-ಡ-ಬೇ-ಕಾ-ಗಿದೆ
ದುಃಖ-ಪ-ಡ-ಬೇಕು
ದುಃಖ-ಪ-ಡು-ತ್ತಾನೆ
ದುಃಖ-ಪ-ಡು-ವನು
ದುಃಖ-ಪ-ರ-ವ-ಶ-ನಾಗಿ
ದುಃಖ-ಪ್ರಾ-ಪ್ತ-ವಾ-ಗು-ವುದು
ದುಃಖ-ಮ-ಜ್ಞಾನಂ
ದುಃಖ-ಮಾಪು
ದುಃಖ-ಮಿ-ತ್ಯೇವ
ದುಃಖ-ಮಿಶ್ರ
ದುಃಖ-ಮಿ-ಶ್ರ-ವಿಲ್ಲ
ದುಃಖ-ಯೋ-ನಯ
ದುಃಖ-ವನ್ನು
ದುಃಖ-ವನ್ನೂ
ದುಃಖ-ವನ್ನೇ
ದುಃಖ-ವಾ-ಗು-ವುದು
ದುಃಖ-ವಾ-ದರೂ
ದುಃಖ-ವಿ-ದೆಯೋ
ದುಃಖ-ವಿ-ರು-ವುದು
ದುಃಖ-ವಿ-ಲ್ಲದ
ದುಃಖ-ವಿ-ಲ್ಲದೆ
ದುಃಖವೂ
ದುಃಖ-ವೆಂ-ತಲೂ
ದುಃಖ-ವೆಂಬ
ದುಃಖ-ವೆಂ-ಬುದು
ದುಃಖ-ವೆಲ್ಲಿ
ದುಃಖವೇ
ದುಃಖ-ಶೋ-ಕಾ-ಮ-ಯ-ಪ್ರ-ದಾಃ
ದುಃಖ-ಸಂ-ಬಂ-ಧ-ದಿಂ-ದಲೂ
ದುಃಖ-ಸ-ಮು-ದ್ರದ
ದುಃಖಹಾ
ದುಃಖಾಂತಂ
ದುಃಖಾ-ಕ್ರಾಂ-ತ-ನಾಗಿ
ದುಃಖಾ-ಲಯ
ದುಃಖಾ-ಲ-ಯ-ಮ-ಶಾ-ಶ್ವ-ತಮ್
ದುಃಖಿ-ತ-ರಾಗಿ
ದುಃಖೇನ
ದುಃಖೇ-ಷ್ವ-ನು-ದ್ವಿ-ಗ್ನ-ಮ-ನಾಃ
ದುಃಸ್ಥಿ-ತಿಗೆ
ದುಃಸ್ವ-ಪ್ನ-ವನ್ನು
ದುಗು-ಡ-ವನ್ನು
ದುಗ್ಧಂ
ದುಡಿ-ತದ
ದುಡಿ-ತ-ದಿಂದ
ದುಡಿ-ತ-ದಿಂ-ದಲೇ
ದುಡಿ-ದಿ-ದ್ದರೆ
ದುಡಿ-ದಿ-ರು-ವರು
ದುಡಿ-ದಿ-ರು-ವರೋ
ದುಡಿದು
ದುಡಿಮೆ
ದುಡಿ-ಮೆ-ಯಿಂದ
ದುಡಿಯ
ದುಡಿ-ಯ-ದ-ವನ
ದುಡಿ-ಯ-ಬೇಕು
ದುಡಿ-ಯುತ್ತ
ದುಡಿ-ಯುತ್ತಾ
ದುಡಿ-ಯು-ತ್ತಾನೆ
ದುಡಿ-ಯು-ತ್ತಿದೆ
ದುಡಿ-ಯು-ತ್ತಿ-ರು-ವೆವು
ದುಡಿ-ಯು-ತ್ತಿವೆ
ದುಡಿ-ಯು-ವನು
ದುಡಿ-ಯು-ವು-ದಿಲ್ಲ
ದುಡಿ-ಯು-ವುದು
ದುಡ್ಡನ್ನು
ದುಡ್ಡ-ನ್ನೆಲ್ಲ
ದುಡ್ಡನ್ನೇ
ದುಡ್ಡಿಗೆ
ದುಡ್ಡಿ-ದ್ದ-ವ-ನಿಗೆ
ದುಡ್ಡಿ-ಲ್ಲ-ದಿ-ದ್ದರೆ
ದುಡ್ಡು
ದುಡ್ಡೂ
ದುಡ್ಡೆಲ್ಲ
ದುದನ್ನು
ದುದ-ನ್ನೆಲ್ಲ
ದುದು
ದುಯೋ-ಧ-ನ-ಸ್ತದಾ
ದುರಂ-ತ-ದಲ್ಲಿ
ದುರ-ತ್ಯಯಾ
ದುರ-ದೃ-ಷ್ಟ-ವನ್ನೋ
ದುರ-ಪ-ಯೋ-ಗ-ಪ-ಡಿ-ಸಿ-ಕೊಂ-ಡಂತೆ
ದುರ-ಭ್ಯಾ-ಸ-ಗಳನ್ನು
ದುರ-ಭ್ಯಾ-ಸ-ಗ-ಳಿಗೆ
ದುರ-ಭ್ಯಾ-ಸ-ಗಳು
ದುರ-ಭ್ಯಾ-ಸದ
ದುರ-ಭ್ಯಾ-ಸ-ದಷ್ಟು
ದುರ-ವ-ಸ್ಥೆಗೆ
ದುರ-ಹಂ-ಕಾರ
ದುರ-ಹಂ-ಕಾ-ರ-ದಲ್ಲಿ
ದುರ-ಹಂ-ಕಾ-ರ-ದಿಂ-ದಲೂ
ದುರ-ಹಂ-ಕಾ-ರಿ-ಗ-ಳಾ-ಗ-ಬ-ಹುದು
ದುರ-ಹಂ-ಕಾ-ರಿ-ಗಳು
ದುರ-ಹಂ-ಕಾ-ರಿಯ
ದುರಾ-ಚಾರ
ದುರಾ-ಚಾರಿ
ದುರಾ-ಚಾ-ರಿ-ಗಳು
ದುರಾ-ಚಾ-ರಿಗೆ
ದುರಾ-ಚಾ-ರಿ-ಯಾ-ಗಿ-ದ್ದರೂ
ದುರಾ-ಚಾ-ರಿ-ಯಾ-ಗಿ-ರ-ಬ-ಹುದು
ದುರಾ-ಚಾ-ರಿಯೇ
ದುರಾತ್ಮ
ದುರಾ-ಸ-ದಮ್
ದುರಿ-ತ-ಗಳಲ್ಲಿ
ದುರಿ-ತ-ಗ-ಳಿ-ಗೆಲ್ಲಾ
ದುರಿ-ತ-ಗಳು
ದುರು-ದ್ದೇ-ಶ-ಗಳನ್ನು
ದುರು-ದ್ದೇ-ಶ-ಗಳೂ
ದುರು-ದ್ದೇ-ಶ-ವಿದೆ
ದುರು-ಪ-ಯೋಗ
ದುರು-ಪ-ಯೋ-ಗ-ಪ-ಡಿ-ಸಿ-ಕೊಂ-ಡರೆ
ದುರು-ಪ-ಯೋ-ಗ-ಪ-ಡಿ-ಸಿ-ಕೊಳ್ಳು
ದುರ್ಗಂಧ
ದುರ್ಗಂ-ಧ-ಗಳನ್ನು
ದುರ್ಗಂ-ಧ-ಮಯ
ದುರ್ಗಂ-ಧ-ವನ್ನು
ದುರ್ಗಂ-ಧ-ವನ್ನೂ
ದುರ್ಗಂ-ಧ-ವಾ-ಗಿ-ರು-ವುದು
ದುರ್ಗತಿ
ದುರ್ಗ-ತಿಗೆ
ದುರ್ಗ-ತಿ-ಯನ್ನು
ದುರ್ಗ-ಮ-ವಾದ
ದುರ್ಗು-ಣ-ಗಳ
ದುರ್ಗು-ಣ-ಗಳು
ದುರ್ಗು-ಣ-ಗ-ಳೆಲ್ಲ
ದುರ್ಜನ
ದುರ್ಜ-ನ-ನಾ-ಗಿಯೇ
ದುರ್ಜ-ನ-ರನ್ನು
ದುರ್ಜಯ
ದುರ್ಜ-ಯವೂ
ದುರ್ನಿ-ಗ್ರಹಂ
ದುರ್ನಿ-ರೀಕ್ಷ್ಯಂ
ದುರ್ಬಲ
ದುರ್ಬ-ಲ-ನನ್ನು
ದುರ್ಬ-ಲ-ನಲ್ಲ
ದುರ್ಬ-ಲ-ನಾಗಿ
ದುರ್ಬ-ಲ-ನಾ-ಗಿ-ರು-ವಾಗ
ದುರ್ಬ-ಲ-ರಾ-ಗು-ತ್ತಾರೆ
ದುರ್ಬ-ಲ-ರಾದ
ದುರ್ಬ-ಲ-ರಾ-ದ-ವ-ರನ್ನು
ದುರ್ಬ-ಲರು
ದುರ್ಬ-ಲ-ವಾಗಿ
ದುರ್ಬ-ಲ-ವಾ-ಗಿದೆ
ದುರ್ಬ-ಲ-ವಾ-ಗಿ-ದ್ದರೆ
ದುರ್ಬ-ಲ-ವಾ-ಗಿ-ರು-ವುದು
ದುರ್ಬ-ಲ-ವಾಗು
ದುರ್ಬ-ಲ-ವಾ-ಗು-ವು-ದಿಲ್ಲ
ದುರ್ಬ-ಲ-ವಾ-ಗು-ವುದು
ದುರ್ಬ-ಲ-ವಾ-ಗು-ವುದೋ
ದುರ್ಬ-ಲ-ವಾ-ಗು-ವುವು
ದುರ್ಬ-ಲ-ವಾದ
ದುರ್ಬ-ಲ-ವಾ-ದರೆ
ದುರ್ಬ-ಲ-ವಾ-ದಾಗ
ದುರ್ಬ-ಲ-ವಾ-ದುದು
ದುರ್ಬುದ್ಧಿ
ದುರ್ಬು-ದ್ಧಿ-ಯನ್ನು
ದುರ್ಬು-ದ್ಧಿ-ಯಾದ
ದುರ್ಬು-ದ್ಧಿ-ಯು-ಳ್ಳ-ವನು
ದುರ್ಬು-ದ್ಧೇ-ರ್ಯುದ್ಧೇ
ದುರ್ಮ-ತಿಃ
ದುರ್ಮಾ-ರ್ಗಿ-ಗ-ಳಾ-ಗು-ವರು
ದುರ್ಮೇಧಾ
ದುರ್ಯೋ-ಧನ
ದುರ್ಯೋ-ಧ-ನನ
ದುರ್ಯೋ-ಧ-ನ-ನಲ್ಲಿ
ದುರ್ಯೋ-ಧ-ನ-ನಾ-ದರೋ
ದುರ್ಯೋ-ಧ-ನ-ನಿಗೆ
ದುರ್ಯೋ-ಧ-ನನು
ದುರ್ಯೋ-ಧ-ನನೇ
ದುರ್ಯೋ-ಧ-ನಾ-ದಿ-ಗಳು
ದುರ್ಯೋ-ಧ-ನಾ-ದಿ-ಗ-ಳೆಲ್ಲಾ
ದುರ್ಯೋ-ಧ-ನಾ-ವ-ರ್ತಿನೀ
ದುರ್ಲಭ
ದುರ್ಲ-ಭ-ತರಂ
ದುರ್ಲ-ಭ-ತ-ರ-ವಾ-ದದ್ದು
ದುರ್ಲ-ಭ-ವಾದ
ದುರ್ಲ-ಭ-ವಾ-ದದ್ದು
ದುರ್ಲ-ಭ-ವಾ-ದುದು
ದುರ್ವಾ-ಸನೆ
ದುರ್ವಾ-ಸ-ನೆಯ
ದುರ್ವಾ-ಸ-ನೆ-ಯನ್ನು
ದುರ್ವ್ಯ-ಸ-ನ-ಗಳನ್ನೆಲ್ಲ
ದುರ್ವ್ಯ-ಸ-ನದ
ದುವು-ಗ-ಳೆಲ್ಲ
ದುಷ್ಕ-ರ್ಮ-ಗ-ಳಿಗೆ
ದುಷ್ಕೃ-ತಾಮ್
ದುಷ್ಕೃ-ತಿನೋ
ದುಷ್ಟ
ದುಷ್ಟ-ಪ್ರಾ-ಣಿ-ಗಳ
ದುಷ್ಟ-ಮೃ-ಗ-ಗಳನ್ನು
ದುಷ್ಟ-ಮೃ-ಗ-ಗ-ಳಿವೆ
ದುಷ್ಟ-ಮೃ-ಗ-ದೊ-ಡನೆ
ದುಷ್ಟರ
ದುಷ್ಟ-ರನ್ನು
ದುಷ್ಟರು
ದುಷ್ಟಾಸು
ದುಷ್ಪೂರಂ
ದುಷ್ಪೂ-ರೇ-ಣಾ-ನ-ಲೇನ
ದುಷ್ಪ್ರಾಪ
ದುಸ್ತರ
ದುಸ್ತ-ರ-ವಾಗಿ
ದುಸ್ತ-ರ-ವಾ-ಗಿ-ರು-ತ್ತದೆ
ದುಸ್ಸಾಧ್ಯ
ದೂಡ-ಬೇ-ಕಾ-ದರೆ
ದೂಡು-ತ್ತಾನೆ
ದೂಡು-ವುದು
ದೂಡು-ವೆವೊ
ದೂತ
ದೂತ-ನನ್ನು
ದೂತರು
ದೂರ
ದೂರಕ್ಕೆ
ದೂರದ
ದೂರ-ದರ್ಶಿ
ದೂರ-ದಲ್ಲಿ
ದೂರ-ದ-ಲ್ಲಿ-ಟ್ಟರೂ
ದೂರ-ದ-ಲ್ಲಿದೆ
ದೂರ-ದ-ಲ್ಲಿ-ದ್ದರೆ
ದೂರ-ದ-ಲ್ಲಿ-ದ್ದಾನೆ
ದೂರ-ದ-ಲ್ಲಿ-ರ-ಬ-ಹುದು
ದೂರ-ದ-ಲ್ಲಿ-ರುವ
ದೂರ-ದ-ಲ್ಲಿ-ರು-ವ-ವ-ರಿಗೆ
ದೂರ-ದ-ಲ್ಲಿ-ರು-ವಾಗ
ದೂರ-ದ-ಲ್ಲಿ-ರು-ವುದನ್ನು
ದೂರ-ದ-ಲ್ಲಿ-ರು-ವುದು
ದೂರ-ದ-ಲ್ಲಿವೆ
ದೂರ-ದ-ಲ್ಲಿ-ವೆ-ಯೆಂ-ದರೆ
ದೂರ-ದಿಂದ
ದೂರ-ದೂ-ರಿಗೆ
ದೂರ-ದೃಷ್ಟಿ
ದೂರದೆ
ದೂರ-ಬ-ಹುದು
ದೂರ-ಬೇ-ಕಾ-ಗಿಲ್ಲ
ದೂರ-ಮಾ-ಡುವ
ದೂರ-ಮಾ-ಡು-ವುದು
ದೂರ-ವಾ-ಗಿ-ದ್ದರೂ
ದೂರ-ವಾ-ಗಿ-ರ-ಬೇಕು
ದೂರ-ವಾ-ಗುತ್ತ
ದೂರ-ವಾ-ಯಿತು
ದೂರ-ವಿರ
ದೂರ-ವಿ-ರ-ಬೇಕು
ದೂರ-ವಿ-ರು-ವ-ವ-ರಿಗೆ
ದೂರ-ವಿ-ರು-ವಾಗ
ದೂರ-ವಿ-ರು-ವು-ದಕ್ಕೆ
ದೂರ-ವು-ದ-ರಿಂದ
ದೂರಸ್ಥಂ
ದೂರಿ
ದೂರಿ-ದರೆ
ದೂರು-ಗಳನ್ನು
ದೂರು-ತ್ತಾನೆ
ದೂರು-ತ್ತಿ-ರ-ಲಿಲ್ಲ
ದೂರು-ತ್ತಿ-ರು-ವು-ದಿಲ್ಲ
ದೂರು-ವು-ದಕ್ಕೂ
ದೂರು-ವು-ದಕ್ಕೆ
ದೂರು-ವುದನ್ನು
ದೂರು-ವು-ದಿಲ್ಲ
ದೂರೇಣ
ದೂರ್ವಾ-ಸರು
ದೃಗ್
ದೃಗ್ಮೇಲೆ
ದೃಢ
ದೃಢ-ಕಾ-ಯ-ನಾ-ಗಿ-ರ-ಬೇಕು
ದೃಢ-ಚಿ-ತ್ತನ
ದೃಢತೆ
ದೃಢ-ತೆಗೂ
ದೃಢ-ತೆ-ಯನ್ನು
ದೃಢ-ತೆ-ಯಿಂದ
ದೃಢ-ನಿ-ಶ್ಚಯಃ
ದೃಢ-ನಿ-ಶ್ಚ-ಯನು
ದೃಢ-ನಿ-ಶ್ಚ-ಯನೋ
ದೃಢ-ಮಾ-ಡು-ವನು
ದೃಢ-ಮಿತಿ
ದೃಢ-ವಾ-ಗ-ಬೇಕು
ದೃಢ-ವಾಗಿ
ದೃಢ-ವಾ-ಗಿದೆ
ದೃಢ-ವಾ-ಗಿಲ್ಲ
ದೃಢ-ವಾದ
ದೃಢ-ವ್ರತ
ದೃಢ-ವ್ರ-ತ-ರಾಗಿ
ದೃಢ-ವ್ರ-ತ-ರಾದ
ದೃಢ-ವ್ರ-ತರು
ದೃಢ-ವ್ರ-ತಾಃ
ದೃಢ-ವ್ರತಿ
ದೃಢೇನ
ದೃಶ್ಯ
ದೃಶ್ಯ-ಗಳನ್ನು
ದೃಶ್ಯ-ಗಳು
ದೃಶ್ಯ-ಗು-ಣ-ಗಳನ್ನೆಲ್ಲ
ದೃಶ್ಯ-ದಂತೆ
ದೃಶ್ಯ-ದಲ್ಲಿ
ದೃಶ್ಯ-ದ-ಲ್ಲಿ-ಮೂರು
ದೃಶ್ಯ-ಪ್ರ-ಪಂಚ
ದೃಶ್ಯ-ಪ್ರ-ಪಂ-ಚದ
ದೃಶ್ಯ-ಪ್ರ-ಪಂ-ಚ-ವನ್ನು
ದೃಶ್ಯ-ಪ್ರ-ಪಂ-ಚವೇ
ದೃಶ್ಯ-ವಂತೂ
ದೃಶ್ಯ-ವನ್ನು
ದೃಶ್ಯ-ವ-ನ್ನೆಲ್ಲ
ದೃಶ್ಯ-ವನ್ನೇ
ದೃಶ್ಯ-ವನ್ನೊ
ದೃಶ್ಯ-ವಲ್ಲ
ದೃಶ್ಯ-ವಸ್ತು
ದೃಶ್ಯ-ವ-ಸ್ತು-ಗಳ
ದೃಶ್ಯ-ವ-ಸ್ತು-ಗ-ಳಂತೆ
ದೃಶ್ಯ-ವ-ಸ್ತು-ಗಳನ್ನು
ದೃಶ್ಯ-ವ-ಸ್ತು-ವನ್ನು
ದೃಶ್ಯ-ವ-ಸ್ತು-ವಾ-ದರೋ
ದೃಶ್ಯ-ವ-ಸ್ತು-ವಿನ
ದೃಶ್ಯ-ವ-ಸ್ತು-ವಿ-ನಂತೆ
ದೃಶ್ಯ-ವಾ-ದ-ಮೇಲೆ
ದೃಶ್ಯ-ವಿ-ರು-ವುದು
ದೃಶ್ಯ-ವೆಲ್ಲಾ
ದೃಶ್ಯವೇ
ದೃಷ್ಚಿ-ಯಿಂದ
ದೃಷ್ಚಿ-ಯೆಲ್ಲ
ದೃಷ್ಟ-ಕೇತು
ದೃಷ್ಟ-ಕೇ-ತು-ಶ್ಚೇ-ಕಿ-ತಾನಃ
ದೃಷ್ಟ-ದ್ಯುಮ್ನ
ದೃಷ್ಟ-ದ್ಯು-ಮ್ನ-ನಿಂದ
ದೃಷ್ಟ-ಪೂರ್ವಂ
ದೃಷ್ಟ-ವಾ-ನಸಿ
ದೃಷ್ಟಾಂ-ತ-ವನ್ನು
ದೃಷ್ಟಿ
ದೃಷ್ಟಿ-ಕೋಣ
ದೃಷ್ಟಿ-ಕೋ-ಣ-ಗಳಿಂದ
ದೃಷ್ಟಿ-ಕೋ-ಣ-ಗ-ಳಿವೆ
ದೃಷ್ಟಿ-ಕೋ-ಣದ
ದೃಷ್ಟಿ-ಕೋ-ಣ-ವನ್ನು
ದೃಷ್ಟಿ-ಕೋನ
ದೃಷ್ಟಿ-ಕೋ-ನಕ್ಕೆ
ದೃಷ್ಟಿ-ಕೋ-ನ-ಗಳಲ್ಲಿ
ದೃಷ್ಟಿ-ಕೋ-ನ-ಗಳಿಂದ
ದೃಷ್ಟಿ-ಕೋ-ನ-ಗ-ಳಿಂ-ದಲೂ
ದೃಷ್ಟಿ-ಕೋ-ನದ
ದೃಷ್ಟಿ-ಕೋ-ನ-ದಿಂದ
ದೃಷ್ಟಿ-ಗಳನ್ನೂ
ದೃಷ್ಟಿ-ಗ-ಳಿವೆ
ದೃಷ್ಟಿ-ಗಳು
ದೃಷ್ಟಿ-ಗಿಂತ
ದೃಷ್ಟಿಗೆ
ದೃಷ್ಟಿ-ಭೇ-ದಕ್ಕೆ
ದೃಷ್ಟಿ-ಮ-ವ-ಷ್ಟಭ್ಯ
ದೃಷ್ಟಿಯ
ದೃಷ್ಟಿ-ಯನ್ನು
ದೃಷ್ಟಿ-ಯನ್ನೂ
ದೃಷ್ಟಿ-ಯ-ಲ್ಲಾ-ದರೊ
ದೃಷ್ಟಿ-ಯಲ್ಲಿ
ದೃಷ್ಟಿ-ಯ-ಲ್ಲಿದೆ
ದೃಷ್ಟಿ-ಯ-ಲ್ಲಿಯೇ
ದೃಷ್ಟಿ-ಯಾ-ಗಲೀ
ದೃಷ್ಟಿ-ಯಿಂದ
ದೃಷ್ಟಿ-ಯಿಂ-ದ-ಲಾ-ದರೂ
ದೃಷ್ಟಿ-ಯಿಂ-ದಲೂ
ದೃಷ್ಟಿ-ಯಿಂ-ದಲೇ
ದೃಷ್ಟಿ-ಯಿಂ-ದಲ್ಲ
ದೃಷ್ಟಿಯೂ
ದೃಷ್ಟಿ-ಯೆಲ್ಲ
ದೃಷ್ಟಿಯೇ
ದೃಷ್ಟೋಂ-ತ-ಸ್ತ್ವ-ನ-ಯೋ-ಸ್ತ-ತ್ತ್ವ-ದ-ರ್ಶಿ-ಭಿಃ
ದೃಷ್ಟ್ವಾ
ದೃಷ್ಟ್ವಾ-ದ್ಭುತಂ
ದೃಷ್ಟ್ವೇದಂ
ದೃಷ್ಟ್ವೇಮಂ
ದೃಷ್ವೈವ
ದೆಯೋ
ದೆಲ್ಲ
ದೆವ್ವ
ದೆವ್ವ-ಗ-ಳಂತೆ
ದೆಸೆ-ದೆ-ಸೆಗೂ
ದೆಸೆ-ದೆ-ಸೆಗೆ
ದೆಸೆ-ಯಿಂದ
ದೇನು
ದೇವ
ದೇವಂ
ದೇವಕಿ
ದೇವ-ಕಿಗೆ
ದೇವ-ಕಿ-ಯ-ರಿಗೆ
ದೇವಕೀ
ದೇವ-ಕೀ-ಪ-ರ-ಮಾ-ನಂದಂ
ದೇವ-ತಾಃ
ದೇವತೆ
ದೇವ-ತೆ-ಗಳ
ದೇವ-ತೆ-ಗಳನ್ನು
ದೇವ-ತೆ-ಗಳನ್ನೂ
ದೇವ-ತೆ-ಗಳನ್ನೆಲ್ಲ
ದೇವ-ತೆ-ಗ-ಳ-ಲ್ಲಾ-ಗಲಿ
ದೇವ-ತೆ-ಗಳಲ್ಲಿ
ದೇವ-ತೆ-ಗ-ಳಾ-ಗಲಿ
ದೇವ-ತೆ-ಗಳಿಂದ
ದೇವ-ತೆ-ಗ-ಳಿಗೂ
ದೇವ-ತೆ-ಗ-ಳಿಗೆ
ದೇವ-ತೆ-ಗ-ಳಿ-ಗೆಲ್ಲ
ದೇವ-ತೆ-ಗ-ಳಿಲ್ಲ
ದೇವ-ತೆ-ಗಳು
ದೇವ-ತೆ-ಗಳೂ
ದೇವ-ತೆ-ಗ-ಳೆಂಬ
ದೇವ-ತೆ-ಗ-ಳೆಲ್ಲ
ದೇವ-ತೆ-ಗ-ಳೆಲ್ಲಾ
ದೇವ-ತೆ-ಗ-ಳೆ-ಲ್ಲೆಲ್ಲ
ದೇವ-ತೆಯ
ದೇವ-ತ್ವಕ್ಕೆ
ದೇವ-ದತ್ತಂ
ದೇವ-ದ-ತ್ತ-ವನ್ನು
ದೇವ-ದಾ-ನ-ವ-ರಲ್ಲಿ
ದೇವ-ದಾ-ನ-ವರು
ದೇವ-ದೂ-ತ-ರಂ-ತಿವೆ
ದೇವ-ದೇವ
ದೇವ-ದೇ-ವತೆ
ದೇವ-ದೇ-ವ-ತೆ-ಗಳ
ದೇವ-ದೇ-ವ-ತೆ-ಗಳು
ದೇವ-ದೇ-ವ-ತೆ-ಗ-ಳೆಲ್ಲ
ದೇವ-ದೇ-ವನ
ದೇವ-ದೇ-ವ-ನಾದ
ದೇವ-ದೇ-ವಿ-ಯರ
ದೇವ-ದ್ವಿ-ಜ-ಗು-ರು-ಪ್ರಾ-ಜ್ಞ-ಪೂ-ಜನಂ
ದೇವನ
ದೇವ-ನಂತೆ
ದೇವ-ನನ್ನು
ದೇವ-ನಿ-ಗಾಗಿ
ದೇವ-ನಿಗೆ
ದೇವನೂ
ದೇವ-ನೆಂದೆ
ದೇವ-ನೆ-ಡೆಗೆ
ದೇವನೇ
ದೇವ-ನೊಬ್ಬ
ದೇವ-ಪು-ಷಿ-ಗಳಲ್ಲಿ
ದೇವ-ಮ-ನಂತಂ
ದೇವ-ಯಜೋ
ದೇವ-ಯಜ್ಞ
ದೇವರ
ದೇವ-ರಂತೆ
ದೇವ-ರ-ಕ-ಡೆಗೆ
ದೇವ-ರ-ಡಿಗೆ
ದೇವ-ರ-ತನ
ದೇವ-ರದು
ದೇವ-ರ-ನ್ನಾ-ದರೂ
ದೇವ-ರನ್ನು
ದೇವ-ರನ್ನೂ
ದೇವ-ರನ್ನೆ
ದೇವ-ರನ್ನೇ
ದೇವ-ರ-ಮ-ನೆಯ
ದೇವ-ರ-ಮ-ನೆ-ಯಲ್ಲಿ
ದೇವ-ರ-ಮೇಲೆ
ದೇವ-ರಲ್ಲ
ದೇವ-ರ-ಲ್ಲದ
ದೇವ-ರ-ಲ್ಲದೆ
ದೇವ-ರಲ್ಲಿ
ದೇವ-ರ-ಲ್ಲಿ-ಡ-ಬೇಕು
ದೇವ-ರ-ಲ್ಲಿಯೇ
ದೇವ-ರ-ಲ್ಲಿ-ರು-ವನು
ದೇವ-ರಲ್ಲೇ
ದೇವ-ರಷ್ಟು
ದೇವ-ರಾ-ಗಲಿ
ದೇವ-ರಾ-ಗು-ವನು
ದೇವ-ರಾ-ದರೊ
ದೇವ-ರಾ-ದರೋ
ದೇವ-ರಿಂದ
ದೇವ-ರಿಂ-ದಲೇ
ದೇವ-ರಿ-ಗಲ್ಲ
ದೇವ-ರಿ-ಗಾಗಿ
ದೇವ-ರಿಗೂ
ದೇವ-ರಿಗೆ
ದೇವ-ರಿ-ಗೇನು
ದೇವ-ರಿ-ಗೊಂದು
ದೇವ-ರಿ-ದ್ದಾ-ನೆಯೆ
ದೇವ-ರಿ-ರ-ಬೇಕು
ದೇವ-ರಿ-ರು-ವನು
ದೇವ-ರಿ-ರು-ವನೊ
ದೇವ-ರಿ-ಲ್ಲದ
ದೇವ-ರಿ-ಲ್ಲದೆ
ದೇವರು
ದೇವ-ರು-ಗಳ
ದೇವ-ರು-ಗಳನ್ನು
ದೇವ-ರು-ಗಳಲ್ಲಿ
ದೇವ-ರು-ಗ-ಳಿಗೆ
ದೇವ-ರು-ಗ-ಳಿ-ಗೆಲ್ಲ
ದೇವ-ರು-ಗಳು
ದೇವ-ರು-ಗ-ಳೆಲ್ಲ
ದೇವ-ರು-ಗಳೇ
ದೇವರೂ
ದೇವರೆ
ದೇವ-ರೆಂಬ
ದೇವ-ರೆಂ-ಬುದು
ದೇವರೆ-ಡೆಗೆ
ದೇವರೆ-ಲ್ಲ-ವನ್ನೂ
ದೇವರೆಲ್ಲೊ
ದೇವರೇ
ದೇವ-ರೇನೋ
ದೇವರೊ
ದೇವ-ರೊಂ-ದಿಗೆ
ದೇವ-ರೊ-ಡನೆ
ದೇವ-ರೊ-ಬ್ಬ-ನಿಗೆ
ದೇವ-ರೊ-ಬ್ಬ-ನಿಗೇ
ದೇವ-ರೊ-ಬ್ಬನೆ
ದೇವ-ರೊ-ಬ್ಬನೇ
ದೇವ-ರ್ಷಿ-ಯಾದ
ದೇವ-ರ್ಷಿ-ರ್ನಾ-ರ-ದ-ಸ್ತಥಾ
ದೇವ-ರ್ಷೀ-ಣಾಂ
ದೇವಲ
ದೇವಲೋ
ದೇವ-ಲೋಕ
ದೇವ-ಲೋ-ಕ-ದಲ್ಲಿ
ದೇವ-ಲೋ-ಕ-ದ-ಲ್ಲಿ-ರ-ಬ-ಹುದು
ದೇವ-ಲೋ-ಕ-ವನ್ನು
ದೇವ-ಲೋ-ಕ-ವೆಂದರೆ
ದೇವ-ವರ
ದೇವ-ವ್ರತಾ
ದೇವ-ಸ್ಥಾ-ನಕ್ಕೆ
ದೇವ-ಸ್ಥಾ-ನದ
ದೇವ-ಸ್ಥಾ-ನ-ದಲ್ಲಿ
ದೇವ-ಸ್ಥಾ-ನ-ದ-ಲ್ಲಿ-ರುವ
ದೇವಾ
ದೇವಾಂ-ಸ್ತವ
ದೇವಾ-ನಾಂ
ದೇವಾ-ನಾ-ಮಸ್ಮಿ
ದೇವಾನ್
ದೇವಾಯ
ದೇವಾ-ಲಯ
ದೇವಾ-ಲ-ಯ-ದಂತೆ
ದೇವಿ
ದೇವಿಯ
ದೇವಿ-ಯನ್ನು
ದೇವಿ-ಯರ
ದೇವೆಂದ್ರ
ದೇವೇಶ
ದೇವೇ-ಶ-ನಾದ
ದೇವೇಷು
ದೇಶ
ದೇಶ-ಕಾಲ
ದೇಶ-ಕಾ-ಲಕ್ಕೆ
ದೇಶ-ಕಾ-ಲ-ನಿ-ಮಿತ್ತ
ದೇಶ-ಕಾ-ಲ-ನಿ-ಮಿ-ತ್ತಕ್ಕೆ
ದೇಶ-ಕಾ-ಲ-ನಿ-ಮಿ-ತ್ತದ
ದೇಶ-ಕಾ-ಲ-ನಿ-ಮಿ-ತ್ತ-ದ-ಲ್ಲಿ-ರುವ
ದೇಶ-ಕಾ-ಲ-ನಿ-ಮಿ-ತ್ತ-ವನ್ನು
ದೇಶ-ಕಾ-ಲ-ಪಾ-ತ್ರ-ಗಳಲ್ಲಿ
ದೇಶ-ಕಾ-ಲಾ-ತೀ-ತ-ವಾ-ದುದು
ದೇಶ-ಕ್ಕಲ್ಲ
ದೇಶಕ್ಕೆ
ದೇಶ-ಗಳಲ್ಲಿ
ದೇಶದ
ದೇಶ-ದ-ಲ್ಲಾ-ಗಲೀ
ದೇಶ-ದಲ್ಲಿ
ದೇಶ-ದ-ಲ್ಲಿಯೂ
ದೇಶ-ದ-ಲ್ಲಿ-ರುವ
ದೇಶ-ದ-ಲ್ಲಿ-ರು-ವ-ವರೆಲ್ಲ
ದೇಶ-ದಲ್ಲೆ
ದೇಶ-ದ-ವರು
ದೇಶ-ದಿಂದ
ದೇಶ-ಭಾ-ಷೆ-ಯಲ್ಲಿ
ದೇಶ-ವನ್ನು
ದೇಶ-ವ-ನ್ನೆಲ್ಲಾ
ದೇಶ-ವಿದೆ
ದೇಶ-ವಿಲ್ಲ
ದೇಶ-ವೆಲ್ಲ
ದೇಶವೇ
ದೇಶೇ
ದೇಹ
ದೇಹಂ
ದೇಹಕ್ಕೂ
ದೇಹಕ್ಕೆ
ದೇಹ-ಗಳ
ದೇಹ-ಗಳನ್ನು
ದೇಹ-ಗ-ಳ-ಲ್ಲಿ-ರುವ
ದೇಹ-ಗ-ಳಲ್ಲೂ
ದೇಹ-ಗಳು
ದೇಹ-ಗ-ಳೆಲ್ಲಾ
ದೇಹ-ತ್ಯಾಗ
ದೇಹ-ತ್ಯಾ-ಗಕ್ಕೆ
ದೇಹದ
ದೇಹ-ದಲ್ಲಿ
ದೇಹ-ದ-ಲ್ಲಿದೆ
ದೇಹ-ದ-ಲ್ಲಿ-ದ್ದರೂ
ದೇಹ-ದ-ಲ್ಲಿ-ದ್ದರೆ
ದೇಹ-ದ-ಲ್ಲಿ-ದ್ದು-ಕೊಂಡು
ದೇಹ-ದ-ಲ್ಲಿಯೂ
ದೇಹ-ದ-ಲ್ಲಿ-ರುವ
ದೇಹ-ದ-ಲ್ಲಿ-ರು-ವಾಗ
ದೇಹ-ದ-ಲ್ಲಿ-ರು-ವು-ದ-ನ್ನೆಲ್ಲ
ದೇಹ-ದಲ್ಲೆ
ದೇಹ-ದಲ್ಲೇ
ದೇಹ-ದಷ್ಟು
ದೇಹ-ದಾರ್ಢ್ಯ
ದೇಹ-ದಾ-ರ್ಢ್ಯ-ವ-ನ್ನೆಲ್ಲ
ದೇಹ-ದಿಂದ
ದೇಹ-ದಿಂ-ದಲೇ
ದೇಹ-ದೊ-ಳಗೆ
ದೇಹ-ಧ-ರ್ಮ-ಗಳು
ದೇಹ-ಧ-ರ್ಮ-ದಿಂದ
ದೇಹ-ಧ-ರ್ಮ-ವ-ನ್ನೆಲ್ಲ
ದೇಹ-ಧಾ-ರ-ಣೆಗೆ
ದೇಹ-ಧಾರಿ
ದೇಹ-ಧಾ-ರಿ-ಗಳಲ್ಲಿ
ದೇಹ-ಧಾ-ರಿ-ಗ-ಳಿಗೆ
ದೇಹ-ಧಾ-ರಿ-ಗಳು
ದೇಹ-ಧಾ-ರಿಗೆ
ದೇಹ-ಪೋ-ಷ-ಣೆ-ಗಾಗಿ
ದೇಹ-ಪೋ-ಷ-ಣೆಗೆ
ದೇಹ-ಭಾವ
ದೇಹ-ಭೃತಾ
ದೇಹ-ಭೃ-ತಾಂ
ದೇಹ-ಭೃತ್
ದೇಹ-ಮಾ-ಶ್ರಿತಃ
ದೇಹ-ರ-ಕ್ಷ-ಣೆಗೆ
ದೇಹ-ವ-ದ್ಭಿ-ರ-ವಾ-ಪ್ಯತೇ
ದೇಹ-ವನ್ನು
ದೇಹ-ವನ್ನೂ
ದೇಹ-ವ-ನ್ನೆಲ್ಲ
ದೇಹ-ವನ್ನೇ
ದೇಹ-ವನ್ನೊ
ದೇಹ-ವಲ್ಲ
ದೇಹ-ವಾ-ದರೂ
ದೇಹ-ವಾ-ದರೊ
ದೇಹ-ವಿ-ರ-ಬೇಕು
ದೇಹವು
ದೇಹವೂ
ದೇಹ-ವೆಂದು
ದೇಹ-ವೆಂಬ
ದೇಹ-ವೆ-ನ್ನು-ವುದು
ದೇಹ-ವೆಲ್ಲ
ದೇಹವೇ
ದೇಹ-ಸ-ಮು-ದ್ಭ-ವಾನ್
ದೇಹಾ
ದೇಹಾಂ-ತ-ರ-ಪ್ರಾ-ಪ್ತಿ-ರ್ಧೀ-ರ-ಸ್ತತ್ರ
ದೇಹಾ-ತೀ-ತ-ವಾದ
ದೇಹಾತ್ಮ
ದೇಹಾ-ದಿ-ಗಳನ್ನು
ದೇಹಾ-ದಿ-ಗ-ಳೆಲ್ಲ
ದೇಹಾ-ಭಿ-ಮಾನ
ದೇಹಾ-ರೋಗ್ಯ
ದೇಹಾ-ಸ-ಕ್ತ-ನಾಗಿ
ದೇಹಾ-ಸ-ಕ್ತ-ನಾ-ಗಿ-ರ-ಬಾ-ರದು
ದೇಹಿ
ದೇಹಿನಃ
ದೇಹಿ-ನ-ಮ-ವ್ಯ-ಯಮ್
ದೇಹಿ-ನಮ್
ದೇಹಿ-ನಾಂ
ದೇಹಿ-ನೋ-ಸ್ಮಿನ್
ದೇಹಿ-ಯನ್ನು
ದೇಹೀ
ದೇಹೇ
ದೇಹೇಂ-ದ್ರಿ-ಯ-ಗಳಿಂದ
ದೇಹೇಂ-ದ್ರಿ-ಯ-ಗ-ಳಿಗೆ
ದೇಹೇಂ-ದ್ರಿ-ಯ-ಗ-ಳೆ-ಲ್ಲ-ಕ್ಕಿಂ-ತಲೂ
ದೇಹೇಂ-ದ್ರಿ-ಯ-ಗ-ಳೊ-ಡನೆ
ದೇಹೇಂ-ದ್ರಿ-ಯದ
ದೇಹೇ-ಽಸ್ಮಿನ್
ದೇಹೊಂದೇ
ದೈತ್ಯ-ರಲ್ಲಿ
ದೈತ್ಯಾ-ನಾಂ
ದೈನಂ-ದಿನ
ದೈನ್ಯತೆ
ದೈನ್ಯ-ತೆ-ಯನ್ನು
ದೈನ್ಯ-ತೆ-ಯಿಂ-ದಿ-ರ-ಬೇಕು
ದೈನ್ಯ-ದಿಂದ
ದೈನ್ಯ-ವೆಂಬ
ದೈವ
ದೈವಂ
ದೈವತ್ವ
ದೈವ-ತ್ವಕ್ಕೆ
ದೈವ-ತ್ವದ
ದೈವ-ತ್ವ-ವನ್ನು
ದೈವದ
ದೈವ-ಭಕ್ತ
ದೈವ-ಭ-ಕ್ತಿ-ಯನ್ನು
ದೈವ-ಭೋ-ಗ-ಗಳೂ
ದೈವ-ಮೇ-ವಾ-ಪರೇ
ದೈವ-ಯ-ಜ್ಞ-ವನ್ನೇ
ದೈವಾ-ಸುರ
ದೈವೀ
ದೈವೀಂ
ದೈವೀ-ಗುಣ
ದೈವೀ-ಮ-ಭಿ-ಜಾ-ತಸ್ಯ
ದೈವೀ-ಮ-ಭಿ-ಜಾ-ತೋಽಸಿ
ದೈವೀ-ಸಂ-ಪ-ತ್ತನ್ನು
ದೈವೀ-ಸಂ-ಪ-ತ್ತಿನ
ದೈವೀ-ಸಂ-ಪ-ದ್ವಿ-ಮೋ-ಕ್ಷಾಯ
ದೈವೀ-ಸ್ವ-ಭಾವ
ದೈವೇ-ಚ್ಛೆ-ಯಂತೆ
ದೈವೋ
ದೈಹಿಕ
ದೈಹಿ-ಕ-ವಾಗಿ
ದೈಹಿ-ಕ-ವಾ-ದು-ದಲ್ಲ
ದೊಂದಿಗೆ
ದೊಂದು
ದೊಂದೇ
ದೊಂಬ-ರಾ-ಟ-ವನ್ನು
ದೊಡನೆ
ದೊಡ್ಡ
ದೊಡ್ಡ-ತನ
ದೊಡ್ಡ-ತ-ನಕ್ಕೆ
ದೊಡ್ಡ-ತ-ನ-ವನ್ನು
ದೊಡ್ಡ-ದ-ರಲ್ಲಿ
ದೊಡ್ಡ-ದಲ್ಲ
ದೊಡ್ಡ-ದ-ಲ್ಲಿ-ರು-ವುದು
ದೊಡ್ಡ-ದಾಗಿ
ದೊಡ್ಡ-ದಾ-ಗಿ-ರ-ಬೇಕು
ದೊಡ್ಡ-ದಾ-ಗಿ-ರು-ವುದು
ದೊಡ್ಡ-ದಾ-ದರೂ
ದೊಡ್ಡ-ದಾ-ದರೆ
ದೊಡ್ಡದು
ದೊಡ್ಡ-ದೊಂದು
ದೊಡ್ಡ-ದೊಡ್ಡ
ದೊಡ್ಡಪ್ಪ
ದೊಡ್ಡ-ಪ್ಪ-ನ-ವರು
ದೊಡ್ಡ-ಮರ
ದೊಡ್ಡವ
ದೊಡ್ಡ-ವ-ನನ್ನು
ದೊಡ್ಡ-ವ-ನಾ-ಗಿ-ರ-ಬೇಕು
ದೊಡ್ಡ-ವ-ನಾದ
ದೊಡ್ಡ-ವನು
ದೊಡ್ಡ-ವರ
ದೊಡ್ಡ-ವ-ರಾ-ಗಿ-ದ್ದ-ರೇ-ನಂತೆ
ದೊಡ್ಡ-ವರು
ದೊಡ್ಡ-ಸ್ತಿ-ಕೆಗೆ
ದೊಡ್ಡ-ಸ್ತಿ-ಕೆ-ಯಲ್ಲ
ದೊರಕ
ದೊರ-ಕದ
ದೊರ-ಕ-ದಿ-ರುವ
ದೊರ-ಕದೆ
ದೊರ-ಕ-ಬಲ್ಲ
ದೊರ-ಕ-ಬೇ-ಕಾ-ದರೆ
ದೊರ-ಕ-ಲಾ-ರದು
ದೊರಕು
ದೊರ-ಕು-ತ್ತದೆ
ದೊರ-ಕು-ತ್ತವೆ
ದೊರ-ಕುವ
ದೊರ-ಕು-ವು-ದಿಲ್ಲ
ದೊರ-ಕು-ವುದು
ದೊರ-ಕು-ವುದೇ
ದೊರ-ಕು-ವು-ದೇನು
ದೊರ-ಕು-ವುದೊ
ದೊರ-ಕು-ವುವು
ದೊರು-ಕು-ವು-ದಿಲ್ಲ
ದೊರೆತ
ದೊರೆ-ತೊ-ಡನೆ
ದೊರೆಯ
ದೊರೆಯು
ದೊರೆ-ಯು-ತ್ತವೆ
ದೊರೆ-ಯುವ
ದೊರೆ-ಯು-ವುದು
ದೋಗ್ಧಾ
ದೋಚಲು
ದೋಚುತ್ತ
ದೋಣಿ
ದೋಣಿ-ಗಳನ್ನು
ದೋಣಿ-ಗಾಗಿ
ದೋಣಿ-ಗಾ-ರ-ನಿಗೆ
ದೋಣಿಗೆ
ದೋಣಿಯ
ದೋಣಿ-ಯನ್ನು
ದೋಣಿ-ಯ-ಮೇಲೆ
ದೋಣಿ-ಯಲ್ಲಿ
ದೋಣಿ-ಯ-ವ-ನಿಗೆ
ದೋಣಿ-ಯ-ವನು
ದೋಣಿ-ಯಾ-ದರೋ
ದೋಣಿ-ಯಿಂ-ದಲೇ
ದೋಣಿಯೂ
ದೋಣಿ-ಯೊ-ಳಗೆ
ದೋಷ
ದೋಷಂ
ದೋಷ-ಗಳನ್ನು
ದೋಷ-ಗಳಿಂದ
ದೋಷ-ಗಳು
ದೋಷ-ಗ-ಳೆಲ್ಲ
ದೋಷ-ಗ-ಳೆಲ್ಲಾ
ದೋಷ-ದಿಂದ
ದೋಷ-ಪೂ-ರಿ-ತ-ವಾ-ಗಿ-ರು-ವು-ದ-ರಿಂದ
ದೋಷ-ರ-ಹಿ-ತ-ವಾ-ಗಿಯೂ
ದೋಷ-ವ-ದಿ-ತ್ಯೇಕೇ
ದೋಷ-ವನ್ನು
ದೋಷ-ವಲ್ಲ
ದೋಷ-ವಿದೆ
ದೋಷ-ವಿ-ದ್ದರೂ
ದೋಷ-ವಿ-ರು-ವು-ದಿಲ್ಲ
ದೋಷವೂ
ದೋಷವೇ
ದೋಷೇಣ
ದೋಷೈ-ರೇ-ತೈಃ
ದೋಸೆಯೂ
ದೌಡಾ-ಯಿಸಿ
ದೌರ್ಬಲ್ಯ
ದೌರ್ಬ-ಲ್ಯ-ಗ-ಳಿ-ಲ್ಲದ
ದೌರ್ಬ-ಲ್ಯ-ಗಳು
ದೌರ್ಬ-ಲ್ಯದ
ದೌರ್ಬ-ಲ್ಯ-ದಿಂದ
ದೌರ್ಬ-ಲ್ಯ-ವನ್ನು
ದೌರ್ಬ-ಲ್ಯ-ವಿ-ದ್ದರೆ
ದೌರ್ಬ-ಲ್ಯ-ವೆಲ್ಲ
ದ್ಗೀತಾ
ದ್ಗೀತೆ-ಯಲ್ಲಿ
ದ್ದಕ್ಕೆ
ದ್ದರು
ದ್ದರೂ
ದ್ದರೆ
ದ್ದಲ್ಲ
ದ್ದವ-ನಿಗೆ
ದ್ದಾನೆ
ದ್ದಾನೆಯೋ
ದ್ದಾರೆ
ದ್ದಿರುವ
ದ್ದೀಪ್ತಾ-ನ-ಲಾ-ರ್ಕ-ದ್ಯು-ತಿ-ಮ-ಪ್ರ-ಮೇ-ಯಮ್
ದ್ದುದು
ದ್ದುವು
ದ್ದೇನು
ದ್ದೇನೆ
ದ್ದೇವೆಯೊ
ದ್ಭಾವನೆ
ದ್ಯಂತವೂ
ದ್ಯಚ್ಛೋ-ಕ-ಮು-ಚ್ಛೋ-ಷ-ಣ-ಮಿಂ-ದ್ರಿ-ಯಾ-ಣಾಮ್
ದ್ಯಾವಾ-ಪೃ-ಥಿ-ವ್ಯೋ-ರಿ-ದ-ಮಂ-ತರಂ
ದ್ಯೂತ
ದ್ಯೂತಂ
ದ್ಯೂತವೂ
ದ್ರಕ್ಷ್ಯ-ಸ್ಯಾ-ತ್ಮ-ನ್ಯಥೋ
ದ್ರವಂತಿ
ದ್ರವ-ಗಳನ್ನು
ದ್ರವದ
ದ್ರವ-ರೂಪ
ದ್ರವ-ರೂ-ಪ-ದಲ್ಲಿ
ದ್ರವ-ರೂ-ಪ-ದ-ಲ್ಲಿದೆ
ದ್ರವ-ರೂ-ಪ-ದ-ಲ್ಲಿ-ರು-ವಾಗ
ದ್ರವ-ರೂ-ಪ-ವಾಗಿ
ದ್ರವ-ವನ್ನು
ದ್ರವ್ಯ
ದ್ರವ್ಯ-ಕ್ಕಾಗಿ
ದ್ರವ್ಯ-ಗಳನ್ನು
ದ್ರವ್ಯ-ಗಳಿಂದ
ದ್ರವ್ಯ-ಗ-ಳಿಗೇ
ದ್ರವ್ಯ-ಗಳು
ದ್ರವ್ಯ-ಗ-ಳೆಲ್ಲ
ದ್ರವ್ಯ-ಗ-ಳೊಂ-ದಿಗೆ
ದ್ರವ್ಯ-ಮ-ಯಾ-ದ್ಯ-ಜ್ಞಾ-ಜ್ಜ್ಞಾ-ನ-ಯಜ್ಞಃ
ದ್ರವ್ಯ-ಯಜ್ಞ
ದ್ರವ್ಯ-ಯ-ಜ್ಞ-ಕ್ಕಿಂತ
ದ್ರವ್ಯ-ಯ-ಜ್ಞ-ವನ್ನು
ದ್ರವ್ಯ-ಯ-ಜ್ಞ-ವಾ-ಗು-ವುದು
ದ್ರವ್ಯ-ಯ-ಜ್ಞಾ-ಸ್ತ-ಪೋ-ಯಜ್ಞಾ
ದ್ರವ್ಯ-ರೂ-ಪ-ದ-ಲ್ಲಿ-ರಲಿ
ದ್ರವ್ಯ-ವನ್ನು
ದ್ರವ್ಯಾ-ರ್ಜನೆ
ದ್ರಷ್ಟಾ-ನು-ಪ-ಶ್ಯತಿ
ದ್ರಷ್ಟಾರ
ದ್ರಷ್ಟಾ-ರ-ನಾದ
ದ್ರಷ್ಟಾ-ರರು
ದ್ರಷ್ಟುಂ
ದ್ರಷ್ಟು-ಮ-ನೇ-ನೈವ
ದ್ರಷ್ಟು-ಮಹಂ
ದ್ರಷ್ಟು-ಮಿ-ಚ್ಛಾಮಿ
ದ್ರಷ್ಟು-ಮಿತಿ
ದ್ರಾಕ್ಷಿ
ದ್ರಾವಿ-ಡ-ರನ್ನು
ದ್ರಿಯ
ದ್ರುಪದ
ದ್ರುಪ-ದನ
ದ್ರುಪ-ದ-ನನ್ನು
ದ್ರುಪ-ದ-ಪು-ತ್ರೇಣ
ದ್ರುಪ-ದ-ರಾಜ
ದ್ರುಪ-ದಶ್ಚ
ದ್ರುಪದೋ
ದ್ರೋಣ
ದ್ರೋಣಂ
ದ್ರೋಣಃ
ದ್ರೋಣನ
ದ್ರೋಣ-ನನ್ನು
ದ್ರೋಣ-ನಿಗೂ
ದ್ರೋಣರ
ದ್ರೋಣ-ರಿ-ಬ್ಬರೂ
ದ್ರೋಣರು
ದ್ರೋಣಾ-ಚಾ-ರ್ಯನ
ದ್ರೋಣಾ-ಚಾ-ರ್ಯರ
ದ್ರೋಣಾ-ಚಾ-ರ್ಯ-ರನ್ನು
ದ್ರೋಣಾ-ಚಾ-ರ್ಯ-ರಿಂ-ದಲೇ
ದ್ರೋಣಾ-ಚಾ-ರ್ಯ-ರಿಗೆ
ದ್ರೋಣಾ-ಚಾ-ರ್ಯರು
ದ್ರೋಣಾ-ದಿ-ಗ-ಳಿಗೆ
ದ್ರೋಣಾ-ದಿ-ಗಳು
ದ್ರೋಹ
ದ್ರೋಹ-ವನ್ನು
ದ್ರೋಹ-ವ-ನ್ನೆ-ಸ-ಗು-ವು-ದಿಲ್ಲ
ದ್ರೌಪದಿ
ದ್ರೌಪ-ದಿಯ
ದ್ರೌಪ-ದೇ-ಯಾಶ್ಚ
ದ್ವಂದ್ವ
ದ್ವಂದ್ವಃ
ದ್ವಂದ್ವ-ಗಳ
ದ್ವಂದ್ವ-ಗಳಿಂದ
ದ್ವಂದ್ವ-ಗ-ಳಿಗೆ
ದ್ವಂದ್ವ-ಗ-ಳಿ-ಲ್ಲವೊ
ದ್ವಂದ್ವ-ಗಳು
ದ್ವಂದ್ವದ
ದ್ವಂದ್ವ-ದಿಂದ
ದ್ವಂದ್ವ-ಮೋ-ಹ-ನಿ-ರ್ಮುಕ್ತಾ
ದ್ವಂದ್ವ-ಮೋ-ಹೇನ
ದ್ವಂದ್ವ-ರ-ಹಿ-ತ-ರಾಗಿ
ದ್ವಂದ್ವ-ವನ್ನು
ದ್ವಂದ್ವ-ವಿಲ್ಲ
ದ್ವಂದ್ವ-ಸ-ಮಾಸ
ದ್ವಂದ್ವಾ-ತೀ-ತ-ನಾ-ಗಿ-ರು-ವನು
ದ್ವಂದ್ವಾ-ತೀತೋ
ದ್ವಂದ್ವೈ-ರ್ವಿ-ಮು-ಕ್ತಾಃ
ದ್ವಾಪರ
ದ್ವಾರಂ
ದ್ವಾರ-ಕೆಗೆ
ದ್ವಾರ-ಗಳಲ್ಲಿ
ದ್ವಾರ-ಗಳು
ದ್ವಾರದ
ದ್ವಾವಿಮೌ
ದ್ವಿಜರು
ದ್ವಿಜೋ-ತ್ತಮ
ದ್ವಿಜೋ-ತ್ತ-ಮರೆ
ದ್ವಿವಿಧಾ
ದ್ವಿಷತಃ
ದ್ವೀಪ
ದ್ವೀಪ-ಗಳು
ದ್ವೀಪ-ದಂತೆ
ದ್ವೀಪ-ದಲ್ಲಿ
ದ್ವೇಷ
ದ್ವೇಷಃ
ದ್ವೇಷ-ಇ-ವೆ-ರಡೂ
ದ್ವೇಷ-ಕ್ಕಿಂತ
ದ್ವೇಷ-ಗಳ
ದ್ವೇಷ-ಗಳನ್ನು
ದ್ವೇಷ-ಗಳಿಂದ
ದ್ವೇಷ-ಗ-ಳಿ-ವೆಯೋ
ದ್ವೇಷ-ಗಳು
ದ್ವೇಷದ
ದ್ವೇಷ-ದಿಂದ
ದ್ವೇಷ-ವಿ-ಲ್ಲದೆ
ದ್ವೇಷವೂ
ದ್ವೇಷಿ
ದ್ವೇಷಿ-ಗ-ಳಾಗಿ
ದ್ವೇಷಿ-ಗಳು
ದ್ವೇಷಿ-ಸದೇ
ದ್ವೇಷಿ-ಸ-ಬೇಡ
ದ್ವೇಷಿ-ಸಿ-ದರೂ
ದ್ವೇಷಿ-ಸಿ-ದರೆ
ದ್ವೇಷಿಸು
ದ್ವೇಷಿ-ಸುತ್ತ
ದ್ವೇಷಿ-ಸು-ತ್ತವೆ
ದ್ವೇಷಿ-ಸು-ತ್ತಾನೆ
ದ್ವೇಷಿ-ಸು-ತ್ತಾ-ನೆಯೋ
ದ್ವೇಷಿ-ಸು-ತ್ತಾರೆ
ದ್ವೇಷಿ-ಸು-ತ್ತಿ-ರುವ
ದ್ವೇಷಿ-ಸು-ತ್ತೇನೆ
ದ್ವೇಷಿ-ಸು-ತ್ತೇವೆ
ದ್ವೇಷಿ-ಸು-ತ್ತೇ-ವೆಯೊ
ದ್ವೇಷಿ-ಸು-ತ್ತೇ-ವೆಯೋ
ದ್ವೇಷಿ-ಸುವ
ದ್ವೇಷಿ-ಸು-ವಂತೆ
ದ್ವೇಷಿ-ಸು-ವನೊ
ದ್ವೇಷಿ-ಸು-ವರು
ದ್ವೇಷಿ-ಸು-ವ-ವ-ನಿಗೆ
ದ್ವೇಷಿ-ಸು-ವ-ವರು
ದ್ವೇಷಿ-ಸು-ವು-ದಕ್ಕೆ
ದ್ವೇಷಿ-ಸು-ವು-ದರ
ದ್ವೇಷಿ-ಸು-ವು-ದಿಲ್ಲ
ದ್ವೇಷಿ-ಸು-ವು-ದಿ-ಲ್ಲವೊ
ದ್ವೇಷಿ-ಸು-ವು-ದಿ-ಲ್ಲವೋ
ದ್ವೇಷಿ-ಸು-ವುದು
ದ್ವೇಷಿ-ಸು-ವುದೂ
ದ್ವೇಷಿ-ಸು-ವೆವೊ
ದ್ವೇಷ್ಟಿ
ದ್ವೇಷ್ಟ್ಯ-ಕು-ಶಲಂ
ದ್ವೇಷ್ಯೋಽಸ್ತಿ
ದ್ವೈತ
ದ್ವೈತ-ದೃ-ಷ್ಟಿ-ಯಿಂದ
ದ್ವೈತವೋ
ದ್ವೈತಿ-ಗ-ಳಾ-ದರೊ
ದ್ವೌ
ಧಕ್ಕೆ
ಧಕ್ಕೆ-ಯನ್ನು
ಧಗ-ಧಗ
ಧಗ-ಧ-ಗಿಸಿ
ಧಗ-ಧ-ಗಿ-ಸು-ತ್ತಿ-ರುವ
ಧಗ-ಧ-ಗಿ-ಸುವ
ಧಗ್
ಧಣಿ
ಧನ
ಧನಂ-ಜಯ
ಧನಂ-ಜಯಃ
ಧನ-ಕ-ನಕ
ಧನ-ಕ-ನ-ಕಾ-ದಿ-ಗಳು
ಧನದ
ಧನ-ಮಾ-ನ-ಗಳ
ಧನ-ಮಾ-ನ-ಮ-ದಾ-ನ್ವಿ-ತಾಃ
ಧನ-ವನ್ನು
ಧನ-ಸಂ-ಗ್ರಹ
ಧನಾ-ಕಾಂಕ್ಷಿ
ಧನು-ರು-ದ್ಯಮ್ಯ
ಧನು-ರ್ಧರಃ
ಧನು-ರ್ಧಾ-ರಿ-ಯಾದ
ಧನು-ರ್ವಿ-ದ್ಯಾ-ಚಾರ್ಯ
ಧನುಸ್ಸು
ಧನ್ಯ-ನಾ-ಗು-ತ್ತೇನೆ
ಧನ್ಯ-ನಾ-ಗು-ವನು
ಧನ್ಯ-ರಾ-ಗ-ಬೇಕು
ಧನ್ಯ-ರಾ-ಗಲಿ
ಧನ್ಯ-ರಾ-ಗಿರೊ
ಧನ್ಯ-ರಾ-ಗಿರೋ
ಧನ್ಯ-ರಾ-ಗು-ತ್ತೇವೆ
ಧನ್ಯ-ರಾ-ಗು-ವ-ವರು
ಧನ್ಯ-ವಾ-ಗಲಿ
ಧನ್ಯ-ವಾ-ದ-ಗಳನ್ನು
ಧನ್ಯ-ವಾ-ದ-ವನ್ನು
ಧರಿ-ಸ-ಬ-ಲ್ಲರು
ಧರಿ-ಸ-ಬ-ಹುದು
ಧರಿ-ಸ-ಬೇಕಾ
ಧರಿ-ಸ-ಬೇ-ಕಾ-ಗಿಲ್ಲ
ಧರಿ-ಸ-ಬೇ-ಕಾ-ಗು-ವುದು
ಧರಿ-ಸ-ಬೇಕು
ಧರಿ-ಸ-ಲ್ಪ-ಟ್ಟಿದೆ
ಧರಿ-ಸ-ಲ್ಪ-ಟ್ಟಿ-ರು-ವುದು
ಧರಿಸಿ
ಧರಿ-ಸಿ-ಕೊಂ-ಡಿ-ರು-ವನು
ಧರಿ-ಸಿದ
ಧರಿ-ಸಿ-ದಂ-ತಿದೆ
ಧರಿ-ಸಿ-ದಂತೆ
ಧರಿ-ಸಿ-ದ-ವನು
ಧರಿ-ಸಿ-ದಾಗ
ಧರಿ-ಸಿ-ದ್ದರೂ
ಧರಿ-ಸಿ-ದ್ದೇನೆ
ಧರಿ-ಸಿ-ರ-ಬೇ-ಕಾ-ಗಿಲ್ಲ
ಧರಿ-ಸಿರು
ಧರಿ-ಸಿ-ರುವ
ಧರಿ-ಸಿ-ರು-ವನು
ಧರಿ-ಸಿ-ರು-ವ-ವನು
ಧರಿ-ಸಿ-ರು-ವು-ದ-ರಿಂದ
ಧರಿ-ಸಿ-ರು-ವು-ದು-ರೂ-ಪ-ವನ್ನು
ಧರಿ-ಸಿಲ್ಲ
ಧರಿಸು
ಧರಿ-ಸು-ತ್ತದೆ
ಧರಿ-ಸು-ತ್ತಾನೆ
ಧರಿ-ಸು-ತ್ತಾ-ನೆಯೊ
ಧರಿ-ಸು-ತ್ತಾನೋ
ಧರಿ-ಸು-ತ್ತೇನೆ
ಧರಿ-ಸು-ತ್ತೇವೆ
ಧರಿ-ಸು-ವನು
ಧರಿ-ಸು-ವರು
ಧರಿ-ಸು-ವು-ದಕ್ಕೆ
ಧರಿ-ಸು-ವು-ದರ
ಧರಿ-ಸು-ವುದು
ಧರಿ-ಸು-ವುವು
ಧರಿ-ಸು-ವುವೋ
ಧರೆಗೆ
ಧರೆಯ
ಧರ್ಮ
ಧರ್ಮ-ಕಂ-ಟ-ಕ-ರನ್ನು
ಧರ್ಮ-ಕಂ-ಟ-ಕ-ರಾದ
ಧರ್ಮ-ಕಾ-ಮಾ-ರ್ಥಾನ್
ಧರ್ಮ-ಕ್ಕಾಗಿ
ಧರ್ಮಕ್ಕೂ
ಧರ್ಮಕ್ಕೆ
ಧರ್ಮಕ್ಕೇ
ಧರ್ಮಕ್ಕೊ
ಧರ್ಮ-ಕ್ಷೇ-ತ್ರ-ವಾದ
ಧರ್ಮ-ಕ್ಷೇತ್ರೇ
ಧರ್ಮ-ಗಳ
ಧರ್ಮ-ಗಳನ್ನು
ಧರ್ಮ-ಗಳನ್ನೂ
ಧರ್ಮ-ಗ-ಳ-ಲ್ಲಿಯೂ
ಧರ್ಮ-ಗ-ಳ-ವರು
ಧರ್ಮ-ಗಳಿಂದ
ಧರ್ಮ-ಗಳು
ಧರ್ಮ-ಗಳೂ
ಧರ್ಮ-ಗ-ಳೆಲ್ಲ
ಧರ್ಮ-ಗ್ರಂಥ
ಧರ್ಮತ್ಮಾ
ಧರ್ಮದ
ಧರ್ಮ-ದಲ್ಲಿ
ಧರ್ಮ-ದ-ಲ್ಲಿಯೂ
ಧರ್ಮ-ದ-ಲ್ಲಿರು
ಧರ್ಮ-ದ-ವ-ರಿಗೆ
ಧರ್ಮ-ದ-ವರು
ಧರ್ಮ-ದಿಂದ
ಧರ್ಮ-ಪ-ರಾ-ಯಣ
ಧರ್ಮ-ಮ-ಧರ್ಮಂ
ಧರ್ಮ-ಮಿತಿ
ಧರ್ಮ-ಯುದ್ಧ
ಧರ್ಮ-ಯು-ದ್ಧ-ಕ್ಕಿಂತ
ಧರ್ಮ-ಯು-ದ್ಧ-ದಲ್ಲಿ
ಧರ್ಮ-ಯು-ದ್ಧ-ವನ್ನು
ಧರ್ಮ-ಯು-ದ್ಧ-ವಲ್ಲ
ಧರ್ಮ-ರ-ಕ್ಷಣೆ
ಧರ್ಮ-ರ-ತ್ನಾ-ಕರ
ಧರ್ಮ-ರಾ-ಯನ
ಧರ್ಮ-ರಾ-ಯ-ನಷ್ಟೇ
ಧರ್ಮ-ವ-ನ್ನಾ-ದರೂ
ಧರ್ಮ-ವನ್ನು
ಧರ್ಮ-ವಲ್ಲ
ಧರ್ಮ-ವಾ-ಗಿದೆ
ಧರ್ಮ-ವಾದ
ಧರ್ಮ-ವಿದೆ
ಧರ್ಮವೂ
ಧರ್ಮವೆ
ಧರ್ಮ-ವೆಂದು
ಧರ್ಮವೇ
ಧರ್ಮ-ವ್ಯಾಧ
ಧರ್ಮ-ವ್ಯಾ-ಧ-ನಾ-ಗು-ತ್ತಾನೆ
ಧರ್ಮ-ಶಾಸ್ತ್ರ
ಧರ್ಮ-ಶಾ-ಸ್ತ್ರ-ಗ-ಳಾ-ಗಿ-ರ-ಬ-ಹುದು
ಧರ್ಮ-ಸಂ-ಕ-ಟ-ದ-ಲ್ಲಿದ್ದ
ಧರ್ಮ-ಸಂ-ಮೂ-ಢ-ಚೇ-ತಾಃ
ಧರ್ಮ-ಸಂ-ಸ್ಥಾ-ಪ-ನಾ-ರ್ಥಾಯ
ಧರ್ಮ-ಸಂ-ಸ್ಥಾ-ಪ-ನೆಗೆ
ಧರ್ಮ-ಸ-ಸಿಗೆ
ಧರ್ಮಸ್ಯ
ಧರ್ಮ-ಸ್ಯಾಸ್ಯ
ಧರ್ಮಾತ್ಮ
ಧರ್ಮಾ-ತ್ಮನ
ಧರ್ಮಾ-ತ್ಮ-ನಾ-ಗು-ತ್ತಾನೆ
ಧರ್ಮಾ-ತ್ಮ-ರನ್ನು
ಧರ್ಮಾ-ನು-ಸಾರ
ಧರ್ಮಾ-ವಿ-ರುದ್ಧೋ
ಧರ್ಮಾ-್ಯ-ಮೃ-ತ-ವನ್ನು
ಧರ್ಮೇ
ಧರ್ಮ್ಯಂ
ಧರ್ಮ್ಯಾದ್ಧಿ
ಧರ್ಮ್ಯಾ-ಮೃ-ತ-ಮಿದಂ
ಧಾತಾ
ಧಾತಾ-ರ-ಮ-ಚಿಂ-ತ್ಯ-ರೂ-ಪ-ಮಾ-ದಿ-ತ್ಯ-ವರ್ಣಂ
ಧಾತಾಽಹಂ
ಧಾತೃ
ಧಾನ್ಯ
ಧಾನ್ಯ-ಗಳಿಂದ
ಧಾಮ
ಧಾರ-ಕರು
ಧಾರಣ
ಧಾರಣೆ
ಧಾರ-ಣೆ-ಮಾಡಿ
ಧಾರ-ಣೆ-ಮಾ-ಡಿದ
ಧಾರ-ಣೆ-ಯಲ್ಲಿ
ಧಾರ-ಯತೇ
ಧಾರ-ಯ-ತೇ-ಽಜುನ
ಧಾರ-ಯನ್
ಧಾರ-ಯ-ನ್ನ-ಚಲಂ
ಧಾರ-ಯಾ-ಮ್ಯ-ಹ-ಮೋ-ಜಸಾ
ಧಾರಾ-ಳ-ವಾಗಿ
ಧಾರೆ
ಧಾರೆ-ಯಂತೆ
ಧಾರೆ-ಯೆ-ರೆದು
ಧಾರೆ-ಯೆ-ರೆ-ಯ-ಬೇಕು
ಧಾರೆ-ಯೆ-ರೆ-ಯು-ವನು
ಧಾರೆ-ಯೆ-ರೆ-ಯು-ವುದು
ಧಾರ್ತ-ರಾ-ಷ್ಟ್ರಸ್ಯ
ಧಾರ್ತ-ರಾಷ್ಟ್ರಾ
ಧಾರ್ತ-ರಾ-ಷ್ಟ್ರಾಃ
ಧಾರ್ತ-ರಾ-ಷ್ಟ್ರಾ-ಣಾಂ
ಧಾರ್ತ-ರಾ-ಷ್ಟ್ರಾನ್
ಧಾರ್ತ-ರಾ-ಷ್ಟ್ರಾನ್ನಃ
ಧಾರ್ಮಿಕ
ಧಾರ್ಮಿ-ಕ-ವಾಗಿ
ಧಾರ್ಯತೇ
ಧಾಳಿ
ಧಾಳಿ-ಯಿ-ಡು-ವುವು
ಧಾವಿಸಿ
ಧಾವಿ-ಸುತ್ತ
ಧಾವಿ-ಸು-ತ್ತಿ-ದ್ದರೆ
ಧಾವಿ-ಸು-ತ್ತಿ-ರ-ಬೇಕು
ಧಾವಿ-ಸು-ತ್ತಿ-ರು-ವಾಗ
ಧಾವಿ-ಸು-ತ್ತಿವೆ
ಧಾವಿ-ಸು-ವನು
ಧಾವಿ-ಸು-ವು-ದಕ್ಕೆ
ಧಾವಿ-ಸು-ವು-ದಿಲ್ಲ
ಧಾವಿ-ಸು-ವುದು
ಧಾವಿ-ಸು-ವುದೊ
ಧಾವಿ-ಸು-ವುದೋ
ಧಾವಿ-ಸು-ವುವು
ಧಾವಿ-ಸು-ವೆವು
ಧಿಃಕ-ರಿ-ಸು-ವು-ದಿಲ್ಲ
ಧಿಕ್ಕ-ರಿ-ಸು-ವು-ದಿಲ್ಲ
ಧೀಮಂ-ತ-ರಾದ
ಧೀಮತಾ
ಧೀಮ-ತಾಮ್
ಧೀರ
ಧೀರಂ
ಧೀರ-ನನ್ನು
ಧೀರ-ನಾಗಿ
ಧೀರ-ನಾದ
ಧೀರ-ನಾ-ದರೊ
ಧೀರನು
ಧೀರ-ಮ-ತಿಗೆ
ಧೀರ-ರಾ-ಗಿ-ರ-ಬೇಕು
ಧೀರ-ರಾದ
ಧೀರ-ರಿಗೆ
ಧೀರ-ರಿ-ದ್ದಾರೆ
ಧೀರರು
ಧೀರರೂ
ಧೀರ-ಸ್ತು-ಲ್ಯ-ನಿಂ-ದಾ-ತ್ಮ-ಸಂ-ಸ್ತು-ತಿಃ
ಧೀರಾ-ಧಿ-ಧೀರ
ಧುಮು-ಕು-ವನು
ಧುಮು-ಕು-ವೆವು
ಧೂನಿ
ಧೂಮ
ಧೂಮದ
ಧೂಮ-ದ-ಲ್ಲಿಯೂ
ಧೂಮೇ-ನಾ-ಗ್ನಿ-ರಿ-ವಾ-ವೃ-ತಾಃ
ಧೂಮೇ-ನಾ-ವ್ರಿ-ಯತೇ
ಧೂಮೋ
ಧೂಳಿ-ಕ-ಣ-ಗ-ಳಾ-ಗು-ವುದು
ಧೂಳೀ-ಕ-ಣ-ಗ-ಳಾಗಿ
ಧೂಳೀ-ಕ-ಣ-ಗಳು
ಧೂಳೀ-ಸಮ
ಧೂಳು
ಧೃತ-ರಾಷ್ಟ್ರ
ಧೃತ-ರಾ-ಷ್ಟ್ರನ
ಧೃತ-ರಾ-ಷ್ಟ್ರ-ನಿಗೆ
ಧೃತ-ರಾ-ಷ್ಟ್ರನು
ಧೃತ-ರಾ-ಷ್ಟ್ರಸ್ಯ
ಧೃತಿ
ಧೃತಿಂ
ಧೃತಿಃ
ಧೃತಿ-ಗೃ-ಹೀ-ತಯಾ
ಧೃತಿ-ಯಿಂದ
ಧೃತಿ-ರ್ದಾಕ್ಷ್ಯಂ
ಧೃತೇ-ಶ್ಚೈವ
ಧೃತ್ಯಾ
ಧೃತ್ಯಾ-ತ್ಮಾನಂ
ಧೃತ್ಯು-ತ್ಸಾ-ಹ-ಸ-ಮ-ನ್ವಿತಃ
ಧೃವ-ತಾರೆ
ಧೃವ-ತಾ-ರೆ-ಯಂತೆ
ಧೃಷ್ಟ-ದ್ಯು-ಮ್ನರು
ಧೃಷ್ಟ-ದ್ಯುಮ್ನೋ
ಧೇನೂ-ನಾ-ಮಸ್ಮಿ
ಧೈರ್ಯ
ಧೈರ್ಯ-ಗೆ-ಡು-ವು-ದಿಲ್ಲ
ಧೈರ್ಯ-ದಿಂದ
ಧೈರ್ಯ-ದೊಂ-ದಿಗೆ
ಧೈರ್ಯ-ವನ್ನು
ಧೈರ್ಯ-ವಾಗಿ
ಧೈರ್ಯ-ವಾ-ಗಿರ
ಧೈರ್ಯ-ವಿದೆ
ಧೈರ್ಯ-ವಿ-ರ-ಬೇಕು
ಧೈರ್ಯ-ವಿ-ರ-ಲಿಲ್ಲ
ಧೈರ್ಯ-ವಿ-ರು-ವುದೊ
ಧೈರ್ಯ-ವಿಲ್ಲ
ಧೈರ್ಯವು
ಧೈರ್ಯವೂ
ಧೈರ್ಯವೇ
ಧ್ಯಾನ
ಧ್ಯಾನ-ಕ್ಕಿಂತ
ಧ್ಯಾನಕ್ಕೆ
ಧ್ಯಾನ-ಜೀ-ವಿಯೂ
ಧ್ಯಾನದ
ಧ್ಯಾನ-ದಲ್ಲಿ
ಧ್ಯಾನ-ದ-ಲ್ಲಿ-ರು-ವಾಗ
ಧ್ಯಾನ-ದಲ್ಲೆ
ಧ್ಯಾನ-ದಿಂದ
ಧ್ಯಾನ-ಮಾ-ಡದೇ
ಧ್ಯಾನ-ಮಾ-ಡ-ಬೇ-ಕಾ-ದರೆ
ಧ್ಯಾನ-ಮಾ-ಡ-ಬೇ-ಕೆಂದು
ಧ್ಯಾನ-ಮಾ-ಡ-ಲಾ-ರರು
ಧ್ಯಾನ-ಮಾಡಿ
ಧ್ಯಾನ-ಮಾಡು
ಧ್ಯಾನ-ಮಾ-ಡು-ತ್ತಿ-ರುವ
ಧ್ಯಾನ-ಮಾ-ಡುವ
ಧ್ಯಾನ-ಮಾ-ಡು-ವನು
ಧ್ಯಾನ-ಮಾ-ಡು-ವ-ವನು
ಧ್ಯಾನ-ಮಾ-ಡು-ವಾಗ
ಧ್ಯಾನ-ಮಾ-ಡು-ವು-ದಕ್ಕೆ
ಧ್ಯಾನ-ಮಾ-ಡು-ವುದು
ಧ್ಯಾನ-ಮಾರ್ಗ
ಧ್ಯಾನ-ಯೋಗ
ಧ್ಯಾನ-ಯೋ-ಗ-ಗಳು
ಧ್ಯಾನ-ಯೋ-ಗ-ದಲ್ಲಿ
ಧ್ಯಾನ-ಯೋ-ಗ-ದಲ್ಲೇ
ಧ್ಯಾನ-ಯೋ-ಗ-ಪರೋ
ಧ್ಯಾನ-ಯೋಗಿ
ಧ್ಯಾನ-ವನ್ನೋ
ಧ್ಯಾನ-ವಾ-ಗಲಿ
ಧ್ಯಾನ-ವಾ-ಗು-ವುದು
ಧ್ಯಾನವು
ಧ್ಯಾನವೂ
ಧ್ಯಾನ-ವೆಂದರೆ
ಧ್ಯಾನವೊ
ಧ್ಯಾನ-ವೊಂದು
ಧ್ಯಾನವೋ
ಧ್ಯಾನ-ಶ್ಲೋ-ಕ-ಗಳನ್ನು
ಧ್ಯಾನ-ಶ್ಲೋ-ಕ-ಗಳಲ್ಲಿ
ಧ್ಯಾನ-ಶ್ಲೋ-ಕ-ಗಳು
ಧ್ಯಾನ-ಸ-ಮ-ಯ-ದಲ್ಲಿ
ಧ್ಯಾನಾ-ಗ್ನಿ-ಯಲ್ಲಿ
ಧ್ಯಾನಾತ್
ಧ್ಯಾನಾ-ನಂದ
ಧ್ಯಾನಾ-ನಂ-ದ-ಕ್ಕಿಂತ
ಧ್ಯಾನಾ-ವ-ಸ್ಥಿ-ತ-ತ-ದ್ಗ-ತೇನ
ಧ್ಯಾನಾ-ವ-ಸ್ಥೆ-ಯಲ್ಲಿ
ಧ್ಯಾನಾ-ವ-ಸ್ಥೆ-ಯ-ಲ್ಲಿಯೇ
ಧ್ಯಾನಾ-ವ-ಸ್ಥೆ-ಯ-ಲ್ಲಿ-ರುವ
ಧ್ಯಾನಾ-ವ-ಸ್ಥೆಯೇ
ಧ್ಯಾನಿ
ಧ್ಯಾನಿಗೆ
ಧ್ಯಾನಿ-ಸ-ಬೇ-ಕಾ-ದರೆ
ಧ್ಯಾನಿ-ಸಲಿ
ಧ್ಯಾನಿಸು
ಧ್ಯಾನಿ-ಸುತ್ತ
ಧ್ಯಾನಿ-ಸು-ತ್ತಿ-ರ-ಬೇಕು
ಧ್ಯಾನಿ-ಸು-ತ್ತಿ-ರುವ
ಧ್ಯಾನಿ-ಸು-ತ್ತೇನೆ
ಧ್ಯಾನಿ-ಸುವ
ಧ್ಯಾನಿ-ಸು-ವರು
ಧ್ಯಾನಿ-ಸು-ವರೊ
ಧ್ಯಾನಿ-ಸು-ವ-ವ-ನಿಗೂ
ಧ್ಯಾನಿ-ಸು-ವ-ವರು
ಧ್ಯಾನಿ-ಸು-ವಾಗ
ಧ್ಯಾನಿ-ಸು-ವು-ದಕ್ಕೆ
ಧ್ಯಾನಿ-ಸು-ವುದು
ಧ್ಯಾನಿ-ಸು-ವು-ದೆಂ-ದರೆ
ಧ್ಯಾನೇ-ನಾ-ತ್ಮನಿ
ಧ್ಯಾಯಂತ
ಧ್ಯಾಯತೋ
ಧ್ಯೆರ್ಯ-ವನ್ನು
ಧ್ಯೇಯ-ವ-ಸ್ತು-ವನ್ನು
ಧ್ಯೇಯ-ವ-ಸ್ತು-ವಿನ
ಧ್ಯೇಯ-ವ-ಸ್ತು-ವಿ-ನಿಂದ
ಧ್ಯೇಯ-ವ-ಸ್ತುವೇ
ಧ್ರುವ
ಧ್ರುವ-ಗಳಲ್ಲಿ
ಧ್ರುವ-ತಾರೆ
ಧ್ರುವ-ತಾ-ರೆ-ಯಂತೆ
ಧ್ರುವ-ತಾ-ರೆ-ಯ-ನ್ನಾಗಿ
ಧ್ರುವದ
ಧ್ರುವಮ್
ಧ್ರುವೋ
ಧ್ವಂಸ
ಧ್ವಂಸ-ಕಾ-ರಿಯೂ
ಧ್ವಂಸ-ಮಾ-ಡ-ಲಾ-ಗು-ವುದೆ
ಧ್ವಂಸ-ಮಾ-ಡಲು
ಧ್ವಂಸ-ಮಾ-ಡಿ-ದಂತೆ
ಧ್ವಂಸ-ಮಾ-ಡಿ-ದರೂ
ಧ್ವಂಸ-ಮಾ-ಡಿ-ರು-ವು-ದಿಲ್ಲ
ಧ್ವಂಸ-ಮಾ-ಡು-ವು-ದಕ್ಕೆ
ಧ್ವಂಸ-ಮಾ-ಡು-ವು-ದಿಲ್ಲ
ಧ್ವಂಸ-ಲೀ-ಲೆ-ಯಲ್ಲಿ
ಧ್ವಂಸ-ವಾ-ಗು-ವುವೊ
ಧ್ವಜ-ಪ-ಟ-ವನ್ನು
ಧ್ವಜ-ವನ್ನು
ಧ್ವಜ-ಸ್ತಂ-ಭದ
ಧ್ವನಿ
ಧ್ವನಿ-ತ-ವಾ-ಗಿದೆ
ಧ್ವನಿ-ಪೂ-ರಿತ
ಧ್ವನಿ-ಪೂ-ರ್ಣ-ವಾ-ಗಿದೆ
ಧ್ವನಿ-ಪೂ-ರ್ಣ-ವಾದ
ಧ್ವನಿ-ಪೂ-ರ್ಣ-ವಾ-ದುದು
ಧ್ವನಿ-ಯನ್ನು
ಧ್ವನಿ-ಯಾದ
ಧ್ವನಿ-ಯೊಂದೇ
ಧ್ವನಿ-ವೂ-ರ್ಣ-ವಾ-ದುದು
ನ
ನಂಟ
ನಂಟ-ರಿ-ಷ್ಟರು
ನಂಟರೋ
ನಂತರ
ನಂತ-ರದ
ನಂತ-ರವೂ
ನಂತೆ
ನಂದದೆ
ನಂದರೇ
ನಂದ-ವನ್ನು
ನಂದಾ-ದೀ-ಪ-ದಂತೆ
ನಂದಿ-ಹೋ-ಗು-ವು-ದಿಲ್ಲ
ನಂಬದ
ನಂಬ-ದ-ವ-ನಿಗೆ
ನಂಬ-ದ-ವರು
ನಂಬದೆ
ನಂಬದೇ
ನಂಬ-ಬ-ಹುದು
ನಂಬ-ಬೇ-ಕೆಂಬ
ನಂಬರು
ನಂಬರ್
ನಂಬಲು
ನಂಬಿ
ನಂಬಿಕೆ
ನಂಬಿ-ಕೆಗೆ
ನಂಬಿ-ಕೆಯ
ನಂಬಿ-ಕೆಯೇ
ನಂಬಿದ
ನಂಬಿ-ದರೆ
ನಂಬಿ-ದಳು
ನಂಬಿ-ದ-ವ-ನಿಗೆ
ನಂಬಿ-ದ-ವನು
ನಂಬಿ-ದ-ವ-ರಿಗೆ
ನಂಬಿ-ದ-ವರು
ನಂಬಿ-ದ್ದೇನೆ
ನಂಬಿ-ದ್ದೇವೆ
ನಂಬಿ-ರ-ಬೇಕು
ನಂಬಿರು
ನಂಬಿ-ರು-ವೆವೊ
ನಂಬಿ-ಹೋ-ದರೆ
ನಂಬು
ನಂಬುತ್ತ
ನಂಬು-ತ್ತಾನೆ
ನಂಬು-ತ್ತಾ-ನೆಯೆ
ನಂಬು-ತ್ತಾ-ನೆಯೇ
ನಂಬು-ತ್ತಾನೋ
ನಂಬು-ತ್ತಾರೆ
ನಂಬು-ತ್ತಾರೊ
ನಂಬು-ತ್ತಾರೋ
ನಂಬು-ತ್ತಿ-ರ-ಲಿಲ್ಲ
ನಂಬುವ
ನಂಬು-ವನು
ನಂಬು-ವನೋ
ನಂಬು-ವರು
ನಂಬು-ವರೋ
ನಂಬು-ವ-ವ-ನಲ್ಲ
ನಂಬು-ವ-ವ-ನಿಗೆ
ನಂಬು-ವ-ವನು
ನಂಬು-ವ-ವ-ರಾ-ದರೊ
ನಂಬು-ವ-ವರು
ನಂಬುವು
ನಂಬು-ವು-ದಕ್ಕೆ
ನಂಬು-ವು-ದಿಲ್ಲ
ನಂಬು-ವು-ದಿ-ಲ್ಲವೊ
ನಂಬು-ವು-ದಿ-ಲ್ಲವೋ
ನಂಬು-ವುದು
ನಂಬು-ವುದೂ
ನಂಬು-ವುದೇ
ನಃ
ನಅವು
ನಕುಲ
ನಕುಲಃ
ನಕ್ಕರು
ನಕ್ಷತ್ರ
ನಕ್ಷ-ತ್ರ-ಗಳ
ನಕ್ಷ-ತ್ರ-ಗಳಲ್ಲಿ
ನಕ್ಷ-ತ್ರ-ಗ-ಳ-ವ-ರೆಗೆ
ನಕ್ಷ-ತ್ರ-ಗಳು
ನಕ್ಷ-ತ್ರ-ಗ-ಳೆಲ್ಲ
ನಕ್ಷ-ತ್ರ-ಗ-ಳೇನೊ
ನಕ್ಷ-ತ್ರ-ದಲ್ಲಿ
ನಕ್ಷ-ತ್ರ-ಲೋ-ಕ-ದಲ್ಲಿ
ನಕ್ಷ-ತ್ರಾ-ಣಾ-ಮಹಂ
ನಕ್ಷೆ
ನಕ್ಷೆ-ಯನ್ನು
ನಕ್ಷೆ-ಯಲ್ಲಿ
ನಕ್ಷೆ-ಯ-ಲ್ಲಿರು
ನಖ
ನಖ-ಗಳಿಂದ
ನಖ-ದಾ-ಡೆ-ಗಳನ್ನು
ನಗ-ಣ್ಯ-ನಾ-ಗು-ವನು
ನಗ-ಣ್ಯ-ರಲ್ಲ
ನಗ-ಣ್ಯ-ವಸ್ತು
ನಗ-ನಾ-ಣ್ಯದ
ನಗರ
ನಗ-ರ-ಗಳಲ್ಲಿ
ನಗ-ರ-ಗಳು
ನಗ-ರದ
ನಗ-ರ-ದಲ್ಲಿ
ನಗ-ಲೇ-ಬೇ-ಕಾ-ಗು-ವುದು
ನಗಿ-ಸ-ಲಾ-ರವು
ನಗಿ-ಸಿ-ದರೆ
ನಗಿ-ಸು-ವಂತೆ
ನಗು
ನಗುತ್ತ
ನಗುತ್ತಾ
ನಗು-ತ್ತಿ-ರು-ವ-ವ-ನಂತೆ
ನಗು-ತ್ತೇವೆ
ನಗು-ವನು
ನಗು-ವ-ವ-ನಂತೆ
ನಗು-ವ-ವ-ನಲ್ಲ
ನಗು-ವಿನ
ನಗು-ವಿ-ನಿಂದ
ನಗು-ವುದೂ
ನಗ್ನ
ನಟ
ನಟ-ನಾ-ಗಿದ್ದ
ನಟನೆ
ನಟಿ-ಯಿಂದ
ನಟಿ-ಸಿ-ದರೆ
ನಟಿ-ಸುತ್ತ
ನಟ್ಟೊ
ನಡತೆ
ನಡ-ತೆಗೂ
ನಡ-ತೆ-ಯಾ-ಗಲೀ
ನಡ-ತೆಯೂ
ನಡ-ವ-ಳಿಕೆ
ನಡ-ವ-ಳಿ-ಕೆಗೂ
ನಡ-ಸು-ವುದು
ನಡು-ಗುತ್ತ
ನಡು-ಗು-ತ್ತಿದೆ
ನಡು-ಗು-ತ್ತಿರು
ನಡು-ಗು-ತ್ತಿವೆ
ನಡು-ನೀ-ರಲಿ
ನಡು-ನೀ-ರಿ-ನಲ್ಲಿ
ನಡು-ರ-ಸ್ತೆ-ಯಲ್ಲಿ
ನಡುವೆ
ನಡೆ
ನಡೆದ
ನಡೆ-ದರೆ
ನಡೆ-ದಳು
ನಡೆ-ದ-ವರು
ನಡೆದು
ನಡೆ-ದು-ಕೊಂ-ಡರೆ
ನಡೆ-ದು-ಕೊಂಡು
ನಡೆ-ದು-ಕೊಂಡೇ
ನಡೆ-ದು-ಕೊ-ಳ್ಳು-ತ್ತಾನೆ
ನಡೆ-ದು-ಹೋದ
ನಡೆ-ನುಡಿ
ನಡೆಯ
ನಡೆ-ಯ-ಬೇ-ಕಾ-ದರೆ
ನಡೆ-ಯ-ಬೇಕು
ನಡೆ-ಯ-ಬೇ-ಕೆಂದು
ನಡೆ-ಯಲು
ನಡೆ-ಯ-ವ-ವರು
ನಡೆ-ಯಿತು
ನಡೆಯು
ನಡೆ-ಯು-ತ್ತದೆ
ನಡೆ-ಯು-ತ್ತಾನೆ
ನಡೆ-ಯು-ತ್ತಿದೆ
ನಡೆ-ಯು-ತ್ತಿ-ದ್ದರೆ
ನಡೆ-ಯು-ತ್ತಿ-ದ್ದಾಗ
ನಡೆ-ಯು-ತ್ತಿ-ರ-ಬೇಕು
ನಡೆ-ಯು-ತ್ತಿ-ರು-ವನೊ
ನಡೆ-ಯು-ತ್ತಿ-ರು-ವಾಗ
ನಡೆ-ಯು-ತ್ತಿ-ರು-ವುದನ್ನು
ನಡೆ-ಯು-ತ್ತಿ-ರು-ವೆವು
ನಡೆ-ಯು-ತ್ತೇನೆ
ನಡೆ-ಯುವ
ನಡೆ-ಯು-ವನು
ನಡೆ-ಯು-ವರು
ನಡೆ-ಯು-ವ-ವ-ರನ್ನು
ನಡೆ-ಯು-ವ-ವರೇ
ನಡೆ-ಯು-ವಾಗ
ನಡೆ-ಯು-ವು-ದಕ್ಕೆ
ನಡೆ-ಯು-ವುದನ್ನು
ನಡೆ-ಯು-ವು-ದ-ರಿಂದ
ನಡೆ-ಯು-ವು-ದಿಲ್ಲ
ನಡೆ-ಯು-ವುದು
ನಡೆ-ಯು-ವುದೇ
ನಡೆ-ಯು-ವು-ದೊಂ-ದ-ರಲ್ಲೇ
ನಡೆ-ಯು-ವುವು
ನಡೆ-ಸ-ದ-ವ-ರಿಗೆ
ನಡೆ-ಸ-ಬ-ಹುದು
ನಡೆ-ಸ-ಬೇಕು
ನಡೆ-ಸಲು
ನಡೆಸಿ
ನಡೆ-ಸಿ-ಕೊಂಡು
ನಡೆ-ಸಿ-ಕೊ-ಡು-ವನು
ನಡೆ-ಸಿ-ದರೆ
ನಡೆ-ಸಿ-ದ್ದರೆ
ನಡೆ-ಸಿ-ದ್ದೇವೆ
ನಡೆ-ಸು-ತ್ತಾನೆ
ನಡೆ-ಸು-ತ್ತಾರೆ
ನಡೆ-ಸು-ತ್ತಿದ್ದ
ನಡೆ-ಸು-ತ್ತಿ-ದ್ದರು
ನಡೆ-ಸು-ತ್ತಿ-ದ್ದರೂ
ನಡೆ-ಸು-ತ್ತಿ-ದ್ದುದೇ
ನಡೆ-ಸು-ತ್ತಿ-ರ-ಬೇಕು
ನಡೆ-ಸು-ತ್ತಿ-ರುವ
ನಡೆ-ಸು-ತ್ತಿ-ರು-ವನು
ನಡೆ-ಸು-ತ್ತಿ-ರು-ವ-ವನು
ನಡೆ-ಸು-ತ್ತಿ-ರು-ವಾಗ
ನಡೆ-ಸು-ತ್ತಿ-ರು-ವುದು
ನಡೆ-ಸು-ತ್ತಿಲ್ಲ
ನಡೆ-ಸುವ
ನಡೆ-ಸು-ವನು
ನಡೆ-ಸು-ವ-ವನು
ನಡೆ-ಸು-ವ-ವರು
ನಡೆ-ಸು-ವು-ದಕ್ಕೆ
ನಡೆ-ಸು-ವುದು
ನದಿ
ನದಿ-ಗಳ
ನದಿ-ಗಳನ್ನೂ
ನದಿ-ಗಳಲ್ಲಿ
ನದಿ-ಗಳಿಂದ
ನದಿ-ಗ-ಳಿವೆ
ನದಿ-ಗಳು
ನದಿ-ಗ-ಳೆಲ್ಲ
ನದಿಗೆ
ನದಿ-ಗೆಲ್ಲ
ನದಿ-ತುಂಬಿ
ನದಿಯ
ನದಿ-ಯನ್ನು
ನದಿ-ಯಲ್ಲಿ
ನದಿ-ಯಾ-ಗ-ಬ-ಹುದು
ನದಿ-ಯಾ-ಗು-ವುದೋ
ನದಿಯೂ
ನದಿಯೇ
ನದಿ-ಯೊಂದು
ನದಿ-ಯೊ-ಡನೆ
ನದೀ
ನದೀ-ತೀರ
ನದೀ-ತೀ-ರಕ್ಕೆ
ನದೀ-ತೀ-ರ-ದ-ಲ್ಲಿದ್ದ
ನದೀ-ನಾಂ
ನದೀ-ಪಾ-ಲಾ-ಗು-ವೆವು
ನದೀ-ಮು-ಖ-ದಲ್ಲಿ
ನನ-ಗ-ರ್ಪಿಸಿ
ನನ-ಗಲ್ಲ
ನನ-ಗಾಗಿ
ನನ-ಗಾ-ಗಿಯೇ
ನನ-ಗಿಂತ
ನನಗೂ
ನನಗೆ
ನನ-ಗೆಲ್ಲ
ನನ-ಗೇಕೆ
ನನ-ಗೇನು
ನನ-ಗೇನೂ
ನನ-ಗೊಂದು
ನನ-ಗೋ-ಸ್ಕರ
ನನ್ನ
ನನ್ನಂ-ತಹ
ನನ್ನ-ದ-ನ್ನೆಲ್ಲಾ
ನನ್ನ-ದ-ಲ್ಲದ
ನನ್ನದು
ನನ್ನ-ದೆಂಬ
ನನ್ನದೇ
ನನ್ನನ್ನು
ನನ್ನನ್ನೂ
ನನ್ನನ್ನೇ
ನನ್ನಲ್ಲಿ
ನನ್ನ-ಲ್ಲಿಗೆ
ನನ್ನ-ಲ್ಲಿದೆ
ನನ್ನ-ಲ್ಲಿಯೂ
ನನ್ನ-ಲ್ಲಿಯೆ
ನನ್ನ-ಲ್ಲಿಯೇ
ನನ್ನ-ಲ್ಲಿ-ರುವ
ನನ್ನ-ಲ್ಲಿ-ರು-ವುದು
ನನ್ನ-ಲ್ಲಿ-ರು-ವು-ದೆಲ್ಲ
ನನ್ನ-ಲ್ಲಿವೆ
ನನ್ನಲ್ಲೇ
ನನ್ನ-ವರು
ನನ್ನವೇ
ನನ್ನಷ್ಟು
ನನ್ನಿಂದ
ನನ್ನಿಂ-ದಲೆ
ನನ್ನಿಂ-ದಲೇ
ನನ್ನು
ನನ್ನೆ-ಡೆಗೆ
ನನ್ನೆ-ಡೆಗೇ
ನನ್ನೊಂ-ದಿಗೆ
ನನ್ನೊ-ಬ್ಬ-ನನ್ನೇ
ನನ್ನೊ-ಬ್ಬ-ನಲ್ಲಿ
ನನ್ನೊ-ಳಗೆ
ನನ್ನೊ-ಳಗೇ
ನಭಃ-ಸ್ಪೃಶಂ
ನಭಶ್ಚ
ನಮಃ
ನಮ-ಗಲ್ಲ
ನಮ-ಗಲ್ಲಿ
ನಮ-ಗಾಗಿ
ನಮ-ಗಾ-ಗು-ವು-ದಿಲ್ಲ
ನಮ-ಗಾ-ದರೆ
ನಮ-ಗಿಂತ
ನಮ-ಗಿಂ-ತಲೂ
ನಮ-ಗಿ-ರುವ
ನಮ-ಗುಂ-ಟಾ-ಗು-ವು-ದಿಲ್ಲ
ನಮಗೂ
ನಮಗೆ
ನಮ-ಗೆ-ಎ-ಚ್ಚ-ರ-ವಾ-ಗು-ವುದು
ನಮ-ಗೆಲ್ಲ
ನಮ-ಗೆಲ್ಲಾ
ನಮ-ಗೆಷ್ಟು
ನಮಗೇ
ನಮ-ಗೇ-ನಾ-ಗು-ವುದು
ನಮ-ಗೇನು
ನಮ-ಗೇನೂ
ನಮ-ಗೊಂದು
ನಮ-ಸ್ಕರಿ
ನಮ-ಸ್ಕ-ರಿಸ
ನಮ-ಸ್ಕ-ರಿ-ಸದೆ
ನಮ-ಸ್ಕ-ರಿ-ಸ-ಬೇಕು
ನಮ-ಸ್ಕ-ರಿಸಿ
ನಮ-ಸ್ಕ-ರಿಸು
ನಮ-ಸ್ಕ-ರಿ-ಸುತ್ತ
ನಮ-ಸ್ಕ-ರಿ-ಸು-ತ್ತಾನೆ
ನಮ-ಸ್ಕ-ರಿ-ಸು-ತ್ತಾರೆ
ನಮ-ಸ್ಕ-ರಿ-ಸು-ತ್ತಿದೆ
ನಮ-ಸ್ಕ-ರಿ-ಸು-ವರು
ನಮ-ಸ್ಕ-ರಿ-ಸು-ವು-ದ-ರೊಂ-ದಿಗೆ
ನಮ-ಸ್ಕ-ರಿ-ಸು-ವುದು
ನಮ-ಸ್ಕಾರ
ನಮ-ಸ್ಕಾ-ರ-ಗಳು
ನಮ-ಸ್ಕಾ-ರ-ದಲ್ಲಿ
ನಮ-ಸ್ಕಾ-ರ-ದಿಂ-ದಲೇ
ನಮ-ಸ್ಕುರು
ನಮ-ಸ್ಕೃತ್ವಾ
ನಮಸ್ತೇ
ನಮ-ಸ್ತೇಽಸ್ತು
ನಮ-ಸ್ಯಂ-ತಶ್ಚ
ನಮ-ಸ್ಯಂತಿ
ನಮಿ-ಸು-ತ್ತೇನೆ
ನಮೇ-ರನ್
ನಮೋ
ನಮೋಸ್ತು
ನಮೋಽಸ್ತು
ನಮ್ಮ
ನಮ್ಮಂ-ತಹ
ನಮ್ಮಂ-ತಿ-ರುವ
ನಮ್ಮಂತೆ
ನಮ್ಮಂ-ತೆಯೆ
ನಮ್ಮಂ-ತೆಯೇ
ನಮ್ಮ-ಅಂ-ತ-ರ್ಯಾ-ಮಿ-ಯಾಗಿ
ನಮ್ಮ-ಕ-ರ್ಮ-ಗ-ಳಿಗೆ
ನಮ್ಮ-ಗಳ
ನಮ್ಮ-ತನ
ನಮ್ಮ-ದ-ನ್ನಾಗಿ
ನಮ್ಮ-ದನ್ನು
ನಮ್ಮ-ದಲ್ಲ
ನಮ್ಮ-ದ-ಲ್ಲ-ದಿ-ರು-ವು-ದ-ನ್ನೆಲ್ಲಾ
ನಮ್ಮ-ದ-ಲ್ಲ-ದು-ದನ್ನು
ನಮ್ಮ-ದಾ-ಗ-ಬೇಕು
ನಮ್ಮ-ದಾ-ಗಿರ
ನಮ್ಮ-ದಾ-ಗಿ-ರು-ವು-ದಿಲ್ಲ
ನಮ್ಮ-ದಾ-ಗಿಲ್ಲ
ನಮ್ಮ-ದಾ-ಗು-ವು-ದಿಲ್ಲ
ನಮ್ಮ-ದಾ-ಗು-ವುದು
ನಮ್ಮದು
ನಮ್ಮ-ದುಃಖ
ನಮ್ಮ-ದೆಂದು
ನಮ್ಮ-ದೆಂ-ಬುದು
ನಮ್ಮ-ದೆಂ-ಬುವ
ನಮ್ಮ-ದೆಲ್ಲ
ನಮ್ಮದೇ
ನಮ್ಮನ್ನು
ನಮ್ಮ-ನ್ನೆಲ್ಲ
ನಮ್ಮ-ನ್ನೆಲ್ಲಾ
ನಮ್ಮನ್ನೇ
ನಮ್ಮ-ಮೇಲೆ
ನಮ್ಮಲ್ಲಿ
ನಮ್ಮ-ಲ್ಲಿ-ಇದೆ
ನಮ್ಮ-ಲ್ಲಿ-ಟ್ಟಿ-ರುವ
ನಮ್ಮ-ಲ್ಲಿಟ್ಟು
ನಮ್ಮ-ಲ್ಲಿತ್ತು
ನಮ್ಮ-ಲ್ಲಿದೆ
ನಮ್ಮ-ಲ್ಲಿ-ದ್ದರೆ
ನಮ್ಮ-ಲ್ಲಿಯೂ
ನಮ್ಮ-ಲ್ಲಿಯೇ
ನಮ್ಮ-ಲ್ಲಿ-ರ-ಬೇಕು
ನಮ್ಮ-ಲ್ಲಿರು
ನಮ್ಮ-ಲ್ಲಿ-ರುವ
ನಮ್ಮ-ಲ್ಲಿ-ರು-ವುದನ್ನು
ನಮ್ಮ-ಲ್ಲಿ-ರು-ವು-ದ-ರಿಂದ
ನಮ್ಮ-ಲ್ಲಿ-ರು-ವುದು
ನಮ್ಮ-ಲ್ಲಿ-ರು-ವುದೊ
ನಮ್ಮ-ಲ್ಲಿ-ರು-ವುದೋ
ನಮ್ಮ-ಲ್ಲಿವೆ
ನಮ್ಮಲ್ಲೆ
ನಮ್ಮ-ಲ್ಲೆಲ್ಲ
ನಮ್ಮ-ಲ್ಲೆಲ್ಲಾ
ನಮ್ಮಲ್ಲೇ
ನಮ್ಮ-ವರ
ನಮ್ಮ-ವರು
ನಮ್ಮ-ವಾ-ಸ-ನೆ-ಯನ್ನು
ನಮ್ಮಷ್ಟು
ನಮ್ಮಿಂದ
ನಮ್ಮಿಂ-ದಲೇ
ನಮ್ಮೆ-ದು-ರಿಗ
ನಮ್ಮೆ-ದು-ರಿಗೆ
ನಮ್ಮೆ-ಲ್ಲರ
ನಮ್ಮೆ-ಲ್ಲ-ರಿ-ಗಿಂತ
ನಮ್ಮೆ-ಲ್ಲ-ರಿಗೂ
ನಮ್ಮೆ-ಲ್ಲ-ರಿಗೆ
ನಮ್ಮೊಂ-ದಿಗೆ
ನಮ್ಮೊ-ಡನೆ
ನಮ್ಮೊ-ಳ-ಗಿ-ರುವ
ನಮ್ಮೊ-ಳಗೆ
ನಯ
ನಯನ
ನಯ-ನ-ಗಳೇ
ನಯೇತ್
ನರಃ
ನರಕ
ನರ-ಕ-ಕ್ಕಿಂತ
ನರ-ಕಕ್ಕೆ
ನರ-ಕದ
ನರ-ಕ-ದಲ್ಲಿ
ನರ-ಕ-ದಿಂದ
ನರ-ಕ-ಪ್ರಾಪ್ತಿ
ನರ-ಕ-ಯಾ-ತನೆ
ನರ-ಕ-ಯಾ-ತ-ನೆ-ಗಿಂತ
ನರ-ಕ-ಲೋ-ಕವೇ
ನರ-ಕ-ವ-ನ್ನುಂ-ಟು-ಮಾ-ಡು-ವುದು
ನರ-ಕ-ವ-ಲ್ಲದೆ
ನರ-ಕ-ವಾ-ಗು-ವುದು
ನರ-ಕ-ವಾಸ
ನರ-ಕವೂ
ನರ-ಕವೆ
ನರ-ಕ-ವೆಂದರೆ
ನರ-ಕ-ಸ-ದೃ-ಶ-ವಾ-ಗು-ವುದು
ನರ-ಕ-ಸ್ಯೇದಂ
ನರ-ಕಾ-ಯೈವ
ನರಕೇ
ನರ-ಕೇ-ಽಶುಚೌ
ನರ-ಗ-ಳಿಗೆ
ನರನ
ನರ-ನಾಗಿ
ನರ-ಪತಿ
ನರ-ಪಿಳ್ಳೆ
ನರ-ಪುಂ-ಗವಃ
ನರ-ಮ-ನು-ಷ್ಯರ
ನರ-ಮೃ-ಗ-ಗ-ಳಾ-ಗು-ತ್ತೇವೆ
ನರ-ರಲ್ಲಿ
ನರ-ರೂಪಿ
ನರ-ಲೋ-ಕ-ವೀ-ರರು
ನರ-ಲೋ-ಕ-ವೀರಾ
ನರ-ಳ-ಕೂ-ಡದು
ನರ-ಳ-ಬೇ-ಕಾ-ಗು-ತ್ತದೆ
ನರ-ಳ-ಬೇ-ಕಾ-ಗು-ವುದು
ನರ-ಳ-ಬೇ-ಕಾದ
ನರಳಿ
ನರ-ಳು-ತ್ತಾನೆ
ನರ-ಳು-ತ್ತಿದೆ
ನರ-ಳು-ತ್ತಿ-ದ್ದರೆ
ನರ-ಳು-ತ್ತಿ-ರಲಿ
ನರ-ಳು-ತ್ತಿರು
ನರ-ಳು-ತ್ತಿ-ರು-ವನು
ನರ-ಳು-ತ್ತಿ-ರು-ವರು
ನರ-ಳು-ತ್ತಿ-ರು-ವ-ವ-ನಿಗೆ
ನರ-ಳು-ತ್ತಿ-ರು-ವ-ವನು
ನರ-ಳು-ತ್ತಿ-ರು-ವ-ವರು
ನರ-ಳು-ತ್ತಿ-ರು-ವಾಗ
ನರ-ಳು-ತ್ತಿ-ರು-ವುವು
ನರ-ಳು-ತ್ತಿ-ರು-ವೆವು
ನರ-ಳು-ತ್ತಿಲ್ಲ
ನರ-ಳು-ತ್ತಿವೆ
ನರ-ಳು-ತ್ತೇವೆ
ನರ-ಳುವ
ನರ-ಳು-ವನು
ನರ-ಳು-ವರು
ನರ-ಳು-ವ-ವ-ನನ್ನು
ನರ-ಳು-ವ-ವನು
ನರ-ಳು-ವ-ವರು
ನರ-ಳು-ವಾಗ
ನರ-ಳು-ವು-ದಕ್ಕೆ
ನರ-ಳು-ವುದು
ನರ-ಳು-ವುದೂ
ನರ-ಳು-ವುದೇ
ನರ-ಳು-ವೆವು
ನರ-ಶ್ರೇ-ಷ್ಠ-ನಾದ
ನರ-ಸಿಂಹ
ನರ-ಸಿಂ-ಹ-ನಂತೆ
ನರಾ-ಣಾಂ
ನರಾ-ಧ-ಮರ
ನರಾ-ಧ-ಮ-ರನ್ನು
ನರಾ-ಧ-ಮ-ರಲ್ಲಿ
ನರಾ-ಧ-ಮರು
ನರಾ-ಧ-ಮಾಃ
ನರಾ-ಧ-ಮಾನ್
ನರಾ-ಧಿ-ಪಮ್
ನರಿ
ನರೇಂದ್ರ
ನರೇಂ-ದ್ರ-ನಾಥ
ನರೇಂ-ದ್ರ-ನಿಗೆ
ನರೇಂ-ದ್ರನು
ನರೈಃ
ನರೋ-ಪ-ರಾಣಿ
ನಲ್ಲ
ನಲ್ಲದ
ನಲ್ಲದೆ
ನಲ್ಲಿ
ನಲ್ಲಿ-ಗ-ಳಿಗೆ
ನಲ್ಲಿ-ಗ-ಳಿ-ಗೆಲ್ಲಾ
ನಲ್ಲಿ-ಟ್ಟು-ಕೊ-ಳ್ಳು-ತ್ತೇವೆ
ನಲ್ಲಿದೆ
ನಲ್ಲಿಯ
ನಲ್ಲಿಯೂ
ನಲ್ಲಿಯೇ
ನಲ್ಲಿ-ರುವ
ನವ-ದ್ವಾ-ರ-ಗ-ಳುಳ್ಳ
ನವ-ದ್ವಾರೇ
ನವ-ರಿಗೆ
ನವಾನಿ
ನವ್ಯ-ತೆ-ಯನ್ನು
ನಶಿಸಿ
ನಶ್ಯ
ನಶ್ಯಕ್ಕೆ
ನಶ್ಯತಿ
ನಶ್ಯತ್ಸು
ನಶ್ವರ
ನಶ್ವ-ರ-ತೆ-ಯನ್ನು
ನಶ್ವ-ರ-ವನ್ನು
ನಶ್ವ-ರ-ವಾ-ಗದೆ
ನಶ್ವ-ರ-ವಾದ
ನಷ್ಟ
ನಷ್ಟಃ
ನಷ್ಟಕ್ಕೂ
ನಷ್ಟ-ಗಳ
ನಷ್ಟ-ಗಳನ್ನು
ನಷ್ಟ-ಗಳು
ನಷ್ಟದ
ನಷ್ಟ-ದಿಂದ
ನಷ್ಟ-ವನ್ನು
ನಷ್ಟ-ವ-ನ್ನುಂ-ಟು-ಮಾ-ಡದೆ
ನಷ್ಟ-ವಲ್ಲ
ನಷ್ಟ-ವಾ-ಗ-ಬೇ-ಕಾ-ದರೆ
ನಷ್ಟ-ವಾ-ಗಲಿ
ನಷ್ಟ-ವಾಗಿ
ನಷ್ಟ-ವಾ-ಗಿ-ದೆಯೋ
ನಷ್ಟ-ವಾ-ಗು-ವು-ದಿಲ್ಲ
ನಷ್ಟ-ವಾ-ಗು-ವುದು
ನಷ್ಟ-ವಾ-ದರೂ
ನಷ್ಟ-ವಾ-ಯಿತು
ನಷ್ಟ-ವಿಲ್ಲ
ನಷ್ಟ-ವುಂ-ಟಾ-ಗು-ವುದು
ನಷ್ಟವೂ
ನಷ್ಟ-ವೆಂ-ಬು-ದಿಲ್ಲ
ನಷ್ಟವೇ
ನಷ್ಟಾ-ತ್ಮರೂ
ನಷ್ಟಾ-ತ್ಮಾ-ನೋ-ಽಲ್ಪ-ಬು-ದ್ಧಯಃ
ನಷ್ಟಾ-ನ-ಚೇ-ತಸಃ
ನಷ್ಟೇ
ನಷ್ಟೋ
ನಸು-ನ-ಗುತ್ತ
ನಾಂತಂ
ನಾಂತೋ
ನಾಂತೋಽಸ್ತಿ
ನಾಕೃ-ತೇ-ನೇಹ
ನಾಗ
ನಾಗ-ಗಳಲ್ಲಿ
ನಾಗ-ರ-ಹಾ-ವಿನ
ನಾಗ-ರ-ಹಾವು
ನಾಗ-ರಿ-ಕತೆ
ನಾಗ-ರಿ-ಕ-ತೆಗೆ
ನಾಗಾ-ನಾಂ
ನಾಗಾ-ಲೋ-ಟ-ದಲ್ಲಿ
ನಾಗಿ
ನಾಗಿದ್ದ
ನಾಗಿ-ದ್ದರೆ
ನಾಗಿ-ದ್ದಾನೆ
ನಾಗಿ-ರ-ಬೇಕು
ನಾಗಿ-ರ-ಲೇ-ಬೇ-ಕೆಂಬ
ನಾಗಿ-ರು-ವನು
ನಾಗಿ-ರು-ವನೆ
ನಾಗಿ-ರು-ವನೊ
ನಾಗಿ-ರು-ವ-ವನು
ನಾಗಿ-ರು-ವು-ದ-ರಿಂದ
ನಾಗು-ತ್ತಾನೆ
ನಾಗು-ವನು
ನಾಗು-ವು-ದಿಲ್ಲ
ನಾಗು-ವೆನು
ನಾಚ-ಬೇ-ಕಾ-ಗಿಲ್ಲ
ನಾಚ-ಬೇ-ಕಾ-ಗು-ವುದು
ನಾಚ-ಬೇಕೋ
ನಾಚಿ
ನಾಚಿಕೆ
ನಾಚಿ-ಕೆ-ಯಿಂದ
ನಾಚಿ-ಕೆ-ಯಿ-ಲ್ಲದೆ
ನಾಚಿ-ದಾಗ
ನಾಚು-ತ್ತಾನೆ
ನಾಚು-ವನು
ನಾಚು-ವು-ದಿಲ್ಲ
ನಾಚು-ವುದು
ನಾಜೂ-ಕಿಲ್ಲ
ನಾಟಕ
ನಾಟ-ಕ-ದಲ್ಲಿ
ನಾಟ-ಕ-ವನ್ನು
ನಾಟ-ಕ-ವೊಂದು
ನಾಟ-ಕೀಯ
ನಾಟ-ಕೀ-ಯ-ವಾಗಿ
ನಾಟಿ-ರು-ವವೊ
ನಾಟು-ತ್ತದೆ
ನಾಟು-ವಂ-ತಹ
ನಾಟು-ವಂತೆ
ನಾಡ-ಲೇ-ಬೇಕು
ನಾಡಿ
ನಾಡಿ-ನಾ-ಡಿ-ಯಲ್ಲಿ
ನಾಡಿ-ಯನ್ನು
ನಾಡೀ-ಶಾಸ್ತ್ರ
ನಾಡು-ತ್ತಾನೆ
ನಾಣ್ಯದ
ನಾಣ್ಯ-ವೊಂದು
ನಾತ-ಪ-ಸ್ಕಾಯ
ನಾತಿ
ನಾತಿ-ಮಾ-ನಿತಾ
ನಾತ್ಮಾ-ನ-ಮ-ವ-ಸಾ-ದ-ಯೇತ್
ನಾತ್ಯ-ಶ್ನ-ತಸ್ತು
ನಾತ್ಯು-ಚ್ಛ್ರಿತಂ
ನಾತ್ರ
ನಾದ
ನಾದತ್ತೇ
ನಾದರೂ
ನಾದರೆ
ನಾದರೊ
ನಾದರೋ
ನಾದ-ವ-ನಿಗೆ
ನಾದ-ವನು
ನಾಧಿಕಂ
ನಾನಂತೂ
ನಾನ-ದನ್ನು
ನಾನನ್ನು
ನಾನಲ್ಲ
ನಾನ-ಲ್ಲದೇ
ನಾನಲ್ಲಿ
ನಾನ-ವಾ-ಪ್ತ-ಮ-ವಾ-ಪ್ತವ್ಯಂ
ನಾನಾ
ನಾನಾ-ಖ್ಯಾ-ನ-ಕ-ಕೇ-ಸರಂ
ನಾನಾಗಿ
ನಾನಾ-ಗಿಯೇ
ನಾನಾತ್ವ
ನಾನಾ-ತ್ವಕ್ಕೆ
ನಾನಾ-ತ್ವದ
ನಾನಾ-ತ್ವ-ದ-ಲ್ಲಿಯೇ
ನಾನಾ-ತ್ವ-ವನ್ನು
ನಾನಾ-ಭಾ-ವಾನ್
ನಾನಾರು
ನಾನಾ-ವ-ರ್ಣಾ-ಕೃ-ತೀನಿ
ನಾನಾ-ವಿ-ಧಾನಿ
ನಾನಾ-ಶ-ಸ್ತ್ರ-ಪ್ರ-ಹ-ರ-ಣಾಃ
ನಾನಿದ್ದ
ನಾನಿ-ದ್ದರೆ
ನಾನಿದ್ದೆ
ನಾನಿ-ದ್ದೆಡೆ
ನಾನಿ-ದ್ದೇನೆ
ನಾನಿ-ರ-ಬೇ-ಕಾ-ದರೆ
ನಾನಿ-ರ-ಬೇಕು
ನಾನಿ-ರುವ
ನಾನಿ-ಲ್ಲಿ-ರುವೆ
ನಾನಿ-ಲ್ಲಿ-ರು-ವೆನು
ನಾನು
ನಾನು-ಣ್ಣ-ಬೇಕು
ನಾನು-ತಿ-ಷ್ಠಂತಿ
ನಾನು-ಪ-ಶ್ಯಂತಿ
ನಾನು-ವ-ರ್ತ-ಯ-ತೀಹ
ನಾನು-ಶೋ-ಚಂತಿ
ನಾನು-ಶೋ-ಚಿ-ತು-ಮ-ರ್ಹಸಿ
ನಾನು-ಷ-ಜ್ಜತೇ
ನಾನೆ
ನಾನೆಂದು
ನಾನೆಂಬ
ನಾನೆಂ-ಬು-ದ-ಕ್ಕಿಂತ
ನಾನೆಂ-ಬುದು
ನಾನೆಂ-ಬುದೇ
ನಾನೆಷ್ಟು
ನಾನೇ
ನಾನೇಕೆ
ನಾನೇ-ನಾ-ದರೂ
ನಾನೇನು
ನಾನೇನೂ
ನಾನೇನೊ
ನಾನೊಂದು
ನಾನೊಬ್ಬ
ನಾನೋ
ನಾನ್ಯಂ
ನಾನ್ಯ-ಗಾ-ಮಿನಾ
ನಾನ್ಯತ್
ನಾನ್ಯ-ದ-ಸ್ತೀತಿ
ನಾನ್ಸ್ಟಾಪ್
ನಾಪಿ
ನಾಪ್ನು-ವಂತಿ
ನಾಭ-ಕ್ತಾಯ
ನಾಭಾವೋ
ನಾಭಿ
ನಾಭಿ-ಜಾ-ನಾತಿ
ನಾಭಿ-ನಂ-ದತಿ
ನಾಭಿ-ಯಂತೆ
ನಾಭಿ-ಯಲ್ಲೇ
ನಾಭಿ-ಯಿಂದ
ನಾಭಿ-ಯಿಂ-ದಲೇ
ನಾಮ
ನಾಮ-ಕ-ರಣ
ನಾಮ-ಗಳನ್ನು
ನಾಮದ
ನಾಮ-ದಲ್ಲಿ
ನಾಮ-ದಲ್ಲೆ
ನಾಮ-ಯಜ್ಞೆ
ನಾಮ-ರೂಪ
ನಾಮ-ರೂ-ಪಕ್ಕೆ
ನಾಮ-ರೂ-ಪ-ಗಳ
ನಾಮ-ರೂ-ಪ-ಗಳನ್ನು
ನಾಮ-ರೂ-ಪ-ಗ-ಳನ್ನೆ
ನಾಮ-ರೂ-ಪ-ಗಳನ್ನೆಲ್ಲ
ನಾಮ-ರೂ-ಪ-ಗ-ಳಾಚೆ
ನಾಮ-ರೂ-ಪ-ಗ-ಳಾ-ದರೂ
ನಾಮ-ರೂ-ಪ-ಗಳಿಂದ
ನಾಮ-ರೂ-ಪ-ಗ-ಳಿಗೆ
ನಾಮ-ರೂ-ಪ-ಗ-ಳಿ-ದ್ದರೂ
ನಾಮ-ರೂ-ಪ-ಗ-ಳಿ-ರ-ಬೇಕು
ನಾಮ-ರೂ-ಪ-ಗ-ಳಿಲ್ಲ
ನಾಮ-ರೂ-ಪ-ಗ-ಳಿ-ಲ್ಲದೆ
ನಾಮ-ರೂ-ಪ-ಗಳು
ನಾಮ-ರೂ-ಪ-ಗಳೂ
ನಾಮ-ರೂ-ಪ-ಗ-ಳೆಲ್ಲ
ನಾಮ-ರೂ-ಪ-ಗು-ಣ-ವಿ-ಲ್ಲದೆ
ನಾಮ-ರೂ-ಪದ
ನಾಮ-ರೂ-ಪ-ದಿಂದ
ನಾಮ-ರೂ-ಪ-ದೊ-ಡನೆ
ನಾಮ-ರೂ-ಪನ್ನು
ನಾಮ-ರೂ-ಪ-ವನ್ನು
ನಾಮ-ರೂ-ಪ-ವನ್ನೆ
ನಾಮ-ರೂ-ಪ-ವನ್ನೇ
ನಾಮ-ರೂ-ಪ-ವುಳ್ಳ
ನಾಮ-ರೂಪು
ನಾಮ-ರೂ-ಪು-ಗಳ
ನಾಮ-ರೂ-ಪು-ಗ-ಳಿಂ-ದಲೇ
ನಾಮ-ವನ್ನು
ನಾಮ-ವೆ-ನ್ನು-ವನು
ನಾಮಾ-ರ್ಚ-ನೆಯೋ
ನಾಮಾ-ವ-ಶೇ-ಷ-ವಾಗಿ
ನಾಮಿ
ನಾಮುತ್ರ
ನಾಯಂ
ನಾಯ-ಕ-ನಾದ
ನಾಯ-ಕ-ನೆಂದು
ನಾಯಕಾ
ನಾಯನ್ನು
ನಾಯಿ
ನಾಯಿ-ಗಳ
ನಾಯಿ-ಗಳನ್ನು
ನಾಯಿ-ಗಳು
ನಾಯಿಗೆ
ನಾಯಿ-ನ-ರಿ-ಗಳ
ನಾಯಿ-ಬಾ-ಲ-ದಂತೆ
ನಾಯಿಯ
ನಾಯಿ-ಯಂತೆ
ನಾಯಿ-ಯನ್ನು
ನಾಯಿ-ಯಲ್ಲಿ
ನಾರದ
ನಾರದಃ
ನಾರ-ದನು
ನಾರ-ದ-ರನ್ನು
ನಾರ-ದ-ರಿಗೆ
ನಾರ-ದರು
ನಾರ-ದರೆ
ನಾರಾ-ಯಣ
ನಾರಾ-ಯ-ಣನ
ನಾರಾ-ಯ-ಣ-ನಿಗೆ
ನಾರಾ-ಯ-ಣನೆ
ನಾರಾ-ಯ-ಣನೇ
ನಾರಾ-ಯ-ಣೇನ
ನಾರಿ
ನಾರಿ-ಯರ
ನಾರಿ-ಯ-ರಲ್ಲಿ
ನಾರಿ-ಯ-ರ-ಲ್ಲಿ-ರುವ
ನಾರೀ-ಣಾಂ
ನಾರು
ನಾರು-ತ್ತಿ-ರು-ವಾಗ
ನಾರು-ವುದು
ನಾಲಗೆ
ನಾಲ-ಗೆಯ
ನಾಲ-ಗೆ-ಯನ್ನು
ನಾಲ-ಗೆ-ಯಲ್ಲಿ
ನಾಲ-ಗೆ-ಯಿಂದ
ನಾಲಿಗೆ
ನಾಲಿ-ಗೆಗೆ
ನಾಲಿ-ಗೆ-ಯಿಂದ
ನಾಲೆ
ನಾಲೆ-ಯಲ್ಲಿ
ನಾಲ್ಕನೆ
ನಾಲ್ಕ-ನೆಯ
ನಾಲ್ಕ-ನೆ-ಯದೆ
ನಾಲ್ಕ-ನೆ-ಯ-ವನೆ
ನಾಲ್ಕ-ನೆ-ಯ-ವನೇ
ನಾಲ್ಕ-ನೆ-ಯ-ವರೇ
ನಾಲ್ಕನೇ
ನಾಲ್ಕು
ನಾಲ್ಕೇ
ನಾಲ್ಕೈದು
ನಾಳೆ
ನಾಳೆಯ
ನಾಳೆ-ಯಂತೂ
ನಾವಾಗಿ
ನಾವಾ-ಗಿ-ರು-ವುದು
ನಾವಾ-ದರೂ
ನಾವಾರು
ನಾವಿಕ
ನಾವಿ-ಕ-ನಿಗೆ
ನಾವಿ-ಚ್ಛಿ-ಸಿ-ದರೆ
ನಾವಿದ್ದು
ನಾವಿನ್ನು
ನಾವಿ-ಬ್ಬರೂ
ನಾವಿ-ರುವ
ನಾವಿ-ರು-ವುದೇ
ನಾವಿಲ್ಲ
ನಾವಿಲ್ಲಿ
ನಾವಿ-ಲ್ಲಿ-ರು-ವಾಗ
ನಾವೀಗ
ನಾವು
ನಾವು-ಗ-ಳೆಲ್ಲ
ನಾವೂ
ನಾವೆ
ನಾವೆಂದರೆ
ನಾವೆಂಬ
ನಾವೆಲ್ಲ
ನಾವೆ-ಲ್ಲ-ಕು-ಡಿ-ಯುವ
ನಾವೆಲ್ಲಾ
ನಾವೆ-ಲ್ಲಿಗೂ
ನಾವೆ-ಲ್ಲಿ-ರು-ವೆವು
ನಾವೆಷ್ಟು
ನಾವೇ
ನಾವೇಕೆ
ನಾವೇ-ನಾ-ದರೂ
ನಾವೇನು
ನಾವೇನೂ
ನಾವೇನೊ
ನಾವೇನೋ
ನಾವೊಂದು
ನಾಶ
ನಾಶ-ಕ-ವಾ-ಗು-ವುದು
ನಾಶ-ಕ್ಕಾಗಿ
ನಾಶಕ್ಕೆ
ನಾಶ-ಗೊ-ಳಿ-ಸ-ಲೆಂದೇ
ನಾಶದ
ನಾಶ-ನ-ಮಾ-ತ್ಮನಃ
ನಾಶ-ಮಾಡ
ನಾಶ-ಮಾ-ಡದೆ
ನಾಶ-ಮಾ-ಡ-ಬೇ-ಕಾ-ಗಿದೆ
ನಾಶ-ಮಾ-ಡ-ಬೇಕು
ನಾಶ-ಮಾ-ಡಲು
ನಾಶ-ಮಾ-ಡಲೂ
ನಾಶ-ಮಾಡಿ
ನಾಶ-ಮಾ-ಡಿ-ಕೊಂ-ಡ-ವನೂ
ನಾಶ-ಮಾ-ಡಿ-ಕೊ-ಳ್ಳುವ
ನಾಶ-ಮಾ-ಡಿ-ಕೊ-ಳ್ಳು-ವು-ದಿಲ್ಲ
ನಾಶ-ಮಾ-ಡಿದ
ನಾಶ-ಮಾ-ಡಿ-ರು-ವನು
ನಾಶ-ಮಾಡು
ನಾಶ-ಮಾ-ಡುತ್ತ
ನಾಶ-ಮಾ-ಡು-ತ್ತಾನೆ
ನಾಶ-ಮಾ-ಡು-ತ್ತೇನೆ
ನಾಶ-ಮಾ-ಡು-ತ್ತೇವೆ
ನಾಶ-ಮಾ-ಡುವ
ನಾಶ-ಮಾ-ಡು-ವಾಗ
ನಾಶ-ಮಾ-ಡು-ವು-ದಕ್ಕೂ
ನಾಶ-ಮಾ-ಡು-ವು-ದಕ್ಕೆ
ನಾಶ-ಮಾ-ಡು-ವುದು
ನಾಶ-ಯಾ-ಮ್ಯಾ-ತ್ಮ-ಭಾ-ವಸ್ಥೋ
ನಾಶ-ವನ್ನು
ನಾಶ-ವಾ-ಗದ
ನಾಶ-ವಾ-ಗ-ದ-ವನು
ನಾಶ-ವಾ-ಗ-ದ-ವನೆ
ನಾಶ-ವಾ-ಗ-ದುದು
ನಾಶ-ವಾ-ಗದೆ
ನಾಶ-ವಾ-ಗ-ಬ-ಹುದು
ನಾಶ-ವಾ-ಗ-ಬೇ-ಕಾ-ದರೆ
ನಾಶ-ವಾ-ಗಲಿ
ನಾಶ-ವಾ-ಗ-ಲಿಲ್ಲ
ನಾಶ-ವಾ-ಗಲೇ
ನಾಶ-ವಾ-ಗ-ಲೇ-ಬೇಕು
ನಾಶ-ವಾಗಿ
ನಾಶ-ವಾ-ಗಿದೆ
ನಾಶ-ವಾ-ಗಿ-ದೆಯೊ
ನಾಶ-ವಾ-ಗಿಯೇ
ನಾಶ-ವಾ-ಗಿ-ರ-ಲಿಲ್ಲ
ನಾಶ-ವಾ-ಗಿ-ರು-ವು-ದಿಲ್ಲ
ನಾಶ-ವಾ-ಗಿಲ್ಲ
ನಾಶ-ವಾಗು
ನಾಶ-ವಾ-ಗು-ತ್ತದೆ
ನಾಶ-ವಾ-ಗು-ತ್ತವೆ
ನಾಶ-ವಾ-ಗು-ತ್ತ-ವೆಯೊ
ನಾಶ-ವಾ-ಗು-ತ್ತಾನೆ
ನಾಶ-ವಾ-ಗು-ತ್ತಾರೆ
ನಾಶ-ವಾ-ಗು-ತ್ತಿವೆ
ನಾಶ-ವಾ-ಗು-ತ್ತೀಯೆ
ನಾಶ-ವಾ-ಗು-ತ್ತೇವೆ
ನಾಶ-ವಾ-ಗುವ
ನಾಶ-ವಾ-ಗು-ವನು
ನಾಶ-ವಾ-ಗು-ವರು
ನಾಶ-ವಾ-ಗು-ವರೋ
ನಾಶ-ವಾ-ಗು-ವ-ವರು
ನಾಶ-ವಾ-ಗು-ವು-ದಂತೂ
ನಾಶ-ವಾ-ಗು-ವು-ದಾ-ದರೂ
ನಾಶ-ವಾ-ಗು-ವು-ದಿಲ್ಲ
ನಾಶ-ವಾ-ಗು-ವು-ದಿ-ಲ್ಲ-ವೆಂದು
ನಾಶ-ವಾ-ಗು-ವು-ದಿ-ಲ್ಲವೇ
ನಾಶ-ವಾ-ಗು-ವು-ದಿ-ಲ್ಲವೋ
ನಾಶ-ವಾ-ಗು-ವುದು
ನಾಶ-ವಾ-ಗು-ವು-ದೆಂದು
ನಾಶ-ವಾ-ಗು-ವುದೊ
ನಾಶ-ವಾ-ಗು-ವುವು
ನಾಶ-ವಾದ
ನಾಶ-ವಾ-ದಂತೆ
ನಾಶ-ವಾ-ದ-ಮೇಲೂ
ನಾಶ-ವಾ-ದ-ಮೇಲೆ
ನಾಶ-ವಾ-ದರೂ
ನಾಶ-ವಾ-ದರೆ
ನಾಶ-ವಾ-ಯಿತು
ನಾಶ-ವಾ-ಯಿತೆ
ನಾಶ-ವಾ-ಯಿತೇ
ನಾಶ-ವಿಲ್ಲ
ನಾಶ-ವಿ-ಲ್ಲ-ದ-ವನು
ನಾಶ-ವು-ಳೃದ್ದು
ನಾಶ-ವು-ಳ್ಳದ್ದು
ನಾಶವೂ
ನಾಶ-ವೆಂದರೆ
ನಾಶ-ವೆಂ-ಬುದು
ನಾಶಾಯ
ನಾಶಿ-ತ-ಮಾ-ತ್ಮನಃ
ನಾಸಂ
ನಾಸತೋ
ನಾಸಾ-ಭ್ಯಂ-ತ-ರ-ಚಾ-ರಿಣೌ
ನಾಸಿ-ಕಾಗ್ರಂ
ನಾಸ್ತಿ
ನಾಸ್ತಿಕ
ನಾಸ್ತಿ-ಕ-ನಾ-ಗಿ-ರ-ಬ-ಹುದು
ನಾಸ್ತ್ಯಂತೋ
ನಾಸ್ತ್ಯತ್ರ
ನಾಹಂ
ನಾಹಂ-ಕೃತೋ
ನಿಂತ
ನಿಂತಂತೆ
ನಿಂತ-ಕಡೆ
ನಿಂತನು
ನಿಂತ-ಮೇಲೆ
ನಿಂತರು
ನಿಂತರೂ
ನಿಂತರೆ
ನಿಂತ-ಲ್ಲಿಯೋ
ನಿಂತ-ವನು
ನಿಂತ-ವರು
ನಿಂತಾಗ
ನಿಂತಿತು
ನಿಂತಿದೆ
ನಿಂತಿ-ದೆಯೋ
ನಿಂತಿ-ದ್ದರೆ
ನಿಂತಿರ
ನಿಂತಿ-ರ-ಬೇಕು
ನಿಂತಿ-ರಲಿ
ನಿಂತಿ-ರಲು
ನಿಂತಿರು
ನಿಂತಿ-ರುವ
ನಿಂತಿ-ರು-ವನು
ನಿಂತಿ-ರು-ವನೋ
ನಿಂತಿ-ರು-ವರು
ನಿಂತಿ-ರು-ವ-ವನು
ನಿಂತಿ-ರು-ವ-ವನೇ
ನಿಂತಿ-ರು-ವ-ವರು
ನಿಂತಿ-ರು-ವಾಗ
ನಿಂತಿ-ರು-ವು-ದಕ್ಕೆ
ನಿಂತಿ-ರು-ವುದು
ನಿಂತಿ-ರು-ವುದೆ
ನಿಂತಿ-ರು-ವುದೇ
ನಿಂತಿಲ್ಲ
ನಿಂತಿವೆ
ನಿಂತು
ನಿಂತು-ಕೊಂ-ಡರೂ
ನಿಂತು-ಕೊಂ-ಡರೆ
ನಿಂತು-ಕೊಂ-ಡಾಗ
ನಿಂತು-ಕೊಂ-ಡಿತು
ನಿಂತು-ಕೊಂ-ಡಿ-ರುವ
ನಿಂತು-ಕೊಂ-ಡಿ-ರು-ವುದು
ನಿಂತು-ಕೊಂಡು
ನಿಂತು-ಕೊ-ಳ್ಳ-ಬೇಕು
ನಿಂತು-ಕೊ-ಳ್ಳು-ವಂತೆ
ನಿಂತು-ಕೊ-ಳ್ಳು-ವರು
ನಿಂತು-ಕೊ-ಳ್ಳುವು
ನಿಂತು-ಕೊ-ಳ್ಳು-ವುದೊ
ನಿಂತು-ಹೋಗಿ
ನಿಂತು-ಹೋ-ದರೂ
ನಿಂತು-ಹೋ-ಯಿತೆ
ನಿಂತೊ-ಡ-ನೆಯೆ
ನಿಂದ
ನಿಂದಂ-ತ-ಸ್ತವ
ನಿಂದಲೂ
ನಿಂದಲೇ
ನಿಂದಾ
ನಿಂದಾದ
ನಿಂದಾ-ಸ್ತುತಿ
ನಿಂದಾ-ಸ್ತು-ತಿ-ಗಳನ್ನು
ನಿಂದಿ-ಸಿ-ಕೊ-ಳ್ಳು-ತ್ತೇ-ನೆಯೋ
ನಿಂದಿ-ಸುತ್ತ
ನಿಂದಿ-ಸು-ವಾಗ
ನಿಂದೆ
ನಿಂದೆ-ಗಳ
ನಿಂದೆ-ಗಳನ್ನು
ನಿಂದೆ-ಗಳಲ್ಲಿ
ನಿಂದೆ-ಗಿಂತ
ನಿಂದೆಯ
ನಿಂದೆ-ಯನ್ನು
ನಿಂದೆ-ಯಿಲ್ಲ
ನಿಂಬಾರ್ಕ
ನಿಃಶ್ರೇ-ಯ-ಸ-ಕ-ರಾ-ವುಭೌ
ನಿಃಸ್ಪೃಹಃ
ನಿಃಸ್ವಾರ್ಥ
ನಿಃಸ್ವಾ-ರ್ಥತೆ
ನಿಃಸ್ವಾ-ರ್ಥ-ತೆ-ಯಿಂದ
ನಿಃಸ್ವಾ-ರ್ಥ-ದೃ-ಷ್ಟಿ-ಯಿಂದ
ನಿಃಸ್ವಾ-ರ್ಥ-ನಾಗಿ
ನಿಃಸ್ವಾ-ರ್ಥ-ನಾ-ಗುತ್ತ
ನಿಕಟ
ನಿಕ-ಟ-ತೆ-ಯಿಂದ
ನಿಕ-ಟ-ವಾ-ಗಿದೆ
ನಿಕ-ಟ-ವಾದ
ನಿಕ-ಟ-ವಾ-ದುದು
ನಿಕೃ-ಷ್ಚ-ವಾಗಿ
ನಿಕೃಷ್ಟ
ನಿಕೃ-ಷ್ಟರು
ನಿಕೃ-ಷ್ಟ-ವಾಗಿ
ನಿಕೃ-ಷ್ಟ-ವಾದ
ನಿಗ-ಚ್ಛತಿ
ನಿಗ-ರ್ವ-ವನ್ನು
ನಿಗ-ರ್ವಿ-ಯಾದ
ನಿಗಿಂತ
ನಿಗೂ
ನಿಗೃ-ಹೀ-ತಾನಿ
ನಿಗೃ-ಹ್ಣಾ-ಮ್ಯು-ತ್ಸೃ-ಜಾಮಿ
ನಿಗೆ
ನಿಗ್ರಹ
ನಿಗ್ರಹಂ
ನಿಗ್ರಹಃ
ನಿಗ್ರ-ಹ-ದಂತೆ
ನಿಗ್ರ-ಹ-ದೊ-ಡನೆ
ನಿಗ್ರಹಿ
ನಿಗ್ರ-ಹಿ-ಸದ
ನಿಗ್ರ-ಹಿ-ಸ-ದ-ವ-ನಿಗೆ
ನಿಗ್ರ-ಹಿ-ಸ-ದ-ವನು
ನಿಗ್ರ-ಹಿ-ಸ-ದಿ-ದ್ದರೆ
ನಿಗ್ರ-ಹಿ-ಸದೆ
ನಿಗ್ರ-ಹಿ-ಸ-ಬ-ಹುದು
ನಿಗ್ರ-ಹಿ-ಸ-ಬೇ-ಕಾ-ದರೆ
ನಿಗ್ರ-ಹಿ-ಸ-ಬೇಕು
ನಿಗ್ರ-ಹಿ-ಸ-ಲಾ-ರದು
ನಿಗ್ರ-ಹಿ-ಸಲು
ನಿಗ್ರ-ಹಿ-ಸಲೂ
ನಿಗ್ರ-ಹಿ-ಸ-ಲ್ಪಟ್ಟ
ನಿಗ್ರ-ಹಿಸಿ
ನಿಗ್ರ-ಹಿ-ಸಿದ
ನಿಗ್ರ-ಹಿ-ಸಿ-ದರೆ
ನಿಗ್ರ-ಹಿ-ಸಿ-ದ-ವನ
ನಿಗ್ರ-ಹಿ-ಸಿ-ದ-ವನು
ನಿಗ್ರ-ಹಿ-ಸಿ-ದ-ವರು
ನಿಗ್ರ-ಹಿ-ಸಿ-ದಾಗ
ನಿಗ್ರ-ಹಿ-ಸಿ-ದು-ದರ
ನಿಗ್ರ-ಹಿ-ಸಿ-ದ್ದರೆ
ನಿಗ್ರ-ಹಿ-ಸಿ-ದ್ದೇನೆ
ನಿಗ್ರ-ಹಿ-ಸಿ-ರ-ಬೇಕು
ನಿಗ್ರ-ಹಿ-ಸಿ-ರು-ವನು
ನಿಗ್ರ-ಹಿ-ಸಿ-ರು-ವ-ನು-ಅ-ವನು
ನಿಗ್ರ-ಹಿ-ಸಿ-ರು-ವರು
ನಿಗ್ರ-ಹಿ-ಸಿ-ರು-ವೆವೊ
ನಿಗ್ರ-ಹಿ-ಸಿಲ್ಲ
ನಿಗ್ರ-ಹಿ-ಸಿ-ಲ್ಲವೊ
ನಿಗ್ರ-ಹಿಸು
ನಿಗ್ರ-ಹಿ-ಸು-ತ್ತಾನೆ
ನಿಗ್ರ-ಹಿ-ಸು-ತ್ತಾ-ನೆಯೊ
ನಿಗ್ರ-ಹಿ-ಸು-ತ್ತೇನೆ
ನಿಗ್ರ-ಹಿ-ಸುವ
ನಿಗ್ರ-ಹಿ-ಸು-ವನು
ನಿಗ್ರ-ಹಿ-ಸು-ವನೊ
ನಿಗ್ರ-ಹಿ-ಸು-ವನೋ
ನಿಗ್ರ-ಹಿ-ಸು-ವಷ್ಟು
ನಿಗ್ರ-ಹಿ-ಸು-ವು-ದ-ಕ್ಕಾಗಿ
ನಿಗ್ರ-ಹಿ-ಸು-ವು-ದ-ಕ್ಕಿಂತ
ನಿಗ್ರ-ಹಿ-ಸು-ವು-ದಕ್ಕೆ
ನಿಗ್ರ-ಹಿ-ಸು-ವುದನ್ನು
ನಿಗ್ರ-ಹಿ-ಸು-ವು-ದರ
ನಿಗ್ರ-ಹಿ-ಸು-ವು-ದ-ರಲ್ಲಿ
ನಿಗ್ರ-ಹಿ-ಸು-ವು-ದ-ರಿಂದ
ನಿಗ್ರ-ಹಿ-ಸು-ವು-ದಿಲ್ಲ
ನಿಗ್ರ-ಹಿ-ಸು-ವುದು
ನಿಗ್ರ-ಹಿ-ಸು-ವೆವೊ
ನಿಜ
ನಿಜ-ದಂತೆ
ನಿಜ-ವಾಗಿ
ನಿಜ-ವಾ-ಗಿ-ದ್ದರೆ
ನಿಜ-ವಾ-ಗಿಯೂ
ನಿಜ-ವಾ-ಗಿ-ರು-ವುದು
ನಿಜ-ವಾ-ಗು-ವುದು
ನಿಜ-ವಾದ
ನಿಜ-ವಾ-ದರೂ
ನಿಜ-ವಾ-ದರೆ
ನಿಜ-ವೆ-ಎಂದು
ನಿಜವೇ
ನಿಜ-ಸ್ವ-ರೂ-ಪ-ವನ್ನು
ನಿಜಾಂಶ
ನಿಟ್ಟು-ಸಿ-ರು-ಬಿ-ಡು-ವುದು
ನಿತು
ನಿತ್ಯ
ನಿತ್ಯಂ
ನಿತ್ಯಃ
ನಿತ್ಯ-ಅ-ನಿ-ತ್ಯ-ವ-ಸ್ತು-ವಿ-ವೇ-ಚನೆ
ನಿತ್ಯ-ಕ-ರ್ಮದ
ನಿತ್ಯ-ಕ-ರ್ಮ-ವನ್ನು
ನಿತ್ಯ-ಗೆ-ಳೆ-ಯ-ನಾಗಿ
ನಿತ್ಯ-ಜಾತಂ
ನಿತ್ಯ-ಜೀ-ವ-ನ-ದಲ್ಲಿ
ನಿತ್ಯ-ತೃ-ಪ್ತ-ನಾಗಿ
ನಿತ್ಯ-ತೃಪ್ತೋ
ನಿತ್ಯ-ದಾ-ಸರು
ನಿತ್ಯ-ಪ್ರೇ-ಕ್ಷಕ
ನಿತ್ಯ-ಭ-ಕ್ತನೂ
ನಿತ್ಯ-ಮ-ನು-ತಿ-ಷ್ಠಂತಿ
ನಿತ್ಯ-ಮ-ವ-ಧ್ಯೋಽಯಂ
ನಿತ್ಯ-ಯುಕ್ತ
ನಿತ್ಯ-ಯು-ಕ್ತ-ನಾದ
ನಿತ್ಯ-ಯು-ಕ್ತನೂ
ನಿತ್ಯ-ಯು-ಕ್ತ-ರಾಗಿ
ನಿತ್ಯ-ಯು-ಕ್ತ-ರಾ-ದ-ವರ
ನಿತ್ಯ-ಯು-ಕ್ತರು
ನಿತ್ಯ-ಯು-ಕ್ತರೂ
ನಿತ್ಯ-ಯು-ಕ್ತಸ್ಯ
ನಿತ್ಯ-ಯುಕ್ತಾ
ನಿತ್ಯ-ವನ್ನು
ನಿತ್ಯ-ವಾದ
ನಿತ್ಯವೂ
ನಿತ್ಯ-ವೈರಿ
ನಿತ್ಯ-ವೈ-ರಿಣಾ
ನಿತ್ಯ-ವೈ-ರಿ-ಯಾ-ಗಿ-ರುವ
ನಿತ್ಯವೊ
ನಿತ್ಯಶಃ
ನಿತ್ಯ-ಶಾಸ್ತ್ರ
ನಿತ್ಯ-ಸಂನ್ಯಾಸಿ
ನಿತ್ಯ-ಸಂ-ನ್ಯಾಸೀ
ನಿತ್ಯ-ಸ-ತ್ತ್ವಸ್ಥೋ
ನಿತ್ಯ-ಸ-ತ್ಯದ
ನಿತ್ಯ-ಸಾ-ಕ್ಷಿ-ಯಾ-ಗಿದೆ
ನಿತ್ಯ-ಸುಖಿ
ನಿತ್ಯ-ಸ್ಯೋ-ಕ್ತಾಃ
ನಿತ್ಯಾ-ಭಿ-ಯು-ಕ್ತಾ-ನಾಂ
ನಿದ-ಧಾ-ಮ್ಯ-ಹಮ್
ನಿದ-ರ್ಶನ
ನಿದ-ರ್ಶ-ನ-ಗಳನ್ನು
ನಿದ-ರ್ಶ-ನ-ಗಳು
ನಿದ-ರ್ಶ-ನ-ವನ್ನು
ನಿದ-ರ್ಶಿ-ಸುವ
ನಿದ-ರ್ಶಿ-ಸು-ವು-ದಕ್ಕೆ
ನಿದಾ-ನಮ್
ನಿದ್ದಾನೆ
ನಿದ್ದೀಶ
ನಿದ್ದೆ
ನಿದ್ರಾ
ನಿದ್ರಾ-ದೇವಿ
ನಿದ್ರಾ-ಲ-ಸ್ಯ-ಪ್ರ-ಮಾ-ದೋತ್ಥಂ
ನಿದ್ರಾ-ಹಾ-ರ-ಗಳನ್ನು
ನಿದ್ರಿ-ಸು-ತ್ತಲೇ
ನಿದ್ರಿ-ಸು-ತ್ತಾನೆ
ನಿದ್ರಿ-ಸು-ತ್ತಿತ್ತು
ನಿದ್ರಿ-ಸು-ತ್ತಿದೆ
ನಿದ್ರಿ-ಸು-ತ್ತಿದ್ದ
ನಿದ್ರಿ-ಸು-ತ್ತಿ-ರು-ವರೊ
ನಿದ್ರಿ-ಸು-ವನು
ನಿದ್ರಿ-ಸು-ವನೋ
ನಿದ್ರಿ-ಸು-ವಾ-ಗಲೂ
ನಿದ್ರೆ
ನಿದ್ರೆ-ಗಳನ್ನು
ನಿದ್ರೆಗೆ
ನಿದ್ರೆ-ಮಾ-ಡು-ತ್ತಾನೆ
ನಿದ್ರೆಯ
ನಿದ್ರೆ-ಯನ್ನು
ನಿದ್ರೆ-ಯಲ್ಲಿ
ನಿದ್ರೆ-ಯ-ಲ್ಲಿಯೂ
ನಿದ್ರೆ-ಯಿಂದ
ನಿದ್ರೆ-ಯಿಂ-ದಲೊ
ನಿದ್ರೆ-ಯೆ-ನ್ನು-ವುದು
ನಿದ್ರೆಯೇ
ನಿಧನಂ
ನಿಧಾನ
ನಿಧಾನಂ
ನಿಧಾ-ನಮ್
ನಿಧಾ-ನ-ವಾ-ಗ-ಬ-ಹುದು
ನಿಧಾ-ನ-ವಾಗಿ
ನಿಧಾ-ನ-ವಾ-ಗಿ-ರು-ವುದು
ನಿಧಾ-ನ-ವಾ-ಗು-ವುದು
ನಿಧಿ
ನಿಧಿಯ
ನಿಧಿ-ಯನ್ನು
ನಿನ-ಗಿಂತ
ನಿನಗೂ
ನಿನಗೆ
ನಿನಾದ
ನಿನ್ನ
ನಿನ್ನಂ-ತಹ
ನಿನ್ನದು
ನಿನ್ನ-ನ್ನಿ-ಟ್ಟು-ಕೊಂಡಿ
ನಿನ್ನನ್ನು
ನಿನ್ನನ್ನೇ
ನಿನ್ನಲ್ಲಿ
ನಿನ್ನಿಂದ
ನಿನ್ನಿಂ-ದಲೇ
ನಿನ್ನೆ
ನಿನ್ನೆಯೇ
ನಿನ್ನೊಂ-ದಿಗೆ
ನಿನ್ನೊ-ಟ್ಟಿ-ಗಿ-ದ್ದಾಗ
ನಿಪುಣ
ನಿಪು-ಣ-ನಾ-ಗ-ಬ-ಹುದು
ನಿಪು-ಣ-ನಾ-ಗು-ವನು
ನಿಪು-ಣ-ರಾದ
ನಿಪು-ಣರು
ನಿಬಂ-ಧಾ-ಯಾ-ಸುರೀ
ನಿಬದ್ಧಃ
ನಿಬ-ಧ್ನಂತಿ
ನಿಬ-ಧ್ಯತೇ
ನಿಬೋಧ
ನಿಮ-ಗದು
ನಿಮಗೆ
ನಿಮಿತ್ತ
ನಿಮಿ-ತ್ತಕ್ಕೆ
ನಿಮಿ-ತ್ತ-ಕ್ಕೇನು
ನಿಮಿ-ತ್ತ-ಗ-ಳಷ್ಟೆ
ನಿಮಿ-ತ್ತ-ಗಳು
ನಿಮಿ-ತ್ತದ
ನಿಮಿ-ತ್ತ-ದಲ್ಲಿ
ನಿಮಿ-ತ್ತ-ದ-ಲ್ಲಿ-ರುವ
ನಿಮಿ-ತ್ತ-ದ-ಲ್ಲಿ-ರು-ವು-ದೆಲ್ಲ
ನಿಮಿ-ತ್ತ-ನಾಗು
ನಿಮಿ-ತ್ತ-ನಾದ
ನಿಮಿ-ತ್ತ-ಮಾ-ಡಿ-ಕೊಂಡು
ನಿಮಿ-ತ್ತ-ಮಾತ್ರ
ನಿಮಿ-ತ್ತ-ಮಾತ್ರಂ
ನಿಮಿ-ತ್ತ-ರೂ-ಪ-ನಾಗು
ನಿಮಿ-ತ್ತ-ವ-ನ್ನಾಗಿ
ನಿಮಿ-ತ್ತ-ವನ್ನು
ನಿಮಿ-ತ್ತ-ವಾ-ಗ-ಬೇ-ಕಾ-ದರೆ
ನಿಮಿ-ತ್ತ-ವಾ-ಗ-ಬೇಕು
ನಿಮಿ-ತ್ತ-ವಾಗಿ
ನಿಮಿ-ತ್ತ-ವಾ-ಗಿ-ಟ್ಟು-ಕೊಂಡು
ನಿಮಿ-ತ್ತ-ವಾ-ಗಿ-ರಲು
ನಿಮಿ-ತ್ತ-ವಾ-ಗಿ-ರುವ
ನಿಮಿ-ತ್ತ-ವಾ-ಗಿ-ರು-ವನು
ನಿಮಿ-ತ್ತ-ವಾಗು
ನಿಮಿ-ತ್ತ-ವಾ-ಗು-ತ್ತಾನೆ
ನಿಮಿ-ತ್ತ-ವಾ-ಗುವ
ನಿಮಿ-ತ್ತ-ವಾ-ಗು-ವನು
ನಿಮಿ-ತ್ತ-ವಾ-ಗು-ವು-ದಕ್ಕೆ
ನಿಮಿ-ತ್ತವೆ
ನಿಮಿ-ತ್ತ-ವೆಲ್ಲ
ನಿಮಿ-ತ್ತವೋ
ನಿಮಿ-ತ್ತಾ-ತೀ-ತ-ವಾ-ಗಿ-ರು-ವುದು
ನಿಮಿ-ತ್ತಾನಿ
ನಿಮಿರಿ
ನಿಮಿ-ಷಕ್ಕೆ
ನಿಮಿ-ಷ-ಗಳಲ್ಲಿ
ನಿಮಿ-ಷ-ಗಳು
ನಿಮಿ-ಷ-ದಂತೆ
ನಿಮಿ-ಷ-ವನ್ನು
ನಿಮಿ-ಷಾ-ರ್ಧ-ದಲ್ಲಿ
ನಿಮ್ಮ
ನಿಮ್ಮದು
ನಿಮ್ಮನ್ನು
ನಿಮ್ಮಲ್ಲಿ
ನಿಮ್ಮ-ಲ್ಲಿ-ರು-ವು-ದ-ನ್ನೆಲ್ಲ
ನಿಮ್ಮಿಂದ
ನಿಮ್ಮೊ-ಡನೆ
ನಿಯಂತ
ನಿಯತ
ನಿಯತಂ
ನಿಯ-ತ-ಕರ್ಮ
ನಿಯ-ತ-ಕ-ರ್ಮ-ವನ್ನು
ನಿಯ-ತ-ಕ-ರ್ಮ-ವಾ-ಗಿ-ರ-ಬ-ಹುದು
ನಿಯ-ತ-ಮಾ-ನಸಃ
ನಿಯ-ತ-ಮಾ-ನ-ಸ-ನಾಗಿ
ನಿಯ-ತ-ವಾದ
ನಿಯ-ತಸ್ಯ
ನಿಯ-ತಾಃ
ನಿಯ-ತಾ-ತ್ಮ-ಭಿಃ
ನಿಯ-ತಾ-ಹಾರ
ನಿಯ-ತಾ-ಹಾ-ರಾಃ
ನಿಯ-ತಾ-ಹಾ-ರಿ-ಗ-ಳಾಗಿ
ನಿಯಮ
ನಿಯ-ಮಕ್ಕೂ
ನಿಯ-ಮಕ್ಕೆ
ನಿಯ-ಮ-ಗಳ
ನಿಯ-ಮ-ಗ-ಳಂತೆ
ನಿಯ-ಮ-ಗಳನ್ನು
ನಿಯ-ಮ-ಗಳನ್ನೆಲ್ಲ
ನಿಯ-ಮ-ಗಳಲ್ಲಿ
ನಿಯ-ಮ-ಗ-ಳಿ-ಗಿಂತ
ನಿಯ-ಮ-ಗ-ಳಿ-ವೆಯೋ
ನಿಯ-ಮ-ಗಳು
ನಿಯ-ಮ-ಗಳೂ
ನಿಯ-ಮ-ಗ-ಳೆಲ್ಲಾ
ನಿಯ-ಮ-ಜಾ-ಲ-ವನ್ನು
ನಿಯ-ಮದ
ನಿಯ-ಮ-ದಂತೆ
ನಿಯ-ಮ-ದಷ್ಟೇ
ನಿಯ-ಮ-ದಿಂದ
ನಿಯ-ಮ-ಮಾ-ಸ್ಥಾಯ
ನಿಯ-ಮ-ವನ್ನು
ನಿಯ-ಮ-ವನ್ನೂ
ನಿಯ-ಮ-ವಿದೆ
ನಿಯ-ಮ-ವಿಲ್ಲ
ನಿಯ-ಮವೂ
ನಿಯ-ಮ-ವೆಂಬ
ನಿಯ-ಮವೇ
ನಿಯ-ಮ-ವೇನೊ
ನಿಯ-ಮಾ-ನು-ಸಾರ
ನಿಯ-ಮಾ-ವಳಿ
ನಿಯ-ಮಿತ
ನಿಯ-ಮಿ-ಸಿ-ರ-ಬ-ಹು-ದೆಂದು
ನಿಯ-ಮಿ-ಸು-ತ್ತಾರೆ
ನಿಯ-ಮಿ-ಸು-ತ್ತಿ-ರು-ವನು
ನಿಯಮ್ಯ
ನಿಯ-ಮ್ಯಾ-ರ-ಭ-ತೇ-ಽಜುನ
ನಿಯ-ಮ್ಯೈ-ತ-ದಾ-ತ್ಮ-ನ್ಯೇವ
ನಿಯಾ-ಮ-ಕ-ನಾ-ಗಿರು
ನಿಯಾ-ಮ-ಕ-ವೆಂದು
ನಿಯೋ-ಕ್ಷ್ಯತಿ
ನಿಯೋ-ಜ-ಯಸಿ
ನಿಯೋ-ಜಿತಃ
ನಿರಂ-ಕುಶ
ನಿರಂ-ತರ
ನಿರಂ-ತ-ರವೂ
ನಿರ-ಕ್ಷ-ರ-ಕು-ಕ್ಷಿ-ಯ-ಲ್ಲಿ-ರುವ
ನಿರ-ಕ್ಷ-ರ-ಸ್ಥ-ನಿಗೆ
ನಿರ-ಗ್ನಿರ್ನ
ನಿರತ
ನಿರ-ತ-ನಾ-ಗ-ದಿ-ದ್ದರೆ
ನಿರ-ತ-ನಾ-ಗ-ಬೇಕು
ನಿರ-ತ-ನಾಗಿ
ನಿರ-ತ-ನಾ-ಗಿ-ದ್ದರೂ
ನಿರ-ತ-ನಾ-ಗಿ-ದ್ದಾನೆ
ನಿರ-ತ-ನಾ-ಗಿರ
ನಿರ-ತ-ನಾ-ಗಿ-ರ-ಬ-ಹುದು
ನಿರ-ತ-ನಾ-ಗಿರು
ನಿರ-ತ-ನಾ-ಗಿ-ರು-ತ್ತಾನೆ
ನಿರ-ತ-ನಾ-ಗಿ-ರು-ವನು
ನಿರ-ತ-ನಾ-ಗಿ-ರು-ವ-ವನು
ನಿರ-ತ-ನಾ-ಗಿ-ರು-ವಾಗ
ನಿರ-ತ-ನಾ-ಗಿ-ರು-ವಾ-ಗಲೂ
ನಿರ-ತ-ನಾ-ಗಿ-ರು-ವೆನು
ನಿರ-ತ-ನಾಗು
ನಿರ-ತ-ನಾ-ಗು-ವಂತೆ
ನಿರ-ತ-ನಾದ
ನಿರ-ತ-ನಾ-ದ-ವನು
ನಿರ-ತ-ನಾ-ದಾಗ
ನಿರ-ತ-ರಾಗಿ
ನಿರ-ತ-ರಾ-ಗಿ-ರುವ
ನಿರ-ತ-ರಾ-ಗಿ-ರು-ವರೋ
ನಿರ-ತ-ರಾ-ಗಿ-ರು-ವ-ವರೂ
ನಿರ-ತ-ರಾ-ಗಿ-ರು-ವಾಗ
ನಿರ-ತ-ರಾ-ಗು-ತ್ತೇವೆ
ನಿರ-ತ-ರಾದ
ನಿರ-ತ-ರಾ-ದರೆ
ನಿರ-ತರು
ನಿರ-ತ-ವಾಗಿ
ನಿರ-ತ-ವಾ-ಗಿದೆ
ನಿರ-ತ-ವಾ-ಗಿರು
ನಿರ-ತ-ವಾ-ಗಿ-ರು-ವುದು
ನಿರ-ತ-ವಾ-ಗಿವೆ
ನಿರ-ತ-ವಾ-ಗು-ವಂತೆ
ನಿರ-ತ-ವಾದ
ನಿರ-ಪ-ರಾ-ಧಿ-ಯ-ದಲ್ಲ
ನಿರ-ರ್ಥ-ಕ-ವಾ-ಗು-ವು-ದಿಲ್ಲ
ನಿರ-ರ್ಥ-ಕ-ವಾ-ಗು-ವುದು
ನಿರ-ಹಂ-ಕಾರ
ನಿರ-ಹಂ-ಕಾರಃ
ನಿರ-ಹಂ-ಕಾ-ರ-ನಾಗಿ
ನಿರಾ-ಕ-ರಿ-ಸದೆ
ನಿರಾ-ಕ-ರಿ-ಸದೇ
ನಿರಾ-ಕ-ರಿಸಿ
ನಿರಾ-ಕ-ರಿ-ಸಿ-ರು-ವರು
ನಿರಾ-ಕ-ರಿ-ಸಿ-ರು-ವ-ವನು
ನಿರಾ-ಕ-ರಿಸು
ನಿರಾ-ಕ-ರಿ-ಸು-ತ್ತೇವೆ
ನಿರಾ-ಕ-ರಿ-ಸು-ವನು
ನಿರಾ-ಕ-ರಿ-ಸು-ವನೆ
ನಿರಾ-ಕ-ರಿ-ಸು-ವ-ವನು
ನಿರಾ-ಕ-ರಿ-ಸು-ವು-ದಿಲ್ಲ
ನಿರಾ-ಕಾರ
ನಿರಾ-ಕಾ-ರನೂ
ನಿರಾ-ಕಾ-ರ-ವನ್ನು
ನಿರಾ-ಕಾ-ರ-ವಾಗಿ
ನಿರಾ-ಕಾ-ರ-ವಾ-ದುದೂ
ನಿರಾ-ಡಂ-ಬ-ರ-ವಾಗಿ
ನಿರಾ-ತಂ-ಕ-ವಾಗಿ
ನಿರಾ-ನಂದ
ನಿರಾ-ಶ-ನಾಗಿ
ನಿರಾ-ಶನೂ
ನಿರಾ-ಶ-ರಾ-ಗದೆ
ನಿರಾ-ಶ-ರಾ-ಗ-ಬೇ-ಕಾ-ಗಿಲ್ಲ
ನಿರಾ-ಶ-ರಾಗಿ
ನಿರಾ-ಶಿ-ಯಾ-ಗಿ-ರ-ಬೇಕು
ನಿರಾಶೀ
ನಿರಾ-ಶೀ-ರ-ಪ-ರಿ-ಗ್ರಹಃ
ನಿರಾ-ಶೀ-ರ್ನಿ-ರ್ಮಮೋ
ನಿರಾ-ಶೀ-ರ್ಯ-ತ-ಚಿ-ತ್ತಾತ್ಮಾ
ನಿರಾಶೆ
ನಿರಾ-ಶೆ-ಯ-ನ್ನಾ-ದರೂ
ನಿರಾ-ಶೆ-ಯಿಂ-ದಲೇ
ನಿರಾ-ಶ್ರಯಃ
ನಿರಾ-ಶ್ರ-ಯ-ನಾ-ಗಿ-ರುವ
ನಿರಾ-ಹಾ-ರ-ನಾದ
ನಿರಾ-ಹಾ-ರಸ್ಯ
ನಿರೀ-ಕ್ಷಣೆ
ನಿರೀ-ಕ್ಷಿ-ಸದೆ
ನಿರೀ-ಕ್ಷಿ-ಸ-ಬಾ-ರದು
ನಿರೀ-ಕ್ಷಿ-ಸ-ಬೇಕು
ನಿರೀ-ಕ್ಷಿ-ಸಿ-ದು-ದ-ಕ್ಕಿಂತ
ನಿರೀ-ಕ್ಷಿ-ಸಿದೆ
ನಿರೀ-ಕ್ಷಿಸು
ನಿರೀ-ಕ್ಷಿ-ಸು-ತ್ತಿದೆ
ನಿರೀ-ಕ್ಷಿ-ಸು-ತ್ತಿಲ್ಲ
ನಿರೀ-ಕ್ಷಿ-ಸುವ
ನಿರೀ-ಕ್ಷಿ-ಸು-ವು-ದಿಲ್ಲ
ನಿರೀ-ಕ್ಷಿ-ಸು-ವುದೋ
ನಿರೀ-ಕ್ಷಿ-ಸು-ವೆವೋ
ನಿರೀ-ಕ್ಷೇಹಂ
ನಿರೀ-ಶ್ವ-ರ-ವಾದಿ
ನಿರು-ತ್ತ-ರ-ನಾದ
ನಿರುದ್ಧಂ
ನಿರುಧ್ಯ
ನಿರು-ವನು
ನಿರೂ-ಪಿ-ಸು-ವುದು
ನಿರೋ-ಧಿ-ಸಲೇ
ನಿರೋ-ಧಿ-ಸ-ಲ್ಪಟ್ಟು
ನಿರೋ-ಧಿಸಿ
ನಿರೋ-ಧಿ-ಸು-ವುದು
ನಿರ್ಗ-ತಿ-ಕ-ನಿಗೆ
ನಿರ್ಗುಣ
ನಿರ್ಗುಣಂ
ನಿರ್ಗು-ಣ-ಗ-ಳೆ-ರ-ಡಕ್ಕೂ
ನಿರ್ಗು-ಣದ
ನಿರ್ಗು-ಣ-ನನ್ನು
ನಿರ್ಗು-ಣ-ನಾ-ಗಿ-ರು-ವು-ದ-ರಿಂ-ದಲೂ
ನಿರ್ಗು-ಣ-ವಾಗಿ
ನಿರ್ಗು-ಣ-ವಾ-ದುದೂ
ನಿರ್ಗು-ಣೋ-ಪಾ-ಸಕ
ನಿರ್ಗು-ಣೋ-ಪಾ-ಸನೆ
ನಿರ್ಜನ
ನಿರ್ಜ-ನ-ಪ್ರ-ದೇಶ
ನಿರ್ಜ-ನ-ಪ್ರ-ದೇ-ಶ-ದ-ಲ್ಲಿಯೂ
ನಿರ್ಜ-ನ-ಪ್ರ-ದೇ-ಶ-ದ-ಲ್ಲಿರ
ನಿರ್ಜ-ನ-ಪ್ರಿಯ
ನಿರ್ಜ-ನ-ಪ್ರಿ-ಯತೆ
ನಿರ್ಜ-ನಾ-ರ-ಣ್ಯ-ಗಳು
ನಿರ್ಣಯ
ನಿರ್ಣ-ಯಕ್ಕೆ
ನಿರ್ಣಯಿ
ನಿರ್ಣ-ಯಿ-ಸ-ಬ-ಹುದು
ನಿರ್ಣ-ಯಿ-ಸು-ತ್ತಾನೆ
ನಿರ್ಣ-ಯಿ-ಸು-ತ್ತಾರೆ
ನಿರ್ಣ-ಯಿ-ಸು-ತ್ತಿ-ದ್ದರು
ನಿರ್ಣ-ಯಿ-ಸುವ
ನಿರ್ಣ-ಯಿ-ಸು-ವು-ದಕ್ಕೆ
ನಿರ್ದ-ಯ-ನಾಗಿ
ನಿರ್ದ-ಯ-ವಾಗಿ
ನಿರ್ದಾ-ಕ್ಷಿ-ಣ್ಯ-ದಿಂದ
ನಿರ್ದಾ-ಕ್ಷಿ-ಣ್ಯ-ವಾಗಿ
ನಿರ್ದಿಷ್ಟ
ನಿರ್ದಿ-ಷ್ಟ-ವಾದ
ನಿರ್ದೇಶ
ನಿರ್ದೇ-ಶಿ-ಸಲು
ನಿರ್ದೇಶೋ
ನಿರ್ದೇ-ಹಿ-ಯಾದ
ನಿರ್ದೋಷಂ
ನಿರ್ದೋ-ಷ-ವಾದ
ನಿರ್ದ್ವಂ-ದ್ವ-ನಾಗು
ನಿರ್ದ್ವಂ-ದ್ವನೂ
ನಿರ್ದ್ವಂದ್ವೋ
ನಿರ್ಧ-ರಿಸ
ನಿರ್ಧ-ರಿ-ಸ-ಬ-ಲ್ಲುದು
ನಿರ್ಧ-ರಿ-ಸ-ಬೇ-ಕಾ-ಗಿದೆ
ನಿರ್ಧ-ರಿ-ಸ-ಬೇ-ಕಾ-ದರೂ
ನಿರ್ಧ-ರಿ-ಸಲಿ
ನಿರ್ಧ-ರಿ-ಸಿ-ಕೊಂ-ಡಿ-ರು-ವನು
ನಿರ್ಧ-ರಿ-ಸಿದೆ
ನಿರ್ಧ-ರಿ-ಸಿಲ್ಲ
ನಿರ್ಧ-ರಿ-ಸುವ
ನಿರ್ಧ-ರಿ-ಸು-ವಂ-ತಿಲ್ಲ
ನಿರ್ಧ-ರಿ-ಸು-ವನು
ನಿರ್ಧ-ರಿ-ಸು-ವು-ದಕ್ಕೆ
ನಿರ್ಧ-ರಿ-ಸು-ವುದು
ನಿರ್ಧ-ರಿ-ಸು-ವುದೇ
ನಿರ್ಧಾರ
ನಿರ್ಧಾ-ರಕ್ಕೆ
ನಿರ್ಧಾ-ರ-ವಾ-ಗಿದೆ
ನಿರ್ಧಾ-ರಿ-ತ-ವಾ-ಗಿದೆ
ನಿರ್ನಾಮ
ನಿರ್ನಾ-ಮ-ಮಾಡಿ
ನಿರ್ನಾ-ಮ-ವಾ-ಗ-ಬ-ಹುದು
ನಿರ್ನಾ-ಮ-ವಾಗಿ
ನಿರ್ನಾ-ಮ-ವಾ-ಗಿ-ಬಿ-ಡು-ವುವು
ನಿರ್ನಾ-ಮ-ವಾ-ಗು-ತ್ತೇವೆ
ನಿರ್ನಾ-ಮ-ವಾ-ಗು-ವರು
ನಿರ್ನಾ-ಮ-ವಾ-ದರೂ
ನಿರ್ಬಂ-ಧಿ-ಸುವ
ನಿರ್ಭಯ
ನಿರ್ಭ-ಯತೆ
ನಿರ್ಭ-ಯ-ನಾ-ಗಿ-ರ-ಬೇಕು
ನಿರ್ಭ-ಯನು
ನಿರ್ಭ-ಯನೂ
ನಿರ್ಭೀತ
ನಿರ್ಭೀ-ತ-ನಾ-ಗು-ವನು
ನಿರ್ಭೀ-ತನು
ನಿರ್ಭೀ-ತಿ-ಯಿಂದ
ನಿರ್ಮಮ
ನಿರ್ಮಮಃ
ನಿರ್ಮಮೋ
ನಿರ್ಮಲ
ನಿರ್ಮಲಂ
ನಿರ್ಮ-ಲ-ತ್ವಾತ್
ನಿರ್ಮ-ಲ-ವಾಗಿ
ನಿರ್ಮ-ಲ-ವಾ-ಗಿದೆ
ನಿರ್ಮ-ಲ-ವಾ-ಗಿ-ರು-ವುವು
ನಿರ್ಮ-ಲ-ವಾದ
ನಿರ್ಮ-ಲ-ವಾ-ದು-ದ-ರಿಂದ
ನಿರ್ಮಾಣ
ನಿರ್ಮಾ-ನ-ಮೋಹಾ
ನಿರ್ಮಿ-ತ-ವಾದ
ನಿರ್ಮಿ-ಸಿದ
ನಿರ್ಮಿ-ಸಿ-ದ-ವನು
ನಿರ್ಮಿ-ಸಿ-ದ್ದಾನೆ
ನಿರ್ಮಿ-ಸಿ-ರು-ವನು
ನಿರ್ಮಿ-ಸು-ತ್ತಾನೆ
ನಿರ್ಮೂಲ
ನಿರ್ಮೂ-ಲ-ವಾ-ಗಿದೆ
ನಿರ್ಮೂ-ಲ-ವಾ-ದ-ಮೇಲೆ
ನಿರ್ಯೋ-ಗ-ಕ್ಷೇಮ
ನಿರ್ಯೋ-ಗ-ಕ್ಷೇ-ಮನೂ
ನಿರ್ಲಕ್ಷ
ನಿರ್ಲ-ಕ್ಷಿ-ಸಲೂ
ನಿರ್ಲ-ಕ್ಷಿ-ಸಿ-ರು-ವನು
ನಿರ್ಲ-ಕ್ಷಿ-ಸು-ವುದು
ನಿರ್ಲ-ಜ್ಜೆ-ಯಿಂದ
ನಿರ್ಲಿಪ್ತ
ನಿರ್ಲಿ-ಪ್ತ-ನಾಗಿ
ನಿರ್ಲಿ-ಪ್ತರು
ನಿರ್ವಂ-ಚ-ನೆ-ಯಿಂದ
ನಿರ್ವ-ಹಣೆ
ನಿರ್ವ-ಹ-ಣೆಗೆ
ನಿರ್ವ-ಹ-ಣೆ-ಯಿಂದ
ನಿರ್ವ-ಹ-ಣೆಯೇ
ನಿರ್ವ-ಹಿ-ಸಲಿ
ನಿರ್ವ-ಹಿ-ಸಿ-ದರೆ
ನಿರ್ವ-ಹಿ-ಸು-ತ್ತಿ-ದ್ದೇವೆ
ನಿರ್ವ-ಹಿ-ಸು-ತ್ತಿ-ರು-ವನು
ನಿರ್ವ-ಹಿ-ಸು-ವನು
ನಿರ್ವ-ಹಿ-ಸು-ವರು
ನಿರ್ವ-ಹಿ-ಸು-ವು-ದಕ್ಕೆ
ನಿರ್ವಾಣ
ನಿರ್ವಾ-ಣ-ಪ-ರಮ
ನಿರ್ವಾ-ಣ-ಪ-ರ-ಮಾಂ
ನಿರ್ವಾ-ಣ-ವನ್ನು
ನಿರ್ವಾ-ಣ-ಶಾಂ-ತಿ-ಯನ್ನು
ನಿರ್ವಾ-ಹ-ವಿ-ಲ್ಲದೆ
ನಿರ್ವಿ-ಕಲ್ಪ
ನಿರ್ವಿ-ಕಾ-ರನೂ
ನಿರ್ವಿ-ವಾ-ದ-ವಾಗಿ
ನಿರ್ವೇದಂ
ನಿರ್ವೈರಃ
ನಿಲು-ಕದ
ನಿಲು-ಕ-ದಂತೆ
ನಿಲು-ಕ-ದ-ವನು
ನಿಲು-ಕ-ದುದೆ
ನಿಲು-ಕದೆ
ನಿಲು-ಕುವ
ನಿಲು-ಕು-ವಂ-ತ-ಹವು
ನಿಲು-ಕು-ವು-ದಿಲ್ಲ
ನಿಲು-ಕು-ವುದು
ನಿಲುವು
ನಿಲುವೇ
ನಿಲ್ದಾಣ
ನಿಲ್ದಾ-ಣ-ಗಳನ್ನು
ನಿಲ್ದಾ-ಣ-ಗಳು
ನಿಲ್ದಾ-ಣ-ದ-ಲ್ಲಿ-ರು-ವಂತೆ
ನಿಲ್ಲ-ಕೂ-ಡದು
ನಿಲ್ಲ-ದಿ-ರಲಿ
ನಿಲ್ಲದೆ
ನಿಲ್ಲ-ದೆ-ಹೋ-ದರೆ
ನಿಲ್ಲ-ಬ-ಲ್ಲನೋ
ನಿಲ್ಲ-ಬ-ಲ್ಲವು
ನಿಲ್ಲ-ಬ-ಹುದು
ನಿಲ್ಲ-ಬೇಕು
ನಿಲ್ಲ-ಲಾರ
ನಿಲ್ಲ-ಲಾ-ರದು
ನಿಲ್ಲ-ಲಾ-ರವು
ನಿಲ್ಲಲಿ
ನಿಲ್ಲ-ಲಿಲ್ಲ
ನಿಲ್ಲ-ವುದು
ನಿಲ್ಲಿಸ
ನಿಲ್ಲಿ-ಸದೇ
ನಿಲ್ಲಿ-ಸ-ಬಲ್ಲ
ನಿಲ್ಲಿ-ಸ-ಬ-ಹುದು
ನಿಲ್ಲಿ-ಸ-ಬೇ-ಕಾ-ದರೆ
ನಿಲ್ಲಿ-ಸ-ಬೇಕು
ನಿಲ್ಲಿ-ಸ-ಬೇ-ಕೆಂದು
ನಿಲ್ಲಿ-ಸಲು
ನಿಲ್ಲಿಸಿ
ನಿಲ್ಲಿ-ಸಿ-ದನು
ನಿಲ್ಲಿ-ಸಿ-ದರೆ
ನಿಲ್ಲಿ-ಸಿ-ಬಿ-ಡು-ವುದು
ನಿಲ್ಲಿ-ಸಿಯೂ
ನಿಲ್ಲಿ-ಸಿ-ರು-ತ್ತಾನೆ
ನಿಲ್ಲಿಸು
ನಿಲ್ಲಿ-ಸು-ತ್ತಾನೆ
ನಿಲ್ಲಿ-ಸು-ತ್ತೇವೆ
ನಿಲ್ಲಿ-ಸು-ತ್ತೇ-ವೆಯೋ
ನಿಲ್ಲಿ-ಸುವ
ನಿಲ್ಲಿ-ಸು-ವನು
ನಿಲ್ಲಿ-ಸು-ವರೋ
ನಿಲ್ಲಿ-ಸು-ವು-ದ-ಕ್ಕಿಂತ
ನಿಲ್ಲಿ-ಸು-ವು-ದಲ್ಲ
ನಿಲ್ಲಿ-ಸು-ವು-ದಿಲ್ಲ
ನಿಲ್ಲಿ-ಸು-ವುದು
ನಿಲ್ಲಿ-ಸೋಣ
ನಿಲ್ಲು
ನಿಲ್ಲು-ತ್ತಾನೋ
ನಿಲ್ಲು-ತ್ತಿ-ದ್ದಳು
ನಿಲ್ಲು-ತ್ತೇ-ವೆಯೋ
ನಿಲ್ಲುವ
ನಿಲ್ಲು-ವಂ-ತಿಲ್ಲ
ನಿಲ್ಲು-ವಂ-ತೆಯೆ
ನಿಲ್ಲು-ವನು
ನಿಲ್ಲು-ವರು
ನಿಲ್ಲು-ವ-ಹಾ-ಗಿಲ್ಲ
ನಿಲ್ಲು-ವು-ದಕ್ಕೂ
ನಿಲ್ಲು-ವು-ದಕ್ಕೆ
ನಿಲ್ಲು-ವು-ದಿಲ್ಲ
ನಿಲ್ಲು-ವುದು
ನಿಲ್ಲು-ವುದೊ
ನಿಲ್ಲು-ವುದೋ
ನಿಲ್ಲು-ವುವು
ನಿವ-ರ್ತಂತಿ
ನಿವ-ರ್ತಂತೇ
ನಿವ-ರ್ತತೇ
ನಿವ-ಸಿ-ಷ್ಯಸಿ
ನಿವಾ-ತಸ್ಥೋ
ನಿವಾ-ರಣೆ
ನಿವಾ-ರ-ಣೆಗೆ
ನಿವಾ-ರ-ಣೆ-ಯಾ-ಗಲಿ
ನಿವಾ-ರಿ-ಸ-ಬೇಕು
ನಿವಾ-ರಿ-ಸಲು
ನಿವಾ-ರಿಸಿ
ನಿವಾ-ರಿ-ಸಿ-ಕೊಂ-ಡಿ-ರು-ವನು
ನಿವಾ-ರಿ-ಸಿ-ಕೊ-ಳ್ಳು-ವನು
ನಿವಾ-ರಿಸು
ನಿವಾ-ರಿ-ಸು-ತ್ತಾನೆ
ನಿವಾ-ರಿ-ಸು-ತ್ತೇನೆ
ನಿವಾ-ರಿ-ಸು-ತ್ತೇ-ವೆಯೋ
ನಿವಾ-ರಿ-ಸುವ
ನಿವಾ-ರಿ-ಸು-ವನು
ನಿವಾ-ರಿ-ಸು-ವುದು
ನಿವಾ-ರಿ-ಸೆಂದು
ನಿವಾಸ
ನಿವಾಸಃ
ನಿವಾ-ಸ-ಸ್ಥಾನ
ನಿವೃತ್ತ
ನಿವೃ-ತ್ತ-ನಾ-ಗಿ-ರು-ವನು
ನಿವೃ-ತ್ತಾನಿ
ನಿವೃತ್ತಿ
ನಿವೃ-ತ್ತಿಂ
ನಿವೃ-ತ್ತಿ-ಗಳನ್ನು
ನಿವೃ-ತ್ತಿ-ಗಾಗಿ
ನಿವೃ-ತ್ತಿಗೆ
ನಿವೃ-ತ್ತಿಯ
ನಿವೃ-ತ್ತಿಯೂ
ನಿವೃ-ತ್ತಿ-ವೇ-ತನ
ನಿವೇ-ದಿ-ಸ-ಬೇಕು
ನಿವೇ-ಶಯ
ನಿಶಾ
ನಿಶ್ಚಯ
ನಿಶ್ಚಯಂ
ನಿಶ್ಚ-ಯ-ಪೂ-ರ್ವಕ
ನಿಶ್ಚ-ಯ-ಬು-ದ್ಧಿ-ಯು-ಳ್ಳ-ವ-ನಾ-ಗಿ-ರು-ತ್ತಾನೆ
ನಿಶ್ಚ-ಯ-ರೂ-ಪ-ವಾ-ಗಿ-ರುವ
ನಿಶ್ಚ-ಯ-ವನ್ನು
ನಿಶ್ಚ-ಯ-ವಾ-ಗಿದೆ
ನಿಶ್ಚ-ಯ-ವಾದ
ನಿಶ್ಚ-ಯಾ-ತ್ಮ-ಕ-ವಾದ
ನಿಶ್ಚ-ಯಿ-ಸಲು
ನಿಶ್ಚ-ಯಿಸಿ
ನಿಶ್ಚ-ಯಿ-ಸಿ-ದಂ-ತಾ-ಗಿದೆ
ನಿಶ್ಚ-ಯಿ-ಸಿ-ದಂತೆ
ನಿಶ್ಚ-ಯಿ-ಸಿ-ರು-ವನು
ನಿಶ್ಚ-ಯಿ-ಸು-ತ್ತಾನೆ
ನಿಶ್ಚ-ಯಿ-ಸು-ತ್ತೇವೆ
ನಿಶ್ಚ-ಯಿ-ಸು-ವರು
ನಿಶ್ಚ-ಯಿ-ಸು-ವು-ದಕ್ಕೆ
ನಿಶ್ಚ-ಯೇನ
ನಿಶ್ಚ-ರತಿ
ನಿಶ್ಚ-ಲ-ವಾಗಿ
ನಿಶ್ಚಲಾ
ನಿಶ್ಚಿಂ-ತ-ನಾ-ಗಿರು
ನಿಶ್ಚಿಂ-ತೆ-ಯಿಂದ
ನಿಶ್ಚಿತಂ
ನಿಶ್ಚಿ-ತ-ವಾದ
ನಿಶ್ಚಿ-ತಾಃ
ನಿಶ್ಚಿತ್ಯ
ನಿಶ್ವಾ-ಸ-ದಂತೆ
ನಿಶ್ಶಬ್ದ
ನಿಶ್ಶೇ-ಷ-ವಾಗಿ
ನಿಷೇ-ಧ-ಗ-ಳಿರ
ನಿಷೇ-ಧದ
ನಿಷೇ-ಧಾ-ತ್ಮ-ಕ-ವಾ-ಯಿತು
ನಿಷ್ಕ-ರು-ಣಿ-ಯಂತೆ
ನಿಷ್ಕ-ರು-ಣೆ-ಯಿಂದ
ನಿಷ್ಕ-ರ್ಷಿಸ
ನಿಷ್ಕ-ರ್ಷಿ-ಸದೆ
ನಿಷ್ಕ-ರ್ಷಿ-ಸ-ಬಲ್ಲ
ನಿಷ್ಕ-ರ್ಷಿ-ಸ-ಬ-ಲ್ಲನೊ
ನಿಷ್ಕ-ರ್ಷಿ-ಸ-ಬೇ-ಕಾ-ದರೆ
ನಿಷ್ಕ-ರ್ಷಿ-ಸ-ಲಾ-ರದೆ
ನಿಷ್ಕ-ರ್ಷಿ-ಸ-ಲ್ಪ-ಟ್ಟಿ-ದ್ದರೂ
ನಿಷ್ಕ-ರ್ಷಿಸಿ
ನಿಷ್ಕ-ರ್ಷಿ-ಸು-ತ್ತಾರೆ
ನಿಷ್ಕ-ರ್ಷಿ-ಸು-ತ್ತೇವೆ
ನಿಷ್ಕ-ರ್ಷಿ-ಸುವ
ನಿಷ್ಕ-ರ್ಷಿ-ಸು-ವುದು
ನಿಷ್ಕಾಮ
ನಿಷ್ಠ-ನಾ-ಗಿ-ದ್ದಾನೆ
ನಿಷ್ಠಾ
ನಿಷ್ಠಾ-ರೂ-ಪ-ವಾದ
ನಿಷ್ಠು-ರ-ವನ್ನು
ನಿಷ್ಠೆ
ನಿಷ್ಠೆಯ
ನಿಷ್ಠೆ-ಯನ್ನು
ನಿಷ್ಠೆ-ಯಾದ
ನಿಷ್ಠೆ-ಯು-ಳ್ಳದ್ದು
ನಿಷ್ಪ-ಕ್ಷ-ಪಾ-ತ-ವಾಗಿ
ನಿಷ್ಪ-ಕ್ಷ-ಪಾ-ತಿಗೆ
ನಿಷ್ಪ-ಕ್ಷ-ಪಾ-ತಿ-ಯಂತೆ
ನಿಷ್ಪ-ಕ್ಷ-ಪಾ-ತಿ-ಯಾದ
ನಿಷ್ಪ್ರ-ಯೋ-ಜ-ಕ-ವಾ-ಗು-ವು-ದಿಲ್ಲ
ನಿಷ್ಪ್ರ-ಯೋ-ಜ-ನ-ವಾ-ಯಿತು
ನಿಷ್ಫ-ಲ-ವಾ-ಗ-ಲಿಲ್ಲ
ನಿಷ್ಫ-ಲ-ವಾ-ಗು-ವು-ದಿಲ್ಲ
ನಿಷ್ಫ-ಲ-ವಾ-ಗು-ವುದು
ನಿಸ್ತ್ರೈ-ಗುಣ್ಯೋ
ನಿಸ್ವಾ-ರ್ಥ-ದೃ-ಷ್ಟಿ-ಯಿಂದ
ನಿಸ್ವಾ-ರ್ಥ-ನಾ-ಗುತ್ತ
ನಿಸ್ಸಂ-ಕೋ-ಚ-ದಿಂದ
ನಿಸ್ಸಂಗ
ನಿಸ್ಸಂ-ಗ-ನಾಗಿ
ನಿಸ್ಸಂ-ಗ-ನಾ-ಗಿ-ರು-ವನು
ನಿಸ್ಸಂ-ಗ-ರಾಗಿ
ನಿಸ್ಸಂಗಿ
ನಿಸ್ಸಂ-ದೇ-ಹ-ವಾಗಿ
ನಿಸ್ಸಂ-ಶಯ
ನಿಸ್ಸಂ-ಶ-ಯ-ವಾಗಿ
ನಿಸ್ಸಂ-ಶ-ಯ-ವಾ-ಗಿಯೂ
ನಿಸ್ಸ-ಹಾ-ಯ-ಕ-ನಾಗಿ
ನಿಸ್ಸ-ಹಾ-ಯ-ಕ-ವಾ-ಗು-ವುದು
ನಿಸ್ಸ-ಹಾ-ಯ-ನಾಗಿ
ನಿಸ್ಸ-ಹಾ-ಯ-ರಾ-ಗಿ-ರು-ತ್ತವೆ
ನಿಸ್ಸಾರ
ನಿಹ-ತಾಃ
ನಿಹತ್ಯ
ನಿಹಾ-ರಿಕೆ
ನಿಹಾ-ರಿ-ಕೆ-ಗ-ಳಿವೆ
ನೀಗಿಸು
ನೀಚ
ನೀಚಂ
ನೀಚ-ಕೃ-ತ್ಯ-ಗಳು
ನೀಡ-ಬ-ಹುದು
ನೀಡ-ಬೇಕು
ನೀಡಲಿ
ನೀಡಲು
ನೀಡಿ
ನೀಡಿದ
ನೀಡಿ-ದನೊ
ನೀಡಿದೆ
ನೀಡಿ-ದ್ದರೆ
ನೀಡು-ತ್ತದೆ
ನೀಡು-ತ್ತಾನೆ
ನೀಡು-ತ್ತಾರೆ
ನೀಡು-ತ್ತಿ-ರು-ವನು
ನೀಡು-ತ್ತೇನೆ
ನೀಡುವ
ನೀಡು-ವನು
ನೀಡು-ವಾಗ
ನೀಡು-ವು-ದ-ಕ್ಕಾಗಿ
ನೀಡು-ವು-ದಕ್ಕೆ
ನೀಡು-ವು-ದ-ರಿಂದ
ನೀಡು-ವುದು
ನೀಡು-ವುದೇ
ನೀಡೆಂದು
ನೀತಿ
ನೀತಿ-ಗಳನ್ನು
ನೀತಿ-ಗಳು
ನೀತಿಗೆ
ನೀತಿ-ನಿ-ಯ-ಮ-ಗಳನ್ನು
ನೀತಿಯ
ನೀತಿ-ಯನ್ನು
ನೀತಿ-ಯನ್ನೆ
ನೀತಿ-ಯಿಂದ
ನೀತಿ-ಯೊಂದೇ
ನೀತಿ-ರಸ್ಮಿ
ನೀತಿ-ರ್ಮ-ತಿ-ರ್ಮಮ
ನೀನಾ-ಗಲೀ
ನೀನಿ-ಲ್ಲದೇ
ನೀನು
ನೀನೂ
ನೀನೆ
ನೀನೇ
ನೀನೇಕೆ
ನೀನೊ-ಬ್ಬನೇ
ನೀನೊ-ಳ್ಳೆಯ
ನೀನ್ಯಾಕೊ
ನೀರ
ನೀರ-ಗುಳ್ಳೆ
ನೀರ-ಗು-ಳ್ಳೆ-ಯಂತೆ
ನೀರದು
ನೀರ-ನ-ಲ್ಲಿ-ರು-ವಾಗ
ನೀರನು
ನೀರ-ನ್ನಾ-ಗಲಿ
ನೀರನ್ನು
ನೀರ-ನ್ನೆಲ್ಲಾ
ನೀರನ್ನೇ
ನೀರ-ಲ್ಲಿ-ರುವ
ನೀರಸ
ನೀರ-ಸ-ವಾ-ಗಲಿ
ನೀರ-ಸ-ವಾ-ಗಲ್ಲ
ನೀರ-ಸ-ವಾಗಿ
ನೀರ-ಸ-ವಾ-ಗಿ-ರುವು
ನೀರ-ಸ-ವಾ-ಗಿ-ರು-ವು-ದಕ್ಕೆ
ನೀರ-ಸ-ವಾ-ಗಿ-ರು-ವು-ದಿಲ್ಲ
ನೀರ-ಸ-ವಾ-ಗು-ವುದು
ನೀರ-ಸ-ವಾದ
ನೀರಾ-ಗು-ವುದೊ
ನೀರಿ
ನೀರಿಗೂ
ನೀರಿಗೆ
ನೀರಿದೆ
ನೀರಿ-ದ್ದರೂ
ನೀರಿ-ದ್ದರೆ
ನೀರಿನ
ನೀರಿ-ನಂ-ತಿದೆ
ನೀರಿ-ನಂತೆ
ನೀರಿ-ನ-ಮೇಲೆ
ನೀರಿ-ನಲ್ಲಿ
ನೀರಿ-ನ-ಲ್ಲಿಟ್ಟು
ನೀರಿ-ನ-ಲ್ಲಿ-ದ್ದರೂ
ನೀರಿ-ನ-ಲ್ಲಿ-ದ್ದರೆ
ನೀರಿ-ನ-ಲ್ಲಿಯೇ
ನೀರಿ-ನ-ಲ್ಲಿ-ರುವ
ನೀರಿ-ನ-ಲ್ಲಿ-ರು-ವಂತೆ
ನೀರಿ-ನಿಂದ
ನೀರಿ-ನಿಂ-ದಲೇ
ನೀರಿ-ನೊಂ-ದಿಗೆ
ನೀರಿ-ನೊ-ಳಗೆ
ನೀರಿ-ರು-ವುದು
ನೀರಿ-ಲ್ಲದೆ
ನೀರು
ನೀರು-ಗುಳ್ಳೆ
ನೀರು-ಬಿಟ್ಟು
ನೀರೂ
ನೀರೂ-ರು-ವಂತೆ
ನೀರೂ-ರು-ವುದು
ನೀರೆ
ನೀರೆ-ರೆ-ಯು-ವನು
ನೀರೆಲ್ಲ
ನೀರೇ
ನೀರೊಂದು
ನೀರ್ಗ-ಲ್ಲಾಗಿ
ನೀರ್ಗ-ಲ್ಲಿಗೆ
ನೀರ್ಗ-ಲ್ಲಿ-ನಂತೆ
ನೀರ್ಗಲ್ಲು
ನೀಲ-ಗಿ-ರಿ-ಯನ್ನೇ
ನೀಲಿ
ನೀವು
ನೀವು-ಗ-ಳೆಲ್ಲ
ನೀವು-ಭೀಷ್ಮ
ನೀವೆ
ನೀವೆಲ್ಲ
ನೀವೇ
ನೀವೇನು
ನೀಹಾ-ರಿ-ಕೆ-ಗಳೂ
ನೀಹಾ-ರಿ-ಕೆ-ಗ-ಳೆಲ್ಲ
ನೀಹಾ-ರಿ-ಕೆ-ಯಾ-ಗಲೀ
ನೀಹಾ-ರಿ-ಕೆ-ವ-ರೆಗೆ
ನು
ನುಂಗಲು
ನುಂಗಲೇ
ನುಂಗಿ
ನುಂಗಿ-ದರೆ
ನುಂಗು-ತ್ತಿ-ರ-ಬೇಕು
ನುಂಗು-ತ್ತಿ-ರುವ
ನುಂಗು-ತ್ತಿ-ರು-ವನು
ನುಂಗು-ವುದು
ನುಗ್ಗ-ಬ-ಹುದು
ನುಗ್ಗ-ಬೇ-ಕಾ-ದರೆ
ನುಗ್ಗ-ಲಾ-ರದು
ನುಗ್ಗಿ
ನುಗ್ಗಿದ
ನುಗ್ಗು
ನುಗ್ಗು-ತ್ತಿದೆ
ನುಗ್ಗು-ವನು
ನುಗ್ಗು-ವೆವು
ನುಡಿ
ನುಡಿದ
ನುಡಿ-ದಂತೆ
ನುಡಿ-ಯನ್ನು
ನುಡಿ-ಯ-ಬೇಕು
ನುಡಿ-ಯು-ತ್ತಾನೆ
ನುಡಿ-ಸು-ತ್ತಿ-ರು-ವ-ವನು
ನುಡಿ-ಸು-ವ-ವ-ನಿಗೆ
ನುಡಿ-ಸು-ವ-ವನು
ನುಣು-ಚಿ-ಕೊಂ-ಡಿ-ರು-ವನು
ನುಣು-ಪಾ-ಗಿದೆ
ನುರಿತ
ನುಸುಳಿ
ನುಸು-ಳಿ-ಕೊಂಡು
ನುಸು-ಳಿ-ದರೆ
ನೂಕ-ಬೇ-ಕಾ-ಗಿದೆ
ನೂಕ-ಬೇ-ಕಾ-ಗಿಲ್ಲ
ನೂಕಿ
ನೂಕಿ-ಕೊಂಡು
ನೂಕಿ-ದ್ದಲ್ಲ
ನೂಕಿ-ಬಿ-ಡು-ವುವು
ನೂಕಿ-ಸಿ-ಕೊಂಡು
ನೂಕು
ನೂಕು-ತ್ತದೆ
ನೂಕು-ತ್ತಾನೆ
ನೂಕು-ತ್ತಿ-ರುವ
ನೂಕು-ತ್ತಿ-ರು-ವಂತೆ
ನೂಕು-ತ್ತಿ-ರು-ವುವು
ನೂಕು-ನು-ಗ್ಗಲೊ
ನೂಕು-ವಳು
ನೂಕು-ವುದನ್ನು
ನೂತನ
ನೂತ-ನತೆ
ನೂರಾಗಿ
ನೂರಾ-ಗು-ವುದು
ನೂರಾರು
ನೂರು
ನೂಲನ್ನು
ನೂಲಿ-ನಿಂದ
ನೂಲು
ನೂಲೇ
ನೃತ್ಯ
ನೃತ್ಯ-ಗಾರ
ನೃಯಜ್ಞ
ನೃಲೋಕೇ
ನೃಷು
ನೆಂಚಿ-ಕೊಂಡು
ನೆಂಟ-ನೊಬ್ಬ
ನೆಂಟ-ರನ್ನು
ನೆಂಟ-ರಿ-ಷ್ಟ-ರನ್ನು
ನೆಂಟ-ರಿ-ಷ್ಟ-ರಲ್ಲಿ
ನೆಂಟ-ರಿ-ಷ್ಟ-ರಿಂದ
ನೆಂಟ-ರಿ-ಷ್ಟ-ರಿಗೇ
ನೆಂಟ-ರಿ-ಷ್ಟರು
ನೆಂಟರು
ನೆಂಟ-ರು-ಗ-ಳೆಲ್ಲ
ನೆಂದಿಗೂ
ನೆಂದು
ನೆಂದೂ
ನೆಂಬ
ನೆಕ್ಕಿ
ನೆಕ್ಕು-ತ್ತಾನೆ
ನೆಕ್ಕು-ತ್ತಿ-ದ್ದೀಯೆ
ನೆಕ್ಕು-ವನು
ನೆಗೆ-ದಾ-ಡು-ವುದು
ನೆಗೆದು
ನೆಗೆ-ಯಲು
ನೆಗೆ-ಯು-ತ್ತಿ-ರು-ವುದು
ನೆಗೆ-ಯು-ವರು
ನೆಗೆ-ಯು-ವು-ದಕ್ಕೆ
ನೆಗೆ-ಯು-ವುದು
ನೆಗ್ಗಲು
ನೆಚ್ಚ-ಬೇಡ
ನೆಚ್ಚಿ
ನೆಚ್ಚಿ-ಕೊಂ-ಡರೆ
ನೆಚ್ಚಿದ
ನೆಚ್ಚಿ-ದರೆ
ನೆಚ್ಚಿ-ದ-ವನ
ನೆಚ್ಚಿ-ದ-ವ-ನಿಗೆ
ನೆಚ್ಚಿ-ದ-ವನು
ನೆಚ್ಚಿ-ದ-ವ-ರನ್ನು
ನೆಚ್ಚಿ-ದ-ವರು
ನೆಚ್ಚಿದ್ದ
ನೆಚ್ಚಿ-ದ್ದರೆ
ನೆಚ್ಚಿ-ದ್ದಾನೆ
ನೆಚ್ಚಿ-ದ್ದೇನೆ
ನೆಚ್ಚಿ-ದ್ದೇವೆ
ನೆಚ್ಚಿನ
ನೆಚ್ಚಿ-ರುವ
ನೆಚ್ಚಿ-ರು-ವನು
ನೆಚ್ಚಿ-ರು-ವರೊ
ನೆಚ್ಚಿ-ರು-ವು-ದ-ರಿಂದ
ನೆಚ್ಚು
ನೆಚ್ಚು-ಗೆಡ
ನೆಚ್ಚು-ತ್ತೇವೋ
ನೆಚ್ಚುವ
ನೆಚ್ಚು-ವುದು
ನೆಟ್ಟ
ನೆಟ್ಟರೆ
ನೆಟ್ಟಾಗ
ನೆಟ್ಟು
ನೆಡ-ಬೇಕು
ನೆಡು-ತ್ತಾನೆ
ನೆಡು-ತ್ತಾ-ನೆಯೋ
ನೆಡು-ತ್ತಿ-ರು-ವನು
ನೆಡುವ
ನೆಡು-ವನು
ನೆಡು-ವು-ದ-ಲ್ಲದೆ
ನೆತ್ತಿಗೆ
ನೆತ್ತಿಯ
ನೆತ್ತಿ-ಯ-ಮೇ-ಲಿ-ರು-ವಾಗ
ನೆತ್ತಿ-ಯ-ಮೇಲೆ
ನೆನ-ಪನ್ನು
ನೆನಪಿ
ನೆನ-ಪಿಗೆ
ನೆನ-ಪಿನ
ನೆನ-ಪಿಲ್ಲ
ನೆನಪು
ನೆನ-ಪೆಲ್ಲ
ನೆನ-ಪೆಲ್ಲಾ
ನೆನಪೇ
ನೆನೆ-ದ-ಮೇಲೆ
ನೆನೆ-ದ-ರೇನೆ
ನೆನೆದು
ನೆನೆ-ನೆ-ನೆದು
ನೆನೆ-ಯಿ-ಸ-ಬ-ಹುದು
ನೆನೆ-ಯಿ-ಸ-ಲಾ-ರದು
ನೆನೆ-ಯು-ತ್ತಿ-ರು-ವರು
ನೆನೆ-ಸಿ-ಕೊಂ-ಡರೆ
ನೆನೆ-ಸಿ-ದಂತೆ
ನೆನೆ-ಸು-ತ್ತೇ-ವೆಯೋ
ನೆನೆ-ಸು-ವು-ದ-ಕ್ಕಾ-ಗು-ವು-ದಿಲ್ಲ
ನೆನೆ-ಹಾಕಿ
ನೆನ್ನೆ
ನೆನ್ನೆಯೂ
ನೆಪೋ-ಲಿ-ಯನ್
ನೆಮ್ಮದಿ
ನೆಮ್ಮ-ದಿಗೆ
ನೆಮ್ಮ-ದಿಯ
ನೆಮ್ಮ-ದಿ-ಯಾ-ಗಿ-ರುವೆ
ನೆಮ್ಮ-ದಿ-ಯಿಂದ
ನೆಯದು
ನೆಯ್ಗೆ
ನೆಯ್ಗೆ-ಯ-ವನು
ನೆಯ್ದ
ನೆಯ್ದಿ-ರುವ
ನೆಯ್ದಿ-ರು-ವ-ವನೇ
ನೆಯ್ದಿ-ರು-ವುದು
ನೆಯ್ದು
ನೆಯ್ದು-ಕೊಂ-ಡಿರು
ನೆಯ್ದು-ಕೊಂಡು
ನೆಯ್ಯು-ವುದು
ನೆರ-ಳಂತೆ
ನೆರ-ಳಿಗೆ
ನೆರ-ಳಿ-ನಂತೆ
ನೆರ-ಳಿ-ನಲ್ಲಿ
ನೆರಳು
ನೆರಳೆ
ನೆರಳೊ
ನೆರ-ವಾ-ಗ-ತಕ್ಕ
ನೆರ-ವಾ-ಗ-ಬೇ-ಕೆಂಬ
ನೆರವಿ
ನೆರ-ವೇ-ರಲಿ
ನೆರ-ವೇ-ರಿ-ಸ-ಬೇಕು
ನೆರ-ವೇ-ರಿ-ಸಿ-ಕೊ-ಳ್ಳು-ವು-ದಕ್ಕೆ
ನೆರ-ವೇ-ರಿ-ಸು-ತ್ತಿ-ರು-ವನು
ನೆರ-ವೇ-ರಿ-ಸು-ವ-ವನು
ನೆರೆ
ನೆರೆದ
ನೆರೆ-ದರು
ನೆರೆ-ದ-ವರ
ನೆರೆ-ದ-ವ-ರಿಗೆ
ನೆರೆ-ದಿದೆ
ನೆರೆ-ದಿದ್ದ
ನೆರೆ-ದಿರು
ನೆರೆ-ದಿ-ರುವ
ನೆರೆ-ದಿ-ರು-ವರು
ನೆರೆ-ದಿ-ರು-ವ-ವ-ರಾರೂ
ನೆರೆ-ದಿ-ರು-ವಾಗ
ನೆರೆ-ಯ-ವನು
ನೆರೆ-ಯು-ತ್ತಾರೆ
ನೆರೆ-ಯು-ವರು
ನೆರೆ-ವೇರಿ
ನೆರೆ-ವೇ-ರಿಸಿ
ನೆರೆ-ಹೊ-ರೆ-ಯ-ವ-ರಿ-ಗೆಲ್ಲ
ನೆರೆ-ಹೊ-ರೆ-ಯ-ವರು
ನೆಲ
ನೆಲಕ್ಕೆ
ನೆಲದ
ನೆಲ-ದಂತೆ
ನೆಲ-ದ-ಮೇಲೆ
ನೆಲ-ದಲ್ಲಿ
ನೆಲ-ದ-ಲ್ಲಿದೆ
ನೆಲ-ದಲ್ಲೇ
ನೆಲ-ದಿಂದ
ನೆಲ-ದೊ-ಳಗೆ
ನೆಲ-ವನ್ನು
ನೆಲ-ವಿತ್ತು
ನೆಲ-ಸಿದೆ
ನೆಲ-ಸಿ-ರುವ
ನೆಲ-ಸಿ-ರು-ವನು
ನೆಲ-ಸಿ-ರು-ವರು
ನೆಲ-ಸಿ-ರು-ವುದು
ನೆಲ-ಸಿ-ರು-ವುದೊ
ನೆಲಸು
ನೆಲ-ಸು-ತ್ತಾ-ನೆಯೊ
ನೆಲ-ಸು-ತ್ತಾನೋ
ನೆಲಿ-ಸಿ-ದಾಗ
ನೆಲೆ
ನೆಲೆ-ಗೊ-ಳಿ-ಸಿದ
ನೆಲೆಯ
ನೆಲೆ-ಯನ್ನೇ
ನೆಲೆಸಿ
ನೆಲೆ-ಸಿದೆ
ನೆಲೆ-ಸಿ-ದ್ದಾನೆ
ನೆಲೆ-ಸಿರು
ನೆಲೆ-ಸಿ-ರು-ತ್ತವೆ
ನೆಲೆ-ಸಿ-ರು-ತ್ತಾನೆ
ನೆಲೆ-ಸಿ-ರುವ
ನೆಲೆ-ಸಿ-ರು-ವನು
ನೆಲೆ-ಸಿ-ರು-ವ-ವರು
ನೆಲೆ-ಸಿ-ರು-ವ-ವರೇ
ನೆಲೆ-ಸಿ-ರು-ವುದನ್ನು
ನೆಲೆ-ಸಿ-ರು-ವುದು
ನೆಲೆ-ಸಿ-ರು-ವುದೊ
ನೆಲೆಸು
ನೆಲೆ-ಸು-ತ್ತಾನೆ
ನೆಲೆ-ಸು-ವಂತೆ
ನೆಲೆ-ಸು-ವುದು
ನೆಲ್ಲಿ-ಕಾ-ಯಂ-ತಾ-ಗು-ವುದು
ನೆಲ್ಲಿ-ಕಾಯಿ
ನೆವ
ನೇ
ನೇಂಗತೇ
ನೇಂದ್ರಿ-ಯಾ-ರ್ಥೇಷು
ನೇಚ್ಛಸಿ
ನೇತಾ-ಡು-ತ್ತವೆ
ನೇತಾ-ಡು-ತ್ತಿ-ರುವ
ನೇತಿ
ನೇತು
ನೇತು-ಹಾ-ಕಿ-ದರೆ
ನೇತೃ-ತ್ವ-ದಲ್ಲಿ
ನೇತ್ರ-ಗ-ಳಾಗಿ
ನೇತ್ರ-ಗ-ಳುಳ್ಳ
ನೇಮಿ-ಸಿ-ರುವ
ನೇಮೇ
ನೇಯು-ವನು
ನೇಯ್ಗೆ-ಯ-ವನು
ನೇಯ್ದಿದ್ದ
ನೇಯ್ದು
ನೇರ
ನೇರಕ್ಕೆ
ನೇರ-ಮಾ-ಡಲು
ನೇರ-ವಾ-ಗ-ಬೇ-ಕಾ-ದರೆ
ನೇರ-ವಾಗಿ
ನೇರ-ವಾ-ಗಿಯೋ
ನೇರ-ವಾ-ಗಿ-ರ-ಬೇಕು
ನೇರ-ವಾ-ಗುವ
ನೇರ-ವಾ-ಗು-ವು-ದಿಲ್ಲ
ನೇರ-ವಾ-ಗು-ವುದು
ನೇರ-ವಾದ
ನೇರ-ವೇ-ರಿ-ಸಿ-ದಂ-ತಾ-ಗು-ವುದು
ನೇಹ
ನೇಹಾ-ಭಿ-ಕ್ರ-ಮ-ನಾ-ಶೋಽಸ್ತಿ
ನೈಜ
ನೈಜ-ವಾದ
ನೈಜ-ಸ್ಥಿತಿ
ನೈಜ-ಸ್ಥಿ-ತಿ-ಯನ್ನು
ನೈಜ-ಸ್ಥಿ-ತಿ-ಯಲ್ಲಿ
ನೈಜ-ಸ್ಥಿ-ತಿ-ಯಾದ
ನೈಜ-ಸ್ವ-ಭಾ-ವ-ವನ್ನು
ನೈಟ್ಸ್
ನೈತ-ತ್ತ್ವ-ಯ್ಯು-ಪ-ಪ-ದ್ಯತೇ
ನೈತಿ
ನೈತಿಕ
ನೈತೇ
ನೈನ
ನೈನಂ
ನೈನಾಂ
ನೈಮಿ-ತ್ತಿಕ
ನೈರ್ಮ-ಲ್ಯದ
ನೈರ್ಮ-ಲ್ಯ-ವಾ-ಗಿ-ದ್ದರೂ
ನೈವ
ನೈವಂ
ನೈವೇದ್ಯ
ನೈವೇ-ದ್ಯಕ್ಕೆ
ನೈವೇ-ದ್ಯ-ದಂತೆ
ನೈವೇ-ದ್ಯಾ-ದಿ-ಗಳನ್ನು
ನೈವೇ-ದ್ಯಾ-ದಿ-ಗಳಲ್ಲಿ
ನೈವೇಹ
ನೈಷ್ಕರ್ಮ
ನೈಷ್ಕ-ರ್ಮ-ವನ್ನು
ನೈಷ್ಕರ್ಮ್ಯ
ನೈಷ್ಕ-ರ್ಮ್ಯ-ಸಿ-ದ್ಧಿಂ
ನೈಷ್ಕೃ-ತಿ-ಕೋ-ಽಲಸಃ
ನೈಷ್ಠಿ-ಕೀಮ್
ನೈಸ-ರ್ಗಿಕ
ನೊಂದಿಗೆ
ನೊಂದಿ-ರು-ವೆವು
ನೊಗ
ನೊಗಕ್ಕೆ
ನೊಗ-ವನ್ನು
ನೊಡನೆ
ನೊಡು-ತ್ತೇವೆ
ನೊಣ
ನೊಣ-ಗಳನ್ನು
ನೊಣ-ದಂತೆ
ನೊಬ್ಬನೆ
ನೊಬ್ಬನೇ
ನೊರೆ
ನೊರೆ-ಗಳು
ನೊರೆ-ಗುಳ್ಳೆ
ನೋ
ನೋಟ
ನೋಟ-ಗಳು
ನೋಟನ್ನು
ನೋಟ-ವಾ-ಗಲಿ
ನೋಟು-ಗಳನ್ನು
ನೋಡ
ನೋಡ-ಕೂ-ಡದು
ನೋಡ-ತಕ್ಕ
ನೋಡ-ತ್ತಿ-ದ್ದರೆ
ನೋಡದ
ನೋಡ-ದಂತೆ
ನೋಡ-ದಿ-ರುವ
ನೋಡ-ದು-ದನ್ನು
ನೋಡದೆ
ನೋಡ-ಬ-ಯ-ಸ-ವನು
ನೋಡ-ಬ-ಯ-ಸು-ವನು
ನೋಡ-ಬಲ್ಲ
ನೋಡ-ಬ-ಲ್ಲನೊ
ನೋಡ-ಬ-ಲ್ಲರು
ನೋಡ-ಬ-ಲ್ಲ-ವ-ನಾ-ಗಿದ್ದ
ನೋಡ-ಬ-ಲ್ಲವು
ನೋಡ-ಬ-ಲ್ಲೆ-ವಾ-ದರೆ
ನೋಡ-ಬ-ಹು-ದಾ-ಗಿತ್ತು
ನೋಡ-ಬ-ಹುದು
ನೋಡ-ಬ-ಹು-ದೆಂದು
ನೋಡ-ಬಾ-ರದು
ನೋಡ-ಬೇ-ಕಾಗಿ
ನೋಡ-ಬೇ-ಕಾ-ಗಿದೆ
ನೋಡ-ಬೇ-ಕಾ-ಗಿಲ್ಲ
ನೋಡ-ಬೇ-ಕಾ-ಗು-ವುದು
ನೋಡ-ಬೇ-ಕಾ-ದರೂ
ನೋಡ-ಬೇ-ಕಾ-ದರೆ
ನೋಡ-ಬೇಕು
ನೋಡ-ಬೇ-ಕೆಂ-ದರೆ
ನೋಡ-ಬೇ-ಕೆಂ-ದಿ-ರು-ವೆವು
ನೋಡ-ಬೇ-ಕೆಂದು
ನೋಡ-ಬೇ-ಕೆಂಬ
ನೋಡ-ಲ-ಸಾ-ಧ್ಯನೂ
ನೋಡ-ಲಾಗಿ
ನೋಡ-ಲಾರ
ನೋಡ-ಲಾ-ರನು
ನೋಡ-ಲಾ-ರರು
ನೋಡ-ಲಾ-ರವು
ನೋಡ-ಲಾರೆ
ನೋಡಲಿ
ನೋಡ-ಲಿಲ್ಲ
ನೋಡಲು
ನೋಡಿ
ನೋಡಿಕೊ
ನೋಡಿ-ಕೊಂ-ಡರೆ
ನೋಡಿ-ಕೊಂ-ಡ-ವನು
ನೋಡಿ-ಕೊಂ-ಡಿತು
ನೋಡಿ-ಕೊಂಡು
ನೋಡಿ-ಕೊಳ್ಳ
ನೋಡಿ-ಕೊ-ಳ್ಳ-ದಂತೆ
ನೋಡಿ-ಕೊ-ಳ್ಳ-ಬ-ಹುದು
ನೋಡಿ-ಕೊ-ಳ್ಳ-ಬೇ-ಕಾ-ಗಿದೆ
ನೋಡಿ-ಕೊ-ಳ್ಳ-ಬೇಕು
ನೋಡಿ-ಕೊ-ಳ್ಳ-ಬೇ-ಕೆಂದು
ನೋಡಿ-ಕೊ-ಳ್ಳ-ವನು
ನೋಡಿ-ಕೊಳ್ಳು
ನೋಡಿ-ಕೊ-ಳ್ಳು-ತ್ತಾನೆ
ನೋಡಿ-ಕೊ-ಳ್ಳು-ತ್ತಿ-ದ್ದರು
ನೋಡಿ-ಕೊ-ಳ್ಳು-ತ್ತಿ-ರು-ತ್ತೇವೆ
ನೋಡಿ-ಕೊ-ಳ್ಳು-ತ್ತಿ-ರು-ವನು
ನೋಡಿ-ಕೊ-ಳ್ಳು-ತ್ತೇ-ವೆಯೆ
ನೋಡಿ-ಕೊ-ಳ್ಳು-ವನು
ನೋಡಿ-ಕೊ-ಳ್ಳು-ವು-ದಕ್ಕೆ
ನೋಡಿ-ಕೊ-ಳ್ಳು-ವು-ದಲ್ಲ
ನೋಡಿ-ಕೊ-ಳ್ಳು-ವೆ-ನೆಂದು
ನೋಡಿ-ಕೊ-ಳ್ಳೋಣ
ನೋಡಿದ
ನೋಡಿ-ದಂತೆ
ನೋಡಿ-ದನು
ನೋಡಿ-ದನೊ
ನೋಡಿ-ದ-ಮೇಲೆ
ನೋಡಿ-ದರು
ನೋಡಿ-ದರೂ
ನೋಡಿ-ದರೆ
ನೋಡಿ-ದರೇ
ನೋಡಿ-ದ-ರೇನೇ
ನೋಡಿ-ದ-ಲ್ಲದೆ
ನೋಡಿ-ದಳು
ನೋಡಿ-ದ-ವ-ನಿಗೆ
ನೋಡಿ-ದ-ವನು
ನೋಡಿ-ದ-ವ-ರಾರೂ
ನೋಡಿ-ದ-ವರು
ನೋಡಿ-ದ-ವರೂ
ನೋಡಿ-ದಾಗ
ನೋಡಿ-ದಾ-ಗಲೂ
ನೋಡಿ-ದಾ-ಗಲೆ
ನೋಡಿ-ದಾ-ಗಲೇ
ನೋಡಿ-ದೆ-ಡೆ-ಯ-ಲ್ಲೆಲ್ಲ
ನೋಡಿ-ದೆಯೊ
ನೋಡಿ-ದೆವು
ನೋಡಿ-ದೊ-ಡನೆ
ನೋಡಿದ್ದ
ನೋಡಿ-ದ್ದನ್ನೇ
ನೋಡಿ-ದ್ದ-ರಲ್ಲಿ
ನೋಡಿ-ದ್ದಾರೆ
ನೋಡಿದ್ದು
ನೋಡಿ-ದ್ದೇನೆ
ನೋಡಿಯೂ
ನೋಡಿ-ರ-ಲಿ-ಕ್ಕಿಲ್ಲ
ನೋಡಿ-ರ-ಲಿಲ್ಲ
ನೋಡಿ-ರು-ವಂ-ತಹ
ನೋಡಿ-ರು-ವನು
ನೋಡಿ-ರು-ವರೋ
ನೋಡಿ-ರು-ವ-ವನು
ನೋಡಿ-ರು-ವೆಯೊ
ನೋಡಿ-ರು-ವೆವು
ನೋಡಿಲ್ಲ
ನೋಡು
ನೋಡುತ್ತ
ನೋಡು-ತ್ತದೆ
ನೋಡುತ್ತಾ
ನೋಡು-ತ್ತಾನೆ
ನೋಡು-ತ್ತಾ-ನೆಯೆ
ನೋಡು-ತ್ತಾ-ನೆಯೇ
ನೋಡು-ತ್ತಾ-ನೆಯೊ
ನೋಡು-ತ್ತಾ-ನೆಯೋ
ನೋಡು-ತ್ತಾನೊ
ನೋಡು-ತ್ತಾರೆ
ನೋಡು-ತ್ತಾ-ರೆಯೇ
ನೋಡು-ತ್ತಾ-ರೆಯೊ
ನೋಡು-ತ್ತಾ-ರೆಯೋ
ನೋಡು-ತ್ತಾರೊ
ನೋಡು-ತ್ತಿತ್ತೊ
ನೋಡು-ತ್ತಿದೆ
ನೋಡು-ತ್ತಿದ್ದ
ನೋಡು-ತ್ತಿ-ದ್ದಂತೆ
ನೋಡು-ತ್ತಿ-ದ್ದರು
ನೋಡು-ತ್ತಿ-ದ್ದರೂ
ನೋಡು-ತ್ತಿ-ದ್ದ-ವನು
ನೋಡು-ತ್ತಿ-ದ್ದ-ವರು
ನೋಡು-ತ್ತಿ-ದ್ದಾರೆ
ನೋಡು-ತ್ತಿ-ದ್ದೆನೋ
ನೋಡು-ತ್ತಿ-ದ್ದೇನೆ
ನೋಡು-ತ್ತಿ-ದ್ದೇವೆ
ನೋಡು-ತ್ತಿ-ರ-ಬ-ಹುದು
ನೋಡು-ತ್ತಿ-ರ-ಬಾ-ರದು
ನೋಡು-ತ್ತಿ-ರ-ಬೇ-ಕಾ-ಗು-ವುದು
ನೋಡು-ತ್ತಿ-ರ-ವ-ವ-ರಲ್ಲ
ನೋಡು-ತ್ತಿರು
ನೋಡು-ತ್ತಿ-ರು-ತ್ತೇನೆ
ನೋಡು-ತ್ತಿ-ರು-ವನು
ನೋಡು-ತ್ತಿ-ರು-ವನೆ
ನೋಡು-ತ್ತಿ-ರು-ವನೇ
ನೋಡು-ತ್ತಿ-ರು-ವನೋ
ನೋಡು-ತ್ತಿ-ರು-ವರು
ನೋಡು-ತ್ತಿ-ರು-ವ-ವ-ನಿಗೆ
ನೋಡು-ತ್ತಿ-ರು-ವ-ವನು
ನೋಡು-ತ್ತಿ-ರು-ವಾಗ
ನೋಡು-ತ್ತಿ-ರು-ವುದು
ನೋಡು-ತ್ತಿ-ರುವೆ
ನೋಡು-ತ್ತಿ-ರು-ವೆನು
ನೋಡು-ತ್ತಿ-ರು-ವೆವು
ನೋಡು-ತ್ತಿ-ರು-ವೆವೊ
ನೋಡು-ತ್ತಿ-ರು-ವೆವೋ
ನೋಡು-ತ್ತಿಲ್ಲ
ನೋಡು-ತ್ತಿವೆ
ನೋಡು-ತ್ತೀಯೆ
ನೋಡು-ತ್ತೇನೆ
ನೋಡು-ತ್ತೇವೆ
ನೋಡು-ತ್ತೇ-ವೆಯೊ
ನೋಡು-ತ್ತೇ-ವೆಯೋ
ನೋಡು-ತ್ತೇವೊ
ನೋಡುವ
ನೋಡು-ವಂತೆ
ನೋಡು-ವನು
ನೋಡು-ವನೆ
ನೋಡು-ವನೊ
ನೋಡು-ವನೋ
ನೋಡು-ವರು
ನೋಡು-ವರೊ
ನೋಡು-ವ-ವನ
ನೋಡು-ವ-ವ-ನಿಗೆ
ನೋಡು-ವ-ವನು
ನೋಡು-ವ-ವನೂ
ನೋಡು-ವ-ವನೆ
ನೋಡು-ವ-ವನೇ
ನೋಡು-ವ-ವ-ರಿಗೆ
ನೋಡು-ವ-ವರು
ನೋಡು-ವಾಗ
ನೋಡು-ವಾ-ಗಲೂ
ನೋಡುವು
ನೋಡು-ವು-ದ-ಕ್ಕಾ-ಗಲಿ
ನೋಡು-ವು-ದ-ಕ್ಕಾಗಿ
ನೋಡು-ವು-ದ-ಕ್ಕಿಂತ
ನೋಡು-ವು-ದಕ್ಕೆ
ನೋಡು-ವುದನ್ನು
ನೋಡು-ವು-ದರ
ನೋಡು-ವು-ದ-ರಿಂದ
ನೋಡು-ವು-ದ-ರೊ-ಳಗೆ
ನೋಡು-ವು-ದಾ-ಗಿದೆ
ನೋಡು-ವು-ದಿಲ್ಲ
ನೋಡು-ವು-ದಿ-ಲ್ಲವೋ
ನೋಡು-ವುದು
ನೋಡು-ವು-ದೆಂ-ದರೆ
ನೋಡು-ವು-ದೆಲ್ಲ
ನೋಡು-ವು-ದೆ-ಲ್ಲವೂ
ನೋಡು-ವುದೇ
ನೋಡುವೆ
ನೋಡು-ವೆನು
ನೋಡು-ವೆವು
ನೋಡು-ವೆವೊ
ನೋಡು-ವೆವೋ
ನೋಡೋಣ
ನೋದ್ವಿ-ಜೇತ್
ನೋಪ-ಜಾ-ಯಂತೇ
ನೋಪ-ಪ-ದ್ಯತೇ
ನೋಪ-ಲಿ-ಪ್ಯತೇ
ನೋಯಿ-ಸ-ಬೇ-ಕಾ-ಗು-ವುದು
ನೋಯಿ-ಸಲು
ನೋಯಿಸಿ
ನೋಯಿ-ಸಿ-ರ-ಬ-ಹುದು
ನೋಯಿ-ಸು-ವನು
ನೋಯಿ-ಸು-ವು-ದ-ಕ್ಕಲ್ಲ
ನೋಯಿ-ಸು-ವು-ದಕ್ಕೆ
ನೋಯಿ-ಸು-ವು-ದಿಲ್ಲ
ನೋಯಿ-ಸು-ವುದೆ
ನೋವ-ನ್ನಾ-ದರೂ
ನೋವನ್ನು
ನೋವನ್ನೇ
ನೋವಾ-ಗದ
ನೋವಾ-ಗು-ವುದು
ನೋವಿ
ನೋವಿಗೆ
ನೋವಿನ
ನೋವು
ನೋವೂ
ನೌಕರ
ನೌಕಾ
ನೌಕಾ-ವಿ-ಹಾ-ರಕ್ಕೆ
ನೌಕೆ
ನ್ನಾಗಿ
ನ್ನಾದರೂ
ನ್ನಾಯಂ
ನ್ನೆಲ್ಲ
ನ್ನೆಲ್ಲಾ
ನ್ಯನ್ಯಾನಿ
ನ್ಯಾಯ
ನ್ಯಾಯ-ವಾಗಿ
ನ್ಯಾಯ-ವಾ-ಗಿದೆ
ನ್ಯಾಯ-ವಾ-ಗಿಯೋ
ನ್ಯಾಯವೊ
ನ್ಯಾಯವೋ
ನ್ಯಾಯ-ಸ್ಥಾ-ನ-ದ-ಲ್ಲಿಯೇ
ನ್ಯಾಯಾ-ಧಿ-ಪತಿ
ನ್ಯಾಯಾ-ಧಿ-ಪ-ತಿಗೆ
ನ್ಯಾಯಾ-ಧಿ-ಪ-ತಿಯು
ನ್ಯಾಯಾ-ಲಯ
ನ್ಯಾಯಾ-ಲ-ಯ-ಗಳು
ನ್ಯಾಯಾ-ಸ್ಥಾನ
ನ್ಯಾಯ್ಯಂ
ನ್ಯಾಸಂ
ನ್ಯೂಟನ್
ನ್ಯೂಟ-ನ್ನ-ನನ್ನು
ನ್ಯೂನತೆ
ನ್ಯೂನ-ತೆ-ಗಳಿಂದ
ನ್ಯೂನ-ತೆ-ಗಳು
ನ್ಯೂನ-ತೆ-ಯನ್ನು
ಪ
ಪಂಕ್ತಿ
ಪಂಕ್ತಿ-ಯನ್ನೇ
ಪಂಕ್ತಿ-ಯಲ್ಲಿ
ಪಂಗ-ಡ-ದ-ವರು
ಪಂಗುಂ
ಪಂಚ
ಪಂಚ-ಕೋ-ಶ-ಗ-ಳೆಂಬ
ಪಂಚ-ಪಾತ್ರೆ
ಪಂಚ-ಭೂತ
ಪಂಚ-ಭೂ-ತ-ಗಳ
ಪಂಚ-ಭೂ-ತ-ಗಳಲ್ಲಿ
ಪಂಚ-ಭೂ-ತ-ಗಳಿಂದ
ಪಂಚ-ಭೂ-ತ-ಗ-ಳಿಂ-ದಾ-ಗಿವೆ
ಪಂಚ-ಭೂ-ತ-ಗ-ಳಿಂ-ದಾದ
ಪಂಚ-ಭೂ-ತ-ಗ-ಳಿಗೆ
ಪಂಚ-ಭೂ-ತ-ಗ-ಳಿವೆ
ಪಂಚ-ಭೂ-ತ-ಗಳು
ಪಂಚ-ಭೂ-ತ-ಗಳೂ
ಪಂಚ-ಮಮ್
ಪಂಚ-ಮ-ವೇದ
ಪಂಚ-ಯ-ಜ್ಞ-ಗಳನ್ನು
ಪಂಚ-ವಟಿ
ಪಂಚೆ
ಪಂಚೆ-ಯನ್ನೂ
ಪಂಚೇಂ-ದ್ರಿಯ
ಪಂಚೇಂ-ದ್ರಿ-ಯ-ಗಳ
ಪಂಚೇಂ-ದ್ರಿ-ಯ-ಗಳನ್ನೆಲ್ಲ
ಪಂಚೇಂ-ದ್ರಿ-ಯ-ಗಳಲ್ಲಿ
ಪಂಚೇಂ-ದ್ರಿ-ಯ-ಗ-ಳಿಗೆ
ಪಂಚೇಂ-ದ್ರಿ-ಯ-ಗ-ಳಿವೆ
ಪಂಚೇಂ-ದ್ರಿ-ಯ-ಗಳು
ಪಂಚೇಂ-ದ್ರಿ-ಯದ
ಪಂಚೈ-ತಾನಿ
ಪಂಚೈತೇ
ಪಂಜರ
ಪಂಜ-ರದ
ಪಂಜ-ರ-ದಲ್ಲಿ
ಪಂಜ-ರ-ದ-ಲ್ಲಿ-ರ-ಲಾ-ರನು
ಪಂಜ-ರ-ದಿಂದ
ಪಂಜ-ರ-ವನ್ನು
ಪಂಜಾ-ಬಿನ
ಪಂಜಿ-ನಂತೆ
ಪಂಡಿತ
ಪಂಡಿತಂ
ಪಂಡಿ-ತನ
ಪಂಡಿ-ತ-ನನ್ನು
ಪಂಡಿ-ತ-ನಾಗಿ
ಪಂಡಿ-ತ-ನಾ-ಗಿ-ರಲಿ
ಪಂಡಿ-ತ-ನಾ-ದ-ವ-ನಿಗೆ
ಪಂಡಿ-ತ-ನಿ-ಗಿಂತ
ಪಂಡಿ-ತ-ನಿಗೆ
ಪಂಡಿ-ತ-ನಿ-ರು-ವನು
ಪಂಡಿ-ತನು
ಪಂಡಿ-ತನೂ
ಪಂಡಿ-ತ-ನೊಂ-ದಿಗೆ
ಪಂಡಿ-ತ-ರಂತೆ
ಪಂಡಿ-ತರು
ಪಂಡಿ-ತಾಃ
ಪಕ-ನಂತೆ
ಪಕ್ಕದ
ಪಕ್ಕ-ದಲ್ಲಿ
ಪಕ್ಕ-ದ-ಲ್ಲಿಯೇ
ಪಕ್ವ-ವಾ-ಗು-ವು-ದಿಲ್ಲ
ಪಕ್ವ-ಸ್ಥಿ-ತಿಗೆ
ಪಕ್ಷ
ಪಕ್ಷದ
ಪಕ್ಷ-ದಲ್ಲಿ
ಪಕ್ಷ-ದ-ಲ್ಲಿ-ದ್ದಾರೆ
ಪಕ್ಷ-ದ-ಲ್ಲಿಯೂ
ಪಕ್ಷ-ದ-ಲ್ಲಿ-ರುವ
ಪಕ್ಷ-ದ-ಲ್ಲಿ-ರು-ವರು
ಪಕ್ಷ-ದ-ವ-ರನ್ನೂ
ಪಕ್ಷ-ಪಾ-ತದ
ಪಕ್ಷ-ಪಾ-ತ-ವಿದೆ
ಪಕ್ಷ-ಪಾ-ತ-ವಿ-ದ್ದರೆ
ಪಕ್ಷ-ಪಾ-ತ-ವಿಲ್ಲ
ಪಕ್ಷ-ಪಾ-ತ-ವಿ-ಲ್ಲದೆ
ಪಕ್ಷ-ಪಾ-ತ-ವಿ-ಲ್ಲದೇ
ಪಕ್ಷ-ಪಾ-ತವೆ
ಪಕ್ಷ-ಪಾತಿ
ಪಕ್ಷ-ಪಾ-ತಿ-ಗಳು
ಪಕ್ಷ-ಪಾ-ತಿ-ಯಲ್ಲ
ಪಕ್ಷ-ಪಾ-ತಿ-ಯಾ-ಗು-ತ್ತಾನೆ
ಪಕ್ಷ-ಪಾ-ತಿ-ಯಾ-ದ-ನಲ್ಲ
ಪಕ್ಷ-ಪಾ-ತಿ-ಯೆಂದು
ಪಕ್ಷ-ವ-ನ್ನಾ-ದರೂ
ಪಕ್ಷ-ವನ್ನೂ
ಪಕ್ಷಿ
ಪಕ್ಷಿ-ಕೀ-ಟ-ಗ-ಳಾ-ಗಲಿ
ಪಕ್ಷಿ-ಗಳಲ್ಲಿ
ಪಕ್ಷಿ-ಗ-ಳಿ-ಗೆಲ್ಲ
ಪಕ್ಷಿ-ಗಳು
ಪಕ್ಷಿ-ಣಾಮ್
ಪಕ್ಷಿ-ರಾಜ
ಪಗ-ಡೆ-ಕಾಯಿ
ಪಚಂ-ತ್ಯಾ-ತ್ಮ-ಕಾರಣಾತ್
ಪಚಾ-ಮ್ಯನ್ನಂ
ಪಟ
ಪಟ-ಗ-ಳಂತೆ
ಪಟ-ವನ್ನು
ಪಟು-ತ್ವ-ವನ್ನು
ಪಟು-ವಾ-ಗಿಯೇ
ಪಟ್ಟ
ಪಟ್ಟರು
ಪಟ್ಟರೆ
ಪಟ್ಟ-ವರು
ಪಟ್ಟಿ-ದ್ದೇವೆ
ಪಟ್ಟಿ-ಯನ್ನು
ಪಟ್ಟಿ-ರು-ವನು
ಪಟ್ಟಿ-ರು-ವೆನು
ಪಟ್ಟು
ಪಟ್ಟು-ಕೊಂಡು
ಪಡ
ಪಡ-ಕೂ-ಡದು
ಪಡದೆ
ಪಡದೇ
ಪಡ-ಪೋಸಿ
ಪಡ-ಬ-ಹುದು
ಪಡ-ಬಾ-ರದ
ಪಡ-ಬಾ-ರದು
ಪಡ-ಬೇ-ಕಾ-ಗಿದೆ
ಪಡ-ಬೇ-ಕಾ-ಗಿಲ್ಲ
ಪಡ-ಬೇ-ಕಾ-ಗಿ-ಲ್ಲ-ವಲ್ಲ
ಪಡ-ಬೇ-ಕಾ-ಗು-ವುದು
ಪಡ-ಬೇ-ಕಾ-ದರೆ
ಪಡ-ಬೇಕು
ಪಡ-ಬೇಡ
ಪಡ-ಬೇಡಿ
ಪಡಲು
ಪಡ-ವ-ಲ-ಕಾಯಿ
ಪಡ-ಸಾಲೆ
ಪಡ-ಸಾ-ಲೆ-ಯಲ್ಲಿ
ಪಡಿ-ಸ-ಬೇ-ಕಾ-ಗಿದೆ
ಪಡಿಸಿ
ಪಡಿ-ಸಿ-ಕೊ-ಳ್ಳಲಿ
ಪಡಿ-ಸಿ-ಕೊ-ಳ್ಳಲು
ಪಡಿ-ಸಿ-ಕೊಳ್ಳು
ಪಡಿ-ಸಿ-ಕೊ-ಳ್ಳು-ವು-ದ-ಕ್ಕಾಗಿ
ಪಡಿ-ಸಿ-ಕೊ-ಳ್ಳು-ವು-ದಕ್ಕೆ
ಪಡಿ-ಸಿ-ದಷ್ಟು
ಪಡಿ-ಸಿ-ದಾ-ಗಲೂ
ಪಡಿ-ಸು-ತ್ತಾ-ನೆಯೋ
ಪಡಿ-ಸು-ವು-ದಕ್ಕೆ
ಪಡಿ-ಸು-ವು-ದಾ-ದರೂ
ಪಡಿ-ಸು-ವುದೇ
ಪಡು
ಪಡು-ತ್ತಾನೆ
ಪಡು-ತ್ತಾರೆ
ಪಡು-ತ್ತಿದ್ದ
ಪಡು-ತ್ತಿ-ದ್ದರೆ
ಪಡು-ತ್ತಿ-ರು-ವನು
ಪಡು-ತ್ತಿ-ರು-ವರು
ಪಡು-ತ್ತೇವೆ
ಪಡು-ತ್ತೇ-ವೆಯೊ
ಪಡು-ತ್ತೇ-ವೆಯೋ
ಪಡು-ವನು
ಪಡು-ವರು
ಪಡು-ವ-ವ-ನಲ್ಲ
ಪಡು-ವ-ವ-ನಿಗೆ
ಪಡು-ವ-ವ-ರಲ್ಲಿ
ಪಡು-ವ-ವ-ರಿಗೆ
ಪಡು-ವ-ವರು
ಪಡು-ವಾಗ
ಪಡು-ವಿರಾ
ಪಡುವು
ಪಡು-ವು-ದ-ಕ್ಕಿಂತ
ಪಡು-ವು-ದಕ್ಕೆ
ಪಡು-ವು-ದಿಲ್ಲ
ಪಡು-ವುದು
ಪಡು-ವುದೂ
ಪಡು-ವು-ದೆಲ್ಲ
ಪಡು-ವು-ದೇ-ನಿದೆ
ಪಡು-ವು-ದೇನು
ಪಡು-ವೆವು
ಪಡೆ
ಪಡೆದ
ಪಡೆ-ದಂತೆ
ಪಡೆ-ದ-ದ್ದಾ-ವುದೂ
ಪಡೆ-ದ-ಮೇಲೂ
ಪಡೆ-ದ-ಮೇಲೆ
ಪಡೆ-ದರು
ಪಡೆ-ದರೂ
ಪಡೆ-ದರೆ
ಪಡೆ-ದ-ವನ
ಪಡೆ-ದ-ವನು
ಪಡೆ-ದ-ವ-ರಲ್ಲಿ
ಪಡೆ-ದ-ವರು
ಪಡೆ-ದಾಗ
ಪಡೆ-ದಾ-ಗಲೆ
ಪಡೆ-ದಾ-ದ-ಮೇಲೆ
ಪಡೆ-ದಿ-ದ್ದನು
ಪಡೆ-ದಿ-ದ್ದರೂ
ಪಡೆ-ದಿ-ದ್ದರೆ
ಪಡೆ-ದಿ-ದ್ದಾನೊ
ಪಡೆ-ದಿ-ದ್ದೇನೆ
ಪಡೆ-ದಿರ
ಪಡೆ-ದಿ-ರ-ಬೇಕು
ಪಡೆ-ದಿ-ರು-ತ್ತಾರೆ
ಪಡೆ-ದಿ-ರು-ತ್ತೇನೆ
ಪಡೆ-ದಿ-ರುವ
ಪಡೆ-ದಿ-ರು-ವನು
ಪಡೆ-ದಿ-ರು-ವನೋ
ಪಡೆ-ದಿ-ರು-ವ-ವನು
ಪಡೆ-ದಿ-ರು-ವುದನ್ನು
ಪಡೆ-ದಿ-ರು-ವುದು
ಪಡೆ-ದಿ-ರು-ವೆವು
ಪಡೆ-ದಿ-ರು-ವೆವೋ
ಪಡೆ-ದಿವೆ
ಪಡೆದು
ಪಡೆ-ದು-ಕೊಂಡ
ಪಡೆ-ದು-ಕೊಂ-ಡರೂ
ಪಡೆ-ದು-ಕೊಂ-ಡರೆ
ಪಡೆ-ದು-ಕೊಂ-ಡ-ವನು
ಪಡೆ-ದು-ಕೊಂ-ಡ-ವರು
ಪಡೆ-ದು-ಕೊಂ-ಡಿ-ದ್ದನೆ
ಪಡೆ-ದು-ಕೊಂ-ಡಿ-ರ-ಬೇಕು
ಪಡೆ-ದು-ಕೊಂ-ಡಿ-ರು-ವಂತೆ
ಪಡೆ-ದು-ಕೊಂ-ಡಿ-ರು-ವನು
ಪಡೆ-ದು-ಕೊಂ-ಡಿ-ರು-ವರೊ
ಪಡೆ-ದು-ಕೊಂ-ಡಿ-ರು-ವು-ದಿಲ್ಲ
ಪಡೆ-ದು-ಕೊಂ-ಡಿ-ರು-ವುದೊ
ಪಡೆ-ದು-ಕೊಂ-ಡಿ-ರು-ವೆವು
ಪಡೆ-ದು-ಕೊಂ-ಡಿ-ರು-ವೆವೊ
ಪಡೆ-ದು-ಕೊಂ-ಡಿಲ್ಲ
ಪಡೆ-ದು-ಕೊಂಡು
ಪಡೆ-ದು-ಕೊ-ಳ್ಳ-ಬೇ-ಕಾ-ಗಿದೆ
ಪಡೆ-ದು-ಕೊ-ಳ್ಳ-ಬೇ-ಕಾ-ಗಿ-ದೆಯೋ
ಪಡೆ-ದು-ಕೊ-ಳ್ಳ-ಬೇ-ಕಾ-ದರೆ
ಪಡೆ-ದು-ಕೊ-ಳ್ಳ-ಬೇಕು
ಪಡೆ-ದು-ಕೊ-ಳ್ಳ-ಬೇಕೊ
ಪಡೆ-ದು-ಕೊಳ್ಳು
ಪಡೆ-ದು-ಕೊ-ಳ್ಳು-ತ್ತಾನೆ
ಪಡೆ-ದು-ಕೊ-ಳ್ಳು-ತ್ತಾ-ನೆಯೊ
ಪಡೆ-ದು-ಕೊ-ಳ್ಳು-ವ-ವನು
ಪಡೆ-ದು-ಕೊ-ಳ್ಳು-ವು-ದಕ್ಕೆ
ಪಡೆ-ದು-ಕೊ-ಳ್ಳು-ವು-ದಲ್ಲ
ಪಡೆ-ದು-ಕೊ-ಳ್ಳು-ವುದು
ಪಡೆದೆ
ಪಡೆ-ದೇನೊ
ಪಡೆಯ
ಪಡೆ-ಯದ
ಪಡೆ-ಯ-ದ-ವರು
ಪಡೆ-ಯದೆ
ಪಡೆ-ಯ-ಬ-ಲ್ಲರು
ಪಡೆ-ಯ-ಬ-ಹುದು
ಪಡೆ-ಯ-ಬ-ಹು-ದೆಂದು
ಪಡೆ-ಯ-ಬ-ಹುದೊ
ಪಡೆ-ಯ-ಬೇ-ಕಾ-ಗಿದೆ
ಪಡೆ-ಯ-ಬೇ-ಕಾ-ಗಿ-ರು-ವುದು
ಪಡೆ-ಯ-ಬೇ-ಕಾ-ಗಿಲ್ಲ
ಪಡೆ-ಯ-ಬೇ-ಕಾದ
ಪಡೆ-ಯ-ಬೇ-ಕಾ-ದದ್ದು
ಪಡೆ-ಯ-ಬೇ-ಕಾ-ದರೂ
ಪಡೆ-ಯ-ಬೇ-ಕಾ-ದರೆ
ಪಡೆ-ಯ-ಬೇ-ಕಾರೆ
ಪಡೆ-ಯ-ಬೇಕು
ಪಡೆ-ಯ-ಬೇ-ಕೆಂ
ಪಡೆ-ಯ-ಬೇ-ಕೆಂ-ದಿ-ರು-ವನೋ
ಪಡೆ-ಯ-ಬೇ-ಕೆಂದು
ಪಡೆ-ಯ-ಬೇ-ಕೆಂಬ
ಪಡೆ-ಯ-ಬೇ-ಕೆಂ-ಬು-ದನ್ನು
ಪಡೆ-ಯ-ಬೇಡಿ
ಪಡೆ-ಯ-ಲಾರ
ಪಡೆ-ಯ-ಲಾ-ರರು
ಪಡೆ-ಯ-ಲಾ-ರೆವು
ಪಡೆ-ಯಲಿ
ಪಡೆ-ಯಲು
ಪಡೆ-ಯಲೇ
ಪಡೆ-ಯ-ಲೇ-ಬೇ-ಕಾ-ದರೆ
ಪಡೆ-ಯಿರಿ
ಪಡೆಯು
ಪಡೆ-ಯು-ತ್ತಾನೆ
ಪಡೆ-ಯು-ತ್ತಾ-ನೆಯೆ
ಪಡೆ-ಯು-ತ್ತಾನೋ
ಪಡೆ-ಯು-ತ್ತಾರೆ
ಪಡೆ-ಯು-ತ್ತಾರೊ
ಪಡೆ-ಯು-ತ್ತಾರೋ
ಪಡೆ-ಯು-ತ್ತಿದ್ದ
ಪಡೆ-ಯು-ತ್ತಿ-ರು-ವನು
ಪಡೆ-ಯು-ತ್ತೀರಿ
ಪಡೆ-ಯು-ತ್ತೇನೆ
ಪಡೆ-ಯು-ತ್ತೇವೆ
ಪಡೆ-ಯು-ತ್ತೇ-ವೆಯೊ
ಪಡೆ-ಯುವ
ಪಡೆ-ಯು-ವಂ-ತಹ
ಪಡೆ-ಯು-ವನು
ಪಡೆ-ಯು-ವನೊ
ಪಡೆ-ಯು-ವನೋ
ಪಡೆ-ಯು-ವರು
ಪಡೆ-ಯು-ವರೊ
ಪಡೆ-ಯು-ವ-ವ-ರೆಗೆ
ಪಡೆ-ಯು-ವವು
ಪಡೆ-ಯು-ವುದ
ಪಡೆ-ಯು-ವು-ದ-ಕ್ಕಲ್ಲ
ಪಡೆ-ಯು-ವು-ದ-ಕ್ಕಾಗಿ
ಪಡೆ-ಯು-ವು-ದ-ಕ್ಕಿಂತ
ಪಡೆ-ಯು-ವು-ದಕ್ಕೆ
ಪಡೆ-ಯು-ವು-ದಕ್ಕೊ
ಪಡೆ-ಯು-ವು-ದ-ರಲ್ಲಿ
ಪಡೆ-ಯು-ವು-ದಿಲ್ಲ
ಪಡೆ-ಯು-ವುದು
ಪಡೆ-ಯು-ವು-ದೆಲ್ಲ
ಪಡೆ-ಯು-ವು-ದೊಂದು
ಪಡೆ-ಯುವೆ
ಪಡೆ-ಯು-ವೆನು
ಪಡೆಯೇ
ಪಣ-ವಾ-ನ-ಕ-ಗೋ-ಮು-ಖಾಃ
ಪತಂಗ
ಪತಂ-ಗಕ್ಕೆ
ಪತಂ-ಗ-ಗ-ಳಂತೆ
ಪತಂ-ಗ-ಗಳು
ಪತಂ-ಗ-ದಂತೆ
ಪತಂಗಾ
ಪತಂ-ಜಲಿ
ಪತಂ-ಜ-ಲಿಯ
ಪತಂತಿ
ಪತ-ನಕ್ಕೆ
ಪತಿ-ತ-ನಾಗ
ಪತಿ-ತ-ರಾಗು
ಪತಿ-ಯಂತೆ
ಪತ್ತೆ
ಪತ್ತೆಯೇ
ಪತ್ಯ-ವನ್ನು
ಪತ್ರಂ
ಪಥ
ಪಥ-ದಲ್ಲಿ
ಪಥ-ವನ್ನು
ಪಥ-ವನ್ನೇ
ಪಥಿ
ಪಥಿಕ
ಪಥ್ಯ-ಗಳನ್ನು
ಪಥ್ಯ-ಗ-ಳು-ಇ-ವು-ಗ-ಳಿ-ಗಾಗಿ
ಪಥ್ಯ-ವನ್ನು
ಪದ
ಪದಂ
ಪದ-ಕ-ವನ್ನು
ಪದಕ್ಕೆ
ಪದ-ಕ್ರಮ
ಪದ-ಕ್ರ-ಮೋ-ಪ-ನಿ-ಷ-ದೈ-ರ್ಗಾ-ಯಂತಿ
ಪದ-ಗಳನ್ನು
ಪದ-ಗಳನ್ನೂ
ಪದ-ಗಳಲ್ಲಿ
ಪದ-ಗ-ಳ-ಲ್ಲೆಲ್ಲಾ
ಪದ-ಗಳಿಂದ
ಪದ-ಗಳು
ಪದ-ಗಳೂ
ಪದ-ತಲ
ಪದ-ತ-ಳ-ದಲ್ಲಿ
ಪದದ
ಪದ-ದಲ್ಲಿ
ಪದ-ದ-ಲ್ಲಿದೆ
ಪದ-ದಿಂದ
ಪದ-ಮ-ವ್ಯಯಂ
ಪದ-ಮ-ವ್ಯ-ಯಮ್
ಪದರ
ಪದ-ರ-ಗ-ಳಾಗಿ
ಪದ-ವನ್ನು
ಪದ-ವಾ-ಗಿತ್ತು
ಪದವಿ
ಪದ-ವಿ-ಯನ್ನು
ಪದ-ವಿ-ಯನ್ನೂ
ಪದ-ವಿಯೋ
ಪದವೂ
ಪದಾ-ತಿ-ಗ-ಳೊ-ಡನೆ
ಪದಾ-ರ್ಥಕ್ಕೆ
ಪದಾ-ರ್ಥ-ಗ-ಳ-ನ್ನು-ತ-ಯಾ-ರು-ಮಾಡಿ
ಪದಾ-ರ್ಥ-ಗ-ಳೆಲ್ಲ
ಪದೇ
ಪದೇ-ಪದೇ
ಪದ್ಮ-ಪ-ತ್ರ-ಮಿ-ವಾಂ-ಭಸಾ
ಪಯ-ಣ-ದಲ್ಲಿ
ಪರ
ಪರಂ
ಪರಂ-ಜ್ಯೋತಿ
ಪರಂ-ತಪ
ಪರಂ-ತಪಃ
ಪರಂ-ತ-ಪ-ನಾದ
ಪರಂ-ಧಾಮ
ಪರಂ-ಧಾ-ಮ-ವನ್ನು
ಪರಂ-ಪ-ರಾ-ಗ-ತ-ವಾಗಿ
ಪರಂ-ಪ-ರಾ-ಪ್ರಾ-ಪ್ತ-ಮಿಮಂ
ಪರಂ-ಪ-ರೆಗೆ
ಪರಂ-ಪ-ರೆ-ಯಿಂದ
ಪರಃ
ಪರ-ಕೀ-ಯರು
ಪರ-ಜ್ಞಾನ
ಪರ-ತಂ-ತ್ರಳು
ಪರ-ತರಂ
ಪರ-ತಸ್ತು
ಪರ-ದಲ್ಲಿ
ಪರದೆ
ಪರ-ದೆಯ
ಪರ-ದೆ-ಯ-ಮೇಲೆ
ಪರ-ಧರ್ಮ
ಪರ-ಧ-ರ್ಮ-ಕ್ಕಿಂತ
ಪರ-ಧ-ರ್ಮಾತ್
ಪರ-ಧ-ರ್ಮಾ-ತ್ಸ್ವ-ನು-ಷ್ಠಿ-ತಾತ್
ಪರ-ಧರ್ಮೋ
ಪರ-ಬ್ರಹ್ಮ
ಪರ-ಬ್ರ-ಹ್ಮದ
ಪರ-ಬ್ರ-ಹ್ಮನ
ಪರ-ಬ್ರ-ಹ್ಮ-ನನ್ನು
ಪರ-ಬ್ರ-ಹ್ಮ-ನಲ್ಲ
ಪರ-ಬ್ರ-ಹ್ಮ-ನಲ್ಲಿ
ಪರ-ಬ್ರ-ಹ್ಮ-ನಿಂದ
ಪರ-ಬ್ರ-ಹ್ಮ-ನಿಗೆ
ಪರ-ಬ್ರ-ಹ್ಮನೇ
ಪರ-ಬ್ರ-ಹ್ಮ-ನೊ-ಬ್ಬನೆ
ಪರ-ಬ್ರ-ಹ್ಮ-ವನ್ನು
ಪರ-ಬ್ರ-ಹ್ಮ-ವ-ಸ್ತುವೆ
ಪರ-ಬ್ರ-ಹ್ಮ-ವೊಂದೇ
ಪರ-ಭಕ್ತಿ
ಪರ-ಭಾ-ವ-ವನ್ನು
ಪರಮ
ಪರಮಂ
ಪರ-ಮ-ಆ-ತ್ಮ-ನನ್ನು
ಪರ-ಮ-ಗತಿ
ಪರ-ಮ-ಗ-ತಿ-ಯನ್ನು
ಪರ-ಮ-ಗ-ತಿ-ಯಾದ
ಪರ-ಮ-ಗ-ತಿ-ಯೆಂದು
ಪರ-ಮ-ಗುರಿ
ಪರ-ಮ-ಗು-ರಿ-ಯನ್ನು
ಪರ-ಮ-ಜ್ಞಾನ
ಪರ-ಮ-ಜ್ಞಾ-ನಿ-ಯಾ-ಗಿದ್ದ
ಪರ-ಮ-ಧಾಮ
ಪರ-ಮ-ಪ-ದ-ವನ್ನು
ಪರ-ಮ-ಪ-ವಿತ್ರ
ಪರ-ಮ-ಪು-ರುಷ
ಪರ-ಮ-ಪು-ರು-ಷ-ನನ್ನು
ಪರ-ಮ-ಪು-ರು-ಷನು
ಪರ-ಮ-ಪು-ರು-ಷ-ನೊ-ಬ್ಬನೇ
ಪರ-ಮ-ಭಕ್ತ
ಪರ-ಮ-ಭೋ-ಗಿ-ಗ-ಳಾಗಿ
ಪರ-ಮ-ವಾ-ಪ್ಸ್ಯಥ
ಪರ-ಮ-ವ್ಯ-ಕ್ತಿಯೇ
ಪರ-ಮ-ವ್ಯ-ಯಮ್
ಪರ-ಮ-ಶಾಂತಿ
ಪರ-ಮ-ಶಾಂ-ತಿ-ಯನ್ನು
ಪರ-ಮ-ಶಿ-ವನ
ಪರ-ಮ-ಶಿಷ್ಯ
ಪರ-ಮ-ಶ್ರದ್ಧೆ
ಪರ-ಮ-ಶ್ರ-ದ್ಧೆ-ಯಿಂದ
ಪರ-ಮ-ಶ್ರೇಷ್ಠ
ಪರ-ಮ-ಸ-ತ್ಯಕ್ಕೆ
ಪರ-ಮ-ಸ-ತ್ಯದ
ಪರ-ಮ-ಸ-ತ್ಯ-ವನ್ನು
ಪರ-ಮ-ಸ-ತ್ಯ-ವೆಂದು
ಪರ-ಮ-ಸುಖ
ಪರ-ಮ-ಸುಖಿ
ಪರ-ಮ-ಸ್ಥಾ-ನ-ವನ್ನು
ಪರ-ಮ-ಹಂ-ಸರ
ಪರ-ಮ-ಹಂ-ಸ-ರಿಗೆ
ಪರ-ಮ-ಹಂ-ಸ-ರೆಂಬ
ಪರಮಾ
ಪರ-ಮಾಂ
ಪರ-ಮಾತ್ಮ
ಪರ-ಮಾ-ತ್ಮ-ದೃಷ್ಟಿ
ಪರ-ಮಾ-ತ್ಮ-ದೃ-ಷ್ಟಿ-ಯಿಂದ
ಪರ-ಮಾ-ತ್ಮನ
ಪರ-ಮಾ-ತ್ಮ-ನದು
ಪರ-ಮಾ-ತ್ಮ-ನದೇ
ಪರ-ಮಾ-ತ್ಮ-ನನ್ನು
ಪರ-ಮಾ-ತ್ಮ-ನ-ಮೇಲೆ
ಪರ-ಮಾ-ತ್ಮ-ನ-ಲ್ಲದ
ಪರ-ಮಾ-ತ್ಮ-ನಲ್ಲಿ
ಪರ-ಮಾ-ತ್ಮ-ನ-ಲ್ಲಿ-ಡ-ಬೇಕು
ಪರ-ಮಾ-ತ್ಮ-ನ-ಲ್ಲಿಯೇ
ಪರ-ಮಾ-ತ್ಮ-ನ-ಲ್ಲಿವೆ
ಪರ-ಮಾ-ತ್ಮ-ನಲ್ಲೆ
ಪರ-ಮಾ-ತ್ಮ-ನಲ್ಲೇ
ಪರ-ಮಾ-ತ್ಮ-ನಾ-ಗಲಿ
ಪರ-ಮಾ-ತ್ಮ-ನಾ-ದರೊ
ಪರ-ಮಾ-ತ್ಮ-ನಾ-ದರೋ
ಪರ-ಮಾ-ತ್ಮ-ನಿಂದ
ಪರ-ಮಾ-ತ್ಮ-ನಿಂ-ದಲೇ
ಪರ-ಮಾ-ತ್ಮ-ನಿ-ಗಾ-ದರೊ
ಪರ-ಮಾ-ತ್ಮ-ನಿಗೂ
ಪರ-ಮಾ-ತ್ಮ-ನಿಗೆ
ಪರ-ಮಾ-ತ್ಮ-ನಿ-ರು-ವನು
ಪರ-ಮಾ-ತ್ಮ-ನಿ-ರು-ವು-ದ-ರಿಂದ
ಪರ-ಮಾ-ತ್ಮ-ನಿ-ರು-ವು-ದ-ರಿಂ-ದಲೇ
ಪರ-ಮಾ-ತ್ಮನು
ಪರ-ಮಾ-ತ್ಮನೂ
ಪರ-ಮಾ-ತ್ಮನೆ
ಪರ-ಮಾ-ತ್ಮ-ನೆಂದು
ಪರ-ಮಾ-ತ್ಮ-ನೆಂಬ
ಪರ-ಮಾ-ತ್ಮ-ನೆ-ಡೆಗೆ
ಪರ-ಮಾ-ತ್ಮನೇ
ಪರ-ಮಾ-ತ್ಮ-ನೊಂ-ದಿಗೆ
ಪರ-ಮಾ-ತ್ಮ-ನೊ-ಡನೆ
ಪರ-ಮಾ-ತ್ಮ-ನೊ-ಬ್ಬ-ನಲ್ಲಿ
ಪರ-ಮಾ-ತ್ಮ-ನೊ-ಬ್ಬನು
ಪರ-ಮಾ-ತ್ಮ-ನೊ-ಬ್ಬನೆ
ಪರ-ಮಾ-ತ್ಮ-ನೊ-ಬ್ಬನೇ
ಪರ-ಮಾ-ತ್ಮ-ವಸ್ತು
ಪರ-ಮಾ-ತ್ಮ-ವ-ಸ್ತು-ವನ್ನು
ಪರ-ಮಾ-ತ್ಮ-ವ-ಸ್ತು-ವಿನ
ಪರ-ಮಾ-ತ್ಮ-ಶಕ್ತಿ
ಪರ-ಮಾತ್ಮಾ
ಪರ-ಮಾ-ತ್ಮಾ-ನಂ-ದ-ವದು
ಪರ-ಮಾ-ತ್ಮೇತಿ
ಪರ-ಮಾ-ತ್ಮೇ-ತ್ಯು-ದಾ-ಹೃತಃ
ಪರ-ಮಾ-ನಂದ
ಪರ-ಮಾ-ನಂ-ದ-ಮಾ-ಧ-ವಮ್
ಪರ-ಮಾ-ನಂ-ದ-ವನ್ನು
ಪರ-ಮಾ-ನಂ-ದ-ವಾ-ಗು-ವುದು
ಪರ-ಮಾಪ್ತ
ಪರ-ಮಾ-ಪ್ತ-ವಾದ
ಪರ-ಮಾ-ಪ್ನೋತಿ
ಪರ-ಮಾರ್ಥ
ಪರ-ಮಾ-ರ್ಥದ
ಪರ-ಮಾ-ರ್ಥ-ದಲ್ಲಿ
ಪರ-ಮಾ-ರ್ಥ-ವಾ-ಗಲಿ
ಪರ-ಮಾ-ವ-ಧಿ-ಯಲ್ಲ
ಪರ-ಮು-ಚ್ಯತೇ
ಪರ-ಮೇ-ಶ್ವರ
ಪರ-ಮೇ-ಶ್ವ-ರನ
ಪರ-ಮೇ-ಶ್ವ-ರ-ನಂತೆ
ಪರ-ಮೇ-ಶ್ವ-ರ-ನನ್ನು
ಪರ-ಮೇ-ಶ್ವ-ರ-ನಾ-ಗಿ-ದ್ದಾನೆ
ಪರ-ಮೇ-ಶ್ವ-ರ-ನಿಗೂ
ಪರ-ಮೇ-ಶ್ವ-ರನು
ಪರ-ಮೇ-ಶ್ವ-ರಮ್
ಪರ-ಮೇ-ಷ್ವಾಸಃ
ಪರಮೋ
ಪರಮ್
ಪರಯಾ
ಪರ-ಯಾ-ವಿಷ್ಟೋ
ಪರ-ಯೋ-ಪೇ-ತಾಸ್ತೇ
ಪರರ
ಪರ-ರನ್ನು
ಪರ-ರಿಗೆ
ಪರ-ಲೋಕ
ಪರ-ಲೋ-ಕಕ್ಕೆ
ಪರ-ಲೋ-ಕ-ಗಳಲ್ಲಿ
ಪರ-ಲೋ-ಕದ
ಪರ-ಲೋ-ಕ-ದ-ಲ್ಲಾ-ಗಲಿ
ಪರ-ಲೋ-ಕ-ದ-ಲ್ಲಿಯೂ
ಪರ-ಲೋ-ಕ-ಯಾ-ತ್ರೆಗೆ
ಪರ-ಲೋ-ಕ-ವನ್ನು
ಪರ-ಲೋ-ಕವೂ
ಪರ-ವ-ಶತೆ
ಪರ-ವ-ಶ-ನಾಗಿ
ಪರ-ವ-ಶ-ನಾ-ಗುತ್ತ
ಪರ-ವ-ಶ-ನಾದ
ಪರ-ವ-ಶ-ರಾ-ಗು-ತ್ತೇವೆ
ಪರ-ವ-ಶ-ರಾ-ಗು-ವು-ದಿಲ್ಲ
ಪರ-ವ-ಶ-ವಾ-ಗು-ವುದು
ಪರವಾ
ಪರ-ವಾಗಿ
ಪರ-ವಾ-ಗಿಯೇ
ಪರ-ವಾ-ಗಿಯೋ
ಪರ-ವಾ-ಗಿಲ್ಲ
ಪರ-ವಾ-ದದ್ದು
ಪರ-ವಾ-ಯಿಲ್ಲ
ಪರ-ಸ್ತ-ಸ್ಮಾತ್ತು
ಪರ-ಸ್ತಾತ್
ಪರ-ಸ್ಪರ
ಪರ-ಸ್ಪರಂ
ಪರ-ಸ್ಪ-ರಮ್
ಪರ-ಸ್ಯೋ-ತ್ಸಾ-ದ-ನಾರ್ಥಂ
ಪರ-ಹಿತ
ಪರ-ಹಿ-ತವೋ
ಪರಾ
ಪರಾಂ
ಪರಾ-ಕಾಷ್ಠೆ
ಪರಾ-ಕಾ-ಷ್ಠೆಗೆ
ಪರಾ-ಕಾ-ಷ್ಠೆ-ಯನ್ನು
ಪರಾ-ಕ್ರಮ
ಪರಾ-ಕ್ರ-ಮ-ಗಳಿಂದ
ಪರಾ-ಕ್ರ-ಮದ
ಪರಾ-ಕ್ರ-ಮ-ದಲ್ಲಿ
ಪರಾ-ಕ್ರ-ಮನು
ಪರಾ-ಕ್ರ-ಮ-ಶಾಲಿ
ಪರಾ-ಗದ
ಪರಾ-ಗ-ವನ್ನು
ಪರಾ-ಣ್ಯಾ-ಹು-ರಿಂ-ದ್ರಿ-ಯೇಭ್ಯಃ
ಪರಾ-ತ್ಪರ
ಪರಾ-ಧೀ-ನತೆ
ಪರಾ-ಧೀ-ನರು
ಪರಾ-ಪ್ರ-ಕೃ-ತಿ-ಯನ್ನು
ಪರಾ-ಭ-ಕ್ತ-ನಾ-ಗು-ತ್ತಾನೆ
ಪರಾ-ಭಕ್ತಿ
ಪರಾ-ಭ-ಕ್ತಿ-ಯನ್ನು
ಪರಾಮ್
ಪರಾ-ಯಣ
ಪರಾ-ಯ-ಣ-ತೆಯ
ಪರಾ-ಯ-ಣ-ನಾಗಿ
ಪರಾ-ಯ-ಣ-ನಾ-ಗಿ-ದ್ದಾನೆ
ಪರಾ-ಯ-ಣ-ನಾ-ಗಿ-ರ-ಬೇಕು
ಪರಾ-ಯ-ಣ-ರಾಗಿ
ಪರಾ-ಯ-ಣ-ರಾ-ಗಿ-ರು-ವ-ವರು
ಪರಾ-ಯ-ಣ-ರಾ-ಗಿ-ರು-ವು-ದಿಲ್ಲ
ಪರಾ-ವಿದ್ಯೆ
ಪರಾ-ಶ-ರರ
ಪರಿ-ಕೀ-ರ್ತಿತಃ
ಪರಿ-ಕ್ಲಿಷ್ಟಂ
ಪರಿ-ಕ್ಲೇ-ಶ-ದಿಂದ
ಪರಿ-ಗ-ಣಿ-ಸ-ಬೇಕು
ಪರಿ-ಗ-ಣಿ-ಸ-ಲ್ಪ-ಟ್ಟಿದೆ
ಪರಿ-ಗ-ಣಿ-ಸ-ಲ್ಪ-ಟ್ಟಿ-ರು-ವನೋ
ಪರಿ-ಗ-ಣಿ-ಸು-ವನು
ಪರಿ-ಗ-ಣಿ-ಸು-ವು-ದಾ-ಗಿದೆ
ಪರಿ-ಗ-ಣಿ-ಸು-ವು-ದಿಲ್ಲ
ಪರಿ-ಗ್ರಹ
ಪರಿ-ಗ್ರ-ಹಮ್
ಪರಿ-ಗ್ರ-ಹ-ವನ್ನು
ಪರಿ-ಚ-ಕ್ಷತೇ
ಪರಿ-ಚಯ
ಪರಿ-ಚ-ಯ-ವನ್ನು
ಪರಿ-ಚ-ಯ-ವಾ-ಗಿತ್ತು
ಪರಿ-ಚ-ಯ-ವಾ-ಗು-ವುದು
ಪರಿ-ಚ-ಯ-ವಾದ
ಪರಿ-ಚ-ಯ-ವಾ-ದ-ರೇನೇ
ಪರಿ-ಚ-ಯ-ವಿದೆ
ಪರಿ-ಚ-ಯ-ವಿ-ರ-ಲಿಲ್ಲ
ಪರಿ-ಚ-ಯ-ವಿ-ರು-ವು-ದಿಲ್ಲ
ಪರಿ-ಚ-ಯ-ವಿ-ಲ್ಲ-ದ-ವರು
ಪರಿ-ಚ-ರ್ಯಾ-ತ್ಮಕಂ
ಪರಿ-ಚಾ-ರಕ
ಪರಿ-ಚಾ-ರ-ಕ-ರದು
ಪರಿ-ಚಾ-ರ-ಕರು
ಪರಿ-ಚಾ-ರಿಕೆ
ಪರಿ-ಚಿಂ-ತ-ಯನ್
ಪರಿ-ಚಿ-ತರ
ಪರಿ-ಜ್ಞಾತಾ
ಪರಿ-ಜ್ಞಾತೃ
ಪರಿ-ಜ್ಞಾ-ತೃ-ಗ-ಳೆಂದು
ಪರಿ-ಜ್ಞಾನ
ಪರಿ-ಜ್ಞಾ-ನ-ವನ್ನು
ಪರಿ-ಜ್ಞಾ-ನ-ವಿ-ರು-ವುದು
ಪರಿ-ಣ-ತ-ರಾ-ಗದೆ
ಪರಿ-ಣ-ತಿ-ಯನ್ನು
ಪರಿ-ಣ-ಮಿಸು
ಪರಿ-ಣಾಮ
ಪರಿ-ಣಾ-ಮ-ಕ-ರ-ವಾದ
ಪರಿ-ಣಾ-ಮ-ಕಾ-ರಿ-ಯಾ-ದುದು
ಪರಿ-ಣಾ-ಮ-ಕಾ-ರಿ-ಯಾ-ರು-ವುದು
ಪರಿ-ಣಾ-ಮಕ್ಕೆ
ಪರಿ-ಣಾ-ಮ-ಕ್ಕೆಲ್ಲ
ಪರಿ-ಣಾ-ಮ-ಗಳನ್ನೂ
ಪರಿ-ಣಾ-ಮ-ಗ-ಳಲ್ಲ
ಪರಿ-ಣಾ-ಮ-ಗ-ಳಿಗೆ
ಪರಿ-ಣಾ-ಮ-ಗಳು
ಪರಿ-ಣಾ-ಮ-ಗ-ಳೆಲ್ಲ
ಪರಿ-ಣಾ-ಮದ
ಪರಿ-ಣಾ-ಮ-ದಂತೆ
ಪರಿ-ಣಾ-ಮ-ದಿಂದ
ಪರಿ-ಣಾ-ಮ-ವನ್ನು
ಪರಿ-ಣಾ-ಮ-ವ-ನ್ನುಂ-ಟು-ಮಾ-ಡು-ವುದು
ಪರಿ-ಣಾ-ಮ-ವನ್ನೂ
ಪರಿ-ಣಾ-ಮ-ವನ್ನೇ
ಪರಿ-ಣಾ-ಮ-ವಾಗಿ
ಪರಿ-ಣಾ-ಮ-ವಾ-ಗಿಯೇ
ಪರಿ-ಣಾ-ಮ-ವಾಗು
ಪರಿ-ಣಾ-ಮ-ವಾ-ಗು-ವು-ದಿಲ್ಲ
ಪರಿ-ಣಾ-ಮ-ವಾ-ಗು-ವುದು
ಪರಿ-ಣಾ-ಮ-ವಾ-ಗು-ವುವು
ಪರಿ-ಣಾ-ಮ-ವಾ-ದರೂ
ಪರಿ-ಣಾ-ಮವೆ
ಪರಿ-ಣಾ-ಮವೇ
ಪರಿ-ಣಾಮೇ
ಪರಿ-ಣಾ-ಮೇ-ಽಮೃ-ತೋ-ಪ-ಮಮ್
ಪರಿ-ತ-ಪಿ-ಸ-ಬೇ-ಕಾ-ಗು-ವುದು
ಪರಿ-ತ-ಪಿ-ಸು-ತ್ತೇವೆ
ಪರಿ-ತ-ಪಿ-ಸು-ವಂ-ತಿ-ರ-ಬೇಕು
ಪರಿ-ತ-ಪಿ-ಸು-ವನು
ಪರಿ-ತಾಪ
ಪರಿ-ತಾ-ಪ-ವನ್ನು
ಪರಿ-ತ್ಯ-ಜಿಸಿ
ಪರಿ-ತ್ಯಜ್ಯ
ಪರಿ-ತ್ಯಾಗ
ಪರಿ-ತ್ಯಾ-ಗ-ಸ್ತಾ-ಮಸಃ
ಪರಿ-ತ್ಯಾ-ಗಿಯೋ
ಪರಿ-ತ್ರಾ-ಣಾಯ
ಪರಿ-ದ-ಹ್ಯತೇ
ಪರಿ-ದೇ-ವನಾ
ಪರಿ-ಪಂ-ಥಿನೌ
ಪರಿ-ಪಾಕ
ಪರಿ-ಪಾ-ಲನೆ
ಪರಿ-ಪಾ-ಲ-ನೆಯೇ
ಪರಿ-ಪಾ-ಲಿಸು
ಪರಿ-ಪಾ-ಲಿ-ಸು-ವುದು
ಪರಿ-ಪು-ಷ್ಟಿಗೆ
ಪರಿ-ಪೂ-ರ್ಣ-ನ-ನ್ನಾಗಿ
ಪರಿ-ಪೂ-ರ್ಣ-ನಾಗಿ
ಪರಿ-ಪೂ-ರ್ಣ-ನಾ-ಗಿಯೇ
ಪರಿ-ಪೂ-ರ್ಣ-ವಾಗಿ
ಪರಿ-ಪೂ-ರ್ಣ-ವಾ-ಗು-ವುದು
ಪರಿ-ಪೂ-ರ್ಣವೂ
ಪರಿ-ಪ್ರಶ್ನೆ
ಪರಿ-ಪ್ರ-ಶ್ನೆ-ಯಿಂದ
ಪರಿ-ಪ್ರ-ಶ್ನೆ-ಯಿಂ-ದಲೂ
ಪರಿ-ಪ್ರ-ಶ್ನೇನ
ಪರಿ-ಮಳ
ಪರಿ-ಮ-ಳ-ದಂತೆ
ಪರಿ-ಮ-ಳ-ದಲ್ಲಿ
ಪರಿ-ಮ-ಳ-ದಿಂದ
ಪರಿ-ಮ-ಳ-ವನ್ನು
ಪರಿ-ಮ-ಳ-ವಿಲ್ಲ
ಪರಿ-ಮ-ಳವೂ
ಪರಿ-ಮಿ-ತ-ವಾ-ಗಿದೆ
ಪರಿ-ಯಂ-ತರ
ಪರಿ-ವ-ರ್ತ-ನ-ಗೊ-ಳಿ-ಸಲು
ಪರಿ-ವ-ರ್ತ-ನ-ವಾ-ಗು-ವುವು
ಪರಿ-ವ-ರ್ತಿ-ಸ-ಬ-ಹುದು
ಪರಿ-ವ-ರ್ತಿ-ಸ-ಬೇ-ಕಾ-ದರೆ
ಪರಿ-ವ-ರ್ತಿ-ಸ-ಬೇಕು
ಪರಿ-ವ-ರ್ತಿ-ಸ-ಬೇ-ಕೆಂದು
ಪರಿ-ವ-ರ್ತಿಸಿ
ಪರಿ-ವ-ರ್ತಿ-ಸಿ-ಕೊಂಡು
ಪರಿ-ವಾ-ರ-ದ-ಲ್ಲಿರು
ಪರಿ-ವಾ-ರ-ವನ್ನು
ಪರಿ-ವ್ಯಾ-ಪ್ತ-ನಾ-ಗಿ-ದ್ದಾನೆ
ಪರಿ-ವ್ಯಾ-ಪ್ತ-ವಾ-ಗಿವೆ
ಪರಿ-ವ್ರಾ-ಜ-ಕ-ನಾಗಿ
ಪರಿ-ಶೀ-ಲಿ-ಸ-ಬೇಕು
ಪರಿ-ಶೀ-ಲಿಸಿ
ಪರಿ-ಶೀ-ಲಿ-ಸು-ವನು
ಪರಿ-ಶುದ್ಧ
ಪರಿ-ಶು-ದ್ಧ-ಗೊ-ಳಿ-ಸ-ಬೇಕು
ಪರಿ-ಶು-ದ್ಧ-ನ-ನ್ನಾಗಿ
ಪರಿ-ಶು-ದ್ಧ-ನಾಗಿ
ಪರಿ-ಶು-ದ್ಧ-ನಾ-ಗುತ್ತ
ಪರಿ-ಶು-ದ್ಧ-ಮಾ-ಡಿದ
ಪರಿ-ಶು-ದ್ಧ-ರಾ-ಗು-ತ್ತಾರೆ
ಪರಿ-ಶು-ದ್ಧ-ವಾಗಿ
ಪರಿ-ಶು-ದ್ಧ-ವಾ-ಗಿದೆ
ಪರಿ-ಶು-ದ್ಧ-ವಾ-ಗಿ-ದ್ದರೆ
ಪರಿ-ಶು-ದ್ಧ-ವಾ-ಗಿ-ರ-ಬೇಕು
ಪರಿ-ಶು-ದ್ಧ-ವಾ-ಗಿ-ರು-ವು-ದ-ರಿಂದ
ಪರಿ-ಶು-ದ್ಧ-ವಾ-ಗಿ-ರು-ವುದು
ಪರಿ-ಶು-ದ್ಧ-ವಾ-ಗುತ್ತ
ಪರಿ-ಶು-ದ್ಧ-ವಾ-ಗು-ವುದು
ಪರಿ-ಶು-ದ್ಧ-ವಾದ
ಪರಿ-ಶು-ದ್ಧ-ವಾ-ದರೆ
ಪರಿ-ಶುದ್ಧಿ
ಪರಿ-ಶು-ದ್ಧಿ-ಯಾ-ಗಿ-ರ-ಬೇಕು
ಪರಿ-ಶು-ಷ್ಯತಿ
ಪರಿ-ಸ-ಮಾ-ಪ್ತ-ವಾ-ಗು-ತ್ತವೆ
ಪರಿ-ಸ-ಮಾ-ಪ್ತಿ-ಯಾ-ಗು-ವುವು
ಪರಿ-ಸ-ಮಾ-ಪ್ಯತೇ
ಪರಿ-ಸ್ಥಿತಿ
ಪರಿ-ಸ್ಥಿ-ತಿಗೆ
ಪರಿ-ಸ್ಥಿ-ತಿ-ಯನ್ನು
ಪರಿ-ಸ್ಥಿ-ತಿ-ಯಲ್ಲಿ
ಪರಿ-ಸ್ಥಿ-ತಿ-ಯ-ಲ್ಲಿ-ಇ-ರು-ವೆವು
ಪರಿ-ಸ್ಥಿ-ತಿ-ಯ-ಲ್ಲಿ-ದ್ದರೂ
ಪರಿ-ಸ್ಥಿ-ತಿ-ಯ-ಲ್ಲಿಯೂ
ಪರಿ-ಸ್ಥಿ-ತಿ-ಯ-ಲ್ಲಿ-ರು-ವೆವು
ಪರಿ-ಸ್ಥಿ-ತಿ-ಯಿಂದ
ಪರಿ-ಸ್ಥಿ-ತಿಯೂ
ಪರಿ-ಸ್ಥಿ-ತಿ-ಯೊಂ-ದಿಗೆ
ಪರಿ-ಹ-ರಿಸ
ಪರಿ-ಹ-ರಿ-ಸ-ಬೇ-ಕಾ-ಗಿದೆ
ಪರಿ-ಹ-ರಿ-ಸ-ಲಾ-ಗದ
ಪರಿ-ಹ-ರಿ-ಸಲು
ಪರಿ-ಹ-ರಿ-ಸಿ-ಕೊ-ಳ್ಳಲು
ಪರಿ-ಹ-ರಿ-ಸಿ-ದ್ದರೆ
ಪರಿ-ಹ-ರಿಸು
ಪರಿ-ಹ-ರಿ-ಸುವ
ಪರಿ-ಹ-ರಿ-ಸು-ವನು
ಪರಿ-ಹ-ರಿ-ಸು-ವು-ದ-ಕ್ಕಾಗಿ
ಪರಿ-ಹ-ರಿ-ಸು-ವು-ದಕ್ಕೆ
ಪರಿ-ಹಾರ
ಪರಿ-ಹಾ-ರ-ವಾ-ಗು-ವುದು
ಪರೀಕ್ಷಾ
ಪರೀ-ಕ್ಷಾ-ಸ-ಮಯ
ಪರೀ-ಕ್ಷಾ-ಸ-ಮ-ಯ-ಗಳು
ಪರೀ-ಕ್ಷಾ-ಸ-ಮ-ಯ-ವನ್ನು
ಪರೀ-ಕ್ಷಿ-ಸಲು
ಪರೀ-ಕ್ಷಿಸಿ
ಪರೀ-ಕ್ಷಿ-ಸಿ-ಕೊಂ-ಡಿ-ರು-ವರು
ಪರೀ-ಕ್ಷಿಸು
ಪರೀ-ಕ್ಷಿ-ಸುತ್ತ
ಪರೀ-ಕ್ಷಿ-ಸು-ತ್ತಾನೆ
ಪರೀ-ಕ್ಷಿ-ಸುವ
ಪರೀ-ಕ್ಷಿ-ಸು-ವಾಗ
ಪರೀ-ಕ್ಷಿ-ಸು-ವು-ದ-ಕ್ಕಾಗಿ
ಪರೀ-ಕ್ಷಿ-ಸು-ವು-ದಕ್ಕೆ
ಪರೀಕ್ಷೆ
ಪರೀ-ಕ್ಷೆಗೆ
ಪರೀ-ಕ್ಷೆ-ಮಾಡಿ
ಪರೀ-ಕ್ಷೆಯ
ಪರೀ-ಕ್ಷೆ-ಯಲ್ಲಿ
ಪರೀ-ಕ್ಷೆ-ಯಾ-ಯಿತು
ಪರು
ಪರುಷ
ಪರೆ-ಯನ್ನು
ಪರೋ
ಪರೋ-ಕ್ಷ-ವಾಗಿ
ಪರೋ-ಕ್ಷ-ವಾದ
ಪರೋ-ತ್ಕರ್ಷ
ಪರೋ-ಪ-ಕಾ-ರ-ಗಳನ್ನು
ಪರ್ಜ-ನ್ಯಾ-ದ-ನ್ನ-ಸಂ-ಭವಃ
ಪರ್ಜನ್ಯೋ
ಪರ್ಣಾನಿ
ಪರ್ಯಂ-ತರ
ಪರ್ಯಂ-ತ-ರವೂ
ಪರ್ಯ-ವ-ತಿ-ಷ್ಠತೇ
ಪರ್ಯ-ವ-ಸಾನ
ಪರ್ಯ-ವ-ಸಾ-ನ-ವಾ-ಗ-ಲೇ-ಬೇಕು
ಪರ್ಯ-ವ-ಸಾ-ನ-ವಾಗು
ಪರ್ಯ-ವ-ಸಾ-ನ-ವಾ-ಗು-ತ್ತವೆ
ಪರ್ಯ-ವ-ಸಾ-ನ-ವಾ-ಗು-ವುದು
ಪರ್ಯ-ವ-ಸಾ-ರ-ವಾ-ಗು-ವುದು
ಪರ್ಯಾಪ್ತಂ
ಪರ್ಯಾ-ಲೋ-ಚ-ನೆ-ಯನ್ನೇ
ಪರ್ಯಾ-ಲೋ-ಚ-ನೆ-ಯಿ-ಲ್ಲದೆ
ಪರ್ಯಾ-ಲೋ-ಚಿಸಿ
ಪರ್ಯಾ-ಲೋ-ಚಿ-ಸು-ವುದೇ
ಪರ್ಯು-ಪಾ-ಸತೇ
ಪರ್ಯು-ಷಿತಂ
ಪರ್ವ-ಕಾ-ಲ-ದಲ್ಲಿ
ಪರ್ವತ
ಪರ್ವ-ತ-ಗಳಲ್ಲಿ
ಪರ್ವ-ತ-ಗ-ಳಿವೆ
ಪರ್ವ-ತ-ದಂತೆ
ಪರ್ವ-ತ-ವನ್ನು
ಪರ್ವ-ದಲ್ಲಿ
ಪಲಾ-ಯನ
ಪಲ್ಟಿ
ಪಲ್ಯ
ಪಳ-ಗಿ-ರ-ಬೇಕು
ಪಳ-ಗಿಸ
ಪಳ-ಗಿಸಿ
ಪಳ-ಗಿ-ಸಿದ
ಪಳ-ಗುತ್ತಾ
ಪವ-ತಾ-ಮಸ್ಮಿ
ಪವನಃ
ಪವಾಡ
ಪವಾ-ಡ-ಗಳನ್ನು
ಪವಾ-ಡ-ಗ-ಳಿಗೆ
ಪವಿತ್ರ
ಪವಿತ್ರಂ
ಪವಿ-ತ್ರ-ತಮ
ಪವಿ-ತ್ರತೆ
ಪವಿ-ತ್ರ-ತೆ-ಗಳು
ಪವಿ-ತ್ರ-ತೆಗೆ
ಪವಿ-ತ್ರ-ತೆಯ
ಪವಿ-ತ್ರ-ತೆ-ಯನ್ನು
ಪವಿ-ತ್ರ-ತೆ-ಯನ್ನೂ
ಪವಿ-ತ್ರ-ತೆ-ಯ-ನ್ನೆಲ್ಲ
ಪವಿ-ತ್ರ-ತೆ-ಯಲ್ಲಿ
ಪವಿ-ತ್ರ-ನಾ-ದರೆ
ಪವಿ-ತ್ರ-ಮಾಡಿ
ಪವಿ-ತ್ರ-ಮಿ-ದ-ಮು-ತ್ತ-ಮಮ್
ಪವಿ-ತ್ರ-ಮಿಹ
ಪವಿ-ತ್ರ-ಮೋಂ-ಕಾರ
ಪವಿ-ತ್ರ-ರಾ-ಗಲಿ
ಪವಿ-ತ್ರ-ರಾಗಿ
ಪವಿ-ತ್ರ-ವ-ದ-ನ-ದಿಂದ
ಪವಿ-ತ್ರ-ವಾ-ಗಿ-ರ-ಬೇಕು
ಪವಿ-ತ್ರ-ವಾ-ಗಿ-ರು-ವುದನ್ನು
ಪವಿ-ತ್ರ-ವಾ-ಗಿ-ರು-ವುವು
ಪವಿ-ತ್ರ-ವಾಗು
ಪವಿ-ತ್ರ-ವಾ-ಗು-ವುದು
ಪವಿ-ತ್ರ-ವಾದ
ಪವಿ-ತ್ರ-ವಾ-ದದ್ದು
ಪವಿ-ತ್ರ-ವಾ-ದು-ದನ್ನು
ಪವಿ-ತ್ರ-ವಾ-ದುದು
ಪವಿ-ತ್ರ-ವೆಂದು
ಪವಿ-ತ್ರ-ವೆಂಬ
ಪವಿ-ತ್ರವೇ
ಪವಿ-ತ್ರವೋ
ಪವಿ-ತ್ರಾತ್ಮ
ಪವಿ-ತ್ರಾ-ತ್ಮನ
ಪವಿ-ತ್ರೋ-ತ್ತ-ಮನೂ
ಪಶು
ಪಶು-ಗಳಲ್ಲಿ
ಪಶು-ಗಳು
ಪಶು-ಪಕ್ಷಿ
ಪಶು-ಪ-ಕ್ಷಿ-ಗಳಲ್ಲಿ
ಪಶು-ಪ-ಕ್ಷಿ-ಗಳು
ಪಶ್ಚಾ-ತ್ತಾಪ
ಪಶ್ಚಾ-ತ್ತಾ-ಪ-ಪ-ಟ್ಟರೂ
ಪಶ್ಚಾ-ತ್ತಾ-ಪ-ಪ-ಡು-ವುದು
ಪಶ್ಚಾ-ತ್ತಾ-ಪ-ವನ್ನಾ
ಪಶ್ಯ
ಪಶ್ಯಂತಿ
ಪಶ್ಯಂ-ತ್ಯ-ಚೇ-ತಸಃ
ಪಶ್ಯಂ-ತ್ಯಾ-ತ್ಮ-ನ್ಯ-ವ-ಸ್ಥಿ-ತಮ್
ಪಶ್ಯತಿ
ಪಶ್ಯತೋ
ಪಶ್ಯ-ತ್ಯ-ಕೃ-ತ-ಬು-ದ್ಧಿ-ತ್ವಾನ್ನ
ಪಶ್ಯನ್
ಪಶ್ಯ-ನ್ನಾ-ತ್ಮನಿ
ಪಶ್ಯಾ-ದಿ-ತ್ಯಾನ್
ಪಶ್ಯಾದ್ಯ
ಪಶ್ಯಾಮಿ
ಪಶ್ಯಾ-ಶ್ಚ-ರ್ಯಾಣಿ
ಪಶ್ಯೇ-ದ-ಕ-ರ್ಮಣಿ
ಪಶ್ಯೈ-ತಾಂ
ಪಶ್ಯೈ-ತಾನ್
ಪಸ-ರಿ-ಸಿ-ದರೂ
ಪಸ-ರಿ-ಸಿದೆ
ಪಸ-ರಿ-ಸಿ-ರುವ
ಪಸ-ರಿ-ಸು-ತ್ತಿವೆ
ಪಹ-ರೆ-ಯ-ವ-ರಿಗೆ
ಪಾಂಚ-ಜನ್ಯಂ
ಪಾಂಚ-ಜ-ನ್ಯ-ವನ್ನು
ಪಾಂಚ-ಭೌ-ತಿ-ಕ-ವಾ-ದುದು
ಪಾಂಚಾಲ
ಪಾಂಚಾ-ಲ-ದೇ-ಶಕ್ಕೆ
ಪಾಂಡವ
ಪಾಂಡವಃ
ಪಾಂಡ-ವರ
ಪಾಂಡ-ವ-ರನ್ನು
ಪಾಂಡ-ವ-ರನ್ನೂ
ಪಾಂಡ-ವ-ರ-ನ್ನೆಲ್ಲ
ಪಾಂಡ-ವ-ರಲ್ಲ
ಪಾಂಡ-ವ-ರಲ್ಲಿ
ಪಾಂಡ-ವ-ರಿಗೆ
ಪಾಂಡ-ವ-ರಿಗೇ
ಪಾಂಡ-ವರು
ಪಾಂಡ-ವರೇ
ಪಾಂಡ-ವ-ಶ್ಚೈವ
ಪಾಂಡ-ವ-ಸ್ತದಾ
ಪಾಂಡ-ವಾ-ನಾಂ
ಪಾಂಡ-ವಾ-ನೀಕಂ
ಪಾಂಡ-ವಾ-ಶ್ಚೈವ
ಪಾಂಡವೈ
ಪಾಂಡಿ-ಚೆ-ರಿಯ
ಪಾಂಡಿತ್ಯ
ಪಾಂಡಿ-ತ್ಯಕ್ಕೂ
ಪಾಂಡಿ-ತ್ಯಕ್ಕೆ
ಪಾಂಡಿ-ತ್ಯದ
ಪಾಂಡಿ-ತ್ಯ-ದಂತೆ
ಪಾಂಡಿ-ತ್ಯ-ದಲ್ಲೇ
ಪಾಂಡಿ-ತ್ಯ-ಪೂರ್ಣ
ಪಾಂಡಿ-ತ್ಯ-ವನ್ನು
ಪಾಂಡಿ-ತ್ಯ-ವಾ-ಗ-ಬ-ಹುದು
ಪಾಂಡಿ-ತ್ಯ-ವಾ-ಗಲೀ
ಪಾಂಡು-ಪು-ತ್ರರ
ಪಾಂಡು-ಪು-ತ್ರಾ-ಣಾ-ಮಾ-ಚಾರ್ಯ
ಪಾಂಡು-ರಾಜ
ಪಾಕ
ಪಾಕ-ದಲ್ಲಿ
ಪಾಕ-ದಿಂದ
ಪಾಕ-ವನ್ನು
ಪಾಕವೆ
ಪಾಚಿ
ಪಾಚಿಯ
ಪಾಚಿ-ಯನ್ನು
ಪಾಚಿ-ಯಿಂದ
ಪಾಠ
ಪಾಠ-ಗ-ಳಿವೆ
ಪಾಠ-ಗಳು
ಪಾಠ-ವ-ನ್ನೆಲ್ಲ
ಪಾಠ-ವ-ನ್ನೆಲ್ಲಾ
ಪಾಡನ್ನು
ಪಾಡಾ-ಗು-ವುದು
ಪಾಡಿಗೆ
ಪಾಡು
ಪಾಡೇ
ಪಾಡೇ-ನಾ-ಗ-ಬೇಕು
ಪಾಡೇ-ನಾ-ಗು-ವುದು
ಪಾಡೇನು
ಪಾಣಿನಿ
ಪಾಣಿ-ನಿಯ
ಪಾತಕ
ಪಾತ-ಕ-ಗ-ಳನ್ನೆ
ಪಾತ-ಕಮ್
ಪಾತ-ದಂತೆ
ಪಾತಾಳ
ಪಾತ್ರ
ಪಾತ್ರಕ್ಕೆ
ಪಾತ್ರ-ಗಳನ್ನು
ಪಾತ್ರ-ಗಳಲ್ಲಿ
ಪಾತ್ರ-ಗ-ಳಿವೆ
ಪಾತ್ರ-ಗಳು
ಪಾತ್ರ-ಧಾರಿ
ಪಾತ್ರ-ಧಾ-ರಿ-ಗಳು
ಪಾತ್ರ-ನಲ್ಲ
ಪಾತ್ರ-ನಾ-ಗಿ-ರು-ವುದೇ
ಪಾತ್ರ-ನಾ-ದ-ವನು
ಪಾತ್ರ-ರಾ-ದ-ವ-ರಿಗೆ
ಪಾತ್ರರು
ಪಾತ್ರ-ವನ್ನು
ಪಾತ್ರ-ವಾ-ಗಿದೆ
ಪಾತ್ರ-ವಾ-ದ-ವು-ಗಳು
ಪಾತ್ರ-ವಿದೆ
ಪಾತ್ರೆ
ಪಾತ್ರೆಗೆ
ಪಾತ್ರೆಯ
ಪಾತ್ರೆ-ಯಂ-ತಾ-ಗು-ತ್ತಾನೆ
ಪಾತ್ರೆ-ಯಂತೆ
ಪಾತ್ರೆ-ಯನ್ನು
ಪಾತ್ರೆ-ಯಲ್ಲಿ
ಪಾತ್ರೆ-ಯ-ಲ್ಲಿಟ್ಟು
ಪಾತ್ರೆ-ಯ-ಲ್ಲಿ-ರು-ವುದೊ
ಪಾತ್ರೆ-ಯಿಂದ
ಪಾತ್ರೇ
ಪಾದ-ಕ-ಮ-ಲ-ಗಳನ್ನು
ಪಾದ-ಗಳಿಂದ
ಪಾದ-ಧೂಳಿ
ಪಾದ-ಪ-ದ್ಮ-ಗಳನ್ನು
ಪಾದ-ಪ-ದ್ಮ-ಗಳಲ್ಲಿ
ಪಾದ-ಪ-ದ್ಮ-ಗಳು
ಪಾದ-ಪ-ದ್ಮ-ವನ್ನು
ಪಾದ-ರಸ
ಪಾದ-ರ-ಸ-ದಂತೆ
ಪಾದ-ರ-ಸ-ವನ್ನು
ಪಾನ
ಪಾನ-ಮಾಡಿ
ಪಾನ-ಮಾ-ಡು-ತ್ತಾರೆ
ಪಾನ-ಮಾ-ಡು-ತ್ತಿ-ರು-ವರು
ಪಾಪ
ಪಾಪಂ
ಪಾಪ-ಕ-ರ್ಮ-ಗಳನ್ನು
ಪಾಪ-ಕ-ರ್ಮ-ಗಳೇ
ಪಾಪ-ಕಾ-ರ್ಯ-ಗಳನ್ನು
ಪಾಪ-ಕಾ-ರ್ಯ-ವನ್ನು
ಪಾಪ-ಕೃ-ತ್ತಮಃ
ಪಾಪ-ಕೃ-ತ್ಯ-ಗಳ
ಪಾಪ-ಕೃ-ತ್ಯ-ಗಳನ್ನು
ಪಾಪ-ಕೃ-ತ್ಯ-ಗಳನ್ನೂ
ಪಾಪ-ಕೃ-ತ್ಯ-ಗಳು
ಪಾಪ-ಕೃ-ತ್ಯ-ವನ್ನು
ಪಾಪಕ್ಕೂ
ಪಾಪಕ್ಕೆ
ಪಾಪ-ಗಳ
ಪಾಪ-ಗಳನ್ನು
ಪಾಪ-ಗಳನ್ನೆಲ್ಲ
ಪಾಪ-ಗ-ಳಾ-ದರೋ
ಪಾಪ-ಗಳಿಂದ
ಪಾಪ-ಗ-ಳಿಂ-ದಲೂ
ಪಾಪ-ಗ-ಳಿಂ-ದೆಲ್ಲಾ
ಪಾಪ-ಗಳು
ಪಾಪ-ಗಳೂ
ಪಾಪ-ಗ-ಳೆಲ್ಲ
ಪಾಪದ
ಪಾಪ-ದಲ್ಲಿ
ಪಾಪ-ದಿಂದ
ಪಾಪ-ಪು-ಣ್ಯ-ಗಳನ್ನು
ಪಾಪ-ಪು-ಣ್ಯ-ಗ-ಳಾ-ವುವೂ
ಪಾಪ-ಪು-ಣ್ಯ-ಗಳು
ಪಾಪ-ಪು-ಣ್ಯ-ಗ-ಳೆ-ರ-ಡನ್ನೂ
ಪಾಪ-ಪು-ಣ್ಯ-ಗ-ಳೆಲ್ಲ
ಪಾಪ-ಭೀತಿ
ಪಾಪ-ಮ-ವಾ-ಪ್ಸ್ಯಸಿ
ಪಾಪ-ಮೇ-ವಾ-ಶ್ರ-ಯೇ-ದ-ಸ್ಮಾನ್
ಪಾಪ-ಯೋ-ನಯಃ
ಪಾಪ-ಯೋ-ನಿ-ಗಳು
ಪಾಪ-ಯೋ-ನಿ-ಜ-ರಾ-ದರೂ
ಪಾಪ-ರ-ಹಿ-ತ-ನಾದ
ಪಾಪ-ರ-ಹಿ-ತನೇ
ಪಾಪ-ರಾಗಿ
ಪಾಪರ್
ಪಾಪ-ವ-ನ್ನಾ-ಗಲೀ
ಪಾಪ-ವನ್ನು
ಪಾಪ-ವನ್ನೂ
ಪಾಪ-ವ-ನ್ನೆಲ್ಲಾ
ಪಾಪ-ವನ್ನೇ
ಪಾಪ-ವ-ನ್ನೇಕೆ
ಪಾಪ-ವಲ್ಲ
ಪಾಪ-ವಾ-ಗು-ವುದು
ಪಾಪ-ವಿಲ್ಲ
ಪಾಪವು
ಪಾಪವೂ
ಪಾಪ-ವೆ-ನ್ನು-ವರು
ಪಾಪ-ವೆ-ನ್ನು-ವುವು
ಪಾಪ-ವೆಲ್ಲ
ಪಾಪವೇ
ಪಾಪವೊ
ಪಾಪ-ಸಾ-ಗರ
ಪಾಪಾ
ಪಾಪಾ-ತ್ಮನ
ಪಾಪಾ-ತ್ಮ-ರಿ-ರು-ವರು
ಪಾಪಾ-ದ-ಸ್ಮಾ-ನ್ನಿ-ವ-ರ್ತಿ-ತುಮ್
ಪಾಪಾ-ಯು-ಗಳೂ
ಪಾಪಾಸು
ಪಾಪಿ
ಪಾಪಿ-ಪು-ಣ್ಯ-ವಂತ
ಪಾಪಿ-ಗಳನ್ನು
ಪಾಪಿ-ಗ-ಳಲ್ಲ
ಪಾಪಿ-ಗ-ಳ-ಲ್ಲಿಯೂ
ಪಾಪಿ-ಗ-ಳಿಗೂ
ಪಾಪಿ-ಗಳು
ಪಾಪಿ-ಗಾ-ದರೂ
ಪಾಪಿ-ಯನ್ನು
ಪಾಪಿ-ಯಾ-ಗು-ತ್ತಾನೆ
ಪಾಪಿಯು
ಪಾಪೇನ
ಪಾಪೇಭ್ಯಃ
ಪಾಪೇಷು
ಪಾಪ್ಮಾನಂ
ಪಾಮರ
ಪಾಮ-ರ-ನನ್ನು
ಪಾಮ-ರ-ನಾ-ಗಿ-ರಲಿ
ಪಾಯ-ವನ್ನು
ಪಾಯಿಂ-ಟಿನ
ಪಾಯು
ಪಾರ
ಪಾರಂ-ಗ-ತ-ರಾ-ಗ-ಬೇ-ಕಾ-ಗಿದೆ
ಪಾರಂ-ಗ-ತ-ರಾ-ಗಿ-ರು-ವು-ದಿಲ್ಲ
ಪಾರ-ಗಂ-ತ-ನಾ-ದ-ವನು
ಪಾರ-ದ-ರ್ಶಕ
ಪಾರ-ದ-ರ್ಶ-ಕ-ವಾ-ಗಿ-ದ್ದರೆ
ಪಾರ-ದ-ರ್ಶಿ-ಕತೆ
ಪಾರ-ಮಾ-ರ್ಥಿಕ
ಪಾರ-ಮಾ-ರ್ಥಿ-ಕ-ವಾ-ಗಿ-ರಲಿ
ಪಾರ-ಮಾ-ರ್ಥಿ-ಕ-ವಾ-ಗು-ವುದು
ಪಾರ-ವಿಲ್ಲ
ಪಾರಾ
ಪಾರಾಗ
ಪಾರಾ-ಗ-ದಂತೆ
ಪಾರಾ-ಗ-ಬ-ಹುದು
ಪಾರಾ-ಗ-ಬೇ-ಕಾ-ಗಿದೆ
ಪಾರಾ-ಗ-ಬೇ-ಕಾ-ದರೂ
ಪಾರಾ-ಗ-ಬೇ-ಕಾ-ದರೆ
ಪಾರಾ-ಗ-ಬೇಕು
ಪಾರಾ-ಗ-ಬೇ-ಕೆಂದು
ಪಾರಾ-ಗ-ಬೇ-ಕೆಂಬ
ಪಾರಾ-ಗ-ಲಾರ
ಪಾರಾ-ಗಲಿ
ಪಾರಾ-ಗಲು
ಪಾರಾಗಿ
ಪಾರಾ-ಗಿದೆ
ಪಾರಾ-ಗಿ-ದ್ದರೆ
ಪಾರಾ-ಗಿ-ದ್ದಾನೆ
ಪಾರಾ-ಗಿ-ದ್ದಾರೆ
ಪಾರಾ-ಗಿ-ರ-ಬೇಕು
ಪಾರಾ-ಗಿರು
ಪಾರಾ-ಗಿ-ರು-ವನು
ಪಾರಾ-ಗಿ-ರು-ವುದು
ಪಾರಾ-ಗಿಲ್ಲ
ಪಾರಾ-ಗಿ-ಲ್ಲವೋ
ಪಾರಾ-ಗಿವೆ
ಪಾರಾ-ಗಿ-ವೆಯೋ
ಪಾರಾಗು
ಪಾರಾ-ಗುತ್ತಾ
ಪಾರಾ-ಗು-ತ್ತಾನೆ
ಪಾರಾ-ಗು-ತ್ತಾರೆ
ಪಾರಾ-ಗು-ತ್ತೇನೆ
ಪಾರಾ-ಗು-ತ್ತೇವೆ
ಪಾರಾ-ಗು-ತ್ತೇ-ವೆಯೋ
ಪಾರಾ-ಗುವ
ಪಾರಾ-ಗು-ವಂತೆ
ಪಾರಾ-ಗು-ವನು
ಪಾರಾ-ಗು-ವನೊ
ಪಾರಾ-ಗು-ವರು
ಪಾರಾ-ಗು-ವ-ವ-ನಲ್ಲ
ಪಾರಾ-ಗು-ವು-ದಕ್ಕೆ
ಪಾರಾ-ಗು-ವು-ದಕ್ಕೇ
ಪಾರಾ-ಗು-ವುದನ್ನು
ಪಾರಾ-ಗು-ವು-ದಿಲ್ಲ
ಪಾರಾ-ಗು-ವುದು
ಪಾರಾ-ಗು-ವುದೊ
ಪಾರಾ-ಗು-ವುದೋ
ಪಾರಾ-ಗುವೆ
ಪಾರಾ-ಗು-ವೆಯೋ
ಪಾರಾ-ಗು-ವೆವು
ಪಾರಾದ
ಪಾರಾ-ದಂತೆ
ಪಾರಾ-ದ-ಮೇ-ಲೆಯೇ
ಪಾರಾ-ದರೆ
ಪಾರಾ-ದ-ವನೊ
ಪಾರಾ-ದ-ವರು
ಪಾರಾ-ದ-ವರೂ
ಪಾರಾದೆ
ಪಾರಾ-ದೆಯಾ
ಪಾರಾ-ಯಣ
ಪಾರಾ-ಯಾಣ
ಪಾರಾ-ಶ-ರ್ಯ-ವ-ಚಃ-ಸ-ರೋ-ಜ-ಮ-ಮಲಂ
ಪಾರಿ-ತೋ-ಷ-ಕ-ವನ್ನು
ಪಾರು
ಪಾರು-ಬೇಕು
ಪಾರು-ಮಾ-ಡ-ಬೇ-ಕೆಂದು
ಪಾರು-ಮಾ-ಡವ
ಪಾರು-ಮಾಡಿ
ಪಾರು-ಮಾ-ಡಿ-ದರು
ಪಾರು-ಮಾ-ಡಿ-ದರೆ
ಪಾರು-ಮಾ-ಡಿ-ದಾಗ
ಪಾರು-ಮಾ-ಡಿ-ದೆವೆ
ಪಾರು-ಮಾಡು
ಪಾರು-ಮಾ-ಡು-ವನು
ಪಾರು-ಮಾ-ಡು-ವು-ದ-ಕ್ಕಾಗಿ
ಪಾರು-ಮಾ-ಡು-ವುದು
ಪಾರು-ಷ್ಯ-ಮೇವ
ಪಾರ್ಥ
ಪಾರ್ಥ-ನಿಗೆ
ಪಾರ್ಥ-ನಿ-ರು-ವನೊ
ಪಾರ್ಥನೇ
ಪಾರ್ಥಸ್ಯ
ಪಾರ್ಥಾ-ನು-ಚಿಂ-ತ-ಯನ್
ಪಾರ್ಥಾಯ
ಪಾರ್ಥಾಸ್ತಿ
ಪಾರ್ಥೋ
ಪಾರ್ವತಿ
ಪಾಲಕ
ಪಾಲನ
ಪಾಲ-ನದ
ಪಾಲ-ನ-ದಲ್ಲಿ
ಪಾಲ-ನಾ-ಶ-ಕ್ತಿ-ಯಂತೆ
ಪಾಲನೆ
ಪಾಲನ್ನು
ಪಾಲಾ-ಗು-ವರು
ಪಾಲಾ-ಗು-ವುದು
ಪಾಲಿಗೂ
ಪಾಲಿಗೆ
ಪಾಲಿ-ತ-ವಾ-ಗುತ್ತಾ
ಪಾಲಿನ
ಪಾಲಿ-ನ-ದನ್ನು
ಪಾಲಿಶ್
ಪಾಲಿ-ಸದೆ
ಪಾಲಿ-ಸ-ಬೇ-ಕಾ-ಗಿತ್ತು
ಪಾಲಿ-ಸು-ತ್ತವೆ
ಪಾಲಿ-ಸು-ತ್ತಾನೆ
ಪಾಲಿ-ಸು-ತ್ತಾ-ರೆಯೆ
ಪಾಲಿ-ಸು-ತ್ತಿ-ರುವ
ಪಾಲಿ-ಸು-ತ್ತಿ-ರು-ವನು
ಪಾಲಿ-ಸು-ತ್ತಿ-ರು-ವ-ವನು
ಪಾಲಿ-ಸುವ
ಪಾಲಿ-ಸು-ವರು
ಪಾಲಿ-ಸು-ವು-ದಿಲ್ಲ
ಪಾಲಿ-ಸು-ವುದು
ಪಾಲಿ-ಸು-ವುದೂ
ಪಾಲಿ-ಸು-ವುವೋ
ಪಾಲೀ
ಪಾಲು
ಪಾವಕಃ
ಪಾವ-ಕ-ಶ್ಚಾಸ್ಮಿ
ಪಾವನ
ಪಾವ-ನ-ಗೊ-ಳಿ-ಸು-ವುದು
ಪಾವ-ನ-ಗೊ-ಳಿ-ಸು-ವುವು
ಪಾವ-ನ-ವ-ನ್ನಾಗಿ
ಪಾವ-ನಾನಿ
ಪಾವು
ಪಾಶ
ಪಾಶಕ್ಕೆ
ಪಾಶ-ಗಳಿಂದ
ಪಾಶ-ದಿಂದ
ಪಾಷಾ-ಣ-ವನ್ನು
ಪಾಷಾ-ಣವೂ
ಪಾಸನೆ
ಪಾಸಾ-ಗಲು
ಪಾಸಾ-ಗಿ-ರು-ವೆನು
ಪಾಸಾದ
ಪಾಸು
ಪಾಸೆ-ಲ್ಲ-ವನ್ನೂ
ಪಿಂಡ
ಪಿಂಡಾಂ-ಡ-ಗ-ಳೆಲ್ಲಾ
ಪಿಂಡಾಂ-ಡ-ದ-ಲ್ಲಿಯೂ
ಪಿಂಡಾಂ-ಡ-ವನ್ನು
ಪಿಂಡಿ
ಪಿಂಡಿ-ಗಳನ್ನೆಲ್ಲ
ಪಿತ
ಪಿತ-ನಾ-ಗಿ-ದ್ದಾನೆ
ಪಿತರಃ
ಪಿತರೋ
ಪಿತಾ
ಪಿತಾ-ಮಹ
ಪಿತಾ-ಮಹಃ
ಪಿತಾ-ಮ-ಹನೂ
ಪಿತಾ-ಮ-ಹಾಃ
ಪಿತಾ-ಮ-ಹಾನ್
ಪಿತಾಽಸಿ
ಪಿತಾ-ಽಹ-ಮಸ್ಯ
ಪಿತೃ
ಪಿತೃ-ಗಳ
ಪಿತೃ-ಗಳನ್ನು
ಪಿತೃ-ಗಳಲ್ಲಿ
ಪಿತೃ-ಗ-ಳಿಗೂ
ಪಿತೃ-ಗ-ಳಿಗೆ
ಪಿತೃ-ಗಳು
ಪಿತೃ-ಗಳೇ
ಪಿತೃ-ಯಜ್ಞ
ಪಿತೃ-ಲೋಕ
ಪಿತೃ-ಲೋ-ಕ-ದಲ್ಲಿ
ಪಿತೃ-ಲೋ-ಕ-ದಿಂದ
ಪಿತೃ-ವ್ರ-ತಾಃ
ಪಿತೄ-ಣಾ-ಮ-ರ್ಯಮಾ
ಪಿತೄ-ನಥ
ಪಿತೄನ್
ಪಿತೇವ
ಪಿತ್ತ
ಪಿಪಾ-ಸು-ಗ-ಳೆಲ್ಲ
ಪಿಪಾಸೆ
ಪಿರ-ಮಿ-ಡ್ಡಿನ
ಪಿಶಾಚಿ
ಪಿಶಾ-ಚಿ-ಗಳು
ಪಿಶಾ-ಚಿ-ಯಂತೆ
ಪೀಠ
ಪೀಠದ
ಪೀಠ-ವನ್ನು
ಪೀಡಿ-ತ-ನಾ-ಗು-ವು-ದಿಲ್ಲ
ಪೀಡಿ-ಸಿ-ದಂತೆ
ಪೀಡಿ-ಸು-ವರು
ಪೀಡು-ಸು-ವು-ದಕ್ಕೆ
ಪೀಡೆ-ಯ-ನ್ನುಂಟು
ಪೀನಲ್
ಪೀಳಿ-ಗೆ-ಯ-ವರು
ಪುಂಸಃ
ಪುಕ್
ಪುಕ್ಕ-ಗಳನ್ನು
ಪುಕ್ಕ-ಗಳು
ಪುಕ್ಕ-ವನ್ನು
ಪುಕ್ಸಾಮ
ಪುಜುತ್ವ
ಪುಟ-ಗೊ-ಳ್ಳ-ಬೇ-ಕಾ-ದರೆ
ಪುಟ್ಟ
ಪುಟ್ಟ-ಗಾ-ಲಿ-ಗ-ಳೆಲ್ಲ
ಪುಟ್ಟ-ದಾ-ದ-ರೇನು
ಪುಟ್ಟ-ಪುಟ್ಟ
ಪುಡಿ
ಪುಡಿ-ಕಾಸು
ಪುಡಿ-ಪುಡಿ
ಪುಡಿ-ಪು-ಡಿ-ಯಾಗಿ
ಪುಡಿ-ಪು-ಡಿ-ಯಾ-ಗು-ವುದು
ಪುಡಿ-ಪು-ಡಿ-ಯಾದ
ಪುಡಿ-ಯನ್ನು
ಪುಡಿ-ಯಾ-ಗು-ತ್ತಿ-ರು-ವರು
ಪುಣ
ಪುಣ-ದಿಂದ
ಪುಣ-ಮುಕ್ತ
ಪುಣ-ಮು-ಕ್ತ-ನಾಗಿ
ಪುಣ-ವನ್ನು
ಪುಣ್ಯ
ಪುಣ್ಯಂ
ಪುಣ್ಯ-ಕರ್ಮ
ಪುಣ್ಯ-ಕ-ರ್ಮ-ಗಳನ್ನು
ಪುಣ್ಯ-ಕ-ರ್ಮ-ಣಾಮ್
ಪುಣ್ಯ-ಕೃ-ತಾಂ
ಪುಣ್ಯ-ಕೆ-ಲಸ
ಪುಣ್ಯ-ಕೆ-ಲ-ಸ-ಗಳನ್ನು
ಪುಣ್ಯ-ಕೆ-ಲ-ಸ-ಗಳೂ
ಪುಣ್ಯ-ಕ್ಕಿಂತ
ಪುಣ್ಯ-ಕ್ಷೀ-ಣ-ವಾ-ದ-ಮೇಲೆ
ಪುಣ್ಯ-ಗಳನ್ನು
ಪುಣ್ಯ-ಗ-ಳಿ-ಗ-ನು-ಸಾರ
ಪುಣ್ಯ-ಗಳು
ಪುಣ್ಯ-ಗ-ಳೆ-ರ-ಡನ್ನೂ
ಪುಣ್ಯ-ಗ-ಳೆಲ್ಲ
ಪುಣ್ಯ-ಗ-ಳೆ-ಲ್ಲ-ಕ್ಕಿಂತ
ಪುಣ್ಯ-ಜ-ನ್ಮವೊ
ಪುಣ್ಯ-ತೀ-ರಿದ
ಪುಣ್ಯದ
ಪುಣ್ಯ-ದಿ-ನ-ಗಳು
ಪುಣ್ಯ-ಫ-ಲ-ಗಳನ್ನು
ಪುಣ್ಯ-ಮಾ-ಸಾದ್ಯ
ಪುಣ್ಯ-ಲೋ-ಕವೊ
ಪುಣ್ಯ-ವಂತ
ಪುಣ್ಯ-ವಂ-ತ-ನಾ-ಗಿ-ದ್ದಾನೆ
ಪುಣ್ಯ-ವಂ-ತರ
ಪುಣ್ಯ-ವಂ-ತ-ರಿ-ರು-ವರು
ಪುಣ್ಯ-ವ-ತಿ-ಯರ
ಪುಣ್ಯ-ವ-ನ್ನಾ-ಗಲೀ
ಪುಣ್ಯ-ವನ್ನು
ಪುಣ್ಯ-ವ-ನ್ನೆಲ್ಲ
ಪುಣ್ಯ-ವಲ್ಲ
ಪುಣ್ಯ-ವಾದ
ಪುಣ್ಯ-ವಾ-ಸ-ನೆಯ
ಪುಣ್ಯ-ವಾ-ಸ-ನೆ-ಯಲ್ಲಿ
ಪುಣ್ಯವೊ
ಪುಣ್ಯ-ಶಾ-ಲಿ-ಗಳ
ಪುಣ್ಯ-ಶಾ-ಲಿ-ಗ-ಳಾದ
ಪುಣ್ಯ-ಸಂ-ಪಾ-ದ-ನೆಗೆ
ಪುಣ್ಯಾ
ಪುಣ್ಯಾತ್ಮ
ಪುಣ್ಯಾ-ತ್ಮನ
ಪುಣ್ಯಾ-ತ್ಮ-ನಾ-ಗು-ವಷ್ಟು
ಪುಣ್ಯಾ-ತ್ಮ-ರೆ-ಲ್ಲರೂ
ಪುಣ್ಯಾ-ಪು-ಣ್ಯ-ಗ-ಳೆಲ್ಲ
ಪುಣ್ಯೇ
ಪುಣ್ಯೋ
ಪುತು
ಪುತು-ಗ-ಳ-ಲ್ಲೆಲ್ಲ
ಪುತು-ಧರ್ಮ
ಪುತು-ಧ-ರ್ಮ-ಗಳು
ಪುತೇಽಪಿ
ಪುತ್ರ
ಪುತ್ರ-ಕಾ-ಮೇಷ್ಠಿ
ಪುತ್ರ-ದಾ-ರ-ಗೃ-ಹಾ-ದಿಷು
ಪುತ್ರ-ನಂ-ತಿ-ರು-ವನು
ಪುತ್ರ-ನಾಗಿ
ಪುತ್ರ-ನಾದ
ಪುತ್ರ-ನಿಂದ
ಪುತ್ರನೇ
ಪುತ್ರರು
ಪುತ್ರರೇ
ಪುತ್ರಸ್ಯ
ಪುತ್ರಾಃ
ಪುತ್ರಾನ್
ಪುತ್ರಾ-ಸ್ತ-ಥೈವ
ಪುತ್ವಿ-ಜರು
ಪುನಃ
ಪುನಃ-ಪುನಃ
ಪುನ-ರಾ-ವ-ರ್ತಿ-ನೋ-ಽಜುನ
ಪುನ-ರುಕ್ತಿ
ಪುನ-ರ್ಜನ್ಮ
ಪುನ-ರ್ಜ-ನ್ಮದ
ಪುನ-ರ್ಜ-ನ್ಮ-ದ-ಲ್ಲಿಯೂ
ಪುನ-ರ್ಜ-ನ್ಮ-ವ-ನ್ನಾ-ಗಲಿ
ಪುನ-ರ್ಜ-ನ್ಮ-ವನ್ನು
ಪುನ-ರ್ಜ-ನ್ಮ-ವಿ-ರು-ವುದೊ
ಪುನ-ರ್ಜ-ನ್ಮ-ವಿ-ಲ್ಲದ
ಪುನ-ರ್ಧ-ನಮ್
ಪುನ-ರ್ಬ್ರಾ-ಹ್ಮ-ಣಾಃ
ಪುನ-ರ್ಮೋ-ಹ-ಮೇವಂ
ಪುನ-ರ್ಯೋಗಂ
ಪುನಶ್ಚ
ಪುನ-ಸ್ತ-ವಾ-ದಿಂ
ಪುನ-ಸ್ತಾನಿ
ಪುನಸ್ತ್ವಂ
ಪುಮಾಂ-ಶ್ಚ-ರತಿ
ಪುರ-ಜಿತ್ತು
ಪುರ-ದಲ್ಲಿ
ಪುರಲೆ
ಪುರ-ಷೋ-ತ್ತಮ
ಪುರ-ಸ್ಕ-ರಿ-ಸ-ಬೇ-ಕೆಂದು
ಪುರ-ಸ್ಕ-ರಿ-ಸಿ-ದರೆ
ಪುರ-ಸ್ಕ-ರಿ-ಸು-ತ್ತಾರೆ
ಪುರ-ಸ್ಕ-ರಿ-ಸು-ವು-ದಿಲ್ಲ
ಪುರ-ಸ್ಕಾರ
ಪುರ-ಸ್ಕಾ-ರ-ಗ-ಳೆ-ರಡೂ
ಪುರ-ಸ್ಕಾ-ರವೂ
ಪುರ-ಸ್ತಾ-ದಥ
ಪುರಾ
ಪುರಾಣ
ಪುರಾ-ಣ-ಗಳಲ್ಲಿ
ಪುರಾ-ಣ-ಗಳು
ಪುರಾ-ಣ-ದಲ್ಲಿ
ಪುರಾ-ಣ-ಪು-ರುಷ
ಪುರಾ-ಣ-ಪು-ರು-ಷನು
ಪುರಾ-ಣ-ಮ-ನು-ಶಾ-ಸಿ-ತಾ-ರ-ಮ-ಣೋ-ರ-ಣೀ-ಯಾಂ-ಸ-ಮ-ನು-ಸ್ಮ-ರೇದ್ಯಃ
ಪುರಾ-ಣ-ಮು-ನಿನಾ
ಪುರಾ-ಣವೂ
ಪುರಾ-ಣ-ಸ್ತ್ವ-ಮಸ್ಯ
ಪುರಾಣೀ
ಪುರಾಣೋ
ಪುರಾ-ತನ
ಪುರಾ-ತನಃ
ಪುರಾ-ತ-ನ-ವಾದ
ಪುರಾ-ತ-ನ-ವಾ-ದುದು
ಪುರಿ
ಪುರಿ-ಯನ್ನೋ
ಪುರು-ಜಿತ್
ಪುರುಷ
ಪುರುಷಂ
ಪುರುಷಃ
ಪುರು-ಷನ
ಪುರು-ಷ-ನಂತೆ
ಪುರು-ಷ-ನನ್ನು
ಪುರು-ಷ-ನನ್ನೂ
ಪುರು-ಷ-ನಾದ
ಪುರು-ಷ-ನಿಗೆ
ಪುರು-ಷನು
ಪುರು-ಷ-ನೆಂದೂ
ಪುರು-ಷನೇ
ಪುರು-ಷ-ಮು-ಪೈತಿ
ಪುರು-ಷರ
ಪುರು-ಷ-ರಲ್ಲಿ
ಪುರು-ಷ-ರಿ-ಗಿ-ರು-ವಷ್ಟೇ
ಪುರು-ಷರು
ಪುರು-ಷ-ರ್ಷಭ
ಪುರು-ಷ-ವ್ಯಾಘ್ರ
ಪುರು-ಷ-ಶ್ಚಾ-ಧಿ-ದೈ-ವ-ತಮ್
ಪುರು-ಷ-ಶ್ರೇ-ಷ್ಠನೆ
ಪುರು-ಷ-ಸಿಂ-ಹ-ರ-ನ್ನಾಗಿ
ಪುರು-ಷ-ಸೂ-ಕ್ತದ
ಪುರು-ಷ-ಸೂ-ಕ್ತ-ದಲ್ಲಿ
ಪುರು-ಷ-ಸ್ತ್ವನ್ಯಃ
ಪುರು-ಷಸ್ಯ
ಪುರುಷಾ
ಪುರು-ಷಾರ್ಥ
ಪುರು-ಷಾ-ರ್ಥ-ಗಳನ್ನು
ಪುರು-ಷಾ-ರ್ಥವೂ
ಪುರುಷೋ
ಪುರು-ಷೋ-ತ್ತಮ
ಪುರು-ಷೋ-ತ್ತಮಃ
ಪುರು-ಷೋ-ತ್ತ-ಮ-ನನ್ನು
ಪುರು-ಷೋ-ತ್ತ-ಮ-ನಲ್ಲಿ
ಪುರು-ಷೋ-ತ್ತ-ಮ-ನೆಂದು
ಪುರು-ಷೋ-ತ್ತ-ಮನೇ
ಪುರು-ಷೋ-ತ್ತ-ಮಮ್
ಪುರು-ಷೋ-ತ್ತ-ಮ-ಯೋಗ
ಪುರು-ಷೋ-ಽಶ್ನುತೇ
ಪುರುಷೌ
ಪುರೇ
ಪುರೋ-ಗ-ಮನ
ಪುರೋ-ಧ-ಸಾಂ
ಪುರೋ-ವಾಚ
ಪುರೋ-ಹಿತ
ಪುರೋ-ಹಿ-ತ-ನನ್ನು
ಪುರೋ-ಹಿ-ತರ
ಪುರೋ-ಹಿ-ತ-ರಿಗೆ
ಪುರೋ-ಹಿ-ತರು
ಪುರ್ಣಾ-ತ್ಮ-ನಾಗಿ
ಪುಲ್ಲಾ-ರ-ವಿಂ-ದಾ-ಯ-ತ-ಪ-ತ್ರ-ನೇತ್ರ
ಪುಷಿ
ಪುಷಿ-ಕ-ವಿ-ಗಳು
ಪುಷಿ-ಗಳ
ಪುಷಿ-ಗಳನ್ನೂ
ಪುಷಿ-ಗ-ಳಲ್ಲ
ಪುಷಿ-ಗಳಲ್ಲಿ
ಪುಷಿ-ಗ-ಳಿ-ಗಿಂತ
ಪುಷಿ-ಗ-ಳಿಗೆ
ಪುಷಿ-ಗಳು
ಪುಷಿ-ಭಿ-ರ್ಬ-ಹುಧಾ
ಪುಷಿ-ಯಜ್ಞ
ಪುಷೀಂಶ್ಚ
ಪುಷ್ಕ-ಲಾ-ಭಿಃ
ಪುಷ್ಕಳ
ಪುಷ್ಟ-ವಾದ
ಪುಷ್ಟಿ-ಕ-ರ-ವಾದ
ಪುಷ್ಣಾಮಿ
ಪುಷ್ಪ
ಪುಷ್ಪಂ
ಪುಷ್ಪ-ಗಳ
ಪುಷ್ಪ-ಗ-ಳಂತೆ
ಪುಷ್ಪ-ಗಳನ್ನು
ಪುಷ್ಪ-ಗಳಲ್ಲಿ
ಪುಷ್ಪ-ಗ-ಳಿವೆ
ಪುಷ್ಪ-ಗಳು
ಪುಷ್ಪ-ಗ-ಳೆಲ್ಲಾ
ಪುಷ್ಪಿ-ತಾಂ
ಪುಷ್ಯಾ-ಶ್ರ-ಮ-ವಲ್ಲ
ಪುಷ್ಯಾ-ಶ್ರ-ಮ-ವಾ-ಗು-ವುದು
ಪುಸ್ತಕ
ಪುಸ್ತ-ಕ-ಗಳನ್ನು
ಪುಸ್ತ-ಕ-ಗಳನ್ನೆಲ್ಲಾ
ಪುಸ್ತ-ಕ-ಗಳಿಂದ
ಪುಸ್ತ-ಕ-ಗಳು
ಪುಸ್ತ-ಕ-ದಂತೆ
ಪುಸ್ತ-ಕ-ದಿಂದ
ಪುಸ್ತ-ಕ-ರೂ-ಪಕ್ಕೆ
ಪುಸ್ತ-ಕ-ರೂ-ಪ-ದಲ್ಲಿ
ಪುಸ್ತ-ಕ-ವನ್ನು
ಪೂಜಾ
ಪೂಜಾ-ದಿ-ಗಳನ್ನು
ಪೂಜಾ-ದೃ-ಷ್ಟಿ-ಯಿಂದ
ಪೂಜಾ-ಪೀ-ಠಕ್ಕೆ
ಪೂಜಾ-ಭಾ-ವ-ನೆ-ಯನ್ನು
ಪೂಜಾ-ರಿ-ಯಷ್ಟೇ
ಪೂಜಾ-ರ್ಹ-ರಾದ
ಪೂಜಾ-ರ್ಹಾ-ವ-ರಿ-ಸೂ-ದನ
ಪೂಜಾ-ಸ್ಥಾ-ನಕ್ಕೆ
ಪೂಜಿ-ಸ-ಕೂ-ಡದು
ಪೂಜಿ-ಸ-ಬೇಕು
ಪೂಜಿ-ಸಲಿ
ಪೂಜಿ-ಸಲು
ಪೂಜಿಸಿ
ಪೂಜಿ-ಸಿ-ದರೂ
ಪೂಜಿ-ಸಿ-ದರೆ
ಪೂಜಿಸು
ಪೂಜಿ-ಸು-ತ್ತಾನೆ
ಪೂಜಿ-ಸು-ತ್ತಾರೆ
ಪೂಜಿ-ಸು-ತ್ತಾರೊ
ಪೂಜಿ-ಸು-ತ್ತಿ-ದ್ದರು
ಪೂಜಿ-ಸು-ತ್ತಿ-ರ-ಬ-ಹುದು
ಪೂಜಿ-ಸು-ತ್ತಿ-ರ-ಲಿ-ಕಲ್ಲು
ಪೂಜಿ-ಸುವ
ಪೂಜಿ-ಸು-ವರು
ಪೂಜಿ-ಸು-ವರೊ
ಪೂಜಿ-ಸು-ವ-ವನು
ಪೂಜಿ-ಸು-ವ-ವ-ರಿಗೆ
ಪೂಜಿ-ಸು-ವುದು
ಪೂಜೆ
ಪೂಜೆ-ಗ-ಳಿ-ಗಾಗಿ
ಪೂಜೆಗೂ
ಪೂಜೆಗೆ
ಪೂಜೆ-ಮಾ-ಡ-ಬೇ-ಕೆಂದು
ಪೂಜೆ-ಮಾ-ಡು-ತ್ತಾರೆ
ಪೂಜೆ-ಮಾ-ಡುವ
ಪೂಜೆ-ಮಾ-ಡು-ವಾಗ
ಪೂಜೆಯ
ಪೂಜೆ-ಯಂತೆ
ಪೂಜೆ-ಯನ್ನು
ಪೂಜೆ-ಯಾಗು
ಪೂಜೆ-ಯಾ-ಗು-ವುದು
ಪೂಜೆಯೂ
ಪೂಜೆಯೇ
ಪೂಜ್ಯ
ಪೂಜ್ಯ-ದೃ-ಷ್ಟಿ-ಯಿಂದ
ಪೂಜ್ಯನೂ
ಪೂಜ್ಯ-ಭಾವ
ಪೂಜ್ಯಸ್ಯ
ಪೂತ-ಪಾಪಾ
ಪೂತಾ
ಪೂತಿ
ಪೂರಕ
ಪೂರ-ಕ-ವಾ-ಗಿದೆ
ಪೂರ-ಕ-ವಾ-ಗಿ-ದೆಯೇ
ಪೂರುಷಃ
ಪೂರೈ-ಸಲು
ಪೂರೈಸಿ
ಪೂರೈ-ಸಿ-ಕೊಂಡು
ಪೂರೈ-ಸಿ-ಕೊ-ಳ್ಳು-ವು-ದಕ್ಕೆ
ಪೂರೈ-ಸಿದ
ಪೂರೈ-ಸಿ-ದಂತೆ
ಪೂರೈ-ಸು-ತ್ತಾನೆ
ಪೂರೈ-ಸು-ತ್ತಿ-ದ್ದ-ನಂತೆ
ಪೂರೈ-ಸುವ
ಪೂರೈ-ಸು-ವನು
ಪೂರೈ-ಸು-ವು-ದ-ಕ್ಕಾ-ಗು-ವು-ದಿಲ್ಲ
ಪೂರೈ-ಸು-ವು-ದಕ್ಕೆ
ಪೂರೈ-ಸು-ವು-ದಲ್ಲ
ಪೂರ್ಣ
ಪೂರ್ಣ-ಕಾಮ
ಪೂರ್ಣ-ಕಾ-ಮ-ನಾ-ಗಿ-ರು-ವನು
ಪೂರ್ಣ-ಕಾ-ಮ-ನಾ-ದ-ವನು
ಪೂರ್ಣಕ್ಕೆ
ಪೂರ್ಣ-ಗೊ-ಳಿ-ಸು-ವು-ದ-ಕ್ಕಾಗಿ
ಪೂರ್ಣ-ಗೊ-ಳಿ-ಸು-ವು-ದಕ್ಕೆ
ಪೂರ್ಣ-ಗೊ-ಳಿ-ಸು-ವುದು
ಪೂರ್ಣ-ಗ್ರಾ-ಹಿ-ಯಾ-ದದ್ದು
ಪೂರ್ಣ-ಜ್ಞಾನ
ಪೂರ್ಣತೆ
ಪೂರ್ಣ-ತೆಗೆ
ಪೂರ್ಣ-ತೆಯ
ಪೂರ್ಣ-ತೆ-ಯನ್ನು
ಪೂರ್ಣ-ತೆ-ಯೆ-ಡೆಗೆ
ಪೂರ್ಣ-ದ-ರ್ಶ-ನ-ವಲ್ಲ
ಪೂರ್ಣ-ದೃಷ್ಚಿ
ಪೂರ್ಣ-ದೃಷ್ಟಿ
ಪೂರ್ಣ-ದೃ-ಷ್ಟಿ-ಯಿಂದ
ಪೂರ್ಣ-ನಾಗಿ
ಪೂರ್ಣ-ನಾ-ಗು-ವು-ದಕ್ಕೆ
ಪೂರ್ಣ-ಭಾವ
ಪೂರ್ಣ-ಭಾ-ವ-ದಿಂದ
ಪೂರ್ಣ-ಮಾ-ಡಿ-ಕೊಂಡು
ಪೂರ್ಣ-ಮಾ-ಡಿ-ಕೊ-ಳ್ಳಲು
ಪೂರ್ಣ-ಮಾ-ಡಿ-ಕೊ-ಳ್ಳು-ತ್ತಾನೆ
ಪೂರ್ಣ-ರಾ-ಗು-ವು-ದಕ್ಕೆ
ಪೂರ್ಣ-ವಾ-ಗ-ಬ-ಹುದು
ಪೂರ್ಣ-ವಾ-ಗ-ಬೇ-ಕಾ-ದರೆ
ಪೂರ್ಣ-ವಾಗಿ
ಪೂರ್ಣ-ವಾ-ಗಿ-ಲ್ಲವೋ
ಪೂರ್ಣ-ವಾ-ಗು-ವು-ದಕ್ಕೆ
ಪೂರ್ಣ-ವಾ-ಗು-ವು-ದಿಲ್ಲ
ಪೂರ್ಣ-ಸ-ತ್ಯದ
ಪೂರ್ಣಾತ್ಮ
ಪೂರ್ಣಾ-ತ್ಮ-ನಾ-ಗ-ಬ-ಹುದು
ಪೂರ್ಣಾ-ತ್ಮ-ನಾ-ಗಿಲ್ಲ
ಪೂರ್ಣಾ-ತ್ಮ-ನಾದ
ಪೂರ್ಣಾ-ತ್ಮ-ನಾ-ದರೊ
ಪೂರ್ಣಾ-ತ್ಮ-ರ-ನ್ನಾಗಿ
ಪೂರ್ಣಾ-ತ್ಮ-ರಾ-ಗು-ವ-ವ-ರೆಗೆ
ಪೂರ್ಣಾ-ವ-ತಾ-ರ-ಗ-ಳಿವೆ
ಪೂರ್ಣಾ-ವ-ತಾ-ರ-ವೆಂದು
ಪೂರ್ಣಿ-ಮೆ-ಗ-ಳಿವೆ
ಪೂರ್ಣಿ-ಮೆಯ
ಪೂರ್ಣಿ-ಮೆ-ಯಾ-ಗಿ-ದ್ದು-ದ-ರಿಂದ
ಪೂರ್ತಿ
ಪೂರ್ತಿ-ಯಾ-ಗು-ವ-ವ-ರೆಗೆ
ಪೂರ್ವ
ಪೂರ್ವಕ
ಪೂರ್ವ-ಕ-ಲ್ಪಿತ
ಪೂರ್ವ-ಕ-ವಾಗಿ
ಪೂರ್ವ-ಕಾ-ಲದ
ಪೂರ್ವ-ಜ-ನ್ಮ-ಗಳ
ಪೂರ್ವ-ತರಂ
ಪೂರ್ವ-ದಲ್ಲಿ
ಪೂರ್ವ-ದಿಂದ
ಪೂರ್ವ-ದಿಂ-ದಲೂ
ಪೂರ್ವ-ಭಾ-ವಿ-ಯಾಗಿ
ಪೂರ್ವ-ಮೇವ
ಪೂರ್ವ-ಯು-ಗ-ದ-ಲ್ಲಿ-ದ್ದ-ವ-ರಿಗೆ
ಪೂರ್ವ-ವಾದ
ಪೂರ್ವ-ಸಂ-ಸ್ಕಾ-ರದ
ಪೂರ್ವಾ-ಚಾ-ರ-ವನ್ನು
ಪೂರ್ವಾ-ಪ-ರ-ವನ್ನು
ಪೂರ್ವಾ-ಭ್ಯಾ-ಸ-ದಿಂದ
ಪೂರ್ವಾ-ಭ್ಯಾ-ಸೇನ
ಪೂರ್ವಿ-ಕರು
ಪೂರ್ವಿ-ಕರೂ
ಪೂರ್ವೇ
ಪೂರ್ವೈಃ
ಪೂರ್ವೈ-ರಪಿ
ಪೂಸಿದ
ಪೃಚ್ಛಾಮಿ
ಪೃಥಕ್
ಪೃಥ-ಕ್ಕೇ-ಶಿ-ನಿ-ಷೂ-ದನ
ಪೃಥ-ಕ್ಚೇಷ್ಟಾ
ಪೃಥ-ಕ್ತ್ವೇನ
ಪೃಥ-ಗ್ಬಾ-ಲಾಃ
ಪೃಥ-ಗ್ಭಾವ
ಪೃಥ-ಗ್ವಿ-ಧಮ್
ಪೃಥ-ಗ್ವಿ-ಧಾಃ
ಪೃಥ-ಗ್ವಿ-ಧಾನ್
ಪೃಥಿ-ವೀಂ
ಪೃಥಿ-ವೀ-ಪತೇ
ಪೃಥಿ-ವ್ಯಾಂ
ಪೃಥ್ವಿ
ಪೃಥ್ವಿ-ಗಳ
ಪೃಥ್ವಿ-ಯಂ-ತಹ
ಪೃಥ್ವಿ-ಯ-ನ್ನೆಲ್ಲಾ
ಪೃಥ್ವಿ-ಯಲ್ಲಿ
ಪೃಥ್ವಿ-ಯ-ಲ್ಲಿಯೇ
ಪೃಥ್ವಿ-ಯಿಂದ
ಪೃಷ್ಠ-ತಸ್ತೇ
ಪೆಟ್ಟನ್ನು
ಪೆಟ್ಟಿಗೆ
ಪೆಟ್ಟಿ-ಗೆ-ಗಳು
ಪೆಟ್ಟಿ-ಗೆಗೆ
ಪೆಟ್ಟಿ-ಗೆ-ಯಂತೆ
ಪೆಟ್ಟಿ-ಗೆ-ಯಲ್ಲಿ
ಪೆಟ್ಟಿ-ಗೆ-ಯ-ಲ್ಲಿ-ರು-ವುದನ್ನು
ಪೆಟ್ಟಿ-ನಂತೆ
ಪೆಟ್ಟು
ಪೆಟ್ಟು-ಗಳು
ಪೆಡಂ-ಭೂ-ತ-ವಾಗಿ
ಪೆಡಲ್
ಪೆದ್ದ-ನಲ್ಲ
ಪೆರೇ-ಡಿ-ನಲ್ಲಿ
ಪೇಚಾ-ಡು-ವನು
ಪೇಚಾ-ಡು-ವ-ವ-ನಲ್ಲ
ಪೇಚಾ-ಡು-ವು-ದಿಲ್ಲ
ಪೇಚಾ-ಡು-ವುದೂ
ಪೇಟೆಗೆ
ಪೇಟೆಯ
ಪೇಪ-ರಿ-ನಲ್ಲಿ
ಪೇಪೀ-ಯ-ಮಾನಂ
ಪೈಪೋಟಿ
ಪೈರನ್ನು
ಪೈರು
ಪೈರೂ
ಪೈರೆಲ್ಲ
ಪೈಲ್ವಾ-ನ-ನಂತೆ
ಪೈಲ್ವಾ-ನ-ನಾ-ಗ-ಬೇ-ಕಾ-ದರೂ
ಪೈಲ್ವಾನ್
ಪೈಶಾ-ಚಿಕ
ಪೊದೆ-ಗಳಿಂದ
ಪೊದೆಗೆ
ಪೊದೆ-ಯ-ನ್ನಾ-ದರೂ
ಪೊರೆ
ಪೊರೆ-ಯನ್ನು
ಪೊರೆ-ಯಿಂದ
ಪೊಳ್ಳು
ಪೋಣಿ-ಸ-ಲ್ಪ-ಟ್ಟಿ-ದೆಯೋ
ಪೋಣಿ-ಸ-ಲ್ಪ-ಟ್ಟಿರು
ಪೋಣಿ-ಸ-ಲ್ಪ-ಟ್ಟಿ-ರುವ
ಪೋಣಿ-ಸ-ಲ್ಪ-ಟ್ಟಿವೆ
ಪೋಣಿ-ಸಿ-ರುವ
ಪೋಣಿ-ಸಿ-ರು-ವನು
ಪೋಣಿ-ಸಿ-ರು-ವರು
ಪೋಣಿ-ಸಿ-ರು-ವುದೇ
ಪೋಣಿ-ಸುವ
ಪೋಣಿ-ಸು-ವನು
ಪೋಲಿ-ಸ-ನ-ವರು
ಪೋಲಿ-ಸಿ-ನ-ವರು
ಪೋಲೀ-ಸಿ-ನ-ವ-ನಾ-ಗಲಿ
ಪೋಲೀ-ಸಿ-ನ-ವನು
ಪೋಲೀ-ಸಿ-ನ-ವ-ರಿಗೆ
ಪೋಲೀ-ಸಿ-ನ-ವರು
ಪೋಲೀಸು
ಪೋಷಕ
ಪೋಷ-ಕ-ದ್ರ-ವ್ಯ-ಗಳನ್ನೆಲ್ಲಾ
ಪೋಷ-ಕ-ವಾ-ಗಿದೆ
ಪೋಷ-ಕ-ವಾ-ಗಿ-ದೆಯೇ
ಪೋಷ-ಕ-ವಾ-ಗಿ-ರು-ವುದು
ಪೋಷಣೆ
ಪೋಷ-ಣೆಗೆ
ಪೋಷಾ-ಕಿ-ನಲ್ಲಿ
ಪೋಷಿ-ಸಿ-ಕೊ-ಳ್ಳ-ಬೇ-ಕಾ-ಗಿದೆ
ಪೋಷಿ-ಸಿ-ದೆವು
ಪೋಷಿಸು
ಪೋಷಿ-ಸು-ತ್ತಿ-ರು-ವುದು
ಪೋಷಿ-ಸು-ತ್ತೇನೆ
ಪೋಷಿ-ಸುವ
ಪೋಷಿ-ಸು-ವು-ದ-ಕ್ಕಾಗಿ
ಪೋಷಿ-ಸು-ವು-ದ-ಕ್ಕಾ-ಗಿಯೇ
ಪೋಷಿ-ಸು-ವು-ದಕ್ಕೆ
ಪೋಷಿ-ಸು-ವುದು
ಪೋಸ್ಟಾ-ಫೀ-ಸಿ-ನ-ವರು
ಪೋಸ್ಟಾ-ಫೀ-ಸಿ-ನಿಂದ
ಪೋಸ್ಟಿನ
ಪೋಸ್ಟ್
ಪೌಂಡ್ರಂ
ಪೌಂಡ್ರ-ಕ-ವನ್ನು
ಪೌತ್ರಾ
ಪೌತ್ರಾನ್
ಪೌರ-ಷ-ವಾ-ಗಿ-ದ್ದೇನೆ
ಪೌರ-ಷವೂ
ಪೌರಾ-ಣಿಕ
ಪೌರುಷ
ಪೌರುಷಂ
ಪೌರು-ಷದ
ಪೌರು-ಷಮ್
ಪೌರು-ಷ-ವನ್ನು
ಪೌರು-ಷವೇ
ಪೌರ್ವ-ದೇ-ಹಿ-ಕಮ್
ಪ್
ಪ್ತತ-ಪಂತಿ
ಪ್ಪಪ-ದ್ಯಂ-ತೇ-ಽನ್ಯ-ದೇ-ವ-ತಾಃ
ಪ್ರಂಚೇಂ-ದ್ರಿಯ
ಪ್ರಕ-ಟ-ಗೊ-ಳಿ-ಸು-ತ್ತಿ-ರು-ವುದು
ಪ್ರಕ-ಟ-ವಾಗಿ
ಪ್ರಕ-ಟ-ವಾಗು
ಪ್ರಕ-ಟ-ವಾ-ದರೆ
ಪ್ರಕ-ಟಿತ
ಪ್ರಕ-ಟಿ-ಸಿದೆ
ಪ್ರಕಾರ
ಪ್ರಕಾ-ರ-ಗ-ಳಿಂ-ದಲೂ
ಪ್ರಕಾ-ರ-ದಿಂ-ದಲೂ
ಪ್ರಕಾ-ರ-ವಾಗಿ
ಪ್ರಕಾಶ
ಪ್ರಕಾಶಂ
ಪ್ರಕಾಶಃ
ಪ್ರಕಾ-ಶ-ಕ-ಮ-ನಾ-ಮ-ಯಮ್
ಪ್ರಕಾ-ಶ-ಕ-ರಿಗೆ
ಪ್ರಕಾ-ಶ-ಕರು
ಪ್ರಕಾ-ಶ-ಗೊ-ಳಿ-ಸು-ವುದೋ
ಪ್ರಕಾ-ಶದ
ಪ್ರಕಾ-ಶ-ಮಾ-ನ-ವಾ-ಗಿದೆ
ಪ್ರಕಾ-ಶ-ಮಾ-ನ-ವಾದ
ಪ್ರಕಾ-ಶ-ಮಾ-ನ-ವಾ-ದು-ದನ್ನು
ಪ್ರಕಾ-ಶ-ಯತಿ
ಪ್ರಕಾ-ಶ-ಯ-ತ್ಯೇಕಃ
ಪ್ರಕಾ-ಶ-ವಾ-ಗಿದೆ
ಪ್ರಕಾ-ಶ-ವಾ-ಗು-ತ್ತಿ-ರು-ವುದು
ಪ್ರಕಾ-ಶ-ವಾ-ದಂತೆ
ಪ್ರಕಾ-ಶ-ವು-ಳ್ಳ-ವನೂ
ಪ್ರಕಾ-ಶಿ-ಸ-ಲಾ-ರದು
ಪ್ರಕಾ-ಶಿಸಿ
ಪ್ರಕಾ-ಶಿ-ಸು-ತ್ತಿದೆ
ಪ್ರಕಾ-ಶಿ-ಸು-ತ್ತಿ-ರುವ
ಪ್ರಕಾ-ಶಿ-ಸು-ತ್ತಿ-ರು-ವನು
ಪ್ರಕಾ-ಶಿ-ಸು-ತ್ತಿ-ರು-ವನೋ
ಪ್ರಕಾ-ಶಿ-ಸು-ತ್ತಿ-ರು-ವುದು
ಪ್ರಕಾ-ಶಿ-ಸು-ತ್ತಿವೆ
ಪ್ರಕಾ-ಶಿ-ಸುವ
ಪ್ರಕಾ-ಶಿ-ಸು-ವಾಗ
ಪ್ರಕಾ-ಶಿ-ಸು-ವುದು
ಪ್ರಕೀರ್ತ್ಯಾ
ಪ್ರಕೃತಿ
ಪ್ರಕೃ-ತಿಂ
ಪ್ರಕೃ-ತಿಃ
ಪ್ರಕೃ-ತಿ-ಗ-ಳೆ-ರ-ಡನ್ನೂ
ಪ್ರಕೃ-ತಿ-ಗಿಂತ
ಪ್ರಕೃ-ತಿಗೂ
ಪ್ರಕೃ-ತಿಗೆ
ಪ್ರಕೃ-ತಿಗೇ
ಪ್ರಕೃ-ತಿ-ಜಾನ್
ಪ್ರಕೃ-ತಿ-ಜೈ-ರ್ಗು-ಣೈಃ
ಪ್ರಕೃ-ತಿ-ಜೈ-ರ್ಮುಕ್ತಂ
ಪ್ರಕೃ-ತಿ-ಮಾ-ಶ್ರಿ-ತಾಃ
ಪ್ರಕೃ-ತಿಯ
ಪ್ರಕೃ-ತಿ-ಯಂತೆ
ಪ್ರಕೃ-ತಿ-ಯನ್ನು
ಪ್ರಕೃ-ತಿ-ಯನ್ನೂ
ಪ್ರಕೃ-ತಿ-ಯಲ್ಲ
ಪ್ರಕೃ-ತಿ-ಯಲ್ಲಿ
ಪ್ರಕೃ-ತಿ-ಯ-ಲ್ಲಿ-ಇವೆ
ಪ್ರಕೃ-ತಿ-ಯ-ಲ್ಲಿದೆ
ಪ್ರಕೃ-ತಿ-ಯ-ಲ್ಲಿ-ರುವ
ಪ್ರಕೃ-ತಿ-ಯ-ಲ್ಲಿವೆ
ಪ್ರಕೃ-ತಿ-ಯ-ವರೇ
ಪ್ರಕೃ-ತಿ-ಯ-ವ-ರೊ-ಡನೆ
ಪ್ರಕೃ-ತಿ-ಯಿಂದ
ಪ್ರಕೃ-ತಿಯು
ಪ್ರಕೃ-ತಿ-ಯು-ಳ್ಳ-ವರು
ಪ್ರಕೃ-ತಿ-ಯೆಲ್ಲ
ಪ್ರಕೃ-ತಿ-ಯೆಲ್ಲಾ
ಪ್ರಕೃ-ತಿಯೇ
ಪ್ರಕೃ-ತಿ-ಯೊ-ಳಗೆ
ಪ್ರಕೃ-ತಿ-ಯೊ-ಳಗೇ
ಪ್ರಕೃ-ತಿ-ರ-ಷ್ಟಧಾ
ಪ್ರಕೃ-ತಿ-ರು-ಚ್ಯತೇ
ಪ್ರಕೃ-ತಿ-ಸಂ-ಭ-ವಾಃ
ಪ್ರಕೃ-ತಿ-ಸಂ-ಭ-ವಾನ್
ಪ್ರಕೃ-ತಿ-ಸ್ತ್ವಾಂ
ಪ್ರಕೃ-ತಿ-ಸ್ಥ-ನಾದ
ಪ್ರಕೃ-ತಿ-ಸ್ಥಾನಿ
ಪ್ರಕೃ-ತಿಸ್ಥೋ
ಪ್ರಕೃ-ತೇಃ
ಪ್ರಕೃ-ತೇ-ರ್ಗು-ಣ-ಸಂ-ಮೂ-ಢಾಃ
ಪ್ರಕೃ-ತೇ-ರ್ಜ್ಞಾ-ನ-ವಾ-ನಪಿ
ಪ್ರಕೃ-ತೇ-ವ-ಶಾತ್
ಪ್ರಕೃತ್ಯಾ
ಪ್ರಕೃ-ತ್ಯೈವ
ಪ್ರಖ್ಯಾತ
ಪ್ರಖ್ಯಾ-ತ-ನಾ-ಗ-ಬೇಕು
ಪ್ರಖ್ಯಾ-ತ-ನಾ-ಗಿ-ದ್ದಾನೆ
ಪ್ರಖ್ಯಾ-ತ-ನಾ-ಗಿ-ರು-ವನು
ಪ್ರಖ್ಯಾ-ತ-ನಾದ
ಪ್ರಖ್ಯಾ-ತ-ನಾ-ದ-ವನು
ಪ್ರಖ್ಯಾ-ತ-ನಾ-ದ-ವನೆ
ಪ್ರಖ್ಯಾ-ತ-ರಾ-ಗಿ-ದ್ದಾರೆ
ಪ್ರಖ್ಯಾ-ತ-ರಾ-ಗಿ-ರು-ವರೋ
ಪ್ರಖ್ಯಾ-ತ-ರಾದ
ಪ್ರಖ್ಯಾ-ತ-ರಾ-ದ-ವರು
ಪ್ರಖ್ಯಾ-ತರು
ಪ್ರಖ್ಯಾ-ತ-ವಾ-ಗಿತ್ತು
ಪ್ರಖ್ಯಾ-ತ-ವಾ-ಗಿದೆ
ಪ್ರಖ್ಯಾ-ತ-ವಾದ
ಪ್ರಖ್ಯಾ-ತ-ವ್ಯ-ಕ್ತಿಯ
ಪ್ರಖ್ಯಾ-ತಿ-ಯನ್ನು
ಪ್ರಖ್ಯಾ-ತಿಯೂ
ಪ್ರಗತಿ
ಪ್ರಚಂಡ
ಪ್ರಚಂ-ಡ-ನಾ-ಗಿ-ರ-ಬ-ಹುದು
ಪ್ರಚಂ-ಡ-ವಾ-ದುದು
ಪ್ರಚ-ಲಿ-ತ-ವಾದ
ಪ್ರಚಾರ
ಪ್ರಚಾ-ರಕ
ಪ್ರಚಾ-ರ-ದಲ್ಲಿ
ಪ್ರಚಾ-ರ-ದ-ಲ್ಲಿದ್ದ
ಪ್ರಚಾ-ರ-ಮಾ-ಡು-ತ್ತಾ-ನೆಯೋ
ಪ್ರಚಾ-ರ-ವಾ-ಗು-ವು-ದಕ್ಕೆ
ಪ್ರಚಾ-ರ-ವಾ-ಗು-ವುದು
ಪ್ರಚು-ರ-ವಾದ
ಪ್ರಚೋ-ದನೆ
ಪ್ರಚೋ-ದ-ನೆಗೆ
ಪ್ರಚೋ-ದ-ನೆಯ
ಪ್ರಚೋ-ದಿ-ತ-ನಾ-ದಾಗ
ಪ್ರಚೋ-ದಿ-ಸ-ಬೇ-ಕೆಂದು
ಪ್ರಚೋ-ದಿ-ಸಲಿ
ಪ್ರಚೋ-ದಿಸಿ
ಪ್ರಚೋ-ದಿ-ಸು-ತ್ತಾರೆ
ಪ್ರಚೋ-ದಿ-ಸುವ
ಪ್ರಚೋ-ದಿ-ಸು-ವುದು
ಪ್ರಜ-ನ-ಶ್ಚಾಸ್ಮಿ
ಪ್ರಜ-ಹಾತಿ
ಪ್ರಜಹಿ
ಪ್ರಜಾಃ
ಪ್ರಜಾ-ನಾತಿ
ಪ್ರಜಾ-ನಾಮಿ
ಪ್ರಜಾ-ಪತಿ
ಪ್ರಜಾ-ಪ-ತಿಃ
ಪ್ರಜಾ-ಪ-ತಿಸ್ತ್ವಂ
ಪ್ರಜೆ-ಗಳನ್ನು
ಪ್ರಜೆ-ಗ-ಳಿಗೆ
ಪ್ರಜೆ-ಗಳು
ಪ್ರಜೆ-ಗ-ಳೆಲ್ಲ
ಪ್ರಜೋ-ತ್ಪ-ತ್ತಿಗೆ
ಪ್ರಜ್ಞಾ
ಪ್ರಜ್ಞಾಂ
ಪ್ರಜ್ಞಾ-ವಾ-ದಾಂಶ್ಚ
ಪ್ರಜ್ಞೆ
ಪ್ರಜ್ಞೆ-ಯನ್ನು
ಪ್ರಜ್ವ-ಲಿ-ಸು-ತ್ತಿ-ರುವ
ಪ್ರಜ್ವಾ-ಲಿತೋ
ಪ್ರಣಮ್ಯ
ಪ್ರಣ-ಯೇನ
ಪ್ರಣವ
ಪ್ರಣವಃ
ಪ್ರಣ-ವ-ದ-ಲ್ಲಿದೆ
ಪ್ರಣ-ವ-ವನ್ನು
ಪ್ರಣ-ಶ್ಯಂತಿ
ಪ್ರಣ-ಶ್ಯತಿ
ಪ್ರಣ-ಶ್ಯಾಮಿ
ಪ್ರಣಾಮ
ಪ್ರಣಿ-ಧಾಯ
ಪ್ರಣಿ-ಪಾ-ತೇನ
ಪ್ರತಾ-ಪ-ವನ್ನು
ಪ್ರತಾ-ಪ-ವಾನ್
ಪ್ರತಾ-ಪ-ಶಾ-ಲಿಯೂ
ಪ್ರತಿ
ಪ್ರತಿ-ಕ್ರಿ-ಯಾ-ರೂ-ಪ-ವಾಗಿ
ಪ್ರತಿ-ಕ್ರಿಯೆ
ಪ್ರತಿ-ಕ್ರಿ-ಯೆಗೆ
ಪ್ರತಿ-ಕ್ರಿ-ಯೆ-ಯನ್ನು
ಪ್ರತಿ-ಕ್ರಿ-ಯೆ-ಯಾ-ಗು-ವುದು
ಪ್ರತಿ-ಕ್ರಿ-ಯೆ-ಯಿಂದ
ಪ್ರತಿ-ಕ್ರಿ-ಯೆಯೇ
ಪ್ರತಿ-ಜ-ನ್ಮ-ದ-ಲ್ಲಿಯೂ
ಪ್ರತಿ-ಜಾನೇ
ಪ್ರತಿಜ್ಞೆ
ಪ್ರತಿ-ಜ್ಞೆಯ
ಪ್ರತಿ-ಜ್ಞೆ-ಯನ್ನು
ಪ್ರತಿ-ಜ್ಞೆ-ಯಿಂದ
ಪ್ರತಿ-ದಿನ
ಪ್ರತಿ-ದಿ-ನವೂ
ಪ್ರತಿ-ನಿ-ತ್ಯವೂ
ಪ್ರತಿ-ನಿಧಿ
ಪ್ರತಿ-ನಿ-ಧಿ-ಗಳೇ
ಪ್ರತಿ-ನಿ-ಧಿ-ಯಂತೆ
ಪ್ರತಿ-ನಿ-ಧಿ-ಯನ್ನು
ಪ್ರತಿ-ನಿ-ಮಿ-ಷವೂ
ಪ್ರತಿ-ಪ-ಕ್ಷದ
ಪ್ರತಿ-ಪ-ದ್ಯತೇ
ಪ್ರತಿ-ಪಾ-ದಿ-ಸಿ-ರು-ವರು
ಪ್ರತಿ-ಪಾ-ದಿ-ಸುವ
ಪ್ರತಿ-ಪಾ-ದಿ-ಸು-ವಾಗ
ಪ್ರತಿ-ಪಾದ್ಯ
ಪ್ರತಿ-ಫಲ
ಪ್ರತಿ-ಫ-ಲಕ್ಕೆ
ಪ್ರತಿ-ಫ-ಲ-ವನ್ನು
ಪ್ರತಿ-ಫ-ಲ-ವ-ನ್ನೆಲ್ಲ
ಪ್ರತಿ-ಫ-ಲವೇ
ಪ್ರತಿ-ಫ-ಲಾ-ಕಾಂ-ಕ್ಷೆಯೂ
ಪ್ರತಿ-ಫ-ಲಾ-ಪೇಕ್ಷೆ
ಪ್ರತಿ-ಫ-ಲಾ-ಪೇ-ಕ್ಷೆ-ಯನ್ನು
ಪ್ರತಿ-ಫ-ಲಾ-ಪೇ-ಕ್ಷೆ-ಯನ್ನೂ
ಪ್ರತಿ-ಫ-ಲಾ-ಪೇ-ಕ್ಷೆ-ಯಿ-ಲ್ಲದೆ
ಪ್ರತಿ-ಫ-ಲಾ-ಪೇ-ಕ್ಷೆಯೂ
ಪ್ರತಿ-ಬ-ಯ-ಸು-ವುದೂ
ಪ್ರತಿ-ಬಿಂಬ
ಪ್ರತಿ-ಬಿಂ-ಬ-ದಲ್ಲಿ
ಪ್ರತಿ-ಬಿಂ-ಬ-ವನ್ನು
ಪ್ರತಿ-ಬಿಂ-ಬಿಸ
ಪ್ರತಿ-ಬಿಂ-ಬಿ-ಸದೇ
ಪ್ರತಿ-ಬಿಂ-ಬಿ-ಸ-ಬೇ-ಕಾ-ದರೂ
ಪ್ರತಿ-ಬಿಂ-ಬಿ-ಸ-ಬೇ-ಕಾ-ದರೆ
ಪ್ರತಿ-ಬಿಂ-ಬಿ-ಸ-ಲಾ-ರದೋ
ಪ್ರತಿ-ಬಿಂ-ಬಿ-ಸ-ಲಾ-ರವು
ಪ್ರತಿ-ಬಿಂ-ಬಿ-ಸಲು
ಪ್ರತಿ-ಬಿಂ-ಬಿ-ಸಲೇ
ಪ್ರತಿ-ಬಿಂ-ಬಿಸು
ಪ್ರತಿ-ಬಿಂ-ಬಿ-ಸು-ತ್ತದೆ
ಪ್ರತಿ-ಬಿಂ-ಬಿ-ಸು-ತ್ತವೆ
ಪ್ರತಿ-ಬಿಂ-ಬಿ-ಸು-ತ್ತಿದೆ
ಪ್ರತಿ-ಬಿಂ-ಬಿ-ಸು-ತ್ತಿ-ರ-ಬ-ಹುದು
ಪ್ರತಿ-ಬಿಂ-ಬಿ-ಸು-ತ್ತಿ-ರು-ವುದು
ಪ್ರತಿ-ಬಿಂ-ಬಿ-ಸು-ತ್ತಿಲ್ಲ
ಪ್ರತಿ-ಬಿಂ-ಬಿ-ಸು-ತ್ತಿವೆ
ಪ್ರತಿ-ಬಿಂ-ಬಿ-ಸುವ
ಪ್ರತಿ-ಬಿಂ-ಬಿ-ಸು-ವಂತೆ
ಪ್ರತಿ-ಬಿಂ-ಬಿ-ಸು-ವನು
ಪ್ರತಿ-ಬಿಂ-ಬಿ-ಸು-ವ-ವರೇ
ಪ್ರತಿ-ಬಿಂ-ಬಿ-ಸು-ವು-ದ-ರಲ್ಲಿ
ಪ್ರತಿ-ಬಿಂ-ಬಿ-ಸು-ವು-ದ-ರಿಂದ
ಪ್ರತಿ-ಬಿಂ-ಬಿ-ಸು-ವು-ದಿಲ್ಲ
ಪ್ರತಿ-ಬಿಂ-ಬಿ-ಸು-ವುದು
ಪ್ರತಿ-ಬಿಂ-ಬಿ-ಸು-ವುದೂ
ಪ್ರತಿ-ಬಿಂ-ಬಿ-ಸು-ವುವು
ಪ್ರತಿ-ಬಿಂ-ಬಿ-ಸು-ವುವೊ
ಪ್ರತಿ-ಬೋ-ಧಿ-ತಾಂ
ಪ್ರತಿ-ಭ-ಟಿ-ಸು-ತ್ತಾರೆ
ಪ್ರತಿ-ಭಾ-ವಂತ
ಪ್ರತಿ-ಭೆಗೆ
ಪ್ರತಿ-ಭೆ-ಯನ್ನು
ಪ್ರತಿ-ಯೊಂ-ದಕ್ಕೂ
ಪ್ರತಿ-ಯೊಂ-ದನ್ನೂ
ಪ್ರತಿ-ಯೊಂ-ದರ
ಪ್ರತಿ-ಯೊಂ-ದ-ರ-ಲ್ಲಿಯೂ
ಪ್ರತಿ-ಯೊಂದು
ಪ್ರತಿ-ಯೊಂದೂ
ಪ್ರತಿ-ಯೊ-ಬ-ಪ್ರಿಗೂ
ಪ್ರತಿ-ಯೊಬ್ಬ
ಪ್ರತಿ-ಯೊ-ಬ್ಬ-ನಿಗೂ
ಪ್ರತಿ-ಯೊ-ಬ್ಬನು
ಪ್ರತಿ-ಯೊ-ಬ್ಬನೂ
ಪ್ರತಿ-ಯೊ-ಬ್ಬರ
ಪ್ರತಿ-ಯೊ-ಬ್ಬ-ರ-ಲ್ಲಿಯೂ
ಪ್ರತಿ-ಯೊ-ಬ್ಬ-ರಿಗೂ
ಪ್ರತಿ-ಯೊ-ಬ್ಬರೂ
ಪ್ರತಿ-ಯೋ-ತ್ಸ್ಯಾಮಿ
ಪ್ರತಿ-ವ-ರ್ಷವೂ
ಪ್ರತಿ-ವಾ-ದ-ವನ್ನು
ಪ್ರತಿ-ವಿಂದ್ಯ
ಪ್ರತಿ-ಷ್ಠಾಪ್ಯ
ಪ್ರತಿ-ಷ್ಠಾ-ಹ-ಮ-ಮೃ-ತ-ಸ್ಯಾ-ವ್ಯ-ಯಸ್ಯ
ಪ್ರತಿ-ಷ್ಠಿ-ತಮ್
ಪ್ರತಿ-ಷ್ಠಿ-ತ-ವಾ-ಗಿದೆ
ಪ್ರತಿ-ಷ್ಠಿ-ತ-ವಾ-ಗಿ-ರು-ವುದು
ಪ್ರತಿ-ಷ್ಠಿ-ತ-ವಾದ
ಪ್ರತಿ-ಷ್ಠಿತಾ
ಪ್ರತಿಷ್ಠೆ
ಪ್ರತಿ-ಷ್ಠೆ-ಮಾ-ಡಿ-ರು-ವನು
ಪ್ರತಿ-ಸ-ಲವೂ
ಪ್ರತೀಕ
ಪ್ರತೀ-ಕಾ-ರ-ವನ್ನು
ಪ್ರತೀ-ಕಾ-ರ-ವನ್ನೂ
ಪ್ರತೀತಿ
ಪ್ರತ್ಯಕ್ಷ
ಪ್ರತ್ಯ-ಕ್ಷ-ನಾ-ದಂತೆ
ಪ್ರತ್ಯ-ಕ್ಷ-ನಾ-ದರೆ
ಪ್ರತ್ಯ-ಕ್ಷ-ನಾ-ದಾಗ
ಪ್ರತ್ಯ-ಕ್ಷ-ಳಾ-ದಳು
ಪ್ರತ್ಯ-ಕ್ಷ-ಳಾ-ದೆ-ಯಲ್ಲ
ಪ್ರತ್ಯ-ಕ್ಷ-ವಾಗಿ
ಪ್ರತ್ಯ-ಕ್ಷ-ವಾ-ಗಿಯೆ
ಪ್ರತ್ಯ-ಕ್ಷ-ವಾ-ಗು-ತ್ತಾನೆ
ಪ್ರತ್ಯ-ಕ್ಷ-ವಾದ
ಪ್ರತ್ಯ-ಕ್ಷ-ವೊಂದೇ
ಪ್ರತ್ಯ-ಕ್ಷಾದಿ
ಪ್ರತ್ಯ-ಕ್ಷಾ-ವ-ಗಮಂ
ಪ್ರತ್ಯ-ನೀ-ಕೇಶು
ಪ್ರತ್ಯಯ
ಪ್ರತ್ಯ-ಯಕ್ಕೆ
ಪ್ರತ್ಯ-ವಾಯೋ
ಪ್ರತ್ಯಾ-ಹಾರ
ಪ್ರತ್ಯು-ಪ-ಕಾ-ರ-ಕ್ಕಾಗಿ
ಪ್ರತ್ಯು-ಪ-ಕಾ-ರ-ವನ್ನು
ಪ್ರತ್ಯು-ಪ-ಕಾ-ರಾರ್ಥಂ
ಪ್ರತ್ಯೇ-ಕ-ವಾಗಿ
ಪ್ರತ್ಯೇ-ಕಿ-ಸ-ಬಲ್ಲ
ಪ್ರತ್ಯೇ-ಕಿ-ಸು-ತ್ತಾನೆ
ಪ್ರಥಮ
ಪ್ರಥ-ಮ-ದಲ್ಲಿ
ಪ್ರಥ-ಮ-ದ-ಲ್ಲಿಯೇ
ಪ್ರಥಿತಃ
ಪ್ರದ-ಕ್ಷಿಣೆ
ಪ್ರದ-ಧ್ಮ-ತುಃ
ಪ್ರದ-ರ್ಶನ
ಪ್ರದ-ರ್ಶ-ನ-ವನ್ನು
ಪ್ರದ-ರ್ಶಿ-ಸು-ತ್ತಾನೆ
ಪ್ರದಿ-ಷ್ಟಮ್
ಪ್ರದೀಪ
ಪ್ರದೀ-ಪಕ್ಕೆ
ಪ್ರದೀಪ್ತಂ
ಪ್ರದೀ-ಪ್ತ-ವಾದ
ಪ್ರದು-ಶ್ಯಂತಿ
ಪ್ರದೇಶ
ಪ್ರದೇ-ಶಕ್ಕೆ
ಪ್ರದೇ-ಶ-ದಲ್ಲಿ
ಪ್ರದೇ-ಶ-ದ-ಲ್ಲಿ-ದ್ದು-ಕೊಂಡು
ಪ್ರದೇ-ಶ-ವನ್ನು
ಪ್ರದ್ವಿ-ಷಂ-ತೋ-ಽಭ್ಯ-ಸೂ-ಯ-ಕಾಃ
ಪ್ರಧಾನ
ಪ್ರಧಾ-ನ-ವಾಗಿ
ಪ್ರಧಾ-ನ-ವಾ-ಗಿ-ರ-ಬೇ-ಕಾ-ದರೆ
ಪ್ರಧಾ-ನ-ವಾ-ಗಿ-ರುವ
ಪ್ರಧಾ-ನ-ವಾ-ಗಿ-ರು-ವ-ವನು
ಪ್ರಧಾ-ನ-ವಾ-ಗಿ-ರು-ವಾಗ
ಪ್ರಧಾ-ನ-ವಾ-ಗಿ-ರು-ವುದು
ಪ್ರಧಾ-ನ-ವಾದ
ಪ್ರಧಾ-ನ-ವಾ-ದ-ವರೇ
ಪ್ರಧಾ-ನ-ವಾ-ದುದು
ಪ್ರನ-ಷ್ಟಸ್ತೇ
ಪ್ರಪಂಚ
ಪ್ರಪಂ-ಚ-ಕ್ಕಿಂತ
ಪ್ರಪಂ-ಚಕ್ಕೂ
ಪ್ರಪಂ-ಚಕ್ಕೆ
ಪ್ರಪಂ-ಚ-ಕ್ಕೆಲ್ಲ
ಪ್ರಪಂ-ಚದ
ಪ್ರಪಂ-ಚ-ದಂತೆ
ಪ್ರಪಂ-ಚ-ದ-ಮೇ-ಲೆಲ್ಲ
ಪ್ರಪಂ-ಚ-ದ-ಲ್ಲಾ-ದರೂ
ಪ್ರಪಂ-ಚ-ದಲ್ಲಿ
ಪ್ರಪಂ-ಚ-ದ-ಲ್ಲಿ-ದ್ದರೂ
ಪ್ರಪಂ-ಚ-ದ-ಲ್ಲಿ-ದ್ದಾನೆ
ಪ್ರಪಂ-ಚ-ದ-ಲ್ಲಿಯೇ
ಪ್ರಪಂ-ಚ-ದ-ಲ್ಲಿ-ರ-ಬೇ-ಕಾ-ದರೆ
ಪ್ರಪಂ-ಚ-ದ-ಲ್ಲಿರು
ಪ್ರಪಂ-ಚ-ದ-ಲ್ಲಿ-ರುವ
ಪ್ರಪಂ-ಚ-ದ-ಲ್ಲಿ-ರು-ವನು
ಪ್ರಪಂ-ಚ-ದ-ಲ್ಲಿ-ರು-ವನೊ
ಪ್ರಪಂ-ಚ-ದ-ಲ್ಲಿ-ರು-ವನೋ
ಪ್ರಪಂ-ಚ-ದ-ಲ್ಲಿ-ರು-ವರು
ಪ್ರಪಂ-ಚ-ದ-ಲ್ಲಿ-ರು-ವ-ವರೆಲ್ಲ
ಪ್ರಪಂ-ಚ-ದ-ಲ್ಲಿ-ರು-ವಾಗ
ಪ್ರಪಂ-ಚ-ದ-ಲ್ಲಿ-ರು-ವಾ-ಗಲೂ
ಪ್ರಪಂ-ಚ-ದ-ಲ್ಲಿ-ರು-ವಾ-ಗಲೇ
ಪ್ರಪಂ-ಚ-ದ-ಲ್ಲಿ-ರು-ವು-ದಕ್ಕೆ
ಪ್ರಪಂ-ಚ-ದ-ಲ್ಲಿ-ರು-ವು-ದ-ನ್ನೆಲ್ಲ
ಪ್ರಪಂ-ಚ-ದ-ಲ್ಲಿ-ರು-ವು-ದೆಲ್ಲ
ಪ್ರಪಂ-ಚ-ದಲ್ಲೆ
ಪ್ರಪಂ-ಚ-ದಲ್ಲೆಲ್ಲ
ಪ್ರಪಂ-ಚ-ದಲ್ಲೆಲ್ಲಾ
ಪ್ರಪಂ-ಚ-ದಲ್ಲೇ
ಪ್ರಪಂ-ಚ-ದಿಂದ
ಪ್ರಪಂ-ಚ-ದೊ-ಳಗೆ
ಪ್ರಪಂ-ಚ-ನ್ನೆಲ್ಲ
ಪ್ರಪಂ-ಚವ
ಪ್ರಪಂ-ಚ-ವನ್ನು
ಪ್ರಪಂ-ಚ-ವನ್ನೂ
ಪ್ರಪಂ-ಚ-ವ-ನ್ನೆಲ್ಲ
ಪ್ರಪಂ-ಚ-ವ-ನ್ನೆಲ್ಲಾ
ಪ್ರಪಂ-ಚ-ವನ್ನೇ
ಪ್ರಪಂ-ಚ-ವಲ್ಲ
ಪ್ರಪಂ-ಚ-ವಾ-ಗಿದೆ
ಪ್ರಪಂ-ಚ-ವಾ-ಗು-ವುದು
ಪ್ರಪಂ-ಚವು
ಪ್ರಪಂ-ಚವೆ
ಪ್ರಪಂ-ಚ-ವೆಂದರೆ
ಪ್ರಪಂ-ಚ-ವೆಂಬ
ಪ್ರಪಂ-ಚ-ವೆಲ್ಲ
ಪ್ರಪಂ-ಚ-ವೆಲ್ಲಾ
ಪ್ರಪಂ-ಚವೇ
ಪ್ರಪಂ-ಚ-ವೇನು
ಪ್ರಪ-ದ್ಯಂತೇ
ಪ್ರಪ-ದ್ಯತೇ
ಪ್ರಪದ್ಯೇ
ಪ್ರಪ-ನ್ನ-ಪಾ-ರಿ-ಜಾ-ತಾಯ
ಪ್ರಪ-ನ್ನಮ್
ಪ್ರಪಶ್ಯ
ಪ್ರಪ-ಶ್ಯ-ದ್ಭಿ-ರ್ಜ-ನಾ-ರ್ದನ
ಪ್ರಪ-ಶ್ಯಾಮಿ
ಪ್ರಪಿ-ತಾ-ಮ-ಹಶ್ಚ
ಪ್ರಪ್ರ-ಥ-ಮ-ವಾದ
ಪ್ರಬಲ
ಪ್ರಬ-ಲ-ವಾಗಿ
ಪ್ರಬ-ಲ-ವಾದ
ಪ್ರಬು-ದ್ಧ-ನಾದ
ಪ್ರಬು-ದ್ಧ-ವಾ-ಗಿ-ರು-ವುದನ್ನು
ಪ್ರಬು-ದ್ಧ-ವಾ-ಗಿಲ್ಲ
ಪ್ರಭವ
ಪ್ರಭವಂ
ಪ್ರಭ-ವಂ-ತ್ಯ-ಹ-ರಾ-ಗಮೇ
ಪ್ರಭ-ವಂ-ತ್ಯು-ಗ್ರ-ಕ-ರ್ಮಾಣಃ
ಪ್ರಭವಃ
ಪ್ರಭ-ವ-ತ್ಯ-ಹ-ರಾ-ಗಮೇ
ಪ್ರಭ-ವಿಷ್ಣು
ಪ್ರಭಾವ
ಪ್ರಭಾ-ವ-ಕಾ-ರಿ-ಯಾ-ಗಿ-ರು-ವುದು
ಪ್ರಭಾ-ವಕ್ಕೂ
ಪ್ರಭಾ-ವಕ್ಕೆ
ಪ್ರಭಾ-ವನೆ
ಪ್ರಭಾ-ವ-ವನ್ನು
ಪ್ರಭಾ-ವ-ವನ್ನೇ
ಪ್ರಭಾ-ವ-ವಿದೆ
ಪ್ರಭಾ-ವ-ವೆಂ-ಬುದು
ಪ್ರಭಾವೋ
ಪ್ರಭಾ-ಷೇತ
ಪ್ರಭಾಸ್ಮಿ
ಪ್ರಭು
ಪ್ರಭುಃ
ಪ್ರಭು-ರೇವ
ಪ್ರಭು-ವಾ-ಗಿದ್ದ
ಪ್ರಭುವು
ಪ್ರಭುವೂ
ಪ್ರಭೆ
ಪ್ರಭೆಯ
ಪ್ರಭೆ-ಯನ್ನೇ
ಪ್ರಭೆಯು
ಪ್ರಭೆ-ಯು-ಳ್ಳ-ವನೂ
ಪ್ರಭೋ
ಪ್ರಮಾಣ
ಪ್ರಮಾಣಂ
ಪ್ರಮಾ-ಣ-ಗಳ
ಪ್ರಮಾ-ಣ-ಗಳು
ಪ್ರಮಾ-ಣ-ದಲ್ಲಿ
ಪ್ರಮಾ-ಣ-ದ-ಲ್ಲಿದೆ
ಪ್ರಮಾ-ಣ-ದ-ಲ್ಲಿ-ರ-ಬ-ಹುದು
ಪ್ರಮಾ-ಣ-ದ-ಲ್ಲಿ-ರು-ವುದು
ಪ್ರಮಾ-ಣ-ದ-ಲ್ಲಿವೆ
ಪ್ರಮಾ-ಣವೆ
ಪ್ರಮಾ-ಣ-ವೆಂದು
ಪ್ರಮಾ-ಣವೇ
ಪ್ರಮಾಥಿ
ಪ್ರಮಾ-ಥೀನಿ
ಪ್ರಮಾದ
ಪ್ರಮಾ-ದ-ದಲ್ಲಿ
ಪ್ರಮಾ-ದ-ದಿಂದ
ಪ್ರಮಾ-ದ-ಮೋಹೌ
ಪ್ರಮಾ-ದಾತ್
ಪ್ರಮಾ-ದಾ-ಲ-ಸ್ಯ-ನಿ-ದ್ರಾ-ಭಿ-ಸ್ತ-ನ್ನಿ-ಬ-ಧ್ನಾತಿ
ಪ್ರಮಾದೇ
ಪ್ರಮಾದೋ
ಪ್ರಮುಖ
ಪ್ರಮು-ಖರು
ಪ್ರಮು-ಖ-ವಾ-ದ-ವು-ಗಳು
ಪ್ರಮುಖೇ
ಪ್ರಮು-ಚ್ಯತೇ
ಪ್ರಮೇ-ಯವೇ
ಪ್ರಮೋ-ಷನ್
ಪ್ರಯ-ಚ್ಛತಿ
ಪ್ರಯ-ತಾ-ತ್ಮನಃ
ಪ್ರಯತ್ನ
ಪ್ರಯ-ತ್ನ-ಗ-ಳೆಲ್ಲ
ಪ್ರಯ-ತ್ನದ
ಪ್ರಯ-ತ್ನ-ದಲ್ಲಿ
ಪ್ರಯ-ತ್ನ-ದ-ಲ್ಲಿ-ರ-ಬೇಕು
ಪ್ರಯ-ತ್ನ-ದಿಂದ
ಪ್ರಯ-ತ್ನ-ದಿಂ-ದಲೇ
ಪ್ರಯ-ತ್ನ-ಪ-ಟ್ಟರೂ
ಪ್ರಯ-ತ್ನ-ಪ-ಟ್ಟರೆ
ಪ್ರಯ-ತ್ನ-ಪ-ಟ್ಟಾ-ಗಲೇ
ಪ್ರಯ-ತ್ನ-ಪಟ್ಟು
ಪ್ರಯ-ತ್ನ-ಪ-ಡ-ಬೇಕು
ಪ್ರಯ-ತ್ನ-ಪ-ಡು-ತ್ತೇವೆ
ಪ್ರಯ-ತ್ನ-ಪ-ಡು-ವನು
ಪ್ರಯ-ತ್ನ-ಪ-ಡು-ವಾಗ
ಪ್ರಯ-ತ್ನ-ಪೂ-ರ್ವಕ
ಪ್ರಯ-ತ್ನ-ಪೂ-ರ್ವ-ಕ-ವಾಗಿ
ಪ್ರಯ-ತ್ನ-ಪೂ-ರ್ವ-ಕ-ವಾ-ಗಿಯೋ
ಪ್ರಯ-ತ್ನ-ಮಾ-ಡದೆ
ಪ್ರಯ-ತ್ನ-ಮಾಡಿ
ಪ್ರಯ-ತ್ನ-ಮಾ-ಡಿ-ದರೂ
ಪ್ರಯ-ತ್ನ-ಮಾ-ಡಿ-ದ್ದೇವೆ
ಪ್ರಯ-ತ್ನ-ಮಾ-ಡು-ತ್ತಿ-ರು-ವನೋ
ಪ್ರಯ-ತ್ನ-ಮಾ-ಡು-ತ್ತೇವೆ
ಪ್ರಯ-ತ್ನ-ಮಾ-ಡು-ವು-ದಿ-ಲ್ಲವೊ
ಪ್ರಯ-ತ್ನ-ಮಾ-ಡು-ವುದು
ಪ್ರಯ-ತ್ನ-ವನ್ನು
ಪ್ರಯ-ತ್ನ-ವನ್ನೂ
ಪ್ರಯ-ತ್ನ-ವ-ನ್ನೆಲ್ಲ
ಪ್ರಯ-ತ್ನ-ವಿ-ಲ್ಲದೆ
ಪ್ರಯ-ತ್ನವೂ
ಪ್ರಯ-ತ್ನವೆ
ಪ್ರಯ-ತ್ನ-ವೆಲ್ಲ
ಪ್ರಯ-ತ್ನ-ಶೀ-ಲರು
ಪ್ರಯ-ತ್ನ-ಶೀ-ಲರೂ
ಪ್ರಯ-ತ್ನಾ-ದ್ಯ-ತ-ಮಾ-ನಸ್ತು
ಪ್ರಯ-ತ್ನಿಸ
ಪ್ರಯ-ತ್ನಿ-ಸ-ಬ-ಹುದು
ಪ್ರಯ-ತ್ನಿ-ಸ-ಬೇ-ಕಾ-ದರೆ
ಪ್ರಯ-ತ್ನಿ-ಸ-ಬೇಕು
ಪ್ರಯ-ತ್ನಿಸಿ
ಪ್ರಯ-ತ್ನಿ-ಸಿ-ದಂತೆ
ಪ್ರಯ-ತ್ನಿ-ಸಿ-ದರೆ
ಪ್ರಯ-ತ್ನಿ-ಸಿ-ದರೋ
ಪ್ರಯ-ತ್ನಿ-ಸಿ-ದಾಗ
ಪ್ರಯ-ತ್ನಿ-ಸಿ-ದಾ-ಗಲೆ
ಪ್ರಯ-ತ್ನಿ-ಸಿದೆ
ಪ್ರಯ-ತ್ನಿ-ಸಿ-ದ್ದೇನೆ
ಪ್ರಯ-ತ್ನಿಸು
ಪ್ರಯ-ತ್ನಿ-ಸು-ತ್ತಾನೆ
ಪ್ರಯ-ತ್ನಿ-ಸು-ತ್ತಿ-ದ್ದರೆ
ಪ್ರಯ-ತ್ನಿ-ಸು-ತ್ತಿ-ರು-ವನೊ
ಪ್ರಯ-ತ್ನಿ-ಸು-ತ್ತಿಲ್ಲ
ಪ್ರಯ-ತ್ನಿ-ಸು-ತ್ತೇವೆ
ಪ್ರಯ-ತ್ನಿ-ಸು-ತ್ತೇ-ವೆಯೊ
ಪ್ರಯ-ತ್ನಿ-ಸು-ತ್ತೇ-ವೆಯೋ
ಪ್ರಯ-ತ್ನಿ-ಸುವ
ಪ್ರಯ-ತ್ನಿ-ಸು-ವನು
ಪ್ರಯ-ತ್ನಿ-ಸು-ವನೊ
ಪ್ರಯ-ತ್ನಿ-ಸು-ವರೊ
ಪ್ರಯ-ತ್ನಿ-ಸು-ವು-ದಿಲ್ಲ
ಪ್ರಯ-ತ್ನಿ-ಸು-ವುದು
ಪ್ರಯ-ತ್ನಿ-ಸು-ವುದೇ
ಪ್ರಯ-ತ್ನಿ-ಸು-ವು-ದೊಂದು
ಪ್ರಯ-ತ್ನಿ-ಸೋಣ
ಪ್ರಯಾಣ
ಪ್ರಯಾ-ಣ-ಕಾಲೇ
ಪ್ರಯಾ-ಣ-ಕಾ-ಲೇಽಪಿ
ಪ್ರಯಾ-ಣಕ್ಕೆ
ಪ್ರಯಾ-ಣ-ದಲ್ಲಿ
ಪ್ರಯಾ-ಣ-ವನ್ನು
ಪ್ರಯಾ-ಣ-ವ-ನ್ನೆಲ್ಲ
ಪ್ರಯಾ-ಣಿಕ
ಪ್ರಯಾ-ಣಿ-ಕನ
ಪ್ರಯಾ-ಣಿ-ಕನು
ಪ್ರಯಾ-ಣಿ-ಕರು
ಪ್ರಯಾತಾ
ಪ್ರಯಾತಿ
ಪ್ರಯಾ-ಯಾ-ರ್ಹಸಿ
ಪ್ರಯಾ-ಸದ
ಪ್ರಯು-ಕ್ತೋಽಯಂ
ಪ್ರಯೋಗ
ಪ್ರಯೋ-ಗಕ್ಕೆ
ಪ್ರಯೋ-ಗ-ಗಳನ್ನು
ಪ್ರಯೋ-ಗದ
ಪ್ರಯೋ-ಗ-ದಲ್ಲಿ
ಪ್ರಯೋ-ಗ-ಶಾ-ಲೆಗೆ
ಪ್ರಯೋ-ಗ-ಶಾ-ಲೆ-ಯನ್ನು
ಪ್ರಯೋ-ಗ-ಶಾ-ಲೆ-ಯಲ್ಲಿ
ಪ್ರಯೋ-ಗಿ-ಸಿ-ಕೊ-ಳ್ಳು-ವುದನ್ನು
ಪ್ರಯೋ-ಗಿ-ಸು-ವನು
ಪ್ರಯೋ-ಜ-ಕ-ವಾ-ಗಿ-ರು-ವುದು
ಪ್ರಯೋ-ಜನ
ಪ್ರಯೋ-ಜ-ನ-ಕ್ಕಲ್ಲ
ಪ್ರಯೋ-ಜ-ನ-ಕ್ಕಾಗಿ
ಪ್ರಯೋ-ಜ-ನಕ್ಕೂ
ಪ್ರಯೋ-ಜ-ನಕ್ಕೆ
ಪ್ರಯೋ-ಜ-ನದ
ಪ್ರಯೋ-ಜ-ನ-ವನ್ನು
ಪ್ರಯೋ-ಜ-ನ-ವಾ-ಗ-ಬೇ-ಕಾ-ದರೆ
ಪ್ರಯೋ-ಜ-ನ-ವಾ-ಗ-ಬೇಕು
ಪ್ರಯೋ-ಜ-ನ-ವಾಗಿ
ಪ್ರಯೋ-ಜ-ನ-ವಾ-ಗು-ತ್ತಿ-ರ-ಲಿಲ್ಲ
ಪ್ರಯೋ-ಜ-ನ-ವಾ-ಗು-ವು-ದಿಲ್ಲ
ಪ್ರಯೋ-ಜ-ನ-ವಾ-ಗು-ವುದು
ಪ್ರಯೋ-ಜ-ನ-ವಾ-ಯಿತು
ಪ್ರಯೋ-ಜ-ನ-ವಿಲ್ಲ
ಪ್ರಯೋ-ಜ-ನ-ವಿ-ಲ್ಲದ
ಪ್ರಯೋ-ಜ-ನ-ವಿ-ಲ್ಲ-ದು-ದನ್ನು
ಪ್ರಯೋ-ಜ-ನ-ವಿ-ಲ್ಲ-ವೆಂದು
ಪ್ರಯೋ-ಜ-ನವೂ
ಪ್ರಯೋ-ಜ-ನ-ವೆಲ್ಲ
ಪ್ರಯೋ-ಜ-ನ-ವೇನು
ಪ್ರಯೋ-ಜ-ನವೊ
ಪ್ರಯೋ-ಜ-ನಾ-ಪೇ-ಕ್ಷಿ-ಗ-ಳಾದ
ಪ್ರಯೋ-ಜ-ವಿಲ್ಲ
ಪ್ರಲ-ಪನ್
ಪ್ರಲಯಂ
ಪ್ರಲಯಃ
ಪ್ರಲ-ಯ-ಸ್ತಥಾ
ಪ್ರಲ-ಯಾಂ-ತಾ-ಮು-ಪಾ-ಶ್ರಿ-ತಾಃ
ಪ್ರಲಯೇ
ಪ್ರಲೀ-ನ-ಸ್ತ-ಮಸಿ
ಪ್ರಲೀ-ಯಂತೇ
ಪ್ರಲೀ-ಯತೇ
ಪ್ರಲೋ-ಭ-ನಗೆ
ಪ್ರಲೋ-ಭನೆ
ಪ್ರಲೋ-ಭ-ನೆ-ಗ-ಳೊಂ-ದಿಗೆ
ಪ್ರಲೋ-ಭ-ನೆಯ
ಪ್ರಳಯ
ಪ್ರಳ-ಯ-ಕಾಲ
ಪ್ರಳ-ಯ-ಕಾ-ಲ-ದಲ್ಲಿ
ಪ್ರಳ-ಯ-ಗಳಲ್ಲಿ
ಪ್ರಳ-ಯ-ಗ-ಳೆಲ್ಲ
ಪ್ರಳ-ಯ-ಜ-ಲ-ದಲ್ಲಿ
ಪ್ರಳ-ಯದ
ಪ್ರಳ-ಯ-ದಲ್ಲಿ
ಪ್ರಳ-ಯ-ದಲ್ಲೂ
ಪ್ರಳ-ಯ-ದ-ವ-ರೆಗೆ
ಪ್ರಳ-ಯ-ಮಾ-ಡು-ವು-ದಕ್ಕೆ
ಪ್ರಳ-ಯ-ವಾದ
ಪ್ರಳ-ಯ-ವಾ-ದರೂ
ಪ್ರಳ-ಯವೇ
ಪ್ರಳ-ಯಾ-ಗ್ನಿ-ಯಂ-ತಿ-ರುವ
ಪ್ರವ-ಕ್ಷ್ಯಾಮಿ
ಪ್ರವ-ಕ್ಷ್ಯಾ-ಮ್ಯ-ನ-ಸೂ-ಯವೇ
ಪ್ರವಕ್ಷ್ಯೇ
ಪ್ರವ-ಚನ
ಪ್ರವ-ಚ-ನ-ವನ್ನು
ಪ್ರವ-ದಂತಿ
ಪ್ರವ-ದಂ-ತ್ಯ-ವಿ-ಪ-ಶ್ಚಿತಃ
ಪ್ರವ-ದ-ತಾ-ಮ-ಹಮ್
ಪ್ರವ-ರ್ತಂತೇ
ಪ್ರವ-ರ್ತಂ-ತೇ-ಽಶು-ಚಿ-ವ್ರ-ತಾಃ
ಪ್ರವ-ರ್ತತೇ
ಪ್ರವ-ರ್ತಿತಂ
ಪ್ರವ-ರ್ತಿ-ತ-ವಾ-ಗಿ-ರುವ
ಪ್ರವ-ರ್ತಿ-ಸ-ಲ್ಪಟ್ಟ
ಪ್ರವ-ರ್ತಿ-ಸು-ತ್ತಿವೆ
ಪ್ರವ-ರ್ಧ-ಮಾ-ನಕ್ಕೆ
ಪ್ರವ-ಹಿ-ಸಿ-ದಾ-ಗಲೆ
ಪ್ರವ-ಹಿ-ಸಿದೆ
ಪ್ರವಾಹ
ಪ್ರವಾ-ಹಕ್ಕೆ
ಪ್ರವಾ-ಹದ
ಪ್ರವಾ-ಹ-ದಲ್ಲಿ
ಪ್ರವಾ-ಹ-ದಿಂದ
ಪ್ರವಾ-ಹ-ವನ್ನು
ಪ್ರವಾ-ಹ-ವಾ-ಗು-ವುದು
ಪ್ರವಿ-ಭ-ಕ್ತ-ಮ-ನೇ-ಕಧಾ
ಪ್ರವಿ-ಭ-ಕ್ತಾನಿ
ಪ್ರವಿ-ಲೀ-ಯತೇ
ಪ್ರವಿ-ಶಂತಿ
ಪ್ರವೃತ್ತ
ಪ್ರವೃತ್ತಃ
ಪ್ರವೃತ್ತಿ
ಪ್ರವೃ-ತ್ತಿಂ
ಪ್ರವೃ-ತ್ತಿಃ
ಪ್ರವೃ-ತ್ತಿ-ಗಳನ್ನು
ಪ್ರವೃ-ತ್ತಿ-ಗಳು
ಪ್ರವೃ-ತ್ತಿಗೆ
ಪ್ರವೃ-ತ್ತಿ-ಮಾರ್ಗ
ಪ್ರವೃ-ತ್ತಿಮ್
ಪ್ರವೃ-ತ್ತಿಯ
ಪ್ರವೃ-ತ್ತಿ-ಯನ್ನು
ಪ್ರವೃ-ತ್ತಿ-ಯ-ವನು
ಪ್ರವೃ-ತ್ತಿ-ಯ-ವರು
ಪ್ರವೃ-ತ್ತಿ-ರಾ-ರಂಭಃ
ಪ್ರವೃ-ತ್ತಿ-ರ್ಭೂ-ತಾ-ನಾಂ
ಪ್ರವೃತ್ತೇ
ಪ್ರವೃದ್ಧೇ
ಪ್ರವೇಶ
ಪ್ರವೇ-ಶಕ್ಕೆ
ಪ್ರವೇ-ಶ-ಮಾಡಿ
ಪ್ರವೇ-ಶ-ವಾ-ಗು-ವುದು
ಪ್ರವೇ-ಶಿ-ಸದೆ
ಪ್ರವೇ-ಶಿ-ಸ-ಬ-ಲ್ಲರೊ
ಪ್ರವೇ-ಶಿ-ಸ-ಬೇ-ಕಾ-ದರೆ
ಪ್ರವೇ-ಶಿ-ಸಲು
ಪ್ರವೇ-ಶಿಸಿ
ಪ್ರವೇ-ಶಿ-ಸಿದ
ಪ್ರವೇ-ಶಿ-ಸಿ-ರ-ಬೇಕು
ಪ್ರವೇ-ಶಿ-ಸಿ-ರು-ವನು
ಪ್ರವೇ-ಶಿಸು
ಪ್ರವೇ-ಶಿ-ಸು-ತ್ತಾನೆ
ಪ್ರವೇ-ಶಿ-ಸು-ತ್ತಾ-ನೆಯೋ
ಪ್ರವೇ-ಶಿ-ಸು-ತ್ತಿ-ರು-ವನು
ಪ್ರವೇ-ಶಿ-ಸು-ತ್ತಿ-ರು-ವರು
ಪ್ರವೇ-ಶಿ-ಸು-ತ್ತಿ-ರು-ವುದೊ
ಪ್ರವೇ-ಶಿ-ಸು-ವನು
ಪ್ರವೇ-ಶಿ-ಸು-ವರು
ಪ್ರವೇ-ಶಿ-ಸು-ವರೋ
ಪ್ರವೇ-ಶಿ-ಸು-ವಾಗ
ಪ್ರವೇ-ಶಿ-ಸು-ವು-ದಕ್ಕೆ
ಪ್ರವೇ-ಶಿ-ಸು-ವು-ದ-ರ-ಲ್ಲಿದೆ
ಪ್ರವೇ-ಶಿ-ಸು-ವುದು
ಪ್ರವೇ-ಶಿ-ಸು-ವುದೊ
ಪ್ರವೇ-ಶಿ-ಸು-ವುದೋ
ಪ್ರವೇ-ಶಿ-ಸು-ವುವು
ಪ್ರವೇ-ಶಿ-ಸು-ವುವೊ
ಪ್ರವೇ-ಶಿ-ಸು-ವೆನು
ಪ್ರವೇ-ಷ್ಟುಂ
ಪ್ರವ್ಯ-ಥಿತಂ
ಪ್ರವ್ಯ-ಥಿ-ತಾಂ-ತ-ರಾತ್ಮಾ
ಪ್ರವ್ಯ-ಥಿ-ತಾ-ಸ್ತ-ಥಾ-ಹಮ್
ಪ್ರಶಸ್ತ
ಪ್ರಶಸ್ತೇ
ಪ್ರಶಾಂತ
ಪ್ರಶಾಂ-ತ-ಚಿ-ತ್ತ-ದ-ವ-ನಾಗ
ಪ್ರಶಾಂ-ತ-ಚಿ-ತ್ತ-ನಾ-ಗಿ-ರು-ವನು
ಪ್ರಶಾಂ-ತ-ಚಿ-ತ್ತ-ನಾ-ಗಿ-ರು-ವನೋ
ಪ್ರಶಾಂ-ತ-ಚಿ-ತ್ತ-ನಾದ
ಪ್ರಶಾಂ-ತ-ಚಿ-ತ್ತನೂ
ಪ್ರಶಾಂ-ತನೂ
ಪ್ರಶಾಂ-ತ-ಮ-ನಸಂ
ಪ್ರಶಾಂ-ತ-ವಾ-ಗಿದೆ
ಪ್ರಶಾಂ-ತ-ವಾ-ಗಿ-ರು-ವುದು
ಪ್ರಶಾಂ-ತ-ವಾದ
ಪ್ರಶಾಂ-ತ-ವಾ-ದರೆ
ಪ್ರಶಾಂ-ತಸ್ಯ
ಪ್ರಶಾಂ-ತಾ-ತ್ಮ-ನಾ-ಗಿ-ರ-ಬೇಕು
ಪ್ರಶಾಂ-ತಾ-ತ್ಮನೂ
ಪ್ರಶಾಂ-ತಾತ್ಮಾ
ಪ್ರಶಾಂತಿ
ಪ್ರಶಾಂ-ತಿ-ಯಂತೆ
ಪ್ರಶ್ನ
ಪ್ರಶ್ನಿ-ಸ-ಬೇ-ಕಾ-ಗು-ವುದು
ಪ್ರಶ್ನಿ-ಸಿ-ದಾಗ
ಪ್ರಶ್ನಿಸು
ಪ್ರಶ್ನಿ-ಸು-ತ್ತಾನೆ
ಪ್ರಶ್ನಿ-ಸು-ತ್ತಿ-ರು-ವನು
ಪ್ರಶ್ನಿ-ಸು-ತ್ತೇವೆ
ಪ್ರಶ್ನಿ-ಸು-ವುದೇ
ಪ್ರಶ್ನೆ
ಪ್ರಶ್ನೆ-ಗಳ
ಪ್ರಶ್ನೆ-ಗಳನ್ನು
ಪ್ರಶ್ನೆ-ಗಳಿಂದ
ಪ್ರಶ್ನೆ-ಗ-ಳಿಗೆ
ಪ್ರಶ್ನೆ-ಗಳು
ಪ್ರಶ್ನೆಗೆ
ಪ್ರಶ್ನೆ-ಯ-ನ್ನಾ-ದರೂ
ಪ್ರಶ್ನೆ-ಯನ್ನು
ಪ್ರಶ್ನೆ-ಯನ್ನೇ
ಪ್ರಶ್ನೆ-ಯಾ-ಗಿದೆ
ಪ್ರಶ್ನೆ-ಯಿಂದ
ಪ್ರಶ್ನೆಯೇ
ಪ್ರಸಂಗ
ಪ್ರಸಂ-ಗ-ಗಳಲ್ಲಿ
ಪ್ರಸಂ-ಗ-ಗ-ಳಿವೆ
ಪ್ರಸಂ-ಗ-ಗಳು
ಪ್ರಸಂ-ಗ-ಗಳೇ
ಪ್ರಸಂ-ಗದ
ಪ್ರಸಂ-ಗವೇ
ಪ್ರಸಂ-ಗ-ವೇ-ಳು-ವುದು
ಪ್ರಸಂ-ಗೇನ
ಪ್ರಸ-ಕ್ತಾಃ
ಪ್ರಸ-ನ್ನ-ಚಿ-ತ್ತ-ನಾಗಿ
ಪ್ರಸ-ನ್ನ-ಚಿ-ತ್ತ-ವು-ಳ್ಳ-ವ-ನಾಗಿ
ಪ್ರಸ-ನ್ನ-ಚೇ-ತಸೋ
ಪ್ರಸ-ನ್ನತೆ
ಪ್ರಸ-ನ್ನ-ತೆ-ಯನ್ನು
ಪ್ರಸ-ನ್ನ-ನಾಗು
ಪ್ರಸ-ನ್ನ-ನಾ-ಗೆಂದು
ಪ್ರಸ-ನ್ನ-ನಾದ
ಪ್ರಸ-ನ್ನ-ವಾ-ಗಿರ
ಪ್ರಸ-ನ್ನ-ವಾ-ಗಿ-ರು-ವುದು
ಪ್ರಸ-ನ್ನ-ವಾ-ದಾಗ
ಪ್ರಸ-ನ್ನಾ-ತ್ಮ-ನಾ-ದ-ವನು
ಪ್ರಸ-ನ್ನಾತ್ಮಾ
ಪ್ರಸ-ನ್ನೇನ
ಪ್ರಸಭಂ
ಪ್ರಸ-ವಿ-ಷ್ಯ-ಧ್ವ-ಮೇಷ
ಪ್ರಸಾದ
ಪ್ರಸಾ-ದ-ದಂತೆ
ಪ್ರಸಾ-ದ-ದಿಂದ
ಪ್ರಸಾ-ದ-ದಿಂ-ದುಂ-ಟಾದ
ಪ್ರಸಾ-ದ-ಮ-ಧಿ-ಗ-ಚ್ಛತಿ
ಪ್ರಸಾ-ದಯೇ
ಪ್ರಸಾ-ದ-ವನ್ನು
ಪ್ರಸಾ-ದ-ವಾ-ಗು-ತ್ತೇವೆ
ಪ್ರಸಾ-ದ-ವಾ-ಗು-ವುದು
ಪ್ರಸಾ-ದ-ವಾ-ದಾ-ಗಲೇ
ಪ್ರಸಾದೇ
ಪ್ರಸಿದ್ಧ
ಪ್ರಸಿ-ದ್ಧ-ನಾ-ಗಿ-ದ್ದೇನೆ
ಪ್ರಸಿ-ದ್ಧ-ನಾ-ದನು
ಪ್ರಸಿ-ದ್ಧ-ವಾ-ಗಿ-ರುವ
ಪ್ರಸಿ-ಧ್ಯೇ-ದ-ಕ-ರ್ಮಣಃ
ಪ್ರಸೀದ
ಪ್ರಸೃತಾ
ಪ್ರಸೃ-ತಾ-ಸ್ತಸ್ಯ
ಪ್ರಸ್ತಾ-ವನೆ
ಪ್ರಸ್ತಾ-ವ-ನೆ-ಯಲ್ಲಿ
ಪ್ರಸ್ಥಾ-ನ-ತ್ರಯ
ಪ್ರಸ್ಥಾ-ನ-ತ್ರ-ಯ-ಗ-ಳಿಗೆ
ಪ್ರಸ್ಥಾ-ನ-ತ್ರ-ಯದ
ಪ್ರಸ್ಥಾ-ನ-ತ್ರ-ಯ-ದಲ್ಲಿ
ಪ್ರಸ್ಥಾ-ನವೇ
ಪ್ರಹ-ಸ-ನ್ನಿವ
ಪ್ರಹಾ-ಸ್ಯಸಿ
ಪ್ರಹೃ-ಷ್ಯ-ತ್ಯ-ನು-ರ-ಜ್ಯತೇ
ಪ್ರಹೃ-ಷ್ಯೇತ್
ಪ್ರಹ್ಲಾದ
ಪ್ರಹ್ಲಾ-ದನ
ಪ್ರಹ್ಲಾ-ದ-ನಾ-ದರೋ
ಪ್ರಹ್ಲಾ-ದ-ನಿಗೆ
ಪ್ರಹ್ಲಾ-ದ-ಶ್ಚಾಸ್ಮಿ
ಪ್ರಾಂಜ-ಲಯೋ
ಪ್ರಾಕಾ-ರ-ದಿಂ-ದಲೂ
ಪ್ರಾಕೃತಃ
ಪ್ರಾಕ್
ಪ್ರಾಚೀ-ನ-ವಾ-ದು-ವು-ಗಳು
ಪ್ರಾಣ
ಪ್ರಾಣಂ
ಪ್ರಾಣ-ಕ-ರ್ಮ-ಗಳನ್ನು
ಪ್ರಾಣ-ಕ-ರ್ಮಾಣಿ
ಪ್ರಾಣ-ಕ್ರಿ-ಯೆಯೆ
ಪ್ರಾಣ-ಗಳನ್ನು
ಪ್ರಾಣ-ಗಳಲ್ಲಿ
ಪ್ರಾಣ-ತ್ಯಾಗ
ಪ್ರಾಣದ
ಪ್ರಾಣ-ದಲ್ಲಿ
ಪ್ರಾಣ-ಪಕ್ಷಿ
ಪ್ರಾಣ-ಪ್ರ-ತಿಷ್ಠೆ
ಪ್ರಾಣ-ಬಿ-ಡ-ಬೇ-ಕೆಂದು
ಪ್ರಾಣ-ಬಿ-ಡು-ತ್ತೇನೆ
ಪ್ರಾಣ-ಮಾ-ವೇಶ್ಯ
ಪ್ರಾಣ-ಮಾ-ಸ್ಥಿತೋ
ಪ್ರಾಣವ
ಪ್ರಾಣ-ವ-ನ್ನಾ-ದರೂ
ಪ್ರಾಣ-ವನ್ನು
ಪ್ರಾಣ-ವನ್ನೂ
ಪ್ರಾಣ-ವಾ-ಗಿದ್ದ
ಪ್ರಾಣ-ವಾ-ಯು-ವನ್ನು
ಪ್ರಾಣ-ವಾ-ಯು-ವಾ-ಗಿ-ರು-ವನು
ಪ್ರಾಣ-ವಾ-ಯುವು
ಪ್ರಾಣ-ವಿದೆ
ಪ್ರಾಣ-ವಿ-ರು-ವುದು
ಪ್ರಾಣ-ವು-ಳ್ಳ-ವ-ರಾಗಿ
ಪ್ರಾಣ-ಸಂ-ಕಟ
ಪ್ರಾಣ-ಸ-ಖ-ನಾ-ಗಿದ್ದ
ಪ್ರಾಣಾಂ-ಸ್ತ್ಯ-ಕ್ತ್ವಾ-ಧ-ನಾನಿ
ಪ್ರಾಣಾನ್
ಪ್ರಾಣಾ-ಪಾ-ನ-ಗತೀ
ಪ್ರಾಣಾ-ಪಾ-ನ-ಗ-ಳೆ-ರ-ಡನ್ನೂ
ಪ್ರಾಣಾ-ಪಾ-ನ-ವಾ-ಯು-ಗಳ
ಪ್ರಾಣಾ-ಪಾ-ನ-ಸ-ಮಾ-ಯುಕ್ತಃ
ಪ್ರಾಣಾ-ಪಾನೌ
ಪ್ರಾಣಾ-ಯಾಮ
ಪ್ರಾಣಾ-ಯಾ-ಮ-ಗಳನ್ನು
ಪ್ರಾಣಾ-ಯಾ-ಮದ
ಪ್ರಾಣಾ-ಯಾ-ಮ-ಪ-ರಾ-ಯ-ಣಾಃ
ಪ್ರಾಣಾ-ಯಾ-ಮ-ವನ್ನು
ಪ್ರಾಣಿ
ಪ್ರಾಣಿ-ಗಳ
ಪ್ರಾಣಿ-ಗ-ಳಂತೆ
ಪ್ರಾಣಿ-ಗಳನ್ನು
ಪ್ರಾಣಿ-ಗಳನ್ನೂ
ಪ್ರಾಣಿ-ಗ-ಳನ್ನೇ
ಪ್ರಾಣಿ-ಗಳಲ್ಲಿ
ಪ್ರಾಣಿ-ಗ-ಳ-ಲ್ಲಿಯೂ
ಪ್ರಾಣಿ-ಗ-ಳ-ಲ್ಲೆಲ್ಲ
ಪ್ರಾಣಿ-ಗ-ಳಾ-ಗಲಿ
ಪ್ರಾಣಿ-ಗ-ಳಾ-ವು-ದಕ್ಕೂ
ಪ್ರಾಣಿ-ಗ-ಳಿಗೆ
ಪ್ರಾಣಿ-ಗ-ಳಿ-ಗೆಲ್ಲ
ಪ್ರಾಣಿ-ಗ-ಳಿವೆ
ಪ್ರಾಣಿ-ಗಳು
ಪ್ರಾಣಿ-ಗಳೂ
ಪ್ರಾಣಿ-ಗ-ಳೆಲ್ಲ
ಪ್ರಾಣಿ-ಗ-ಳೆ-ಲ್ಲವೂ
ಪ್ರಾಣಿ-ಗಿಂತ
ಪ್ರಾಣಿಗೂ
ಪ್ರಾಣಿ-ದೇ-ಹ-ವನ್ನೊ
ಪ್ರಾಣಿ-ನಾಂ
ಪ್ರಾಣಿಯ
ಪ್ರಾಣಿ-ಯಂತೆ
ಪ್ರಾಣಿ-ಯನ್ನು
ಪ್ರಾಣಿ-ಯನ್ನೂ
ಪ್ರಾಣಿ-ಯಾ-ಗಲಿ
ಪ್ರಾಣಿ-ಯಾ-ಗಿ-ರ-ಬೇಕು
ಪ್ರಾಣಿ-ಯಾ-ದರೂ
ಪ್ರಾಣಿ-ಯಿಂದ
ಪ್ರಾಣಿಯು
ಪ್ರಾಣಿಯೂ
ಪ್ರಾಣಿ-ರಾ-ಶಿ-ಗ-ಳು-ಅ-ವರು
ಪ್ರಾಣಿ-ರೂ-ಪ-ದ-ಲ್ಲಿ-ರಲಿ
ಪ್ರಾಣಿ-ವರ್ಗ
ಪ್ರಾಣಿ-ವ-ರ್ಗ-ಕ್ಕಾಗಿ
ಪ್ರಾಣಿ-ವ-ರ್ಗ-ಗಳನ್ನೆಲ್ಲಾ
ಪ್ರಾಣಿ-ವ-ರ್ಗ-ಗಳು
ಪ್ರಾಣಿ-ಸ-ಮಾ-ನ-ವಾದ
ಪ್ರಾಣೇಶು
ಪ್ರಾಣೇ-ಽಪಾನಂ
ಪ್ರಾಧ್ಯಾ
ಪ್ರಾಧ್ಯಾ-ನ್ಯತಃ
ಪ್ರಾಪಂ-ಚಿಕ
ಪ್ರಾಪಂ-ಚಿ-ಕ-ತೆ-ಯನ್ನೇ
ಪ್ರಾಪಂ-ಚಿ-ಕರು
ಪ್ರಾಪಂ-ಚಿ-ಕ-ವಾದ
ಪ್ರಾಪ್ತ
ಪ್ರಾಪ್ತ-ವಾ-ಗ-ಬ-ಹುದು
ಪ್ರಾಪ್ತ-ವಾ-ಗ-ಬೇ-ಕಾ-ದರೆ
ಪ್ರಾಪ್ತ-ವಾ-ಗಿದೆ
ಪ್ರಾಪ್ತ-ವಾ-ಗಿ-ದೆಯೇ
ಪ್ರಾಪ್ತ-ವಾಗು
ಪ್ರಾಪ್ತ-ವಾ-ಗು-ತ್ತದೆ
ಪ್ರಾಪ್ತ-ವಾ-ಗು-ವು-ದಕ್ಕೆ
ಪ್ರಾಪ್ತ-ವಾ-ಗು-ವು-ದಿಲ್ಲ
ಪ್ರಾಪ್ತ-ವಾ-ಗು-ವುದು
ಪ್ರಾಪ್ತ-ವಾ-ಗು-ವುದೊ
ಪ್ರಾಪ್ತ-ವಾ-ಗು-ವುವು
ಪ್ರಾಪ್ತ-ವಾದ
ಪ್ರಾಪ್ತ-ವಾ-ದರೂ
ಪ್ರಾಪ್ತ-ವಾ-ದಾಗ
ಪ್ರಾಪ್ತ-ವಾ-ದೀತು
ಪ್ರಾಪ್ತ-ವಾ-ಯಿತು
ಪ್ರಾಪ್ತಿ
ಪ್ರಾಪ್ತೋ
ಪ್ರಾಪ್ನು-ಯಾತ್
ಪ್ರಾಪ್ನು-ವಂತಿ
ಪ್ರಾಪ್ಯ
ಪ್ರಾಪ್ಯತೇ
ಪ್ರಾಪ್ಯಸಿ
ಪ್ರಾಪ್ಸ್ಯಸಿ
ಪ್ರಾಪ್ಸ್ಯೇ
ಪ್ರಾಮು-ಖ್ಯತೆ
ಪ್ರಾಮು-ಖ್ಯ-ತೆ-ಯನ್ನು
ಪ್ರಾರಂಭ
ಪ್ರಾರಂ-ಭದ
ಪ್ರಾರಂ-ಭ-ದಲ್ಲಿ
ಪ್ರಾರಂ-ಭ-ದ-ಲ್ಲಿಯೇ
ಪ್ರಾರಂ-ಭ-ಮಾ-ಡಿ-ದರೆ
ಪ್ರಾರಂ-ಭ-ವಾ-ಗ-ಬೇ-ಕಾ-ದರೆ
ಪ್ರಾರಂ-ಭ-ವಾ-ಗಲಿ
ಪ್ರಾರಂ-ಭ-ವಾ-ಗ-ವುದು
ಪ್ರಾರಂ-ಭ-ವಾಗಿ
ಪ್ರಾರಂ-ಭ-ವಾ-ಗಿ-ದೆ-ಯಲ್ಲ
ಪ್ರಾರಂ-ಭ-ವಾ-ಗಿದ್ದು
ಪ್ರಾರಂ-ಭ-ವಾಗು
ಪ್ರಾರಂ-ಭ-ವಾ-ಗು-ತ್ತದೆ
ಪ್ರಾರಂ-ಭ-ವಾ-ಗು-ವಾಗ
ಪ್ರಾರಂ-ಭ-ವಾ-ಗು-ವು-ದ-ಕ್ಕಾ-ಗಿಯೇ
ಪ್ರಾರಂ-ಭ-ವಾ-ಗು-ವು-ದಕ್ಕೆ
ಪ್ರಾರಂ-ಭ-ವಾ-ಗು-ವು-ದಿಲ್ಲ
ಪ್ರಾರಂ-ಭ-ವಾ-ಗು-ವುದು
ಪ್ರಾರಂ-ಭ-ವಾ-ಗು-ವುದೋ
ಪ್ರಾರಂ-ಭ-ವಾದ
ಪ್ರಾರಂ-ಭ-ವಾ-ದದ್ದು
ಪ್ರಾರಂ-ಭ-ವಾ-ದದ್ದೇ
ಪ್ರಾರಂ-ಭ-ವಾ-ದರೆ
ಪ್ರಾರಂ-ಭ-ವಾ-ದಾಗ
ಪ್ರಾರಂ-ಭ-ವಾ-ದಾ-ಗಲೇ
ಪ್ರಾರಂ-ಭ-ವಾ-ಯಿತು
ಪ್ರಾರಂ-ಭಿ-ಸ-ಬೇಕು
ಪ್ರಾರಂ-ಭಿ-ಸಲು
ಪ್ರಾರಂ-ಭಿ-ಸಿದ
ಪ್ರಾರಂ-ಭಿ-ಸಿ-ದರೂ
ಪ್ರಾರಂ-ಭಿ-ಸಿ-ದರೆ
ಪ್ರಾರಂ-ಭಿ-ಸಿ-ದೆನೋ
ಪ್ರಾರಂ-ಭಿ-ಸಿ-ದೊ-ಡನೆ
ಪ್ರಾರಂ-ಭಿ-ಸಿ-ದ್ದಾನೆ
ಪ್ರಾರಂ-ಭಿ-ಸಿ-ದ್ದೇವೆ
ಪ್ರಾರಂ-ಭಿ-ಸು-ತ್ತಾನೆ
ಪ್ರಾರಂ-ಭಿ-ಸು-ತ್ತಾರೆ
ಪ್ರಾರಂ-ಭಿ-ಸು-ತ್ತಿ-ದ್ದೇವೆ
ಪ್ರಾರಂ-ಭಿ-ಸು-ವನು
ಪ್ರಾರಂ-ಭಿ-ಸು-ವನೋ
ಪ್ರಾರಂ-ಭಿ-ಸು-ವರೊ
ಪ್ರಾರಂ-ಭಿ-ಸು-ವು-ದಿಲ್ಲ
ಪ್ರಾರಂ-ಭಿ-ಸು-ವುದು
ಪ್ರಾರಂ-ಭಿ-ಸು-ವೆವು
ಪ್ರಾರ-ಭತೇ
ಪ್ರಾರ್ಥ-ನಾ-ಲ-ಯಕ್ಕೆ
ಪ್ರಾರ್ಥ-ನಾ-ಸ-ಭೆಗೆ
ಪ್ರಾರ್ಥನೆ
ಪ್ರಾರ್ಥ-ನೆಯ
ಪ್ರಾರ್ಥ-ನೆ-ಯಿಂದ
ಪ್ರಾರ್ಥ-ಯಂತೇ
ಪ್ರಾರ್ಥಿಸ
ಪ್ರಾರ್ಥಿಸಿ
ಪ್ರಾರ್ಥಿ-ಸಿ-ಕೊಂ-ಡಾಗ
ಪ್ರಾರ್ಥಿ-ಸಿ-ದರೂ
ಪ್ರಾರ್ಥಿ-ಸಿ-ದರೆ
ಪ್ರಾರ್ಥಿಸು
ಪ್ರಾರ್ಥಿ-ಸು-ತ್ತಾನೆ
ಪ್ರಾರ್ಥಿ-ಸು-ತ್ತಾರೆ
ಪ್ರಾರ್ಥಿ-ಸು-ತ್ತೇ-ವೆಯೋ
ಪ್ರಾರ್ಥಿ-ಸು-ವನು
ಪ್ರಾರ್ಥಿ-ಸು-ವರು
ಪ್ರಾರ್ಥಿ-ಸು-ವಾಗ
ಪ್ರಾರ್ಥಿ-ಸು-ವುದು
ಪ್ರಾರ್ಥೀ-ಸು-ತ್ತೇನೆ
ಪ್ರಾಸದ
ಪ್ರಾಹ
ಪ್ರಾಹುಃ
ಪ್ರಾಹು-ರ-ವ್ಯ-ಯಮ್
ಪ್ರಾಹು-ರ್ಮ-ನೀ-ಷಿಣಃ
ಪ್ರಾಹು-ರ್ಯೋಗಂ
ಪ್ರಾಹು-ಸ್ತ್ಯಾಗಂ
ಪ್ರಿಂಟು
ಪ್ರಿಯ
ಪ್ರಿಯಂ
ಪ್ರಿಯಃ
ಪ್ರಿಯ-ಕ-ರ-ನಾ-ದ-ವನು
ಪ್ರಿಯ-ಕೃ-ತ್ತಮಃ
ಪ್ರಿಯ-ಚಿ-ಕೀ-ರ್ಷವಃ
ಪ್ರಿಯ-ತ-ಮ-ನಾದ
ಪ್ರಿಯ-ತ-ಮೆ-ಯನ್ನು
ಪ್ರಿಯ-ತರೋ
ಪ್ರಿಯ-ನಾ-ಗಿ-ರು-ವು-ದ-ರಿಂದ
ಪ್ರಿಯ-ನಾ-ಗಿ-ರುವೆ
ಪ್ರಿಯ-ನಾದ
ಪ್ರಿಯ-ನಾ-ದ-ವನು
ಪ್ರಿಯ-ನಾ-ದು-ದ-ರಿಂದ
ಪ್ರಿಯನು
ಪ್ರಿಯ-ಮಿತ್ರ
ಪ್ರಿಯ-ರಾ-ಗಿ-ರು-ವರೊ
ಪ್ರಿಯ-ರಾ-ದ-ವರೆ
ಪ್ರಿಯರು
ಪ್ರಿಯರೂ
ಪ್ರಿಯರೊ
ಪ್ರಿಯಳ
ಪ್ರಿಯ-ಳನ್ನು
ಪ್ರಿಯ-ಳಾದ
ಪ್ರಿಯ-ವನ್ನು
ಪ್ರಿಯ-ವ-ನ್ನುಂ-ಟು-ಮಾ-ಡು-ವ-ವರು
ಪ್ರಿಯ-ವನ್ನೂ
ಪ್ರಿಯ-ವಾಗಿ
ಪ್ರಿಯ-ವಾ-ಗಿ-ರ-ಬ-ಹುದು
ಪ್ರಿಯ-ವಾ-ಗಿ-ರ-ಬೇಕು
ಪ್ರಿಯ-ವಾ-ಗಿ-ರಲಿ
ಪ್ರಿಯ-ವಾ-ಗಿರು
ಪ್ರಿಯ-ವಾ-ಗಿ-ರು-ತ್ತದೆ
ಪ್ರಿಯ-ವಾ-ಗಿ-ರು-ತ್ತವೆ
ಪ್ರಿಯ-ವಾ-ಗಿ-ರುವ
ಪ್ರಿಯ-ವಾ-ಗಿ-ರು-ವುದನ್ನು
ಪ್ರಿಯ-ವಾ-ಗಿ-ರು-ವು-ದ-ನ್ನೆಲ್ಲಾ
ಪ್ರಿಯ-ವಾ-ಗಿ-ರು-ವು-ದನ್ನೇ
ಪ್ರಿಯ-ವಾ-ಗಿ-ರು-ವುದು
ಪ್ರಿಯ-ವಾ-ಗಿ-ರು-ವುದೂ
ಪ್ರಿಯ-ವಾ-ಗಿ-ರು-ವುದೇ
ಪ್ರಿಯ-ವಾ-ಗಿ-ರು-ವುದೊ
ಪ್ರಿಯ-ವಾ-ಗಿ-ರು-ವುವು
ಪ್ರಿಯ-ವಾ-ಗು-ವುದೊ
ಪ್ರಿಯ-ವಾದ
ಪ್ರಿಯ-ವಾ-ದದು
ಪ್ರಿಯ-ವಾ-ದ-ವ-ರನ್ನು
ಪ್ರಿಯ-ವಾ-ದ-ವು-ಗಳು
ಪ್ರಿಯ-ವಾ-ದು-ದನ್ನು
ಪ್ರಿಯ-ವಾ-ದುದು
ಪ್ರಿಯ-ವಾ-ದು-ವು-ಗಳನ್ನು
ಪ್ರಿಯವೂ
ಪ್ರಿಯ-ಹಿತಂ
ಪ್ರಿಯಾಃ
ಪ್ರಿಯೋ
ಪ್ರಿಯೋಽಸಿ
ಪ್ರೀತ-ಮ-ನಾಃ
ಪ್ರೀತಿ
ಪ್ರೀತಿಃ
ಪ್ರೀತಿ-ಗಳು
ಪ್ರೀತಿ-ಗಾಗಿ
ಪ್ರೀತಿ-ಗಿಂತ
ಪ್ರೀತಿಗೆ
ಪ್ರೀತಿ-ಪಾ-ತ್ರನು
ಪ್ರೀತಿ-ಪೂ-ರ್ವ-ಕಮ್
ಪ್ರೀತಿ-ಪೂ-ರ್ವ-ಕ-ವಾಗಿ
ಪ್ರೀತಿಯ
ಪ್ರೀತಿ-ಯಂತೆ
ಪ್ರೀತಿ-ಯನ್ನು
ಪ್ರೀತಿ-ಯಲ್ಲ
ಪ್ರೀತಿ-ಯಲ್ಲಿ
ಪ್ರೀತಿ-ಯಷ್ಟು
ಪ್ರೀತಿ-ಯಿಂದ
ಪ್ರೀತಿ-ಯಿಂ-ದಲ್ಲ
ಪ್ರೀತಿ-ಯಿ-ಲ್ಲದ
ಪ್ರೀತಿ-ಯುಳ್ಳ
ಪ್ರೀತಿಯೂ
ಪ್ರೀತಿಯೆ
ಪ್ರೀತಿ-ಯೆಲ್ಲ
ಪ್ರೀತಿಯೇ
ಪ್ರೀತಿ-ಯೊಂ-ದಿ-ದ್ದರೆ
ಪ್ರೀತಿ-ಯೊಂದೇ
ಪ್ರೀತಿ-ಶಿ-ಖೆ-ಯಲ್ಲಿ
ಪ್ರೀತಿಸ
ಪ್ರೀತಿ-ಸದೇ
ಪ್ರೀತಿ-ಸ-ಬಲ್ಲ
ಪ್ರೀತಿ-ಸ-ಬ-ಹುದು
ಪ್ರೀತಿ-ಸ-ಬಾ-ರದು
ಪ್ರೀತಿ-ಸ-ಬೇ-ಕಾ-ಗಿದೆ
ಪ್ರೀತಿ-ಸ-ಬೇ-ಕಾ-ಗು-ವುದು
ಪ್ರೀತಿ-ಸ-ಬೇಕು
ಪ್ರೀತಿ-ಸ-ಲಾ-ರನು
ಪ್ರೀತಿ-ಸಲಿ
ಪ್ರೀತಿ-ಸಲು
ಪ್ರೀತಿ-ಸ-ಲೆ-ತ್ನಿ-ಸು-ವನು
ಪ್ರೀತಿಸಿ
ಪ್ರೀತಿ-ಸಿ-ದಂತೆ
ಪ್ರೀತಿ-ಸಿ-ದರೂ
ಪ್ರೀತಿ-ಸಿ-ದರೆ
ಪ್ರೀತಿ-ಸಿ-ದ-ವರು
ಪ್ರೀತಿ-ಸಿ-ದ್ದರು
ಪ್ರೀತಿ-ಸಿಯೇ
ಪ್ರೀತಿ-ಸಿ-ರು-ವನು
ಪ್ರೀತಿಸು
ಪ್ರೀತಿ-ಸುತ್ತಾ
ಪ್ರೀತಿ-ಸು-ತ್ತಾನೆ
ಪ್ರೀತಿ-ಸು-ತ್ತಾನೊ
ಪ್ರೀತಿ-ಸು-ತ್ತಾರೆ
ಪ್ರೀತಿ-ಸು-ತ್ತಾರೊ
ಪ್ರೀತಿ-ಸು-ತ್ತಿ-ದ್ದಂತೆ
ಪ್ರೀತಿ-ಸು-ತ್ತಿ-ದ್ದರು
ಪ್ರೀತಿ-ಸು-ತ್ತಿ-ದ್ದರೆ
ಪ್ರೀತಿ-ಸು-ತ್ತಿ-ದ್ದ-ವನು
ಪ್ರೀತಿ-ಸು-ತ್ತಿ-ದ್ದಷ್ಟು
ಪ್ರೀತಿ-ಸು-ತ್ತಿ-ದ್ದಾರೆ
ಪ್ರೀತಿ-ಸು-ತ್ತಿ-ದ್ದೇವೆ
ಪ್ರೀತಿ-ಸು-ತ್ತಿ-ರಲಿ
ಪ್ರೀತಿ-ಸು-ತ್ತಿ-ರ-ಲಿಲ್ಲ
ಪ್ರೀತಿ-ಸು-ತ್ತೇವೆ
ಪ್ರೀತಿ-ಸು-ತ್ತೇ-ವೆಯೊ
ಪ್ರೀತಿ-ಸುನ
ಪ್ರೀತಿ-ಸುವ
ಪ್ರೀತಿ-ಸು-ವಂತೆ
ಪ್ರೀತಿ-ಸು-ವನು
ಪ್ರೀತಿ-ಸು-ವನೋ
ಪ್ರೀತಿ-ಸು-ವರೊ
ಪ್ರೀತಿ-ಸು-ವರೋ
ಪ್ರೀತಿ-ಸು-ವ-ವ-ನಾ-ದರೊ
ಪ್ರೀತಿ-ಸು-ವ-ವನು
ಪ್ರೀತಿ-ಸು-ವ-ವನೇ
ಪ್ರೀತಿ-ಸು-ವ-ವರು
ಪ್ರೀತಿ-ಸು-ವಷ್ಟು
ಪ್ರೀತಿ-ಸು-ವಾಗ
ಪ್ರೀತಿ-ಸು-ವು-ದ-ಕ್ಕಲ್ಲ
ಪ್ರೀತಿ-ಸು-ವು-ದ-ಕ್ಕಿಂತ
ಪ್ರೀತಿ-ಸು-ವು-ದಕ್ಕೆ
ಪ್ರೀತಿ-ಸು-ವುದನ್ನು
ಪ್ರೀತಿ-ಸು-ವು-ದಿಲ್ಲ
ಪ್ರೀತಿ-ಸು-ವುದು
ಪ್ರೀತಿ-ಸು-ವುದೂ
ಪ್ರೀತಿ-ಸು-ವುದೊ
ಪ್ರೀತ್ಯರ್ಥಂ
ಪ್ರೀತ್ಯ-ರ್ಥ-ವಾಗಿ
ಪ್ರೀಯ-ಮಾ-ಣಾಯ
ಪ್ರೇಕ್ಷ-ಕ-ನಂತೆ
ಪ್ರೇಕ್ಷ-ಕ-ನಾ-ಗಿ-ರು-ತ್ತಾನೆ
ಪ್ರೇಕ್ಷ-ಕನು
ಪ್ರೇಕ್ಷ-ಕ-ರಿಗೆ
ಪ್ರೇಕ್ಷ-ಕರು
ಪ್ರೇತ
ಪ್ರೇತ-ಭೂ-ತ-ಗಳು
ಪ್ರೇತ-ರಾ-ಶಿ-ಗಳು
ಪ್ರೇತಾನ್
ಪ್ರೇತ್ಯ
ಪ್ರೇಮ
ಪ್ರೇಮ-ಕ್ಕಾಗಿ
ಪ್ರೇಮ-ಚಿ-ಲುಮೆ
ಪ್ರೇಮ-ದಿಂದ
ಪ್ರೇಮ-ಮಯ
ಪ್ರೇಮ-ಮ-ಯ-ನಾ-ಗಿ-ರುವ
ಪ್ರೇಮ-ಮೂರ್ತಿ
ಪ್ರೇಮ-ಮೂ-ರ್ತಿ-ಯನ್ನು
ಪ್ರೇಮ-ಮೂ-ರ್ತಿಯೇ
ಪ್ರೇಮ-ವನ್ನು
ಪ್ರೇಮ-ವೆಲ್ಲ
ಪ್ರೇಮವೇ
ಪ್ರೇಮ-ಶಿ-ಖೆ-ಯೆ-ಡೆಗೆ
ಪ್ರೇಮ-ಸ್ಮಾ-ರ-ಕ-ವೆ-ನ್ನು-ತ್ತೇವೆ
ಪ್ರೇಯಸ್
ಪ್ರೇಯ-ಸ್ಸನ್ನು
ಪ್ರೇಯ-ಸ್ಸಿನ
ಪ್ರೇಯಸ್ಸು
ಪ್ರೇರ-ಣೆ-ಯಲ್ಲಿ
ಪ್ರೇರಿತ
ಪ್ರೇರಿ-ತ-ನಾಗಿ
ಪ್ರೇರಿ-ತ-ರಾಗಿ
ಪ್ರೇರಿ-ತ-ರಾ-ಗಿ-ರು-ವರು
ಪ್ರೇರಿ-ತ-ವಾಗಿ
ಪ್ರೇರಿ-ತ-ವಾ-ಗಿ-ರುವ
ಪ್ರೇರಿ-ತ-ವಾದ
ಪ್ರೇರಿ-ತ-ವಾ-ದುವು
ಪ್ರೇರೇ-ಪಿ-ತ-ನಾಗಿ
ಪ್ರೇರೇ-ಪಿ-ತ-ರಾಗಿ
ಪ್ರೇರೇ-ಪಿ-ತ-ವಾದ
ಪ್ರೇರೇ-ಪಿ-ಸಿ-ದಂತೆ
ಪ್ರೇರೇ-ಪಿ-ಸುತ್ತ
ಪ್ರೇರೇ-ಪಿ-ಸು-ತ್ತಾನೆ
ಪ್ರೇರೇ-ಪಿ-ಸು-ತ್ತಿ-ರು-ವುದು
ಪ್ರೇರೇ-ಪಿ-ಸು-ವಂತೆ
ಪ್ರೇರೇ-ಪಿ-ಸು-ವನು
ಪ್ರೇರೇ-ಪಿ-ಸು-ವುದು
ಪ್ರೇರೇ-ಪಿ-ಸು-ವುದೇ
ಪ್ರೈಮರಿ
ಪ್ರೊಜೆ-ಕ್ಟ-ರಿನ
ಪ್ರೊಜೆ-ಕ್ಟರ್
ಪ್ರೋಕ್ತ
ಪ್ರೋಕ್ತಂ
ಪ್ರೋಕ್ತಃ
ಪ್ರೋಕ್ತ-ಮ-ಜ್ಞಾನಂ
ಪ್ರೋಕ್ತ-ಮ-ಧಿ-ದೈವಂ
ಪ್ರೋಕ್ತ-ಮಾ-ತ್ಮ-ಬು-ದ್ಧಿ-ಪ್ರ-ಸಾ-ದ-ಜಮ್
ಪ್ರೋಕ್ತ-ವಾ-ನ-ಹ-ಮ-ವ್ಯ-ಯಮ್
ಪ್ರೋಕ್ತ-ವಾ-ನಿತಿ
ಪ್ರೋಕ್ತಾ
ಪ್ರೋಕ್ತಾನಿ
ಪ್ರೋಕ್ತೋ
ಪ್ರೋಕ್ಷ-ಣೆ-ಮಾಡಿ
ಪ್ರೋಚ್ಯತೇ
ಪ್ರೋಚ್ಯ-ಮಾ-ನ-ಮ-ಶೇ-ಷೇಣ
ಪ್ರೋತಂ
ಪ್ರೋತ-ವಾ-ಗಿ-ರು-ವುದನ್ನು
ಪ್ರೋತ್ಸಾಹ
ಪ್ರೋತ್ಸಾ-ಹ-ವನ್ನು
ಪ್ರೋತ್ಸಾ-ಹ-ವಿದೆ
ಪ್ರೋತ್ಸಾ-ಹವೂ
ಪ್ರೋತ್ಸಾ-ಹಿ-ಸು-ವು-ದಕ್ಕೂ
ಪ್ರೋತ್ಸಾ-ಹಿ-ಸು-ವು-ದಕ್ಕೆ
ಪ್ಲೇಗು
ಫಕೀರ
ಫರ್ಷಣೆ
ಫಲ
ಫಲಂ
ಫಲ-ಕಾಂ-ಕ್ಷಿ-ಗ-ಳಾದ
ಫಲ-ಕಾರಿ
ಫಲ-ಕಾ-ರಿ-ಯಾ-ಗು-ವು-ದಿಲ್ಲ
ಫಲ-ಕೊ-ಡ-ಬೇ-ಕಾ-ದರೆ
ಫಲ-ಕ್ಕಾ-ಗಲಿ
ಫಲ-ಕ್ಕಾಗಿ
ಫಲ-ಕ್ಕಾ-ಗಿಯೂ
ಫಲ-ಕ್ಕಾ-ಗಿಯೇ
ಫಲಕ್ಕೂ
ಫಲಕ್ಕೆ
ಫಲ-ಕ್ಕೆಲ್ಲ
ಫಲ-ಗಳ
ಫಲ-ಗಳನ್ನು
ಫಲ-ಗಳನ್ನೆಲ್ಲ
ಫಲ-ಗಳನ್ನೆಲ್ಲಾ
ಫಲ-ಗ-ಳಿ-ಗಲ್ಲ
ಫಲ-ಗ-ಳಿಗೂ
ಫಲ-ಗ-ಳಿಗೆ
ಫಲ-ಗಳು
ಫಲ-ಗ-ಳುಳ್ಳ
ಫಲ-ಗ-ಳೆ-ಲ್ಲವೂ
ಫಲ-ತ್ಯಾ-ಗ-ರೂ-ಪದ
ಫಲ-ತ್ಯಾ-ಗ-ವನ್ನು
ಫಲದ
ಫಲ-ದಲ್ಲಿ
ಫಲ-ಪು-ಷ್ಪ-ಗಳನ್ನು
ಫಲ-ಪು-ಷ್ಪ-ಗಳು
ಫಲ-ಪ್ರ-ದ-ವಾ-ಗಿ-ರುವ
ಫಲ-ಮು-ದ್ದಿಶ್ಯ
ಫಲಮ್
ಫಲ-ರೂ-ಪ-ವಾಗಿ
ಫಲ-ವಂತೂ
ಫಲ-ವನ್ನು
ಫಲ-ವ-ನ್ನು-ಣ್ಣಲು
ಫಲ-ವನ್ನೂ
ಫಲ-ವ-ನ್ನೆಲ್ಲ
ಫಲ-ವ-ನ್ನೆಲ್ಲಾ
ಫಲ-ವಾ-ಗಲಿ
ಫಲ-ವಾಗಿ
ಫಲ-ವಾ-ಗು-ವು-ದಿಲ್ಲ
ಫಲವೂ
ಫಲ-ವೆಂಬ
ಫಲ-ವೆಲ್ಲಿ
ಫಲವೇ
ಫಲ-ವೇನು
ಫಲ-ಸಂ-ಗ-ರ-ಹಿ-ತನೂ
ಫಲ-ಸ್ವ-ರೂ-ಪ-ನಾ-ಗಿ-ದ್ದಾನೆ
ಫಲ-ಹೇ-ತವಃ
ಫಲಾ-ಕಾಂಕ್ಷಿ
ಫಲಾ-ಕಾಂ-ಕ್ಷಿ-ಗ-ಳಾ-ದ-ವ-ರಿಗೆ
ಫಲಾ-ಕಾಂ-ಕ್ಷಿ-ತ-ನಲ್ಲಿ
ಫಲಾ-ಕಾಂ-ಕ್ಷಿಯು
ಫಲಾ-ಕಾಂಕ್ಷೀ
ಫಲಾ-ಕಾಂಕ್ಷೆ
ಫಲಾ-ಕಾಂ-ಕ್ಷೆ-ಯು-ಳ್ಳ-ವನು
ಫಲಾ-ಕಾಂ-ಕ್ಷೆಯೂ
ಫಲಾನಿ
ಫಲಾ-ಪೇ-ಕ್ಷಿ-ಗಳ
ಫಲಾ-ಪೇಕ್ಷೆ
ಫಲಾ-ಪೇ-ಕ್ಷೆಗೆ
ಫಲಾ-ಪೇ-ಕ್ಷೆಯ
ಫಲಾ-ಪೇ-ಕ್ಷೆ-ಯನ್ನು
ಫಲಾ-ಪೇ-ಕ್ಷೆ-ಯ-ನ್ನೆಲ್ಲ
ಫಲಾ-ಪೇ-ಕ್ಷೆ-ಯಿಂದ
ಫಲಾ-ಪೇ-ಕ್ಷೆ-ಯಿಂ-ದಲೇ
ಫಲಾ-ಪೇ-ಕ್ಷೆ-ಯಿ-ಲ್ಲದೆ
ಫಲಾ-ಪೇ-ಕ್ಷೆಯೇ
ಫಲಾ-ಫ-ಲ-ಗಳಿಂದ
ಫಲಾ-ಫ-ಲ-ಗ-ಳಿಗೆ
ಫಲಾ-ಫ-ಲ-ಗಳು
ಫಲಾ-ಫ-ಲದ
ಫಲಾ-ಸ-ಕ್ತ-ನಾ-ಗ-ದಿ-ರಲು
ಫಲಾ-ಸ-ಕ್ತನೂ
ಫಲಾ-ಸಕ್ತಿ
ಫಲಾ-ಸ-ಕ್ತಿ-ಯಿಂದ
ಫಲಾ-ಸ-ಕ್ತಿ-ಯಿ-ಲ್ಲದೆ
ಫಲಾ-ಸ-ಕ್ತಿಯೇ
ಫಲಾ-ಹಾರ
ಫಲಿ-ಸಿ-ದಾಗ
ಫಲಿ-ಸು-ವು-ದ-ರಲ್ಲಿ
ಫಲೇ
ಫಲೇಷು
ಫಳಾರ್
ಫಾಸಿ
ಫಿರಂ-ಗಿಯ
ಫಿಲಂ
ಫಿಲಂನ
ಫಿಲಂ-ನಿಂದ
ಫಿಸಿ-ಕ್ಸ್
ಫೀಜು-ಕೊಟ್ಟು
ಫುಟ್ಬಾಲ್
ಫೇಲಾ-ಗಿ-ರು-ವನೊ
ಫೇಲಾದ
ಫೇಲಾ-ದ-ವನು
ಫೇಲ್
ಫೋಕಲ್
ಫ್ಯಾಕ್ಟರಿ
ಫ್ಯಾಕ್ಟ-ರಿ-ಗಳಲ್ಲಿ
ಫ್ಯಾಕ್ಟ-ರಿಯ
ಫ್ಯಾಕ್ಟ-ರಿ-ಯನ್ನು
ಫ್ಯಾಕ್ಟ-ರಿ-ಯಲ್ಲಿ
ಫ್ಯಾಕ್ಟ-ರಿಯೂ
ಫ್ಯಾಕ್ಟ-ರಿಯೇ
ಫ್ಯಾಶನ್
ಬಂಗಲೆ
ಬಂಗಾ-ರದ
ಬಂಡ-ವಾಳ
ಬಂಡ-ವಾ-ಳ-ವಿ-ಲ್ಲದ
ಬಂಡಿ-ಯಲ್ಲಿ
ಬಂಡೆ
ಬಂಡೆ-ಗಳನ್ನು
ಬಂಡೆಗೆ
ಬಂಡೆಯ
ಬಂಡೆ-ಯಂ-ತಿ-ರು-ವರು
ಬಂಡೆ-ಯಂತೆ
ಬಂಡೆ-ಯ-ಮೇಲೆ
ಬಂಡೆ-ಯಿಂದ
ಬಂಡೆಯೇ
ಬಂತಲ್ಲ
ಬಂತು
ಬಂತೆ
ಬಂತೇನು
ಬಂತೊ
ಬಂತೋ
ಬಂದ
ಬಂದಂತೆ
ಬಂದ-ದ್ದ-ನ್ನೆಲ್ಲಾ
ಬಂದ-ದ್ದ-ರಿಂದ
ಬಂದದ್ದು
ಬಂದ-ದ್ದೇನು
ಬಂದನು
ಬಂದ-ನೆಂ-ದರೆ
ಬಂದನೊ
ಬಂದನೋ
ಬಂದ-ಮೇಲೂ
ಬಂದ-ಮೇಲೆ
ಬಂದ-ರಲ್ಲ
ಬಂದರು
ಬಂದರೂ
ಬಂದರೆ
ಬಂದ-ಲ್ಲದೆ
ಬಂದ-ಲ್ಲಿಗೆ
ಬಂದವ
ಬಂದ-ವನ
ಬಂದ-ವ-ನಲ್ಲ
ಬಂದ-ವ-ನಲ್ಲಿ
ಬಂದ-ವ-ನಿಗೆ
ಬಂದ-ವನು
ಬಂದ-ವನೇ
ಬಂದ-ವರ
ಬಂದ-ವ-ರನ್ನು
ಬಂದ-ವ-ರಲ್ಲ
ಬಂದ-ವ-ರಿ-ಗೆಲ್ಲಾ
ಬಂದ-ವರು
ಬಂದ-ವರೇ
ಬಂದ-ವಲ್ಲ
ಬಂದವು
ಬಂದಾಗ
ಬಂದಾ-ಗಲೂ
ಬಂದಾ-ಗಲೆ
ಬಂದಾ-ಗಲೇ
ಬಂದಾ-ಗಿ-ನಿಂ-ದಲೂ
ಬಂದಾದ
ಬಂದಾ-ದ-ಮೇಲೆ
ಬಂದಾವು
ಬಂದಿತು
ಬಂದಿತೋ
ಬಂದಿ-ತ್ತಲ್ಲ
ಬಂದಿತ್ತು
ಬಂದಿದೆ
ಬಂದಿ-ದೆಯೊ
ಬಂದಿ-ದೆಯೋ
ಬಂದಿ-ದ್ದರೆ
ಬಂದಿ-ದ್ದಾನೆ
ಬಂದಿ-ದ್ದಾರೆ
ಬಂದಿದ್ದು
ಬಂದಿ-ದ್ದೇನೆ
ಬಂದಿ-ದ್ದೇವೆ
ಬಂದಿ-ಯಾ-ಗು-ವುದು
ಬಂದಿರ
ಬಂದಿ-ರ-ಬ-ಹುದು
ಬಂದಿ-ರ-ಬೇಕು
ಬಂದಿ-ರ-ಬೇ-ಕೆಂದು
ಬಂದಿ-ರಲಿ
ಬಂದಿ-ರ-ಲಿಲ್ಲ
ಬಂದಿ-ರುವ
ಬಂದಿ-ರು-ವಂತೆ
ಬಂದಿ-ರು-ವನು
ಬಂದಿ-ರು-ವ-ರಲ್ಲ
ಬಂದಿ-ರು-ವರು
ಬಂದಿ-ರು-ವರೊ
ಬಂದಿ-ರು-ವರೋ
ಬಂದಿ-ರು-ವ-ವನ
ಬಂದಿ-ರು-ವ-ವರು
ಬಂದಿ-ರು-ವಾಗ
ಬಂದಿ-ರು-ವು-ದ-ರಲ್ಲಿ
ಬಂದಿ-ರು-ವು-ದ-ರಿಂದ
ಬಂದಿ-ರು-ವು-ದಷ್ಟೆ
ಬಂದಿ-ರು-ವುದು
ಬಂದಿ-ರು-ವು-ದೆ-ಲ್ಲಿಂದ
ಬಂದಿ-ರು-ವುದೇ
ಬಂದಿ-ರು-ವುದೋ
ಬಂದಿ-ರು-ವುವು
ಬಂದಿ-ರು-ವೆವು
ಬಂದಿಲ್ಲ
ಬಂದಿ-ಲ್ಲದೇ
ಬಂದಿ-ಲ್ಲ-ವೆಂ-ಬುದು
ಬಂದಿ-ಲ್ಲವೋ
ಬಂದಿವೆ
ಬಂದಿ-ವೆಯೊ
ಬಂದಿ-ವೆಯೋ
ಬಂದೀತು
ಬಂದೀ-ತು-ಎಂದು
ಬಂದು
ಬಂದು-ದನ್ನು
ಬಂದು-ದ-ರಲ್ಲಿ
ಬಂದು-ದ-ರಿಂದ
ಬಂದು-ದಲ್ಲ
ಬಂದುದು
ಬಂದು-ದೆಲ್ಲ
ಬಂದುದೇ
ಬಂದು-ಬಿ-ಟ್ಟರೆ
ಬಂದು-ಬಿ-ಡುವ
ಬಂದು-ಬಿ-ಡು-ವುದು
ಬಂದು-ಬಿ-ದ್ದರೆ
ಬಂದು-ಹೋ-ಗಿದೆ
ಬಂದು-ಹೋ-ಗಿ-ರು-ವುದು
ಬಂದು-ಹೋ-ಗುತ್ತ
ಬಂದು-ಹೋ-ಗು-ತ್ತಿ-ರು-ವುವು
ಬಂದು-ಹೋ-ಗು-ವುವು
ಬಂದೆ
ಬಂದೆಡೆ
ಬಂದೆ-ನಲ್ಲ
ಬಂದೆ-ಯಲ್ಲ
ಬಂದೆಯಾ
ಬಂದೆವು
ಬಂದೆವೋ
ಬಂದೇ
ಬಂದೊ-ಡನೆ
ಬಂದೊ-ಡ-ನೆಯೆ
ಬಂದೊ-ಡ-ನೆಯೇ
ಬಂಧಂ
ಬಂಧನ
ಬಂಧ-ನ-ಮೋ-ಕ್ಷಕ್ಕೆ
ಬಂಧ-ನ-ಕಾರಿ
ಬಂಧ-ನಕ್ಕೂ
ಬಂಧ-ನಕ್ಕೆ
ಬಂಧ-ನ-ಗಳನ್ನು
ಬಂಧ-ನ-ಗಳು
ಬಂಧ-ನ-ಗಳೇ
ಬಂಧ-ನದ
ಬಂಧ-ನ-ದಲ್ಲಿ
ಬಂಧ-ನ-ದ-ಲ್ಲಿ-ರು-ವಾಗ
ಬಂಧ-ನ-ದಿಂದ
ಬಂಧ-ನ-ಮೋ-ಕ್ಷ-ಗಳು
ಬಂಧ-ನ-ವನ್ನು
ಬಂಧ-ನವೂ
ಬಂಧ-ನವೇ
ಬಂಧ-ನವೋ
ಬಂಧ-ಮು-ಕ್ತ-ನಾ-ಗು-ತ್ತಾನೆ
ಬಂಧ-ಮೋ-ಕ್ಷ-ಗಳನ್ನು
ಬಂಧಾತ್
ಬಂಧಿ
ಬಂಧಿ-ಗಳು
ಬಂಧಿ-ಗಳೊ
ಬಂಧಿ-ತ-ರಾ-ಗು-ವೆವು
ಬಂಧಿ-ಯಲ್ಲ
ಬಂಧಿ-ಯಾಗಿದ್ದಾನೆ
ಬಂಧಿ-ಯಾ-ಗಿ-ರು-ವರು
ಬಂಧಿ-ಯಾ-ಗು-ವುದು
ಬಂಧಿ-ಸ-ಲಾ-ರದು
ಬಂಧಿ-ಸ-ಲಾ-ರದೆ
ಬಂಧಿ-ಸ-ಲಾ-ರವು
ಬಂಧಿ-ಸವು
ಬಂಧಿ-ಸಿದೆ
ಬಂಧಿ-ಸಿ-ರು-ವನು
ಬಂಧಿ-ಸಿ-ರು-ವುದು
ಬಂಧಿಸು
ಬಂಧಿ-ಸು-ತ್ತದೆ
ಬಂಧಿ-ಸು-ತ್ತವೆ
ಬಂಧಿ-ಸುವ
ಬಂಧಿ-ಸು-ವು-ದಾ-ವುದೂ
ಬಂಧಿ-ಸು-ವು-ದಿಲ್ಲ
ಬಂಧಿ-ಸು-ವುದು
ಬಂಧಿ-ಸು-ವುವು
ಬಂಧು
ಬಂಧು-ಇ-ವ-ರ-ಲ್ಲಿಯೂ
ಬಂಧು-ಗಳನ್ನು
ಬಂಧು-ಬ-ಳ-ಗ-ದ-ವ-ರನ್ನು
ಬಂಧು-ಬಾಂ-ಧ-ವ-ರನ್ನು
ಬಂಧು-ಬಾಂ-ಧ-ವ-ರಾ-ಗಿ-ರಲೀ
ಬಂಧು-ಬಾಂ-ಧ-ವ-ರಾ-ಗು-ತ್ತಾರೆ
ಬಂಧು-ಬಾಂ-ಧ-ವರು
ಬಂಧು-ರಾ-ತ್ಮಾ-ತ್ಮ-ನ-ಸ್ತಸ್ಯ
ಬಂಧು-ರಾ-ತ್ಮೈವ
ಬಂಧು-ವಾ-ಗಿ-ರು-ವುದು
ಬಂಧೂ
ಬಂಧೂ-ನ-ವ-ಸ್ಥಿ-ತಾನ್
ಬಗುಳಿ
ಬಗು-ಳು-ವೆವು
ಬಗೆ
ಬಗೆ-ಗ-ಳಿವೆ
ಬಗೆ-ದರೆ
ಬಗೆ-ದಿದ್ದ
ಬಗೆದು
ಬಗೆ-ಬ-ಗೆಯ
ಬಗೆ-ಬ-ಗೆ-ಯಾದ
ಬಗೆಯ
ಬಗೆ-ಯನ್ನು
ಬಗೆ-ಯಾಗಿ
ಬಗೆ-ಯಾ-ಗಿದೆ
ಬಗೆ-ಯಾ-ಗಿ-ರು-ತ್ತವೆ
ಬಗೆ-ಯಾ-ಗಿವೆ
ಬಗೆ-ಯಿದೆ
ಬಗೆ-ಹ-ರಿ-ಸು-ವು-ದಿಲ್ಲ
ಬಗ್ಗ-ಕೂ-ಡದು
ಬಗ್ಗ-ಡದ
ಬಗ್ಗ-ಡ-ದಲ್ಲಿ
ಬಗ್ಗ-ಡ-ವಾ-ಗಿದ್ದು
ಬಗ್ಗದೆ
ಬಗ್ಗಿ-ಸಿ-ದಾಗ
ಬಗ್ಗಿ-ಸು-ವು-ದಕ್ಕೆ
ಬಗ್ಗಿ-ಸು-ವುದು
ಬಗ್ಗಿ-ಹೋ-ಗು-ವುದು
ಬಗ್ಗೆ
ಬಚ್ಚಿ-ಟ್ಟರೆ
ಬಚ್ಚಿ-ಟ್ಟು-ಕೊ-ಳ್ಳ-ಬಲ್ಲ
ಬಚ್ಚಿ-ಡ-ಬೇ-ಕಾ-ದರೆ
ಬಚ್ಚಿ-ಡ-ಲಾ-ರದು
ಬಚ್ಚಿ-ಡಲು
ಬಚ್ಚಿ-ಡು-ವಂತೆ
ಬಚ್ಚಿ-ಡು-ವು-ದ-ಕ್ಕಾ-ಗು-ವು-ದಿಲ್ಲ
ಬಚ್ಚಿ-ಡು-ವು-ದಕ್ಕೆ
ಬಚ್ಚಿ-ಡು-ವು-ದಿಲ್ಲ
ಬಟ್ಟೆ
ಬಟ್ಟೆ-ಗಳನ್ನು
ಬಟ್ಟೆ-ಗಳು
ಬಟ್ಟೆ-ಗ-ಳೆಲ್ಲ
ಬಟ್ಟೆಗೆ
ಬಟ್ಟೆ-ಬರೆ
ಬಟ್ಟೆ-ಬ-ರೆ-ಗಳನ್ನು
ಬಟ್ಟೆ-ಬ-ರೆ-ಗಳೂ
ಬಟ್ಟೆಯ
ಬಟ್ಟೆ-ಯಂತೆ
ಬಟ್ಟೆ-ಯನ್ನು
ಬಟ್ಟೆ-ಯ-ಲ್ಲೆಲ್ಲಾ
ಬಟ್ಟೆ-ಯಾ-ಗಲಿ
ಬಟ್ಟೆಯು
ಬಡ
ಬಡಗಿ
ಬಡ-ತನ
ಬಡ-ತ-ನವೂ
ಬಡ-ಪಾ-ಯಿಗೆ
ಬಡ-ಪೆ-ಟ್ಟಗೆ
ಬಡ-ಪೆ-ಟ್ಟಿಗೆ
ಬಡ-ಭಾಷೆ
ಬಡ-ವನ
ಬಡ-ವ-ನಂತೆ
ಬಡ-ವ-ನನ್ನು
ಬಡ-ವ-ನಲ್ಲಿ
ಬಡ-ವ-ನಾ-ಗಿರ
ಬಡ-ವ-ನಾ-ಗಿ-ರಲಿ
ಬಡ-ವ-ನಿಗೆ
ಬಡಾಯಿ
ಬಡಿ-ತ-ವನ್ನು
ಬಡಿದ
ಬಡಿ-ದರೂ
ಬಡಿ-ದರೆ
ಬಡಿ-ದಾಗ
ಬಡಿ-ದಾ-ಗಲೇ
ಬಡಿದು
ಬಡಿ-ದು-ಕೊಂಡು
ಬಡಿ-ಯ-ದಂತೆ
ಬಡಿ-ಯ-ಬ-ಹುದು
ಬಡಿ-ಯ-ಬೇಕು
ಬಡಿ-ಯು-ತ್ತಿ-ರು-ವುದು
ಬಡಿ-ಯು-ವ-ವರು
ಬಡಿ-ಯು-ವು-ದಕ್ಕೆ
ಬಡಿ-ಯು-ವುದು
ಬಡಿ-ಸ-ಲಾ-ರರು
ಬಡಿ-ಸು-ತ್ತಾನೆ
ಬಡಿ-ಸು-ವರು
ಬಡಿ-ಸು-ವಳೆ
ಬಡ್ಜೆಟ್
ಬಡ್ಡಿ
ಬಡ್ಡಿ-ಯನ್ನು
ಬಡ್ಡಿ-ಯನ್ನೂ
ಬಡ್ಡಿಯೇ
ಬಡ್ಡಿ-ಸ-ಹಿತ
ಬಣ್ಣ
ಬಣ್ಣಕ್ಕೆ
ಬಣ್ಣ-ಗಳ
ಬಣ್ಣ-ಗಳನ್ನು
ಬಣ್ಣ-ಗಳಿಂದ
ಬಣ್ಣ-ಗ-ಳಿಗೆ
ಬಣ್ಣ-ಗಳು
ಬಣ್ಣದ
ಬಣ್ಣ-ದಂತೆ
ಬಣ್ಣ-ದಲ್ಲಿ
ಬಣ್ಣ-ದಿಂದ
ಬಣ್ಣದ್ದು
ಬಣ್ಣ-ನೆಗೆ
ಬಣ್ಣ-ನೆಯ
ಬಣ್ಣ-ಬ-ಣ್ಣದ
ಬಣ್ಣ-ವನ್ನು
ಬಣ್ಣ-ವನ್ನೂ
ಬಣ್ಣ-ವನ್ನೇ
ಬಣ್ಣ-ವಲ್ಲ
ಬಣ್ಣ-ವಾ-ಗಲಿ
ಬಣ್ಣ-ವಿಲ್ಲ
ಬಣ್ಣವೂ
ಬಣ್ಣ-ಸು-ಖ-ವೆಂ-ಬುದು
ಬಣ್ಣ-ಹಾ-ಕ-ಬೇ-ಕಾ-ಗಿದೆ
ಬಣ್ಣಿ-ಸಿ-ರು-ವರು
ಬಣ್ಣಿ-ಸು-ತ್ತದೆ
ಬಣ್ಣಿ-ಸು-ವರು
ಬಣ್ಣಿ-ಸು-ವಾಗ
ಬಣ್ಣಿ-ಸು-ವು-ದಕ್ಕೆ
ಬತ
ಬತ್ತದ
ಬತ್ತಿ
ಬತ್ತಿ-ಹೋ-ಗ-ಬ-ಹುದು
ಬತ್ತಿ-ಹೋಗಿ
ಬತ್ತಿ-ಹೋ-ದರೂ
ಬದ-ನೆ-ಕಾ-ಯನ್ನು
ಬದ-ನೆ-ಕಾಯಿ
ಬದ-ಲಾಗಿ
ಬದ-ಲಾ-ಗು-ವುದು
ಬದ-ಲಾಯಿ
ಬದ-ಲಾ-ಯಿಸ
ಬದ-ಲಾ-ಯಿ-ಸದೆ
ಬದ-ಲಾ-ಯಿ-ಸ-ಬ-ಲ್ಲುದು
ಬದ-ಲಾ-ಯಿ-ಸ-ಬ-ಹುದು
ಬದ-ಲಾ-ಯಿ-ಸ-ಬ-ಹು-ದೇನೊ
ಬದ-ಲಾ-ಯಿ-ಸ-ಬೇ-ಕಾ-ದರೆ
ಬದ-ಲಾ-ಯಿ-ಸ-ಬೇಕು
ಬದ-ಲಾ-ಯಿ-ಸ-ಬೇ-ಕೆಂ-ದಾ-ಗಲಿ
ಬದ-ಲಾ-ಯಿಸಿ
ಬದ-ಲಾ-ಯಿ-ಸಿ-ಕೊಂಡು
ಬದ-ಲಾ-ಯಿ-ಸಿ-ಕೊ-ಳ್ಳ-ಬ-ಹುದು
ಬದ-ಲಾ-ಯಿ-ಸಿ-ಕೊ-ಳ್ಳು-ವನು
ಬದ-ಲಾ-ಯಿ-ಸಿ-ದರೆ
ಬದ-ಲಾ-ಯಿ-ಸಿದೆ
ಬದ-ಲಾ-ಯಿಸು
ಬದ-ಲಾ-ಯಿ-ಸುತ್ತ
ಬದ-ಲಾ-ಯಿ-ಸುತ್ತಾ
ಬದ-ಲಾ-ಯಿ-ಸು-ತ್ತಾನೆ
ಬದ-ಲಾ-ಯಿ-ಸು-ತ್ತಿದೆ
ಬದ-ಲಾ-ಯಿ-ಸು-ತ್ತಿ-ದ್ದರೂ
ಬದ-ಲಾ-ಯಿ-ಸು-ತ್ತಿ-ರು-ವುದು
ಬದ-ಲಾ-ಯಿ-ಸು-ತ್ತಿ-ರು-ವುದೇ
ಬದ-ಲಾ-ಯಿ-ಸು-ತ್ತಿ-ರು-ವುವು
ಬದ-ಲಾ-ಯಿ-ಸು-ತ್ತಿವೆ
ಬದ-ಲಾ-ಯಿ-ಸು-ವನು
ಬದ-ಲಾ-ಯಿ-ಸುವು
ಬದ-ಲಾ-ಯಿ-ಸು-ವು-ದಕ್ಕೆ
ಬದ-ಲಾ-ಯಿ-ಸು-ವು-ದಿಲ್ಲ
ಬದ-ಲಾ-ಯಿ-ಸು-ವುದು
ಬದ-ಲಾ-ಯಿ-ಸು-ವುವು
ಬದ-ಲಾ-ವಣೆ
ಬದ-ಲಾ-ವ-ಣೆ-ಗಳ
ಬದ-ಲಾ-ವ-ಣೆ-ಗಳನ್ನೆಲ್ಲಾ
ಬದ-ಲಾ-ವ-ಣೆ-ಗ-ಳಿಗೆ
ಬದ-ಲಾ-ವ-ಣೆ-ಗ-ಳಿ-ಗೆಲ್ಲ
ಬದ-ಲಾ-ವ-ಣೆ-ಗ-ಳಿವೆ
ಬದ-ಲಾ-ವ-ಣೆ-ಗಳು
ಬದ-ಲಾ-ವ-ಣೆ-ಗ-ಳೆಲ್ಲ
ಬದ-ಲಾ-ವ-ಣೆಗೂ
ಬದ-ಲಾ-ವ-ಣೆಗೆ
ಬದ-ಲಾ-ವ-ಣೆ-ಗೆಲ್ಲ
ಬದ-ಲಾ-ವ-ಣೆಯ
ಬದ-ಲಾ-ವ-ಣೆ-ಯನ್ನು
ಬದ-ಲಾ-ವ-ಣೆ-ಯನ್ನೂ
ಬದ-ಲಾ-ವ-ಣೆ-ಯಾಗಿ
ಬದ-ಲಾ-ವ-ಣೆ-ಯಾ-ಗು-ತ್ತಿದೆ
ಬದ-ಲಾ-ವ-ಣೆ-ಯಿಂ-ದಲೂ
ಬದ-ಲಾ-ವ-ಣೆಯೂ
ಬದ-ಲಾ-ವ-ಣೆಯೆ
ಬದ-ಲಾ-ವ-ಣೆ-ಯೆಲ್ಲ
ಬದಲು
ಬದ-ವಾ-ವಣೆ
ಬದಿ-ಗೊಡ್ಡ
ಬದಿ-ಗೊ-ಡ್ಡು-ತ್ತಾನೆ
ಬದಿ-ಗೊ-ಡ್ಡು-ವನು
ಬದು-ಕ-ಬೇ-ಕಾ-ದರೆ
ಬದು-ಕ-ಲಾ-ರದೊ
ಬದು-ಕ-ಲಾ-ರೆವು
ಬದು-ಕ-ಲಾ-ರೆವೊ
ಬದು-ಕಲು
ಬದುಕಿ
ಬದು-ಕಿದ
ಬದು-ಕಿ-ದರೆ
ಬದು-ಕಿ-ದ-ವರು
ಬದು-ಕಿ-ದೆ-ಯೆಂದು
ಬದು-ಕಿ-ದೆವು
ಬದು-ಕಿ-ದ್ದರೆ
ಬದು-ಕಿ-ದ್ದೇವೆ
ಬದು-ಕಿ-ರ-ಬೇ-ಕಾ-ದರೆ
ಬದು-ಕಿ-ರ-ಬೇಕು
ಬದು-ಕಿ-ರ-ಲಾ-ರದು
ಬದು-ಕಿರು
ಬದು-ಕಿ-ರುವ
ಬದು-ಕಿ-ರು-ವನು
ಬದು-ಕಿ-ರು-ವನೊ
ಬದು-ಕಿ-ರು-ವರು
ಬದು-ಕಿ-ರು-ವ-ವ-ನಂತೆ
ಬದು-ಕಿ-ರು-ವ-ವ-ರಿ-ಗಾ-ಗಲಿ
ಬದು-ಕಿ-ರು-ವ-ವ-ರೆಗೆ
ಬದು-ಕಿ-ರು-ವಾಗ
ಬದು-ಕಿ-ರು-ವಾ-ಗಲೂ
ಬದು-ಕಿ-ರು-ವಾ-ಗಲೆ
ಬದು-ಕಿ-ರು-ವಾ-ಗಲೇ
ಬದು-ಕಿ-ರು-ವು-ದಕ್ಕೂ
ಬದು-ಕಿ-ರು-ವುದು
ಬದು-ಕಿ-ರು-ವುದೂ
ಬದು-ಕಿ-ರು-ವುದೇ
ಬದು-ಕಿ-ಸುವ
ಬದುಕು
ಬದು-ಕು-ತ್ತೇವೆ
ಬದು-ಕು-ವ-ವ-ರಲ್ಲ
ಬದು-ಕು-ವು-ದಕ್ಕೆ
ಬದು-ಕು-ವುದು
ಬದ್ಧ
ಬದ್ಧ-ಜೀವಿ
ಬದ್ಧ-ಜೀ-ವಿ-ಗ-ಳಂತೆ
ಬದ್ಧ-ಜೀ-ವಿ-ಗ-ಳಾಗಿ
ಬದ್ಧ-ಜೀ-ವಿ-ಗಳು
ಬದ್ಧ-ಜೀ-ವಿ-ಗ-ಳೆಲ್ಲ
ಬದ್ಧ-ಜೀ-ವಿ-ಯನ್ನು
ಬದ್ಧನ
ಬದ್ಧ-ನಲ್ಲ
ಬದ್ಧ-ನಾಗ
ಬದ್ಧ-ನಾ-ಗದೆ
ಬದ್ಧ-ನಾ-ಗ-ಬ-ಹುದು
ಬದ್ಧ-ನಾಗಿ
ಬದ್ಧ-ನಾ-ಗಿ-ದ್ದರೆ
ಬದ್ಧ-ನಾಗು
ಬದ್ಧ-ನಾ-ಗು-ತ್ತಾನೆ
ಬದ್ಧ-ನಾ-ಗುವು
ಬದ್ಧ-ನಾ-ಗು-ವು-ದಿಲ್ಲ
ಬದ್ಧನೂ
ಬದ್ಧ-ರಲ್ಲ
ಬದ್ಧ-ರಾಗಿ
ಬದ್ಧ-ರಾ-ಗಿ-ರುವ
ಬದ್ಧ-ರಾಗು
ಬದ್ಧ-ರಾ-ಗು-ತ್ತೇವೆ
ಬದ್ಧ-ರಾ-ಗು-ವರು
ಬದ್ಧ-ರಾ-ಗು-ವು-ದಕ್ಕೆ
ಬದ್ಧ-ರಾ-ಗು-ವು-ದಿಲ್ಲ
ಬದ್ಧ-ರಾ-ಗು-ವೆವು
ಬದ್ಧ-ರಾದ
ಬದ್ಧರು
ಬದ್ಧರೋ
ಬದ್ಧ-ವಾ-ಗಿದೆ
ಬದ್ಧ-ವಾ-ಗಿ-ರು-ವು-ದೆಲ್ಲ
ಬದ್ಧ-ವಾ-ಗಿಲ್ಲ
ಬದ್ಧ-ವಾ-ಗಿವೆ
ಬದ್ಧ-ವೈರಿ
ಬಧ್ನಾತಿ
ಬಧ್ಯತೇ
ಬನ್ನಿ
ಬಭೂವ
ಬಯಕೆ
ಬಯ-ಕೆ-ಗಳನ್ನು
ಬಯ-ಕೆ-ಗಳನ್ನೆಲ್ಲ
ಬಯ-ಕೆ-ಗ-ಳಿ-ದ್ದರೆ
ಬಯ-ಕೆ-ಗಳು
ಬಯ-ಕೆ-ಗಳೂ
ಬಯ-ಕೆ-ಗ-ಳೆಲ್ಲಾ
ಬಯ-ಕೆ-ಗಳೇ
ಬಯ-ಕೆಯ
ಬಯ-ಕೆ-ಯನ್ನು
ಬಯ-ಕೆ-ಯಾ-ದರೊ
ಬಯ-ಲಲ್ಲಿ
ಬಯ-ಲಾ-ಗ-ಬ-ಹುದೊ
ಬಯ-ಲಾ-ಗ-ಲೇ-ಬೇಕು
ಬಯ-ಲಾ-ಗು-ವುದೋ
ಬಯಲೇ
ಬಯ-ಸ-ಕೂ-ಡದು
ಬಯ-ಸದೆ
ಬಯ-ಸದೇ
ಬಯ-ಸ-ಬಾ-ರದು
ಬಯ-ಸ-ಲಿಲ್ಲ
ಬಯಸಿ
ಬಯ-ಸಿದ
ಬಯ-ಸಿ-ದನು
ಬಯ-ಸಿ-ದರೂ
ಬಯ-ಸಿ-ದರೆ
ಬಯ-ಸಿ-ದ-ವರು
ಬಯ-ಸಿ-ದಾಗ
ಬಯ-ಸಿದ್ದು
ಬಯಸು
ಬಯ-ಸು-ತ್ತಾರೆ
ಬಯ-ಸು-ತ್ತಿ-ದ್ದರು
ಬಯ-ಸು-ತ್ತಿ-ರು-ವರು
ಬಯ-ಸು-ತ್ತೇನೆ
ಬಯ-ಸು-ತ್ತೇ-ವೆಯೋ
ಬಯ-ಸುವ
ಬಯ-ಸು-ವನು
ಬಯ-ಸು-ವನೊ
ಬಯ-ಸು-ವ-ವನ
ಬಯ-ಸು-ವ-ವ-ನಿಗೆ
ಬಯ-ಸು-ವ-ವನು
ಬಯ-ಸು-ವ-ವರು
ಬಯ-ಸುವು
ಬಯ-ಸು-ವು-ದ-ಕ್ಕಾಗಿ
ಬಯ-ಸು-ವು-ದ-ರಿಂದ
ಬಯ-ಸು-ವು-ದಿಲ್ಲ
ಬಯ-ಸು-ವು-ದಿ-ಲ್ಲ-ವೊ-
ಬಯ-ಸು-ವು-ದಿ-ಲ್ಲವೋ
ಬಯ-ಸು-ವುದು
ಬಯ-ಸು-ವೆನು
ಬಯ-ಸು-ವೆಯೊ
ಬರ
ಬರ-ಕೂ-ಡದು
ಬರ-ಗಾಲ
ಬರ-ಗಾ-ಲದ
ಬರ-ಗಾ-ಲ-ವಿ-ರ-ಲಿಲ್ಲ
ಬರ-ಗಾ-ಲ-ವಿಲ್ಲ
ಬರ-ಡಾ-ಗು-ವುದು
ಬರದ
ಬರ-ದಂತೆ
ಬರ-ದಾಗ
ಬರ-ದಿ-ರಲಿ
ಬರ-ದಿ-ರುವ
ಬರ-ದಿ-ರು-ವು-ದಕ್ಕೆ
ಬರದೆ
ಬರದೇ
ಬರ-ಬ-ರುತ್ತ
ಬರ-ಬ-ಲ್ಲುದು
ಬರ-ಬ-ಹುದು
ಬರ-ಬ-ಹುದೆ
ಬರ-ಬ-ಹು-ದೆಂದು
ಬರ-ಬ-ಹುದೇ
ಬರ-ಬಾ-ರದು
ಬರ-ಬೇ-ಕಾ-ಗಿ-ತ್ತಲ್ಲ
ಬರ-ಬೇ-ಕಾ-ಗಿತ್ತು
ಬರ-ಬೇ-ಕಾ-ಗಿದೆ
ಬರ-ಬೇ-ಕಾ-ಗಿ-ರು-ವುದನ್ನು
ಬರ-ಬೇ-ಕಾ-ಗಿ-ರು-ವುದು
ಬರ-ಬೇ-ಕಾ-ಗಿಲ್ಲ
ಬರ-ಬೇ-ಕಾ-ಗು-ವುದು
ಬರ-ಬೇ-ಕಾದ
ಬರ-ಬೇ-ಕಾ-ದರೂ
ಬರ-ಬೇ-ಕಾ-ದರೆ
ಬರ-ಬೇಕು
ಬರ-ಬೇ-ಕೆಂ-ದಾಗ
ಬರ-ಬೇ-ಕೆಂ-ದಿಲ್ಲ
ಬರ-ಬೇ-ಕೆಂದು
ಬರ-ಬೇಡ
ಬರ-ಮಾಡಿ
ಬರ-ಮಾ-ಡಿ-ಕೊಂಡು
ಬರ-ಲಾ-ರದು
ಬರ-ಲಾ-ರರು
ಬರ-ಲಾ-ರವು
ಬರ-ಲಾ-ರೆವೊ
ಬರಲಿ
ಬರ-ಲಿ-ರು-ವರು
ಬರ-ಲಿಲ್ಲ
ಬರ-ಲಿ-ಲ್ಲ-ವಲ್ಲ
ಬರ-ಲಿ-ಲ್ಲ-ವೆಂಬ
ಬರಲು
ಬರಲೇ
ಬರ-ಲೇ-ಬೇ-ಕಾ-ಗಿದೆ
ಬರ-ಲೇ-ಬೇಕು
ಬರ-ಲೊ-ಲ್ಲದು
ಬರ-ವ-ಣಿ-ಗೆಯ
ಬರ-ವುದೊ
ಬರಹ
ಬರಿ
ಬರಿ-ದಾ-ದರೆ
ಬರಿಯ
ಬರಿ-ಸು-ವುದು
ಬರೀ
ಬರು
ಬರು-ಡೆಯ
ಬರುತ್ತ
ಬರು-ತ್ತ-ದಲ್ಲ
ಬರು-ತ್ತದೆ
ಬರು-ತ್ತ-ದೆಯೆ
ಬರು-ತ್ತಲೇ
ಬರು-ತ್ತವೆ
ಬರು-ತ್ತ-ವೆಯೊ
ಬರುತ್ತಾ
ಬರು-ತ್ತಾನೆ
ಬರು-ತ್ತಾ-ನೆಯೋ
ಬರು-ತ್ತಾನೋ
ಬರು-ತ್ತಾರೆ
ಬರು-ತ್ತಿತ್ತು
ಬರು-ತ್ತಿತ್ತೊ
ಬರು-ತ್ತಿದೆ
ಬರು-ತ್ತಿ-ದೆಯೆ
ಬರು-ತ್ತಿ-ದೆಯೋ
ಬರು-ತ್ತಿದ್ದ
ಬರು-ತ್ತಿ-ದ್ದರು
ಬರು-ತ್ತಿ-ದ್ದರೂ
ಬರು-ತ್ತಿ-ದ್ದರೆ
ಬರು-ತ್ತಿ-ದ್ದರೊ
ಬರು-ತ್ತಿ-ರ-ಬೇಕು
ಬರು-ತ್ತಿರು
ಬರು-ತ್ತಿ-ರುವ
ಬರು-ತ್ತಿ-ರು-ವರು
ಬರು-ತ್ತಿ-ರು-ವ-ವ-ನಿಗೆ
ಬರು-ತ್ತಿ-ರು-ವಾಗ
ಬರು-ತ್ತಿ-ರು-ವುದು
ಬರು-ತ್ತಿ-ರು-ವುವು
ಬರು-ತ್ತಿ-ರು-ವೆವು
ಬರು-ತ್ತಿಲ್ಲ
ಬರು-ತ್ತಿವೆ
ಬರು-ತ್ತಿ-ವೆಯೊ
ಬರು-ತ್ತೇನೆ
ಬರು-ತ್ತೇವೆ
ಬರು-ತ್ತೇ-ವೆಯೋ
ಬರು-ಬ-ರುತ್ತ
ಬರುವ
ಬರು-ವಂ-ತಿ-ಲ್ಲ-ಇ-ದನ್ನು
ಬರು-ವಂತೆ
ಬರು-ವಂ-ತೆಯೇ
ಬರು-ವನು
ಬರು-ವನೆ
ಬರು-ವನೊ
ಬರು-ವರು
ಬರು-ವ-ವ-ನಲ್ಲಿ
ಬರು-ವ-ವ-ನಿ-ಗಿಂತ
ಬರು-ವ-ವನು
ಬರು-ವ-ವನೂ
ಬರು-ವ-ವನೆ
ಬರು-ವ-ವರ
ಬರು-ವ-ವ-ರಲ್ಲ
ಬರು-ವ-ವ-ರಿಗೆ
ಬರು-ವ-ವರು
ಬರು-ವ-ವ-ರೆಗೆ
ಬರು-ವಾಗ
ಬರು-ವಾ-ಗಲೂ
ಬರು-ವಾ-ಗಲೇ
ಬರು-ವು-ದ-ಕ್ಕಾ-ಗು-ವು-ದಿಲ್ಲ
ಬರು-ವು-ದ-ಕ್ಕಾ-ದರೂ
ಬರು-ವು-ದ-ಕ್ಕಿಂತ
ಬರು-ವು-ದಕ್ಕೂ
ಬರು-ವು-ದಕ್ಕೆ
ಬರು-ವುದನ್ನು
ಬರು-ವು-ದ-ನ್ನೆಲ್ಲ
ಬರು-ವು-ದ-ನ್ನೆಲ್ಲಾ
ಬರು-ವು-ದರ
ಬರು-ವು-ದ-ರಲ್ಲಿ
ಬರು-ವು-ದಲ್ಲ
ಬರು-ವು-ದ-ವ-ನಿಂದ
ಬರು-ವು-ದಿದೆ
ಬರು-ವು-ದಿಲ್ಲ
ಬರು-ವು-ದಿ-ಲ್ಲವೆ
ಬರು-ವು-ದಿ-ಲ್ಲ-ವೆಂದು
ಬರು-ವು-ದಿ-ಲ್ಲವೇ
ಬರು-ವು-ದಿ-ಲ್ಲವೊ
ಬರು-ವುದು
ಬರು-ವು-ದು-ಹೋ-ಗು-ವು-ದ-ರಿಂದ
ಬರು-ವುದೂ
ಬರು-ವುದೆ
ಬರು-ವು-ದೆಂದು
ಬರು-ವು-ದೆಲ್ಲಾ
ಬರು-ವುದೇ
ಬರು-ವು-ದೇನೊ
ಬರು-ವುದೊ
ಬರು-ವುದೋ
ಬರು-ವುವು
ಬರು-ವುವೊ
ಬರು-ವುವೋ
ಬರು-ವೆವು
ಬರು-ವೆವೊ
ಬರು-ವೆವೋ
ಬರೆ
ಬರೆ-ಗಳನ್ನು
ಬರೆದ
ಬರೆ-ದಂತೆ
ಬರೆ-ದರು
ಬರೆ-ದರೂ
ಬರೆ-ದರೆ
ಬರೆ-ದರೊ
ಬರೆ-ದರೋ
ಬರೆ-ದ-ವನು
ಬರೆ-ದ-ವರು
ಬರೆದಿ
ಬರೆ-ದಿ-ಟ್ಟಿರು
ಬರೆ-ದಿ-ಡು-ತ್ತೇವೆ
ಬರೆ-ದಿದ್ದ
ಬರೆ-ದಿ-ದ್ದರೆ
ಬರೆ-ದಿ-ದ್ದಾನೆ
ಬರೆ-ದಿ-ರಲಿ
ಬರೆ-ದಿ-ರುವ
ಬರೆ-ದಿ-ರು-ವನು
ಬರೆ-ದಿ-ರು-ವರು
ಬರೆ-ದಿ-ರು-ವುದನ್ನು
ಬರೆ-ದಿ-ರು-ವುದು
ಬರೆ-ದಿ-ರು-ವುದೋ
ಬರೆ-ದಿ-ರು-ವೆವು
ಬರೆ-ದಿಲ್ಲ
ಬರೆದು
ಬರೆ-ದು-ಕೊ-ಡು-ತ್ತಾನೆ
ಬರೆ-ದು-ಕೊ-ಳ್ಳು-ವರು
ಬರೆ-ಯ-ತೊ-ಡ-ಗಿ-ದರು
ಬರೆ-ಯದ
ಬರೆ-ಯ-ಬ-ಹುದು
ಬರೆ-ಯ-ಬೇ-ಕಾ-ಗಿದೆ
ಬರೆ-ಯ-ಬೇ-ಕೆಂದು
ಬರೆ-ಯಲಿ
ಬರೆ-ಯ-ಲಿಲ್ಲ
ಬರೆ-ಯಲು
ಬರೆ-ಯ-ವ-ವನು
ಬರೆಯು
ಬರೆ-ಯುತ್ತಾ
ಬರೆ-ಯು-ತ್ತಾನೆ
ಬರೆ-ಯು-ತ್ತಾರೆ
ಬರೆ-ಯು-ವನು
ಬರೆ-ಯು-ವ-ವನು
ಬರೆ-ಯು-ವಾಗ
ಬರೆ-ಯು-ವಾ-ಗಲೇ
ಬರೆ-ಸು-ತ್ತಾರೆ
ಬರೆ-ಸು-ತ್ತಿ-ರು-ವರು
ಬರೆ-ಸು-ತ್ತೇವೆ
ಬರೇ
ಬಲ
ಬಲಂ
ಬಲ-ಗ-ಳಿಂ-ದಲೂ
ಬಲ-ಗ-ಳಿವೆ
ಬಲಗೈ
ಬಲ-ತ್ಕಾ-ರ-ವಾಗಿ
ಬಲದ
ಬಲ-ದಲ್ಲಿ
ಬಲ-ದಿಂದ
ಬಲ-ದಿಂ-ದಲೇ
ಬಲ-ದೇ-ವನ
ಬಲ-ಪ್ರ-ಯೋ-ಗ-ವೆಲ್ಲ
ಬಲ-ರಾ-ಮನ
ಬಲ-ವಂತ
ಬಲ-ವಂ-ತ-ನಾ-ಗಿ-ರು-ತ್ತಾನೆ
ಬಲ-ವಂ-ತರ
ಬಲ-ವಂ-ತ-ರ-ಲ್ಲಿ-ರುವ
ಬಲ-ವ-ತಾಂ
ಬಲ-ವ-ದ್ದೃ-ಢಮ್
ಬಲ-ವನ್ನು
ಬಲ-ವಾಗಿ
ಬಲ-ವಾ-ಗಿದೆ
ಬಲ-ವಾ-ಗಿ-ದ್ದರೂ
ಬಲ-ವಾ-ಗಿ-ದ್ದರೆ
ಬಲ-ವಾ-ಗಿ-ದ್ದ-ರೇನೆ
ಬಲ-ವಾ-ಗಿ-ದ್ದಾ-ಗಲೆ
ಬಲ-ವಾ-ಗಿರ
ಬಲ-ವಾ-ಗಿ-ರ-ಬೇ-ಕಾ-ದರೆ
ಬಲ-ವಾ-ಗಿ-ರ-ಬೇಕು
ಬಲ-ವಾ-ಗಿ-ರುವ
ಬಲ-ವಾ-ಗಿ-ರು-ವಾಗ
ಬಲ-ವಾ-ಗಿ-ರು-ವಾ-ಗಲೆ
ಬಲ-ವಾ-ಗಿ-ರು-ವುದು
ಬಲ-ವಾ-ಗಿ-ರು-ವುದೇ
ಬಲ-ವಾ-ಗುತ್ತ
ಬಲ-ವಾ-ಗುತ್ತಾ
ಬಲ-ವಾ-ಗು-ವುದು
ಬಲ-ವಾದ
ಬಲ-ವಾ-ದಂತೆ
ಬಲ-ವಾ-ದರೆ
ಬಲ-ವಾ-ದಾಗ
ಬಲ-ವಾ-ದುದು
ಬಲ-ವಾನ್
ಬಲ-ವಾ-ಯಿತು
ಬಲ-ವಿದೆ
ಬಲ-ವಿ-ದ್ದರೆ
ಬಲ-ವಿ-ದ್ದ-ವನೇ
ಬಲ-ವು-ಳ್ಳದ್ದು
ಬಲವೂ
ಬಲ-ವೊಂ-ದಿ-ದ್ದರೆ
ಬಲವೋ
ಬಲ-ಶಾ-ಲಿ-ಗ-ಳಾ-ಗ-ಬೇ-ಕಾ-ದರೆ
ಬಲಾ-ಢ್ಯ-ರಾ-ಗಿ-ರ-ಬ-ಹುದು
ಬಲಾ-ತ್ಕ-ರಿ-ಸ-ಲಾ-ರರು
ಬಲಾ-ತ್ಕ-ರಿ-ಸಿ-ದಾಗ
ಬಲಾ-ತ್ಕ-ರಿ-ಸು-ವುದು
ಬಲಾ-ತ್ಕಾರ
ಬಲಾ-ತ್ಕಾ-ರಕ್ಕೆ
ಬಲಾ-ತ್ಕಾ-ರ-ದಿಂದ
ಬಲಾ-ತ್ಕಾ-ರ-ದಿಂ-ದಲೋ
ಬಲಾ-ತ್ಕಾ-ರ-ವಾಗಿ
ಬಲಾ-ತ್ಕಾ-ರ-ವಾ-ಗಿಯೋ
ಬಲಾ-ತ್ಕಾ-ರ-ವಿಲ್ಲ
ಬಲಾ-ತ್ಕಾ-ರವೂ
ಬಲಾ-ದಿವ
ಬಲಾ-ಬ-ಲ-ಗಳನ್ನು
ಬಲಾ-ಬ-ಲ-ಗಳು
ಬಲಿ
ಬಲಿ-ಕೊ-ಡ-ಬೇ-ಕಾ-ಗು-ವುದು
ಬಲಿ-ಕೊ-ಡಲು
ಬಲಿ-ಕೊ-ಡು-ವನು
ಬಲಿ-ಕೊ-ಡು-ವುದು
ಬಲಿ-ಗೆಂದೇ
ಬಲಿತ
ಬಲಿತು
ಬಲಿ-ದಾನ
ಬಲಿ-ಪ್ರಿಯ
ಬಲಿ-ಯಾಗಿ
ಬಲಿಯೇ
ಬಲಿಷ್ಠ
ಬಲಿ-ಷ್ಠ-ರಾ-ದ-ವ-ರನ್ನು
ಬಲಿ-ಷ್ಠರು
ಬಲಿ-ಷ್ಠ-ವಾ-ಗಿದ್ದು
ಬಲಿ-ಷ್ಠ-ವಾ-ದುದು
ಬಲು
ಬಲೆ
ಬಲೆಗೂ
ಬಲೆಗೆ
ಬಲೆ-ಯನ್ನು
ಬಲೆ-ಯಲ್ಲಿ
ಬಲೆ-ಯಾ-ಗು-ವುವು
ಬಲೆ-ಯಿಂದ
ಬಲೆ-ಯೊ-ಳಗೆ
ಬಲ್ಬನ್ನು
ಬಲ್ಬಲ್ಲ
ಬಲ್ಬಿ
ಬಲ್ಬಿಗೆ
ಬಲ್ಬಿ-ದ್ದರೂ
ಬಲ್ಬಿನ
ಬಲ್ಬಿ-ನಂತೆ
ಬಲ್ಬಿ-ನಂ-ತೆ-ಅ-ವ-ರಲ್ಲಿ
ಬಲ್ಬಿ-ನಲ್ಲಿ
ಬಲ್ಬಿ-ನ-ಲ್ಲಿದೆ
ಬಲ್ಬಿ-ನ-ವ-ರೆಗೆ
ಬಲ್ಬಿ-ನಿಂದ
ಬಲ್ಬಿ-ನೊ-ಳಗೆ
ಬಲ್ಬು
ಬಲ್ಬು-ಗಳ
ಬಲ್ಬು-ಗಳನ್ನು
ಬಲ್ಬು-ಗಳಲ್ಲಿ
ಬಲ್ಬು-ಗಳೇ
ಬಲ್ಬೇ
ಬಲ್ಬ್
ಬಲ್ಲ
ಬಲ್ಲದು
ಬಲ್ಲದೇ
ಬಲ್ಲನು
ಬಲ್ಲನೊ
ಬಲ್ಲರು
ಬಲ್ಲರೋ
ಬಲ್ಲ-ವ-ನಲ್ಲಿ
ಬಲ್ಲ-ವ-ನಾ-ಗಿದ್ದ
ಬಲ್ಲ-ವ-ನಾ-ಗಿ-ರ-ಬೇಕು
ಬಲ್ಲ-ವ-ನಿಗೆ
ಬಲ್ಲ-ವನು
ಬಲ್ಲ-ವನೂ
ಬಲ್ಲ-ವರ
ಬಲ್ಲ-ವ-ರಾ-ಗಿ-ದ್ದರು
ಬಲ್ಲ-ವ-ರಿ-ಗೆಲ್ಲ
ಬಲ್ಲ-ವ-ರಿ-ಗೆಲ್ಲಾ
ಬಲ್ಲ-ವರು
ಬಲ್ಲ-ವರೆಲ್ಲೊ
ಬಲ್ಲವೆ
ಬಲ್ಲುದು
ಬಲ್ಲೆ
ಬಳ-ಕೆಗೆ
ಬಳ-ಕೆಯ
ಬಳ-ಕೆ-ಯ-ಲ್ಲಿತ್ತೊ
ಬಳ-ಕೆ-ಯ-ಲ್ಲಿ-ರುವ
ಬಳ-ಕೆ-ಯ-ಲ್ಲಿ-ರು-ವುದು
ಬಳಗ
ಬಳ-ಗ-ದ-ವ-ರಾ-ಗಿ-ರ-ಬ-ಹುದು
ಬಳ-ಗ-ದ-ವರೆಲ್ಲ
ಬಳ-ಲ-ಬೇ-ಕಾ-ಗಿಲ್ಲ
ಬಳಲಿ
ಬಳ-ಸದೆ
ಬಳ-ಸ-ಬೇಕು
ಬಳ-ಸಿ-ಕೊಂ-ಡಿ-ರು-ವೆನು
ಬಳ-ಸಿ-ಕೊಂಡು
ಬಳ-ಸಿ-ಕೊಂಡೋ
ಬಳ-ಸಿ-ದರೂ
ಬಳ-ಸಿ-ದರೆ
ಬಳ-ಸು-ತ್ತಾನೆ
ಬಳ-ಸು-ತ್ತಿ-ದ್ದರು
ಬಳ-ಸು-ವುದು
ಬಳಿ
ಬಳಿಕ
ಬಳಿಗೆ
ಬಳಿಗೇ
ಬಳಿದ
ಬಳಿ-ದಂತೆ
ಬಳಿ-ದಿ-ರು-ವು-ದ-ರಿಂದ
ಬಳಿದು
ಬಳಿ-ದು-ಕೊ-ಳ್ಳು-ತ್ತಾನೆ
ಬಳಿ-ಯಿಂದ
ಬಳಿ-ಯು-ವುದು
ಬಳಿ-ಯು-ವೆವು
ಬಳಿ-ಸಿ-ಕೊಂಡು
ಬಳೆ-ದಿ-ದ್ದೇವೆ
ಬಳೆಯ
ಬಳ್ಳಿ
ಬಸ್ಸಿ-ನಂತೆ
ಬಸ್ಸಿ-ನಲ್ಲಿ
ಬಸ್ಸು
ಬಹಳ
ಬಹ-ಳ-ಕಾಲ
ಬಹ-ಳ-ಕಾ-ಲ-ದಿಂದ
ಬಹ-ಳ-ಮಂದಿ
ಬಹ-ಳ-ಹೊತ್ತು
ಬಹವಃ
ಬಹವೋ
ಬಹಿ-ರಂಗ
ಬಹಿ-ರಂ-ತಶ್ಚ
ಬಹಿ-ರ್ಬಾ-ಹ್ಯಾಂ-ಶ್ಚ-ಕ್ಷು-ಶ್ಚೈ-ವಾಂ-ತರೇ
ಬಹಿ-ರ್ಮುಖ
ಬಹಿ-ರ್ಮು-ಖತೆ
ಬಹಿ-ರ್ಮು-ಖ-ವಾಗಿ
ಬಹಿ-ರ್ಮು-ಖ-ವಾ-ಗು-ವು-ದಕ್ಕೆ
ಬಹು
ಬಹು-ಕಾಲ
ಬಹು-ಕಾ-ಲ-ದಿಂದ
ಬಹು-ಜ-ನರ
ಬಹು-ಜ-ನ-ರಿಗೆ
ಬಹು-ದಂ-ಷ್ಟ್ರಾ-ಕ-ರಾಲಂ
ಬಹುದು
ಬಹು-ದೂರ
ಬಹು-ದೆಂದು
ಬಹುದೇ
ಬಹು-ದೇನೊ
ಬಹುದೊ
ಬಹುಧಾ
ಬಹು-ನೈ-ತೇನ
ಬಹು-ಪಾಲು
ಬಹು-ಬ-ಗೆಯ
ಬಹು-ಬಾ-ಹೂ-ರು-ಪಾ-ದಮ್
ಬಹು-ಭಾಗ
ಬಹು-ಭಾ-ಗ-ವನ್ನು
ಬಹು-ಭಾ-ಗ-ವೆಲ್ಲ
ಬಹು-ಮಂದಿ
ಬಹು-ಮತೋ
ಬಹು-ಮಾನ
ಬಹು-ಮಾ-ನ-ವನ್ನೂ
ಬಹು-ಮಾ-ನ-ವನ್ನೋ
ಬಹು-ಮಾ-ನ್ಯ-ನಾ-ಗಿ-ದ್ದೆಯೊ
ಬಹು-ಮು-ಖ-ತೆ-ಯನ್ನು
ಬಹು-ಮು-ಖದ
ಬಹು-ಮು-ಖ-ವಾದ
ಬಹು-ರೂ-ಪ-ವನ್ನು
ಬಹು-ಲಾ-ಯಾಸಂ
ಬಹು-ವ-ಕ್ತ್ರ-ನೇತ್ರಂ
ಬಹು-ವಿ-ಧ-ವಾದ
ಬಹು-ವಿಧಾ
ಬಹು-ವೋಂ-ಬು-ವೇ-ಗಾಃ
ಬಹು-ವ್ರೀ-ಹಿ-ಯಲ್ಲಿ
ಬಹುಶಃ
ಬಹು-ಶಾಖಾ
ಬಹೂ-ದರಂ
ಬಹೂ-ನಾಂ
ಬಹೂನಿ
ಬಹೂನ್
ಬಹೂ-ನ್ಯ-ದೃ-ಷ್ಟ-ಪೂ-ರ್ವಾಣಿ
ಬಾ
ಬಾಂಡ-ಲೆಯ
ಬಾಂಡ-ಲೆ-ಯಲ್ಲಿ
ಬಾಂಡ-ಲೆ-ಯಿಂದ
ಬಾಂಧವ
ಬಾಂಧ-ವ-ರನ್ನು
ಬಾಂಧ-ವರು
ಬಾಂಧ-ವರೂ
ಬಾಂಬನ್ನು
ಬಾಂಬು
ಬಾಕಿ
ಬಾಕ್ಸ್ಗೆ
ಬಾಗ-ಬೇಕು
ಬಾಗ-ಲಿಗೆ
ಬಾಗಿ
ಬಾಗಿ-ದರೆ
ಬಾಗಿ-ದ್ದರೆ
ಬಾಗಿ-ರ-ಬಾ-ರದು
ಬಾಗಿ-ಲನ್ನು
ಬಾಗಿ-ಲಿಗೆ
ಬಾಗಿ-ಲಿನ
ಬಾಗಿ-ಲಿ-ನಂ-ತಿದೆ
ಬಾಗಿ-ಲಿ-ನಂತೆ
ಬಾಗಿ-ಲಿ-ನಲ್ಲಿ
ಬಾಗಿ-ಲಿ-ನಲ್ಲೆ
ಬಾಗಿ-ಲಿ-ನಿಂದ
ಬಾಗಿಲು
ಬಾಗಿ-ಲು-ಗಳನ್ನು
ಬಾಗಿ-ಲು-ಗ-ಳುಳ್ಳ
ಬಾಗಿ-ಲು-ಗ-ಳೆಲ್ಲ
ಬಾಗಿ-ಲು-ಗಳೇ
ಬಾಗಿ-ಸ-ಬೇಕು
ಬಾಗು-ತ್ತಾನೆ
ಬಾಗು-ತ್ತಿ-ದ್ದರು
ಬಾಗು-ವನು
ಬಾಗು-ವುದು
ಬಾಗು-ವುದೊ
ಬಾಗು-ವುವು
ಬಾಚಿ
ಬಾಚಿ-ಕೊ-ಳ್ಳಲು
ಬಾಟ-ಲಿನ
ಬಾಟ-ಲು-ಗ-ಳಿಗೆ
ಬಾಡಬ
ಬಾಡಿ-ಗೆಗೆ
ಬಾಡಿದ
ಬಾಡಿ-ಹೋ-ಗಿದೆ
ಬಾಡಿ-ಹೋ-ಗು-ವುದು
ಬಾಡಿ-ಹೋ-ದಂತೆ
ಬಾಡುವ
ಬಾಡು-ವುದು
ಬಾಣ
ಬಾಣ-ಗಳನ್ನು
ಬಾಣದ
ಬಾಣ-ದಂತೆ
ಬಾಣ-ದಿಂದ
ಬಾಣ-ವನ್ನು
ಬಾಣವೇ
ಬಾತಿಗೆ
ಬಾಧ-ಕ-ವಿಲ್ಲ
ಬಾಧ-ಕವೂ
ಬಾಧಿತ
ಬಾಧಿ-ತ-ನಾ-ಗ-ದ-ವನು
ಬಾಧಿ-ತ-ನಾ-ಗದೆ
ಬಾಧಿ-ತ-ನಾಗಿ
ಬಾಧಿ-ತ-ನಾ-ಗು-ವು-ದಿಲ್ಲ
ಬಾಧಿ-ತ-ರಾ-ಗ-ದ-ವರು
ಬಾಧಿ-ತ-ರಾ-ಗು-ವು-ದಿಲ್ಲ
ಬಾಧಿ-ತ-ವಾ-ಗದೆ
ಬಾಧಿ-ತ-ವಾಗಿ
ಬಾಧಿ-ತ-ವಾ-ಗಿಲ್ಲ
ಬಾಧಿ-ತ-ವಾ-ಗು-ವು-ದಿಲ್ಲ
ಬಾಧಿ-ತ-ವಾ-ಗು-ವುದು
ಬಾಧಿ-ದ-ವಾ-ಗು-ವು-ದಿಲ್ಲ
ಬಾಧಿ-ಸ-ದಂತೆ
ಬಾಧಿ-ಸ-ಲಾ-ರವು
ಬಾಧಿ-ಸಿ-ದು-ದ-ರಲ್ಲಿ
ಬಾಧಿ-ಸು-ತ್ತಿದ್ದ
ಬಾಧಿ-ಸು-ತ್ತಿ-ದ್ದು-ದನ್ನು
ಬಾಧಿ-ಸು-ತ್ತಿ-ರು-ವುದು
ಬಾಧಿ-ಸು-ತ್ತಿ-ರು-ವುವು
ಬಾಧಿ-ಸು-ವುದು
ಬಾಧೆಯೂ
ಬಾಧ್ಯ-ತೆ-ಗಳು
ಬಾಯನ್ನು
ಬಾಯಲ್ಲಿ
ಬಾಯ-ಲ್ಲಿ-ರುವ
ಬಾಯಾರಿ
ಬಾಯಾ-ರಿಕೆ
ಬಾಯಾ-ರಿ-ಕೆ-ಯಿಂದ
ಬಾಯಾ-ರಿ-ದರೆ
ಬಾಯಿ
ಬಾಯಿಂದ
ಬಾಯಿಂ-ದಲೂ
ಬಾಯಿಂ-ದಲೆ
ಬಾಯಿಂ-ದಲೇ
ಬಾಯಿ-ಗ-ಳು-ಳ್ಳದ್ದು
ಬಾಯಿಗೆ
ಬಾಯಿ-ಗೆಲ್ಲಾ
ಬಾಯಿ-ತೆ-ರೆದು
ಬಾಯಿ-ತೆ-ರೆ-ದು-ಕೊಂ-ಡಿ-ರು-ವುದು
ಬಾಯಿನ
ಬಾಯಿ-ನಲ್ಲಿ
ಬಾಯಿ-ನಿಂದ
ಬಾಯಿ-ನಿಂ-ದಲೇ
ಬಾಯಿ-ಬಿ-ಟ್ಟು-ಕೊಂ-ಡಿ-ರು-ವ-ವನು
ಬಾಯಿಯ
ಬಾಯಿ-ಯನ್ನು
ಬಾಯಿ-ಯಲ್ಲಿ
ಬಾಯಿ-ಯ-ಲ್ಲಿ-ರುವ
ಬಾಯಿ-ಯಿಂ-ದಲೂ
ಬಾಯಿ-ಯಿಂ-ದಲೇ
ಬಾಯಿ-ಯು-ಳ್ಳ-ವನು
ಬಾಯಿ-ಯೊ-ಳಗೆ
ಬಾಯಿ-ಹಾ-ಕಿ-ದೊ-ಡ-ನೆಯೇ
ಬಾಯೊ-ಳಗೆ
ಬಾರದ
ಬಾರ-ದಂತೆ
ಬಾರ-ದವು
ಬಾರ-ದಿ-ರಲಿ
ಬಾರ-ದಿ-ರುವ
ಬಾರದು
ಬಾರ-ದು-ದಲ್ಲ
ಬಾರ-ದುದು
ಬಾರದೆ
ಬಾರನು
ಬಾರವು
ಬಾರಿ
ಬಾರಿಗೆ
ಬಾರಿ-ಯಲ್ಲ
ಬಾರಿ-ಸಿ-ದಂತೆ
ಬಾರಿ-ಸು-ವಂತೆ
ಬಾರಿ-ಸು-ವು-ದಕ್ಕೆ
ಬಾಲ
ಬಾಲ-ಗಂ-ಗಾ-ಧರ
ಬಾಲ-ಗೋ-ಪಾಲ
ಬಾಲದ
ಬಾಲ-ದಂತೆ
ಬಾಲ-ದಲ್ಲಿ
ಬಾಲ-ದ-ಲ್ಲಿ-ರುವ
ಬಾಲ-ಲೀ-ಲೆ-ಯನ್ನು
ಬಾಲ-ವನ್ನು
ಬಾಲ್ಯ
ಬಾಲ್ಯ-ದಲ್ಲಿ
ಬಾಲ್ಯ-ದಿಂದ
ಬಾಲ್ಯ-ದಿಂ-ದಲೂ
ಬಾಳ-ತ-ಕ್ಕಂ-ತಹ
ಬಾಳನ್ನು
ಬಾಳ-ನ್ನೆಲ್ಲ
ಬಾಳನ್ನೇ
ಬಾಳ-ಬೇಕು
ಬಾಳ-ಬೇ-ಕೆಂದು
ಬಾಳಲು
ಬಾಳಾ-ಗಲೀ
ಬಾಳಿ
ಬಾಳಿಗೆ
ಬಾಳಿದ
ಬಾಳಿ-ದರೆ
ಬಾಳಿನ
ಬಾಳಿ-ನಲ್ಲಿ
ಬಾಳಿ-ನ-ಲ್ಲಿಯೇ
ಬಾಳು
ಬಾಳು-ತ್ತಾನೆ
ಬಾಳು-ತ್ತಿ-ದ್ದರೆ
ಬಾಳು-ತ್ತಿ-ರು-ವರೊ
ಬಾಳು-ತ್ತಿ-ರು-ವ-ವರು
ಬಾಳು-ತ್ತಿ-ರು-ವೆವು
ಬಾಳು-ತ್ತಿವೆ
ಬಾಳು-ತ್ತೇವೆ
ಬಾಳು-ವಂ-ತೆಯೇ
ಬಾಳು-ವು-ದಕ್ಕೆ
ಬಾಳು-ವುದು
ಬಾಳು-ವುದೇ
ಬಾಳು-ವುದೋ
ಬಾಳುವೆ
ಬಾಳು-ವೆಗೆ
ಬಾಳು-ವೆಯೆ
ಬಾಳು-ವೆವೊ
ಬಾಳೆ
ಬಾಳೆಲ್ಲ
ಬಾಳೆಲ್ಲಾ
ಬಾಳೇ
ಬಾವಿ
ಬಾವಿಗೆ
ಬಾವಿಗೊ
ಬಾವಿ-ಯಲ್ಲಿ
ಬಾಸುಂದಿ
ಬಾಸುಂ-ದಿ-ಮಾಡಿ
ಬಾಹಿ-ರ-ರಲ್ಲ
ಬಾಹು-ಗಳನ್ನು
ಬಾಹು-ಗಳು
ಬಾಹು-ಗ-ಳು-ಳ್ಳ-ವನು
ಬಾಹು-ವಾದ
ಬಾಹುವೆ
ಬಾಹ್ಯ
ಬಾಹ್ಯ-ಘ-ಟ-ನೆಗೂ
ಬಾಹ್ಯ-ಘ-ಟ-ನೆಯೂ
ಬಾಹ್ಯ-ತೃ-ಪ್ತಿ-ಯನ್ನು
ಬಾಹ್ಯದ
ಬಾಹ್ಯ-ದಲ್ಲಿ
ಬಾಹ್ಯ-ದ-ಲ್ಲಿ-ಇಲ್ಲ
ಬಾಹ್ಯ-ದ-ಲ್ಲಿಯೇ
ಬಾಹ್ಯ-ದೃಷ್ಟಿ
ಬಾಹ್ಯ-ದೃ-ಷ್ಟಿ-ಯಿಂದ
ಬಾಹ್ಯ-ದೃ-ಷ್ಟಿ-ಯಿಂ-ದಲ್ಲ
ಬಾಹ್ಯ-ಪ್ರ-ಪಂ-ಚದ
ಬಾಹ್ಯ-ಬಲ
ಬಾಹ್ಯ-ಮು-ಖ-ವಾಗಿ
ಬಾಹ್ಯ-ಮು-ಖ-ವಾ-ಗಿದೆ
ಬಾಹ್ಯ-ಮು-ಖ-ವಾ-ಗು-ವುದು
ಬಾಹ್ಯ-ವನ್ನು
ಬಾಹ್ಯ-ವಲ್ಲ
ಬಾಹ್ಯ-ವಸ್ತು
ಬಾಹ್ಯ-ವ-ಸ್ತು-ಗಳ
ಬಾಹ್ಯ-ವ-ಸ್ತು-ವನ್ನು
ಬಾಹ್ಯ-ವ-ಸ್ತು-ವಿನ
ಬಾಹ್ಯ-ವ-ಸ್ತು-ವಿ-ನೊಂ-ದಿಗೆ
ಬಾಹ್ಯ-ಶೌ-ಚ-ಕ್ಕಿಂತ
ಬಾಹ್ಯ-ಸು-ಖದ
ಬಾಹ್ಯ-ಸ್ಪ-ರ್ಶೇ-ಷ್ವ-ಸ-ಕ್ತಾತ್ಮಾ
ಬಾಹ್ಯಾ-ಚಾರ
ಬಾಹ್ಯಾ-ಚಾ-ರ-ಗಳನ್ನು
ಬಾಹ್ಯೇಂ-ದ್ರಿ-ಯ-ನಿ-ಗ್ರಹ
ಬಾಹ್ಲೀಕ
ಬಿಂಬಿ-ಸು-ವನು
ಬಿಕರಿ
ಬಿಕ್ಕಿ-ಬಿಕ್ಕಿ
ಬಿಗಿ
ಬಿಗಿದ
ಬಿಗಿ-ದಂತೆ
ಬಿಗಿ-ದಿ-ಡು-ವುದು
ಬಿಗಿ-ದಿದೆ
ಬಿಗಿ-ದಿ-ದೆಯೊ
ಬಿಗಿ-ದಿರು
ಬಿಗಿ-ದಿ-ರುವ
ಬಿಗಿ-ದಿ-ರು-ವನು
ಬಿಗಿ-ದಿ-ರು-ವುದೆ
ಬಿಗಿ-ದಿ-ರು-ವುದೊ
ಬಿಗಿದು
ಬಿಗಿ-ದು-ಕೊಂಡು
ಬಿಗಿ-ಯಾಗಿ
ಬಿಗಿ-ಯಾ-ದರೆ
ಬಿಗಿ-ಯುವ
ಬಿಗಿ-ಯು-ವು-ದಕ್ಕೆ
ಬಿಗಿ-ಯು-ವುದು
ಬಿಗಿ-ಯು-ವುದೂ
ಬಿಗಿ-ಹಿ-ಡಿ-ದು-ಕೊಂಡು
ಬಿಚ್ಚದೆ
ಬಿಚ್ಚ-ಬ-ಹುದು
ಬಿಚ್ಚಿ
ಬಿಚ್ಚಿ-ಕೊ-ಳ್ಳ-ಬೇ-ಕೆಂಬ
ಬಿಚ್ಚಿ-ಕೊ-ಳ್ಳಲು
ಬಿಚ್ಚಿ-ಕೊ-ಳ್ಳಲೂ
ಬಿಚ್ಚಿ-ಕೊಳ್ಳು
ಬಿಚ್ಚಿ-ನೋ-ಡುವ
ಬಿಚ್ಚುತ್ತಾ
ಬಿಚ್ಚುವ
ಬಿಚ್ಚು-ವಂತೆ
ಬಿಚ್ಚು-ವುದೂ
ಬಿಟ್ಚು
ಬಿಟ್ಚು-ಪ-ರಿ-ಶು-ದ್ಧ-ವಾಗಿ
ಬಿಟ್ಟ
ಬಿಟ್ಟಂ-ತಿದೆ
ಬಿಟ್ಟಂತೆ
ಬಿಟ್ಟ-ದ್ದನ್ನು
ಬಿಟ್ಟ-ಬ-ರಲು
ಬಿಟ್ಟ-ಮೇಲೆ
ಬಿಟ್ಟರು
ಬಿಟ್ಟರೂ
ಬಿಟ್ಟರೆ
ಬಿಟ್ಟ-ರೆಷ್ಟು
ಬಿಟ್ಟ-ರೇನೆ
ಬಿಟ್ಟ-ರೇನೇ
ಬಿಟ್ಟ-ವ-ನ-ವನು
ಬಿಟ್ಟ-ವ-ನಿಗೆ
ಬಿಟ್ಟ-ವ-ನಿಗೇ
ಬಿಟ್ಟ-ವನು
ಬಿಟ್ಟ-ವರು
ಬಿಟ್ಟಾಗ
ಬಿಟ್ಟಾದ
ಬಿಟ್ಟಾ-ಯಿತು
ಬಿಟ್ಟಿ
ಬಿಟ್ಟಿತು
ಬಿಟ್ಟಿದೆ
ಬಿಟ್ಟಿ-ದ್ದನೊ
ಬಿಟ್ಟಿ-ದ್ದರೆ
ಬಿಟ್ಟಿ-ದ್ದಲ್ಲ
ಬಿಟ್ಟಿ-ದ್ದಾನೆ
ಬಿಟ್ಟಿದ್ದು
ಬಿಟ್ಟಿ-ದ್ದೇವೆ
ಬಿಟ್ಟಿ-ಯಾಗಿ
ಬಿಟ್ಟಿ-ರ-ಬೇಕು
ಬಿಟ್ಟಿ-ರು-ತ್ತಾನೆ
ಬಿಟ್ಟಿ-ರುವ
ಬಿಟ್ಟಿ-ರು-ವನು
ಬಿಟ್ಟಿ-ರು-ವನೊ
ಬಿಟ್ಟಿ-ರು-ವರು
ಬಿಟ್ಟಿ-ರು-ವ-ವರೂ
ಬಿಟ್ಟಿ-ರು-ವು-ದಕ್ಕೆ
ಬಿಟ್ಟಿ-ರು-ವು-ದ-ರಿಂದ
ಬಿಟ್ಟಿ-ರು-ವು-ದಿಲ್ಲ
ಬಿಟ್ಟಿ-ರು-ವುದು
ಬಿಟ್ಟಿ-ರು-ವೆವು
ಬಿಟ್ಟಿ-ರು-ವೆವೆ
ಬಿಟ್ಟಿ-ರು-ವೆವೊ
ಬಿಟ್ಟಿ-ರು-ವೆವೋ
ಬಿಟ್ಟಿಲ್ಲ
ಬಿಟ್ಟು
ಬಿಟ್ಟು-ಕೊಂಡು
ಬಿಟ್ಟು-ಕೊ-ಡು-ವರು
ಬಿಟ್ಟು-ಕೊ-ಡು-ವು-ದಿಲ್ಲ
ಬಿಟ್ಟು-ಬ-ರು-ವು-ದಕ್ಕೆ
ಬಿಟ್ಟು-ಬಿ-ಟ್ಟರೆ
ಬಿಟ್ಟು-ಬಿ-ಟ್ಟಿ-ರು-ತ್ತಾನೆ
ಬಿಟ್ಟು-ಬಿಡಿ
ಬಿಟ್ಟು-ಬಿಡು
ಬಿಟ್ಟು-ಬಿ-ಡು-ವು-ದಕ್ಕೆ
ಬಿಟ್ಟು-ಬಿ-ಡು-ವುದು
ಬಿಟ್ಟು-ಬಿ-ಡು-ವೆವು
ಬಿಟ್ಟು-ಹೋ-ಗ-ಕೂ-ಡದು
ಬಿಟ್ಟು-ಹೋ-ಗ-ದಂತೆ
ಬಿಟ್ಟು-ಹೋ-ಗದು
ಬಿಟ್ಟು-ಹೋ-ಗ-ಬೇ-ಕಾ-ಗಿದೆ
ಬಿಟ್ಟು-ಹೋ-ಗ-ಬೇ-ಕಾ-ದರೆ
ಬಿಟ್ಟು-ಹೋ-ಗ-ಬೇಕೊ
ಬಿಟ್ಟು-ಹೋ-ಗಲು
ಬಿಟ್ಟು-ಹೋ-ಗಿಲ್ಲ
ಬಿಟ್ಟು-ಹೋ-ಗಿವೆ
ಬಿಟ್ಟು-ಹೋಗು
ಬಿಟ್ಟು-ಹೋ-ಗು-ತ್ತಾರೆ
ಬಿಟ್ಟು-ಹೋ-ಗು-ತ್ತಿ-ರ-ಲಿ-ಲ್ಲ-ವಂತೆ
ಬಿಟ್ಟು-ಹೋ-ಗು-ವು-ದಕ್ಕೆ
ಬಿಟ್ಟು-ಹೋ-ಗು-ವು-ದಲ್ಲ
ಬಿಟ್ಟು-ಹೋ-ಗು-ವು-ದಿಲ್ಲ
ಬಿಟ್ಟು-ಹೋ-ಗು-ವುದು
ಬಿಟ್ಟು-ಹೋ-ಗು-ವುದೋ
ಬಿಟ್ಟು-ಹೋ-ಗು-ವೆನೋ
ಬಿಟ್ಟು-ಹೋ-ಗು-ವೆವು
ಬಿಟ್ಟು-ಹೋದ
ಬಿಟ್ಟು-ಹೋ-ಯಿತು
ಬಿಟ್ಟೇ-ಬಿ-ಡು-ವರು
ಬಿಟ್ಟೇ-ಬಿ-ಡು-ವುವು
ಬಿಟ್ಟೇ-ಳು-ವುದು
ಬಿಟ್ಟೊ-ಡ-ನೆಯೆ
ಬಿಡ
ಬಿಡ-ಕೂ-ಡದು
ಬಿಡ-ಕೂ-ಡ-ದೆಂದು
ಬಿಡ-ಗಡೆ
ಬಿಡದ
ಬಿಡ-ದಂತೆ
ಬಿಡ-ದ-ವನು
ಬಿಡದು
ಬಿಡದೆ
ಬಿಡದೇ
ಬಿಡ-ಬ-ಹು-ದಲ್ಲ
ಬಿಡ-ಬ-ಹು-ದಾ-ಗಿತ್ತು
ಬಿಡ-ಬ-ಹುದು
ಬಿಡ-ಬ-ಹುದೆ
ಬಿಡ-ಬಾ-ರದು
ಬಿಡ-ಬೇ-ಕಲ್ಲ
ಬಿಡ-ಬೇ-ಕಾಗಿ
ಬಿಡ-ಬೇ-ಕಾ-ಗಿಲ್ಲ
ಬಿಡ-ಬೇ-ಕಾ-ಗು-ವುದು
ಬಿಡ-ಬೇ-ಕಾ-ದರೆ
ಬಿಡ-ಬೇ-ಕಾ-ದು-ದನ್ನು
ಬಿಡ-ಬೇ-ಕಾ-ದು-ದೇನೊ
ಬಿಡ-ಬೇಕು
ಬಿಡ-ಬೇಕೆ
ಬಿಡ-ಬೇ-ಕೆಂ-ದಿ-ರು-ವೆವೊ
ಬಿಡ-ಬೇ-ಕೆಂದು
ಬಿಡ-ಬೇಡ
ಬಿಡ-ಲಾ-ಗು-ವುದೆ
ಬಿಡ-ಲಾ-ರದೊ
ಬಿಡ-ಲಾ-ರವು
ಬಿಡ-ಲಾರೆ
ಬಿಡಲಿ
ಬಿಡ-ಲಿಲ್ಲ
ಬಿಡಲು
ಬಿಡಲೆ
ಬಿಡ-ಲೆ-ತ್ನಿ-ಸು-ವುದು
ಬಿಡ-ಲ್ಪಟ್ಟ
ಬಿಡ-ಲ್ಪ-ಟ್ಟ-ವನು
ಬಿಡ-ಲ್ಪ-ಟ್ಟ-ವರೂ
ಬಿಡಿ
ಬಿಡಿ-ಭಾ-ಗ-ಗಳು
ಬಿಡಿ-ಸ-ಬಲ್ಲ
ಬಿಡಿ-ಸ-ಬೇ-ಕಾ-ದರೆ
ಬಿಡಿ-ಸ-ಬೇಕು
ಬಿಡಿ-ಸ-ಬೇ-ಕೆಂದು
ಬಿಡಿ-ಸಲು
ಬಿಡಿಸಿ
ಬಿಡಿ-ಸಿ-ಕೊಂಡು
ಬಿಡಿ-ಸಿ-ಕೊ-ಳ್ಳ-ಬೇ-ಕಾ-ಗಿದೆ
ಬಿಡಿ-ಸಿ-ಕೊ-ಳ್ಳ-ಬೇಕು
ಬಿಡಿ-ಸಿ-ಕೊ-ಳ್ಳು-ವಂತೆ
ಬಿಡಿ-ಸಿ-ಕೊ-ಳ್ಳು-ವುದೂ
ಬಿಡಿಸು
ಬಿಡಿ-ಸು-ತ್ತಾನೆ
ಬಿಡಿ-ಸು-ತ್ತಾರೆ
ಬಿಡಿ-ಸು-ತ್ತಿ-ರು-ವನು
ಬಿಡಿ-ಸುವ
ಬಿಡಿ-ಸು-ವನು
ಬಿಡಿ-ಸು-ವಾಗ
ಬಿಡಿ-ಸು-ವು-ದ-ಕ್ಕಾಗಿ
ಬಿಡಿ-ಸು-ವು-ದಕ್ಕೆ
ಬಿಡಿ-ಸು-ವು-ದಿಲ್ಲ
ಬಿಡಿ-ಸು-ವುದು
ಬಿಡಿ-ಸು-ವು-ದುಈ
ಬಿಡಿ-ಸು-ವುದೂ
ಬಿಡಿ-ಸು-ವುದೊ
ಬಿಡು
ಬಿಡು-ಗಡೆ
ಬಿಡು-ಗ-ಡೆ-ಯನ್ನು
ಬಿಡು-ಗ-ಡೆ-ಯಲ್ಲ
ಬಿಡು-ಗ-ಡೆ-ಯಾ-ಗ-ಬೇ-ಕಾ-ದರೂ
ಬಿಡು-ಗ-ಡೆ-ಯಾಗಿ
ಬಿಡು-ಗ-ಡೆ-ಯಾದ
ಬಿಡುತ್ತ
ಬಿಡು-ತ್ತದೆ
ಬಿಡು-ತ್ತವೆ
ಬಿಡುತ್ತಾ
ಬಿಡು-ತ್ತಾನೆ
ಬಿಡು-ತ್ತಾ-ನೆಯೆ
ಬಿಡು-ತ್ತಾ-ನೆಯೊ
ಬಿಡು-ತ್ತಾ-ನೆಯೋ
ಬಿಡು-ತ್ತಾರೆ
ಬಿಡುತ್ತಿ
ಬಿಡು-ತ್ತಿಲ್ಲ
ಬಿಡು-ತ್ತೇನೆ
ಬಿಡು-ತ್ತೇವೆ
ಬಿಡು-ತ್ತೇ-ವೆಯೊ
ಬಿಡುವ
ಬಿಡು-ವಂ-ತಹ
ಬಿಡು-ವಂ-ತಿಲ್ಲ
ಬಿಡು-ವಂತೆ
ಬಿಡು-ವನು
ಬಿಡು-ವನೆ
ಬಿಡು-ವನೊ
ಬಿಡು-ವನೋ
ಬಿಡು-ವರು
ಬಿಡು-ವ-ರೇನೋ
ಬಿಡು-ವ-ವ-ನಲ್ಲ
ಬಿಡು-ವ-ವನೂ
ಬಿಡು-ವಾಗ
ಬಿಡು-ವಾ-ಗಲೂ
ಬಿಡು-ವಾ-ಗಲೆ
ಬಿಡು-ವಿ-ಲ್ಲದೆ
ಬಿಡುವು
ಬಿಡು-ವು-ದ-ಕ್ಕಿಂತ
ಬಿಡು-ವು-ದಕ್ಕೂ
ಬಿಡು-ವು-ದಕ್ಕೆ
ಬಿಡು-ವುದನ್ನು
ಬಿಡು-ವು-ದ-ರಲ್ಲಿ
ಬಿಡು-ವು-ದಲ್ಲ
ಬಿಡು-ವು-ದಿಲ್ಲ
ಬಿಡು-ವು-ದಿ-ಲ್ಲ-ವಲ್ಲ
ಬಿಡು-ವು-ದಿ-ಲ್ಲವೊ
ಬಿಡು-ವುದು
ಬಿಡು-ವುದೂ
ಬಿಡು-ವುದೆ
ಬಿಡು-ವುದೇ
ಬಿಡು-ವುದೊ
ಬಿಡು-ವುವು
ಬಿಡು-ವುವೇ
ಬಿಡು-ವೆನು
ಬಿಡು-ವೆವು
ಬಿಡು-ವೆವೋ
ಬಿಡೋಣ
ಬಿತ್ತದೆ
ಬಿತ್ತ-ಬೇ-ಕಾ-ಗಿಲ್ಲ
ಬಿತ್ತ-ಬೇಕು
ಬಿತ್ತಲು
ಬಿತ್ತಿ
ಬಿತ್ತಿದ
ಬಿತ್ತಿ-ದಂತೆ
ಬಿತ್ತಿ-ದರೆ
ಬಿತ್ತಿ-ದಾಗ
ಬಿತ್ತಿದೆ
ಬಿತ್ತಿ-ದ್ದನ್ನು
ಬಿತ್ತಿದ್ದು
ಬಿತ್ತಿ-ದ್ದೆವೋ
ಬಿತ್ತಿ-ರು-ವನು
ಬಿತ್ತಿ-ರು-ವರೊ
ಬಿತ್ತಿ-ರು-ವರೋ
ಬಿತ್ತಿ-ರು-ವುದೇ
ಬಿತ್ತಿ-ರು-ವೆವೊ
ಬಿತ್ತು
ಬಿತ್ತು-ತ್ತಾನೆ
ಬಿತ್ತು-ವನು
ಬಿತ್ತು-ವು-ದಿಲ್ಲ
ಬಿತ್ತು-ವುದು
ಬಿತ್ತು-ವುದೊ
ಬಿದ-ರನ್ನು
ಬಿದಿ-ರನ್ನು
ಬಿದಿರು
ಬಿದ್ದ
ಬಿದ್ದಂ-ತಿದೆ
ಬಿದ್ದಂತೆ
ಬಿದ್ದದ್ದು
ಬಿದ್ದ-ದ್ದೆಲ್ಲಾ
ಬಿದ್ದ-ಮೇಲೆ
ಬಿದ್ದರೂ
ಬಿದ್ದರೆ
ಬಿದ್ದ-ವ-ನಿಗೆ
ಬಿದ್ದ-ವನು
ಬಿದ್ದಾಗ
ಬಿದ್ದಾ-ಗಲೇ
ಬಿದ್ದಾದ
ಬಿದ್ದಾ-ದ-ಮೇ-ಲೆಯೇ
ಬಿದ್ದಿತ್ತು
ಬಿದ್ದಿದೆ
ಬಿದ್ದಿ-ದೆಯೆ
ಬಿದ್ದಿದ್ದ
ಬಿದ್ದಿ-ದ್ದಾರೆ
ಬಿದ್ದಿ-ರ-ಬ-ಹುದು
ಬಿದ್ದಿ-ರ-ಬಾ-ರದು
ಬಿದ್ದಿ-ರಲಿ
ಬಿದ್ದಿರು
ಬಿದ್ದಿ-ರುವ
ಬಿದ್ದಿ-ರು-ವರು
ಬಿದ್ದಿ-ರು-ವ-ರೆಂ-ಬ-ದನ್ನು
ಬಿದ್ದಿ-ರು-ವು-ದಕ್ಕೆ
ಬಿದ್ದಿ-ರು-ವು-ದಿಲ್ಲ
ಬಿದ್ದಿ-ರು-ವುದು
ಬಿದ್ದಿ-ರು-ವೆವು
ಬಿದ್ದಿ-ರು-ವೆವೋ
ಬಿದ್ದಿಲ್ಲ
ಬಿದ್ದಿವೆ
ಬಿದ್ದು
ಬಿದ್ದು-ದ-ನ್ನೆಲ್ಲ
ಬಿದ್ದು-ಬಿ-ಡು-ವು-ದೇನೋ
ಬಿದ್ದು-ಹೋ-ಗ-ಬೇ-ಕಾ-ದರೆ
ಬಿದ್ದು-ಹೋ-ಗಿದೆ
ಬಿದ್ದು-ಹೋ-ಗಿ-ರು-ವಂತೆ
ಬಿದ್ದು-ಹೋ-ಗುವ
ಬಿದ್ದು-ಹೋ-ಗು-ವುದು
ಬಿದ್ದು-ಹೋ-ಗು-ವುವು
ಬಿದ್ದೊ-ಡನೆ
ಬಿನ್ನ-ವ-ತ್ತ-ಳೆ-ಗಳನ್ನು
ಬಿಭ-ರ್ತ್ಯ-ವ್ಯಯ
ಬಿರಿ-ಯು-ವಂತೆ
ಬಿರುಕು
ಬಿರು-ಕು-ಬಿಟ್ಟು
ಬಿರು-ಗಾಳಿ
ಬಿರು-ಗಾ-ಳಿಗೆ
ಬಿರು-ಗಾ-ಳಿಯ
ಬಿರು-ಗಾ-ಳಿ-ಯ-ನ್ನಾ-ದರೂ
ಬಿರು-ಗಾ-ಳಿ-ಯಿಂದ
ಬಿರು-ಗಾ-ಳಿ-ಯೊಂದು
ಬಿರು-ದನ್ನು
ಬಿರುದು
ಬಿರು-ಸಾದ
ಬಿಲ
ಬಿಲದ
ಬಿಲ-ದಲ್ಲಿ
ಬಿಲ-ದಿಂದ
ಬಿಲ-ವನ್ನು
ಬಿಲ್ಗಾ-ರರು
ಬಿಲ್ಲನ್ನು
ಬಿಲ್ಲಾ-ಳ-ನ್ನಾಗಿ
ಬಿಲ್ಲಿನ
ಬಿಲ್ಲಿ-ನಿಂದ
ಬಿಲ್ಲು
ಬಿಲ್ಲು-ಗಳನ್ನು
ಬಿಲ್ಲು-ಗಾರ
ಬಿಲ್ಲು-ಗಾ-ರ-ನ-ನ್ನಾಗಿ
ಬಿಲ್ಲು-ಗಾ-ರ-ನಾದ
ಬಿಲ್ಲು-ಗಾ-ರ-ನಾ-ದಾಗ
ಬಿಲ್ಲು-ಬಾ-ಣ-ಗಳನ್ನು
ಬಿಲ್ವ
ಬಿಲ್ವ-ಪ-ತ್ರೆಯೋ
ಬಿಳ-ಲು-ಗಳು
ಬಿಳ-ಲು-ಗ-ಳೆಲ್ಲಾ
ಬಿಳಿ
ಬಿಳಿಯ
ಬಿಳಿ-ಯಾ-ಗ-ಬೇ-ಕಾ-ಗಿಲ್ಲ
ಬಿಸಾ-ಡ-ಬ-ಲ್ಲುದು
ಬಿಸಾ-ಡಲು
ಬಿಸಾ-ಡಿ-ದರೂ
ಬಿಸಾ-ಡು-ತ್ತಿ-ದ್ದರು
ಬಿಸಾ-ಡು-ವು-ದಕ್ಕೂ
ಬಿಸಾ-ಡು-ವು-ದಕ್ಕೆ
ಬಿಸಿ
ಬಿಸಿ-ನೀ-ರಿ-ನ-ಲ್ಲಿಟ್ಟು
ಬಿಸಿ-ಬಿ-ಸಿ-ಯಾ-ಗಿ-ರು-ವುದನ್ನು
ಬಿಸಿ-ಮಂ-ಜಿ-ನಂತೆ
ಬಿಸಿ-ಯನ್ನು
ಬಿಸಿ-ಯಾ-ಗಲಿ
ಬಿಸಿ-ಯಾಗಿ
ಬಿಸಿ-ಯಾ-ಗಿ-ದ್ದರೆ
ಬಿಸಿ-ಯಾ-ಗಿ-ರ-ಬೇಕು
ಬಿಸಿ-ಯಾ-ಗಿ-ರು-ವುದೆ
ಬಿಸಿಯೂ
ಬಿಸಿ-ಲ-ಲ್ಲದೆ
ಬಿಸಿ-ಲಿ-ನಂತೆ
ಬಿಸಿ-ಲಿ-ನಲ್ಲಿ
ಬಿಸಿಲು
ಬಿಸಿ-ಲು-ಕಾ-ಲ-ದಲ್ಲಿ
ಬಿಸಿ-ಲು-ಕಾ-ಲ-ವೆ-ನ್ನದೆ
ಬಿಸಿ-ಲು-ಗಾ-ಲ-ವೆ-ನ್ನದೆ
ಬಿಸಿಲೂ
ಬಿಸಿ-ಲೆಂ-ದರೆ
ಬಿಸಿಲೇ
ಬಿಸಿ-ಹಾ-ಲನ್ನು
ಬಿಸು-ಟರೂ
ಬಿಸು-ಟಿ-ರುವ
ಬಿಸು-ಡ-ಬೇ-ಕಾ-ಗಿದೆ
ಬಿಸು-ಡ-ಬೇಕು
ಬಿಸು-ಡು-ವನು
ಬೀಗ
ಬೀಗದ
ಬೀಗ-ಹಾಕಿ
ಬೀಜ
ಬೀಜಂ
ಬೀಜಕ್ಕೆ
ಬೀಜ-ಗಳ
ಬೀಜ-ಗಳನ್ನು
ಬೀಜ-ಗ-ಳಿ-ವೆಯೋ
ಬೀಜ-ಗಳು
ಬೀಜ-ಗಳೇ
ಬೀಜ-ದಂತೆ
ಬೀಜ-ದಲ್ಲಿ
ಬೀಜ-ದ-ಲ್ಲಿ-ರುವ
ಬೀಜ-ದ-ಲ್ಲಿ-ರು-ವುದನ್ನು
ಬೀಜ-ಪ್ರದಃ
ಬೀಜ-ಪ್ರ-ದ-ನಾದ
ಬೀಜ-ಮ-ವ್ಯ-ಯಮ್
ಬೀಜ-ರೂ-ಪ-ದಲ್ಲಿ
ಬೀಜ-ರೂ-ಪ-ದ-ಲ್ಲಿದೆ
ಬೀಜ-ರೂ-ಪ-ದ-ಲ್ಲಿ-ರು-ತ್ತದೆ
ಬೀಜ-ರೂ-ಪ-ದ-ಲ್ಲಿವೆ
ಬೀಜ-ರೂ-ಪ-ದಿಂದ
ಬೀಜ-ವನ್ನು
ಬೀಜ-ವನ್ನೆ
ಬೀಜ-ವನ್ನೇ
ಬೀಜ-ವಾ-ಗು-ವುದು
ಬೀಜ-ವಾ-ದರೆ
ಬೀಜ-ವಿ-ರು-ವುದು
ಬೀಜವೂ
ಬೀಜ-ವೆಲ್ಲ
ಬೀಜ-ಸ್ವ-ರೂ-ಪ-ನಾ-ಗಿ-ರು-ವನು
ಬೀಡ-ಬೇಕು
ಬೀಡಿ
ಬೀಡಿಗೆ
ಬೀಡು
ಬೀದಿ-ಯಲ್ಲಿ
ಬೀರ-ಬ-ಲ್ಲರು
ಬೀರ-ಬೇ-ಕಾ-ದರೆ
ಬೀರ-ಬೇಕು
ಬೀರ-ಬೇಕೇ
ಬೀರಲು
ಬೀರಿ
ಬೀರಿ-ದರೂ
ಬೀರಿ-ದರೆ
ಬೀರಿ-ದಾಗ
ಬೀರಿಯೆ
ಬೀರಿಯೇ
ಬೀರಿ-ರು-ವನು
ಬೀರು
ಬೀರು-ತ್ತದೆ
ಬೀರು-ತ್ತವೆ
ಬೀರು-ತ್ತಿದೆ
ಬೀರು-ತ್ತಿ-ದ್ದನು
ಬೀರು-ತ್ತಿ-ರುವ
ಬೀರು-ತ್ತಿ-ರು-ವುದು
ಬೀರು-ತ್ತಿ-ರು-ವುವು
ಬೀರು-ತ್ತಿವೆ
ಬೀರುವ
ಬೀರು-ವಂತೆ
ಬೀರು-ವನು
ಬೀರು-ವುದು
ಬೀರು-ವುವು
ಬೀಳ-ಕೂ-ಡದು
ಬೀಳದ
ಬೀಳ-ದಂತೆ
ಬೀಳ-ದ-ವನು
ಬೀಳ-ದಿ-ರುವ
ಬೀಳ-ದಿ-ರು-ವುದು
ಬೀಳದೆ
ಬೀಳದೇ
ಬೀಳನು
ಬೀಳ-ಬ-ಲ್ಲದು
ಬೀಳ-ಬ-ಹುದು
ಬೀಳ-ಬಾ-ರದು
ಬೀಳ-ಬೇ-ಕಾ-ಗಿದೆ
ಬೀಳ-ಬೇ-ಕಾ-ಗು-ವುದು
ಬೀಳ-ಬೇ-ಕಾ-ದರೆ
ಬೀಳ-ಬೇಕು
ಬೀಳ-ಲಾ-ರದು
ಬೀಳಲಿ
ಬೀಳಲು
ಬೀಳಿ-ಸದೆ
ಬೀಳಿ-ಸು-ವು-ದಕ್ಕೆ
ಬೀಳಿ-ಸು-ವುದು
ಬೀಳಿ-ಸು-ವು-ದೆಲ್ಲ
ಬೀಳು
ಬೀಳು-ಕೊ-ಡು-ವು-ದಿಲ್ಲ
ಬೀಳು-ಗ-ಳಿವೆ
ಬೀಳುತ್ತ
ಬೀಳು-ತ್ತದೆ
ಬೀಳು-ತ್ತಲೇ
ಬೀಳು-ತ್ತವೆ
ಬೀಳುತ್ತಾ
ಬೀಳು-ತ್ತಾನೆ
ಬೀಳು-ತ್ತಾರೆ
ಬೀಳು-ತ್ತಿದೆ
ಬೀಳು-ತ್ತಿ-ದ್ದರೂ
ಬೀಳು-ತ್ತಿ-ದ್ದರೆ
ಬೀಳು-ತ್ತಿ-ರ-ಬೇಕು
ಬೀಳು-ತ್ತಿ-ರ-ಲಿಲ್ಲ
ಬೀಳು-ತ್ತಿ-ರು-ತ್ತದೆ
ಬೀಳು-ತ್ತಿ-ರು-ತ್ತವೆ
ಬೀಳು-ತ್ತಿ-ರುವ
ಬೀಳು-ತ್ತಿ-ರು-ವಾಗ
ಬೀಳು-ತ್ತಿ-ರು-ವುದು
ಬೀಳು-ತ್ತಿ-ರು-ವು-ದೊಂದೇ
ಬೀಳು-ತ್ತಿ-ರು-ವುವು
ಬೀಳು-ತ್ತಿವೆ
ಬೀಳು-ತ್ತೇನೆ
ಬೀಳು-ತ್ತೇವೆ
ಬೀಳು-ತ್ತೇ-ವೆಯೊ
ಬೀಳು-ತ್ತೇ-ವೆಯೋ
ಬೀಳುವ
ಬೀಳು-ವಂ-ತಹ
ಬೀಳು-ವಂತೆ
ಬೀಳು-ವನು
ಬೀಳು-ವರು
ಬೀಳು-ವ-ವ-ನಲ್ಲ
ಬೀಳು-ವಷ್ಟು
ಬೀಳು-ವಾಗ
ಬೀಳುವು
ಬೀಳು-ವುದ
ಬೀಳು-ವು-ದಕ್ಕೆ
ಬೀಳು-ವು-ದ-ಕ್ಕೆಲ್ಲ
ಬೀಳು-ವು-ದ-ನ್ನೆಲ್ಲ
ಬೀಳು-ವು-ದಲ್ಲ
ಬೀಳು-ವು-ದಿಲ್ಲ
ಬೀಳು-ವು-ದಿ-ಲ್ಲವೊ
ಬೀಳು-ವು-ದಿ-ಲ್ಲವೋ
ಬೀಳು-ವುದು
ಬೀಳು-ವುದೊ
ಬೀಳು-ವುದೋ
ಬೀಳು-ವುವು
ಬೀಳು-ವುವೊ
ಬೀಳು-ವುವೋ
ಬೀಳುವೆ
ಬೀಳು-ವೆನೋ
ಬೀಳು-ವೆವು
ಬೀಳ್ಕೊಡು
ಬೀಸ-ಣಿ-ಗೆ-ಯನ್ನೊ
ಬೀಸದ
ಬೀಸ-ದ-ಕಡೆ
ಬೀಸ-ಬ-ಹುದು
ಬೀಸಲಿ
ಬೀಸಿ
ಬೀಸಿತು
ಬೀಸಿ-ದರೂ
ಬೀಸಿ-ದರೆ
ಬೀಸಿ-ದೊ-ಡ-ನೆಯೆ
ಬೀಸು-ತ್ತಿದೆ
ಬೀಸು-ತ್ತಿ-ದ್ದರೆ
ಬೀಸು-ತ್ತಿ-ರುವ
ಬೀಸುವ
ಬೀಸು-ವ-ವನು
ಬೀಸು-ವು-ದ-ವ-ನಿಂದ
ಬೀಸು-ವುದು
ಬುಗು-ರಿ-ಯನ್ನೋ
ಬುಗ್ಗೆ
ಬುಟ್ಟಿ
ಬುಟ್ಟಿ-ಯಲ್ಲಿ
ಬುಡದ
ಬುಡ-ದಲ್ಲಿ
ಬುಡ-ವುಳ್ಳ
ಬುಡ್ಡಿ
ಬುಡ್ಡಿ-ಗಳನ್ನು
ಬುಡ್ಡಿ-ಗ-ಳಿಗೆ
ಬುಡ್ಡಿಯ
ಬುಡ್ಡಿ-ಯಲ್ಲಿ
ಬುಡ್ಡಿ-ಯೊ-ಳಗೆ
ಬುತ್ತಿ
ಬುತ್ತಿ-ಯಲ್ಲಿ
ಬುತ್ತಿ-ಯಾಗಿ
ಬುತ್ತಿ-ಯಾ-ಗು-ವುದು
ಬುದ್ಧ
ಬುದ್ಧನ
ಬುದ್ಧ-ನಂ-ತಹ
ಬುದ್ಧ-ನನ್ನು
ಬುದ್ಧ-ನಿ-ಗಿಂತ
ಬುದ್ಧ-ಯೋ-ಽವ್ಯ-ವ-ಸಾ-ಯಿ-ನಾಮ್
ಬುದ್ಧಿ
ಬುದ್ಧಿಂ
ಬುದ್ಧಿಃ
ಬುದ್ಧಿ-ಗಳ
ಬುದ್ಧಿ-ಗಳನ್ನು
ಬುದ್ಧಿ-ಗ-ಳಿಗೆ
ಬುದ್ಧಿ-ಗ-ಳಿ-ರ-ಬೇಕು
ಬುದ್ಧಿ-ಗಳು
ಬುದ್ಧಿ-ಗಿಂತ
ಬುದ್ಧಿಗೆ
ಬುದ್ಧಿ-ಗ್ರಾಹ್ಯ
ಬುದ್ಧಿ-ಗ್ರಾ-ಹ್ಯವೂ
ಬುದ್ಧಿ-ಜೀ-ವಿಗೆ
ಬುದ್ಧಿ-ನಾಶ
ಬುದ್ಧಿ-ನಾ-ಶ-ದಿಂದ
ಬುದ್ಧಿ-ನಾ-ಶಾತ್
ಬುದ್ಧಿ-ಪ್ರ-ಧಾ-ನ-ವಾ-ದುದು
ಬುದ್ಧಿ-ಬ-ಲ-ದಿಂ-ದಲೆ
ಬುದ್ಧಿ-ಭೇದಂ
ಬುದ್ಧಿ-ಭೇ-ದ-ವನ್ನು
ಬುದ್ಧಿ-ಮಾನ್
ಬುದ್ಧಿ-ಮಾ-ಲಿನ್ಯ
ಬುದ್ಧಿಯ
ಬುದ್ಧಿ-ಯಂತೆ
ಬುದ್ಧಿ-ಯನ್ನು
ಬುದ್ಧಿ-ಯಲ್ಲಿ
ಬುದ್ಧಿ-ಯಾ-ಗಿ-ರ-ಬ-ಹುದು
ಬುದ್ಧಿ-ಯಾ-ಗು-ವುದು
ಬುದ್ಧಿ-ಯಾ-ದರೊ
ಬುದ್ಧಿ-ಯಿಂದ
ಬುದ್ಧಿ-ಯಿಂ-ದಲೇ
ಬುದ್ಧಿ-ಯಿಲ್ಲ
ಬುದ್ಧಿಯು
ಬುದ್ಧಿ-ಯುಕ್ತಾ
ಬುದ್ಧಿ-ಯುಕ್ತೋ
ಬುದ್ಧಿ-ಯು-ಳ್ಳ-ವನು
ಬುದ್ಧಿ-ಯು-ಳ್ಳ-ವರು
ಬುದ್ಧಿ-ಯೆಂಬ
ಬುದ್ಧಿ-ಯೆಲ್ಲಾ
ಬುದ್ಧಿಯೇ
ಬುದ್ಧಿ-ಯೋಗಂ
ಬುದ್ಧಿ-ಯೋ-ಗ-ಮು-ಪಾ-ಶ್ರಿತ್ಯ
ಬುದ್ಧಿ-ಯೋ-ಗ-ವನ್ನು
ಬುದ್ಧಿ-ಯೋ-ಗ-ವಾ-ಗಿಲ್ಲ
ಬುದ್ಧಿ-ಯೋ-ಗಾ-ದ್ಧ-ನಂ-ಜಯ
ಬುದ್ಧಿ-ರ-ಯು-ಕ್ತಸ್ಯ
ಬುದ್ಧಿ-ರ-ವ್ಯ-ಕ್ತ-ಮೇವ
ಬುದ್ಧಿ-ರ-ಸ್ಯಾ-ಧಿ-ಷ್ಠಾ-ನ-ಮು-ಚ್ಯತೇ
ಬುದ್ಧಿ-ರೇ-ಕೇಹ
ಬುದ್ಧಿ-ರೇವ
ಬುದ್ಧಿ-ರ್ಜ-ನಾ-ರ್ದನ
ಬುದ್ಧಿ-ರ್ಜ್ಞಾ-ನ-ಮ-ಸಂ-ಮೋಹಃ
ಬುದ್ಧಿ-ರ್ಬು-ದ್ಧಿ-ಮ-ತಾ-ಮಸ್ಮಿ
ಬುದ್ಧಿ-ರ್ಯಸ್ಯ
ಬುದ್ಧಿರ್ಯೋ
ಬುದ್ಧಿ-ರ್ಯೋಗೇ
ಬುದ್ಧಿ-ರ್ವ್ಯ-ತಿ-ತ-ರಿ-ಷ್ಯತಿ
ಬುದ್ಧಿ-ವಂತ
ಬುದ್ಧಿ-ವಂ-ತ-ದಡ್ಡ
ಬುದ್ಧಿ-ವಂ-ತ-ನಾ-ಗ-ಲಾರ
ಬುದ್ಧಿ-ವಂ-ತ-ನಾಗಿ
ಬುದ್ಧಿ-ವಂ-ತ-ನಾ-ಗು-ತ್ತಾನೆ
ಬುದ್ಧಿ-ವಂ-ತ-ನಾದ
ಬುದ್ಧಿ-ವಂ-ತ-ನಿಗೆ
ಬುದ್ಧಿ-ವಂ-ತನೂ
ಬುದ್ಧಿ-ವಂ-ತರ
ಬುದ್ಧಿ-ವಂ-ತ-ರ-ನ್ನಾಗಿ
ಬುದ್ಧಿ-ವಂ-ತ-ರಲ್ಲ
ಬುದ್ಧಿ-ವಂ-ತ-ರಲ್ಲಿ
ಬುದ್ಧಿ-ವಂ-ತ-ರಾ-ಗಿ-ದ್ದರೆ
ಬುದ್ಧಿ-ವಂ-ತ-ರಾದ
ಬುದ್ಧಿ-ವಂ-ತ-ರಿ-ರು-ವರು
ಬುದ್ಧಿ-ವಂ-ತರು
ಬುದ್ಧಿ-ವಂ-ತರೂ
ಬುದ್ಧಿ-ವಂ-ತಿ-ಕ-ಯಿಂದ
ಬುದ್ಧಿ-ವಂ-ತಿಕೆ
ಬುದ್ಧಿ-ವಂ-ತಿ-ಕೆ-ಯನ್ನು
ಬುದ್ಧಿ-ವಂ-ತಿ-ಕೆ-ಯ-ನ್ನೆಲ್ಲ
ಬುದ್ಧಿ-ವಂ-ತಿ-ಕೆಯೇ
ಬುದ್ಧಿ-ವಾದ
ಬುದ್ಧಿ-ವಾ-ದ-ವನ್ನು
ಬುದ್ಧಿ-ಶಕ್ತಿ
ಬುದ್ಧಿ-ಶ-ಕ್ತಿಯ
ಬುದ್ಧಿ-ಶ-ಕ್ತಿ-ಯನ್ನು
ಬುದ್ಧಿ-ಶ-ಕ್ತಿ-ಯಿಂದ
ಬುದ್ಧಿ-ಶ-ಕ್ತಿ-ಯೊಂದೇ
ಬುದ್ಧಿ-ಸಂ-ಯೋಗಂ
ಬುದ್ಧಿ-ಸ್ತದಾ
ಬುದ್ಧಿ-ಹೇ-ಳಿ-ಕೊಂಡು
ಬುದ್ಧೇಃ
ಬುದ್ಧೇ-ರ್ಭೇದಂ
ಬುದ್ಧೌ
ಬುದ್ಧ್ಯಾ
ಬುದ್ಧ್ವಾ
ಬುದ್ಬು-ದ-ಗಳು
ಬುಧಃ
ಬುಧಾ
ಬುಧಾಃ
ಬುರುಡೆ
ಬುಲೆ-ಟ್ಟಿಗೇ
ಬುಸು-ಗು-ಟ್ಟು-ವುದು
ಬೂದಿ
ಬೂದಿ-ಪಾ-ಲಾ-ಗು-ವುದು
ಬೂದಿ-ಯನ್ನು
ಬೂದಿ-ಯಾಗಿ
ಬೂದಿ-ಯಾ-ಗಿವೆ
ಬೂದಿ-ಯಾ-ಗುವ
ಬೂದಿ-ಯಿಂದ
ಬೂದು-ಗ-ನ್ನ-ಡಿಯ
ಬೃಂದಾ-ವ-ನದ
ಬೃಂದಾ-ವ-ನ-ದಲ್ಲಿ
ಬೃಹತ್
ಬೃಹ-ತ್ತಾ-ಗಿ-ರು-ವುದು
ಬೃಹ-ತ್ತಾದ
ಬೃಹ-ತ್ಸಾಮ
ಬೃಹ-ದಾ-ಕಾ-ರ-ವನ್ನು
ಬೃಹ-ದಾ-ರ-ಣ್ಯಕ
ಬೃಹ-ಸ್ಪತಿ
ಬೃಹ-ಸ್ಪ-ತಿಮ್
ಬೆಂಕಿ
ಬೆಂಕಿಗೆ
ಬೆಂಕಿಯ
ಬೆಂಕಿ-ಯಂತೆ
ಬೆಂಕಿ-ಯ-ನ್ನಾಗಿ
ಬೆಂಕಿ-ಯನ್ನು
ಬೆಂಕಿ-ಯ-ಮೇಲೆ
ಬೆಂಕಿ-ಯಲ್ಲಿ
ಬೆಂಕಿ-ಯಾ-ದರೋ
ಬೆಂಕಿ-ಯಿಂದ
ಬೆಂಕಿ-ಯು-ರಿಸಿ
ಬೆಂಕಿಯೇ
ಬೆಂಕಿ-ಯೊ-ಳಗೆ
ಬೆಂಕಿ-ಶೇ-ಷ-ಗಳನ್ನು
ಬೆಂಗ-ಳೂ-ರನ್ನು
ಬೆಂಗ-ಳೂ-ರಿಗೆ
ಬೆಂದ
ಬೆಂದರೆ
ಬೆಂದಾದ
ಬೆಂದಿ-ರ-ಬೇಕು
ಬೆಂದಿ-ರು-ವೆವು
ಬೆಂದಿ-ಲ್ಲವೋ
ಬೆಂದಿವೆ
ಬೆಂದು
ಬೆಂದು-ಹೋ-ಗು-ವುದೊ
ಬೆಂಬ-ಲ-ವಾಗಿ
ಬೆಂಬ-ಲಿ-ಗ-ರಾಗಿ
ಬೆಂಬ-ಲಿ-ಗ-ರಾ-ಗಿ-ರು-ವಾಗ
ಬೆಂಬಿ-ಡದೆ
ಬೆಕ್ಕು
ಬೆಟ್ಟ
ಬೆಟ್ಟಕ್ಕೆ
ಬೆಟ್ಟ-ಗಳು
ಬೆಟ್ಟದ
ಬೆಟ್ಟ-ದ-ಮೇಲೆ
ಬೆಟ್ಟ-ದಲ್ಲಿ
ಬೆಟ್ಟ-ವಾಗ
ಬೆಡ-ಗಿನ
ಬೆಣೆ-ಯನ್ನು
ಬೆಣ್ಣೆ
ಬೆಣ್ಣೆಗೂ
ಬೆಣ್ಣೆ-ಯಂತೆ
ಬೆಣ್ಣೆ-ಯನ್ನು
ಬೆಣ್ಣೆ-ಯಲ್ಲಿ
ಬೆಣ್ಣೆ-ಯ-ಲ್ಲಿ-ರುವ
ಬೆತ್ತ-ಲೆ-ಯಾಗಿ
ಬೆದ-ರಿ-ಕೆಗೆ
ಬೆನ್ನ
ಬೆನ್ನಟ್ಟಿ
ಬೆನ್ನ-ಟ್ಟಿ-ಕೊಂಡು
ಬೆನ್ನ-ಟ್ಟು-ವುದನ್ನು
ಬೆನ್ನನ್ನು
ಬೆನ್ನ-ಮೇ-ಲಿ-ರುವ
ಬೆನ್ನಿನ
ಬೆನ್ನಿ-ನ-ಮೇ-ಲೆಯೇ
ಬೆನ್ನು
ಬೆನ್ನು-ಮೂಳೆ
ಬೆರಕೆ
ಬೆರ-ಗಾ-ಗಿ-ಹೋ-ದನೆ
ಬೆರ-ಗಾ-ಗು-ವರು
ಬೆರ-ತಿ-ರು-ವುದು
ಬೆರತು
ಬೆರ-ತು-ಹೋಗಿ
ಬೆರ-ತು-ಹೋ-ಗಿ-ದೆಯೋ
ಬೆರ-ಳನ್ನೇ
ಬೆರ-ಳ-ಸ-ನ್ನೆಗೆ
ಬೆರ-ಳಿ-ನಂತೆ
ಬೆರ-ಳಿಲ್ಲ
ಬೆರಳು
ಬೆರ-ಳು-ಗಳಲ್ಲಿ
ಬೆರ-ಳು-ಗಳಿಂದ
ಬೆರ-ಸಿ-ದರೆ
ಬೆರ-ಸಿ-ರು-ವುದನ್ನು
ಬೆರಸು
ಬೆರ-ಸು-ತ್ತಾನೆ
ಬೆರ-ಸು-ವನು
ಬೆರಿಕೆ
ಬೆರೆತ
ಬೆರೆ-ತಿದೆ
ಬೆರೆ-ತಿ-ದೆಯೆ
ಬೆರೆ-ತಿ-ದ್ದರೆ
ಬೆರೆ-ತಿ-ರು-ವುದು
ಬೆರೆ-ತಿವೆ
ಬೆರೆ-ತು-ಕೊಂ-ಡಿಲ್ಲ
ಬೆರೆ-ತು-ಹೋ-ಗಿದೆ
ಬೆರೆ-ತು-ಹೋ-ದಂತೆ
ಬೆರೆ-ಯ-ದಂತೆ
ಬೆರೆ-ಯದೆ
ಬೆರೆ-ಯ-ಬೇ-ಕಾ-ಗಿದೆ
ಬೆರೆ-ಯ-ಬೇ-ಕಾ-ಗು-ವುದು
ಬೆರೆ-ಯು-ತ್ತಿ-ರು-ವುದು
ಬೆರೆಸಿ
ಬೆರೆ-ಸಿ-ರ-ಬೇಕು
ಬೆರೆ-ಸು-ತ್ತಾರೆ
ಬೆರೆ-ಸು-ವರು
ಬೆಲೆ
ಬೆಲೆ-ಕ-ಟ್ಟು-ವಂತೆ
ಬೆಲೆ-ಕೊಟ್ಟು
ಬೆಲೆಗೂ
ಬೆಲೆ-ಬಾ-ಳುವ
ಬೆಲೆ-ಯನ್ನು
ಬೆಲೆ-ಯನ್ನೇ
ಬೆಲೆ-ಯಾಗಿ
ಬೆಲೆ-ಯಿಲ್ಲ
ಬೆಲೆ-ಯಿ-ಲ್ಲ-ದೇ-ಹೋ-ದರೂ
ಬೆಲೆಯೂ
ಬೆಲೆ-ಯೆಲ್ಲ
ಬೆಲೆಯೇ
ಬೆಲ್ಟಿ-ನಂತೆ
ಬೆಲ್ಲ
ಬೆಲ್ಲದ
ಬೆಲ್ಲ-ದ-ಚ್ಚು-ಗಳ
ಬೆಲ್ಲ-ವನ್ನು
ಬೆಲ್ಲ-ವಲ್ಲ
ಬೆಲ್ಲು
ಬೆಳ
ಬೆಳ-ಕನ್ನು
ಬೆಳ-ಕಾ-ಗ-ಬ-ಲ್ಲಂ-ತಹ
ಬೆಳ-ಕಾ-ಗ-ಬೇಕು
ಬೆಳ-ಕಾ-ಗಿ-ರ-ಬ-ಹುದು
ಬೆಳ-ಕಾ-ಗಿ-ರು-ವ-ವನೂ
ಬೆಳ-ಕಾ-ಗು-ವನು
ಬೆಳ-ಕಿ-ಗಾಗಿ
ಬೆಳ-ಕಿಗೂ
ಬೆಳ-ಕಿಗೆ
ಬೆಳ-ಕಿ-ದ್ದರೆ
ಬೆಳ-ಕಿನ
ಬೆಳ-ಕಿ-ನಂ-ತಲ್ಲ
ಬೆಳ-ಕಿ-ನಲ್ಲಿ
ಬೆಳ-ಕಿ-ನ-ಲ್ಲಿ-ರುವ
ಬೆಳ-ಕಿ-ನಿಂದ
ಬೆಳ-ಕಿ-ಲ್ಲವೊ
ಬೆಳಕು
ಬೆಳಕೇ
ಬೆಳ-ಗದೆ
ಬೆಳ-ಗ-ಬೇ-ಕಾ-ಗಿಲ್ಲ
ಬೆಳ-ಗ-ಬೇಕು
ಬೆಳ-ಗಲು
ಬೆಳ-ಗಾ-ದರೆ
ಬೆಳ-ಗಾ-ಯಿತು
ಬೆಳಗಿ
ಬೆಳ-ಗಿ-ದರೂ
ಬೆಳ-ಗಿ-ದರೆ
ಬೆಳ-ಗಿ-ದರೇ
ಬೆಳ-ಗಿ-ದಾಗ
ಬೆಳ-ಗಿ-ದಾ-ಗಲೂ
ಬೆಳ-ಗಿ-ನಿಂದ
ಬೆಳ-ಗಿ-ಸ-ಬ-ಹುದು
ಬೆಳ-ಗಿ-ಸಲು
ಬೆಳ-ಗಿ-ಸು-ತ್ತಿದೆ
ಬೆಳ-ಗಿ-ಸು-ತ್ತಿ-ರು-ವನು
ಬೆಳಗು
ಬೆಳ-ಗು-ತ್ತಾನೆ
ಬೆಳ-ಗು-ತ್ತಾ-ನೆಂದು
ಬೆಳ-ಗು-ತ್ತಿದೆ
ಬೆಳ-ಗು-ತ್ತಿ-ದೆಯೋ
ಬೆಳ-ಗು-ತ್ತಿದ್ದ
ಬೆಳ-ಗು-ತ್ತಿ-ದ್ದನೊ
ಬೆಳ-ಗು-ತ್ತಿ-ದ್ದರೂ
ಬೆಳ-ಗು-ತ್ತಿ-ದ್ದರೆ
ಬೆಳ-ಗು-ತ್ತಿ-ರ-ಬ-ಹುದು
ಬೆಳ-ಗು-ತ್ತಿ-ರುವ
ಬೆಳ-ಗು-ತ್ತಿ-ರು-ವನು
ಬೆಳ-ಗು-ತ್ತಿ-ರು-ವನೋ
ಬೆಳ-ಗು-ತ್ತಿ-ರು-ವಳು
ಬೆಳ-ಗು-ತ್ತಿ-ರು-ವುದ
ಬೆಳ-ಗು-ತ್ತಿ-ರು-ವುದನ್ನು
ಬೆಳ-ಗು-ತ್ತಿ-ರು-ವುದು
ಬೆಳ-ಗು-ತ್ತಿ-ರುವೆ
ಬೆಳ-ಗು-ತ್ತಿವೆ
ಬೆಳ-ಗು-ತ್ತಿ-ವೆ-ಯೇನೊ
ಬೆಳ-ಗು-ತ್ತೇವೆ
ಬೆಳ-ಗುವ
ಬೆಳ-ಗು-ವಂತೆ
ಬೆಳ-ಗು-ವನು
ಬೆಳ-ಗು-ವ-ವನು
ಬೆಳ-ಗು-ವ-ವನೂ
ಬೆಳ-ಗು-ವ-ವ-ರಾರೂ
ಬೆಳ-ಗು-ವಾಗ
ಬೆಳ-ಗು-ವುದನ್ನು
ಬೆಳ-ಗು-ವು-ದ-ರಲ್ಲಿ
ಬೆಳ-ಗು-ವು-ದ-ರಿಂದ
ಬೆಳ-ಗು-ವು-ದಿಲ್ಲ
ಬೆಳ-ಗು-ವುದು
ಬೆಳ-ಗು-ವುದೇ
ಬೆಳ-ಗು-ವುದೊ
ಬೆಳ-ಗು-ವು-ದೊಂದೇ
ಬೆಳ-ಗು-ವುದೋ
ಬೆಳ-ಗು-ವುವು
ಬೆಳ-ಗೆ-ದ್ದ-ಮೇಲೆ
ಬೆಳ-ಗ್ಗಿ-ನಿಂದ
ಬೆಳಗ್ಗೆ
ಬೆಳ-ಗ್ಗೆ-ದ್ದರೆ
ಬೆಳ-ವ-ಣಿ-ಗಾಗಿ
ಬೆಳ-ವ-ಣಿ-ಗೆಗೂ
ಬೆಳ-ವ-ಣಿ-ಗೆಗೆ
ಬೆಳ-ವ-ಣಿ-ಗೆಯ
ಬೆಳ-ವ-ಣಿ-ಗೆ-ಯನ್ನು
ಬೆಳ-ಸ-ಬ-ಹುದು
ಬೆಳ-ಸ-ಬೇಕು
ಬೆಳ-ಸಿ-ಕೊ-ಳ್ಳ-ಬೇಕು
ಬೆಳಿಗ್ಗೆ
ಬೆಳು-ಗು-ತ್ತಿ-ರು-ವಂತೆ
ಬೆಳು-ಗು-ತ್ತೇನೆ
ಬೆಳು-ಗು-ವನು
ಬೆಳು-ಗು-ವ-ವನೆ
ಬೆಳು-ಗು-ವ-ವನೇ
ಬೆಳೆ
ಬೆಳೆ-ಗಳನ್ನು
ಬೆಳೆಗೆ
ಬೆಳೆದ
ಬೆಳೆ-ದರೆ
ಬೆಳೆ-ದ-ವರು
ಬೆಳೆ-ದಾಗ
ಬೆಳೆ-ದಾ-ಗ-ದೇ-ವರೇ
ಬೆಳೆ-ದಿ-ರ-ಬೇಕು
ಬೆಳೆ-ದಿ-ರು-ವರು
ಬೆಳೆ-ದಿ-ರು-ವುದು
ಬೆಳೆ-ದಿ-ರು-ವೆವು
ಬೆಳೆ-ದಿಲ್ಲ
ಬೆಳೆದು
ಬೆಳೆ-ದು-ಕೊಂ-ಡಿ-ರುವ
ಬೆಳೆ-ದು-ಕೊಂಡು
ಬೆಳೆ-ದು-ದನ್ನು
ಬೆಳೆ-ದು-ದೆಲ್ಲ
ಬೆಳೆದೆ
ಬೆಳೆ-ದೆವು
ಬೆಳೆ-ಯಂತೆ
ಬೆಳೆ-ಯದೆ
ಬೆಳೆ-ಯನ್ನು
ಬೆಳೆ-ಯ-ಬ-ಹುದು
ಬೆಳೆ-ಯ-ಬೇ-ಕಾ-ಗಿದೆ
ಬೆಳೆ-ಯ-ಬೇ-ಕಾ-ದರೆ
ಬೆಳೆ-ಯ-ಬೇಕು
ಬೆಳೆ-ಯ-ಬೇ-ಕೆಂದು
ಬೆಳೆ-ಯಲು
ಬೆಳೆ-ಯಲ್ಲಿ
ಬೆಳೆ-ಯಾ-ಗು-ವುದು
ಬೆಳೆಯು
ಬೆಳೆ-ಯುತ್ತ
ಬೆಳೆ-ಯು-ತ್ತವೆ
ಬೆಳೆ-ಯುತ್ತಾ
ಬೆಳೆ-ಯು-ತ್ತಾನೆ
ಬೆಳೆ-ಯು-ತ್ತಾರೆ
ಬೆಳೆ-ಯು-ತ್ತಿದೆ
ಬೆಳೆ-ಯು-ತ್ತಿ-ದ್ದರೆ
ಬೆಳೆ-ಯು-ತ್ತಿ-ರು-ವಂ-ತಿದೆ
ಬೆಳೆ-ಯು-ತ್ತಿ-ರು-ವಾಗ
ಬೆಳೆ-ಯು-ತ್ತಿ-ರು-ವೆವು
ಬೆಳೆ-ಯು-ತ್ತಿಲ್ಲ
ಬೆಳೆ-ಯು-ತ್ತೇವೆ
ಬೆಳೆ-ಯು-ವನು
ಬೆಳೆ-ಯು-ವರು
ಬೆಳೆ-ಯು-ವು-ದಕ್ಕೂ
ಬೆಳೆ-ಯು-ವು-ದಕ್ಕೆ
ಬೆಳೆ-ಯು-ವುದನ್ನು
ಬೆಳೆ-ಯು-ವು-ದಿಲ್ಲ
ಬೆಳೆ-ಯು-ವುದು
ಬೆಳೆ-ಯು-ವುವು
ಬೆಳೆ-ಸ-ದ-ವರು
ಬೆಳೆ-ಸದೆ
ಬೆಳೆ-ಸ-ಬೇ-ಕಾ-ದರೆ
ಬೆಳೆ-ಸಿ-ಕೊಂ-ಡಂತೆ
ಬೆಳೆ-ಸಿ-ದರೆ
ಬೆಳೆ-ಸಿ-ದ-ವರು
ಬೆಳೆ-ಸಿ-ದ್ದಾ-ನೆಯೋ
ಬೆಳೆ-ಸು-ತ್ತಾನೆ
ಬೆಳೆ-ಸು-ವರು
ಬೆಳೆ-ಸು-ವರೊ
ಬೆಳೆ-ಸು-ವಳೋ
ಬೆಳೆ-ಸು-ವು-ದಕ್ಕೆ
ಬೆಳೆ-ಸು-ವು-ದ-ರಲ್ಲಿ
ಬೆಳೆ-ಸು-ವುದು
ಬೆಳೆ-ಸು-ವೆವು
ಬೆಳ್ಳುಳ್ಳಿ
ಬೆವ-ರನ್ನು
ಬೆವ-ರಿ-ನಲ್ಲಿ
ಬೆವರು
ಬೆಸ್
ಬೆಸ್ತ
ಬೆಸ್ತ-ರಲ್ಲಿ
ಬೇಕಲ್ಲ
ಬೇಕಾ-ಗದೆ
ಬೇಕಾಗಿ
ಬೇಕಾ-ಗಿತ್ತು
ಬೇಕಾ-ಗಿದೆ
ಬೇಕಾ-ಗಿ-ದ್ದನು
ಬೇಕಾ-ಗಿ-ರ-ಲಿಲ್ಲ
ಬೇಕಾ-ಗಿ-ರುವ
ಬೇಕಾ-ಗಿ-ರು-ವರೊ
ಬೇಕಾ-ಗಿ-ರು-ವ-ವರು
ಬೇಕಾ-ಗಿ-ರು-ವು-ದ-ರಿಂದ
ಬೇಕಾ-ಗಿ-ರು-ವು-ದ-ರಿಂ-ದಲೇ
ಬೇಕಾ-ಗಿ-ರು-ವು-ದಿಲ್ಲ
ಬೇಕಾ-ಗಿ-ರು-ವುದು
ಬೇಕಾ-ಗಿಲ್ಲ
ಬೇಕಾ-ಗಿ-ಲ್ಲ-ವಲ್ಲ
ಬೇಕಾ-ಗಿ-ಲ್ಲವೊ
ಬೇಕಾ-ಗಿವೆ
ಬೇಕಾ-ಗು-ತ್ತವೆ
ಬೇಕಾ-ಗುವ
ಬೇಕಾ-ಗು-ವು-ದಿಲ್ಲ
ಬೇಕಾ-ಗು-ವುದು
ಬೇಕಾ-ಗು-ವು-ದು-ಇನ್ನು
ಬೇಕಾ-ಗು-ವು-ದು-ಬ-ದ್ಧ-ಜೀ-ವಿ-ಗಳು
ಬೇಕಾ-ಗು-ವು-ದೆ-ಲ್ಲ-ವನ್ನೂ
ಬೇಕಾದ
ಬೇಕಾ-ದಂತೆ
ಬೇಕಾ-ದ-ರಂತೂ
ಬೇಕಾ-ದರೂ
ಬೇಕಾ-ದರೆ
ಬೇಕಾ-ದ-ವರು
ಬೇಕಾ-ದ-ಷ್ಟನ್ನು
ಬೇಕಾ-ದಷ್ಟು
ಬೇಕಾ-ದಾಗ
ಬೇಕಾದು
ಬೇಕಾ-ದುದ
ಬೇಕಾ-ದು-ದನ್ನು
ಬೇಕಾ-ದು-ದ-ನ್ನೆ-ಲ್ಲ-ವನ್ನೂ
ಬೇಕಾ-ದು-ದ-ನ್ನೆಲ್ಲಾ
ಬೇಕಾ-ದುದು
ಬೇಕಾ-ದು-ದೆಲ್ಲ
ಬೇಕಾ-ದು-ದೆಲ್ಲಾ
ಬೇಕಾ-ದು-ದೇನೊ
ಬೇಕಾದ್ದು
ಬೇಕಾ-ಬಿಟ್ಟಿ
ಬೇಕಾ-ಬಿ-ಟ್ಟಿ-ಯಾಗಿ
ಬೇಕಿಲ್ಲ
ಬೇಕು
ಬೇಕೆಂ-ತಲೆ
ಬೇಕೆಂ-ತಲ್ಲ
ಬೇಕೆಂ-ದಿರು
ಬೇಕೆಂ-ದಿ-ರುವೆ
ಬೇಕೆಂದು
ಬೇಕೆಂದೇ
ಬೇಕೆಂಬ
ಬೇಕೆಂ-ಬುದೇ
ಬೇಕೆ-ನಿ-ಸ-ಬ-ಹುದೊ
ಬೇಕೆ-ನಿ-ಸು-ವು-ದಿಲ್ಲ
ಬೇಕೆ-ನಿ-ಸು-ವುದೇ
ಬೇಕೊ
ಬೇಕೋ
ಬೇಗ
ಬೇಗ-ಬೇಗ
ಬೇಗಲೆ
ಬೇಜ-ರಾಗಿ
ಬೇಜಾ-ರಾಗಿ
ಬೇಜಾ-ರಾ-ಗು-ತ್ತದೆ
ಬೇಜಾ-ರಾ-ಗು-ವು-ದಿಲ್ಲ
ಬೇಜಾ-ರಾ-ಗು-ವುದು
ಬೇಜಾ-ರಾ-ದ-ಮೇಲೆ
ಬೇಜಾ-ರಾ-ದರೆ
ಬೇಜಾರು
ಬೇಟೆ
ಬೇಟೆ-ಗಾರ
ಬೇಟೆ-ಗಾ-ರನ
ಬೇಟೆ-ಗಾ-ರ-ನೊ-ಬ್ಬನ
ಬೇಟೆಗೆ
ಬೇಟೆ-ನಾಯಿ
ಬೇಟೆ-ನಾ-ಯಿ-ಯನ್ನು
ಬೇಟೆಯ
ಬೇಟೆ-ಯಾ-ಡು-ತ್ತಿ-ರು-ವಾಗ
ಬೇಡ
ಬೇಡ-ಎಂಬ
ಬೇಡದ
ಬೇಡ-ದಿ-ದ್ದರೆ
ಬೇಡ-ದು-ದ-ರಿಂದ
ಬೇಡ-ದುದು
ಬೇಡದೆ
ಬೇಡದೇ
ಬೇಡದ್ದು
ಬೇಡ-ಬ-ಹುದು
ಬೇಡಲು
ಬೇಡ-ವಾ-ಗಿ-ದ್ದೇವೆ
ಬೇಡ-ವಾ-ದರೂ
ಬೇಡ-ವಾ-ದರೆ
ಬೇಡ-ವಾ-ದಾಗ
ಬೇಡ-ವಾ-ದು-ದನ್ನು
ಬೇಡ-ವಾ-ದುದು
ಬೇಡ-ವಾ-ದುದೂ
ಬೇಡ-ವೆಂ-ದರೂ
ಬೇಡ-ವೆಂದು
ಬೇಡವೇ
ಬೇಡವೊ
ಬೇಡಿ
ಬೇಡಿ-ಕೆ-ಗಳನ್ನು
ಬೇಡಿ-ಕೆ-ಯನ್ನೂ
ಬೇಡಿ-ಕೊಂ-ಡಿದ್ದ
ಬೇಡಿ-ಕೊ-ಳ್ಳು-ತ್ತಾನೆ
ಬೇಡಿ-ಕೊ-ಳ್ಳು-ತ್ತಾರೆ
ಬೇಡಿದ
ಬೇಡಿ-ದರೆ
ಬೇಡಿ-ದಾಗ
ಬೇಡಿದ್ದು
ಬೇಡಿಯೇ
ಬೇಡುತ್ತಾ
ಬೇಡು-ತ್ತಾನೆ
ಬೇಡು-ತ್ತಾ-ನೆಯೋ
ಬೇಡು-ತ್ತಾರೆ
ಬೇಡು-ತ್ತಿ-ರು-ವನು
ಬೇಡು-ತ್ತಿ-ರು-ವರು
ಬೇಡು-ತ್ತಿ-ರು-ವುದು
ಬೇಡು-ತ್ತೇವೆ
ಬೇಡು-ತ್ತೇ-ವೆಯೊ
ಬೇಡು-ವನು
ಬೇಡು-ವರು
ಬೇಡು-ವ-ವರು
ಬೇಡು-ವಾಗ
ಬೇಡು-ವು-ದ-ಕ್ಕಿಂತ
ಬೇಡು-ವು-ದಕ್ಕೆ
ಬೇಡು-ವುದನ್ನು
ಬೇಡು-ವು-ದಿಲ್ಲ
ಬೇಡು-ವುದು
ಬೇಡು-ವುದೂ
ಬೇಡು-ವೆವು
ಬೇಧಿ
ಬೇಧಿಯೋ
ಬೇಯಲಿ
ಬೇಯಿ-ಸ-ಬೇಕು
ಬೇಯಿಸಿ
ಬೇಯಿ-ಸು-ತ್ತಾನೆ
ಬೇಯಿ-ಸು-ತ್ತಿ-ದ್ದರೆ
ಬೇಯಿ-ಸು-ತ್ತೇ-ವೆಯೊ
ಬೇಯು-ತ್ತಿ-ರುವ
ಬೇಯು-ವ-ವ-ರೆಗೆ
ಬೇಯು-ವುದೋ
ಬೇರನ್ನು
ಬೇರ-ನ್ನೆಲ್ಲಾ
ಬೇರಲ್ಲ
ಬೇರಾವ
ಬೇರಾ-ವು-ದಾ-ದರೂ
ಬೇರಾ-ವುದೂ
ಬೇರಿಗೆ
ಬೇರಿಲ್ಲ
ಬೇರು
ಬೇರು-ಗಳು
ಬೇರು-ಸ-ಹಿತ
ಬೇರೂರಿ
ಬೇರೂ-ರಿದೆ
ಬೇರೂ-ರಿ-ರುವ
ಬೇರೂ-ರು-ವುದು
ಬೇರೆ
ಬೇರೆ-ಕ-ಡೆಗೆ
ಬೇರೆ-ಬೇರೆ
ಬೇರೆ-ಬೇ-ರೆ-ಯಾಗಿ
ಬೇರೆ-ಮಾ-ಡು-ವೆವು
ಬೇರೆ-ಯಂತೆ
ಬೇರೆ-ಯಲ್ಲ
ಬೇರೆ-ಯ-ವರ
ಬೇರೆ-ಯ-ವ-ರನ್ನು
ಬೇರೆ-ಯ-ವರು
ಬೇರೆ-ಯಾಗಿ
ಬೇರೆ-ಯಾ-ಗಿದೆ
ಬೇರೆ-ಯಾ-ಗಿಯೇ
ಬೇರೆ-ಯಾ-ಗಿ-ರು-ವಂತೆ
ಬೇರೆ-ಯಾ-ಗಿ-ರು-ವನು
ಬೇರೆ-ಯಾ-ಗಿ-ರು-ವುದು
ಬೇರೆ-ಯಾ-ಗು-ವುದನ್ನು
ಬೇರೆ-ಯಾ-ಗು-ವುದು
ಬೇರೆ-ಯಾ-ಗು-ವುವು
ಬೇರೆ-ಯಾದ
ಬೇರೆ-ಯಾ-ದರೂ
ಬೇರೆಯೋ
ಬೇರೆ-ಲ್ಲಿ-ಯಾ-ದರೂ
ಬೇರೆಲ್ಲೊ
ಬೇರೆ-ಲ್ಲೊ-ಇತ್ತು
ಬೇರೇನೂ
ಬೇರೊಂ-ದನ್ನು
ಬೇರೊಂ-ದಿಲ್ಲ
ಬೇರೊಂದು
ಬೇರೊಬ್ಬ
ಬೇರ್ಪ-ಡಿ-ಸ-ಬ-ಹುದು
ಬೇರ್ಪ-ಡಿ-ಸ-ಬೇಕು
ಬೇರ್ಪ-ಡಿ-ಸಲು
ಬೇರ್ಪ-ಡಿಸಿ
ಬೇರ್ಪ-ಡಿ-ಸು-ತ್ತಾ-ನೆಯೊ
ಬೇರ್ಪ-ಡಿ-ಸು-ತ್ತಾರೆ
ಬೇರ್ಪ-ಡಿ-ಸುವ
ಬೇಲಿ
ಬೇಲಿಯ
ಬೇಲಿ-ಯನ್ನು
ಬೇಲಿ-ಯ-ಲ್ಲಿ-ರು-ವುದು
ಬೇಲಿ-ಯಿಂದ
ಬೇಲಿಯೋ
ಬೇವಿನ
ಬೇಸತ್ತು
ಬೇಸ-ರ-ವಾ-ಗು-ವು-ದಿಲ್ಲ
ಬೇಸ-ರ-ವಿಲ್ಲ
ಬೇಸಾಯ
ಬೇಸಿಗೆ
ಬೇಸಿ-ಗೆಯ
ಬೇಸೆಗೆ
ಬೈದರೂ
ಬೈದರೆ
ಬೈದು
ಬೈಬ-ಲ್ಲಿ-ನಲ್ಲಿ
ಬೈಲು
ಬೊಂಬೆಯ
ಬೊಂಬೆ-ಯಂತೆ
ಬೊಗ-ಳ-ಬೇಡ
ಬೊಗಳಿ
ಬೊಗ-ಳು-ತ್ತಿ-ರು-ತ್ತವೆ
ಬೊಗಸೆ
ಬೊಗುಳು
ಬೋದ್ಧವ್ಯಂ
ಬೋಧನೆ
ಬೋಧ-ನೆ-ಗಳನ್ನು
ಬೋಧ-ನೆ-ಗಳನ್ನೆಲ್ಲ
ಬೋಧ-ನೆ-ಗಳೂ
ಬೋಧ-ನೆಗೆ
ಬೋಧ-ನೆಯ
ಬೋಧ-ನೆ-ಯನ್ನು
ಬೋಧ-ನೆ-ಯಲ್ಲಿ
ಬೋಧ-ನೆ-ಯಿಂದ
ಬೋಧ-ನೆಯೂ
ಬೋಧ-ಯಂತಃ
ಬೋಧಿಸ
ಬೋಧಿ-ಸಲು
ಬೋಧಿಸಿ
ಬೋಧಿ-ಸಿದ
ಬೋಧಿ-ಸಿ-ದನು
ಬೋಧಿ-ಸಿ-ದರು
ಬೋಧಿ-ಸಿ-ದ-ವನು
ಬೋಧಿ-ಸಿದ್ದು
ಬೋಧಿ-ಸಿದ್ದೂ
ಬೋಧಿಸು
ಬೋಧಿ-ಸುತ್ತ
ಬೋಧಿ-ಸು-ತ್ತಾನೆ
ಬೋಧಿ-ಸು-ತ್ತಿ-ರ-ಬ-ಹುದು
ಬೋಧಿ-ಸು-ತ್ತಿವೆ
ಬೋಧಿ-ಸು-ತ್ತೇವೆ
ಬೋಧಿ-ಸು-ವನು
ಬೋಧಿ-ಸು-ವ-ವನು
ಬೋಧಿ-ಸು-ವ-ವರು
ಬೋಧಿ-ಸು-ವು-ದಕ್ಕೆ
ಬೋಧಿ-ಸು-ವುದು
ಬೋಧಿ-ಸು-ವುದೇ
ಬೋಧೆ-ಯ-ಲ್ಲೆಲ್ಲ
ಬೋನಿ-ನಿಂದ
ಬೋನಿ-ನೊ-ಳಗೆ
ಬೋಲ್ಟೊ
ಬೋಸ್
ಬೌದ್ಧ-ಗ್ರಂ-ಥ-ವಾದ
ಬೌದ್ಧ-ರಲ್ಲಿ
ಬೌದ್ಧಿ-ಕ-ವಾಗಿ
ಬೌದ್ಧಿ-ಕ-ವಾ-ಗಿ-ರು-ವುದು
ಬ್ಬರು
ಬ್ಯಾಂಕಿ-ನಲ್ಲಿ
ಬ್ಯಾಂಕಿ-ನ-ಲ್ಲಿ-ಟ್ಟಂತೆ
ಬ್ಯಾಂಕಿ-ನ-ಲ್ಲಿ-ರುವ
ಬ್ಯಾಂಕೇ
ಬ್ಯಾಂಡನ್ನು
ಬ್ರವೀಮಿ
ಬ್ರವೀಷಿ
ಬ್ರಹ್ಮ
ಬ್ರಹ್ಮ-ಕರ್ಮ
ಬ್ರಹ್ಮ-ಕ-ರ್ಮ-ಸ-ಮಾ-ಧಿನಾ
ಬ್ರಹ್ಮಕ್ಕೂ
ಬ್ರಹ್ಮಕ್ಕೆ
ಬ್ರಹ್ಮ-ಚರ್ಯ
ಬ್ರಹ್ಮ-ಚರ್ಯಂ
ಬ್ರಹ್ಮ-ಚ-ರ್ಯ-ಪಾ-ಲನೆ
ಬ್ರಹ್ಮ-ಚ-ರ್ಯ-ಮ-ಹಿಂಸಾ
ಬ್ರಹ್ಮ-ಚಾರಿ
ಬ್ರಹ್ಮ-ಚಾ-ರಿ-ವ್ರ-ತ-ದಲ್ಲಿ
ಬ್ರಹ್ಮ-ಚಿಂ-ತ-ನೆ-ಯನ್ನು
ಬ್ರಹ್ಮ-ಜ್ಞಾ-ನ-ದಲ್ಲಿ
ಬ್ರಹ್ಮ-ಜ್ಞಾ-ನ-ವನ್ನು
ಬ್ರಹ್ಮ-ಜ್ಞಾನಿ
ಬ್ರಹ್ಮ-ಜ್ಞಾ-ನಿ-ಗ-ಳಾ-ಗಿ-ರು-ವರೊ
ಬ್ರಹ್ಮ-ಜ್ಞಾ-ನಿ-ಗಳು
ಬ್ರಹ್ಮ-ಜ್ಞಾ-ನಿ-ಯಲ್ಲಿ
ಬ್ರಹ್ಮಣಃ
ಬ್ರಹ್ಮ-ಣ-ಸ್ತ್ರಿ-ವಿಧಃ
ಬ್ರಹ್ಮಣಾ
ಬ್ರಹ್ಮಣೋ
ಬ್ರಹ್ಮ-ಣೋ-ಽಪ್ಯಾ-ದಿ-ಕರ್ತ್ರೇ
ಬ್ರಹ್ಮ-ಣ್ಯಾ-ಧಾಯ
ಬ್ರಹ್ಮ-ತ-ತ್ವ-ವನ್ನು
ಬ್ರಹ್ಮ-ತೇ-ಜ-ಸ್ಸಾ-ಗಲೀ
ಬ್ರಹ್ಮದ
ಬ್ರಹ್ಮ-ದ-ರ್ಶ-ನ-ವಾ-ಗು-ವುದು
ಬ್ರಹ್ಮ-ದ-ರ್ಶ-ನವೇ
ಬ್ರಹ್ಮ-ದಲ್ಲಿ
ಬ್ರಹ್ಮ-ದ-ಲ್ಲಿಯೇ
ಬ್ರಹ್ಮ-ದಿಂದ
ಬ್ರಹ್ಮ-ದಿಂ-ದಾ-ಗು-ವುದು
ಬ್ರಹ್ಮ-ದೃಷ್ಟಿ
ಬ್ರಹ್ಮನ
ಬ್ರಹ್ಮ-ನನ್ನು
ಬ್ರಹ್ಮ-ನನ್ನೂ
ಬ್ರಹ್ಮ-ನಲ್ಲಿ
ಬ್ರಹ್ಮ-ನ-ಲ್ಲಿಗೆ
ಬ್ರಹ್ಮ-ನಾಗ
ಬ್ರಹ್ಮ-ನಾ-ದರೋ
ಬ್ರಹ್ಮ-ನಿಂದ
ಬ್ರಹ್ಮ-ನಿಗೂ
ಬ್ರಹ್ಮ-ನಿಗೆ
ಬ್ರಹ್ಮ-ನಿ-ರ್ವಾಣ
ಬ್ರಹ್ಮ-ನಿ-ರ್ವಾಣಂ
ಬ್ರಹ್ಮ-ನಿ-ರ್ವಾ-ಣ-ಮೃ-ಚ್ಛತಿ
ಬ್ರಹ್ಮ-ನಿ-ರ್ವಾ-ಣ-ಮೃ-ಷಯಃ
ಬ್ರಹ್ಮ-ನಿ-ರ್ವಾ-ಣ-ವನ್ನು
ಬ್ರಹ್ಮನೇ
ಬ್ರಹ್ಮ-ಪ-ದ-ವಿ-ಯನ್ನೂ
ಬ್ರಹ್ಮ-ಪುಷಿ
ಬ್ರಹ್ಮ-ಪ್ರಾ-ಪ್ತಿಯ
ಬ್ರಹ್ಮ-ಭಾ-ವ-ವನ್ನು
ಬ್ರಹ್ಮ-ಭೂತಃ
ಬ್ರಹ್ಮ-ಭೂ-ತ-ನಾಗಿ
ಬ್ರಹ್ಮ-ಭೂ-ತ-ಮ-ಕ-ಲ್ಮ-ಶಮ್
ಬ್ರಹ್ಮ-ಭೂ-ತೋ-ಽಧಿ-ಗ-ಚ್ಛತಿ
ಬ್ರಹ್ಮ-ಭೂ-ಯಾಯ
ಬ್ರಹ್ಮ-ಮಯ
ಬ್ರಹ್ಮ-ಯೋ-ಗ-ದಿಂದ
ಬ್ರಹ್ಮ-ಯೋ-ಗ-ಯು-ಕ್ತಾತ್ಮಾ
ಬ್ರಹ್ಮ-ಲೋಕ
ಬ್ರಹ್ಮ-ವನ್ನು
ಬ್ರಹ್ಮ-ವನ್ನೂ
ಬ್ರಹ್ಮ-ವನ್ನೇ
ಬ್ರಹ್ಮ-ವ-ಸ್ತುವೇ
ಬ್ರಹ್ಮ-ವಾ-ಗು-ತ್ತಾನೆ
ಬ್ರಹ್ಮ-ವಾ-ದಿ-ಗಳು
ಬ್ರಹ್ಮ-ವಾ-ದಿಗೆ
ಬ್ರಹ್ಮ-ವಾ-ದಿ-ನಾಮ್
ಬ್ರಹ್ಮ-ವಿ-ದ-ನಂತೆ
ಬ್ರಹ್ಮ-ವಿದೋ
ಬ್ರಹ್ಮ-ವಿ-ದ್ಬ್ರ-ಹ್ಮಣಿ
ಬ್ರಹ್ಮ-ವಿ-ದ್ಯೆ-ಜೀ-ವ-ಜ-ಗ-ತ್ತಿನ
ಬ್ರಹ್ಮ-ವಿ-ದ್ಯೆ-ಯನ್ನು
ಬ್ರಹ್ಮ-ವಿ-ಷ-ಯ-ವನ್ನು
ಬ್ರಹ್ಮ-ವೆಂದು
ಬ್ರಹ್ಮ-ವೆಂದೂ
ಬ್ರಹ್ಮ-ವೆಂಬ
ಬ್ರಹ್ಮವೇ
ಬ್ರಹ್ಮ-ಸಂ-ಸ್ಪರ್ಶ
ಬ್ರಹ್ಮ-ಸಂ-ಸ್ಪ-ರ್ಶ-ಮ-ತ್ಯಂತಂ
ಬ್ರಹ್ಮ-ಸಂ-ಸ್ಪ-ರ್ಶ-ವೆಂಬ
ಬ್ರಹ್ಮ-ಸ-ಮಾ-ಧಿಯೆ
ಬ್ರಹ್ಮ-ಸಾ-ಕ್ಷಾ-ತ್ಕಾರ
ಬ್ರಹ್ಮ-ಸಾ-ಕ್ಷಾ-ತ್ಕಾ-ರ-ವನ್ನು
ಬ್ರಹ್ಮ-ಸೂತ್ರ
ಬ್ರಹ್ಮ-ಸೂ-ತ್ರ-ಗಳು
ಬ್ರಹ್ಮ-ಸೂ-ತ್ರದ
ಬ್ರಹ್ಮ-ಸೂ-ತ್ರ-ಪ-ದ-ಗಳು
ಬ್ರಹ್ಮ-ಸೂ-ತ್ರ-ಪ-ದೈ-ಶ್ಚೈವ
ಬ್ರಹ್ಮ-ಸ್ಥಿ-ತಿಗೆ
ಬ್ರಹ್ಮಾ
ಬ್ರಹ್ಮಾಂಡ
ಬ್ರಹ್ಮಾಂ-ಡಕ್ಕೆ
ಬ್ರಹ್ಮಾಂ-ಡ-ಕ್ಕೆಲ್ಲ
ಬ್ರಹ್ಮಾಂ-ಡ-ಕ್ಕೆಲ್ಲಾ
ಬ್ರಹ್ಮಾಂ-ಡ-ಗ-ಳೆಲ್ಲ
ಬ್ರಹ್ಮಾಂ-ಡದ
ಬ್ರಹ್ಮಾಂ-ಡ-ದಲ್ಲಿ
ಬ್ರಹ್ಮಾಂ-ಡ-ದ-ಲ್ಲಿ-ರುವ
ಬ್ರಹ್ಮಾಂ-ಡ-ವನ್ನು
ಬ್ರಹ್ಮಾಂ-ಡ-ವ-ನ್ನೆಲ್ಲ
ಬ್ರಹ್ಮಾಂ-ಡ-ವ-ನ್ನೆಲ್ಲಾ
ಬ್ರಹ್ಮಾಂ-ಡ-ವನ್ನೇ
ಬ್ರಹ್ಮಾಂ-ಡ-ವಾ-ಗಿದೆ
ಬ್ರಹ್ಮಾಂ-ಡ-ವಾ-ಗಿ-ರು-ವುದು
ಬ್ರಹ್ಮಾಂ-ಡ-ವೆಂಬ
ಬ್ರಹ್ಮಾಂ-ಡ-ವೆಲ್ಲ
ಬ್ರಹ್ಮಾಂ-ಡ-ವೆ-ಲ್ಲವೂ
ಬ್ರಹ್ಮಾಂ-ಡ-ವೆಲ್ಲಾ
ಬ್ರಹ್ಮಾಂ-ಡವೇ
ಬ್ರಹ್ಮಾಂ-ಡ-ವೇನು
ಬ್ರಹ್ಮಾ-ಕ್ಷ-ರ-ಸ-ಮು-ದ್ಭ-ವಮ್
ಬ್ರಹ್ಮಾ-ಗ್ನಾ-ವ-ಪರೇ
ಬ್ರಹ್ಮಾಗ್ನಿ
ಬ್ರಹ್ಮಾ-ಗ್ನಿ-ಯಲ್ಲಿ
ಬ್ರಹ್ಮಾ-ಣ-ಮೀಶಂ
ಬ್ರಹ್ಮಾ-ನು-ಭವ
ಬ್ರಹ್ಮಾ-ನು-ಭ-ವದ
ಬ್ರಹ್ಮಾ-ನು-ಭ-ವ-ವನ್ನು
ಬ್ರಹ್ಮಾ-ರ್ಪಣಂ
ಬ್ರಹ್ಮಾ-ಸ್ತ್ರ-ವನ್ನು
ಬ್ರಹ್ಮೈವ
ಬ್ರಹ್ಮೋ-ಪಾ-ಸ-ನೆ-ಯಲ್ಲಿ
ಬ್ರಾಡ್ಕಾ-ಸ್ಟಿಂಗ್
ಬ್ರಾಹ್ಮಣ
ಬ್ರಾಹ್ಮ-ಣ-ಕ್ಷ-ತ್ರಿ-ಯ-ವಿ-ಶಾಂ
ಬ್ರಾಹ್ಮ-ಣನ
ಬ್ರಾಹ್ಮ-ಣ-ನಲ್ಲಿ
ಬ್ರಾಹ್ಮ-ಣ-ನಾ-ದರೆ
ಬ್ರಾಹ್ಮ-ಣ-ನಿಗೆ
ಬ್ರಾಹ್ಮ-ಣರ
ಬ್ರಾಹ್ಮ-ಣರು
ಬ್ರಾಹ್ಮ-ಣ-ವ-ರ್ಗ-ದ-ವ-ರೊಂ-ದಿಗೆ
ಬ್ರಾಹ್ಮ-ಣಸ್ಯ
ಬ್ರಾಹ್ಮ-ಣಾ-ಸ್ತೇನ
ಬ್ರಾಹ್ಮಣೇ
ಬ್ರಾಹ್ಮ-ಣೇ-ತರ
ಬ್ರಾಹ್ಮೀ
ಬ್ರೂಹಿ
ಬ್ರೇಕ್
ಭಂಗ
ಭಂಗ-ತಾ-ರವು
ಭಂಗ-ಬ-ರುವ
ಭಂಗ-ವನ್ನು
ಭಂಗ-ವಾ-ಗದ
ಭಂಗ-ವಾ-ಗು-ವುದು
ಭಂಗ-ವಾ-ಯಿತು
ಭಂಗವೂ
ಭಂಗಿ-ಯನ್ನು
ಭಂಗಿ-ಸು-ತ್ತೇನೆ
ಭಂಗಿ-ಸು-ವು-ದಾ-ಗಲಿ
ಭಂಟ
ಭಂಡ
ಭಂಡ-ತನ
ಭಂಡ-ತ-ನ-ವನ್ನು
ಭಂಡ-ನಾದ
ಭಂಡ-ರಂತೆ
ಭಂಡ-ರಾಗಿ
ಭಂಡ-ರಾ-ದರೆ
ಭಂಡರು
ಭಂಡಾ-ರಕ್ಕೇ
ಭಂಡಾ-ರ-ದಲ್ಲಿ
ಭಂಡಾ-ರ-ವನ್ನು
ಭಂಡಾ-ಸು-ರ-ನಂತೆ
ಭಕ್ತ
ಭಕ್ತಃ
ಭಕ್ತ-ಗೋರ
ಭಕ್ತನ
ಭಕ್ತ-ನದು
ಭಕ್ತ-ನನ್ನು
ಭಕ್ತ-ನ-ಲ್ಲ-ದ-ವ-ನಿಗೆ
ಭಕ್ತ-ನ-ಲ್ಲವೊ
ಭಕ್ತ-ನಲ್ಲಿ
ಭಕ್ತ-ನಾ-ಗಿ-ದ್ದಾನೆ
ಭಕ್ತ-ನಾ-ಗಿ-ರ-ಬೇಕು
ಭಕ್ತ-ನಾ-ಗಿ-ರು-ವನು
ಭಕ್ತ-ನಾಗು
ಭಕ್ತ-ನಾ-ಗು-ತ್ತಾನೆ
ಭಕ್ತ-ನಾದ
ಭಕ್ತ-ನಾ-ದರೆ
ಭಕ್ತ-ನಾ-ದರೊ
ಭಕ್ತ-ನಾ-ದರೋ
ಭಕ್ತ-ನಾ-ದ-ವನು
ಭಕ್ತ-ನಿ-ಗಾ-ದರೋ
ಭಕ್ತ-ನಿಗೂ
ಭಕ್ತ-ನಿಗೆ
ಭಕ್ತನು
ಭಕ್ತನೂ
ಭಕ್ತ-ನೆಂ-ದರೆ
ಭಕ್ತ-ನೆಂದು
ಭಕ್ತನೇ
ಭಕ್ತನೋ
ಭಕ್ತರ
ಭಕ್ತ-ರನ್ನು
ಭಕ್ತ-ರಲ್ಲಿ
ಭಕ್ತ-ರಾ-ಗ-ಕೂ-ಡದು
ಭಕ್ತ-ರಾ-ಗ-ಬೇಕು
ಭಕ್ತ-ರಾ-ಗಿ-ರು-ವುದನ್ನು
ಭಕ್ತ-ರಾದ
ಭಕ್ತ-ರಿಗೆ
ಭಕ್ತರು
ಭಕ್ತ-ರೆ-ಲ್ಲರೂ
ಭಕ್ತರೊ
ಭಕ್ತ-ರೊ-ಡನೆ
ಭಕ್ತ-ರೊ-ಡ-ನೆಯೂ
ಭಕ್ತಾ
ಭಕ್ತಾ-ಸ್ತೇ-ಽತೀವ
ಭಕ್ತಾ-ಸ್ತ್ವಾಂ
ಭಕ್ತಿ
ಭಕ್ತಿಂ
ಭಕ್ತಿ-ಅ-ಮೃ-ತ-ದಿಂದ
ಭಕ್ತಿ-ಗಾಗಿ
ಭಕ್ತಿಗೆ
ಭಕ್ತಿ-ಜೀ-ವ-ನಕ್ಕೆ
ಭಕ್ತಿ-ಪೂ-ರ್ಣ-ವಾಗಿ
ಭಕ್ತಿ-ಪೂ-ರ್ವಕ
ಭಕ್ತಿ-ಬೀಜ
ಭಕ್ತಿ-ಮಾನ್
ಭಕ್ತಿ-ಮಾ-ರ್ಗ-ವಿದೆ
ಭಕ್ತಿ-ಮು-ಕ್ತಿ-ಗಳನ್ನು
ಭಕ್ತಿಯ
ಭಕ್ತಿ-ಯನ್ನು
ಭಕ್ತಿ-ಯನ್ನೋ
ಭಕ್ತಿ-ಯಲ್ಲ
ಭಕ್ತಿ-ಯಲ್ಲಿ
ಭಕ್ತಿ-ಯಾ-ಗಲಿ
ಭಕ್ತಿ-ಯಾ-ಗಿ-ರ-ಬೇಕು
ಭಕ್ತಿ-ಯಿಂದ
ಭಕ್ತಿ-ಯಿಂ-ದಲೇ
ಭಕ್ತಿ-ಯು-ಳ್ಳ-ವ-ನಾಗಿ
ಭಕ್ತಿ-ಯು-ಳ್ಳ-ವನೂ
ಭಕ್ತಿಯೂ
ಭಕ್ತಿ-ಯೇನೊ
ಭಕ್ತಿ-ಯೊಂದೇ
ಭಕ್ತಿಯೋ
ಭಕ್ತಿ-ಯೋಗ
ಭಕ್ತಿ-ಯೋ-ಗೇನ
ಭಕ್ತಿ-ರ-ವ್ಯ-ಭಿ-ಚಾ-ರಿಣೀ
ಭಕ್ತಿ-ವಂ-ತನೂ
ಭಕ್ತಿ-ಸೂ-ತ್ರ-ದಲ್ಲಿ
ಭಕ್ತೋಽಸಿ
ಭಕ್ತ್ಯಾ
ಭಕ್ತ್ಯು-ಪ-ಹೃ-ತ-ಮ-ಶ್ನಾಮಿ
ಭಕ್ಷಿ-ಸುವ
ಭಕ್ಷ್ಯ-ಗಳನ್ನು
ಭಕ್ಷ್ಯ-ಭೋ-ಜ್ಯ-ಗಳನ್ನು
ಭಗ
ಭಗ-ನಂ-ತ-ನಿಗೆ
ಭಗವ
ಭಗ-ವಂತ
ಭಗ-ವಂ-ತನ
ಭಗ-ವಂ-ತ-ನಂತೆ
ಭಗ-ವಂ-ತ-ನ-ಕ-ಡೆಗೆ
ಭಗ-ವಂ-ತ-ನ-ಡಿ-ಯಲ್ಲಿ
ಭಗ-ವಂ-ತ-ನ-ದಾ-ಗಿದೆ
ಭಗ-ವಂ-ತ-ನದು
ಭಗ-ವಂ-ತ-ನನ್ನು
ಭಗ-ವಂ-ತ-ನನ್ನೆ
ಭಗ-ವಂ-ತ-ನನ್ನೇ
ಭಗ-ವಂ-ತ-ನ-ಮೇಲೆ
ಭಗ-ವಂ-ತ-ನಲ್ಲಿ
ಭಗ-ವಂ-ತ-ನ-ಲ್ಲಿ-ಟ್ಟಿ-ರುವ
ಭಗ-ವಂ-ತ-ನ-ಲ್ಲಿದೆ
ಭಗ-ವಂ-ತ-ನ-ಲ್ಲಿಯೇ
ಭಗ-ವಂ-ತ-ನ-ಲ್ಲಿ-ರುವ
ಭಗ-ವಂ-ತ-ನ-ಲ್ಲಿ-ರು-ವುದು
ಭಗ-ವಂ-ತ-ನಲ್ಲೆ
ಭಗ-ವಂ-ತ-ನಲ್ಲೇ
ಭಗ-ವಂ-ತ-ನಷ್ಟೇ
ಭಗ-ವಂ-ತ-ನಾ-ಗು-ವನು
ಭಗ-ವಂ-ತ-ನಾ-ದರೂ
ಭಗ-ವಂ-ತ-ನಾ-ದರೊ
ಭಗ-ವಂ-ತ-ನಾ-ದರೋ
ಭಗ-ವಂ-ತ-ನಿಂದ
ಭಗ-ವಂ-ತ-ನಿಂ-ದಲೇ
ಭಗ-ವಂ-ತ-ನಿ-ಗಾಗಿ
ಭಗ-ವಂ-ತ-ನಿ-ಗಾ-ಗಿಯೇ
ಭಗ-ವಂ-ತ-ನಿ-ಗಾ-ದರೊ
ಭಗ-ವಂ-ತ-ನಿ-ಗಿಂತ
ಭಗ-ವಂ-ತ-ನಿಗೂ
ಭಗ-ವಂ-ತ-ನಿಗೆ
ಭಗ-ವಂ-ತ-ನಿಗೇ
ಭಗ-ವಂ-ತ-ನಿ-ಗೊ-ಬ್ಬ-ನಿಗೇ
ಭಗ-ವಂ-ತ-ನಿ-ಟ್ಟಿ-ರುವ
ಭಗ-ವಂ-ತ-ನಿ-ರುವ
ಭಗ-ವಂ-ತ-ನಿ-ರು-ವನೋ
ಭಗ-ವಂ-ತ-ನಿ-ಲ್ಲದ
ಭಗ-ವಂ-ತನು
ಭಗ-ವಂ-ತನೆ
ಭಗ-ವಂ-ತ-ನೆಂಬ
ಭಗ-ವಂ-ತ-ನೆಂ-ಬುದು
ಭಗ-ವಂ-ತ-ನೆಡೆ
ಭಗ-ವಂ-ತ-ನೆ-ಡೆಗೆ
ಭಗ-ವಂ-ತ-ನೆ-ದು-ರಿಗೆ
ಭಗ-ವಂ-ತನೇ
ಭಗ-ವಂ-ತನೊ
ಭಗ-ವಂ-ತ-ನೊ-ಡನೆ
ಭಗ-ವಂ-ತ-ನೊ-ಬ್ಬನು
ಭಗ-ವಂ-ತ-ನೊ-ಬ್ಬನೆ
ಭಗ-ವಂ-ತ-ನೊ-ಬ್ಬನೇ
ಭಗ-ವಂ-ತನ್ನು
ಭಗ-ವ-ಚ್ಚಿಂ-ತ-ನೆಯ
ಭಗ-ವ-ಚ್ಛ-ಕ್ತಿಯ
ಭಗ-ವತಾ
ಭಗ-ವ-ತಿಯು
ಭಗ-ವ-ತೀಂ
ಭಗ-ವತ್
ಭಗ-ವ-ತ್ಕೃಪೆ
ಭಗ-ವ-ತ್ಪೂ-ಜೆ-ಯಾ-ಗಿ-ರ-ಬ-ಹುದು
ಭಗ-ವ-ತ್ಪ್ರೀ-ತಿ-ಗಾಗಿ
ಭಗ-ವ-ತ್ಪ್ರೇಮ
ಭಗ-ವ-ತ್ಭಕ್ತ
ಭಗ-ವ-ತ್ಸಾ-ಕ್ಷಾ-ತ್ಕಾರ
ಭಗ-ವ-ದಂಶ
ಭಗ-ವ-ದ-ನು-ಭವ
ಭಗ-ವ-ದ-ರ್ಪಣ
ಭಗ-ವ-ದ-ರ್ಪ-ಣ-ಭಾ-ವ-ದಿಂದ
ಭಗ-ವ-ದ-ರ್ಪಣೆ
ಭಗ-ವ-ದ-ರ್ಪಿತ
ಭಗ-ವ-ದ-ರ್ಪಿ-ತ-ವಾ-ಗಲಿ
ಭಗ-ವ-ದಾ-ನಂದ
ಭಗ-ವ-ದಾ-ನಂ-ದ-ದಷ್ಟು
ಭಗ-ವ-ದಿ-ಚ್ಛೆಗೆ
ಭಗ-ವ-ದಿ-ಚ್ಛೆ-ಯಿಂದ
ಭಗ-ವ-ದು-ಪಾ-ಸ-ನೆಗೆ
ಭಗ-ವದ್
ಭಗ-ವ-ದ್ಗೀತಾ
ಭಗ-ವ-ದ್ಗೀತೆ
ಭಗ-ವ-ದ್ಗೀ-ತೆಯ
ಭಗ-ವ-ದ್ಗೀ-ತೆ-ಯನ್ನು
ಭಗ-ವ-ದ್ಗೀ-ತೆ-ಯಲ್ಲಿ
ಭಗ-ವ-ದ್ಗೀ-ತೆ-ಯ-ಲ್ಲೆಲ್ಲ
ಭಗ-ವ-ದ್ಗೀ-ತೆ-ಯೆಂಬ
ಭಗ-ವ-ದ್ಗೀತೇ
ಭಗ-ವ-ದ್ದ-ರ್ಶ-ನ-ಕ್ಕಾಗಿ
ಭಗ-ವ-ದ್ಭ-ಕ್ತ-ರಿಗೆ
ಭಗ-ವ-ದ್ಭ-ಕ್ತರೊ
ಭಗ-ವ-ದ್ಭ-ಕ್ತಿಯ
ಭಗ-ವ-ದ್ಭ-ಕ್ತಿ-ಯಿಂದ
ಭಗ-ವ-ದ್ಭಾ-ವ-ನೆ-ಯನ್ನು
ಭಗ-ವ-ದ್ವಾ-ಣಿ-ಯನ್ನು
ಭಗ-ವ-ದ್ಸಾಂ-ನಿ-ಧ್ಯ-ದ-ಲ್ಲಿ-ರು-ವನು
ಭಗ-ವನ್
ಭಗ-ವ-ನ್ಮ-ಯ-ವಾಗಿ
ಭಗ-ವ-ನ್ಮಯಾ
ಭಗ-ವಾನ್
ಭಗೀ-ರಥ
ಭಗ್
ಭಜಂತಿ
ಭಜಂತೇ
ಭಜಂ-ತ್ಯ-ನ-ನ್ಯ-ಮ-ನಸೋ
ಭಜ-ತಾಂ
ಭಜತೇ
ಭಜ-ತ್ಯೇ-ಕ-ತ್ವ-ಮಾ-ಸ್ಥಿತಃ
ಭಜನೆ
ಭಜ-ನೆಯ
ಭಜಸ್ವ
ಭಜಾ-ಮ್ಯ-ಹಮ್
ಭಜಿ-ಸ-ಬೇ-ಕಾ-ದರೆ
ಭಜಿ-ಸಿ-ದರೆ
ಭಜಿಸು
ಭಜಿ-ಸು-ತ್ತಾನೆ
ಭಜಿ-ಸು-ತ್ತಾರೆ
ಭಜಿ-ಸು-ತ್ತಾ-ರೆಯೋ
ಭಜಿ-ಸು-ತ್ತಾರೊ
ಭಜಿ-ಸು-ತ್ತಿ-ರು-ವನು
ಭಜಿ-ಸು-ತ್ತಿ-ರು-ವ-ವರೂ
ಭಜಿ-ಸು-ವನು
ಭಜಿ-ಸು-ವನೋ
ಭಜಿ-ಸು-ವರು
ಭಜಿ-ಸು-ವರೊ
ಭಜಿ-ಸು-ವರೋ
ಭಜಿ-ಸು-ವ-ವರು
ಭಜಿ-ಸು-ವುದು
ಭತ್ತ
ಭತ್ತದ
ಭತ್ತ-ವನ್ನು
ಭದ್ರ
ಭದ್ರ-ತೆಗೆ
ಭದ್ರ-ಮ-ನು-ಷ್ಯ-ನಿಗೆ
ಭದ್ರ-ವಾಗಿ
ಭದ್ರ-ವಾ-ಗಿ-ರ-ಬೇ-ಕಾ-ದರೆ
ಭದ್ರ-ವಾ-ಗಿ-ರ-ಬೇಕು
ಭದ್ರ-ವಾ-ಗಿ-ರು-ವುದು
ಭದ್ರ-ವಾ-ಗಿ-ರು-ವುದೊ
ಭದ್ರ-ವಾ-ಗಿ-ರು-ವೆವು
ಭದ್ರ-ವಾ-ಗು-ವುದು
ಭದ್ರ-ವಾದ
ಭಯ
ಭಯಂ
ಭಯಂ-ಕರ
ಭಯಂ-ಕ-ರ-ವಾ-ಗಿತ್ತು
ಭಯಂ-ಕ-ರ-ವಾ-ಗಿ-ರು-ವುದು
ಭಯಂ-ಕ-ರ-ವಾದ
ಭಯಂ-ಕ-ರ-ವಾ-ದು-ವು-ಗಳು
ಭಯಂ-ಕ-ರವೂ
ಭಯಕ್ಕೆ
ಭಯ-ಗಳಿಂದ
ಭಯ-ಗಳು
ಭಯ-ಗೊಂ-ಡಿದೆ
ಭಯ-ಗೊಂ-ಡಿ-ದ್ದೇವೆ
ಭಯ-ಗೊಂ-ಡಿ-ರು-ವರು
ಭಯ-ಗೊಂ-ಡಿಲ್ಲ
ಭಯ-ದಿಂದ
ಭಯ-ನ-ಕಾನಿ
ಭಯ-ಪ-ಡ-ಬಾ-ರದು
ಭಯ-ಪ-ಡ-ಬೇಕು
ಭಯ-ಭೀ-ತ-ನಾದ
ಭಯ-ವನ್ನು
ಭಯ-ವಲ್ಲ
ಭಯ-ವಿಲ್ಲ
ಭಯ-ವುಂ-ಟಾ-ಗು-ವುದು
ಭಯವೂ
ಭಯ-ವೆಂ-ಬುದು
ಭಯಾತ್
ಭಯಾ-ದ್ರ-ಣಾ-ದು-ಪ-ರತಂ
ಭಯಾ-ನಕ
ಭಯಾ-ನ-ಕ-ವಲ್ಲ
ಭಯಾ-ನ-ಕ-ವಾಗಿ
ಭಯಾ-ನ-ಕ-ವಾ-ಗಿದೆ
ಭಯಾ-ನ-ಕ-ವಾ-ಗಿ-ರು-ವುದು
ಭಯಾ-ನ-ಕ-ವಾ-ಗು-ವುದು
ಭಯಾ-ನ-ಕ-ವಾದ
ಭಯಾ-ನ-ಕ-ವಾ-ದು-ದನ್ನೂ
ಭಯಾ-ಭ-ಯ-ಗಳನ್ನು
ಭಯಾ-ಭಯೇ
ಭಯಾ-ವಹಃ
ಭಯೇನ
ಭರ-ತ-ಖಂಡ
ಭರ-ತ-ಖಂ-ಡದ
ಭರ-ತ-ಖಂ-ಡ-ದಲ್ಲಿ
ಭರ-ತ-ರ್ಷಭ
ಭರ-ತ-ಶ್ರೇಷ್ಠ
ಭರ-ತ-ಸ-ತ್ತಮ
ಭರ-ದಿಂದ
ಭರ-ವಸೆ
ಭರ-ವ-ಸೆಯ
ಭರ-ವ-ಸೆ-ಯನ್ನು
ಭರ-ವ-ಸೆಯೆ
ಭರ-ವ-ಸೆಯೇ
ಭರ್ತಾ
ಭರ್ತಿ
ಭರ್ತಿ-ಮಾಡ
ಭರ್ತಿ-ಮಾ-ಡಲು
ಭರ್ತಿ-ಯಾ-ಗು-ವು-ದಿಲ್ಲ
ಭರ್ತೃ
ಭಲೆ
ಭವ
ಭವಂತಃ
ಭವಂ-ತ-ಮಾದ್ಯಂ
ಭವಂತಿ
ಭವಕ್ಕೆ
ಭವ-ಜೀ-ವಿ-ಗಳ
ಭವ-ಜೀ-ವಿ-ಗಳನ್ನು
ಭವ-ಜೀ-ವಿ-ಗಳಲ್ಲಿ
ಭವ-ಜೀ-ವಿ-ಗ-ಳಿಗೂ
ಭವ-ಜೀ-ವಿ-ಗ-ಳಿಗೆ
ಭವ-ಜೀ-ವಿ-ಗ-ಳೆಂಬ
ಭವತಿ
ಭವ-ತೀ-ತ್ಯ-ನು-ಶು-ಶ್ರುಮ
ಭವತೋ
ಭವ-ತೋ-ಽಜ್ಞಾ-ನ-ಮೇವ
ಭವತ್
ಭವ-ತ್ಯ-ತ್ಯಾ-ಗಿ-ನಾಂ
ಭವದ
ಭವ-ದ್ವೇ-ಷಿ-ಣೀಮ್
ಭವ-ಬಂ-ಧ-ನ-ದಿಂದ
ಭವ-ಭೂತಿ
ಭವ-ರೋ-ಗದ
ಭವ-ರೋ-ಗ-ವನ್ನು
ಭವ-ವೈದ್ಯ
ಭವ-ವೈ-ದ್ಯ-ನಾದ
ಭವ-ಸಾ-ಗ-ರ-ದಿಂದ
ಭವ-ಸಾ-ಗ-ರ-ವನ್ನು
ಭವಾ-ನಿ-ಯನ್ನು
ಭವಾ-ನು-ಗ್ರ-ರೂಪೋ
ಭವಾನ್
ಭವಾ-ಪ್ಯಯೌ
ಭವಾಮಿ
ಭವಾ-ರ್ಜುನ
ಭವಿತಾ
ಭವಿಷ್ಯ
ಭವಿ-ಷ್ಯಂತಿ
ಭವಿ-ಷ್ಯ-ತಾಮ್
ಭವಿ-ಷ್ಯತಿ
ಭವಿ-ಷ್ಯತ್
ಭವಿ-ಷ್ಯ-ತ್ತು-ಗ-ಳಿಗೆ
ಭವಿ-ಷ್ಯದ
ಭವಿ-ಷ್ಯ-ದಲ್ಲಿ
ಭವಿ-ಷ್ಯ-ವನ್ನು
ಭವಿ-ಷ್ಯ-ವಾ-ಗಿ-ರ-ಬ-ಹುದು
ಭವಿ-ಷ್ಯ-ವಾ-ಯಿತು
ಭವಿ-ಷ್ಯಾಣಿ
ಭವಿ-ಷ್ಯಾಮಃ
ಭವಿ-ಸಿ-ಯೇನೊ
ಭವಿ-ಸು-ವ-ವ-ರನ್ನು
ಭವಿ-ಸು-ವು-ದ-ಕ್ಕಿಂತ
ಭವೇತ್
ಭವೇ-ದ್ಯು-ಗ-ಪ-ದು-ತ್ಥಿತಾ
ಭವೋ-ಽಭಾವೋ
ಭವ್ಯ
ಭವ್ಯ-ವಾ-ಗಿ-ರ-ಬೇಕು
ಭವ್ಯ-ವಾ-ಗಿ-ರು-ವುದು
ಭವ್ಯ-ವಾದ
ಭವ್ಯ-ವಾ-ದುದು
ಭವ್ಯ-ವಾ-ದು-ವು-ಗಳು
ಭಸ್ಮ-ಮಾ-ಡಿ-ಬಿ-ಡು-ವುದು
ಭಸ್ಮ-ವಾಗಿ
ಭಸ್ಮ-ಸಾತ್
ಭಸ್ಮೀ-ಭೂತ
ಭಸ್ಮೀ-ಭೂ-ತ-ಮಾ-ಡು-ವುದು
ಭಾಂಗ್
ಭಾಃ
ಭಾಗ
ಭಾಗ-ಕ್ಕಿಂತ
ಭಾಗಕ್ಕೆ
ಭಾಗ-ಗ-ಳಂತೆ
ಭಾಗ-ಗಳನ್ನು
ಭಾಗ-ಗಳನ್ನೂ
ಭಾಗ-ಗಳಲ್ಲಿ
ಭಾಗ-ಗಳಿಂದ
ಭಾಗ-ಗ-ಳಿ-ಗೆಲ್ಲ
ಭಾಗ-ಗ-ಳಿವೆ
ಭಾಗ-ಗಳು
ಭಾಗದ
ಭಾಗ-ದಲ್ಲಿ
ಭಾಗ-ದಷ್ಟೇ
ಭಾಗ-ಮಾ-ಡ-ಲಾ-ರರು
ಭಾಗ-ವತ
ಭಾಗ-ವ-ತ-ಗಳಲ್ಲಿ
ಭಾಗ-ವ-ತ-ದಲ್ಲಿ
ಭಾಗ-ವ-ತ-ನಾದ
ಭಾಗ-ವ-ತನು
ಭಾಗ-ವ-ತ-ವನ್ನು
ಭಾಗ-ವನ್ನು
ಭಾಗ-ವ-ನ್ನೆಲ್ಲಾ
ಭಾಗ-ವ-ಹಿ-ಸ-ಬೇ-ಕಾ-ದುದು
ಭಾಗ-ವ-ಹಿ-ಸಲು
ಭಾಗ-ವ-ಹಿ-ಸು-ವುದು
ಭಾಗ-ವಾಗಿ
ಭಾಗ-ವಾ-ಗಿದೆ
ಭಾಗ-ವಾ-ಗಿ-ರು-ವಂತೆ
ಭಾಗ-ವಾ-ಗಿಲ್ಲ
ಭಾಗ-ವಾ-ಗಿವೆ
ಭಾಗ-ವಾ-ಗು-ವು-ದಿಲ್ಲ
ಭಾಗ-ವಾದ
ಭಾಗ-ವಾ-ದಂತೆ
ಭಾಗ-ವಿದೆ
ಭಾಗವೂ
ಭಾಗವೇ
ಭಾಗಿ
ಭಾಗಿ-ಗ-ಳಾಗಿ
ಭಾಗಿ-ಗ-ಳಾ-ಗು-ತ್ತೇವೆ
ಭಾಗಿ-ಗಳು
ಭಾಗಿ-ಯಾ-ಗು-ವಂ-ತಹ
ಭಾಗಿ-ಯಾ-ಗು-ವನು
ಭಾಗ್ಯ
ಭಾಗ್ಯ-ವಂ-ತ-ನಿಗೆ
ಭಾರ
ಭಾರಕ್ಕೆ
ಭಾರತ
ಭಾರ-ತ-ತೈ-ಲ-ಪೂರ್ಣಃ
ಭಾರ-ತ-ವನ್ನು
ಭಾರ-ತ-ವಾ-ಯಿತು
ಭಾರ-ತೀಯ
ಭಾರ-ದಲ್ಲಿ
ಭಾರ-ದಿಂದ
ಭಾರ-ವನ್ನು
ಭಾರ-ವಾಗಿ
ಭಾರ-ವಾ-ಗು-ತ್ತೇವೆ
ಭಾರ-ವಾ-ಗು-ವುದು
ಭಾರವೇ
ಭಾರಿ
ಭಾವ
ಭಾವ-ಇವು
ಭಾವಂ
ಭಾವಃ
ಭಾವಕ್ಕೂ
ಭಾವಕ್ಕೆ
ಭಾವ-ಗಳ
ಭಾವ-ಗಳನ್ನು
ಭಾವ-ಗಳಿಂದ
ಭಾವ-ಗಳು
ಭಾವ-ಗ-ಳೆಲ್ಲ
ಭಾವ-ಚಿ-ತ್ರ-ವಲ್ಲ
ಭಾವ-ಜೀವಿ
ಭಾವದ
ಭಾವ-ದಂ-ತೆಯೇ
ಭಾವ-ದಲ್ಲಿ
ಭಾವ-ದ-ಲ್ಲಿಯೂ
ಭಾವ-ದಿಂದ
ಭಾವ-ದಿಂ-ದ-ಲಾ-ದರೂ
ಭಾವ-ಧಾರೆ
ಭಾವನಾ
ಭಾವನೆ
ಭಾವ-ನೆ-ಗಳ
ಭಾವ-ನೆ-ಗಳನ್ನು
ಭಾವ-ನೆ-ಗಳನ್ನೆಲ್ಲ
ಭಾವ-ನೆ-ಗಳನ್ನೆಲ್ಲಾ
ಭಾವ-ನೆ-ಗಳಲ್ಲಿ
ಭಾವ-ನೆ-ಗಳಿಂದ
ಭಾವ-ನೆ-ಗ-ಳಿಂ-ದಲೇ
ಭಾವ-ನೆ-ಗ-ಳಿ-ಗಿಂತ
ಭಾವ-ನೆ-ಗ-ಳಿ-ದ್ದರೆ
ಭಾವ-ನೆ-ಗಳು
ಭಾವ-ನೆ-ಗಳೂ
ಭಾವ-ನೆ-ಗ-ಳೆಲ್ಲ
ಭಾವ-ನೆ-ಗಳೇ
ಭಾವ-ನೆಗೆ
ಭಾವ-ನೆಯ
ಭಾವ-ನೆ-ಯನ್ನು
ಭಾವ-ನೆ-ಯನ್ನೂ
ಭಾವ-ನೆ-ಯನ್ನೇ
ಭಾವ-ನೆ-ಯಲ್ಲಿ
ಭಾವ-ನೆ-ಯ-ಲ್ಲಿದೆ
ಭಾವ-ನೆ-ಯ-ವ-ರೆಗೆ
ಭಾವ-ನೆ-ಯಾ-ದರೋ
ಭಾವ-ನೆ-ಯಿಂದ
ಭಾವ-ನೆ-ಯಿಂ-ದಲ್ಲ
ಭಾವ-ನೆಯೂ
ಭಾವ-ನೆಯೇ
ಭಾವ-ಪೂ-ರ್ವಕ
ಭಾವ-ಪ್ರ-ಧಾನ
ಭಾವ-ಮ-ಜಾ-ನಂತೋ
ಭಾವ-ಮ-ವ್ಯ-ಯ-ಮೀ-ಕ್ಷತೇ
ಭಾವ-ಮಾ-ಶ್ರಿ-ತಾಃ
ಭಾವ-ಮು-ಖ-ದಿಂದ
ಭಾವ-ಮೈ-ದಂ-ದಿರು
ಭಾವ-ಯಂತಃ
ಭಾವ-ಯಂತು
ಭಾವ-ಯ-ತಾ-ನೇನ
ಭಾವ-ವ-ಗಳಲ್ಲಿ
ಭಾವ-ವನ್ನು
ಭಾವ-ವನ್ನೂ
ಭಾವ-ವನ್ನೇ
ಭಾವ-ವಿದೆ
ಭಾವ-ವಿ-ದ್ದರೆ
ಭಾವ-ವಿಲ್ಲ
ಭಾವ-ವು-ಳ್ಳ-ವ-ರಾಗಿ
ಭಾವವೂ
ಭಾವವೇ
ಭಾವ-ವೊಂದೇ
ಭಾವವೋ
ಭಾವ-ಶುದ್ಧಿ
ಭಾವ-ಸಂ-ಶು-ದ್ಧಿ-ರಿ-ತ್ಯೇ-ತ-ತ್ತಪೋ
ಭಾವ-ಸ-ಮ-ನ್ವಿ-ತಾಃ
ಭಾವಾ
ಭಾವಿ
ಭಾವಿಗೆ
ಭಾವಿಯ
ಭಾವಿ-ಯಲ್ಲಿ
ಭಾವಿಸ
ಭಾವಿ-ಸದೆ
ಭಾವಿ-ಸ-ಬ-ಹುದು
ಭಾವಿ-ಸ-ಬಾ-ರದು
ಭಾವಿ-ಸ-ಬೇ-ಕಾ-ಗಿಲ್ಲ
ಭಾವಿ-ಸ-ಬೇ-ಕಾ-ಗು-ವುದು
ಭಾವಿ-ಸ-ಬೇಕು
ಭಾವಿ-ಸ-ಬೇಡ
ಭಾವಿ-ಸಲು
ಭಾವಿಸಿ
ಭಾವಿ-ಸಿ-ಕೊಂಡು
ಭಾವಿ-ಸಿ-ಕೊ-ಳ್ಳಲು
ಭಾವಿ-ಸಿದ
ಭಾವಿ-ಸಿ-ದಂತೆ
ಭಾವಿ-ಸಿ-ದನು
ಭಾವಿ-ಸಿ-ದರು
ಭಾವಿ-ಸಿ-ದರೂ
ಭಾವಿ-ಸಿ-ದರೆ
ಭಾವಿ-ಸಿ-ದಾ-ಗಲೂ
ಭಾವಿ-ಸಿದ್ದ
ಭಾವಿ-ಸಿ-ದ್ದ-ನಲ್ಲ
ಭಾವಿ-ಸಿ-ದ್ದರು
ಭಾವಿ-ಸಿ-ದ್ದಾನೆ
ಭಾವಿ-ಸಿದ್ದು
ಭಾವಿ-ಸಿದ್ದೆ
ಭಾವಿ-ಸಿ-ದ್ದೆ-ನಲ್ಲ
ಭಾವಿ-ಸಿ-ದ್ದೆವು
ಭಾವಿ-ಸಿ-ದ್ದೇವೆ
ಭಾವಿ-ಸಿ-ರ-ಬೇಕು
ಭಾವಿ-ಸಿ-ರ-ಲಿಲ್ಲ
ಭಾವಿ-ಸಿ-ರು-ವ-ವನೂ
ಭಾವಿ-ಸಿ-ರು-ವುದು
ಭಾವಿ-ಸಿ-ರು-ವೆನು
ಭಾವಿ-ಸಿ-ರು-ವೆವು
ಭಾವಿಸು
ಭಾವಿ-ಸು-ತ್ತವೆ
ಭಾವಿ-ಸು-ತ್ತಾನೆ
ಭಾವಿ-ಸು-ತ್ತಾ-ನೆಯೊ
ಭಾವಿ-ಸು-ತ್ತಾ-ನೆಯೋ
ಭಾವಿ-ಸು-ತ್ತಾರೆ
ಭಾವಿ-ಸು-ತ್ತಿ-ದ್ದರೆ
ಭಾವಿ-ಸು-ತ್ತಿ-ದ್ದಾನೆ
ಭಾವಿ-ಸು-ತ್ತಿ-ರು-ವಾ-ಗಲೆ
ಭಾವಿ-ಸು-ತ್ತಿ-ರು-ವೆವೋ
ಭಾವಿ-ಸು-ತ್ತೇನೆ
ಭಾವಿ-ಸು-ತ್ತೇವೆ
ಭಾವಿ-ಸು-ತ್ತೇ-ವೆಯೊ
ಭಾವಿ-ಸುವ
ಭಾವಿ-ಸು-ವಂತೆ
ಭಾವಿ-ಸು-ವನು
ಭಾವಿ-ಸು-ವನೆ
ಭಾವಿ-ಸು-ವನೊ
ಭಾವಿ-ಸು-ವನೋ
ಭಾವಿ-ಸು-ವರು
ಭಾವಿ-ಸು-ವರೊ
ಭಾವಿ-ಸು-ವ-ವನ
ಭಾವಿ-ಸು-ವು-ದಕ್ಕೆ
ಭಾವಿ-ಸು-ವು-ದ-ರಿಂದ
ಭಾವಿ-ಸು-ವು-ದಿಲ್ಲ
ಭಾವಿ-ಸು-ವುದು
ಭಾವಿ-ಸು-ವೆನು
ಭಾವಿ-ಸು-ವೆವು
ಭಾವಿ-ಸು-ವೆವೊ
ಭಾವಿ-ಸು-ವೆವೋ
ಭಾವಿ-ಸೋಣ
ಭಾವೇಷು
ಭಾವೋ
ಭಾವೋ-ಽನ್ಯೋ-ಽವ್ಯ-ಕ್ತೋ-ಽವ್ಯ-ಕ್ತಾತ್
ಭಾಷಸೇ
ಭಾಷಾ
ಭಾಷೆ
ಭಾಷೆಯ
ಭಾಷೆ-ಯನ್ನು
ಭಾಷೆ-ಯನ್ನೇ
ಭಾಷೆ-ಯಲ್ಲಿ
ಭಾಷೆ-ಯ-ಲ್ಲಿ-ಡ-ಬೇ-ಕಾ-ದಾಗ
ಭಾಷೆ-ಯಾ-ದರೋ
ಭಾಷೆ-ಯೊ-ಳಗೂ
ಭಾಷ್ಯ-ಕಾ-ರರ
ಭಾಷ್ಯ-ಕಾ-ರ-ರಿಗೆ
ಭಾಷ್ಯ-ಗಳನ್ನು
ಭಾಷ್ಯ-ಗ-ಳಿಗೆ
ಭಾಷ್ಯ-ಗಳು
ಭಾಷ್ಯ-ದಂ-ತಿದೆ
ಭಾಷ್ಯ-ವನ್ನು
ಭಾಷ್ಯವೇ
ಭಾಸ-ವಾ-ಗು-ವುದು
ಭಾಸ-ವಾ-ಗು-ವುವು
ಭಾಸ-ಸ್ತ-ವೋ-ಗ್ರಾಃ
ಭಾಸ್ವತಾ
ಭಿಕಾರಿ
ಭಿಕಾ-ರಿ-ಗ-ಳಂತೆ
ಭಿಕಾ-ರಿಯು
ಭಿಕ್ಷ-ಕ-ರಂತೆ
ಭಿಕ್ಷಾ-ನ್ನ-ದಿಂ-ದ-ಲಾ-ದರೂ
ಭಿಕ್ಷಾ-ಪಾ-ತ್ರೆ-ಯನ್ನು
ಭಿಕ್ಷು-ಕ-ನಂತೆ
ಭಿಕ್ಷು-ಕ-ನಾ-ಗಿದ್ದು
ಭಿಕ್ಷು-ಕ-ನಿಂದ
ಭಿಕ್ಷು-ಕ-ನಿಗೆ
ಭಿಕ್ಷೆ
ಭಿನ್ನ
ಭಿನ್ನತೆ
ಭಿನ್ನ-ತೆ-ಗ-ಳೆಲ್ಲ
ಭಿನ್ನ-ಭಿನ್ನ
ಭಿನ್ನ-ವಾ-ಗಿದೆ
ಭಿನ್ನವೂ
ಭಿನ್ನಾ
ಭಿನ್ನಾ-ಭಿ-ಪ್ರಾ-ಯ-ಗಳು
ಭೀಕರ
ಭೀಕ-ರ-ವಾ-ಗಿದೆ
ಭೀಕ-ರ-ವಾ-ಗಿ-ರು-ವುದು
ಭೀಕ-ರ-ವಾ-ಗು-ವುದು
ಭೀಕ-ರ-ವಾದ
ಭೀತ-ಗೊಂ-ಡಿದೆ
ಭೀತ-ನಾಗಿ
ಭೀತ-ಭೀತಃ
ಭೀತ-ಮೇನಂ
ಭೀತ-ರಾಗಿ
ಭೀತಾನಿ
ಭೀಮ
ಭೀಮ-ಕರ್ಮಾ
ಭೀಮ-ನಂತೆ
ಭೀಮ-ನಿಂದ
ಭೀಮನು
ಭೀಮ-ಬಂ-ಡೆ-ಯಂತೆ
ಭೀಮ-ವೃ-ಕ್ಷದ
ಭೀಮ-ಸೇನ
ಭೀಮಾ-ಭಿ-ರ-ಕ್ಷಿ-ತಮ್
ಭೀಮಾ-ರ್ಜು-ನ-ರಿಗೆ
ಭೀಮಾ-ರ್ಜು-ನ-ಸಮಾ
ಭೀಷ್ಮ
ಭೀಷ್ಮಂ
ಭೀಷ್ಮ-ಅ-ವನೇ
ಭೀಷ್ಮ-ಇ-ಚ್ಛಾ-ಮ-ರಣಿ
ಭೀಷ್ಮ-ದ್ರೋ-ಣ-ತಟಾ
ಭೀಷ್ಮ-ದ್ರೋ-ಣ-ಪ್ರ-ಮು-ಖತಃ
ಭೀಷ್ಮ-ದ್ರೋ-ಣರೇ
ಭೀಷ್ಮ-ದ್ರೋ-ಣಾ-ದಿ-ಗಳು
ಭೀಷ್ಮ-ನಂ-ತಹ
ಭೀಷ್ಮ-ನಿಂದ
ಭೀಷ್ಮನು
ಭೀಷ್ಮ-ಪ-ರ್ವ-ದಲ್ಲಿ
ಭೀಷ್ಮ-ಪ್ರ-ತಿಜ್ಞೆ
ಭೀಷ್ಮ-ಮಹಂ
ಭೀಷ್ಮ-ಮೇ-ವಾ-ಭಿ-ರ-ಕ್ಷಂತು
ಭೀಷ್ಮ-ರನ್ನು
ಭೀಷ್ಮ-ರಿಗೂ
ಭೀಷ್ಮರು
ಭೀಷ್ಮಶ್ಚ
ಭೀಷ್ಮಾ-ಚಾ-ರ್ಯ-ರಾ-ದರೋ
ಭೀಷ್ಮಾ-ಚಾ-ರ್ಯರು
ಭೀಷ್ಮಾ-ಭಿ-ರ-ಕ್ಷಿ-ತಮ್
ಭೀಷ್ಮೋ
ಭುಂಕ್ತೇ
ಭುಂಕ್ಷ್ವ
ಭುಂಜತೇ
ಭುಂಜಾನಂ
ಭುಂಜಿ-ಸ-ಬೇ-ಕಾ-ದರೂ
ಭುಂಜಿ-ಸ-ಬೇಕು
ಭುಂಜಿಸಿ
ಭುಂಜಿ-ಸು-ತ್ತಾರೆ
ಭುಂಜಿ-ಸು-ವರೋ
ಭುಂಜೀಯ
ಭುಕ್ತ್ವಾ
ಭುಗಿ-ಲೆಂದು
ಭುಜ-ದಿಂದ
ಭುವ-ನ-ವೆಂಬ
ಭುವಿ
ಭೂಕಂಪ
ಭೂಕಂ-ಪ-ವಾಗಿ
ಭೂಕಂ-ಪವು
ಭೂಗರ್ಭ
ಭೂಗೋಳ
ಭೂಚಃ
ಭೂತ
ಭೂತಂ
ಭೂತ-ಕ-ನ್ನಡಿ
ಭೂತ-ಕಾ-ಲದ್ದು
ಭೂತ-ಕಾ-ಲ-ವಿದೆ
ಭೂತಕ್ಕೆ
ಭೂತ-ಗ-ಣಾಂ-ಶ್ಚಾನ್ಯೇ
ಭೂತ-ಗಳ
ಭೂತ-ಗಳನ್ನು
ಭೂತ-ಗಳನ್ನೂ
ಭೂತ-ಗಳಲ್ಲಿ
ಭೂತ-ಗ-ಳ-ಲ್ಲಿಯೂ
ಭೂತ-ಗ-ಳ-ಲ್ಲಿ-ರುವ
ಭೂತ-ಗಳಿಂದ
ಭೂತ-ಗ-ಳಿಂ-ದಲೂ
ಭೂತ-ಗ-ಳಿಗೂ
ಭೂತ-ಗ-ಳಿಗೆ
ಭೂತ-ಗಳು
ಭೂತ-ಗ-ಳೆಲ್ಲಾ
ಭೂತ-ಗ್ರಾಮಃ
ಭೂತ-ಗ್ರಾ-ಮ-ಮ-ಚೇ-ತಸಃ
ಭೂತ-ಗ್ರಾ-ಮ-ಮಿಮಂ
ಭೂತದ
ಭೂತ-ದಯೆ
ಭೂತ-ನಾ-ಮಂತ
ಭೂತ-ಪೃ-ಥ-ಗ್ಭಾ-ವ-ಮೇ-ಕ-ಸ್ಥ-ಮ-ನು-ಪ-ಶ್ಯತಿ
ಭೂತ-ಪ್ರ-ಕೃ-ತಿ-ಮೋಕ್ಷಂ
ಭೂತ-ಪ್ರೇ-ತಾ-ದಿ-ಗಳನ್ನು
ಭೂತ-ಭರ್ತೃ
ಭೂತ-ಭ-ವಿ-ಷ್ಯ-ತ್ತು-ಗಳನ್ನೆಲ್ಲಾ
ಭೂತ-ಭಾ-ವನ
ಭೂತ-ಭಾ-ವನಃ
ಭೂತ-ಭಾ-ವೋ-ದ್ಭ-ವ-ಕರೋ
ಭೂತ-ಭೃನ್ನ
ಭೂತ-ಮ-ಹೇ-ಶ್ವ-ರಮ್
ಭೂತ-ಯಜ್ಞ
ಭೂತ-ವಿ-ಶೇ-ಷ-ಸಂ-ಘಾನ್
ಭೂತ-ಸರ್ಗೌ
ಭೂತಸ್ಥೋ
ಭೂತಾ-ದಿ-ಮ-ವ್ಯ-ಯಮ್
ಭೂತಾ-ನಾಂ
ಭೂತಾ-ನಾ-ಮ-ಚರಂ
ಭೂತಾ-ನಾ-ಮಸ್ಮಿ
ಭೂತಾ-ನಾ-ಮೀ-ಶ್ವ-ರೋಽಪಿ
ಭೂತಾನಿ
ಭೂತಾ-ನ್ಯ-ಶೇ-ಷೇಣ
ಭೂತಾ-ವ-ಸ್ಥೆ-ಯ-ಲ್ಲಿ-ರುವ
ಭೂತಿ-ರ್ಧ್ರುವಾ
ಭೂತೇಜ್ಯಾ
ಭೂತೇಶ
ಭೂತೇ-ಶ್ವ-ಲೋ-ಲುಪ್ತ್ವಂ
ಭೂತೇಷು
ಭೂತ್ವಾ
ಭೂಮ
ಭೂಮಂ-ಡ-ಲವೇ
ಭೂಮಕ್ಕೆ
ಭೂಮದ
ಭೂಮ-ದ-ರ್ಶ-ನ-ವನ್ನು
ಭೂಮ-ದಲ್ಲಿ
ಭೂಮ-ದೃ-ಷ್ಟಿಗೆ
ಭೂಮ-ದೃ-ಷ್ಟಿ-ಯಿಂದ
ಭೂಮ-ದೆ-ದು-ರಿಗೆ
ಭೂಮ-ದೊಂ-ದಿಗೆ
ಭೂಮ-ನಾದ
ಭೂಮ-ಯಜ್ಞ
ಭೂಮ-ವ-ನ್ನಾಗಿ
ಭೂಮ-ವನ್ನು
ಭೂಮ-ವಾದ
ಭೂಮ-ವ್ಯ-ಕ್ತಿತ್ವ
ಭೂಮಾ-ನಂ-ದ-ವಿದೆ
ಭೂಮಾ-ನು-ಭವ
ಭೂಮಾ-ನು-ಭೂತಿ
ಭೂಮಾ-ವ-ಸ-ಪ-ತ್ನ-ಮೃದ್ಧಂ
ಭೂಮಿ
ಭೂಮಿಕೆ
ಭೂಮಿ-ಕೆಗೆ
ಭೂಮಿ-ಕೆಯ
ಭೂಮಿ-ಕೆ-ಯಲ್ಲಿ
ಭೂಮಿ-ಕೆ-ಯ-ಲ್ಲಿ-ರು-ವಾ-ಗಲೂ
ಭೂಮಿ-ಗಾಗಿ
ಭೂಮಿ-ಗಿಂತ
ಭೂಮಿಗೆ
ಭೂಮಿ-ದೇವಿ
ಭೂಮಿಯ
ಭೂಮಿ-ಯಂ-ತಹ
ಭೂಮಿ-ಯನ್ನು
ಭೂಮಿ-ಯ-ನ್ನೆಲ್ಲ
ಭೂಮಿ-ಯ-ಮೇಲೆ
ಭೂಮಿ-ಯಲ್ಲಿ
ಭೂಮಿ-ಯ-ಲ್ಲಿದ್ದು
ಭೂಮಿ-ಯ-ಲ್ಲಿ-ರುವ
ಭೂಮಿ-ಯ-ಲ್ಲಿ-ಲ್ಲದ
ಭೂಮಿ-ಯ-ಲ್ಲೆಲ್ಲಾ
ಭೂಮಿ-ಯಾ-ಗಿತ್ತು
ಭೂಮಿ-ಯಿಂದ
ಭೂಮಿಯೆ
ಭೂಮಿ-ರಾ-ಪೋ-ಽನಲೋ
ಭೂಯ
ಭೂಯಃ
ಭೂಯಾ-ದ್ಭಾ-ರ-ತ-ಪಂ-ಕಜಂ
ಭೂಯೋ-ನ್ಯ-ಜ್ಜ್ಞಾ-ತ-ವ್ಯ-ಮ-ವ-ಶಿ-ಷ್ಯತೇ
ಭೂಯೋಽಪಿ
ಭೂಯೋ-ಽಭಿ-ಜಾ-ಯತೇ
ಭೂರಿ
ಭೂರಿ-ಶ್ರ-ವ-ಸ್ಸಿನ
ಭೂಲೋಕ
ಭೂಲೋ-ಕದ
ಭೂಲೋ-ಕ-ದಲ್ಲಿ
ಭೂಷ-ಣ-ಗಳಲ್ಲಿ
ಭೂಷ-ಣ-ವಲ್ಲ
ಭೃಗು
ಭೃಗು-ರಹಂ
ಭೃತ್ಯ-ನಾ-ಗು-ವು-ದಕ್ಕೆ
ಭೇದ
ಭೇದ-ಗಳನ್ನು
ಭೇದ-ಭಾವ
ಭೇದ-ಮಿಮಂ
ಭೇದ-ವನ್ನು
ಭೇದ-ವಷ್ಟೆ
ಭೇದ-ವಿಲ್ಲ
ಭೇದಿ
ಭೇದಿ-ಸ-ಲಾ-ಗದ
ಭೇದಿ-ಸಲು
ಭೇದಿಸಿ
ಭೇದಿ-ಸಿ-ಕೊಂಡು
ಭೇದಿ-ಸಿ-ರು-ವನು
ಭೇದಿ-ಸುವ
ಭೇದಿ-ಸು-ವಂತೆ
ಭೇದಿ-ಸು-ವುದೊ
ಭೇರಿ
ಭೇರ್ಯಶ್ಚ
ಭೇಷ್
ಭೈಕ್ಷ್ಯ-ಮ-ಪೀಹ
ಭೈರವ
ಭೋಕ್ಚ
ಭೋಕ್ತಾ
ಭೋಕ್ತಾರಂ
ಭೋಕ್ತುಂ
ಭೋಕ್ತೃ
ಭೋಕ್ತೃತ್ವೇ
ಭೋಕ್ತೃವೂ
ಭೋಕ್ತೃ-ವೆಂದೂ
ಭೋಕ್ಷ್ಯಸೇ
ಭೋಗ
ಭೋಗ-ಕ್ಕಲ್ಲ
ಭೋಗಕ್ಕೆ
ಭೋಗ-ಗಳ
ಭೋಗ-ಗಳನ್ನು
ಭೋಗ-ಗಳೂ
ಭೋಗ-ಗ-ಳೆಲ್ಲ
ಭೋಗದ
ಭೋಗ-ದಲ್ಲಿ
ಭೋಗ-ದಾ-ಸೆಯ
ಭೋಗ-ದಿಂ-ದೇನು
ಭೋಗ-ವನ್ನು
ಭೋಗ-ವ-ಸ್ತು-ಗಳನ್ನು
ಭೋಗ-ವ-ಸ್ತು-ಗಳು
ಭೋಗ-ವ-ಸ್ತು-ವಿಗೂ
ಭೋಗ-ವಾ-ಸನೆ
ಭೋಗ-ವಾ-ಸ-ನೆ-ಯಿಂದ
ಭೋಗ-ವಿ-ಲ್ಲದ
ಭೋಗವೇ
ಭೋಗಾ
ಭೋಗಾಃ
ಭೋಗಾನ್
ಭೋಗಾ-ನ್ವೇ-ಷಣೆ
ಭೋಗಿ
ಭೋಗಿಗೆ
ಭೋಗಿಸಿ
ಭೋಗಿ-ಸುವ
ಭೋಗಿ-ಸು-ವ-ವರು
ಭೋಗಿ-ಸು-ವು-ದಕ್ಕೆ
ಭೋಗಿ-ಸು-ವುದು
ಭೋಗೀ
ಭೋಗೈ-ಶ್ವ-ರ್ಯ-ಗ-ತಿಂ
ಭೋಗೈ-ಶ್ವ-ರ್ಯ-ದಲ್ಲಿ
ಭೋಗೈ-ಶ್ವ-ರ್ಯ-ಪ್ರ-ಸ-ಕ್ತಾ-ನಾಂ
ಭೋಗೈ-ಶ್ವ-ರ್ಯ-ವನ್ನು
ಭೋಗ್ಯೆ-ರ್ಜೀ-ವಿ-ತೇನ
ಭೋಜನ
ಭೋಜನಂ
ಭೋಜ-ನ-ವನ್ನು
ಭೋಜ-ನ-ವಾ-ಗು-ವುದು
ಭೋರ್ಗ-ರೆದು
ಭೋರ್ಗ-ರೆ-ಯು-ತ್ತಿ-ರುವ
ಭೌತಿಕ
ಭೌತಿ-ಕ-ಪ್ರ-ಪಂ-ಚ-ದ-ಮೇಲೆ
ಭೌತಿ-ಕ-ವಾದ
ಭ್ರಮ-ತೀವ
ಭ್ರಮ-ರದ
ಭ್ರಮ-ರ-ವನ್ನು
ಭ್ರಮ-ರ-ವಾಗಿ
ಭ್ರಮ-ರ-ವಾ-ಗು-ವಂತೆ
ಭ್ರಮ-ರವೇ
ಭ್ರಮಿಸಿ
ಭ್ರಮಿ-ಸಿ-ದಂತೆ
ಭ್ರಮಿ-ಸಿ-ದೊ-ಡ-ನೆಯೆ
ಭ್ರಮಿ-ಸಿ-ದ್ದೆವು
ಭ್ರಮಿ-ಸಿ-ರು-ವೆವು
ಭ್ರಮಿ-ಸು-ತ್ತಾನೆ
ಭ್ರಮಿ-ಸು-ತ್ತಿ-ರು-ವನು
ಭ್ರಮಿ-ಸು-ತ್ತೇವೆ
ಭ್ರಮಿ-ಸು-ವನು
ಭ್ರಮಿ-ಸು-ವುದೇ
ಭ್ರಮೆ
ಭ್ರಮೆ-ಯಾ-ಗಿ-ದ್ದರೆ
ಭ್ರಮೆ-ಯಿಂದ
ಭ್ರಷ್ಟ
ಭ್ರಷ್ಟ-ನಾದ
ಭ್ರಷ್ಟ-ನಾ-ದ-ನಲ್ಲ
ಭ್ರಷ್ಟ-ನಾ-ದರೆ
ಭ್ರಾಂತ-ರಾಗಿ
ಭ್ರಾಂತ-ರಾ-ಗಿ-ದ್ದಾರೆ
ಭ್ರಾಂತಿ
ಭ್ರಾಂತಿ-ಗೊ-ಳಿ-ಸು-ತ್ತಿ-ರು-ವಂತೆ
ಭ್ರಾಂತಿ-ಗೊ-ಳಿ-ಸು-ವಂತೆ
ಭ್ರಾಂತಿ-ಜೀ-ವ-ನ-ದಲ್ಲಿ
ಭ್ರಾಂತಿ-ಪ-ಡು-ವು-ದಿಲ್ಲ
ಭ್ರಾಂತಿಯ
ಭ್ರಾಂತಿ-ಯನ್ನು
ಭ್ರಾಂತಿ-ಯಲ್ಲಿ
ಭ್ರಾಂತಿ-ಯಾ-ಗು-ವುದು
ಭ್ರಾಂತಿ-ಯಾ-ದರೆ
ಭ್ರಾಂತಿ-ಯಿಂದ
ಭ್ರಾಂತಿಯೂ
ಭ್ರಾತೄನ್
ಭ್ರಾಮ-ಯನ್
ಭ್ರುವೋಃ
ಭ್ರುವೋ-ರ್ಮಧ್ಯೇ
ಮಂಕಾಗಿ
ಮಂಗನ
ಮಂಗ-ಮಾ-ಯ-ವಾ-ಗು-ವುದು
ಮಂಗ-ಮಾ-ಯ-ವಾ-ಗು-ವುವು
ಮಂಗಳ
ಮಂಗ-ಳ-ದಿಂದ
ಮಂಗ-ಳ-ವನ್ನು
ಮಂಗ-ಳ-ವಾದ
ಮಂಗ-ಳ-ವೊಂದೇ
ಮಂಚಕ್ಕೆ
ಮಂಚದ
ಮಂಜನ್ನು
ಮಂಜಾ-ಗು-ವುದೆ
ಮಂಜಿನ
ಮಂಜಿ-ನಂತೆ
ಮಂಜಿ-ನ-ಗಡ್ಡೆ
ಮಂಜಿ-ನ-ಗ-ಡ್ಡೆ-ಯಲ್ಲಿ
ಮಂಜಿ-ನ-ಗ-ಡ್ಡೆ-ಯಾಗಿ
ಮಂಜು
ಮಂಜು-ಗೆ-ಡ್ಡೆ-ಯಂತೆ
ಮಂಡಿ-ಸಿ-ರು-ವಾಗ
ಮಂಡಿ-ಸಿ-ರು-ವುದನ್ನು
ಮಂತವ್ಯಃ
ಮಂತಾ-ದು-ವು-ಗಳನ್ನು
ಮಂತ್ರ
ಮಂತ್ರ-ಕ್ಕಿಂತ
ಮಂತ್ರ-ಗಳ
ಮಂತ್ರ-ಗಳನ್ನು
ಮಂತ್ರ-ಗಾ-ರನೂ
ಮಂತ್ರ-ಜ-ಲ-ವನ್ನು
ಮಂತ್ರದ
ಮಂತ್ರ-ದೇ-ವ-ತೆಗೆ
ಮಂತ್ರ-ದೇ-ವ-ತೆಗೇ
ಮಂತ್ರ-ದ್ರಷ್ಟ
ಮಂತ್ರ-ದ್ರ-ಷ್ಟಾ-ರರು
ಮಂತ್ರ-ವನ್ನು
ಮಂತ್ರವೂ
ಮಂತ್ರ-ಹೀನ
ಮಂತ್ರ-ಹೀ-ನ-ಮ-ದ-ಕ್ಷಿ-ಣಮ್
ಮಂತ್ರಾ-ಕ್ಷತೆ
ಮಂತ್ರಾ-ಕ್ಷ-ತೆ-ಗಳನ್ನು
ಮಂತ್ರಿ
ಮಂತ್ರಿ-ಯಂತೆ
ಮಂತ್ರೋ-ಚ್ಚಾ-ರ-ಣೆ-ಯನ್ನು
ಮಂತ್ರೋ-ಽಹ-ಮ-ಹ-ಮೇ-ವಾ-ಜ್ಯ-ಮ-ಹ-ಮ-ಗ್ನಿ-ರ-ಹಮ್
ಮಂದ-ಮ-ತಿ-ಗಳನ್ನು
ಮಂದರ
ಮಂದ-ವಾಗಿ
ಮಂದ-ಸ್ವ-ಭಾ-ವ-ವು-ಳ್ಳ-ವನು
ಮಂದ-ಹಾ-ಸ-ದಿಂದ
ಮಂದ-ಹಾ-ಸ-ಪೂ-ರಿ-ತ-ನಾದ
ಮಂದಾನ್
ಮಂದಿ
ಮಂದಿ-ಗಳು
ಮಂದಿಗೆ
ಮಂದಿ-ರ-ದಲ್ಲಿ
ಮಂದೆ-ಯಂತೆ
ಮಂದೆಯೋ
ಮಂಸ್ಯಂತೇ
ಮಕರ
ಮಕ-ರಂ-ದ-ವನ್ನು
ಮಕ-ರ-ಶ್ಚಾಸ್ಮಿ
ಮಕ್ಕಳ
ಮಕ್ಕ-ಳನ್ನು
ಮಕ್ಕ-ಳಾ-ಗಿ-ದ್ದಾಗ
ಮಕ್ಕ-ಳಾ-ಗಿಯೇ
ಮಕ್ಕ-ಳಾಟ
ಮಕ್ಕ-ಳಾ-ಟ-ದಂತೆ
ಮಕ್ಕ-ಳಾ-ಟ-ವಲ್ಲ
ಮಕ್ಕ-ಳಾದ
ಮಕ್ಕ-ಳಿಗೆ
ಮಕ್ಕ-ಳಿ-ಗೆಲ್ಲಾ
ಮಕ್ಕ-ಳಿ-ದ್ದರು
ಮಕ್ಕ-ಳಿ-ದ್ದ-ವ-ರಿಗೆ
ಮಕ್ಕ-ಳಿ-ಲ್ಲ-ದ-ವ-ರಿಗೆ
ಮಕ್ಕ-ಳಿ-ಲ್ಲ-ದ-ವರು
ಮಕ್ಕಳು
ಮಕ್ಕ-ಳು-ಆಸ್ತಿ
ಮಕ್ಕ-ಳು-ಗಳು
ಮಕ್ಕಳೂ
ಮಕ್ಕ-ಳೆಲ್ಲಾ
ಮಕ್ಕಳೇ
ಮಕ್ಕ-ಳೊ-ಡನೆ
ಮಗ
ಮಗ-ದೊಂದು
ಮಗನ
ಮಗ-ನನ್ನು
ಮಗ-ನಾಗಿ
ಮಗ-ನಾ-ಗಿಯೋ
ಮಗ-ನಾದ
ಮಗ-ನಾ-ದರೊ
ಮಗ-ನಾ-ದರೋ
ಮಗ-ನಾ-ದ್ದ-ರಿಂದ
ಮಗ-ನಿ-ಗಿಂತ
ಮಗ-ನಿಗೆ
ಮಗ-ನಿಗೋ
ಮಗನೂ
ಮಗನೆ
ಮಗನೇ
ಮಗ-ಳಾ-ಗಿಯೋ
ಮಗ-ಳಾದ
ಮಗಳು
ಮಗು
ಮಗು-ವನ್ನು
ಮಗು-ವಲ್ಲ
ಮಗು-ವಾ-ದರೊ
ಮಗು-ವಾ-ದಾಗ
ಮಗುವಿ
ಮಗು-ವಿ-ಗಿಂ-ತಲೂ
ಮಗು-ವಿಗೆ
ಮಗು-ವಿ-ಗೇನೂ
ಮಗು-ವಿ-ಗೇನೋ
ಮಗು-ವಿನ
ಮಗು-ವಿ-ನಂತೆ
ಮಗು-ವಿ-ನಿಂದ
ಮಗುವು
ಮಗು-ವೆಂದೂ
ಮಗು-ವೊಂ-ದಿದೆ
ಮಗು-ವೊಂದು
ಮಗುವೋ
ಮಗ್ನ-ನಾ-ಗಿ-ದ್ದರೂ
ಮಗ್ನ-ನಾ-ಗಿ-ರು-ವನು
ಮಗ್ನ-ನಾ-ಗಿ-ರು-ವನೋ
ಮಗ್ನ-ನಾ-ಗಿ-ರು-ವುದು
ಮಗ್ನ-ರಾ-ಗಿ-ರು-ವರೋ
ಮಚ್ಚಿತ್ತಃ
ಮಚ್ಚಿತ್ತಾ
ಮಚ್ಚಿತ್ತೋ
ಮಚ್ಚೆ
ಮಚ್ಚೆ-ಯನ್ನು
ಮಜ್ಜಿಗೆ
ಮಜ್ಜಿ-ಗೆ-ಯನ್ನು
ಮಜ್ಜಿ-ಗೆ-ಯ-ಲ್ಲಿ-ರುವ
ಮಟ್ಟ
ಮಟ್ಟಕ್ಕೆ
ಮಟ್ಟದ
ಮಟ್ಟ-ದಲ್ಲಿ
ಮಟ್ಟ-ದ-ಲ್ಲಿ-ದ್ದಾಗ
ಮಟ್ಟದ್ದು
ಮಟ್ಟದ್ದೆ
ಮಟ್ಟ-ವನ್ನು
ಮಟ್ಟಿಗೆ
ಮಡಕೆ
ಮಡ-ಕೆ-ಕು-ಡಿ-ಕೆ-ಗಳನ್ನು
ಮಡ-ಕೆ-ಕು-ಡಿ-ಕೆಯ
ಮಡ-ಕೆಯ
ಮಡ-ಕೆ-ಯಂತೆ
ಮಡ-ಕೆ-ಯನ್ನು
ಮಡ-ಕೆ-ಯ-ಲ್ಲಿ-ರುವ
ಮಡ-ಕೆಯು
ಮಡಿ
ಮಡಿಕೆ
ಮಡಿ-ಕೆ-ಕು-ಡಿ-ಕೆ-ಗಳು
ಮಡಿ-ಕೆ-ಗಳನ್ನೆಲ್ಲಾ
ಮಡಿ-ಕೆ-ಯ-ಲ್ಲಿ-ಡು-ವಳು
ಮಡಿದ
ಮಡಿ-ದರೂ
ಮಡಿ-ದರೆ
ಮಡಿ-ಯ-ಬ-ಹುದು
ಮಡಿ-ಯಾಗಿ
ಮಡಿ-ಯಾ-ಗಿ-ರ-ಬೇಕು
ಮಡಿ-ಯಿಂದ
ಮಡಿ-ಯು-ವನು
ಮಡಿ-ಯು-ವು-ದಕ್ಕೆ
ಮಡಿ-ಯು-ವುದು
ಮಡಿಯೇ
ಮಡಿ-ಲಿ-ನಲ್ಲಿ
ಮಣಿ
ಮಣಿ-ಗಣಾ
ಮಣಿ-ಗ-ಳಂತೆ
ಮಣಿ-ಗಳು
ಮಣಿ-ಗ-ಳೊಂದೇ
ಮಣಿಗೂ
ಮಣಿಗೆ
ಮಣಿ-ದರು
ಮಣಿ-ದರೆ
ಮಣಿ-ಪು-ಷ್ಪ-ಗ-ಳೆಂಬ
ಮಣಿ-ಯದೆ
ಮಣಿ-ಯ-ಬೇ-ಕಾ-ದರೆ
ಮಣಿ-ಯ-ಬೇಕು
ಮಣಿ-ಯು-ವಂತೆ
ಮಣಿ-ಯು-ವುದು
ಮಣಿ-ಯು-ವೆವು
ಮಣ್ಣನ್ನು
ಮಣ್ಣಿನ
ಮಣ್ಣಿ-ನಲ್ಲಿ
ಮಣ್ಣಿ-ನಿಂದ
ಮಣ್ಣು
ಮಣ್ಣೆ-ರಚು
ಮಣ್ಣೆ-ರ-ಚು-ವುದು
ಮಣ್ಣೆ-ರೆ-ಚಲು
ಮತ
ಮತಂ
ಮತಃ
ಮತ-ಭ್ರಾಂತಿ
ಮತ-ಮಿದಂ
ಮತ-ಮು-ತ್ತ-ಮಮ್
ಮತಮ್
ಮತ-ವನ್ನು
ಮತಾ
ಮತಾಃ
ಮತಿಃ
ಮತಿಗೆ
ಮತೇ
ಮತೋ
ಮತೋ-ಽಧಿಕಃ
ಮತ್ಕ-ರ್ಮ-ಕೃ-ನ್ಮ-ತ್ಪ-ರಮೋ
ಮತ್ಕ-ರ್ಮ-ಪ-ರಮೋ
ಮತ್ತ
ಮತ್ತಃ
ಮತ್ತ-ರಾಗಿ
ಮತ್ತರ್ಧ
ಮತ್ತಷ್ಚು
ಮತ್ತ-ಷ್ಟನ್ನು
ಮತ್ತಷ್ಟು
ಮತ್ತಾ-ರದೊ
ಮತ್ತಾ-ರದೋ
ಮತ್ತಾ-ರನ್ನೋ
ಮತ್ತಾ-ರಾ-ದರೂ
ಮತ್ತಾ-ರಿಗೊ
ಮತ್ತಾ-ರಿಗೋ
ಮತ್ತಾರೂ
ಮತ್ತಾರೊ
ಮತ್ತಾರೋ
ಮತ್ತಾವ
ಮತ್ತಾ-ವು-ದಕ್ಕೂ
ಮತ್ತಾ-ವುದನ್ನು
ಮತ್ತಾ-ವು-ದನ್ನೂ
ಮತ್ತಾ-ವು-ದನ್ನೋ
ಮತ್ತಾ-ವು-ದ-ರಲ್ಲೋ
ಮತ್ತಾ-ವು-ದ-ರಿಂ-ದಲೂ
ಮತ್ತಾ-ವು-ದಾ-ದರೂ
ಮತ್ತಾ-ವುದು
ಮತ್ತಾ-ವುದೂ
ಮತ್ತಾ-ವುದೊ
ಮತ್ತಾ-ವುದೋ
ಮತ್ತಿ-ನಲ್ಲಿ
ಮತ್ತು
ಮತ್ತು-ಅದು
ಮತ್ತು-ಇ-ತ-ರ-ರಿಂದ
ಮತ್ತು-ಸ್ಥಿತಿ
ಮತ್ತೂ
ಮತ್ತೆ
ಮತ್ತೆ-ರಡು
ಮತ್ತೆ-ಲ್ಲಿಯೂ
ಮತ್ತೆಲ್ಲೊ
ಮತ್ತೆಷ್ಟು
ಮತ್ತೇ-ನನ್ನೊ
ಮತ್ತೇ-ನನ್ನೋ
ಮತ್ತೇನು
ಮತ್ತೇನೊ
ಮತ್ತೊಂ
ಮತ್ತೊಂ-ದ-ಕ್ಕಾಗಿ
ಮತ್ತೊಂ-ದ-ಕ್ಕಿಂತ
ಮತ್ತೊಂ-ದಕ್ಕೂ
ಮತ್ತೊಂ-ದಕ್ಕೆ
ಮತ್ತೊಂ-ದನ್ನು
ಮತ್ತೊಂ-ದನ್ನೂ
ಮತ್ತೊಂ-ದರ
ಮತ್ತೊಂ-ದ-ರಲ್ಲಿ
ಮತ್ತೊಂ-ದ-ರಿಂದ
ಮತ್ತೊಂ-ದ-ರಿಂ-ದ-ಲಾ-ದರೂ
ಮತ್ತೊಂ-ದಾ-ದರೂ
ಮತ್ತೊಂ-ದಿ-ರ-ಲಾ-ರದು
ಮತ್ತೊಂ-ದಿಲ್ಲ
ಮತ್ತೊಂ-ದಿ-ಲ್ಲ-ವೆಂದು
ಮತ್ತೊಂದು
ಮತ್ತೊಂದೂ
ಮತ್ತೊಂ-ದೆಡೆ
ಮತ್ತೊಂ-ದೆ-ಡೆಗೆ
ಮತ್ತೊಬ್ಬ
ಮತ್ತೊ-ಬ್ಬನ
ಮತ್ತೊ-ಬ್ಬ-ನದು
ಮತ್ತೊ-ಬ್ಬ-ನದೇ
ಮತ್ತೊ-ಬ್ಬ-ನನ್ನು
ಮತ್ತೊ-ಬ್ಬ-ನಿಂದ
ಮತ್ತೊ-ಬ್ಬ-ನಿಗೆ
ಮತ್ತೊ-ಬ್ಬ-ನಿಲ್ಲ
ಮತ್ತೊ-ಬ್ಬನು
ಮತ್ತೊ-ಬ್ಬನೇ
ಮತ್ತೊ-ಬ್ಬರ
ಮತ್ತೊ-ಬ್ಬ-ರ-ದನ್ನು
ಮತ್ತೊ-ಬ್ಬ-ರನ್ನು
ಮತ್ತೊ-ಬ್ಬ-ರಿಂದ
ಮತ್ತೊ-ಬ್ಬ-ರಿಗೆ
ಮತ್ತೊ-ಬ್ಬ-ರಿಲ್ಲ
ಮತ್ತೊ-ಬ್ಬರು
ಮತ್ತೊ-ಬ್ಬ-ರೊ-ಡನೆ
ಮತ್ತೊಮ್ಮೆ
ಮತ್ಪರಃ
ಮತ್ಪ-ರ-ನಾ-ಗಿ-ರ-ಬೇಕು
ಮತ್ಪ-ರಮಾ
ಮತ್ಪ-ರ-ರಾಗಿ
ಮತ್ಪ-ರಾಃ
ಮತ್ಪ-ರಾ-ಯಣಃ
ಮತ್ಪ-ರಾ-ಯ-ಣ-ನಾಗಿ
ಮತ್ಪ-ರಾ-ಯ-ಣ-ನಾ-ಗಿ-ರ-ಬೇಕು
ಮತ್ಪ-ರಾ-ಯ-ಣ-ರಾಗಿ
ಮತ್ಪ್ರ-ಸಾ-ದಾ-ತ್ತ-ರಿ-ಷ್ಯಸಿ
ಮತ್ಪ್ರ-ಸಾ-ದಾ-ದ-ವಾ-ಪ್ನೋತಿ
ಮತ್ವಾ
ಮತ್ಸಂ-ಸ್ಥಾ-ಮ-ಧಿ-ಗ-ಚ್ಛತಿ
ಮತ್ಸಾ-್ಥನಿ
ಮತ್ಸಾ-್ಥ-ನೀ-ತ್ಯು-ಪ-ಧಾ-ರಯ
ಮಥುರ
ಮಥು-ರ-ನಾಥ
ಮಥು-ರ-ನಾ-ಥನ
ಮಥು-ರ-ನಾ-ಥ-ನಿಗೆ
ಮಥು-ರ-ನಿಗೆ
ಮದ
ಮದ-ಗಳಿಂದ
ಮದ-ದಲ್ಲಿ
ಮದ-ದಿಂದ
ಮದ-ನ-ಮೋ-ಹನ
ಮದ-ನು-ಗ್ರ-ಹಾಯ
ಮದ-ಮೇವ
ಮದ-ರ್ಥ-ಮಪಿ
ಮದರ್ಥೇ
ಮದ-ರ್ಪ-ಣಮ್
ಮದಾ-ಶ್ರಯಃ
ಮದುವೆ
ಮದು-ವೆ-ಯಾಗಿ
ಮದು-ವೆ-ಯಾ-ಗು-ವಾ-ಗಲೂ
ಮದು-ವೆ-ಯಾ-ದನು
ಮದ್ಗ-ತ-ಪ್ರಾಣಾ
ಮದ್ಗ-ತೇ-ನಾಂ-ತ-ರಾ-ತ್ಮನಾ
ಮದ್ದನ್ನು
ಮದ್ದಿನ
ಮದ್ದಿ-ನಿಂದ
ಮದ್ದಿ-ರು-ವು-ದ-ರಿಂದ
ಮದ್ದು
ಮದ್ದೇ
ಮದ್ಭಕ್ತ
ಮದ್ಭಕ್ತಃ
ಮದ್ಭಕ್ತಾ
ಮದ್ಭ-ಕ್ತಿಂ
ಮದ್ಭ-ಕ್ತೇ-ಷ್ವ-ಭಿ-ಧಾ-ಸ್ಯತಿ
ಮದ್ಭಕ್ತೋ
ಮದ್ಭಾವಂ
ಮದ್ಭಾ-ವ-ಮಾ-ಗ-ತಾಃ
ಮದ್ಭಾವಾ
ಮದ್ಭಾ-ವಾ-ಯೋ-ಪ-ಪ-ದ್ಯತೇ
ಮದ್ಯಾ-ಜಿ-ನೋಽಪಿ
ಮದ್ಯಾಜೀ
ಮದ್ಯೋ-ಗ-ಮಾ-ಶ್ರಿತಃ
ಮದ್ವ್ಯ-ಪಾ-ಶ್ರಯಃ
ಮಧು-ಪಾ-ನ-ಮಾಡಿ
ಮಧು-ಮೇಹ
ಮಧು-ಮೇ-ಹ-ಗಳು
ಮಧು-ರ-ವಾ-ಗಿ-ರ-ಬೇಕು
ಮಧು-ಸೂ-ದನ
ಮಧು-ಸೂ-ದನಃ
ಮಧ್ಯ
ಮಧ್ಯಂ
ಮಧ್ಯದ
ಮಧ್ಯ-ದಲ್ಲಿ
ಮಧ್ಯ-ದ-ಲ್ಲಿ-ದ್ದರೂ
ಮಧ್ಯ-ದ-ಲ್ಲಿಯೇ
ಮಧ್ಯ-ದ-ಲ್ಲಿ-ರ-ಬೇಕು
ಮಧ್ಯ-ದ-ಲ್ಲಿ-ರುವ
ಮಧ್ಯ-ದ-ಲ್ಲಿ-ರು-ವನು
ಮಧ್ಯ-ದ-ಲ್ಲಿ-ರು-ವು-ದಾ-ಗಿಯೂ
ಮಧ್ಯ-ದ-ಲ್ಲಿ-ರು-ವುದು
ಮಧ್ಯ-ದ-ಲ್ಲಿ-ರು-ವೆವು
ಮಧ್ಯ-ಭಾ-ಗ-ದಲ್ಲಿ
ಮಧ್ಯಮ
ಮಧ್ಯ-ವರ್ತಿ
ಮಧ್ಯ-ವ-ರ್ತಿಗೆ
ಮಧ್ಯ-ವ-ರ್ತಿಯ
ಮಧ್ಯ-ವ-ರ್ತಿ-ಯನ್ನು
ಮಧ್ಯ-ವಾ-ಗಲೀ
ಮಧ್ಯಸ್ಥ
ಮಧ್ಯಾ-ಹ್ನದ
ಮಧ್ಯೆ
ಮಧ್ಯೆ-ಮಧ್ಯೆ
ಮಧ್ಯೇ
ಮಧ್ವಾ-ಚಾ-ರ್ಯರು
ಮನ
ಮನಃ
ಮನಃ-ಪ್ರ-ಸಾದಃ
ಮನಃ-ಪ್ರಾ-ಣೇಂ-ದ್ರಿ-ಯ-ಕ್ರಿ-ಯಾಃ
ಮನಃ-ಶಾ-ಸ್ತ್ರದ
ಮನಃ-ಷ-ಷ್ಠಾ-ನೀಂ-ದ್ರಿ-ಯಾಣಿ
ಮನ-ಗಂ-ಡ-ವನು
ಮನ-ಗಂ-ಡ-ವರು
ಮನ-ಗಂ-ಡಿದೆ
ಮನ-ಗಂ-ಡಿರು
ಮನ-ಗಂ-ಡಿ-ರುವ
ಮನ-ಗಂ-ಡಿ-ರು-ವನು
ಮನ-ಗಳನ್ನು
ಮನ-ಗೊ-ಡನು
ಮನದ
ಮನ-ದ-ಟ್ಟಾ-ದ-ಮೇಲೆ
ಮನ-ದ-ಣಿಯ
ಮನ-ದಲ್ಲಿ
ಮನ-ದ-ಲ್ಲಿಯೇ
ಮನ-ದ-ಲ್ಲಿ-ರ-ಬೇಕು
ಮನ-ದಲ್ಲೇ
ಮನನ
ಮನ-ನದ
ಮನ-ನ-ದಿಂದ
ಮನ-ಬಂ-ದಂತೆ
ಮನ-ಮೋ-ಹ-ಕ-ವಾದ
ಮನ-ರಂ-ಜ-ನೆಯ
ಮನ-ವ-ಸ್ತಥಾ
ಮನವೇ
ಮನ-ಶ್ಚಂ-ಚ-ಲ-ಮ-ಸ್ಥಿ-ರಮ್
ಮನ-ಶ್ಚಾಯಂ
ಮನ-ಶ್ಚಾಸ್ಮಿ
ಮನ-ಶ್ಶಾಂತಿ
ಮನ-ಷ್ಯ-ರಾ-ಗಲಿ
ಮನ-ಸಸ್ತು
ಮನಸಾ
ಮನ-ಸಾ-ಚ-ಲೇನ
ಮನ-ಸೈ-ವೇಂ-ದ್ರಿ-ಯ-ಗ್ರಾಮಂ
ಮನ-ಸೋತು
ಮನ-ಸೋ-ಲು-ತ್ತೇವೆ
ಮನ-ಸೋ-ಲು-ವುದೂ
ಮನ-ಸ್ತಾ-ಪಕ್ಕೆ
ಮನ-ಸ್ಶ-ಕ್ತಿ-ಯನ್ನು
ಮನ-ಸ್ಸ-ನ್ನಿಟ್ಟು
ಮನ-ಸ್ಸನ್ನು
ಮನ-ಸ್ಸ-ನ್ನೆಲ್ಲ
ಮನ-ಸ್ಸ-ನ್ನೆಲ್ಲಾ
ಮನ-ಸ್ಸನ್ನೇ
ಮನ-ಸ್ಸಾ-ಗು-ವುದು
ಮನ-ಸ್ಸಾ-ದರೂ
ಮನ-ಸ್ಸಾ-ದರೆ
ಮನ-ಸ್ಸಾ-ದರೋ
ಮನ-ಸ್ಸಾ-ಯಿತು
ಮನಸ್ಸಿ
ಮನ-ಸ್ಸಿ-ಗಿಂತ
ಮನ-ಸ್ಸಿಗೂ
ಮನ-ಸ್ಸಿಗೆ
ಮನ-ಸ್ಸಿ-ಗೇನೊ
ಮನ-ಸ್ಸಿ-ಟ್ಟರೆ
ಮನ-ಸ್ಸಿ-ಟ್ಟ-ವನೇ
ಮನ-ಸ್ಸಿಟ್ಟು
ಮನ-ಸ್ಸಿಡು
ಮನ-ಸ್ಸಿದೆ
ಮನ-ಸ್ಸಿ-ದ್ದರೆ
ಮನ-ಸ್ಸಿ-ದ್ದ-ರೇನೇ
ಮನ-ಸ್ಸಿನ
ಮನ-ಸ್ಸಿ-ನ-ಮೇಲೆ
ಮನ-ಸ್ಸಿ-ನಲ್ಲಿ
ಮನ-ಸ್ಸಿ-ನ-ಲ್ಲಿ-ಟ್ಟಿ-ರು-ವನು
ಮನ-ಸ್ಸಿ-ನ-ಲ್ಲಿ-ಟ್ಟು-ಕೊಂ-ಡಿಲ್ಲ
ಮನ-ಸ್ಸಿ-ನ-ಲ್ಲಿ-ಟ್ಟು-ಕೊಂಡು
ಮನ-ಸ್ಸಿ-ನ-ಲ್ಲಿದೆ
ಮನ-ಸ್ಸಿ-ನ-ಲ್ಲಿದ್ದ
ಮನ-ಸ್ಸಿ-ನ-ಲ್ಲಿ-ದ್ದರೂ
ಮನ-ಸ್ಸಿ-ನ-ಲ್ಲಿ-ದ್ದರೆ
ಮನ-ಸ್ಸಿ-ನ-ಲ್ಲಿಯೂ
ಮನ-ಸ್ಸಿ-ನ-ಲ್ಲಿಯೇ
ಮನ-ಸ್ಸಿ-ನ-ಲ್ಲಿರ
ಮನ-ಸ್ಸಿ-ನ-ಲ್ಲಿ-ರುವ
ಮನ-ಸ್ಸಿ-ನ-ಲ್ಲಿ-ರು-ವನು
ಮನ-ಸ್ಸಿ-ನ-ಲ್ಲಿ-ರು-ವು-ದ-ರಿಂದ
ಮನ-ಸ್ಸಿ-ನಲ್ಲೂ
ಮನ-ಸ್ಸಿ-ನಲ್ಲೆ
ಮನ-ಸ್ಸಿ-ನಷ್ಟು
ಮನ-ಸ್ಸಿ-ನಿಂದ
ಮನ-ಸ್ಸಿ-ನಿಂ-ದಲೆ
ಮನ-ಸ್ಸಿ-ನಿಂ-ದಲೇ
ಮನ-ಸ್ಸಿ-ನೊಂ-ದಿಗೆ
ಮನ-ಸ್ಸಿ-ನೊ-ಳಗೆ
ಮನ-ಸ್ಸಿ-ರ-ಬೇಕು
ಮನ-ಸ್ಸಿ-ರು-ವ-ವ-ರಿಗೆ
ಮನ-ಸ್ಸಿಲ್ಲ
ಮನ-ಸ್ಸಿ-ಲ್ಲದ
ಮನ-ಸ್ಸಿ-ಲ್ಲ-ದಿ-ದ್ದರೆ
ಮನ-ಸ್ಸಿ-ಲ್ಲದೇ
ಮನಸ್ಸು
ಮನ-ಸ್ಸು-ಗಳನ್ನು
ಮನ-ಸ್ಸು-ಗಳನ್ನೆಲ್ಲಾ
ಮನ-ಸ್ಸು-ಗ-ಳಿಗೆ
ಮನ-ಸ್ಸು-ಮಾಡ
ಮನ-ಸ್ಸು-ಮಾಡಿ
ಮನ-ಸ್ಸುಳ್ಳ
ಮನ-ಸ್ಸು-ಳ್ಳ-ವ-ನಾಗಿ
ಮನ-ಸ್ಸು-ಳ್ಳ-ವ-ನಾ-ಗಿ-ರು-ವನು
ಮನ-ಸ್ಸು-ಳ್ಳ-ವ-ನಾಗು
ಮನ-ಸ್ಸು-ಳ್ಳ-ವನು
ಮನ-ಸ್ಸು-ಳ್ಳ-ವ-ರಾಗಿ
ಮನಸ್ಸೂ
ಮನಸ್ಸೆ
ಮನ-ಸ್ಸೆಂಬ
ಮನ-ಸ್ಸೆಲ್ಲ
ಮನ-ಸ್ಸೆಲ್ಲಾ
ಮನಸ್ಸೇ
ಮನ-ಸ್ಸೇನೂ
ಮನೀ-ಷಿಣಃ
ಮನೀ-ಷಿ-ಣಾಮ್
ಮನು
ಮನು-ಗಳು
ಮನು-ಜರು
ಮನು-ರಿಕ್ಷಾ
ಮನು-ವಿಗೆ
ಮನು-ಶಾ-ಸ್ತ್ರ-ದಲ್ಲಿ
ಮನುಷೃ
ಮನು-ಷೃ-ನಿಗೆ
ಮನು-ಷೃನೇ
ಮನು-ಷೃ-ರಂತೆ
ಮನುಷ್ಯ
ಮನು-ಷ್ಯ-ಎಂ-ದರೆ
ಮನು-ಷ್ಯ-ಜೀ-ವನ
ಮನು-ಷ್ಯನ
ಮನು-ಷ್ಯ-ನಂತೆ
ಮನು-ಷ್ಯ-ನಂ-ತೆಯೇ
ಮನು-ಷ್ಯ-ನನ್ನು
ಮನು-ಷ್ಯ-ನಲ್ಲ
ಮನು-ಷ್ಯ-ನಲ್ಲಿ
ಮನು-ಷ್ಯ-ನ-ಲ್ಲಿಯೇ
ಮನು-ಷ್ಯ-ನ-ಲ್ಲಿ-ರುವ
ಮನು-ಷ್ಯ-ನಾಗಿ
ಮನು-ಷ್ಯ-ನಾ-ಗಿ-ರ-ಬ-ಹುದು
ಮನು-ಷ್ಯ-ನಾ-ದರೂ
ಮನು-ಷ್ಯ-ನಿ-ಗಾ-ದರೊ
ಮನು-ಷ್ಯ-ನಿ-ಗಾ-ದರೋ
ಮನು-ಷ್ಯ-ನಿಗೆ
ಮನು-ಷ್ಯ-ನಿ-ಗೊ-ಬ್ಬ-ನಿಗೇ
ಮನು-ಷ್ಯನು
ಮನು-ಷ್ಯನೂ
ಮನು-ಷ್ಯನೇ
ಮನು-ಷ್ಯ-ನೇ-ಳು-ವನು
ಮನು-ಷ್ಯರ
ಮನು-ಷ್ಯ-ರಂತೆ
ಮನು-ಷ್ಯ-ರನ್ನು
ಮನು-ಷ್ಯ-ರ-ಲ್ಲದ
ಮನು-ಷ್ಯ-ರಲ್ಲಿ
ಮನು-ಷ್ಯ-ರ-ಲ್ಲಿ-ರುವ
ಮನು-ಷ್ಯ-ರಾ-ಗಲಿ
ಮನು-ಷ್ಯ-ರಾಗಿ
ಮನು-ಷ್ಯ-ರಾ-ಗಿ-ರು-ವುದನ್ನು
ಮನು-ಷ್ಯ-ರಾ-ಗು-ವು-ದಿಲ್ಲ
ಮನು-ಷ್ಯ-ರಿಗೆ
ಮನು-ಷ್ಯ-ರಿ-ಗೆಲ್ಲಾ
ಮನು-ಷ್ಯರು
ಮನು-ಷ್ಯ-ರೂ-ಪ-ವನ್ನು
ಮನು-ಷ್ಯ-ರೆಲ್ಲ
ಮನು-ಷ್ಯ-ರೆ-ಲ್ಲರೂ
ಮನು-ಷ್ಯರೇ
ಮನು-ಷ್ಯ-ಲೋಕ
ಮನು-ಷ್ಯ-ಲೋ-ಕ-ದಲ್ಲಿ
ಮನು-ಷ್ಯ-ಲೋ-ಕ-ದಲ್ಲೆ
ಮನು-ಷ್ಯ-ಲೋಕೇ
ಮನು-ಷ್ಯ-ವನ್ನು
ಮನು-ಷ್ಯಾಃ
ಮನು-ಷ್ಯಾ-ಣಾಂ
ಮನು-ಷ್ಯೇಷು
ಮನುಸ್ಸು
ಮನೆ
ಮನೆ-ಗಲ್ಲ
ಮನೆ-ಗಳನ್ನು
ಮನೆ-ಗಳಲ್ಲಿ
ಮನೆ-ಗ-ಳಿಗೆ
ಮನೆ-ಗ-ಳಿವೆ
ಮನೆಗೂ
ಮನೆಗೆ
ಮನೆಗೇ
ಮನೆ-ತ-ನಕ್ಕೆ
ಮನೆ-ಬಾ-ಗಿಲ
ಮನೆ-ಬಾ-ಗಿ-ಲನ್ನು
ಮನೆ-ಮ-ಠ-ಗ-ಳಿ-ರ-ಬೇಕು
ಮನೆ-ಮ-ನೆಗೆ
ಮನೆ-ಮ-ನೆ-ಯಲ್ಲಿ
ಮನೆ-ಮ-ನೆ-ಯ-ಲ್ಲಿ-ರುವ
ಮನೆ-ಮಾ-ಡಿ-ಕೊಂ-ಡಿತು
ಮನೆ-ಮಾ-ಡಿ-ಕೊಂ-ಡಿ-ರು-ವುದು
ಮನೆ-ಮಾ-ಡಿ-ಕೊಂ-ಡಿ-ರು-ವುವು
ಮನೆಯ
ಮನೆ-ಯಂ-ತಾ-ಗು-ವುದು
ಮನೆ-ಯಂತೆ
ಮನೆ-ಯನ್ನು
ಮನೆ-ಯ-ನ್ನೆಲ್ಲಾ
ಮನೆ-ಯ-ಲ್ಲಾ-ಗಲೀ
ಮನೆ-ಯಲ್ಲಿ
ಮನೆ-ಯ-ಲ್ಲಿಟ್ಟು
ಮನೆ-ಯ-ಲ್ಲಿ-ಡ-ಬ-ಹುದು
ಮನೆ-ಯ-ಲ್ಲಿಯೂ
ಮನೆ-ಯ-ಲ್ಲಿಯೇ
ಮನೆ-ಯ-ಲ್ಲಿ-ರುವ
ಮನೆ-ಯ-ಲ್ಲಿ-ರು-ವಾಗ
ಮನೆ-ಯ-ಲ್ಲಿ-ರು-ವೆವು
ಮನೆ-ಯಲ್ಲೇ
ಮನೆ-ಯ-ವ-ರಿಗೆ
ಮನೆ-ಯ-ವರು
ಮನೆ-ಯಾ-ಗ-ಬೇ-ಕಾ-ದರೆ
ಮನೆ-ಯಿಂದ
ಮನೆಯೂ
ಮನೆ-ಯೆಲ್ಲ
ಮನೆಯೇ
ಮನೆ-ಯೊಂ-ದ-ರಲ್ಲಿ
ಮನೆ-ಯೊ-ಳಗೆ
ಮನೋ
ಮನೋ-ಗ-ತಾನ್
ಮನೋ-ಜ್ಞ-ವಾಗಿ
ಮನೋ-ದಾ-ರ್ಢ್ಯ-ವನ್ನು
ಮನೋ-ದೌ-ರ್ಬಲ್ಯ
ಮನೋ-ದೌ-ರ್ಬ-ಲ್ಯದ
ಮನೋ-ದೌ-ರ್ಬ-ಲ್ಯ-ದಿಂದ
ಮನೋ-ಧರ್ಮ
ಮನೋ-ನಿ-ಗ್ರ-ಹ-ದಿಂದ
ಮನೋ-ನಿ-ರೋಧ
ಮನೋ-ಬುದ್ಧಿ
ಮನೋ-ಬು-ದ್ಧಿ-ಯು-ಳ್ಳ-ವನೋ
ಮನೋ-ಭಾವ
ಮನೋ-ಭಾ-ವ-ದ-ವ-ರಿಗೂ
ಮನೋ-ಭಾ-ವ-ವನ್ನು
ಮನೋ-ರ-ಥಮ್
ಮನೋ-ರು-ಚಿಯ
ಮನೋ-ರೋ-ಗ-ಗಳೂ
ಮನೋ-ವೃತ್ತಿ
ಮನೋ-ಶ-ಕ್ತಿ-ಯನ್ನು
ಮನೋ-ಹರ
ಮನೋ-ಹ-ರ-ವಾಗಿ
ಮನೋ-ಹ-ರ-ವಾ-ಗಿ-ದ್ದನು
ಮನೋ-ಹ-ರ-ವಾ-ಗಿ-ರಲಿ
ಮನೋ-ಹ-ರ-ವಾ-ಗಿ-ರು-ವುದು
ಮನೋ-ಹ-ರ-ವಾದ
ಮನ್ನಣೆ
ಮನ್ನಿಸಿ
ಮನ್ನಿಸು
ಮನ್ನಿ-ಸು-ತ್ತಾನೆ
ಮನ್ನಿ-ಸು-ವಂತೆ
ಮನ್ಮಥ
ಮನ್ಮನಾ
ಮನ್ಮಯಾ
ಮನ್ಯಂತೇ
ಮನ್ಯತೇ
ಮನ್ಯಸೇ
ಮನ್ಯೇ
ಮನ್ಯೇತ
ಮಮ
ಮಮ-ಕಾರ
ಮಮ-ಕಾ-ರ-ಗ-ಳಿ-ಲ್ಲ-ದ-ವನೋ
ಮಮ-ಕಾ-ರವೂ
ಮಮತೆ
ಮಮ-ತೆ-ಗಳನ್ನು
ಮಮ-ತೆಯ
ಮಮ-ತೆ-ಯನ್ನು
ಮಮ-ತೆ-ಯಿಂದ
ಮಮಾತ್ಮಾ
ಮಮಾ-ಪ-ನುದ್ಯಾ
ಮಮಾ-ವ್ಯ-ಯ-ಮ-ನು-ತ್ತ-ಮಮ್
ಮಮೈ-ವಾಂಶೋ
ಮಯ
ಮಯನೂ
ಮಯಾ
ಮಯಾ-ಧ್ಯ-ಕ್ಷೇಣ
ಮಯಾ-ನಘ
ಮಯಿ
ಮಯೈವ
ಮಯೈ-ವೈತೇ
ಮಯ್ಯ-ರ್ಪಿ-ತ-ಮ-ನೋ-ಬು-ದ್ಧಿ-ರ್ಮಾ-ಮೇ-ವೈ-ಷ್ಯ-ಸ್ಯ-ಸಂ-ಶಯಃ
ಮಯ್ಯ-ರ್ಪಿ-ತ-ಮ-ನೋ-ಬು-ದ್ಧಿ-ರ್ಯೋ-ಮ-ದ್ಭಕ್ತಃ
ಮಯ್ಯಾ-ವೇ-ಶಿ-ತ-ಚೇ-ತ-ಸಾಮ್
ಮಯ್ಯಾ-ವೇಶ್ಯ
ಮಯ್ಯಾ-ಸ-ಕ್ತ-ಮ-ನಾಃ
ಮಯ್ಯೇವ
ಮರ
ಮರಕ್ಕೆ
ಮರ-ಗ-ಳನ್ನೇ
ಮರ-ಗಳು
ಮರ-ಗಿ-ಡ-ಗ-ಳೆಲ್ಲಾ
ಮರಣ
ಮರ-ಣ-ಕಾ-ಲ-ದಲ್ಲಿ
ಮರ-ಣ-ಕಾ-ಲ-ದ-ಲ್ಲಿಯೂ
ಮರ-ಣ-ಕ್ಕಿಂತ
ಮರ-ಣಕ್ಕೂ
ಮರ-ಣಕ್ಕೆ
ಮರ-ಣ-ಗಳ
ಮರ-ಣ-ಗಳಿಂದ
ಮರ-ಣ-ಗ-ಳೊಂದು
ಮರ-ಣದ
ಮರ-ಣ-ದಲ್ಲಿ
ಮರ-ಣ-ದಿಂದ
ಮರ-ಣ-ವನ್ನು
ಮರ-ಣ-ವ-ನ್ನೈ-ದಿದ
ಮರ-ಣ-ವೆ-ನ್ನು-ವುದು
ಮರ-ಣ-ಸ-ಮ-ಯ-ದಲ್ಲಿ
ಮರ-ಣಾ-ತೀ-ತ-ರ-ನ್ನಾಗಿ
ಮರದ
ಮರ-ದಂತೆ
ಮರ-ದಲ್ಲಿ
ಮರ-ದ-ಲ್ಲಿ-ರು-ವನು
ಮರ-ದಲ್ಲೆ
ಮರ-ದಿಂದ
ಮರ-ದೊ-ಳಗೆ
ಮರಳ
ಮರ-ಳನ್ನು
ಮರ-ಳ-ಮೇಲೆ
ಮರಳಿ
ಮರ-ಳಿನ
ಮರ-ಳಿಸಿ
ಮರಳು
ಮರ-ಳು-ಕಾ-ಡಿ-ನಲ್ಲಿ
ಮರ-ಳು-ಗಾ-ಡಿ-ನಲ್ಲಿ
ಮರ-ವನ್ನು
ಮರ-ವನ್ನೇ
ಮರ-ವಾಗಿ
ಮರ-ವಾ-ದರೋ
ಮರವು
ಮರವೂ
ಮರಿ
ಮರಿ-ಗಳನ್ನು
ಮರಿ-ಗ-ಳಾ-ಗು-ತ್ತವೆ
ಮರಿ-ಗಳಿಂದ
ಮರಿ-ಗ-ಳಿಗೆ
ಮರಿ-ಗಳು
ಮರಿಗೆ
ಮರಿ-ಯನ್ನು
ಮರಿ-ಯಾ-ಗದೆ
ಮರಿ-ಹಾಕಿ
ಮರಿ-ಹಾ-ಕು-ವುದು
ಮರೀಚಿ
ಮರೀ-ಚಿ-ಕಾ-ಮ-ಯ-ವಾದ
ಮರೀ-ಚಿಕೆ
ಮರೀ-ಚಿ-ಕೆ-ಗಳನ್ನು
ಮರೀ-ಚಿ-ಕೆ-ಯಂತೆ
ಮರೀ-ಚಿ-ಕೆ-ಯನ್ನು
ಮರೀ-ಚಿ-ರ್ಮ-ರು-ತಾ-ಮಸ್ಮಿ
ಮರುಕ
ಮರು-ಕಕ್ಕೆ
ಮರು-ಕದ
ಮರು-ಕ-ದಿಂದ
ಮರು-ಕ-ಪ-ಡು-ವುದು
ಮರು-ಗನು
ಮರು-ಗ-ಬೇ-ಕಾ-ಗಿಲ್ಲ
ಮರುಗು
ಮರು-ಗು-ವನು
ಮರು-ಗು-ವು-ದಿಲ್ಲ
ಮರು-ಗು-ವುದೊ
ಮರು-ತ-ಶ್ಚೋ-ಷ್ಮ-ಪಾಶ್ಚ
ಮರು-ತ-ಸ್ತಥಾ
ಮರುತ್
ಮರು-ದ್ಗ-ಣ-ಗಳನ್ನು
ಮರು-ದ್ಗ-ಣ-ಗಳು
ಮರು-ದ್ಗ-ಣ-ಗ-ಳೆ-ಲ್ಲರೂ
ಮರು-ಧ್ವ-ನಿ-ಯಂತೆ
ಮರು-ಮಾ-ತಾ-ಡದೆ
ಮರು-ಮಾ-ತಿ-ಲ್ಲದೆ
ಮರು-ಳನೋ
ಮರು-ಳಾಗು
ಮರು-ಳು-ಗೊ-ಳಿ-ಸ-ಲಾ-ರವು
ಮರೆ
ಮರೆ-ತರೂ
ಮರೆ-ತರೆ
ಮರೆ-ತ-ವ-ನಲ್ಲ
ಮರೆ-ತ-ವ-ನಿಗೆ
ಮರೆ-ತಾಗ
ಮರೆ-ತಾ-ಗಲೇ
ಮರೆ-ತಿದ್ದ
ಮರೆ-ತಿ-ದ್ದಾನೆ
ಮರೆ-ತಿದ್ದೇ
ಮರೆ-ತಿರು
ಮರೆ-ತಿ-ರುವ
ಮರೆ-ತಿ-ರು-ವನು
ಮರೆ-ತಿ-ರು-ವರೋ
ಮರೆ-ತಿ-ರು-ವು-ದಿಲ್ಲ
ಮರೆ-ತಿ-ರು-ವುದು
ಮರೆ-ತಿ-ರು-ವೆವು
ಮರೆ-ತಿ-ರು-ವೆವೊ
ಮರೆ-ತಿಲ್ಲ
ಮರೆತು
ಮರೆ-ತು-ಹೋಗಿ
ಮರೆ-ತು-ಹೋ-ಗಿದೆ
ಮರೆ-ತು-ಹೋ-ಗಿ-ರ-ಲಿಲ್ಲ
ಮರೆ-ತು-ಹೋ-ಗು-ವಂತೆ
ಮರೆ-ತು-ಹೋ-ಗು-ವುದು
ಮರೆ-ತು-ಹೋ-ದರೆ
ಮರೆತೇ
ಮರೆ-ಮಾ-ಚಿ-ಕೊಂಡು
ಮರೆ-ಮಾ-ಚು-ವು-ದಕ್ಕೆ
ಮರೆಯ
ಮರೆ-ಯ-ಕೂ-ಡದು
ಮರೆ-ಯ-ದಂತೆ
ಮರೆ-ಯದು
ಮರೆ-ಯದೆ
ಮರೆ-ಯ-ಬ-ಹುದು
ಮರೆ-ಯ-ಬೇ-ಕಾ-ಗಿದೆ
ಮರೆ-ಯ-ಬೇಕು
ಮರೆ-ಯ-ಬೇಡಿ
ಮರೆ-ಯಲು
ಮರೆಯು
ಮರೆ-ಯು-ತ್ತಾನೆ
ಮರೆ-ಯು-ತ್ತಾಳೆ
ಮರೆ-ಯು-ತ್ತಿ-ರು-ವೆವು
ಮರೆ-ಯು-ತ್ತೇವೆ
ಮರೆ-ಯು-ತ್ತೇ-ವೆಯೊ
ಮರೆ-ಯು-ತ್ತೇ-ವೆಯೋ
ಮರೆ-ಯುವ
ಮರೆ-ಯು-ವನು
ಮರೆ-ಯು-ವರು
ಮರೆ-ಯು-ವ-ವ-ನಲ್ಲ
ಮರೆ-ಯು-ವ-ವ-ನಿಗೆ
ಮರೆ-ಯು-ವು-ದಕ್ಕೆ
ಮರೆ-ಯು-ವು-ದಿಲ್ಲ
ಮರೆ-ಯು-ವು-ದಿ-ಲ್ಲವೊ
ಮರೆ-ಯು-ವುದು
ಮರೆ-ಯು-ವೆವು
ಮರೆ-ಯು-ವೆವೊ
ಮರೆ-ಯು-ವೆವೋ
ಮರೆ-ವನ್ನು
ಮರೆ-ವಿ-ನಲ್ಲಿ
ಮರೆ-ವಿ-ನಿಂದ
ಮರೆವು
ಮರೆ-ವು-ಗಳಿಂದ
ಮರೆ-ಸಿ-ರು-ವುದು
ಮರೆ-ಸುವ
ಮರೆ-ಸು-ವು-ದಕ್ಕೆ
ಮರೆ-ಸು-ವುದು
ಮರೆ-ಸು-ವುವು
ಮರ್ತ್ಯ
ಮರ್ತ್ಯ-ನಲ್ಲ
ಮರ್ತ್ಯರ
ಮರ್ತ್ಯ-ರನ್ನು
ಮರ್ತ್ಯ-ರಿಗೆ
ಮರ್ತ್ಯ-ಲೋಕಂ
ಮರ್ತ್ಯ-ಲೋ-ಕಕ್ಕೆ
ಮರ್ತ್ಯ-ಲೋ-ಕದ
ಮರ್ತ್ಯ-ಲೋ-ಕ-ದ-ಲ್ಲಿಯೇ
ಮರ್ತ್ಯ-ಲೋ-ಕ-ವನ್ನು
ಮರ್ತ್ಯೇಷು
ಮರ್ಮ
ಮರ್ಮ-ವನ್ನು
ಮಲ-ಕ-ವಾಗಿ
ಮಲ-ಗಲು
ಮಲಗಿ
ಮಲ-ಗಿ-ಕೊಂ-ಡಿ-ರುವ
ಮಲ-ಗಿ-ಕೊ-ಳ್ಳು-ವಾಗ
ಮಲ-ಗಿದ
ಮಲ-ಗಿ-ದ-ವ-ನಲ್ಲ
ಮಲ-ಗಿ-ದಾಗ
ಮಲ-ಗಿ-ದ್ದಾಗ
ಮಲ-ಗಿ-ರು-ವನು
ಮಲ-ಗಿ-ರು-ವಾಗ
ಮಲ-ಗು-ತ್ತಾನೆ
ಮಲ-ಗುವ
ಮಲ-ಗು-ವಾ-ಗಲೊ
ಮಲ-ಗು-ವು-ದಕ್ಕೆ
ಮಲ-ಗು-ವುದು
ಮಲ-ಮೂತ್ರ
ಮಲ-ಮೂ-ತ್ರ-ಗ-ಳ-ಲ್ಲಿಯೂ
ಮಲಿ-ನತೆ
ಮಲಿ-ನ-ವಾ-ಗಿದೆ
ಮಲಿ-ನ-ವಾ-ಗಿ-ಲ್ಲವೊ
ಮಲೇನ
ಮಲೇ-ರಿಯಾ
ಮಲ್ಲಿಗೆ
ಮಲ್ಲಿ-ಗೆಯ
ಮಲ್ಲಿ-ಗೆ-ಯೊಂದು
ಮಳಿ-ಗೆ-ಗ-ಳಿಗೆ
ಮಳೆ
ಮಳೆ-ಬಿ-ಸಿ-ಲು-ಗಾಳಿ
ಮಳೆ-ಗ-ರೆಯು
ಮಳೆ-ಗ-ರೆ-ಯು-ವನು
ಮಳೆ-ಗ-ರೆ-ಯು-ವುದು
ಮಳೆ-ಗ-ರೆ-ಯು-ವುದೆ
ಮಳೆಗೂ
ಮಳೆ-ಬ-ರು-ವುದು
ಮಳೆಯ
ಮಳೆ-ಯಂತೆ
ಮಳೆ-ಯನ್ನು
ಮಳೆ-ಯಲ್ಲಿ
ಮಳೆ-ಯಾಗಿ
ಮಳೆ-ಯಾ-ಗು-ವುದು
ಮಳೆ-ಯಿಂದ
ಮಳೆಯೂ
ಮಳೆಯೇ
ಮಸ-ಣಮ್ಮ
ಮಸಾ-ಲೆಯ
ಮಸಾ-ಲೆ-ಯಲ್ಲಿ
ಮಸಿ
ಮಸಿ-ಕಡ್ಡಿ
ಮಸಿ-ಯ-ನ್ನಾಗಿ
ಮಸಿ-ಯನ್ನೋ
ಮಸಿ-ಯಿಂದ
ಮಸೆಯು
ಮಸ್ತ-ಕ-ದ-ವ-ರೆಗೆ
ಮಹಡಿ
ಮಹತಾ
ಮಹತಿ
ಮಹ-ತೀಂ
ಮಹತೋ
ಮಹತ್
ಮಹ-ತ್ತಾದ
ಮಹ-ತ್ತಿಗೆ
ಮಹ-ತ್ತಿ-ನಲ್ಲಿ
ಮಹ-ತ್ತಿ-ನೊ-ಳಗೆ
ಮಹತ್ತು
ಮಹತ್ತೇ
ಮಹ-ತ್ಪಾಪಂ
ಮಹ-ತ್ವ-ವನ್ನು
ಮಹ-ದು-ಪ-ಕಾ-ರ-ವನ್ನು
ಮಹ-ದ್ಯೋ-ನಿ-ರಹಂ
ಮಹ-ದ್ರೂ-ಪ-ವನ್ನು
ಮಹ-ದ್ವ್ಯ-ಕ್ತಿ-ಗಳು
ಮಹ-ಮ್ಮ-ದೀಯ
ಮಹ-ಮ್ಮ-ದೀ-ಯ-ನಿಗೂ
ಮಹ-ಮ್ಮದ್
ಮಹ-ರಾಯ
ಮಹ-ರ್ಷಯಃ
ಮಹರ್ಷಿ
ಮಹ-ರ್ಷಿ-ಗ-ಳಾ-ಗಲಿ
ಮಹ-ರ್ಷಿ-ಗ-ಳಿಗೂ
ಮಹ-ರ್ಷಿ-ಗ-ಳಿಗೆ
ಮಹ-ರ್ಷಿ-ಗಳು
ಮಹ-ರ್ಷಿ-ಸಿ-ದ್ಧ-ಸಂ-ಘಾಃ
ಮಹ-ರ್ಷೀ-ಣಾಂ
ಮಹ-ಲಿಂ-ಗ-ರಂಗ
ಮಹಾ
ಮಹಾ-ಕವಿ
ಮಹಾ-ಕ-ವಿ-ಯೊಬ್ಬ
ಮಹಾ-ಕಾ-ಲ-ದಲ್ಲಿ
ಮಹಾ-ಕಾವ್ಯ
ಮಹಾ-ಕಾ-ವ್ಯ-ದಂ-ತಾಗು
ಮಹಾ-ಕಾ-ವ್ಯ-ವನ್ನು
ಮಹಾ-ಕಾ-ವ್ಯವೇ
ಮಹಾ-ಕಾ-ಶ-ದಲ್ಲಿ
ಮಹಾ-ಕಾ-ಶ-ದೊಂ-ದಿಗೆ
ಮಹಾ-ಕಾ-ಶವೇ
ಮಹಾ-ಕೊ-ಲೆ-ಯಾದ
ಮಹಾ-ಗರ್ವ
ಮಹಾ-ಗು-ರು-ವಾದ
ಮಹಾ-ಗು-ರು-ವಿಗೆ
ಮಹಾ-ಗು-ರು-ವಿ-ನೆ-ದು-ರಿಗೆ
ಮಹಾ-ಜ್ಞಾ-ನ-ದೆ-ದು-ರಿಗೆ
ಮಹಾ-ಜ್ಞಾ-ನಿ-ಗಳ
ಮಹಾ-ಜ್ಞಾ-ನಿ-ಗಳು
ಮಹಾ-ಜ್ಞಾ-ನಿ-ಗಳೋ
ಮಹಾ-ತ-ಪಸ್ವಿ
ಮಹಾತ್ಮ
ಮಹಾ-ತ್ಮನ
ಮಹಾ-ತ್ಮನಃ
ಮಹಾ-ತ್ಮ-ನನ್ನು
ಮಹಾ-ತ್ಮ-ನಾ-ಗಿ-ದ್ದರೆ
ಮಹಾ-ತ್ಮ-ನಾದ
ಮಹಾ-ತ್ಮ-ನೆಂದು
ಮಹಾ-ತ್ಮನೇ
ಮಹಾ-ತ್ಮನ್
ಮಹಾ-ತ್ಮರ
ಮಹಾ-ತ್ಮರು
ಮಹಾತ್ಮಾ
ಮಹಾ-ತ್ಮಾನಃ
ಮಹಾ-ತ್ಮಾ-ನಸ್ತು
ಮಹಾತ್ಮ್ಯ
ಮಹಾ-ತ್ಮ್ಯ-ವನ್ನು
ಮಹಾತ್ಮ್ಯೆ
ಮಹಾ-ತ್ಮ್ಯೆಗೆ
ಮಹಾ-ತ್ಯಾ-ಗವೇ
ಮಹಾ-ದು-ರ್ಗು-ಣ-ಗಳನ್ನು
ಮಹಾ-ದೋಷ
ಮಹಾ-ಧೀ-ರ-ರಿಗೆ
ಮಹಾ-ನೀತಿ
ಮಹಾ-ನು-ಭಾವ
ಮಹಾ-ನು-ಭಾ-ವ-ರಾದ
ಮಹಾ-ನು-ಭಾ-ವಾನ್
ಮಹಾನ್
ಮಹಾ-ಪ-ರಾ-ಧ-ಗಳನ್ನೆಲ್ಲಾ
ಮಹಾ-ಪ-ರಾ-ಧ-ಮಾ-ಡಿದೆ
ಮಹಾ-ಪ-ರಾ-ಧವೂ
ಮಹಾ-ಪಾಪ
ಮಹಾ-ಪಾ-ಪ-ವಿ-ನ್ನಿಲ್ಲ
ಮಹಾ-ಪಾಪಿ
ಮಹಾ-ಪಾಪ್ಮಾ
ಮಹಾ-ಪುಣ್ಯ
ಮಹಾ-ಪು-ರುಷ
ಮಹಾ-ಪುಷಿ
ಮಹಾ-ಪು-ಷಿ-ಗಳು
ಮಹಾ-ಬಾಹು
ಮಹಾ-ಬಾ-ಹುಃ
ಮಹಾ-ಬಾ-ಹುವೆ
ಮಹಾ-ಬಾಹೋ
ಮಹಾ-ಬ್ರ-ಹ್ಮಾಂಡ
ಮಹಾ-ಭ-ಕ್ತರು
ಮಹಾ-ಭ-ಯ-ದಿಂದ
ಮಹಾ-ಭಾ-ರತ
ಮಹಾ-ಭಾ-ರ-ತಕ್ಕೆ
ಮಹಾ-ಭಾ-ರ-ತದ
ಮಹಾ-ಭಾ-ರ-ತ-ದಲ್ಲಿ
ಮಹಾ-ಭಾ-ರ-ತಮ್
ಮಹಾ-ಭಾ-ರ-ತ-ವನ್ನು
ಮಹಾ-ಭಾ-ರ-ತ-ವಾ-ಗಿದೆ
ಮಹಾ-ಭಾ-ರ-ತ-ವಾ-ಯಿತು
ಮಹಾ-ಭಾ-ರ-ತ-ವೆಂಬ
ಮಹಾ-ಭಾ-ರ-ತ-ವೆಂ-ಬುದು
ಮಹಾ-ಭಾ-ರ-ತ-ವೆಲ್ಲಾ
ಮಹಾ-ಭಾ-ಷ್ಯ-ದಲ್ಲಿ
ಮಹಾ-ಭೂ-ತ-ಗಳು
ಮಹಾ-ಭೂ-ತಾ-ನ್ಯ-ಹಂ-ಕಾರೋ
ಮಹಾ-ಮಂ-ತ್ರವೇ
ಮಹಾ-ಮಹಾ
ಮಹಾ-ಮ-ಹಿ-ಮ-ನಿಗೆ
ಮಹಾ-ಮ-ಹಿ-ಮರ
ಮಹಾ-ಮ-ಹಿ-ಮ-ರಲ್ಲಿ
ಮಹಾ-ಮು-ನಿ-ಗಳು
ಮಹಾ-ಯಂ-ತ್ರದ
ಮಹಾ-ಯ-ಜ್ಞ-ಗಳು
ಮಹಾ-ಯುಗ
ಮಹಾ-ಯು-ಗ-ವಾ-ದರೆ
ಮಹಾ-ಯೋ-ಗೇ-ಶ್ವ-ರ-ನಾದ
ಮಹಾ-ಯೋ-ಗೇ-ಶ್ವರೋ
ಮಹಾ-ರಥಃ
ಮಹಾ-ರ-ಥ-ದಲ್ಲಿ
ಮಹಾ-ರ-ಥ-ನಾದ
ಮಹಾ-ರ-ಥ-ರಾದ
ಮಹಾ-ರ-ಥರು
ಮಹಾ-ರ-ಥ-ರು-ಗಳಲ್ಲಿ
ಮಹಾ-ರ-ಥ-ರೆ-ಲ್ಲರ
ಮಹಾ-ರ-ಥರೇ
ಮಹಾ-ರ-ಥಾಃ
ಮಹಾ-ರ-ಥಿ-ಗಳು
ಮಹಾ-ರ-ಥಿ-ಯೆಂದು
ಮಹಾ-ರ-ಥಿ-ಯೊ-ಡನೆ
ಮಹಾ-ರಾ-ಜನ
ಮಹಾ-ವಾಯು
ಮಹಾ-ವಿ-ಸ್ಮ-ಯ-ವಾ-ಗು-ತ್ತಿದೆ
ಮಹಾ-ವೀರ
ಮಹಾ-ವೈ-ದ್ಯ-ರೆಲ್ಲಾ
ಮಹಾ-ವ್ಯ-ಕ್ತಿ-ಗಳನ್ನು
ಮಹಾ-ವ್ಯ-ಕ್ತಿ-ಗಳು
ಮಹಾ-ವ್ಯ-ಕ್ತಿ-ಯಲ್ಲಿ
ಮಹಾ-ಶಂಖಂ
ಮಹಾ-ಶನೋ
ಮಹಾ-ಸ-ತ್ಯ-ವನ್ನು
ಮಹಾ-ಸಾ-ಗ-ರಕ್ಕೆ
ಮಹಾ-ಸಾ-ಗ-ರ-ದ-ಲ್ಲಿ-ರುವ
ಮಹಾ-ಸಾ-ಗ-ರ-ದಿಂದ
ಮಹಾ-ಸಾ-ಗ-ರ-ವನ್ನೇ
ಮಹಾ-ಸಾ-ಗ-ರ-ವೆಂದು
ಮಹಾ-ಸ್ತ್ರ-ದಂತೆ
ಮಹಿ-ಮರ
ಮಹಿ-ಮಾನಂ
ಮಹಿ-ಮಾ-ವಂತ
ಮಹಿಮೆ
ಮಹಿ-ಮೆಗೆ
ಮಹಿ-ಮೆ-ಯನ್ನು
ಮಹಿ-ಮೆಯೇ
ಮಹಿ-ಮೋ-ನ್ನತ
ಮಹೀ-ಕೃತೇ
ಮಹೀ-ಕ್ಷಿ-ತಾಮ್
ಮಹೀ-ಪತೇ
ಮಹೀಮ್
ಮಹೇ-ಶ್ವರ
ಮಹೇ-ಶ್ವರಃ
ಮಹೇ-ಶ್ವ-ರನ
ಮಹೇ-ಶ್ವ-ರ-ನಾದ
ಮಹೇ-ಶ್ವ-ರ-ನೆಂದು
ಮಹೇ-ಶ್ವ-ರ-ನೆಂದೂ
ಮಹೇ-ಷ್ವಾಸಾ
ಮಹೋ-ನ್ನತ
ಮಹೋ-ಪ-ಕಾರ
ಮಾ
ಮಾಂ
ಮಾಂಡೂಕ್ಯ
ಮಾಂಸದ
ಮಾಂಸ-ವನ್ನು
ಮಾಂಸ-ವಾಗಿ
ಮಾಂಸ-ವಾ-ಗು-ವುದು
ಮಾಂಸಾ-ಹಾರ
ಮಾಗ-ಬೇಕು
ಮಾಗಲು
ಮಾಗಿ
ಮಾಗಿದ
ಮಾಗಿ-ದರೂ
ಮಾಗಿ-ದಾಗ
ಮಾಗು
ಮಾಗುತ್ತ
ಮಾಗು-ವು-ದಕ್ಕೆ
ಮಾಗು-ವು-ದಿಲ್ಲ
ಮಾಗು-ವುದು
ಮಾಟ
ಮಾಟ-ವನ್ನೊ
ಮಾಡ
ಮಾಡ-ಕೂ-ಡದು
ಮಾಡ-ತಕ್ಕ
ಮಾಡ-ತ-ಕ್ಕದ್ದು
ಮಾಡ-ತೊ-ಡ-ಗಿ-ದರು
ಮಾಡ-ತ್ತಿ-ರು-ವನೊ
ಮಾಡದ
ಮಾಡ-ದಂತೆ
ಮಾಡ-ದ-ವ-ನಂತೆ
ಮಾಡ-ದ-ವ-ನಿ-ಗಿಂತ
ಮಾಡ-ದ-ವ-ನಿಗೂ
ಮಾಡ-ದ-ವ-ನಿಗೆ
ಮಾಡ-ದ-ವರು
ಮಾಡ-ದ-ವರೇ
ಮಾಡ-ದಾಗ
ಮಾಡ-ದಿ-ರು-ವು-ದ-ಕ್ಕಿಂತ
ಮಾಡ-ದು-ದನ್ನು
ಮಾಡದೆ
ಮಾಡದೇ
ಮಾಡ-ಬ-ಯ-ಸು-ತ್ತಾರೆ
ಮಾಡ-ಬಲ್ಲ
ಮಾಡ-ಬ-ಲ್ಲಂ-ತಹ
ಮಾಡ-ಬ-ಲ್ಲದು
ಮಾಡ-ಬ-ಲ್ಲನು
ಮಾಡ-ಬ-ಲ್ಲ-ಯಾ-ವುದು
ಮಾಡ-ಬ-ಲ್ಲ-ವ-ನಾ-ಗಿ-ದ್ದಾನೊ
ಮಾಡ-ಬ-ಲ್ಲುದು
ಮಾಡ-ಬ-ಲ್ಲೆಯಾ
ಮಾಡ-ಬ-ಹು-ದಲ್ಲ
ಮಾಡ-ಬ-ಹು-ದಷ್ಟೆ
ಮಾಡ-ಬ-ಹುದು
ಮಾಡ-ಬ-ಹು-ದೆಂದು
ಮಾಡ-ಬ-ಹುದೊ
ಮಾಡ-ಬಾ-ರದ
ಮಾಡ-ಬಾ-ರ-ದಾ-ಗಿತ್ತು
ಮಾಡ-ಬಾ-ರದು
ಮಾಡ-ಬಾ-ರ-ದು-ದನ್ನು
ಮಾಡ-ಬಾ-ರ-ದು-ದನ್ನೇ
ಮಾಡ-ಬಾ-ರ-ದೇಕೆ
ಮಾಡ-ಬಾ-ರದೊ
ಮಾಡ-ಬಾ-ರದೋ
ಮಾಡ-ಬಾ-ರ-ದ್ದನ್ನು
ಮಾಡ-ಬೇ-ಕಲ್ಲ
ಮಾಡ-ಬೇಕಾ
ಮಾಡ-ಬೇ-ಕಾಗಿ
ಮಾಡ-ಬೇ-ಕಾ-ಗಿತ್ತು
ಮಾಡ-ಬೇ-ಕಾ-ಗಿದೆ
ಮಾಡ-ಬೇ-ಕಾ-ಗಿ-ದೆಯೊ
ಮಾಡ-ಬೇ-ಕಾ-ಗಿ-ರುವ
ಮಾಡ-ಬೇ-ಕಾ-ಗಿ-ರು-ವುದನ್ನು
ಮಾಡ-ಬೇ-ಕಾ-ಗಿ-ರು-ವುದು
ಮಾಡ-ಬೇ-ಕಾ-ಗಿಲ್ಲ
ಮಾಡ-ಬೇ-ಕಾ-ಗು-ವುದು
ಮಾಡ-ಬೇ-ಕಾದ
ಮಾಡ-ಬೇ-ಕಾ-ದರೂ
ಮಾಡ-ಬೇ-ಕಾ-ದರೆ
ಮಾಡ-ಬೇ-ಕಾ-ದು-ದನ್ನು
ಮಾಡ-ಬೇ-ಕಾ-ದು-ದ-ನ್ನೆಲ್ಲಾ
ಮಾಡ-ಬೇ-ಕಾ-ದು-ದೇನು
ಮಾಡ-ಬೇ-ಕಾದ್ದು
ಮಾಡ-ಬೇ-ಕಾ-ಯಿತು
ಮಾಡ-ಬೇಕು
ಮಾಡ-ಬೇ-ಕು-ಇ-ದೇನು
ಮಾಡ-ಬೇಕೆ
ಮಾಡ-ಬೇ-ಕೆಂ-ದಲ್ಲ
ಮಾಡ-ಬೇ-ಕೆಂ-ದಿ-ರು-ವನೊ
ಮಾಡ-ಬೇ-ಕೆಂ-ದಿ-ರುವೆ
ಮಾಡ-ಬೇ-ಕೆಂದು
ಮಾಡ-ಬೇ-ಕೆಂಬ
ಮಾಡ-ಬೇ-ಕೆಂ-ಬು-ದನ್ನು
ಮಾಡ-ಬೇ-ಕೆ-ನಿ-ಸಿ-ದರೆ
ಮಾಡ-ಬೇಕೊ
ಮಾಡ-ಬೇಕೋ
ಮಾಡ-ಬೇಡ
ಮಾಡ-ಲಾಗು
ಮಾಡ-ಲಾ-ಗು-ವು-ದಿಲ್ಲ
ಮಾಡ-ಲಾ-ಗು-ವು-ದಿ-ಲ್ಲವೊ
ಮಾಡ-ಲಾ-ಗು-ವುದು
ಮಾಡ-ಲಾರ
ಮಾಡ-ಲಾ-ರದ
ಮಾಡ-ಲಾ-ರದು
ಮಾಡ-ಲಾ-ರದೊ
ಮಾಡ-ಲಾ-ರನು
ಮಾಡ-ಲಾ-ರರು
ಮಾಡ-ಲಾ-ರರೋ
ಮಾಡ-ಲಾ-ರಳು
ಮಾಡ-ಲಾ-ರವು
ಮಾಡ-ಲಾ-ರವೊ
ಮಾಡ-ಲಾರೆ
ಮಾಡ-ಲಾ-ರೆವು
ಮಾಡಲಿ
ಮಾಡ-ಲಿಲ್ಲ
ಮಾಡಲು
ಮಾಡಲೂ
ಮಾಡಲೆ
ಮಾಡ-ಲೆ-ತ್ನಿ-ಸು-ವನು
ಮಾಡ-ಲೆ-ತ್ನಿ-ಸು-ವರು
ಮಾಡ-ಲೆ-ತ್ನಿ-ಸು-ವು-ದಿಲ್ಲ
ಮಾಡಲೇ
ಮಾಡ-ಲೇ-ಬೇ-ಕಾ-ಗಿದೆ
ಮಾಡ-ಲೇ-ಬೇ-ಕಾ-ಗಿ-ರು-ವು-ದ-ರಿಂದ
ಮಾಡ-ಲೇ-ಬೇ-ಕಾ-ಗಿ-ರು-ವುದು
ಮಾಡ-ಲೇ-ಬೇ-ಕಾದ
ಮಾಡ-ಲೇ-ಬೇ-ಕಾ-ದುದು
ಮಾಡ-ಲೇ-ಬೇಕು
ಮಾಡ-ಲ್ಪಟ್ಟ
ಮಾಡ-ಲ್ಪ-ಟ್ಟಂತೆ
ಮಾಡ-ಲ್ಪ-ಟ್ಟಿದೆ
ಮಾಡ-ವ-ವರು
ಮಾಡ-ವುದು
ಮಾಡಿ
ಮಾಡಿಕೊ
ಮಾಡಿ-ಕೊಂಡ
ಮಾಡಿ-ಕೊಂ-ಡರು
ಮಾಡಿ-ಕೊಂ-ಡರೆ
ಮಾಡಿ-ಕೊಂ-ಡ-ವ-ನಲ್ಲ
ಮಾಡಿ-ಕೊಂ-ಡ-ವನು
ಮಾಡಿ-ಕೊಂ-ಡ-ವನೆ
ಮಾಡಿ-ಕೊಂ-ಡ-ವರು
ಮಾಡಿ-ಕೊಂ-ಡಿತು
ಮಾಡಿ-ಕೊಂ-ಡಿದ್ದ
ಮಾಡಿ-ಕೊಂ-ಡಿ-ದ್ದರೆ
ಮಾಡಿ-ಕೊಂ-ಡಿ-ದ್ದೆನೊ
ಮಾಡಿ-ಕೊಂ-ಡಿ-ದ್ದೇವೆ
ಮಾಡಿ-ಕೊಂ-ಡಿ-ರ-ಬೇಕು
ಮಾಡಿ-ಕೊಂ-ಡಿರು
ಮಾಡಿ-ಕೊಂ-ಡಿ-ರು-ವನು
ಮಾಡಿ-ಕೊಂ-ಡಿ-ರು-ವನೋ
ಮಾಡಿ-ಕೊಂ-ಡಿ-ರು-ವ-ವರು
ಮಾಡಿ-ಕೊಂ-ಡಿ-ರು-ವು-ದನ್ನೇ
ಮಾಡಿ-ಕೊಂ-ಡಿ-ರು-ವು-ದಿಲ್ಲ
ಮಾಡಿ-ಕೊಂ-ಡಿ-ರು-ವುದೂ
ಮಾಡಿ-ಕೊಂ-ಡಿ-ರು-ವೆನು
ಮಾಡಿ-ಕೊಂ-ಡಿ-ರು-ವೆವು
ಮಾಡಿ-ಕೊಂ-ಡಿ-ರು-ವೆವೊ
ಮಾಡಿ-ಕೊಂ-ಡಿಲ್ಲ
ಮಾಡಿ-ಕೊಂ-ಡಿ-ಲ್ಲವೊ
ಮಾಡಿ-ಕೊಂ-ಡಿವೆ
ಮಾಡಿ-ಕೊಂಡು
ಮಾಡಿ-ಕೊಂಡೆ
ಮಾಡಿ-ಕೊಂ-ಡೆಯಾ
ಮಾಡಿ-ಕೊಂ-ಡೆವು
ಮಾಡಿ-ಕೊ-ಟ್ಟಂತೆ
ಮಾಡಿ-ಕೊ-ಡ-ಬೇ-ಕೆಂದು
ಮಾಡಿ-ಕೊಡು
ಮಾಡಿ-ಕೊ-ಡು-ವನು
ಮಾಡಿ-ಕೊ-ಡು-ವುದು
ಮಾಡಿ-ಕೊಳ್ಳ
ಮಾಡಿ-ಕೊ-ಳ್ಳ-ಬ-ಲ್ಲರು
ಮಾಡಿ-ಕೊ-ಳ್ಳ-ಬ-ಲ್ಲರೋ
ಮಾಡಿ-ಕೊ-ಳ್ಳ-ಬ-ಹುದು
ಮಾಡಿ-ಕೊ-ಳ್ಳ-ಬಾ-ರದು
ಮಾಡಿ-ಕೊ-ಳ್ಳ-ಬೇ-ಕಾ-ಗಿದೆ
ಮಾಡಿ-ಕೊ-ಳ್ಳ-ಬೇ-ಕಾ-ದರೆ
ಮಾಡಿ-ಕೊ-ಳ್ಳ-ಬೇಕು
ಮಾಡಿ-ಕೊ-ಳ್ಳ-ಲಾ-ರದೆ
ಮಾಡಿ-ಕೊ-ಳ್ಳಲು
ಮಾಡಿ-ಕೊಳ್ಳಿ
ಮಾಡಿ-ಕೊಳ್ಳು
ಮಾಡಿ-ಕೊ-ಳ್ಳು-ತ್ತಾನೆ
ಮಾಡಿ-ಕೊ-ಳ್ಳು-ತ್ತಾ-ನೆಯೊ
ಮಾಡಿ-ಕೊ-ಳ್ಳು-ತ್ತಾರೊ
ಮಾಡಿ-ಕೊ-ಳ್ಳು-ತ್ತಿ-ರು-ವರು
ಮಾಡಿ-ಕೊ-ಳ್ಳು-ತ್ತಿ-ರು-ವೆವು
ಮಾಡಿ-ಕೊ-ಳ್ಳು-ತ್ತಿಲ್ಲ
ಮಾಡಿ-ಕೊ-ಳ್ಳು-ತ್ತೇವೆ
ಮಾಡಿ-ಕೊ-ಳ್ಳು-ತ್ತೇ-ವೆಯೊ
ಮಾಡಿ-ಕೊ-ಳ್ಳುವ
ಮಾಡಿ-ಕೊ-ಳ್ಳು-ವನು
ಮಾಡಿ-ಕೊ-ಳ್ಳು-ವನೊ
ಮಾಡಿ-ಕೊ-ಳ್ಳು-ವರು
ಮಾಡಿ-ಕೊ-ಳ್ಳು-ವ-ವ-ನಲ್ಲ
ಮಾಡಿ-ಕೊ-ಳ್ಳು-ವ-ವರು
ಮಾಡಿ-ಕೊ-ಳ್ಳು-ವಾ-ಗಲೂ
ಮಾಡಿ-ಕೊ-ಳ್ಳು-ವು-ದಕ್ಕೆ
ಮಾಡಿ-ಕೊ-ಳ್ಳು-ವುದನ್ನು
ಮಾಡಿ-ಕೊ-ಳ್ಳು-ವು-ದರ
ಮಾಡಿ-ಕೊ-ಳ್ಳು-ವು-ದಿಲ್ಲ
ಮಾಡಿ-ಕೊ-ಳ್ಳು-ವುದು
ಮಾಡಿ-ಕೊ-ಳ್ಳು-ವುದೇ
ಮಾಡಿ-ಕೊ-ಳ್ಳು-ವೆನು
ಮಾಡಿ-ಕೊ-ಳ್ಳು-ವೆವು
ಮಾಡಿ-ಕೊ-ಳ್ಳು-ವೆವೋ
ಮಾಡಿತು
ಮಾಡಿತೆ
ಮಾಡಿತ್ತು
ಮಾಡಿದ
ಮಾಡಿ-ದಂ-ತಾ-ಗು-ವು-ದಿಲ್ಲ
ಮಾಡಿ-ದಂ-ತಾ-ಗು-ವುದು
ಮಾಡಿ-ದಂತೆ
ಮಾಡಿ-ದ-ನಲ್ಲ
ಮಾಡಿ-ದನು
ಮಾಡಿ-ದ-ನೆಂದು
ಮಾಡಿ-ದನೋ
ಮಾಡಿ-ದ-ಮೇಲೆ
ಮಾಡಿ-ದ-ರಾ-ಯಿತು
ಮಾಡಿ-ದರು
ಮಾಡಿ-ದರೂ
ಮಾಡಿ-ದರೆ
ಮಾಡಿ-ದರೇ
ಮಾಡಿ-ದ-ರೇನೆ
ಮಾಡಿ-ದರೋ
ಮಾಡಿ-ದ-ಲ್ಲದೆ
ಮಾಡಿ-ದವ
ಮಾಡಿ-ದ-ವ-ನನ್ನು
ಮಾಡಿ-ದ-ವ-ನಲ್ಲ
ಮಾಡಿ-ದ-ವ-ನಾ-ಗು-ತ್ತಾನೆ
ಮಾಡಿ-ದ-ವ-ನಾ-ವನೂ
ಮಾಡಿ-ದ-ವ-ನಿಗೆ
ಮಾಡಿ-ದ-ವನು
ಮಾಡಿ-ದ-ವ-ನೆಂದು
ಮಾಡಿ-ದ-ವ-ನೊ-ಬ್ಬ-ನಿಗೆ
ಮಾಡಿ-ದ-ವರ
ಮಾಡಿ-ದ-ವ-ರನ್ನು
ಮಾಡಿ-ದ-ವ-ರ-ನ್ನೆಲ್ಲ
ಮಾಡಿ-ದ-ವ-ರಲ್ಲ
ಮಾಡಿ-ದ-ವ-ರಲ್ಲಿ
ಮಾಡಿ-ದ-ವ-ರಿಗೆ
ಮಾಡಿ-ದ-ವರು
ಮಾಡಿ-ದ-ವರೆ
ಮಾಡಿ-ದಷ್ಟೂ
ಮಾಡಿ-ದಾಗ
ಮಾಡಿ-ದಾ-ಗಲೂ
ಮಾಡಿ-ದಾ-ಗಲೆ
ಮಾಡಿ-ದಿರಿ
ಮಾಡಿ-ದು-ದಕ್ಕೆ
ಮಾಡಿ-ದು-ದನ್ನು
ಮಾಡಿ-ದು-ದನ್ನೇ
ಮಾಡಿ-ದು-ದ-ರಿಂದ
ಮಾಡಿ-ದು-ದಲ್ಲ
ಮಾಡಿದೆ
ಮಾಡಿ-ದೆನೊ
ಮಾಡಿ-ದೆಯೊ
ಮಾಡಿ-ದೆಯೋ
ಮಾಡಿ-ದೆವು
ಮಾಡಿ-ದೆವೋ
ಮಾಡಿ-ದೊ-ಡ-ನೆಯೇ
ಮಾಡಿದ್ದ
ಮಾಡಿ-ದ್ದಕ್ಕೆ
ಮಾಡಿ-ದ್ದನ್ನು
ಮಾಡಿ-ದ್ದ-ನ್ನೆಲ್ಲ
ಮಾಡಿ-ದ್ದ-ರಿಂದ
ಮಾಡಿ-ದ್ದ-ರಿಂ-ದಲೇ
ಮಾಡಿ-ದ್ದರು
ಮಾಡಿ-ದ್ದರೂ
ಮಾಡಿ-ದ್ದರೆ
ಮಾಡಿ-ದ್ದರೊ
ಮಾಡಿ-ದ್ದ-ವ-ರ-ನ್ನೆಲ್ಲಾ
ಮಾಡಿ-ದ್ದಾ-ಗಿ-ರ-ಬೇಕು
ಮಾಡಿ-ದ್ದಾನೆ
ಮಾಡಿ-ದ್ದಾ-ನೆಯೆ
ಮಾಡಿ-ದ್ದಾ-ನೆಯೋ
ಮಾಡಿ-ದ್ದಾರೆ
ಮಾಡಿದ್ದಿ
ಮಾಡಿ-ದ್ದಿ-ದ್ದನೊ
ಮಾಡಿದ್ದು
ಮಾಡಿ-ದ್ದುಣ್ಣೊ
ಮಾಡಿ-ದ್ದುಣ್ಣೋ
ಮಾಡಿದ್ದೆ
ಮಾಡಿ-ದ್ದೆಲ್ಲಾ
ಮಾಡಿ-ದ್ದೆವೊ
ಮಾಡಿದ್ದೇ
ಮಾಡಿ-ದ್ದೇನೆ
ಮಾಡಿ-ದ್ದೇನೋ
ಮಾಡಿ-ದ್ದೇವೆ
ಮಾಡಿ-ಬಿ-ಟ್ಟರೆ
ಮಾಡಿ-ಬಿ-ಟ್ಟಿರು
ಮಾಡಿ-ಬಿಟ್ಟು
ಮಾಡಿ-ಬಿಡು
ಮಾಡಿ-ಬಿ-ಡು-ತ್ತಾನೆ
ಮಾಡಿ-ಬಿ-ಡು-ವೆವು
ಮಾಡಿ-ಯಾದ
ಮಾಡಿ-ಯಾ-ದ-ಮೇಲೆ
ಮಾಡಿಯೂ
ಮಾಡಿಯೇ
ಮಾಡಿ-ಯೇನು
ಮಾಡಿಯೋ
ಮಾಡಿರ
ಮಾಡಿ-ರ-ಬ-ಹುದು
ಮಾಡಿ-ರ-ಬೇಕು
ಮಾಡಿ-ರ-ಲಿಲ್ಲ
ಮಾಡಿರು
ಮಾಡಿ-ರು-ತ್ತಾನೆ
ಮಾಡಿ-ರು-ತ್ತಾರೆ
ಮಾಡಿ-ರುವ
ಮಾಡಿ-ರು-ವನು
ಮಾಡಿ-ರು-ವನೊ
ಮಾಡಿ-ರು-ವನೋ
ಮಾಡಿ-ರು-ವರು
ಮಾಡಿ-ರು-ವರೊ
ಮಾಡಿ-ರು-ವರೋ
ಮಾಡಿ-ರು-ವ-ವ-ನದು
ಮಾಡಿ-ರು-ವ-ವನು
ಮಾಡಿ-ರು-ವ-ವನೆ
ಮಾಡಿ-ರು-ವ-ವರು
ಮಾಡಿ-ರು-ವಷ್ಟು
ಮಾಡಿ-ರು-ವುದನ್ನು
ಮಾಡಿ-ರು-ವು-ದ-ನ್ನೆಲ್ಲ
ಮಾಡಿ-ರು-ವು-ದನ್ನೇ
ಮಾಡಿ-ರು-ವು-ದ-ರಿಂದ
ಮಾಡಿ-ರು-ವು-ದಾ-ವುದು
ಮಾಡಿ-ರು-ವು-ದಾ-ವುದೂ
ಮಾಡಿ-ರು-ವು-ದಿಲ್ಲ
ಮಾಡಿ-ರು-ವುದು
ಮಾಡಿ-ರು-ವುದೆ
ಮಾಡಿ-ರು-ವು-ದೆಲ್ಲ
ಮಾಡಿ-ರು-ವು-ದೆ-ಲ್ಲಾ-ಅದು
ಮಾಡಿ-ರು-ವುದೇ
ಮಾಡಿ-ರುವೆ
ಮಾಡಿ-ರು-ವೆನು
ಮಾಡಿ-ರು-ವೆವು
ಮಾಡಿ-ರು-ವೆವೆ
ಮಾಡಿ-ರು-ವೆವೋ
ಮಾಡಿಲ್ಲ
ಮಾಡಿ-ಲ್ಲ-ದಂ-ತೆಯೆ
ಮಾಡಿ-ಲ್ಲದೆ
ಮಾಡಿ-ಲ್ಲವೋ
ಮಾಡಿ-ಸದೆ
ಮಾಡಿ-ಸದೇ
ಮಾಡಿ-ಸನು
ಮಾಡಿ-ಸ-ಬೇ-ಕಾ-ದರೆ
ಮಾಡಿ-ಸ-ಬೇಕು
ಮಾಡಿ-ಸ-ಬೇಕೋ
ಮಾಡಿ-ಸಲಿ
ಮಾಡಿ-ಸಲು
ಮಾಡಿಸಿ
ಮಾಡಿ-ಸಿ-ಕೊಂಡು
ಮಾಡಿ-ಸಿ-ಕೊ-ಳ್ಳು-ತ್ತೇವೆ
ಮಾಡಿ-ಸಿ-ಕೊ-ಳ್ಳು-ವನು
ಮಾಡಿ-ಸಿ-ಕೊ-ಳ್ಳು-ವು-ದಕ್ಕೆ
ಮಾಡಿ-ಸಿ-ಕೊ-ಳ್ಳು-ವುದು
ಮಾಡಿ-ಸಿದ
ಮಾಡಿ-ಸಿ-ದಂತೆ
ಮಾಡಿ-ಸಿ-ದರು
ಮಾಡಿ-ಸಿದೆ
ಮಾಡಿ-ಸಿದ್ದು
ಮಾಡಿ-ಸಿ-ಬಿಟ್ಟ
ಮಾಡಿ-ಸಿ-ಬಿ-ಡು-ವರು
ಮಾಡಿ-ಸಿಯೇ
ಮಾಡಿ-ಸು-ತ್ತಾನೆ
ಮಾಡಿ-ಸು-ತ್ತಿ-ರು-ವನು
ಮಾಡಿ-ಸು-ತ್ತಿ-ರು-ವ-ವನು
ಮಾಡಿ-ಸು-ತ್ತಿ-ರು-ವುದು
ಮಾಡಿ-ಸುವ
ಮಾಡಿ-ಸು-ವನು
ಮಾಡಿ-ಸು-ವನೊ
ಮಾಡಿ-ಸು-ವ-ವನು
ಮಾಡಿ-ಸು-ವಾಗ
ಮಾಡಿ-ಸು-ವಾ-ತನು
ಮಾಡಿ-ಸು-ವು-ದಕ್ಕೆ
ಮಾಡಿ-ಸು-ವು-ದ-ರಲ್ಲಿ
ಮಾಡಿ-ಸು-ವುದು
ಮಾಡಿ-ಸು-ವುವು
ಮಾಡಿ-ಸು-ವೆನು
ಮಾಡಿ-ಹಾಕು
ಮಾಡಿ-ಹಾ-ಕು-ತ್ತದೆ
ಮಾಡಿ-ಹಾ-ಕು-ತ್ತಾನೆ
ಮಾಡಿ-ಹಾ-ಕು-ತ್ತೇವೆ
ಮಾಡಿ-ಹಾ-ಕು-ವನು
ಮಾಡಿ-ಹಾ-ಕು-ವುದು
ಮಾಡಿ-ಹಾ-ಕು-ವೆವು
ಮಾಡಿ-ಹಾ-ಕೋಣ
ಮಾಡೀತು
ಮಾಡು
ಮಾಡುತ್ತ
ಮಾಡು-ತ್ತದೆ
ಮಾಡು-ತ್ತಲೇ
ಮಾಡು-ತ್ತವೆ
ಮಾಡುತ್ತಾ
ಮಾಡು-ತ್ತಾನೆ
ಮಾಡು-ತ್ತಾ-ನೆಯೊ
ಮಾಡು-ತ್ತಾ-ನೆಯೋ
ಮಾಡು-ತ್ತಾನೊ
ಮಾಡು-ತ್ತಾನೋ
ಮಾಡು-ತ್ತಾರೆ
ಮಾಡು-ತ್ತಾ-ರೆಯೊ
ಮಾಡು-ತ್ತಾ-ರೆಯೋ
ಮಾಡು-ತ್ತಾರೊ
ಮಾಡು-ತ್ತಾರೋ
ಮಾಡು-ತ್ತಾಳೆ
ಮಾಡು-ತ್ತಿದೆ
ಮಾಡು-ತ್ತಿ-ದೆ-ಣಿ-ಜ್ಞ್ಜ-ಕಿ-ಖಿಗಿ
ಮಾಡು-ತ್ತಿ-ದೆಯೋ
ಮಾಡು-ತ್ತಿದ್ದ
ಮಾಡು-ತ್ತಿ-ದ್ದನೋ
ಮಾಡು-ತ್ತಿ-ದ್ದರು
ಮಾಡು-ತ್ತಿ-ದ್ದರೂ
ಮಾಡು-ತ್ತಿ-ದ್ದರೆ
ಮಾಡು-ತ್ತಿ-ದ್ದ-ರೆಂದೂ
ಮಾಡು-ತ್ತಿ-ದ್ದರೇ
ಮಾಡು-ತ್ತಿ-ದ್ದರೊ
ಮಾಡು-ತ್ತಿ-ದ್ದ-ವನ
ಮಾಡು-ತ್ತಿ-ದ್ದ-ವನು
ಮಾಡು-ತ್ತಿ-ದ್ದಾಗ
ಮಾಡು-ತ್ತಿ-ದ್ದಾನೆ
ಮಾಡು-ತ್ತಿ-ದ್ದೀಯೊ
ಮಾಡು-ತ್ತಿ-ದ್ದು-ದಕ್ಕೆ
ಮಾಡು-ತ್ತಿ-ದ್ದೇನೆ
ಮಾಡು-ತ್ತಿರ
ಮಾಡು-ತ್ತಿ-ರ-ಬ-ಹುದು
ಮಾಡು-ತ್ತಿ-ರ-ಬೇ-ಕಾ-ಗು-ವುದು
ಮಾಡು-ತ್ತಿ-ರ-ಬೇಕು
ಮಾಡು-ತ್ತಿ-ರಲಿ
ಮಾಡು-ತ್ತಿರು
ಮಾಡು-ತ್ತಿ-ರು-ತ್ತದೆ
ಮಾಡು-ತ್ತಿ-ರು-ತ್ತವೆ
ಮಾಡು-ತ್ತಿ-ರುವ
ಮಾಡು-ತ್ತಿ-ರು-ವಂತೆ
ಮಾಡು-ತ್ತಿ-ರು-ವನು
ಮಾಡು-ತ್ತಿ-ರು-ವನೊ
ಮಾಡು-ತ್ತಿ-ರು-ವನೋ
ಮಾಡು-ತ್ತಿ-ರು-ವರು
ಮಾಡು-ತ್ತಿ-ರು-ವರೊ
ಮಾಡು-ತ್ತಿ-ರು-ವ-ವ-ನಲ್ಲ
ಮಾಡು-ತ್ತಿ-ರು-ವ-ವ-ನಲ್ಲಿ
ಮಾಡು-ತ್ತಿ-ರು-ವ-ವ-ನಿಗೆ
ಮಾಡು-ತ್ತಿ-ರು-ವ-ವನು
ಮಾಡು-ತ್ತಿ-ರು-ವ-ವನೂ
ಮಾಡು-ತ್ತಿ-ರು-ವ-ವ-ರಿಗೆ
ಮಾಡು-ತ್ತಿ-ರು-ವ-ವರು
ಮಾಡು-ತ್ತಿ-ರು-ವಾಗ
ಮಾಡು-ತ್ತಿ-ರು-ವಾ-ಗಲೂ
ಮಾಡು-ತ್ತಿ-ರು-ವಾ-ಗಲೇ
ಮಾಡು-ತ್ತಿ-ರು-ವಿರಿ
ಮಾಡು-ತ್ತಿ-ರು-ವುದನ್ನು
ಮಾಡು-ತ್ತಿ-ರು-ವು-ದನ್ನೇ
ಮಾಡು-ತ್ತಿ-ರು-ವು-ದ-ರಿಂದ
ಮಾಡು-ತ್ತಿ-ರು-ವು-ದಾ-ದರೂ
ಮಾಡು-ತ್ತಿ-ರು-ವುದು
ಮಾಡು-ತ್ತಿ-ರು-ವುದೂ
ಮಾಡು-ತ್ತಿ-ರು-ವುದೇ
ಮಾಡು-ತ್ತಿ-ರು-ವುವು
ಮಾಡು-ತ್ತಿ-ರುವೆ
ಮಾಡು-ತ್ತಿ-ರು-ವೆನು
ಮಾಡು-ತ್ತಿ-ರು-ವೆನೊ
ಮಾಡು-ತ್ತಿ-ರು-ವೆಯೋ
ಮಾಡು-ತ್ತಿ-ರು-ವೆವು
ಮಾಡು-ತ್ತಿಲ್ಲ
ಮಾಡು-ತ್ತಿವೆ
ಮಾಡು-ತ್ತಿ-ವೆಯೆ
ಮಾಡು-ತ್ತೀಯೆ
ಮಾಡು-ತ್ತೀಯೊ
ಮಾಡು-ತ್ತೀಯೋ
ಮಾಡು-ತ್ತೇನೆ
ಮಾಡು-ತ್ತೇ-ನೆಯೋ
ಮಾಡು-ತ್ತೇವೆ
ಮಾಡು-ತ್ತೇ-ವೆಯೊ
ಮಾಡು-ತ್ತೇ-ವೆಯೋ
ಮಾಡುಲು
ಮಾಡುವ
ಮಾಡು-ವಂ-ತಹ
ಮಾಡು-ವಂ-ತ-ಹ-ವ-ರಲ್ಲ
ಮಾಡು-ವಂ-ತ-ಹು-ದಲ್ಲ
ಮಾಡು-ವಂ-ತಿಲ್ಲ
ಮಾಡು-ವಂತೆ
ಮಾಡು-ವಂ-ತೆಯೇ
ಮಾಡು-ವನು
ಮಾಡು-ವನೆ
ಮಾಡು-ವನೇ
ಮಾಡು-ವನೊ
ಮಾಡು-ವನೋ
ಮಾಡು-ವರು
ಮಾಡು-ವರೆ
ಮಾಡು-ವರೊ
ಮಾಡು-ವರೋ
ಮಾಡು-ವಳು
ಮಾಡು-ವವ
ಮಾಡು-ವ-ವನ
ಮಾಡು-ವ-ವ-ನಂತೆ
ಮಾಡು-ವ-ವ-ನ-ದಾ-ಯಿತು
ಮಾಡು-ವ-ವ-ನನ್ನು
ಮಾಡು-ವ-ವ-ನಲ್ಲ
ಮಾಡು-ವ-ವ-ನಿ-ಗಿಂತ
ಮಾಡು-ವ-ವ-ನಿಗೂ
ಮಾಡು-ವ-ವ-ನಿಗೆ
ಮಾಡು-ವ-ವನು
ಮಾಡು-ವ-ವನೂ
ಮಾಡು-ವ-ವನೆ
ಮಾಡು-ವ-ವನೇ
ಮಾಡು-ವ-ವ-ರನ್ನು
ಮಾಡು-ವ-ವ-ರ-ನ್ನೆಲ್ಲ
ಮಾಡು-ವ-ವ-ರಲ್ಲಿ
ಮಾಡು-ವ-ವ-ರಿಗೂ
ಮಾಡು-ವ-ವ-ರಿಗೆ
ಮಾಡು-ವ-ವ-ರಿ-ಲ್ಲದೆ
ಮಾಡು-ವ-ವರು
ಮಾಡು-ವ-ವರೂ
ಮಾಡು-ವ-ವರೇ
ಮಾಡು-ವಾಗ
ಮಾಡು-ವಾ-ಗಲೂ
ಮಾಡು-ವಾ-ಗಲೆ
ಮಾಡುವು
ಮಾಡು-ವುದ
ಮಾಡು-ವು-ದ-ಕ್ಕಲ್ಲ
ಮಾಡು-ವು-ದ-ಕ್ಕಾ-ಗಲಿ
ಮಾಡು-ವು-ದ-ಕ್ಕಾಗಿ
ಮಾಡು-ವು-ದ-ಕ್ಕಾ-ಗು-ವು-ದಿಲ್ಲ
ಮಾಡು-ವು-ದ-ಕ್ಕಿಂತ
ಮಾಡು-ವು-ದ-ಕ್ಕಿಂ-ತಲೂ
ಮಾಡು-ವು-ದಕ್ಕೂ
ಮಾಡು-ವು-ದಕ್ಕೆ
ಮಾಡು-ವು-ದಕ್ಕೇ
ಮಾಡು-ವುದನ್ನು
ಮಾಡು-ವು-ದ-ನ್ನೆಲ್ಲಾ
ಮಾಡು-ವು-ದನ್ನೇ
ಮಾಡು-ವು-ದರ
ಮಾಡು-ವು-ದ-ರಲ್ಲಿ
ಮಾಡು-ವು-ದ-ರ-ಲ್ಲಿಯೇ
ಮಾಡು-ವು-ದ-ರಿಂದ
ಮಾಡು-ವು-ದಲ್ಲ
ಮಾಡು-ವು-ದಾ-ದರೂ
ಮಾಡು-ವು-ದಿಲ್ಲ
ಮಾಡು-ವು-ದಿ-ಲ್ಲವೆ
ಮಾಡು-ವು-ದಿ-ಲ್ಲ-ವೆಂ-ದರೂ
ಮಾಡು-ವು-ದಿ-ಲ್ಲ-ವೆಂದು
ಮಾಡು-ವು-ದಿ-ಲ್ಲವೊ
ಮಾಡು-ವು-ದಿ-ಲ್ಲವೋ
ಮಾಡು-ವುದು
ಮಾಡು-ವುದೂ
ಮಾಡು-ವುದೆ
ಮಾಡು-ವು-ದೆಲ್ಲ
ಮಾಡು-ವು-ದೆಲ್ಲಾ
ಮಾಡು-ವುದೇ
ಮಾಡು-ವು-ದೇನು
ಮಾಡು-ವುದೊ
ಮಾಡು-ವು-ದೊಂದು
ಮಾಡು-ವುದೋ
ಮಾಡು-ವುವು
ಮಾಡು-ವುವೋ
ಮಾಡುವೆ
ಮಾಡು-ವೆವು
ಮಾಡು-ವೆವೊ
ಮಾಡು-ವೆವೋ
ಮಾಡೆ-ನೆಂ-ದರೂ
ಮಾಡೇ-ತೀ-ರ-ಬೇಕು
ಮಾಡೋಣ
ಮಾಣಿಕ್ಯ
ಮಾಣಿ-ಕ್ಯ-ವನ್ನು
ಮಾತ
ಮಾತ-ನಾಡ
ಮಾತ-ನಾ-ಡದೆ
ಮಾತ-ನಾ-ಡ-ಬ-ಹುದು
ಮಾತ-ನಾ-ಡ-ಬೇ-ಕಾ-ಗು-ವುದು
ಮಾತ-ನಾ-ಡ-ಬೇಡಿ
ಮಾತ-ನಾಡಿ
ಮಾತ-ನಾ-ಡಿ-ಕೊಂಡು
ಮಾತ-ನಾ-ಡಿ-ದರೆ
ಮಾತ-ನಾ-ಡಿ-ದಾಗ
ಮಾತ-ನಾ-ಡಿ-ರು-ವೆನು
ಮಾತ-ನಾಡು
ಮಾತ-ನಾ-ಡುತ್ತಾ
ಮಾತ-ನಾ-ಡು-ತ್ತಾನೆ
ಮಾತ-ನಾ-ಡು-ತ್ತಾ-ರೆಯೋ
ಮಾತ-ನಾ-ಡು-ತ್ತಿ-ದ್ದರು
ಮಾತ-ನಾ-ಡು-ತ್ತಿ-ದ್ದಾಗ
ಮಾತ-ನಾ-ಡು-ತ್ತಿ-ರು-ವನು
ಮಾತ-ನಾ-ಡು-ತ್ತಿ-ರು-ವಾಗ
ಮಾತ-ನಾ-ಡು-ತ್ತಿ-ರು-ವು-ದರ
ಮಾತ-ನಾ-ಡು-ತ್ತಿ-ರುವೆ
ಮಾತ-ನಾ-ಡು-ತ್ತಿ-ರು-ವೆವು
ಮಾತ-ನಾ-ಡುವ
ಮಾತ-ನಾ-ಡು-ವನು
ಮಾತ-ನಾ-ಡು-ವರು
ಮಾತ-ನಾ-ಡು-ವ-ವ-ರೆಷ್ಟೊ
ಮಾತ-ನಾ-ಡು-ವಾಗ
ಮಾತ-ನಾ-ಡು-ವಾ-ಗಲೂ
ಮಾತ-ನಾ-ಡು-ವುದನ್ನು
ಮಾತ-ನಾ-ಡು-ವು-ದಿಲ್ಲ
ಮಾತ-ನಾ-ಡು-ವುದು
ಮಾತ-ನಾ-ಡೋಣ
ಮಾತ-ನ್ನಾ-ದರೂ
ಮಾತನ್ನು
ಮಾತನ್ನೂ
ಮಾತನ್ನೆ
ಮಾತನ್ನೇ
ಮಾತನ್ನೊ
ಮಾತಲ್ಲ
ಮಾತಾ
ಮಾತಾ-ಗಲೀ
ಮಾತಾ-ಡು-ತ್ತಾನೆ
ಮಾತಾ-ಡು-ತ್ತೇವೆ
ಮಾತಾ-ಡು-ವನು
ಮಾತಾ-ಡು-ವು-ದ-ರಲ್ಲಿ
ಮಾತಾ-ಡು-ವುದು
ಮಾತಾ-ದರೂ
ಮಾತಿ
ಮಾತಿಗೂ
ಮಾತಿಗೆ
ಮಾತಿನ
ಮಾತಿ-ನಂತೆ
ಮಾತಿ-ನಲ್ಲಿ
ಮಾತಿ-ನ-ಲ್ಲಿ-ರುವ
ಮಾತಿ-ನಲ್ಲೆ
ಮಾತಿ-ನ-ಲ್ಲೆಲ್ಲ
ಮಾತಿ-ನಷ್ಟು
ಮಾತಿ-ನಿಂದ
ಮಾತಿ-ನಿಂ-ದಲೂ
ಮಾತಿಲ್ಲ
ಮಾತು
ಮಾತು-ಕತೆ
ಮಾತು-ಕ-ತೆ-ಗಳನ್ನು
ಮಾತು-ಕ-ತೆ-ಗಳಲ್ಲಿ
ಮಾತು-ಕ-ತೆ-ಯ-ಲ್ಲದೆ
ಮಾತು-ಕ-ತೆ-ಯಲ್ಲಿ
ಮಾತು-ಕ-ತೆ-ಯಾಡು
ಮಾತು-ಕ-ತೆ-ಯಾ-ಡು-ವನು
ಮಾತು-ಗಳನ್ನು
ಮಾತು-ಗಳು
ಮಾತು-ಗಳೇ
ಮಾತು-ಲಾಃ
ಮಾತೆ
ಮಾತೆಂಬ
ಮಾತೆತ್ತಿ
ಮಾತೆ-ತ್ತಿ-ದರೆ
ಮಾತೆಲ್ಲ
ಮಾತೊಂದು
ಮಾತ್ತ
ಮಾತ್ಮನ
ಮಾತ್ಮ-ನಲ್ಲಿ
ಮಾತ್ರ
ಮಾತ್ರಕ್ಕೆ
ಮಾತ್ರ-ದಲ್ಲಿ
ಮಾತ್ರ-ನಾ-ಗಿದ್ದ
ಮಾತ್ರ-ವಲ್ಲ
ಮಾತ್ರ-ವಾಗಿ
ಮಾತ್ರವೆ
ಮಾತ್ರ-ವೆಲ್ಲ
ಮಾತ್ರವೇ
ಮಾತ್ರಾ
ಮಾತ್ರಾ-ಸ್ಪ-ರ್ಶಾಸ್ತು
ಮಾತ್ಸರ್ಯ
ಮಾತ್ಸ-ರ್ಯದ
ಮಾತ್ಸ-ರ್ಯ-ವಿಲ್ಲ
ಮಾತ್ಸ-ರ್ಯ-ವಿ-ಲ್ಲದೆ
ಮಾದಕ
ಮಾದ-ರಿಯ
ಮಾಧವ
ಮಾಧವಃ
ಮಾಧ-ವ-ನನ್ನು
ಮಾನ
ಮಾನಕ್ಕೆ
ಮಾನದ
ಮಾನವ
ಮಾನವಃ
ಮಾನ-ವ-ಕೋಟಿ
ಮಾನ-ವ-ಕೋ-ಟಿಗೆ
ಮಾನ-ವ-ಕೋ-ಟಿಯ
ಮಾನ-ವ-ಕೋ-ಟಿ-ಯನ್ನೇ
ಮಾನ-ವ-ಕೋ-ಟಿ-ಯಲ್ಲಿ
ಮಾನ-ವನ
ಮಾನ-ವ-ನಂ-ತಿ-ರುವ
ಮಾನ-ವ-ನಂತೆ
ಮಾನ-ವ-ನ-ಕೋ-ಟಿಗೆ
ಮಾನ-ವ-ನನ್ನು
ಮಾನ-ವ-ನಲ್ಲಿ
ಮಾನ-ವ-ನ-ವ-ರೆಗೆ
ಮಾನ-ವ-ನಾ-ಗಿ-ರು-ವನೋ
ಮಾನ-ವ-ನಿಗೆ
ಮಾನ-ವನ್ನು
ಮಾನ-ವರ
ಮಾನ-ವ-ರಲ್ಲಿ
ಮಾನ-ವ-ರಾಗಿ
ಮಾನ-ವ-ರಿ-ಗಾಗಿ
ಮಾನ-ವ-ರಿಗೆ
ಮಾನ-ವ-ರಿ-ಗೆಲ್ಲ
ಮಾನ-ವ-ರಿಲ್ಲ
ಮಾನ-ವರು
ಮಾನ-ವರೆ-ಲ್ಲ-ರಿ-ಗಿಂ-ತಲೂ
ಮಾನ-ವರೆ-ಲ್ಲರೂ
ಮಾನ-ವರೇ
ಮಾನ-ವ-ಸ-ಹ-ಜ-ವಾ-ಗಿ-ರುವ
ಮಾನ-ವ-ಸ-ಹ-ಜ-ವಾದ
ಮಾನ-ವ-ಸ-ಹ-ಜ-ವಾ-ದುದು
ಮಾನ-ವಾಃ
ಮಾನ-ವಾ-ಕಾ-ರ-ವನ್ನು
ಮಾನ-ವಾ-ಕೃತಿ
ಮಾನ-ವಿಲ್ಲ
ಮಾನ-ವೀಯ
ಮಾನಸ
ಮಾನ-ಸ-ಮು-ಚ್ಯತೇ
ಮಾನಸಾ
ಮಾನ-ಸಿಕ
ಮಾನ-ಸಿ-ಕ-ವಾ-ದುದು
ಮಾನಾ-ಪ-ಮಾ-ನ-ಗಳಲ್ಲಿ
ಮಾನಾ-ಪ-ಮಾ-ನ-ಗ-ಳ-ಲ್ಲಿಯೂ
ಮಾನಾ-ಪ-ಮಾ-ನ-ಯೋಃ
ಮಾನಾ-ಪ-ಮಾ-ನ-ಯೋ-ಸ್ತು-ಲ್ಯ-ಸ್ತುಲ್ಯೋ
ಮಾನಿ-ಸು-ವು-ದಿಲ್ಲ
ಮಾನುಷ
ಮಾನುಷಂ
ಮಾನು-ಷೀಂ
ಮಾನುಷೇ
ಮಾನ್ಯತೆ
ಮಾನ್ಯ-ಮಾ-ಡು-ವು-ದಿಲ್ಲ
ಮಾಮ-ಕಮ್
ಮಾಮ-ಕಾಃ
ಮಾಮ-ಜ-ಮ-ನಾ-ದಿಂ
ಮಾಮ-ಜ-ಮ-ವ್ಯ-ಯಮ್
ಮಾಮ-ನ-ನ್ಯ-ಭಾಕ್
ಮಾಮ-ನು-ಸ್ಮರ
ಮಾಮ-ನು-ಸ್ಮ-ರನ್
ಮಾಮಪಿ
ಮಾಮ-ಪ್ರ-ತೀ-ಕಾ-ರ-ಮ-ಶಸ್ತ್ರಂ
ಮಾಮ-ಪ್ರಾ-ಪ್ಯೈವ
ಮಾಮ-ಬು-ದ್ಧಯಃ
ಮಾಮ-ಭಿ-ಜಾ-ನಂತಿ
ಮಾಮ-ಭಿ-ಜಾ-ನಾತಿ
ಮಾಮ-ಮೃ-ತೋ-ದ್ಭ-ವಮ್
ಮಾಮಾ-ತ್ಮ-ಪ-ರ-ದೇ-ಹೇಷು
ಮಾಮಾ-ಶ್ರಿತ್ಯ
ಮಾಮಿ-ಕಾಮ್
ಮಾಮಿ-ಚ್ಛಾ-ಪ್ತುಂ
ಮಾಮು-ಪ-ಯಾಂತಿ
ಮಾಮು-ಪಾ-ಶ್ರಿ-ತಾಃ
ಮಾಮು-ಪಾ-ಸತೇ
ಮಾಮು-ಪೇತ್ಯ
ಮಾಮು-ಪೈ-ಷ್ಯಸಿ
ಮಾಮೂಲು
ಮಾಮೇಕಂ
ಮಾಮೇತಿ
ಮಾಮೇಭ್ಯಃ
ಮಾಮೇವ
ಮಾಮೇ-ವ-ಮ-ಸಂ-ಮೂಢೋ
ಮಾಮೇ-ವಾ-ನು-ತ್ತ-ಮಾಂ
ಮಾಮೇ-ವೈ-ಷ್ಯ-ತ್ಯ-ಸಂ-ಶಯಃ
ಮಾಮೇ-ವೈ-ಷ್ಯಸಿ
ಮಾಮ್
ಮಾಯ
ಮಾಯಯಾ
ಮಾಯ-ಯಾ-ಪ-ಹೃ-ತ-ಜ್ಞಾನಾ
ಮಾಯ-ವಾ-ಗ-ಬೇ-ಕಾ-ದರೆ
ಮಾಯ-ವಾ-ಗ-ಲಾರ
ಮಾಯ-ವಾಗಿ
ಮಾಯ-ವಾ-ಗಿ-ದೆಯೋ
ಮಾಯ-ವಾ-ಗಿವೆ
ಮಾಯ-ವಾ-ಗು-ವಂತೆ
ಮಾಯ-ವಾ-ಗು-ವನು
ಮಾಯ-ವಾ-ಗು-ವು-ದಿಲ್ಲ
ಮಾಯ-ವಾ-ಗು-ವುದು
ಮಾಯ-ವಾ-ಗು-ವುದೋ
ಮಾಯ-ವಾ-ಗು-ವುವು
ಮಾಯ-ವಾ-ಗು-ವುವೋ
ಮಾಯ-ವಾದ
ಮಾಯ-ವಾ-ದರೂ
ಮಾಯಾ
ಮಾಯಾ-ತೆ-ರೆ-ಯನ್ನು
ಮಾಯಾ-ಧಿ-ಪ-ತಿ-ಯನ್ನು
ಮಾಯಾ-ಧೀ-ಶನೋ
ಮಾಯಾ-ಮೇ-ತಾಂ
ಮಾಯಾವಿ
ಮಾಯೆ
ಮಾಯೆಗೆ
ಮಾಯೆಯ
ಮಾಯೆ-ಯನ್ನು
ಮಾಯೆ-ಯಲ್ಲಿ
ಮಾಯೆ-ಯಿಂದ
ಮಾಯೆ-ಯೆಂಬ
ಮಾರನೆ
ಮಾರ-ನೆಯ
ಮಾರಮ್ಮ
ಮಾರ-ಮ್ಮ-ನನ್ನೋ
ಮಾರಲು
ಮಾರಾ-ಟದ
ಮಾರಿ
ಮಾರಿ-ಕೊಂಡ
ಮಾರಿ-ಕೊಂ-ಡ-ವನ
ಮಾರಿ-ಕೊ-ಳ್ಳ-ಬೇಕು
ಮಾರಿ-ಕೊ-ಳ್ಳು-ತ್ತೇವೆ
ಮಾರಿಗೆ
ಮಾರಿ-ಯನ್ನು
ಮಾರಿಯೇ
ಮಾರಿ-ಹಾ-ಕು-ವನು
ಮಾರೀಚ
ಮಾರೀ-ಚ-ನಲ್ಲಿ
ಮಾರು
ಮಾರುತಃ
ಮಾರುದ್ದ
ಮಾರುವ
ಮಾರು-ವ-ವ-ನಿ-ಗಲ್ಲ
ಮಾರು-ಹೋ-ಗು-ತ್ತಾನೆ
ಮಾರು-ಹೋ-ಗು-ತ್ತಿದ್ದ
ಮಾರು-ಹೋ-ಗು-ತ್ತೇವೆ
ಮಾರ್ಕೆ-ಟ್ಟಿ-ನಲ್ಲಿ
ಮಾರ್ಗ
ಮಾರ್ಗಕ್ಕೆ
ಮಾರ್ಗ-ಗಳನ್ನು
ಮಾರ್ಗ-ಗ-ಳ-ಲ್ಲಿಯೂ
ಮಾರ್ಗ-ಗ-ಳ-ಲ್ಲೆಲ್ಲ
ಮಾರ್ಗ-ಗ-ಳಿವೆ
ಮಾರ್ಗ-ಗಳು
ಮಾರ್ಗ-ಗಳೂ
ಮಾರ್ಗದ
ಮಾರ್ಗ-ದ-ರ್ಶ-ಕ-ನಂತೆ
ಮಾರ್ಗ-ದ-ರ್ಶ-ಕ-ನಾ-ಗು-ತ್ತಾನೆ
ಮಾರ್ಗ-ದ-ರ್ಶ-ಕನೇ
ಮಾರ್ಗ-ದ-ರ್ಶ-ಕ-ರಂತೆ
ಮಾರ್ಗ-ದ-ರ್ಶನ
ಮಾರ್ಗ-ದರ್ಶಿ
ಮಾರ್ಗ-ದ-ಲ್ಲಾ-ದರೂ
ಮಾರ್ಗ-ದಲ್ಲಿ
ಮಾರ್ಗ-ದ-ಲ್ಲಿಯೂ
ಮಾರ್ಗ-ದ-ಲ್ಲಿ-ರುವ
ಮಾರ್ಗ-ವನ್ನು
ಮಾರ್ಗ-ವನ್ನೂ
ಮಾರ್ಗ-ವನ್ನೇ
ಮಾರ್ಗ-ವಾಗಿ
ಮಾರ್ಗ-ವಿದೆ
ಮಾರ್ಗ-ವಿಲ್ಲ
ಮಾರ್ಗ-ವಿ-ಲ್ಲದೆ
ಮಾರ್ಗವೇ
ಮಾರ್ಗ-ಶಿರ
ಮಾರ್ಗ-ಶೀರ್ಷ
ಮಾರ್ಗ-ಶೀ-ರ್ಷೋ-ಽಹ-ಮೃ-ತೂ-ನಾಂ
ಮಾರ್ಗಾ-ವ-ಲಂ-ಬಿ-ಗಳು
ಮಾರ್ಥಿ-ಕವೇ
ಮಾರ್ದವಂ
ಮಾರ್ಪ-ಡಿ-ಸಿ-ಕೊಂ-ಡಿ-ರು-ವರೊ
ಮಾರ್ಪ-ಡಿಸು
ಮಾರ್ಪ-ಡಿ-ಸು-ತ್ತದೆ
ಮಾರ್ಪ-ಡಿ-ಸು-ವು-ದಕ್ಕೆ
ಮಾರ್ಪ-ಡಿ-ಸು-ವುದು
ಮಾರ್ಪಾ-ಡಾಗಿ
ಮಾರ್ಪಾ-ಡಾ-ಗಿದೆ
ಮಾರ್ಪಾ-ಡಾ-ಗು-ತ್ತಾನೆ
ಮಾಲಕ
ಮಾಲ-ವೀಯ
ಮಾಲೆ
ಮಾಲೆ-ಗಳನ್ನು
ಮಾಲೆ-ಯಂತೆ
ಮಾಲೆ-ಯನ್ನು
ಮಾಲೆ-ಯಲ್ಲ
ಮಾವಂ-ದಿರು
ಮಾವಿನ
ಮಾಸ-ಗಳಲ್ಲಿ
ಮಾಸ-ಗಳು
ಮಾಸ-ಗ-ಳು-ಇ-ವು-ಗಳಲ್ಲಿ
ಮಾಸ-ದಲ್ಲಿ
ಮಾಸ-ದ-ಲ್ಲಿಯೇ
ಮಾಸ-ದಿಂ-ದಲೇ
ಮಾಸ-ಲಾ-ಗು-ವಂತೆ
ಮಾಸಾ-ನಾಂ
ಮಾಹಾ-ತ್ಮ್ಯ-ಮಪಿ
ಮಾಹುತ
ಮಿಂಚಿ
ಮಿಂಚಿ-ದಂತೆ
ಮಿಂಚಿನ
ಮಿಂಚಿ-ನಂತೆ
ಮಿಂಚು
ಮಿಂಚು-ತ್ತಿ-ರುವ
ಮಿಂಚು-ಹು-ಳು-ಗ-ಳಂತೆ
ಮಿಂಚು-ಹು-ಳು-ವಿನ
ಮಿಂದರೆ
ಮಿಂದು
ಮಿಕ್ಕ
ಮಿಕ್ಕ-ದ್ದೆಲ್ಲ
ಮಿಕ್ಕಿದ್ದು
ಮಿಕ್ಕಿರು
ಮಿಕ್ಕಿ-ರುವ
ಮಿಕ್ಕಿ-ರು-ವುದನ್ನು
ಮಿಕ್ಕಿ-ರು-ವು-ದಾ-ವುದೂ
ಮಿಕ್ಕಿ-ರು-ವುದು
ಮಿಕ್ಕಿ-ರು-ವು-ದೆಲ್ಲ
ಮಿಕ್ಕಿ-ರು-ವು-ದೆಲ್ಲಾ
ಮಿಗಿ-ಲಾ-ಗಿತ್ತು
ಮಿಗಿ-ಲಾ-ಗಿಲ್ಲ
ಮಿಗಿ-ಲಾದ
ಮಿಗಿ-ಲಾ-ದ-ವನು
ಮಿಗಿ-ಲಾ-ದುದು
ಮಿಗಿಲು
ಮಿಗು-ವುದನ್ನು
ಮಿಗು-ವುದೇ
ಮಿಚ್ಛಾಮಿ
ಮಿಟುಕು
ಮಿಡಿ-ದಾಗ
ಮಿಡಿ-ಯಾ-ಗು-ವುದು
ಮಿಡಿ-ಯು-ತ್ತಿ-ರು-ವನು
ಮಿಡಿ-ಯು-ವನು
ಮಿಡಿ-ಯು-ವುದು
ಮಿಣುಕು
ಮಿತ-ದಲ್ಲಿ
ಮಿತ-ವನ್ನು
ಮಿತ-ವಾಗಿ
ಮಿತ-ವಾ-ಗಿ-ರ-ಬೇಕು
ಮಿತ-ವಾದ
ಮಿತಾ-ಹಾರ
ಮಿತಾ-ಹಾರಿ
ಮಿತಾ-ಹಾ-ರಿ-ಯಾಗಿ
ಮಿತಾ-ಹಾ-ರಿ-ಯಾ-ಗಿ-ರ-ಬೇಕು
ಮಿತಿ
ಮಿತಿ-ಗ-ಳಿವೆ
ಮಿತಿ-ಮೀರಿ
ಮಿತಿ-ಯನ್ನು
ಮಿತಿ-ಯನ್ನೂ
ಮಿತಿ-ಯಲ್ಲಿ
ಮಿತಿ-ಯಿಂದ
ಮಿತ್ರ
ಮಿತ್ರ-ದ್ರೋ-ಹ-ದ-ಲ್ಲಿ-ರುವ
ಮಿತ್ರ-ದ್ರೋಹೇ
ಮಿತ್ರನೂ
ಮಿತ್ರ-ನೆಂದು
ಮಿತ್ರ-ಭಾ-ವ-ದಿಂದ
ಮಿತ್ರ-ರನ್ನು
ಮಿತ್ರ-ರಲ್ಲಿ
ಮಿತ್ರ-ರಾಗಿ
ಮಿತ್ರ-ರಾರು
ಮಿತ್ರಾ-ರಿ-ಪ-ಕ್ಷ-ಯೋಃ
ಮಿತ್ರೇ
ಮಿಥ್ಯ
ಮಿಥ್ಯದ
ಮಿಥ್ಯ-ದೊ-ಡನೆ
ಮಿಥ್ಯ-ವನ್ನು
ಮಿಥ್ಯ-ವಾದ
ಮಿಥ್ಯಾ
ಮಿಥ್ಯಾ-ಚಾರ
ಮಿಥ್ಯಾ-ಚಾರಃ
ಮಿಥ್ಯಾ-ಚಾರಿ
ಮಿಥ್ಯಾ-ಚಾ-ರಿ-ಗಿಂತ
ಮಿಥ್ಯಾ-ಚಾ-ರಿಗೆ
ಮಿಥ್ಯಾ-ಮೋ-ಹದ
ಮಿಥ್ಯಾ-ವ-ಸ್ತು-ಗಳನ್ನು
ಮಿಥ್ಯಾ-ವ-ಸ್ತು-ವನ್ನು
ಮಿಥ್ಯಾ-ವೇ-ಷ-ದಿಂದ
ಮಿಥ್ಯೈಷ
ಮಿನು-ಗು-ತ್ತಿ-ರುವ
ಮಿಲಿ-ಟರಿ
ಮಿಲಿ-ಟ-ರಿಗೆ
ಮಿಲಿ-ಟ-ರಿಯ
ಮಿಲಿ-ಟ-ರಿ-ಯ-ವರು
ಮಿಲ್ಲಿಗೆ
ಮಿಶ್ರ
ಮಿಶ್ರಂ
ಮಿಶ್ರದ
ಮಿಶ್ರ-ದಿಂದ
ಮಿಶ್ರ-ಫ-ಲ-ವಾ-ದರೆ
ಮಿಶ್ರ-ಮಾ-ಡಿದ
ಮಿಶ್ರ-ವನ್ನು
ಮಿಶ್ರ-ವಾಗಿ
ಮಿಶ್ರ-ವಾ-ಗಿದೆ
ಮಿಶ್ರ-ವಾ-ಗಿ-ರಲಿ
ಮಿಶ್ರ-ವಾ-ಗಿ-ರು-ವುದು
ಮಿಶ್ರ-ವಾದ
ಮಿಶ್ರ-ವಾ-ದಂ-ತಿದೆ
ಮಿಶ್ರ-ವೆಲ್ಲ
ಮಿಶ್ರವೇ
ಮೀಟುತ್ತ
ಮೀಟು-ವನು
ಮೀನನ್ನು
ಮೀನಿನ
ಮೀನಿ-ನಂತೆ
ಮೀನು
ಮೀನು-ಗ-ಳಂತೆ
ಮೀನು-ಗಳನ್ನು
ಮೀನು-ಗಳಲ್ಲಿ
ಮೀನು-ಗ-ಳಿ-ಗಿಂತ
ಮೀಯಲು
ಮೀಯು-ತ್ತಿ-ದ್ದರು
ಮೀಯು-ತ್ತಿದ್ದೆ
ಮೀಯು-ವರು
ಮೀರ-ಲಾ-ರರು
ಮೀರಲು
ಮೀರಿ
ಮೀರಿತು
ಮೀರಿದ
ಮೀರಿ-ದಂ-ತೆಯೂ
ಮೀರಿ-ದರೆ
ಮೀರಿ-ದ-ವ-ನನ್ನು
ಮೀರಿ-ದ-ವನು
ಮೀರಿ-ದ-ವ-ರಿಲ್ಲ
ಮೀರಿ-ದ-ವರು
ಮೀರಿದೆ
ಮೀರಿ-ದ್ದರೆ
ಮೀರಿಯೂ
ಮೀರಿ-ರು-ವನು
ಮೀರಿ-ರು-ವನೆ
ಮೀರಿ-ರು-ವುದು
ಮೀರಿಲ್ಲ
ಮೀರಿಸಿ
ಮೀರಿ-ಸುವ
ಮೀರಿ-ಹೋ-ಗ-ಬೇಕು
ಮೀರಿ-ಹೋ-ಗಿಲ್ಲ
ಮೀರಿ-ಹೋ-ಗು-ತ್ತಾನೆ
ಮೀರಿ-ಹೋ-ಗು-ವು-ದಕ್ಕೆ
ಮೀರು
ಮೀರುವ
ಮೀರು-ವಂತೆ
ಮೀರು-ವ-ವ-ರಾರೂ
ಮೀರು-ವ-ವ-ರಿಲ್ಲ
ಮೀಸ-ಲಲ್ಲ
ಮೀಸ-ಲಾಗಿ
ಮೀಸ-ಲಾ-ಗಿ-ಟ್ಟಿ-ದ್ದರೆ
ಮೀಸ-ಲಾ-ಗಿ-ಟ್ಟಿರು
ಮೀಸ-ಲಾ-ಗಿ-ಟ್ಟಿಲ್ಲ
ಮೀಸ-ಲಾ-ಗಿ-ಡ-ಬೇಕು
ಮೀಸ-ಲಾ-ಗಿತ್ತು
ಮೀಸ-ಲಾಗಿದೆ
ಮೀಸ-ಲಾ-ಗಿ-ರು-ವುದು
ಮೀಸ-ಲಾ-ಗಿಲ್ಲ
ಮೀಸ-ಲಿಲ್ಲ
ಮೀಸಲು
ಮುಂಚಿನ
ಮುಂಚಿ-ನಿಂ-ದಲೇ
ಮುಂಚೆ
ಮುಂಚೆಯೂ
ಮುಂಚೆಯೆ
ಮುಂಚೆಯೇ
ಮುಂಜಾ-ಗ್ರ-ತೆ-ಯ-ನ್ನೆಲ್ಲ
ಮುಂಜಿ
ಮುಂಡಕ
ಮುಂತಾ
ಮುಂತಾದ
ಮುಂತಾ-ದವ
ಮುಂತಾ-ದ-ವ-ರನ್ನು
ಮುಂತಾ-ದ-ವ-ರ-ನ್ನೆಲ್ಲ
ಮುಂತಾ-ದ-ವ-ರಿಗೆ
ಮುಂತಾ-ದ-ವರು
ಮುಂತಾ-ದ-ವರೆ
ಮುಂತಾ-ದ-ವರೆಲ್ಲ
ಮುಂತಾ-ದ-ವ-ರೊ-ಡನೆ
ಮುಂತಾ-ದವು
ಮುಂತಾ-ದ-ವು-ಗಳನ್ನು
ಮುಂತಾ-ದ-ವು-ಗಳಾ
ಮುಂತಾ-ದ-ವು-ಗಳಿಂದ
ಮುಂತಾ-ದ-ವು-ಗ-ಳಿಗೆ
ಮುಂತಾ-ದ-ವು-ಗಳು
ಮುಂತಾ-ದ-ವು-ಗಳೆ
ಮುಂತಾ-ದ-ವು-ಗ-ಳೆಲ್ಲ
ಮುಂತಾದು
ಮುಂತಾ-ದು-ವಕ್ಕೆ
ಮುಂತಾ-ದು-ವನ್ನು
ಮುಂತಾ-ದುವು
ಮುಂತಾ-ದು-ವು-ಗಳ
ಮುಂತಾ-ದು-ವು-ಗಳನ್ನು
ಮುಂತಾ-ದು-ವು-ಗಳನ್ನೆಲ್ಲ
ಮುಂತಾ-ದು-ವು-ಗ-ಳನ್ನೇ
ಮುಂತಾ-ದು-ವು-ಗ-ಳ-ಲ್ಲದೆ
ಮುಂತಾ-ದು-ವು-ಗಳಲ್ಲಿ
ಮುಂತಾ-ದು-ವು-ಗಳಿಂದ
ಮುಂತಾ-ದು-ವು-ಗ-ಳಿಗೆ
ಮುಂತಾ-ದು-ವು-ಗಳು
ಮುಂತಾ-ದು-ವು-ಗ-ಳೆಲ್ಲ
ಮುಂತಾ-ದು-ವು-ಗ-ಳೆ-ಲ್ಲವೂ
ಮುಂತಾ-ದು-ವು-ಗ-ಳೆಲ್ಲಾ
ಮುಂತಾ-ದು-ವು-ಗಳೇ
ಮುಂತಾ-ದುವೆ
ಮುಂತಾ-ದು-ವೆಲ್ಲ
ಮುಂತಾ-ದುವೇ
ಮುಂದಕ್ಕೆ
ಮುಂದಡಿ
ಮುಂದಾ-ಗಿ-ದೆಯೋ
ಮುಂದಾ-ಗು-ವುದನ್ನು
ಮುಂದಾ-ಗು-ವು-ದ-ನ್ನೆಲ್ಲ
ಮುಂದಾ-ಲೋ-ಚನೆ
ಮುಂದಾ-ಳಾ-ಗಿ-ದ್ದಾ-ನೆಯೋ
ಮುಂದಾ-ಳಾ-ಗಿ-ರು-ವನೋ
ಮುಂದಾಳು
ಮುಂದಿ-ಟ್ಟು-ಕೊಂಡು
ಮುಂದಿ-ಡ-ಬೇ-ಕಾ-ಗಿದೆ
ಮುಂದಿ-ಡು-ವು-ದಕ್ಕೆ
ಮುಂದಿನ
ಮುಂದಿ-ನ-ದಕ್ಕೆ
ಮುಂದಿ-ನ-ದನ್ನು
ಮುಂದಿ-ನದು
ಮುಂದಿ-ನ-ವ-ರನ್ನು
ಮುಂದಿ-ನ-ವ-ರಿಗೆ
ಮುಂದಿ-ನ-ವರು
ಮುಂದಿ-ರುವ
ಮುಂದಿ-ರು-ವನು
ಮುಂದಿ-ರು-ವು-ದ-ನ್ನೆಲ್ಲ
ಮುಂದಿ-ರು-ವುದು
ಮುಂದಿಲ್ಲ
ಮುಂದು
ಮುಂದು-ಗಡೆ
ಮುಂದು-ಗಾ-ಣದ
ಮುಂದು-ವರಿ
ಮುಂದು-ವ-ರಿದ
ಮುಂದು-ವ-ರಿ-ದಂತೆ
ಮುಂದು-ವ-ರಿ-ದರೂ
ಮುಂದು-ವ-ರಿ-ದರೆ
ಮುಂದು-ವ-ರಿ-ದ-ವನು
ಮುಂದು-ವ-ರಿ-ದಿ-ದ್ದರೆ
ಮುಂದು-ವ-ರಿದು
ಮುಂದು-ವ-ರಿ-ಯ-ಬೇ-ಕಾ-ಗಿತ್ತು
ಮುಂದು-ವ-ರಿ-ಯ-ಬೇ-ಕಾ-ಗಿ-ರು-ವುದ
ಮುಂದು-ವ-ರಿ-ಯ-ಬೇ-ಕಾ-ದರೆ
ಮುಂದು-ವ-ರಿ-ಯ-ಬೇಕು
ಮುಂದು-ವ-ರಿ-ಯ-ಬೇ-ಕೆಂದು
ಮುಂದು-ವ-ರಿ-ಯ-ಲಾರ
ಮುಂದು-ವ-ರಿ-ಯ-ಲಾ-ರದೆ
ಮುಂದು-ವ-ರಿಯು
ಮುಂದು-ವ-ರಿ-ಯುತ್ತ
ಮುಂದು-ವ-ರಿ-ಯು-ತ್ತಿದೆ
ಮುಂದು-ವ-ರಿ-ಯು-ವಂತೆ
ಮುಂದು-ವ-ರಿ-ಯು-ವಂ-ತೆಯೇ
ಮುಂದು-ವ-ರಿ-ಯು-ವನು
ಮುಂದು-ವ-ರಿ-ಯು-ವಾಗ
ಮುಂದು-ವ-ರಿ-ಯು-ವು-ದ-ಕ್ಕಾ-ಗು-ವು-ದಿಲ್ಲ
ಮುಂದು-ವ-ರಿ-ಯು-ವು-ದಕ್ಕೆ
ಮುಂದು-ವ-ರಿ-ಯು-ವು-ದಕ್ಕೇ
ಮುಂದು-ವ-ರಿ-ಯು-ವು-ದಿಲ್ಲ
ಮುಂದು-ವ-ರಿ-ಯು-ವುದು
ಮುಂದು-ವ-ರಿ-ಸ-ಬೇಕು
ಮುಂದು-ವ-ರಿಸಿ
ಮುಂದು-ವ-ರಿ-ಸಿ-ಕೊಂಡು
ಮುಂದು-ವ-ರಿ-ಸಿ-ದರೆ
ಮುಂದು-ವ-ರಿ-ಸು-ತ್ತಾನೆ
ಮುಂದು-ವ-ರಿ-ಸು-ತ್ತಿ-ರು-ವರು
ಮುಂದು-ವ-ರಿ-ಸು-ವನು
ಮುಂದು-ವರೆ-ದಿ-ರು-ವರೋ
ಮುಂದು-ವರೆ-ಯ-ಬ-ಹುದು
ಮುಂದೂ-ಡಿದೆ
ಮುಂದೂ-ಡು-ತ್ತ-ದೆಯೆ
ಮುಂದೂ-ಡು-ವನು
ಮುಂದೆ
ಮುಂದೆ-ಯಾ-ದರೂ
ಮುಂದೆಯೂ
ಮುಂದೆಯೇ
ಮುಂದೆಲ್ಲ
ಮುಂದೆಲ್ಲಾ
ಮುಂದೇ-ನಾ-ಗು-ವುದು
ಮುಂದೇನು
ಮುಕ್ಕಾಲು
ಮುಕ್ಕಾ-ಲು-ಪಾಲು
ಮುಕ್ತ
ಮುಕ್ತಃ
ಮುಕ್ತ-ಜೀವಿ
ಮುಕ್ತ-ಜೀ-ವಿ-ಗ-ಳಲ್ಲ
ಮುಕ್ತ-ಜೀ-ವಿ-ಗಳು
ಮುಕ್ತ-ಜೀ-ವಿ-ಗ-ಳೆಂ-ದರೆ
ಮುಕ್ತನ
ಮುಕ್ತ-ನನ್ನು
ಮುಕ್ತ-ನ-ಲ್ಲಿ-ರುವ
ಮುಕ್ತ-ನಾ-ಗ-ಬೇ-ಕಾ-ಗಿ-ರು-ವುದು
ಮುಕ್ತ-ನಾ-ಗ-ಬೇ-ಕಾ-ದರೆ
ಮುಕ್ತ-ನಾ-ಗ-ಬೇ-ಕಿಲ್ಲ
ಮುಕ್ತ-ನಾಗಿ
ಮುಕ್ತ-ನಾ-ಗಿ-ದ್ದಾನೆ
ಮುಕ್ತ-ನಾ-ಗಿ-ಲ್ಲವೋ
ಮುಕ್ತ-ನಾಗು
ಮುಕ್ತ-ನಾ-ಗು-ತ್ತಾನೆ
ಮುಕ್ತ-ನಾ-ಗು-ತ್ತಾ-ನೆಯೊ
ಮುಕ್ತ-ನಾ-ಗು-ತ್ತೀಯೆ
ಮುಕ್ತ-ನಾ-ಗು-ವನು
ಮುಕ್ತ-ನಾ-ಗು-ವು-ದಕ್ಕೂ
ಮುಕ್ತ-ನಾ-ಗುವೆ
ಮುಕ್ತ-ನಾ-ಗು-ವೆಯೊ
ಮುಕ್ತ-ನಾದ
ಮುಕ್ತ-ನಾ-ದಂ-ತೆಯೇ
ಮುಕ್ತನೂ
ಮುಕ್ತನೆ
ಮುಕ್ತ-ನೆಂ-ದರೆ
ಮುಕ್ತ-ಪು-ರು-ಷರು
ಮುಕ್ತ-ರ-ನ್ನಾಗಿ
ಮುಕ್ತ-ರಾ-ಗಲು
ಮುಕ್ತ-ರಾಗಿ
ಮುಕ್ತ-ರಾ-ಗಿ-ಹೋಗ
ಮುಕ್ತ-ರಾಗು
ಮುಕ್ತ-ರಾ-ಗು-ತ್ತಾರೆ
ಮುಕ್ತ-ರಾ-ಗು-ತ್ತೇವೆ
ಮುಕ್ತ-ರಾ-ಗು-ವಾಗ
ಮುಕ್ತ-ರಾ-ಗು-ವು-ದಕ್ಕೆ
ಮುಕ್ತ-ರಾ-ಗು-ವು-ದಿಲ್ಲ
ಮುಕ್ತ-ರಾ-ಗು-ವುದು
ಮುಕ್ತ-ರಾ-ಗೆವು
ಮುಕ್ತ-ರಾದ
ಮುಕ್ತ-ರಾ-ದರು
ಮುಕ್ತ-ರಾ-ದ-ವರೆಲ್ಲ
ಮುಕ್ತರೂ
ಮುಕ್ತ-ವಾ-ಗಿ-ರುವ
ಮುಕ್ತ-ಸಂಗಃ
ಮುಕ್ತ-ಸಂ-ಗೋ-ಽನ-ಹಂ-ವಾದೀ
ಮುಕ್ತಸ್ಯ
ಮುಕ್ತಾತ್ಮ
ಮುಕ್ತಾ-ತ್ಮರು
ಮುಕ್ತಾ-ವ-ಸ್ಥೆಗೆ
ಮುಕ್ತಾ-ವ-ಸ್ಥೆ-ಯ-ಲ್ಲಿಯೇ
ಮುಕ್ತಿ
ಮುಕ್ತಿ-ಕಾಮಿ
ಮುಕ್ತಿ-ಗಳನ್ನು
ಮುಕ್ತಿ-ಗಾಗಿ
ಮುಕ್ತಿಗೆ
ಮುಕ್ತಿಯ
ಮುಕ್ತಿ-ಯನ್ನು
ಮುಕ್ತಿ-ಯಲ್ಲ
ಮುಕ್ತಿಯೂ
ಮುಕ್ತ್ವಾ
ಮುಖ
ಮುಖಂ
ಮುಖಕ್ಕೆ
ಮುಖ-ಗ-ಳಂತೆ
ಮುಖ-ಗಳನ್ನು
ಮುಖ-ಗಳು
ಮುಖ-ಗಳೇ
ಮುಖದ
ಮುಖ-ದಲ್ಲಿ
ಮುಖ-ದಿಂದ
ಮುಖ-ಬಿಂ-ಬ-ಗಳೇ
ಮುಖ-ವನ್ನು
ಮುಖ-ವಲ್ಲ
ಮುಖ-ವಾಗಿ
ಮುಖ-ವಾಡ
ಮುಖ-ವಾ-ಡ-ಗಳಲ್ಲಿ
ಮುಖ-ವಾ-ಡ-ವನ್ನು
ಮುಖವೋ
ಮುಖಾನಿ
ಮುಖಿಯ
ಮುಖೇ
ಮುಖೇನ
ಮುಖ್ಯ
ಮುಖ್ಯಂ
ಮುಖ್ಯ-ಅ-ವನು
ಮುಖ್ಯ-ಗುರಿ
ಮುಖ್ಯ-ನಾದ
ಮುಖ್ಯ-ರಾದ
ಮುಖ್ಯ-ವಲ್ಲ
ಮುಖ್ಯ-ವಾಗಿ
ಮುಖ್ಯ-ವಾ-ಗಿರು
ಮುಖ್ಯ-ವಾ-ಗಿ-ರು-ವುದನ್ನು
ಮುಖ್ಯ-ವಾ-ಗಿ-ರು-ವುದು
ಮುಖ್ಯ-ವಾ-ಗಿ-ರು-ವುದೇ
ಮುಖ್ಯ-ವಾದ
ಮುಖ್ಯ-ವಾ-ದುದು
ಮುಖ್ಯವೇ
ಮುಖ್ಯವೋ
ಮುಗಿ-ದ-ಮೇಲೆ
ಮುಗಿ-ದರೆ
ಮುಗಿ-ದಾದ
ಮುಗಿ-ಯ-ದಿ-ರುವ
ಮುಗಿ-ಯ-ಲಿಲ್ಲ
ಮುಗಿಯು
ಮುಗಿ-ಯುತು
ಮುಗಿ-ಯುತ್ತ
ಮುಗಿ-ಯುವ
ಮುಗಿ-ಯು-ವುದು
ಮುಗಿಲ
ಮುಗಿ-ಲು-ಗ-ಳಾಚೆ
ಮುಗಿ-ಸ-ಬೇಕು
ಮುಗಿ-ಸಿ-ದ-ಮೇಲೆ
ಮುಗಿ-ಸುತ್ತಾ
ಮುಗಿ-ಸು-ತ್ತಾ-ನೆಯೋ
ಮುಗಿ-ಸು-ವು-ದಕ್ಕೆ
ಮುಗಿ-ಸು-ವುದು
ಮುಗ್ಗ-ರಿ-ಸು-ವೆವು
ಮುಗ್ಧ
ಮುಗ್ಧನೂ
ಮುಚ್ಚದೇ
ಮುಚ್ಚ-ಬೇಕು
ಮುಚ್ಚಲೂ
ಮುಚ್ಚ-ಲ್ಪ-ಟ್ಟಿದೆ
ಮುಚ್ಚ-ಲ್ಪ-ಟ್ಟಿರು
ಮುಚ್ಚಿ
ಮುಚ್ಚಿ-ಕೊಂ-ಡರೆ
ಮುಚ್ಚಿ-ಕೊಂ-ಡಿದೆ
ಮುಚ್ಚಿ-ಕೊಂ-ಡಿ-ರು-ವಂತೆ
ಮುಚ್ಚಿ-ಕೊಂ-ಡಿ-ರು-ವರು
ಮುಚ್ಚಿ-ಕೊಂ-ಡಿ-ರು-ವು-ದ-ರಿಂದ
ಮುಚ್ಚಿ-ಕೊಂಡು
ಮುಚ್ಚಿ-ಕೊ-ಳ್ಳುವ
ಮುಚ್ಚಿ-ಕೊ-ಳ್ಳು-ವು-ದಕ್ಕೆ
ಮುಚ್ಚಿ-ಕೊ-ಳ್ಳು-ವುದು
ಮುಚ್ಚಿ-ಕೊ-ಳ್ಳು-ವುದೋ
ಮುಚ್ಚಿ-ಕೊ-ಳ್ಳು-ವುವು
ಮುಚ್ಚಿ-ಕೊ-ಳ್ಳು-ವೆವು
ಮುಚ್ಚಿ-ಡು-ವುದೇ
ಮುಚ್ಚಿದ
ಮುಚ್ಚಿ-ದರೆ
ಮುಚ್ಚಿದ್ದ
ಮುಚ್ಚಿ-ರುವ
ಮುಚ್ಚಿ-ರು-ವಂತೆ
ಮುಚ್ಚಿ-ರು-ವುದನ್ನು
ಮುಚ್ಚಿ-ರು-ವುದೇ
ಮುಚ್ಚಿ-ರು-ವೆವು
ಮುಚ್ಚಿವೆ
ಮುಚ್ಚಿ-ಹೋ-ಗಿದೆ
ಮುಚ್ಚಿ-ಹೋ-ಗು-ವುದು
ಮುಚ್ಚು
ಮುಚ್ಚು-ತ್ತಾನೆ
ಮುಚ್ಚು-ತ್ತಿದ್ದ
ಮುಚ್ಚು-ಮ-ರೆ-ಯಿಲ್ಲ
ಮುಚ್ಚು-ಮ-ರೆ-ಯಿ-ಲ್ಲದೆ
ಮುಚ್ಚು-ಮ-ರೆಯೂ
ಮುಚ್ಚು-ವಂತೆ
ಮುಚ್ಚು-ವನು
ಮುಚ್ಚು-ವಾ-ಗಲೂ
ಮುಚ್ಚು-ವು-ದಕ್ಕೆ
ಮುಚ್ಚು-ವು-ದಿಲ್ಲ
ಮುಚ್ಚು-ವುದು
ಮುಚ್ಚು-ವುದೊ
ಮುಚ್ಯಂತೇ
ಮುಟ್ಟ
ಮುಟ್ಟ-ಕೂ-ಡದು
ಮುಟ್ಟದ
ಮುಟ್ಟ-ದಂತೆ
ಮುಟ್ಟದೆ
ಮುಟ್ಟದೇ
ಮುಟ್ಟ-ಬ-ಹುದು
ಮುಟ್ಟ-ಬೇ-ಕಾದ
ಮುಟ್ಟ-ಬೇ-ಕಾ-ದರೆ
ಮುಟ್ಟ-ಬೇಕು
ಮುಟ್ಟ-ಲಿಲ್ಲ
ಮುಟ್ಟಲು
ಮುಟ್ಟ-ವುದು
ಮುಟ್ಟಿ
ಮುಟ್ಟಿತು
ಮುಟ್ಟಿದ
ಮುಟ್ಟಿ-ದಂ-ತಹ
ಮುಟ್ಟಿ-ದಂ-ತಾ-ಗು-ವುದು
ಮುಟ್ಟಿ-ದರೆ
ಮುಟ್ಟಿ-ದ-ವನು
ಮುಟ್ಟಿ-ದ-ವರು
ಮುಟ್ಟಿ-ದಾಗ
ಮುಟ್ಟಿ-ದಾ-ಗಲೆ
ಮುಟ್ಟಿದೆ
ಮುಟ್ಟಿ-ದೆವೆ
ಮುಟ್ಟಿ-ದೊ-ಡನೆ
ಮುಟ್ಟಿದ್ದು
ಮುಟ್ಟಿ-ಬ-ರುವ
ಮುಟ್ಟಿ-ರ-ಬಾ-ರದು
ಮುಟ್ಟಿ-ರ-ಬೇಕು
ಮುಟ್ಟಿ-ರುವ
ಮುಟ್ಟಿ-ರು-ವನು
ಮುಟ್ಟಿ-ರು-ವರು
ಮುಟ್ಟಿಲ್ಲ
ಮುಟ್ಟಿ-ಸ-ಬ-ಹುದು
ಮುಟ್ಟಿ-ಸು-ವನು
ಮುಟ್ಟು
ಮುಟ್ಟು-ತ್ತದೆ
ಮುಟ್ಟು-ತ್ತಾನೆ
ಮುಟ್ಟು-ತ್ತಾ-ನೆಯೋ
ಮುಟ್ಟು-ತ್ತಾರೆ
ಮುಟ್ಟು-ತ್ತಾರೊ
ಮುಟ್ಟು-ತ್ತಾರೋ
ಮುಟ್ಟು-ತ್ತಿ-ರು-ವರು
ಮುಟ್ಟು-ತ್ತೇ-ನಲ್ಲ
ಮುಟ್ಟು-ತ್ತೇ-ವೆಯೊ
ಮುಟ್ಟುವ
ಮುಟ್ಟು-ವನು
ಮುಟ್ಟು-ವರು
ಮುಟ್ಟು-ವ-ವ-ರೆಗೆ
ಮುಟ್ಟು-ವಾಗ
ಮುಟ್ಟು-ವಾ-ಗಲೂ
ಮುಟ್ಟು-ವು-ದ-ಕ್ಕಾ-ಗು-ವು-ದಿಲ್ಲ
ಮುಟ್ಟು-ವು-ದಕ್ಕೆ
ಮುಟ್ಟು-ವು-ದಿಲ್ಲ
ಮುಟ್ಟು-ವುದು
ಮುಟ್ಟು-ವುದೋ
ಮುಟ್ಟು-ವೆವು
ಮುಡು-ಪಾಗಿ
ಮುಡು-ಪಾ-ಗಿ-ಟ್ಟರೆ
ಮುಡು-ಪಾ-ಗಿ-ರ-ಬೇಕು
ಮುತು-ವ-ರ್ಜಿ-ಯಿಂದ
ಮುತ್ತನ್ನು
ಮುತ್ತ-ಲ್ಪಟ್ಟ
ಮುತ್ತಿ
ಮುತ್ತಿ-ದ್ದರೆ
ಮುತ್ತಿ-ರುವ
ಮುತ್ತಿ-ರು-ವುವು
ಮುತ್ತಿವೆ
ಮುತ್ತು
ಮುತ್ತು-ತ್ತವೆ
ಮುತ್ತು-ರ-ತ್ನ-ಗ-ಳೆಲ್ಲ
ಮುತ್ತುವ
ಮುತ್ತು-ವು-ದಕ್ಕೆ
ಮುತ್ತು-ವುವು
ಮುದಾ
ಮುದಿತ
ಮುದುಕ
ಮುದು-ಕ-ನ-ವ-ರೆಗೆ
ಮುದ್ದಾ-ಡು-ವಾಗ
ಮುದ್ದಿ-ಗಾಗಿ
ಮುದ್ದಿಸಿ
ಮುದ್ದಿ-ಸಿ-ದರೆ
ಮುದ್ದಿ-ಸು-ತ್ತಿ-ದ್ದೆವು
ಮುದ್ದಿ-ಸು-ವುದು
ಮುದ್ದಿ-ಸು-ವೆವು
ಮುದ್ದೆ
ಮುದ್ರ-ಣಾ-ಲ-ಯ-ಗಳಲ್ಲಿ
ಮುದ್ರೆ
ಮುದ್ರೆ-ಯನ್ನು
ಮುನಯಃ
ಮುನಿ
ಮುನಿಃ
ಮುನಿ-ಗಳಲ್ಲಿ
ಮುನಿ-ಗ-ಳಾದ
ಮುನಿ-ಗಳು
ಮುನಿಗೆ
ಮುನಿಯ
ಮುನಿ-ಯಾ-ದರೂ
ಮುನಿ-ರ್ಬ್ರಹ್ಮ
ಮುನಿ-ಸಿ-ನಿಂದ
ಮುನೀ-ನಾ-ಮ-ಪ್ಯಹಂ
ಮುನೇಃ
ಮುನ್ನ
ಮುಪ್ಪು
ಮುಪ್ಪು-ಸಾ-ವು-ಗಳಿಂದ
ಮುಮು-ಕ್ಷು-ಗಳು
ಮುಮು-ಕ್ಷು-ಭಿಃ
ಮುರಿದ
ಮುರಿ-ದಿ-ರು-ವುದು
ಮುರಿದು
ಮುಲಾ-ಮನ್ನು
ಮುಲಾಮು
ಮುಳ-ಗ-ಬೇಕು
ಮುಳ-ಗಿ-ಸಲು
ಮುಳ-ಗು-ವು-ದಕ್ಕೆ
ಮುಳು-ಗದೆ
ಮುಳು-ಗ-ಬ-ಹುದು
ಮುಳು-ಗ-ಬೇಕು
ಮುಳುಗಿ
ಮುಳು-ಗಿ-ದಂತೆ
ಮುಳು-ಗಿ-ದರೆ
ಮುಳು-ಗಿ-ದ-ವರು
ಮುಳು-ಗಿ-ದ್ದರೂ
ಮುಳು-ಗಿ-ರ-ಬೇ-ಕಾ-ಗಿದೆ
ಮುಳು-ಗಿ-ರ-ಬೇಕು
ಮುಳು-ಗಿ-ರು-ತ್ತದೆ
ಮುಳು-ಗಿ-ರುವ
ಮುಳು-ಗಿ-ರು-ವನು
ಮುಳು-ಗಿ-ರು-ವರು
ಮುಳು-ಗಿ-ರು-ವ-ವ-ರಲ್ಲ
ಮುಳು-ಗಿ-ರು-ವ-ವರು
ಮುಳು-ಗಿ-ರು-ವಾಗ
ಮುಳು-ಗಿ-ರು-ವಾ-ಗಲೂ
ಮುಳು-ಗಿ-ರು-ವುದನ್ನು
ಮುಳು-ಗಿ-ರು-ವುದು
ಮುಳು-ಗಿ-ಸದೆ
ಮುಳು-ಗಿ-ಸ-ಬೇಕು
ಮುಳು-ಗಿ-ಸ-ಲಾ-ರದು
ಮುಳು-ಗಿ-ಸಲು
ಮುಳು-ಗಿ-ಸಿ-ದರೆ
ಮುಳು-ಗಿ-ಸಿ-ಬಿ-ಡು-ವುದು
ಮುಳು-ಗಿ-ಸಿ-ರು-ವುದು
ಮುಳು-ಗಿ-ಸು-ವನು
ಮುಳು-ಗಿ-ಸು-ವು-ದಕ್ಕೆ
ಮುಳು-ಗಿ-ಸು-ವುದು
ಮುಳು-ಗಿ-ಸು-ವುವು
ಮುಳು-ಗಿ-ಹೋ-ಗಿದೆ
ಮುಳು-ಗಿ-ಹೋ-ಗಿ-ದ್ದಾನೆ
ಮುಳು-ಗಿ-ಹೋ-ಗಿರು
ಮುಳು-ಗಿ-ಹೋ-ಗು-ವುದು
ಮುಳು-ಗಿ-ಹೋದ
ಮುಳು-ಗು-ತ್ತಲೂ
ಮುಳು-ಗು-ತ್ತಾನೆ
ಮುಳು-ಗು-ತ್ತಿದೆ
ಮುಳು-ಗು-ತ್ತಿ-ದ್ದೇ-ನಲ್ಲಾ
ಮುಳು-ಗು-ತ್ತಿ-ರು-ವಾಗ
ಮುಳು-ಗು-ತ್ತೇವೆ
ಮುಳು-ಗುವ
ಮುಳು-ಗು-ವು-ದಿಲ್ಲ
ಮುಳು-ಗು-ವುದು
ಮುಳು-ಗು-ವುದೂ
ಮುಳ್ಳನ್ನು
ಮುಳ್ಳಿನ
ಮುಳ್ಳಿ-ನಿಂದ
ಮುಳ್ಳಿ-ನಿಂ-ದಲೂ
ಮುಳ್ಳು
ಮುಳ್ಳು-ಗಳಿಂದ
ಮುಳ್ಳು-ಗಳು
ಮುಷ್ಟಾ
ಮುಷ್ಟಿ
ಮುಷ್ಟಿಗೆ
ಮುಷ್ಟಿ-ಯಲ್ಲಿ
ಮುಷ್ಟಿ-ಯುದ್ಧ
ಮುಸಲ
ಮುಸ-ಲ-ಧಾರೆ
ಮುಸಾ-ಫಿ-ರ-ಖಾ-ನೆ-ಯಂತೆ
ಮುಸು-ಕಿ-ರುವ
ಮುಹು-ರ್ಮು-ಹುಃ
ಮುಹ್ಯಂತಿ
ಮುಹ್ಯತಿ
ಮೂಕ
ಮೂಕಂ
ಮೂಕ-ನನ್ನು
ಮೂಕ-ರಾ-ಗು-ತ್ತೇವೆ
ಮೂಕ-ವಾ-ಗಿ-ಹೋ-ಗು-ವುದು
ಮೂಕ-ವಾ-ಗು-ವುದು
ಮೂಗಿಗೆ
ಮೂಗಿದೆ
ಮೂಗಿನ
ಮೂಗಿ-ನ-ಲ್ಲಿ-ರುವ
ಮೂಗು
ಮೂಗು-ದಾ-ರ-ಗಳು
ಮೂಟೆ
ಮೂಡ-ಬ-ಹುದು
ಮೂಡ-ಬೇಕು
ಮೂಡ-ಲಂ-ಚಿ-ನಲ್ಲಿ
ಮೂಡಲು
ಮೂಡಿ
ಮೂಡಿ-ಬಂ-ದಿದೆ
ಮೂಡಿವೆ
ಮೂಡುವ
ಮೂಡು-ವುದು
ಮೂಢ
ಮೂಢ-ಗ್ರಾ-ಹೇ-ಣಾ-ತ್ಮನೋ
ಮೂಢ-ನಂ-ಬಿ-ಕೆಗೆ
ಮೂಢ-ನಾ-ಗಿ-ರು-ವು-ದಿಲ್ಲ
ಮೂಢ-ನಾ-ಗು-ವು-ದಿಲ್ಲ
ಮೂಢ-ನಾ-ದ-ವನು
ಮೂಢ-ಬು-ದ್ಧಿ-ಯಿಂದ
ಮೂಢ-ಯೋ-ನಿಷು
ಮೂಢರ
ಮೂಢ-ರಾ-ಗು-ತ್ತೇವೆ
ಮೂಢರು
ಮೂಢ-ವಾದ
ಮೂಢಾ
ಮೂಢಾಃ
ಮೂಢೋಽಯಂ
ಮೂರನೆ
ಮೂರ-ನೆಯ
ಮೂರ-ನೆ-ಯದು
ಮೂರ-ನೆ-ಯದೆ
ಮೂರ-ನೆ-ಯದೇ
ಮೂರ-ನೆ-ಯ-ವ-ನಿಗೆ
ಮೂರ-ನೆ-ಯ-ವನು
ಮೂರ-ನೆ-ಯ-ವನೆ
ಮೂರ-ನೆ-ಯ-ವರು
ಮೂರ-ನೆ-ಯ-ವರೆ
ಮೂರನೇ
ಮೂರನ್ನು
ಮೂರನ್ನೂ
ಮೂರರ
ಮೂರಷ್ಟು
ಮೂರು
ಮೂರೂ
ಮೂರೇ
ಮೂರ್ಛೆ
ಮೂರ್ತಯಃ
ಮೂರ್ತಿ-ಯನ್ನು
ಮೂರ್ತಿ-ಯಾ-ಗಿ-ರು-ವನು
ಮೂರ್ತಿ-ವೆ-ತ್ತಂತೆ
ಮೂರ್ಧ್ನ-್ಯಾ-ಧಾ-ಯಾ-ತ್ಮನಃ
ಮೂಲ
ಮೂಲಕ
ಮೂಲ-ಕ-ವಾಗಿ
ಮೂಲ-ಕ-ವಾ-ಗಿಯೂ
ಮೂಲ-ಕ-ವಾ-ಗಿಯೇ
ಮೂಲ-ಕ-ವಾ-ಗಿಯೋ
ಮೂಲ-ಕವೂ
ಮೂಲ-ಕವೆ
ಮೂಲ-ಕವೇ
ಮೂಲಕ್ಕೆ
ಮೂಲ-ಗ್ರಂಥ
ಮೂಲ-ಚೈ-ತ-ನ್ಯಕ್ಕೆ
ಮೂಲ-ಚೈ-ತ-ನ್ಯದ
ಮೂಲದ
ಮೂಲ-ದಲ್ಲಿ
ಮೂಲ-ದಿಂದ
ಮೂಲ-ದಿಂ-ದಲೇ
ಮೂಲ-ರೂ-ಪಕ್ಕೆ
ಮೂಲ-ವಸ್ತು
ಮೂಲ-ವ-ಸ್ತು-ವಿ-ನಿಂ-ದಲೇ
ಮೂಲ-ವ-ಸ್ತುವೆ
ಮೂಲ-ವಾದ
ಮೂಲವೂ
ಮೂಲವೋ
ಮೂಲಾ-ಧಾರ
ಮೂಲಾ-ನ್ಯ-ನು-ಸಂ-ತ-ತಾನಿ
ಮೂಲೆ
ಮೂಲೆ-ಮೂ-ಲೆ-ಯಲ್ಲಿ
ಮೂಲೆಯ
ಮೂಲೆ-ಯಲ್ಲಿ
ಮೂಲೆ-ಯ-ಲ್ಲಿದೆ
ಮೂಲೆ-ಯಿಂದ
ಮೂಳನ್ನು
ಮೂಳು
ಮೂಳೆ
ಮೂಳೆ-ಗ-ಳ-ಲ್ಲದೆ
ಮೂಳೆ-ಗಳು
ಮೂಳೆಯ
ಮೂಳೆ-ಯನ್ನೆ
ಮೂಸಿ
ಮೂಸಿದ್ದು
ಮೂಸಿ-ನೋ-ಡು-ವುದು
ಮೂಸು-ತ್ತಾನೆ
ಮೂಸು-ವಂತೆ
ಮೂಸು-ವಾಗ
ಮೂಸು-ವಾ-ಗಲೂ
ಮೂಸು-ವುದು
ಮೂಸೆ-ಯಲ್ಲಿ
ಮೂಸೆ-ಯ-ಲ್ಲಿಟ್ಟು
ಮೃಗ
ಮೃಗಕ್ಕೆ
ಮೃಗ-ಗ-ಳಿ-ಗಿಂತ
ಮೃಗದ
ಮೃಗ-ದಂತೆ
ಮೃಗ-ರಾಜ
ಮೃಗ-ವನ್ನು
ಮೃಗ-ವಾಗಿ
ಮೃಗ-ಸ-ದೃಶ
ಮೃಗ-ಸ-ದೃ-ಶ-ರಾ-ಗು-ವೆವು
ಮೃಗ-ಸ-ದೃ-ಶರು
ಮೃಗಾ-ಣಾಂ
ಮೃಗಾ-ಲ-ಯ-ದ-ಲ್ಲಿ-ರುವ
ಮೃಗೀಯ
ಮೃಗೇಂ-ದ್ರೋಽಹಂ
ಮೃತಮ್
ಮೃತಸ್ಯ
ಮೃತ್ಯು
ಮೃತ್ಯುಂ
ಮೃತ್ಯುಃ
ಮೃತ್ಯು-ಮ-ಯ-ವಾದ
ಮೃತ್ಯು-ಮು-ಖ-ದಲ್ಲಿ
ಮೃತ್ಯು-ರ್ಧ್ರುವಂ
ಮೃತ್ಯು-ವನ್ನು
ಮೃತ್ಯು-ವನ್ನೇ
ಮೃತ್ಯು-ವಾದ
ಮೃತ್ಯು-ವಾ-ದರೆ
ಮೃತ್ಯು-ವಿಗೆ
ಮೃತ್ಯು-ವಿನ
ಮೃತ್ಯು-ವಿ-ನಷ್ಟು
ಮೃತ್ಯುವೂ
ಮೃತ್ಯುಶ್ಚ
ಮೃತ್ಯು-ಸಂ-ಸಾರ
ಮೃತ್ಯು-ಸಂ-ಸಾ-ರ-ವ-ರ್ತ್ಮನಿ
ಮೃತ್ಯು-ಸಂ-ಸಾ-ರ-ಸಾ-ಗ-ರಾತ್
ಮೃದಂಗ
ಮೃದು
ಮೃದು-ತ್ವ-ವನ್ನು
ಮೃದು-ವಾ-ಗಿ-ದೆಯೆ
ಮೃದು-ವಾ-ಗಿ-ರ-ಬೇಕು
ಮೃದು-ವಾ-ಗಿ-ರು-ವುದು
ಮೃದು-ವಾ-ಗು-ವು-ದಕ್ಕೆ
ಮೃದು-ವಾ-ಗು-ವುದು
ಮೃದು-ಸ್ವ-ಭಾವ
ಮೃದೂನಿ
ಮೃಷ್ಟಾನ್ನ
ಮೃಷ್ಟಾ-ನ್ನ-ವನ್ನು
ಮೆಗ-ಸ್ಥ-ನೀಸ್
ಮೆಚ್ಚ-ಬೇ-ಕಾ-ದರೆ
ಮೆಚ್ಚ-ಬೇಕು
ಮೆಚ್ಚಿ
ಮೆಚ್ಚಿನ
ಮೆಚ್ಚಿಸು
ಮೆಚ್ಚಿ-ಸು-ವು-ದಕ್ಕೆ
ಮೆಚ್ಚುಗೆ
ಮೆಚ್ಚು-ಗೆ-ಯಾ-ದುದು
ಮೆಚ್ಚು-ತ್ತಾನೆ
ಮೆಚ್ಚು-ತ್ತಾರೆ
ಮೆಚ್ಚು-ತ್ತೇವೆ
ಮೆಚ್ಚು-ವನು
ಮೆಚ್ಚು-ವ-ನು-ದೇ-ವರೇ
ಮೆಚ್ಚು-ವರು
ಮೆಚ್ಚು-ವು-ದಕ್ಕೆ
ಮೆಚ್ಚು-ವುದು
ಮೆಟ್ಚ-ಲನ್ನು
ಮೆಟ್ಟ-ಲನ್ನು
ಮೆಟ್ಟ-ಲನ್ನೇ
ಮೆಟ್ಟ-ಲಲ್ಲಿ
ಮೆಟ್ಟ-ಲಾಗಿ
ಮೆಟ್ಟಲಿ
ಮೆಟ್ಟ-ಲಿನ
ಮೆಟ್ಟ-ಲಿ-ನಲ್ಲಿ
ಮೆಟ್ಟ-ಲಿ-ನಿಂದ
ಮೆಟ್ಟಲು
ಮೆಟ್ಟ-ಲು-ಗಳನ್ನು
ಮೆಟ್ಟ-ಲು-ಗಳು
ಮೆಟ್ಟಲೂ
ಮೆಟ್ಟಲೇ
ಮೆಟ್ಟಿ
ಮೆಟ್ಟಿ-ಕೊಂಡ
ಮೆಟ್ಟಿ-ಕೊಂ-ಡರೆ
ಮೆಟ್ಟಿ-ಕೊಂ-ಡಿತು
ಮೆಟ್ಟಿ-ಕೊಂ-ಡಿದೆ
ಮೆಟ್ಟಿ-ಕೊಂ-ಡಿ-ದ್ದರೆ
ಮೆಟ್ಟಿ-ಕೊಂ-ಡಿ-ರು-ವು-ದ-ರಿಂ-ದಲೇ
ಮೆಟ್ಟಿ-ಕೊಂ-ಡಿ-ರು-ವುದು
ಮೆಟ್ಟಿ-ಕೊಳ್ಳು
ಮೆಟ್ಟಿ-ಕೊ-ಳ್ಳು-ವು-ದೆಂದು
ಮೆಟ್ಟಿತು
ಮೆಟ್ಟಿದೆ
ಮೆಟ್ಟಿ-ರು-ವಾಗ
ಮೆಟ್ಟಿ-ಲನ್ನು
ಮೆಟ್ಟಿ-ಲ-ಲ್ಲಿ-ರು-ವನು
ಮೆಟ್ಟಿ-ಲಿಗೆ
ಮೆಟ್ಟಿ-ಲಿ-ನಲ್ಲಿ
ಮೆಟ್ಟಿಲು
ಮೆಟ್ಟಿ-ಲು-ಗಳ
ಮೆಟ್ಟಿ-ಲು-ಗಳನ್ನು
ಮೆಟ್ಟು-ವುವು
ಮೆಡ-ಲು-ಗಳನ್ನೆಲ್ಲ
ಮೆಡಿ-ಕಲ್
ಮೆಣ-ಸಿನ
ಮೆತ್ತಿ-ಕೊಂ-ಡಿತು
ಮೆತ್ತಿ-ಕೊಂ-ಡಿ-ರುವ
ಮೆತ್ತಿ-ಕೊ-ಳ್ಳು-ವುದು
ಮೆತ್ತಿದ್ದ
ಮೆದು-ಳಿನ
ಮೆದು-ಳಿ-ನಲ್ಲಿ
ಮೆದುಳು
ಮೆರ-ವ-ಣಿಗೆ
ಮೆರ-ವ-ಣಿ-ಗೆಗೆ
ಮೆರೆ-ದರು
ಮೆರೆ-ದ-ವರು
ಮೆರೆ-ದಾಡು
ಮೆರೆ-ಯ-ಬಾ-ರದು
ಮೆರೆ-ಯಲೂ
ಮೆರೆಯು
ಮೆರೆ-ಯು-ತ್ತಾನೆ
ಮೆರೆ-ಯು-ತ್ತಿತ್ತು
ಮೆರೆ-ಯು-ತ್ತಿ-ರ-ಬಾ-ರದು
ಮೆರೆ-ಯು-ತ್ತಿ-ರು-ವರು
ಮೆರೆ-ಯು-ವನು
ಮೆರೆ-ಯು-ವರು
ಮೆರೆ-ಯು-ವು-ದಿಲ್ಲ
ಮೆರೆ-ಯು-ವುದು
ಮೆರೆ-ಯು-ವುದೂ
ಮೆರೆ-ಯು-ವೆನು
ಮೆರೆ-ಸ-ಬೇಕು
ಮೆರೆ-ಸಲು
ಮೆರೆ-ಸಿ-ಕೊ-ಳ್ಳು-ವು-ದಲ್ಲ
ಮೆರೆ-ಸು-ತ್ತಿದೆ
ಮೆರೆ-ಸು-ತ್ತಿ-ರು-ವನು
ಮೆರೆ-ಸು-ವು-ದಿಲ್ಲ
ಮೆರೆ-ಸು-ವುದು
ಮೆಲಕು
ಮೆಲ-ಕು-ತ್ತಿ-ದ್ದರೆ
ಮೆಲ-ಕು-ತ್ತಿ-ರು-ವುದು
ಮೆಲ-ಕು-ಹಾ-ಕು-ತ್ತಿ-ರು-ವೆವು
ಮೆಲ-ಕು-ಹಾ-ಕು-ವು-ದಿ-ಲ್ಲವೊ
ಮೆಲುಕು
ಮೆಲು-ಕುತ್ತಾ
ಮೆಲು-ಕು-ತ್ತಿ-ರು-ವುದು
ಮೆಲ್ಲ-ಬ-ಹುದು
ಮೆಲ್ಲ-ಮೆ-ಲ್ಲಗೆ
ಮೆಲ್ಲ-ಮೆ-ಲ್ಲನೆ
ಮೆಲ್ಲು
ಮೆಲ್ಲು-ತ್ತಿ-ರ-ಬೇಕು
ಮೆಲ್ಲು-ತ್ತಿ-ರು-ವನು
ಮೆಲ್ಲು-ವನು
ಮೆಲ್ಲು-ವ-ವನು
ಮೇ
ಮೇಕೆ-ಯನ್ನು
ಮೇಘ-ದಂತೆ
ಮೇಚ್ಯುತ
ಮೇಜು
ಮೇಧಸ್ಸು
ಮೇಧಾ
ಮೇಧಾವಿ
ಮೇಧಾ-ವಿ-ಗಳು
ಮೇಧಾ-ವಿಯ
ಮೇಧಾ-ವಿ-ಯಾ-ದ-ವ-ನಿಗೆ
ಮೇಧಾ-ವಿಯೂ
ಮೇಧಾವೀ
ಮೇಯ-ಬ-ಹುದು
ಮೇಯಲು
ಮೇಯಿಸು
ಮೇಯು-ತ್ತ-ವೆಯೊ
ಮೇಯು-ತ್ತಿದ್ದ
ಮೇಯು-ತ್ತಿ-ರುವ
ಮೇಯು-ತ್ತಿ-ರು-ವು-ದಕ್ಕೆ
ಮೇಯು-ವು-ದಕ್ಕೆ
ಮೇಯು-ವುದು
ಮೇರು
ಮೇರುಃ
ಮೇರೆ-ಗ-ಳೇನು
ಮೇರೆ-ಯನ್ನು
ಮೇರೆ-ಯ-ಲ್ಲಿಯೇ
ಮೇಲಕ್ಕೂ
ಮೇಲಕ್ಕೆ
ಮೇಲ-ಕ್ಕೆ-ತ್ತು-ತ್ತಿ-ದ್ದರೆ
ಮೇಲ-ಕ್ಕೇ-ಳು-ವುದು
ಮೇಲಲ್ಲ
ಮೇಲ-ಲ್ಲವೆ
ಮೇಲಾ-ಗ-ಬ-ಹುದು
ಮೇಲಾ-ಗಲಿ
ಮೇಲಾ-ಗಿ-ರು-ವುದೊ
ಮೇಲಾ-ಗು-ತ್ತಿತ್ತು
ಮೇಲಾ-ಗು-ವನು
ಮೇಲಾ-ಗು-ವುದು
ಮೇಲಾ-ದರೂ
ಮೇಲಾ-ದ-ವನು
ಮೇಲಿ
ಮೇಲಿಂದ
ಮೇಲಿಟ್ಟ
ಮೇಲಿಟ್ಟು
ಮೇಲಿದೆ
ಮೇಲಿ-ದ್ದರೆ
ಮೇಲಿ-ದ್ದಾ-ಗಲೆ
ಮೇಲಿನ
ಮೇಲಿ-ನದು
ಮೇಲಿ-ನ-ಮ-ಟ್ಟ-ದಲ್ಲಿ
ಮೇಲಿ-ನ-ವರು
ಮೇಲಿ-ನಿಂದ
ಮೇಲಿ-ರ-ಬ-ಹುದು
ಮೇಲಿ-ರಲಿ
ಮೇಲಿರು
ಮೇಲಿ-ರುವ
ಮೇಲಿ-ರು-ವನು
ಮೇಲಿ-ರು-ವ-ವ-ರನ್ನು
ಮೇಲಿ-ರು-ವ-ವ-ರೊ-ಡನೆ
ಮೇಲಿ-ರು-ವಾಗ
ಮೇಲಿ-ರುವು
ಮೇಲಿ-ರು-ವುದನ್ನು
ಮೇಲಿ-ರು-ವು-ದ-ನ್ನೆಲ್ಲ
ಮೇಲಿ-ರು-ವುದು
ಮೇಲಿ-ರು-ವುದೇ
ಮೇಲಿ-ರು-ವುದೋ
ಮೇಲಿ-ರು-ವುವು
ಮೇಲಿಲ್ಲ
ಮೇಲಿವೆ
ಮೇಲು
ಮೇಲು-ಗಡೆ
ಮೇಲು-ಭಾ-ಗಕ್ಕೆ
ಮೇಲು-ಮೇ-ಲಕ್ಕೆ
ಮೇಲೂ
ಮೇಲೆ
ಮೇಲೆಂದು
ಮೇಲೆಕ್ಕೆ
ಮೇಲೆ-ಣಿ-ಸು-ವು-ದಿ-ಲ್ಲವೊ
ಮೇಲೆ-ತ್ತಲು
ಮೇಲೆತ್ತಿ
ಮೇಲೆ-ತ್ತಿ-ಬಿ-ಡು-ವುದು
ಮೇಲೆ-ತ್ತು-ತ್ತಾನೆ
ಮೇಲೆ-ತ್ತು-ವನು
ಮೇಲೆ-ತ್ತು-ವುದು
ಮೇಲೆದ್ದ
ಮೇಲೆ-ದ್ದರೆ
ಮೇಲೆ-ದ್ದಿ-ರು-ವನು
ಮೇಲೆದ್ದು
ಮೇಲೆನ್ನು
ಮೇಲೆ-ಬ್ಬಿ-ಸಲು
ಮೇಲೆ-ಮೇಲೆ
ಮೇಲೆಯೂ
ಮೇಲೆಯೆ
ಮೇಲೆಯೇ
ಮೇಲೆಲ್ಲ
ಮೇಲೇನೊ
ಮೇಲೇ-ರು-ವು-ದಕ್ಕೆ
ಮೇಲೇ-ರು-ವುದು
ಮೇಲೇ-ಳ-ಬೇ-ಕಾ-ದರೆ
ಮೇಲೇ-ಳ-ಬೇಕು
ಮೇಲೇ-ಳ-ಲಾ-ರರು
ಮೇಲೇ-ಳಲು
ಮೇಲೇ-ಳು-ತ್ತದೆ
ಮೇಲೇ-ಳು-ತ್ತವೆ
ಮೇಲೇ-ಳು-ತ್ತವೋ
ಮೇಲೇ-ಳು-ತ್ತಾನೆ
ಮೇಲೇ-ಳು-ವನು
ಮೇಲೇ-ಳು-ವುದು
ಮೇಲೇ-ಳು-ವುದೇ
ಮೇಲೊ
ಮೇಲೊಂ-ದ-ರಂತೆ
ಮೇಲೊಂದು
ಮೇಲೋ
ಮೇಲೋ-ಗರ
ಮೇಲ್ಪಂ-ಕ್ತಿ-ಯನ್ನು
ಮೇಲ್ಪಂ-ಕ್ತಿ-ಯಾ-ಗ-ಬೇಕು
ಮೇಲ್ವಿ-ಚಾ-ರಣೆ
ಮೇಳ-ಗಳು
ಮೇಳ-ಗ-ಳೊ-ಡನೆ
ಮೇವಿನ
ಮೇಽಮೃ-ತಮ್
ಮೈ
ಮೈಕ್ರಾ-ಸ್ಕೋ-ಪಿನ
ಮೈತ್ರಃ
ಮೈತ್ರಿ
ಮೈತ್ರೀ
ಮೈದಾ-ನ-ದಲ್ಲಿ
ಮೈದೋ-ರಿ-ರು-ವನು
ಮೈಬಗ್ಗಿ
ಮೈಮ-ರೆತ
ಮೈಮ-ರೆ-ಯು-ವ-ನಲ್ಲ
ಮೈಯೆಲ್ಲ
ಮೈಲಿ
ಮೈಲಿ-ಗಳ
ಮೈಲಿ-ಗ-ಳಷ್ಟು
ಮೈಲಿ-ಗಳಿಂದ
ಮೈಲಿಗೆ
ಮೈಲಿ-ಗೆಯ
ಮೈಲಿ-ಗೆ-ಯಾ-ಗುವ
ಮೈಲಿ-ಗೆಯೆ
ಮೈಸೂ-ರಿ-ನಲ್ಲಿ
ಮೈಸೂ-ರಿ-ನಿಂದ
ಮೈಸೂರು
ಮೊಂಡ-ತನ
ಮೊಂಡ-ನಾಗಿ
ಮೊಂಡಾಟ
ಮೊಂಡು
ಮೊಂಡು-ತ-ನ-ದಿಂದ
ಮೊಗ-ಚುವ
ಮೊಗ-ಚೆ-ಯ-ಕಾ-ಯಿ-ಯನ್ನು
ಮೊಗ-ದಾ-ವರೆ
ಮೊಟ್ಟೆ
ಮೊಟ್ಟೆ-ಗ-ಳಂತೆ
ಮೊಟ್ಟೆ-ಗಳನ್ನೆಲ್ಲ
ಮೊಟ್ಟೆ-ಗ-ಳ-ನ್ನೊ-ಡೆದು
ಮೊತ್ತ
ಮೊತ್ತ-ವಾದ
ಮೊದ
ಮೊದ-ಮೊ-ದ-ಲಲ್ಲಿ
ಮೊದ-ಮೊ-ದಲು
ಮೊದಲ
ಮೊದ-ಲನೆ
ಮೊದ-ಲ-ನೆಯ
ಮೊದ-ಲ-ನೆ-ಯ-ದನ್ನು
ಮೊದ-ಲ-ನೆ-ಯ-ದ-ರಷ್ಟೆ
ಮೊದ-ಲ-ನೆ-ಯ-ದಾಗಿ
ಮೊದ-ಲ-ನೆ-ಯದು
ಮೊದ-ಲ-ನೆ-ಯದೆ
ಮೊದ-ಲ-ನೆ-ಯದೇ
ಮೊದ-ಲ-ನೆ-ಯ-ವನು
ಮೊದ-ಲ-ನೆ-ಯ-ವನೆ
ಮೊದ-ಲ-ನೆ-ಯ-ವನೇ
ಮೊದ-ಲನೇ
ಮೊದ-ಲಲ್ಲಿ
ಮೊದ-ಲ-ಲ್ಲಿಯೇ
ಮೊದ-ಲಾ-ಗ-ಬೇ-ಕಾ-ದರೆ
ಮೊದ-ಲಾಗು
ಮೊದ-ಲಾ-ಗು-ವುದು
ಮೊದ-ಲಾದ
ಮೊದ-ಲಾ-ದ-ವು-ಗ-ಳೆಲ್ಲ
ಮೊದ-ಲಾ-ದುದು
ಮೊದ-ಲಾ-ದುವು
ಮೊದ-ಲಾ-ದು-ವು-ಗಳನ್ನು
ಮೊದ-ಲಾ-ದು-ವು-ಗಳಲ್ಲಿ
ಮೊದ-ಲಾ-ದು-ವು-ಗಳಿಂದ
ಮೊದಲಿ
ಮೊದ-ಲಿ-ಗ-ನಲ್ಲ
ಮೊದ-ಲಿ-ಗ-ರಿ-ರ-ಬ-ಹುದು
ಮೊದ-ಲಿಡು
ಮೊದ-ಲಿ-ನಂತೆ
ಮೊದ-ಲಿ-ನದು
ಮೊದ-ಲಿ-ನಿಂದ
ಮೊದ-ಲಿ-ನಿಂ-ದಲೂ
ಮೊದ-ಲಿ-ರುವ
ಮೊದಲು
ಮೊದ-ಲು-ಮಾ-ಡಿ-ದರೆ
ಮೊದಲೂ
ಮೊದಲೆ
ಮೊದಲೇ
ಮೊದ-ಲ್ಗೊಂಡು
ಮೊನ-ಚಾ-ದರೆ
ಮೊಬ್ಬಾಗಿ
ಮೊಬ್ಬಾ-ಗಿಲ್ಲ
ಮೊಬ್ಬು
ಮೊಬ್ಬು-ಮೊ-ಬ್ಬಾಗಿ
ಮೊಮ್ಮ-ಕ್ಕಳು
ಮೊಮ್ಮ-ಗ-ನಾದ
ಮೊರೆ
ಮೊರೆ-ಯಿ-ಡ-ಬೇಕು
ಮೊರೆ-ಯು-ತ್ತಿ-ರುವ
ಮೊರೆ-ಯು-ವುದು
ಮೊರೆ-ಹೊ-ಕ್ಕರೆ
ಮೊಲೆ-ಯನ್ನು
ಮೊಳಕೆ
ಮೊಳ-ಕೆಯೇ
ಮೊಳ-ಗಿ-ದುವು
ಮೊಳ-ಗು-ತ್ತಿ-ರುವ
ಮೊಳ-ಗುವ
ಮೊಳೆ
ಮೊಳೆಯ
ಮೊಳೆ-ಯನ್ನು
ಮೊಳೆ-ಯ-ಬ-ಲ್ಲದು
ಮೊಳೆ-ಯ-ಬೇ-ಕಾ-ದರೆ
ಮೊಳೆ-ಯ-ಲಾ-ರದು
ಮೊಳೆ-ಯು-ತ್ತಿತ್ತು
ಮೊಳೆ-ಯು-ವು-ದ-ರಲ್ಲಿ
ಮೊಳೆ-ಯು-ವು-ದಿಲ್ಲ
ಮೊಳೆ-ಯು-ವುದು
ಮೊಳೆಯೇ
ಮೊಸ-ರನ್ನ
ಮೊಸ-ರನ್ನು
ಮೊಸ-ರಿನ
ಮೊಸ-ರಿ-ನಲ್ಲಿ
ಮೊಸರು
ಮೊಸಳೆ
ಮೊಸ-ಳೆ-ಗಳು
ಮೋಕ್ಷ
ಮೋಕ್ಷಂ
ಮೋಕ್ಷ-ಕಾಂ-ಕ್ಷಿ-ಭಿಃ
ಮೋಕ್ಷ-ಕಾರಿ
ಮೋಕ್ಷಕ್ಕೆ
ಮೋಕ್ಷ-ಗಳನ್ನು
ಮೋಕ್ಷ-ಗಳನ್ನೆಲ್ಲಾ
ಮೋಕ್ಷ-ಗ-ಳೆಲ್ಲ
ಮೋಕ್ಷ-ದಾ-ಯಕ
ಮೋಕ್ಷ-ಪ-ರಾ-ಯಣ
ಮೋಕ್ಷ-ಪ-ರಾ-ಯ-ಣ-ನಾಗಿ
ಮೋಕ್ಷ-ಯಿ-ಷ್ಯಾಮಿ
ಮೋಕ್ಷ-ವನ್ನು
ಮೋಕ್ಷ-ವೆಂಬ
ಮೋಕ್ಷಾ-ಕಾಂ-ಕ್ಷಿ-ಗಳು
ಮೋಕ್ಷ್ಯಸೇ
ಮೋಕ್ಷ್ಯ-ಸೇ-ಽಶು-ಭಾತ್
ಮೋಘಂ
ಮೋಘ-ಕ-ರ್ಮಾಣೋ
ಮೋಘ-ಜ್ಞಾನಾ
ಮೋಘಾಶಾ
ಮೋಚಿ-ಗಳಲ್ಲಿ
ಮೋಟು
ಮೋಡ
ಮೋಡಕ್ಕೆ
ಮೋಡ-ಗಳ
ಮೋಡ-ಗ-ಳಾಚೆ
ಮೋಡ-ಗಳಿಂದ
ಮೋಡ-ಗಳು
ಮೋಡದ
ಮೋಡ-ದಂತೆ
ಮೋಡ-ದ-ಲ್ಲಿ-ರುವ
ಮೋಡ-ದಿಂದ
ಮೋಡ-ವನ್ನು
ಮೋದ-ಪ-ಡು-ತ್ತೇನೆ
ಮೋದಿಷ್ಯ
ಮೋರೆ
ಮೋಸ
ಮೋಸ-ಗಾ-ರನೂ
ಮೋಸದ
ಮೋಸ-ಮಾ-ಡಿ-ರ-ಬ-ಹುದು
ಮೋಸ-ಮಾ-ಡು-ವಂ-ತಹ
ಮೋಸ-ವನ್ನು
ಮೋಹ
ಮೋಹಃ
ಮೋಹ-ಕ-ಲಿಲಂ
ಮೋಹ-ಕ-ವಾ-ಗಿದೆ
ಮೋಹ-ಕ-ವಾ-ಗಿ-ರು-ವುದು
ಮೋಹ-ಕ-ವಾದ
ಮೋಹಕ್ಕೆ
ಮೋಹ-ಗಳನ್ನೂ
ಮೋಹ-ಗ-ಳಿ-ಲ್ಲ-ದ-ವರೂ
ಮೋಹ-ಗೊ-ಳಿ-ಸು-ವುದು
ಮೋಹ-ಗೊ-ಳಿ-ಸು-ವುದೊ
ಮೋಹ-ಗೊ-ಳ್ಳು-ತ್ತಿವೆ
ಮೋಹ-ಗೊ-ಳ್ಳು-ವು-ದಿಲ್ಲ
ಮೋಹ-ಜಾ-ಲ-ದಿಂದ
ಮೋಹ-ಜಾ-ಲ-ಸ-ಮಾ-ವೃ-ತಾಃ
ಮೋಹದ
ಮೋಹ-ದಿಂದ
ಮೋಹನಂ
ಮೋಹ-ನ-ಮಾ-ತ್ಮನಃ
ಮೋಹ-ಪ-ಡಿ-ಸು-ವುದು
ಮೋಹ-ಮೇವ
ಮೋಹ-ಯ-ಸೀವ
ಮೋಹ-ವನ್ನು
ಮೋಹ-ವ-ಶ-ನಾಗಿ
ಮೋಹ-ವ-ಶ-ನಾ-ಗು-ವು-ದಿಲ್ಲ
ಮೋಹ-ವಿಲ್ಲ
ಮೋಹ-ವಿ-ಲ್ಲ-ದ-ವನು
ಮೋಹ-ವಿ-ಲ್ಲದೆ
ಮೋಹ-ವಿ-ಲ್ಲವೋ
ಮೋಹವೂ
ಮೋಹ-ಶೂ-ನ್ಯ-ನಾಗಿ
ಮೋಹಾ-ತ್ತಸ್ಯ
ಮೋಹಾ-ದಾ-ರ-ಭ್ಯತೇ
ಮೋಹಾ-ದ್ಗೃ-ಹೀತ್ವಾ
ಮೋಹಾ-ಸ-ಕ್ತಿ-ಯಲ್ಲಿ
ಮೋಹಿತಂ
ಮೋಹಿ-ತ-ನಾ-ಗದೆ
ಮೋಹಿ-ತ-ರಾಗಿ
ಮೋಹಿ-ತ-ರಾ-ಗಿ-ದ್ದಾನೆ
ಮೋಹಿ-ತ-ರಾ-ದ-ವರು
ಮೋಹಿ-ತ-ವಾ-ಗಿದೆ
ಮೋಹಿ-ತಾಃ
ಮೋಹಿ-ನೀಂ
ಮೋಹೋಽಯಂ
ಮೌಢ್ಯ
ಮೌಢ್ಯ-ತೆ-ಯಲ್ಲಿ
ಮೌನ
ಮೌನಂ
ಮೌನದ
ಮೌನ-ಮಾ-ತ್ಮ-ವಿ-ನಿ-ಗ್ರಹಃ
ಮೌನ-ವಾಗಿ
ಮೌನ-ವಾ-ಗಿ-ರ-ಬೇಕು
ಮೌನ-ವಾ-ಗಿ-ರು-ತ್ತಾನೆ
ಮೌನ-ವಾ-ಗಿ-ರು-ವುದು
ಮೌನ-ವಾ-ಗು-ವುದು
ಮೌನಿಯೂ
ಮ್ಯಾಗ್ನೆ-ಟಿಕ್
ಮ್ಯಾಚನ್ನು
ಮ್ಯಾಚ್
ಮ್ಯಾಟಿನಿ
ಮ್ಯಾನೇ-ಜರ್
ಮ್ರಿಯತೇ
ಮ್ಲಾನ-ವಾಗು
ಯ
ಯಂ
ಯಂತಹ
ಯಂತೆ
ಯಂತ್ರ
ಯಂತ್ರಕ್ಕೆ
ಯಂತ್ರ-ಗಳು
ಯಂತ್ರದ
ಯಂತ್ರ-ದಂತೆ
ಯಂತ್ರ-ದಲ್ಲಿ
ಯಂತ್ರ-ದಿಂದ
ಯಂತ್ರ-ವನ್ನು
ಯಂತ್ರ-ವನ್ನೋ
ಯಂತ್ರ-ವಾ-ಗ-ಬೇ-ಕಾ-ಗಿದೆ
ಯಂತ್ರ-ವಿದೆ
ಯಂತ್ರವೂ
ಯಂತ್ರಾ-ರೂ-ಢ-ವಾದ
ಯಂತ್ರಾ-ರೂ-ಢಾನಿ
ಯಃ
ಯಃಕ-ಶ್ಚಿತ್
ಯಕೃತ್
ಯಕ್ಷ
ಯಕ್ಷನ
ಯಕ್ಷ-ರ-ಕ್ಷ-ಸಾಮ್
ಯಕ್ಷ-ರ-ಕ್ಷಾಂಸಿ
ಯಕ್ಷ-ರಾ-ಕ್ಷ-ಸರ
ಯಕ್ಷ-ರಾ-ಕ್ಷ-ಸ-ರನ್ನು
ಯಕ್ಷರು
ಯಕ್ಷಾ-ಸುರ
ಯಕ್ಷ್ಯೇ
ಯಚ್ಚ
ಯಚ್ಚಂ-ದ್ರ-ಮಸಿ
ಯಚ್ಚಾಗ್ನೌ
ಯಚ್ಚಾ-ನ್ಯ-ದ್ದ್ರ-ಷ್ಟು-ಮಿ-ಚ್ಛಸಿ
ಯಚ್ಚಾಪಿ
ಯಚ್ಚಾ-ಪ್ಯುತ್ಕಾ
ಯಚ್ಚಾ-ವ-ಹಾ-ಸಾ-ರ್ಥ-ಮ-ಸ-ತ್ಕೃ-ತೋಽಸಿ
ಯಚ್ಛ್ರದ್ಧಃ
ಯಚ್ಛ್ರೇಯ
ಯಚ್ಛ್ರೇಯಃ
ಯಜ
ಯಜಂತ
ಯಜಂತೇ
ಯಜಂತೋ
ಯಜಂ-ತ್ಯ-ವಿ-ಧಿ-ಪೂ-ರ್ವ-ಕಮ್
ಯಜ-ಮಾನ
ಯಜ-ಮಾ-ನನ
ಯಜ-ಮಾ-ನ-ನಂತೆ
ಯಜ-ಮಾ-ನ-ನನ್ನು
ಯಜ-ಮಾ-ನ-ನಲ್ಲ
ಯಜ-ಮಾ-ನ-ನಾಗಿ
ಯಜ-ಮಾ-ನ-ನಾ-ಗಿದೆ
ಯಜ-ಮಾ-ನ-ನಿಗೆ
ಯಜ-ಮಾ-ನ-ನಿಗೇ
ಯಜ-ಮಾ-ನ-ನಿ-ರುವ
ಯಜ-ಮಾ-ನನು
ಯಜ-ಮಾ-ನನೇ
ಯಜ-ಮಾ-ನ-ರಂತೆ
ಯಜಸ್ಸು
ಯಜು-ರೇವ
ಯಜ್ಜು-ಹೋಷಿ
ಯಜ್ಜ್ಞಾತ್ವಾ
ಯಜ್ಜ್ಞಾ-ತ್ವಾ-ಮೃ-ತ-ಮ-ಶ್ನುತೇ
ಯಜ್ಜ್ಞಾನಂ
ಯಜ್ಞ
ಯಜ್ಞಂ
ಯಜ್ಞಃ
ಯಜ್ಞ-ಕಾ-ಲ-ದಲ್ಲಿ
ಯಜ್ಞ-ಕ್ಕಾಗಿ
ಯಜ್ಞಕ್ಕೂ
ಯಜ್ಞಕ್ಕೆ
ಯಜ್ಞ-ಕ್ಕೆಲ್ಲ
ಯಜ್ಞ-ಕ್ಕೋ-ಸುಗ
ಯಜ್ಞ-ಕ್ಕೋ-ಸ್ಕರ
ಯಜ್ಞ-ಕ್ಷ-ಪಿತ
ಯಜ್ಞ-ಗಳ
ಯಜ್ಞ-ಗಳನ್ನು
ಯಜ್ಞ-ಗಳನ್ನೆಲ್ಲ
ಯಜ್ಞ-ಗಳನ್ನೆಲ್ಲಾ
ಯಜ್ಞ-ಗಳಲ್ಲಿ
ಯಜ್ಞ-ಗಳಿಂದ
ಯಜ್ಞ-ಗ-ಳಿಗೂ
ಯಜ್ಞ-ಗ-ಳಿವೆ
ಯಜ್ಞ-ಗಳು
ಯಜ್ಞ-ಗ-ಳೆಲ್ಲ
ಯಜ್ಞ-ಗಳೇ
ಯಜ್ಞ-ಗ-ಳೊಂ-ದಿಗೆ
ಯಜ್ಞ-ಗ-ಳೊ-ಡನೆ
ಯಜ್ಞ-ತತ್ತ್ವ
ಯಜ್ಞದ
ಯಜ್ಞ-ದಂತೆ
ಯಜ್ಞ-ದಲ್ಲಿ
ಯಜ್ಞ-ದಾ-ನ-ತ-ಪಃ-ಕರ್ಮ
ಯಜ್ಞ-ದಾ-ನ-ತ-ಪಃ-ಕ್ರಿ-ಯಾಃ
ಯಜ್ಞ-ದಾ-ನ-ತ-ಪಸ್ಸು
ಯಜ್ಞ-ದಿಂದ
ಯಜ್ಞ-ದಿಂ-ದಲೇ
ಯಜ್ಞ-ದೃಷ್ಟಿ
ಯಜ್ಞ-ದೃ-ಷ್ಟಿಯ
ಯಜ್ಞ-ದೃ-ಷ್ಟಿ-ಯಂತೆ
ಯಜ್ಞ-ದೃ-ಷ್ಟಿ-ಯನ್ನು
ಯಜ್ಞ-ದೃ-ಷ್ಟಿ-ಯಿಂದ
ಯಜ್ಞ-ಪ್ರ-ಸಾ-ದ-ವನ್ನು
ಯಜ್ಞ-ಭಾ-ವ-ದಿಂದ
ಯಜ್ಞ-ಭಾ-ವನೆ
ಯಜ್ಞ-ಭಾ-ವ-ನೆ-ಯನ್ನು
ಯಜ್ಞ-ಭಾ-ವ-ನೆ-ಯೊಂ-ದನ್ನು
ಯಜ್ಞ-ಭಾ-ವಿ-ತಾಃ
ಯಜ್ಞ-ರೂ-ಪಕ್ಕೆ
ಯಜ್ಞ-ರೂ-ಪ-ದಂತೆ
ಯಜ್ಞ-ರೂ-ಪ-ದಲ್ಲಿ
ಯಜ್ಞ-ರೂ-ಪ-ದಿಂದ
ಯಜ್ಞ-ರೂ-ಪ-ವಾದ
ಯಜ್ಞ-ವನ್ನು
ಯಜ್ಞ-ವನ್ನೂ
ಯಜ್ಞ-ವನ್ನೋ
ಯಜ್ಞ-ವಾ-ಗ-ಬೇ-ಕಾ-ದರೆ
ಯಜ್ಞ-ವಾ-ಗಲಿ
ಯಜ್ಞ-ವಾ-ಗು-ವು-ದಿಲ್ಲ
ಯಜ್ಞ-ವಾ-ಗು-ವುದು
ಯಜ್ಞ-ವಾ-ದರೂ
ಯಜ್ಞ-ವಿದೋ
ಯಜ್ಞವೆ
ಯಜ್ಞ-ವೆಂದರೆ
ಯಜ್ಞ-ವೆಂದು
ಯಜ್ಞ-ವೆಂಬ
ಯಜ್ಞವೇ
ಯಜ್ಞ-ಶಿ-ಷ್ಟ-ವಾದ
ಯಜ್ಞ-ಶಿ-ಷ್ಟಾ-ಶಿನಃ
ಯಜ್ಞ-ಶೇ-ಷ-ವನ್ನು
ಯಜ್ಞಶ್ಚ
ಯಜ್ಞ-ಸ್ತ-ಪ-ಸ್ತಥಾ
ಯಜ್ಞ-ಸ್ವ-ರೂ-ಪ-ನಾ-ಗಿ-ರು-ವನು
ಯಜ್ಞಾ
ಯಜ್ಞಾತ್ವಾ
ಯಜ್ಞಾ-ದಿ-ಗಳಿಂದ
ಯಜ್ಞಾ-ದ್ಭ-ವತಿ
ಯಜ್ಞಾ-ನಾಂ
ಯಜ್ಞಾ-ಯಾ-ಚ-ರತಃ
ಯಜ್ಞಾ-ರ್ಥ-ವಾಗಿ
ಯಜ್ಞಾ-ರ್ಥಾತ್
ಯಜ್ಞಾಶ್ಚ
ಯಜ್ಞೇ
ಯಜ್ಞೇ-ನೈ-ವೋ-ಪ-ಜು-ಹ್ವತಿ
ಯಜ್ಞೇಷು
ಯಜ್ಞೈ-ರಿಷ್ಟ್ವಾ
ಯಜ್ಞೋ
ಯತಂ-ತಶ್ಚ
ಯತಂತಿ
ಯತಂತೋ
ಯತಂ-ತೋ-ಽಪ್ಯ-ಕೃ-ತಾ-ತ್ಮಾನೋ
ಯತಃ
ಯತ-ಚಿ-ತ್ತಸ್ಯ
ಯತ-ಚಿ-ತ್ತಾತ್ಮಾ
ಯತ-ಚಿ-ತ್ತೇಂ-ದ್ರಿ-ಯ-ಕ್ರಿಯಃ
ಯತ-ಚೇ-ತ-ಸಾಮ್
ಯತ-ತಶ್ಚ
ಯತತಾ
ಯತ-ತಾ-ಮಪಿ
ಯತತೇ
ಯತತೋ
ಯತಯಃ
ಯತ-ವಾ-ಕ್ಕಾ-ಯ-ಮಾ-ನಸಃ
ಯತಾತ್ಮ
ಯತಾ-ತ್ಮನೋ
ಯತಾ-ತ್ಮ-ವಾನ್
ಯತಾತ್ಮಾ
ಯತಾ-ತ್ಮಾನಃ
ಯತಿ
ಯತಿ-ಗಳು
ಯತಿ-ಗಳೇ
ಯತೀ-ನಾಂ
ಯತೇಂ-ದ್ರಿ-ಯ-ಮ-ನೋ-ಬು-ದ್ಧಿ-ರ್ಮು-ನಿ-ರ್ಮೋ-ಕ್ಷ-ಪ-ರಾ-ಯಣಃ
ಯತೋ
ಯತ್
ಯತ್ಕ-ರೋಷಿ
ಯತ್ಕರ್ಮ
ಯತ್ಕೃಪಾ
ಯತ್ತ-ಜ್ಜ್ಞಾನಂ
ಯತ್ತ-ತ್ತಾ-ಮ-ಸ-ಮು-ಚ್ಯತೇ
ಯತ್ತ-ತ್ಪ್ರ-ವ-ಕ್ಷ್ಯಾಮಿ
ಯತ್ತ-ತ್ಸಾ-ತ್ತ್ವಿ-ಕ-ಮು-ಚ್ಯತೇ
ಯತ್ತ-ದಗ್ರೇ
ಯತ್ತ-ದ್ಬು-ದ್ಧಿ-ಗ್ರಾ-ಹ್ಯ-ಮ-ತೀಂ-ದ್ರಿ-ಯಮ್
ಯತ್ತ-ಪ-ಸ್ಯಸಿ
ಯತ್ತು
ಯತ್ತೇಽಹಂ
ಯತ್ತ್ವ-ಯೋಕ್ತಂ
ಯತ್ನ-ಪೂ-ರ್ವಕ
ಯತ್ನವ
ಯತ್ನ-ಶೀ-ಲ-ನಾದ
ಯತ್ನಿ-ಸ-ಬ-ಹುದು
ಯತ್ನಿ-ಸ-ಬೇ-ಕಾ-ದರೆ
ಯತ್ನಿ-ಸ-ಬೇಕು
ಯತ್ನಿ-ಸ-ಬೇಡ
ಯತ್ನಿಸಿ
ಯತ್ನಿ-ಸಿತು
ಯತ್ನಿ-ಸಿ-ದಂತೆ
ಯತ್ನಿ-ಸಿ-ದರೂ
ಯತ್ನಿ-ಸಿ-ದರೆ
ಯತ್ನಿ-ಸಿ-ದ-ವನು
ಯತ್ನಿ-ಸಿ-ದಾಗ
ಯತ್ನಿ-ಸಿ-ದ್ದಾ-ಯಿತು
ಯತ್ನಿ-ಸಿ-ರು-ವರು
ಯತ್ನಿಸು
ಯತ್ನಿ-ಸು-ತ್ತಾನೆ
ಯತ್ನಿ-ಸು-ತ್ತಾರೆ
ಯತ್ನಿ-ಸು-ತ್ತಾ-ರೆಯೋ
ಯತ್ನಿ-ಸು-ತ್ತಿರ
ಯತ್ನಿ-ಸು-ತ್ತಿ-ರುವ
ಯತ್ನಿ-ಸು-ತ್ತಿ-ರು-ವನು
ಯತ್ನಿ-ಸು-ತ್ತಿ-ರು-ವರು
ಯತ್ನಿ-ಸು-ತ್ತಿ-ರು-ವ-ವನು
ಯತ್ನಿ-ಸು-ತ್ತಿ-ರು-ವ-ವರ
ಯತ್ನಿ-ಸು-ತ್ತಿ-ರು-ವೆ-ವಲ್ಲ
ಯತ್ನಿ-ಸು-ತ್ತಿಲ್ಲ
ಯತ್ನಿ-ಸು-ತ್ತೇವೆ
ಯತ್ನಿ-ಸು-ವನು
ಯತ್ನಿ-ಸು-ವನೋ
ಯತ್ನಿ-ಸು-ವರು
ಯತ್ನಿ-ಸು-ವು-ದಿಲ್ಲ
ಯತ್ನಿ-ಸು-ವುದು
ಯತ್ನಿ-ಸು-ವು-ದೆಲ್ಲ
ಯತ್ನಿ-ಸು-ವುದೇ
ಯತ್ನಿ-ಸು-ವುದೋ
ಯತ್ನಿ-ಸು-ವೆನು
ಯತ್ನಿ-ಸು-ವೆವು
ಯತ್ನಿ-ಸು-ವೆವೊ
ಯತ್ಪ-ಭಾ-ವಶ್ಚ
ಯತ್ಪೀ-ಡಯಾ
ಯತ್ಪು-ಣ್ಯ-ಫಲಂ
ಯತ್ಪ್ರ-ಮಾಣಂ
ಯತ್ರ
ಯತ್ರ-ಚೈ-ವಾ-ತ್ಮ-ನಾ-ತ್ಮಾನಂ
ಯತ್ರೋ-ಪ-ರ-ಮತೇ
ಯತ್ಸು-ಖಮ್
ಯಥಾ
ಯಥಾ-ಕಾ-ಶ-ಸ್ಥಿತೋ
ಯಥಾ-ಭಾ-ಗ-ಮ-ವ-ಸ್ಥಿ-ತಾಃ
ಯಥಾರ್ಥ
ಯಥಾ-ರ್ಥ-ವಾಗಿ
ಯಥಾ-ರ್ಥ-ಸ್ಥಿ-ತಿ-ಯನ್ನು
ಯಥಾ-ವ-ಚ್ಛೃಣು
ಯಥೇ-ಚ್ಛಸಿ
ಯಥೈ-ಧಾಂಸಿ
ಯಥೋಕ್ತಂ
ಯಥೋ-ಲ್ಬೇ-ನಾ-ವೃತೋ
ಯದ-ಕ್ಷರಂ
ಯದಗ್ರೇ
ಯದ-ತೋ-ಽನ್ಯಥಾ
ಯದ-ವಾ-ಪ್ನೋತಿ
ಯದ-ಶ್ನಾಸಿ
ಯದ-ಹಂ-ಕಾ-ರ-ಮಾ-ಶ್ರಿತ್ಯ
ಯದಾ
ಯದಾ-ದಿ-ತ್ಯ-ಗತಂ
ಯದಿ
ಯದಿ-ಚ್ಛಂತೋ
ಯದೀ-ದೃ-ಶಮ್
ಯದು
ಯದು-ಕು-ಲದ
ಯದುಕ್ತಂ
ಯದೃ-ಚ್ಛಯಾ
ಯದೃ-ಚ್ಛಾ-ಲಾ-ಭ-ಸಂ-ತುಷ್ಟೋ
ಯದೇ-ಭಿಃ
ಯದ್ಗತ್ವಾ
ಯದ್ದಾನಂ
ಯದ್ದಾ-ನ-ಮ-ಪಾ-ತ್ರೇ-ಭ್ಯಶ್ಚ
ಯದ್ಯ-ತಯೋ
ಯದ್ಯ-ದಾ-ಚ-ರತಿ
ಯದ್ಯ-ದ್ವಿ-ಭೂ-ತಿ-ಮತ್
ಯದ್ಯ-ಪ್ಯೇತೇ
ಯದ್ರಾ-ಜ್ಯ-ಸು-ಖ-ಲೋ-ಭೇನ
ಯದ್ವತ್
ಯದ್ವಾ
ಯನ್ನಾ-ದರೂ
ಯನ್ನು
ಯನ್ನೂ
ಯನ್ಮ-ನೋ-ಽನು-ವಿ-ಧೀ-ಯತೇ
ಯನ್ಮಮ
ಯನ್ಮಾಂ
ಯನ್ಮೇ
ಯನ್ಮೋ-ಹಾತ್
ಯಮ
ಯಮಃ
ಯಮ-ಕಾ-ಟ-ದಿಂದ
ಯಮ-ಧ-ರ್ಮ-ರಾಯ
ಯಮನ
ಯಮ-ನನ್ನು
ಯಮ-ಯಾ-ತನೆ
ಯಮ-ಲೋ-ಕ-ದಲ್ಲಿ
ಯಮುನೆ
ಯಯಾ
ಯಯಾತಿ
ಯಯೇದಂ
ಯರು
ಯಲ್ಲ
ಯಲ್ಲಿ
ಯಲ್ಲಿತ್ತೋ
ಯಲ್ಲಿದೆ
ಯಲ್ಲಿ-ದೆಯೊ
ಯಲ್ಲಿಯೂ
ಯಲ್ಲಿಯೆ
ಯಲ್ಲಿಯೇ
ಯಲ್ಲಿ-ರುವ
ಯವನು
ಯವನೆ
ಯವ-ರಿಗೂ
ಯವೋ
ಯಶಸ್ವಿ
ಯಶ-ಸ್ವಿ-ಯಾಗಿ
ಯಶ-ಸ್ಸನ್ನು
ಯಶ-ಸ್ಸಿಗೆ
ಯಶ-ಸ್ಸಿ-ನಂ-ತೆಯೇ
ಯಶಸ್ಸು
ಯಶೋ
ಯಶೋ-ದೆಗೆ
ಯಶೋ-ಽಯಶಃ
ಯಶ್ಚೈನಂ
ಯಷ್ಟ-ವ್ಯ-ಮೇ-ವೇತಿ
ಯಷ್ಟು
ಯಸ್ತಂ
ಯಸ್ತು
ಯಸ್ತ್ವಾ-ತ್ಮ-ರ-ತಿ-ರೇವ
ಯಸ್ತ್ವಿಂ-ದ್ರಿ-ಯಾಣಿ
ಯಸ್ಮಾತ್
ಯಸ್ಮಾ-ನ್ನೋ-ದ್ವಿ-ಜತೇ
ಯಸ್ಮಿಂ-ಸ್ಥಿತೋ
ಯಸ್ಮಿನ್
ಯಸ್ಯ
ಯಸ್ಯಾಂ
ಯಸ್ಯಾಂತಂ
ಯಸ್ಯಾಂ-ತಃ-ಸ್ಥಾನಿ
ಯಸ್ಯೇಂ-ದ್ರಿ-ಯಾಣಿ
ಯಾ
ಯಾಂ
ಯಾಂತಿ
ಯಾಂತ್ಯ-ಧ-ಮಾಂ
ಯಾಂತ್ರಿ-ಕ-ನಂತೆ
ಯಾಂತ್ರಿ-ಕ-ವಾ-ಗಲ್ಲ
ಯಾಂತ್ರಿ-ಕ-ವಾಗಿ
ಯಾಂತ್ರಿ-ಕ-ವಾ-ಗಿ-ರುವ
ಯಾಃ
ಯಾಕೆಂ-ದರೆ
ಯಾಗ
ಯಾಗ-ಬ-ಲ್ಲದು
ಯಾಗ-ಬೇಕು
ಯಾಗ-ಯಜ್ಞ
ಯಾಗ-ಯ-ಜ್ಞ-ಗಳನ್ನು
ಯಾಗ-ಯ-ಜ್ಞ-ವನ್ನು
ಯಾಗ-ಯ-ಜ್ಞಾ-ದಿ-ಗಳನ್ನು
ಯಾಗ-ವನ್ನು
ಯಾಗ-ವನ್ನೋ
ಯಾಗಿ
ಯಾಗಿದೆ
ಯಾಗಿ-ದ್ದಾನೆ
ಯಾಗಿ-ರ-ಬೇಕು
ಯಾಗಿ-ರು-ವನು
ಯಾಗಿ-ರು-ವನೊ
ಯಾಗಿ-ರು-ವು-ದ-ರಿಂದ
ಯಾಗಿ-ರು-ವುದು
ಯಾಗಿಲ್ಲ
ಯಾಗುತ್ತ
ಯಾಗು-ವನೊ
ಯಾಗು-ವು-ದಿಲ್ಲ
ಯಾಗು-ವುದು
ಯಾಚಿ-ಸಿ-ದರೆ
ಯಾಚಿ-ಸು-ತ್ತಿ-ರು-ವ-ವನು
ಯಾಚಿ-ಸು-ತ್ತೇವೆ
ಯಾಚಿ-ಸು-ವನು
ಯಾಚಿ-ಸು-ವರು
ಯಾಚಿ-ಸು-ವು-ದಕ್ಕೆ
ಯಾಚಿ-ಸು-ವು-ದಿಲ್ಲ
ಯಾತ-ಕ್ಕಾಗಿ
ಯಾತ-ಕ್ಕಾ-ದರೂ
ಯಾತಕ್ಕೆ
ಯಾತ-ನಾ-ಮಯ
ಯಾತ-ನಾ-ಮ-ಯ-ವಾ-ಗಿ-ದ್ದರೂ
ಯಾತನೆ
ಯಾತ-ನೆ-ಗಳನ್ನೆಲ್ಲ
ಯಾತ-ನೆಯ
ಯಾತ-ನೆ-ಯನ್ನು
ಯಾತ-ನೆ-ಯ-ನ್ನೆಲ್ಲ
ಯಾತ-ನೆ-ಯ-ನ್ನೆಲ್ಲಾ
ಯಾತ-ನೆ-ಯಿಂದ
ಯಾತ-ನೆ-ಯಿ-ಲ್ಲದೆ
ಯಾತ-ಯಾಮಂ
ಯಾತಿ
ಯಾತ್ಯ-ನಾ-ವೃ-ತ್ತಿ-ಮ-ನ್ಯ-ಯಾ-ವ-ರ್ತತೇ
ಯಾತ್ರಿಕ
ಯಾತ್ರಿ-ಕ-ನಲ್ಲಿ
ಯಾತ್ರಿ-ಕ-ನಾ-ಗ-ಬ-ಹುದು
ಯಾತ್ರಿ-ಕ-ರಲ್ಲಿ
ಯಾತ್ರಿ-ಕ-ರಿಗೆ
ಯಾತ್ರಿ-ಕರು
ಯಾತ್ರೆ
ಯಾತ್ರೆಗೆ
ಯಾದ
ಯಾದರೂ
ಯಾದರೆ
ಯಾದರೋ
ಯಾದವ
ಯಾದ-ವ-ಕು-ಲ-ದಲ್ಲೆಲ್ಲಾ
ಯಾದ-ವ-ಗಿರಿ
ಯಾದ-ವನು
ಯಾದ-ವ-ರಲ್ಲಿ
ಯಾದ-ಸಾ-ಮ-ಹಮ್
ಯಾದಾಗ
ಯಾನೇನ
ಯಾಪನೆ
ಯಾಭಿ-ರ್ವಿ-ಭೂ-ತಿ-ಭಿ-ರ್ಲೋ-ಕಾ-ನಿ-ಮಾಂಸ್ತ್ವಂ
ಯಾಮಿ-ಮಾಂ
ಯಾರ
ಯಾರ-ದನ್ನೋ
ಯಾರದು
ಯಾರದೊ
ಯಾರದೋ
ಯಾರ-ನ್ನಾ-ದರೂ
ಯಾರನ್ನು
ಯಾರನ್ನೂ
ಯಾರನ್ನೊ
ಯಾರನ್ನೋ
ಯಾರ-ಮೇಲೂ
ಯಾರ-ಮೇಲೆ
ಯಾರ-ಮೇಲೊ
ಯಾರ-ಲ್ಲವೋ
ಯಾರಲ್ಲಿ
ಯಾರ-ಲ್ಲಿಯೂ
ಯಾರಾ
ಯಾರಾ-ಗಿ-ದ್ದಾರೆ
ಯಾರಾ-ದರೂ
ಯಾರಾ-ದ-ರೊ-ಬ್ಬರು
ಯಾರಾರು
ಯಾರಿ
ಯಾರಿಂ
ಯಾರಿಂದ
ಯಾರಿಂ-ದಲೂ
ಯಾರಿಂ-ದಲೋ
ಯಾರಿ-ಗಾಗಿ
ಯಾರಿ-ಗಾ-ದರೂ
ಯಾರಿಗೂ
ಯಾರಿಗೆ
ಯಾರಿಗೊ
ಯಾರಿಗೋ
ಯಾರಿ-ದ್ದರೆ
ಯಾರಿ-ದ್ದಾರೆ
ಯಾರಿ-ರು-ವರು
ಯಾರಿ-ರು-ವರೊ
ಯಾರು
ಯಾರು-ಯಾ-ರನ್ನು
ಯಾರು-ಯಾರು
ಯಾರೂ
ಯಾರೆ-ದು-ರಿಗೆ
ಯಾರೇ
ಯಾರೇನು
ಯಾರೊ
ಯಾರೊ-ಡನೆ
ಯಾರೊ-ಬ್ಬರ
ಯಾರೋ
ಯಾರ್ಯಾರು
ಯಾವ
ಯಾವ-ಜ್ಜೀ-ವ-ನವೂ
ಯಾವತ್
ಯಾವ-ದಾರಿ
ಯಾವ-ದೇ-ತಾನ್
ಯಾವನು
ಯಾವನೂ
ಯಾವನೊ
ಯಾವನೋ
ಯಾವ-ಯಾವ
ಯಾವಳೋ
ಯಾವಾ
ಯಾವಾಗ
ಯಾವಾ-ಗ-ಲಾ-ದರೂ
ಯಾವಾ-ಗ-ಲಾ-ದ-ರೊಮ್ಮೆ
ಯಾವಾ-ಗಲೂ
ಯಾವಾ-ಗಲೋ
ಯಾವಾ-ನರ್ಥ
ಯಾವಾ-ನ್ಯ-ಶ್ಚಾಸ್ಮಿ
ಯಾವು
ಯಾವುಕ್ಕೆ
ಯಾವುದ
ಯಾವು-ದ-ಕ್ಕಾಗಿ
ಯಾವು-ದ-ಕ್ಕಾ-ದರೂ
ಯಾವು-ದ-ಕ್ಕಿಂತ
ಯಾವು-ದಕ್ಕೂ
ಯಾವು-ದಕ್ಕೆ
ಯಾವು-ದ-ಕ್ಕೆ-ಕೇ-ವಲ
ಯಾವು-ದಕ್ಕೊ
ಯಾವು-ದ-ನ್ನಾ-ದರೂ
ಯಾವುದನ್ನು
ಯಾವು-ದನ್ನೂ
ಯಾವು-ದನ್ನೊ
ಯಾವು-ದನ್ನೋ
ಯಾವು-ದರ
ಯಾವು-ದ-ರ-ಮೇಲೆ
ಯಾವು-ದ-ರಲ್ಲಿ
ಯಾವು-ದ-ರ-ಲ್ಲಿ-ದ್ದು-ಕೊಂಡು
ಯಾವು-ದ-ರ-ಲ್ಲಿಯೂ
ಯಾವು-ದ-ರಿಂದ
ಯಾವು-ದ-ರಿಂ-ದ-ಲಾ-ದರೂ
ಯಾವು-ದ-ರಿಂ-ದಲೂ
ಯಾವು-ದ-ರಿಂ-ದಲೋ
ಯಾವು-ದಾ-ದರೂ
ಯಾವು-ದಾ-ದರೊ
ಯಾವು-ದಾ-ದ-ರೊಂದು
ಯಾವು-ದಿದೆ
ಯಾವು-ದಿ-ದೆಯೋ
ಯಾವು-ದಿ-ದ್ದರೂ
ಯಾವು-ದಿ-ಲ್ಲದೇ
ಯಾವುದು
ಯಾವುದೂ
ಯಾವು-ದೆಂ-ದರೆ
ಯಾವುದೇ
ಯಾವುದೊ
ಯಾವು-ದೊಂ-ದ-ರಿಂದ
ಯಾವು-ದೊಂದೂ
ಯಾವುದೋ
ಯಾವು-ಯಾ-ವುದು
ಯಾವುವು
ಯಾವುವೂ
ಯಾವೊಂದು
ಯಾಸ್ಯಸಿ
ಯಿಂದ
ಯಿಂದಲೂ
ಯಿಂದಲೇ
ಯಿಂದಲೋ
ಯಿಂದೆ-ದ್ದಿವೆ
ಯಿತ್ತೊ
ಯಿದೆ
ಯಿಲ್ಲ
ಯುಂಜತೋ
ಯುಂಜನ್
ಯುಂಜ-ನ್ನೇವಂ
ಯುಂಜೀತ
ಯುಂಜ್ಯಾ-ದ್ಯೋ-ಗ-ಮಾ-ತ್ಮ-ವಿ-ಶು-ದ್ಧಯೇ
ಯುಕ್ತ
ಯುಕ್ತಃ
ಯುಕ್ತ-ಚೇ-ಷ್ಟಸ್ಯ
ಯುಕ್ತ-ತಮಾ
ಯುಕ್ತ-ತಮೋ
ಯುಕ್ತ-ನಾ-ಗ-ದ-ವನು
ಯುಕ್ತ-ನಾಗಿ
ಯುಕ್ತ-ನಾ-ಗು-ತ್ತಾನೆ
ಯುಕ್ತ-ನಾದ
ಯುಕ್ತ-ನೆಂದು
ಯುಕ್ತ-ವಲ್ಲ
ಯುಕ್ತವೇ
ಯುಕ್ತ-ಸ್ತ-ಸ್ಯಾ-ರಾ-ಧ-ನ-ಮೀ-ಹತೇ
ಯುಕ್ತ-ಸ್ವ-ಪ್ನಾ-ವ-ಬೋ-ಧಸ್ಯ
ಯುಕ್ತಾತ್ಮ
ಯುಕ್ತಾತ್ಮಾ
ಯುಕ್ತಾ-ಯುಕ್ತ
ಯುಕ್ತಾ-ಯು-ಕ್ತ-ಪ-ರಿ-ಜ್ಞಾ-ನ-ವನ್ನು
ಯುಕ್ತಾ-ಯುಕ್ತಾ
ಯುಕ್ತಾ-ಹಾರ
ಯುಕ್ತಾ-ಹಾ-ರ-ವಿ-ಹಾ-ರಸ್ಯ
ಯುಕ್ತಿ
ಯುಕ್ತಿ-ಗಿಂತ
ಯುಕ್ತಿಗೆ
ಯುಕ್ತಿಯ
ಯುಕ್ತಿ-ಯನ್ನು
ಯುಕ್ತೋ
ಯುಕ್ತ್ವೈ-ವ-ಮಾ-ತ್ಮಾನಂ
ಯುಗ
ಯುಗಕ್ಕೆ
ಯುಗ-ಗಳಲ್ಲಿ
ಯುಗ-ಗ-ಳಾ-ದರೂ
ಯುಗ-ಗಳು
ಯುಗದ
ಯುಗ-ದಲ್ಲಿ
ಯುಗ-ದ-ಲ್ಲಿಯೂ
ಯುಗ-ವನ್ನು
ಯುಗ-ಸ-ಹ-ಸ್ರಾಂ-ತಾಂ
ಯುಗೇ
ಯುಜ್ಯತೇ
ಯುಜ್ಯಸ್ವ
ಯುದ್ಧ
ಯುದ್ಧ-ಕ್ಕಾಗಿ
ಯುದ್ಧಕ್ಕೆ
ಯುದ್ಧ-ಕ್ಷೇ-ತ್ರ-ದಲ್ಲಿ
ಯುದ್ಧ-ಗಳನ್ನು
ಯುದ್ಧ-ಗಳಲ್ಲಿ
ಯುದ್ಧ-ಗಳು
ಯುದ್ಧದ
ಯುದ್ಧ-ದಂತೆ
ಯುದ್ಧ-ದ-ಲ್ಲಾ-ಗಲೀ
ಯುದ್ಧ-ದಲ್ಲಿ
ಯುದ್ಧ-ದ-ಲ್ಲಿದ್ದು
ಯುದ್ಧ-ದಿಂದ
ಯುದ್ಧ-ನೀ-ತಿ-ಯಲ್ಲಿ
ಯುದ್ಧ-ಭೂಮಿ
ಯುದ್ಧ-ಭೂ-ಮಿಗೆ
ಯುದ್ಧ-ಭೂ-ಮಿ-ಯಲ್ಲಿ
ಯುದ್ಧ-ಮಾಡ
ಯುದ್ಧ-ಮಾ-ಡದೆ
ಯುದ್ಧ-ಮಾ-ಡ-ಬೇ-ಕಾ-ಗಿದೆ
ಯುದ್ಧ-ಮಾ-ಡ-ಬೇ-ಕೆಂದು
ಯುದ್ಧ-ಮಾ-ಡ-ಬೇ-ಕೆಂದೂ
ಯುದ್ಧ-ಮಾ-ಡಲು
ಯುದ್ಧ-ಮಾಡಿ
ಯುದ್ಧ-ಮಾ-ಡಿದ
ಯುದ್ಧ-ಮಾ-ಡಿ-ದರೆ
ಯುದ್ಧ-ಮಾಡು
ಯುದ್ಧ-ಮಾ-ಡು-ವನೋ
ಯುದ್ಧ-ಮಾ-ಡು-ವರೆ
ಯುದ್ಧ-ಮಾ-ಡು-ವರೋ
ಯುದ್ಧ-ಮಾ-ಡುವು
ಯುದ್ಧ-ಮಾ-ಡು-ವು-ದ-ಕ್ಕಾಗಿ
ಯುದ್ಧ-ಮಾ-ಡು-ವು-ದಕ್ಕೆ
ಯುದ್ಧ-ಮಾ-ಡು-ವುದನ್ನು
ಯುದ್ಧ-ಮಾ-ಡು-ವು-ದಿಲ್ಲ
ಯುದ್ಧ-ಮಾ-ರಿಗೆ
ಯುದ್ಧ-ಮೀ-ದೃ-ಶಮ್
ಯುದ್ಧ-ರಂ-ಗ-ದಲ್ಲಿ
ಯುದ್ಧ-ರಂ-ಗ-ದಿಂದ
ಯುದ್ಧ-ವನ್ನು
ಯುದ್ಧ-ವನ್ನೂ
ಯುದ್ಧ-ವನ್ನೇ
ಯುದ್ಧ-ವಲ್ಲ
ಯುದ್ಧ-ವಾ-ಗು-ತ್ತಿ-ರು-ವಾಗ
ಯುದ್ಧ-ವಾದ
ಯುದ್ಧ-ವಿದೆ
ಯುದ್ಧ-ವಿ-ಶಾ-ರ-ದಾಃ
ಯುದ್ಧವೂ
ಯುದ್ಧ-ವೆಲ್ಲ
ಯುದ್ಧಾ-ಚ್ಛ್ರೇ-ಯೋ-ಽನ್ಯತ್
ಯುದ್ಧಾಯ
ಯುದ್ಧೇ
ಯುದ್ಧೋ-ನ್ಮ-ಖ-ನ-ನ್ನಾಗಿ
ಯುಧಾ-ಮನ್ಯು
ಯುಧಾ-ಮ-ನ್ಯುಶ್ಚ
ಯುಧಿ
ಯುಧಿ-ಷ್ಠಿರಃ
ಯುಧಿ-ಷ್ಠಿ-ರನ
ಯುಧಿ-ಷ್ಠಿ-ರನು
ಯುಧ್ಯ
ಯುಧ್ಯಸ್ವ
ಯುಯು-ತ್ಸವಃ
ಯುಯು-ತ್ಸುಂ
ಯುಯು-ಧಾನ
ಯುಯು-ಧಾನೋ
ಯುಳ್ಳ-ವ-ನಾಗಿ
ಯುಳ್ಳ-ವನು
ಯುಳ್ಳ-ವನೂ
ಯುವಂ-ತಿಲ್ಲ
ಯುವಕ
ಯುವ-ಕ-ನಾಗಿ
ಯುವುದು
ಯೂನಿ-ಫಾರಂ
ಯೆಲ್ಲ
ಯೇ
ಯೇನ
ಯೇನಾ-ತ್ಮೈ-ವಾ-ತ್ಮನಾ
ಯೇನೈಕಂ
ಯೇಷಾಂ
ಯೇಷಾ-ಮರ್ಥೇ
ಯೇಽಪಿ
ಯೇಽಪ್ಯ-ನ್ಯ-ದೇ-ವತಾ
ಯೇಽವ-ಸ್ಥಿ-ತಾಃ
ಯೈದೃ-ಕ್ಚ-ಯ-ದ್ವಿ-ಕಾರಿ
ಯೊಂದನ್ನೂ
ಯೊಂದನ್ನೇ
ಯೊಂದಿಗೆ
ಯೊಂದಿ-ದ್ದರೆ
ಯೊಂದು
ಯೊಂದೂ
ಯೊಬ್ಬನೂ
ಯೋ
ಯೋಕ್ತವ್ಯೋ
ಯೋಗ
ಯೋಗಂ
ಯೋಗಃ
ಯೋಗ-ಕ್ಕಾಗಿ
ಯೋಗಕ್ಕೆ
ಯೋಗ-ಕ್ಷೇಮಂ
ಯೋಗ-ಕ್ಷೇ-ಮ-ವನ್ನು
ಯೋಗ-ಕ್ಷೇ-ಮ-ವಿ-ಲ್ಲ-ದ-ವ-ನಾಗು
ಯೋಗ-ಗಳ
ಯೋಗ-ಗ-ಳಿವೆ
ಯೋಗ-ಗಳು
ಯೋಗ-ಗ-ಳೊಂ-ದಿಗೆ
ಯೋಗ-ಜೀ-ವನ
ಯೋಗ-ಜೀ-ವ-ನಕ್ಕೆ
ಯೋಗ-ಜೀ-ವ-ನ-ದಲ್ಲಿ
ಯೋಗದ
ಯೋಗ-ದಲ್ಲಿ
ಯೋಗ-ದ-ಲ್ಲಿ-ಟ್ಟಿ-ರ-ಬೇಕು
ಯೋಗ-ದ-ಲ್ಲಿ-ಟ್ಟು-ಕೊಂಡು
ಯೋಗ-ದ-ಲ್ಲಿ-ಟ್ಟು-ಕೊ-ಳ್ಳುವ
ಯೋಗ-ದ-ಲ್ಲಿ-ಡ-ಬೇಕು
ಯೋಗ-ದ-ಲ್ಲಿಯೂ
ಯೋಗ-ದಿಂದ
ಯೋಗ-ದಿಂ-ದಲೇ
ಯೋಗ-ದೃ-ಷ್ಟಿ-ಯಿಂದ
ಯೋಗ-ದೊ-ಡನೆ
ಯೋಗ-ಧಾ-ರ-ಣಾಮ್
ಯೋಗ-ಬ-ಲ-ದಿಂದ
ಯೋಗ-ಬ-ಲೇನ
ಯೋಗ-ಭ್ರಷ್ಟ
ಯೋಗ-ಭ್ರ-ಷ್ಟ-ನಾ-ಗು-ವುದು
ಯೋಗ-ಭ್ರ-ಷ್ಟ-ನಿಗೆ
ಯೋಗ-ಭ್ರ-ಷ್ಟೋ-ಽಭಿ-ಜಾ-ಯತೇ
ಯೋಗ-ಭ್ರ-ಷ್ಠ-ನಿಗೆ
ಯೋಗ-ಭ್ರ-ಷ್ಠನು
ಯೋಗ-ಮ-ವಾ-ಪ್ಸ್ಯಸಿ
ಯೋಗ-ಮ-ಹಾ-ನ-ದಿ-ಗಳು
ಯೋಗ-ಮಾ-ತಿ-ಷ್ಠೋ-ತ್ತಿಷ್ಠ
ಯೋಗ-ಮಾ-ತ್ಮನಃ
ಯೋಗ-ಮಾ-ಯಾ-ಸ-ಮಾ-ವೃತಃ
ಯೋಗ-ಮಾಯೆ
ಯೋಗ-ಮಾ-ಯೆ-ಯಲ್ಲಿ
ಯೋಗ-ಮಾ-ಯೆ-ಯಿಂದ
ಯೋಗ-ಮೈ-ಶ್ವ-ರಮ್
ಯೋಗ-ಯಜ್ಞ
ಯೋಗ-ಯ-ಜ್ಞ-ವನ್ನು
ಯೋಗ-ಯ-ಜ್ಞಾ-ಸ್ತ-ಥಾ-ಪರೇ
ಯೋಗ-ಯು-ಕ್ತಾತ್ಮಾ
ಯೋಗ-ಯುಕ್ತೋ
ಯೋಗ-ವ-ನ್ನಾ-ದರೂ
ಯೋಗ-ವನ್ನು
ಯೋಗ-ವನ್ನೇ
ಯೋಗ-ವಾಗ
ಯೋಗ-ವಾ-ಗ-ಬೇ-ಕಾ-ದರೆ
ಯೋಗ-ವಿ-ತ್ತ-ಮಾಃ
ಯೋಗವೂ
ಯೋಗ-ವೆಂದು
ಯೋಗ-ವೆಂಬ
ಯೋಗ-ವೆ-ನ್ನು-ವುದು
ಯೋಗವೇ
ಯೋಗ-ಶಕ್ತಿ
ಯೋಗ-ಶ-ಕ್ತಿ-ಯನ್ನು
ಯೋಗ-ಶಾಸ್ತ್ರ
ಯೋಗ-ಶಾಸ್ತ್ರೇ
ಯೋಗ-ಸಂ-ಜ್ಞಿ-ತಮ್
ಯೋಗ-ಸಂ-ನ್ಯ-ಸ್ತ-ಕ-ರ್ಮಾಣಂ
ಯೋಗ-ಸಂ-ಸಿದ್ಧಃ
ಯೋಗ-ಸಂ-ಸಿ-ದ್ಧಿಂ
ಯೋಗ-ಸಾ-ಧನೆ
ಯೋಗ-ಸೇ-ವಯಾ
ಯೋಗ-ಸ್ತ್ವಯಾ
ಯೋಗಸ್ಥಃ
ಯೋಗಸ್ಯ
ಯೋಗಾ-ಗ್ನಿ-ಯಲ್ಲಿ
ಯೋಗಾ-ಚ್ಚ-ಲಿ-ತ-ಮಾ-ನಸಃ
ಯೋಗಾ-ಭ್ಯಾಸ
ಯೋಗಾ-ಭ್ಯಾ-ಸ-ದಿಂದ
ಯೋಗಾ-ರೂ-ಢ-ಸ್ತ-ದೋ-ಚ್ಯತೇ
ಯೋಗಾ-ರೂ-ಢಸ್ಯ
ಯೋಗಾ-ವ-ಸ್ಥೆ-ಯ-ಲ್ಲಿ-ದ್ದಾಗ
ಯೋಗಿ
ಯೋಗಿಂ-ಸ್ತ್ವಾಂ
ಯೋಗಿ-ಗಳ
ಯೋಗಿ-ಗಳಲ್ಲಿ
ಯೋಗಿ-ಗ-ಳಾ-ಗ-ಲಾ-ರರು
ಯೋಗಿ-ಗ-ಳಾದ
ಯೋಗಿ-ಗ-ಳಿಗೆ
ಯೋಗಿ-ಗಳು
ಯೋಗಿ-ಗ-ಳೆಲ್ಲ
ಯೋಗಿ-ಗಾ-ದರೊ
ಯೋಗಿಗೆ
ಯೋಗಿನಂ
ಯೋಗಿನಃ
ಯೋಗಿ-ನ-ಶ್ಚೈನಂ
ಯೋಗಿ-ನಾ-ಮಪಿ
ಯೋಗಿ-ನಾ-ಮೇವ
ಯೋಗಿ-ನಾಮ್
ಯೋಗಿನೋ
ಯೋಗಿಯ
ಯೋಗಿ-ಯನ್ನು
ಯೋಗಿ-ಯಲ್ಲಿ
ಯೋಗಿ-ಯಾ-ಗ-ಬ-ಯ-ಸು-ವ-ವನು
ಯೋಗಿ-ಯಾ-ಗ-ಬೇಕು
ಯೋಗಿ-ಯಾ-ಗ-ಬೇ-ಕೆಂದು
ಯೋಗಿ-ಯಾ-ಗ-ಲಾರ
ಯೋಗಿ-ಯಾ-ಗ-ಲಾ-ರರು
ಯೋಗಿ-ಯಾ-ಗಲು
ಯೋಗಿ-ಯಾ-ಗಿ-ರು-ವ-ವನು
ಯೋಗಿ-ಯಾಗು
ಯೋಗಿ-ಯಾ-ಗು-ವು-ದಕ್ಕೆ
ಯೋಗಿ-ಯಾ-ಗು-ವುದು
ಯೋಗಿ-ಯಾ-ದರೊ
ಯೋಗಿ-ಯಾ-ದರೋ
ಯೋಗಿ-ಯಾ-ದ-ವನು
ಯೋಗಿಯು
ಯೋಗಿಯೂ
ಯೋಗಿಯೊ
ಯೋಗಿ-ಸ-ಬೇಕು
ಯೋಗಿ-ಸಿ-ಕೊ-ಳ್ಳ-ಬ-ಹುದು
ಯೋಗಿ-ಸಿ-ಕೊ-ಳ್ಳು-ತ್ತೇ-ವೆಯೊ
ಯೋಗಿ-ಸು-ತ್ತಾನೆ
ಯೋಗಿ-ಸು-ತ್ತೇವೆ
ಯೋಗಿ-ಸುವ
ಯೋಗೀ
ಯೋಗೇನ
ಯೋಗೇ-ನಾ-ವ್ಯ-ಭಿ-ಚಾ-ರಿಣ್ಯಾ
ಯೋಗೇ-ಶ್ವರ
ಯೋಗೇ-ಶ್ವರಃ
ಯೋಗೇ-ಶ್ವ-ರ-ನನ್ನು
ಯೋಗೇ-ಶ್ವ-ರ-ನಾದ
ಯೋಗೇ-ಶ್ವ-ರನೆ
ಯೋಗೇ-ಶ್ವ-ರಾತ್
ಯೋಗೈ-ಶ್ವರ್ಯ
ಯೋಗೋ
ಯೋಗೋ-ಽನಿ-ರ್ವಿ-ಣ್ಣ-ಚೇ-ತಸಾ
ಯೋಗೋಽಸ್ತಿ
ಯೋಗ್ಯ
ಯೋಗ್ಯ-ತಾನು
ಯೋಗ್ಯ-ತಾ-ನು-ಸಾರ
ಯೋಗ್ಯತೆ
ಯೋಗ್ಯ-ತೆ-ಗಳನ್ನು
ಯೋಗ್ಯ-ತೆ-ಗ-ಳ-ಲ್ಲೆಲ್ಲಾ
ಯೋಗ್ಯ-ತೆ-ಗ-ಳಿರ
ಯೋಗ್ಯ-ತೆ-ಗಳು
ಯೋಗ್ಯ-ತೆ-ಗ-ಳೇ-ನೇನು
ಯೋಗ್ಯ-ತೆ-ಗಿಂತ
ಯೋಗ್ಯ-ತೆಗೆ
ಯೋಗ್ಯ-ತೆಯ
ಯೋಗ್ಯ-ತೆ-ಯನ್ನು
ಯೋಗ್ಯ-ತೆಯೂ
ಯೋಗ್ಯ-ತೆಯೇ
ಯೋಗ್ಯ-ನಲ್ಲ
ಯೋಗ್ಯ-ನ-ಲ್ಲದೇ
ಯೋಗ್ಯ-ನಾ-ಗಿ-ರ-ಬೇಕು
ಯೋಗ್ಯ-ನಾ-ಗಿ-ರು-ತ್ತಾನೆ
ಯೋಗ್ಯ-ನಾ-ಗಿ-ರು-ವನು
ಯೋಗ್ಯ-ನಾ-ಗು-ತ್ತಾನೆ
ಯೋಗ್ಯ-ನಾದ
ಯೋಗ್ಯನೇ
ಯೋಗ್ಯನೊ
ಯೋಗ್ಯರ
ಯೋಗ್ಯ-ರಲ್ಲ
ಯೋಗ್ಯ-ರಾದ
ಯೋಗ್ಯ-ರಾ-ದರು
ಯೋಗ್ಯ-ರಾ-ದ-ವ-ರಿಗೆ
ಯೋಗ್ಯ-ರಿಗೆ
ಯೋಗ್ಯರು
ಯೋಗ್ಯರೋ
ಯೋಗ್ಯ-ವಲ್ಲ
ಯೋಗ್ಯ-ವಾಗಿ
ಯೋಗ್ಯ-ವಾ-ಗಿ-ದ್ದರೆ
ಯೋಗ್ಯ-ವಾದ
ಯೋಗ್ಯ-ವಾ-ದು-ದನ್ನು
ಯೋಗ್ಯ-ವಾ-ದುದು
ಯೋಗ್ಯ-ವ್ಯ-ಕ್ತಿ-ಗ-ಳಿಗೆ
ಯೋಗ್ಯ-ಸಂಸ್ಥೆ
ಯೋಗ್ಯಾ-ತಾ-ನು-ಸಾರ
ಯೋಚನೆ
ಯೋಚ-ನೆ-ಗಳು
ಯೋಚ-ನೆ-ಯನ್ನು
ಯೋಚಿಸ
ಯೋಚಿ-ಸ-ಬ-ಹುದು
ಯೋಚಿ-ಸ-ಬೇ-ಕಾ-ದರೆ
ಯೋಚಿ-ಸ-ಬೇಕು
ಯೋಚಿಸಿ
ಯೋಚಿ-ಸಿ-ದಾಗ
ಯೋಚಿ-ಸಿಲ್ಲ
ಯೋಚಿಸು
ಯೋಚಿ-ಸು-ತ್ತಿ-ರು-ವಾಗ
ಯೋಚಿ-ಸು-ತ್ತಿ-ರು-ವಾ-ಗಲೆ
ಯೋಚಿ-ಸು-ತ್ತಿ-ರು-ವು-ದರ
ಯೋಚಿ-ಸು-ತ್ತೇವೆ
ಯೋಚಿ-ಸು-ವನು
ಯೋಚಿ-ಸು-ವರು
ಯೋಚಿ-ಸುವು
ಯೋಚಿ-ಸು-ವು-ದಿಲ್ಲ
ಯೋಚಿ-ಸು-ವುದು
ಯೋಚಿ-ಸು-ವುದೇ
ಯೋಜನೆ
ಯೋತ್ಸ್ಯ
ಯೋತ್ಸ್ಯ-ಮಾ-ನಾ-ನ-ವೇ-ಕ್ಷೇಹಂ
ಯೋದ್ಧ-ವ್ಯ-ಮ-ಸ್ಮಿನ್
ಯೋದ್ಧು-ಕಾ-ಮಾ-ನ-ವ-ಸ್ಥಿ-ತಾನ್
ಯೋಧ
ಯೋಧ-ನಾದ
ಯೋಧ-ಮು-ಖ್ಯೈಃ
ಯೋಧ-ರಿಂದ
ಯೋಧ-ರೆ-ಲ್ಲರೂ
ಯೋಧ-ವೀ-ರಾನ್
ಯೋಧಾಃ
ಯೋನಿ
ಯೋನಿ-ಮಾ-ಪನ್ನಾ
ಯೋನಿ-ಯನ್ನು
ಯೋನಿ-ಯಲ್ಲಿ
ಯೋನಿ-ರ್ಮ-ಹ-ದ್ಬ್ರಹ್ಮ
ಯೋನಿಷು
ಯೋಽಂತಃ-ಸು-ಖೋ-ಽಂತ-ರಾ-ಮಾ-ಸ್ತ-ಥಾ-ರ್ಂ-ಜ್ಯೋ-ತಿ-ರೇವ
ಯೋಽಜುನ
ಯೋಽಭಿ-ಜಾ-ನಾತಿ
ಯೋಽಭ್ಯ-ಸೂ-ಯತಿ
ಯೋಽಯಂ
ಯೋಽವ-ತಿ-ಷ್ಠತಿ
ಯೋಽವ್ಯ-ಭಿ-ಚಾ-ರೇಣ
ಯೌವನ
ಯೌವನಂ
ಯೌವ-ನಕ್ಕೆ
ಯೌವ-ನ-ದಲ್ಲಿ
ಯೌವ-ನ-ದ-ಲ್ಲಿಯೆ
ಯೌವ-ನ-ವನ್ನು
ರಂಗ
ರಂಗ-ದಲ್ಲಿ
ರಂಗ-ಭೂಮಿ
ರಂಗ-ಭೂ-ಮಿಗೆ
ರಂಗ-ಭೂ-ಮಿಯ
ರಂಗೋಲಿ
ರಂಗೋ-ಲಿಯ
ರಂತಿ-ದ್ದ-ವರು
ರಂತೆಯೇ
ರಂಧ್ರ
ರಂಧ್ರ-ಗ-ಳಂತೆ
ರಂಧ್ರ-ಗಳನ್ನೂ
ರಂಧ್ರ-ಗಳಿಂದ
ರಂಧ್ರ-ಗ-ಳಿಂ-ದಲೂ
ರಂಧ್ರ-ಗಳು
ರಂಧ್ರದ
ರಂಧ್ರ-ದಂ-ತಿದೆ
ರಂಧ್ರ-ದಲ್ಲಿ
ರಂಧ್ರ-ದಲ್ಲೆಲ್ಲಾ
ರಂಧ್ರ-ದಿಂದ
ರಂಧ್ರ-ವನ್ನು
ರಕ್ತ
ರಕ್ತ-ಚ-ಲ-ನೆ-ಯಾ-ಗು-ವುದು
ರಕ್ತದ
ರಕ್ತ-ದಿಂದ
ರಕ್ತ-ಮಾಂಸ
ರಕ್ತ-ವನ್ನು
ರಕ್ಷಕ
ರಕ್ಷಣೆ
ರಕ್ಷ-ಣೆ-ಗಾಗಿ
ರಕ್ಷ-ಣೆಗೆ
ರಕ್ಷ-ಣೆ-ಮಾ-ಡು-ವು-ದಕ್ಕೆ
ರಕ್ಷ-ಣೆಯ
ರಕ್ಷ-ಣೆಯೂ
ರಕ್ಷ-ಣೆಯೇ
ರಕ್ಷಾಂಸಿ
ರಕ್ಷಿ-ತ-ವಾದ
ರಕ್ಷಿ-ಸ-ಬ-ಹುದು
ರಕ್ಷಿ-ಸ-ಬೇ-ಕಾ-ಗಿದೆ
ರಕ್ಷಿ-ಸ-ಬೇ-ಕಾ-ದರೆ
ರಕ್ಷಿ-ಸ-ಬೇಕು
ರಕ್ಷಿ-ಸಲು
ರಕ್ಷಿಸಿ
ರಕ್ಷಿ-ಸಿ-ಕೊಂ-ಡಿ-ದ್ದೇವೋ
ರಕ್ಷಿ-ಸಿ-ಕೊಂ-ಡಿ-ರ-ಬೇಕು
ರಕ್ಷಿ-ಸಿ-ಕೊಂಡು
ರಕ್ಷಿ-ಸಿ-ಕೊ-ಳ್ಳು-ವು-ದಕ್ಕೆ
ರಕ್ಷಿ-ಸಿ-ದರೂ
ರಕ್ಷಿಸು
ರಕ್ಷಿ-ಸು-ತ್ತಾನೆ
ರಕ್ಷಿ-ಸು-ತ್ತಿ-ದ್ದರು
ರಕ್ಷಿ-ಸು-ತ್ತಿ-ದ್ದರೊ
ರಕ್ಷಿ-ಸು-ತ್ತಿ-ರುವ
ರಕ್ಷಿ-ಸುವ
ರಕ್ಷಿ-ಸು-ವನು
ರಕ್ಷಿ-ಸು-ವ-ನು-ಮುಂ-ತಾದ
ರಕ್ಷಿ-ಸು-ವ-ವನು
ರಕ್ಷಿ-ಸು-ವಾಗ
ರಕ್ಷಿ-ಸು-ವು-ದ-ಕ್ಕಾ-ಗಲಿ
ರಕ್ಷಿ-ಸು-ವು-ದ-ಕ್ಕಾ-ಗಿಯೆ
ರಕ್ಷಿ-ಸು-ವು-ದಕ್ಕೆ
ರಕ್ಷಿ-ಸು-ವುದು
ರಕ್ಷಿ-ಸು-ವೆವು
ರಕ್ಷೆ
ರಕ್ಷೆ-ಯನ್ನು
ರಚಿ-ಸ-ಲ್ಪಟ್ಟ
ರಚಿ-ಸ-ಲ್ಪ-ಟ್ಟಿವೆ
ರಚಿ-ಸಿದ
ರಚಿ-ಸಿದ್ದ
ರಚಿ-ಸಿ-ರುವ
ರಚಿ-ಸು-ವನು
ರಜ
ರಜಃ
ರಜ-ವನ್ನು
ರಜವೇ
ರಜ-ಸಸ್ತು
ರಜಸಿ
ರಜಸೋ
ರಜಸ್
ರಜ-ಸ್ತಥಾ
ರಜ-ಸ್ತಮ
ರಜ-ಸ್ತಮಃ
ರಜ-ಸ್ತ-ಮ-ಶ್ಚಾ-ಭಿ-ಭೂಯ
ರಜ-ಸ್ಯೇ-ತಾನಿ
ರಜ-ಸ್ಸಿ-ಗಿಂತ
ರಜ-ಸ್ಸಿಗೆ
ರಜಸ್ಸು
ರಜಾ
ರಜೊ-ಗುಣ
ರಜೊ-ಗು-ಣದ
ರಜೋ
ರಜೋ-ಗುಣ
ರಜೋ-ಗು-ಣ-ಕ್ಕಿಂತ
ರಜೋ-ಗು-ಣಕ್ಕೆ
ರಜೋ-ಗು-ಣಕ್ಕೋ
ರಜೋ-ಗು-ಣ-ಗಳನ್ನು
ರಜೋ-ಗು-ಣ-ಗಳು
ರಜೋ-ಗು-ಣದ
ರಜೋ-ಗು-ಣ-ದಲ್ಲಿ
ರಜೋ-ಗು-ಣ-ದ-ಲ್ಲಿ-ರು-ವ-ವ-ರಿಗೆ
ರಜೋ-ಗು-ಣ-ದಿಂದ
ರಜೋ-ಗು-ಣ-ದಿಂ-ದ-ಲಾ-ದರೂ
ರಜೋ-ಗು-ಣ-ವನ್ನು
ರಜೋ-ಗು-ಣ-ವಿದೆ
ರಜೋ-ಗು-ಣ-ವಿ-ದ್ದರೆ
ರಜೋ-ಗು-ಣ-ವಿಲ್ಲ
ರಜೋ-ಗು-ಣ-ವಿ-ಲ್ಲ-ದ-ವನೂ
ರಜೋ-ಗು-ಣವೇ
ರಜೋ-ಗು-ಣ-ಸ-ಮು-ದ್ಭವಃ
ರಜೋ-ಗುಣಿ
ರಜೋ-ಗು-ಣಿ-ಗಳಲ್ಲಿ
ರಜೋ-ಗು-ಣಿ-ಗಳು
ರಜೋ-ಗು-ಣಿಯ
ರಜೋ-ಗು-ಣಿ-ಯಾ-ದರೂ
ರಜೋ-ಧೈರ್ಯ
ರಜೋ-ಪ್ರ-ವೃತ್ತಿ
ರಟ್ಟಾ-ಗಿದೆ
ರಟ್ಟಾ-ಗು-ವುದು
ರಣ
ರಣ-ದಲ್ಲಿ
ರಣ-ದೇ-ವ-ತೆಗೆ
ರಣ-ದೇ-ವಿಗೆ
ರಣ-ನದಿ
ರಣ-ನ-ದಿ-ಯ-ನ್ನಾಗಿ
ರಣ-ನ-ದಿ-ಯನ್ನು
ರಣ-ನ-ದಿ-ಯಲ್ಲಿ
ರಣ-ನದೀ
ರಣ-ರಂ-ಗ-ದ-ಲ್ಲಾ-ಗಲೀ
ರಣ-ಶಬ್ದ
ರಣ-ಸ-ಮು-ದ್ಯಮೇ
ರಣ-ಹ-ದ್ದಿ-ನಂತೆ
ರಣ-ಹದ್ದು
ರಣೇ
ರಣೋ-ತ್ಸಾ-ಹ-ವನ್ನು
ರತಾಃ
ರತ್ನ
ರತ್ನ-ಗಳನ್ನು
ರತ್ನ-ಗಳು
ರತ್ನ-ದಂತೆ
ರತ್ನ-ವನ್ನೆ
ರತ್ನ-ವಿದೆ
ರಥಂ
ರಥದ
ರಥ-ದಲ್ಲಿ
ರಥ-ವನ್ನು
ರಥವೇ
ರಥೋ-ತ್ತ-ಮಮ್
ರಥೋ-ಪಸ್ಥ
ರನ್ನು
ರನ್ನೂ
ರಬ್ಬ-ರಿಗೂ
ರಬ್ಬ-ರಿ-ನಂತೆ
ರಭ-ಸ-ದಿಂದ
ರಮಂತಿ
ರಮತೇ
ರಮಿ-ಸು-ತ್ತಾ-ನೆಯೋ
ರಮಿ-ಸು-ವು-ದಿಲ್ಲ
ರಲಿ
ರಲ್ಲಿ
ರಲ್ಲಿತ್ತು
ರಲ್ಲಿಯೂ
ರವಾ-ನಿ-ಸ-ಬೇ-ಕಾ-ಗಿದೆ
ರವಾ-ನಿಸು
ರವಾ-ನಿ-ಸು-ವುದು
ರವಿ
ರವಿಃ
ರವಿಯ
ರವಿ-ಯಾ-ದರೋ
ರವಿ-ರಂ-ಶು-ಮಾನ್
ರಶ್ಮಿ
ರಸ
ರಸಕ್ಕೂ
ರಸ-ಗಳನ್ನೂ
ರಸ-ಗಳನ್ನೆಲ್ಲ
ರಸ-ಗ-ಳೊಂ-ದಿಗೆ
ರಸ-ದಂತೆ
ರಸ-ದಲ್ಲಿ
ರಸ-ದ-ಲ್ಲಿ-ರು-ವು-ದೆಲ್ಲಾ
ರಸನಂ
ರಸ-ನೇಂ-ದ್ರಿ-ಯ-ಗಳು
ರಸ-ಯುಕ್ತ
ರಸ-ಯು-ಕ್ತ-ವಾದ
ರಸ-ವ-ತ್ತಾದ
ರಸ-ವನ್ನು
ರಸ-ವರ್ಜಂ
ರಸವು
ರಸವೂ
ರಸ-ಸ್ವ-ರೂ-ಪದ
ರಸ-ಹೀ-ನ-ವಾ-ಗಿ-ರು-ವುದು
ರಸಾ-ತ್ಮಕಃ
ರಸಾ-ಯ-ನ-ಶಾಸ್ತ್ರ
ರಸಾ-ಯ-ನಿಕ
ರಸಾ-ಸ್ವಾ-ದನೆ
ರಸೋ-ಽಪ್ಯಸ್ಯ
ರಸೋ-ಽಹ-ಮಪ್ಸು
ರಸ್ತೆ-ಮುಂ-ತಾ-ದು-ವು-ಗಳನ್ನು
ರಸ್ತೆಯ
ರಸ್ತೆ-ಯಲ್ಲಿ
ರಸ್ಯಾಃ
ರಹಸಿ
ರಹಸ್ಯ
ರಹಸ್ಯಂ
ರಹ-ಸ್ಯ-ಕ್ಕೆಲ್ಲ
ರಹ-ಸ್ಯ-ಗಳನ್ನು
ರಹ-ಸ್ಯದ
ರಹ-ಸ್ಯ-ವನ್ನು
ರಹ-ಸ್ಯ-ವನ್ನೂ
ರಹ-ಸ್ಯ-ವ-ನ್ನೆಲ್ಲಾ
ರಹ-ಸ್ಯ-ವಾಗಿ
ರಹ-ಸ್ಯ-ವಾ-ಗಿ-ರುವ
ರಹ-ಸ್ಯ-ವಾದ
ರಹ-ಸ್ಯ-ವೆಂದರೆ
ರಹ-ಸ್ಯ-ವೆಲ್ಲಾ
ರಹ-ಸ್ಯವೇ
ರಹ-ಸ್ಯ-ವೇನು
ರಹ-ಸ್ಯಾ-ತಿ-ರ-ಹ-ಸ್ಯ-ವಾದ
ರಹಿತ
ರಹಿ-ತ-ನಾಗಿ
ರಹಿ-ತನು
ರಹಿ-ತನೂ
ರಹಿ-ತ-ವಾ-ಗಿದೆ
ರಾಕ್ಷಸ
ರಾಕ್ಷ-ಸ-ಕು-ಲ-ದಲ್ಲಿ
ರಾಕ್ಷ-ಸನ
ರಾಕ್ಷ-ಸ-ನನ್ನು
ರಾಕ್ಷ-ಸ-ನಿಗೆ
ರಾಕ್ಷ-ಸರ
ರಾಕ್ಷ-ಸ-ರನ್ನು
ರಾಕ್ಷ-ಸ-ರಲ್ಲಿ
ರಾಕ್ಷ-ಸರು
ರಾಕ್ಷ-ಸಿ-ಯ-ರಿಗೆ
ರಾಕ್ಷ-ಸಿ-ಯರು
ರಾಕ್ಷಸೀ
ರಾಕ್ಷ-ಸೀ-ಮಾ-ಸು-ರೀಂ
ರಾಕ್ಷ-ಸೀಯ
ರಾಗ
ರಾಗ-ದಿಂದ
ರಾಗ-ದ್ವೇ-ಷ-ಗಳಿಂದ
ರಾಗ-ದ್ವೇ-ಷ-ಗಳು
ರಾಗ-ದ್ವೇ-ಷ-ವಿ-ಯು-ಕ್ತೈಸ್ತು
ರಾಗ-ದ್ವೇಷೌ
ರಾಗ-ವಾಗಿ
ರಾಗ-ವು-ಳ್ಳ-ವನೂ
ರಾಗವೇ
ರಾಗ-ವೇನೂ
ರಾಗಾ-ತ್ಮಕಂ
ರಾಗಿ
ರಾಗಿ-ರ-ಬ-ಹುದು
ರಾಗಿ-ರ-ಬೇಕು
ರಾಗಿ-ರು-ವ-ವ-ರೆಷ್ಟೊ
ರಾಗೀ
ರಾಗು-ವರು
ರಾಜ
ರಾಜ-ಕೀಯ
ರಾಜ-ಗುಹ್ಯಂ
ರಾಜ-ಗು-ಹ್ಯ-ಯೋಗ
ರಾಜ-ಧಾ-ನಿಗೆ
ರಾಜನ
ರಾಜ-ನನ್ನು
ರಾಜ-ನಾ-ಗ-ಲಿಲ್ಲ
ರಾಜ-ನಾಗಿ
ರಾಜ-ನಾ-ಗಿದ್ದ
ರಾಜ-ನಾ-ಗಿ-ರು-ವನು
ರಾಜ-ನಾ-ಗು-ವನು
ರಾಜ-ನಾದ
ರಾಜ-ನಾ-ದರೆ
ರಾಜ-ನಿಗೆ
ರಾಜ-ನಿ-ರ್ಮಾ-ಪ-ಕ-ನಾ-ಗು-ವನು
ರಾಜನೆ
ರಾಜನೇ
ರಾಜನ್
ರಾಜ-ಪು-ರೋ-ಹಿ-ತ-ರಲ್ಲಿ
ರಾಜ-ಪು-ಷಿ-ಗಳು
ರಾಜ-ಪು-ಷಿ-ಯಾ-ದರೋ
ರಾಜ-ಯೋ-ಗ-ಗ-ಳೆಂಬ
ರಾಜ-ಯೋ-ಗ-ಸೂ-ತ್ರ-ದಲ್ಲಿ
ರಾಜರ
ರಾಜ-ರನ್ನು
ರಾಜರು
ರಾಜ-ರು-ಗ-ಳಾ-ಗಲೀ
ರಾಜ-ರು-ಗ-ಳಿಗೆ
ರಾಜ-ರೆ-ದು-ರಿಗೆ
ರಾಜ-ರ್ಷ-ಯ-ಸ್ತಥಾ
ರಾಜ-ರ್ಷಯೋ
ರಾಜ-ರ್ಷಿ-ಗಳು
ರಾಜ-ವಿದ್ಯಾ
ರಾಜ-ವಿದ್ಯೆ
ರಾಜಸ
ರಾಜಸಂ
ರಾಜಸಃ
ರಾಜ-ಸ-ಮೂ-ಹ-ಗಳಿಂದ
ರಾಜ-ಸಮ್
ರಾಜ-ಸ-ಸ್ಯೇಷ್ಟಾ
ರಾಜ-ಸಾಃ
ರಾಜ-ಸಾ-ಸ್ತಾ-ಮ-ಸಾಶ್ಚ
ರಾಜ-ಸಿಕ
ರಾಜ-ಸಿ-ಕ-ದಾನಿ
ರಾಜ-ಸಿ-ಕನ
ರಾಜ-ಸಿ-ಕ-ನ-ಲ್ಲಿಯೂ
ರಾಜ-ಸಿ-ಕ-ನಿಗೆ
ರಾಜ-ಸಿ-ಕನು
ರಾಜ-ಸಿ-ಕರು
ರಾಜ-ಸಿ-ಕ-ವಾ-ಗಿಯೂ
ರಾಜ-ಸಿ-ಕವೇ
ರಾಜ-ಸಿ-ಕವೊ
ರಾಜ-ಸಿ-ಕವೋ
ರಾಜಸೀ
ರಾಜ-ಸೂಯ
ರಾಜ-ಸೂ-ಯ-ಯಾ-ಗ-ದಲ್ಲಿ
ರಾಜಾ
ರಾಜಾ-ಧಿ-ರಾ-ಜ-ನಾದ
ರಾಜಾ-ಧಿ-ರಾ-ಜರು
ರಾಜಾ-ಸ್ಥಾ-ನಕ್ಕೆ
ರಾಜಿ
ರಾಜಿ-ಸು-ತ್ತಿ-ರುವ
ರಾಜ್ಯ
ರಾಜ್ಯಂ
ರಾಜ್ಯ-ಗಳನ್ನು
ರಾಜ್ಯ-ಗಳು
ರಾಜ್ಯದ
ರಾಜ್ಯ-ದಾ-ಸೆಗೆ
ರಾಜ್ಯ-ದಿಂ-ದೇನು
ರಾಜ್ಯ-ಭೋಗ
ರಾಜ್ಯ-ಭೋ-ಗ-ಗ-ಳಿಂ-ದಲೂ
ರಾಜ್ಯ-ಲಾಭ
ರಾಜ್ಯ-ವನ್ನು
ರಾಜ್ಯ-ವಲ್ಲ
ರಾಜ್ಯ-ವಿದೆ
ರಾಜ್ಯೇನ
ರಾಣಿ-ಜೇನು
ರಾತ್ರ-ವಿ-ದರು
ರಾತ್ರಿ
ರಾತ್ರಿಂ
ರಾತ್ರಿ-ಗ-ಳಂತೆ
ರಾತ್ರಿ-ಗ-ಳಾ-ಗ-ಲೆಂದು
ರಾತ್ರಿ-ಗ-ಳಾ-ಗು-ವುದು
ರಾತ್ರಿ-ಯಂತೆ
ರಾತ್ರಿ-ಯಲ್ಲಿ
ರಾತ್ರಿ-ಯಾ-ಗಿ-ರು-ವುದು
ರಾತ್ರಿ-ಯಾದ
ರಾತ್ರಿ-ಯಾ-ದ-ಮೇಲೆ
ರಾತ್ರಿ-ಯಾ-ದೊ-ಡನೆ
ರಾತ್ರಿಯೊ
ರಾತ್ರಿ-ಸ್ತಥಾ
ರಾತ್ರ್ಯಾ-ಗಮೇ
ರಾತ್ರ್ಯಾ-ಗ-ಮೇ-ಽವಶಃ
ರಾದ
ರಾದ-ವ-ರನ್ನು
ರಾದ-ವರೆ
ರಾದು-ದ-ರಿಂದ
ರಾಧ-ಕೃಷ್ಣ
ರಾಧಾ-ಕೃಷ್ಣ
ರಾಮ
ರಾಮಃ
ರಾಮ-ಕೃ-ಷ್ಣರ
ರಾಮ-ಕೃ-ಷ್ಣರು
ರಾಮನ
ರಾಮ-ನನ್ನು
ರಾಮ-ನಿಗೆ
ರಾಮನೆ
ರಾಮಾ-ನು-ಜಾ-ಚಾ-ರ್ಯರು
ರಾಮಾ-ಯಣ
ರಾಮಾ-ಯ-ಣ-ದಲ್ಲಿ
ರಾಮಾ-ವ-ತಾ-ರ-ದಲ್ಲಿ
ರಾಮ್
ರಾರ್ಥ-ವನ್ನು
ರಾವಣ
ರಾವ-ಣನ
ರಾಶಿ
ರಾಶಿ-ಯ-ನ್ನೆಲ್ಲ
ರಾಶಿ-ಯಲ್ಲಿ
ರಾಶಿ-ಯಾಗಿ
ರಾಶಿ-ಯಿಂದ
ರಾಶಿಯೇ
ರಾಶಿ-ರಾಶಿ
ರಾಷ್ಟ್ರ-ವನ್ನು
ರಿಂದ
ರಿಂದಲೂ
ರಿಂದಲೇ
ರಿಕಾ-ರ್ಡನ್ನು
ರಿಕೆಯ
ರಿಗೂ
ರಿಗೆ
ರಿಪು-ರಾ-ತ್ಮನಃ
ರಿಪೇರಿ
ರಿಪೇ-ರಿಗೆ
ರಿಪೇ-ರಿಗೊ
ರಿಪೇ-ರಿ-ಮಾ-ಡು-ತ್ತೇ-ವೆಯೋ
ರೀತಿ
ರೀತಿ-ಗ-ಳಿವೆ
ರೀತಿಗೂ
ರೀತಿಯ
ರೀತಿ-ಯಲ್ಲಿ
ರೀತಿ-ಯ-ಲ್ಲಿ-ಇ-ವ-ನನ್ನು
ರೀತಿ-ಯ-ಲ್ಲಿ-ದ್ದಾನೆ
ರೀತಿ-ಯ-ಲ್ಲಿಯೇ
ರೀತಿ-ಯ-ಲ್ಲಿ-ರ-ಬೇಕು
ರೀತಿ-ಯಲ್ಲೊ
ರೀತಿ-ಯಾಗಿ
ರೀತಿ-ಯಾ-ಗಿದೆ
ರೀತಿ-ಯಿಂದ
ರೀತಿ-ಯಿಂ-ದ-ಲಾ-ದರೂ
ರೀತಿ-ಯಿಂ-ದಲೂ
ರೀತಿಯೇ
ರೀತೆಯೇ
ರೀಪೇರಿ
ರುಂಡ-ಮಾಲಿ
ರುಂಡ-ಮಾ-ಲೆಯ
ರುಚಿ
ರುಚಿ-ಕರ
ರುಚಿ-ಕ-ರ-ವಾ-ಗಿ-ರ-ಬೇಕು
ರುಚಿ-ಕ-ರ-ವಾ-ಗಿ-ರುವ
ರುಚಿ-ಕ-ರ-ವಾದ
ರುಚಿ-ಕ-ರ-ವಾ-ದು-ದನ್ನು
ರುಚಿಗೆ
ರುಚಿ-ನೋ-ಡದ
ರುಚಿ-ನೋ-ಡು-ವುದು
ರುಚಿಯ
ರುಚಿ-ಯನ್ನು
ರುಚಿ-ಯಾ-ಗಲಿ
ರುಚಿ-ಯಾ-ಗಿದೆ
ರುಚಿ-ಯಾ-ಗಿ-ದ್ದರೇ
ರುಚಿ-ಯಾ-ಗಿ-ರುವ
ರುಚಿ-ಯಾ-ಗಿ-ಲ್ಲದೆ
ರುಚಿ-ಯಾ-ಸೆಗೆ
ರುಚಿ-ಯೊಂದೇ
ರುಚಿ-ಸ-ಬ-ಲ್ಲುದು
ರುಚಿ-ಸು-ವು-ದಿಲ್ಲ
ರುಜಿನ
ರುಜಿ-ನ-ಗ-ಳಿಗೆ
ರುಜಿ-ನ-ಗಳು
ರುತ್ತವೆ
ರುದ್ಧ್ವಾ
ರುದ್ರ
ರುದ್ರನ
ರುದ್ರ-ರನ್ನು
ರುದ್ರ-ರಲ್ಲಿ
ರುದ್ರರು
ರುದ್ರ-ರೂ-ಪ-ವನ್ನು
ರುದ್ರ-ಶಕ್ತಿ
ರುದ್ರಾಕ್ಷಿ
ರುದ್ರಾ-ಣಾಂ
ರುದ್ರಾ-ದಿತ್ಯಾ
ರುದ್ರಾ-ನ-ಶ್ವಿನೌ
ರುದ್ರಾ-ವ-ತಾ-ರವೆ
ರುಧಿ-ರ-ಪ್ರ-ದಿ-ಗ್ಧಾನ್
ರುಬ್ಬು-ತ್ತಿದೆ
ರುವ
ರುವನು
ರುವರು
ರುವುದನ್ನು
ರುವುದು
ರುವೆ
ರೂಢಿ
ರೂಢಿ-ಯಾ-ಗಿದೆ
ರೂಢಿ-ಸ-ಬ-ಹುದು
ರೂಢಿ-ಸ-ಬೇಕು
ರೂಢಿ-ಸಿ-ಕೊಂ-ಡಿ-ರ-ಬೇಕು
ರೂಢಿ-ಸಿ-ಕೊಂ-ಡಿ-ರು-ವರು
ರೂಢಿ-ಸಿ-ಕೊ-ಳ್ಳ-ಬೇ-ಕಾಗಿ
ರೂಢಿ-ಸಿ-ಕೊ-ಳ್ಳ-ಬೇ-ಕಾ-ಗಿದೆ
ರೂಢಿ-ಸಿ-ಕೊ-ಳ್ಳ-ಬೇ-ಕಾದ
ರೂಢಿ-ಸಿ-ಕೊ-ಳ್ಳ-ಬೇಕು
ರೂಢಿ-ಸಿ-ಕೊ-ಳ್ಳು-ತ್ತಾನೆ
ರೂಢಿ-ಸಿ-ಕೊ-ಳ್ಳುವ
ರೂಢಿ-ಸಿ-ಕೊ-ಳ್ಳು-ವಾಗ
ರೂಢಿ-ಸಿ-ಕೊ-ಳ್ಳು-ವು-ದ-ಕ್ಕಾಗಿ
ರೂಢಿ-ಸಿ-ಕೊ-ಳ್ಳು-ವು-ದಕ್ಕೆ
ರೂಢಿ-ಸುವ
ರೂಢಿ-ಸು-ವನು
ರೂಢಿ-ಸು-ವನೊ
ರೂಢಿ-ಸು-ವು-ದಕ್ಕೆ
ರೂಢಿ-ಸು-ವುದು
ರೂಪ
ರೂಪಂ
ರೂಪಕ್ಕೆ
ರೂಪ-ಗಳ
ರೂಪ-ಗಳನ್ನು
ರೂಪ-ಗ-ಳಾ-ದರೂ
ರೂಪ-ಗಳಿಂದ
ರೂಪ-ಗ-ಳಿವೆ
ರೂಪ-ಗಳು
ರೂಪ-ಗ-ಳೆಲ್ಲ
ರೂಪ-ಗಳೋ
ರೂಪದ
ರೂಪ-ದಂತೆ
ರೂಪ-ದಲ್ಲಿ
ರೂಪ-ದ-ಲ್ಲಿತ್ತು
ರೂಪ-ದ-ಲ್ಲಿದೆ
ರೂಪ-ದ-ಲ್ಲಿಯೂ
ರೂಪ-ದ-ಲ್ಲಿಯೇ
ರೂಪ-ದ-ಲ್ಲಿ-ರುವ
ರೂಪ-ದಿಂದ
ರೂಪ-ದಿಂ-ದಲೋ
ರೂಪನು
ರೂಪನ್ನು
ರೂಪ-ಮ-ತ್ಯ-ದ್ಭುತಂ
ರೂಪ-ಮ-ಸ್ಯೇಹ
ರೂಪ-ಮಿದಂ
ರೂಪ-ಮುಗ್ರಂ
ರೂಪ-ಮೈ-ಶ್ವರಂ
ರೂಪ-ಮೈ-ಶ್ವ-ರಮ್
ರೂಪ-ವನ್ನು
ರೂಪ-ವನ್ನೇ
ರೂಪ-ವಲ್ಲ
ರೂಪ-ವಾದ
ರೂಪ-ವಾ-ದರೊ
ರೂಪ-ವುಳ್ಳ
ರೂಪವೂ
ರೂಪಸ್ಯ
ರೂಪಾಂ-ತರ
ರೂಪಾಣಿ
ರೂಪಾ-ತೀತ
ರೂಪಾಯಿ
ರೂಪಾ-ಯಿ-ಗಳ
ರೂಪಾ-ಯಿ-ಗಳು
ರೂಪಾ-ಯಿ-ಯನ್ನು
ರೂಪಿ-ತ-ವಾ-ದಂತೆ
ರೂಪಿ-ನಂತೆ
ರೂಪಿ-ನಲ್ಲಿ
ರೂಪಿ-ನ-ಲ್ಲಿಯೂ
ರೂಪಿ-ನಿಂದ
ರೂಪಿ-ನಿಂ-ದಲೇ
ರೂಪಿ-ಯಾ-ಗಿ-ರುವ
ರೂಪಿಸಿ
ರೂಪಿ-ಸಿ-ಕೊಂ-ಡಿ-ದ್ದರೆ
ರೂಪಿ-ಸಿ-ಕೊಳ್ಳ
ರೂಪಿ-ಸಿ-ಕೊ-ಳ್ಳ-ಬ-ಹುದು
ರೂಪಿ-ಸು-ತ್ತಿದೆ
ರೂಪಿ-ಸುವ
ರೂಪಿ-ಸು-ವು-ದಕ್ಕೆ
ರೂಪಿ-ಸು-ವುದು
ರೂಪಿ-ಸು-ವುವು
ರೂಪು
ರೂಪು-ಗಳ
ರೂಪು-ಗಳನ್ನು
ರೂಪು-ಗ-ಳ-ಲ್ಲಿಯೋ
ರೂಪು-ಗ-ಳಿವೆ
ರೂಪು-ಗೊಂ-ಡ-ವನು
ರೂಪು-ತಾ-ಳಿದೆ
ರೂಪೇಣ
ರೂಮಿ-ನಲ್ಲಿ
ರೂಮಿ-ನ-ಲ್ಲಿದೆ
ರೆಂದು
ರೆಂಬೆ
ರೆಂಬೆ-ಕೊಂ-ಬೆ-ಗಳನ್ನು
ರೆಂಬೆ-ಗಳನ್ನು
ರೆಂಬೆ-ಗಳು
ರೆಂಬೆಗೆ
ರೆಕ್ಕೆ
ರೆಕ್ಕೆ-ಗಳನ್ನು
ರೆಕ್ಕೆಗೆ
ರೆಕ್ಕೆಯ
ರೆಕ್ಕೆ-ಯನ್ನು
ರೆನ್ನು-ತ್ತಾರೆ
ರೆಫ-ರೆ-ನ್ಸ್
ರೆಲ್ಲ
ರೆಲ್ಲರೂ
ರೇ
ರೇಕು-ಗಳು
ರೇಗಾ-ಡು-ತ್ತಿ-ದ್ದರೆ
ರೇಗಿ
ರೇಗು-ವೆವು
ರೇಚಕ
ರೇಡಿ-ಯೇ-ಷ-ನ್ನಂತೆ
ರೇಡಿಯೊ
ರೇಡಿಯೋ
ರೇಡಿ-ಯೋ-ವನ್ನು
ರೇಡಿ-ಯೋ-ವನ್ನೊ
ರೇಶ್ಮೆಯ
ರೇಶ್ಮೆ-ಹುಳು
ರೇಷ್ಮೆ
ರೇಷ್ಮೆಯ
ರೇಷ್ಮೆ-ಹುಳು
ರೈತ
ರೈತನ
ರೈಲನ್ನು
ರೈಲಿನ
ರೈಲಿ-ನಲ್ಲಿ
ರೈಲು
ರೈಲೆ
ರೈಲ್ವೆ
ರೈಲ್ವೇ
ರೊಂದಿಕೆ
ರೊಂದಿಗೆ
ರೊಚ್ಚಿ-ಗೆದ್ದು
ರೊಚ್ಚಿ-ದಾಗ
ರೊಚ್ಚಿ-ನ-ಮೇಲೆ
ರೊಟ್ಟಿ
ರೋಗ
ರೋಗಕ್ಕೂ
ರೋಗಕ್ಕೆ
ರೋಗ-ಕ್ಕೆಲ್ಲ
ರೋಗ-ಗ-ಳ-ನ್ನುಂಟು
ರೋಗ-ಗಳಿಂದ
ರೋಗ-ಗಳು
ರೋಗದ
ರೋಗ-ದಂತೆ
ರೋಗ-ದಲ್ಲಿ
ರೋಗ-ದಿಂದ
ರೋಗ-ದೊಂ-ದಿಗೆ
ರೋಗ-ರು-ಜಿನ
ರೋಗ-ರು-ಜಿ-ನ-ಗಳ
ರೋಗ-ರು-ಜಿ-ನ-ಗಳನ್ನು
ರೋಗ-ರು-ಜಿ-ನ-ಗ-ಳಿಗೆ
ರೋಗ-ರು-ಜಿ-ನ-ಗಳು
ರೋಗ-ವನ್ನು
ರೋಗ-ವ-ನ್ನೆಲ್ಲಾ
ರೋಗ-ವಿರು
ರೋಗ-ವಿ-ರು-ವಾಗ
ರೋಗ-ಶೇಷ
ರೋಗಾ-ದಿ-ಗ-ಳೆಲ್ಲ
ರೋಗಿ
ರೋಗಿ-ಗ-ಳಿಗೆ
ರೋಗಿಗೆ
ರೋಗಿಯ
ರೋಗಿ-ಯಂತೆ
ರೋಗಿ-ಯನ್ನು
ರೋಗಿ-ಯಾ-ಗಿ-ರ-ಬ-ಹುದು
ರೋಗಿ-ಯಾ-ದರೊ
ರೋಗಿಯು
ರೋಗಿ-ಯೊ-ಬ್ಬ-ನಿಗೆ
ರೋಮ
ರೋಮ-ಕೂ-ಪ-ದಂತೆ
ರೋಮ-ದಂತೆ
ರೋಮ-ಹ-ರ್ಷ-ಣಮ್
ರೋಮ-ಹ-ರ್ಷಶ್ಚ
ರೋಮಾಂ-ಚ-ಕಾ-ರಿ-ಯಾದ
ರೋಮಾಂ-ಚನ
ರೋಮಾಂ-ಚ-ನ-ವಾ-ಗು-ತ್ತಿದೆ
ರೋಮಾಂ-ಚ-ನ-ವಾ-ಗು-ವು-ದ-ರಲ್ಲಿ
ರೋಮಾಂ-ಚ-ವು-ಳ್ಳ-ವನೂ
ರೋಲಿ-ನಲ್ಲಿ
ರೋಸಿ-ದಾಗ
ರ್ಯಾಂಕ್
ಲಂಕೆಗೆ
ಲಂಗ-ರನ್ನು
ಲಂಗು
ಲಂಘ-ಯತೇ
ಲಂಚ
ಲಂಚ-ವನ್ನು
ಲಂಪ-ಟ-ನಲ್ಲ
ಲಕ್ಷ
ಲಕ್ಷಣ
ಲಕ್ಷ-ಣ-ಗಳನ್ನು
ಲಕ್ಷ-ಣ-ಗಳನ್ನೆಲ್ಲ
ಲಕ್ಷ-ಣ-ಗ-ಳಿ-ರ-ಬೇಕು
ಲಕ್ಷ-ಣ-ಗ-ಳೇನು
ಲಕ್ಷ-ಣ-ದಲ್ಲಿ
ಲಕ್ಷ-ಣ-ವನ್ನು
ಲಕ್ಷ-ಣ-ವಲ್ಲ
ಲಕ್ಷ-ಣ-ವುಳ್ಳ
ಲಕ್ಷ-ಣವೇ
ಲಕ್ಷದ
ಲಕ್ಷಾಂ-ತರ
ಲಕ್ಷಾ-ಧೀ-ಶ್ವ-ರ-ನ-ವ-ರೆಗೆ
ಲಕ್ಷಿ
ಲಕ್ಷಿ-ಸು-ವು-ದಿಲ್ಲ
ಲಕ್ಷೋ-ಪ-ಲಕ್ಷ
ಲಕ್ಷ್ಮಣ
ಲಕ್ಷ್ಮಿ
ಲಕ್ಷ್ಮಿ-ಪೂಜೆ
ಲಕ್ಷ್ಮಿಯ
ಲಕ್ಷ್ಮಿ-ಯಷ್ಟು
ಲಕ್ಷ್ಯ-ದ-ಲ್ಲಿ-ಟ್ಟರೆ
ಲಕ್ಷ್ಯ-ದ-ಲ್ಲಿ-ಡು-ವು-ದಿಲ್ಲ
ಲಗಾಮೂ
ಲಘು-ವಾಗಿ
ಲಘು-ವಾದ
ಲಘ್ವಾಶೀ
ಲಜ್ಜೆ-ಯಿಂದ
ಲಜ್ಜೆ-ಯಿಲ್ಲ
ಲಟು
ಲಬ್ಧ
ಲಬ್ಧ-ಮಿದಂ
ಲಬ್ಧ್ವಾ
ಲಭಂತೇ
ಲಭತೇ
ಲಭಸ್ವ
ಲಭಿ-ಸದು
ಲಭಿ-ಸ-ಬ-ಲ್ಲದು
ಲಭಿ-ಸಲಿ
ಲಭಿ-ಸ-ಲಿಲ್ಲ
ಲಭಿ-ಸಿ-ದರೂ
ಲಭಿ-ಸಿದೆ
ಲಭಿ-ಸು-ತ್ತದೆ
ಲಭಿ-ಸು-ತ್ತವೆ
ಲಭಿ-ಸು-ವು-ದಿಲ್ಲ
ಲಭಿ-ಸು-ವುದು
ಲಭಿ-ಸು-ವುವು
ಲಭೇ
ಲಭೇತ್
ಲಭ್ಯ
ಲಭ್ಯ-ಸ್ತ್ವ-ನ-ನ್ಯಯಾ
ಲಯ-ಗ-ಳಿಗೆ
ಲಯ-ದಲ್ಲಿ
ಲಯ-ವಾಗ
ಲಯ-ವಾ-ಗ-ಬೇಕು
ಲಯ-ವಾಗಿ
ಲಯ-ವಾ-ಗು-ವುದು
ಲಯ-ವೆಲ್ಲ
ಲಲಿತ
ಲವ-ಲ-ವಿ-ಕೆ-ಯಿಂದ
ಲವ-ಲೇ-ಶವೂ
ಲವ್ಕಿವ್
ಲಾಗಲೀ
ಲಾಗು-ವು-ದಿಲ್ಲ
ಲಾಗು-ವುದು
ಲಾಘ-ವಮ್
ಲಾಟೀನ್
ಲಾಭ
ಲಾಭಂ
ಲಾಭ-ಕ್ಕಾ-ಗಲಿ
ಲಾಭ-ಕ್ಕಾಗಿ
ಲಾಭಕ್ಕೂ
ಲಾಭಕ್ಕೆ
ಲಾಭಕ್ಕೋ
ಲಾಭದ
ಲಾಭ-ದಿಂ-ದೇನು
ಲಾಭ-ನ-ಷ್ಟ-ಗಳ
ಲಾಭ-ನ-ಷ್ಟ-ಗಳಲ್ಲಿ
ಲಾಭ-ನ-ಷ್ಟದ
ಲಾಭ-ನ-ಷ್ಟ-ದಲ್ಲಿ
ಲಾಭ-ವನ್ನು
ಲಾಭ-ವಿಲ್ಲ
ಲಾಭ-ವಿ-ಲ್ಲದೇ
ಲಾಭವೂ
ಲಾಭವೇ
ಲಾಭಾ-ಲಾಭೌ
ಲಾಯ-ರ-ನ್ನಿಟ್ಟು
ಲಾಯ-ರನ್ನು
ಲಾಯ-ರಿಗೆ
ಲಾಯ-ರಿನ
ಲಾಯರು
ಲಾಯ-ರು-ಗಳನ್ನು
ಲಾಯ-ರು-ಗಳು
ಲಾಯರ್
ಲಾರದ
ಲಾರ-ದ-ವ-ರಿಗೆ
ಲಾರದು
ಲಾರನೊ
ಲಾರವು
ಲಾರಿಯೇ
ಲಾರೆವು
ಲಾಲಸೆ
ಲಾಲಿ-ಸು-ತ್ತಾನೆ
ಲಾವಾ-ಗ್ರಂ-ಥಿ-ಗಳು
ಲಿಂಗವೂ
ಲಿಂಪಂತಿ
ಲಿಪ್ತ-ನಾ-ಗು-ವು-ದಿಲ್ಲ
ಲಿಪ್ತ-ನಾ-ಗು-ವು-ದಿ-ಲ್ಲವೋ
ಲಿಪ್ತ-ವಾ-ಗಿ-ರು-ವು-ದಿ-ಲ್ಲವೋ
ಲಿಪ್ಯತೇ
ಲಿರುವ
ಲಿಲ್ಲ
ಲಿಲ್ಲ-ವಂತೆ
ಲಿಲ್ಲವೆ
ಲೀನ-ಗೊ-ಳಿ-ಸು-ವನು
ಲೀನ-ಮಾಡಿ
ಲೀನ-ಮಾ-ಡು-ವುದು
ಲೀನ-ವಾ-ಗು-ತ್ತವೆ
ಲೀಲಾ-ನಾ-ಟ-ಕ-ದಲ್ಲಿ
ಲೀಲೆ
ಲೀಲೆ-ಗಳನ್ನು
ಲೀಲೆ-ಯನ್ನು
ಲುಪ್ತ-ಪಿಂ-ಡೋ-ದ-ಕ-ಕ್ರಿ-ಯಾಃ
ಲೂಟಿ
ಲೂಟಿಗೆ
ಲೂಟಿ-ಮಾ-ಡು-ವು-ದಕ್ಕೆ
ಲೆಕ್ಕದ
ಲೆಕ್ಕ-ವನ್ನು
ಲೆಕ್ಕ-ವಿ-ಲ್ಲ-ದಷ್ಟು
ಲೆಕ್ಕಾ-ಚಾರ
ಲೆಕ್ಕಿ-ಸದೆ
ಲೆಕ್ಕಿ-ಸನು
ಲೆಕ್ಕಿ-ಸು-ವುದೇ
ಲೆನ್ಸ್
ಲೇಕ
ಲೇಖ-ನಿಗೆ
ಲೇಖ-ನಿಯ
ಲೇಖ-ನಿ-ಯ-ನ್ನಾಗಿ
ಲೇದು
ಲೇನೋ
ಲೇಪ-ದಿಂದ
ಲೇಪ-ನ-ವಾದ
ಲೇಪ-ವಿ-ರು-ವು-ದಿಲ್ಲ
ಲೇಪ-ವಿ-ಲ್ಲ-ದಿ-ರು-ವಂತೆ
ಲೇಪವೂ
ಲೇಪಿಸಿ
ಲೇಲಿ-ಹ್ಯಸೇ
ಲೇವಾ-ದೇವಿ
ಲೇಸು
ಲೈಫ್
ಲೋಕ
ಲೋಕ-ಕಂ-ಟಕ
ಲೋಕ-ಕಂ-ಟ-ಕ-ರಾ-ಗು-ತ್ತಾರೆ
ಲೋಕ-ಕಂ-ಟ-ಕ-ರಾ-ದರು
ಲೋಕ-ಕಂ-ಟ-ರಾ-ಗು-ವುದನ್ನು
ಲೋಕ-ಕ-ಲ್ಯಾಣ
ಲೋಕ-ಕ-ಲ್ಯಾ-ಣ-ಕ್ಕಾಗಿ
ಲೋಕ-ಕ-ಲ್ಯಾ-ಣಕ್ಕೆ
ಲೋಕಕ್ಕೆ
ಲೋಕ-ಕ್ಷ-ಯ-ಕೃ-ತ್ಪ್ರ-ವೃದ್ಧೋ
ಲೋಕ-ಗಳನ್ನು
ಲೋಕ-ಗಳನ್ನೂ
ಲೋಕ-ಗಳನ್ನೆಲ್ಲ
ಲೋಕ-ಗಳನ್ನೆಲ್ಲಾ
ಲೋಕ-ಗಳಲ್ಲಿ
ಲೋಕ-ಗ-ಳ-ಲ್ಲಿಯೂ
ಲೋಕ-ಗ-ಳ-ಲ್ಲೆಲ್ಲ
ಲೋಕ-ಗ-ಳಿ-ಗಿಂ-ತಲೂ
ಲೋಕ-ಗ-ಳಿಗೆ
ಲೋಕ-ಗ-ಳಿವೆ
ಲೋಕ-ಗಳು
ಲೋಕ-ಗಳೂ
ಲೋಕ-ಗ-ಳೆಲ್ಲ
ಲೋಕ-ಗಳೇ
ಲೋಕ-ಗುರು
ಲೋಕ-ಗು-ರು-ವಾದ
ಲೋಕ-ಗು-ರುವೇ
ಲೋಕ-ತ್ರಯಂ
ಲೋಕ-ತ್ರ-ಯ-ದಲ್ಲಿ
ಲೋಕ-ತ್ರ-ಯ-ಮಾ-ವಿಶ್ಯ
ಲೋಕ-ತ್ರ-ಯೇ-ಽಪ್ಯ-ಪ್ರ-ತಿ-ಮ-ಪ್ರ-ಭಾವ
ಲೋಕದ
ಲೋಕ-ದ-ಲ್ಲಾ-ಗಲಿ
ಲೋಕ-ದಲ್ಲಿ
ಲೋಕ-ದ-ಲ್ಲಿ-ದ್ದರೆ
ಲೋಕ-ದ-ಲ್ಲಿಯೇ
ಲೋಕ-ದಲ್ಲೆ
ಲೋಕ-ದಲ್ಲೊ
ಲೋಕ-ದಿಂ-ದಲೂ
ಲೋಕ-ಪಾ-ಲ-ಕ-ನಾದ
ಲೋಕ-ಮ-ಹೇ-ಶ್ವ-ರಮ್
ಲೋಕ-ಮಿಮಂ
ಲೋಕ-ಲೀ-ಲೆಯ
ಲೋಕ-ವನ್ನು
ಲೋಕ-ವ-ನ್ನೆಲ್ಲ
ಲೋಕ-ವನ್ನೇ
ಲೋಕ-ವಿಲ್ಲ
ಲೋಕವೂ
ಲೋಕ-ವೆಂದರೆ
ಲೋಕವೇ
ಲೋಕವೋ
ಲೋಕ-ವ್ಯ-ವ-ಹಾ-ರದ
ಲೋಕ-ಸಂ-ಗ್ರಹ
ಲೋಕ-ಸಂ-ಗ್ರ-ಹ-ಕ್ಕಾಗಿ
ಲೋಕ-ಸಂ-ಗ್ರ-ಹಕ್ಕೆ
ಲೋಕ-ಸಂ-ಗ್ರ-ಹದ
ಲೋಕ-ಸಂ-ಗ್ರ-ಹ-ಮೇ-ವಾಪಿ
ಲೋಕ-ಸಂ-ಗ್ರ-ಹ-ವಾದ
ಲೋಕ-ಸಂ-ಹಾರ
ಲೋಕ-ಸ್ತ-ದ-ನು-ವ-ರ್ತತೇ
ಲೋಕಸ್ಯ
ಲೋಕ-ಹಾನಿ
ಲೋಕ-ಹಾ-ನಿ-ಕ-ರ-ವಾದ
ಲೋಕ-ಹಾ-ನಿಯೂ
ಲೋಕಾ
ಲೋಕಾಃ
ಲೋಕಾ-ನ-ಮ-ಲಾನ್
ಲೋಕಾ-ನು-ಕಂ-ಪೆ-ಯಿಂದ
ಲೋಕಾ-ನು-ಷಿತ್ವಾ
ಲೋಕಾನ್
ಲೋಕಾ-ನ್ನೋ-ದ್ವಿ-ಜತೇ
ಲೋಕಾ-ಪ-ವಾದ
ಲೋಕೇ
ಲೋಕೇಷು
ಲೋಕೇ-ಸ-ಜ್ಜ-ನ-ಷ-ಟ್ಪ-ದೈ-ರ-ಹ-ರಹಃ
ಲೋಕೇ-ಽಸ್ಮಿನ್
ಲೋಕೈಕ
ಲೋಕೋ
ಲೋಕೋ-ದ್ಧಾ-ರ-ಕ್ಕಾಗಿ
ಲೋಕೋ-ದ್ಧಾ-ರವೇ
ಲೋಕೋ-ಪ-ಕಾ-ರ-ವಾ-ಗು-ವುದು
ಲೋಕೋಽಯಂ
ಲೋಕೋಽಸ್ತಿ
ಲೋಕೋ-ಽಸ್ತ್ಯ-ಯ-ಜ್ಞಸ್ಯ
ಲೋಟ
ಲೋಟ-ಗ-ಟ್ಟಲೆ
ಲೋಪ
ಲೋಪ-ದೋ-ಷ-ಗಳನ್ನೂ
ಲೋಪ-ದೋ-ಷ-ಗಳು
ಲೋಪ-ಬ-ರು-ವುದೋ
ಲೋಪ-ವಾ-ಗು-ವುದು
ಲೋಪ-ವಿಲ್ಲ
ಲೋಭ
ಲೋಭ-ಇವು
ಲೋಭಃ
ಲೋಭ-ದಿಂದ
ಲೋಭ-ಸ್ತ-ಸ್ಮಾ-ದೇ-ತ-ತ್ತ್ರಯಂ
ಲೋಭಿ
ಲೋಭಿಗೆ
ಲೋಭಿಯೂ
ಲೋಭೋ-ಪ-ಹ-ತ-ಚೇ-ತಸಃ
ಲೋಲು-ಪ-ತೆ-ಯೊಂದೇ
ಲೋಹ
ಲೋಹದ
ಲೋಹ-ವನ್ನು
ಲೋಹಾ-ದಿ-ಗಳಲ್ಲಿ
ಲೌಕಿಕ
ಲೌಕಿ-ಕಕ್ಕೆ
ಲೌಕಿ-ಕ-ವ-ಸ್ತು-ವನ್ನೇ
ಲೌಕಿ-ಕ-ವಾ-ಗಲಿ
ಲೌಕಿ-ಕ-ವಾ-ಗಿ-ರ-ಬ-ಹುದು
ಲೌಕಿ-ಕ-ವಾ-ಗಿ-ರಲಿ
ಲೌಕಿ-ಕ-ವಾ-ಗಿ-ರು-ವುದನ್ನು
ಲೌಕಿ-ಕ-ವಾ-ಗಿ-ರು-ವು-ದೇ-ನನ್ನೂ
ಲೌಕಿ-ಕ-ವಾದ
ಲೌಕಿ-ಕ-ವಾ-ದವು
ಲೌಕಿ-ಕ-ವಾ-ದ-ವು-ಗಳನ್ನು
ಲೌಕಿ-ಕ-ವಾ-ದ-ವು-ಗಳನ್ನೂ
ಲೌಕಿ-ಕ-ವಾ-ದು-ದನ್ನು
ಲೌಕಿ-ಕ-ವಾ-ದು-ದನ್ನೂ
ಲೌಕಿ-ಕ-ವಾ-ದು-ದಾ-ವು-ದನ್ನೂ
ಲೌಕಿ-ಕ-ವಾ-ದು-ದೇನೂ
ಲೌಕಿ-ಕ-ವಾ-ದು-ವು-ಗಳು
ಲೌಕಿ-ಕ-ಸಂ-ಪತ್ತು
ಲ್ಪಟ್ಟ
ಲ್ಪಡುತ್ತಾ
ಲ್ಲದ
ಲ್ಲದೆ
ಲ್ಲವೆ
ಲ್ಲಿದೆಯೋ
ಲ್ಲಿಯೂ
ಲ್ಲಿರುವ
ಲ್ಲೆಲ್ಲಾ
ಲ್ಲೋಕಾನ್
ಳ್ಳವನೂ
ವಂಚ-ನ-ಪರ
ವಂಚ-ನೆ-ಗಳನ್ನೆಲ್ಲ
ವಂಚಿ-ಸ-ಬ-ಲ್ಲನೆ
ವಂಚಿ-ಸಲು
ವಂಚಿ-ಸಿ-ಕೊ-ಳ್ಳು-ತ್ತಿ-ರು-ವನು
ವಂಚಿ-ಸು-ವನು
ವಂತ
ವಂತನ
ವಂತ-ನಾದ
ವಂತನೇ
ವಂತರ
ವಂತಹ
ವಂತಿಲ್ಲ
ವಂತೆ
ವಂದಿ-ಸು-ತ್ತೇನೆ
ವಂದಿ-ಸು-ವೆನು
ವಂದೇ
ವಂಶ
ವಂಶದ
ವಂಶ-ದಿಂದ
ವಂಶ-ವನ್ನೇ
ವಃ
ವಕಾ-ಲ-ತ್ತಿಗೆ
ವಕ್ತು-ಮ-ರ್ಹ-ಸ್ಯ-ಶೇ-ಷೇಣ
ವಕ್ತ್ರಾಣಿ
ವಕ್ತ್ರಾ-ಣ್ಯ-ಭಿ-ವಿ-ಜ್ವ-ಲಂತಿ
ವಕ್ರ-ವಾ-ಗಿಯೋ
ವಕ್ಷದ
ವಕ್ಷ್ಯಾಮಿ
ವಕ್ಷ್ಯಾ-ಮ್ಯ-ಶೇ-ಷತಃ
ವಚಃ
ವಚನಂ
ವಚ-ನ-ಮ-ಬ್ರ-ವೀತ್
ವಚ-ನ-ವನ್ನು
ವಚ-ಸ್ತೇನ
ವಜ್ರ
ವಜ್ರಂ
ವಜ್ರಕ್ಕೆ
ವಜ್ರದ
ವಜ್ರ-ದಂತೆ
ವಜ್ರ-ಪಡಿ
ವಜ್ರ-ಮು-ಷ್ಟಿ-ಯಲ್ಲಿ
ವಜ್ರ-ಮು-ಷ್ಠಿ-ಯಲ್ಲಿ
ವಜ್ರ-ವನ್ನು
ವಜ್ರಾ-ದಪಿ
ವಜ್ರಾ-ಯುಧ
ವಜ್ರಾ-ಯು-ಧ-ದಿಂದ
ವಟ-ವೃಕ್ಷ
ವತ್ಸಃ
ವತ್ಸ-ರ-ಗ-ಳಾ-ಚೆ-ಯಿಂದ
ವದ
ವದಂತಿ
ವದ-ನೈ-ರ್ಜ್ವ-ಲ-ದ್ಭಿಃ
ವದಸಿ
ವದಿ-ಷ್ಯಂತಿ
ವಧೆ
ವನನ್ನು
ವನ-ಮಾಲಿ
ವನ-ಮಾ-ಲೆ-ಯನ್ನು
ವನ-ವಾಸ
ವನ-ಸ್ಪ-ತಿ-ಗಳನ್ನು
ವನ-ಸ್ಪ-ತಿ-ಗ-ಳ-ಲ್ಲಿಯೂ
ವನಿಗೆ
ವನು
ವನೂ
ವನೇ
ವನೊ
ವನೋ
ವನ್ನಾ-ಗಲಿ
ವನ್ನಾಗಿ
ವನ್ನಾ-ದರೂ
ವನ್ನು
ವನ್ನೂ
ವನ್ನೆಲ್ಲ
ವನ್ನೇ
ವನ್ನೋ
ವಯಂ
ವಯ-ಮತಃ
ವಯಮ್
ವಯ-ಸ್ಕ-ನಾದ
ವಯ-ಸ್ಸಾ-ದಂ-ತೆಲ್ಲ
ವಯ-ಸ್ಸಿ-ನಲ್ಲಿ
ವಯಸ್ಸು
ವರ
ವರ-ಗಳನ್ನು
ವರ-ಣ-ದಲ್ಲಿ
ವರದ
ವರ-ದಂತೆ
ವರ-ಮಾನ
ವರ-ರೂ-ಪ-ದಲ್ಲಿ
ವರ-ವನ್ನು
ವರ-ವನ್ನೇ
ವರವೆ
ವರವೇ
ವರಾಹ
ವರಿಗೆ
ವರಿ-ದರೆ
ವರಿ-ದ-ವರು
ವರಿದು
ವರಿಸಿ
ವರಿ-ಸಿ-ಕೊಂ-ಡ-ವನು
ವರಿ-ಸು-ವನು
ವರು
ವರುಣ
ವರು-ಣೇಂ-ದ್ರ-ರು-ದ್ರ-ಮ-ರುತಃ
ವರುಣೋ
ವರುಷ
ವರು-ಷಕ್ಕೆ
ವರು-ಷ-ಗಳ
ವರು-ಷ-ಗಳಲ್ಲಿ
ವರು-ಷ-ಗ-ಳಾದ
ವರು-ಷ-ಗಳಿಂದ
ವರು-ಷ-ಗಳು
ವರು-ಷದ
ವರೂ
ವರೆ
ವರೆಗೂ
ವರೆಗೆ
ವರೊ
ವರೋ
ವರ್ಗ
ವರ್ಗ-ದ-ವರೆ
ವರ್ಗ-ದ-ವರೇ
ವರ್ಗೀ-ಕ-ರಿ-ಸುವ
ವರ್ಗೀ-ಕ-ರಿ-ಸು-ವುದೇ
ವರ್ಣ
ವರ್ಣ-ಕ್ಕಾ-ಗಲಿ
ವರ್ಣಕ್ಕೆ
ವರ್ಣಕ್ಕೊ
ವರ್ಣ-ಗಳನ್ನು
ವರ್ಣ-ಗಳಲ್ಲಿ
ವರ್ಣ-ಗ-ಳಿ-ಲ್ಲದೆ
ವರ್ಣ-ಗಳು
ವರ್ಣ-ಗ-ಳುಳ್ಳ
ವರ್ಣ-ಗಳೇ
ವರ್ಣದ
ವರ್ಣ-ದ-ಲ್ಲಾ-ದರೂ
ವರ್ಣ-ದಲ್ಲಿ
ವರ್ಣ-ದ-ವ-ರಿಗೂ
ವರ್ಣ-ದ-ವ-ರಿಗೆ
ವರ್ಣ-ದ-ವರು
ವರ್ಣನೆ
ವರ್ಣ-ನೆ-ಗ-ಳಿವೆ
ವರ್ಣ-ನೆಗೆ
ವರ್ಣ-ನೆ-ಯನ್ನು
ವರ್ಣ-ವನ್ನು
ವರ್ಣ-ಸಂ-ಕರ
ವರ್ಣ-ಸಂ-ಕರಃ
ವರ್ಣ-ಸಂ-ಕ-ರ-ಕಾ-ರ-ಕೈಃ
ವರ್ಣ-ಸಂ-ಕ-ರಕ್ಕೆ
ವರ್ಣ-ಸಂ-ಕ-ರ-ದಿಂದ
ವರ್ಣ-ಸಂ-ಕ-ರ-ವಾ-ಗು-ವುದು
ವರ್ಣ-ಸಂ-ಕ-ರವು
ವರ್ಣ-ಸಂ-ಕ-ರ-ವುಂ-ಟಾ-ಗು-ವುದು
ವರ್ಣಾ-ಶ್ರ-ಮ-ಗ-ಳಿ-ಗ-ನು-ಸಾ-ರ-ವಾಗಿ
ವರ್ಣಾ-ಶ್ರ-ಮದ
ವರ್ಣಿ-ಸು-ವನು
ವರ್ಣಿ-ಸು-ವು-ದಕ್ಕೆ
ವರ್ತ
ವರ್ತಂತ
ವರ್ತಕ
ವರ್ತ-ಕರು
ವರ್ತತೇ
ವರ್ತ-ಮಾನ
ವರ್ತ-ಮಾ-ನಾನಿ
ವರ್ತ-ಮಾ-ನೋಽಪಿ
ವರ್ತಿ-ಸ-ಬ-ಹುದು
ವರ್ತಿಸಿ
ವರ್ತಿ-ಸಿ-ದರೂ
ವರ್ತಿ-ಸು-ತ್ತವೆ
ವರ್ತಿ-ಸು-ತ್ತಾನೆ
ವರ್ತಿ-ಸು-ತ್ತಾರೆ
ವರ್ತಿ-ಸು-ತ್ತಿ-ರು-ತ್ತವೆ
ವರ್ತಿ-ಸು-ತ್ತೇನೆ
ವರ್ತಿ-ಸು-ವನು
ವರ್ತಿ-ಸು-ವರೊ
ವರ್ತಿ-ಸು-ವುದು
ವರ್ತೇ-ತಾ-ತ್ಮೈವ
ವರ್ತೇಯಂ
ವರ್ತ್ಮಾ-ನು-ವ-ರ್ತಂತೇ
ವರ್ಷ
ವರ್ಷಂ
ವರ್ಷ-ಗಳ
ವರ್ಷ-ಗಳಲ್ಲಿ
ವರ್ಷ-ಗ-ಳ-ವ-ರೆಗೆ
ವರ್ಷ-ಗ-ಳಾದ
ವರ್ಷ-ಗಳಿಂದ
ವರ್ಷ-ಗಳು
ವರ್ಷಾ-ನು-ಗ-ಟ್ಟಲೆ
ವಲ್ಲ
ವಲ್ಲಭ
ವವನ
ವವ-ನನ್ನೂ
ವವ-ನಿ-ಗಿಂತ
ವವನು
ವವನೂ
ವವನೇ
ವವರ
ವವ-ರನ್ನು
ವವ-ರಲ್ಲಿ
ವವ-ರಾ-ಗ-ಬೇಕು
ವವ-ರಿಲ್ಲ
ವವರು
ವವರೂ
ವವರೆ
ವವರೆಲ್ಲ
ವಶಂ
ವಶಕ್ಕೆ
ವಶ-ದಲ್ಲಿ
ವಶ-ದ-ಲ್ಲಿ-ಟ್ಟು-ಕೊಂ-ಡಿ-ರು-ವನು
ವಶ-ದ-ಲ್ಲಿ-ರುವ
ವಶ-ದ-ಲ್ಲಿ-ರು-ವುದೋ
ವಶ-ನಲ್ಲ
ವಶ-ನಾಗ
ವಶ-ನಾಗಿ
ವಶ-ನಾ-ಗಿ-ರು-ವು-ದಿಲ್ಲ
ವಶ-ಮಾ-ಗ-ಚ್ಛೇತ್
ವಶ-ರಾಗಿ
ವಶ-ರಾ-ಗು-ತ್ತಾರೆ
ವಶ-ವಾ-ಗ-ದ-ವರು
ವಶ-ವಾ-ಗದೆ
ವಶ-ವಾಗಿ
ವಶ-ವಾ-ಗಿವೆ
ವಶ-ವಾ-ಗು-ವನು
ವಶ-ವಾ-ಗು-ವು-ದಿಲ್ಲ
ವಶ-ವಾದ
ವಶೀ
ವಶೇ
ವಶ್ಯಾ-ತ್ಮನಾ
ವಸಂತ
ವಸ-ತಿಯ
ವಸವೋ
ವಸಿಷ್ಠ
ವಸು-ಗಳನ್ನು
ವಸು-ಗಳಲ್ಲಿ
ವಸು-ಗಳು
ವಸು-ದೇವ
ವಸು-ದೇ-ವನ
ವಸು-ದೇ-ವ-ನಿಗೆ
ವಸು-ದೇ-ವರ
ವಸು-ದೇ-ವ-ರಿಗೆ
ವಸು-ದೇ-ವ-ಸುತಂ
ವಸೂ-ನಾಂ
ವಸೂನ್
ವಸೂಲಿ
ವಸೂ-ಲಿ-ಮಾ-ಡ-ಬೇಕೇ
ವಸೂ-ಲಿ-ಮಾ-ಡು-ವುದು
ವಸ್ತ-ವಿನ
ವಸ್ತು
ವಸ್ತು-ಇ-ಲ್ಲದೇ
ವಸ್ತು-ಎಂದು
ವಸ್ತು-ಗಳ
ವಸ್ತು-ಗ-ಳಂತೆ
ವಸ್ತು-ಗಳನ್ನು
ವಸ್ತು-ಗಳನ್ನೂ
ವಸ್ತು-ಗಳನ್ನೆಲ್ಲಾ
ವಸ್ತು-ಗ-ಳನ್ನೇ
ವಸ್ತು-ಗಳಲ್ಲಿ
ವಸ್ತು-ಗ-ಳ-ಲ್ಲಿಯೂ
ವಸ್ತು-ಗ-ಳ-ಲ್ಲೆಲ್ಲಾ
ವಸ್ತು-ಗ-ಳಾ-ಗ-ಬ-ಹುದು
ವಸ್ತು-ಗ-ಳಾಗಿ
ವಸ್ತು-ಗ-ಳಾ-ಗಿವೆ
ವಸ್ತು-ಗಳಿಂದ
ವಸ್ತು-ಗ-ಳಿ-ಗಾಗಿ
ವಸ್ತು-ಗ-ಳಿಗೂ
ವಸ್ತು-ಗ-ಳಿಗೆ
ವಸ್ತು-ಗ-ಳಿ-ಗೆಲ್ಲಾ
ವಸ್ತು-ಗ-ಳಿ-ರ-ಬೇ-ಕು-ಅವೇ
ವಸ್ತು-ಗ-ಳಿ-ರು-ವಂತೆ
ವಸ್ತು-ಗ-ಳಿವೆ
ವಸ್ತು-ಗಳು
ವಸ್ತು-ಗಳೂ
ವಸ್ತು-ಗಳೆ
ವಸ್ತು-ಗ-ಳೆಂಬ
ವಸ್ತು-ಗ-ಳೆಲ್ಲ
ವಸ್ತು-ಗ-ಳೆಲ್ಲಾ
ವಸ್ತು-ಗಳೇ
ವಸ್ತು-ಗ-ಳೊಂ-ದಿಗೆ
ವಸ್ತು-ನಿ-ನಲ್ಲೂ
ವಸ್ತು-ವ-ನ್ನಾಗಿ
ವಸ್ತು-ವನ್ನು
ವಸ್ತು-ವ-ನ್ನು-ಸೃಷ್ಟಿ
ವಸ್ತು-ವನ್ನೂ
ವಸ್ತು-ವ-ನ್ನೆಲ್ಲ
ವಸ್ತು-ವನ್ನೇ
ವಸ್ತು-ವಲ್ಲ
ವಸ್ತು-ವಾ-ಗಲಿ
ವಸ್ತು-ವಾ-ಗಿ-ರ-ಬ-ಹುದು
ವಸ್ತು-ವಾಗು
ವಸ್ತು-ವಾ-ದರೊ
ವಸ್ತುವಿ
ವಸ್ತು-ವಿಗೂ
ವಸ್ತು-ವಿಗೆ
ವಸ್ತು-ವಿ-ಗೆಲ್ಲ
ವಸ್ತು-ವಿದೆ
ವಸ್ತು-ವಿ-ದ್ದರೆ
ವಸ್ತು-ವಿನ
ವಸ್ತು-ವಿ-ನಂತೆ
ವಸ್ತು-ವಿ-ನಂ-ತೆಯೂ
ವಸ್ತು-ವಿ-ನಂ-ತೆಯೋ
ವಸ್ತು-ವಿ-ನ-ಮೇಲೆ
ವಸ್ತು-ವಿ-ನಲ್ಲಿ
ವಸ್ತು-ವಿ-ನ-ಲ್ಲಿ-ರುವ
ವಸ್ತು-ವಿ-ನಿಂದ
ವಸ್ತು-ವಿ-ನೊಂ-ದಿಗೆ
ವಸ್ತುವು
ವಸ್ತುವೂ
ವಸ್ತುವೇ
ವಸ್ತ್ರ
ವಸ್ತ್ರ-ಗಳನ್ನು
ವಸ್ತ್ರ-ಗಳಿಂದ
ವಸ್ಥೆಯ
ವಹನೀ
ವಹಾ-ಮ್ಯಹಂ
ವಹಿ-ಸ-ದ-ವನು
ವಹಿ-ಸ-ಬಿ-ಡ-ಬ-ಹುದು
ವಹಿ-ಸ-ಬೇ-ಕಾ-ಗು-ವುದು
ವಹಿ-ಸಿ-ಕೊಂಡು
ವಹಿ-ಸಿ-ರು-ವುದು
ವಹಿ-ಸು-ತ್ತೇನೆ
ವಹಿ-ಸು-ವು-ದಕ್ಕೆ
ವಹಿ-ಸು-ವು-ದಿಲ್ಲ
ವಹ್ನಿ-ರ್ಯ-ಥಾ-ದರ್ಶೋ
ವಾ
ವಾಂಛ-ನೆ-ಗಳೂ
ವಾಂತಿ
ವಾಂತಿ-ಯನ್ನು
ವಾಂತಿಯೋ
ವಾಕ-ರಿಕೆ
ವಾಕ್
ವಾಕ್ಕು
ವಾಕ್ಯ
ವಾಕ್ಯಂ
ವಾಕ್ಯ-ಗಳಿಂದ
ವಾಕ್ಯ-ಗ-ಳಿಗೆ
ವಾಕ್ಯ-ಗಳು
ವಾಕ್ಯ-ಮಿ-ದ-ಮಾಹ
ವಾಕ್ಯ-ಮು-ವಾಚ
ವಾಕ್ಯೇನ
ವಾಗ
ವಾಗ-ತೊ-ಡ-ಗಿತು
ವಾಗ-ಬೇಕು
ವಾಗಲಿ
ವಾಗಲೂ
ವಾಗಲೇ
ವಾಗಿ
ವಾಗಿತ್ತು
ವಾಗಿದೆ
ವಾಗಿ-ದೆಯೊ
ವಾಗಿ-ದ್ದರೆ
ವಾಗಿ-ದ್ದಾನೆ
ವಾಗಿಯೂ
ವಾಗಿ-ರ-ಬೇಕು
ವಾಗಿ-ರುವ
ವಾಗಿ-ರು-ವು-ದನ್ನೇ
ವಾಗಿ-ರು-ವು-ದ-ರಿಂದ
ವಾಗಿ-ರು-ವುದು
ವಾಗಿ-ರು-ವುದೇ
ವಾಗಿಲ್ಲ
ವಾಗಿವೆ
ವಾಗುತ್ತ
ವಾಗು-ತ್ತವೆ
ವಾಗುತ್ತಾ
ವಾಗು-ವನು
ವಾಗು-ವ-ವ-ರೆಗೆ
ವಾಗು-ವು-ದಿಲ್ಲ
ವಾಗು-ವುದು
ವಾಗು-ವುವು
ವಾಙ್ಮಯಂ
ವಾಚಂ
ವಾಚಾಲಂ
ವಾಚಾ-ಳಿ-ಯ-ನ್ನಾಗಿ
ವಾಚಿಕ
ವಾಚಿ-ಕ-ಕ್ಕಿಂತ
ವಾಚಿ-ಕ-ವಾದ
ವಾಚ್ಯಂ
ವಾಡು-ತ್ತಿ-ರು-ವ-ವನು
ವಾಣಿ
ವಾಣಿಗೆ
ವಾಣಿಜ್ಯ
ವಾಣಿ-ಜ್ಯವೇ
ವಾಣಿಯ
ವಾಣಿ-ಯಂತೆ
ವಾಣಿ-ಯನ್ನು
ವಾಣಿ-ಯಲ್ಲಿ
ವಾಣಿ-ಯಿಂದ
ವಾಣಿಯೆ
ವಾಣಿಯೇ
ವಾತ-ರೋ-ಗ-ದಂತೆ
ವಾತ-ವ-ರಣ
ವಾತಾ
ವಾತಾ-ವ-ರಣ
ವಾತಾ-ವ-ರ-ಣಕ್ಕೆ
ವಾತಾ-ವ-ರ-ಣದ
ವಾತಾ-ವ-ರ-ಣ-ದಲ್ಲಿ
ವಾತಾ-ವ-ರ-ಣ-ದ-ಲ್ಲಿದೆ
ವಾತಾ-ವ-ರ-ಣ-ದ-ಲ್ಲಿ-ದ್ದರೂ
ವಾತಾ-ವ-ರ-ಣ-ದ-ಲ್ಲಿಯೇ
ವಾತಾ-ವ-ರ-ಣ-ದ-ಲ್ಲಿ-ರು-ವನೊ
ವಾತಾ-ವ-ರ-ಣ-ದ-ಲ್ಲಿ-ರು-ವರೊ
ವಾತಾ-ವ-ರ-ಣ-ದಿಂದ
ವಾತಾ-ವ-ರ-ಣ-ವನ್ನು
ವಾತಾ-ವ-ರ-ಣ-ವನ್ನೋ
ವಾತಾ-ವಾ-ರ-ಣ-ದಲ್ಲೆ
ವಾತ್ಸಲ್ಯ
ವಾತ್ಸ-ಲ್ಯ-ಕ್ಕಿಂತ
ವಾದ
ವಾದಂತೆ
ವಾದಃ
ವಾದ-ಜಾ-ಲ-ವನ್ನು
ವಾದದ
ವಾದ-ದಿಂದ
ವಾದದು
ವಾದದ್ದು
ವಾದ-ದ್ದೆಂದು
ವಾದ-ಮಾ-ಡುವ
ವಾದರೂ
ವಾದರೆ
ವಾದರೊ
ವಾದರೋ
ವಾದ-ವನ್ನು
ವಾದ-ವು-ಗಳು
ವಾದಷ್ಟು
ವಾದ-ಸ-ರ-ಣಿ-ಯನ್ನು
ವಾದಾಗ
ವಾದಿನಃ
ವಾದಿಯೂ
ವಾದಿ-ಸ-ಬೇ-ಕಾ-ಗಿದೆ
ವಾದಿ-ಸಿ-ದರೆ
ವಾದಿ-ಸು-ತ್ತಿ-ರು-ವ-ವರು
ವಾದಿ-ಸುವ
ವಾದಿ-ಸು-ವನು
ವಾದಿ-ಸು-ವರು
ವಾದುದು
ವಾದ್ಯ-ಗಳು
ವಾದ್ಯ-ಗಳೂ
ವಾಪಿ
ವಾಯಿತು
ವಾಯಿತೆ
ವಾಯು
ವಾಯುಃ
ವಾಯು-ಗ-ಳಂತೆ
ವಾಯು-ಗಳಲ್ಲಿ
ವಾಯು-ರ್ಗಂ-ಧಾ-ನಿ-ವಾ-ಶ-ಯಾತ್
ವಾಯು-ರ್ನಾ-ವ-ಮಿ-ವಾಂ-ಭಸಿ
ವಾಯು-ರ್ಯ-ಮೋ-ಽಗ್ನಿ-ರ್ವ-ರುಣಃ
ವಾಯು-ವನ್ನು
ವಾಯು-ವಿನ
ವಾಯು-ವಿ-ನಂತೆ
ವಾಯೋ-ರಿವ
ವಾರು
ವಾರ್ತೆ-ಯನ್ನು
ವಾರ್ಧಕ್ಯ
ವಾರ್ಷ್ಣೇಯ
ವಾಲದೆ
ವಾಲ-ಬಾ-ರದು
ವಾಲುತ್ತ
ವಾಲು-ತ್ತಾನೆ
ವಾಲು-ತ್ತಿ-ದ್ದು-ದನ್ನು
ವಾಲು-ವನು
ವಾಲು-ವ-ವ-ನಲ್ಲ
ವಾಲು-ವು-ದಿಲ್ಲ
ವಾಲು-ವುದು
ವಾಲು-ವುದೇ
ವಾಲು-ವುದೊ
ವಾಲ್ಮೀಕಿ
ವಾಲ್ಮೀ-ಕಿ-ಗ-ಳಂ-ತಹ
ವಾಲ್ಮೀ-ಕಿಯ
ವಾಲ್ಮೀ-ಕಿ-ಯನ್ನು
ವಾಸ
ವಾಸನಾ
ವಾಸನೆ
ವಾಸ-ನೆ-ಗಳನ್ನು
ವಾಸ-ನೆ-ಗಳನ್ನೆಲ್ಲ
ವಾಸ-ನೆ-ಗಳಲ್ಲಿ
ವಾಸ-ನೆ-ಗ-ಳಿಗೆ
ವಾಸ-ನೆ-ಗ-ಳಿವೆ
ವಾಸ-ನೆ-ಗಳು
ವಾಸ-ನೆ-ಗಳೂ
ವಾಸ-ನೆ-ಗ-ಳೆ-ಲ್ಲ-ದ-ರಿಂದ
ವಾಸ-ನೆ-ಗಳೇ
ವಾಸ-ನೆ-ಗಾ-ಗಿಯೇ
ವಾಸ-ನೆಗೂ
ವಾಸ-ನೆ-ಗೊಂದು
ವಾಸ-ನೆಯ
ವಾಸ-ನೆ-ಯನ್ನು
ವಾಸ-ನೆ-ಯಾ-ಗಲಿ
ವಾಸ-ನೆ-ಯಾಗಿ
ವಾಸ-ನೆ-ಯು-ಳ್ಳ-ವರು
ವಾಸ-ನೆಯೂ
ವಾಸ-ನೆಯೇ
ವಾಸ-ಮಾ-ಡಲು
ವಾಸ-ಮಾ-ಡಿ-ಕೊಂ-ಡಿರು
ವಾಸ-ಮಾ-ಡುತ್ತ
ವಾಸ-ಮಾ-ಡು-ತ್ತಿ-ರು-ವನು
ವಾಸ-ಮಾ-ಡು-ತ್ತಿ-ರು-ವೆವು
ವಾಸ-ಮಾ-ಡುವ
ವಾಸ-ಮಾ-ಡು-ವಂತೆ
ವಾಸ-ಮಾ-ಡು-ವ-ವನು
ವಾಸ-ಮಾ-ಡು-ವು-ದಕ್ಕೆ
ವಾಸ-ಮಾ-ಡುವೆ
ವಾಸವಃ
ವಾಸ-ವಾ-ಗಿದ್ದು
ವಾಸ-ವಾ-ಗಿ-ರು-ವನು
ವಾಸ-ಸ್ಥಾನ
ವಾಸ-ಸ್ಥಾ-ನ-ವಾ-ಗಿ-ರು-ವ-ವನು
ವಾಸ-ಸ್ಥಾ-ನ-ವಾದ
ವಾಸ-ಸ್ಥಾ-ನ-ವೆಂದು
ವಾಸಾಂಸಿ
ವಾಸಿ
ವಾಸಿಸು
ವಾಸಿ-ಸುವ
ವಾಸಿ-ಸು-ವ-ವ-ನಿಗೆ
ವಾಸಿ-ಸು-ವ-ವನು
ವಾಸಿ-ಸು-ವ-ವರು
ವಾಸುಕಿ
ವಾಸು-ಕಿಃ
ವಾಸು-ಕಿ-ಯ-ಲ್ಲಿ-ರು-ವಷ್ಟು
ವಾಸು-ದೇವ
ವಾಸು-ದೇವಃ
ವಾಸು-ದೇ-ವನ
ವಾಸು-ದೇ-ವ-ನನ್ನು
ವಾಸು-ದೇ-ವ-ನಿಗೆ
ವಾಸು-ದೇ-ವನೇ
ವಾಸು-ದೇ-ವ-ಮಯ
ವಾಸು-ದೇ-ವ-ಸ್ತ-ಥೋಕ್ತ್ವಾ
ವಾಸು-ದೇ-ವಸ್ಯ
ವಾಸು-ದೇ-ವೋಽಸ್ಮಿ
ವಾಸೋ
ವಾಸ್ತ-ವದ
ವಾಸ್ತ-ವಿ-ಕ-ವಾಗಿ
ವಾಹನ
ವಿಂದತಿ
ವಿಂದ-ತ್ಯಾ-ತ್ಮನಿ
ವಿಂದಾಮಿ
ವಿಕಂ-ಪಿ-ತು-ಮ-ರ್ಹಸಿ
ವಿಕರ್ಣ
ವಿಕ-ರ್ಣರು
ವಿಕ-ರ್ಣಶ್ಚ
ವಿಕರ್ಮ
ವಿಕ-ರ್ಮ-ಗಳು
ವಿಕ-ರ್ಮಣಃ
ವಿಕ-ರ್ಷ-ಣೆ-ಗಳಿಂದ
ವಿಕ-ಸಿ-ತ-ವಾದ
ವಿಕಾರ
ವಿಕಾ-ರಕ್ಕೂ
ವಿಕಾ-ರಕ್ಕೆ
ವಿಕಾ-ರ-ಗಳನ್ನು
ವಿಕಾ-ರ-ಗ-ಳಾ-ವು-ದಕ್ಕೂ
ವಿಕಾ-ರ-ಗಳು
ವಿಕಾ-ರ-ಗಳೂ
ವಿಕಾ-ರ-ಗ-ಳೆಲ್ಲ
ವಿಕಾ-ರ-ಗ-ಳೇನು
ವಿಕಾ-ರ-ದಿಂದ
ವಿಕಾ-ರ-ವಾ-ಗಿದೆ
ವಿಕಾ-ರ-ವೆಂ-ಥದ್ದು
ವಿಕಾ-ರಾಂಶ್ಚ
ವಿಕಾಸ
ವಿಕಾ-ಸಕ್ಕೆ
ವಿಕಾ-ಸದ
ವಿಕಾ-ಸ-ದಲ್ಲಿ
ವಿಕಾ-ಸ-ವನ್ನು
ವಿಕಾ-ಸ-ವಾ-ಗ-ಬೇ-ಕಾ-ಗಿದೆ
ವಿಕಾ-ಸ-ವಾ-ಗ-ಬೇ-ಕಾ-ದರೆ
ವಿಕಾ-ಸ-ವಾ-ಗ-ಬೇಕು
ವಿಕಾ-ಸ-ವಾ-ಗಲು
ವಿಕಾ-ಸ-ವಾಗಿ
ವಿಕಾ-ಸ-ವಾ-ಗಿಲ್ಲ
ವಿಕಾ-ಸ-ವಾ-ಗಿ-ಲ್ಲವೋ
ವಿಕಾ-ಸ-ವಾ-ಗುತ್ತ
ವಿಕಾ-ಸ-ವಾ-ಗು-ತ್ತದೆ
ವಿಕಾ-ಸ-ವಾ-ಗು-ತ್ತಿರು
ವಿಕಾ-ಸ-ವಾ-ಗು-ತ್ತಿ-ರುವ
ವಿಕಾ-ಸ-ವಾ-ಗು-ತ್ತಿ-ರು-ವುದು
ವಿಕಾ-ಸ-ವಾ-ಗುವ
ವಿಕಾ-ಸ-ವಾ-ಗು-ವು-ದಕ್ಕೆ
ವಿಕಾ-ಸ-ವಾ-ಗು-ವುದು
ವಿಕಾ-ಸ-ವಾದ
ವಿಕಾ-ಸ-ವಾ-ದಾಗ
ವಿಕಾ-ಸ-ವೆ-ನ್ನು-ವುದು
ವಿಕಾ-ಸ-ವೆಲ್ಲ
ವಿಕ್ರ-ಮಿ-ಯಾದ
ವಿಕ್ರಾಂತ
ವಿಗ-ತ-ಕ-ಲ್ಮಷಃ
ವಿಗ-ತ-ಜ್ವರಃ
ವಿಗ-ತ-ಭೀ-ರ್ಬ್ರ-ಹ್ಮ-ಚಾ-ರಿ-ವ್ರತೇ
ವಿಗ-ತ-ಸ್ಪೃಹಃ
ವಿಗ-ತೇ-ಚ್ಛಾ-ಭ-ಯ-ಕ್ರೋಧೋ
ವಿಗತೋ
ವಿಗುಣಃ
ವಿಗ್ರಹ
ವಿಗ್ರ-ಹ-ಗಳನ್ನು
ವಿಗ್ರ-ಹಾ-ರಾ-ಧ-ನೆ-ಯನ್ನು
ವಿಘ್ನ-ಗ-ಳಿಗೆ
ವಿಘ್ನ-ಗಳು
ವಿಚ-ಕ್ಷ-ಣಾಃ
ವಿಚ-ಲಿತ
ವಿಚ-ಲಿ-ತ-ನಾ-ಗದೆ
ವಿಚ-ಲಿ-ತ-ನಾಗು
ವಿಚ-ಲಿ-ತ-ನಾ-ಗು-ವು-ದಿಲ್ಲ
ವಿಚ-ಲಿ-ತ-ವಾ-ಗು-ವುದು
ವಿಚಾರ
ವಿಚಾ-ರ-ಗಳ
ವಿಚಾ-ರ-ಣೆಗೆ
ವಿಚಾ-ರದ
ವಿಚಾ-ರ-ದಿಂದ
ವಿಚಾ-ರ-ಪರ
ವಿಚಾ-ರ-ಪ-ರ-ನಾ-ಗಿ-ರುವ
ವಿಚಾ-ರ-ಪ-ರ-ನಿಗೆ
ವಿಚಾ-ರ-ಪ್ರ-ಣಾ-ಳಿಗೆ
ವಿಚಾ-ರ-ಪ್ರ-ಧಾ-ನದ
ವಿಚಾ-ರ-ಪ್ರ-ಧಾ-ನ-ವಾದ
ವಿಚಾ-ರ-ಮತಿ
ವಿಚಾ-ರ-ಮಾ-ಡದೆ
ವಿಚಾ-ರ-ಮಾ-ಡ-ಬೇಕು
ವಿಚಾ-ರ-ಮಾಡಿ
ವಿಚಾ-ರ-ಮಾ-ಡುವ
ವಿಚಾ-ರ-ಮಾ-ಡು-ವ-ವನೂ
ವಿಚಾ-ರ-ಮಾ-ಡು-ವುದು
ವಿಚಾ-ರ-ವನ್ನು
ವಿಚಾ-ರ-ವನ್ನೇ
ವಿಚಾ-ರ-ವಾಗಿ
ವಿಚಾ-ರ-ವಾದಿ
ವಿಚಾ-ರವೂ
ವಿಚಾ-ರವೇ
ವಿಚಾ-ರ-ಶ-ಕ್ತಿ-ಯನ್ನು
ವಿಚಾ-ರಿ-ಸ-ಬೇ-ಕಾ-ಗಿದೆ
ವಿಚಾ-ರಿ-ಸ-ಲಾರ
ವಿಚಾ-ರಿ-ಸಲೂ
ವಿಚಾ-ರಿಸಿ
ವಿಚಾ-ರಿ-ಸಿ-ದರೆ
ವಿಚಾ-ರಿ-ಸು-ತ್ತಾನೆ
ವಿಚಾ-ರಿ-ಸು-ತ್ತಿ-ದ್ದರೆ
ವಿಚಾ-ರಿ-ಸು-ವ-ವನು
ವಿಚಾ-ರಿ-ಸು-ವು-ದಕ್ಕೆ
ವಿಚಾ-ರಿ-ಸು-ವು-ದಿಲ್ಲ
ವಿಚಾ-ಲ-ಯೇತ್
ವಿಚಾ-ಲ್ಯತೇ
ವಿಚಿತ್ರ
ವಿಚಿ-ತ್ರ-ವನ್ನು
ವಿಚಿ-ತ್ರ-ವಾ-ಯಿ-ತಲ್ಲ
ವಿಚೇ-ತಸಃ
ವಿಜಯ
ವಿಜಯಂ
ವಿಜ-ಯಕ್ಕೆ
ವಿಜ-ಯ-ವನ್ನು
ವಿಜಾ-ನತಃ
ವಿಜಾ-ನೀತೋ
ವಿಜಿ-ತಾ-ತ್ಮ-ನಾ-ಗಿ-ರು-ವನು
ವಿಜಿ-ತಾ-ತ್ಮನೂ
ವಿಜಿ-ತಾತ್ಮಾ
ವಿಜಿ-ತೇಂ-ದ್ರಿಯಃ
ವಿಜೃಂ-ಭಣೆ
ವಿಜ್ಞಾ-ತು-ಮಿ-ಚ್ಛಾಮಿ
ವಿಜ್ಞಾನ
ವಿಜ್ಞಾ-ನಕ್ಕೂ
ವಿಜ್ಞಾ-ನ-ದಿಂದ
ವಿಜ್ಞಾ-ನ-ಮಾ-ಸ್ತಿಕ್ಯಂ
ವಿಜ್ಞಾ-ನ-ವನ್ನು
ವಿಜ್ಞಾ-ನ-ವಿದೆ
ವಿಜ್ಞಾ-ನ-ವೆಂದರೆ
ವಿಜ್ಞಾ-ನ-ವೆಂದು
ವಿಜ್ಞಾ-ನ-ಸ-ಹಿತಂ
ವಿಜ್ಞಾನಿ
ವಿಜ್ಞಾ-ನಿ-ಗಳು
ವಿಜ್ಞಾ-ನಿಗೆ
ವಿಜ್ಞಾ-ನಿ-ಯಲ್ಲಿ
ವಿಜ್ಞಾ-ನಿ-ಯಾ-ಗ-ಬ-ಹುದು
ವಿಜ್ಞಾ-ನಿ-ಯಾ-ಗಿ-ರ-ಬ-ಹುದು
ವಿಜ್ಞಾ-ನಿಯೂ
ವಿಟ-ಮಿ-ನ್ಗಳು
ವಿತಂಡ
ವಿತತಾ
ವಿತ್ತೇಶೋ
ವಿದಿ-ತಾ-ತ್ಮ-ನಾಮ್
ವಿದಿತ್ವಾ
ವಿದಿ-ತ್ವೈನಂ
ವಿದುಃ
ವಿದುರ
ವಿದು-ರಾ-ಸು-ರಾಃ
ವಿದು-ರ್ದೇವಾ
ವಿದು-ರ್ಯಾಂತಿ
ವಿದು-ರ್ಯು-ಕ್ತ-ಚೇ-ತಸಃ
ವಿದೆ
ವಿದೆಯೋ
ವಿದ್ದರೆ
ವಿದ್ದ-ರೇನೆ
ವಿದ್ಧಿ
ವಿದ್ಧ್ಯ-ಕ-ರ್ತಾ-ರ-ಮ-ವ್ಯ-ಯಮ್
ವಿದ್ಧ್ಯ-ನಾದೀ
ವಿದ್ಧ್ಯಾ-ಸು-ರ-ನಿ-ಶ್ಚ-ಯಾನ್
ವಿದ್ಧ್ಯೇ-ನ-ಮಿಹ
ವಿದ್ಯತೇ
ವಿದ್ಯಾ-ದಾ-ನ-ವಿ-ರ-ಬ-ಹುದು
ವಿದ್ಯಾ-ದ್ದುಃ-ಖ-ಸಂ-ಯೋ-ಗ-ವಿ-ಯೋಗಂ
ವಿದ್ಯಾ-ದ್ವಿ-ವೃದ್ಧಂ
ವಿದ್ಯಾ-ನಾಂ
ವಿದ್ಯಾ-ಭ್ಯಾ-ಸ-ವನ್ನು
ವಿದ್ಯಾ-ಮಹಂ
ವಿದ್ಯಾ-ಯಾಂ
ವಿದ್ಯಾರ್ಥಿ
ವಿದ್ಯಾ-ರ್ಥಿ-ಯಂತೆ
ವಿದ್ಯಾ-ವಂತ
ವಿದ್ಯಾ-ವಂ-ತ-ನಾ-ಗಿ-ರ-ಬ-ಹುದು
ವಿದ್ಯಾ-ವಂ-ತನೂ
ವಿದ್ಯಾ-ವಂ-ತನೇ
ವಿದ್ಯಾ-ವಂ-ತರು
ವಿದ್ಯಾ-ವಿ-ನ-ಯ-ಗಳಿಂದ
ವಿದ್ಯಾ-ವಿ-ನ-ಯ-ಸಂ-ಪನ್ನೇ
ವಿದ್ಯು-ಚ್ಛಕ್ತಿ
ವಿದ್ಯು-ಚ್ಛ-ಕ್ತಿಯ
ವಿದ್ಯು-ಚ್ಛ-ಕ್ತಿ-ಯನ್ನೋ
ವಿದ್ಯುತ್
ವಿದ್ಯು-ತ್ತಿಗೆ
ವಿದ್ಯು-ತ್ತಿ-ನಂತೆ
ವಿದ್ಯು-ತ್ದೀ-ಪ-ಗ-ಳಿವೆ
ವಿದ್ಯು-ತ್ಶಕ್ತಿ
ವಿದ್ಯು-ತ್ಶ-ಕ್ತಿಗೆ
ವಿದ್ಯು-ತ್ಶ-ಕ್ತಿ-ಯನ್ನು
ವಿದ್ಯು-ತ್ಶ-ಕ್ತಿ-ಯಾ-ದರೆ
ವಿದ್ಯು-ತ್ಶ-ಕ್ತಿಯೇ
ವಿದ್ಯು-ದ್ದೀಪ
ವಿದ್ಯೆ
ವಿದ್ಯೆ-ಗಳನ್ನು
ವಿದ್ಯೆ-ಗಳಲ್ಲಿ
ವಿದ್ಯೆ-ಗ-ಳಾ-ವುವೂ
ವಿದ್ಯೆ-ಗಳಿಂದ
ವಿದ್ಯೆ-ಗ-ಳಿ-ಗಿಂತ
ವಿದ್ಯೆ-ಗ-ಳಿಗೂ
ವಿದ್ಯೆ-ಗ-ಳೆಲ್ಲ
ವಿದ್ಯೆಗೆ
ವಿದ್ಯೆಯ
ವಿದ್ಯೆ-ಯ-ನ್ನಾ-ದರೂ
ವಿದ್ಯೆ-ಯನ್ನು
ವಿದ್ಯೆ-ಯ-ನ್ನೆಲ್ಲ
ವಿದ್ಯೆ-ಯನ್ನೇ
ವಿದ್ಯೆ-ಯಲ್ಲ
ವಿದ್ಯೆ-ಯ-ಲ್ಲಾ-ದರೂ
ವಿದ್ಯೆ-ಯಲ್ಲಿ
ವಿದ್ಯೆ-ಯ-ಲ್ಲಿಯೂ
ವಿದ್ಯೆ-ಯಾ-ಗಿ-ರ-ಬ-ಹುದು
ವಿದ್ಯೆ-ಯಾ-ದರೊ
ವಿದ್ಯೆಯೇ
ವಿದ್ವತ್
ವಿದ್ವ-ತ್ತಿ-ಗಾ-ಗಲೀ
ವಿದ್ವ-ತ್ಪೂ-ರ್ಣ-ವಾದ
ವಿದ್ವಾಂ-ಸ-ನಂತೆ
ವಿದ್ವಾಂ-ಸ-ನಿದ್ದ
ವಿದ್ವಾಂ-ಸ-ರಿಗೇ
ವಿದ್ವಾಂ-ಸರು
ವಿದ್ವಾಂ-ಸರೇ
ವಿದ್ವಾನ್
ವಿಧ
ವಿಧ-ಗ-ಳಿಂ-ದಲೂ
ವಿಧ-ಗ-ಳಿವೆ
ವಿಧ-ಗ-ಳಿ-ವೆಯೊ
ವಿಧ-ಗಳು
ವಿಧದ
ವಿಧ-ದ-ಗ-ಳಾ-ವುವು
ವಿಧ-ದಲ್ಲಿ
ವಿಧ-ದ-ಲ್ಲಿಯೂ
ವಿಧ-ದಿಂದ
ವಿಧ-ದಿಂ-ದಲೂ
ವಿಧದ್ದು
ವಿಧ-ವಾಗಿ
ವಿಧ-ವಾ-ಗಿದೆ
ವಿಧ-ವಾ-ಗಿ-ರುವ
ವಿಧ-ವಾ-ಗಿ-ರು-ವುದು
ವಿಧ-ವಾದ
ವಿಧ-ವಾ-ದರೂ
ವಿಧ-ವಿ-ಧದ
ವಿಧ-ವಿ-ಧ-ವಾಗಿ
ವಿಧ-ವಿ-ಧ-ವಾದ
ವಿಧವೂ
ವಿಧ-ವೆಂದು
ವಿಧ-ವೆ-ಯರ
ವಿಧ-ವೆ-ಯ-ರಾ-ಗ-ಬ-ಹುದು
ವಿಧ-ವೆ-ಯರು
ವಿಧಾನ
ವಿಧಾ-ನ-ದ-ಲ್ಲಿದೆ
ವಿಧಾ-ನ-ವನ್ನು
ವಿಧಾ-ನೋ-ಕ್ತಾಃ
ವಿಧಿ
ವಿಧಿ-ದೃಷ್ಟೋ
ವಿಧಿ-ನಿ-ಯಮ
ವಿಧಿ-ನಿ-ಯ-ಮ-ವನ್ನೂ
ವಿಧಿ-ಪ್ರ-ಕಾರ
ವಿಧಿಯ
ವಿಧಿ-ಯಿ-ಲ್ಲದೆ
ವಿಧಿಯೇ
ವಿಧಿ-ವ-ಶ-ದಿಂದ
ವಿಧಿ-ವಿ-ಹೀ-ನ-ವಾಗಿ
ವಿಧಿ-ಸಿ-ರುವ
ವಿಧಿ-ಸು-ತ್ತಾನೆ
ವಿಧಿ-ಸು-ತ್ತಿ-ರು-ವನು
ವಿಧಿ-ಸು-ತ್ತಿ-ರುವೆ
ವಿಧಿ-ಸು-ತ್ತಿಲ್ಲ
ವಿಧಿ-ಸುವ
ವಿಧಿ-ಸು-ವರು
ವಿಧಿ-ಸು-ವ-ವರೆ
ವಿಧಿ-ಸುವು
ವಿಧಿ-ಸು-ವು-ದಕ್ಕೊ
ವಿಧಿ-ಹೀನ
ವಿಧಿ-ಹೀ-ನ-ಮ-ಸೃ-ಷ್ವಾನ್ನಂ
ವಿಧಿ-ಹೀ-ನ-ವಾ-ದುದು
ವಿಧೀ-ಯತೇ
ವಿಧೇ-ಯತೆ
ವಿಧೇ-ಯಾ-ತ್ಮನು
ವಿನ
ವಿನಂ-ಕ್ಷ್ಯಸಿ
ವಿನಃ
ವಿನ-ದ್ಯೋ-ಚ್ಚೈಃ
ವಿನಯ
ವಿನ-ಯ-ದಿಂದ
ವಿನ-ಯ-ವಿಲ್ಲ
ವಿನ-ಲ್ಲಿಯೂ
ವಿನ-ಶ್ಯತಿ
ವಿನ-ಶ್ಯ-ತ್ಸ್ವ-ವಿ-ನ-ಶ್ಯಂತಂ
ವಿನಹ
ವಿನಾ
ವಿನಾ-ಯ-ಕನ
ವಿನಾ-ಯ-ಕ-ನಂತೆ
ವಿನಾ-ಯತಿ
ವಿನಾಶ
ವಿನಾ-ಶ-ಮ-ವ್ಯ-ಯ-ಸ್ಯಾಸ್ಯ
ವಿನಾ-ಶ-ಸ್ತಸ್ಯ
ವಿನಾ-ಶಾಯ
ವಿನಿ-ಯತಂ
ವಿನಿ-ಯಮ್ಯ
ವಿನಿ-ಯೋ-ಗಿಸಿ
ವಿನಿ-ಯೋ-ಗಿ-ಸು-ವರೋ
ವಿನಿ-ಯೋ-ಗಿ-ಸು-ವೆವು
ವಿನಿ-ವ-ರ್ತಂತೇ
ವಿನಿ-ವೃ-ತ್ತ-ಕಾ-ಮಾಃ
ವಿನೀ-ತ-ನಾಗಿ
ವಿನೋ-ದ-ವಾಗಿ
ವಿಪ-ತ್ತನ್ನು
ವಿಪ-ತ್ತಿ-ನಿಂದ
ವಿಪತ್ತು
ವಿಪ-ತ್ತು-ಗಳಿಂದ
ವಿಪ-ತ್ತು-ಗ-ಳಿಂ-ದಲೂ
ವಿಪತ್ತೇ
ವಿಪ-ರೀತ
ವಿಪ-ರೀತಂ
ವಿಪ-ರೀ-ತ-ವನ್ನು
ವಿಪ-ರೀ-ತ-ವಾಗಿ
ವಿಪ-ರೀ-ತಾಂಶ್ಚ
ವಿಪ-ರೀ-ತಾನಿ
ವಿಪ-ಶ್ಚಿತಃ
ವಿಫ-ಲ-ವಾ-ಗ-ಕೂ-ಡದು
ವಿಫ-ಲ-ವಾ-ಗು-ವು-ದಿಲ್ಲ
ವಿಭ-ಕ್ತ-ಮಿವ
ವಿಭ-ಕ್ತೇಷು
ವಿಭ-ಜನೆ
ವಿಭ-ಜ-ನೆ-ಮಾ-ಡಿ-ಕೊ-ಳ್ಳ-ಬ-ಹುದು
ವಿಭ-ಜ-ನೆ-ಮಾ-ಡು-ವ-ವರು
ವಿಭ-ಜ-ನೆ-ಮಾ-ಡು-ವುದು
ವಿಭಾಗ
ವಿಭಾ-ಗ-ಗಳನ್ನು
ವಿಭಾ-ಗದ
ವಿಭಾ-ಗ-ದಂತೆ
ವಿಭಾ-ಗ-ಮಾ-ಡು-ವುದು
ವಿಭಾ-ಗ-ಯೋಃ
ವಿಭಾ-ಗ-ವನ್ನು
ವಿಭಾ-ಗ-ವಾ-ಗದೆ
ವಿಭಾ-ಗ-ವಾ-ಗಿ-ರುವ
ವಿಭಾ-ಗ-ವಾ-ಗಿಲ್ಲ
ವಿಭಾ-ಗ-ವಾ-ದಂತೆ
ವಿಭಾ-ಗ-ವಿ-ಲ್ಲದೆ
ವಿಭಾ-ಗಿ-ಸ-ಲ್ಪ-ಟ್ಟಿವೆ
ವಿಭಾ-ಗಿ-ಸಿ-ರು-ವೆವೆ
ವಿಭಾ-ವಸೌ
ವಿಭೀ-ಷ-ಣ-ನನ್ನು
ವಿಭು
ವಿಭುಃ
ವಿಭುಮ್
ವಿಭು-ರೂಪ
ವಿಭು-ವೆಂದೂ
ವಿಭೂತಿ
ವಿಭೂ-ತಿಂ
ವಿಭೂ-ತಿ-ಗಳ
ವಿಭೂ-ತಿ-ಗಳನ್ನು
ವಿಭೂ-ತಿ-ಗಳನ್ನೆಲ್ಲ
ವಿಭೂ-ತಿ-ಗಳಲ್ಲಿ
ವಿಭೂ-ತಿ-ಗ-ಳಿಗೆ
ವಿಭೂ-ತಿ-ಗಳು
ವಿಭೂ-ತಿ-ಗಳೂ
ವಿಭೂ-ತಿಗೆ
ವಿಭೂ-ತಿ-ಪು-ರು-ಷರು
ವಿಭೂ-ತಿಯ
ವಿಭೂ-ತಿ-ಯನ್ನು
ವಿಭೂ-ತಿ-ಯಲ್ಲಿ
ವಿಭೂ-ತಿ-ಯಿಂದ
ವಿಭೂ-ತಿಯು
ವಿಭೂ-ತಿ-ಯು-ಳ್ಳದ್ದೊ
ವಿಭೂ-ತಿಯೂ
ವಿಭೂ-ತಿಯೆ
ವಿಭೂ-ತಿ-ಯೆಲ್ಲಾ
ವಿಭೂ-ತಿ-ಯೋಗ
ವಿಭೂ-ತಿ-ಯೋ-ಗದ
ವಿಭೂ-ತಿ-ಯೋ-ಗ-ದಲ್ಲಿ
ವಿಭೂ-ತೀ-ನಾಂ
ವಿಭೂ-ತೇ-ರ್ವಿ-ಸ್ತರೋ
ವಿಮ-ತ್ಸರಃ
ವಿಮ-ರ್ಶ-ಕರು
ವಿಮ-ರ್ಶಿ-ಸ-ಬಲ್ಲ
ವಿಮ-ರ್ಶಿ-ಸ-ಬೇ-ಕಾ-ಗಿದೆ
ವಿಮ-ರ್ಶಿ-ಸ-ಲಾ-ರರು
ವಿಮ-ರ್ಶಿಸಿ
ವಿಮ-ರ್ಶಿ-ಸು-ತ್ತಾನೆ
ವಿಮ-ರ್ಶಿ-ಸು-ತ್ತಿ-ರ-ಬೇಕು
ವಿಮ-ರ್ಶಿ-ಸುವ
ವಿಮ-ರ್ಶಿ-ಸು-ವನು
ವಿಮ-ರ್ಶಿ-ಸು-ವಾಗ
ವಿಮ-ರ್ಶೆ-ಮಾಡಿ
ವಿಮ-ರ್ಶೆಯ
ವಿಮಾನ
ವಿಮುಂ-ಚತಿ
ವಿಮು-ಕ್ತ-ರಾ-ಗ-ಬೇಕು
ವಿಮು-ಕ್ತ-ರಾ-ಗು-ತ್ತೇವೆ
ವಿಮುಕ್ತೋ
ವಿಮು-ಖ-ನಾ-ಗಿ-ದ್ದಾನೆ
ವಿಮು-ಖ-ರ-ನ್ನಾಗಿ
ವಿಮು-ಖ-ರಾಗಿ
ವಿಮು-ಖ-ರಾ-ದರೆ
ವಿಮುಚ್ಯ
ವಿಮು-ಹ್ಯತಿ
ವಿಮೂ-ಢ-ಭಾವೋ
ವಿಮೂ-ಢ-ರೆಂದೂ
ವಿಮೂಢಾ
ವಿಮೂ-ಢಾತ್ಮ
ವಿಮೂ-ಢಾತ್ಮಾ
ವಿಮೂಢೋ
ವಿಮೃ-ಶ್ಯೈ-ತ-ದ-ಶೇ-ಷೇಣ
ವಿಮೋ-ಕ್ಷ್ಯಸೇ
ವಿಯೋ-ಗ-ದ-ಲ್ಲಿ-ರುವ
ವಿಯೋ-ಗ-ದ-ಲ್ಲಿ-ರು-ವೆವು
ವಿಯೋ-ಗ-ವಾ-ಗಲಿ
ವಿರ-ಕ್ತನ
ವಿರಕ್ತಿ
ವಿರ-ಬ-ಹುದು
ವಿರಲಿ
ವಿರಳ
ವಿರ-ಳರ
ವಿರ-ಳವೇ
ವಿರಾ-ಜ-ಮಾ-ನ-ನಾದ
ವಿರಾಟ
ವಿರಾ-ಟಶ್ಚ
ವಿರಾಟ್
ವಿರಾ-ಟ್ರೂ-ಪಿ-ಯಾದ
ವಿರಾ-ಡ್ರೂ-ಪ-ವನ್ನು
ವಿರಾ-ಡ್ರೂ-ಪಿ-ಯಾದ
ವಿರಾಮ
ವಿರಾ-ಮ-ವಷ್ಟೆ
ವಿರಾ-ಮ-ವಾ-ಗಿಲ್ಲ
ವಿರುದ್ಧ
ವಿರು-ದ್ಧ-ನಾ-ಗದೆ
ವಿರು-ದ್ಧ-ವ-ಲ್ಲದ
ವಿರು-ದ್ಧ-ವಾಗಿ
ವಿರು-ದ್ಧ-ವಾದ
ವಿರು-ವಾಗ
ವಿರೋಧ
ವಿರೋ-ಧ-ಗಳಿಂದ
ವಿರೋ-ಧ-ಪ-ಕ್ಷ-ದ-ಲ್ಲಿ-ರುವ
ವಿರೋ-ಧ-ವ-ಲ್ಲದ
ವಿರೋ-ಧ-ವಾ-ಗ-ಲಿ-ಲ್ಲವೆ
ವಿರೋ-ಧ-ವಾಗಿ
ವಿರೋ-ಧ-ವಾ-ಗಿ-ದ್ದರೆ
ವಿರೋ-ಧ-ವಾ-ಗಿ-ರುವ
ವಿರೋ-ಧ-ವಾ-ಗಿ-ರು-ವ-ವರೆ
ವಿರೋ-ಧ-ವಾ-ಗಿ-ರು-ವ-ವು-ಗ-ಳೆಲ್ಲಾ
ವಿರೋ-ಧ-ವಾ-ಗಿ-ರು-ವುದು
ವಿರೋ-ಧ-ವಾ-ಗಿ-ರು-ವುದೋ
ವಿರೋ-ಧ-ವಾ-ಗಿ-ರು-ವುವು
ವಿರೋ-ಧ-ವಾ-ಗಿಲ್ಲ
ವಿರೋ-ಧ-ವಾದ
ವಿರೋ-ಧ-ವಾ-ದರೂ
ವಿರೋ-ಧ-ವಾ-ದು-ದನ್ನು
ವಿರೋ-ಧ-ವಾ-ದುದು
ವಿರೋ-ಧ-ವು-ಳ್ಳ-ವು-ಗಳು
ವಿರೋ-ಧವೂ
ವಿರೋ-ಧಾ-ಭಾ-ಸ-ಗ-ಳೆಲ್ಲ
ವಿರೋ-ಧಿ-ಯಾದ
ವಿರೋ-ಧಿ-ಸ-ದಿ-ದ್ದರೆ
ವಿರೋ-ಧಿ-ಸು-ತ್ತೇವೆ
ವಿರೋ-ಧಿ-ಸು-ವುದು
ವಿರೋ-ಧಿ-ಸು-ವುದೇ
ವಿರೋ-ಧಿ-ಸು-ವುದೋ
ವಿರ್ಮ-ರ್ಶಿ-ಸ-ಬೇಕು
ವಿಲ-ಕ್ಷ-ಣನೂ
ವಿಲೇ-ವಾರಿ
ವಿಲ್ಲ
ವಿಲ್ಲದೆ
ವಿಳಂ-ಬ-ವನ್ನು
ವಿಳಾಸ
ವಿವ-ರ-ಗಳನ್ನು
ವಿವ-ರ-ಗಳನ್ನೆಲ್ಲಾ
ವಿವ-ರ-ಗಳು
ವಿವ-ರ-ಗ-ಳೆಲ್ಲ
ವಿವ-ರಣೆ
ವಿವ-ರ-ಣೆಗೂ
ವಿವ-ರ-ಣೆ-ಯನ್ನು
ವಿವ-ರ-ವಾಗಿ
ವಿವ-ರಿಸ
ವಿವ-ರಿ-ಸ-ಬ-ಲ್ಲುದು
ವಿವ-ರಿ-ಸ-ಬ-ಹುದು
ವಿವ-ರಿ-ಸ-ಬೇ-ಕಾ-ದರೆ
ವಿವ-ರಿ-ಸ-ಲಾ-ಗದು
ವಿವ-ರಿ-ಸ-ಲಾ-ರದು
ವಿವ-ರಿ-ಸ-ಲಾ-ರೆವು
ವಿವ-ರಿ-ಸಲು
ವಿವ-ರಿಸಿ
ವಿವ-ರಿ-ಸಿ-ಕೊಂ-ಡಿ-ರು-ವೆಯೋ
ವಿವ-ರಿ-ಸಿ-ಕೊಂಡು
ವಿವ-ರಿ-ಸಿ-ದರೆ
ವಿವ-ರಿ-ಸಿದ್ದು
ವಿವ-ರಿ-ಸಿ-ರು-ವನು
ವಿವ-ರಿಸು
ವಿವ-ರಿ-ಸು-ತ್ತವೆ
ವಿವ-ರಿ-ಸು-ತ್ತಾನೆ
ವಿವ-ರಿ-ಸು-ತ್ತಾರೆ
ವಿವ-ರಿ-ಸು-ತ್ತಿದ್ದೆ
ವಿವ-ರಿ-ಸು-ತ್ತಿ-ರು-ವನೊ
ವಿವ-ರಿ-ಸು-ತ್ತೇನೆ
ವಿವ-ರಿ-ಸುವ
ವಿವ-ರಿ-ಸು-ವನು
ವಿವ-ರಿ-ಸು-ವರು
ವಿವ-ರಿ-ಸು-ವಾಗ
ವಿವ-ರಿ-ಸು-ವಾ-ಗಲೇ
ವಿವ-ರಿ-ಸು-ವು-ದ-ಕ್ಕಾ-ಗು-ವು-ದಿಲ್ಲ
ವಿವ-ರಿ-ಸು-ವು-ದಕ್ಕೆ
ವಿವ-ರಿ-ಸು-ವುದನ್ನು
ವಿವ-ರಿ-ಸು-ವು-ದಿಲ್ಲ
ವಿವ-ರಿ-ಸು-ವುದು
ವಿವ-ರಿ-ಸು-ವುದೊ
ವಿವ-ಸ್ವಂತ
ವಿವ-ಸ್ವತಃ
ವಿವ-ಸ್ವತೇ
ವಿವ-ಸ್ವಾನ್
ವಿವಿ-ಕ್ತ-ದೇ-ಶ-ಸೇ-ವಿ-ತ್ವ-ಮ-ರ-ತಿ-ರ್ಜ-ನ-ಸಂ-ಸದಿ
ವಿವಿ-ಕ್ತ-ಸೇವೀ
ವಿವಿಧ
ವಿವಿ-ಧ-ಮಾ-ರ್ಗ-ಗಳಿಂದ
ವಿವಿ-ಧ-ವಾದ
ವಿವಿ-ಧಾಃ
ವಿವಿ-ಧಾಶ್ಚ
ವಿವೃದ್ಧೇ
ವಿವೇಕ
ವಿವೇ-ಕ-ದಿಂದ
ವಿವೇಕಾ
ವಿವೇಕಾನಂದ
ವಿವೇಕಾನಂದ-ರಾದ
ವಿವೇಕಾನಂದರು
ವಿವೇ-ಕಿ-ಗಳನ್ನು
ವಿವೇ-ಕಿ-ಯಾ-ದರೂ
ವಿವೇ-ಚನಾ
ವಿಶಂತಿ
ವಿಶಂಶಿ
ವಿಶತೇ
ವಿಶ-ದ-ಪ-ಡಿ-ಸು-ವರು
ವಿಶಾಲ
ವಿಶಾಲಂ
ವಿಶಾ-ಲ-ಬುದ್ಧೇ
ವಿಶಾ-ಲ-ವಾಗಿ
ವಿಶಾ-ಲ-ವಾ-ಗಿದೆ
ವಿಶಾ-ಲ-ವಾ-ಗಿ-ರುವ
ವಿಶಾ-ಲ-ವಾ-ಗಿ-ರು-ವುದು
ವಿಶಾ-ಲ-ವಾ-ಗುತ್ತ
ವಿಶಾ-ಲ-ವಾ-ಗುತ್ತಾ
ವಿಶಾ-ಲ-ವಾ-ಗು-ವುದು
ವಿಶಾ-ಲ-ವಾ-ಗು-ವುದೋ
ವಿಶಾ-ಲ-ವಾದ
ವಿಶಾ-ಲ-ವಾ-ದುದು
ವಿಶಾ-ಲ-ಹೃ-ದ-ಯನು
ವಿಶಿಷ್ಟಾ
ವಿಶಿ-ಷ್ಟಾ-ದ್ವೈತ
ವಿಶಿ-ಷ್ಟಾ-ದ್ವೈ-ತದ
ವಿಶಿ-ಷ್ಟಾ-ದ್ವೈ-ತಿ-ಗಳು
ವಿಶಿ-ಷ್ಯತೇ
ವಿಶು-ದ್ಧಯಾ
ವಿಶು-ದ್ಧ-ವಾದ
ವಿಶು-ದ್ಧಾ-ತ್ಮ-ನಾ-ಗಿ-ರು-ವನು
ವಿಶು-ದ್ಧಾ-ತ್ಮನೂ
ವಿಶು-ದ್ಧಾತ್ಮಾ
ವಿಶೇಷ
ವಿಶೇ-ಷ-ರೂಪ
ವಿಶೇ-ಷ-ವಾಗಿ
ವಿಶೇ-ಷ-ವಾ-ಗಿದೆ
ವಿಶೇ-ಷ-ವಾದ
ವಿಶ್ರಾಂತಿ
ವಿಶ್ರಾಂ-ತಿ-ಯನ್ನು
ವಿಶ್ವ
ವಿಶ್ವ-ಕಾ-ರು-ಣ್ಯ-ದಿಂದ
ವಿಶ್ವಕ್ಕೆ
ವಿಶ್ವ-ಚೈ-ತನ್ಯ
ವಿಶ್ವ-ತೋ-ಮುಖಃ
ವಿಶ್ವ-ತೋ-ಮು-ಖನೂ
ವಿಶ್ವ-ತೋ-ಮು-ಖಮ್
ವಿಶ್ವದ
ವಿಶ್ವ-ದಲ್ಲಿ
ವಿಶ್ವ-ದ-ಲ್ಲಿ-ರುವ
ವಿಶ್ವ-ದಲ್ಲೆಲ್ಲಾ
ವಿಶ್ವ-ದೇವಾ
ವಿಶ್ವ-ನಿ-ಯಾ-ಮ-ಕ-ವಿ-ಶ್ವ-ವನ್ನು
ವಿಶ್ವ-ಮ-ನಂ-ತ-ಮಾದ್ಯಂ
ವಿಶ್ವ-ಮ-ನಂ-ತ-ರೂಪ
ವಿಶ್ವ-ಮಿದಂ
ವಿಶ್ವ-ಮೂರ್ತಿ
ವಿಶ್ವ-ಮೂರ್ತೆ
ವಿಶ್ವ-ಮೂರ್ತೇ
ವಿಶ್ವ-ರೂಪ
ವಿಶ್ವ-ರೂ-ಪಕ್ಕೆ
ವಿಶ್ವ-ರೂ-ಪ-ಗಳಲ್ಲಿ
ವಿಶ್ವ-ರೂ-ಪದ
ವಿಶ್ವ-ರೂ-ಪ-ದಂತೆ
ವಿಶ್ವ-ರೂ-ಪ-ದ-ರ್ಶನ
ವಿಶ್ವ-ರೂ-ಪ-ದ-ರ್ಶ-ನದ
ವಿಶ್ವ-ರೂ-ಪ-ದ-ರ್ಶ-ನ-ಯೋಗ
ವಿಶ್ವ-ರೂ-ಪ-ದ-ರ್ಶ-ನ-ವೆಂ-ಬುದು
ವಿಶ್ವ-ರೂ-ಪ-ದಲ್ಲಿ
ವಿಶ್ವ-ರೂ-ಪನೆ
ವಿಶ್ವ-ರೂ-ಪ-ವನ್ನು
ವಿಶ್ವ-ರೂ-ಪ-ವನ್ನೇ
ವಿಶ್ವ-ರೂ-ಪವು
ವಿಶ್ವ-ರೂ-ಪ-ವೆಂಬ
ವಿಶ್ವ-ರೂ-ಪ-ವೆಲ್ಲ
ವಿಶ್ವ-ರೂಪಿ
ವಿಶ್ವ-ರೂಪು
ವಿಶ್ವ-ವನ್ನು
ವಿಶ್ವ-ವ-ನ್ನೆಲ್ಲ
ವಿಶ್ವ-ವ-ನ್ನೆಲ್ಲಾ
ವಿಶ್ವ-ವಿದೆ
ವಿಶ್ವ-ವೆಲ್ಲ
ವಿಶ್ವ-ವೆಲ್ಲಾ
ವಿಶ್ವವೇ
ವಿಶ್ವ-ವ್ಯಾ-ಪಿ-ಯಾ-ಗಿದೆ
ವಿಶ್ವ-ವ್ಯಾ-ಪಿ-ಯಾದ
ವಿಶ್ವಸ್ಯ
ವಿಶ್ವಾ
ವಿಶ್ವಾ-ನು-ಕಂ-ಪ-ದಿಂದ
ವಿಶ್ವಾ-ನು-ಕಂಪೆ
ವಿಶ್ವಾಸ
ವಿಶ್ವಾ-ಸ-ಗಳು
ವಿಶ್ವಾ-ಸ-ವಿಲ್ಲ
ವಿಶ್ವೇ-ದೇ-ವ-ತೆ-ಗಳು
ವಿಶ್ವೇ-ದೇವಾ
ವಿಶ್ವೇ-ಶ್ವರ
ವಿಶ್ವೇ-ಶ್ವರಾ
ವಿಶ್ವೇ-ಽಶ್ವಿನೌ
ವಿಷ
ವಿಷ-ಕ್ಕಿಂತ
ವಿಷ-ಕ್ರಿಮಿ
ವಿಷ-ಕ್ರಿ-ಮಿ-ಗ-ಳಂತೆ
ವಿಷ-ಕ್ರಿ-ಮಿ-ಗಳು
ವಿಷ-ಗಾ-ಳಿ-ಯಾ-ದರೂ
ವಿಷದ
ವಿಷ-ದಂತೆ
ವಿಷ-ದಲ್ಲಿ
ವಿಷ-ಪೂ-ರಿತ
ವಿಷಮ
ವಿಷ-ಮಿವ
ವಿಷಮೇ
ವಿಷಯ
ವಿಷ-ಯಕ್ಕೂ
ವಿಷ-ಯಕ್ಕೆ
ವಿಷ-ಯ-ಗಳ
ವಿಷ-ಯ-ಗ-ಳ-ನ್ನಾ-ದರೂ
ವಿಷ-ಯ-ಗಳನ್ನು
ವಿಷ-ಯ-ಗಳನ್ನೆಲ್ಲ
ವಿಷ-ಯ-ಗ-ಳನ್ನೇ
ವಿಷ-ಯ-ಗಳಲ್ಲಿ
ವಿಷ-ಯ-ಗ-ಳ-ಲ್ಲಿಯೂ
ವಿಷ-ಯ-ಗ-ಳಾ-ದರೆ
ವಿಷ-ಯ-ಗ-ಳಿಗೂ
ವಿಷ-ಯ-ಗ-ಳಿಗೆ
ವಿಷ-ಯ-ಗ-ಳಿ-ರು-ತ್ತವೆ
ವಿಷ-ಯ-ಗ-ಳಿವೆ
ವಿಷ-ಯ-ಗ-ಳಿ-ವೆಯೋ
ವಿಷ-ಯ-ಗಳು
ವಿಷ-ಯ-ಗ-ಳೆಲ್ಲ
ವಿಷ-ಯ-ಗ-ಳೆಲ್ಲಾ
ವಿಷ-ಯ-ಗ-ಳೊಂ-ದಿಗೆ
ವಿಷ-ಯದ
ವಿಷ-ಯ-ದಲ್ಲಿ
ವಿಷ-ಯ-ದ-ಲ್ಲಿಯೂ
ವಿಷ-ಯ-ದಿಂದ
ವಿಷ-ಯನ್ನು
ವಿಷ-ಯ-ಪ್ರ-ವಾ-ಲಾಃ
ವಿಷ-ಯಳ
ವಿಷ-ಯ-ವನ್ನು
ವಿಷ-ಯ-ವ-ನ್ನೆಲ್ಲ
ವಿಷ-ಯ-ವನ್ನೇ
ವಿಷ-ಯ-ವಲ್ಲ
ವಿಷ-ಯ-ವಸ್ತು
ವಿಷ-ಯ-ವ-ಸ್ತು-ಗಳ
ವಿಷ-ಯ-ವ-ಸ್ತು-ಗಳನ್ನು
ವಿಷ-ಯ-ವ-ಸ್ತು-ಗಳಲ್ಲಿ
ವಿಷ-ಯ-ವ-ಸ್ತು-ಗಳಿಂದ
ವಿಷ-ಯ-ವ-ಸ್ತು-ಗ-ಳಿಗೆ
ವಿಷ-ಯ-ವ-ಸ್ತು-ಗಳು
ವಿಷ-ಯ-ವ-ಸ್ತು-ಗ-ಳೆಲ್ಲ
ವಿಷ-ಯ-ವ-ಸ್ತು-ಗ-ಳೊ-ಡನೆ
ವಿಷ-ಯ-ವ-ಸ್ತು-ವನ್ನು
ವಿಷ-ಯ-ವ-ಸ್ತು-ವಿಗೂ
ವಿಷ-ಯ-ವ-ಸ್ತು-ವಿಗೆ
ವಿಷ-ಯ-ವ-ಸ್ತು-ವಿನ
ವಿಷ-ಯ-ವ-ಸ್ತು-ವಿ-ನಲ್ಲಿ
ವಿಷ-ಯ-ವ-ಸ್ತು-ವಿ-ನಿಂದ
ವಿಷ-ಯ-ವ-ಸ್ತು-ವಿ-ನೆ-ಡೆಗೆ
ವಿಷ-ಯ-ವಾಗಿ
ವಿಷ-ಯ-ವಾ-ಗಿಯೇ
ವಿಷ-ಯ-ವಾ-ದರೋ
ವಿಷ-ಯ-ವಿ-ದ್ದರೂ
ವಿಷ-ಯ-ವಿಲ್ಲ
ವಿಷ-ಯ-ವೆಂದು
ವಿಷ-ಯ-ವೆಂಬ
ವಿಷ-ಯ-ವೆಲ್ಲ
ವಿಷ-ಯ-ಸು-ಖದ
ವಿಷ-ಯ-ಸು-ಖ-ವನ್ನು
ವಿಷಯಾ
ವಿಷ-ಯಾ-ನನ್ಯ
ವಿಷ-ಯಾ-ನಿಂ-ದ್ರಿ-ಯೈ-ಶ್ಚ-ರನ್
ವಿಷ-ಯಾ-ನು-ಪ-ಸೇ-ವತೇ
ವಿಷ-ಯಾನ್
ವಿಷ-ಯೇಂ-ದ್ರಿಯ
ವಿಷ-ಯೇಂ-ದ್ರಿ-ಯ-ಸಂ-ಯೋ-ಗಾ-ದ್ಯ-ತ್ತ-ದ-ಗ್ರೇ-ಽಮೃ-ತೋ-ಪ-ಮಮ್
ವಿಷ-ವನ್ನು
ವಿಷ-ವಾ-ಗು-ವುದು
ವಿಷ-ವಿ-ರುವ
ವಿಷ-ವಿಲ್ಲ
ವಿಷ-ವಿ-ಲ್ಲದ
ವಿಷಾದ
ವಿಷಾದಂ
ವಿಷಾ-ದಕ್ಕೆ
ವಿಷಾ-ದದ
ವಿಷಾ-ದ-ದಿಂದ
ವಿಷಾ-ದ-ಪ-ಡ-ಬೇ-ಕಾದ
ವಿಷಾ-ದ-ಪ-ಡು-ತ್ತಿ-ರುವ
ವಿಷಾ-ದ-ಪ-ಡು-ತ್ತಿ-ರು-ವನು
ವಿಷಾ-ದ-ಪರ
ವಿಷಾ-ದ-ಯೋಗ
ವಿಷಾ-ದ-ಯೋ-ಗ-ವೆಂದು
ವಿಷಾ-ದ-ಯೋ-ಗ-ವೆಂಬ
ವಿಷಾ-ದವೇ
ವಿಷಾ-ದಿಸ
ವಿಷಾ-ದಿಸಿ
ವಿಷಾದೀ
ವಿಷೀ-ದಂ-ತ-ಮಿದಂ
ವಿಷೀ-ದ-ನ್ನಿ-ದ-ಮ-ಬ್ರ-ವೀತ್
ವಿಷ್ಟ-ಭ್ಯಾ-ಹ-ಮಿದಂ
ವಿಷ್ಟಾ-ದ್ವೈತ
ವಿಷ್ಠಿ-ತಮ್
ವಿಷ್ಣು
ವಿಷ್ಣು-ರ್ಜ್ಯೋ-ತಿ-ಷಾಂ
ವಿಷ್ಣು-ವಿನ
ವಿಷ್ಣು-ವಿ-ನಲ್ಲಿ
ವಿಷ್ಣುವೆ
ವಿಷ್ಣೋ
ವಿಸರ್ಗಃ
ವಿಸ-ರ್ಜಿ-ಸಿ-ದನು
ವಿಸ-ರ್ಜಿಸು
ವಿಸ-ರ್ಜಿ-ಸು-ವನು
ವಿಸ-ರ್ಜಿ-ಸು-ವುದು
ವಿಸೃ-ಜನ್
ವಿಸೃ-ಜಾಮಿ
ವಿಸೃ-ಜಾ-ಮ್ಯ-ಹಮ್
ವಿಸೃಜ್ಯ
ವಿಸ್ತ-ರಶಃ
ವಿಸ್ತ-ರಶೋ
ವಿಸ್ತ-ರಸ್ಯ
ವಿಸ್ತ-ರಿಸಿ
ವಿಸ್ತ-ರಿ-ಸಿ-ಕೊ-ಳ್ಳು-ವು-ದಕ್ಕೆ
ವಿಸ್ತ-ರಿ-ಸು-ತ್ತಿರು
ವಿಸ್ತ-ರೇ-ಣಾ-ತ್ಮನೋ
ವಿಸ್ತಾರ
ವಿಸ್ತಾರಂ
ವಿಸ್ತಾ-ರಕ್ಕೆ
ವಿಸ್ತಾ-ರ-ವನ್ನು
ವಿಸ್ತಾ-ರ-ವಾ-ಗ-ಬೇ-ಕೆಂದು
ವಿಸ್ತಾ-ರ-ವಾಗಿ
ವಿಸ್ತಾ-ರ-ವಾ-ಗಿದೆ
ವಿಸ್ತಾ-ರ-ವಾ-ಗಿ-ರು-ವುದು
ವಿಸ್ತಾ-ರ-ವಾ-ಗುತ್ತ
ವಿಸ್ತಾ-ರ-ವಾ-ಗು-ವುದು
ವಿಸ್ತಾ-ರ-ವಾ-ಗು-ವುದೇ
ವಿಸ್ಮ-ಯ-ಗ-ಳೆ-ರ-ಡನ್ನೂ
ವಿಸ್ಮ-ಯಾ-ವಿಷ್ಟೋ
ವಿಸ್ಮಯೋ
ವಿಸ್ಮಿ-ತಾ-ಶ್ಚೈವ
ವಿಹ-ರಿ-ಸ-ಬೇ-ಕೆಂಬ
ವಿಹ-ರಿಸಿ
ವಿಹ-ರಿ-ಸು-ತ್ತಿದ್ದ
ವಿಹ-ರಿ-ಸು-ತ್ತಿ-ರ-ವ-ವ-ರನ್ನು
ವಿಹ-ರಿ-ಸು-ತ್ತಿ-ರುವ
ವಿಹ-ರಿ-ಸು-ತ್ತಿ-ರು-ವನು
ವಿಹ-ರಿ-ಸು-ತ್ತಿ-ರು-ವಾಗ
ವಿಹ-ರಿ-ಸು-ತ್ತಿ-ರು-ವುದು
ವಿಹ-ರಿ-ಸು-ವರೊ
ವಿಹ-ರಿ-ಸು-ವ-ವ-ನಿಗೂ
ವಿಹ-ರಿ-ಸು-ವು-ದಕ್ಕೆ
ವಿಹ-ರಿ-ಸು-ವುದು
ವಿಹಾಯ
ವಿಹಾರ
ವಿಹಾ-ರಕ್ಕೆ
ವಿಹಾ-ರ-ಗಳನ್ನು
ವಿಹಾ-ರ-ದಲ್ಲಿ
ವಿಹಾ-ರ-ವಾ-ಗು-ವುದು
ವಿಹಾ-ರ-ಶ-ಯ್ಯಾ-ಸ-ನ-ಭೋ-ಜ-ನೇಷು
ವಿಹಾ-ರಿ-ಯಾಗಿ
ವಿಹಿ-ತ-ವಾ-ಗಿದೆ
ವಿಹಿ-ತಾಃ
ವಿಹಿ-ತಾನ್
ವಿಹ್ವ-ಲ-ನಾಗಿ
ವೀಕ್ಷಂತೇ
ವೀಣೆ
ವೀಣೆಗೆ
ವೀತ-ರಾ-ಗ-ಭ-ಯ-ಕ್ರೋಧಃ
ವೀತ-ರಾ-ಗ-ಭ-ಯ-ಕ್ರೋಧಾ
ವೀತ-ರಾ-ಗಾಃ
ವೀತ-ರಾ-ಗಿ-ಗ-ಳಾದ
ವೀರ
ವೀರ-ಕ್ಷ-ತ್ರಿಯ
ವೀರನ
ವೀರ-ನಾ-ಗು-ವನು
ವೀರ-ನೆಂಬ
ವೀರ-ಪ-ದ-ಕ-ವನ್ನು
ವೀರ-ಯೋಧ
ವೀರರ
ವೀರ-ರನ್ನು
ವೀರ-ರಲ್ಲಿ
ವೀರ-ರಾ-ಗ-ಬ-ಹು-ದಿತ್ತು
ವೀರ-ರಿಂದ
ವೀರರು
ವೀರ-ರೆಲ್ಲ
ವೀರ-ಸ್ವ-ರ್ಗ-ವಿದೆ
ವೀರಾ-ಧಿ-ವೀ-ರನ
ವೀರಾ-ಧಿ-ವೀ-ರ-ನಾದ
ವೀರಾ-ಧಿ-ವೀ-ರ-ನಿಗೆ
ವೀರಾ-ಧಿ-ವೀ-ರ-ರನ್ನು
ವೀರಾ-ಧಿ-ವೀ-ರ-ರಾ-ದ-ವರು
ವೀರಾ-ಧಿ-ವೀ-ರ-ರಿಗೆ
ವೀರಾ-ಧಿ-ವೀ-ರರು
ವೀರ್ಯ
ವೀರ್ಯದ
ವೀರ್ಯನು
ವೀರ್ಯ-ವಂ-ತ-ನಾದ
ವೀರ್ಯ-ವಾನ್
ವುದ-ಕ್ಕಾ-ಗಲಿ
ವುದ-ಕ್ಕಾಗಿ
ವುದಕ್ಕೂ
ವುದಕ್ಕೆ
ವುದನ್ನು
ವುದ-ನ್ನೆಲ್ಲ
ವುದ-ರಲ್ಲಿ
ವುದ-ರಲ್ಲೇ
ವುದ-ರಿಂದ
ವುದಲ್ಲ
ವುದಿಲ್ಲ
ವುದು
ವುದೂ
ವುದೆ
ವುದೇ
ವುದೊ
ವುದೋ
ವುವು
ವುವೂ
ವುವೊ
ವೃಕೋ-ದರಃ
ವೃಕ್ಷ
ವೃಕ್ಷಕ್ಕೂ
ವೃಕ್ಷಕ್ಕೆ
ವೃಕ್ಷ-ಗಳಲ್ಲಿ
ವೃಕ್ಷ-ಗ-ಳಿವೆ
ವೃಕ್ಷದ
ವೃಕ್ಷ-ದಲ್ಲಿ
ವೃಕ್ಷ-ದ-ಲ್ಲಿದೆ
ವೃಕ್ಷ-ದೊ-ಳಗೆ
ವೃಕ್ಷ-ವನ್ನು
ವೃಕ್ಷ-ವ-ನ್ನೆಲ್ಲಾ
ವೃಕ್ಷ-ವಾ-ಗಿತ್ತು
ವೃಕ್ಷ-ವೇನೊ
ವೃಜಿನಂ
ವೃತ್ತಕ್ಕೆ
ವೃತ್ತದ
ವೃತ್ತ-ದಲ್ಲಿ
ವೃತ್ತ-ವನ್ನು
ವೃತ್ತಿ
ವೃತ್ತಿ-ಗಳನ್ನು
ವೃತ್ತಿ-ಗ-ಳಿಗೆ
ವೃತ್ತಿಗೆ
ವೃತ್ತಿಯ
ವೃತ್ತಿ-ಯನ್ನು
ವೃತ್ತಿ-ಯಲ್ಲಿ
ವೃತ್ತಿ-ಯ-ಲ್ಲಿಯೂ
ವೃತ್ತಿ-ಯ-ಲ್ಲಿ-ರ-ಬೇಕು
ವೃತ್ತಿ-ಯ-ವರು
ವೃತ್ರಾ-ಸು-ರ-ನನ್ನು
ವೃಥಾ
ವೃದ್ಧ
ವೃದ್ಧ-ರಿಗೆ
ವೃದ್ಧಾಪ್ಯ
ವೃದ್ಧಾ-ಪ್ಯ-ಗಳು
ವೃದ್ಧಾ-ಪ್ಯದ
ವೃದ್ಧಾ-ಪ್ಯ-ದಲ್ಲಿ
ವೃದ್ಧಾ-ಪ್ಯ-ದ-ವ-ರೆಗೆ
ವೃದ್ಧಿ
ವೃದ್ಧಿ-ಕ್ಷಯ
ವೃದ್ಧಿ-ಗೊ-ಳಿ-ಸ-ಬೇ-ಕಾ-ಗಿದೆ
ವೃದ್ಧಿ-ಗೊ-ಳಿ-ಸುವ
ವೃದ್ಧಿ-ಮಾಡ
ವೃದ್ಧಿ-ಮಾ-ಡಿ-ಕೊ-ಳ್ಳ-ಬ-ಹುದು
ವೃದ್ಧಿ-ಮಾ-ಡಿ-ಕೊ-ಳ್ಳ-ಬೇಕು
ವೃದ್ಧಿ-ಯಾ-ಗ-ಬೇ-ಕಾ-ದರೆ
ವೃದ್ಧಿ-ಯಾ-ಗ-ಬೇಕು
ವೃದ್ಧಿ-ಯಾಗಿ
ವೃದ್ಧಿ-ಯಾ-ಗಿದೆ
ವೃದ್ಧಿ-ಯಾ-ಗಿರಿ
ವೃದ್ಧಿ-ಯಾ-ಗಿ-ರು-ವಾಗ
ವೃದ್ಧಿ-ಯಾಗು
ವೃದ್ಧಿ-ಯಾ-ಗುವ
ವೃದ್ಧಿ-ಯಾ-ಗು-ವು-ದಕ್ಕೆ
ವೃದ್ಧಿ-ಯಾ-ಗು-ವುದು
ವೃದ್ಧಿ-ಯಾ-ಗು-ವುದೇ
ವೃದ್ಧಿ-ಯಾ-ಗು-ವುದೊ
ವೃದ್ಧಿ-ಯಾ-ಗು-ವುದೋ
ವೃದ್ಧಿ-ಯಾ-ದರೆ
ವೃದ್ಧಿ-ಯಾ-ದಾಗ
ವೃಷ್ಣಿ
ವೃಷ್ಣೀ-ನಾಂ
ವೆಂಕ-ಟ-ರ-ಮಣ
ವೆಂಕ-ಟ-ರ-ಮ-ಣನ
ವೆಂದರೂ
ವೆಂದರೆ
ವೆಂದು
ವೆಂಬ
ವೆಂಬುದು
ವೆಚ್ಚ
ವೆಚ್ಚ-ಮಾ-ಡಲಿ
ವೆಚ್ಚ-ವಾ-ಗು-ವು-ದಿಲ್ಲ
ವೆಚ್ಚ-ವಾ-ಗು-ವುದು
ವೆದೈಃ
ವೆನು
ವೆನೊ
ವೆನೋ
ವೆಯೋ
ವೆವು
ವೆವೆ
ವೆವೊ
ವೆವೋ
ವೇಗ
ವೇಗಂ
ವೇಗ-ಕ್ಕಿಂತ
ವೇಗಕ್ಕೆ
ವೇಗ-ದಂತೆ
ವೇಗ-ದಲ್ಲಿ
ವೇಗ-ದಿಂದ
ವೇಗ-ವನ್ನು
ವೇಗ-ವಾಗಿ
ವೇಗವೋ
ವೇತ್ತಾಽಸಿ
ವೇತ್ತಿ
ವೇತ್ಥ
ವೇದ
ವೇದ-ಗಳ
ವೇದ-ಗಳನ್ನು
ವೇದ-ಗಳನ್ನೆಲ್ಲ
ವೇದ-ಗಳಲ್ಲಿ
ವೇದ-ಗ-ಳ-ಲ್ಲಿ-ರುವ
ವೇದ-ಗಳಿಂದ
ವೇದ-ಗ-ಳಿಂ-ದಲೂ
ವೇದ-ಗಳು
ವೇದ-ಗ-ಳೆಂಬ
ವೇದ-ಗ-ಳೆಲ್ಲ
ವೇದ-ದಲ್ಲಿ
ವೇದ-ದ-ಲ್ಲಿದೆ
ವೇದನೆ
ವೇದ-ನೆ-ಗಳನ್ನು
ವೇದ-ನೆ-ಗಳಿಂದ
ವೇದ-ನೆ-ಗ-ಳಿಗೆ
ವೇದ-ನೆ-ಗಳು
ವೇದ-ನೆ-ಗಳೂ
ವೇದ-ನೆ-ಗ-ಳೆ-ಲ್ಲವೂ
ವೇದ-ನೆಯ
ವೇದ-ನೆ-ಯನ್ನು
ವೇದ-ನೆ-ಯನ್ನೂ
ವೇದ-ನೆ-ಯನ್ನೇ
ವೇದ-ಮಂ-ತ್ರ-ಗಳನ್ನು
ವೇದ-ಯ-ಜ್ಞಾ-ಧ್ಯ-ಯ-ನೈ-ರ್ನ-ದಾ-ನೈರ್ನ
ವೇದ-ವನ್ನು
ವೇದ-ವಾ-ದ-ರ-ತಾಃ
ವೇದ-ವಿತ್
ವೇದ-ವಿದೋ
ವೇದ-ವೇ-ದಾಂತ
ವೇದ-ವೇ-ದಾಂ-ತ-ಗಳ
ವೇದ-ವೇ-ದಾಂ-ತ-ಗಳನ್ನೂ
ವೇದ-ವೇ-ದಾಂ-ತ-ದಲ್ಲಿ
ವೇದ-ವ್ಯಾಸ
ವೇದ-ಶಾ-ಖೆಯ
ವೇದಾ
ವೇದಾಂತ
ವೇದಾಂ-ತ-ಕೃ-ದ್ವೇ-ದ-ವಿ-ದೇವ
ವೇದಾಂ-ತಕ್ಕೆ
ವೇದಾಂ-ತದ
ವೇದಾಂ-ತ-ವನ್ನು
ವೇದಾಂತಿ
ವೇದಾಂ-ತಿ-ಗ-ಳಂತೆ
ವೇದಾ-ಧ್ಯ-ಯನ
ವೇದಾ-ಧ್ಯ-ಯ-ನ-ದಿಂದ
ವೇದಾ-ನಾಂ
ವೇದಾರ್ಥ
ವೇದಾ-ರ್ಥ-ಸಾ-ರವೂ
ವೇದಾ-ವಿ-ನಾ-ಶಿನಂ
ವೇದಾಶ್ಚ
ವೇದಾಹಂ
ವೇದಿ-ಕೆಯ
ವೇದಿ-ತವ್ಯಂ
ವೇದಿ-ತುಮ್
ವೇದೇ
ವೇದೇಷು
ವೇದೈರ್ನ
ವೇದೈಶ್ಚ
ವೇದೋ-ಪ-ನಿ-ಷ-ತ್ತಿನ
ವೇದ್ಯ
ವೇದ್ಯಂ
ವೇದ್ಯ-ವಾ-ಗಿದೆ
ವೇದ್ಯ-ವಾ-ಗು-ವುದು
ವೇದ್ಯ-ವಾದ
ವೇದ್ಯೋ
ವೇನು
ವೇಪ-ಥುಶ್ಚ
ವೇಲಾ-ಕುಲಾ
ವೇಳೆ
ವೇಳೆ-ಯಲ್ಲ
ವೇಷ
ವೇಷ-ಕ್ಕಾಗಿ
ವೇಷಕ್ಕೂ
ವೇಷಕ್ಕೆ
ವೇಷ-ಗಳನ್ನು
ವೇಷ-ಗಳನ್ನೆಲ್ಲ
ವೇಷ-ಗ-ಳಿಗೆ
ವೇಷ-ಗ-ಳಿವೆ
ವೇಷ-ಗಳು
ವೇಷದ
ವೇಷ-ದಲ್ಲಿ
ವೇಷ-ದಲ್ಲೆ
ವೇಷ-ಧಾ-ರಿ-ಯೊ-ಡನೆ
ವೇಷ-ಭೂ-ಷ-ಣ-ಗ-ಳೆಲ್ಲಾ
ವೇಷ-ವ-ನ್ನಲ್ಲ
ವೇಷ-ವನ್ನು
ವೇಷ-ವ-ನ್ನೆಲ್ಲಾ
ವೇಷ-ವೆಲ್ಲ
ವೈಕುಂ-ಠ-ಕ್ಕಿಂತ
ವೈಕುಂ-ಠಕ್ಕೆ
ವೈಕುಂ-ಠ-ದಲ್ಲಿ
ವೈಕುಂ-ಠ-ದಲ್ಲೋ
ವೈಜ್ಞಾ-ನಿಕ
ವೈಜ್ಞಾ-ನಿ-ಕ-ವಲ್ಲ
ವೈಜ್ಞಾ-ನಿ-ಕ-ವಾಗಿ
ವೈಜ್ಞಾ-ನಿ-ಕ-ವಾದ
ವೈದಿಕ
ವೈದ್ಯ
ವೈದ್ಯ-ಕೀಯ
ವೈದ್ಯನ
ವೈದ್ಯ-ನನ್ನು
ವೈದ್ಯ-ನಿ-ಗಾ-ದರೊ
ವೈದ್ಯ-ನಿ-ಗಿಂತ
ವೈದ್ಯ-ನಿಗೆ
ವೈದ್ಯನು
ವೈದ್ಯನೇ
ವೈದ್ಯರ
ವೈದ್ಯ-ರಿಗೆ
ವೈದ್ಯರು
ವೈನ-ತೇ-ಯಶ್ಚ
ವೈಭವ
ವೈಯ-ಕ್ತಿಕ
ವೈರ
ವೈರತ್ವ
ವೈರದ
ವೈರ-ವಿಲ್ಲ
ವೈರ-ವಿ-ಲ್ಲ-ದ-ವನೋ
ವೈರಾಗ್ಯ
ವೈರಾಗ್ಯಂ
ವೈರಾ-ಗ್ಯ-ಗಳನ್ನು
ವೈರಾ-ಗ್ಯ-ಗಳನ್ನೆಲ್ಲ
ವೈರಾ-ಗ್ಯದ
ವೈರಾ-ಗ್ಯ-ದಿಂದ
ವೈರಾ-ಗ್ಯ-ದಿಂ-ದಲೂ
ವೈರಾ-ಗ್ಯ-ಪ-ರ-ನಾಗಿ
ವೈರಾ-ಗ್ಯ-ಮ-ನ-ಹಂ-ಕಾರ
ವೈರಾ-ಗ್ಯ-ವನ್ನು
ವೈರಾ-ಗ್ಯವೇ
ವೈರಾ-ಗ್ಯೇಣ
ವೈರಿ
ವೈರಿ-ಗಳ
ವೈರಿ-ಗಳೂ
ವೈರಿ-ಣಮ್
ವೈವಿಧ್ಯ
ವೈವಿ-ಧ್ಯತೆ
ವೈವಿ-ಧ್ಯ-ತೆ-ಗಳ
ವೈವಿ-ಧ್ಯ-ತೆ-ಗಳನ್ನು
ವೈವಿ-ಧ್ಯ-ತೆ-ಗಳನ್ನೆಲ್ಲಾ
ವೈವಿ-ಧ್ಯ-ತೆ-ಗಳಲ್ಲಿ
ವೈವಿ-ಧ್ಯ-ತೆ-ಗಳಿಂದ
ವೈವಿ-ಧ್ಯ-ತೆ-ಗ-ಳಿವೆ
ವೈವಿ-ಧ್ಯ-ತೆ-ಗಳು
ವೈವಿ-ಧ್ಯ-ತೆ-ಗ-ಳೆಲ್ಲಾ
ವೈವಿ-ಧ್ಯ-ತೆಯ
ವೈವಿ-ಧ್ಯ-ತೆ-ಯಿಂದ
ವೈವಿ-ಧ್ಯ-ತೆ-ಯೆಲ್ಲಾ
ವೈವಿ-ಧ್ಯ-ತೆಯೇ
ವೈಶಿ-ಷ್ಟ್ಯ-ವಿ-ರು-ವಂತೆ
ವೈಶ್ಯ
ವೈಶ್ಯ-ಕರ್ಮ
ವೈಶ್ಯನ
ವೈಶ್ಯ-ನಿಗೆ
ವೈಶ್ಯರು
ವೈಶ್ಯಾ-ಸ್ತಥಾ
ವೈಶ್ವ-ದೇ-ವಾದಿ
ವೈಶ್ವಾ-ನರೋ
ವೈಷ-ಯಿಕ
ವೈಷ್ಣ-ವ-ರಿಗೆ
ವೊಂದೇ
ವೋ
ವೋಽಸ್ತಿ-ಷ್ಟ-ಕಾ-ಮ-ಧುಕ್
ವ್ಟಕ್ತಿಯ
ವ್ಯಕ್ತ
ವ್ಯಕ್ತ-ಗೊ-ಳಿ-ಸ-ಬ-ಲ್ಲುದು
ವ್ಯಕ್ತ-ಗೊ-ಳಿ-ಸು-ತ್ತದೆ
ವ್ಯಕ್ತ-ಗೊ-ಳಿ-ಸು-ತ್ತಿರು
ವ್ಯಕ್ತ-ಗೊ-ಳಿ-ಸು-ತ್ತಿ-ರು-ವನೊ
ವ್ಯಕ್ತ-ಗೊ-ಳಿ-ಸು-ತ್ತಿಲ್ಲ
ವ್ಯಕ್ತ-ದಿಂದ
ವ್ಯಕ್ತ-ನಲ್ಲ
ವ್ಯಕ್ತ-ನಾಗಿ
ವ್ಯಕ್ತ-ನಾ-ಗಿ-ರು-ವನು
ವ್ಯಕ್ತ-ನಾ-ಗು-ತ್ತಾನೆ
ವ್ಯಕ್ತ-ಪ-ಡಿ-ಸಲು
ವ್ಯಕ್ತ-ಪ-ಡಿ-ಸಿ-ಕೊ-ಳ್ಳ-ಬೇಕು
ವ್ಯಕ್ತ-ಪ-ಡಿ-ಸಿ-ದರೆ
ವ್ಯಕ್ತ-ಪ-ಡಿಸು
ವ್ಯಕ್ತ-ಪ-ಡಿ-ಸು-ತ್ತಿ-ದ್ದರೂ
ವ್ಯಕ್ತ-ಪ-ಡಿ-ಸು-ತ್ತಿ-ರು-ವರು
ವ್ಯಕ್ತ-ಪ-ಡಿ-ಸು-ತ್ತಿ-ರು-ವ-ರೆಂದು
ವ್ಯಕ್ತ-ಪ-ಡಿ-ಸು-ವನು
ವ್ಯಕ್ತ-ಪ-ಡಿ-ಸು-ವು-ದಿಲ್ಲ
ವ್ಯಕ್ತ-ಪ-ಡಿ-ಸು-ವುದು
ವ್ಯಕ್ತ-ಪ-ಡಿ-ಸು-ವೆವೊ
ವ್ಯಕ್ತ-ಮ-ಧ್ಯಾನಿ
ವ್ಯಕ್ತ-ಮಾ-ಡು-ವುದು
ವ್ಯಕ್ತ-ವಾಗ
ವ್ಯಕ್ತ-ವಾ-ಗದೆ
ವ್ಯಕ್ತ-ವಾಗಿ
ವ್ಯಕ್ತ-ವಾ-ಗಿ-ದೆಯೊ
ವ್ಯಕ್ತ-ವಾ-ಗಿ-ರುವ
ವ್ಯಕ್ತ-ವಾಗು
ವ್ಯಕ್ತ-ವಾ-ಗು-ತ್ತದೆ
ವ್ಯಕ್ತ-ವಾ-ಗು-ತ್ತವೆ
ವ್ಯಕ್ತ-ವಾ-ಗು-ತ್ತಾನೆ
ವ್ಯಕ್ತ-ವಾ-ಗು-ತ್ತಿದೆ
ವ್ಯಕ್ತ-ವಾ-ಗು-ತ್ತಿ-ರು-ವನು
ವ್ಯಕ್ತ-ವಾ-ಗು-ತ್ತಿ-ರು-ವುದು
ವ್ಯಕ್ತ-ವಾ-ಗುವ
ವ್ಯಕ್ತ-ವಾ-ಗು-ವಂತೆ
ವ್ಯಕ್ತ-ವಾ-ಗು-ವನು
ವ್ಯಕ್ತ-ವಾ-ಗು-ವು-ದಕ್ಕೆ
ವ್ಯಕ್ತ-ವಾ-ಗು-ವು-ದಿಲ್ಲ
ವ್ಯಕ್ತ-ವಾ-ಗು-ವುದು
ವ್ಯಕ್ತ-ವಾ-ಗು-ವುದೊ
ವ್ಯಕ್ತ-ವಾ-ಗು-ವುದೋ
ವ್ಯಕ್ತ-ವಾ-ಗು-ವುವು
ವ್ಯಕ್ತ-ವಾ-ದಾಗ
ವ್ಯಕ್ತಿ
ವ್ಯಕ್ತಿಂ
ವ್ಯಕ್ತಿ-ಗಳ
ವ್ಯಕ್ತಿ-ಗಳನ್ನು
ವ್ಯಕ್ತಿ-ಗಳನ್ನೂ
ವ್ಯಕ್ತಿ-ಗಳನ್ನೆಲ್ಲ
ವ್ಯಕ್ತಿ-ಗ-ಳನ್ನೇ
ವ್ಯಕ್ತಿ-ಗ-ಳಲ್ಲ
ವ್ಯಕ್ತಿ-ಗಳಲ್ಲಿ
ವ್ಯಕ್ತಿ-ಗಳಿಂದ
ವ್ಯಕ್ತಿ-ಗ-ಳಿಗೆ
ವ್ಯಕ್ತಿ-ಗ-ಳಿಗೇ
ವ್ಯಕ್ತಿ-ಗ-ಳಿ-ದ್ದರೆ
ವ್ಯಕ್ತಿ-ಗ-ಳಿ-ರು-ವರೊ
ವ್ಯಕ್ತಿ-ಗ-ಳಿ-ವರು
ವ್ಯಕ್ತಿ-ಗಳು
ವ್ಯಕ್ತಿ-ಗ-ಳೆಲ್ಲ
ವ್ಯಕ್ತಿ-ಗ-ಳೆ-ಲ್ಲರೂ
ವ್ಯಕ್ತಿ-ಗಳೇ
ವ್ಯಕ್ತಿಗೂ
ವ್ಯಕ್ತಿಗೆ
ವ್ಯಕ್ತಿತ್ವ
ವ್ಯಕ್ತಿ-ತ್ವಕ್ಕೆ
ವ್ಯಕ್ತಿ-ತ್ವ-ಗ-ಳಾ-ಗ-ಬ-ಹುದು
ವ್ಯಕ್ತಿ-ತ್ವ-ಗ-ಳಿವೆ
ವ್ಯಕ್ತಿ-ತ್ವ-ಗಳು
ವ್ಯಕ್ತಿ-ತ್ವ-ಗ-ಳೆಲ್ಲಾ
ವ್ಯಕ್ತಿ-ತ್ವದ
ವ್ಯಕ್ತಿ-ತ್ವ-ದಂತೆ
ವ್ಯಕ್ತಿ-ತ್ವ-ದಲ್ಲಿ
ವ್ಯಕ್ತಿ-ತ್ವ-ದಲ್ಲೆಲ್ಲ
ವ್ಯಕ್ತಿ-ತ್ವ-ವನ್ನು
ವ್ಯಕ್ತಿ-ತ್ವ-ವಾ-ಗು-ವುದು
ವ್ಯಕ್ತಿ-ತ್ವ-ವಿದೆ
ವ್ಯಕ್ತಿ-ತ್ವ-ವಿಲ್ಲ
ವ್ಯಕ್ತಿ-ತ್ವ-ವೆಲ್ಲ
ವ್ಯಕ್ತಿ-ತ್ವವೇ
ವ್ಯಕ್ತಿ-ತ್ವ-ವೇನು
ವ್ಯಕ್ತಿ-ಮಾ-ಪನ್ನಂ
ವ್ಯಕ್ತಿಯ
ವ್ಯಕ್ತಿ-ಯಂ-ತಹ
ವ್ಯಕ್ತಿ-ಯಂತೆ
ವ್ಯಕ್ತಿ-ಯಂ-ತೆಯೋ
ವ್ಯಕ್ತಿ-ಯ-ನ್ನಾಗಿ
ವ್ಯಕ್ತಿ-ಯನ್ನು
ವ್ಯಕ್ತಿ-ಯನ್ನೇ
ವ್ಯಕ್ತಿ-ಯಲ್ಲ
ವ್ಯಕ್ತಿ-ಯ-ಲ್ಲಾ-ದರೋ
ವ್ಯಕ್ತಿ-ಯಲ್ಲಿ
ವ್ಯಕ್ತಿ-ಯ-ಲ್ಲಿ-ರುವ
ವ್ಯಕ್ತಿ-ಯಾ-ಗ-ಬ-ಹುದು
ವ್ಯಕ್ತಿ-ಯಾ-ಗಲೀ
ವ್ಯಕ್ತಿ-ಯಾಗಿ
ವ್ಯಕ್ತಿ-ಯಾ-ಗಿದ್ದ
ವ್ಯಕ್ತಿ-ಯಾ-ಗಿ-ರು-ವನು
ವ್ಯಕ್ತಿ-ಯಾ-ಗು-ವನು
ವ್ಯಕ್ತಿ-ಯಾ-ಗು-ವು-ದಲ್ಲ
ವ್ಯಕ್ತಿ-ಯಿಂ-ದಲೂ
ವ್ಯಕ್ತಿಯೂ
ವ್ಯಕ್ತಿಯೆ
ವ್ಯಕ್ತಿಯೇ
ವ್ಯತಿ-ರಿ-ಕ್ತ-ವಾಗಿ
ವ್ಯತೀ-ತಾನಿ
ವ್ಯತ್ಯಾಸ
ವ್ಯತ್ಯಾ-ಸ-ದಂತೆ
ವ್ಯತ್ಯಾ-ಸ-ವನ್ನು
ವ್ಯತ್ಯಾ-ಸ-ವನ್ನೂ
ವ್ಯತ್ಯಾ-ಸ-ವಾ-ಗ-ಬ-ಹುದು
ವ್ಯತ್ಯಾ-ಸ-ವಾ-ಗು-ತ್ತಿ-ರು-ತ್ತದೆ
ವ್ಯತ್ಯಾ-ಸ-ವಾ-ಗು-ವುದು
ವ್ಯತ್ಯಾ-ಸ-ವಾ-ದಂತೆ
ವ್ಯತ್ಯಾ-ಸ-ವಿದೆ
ವ್ಯತ್ಯಾ-ಸವೂ
ವ್ಯತ್ಯಾ-ಸ-ವೆಲ್ಲ
ವ್ಯತ್ಯಾ-ಸವೇ
ವ್ಯತ್ಯಾ-ಸ-ವೇನೂ
ವ್ಯಥಂತಿ
ವ್ಯಥ-ಯಂ-ತ್ಯೇತೇ
ವ್ಯಥಾ
ವ್ಯಥಿಷ್ಠಾ
ವ್ಯಥೆ
ವ್ಯಥೆ-ಗೀ-ಡಾ-ಗು-ತ್ತೇವೆ
ವ್ಯಥೆ-ಗೊ-ಳಿ-ಸ-ಲಾ-ರವು
ವ್ಯಥೆ-ಗೊ-ಳಿ-ಸುವು
ವ್ಯಥೆ-ಪಟ್ಟು
ವ್ಯಥೆ-ಪ-ಟ್ಟು-ಕೊಂಡು
ವ್ಯಥೆ-ಪಡ
ವ್ಯಥೆ-ಪ-ಡ-ಬಾ-ರದೊ
ವ್ಯಥೆ-ಪ-ಡ-ಬೇ-ಕಾ-ಗು-ವುದು
ವ್ಯಥೆ-ಪ-ಡ-ಬೇಡ
ವ್ಯಥೆ-ಪಡು
ವ್ಯಥೆ-ಪ-ಡು-ತ್ತಾನೆ
ವ್ಯಥೆ-ಪ-ಡು-ತ್ತಾಳೆ
ವ್ಯಥೆ-ಪ-ಡು-ತ್ತಿ-ರು-ವನು
ವ್ಯಥೆ-ಪ-ಡು-ತ್ತಿ-ರು-ವುದು
ವ್ಯಥೆ-ಪ-ಡು-ತ್ತೇನೆ
ವ್ಯಥೆ-ಪ-ಡು-ತ್ತೇವೆ
ವ್ಯಥೆ-ಪ-ಡು-ವನು
ವ್ಯಥೆ-ಪ-ಡು-ವರು
ವ್ಯಥೆ-ಪ-ಡು-ವ-ವ-ನಲ್ಲ
ವ್ಯಥೆ-ಪ-ಡು-ವ-ವನು
ವ್ಯಥೆ-ಪ-ಡು-ವು-ದಕ್ಕೆ
ವ್ಯಥೆ-ಪ-ಡು-ವು-ದಿಲ್ಲ
ವ್ಯಥೆ-ಪ-ಡು-ವೆವು
ವ್ಯಥೆ-ಯನ್ನು
ವ್ಯಥೆ-ಯನ್ನೂ
ವ್ಯಥೆ-ಯಾ-ಗು-ವಾಗ
ವ್ಯಥೆ-ಯಾ-ಗು-ವುದು
ವ್ಯಥೆ-ಯಾ-ದರೂ
ವ್ಯಥೆ-ಯಾ-ಯಿತು
ವ್ಯಥೆಯೂ
ವ್ಯದಾ-ರ-ಯತ್
ವ್ಯನು-ನಾ-ದ-ಯನ್
ವ್ಯಪಾ-ಶ್ರಿತ್ಯ
ವ್ಯಪೇ-ತ-ಭೀಃ
ವ್ಯಯ
ವ್ಯಯ-ಮಾ-ಡಿ-ಕೊ-ಳ್ಳು-ವೆವು
ವ್ಯಯ-ಮಾ-ಡು-ವುದು
ವ್ಯಯ-ವಾಗಿ
ವ್ಯಯ-ವಾ-ಗು-ತ್ತಿದೆ
ವ್ಯಯ-ವಾ-ಗು-ವು-ದಕ್ಕೆ
ವ್ಯಯ-ವಾ-ಗು-ವುದನ್ನು
ವ್ಯಯ-ವಾ-ಗು-ವು-ದ-ನ್ನೆಲ್ಲ
ವ್ಯಯ-ವಾ-ಗು-ವುದು
ವ್ಯರ್ಥ
ವ್ಯರ್ಥ-ಮಾ-ಡಿ-ಕೊ-ಳ್ಳ-ಬಾ-ರದು
ವ್ಯರ್ಥ-ಮಾ-ಡಿ-ಕೊ-ಳ್ಳು-ವೆವು
ವ್ಯರ್ಥ-ಮಾ-ಡು-ವು-ದಿಲ್ಲ
ವ್ಯರ್ಥ-ಮಾ-ಡು-ವುದು
ವ್ಯರ್ಥ-ವಲ್ಲ
ವ್ಯರ್ಥ-ವಾಗ
ವ್ಯರ್ಥ-ವಾ-ಗ-ದಂತೆ
ವ್ಯರ್ಥ-ವಾ-ಗ-ಲಿಲ್ಲ
ವ್ಯರ್ಥ-ವಾ-ಗ-ಲಿ-ಲ್ಲವೆ
ವ್ಯರ್ಥ-ವಾಗಿ
ವ್ಯರ್ಥ-ವಾ-ಗಿ-ರು-ವು-ದಿಲ್ಲ
ವ್ಯರ್ಥ-ವಾ-ಗಿ-ರು-ವುದು
ವ್ಯರ್ಥ-ವಾಗು
ವ್ಯರ್ಥ-ವಾ-ಗು-ತ್ತಿದೆ
ವ್ಯರ್ಥ-ವಾ-ಗು-ತ್ತಿದ್ದ
ವ್ಯರ್ಥ-ವಾ-ಗು-ವಂ-ತಿಲ್ಲ
ವ್ಯರ್ಥ-ವಾ-ಗು-ವು-ದಿಲ್ಲ
ವ್ಯರ್ಥ-ವಾ-ಗು-ವುದು
ವ್ಯರ್ಥ-ವಾ-ಗು-ವುದೋ
ವ್ಯರ್ಥ-ವಾ-ಗು-ವುವು
ವ್ಯರ್ಥ-ವಾ-ದಂತೆ
ವ್ಯರ್ಥ-ವಾ-ಯಿ-ತಲ್ಲ
ವ್ಯವ
ವ್ಯವ-ಸಾಯ
ವ್ಯವ-ಸಾ-ಯಸ್ತೇ
ವ್ಯವ-ಸಾ-ಯಾ-ತ್ಮಿಕಾ
ವ್ಯವ-ಸಾ-ಯಿ-ಗಳ
ವ್ಯವ-ಸಾ-ಯಿ-ಗಳಲ್ಲಿ
ವ್ಯವ-ಸಾ-ಯೋಽಸ್ಮಿ
ವ್ಯವ-ಸಿತಾ
ವ್ಯವ-ಸ್ಥಿ-ತಾನ್
ವ್ಯವ-ಸ್ಥಿತೌ
ವ್ಯವ-ಹ-ರಿಸಿ
ವ್ಯವ-ಹ-ರಿ-ಸಿ-ದರೂ
ವ್ಯವ-ಹ-ರಿ-ಸು-ತ್ತಾನೆ
ವ್ಯವ-ಹ-ರಿ-ಸು-ತ್ತಿ-ರು-ವರು
ವ್ಯವ-ಹ-ರಿ-ಸು-ತ್ತಿ-ರು-ವಾ-ಗಲೂ
ವ್ಯವ-ಹ-ರಿ-ಸು-ವುವು
ವ್ಯವ-ಹಾರ
ವ್ಯವ-ಹಾ-ರಕ್ಕೆ
ವ್ಯವ-ಹಾ-ರಕ್ಕೇ
ವ್ಯವ-ಹಾ-ರ-ಗಳಲ್ಲಿ
ವ್ಯವ-ಹಾ-ರದ
ವ್ಯವ-ಹಾ-ರ-ದ-ದೃಷ್ಟಿ
ವ್ಯವ-ಹಾ-ರ-ದಲ್ಲಿ
ವ್ಯವ-ಹಾ-ರ-ದ-ಲ್ಲಿ-ರು-ವಾಗ
ವ್ಯವ-ಹಾ-ರ-ದಲ್ಲೆ
ವ್ಯವ-ಹಾ-ರ-ವನ್ನು
ವ್ಯವ-ಹಾ-ರ-ವೆಲ್ಲ
ವ್ಯಷ್ಟಿ
ವ್ಯಷ್ಟಿ-ಗಳು
ವ್ಯಷ್ಟಿಗೆ
ವ್ಯಷ್ಟಿ-ದೃ-ಷ್ಟಿ-ಯಿಂದ
ವ್ಯಷ್ಟಿಯ
ವ್ಯಷ್ಟಿಯು
ವ್ಯಸ್ತ-ನಾ-ಗದೆ
ವ್ಯಸ್ತ-ರಾ-ಗದೆ
ವ್ಯಸ್ತ-ರಾ-ಗು-ವ-ವರು
ವ್ಯಸ್ತ-ಳಾಗಿ
ವ್ಯಸ್ತ-ವಾ-ಗು-ವು-ದಿಲ್ಲ
ವ್ಯಸ್ಥ-ಗೊ-ಳಿ-ಸ-ಲಾ-ರದು
ವ್ಯಾಕುಲ
ವ್ಯಾಕು-ಲತೆ
ವ್ಯಾಕು-ಲ-ತೆ-ಯಲ್ಲಿ
ವ್ಯಾಕು-ಲ-ತೆ-ಯಿಂದ
ವ್ಯಾಕು-ಲ-ದ-ಲ್ಲಿರ
ವ್ಯಾಕು-ಲ-ದ-ಲ್ಲಿ-ರು-ವುದು
ವ್ಯಾಕು-ಲ-ವಿ-ಲ್ಲದೆ
ವ್ಯಾಘ್ರ
ವ್ಯಾತ್ತಾ-ನನಂ
ವ್ಯಾಧ
ವ್ಯಾಧ-ನಿಗೆ
ವ್ಯಾಧಿ
ವ್ಯಾಧಿ-ಇ-ವು-ಗಳಲ್ಲಿ
ವ್ಯಾಧಿ-ಗಳನ್ನು
ವ್ಯಾಧಿ-ಗಳಲ್ಲಿ
ವ್ಯಾಧಿ-ಗ-ಳಿಗೆ
ವ್ಯಾಧಿಗೆ
ವ್ಯಾಧಿಯ
ವ್ಯಾಧಿ-ಯಿಂದ
ವ್ಯಾನ
ವ್ಯಾಪ-ಕ-ತೆ-ಯಲ್ಲಿ
ವ್ಯಾಪಾರ
ವ್ಯಾಪಾ-ರಕ್ಕೆ
ವ್ಯಾಪಾ-ರದ
ವ್ಯಾಪಾ-ರ-ದಲ್ಲಿ
ವ್ಯಾಪಾ-ರ-ವನ್ನೇ
ವ್ಯಾಪಾ-ರ-ವಲ್ಲ
ವ್ಯಾಪಾ-ರ-ವಾ-ಗು-ವುದು
ವ್ಯಾಪಾ-ರವೆ
ವ್ಯಾಪಾ-ರಾದಿ
ವ್ಯಾಪಾರಿ
ವ್ಯಾಪಿ-ಯಾಗಿ
ವ್ಯಾಪಿ-ಸ-ಲ್ಪ-ಟ್ಟಿದೆ
ವ್ಯಾಪಿಸಿ
ವ್ಯಾಪಿ-ಸಿ-ಕೊಂ-ಡಿದೆ
ವ್ಯಾಪಿ-ಸಿ-ಕೊಂ-ಡಿ-ದೆಯೊ
ವ್ಯಾಪಿ-ಸಿ-ಕೊಂ-ಡಿ-ದೆಯೋ
ವ್ಯಾಪಿ-ಸಿ-ಕೊಂ-ಡಿರು
ವ್ಯಾಪಿ-ಸಿ-ಕೊಂ-ಡಿ-ರುವ
ವ್ಯಾಪಿ-ಸಿ-ಕೊಂ-ಡಿ-ರು-ವಂತೆ
ವ್ಯಾಪಿ-ಸಿ-ಕೊಂ-ಡಿ-ರು-ವನು
ವ್ಯಾಪಿ-ಸಿ-ಕೊಂ-ಡಿ-ರು-ವನೊ
ವ್ಯಾಪಿ-ಸಿ-ಕೊಂ-ಡಿ-ರು-ವು-ದ-ರಿಂದ
ವ್ಯಾಪಿ-ಸಿ-ಕೊಂ-ಡಿ-ರು-ವುದು
ವ್ಯಾಪಿ-ಸಿ-ಕೊಂ-ಡಿ-ರು-ವುದೊ
ವ್ಯಾಪಿ-ಸಿ-ಕೊಂ-ಡಿ-ರು-ವುದೋ
ವ್ಯಾಪಿ-ಸಿ-ಕೊಂ-ಡಿ-ರು-ವೆಯೊ
ವ್ಯಾಪಿ-ಸಿ-ಕೊಂಡು
ವ್ಯಾಪಿ-ಸಿ-ಕೊ-ಳ್ಳದೆ
ವ್ಯಾಪಿ-ಸಿ-ಕೊ-ಳ್ಳು-ವು-ದ-ರಿಂದ
ವ್ಯಾಪಿ-ಸಿ-ಕೊ-ಳ್ಳು-ವುದು
ವ್ಯಾಪಿ-ಸಿ-ಕೊ-ಳ್ಳು-ವುವು
ವ್ಯಾಪಿ-ಸಿ-ಗೊಂ-ಡಿ-ರು-ವನು
ವ್ಯಾಪಿ-ಸಿದೆ
ವ್ಯಾಪಿ-ಸಿ-ದ್ದರೆ
ವ್ಯಾಪಿ-ಸಿ-ರು-ತ್ತದೆ
ವ್ಯಾಪಿ-ಸಿ-ರುವ
ವ್ಯಾಪಿ-ಸಿ-ರು-ವನು
ವ್ಯಾಪಿ-ಸಿಲ್ಲ
ವ್ಯಾಪಿ-ಸು-ವುದು
ವ್ಯಾಪ್ತಂ
ವ್ಯಾಪ್ತ-ನಾ-ಗಿ-ದ್ದೀಯೆ
ವ್ಯಾಪ್ತ-ವಾ-ಗಿದೆ
ವ್ಯಾಪ್ತ-ವಾ-ಗಿ-ರು-ವುದೊ
ವ್ಯಾಪ್ತಿ
ವ್ಯಾಪ್ತಿ-ಯಲ್ಲಿ
ವ್ಯಾಪ್ಯ
ವ್ಯಾಮಿ-ಶ್ರೇ-ಣೇವ
ವ್ಯಾಮೋಹ
ವ್ಯಾಮೋ-ಹಕ್ಕೆ
ವ್ಯಾಮೋ-ಹದ
ವ್ಯಾಮೋ-ಹ-ದಂತೆ
ವ್ಯಾಮೋ-ಹ-ದಿಂದ
ವ್ಯಾಮೋ-ಹ-ವನ್ನು
ವ್ಯಾಮೋ-ಹ-ವೆಲ್ಲ
ವ್ಯಾವ
ವ್ಯಾವ-ಹಾ-ರಿಕ
ವ್ಯಾವ-ಹಾ-ರಿ-ಕ-ವಾಗಿ
ವ್ಯಾಸ
ವ್ಯಾಸಃ
ವ್ಯಾಸನೇ
ವ್ಯಾಸ-ಪ್ರ-ಸಾ-ದು-ಚ್ಛ್ರ-ತ-ವಾ-ನಿಮಂ
ವ್ಯಾಸರ
ವ್ಯಾಸ-ರಂ-ತಹ
ವ್ಯಾಸ-ರಿಗೆ
ವ್ಯಾಸರು
ವ್ಯಾಸ-ರು-ಏ-ಳು-ನೂರು
ವ್ಯಾಸ-ರೆಂಬ
ವ್ಯಾಸರೇ
ವ್ಯಾಸೇನ
ವ್ಯಾಹ-ರನ್
ವ್ಯುದಸ್ಯ
ವ್ಯೂಢಂ
ವ್ಯೂಢಾಂ
ವ್ಯೂಹ
ವ್ಯೂಹಾ-ಕಾ-ರ-ದಲ್ಲಿ
ವ್ಯೂಹಾ-ಕಾ-ರ-ವಾಗಿ
ವ್ಯೆಶ್ಯರು
ವ್ರಜ
ವ್ರಜೇತ
ವ್ರಣ
ವ್ರತ
ವ್ರತ-ಗಳು
ವ್ರತ-ದಲ್ಲಿ
ವ್ರತ-ನಿ-ಯ-ಮ-ಗಳು
ವ್ರತ-ವನ್ನು
ವ್ರತಾಃ
ವ್ರಯ-ಮಾ-ಡದೆ
ವ್ರಯ-ಮಾ-ಡುವ
ಶಂಕರ
ಶಂಕ-ರ-ಶ್ಚಾಸ್ಮಿ
ಶಂಕ-ರಾ-ಚಾ-ರ್ಯರ
ಶಂಕ-ರಾ-ಚಾ-ರ್ಯರು
ಶಂಕಿ-ಸ-ಬ-ಹುದು
ಶಂಕೆ
ಶಂಖ
ಶಂಖಂ
ಶಂಖ-ಗಳನ್ನು
ಶಂಖ-ಚ-ಕ್ರ-ಗ-ದಾ-ಪ-ದ್ಮವೇ
ಶಂಖದ
ಶಂಖ-ಧ್ವನಿ
ಶಂಖ-ವನ್ನು
ಶಂಖಾನ್
ಶಂಖಾಶ್ಚ
ಶಂಖೌ
ಶಂಸಸಿ
ಶಕು-ನ-ಗಳನ್ನು
ಶಕು-ನ-ಗಳನ್ನೆಲ್ಲ
ಶಕ್ತ-ನಾ-ಗ-ಲಾರೆ
ಶಕ್ತ-ನಾ-ಗಿ-ದ್ದೇನೆ
ಶಕ್ತ-ನಾ-ಗು-ವನೋ
ಶಕ್ತಿ
ಶಕ್ತಿ-ಗಳ
ಶಕ್ತಿ-ಗಳನ್ನು
ಶಕ್ತಿ-ಗಳು
ಶಕ್ತಿ-ಗ-ಳೆಲ್ಲಾ
ಶಕ್ತಿ-ಗಿಂತ
ಶಕ್ತಿಗೂ
ಶಕ್ತಿಗೆ
ಶಕ್ತಿಯ
ಶಕ್ತಿ-ಯಂತೆ
ಶಕ್ತಿ-ಯನ್ನು
ಶಕ್ತಿ-ಯನ್ನೂ
ಶಕ್ತಿ-ಯ-ನ್ನೆಲ್ಲ
ಶಕ್ತಿ-ಯ-ನ್ನೆಲ್ಲಾ
ಶಕ್ತಿ-ಯನ್ನೇ
ಶಕ್ತಿ-ಯಲ್ಲಿ
ಶಕ್ತಿ-ಯಾ-ಗ-ಬ-ಹುದು
ಶಕ್ತಿ-ಯಾ-ಗ-ಬೇ-ಕಾ-ದರೆ
ಶಕ್ತಿ-ಯಾಗು
ಶಕ್ತಿ-ಯಾ-ಗು-ವುದು
ಶಕ್ತಿ-ಯಾ-ದರೊ
ಶಕ್ತಿ-ಯಿಂದ
ಶಕ್ತಿ-ಯಿಲ್ಲ
ಶಕ್ತಿ-ಯು-ತ-ವಾ-ದುದೇ
ಶಕ್ತಿಯೂ
ಶಕ್ತಿಯೆ
ಶಕ್ತಿ-ಯೆಲ್ಲ
ಶಕ್ತಿಯೇ
ಶಕ್ತಿ-ಯೊಂ-ದಿಗೆ
ಶಕ್ನೋ-ತೀ-ಹೈವ
ಶಕ್ನೋ-ಮ್ಯ-ವ-ಸ್ಥಾ-ತುಂ
ಶಕ್ನೋಷಿ
ಶಕ್ಯ
ಶಕ್ಯಂ
ಶಕ್ಯ-ವಿಲ್ಲ
ಶಕ್ಯಸೇ
ಶಕ್ಯೋ-ಽವಾ-ಪ್ತು-ಮು-ಪಾ-ಯತಃ
ಶಟಲ್
ಶಠೋ
ಶಡ್ವಿ-ಕಾ-ರಾ-ತ್ಮ-ಕ-ವಾದ
ಶತ
ಶತ-ಮಾನ
ಶತ-ಮಾ-ನಕ್ಕೆ
ಶತ-ಮಾ-ನ-ಗಳು
ಶತ-ಮಾ-ನ-ದ-ಲ್ಲಿ-ರುವ
ಶತ-ಶತ
ಶತ-ಶೋಽಥ
ಶತಾ-ನೀಕ
ಶತ್ರು
ಶತ್ರುಂ
ಶತ್ರು-ಗಳ
ಶತ್ರು-ಗಳನ್ನು
ಶತ್ರು-ಗ-ಳಾಗಿ
ಶತ್ರು-ಗ-ಳಾರು
ಶತ್ರು-ಗ-ಳಾರೂ
ಶತ್ರು-ಗ-ಳಿಂ-ದೆಲ್ಲ
ಶತ್ರು-ಗ-ಳಿಗೆ
ಶತ್ರು-ಗ-ಳಿ-ಲ್ಲದೆ
ಶತ್ರು-ಗಳು
ಶತ್ರು-ಗಳೂ
ಶತ್ರು-ಗಳೇ
ಶತ್ರು-ತಾ-ಪ-ನನೇ
ಶತ್ರು-ತ್ವ-ವನ್ನು
ಶತ್ರುತ್ವೇ
ಶತ್ರು-ಪ-ಕ್ಷ-ದಲ್ಲಿ
ಶತ್ರು-ಮಿ-ತ್ರ-ರನ್ನು
ಶತ್ರು-ರ್ಹ-ನಿಷ್ಯೇ
ಶತ್ರು-ವತ್
ಶತ್ರು-ವನ್ನು
ಶತ್ರು-ವಾ-ದರೂ
ಶತ್ರು-ವಾ-ದರೋ
ಶತ್ರು-ವಿದೆ
ಶತ್ರು-ವಿ-ನಂತೆ
ಶತ್ರು-ವಿ-ನಷ್ಟು
ಶತ್ರು-ವಿ-ನೊಂ-ದಿಗೆ
ಶತ್ರುವೂ
ಶತ್ರು-ಶೇಷ
ಶತ್ರು-ಸೇ-ನೆಯ
ಶತ್ರು-ಸೇ-ನೆ-ಯಲ್ಲಿ
ಶತ್ರೂನ್
ಶತ್ರೌ
ಶನಿ
ಶನಿ-ಯನ್ನೊ
ಶನೈಃ-ಶ-ನೈ-ರು-ಪ-ರ-ಮೇ-ದ್ಬುದ್ಧ್ಯಾ
ಶಪಥ
ಶಪ-ಥ-ಗಳ
ಶಪ-ಥ-ಮಾ-ಡಿ-ದರೂ
ಶಪ-ಥ-ಮಾ-ಡು-ವನು
ಶಪ-ಥ-ವನ್ನು
ಶಪಿ-ಸಿ-ಕೊ-ಳ್ಳು-ತ್ತೇನೆ
ಶಪಿ-ಸು-ವೆವು
ಶಬರಿ
ಶಬ್ದ
ಶಬ್ದಃ
ಶಬ್ದ-ಗಳ
ಶಬ್ದ-ಗಳು
ಶಬ್ದ-ಗ-ಳೆ-ಲ್ಲವೂ
ಶಬ್ದ-ದಂತೆ
ಶಬ್ದ-ದಿಂ-ದಲೇ
ಶಬ್ದ-ಬ್ರಹ್ಮ
ಶಬ್ದ-ಬ್ರ-ಹ್ಮ-ನನ್ನು
ಶಬ್ದ-ಬ್ರ-ಹ್ಮ-ವನ್ನು
ಶಬ್ದ-ಬ್ರ-ಹ್ಮಾ-ತಿ-ವ-ರ್ತತೇ
ಶಬ್ದ-ಮಾ-ಡ-ದಂತೆ
ಶಬ್ದ-ಮಾ-ಡಿ-ಕೊಂಡು
ಶಬ್ದ-ಮಾ-ಡು-ವುದು
ಶಬ್ದ-ರೂಪ
ಶಬ್ದ-ರೂ-ಪ-ವನ್ನು
ಶಬ್ದ-ವನ್ನು
ಶಬ್ದ-ವಾ-ಗಲಿ
ಶಬ್ದ-ವಿದೆ
ಶಬ್ದವೇ
ಶಬ್ದ-ಸ್ತು-ಮುಲೋ
ಶಬ್ದಾದಿ
ಶಬ್ದಾ-ದೀನ್
ಶಬ್ದಾ-ದೀ-ನ್ವಿ-ಷ-ಯಾಂ-ಸ್ತ್ಯಕ್ತ್ವಾ
ಶಮ
ಶಮಂ
ಶಮಃ
ಶಮನ
ಶಮ-ನಕ್ಕೆ
ಶಮ-ನ-ವಾ-ಗದು
ಶಮ-ನ-ವಾ-ಗು-ವುದು
ಶಮ-ನ-ವಾ-ದೀತೇ
ಶಮೋ
ಶಯ್ಯೆ-ಯ-ಮೇಲೆ
ಶರಣಂ
ಶರ-ಣ-ಮ-ನ್ವಿಚ್ಛ
ಶರ-ಣಾ-ಗ-ತ-ನಾ-ಗಲು
ಶರ-ಣಾ-ಗ-ತ-ನಾ-ಗು-ವನೊ
ಶರ-ಣಾ-ಗ-ತ-ನಾ-ದರೆ
ಶರ-ಣಾ-ಗ-ತ-ರಾ-ಗ-ಬೇಕು
ಶರ-ಣಾ-ಗ-ತ-ರಾಗಿ
ಶರ-ಣಾ-ಗ-ತ-ರಾ-ದ-ವರು
ಶರ-ಣಾ-ಗತಿ
ಶರ-ಣಾ-ಗ-ತಿಯ
ಶರ-ಣಾ-ಗ-ತಿ-ಯನ್ನು
ಶರ-ಣಾ-ಗ-ತಿ-ಯಿಂದ
ಶರ-ಣಾ-ಗ-ಬ-ಲ್ಲ-ವರು
ಶರ-ಣಾ-ಗ-ಬೇ-ಕಾ-ದರೆ
ಶರ-ಣಾ-ಗರು
ಶರ-ಣಾ-ಗಲು
ಶರ-ಣಾಗಿ
ಶರ-ಣಾ-ಗಿದೆ
ಶರ-ಣಾ-ಗಿ-ದ್ದಾನೆ
ಶರ-ಣಾ-ಗಿ-ದ್ದೇನೆ
ಶರ-ಣಾ-ಗಿ-ದ್ದೇ-ನೆಂದು
ಶರ-ಣಾ-ಗಿ-ರ-ಬೇಕು
ಶರ-ಣಾ-ಗಿ-ರುವ
ಶರ-ಣಾ-ಗಿ-ರು-ವಾಗ
ಶರ-ಣಾ-ಗಿ-ರು-ವು-ದ-ರಿಂದ
ಶರ-ಣಾಗು
ಶರ-ಣಾ-ಗು-ತ್ತಾನೆ
ಶರ-ಣಾ-ಗು-ತ್ತಾ-ನೆಯೋ
ಶರ-ಣಾ-ಗು-ತ್ತಾನೊ
ಶರ-ಣಾ-ಗು-ತ್ತೇ-ವೆಯೊ
ಶರ-ಣಾ-ಗು-ತ್ತೇ-ವೆಯೋ
ಶರ-ಣಾ-ಗು-ವರೊ
ಶರ-ಣಾ-ಗು-ವರೋ
ಶರ-ಣಾ-ಗು-ವವ
ಶರ-ಣಾ-ಗು-ವ-ವನು
ಶರ-ಣಾ-ಗು-ವ-ವರು
ಶರ-ಣಾ-ಗು-ವು-ದಕ್ಕೆ
ಶರ-ಣಾ-ಗು-ವುದು
ಶರ-ಣಾ-ಗು-ವೆವೊ
ಶರ-ಣಾ-ಗು-ವೆವೋ
ಶರ-ಣಾದ
ಶರ-ಣಾ-ದರೆ
ಶರ-ಣಾ-ದರೇ
ಶರ-ಣಾ-ದ-ವನ
ಶರ-ಣಾ-ದ-ವರ
ಶರ-ಣಾ-ದ-ವ-ರನ್ನು
ಶರ-ಣಾದೆ
ಶರಣು
ಶರ-ಣು-ಹೊಂ-ದಿದ
ಶರ-ಣ್ಯ-ರಿಗೆ
ಶರಾ-ಣಾ-ದ-ವನು
ಶರಾ-ವ-ತಿಯ
ಶರಾ-ವ-ತಿ-ಯಲ್ಲಿ
ಶರೀರ
ಶರೀರಂ
ಶರೀ-ರ-ಕ್ಕಿಂತ
ಶರೀ-ರಕ್ಕೆ
ಶರೀ-ರ-ಗಳು
ಶರೀ-ರದ
ಶರೀ-ರ-ದಲ್ಲಿ
ಶರೀ-ರ-ದ-ಲ್ಲಿ-ದ್ದರೂ
ಶರೀ-ರ-ದ-ಲ್ಲಿ-ರುವ
ಶರೀ-ರ-ದಾರ್ಢ್ಯ
ಶರೀ-ರ-ಯಾ-ತ್ರಾಪಿ
ಶರೀ-ರ-ವನ್ನು
ಶರೀ-ರ-ವಾ-ಙ್ಮ-ನೋ-ಭಿ-ರ್ಯ-ತ್ಕರ್ಮ
ಶರೀ-ರ-ವಿ-ಮೋ-ಕ್ಷ-ಣಾತ್
ಶರೀ-ರ-ವೆಂ-ಬುದು
ಶರೀ-ರ-ಶಾಸ್ತ್ರ
ಶರೀ-ರಸ್ಥಂ
ಶರೀ-ರ-ಸ್ಥೋಽಪಿ
ಶರೀರಾ
ಶರೀ-ರಾಣಿ
ಶರೀ-ರಿಣಃ
ಶರೀರೇ
ಶರ್ಮ
ಶಲ್ಯ
ಶಲ್ಯ-ಗ್ರಾ-ಹ-ವತೀ
ಶಲ್ಯನೇ
ಶವದ
ಶವ-ವನ್ನು
ಶವ-ವಾಗಿ
ಶಶಾಂಕಃ
ಶಶಾಂಕೋ
ಶಶಿ-ಸೂ-ರ್ಯ-ನೇ-ತ್ರಮ್
ಶಶಿ-ಸೂ-ರ್ಯ-ಯೋಃ
ಶಶೀ
ಶಶ್ವ-ಚ್ಛಾಂ-ತಿಂ
ಶಸ್ತ್ರ
ಶಸ್ತ್ರ-ಕ್ಕಿಂತ
ಶಸ್ತ್ರ-ಕ್ರಿ-ಯೆಗೆ
ಶಸ್ತ್ರ-ಗಳನ್ನು
ಶಸ್ತ್ರ-ಗಳು
ಶಸ್ತ್ರ-ಚಿ-ಕಿ-ತ್ಸಕ
ಶಸ್ತ್ರ-ಚಿ-ಕಿತ್ಸೆ
ಶಸ್ತ್ರ-ಚಿ-ಕಿ-ತ್ಸೆ-ಯನ್ನು
ಶಸ್ತ್ರ-ಚಿ-ಕಿ-ತ್ಸೆ-ಯಲ್ಲಿ
ಶಸ್ತ್ರ-ದಿಂದ
ಶಸ್ತ್ರ-ಧಾ-ರಿ-ಗಳಲ್ಲಿ
ಶಸ್ತ್ರ-ಪಾ-ಣಯಃ
ಶಸ್ತ್ರ-ಪಾ-ಣಿ-ಗ-ಳಾದ
ಶಸ್ತ್ರ-ಪ್ರ-ಯೋ-ಗ-ದಿಂದ
ಶಸ್ತ್ರ-ಭೃ-ತಾ-ಮ-ಹಮ್
ಶಸ್ತ್ರ-ವನ್ನು
ಶಸ್ತ್ರ-ವ-ನ್ನೆತ್ತಿ
ಶಸ್ತ್ರ-ವಿದ್ಯಾ
ಶಸ್ತ್ರ-ಸಂ-ಪಾತೇ
ಶಸ್ತ್ರ-ಸ-ನ್ನದ್ಧ
ಶಸ್ತ್ರಾಣಿ
ಶಸ್ತ್ರಾ-ಸ್ತ್ರ-ಗಳ
ಶಸ್ತ್ರಾ-ಸ್ತ್ರ-ಗಳನ್ನು
ಶಸ್ತ್ರಾ-ಸ್ತ್ರ-ಗಳಲ್ಲಿ
ಶಸ್ತ್ರಾ-ಸ್ತ್ರ-ಗಳು
ಶಾಂತ
ಶಾಂತ-ಚಿ-ತ್ತ-ರಾ-ಗಿ-ದ್ದಾಗ
ಶಾಂತ-ನಾಗಿ
ಶಾಂತ-ನಾ-ಗಿ-ರ-ಬೇಕು
ಶಾಂತಮು
ಶಾಂತ-ರ-ಜಸಂ
ಶಾಂತ-ರೂ-ಪ-ವನ್ನು
ಶಾಂತ-ವಾಗಿ
ಶಾಂತ-ವಾ-ಗಿತ್ತು
ಶಾಂತ-ವಾ-ಗಿದೆ
ಶಾಂತ-ವಾ-ಗಿ-ದ್ದರೆ
ಶಾಂತ-ವಾ-ಗಿ-ರ-ಬೇಕು
ಶಾಂತ-ವಾ-ಗಿರು
ಶಾಂತ-ವಾ-ಗಿ-ರು-ವಾಗ
ಶಾಂತ-ವಾ-ಗಿ-ರು-ವು-ದಕ್ಕೆ
ಶಾಂತ-ವಾ-ಗಿ-ರು-ವುದು
ಶಾಂತ-ವಾ-ಗು-ವು-ದಿಲ್ಲ
ಶಾಂತ-ವಾ-ಗು-ವುದು
ಶಾಂತ-ವಾ-ಗು-ವುದೊ
ಶಾಂತ-ವಾ-ದೊ-ಡನೆ
ಶಾಂತಿ
ಶಾಂತಿಂ
ಶಾಂತಿಗೆ
ಶಾಂತಿ-ಪ-ರ್ವ-ದಲ್ಲಿ
ಶಾಂತಿ-ಮ-ಚಿ-ರೇ-ಣಾ-ಧಿ-ಗ-ಚ್ಛತಿ
ಶಾಂತಿ-ಮ-ಧಿ-ಗ-ಚ್ಛತಿ
ಶಾಂತಿ-ಮಾ-ಪ್ನೋತಿ
ಶಾಂತಿ-ಮೃ-ಚ್ಛತಿ
ಶಾಂತಿಯ
ಶಾಂತಿ-ಯಂ-ತಿದೆ
ಶಾಂತಿ-ಯನ್ನು
ಶಾಂತಿ-ಯಾ-ದರೂ
ಶಾಂತಿ-ಯಿಂದ
ಶಾಂತಿ-ಯಿಲ್ಲ
ಶಾಂತಿಯೂ
ಶಾಂತಿ-ಯೊಂದೇ
ಶಾಂತಿ-ರ-ಪೈ-ಶು-ನಮ್
ಶಾಂತಿ-ರ-ಶಾಂ-ತಸ್ಯ
ಶಾಂತೋ
ಶಾಕ್ತ
ಶಾಖ
ಶಾಖದ
ಶಾಖ-ಯು-ಕ್ತ-ವಾ-ಗಿಯೂ
ಶಾಖ-ವನ್ನು
ಶಾಖ-ವಾ-ಗಿ-ದೆಯೆ
ಶಾಖಾ
ಶಾಖೆ-ಗ-ಳಾ-ಗಿವೆ
ಶಾಖೆ-ಗ-ಳಿಲ್ಲ
ಶಾಖೆಗೆ
ಶಾಖೋ-ಪ-ಶಾ-ಖೆ-ಗಳು
ಶಾಖೋ-ಪ-ಶಾ-ಖೆ-ಗ-ಳೆಲ್ಲ
ಶಾಜ-ಹಾನ್
ಶಾಧಿ
ಶಾಪ
ಶಾರ-ದಾ-ದೇವಿ
ಶಾರ-ದಾ-ದೇ-ವಿ-ಯ-ವರು
ಶಾರದೆ
ಶಾರೀರಂ
ಶಾರೀ-ರಕ
ಶಾರೀ-ರ-ಕ-ವಾಗಿ
ಶಾಲಿ-ಗಳ
ಶಾಲೆ
ಶಾಲೆಗೆ
ಶಾಲೆಯ
ಶಾಲೆ-ಯನ್ನು
ಶಾಲೆ-ಯಲ್ಲಿ
ಶಾಶ್ವತ
ಶಾಶ್ವತಂ
ಶಾಶ್ವ-ತ-ಧ-ರ್ಮ-ಗೋಪ್ತಾ
ಶಾಶ್ವ-ತ-ಧ-ರ್ಮ-ವನ್ನು
ಶಾಶ್ವ-ತ-ನಾ-ಗಿ-ರು-ವುದು
ಶಾಶ್ವ-ತ-ನೆಂದೂ
ಶಾಶ್ವ-ತಮ್
ಶಾಶ್ವ-ತ-ವಲ್ಲ
ಶಾಶ್ವ-ತ-ವಾಗಿ
ಶಾಶ್ವ-ತ-ವಾ-ಗಿ-ಟ್ಟಿ-ರು-ವು-ದಕ್ಕೆ
ಶಾಶ್ವ-ತ-ವಾ-ಗಿರು
ಶಾಶ್ವ-ತ-ವಾ-ಗಿ-ರು-ವಂ-ತ-ಹು-ದಲ್ಲ
ಶಾಶ್ವ-ತ-ವಾ-ಗಿ-ರು-ವು-ದ-ಕ್ಕಾ-ಗು-ವು-ದಿಲ್ಲ
ಶಾಶ್ವ-ತ-ವಾ-ಗಿ-ರು-ವು-ದಿಲ್ಲ
ಶಾಶ್ವ-ತ-ವಾ-ಗಿ-ರು-ವುದು
ಶಾಶ್ವ-ತ-ವಾ-ಗಿವೆ
ಶಾಶ್ವ-ತ-ವಾದ
ಶಾಶ್ವ-ತ-ವಾ-ದುದು
ಶಾಶ್ವ-ತವೂ
ಶಾಶ್ವ-ತಸ್ಯ
ಶಾಶ್ವ-ತಾಃ
ಶಾಶ್ವ-ತೀಃ
ಶಾಶ್ವತೇ
ಶಾಶ್ವ-ತೋಯಂ
ಶಾಸನ
ಶಾಸ-ನ-ಕರ್ತೃ
ಶಾಸ-ನಕ್ಕೆ
ಶಾಸ-ನ-ಗಳ
ಶಾಸ-ನದ
ಶಾಸ-ನ-ದಲ್ಲಿ
ಶಾಸ-ನ-ದಿಂದ
ಶಾಸ-ನ-ವನ್ನು
ಶಾಸಿ-ಸು-ವ-ವ-ರಲ್ಲಿ
ಶಾಸ್ತ್ರ
ಶಾಸ್ತ್ರಕ್ಕೂ
ಶಾಸ್ತ್ರಕ್ಕೆ
ಶಾಸ್ತ್ರ-ಗಳ
ಶಾಸ್ತ್ರ-ಗಳನ್ನು
ಶಾಸ್ತ್ರ-ಗಳನ್ನೂ
ಶಾಸ್ತ್ರ-ಗ-ಳಿಗೆ
ಶಾಸ್ತ್ರ-ಗಳು
ಶಾಸ್ತ್ರದ
ಶಾಸ್ತ್ರ-ದಲ್ಲಿ
ಶಾಸ್ತ್ರ-ಮಿ-ದ-ಮುಕ್ತಂ
ಶಾಸ್ತ್ರ-ವನ್ನು
ಶಾಸ್ತ್ರ-ವಾ-ಗಲಿ
ಶಾಸ್ತ್ರ-ವಾ-ದರೋ
ಶಾಸ್ತ್ರ-ವಿ-ಧಾ-ನೋಕ್ತಂ
ಶಾಸ್ತ್ರ-ವಿ-ಧಿ-ಮು-ತ್ಸೃಜ್ಯ
ಶಾಸ್ತ್ರ-ವಿ-ಧಿ-ಯನ್ನು
ಶಾಸ್ತ್ರ-ವಿ-ಹಿ-ತ-ವ-ಲ್ಲದ
ಶಾಸ್ತ್ರವೇ
ಶಾಸ್ತ್ರ-ಸ-ಮ್ಮ-ತ-ವಾದ
ಶಾಸ್ತ್ರ-ಸಾ-ರವೇ
ಶಾಸ್ತ್ರಾದಿ
ಶಾಸ್ತ್ರಾ-ದಿ-ಗಳನ್ನು
ಶಾಸ್ತ್ರಾ-ದಿ-ಗಳಿಂದ
ಶಾಸ್ತ್ರಾ-ದಿ-ಗ-ಳಿಗೆ
ಶಾಸ್ತ್ರಾ-ದಿ-ಗಳು
ಶಾಸ್ತ್ರಾ-ದಿ-ಗ-ಳೆಲ್ಲ
ಶಾಸ್ತ್ರೀ-ಯ-ವಾಗಿ
ಶಿಕ್ಷಿ-ಸದೆ
ಶಿಕ್ಷಿ-ಸಿದ
ಶಿಕ್ಷಿ-ಸಿ-ದರೆ
ಶಿಕ್ಷಿ-ಸು-ತ್ತಾನೆ
ಶಿಕ್ಷಿ-ಸು-ತ್ತಿಲ್ಲ
ಶಿಕ್ಷಿ-ಸು-ವು-ದಕ್ಕೆ
ಶಿಕ್ಷಿ-ಸು-ವು-ದಿಲ್ಲ
ಶಿಕ್ಷಿ-ಸೆಂದು
ಶಿಕ್ಷೆ
ಶಿಕ್ಷೆಗೆ
ಶಿಕ್ಷೆಯ
ಶಿಕ್ಷೆ-ಯಂತೆ
ಶಿಕ್ಷೆ-ಯನ್ನು
ಶಿಕ್ಷೆ-ಯನ್ನೋ
ಶಿಕ್ಷೆ-ಯಿಂದ
ಶಿಖಂಡಿ
ಶಿಖಂಡೀ
ಶಿಖರ
ಶಿಖ-ರಕ್ಕೆ
ಶಿಖ-ರ-ಗಳಲ್ಲಿ
ಶಿಖ-ರ-ಗ-ಳಿವೆ
ಶಿಖ-ರ-ಗಳು
ಶಿಖ-ರ-ಗ-ಳುಳ್ಳ
ಶಿಖ-ರ-ಗ-ಳೆ-ಷ್ಟೊಂದು
ಶಿಖ-ರದ
ಶಿಖ-ರ-ದಷ್ಟು
ಶಿಖ-ರ-ವನ್ನು
ಶಿಖ-ರಿ-ಣಾ-ಮ-ಹಮ್
ಶಿಖೆಗೆ
ಶಿಬಿ
ಶಿರಕ್ಕೆ
ಶಿರದ
ಶಿರ-ದಲ್ಲಿ
ಶಿರವೋ
ಶಿರ-ಶಾ-ಸ-ನ-ದಲ್ಲೆ
ಶಿರಸಾ
ಶಿರ-ಸಾ-ಧ-ರಿಸಿ
ಶಿರಸ್ಸು
ಶಿರೋ-ಮ-ಧ್ಯ-ದ-ಲ್ಲಿ-ರು-ವುದು
ಶಿಲಾ-ಪ್ರ-ತಿ-ಮೆ-ಯಂತೆ
ಶಿಲೆಗೆ
ಶಿಲ್ಪಿ
ಶಿಲ್ಪಿ-ಗಳು
ಶಿಲ್ಪಿಯ
ಶಿಲ್ಪಿ-ಯಾ-ದರೋ
ಶಿವ
ಶಿವನ
ಶಿವ-ನನ್ನು
ಶಿವ-ನಿಂದ
ಶಿವ-ನೊ-ಡನೆ
ಶಿವ-ಮ-ಹಿಮ್ನಾ
ಶಿವ-ರಾತ್ರಿ
ಶಿಶು
ಶಿಶು-ಪಾಲ
ಶಿಶು-ಪಾ-ಲನ
ಶಿಷ್ಟಾ-ಮೃ-ತ-ಭುಜೋ
ಶಿಷ್ಯ
ಶಿಷ್ಯನ
ಶಿಷ್ಯ-ನಂತೆ
ಶಿಷ್ಯ-ನನ್ನು
ಶಿಷ್ಯ-ನ-ಲ್ಲ-ದ-ವ-ನಿಗೆ
ಶಿಷ್ಯ-ನಲ್ಲಿ
ಶಿಷ್ಯ-ನ-ಲ್ಲಿ-ರ-ಬೇ-ಕಾದ
ಶಿಷ್ಯ-ನ-ಲ್ಲಿ-ರ-ಬೇಕು
ಶಿಷ್ಯ-ನಾ-ಗ-ಬೇಕು
ಶಿಷ್ಯ-ನಾಗಿ
ಶಿಷ್ಯ-ನಾ-ಗಿ-ದ್ದನು
ಶಿಷ್ಯ-ನಾ-ಗಿ-ದ್ದಾನೆ
ಶಿಷ್ಯ-ನಾ-ಗು-ತ್ತಾ-ನೆಯೋ
ಶಿಷ್ಯ-ನಾ-ಗು-ವುದು
ಶಿಷ್ಯ-ನಾದ
ಶಿಷ್ಯ-ನಾ-ದರೆ
ಶಿಷ್ಯ-ನಿಂದ
ಶಿಷ್ಯ-ನಿಗೆ
ಶಿಷ್ಯನೇ
ಶಿಷ್ಯ-ನೊಬ್ಬ
ಶಿಷ್ಯರ
ಶಿಷ್ಯ-ರನ್ನು
ಶಿಷ್ಯ-ರಲ್ಲಿ
ಶಿಷ್ಯ-ರಾ-ಗ-ಬೇಕು
ಶಿಷ್ಯ-ರಿಗೆ
ಶಿಷ್ಯ-ರಿ-ದ್ದಾರೆ
ಶಿಷ್ಯರು
ಶಿಷ್ಯ-ಸ್ತೇಹಂ
ಶಿಷ್ಯೇಣ
ಶಿಸ್ತು
ಶೀಘ್ರ-ವಾಗಿ
ಶೀತ
ಶೀತ-ವಾ-ಗಿ-ದೆಯೆ
ಶೀತೋಷ್ಣ
ಶೀತೋ-ಷ್ಣ-ಗಳಲ್ಲಿ
ಶೀತೋ-ಷ್ಣ-ಗಳು
ಶೀತೋ-ಷ್ಣ-ದಲ್ಲಿ
ಶೀತೋ-ಷ್ಣ-ದಿಂದ
ಶೀತೋ-ಷ್ಣ-ಸು-ಖ-ದುಃ-ಖ-ದಾಃ
ಶೀತೋ-ಷ್ಣ-ಸು-ಖ-ದುಃ-ಖೇಷು
ಶೀಲ
ಶೀಲದ
ಶೀಲ-ದಲ್ಲಿ
ಶೀಲ-ದ-ಲ್ಲಿಯೂ
ಶೀಲರೂ
ಶೀಲ-ವನ್ನು
ಶುಂಠಿ-ಕಾಯಿ
ಶುಕ
ಶುಕ-ನಂತೆ
ಶುಕ-ನನ್ನು
ಶುಕ-ನಿಗೆ
ಶುಕ್ರ-ವರ್ಮ
ಶುಕ್ರಾ-ಚಾ-ರ್ಯನ
ಶುಕ್ರಾ-ಚಾ-ರ್ಯರು
ಶುಕ್ಲಃ
ಶುಕ್ಲ-ಕೃಷ್ಣೇ
ಶುಕ್ಲ-ಪಕ್ಷ
ಶುಕ್ಲ-ಪ-ಕ್ಷದ
ಶುಚಃ
ಶುಚಿ
ಶುಚಿ-ಯಲ್ಲ
ಶುಚಿಯಾ
ಶುಚಿ-ಯಾಗಿ
ಶುಚಿ-ಯಾ-ಗಿ-ಟ್ಟಿ-ರ-ಬೇ-ಕಾ-ದರೆ
ಶುಚಿ-ಯಾ-ಗಿ-ಟ್ಟಿ-ರ-ಬೇಕು
ಶುಚಿ-ಯಾ-ಗಿ-ಟ್ಟಿರು
ಶುಚಿ-ಯಾ-ಗಿ-ಡ-ಬೇ-ಕಾ-ದರೆ
ಶುಚಿ-ಯಾ-ಗಿ-ಡು-ತ್ತಾನೆ
ಶುಚಿ-ಯಾ-ಗಿ-ಡು-ವು-ದಲ್ಲ
ಶುಚಿ-ಯಾ-ಗಿ-ಡು-ವುದು
ಶುಚಿ-ಯಾ-ಗಿ-ರ-ಬೇಕು
ಶುಚಿ-ಯಾ-ಗಿ-ರು-ತ್ತಾನೆ
ಶುಚಿ-ಯಾ-ಗಿ-ರುವ
ಶುಚಿ-ಯಾ-ಗಿ-ರು-ವನು
ಶುಚಿ-ಯಾ-ಗಿ-ರು-ವು-ದಲ್ಲ
ಶುಚಿ-ಯಾ-ಗಿ-ರು-ವು-ದಿಲ್ಲ
ಶುಚಿ-ಯಾ-ಗಿ-ರು-ವುದು
ಶುಚಿ-ಯಾ-ಗಿ-ರು-ವುವು
ಶುಚಿ-ಯಾದ
ಶುಚಿಯೋ
ಶುಚಿ-ರ್ದಕ್ಷ
ಶುಚೀ-ನಾಂ
ಶುಚೌ
ಶುದ್ಧ
ಶುದ್ಧ-ಗೊ-ಳಿಸಿ
ಶುದ್ಧ-ಚಾ-ರಿ-ತ್ರ್ಯದ
ಶುದ್ಧ-ಚಾ-ರಿ-ತ್ರ್ಯವೇ
ಶುದ್ಧ-ಚಿ-ತ್ತಕ್ಕೆ
ಶುದ್ಧ-ಮಾಡಿ
ಶುದ್ಧ-ಮಾ-ಡಿ-ಕೊಂ-ಡಿ-ರು-ವನೊ
ಶುದ್ಧ-ಮಾ-ಡಿ-ಕೊಂ-ಡಿ-ರು-ವರು
ಶುದ್ಧ-ಮಾ-ಡಿ-ಕೊ-ಳ್ಳ-ಬೇ-ಕಾ-ದರೆ
ಶುದ್ಧ-ಮಾ-ಡಿ-ದ-ಲ್ಲದೆ
ಶುದ್ಧ-ಮಾ-ಡು-ವಂತೆ
ಶುದ್ಧ-ಮಾ-ಡು-ವುದು
ಶುದ್ಧ-ವ-ಲ್ಲದ
ಶುದ್ಧ-ವಾಗಿ
ಶುದ್ಧ-ವಾ-ಗಿ-ಟ್ಟು-ಕೊಂ-ಡಿ-ರು-ವು-ದಿಲ್ಲ
ಶುದ್ಧ-ವಾ-ಗಿದೆ
ಶುದ್ಧ-ವಾ-ಗಿ-ದೆಯೊ
ಶುದ್ಧ-ವಾ-ಗಿ-ದೆಯೋ
ಶುದ್ಧ-ವಾ-ಗಿ-ದ್ದರೆ
ಶುದ್ಧ-ವಾ-ಗಿ-ರ-ಬ-ಲ್ಲದು
ಶುದ್ಧ-ವಾ-ಗಿ-ರ-ಬೇಕು
ಶುದ್ಧ-ವಾ-ಗಿ-ರುವ
ಶುದ್ಧ-ವಾ-ಗಿ-ರು-ವುದು
ಶುದ್ಧ-ವಾ-ಗಿಲ್ಲ
ಶುದ್ಧ-ವಾ-ಗಿ-ಲ್ಲವೋ
ಶುದ್ಧ-ವಾ-ಗುತ್ತ
ಶುದ್ಧ-ವಾ-ಗು-ವುದು
ಶುದ್ಧ-ವಾ-ಗು-ವುದೋ
ಶುದ್ಧ-ವಾದ
ಶುದ್ಧ-ವಾ-ದಂತೆ
ಶುದ್ಧ-ವಾ-ದ-ಮೇಲೆ
ಶುದ್ಧ-ವಾ-ದರೆ
ಶುದ್ಧ-ವಿ-ಲ್ಲದೆ
ಶುದ್ಧಿ
ಶುದ್ಧಿ-ಗಾಗಿ
ಶುದ್ಧಿ-ಗೊ-ಳಿಸು
ಶುದ್ಧಿ-ಮಾ-ಡದೆ
ಶುದ್ಧಿ-ಮಾ-ಡ-ಬೇಕು
ಶುದ್ಧಿ-ಮಾಡಿ
ಶುದ್ಧಿ-ಮಾ-ಡಿ-ಕೊಂ-ಡಿ-ರು-ವನು
ಶುದ್ಧಿ-ಮಾ-ಡಿ-ಕೊಂ-ಡಿಲ್ಲ
ಶುದ್ಧಿ-ಮಾ-ಡಿ-ಕೊ-ಳ್ಳು-ವು-ದಕ್ಕೆ
ಶುದ್ಧಿ-ಮಾ-ಡು-ವುದು
ಶುದ್ಧಿ-ಮಾ-ಡು-ವುವು
ಶುದ್ಧಿ-ಯನ್ನು
ಶುದ್ಧಿ-ಯಾ-ಗ-ಬೇಕು
ಶುದ್ಧಿ-ಯಾಗಿ
ಶುದ್ಧಿ-ಯಾ-ಗಿದೆ
ಶುದ್ಧಿ-ಯಾಗಿಯೂ
ಶುದ್ಧಿ-ಯಾ-ಗಿ-ರ-ಬೇಕು
ಶುದ್ಧಿ-ಯಾ-ಗುತ್ತ
ಶುದ್ಧಿ-ಯಾ-ಗು-ವುದು
ಶುದ್ಧಿ-ಯಾ-ಯಿ-ತೆಂ-ದರೆ
ಶುನಿ
ಶುಭ
ಶುಭ-ಕ-ರ್ಮ-ವನ್ನು
ಶುಭದ
ಶುಭ-ವನ್ನು
ಶುಭ-ವಾದ
ಶುಭಾಂ-ಲ್ಲೋ-ಕಾನ್
ಶುಭಾ-ಶಯ
ಶುಭಾ-ಶು-ಭ-ಪ-ರಿ-ತ್ಯಾಗೀ
ಶುಭಾ-ಶು-ಭ-ಫ-ಲೈ-ರೇವಂ
ಶುಭಾ-ಶು-ಭಮ್
ಶುಭೇಚ್ಛೆ
ಶುಭ್ರತೆ
ಶುಭ್ರ-ವಾಗಿ
ಶುಭ್ರ-ವಾ-ಗಿ-ದ್ದರೆ
ಶುಭ್ರ-ವಾ-ಗಿ-ರು-ವುದು
ಶುಭ್ರ-ವಾದ
ಶುರು
ಶುರು-ಮಾ-ಡು-ವನು
ಶುರು-ವಾ-ಗ-ಬೇ-ಕಾ-ಗಿದೆ
ಶುರು-ವಾ-ದರೆ
ಶುರು-ವಾ-ದಾಗ
ಶುಶ್ರೂಷೆ
ಶುಶ್ರೂ-ಷೆ-ಯಿಂದ
ಶೂದ್ರ
ಶೂದ್ರನ
ಶೂದ್ರ-ನಿಗೆ
ಶೂದ್ರರ
ಶೂದ್ರ-ರಿಗೆ
ಶೂದ್ರರು
ಶೂದ್ರರೂ
ಶೂದ್ರ-ಸ್ಯಾಪಿ
ಶೂದ್ರಾ-ಣಾಂ
ಶೂದ್ರಾ-ಸ್ತೇಽಪಿ
ಶೂನ್ಯ
ಶೂನ್ಯ-ದಿಂದ
ಶೂನ್ಯವೋ
ಶೂರ
ಶೂರ-ನಾದ
ಶೂರ-ರಿಂದ
ಶೂರ-ರಿಗೆ
ಶೂರರು
ಶೂರಾ
ಶೃಂಖಲೆ
ಶೃಂಗಾರ
ಶೃಣು
ಶೃಣು-ಯಾ-ದಪಿ
ಶೃಣೋತಿ
ಶೃಣ್ವತೋ
ಶೃಣ್ವ-ನ್ಸ್ಪೃ-ಶನ್
ಶೇಕಡ
ಶೇಖ-ರ-ವಾ-ಗು-ತ್ತಿ-ರ-ಬೇಕು
ಶೇಖ-ರಿಸಿ
ಶೇಖ-ರಿ-ಸಿ-ಡು-ವನು
ಶೇಖ-ರಿ-ಸುತ್ತಾ
ಶೇಷ
ಶೇಷ-ವೆಂದು
ಶೇಷವೇ
ಶೈಬ್ಯ-ನೆನ್ನು
ಶೈಬ್ಯಶ್ಚ
ಶೈಭ್ಯ
ಶೈಲಿಯ
ಶೈಲಿ-ಯಲ್ಲಿ
ಶೈವ-ರಿಗೆ
ಶೈಶ-ವಾ-ವ-ಸ್ಥೆ-ಯ-ಲ್ಲಿ-ರು-ವಾಗ
ಶೋಕ
ಶೋಕಂ
ಶೋಕಕ್ಕೆ
ಶೋಕ-ಕ್ಕೆಲ್ಲ
ಶೋಕ-ಗಳಿಂದ
ಶೋಕ-ಗಳು
ಶೋಕದ
ಶೋಕ-ದಲ್ಲಿ
ಶೋಕ-ದಿಂದ
ಶೋಕ-ವನ್ನು
ಶೋಕ-ವೇಕೆ
ಶೋಕ-ಸಂ-ತ-ಪ್ತ-ನಾ-ಗಿ-ರು-ವು-ದ-ರಿಂದ
ಶೋಕ-ಸಂ-ವಿ-ಗ್ನ-ಮಾ-ನಸಃ
ಶೋಕಾ-ಕು-ಲ-ನಾ-ಗಿ-ರು-ವಾಗ
ಶೋಕಿ
ಶೋಕಿ-ಯಲ್ಲ
ಶೋಕಿ-ಸ-ಬಾ-ರದು
ಶೋಕಿ-ಸ-ಬೇಡ
ಶೋಕಿಸು
ಶೋಕಿ-ಸು-ತ್ತಿ-ದ್ದೀಯೆ
ಶೋಕಿ-ಸು-ವು-ದಿಲ್ಲ
ಶೋಕಿ-ಸು-ವು-ದಿ-ಲ್ಲವೊ
ಶೋಕಿ-ಸು-ವುದೂ
ಶೋಚತಿ
ಶೋಚ-ನೀಯ
ಶೋಚ-ನೀ-ಯ-ವಾ-ಗು-ವುದು
ಶೋಚ-ನೀ-ಯ-ವಾದ
ಶೋಚಿ-ತು-ಮ-ರ್ಹಸಿ
ಶೋಧಿ-ಸಿ-ಕೊಳ್ಳ
ಶೋಧಿ-ಸಿ-ಕೊ-ಳ್ಳ-ಬೇ-ಕಾ-ದರೆ
ಶೋಧಿ-ಸಿ-ದರೆ
ಶೋಧಿ-ಸು-ವರು
ಶೋಭಿ-ಸು-ವುದು
ಶೋಷ-ಯತಿ
ಶೌಚ
ಶೌಚಂ
ಶೌಚ-ಮ-ದ್ರೋಹೋ
ಶೌಚ-ಮಾ-ರ್ಜ-ವಮ್
ಶೌಚ-ವನ್ನು
ಶೌಚ-ವಿ-ರ-ಬೇಕು
ಶೌಚ-ವಿಲ್ಲ
ಶೌರ-ಸೇ-ನಿ-ಯರು
ಶೌರ್ಯ
ಶೌರ್ಯಂ
ಶೌರ್ಯ-ವಾ-ಗಿ-ರ-ಬ-ಹುದು
ಶ್ಚರ್ಯ-ಕ-ರ-ವಾ-ಗಿದೆ
ಶ್ಯಕ-ವಾದ
ಶ್ಯಾಲಾಃ
ಶ್ರದ್ದ-ಧಾನಾ
ಶ್ರದ್ದೆ
ಶ್ರದ್ಧಯಾ
ಶ್ರದ್ಧ-ಯಾ-ನ್ವಿ-ತಾಃ
ಶ್ರದ್ಧ-ಯಾ-ರ್ಚಿ-ತು-ಮಿ-ಚ್ಛತಿ
ಶ್ರದ್ಧ-ಯೋ-ಪೇತೋ
ಶ್ರದ್ಧಾ
ಶ್ರದ್ಧಾಂ
ಶ್ರದ್ಧಾ-ತ್ರ-ಯ-ವಿ-ಭಾ-ಗ-ಯೋಗ
ಶ್ರದ್ಧಾ-ಪೂ-ರ್ವಕ
ಶ್ರದ್ಧಾ-ಭ-ಕ್ತಿ-ಯಿಂದ
ಶ್ರದ್ಧಾ-ಮಯ
ಶ್ರದ್ಧಾ-ಮ-ಯೋಽಯಂ
ಶ್ರದ್ಧಾ-ಯು-ಕ್ತ-ನಾ-ಗಿ-ದ್ದರೂ
ಶ್ರದ್ಧಾ-ಯು-ಕ್ತ-ನಾ-ದರೂ
ಶ್ರದ್ಧಾ-ರ-ಹಿ-ತ-ವಾದ
ಶ್ರದ್ಧಾ-ವಂ-ತ-ನಾ-ಗಿ-ರ-ಬೇಕು
ಶ್ರದ್ಧಾ-ವಂ-ತನೂ
ಶ್ರದ್ಧಾ-ವಂ-ತೋ-ಽನ-ಸೂ-ಯಂತೋ
ಶ್ರದ್ಧಾ-ವಾಂ-ಲ್ಲ-ಭತೇ
ಶ್ರದ್ಧಾ-ವಾ-ನ-ನ-ಸೂ-ಯಶ್ಚ
ಶ್ರದ್ಧಾ-ವಾನ್
ಶ್ರದ್ಧಾ-ವಿ-ರ-ಹಿತಂ
ಶ್ರದ್ಧೆ
ಶ್ರದ್ಧೆಗೂ
ಶ್ರದ್ಧೆಗೆ
ಶ್ರದ್ಧೆಯ
ಶ್ರದ್ಧೆ-ಯನ್ನು
ಶ್ರದ್ಧೆ-ಯಿಂದ
ಶ್ರದ್ಧೆ-ಯಿಟ್ಟು
ಶ್ರದ್ಧೆ-ಯಿಲ್ಲ
ಶ್ರದ್ಧೆ-ಯಿ-ಲ್ಲ-ದ-ವನು
ಶ್ರದ್ಧೆ-ಯು-ಳ್ಳ-ವನು
ಶ್ರದ್ಧೆ-ಯು-ಳ್ಳ-ವ-ರಾಗಿ
ಶ್ರದ್ಧೆಯೂ
ಶ್ರದ್ಧೆ-ಯೆಂಬ
ಶ್ರದ್ಧೆಯೇ
ಶ್ರದ್ಧೆ-ಯೊಂ-ದಿದೆ
ಶ್ರದ್ಧೆ-ಯೊಂ-ದಿ-ದ್ದರೆ
ಶ್ರಮ
ಶ್ರಮಕ್ಕೆ
ಶ್ರಮ-ಜೀ-ವಿ-ಗಳ
ಶ್ರಮ-ಜೀ-ವಿ-ಗಳು
ಶ್ರಮದ
ಶ್ರಮ-ದಲ್ಲಿ
ಶ್ರಮ-ದಾ-ನ-ವಿ-ರ-ಬ-ಹುದು
ಶ್ರಮ-ದಿಂದ
ಶ್ರಮ-ದಿಂ-ದಲೆ
ಶ್ರಮ-ಪ-ಟ್ಟ-ವ-ನಿಗೆ
ಶ್ರಮ-ಪಟ್ಟು
ಶ್ರಮ-ಪ-ಡ-ಬೇಕೆ
ಶ್ರಮ-ವನ್ನು
ಶ್ರಮ-ವಾಗ
ಶ್ರಮ-ವಾ-ಗ-ಬ-ಹುದು
ಶ್ರಮ-ವಾ-ಗಿ-ರ-ಬ-ಹುದು
ಶ್ರಮ-ವಿಲ್ಲ
ಶ್ರಮ-ಸ-ಹಿಷ್ಣು
ಶ್ರಮಿ-ಸು-ತ್ತಿ-ರುವ
ಶ್ರಮಿ-ಸು-ತ್ತಿ-ರು-ವನೋ
ಶ್ರಮಿ-ಸು-ವು-ದಕ್ಕೆ
ಶ್ರವಸ್ಸು
ಶ್ರಾದ್ಧ
ಶ್ರಾದ್ಧ-ತ-ರ್ಪ-ಣ-ಗ-ಳಿ-ಲ್ಲದೆ
ಶ್ರಾದ್ಧ-ತ-ರ್ಪ-ಣಾ-ದಿ-ಗಳು
ಶ್ರಿತಾಃ
ಶ್ರೀ
ಶ್ರೀಕೃಷ್ಣ
ಶ್ರೀಕೃ-ಷ್ಣ-ಅ-ರ್ಜು-ನರ
ಶ್ರೀಕೃ-ಷ್ಣ-ದೇಹ
ಶ್ರೀಕೃ-ಷ್ಣ-ದ್ವೈ-ಪಾ-ಯ-ನರು
ಶ್ರೀಕೃ-ಷ್ಣನ
ಶ್ರೀಕೃ-ಷ್ಣ-ನಂ-ತಹ
ಶ್ರೀಕೃ-ಷ್ಣ-ನಂತೆ
ಶ್ರೀಕೃ-ಷ್ಣ-ನಂ-ಥ-ವನೇ
ಶ್ರೀಕೃ-ಷ್ಣ-ನನ್ನು
ಶ್ರೀಕೃ-ಷ್ಣ-ನಲ್ಲಿ
ಶ್ರೀಕೃ-ಷ್ಣ-ನ-ಲ್ಲಿ-ರುವ
ಶ್ರೀಕೃ-ಷ್ಣ-ನಷ್ಟು
ಶ್ರೀಕೃ-ಷ್ಣ-ನಾ-ದರೋ
ಶ್ರೀಕೃ-ಷ್ಣ-ನಿಂದ
ಶ್ರೀಕೃ-ಷ್ಣ-ನಿಗೆ
ಶ್ರೀಕೃ-ಷ್ಣ-ನಿ-ಗೊ-ಬ್ಬ-ನಿಗೇ
ಶ್ರೀಕೃ-ಷ್ಣ-ನಿ-ರು-ವನೊ
ಶ್ರೀಕೃ-ಷ್ಣ-ನಿ-ರು-ವನೋ
ಶ್ರೀಕೃ-ಷ್ಣನು
ಶ್ರೀಕೃ-ಷ್ಣನೂ
ಶ್ರೀಕೃ-ಷ್ಣನೆ
ಶ್ರೀಕೃ-ಷ್ಣ-ನೆಂಬ
ಶ್ರೀಕೃ-ಷ್ಣನೇ
ಶ್ರೀಕೃ-ಷ್ಣರು
ಶ್ರೀಕೃ-ಷ್ಣಾ-ರ್ಜುನ
ಶ್ರೀಕೃ-ಷ್ಣಾ-ರ್ಜು-ನರ
ಶ್ರೀಕೃ-ಷ್ಣಾ-ರ್ಜು-ನರು
ಶ್ರೀಕೃ-ಷ್ಣಾ-ರ್ಪ-ಣ-ಮಸ್ತು
ಶ್ರೀಕೃ-ಷ್ಣಾ-ರ್ಪ-ಣ-ವಾ-ಗಲಿ
ಶ್ರೀಚೈ-ತ-ನ್ಯನ
ಶ್ರೀಧರ
ಶ್ರೀಧ-ರ-ಸ್ವಾ-ಮಿ-ಗಳು
ಶ್ರೀಮಂತ
ಶ್ರೀಮಂ-ತನ
ಶ್ರೀಮಂ-ತ-ನಾ-ಗ-ಬೇ-ಕೆಂದು
ಶ್ರೀಮಂ-ತ-ನಾ-ಗಿ-ರಲಿ
ಶ್ರೀಮಂ-ತ-ನಾ-ಗು-ವು-ದಕ್ಕೆ
ಶ್ರೀಮಂ-ತ-ನಾ-ದರೆ
ಶ್ರೀಮಂ-ತ-ನೆಂದು
ಶ್ರೀಮಂ-ತರ
ಶ್ರೀಮಂ-ತರು
ಶ್ರೀಮಂ-ತರೂ
ಶ್ರೀಮಂ-ತ-ರೆಲ್ಲ
ಶ್ರೀಮಂ-ತಿಕೆ
ಶ್ರೀಮಂ-ತಿ-ಕೆ-ಯನ್ನು
ಶ್ರೀಮ-ತಾಂ
ಶ್ರೀಮ-ದೂ-ರ್ಜಿ-ತ-ಮೇವ
ಶ್ರೀಮ-ದ್ಭ-ಗ-ವ-ದ್ಗೀ-ತಾಸು
ಶ್ರೀಮ-ನ್ನಾ-ರಾ-ಯ-ಣನ
ಶ್ರೀಮ-ನ್ನಾ-ರಾ-ಯ-ಣ-ನನ್ನು
ಶ್ರೀಮ-ನ್ನಾ-ರಾ-ಯ-ಣ-ನಿಗೆ
ಶ್ರೀಯು-ಕ್ತವೊ
ಶ್ರೀಯು-ಕ್ತವೋ
ಶ್ರೀರಾಮ
ಶ್ರೀರಾ-ಮ-ಕೃಷ್ಣ
ಶ್ರೀರಾ-ಮ-ಕೃ-ಷ್ಣರ
ಶ್ರೀರಾ-ಮ-ಕೃ-ಷ್ಣ-ರಂ-ದಂತೆ
ಶ್ರೀರಾ-ಮ-ಕೃ-ಷ್ಣ-ರನ್ನು
ಶ್ರೀರಾ-ಮ-ಕೃ-ಷ್ಣ-ರಲ್ಲಿ
ಶ್ರೀರಾ-ಮ-ಕೃ-ಷ್ಣ-ರಿ-ಗಾ-ದರೊ
ಶ್ರೀರಾ-ಮ-ಕೃ-ಷ್ಣ-ರಿಗೆ
ಶ್ರೀರಾ-ಮ-ಕೃ-ಷ್ಣರು
ಶ್ರೀರಾ-ಮ-ಚಂ-ದ್ರನ
ಶ್ರೀರಾ-ಮ-ನನ್ನು
ಶ್ರೀರಾ-ಮ-ನಷ್ಟೇ
ಶ್ರೀರಾ-ಮನು
ಶ್ರೀರ್ವಾಕ್
ಶ್ರೀರ್ವಿ-ಜಯೋ
ಶ್ರೀಶಾ-ರ-ದಾ-ದೇವಿ
ಶ್ರೀಶಾ-ರ-ದಾ-ದೇ-ವಿ-ಯ-ವರ
ಶ್ರುತ-ಕೀರ್ತಿ
ಶ್ರುತಸ್ಯ
ಶ್ರುತಿ
ಶ್ರುತಿ-ಗಳ
ಶ್ರುತಿಗೆ
ಶ್ರುತಿ-ಪ-ಡಿ-ಸಿ-ದಂತೆ
ಶ್ರುತಿ-ಪ-ರಾ-ಯ-ಣಾಃ
ಶ್ರುತಿ-ಪಾ-ರಾ-ಯಣ
ಶ್ರುತಿ-ವಿ-ಪ್ರ-ತಿ-ಪನ್ನಾ
ಶ್ರುತೌ
ಶ್ರುತ್ವಾ-ಪ್ಯೇನಂ
ಶ್ರುತ್ವಾ-ಽನ್ಯೇಭ್ಯ
ಶ್ರೇಯ
ಶ್ರೇಯಃ
ಶ್ರೇಯಸೇ
ಶ್ರೇಯಸ್
ಶ್ರೇಯ-ಸ್ಕರ
ಶ್ರೇಯ-ಸ್ಕ-ರ-ವಲ್ಲ
ಶ್ರೇಯ-ಸ್ಕ-ರ-ವಾ-ಗ-ಬೇ-ಕೆಂದು
ಶ್ರೇಯ-ಸ್ಕ-ರ-ವಾ-ಗಿ-ರು-ವುದನ್ನು
ಶ್ರೇಯ-ಸ್ಕ-ರ-ವಾ-ಗಿ-ರು-ವುದು
ಶ್ರೇಯ-ಸ್ಕ-ರ-ವಾ-ದುದು
ಶ್ರೇಯ-ಸ್ಕ-ರವೊ
ಶ್ರೇಯ-ಸ್ಕ-ರವೋ
ಶ್ರೇಯ-ಸ್ತತೋ
ಶ್ರೇಯ-ಸ್ಸನ್ನು
ಶ್ರೇಯ-ಸ್ಸ-ನ್ನುಂ-ಟು-ಮಾ-ಡಲು
ಶ್ರೇಯ-ಸ್ಸಿ-ಗಾಗಿ
ಶ್ರೇಯ-ಸ್ಸಿಗೆ
ಶ್ರೇಯ-ಸ್ಸಿನ
ಶ್ರೇಯ-ಸ್ಸಿ-ನ-ಲ್ಲೆಲ್ಲಾ
ಶ್ರೇಯಸ್ಸು
ಶ್ರೇಯಸ್ಸೇ
ಶ್ರೇಯಾನ್
ಶ್ರೇಯೋ
ಶ್ರೇಯೋ-ನು-ಪ-ಶ್ಯಾಮಿ
ಶ್ರೇಯೋ-ಽಹ-ಮಾ-ಪ್ನು-ಯಾಮ್
ಶ್ರೇಷ್ಠ
ಶ್ರೇಷ್ಠ-ಗತಿ
ಶ್ರೇಷ್ಠ-ಜ್ಞಾ-ನ-ದಲ್ಲಿ
ಶ್ರೇಷ್ಠ-ತಮ
ಶ್ರೇಷ್ಠ-ನಾದ
ಶ್ರೇಷ್ಠ-ನಾ-ದ-ವನು
ಶ್ರೇಷ್ಠನೂ
ಶ್ರೇಷ್ಠ-ನೆಂದು
ಶ್ರೇಷ್ಠ-ನೆಂ-ಬುದು
ಶ್ರೇಷ್ಠ-ಫ-ಲವೇ
ಶ್ರೇಷ್ಠ-ಭಾ-ವನೆ
ಶ್ರೇಷ್ಠ-ರಾದ
ಶ್ರೇಷ್ಠರು
ಶ್ರೇಷ್ಠ-ವನ್ನೂ
ಶ್ರೇಷ್ಠ-ವಾಗಿ
ಶ್ರೇಷ್ಠ-ವಾ-ಗಿ-ರುವ
ಶ್ರೇಷ್ಠ-ವಾ-ಗಿ-ರು-ವುದನ್ನು
ಶ್ರೇಷ್ಠ-ವಾ-ಗಿ-ರು-ವುದು
ಶ್ರೇಷ್ಠ-ವಾ-ಗಿ-ರು-ವುದೇ
ಶ್ರೇಷ್ಠ-ವಾ-ಗು-ವುದು
ಶ್ರೇಷ್ಠ-ವಾದ
ಶ್ರೇಷ್ಠ-ವಾ-ದು-ದನ್ನು
ಶ್ರೇಷ್ಠ-ವಾ-ದುದು
ಶ್ರೇಷ್ಠ-ವಾ-ದುದೇ
ಶ್ರೇಷ್ಠವೂ
ಶ್ರೇಷ್ಠ-ವೆ-ನ್ನು-ವರು
ಶ್ರೇಷ್ಠವೇ
ಶ್ರೇಷ್ಠವೊ
ಶ್ರೇಷ್ಠವೋ
ಶ್ರೇಷ್ಠ-ವ್ಯಕ್ತಿ
ಶ್ರೇಷ್ಠ-ವ್ಯ-ಕ್ತಿ-ಗಳೇ
ಶ್ರೇಷ್ಠ-ಸುಖ
ಶ್ರೇಷ್ಠ-ಸ್ತ-ತ್ತ-ದೇ-ವೇ-ತರೋ
ಶ್ರೋತ-ವ್ಯಸ್ಯ
ಶ್ರೋತೃ-ಗ-ಳಿಗೆ
ಶ್ರೋತ್ರಂ
ಶ್ರೋತ್ರವೇ
ಶ್ರೋತ್ರಾ-ದೀ-ನೀಂ-ದ್ರಿ-ಯಾ-ಣ್ಯನ್ಯೇ
ಶ್ರೋಷ್ಯಸಿ
ಶ್ರೌತ
ಶ್ಲಾಘಿ-ಸು-ವರು
ಶ್ಲೋಕ
ಶ್ಲೋಕ-ಗಳ
ಶ್ಲೋಕ-ಗ-ಳಂತೂ
ಶ್ಲೋಕ-ಗಳನ್ನು
ಶ್ಲೋಕ-ಗ-ಳ-ನ್ನೊಳ
ಶ್ಲೋಕ-ಗಳಲ್ಲಿ
ಶ್ಲೋಕ-ಗ-ಳಿವೆ
ಶ್ಲೋಕ-ಗಳು
ಶ್ಲೋಕದ
ಶ್ಲೋಕ-ದಲ್ಲಿ
ಶ್ಲೋಕ-ದ-ವ-ರೆಗೆ
ಶ್ಲೋಕ-ದಿಂದ
ಶ್ಲೋಕ-ವಲ್ಲ
ಶ್ವಪಾಕೇ
ಶ್ವಶು-ರಾಃ
ಶ್ವಶು-ರಾನ್
ಶ್ವಸನ್
ಶ್ವಾಸ-ಕೋಶ
ಶ್ವಾಸ-ಕೋ-ಶ-ಗ-ಳಿಗೆ
ಶ್ವಾಸ-ಕೋ-ಶ-ಗಳು
ಶ್ವೇತೈ-ರ್ಹ-ಯೈ-ರ್ಯುಕ್ತೇ
ಷಂಡ
ಷಂಡ-ತ-ನ-ವನ್ನು
ಷಣ್ಮಾಸಾ
ಷರ-ತ್ತನ್ನೂ
ಸ
ಸಂಕಟ
ಸಂಕ-ಟಕ್ಕೆ
ಸಂಕ-ಟ-ಗಳನ್ನು
ಸಂಕ-ಟ-ಗಳು
ಸಂಕ-ಟದ
ಸಂಕ-ಟ-ದಿಂದ
ಸಂಕ-ಟ-ಪ-ಡ-ಬೇ-ಕಾ-ಗು-ವುದು
ಸಂಕ-ಟ-ಪ-ಡ-ಬೇಕು
ಸಂಕ-ಟ-ಪ-ಡು-ವುದೇ
ಸಂಕ-ಟ-ವನ್ನು
ಸಂಕ-ಟ-ವಿಲ್ಲ
ಸಂಕ-ರಸ್ಯ
ಸಂಕರೋ
ಸಂಕಲ್ಪ
ಸಂಕ-ಲ್ಪ-ಗಳಿಂದ
ಸಂಕ-ಲ್ಪ-ಗಳು
ಸಂಕ-ಲ್ಪ-ದಿಂದ
ಸಂಕ-ಲ್ಪ-ಪ್ರ-ಭ-ವಾನ್
ಸಂಕ-ಲ್ಪ-ವನ್ನು
ಸಂಕ-ಲ್ಪ-ವಾ-ಗಿದೆ
ಸಂಕ-ಲ್ಪವೇ
ಸಂಕು-ಚಿತ
ಸಂಕು-ಚಿ-ತ-ವಾ-ಗುತ್ತ
ಸಂಕು-ಚಿ-ತ-ವಾ-ಗು-ವುದು
ಸಂಕೇ-ತ-ರೂ-ಪ-ವಾಗಿ
ಸಂಕೇ-ತ-ವಾ-ಗಿ-ರು-ವುದು
ಸಂಕೊ-ಚ-ಪ-ಡ-ಬೇ-ಕಾ-ಗಿಲ್ಲ
ಸಂಕೋಚ
ಸಂಕೋ-ಚ-ವಿ-ಲ್ಲದೆ
ಸಂಕೋಲೆ
ಸಂಕ್ರಾಂತಿ
ಸಂಕ್ರಾಂ-ತಿಯ
ಸಂಕ್ರಾ-ಮಿಕ
ಸಂಕ್ಷಿಪ್ತ
ಸಂಕ್ಷೇಪ
ಸಂಕ್ಷೇ-ಪ-ದಿಂದ
ಸಂಕ್ಷೇ-ಪ-ವಾಗಿ
ಸಂಕ್ಷೇ-ಪ-ವಾದ
ಸಂಖ್ಯೆ
ಸಂಖ್ಯೆ-ಯಲ್ಲ
ಸಂಖ್ಯೆ-ಯಲ್ಲಿ
ಸಂಖ್ಯೇ
ಸಂಗ
ಸಂಗಂ
ಸಂಗದ
ಸಂಗ-ದಿಂದ
ಸಂಗಮ
ಸಂಗ-ಮ-ವಾ-ಗಿವೆ
ಸಂಗ-ಮ-ವಾದ
ಸಂಗ-ರ-ಹಿ-ತ-ನಾ-ಗಿ-ರ-ಬೇಕು
ಸಂಗ-ರ-ಹಿ-ತನೂ
ಸಂಗ-ರ-ಹಿ-ತನೋ
ಸಂಗ-ರ-ಹಿ-ತ-ಮ-ರಾ-ಗ-ದ್ವೇ-ಷತಃ
ಸಂಗ-ರ-ಹಿ-ತ-ವಾದ
ಸಂಗ-ವನ್ನು
ಸಂಗ-ವ-ರ್ಜಿತಃ
ಸಂಗ-ವಿ-ವ-ರ್ಜಿತಃ
ಸಂಗ-ಸ್ತೇ-ಷೂ-ಪ-ಜಾ-ಯತೇ
ಸಂಗಾತ್
ಸಂಗೀತ
ಸಂಗೀ-ತ-ಗಾರ
ಸಂಗೀ-ತ-ಗಾ-ರನ
ಸಂಗೀ-ತ-ಗಾ-ರ-ನಾ-ಗಿದ್ದ
ಸಂಗೀ-ತ-ಗಾ-ರ-ನಾ-ಗಿ-ರ-ಬ-ಹುದು
ಸಂಗೀ-ತ-ಗಾ-ರ-ನಿದ್ದ
ಸಂಗೀ-ತ-ದಂತೆ
ಸಂಗೀ-ತ-ವನ್ನು
ಸಂಗೀ-ತ-ವಾ-ಗಿ-ರ-ಬ-ಹುದು
ಸಂಗೀ-ತ-ವೇನೊ
ಸಂಗೋ-ಽಸ್ತ್ವ-ಕ-ರ್ಮಣಿ
ಸಂಗ್ರ-ಸಿ-ಸು-ತ್ತಿ-ರು-ವುದು
ಸಂಗ್ರಹ
ಸಂಗ್ರ-ಹ-ಕ್ಕಲ್ಲ
ಸಂಗ್ರ-ಹಕ್ಕೆ
ಸಂಗ್ರ-ಹ-ಭೂತಂ
ಸಂಗ್ರ-ಹ-ವಾಗಿ
ಸಂಗ್ರ-ಹ-ವಾ-ಗುತ್ತ
ಸಂಗ್ರ-ಹ-ವಾ-ಗು-ತ್ತಿದೆ
ಸಂಗ್ರ-ಹ-ವಾ-ಗು-ವುದು
ಸಂಗ್ರಹಿ
ಸಂಗ್ರ-ಹಿಸ
ಸಂಗ್ರ-ಹಿ-ಸ-ಬೇಕು
ಸಂಗ್ರ-ಹಿ-ಸ-ಲಿಲ್ಲ
ಸಂಗ್ರ-ಹಿಸಿ
ಸಂಗ್ರ-ಹಿ-ಸಿ-ಕೊಂಡ
ಸಂಗ್ರ-ಹಿ-ಸಿ-ಕೊಂ-ಡಿ-ರುವ
ಸಂಗ್ರ-ಹಿ-ಸಿ-ಕೊಂ-ಡಿ-ರು-ವೆವು
ಸಂಗ್ರ-ಹಿ-ಸಿ-ಕೊಂ-ಡಿ-ರು-ವೆವೊ
ಸಂಗ್ರ-ಹಿ-ಸಿಟ್ಟ
ಸಂಗ್ರ-ಹಿ-ಸಿ-ಟ್ಟಿ-ರು-ವೆವೊ
ಸಂಗ್ರ-ಹಿ-ಸಿ-ಟ್ಟು-ಕೊ-ಳ್ಳು-ತ್ತೇವೆ
ಸಂಗ್ರ-ಹಿ-ಸಿದ
ಸಂಗ್ರ-ಹಿ-ಸಿ-ದಂತೆ
ಸಂಗ್ರ-ಹಿ-ಸಿ-ದ್ದೇವೆ
ಸಂಗ್ರ-ಹಿ-ಸಿ-ರು-ವನು
ಸಂಗ್ರ-ಹಿ-ಸಿ-ರು-ವುದನ್ನು
ಸಂಗ್ರ-ಹಿ-ಸುತ್ತ
ಸಂಗ್ರ-ಹಿ-ಸುತ್ತಾ
ಸಂಗ್ರ-ಹಿ-ಸು-ತ್ತಿ-ದ್ದಾಗ
ಸಂಗ್ರ-ಹಿ-ಸು-ತ್ತಿ-ರುವ
ಸಂಗ್ರ-ಹಿ-ಸು-ತ್ತಿ-ರುವೆ
ಸಂಗ್ರ-ಹಿ-ಸು-ತ್ತಿಲ್ಲ
ಸಂಗ್ರ-ಹಿ-ಸು-ತ್ತೇವೆ
ಸಂಗ್ರ-ಹಿ-ಸುವ
ಸಂಗ್ರ-ಹಿ-ಸು-ವಾಗ
ಸಂಗ್ರ-ಹಿ-ಸು-ವು-ದಕ್ಕೆ
ಸಂಗ್ರ-ಹಿ-ಸು-ವು-ದ-ರಲ್ಲಿ
ಸಂಗ್ರ-ಹಿ-ಸು-ವುದು
ಸಂಗ್ರ-ಹಿ-ಸು-ವುದೊ
ಸಂಗ್ರ-ಹಿ-ಸೋಣ
ಸಂಗ್ರ-ಹೇಣ
ಸಂಗ್ರಾಮಂ
ಸಂಘ
ಸಂಘ-ಗಳು
ಸಂಘಾತ
ಸಂಘಾ-ತ-ದಿಂದ
ಸಂಘಾ-ತ-ವಿ-ಲ್ಲದೆ
ಸಂಘಾ-ತ-ಶ್ಚೇ-ತನಾ
ಸಂಚ-ಯಾನ್
ಸಂಚ-ರಿ-ಸು-ತ್ತಿ-ರುವ
ಸಂಚ-ರಿ-ಸು-ತ್ತಿ-ರು-ವನು
ಸಂಚ-ರಿ-ಸು-ತ್ತಿ-ರು-ವುದು
ಸಂಚ-ರಿ-ಸುವ
ಸಂಚ-ರಿ-ಸು-ವಂತೆ
ಸಂಚ-ರಿ-ಸು-ವುದು
ಸಂಚ-ರಿ-ಸು-ವುವು
ಸಂಚಾರ
ಸಂಚಾ-ರ-ಮಾ-ಡು-ತ್ತಿ-ದ್ದಾಗ
ಸಂಚಾ-ರ-ಮಾ-ಡುವ
ಸಂಜ-ನ-ಯನ್
ಸಂಜಯ
ಸಂಜ-ಯತಿ
ಸಂಜ-ಯ-ತ್ಯುತ
ಸಂಜ-ಯ-ನನ್ನು
ಸಂಜ-ಯ-ನಿಗೂ
ಸಂಜ-ಯ-ನಿಗೆ
ಸಂಜ-ಯನೂ
ಸಂಜಾ-ಯತೇ
ಸಂಜೀ-ವಿ-ನಿ-ಯಂತೆ
ಸಂಜೀ-ವಿ-ನಿ-ಯನ್ನು
ಸಂಜೆ
ಸಂಜೆಯ
ಸಂಜ್ಞಾರ್ಥಂ
ಸಂತ-ರಿ-ಷ್ಯಸಿ
ಸಂತ-ರ್ಪ-ಣೆ-ಯನ್ನು
ಸಂತಾ-ಪ-ವನ್ನು
ಸಂತುಷ್ಟ
ಸಂತುಷ್ಟಃ
ಸಂತು-ಷ್ಟ-ನಾ-ದಾಗ
ಸಂತು-ಷ್ಟನೊ
ಸಂತು-ಷ್ಟ-ಸ್ತಸ್ಯ
ಸಂತುಷ್ಟೋ
ಸಂತೆ
ಸಂತೆಗೆ
ಸಂತೆಯ
ಸಂತೆ-ಯಂತೆ
ಸಂತೆ-ಯಲ್ಲಿ
ಸಂತೆಯೇ
ಸಂತೋ
ಸಂತೋಷ
ಸಂತೋ-ಷ-ಕ್ಕಲ್ಲ
ಸಂತೋ-ಷಕ್ಕೆ
ಸಂತೋ-ಷ-ದಲ್ಲಿ
ಸಂತೋ-ಷ-ದಿಂದ
ಸಂತೋ-ಷ-ಪ-ಟ್ಟರೆ
ಸಂತೋ-ಷ-ಪ-ಡ-ಬ-ಹುದು
ಸಂತೋ-ಷ-ಪ-ಡು-ತ್ತಾನೆ
ಸಂತೋ-ಷ-ಪ-ಡು-ತ್ತಾರೆ
ಸಂತೋ-ಷ-ಪ-ಡು-ತ್ತಿದೆ
ಸಂತೋ-ಷ-ಪ-ಡು-ತ್ತಿ-ರು-ವನು
ಸಂತೋ-ಷ-ಪ-ಡು-ತ್ತೇನೆ
ಸಂತೋ-ಷ-ಪ-ಡು-ತ್ತೇ-ನೆ-ಎಂದು
ಸಂತೋ-ಷ-ಪ-ಡು-ತ್ತೇವೆ
ಸಂತೋ-ಷ-ಪ-ಡು-ವನು
ಸಂತೋ-ಷ-ಪ-ಡು-ವನೊ
ಸಂತೋ-ಷ-ಪ-ಡು-ವು-ದಿಲ್ಲ
ಸಂತೋ-ಷ-ಪ-ಡು-ವುದು
ಸಂತೋ-ಷ-ಪ-ಡು-ವುದೂ
ಸಂತೋ-ಷ-ಭ-ರಿತ
ಸಂತೋ-ಷ-ವನ್ನು
ಸಂತೋ-ಷ-ವಾ-ಗು-ವುದು
ಸಂತೋ-ಷ-ವಿದೆ
ಸಂತೋ-ಷವೇ
ಸಂತೋ-ಷಿ-ಸು-ತ್ತಿ-ರು-ವರು
ಸಂತೋ-ಷಿ-ಸು-ವುದೂ
ಸಂತೋ-ಷಿ-ಸು-ವುದೇ
ಸಂದಿಗ್ಧ
ಸಂದಿ-ನಿಂದ
ಸಂದಿ-ಯಲ್ಲಿ
ಸಂದು-ಗಳಲ್ಲಿ
ಸಂದು-ಗೊಂ-ದು-ಗಳಲ್ಲಿ
ಸಂದೂ-ಕ-ದ-ಲ್ಲಿಟ್ಟು
ಸಂದೂ-ಕಿ-ನಲ್ಲಿ
ಸಂದೃ-ಶ್ಯಂತೇ
ಸಂದೇಶ
ಸಂದೇ-ಶಕ್ಕೆ
ಸಂದೇ-ಶ-ಗಳನ್ನು
ಸಂದೇ-ಶದ
ಸಂದೇ-ಶ-ದಿಂದ
ಸಂದೇ-ಶ-ವನ್ನು
ಸಂದೇ-ಶವೂ
ಸಂದೇಹ
ಸಂದೇ-ಹ-ಕ್ಕಿಂತ
ಸಂದೇ-ಹಕ್ಕೆ
ಸಂದೇ-ಹ-ಕ್ಕೆಲ್ಲ
ಸಂದೇ-ಹ-ಗಳನ್ನು
ಸಂದೇ-ಹ-ಗಳನ್ನೆಲ್ಲಾ
ಸಂದೇ-ಹ-ಗ-ಳಿಂ-ದಲೂ
ಸಂದೇ-ಹ-ಗಳು
ಸಂದೇ-ಹ-ಗ-ಳೆಲ್ಲ
ಸಂದೇ-ಹದ
ಸಂದೇ-ಹ-ದಿಂದ
ಸಂದೇ-ಹ-ವನ್ನು
ಸಂದೇ-ಹ-ವಾದಿ
ಸಂದೇ-ಹ-ವಾ-ದಿ-ಗಳು
ಸಂದೇ-ಹ-ವಾ-ದಿಗೆ
ಸಂದೇ-ಹ-ವಾ-ದಿ-ಯಾ-ಗಲಿ
ಸಂದೇ-ಹ-ವಿತ್ತು
ಸಂದೇ-ಹ-ವಿಲ್ಲ
ಸಂದೇ-ಹ-ವಿ-ಲ್ಲದೆ
ಸಂದೇ-ಹವೂ
ಸಂದೇ-ಹವೇ
ಸಂಧಿ
ಸಂಧಿಗೆ
ಸಂಧಿಗ್ಧ
ಸಂಧಿಯ
ಸಂಧಿ-ಸು-ತ್ತಾರೆ
ಸಂಧಿ-ಸು-ತ್ತೇ-ವೆ-ಎಂ-ದೆಂ-ದಿಗೂ
ಸಂಧಿ-ಸುವ
ಸಂಧಿ-ಸು-ವರು
ಸಂಧಿ-ಸು-ವ-ವ-ರೆಗೆ
ಸಂಧಿ-ಸು-ವುವು
ಸಂಧ್ಯಾ-ರಾ-ಗದ
ಸಂನಿ-ಯ-ಮ್ಯೇಂ-ದ್ರಿ-ಯ-ಗ್ರಾಮಂ
ಸಂನಿ-ವಿಷ್ಟೋ
ಸಂನ್ಯ-ಸ-ನಾ-ದೇವ
ಸಂನ್ಯಸ್ಯ
ಸಂನ್ಯ-ಸ್ಯಾ-ಧ್ಯಾ-ತ್ಮ-ಚೇ-ತಸಾ
ಸಂನ್ಯ-ಸ್ಯಾಸ್ತೇ
ಸಂನ್ಯಾಸ
ಸಂನ್ಯಾಸಂ
ಸಂನ್ಯಾಸಃ
ಸಂನ್ಯಾ-ಸದ
ಸಂನ್ಯಾ-ಸ-ಮಿತಿ
ಸಂನ್ಯಾ-ಸ-ಯೋಗ
ಸಂನ್ಯಾ-ಸ-ಯೋ-ಗ-ಯು-ಕ್ತಾತ್ಮಾ
ಸಂನ್ಯಾ-ಸ-ಯೋ-ಗ-ವೆಂಬ
ಸಂನ್ಯಾ-ಸ-ವನ್ನು
ಸಂನ್ಯಾ-ಸ-ವೆಂದು
ಸಂನ್ಯಾ-ಸಸ್ತು
ಸಂನ್ಯಾ-ಸಸ್ಯ
ಸಂನ್ಯಾಸಿ
ಸಂನ್ಯಾ-ಸಿ-ಗಳು
ಸಂನ್ಯಾ-ಸಿ-ನಾಂ
ಸಂನ್ಯಾ-ಸಿಯ
ಸಂನ್ಯಾ-ಸಿ-ಯಂತೆ
ಸಂನ್ಯಾ-ಸಿ-ಯಾ-ಗಲಿ
ಸಂನ್ಯಾ-ಸಿ-ಯಾ-ಗಿ-ರು-ವನೊ
ಸಂನ್ಯಾ-ಸಿಯೆ
ಸಂನ್ಯಾ-ಸಿಯೇ
ಸಂನ್ಯಾಸೀ
ಸಂನ್ಯಾ-ಸೇ-ನಾ-ಧಿ-ಗ-ಚ್ಛತಿ
ಸಂಪ-ತ್ತನ್ನು
ಸಂಪ-ತ್ತಿಗೆ
ಸಂಪ-ತ್ತಿನ
ಸಂಪ-ತ್ತಿ-ನಿಂದ
ಸಂಪತ್ತು
ಸಂಪ-ತ್ತು-ಗಳು
ಸಂಪತ್ತೂ
ಸಂಪ-ತ್ಸ-ಮೃ-ದ್ಧ-ವಾ-ಗಿ-ರುವ
ಸಂಪದಂ
ಸಂಪ-ದ-ಮಾ-ಸು-ರೀಮ್
ಸಂಪ-ದ್ಯತೇ
ಸಂಪ-ದ್ವಿ-ಭಾ-ಗ-ಯೋಗ
ಸಂಪರ್ಕ
ಸಂಪ-ರ್ಕಕ್ಕೆ
ಸಂಪ-ರ್ಕ-ದಿಂದ
ಸಂಪ-ರ್ಕ-ದಿಂ-ದಲೇ
ಸಂಪ-ರ್ಕ-ವನ್ನು
ಸಂಪ-ರ್ಕ-ವಿ-ರು-ವು-ದ-ರಿಂದ
ಸಂಪ-ರ್ಕವೇ
ಸಂಪ-ಶ್ಯನ್
ಸಂಪಾ-ದನೆ
ಸಂಪಾ-ದ-ನೆಗೆ
ಸಂಪಾ-ದ-ನೆಯ
ಸಂಪಾ-ದಿ-ಸ-ಬ-ಹುದೇ
ಸಂಪಾ-ದಿ-ಸ-ಬೇ-ಕಾದ
ಸಂಪಾ-ದಿ-ಸ-ಬೇ-ಕಾ-ದರೆ
ಸಂಪಾ-ದಿ-ಸ-ಬೇಕು
ಸಂಪಾ-ದಿ-ಸ-ಬೇ-ಕೆಂ-ದಿ-ರು-ವುದು
ಸಂಪಾ-ದಿ-ಸಲಿ
ಸಂಪಾ-ದಿ-ಸಲು
ಸಂಪಾ-ದಿ-ಸ-ಲೆ-ತ್ನಿ-ಸು-ವನು
ಸಂಪಾ-ದಿಸಿ
ಸಂಪಾ-ದಿ-ಸಿ-ಕೊಂ-ಡರೆ
ಸಂಪಾ-ದಿ-ಸಿ-ಕೊಂಡಿ
ಸಂಪಾ-ದಿ-ಸಿ-ಕೊಂ-ಡಿ-ರ-ಬೇಕು
ಸಂಪಾ-ದಿ-ಸಿ-ಕೊಂ-ಡಿ-ರುವ
ಸಂಪಾ-ದಿ-ಸಿ-ಕೊಂ-ಡಿ-ರು-ವೆವೊ
ಸಂಪಾ-ದಿ-ಸಿ-ಕೊಂ-ಡಿಲ್ಲ
ಸಂಪಾ-ದಿ-ಸಿ-ಕೊಂಡು
ಸಂಪಾ-ದಿ-ಸಿ-ಕೊ-ಳ್ಳು-ವು-ದಕ್ಕೆ
ಸಂಪಾ-ದಿ-ಸಿದ
ಸಂಪಾ-ದಿ-ಸಿ-ದ-ನಲ್ಲ
ಸಂಪಾ-ದಿ-ಸಿ-ದರು
ಸಂಪಾ-ದಿ-ಸಿ-ದ-ವನು
ಸಂಪಾ-ದಿ-ಸಿ-ದ್ದನ್ನು
ಸಂಪಾ-ದಿ-ಸಿ-ದ್ದಾನೆ
ಸಂಪಾ-ದಿ-ಸಿದ್ದು
ಸಂಪಾ-ದಿ-ಸಿಯೂ
ಸಂಪಾ-ದಿ-ಸಿರ
ಸಂಪಾ-ದಿ-ಸಿ-ರ-ಬೇಕು
ಸಂಪಾ-ದಿ-ಸಿರು
ಸಂಪಾ-ದಿ-ಸಿ-ರು-ವನು
ಸಂಪಾ-ದಿ-ಸಿಲ್ಲ
ಸಂಪಾ-ದಿ-ಸು-ತ್ತೇವೆ
ಸಂಪಾ-ದಿ-ಸುವ
ಸಂಪಾ-ದಿ-ಸು-ವನು
ಸಂಪಾ-ದಿ-ಸು-ವು-ದಕ್ಕೆ
ಸಂಪಾ-ದಿ-ಸು-ವು-ದ-ರಲ್ಲಿ
ಸಂಪಾ-ದಿ-ಸು-ವು-ದಾ-ಗಿ-ರ-ಬಾ-ರದು
ಸಂಪಾ-ದಿ-ಸು-ವೆನು
ಸಂಪಿಗೆ
ಸಂಪಿ-ಗೆ-ಯಿಂದ
ಸಂಪೂರ್ಣ
ಸಂಪೂ-ರ್ಣ-ವಾಗಿ
ಸಂಪೂ-ರ್ಣ-ವಾ-ಗಿದೆ
ಸಂಪೂ-ರ್ಣ-ವಾ-ಗಿಯೂ
ಸಂಪೂ-ರ್ಣ-ವಾ-ಗಿ-ರ-ವಂತೆ
ಸಂಪೂ-ರ್ಣ-ವಿ-ಚಾ-ರ-ಮಾಡಿ
ಸಂಪ್ರ-ಕೀ-ರ್ತಿತಃ
ಸಂಪ್ರ-ತಿಷ್ಠಾ
ಸಂಪ್ರ-ದಾ-ಯ-ದಲ್ಲಿ
ಸಂಪ್ರ-ವೃ-ತ್ತಾನಿ
ಸಂಪ್ರೇಕ್ಷ್ಯ
ಸಂಬಂ-ದ-ಪ-ಟ್ಟಿ-ರ-ಬ-ಹುದು
ಸಂಬಂಧ
ಸಂಬಂ-ಧ-ಕ್ಕಿಂತ
ಸಂಬಂ-ಧಕ್ಕೂ
ಸಂಬಂ-ಧ-ಗ-ಳಾ-ವುವು
ಸಂಬಂ-ಧ-ಗಳಿಂದ
ಸಂಬಂ-ಧ-ಗ-ಳೇನು
ಸಂಬಂ-ಧದ
ಸಂಬಂ-ಧ-ದಲ್ಲಿ
ಸಂಬಂ-ಧ-ದಿಂದ
ಸಂಬಂ-ಧ-ಪಟ್ಟ
ಸಂಬಂ-ಧ-ಪ-ಟ್ಟ-ದ್ದನ್ನು
ಸಂಬಂ-ಧ-ಪ-ಟ್ಟ-ದ್ದಲ್ಲ
ಸಂಬಂ-ಧ-ಪ-ಟ್ಟದ್ದು
ಸಂಬಂ-ಧ-ಪ-ಟ್ಟ-ವರು
ಸಂಬಂ-ಧ-ಪ-ಟ್ಟಿದ್ದು
ಸಂಬಂ-ಧ-ಪ-ಟ್ಟಿದ್ದೆ
ಸಂಬಂ-ಧ-ಪ-ಟ್ಟಿ-ರಲಿ
ಸಂಬಂ-ಧ-ಪ-ಟ್ಟಿವೆ
ಸಂಬಂ-ಧ-ಪ-ಟ್ಟು-ದನ್ನು
ಸಂಬಂ-ಧ-ಪ-ಟ್ಟು-ದೆಲ್ಲಾ
ಸಂಬಂ-ಧ-ವನ್ನು
ಸಂಬಂ-ಧ-ವಾದ
ಸಂಬಂ-ಧ-ವಾ-ದರೆ
ಸಂಬಂ-ಧ-ವಿದೆ
ಸಂಬಂ-ಧ-ವಿ-ದೆಯೋ
ಸಂಬಂ-ಧ-ವಿ-ದ್ದರೆ
ಸಂಬಂ-ಧ-ವಿ-ರ-ಲಿಲ್ಲ
ಸಂಬಂ-ಧ-ವಿ-ರು-ವಂತೆ
ಸಂಬಂ-ಧ-ವಿಲ್ಲ
ಸಂಬಂ-ಧ-ವಿ-ಲ್ಲದೆ
ಸಂಬಂ-ಧ-ವುಳ್ಳ
ಸಂಬಂ-ಧವೂ
ಸಂಬಂ-ಧ-ವೆಂ-ತ-ಹುದು
ಸಂಬಂ-ಧ-ವೇನು
ಸಂಬಂ-ಧ-ಸೂ-ತ್ರ-ದಿಂದ
ಸಂಬಂ-ಧಿ-ಕರು
ಸಂಬಂ-ಧಿ-ನ-ಸ್ತಥಾ
ಸಂಬಂ-ಧಿ-ಯಾದ
ಸಂಬಂ-ಧಿ-ಸಿದ
ಸಂಬಂ-ಧಿ-ಸಿ-ದಂತೆ
ಸಂಬಂ-ಧಿ-ಸಿ-ರು-ವು-ದಾ-ಯಿತು
ಸಂಬಂ-ಧಿ-ಸು-ವನು
ಸಂಬಳ
ಸಂಬ-ಳಕ್ಕೆ
ಸಂಬ-ಳ-ರೂ-ಪ-ವಾಗಿ
ಸಂಬ-ಳ-ವೊಂದೆ
ಸಂಬೋ-ಧಿ-ಸು-ವರು
ಸಂಭವ
ಸಂಭ-ವಂತಿ
ಸಂಭವಃ
ಸಂಭ-ವ-ವಿ-ರ-ಬ-ಹು-ದೆಂದು
ಸಂಭ-ವವೂ
ಸಂಭ-ವವೇ
ಸಂಭ-ವಾಮಿ
ಸಂಭ-ವಾ-ಮ್ಯಾ-ತ್ಮ-ಮಾ-ಯಯಾ
ಸಂಭ-ವಿ-ಸು-ವುದು
ಸಂಭಾ-ವ-ನೆ-ಯನ್ನು
ಸಂಭಾ-ವಿ-ತಸ್ಯ
ಸಂಮಿ-ಶ್ರ-ಣ-ದಿಂದ
ಸಂಮೋಹ
ಸಂಮೋಹಂ
ಸಂಮೋಹಃ
ಸಂಮೋ-ಹ-ವುಂ-ಟಾ-ಗು-ವುದು
ಸಂಮೋ-ಹಾತ್
ಸಂಯ-ತೇಂ-ದ್ರಿಯಃ
ಸಂಯಮ
ಸಂಯ-ಮ-ತಾ-ಮ-ಹಮ್
ಸಂಯ-ಮ-ಮಾಡಿ
ಸಂಯ-ಮ-ಮಾ-ಡಿ-ಕೊಂಡು
ಸಂಯ-ಮಾ-ಗ್ನಿ-ಯಲ್ಲಿ
ಸಂಯ-ಮಾ-ಗ್ನಿಷು
ಸಂಯಮಿ
ಸಂಯಮೀ
ಸಂಯಮ್ಯ
ಸಂಯಾತಿ
ಸಂಯುಕ್ತ
ಸಂಯೋ-ಗ-ದಿಂದ
ಸಂಯೋ-ಗಾತ್
ಸಂರ-ಕ್ಷಣ
ಸಂರ-ಕ್ಷಿ-ಸ-ಬೇ-ಕಾ-ಗಿ-ದೆಯೊ
ಸಂರ-ಕ್ಷಿ-ಸಿ-ಕೊಂ-ಡಿ-ರು-ವುದು
ಸಂರ-ಕ್ಷಿ-ಸು-ವನು
ಸಂರ-ಕ್ಷಿ-ಸು-ವುದು
ಸಂಲ-ಗ್ನ-ವಾ-ಗಿವೆ
ಸಂಲ-ಗ್ನ-ವಾ-ಗುವ
ಸಂವಾದ
ಸಂವಾ-ದ-ಮಾ-ವ-ಯೋಃ
ಸಂವಾ-ದ-ಮಿ-ಮ-ಮ-ದ್ಭು-ತಮ್
ಸಂವಾ-ದ-ಮಿ-ಮ-ಮ-ಶ್ರೌ-ಷ-ಮ-ದ್ಭುತಂ
ಸಂವಾ-ದ-ರೂ-ಪದ್ದು
ಸಂವಾ-ದ-ರೂ-ಪ-ವಾ-ಗಿ-ರು-ವುದು
ಸಂವಾ-ದ-ವನ್ನು
ಸಂವಾದೇ
ಸಂವೃತ್ತಃ
ಸಂವೇ-ದನೆ
ಸಂವೇ-ದ-ನೆಗೆ
ಸಂಶಯ
ಸಂಶಯಂ
ಸಂಶಯಃ
ಸಂಶ-ಯಕ್ಕೆ
ಸಂಶ-ಯ-ಗಳ
ಸಂಶ-ಯ-ಗಳನ್ನು
ಸಂಶ-ಯ-ಗಳು
ಸಂಶ-ಯ-ಗಳೂ
ಸಂಶ-ಯ-ಗ-ಳೆಲ್ಲ
ಸಂಶ-ಯ-ಗ-ಳೊಂ-ದಿಗೆ
ಸಂಶ-ಯ-ಗ್ರ-ಸ್ಥ-ವಾಗು
ಸಂಶ-ಯದ
ಸಂಶ-ಯ-ದಿಂದ
ಸಂಶ-ಯ-ಪ-ಡ-ಬ-ಹುದು
ಸಂಶ-ಯ-ರ-ಹಿತ
ಸಂಶ-ಯ-ವನ್ನು
ಸಂಶ-ಯ-ವಾ-ದಿ-ಗಳು
ಸಂಶ-ಯ-ವಿಲ್ಲ
ಸಂಶ-ಯ-ವಿ-ಲ್ಲ-ದಿ-ರಲಿ
ಸಂಶ-ಯವೂ
ಸಂಶ-ಯಾತ್ಮ
ಸಂಶ-ಯಾ-ತ್ಮ-ಕ-ವಾದ
ಸಂಶ-ಯಾ-ತ್ಮನಃ
ಸಂಶ-ಯಾ-ತ್ಮ-ನಿಗೆ
ಸಂಶ-ಯಾ-ತ್ಮನು
ಸಂಶ-ಯಾ-ತ್ಮನೋ
ಸಂಶ-ಯಾತ್ಮಾ
ಸಂಶ-ಯಾ-ವ-ಸ್ಥೆ-ಯನ್ನು
ಸಂಶ-ಯಾ-ಸ್ಯಾಸ್ಯ
ಸಂಶಿತ
ಸಂಶು-ದ್ಧ-ಕಿ-ಲ್ಬಿಷಃ
ಸಂಶು-ದ್ಧಿ-ಯಾ-ಗ-ಬೇ-ಕಾ-ದರೆ
ಸಂಶ್ರಿ-ತಾಃ
ಸಂಸಾರ
ಸಂಸಾ-ರಕ್ಕೆ
ಸಂಸಾ-ರ-ಗಳ
ಸಂಸಾ-ರ-ಚ-ಕ್ರಕ್ಕೆ
ಸಂಸಾ-ರದ
ಸಂಸಾ-ರ-ದಲ್ಲಿ
ಸಂಸಾ-ರ-ದ-ಲ್ಲಿ-ರು-ತ್ತಾನೆ
ಸಂಸಾ-ರ-ದಿಂದ
ಸಂಸಾ-ರ-ವನ್ನು
ಸಂಸಾ-ರ-ವೆಂದರೆ
ಸಂಸಾ-ರ-ವೆಂಬ
ಸಂಸಾ-ರವೇ
ಸಂಸಾ-ರ-ಸಾ-ಗ-ರ-ದ-ಲ್ಲಿ-ರುವ
ಸಂಸಾ-ರೇಷು
ಸಂಸಿದ್ಧ
ಸಂಸಿ-ದ್ಧಿಂ
ಸಂಸಿ-ದ್ಧಿ-ಗಾಗಿ
ಸಂಸಿ-ದ್ಧಿ-ಮಾ-ಸ್ಥಿತಾ
ಸಂಸಿ-ದ್ಧಿ-ಯನ್ನು
ಸಂಸಿದ್ಧೌ
ಸಂಸ್ಕಾರ
ಸಂಸ್ಕಾ-ರಕ್ಕೆ
ಸಂಸ್ಕಾ-ರ-ಗಳ
ಸಂಸ್ಕಾ-ರ-ಗಳನ್ನು
ಸಂಸ್ಕಾ-ರ-ಗಳನ್ನೂ
ಸಂಸ್ಕಾ-ರ-ಗಳನ್ನೆಲ್ಲ
ಸಂಸ್ಕಾ-ರ-ಗಳನ್ನೆಲ್ಲಾ
ಸಂಸ್ಕಾ-ರ-ಗ-ಳಾ-ವು-ದನ್ನೂ
ಸಂಸ್ಕಾ-ರ-ಗಳಿಂದ
ಸಂಸ್ಕಾ-ರ-ಗ-ಳಿಗೆ
ಸಂಸ್ಕಾ-ರ-ಗ-ಳಿವೆ
ಸಂಸ್ಕಾ-ರ-ಗಳು
ಸಂಸ್ಕಾ-ರ-ಗಳೂ
ಸಂಸ್ಕಾ-ರ-ಗ-ಳೆಂಬ
ಸಂಸ್ಕಾ-ರ-ಗ-ಳೆಲ್ಲ
ಸಂಸ್ಕಾ-ರ-ಗಳೇ
ಸಂಸ್ಕಾ-ರದ
ಸಂಸ್ಕಾ-ರ-ದಂತೆ
ಸಂಸ್ಕಾ-ರ-ದಿಂದ
ಸಂಸ್ಕಾ-ರ-ದೊ-ಡನೆ
ಸಂಸ್ಕಾ-ರ-ವನ್ನು
ಸಂಸ್ಕಾ-ರ-ವಾಗಿ
ಸಂಸ್ಕಾ-ರವೂ
ಸಂಸ್ಕಾ-ರ-ವೆಲ್ಲ
ಸಂಸ್ಕೃ-ತ-ಭಾಷೆ
ಸಂಸ್ಕೃತಿ
ಸಂಸ್ಕೃ-ತಿಗೆ
ಸಂಸ್ಕೃ-ತಿ-ಯಲ್ಲಿ
ಸಂಸ್ತ-ಭ್ಯಾ-ತ್ಮಾ-ನ-ಮಾ-ತ್ಮನಾ
ಸಂಸ್ಥಾ-ಪನೆ
ಸಂಸ್ಥಾ-ಪ-ನೆಗೆ
ಸಂಸ್ಥಾ-ಪ-ನೆಯ
ಸಂಸ್ಪ-ರ್ಶಜಾ
ಸಂಸ್ಮೃತ್ಯ
ಸಂಹ-ರತೇ
ಸಂಹ-ರಿ-ಸಿ-ದ್ದೇನೆ
ಸಂಹ-ರಿ-ಸು-ತ್ತಾನೆ
ಸಂಹ-ರಿ-ಸುವ
ಸಂಹಾರ
ಸಂಹಾ-ರಕ
ಸಂಹಾ-ರ-ಕನ
ಸಂಹಾ-ರ-ಕ-ಶಕ್ತಿ
ಸಂಹಾ-ರಕ್ಕೆ
ಸಂಹಾ-ರ-ಗಳು
ಸಂಹಾ-ರದ
ಸಂಹಾ-ರ-ವಾ-ದರೆ
ಸಃ
ಸಕಲ
ಸಕ-ಲ-ವನ್ನೂ
ಸಕ-ಲವೂ
ಸಕ-ಲಾ-ಧಾರ
ಸಕಾ-ಮ-ನಾಗಿ
ಸಕಾಲ
ಸಕಾ-ಲ-ದಲ್ಲಿ
ಸಕ್ಕರೆ
ಸಕ್ಕ-ರೆಯ
ಸಕ್ಕ-ರೆ-ಯಂ-ತಿದೆ
ಸಕ್ಕ-ರೆ-ಯನ್ನು
ಸಕ್ಕ-ರೆ-ಯ-ಲ್ಲದೆ
ಸಕ್ಕ-ರೆ-ಯಿಂದ
ಸಕ್ಕ-ರೆಯೇ
ಸಕ್ಕ-ರೆ-ಯೊಂದೇ
ಸಕ್ತ-ಮ-ಹೈ-ತು-ಕಮ್
ಸಕ್ತಾಃ
ಸಕ್ತೋ
ಸಖ
ಸಖ-ನಲ್ಲ
ಸಖ-ನಾ-ಗಿದ್ದ
ಸಖ-ನಾದ
ಸಖಾ
ಸಖೀಂ-ಸ್ತಥಾ
ಸಖೇತಿ
ಸಖೇವ
ಸಖ್ಯುಃ
ಸಗ-ದ್ಗದಂ
ಸಗುಣ
ಸಗು-ಣ
ಸಗು-ಣ-ದಲ್ಲಿ
ಸಗು-ಣ-ದಿಂದ
ಸಗು-ಣ-ದೇ-ವ-ರನ್ನು
ಸಗು-ಣ-ವನ್ನು
ಸಗು-ಣೋ-ಪಾ-ಸನೆ
ಸಗು-ಣೋ-ಪಾ-ಸ-ನೆಯೇ
ಸಚ-ರಾ-ಚ-ರಮ್
ಸಚ-ಲ-ದೇ-ವಾ-ಲಯ
ಸಚೇ-ತನ
ಸಚೇ-ತಾಃ
ಸಚ್ಚಿ-ದಾ-ನಂದ
ಸಚ್ಚಿ-ದಾ-ನಂ-ದ-ವಾ-ಗಿ-ರು-ವುದು
ಸಚ್ಚಿ-ದಾ-ನಂ-ದ-ಸ್ವ-ರೂ-ಪ-ನಾದ
ಸಚ್ಚಿ-ದಾ-ನಂ-ದ-ಸ್ವ-ರೂ-ಪ-ವನ್ನು
ಸಚ್ಛಬ್ದಃ
ಸಜ
ಸಜ್ಜಂತೇ
ಸಜ್ಜತೇ
ಸಜ್ಜ-ನ-ನಾ-ಗು-ತ್ತಾನೆ
ಸಜ್ಜ-ನ-ರನ್ನು
ಸಜ್ಜ-ನ-ರಿಗೆ
ಸಜ್ಜ-ನರು
ಸಜ್ಜ-ನ-ರು-ಎಂ-ದರೆ
ಸಡಿಲ
ಸಡಿ-ಲ-ವಾ-ಗುತ್ತ
ಸಡಿ-ಲ-ವಾ-ದರೆ
ಸಣ್ಣ
ಸಣ್ಣ-ಕ-ಣ-ವಾ-ಗ-ಬ-ಹುದು
ಸಣ್ಣ-ಕ-ವ-ನವೊ
ಸಣ್ಣ-ತನ
ಸಣ್ಣ-ತ-ನಕ್ಕೆ
ಸಣ್ಣ-ತ-ನ-ವಿ-ಲ್ಲದೆ
ಸಣ್ಣ-ದನ್ನು
ಸಣ್ಣ-ದರ
ಸಣ್ಣ-ದ-ರಲ್ಲಿ
ಸಣ್ಣ-ದ-ರೊ-ಳಗೆ
ಸಣ್ಣ-ದಾ-ಗಿ-ದೆಯೋ
ಸಣ್ಣ-ದಾ-ಗಿ-ರ-ಬ-ಹುದು
ಸಣ್ಣ-ದಾದ
ಸಣ್ಣ-ದಾ-ದರೂ
ಸಣ್ಣದು
ಸಣ್ಣ-ದು-ದೊ-ಡ್ಡದು
ಸಣ್ಣ-ದೊಂದು
ಸಣ್ಣ-ಪುಟ್ಟ
ಸಣ್ಣ-ಬೀ-ಜ-ದ-ಲ್ಲಿ-ರು-ವುದೊ
ಸಣ್ಣವು
ಸಣ್ಣ-ಸಣ್ಣ
ಸಣ್ಣಾ-ದಾಗಿ
ಸತಃ
ಸತತ
ಸತತಂ
ಸತ-ತ-ಮಾ-ತ್ಮಾನಂ
ಸತ-ತ-ಯುಕ್ತಾ
ಸತ-ತ-ಯು-ಕ್ತಾ-ನಾಂ
ಸತ-ತವೂ
ಸತಿ
ಸತಿ-ಪತಿ
ಸತ್
ಸತ್ಕ-ರ್ಮದ
ಸತ್ಕಾರ
ಸತ್ಕಾ-ರ-ಮಾ-ನ-ಪೂ-ಜಾರ್ಥಂ
ಸತ್ಕಾ-ರ-ವಿ-ಲ್ಲದೆ
ಸತ್ಕು-ಲ-ದ-ಲ್ಲಿಯೇ
ಸತ್ತ
ಸತ್ತಂ-ತಿ-ರು-ವನು
ಸತ್ತ-ನ್ನಾ-ಸ-ದು-ಚ್ಯತೇ
ಸತ್ತ-ಮೇಲೆ
ಸತ್ತ-ಮೇ-ಲೆಯೂ
ಸತ್ತರೆ
ಸತ್ತ-ವ-ನನ್ನು
ಸತ್ತ-ವ-ನಿಗೆ
ಸತ್ತ-ವನು
ಸತ್ತ-ವ-ರನ್ನೆ
ಸತ್ತ-ವ-ರಾರೂ
ಸತ್ತ-ವ-ರಿ-ಗಾ-ಗಲಿ
ಸತ್ತ-ವರೆಲ್ಲ
ಸತ್ತಾಗ
ಸತ್ತಾ-ದ-ಮೇಲೆ
ಸತ್ತಿಲ್ಲ
ಸತ್ತು
ಸತ್ತು-ಬೀ-ಳು-ವರು
ಸತ್ತು-ಹೋ-ದರೆ
ಸತ್ತೇ
ಸತ್ತ್ವ
ಸತ್ತ್ವಂ
ಸತ್ತ್ವಕ್ಕೂ
ಸತ್ತ್ವಕ್ಕೆ
ಸತ್ತ್ವ-ಗುಣ
ಸತ್ತ್ವ-ಗು-ಣಕ್ಕೆ
ಸತ್ತ್ವ-ಗು-ಣಕ್ಕೋ
ಸತ್ತ್ವ-ಗು-ಣ-ಗಳನ್ನು
ಸತ್ತ್ವ-ಗು-ಣ-ಗಳಲ್ಲಿ
ಸತ್ತ್ವ-ಗು-ಣದ
ಸತ್ತ್ವ-ಗು-ಣ-ದಲ್ಲಿ
ಸತ್ತ್ವ-ಗು-ಣ-ದ-ಲ್ಲಿ-ದ್ದರೂ
ಸತ್ತ್ವ-ಗು-ಣ-ದ-ಲ್ಲಿ-ರು-ವ-ವರು
ಸತ್ತ್ವ-ಗು-ಣ-ದಿಂದ
ಸತ್ತ್ವ-ಗು-ಣ-ವನ್ನು
ಸತ್ತ್ವ-ಗು-ಣ-ವಿದೆ
ಸತ್ತ್ವ-ಗು-ಣವೇ
ಸತ್ತ್ವ-ಗುಣಿ
ಸತ್ತ್ವ-ಗು-ಣಿಯ
ಸತ್ತ್ವ-ಗು-ಣಿ-ಯಾ-ದರೊ
ಸತ್ತ್ವದ
ಸತ್ತ್ವ-ದೃ-ಷ್ಟಿ-ಯಿಂದ
ಸತ್ತ್ವ-ಮಾಹೋ
ಸತ್ತ್ವ-ಮಿ-ತ್ಯುತ
ಸತ್ತ್ವ-ವ-ತಾ-ಮ-ಹಮ್
ಸತ್ತ್ವ-ವಿ-ಲ್ಲ-ದುವು
ಸತ್ತ್ವ-ವೆಂಬ
ಸತ್ತ್ವ-ಸಂ-ಶುದ್ಧಿ
ಸತ್ತ್ವ-ಸಂ-ಶು-ದ್ಧಿ-ರ್ಜ್ಞಾ-ನ-ಯೋ-ಗ-ವ್ಯ-ವ-ಸ್ಥಿ-ತಿಃ
ಸತ್ತ್ವ-ಸ-ಮಾ-ವಿಷ್ಟೋ
ಸತ್ತ್ವ-ಸ್ಥನೂ
ಸತ್ತ್ವಸ್ಥಾ
ಸತ್ತ್ವಾತ್
ಸತ್ತ್ವಾ-ನು-ರೂಪಾ
ಸತ್ತ್ವೇ
ಸತ್ಪು-ತ್ರನ
ಸತ್ಯ
ಸತ್ಯಂ
ಸತ್ಯ-ಕನ
ಸತ್ಯ-ಕಾಮ
ಸತ್ಯ-ಕಾ-ಮನ
ಸತ್ಯ-ಕ್ಕಿಂತ
ಸತ್ಯಕ್ಕೂ
ಸತ್ಯಕ್ಕೆ
ಸತ್ಯ-ಗಳನ್ನು
ಸತ್ಯ-ಗ-ಳ-ನ್ನೊ-ಳ-ಗೊಂಡ
ಸತ್ಯ-ಗ-ಳಿಲ್ಲ
ಸತ್ಯ-ಗ-ಳಿವೆ
ಸತ್ಯ-ಗಳು
ಸತ್ಯ-ಗಳೂ
ಸತ್ಯತ್ವ
ಸತ್ಯದ
ಸತ್ಯ-ದಂತೆ
ಸತ್ಯ-ದಲ್ಲಿ
ಸತ್ಯ-ದಿಂದ
ಸತ್ಯ-ದೆ-ಡೆಗೆ
ಸತ್ಯ-ದೊಂ-ದಿಗೆ
ಸತ್ಯ-ನಿ-ಷ್ಠ-ರಿಗೆ
ಸತ್ಯ-ಮ-ಕ್ರೋ-ಧ-ಸ್ತ್ಯಾಗಃ
ಸತ್ಯ-ವನ್ನು
ಸತ್ಯ-ವನ್ನೇ
ಸತ್ಯ-ವಲ್ಲ
ಸತ್ಯ-ವ-ಲ್ಲವೋ
ಸತ್ಯ-ವಾ-ಗ-ಬೇ-ಕಾ-ದರೆ
ಸತ್ಯ-ವಾಗಿ
ಸತ್ಯ-ವಾ-ಗಿದೆ
ಸತ್ಯ-ವಾ-ಗಿ-ದ್ದರೆ
ಸತ್ಯ-ವಾ-ಗಿಯೂ
ಸತ್ಯ-ವಾ-ಗಿ-ರು-ವ-ವನು
ಸತ್ಯ-ವಾ-ಗಿ-ರು-ವುದು
ಸತ್ಯ-ವಾ-ಗು-ವು-ದಿಲ್ಲ
ಸತ್ಯ-ವಾದ
ಸತ್ಯ-ವಾ-ದರೂ
ಸತ್ಯ-ವಾ-ದರೆ
ಸತ್ಯ-ವಾ-ದುದು
ಸತ್ಯ-ವಿದೆ
ಸತ್ಯವೂ
ಸತ್ಯವೆ
ಸತ್ಯ-ವೆಂದು
ಸತ್ಯ-ವೆಂ-ಬಂತೆ
ಸತ್ಯ-ವೆಲ್ಲ
ಸತ್ಯ-ವೆ-ಲ್ಲಿಗೆ
ಸತ್ಯವೇ
ಸತ್ಯ-ವೇನು
ಸತ್ಯ-ವೇನೋ
ಸತ್ಯ-ವೊಂದೇ
ಸತ್ಯವೋ
ಸತ್ಯ-ಸಂ-ಕಲ್ಪ
ಸತ್ಯ-ಸಾ-ಕ್ಷಾ-ತ್ಕಾರ
ಸತ್ಯ-ಸೂರ್ಯ
ಸತ್ಯ-ಸ್ಥಿತಿ
ಸತ್ಯಸ್ಯ
ಸತ್ಯ-ಸ್ಯ-ಸ-ತ್ಯ-ವಾದ
ಸತ್ಯಾ-ಕಾಂ-ಕ್ಷಿ-ಯಾ-ಗಿಲ್ಲ
ಸತ್ಯಾ-ನ್ವೇ-ಷ-ಣೆ-ಯಲ್ಲ
ಸತ್ವ
ಸತ್ವ-ಗುಣ
ಸತ್ವ-ಗು-ಣ-ವನ್ನು
ಸತ್ವದ
ಸತ್ವ-ವನ್ನು
ಸತ್ವ-ಸ್ತು-ವಿಗೆ
ಸತ್ಸಂ-ಸ್ಕಾ-ರ-ಗಳ
ಸದ-ಭ್ಯಾ-ಸದ
ಸದರ
ಸದ-ರ-ದಿಂದ
ಸದ-ರ-ವನ್ನು
ಸದ-ವ-ಕಾಶ
ಸದ-ಸ-ಚ್ಚಾ-ಹ-ಮ-ರ್ಜುನ
ಸದ-ಸ-ತ್ತ-ತ್ಪರಂ
ಸದ-ಸ-ದ್ಯೋ-ನಿ-ಜ-ನ್ಮಸು
ಸದಾ
ಸದಾ-ಕಾಲ
ಸದಾ-ಕಾ-ಲ-ದಲ್ಲಿ
ಸದಾ-ಕಾ-ಲ-ದ-ಲ್ಲಿಯೂ
ಸದಾ-ಕಾ-ಲವೂ
ಸದಾ-ಚಾರ
ಸದಾ-ಚಾ-ರ-ಶೀ-ಲರ
ಸದಾ-ಚಾ-ರ-ಶೀ-ಲ-ರಲ್ಲ
ಸದಾ-ಚಾ-ರ-ಶೀ-ಲರು
ಸದಾ-ಚಾ-ರ-ಶೀ-ಲರೂ
ಸದಾ-ಚಾ-ರಿ-ಗ-ಳಾದ
ಸದಾ-ಚಾ-ರಿ-ಗಳು
ಸದಾ-ಚಾ-ರಿಗೆ
ಸದಾ-ತ್ಮಾನಂ
ಸದಿತಿ
ಸದಿ-ತ್ಯೇ-ತ-ತ್ಪ್ರ-ಯು-ಜ್ಯತೇ
ಸದಿ-ತ್ಯೇ-ವಾ-ಭಿ-ಧೀ-ಯತೇ
ಸದೃಶ
ಸದೃಶಂ
ಸದೃ-ಶ-ರಾ-ಗು-ವೆವು
ಸದೃ-ಶ-ವಾಗು
ಸದೃ-ಶ-ವಾ-ಗು-ವುದು
ಸದೃಶೀ
ಸದೃಶೋ
ಸದೆ
ಸದೋ-ಷ-ಮಪಿ
ಸದ್
ಸದ್ಗತಿ
ಸದ್ಗು-ಣ-ಗಳನ್ನು
ಸದ್ಗು-ಣ-ಗಳನ್ನೂ
ಸದ್ಗು-ಣ-ಗ-ಳೆಂಬ
ಸದ್ಗುರು
ಸದ್ಗ್ರಾ-ಹಾನ್
ಸದ್ದು-ಮಾ-ಡಿ-ಕೊಂಡು
ಸದ್ದೇ
ಸದ್ಬು-ದ್ಧಿ-ಯನ್ನು
ಸದ್ಭಾವೇ
ಸದ್ಯಕ್ಕೆ
ಸದ್ವಿ-ನಿ-ಯೋಗ
ಸದ್ವೃ-ತ್ತಿ-ಗಳನ್ನೆಲ್ಲ
ಸಧ್ಯಕ್ಕೆ
ಸನಾ-ತನ
ಸನಾ-ತನಃ
ಸನಾ-ತ-ನ-ನಾದ
ಸನಾ-ತ-ನನೊ
ಸನಾ-ತ-ನಮ್
ಸನಾ-ತ-ನ-ವಾದ
ಸನಾ-ತ-ನ-ವಾ-ದುದು
ಸನಾ-ತ-ನಸ್ತ್ವಂ
ಸನಾ-ತ-ನಾಃ
ಸನ್
ಸನ್ನ-ದ್ಧ-ನಾ-ಗಿ-ರು-ವನೋ
ಸನ್ನ-ವ್ಯ-ಯಾತ್ಮಾ
ಸನ್ನಾಹ
ಸನ್ನಾ-ಹ-ವಾ-ಗು-ತ್ತಿ-ರು-ವುದು
ಸನ್ನಿ
ಸನ್ನಿ-ಧಿ-ಯಲ್ಲಿ
ಸನ್ನಿ-ವೇಶ
ಸನ್ನಿ-ವೇ-ಶ-ಗಳಲ್ಲಿ
ಸನ್ನಿ-ವೇ-ಶ-ಗಳು
ಸನ್ನಿ-ವೇ-ಶದ
ಸನ್ನಿ-ವೇ-ಶ-ದಲ್ಲಿ
ಸನ್ನಿ-ವೇ-ಶ-ವನ್ನೊ
ಸನ್ನಿ-ಹಿ-ತ-ವಾ-ಯಿ-ತೆಂದು
ಸನ್ನೆ
ಸನ್ನೆಗೆ
ಸನ್ನೆಯ
ಸನ್ಮಾನ
ಸನ್ಮಾ-ನ-ವೇನೋ
ಸನ್ಮಾ-ನಿ-ಸು-ತ್ತಾ-ನೆಯೆ
ಸನ್ಮಾ-ರ್ಗಿ-ಗಳ
ಸಪ-ತ್ನಾನ್
ಸಪ್ತ
ಸಪ್ತ-ಪು-ಷಿ-ಗಳ
ಸಪ್ತ-ಪು-ಷಿ-ಗಳು
ಸಪ್ತ-ಮ-ಹ-ರ್ಷಿ-ಗಳಲ್ಲಿ
ಸಪ್ತ-ಮ-ಹ-ರ್ಷಿ-ಗಳು
ಸಪ್ಪಳ
ಸಪ್ಪೆ
ಸಪ್ಪೆ-ಯಾಗಿ
ಸಪ್ಪೆ-ಯಾ-ಗು-ವು-ದಿಲ್ಲ
ಸಪ್ಪೆ-ಯಾ-ಗು-ವುದು
ಸಪ್ಪೆ-ಯಾ-ಗು-ವುವು
ಸಬ-ಲ-ನ-ನ್ನಾಗಿ
ಸಭೆ-ಯಲ್ಲಿ
ಸಮ
ಸಮಂ
ಸಮಂ-ತತಃ
ಸಮಂತಾ
ಸಮಃ
ಸಮ-ಕಾ-ಲೀನ
ಸಮಗ್ರ
ಸಮಗ್ರಂ
ಸಮ-ಗ್ರಾನ್
ಸಮ-ಚಿ-ತ್ತತೆ
ಸಮ-ಚಿ-ತ್ತ-ತ್ವ-ಮಿ-ಷ್ಟಾ-ನಿ-ಷ್ಟೋ-ಪ-ಪ-ತ್ತಿಷು
ಸಮ-ಚಿ-ತ್ತ-ನಾ-ಗಿ-ರು-ವುದು
ಸಮತಾ
ಸಮ-ತೀ-ತಾನಿ
ಸಮ-ತೀ-ತ್ಯೈ-ತಾನ್
ಸಮ-ತೂ-ಕ-ದ-ಲ್ಲಿವೆ
ಸಮ-ತೂ-ಕ-ವಾ-ಗಿ-ರ-ಬೇಕು
ಸಮತ್ವ
ಸಮತ್ವಂ
ಸಮ-ತ್ವ-ಗುಣ
ಸಮ-ತ್ವದ
ಸಮ-ತ್ವ-ದಿಂದ
ಸಮ-ತ್ವ-ದೃ-ಷ್ಟಿ-ಯಿಂದ
ಸಮ-ತ್ವ-ದೃ-ಷ್ಟಿ-ಯೊಂದು
ಸಮ-ತ್ವ-ಬು-ದ್ಧಿ-ಯನ್ನು
ಸಮ-ತ್ವ-ಬು-ದ್ಧಿ-ಯಲ್ಲಿ
ಸಮ-ತ್ವ-ಬು-ದ್ಧಿ-ಯಿಂದ
ಸಮ-ತ್ವ-ಬು-ದ್ಧಿ-ಯು-ಳ್ಳ-ವನು
ಸಮ-ತ್ವ-ವನ್ನು
ಸಮ-ತ್ವ-ವನ್ನೇ
ಸಮ-ತ್ವ-ವ-ನ್ನೈದಿ
ಸಮ-ತ್ವವೇ
ಸಮ-ದ-ರ್ಶನಃ
ಸಮ-ದ-ರ್ಶಿ-ಗ-ಳಾ-ಗಿ-ರು-ವರು
ಸಮ-ದ-ರ್ಶಿನಃ
ಸಮ-ದ-ರ್ಶಿ-ಯಾ-ಗಿ-ರುವ
ಸಮ-ದ-ರ್ಶಿ-ಯಾದ
ಸಮ-ದುಃ-ಖ-ಸುಖಂ
ಸಮ-ದುಃ-ಖ-ಸುಖಃ
ಸಮ-ದೃಷ್ಟಿ
ಸಮ-ಧಿ-ಗ-ಚ್ಛತಿ
ಸಮನಾ
ಸಮ-ನಾಗಿ
ಸಮ-ನಾ-ಗಿದೆ
ಸಮ-ನಾ-ಗಿ-ದ್ದಾನೆ
ಸಮ-ನಾ-ಗಿ-ಬೀ-ಳು-ತ್ತದೆ
ಸಮ-ನಾ-ಗಿ-ರಲೇ
ಸಮ-ನಾ-ಗಿರು
ಸಮ-ನಾ-ಗಿ-ರು-ತ್ತದೆ
ಸಮ-ನಾ-ಗಿ-ರು-ತ್ತಾನೆ
ಸಮ-ನಾ-ಗಿ-ರು-ತ್ತೇನೆ
ಸಮ-ನಾ-ಗಿ-ರುವ
ಸಮ-ನಾ-ಗಿ-ರು-ವನು
ಸಮ-ನಾ-ಗಿ-ರು-ವನೋ
ಸಮ-ನಾ-ಗಿ-ರು-ವ-ವ-ನಿಗೆ
ಸಮ-ನಾ-ಗಿ-ರು-ವ-ವನು
ಸಮ-ನಾ-ಗಿ-ರು-ವ-ವನೂ
ಸಮ-ನಾ-ಗಿ-ರು-ವು-ದಿಲ್ಲ
ಸಮ-ನಾ-ಗಿ-ರು-ವುದು
ಸಮ-ನಾ-ಗಿ-ರು-ವು-ದೊಂದು
ಸಮ-ನಾ-ಗು-ವುದು
ಸಮ-ನಾದ
ಸಮ-ನಾ-ದ-ವರು
ಸಮ-ನಾ-ದ-ವರೂ
ಸಮ-ನಾ-ದುದು
ಸಮನೆ
ಸಮನೇ
ಸಮನ್
ಸಮ-ನ್ನನ್ನು
ಸಮ-ನ್ವ-ಯ-ವನ್ನು
ಸಮ-ನ್ವ-ಯವೇ
ಸಮ-ಪ್ರ-ಮಾ-ಣ-ದಲ್ಲಿ
ಸಮ-ಬು-ದ್ಧಯಃ
ಸಮ-ಬುದ್ಧಿ
ಸಮ-ಬು-ದ್ಧಿ-ಯು-ಳ್ಳ-ವನು
ಸಮ-ಬು-ದ್ಧಿ-ಯು-ಳ್ಳ-ವನೊ
ಸಮ-ಬು-ದ್ಧಿ-ಯು-ಳ್ಳ-ವ-ರಾಗಿ
ಸಮ-ಬು-ದ್ಧಿ-ರ್ವಿ-ಶಿ-ಷ್ಯತೇ
ಸಮ-ಭಾವ
ಸಮ-ಭಾ-ವ-ದಿಂದ
ಸಮ-ಭಾ-ವ-ನೆ-ಯು-ಳ್ಳ-ವನು
ಸಮ-ಮಾಡಿ
ಸಮಯ
ಸಮ-ಯಕ್ಕೆ
ಸಮ-ಯ-ಗಳಲ್ಲಿ
ಸಮ-ಯ-ಗ-ಳ-ಲ್ಲಿಯೂ
ಸಮ-ಯದ
ಸಮ-ಯ-ದಲ್ಲಿ
ಸಮ-ಯ-ದ-ಲ್ಲಿಯೂ
ಸಮ-ಯ-ದ-ಲ್ಲಿ-ರು-ವನು
ಸಮ-ಯ-ದಲ್ಲೆ
ಸಮ-ಯ-ವನ್ನು
ಸಮ-ಯ-ವಿಲ್ಲ
ಸಮ-ಯ-ವೆ-ಲ್ಲಿತ್ತು
ಸಮ-ಯ-ವೆ-ಲ್ಲಿದೆ
ಸಮ-ಯವೇ
ಸಮರ
ಸಮ-ರ-ಜ-ಯಿ-ಯಾ-ಗಿರು
ಸಮ-ರ-ಜ-ಯಿ-ಯಾದ
ಸಮ-ರಥ
ಸಮ-ರ-ಥ-ರೊ-ಡನೆ
ಸಮ-ರ-ದಲ್ಲಿ
ಸಮ-ರಾಂ-ಗಣ
ಸಮ-ರಾಂ-ಗ-ಣ-ದಲ್ಲಿ
ಸಮ-ರ್ಥ-ನಾ-ಗ-ದಿ-ದ್ದರೆ
ಸಮ-ರ್ಥ-ನಾ-ಗು-ತ್ತಾನೆ
ಸಮ-ರ್ಥನೆ
ಸಮ-ರ್ಥಿ-ಸ-ಕೂ-ಡದು
ಸಮ-ರ್ಥಿಸಿ
ಸಮ-ರ್ಥಿ-ಸಿದ್ದು
ಸಮ-ರ್ಥಿ-ಸಿ-ರು-ವರು
ಸಮ-ರ್ಥಿ-ಸು-ತ್ತಾನೆ
ಸಮ-ರ್ಥಿ-ಸು-ವನು
ಸಮ-ರ್ಥಿ-ಸು-ವು-ದಕ್ಕೆ
ಸಮ-ರ್ಥಿ-ಸು-ವುದು
ಸಮ-ರ್ಪ-ಕ-ವಾ-ಗಿ-ರು-ವಂತೆ
ಸಮ-ರ್ಪ-ಕ-ವಾ-ಗಿ-ಲ್ಲದೆ
ಸಮ-ರ್ಪ-ಕ-ವಾದ
ಸಮ-ರ್ಪಿ-ಸಿದ
ಸಮ-ರ್ಪಿ-ಸು-ವಷ್ಟು
ಸಮ-ಲೋ-ಷ್ವಾಶ್ಮ
ಸಮ-ಲೋ-ಷ್ವಾ-ಶ್ಮ-ಕಾಂ-ಚನಃ
ಸಮ-ವ-ಸ್ಥಿ-ತ-ಮೀ-ಶ್ವ-ರಮ್
ಸಮ-ವಾಗಿ
ಸಮ-ವಾ-ಗಿಯೂ
ಸಮ-ವೇತಾ
ಸಮ-ವೇ-ತಾನ್
ಸಮಷ್ಟಿ
ಸಮ-ಷ್ಟಿಗೆ
ಸಮ-ಷ್ಟಿ-ಜೀ-ವ-ನಕ್ಕೆ
ಸಮ-ಷ್ಟಿ-ಜೀ-ವ-ನವೇ
ಸಮ-ಷ್ಟಿಯ
ಸಮಸ್ತ
ಸಮ-ಸ್ತ-ಪು-ರು-ಷಾ-ರ್ಥ-ಸಿ-ದ್ಧಿಃ
ಸಮ-ಸ್ಪ-ರ್ಧಿಯೇ
ಸಮಸ್ಯೆ
ಸಮ-ಸ್ಯೆ-ಗಳು
ಸಮ-ಸ್ಯೆ-ಗೆಲ್ಲ
ಸಮ-ಸ್ಯೆ-ಯನ್ನು
ಸಮ-ಸ್ಯೆ-ಯಿಂದ
ಸಮ-ಸ್ಯೆಯೇ
ಸಮ-ಸ್ಯೆ-ಯೊಂ-ದಿಗೆ
ಸಮಾಃ
ಸಮಾ-ಗ-ತಾಃ
ಸಮಾ-ಚರ
ಸಮಾ-ಚ-ರನ್
ಸಮಾ-ಚಾರ
ಸಮಾ-ಚಾ-ರ-ಗಳನ್ನು
ಸಮಾ-ಚಾ-ರ-ಗಳು
ಸಮಾ-ಚಾ-ರ-ವನ್ನು
ಸಮಾ-ಚಾ-ರ-ವನ್ನೂ
ಸಮಾ-ಚಾ-ರವೂ
ಸಮಾಜ
ಸಮಾ-ಜ-ಕಂ-ಟ-ಕ-ನಿಗೆ
ಸಮಾ-ಜ-ಕಂ-ಟ-ಕ-ರಾದ
ಸಮಾ-ಜಕ್ಕೂ
ಸಮಾ-ಜಕ್ಕೆ
ಸಮಾ-ಜ-ಘಾ-ತುಕ
ಸಮಾ-ಜ-ಘಾ-ತು-ಕ-ನಾ-ಗು-ತ್ತಾನೆ
ಸಮಾ-ಜ-ಘಾ-ತು-ಕ-ರಂತೆ
ಸಮಾ-ಜದ
ಸಮಾ-ಜ-ದ-ಮೇಲೆ
ಸಮಾ-ಜ-ದಲ್ಲಿ
ಸಮಾ-ಜ-ದ-ಲ್ಲಿಯೂ
ಸಮಾ-ಜ-ದ-ಲ್ಲಿ-ರು-ವನು
ಸಮಾ-ಜ-ದ-ಲ್ಲಿ-ರು-ವನೊ
ಸಮಾ-ಜ-ದಲ್ಲೆಲ್ಲ
ಸಮಾ-ಜ-ದಿಂದ
ಸಮಾ-ಜ-ದೊ-ಳಗೆ
ಸಮಾ-ಜ-ಬಾ-ಹಿ-ರ-ರಿಗೂ
ಸಮಾ-ಜ-ಯಂ-ತ್ರವೂ
ಸಮಾ-ಜ-ವ-ನ್ನಾ-ಗಲಿ
ಸಮಾ-ಜ-ವನ್ನು
ಸಮಾ-ಜ-ವ-ನ್ನೆಲ್ಲ
ಸಮಾ-ಜ-ವನ್ನೇ
ಸಮಾ-ಜ-ವಾ-ಗಲೀ
ಸಮಾ-ಜ-ವಿದೆ
ಸಮಾ-ಜವೂ
ಸಮಾ-ಜವೇ
ಸಮಾ-ಧಾ-ತುಂ
ಸಮಾ-ಧಾನ
ಸಮಾ-ಧಾ-ನ-ಗಳನ್ನು
ಸಮಾ-ಧಾ-ನ-ಗೊ-ಳಿ-ಸಿ-ದನು
ಸಮಾ-ಧಾ-ನ-ಗೊ-ಳ್ಳು-ವುದು
ಸಮಾ-ಧಾ-ನ-ದಿಂದ
ಸಮಾ-ಧಾ-ನ-ಪಟ್ಟು
ಸಮಾ-ಧಾ-ನ-ವನ್ನು
ಸಮಾ-ಧಾ-ನ-ವಷ್ಟೆ
ಸಮಾ-ಧಾ-ನ-ವಾ-ಗ-ಬ-ಹುದು
ಸಮಾ-ಧಾ-ನ-ವಾ-ಗಲಿ
ಸಮಾ-ಧಾ-ನ-ವಾ-ಗಿ-ರುವ
ಸಮಾ-ಧಾ-ನ-ವಾ-ಗು-ವು-ದಿಲ್ಲ
ಸಮಾ-ಧಾ-ನ-ವಾ-ಗು-ವುದು
ಸಮಾ-ಧಾ-ನ-ವಾ-ದರೂ
ಸಮಾ-ಧಾ-ನ-ವಾ-ದು-ದನ್ನು
ಸಮಾ-ಧಾ-ನ-ವಿದೆ
ಸಮಾ-ಧಾಯ
ಸಮಾ-ಧಾ-ವ-ಚಲಾ
ಸಮಾಧಿ
ಸಮಾ-ಧಿ-ಗಳು
ಸಮಾ-ಧಿಗೆ
ಸಮಾ-ಧಿ-ಯನ್ನು
ಸಮಾ-ಧಿ-ಯಲ್ಲಿ
ಸಮಾ-ಧಿ-ಯ-ಲ್ಲಿ-ರುವ
ಸಮಾ-ಧಿ-ಯಿಂದ
ಸಮಾ-ಧಿ-ಸ್ಥ-ನಾಗಿ
ಸಮಾ-ಧಿ-ಸ್ಥಸ್ಯ
ಸಮಾಧೌ
ಸಮಾನ
ಸಮಾ-ನ-ರಾರು
ಸಮಾ-ನ-ರಿಲ್ಲ
ಸಮಾ-ನ-ವಾಗಿ
ಸಮಾ-ನ-ವಾದ
ಸಮಾ-ನ-ವಿಲ್ಲ
ಸಮಾ-ನ-ವಿ-ಲ್ಲವೋ
ಸಮಾ-ಪ್ನೋಷಿ
ಸಮಾ-ರಂ-ಭಾಃ
ಸಮಾರು
ಸಮಾ-ಸ-ಗಳಲ್ಲಿ
ಸಮಾ-ಸ-ಗ-ಳಿವೆ
ಸಮಾ-ಸತಃ
ಸಮಾ-ಸ-ದಲ್ಲಿ
ಸಮಾ-ಸೇನ
ಸಮಾ-ಸೇ-ನೈವ
ಸಮಾ-ಹ-ರ್ತು-ಮಿಹ
ಸಮಾ-ಹಿತ
ಸಮಾ-ಹಿತಃ
ಸಮಾ-ಹಿ-ತ-ವಾ-ಗಿದೆ
ಸಮಾ-ಹಿ-ತ-ವಾ-ಗಿ-ರು-ವುದು
ಸಮಿ-ತಿಂ-ಜಯಃ
ಸಮಿ-ದ್ಧೋ-ಽಗ್ನಿ-ರ್ಭ-ಸ್ಮ-ಸಾತ್
ಸಮೀಕ್ಷ್ಯ
ಸಮೀಪ
ಸಮೀ-ಪಕ್ಕೆ
ಸಮೀ-ಪದ
ಸಮೀ-ಪ-ದಲ್ಲಿ
ಸಮೀ-ಪ-ದ-ಲ್ಲಿ-ಟ್ಟರೆ
ಸಮೀ-ಪ-ದ-ಲ್ಲಿದೆ
ಸಮೀ-ಪ-ದ-ಲ್ಲಿ-ದ್ದರೆ
ಸಮೀ-ಪ-ದ-ಲ್ಲಿಯೂ
ಸಮೀ-ಪ-ದ-ಲ್ಲಿಯೇ
ಸಮೀ-ಪ-ದ-ಲ್ಲಿರು
ಸಮೀ-ಪ-ದ-ಲ್ಲಿ-ರು-ವನು
ಸಮೀ-ಪ-ದ-ಲ್ಲಿ-ರು-ವರೋ
ಸಮೀ-ಪ-ದ-ಲ್ಲಿ-ರು-ವುದೇ
ಸಮೀ-ಪ-ದ-ಲ್ಲಿವೆ
ಸಮೀ-ಪ-ದಲ್ಲೇ
ಸಮೀ-ಪ-ವಾ-ಗಿ-ರು-ವ-ವನು
ಸಮೀ-ಪ-ವಾ-ಗಿ-ರು-ವುದು
ಸಮೀ-ಪ-ವಾದ
ಸಮೀ-ಪಿ-ಸು-ವು-ದರ
ಸಮೀ-ಪಿ-ಸು-ವುವು
ಸಮು-ದಾಯ
ಸಮು-ದಾ-ಯದ
ಸಮು-ದಾ-ಯ-ವನ್ನು
ಸಮು-ದಾ-ಯ-ವನ್ನೂ
ಸಮು-ದಾ-ಯ-ವ-ನ್ನೆಲ್ಲ
ಸಮು-ದಾ-ಯ-ವೆಲ್ಲ
ಸಮು-ದ್ಧರ್ತಾ
ಸಮುದ್ರ
ಸಮು-ದ್ರಕ್ಕೆ
ಸಮು-ದ್ರ-ತೀ-ರ-ದ-ಲ್ಲಿ-ರುವ
ಸಮು-ದ್ರದ
ಸಮು-ದ್ರ-ದ-ಮೇಲೆ
ಸಮು-ದ್ರ-ದಲ್ಲಿ
ಸಮು-ದ್ರ-ಮಾಪಃ
ಸಮು-ದ್ರ-ಮೇ-ವಾ-ಭಿ-ಮುಖಾ
ಸಮು-ದ್ರ-ವನ್ನು
ಸಮು-ದ್ರ-ವನ್ನೇ
ಸಮು-ದ್ರ-ವಾ-ಗು-ವುದು
ಸಮು-ದ್ರ-ವಿತ್ತು
ಸಮು-ದ್ರ-ವಿ-ದ್ದರೆ
ಸಮು-ದ್ರವೇ
ಸಮು-ಪ-ಸ್ಥಿ-ತಮ್
ಸಮು-ಪಾ-ಶ್ರಿತಃ
ಸಮೂಹ
ಸಮೂ-ಹ-ದ-ಲ್ಲಿದ್ದ
ಸಮೂ-ಹ-ವನ್ನು
ಸಮೂ-ಹ-ವನ್ನೂ
ಸಮೂ-ಹವು
ಸಮೂ-ಹವೇ
ಸಮೃ-ದ್ಧಮ್
ಸಮೃ-ದ್ಧ-ವಾದ
ಸಮೃ-ದ್ಧ-ವೇ-ಗಾಃ
ಸಮೃ-ದ್ಧಿ-ಯಾಗಿ
ಸಮೃ-ದ್ಧಿ-ಯಾದ
ಸಮೆಯು
ಸಮೆ-ಯುತ್ತ
ಸಮೆ-ಸ-ಬೇ-ಕಾ-ಗಿದೆ
ಸಮೆ-ಸ-ಬೇಕು
ಸಮೆ-ಸಲು
ಸಮೆ-ಸು-ತ್ತಾನೆ
ಸಮೆ-ಸು-ವು-ದ-ಕ್ಕಾಗಿ
ಸಮೇ
ಸಮೇತ
ಸಮೋ
ಸಮೋಽಹಂ
ಸಮೌ
ಸಮ್ಮ-ತ-ವಾದ
ಸಮ್ಮ-ತ-ವಿದೆ
ಸಮ್ಮ-ತ-ವಿಲ್ಲ
ಸಮ್ಮೋಹ
ಸಮ್ಯಕ್
ಸಮ್ಯ-ಗು-ಭ-ಯೋ-ರ್ವಿಂ-ದತೇ
ಸಮ್ಯ-ಗ್ವ್ಯ-ವ-ಸಿತೋ
ಸರ
ಸರ-ಕನ್ನು
ಸರ-ಕ-ನ್ನೆಲ್ಲಾ
ಸರ-ಕಲ್ಲ
ಸರ-ಕಾರ
ಸರ-ಕಿ-ಗಲ್ಲ
ಸರ-ಕಿ-ನ-ಲ್ಲಿಯೇ
ಸರ-ಕು-ಗ-ಳಿವೆ
ಸರ-ಕು-ಗಳೇ
ಸರದಿ
ಸರ-ದಿ-ಯ-ಮೇಲೆ
ಸರ-ಪಣಿ
ಸರ-ಪ-ಣಿಯೂ
ಸರ-ಪಳಿ
ಸರ-ಪ-ಳಿಗೆ
ಸರ-ಪ-ಳಿ-ಯನ್ನು
ಸರ-ಪ-ಳಿಯೇ
ಸರ-ಬ-ರಾ-ಜಾ-ಗು-ವುದು
ಸರ-ಬ-ರಾಜು
ಸರಳ
ಸರ-ಳತೆ
ಸರ-ಳ-ತೆ-ಗಳೇ
ಸರ-ಳ-ತೆ-ಯನ್ನು
ಸರ-ಳ-ರೇ-ಖೆ-ಯಲ್ಲಿ
ಸರ-ಳ-ರೇ-ಖೆ-ಯ-ಲ್ಲಿ-ರ-ಬೇಕು
ಸರ-ಳ-ವಾಗಿ
ಸರ-ಳ-ವಾ-ಗಿಯೇ
ಸರ-ಳ-ವಾ-ಗಿ-ರು-ತ್ತದೆ
ಸರ-ಳ-ವಾ-ಗಿ-ರು-ವನೊ
ಸರ-ಳ-ವಾ-ಗಿ-ರು-ವುದು
ಸರ-ಸ-ವಾ-ಗಲಿ
ಸರ-ಸ-ವಾ-ಡು-ವುದು
ಸರ-ಸಾ-ಮಸ್ಮಿ
ಸರ-ಸ್ವತಿ
ಸರ-ಸ್ವ-ತಿಯೆ
ಸರ-ಸ್ಸು-ಗಳಲ್ಲಿ
ಸರಾಗ
ಸರಾ-ಗ-ವಾಗಿ
ಸರಾ-ಗ-ವಾ-ಗು-ವು-ದಕ್ಕೆ
ಸರಾಯಿ
ಸರಿ
ಸರಿ-ದರೆ
ಸರಿದು
ಸರಿ-ದು-ಹೋ-ಗಿದೆ
ಸರಿ-ದೊ-ಡ-ನೆಯೆ
ಸರಿ-ದೋ-ರುವ
ಸರಿ-ಮಾ-ಡಿ-ರು-ವನು
ಸರಿ-ಯನ್ನು
ಸರಿ-ಯ-ಬಾ-ರದು
ಸರಿ-ಯ-ಬೇ-ಕೆಂದು
ಸರಿ-ಯಲ್ಲ
ಸರಿ-ಯಾಗಿ
ಸರಿ-ಯಾ-ಗಿ-ದ್ದರೆ
ಸರಿ-ಯಾ-ಗಿ-ರ-ಬೇ-ಕಾ-ದರೆ
ಸರಿ-ಯಾ-ಗಿ-ರು-ವು-ದಿಲ್ಲ
ಸರಿ-ಯಾ-ಗಿ-ರು-ವುದು
ಸರಿ-ಯಾ-ಗಿ-ರು-ವುದೇ
ಸರಿ-ಯಾ-ಗಿಲ್ಲ
ಸರಿ-ಯಾದ
ಸರಿ-ಯು-ತ್ತಿ-ರ-ಬೇಕು
ಸರಿ-ಯು-ವನು
ಸರಿ-ಯು-ವು-ದಿಲ್ಲ
ಸರಿ-ಯು-ವುದು
ಸರಿ-ಯು-ವುದೋ
ಸರಿ-ಯು-ವುವು
ಸರಿಯೆ
ಸರಿ-ಯೆಂದು
ಸರಿಯೇ
ಸರಿಯೋ
ಸರಿ-ಸ-ಬೇಕು
ಸರಿ-ಸಮ
ಸರಿ-ಸ-ಮಾ-ನ-ನಲ್ಲ
ಸರಿ-ಸ-ಮಾ-ನ-ರಾದ
ಸರಿ-ಸ-ಮಾ-ನ-ರಾ-ದ-ವ-ರಿಂದ
ಸರಿ-ಸ-ಮಾ-ನರು
ಸರಿ-ಸ-ಮಾ-ನರೂ
ಸರಿಸಿ
ಸರಿ-ಸು-ತ್ತೇ-ವೆಯೋ
ಸರಿ-ಸು-ವೆವೋ
ಸರಿ-ಹೊಂ-ದಿ-ಕೊ-ಳ್ಳು-ವುದು
ಸರೋ-ವರ
ಸರೋ-ವ-ರಕ್ಕೆ
ಸರೋ-ವ-ರದ
ಸರೋ-ವ-ರ-ದಂತೆ
ಸರೋ-ವ-ರ-ದಲ್ಲಿ
ಸರೋ-ವ-ರವೂ
ಸರ್ಕ-ಸ್ಸಿ-ನಲ್ಲಿ
ಸರ್ಕ-ಸ್ಸಿ-ನ-ಲ್ಲಿ-ರುವ
ಸರ್ಕ-ಸ್ಸಿ-ನಿಂದ
ಸರ್ಕಾರ
ಸರ್ಕಾ-ರದ
ಸರ್ಕಾ-ರ-ದಲ್ಲಿ
ಸರ್ಕಾ-ರ-ದಿಂದ
ಸರ್ಗಾ-ಣಾ-ಮಾ-ದಿ-ರಂ-ತಶ್ಚ
ಸರ್ಗೇ
ಸರ್ಗೇಽಪಿ
ಸರ್ಗೋ
ಸರ್ಜನ್
ಸರ್ಪ-ಗಳನ್ನೂ
ಸರ್ಪ-ಗಳಲ್ಲಿ
ಸರ್ಪ-ಗಳು
ಸರ್ಪ-ದಲ್ಲಿ
ಸರ್ಪಾ-ಣಾ-ಮಸ್ಮಿ
ಸರ್ವ
ಸರ್ವಂ
ಸರ್ವಃ
ಸರ್ವ-ಕರ್ಮ
ಸರ್ವ-ಕ-ರ್ಮ-ಗಳನ್ನು
ಸರ್ವ-ಕ-ರ್ಮ-ಣಾಮ್
ಸರ್ವ-ಕ-ರ್ಮ-ಫ-ಲ-ತ್ಯಾಗಂ
ಸರ್ವ-ಕ-ರ್ಮಾಣಿ
ಸರ್ವ-ಕ-ರ್ಮಾ-ಣ್ಯಪಿ
ಸರ್ವ-ಕಾ-ಮ-ಗಳಲ್ಲಿ
ಸರ್ವ-ಕಾ-ಮೇಭ್ಯೋ
ಸರ್ವ-ಕಿ-ಲ್ಬಿ-ಷೈಃ
ಸರ್ವಕ್ಕೂ
ಸರ್ವ-ಕ್ಷೇ-ತ್ರೇಷು
ಸರ್ವ-ಗತ
ಸರ್ವ-ಗತಂ
ಸರ್ವ-ಗತಃ
ಸರ್ವ-ಗು-ಹ್ಯ-ತಮಂ
ಸರ್ವ-ಜೀ-ವರ
ಸರ್ವಜ್ಞ
ಸರ್ವ-ಜ್ಞತೆ
ಸರ್ವ-ಜ್ಞತ್ವ
ಸರ್ವ-ಜ್ಞ-ನಾ-ಗು-ವನು
ಸರ್ವ-ಜ್ಞ-ನಾದ
ಸರ್ವ-ಜ್ಞ-ನಾ-ದರೆ
ಸರ್ವ-ಜ್ಞನು
ಸರ್ವ-ಜ್ಞ-ರಾ-ಗು-ತ್ತೇವೆ
ಸರ್ವ-ಜ್ಞಾನ
ಸರ್ವ-ಜ್ಞಾ-ನ-ವಿ-ಮೂ-ಢಾಂ-ಸ್ತಾನ್
ಸರ್ವತ
ಸರ್ವ-ತಃ-ಪಾ-ಣಿ-ಪಾದಂ
ಸರ್ವ-ತಃ-ಸಂ-ಪ್ಲು-ತೋ-ದಕೇ
ಸರ್ವ-ತ-ಶ್ಶ್ರು-ತಿ-ಮ-ಲ್ಲೋಕೇ
ಸರ್ವತೋ
ಸರ್ವ-ತೋ-ಮುಖ
ಸರ್ವ-ತೋ-ಮು-ಖ-ನಾದ
ಸರ್ವ-ತೋ-ಮು-ಖ-ವಾಗಿ
ಸರ್ವ-ತೋ-ಽನಂ-ತ-ರೂ-ಪಮ್
ಸರ್ವತ್ರ
ಸರ್ವ-ತ್ರ-ಗ-ಮ-ಚಿಂತ್ಯಂ
ಸರ್ವ-ತ್ರಗೋ
ಸರ್ವ-ತ್ರಾ-ನ-ಭಿ-ಸ್ನೇ-ಹ-ಸ್ತ-ತ್ತತ್
ಸರ್ವ-ತ್ರಾ-ವ-ಸ್ಥಿತೋ
ಸರ್ವಥಾ
ಸರ್ವದಾ
ಸರ್ವ-ದುಃ-ಖ-ಗಳ
ಸರ್ವ-ದುಃ-ಖಾ-ನಾಂ
ಸರ್ವ-ದು-ರ್ಗಾಣಿ
ಸರ್ವ-ದೇ-ಹಿ-ನಾಮ್
ಸರ್ವ-ದ್ವಾ-ರ-ಗಳನ್ನು
ಸರ್ವ-ದ್ವಾ-ರಾಣಿ
ಸರ್ವ-ದ್ವಾ-ರೇಷು
ಸರ್ವ-ಧ-ರ್ಮಾನ್
ಸರ್ವ-ನಾಶ
ಸರ್ವ-ನಾ-ಶ-ವಾಗಿ
ಸರ್ವ-ಪ-ರಿ-ಗ್ರ-ಹ-ವನ್ನೂ
ಸರ್ವ-ಪಾ-ಪೇಭ್ಯೋ
ಸರ್ವ-ಪಾ-ಪೈಃ
ಸರ್ವ-ಪ್ರ-ಯ-ತ್ನ-ದಿಂದ
ಸರ್ವ-ಪ್ರಾ-ಣಿ-ಗಳ
ಸರ್ವ-ಪ್ರಾ-ಣಿ-ಗಳನ್ನು
ಸರ್ವ-ಪ್ರಾ-ಣಿ-ಗಳಲ್ಲಿ
ಸರ್ವ-ಪ್ರಾ-ಣಿ-ಗ-ಳ-ಲ್ಲಿಯೂ
ಸರ್ವ-ಭ-ಕ್ಷ-ಕ-ನಾದ
ಸರ್ವ-ಭಾ-ವೇನ
ಸರ್ವ-ಭೂ-ತ-ಗಳ
ಸರ್ವ-ಭೂ-ತ-ಗಳನ್ನು
ಸರ್ವ-ಭೂ-ತ-ಗ-ಳ-ಲ್ಲಿಯೂ
ಸರ್ವ-ಭೂ-ತ-ಗ-ಳಿಗೆ
ಸರ್ವ-ಭೂ-ತ-ಸ್ಥ-ಮಾ-ತ್ಮಾನಂ
ಸರ್ವ-ಭೂ-ತ-ಸ್ಥಿತಂ
ಸರ್ವ-ಭೂ-ತ-ಹಿ-ತ-ದಲ್ಲಿ
ಸರ್ವ-ಭೂ-ತ-ಹಿತೇ
ಸರ್ವ-ಭೂ-ತಾ-ತ್ಮ-ಭೂ-ತಾತ್ಮಾ
ಸರ್ವ-ಭೂ-ತಾ-ನಾಂ
ಸರ್ವ-ಭೂ-ತಾನಿ
ಸರ್ವ-ಭೂ-ತಾ-ಶ-ಯ-ಸ್ಥಿತಃ
ಸರ್ವ-ಭೂ-ತೇಷು
ಸರ್ವ-ಭೃ-ಚ್ಚೈವ
ಸರ್ವ-ಮಾ-ವೃತ್ಯ
ಸರ್ವ-ಮಿತಿ
ಸರ್ವ-ಮಿದಂ
ಸರ್ವ-ಮೇ-ತ-ದೃತಂ
ಸರ್ವ-ಯ-ಜ್ಞಾ-ನಾಂ
ಸರ್ವ-ಯೋ-ನಿ-ಗ-ಳಲ್ಲೂ
ಸರ್ವ-ಯೋ-ನಿಷು
ಸರ್ವರ
ಸರ್ವ-ರ-ಲ್ಲಿಯೂ
ಸರ್ವ-ಲೋಕ
ಸರ್ವ-ಲೋ-ಕ-ಗಳ
ಸರ್ವ-ಲೋ-ಕ-ಮ-ಹೇ-ಶ್ವ-ರಮ್
ಸರ್ವ-ವನ್ನು
ಸರ್ವ-ವನ್ನೂ
ಸರ್ವ-ವ-ಸ್ತು-ಗಳ
ಸರ್ವ-ವಿ-ದ್ಭ-ಜತಿ
ಸರ್ವವೂ
ಸರ್ವ-ವೃ-ಕ್ಷಾ-ಣಾಂ
ಸರ್ವ-ವೇ-ದ-ಗಳ
ಸರ್ವ-ವೇ-ದೇಷು
ಸರ್ವ-ವ್ಯಾಪಿ
ಸರ್ವ-ವ್ಯಾ-ಪಿತ್ವ
ಸರ್ವ-ವ್ಯಾ-ಪಿ-ಯಾಗಿ
ಸರ್ವ-ವ್ಯಾ-ಪಿ-ಯಾ-ಗಿ-ರುವ
ಸರ್ವ-ವ್ಯಾ-ಪಿ-ಯಾ-ಗಿ-ರು-ವನು
ಸರ್ವ-ವ್ಯಾ-ಪಿ-ಯಾದ
ಸರ್ವ-ವ್ಯಾ-ಪಿ-ಯಾ-ದ-ವನು
ಸರ್ವ-ವ್ಯಾ-ಪಿಯೂ
ಸರ್ವ-ವ್ಯಾ-ಪ್ತಿ-ತ್ವಕ್ಕೆ
ಸರ್ವ-ವ್ಯಾ-ಪ್ತಿ-ತ್ವದ
ಸರ್ವಶಃ
ಸರ್ವ-ಶಕ್ತ
ಸರ್ವ-ಶ-ಕ್ತ-ನಾ-ಗಿ-ರುವ
ಸರ್ವ-ಶ್ರೇಷ್ಠ
ಸರ್ವ-ಶ್ರೇ-ಷ್ಠ-ನಾ-ದ-ವನೆ
ಸರ್ವ-ಶ್ರೇ-ಷ್ಠ-ವಾದ
ಸರ್ವ-ಶ್ರೇ-ಷ್ಠ-ವಾ-ದು-ದನ್ನು
ಸರ್ವ-ಶ್ರೇ-ಷ್ಠ-ವಾ-ದು-ದೆಂದು
ಸರ್ವ-ಶ್ರೇ-ಷ್ಠವೇ
ಸರ್ವ-ಸಂ-ಕ-ಲ್ಪ-ಗಳನ್ನೂ
ಸರ್ವ-ಸಂ-ಕ-ಲ್ಪ-ಸಂ-ನ್ಯಾಸೀ
ಸರ್ವ-ಸಂ-ಶ-ಯ-ಗಳನ್ನು
ಸರ್ವ-ಸಾ-ಮಾನ್ಯ
ಸರ್ವ-ಸಾ-ಮಾ-ನ್ಯ-ನಂತೆ
ಸರ್ವ-ಸಾ-ಮಾ-ನ್ಯ-ವಾದ
ಸರ್ವಸ್ಯ
ಸರ್ವಸ್ವ
ಸರ್ವ-ಸ್ವ-ರೂ-ಪ-ನಾ-ಗಿ-ರುವೆ
ಸರ್ವ-ಸ್ವ-ವನ್ನು
ಸರ್ವ-ಸ್ವ-ವನ್ನೂ
ಸರ್ವ-ಸ್ವ-ವನ್ನೇ
ಸರ್ವ-ಸ್ವವೂ
ಸರ್ವ-ಹ-ರ-ನಾದ
ಸರ್ವ-ಹ-ರ-ಶ್ಚಾ-ಹ-ಮು-ದ್ಭ-ವಶ್ಚ
ಸರ್ವಾಂಗ
ಸರ್ವಾಂ-ತ-ರ್ಯಾಮಿ
ಸರ್ವಾಂ-ತ-ರ್ಯಾ-ಮಿ-ಯಂತೆ
ಸರ್ವಾಂ-ತ-ರ್ಯಾ-ಮಿ-ಯಾಗಿ
ಸರ್ವಾಂ-ತ-ರ್ಯಾ-ಮಿ-ಯಾ-ಗಿದೆ
ಸರ್ವಾಂ-ತ-ರ್ಯಾ-ಮಿ-ಯಾದ
ಸರ್ವಾಂ-ಸ್ತಥಾ
ಸರ್ವಾಃ
ಸರ್ವಾಣಿ
ಸರ್ವಾ-ಣೀಂ-ದ್ರಿ-ಯ-ಕ-ರ್ಮಾಣಿ
ಸರ್ವಾ-ಣೀ-ತ್ಯು-ಪ-ಧಾ-ರಯ
ಸರ್ವಾ-ಧಿ-ಕಾ-ರಿ-ಗ-ಳಾಗಿ
ಸರ್ವಾ-ನು-ರ-ಗಾಂಶ್ಚ
ಸರ್ವಾ-ನೇವಂ
ಸರ್ವಾನ್
ಸರ್ವಾ-ರಂ-ಭ-ಪ-ರಿ-ತ್ಯಾಗೀ
ಸರ್ವಾ-ರಂಭಾ
ಸರ್ವಾ-ರ್ಥಾನ್
ಸರ್ವಾ-ಶ್ಚ-ರ್ಯ-ಮಯಂ
ಸರ್ವೆ-ಸಾ-ಮಾನ್ಯ
ಸರ್ವೇ
ಸರ್ವೇಂ-ದ್ರಿಯ
ಸರ್ವೇಂ-ದ್ರಿ-ಯ-ಗು-ಣಾ-ಭಾಸಂ
ಸರ್ವೇಂ-ದ್ರಿ-ಯ-ವಿ-ವ-ರ್ಜಿ-ತಮ್
ಸರ್ವೇಭ್ಯಃ
ಸರ್ವೇ-ಶ್ವರ
ಸರ್ವೇ-ಶ್ವ-ರ-ನಾದ
ಸರ್ವೇ-ಷಾಂ
ಸರ್ವೇಷು
ಸರ್ವೇ-ಽಪ್ಯೇತೇ
ಸರ್ವೈ-ರ-ಹ-ಮೇವ
ಸರ್ವೋ-ತ್ಕೃಷ್ಟ
ಸರ್ವೋ-ತ್ಕೃ-ಷ್ಟ-ವಾ-ದು-ದನ್ನು
ಸರ್ವೋ-ತ್ತ-ಮ-ವಾ-ದುದು
ಸರ್ವೋ-ತ್ತ-ಮವೊ
ಸರ್ವೋ-ಪ-ನಿ-ಷದೋ
ಸಲ
ಸಲ-ಕ-ರ-ಣೆ-ಗಳು
ಸಲಕ್ಕೆ
ಸಲ-ವಾ-ದರೂ
ಸಲವೂ
ಸಲ-ಹಿದ
ಸಲ-ಹಿದ್ದು
ಸಲ-ಹು-ವ-ವನು
ಸಲಹೆ
ಸಲ-ಹೆ-ಗಳನ್ನು
ಸಲ-ಹೆ-ಗಳು
ಸಲ-ಹೆ-ಯನ್ನು
ಸಲು
ಸಲ್ಲದ
ಸಲ್ಲ-ಬೇ-ಕಾ-ಗಿ-ರು-ವುದು
ಸಲ್ಲ-ಬೇಕೋ
ಸಲ್ಲಿ-ಸು-ತ್ತಾನೆ
ಸಲ್ಲಿ-ಸು-ತ್ತಿ-ದ್ದರು
ಸಲ್ಲಿ-ಸುವ
ಸಲ್ಲು-ತ್ತದೆ
ಸಲ್ಲು-ವುದು
ಸಲ್ಲು-ವುದೇ
ಸವ-ಕ-ಲಾ-ಗು-ವುದು
ಸವ-ರಿ-ಕೊಂ-ಡರೆ
ಸವ-ರಿ-ಕೊಂ-ಡಿ-ರ-ಬೇಕು
ಸವ-ರಿ-ಕೊಂಡು
ಸವ-ರಿ-ಕೊಳ್ಳಿ
ಸವ-ರಿ-ದ್ದರೆ
ಸವ-ರಿ-ರು-ವನು
ಸವ-ರಿ-ಲ್ಲವೋ
ಸವ-ರು-ವನು
ಸವಾ-ರನ
ಸವಾರಿ
ಸವಾ-ಲನ್ನು
ಸವಿ
ಸವಿ-ಕ-ನ-ಸನ್ನು
ಸವಿ-ಕಾ-ರ-ಮು-ದಾ-ಹೃ-ತಮ್
ಸವಿ-ಗ-ನ-ಸು-ಗಳು
ಸವಿ-ಜ್ಞಾ-ನ-ಮಿದಂ
ಸವಿ-ದ-ವನು
ಸವಿ-ಯನ್ನು
ಸವಿ-ಯು-ವನು
ಸವೆ-ದಿ-ದ್ದರೆ
ಸವೆ-ದು-ಹೋ-ಗು-ವುದು
ಸವೆ-ಯದೆ
ಸವೆ-ಸ-ಬೇ-ಕಾ-ಗಿದೆ
ಸವೆ-ಸ-ಬೇ-ಕಾ-ದರೂ
ಸವೆ-ಸಲು
ಸವೆ-ಸು-ವು-ದಕ್ಕೆ
ಸವ್ಯ-ಸಾಚಿ
ಸವ್ಯ-ಸಾ-ಚಿನ್
ಸಶರಂ
ಸಸಿ
ಸಸಿ-ಯನ್ನು
ಸಸಿ-ಯಾ-ದಾಗ
ಸಸಿಯೂ
ಸಸ್ಯ
ಸಸ್ಯ-ಗಳ
ಸಸ್ಯ-ಜೀ-ವ-ನದ
ಸಸ್ಯ-ರಾಶಿ
ಸಸ್ಯ-ರೂ-ಪ-ದ-ಲ್ಲಿ-ರಲಿ
ಸಸ್ಯಾ-ದಿ-ಗ-ಳ-ಲ್ಲಿಯೂ
ಸಸ್ಯಾ-ದಿ-ಗಳು
ಸಸ್ಯಾ-ಹಾರ
ಸಹ
ಸಹ-ಕ-ರಿ-ಸದೆ
ಸಹ-ಕ-ರಿ-ಸ-ಬೇಕು
ಸಹ-ಕ-ರಿ-ಸಿ-ದರೆ
ಸಹ-ಕ-ರಿ-ಸುವು
ಸಹ-ಕಾರ
ಸಹ-ಕಾ-ರ-ದಿಂದ
ಸಹ-ಕಾ-ರ-ವಿ-ಲ್ಲದೆ
ಸಹ-ಕಾರಿ
ಸಹಜ
ಸಹಜಂ
ಸಹ-ಜ-ವಾಗಿ
ಸಹ-ಜ-ವಾ-ಗಿದೆ
ಸಹ-ಜ-ವಾ-ಗಿ-ರು-ವುದು
ಸಹ-ಜ-ವಾದ
ಸಹ-ಜವೇ
ಸಹ-ದೇ-ವರು
ಸಹ-ದೇ-ವಶ್ಚ
ಸಹ-ನೆ-ಯಿಂದ
ಸಹ-ಯ-ಜ್ಞಾಃ
ಸಹ-ವಾಸ
ಸಹ-ವಾ-ಸ-ದ-ಲ್ಲಿ-ರು-ವನು
ಸಹ-ವಾ-ಸ-ವನ್ನು
ಸಹ-ವಾ-ಸ-ವಾ-ದರೂ
ಸಹ-ಸೈ-ವಾ-ಭ್ಯ-ಹ-ನ್ಯಂತ
ಸಹಸ್ರ
ಸಹ-ಸ್ರ-ಕೃತ್ವಃ
ಸಹ-ಸ್ರ-ಕ್ಕಿಂತ
ಸಹ-ಸ್ರ-ನಾಮ
ಸಹ-ಸ್ರ-ನಾ-ಮಾ-ರ್ಚ-ನೆಯೋ
ಸಹ-ಸ್ರ-ಪಾಲು
ಸಹ-ಸ್ರ-ಬಾ-ಹು-ಗ-ಳುಳ್ಳು
ಸಹ-ಸ್ರ-ಬಾಹೋ
ಸಹ-ಸ್ರ-ಯು-ಗ-ಪ-ರ್ಯಂ-ತ-ಮ-ಹ-ರ್ಯ-ದ್ಬ್ರ-ಹ್ಮಣೋ
ಸಹ-ಸ್ರಶಃ
ಸಹ-ಸ್ರಾರು
ಸಹ-ಸ್ರೇಷು
ಸಹಾ-ನು-ಭೂತಿ
ಸಹಾ-ನು-ಭೂ-ತಿ-ಯನ್ನು
ಸಹಾಯ
ಸಹಾ-ಯ-ಕ-ನಾ-ಗು-ವನು
ಸಹಾ-ಯ-ಕ-ನಾದ
ಸಹಾ-ಯ-ಕ-ವಾ-ಗಲಿ
ಸಹಾ-ಯ-ಕ-ವಾಗಿ
ಸಹಾ-ಯ-ಕ-ವಾ-ಗಿ-ರುವ
ಸಹಾ-ಯ-ಕ-ವಾ-ಗಿ-ರು-ವುದು
ಸಹಾ-ಯಕ್ಕೆ
ಸಹಾ-ಯ-ದಿಂದ
ಸಹಾ-ಯ-ದಿಂ-ದಲೇ
ಸಹಾ-ಯ-ಮಾ-ಡ-ಲಾ-ರವು
ಸಹಾ-ಯ-ಮಾ-ಡು-ವುದು
ಸಹಾ-ಯ-ವನ್ನು
ಸಹಾ-ಯ-ವಾ-ಗ-ಬೇಕು
ಸಹಾ-ಯ-ವಾ-ಗಲಿ
ಸಹಾ-ಯ-ವಾಗು
ಸಹಾ-ಯ-ವಾ-ಗು-ತ್ತದೆ
ಸಹಾ-ಯ-ವಾ-ಗು-ವುದು
ಸಹಾ-ಯ-ವಿ-ಲ್ಲದೆ
ಸಹಾ-ಯ-ವಿ-ಲ್ಲದೇ
ಸಹಾ-ಯವೂ
ಸಹಾ-ಸ್ಮ-ದೀ-ಯೈ-ರಪಿ
ಸಹಿತ
ಸಹಿ-ತ-ವಾದ
ಸಹಿ-ಷ್ಣತೆ
ಸಹಿ-ಷ್ಣುತೆ
ಸಹಿ-ಸ-ಬಲ್ಲ
ಸಹಿ-ಸ-ಬ-ಲ್ಲದು
ಸಹಿ-ಸ-ಬೇ-ಕಾ-ಗು-ವುದು
ಸಹಿ-ಸ-ಬೇ-ಕಾ-ದರೆ
ಸಹಿ-ಸ-ಬೇಕು
ಸಹಿ-ಸ-ಲಾ-ಗ-ಲಿಲ್ಲ
ಸಹಿ-ಸ-ಲಾ-ರದ
ಸಹಿ-ಸ-ಲಾ-ರದೆ
ಸಹಿ-ಸ-ಲಾ-ರನು
ಸಹಿ-ಸ-ಲಾ-ರನೋ
ಸಹಿ-ಸಲು
ಸಹಿ-ಸ-ಲೇ-ಬೇ-ಕಾ-ಗಿದೆ
ಸಹಿ-ಸಿಕೊ
ಸಹಿ-ಸಿ-ಕೊಂಡು
ಸಹಿ-ಸಿ-ಕೊ-ಳ್ಳ-ಬೇಕು
ಸಹಿ-ಸಿ-ಕೊ-ಳ್ಳಲು
ಸಹಿ-ಸಿ-ಕೊ-ಳ್ಳು-ವಂತೆ
ಸಹಿ-ಸು-ತ್ತೇವೆ
ಸಹಿ-ಸುವ
ಸಹಿ-ಸು-ವನು
ಸಹಿ-ಸು-ವನೆ
ಸಹಿ-ಸು-ವ-ವ-ನಲ್ಲ
ಸಹಿ-ಸು-ವ-ವ-ನಿಗೆ
ಸಹಿ-ಸು-ವು-ದಕ್ಕೆ
ಸಹಿ-ಸು-ವುದನ್ನು
ಸಹಿ-ಸು-ವು-ದಿಲ್ಲ
ಸಹಿ-ಸು-ವುದು
ಸಹೈ-ವಾ-ವ-ನಿ-ಪಾ-ಲ-ಸಂ-ಘೈಃ
ಸಹೋ-ದರ
ಸಹೋ-ದ-ರ-ರಾ-ಗು-ವರು
ಸಹೋ-ದ-ರರು
ಸಹೋ-ದ-ರಿ-ಯರು
ಸಾ
ಸಾಂಕ್ರಾ-ಮಿಕ
ಸಾಂಖ್ಯ
ಸಾಂಖ್ಯಂ
ಸಾಂಖ್ಯ-ದೃ-ಷ್ಟಿಗೆ
ಸಾಂಖ್ಯ-ಯೋಗ
ಸಾಂಖ್ಯ-ಯೋಗೌ
ಸಾಂಖ್ಯ-ರಿಗೆ
ಸಾಂಖ್ಯಾ-ನಾಂ
ಸಾಂಖ್ಯೇ
ಸಾಂಖ್ಯೇನ
ಸಾಂಖ್ಯೈಃ
ಸಾಂಗ
ಸಾಂಗ-ವಾಗಿ
ಸಾಂತ
ಸಾಂತ-ತ್ವ-ವನ್ನು
ಸಾಂತದ
ಸಾಂತ-ದಲ್ಲಿ
ಸಾಂತ-ಮ-ತಿಗೆ
ಸಾಂತ-ವನ್ನು
ಸಾಂತ-ವ-ಸ್ತು-ವನ್ನು
ಸಾಂತ-ವ-ಸ್ತು-ವಿಗೆ
ಸಾಂತ-ವಾಗಿ
ಸಾಂತ-ವಾ-ಗಿ-ರ-ಬೇಕು
ಸಾಂತ-ವಾ-ಗು-ವು-ದಿ-ಲ್ಲವೇ
ಸಾಂತ-ವಾ-ದುದು
ಸಾಂತ-ವಾ-ದು-ದೆಂದು
ಸಾಂತ್ವನ
ಸಾಂನಿ-ಧ್ಯ-ವನ್ನು
ಸಾಂಸಾ-ರಿಕ
ಸಾಂಸಾ-ರಿ-ಕತೆ
ಸಾಂಸಾ-ರಿ-ಕ-ತೆ-ಯಿಂದ
ಸಾಕಪ್ಪ
ಸಾಕಲ್ಲ
ಸಾಕಷ್ಚು
ಸಾಕಷ್ಟು
ಸಾಕಾಗಿ
ಸಾಕಾ-ಗಿತ್ತು
ಸಾಕಾ-ಗಿದೆ
ಸಾಕಾ-ದ-ವನು
ಸಾಕಾ-ದಷ್ಟು
ಸಾಕಾ-ದಾಗ
ಸಾಕಾರ
ಸಾಕಾ-ರ-ದಲ್ಲಿ
ಸಾಕಾ-ರ-ವಾಗಿ
ಸಾಕಾ-ರ-ವೆಂದು
ಸಾಕಿ
ಸಾಕಿದ
ಸಾಕು
ಸಾಕು-ತಂದೆ
ಸಾಕು-ತ್ತಿದ್ದ
ಸಾಕು-ತ್ತೇವೆ
ಸಾಕೆಂ-ದರೆ
ಸಾಕೆ-ನಿ-ಸು-ವು-ದಿಲ್ಲ
ಸಾಕೊ
ಸಾಕ್ಷಾ
ಸಾಕ್ಷಾತ್
ಸಾಕ್ಷಾ-ತ್ಕ-ರಿ-ಸಲು
ಸಾಕ್ಷಾ-ತ್ಕಾರ
ಸಾಕ್ಷಾ-ತ್ಕಾ-ರದ
ಸಾಕ್ಷಾ-ತ್ಕಾ-ರ-ವನ್ನು
ಸಾಕ್ಷಾ-ತ್ಕಾ-ರ-ವಾ-ಗ-ಬೇಕು
ಸಾಕ್ಷಾ-ತ್ಕಾ-ರ-ವಾ-ಗು-ತ್ತದೆ
ಸಾಕ್ಷಾ-ತ್ಕಾ-ರ-ವಾ-ಗು-ವು-ದಿಲ್ಲ
ಸಾಕ್ಷಾ-ತ್ಕಾ-ರ-ವಾ-ಗು-ವುದು
ಸಾಕ್ಷಾ-ತ್ಕಾ-ರ-ವಾದ
ಸಾಕ್ಷಾ-ತ್ಕಾ-ರ-ವೆಲ್ಲ
ಸಾಕ್ಷಾ-ತ್ತಾಗಿ
ಸಾಕ್ಷಿ
ಸಾಕ್ಷಿಯ
ಸಾಕ್ಷಿ-ಯಂತೆ
ಸಾಕ್ಷಿ-ಯಾಗಿ
ಸಾಕ್ಷಿ-ಯಾ-ಗಿ-ರು-ವನು
ಸಾಕ್ಷಿ-ಯಾ-ಗಿ-ರು-ವ-ವನು
ಸಾಕ್ಷಿ-ಯಾ-ಗಿ-ರು-ವುದು
ಸಾಕ್ಷಿಯೇ
ಸಾಕ್ಷೀ
ಸಾಕ್ಷೀ-ಭಾವ
ಸಾಕ್ಷೀ-ಭಾ-ವ-ವನ್ನು
ಸಾಗರ
ಸಾಗರಃ
ಸಾಗ-ರಕ್ಕೂ
ಸಾಗ-ರಕ್ಕೆ
ಸಾಗ-ರದ
ಸಾಗ-ರ-ದಂ-ತಿ-ರುವ
ಸಾಗ-ರ-ದಂತೆ
ಸಾಗ-ರ-ದಲ್ಲಿ
ಸಾಗ-ರ-ದ-ಲ್ಲಿತ್ತು
ಸಾಗ-ರ-ದ-ಲ್ಲಿ-ರುವ
ಸಾಗ-ರ-ದ-ಲ್ಲಿ-ರು-ವುದೋ
ಸಾಗ-ರ-ದಷ್ಟು
ಸಾಗ-ರ-ದಿಂದ
ಸಾಗ-ರ-ಮುಖ
ಸಾಗ-ರ-ವನ್ನು
ಸಾಗ-ರ-ವನ್ನೆ
ಸಾಗ-ರ-ವಾ-ಗು-ವಂತೆ
ಸಾಗ-ರ-ವಾ-ದರೊ
ಸಾಗ-ರ-ವಿದೆ
ಸಾಗ-ರವೆ
ಸಾಗ-ರ-ವೆಲ್ಲಾ
ಸಾಗ-ರವೇ
ಸಾಗಿ
ಸಾಗಿ-ಸಿ-ಕೊಂಡೇ
ಸಾಗಿ-ಸಿ-ದರೆ
ಸಾಗಿ-ಸು-ತ್ತಿ-ರು-ವರು
ಸಾಗಿ-ಸು-ತ್ತಿ-ರು-ವೆನು
ಸಾಗಿ-ಸು-ವುದು
ಸಾಗಿ-ಹೋ-ಗು-ತ್ತಿ-ರು-ವರು
ಸಾಗಿ-ಹೋ-ಗು-ವನು
ಸಾಗು-ತ್ತಾನೆ
ಸಾಗು-ತ್ತಿದೆ
ಸಾಗು-ತ್ತಿ-ರು-ವರು
ಸಾಗು-ತ್ತಿ-ರು-ವುದು
ಸಾಗು-ತ್ತಿವೆ
ಸಾಗು-ವನು
ಸಾಗು-ವಾಗ
ಸಾಗು-ವು-ದಿಲ್ಲ
ಸಾಗು-ವುದು
ಸಾಗು-ವು-ದೆಂತು
ಸಾಗು-ವುದೇ
ಸಾತ್ತಿ
ಸಾತ್ತ್ವ
ಸಾತ್ತ್ವಿಕ
ಸಾತ್ತ್ವಿಕಂ
ಸಾತ್ತ್ವಿಕಃ
ಸಾತ್ತ್ವಿ-ಕ-ಗು-ಣದ
ಸಾತ್ತ್ವಿ-ಕ-ನಿಗೆ
ಸಾತ್ತ್ವಿ-ಕ-ಪ್ರಿ-ಯಾಃ
ಸಾತ್ತ್ವಿ-ಕ-ಬುದ್ಧಿ
ಸಾತ್ತ್ವಿ-ಕ-ಬು-ದ್ಧಿಗೆ
ಸಾತ್ತ್ವಿ-ಕ-ಬು-ದ್ಧಿ-ಯ-ವನು
ಸಾತ್ತ್ವಿ-ಕಮ್
ಸಾತ್ತ್ವಿ-ಕರು
ಸಾತ್ತ್ವಿ-ಕ-ವಾ-ಗಿಯೂ
ಸಾತ್ತ್ವಿ-ಕ-ವಾ-ದದ್ದು
ಸಾತ್ತ್ವಿ-ಕ-ವೆಂದು
ಸಾತ್ತ್ವಿ-ಕವೇ
ಸಾತ್ತ್ವಿ-ಕವೊ
ಸಾತ್ತ್ವಿ-ಕವೋ
ಸಾತ್ತ್ವಿಕಾ
ಸಾತ್ತ್ವಿಕೀ
ಸಾತ್ತ್ವಿಕೋ
ಸಾತ್ಯಕಿ
ಸಾತ್ಯ-ಶ್ಚಾ-ಪ-ರಾ-ಜಿತಃ
ಸಾತ್ವಿಕ
ಸಾತ್ವಿ-ಕದ
ಸಾತ್ವಿ-ಕ-ದೃ-ಷ್ಟಿ-ಯಿಂದ
ಸಾತ್ವಿ-ಕರ
ಸಾದಾ
ಸಾಧಕ
ಸಾಧ-ಕನ
ಸಾಧ-ಕ-ನನ್ನು
ಸಾಧ-ಕ-ನಾ-ಗಿ-ದ್ದರೆ
ಸಾಧ-ಕ-ನಿಗೆ
ಸಾಧ-ಕ-ಬಾ-ಧ-ಕ-ಗಳನ್ನು
ಸಾಧ-ಕ-ಬಾ-ಧ-ಕ-ಗ-ಳೇನು
ಸಾಧ-ಕ-ರಿಗೂ
ಸಾಧ-ಕರು
ಸಾಧ-ಕರೆ
ಸಾಧನ
ಸಾಧ-ನ-ಗಳನ್ನು
ಸಾಧ-ನ-ಗಳು
ಸಾಧ-ನ-ದಿಂದ
ಸಾಧ-ನ-ಬ-ಲ-ದಿಂದ
ಸಾಧ-ನ-ವಾ-ಗು-ವುದು
ಸಾಧ-ನವೇ
ಸಾಧನಾ
ಸಾಧ-ನಾ-ಫ-ಲ-ದಿಂದ
ಸಾಧ-ನಾ-ಬ-ಲ-ದಿಂ-ದಲೇ
ಸಾಧ-ನಾ-ವ-ಸ್ಥೆ-ಯ-ಲ್ಲಿ-ರು-ವನು
ಸಾಧನೆ
ಸಾಧ-ನೆ-ಗಳನ್ನು
ಸಾಧ-ನೆ-ಗಳಲ್ಲಿ
ಸಾಧ-ನೆ-ಗ-ಳಿ-ಗಿಂ-ತಲೂ
ಸಾಧ-ನೆ-ಗ-ಳೆಲ್ಲಾ
ಸಾಧ-ನೆಗೂ
ಸಾಧ-ನೆಗೆ
ಸಾಧ-ನೆ-ಮಾಡಿ
ಸಾಧ-ನೆ-ಮಾ-ಡಿ-ದೊ-ಡ-ನೆಯೇ
ಸಾಧ-ನೆ-ಮಾ-ಡು-ತ್ತಿದ್ದ
ಸಾಧ-ನೆ-ಮಾ-ಡು-ತ್ತಿ-ರು-ವ-ವನು
ಸಾಧ-ನೆಯ
ಸಾಧ-ನೆ-ಯ-ನ್ನಾಗಿ
ಸಾಧ-ನೆ-ಯ-ನ್ನಾ-ದರೂ
ಸಾಧ-ನೆ-ಯನ್ನು
ಸಾಧ-ನೆ-ಯಲ್ಲ
ಸಾಧ-ನೆ-ಯಲ್ಲಿ
ಸಾಧ-ನೆ-ಯ-ಲ್ಲಿಯೂ
ಸಾಧ-ನೆ-ಯಾ-ದರೂ
ಸಾಧ-ನೆ-ಯಿಂದ
ಸಾಧ-ನೆ-ಯಿಂ-ದಲೇ
ಸಾಧ-ನೆ-ಯೆಲ್ಲ
ಸಾಧ-ನೆ-ಯೆಲ್ಲಾ
ಸಾಧರ್ಮ್ಯ
ಸಾಧ-ರ್ಮ್ಯ-ಮಾ-ಗ-ತಾಃ
ಸಾಧ-ರ್ಮ್ಯ-ವನ್ನು
ಸಾಧಾ-ರಣ
ಸಾಧಾ-ರ-ಣ-ಮ-ನು-ಷ್ಯ-ನನ್ನು
ಸಾಧಾ-ರ-ಣರು
ಸಾಧಾ-ರ-ಣ-ವಾಗಿ
ಸಾಧಾ-ರ-ಣ-ವಾದ
ಸಾಧಾ-ರ-ಣ-ಶ್ರ-ದ್ಧೆ-ಯಲ್ಲ
ಸಾಧಾ-ರಾ-ಣ-ವಾಗಿ
ಸಾಧಿ-ಭೂ-ತಾ-ಧಿ-ದೈವಂ
ಸಾಧಿ-ಯಜ್ಞಂ
ಸಾಧಿ-ಸದೆ
ಸಾಧಿ-ಸ-ಬಲ್ಲ
ಸಾಧಿ-ಸ-ಬ-ಹುದು
ಸಾಧಿ-ಸ-ಬೇ-ಕಾ-ಗಿ-ರು-ವುದನ್ನು
ಸಾಧಿ-ಸ-ಬೇ-ಕಾ-ಗಿ-ರು-ವು-ದ-ರೊಂ-ದಿಗೆ
ಸಾಧಿ-ಸ-ಬೇ-ಕಾ-ಗಿ-ರು-ವುದೋ
ಸಾಧಿ-ಸ-ಬೇ-ಕಾ-ದರೆ
ಸಾಧಿ-ಸ-ಬೇಕು
ಸಾಧಿ-ಸಲು
ಸಾಧಿ-ಸಲೂ
ಸಾಧಿ-ಸ-ಲೆ-ತ್ನಿ-ಸು-ವರು
ಸಾಧಿ-ಸಿ-ಕೊ-ಳ್ಳ-ಬೇ-ಕಾ-ದರೂ
ಸಾಧಿ-ಸಿ-ಕೊ-ಳ್ಳು-ವನು
ಸಾಧಿ-ಸಿದ
ಸಾಧಿ-ಸಿ-ದರೆ
ಸಾಧಿ-ಸಿ-ದ-ವನು
ಸಾಧಿ-ಸಿ-ದು-ದನ್ನು
ಸಾಧಿ-ಸಿದೆ
ಸಾಧಿ-ಸಿ-ದೆವು
ಸಾಧಿ-ಸಿ-ದ್ದರೆ
ಸಾಧಿ-ಸಿದ್ದು
ಸಾಧಿ-ಸಿದ್ದೇ
ಸಾಧಿ-ಸಿ-ದ್ದೇನು
ಸಾಧಿ-ಸಿ-ರು-ವುದನ್ನು
ಸಾಧಿ-ಸು-ತ್ತಾನೆ
ಸಾಧಿ-ಸುವ
ಸಾಧಿ-ಸು-ವರೋ
ಸಾಧಿ-ಸು-ವು-ದಕ್ಕೆ
ಸಾಧಿ-ಸು-ವು-ದಿಲ್ಲ
ಸಾಧಿ-ಸು-ವುದು
ಸಾಧಿ-ಸು-ವೆನು
ಸಾಧು
ಸಾಧು-ಸಂ-ತರು
ಸಾಧು-ಗಳ
ಸಾಧು-ಗಳನ್ನು
ಸಾಧು-ಗ-ಳ-ಲ್ಲಿಯೂ
ಸಾಧು-ಭಾವೇ
ಸಾಧು-ರೇವ
ಸಾಧು-ವಾ-ಗಿ-ದ್ದರೆ
ಸಾಧು-ವಾದ
ಸಾಧು-ವಿನ
ಸಾಧು-ವಿ-ನಲ್ಲಿ
ಸಾಧು-ವೃ-ತ್ತಿ-ಯು-ಳ್ಳ-ವನು
ಸಾಧು-ವೆಂದೇ
ಸಾಧು-ವೇಷ
ಸಾಧು-ಷ್ವಪಿ
ಸಾಧು-ಸಂ-ತರು
ಸಾಧು-ಸ-ಜ್ಜ-ನರು
ಸಾಧೂ-ನಾಂ
ಸಾಧ್ಯ
ಸಾಧ್ಯರು
ಸಾಧ್ಯ-ವ-ನ್ನಾಗಿ
ಸಾಧ್ಯ-ವಾಗ
ಸಾಧ್ಯ-ವಾ-ಗದ
ಸಾಧ್ಯ-ವಾ-ಗ-ದುದು
ಸಾಧ್ಯ-ವಾ-ಗದೆ
ಸಾಧ್ಯ-ವಾ-ಗದೇ
ಸಾಧ್ಯ-ವಾ-ಗ-ಬ-ಹುದು
ಸಾಧ್ಯ-ವಾ-ಗಲಿ
ಸಾಧ್ಯ-ವಾ-ಗ-ಲಿಲ್ಲ
ಸಾಧ್ಯ-ವಾ-ಗ-ಲಿ-ಲ್ಲ-ವಲ್ಲ
ಸಾಧ್ಯ-ವಾ-ಗಲೇ
ಸಾಧ್ಯ-ವಾ-ಗುವ
ಸಾಧ್ಯ-ವಾ-ಗು-ವು-ದಿಲ್ಲ
ಸಾಧ್ಯ-ವಾ-ಗು-ವುದು
ಸಾಧ್ಯ-ವಾ-ಗು-ವುದೊ
ಸಾಧ್ಯ-ವಾ-ಗು-ವುದೋ
ಸಾಧ್ಯ-ವಾ-ದರೆ
ಸಾಧ್ಯ-ವಾ-ದ-ಷ್ಟಾ-ದರೂ
ಸಾಧ್ಯ-ವಾ-ದಷ್ಟು
ಸಾಧ್ಯ-ವಾ-ದು-ದನ್ನು
ಸಾಧ್ಯ-ವಾ-ದೆ-ಡೆ-ಯ-ಲ್ಲೆಲ್ಲಾ
ಸಾಧ್ಯ-ವಾ-ಯಿತು
ಸಾಧ್ಯ-ವಿ-ದೆಯೇ
ಸಾಧ್ಯ-ವಿ-ರ-ಲಿಲ್ಲ
ಸಾಧ್ಯ-ವಿ-ರು-ವಾಗ
ಸಾಧ್ಯ-ವಿಲ್ಲ
ಸಾಧ್ಯ-ವಿ-ಲ್ಲದೆ
ಸಾಧ್ಯ-ವಿ-ಲ್ಲದೇ
ಸಾಧ್ಯ-ವಿ-ಲ್ಲವೊ
ಸಾಧ್ಯ-ವಿ-ಲ್ಲವೋ
ಸಾಧ್ಯವೆ
ಸಾಧ್ಯವೇ
ಸಾಧ್ಯವೋ
ಸಾಧ್ಯಾ
ಸಾನ್ನಿಧ್ಯ
ಸಾನ್ನಿ-ಧ್ಯ-ದಲ್ಲಿ
ಸಾನ್ನಿ-ಧ್ಯ-ದ-ಲ್ಲಿ-ದ್ದರೂ
ಸಾನ್ನಿ-ಧ್ಯ-ದ-ಲ್ಲಿ-ರು-ವನು
ಸಾನ್ನಿ-ಧ್ಯ-ವನ್ನು
ಸಾನ್ನಿ-ಧ್ಯವೇ
ಸಾಪೇಕ್ಷ
ಸಾಪೇ-ಕ್ಷ-ವಸ್ತು
ಸಾಮ
ಸಾಮ-ಗರು
ಸಾಮ-ಗಳಲ್ಲಿ
ಸಾಮ-ಗಾಃ
ಸಾಮ-ಗ್ರಿ-ಗಳನ್ನು
ಸಾಮ-ಗ್ರಿ-ಗಳು
ಸಾಮ-ಗ್ರಿ-ಗ-ಳೆಲ್ಲ
ಸಾಮ-ರಸ್ಯ
ಸಾಮ-ರ-ಸ್ಯ-ವನ್ನು
ಸಾಮರ್ಥ್ಯ
ಸಾಮರ್ಥ್ಯಂ
ಸಾಮ-ರ್ಥ್ಯ-ಗಳ
ಸಾಮ-ರ್ಥ್ಯದ
ಸಾಮ-ರ್ಥ್ಯ-ದಿಂದ
ಸಾಮ-ರ್ಥ್ಯ-ವನ್ನು
ಸಾಮ-ರ್ಥ್ಯ-ವಾ-ಗಲಿ
ಸಾಮ-ರ್ಥ್ಯವೂ
ಸಾಮ-ವೇದ
ಸಾಮ-ವೇ-ದ-ದಲ್ಲಿ
ಸಾಮ-ವೇ-ದ-ವನ್ನು
ಸಾಮ-ವೇ-ದ-ವೆಂದು
ಸಾಮ-ವೇ-ದೋಽಸ್ಮಿ
ಸಾಮಾ-ನನ್ನು
ಸಾಮಾ-ನಿಗೆ
ಸಾಮಾ-ನಿನ
ಸಾಮಾ-ನಿ-ನಲ್ಲಿ
ಸಾಮಾ-ನಿ-ರು-ವಾ-ಗಲೂ
ಸಾಮಾನು
ಸಾಮಾ-ನು-ಗಳನ್ನು
ಸಾಮಾ-ನು-ಗ-ಳಿವೆ
ಸಾಮಾ-ನು-ಗಳು
ಸಾಮಾ-ನು-ಗ-ಳೆಲ್ಲ
ಸಾಮಾನ್ಯ
ಸಾಮಾ-ನ್ಯ-ರಿಗೆ
ಸಾಮಾ-ನ್ಯ-ವಾಗಿ
ಸಾಮಾ-ನ್ಯ-ವಾ-ಗಿ-ರು-ವುದು
ಸಾಮಾ-ನ್ಯ-ವಾದ
ಸಾಮಾ-ನ್ಯ-ವಾ-ದು-ದನ್ನು
ಸಾಮಾ-ನ್ಯವೆ
ಸಾಮಾ-ಸಿ-ಕಸ್ಯ
ಸಾಮ್ನಾಂ
ಸಾಮ್ಯ-ದಲ್ಲಿ
ಸಾಮ್ಯ-ದೃಷ್ಟಿ
ಸಾಮ್ಯ-ವನ್ನು
ಸಾಮ್ಯೇ
ಸಾಮ್ಯೇನ
ಸಾಯ
ಸಾಯಂ-ಕಾಲ
ಸಾಯಂ-ಕಾ-ಲದ
ಸಾಯಂ-ಕಾ-ಲ-ದ-ವ-ರೆಗೆ
ಸಾಯದ
ಸಾಯ-ಬ-ಹುದು
ಸಾಯ-ಬೇ-ಕಾ-ಗಿಲ್ಲ
ಸಾಯ-ಬೇ-ಕಾ-ಗು-ವುದು
ಸಾಯ-ಬೇ-ಕಾ-ಯಿತು
ಸಾಯ-ಬೇಕು
ಸಾಯ-ಲಿ-ಚ್ಛಿ-ಸು-ವನು
ಸಾಯ-ಲಿಲ್ಲ
ಸಾಯ-ಲೇ-ಬೇಕು
ಸಾಯಿ
ಸಾಯಿ-ಸ-ಬೇಕೇ
ಸಾಯಿ-ಸುವ
ಸಾಯಿ-ಸು-ವು-ದಕ್ಕೆ
ಸಾಯಿ-ಸು-ವುದೇ
ಸಾಯು
ಸಾಯು-ತ್ತದೆ
ಸಾಯುತ್ತಾ
ಸಾಯು-ತ್ತಾ-ನೆಂದು
ಸಾಯು-ತ್ತಿ-ರುವ
ಸಾಯು-ತ್ತಿ-ರು-ವನು
ಸಾಯು-ತ್ತಿ-ರು-ವರು
ಸಾಯು-ತ್ತೇನೆ
ಸಾಯು-ತ್ತೇ-ವೆಯೊ
ಸಾಯುವ
ಸಾಯು-ವ-ತ-ನಕ
ಸಾಯು-ವರು
ಸಾಯು-ವಾಗ
ಸಾಯುವು
ಸಾಯು-ವು-ದ-ಕ್ಕಿಂತ
ಸಾಯು-ವು-ದಕ್ಕೆ
ಸಾಯು-ವು-ದರ
ಸಾಯು-ವು-ದಿಲ್ಲ
ಸಾಯು-ವುದು
ಸಾಯು-ವುದೂ
ಸಾಯು-ವುದೆ
ಸಾಯು-ವು-ದೊಂದೇ
ಸಾಯುವೆ
ಸಾಯು-ವೆವು
ಸಾರ
ಸಾರಥಿ
ಸಾರ-ಥಿ-ಯ-ನ್ನಾಗಿ
ಸಾರ-ಥಿ-ಯಾಗಿ
ಸಾರ-ಥಿ-ಯಾ-ಗಿ-ರು-ವನು
ಸಾರ-ಥಿ-ಯಾ-ಗಿ-ರು-ವುದು
ಸಾರ-ಥಿ-ಯಾ-ಗು-ವನು
ಸಾರ-ಥಿ-ಯಾ-ದ-ವನು
ಸಾರ-ಥ್ಯ-ವನ್ನು
ಸಾರ-ದಂತೆ
ಸಾರ-ವನ್ನು
ಸಾರ-ವ-ನ್ನೆಲ್ಲ
ಸಾರ-ವನ್ನೇ
ಸಾರ-ವ-ಸ್ತು-ಗಳನ್ನು
ಸಾರ-ವ-ಸ್ತು-ವನ್ನು
ಸಾರ-ವಾದ
ಸಾರವೂ
ಸಾರ-ವೆಲ್ಲ
ಸಾರವೇ
ಸಾರವೋ
ಸಾರ-ಸಂ-ಗ್ರ-ಹ-ವಿದೆ
ಸಾರಿ
ಸಾರಿದ
ಸಾರಿ-ರು-ವನು
ಸಾರಿಲ್ಲ
ಸಾರಿಸಿ
ಸಾರು-ತ್ತಲೇ
ಸಾರು-ತ್ತಾನೆ
ಸಾರು-ತ್ತಾರೆ
ಸಾರು-ತ್ತಿದೆ
ಸಾರು-ತ್ತಿ-ದ್ದವು
ಸಾರು-ತ್ತಿ-ರು-ವನು
ಸಾರು-ತ್ತಿವೆ
ಸಾರುವ
ಸಾರು-ವನು
ಸಾರು-ವಾಗ
ಸಾರು-ವು-ದ-ಕ್ಕಾಗಿ
ಸಾರು-ವು-ದ-ಕ್ಕಾ-ಗಿಯೇ
ಸಾರು-ವು-ದಕ್ಕೆ
ಸಾರು-ವುದು
ಸಾರು-ವುದೇ
ಸಾರು-ವುವು
ಸಾರ್ಥಕ
ಸಾರ್ಥ-ಕ-ವಾ-ಗು-ತ್ತಿತ್ತು
ಸಾರ್ಥ-ಕ-ವಾ-ಗು-ವುದು
ಸಾರ್ಥ-ಕ-ವಾ-ಗು-ವುವು
ಸಾರ್ಥ-ಕ-ವಾದ
ಸಾರ್ಥ-ಕ-ವಾ-ದು-ದನ್ನು
ಸಾರ್ಥ-ಕ-ವಾ-ದು-ದಲ್ಲ
ಸಾರ್ಥ-ಕ-ವಾ-ಯಿತು
ಸಾರ್ಥ-ಕವೇ
ಸಾರ್ವ-ಜ-ನಿಕ
ಸಾರ್ವ-ಭೌ-ಮ-ನಿಗೆ
ಸಾರ್ವ-ಭೌ-ಮನೂ
ಸಾಲ
ಸಾಲಕ್ಕೆ
ಸಾಲ-ಗಾ-ರರ
ಸಾಲ-ಗಾ-ರ-ರಾಗಿ
ಸಾಲ-ದ-ಲ್ಲಿ-ದ್ದೇವೆ
ಸಾಲ-ದಿಂದ
ಸಾಲದು
ಸಾಲ-ದ್ದಕ್ಕೆ
ಸಾಲ-ಮಾಡಿ
ಸಾಲ-ಮಾ-ಡುವ
ಸಾಲ-ವನ್ನು
ಸಾಲ-ವನ್ನೂ
ಸಾಲ-ವ-ನ್ನೆಲ್ಲ
ಸಾಲ-ವಾಗಿ
ಸಾಲ-ವಿ-ದ್ದರೆ
ಸಾಲವು
ಸಾಲವೂ
ಸಾವ-ಧಾ-ನ-ದಿಂದ
ಸಾವನ್ನು
ಸಾವಿಗೆ
ಸಾವಿನ
ಸಾವಿ-ನಲ್ಲಿ
ಸಾವಿರ
ಸಾವಿ-ರದ
ಸಾವಿ-ರ-ವಾಗಿ
ಸಾವಿ-ರ-ವಾ-ಗು-ವುದು
ಸಾವಿ-ರಾರು
ಸಾವು
ಸಾವು-ಗಳಿಂದ
ಸಾವೆ
ಸಾವೆಂದರೆ
ಸಾಷ್ಟಾಂಗ
ಸಾಸಿವೆ
ಸಾಸುವೆ
ಸಾಹಂ-ಕಾ-ರೇಣ
ಸಾಹಸ
ಸಾಹ-ಸ-ಕ್ಕಿಂತ
ಸಾಹ-ಸದ
ಸಾಹ-ಸ-ದಲ್ಲಿ
ಸಾಹ-ಸ-ದಿಂದ
ಸಾಹ-ಸ-ಪ-ರನೆ
ಸಾಹ-ಸ-ಯಾತ್ರೆ
ಸಾಹ-ಸ-ವನ್ನು
ಸಾಹ-ಸ-ವಲ್ಲ
ಸಾಹ-ಸ-ವೆಲ್ಲ
ಸಾಹ-ಸ-ವೇನೂ
ಸಾಹಸಿ
ಸಾಹಿ-ತಿ-ಯಾ-ಗಿ-ರ-ಬ-ಹುದು
ಸಾಹಿತ್ಯ
ಸಾಹಿ-ತ್ಯದ
ಸಾಹಿ-ತ್ಯ-ದ-ಲ್ಲಿಯೂ
ಸಾಹೇ-ಬ-ನೇನೂ
ಸಿ
ಸಿಂಹ
ಸಿಂಹ-ಗ-ಳಿ-ಗಿಂತ
ಸಿಂಹ-ಗ-ಳಿವೆ
ಸಿಂಹ-ನಾದಂ
ಸಿಂಹ-ನಾ-ದ-ವನ್ನು
ಸಿಂಹಾ-ಸನ
ಸಿಂಹಾ-ಸ-ನದ
ಸಿಂಹಾ-ಸ-ನ-ದಿಂದ
ಸಿಕ್ಕ-ತಕ್ಕ
ಸಿಕ್ಕದ
ಸಿಕ್ಕ-ದಂತೆ
ಸಿಕ್ಕ-ದ-ವನು
ಸಿಕ್ಕ-ದ-ವ-ರಿ-ಗೆಲ್ಲ
ಸಿಕ್ಕದು
ಸಿಕ್ಕದೆ
ಸಿಕ್ಕದೇ
ಸಿಕ್ಕ-ಬ-ಲ್ಲದು
ಸಿಕ್ಕ-ಬ-ಹು-ದಾದ
ಸಿಕ್ಕ-ಬೇ-ಕಾ-ದರೆ
ಸಿಕ್ಕ-ಬೇ-ಕಾ-ದುದು
ಸಿಕ್ಕ-ಬೇಕು
ಸಿಕ್ಕ-ಲಾ-ರದು
ಸಿಕ್ಕ-ಲಾ-ರವು
ಸಿಕ್ಕಲಿ
ಸಿಕ್ಕ-ಲಿಲ್ಲ
ಸಿಕ್ಕ-ಲೆಂದು
ಸಿಕ್ಕಾ-ಪಟ್ಟೆ
ಸಿಕ್ಕಿ
ಸಿಕ್ಕಿ-ಕೊಂಡ
ಸಿಕ್ಕಿ-ಕೊಂ-ಡರೆ
ಸಿಕ್ಕಿ-ಕೊಂ-ಡ-ವರು
ಸಿಕ್ಕಿ-ಕೊಂ-ಡಿರು
ಸಿಕ್ಕಿ-ಕೊಂ-ಡಿ-ರು-ವು-ದಿಲ್ಲ
ಸಿಕ್ಕಿ-ಕೊಂ-ಡಿ-ರು-ವೆವು
ಸಿಕ್ಕಿ-ಕೊಂಡು
ಸಿಕ್ಕಿ-ಕೊ-ಳ್ಳ-ದಂತೆ
ಸಿಕ್ಕಿ-ಕೊ-ಳ್ಳ-ಬಾ-ರದು
ಸಿಕ್ಕಿ-ಕೊ-ಳ್ಳು-ತ್ತಾನೆ
ಸಿಕ್ಕಿ-ಕೊ-ಳ್ಳು-ತ್ತೇ-ನೆಯೋ
ಸಿಕ್ಕಿ-ಕೊ-ಳ್ಳು-ತ್ತೇವೆ
ಸಿಕ್ಕಿ-ಕೊ-ಳ್ಳು-ತ್ತೇ-ವೆಯೇ
ಸಿಕ್ಕಿ-ಕೊ-ಳ್ಳು-ವನೊ
ಸಿಕ್ಕಿ-ಕೊ-ಳ್ಳು-ವರು
ಸಿಕ್ಕಿ-ಕೊ-ಳ್ಳು-ವು-ದಿಲ್ಲ
ಸಿಕ್ಕಿ-ಕೊ-ಳ್ಳು-ವುದು
ಸಿಕ್ಕಿ-ಕೊ-ಳ್ಳು-ವುವು
ಸಿಕ್ಕಿ-ಕೊ-ಳ್ಳು-ವೆವು
ಸಿಕ್ಕಿತು
ಸಿಕ್ಕಿದ
ಸಿಕ್ಕಿ-ದಂ-ತಾ-ಗಿ-ರು-ವನು
ಸಿಕ್ಕಿ-ದಂತೆ
ಸಿಕ್ಕಿ-ದರೂ
ಸಿಕ್ಕಿ-ದರೆ
ಸಿಕ್ಕಿ-ದ-ಷ್ಟ-ನ್ನೆಲ್ಲಾ
ಸಿಕ್ಕಿ-ದಾಗ
ಸಿಕ್ಕಿ-ದಾ-ಗಲೆ
ಸಿಕ್ಕಿ-ದು-ದನ್ನು
ಸಿಕ್ಕಿ-ದು-ದ-ರಲ್ಲಿ
ಸಿಕ್ಕಿದೆ
ಸಿಕ್ಕಿ-ದೆಯೆ
ಸಿಕ್ಕಿ-ದ್ದ-ರೇನೇ
ಸಿಕ್ಕಿ-ದ್ದಾರೆ
ಸಿಕ್ಕಿ-ಬಿ-ದ್ದಿಲ್ಲ
ಸಿಕ್ಕಿ-ಬಿದ್ದು
ಸಿಕ್ಕಿ-ಬೀ-ಳು-ವನು
ಸಿಕ್ಕಿ-ಬೀ-ಳು-ವುದು
ಸಿಕ್ಕಿ-ಬೀ-ಳು-ವೆವೋ
ಸಿಕ್ಕಿಯೇ
ಸಿಕ್ಕಿ-ರ-ಕೂ-ಡದು
ಸಿಕ್ಕಿ-ರ-ಬೇಕು
ಸಿಕ್ಕಿ-ರು-ತ್ತದೆ
ಸಿಕ್ಕಿ-ರು-ವಷ್ಟೆ
ಸಿಕ್ಕಿ-ರು-ವು-ದ-ರಲ್ಲಿ
ಸಿಕ್ಕಿ-ರು-ವು-ದಿಲ್ಲ
ಸಿಕ್ಕಿ-ರು-ವುದು
ಸಿಕ್ಕಿಲ್ಲ
ಸಿಕ್ಕಿ-ಲ್ಲದೆ
ಸಿಕ್ಕಿ-ಲ್ಲವೊ
ಸಿಕ್ಕಿಸಿ
ಸಿಕ್ಕಿ-ಸಿ-ಕೊಂ-ಡಂತೆ
ಸಿಕ್ಕಿ-ಸಿ-ಕ್ಕಿ-ದ-ವ-ರಿ-ಗೆಲ್ಲಾ
ಸಿಕ್ಕಿ-ಸಿ-ರುವ
ಸಿಕ್ಕಿ-ಸು-ತ್ತಾನೆ
ಸಿಕ್ಕಿ-ಸು-ವುದು
ಸಿಕ್ಕಿ-ಹಾ-ಕಿ-ಕೊ-ಳ್ಳು-ತ್ತೇವೆ
ಸಿಕ್ಕೀತೇ
ಸಿಕ್ಕು-ತ್ತದೆ
ಸಿಕ್ಕು-ತ್ತವೆ
ಸಿಕ್ಕು-ತ್ತಾನೆ
ಸಿಕ್ಕು-ತ್ತಿ-ದ್ದುದು
ಸಿಕ್ಕು-ತ್ತಿ-ರು-ವನು
ಸಿಕ್ಕುವ
ಸಿಕ್ಕು-ವಂ-ತಿಲ್ಲ
ಸಿಕ್ಕು-ವಂತೆ
ಸಿಕ್ಕು-ವನು
ಸಿಕ್ಕು-ವರು
ಸಿಕ್ಕು-ವ-ವ-ನಲ್ಲ
ಸಿಕ್ಕು-ವ-ವ-ನೊ-ಬ್ಬನೇ
ಸಿಕ್ಕುವು
ಸಿಕ್ಕು-ವು-ದಂತೆ
ಸಿಕ್ಕು-ವು-ದಕ್ಕೆ
ಸಿಕ್ಕು-ವು-ದಲ್ಲ
ಸಿಕ್ಕು-ವು-ದಿಲ್ಲ
ಸಿಕ್ಕು-ವುದು
ಸಿಕ್ಕು-ವುದೂ
ಸಿಕ್ಕು-ವುದೇ
ಸಿಕ್ಕು-ವು-ದೇನು
ಸಿಕ್ಕು-ವು-ದೇನೊ
ಸಿಕ್ಕು-ವುದೊ
ಸಿಕ್ಕು-ವುದೋ
ಸಿಕ್ಕು-ವುವು
ಸಿಗ-ಬ-ಲ್ಲದು
ಸಿಗ-ಬೇಕು
ಸಿಗ-ರೇ-ಟಿಗೆ
ಸಿಗ-ಲಾ-ರದು
ಸಿಗ-ಲಿಲ್ಲ
ಸಿಗಿ-ದಾಗ
ಸಿಗಿದು
ಸಿಗಿ-ದು-ಹಾ-ಕ-ಬ-ಲ್ಲುದು
ಸಿಗುವ
ಸಿಗು-ವು-ದಿಲ್ಲ
ಸಿಡಿದ
ಸಿಡಿದು
ಸಿಡಿಯು
ಸಿಡಿ-ಯುವ
ಸಿಡಿ-ಯು-ವುದು
ಸಿಡಿ-ಲನ್ನು
ಸಿಡಿ-ಲಿನ
ಸಿಡಿಲು
ಸಿಡಿ-ಲು-ಗಳು
ಸಿಡುಬು
ಸಿದರೂ
ಸಿದ್ಧ
ಸಿದ್ಧ-ತೆ-ಗ-ಳೆಲ್ಲ
ಸಿದ್ಧ-ನಾ-ಗ-ಬೇ-ಕಾ-ದರೆ
ಸಿದ್ಧ-ನಾ-ಗ-ಬೇಕು
ಸಿದ್ಧ-ನಾಗಿ
ಸಿದ್ಧ-ನಾ-ಗಿ-ದ್ದನು
ಸಿದ್ಧ-ನಾ-ಗಿ-ದ್ದರೂ
ಸಿದ್ಧ-ನಾ-ಗಿ-ದ್ದೇನೆ
ಸಿದ್ಧ-ನಾ-ಗಿರ
ಸಿದ್ಧ-ನಾ-ಗಿ-ರ-ಬೇಕು
ಸಿದ್ಧ-ನಾ-ಗಿರು
ಸಿದ್ಧ-ನಾ-ಗಿ-ರು-ತ್ತಾನೆ
ಸಿದ್ಧ-ನಾ-ಗಿ-ರು-ವನು
ಸಿದ್ಧ-ನಾ-ಗಿ-ರು-ವನೊ
ಸಿದ್ಧ-ನಾ-ಗಿ-ರು-ವೆನು
ಸಿದ್ಧ-ನಾ-ಗಿಲ್ಲ
ಸಿದ್ಧ-ನಾಗು
ಸಿದ್ಧ-ನಾ-ಗು-ತ್ತಾನೆ
ಸಿದ್ಧ-ನಾ-ಗು-ವನು
ಸಿದ್ಧ-ನಾದ
ಸಿದ್ಧ-ನಾ-ದಂತೆ
ಸಿದ್ಧ-ನಿ-ರು-ವನು
ಸಿದ್ಧ-ಪು-ರು-ಷ-ನಾ-ದರೋ
ಸಿದ್ಧ-ಪು-ರು-ಷ-ನಿಗೆ
ಸಿದ್ಧ-ಪು-ರು-ಷ-ರಾಗಿ
ಸಿದ್ಧ-ಪು-ರು-ಷರು
ಸಿದ್ಧ-ಮಾ-ಡಲು
ಸಿದ್ಧ-ಮಾ-ಡು-ವ-ವನು
ಸಿದ್ಧಯೇ
ಸಿದ್ಧರ
ಸಿದ್ಧ-ರಲ್ಲಿ
ಸಿದ್ಧ-ರಾ-ಗ-ಬ-ಹುದು
ಸಿದ್ಧ-ರಾ-ಗಿ-ರ-ಬ-ಹುದು
ಸಿದ್ಧ-ರಾ-ಗಿ-ರ-ಬೇಕು
ಸಿದ್ಧ-ರಾ-ಗಿ-ರ-ಲಿಲ್ಲ
ಸಿದ್ಧ-ರಾ-ಗಿ-ರು-ತ್ತಾರೆ
ಸಿದ್ಧ-ರಾ-ಗಿ-ರು-ವರು
ಸಿದ್ಧ-ರಾ-ಗಿ-ರು-ವರೊ
ಸಿದ್ಧ-ರಾ-ಗಿ-ರು-ವರೋ
ಸಿದ್ಧ-ರಾ-ಗಿ-ರು-ವ-ವರು
ಸಿದ್ಧ-ರಾ-ಗಿ-ರು-ವೆವು
ಸಿದ್ಧ-ರಾ-ಗಿಲ್ಲ
ಸಿದ್ಧ-ರಾ-ಗು-ತ್ತೇವೆ
ಸಿದ್ಧ-ರಾ-ಗು-ವು-ದಲ್ಲ
ಸಿದ್ಧರು
ಸಿದ್ಧ-ಳಾ-ಗಿ-ರು-ವಳು
ಸಿದ್ಧ-ವಾ-ಗಿತ್ತು
ಸಿದ್ಧ-ವಾ-ಗಿದೆ
ಸಿದ್ಧ-ವಾ-ಗಿ-ದ್ದೇವೆ
ಸಿದ್ಧ-ವಾ-ಗಿ-ರ-ಬ-ಹುದು
ಸಿದ್ಧ-ವಾ-ಗಿ-ರ-ಬೇಕು
ಸಿದ್ಧ-ವಾ-ಗಿ-ರ-ಲಿಲ್ಲ
ಸಿದ್ಧ-ವಾ-ಗಿರು
ಸಿದ್ಧ-ವಾ-ಗಿ-ರುವ
ಸಿದ್ಧ-ವಾ-ಗಿ-ರು-ವನು
ಸಿದ್ಧ-ವಾ-ಗಿ-ರು-ವರು
ಸಿದ್ಧ-ವಾ-ಗಿ-ರುವೆ
ಸಿದ್ಧ-ವಾ-ಗಿಲ್ಲ
ಸಿದ್ಧ-ವಾ-ಗಿವೆ
ಸಿದ್ಧ-ವಾ-ಗು-ತ್ತಿದೆ
ಸಿದ್ಧ-ವಾದ
ಸಿದ್ಧ-ವಾ-ದ-ವ-ನಿಗೆ
ಸಿದ್ಧ-ಸಂ-ಘ-ಗಳು
ಸಿದ್ಧ-ಸಂ-ಘಾಃ
ಸಿದ್ಧಾಂತ
ಸಿದ್ಧಾಂ-ತ-ಗಳನ್ನು
ಸಿದ್ಧಾಂ-ತ-ಗ-ಳಿ-ಗಿಂತ
ಸಿದ್ಧಾಂ-ತದ
ಸಿದ್ಧಾಂ-ತ-ದಂತೆ
ಸಿದ್ಧಾಂ-ತ-ದ-ವ-ರಿಗೂ
ಸಿದ್ಧಾಂ-ತ-ವ-ನ್ನಾಗಿ
ಸಿದ್ಧಾಂ-ತ-ವನ್ನು
ಸಿದ್ಧಾಂ-ತ-ವಾ-ದಿ-ಗಳೂ
ಸಿದ್ಧಾಂ-ತ-ವಿದೆ
ಸಿದ್ಧಾ-ನಾಂ
ಸಿದ್ಧಾ-ವ-ಸಿದ್ಧೌ
ಸಿದ್ಧಿ
ಸಿದ್ಧಿಂ
ಸಿದ್ಧಿ-ಗಳನ್ನು
ಸಿದ್ಧಿ-ಗಳು
ಸಿದ್ಧಿ-ಗಾಗಿ
ಸಿದ್ಧಿ-ಮ-ವಾ-ಪ್ನೋತಿ
ಸಿದ್ಧಿ-ಮ-ವಾ-ಪ್ಸ್ಯಸಿ
ಸಿದ್ಧಿ-ಮಿತೋ
ಸಿದ್ಧಿಯ
ಸಿದ್ಧಿ-ಯನ್ನು
ಸಿದ್ಧಿ-ಯಲ್ಲಿ
ಸಿದ್ಧಿ-ಯಾ-ಗು-ವು-ದಿಲ್ಲ
ಸಿದ್ಧಿಯೂ
ಸಿದ್ಧಿ-ರ್ಭ-ವತಿ
ಸಿದ್ಧಿ-ಸದೆ
ಸಿದ್ಧಿ-ಸದೇ
ಸಿದ್ಧಿ-ಸ-ಬೇಕು
ಸಿದ್ಧಿ-ಸ-ಲಾ-ರದು
ಸಿದ್ಧಿ-ಸಲಿ
ಸಿದ್ಧಿ-ಸಿ-ದರೆ
ಸಿದ್ಧಿ-ಸಿದೆ
ಸಿದ್ಧಿ-ಸಿ-ದ್ದರೆ
ಸಿದ್ಧಿಸು
ಸಿದ್ಧಿ-ಸು-ತ್ತದೆ
ಸಿದ್ಧಿ-ಸು-ವಂ-ತ-ಹ-ದಲ್ಲ
ಸಿದ್ಧಿ-ಸುವು
ಸಿದ್ಧಿ-ಸು-ವು-ದಿಲ್ಲ
ಸಿದ್ಧಿ-ಸು-ವುದು
ಸಿದ್ಧಿ-ಸು-ವು-ದೆಂ-ದರೆ
ಸಿದ್ಧಿ-ಸು-ವುವು
ಸಿದ್ಧೋಽಹಂ
ಸಿದ್ಧ್ಯ-ಸಿ-ದ್ಧ್ಯೊ-ರ್ನಿ-ರ್ವಿ-ಕಾರಃ
ಸಿದ್ಧ್ಯ-ಸಿ-ದ್ಧ್ಯೋಃ
ಸಿನಿಮ
ಸಿನಿ-ಮ-ದಲ್ಲಿ
ಸಿನಿಮಾ
ಸಿನಿ-ಮಾದ
ಸಿನಿ-ಮಾ-ದಲ್ಲಿ
ಸಿಪಾಯಿ
ಸಿಪಾ-ಯಿಯೂ
ಸಿಪ್ಪೆ-ಗಳು
ಸಿಪ್ಪೆ-ಯಂತೆ
ಸಿಬ್ಬಂದಿ
ಸಿಮೆಂ-ಟ್ರೋಡ್
ಸಿರ-ಬೇಕು
ಸಿರಿ
ಸಿರು-ವನು
ಸಿರು-ವನೊ
ಸಿಲು-ಕದ
ಸಿಲು-ಕದು
ಸಿಲು-ಕ-ದುದು
ಸಿಲು-ಕದೆ
ಸಿಲುಕಿ
ಸಿಲು-ಕಿ-ದರೆ
ಸಿಲು-ಕಿ-ದ್ದಾನೆ
ಸಿಲು-ಕಿ-ರು-ವ-ವನು
ಸಿಲು-ಕಿಲ್ಲ
ಸಿಲು-ಕು-ವು-ದಿಲ್ಲ
ಸಿಹಿ
ಸಿಹಿಗೂ
ಸಿಹಿಗೆ
ಸಿಹಿ-ತಿಂ-ಡಿ-ಯನ್ನು
ಸಿಹಿ-ಯನ್ನೇ
ಸಿಹಿ-ಯಾ-ಗ-ಬೇ-ಕಾ-ದರೆ
ಸಿಹಿ-ಯಾಗಿ
ಸಿಹಿ-ಯಾ-ಗಿ-ದ್ದರೆ
ಸಿಹಿ-ಯಾ-ಗಿ-ರು-ವುದನ್ನು
ಸಿಹಿ-ಯಾ-ಗಿ-ರು-ವುದು
ಸಿಹಿ-ಯಾ-ಗು-ವುದು
ಸಿಹಿ-ಯಾದ
ಸಿಹಿಯೇ
ಸೀಟಿನ
ಸೀತೆ
ಸೀತೆಗೆ
ಸೀತೆಯ
ಸೀದಂತಿ
ಸೀದು
ಸೀದು-ಹೋ-ಗಿದೆ
ಸೀದು-ಹೋ-ಗಿ-ರು-ವುದು
ಸೀರೆ
ಸೀಲನ್ನು
ಸೀಳಿ
ಸೀಳಿ-ಕೊಂಡು
ಸೀಳಿದ
ಸೀಳು-ವಂತೆ
ಸುಂಕ
ಸುಂಕ-ದಂತೆ
ಸುಂಟರ
ಸುಂಟ-ರ-ಗಾಳಿ
ಸುಂಟ-ರ-ಗಾ-ಳಿ-ಯಂತೆ
ಸುಂಟ-ರ-ಗಾ-ಳಿ-ಯಲ್ಲಿ
ಸುಂಟು-ರು-ಗಾ-ಳಿ-ಯಂತೆ
ಸುಂದರ
ಸುಂದ-ರನು
ಸುಂದ-ರ-ವಾಗಿ
ಸುಂದ-ರ-ವಾ-ಗಿದೆ
ಸುಂದ-ರ-ವಾ-ಗಿ-ರಲಿ
ಸುಂದ-ರ-ವಾ-ಗಿ-ರುವ
ಸುಂದ-ರ-ವಾ-ಗಿ-ರು-ವಂತೆ
ಸುಂದ-ರ-ವಾ-ಗಿ-ರು-ವನು
ಸುಂದ-ರ-ವಾ-ಗಿಲ್ಲ
ಸುಂದ-ರ-ವಾದ
ಸುಂದ-ರ-ವಾ-ದು-ವು-ಗಳು
ಸುಕೃತಂ
ಸುಕೃ-ತ-ದು-ಷ್ಕೃತೇ
ಸುಕೃ-ತ-ಸ್ಯಾ-ಹುಃ
ಸುಕೃ-ತಿ-ನೋ-ಽಜುನ
ಸುಖ
ಸುಖಂ
ಸುಖ-ಕ್ಕಾಗಿ
ಸುಖ-ಕ್ಕಾ-ದರೂ
ಸುಖಕ್ಕೂ
ಸುಖಕ್ಕೆ
ಸುಖ-ಕ್ಕೆಲ್ಲ
ಸುಖ-ಗಳನ್ನು
ಸುಖ-ಗಳನ್ನೆಲ್ಲ
ಸುಖ-ಗ-ಳಾ-ದರೊ
ಸುಖ-ಗಳು
ಸುಖ-ಗ-ಳೆಲ್ಲ
ಸುಖದ
ಸುಖ-ದ-ಲ್ಲಾ-ದರೊ
ಸುಖ-ದಲ್ಲಿ
ಸುಖ-ದ-ಲ್ಲಿಯೂ
ಸುಖ-ದಿಂದ
ಸುಖ-ದುಃಖ
ಸುಖ-ದುಃ-ಖ-ಗಳ
ಸುಖ-ದುಃ-ಖ-ಗಳನ್ನು
ಸುಖ-ದುಃ-ಖ-ಗಳಲ್ಲಿ
ಸುಖ-ದುಃ-ಖ-ಗ-ಳ-ಲ್ಲಿಯೂ
ಸುಖ-ದುಃ-ಖ-ಗ-ಳಲ್ಲೂ
ಸುಖ-ದುಃ-ಖ-ಗ-ಳಿ-ಗೆಲ್ಲಾ
ಸುಖ-ದುಃ-ಖ-ಗಳು
ಸುಖ-ದುಃ-ಖ-ಗ-ಳೆಂಬ
ಸುಖ-ದುಃ-ಖ-ದಲ್ಲಿ
ಸುಖ-ದುಃ-ಖ-ವನ್ನು
ಸುಖ-ದುಃ-ಖ-ಸಂಜ್ಞೆ
ಸುಖ-ದುಃ-ಖಾ-ನಾಂ
ಸುಖ-ದುಃಖೇ
ಸುಖ-ದ್ವೀ-ಪ-ವನ್ನು
ಸುಖ-ಪ-ಟ್ಟಿ-ದ್ದಕ್ಕೆ
ಸುಖ-ಪಟ್ಟು
ಸುಖ-ಪ-ಡು-ತ್ತೇನೆ
ಸುಖ-ಪ-ಡು-ತ್ತೇ-ವೆಯೋ
ಸುಖ-ಪ-ಡು-ವನು
ಸುಖ-ಪ-ಡು-ವು-ದ-ಕ್ಕಾ-ಗುವು
ಸುಖ-ಪ-ಡು-ವು-ದಕ್ಕೆ
ಸುಖ-ಪ-ಡು-ವುದು
ಸುಖ-ಪ-ಡೋಣ
ಸುಖ-ಪು-ರುಷ
ಸುಖ-ಪ್ರಾ-ಪ್ತಿ-ಗಾಗಿ
ಸುಖ-ಮ-ಕ್ಷ-ಯ-ಮ-ಶ್ನುತೇ
ಸುಖ-ಮ-ಶ್ನುತೇ
ಸುಖ-ಮಾ-ತ್ಯಂ-ತಿಕಂ
ಸುಖ-ಮು-ತ್ತ-ಮಮ್
ಸುಖಮ್
ಸುಖ-ಲೋ-ಭ-ದಿಂದ
ಸುಖ-ವನ್ನು
ಸುಖ-ವನ್ನೂ
ಸುಖ-ವ-ನ್ನೆಲ್ಲ
ಸುಖ-ವಲ್ಲ
ಸುಖ-ವಾಗಿ
ಸುಖ-ವಾ-ಗಿ-ರ-ಬೇ-ಕಾ-ದರೆ
ಸುಖ-ವಾ-ಗಿ-ರ-ಬೇ-ಕೆಂ-ಬು-ದಕ್ಕೆ
ಸುಖ-ವಾ-ಗಿ-ರ-ಲಾರ
ಸುಖ-ವಾ-ಗಿ-ರಲಿ
ಸುಖ-ವಾ-ಗಿರು
ಸುಖ-ವಾ-ಗಿ-ರು-ತ್ತಾನೆ
ಸುಖ-ವಾ-ಗಿ-ರು-ವು-ದಕ್ಕೆ
ಸುಖ-ವಾ-ಗಿ-ರು-ವುದನ್ನು
ಸುಖ-ವಾ-ಗು-ವುದು
ಸುಖ-ವಾದ
ಸುಖ-ವಾ-ದರೂ
ಸುಖ-ವಾ-ದರೋ
ಸುಖ-ವಿ-ದೆಯೊ
ಸುಖ-ವಿ-ದ್ದರೆ
ಸುಖ-ವಿಲ್ಲ
ಸುಖವೂ
ಸುಖವೆ
ಸುಖ-ವೆಂ-ತಲೂ
ಸುಖ-ವೆಂಬ
ಸುಖ-ವೆಂ-ಬುದು
ಸುಖ-ವೆ-ನ್ನು-ವುದನ್ನು
ಸುಖ-ವೆಲ್ಲ
ಸುಖ-ವೆಲ್ಲಿ
ಸುಖವೇ
ಸುಖವೊ
ಸುಖ-ವೊಂದೇ
ಸುಖ-ಶಾಂ-ತಿ-ಗಳನ್ನು
ಸುಖ-ಶಾಂ-ತಿಗೆ
ಸುಖ-ಸಂಗ
ಸುಖ-ಸಂ-ಗ-ದಿಂ-ದಲೂ
ಸುಖ-ಸಂ-ಗೇನ
ಸುಖ-ಸಂ-ತೋ-ಷ-ವನ್ನು
ಸುಖ-ಸಾ-ಧ-ನ-ಗ-ಳಿ-ಗಾಗಿ
ಸುಖ-ಸ್ಯೈ-ಕಾಂ-ತಿ-ಕಸ್ಯ
ಸುಖಾನಿ
ಸುಖಾ-ನು-ಭ-ವ-ದಲ್ಲಿ
ಸುಖಾ-ಭಿ-ಲಾ-ಶೆ-ಯ-ಲ್ಲಿಯೇ
ಸುಖಾ-ಭಿ-ಲಾ-ಷಿಯೂ
ಸುಖಾ-ಭಿ-ಲಾಷೆ
ಸುಖಿ
ಸುಖಿ-ಗ-ಳಾ-ದೇವು
ಸುಖಿನಃ
ಸುಖಿ-ಯಾ-ಗಿ-ದ್ದಾನು
ಸುಖಿ-ಯಾ-ಗಿ-ರ-ಬೇ-ಕಾ-ದರೆ
ಸುಖಿ-ಯಾ-ಗಿ-ರು-ವನು
ಸುಖೀ
ಸುಖೇ
ಸುಖೇನ
ಸುಖೇಷು
ಸುಗಂಧ
ಸುಗಂ-ಧ-ವನ್ನು
ಸುಗುಣ
ಸುಗು-ಣ-ಗಳು
ಸುಗು-ಣನೂ
ಸುಘೋಷ
ಸುಘೋ-ಷ-ಮ-ಣಿ-ಪು-ಷ್ಪಕೌ
ಸುಜ್ಞಾ-ನಕ್ಕೆ
ಸುಜ್ಞಾ-ನಿ-ಯಂತೆ
ಸುಟ್ಟಿದ್ದು
ಸುಟ್ಟು
ಸುಟ್ಟು-ಕೊಂಡ
ಸುಟ್ಟು-ಕೊಂ-ಡರೆ
ಸುಟ್ಟು-ಕೊಳ್ಳ
ಸುಟ್ಟು-ಹಾ-ಕ-ಬ-ಹುದು
ಸುಟ್ಟು-ಹೋ-ಗಿದೆ
ಸುಟ್ಟು-ಹೋ-ಗು-ವುದು
ಸುಟ್ಟು-ಹೋದ
ಸುಡ-ಬ-ಹುದು
ಸುಡ-ಲಾ-ರದು
ಸುಡ-ಲ್ಪ-ಟ್ಟಿ-ವೆಯೋ
ಸುಡು-ತ್ತಿದೆ
ಸುಡುವ
ಸುಡು-ವು-ದ-ಕ್ಕಾ-ಗು-ವು-ದಿಲ್ಲ
ಸುಡು-ವುದು
ಸುಣ್ಣ
ಸುಣ್ಣದ
ಸುತ-ಸೋಮ
ಸುತ್ತ
ಸುತ್ತ-ಮುತ್ತ
ಸುತ್ತ-ಮು-ತ್ತಲ
ಸುತ್ತ-ಮು-ತ್ತ-ಲಿನ
ಸುತ್ತ-ಮು-ತ್ತ-ಲಿ-ರುವ
ಸುತ್ತ-ಮು-ತ್ತ-ಲಿ-ರು-ವ-ವರು
ಸುತ್ತ-ಮು-ತ್ತ-ಲಿ-ರು-ವು-ದ-ನ್ನೆಲ್ಲ
ಸುತ್ತ-ಮು-ತ್ತಲು
ಸುತ್ತ-ಮು-ತ್ತಲೂ
ಸುತ್ತ-ಲಿ-ರುವ
ಸುತ್ತಲು
ಸುತ್ತಲೂ
ಸುತ್ತಲೇ
ಸುತ್ತ-ವುದೂ
ಸುತ್ತಾ
ಸುತ್ತಾಡಿ
ಸುತ್ತಾರೆ
ಸುತ್ತಿ
ಸುತ್ತಿ-ಕೊ-ಳ್ಳು-ವುದು
ಸುತ್ತಿಗೆ
ಸುತ್ತಿ-ಗೆಯ
ಸುತ್ತಿ-ಟ್ಟಿ-ರು-ವರು
ಸುತ್ತಿದ
ಸುತ್ತಿ-ರು-ವನು
ಸುತ್ತಿ-ರು-ವುದು
ಸುತ್ತಿ-ಸು-ತ್ತಿರು
ಸುತ್ತು
ಸುತ್ತು-ತ್ತಿದೆ
ಸುತ್ತು-ತ್ತಿ-ದ್ದರೆ
ಸುತ್ತು-ತ್ತಿ-ದ್ದ-ರೆಂದೂ
ಸುತ್ತು-ತ್ತಿರ
ಸುತ್ತು-ತ್ತಿ-ರ-ಬ-ಹುದೇ
ಸುತ್ತು-ತ್ತಿ-ರ-ಬೇ-ಕಾಗು
ಸುತ್ತು-ತ್ತಿ-ರ-ಬೇ-ಕಾ-ಗು-ವುದು
ಸುತ್ತು-ತ್ತಿರು
ಸುತ್ತು-ತ್ತಿ-ರು-ಬೇಕೋ
ಸುತ್ತು-ತ್ತಿ-ರುವ
ಸುತ್ತು-ತ್ತಿ-ರು-ವಂತೆ
ಸುತ್ತು-ತ್ತಿ-ರು-ವಂ-ತೆ-ಆ-ಗಿ-ಹೋದ
ಸುತ್ತು-ತ್ತಿ-ರು-ವ-ವನು
ಸುತ್ತು-ತ್ತಿ-ರು-ವು-ದಿಲ್ಲ
ಸುತ್ತು-ತ್ತಿ-ರು-ವುದು
ಸುತ್ತು-ತ್ತಿ-ರು-ವುದೋ
ಸುತ್ತು-ತ್ತಿಲ್ಲ
ಸುತ್ತು-ತ್ತಿವೆ
ಸುತ್ತು-ಮು-ತ್ತ-ಲನ್ನು
ಸುತ್ತುವ
ಸುತ್ತು-ವಂತೆ
ಸುತ್ತು-ವು-ದ-ಕ್ಕಿಂತ
ಸುತ್ತು-ವು-ದಕ್ಕೆ
ಸುತ್ತು-ವುದು
ಸುತ್ತು-ವುದೂ
ಸುದು-ರ್ದ-ರ್ಶ-ಮಿದಂ
ಸುದು-ರ್ಲಭಃ
ಸುದು-ಷ್ಕ-ರಮ್
ಸುದ್ದಿ
ಸುದ್ದಿ-ಯನ್ನು
ಸುದ್ದಿ-ಸ-ಮಾ-ಚಾರ
ಸುದ್ದಿ-ಸ-ಮಾ-ಚಾ-ರ-ಗ-ಳಿಗೆ
ಸುದ್ಧಿ
ಸುಧಾ-ರ-ಣೆಯ
ಸುಧಾ-ರಿ-ಸಿ-ಕೊಂಡು
ಸುಧಾ-ರಿ-ಸಿ-ಕೊ-ಳ್ಳು-ವಂತೆ
ಸುಧಾ-ರಿ-ಸು-ವು-ದೇನು
ಸುಧೀ-ರ್ಭೋಕ್ತಾ
ಸುನಿ-ಶ್ಚಿತ
ಸುನಿ-ಶ್ಚಿ-ತಮ್
ಸುಪುತ್ರ
ಸುಪ್ತ
ಸುಪ್ತ-ವಾ-ಗಿದೆ
ಸುಪ್ತ-ವಾ-ಗಿ-ದೆಯೊ
ಸುಪ್ತ-ವಾ-ಗಿದ್ದ
ಸುಪ್ತ-ವಾ-ಗಿ-ರುವ
ಸುಪ್ತ-ವಾದ
ಸುಪ್ತಾ-ವಸ್ಥೆ
ಸುಪ್ತಾ-ವ-ಸ್ಥೆಗೆ
ಸುಪ್ತಾ-ವ-ಸ್ಥೆ-ಯಲ್ಲಿ
ಸುಪ್ತಾ-ವ-ಸ್ಥೆ-ಯ-ಲ್ಲಿತ್ತು
ಸುಪ್ತಾ-ವ-ಸ್ಥೆ-ಯ-ಲ್ಲಿದೆ
ಸುಪ್ತಾ-ವ-ಸ್ಥೆ-ಯ-ಲ್ಲಿದ್ದು
ಸುಪ್ತಾ-ವ-ಸ್ಥೆ-ಯ-ಲ್ಲಿ-ರು-ತ್ತವೆ
ಸುಪ್ತಾ-ವ-ಸ್ಥೆ-ಯ-ಲ್ಲಿ-ರು-ವುದು
ಸುಪ್ರೀ-ತ-ನಾಗಿ
ಸುಪ್ರೀ-ತ-ನಾ-ಗು-ತ್ತಾನೆ
ಸುಪ್ರೀ-ತ-ನಾ-ಗು-ವನು
ಸುಪ್ರೀ-ತ-ನಾ-ಗು-ವುದು
ಸುಪ್ರೀ-ತ-ನಾದ
ಸುಭ-ದ್ರೆಯ
ಸುಮಾರು
ಸುಮ್ಮ-ನಾ-ಗು-ವುದು
ಸುಮ್ಮ-ನಾದ
ಸುಮ್ಮ-ನಾ-ದನು
ಸುಮ್ಮ-ನಿ-ರ-ಬೇಕು
ಸುಮ್ಮ-ನಿ-ರ-ಬೇ-ಕೆಂದು
ಸುಮ್ಮ-ನಿ-ರ-ಲಾರ
ಸುಮ್ಮ-ನಿ-ರ-ಲಾ-ರದೆ
ಸುಮ್ಮ-ನಿ-ರಲು
ಸುಮ್ಮ-ನಿ-ರಿ-ಸ-ಬ-ಹುದೆ
ಸುಮ್ಮ-ನಿ-ರಿಸಿ
ಸುಮ್ಮ-ನಿ-ರಿ-ಸು-ವುದು
ಸುಮ್ಮ-ನಿರು
ಸುಮ್ಮ-ನಿ-ರು-ತ್ತಾ-ನೆಯೊ
ಸುಮ್ಮ-ನಿ-ರು-ತ್ತಾರೆ
ಸುಮ್ಮ-ನಿ-ರುವ
ಸುಮ್ಮ-ನಿ-ರು-ವನು
ಸುಮ್ಮ-ನಿ-ರು-ವಾಗ
ಸುಮ್ಮ-ನಿ-ರು-ವು-ದಕ್ಕೆ
ಸುಮ್ಮ-ನಿ-ರು-ವು-ದಿಲ್ಲ
ಸುಮ್ಮ-ನಿ-ರು-ವುದು
ಸುಮ್ಮ-ನಿ-ರು-ವುದೇ
ಸುಮ್ಮನೆ
ಸುಮ್ಮನೇ
ಸುರ
ಸುರಂ-ಗ-ವನ್ನು
ಸುರ-ಕ್ಷಿತ
ಸುರ-ಕ್ಷಿ-ತರು
ಸುರ-ಕ್ಷಿ-ತ-ವಾಗಿ
ಸುರ-ಕ್ಷಿ-ತ-ವಾ-ಗಿ-ಟ್ಟಿ-ರು-ವುದು
ಸುರ-ಕ್ಷಿ-ತ-ವಾ-ಗಿ-ರು-ವಷ್ಟು
ಸುರ-ಕ್ಷಿ-ತ-ವಾದ
ಸುರ-ಗ-ಣಾಃ
ಸುರ-ತ-ರು-ವನ್ನೇ
ಸುರ-ಸಂಘಾ
ಸುರಾ-ಣಾ-ಮಪಿ
ಸುರಾ-ಸುರ
ಸುರಾ-ಸು-ರ-ಗಣಾ
ಸುರಿ-ದರೆ
ಸುರಿದು
ಸುರಿ-ಮ-ಳೆ-ಯನ್ನೇ
ಸುರಿ-ಮ-ಳೆಯೇ
ಸುರಿಯು
ಸುರಿ-ಯು-ವ-ತ-ನಕ
ಸುರಿ-ಯು-ವುದೊ
ಸುರಿ-ಸುತ್ತ
ಸುರಿ-ಸು-ತ್ತಾನೆ
ಸುರಿ-ಸು-ತ್ತಿವೆ
ಸುರಿ-ಸು-ತ್ತೇನೆ
ಸುರೆಯ
ಸುರೆ-ಯಂತೆ
ಸುರೆ-ಯನ್ನು
ಸುರೆ-ಯಲ್ಲ
ಸುರೆ-ಯಾ-ದರೋ
ಸುರೆಯೆ
ಸುರೇಂ-ದ್ರ-ಲೋ-ಕ-ಮ-ಶ್ನಂತಿ
ಸುಲಭ
ಸುಲಭಃ
ಸುಲ-ಭದ
ಸುಲ-ಭ-ವಲ್ಲ
ಸುಲ-ಭ-ವಾಗಿ
ಸುಲ-ಭ-ವಾ-ಗಿತ್ತು
ಸುಲ-ಭ-ವಾ-ಗಿ-ದೆಯೋ
ಸುಲ-ಭ-ವಾ-ಗಿ-ರ-ಬೇಕು
ಸುಲ-ಭ-ವಾ-ಗಿ-ರು-ವುದನ್ನು
ಸುಲ-ಭ-ವಾ-ಗಿ-ರು-ವುದು
ಸುಲ-ಭ-ವಾ-ಗು-ವುದು
ಸುಲ-ಭ-ವಾದ
ಸುಲ-ಭ-ವಾ-ದಾಗ
ಸುಲ-ಭ-ವಾ-ದುದು
ಸುಲ-ಭವೊ
ಸುಲಿಗೆ
ಸುಲಿ-ಯ-ಬ-ಲ್ಲ-ವನೇ
ಸುಲಿ-ಯು-ವು-ದಕ್ಕೆ
ಸುಳಿ
ಸುಳಿಗೆ
ಸುಳಿ-ಯನ್ನು
ಸುಳಿ-ಯನ್ನೇ
ಸುಳಿ-ಯಲೇ
ಸುಳಿ-ಯಲ್ಲಿ
ಸುಳಿ-ಯು-ವರು
ಸುಳಿ-ಯು-ವು-ದಿಲ್ಲ
ಸುಳಿಯೇ
ಸುಳಿವು
ಸುಳಿವೂ
ಸುಳಿ-ವೆಲ್ಲಾ
ಸುಳ್ಳನೊ
ಸುಳ್ಳನ್ನು
ಸುಳ್ಳಲ್ಲ
ಸುಳ್ಳಾ-ಗ-ಬೇಕು
ಸುಳ್ಳಾ-ಗಿ-ದ್ದರೆ
ಸುಳ್ಳಾ-ಗಿ-ರ-ಬ-ಹುದು
ಸುಳ್ಳಾ-ಗಿ-ರ-ಬೇ-ಕೆಂದು
ಸುಳ್ಳಿಗೆ
ಸುಳ್ಳಿ-ನಿಂದ
ಸುಳ್ಳು
ಸುಳ್ಳೇ
ಸುಳ್ಳೋ
ಸುವ
ಸುವನು
ಸುವರು
ಸುವ-ವ-ನಲ್ಲ
ಸುವ-ವನು
ಸುವಾ-ಸನೆ
ಸುವಾ-ಸ-ನೆ-ಯನ್ನು
ಸುವಿ-ರೂ-ಢ-ಮೂ-ಲ-ಮ-ಸಂ-ಗ-ಶ-ಸ್ತ್ರೇಣ
ಸುವು-ದಕ್ಕೆ
ಸುವುದು
ಸುವು-ದೊಂದು
ಸುವ್ಯ-ವ-ಸ್ಥಿತ
ಸುವ್ಯ-ವ-ಸ್ಥಿ-ತ-ವಾ-ಗಿ-ರು-ವು-ದಿಲ್ಲ
ಸುವ್ಯ-ವಸ್ಥೆ
ಸುಷು-ಪ್ತಿಗೆ
ಸುಸಂ-ಸ್ಕೃ-ತರು
ಸುಸುಖಂ
ಸುಸೂ-ತ್ರ-ವಾಗಿ
ಸುಸ್ತಾಗಿ
ಸುಸ್ತಿ-ಬ-ಡ್ಡಿ-ಯನ್ನು
ಸುಹೃತ್
ಸುಹೃ-ತ್ತಿನ
ಸುಹೃದಂ
ಸುಹೃ-ದ-ಶ್ಚೈವ
ಸುಹೃ-ನ್ಮಿ-ತ್ರಾ-ರ್ಯು-ದಾ-ಸೀ-ನ-ಮ-ಧ್ಯ-ಸ್ಥ-ದ್ವೇ-ಷ್ಯ-ಬಂ-ಧುಷು
ಸೂಕ್ತ-ವಾದ
ಸೂಕ್ಷ
ಸೂಕ್ಷ್ಮ
ಸೂಕ್ಷ್ಮ-ಕ-ಣ-ದಿಂದ
ಸೂಕ್ಷ್ಮ-ಕರ್ಮ
ಸೂಕ್ಷ್ಮಕ್ಕೆ
ಸೂಕ್ಷ್ಮ-ತಮ
ಸೂಕ್ಷ್ಮ-ತ್ವಾತ್
ಸೂಕ್ಷ್ಮದ
ಸೂಕ್ಷ್ಮ-ದ-ರ್ಶಕ
ಸೂಕ್ಷ್ಮ-ದ-ಲ್ಲಿಯೂ
ಸೂಕ್ಷ್ಮ-ನಾ-ಗ-ಬೇ-ಕಾ-ಯಿತು
ಸೂಕ್ಷ್ಮ-ನಿ-ಯ-ಮ-ಗಳನ್ನು
ಸೂಕ್ಷ್ಮ-ನಿ-ಯ-ಮ-ಗಳು
ಸೂಕ್ಷ್ಮನು
ಸೂಕ್ಷ್ಮ-ಪಂ-ಚ-ಭೂ-ತ-ಗಳು
ಸೂಕ್ಷ್ಮ-ಭಾ-ವ-ಗಳು
ಸೂಕ್ಷ್ಮ-ಯಂತ್ರ
ಸೂಕ್ಷ್ಮ-ರೂ-ಪ-ದಲ್ಲಿ
ಸೂಕ್ಷ್ಮ-ವನ್ನು
ಸೂಕ್ಷ್ಮ-ವ-ಸ್ತು-ಗ-ಳಿವೆ
ಸೂಕ್ಷ್ಮ-ವ-ಸ್ತು-ಗ-ಳೆಲ್ಲ
ಸೂಕ್ಷ್ಮ-ವಾಗಿ
ಸೂಕ್ಷ್ಮ-ವಾ-ಗಿ-ದ್ದರೆ
ಸೂಕ್ಷ್ಮ-ವಾ-ಗಿ-ರ-ಬೇಕು
ಸೂಕ್ಷ್ಮ-ವಾ-ಗಿ-ರು-ವು-ದಕ್ಕೆ
ಸೂಕ್ಷ್ಮ-ವಾ-ಗಿ-ರು-ವುದನ್ನು
ಸೂಕ್ಷ್ಮ-ವಾ-ಗಿ-ರು-ವು-ದ-ರಿಂದ
ಸೂಕ್ಷ್ಮ-ವಾ-ಗಿ-ರು-ವುದು
ಸೂಕ್ಷ್ಮ-ವಾ-ಗಿ-ರು-ವುದೇ
ಸೂಕ್ಷ್ಮ-ವಾ-ಗಿವೆ
ಸೂಕ್ಷ್ಮ-ವಾ-ಗುತ್ತ
ಸೂಕ್ಷ್ಮ-ವಾ-ಗು-ವುದು
ಸೂಕ್ಷ್ಮ-ವಾದ
ಸೂಕ್ಷ್ಮ-ವಾ-ದದ್ದು
ಸೂಕ್ಷ್ಮ-ವಾ-ದುದು
ಸೂಕ್ಷ್ಮ-ವೆಂದು
ಸೂಕ್ಷ್ಮವೋ
ಸೂಕ್ಷ್ಮ-ಸ್ವ-ರೂಪ
ಸೂಕ್ಷ್ಮಾತಿ
ಸೂಕ್ಷ್ಮಾ-ತಿ-ಸೂಕ್ಷ
ಸೂಕ್ಷ್ಮಾ-ತಿ-ಸೂಕ್ಷ್ಮ
ಸೂಕ್ಷ್ಮಾ-ತಿ-ಸೂ-ಕ್ಷ್ಮ-ವಾದ
ಸೂಕ್ಷ್ಮಾ-ವಸ್ಥೆ
ಸೂಕ್ಷ್ಮಾ-ವ-ಸ್ಥೆ-ಯಲ್ಲಿ
ಸೂಕ್ಷ್ಮಾ-ವ-ಸ್ಥೆ-ಯ-ಲ್ಲಿತ್ತು
ಸೂಕ್ಷ್ಮಾ-ವ-ಸ್ಥೆ-ಯ-ಲ್ಲಿ-ದ್ದರೊ
ಸೂಕ್ಷ್ಮೇಂ-ದ್ರಿ-ಯ-ಗಳನ್ನು
ಸೂಕ್ಷ್ಮೇಂ-ದ್ರಿ-ಯ-ಗಳು
ಸೂಚನೆ
ಸೂಚಿ-ಸಿ-ದರು
ಸೂಚಿ-ಸಿ-ದರೊ
ಸೂಚಿ-ಸು-ತ್ತಾನೆ
ಸೂಚಿ-ಸು-ತ್ತಾ-ರೆಯೋ
ಸೂಚಿ-ಸು-ತ್ತಿ-ದ್ದರು
ಸೂಚಿ-ಸು-ತ್ತಿ-ರು-ವುದೊ
ಸೂಚಿ-ಸು-ವು-ದ-ಕ್ಕಾಗಿ
ಸೂಚಿ-ಸು-ವು-ದಕ್ಕೆ
ಸೂಚಿ-ಸು-ವು-ದ-ಕ್ಕೋ-ಸ್ಕರ
ಸೂಚಿ-ಸು-ವುದು
ಸೂಜಿ-ಯನ್ನು
ಸೂಜಿ-ಯನ್ನೂ
ಸೂಜಿ-ಯಿಂದ
ಸೂತ-ಪು-ತ್ರ-ಸ್ತ-ಥಾಸೌ
ಸೂತ್ರ
ಸೂತ್ರದ
ಸೂತ್ರ-ದಂ-ತಹ
ಸೂತ್ರ-ದಂತೆ
ಸೂತ್ರ-ದಲ್ಲಿ
ಸೂತ್ರ-ದಿಂದ
ಸೂತ್ರ-ಧಾರ
ಸೂತ್ರ-ಧಾ-ರನ
ಸೂತ್ರ-ಧಾ-ರ-ನಂತೆ
ಸೂತ್ರ-ಧಾ-ರ-ನನ್ನು
ಸೂತ್ರ-ರೂ-ಪ-ದಲ್ಲಿ
ಸೂತ್ರ-ವನ್ನು
ಸೂತ್ರೇ
ಸೂಯತೇ
ಸೂರ್ಯ
ಸೂರ್ಯ-ಕಿ-ರಣ
ಸೂರ್ಯ-ಕಿ-ರ-ಣ-ಗಳು
ಸೂರ್ಯ-ಚಂ-ದ್ರರ
ಸೂರ್ಯ-ಚಂ-ದ್ರರು
ಸೂರ್ಯ-ಚಂ-ದ್ರರೇ
ಸೂರ್ಯನ
ಸೂರ್ಯ-ನಂತೆ
ಸೂರ್ಯ-ನನ್ನು
ಸೂರ್ಯ-ನನ್ನೇ
ಸೂರ್ಯ-ನಲ್ಲಿ
ಸೂರ್ಯ-ನ-ಲ್ಲಿ-ರುವ
ಸೂರ್ಯ-ನಷ್ಟು
ಸೂರ್ಯ-ನಾ-ರಾ-ಯಣ
ಸೂರ್ಯ-ನಿಂದ
ಸೂರ್ಯ-ನಿಂ-ದಲೇ
ಸೂರ್ಯ-ನಿಗೆ
ಸೂರ್ಯ-ನಿ-ಗೇನೂ
ಸೂರ್ಯ-ನಿ-ರು-ವನು
ಸೂರ್ಯ-ನಿ-ಲ್ಲದೆ
ಸೂರ್ಯನು
ಸೂರ್ಯ-ಪ್ರ-ಭೆಯ
ಸೂರ್ಯ-ರಂತೆ
ಸೂರ್ಯ-ರ-ನ್ನೊ-ಳ-ಗೊಂಡ
ಸೂರ್ಯ-ರಿಗೆ
ಸೂರ್ಯ-ರಿ-ರು-ವರೆ
ಸೂರ್ಯ-ರೊ-ಡನೆ
ಸೂರ್ಯ-ಲೋ-ಕ-ವಾ-ಗಿವೆ
ಸೂರ್ಯ-ಶಕ್ತಿ
ಸೂರ್ಯ-ಸ-ಹ-ಸ್ರಸ್ಯ
ಸೂರ್ಯೋ
ಸೂರ್ಯೋ-ದಯ
ಸೂರ್ಯೋ-ದ-ಯದ
ಸೂಸುತ್ತ
ಸೃಜತಿ
ಸೃಜಾ-ಮ್ಯ-ಹಮ್
ಸೃಜಿಸಿ
ಸೃತೀ
ಸೃಷ್ಚಿ
ಸೃಷ್ಚಿ-ಸಿ-ಕೊ-ಳ್ಳು-ವುದು
ಸೃಷ್ಟಂ
ಸೃಷ್ಟಿ
ಸೃಷ್ಟಿ-ಕ-ರ್ತನ
ಸೃಷ್ಟಿ-ಕ-ರ್ತ-ನನ್ನು
ಸೃಷ್ಟಿ-ಕ-ರ್ತ-ನಿಗೆ
ಸೃಷ್ಟಿ-ಕ-ರ್ತನೆ
ಸೃಷ್ಟಿ-ಕ-ರ್ತರೆ
ಸೃಷ್ಟಿ-ಕಾ-ಲ-ದಲ್ಲಿ
ಸೃಷ್ಟಿ-ಕಾ-ಲ-ದ-ಲ್ಲಿಯೇ
ಸೃಷ್ಟಿ-ಗಳು
ಸೃಷ್ಟಿಗೂ
ಸೃಷ್ಟಿಗೆ
ಸೃಷ್ಟಿ-ಗೊ-ಡೆ-ಯನೆ
ಸೃಷ್ಟಿ-ಚಕ್ರ
ಸೃಷ್ಟಿ-ನಾ-ಟ-ಕದ
ಸೃಷ್ಟಿ-ಮಾ-ಡು-ತ್ತಿ-ದ್ದರೆ
ಸೃಷ್ಟಿಯ
ಸೃಷ್ಟಿ-ಯನ್ನು
ಸೃಷ್ಟಿ-ಯ-ನ್ನೆಲ್ಲ
ಸೃಷ್ಟಿ-ಯ-ನ್ನೆಲ್ಲಾ
ಸೃಷ್ಟಿ-ಯ-ಮೇಲೆ
ಸೃಷ್ಟಿ-ಯಲ್ಲ
ಸೃಷ್ಟಿ-ಯಲ್ಲಿ
ಸೃಷ್ಟಿ-ಯ-ಲ್ಲಿ-ರುವ
ಸೃಷ್ಟಿ-ಯ-ಲ್ಲಿ-ರು-ವ-ವರು
ಸೃಷ್ಟಿ-ಯ-ಲ್ಲೆಲ್ಲಾ
ಸೃಷ್ಟಿ-ಯಾ-ಗಿ-ರ-ಬ-ಹುದೆ
ಸೃಷ್ಟಿ-ಯಾ-ಗು-ವುದು
ಸೃಷ್ಟಿ-ಯಾದ
ಸೃಷ್ಟಿ-ಯಾ-ದ-ಮೇಲೆ
ಸೃಷ್ಟಿ-ಯಾ-ದಾಗ
ಸೃಷ್ಟಿ-ಯಾ-ದಾ-ಗಲೂ
ಸೃಷ್ಟಿ-ಯಾ-ಯಿತು
ಸೃಷ್ಟಿ-ಯಿಂದ
ಸೃಷ್ಟಿ-ಯೆ-ನ್ನು-ವುದು
ಸೃಷ್ಟಿಯೇ
ಸೃಷ್ಟಿ-ರಾ-ಶಿ-ಯಲ್ಲಿ
ಸೃಷ್ಟಿ-ಸ-ಬೇ-ಕಾ-ಗಿಲ್ಲ
ಸೃಷ್ಟಿ-ಸಲಿ
ಸೃಷ್ಟಿ-ಸಲು
ಸೃಷ್ಟಿ-ಸ-ಲ್ಪ-ಟ್ಟಿತು
ಸೃಷ್ಟಿ-ಸ-ಲ್ಪ-ಟ್ಟಿದೆ
ಸೃಷ್ಟಿಸಿ
ಸೃಷ್ಟಿ-ಸಿ-ಕೊಂಡು
ಸೃಷ್ಟಿ-ಸಿ-ಕೊ-ಳ್ಳು-ವುದು
ಸೃಷ್ಟಿ-ಸಿದ
ಸೃಷ್ಟಿ-ಸಿ-ದನು
ಸೃಷ್ಟಿ-ಸಿ-ದನೋ
ಸೃಷ್ಟಿ-ಸಿ-ದರೂ
ಸೃಷ್ಟಿ-ಸಿ-ದರೆ
ಸೃಷ್ಟಿ-ಸಿ-ದವ
ಸೃಷ್ಟಿ-ಸಿ-ದ-ವ-ನಿಗೆ
ಸೃಷ್ಟಿ-ಸಿ-ದ-ವನು
ಸೃಷ್ಟಿ-ಸಿ-ದಾ-ತನೆ
ಸೃಷ್ಟಿ-ಸಿ-ದಾ-ತ-ನೊ-ಬ್ಬನೇ
ಸೃಷ್ಟಿ-ಸಿದೆ
ಸೃಷ್ಟಿ-ಸಿ-ದ್ದ-ಕ್ಕೆಲ್ಲ
ಸೃಷ್ಟಿ-ಸಿದ್ದೇ
ಸೃಷ್ಟಿ-ಸಿ-ದ್ದೇನೆ
ಸೃಷ್ಟಿ-ಸಿ-ರ-ಬೇಕು
ಸೃಷ್ಟಿ-ಸಿ-ರುವ
ಸೃಷ್ಟಿ-ಸಿ-ರು-ವನು
ಸೃಷ್ಟಿ-ಸಿ-ರು-ವುದು
ಸೃಷ್ಟಿ-ಸಿಲ್ಲ
ಸೃಷ್ಟಿ-ಸು-ತ್ತಾನೆ
ಸೃಷ್ಟಿ-ಸು-ತ್ತಿ-ರು-ವೆನು
ಸೃಷ್ಟಿ-ಸು-ತ್ತೇನೆ
ಸೃಷ್ಟಿ-ಸುವ
ಸೃಷ್ಟಿ-ಸು-ವಾ-ಗಲೇ
ಸೃಷ್ಟಿ-ಸು-ವು-ದಕ್ಕೆ
ಸೃಷ್ಟಿ-ಸು-ವು-ದಿಲ್ಲ
ಸೃಷ್ಟಿ-ಸು-ವುದು
ಸೃಷ್ಟಿ-ಸ್ಥಿತಿ
ಸೃಷ್ಟ್ವಾ
ಸೆಕೆಂ-ಡಿಕೆ
ಸೆಕ್ರೆ-ಟರಿ
ಸೆಖೆ
ಸೆಖೆ-ಯಾ-ಗು-ವುದು
ಸೆಟ್ಟಿ-ನಂತೆ
ಸೆಣ-ಸಿ-ದರೂ
ಸೆರೆ
ಸೆರೆ-ಮ-ನೆ-ಗ-ಳಿ-ಗಿಂ-ತಲೂ
ಸೆರೆ-ಮ-ನೆಯ
ಸೆರೆ-ಮ-ನೆ-ಯಲ್ಲಿ
ಸೆರೆ-ಮ-ನೆ-ಯ-ಲ್ಲಿ-ಟ್ಟರು
ಸೆರೆ-ಮ-ನೆ-ಯಾ-ಗು-ವುದು
ಸೆರೆ-ಮ-ನೆ-ಯಿಂದ
ಸೆರೆ-ಯ-ಲ್ಲಿ-ರು-ವು-ದಿಲ್ಲ
ಸೆರೆ-ಯಾಳು
ಸೆಲ್ಗಳು
ಸೆಲ್ಲು-ಗಳು
ಸೆಳೆತ
ಸೆಳೆ-ತಕ್ಕೆ
ಸೆಳೆ-ತ-ದಿಂದ
ಸೆಳೆ-ತ-ವನ್ನು
ಸೆಳೆದ
ಸೆಳೆ-ದರೆ
ಸೆಳೆದು
ಸೆಳೆ-ದು-ಕೊಂ-ಡಿ-ರುವೆ
ಸೆಳೆ-ದು-ಕೊಂಡು
ಸೆಳೆ-ದು-ಕೊ-ಳ್ಳುವ
ಸೆಳೆ-ದು-ಕೊ-ಳ್ಳು-ವನು
ಸೆಳೆ-ದು-ಕೊ-ಳ್ಳು-ವನೊ
ಸೆಳೆ-ದು-ಕೊ-ಳ್ಳು-ವ-ವನು
ಸೆಳೆ-ದು-ಕೊ-ಳ್ಳು-ವಾಗ
ಸೆಳೆ-ದು-ಕೊ-ಳ್ಳು-ವು-ದಕ್ಕೆ
ಸೆಳೆ-ದು-ಕೊ-ಳ್ಳು-ವುದು
ಸೆಳೆ-ದು-ಕೊ-ಳ್ಳು-ವುದೊ
ಸೆಳೆಯ
ಸೆಳೆ-ಯ-ಕೂ-ಡದು
ಸೆಳೆ-ಯ-ಬ-ಹುದು
ಸೆಳೆ-ಯ-ಬೇಕು
ಸೆಳೆ-ಯ-ಲಾ-ರದು
ಸೆಳೆ-ಯ-ಲ್ಪ-ಡು-ತ್ತಾನೆ
ಸೆಳೆ-ಯು-ತ್ತ-ದೆಯೊ
ಸೆಳೆ-ಯು-ತ್ತವೆ
ಸೆಳೆ-ಯು-ತ್ತಿ-ದೆಯೊ
ಸೆಳೆ-ಯು-ತ್ತಿ-ರುವ
ಸೆಳೆ-ಯು-ತ್ತಿ-ರು-ವಂತೆ
ಸೆಳೆ-ಯು-ತ್ತಿವೆ
ಸೆಳೆ-ಯುವ
ಸೆಳೆ-ಯು-ವಂ-ತಹ
ಸೆಳೆ-ಯು-ವ-ದಿ-ಲ್ಲವೇ
ಸೆಳೆ-ಯು-ವುದು
ಸೆಳೆ-ಯು-ವುದೇ
ಸೆಳೆ-ಯು-ವುವು
ಸೇಡನ್ನು
ಸೇಡು
ಸೇತುವೆ
ಸೇತು-ವೆಯ
ಸೇತು-ವೆ-ಯಂ-ತಿದೆ
ಸೇತು-ವೆ-ಯಂತೆ
ಸೇದಲು
ಸೇದು-ತ್ತಿ-ದ್ದರು
ಸೇದು-ತ್ತೇವೆ
ಸೇದು-ವು-ದಕ್ಕೆ
ಸೇನ-ಯೋ-ರು-ಭ-ಯೋ-ರಪಿ
ಸೇನ-ಯೋ-ರು-ಭ-ಯೋ-ರ್ಮಧ್ಯೇ
ಸೇನಾ
ಸೇನಾನಿ
ಸೇನಾ-ನಿಯ
ಸೇನಾ-ನಿ-ಯಾಗಿ
ಸೇನಾ-ನೀ-ನಾ-ಮಹಂ
ಸೇನಾ-ಪ-ತಿ-ಗಳಲ್ಲಿ
ಸೇನಾ-ಪ-ತಿ-ಗಳೊ
ಸೇನೆ
ಸೇನೆ-ಗಳ
ಸೇನೆ-ಗ-ಳ-ಲ್ಲಿಯೂ
ಸೇನೆ-ಗಳು
ಸೇನೆ-ಗ-ಳೆಲ್ಲ
ಸೇನೆಗೆ
ಸೇನೆಯ
ಸೇನೆ-ಯನ್ನು
ಸೇನೆ-ಯನ್ನೇ
ಸೇನೆ-ಯಲ್ಲಿ
ಸೇನೆ-ಯಾ-ದರೊ
ಸೇನೆ-ಯೆಲ್ಲಾ
ಸೇಬಿನ
ಸೇರ
ಸೇರ-ದಂತೆ
ಸೇರ-ದ-ವನು
ಸೇರದೆ
ಸೇರದೇ
ಸೇರ-ಬ-ಹುದು
ಸೇರ-ಬೇ-ಕಾದ
ಸೇರ-ಬೇ-ಕಾ-ದರೆ
ಸೇರ-ಬೇಕು
ಸೇರ-ಲಾ-ರೆವು
ಸೇರಲಿ
ಸೇರ-ಲಿಲ್ಲ
ಸೇರಲು
ಸೇರಿ
ಸೇರಿ-ಕೊಂ-ಡರೆ
ಸೇರಿ-ಕೊಂ-ಡಿ-ರು-ವು-ದ-ರಿಂದ
ಸೇರಿ-ಕೊಂ-ಡಿ-ರು-ವುದು
ಸೇರಿ-ಕೊಂಡು
ಸೇರಿದ
ಸೇರಿ-ದಂತೆ
ಸೇರಿ-ದ-ಮೇಲೆ
ಸೇರಿ-ದರೂ
ಸೇರಿ-ದರೆ
ಸೇರಿ-ದವ
ಸೇರಿ-ದ-ವ-ನಲ್ಲ
ಸೇರಿ-ದ-ವನು
ಸೇರಿ-ದ-ವನೆ
ಸೇರಿ-ದ-ವ-ನೆಂಬ
ಸೇರಿ-ದ-ವರ
ಸೇರಿ-ದ-ವ-ರಿಗೆ
ಸೇರಿ-ದ-ವರು
ಸೇರಿ-ದ-ವ-ರೆಂದು
ಸೇರಿ-ದ-ವರೇ
ಸೇರಿ-ದವು
ಸೇರಿ-ದಾಗ
ಸೇರಿ-ದಾ-ಗಲೇ
ಸೇರಿ-ದು-ದಲ್ಲ
ಸೇರಿದೆ
ಸೇರಿ-ದೆ-ಯೇನು
ಸೇರಿ-ದೆಯೊ
ಸೇರಿದ್ದ
ಸೇರಿ-ದ್ದರೂ
ಸೇರಿ-ದ್ದರೊ
ಸೇರಿ-ದ್ದಾರೆ
ಸೇರಿದ್ದು
ಸೇರಿ-ನಲ್ಲಿ
ಸೇರಿಯೇ
ಸೇರಿ-ರ-ಬ-ಹುದು
ಸೇರಿ-ರಲಿ
ಸೇರಿ-ರ-ಲಿ-ಅದು
ಸೇರಿರು
ಸೇರಿ-ರು-ವರು
ಸೇರಿ-ರು-ವರೆಲ್ಲ
ಸೇರಿ-ರು-ವುದು
ಸೇರಿ-ರು-ವೆವೊ
ಸೇರಿಲ್ಲ
ಸೇರಿ-ಲ್ಲದ
ಸೇರಿ-ಲ್ಲವೋ
ಸೇರಿವೆ
ಸೇರಿ-ವೆಯೊ
ಸೇರಿ-ಸದೆ
ಸೇರಿಸಿ
ಸೇರಿ-ಸಿ-ಕೊಂಡು
ಸೇರಿ-ಸಿ-ಕೊ-ಳ್ಳಲು
ಸೇರಿ-ಸಿ-ಡು-ವನು
ಸೇರಿ-ಸಿ-ದಂತೆ
ಸೇರಿ-ಸಿ-ದರು
ಸೇರಿ-ಸಿ-ದಾಗ
ಸೇರಿ-ಸಿಯೊ
ಸೇರಿ-ಸಿಯೋ
ಸೇರಿ-ಸಿ-ರು-ವರು
ಸೇರಿಸು
ಸೇರಿ-ಸುತ್ತಾ
ಸೇರಿ-ಸು-ತ್ತಾನೆ
ಸೇರಿ-ಸುವ
ಸೇರಿ-ಸು-ವನು
ಸೇರಿ-ಸು-ವು-ದಕ್ಕೆ
ಸೇರಿ-ಸು-ವು-ದಿಲ್ಲ
ಸೇರಿ-ಸು-ವುದು
ಸೇರು
ಸೇರು-ತ್ತವೆ
ಸೇರು-ತ್ತಾನೆ
ಸೇರು-ತ್ತಾರೆ
ಸೇರು-ತ್ತಿದೆ
ಸೇರು-ತ್ತಿ-ದ್ದರೆ
ಸೇರು-ತ್ತಿ-ರ-ಬೇಕು
ಸೇರು-ತ್ತಿ-ರು-ವಾಗ
ಸೇರು-ತ್ತಿ-ರು-ವುದೊ
ಸೇರು-ತ್ತೀಯೆ
ಸೇರು-ತ್ತೇವೆ
ಸೇರುವ
ಸೇರು-ವಂತೆ
ಸೇರು-ವನು
ಸೇರು-ವರು
ಸೇರು-ವರೊ
ಸೇರು-ವ-ವ-ರೆಗೂ
ಸೇರು-ವ-ವ-ರೆಗೆ
ಸೇರು-ವು-ದಕ್ಕೆ
ಸೇರು-ವು-ದ-ರಲ್ಲಿ
ಸೇರು-ವು-ದ-ಲ್ಲದೆ
ಸೇರು-ವು-ದಿಲ್ಲ
ಸೇರು-ವುದು
ಸೇರು-ವುದೇ
ಸೇರು-ವುದೋ
ಸೇರು-ವುವು
ಸೇರುವೆ
ಸೇಲ್ಸ್
ಸೇವ-ಕ-ನ-ನ್ನಾಗಿ
ಸೇವ-ಕ-ರಾ-ಗ-ಬೇಕು
ಸೇವ-ಕರು
ಸೇವತೇ
ಸೇವನೆ
ಸೇವ-ನೆ-ಯಿಂ-ದಲೇ
ಸೇವಯಾ
ಸೇವಿ-ಸ-ಬೇಕು
ಸೇವಿ-ಸಿದ
ಸೇವಿ-ಸಿ-ದರೆ
ಸೇವಿ-ಸಿ-ದಾಗ
ಸೇವಿ-ಸು-ತ್ತಾ-ನೆಯೋ
ಸೇವಿ-ಸು-ತ್ತಿ-ರು-ವೆವು
ಸೇವಿ-ಸುವ
ಸೇವಿ-ಸು-ವನು
ಸೇವಿ-ಸು-ವರೋ
ಸೇವಿ-ಸು-ವು-ದಕ್ಕೆ
ಸೇವಿ-ಸು-ವುದೇ
ಸೇವೆ
ಸೇವೆಗೆ
ಸೇವೆಯ
ಸೇವೆ-ಯಂತೆ
ಸೇವೆ-ಯನ್ನು
ಸೇವೆ-ಯಲ್ಲಿ
ಸೇವೆ-ಯಾ-ಗಿ-ರ-ಬ-ಹುದು
ಸೇವೆ-ಯಿಂದ
ಸೇವೆ-ಯಿಂ-ದಲೇ
ಸೇವೆಯೂ
ಸೇವೆ-ಯೆಂ-ಬುದೇ
ಸೇವೆಯೇ
ಸೈಕ-ಲ್ಮೇಲೆ
ಸೈಕ-ಲ್ಲನ್ನು
ಸೈಕ-ಲ್ಲಿನ
ಸೈನ್ಯ
ಸೈನ್ಯಕ್ಕೆ
ಸೈನ್ಯ-ಗಳ
ಸೈನ್ಯದ
ಸೈನ್ಯ-ದಲ್ಲಿ
ಸೈನ್ಯ-ವನ್ನು
ಸೈನ್ಯ-ವಾ-ದರೋ
ಸೈನ್ಯ-ವಿದೆ
ಸೈನ್ಯಸ್ಯ
ಸೊಂಟ
ಸೊಂಡಿಲು
ಸೊಗ-ಸು-ಗಾ-ರ-ನಂತೆ
ಸೊನ್ನೆ
ಸೊನ್ನೆ-ಗ-ಳಾ-ಗು-ತ್ತವೆ
ಸೊನ್ನೆಗೆ
ಸೊನ್ನೆಯ
ಸೊನ್ನೆ-ಯನ್ನು
ಸೊಪ್ಪು
ಸೊರ-ಗನ್ನು
ಸೊರಗಿ
ಸೊರ-ಗಿ-ಸು-ತ್ತಿ-ರುವ
ಸೊರ-ಗು-ತ್ತಿದೆ
ಸೊಳ್ಳೆ-ಗಳು
ಸೋಕು-ವು-ದಿಲ್ಲ
ಸೋಗನ್ನು
ಸೋಗು
ಸೋಜಿಗ
ಸೋಜಿ-ಗ-ವಾಗಿ
ಸೋಡಿ
ಸೋಢುಂ
ಸೋಢುಮ್
ಸೋತ-ನೆಂದು
ಸೋತ-ಮೇಲೂ
ಸೋತರೂ
ಸೋತರೆ
ಸೋತ-ವ-ನನ್ನು
ಸೋತಿ-ದ್ದಾರೆ
ಸೋತಿ-ರು-ವನು
ಸೋತು
ಸೋತು-ಹೋಗು
ಸೋತು-ಹೋ-ಗು-ತ್ತೇವೆ
ಸೋತು-ಹೋ-ದರೆ
ಸೋತೆ
ಸೋತ್ತೀರ್ಣಾ
ಸೋದರ
ಸೋದ-ರ-ಮಾ-ವ-ನಾ-ದರೂ
ಸೋದ-ರ-ಳಿಯ
ಸೋಪಮಾ
ಸೋಪಾನ
ಸೋಪಿನ
ಸೋಪು
ಸೋಮ-ದ-ತ್ತನ
ಸೋಮ-ನಾ-ಥಾ-ನಂದ
ಸೋಮ-ಪಾಃ
ಸೋಮ-ಪಾನ
ಸೋಮ-ರಸ
ಸೋಮಾರಿ
ಸೋಮಾ-ರಿ-ಗ-ಳಾಗಿ
ಸೋಮಾ-ರಿ-ಗ-ಳಿಗೆ
ಸೋಮಾ-ರಿಗೆ
ಸೋಮಾ-ರಿ-ತನ
ಸೋಮಾ-ರಿ-ಯಾಗಿ
ಸೋಮಾ-ರಿ-ಯಾಗು
ಸೋಮಾ-ರಿ-ಯಾ-ಗು-ವುದು
ಸೋಮೃ-ತ-ತ್ವಾಯ
ಸೋಮೋ
ಸೋರಿ-ಹೋ-ಗು-ವುದು
ಸೋರು-ವುದು
ಸೋರೆ
ಸೋರೆ-ಕಾಯಿ
ಸೋರೆ-ಬು-ರುಡೆ
ಸೋರೆ-ಬು-ರು-ಡೆ-ಯಂತೆ
ಸೋರೆಯ
ಸೋಲದೆ
ಸೋಲನ್ನು
ಸೋಲನ್ನೂ
ಸೋಲಲಿ
ಸೋಲಾ-ದರೂ
ಸೋಲಿನ
ಸೋಲಿ-ನಿಂದ
ಸೋಲಿ-ಸ-ಬೇ-ಕಾ-ದರೆ
ಸೋಲಿ-ಸ-ಬೇಕು
ಸೋಲಿ-ಸ-ಲಾರ
ಸೋಲಿಸಿ
ಸೋಲಿ-ಸಿದ
ಸೋಲಿ-ಸಿದ್ದು
ಸೋಲಿ-ಸು-ತ್ತೇನೆ
ಸೋಲಿ-ಸು-ವಂತೆ
ಸೋಲಿ-ಸು-ವುದೇ
ಸೋಲಿ-ಸು-ವು-ದೊಂದೇ
ಸೋಲು
ಸೋಲು-ಗೆ-ಲ-ವು-ಗಳು
ಸೋಲು-ತ್ತಾರೆ
ಸೋಲು-ತ್ತೇವೆ
ಸೋಲುವ
ಸೋಲು-ವು-ದ-ರಲ್ಲಿ
ಸೋಲು-ವು-ದಿ-ಲ್ಲವೊ
ಸೋಲು-ವುದು
ಸೋಲು-ವುದೇ
ಸೋಲೂ
ಸೋಲೆಂ-ಬು-ದಿಲ್ಲ
ಸೋಲೇ
ಸೋಽಜುನ
ಸೋಽಧಿ-ಗ-ಚ್ಛತಿ
ಸೋಽಪಿ
ಸೋಽವಿ-ಕಂ-ಪೇನ
ಸೌಂದರ್ಯ
ಸೌಂದ-ರ್ಯಕ್ಕೆ
ಸೌಂದ-ರ್ಯದ
ಸೌಂದ-ರ್ಯ-ದಲ್ಲಿ
ಸೌಂದ-ರ್ಯ-ವನ್ನು
ಸೌಕ-ರ್ಯ-ಕ್ಕಾಗಿ
ಸೌಕ-ರ್ಯ-ಗಳನ್ನು
ಸೌಕ-ರ್ಯ-ಗ-ಳಿವೆ
ಸೌಕ್ಷ್ಮ-್ಯಾ-ದಾ-ಕಾಶಂ
ಸೌಖ್ಯದ
ಸೌಖ್ಯಮು
ಸೌಟಿ-ನಿಂದ
ಸೌದೆ
ಸೌದೆಗೆ
ಸೌದೆಯ
ಸೌದೆ-ಯಂತೆ
ಸೌದೆ-ಯನ್ನು
ಸೌದೆ-ಯೆಲ್ಲ
ಸೌಧ
ಸೌಧ-ವನ್ನು
ಸೌಭ-ದ್ರಶ್ಚ
ಸೌಭದ್ರೋ
ಸೌಮ-ದತ್ತಿ
ಸೌಮ-ದ-ತ್ತಿ-ರ್ಜ-ಯ-ದ್ರಥಃ
ಸೌಮ್ಯ
ಸೌಮ್ಯಂ
ಸೌಮ್ಯತ್ವಂ
ಸೌಮ್ಯ-ಭಾವ
ಸೌಮ್ಯ-ಮೂರ್ತಿ
ಸೌಮ್ಯ-ರೂ-ಪ-ವನ್ನು
ಸೌಮ್ಯ-ವ-ಪು-ರ್ಮ-ಹಾತ್ಮಾ
ಸೌಮ್ಯ-ವಾ-ಗಿ-ರ-ಬೆಕು
ಸೌಮ್ಯ-ವಾ-ಗಿ-ರ-ಬೇಕು
ಸೌಮ್ಯ-ವಾ-ಗಿ-ರು-ವನು
ಸೌಮ್ಯ-ವಾದ
ಸೌರ-ಭ-ವನ್ನು
ಸೌಲ-ಭ್ಯ-ಗ-ಳಿತ್ತೊ
ಸೌಲ-ಭ್ಯ-ಗಳು
ಸೌಲ-ಭ್ಯ-ಗಳೂ
ಸ್ಕಂದ
ಸ್ಕಂದಃ
ಸ್ಕಂದನ
ಸ್ಕಂದನೇ
ಸ್ಕರ
ಸ್ಕರವೋ
ಸ್ಕಾಂತಕ್ಕೂ
ಸ್ಕ್ರೂ
ಸ್ಕ್ರೂಡ್ರೈ-ವ-ರಿ-ನಿಂದ
ಸ್ಟೇಷನ್
ಸ್ಟೇಷ-ನ್ಗ-ಳಿ-ಗಿಂತ
ಸ್ಟೇಷ-ನ್ನಿಗೆ
ಸ್ಟೋರೇ-ಜ್ನಂತೆ
ಸ್ಟ್ರೈಕು-ಗಳು
ಸ್ತಬ್ಧ
ಸ್ತಬ್ಧಃ
ಸ್ತಬ್ಧರು
ಸ್ತಬ್ಧಾ
ಸ್ತವಾಪಿ
ಸ್ತವೈ
ಸ್ತಿಮಿ-ತಕ್ಕೆ
ಸ್ತುತಿ
ಸ್ತುತಿ-ಗಳಲ್ಲಿ
ಸ್ತುತಿ-ಗಳಿಂದ
ಸ್ತುತಿ-ನಿಂದೆ
ಸ್ತುತಿ-ಭಿಃ
ಸ್ತುತಿ-ಸ-ಬೇ-ಕಾ-ದರೆ
ಸ್ತುತಿ-ಸು-ತ್ತ-ವೆಯೋ
ಸ್ತುತಿ-ಸು-ತ್ತಾನೆ
ಸ್ತುತಿ-ಸು-ತ್ತಿ-ರು-ವರು
ಸ್ತುತಿ-ಸು-ವನು
ಸ್ತುತಿ-ಸು-ವರು
ಸ್ತುತ್ಯ-ರ್ಹನೂ
ಸ್ತುನ್ವಂತಿ
ಸ್ತುವಂತಿ
ಸ್ತೇನ
ಸ್ತೇವ-ಸ್ಥಿ-ತಾಃ
ಸ್ತೋತ್ರ
ಸ್ತೋತ್ರ-ದಲ್ಲಿ
ಸ್ತ್ರಿಯೋ
ಸ್ತ್ರೀ
ಸ್ತ್ರೀಯ
ಸ್ತ್ರೀಯನ್ನು
ಸ್ತ್ರೀಯ-ರನ್ನು
ಸ್ತ್ರೀಯ-ರಲ್ಲಿ
ಸ್ತ್ರೀಯ-ರಿ-ಗಿ-ರ-ಲಿಲ್ಲ
ಸ್ತ್ರೀಯ-ರಿಗೂ
ಸ್ತ್ರೀಯ-ರಿಗೆ
ಸ್ತ್ರೀಯರು
ಸ್ತ್ರೀಷು
ಸ್ಥಳ
ಸ್ಥಳ-ಕ್ಕಿಂತ
ಸ್ಥಳಕ್ಕೆ
ಸ್ಥಳ-ಕ್ಕೆ-ತ-ಮ್ಮನ್ನು
ಸ್ಥಳ-ಗಳನ್ನು
ಸ್ಥಳ-ಗಳಲ್ಲಿ
ಸ್ಥಳ-ಗ-ಳ-ಲ್ಲಿಯೂ
ಸ್ಥಳ-ಗ-ಳಿಗೆ
ಸ್ಥಳ-ಗ-ಳಿವೆ
ಸ್ಥಳ-ಗಳು
ಸ್ಥಳ-ಗ-ಳೆಲ್ಲ
ಸ್ಥಳದ
ಸ್ಥಳ-ದ-ಲ್ಲಾ-ಗಲಿ
ಸ್ಥಳ-ದ-ಲ್ಲಾ-ದರೂ
ಸ್ಥಳ-ದಲ್ಲಿ
ಸ್ಥಳ-ದ-ಲ್ಲಿಯೂ
ಸ್ಥಳ-ದ-ಲ್ಲಿಯೇ
ಸ್ಥಳ-ದ-ಲ್ಲಿ-ರುವ
ಸ್ಥಳ-ದಲ್ಲೆ
ಸ್ಥಳ-ದಲ್ಲೊ
ಸ್ಥಳ-ದಷ್ಟು
ಸ್ಥಳ-ದಿಂದ
ಸ್ಥಳ-ವನ್ನು
ಸ್ಥಳ-ವಲ್ಲ
ಸ್ಥಳ-ವಾ-ಗಲೀ
ಸ್ಥಳ-ವಾ-ಗಿ-ರ-ಬಾ-ರದು
ಸ್ಥಳ-ವಾ-ಗಿ-ರ-ಬೇಕು
ಸ್ಥಳ-ವಿದೆ
ಸ್ಥಳ-ವಿ-ದೆ-ಹಿಂ-ದಿನ
ಸ್ಥಳ-ವಿ-ದ್ದರೆ
ಸ್ಥಳ-ವಿಲ್ಲ
ಸ್ಥಳವೂ
ಸ್ಥಳವೆ
ಸ್ಥಳವೇ
ಸ್ಥಾಣು
ಸ್ಥಾಣು-ರ-ಚ-ಲೋಯಂ
ಸ್ಥಾನ
ಸ್ಥಾನಂ
ಸ್ಥಾನಕ್ಕೆ
ಸ್ಥಾನ-ಗ-ಳಿವೆ
ಸ್ಥಾನ-ಗಳು
ಸ್ಥಾನದ
ಸ್ಥಾನ-ದಲ್ಲಿ
ಸ್ಥಾನ-ದ-ಲ್ಲಿಯೇ
ಸ್ಥಾನ-ದೊಂ-ದಿಗೆ
ಸ್ಥಾನ-ಮು-ಪೈತಿ
ಸ್ಥಾನ-ವನ್ನು
ಸ್ಥಾನ-ವಿದೆ
ಸ್ಥಾನ-ವಿ-ಲ್ಲ-ದ-ವನೂ
ಸ್ಥಾನೇ
ಸ್ಥಾಪನೆ
ಸ್ಥಾಪ-ನೆಗೆ
ಸ್ಥಾಪಯ
ಸ್ಥಾಪ-ಯಿತ್ವಾ
ಸ್ಥಾಪಿ-ತ-ವಾಗಿ
ಸ್ಥಾಪಿ-ಸ-ಬಲ್ಲ
ಸ್ಥಾಪಿ-ಸು-ತ್ತೇವೆ
ಸ್ಥಾಪಿ-ಸು-ವರು
ಸ್ಥಾಪಿ-ಸು-ವು-ದ-ಕ್ಕಾಗಿ
ಸ್ಥಾವರ
ಸ್ಥಾವ-ರ-ಗಳಲ್ಲಿ
ಸ್ಥಾವ-ರ-ಜಂ-ಗ-ಮಮ್
ಸ್ಥಾವ-ರಾ-ಣಾಂ
ಸ್ಥಾಸ್ಯತಿ
ಸ್ಥಿತಂ
ಸ್ಥಿತಃ
ಸ್ಥಿತ-ಧೀಃ
ಸ್ಥಿತ-ಧೀ-ರ್ಮು-ನಿ-ರು-ಚ್ಯತೇ
ಸ್ಥಿತ-ಪ್ರಜ್ಞ
ಸ್ಥಿತ-ಪ್ರ-ಜ್ಞನ
ಸ್ಥಿತ-ಪ್ರ-ಜ್ಞ-ನಾ-ಗಿ-ರ-ಬೇಕು
ಸ್ಥಿತ-ಪ್ರ-ಜ್ಞ-ನಾ-ದರೂ
ಸ್ಥಿತ-ಪ್ರ-ಜ್ಞ-ನಿಗೆ
ಸ್ಥಿತ-ಪ್ರ-ಜ್ಞನೂ
ಸ್ಥಿತ-ಪ್ರ-ಜ್ಞ-ಸ್ತ-ದೋ-ಚ್ಯತೇ
ಸ್ಥಿತ-ಪ್ರ-ಜ್ಞಸ್ಯ
ಸ್ಥಿತಮ್
ಸ್ಥಿತ-ಶ್ಚ-ಲತಿ
ಸ್ಥಿತಾಃ
ಸ್ಥಿತಾನ್
ಸ್ಥಿತಿ
ಸ್ಥಿತಿಂ
ಸ್ಥಿತಿಃ
ಸ್ಥಿತಿ-ಗ-ತಿ-ಗಳನ್ನು
ಸ್ಥಿತಿ-ಗ-ಳ-ಲ್ಲಿದೆ
ಸ್ಥಿತಿ-ಗ-ಳಿಗೆ
ಸ್ಥಿತಿಗೂ
ಸ್ಥಿತಿಗೆ
ಸ್ಥಿತಿಯ
ಸ್ಥಿತಿ-ಯನ್ನು
ಸ್ಥಿತಿ-ಯನ್ನೇ
ಸ್ಥಿತಿ-ಯಲ್ಲ
ಸ್ಥಿತಿ-ಯಲ್ಲಿ
ಸ್ಥಿತಿ-ಯ-ಲ್ಲಿ-ಟ್ಟರೂ
ಸ್ಥಿತಿ-ಯ-ಲ್ಲಿಟ್ಟು
ಸ್ಥಿತಿ-ಯ-ಲ್ಲಿದೆ
ಸ್ಥಿತಿ-ಯ-ಲ್ಲಿ-ದ್ದರೂ
ಸ್ಥಿತಿ-ಯ-ಲ್ಲಿ-ದ್ದಾನೆ
ಸ್ಥಿತಿ-ಯ-ಲ್ಲಿ-ದ್ದೇನೆ
ಸ್ಥಿತಿ-ಯ-ಲ್ಲಿಯೂ
ಸ್ಥಿತಿ-ಯ-ಲ್ಲಿಯೇ
ಸ್ಥಿತಿ-ಯ-ಲ್ಲಿ-ರ-ಬ-ಹುದು
ಸ್ಥಿತಿ-ಯ-ಲ್ಲಿರು
ಸ್ಥಿತಿ-ಯ-ಲ್ಲಿ-ರು-ತ್ತವೆ
ಸ್ಥಿತಿ-ಯ-ಲ್ಲಿ-ರುವ
ಸ್ಥಿತಿ-ಯ-ಲ್ಲಿ-ರು-ವನು
ಸ್ಥಿತಿ-ಯ-ಲ್ಲಿ-ರು-ವರು
ಸ್ಥಿತಿ-ಯ-ಲ್ಲಿ-ರು-ವ-ವ-ನಿಗೆ
ಸ್ಥಿತಿ-ಯ-ಲ್ಲಿ-ರು-ವೆವೋ
ಸ್ಥಿತಿ-ಯ-ಲ್ಲಿಲ್ಲ
ಸ್ಥಿತಿ-ಯಲ್ಲೂ
ಸ್ಥಿತಿ-ಯಲ್ಲೇ
ಸ್ಥಿತಿ-ಯಾ-ಗ-ಬೇ-ಕಾ-ದರೆ
ಸ್ಥಿತಿ-ಯಾ-ಗುವು
ಸ್ಥಿತಿ-ಯಾ-ದರೂ
ಸ್ಥಿತಿ-ಯಿಂದ
ಸ್ಥಿತಿ-ಯಿಲ್ಲ
ಸ್ಥಿತಿಯು
ಸ್ಥಿತಿಯೂ
ಸ್ಥಿತಿಯೇ
ಸ್ಥಿತೋ
ಸ್ಥಿತೋಽಸ್ಮಿ
ಸ್ಥಿತೌ
ಸ್ಥಿತ್ವಾ-ಸ್ಯಾ-ಮಂ-ತ-ಕಾ-ಲೇಽಪಿ
ಸ್ಥಿಮಿ-ತ-ವಿಲ್ಲ
ಸ್ಥಿರಃ
ಸ್ಥಿರ-ಗೊ-ಳಿ-ಸಲು
ಸ್ಥಿರ-ಗೊ-ಳಿಸಿ
ಸ್ಥಿರ-ಗೊ-ಳಿ-ಸು-ತ್ತೇನೆ
ಸ್ಥಿರ-ನಾಗಿ
ಸ್ಥಿರ-ನಾ-ಗಿ-ದ್ದೀಯೆ
ಸ್ಥಿರ-ನಾ-ಗಿ-ರು-ವನು
ಸ್ಥಿರ-ಬು-ದ್ಧಿಗೆ
ಸ್ಥಿರ-ಬು-ದ್ಧಿ-ಯು-ಳ್ಳ-ವನು
ಸ್ಥಿರ-ಬು-ದ್ಧಿ-ಯು-ಳ್ಳ-ವನೂ
ಸ್ಥಿರ-ಬು-ದ್ಧಿ-ರ-ಸಂ-ಮೂಢೋ
ಸ್ಥಿರ-ಮ-ತಿ-ರ್ಭ-ಕ್ತಿ-ಮಾನ್
ಸ್ಥಿರ-ಮಾ-ಡಿ-ಕೊಂ-ಡಿರು
ಸ್ಥಿರ-ಮಾ-ಸ-ನ-ಮಾ-ತ್ಮನಃ
ಸ್ಥಿರಮ್
ಸ್ಥಿರ-ವಲ್ಲ
ಸ್ಥಿರ-ವಾಗಿ
ಸ್ಥಿರ-ವಾ-ಗಿದೆ
ಸ್ಥಿರ-ವಾ-ಗಿ-ರ-ಬ-ಹುದು
ಸ್ಥಿರ-ವಾ-ಗಿ-ರ-ಬೇಕು
ಸ್ಥಿರ-ವಾ-ಗಿ-ರು-ವನು
ಸ್ಥಿರ-ವಾ-ಗಿ-ರು-ವುದು
ಸ್ಥಿರ-ವಾ-ಗು-ವುದು
ಸ್ಥಿರ-ವಾದ
ಸ್ಥಿರಾ
ಸ್ಥಿರಾಮ್
ಸ್ಥೂಲ
ಸ್ಥೂಲದ
ಸ್ಥೂಲ-ದಿಂದ
ಸ್ಥೂಲ-ದೇಹ
ಸ್ಥೂಲ-ರೂ-ಪಕ್ಕೆ
ಸ್ಥೂಲ-ರೂ-ಪ-ವಾ-ಗಿಯೂ
ಸ್ಥೂಲ-ವನ್ನು
ಸ್ಥೂಲ-ವಾಗಿ
ಸ್ಥೂಲ-ವಾ-ಗಿತ್ತು
ಸ್ಥೂಲ-ವಾ-ಗಿ-ರು-ವುದನ್ನು
ಸ್ಥೂಲ-ವಾ-ಗಿ-ರು-ವುದು
ಸ್ಥೂಲ-ವಾ-ಗಿ-ರು-ವುದೇ
ಸ್ಥೂಲ-ವಾ-ಗಿ-ರು-ವುವು
ಸ್ಥೂಲ-ವಾದ
ಸ್ಥೂಲ-ಶಕ್ತಿ
ಸ್ಥೂಲಾ-ವ-ಸ್ಥೆಗೆ
ಸ್ಥೂಲೇಂ-ದ್ರಿಯ
ಸ್ಥೈರ್ಯ
ಸ್ಥೈರ್ಯ-ಮಾ-ತ್ಮ-ವಿ-ನಿ-ಗ್ರಹಃ
ಸ್ನಾನ
ಸ್ನಾನಕ್ಕೆ
ಸ್ನಾನ-ಮಾ-ಡ-ಬೇ-ಕಾ-ಗು-ವುದು
ಸ್ನಾನ-ಮಾ-ಡ-ಬೇಕು
ಸ್ನಾನ-ಮಾ-ಡಿ-ದರೂ
ಸ್ನಾನ-ಮಾ-ಡು-ತ್ತಾನೆ
ಸ್ನಾನ-ವನ್ನು
ಸ್ನಾನವೂ
ಸ್ನಾನಾ-ದಿ-ಗಳನ್ನು
ಸ್ನಾನಾ-ದಿ-ಗಳಿಂದ
ಸ್ನಿಗ್ಧಾಃ
ಸ್ನೇಹ
ಸ್ನೇಹ-ವನ್ನು
ಸ್ನೇಹವೇ
ಸ್ನೇಹಿತ
ಸ್ನೇಹಿ-ತನ
ಸ್ನೇಹಿ-ತ-ನಂತೆ
ಸ್ನೇಹಿ-ತ-ನನ್ನು
ಸ್ನೇಹಿ-ತ-ನಾ-ಗಿ-ದ್ದಾನೆ
ಸ್ನೇಹಿ-ತ-ನಾ-ಗಿ-ರ-ಬ-ಹುದು
ಸ್ನೇಹಿ-ತ-ನಾ-ಗು-ವೆನೋ
ಸ್ನೇಹಿ-ತ-ನಾದ
ಸ್ನೇಹಿ-ತ-ನಿಗೆ
ಸ್ನೇಹಿ-ತ-ನಿಗೋ
ಸ್ನೇಹಿ-ತನೂ
ಸ್ನೇಹಿ-ತ-ನೆಂದು
ಸ್ನೇಹಿ-ತ-ನೆಂದೂ
ಸ್ನೇಹಿ-ತ-ರನ್ನು
ಸ್ನೇಹಿ-ತ-ರಲ್ಲ
ಸ್ನೇಹಿ-ತ-ರಾಗಿ
ಸ್ನೇಹಿ-ತ-ರಾ-ಗಿ-ದ್ದರು
ಸ್ನೇಹಿ-ತ-ರಾ-ಗು-ವರು
ಸ್ನೇಹಿ-ತರು
ಸ್ನೇಹಿ-ತರೊ
ಸ್ಪಂಜನ್ನು
ಸ್ಪಂದ-ನ-ವನ್ನು
ಸ್ಪಂದ-ನವೇ
ಸ್ಪಂದಿ-ಸ-ಲಾ-ರದು
ಸ್ಪಂದಿಸು
ಸ್ಪಂದಿ-ಸು-ತ್ತದೆ
ಸ್ಪಂದಿ-ಸು-ತ್ತಿದೆ
ಸ್ಪಂದಿ-ಸು-ತ್ತಿ-ರು-ತ್ತದೆ
ಸ್ಪಂದಿ-ಸು-ತ್ತಿ-ರು-ವನು
ಸ್ಪಂದಿ-ಸು-ತ್ತಿ-ರು-ವುದು
ಸ್ಪಂದಿ-ಸು-ವಂತೆ
ಸ್ಪಂದಿ-ಸು-ವುದನ್ನು
ಸ್ಪರ್ಧೆ
ಸ್ಪರ್ಶ
ಸ್ಪರ್ಶಕ್ಕೆ
ಸ್ಪರ್ಶ-ದಿಂದ
ಸ್ಪರ್ಶನಂ
ಸ್ಪರ್ಶ-ಮಾ-ಡಿದ
ಸ್ಪರ್ಶ-ಶಿಲೆ
ಸ್ಪರ್ಶ-ಶಿ-ಲೆಗೆ
ಸ್ಪರ್ಶ-ಶಿ-ಲೆ-ಯನ್ನು
ಸ್ಪರ್ಶಾನ್
ಸ್ಪರ್ಶಿ-ಸಿ-ದರು
ಸ್ಪರ್ಶಿ-ಸು-ವನು
ಸ್ಪಲ್ಪ
ಸ್ಪಲ್ಪ-ವನ್ನು
ಸ್ಪಷ್ಟತೆ
ಸ್ಪಷ್ಟ-ಪ-ಡಿ-ಸು-ವನು
ಸ್ಪಷ್ಟ-ವಾಗಿ
ಸ್ಪುಟ್ನಿಕ್
ಸ್ಪುಟ್ನಿ-ಕ್ನಲ್ಲಿ
ಸ್ಪೃಹಾ
ಸ್ಪೃಹೆ
ಸ್ಪ್ರಿಂಗ್
ಸ್ಪ್ರಿಂಗ್ನಿಂದ
ಸ್ಫುರಿಸು
ಸ್ಫುರಿ-ಸು-ತ್ತಿ-ರು-ವುದು
ಸ್ಫುರಿ-ಸು-ವುದು
ಸ್ಫುರಿ-ಸು-ವುವು
ಸ್ಫೂರ್ತಿ
ಸ್ಫೂರ್ತಿ-ಗೊಂಡು
ಸ್ಫೂರ್ತಿ-ಯನ್ನು
ಸ್ಮ
ಸ್ಮರಣೆ
ಸ್ಮರ-ಣೆ-ಮಾ-ಡುತ್ತಾ
ಸ್ಮರ-ಣೆಯ
ಸ್ಮರ-ಣೆ-ಯನ್ನು
ಸ್ಮರತಿ
ಸ್ಮರನ್
ಸ್ಮರಿ-ಸಿ-ಕೊಂ-ಡರೂ
ಸ್ಮರಿ-ಸಿ-ಕೊಂಡು
ಸ್ಮರಿ-ಸಿ-ಕೊಳ್ಳ
ಸ್ಮರಿ-ಸಿ-ಕೊ-ಳ್ಳು-ವುದು
ಸ್ಮರಿ-ಸುತ್ತ
ಸ್ಮರಿ-ಸು-ತ್ತಿ-ರು-ವನೋ
ಸ್ಮರಿ-ಸು-ವ-ವನು
ಸ್ಮರಿ-ಸು-ವು-ದಕ್ಕೆ
ಸ್ಮಶಾನ
ಸ್ಮಶಾ-ನ-ಭೂ-ಮಿ-ಯಾ-ಗು-ವುದನ್ನು
ಸ್ಮಾರಕ
ಸ್ಮೃತಃ
ಸ್ಮೃತಮ್
ಸ್ಮೃತಾ
ಸ್ಮೃತಿ
ಸ್ಮೃತಿ-ಗಳನ್ನು
ಸ್ಮೃತಿ-ಭ್ರಂ-ಶಾ-ದ್ಬು-ದ್ಧಿ-ನಾಶೋ
ಸ್ಮೃತಿ-ಭ್ರಮೆ
ಸ್ಮೃತಿ-ಭ್ರ-ಮೆ-ಯಿಂದ
ಸ್ಮೃತಿಯ
ಸ್ಮೃತಿ-ಯಲ್ಲಿ
ಸ್ಮೃತಿಯೇ
ಸ್ಮೃತಿ-ರ್ಜ್ಞಾ-ನ-ಮ-ಪೋ-ಹನಂ
ಸ್ಮೃತಿ-ರ್ಮೇಧಾ
ಸ್ಮೃತಿ-ರ್ಲಬ್ಧಾ
ಸ್ಮೃತಿ-ವಿ-ಭ್ರಮಃ
ಸ್ಮೃತಿ-ಶಕ್ತಿ
ಸ್ಯಂದನೇ
ಸ್ಯಾಂಪಲ್
ಸ್ಯಾಜ್ಜ-ನಾ-ರ್ದನ
ಸ್ಯಾತ್
ಸ್ಯಾತ್ತ್ರಿ-ಭಿ-ರ್ಗು-ಣೈಃ
ಸ್ಯಾದಾ-ತ್ಮ-ತೃ-ಪ್ತಶ್ಚ
ಸ್ಯಾದ್ಭಾ-ಸ-ಸ್ತಸ್ಯ
ಸ್ಯಾನ್ನಿ-ಶ್ಚಿತಂ
ಸ್ಯಾನ್ಮಯಾ
ಸ್ಯಾಮ
ಸ್ಯಾಮಿತಿ
ಸ್ಯಾಮು-ಪ-ಹ-ನ್ಯಾ-ಮಿ-ಮಾಃ
ಸ್ಯುಃ
ಸ್ರಂಸತೇ
ಸ್ರೋತ-ಸಾ-ಮಸ್ಮಿ
ಸ್ಲೇಟಿನ
ಸ್ವ
ಸ್ವಂ
ಸ್ವಂತ
ಸ್ವಂತ-ದ್ದ-ನ್ನಾಗಿ
ಸ್ವಂತ-ದ್ದಾ-ಗಿ-ರ-ಬೇಕು
ಸ್ವಕಂ
ಸ್ವಕ-ರ್ಮಣಾ
ಸ್ವಕ-ರ್ಮದ
ಸ್ವಕ-ರ್ಮ-ದಲ್ಲಿ
ಸ್ವಕ-ರ್ಮ-ದಿಂದ
ಸ್ವಕ-ರ್ಮ-ನಿ-ರತಃ
ಸ್ವಚ-ಕ್ಷುಷಾ
ಸ್ವಜನಂ
ಸ್ವಜ-ನ-ಮಾ-ಹವೇ
ಸ್ವಜ-ನ-ಮು-ದ್ಯ-ತಾಃ
ಸ್ವಜ-ನ-ರನ್ನು
ಸ್ವಜ-ನ-ರಾದ
ಸ್ವಜ-ನರೇ
ಸ್ವತಂತ್ರ
ಸ್ವತಂ-ತ್ರ-ರಲ್ಲ
ಸ್ವತಂ-ತ್ರ-ರಾಗಿ
ಸ್ವತಂ-ತ್ರ-ರಾ-ಗಿ-ರು-ವ-ರೆಂದು
ಸ್ವತಂ-ತ್ರರು
ಸ್ವತಂ-ತ್ರ-ವಾಗಿ
ಸ್ವತಂತ್ರಿ
ಸ್ವತಃ
ಸ್ವತಃ-ಸಿ-ದ್ಧ-ವಸ್ತು
ಸ್ವತೇ-ಜಸಾ
ಸ್ವಧರ್ಮ
ಸ್ವಧರ್ಮಂ
ಸ್ವಧ-ರ್ಮ-ಕ್ಕೋ-ಸ್ಕರ
ಸ್ವಧ-ರ್ಮ-ಚ್ಯು-ತ-ನಾಗಿ
ಸ್ವಧ-ರ್ಮದ
ಸ್ವಧ-ರ್ಮ-ದಲ್ಲಿ
ಸ್ವಧ-ರ್ಮ-ಮಪಿ
ಸ್ವಧ-ರ್ಮವೇ
ಸ್ವಧರ್ಮೇ
ಸ್ವಧರ್ಮೋ
ಸ್ವಧಾ
ಸ್ವಧಾ-ಮಕ್ಕೆ
ಸ್ವಧಾ-ಹ-ಮ-ಹ-ಮೌ-ಷ-ಧಮ್
ಸ್ವನು-ಷ್ಠಿ-ತಾತ್
ಸ್ವಪನ್
ಸ್ವಪ್ನ
ಸ್ವಪ್ನಂ
ಸ್ವಪ್ನ-ಲೋ-ಕಕ್ಕೆ
ಸ್ವಪ್ರ-ಯತ್ನ
ಸ್ವಪ್ರ-ಯ-ತ್ನದ
ಸ್ವಪ್ರ-ಯ-ತ್ನ-ದಲ್ಲಿ
ಸ್ವಪ್ರ-ಯ-ತ್ನ-ವನ್ನು
ಸ್ವಬಾಂ-ಧ-ವಾನ್
ಸ್ವಭಾವ
ಸ್ವಭಾ-ವಕ್ಕೆ
ಸ್ವಭಾ-ವ-ಗಳನ್ನು
ಸ್ವಭಾ-ವ-ಗ-ಳಾ-ವುವೂ
ಸ್ವಭಾ-ವ-ಗ-ಳಿವೆ
ಸ್ವಭಾ-ವ-ಗಳು
ಸ್ವಭಾ-ವ-ಗ-ಳೆಲ್ಲಾ
ಸ್ವಭಾ-ವ-ಗ-ಳೇನು
ಸ್ವಭಾ-ವ-ಜಮ್
ಸ್ವಭಾ-ವಜಾ
ಸ್ವಭಾ-ವ-ಜೇನ
ಸ್ವಭಾ-ವತಃ
ಸ್ವಭಾ-ವದ
ಸ್ವಭಾ-ವ-ದಂತೆ
ಸ್ವಭಾ-ವ-ದಲ್ಲಿ
ಸ್ವಭಾ-ವ-ದ-ವ-ನಿಗೆ
ಸ್ವಭಾ-ವ-ದ-ವನು
ಸ್ವಭಾ-ವ-ದ-ವರ
ಸ್ವಭಾ-ವ-ದ-ವ-ರನ್ನು
ಸ್ವಭಾ-ವ-ದ-ವ-ರಿಗೆ
ಸ್ವಭಾ-ವ-ದ-ವರು
ಸ್ವಭಾ-ವ-ದಿಂದ
ಸ್ವಭಾ-ವ-ದೊ-ಡನೆ
ಸ್ವಭಾ-ವದ್ದು
ಸ್ವಭಾ-ವ-ನಿ-ಯತಂ
ಸ್ವಭಾ-ವ-ಪ್ರ-ಭ-ವೈ-ರ್ಗು-ಣೈಃ
ಸ್ವಭಾ-ವ-ವನ್ನು
ಸ್ವಭಾ-ವ-ವನ್ನೂ
ಸ್ವಭಾ-ವ-ವ-ನ್ನೆಲ್ಲ
ಸ್ವಭಾ-ವ-ವನ್ನೇ
ಸ್ವಭಾ-ವ-ವಲ್ಲ
ಸ್ವಭಾ-ವ-ವಾಗಿ
ಸ್ವಭಾ-ವ-ವಾ-ಗಿದೆ
ಸ್ವಭಾ-ವ-ವಾ-ಗಿ-ರು-ವುದು
ಸ್ವಭಾ-ವ-ವಾ-ಗಿ-ಹೋ-ಗು-ವುದೊ
ಸ್ವಭಾ-ವ-ವಾ-ಗುತ್ತ
ಸ್ವಭಾ-ವ-ವಾ-ಗುತ್ತಾ
ಸ್ವಭಾ-ವ-ವಾ-ಗು-ವುದು
ಸ್ವಭಾ-ವ-ವಾ-ದರೆ
ಸ್ವಭಾ-ವ-ವಾ-ದರೊ
ಸ್ವಭಾ-ವ-ವಾ-ದರೋ
ಸ್ವಭಾ-ವ-ವು-ಳ್ಳ-ವನ
ಸ್ವಭಾ-ವ-ವು-ಳ್ಳ-ವ-ನಾಗಿ
ಸ್ವಭಾ-ವ-ವು-ಳ್ಳ-ವನೂ
ಸ್ವಭಾ-ವ-ವು-ಳ್ಳ-ವರ
ಸ್ವಭಾ-ವ-ವು-ಳ್ಳ-ವರು
ಸ್ವಭಾ-ವ-ವು-ಳ್ಳ-ವರೇ
ಸ್ವಭಾ-ವ-ವು-ಳ್ಳ-ವು-ಗಳು
ಸ್ವಭಾ-ವವೆ
ಸ್ವಭಾ-ವವೇ
ಸ್ವಭಾ-ವ-ವೇನು
ಸ್ವಭಾ-ವಸ್ತು
ಸ್ವಭಾ-ವೋ-ಽಧ್ಯಾ-ತ್ಮ-ಮು-ಚ್ಯತೇ
ಸ್ವಯಂ
ಸ್ವಯಂ-ಜ್ಯೋತಿ
ಸ್ವಯಂ-ಜ್ಯೋ-ತಿಃ
ಸ್ವಯಂ-ಜ್ಯೋ-ತಿಃ-ಸ್ವ-ರೂಪ
ಸ್ವಯಂ-ಜ್ಯೋ-ತಿ-ಯನ್ನು
ಸ್ವಯಂ-ಪ್ರ-ಕಾಶ
ಸ್ವಯಂ-ಪ್ರ-ಕಾ-ಶ-ಮಾ-ನನು
ಸ್ವಯಂ-ವೇದ್ಯ
ಸ್ವಯ-ಮೇ-ವಾ-ತ್ಮ-ನಾ-ತ್ಮಾನಂ
ಸ್ವಯಮ್
ಸ್ವಯಾ
ಸ್ವರಾ-ಜ್ಯದ
ಸ್ವರೂಪ
ಸ್ವರೂ-ಪಕ್ಕಿಂತ
ಸ್ವರೂ-ಪ-ಗಳು
ಸ್ವರೂ-ಪದ
ಸ್ವರೂ-ಪ-ನ-ವನು
ಸ್ವರೂ-ಪ-ನಾದ
ಸ್ವರೂ-ಪನು
ಸ್ವರೂ-ಪ-ವನ್ನು
ಸ್ವರೂ-ಪ-ವನ್ನೇ
ಸ್ವರೂ-ಪ-ವಾ-ಗಿದೆ
ಸ್ವರೂ-ಪ-ವಾ-ದರೊ
ಸ್ವರೂ-ಪ-ವುಳ್ಳ
ಸ್ವರೂ-ಪವೆ
ಸ್ವರೂ-ಪ-ವೇನು
ಸ್ವರೂಪಿ
ಸ್ವರೂ-ಪಿ-ಯನ್ನು
ಸ್ವರ್ಗ
ಸ್ವರ್ಗಂ
ಸ್ವರ್ಗಕ್ಕೆ
ಸ್ವರ್ಗ-ಗ-ತಿಗೆ
ಸ್ವರ್ಗ-ಗ-ಳೆ-ರಡೂ
ಸ್ವರ್ಗ-ತಿಂ
ಸ್ವರ್ಗದ
ಸ್ವರ್ಗ-ದ-ಲ್ಲಾ-ಗಲಿ
ಸ್ವರ್ಗ-ದಲ್ಲಿ
ಸ್ವರ್ಗ-ದ-ಲ್ಲಿಯೇ
ಸ್ವರ್ಗ-ದ-ಲ್ಲಿ-ರುವ
ಸ್ವರ್ಗ-ದಿಂದ
ಸ್ವರ್ಗ-ದ್ವಾ-ರ-ಮ-ಪಾ-ವೃ-ತಮ್
ಸ್ವರ್ಗ-ಪರಾ
ಸ್ವರ್ಗ-ಪ್ರಾ-ಪ್ತಿ-ಯಾ-ಗುವ
ಸ್ವರ್ಗ-ಲೋಕ
ಸ್ವರ್ಗ-ಲೋಕಂ
ಸ್ವರ್ಗ-ಲೋ-ಕಕ್ಕೆ
ಸ್ವರ್ಗ-ಲೋ-ಕ-ಗಳನ್ನು
ಸ್ವರ್ಗ-ಲೋ-ಕದ
ಸ್ವರ್ಗ-ಲೋ-ಕ-ದಲ್ಲಿ
ಸ್ವರ್ಗ-ಲೋ-ಕ-ದಿಂದ
ಸ್ವರ್ಗ-ಲೋ-ಕ-ವನ್ನು
ಸ್ವರ್ಗ-ವನ್ನು
ಸ್ವರ್ಗ-ವನ್ನೇ
ಸ್ವರ್ಗವೂ
ಸ್ವರ್ಗ-ಸಿ-ಕ್ಕು-ವಂ-ತಹ
ಸ್ವರ್ಗ-ಸುಖ
ಸ್ವರ್ಗ-ಸು-ಖದ
ಸ್ವರ್ಗಾ-ಕಾಂ-ಕ್ಷಿ-ಗಳಲ್ಲಿ
ಸ್ವರ್ಗಾದಿ
ಸ್ವಲ್ಪ
ಸ್ವಲ್ಪ-ಕಾಲ
ಸ್ವಲ್ಪ-ಕಾ-ಲ-ವಾದ
ಸ್ವಲ್ಪಕ್ಕೇ
ಸ್ವಲ್ಪ-ದೂರ
ಸ್ವಲ್ಪ-ಮ-ಪ್ಯಸ್ಯ
ಸ್ವಲ್ಪ-ವನ್ನು
ಸ್ವಲ್ಪ-ವಾಗಿ
ಸ್ವಲ್ಪ-ವಾ-ದರೂ
ಸ್ವಲ್ಪ-ವಾ-ದರೆ
ಸ್ವಲ್ಪವೂ
ಸ್ವಲ್ಪ-ಹೊತ್ತಿ
ಸ್ವಲ್ಪ-ಹೊ-ತ್ತಿಗೆ
ಸ್ವಲ್ಪ-ಹೊ-ತ್ತಿ-ನಲ್ಲಿ
ಸ್ವಲ್ಪ-ಹೊ-ತ್ತಿ-ನ-ಲ್ಲಿಯೇ
ಸ್ವಲ್ಪ-ಹೊತ್ತು
ಸ್ವಲ್ಲ
ಸ್ವಸ್ತೀ-ತ್ಯುಕ್ತ್ವಾ
ಸ್ವಸ್ಥ
ಸ್ವಸ್ಥಃ
ಸ್ವಸ್ಥನು
ಸ್ವಸ್ಯಾಃ
ಸ್ವಾಗ-ತಿ-ಸುವ
ಸ್ವಾಗ-ತಿ-ಸು-ವನು
ಸ್ವಾತಂತ್ರ್ಯ
ಸ್ವಾತಂ-ತ್ರ್ಯ-ವನ್ನು
ಸ್ವಾತಂ-ತ್ರ್ಯ-ವಿದೆ
ಸ್ವಾತಂ-ತ್ರ್ಯ-ವಿ-ದ್ದರೆ
ಸ್ವಾತಂ-ತ್ರ್ಯ-ವಿ-ರು-ವು-ದ-ರಿಂದ
ಸ್ವಾತಂ-ತ್ರ್ಯ-ವಿಲ್ಲ
ಸ್ವಾತಂ-ತ್ರ್ಯ-ವೆಂ-ಬು-ದೆ-ಲ್ಲಿದೆ
ಸ್ವಾತಂ-ತ್ರ್ಯ-ವೆಲ್ಲ
ಸ್ವಾಧೀನ
ಸ್ವಾಧೀ-ನಕ್ಕೆ
ಸ್ವಾಧೀ-ನ-ದ-ಲ್ಲಿ-ಟ್ಟು-ಕೊಂ-ಡಿ-ರು-ವನೊ
ಸ್ವಾಧೀ-ನ-ದ-ಲ್ಲಿ-ಟ್ಟು-ಕೊಂಡು
ಸ್ವಾಧೀ-ನ-ವಾ-ಗಿತ್ತು
ಸ್ವಾಧೀ-ನ-ವಾ-ಗಿ-ದ್ದರೆ
ಸ್ವಾಧೀ-ನ-ವಾಗು
ಸ್ವಾಧೀ-ನ-ವಿದೆ
ಸ್ವಾಧ್ಯಾಯ
ಸ್ವಾಧ್ಯಾ-ಯ-ಜ್ಞಾ-ನ-ಯ-ಜ್ಞಾಶ್ಚ
ಸ್ವಾಧ್ಯಾ-ಯ-ಯಜ್ಞ
ಸ್ವಾಧ್ಯಾ-ಯ-ಸ್ತಪ
ಸ್ವಾಧ್ಯಾ-ಯಾ-ಭ್ಯ-ಸನಂ
ಸ್ವಾಭಾ-ವವೇ
ಸ್ವಾಭಾ-ವಿಕ
ಸ್ವಾಭಾ-ವಿ-ಕ-ವಾಗಿ
ಸ್ವಾಭಾ-ವಿ-ಕ-ವಾ-ಗಿ-ರ-ಬೇಕು
ಸ್ವಾಭಾ-ವಿ-ಕ-ವಾ-ಗಿ-ರು-ವುದು
ಸ್ವಾಭಾ-ವಿ-ಕ-ವಾ-ಗು-ವುದು
ಸ್ವಾಭಾ-ವಿ-ಕ-ವಾದ
ಸ್ವಾಭಾ-ವಿ-ಕ-ವಾ-ದುದು
ಸ್ವಾಭಾ-ವಿ-ಕವೊ
ಸ್ವಾಭಿ-ಮಾ-ನಿ-ಯಾದ
ಸ್ವಾಮ-ಧಿ-ಷ್ಠಾಯ
ಸ್ವಾಮ-ವ-ಷ್ಟಭ್ಯ
ಸ್ವಾಮಿ
ಸ್ವಾಮಿ-ತ್ವ-ವನ್ನು
ಸ್ವಾರ್ಥ
ಸ್ವಾರ್ಥ-ಕಾ-ರ್ಯ-ವನ್ನೂ
ಸ್ವಾರ್ಥ-ಕ್ಕಾಗಿ
ಸ್ವಾರ್ಥಕ್ಕೆ
ಸ್ವಾರ್ಥಕ್ಕೋ
ಸ್ವಾರ್ಥತೆ
ಸ್ವಾರ್ಥ-ತೆಗೆ
ಸ್ವಾರ್ಥ-ತೆ-ಯಿಂದ
ಸ್ವಾರ್ಥದ
ಸ್ವಾರ್ಥ-ದಲ್ಲಿ
ಸ್ವಾರ್ಥ-ದಿಂದ
ಸ್ವಾರ್ಥ-ದೃಷ್ಟಿ
ಸ್ವಾರ್ಥ-ದೃ-ಷ್ಟಿ-ಯನ್ನು
ಸ್ವಾರ್ಥ-ದೃ-ಷ್ಟಿ-ಯಿಂದ
ಸ್ವಾರ್ಥ-ನಾಗಿ
ಸ್ವಾರ್ಥನೋ
ಸ್ವಾರ್ಥ-ರಾ-ಗಿ-ರು-ವು-ದ-ಕ್ಕಿಂತ
ಸ್ವಾರ್ಥ-ವನ್ನು
ಸ್ವಾರ್ಥ-ವಾ-ಗಿ-ರು-ವುದನ್ನು
ಸ್ವಾರ್ಥ-ವಾ-ಗು-ವುದು
ಸ್ವಾರ್ಥವು
ಸ್ವಾರ್ಥವೂ
ಸ್ವಾರ್ಥವೇ
ಸ್ವಾರ್ಥವೋ
ಸ್ವಾರ್ಥಿ
ಸ್ವಾರ್ಥಿ-ಯಲ್ಲಿ
ಸ್ವಾರ್ಥಿಯೇ
ಸ್ವಾಸ್ಥ್ಯ
ಸ್ವಾಸ್ಥ್ಯಕ್ಕೆ
ಸ್ವಾಸ್ಥ್ಯ-ವನ್ನು
ಸ್ವಾಸ್ಥ್ಯ-ವಿ-ರು-ವು-ದಿಲ್ಲ
ಸ್ವಾಸ್ಥ್ಯವೂ
ಸ್ವೀಕ-ರ-ಸಿ-ಲಿ-ಲ್ಲವೆ
ಸ್ವೀಕರಿ
ಸ್ವೀಕ-ರಿಸ
ಸ್ವೀಕ-ರಿ-ಸದೆ
ಸ್ವೀಕ-ರಿ-ಸ-ಬ-ಹುದು
ಸ್ವೀಕ-ರಿ-ಸ-ಬಾ-ರದು
ಸ್ವೀಕ-ರಿ-ಸ-ಬೇ-ಕಾ-ಗಿದೆ
ಸ್ವೀಕ-ರಿ-ಸ-ಬೇ-ಕಾ-ಗು-ವುದು
ಸ್ವೀಕ-ರಿ-ಸ-ಬೇಕು
ಸ್ವೀಕ-ರಿ-ಸಲು
ಸ್ವೀಕ-ರಿ-ಸ-ಲೇ-ಬೇ-ಕಾ-ಗಿದೆ
ಸ್ವೀಕ-ರಿ-ಸ-ವುದೇ
ಸ್ವೀಕ-ರಿಸಿ
ಸ್ವೀಕ-ರಿ-ಸಿದ
ಸ್ವೀಕ-ರಿ-ಸಿ-ದರೆ
ಸ್ವೀಕ-ರಿ-ಸಿ-ದ-ರೆಷ್ಟು
ಸ್ವೀಕ-ರಿ-ಸಿ-ರು-ವರೊ
ಸ್ವೀಕ-ರಿಸು
ಸ್ವೀಕ-ರಿ-ಸು-ತ್ತಾನೆ
ಸ್ವೀಕ-ರಿ-ಸು-ತ್ತಾ-ನೆಯೇ
ಸ್ವೀಕ-ರಿ-ಸು-ತ್ತಾ-ನೆಯೋ
ಸ್ವೀಕ-ರಿ-ಸು-ತ್ತಾರೆ
ಸ್ವೀಕ-ರಿ-ಸು-ತ್ತಿ-ರು-ವನು
ಸ್ವೀಕ-ರಿ-ಸು-ತ್ತಿ-ರು-ವ-ವನೇ
ಸ್ವೀಕ-ರಿ-ಸು-ತ್ತೇನೆ
ಸ್ವೀಕ-ರಿ-ಸು-ತ್ತೇವೆ
ಸ್ವೀಕ-ರಿ-ಸು-ತ್ತೇ-ವೆಯೊ
ಸ್ವೀಕ-ರಿ-ಸು-ತ್ತೇ-ವೆಯೋ
ಸ್ವೀಕ-ರಿ-ಸು-ವನು
ಸ್ವೀಕ-ರಿ-ಸು-ವರು
ಸ್ವೀಕ-ರಿ-ಸು-ವ-ವ-ನನ್ನು
ಸ್ವೀಕ-ರಿ-ಸು-ವ-ವ-ನಿ-ಗಿಂತ
ಸ್ವೀಕ-ರಿ-ಸು-ವ-ವನು
ಸ್ವೀಕ-ರಿ-ಸು-ವ-ವ-ರಿಗೆ
ಸ್ವೀಕ-ರಿ-ಸು-ವು-ದಕ್ಕೆ
ಸ್ವೀಕ-ರಿ-ಸು-ವುದನ್ನು
ಸ್ವೀಕ-ರಿ-ಸು-ವು-ದಲ್ಲ
ಸ್ವೀಕ-ರಿ-ಸು-ವು-ದಿಲ್ಲ
ಸ್ವೀಕ-ರಿ-ಸು-ವುದು
ಸ್ವೀಕ-ರಿ-ಸು-ವುದೂ
ಸ್ವೀಕಾ-ರಕ್ಕೆ
ಸ್ವೇ
ಸ್ವೇಚ್ಛಾ
ಸ್ವೇಚ್ಛಾ-ನು-ಸಾರ
ಸ್ವೇಚ್ಛೆ-ಯಿಂದ
ಸ್ವೇನ
ಸ್ಸನ್ನು
ಸ್ಸನ್ನೇ
ಹ
ಹಂಗನ್ನು
ಹಂಗನ್ನೇ
ಹಂಗಿಗೂ
ಹಂಗಿಗೆ
ಹಂಗಿ-ಲ್ಲದೆ
ಹಂಗಿ-ಸ-ಬಾ-ರದು
ಹಂಗಿ-ಸು-ತ್ತಾನೆ
ಹಂಗಿ-ಸು-ವನು
ಹಂಗಿ-ಸು-ವು-ದಾ-ಗಲಿ
ಹಂಗು
ಹಂಗೂ
ಹಂಗೇ
ಹಂಗೇಕೆ
ಹಂಗ್ಯಾಕೊ
ಹಂಚ-ಬೇಕು
ಹಂಚಿ
ಹಂಚಿ-ಕೊಂ-ಡಾಗ
ಹಂಚಿ-ಕೊಂಡು
ಹಂಚಿ-ಕೊ-ಳ್ಳ-ಬೇ-ಕಾಗಿ
ಹಂಚಿ-ಕೊ-ಳ್ಳ-ಬೇಕು
ಹಂಚಿ-ಕೊ-ಳ್ಳು-ತ್ತಿ-ರು-ವಾಗ
ಹಂಚಿ-ಕೊ-ಳ್ಳು-ವು-ದ-ರಿಂದ
ಹಂಚಿ-ದರೆ
ಹಂಚುವ
ಹಂಚು-ವ-ವನು
ಹಂಚು-ವು-ದ-ಕ್ಕಾಗಿ
ಹಂಚು-ವು-ದಕ್ಕೆ
ಹಂಚು-ವು-ದರ
ಹಂಚು-ವು-ದ-ರಲ್ಲಿ
ಹಂಚು-ವು-ದಿ-ಲ್ಲವೊ
ಹಂಚು-ವುದು
ಹಂಡೆ
ಹಂಡೆ-ಗ-ಳಷ್ಟು
ಹಂತ
ಹಂತಕ್ಕೆ
ಹಂತ-ದ-ಲ್ಲಿ-ರು-ವೆವೊ
ಹಂತಾರಂ
ಹಂತಿ
ಹಂತುಂ
ಹಂತು-ಮಿ-ಚ್ಛಾಮಿ
ಹಂದಿಗೆ
ಹಂದಿಯ
ಹಂದಿ-ಯೊಂ-ದಿಗೆ
ಹಂಪ-ಲು-ಗಳು
ಹಂಬಲ
ಹಂಬ-ಲ-ವಿಲ್ಲ
ಹಂಸ-ತೂ-ಲಿ-ಕಾ-ತಲ್ಪ
ಹಂಸ-ತೂ-ಲಿ-ಕಾ-ತ-ಲ್ಪ-ವಿದೆ
ಹಕ್ಕನ್ನು
ಹಕ್ಕಿ
ಹಕ್ಕಿ-ಗಳಲ್ಲಿ
ಹಕ್ಕಿ-ಗ-ಳಿ-ಗಿಂತ
ಹಕ್ಕಿ-ಗ-ಳಿಗೆ
ಹಕ್ಕಿ-ಗಳು
ಹಕ್ಕಿಗೆ
ಹಕ್ಕಿದೆ
ಹಕ್ಕಿ-ಯಂತೆ
ಹಕ್ಕಿ-ಯ-ಗೂ-ಡಿ-ನಂತೆ
ಹಕ್ಕಿ-ಯನ್ನು
ಹಕ್ಕಿ-ಯಲ್ಲಿ
ಹಕ್ಕಿಯೊ
ಹಕ್ಕು
ಹಕ್ಕು-ಗಳೂ
ಹಕ್ಕು-ದಾರ
ಹಕ್ಕು-ದಾ-ರ-ನಾ-ಗು-ವನು
ಹಕ್ಕು-ಬಾ-ಧ್ಯ-ತೆ-ಗಳನ್ನು
ಹಗ-ಲಲ್ಲಿ
ಹಗ-ಲಾ-ಗು-ತ್ತಲೆ
ಹಗ-ಲಾ-ದ-ಮೇಲೆ
ಹಗ-ಲಾ-ದೊ-ಡನೆ
ಹಗ-ಲಿಗೆ
ಹಗ-ಲಿ-ನಲ್ಲಿ
ಹಗ-ಲಿ-ರುಳು
ಹಗ-ಲಿ-ರುಳೂ
ಹಗಲು
ಹಗ-ಲು-ರಾತ್ರಿ
ಹಗ-ಲು-ಹೊತ್ತು
ಹಗಲೂ
ಹಗುರ
ಹಗು-ರ-ವಾ-ಗ-ಬೇಕು
ಹಗು-ರ-ವಾ-ಗ-ವುದು
ಹಗು-ರ-ವಾಗಿ
ಹಗು-ರ-ವಾಗು
ಹಗ್ಗ
ಹಗ್ಗ-ದಿಂದ
ಹಗ್ಗ-ವನ್ನು
ಹಚ್ಚ-ಬ-ಲ್ಲನು
ಹಚ್ಚಲು
ಹಚ್ಚಿ
ಹಚ್ಚಿ-ಕೊ-ಳ್ಳು-ವನು
ಹಚ್ಚಿ-ಕೊ-ಳ್ಳು-ವು-ದಿಲ್ಲ
ಹಚ್ಚಿ-ದೊ-ಡನೆ
ಹಟ
ಹಟ-ಮಾ-ಡಿಯೇ
ಹಡ-ಗನ್ನು
ಹಡ-ಗಿನ
ಹಡಗು
ಹಡ-ಗು-ಗಳನ್ನು
ಹಣ
ಹಣಕ್ಕೆ
ಹಣಕ್ಕೋ
ಹಣತೆ
ಹಣ-ತೆ-ಯಂತೆ
ಹಣ-ತೆ-ಯನ್ನು
ಹಣದ
ಹಣ-ದಲ್ಲಿ
ಹಣ-ದಿಂದ
ಹಣ-ದಿಂ-ದಲ್ಲ
ಹಣ-ವಂತ
ಹಣ-ವನ್ನು
ಹಣ-ವನ್ನೊ
ಹಣ-ವಾ-ಗ-ಬ-ಹುದು
ಹಣ-ವಿಟ್ಟ
ಹಣ-ವಿ-ಲ್ಲ-ದ-ವ-ನಿಗೆ
ಹಣವೋ
ಹಣೆಯ
ಹಣೆ-ಯಲ್ಲಿ
ಹಣೆ-ಯಿಂದ
ಹಣ್ಣಂತೆ
ಹಣ್ಣ-ನ್ನಾ-ಗಲಿ
ಹಣ್ಣನ್ನು
ಹಣ್ಣಾ-ಗು-ವುದು
ಹಣ್ಣಾದ
ಹಣ್ಣಾ-ದಂತೆ
ಹಣ್ಣಿಗೆ
ಹಣ್ಣಿನ
ಹಣ್ಣಿ-ನಿಂದ
ಹಣ್ಣು
ಹಣ್ಣು-ಕೊ-ಡು-ವುದು
ಹಣ್ಣು-ಗಳನ್ನು
ಹಣ್ಣು-ಗಳು
ಹತ
ಹತಃ
ಹತ-ನಾ-ದನು
ಹತಮ್
ಹತ-ರಾ-ಗಿ-ದ್ದಾರೆ
ಹತ-ರಾದ
ಹತ-ರಾ-ದರು
ಹತ-ವಾ-ಗು-ತ್ತಿ-ದ್ದರೂ
ಹತ-ಸ್ವ-ಭಾ-ವ-ದ-ವ-ನಾ-ಗಿ-ದ್ದೇನೆ
ಹತಾಂಸ್ತ್ವಂ
ಹತಾ-ಶ-ನಾಗಿ
ಹತಾ-ಶ-ನಾ-ಗು-ವನು
ಹತಾ-ಶ-ನಾ-ಗು-ವು-ದಿಲ್ಲ
ಹತಾ-ಶನೂ
ಹತಾ-ಶ-ರಾ-ಗಿಯೂ
ಹತಾ-ಶ-ರಾಗು
ಹತಾ-ಶ-ರಾ-ಗು-ತ್ತೇವೆ
ಹತೋ
ಹತೋಟಿ
ಹತೋ-ಟಿ-ಯ-ಲ್ಲಿ-ರು-ವುದು
ಹತ್ತನ್ನು
ಹತ್ತ-ಬೇ-ಕಾ-ದರೆ
ಹತ್ತ-ರಲ್ಲಿ
ಹತ್ತ-ರಷ್ಟು
ಹತ್ತರಿ
ಹತ್ತಲು
ಹತ್ತಾಗಿ
ಹತ್ತಾ-ಗು-ವುದು
ಹತ್ತಿ
ಹತ್ತಿ-ಕೊಂಡು
ಹತ್ತಿ-ಕೊ-ಳ್ಳು-ವು-ದಕ್ಕೆ
ಹತ್ತಿ-ಕೊ-ಳ್ಳು-ವುದು
ಹತ್ತಿ-ಬಂದ
ಹತ್ತಿ-ಯನ್ನು
ಹತ್ತಿರ
ಹತ್ತಿ-ರಕ್ಕೆ
ಹತ್ತಿ-ರದ
ಹತ್ತಿ-ರ-ದ-ಲ್ಲಿದೆ
ಹತ್ತಿ-ರ-ದ-ಲ್ಲಿ-ರುವ
ಹತ್ತಿ-ರ-ದ-ಲ್ಲಿ-ರು-ವುದು
ಹತ್ತಿ-ರ-ದಲ್ಲೆ
ಹತ್ತಿ-ರ-ದಲ್ಲೇ
ಹತ್ತಿ-ರ-ದ-ವ-ರಾ-ಗಿ-ರ-ಬೇಕು
ಹತ್ತಿ-ರ-ದ-ವರು
ಹತ್ತಿ-ರ-ವಾ-ಗಿ-ದ್ದರೂ
ಹತ್ತಿ-ರ-ವಾ-ಗಿ-ರು-ವುದು
ಹತ್ತಿ-ರ-ವಾದ
ಹತ್ತಿ-ರವೂ
ಹತ್ತಿ-ರವೆ
ಹತ್ತಿ-ರ-ವೆಲ್ಲ
ಹತ್ತಿ-ರವೇ
ಹತ್ತಿ-ರು-ವಾಗ
ಹತ್ತಿಲ್ಲ
ಹತ್ತಿ-ಸದೆ
ಹತ್ತಿ-ಸ-ಲ್ಪ-ಟ್ಟಿತು
ಹತ್ತಿ-ಸುವ
ಹತ್ತಿ-ಸು-ವಂತೆ
ಹತ್ತಿ-ಹೋ-ಗು-ವಾಗ
ಹತ್ತು
ಹತ್ತು-ವ-ವ-ರಿ-ದ್ದಾರೆ
ಹತ್ತು-ವಾಗ
ಹತ್ತು-ವುದು
ಹತ್ತು-ವುದೇ
ಹತ್ತು-ಸಾ-ವಿರ
ಹತ್ಯ-ದಿಂದ
ಹತ್ಯೆ
ಹತ್ವಾ
ಹತ್ವಾನ
ಹತ್ವಾಪಿ
ಹತ್ವಾ-ರ್ಥ-ಕಾ-ಮಾಂಸ್ತು
ಹತ್ವೈ-ತಾ-ನಾ-ತ-ತಾ-ಯಿನಃ
ಹದ
ಹದ-ಮಾ-ಡದ
ಹದ-ವಾ-ಗಿಲ್ಲ
ಹದ-ವಾದ
ಹದಿ-ನಾ-ರ-ನೆಯ
ಹದಿ-ನಾಲ್ಕು
ಹದಿ-ನೆಂಟು
ಹದಿ-ನೇಳು
ಹದಿ-ಮೂ-ರನೆ
ಹದ್ದು
ಹನಿ
ಹನಿ-ಗಳನ್ನು
ಹನಿ-ಗೂ-ಡಿತು
ಹನಿ-ಗೂ-ಡಿ-ದರೆ
ಹನಿ-ಗೂ-ಡು-ವುದು
ಹನಿಗೆ
ಹನಿ-ಯಾಗಿ
ಹನಿ-ಯಿಂದ
ಹನಿ-ಯೊಂದು
ಹನು-ಮಂತ
ಹನು-ಮಂ-ತ-ನಾ-ದರೋ
ಹನ್ನೆ-ರ-ಡನೆ
ಹನ್ನೆ-ರಡು
ಹನ್ನೊಂ-ದನೆ
ಹನ್ನೊಂ-ದನೇ
ಹನ್ನೊಂ-ದ-ರಂತೆ
ಹನ್ನೊಂದು
ಹನ್ಯತೇ
ಹನ್ಯ-ಮಾನೇ
ಹನ್ಯು-ಸ್ತನ್ಮೇ
ಹಬ್ಬ
ಹಬ್ಬಿ
ಹಬ್ಬಿ-ರು-ವನೋ
ಹಬ್ಬಿ-ರು-ವುದು
ಹಬ್ಬಿವೆ
ಹಬ್ಬುತ್ತಾ
ಹಬ್ಬುವ
ಹಬ್ಬು-ವುದು
ಹರಂತಿ
ಹರ-ಕಲು
ಹರಕೆ
ಹರ-ಕೆಯ
ಹರ-ಕೆ-ಹೊ-ರು-ತ್ತಾನೆ
ಹರಟೆ
ಹರ-ಟೆ-ಯ-ನ್ನಲ್ಲ
ಹರ-ಡ-ಬೇ-ಕಾ-ದರೆ
ಹರ-ಡಲು
ಹರಡಿ
ಹರ-ಡಿ-ಕೊಂ-ಡಿವೆ
ಹರ-ಡಿ-ದರೆ
ಹರ-ಡಿ-ರು-ವನು
ಹರ-ಡಿವೆ
ಹರ-ಡುವ
ಹರ-ಡು-ವಂತೆ
ಹರ-ಡು-ವರು
ಹರ-ಡು-ವು-ದಕ್ಕೆ
ಹರ-ಡು-ವುದು
ಹರ-ಡು-ವುವು
ಹರಣ
ಹರ-ಣ-ವಾ-ಗಿದೆ
ಹರತಿ
ಹರ-ಸ-ಬೇಕು
ಹರಿ
ಹರಿಃ
ಹರಿ-ಕ-ಥಾ-ಸಂ-ಬೋ-ಧ-ನಾ-ಬೋ-ಧಿ-ತಮ್
ಹರಿ-ಕಥೆ
ಹರಿತ
ಹರಿ-ತ-ವಾ-ಗಿದೆ
ಹರಿ-ತ-ವಾ-ಗಿರ
ಹರಿ-ತ-ವಾ-ಗಿ-ರ-ಬೇಕು
ಹರಿ-ತ-ವಾ-ಗಿ-ರು-ವುದು
ಹರಿ-ತ-ವಾದ
ಹರಿದ
ಹರಿ-ದಂತೆ
ಹರಿ-ದಾ-ಡುವ
ಹರಿ-ದಿ-ದೆಯೇ
ಹರಿದು
ಹರಿ-ದು-ಕೊಂ-ಡಿವೆ
ಹರಿ-ದು-ಕೊಂಡು
ಹರಿ-ದು-ಬರು
ಹರಿ-ದು-ಹೋ-ಗ-ದಂತೆ
ಹರಿ-ದು-ಹೋ-ಗಲು
ಹರಿ-ದು-ಹೋಗಿ
ಹರಿ-ದು-ಹೋಗು
ಹರಿ-ದು-ಹೋ-ಗು-ತ್ತಿ-ರು-ವುದು
ಹರಿ-ದು-ಹೋ-ಗುವ
ಹರಿ-ದು-ಹೋ-ಗು-ವಂತೆ
ಹರಿ-ದು-ಹೋ-ಗು-ವು-ದಕ್ಕೆ
ಹರಿ-ದು-ಹೋ-ಗು-ವುದನ್ನು
ಹರಿ-ದು-ಹೋ-ಗು-ವುದು
ಹರಿ-ಬಿ-ಡು-ತ್ತೇ-ವೆಯೊ
ಹರಿ-ಬಿ-ಡು-ವು-ದಿಲ್ಲ
ಹರಿಯ
ಹರಿ-ಯ-ತೊ-ಡ-ಗಿ-ದರೆ
ಹರಿ-ಯ-ದಿ-ರು-ವು-ದಲ್ಲ
ಹರಿ-ಯ-ಬಿ-ಡದೆ
ಹರಿ-ಯ-ಬೇ-ಕಾ-ದರೆ
ಹರಿ-ಯ-ಬೇಕು
ಹರಿ-ಯ-ಲಾ-ರಂ-ಭಿ-ಸು-ವುದು
ಹರಿ-ಯ-ಲಾ-ರದು
ಹರಿ-ಯಲಿ
ಹರಿ-ಯಲು
ಹರಿಯು
ಹರಿ-ಯುತ್ತ
ಹರಿ-ಯು-ತ್ತಿದೆ
ಹರಿ-ಯು-ತ್ತಿ-ದ್ದರೂ
ಹರಿ-ಯು-ತ್ತಿದ್ದು
ಹರಿ-ಯು-ತ್ತಿ-ರ-ಬ-ಹುದು
ಹರಿ-ಯು-ತ್ತಿ-ರ-ಬೇಕು
ಹರಿ-ಯು-ತ್ತಿ-ರುವ
ಹರಿ-ಯು-ತ್ತಿ-ರು-ವನು
ಹರಿ-ಯು-ತ್ತಿ-ರು-ವಳು
ಹರಿ-ಯು-ತ್ತಿ-ರು-ವಾಗ
ಹರಿ-ಯು-ತ್ತಿ-ರು-ವು-ದ-ರಿಂ-ದಲೇ
ಹರಿ-ಯು-ತ್ತಿ-ರು-ವುದು
ಹರಿ-ಯು-ತ್ತಿ-ರು-ವುದೊ
ಹರಿ-ಯು-ತ್ತಿ-ರು-ವುದೋ
ಹರಿ-ಯು-ತ್ತಿಲ್ಲ
ಹರಿ-ಯುವ
ಹರಿ-ಯು-ವಂತೆ
ಹರಿ-ಯು-ವ-ವನು
ಹರಿ-ಯು-ವಾಗ
ಹರಿ-ಯು-ವು-ದಕ್ಕೆ
ಹರಿ-ಯು-ವುದನ್ನು
ಹರಿ-ಯು-ವುದು
ಹರಿ-ಯು-ವುದೊ
ಹರಿ-ಯು-ವುದೋ
ಹರಿಯೂ
ಹರಿವ
ಹರಿ-ವಂಶ
ಹರಿಸ
ಹರಿ-ಸ-ಬ-ಹುದು
ಹರಿ-ಸ-ಬೇ-ಕಾ-ಗು-ವುದು
ಹರಿ-ಸ-ಬೇ-ಕಾ-ದರೆ
ಹರಿ-ಸ-ಬೇಕು
ಹರಿಸಿ
ಹರಿ-ಸಿ-ದರೆ
ಹರಿ-ಸಿ-ರು-ವೆವು
ಹರಿ-ಸುತ್ತ
ಹರಿ-ಸು-ತ್ತಾನೆ
ಹರಿ-ಸು-ತ್ತಾನೋ
ಹರಿ-ಸು-ತ್ತಿ-ರ-ಬೇಕು
ಹರಿ-ಸು-ತ್ತಿ-ರು-ವನು
ಹರಿ-ಸು-ತ್ತಿ-ರು-ವ-ವಳು
ಹರಿ-ಸು-ತ್ತೇವೆ
ಹರಿ-ಸು-ವನು
ಹರಿ-ಸು-ವಾಗ
ಹರಿ-ಸು-ವು-ದಕ್ಕೆ
ಹರಿ-ಸು-ವುದನ್ನು
ಹರಿ-ಸು-ವು-ದಿಲ್ಲ
ಹರಿ-ಸು-ವುದು
ಹರಿ-ಸು-ವೆವೋ
ಹರು-ಡು-ತ್ತಿ-ದ್ದುದೇ
ಹರೇಃ
ಹರ್ಷ
ಹರ್ಷಂ
ಹರ್ಷ-ದಿಂದ
ಹರ್ಷ-ಪ-ಡು-ತ್ತಿ-ದ್ದೇನೆ
ಹರ್ಷ-ಪ-ಡು-ತ್ತಿ-ರು-ವೆನು
ಹರ್ಷ-ವನ್ನು
ಹರ್ಷ-ವಿಲ್ಲ
ಹರ್ಷವೂ
ಹರ್ಷ-ಶೋ-ಕಾ-ನ್ವಿತಃ
ಹರ್ಷಾ-ಮ-ರ್ಷ-ಭ-ಯೋ-ದ್ವೇ-ಗೈ-ರ್ಮುಕ್ತೋ
ಹರ್ಷಿ-ತ-ವಾ-ಗಿ-ದ್ದೇನೆ
ಹರ್ಷಿ-ಸು-ವು-ದಿ-ಲ್ಲವೊ
ಹಲ
ಹಲ-ವನ್ನು
ಹಲ-ವನ್ನೇ
ಹಲ-ವರ
ಹಲ-ವ-ರಂತೆ
ಹಲ-ವ-ರನ್ನು
ಹಲ-ವ-ರ-ಲ್ಲಿದೆ
ಹಲ-ವ-ರಿಗೆ
ಹಲ-ವರು
ಹಲ-ವಾರು
ಹಲವು
ಹಲ-ಸಿನ
ಹಲಿ-ಯ-ದೋರ
ಹಲು-ಬು-ವನು
ಹಲ್ಲನ್ನು
ಹಲ್ಲಿಗೆ
ಹಲ್ಲಿನ
ಹಲ್ಲಿ-ನಂತೆ
ಹಲ್ಲಿ-ನಲ್ಲಿ
ಹಲ್ಲು
ಹಲ್ಲು-ಜ್ಜು-ವುದು
ಹಲ್ಲು-ನೋ-ವಿನ
ಹಲ್ಲೆಲ್ಲ
ಹಳ-ತಾಗು
ಹಳ-ತಾ-ಗು-ವು-ದಿಲ್ಲ
ಹಳದಿ
ಹಳ-ಬರು
ಹಳಸಿ
ಹಳಿ-ದು-ಕೊ-ಳ್ಳು-ತ್ತೇನೆ
ಹಳಿ-ಯ-ಬೇ-ಕಾ-ಗಿಲ್ಲ
ಹಳಿ-ಯು-ವು-ದಿಲ್ಲ
ಹಳೆಯ
ಹಳೆ-ಯ-ದನ್ನು
ಹಳೆ-ಯ-ದ-ರಷ್ಟೇ
ಹಳೆ-ಯ-ದಾ-ಗದ
ಹಳೆ-ಯ-ದಾ-ಗಿದೆ
ಹಳೆ-ಯ-ದಾಗು
ಹಳೆ-ಯ-ದಾ-ಗುತ್ತ
ಹಳೆ-ಯ-ದಾ-ಗು-ವು-ದಿಲ್ಲ
ಹಳೆ-ಯ-ದಾ-ದರೆ
ಹಳೆ-ಯದು
ಹಳೆ-ಯ-ದೆಲ್ಲ
ಹಳೆ-ಯ-ವನು
ಹಳೆ-ಯ-ವರು
ಹಳೇ
ಹಳ್ಳ
ಹಳ್ಳಕ್ಕೆ
ಹಳ್ಳ-ದಲ್ಲಿ
ಹಳ್ಳ-ದಿಂದ
ಹಳ್ಳ-ದೊ-ಳಕ್ಕೆ
ಹಳ್ಳ-ವನ್ನು
ಹಳ್ಳ-ವಾಗು
ಹಳ್ಳಿ-ಯನ್ನೂ
ಹಳ್ಳಿ-ಯ-ವ-ನೂ-ನ-ಮ್ಮಮ್ಮ
ಹಳ್ಳಿ-ಯಿಂದ
ಹವ-ಣಿ-ಸು-ತ್ತಿ-ರು-ವನು
ಹವಾ-ಲನ್ನು
ಹವಿ-ರ್ಬ್ರ-ಹ್ಮಾಗ್ನೌ
ಹವಿ-ಸ್ಸನ್ನು
ಹವಿಸ್ಸು
ಹವಿ-ಸ್ಸು-ಗಳನ್ನು
ಹವ್ಯಾ-ಸ-ದಿಂದ
ಹಸ-ನ್ಮು-ಖ-ದಿಂದ
ಹಸ-ನ್ಮು-ಖಿ-ಯಾದ
ಹಸಿ
ಹಸಿ-ದರೆ
ಹಸಿ-ದಾಗ
ಹಸಿ-ಮೆ-ಣ-ಸಿನ
ಹಸಿ-ಮೆ-ಣ-ಸಿ-ನ-ಕಾಯಿ
ಹಸಿಯ
ಹಸಿ-ಯ-ದನ್ನು
ಹಸಿ-ಯ-ನ್ನೆಲ್ಲ
ಹಸಿ-ಯಾ-ಗಿ-ದ್ದರೆ
ಹಸಿ-ಯಾ-ಗಿ-ರು-ವುದು
ಹಸಿ-ಯು-ವು-ದಕ್ಕೆ
ಹಸಿ-ಯೆಲ್ಲ
ಹಸಿಯೇ
ಹಸಿ-ರಿ-ನಿಂದ
ಹಸಿರು
ಹಸಿ-ವನ್ನು
ಹಸಿ-ವಾ-ಗಿ-ರು-ವುದೋ
ಹಸಿ-ವಾ-ಯಿತು
ಹಸಿ-ವಿನ
ಹಸಿವು
ಹಸು
ಹಸು-ಗ-ಳಿಗೆ
ಹಸು-ಗಳು
ಹಸುಳೆ
ಹಸು-ಳೆ-ಗಳನ್ನು
ಹಸು-ಳೆ-ಗಳು
ಹಸು-ವನ್ನು
ಹಸು-ವಲ್ಲ
ಹಸು-ವಿನ
ಹಸು-ವಿ-ನಲ್ಲಿ
ಹಸು-ವಿ-ನ-ಲ್ಲಿದೆ
ಹಸ್ತ
ಹಸ್ತಾತ್
ಹಸ್ತಿ-ನಾ-ವತಿ
ಹಸ್ತಿ-ನಾ-ವ-ತಿಗೆ
ಹಸ್ತಿನಿ
ಹಾಕ-ದಿ-ದ್ದರೆ
ಹಾಕದೆ
ಹಾಕದೇ
ಹಾಕ-ಬ-ಹು-ದಂತೆ
ಹಾಕ-ಬ-ಹುದು
ಹಾಕ-ಬೇ-ಕಾ-ಗಿದೆ
ಹಾಕ-ಬೇ-ಕಾ-ದರೆ
ಹಾಕ-ಬೇಕು
ಹಾಕ-ಲಿ-ರು-ವನು
ಹಾಕ-ಲಿಲ್ಲ
ಹಾಕಲು
ಹಾಕಿ
ಹಾಕಿ-ಕೊಂಡ
ಹಾಕಿ-ಕೊಂ-ಡರೆ
ಹಾಕಿ-ಕೊಂ-ಡ-ವನು
ಹಾಕಿ-ಕೊಂ-ಡಾಗ
ಹಾಕಿ-ಕೊಂ-ಡಿ-ದ್ದರೂ
ಹಾಕಿ-ಕೊಂ-ಡಿ-ದ್ದೆ-ನಲ್ಲ
ಹಾಕಿ-ಕೊಂ-ಡಿರು
ಹಾಕಿ-ಕೊಂ-ಡಿ-ರುವ
ಹಾಕಿ-ಕೊಂ-ಡಿ-ರು-ವನು
ಹಾಕಿ-ಕೊಂ-ಡಿ-ರು-ವಾಗ
ಹಾಕಿ-ಕೊಂಡು
ಹಾಕಿ-ಕೊಂಡೂ
ಹಾಕಿ-ಕೊಂ-ಡೆವು
ಹಾಕಿ-ಕೊಂ-ಡೊ-ಡ-ನೆಯೇ
ಹಾಕಿ-ಕೊ-ಳ್ಳದೆ
ಹಾಕಿ-ಕೊ-ಳ್ಳ-ಬ-ಹುದು
ಹಾಕಿ-ಕೊ-ಳ್ಳ-ಬೇಕು
ಹಾಕಿ-ಕೊ-ಳ್ಳ-ವು-ದಿಲ್ಲ
ಹಾಕಿ-ಕೊಳ್ಳು
ಹಾಕಿ-ಕೊ-ಳ್ಳು-ತ್ತಾನೆ
ಹಾಕಿ-ಕೊ-ಳ್ಳು-ತ್ತಿ-ರ-ಲಿಲ್ಲ
ಹಾಕಿ-ಕೊ-ಳ್ಳು-ತ್ತೇವೆ
ಹಾಕಿ-ಕೊ-ಳ್ಳುವ
ಹಾಕಿ-ಕೊ-ಳ್ಳು-ವನು
ಹಾಕಿ-ಕೊ-ಳ್ಳು-ವು-ದಕ್ಕೆ
ಹಾಕಿ-ಕೊ-ಳ್ಳು-ವು-ದಿಲ್ಲ
ಹಾಕಿ-ಕೊ-ಳ್ಳು-ವುದೇ
ಹಾಕಿ-ಕೊ-ಳ್ಳು-ವೆನೊ
ಹಾಕಿ-ಟ್ಟರೆ
ಹಾಕಿ-ಡು-ತ್ತೇವೆ
ಹಾಕಿತು
ಹಾಕಿದ
ಹಾಕಿ-ದಂತೆ
ಹಾಕಿ-ದರು
ಹಾಕಿ-ದರೂ
ಹಾಕಿ-ದರೆ
ಹಾಕಿ-ದ-ರೆಂದು
ಹಾಕಿ-ದ-ರೇನೆ
ಹಾಕಿ-ದ-ವನ
ಹಾಕಿ-ದ-ವರು
ಹಾಕಿ-ದ-ಷ್ಟನ್ನು
ಹಾಕಿ-ದಾಗ
ಹಾಕಿ-ದಾ-ಗಲೂ
ಹಾಕಿದೆ
ಹಾಕಿದ್ದ
ಹಾಕಿ-ದ್ದರು
ಹಾಕಿ-ದ್ದರೆ
ಹಾಕಿ-ದ್ದಾರೆ
ಹಾಕಿ-ಬಿ-ಟ್ಟಿ-ರು-ವನು
ಹಾಕಿ-ರಲಿ
ಹಾಕಿ-ರುವ
ಹಾಕಿ-ರು-ವನು
ಹಾಕಿ-ರು-ವರು
ಹಾಕಿ-ರು-ವರೊ
ಹಾಕಿ-ರು-ವ-ವನು
ಹಾಕಿಲ್ಲ
ಹಾಕಿ-ಸಿ-ಕೊಂ-ಡಿ-ದ್ದರೆ
ಹಾಕಿ-ಸಿ-ಕೊ-ಳ್ಳು-ವಾಗ
ಹಾಕಿ-ಸಿ-ಕೊ-ಳ್ಳು-ವೆವು
ಹಾಕು
ಹಾಕುತ್ತ
ಹಾಕುತ್ತಾ
ಹಾಕು-ತ್ತಾನೆ
ಹಾಕು-ತ್ತಾ-ನೆಯೆ
ಹಾಕು-ತ್ತಾ-ನೆಯೋ
ಹಾಕು-ತ್ತಾರೆ
ಹಾಕು-ತ್ತಿತ್ತು
ಹಾಕು-ತ್ತಿದ್ದ
ಹಾಕು-ತ್ತಿ-ದ್ದರು
ಹಾಕು-ತ್ತಿ-ರ-ಬೇಕು
ಹಾಕು-ತ್ತಿ-ರ-ವನೆ
ಹಾಕು-ತ್ತಿರು
ಹಾಕು-ತ್ತಿ-ರು-ವಂ-ತಿದೆ
ಹಾಕು-ತ್ತಿ-ರು-ವನು
ಹಾಕು-ತ್ತಿ-ರು-ವ-ವನು
ಹಾಕು-ತ್ತಿ-ರು-ವುದು
ಹಾಕು-ತ್ತಿ-ರು-ವೆವೊ
ಹಾಕು-ತ್ತೇನೆ
ಹಾಕು-ತ್ತೇವೆ
ಹಾಕು-ತ್ತೇ-ವೆಯೋ
ಹಾಕುವ
ಹಾಕು-ವನು
ಹಾಕು-ವನೊ
ಹಾಕು-ವನೋ
ಹಾಕು-ವರು
ಹಾಕು-ವಳು
ಹಾಕು-ವ-ವ-ನಲ್ಲ
ಹಾಕು-ವ-ವ-ನಿಗೆ
ಹಾಕು-ವ-ವನು
ಹಾಕು-ವಾಗ
ಹಾಕು-ವು-ದಕ್ಕೆ
ಹಾಕು-ವುದನ್ನು
ಹಾಕು-ವು-ದಿಲ್ಲ
ಹಾಕು-ವು-ದಿ-ಲ್ಲವೋ
ಹಾಕು-ವುದು
ಹಾಕು-ವುದೆ
ಹಾಕು-ವು-ದೇನೋ
ಹಾಕು-ವುವು
ಹಾಕು-ವೆವು
ಹಾಕು-ವೆವೋ
ಹಾಕೋಣ
ಹಾಗ-ಲ-ಕಾಯಿ
ಹಾಗ-ಲ-ಕಾ-ಯಿಯ
ಹಾಗಲ್ಲ
ಹಾಗಾ
ಹಾಗಾ-ಗಿದೆ
ಹಾಗಾ-ದರೂ
ಹಾಗಾ-ದರೆ
ಹಾಗಾ-ಯಿತು
ಹಾಗಿದೆ
ಹಾಗಿ-ದ್ದರೂ
ಹಾಗಿ-ದ್ದರೆ
ಹಾಗಿ-ರ-ಬ-ಹುದು
ಹಾಗಿ-ರ-ಬೇಕು
ಹಾಗಿ-ರು-ವಾಗ
ಹಾಗಿ-ರು-ವು-ದಿಲ್ಲ
ಹಾಗಿಲ್ಲ
ಹಾಗಿ-ಲ್ಲದೆ
ಹಾಗೂ
ಹಾಗೆ
ಹಾಗೆಂದು
ಹಾಗೆಂದೆ
ಹಾಗೆ-ನ್ನ-ಬೇ-ಕಾ-ದರೆ
ಹಾಗೆ-ಮಾ-ಡಿ-ದರೆ
ಹಾಗೆಯ
ಹಾಗೆಯೆ
ಹಾಗೆಯೇ
ಹಾಗೇ
ಹಾಗೇ-ನಾ-ದರೂ
ಹಾಜರ್
ಹಾಡನ್ನು
ಹಾಡ-ಲಿ-ಲ್ಲ-ವಲ್ಲ
ಹಾಡಲು
ಹಾಡಾ-ದರೂ
ಹಾಡಿ-ಕೊಂಡು
ಹಾಡಿ-ದಂತೆ
ಹಾಡಿ-ದ-ಮೇಲೆ
ಹಾಡಿ-ದರು
ಹಾಡಿ-ದ್ದಾರೆ
ಹಾಡಿ-ರ-ಲಿಲ್ಲ
ಹಾಡಿವೆ
ಹಾಡು
ಹಾಡು-ತ್ತಾನೆ
ಹಾಡು-ತ್ತಾರೆ
ಹಾಡು-ತ್ತಿ-ದ್ದರೆ
ಹಾಡು-ತ್ತಿ-ದ್ದಾಗ
ಹಾಡು-ತ್ತಿಲ್ಲ
ಹಾಡು-ತ್ತೇನೆ
ಹಾಡು-ವಂತೆ
ಹಾಡು-ವ-ವನು
ಹಾಡುವು
ಹಾಡು-ವು-ದಕ್ಕೆ
ಹಾಡು-ವುವು
ಹಾತೊ-ರೆ-ಯುತ್ತಾ
ಹಾತೊ-ರೆ-ಯು-ತ್ತಿ-ರು-ವರು
ಹಾತೊ-ರೆ-ಯು-ವುದು
ಹಾದಿ
ಹಾದಿ-ಗಳು
ಹಾದಿಗೆ
ಹಾದಿ-ಯದು
ಹಾದಿ-ಯನ್ನು
ಹಾದಿ-ಯನ್ನೇ
ಹಾದಿ-ಯಲ್ಲಿ
ಹಾನಿ
ಹಾನಿ-ಯನ್ನು
ಹಾನಿ-ಯಲ್ಲ
ಹಾನಿ-ಯಾ-ಗದ
ಹಾನಿ-ಯಾ-ಗು-ವುದು
ಹಾನಿ-ಯಾ-ದೀತು
ಹಾನಿ-ಯಿಲ್ಲ
ಹಾನಿಯೂ
ಹಾನಿಯೇ
ಹಾನಿಯೋ
ಹಾನಿ-ರ-ಸ್ಯೋ-ಪ-ಜಾ-ಯತೇ
ಹಾಯಾಗಿ
ಹಾಯಾ-ಗಿದೆ
ಹಾಯಾ-ಗಿರ
ಹಾಯಾ-ಗಿರು
ಹಾಯಿ-ಸಿ-ದರೆ
ಹಾಯಿ-ಸು-ವು-ದಿಲ್ಲ
ಹಾರ
ಹಾರ-ಬ-ಹುದು
ಹಾರಲು
ಹಾರ-ವನ್ನು
ಹಾರಾಡಿ
ಹಾರಾಡು
ಹಾರಾ-ಡು-ತ್ತದೆ
ಹಾರಾ-ಡು-ತ್ತಿ-ರು-ತ್ತವೆ
ಹಾರಾ-ಡು-ತ್ತಿ-ರುವ
ಹಾರಾ-ಡು-ತ್ತಿ-ರು-ವು-ದಲ್ಲ
ಹಾರಾ-ಡು-ತ್ತಿ-ರು-ವುದು
ಹಾರಿ
ಹಾರಿಕ
ಹಾರಿ-ಬಂದು
ಹಾರಿ-ಸಿ-ದಂತೆ
ಹಾರಿ-ಸು-ವರು
ಹಾರಿ-ಹೋ-ಗದೆ
ಹಾರಿ-ಹೋ-ಗ-ಬೇಕು
ಹಾರಿ-ಹೋ-ಗ-ಬೇ-ಕೆಂದು
ಹಾರಿ-ಹೋಗಿ
ಹಾರಿ-ಹೋ-ಗಿದೆ
ಹಾರಿ-ಹೋ-ಗು-ತ್ತಿದೆ
ಹಾರಿ-ಹೋ-ಗು-ವನು
ಹಾರಿ-ಹೋ-ಗು-ವು-ದಕ್ಕೆ
ಹಾರಿ-ಹೋ-ಗು-ವುದು
ಹಾರಿ-ಹೋ-ದ-ವನು
ಹಾರಿ-ಹೋ-ಯಿತು
ಹಾರು-ತ್ತಾನೆ
ಹಾರು-ತ್ತಿ-ರು-ವಂತೆ
ಹಾರು-ತ್ತಿ-ರು-ವುದು
ಹಾರುವ
ಹಾರು-ವನು
ಹಾರು-ವುದು
ಹಾರು-ವುವು
ಹಾಲನ್ನು
ಹಾಲಿಗೂ
ಹಾಲಿಗೆ
ಹಾಲಿದೆ
ಹಾಲಿನ
ಹಾಲಿ-ನ-ದಲ್ಲ
ಹಾಲಿ-ನಲ್ಲಿ
ಹಾಲಿ-ನ-ಲ್ಲಿದೆ
ಹಾಲಿ-ನ-ಲ್ಲಿ-ದೆಯೋ
ಹಾಲಿ-ನ-ವಳ
ಹಾಲಿ-ರುವ
ಹಾಲು
ಹಾಲೂ
ಹಾಲೇ
ಹಾಲೇನೊ
ಹಾಲೇನೋ
ಹಾಳಾ-ಗದ
ಹಾಳಾಗಿ
ಹಾಳಾ-ಗಿ-ರು-ವುದು
ಹಾಳಾ-ಗಿ-ರು-ವೆವೆ
ಹಾಳಾ-ಗು-ತ್ತಾನೆ
ಹಾಳಾ-ಗು-ತ್ತೀಯೆ
ಹಾಳಾ-ಗು-ತ್ತೇವೆ
ಹಾಳಾ-ಗು-ವರೆ
ಹಾಳಾ-ಗು-ವು-ದಕ್ಕೆ
ಹಾಳಾ-ಗು-ವು-ದ-ಲ್ಲದೆ
ಹಾಳಾ-ಗು-ವುದು
ಹಾಳಾ-ಗು-ವುದೋ
ಹಾಳಾ-ದರೂ
ಹಾಳಾ-ದ-ವರು
ಹಾಳಾ-ದ-ವ-ರೆಂದೂ
ಹಾಳಾ-ಯಿತು
ಹಾಳು
ಹಾಳು-ಮಾ-ಡ-ಬೇಕು
ಹಾಳು-ಮಾ-ಡಿ-ಕೊಂ-ಡು-ಬಿ-ಡ-ಬ-ಹುದು
ಹಾಳು-ಮಾ-ಡಿ-ಕೊಂ-ಡೆ-ಯಲ್ಲ
ಹಾಳು-ಮಾ-ಡಿ-ಕೊ-ಳ್ಳ-ಬ-ಹುದು
ಹಾಳು-ಮಾ-ಡಿ-ಕೊ-ಳ್ಳುವ
ಹಾಳು-ಮಾ-ಡಿ-ಕೊ-ಳ್ಳು-ವನು
ಹಾಳು-ಮಾ-ಡಿ-ಕೊ-ಳ್ಳು-ವುದು
ಹಾಳು-ಮಾ-ಡಿ-ದಂತೆ
ಹಾಳು-ಮಾ-ಡು-ವು-ದಕ್ಕೆ
ಹಾಳು-ಮಾ-ಡು-ವುದು
ಹಾಳು-ಮಾ-ಡು-ವೆವು
ಹಾಳೂ-ರಲ್ಲ
ಹಾಳೂ-ರಿಗೆ
ಹಾವನ್ನು
ಹಾವಳಿ
ಹಾವಾ-ಡಿಸು
ಹಾವಾ-ಡಿ-ಸು-ವ-ವನು
ಹಾವಿಗೆ
ಹಾವಿನ
ಹಾವಿ-ನಂತೆ
ಹಾವಿ-ನೊ-ಡನೆ
ಹಾವು
ಹಾವು-ಗಳನ್ನೂ
ಹಾವೇ
ಹಾಸಿ
ಹಾಸಿಗೆ
ಹಾಸು
ಹಾಸು-ಹೊ-ಕ್ಕಾಗಿ
ಹಾಸು-ಹೊ-ಕ್ಕಾ-ಗಿ-ದ್ದಾನೆ
ಹಾಸು-ಹೊ-ಕ್ಕಾ-ಗಿ-ರುವ
ಹಾಸು-ಹೊ-ಕ್ಕಾ-ಗಿ-ರು-ವನು
ಹಾಸು-ಹೋ-ಕ್ಕಾ-ಗಿ-ರು-ವನು
ಹಾಸ್ಯ
ಹಾಸ್ಯ-ವಾಗಿ
ಹಾಸ್ಯಾ-ಸ್ಪ-ದ-ವಾಗಿ
ಹಾಸ್ಯಾ-ಸ್ಪ-ದ-ವಾ-ಗು-ವುದು
ಹಾಸ್ಯಾ-ಸ್ಪ-ದ-ವಾ-ಗು-ವುದೋ
ಹಾಹಾ-ಕಾರ
ಹಿ
ಹಿಂಗಿ
ಹಿಂಗಿಲ್ಲ
ಹಿಂಗು-ವುದು
ಹಿಂಜ-ರಿದ
ಹಿಂಜ-ರಿ-ದರೆ
ಹಿಂಜ-ರಿದೆ
ಹಿಂಡ
ಹಿಂಡ-ಬೇಕು
ಹಿಂಡಿ-ದಂ-ತಿದೆ
ಹಿಂಡಿ-ದರೂ
ಹಿಂಡಿ-ದರೆ
ಹಿಂಡಿ-ದಾಗ
ಹಿಂಡುವ
ಹಿಂಡು-ವುದು
ಹಿಂಡು-ವುವೋ
ಹಿಂತಿ-ರು-ಗದ
ಹಿಂತಿ-ರು-ಗ-ದಿ-ರು-ವಿ-ಕೆ-ಯನ್ನು
ಹಿಂತಿ-ರು-ಗ-ಬೇಕು
ಹಿಂತಿ-ರು-ಗ-ಲಾ-ಗು-ವು-ದಿ-ಲ್ಲವೊ
ಹಿಂತಿ-ರು-ಗ-ಲಾ-ರದು
ಹಿಂತಿ-ರುಗಿ
ಹಿಂತಿ-ರು-ಗಿ-ದರೂ
ಹಿಂತಿ-ರು-ಗಿ-ದರೆ
ಹಿಂತಿ-ರು-ಗಿ-ಸ-ಬೇಕು
ಹಿಂತಿ-ರು-ಗಿ-ಸು-ತ್ತಾರೆ
ಹಿಂತಿ-ರುಗು
ಹಿಂತಿ-ರು-ಗು-ತ್ತಾನೆ
ಹಿಂತಿ-ರು-ಗು-ತ್ತಾರೆ
ಹಿಂತಿ-ರು-ಗು-ವು-ದಿಲ್ಲ
ಹಿಂತಿ-ರು-ಗು-ವು-ದಿ-ಲ್ಲವೋ
ಹಿಂತಿ-ರು-ಗು-ವುದು
ಹಿಂತೆ-ಗೆ-ಯದೆ
ಹಿಂದಕ್ಕೆ
ಹಿಂದಿ-ಗಿಂತ
ಹಿಂದಿದೆ
ಹಿಂದಿ-ದ್ದರೂ
ಹಿಂದಿ-ದ್ದು-ದನ್ನೇ
ಹಿಂದಿನ
ಹಿಂದಿ-ನಂತೆ
ಹಿಂದಿ-ನಂ-ತೆಯೇ
ಹಿಂದಿ-ನ-ದಕ್ಕೂ
ಹಿಂದಿ-ನ-ದಕ್ಕೆ
ಹಿಂದಿ-ನ-ದನ್ನು
ಹಿಂದಿ-ನ-ದ-ನ್ನೆಲ್ಲಾ
ಹಿಂದಿ-ನ-ದ-ರಿಂದ
ಹಿಂದಿ-ನದು
ಹಿಂದಿ-ನ-ದೆಲ್ಲ
ಹಿಂದಿ-ನ-ವರ
ಹಿಂದಿ-ನ-ವರು
ಹಿಂದಿ-ನಿಂದ
ಹಿಂದಿ-ನಿಂ-ದಲೂ
ಹಿಂದಿ-ರುಗಿ
ಹಿಂದಿ-ರು-ಗಿಸಿ
ಹಿಂದಿ-ರು-ಗು-ತ್ತವೆ
ಹಿಂದಿ-ರು-ಗು-ತ್ತಿ-ದ್ದಾಗ
ಹಿಂದಿ-ರು-ಗು-ವುದು
ಹಿಂದಿ-ರುವ
ಹಿಂದಿ-ರು-ವ-ವರು
ಹಿಂದಿ-ರು-ವುದು
ಹಿಂದಿಲ್ಲ
ಹಿಂದು
ಹಿಂದು-ಗಡೆ
ಹಿಂದು-ಗ-ಡೆಯೂ
ಹಿಂದು-ಗ-ಡೆಯೇ
ಹಿಂದು-ಗಳ
ಹಿಂದು-ಗ-ಳ-ಲ್ಲೆಲ್ಲ
ಹಿಂದು-ಗ-ಳಿಗೆ
ಹಿಂದು-ಧರ್ಮ
ಹಿಂದುವೂ
ಹಿಂದೂ
ಹಿಂದೂ-ಗಳ
ಹಿಂದೂ-ಗಳಲ್ಲಿ
ಹಿಂದೂ-ಗ-ಳಾದ
ಹಿಂದೂ-ಗ-ಳಾ-ದರೋ
ಹಿಂದೂ-ಗ-ಳಿಗೂ
ಹಿಂದೂ-ಗ-ಳಿಗೆ
ಹಿಂದೂ-ಗಳು
ಹಿಂದೂ-ಧ-ರ್ಮ-ದ-ಲ್ಲಿ-ರುವ
ಹಿಂದೂ-ವಿಗೂ
ಹಿಂದೂ-ವಿ-ನಂತೆ
ಹಿಂದೆ
ಹಿಂದೆ-ಗೆ-ಯು-ವ-ವರು
ಹಿಂದೆ-ಗೆ-ಯು-ವು-ದಿಲ್ಲ
ಹಿಂದೆ-ಮುಂದೆ
ಹಿಂದೆಯೂ
ಹಿಂದೆಯೆ
ಹಿಂದೆಯೇ
ಹಿಂದೆಲ್ಲ
ಹಿಂದೆಲ್ಲಾ
ಹಿಂಬ-ದಿ-ಯಲ್ಲಿ
ಹಿಂಬ-ದಿ-ಯ-ಲ್ಲಿ-ಇ-ರು-ತ್ತವೆ
ಹಿಂಬ-ದಿ-ಯ-ಲ್ಲಿದೆ
ಹಿಂಬ-ದಿ-ಯ-ಲ್ಲಿ-ರು-ತ್ತವೆ
ಹಿಂಬ-ದಿ-ಯಾಗಿ
ಹಿಂಬಾ-ಲಿ-ಸಿದ
ಹಿಂಬಾ-ಲಿ-ಸು-ವುದು
ಹಿಂಸಾ
ಹಿಂಸಾ-ತ್ಮ-ಕೋ-ಽಶು-ಚಿಃ
ಹಿಂಸಾ-ಮ-ನ-ವೇಕ್ಷ್ಯ
ಹಿಂಸಿ-ಸಿ-ದರೂ
ಹಿಂಸಿ-ಸು-ತ್ತಿದ್ದ
ಹಿಂಸೆ
ಹಿಂಸೆ-ಗಿಂತ
ಹಿಂಸೆಗೆ
ಹಿಂಸೆ-ಯನ್ನು
ಹಿಂಸೆ-ಯಿಂದ
ಹಿಗ್ಗ-ಕೂ-ಡದು
ಹಿಗ್ಗಿ-ಬಿ-ಡು-ವು-ದಿಲ್ಲ
ಹಿಗ್ಗು
ಹಿಗ್ಗು-ವು-ದಿಲ್ಲ
ಹಿಗ್ಗು-ವುದೂ
ಹಿಟ್ಟಾದ
ಹಿಟ್ಟು
ಹಿಡಿ
ಹಿಡಿ-ತಕ್ಕೆ
ಹಿಡಿ-ತಕ್ಕೇ
ಹಿಡಿ-ತ-ದಲ್ಲಿ
ಹಿಡಿ-ತ-ದಿಂದ
ಹಿಡಿದ
ಹಿಡಿ-ದರೂ
ಹಿಡಿ-ದರೆ
ಹಿಡಿ-ದ-ವ-ನನ್ನು
ಹಿಡಿ-ದ-ವ-ನಿಗೆ
ಹಿಡಿ-ದ-ವನು
ಹಿಡಿ-ದ-ವರ
ಹಿಡಿ-ದಾಗ
ಹಿಡಿ-ದಿ-ಟ್ಟಿ-ರು-ವರು
ಹಿಡಿ-ದಿ-ಟ್ಟು-ಕೊ-ಳ್ಳು-ತ್ತೇವೆ
ಹಿಡಿ-ದಿದೆ
ಹಿಡಿ-ದಿದ್ದ
ಹಿಡಿ-ದಿ-ದ್ದನೊ
ಹಿಡಿ-ದಿ-ದ್ದರೂ
ಹಿಡಿ-ದಿ-ರ-ಬ-ಹುದು
ಹಿಡಿ-ದಿ-ರ-ಬೇಕು
ಹಿಡಿ-ದಿ-ರುವ
ಹಿಡಿ-ದಿ-ರು-ವನು
ಹಿಡಿ-ದಿ-ರು-ವರು
ಹಿಡಿ-ದಿ-ರು-ವರೆ
ಹಿಡಿ-ದಿ-ರು-ವೆವು
ಹಿಡಿ-ದಿವೆ
ಹಿಡಿ-ದೀತು
ಹಿಡಿದು
ಹಿಡಿ-ದುಕೊ
ಹಿಡಿ-ದು-ಕೊಂಡ
ಹಿಡಿ-ದು-ಕೊಂ-ಡರೆ
ಹಿಡಿ-ದು-ಕೊಂ-ಡ-ರೇನೇ
ಹಿಡಿ-ದು-ಕೊಂ-ಡ-ವನು
ಹಿಡಿ-ದು-ಕೊಂ-ಡ-ವರು
ಹಿಡಿ-ದು-ಕೊಂಡಿ
ಹಿಡಿ-ದು-ಕೊಂ-ಡಿರು
ಹಿಡಿ-ದು-ಕೊಂ-ಡಿ-ರು-ತ್ತಾನೆ
ಹಿಡಿ-ದು-ಕೊಂ-ಡಿ-ರು-ವನು
ಹಿಡಿ-ದು-ಕೊಂ-ಡಿ-ರು-ವನೊ
ಹಿಡಿ-ದು-ಕೊಂ-ಡಿ-ರು-ವನೋ
ಹಿಡಿ-ದು-ಕೊಂಡು
ಹಿಡಿ-ದು-ಕೊಂ-ಡು-ಬಿ-ಡು-ವರು
ಹಿಡಿ-ದು-ಕೊ-ಳ್ಳ-ಬ-ಲ್ಲನೊ
ಹಿಡಿ-ದು-ಕೊ-ಳ್ಳ-ಬೇ-ಕಾ-ಗಿದೆ
ಹಿಡಿ-ದು-ಕೊ-ಳ್ಳ-ಬೇಕು
ಹಿಡಿ-ದು-ಕೊ-ಳ್ಳು-ತ್ತಾನೆ
ಹಿಡಿ-ದು-ಕೊ-ಳ್ಳು-ತ್ತೇವೆ
ಹಿಡಿ-ದು-ಕೊ-ಳ್ಳುವ
ಹಿಡಿ-ದು-ಕೊ-ಳ್ಳು-ವನು
ಹಿಡಿ-ದು-ಕೊ-ಳ್ಳು-ವು-ದಕ್ಕೆ
ಹಿಡಿ-ದು-ಕೊ-ಳ್ಳು-ವುದು
ಹಿಡಿ-ದು-ದನ್ನು
ಹಿಡಿ-ದು-ಹೋ-ಗಿಲ್ಲ
ಹಿಡಿ-ದು-ಹೋ-ಗು-ವು-ದ-ಕ್ಕಿಂತ
ಹಿಡಿ-ದು-ಹೋ-ಗು-ವು-ದಿಲ್ಲ
ಹಿಡಿಯ
ಹಿಡಿ-ಯದೆ
ಹಿಡಿ-ಯದೇ
ಹಿಡಿ-ಯ-ಬ-ಲ್ಲದೆ
ಹಿಡಿ-ಯ-ಬ-ಲ್ಲುವು
ಹಿಡಿ-ಯ-ಬ-ಹುದು
ಹಿಡಿ-ಯ-ಬಾ-ರದು
ಹಿಡಿ-ಯ-ಬಾ-ರದೊ
ಹಿಡಿ-ಯ-ಬೇ-ಕಷ್ಟೆ
ಹಿಡಿ-ಯ-ಬೇ-ಕಾ-ಗಿದೆ
ಹಿಡಿ-ಯ-ಬೇ-ಕಾ-ದರೆ
ಹಿಡಿ-ಯ-ಬೇಕು
ಹಿಡಿ-ಯ-ಬೇ-ಕೆಂದು
ಹಿಡಿ-ಯ-ಬೇ-ಕೆಂ-ಬುದೇ
ಹಿಡಿ-ಯ-ಲಾ-ಗು-ವು-ದಿಲ್ಲ
ಹಿಡಿ-ಯ-ಲಾ-ರದು
ಹಿಡಿ-ಯಲಿ
ಹಿಡಿ-ಯ-ಲಿಲ್ಲ
ಹಿಡಿ-ಯಲು
ಹಿಡಿ-ಯ-ಲ್ಪಟ್ಟ
ಹಿಡಿ-ಯ-ಲ್ಪ-ಟ್ಟಿವೆ
ಹಿಡಿಯು
ಹಿಡಿ-ಯು-ತ್ತಾನೆ
ಹಿಡಿ-ಯು-ತ್ತಾ-ನೆಯೋ
ಹಿಡಿ-ಯು-ತ್ತಾರೆ
ಹಿಡಿ-ಯು-ತ್ತಿರ
ಹಿಡಿ-ಯು-ತ್ತೇನೆ
ಹಿಡಿ-ಯು-ತ್ತೇವೆ
ಹಿಡಿ-ಯುವ
ಹಿಡಿ-ಯು-ವಂತೆ
ಹಿಡಿ-ಯು-ವನು
ಹಿಡಿ-ಯು-ವರು
ಹಿಡಿ-ಯು-ವ-ವ-ರಲ್ಲಿ
ಹಿಡಿ-ಯು-ವ-ವರು
ಹಿಡಿ-ಯು-ವು-ದ-ಕ್ಕಾಗಿ
ಹಿಡಿ-ಯು-ವು-ದಕ್ಕೆ
ಹಿಡಿ-ಯು-ವು-ದಿಲ್ಲ
ಹಿಡಿ-ಯು-ವುದು
ಹಿಡಿ-ಯು-ವು-ದೆಲ್ಲಾ
ಹಿಡಿ-ಯು-ವುವು
ಹಿಡಿ-ಯೋಣ
ಹಿಡಿ-ಸ-ಲಾ-ರದು
ಹಿಡಿ-ಸು-ವು-ದಿಲ್ಲ
ಹಿಡಿ-ಸು-ವುದು
ಹಿಡಿ-ಸು-ವುದೊ
ಹಿಡಿ-ಸು-ವುದೋ
ಹಿತ-ಕ-ರ-ವಾ-ಗಿ-ದ್ದರೆ
ಹಿತ-ಕ-ರ-ವಾ-ಗಿರು
ಹಿತ-ಕಾ-ಮ್ಯಯಾ
ಹಿತ-ಕಾ-ರಿಯೂ
ಹಿತ-ಕ್ಕಾಗಿ
ಹಿತಕ್ಕೆ
ಹಿತ-ಕ್ಕೋ-ಸ್ಕರ
ಹಿತದ
ಹಿತ-ದಲ್ಲಿ
ಹಿತ-ದಲ್ಲೂ
ಹಿತಮ್
ಹಿತ-ವ-ಚನ
ಹಿತ-ವ-ಚ-ನ-ವನ್ನು
ಹಿತ-ವನ್ನು
ಹಿತ-ವಾ-ಗ-ಬ-ಹುದು
ಹಿತ-ವಾಗಿ
ಹಿತ-ವಾ-ಗು-ವುದು
ಹಿತ-ವಾದ
ಹಿತವೂ
ಹಿತೈಷಿ
ಹಿತೈ-ಷಿ-ಯಾ-ಗಿ-ದ್ದರೆ
ಹಿತ್ತ-ಲಿ-ನಲ್ಲಿ
ಹಿತ್ತಾಳೆ
ಹಿತ್ವಾ
ಹಿನ-ಸ್ತ್ಯಾ-ತ್ಮ-ನಾ-ತ್ಮಾನಂ
ಹಿನ್ನೆಲೆ
ಹಿನ್ನೆ-ಲೆಯ
ಹಿನ್ನೆ-ಲೆ-ಯಂತೆ
ಹಿನ್ನೆ-ಲೆ-ಯನ್ನು
ಹಿನ್ನೆ-ಲೆ-ಯಾ-ಗಿ-ರ-ಬೇಕು
ಹಿನ್ನೆ-ಲೆಯೆ
ಹಿಮ
ಹಿಮ-ದಂತೆ
ಹಿಮ-ದ-ಗಡ್ಡೆ
ಹಿಮ-ಮ-ಣಿ-ಗ-ಳ-ಲ್ಲಿಯೂ
ಹಿಮ-ಮ-ಣಿ-ಗಳು
ಹಿಮ-ಮ-ಣಿ-ಯಲ್ಲಿ
ಹಿಮ-ಮ-ಣಿಯೂ
ಹಿಮ-ರಾಶಿ
ಹಿಮ-ರಾ-ಶಿ-ಯಾಗಿ
ಹಿಮಾ
ಹಿಮಾ-ಲಯ
ಹಿಮಾ-ಲಯಃ
ಹಿಮಾ-ಲ-ಯದ
ಹಿಮಾ-ಲ-ಯ-ದಲ್ಲಿ
ಹಿಮಾ-ಲ-ಯವೇ
ಹಿಮ್ಮೆ-ಟ್ಟಿದೆ
ಹಿರ-ಣ್ಯ-ಕ-ಶಿ-ಪು-ಗಳು
ಹಿರ-ಣ್ಯ-ಕ-ಷ್ಯ-ಪ-ನೆಂಬ
ಹಿರ-ಣ್ಯಾಕ್ಷ
ಹಿರಿ-ಮೆಯ
ಹಿರಿ-ಮೆ-ಯನ್ನು
ಹಿರಿಯ
ಹಿರಿ-ಯ-ನಿ-ರ-ಬ-ಹುದು
ಹಿರಿ-ಯರ
ಹಿರಿ-ಯ-ರನ್ನು
ಹಿರಿ-ಯ-ರಾ-ಗಿ-ರ-ಬ-ಹುದು
ಹಿರಿ-ಯ-ರಾ-ಗಿ-ರಲೀ
ಹಿರಿ-ಯ-ರಿಂದ
ಹಿರಿ-ಯ-ರಿಗೆ
ಹಿರಿ-ಯರು
ಹಿರಿ-ಯಾ-ನಂದ
ಹೀಗಲ್ಲ
ಹೀಗಾ-ಗು-ವುದು
ಹೀಗಾ-ದರೆ
ಹೀಗಾ-ದಾಗ
ಹೀಗಿದೆ
ಹೀಗಿ-ದ್ದರೆ
ಹೀಗಿ-ರ-ಬ-ಹುದು
ಹೀಗಿ-ರು-ವಾಗ
ಹೀಗಿವೆ
ಹೀಗೆ
ಹೀಗೆಂದು
ಹೀಗೆಂದೆ
ಹೀಗೆಯೆ
ಹೀಗೆಯೇ
ಹೀಗೇ
ಹೀಚೂ
ಹೀನ
ಹೀನ-ಕಳೆ
ಹೀನ-ಕೃ-ತ್ಯ-ಗಳ
ಹೀನ-ಬು-ದ್ಧಿ-ಯು-ಳ್ಳ-ವನು
ಹೀನರ
ಹೀನ-ಲೋ-ಕಕ್ಕೆ
ಹೀನ-ಲೋಹ
ಹೀನ-ವಾದ
ಹೀನ-ವಾ-ಸನೆ
ಹೀನ-ವಾ-ಸ-ನೆ-ಗಳೇ
ಹೀನ-ವೃ-ತ್ತಿ-ಗಳನ್ನು
ಹೀನ-ಸಂ-ಸ್ಕಾ-ರ-ಗ-ಳೆಲ್ಲ
ಹೀನ-ಸ್ಥಿ-ತಿಗೆ
ಹೀಯಾಳಿ
ಹೀರ-ಬ-ಹುದು
ಹೀರಲು
ಹೀರಿ
ಹೀರಿ-ಕೊಂಡು
ಹೀರಿ-ಕೊಳ್ಳು
ಹೀರಿ-ಕೊ-ಳ್ಳು-ತ್ತಾನೆ
ಹೀರಿ-ಕೊ-ಳ್ಳು-ತ್ತಿ-ರ-ಬೇಕು
ಹೀರಿ-ಕೊ-ಳ್ಳು-ವುದು
ಹೀರಿ-ಕೊ-ಳ್ಳು-ವುದೊ
ಹೀರಿ-ಕೊ-ಳ್ಳು-ವುದೋ
ಹೀರಿ-ದ-ವರು
ಹೀರಿ-ಬಿ-ಡು-ವುದು
ಹೀರು-ತ್ತದೆ
ಹೀರು-ತ್ತಾನೆ
ಹೀರು-ತ್ತಿದೆ
ಹೀರು-ತ್ತಿ-ರ-ಬ-ಹುದು
ಹೀರು-ತ್ತಿ-ರು-ವನು
ಹೀರು-ತ್ತಿ-ರು-ವರು
ಹೀರು-ತ್ತಿ-ರು-ವೆವು
ಹೀರು-ವು-ದ-ರಲ್ಲಿ
ಹೀರು-ವುದು
ಹೀರು-ವುವು
ಹುಚ್ಚಂತೂ
ಹುಚ್ಚ-ನಾ-ಗು-ತ್ತಾನೆ
ಹುಚ್ಚ-ನಾ-ಗುವ
ಹುಚ್ಚ-ನೊ-ಬ್ಬನೆ
ಹುಚ್ಚನೋ
ಹುಚ್ಚರ
ಹುಚ್ಚ-ರಾಗಿ
ಹುಚ್ಚು
ಹುಟ್ಟ-ದ-ವನು
ಹುಟ್ಟ-ದಿ-ರಲಿ
ಹುಟ್ಟ-ಬ-ಹುದು
ಹುಟ್ಟ-ಬೇ-ಕಾ-ಗಿದೆ
ಹುಟ್ಟ-ಬೇ-ಕಾ-ಗಿಲ್ಲ
ಹುಟ್ಟ-ಬೇ-ಕಾ-ದರೆ
ಹುಟ್ಟ-ಬೇಕು
ಹುಟ್ಟ-ಲಾ-ರವು
ಹುಟ್ಟಲೂ
ಹುಟ್ಟ-ವಾಗ
ಹುಟ್ಟ-ವಾ-ಗಲೇ
ಹುಟ್ಟಿ
ಹುಟ್ಟಿ-ಕೊಂ-ಡಿವೆ
ಹುಟ್ಟಿ-ಕೊ-ಳ್ಳು-ವುದು
ಹುಟ್ಟಿತು
ಹುಟ್ಟಿದ
ಹುಟ್ಟಿ-ದಂತೆ
ಹುಟ್ಟಿ-ದಂ-ತೆಯೇ
ಹುಟ್ಟಿ-ದನು
ಹುಟ್ಟಿ-ದರೆ
ಹುಟ್ಟಿ-ದ-ವ-ನಲ್ಲ
ಹುಟ್ಟಿ-ದ-ವ-ನಿಗೆ
ಹುಟ್ಟಿ-ದ-ವನು
ಹುಟ್ಟಿ-ದ-ವ-ನೇನೊ
ಹುಟ್ಟಿ-ದ-ವರ
ಹುಟ್ಟಿ-ದ-ವ-ರಾರೂ
ಹುಟ್ಟಿ-ದವು
ಹುಟ್ಟಿ-ದ-ವು-ಗಳು
ಹುಟ್ಟಿ-ದಾಗ
ಹುಟ್ಟಿ-ದಾ-ರಭ್ಯ
ಹುಟ್ಟಿ-ದು-ದಕ್ಕೆ
ಹುಟ್ಟಿದೆ
ಹುಟ್ಟಿ-ದೆವು
ಹುಟ್ಟಿ-ದೊ-ಡನೆ
ಹುಟ್ಟಿ-ದ್ದಕ್ಕೆ
ಹುಟ್ಟಿ-ದ್ದಕ್ಕೇ
ಹುಟ್ಟಿ-ದ್ದಲ್ಲ
ಹುಟ್ಟಿ-ದ್ದಾನೆ
ಹುಟ್ಟಿ-ದ್ದೀಯೆ
ಹುಟ್ಟಿದ್ದು
ಹುಟ್ಟಿ-ನಿಂದ
ಹುಟ್ಟಿಯೇ
ಹುಟ್ಟಿರು
ಹುಟ್ಟಿ-ರುವ
ಹುಟ್ಟಿ-ರು-ವನು
ಹುಟ್ಟಿ-ರು-ವರೊ
ಹುಟ್ಟಿ-ರು-ವ-ವನು
ಹುಟ್ಟಿ-ರು-ವಾಗ
ಹುಟ್ಟಿ-ರು-ವುದು
ಹುಟ್ಟಿ-ರುವೆ
ಹುಟ್ಟಿ-ರು-ವೆವು
ಹುಟ್ಟಿ-ರು-ವೆವೋ
ಹುಟ್ಟಿಲ್ಲ
ಹುಟ್ಟಿ-ಸ-ಬಲ್ಲ
ಹುಟ್ಟಿಸಿ
ಹುಟ್ಟಿ-ಸಿ-ಕೊ-ಳ್ಳ-ಬೇ-ಕಾ-ಗಿದೆ
ಹುಟ್ಟಿ-ಸಿ-ಕೊ-ಳ್ಳ-ಬೇಕು
ಹುಟ್ಟಿ-ಸಿ-ಕೊ-ಳ್ಳು-ವನು
ಹುಟ್ಟಿ-ಸು-ತ್ತಿದೆ
ಹುಟ್ಟಿ-ಸು-ವಂ-ತಿಲ್ಲ
ಹುಟ್ಟಿ-ಸು-ವ-ವನು
ಹುಟ್ಟಿ-ಸು-ವುದು
ಹುಟ್ಟಿ-ಹಾ-ಕಿ-ದರೂ
ಹುಟ್ಟಿ-ಹಾ-ಕಿ-ದರೆ
ಹುಟ್ಟು
ಹುಟ್ಟು-ಸಾವು
ಹುಟ್ಟು-ತ್ತದೆ
ಹುಟ್ಟು-ತ್ತಲೂ
ಹುಟ್ಟು-ತ್ತಲೆ
ಹುಟ್ಟು-ತ್ತಲೇ
ಹುಟ್ಟು-ತ್ತವೆ
ಹುಟ್ಟುತ್ತಾ
ಹುಟ್ಟು-ತ್ತಾನೆ
ಹುಟ್ಟು-ತ್ತಾರೆ
ಹುಟ್ಟು-ತ್ತಿರು
ಹುಟ್ಟು-ತ್ತಿ-ರು-ವನು
ಹುಟ್ಟು-ತ್ತಿ-ರು-ವಾಗ
ಹುಟ್ಟು-ತ್ತಿ-ರು-ವುವು
ಹುಟ್ಟು-ತ್ತಿ-ರು-ವುವೋ
ಹುಟ್ಟು-ತ್ತೀರೋ
ಹುಟ್ಟು-ತ್ತೇವೆ
ಹುಟ್ಟು-ಬೇ-ಕಾ-ದರೆ
ಹುಟ್ಟುವ
ಹುಟ್ಟು-ವಂ-ತಹ
ಹುಟ್ಟು-ವಂತೆ
ಹುಟ್ಟು-ವನು
ಹುಟ್ಟು-ವರು
ಹುಟ್ಟು-ವ-ವ-ರಲ್ಲ
ಹುಟ್ಟು-ವ-ವರೆಲ್ಲ
ಹುಟ್ಟು-ವ-ವರೆಲ್ಲಾ
ಹುಟ್ಟು-ವಾಗ
ಹುಟ್ಟು-ವಾ-ಗಲೆ
ಹುಟ್ಟು-ವಾ-ಗಲೇ
ಹುಟ್ಟು-ವಿರೋ
ಹುಟ್ಟುವು
ಹುಟ್ಟು-ವು-ದ-ಕ್ಕಿಂತ
ಹುಟ್ಟು-ವು-ದಕ್ಕೆ
ಹುಟ್ಟು-ವು-ದಲ್ಲ
ಹುಟ್ಟು-ವು-ದಿಲ್ಲ
ಹುಟ್ಟು-ವು-ದಿ-ಲ್ಲವೆ
ಹುಟ್ಟು-ವುದು
ಹುಟ್ಟು-ವುದೂ
ಹುಟ್ಟು-ವುದೇ
ಹುಟ್ಟು-ವುದೊ
ಹುಟ್ಟು-ವೆವು
ಹುಟ್ಟು-ಸು-ತ್ತಿ-ರು-ವನು
ಹುಟ್ಟು-ಹಾ-ಕಿ-ದ-ಮೇಲೆ
ಹುಡುಕ
ಹುಡು-ಕ-ಬೇ-ಕಾ-ಗಿಲ್ಲ
ಹುಡು-ಕ-ಬೇ-ಕಾ-ದರೆ
ಹುಡು-ಕ-ಬೇಕು
ಹುಡು-ಕ-ಲಾ-ರಂ-ಭಿ-ಸು-ವುದು
ಹುಡು-ಕಲು
ಹುಡು-ಕಾ-ಡು-ತ್ತಿ-ರು-ತ್ತೇವೆ
ಹುಡುಕಿ
ಹುಡು-ಕಿ-ಕೊಂ-ಡರು
ಹುಡು-ಕಿ-ಕೊಂಡು
ಹುಡು-ಕಿ-ಕೊಂಡೂ
ಹುಡು-ಕಿ-ದರೂ
ಹುಡುಕು
ಹುಡು-ಕು-ತ್ತಿ-ರು-ವರು
ಹುಡು-ಕು-ತ್ತಿ-ರು-ವಾಗ
ಹುಡು-ಕು-ತ್ತಿ-ರು-ವೆವು
ಹುಡು-ಕುವ
ಹುಡು-ಕು-ವಂತೆ
ಹುಡು-ಕು-ವು-ದ-ಕ್ಕಾಗಿ
ಹುಡು-ಕು-ವು-ದಕ್ಕೆ
ಹುಡು-ಕು-ವುದನ್ನು
ಹುಡು-ಕು-ವು-ದಲ್ಲ
ಹುಡು-ಕು-ವುದು
ಹುಡುಗ
ಹುಡು-ಗ-ನಂತೆ
ಹುಡು-ಗ-ನನ್ನು
ಹುಡು-ಗ-ನಿಗೆ
ಹುಡು-ಗ-ರಾ-ದಾ-ಗಿ-ನಿಂ-ದಲೂ
ಹುಡು-ಗ-ರಿಗೆ
ಹುಡು-ಗರು
ಹುಡು-ಗಾ-ಟಿ-ಕೆ-ಯಂತೆ
ಹುಡು-ಗಿ-ಯನ್ನು
ಹುಣಸೇ
ಹುಣ್ಣನ್ನು
ಹುಣ್ಣು
ಹುತ
ಹುತಂ
ಹುತಮ್
ಹುತ್ತ
ಹುತ್ತ-ವನ್ನು
ಹುದು-ಕಿ-ಕೊಂ-ಡಿದೆ
ಹುದು-ಗಿ-ಕೊಂ-ಡಿ-ರುವ
ಹುದು-ಗಿ-ಕೊಂ-ಡಿ-ರು-ವನು
ಹುದು-ಗಿ-ಕೊಂ-ಡಿವೆ
ಹುದು-ಗಿದೆ
ಹುದು-ಗಿ-ದೆಯೋ
ಹುದು-ಗಿ-ರುವ
ಹುದು-ಗಿ-ರು-ವುದು
ಹುದು-ಗಿ-ರು-ವುದೇ
ಹುದು-ಗಿವೆ
ಹುದು-ಗಿ-ಸಿ-ಟ್ಟು-ಕೊಂ-ಡಿಲ್ಲ
ಹುದ್ದೆಯ
ಹುದ್ದೆ-ಯ-ಲ್ಲಿ-ರುವ
ಹುದ್ದೆ-ಯ-ಲ್ಲಿ-ರು-ವ-ವರೆಲ್ಲ
ಹುಬ್ಬಿನ
ಹುಬ್ಬು-ಗಳ
ಹುರಿದ
ಹುರಿ-ದರೆ
ಹುರಿ-ದಿದೆ
ಹುರಿ-ದುಂ-ಬಿ-ಸು-ತ್ತಿ-ದ್ದರು
ಹುರಿ-ದು-ಹಾ-ಕಿ-ದೆಯೊ
ಹುರಿ-ಯ-ಬೇ-ಕಾ-ದರೆ
ಹುರಿಯು
ಹುರಿ-ಯು-ತ್ತಿ-ರುವ
ಹುರಿ-ಯು-ವ-ವನು
ಹುರಿವ
ಹುಲ-ಸಾಗಿ
ಹುಲಿ
ಹುಲು
ಹುಲು-ಮ-ನು-ಜ-ರ-ನ್ನಲ್ಲ
ಹುಲು-ಸಾಗಿ
ಹುಲು-ಸಾದ
ಹುಲ್ಲನ್ನು
ಹುಲ್ಲಿದೆ
ಹುಲ್ಲಿನ
ಹುಲ್ಲಿ-ನಂತೆ
ಹುಲ್ಲಿ-ನೆ-ಸ-ಳಿ-ನಂತೆ
ಹುಲ್ಲಿ-ನೆ-ಸಳು
ಹುಲ್ಲು
ಹುಳ
ಹುಳಕ್ಕೆ
ಹುಳದ
ಹುಳ-ವನ್ನು
ಹುಳಿ
ಹುಳಿ-ಯನ್ನು
ಹುಳಿ-ಯಾ-ಗಿದೆ
ಹುಳಿಯೂ
ಹುಳಿಯೆ
ಹುಳಿ-ಯೆಲ್ಲ
ಹುಳು
ಹುಳುಕು
ಹುಳು-ಗ-ಳಂತೆ
ಹುಳು-ವನ್ನು
ಹುಳು-ವಿ-ನಂತೆ
ಹುಳು-ಹು-ಪ್ಪ-ಡೆ-ಗ-ಳೆಲ್ಲ
ಹೂ
ಹೂಗಳ
ಹೂಗಳಿಂದ
ಹೂಗಳು
ಹೂಗಳೂ
ಹೂಡದೆ
ಹೂಡಿದ
ಹೂಡಿ-ರುವ
ಹೂಡುವ
ಹೂದೋಟ
ಹೂದೋ-ಟ-ದಲ್ಲಿ
ಹೂದೋ-ಟ-ದಿಂದ
ಹೂವಾ-ಡಿಗ
ಹೂವಿನ
ಹೂವಿ-ನಲ್ಲಿ
ಹೂವಿಲ್ಲ
ಹೂವು
ಹೂವು-ಗ-ಳಂ-ತೆ-ಹುಟ್ಟು
ಹೂವು-ಗಳನ್ನು
ಹೂವು-ಗ-ಳ-ಲ್ಲಿಯೂ
ಹೂವು-ಗ-ಳಿವೆ
ಹೂವು-ಗಳು
ಹೃತ್ಪೂ-ರ್ವ-ಕ-ವಾಗಿ
ಹೃತ್ಪೂ-ರ್ವ-ಕ-ವಾದ
ಹೃತ್ಸ್ಥಂ
ಹೃದಯ
ಹೃದ-ಯಕ್ಕೆ
ಹೃದ-ಯ-ಕ್ಕೆಲ್ಲಾ
ಹೃದ-ಯ-ಕ್ರಿ-ಯೆ-ಯನ್ನು
ಹೃದ-ಯದ
ಹೃದ-ಯ-ದಲ್ಲಿ
ಹೃದ-ಯ-ದ-ಲ್ಲಿದ್ದು
ಹೃದ-ಯ-ದ-ಲ್ಲಿಯೂ
ಹೃದ-ಯ-ದ-ಲ್ಲಿಯೇ
ಹೃದ-ಯ-ದ-ಲ್ಲಿ-ರುವ
ಹೃದ-ಯ-ದ-ಲ್ಲಿ-ರು-ವನು
ಹೃದ-ಯ-ದ-ಲ್ಲಿ-ರು-ವಾಗ
ಹೃದ-ಯ-ದ-ಲ್ಲಿ-ರು-ವುದು
ಹೃದ-ಯ-ದಲ್ಲೇ
ಹೃದ-ಯ-ದಿಂದ
ಹೃದ-ಯ-ದೌ-ರ್ಬಲ್ಯ
ಹೃದ-ಯ-ದೌ-ರ್ಬಲ್ಯಂ
ಹೃದ-ಯ-ದೌ-ರ್ಬ-ಲ್ಯಕ್ಕೆ
ಹೃದ-ಯ-ದೌ-ರ್ಬ-ಲ್ಯದ
ಹೃದ-ಯನ
ಹೃದ-ಯ-ನಾ-ಡಿ-ಯನ್ನು
ಹೃದ-ಯ-ನಿಗೆ
ಹೃದ-ಯನು
ಹೃದ-ಯ-ಯಾಂ-ತ-ರಾ-ಳ-ದಲ್ಲಿ
ಹೃದ-ಯ-ವನ್ನು
ಹೃದ-ಯ-ವ-ನ್ನೆಲ್ಲಾ
ಹೃದ-ಯ-ವಾ-ದರೋ
ಹೃದ-ಯವು
ಹೃದ-ಯವೂ
ಹೃದ-ಯ-ವೆಂಬ
ಹೃದ-ಯವೇ
ಹೃದ-ಯಾಂ-ತ-ರಾ-ಳ-ದಲ್ಲಿ
ಹೃದ-ಯಾಂ-ತ-ರಾ-ಳ-ದ-ಲ್ಲಿಯೂ
ಹೃದ-ಯಾನಿ
ಹೃದಯಿ
ಹೃದಿ
ಹೃದ್ಗ-ತ-ವಾ-ಗಿರ
ಹೃದ್ಗ-ತ-ವಾ-ಗಿಲ್ಲ
ಹೃದ್ಗ-ತ-ವಾದ
ಹೃದ್ದೇ-ಶೇ-ಽಜುನ
ಹೃದ್ಯಾ
ಹೃಷಿ-ಕೇಶ
ಹೃಷಿ-ತೋಽಸ್ಮಿ
ಹೃಷೀ-ಕೇಶ
ಹೃಷೀ-ಕೇಶಂ
ಹೃಷೀ-ಕೇಶಃ
ಹೃಷೀ-ಕೇಶೋ
ಹೃಷ್ಟ-ರೋಮಾ
ಹೃಷ್ಯತಿ
ಹೃಷ್ಯಾಮಿ
ಹೆಂಗ-ಸ-ರನ್ನು
ಹೆಂಗ-ಸರು
ಹೆಂಗ-ಸಾ-ದರೂ
ಹೆಂಗ-ಸಾ-ದರೋ
ಹೆಂಗ-ಸಿಗೆ
ಹೆಂಗ-ಸಿನ
ಹೆಂಗ-ಸಿ-ನಂತೆ
ಹೆಂಗಸು
ಹೆಂಗಸೊ
ಹೆಂಗ-ಸೊ-ಬ್ಬಳು
ಹೆಂಡತಿ
ಹೆಂಡ-ತಿಗೆ
ಹೆಂಡ-ತಿಗೋ
ಹೆಂಡ-ತಿಯ
ಹೆಂಡದ
ಹೆಗ-ಲನ್ನು
ಹೆಗ-ಲ-ನ್ನೇರಿ
ಹೆಗ್ಗಣ
ಹೆಗ್ಗ-ಣ-ಗಳು
ಹೆಗ್ಗು-ರುತು
ಹೆಚ್ಚನ್ನು
ಹೆಚ್ಚನ್ನೂ
ಹೆಚ್ಚಲ್ಲ
ಹೆಚ್ಚಾ-ಗ-ಲಾ-ರದು
ಹೆಚ್ಚಾಗಿ
ಹೆಚ್ಚಾ-ಗಿದೆ
ಹೆಚ್ಚಾ-ಗಿ-ದೆಯೊ
ಹೆಚ್ಚಾ-ಗಿ-ದ್ದರೆ
ಹೆಚ್ಚಾ-ಗಿಯೂ
ಹೆಚ್ಚಾ-ಗಿ-ರು-ವುದು
ಹೆಚ್ಚಾ-ಗಿ-ರು-ವುದೊ
ಹೆಚ್ಚಾ-ಗಿವೆ
ಹೆಚ್ಚಾ-ಗುತ್ತ
ಹೆಚ್ಚಾ-ಗು-ವಂತೆ
ಹೆಚ್ಚಾ-ಗು-ವು-ದಿಲ್ಲ
ಹೆಚ್ಚಾ-ಗು-ವುದು
ಹೆಚ್ಚಾ-ಗು-ವುದೂ
ಹೆಚ್ಚಾ-ಗು-ವುದೋ
ಹೆಚ್ಚಾದ
ಹೆಚ್ಚಾ-ದಂತೆ
ಹೆಚ್ಚಿ
ಹೆಚ್ಚಿ-ನ-ದನ್ನು
ಹೆಚ್ಚಿ-ನದು
ಹೆಚ್ಚಿ-ನ-ದೆಂದು
ಹೆಚ್ಚಿ-ನ-ವನೋ
ಹೆಚ್ಚಿ-ನವು
ಹೆಚ್ಚಿಲ್ಲ
ಹೆಚ್ಚಿಸಿ
ಹೆಚ್ಚಿ-ಸಿ-ಕೊ-ಳ್ಳು-ವು-ದ-ಕ್ಕಲ್ಲ
ಹೆಚ್ಚಿ-ಸುತ್ತ
ಹೆಚ್ಚಿ-ಸುವ
ಹೆಚ್ಚಿ-ಸು-ವಂ-ತಹ
ಹೆಚ್ಚಿ-ಸು-ವುದು
ಹೆಚ್ಚು
ಹೆಚ್ಚು-ಕಾಲ
ಹೆಚ್ಚುತ್ತಾ
ಹೆಚ್ಚು-ತ್ತಿ-ರು-ವುದು
ಹೆಚ್ಚು-ವುದು
ಹೆಚ್ಚು-ವೆವು
ಹೆಚ್ಚು-ಹೆ-ಚ್ಚಾಗಿ
ಹೆಚ್ಚು-ಹೆಚ್ಚು
ಹೆಚ್ಚು-ಹೊತ್ತು
ಹೆಚ್ಟಾ-ಗುತ್ತ
ಹೆಜ್ಜೆ
ಹೆಜ್ಜೆ-ಯನ್ನೂ
ಹೆಜ್ಜೆಯೂ
ಹೆಜ್ಜೆಯೇ
ಹೆಜ್ಜೆ-ಹೆ-ಜ್ಜೆಗೆ
ಹೆಡೆ
ಹೆಡೆಯ
ಹೆಣ-ಗಳ
ಹೆಣ-ಗಳು
ಹೆಣ-ಗಳೇ
ಹೆಣದ
ಹೆಣ-ದ-ಮೇಲೆ
ಹೆಣ-ಭಾ-ರ-ವನ್ನು
ಹೆಣ-ಭಾ-ರ-ವಾ-ಗು-ವುದು
ಹೆಣ-ವನ್ನು
ಹೆಣ್ಣು-ಮ-ಗಳು
ಹೆಣ್ಣು-ಮಗು
ಹೆತ್ತರೆ
ಹೆತ್ತ-ವ-ರನ್ನೋ
ಹೆತ್ತ-ವ-ರಿ-ಗಂತೂ
ಹೆತ್ತ-ವ-ರಿಗೆ
ಹೆತ್ತ-ವರೇ
ಹೆದ-ರಿ-ಸು-ವನು
ಹೆದೆ-ಯೇ-ರಿಸಿ
ಹೆಪ್ಪು
ಹೆಪ್ಪು-ಹಾಕಿ
ಹೆಬ್ಬಾ-ಗಿ-ಲನ್ನು
ಹೆಬ್ಬಾ-ಗಿ-ಲಿನ
ಹೆಬ್ಬಾವು
ಹೆಬ್ರು
ಹೆಮ್ಮೆ
ಹೆಮ್ಮೆ-ಕೊ-ಚ್ಚಿ-ಕೊ-ಳ್ಳ-ಬ-ಹುದು
ಹೆಮ್ಮೆ-ಪ-ಟ್ಟಂತೆ
ಹೆಮ್ಮೆ-ಪ-ಟ್ಟರೆ
ಹೆಮ್ಮೆ-ಪ-ಡು-ತ್ತೇವೆ
ಹೆಮ್ಮೆ-ಪ-ಡುವ
ಹೆಮ್ಮೆ-ಪ-ಡು-ವನು
ಹೆಸ-ರನ್ನು
ಹೆಸ-ರಲ್ಲಿ
ಹೆಸ-ರಾಂತ
ಹೆಸ-ರಿ-ಗಾ-ಗಲಿ
ಹೆಸ-ರಿಗೆ
ಹೆಸ-ರಿದೆ
ಹೆಸ-ರಿನ
ಹೆಸ-ರಿ-ನಲ್ಲಿ
ಹೆಸ-ರಿ-ನ-ಲ್ಲಿದೆ
ಹೆಸ-ರಿ-ನಿಂದ
ಹೆಸ-ರಿ-ನಿಂ-ದ-ಲಾ-ದರೂ
ಹೆಸರು
ಹೆಸ-ರು-ಗಳನ್ನು
ಹೆಸ-ರು-ಗ-ಳ-ನ್ನುಳ್ಳ
ಹೆಸ-ರು-ಗಳಿಂದ
ಹೆಸ-ರು-ಗಳು
ಹೆಸ-ರುಳ್ಳ
ಹೆಸ-ರು-ಳ್ಳದ್ದೂ
ಹೆಸರೂ
ಹೆಸರೇ
ಹೇ
ಹೇಗಾ
ಹೇಗಾ-ದರೂ
ಹೇಗಾ-ಯಿತು
ಹೇಗಿ
ಹೇಗಿತ್ತು
ಹೇಗಿದೆ
ಹೇಗಿ-ದೆಯೊ
ಹೇಗಿ-ದೆಯೋ
ಹೇಗಿ-ದ್ದರೂ
ಹೇಗಿ-ದ್ದ-ರೇ-ನಂತೆ
ಹೇಗಿ-ದ್ದಾನೆ
ಹೇಗಿ-ದ್ದಾ-ನೆಯೋ
ಹೇಗಿ-ದ್ದೇನೆ
ಹೇಗಿ-ರ-ಬ-ಲ್ಲದು
ಹೇಗಿ-ರ-ಬ-ಹುದು
ಹೇಗಿ-ರು-ತ್ತವೆ
ಹೇಗಿ-ರು-ತ್ತಾನೆ
ಹೇಗಿ-ರು-ವನು
ಹೇಗಿ-ರು-ವನೊ
ಹೇಗಿ-ರು-ವನೋ
ಹೇಗಿ-ರು-ವರು
ಹೇಗಿ-ರು-ವುದು
ಹೇಗಿವೆ
ಹೇಗೂ
ಹೇಗೆ
ಹೇಗೆಂ-ಬು-ದನ್ನು
ಹೇಗೊ
ಹೇಗೋ
ಹೇಡಿ
ಹೇಡಿ-ಗಲ್ಲ
ಹೇಡಿ-ಗಳು
ಹೇಡಿಗೆ
ಹೇಡಿ-ತನ
ಹೇಡಿ-ತ-ನ-ವನ್ನೂ
ಹೇಡಿ-ತ-ನ-ವಲ್ಲ
ಹೇಡಿ-ತ-ನ-ವೆಂದು
ಹೇಡಿ-ಯಂತೆ
ಹೇಡಿ-ಯಾ-ಗ-ಬೇಡ
ಹೇಡಿ-ಯಾ-ದರೋ
ಹೇತವಃ
ಹೇತು
ಹೇತುಃ
ಹೇತು-ಗಳಿಂದ
ಹೇತು-ನಾ-ನೇನ
ಹೇತು-ಮ-ದ್ಭಿ-ರ್ವಿ-ನಿ-ಶ್ಚಿ-ತೈಃ
ಹೇತು-ರು-ಚ್ಯತೇ
ಹೇತು-ಶಾಸ್ತ್ರ
ಹೇತು-ಶಾ-ಸ್ತ್ರ-ವ-ನ್ನಾಗಿ
ಹೇತೋಃ
ಹೇಯ
ಹೇಯ-ವಾದ
ಹೇಯ-ವಾ-ದುದು
ಹೇರ-ಳ-ವಾ-ಗಿ-ರು-ತ್ತವೆ
ಹೇರ-ಳ-ವಾ-ಗಿವೆ
ಹೇರಿ-ದಂ-ತೆಲ್ಲಾ
ಹೇರಿ-ದ್ದಲ್ಲ
ಹೇರು-ವರು
ಹೇಳ
ಹೇಳ-ಕೂ-ಡದು
ಹೇಳ-ತೀ-ರದು
ಹೇಳ-ದಿ-ರು-ವುದು
ಹೇಳದೆ
ಹೇಳ-ಬಲ್ಲ
ಹೇಳ-ಬ-ಲ್ಲರು
ಹೇಳ-ಬ-ಹುದು
ಹೇಳ-ಬಾ-ರದು
ಹೇಳ-ಬೇಕಾ
ಹೇಳ-ಬೇ-ಕಾಗಿ
ಹೇಳ-ಬೇ-ಕಾ-ಗಿದೆ
ಹೇಳ-ಬೇ-ಕಾ-ಗಿಲ್ಲ
ಹೇಳ-ಬೇ-ಕಾ-ಗು-ವುದು
ಹೇಳ-ಬೇ-ಕಾ-ದರೂ
ಹೇಳ-ಬೇ-ಕಾ-ದರೆ
ಹೇಳ-ಬೇ-ಕಾ-ದು-ದ-ನ್ನೆಲ್ಲ
ಹೇಳ-ಬೇ-ಕಾ-ಯಿತು
ಹೇಳ-ಬೇಕು
ಹೇಳ-ಬೇ-ಕೆಂದು
ಹೇಳ-ಬೇಡ
ಹೇಳ-ಲ-ಸಾಧ್ಯ
ಹೇಳ-ಲಾಗು
ಹೇಳ-ಲಾ-ಗು-ವು-ದಿಲ್ಲ
ಹೇಳ-ಲಾ-ರೆವು
ಹೇಳ-ಲಾ-ರೆವೊ
ಹೇಳಲಿ
ಹೇಳಲು
ಹೇಳ-ಲೂ-ಬ-ಹುದು
ಹೇಳ-ಲೇ-ಬೇ-ಕಾ-ಗಿಲ್ಲ
ಹೇಳ-ಲ್ಪ-ಟ್ಚಿತು
ಹೇಳ-ಲ್ಪ-ಟ್ಟಿತು
ಹೇಳ-ಲ್ಪ-ಟ್ಟಿದೆ
ಹೇಳ-ಲ್ಪ-ಟ್ಟಿವೆ
ಹೇಳ-ಲ್ಪ-ಡು-ತ್ತದೆ
ಹೇಳ-ಲ್ಪ-ಡು-ವನು
ಹೇಳಿ
ಹೇಳಿ-ಕ-ಳಿ-ಸ-ಲಿಲ್ಲ
ಹೇಳಿ-ಕ-ಳು-ಹಿ-ಸು-ತ್ತಾನೆ
ಹೇಳಿ-ಕೊಂ-ಡಂತೆ
ಹೇಳಿ-ಕೊಂ-ಡರೆ
ಹೇಳಿ-ಕೊಂ-ಡಿತು
ಹೇಳಿ-ಕೊಂಡು
ಹೇಳಿ-ಕೊ-ಟ್ಟಾ-ಗಲೇ
ಹೇಳಿ-ಕೊ-ಡ-ಬ-ಹುದು
ಹೇಳಿ-ಕೊ-ಡ-ಬೇ-ಕಾ-ಗಿಲ್ಲ
ಹೇಳಿ-ಕೊ-ಡ-ಬೇಕು
ಹೇಳಿ-ಕೊಡು
ಹೇಳಿ-ಕೊ-ಡು-ತ್ತಾನೆ
ಹೇಳಿ-ಕೊ-ಡುವ
ಹೇಳಿ-ಕೊ-ಡು-ವು-ದಕ್ಕೆ
ಹೇಳಿ-ಕೊ-ಳ್ಳ-ಬ-ಹುದು
ಹೇಳಿ-ಕೊ-ಳ್ಳಲು
ಹೇಳಿ-ಕೊ-ಳ್ಳಲೇ
ಹೇಳಿ-ಕೊ-ಳ್ಳ-ಲೇ-ಬೇ-ಕಾ-ಗಿಲ್ಲ
ಹೇಳಿ-ಕೊ-ಳ್ಳ-ಲೇ-ಬೇಕೆ
ಹೇಳಿ-ಕೊಳ್ಳು
ಹೇಳಿ-ಕೊ-ಳ್ಳು-ತ್ತಾನೆ
ಹೇಳಿ-ಕೊ-ಳ್ಳು-ತ್ತಾ-ನೆಯೋ
ಹೇಳಿ-ಕೊ-ಳ್ಳು-ತ್ತಾರೆ
ಹೇಳಿ-ಕೊ-ಳ್ಳು-ತ್ತಿದ್ದ
ಹೇಳಿ-ಕೊ-ಳ್ಳು-ತ್ತಿ-ದ್ದೆ-ಯಲ್ಲ
ಹೇಳಿ-ಕೊ-ಳ್ಳು-ತ್ತಿ-ರು-ತ್ತೇನೆ
ಹೇಳಿ-ಕೊ-ಳ್ಳು-ತ್ತಿ-ರು-ವನು
ಹೇಳಿ-ಕೊ-ಳ್ಳು-ತ್ತಿ-ರು-ವನೆ
ಹೇಳಿ-ಕೊ-ಳ್ಳು-ತ್ತೇವೆ
ಹೇಳಿ-ಕೊ-ಳ್ಳು-ತ್ತೇ-ವೆಯೊ
ಹೇಳಿ-ಕೊ-ಳ್ಳು-ವಂತೆ
ಹೇಳಿ-ಕೊ-ಳ್ಳು-ವನು
ಹೇಳಿ-ಕೊ-ಳ್ಳು-ವು-ದಕ್ಕೆ
ಹೇಳಿ-ಕೊ-ಳ್ಳು-ವು-ದಿಲ್ಲ
ಹೇಳಿತು
ಹೇಳಿದ
ಹೇಳಿ-ದಂತೆ
ಹೇಳಿ-ದತ್ತ
ಹೇಳಿ-ದನು
ಹೇಳಿ-ದನೊ
ಹೇಳಿ-ದ-ಮೇಲೆ
ಹೇಳಿ-ದರು
ಹೇಳಿ-ದರೂ
ಹೇಳಿ-ದರೆ
ಹೇಳಿ-ದರೇ
ಹೇಳಿ-ದಳು
ಹೇಳಿ-ದ-ವರು
ಹೇಳಿ-ದಾಗ
ಹೇಳಿ-ದಾ-ಗಲೂ
ಹೇಳಿ-ದು-ದನ್ನು
ಹೇಳಿ-ದು-ವಂತೆ
ಹೇಳಿ-ದುವು
ಹೇಳಿದೆ
ಹೇಳಿ-ದೆನು
ಹೇಳಿ-ದೆ-ಯಲ್ಲ
ಹೇಳಿ-ದೆಯೊ
ಹೇಳಿ-ದೆಯೋ
ಹೇಳಿ-ದೊ-ಡನೆ
ಹೇಳಿದ್ದ
ಹೇಳಿ-ದ್ದ-ರ-ಲ್ಲಿಯೇ
ಹೇಳಿ-ದ್ದರೆ
ಹೇಳಿ-ದ್ದಾನೆ
ಹೇಳಿ-ದ್ದಾ-ಯಿತು
ಹೇಳಿದ್ದು
ಹೇಳಿದ್ದೂ
ಹೇಳಿ-ದ್ದೆಲ್ಲ
ಹೇಳಿ-ದ್ದೇನೆ
ಹೇಳಿ-ಬಿ-ಡ-ಬ-ಹು-ದಾ-ಗಿ-ತ್ತಲ್ಲ
ಹೇಳಿ-ಬಿ-ಡ-ಬ-ಹುದು
ಹೇಳಿ-ಬಿ-ಡು-ತ್ತೇನೆ
ಹೇಳಿ-ಬಿ-ಡು-ವುದು
ಹೇಳಿ-ಯಾದ
ಹೇಳಿ-ರ-ಬ-ಹುದು
ಹೇಳಿ-ರಲಿ
ಹೇಳಿ-ರ-ಲಿಲ್ಲ
ಹೇಳಿ-ರು-ತ್ತಾನೆ
ಹೇಳಿ-ರು-ತ್ತಾ-ನೆಯೊ
ಹೇಳಿ-ರುವ
ಹೇಳಿ-ರು-ವಂತೆ
ಹೇಳಿ-ರು-ವನು
ಹೇಳಿ-ರು-ವರು
ಹೇಳಿ-ರು-ವುದನ್ನು
ಹೇಳಿ-ರು-ವು-ದ-ರಿಂದ
ಹೇಳಿ-ರು-ವುದು
ಹೇಳಿ-ರು-ವೆನೊ
ಹೇಳಿ-ರು-ವೆಯೊ
ಹೇಳಿಲ್ಲ
ಹೇಳಿವೆ
ಹೇಳಿ-ಸಿ-ಕೊ-ಳ್ಳ-ಬೇಕು
ಹೇಳು
ಹೇಳುತ್ತ
ಹೇಳು-ತ್ತದೆ
ಹೇಳು-ತ್ತ-ವೆಯೆ
ಹೇಳುತ್ತಾ
ಹೇಳು-ತ್ತಾ-ನಲ್ಲ
ಹೇಳು-ತ್ತಾನೆ
ಹೇಳು-ತ್ತಾ-ನೆಯೆ
ಹೇಳು-ತ್ತಾ-ನೆಯೊ
ಹೇಳು-ತ್ತಾ-ನೆಯೋ
ಹೇಳು-ತ್ತಾನೊ
ಹೇಳು-ತ್ತಾನೋ
ಹೇಳು-ತ್ತಾರೆ
ಹೇಳು-ತ್ತಾ-ರೆಯೇ
ಹೇಳು-ತ್ತಾ-ರೆಯೊ
ಹೇಳು-ತ್ತಾ-ರೆಯೋ
ಹೇಳು-ತ್ತಿ-ದ್ದಂತೆ
ಹೇಳು-ತ್ತಿ-ದ್ದರು
ಹೇಳು-ತ್ತಿ-ದ್ದರೂ
ಹೇಳು-ತ್ತಿ-ದ್ದರೆ
ಹೇಳು-ತ್ತಿ-ದ್ದಾನೆ
ಹೇಳು-ತ್ತಿ-ದ್ದೆನೊ
ಹೇಳು-ತ್ತಿರು
ಹೇಳು-ತ್ತಿ-ರುವ
ಹೇಳು-ತ್ತಿ-ರು-ವನು
ಹೇಳು-ತ್ತಿ-ರು-ವನೊ
ಹೇಳು-ತ್ತಿ-ರು-ವ-ರಲ್ಲ
ಹೇಳು-ತ್ತಿ-ರು-ವರೋ
ಹೇಳು-ತ್ತಿ-ರು-ವ-ವನೂ
ಹೇಳು-ತ್ತಿ-ರು-ವಾಗ
ಹೇಳು-ತ್ತಿ-ರು-ವುದನ್ನು
ಹೇಳು-ತ್ತಿ-ರು-ವುದು
ಹೇಳು-ತ್ತಿ-ರುವೆ
ಹೇಳು-ತ್ತಿ-ರು-ವೆಯೋ
ಹೇಳು-ತ್ತಿ-ರು-ವೆವು
ಹೇಳು-ತ್ತಿಲ್ಲ
ಹೇಳು-ತ್ತೀಯೆ
ಹೇಳು-ತ್ತೇನೆ
ಹೇಳು-ತ್ತೇವೆ
ಹೇಳು-ತ್ತೇ-ವೆಯೊ
ಹೇಳು-ತ್ತೇ-ವೆಯೋ
ಹೇಳು-ತ್ತೇವೋ
ಹೇಳುವ
ಹೇಳು-ವಂ-ತಹ
ಹೇಳು-ವಂ-ತಿಲ್ಲ
ಹೇಳು-ವಂತೆ
ಹೇಳು-ವನು
ಹೇಳು-ವನೊ
ಹೇಳು-ವನೋ
ಹೇಳು-ವರು
ಹೇಳು-ವರೊ
ಹೇಳು-ವಳು
ಹೇಳು-ವ-ವನ
ಹೇಳು-ವ-ವ-ನಿಗೂ
ಹೇಳು-ವ-ವನು
ಹೇಳು-ವ-ವ-ರಿಗೆ
ಹೇಳು-ವ-ವರು
ಹೇಳು-ವ-ಹಾ-ಗಿಲ್ಲ
ಹೇಳು-ವಾಗ
ಹೇಳು-ವಾ-ಗಲೂ
ಹೇಳು-ವು-ದಕ್ಕೂ
ಹೇಳು-ವು-ದಕ್ಕೆ
ಹೇಳು-ವುದನ್ನು
ಹೇಳು-ವು-ದ-ರಲ್ಲಿ
ಹೇಳು-ವು-ದಲ್ಲ
ಹೇಳು-ವು-ದಿಲ್ಲ
ಹೇಳು-ವುದು
ಹೇಳು-ವುದೇ
ಹೇಳು-ವು-ದೇ-ನಿದೆ
ಹೇಳು-ವು-ದೇನು
ಹೇಳು-ವು-ದೊಂದು
ಹೇಳು-ವುವು
ಹೇಳು-ವೆವು
ಹೇಳು-ವೆವೊ
ಹೇಳೋ-ಣವೇ
ಹೈಡ್ರೊ-ಜನ್
ಹೊಂಚು
ಹೊಂಚು-ಕಾ-ಯು-ತ್ತಿ-ರ-ಬ-ಹುದು
ಹೊಂಚು-ತ್ತಿ-ರು-ವನು
ಹೊಂಚು-ಹಾ-ಕಿ-ಕೊಂ-ಡಿದ್ದು
ಹೊಂದ-ತಕ್ಕ
ಹೊಂದದೆ
ಹೊಂದದೇ
ಹೊಂದ-ಬೇಕು
ಹೊಂದ-ಬೇ-ಕೆಂದು
ಹೊಂದ-ಬೇಡ
ಹೊಂದಲು
ಹೊಂದ-ಲ್ಪ-ಟ್ಟಿತು
ಹೊಂದಿ
ಹೊಂದಿ-ಕೊಂ-ಡರೆ
ಹೊಂದಿ-ಕೊಂಡು
ಹೊಂದಿ-ಕೊ-ಳ್ಳ-ಬೇ-ಕಾ-ಗು-ವುದು
ಹೊಂದಿ-ಕೊ-ಳ್ಳುವ
ಹೊಂದಿ-ಕೊ-ಳ್ಳು-ವಾಗ
ಹೊಂದಿ-ಕೊ-ಳ್ಳು-ವು-ದಿಲ್ಲ
ಹೊಂದಿದ
ಹೊಂದಿ-ದರೆ
ಹೊಂದಿ-ದರೋ
ಹೊಂದಿ-ದ-ವನ
ಹೊಂದಿ-ದ-ವನು
ಹೊಂದಿ-ದ-ವನೂ
ಹೊಂದಿ-ದ-ವರು
ಹೊಂದಿದೆ
ಹೊಂದಿ-ರುವ
ಹೊಂದಿ-ರು-ವ-ವನು
ಹೊಂದಿ-ರು-ವುದು
ಹೊಂದಿಲ್ಲ
ಹೊಂದಿ-ಸಿ-ರು-ವನು
ಹೊಂದಿ-ಸಿ-ರು-ವನೋ
ಹೊಂದು
ಹೊಂದು-ಕೊಂ-ಡಿ-ರು-ತ್ತದೆ
ಹೊಂದು-ತ್ತವೆ
ಹೊಂದು-ತ್ತಾನೆ
ಹೊಂದು-ತ್ತಾ-ನೆಯೊ
ಹೊಂದು-ತ್ತಾರೆ
ಹೊಂದು-ತ್ತಿದೆ
ಹೊಂದು-ತ್ತೀಯೆ
ಹೊಂದು-ತ್ತೀರಿ
ಹೊಂದು-ವನು
ಹೊಂದು-ವರು
ಹೊಂದು-ವರೊ
ಹೊಂದು-ವ-ವನು
ಹೊಂದು-ವಾಗ
ಹೊಂದು-ವು-ದಕ್ಕೆ
ಹೊಂದು-ವು-ದಿಲ್ಲ
ಹೊಂದು-ವು-ದಿ-ಲ್ಲವೋ
ಹೊಂದು-ವುದು
ಹೊಂದುವೆ
ಹೊಕ್ಕಾ-ಗಿದೆ
ಹೊಕ್ಕಾ-ಗಿ-ರು-ವನು
ಹೊಕ್ಕಿ-ಕೊಂ-ಡಿವೆ
ಹೊಕ್ಕಿ-ರು-ವುದನ್ನು
ಹೊಕ್ಕು
ಹೊಗ-ಳ-ಕೂ-ಡದು
ಹೊಗ-ಳ-ಬ-ಹುದು
ಹೊಗ-ಳ-ಬೇಕು
ಹೊಗ-ಳಲಿ
ಹೊಗಳಿ
ಹೊಗ-ಳಿಕೆ
ಹೊಗ-ಳಿ-ಕೆ-ಗ-ಳಿಗೆ
ಹೊಗ-ಳಿ-ಕೆಗೆ
ಹೊಗ-ಳಿ-ಕೆಯ
ಹೊಗ-ಳಿ-ಕೆ-ಯ-ನ್ನಲ್ಲ
ಹೊಗ-ಳಿ-ಕೆ-ಯಿಂ-ದಲ್ಲ
ಹೊಗ-ಳಿ-ಕೊ-ಳ್ಳು-ತ್ತಿ-ರು-ವನು
ಹೊಗ-ಳಿ-ಕೊ-ಳ್ಳು-ತ್ತಿ-ರು-ವರು
ಹೊಗ-ಳಿ-ಕೊ-ಳ್ಳು-ವು-ದೇನೂ
ಹೊಗ-ಳಿ-ದರೂ
ಹೊಗ-ಳಿ-ದರೆ
ಹೊಗ-ಳಿ-ದ-ರೇನು
ಹೊಗ-ಳಿ-ಸಿ-ಕೊ-ಳ್ಳ-ಬೇ-ಕಾ-ದರೆ
ಹೊಗ-ಳಿ-ಸಿ-ಕೊ-ಳ್ಳ-ಬೇಕು
ಹೊಗ-ಳಿ-ಸಿ-ಕೊ-ಳ್ಳು-ತ್ತಿದ್ದ
ಹೊಗ-ಳಿ-ಸಿ-ಕೊ-ಳ್ಳು-ವನು
ಹೊಗ-ಳಿ-ಸಿ-ಕೊ-ಳ್ಳು-ವು-ದ-ಕ್ಕಾಗಿ
ಹೊಗಳು
ಹೊಗ-ಳು-ತ್ತಾನೆ
ಹೊಗ-ಳು-ತ್ತಿ-ದ್ದರು
ಹೊಗ-ಳು-ತ್ತಿ-ರು-ವರು
ಹೊಗ-ಳು-ತ್ತೇವೆ
ಹೊಗ-ಳುವ
ಹೊಗ-ಳು-ವರು
ಹೊಗ-ಳು-ವ-ವರು
ಹೊಗ-ಳು-ವಾಗ
ಹೊಗ-ಳು-ವು-ದಕ್ಕೂ
ಹೊಗ-ಳು-ವು-ದಕ್ಕೆ
ಹೊಗ-ಳು-ವು-ದಿಲ್ಲ
ಹೊಗ-ಳು-ವುದು
ಹೊಗು-ಳು-ವಾಗ
ಹೊಗೆ
ಹೊಗೆಯ
ಹೊಗೆ-ಯಾಡಿ
ಹೊಗೆ-ಯಾ-ಡು-ವುದು
ಹೊಗೆ-ಯಿಂದ
ಹೊಟ್ಟನ್ನು
ಹೊಟ್ಟಿಗೆ
ಹೊಟ್ಟಿ-ರುವ
ಹೊಟ್ಟು
ಹೊಟ್ಟೆ
ಹೊಟ್ಟೆ-ಕಿಚ್ಚು
ಹೊಟ್ಟೆ-ಗಳು
ಹೊಟ್ಟೆ-ಗಿ-ಲ್ಲದೆ
ಹೊಟ್ಟೆಗೆ
ಹೊಟ್ಟೆ-ತುಂಬ
ಹೊಟ್ಟೆ-ಪಾ-ಡಿಗೆ
ಹೊಟ್ಟೆ-ಬಾಕ
ಹೊಟ್ಟೆಯ
ಹೊಟ್ಟೆ-ಯಲ್ಲಿ
ಹೊಟ್ಟೇ
ಹೊಡ-ತಕ್ಕೆ
ಹೊಡೆ-ತ-ದಿಂದ
ಹೊಡೆದ
ಹೊಡೆ-ದಂತೆ
ಹೊಡೆದು
ಹೊಡೆ-ದು-ಕೊಂಡು
ಹೊಡೆ-ದು-ಹಾ-ಕ-ಬ-ಹುದು
ಹೊಡೆದೊ
ಹೊಡೆ-ಯದೆ
ಹೊಡೆ-ಯ-ಬ-ಹುದು
ಹೊಡೆ-ಯ-ಬೇ-ಕಾ-ದರೂ
ಹೊಡೆ-ಯಲೂ
ಹೊಡೆ-ಯಿತು
ಹೊಡೆ-ಯು-ತ್ತಾನೆ
ಹೊಡೆ-ಯು-ತ್ತಿದ್ದ
ಹೊಡೆ-ಯು-ತ್ತಿ-ದ್ದರು
ಹೊಡೆ-ಯು-ತ್ತಿ-ರು-ವ-ವನೂ
ಹೊಡೆ-ಯು-ತ್ತಿ-ರು-ವು-ದ-ರಿಂದ
ಹೊಡೆ-ಯುವ
ಹೊಡೆ-ಯು-ವ-ವನ
ಹೊಡೆ-ಯು-ವ-ವನು
ಹೊಡೆ-ಯು-ವು-ದಕ್ಕೆ
ಹೊಡೆ-ಯು-ವುದು
ಹೊಣೆ
ಹೊಣೆ-ಗಳನ್ನು
ಹೊಣೆ-ಗಳೂ
ಹೊಣೆ-ಗಾ-ರ-ರ-ನ್ನಾಗಿ
ಹೊಣೆ-ಯ-ನ್ನೆಲ್ಲ
ಹೊಣೆ-ಯಲ್ಲ
ಹೊತ್ತ
ಹೊತ್ತಂತೆ
ಹೊತ್ತ-ವ-ರಿ-ಗೆಲ್ಲ
ಹೊತ್ತಾಗಿ
ಹೊತ್ತಾದ
ಹೊತ್ತಾ-ಯಿತು
ಹೊತ್ತಿ
ಹೊತ್ತಿ-ಕೊಂಡು
ಹೊತ್ತಿ-ಕೊ-ಳ್ಳುವ
ಹೊತ್ತಿ-ಕೊ-ಳ್ಳು-ವುದು
ಹೊತ್ತಿಗೆ
ಹೊತ್ತಿದ್ದ
ಹೊತ್ತಿನ
ಹೊತ್ತಿ-ನಲ್ಲಿ
ಹೊತ್ತಿ-ನ-ಲ್ಲಿಯೆ
ಹೊತ್ತಿ-ನ-ಲ್ಲಿಯೇ
ಹೊತ್ತಿರು
ಹೊತ್ತಿ-ರುವ
ಹೊತ್ತಿ-ಸಿ-ರುವ
ಹೊತ್ತು
ಹೊತ್ತು-ಕೊಂ-ಡಿ-ರು-ವಂತೆ
ಹೊತ್ತು-ಕೊಂಡು
ಹೊತ್ತು-ಬಂದ
ಹೊದಿಸಿ
ಹೊದೆ-ಯು-ವನು
ಹೊದೆ-ಯು-ವು-ದಕ್ಕೆ
ಹೊನ-ಲನ್ನು
ಹೊನ-ಲಾ-ಗು-ವುದು
ಹೊನ್ನಾ-ಗಿದೆ
ಹೊನ್ನಿನ
ಹೊನ್ನು-ಇದು
ಹೊಮ್ಮು-ತ್ತದೆ
ಹೊಮ್ಮು-ವಂತೆ
ಹೊಮ್ಮು-ವುದು
ಹೊರ-ಗ-ಟ್ಟ-ಬೇಕು
ಹೊರ-ಗಡೆ
ಹೊರ-ಗ-ಡೆಯ
ಹೊರ-ಗ-ಡೆ-ಯ-ವ-ರಿಂದ
ಹೊರ-ಗ-ಡೆ-ಯ-ವರು
ಹೊರ-ಗ-ಡೆ-ಯಿಂದ
ಹೊರ-ಗ-ಡೆ-ಯೆಲ್ಲ
ಹೊರ-ಗ-ಡೆಯೇ
ಹೊರ-ಗಲ್ಲ
ಹೊರ-ಗಿ-ಗಿಂತ
ಹೊರ-ಗಿನ
ಹೊರ-ಗಿ-ನ-ದಕ್ಕೆ
ಹೊರ-ಗಿ-ನ-ದನ್ನು
ಹೊರ-ಗಿ-ನ-ದಲ್ಲ
ಹೊರ-ಗಿ-ನದು
ಹೊರ-ಗಿ-ನದೆ
ಹೊರ-ಗಿ-ನ-ವರು
ಹೊರ-ಗಿ-ನಿಂದ
ಹೊರ-ಗಿ-ರುವ
ಹೊರ-ಗಿ-ರು-ವ-ವ-ನಲ್ಲ
ಹೊರ-ಗಿವೆ
ಹೊರಗೂ
ಹೊರಗೆ
ಹೊರ-ಗೆಯೇ
ಹೊರ-ಗೆಲ್ಲ
ಹೊರ-ಗೆ-ಲ್ಲ-ವನ್ನು
ಹೊರ-ಗೆಲ್ಲಾ
ಹೊರ-ಗೊಂದು
ಹೊರಟ
ಹೊರ-ಟಂ-ತಿದೆ
ಹೊರ-ಟಂತೆ
ಹೊರ-ಟ-ಮೇಲೆ
ಹೊರ-ಟರು
ಹೊರ-ಟರೂ
ಹೊರ-ಟರೆ
ಹೊರ-ಟ-ವ-ನನ್ನು
ಹೊರ-ಟ-ವ-ನಿಗೆ
ಹೊರ-ಟ-ವ-ರಲ್ಲಿ
ಹೊರ-ಟಿತು
ಹೊರ-ಟಿ-ರುವ
ಹೊರ-ಟಿ-ರು-ವನು
ಹೊರ-ಟಿ-ರು-ವ-ವನು
ಹೊರ-ಟಿ-ರು-ವ-ವರು
ಹೊರ-ಟಿ-ರು-ವಿರಿ
ಹೊರ-ಟಿ-ರು-ವುದು
ಹೊರ-ಟಿ-ರುವೆ
ಹೊರ-ಟಿಲ್ಲ
ಹೊರಟು
ಹೊರ-ಟು-ಹೋ-ಗ-ಬ-ಲ್ಲುದು
ಹೊರ-ಟು-ಹೋ-ಗಿ-ಬಿ-ಡು-ವು-ದಿಲ್ಲ
ಹೊರ-ಟು-ಹೋ-ಗು-ವು-ದಕ್ಕೆ
ಹೊರ-ಟು-ಹೋ-ಗು-ವುದು
ಹೊರ-ಟು-ಹೋ-ದನು
ಹೊರ-ಟು-ಹೋ-ಯಿತು
ಹೊರಟೇ
ಹೊರ-ಡ-ಬ-ಹುದು
ಹೊರ-ಡ-ಬೇಕು
ಹೊರ-ಡಲಿ
ಹೊರ-ಡ-ಲು-ದ್ಯು-ಕ್ತ-ನಾ-ದಾಗ
ಹೊರ-ಡು-ವನು
ಹೊರ-ಡು-ವು-ದಕ್ಕೆ
ಹೊರ-ಡು-ವು-ದಿಲ್ಲ
ಹೊರ-ಡು-ವುದು
ಹೊರತು
ಹೊರ-ದೂ-ಡು-ವುದು
ಹೊರ-ಬರು
ಹೊರ-ಬ-ರು-ತ್ತೇವೆ
ಹೊರ-ಬ-ರು-ವನು
ಹೊರ-ಬೇ-ಕಾ-ಯಿತು
ಹೊರ-ಳಿ-ಸಿ-ದರೆ
ಹೊರ-ಳಿ-ಸು-ವು-ದಿಲ್ಲ
ಹೊರ-ವ-ಲ-ಯ-ದ-ಲ್ಲಿ-ರು-ವೆವು
ಹೊರ-ಸೂ-ಸು-ವನು
ಹೊರ-ಸೆ-ಳೆ-ಯು-ವುದನ್ನು
ಹೊರ-ಹೊ-ಮ್ಮಿದ
ಹೊರ-ಹೊ-ಮ್ಮು-ತ್ತಿ-ರು-ವುದು
ಹೊರ-ಹೊ-ಮ್ಮು-ವಾಗ
ಹೊರ-ಹೊ-ಮ್ಮು-ವುದು
ಹೊರುತ್ತಿ
ಹೊರು-ತ್ತಿ-ರುವ
ಹೊರು-ವನು
ಹೊರು-ವುದು
ಹೊರು-ವುದೇ
ಹೊರೆ
ಹೊರೆಗೆ
ಹೊರೆ-ಯನ್ನು
ಹೊರೆ-ಯನ್ನೆ
ಹೊಲ
ಹೊಲಕ್ಕೆ
ಹೊಲ-ಗ-ದ್ದೆಗೆ
ಹೊಲ-ದಂತೆ
ಹೊಲ-ದಲ್ಲಿ
ಹೊಲ-ವನ್ನು
ಹೊಲ-ಸನ್ನೂ
ಹೊಲಸು
ಹೊಲಿ-ಸು-ತ್ತಾರೆ
ಹೊಳೆ-ದಿಲ್ಲ
ಹೊಳೆ-ಯದೆ
ಹೊಳೆ-ಯು-ತ್ತಿ-ರು-ವುದು
ಹೊಳೆ-ಯು-ವು-ದಿಲ್ಲ
ಹೊಳೆ-ಯು-ವುದು
ಹೊಳೆ-ಯು-ವುದೆ
ಹೊಳೆವ
ಹೊಸ
ಹೊಸ-ದಕ್ಕೆ
ಹೊಸ-ದನ್ನು
ಹೊಸ-ದ-ರಲ್ಲಿ
ಹೊಸ-ದ-ರಲ್ಲೇ
ಹೊಸ-ದಾ-ಗಲೆ
ಹೊಸ-ದಾಗಿ
ಹೊಸ-ದಾ-ಗಿ-ರುವ
ಹೊಸ-ದಾ-ಗಿ-ರು-ವಾಗ
ಹೊಸದು
ಹೊಸದೆ
ಹೊಸ-ದೇನು
ಹೊಸ-ದೇನೂ
ಹೊಸ-ದೊಂದು
ಹೊಸ-ನಿ-ನ-ಲ್ಲಿಯೂ
ಹೊಸಬ
ಹೊಸ-ಬ-ಟ್ಟೆ-ಯನ್ನು
ಹೊಸಲ
ಹೊಸ-ಲಿನ
ಹೊಸ-ಹೊಸ
ಹೊಸಿಲು
ಹೋಗ
ಹೋಗ-ಕೂ-ಡದು
ಹೋಗ-ಗೊ-ಡದೆ
ಹೋಗ-ಗೊ-ಡುವು
ಹೋಗ-ತ-ಕ್ಕ-ವು-ಗಳು
ಹೋಗ-ತ್ತಾನೆ
ಹೋಗ-ತ್ತೇನೆ
ಹೋಗದ
ಹೋಗ-ದಂತೆ
ಹೋಗ-ದ-ವನು
ಹೋಗದೆ
ಹೋಗ-ಬ-ಯ-ಸು-ವನು
ಹೋಗ-ಬಲ್ಲ
ಹೋಗ-ಬ-ಹುದು
ಹೋಗ-ಬ-ಹುದೇ
ಹೋಗ-ಬಾ-ರದು
ಹೋಗ-ಬಾ-ರ-ದೆಂದು
ಹೋಗ-ಬೇ-ಕಲ್ಲ
ಹೋಗ-ಬೇಕಾ
ಹೋಗ-ಬೇ-ಕಾ-ಗಿದೆ
ಹೋಗ-ಬೇ-ಕಾ-ಗಿಲ್ಲ
ಹೋಗ-ಬೇ-ಕಾ-ಗು-ವುದು
ಹೋಗ-ಬೇ-ಕಾ-ದರೂ
ಹೋಗ-ಬೇ-ಕಾ-ದರೆ
ಹೋಗ-ಬೇ-ಕಾ-ದಾಗ
ಹೋಗ-ಬೇಕು
ಹೋಗ-ಬೇ-ಕೆಂದು
ಹೋಗ-ಬೇಕೇ
ಹೋಗ-ಬೇಡ
ಹೋಗ-ಲಾ-ಗದು
ಹೋಗ-ಲಾ-ಡಿ-ಸಲು
ಹೋಗ-ಲಾ-ಡಿ-ಸು-ವನು
ಹೋಗ-ಲಾ-ಡಿ-ಸುವು
ಹೋಗ-ಲಾ-ಡಿ-ಸು-ವುದೊ
ಹೋಗ-ಲಾರ
ಹೋಗ-ಲಾ-ರ-ದ-ವನು
ಹೋಗ-ಲಾ-ರದು
ಹೋಗ-ಲಾ-ರರು
ಹೋಗ-ಲಾರೆ
ಹೋಗ-ಲಾ-ರೆವು
ಹೋಗಲಿ
ಹೋಗ-ಲಿಲ್ಲ
ಹೋಗಲು
ಹೋಗ-ಲೆ-ಣಿ-ಸು-ವನು
ಹೋಗ-ಲೇ-ಬೇ-ಕಾ-ಗು-ವುದು
ಹೋಗ-ವರೋ
ಹೋಗ-ವುದು
ಹೋಗಿ
ಹೋಗಿತ್ತು
ಹೋಗಿದೆ
ಹೋಗಿ-ದೆಯೆ
ಹೋಗಿದ್ದ
ಹೋಗಿ-ದ್ದರು
ಹೋಗಿ-ದ್ದರೆ
ಹೋಗಿ-ದ್ದಾನೆ
ಹೋಗಿ-ದ್ದಾರೆ
ಹೋಗಿ-ದ್ದುವು
ಹೋಗಿ-ಬಂ-ದರೆ
ಹೋಗಿ-ಬ-ರು-ವರು
ಹೋಗಿ-ಬಿಟ್ಟು
ಹೋಗಿ-ಬಿ-ಡು-ತ್ತಿತ್ತು
ಹೋಗಿ-ಬಿ-ಡು-ವನು
ಹೋಗಿ-ಬಿ-ಡು-ವರು
ಹೋಗಿ-ಬಿ-ಡು-ವ-ವ-ರಲ್ಲ
ಹೋಗಿ-ಬಿ-ಡು-ವು-ದಿಲ್ಲ
ಹೋಗಿಯೇ
ಹೋಗಿ-ರ-ಬೇಕು
ಹೋಗಿ-ರ-ಲಿಲ್ಲ
ಹೋಗಿ-ರು-ತ್ತದೆ
ಹೋಗಿ-ರುವ
ಹೋಗಿ-ರು-ವಂತೆ
ಹೋಗಿ-ರು-ವನು
ಹೋಗಿ-ರು-ವನೆ
ಹೋಗಿ-ರು-ವನೇ
ಹೋಗಿ-ರು-ವರೊ
ಹೋಗಿ-ರು-ವ-ವರು
ಹೋಗಿ-ರು-ವುದನ್ನು
ಹೋಗಿ-ರು-ವು-ದಿಲ್ಲ
ಹೋಗಿ-ರು-ವುದು
ಹೋಗಿ-ರು-ವುವು
ಹೋಗಿಲ್ಲ
ಹೋಗಿ-ಲ್ಲ-ವಲ್ಲ
ಹೋಗಿವೆ
ಹೋಗು
ಹೋಗುತ್ತ
ಹೋಗು-ತ್ತದೆ
ಹೋಗು-ತ್ತ-ದೆಯೊ
ಹೋಗು-ತ್ತವೆ
ಹೋಗುತ್ತಾ
ಹೋಗು-ತ್ತಾ-ನಲ್ಲ
ಹೋಗು-ತ್ತಾನೆ
ಹೋಗು-ತ್ತಾರೆ
ಹೋಗು-ತ್ತಾರೋ
ಹೋಗು-ತ್ತಿತ್ತು
ಹೋಗು-ತ್ತಿದೆ
ಹೋಗು-ತ್ತಿ-ದೆಯೊ
ಹೋಗು-ತ್ತಿದ್ದ
ಹೋಗು-ತ್ತಿ-ದ್ದಂತೆ
ಹೋಗು-ತ್ತಿ-ದ್ದನು
ಹೋಗು-ತ್ತಿ-ದ್ದರು
ಹೋಗು-ತ್ತಿ-ದ್ದರೂ
ಹೋಗು-ತ್ತಿ-ದ್ದರೆ
ಹೋಗು-ತ್ತಿ-ದ್ದ-ವನು
ಹೋಗು-ತ್ತಿ-ದ್ದಾಗ
ಹೋಗು-ತ್ತಿದ್ದು
ಹೋಗು-ತ್ತಿ-ದ್ದೇವೆ
ಹೋಗು-ತ್ತಿರ
ಹೋಗು-ತ್ತಿ-ರ-ಬೇ-ಕಾ-ಗು-ವುದು
ಹೋಗು-ತ್ತಿರು
ಹೋಗು-ತ್ತಿ-ರು-ತ್ತದೆ
ಹೋಗು-ತ್ತಿ-ರು-ತ್ತವೆ
ಹೋಗು-ತ್ತಿ-ರುವ
ಹೋಗು-ತ್ತಿ-ರು-ವನು
ಹೋಗು-ತ್ತಿ-ರು-ವರು
ಹೋಗು-ತ್ತಿ-ರು-ವರೋ
ಹೋಗು-ತ್ತಿ-ರು-ವ-ವರು
ಹೋಗು-ತ್ತಿ-ರು-ವಾಗ
ಹೋಗು-ತ್ತಿ-ರು-ವು-ದಕ್ಕೆ
ಹೋಗು-ತ್ತಿ-ರು-ವುದು
ಹೋಗು-ತ್ತಿ-ರು-ವುದೋ
ಹೋಗು-ತ್ತಿ-ರು-ವುವು
ಹೋಗು-ತ್ತಿ-ರು-ವುವೋ
ಹೋಗು-ತ್ತಿ-ರುವೆ
ಹೋಗು-ತ್ತಿ-ರು-ವೆನು
ಹೋಗು-ತ್ತಿ-ರು-ವೆವು
ಹೋಗು-ತ್ತಿ-ರು-ವೆವೊ
ಹೋಗು-ತ್ತಿವೆ
ಹೋಗು-ತ್ತೇನೆ
ಹೋಗು-ತ್ತೇವೆ
ಹೋಗು-ತ್ತೇ-ವೆಯೇ
ಹೋಗು-ತ್ತೇ-ವೆಯೋ
ಹೋಗು-ತ್ತೇವೊ
ಹೋಗುವ
ಹೋಗು-ವಂತೆ
ಹೋಗು-ವ-ತ-ನಕ
ಹೋಗು-ವನು
ಹೋಗು-ವನೊ
ಹೋಗು-ವನೋ
ಹೋಗು-ವರು
ಹೋಗು-ವರೊ
ಹೋಗು-ವರೋ
ಹೋಗು-ವಳು
ಹೋಗು-ವ-ವನ
ಹೋಗು-ವ-ವ-ನನ್ನು
ಹೋಗು-ವ-ವ-ನಿಗೆ
ಹೋಗು-ವ-ವ-ನಿ-ವ-ನೊ-ಬ್ಬನೆ
ಹೋಗು-ವ-ವನು
ಹೋಗು-ವ-ವನೇ
ಹೋಗು-ವ-ವರ
ಹೋಗು-ವ-ವ-ರನ್ನು
ಹೋಗು-ವ-ವ-ರಲ್ಲ
ಹೋಗು-ವ-ವ-ರಲ್ಲಿ
ಹೋಗು-ವ-ವ-ರಿಗೆ
ಹೋಗು-ವ-ವರು
ಹೋಗು-ವ-ವರೆ
ಹೋಗು-ವ-ವರೇ
ಹೋಗು-ವಷ್ಟು
ಹೋಗು-ವ-ಹಾ-ಗಿ-ದ್ದರೆ
ಹೋಗು-ವಾಗ
ಹೋಗು-ವಾ-ಗಲೂ
ಹೋಗುವು
ಹೋಗು-ವು-ದ-ಕ್ಕಾ-ಗಿಯೇ
ಹೋಗು-ವು-ದ-ಕ್ಕಾಗು
ಹೋಗು-ವು-ದ-ಕ್ಕಾ-ಗು-ವು-ದಿಲ್ಲ
ಹೋಗು-ವು-ದ-ಕ್ಕಿಂತ
ಹೋಗು-ವು-ದಕ್ಕೂ
ಹೋಗು-ವು-ದಕ್ಕೆ
ಹೋಗು-ವು-ದಕ್ಕೇ
ಹೋಗು-ವುದನ್ನು
ಹೋಗು-ವು-ದನ್ನೂ
ಹೋಗು-ವು-ದ-ರಲ್ಲಿ
ಹೋಗು-ವು-ದ-ರ-ಲ್ಲಿದೆ
ಹೋಗು-ವು-ದ-ರಿಂದ
ಹೋಗು-ವು-ದ-ರೊ-ಳಗೆ
ಹೋಗು-ವು-ದಲ್ಲ
ಹೋಗು-ವು-ದಾ-ಗಿದೆ
ಹೋಗು-ವು-ದಿಲ್ಲ
ಹೋಗು-ವು-ದಿ-ಲ್ಲವೋ
ಹೋಗು-ವುದು
ಹೋಗು-ವುದೂ
ಹೋಗು-ವುದೆ
ಹೋಗು-ವುದೇ
ಹೋಗು-ವುದೊ
ಹೋಗು-ವುದೋ
ಹೋಗು-ವುವು
ಹೋಗು-ವುವೊ
ಹೋಗು-ವುವೋ
ಹೋಗುವೆ
ಹೋಗು-ವೆವು
ಹೋಗು-ವೆವೊ
ಹೋಗು-ವೆವೋ
ಹೋಗೊ
ಹೋಗೋಣ
ಹೋತ್ತು
ಹೋದ
ಹೋದಂ-ತಿದೆ
ಹೋದಂ-ತಿರು
ಹೋದಂ-ತಿ-ರು-ವುದು
ಹೋದಂತೆ
ಹೋದಂ-ತೆಲ್ಲಾ
ಹೋದ-ನಲ್ಲ
ಹೋದನು
ಹೋದ-ಮೇಲೆ
ಹೋದರು
ಹೋದರೂ
ಹೋದರೆ
ಹೋದ-ರೆಷ್ಟು
ಹೋದರೊ
ಹೋದ-ಲ್ಲದೆ
ಹೋದಲ್ಲಿ
ಹೋದಳು
ಹೋದ-ವನು
ಹೋದ-ವರು
ಹೋದ-ವರೇ
ಹೋದಷ್ಟು
ಹೋದಷ್ಟೂ
ಹೋದಾಗ
ಹೋದಾ-ಗಲೂ
ಹೋದೊ-ಡನೆ
ಹೋದೊ-ಡ-ನೆಯೆ
ಹೋದೊ-ಡ-ನೆಯೇ
ಹೋಮ
ಹೋಮ-ಕರ್ಮ
ಹೋಮದ
ಹೋಮ-ಮಾ-ಡು-ತ್ತಾರೆ
ಹೋಮ-ವನ್ನೊ
ಹೋಯಿ-ತಂತೆ
ಹೋಯಿತು
ಹೋರಾಟ
ಹೋರಾ-ಟ-ಕ್ಕಿಂತ
ಹೋರಾ-ಟಕ್ಕೆ
ಹೋರಾ-ಟ-ದಲ್ಲಿ
ಹೋರಾ-ಟ-ದ-ಲ್ಲಿಯೇ
ಹೋರಾ-ಟ-ದಲ್ಲೆ
ಹೋರಾ-ಟ-ದಿಂದ
ಹೋರಾ-ಟ-ವನ್ನು
ಹೋರಾ-ಟ-ವಾ-ಡು-ತ್ತಿ-ದ್ದರೆ
ಹೋರಾ-ಟ-ವಿ-ರ-ಬೇಕು
ಹೋರಾ-ಟವೇ
ಹೋರಾ-ಡ-ಬೇ-ಕಾ-ಗಿದೆ
ಹೋರಾ-ಡ-ಬೇ-ಕಾ-ಗು-ವುದು
ಹೋರಾ-ಡ-ಬೇಕು
ಹೋರಾ-ಡಲು
ಹೋರಾ-ಡಲೇ
ಹೋರಾಡಿ
ಹೋರಾ-ಡಿ-ದರೆ
ಹೋರಾ-ಡು-ತ್ತಿ-ದ್ದರೂ
ಹೋರಾ-ಡು-ತ್ತಿ-ದ್ದರೆ
ಹೋರಾ-ಡು-ತ್ತಿ-ದ್ದೇನೆ
ಹೋರಾ-ಡು-ತ್ತಿ-ದ್ದೇವೆ
ಹೋರಾ-ಡು-ತ್ತಿ-ರ-ಬೇ-ಕಾ-ಗು-ವುದು
ಹೋರಾ-ಡು-ತ್ತಿ-ರ-ಬೇಕು
ಹೋರಾ-ಡು-ತ್ತಿ-ರು-ವರೋ
ಹೋರಾ-ಡು-ತ್ತಿ-ರು-ವ-ವರು
ಹೋರಾ-ಡು-ತ್ತೇನೆ
ಹೋರಾ-ಡು-ತ್ತೇವೆ
ಹೋರಾ-ಡುವ
ಹೋರಾ-ಡು-ವನು
ಹೋರಾ-ಡು-ವರು
ಹೋರಾ-ಡು-ವ-ವ-ನಿಗೆ
ಹೋರಾ-ಡು-ವ-ವರು
ಹೋರಾ-ಡು-ವಾಗ
ಹೋರಾ-ಡು-ವು-ದಕ್ಕೆ
ಹೋರಾ-ಡು-ವು-ದ-ರಲ್ಲಿ
ಹೋರಾ-ಡು-ವು-ದಲ್ಲ
ಹೋರಾ-ಡು-ವುದು
ಹೋರಾ-ಡು-ವೆವು
ಹೋಲ-ಬ-ಹುದು
ಹೋಲಿಕೆ
ಹೋಲಿಸ
ಹೋಲಿ-ಸ-ಬ-ಹುದು
ಹೋಲಿ-ಸ-ಬಾ-ರದು
ಹೋಲಿ-ಸ-ಬೇ-ಕಾ-ದರೆ
ಹೋಲಿ-ಸ-ಲಾ-ಗು-ವು-ದಿಲ್ಲ
ಹೋಲಿಸಿ
ಹೋಲಿ-ಸಿ-ಕೊಂ-ಡರೆ
ಹೋಲಿ-ಸಿ-ಕೊಂ-ಡಾಗ
ಹೋಲಿ-ಸಿ-ಕೊಂಡು
ಹೋಲಿ-ಸಿ-ಕೊ-ಳ್ಳು-ತ್ತಾನೆ
ಹೋಲಿ-ಸಿ-ಕೊ-ಳ್ಳು-ವು-ದಕ್ಕೆ
ಹೋಲಿ-ಸಿ-ದರೆ
ಹೋಲಿ-ಸಿ-ದಾಗ
ಹೋಲಿ-ಸಿ-ರು-ವನು
ಹೋಲಿಸು
ಹೋಲಿ-ಸು-ತ್ತಾರೆ
ಹೋಲಿ-ಸು-ವರು
ಹೋಲು-ತ್ತಾನೆ
ಹೋಲುವ
ಹೌದು
ಹ್ಯಕ-ರ್ಮಣಃ
ಹ್ಯನಂ-ತಾಶ್ಚ
ಹ್ಯಪಿ
ಹ್ಯವರಂ
ಹ್ಯವಶಃ
ಹ್ಯವ-ಶೋಽಪಿ
ಹ್ಯಸಂ-ನ್ಯ-ಸ್ತ-ಸಂ-ಕಲ್ಪೋ
ಹ್ಯಸ್ಯ
ಹ್ಯಹಂ
ಹ್ಯಾಚ-ರನ್
ಹ್ಯಾತ್ಮನೋ
ಹ್ಯಾತ್ಮ-ವಿ-ಭೂ-ತಯಃ
ಹ್ಯಾತ್ಮಾ-ವಿ-ಭೂ-ತಯಃ
ಹ್ಯಾಶು
ಹ್ಯುಪ-ಪ-ದ್ಯತೇ
ಹ್ಯೇತ-ದು-ತ್ತ-ಮಮ್
ಹ್ಯೇತೇ
ಹ್ಯೇನಂ
ಹ್ಯೇಷಾ
ಹ್ಯೇಷಾಂ
ಹ್ರಿಯತೇ
ಹ್ರೀರ-ಚಾ-ಪ-ಲಮ್
ೈ
ೈಸ್ತ
ೈಸ್ತೇ
್ಣ
್ಣವಾದ
್ತಮ-ಯೋ-ಗತಃ
್ಮ
್ಮದಲ್ಲಿ
್ಮವಾಗಿ
್ಮವಾ-ಗಿದೆ
್ಮವಾದ
್ಮವಾ-ದು-ದನ್ನು
್ಯ
್ಯದ-ಲ್ಲಿ-ಡದೇ
್ಯವನ್ನು
್ಯವಾ-ಗಿ-ರು-ವನು
್ರಮ-ತೀ-ಶ್ವರಃ
್ವ
್ವಕ
್ವಕ-ವಾಗಿ
್ವಕ-ವೇ-ಽಬ್ರ-ವೀತ್
್ವಕು
್ವಕ್ಕೆ
್ವಗು-ಣಕ್ಕೆ
್ವಗುಣಿ
್ವಜ್ಞಾನಿ
್ವವನ್ನು
್ವಶಾ-ಸ್ತ್ರವೇ
್ಷಮಯೇ
}
