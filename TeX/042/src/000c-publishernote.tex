
\chapter*{PUBLISHER'S NOTE}

Sriman Mahabharata Tatparyanirnaya is one of the thirty–seven works by Sri Madhwa. He, therein, gives the correct interpretation of Mahabharata, authored by Sri Vedavyasa (“Swayam Narayanah Prabhu."'). The passage of time has added its own moss by way of additions causing confusion to the Satwiks. It is to remove the confusions and contradictions. Sri Madhwa had undertaken the writing of Mahabharata Tatparyanirnaya giving authenticated interpretation of paratatwa.

There are six commentaries on this work: Bhavaprakasika (Sri Vadiraja), Padarthadeepika (Janardhana Bhatta), Sarasangraha (Varadaraja, Srinivasa Theertha, Sri Raghavendra and Sri Vedanga Theertha).

The present work by Sri A. R. Subbannachar occupying the gadi of Guruji in Sri Madhva Samaj, West Mambalam is the third Kusuma offered to the Lord, thanks to the munificent grant by Tirumala Tirupati Devastanam, which, inter–alia, has been doing yeoman service to the cause of Hindu Dharma. Sri Subbannachar has done a great service to the Satwik public in general and to the Madhwa public in particular by his true and facile translation of Sri Raghavendra's illuminating abridgement on Mahabharata Tatparyanirnaya.

I am sure this Kusuma will be accepted by the Lord of the Seven Hills and that our Samaj activities will be blessed. The Samaj has dedicated itself for the propagation of Madhwa Philosophy.

I join my colleagues in paying our pranams to Sri Subbannachar, who, as the preceptor of our Samaj, has enabled the Samaj to carve out a niche for itself among the Madhwa institutions.

Our thanks are due to Prabha Printers for the excellent printing and get–up of this volume.

\begin{center}
\textbf{Srikrishnarpanamastu}
\end{center}

\begin{flushright}
\textbf{Dr. C. V. RAMADAS}\\\textit{Vice President}\\ Sri Madhwa Samaja, Madras
\end{flushright}

\begin{flushleft}
Madras \\ 09–08–1988
\end{flushleft}

