

\begin{center}
\phantom{\dev{श्रीः}}
\end{center}

\begin{center}
\phantom{\dev{।। श्री गुरुराजो विजयते~।।}}
\end{center}

\chapter{ ಶ‍್ರೀಮನ್ಮಹಾಭಾರತತಾತ್ಪರ್ಯನಿರ್ಣಯ ಭಾವಸಂಗ್ರಹಃ}

\section*{ಅಧ್ಯಾಯ\enginline{-}೧}

\begin{verse}
\textbf{ಯೋಽಗ್ರೇಽಭೂದ್ವಿಶ್ವಗರ್ಭಃ ಸುಖನಿಧಿರಮಿತೈರ್ವಾಸುದೇವಾದಿರೂಪೈಃ }\\\textbf{ಕ್ರೀಡನ್ದೇವೈರಜಾದ್ಯೈರಗಣಿತಸುಗುಣೋ ನಿತ್ಯ ನೀಚೋಚ್ಚಭಾವೈಃ~।} \\\textbf{ವೇದೈರ್ವೆದ್ಯೋಽಸ್ತದೋಷೋಽಪ್ಯ ಸುರಜನಮನೋ ಮೋಹಯನ್ಮರ್ತ್ಯವೃತ್ತ್ಯಾ~।} \\\textbf{ಭಕ್ತಾನಾಂ ಮುಕ್ತಿದಾತಾ ದ್ವಿಷದಸುಖಕರಃ ಪಾತು ಸೋಽಸ್ಮಾನ್ ರಮೇಶಃ~।।}
\end{verse}

\textbf{ತಾತ್ಪರ್ಯ:\enginline{-} }ಜೀವಜಡಾತ್ಮಕವಾದ ಜಗತ್ತನ್ನು ಸೃಷ್ಟಿಗೆ ಪೂರ್ವದಲ್ಲಿ ತನ್ನ ಉದರದಲ್ಲಿ ಧರಿಸಿ ಕ್ರಿಯಾತ್ಮಕವಾಗಿರುವ, ಅಪರಿಮಿತ ಸುಖಾತ್ಮನಾದ, ತಾರತಮ್ಯದಿಂದ ಕೂಡಿದ\break ಚತುರ್ಮುಖಬ್ರಹ್ಮದೇವರೇ ಮೊದಲಾದ ದೇವತೆಗಳೊಡನೆ ವಾಸುದೇವನೇ ಮೊದಲಾದ ತನ್ನ ರೂಪಗಳಿಂದ ಕ್ರೀಡಿಸುವ, ಅಸಂಖ್ಯಾತವಾದ ಶುಭಗುಣಗಳಿಂದ ಪರಿಪೂರ್ಣನಾದ, ವೇದಗಳಿಂದಲೇ ತಿಳಿಯಲ್ಪಡಬೇಕಾದ, ಸಕಲದೋಷವರ್ಜಿತನಾದ, ತನ್ನ ಅವತಾರಗಳಲ್ಲಿ ಮನುಷ್ಯರಂತೆ ನಟಿಸಿ ಅಸುರ ಜನರ ಮನಸ್ಸನ್ನು ಮೋಹಗೊಳಿಸುವ, ತನ್ನ ಭಕ್ತರಿಗೆ ಮೋಕ್ಷವನ್ನು ದಯಪಾಲಿಸುವ, ಹಾಗೂ ತನ್ನನ್ನು ದ್ವೇಷಿಸುವವರಿಗೆ ನರಕಾದಿ ದುಃಖಗಳನ್ನು ಕೊಡುವ ಲಕ್ಷ್ಮೀರಮಣನಾದ ಶ‍್ರೀಹರಿಯು ನಮ್ಮನ್ನು ರಕ್ಷಿಸಲಿ.

ಮೇಲಿನ ಶ್ಲೋಕವು ಶ‍್ರೀಮದಾನಂದತೀರ್ಥರಿಂದ ರಚಿತವಾಗಿರುವ ಮಂಗಳಾ ಚರಣೆಯ ಶ್ಲೋಕದ ಅರ್ಥವನ್ನೇ ಒಳಗೊಂಡಿದೆ. ಮೊದಲನೇ ಅಧ್ಯಾಯದಲ್ಲಿ ೧೪೧ ಶ್ಲೋಕಗಳಿವೆ.

'ಶ‍್ರೀ ಹರಿಯು ವಾಸುದೇವ, ಸಂಕರ್ಷಣ, ಪ್ರದ್ಯುಮ್ನ, ಅನಿರುದ್ಧ – ಈ ರೂಪಗಳನ್ನು ಧರಿಸಿದುದೂ, ಇದಕ್ಕೆ ಕ್ರಮವಾಗಿ ಲಕ್ಷ್ಮಿದೇವಿಯು ಮಾಯಾ, ಜಯಾ, ಕೃತಿ, ಶಾಂತಿ ರೂಪಗಳನ್ನು ಪಡೆದುದೂ, ಶುದ್ಧ ಸೃಷ್ಟಿ, ಪರಾಧೀನವಿಶೇಷಾವಾಪ್ತಿ ರೂಪಸೃಷ್ಟಿ, ಮಿಶ್ರಸೃಷ್ಟಿ, ಕೇವಲ ಸೃಷ್ಟಿಗಳನ್ನು ಮಾಡಿದುದೂ, ಶ‍್ರೀಹರಿಯ ಸ್ವಗತಭೇದ ವಿವರ್ಜಿತತ್ವ, ಶ‍್ರೀಹರಿಯ ಸರ್ವೊತ್ತಮತ್ವ, ಈ ವಿಷಯಕ್ಕೆ ಶಾಸ್ತ್ರದ ಪ್ರಮಾಣ, ಪ್ರಮಾಣಗ್ರಂಥಗಳು ಯಾವವು, ಬ್ರಹ್ಮ ಸೂತ್ರಗಳ ರಚನೆಗೆ ಕಾರಣ, ರುದ್ರದೇವರಿಂದ ಮೋಹ ಶಾಸ್ತ್ರಗಳ ರಚನೆ, ವಿಷ್ಣು ಸರ್ವೋತ್ತಮನೆಂದು ಸಾರುವ ವೇದವಾಕ್ಯಗಳ ಉದಾಹರಣೆ, ಪಂಚ ಭೇದಗಳು, ಜೀವರಲ್ಲಿ ನಿತ್ಯವಾದ ತಾರತಮ್ಯ, ಶ‍್ರೀ ಹರಿಯ ವಿಶೇಷ ಪ್ರಸಾದದಿಂದಲೇ ಮೋಕ್ಷ, ವಿಷ್ಣುವೇ ಮೋಕ್ಷದಾತ, ಸೃಷ್ಟ್ಯಾದಿ ಅಷ್ಟಕರ್ತತ್ವವು ಶ‍್ರೀ ಹರಿಗೇ ವಿನಹ ಅನ್ಯರಿಗಲ್ಲ. ಭಕ್ತಿಯ ಸ್ವರೂಪ, ತ್ರಿವಿಧರಾದ ಜೀವರ ಸ್ವರೂಪ ಯೋಗ್ಯತೆಗಳು, ದೇವತಾ ಪದವಿಗಳಿಗೆ ಬರುವ ಜೀವ ಸಮುದಾಯದ ವಿವರಣೆ, ಸತ್ಕರ್ಮಗಳು ಭಕ್ತಿಗೆ ಅಂಗಗಳೇ, ಜ್ಞಾನದಿಂದಲೇ ಮೋಕ್ಷ, ಗುರುಗಳಲ್ಲಿಯೂ ಯಥಾಯೋಗ್ಯವಾದ ಭಕ್ತಿಯ ಅವಶ್ಯಕತೆ, ದೈತ್ಯರಿಗೆ ಆಗುವ ಅಂತಿಮಗತಿ, ನವವಿಧವಾದ ದ್ವೇಷಗಳು, ಗುರುಗಳ ಲಕ್ಷಣ, ದೈತ್ಯರಲ್ಲಿ ತಾರತಮ್ಮ, ಸದಾಗಮಗಳ ಅರ್ಥ, ಶ‍್ರೀ ವೇದವ್ಯಾಸ ದೇವರ ಆಜ್ಞಾನುಸಾರವಾಗಿ ಸಚ್ಛಾಸ್ತ್ರಗಳ ತಾತ್ಪರ್ಯವು ತಮ್ಮಿಂದ ಹೇಳಲ್ಪಟ್ಟಿವೆ ಮುಂತಾದ ವೇದಾಂತ ಸಾರವು ಶ‍್ರೀಮನ್ಮಹಾಭಾರತತಾತ್ಪರ್ಯನಿರ್ಣಯದ ಮೊದಲನೇ ಅಧ್ಯಾಯದಲ್ಲಿ ನಿರೂಪಿತ\-ವಾಗಿದೆ.


\section*{ಅಧ್ಯಾಯ\enginline{-}೨}

\textbf{ಸದ್ಗ್ರಂಥಾನಾಂ ಸಮೂಹೇ ಜಗತಿ ವಿಲುಲಿತೇ ಯೇನ ತದ್ಭಾವಮುಚ್ಚೈಃ\\ ವಕ್ತುಂ ಮಧ್ವೋ ನಿಯುಕ್ತೋ ವ್ಯಧಿತಸುವಚಸಾಮುದ್ಧೃತಿಂ ಭಾರತಸ್ಯ~।}

\noindent
\textbf{ವೇದೋತ್ಕೃಷ್ಟಸ್ಯ ವಿಷ್ಣೋಃ ಪರಮಪುರುಷತಾಂ ತಾರತಮ್ಯಂ ಸುರಾಣಾಂ\\ ವಾಯೋರ್ಜೀವೋ\-ತ್ತಮುತ್ಯಾದಿಕಮಪಿ ವದತಾಂ ವ್ಯಾಸಮೀಡೇ ತಮೀಶಮ್~।। }

ಮಹಾಭಾರತವೇ ಮೊದಲಾದ ಸಚ್ಛಾಸ್ತ್ರ ಗ್ರಂಥಗಳ ಅರ್ಥಗಳನ್ನು, ಜನರು ತಪ್ಪಾಗಿ ತಿಳಿಯಲು, ಶ‍್ರೀ ವೇದವ್ಯಾಸರು ಆಚಾರ್ಯ ಶ‍್ರೀಮಧ್ವರಿಗೆ ಸಮೀಚೀನ ಅರ್ಥಗಳನ್ನು ಬೋಧಿಸಲು ಆಜ್ಞೆ ಮಾಡಲಾಗಿ ಶ‍್ರೀಮದಾಚಾರ್ಯರು ಭಾರತದ ಪ್ರಮಾಣ ವಾಕ್ಯಗಳನ್ನು ಉದಹರಿಸಿ, ವೇದಗಳಲ್ಲಿ ಪ್ರತಿಪಾದಿತವಾಗಿರುವ ಶ‍್ರೀ ವಿಷ್ಣುವಿನ ಸರ್ವೊತ್ತಮತ್ವವನ್ನೂ ದೇವತೆಗಳಲ್ಲಿರುವ ತಾರತಮ್ಯವನ್ನೂ, ಸಕಲ ಜೀವರಲ್ಲಿ ಶ‍್ರೀವಾಯುದೇವರ ಉತ್ತಮತ್ವವೇ ಮೊದಲಾದ ಅನೇಕ ಯಥಾರ್ಥ ವಿಷಯಗಳನ್ನೂ, ಉಪಪಾದನೆ ಮಾಡಿರುತ್ತಾರೆ. ಶ‍್ರೀ ಮಧ್ವರಿಗೆ ಇಂತಹ ಆಜ್ಞೆಯನ್ನು ಮಾಡಿದ ಶ‍್ರೀ ವೇದವ್ಯಾಸರನ್ನು ಸ್ತುತಿಸುತ್ತೇನೆ.

ಎರಡನೇ ಅಧ್ಯಾಯದಲ್ಲಿ ೧೮೬ ಶ್ಲೋಕಗಳಿವೆ.

ನಿರ್ಣಯಗ್ರಂಥವನ್ನು ರಚಿಸಲು ಕಾರಣ, ಮೂಲಗ್ರಂಥಗಳಲ್ಲಿನ ಶ್ಲೋಕಗಳನ್ನು ಕೆಲವರು ತೆಗೆದು ಹಾಕಿ, ತಮ್ಮ ಮತಕ್ಕೆ ಅನುಸಾರವಾಗಿ ಇತರ ಶ್ಲೋಕಗಳನ್ನು ವ್ಯಾಖ್ಯಾನ ಮಾಡುವುದು, ಮಹಾಭಾರತ ಗ್ರಂಥದ ಶ್ರೇಷ್ಠತೆ, ದೇವತೆಗಳಲ್ಲಿ ತಾರತಮ್ಯ, ಶ‍್ರೀಹರಿಯ ವಿಶೇಷ ಆವೇಶಉಳ್ಳ ದೇವತೆಗಳು, ವಾಯು ದೇವರಲ್ಲಿ ಚತುರ್ಮುಖ ಬ್ರಹ್ಮದೇವರ ನಾಮ ಪ್ರಯೋಗ, ದ್ರೌಪದೀ ದೇವಿಯರ ಭಕ್ತ್ಯಾದಿಗಳು, ಸರ್ವಶಾಸ್ತ್ರಗಳನ್ನೂ ಪರಿಶೀಲಿಸಿದರೆ ಶ‍್ರೀಮನ್ನಾರಾಯಣನೇ ಧ್ಯೇಯನು, ಅವನನ್ನು ಸಾಕಲ್ಯೇನ ತಿಳಿಯಲು ಯಾರಿಗೂ ಸಾಧ್ಯವಿಲ್ಲ, ಅವನು ಚತುರ್ವಿಧ ನಾಶರಹಿತನು, ಶ‍್ರೀಹರಿಯ ಎಲ್ಲ ಅವತಾರ ಶರೀರಗಳೂ ಸಹ ಅಪ್ರಾಕೃತಗಳೇ – ಜ್ಞಾನಾನಂದಾದಿ ಗುಣಪೂರ್ಣಗಳೇ, ಈ ಪ್ರಮೇಯಕ್ಕೆ ಭಗವದ್ಗೀತೆಯ ವಚನ, ಆಸುರೀ ಭಾವದ ಜನರಿಗೆ ಅಗುವ ಅಂತಿಮ ಗತಿಗೆ ಭಗವದ್ಗೀತೆಯ ಸಂವಾದ, ದರ್ಶನಭಾಷೆ, ಸಮಾಧಿಭಾಷೆ, ಗುಹ್ಯಭಾಷೆಗಳ ಲಕ್ಷಣ, ವಾಯುದೇವರು ಶ‍್ರೀಹರಿಯ ಪ್ರಧಾನ ಅಂಗಭೂತರು, ಶ‍್ರೀವಾಯುದೇವರ ಮೂರು ಅವತಾರಗಳು, ಭಾರತಾರ್ಥವು ಮೂರು ಪ್ರಕಾರ, ಭೀಮಸೇನನಿಗೆ ಭಕ್ತ್ಯಾದಿಶಬ್ದವಾಚ್ಯತ್ವ, ದುರ್ಯೊಧನ - ದುಃಶಾಸನರ ಸ್ವರೂಪ, ಶಕುನಿಯು ನಾಸ್ತಿಕವಾದಕ್ಕೆ ಅಭಿಮಾನಿ: ದೇಹದಲ್ಲಿರುವ ಇಂದ್ರಿಯಗಳು ದ್ರೋಣ, ಭೀಷ್ಮ ಮುಂತಾದವರಿಗೆ ಸಮಾನ, ಜ್ಞಾನ-ಭಕ್ತಿ, ವೈರಾಗ್ಯಗಳಲ್ಲಿ ಶ‍್ರೀ ಹನುಮಂತದೇವರಿಗೆ ಸರಿಸಮರಿಲ್ಲ, ಬಳಿತ್ಥಾಸೂಕ್ತವನ್ನು ಉದಹರಿಸಿ ವಾಯುದೇವರ ಮೂರು ಅವತಾರಗಳ ಶ್ರೇಷ್ಠತೆ ಹಾಗೂ ಶ‍್ರೀವಾಯುದೇವರಿಗೆ ವೇದಪ್ರತಿಪಾದ್ಯತ್ವ, ದ್ರೌಪದೀದೇವಿಯರು ಶೇಷಾದಿಗಳಿಗಿಂತ ಉತ್ತಮರು, ಶೇಷದೇವರ, ಇಂದ್ರದೇವರ ಸ್ವರೂಪ ವಿವರಣೆ, ಪಾಂಡವರು ಸರ್ವದಾ ಶ‍್ರೀ ಹರಿಯ ತತ್ಪರರಾಗಿರುವರು-ಇವೇ ಮುಂತಾದ ಪ್ರಮೇಯಗಳು ಎರಡನೇ ಅಧ್ಯಾಯದಲ್ಲಿ ನಿರೂಪಿತವಾಗಿವೆ.


\section*{ಅಧ್ಯಾಯ\enginline{-}೩}

\begin{verse}
\textbf{ಆದೌ ರೂಪಚತುಷ್ಟಯಿಾಂ ಸೃಜತಿ ಯೋ ದೇವಾನ್ಪುರಾನುಕ್ರಮಾತ್\\ ಬ್ರಹ್ಮಾಂಡಂ ಪುರಮಬ್ಜ ಜಾದಿನಿಬುಧಾನ್ ಸೃಷ್ಟ್ವಾ ಹರತ್ಯಂತತಃ~।}\\\textbf{ಸ್ರಷ್ಟಾ ಪೂರ್ವವದಸ್ಯ ಸರ್ವಜಗತೋ ಮತ್ಸ್ಯಾದಿರೂಪೋಽಭವನ್\break ರಾಮೋsಭೂದನುಜಾನ್ವಿತೋ ದಶರಥಾತ್ಪಾಯಾತ್ಸ ನಃ ಶ‍್ರೀಪತಿಃ~।।}
\end{verse}

ಸೃಷ್ಟಿಗೆ ಪೂರ್ವದಲ್ಲಿ ವಾಸುದೇವಾದಿ ನಾಲ್ಕು ರೂಪಗಳನ್ನು ಧರಿಸಿ, ಹಿಂದಿನ ಕಲ್ಪದಂತೆಯೇ ಬ್ರಹ್ಮಾಂಡವನ್ನೂ, ಬ್ರಹ್ಮಾದಿ ದೇವತೆಗಳನ್ನೂ ಸೃಷ್ಟಿಸಿ, ನಂತರ ಅವೆಲ್ಲವನ್ನೂ ತನ್ನಲ್ಲಿಯೇ ಇಟ್ಟುಕೊಂಡು, ಸೃಷ್ಟಿ ಕಾಲ ಬಂದಾಗ ಪುನಃ ಅವೆಲ್ಲವನ್ನೂ ಸೃಜಿಸಿ, ಮತ್ಸ್ಯಾದಿ ಅವತಾರಗಳನ್ನು ಮಾಡಿ, ತಮ್ಮಂದಿರೊಡನೆ ದಶರಥರಾಜನ ನಿಮಿತ್ತದಿಂದ ರಾಮನಾಗಿ ಅವತರಿಸಿದ ಲಕ್ಷ್ಮಿಪತಿಯು ನಮ್ಮನ್ನು ಸಲಹಲಿ.

ಈ ಅಧ್ಯಾಯದಲ್ಲಿ ೮೪ ಶ್ಲೋಕಗಳಿವೆ.

ವೇದವ್ಯಾಸರ ಸ್ತೋತ್ರ, ಶ‍್ರೀರಾಮದೇವರ ಸ್ತೋತ್ರ, ಶ‍್ರೀಕೃಷ್ಣನ ಸ್ತೋತ್ರ, ವಾಯುದೇವರು ಸಂಕರ್ಷಣ-ಜಯಾ ಇವರಿಂದ ಮಗನಾಗಿ ಹುಟ್ಟುವುದು, ಗರುಡ-ಶೇಷರ ಜನನ, ಜಯ-ವಿಜಯರ ಉತ್ಪತ್ತಿ, ರುದ್ರದೇವರ ಉತ್ಪತ್ತಿ, ದೇವತೆಗಳು ಮಾಡುವ ಸ್ತೋತ್ರ, ಬ್ರಹ್ಮಾಂಡದಲ್ಲಿ ಶ‍್ರೀಹರಿಯಿಂದ ಕೂಡಿಕೊಂಡು ದೇವತೆಗಳ ಪ್ರವೇಶ, ವಾಯುದೇವರ ಗಮನಾಗಮನಗಳಿಂದ ಶರೀರವು ಬೀಳುವುದು ಏಳುವುದೆಂಬ ವಾಯುದೇವರ ಶ್ರೇಷ್ಠತೆ, ತತ್ವಾಭಿಮಾನಿ ದೇವತೆಗಳ ಉತ್ಪತ್ತಿ ಕ್ರಮ; ರಾಕ್ಷಸರು, ಋಷಿಗಳು, ಮನುಷ್ಯರು, ಇತರ ಪ್ರಾಣಿಗಳು ಹುಟ್ಟುವುದು; ಜಗತ್ ಪ್ರವಾಹಕ್ಕೆ ಅಂತ್ಯವೇ ಇಲ್ಲ, ದಿತಿ-ಅದಿತಿದೇವಿಯರಿಂದ ಪ್ರಸವ, ಮತ್ಸ್ಯಾವತಾರ, ಶ್ವೇತವರಾಹ ಅವತಾರ, ನರಸಿಂಹಾವತಾರ, ಕೂರ್ಮಾವತಾರ, ವಾಮನಾವತಾರ, ಪರಶುರಾಮಾವತಾರ; ರಾವಣ-ಕುಂಭಕರ್ಣರ ಜನನ, ಕೃತಾದಿ ನಾಲ್ಕು ಯುಗಗಳ ಕಾಲಾವಧಿ, ಸೂರ್ಯವಂಶಜನಾದ ದಶರಥರಾಜ ಅವನ ಪತ್ನಿ ಕೌಸಲ್ಯೆಯ ನಿಮಿತ್ತದಿಂದ ಶ‍್ರೀಹರಿಯು ರಾಮನಾಗಿ ಅವತರಿಸುವುದು, ವಾಯು ದೇವರು ಅಂಜನಾದೇವಿಯ ನಿಮಿತ್ತದಿಂದ ಹನುಮಂತನಾಗಿ ಅವತರಿಸುವುದು, ಸೂರ್ಯನು ಸುಗ್ರೀವನಾಗಿ, ಆಗ್ನಿಯು ನೀಲನಾಗಿ ಅವತಾರ ಮಾಡಿದುದು, ಭರತ, ಲಕ್ಷ್ಮಣ, ಶತ್ರುಘ್ನರ ಜನನ; ಮಹಾಲಕ್ಷ್ಮಿಯು ಸೀತಾದೇವಿಯಾಗಿ ಅವತರಿಸುವುದು ಮುಂತಾದ ವಿಷಯಗಳು ಮೂರನೇ ಅಧ್ಯಾಯದಲ್ಲಿ ನಿರೂಪಿತ\-ವಾಗಿವೆ.


\section*{ಅಧ್ಯಾಯ\enginline{-}೪}

\begin{verse}
\textbf{ಯದ್ವೃದ್ಧಿರ್ಜನಮೋಹಿನೀ ಮುದಮಿತಾ ಯದ್ದರ್ಶನಾತ್ಸಜ್ಜನಾಃ}\\\textbf{ಯೇನರ್ಷಿಪ್ರಿಯಕಾರಿಣಾ ನಿಶಿಚರೀ ಹನ್ತ್ರಾ ಕ್ರತೂ ರಕ್ಷಿತಃ~।}\\\textbf{ಯೋsಹಲ್ಯಾಂ ಸಪತಿಂ ವ್ಯಧಾತ್‌ ಹರದನುರ್ಭಂಕ್ತ್ವಾಽವಹಜ್ಜಾನಕೀಂ}\\\textbf{ಜೇತಾ ವರ್ತ್ಮನಿ ಭಾರ್ಗವಸ್ಯ ನಗರೀಂ ರಾಮೋ ಗತೋಽವ್ಯಾತ್ಸ ಮಾಮ್~।।}
\end{verse}

ಶ‍್ರೀರಾಮನಾಗಿ ಅವತಾರ ಮಾಡಿದಾಗ ಅಜ್ಞಜನರ ಮೋಹನಾರ್ಥವಾಗಿ ದಿನದಿನಕ್ಕೆ\break ಅಭಿವೃದ್ಧಿ ಹೊಂದಿದಂತೆ ತೋರಿಸಿದ, ತನ್ನ ರೂಪಲಾವಣ್ಯಗಳಿಂದ ಸಜ್ಜನರಿಗೆ ಹರ್ಷವನ್ನುಂಟುಮಾಡಿದ, ವಿಶ್ವಾಮಿತ್ರ ಋಷಿಗಳ ಅಭೀಷ್ಟವನ್ನು ಈಡೇರಿಸಿದ, ರಾಕ್ಷಸಿಯಾದ ತಾಟಕಿಯನ್ನು ಸಂಹರಿಸಿ ವಿಶ್ವಾಮಿತ್ರರ ಯಜ್ಞವನ್ನು ರಕ್ಷಿಸಿದ, ಅಹಲ್ಯೆಯನ್ನು ಪತಿಯೊಡನೆ ಜತೆಗೂಡಿಸಿದ, (ಮಿಥಿಲಾನಗರದಲ್ಲಿ) ಶಿವ ಧನುಸ್ಸನ್ನು ಮುರಿದು ಜಾನಕಿದೇವಿಯನ್ನು ವರಿಸಿದ, ಮಾರ್ಗದಲ್ಲಿ ಪರಶುರಾಮರನ್ನು ಜಯಿಸಿ ಅಯೋಧ್ಯಾ ನಗರಕ್ಕೆ ಹಿಂತಿರುಗಿದ ಶ‍್ರೀರಾಮನು ನನ್ನನ್ನು ಸಲಹಲಿ.

ಈ ಅಧ್ಯಾಯದಲ್ಲಿ ೬೫ ಶ್ಲೋಕಗಳಿವೆ.

ದಶರಥರಾಜನ ಕುಮಾರರಾಗಿ ಜನಿಸಿದವರು ದಿನದಿನಕ್ಕೆ ಅಭಿವೃದ್ಧಿ ಹೊಂದುತ್ತಿರುವುದನ್ನು ಕಂಡು ರಾಜನೂ ಇತರರೂ ಸಂತೋಷಪಡುವುದು, ಯಜ್ಞ ರಕ್ಷಣಾರ್ಥವಾಗಿ ವಿಶ್ವಾಮಿತ್ರರಿಂದ ರಾಮನ ಸಹಾಯಕ್ಕಾಗಿ ಪ್ರಾರ್ಥನೆ, ಲಕ್ಷ್ಮಣನಿಂದ ಸಹಿತನಾದ ರಾಮನು ವಿಶ್ವಾಮಿತ್ರರಿಂದ ಅವರ ಅನುಗ್ರಹಾರ್ಥವಾಗಿ ಅಸ್ತ್ರಮಂತ್ರಗಳನ್ನು ಸ್ವೀಕರಿಸುವುದು, ತಾಟಕಿ-ಸುಬಾಹುಗಳನ್ನು ಸಂಹರಿಸಿ ರಾಮನು ಯಜ್ಞವನ್ನು ರಕ್ಷಿಸಿದುದು. ಮಾರೀಚನನ್ನು ಸಮುದ್ರದಲ್ಲಿ ಬಿಸಾಡುವುದು, ವಿಶ್ವಾಮಿತ್ರರೊಡನೆ ಜನಕರಾಜನ ನಗರಕ್ಕೆ ಪ್ರಯಾಣ, ದಾರಿಯಲ್ಲಿ ಶಿಲೆಯಾಗಿದ್ದ ಅಹಲ್ಯೆಯನ್ನು ತನ್ನ ಕೃಪಾದೃಷ್ಟಿಯಿಂದಲೇ ಮೊದಲಿನಂತೆ ಸ್ತ್ರೀಯನ್ನಾಗಿ ಮಾಡಿ ಗೌತಮರ ಜೊತೆಯಲ್ಲಿ ಬಿಟ್ಟ ವಿಷಯ, ಜನಕರಾಜನ ನಗರದಲ್ಲಿ ಶ‍್ರೀ ರಾಮನ ಸೌಂದರ್ಯರಾಶಿಯನ್ನು ಕಂಡು ಜನರು ಮುಗ್ಧರಾಗುವುದು, ಶಿವಧನುಸ್ಸಿನ ಪೂರ್ವಕಥೆ, ಜನಕರಾಜನ ಸಂಕಲ್ಪ, ಧನುಸ್ಸನ್ನು ಹೆದೆಯೇರಿಸುವಲ್ಲಿ ಎಲ್ಲರ ಪರಾಭವ, ಶಿವಧನುಸ್ಸನ್ನು ಶ‍್ರೀ ರಾಮನು ಕ್ರೀಡೆಯಿಂದ ಭಗ್ನಗೊಳಿಸುವುದು, ಜಾನಕಿದೇವಿಯಿಂದ ಪುಷ್ಪಮಾಲಾರ್ಪಣೆ, ದಶರಥನು ಪರಿವಾರ ಸಮೇತ ಅಯೋಧ್ಯಾ ನಗರದಿಂದ ಬರುವುದು, ಶ‍್ರೀ ರಾಮ-ಸೀತಾ ವಿವಾಹ, ಅಯೋಧ್ಯಾ ನಗರಕ್ಕೆ ಹಿಂತಿರುಗುವಾಗ ಪರಶುರಾಮ ದೇವರ ಭೇಟಿ, ರುದ್ರ ದೇವರು ಶ‍್ರೀ ಹರಿ ಇವರಿಬ್ಬರಲ್ಲಿ ಯಾರು ಶ್ರೇಷ್ಠರೆಂದು ತೀರ್ಮಾನವಾದ ಬಗೆ, ವೈಷ್ಣವ ಧನುಸ್ಸನ್ನು ಶ‍್ರೀ ರಾಮನು ಸ್ವೀಕರಿಸಿ, ಬಾಣವನ್ನು ಜೋಡಿಸಿ ಎಳೆದು ಪರಶುರಾಮ ದೇವರ ಹೊಟ್ಟೆಯಲ್ಲಿ ಬಿಟ್ಟು ಅಲ್ಲಿದ್ದ ಅತುಲನೆಂಬ ಅಸುರನನ್ನು ಸಂಹಾರ ಮಾಡಿದುದು. ಪರಶುರಾಮ ದೇವರು ಹಾಗೂ ಶ‍್ರೀ ರಾಮ ಇವರಲ್ಲಿ ಯಾವ ಭೇದವೂ ಇಲ್ಲವೆಂದು ಪ್ರದರ್ಶನ ಮಾಡುವುದು, ಅಯೋಧ್ಯಾ ನಗರಕ್ಕೆ ತೆರಳಿ ಸೀತಾದೇವಿಯೊಡನೆ ಕ್ರೀಡಿಸುವುದು-ಈ ಬಾಲಕಾಂಡದ ಕಥಾ ವಿವರಣೆಯನ್ನು ಈ ಅಧ್ಯಾಯದಲ್ಲಿ ಕಾಣಬಹುದು.


\section*{ಅಧ್ಯಾಯ\enginline{-}೫}

\begin{verse}
\textbf{ತ್ಯಕ್ತ್ವಾ ರಾಜ್ಯವಿತೊ ವನಂ ವಚನತೋ ಮಾತುಶ್ಚ ಕಾಕಾಕ್ಷಿಗಂ} \\\textbf{ದೈತ್ಯಂ ವ್ಯಸ್ಯ ವಿಕರ್ಣಘೋಣಖಚರೀಬಂಧೂನ್ ಖರಾದೀನ್ಖಲಾನ್~।}\\\textbf{ಮಾರೀಚಂ ಚ ನಿಹತ್ಯ ರಾವಣಹೃತಾಂ ಸೀತಾಂ ವಿಚಿನ್ವನ್ನಿವ} \\\textbf{ಪ್ರಾಪ್ತೋ ವಾಯುಸುತೇನ ಸೂರ್ಯಜಯುಜಾ ರಾಮೋಽವತಾದ್ವಂದಿತಃ~।।}
\end{verse}

ತಾಯಿಯ ಮಾತಿನಂತೆ ರಾಜ್ಯವನ್ನು ಪರಿತ್ಯಜಿಸಿ ಅರಣ್ಯಕ್ಕೆ ತೆರಳಿದ, ಅರಣ್ಯದಲ್ಲಿ ಕಾಗೆಯ ಕಣ್ಣಿನಲ್ಲಿದ್ದ ಕುರಂಗಾಸುರನನ್ನು ಸಂಹರಿಸಿದ, ರಾಕ್ಷಸಿಯಾದ ಶೂರ್ಪಣಖಿಯ ಮೂಗು-ಕಿವಿಗಳನ್ನು ಕತ್ತರಿಸಿದ, ಅವಳ ಬಂಧುಗಳಾದ ಖರನೇ ಮುಂತಾದ ರಾಕ್ಷಸರನ್ನು ಸಂಹರಿಸಿದ, ಮಾರೀಚನನ್ನು ಕೊಂದ, ರಾವಣನಿಂದ ಒಯ್ಯಲ್ಪಟ್ಟ ಸೀತಾದೇವಿಯನ್ನು ಹುಡುಕುವವನಂತೆ ನಟಿಸಿದ, ಹುಡುಕುವ ಕಾಲದಲ್ಲಿ ವಾಯುಕುಮಾರನಾದ ಹನುಮಂತನೊಡನೆ ಇರುವ ಸೂರ್ಯನ ಅಂಶಭೂತನಾದ ಸುಗ್ರೀವನನ್ನು ಭೇಟಿಮಾಡಿದ, ನಂತರ ಅವರಿಂದ ನಮಸ್ಕರಿಸಿ ಕೊಳ್ಳಲ್ಪಟ್ಟ ಶ‍್ರೀರಾಮನು ನಮ್ಮನ್ನು ಸಲಹಲಿ.

ಈ ಅಧ್ಯಾಯದಲ್ಲಿ ೫೧ ಶ್ಲೋಕಗಳಿವೆ.

ಶ‍್ರೀರಾಮನ ಪಟ್ಟಾಭಿಷೇಕಕ್ಕೆ ದಶರಥನ ಯೋಚನೆ, ಮಂಥರೆಯ ಕುತಂತ್ರ, ಅವಳ ಸ್ವರೂಪ ವಿವರಣೆ, ಸೀತಾಲಕ್ಷ್ಮಣರಿಂದ ಸಹಿತನಾಗಿ ಶ‍್ರೀರಾಮನು ಅರಣ್ಯಕ್ಕೆ ತೆರಳುವುದು, ಗುಹನಿಂದ ಪೂಜೆ, ಗಂಗೆಯನ್ನು ದಾಟಿ ಚಿತ್ರಕೂಟಪರ್ವತದಲ್ಲಿ ವಾಸ, ದಶರಥನ ನಿಧನ, ಭರತ-ಶತ್ರುಘ್ನರು ಶ‍್ರೀರಾಮನನ್ನು ನೋಡುವುದು, ಭರತನು ಶ‍್ರೀರಾಮನನ್ನು ಅಯೋಧ್ಯೆಗೆ ಆಹ್ವಾನಿಸುವುದು, ಹದಿನಾಲ್ಕು ವರ್ಷಗಳ ನಂತರ ಬರುವುದಾಗಿ ಶ‍್ರೀರಾಮನ ಆಶ್ವಾಸನೆ, ಪಾದುಕಾ ಪ್ರದಾನ, ಜಯಂತನು ಕಾಗೆಯಾಗಿ ಸೀತಾದೇವಿಗೆ ಹಿಂಸೆಕೊಡಲು ಬರುವುದು, ಅವನ ಕಣ್ಣಿನಲ್ಲಿದ್ದ ಕುರಂಗನೆಂಬ ಅಸುರನ ಸಂಹಾರ, ಶರಭಂಗರು ಮಾಡಿದ ಆತಿಥ್ಯ ಸ್ವೀಕಾರ, ಅವರ ಪ್ರಾಣತ್ಯಾಗ, ಅಗಸ್ತ್ಯರ ಪೂಜಾ ಮತ್ತು ಅವರಿಂದ ಕೊಡಲ್ಪಟ್ಟ ಧನುಸ್ಸಿನ ಸ್ವೀಕಾರ, ಖರ-ದೂಷಣರಿಂದ ಸಹಿತಳಾದ ಶೂರ್ಪಣಖಿಯ ಆಗಮನ, ಲಕ್ಷ್ಮಣನಿಂದ ಶೂರ್ಪಣಖಿಯ ಕಿವಿ-ಮೂಗುಗಳ ಛೇದನ, ಖರ-ದೂಷಣಾದಿಗಳ ಸಂಹಾರ, ಸುವರ್ಣಮಯವಾದ ಜಿಂಕೆಯ ರೂಪದಿಂದ ಮಾರೀಚನ ಆಗಮನ, ಸೀತಾ ದೇವಿಯು ಜಿಂಕೆಗಾಗಿ ಆಸೆಪಟ್ಟವಳಂತೆ ನಟಿಸಿದುದು, ಶ‍್ರೀರಾಮನಿಂದ ಮಾರೀಚನ ಸಂಹಾರ, ಶ‍್ರೀಹರಿಯ ವ್ಯಾಪಾರದಂತೆಯೇ ಶ‍್ರೀಮಹಾಲಕ್ಷ್ಮಿಯೂ ಸಹ ನಡೆಯುತ್ತಾಳೆಂಬ ಪ್ರಮೇಯ, ಯತಿಯ ವೇಷದಿಂದ ರಾವಣನು ಸೀತಾಕೃತಿಯನ್ನು ಅಪಹರಿಸುವುದು, ಸೀತಾದೇವಿಯು ಕೈಲಾಸಕ್ಕೆ ತೆರಳುವುದು, ಸೀತಾಕೃತಿಯನ್ನು ರಾವಣನು ಲಂಕೆಯಲ್ಲಿ ಇಡುವುದು, ಸೀತಾದೇವಿಯನ್ನು ಹುಡುಕುವವನಂತೆ ಶ‍್ರೀರಾಮನು ನಟಿಸುವುದು, ಜಟಾಯುವಿನ ದೇಹ ಸಂಸ್ಕಾರ, ಕಬಂಧನ ಬಾಹುಗಳ ಛೇದನ, ಶಬರಿಯ ವರ್ಣನೆ, ಅವಳಿಗೆ ಸದ್ಗತಿಯನ್ನು ಶ‍್ರೀರಾಮನು ದಯಪಾಲಿಸುವುದು, ರಾಮಲಕ್ಷ್ಮಣರು ಋಷ್ಯಮೂಕ ಪರ್ವತಕ್ಕೆ ಬರುವುದು, ಹನುಮಂತನು ಶ‍್ರೀರಾಮನ ಪಾದಕ್ಕೆ ವಂದಿಸುವುದು- ಈ ವಿಷಯಗಳು ಈ ಅಧ್ಯಾಯದಲ್ಲಿ ನಿರೂಪಿತವಾಗಿವೆ.


\section*{ಅಧ್ಯಾಯ\enginline{-}೬}

\begin{verse}
\textbf{ಸುಗ್ರೀವೇಣ ಸಖಿತ್ವಮಾಪ್ಯ ಶಪಥಂ ಕೃತ್ವಾ ವಧೇ ವಾಲಿನಃ}\\\textbf{ತಾಲಾನ್ಸಪ್ತ ವಿಭಿದ್ಯ ವಾಲಿನಿಧನಂ ಕೃತ್ವಾ ಸ್ವರಾಜ್ಯೇ ಸ್ಥಿತಮ್~।}\\\textbf{ಮಾರ್ತಾಂಡಿಂ ಚ ವಿಧಾಯ ಮಾರುತಿಯುಜಾ ಯಾಮ್ಯಾಂ ದಿಶಂ ಗಚ್ಛತಾ}\\\textbf{ಸೀತಾನ್ವೇಷಣಮಿಚ್ಛತಾಬ್ಧಿತರಣೇ ರಾಮೋಽವತಾತ್ಸೋಸ್ತು ನಃ~।।}
\end{verse}

ಸುಗ್ರೀವನೊಡನೆ ಸ್ನೇಹವನ್ನು ಬೆಳೆಸಿದ, ವಾಲಿಯ ಸಂಹಾರಕ್ಕೆ ಪ್ರತಿಜ್ಞೆ ಮಾಡಿದ, ಏಳು ತಾಳವೃಕ್ಷಗಳನ್ನು (ಒಂದೇ ಬಾಣದಿಂದ) ಕತ್ತರಿಸಿದ, ವಾಲಿಯನ್ನು ಸಂಹರಿಸಿ ಸುಗ್ರೀವನನ್ನು ರಾಜ್ಯದಲ್ಲಿ ಸ್ಥಾಪಿಸಿದ, ಸೀತೆಯನ್ನು ಹುಡುಕುವುದಕ್ಕಾಗಿ ಸಮುದ್ರವನ್ನು ದಾಟಲು ದಕ್ಷಿಣ ದಿಕ್ಕನ್ನು ಕುರಿತು ಹೋಗುವ ಹನುಮಂತ ದೇವರಿಂದ ಸ್ತುತಿಸಲ್ಪಟ್ಟ ಶ‍್ರೀರಾಮನು ನಮ್ಮನ್ನು ಸಲಹಲಿ.

ಈ ಅಧ್ಯಾಯದಲ್ಲಿ ೫೯ ಶ್ಲೋಕಗಳಿವೆ.

ಹನುಮಂತ ದೇವರ ಭುಜಗಳ ಮೇಲೆ ಕುಳಿತುಕೊಂಡು ಶ‍್ರೀರಾಮಲಕ್ಷ್ಮಣರು ಸುಗ್ರೀವನ ಬಳಿಗೆ ಬರುವುದು, ಅಗ್ನಿ ಸಾಕ್ಷಿಯಾಗಿ ಶ‍್ರೀರಾಮ-ಸುಗ್ರೀವರ ಸಖ್ಯ, ವಾಲಿಯ ಸಂಹಾರಕ್ಕಾಗಿ ಶ‍್ರೀ ರಾಮನ ಪ್ರತಿಜ್ಞೆ, ದುಂದುಭಿಯ ಶರೀರವನ್ನು ನೂರು ಯೋಜನ ದೂರಕ್ಕೆ ಬಿಸಾಡುವುದು, ಒಂದೇ ಬಾಣದಿಂದ ಏಳು ತಾಳವೃಕ್ಷಗಳನ್ನು ಛೇದಿಸುವುದು, ವಾಲಿ-ಸುಗ್ರೀವರ ಕದನ, ಶ‍್ರೀರಾಮನಿಂದ ವಾಲಿಯ ಸಂಹಾರ, ಸುಗ್ರೀವನಿಗೆ ರಾಜ್ಯಾಭಿಷೇಕ, ಅಪರಿಮಿತ ಸಂಖ್ಯೆಯಲ್ಲಿ ಕಪಿಗಳು ಬಂದು ಸಭೆ ಸೇರುವುದು, ಎಲ್ಲ ದಿಕ್ಕುಗಳಿಗೂ ಕಪಿಗಳನ್ನು ಕಳಿಸುವುದು, ಶ‍್ರೀ ರಾಮನು ಮುದ್ರೆಯುಂಗುರವನ್ನು ಕೊಟ್ಟು ಹನುಮಂತ ದೇವರನ್ನು ದಕ್ಷಿಣ ದಿಕ್ಕಿಗೆ ಕಳಿಸುವುದು, ಕೆಲವು ಕಪಿಗಳು ಮಯನಿಂದ ನಿರ್ಮಿತವಾದ ಗುಹೆಗೆ ಹೋಗುವುದು, ಹನುಮಂತ ದೇವರಿಂದ ಕಪಿಗಳಿಗೆ ಹಿತೋಕ್ತಿ, ಸಮುದ್ರದ ಬಳಿಗೆ ಬಂದು ಕಪಿಗಳು ನಿರಶನವ್ರತ ಮಾಡುವುದು, ಸಂಪಾತಿಯ ವೃತ್ತಾಂತ, ಸಮುದ್ರವನ್ನು ದಾಟುವುದರಲ್ಲಿ ಕಪಿಗಳು ತಮ್ಮ ತಮ್ಮ ಸಾಮರ್ಥ್ಯವನ್ನು ಹೇಳಿ ಕೊಳ್ಳುವುದು, ಎಲ್ಲ ಕಪಿಗಳೂ ಒಂದಾಗಿ ಬಂದು ಹನುಮಂತ ದೇವರನ್ನು ಸಮುದ್ರವನ್ನು ದಾಟಿ ಸೀತೆಯನ್ನು ನೋಡಿ ಬರಲು ಪ್ರಾರ್ಥಿಸುವುದು, ಈ ವಿಷಯಗಳು ಈ ಅಧ್ಯಾಯದಲ್ಲಿ ನಿರೂಪಿತವಾಗಿವೆ.


\section*{ಅಧ್ಯಾಯ\enginline{-}೭}

\begin{verse}
\textbf{ಯಸ್ಯ ಶ‍್ರೀಹನುಮಾನನುಗ್ರಹಬಲಾತ್ತೀರ್ಣಾಂಬುಧಿರ್ಲಿಲಯಾ }\\\textbf{ಲಂಕಾಂ ಪ್ರಾಪ್ಯ ನಿಶಾಮ್ಯ ರಾಮದಯಿತಾಂ ಭಂಕ್ತ್ವಾವನಂ ರಾಕ್ಷಸಾನ್।} \\\textbf{ಅಕ್ಷಾದೀನ್ವಿನಿಹತ್ಯ ವೀಕ್ಷ್ಯ ದಶಕಂ ದಗ್ಧ್ವಾ ಪುರೀಂ ತಾಂ ಪುನಃ}\\\textbf{ತೀರ್ಣಾಬ್ಧಿಃ ಕಪಿಭಿರ್ಯುತೋ ಯಮನಮತ್ತಂ ರಾಮಚಂದ್ರಂ ಭಜೇ~।।}
\end{verse}

ಯಾವ ಶ‍್ರೀರಾಮನ ಅನುಗ್ರಹದಿಂದ ಶ‍್ರೀಹನುಮಂತನು ನಿರಾಯಾಸವಾಗಿ ಸಮುದ್ರವನ್ನು ದಾಟಿ, ಲಂಕೆಗೆ ಹೋಗಿ, ಸೀತಾಕೃತಿಯನ್ನು ನೋಡಿ, ಅಶೋಕವನವನ್ನು ಧ್ವಂಸ ಮಾಡಿ, ಅಕ್ಷಯಕುಮಾರನೇ ಮೊದಲಾದ ಅನೇಕ ರಾಕ್ಷಸರನ್ನು ಸಂಹರಿಸಿ, ರಾವಣನನ್ನು ನೋಡಿ, ಲಂಕಾಪಟ್ಟಣವನ್ನು ಸುಟ್ಟು, ಪುನಃ ಸಮುದ್ರವನ್ನು ದಾಟಿ ಬಂದು, ಕಪಿಗಳೊಡನೆ ಶ‍್ರೀರಾಮನನ್ನು ನಮಸ್ಕರಿಸಿದನೋ, ಅಂತಹ ಶ‍್ರೀರಾಮಚಂದ್ರನನ್ನು ಸ್ತುತಿಸುತ್ತೇನೆ.

ಈ ಅಧ್ಯಾಯದಲ್ಲಿ ೫೦ ಶ್ಲೋಕಗಳಿವೆ.

ಹನುಮಂತನು ಶ‍್ರೀ ರಾಮನಿಗೆ ನಮಸ್ಕರಿಸಿ ಮಹೇಂದ್ರ ಪರ್ವತವನ್ನು ತುಳಿದು ಹಾರಿದುದು, ಮೈನಾಕಪರ್ವತವು ಉಪಚರಿಸುವುದು, ಕದ್ರೂದೇವಿಯ ಮುಖದಲ್ಲಿ ಪ್ರವೇಶಿಸಿ ಪುನಃ ಹೊರಕ್ಕೆ ಬಂದುದು, ಸಿಂಹಿಕಾ ಎಂಬ ರಾಕ್ಷಸಿಯ ಶರೀರವನ್ನು ಪ್ರವೇಶಿಸಿ ಅದನ್ನು ಸೀಳಿದುದು, ಲಂಬನಾಮಕ ಗಿರಿಗೆ ಬಂದು ತನ್ನ ದೇಹವನ್ನು ಮಾರ್ಜಾಲದಷ್ಟು ಸೂಕ್ಷ್ಮವಾಗಿ ಮಾಡಿಕೊಂಡದ್ದು, ಅಶೋಕವನದಲ್ಲಿ ಶಿಂಶುಪಾವೃಕ್ಷದ ಕೆಳಗೆ ಸೀತಾಕೃತಿಯನ್ನು ನೋಡಿದುದು, ಮುದ್ರೆಯುಂಗುರವನ್ನು ಕೊಟ್ಟು ಸೀತೆಯಿಂದ ಶಿರೋರತ್ನವನ್ನು ಸ್ವೀಕರಿಸಿದುದು, ದೈತ್ಯರ ಮೋಹನಾರ್ಥವಾಗಿ ಕೃತ್ಯಗಳನ್ನು ಮಾಡಿದುದು, ಸಿಂಹನಾದದಿಂದ ಅಶೋಕವನವನ್ನು ನಾಶಮಾಡುವುದು, ರಾಕ್ಷಸರು ಬಂದು ಹನುಮಂತನನ್ನು ಆವರಿಸುವುದು, ಏಳುಮಂದಿ ರಾವಣನ ಮಂತ್ರಿ ಪುತ್ರರನ್ನು ಸಂಹರಿಸುವುದು, ಅಕ್ಷಯಕುಮಾರನು ಬಂದು ಯುದ್ಧ ಮಾಡುವುದು, ಅವನನ್ನು ಎತ್ತಿ ಹಾರಿಸಿ ಸಂಹರಿಸುವುದು, ನಂತರ ಇಂದ್ರಜಿತುವು ಬಂದು ಬ್ರಹ್ಮಾಸ್ತ್ರವನ್ನು ಪ್ರಯೋಗ ಮಾಡುವುದು, ಅದರಲ್ಲಿ ತಾನು ಸಿಕ್ಕಿದಂತೆ ನಟಿಸಿದುದು, ರಾವಣನ ಆಸ್ಥಾನಕ್ಕೆ ಬಂದು, ರಾವಣನನ್ನು ಧಿಕ್ಕರಿಸಿ ಶ‍್ರೀರಾಮನನ್ನು ಸ್ತೋತ್ರಮಾಡುವುದು, ಹನುಮಂತನ ಬಾಲಕ್ಕೆ ರಾವಣನು ಬೆಂಕಿಯನ್ನು ಇಡಿಸುವುದು, ಈ ಅಗ್ನಿಯಿಂದ ಲಂಕಾಪಟ್ಟಣವನ್ನು ಸುಟ್ಟು, ಪುನಃ ಸಮುದ್ರವನ್ನು ದಾಟಿ ಬಂದು ಕಪಿಶ್ರೇಷ್ಠರೊಂದಿಗೆ ಶ‍್ರೀರಾಮನನ್ನು ನಮಸ್ಕರಿಸಿ ಚೂಡಾಮಣಿಯನ್ನು ಸಮರ್ಪಿಸುವುದು, ಈ ಸೇವೆಯಿಂದ ಪ್ರೀತನಾದ ಶ‍್ರೀ ರಾಮನು ಹನುಮಂತನನ್ನು ಆಲಂಗಿಸಿಕೊಳ್ಳು ವುದು, ಈ ವಿಷಯಗಳು ನಿರೂಪಿತವಾಗಿವೆ.


\section*{ಅಧ್ಯಾಯ\enginline{-}೮}

\begin{verse}
\textbf{ಸಿಂಧುಂ ದಕ್ಷಿಣಮಾಗತೋ ದಶಮುಖಭ್ರಾತ್ರಿಷ್ಟತೋಽಭ್ಯರ್ಥಿತೋ} \\\textbf{ಬಧ್ವಾ ಸೇತುಮವಾಪ್ಯ ರಾಕ್ಷಸಪುರೀಂ ಸೈನ್ಯೈಃ ಕಪೀನಾಂ ಯುತಃ~। }\\\textbf{ಹತ್ವೇಂದ್ರಾರಿಕುಂಭಕರ್ಣದಶಕಾದೀನ್ ರಾಕ್ಷಸಾನ್ ಜಾನಕೀಂ }\\\textbf{ಆಧಾಯಾಪ್ಯ ಪುರೀಂ ಸ್ವರಾಜ್ಯ ಪದವೀಂ ಪ್ರಾಪ್ತೋಽವತಾದ್ರಾಘವಃ~।}
\end{verse}

ದಕ್ಷಿಣಸಮುದ್ರಕ್ಕೆ ಆಗಮಿಸಿದ, ರಾವಣನ ತಮ್ಮನಾದ ವಿಭೀಷಣನಿಂದ ಪ್ರಾರ್ಥಿತನಾಗಿ ಅವನ ಇಷ್ಟವನ್ನು ಪ್ರದಾನಮಾಡಿದ, ಸಮುದ್ರಕ್ಕೆ ಸೇತುವೆಯನ್ನು ಕಟ್ಟಿ ಕಪಿಗಳ ಸೈನ್ಯದಿಂದ ಕೂಡಿಕೊಂಡು ರಾಕ್ಷಸನಗರವಾದ ಲಂಕಾಪಟ್ಟಣವನ್ನು ಪ್ರವೇಶಮಾಡಿ, ಇಂದ್ರಜಿತ್, ಕುಂಭಕರ್ಣ, ರಾವಣನೇ ಮುಂತಾದ ರಾಕ್ಷಸರನ್ನು ಸಂಹರಿಸಿ, ಸೀತಾದೇವಿಯನ್ನು ಹೊಂದಿ, ಅಯೋಧ್ಯಾನಗರಕ್ಕೆ ಬಂದು, ತನ್ನದೇ ಆದ ರಾಜ್ಯ ಪದವಿಯನ್ನು ಸ್ವೀಕರಿಸಿದ ಶ‍್ರೀರಾಮನು ನಮ್ಮನ್ನು ಸಲಹಲಿ.

ಈ ಅಧ್ಯಾಯದಲ್ಲಿ ೨೪೮ ಶ್ಲೋಕಗಳಿವೆ.

ಲಕ್ಷ್ಮಣ, ಅಂಗದ, ಹನುಮಂತ ಇವರೊಡನೆ ಶ‍್ರೀ ರಾಮನು ದಕ್ಷಿಣ ಸಮುದ್ರಕ್ಕೆ ಬರುವುದು, ವಿಭೀಷಣನಿಗೆ ಆಶ್ವಾಸನೆ ನೀಡುವುದು, ಸಮುದ್ರ ರಾಜನು ಶ‍್ರೀ ರಾಮನನ್ನು ಪ್ರಾರ್ಥಿಸುವುದು, ಮರುಭೂಮಿಯನ್ನು ತನ್ನ ಬಾಣದ ಹೊಡೆತದಿಂದ ಫಲವತ್ತಾಗಿ ಮಾಡುವುದು, ಸಮುದ್ರಕ್ಕೆ ಸೇತುವೆಯನ್ನು ಕಟ್ಟಿ ಲಂಕಾಪಟ್ಟಣಕ್ಕೆ ಕಪಿ ಸೈನ್ಯದೊಡನೆ ಪ್ರವೇಶ ಮಾಡುವುದು, ಅಂಗದನನ್ನು ಸಂಧಾನಕ್ಕೆ ರಾವಣನ ಬಳಿಗೆ ಕಳಿಸುವುದು, ರಾವಣನು ಒಂದೊಂದು ದಿಕ್ಕಿಗೆ ಒಬ್ಬೊಬ್ಬ ಸೇನಾನಿಯನ್ನೂ ಅಪರಿಮಿತವಾದ ಸೈನ್ಯವನ್ನೂ ಕಳಿಸುವುದು, ನೀಲ, ಹನುಮಂತ, ಅಂಗದ, ಸುಗ್ರೀವ, ಜಾಂಬವಂತ ಮುಂತಾದ ಕಪಿ ಶ್ರೇಷ್ಠರಿಂದ ಪ್ರಹಸ್ಯ, ಅಕಂಪನ, ನಿಕುಂಭ, ಕುಂಭ ಮೊದಲಾದ ರಾಕ್ಷಸರ ಹನನ, ಲಕ್ಷ್ಮಣನಿಂದ ಅತಿಕಾಯನ ಮೇಲೆ ಬ್ರಹ್ಮಾಸ್ತ್ರ ಪ್ರಯೋಗ, ಶ‍್ರೀ ರಾಮನಿಂದ ಮಕ ರಾಕ್ಷಸನ ಸಂಹಾರ, ರಾವಣನೇ ಯುದ್ಧಕ್ಕೆ ಬರುವುದು, ಹನುಮಂತನಿಂದ ರಾವಣನಿಗೆ ಪ್ರಹಾರಗಳು, ರಾವಣನ ಬಾಣದ ಪೆಟ್ಟಿನಿಂದ ಲಕ್ಷ್ಮಣನು ಮೂರ್ಛಿತನಾಗುವುದು, ಶ‍್ರೀರಾಮನ ಹಸ್ತಸ್ಪರ್ಶದಿಂದ ಲಕ್ಷ್ಮಣನು ಏಳುವುದು, ಕುಂಭಕರ್ಣನನ್ನು ನಿದ್ದೆಯಿಂದ ಎಚ್ಚರಿಸಿ ಯುದ್ಧಕ್ಕೆ ಕಳಿಸುವುದು, ಕಪಿಗಳು ಭಯದಿಂದ ಓಡುವುದು, ಸುಗ್ರೀವ-ಕುಂಭಕರ್ಣರ ಯುದ್ಧ, ಹನುಮಂತದೇವರಿಂದ ಕುಂಭಕರ್ಣನಿಗೆ ಮೂರ್ಛೆ, ಶ‍್ರೀರಾಮನು ಕುಂಭಕರ್ಣನನ್ನು ಸಂಹರಿಸುವುದು, ಇದರಿಂದ ರಾವಣನಿಗಾದ ಶೋಕ, ಇಂದ್ರಜಿತುವು ಕಪಿಗಳನ್ನು ಸರ್ಪಾಸ್ತ್ರದಿಂದ ಕಟ್ಟಿ ಹಾಕುವುದು, ಗರುಡನ ರೆಕ್ಕೆಗಳ ಗಾಳಿ ಹೊಡೆತದಿಂದಲೇ ಕಪಿಗಳ ಬಂಧನ ನಿವೃತ್ತಿ, ಪುನಃ ಇಂದ್ರಜಿತುವಿನಿಂದ ಬಿಡಲ್ಪಟ್ಟ ಅಸ್ತ್ರಗಳಿಂದ ಲಕ್ಷ್ಮಣನೂ ಸೇರಿದಂತೆ ಕಪಿ ಶ್ರೇಷ್ಠರುಗಳು ಮೂರ್ಛೆ\break ಹೋಗುವುದು, ಹನುಮಂತನು ಗಂಧಮಾದನ ಪರ್ವತವನ್ನು ಕ್ಷಣದೊಳಗೆ ತರುವುದು, ಅದರ ವಾಯುಸ್ಪರ್ಶದಿಂದ ಎಲ್ಲರೂ ಏಳುವುದು, ಶ‍್ರೀರಾಮನಿಂದ ಇಂದ್ರಜಿತುವಿನ ಪರಾಭವ, ಲಕ್ಷ್ಮಣ-ಇಂದ್ರಜಿತು ಇವರ ಯುದ್ಧ, ಇಂದ್ರಜಿತುವಿನ ಮರಣ, ಶ‍್ರೀರಾಮನಿಂದ ಅನೇಕ ರಾಕ್ಷಸರ ಸಂಹಾರ, ಮಹೋದರ, ಮಹಾಪಾರ್ಶ್ವ, ವಿರೂಪನೇತ್ರ, ಯೂಪನೇತ್ರ ಇವರ ಮರಣ, ಶ‍್ರೀರಾಮ-ರಾವಣರ ಘೋರವಾದ ಯುದ್ಧ, ರಾವಣನಿಂದ ಬಿಡಲ್ಪಟ್ಟ ಶಕ್ಯಾಖ್ಯ ಆಯುಧದಿಂದ ಲಕ್ಷ್ಮಣನು ಪುನಃ ಮೂರ್ಛಿತನಾಗುವುದು, ಹನುಮಂತನು ಪುನಃ ಶ‍್ರೀರಾಮನ ಆಜ್ಞೆಯಂತೆ ಔಷಧಗಳನ್ನು ತರುವುದು, ಮೃತಸಂಜೀವಿನಿ ಔಷಧದಿಂದ ಲಕ್ಷ್ಮಣನು ಬಲಾಡ್ಯನಾಗುವುದು, ಇಂದ್ರನು ಶ‍್ರೀರಾಮನಿಗೆ ರಥವನ್ನು ಕಳಿಸುವುದು, ಶ‍್ರೀರಾಮನಿಂದ ರಾವಣನ ಸಂಹಾರ, ಬ್ರಹ್ಮದೇವರ ಸ್ತೋತ್ರ, ರುದ್ರ ದೇವರು ಶ‍್ರೀರಾಮನನ್ನು ಯುದ್ದಕ್ಕಾಗಿ ಕರೆಯುವುದು, ಸೀತಾಕೃತಿಯು ಅಗ್ನಿ ಪ್ರವೇಶ ಮಾಡುವುದು, ನಿಜವಾದ ಸೀತಾದೇವಿಯು ಕೈಲಾಸದಿಂದ ಬರುವುದು, ಶ‍್ರೀರಾಮನು ಮೃತರಾದ ಕಪಿಗಳನ್ನು ಪುನರುಜ್ಜೀವಿಸುವುದು, ವಿಭೀಷಣನಿಂದ ಕೊಡಲ್ಪಟ್ಟ ಪುಷ್ಪಕ ವಿಮಾನದಲ್ಲಿ ಎಲ್ಲ ಕಪಿಗಳಿಂದಲೂ ಯುಕ್ತನಾಗಿ ಶ‍್ರೀರಾಮನು ಅಯೋಧ್ಯೆಗೆ ಪ್ರಯಾಣ ಮಾಡುವುದು, ಹನುಮಂತನು ಭರತನಿಗೆ ಮುಂಚೆಯೇ ಈ ವಿಷಯವನ್ನು ತಿಳಿಸುವುದು, ಭರತನು ಅತ್ಯಂತ ಭಕ್ತಿಯಿಂದ ಶ‍್ರೀರಾಮನನ್ನು ಸ್ವಾಗತಿಸಿ ನಮಸ್ಕರಿಸುವುದು, ಶ‍್ರೀರಾಮನ ಪಟ್ಟಾಭಿಷೇಕ, ಶ‍್ರೀರಾಮನು ಹನುಮಂತನ ಸೇವೆಯನ್ನು ಹೊಗಳುವುದು, ಬ್ರಹ್ಮಪದವಿಯ ವಾಗ್ದಾನ, ಹನುಮಂತದೇವರು ಶ‍್ರೀರಾಮನ ಪಾದಸೇವೆಯನ್ನು ಮಾತ್ರ ಬಯಸುವುದು ಮತ್ತು ಶ‍್ರೀರಾಮನನ್ನು ಸ್ತೋತ್ರ ಮಾಡುವುದು, ಶ‍್ರೀರಾಮನು ಹನುಮಂತನನ್ನು ಆಲಂಗಿಸಿಕೊಳ್ಳುವುದು ಈ ವಿಷಯಗಳು ಈ ಅಧ್ಯಾಯದಲ್ಲಿ ನಿರೂಪಿತವಾಗಿವೆ.


\section*{ಅಧ್ಯಾಯ\enginline{-}೯}

\begin{verse}
\textbf{ಪ್ರಾಪ್ತಃ ಸಾಮ್ರಾಜ್ಯ ಲಕ್ಷ್ಮೀಂ ಪ್ರಿಯತಮಭರತಂ ಯವರಾಜ್ಯೇಽಭಿಷಿಚ್ಯ}\\\textbf{ಸ್ವೀಯಾನ್ರಕ್ಷನ್ ಸುತೌ ದ್ವೌ ಜನಕದುಹಿತರಿ ಪ್ರಾಪ್ಯಯಜ್ಞ್ಯೈರ್ಯಜನ್ ಸ್ವಮ್~।}\\\textbf{ಸೀತಾಹೇತೋರ್ವಿಮೋಹ್ಯ ಕ್ಷಿತಿಜದಿತಿಸುತಾನಾರ್ಥಿತೋ ದೇವಸಂಘೈಃ }\\\textbf{ಸದ್ಭಿರ್ಯುಕ್ತೋ ಹನೂಮದ್ವರದ ಉಪಗತಃ ಸ್ವಂ ಪದಂ ಪಾತು ರಾಮಃ~।।}
\end{verse}

ರಾಜ್ಯವನ್ನು ಪಡೆದು, ತನಗೆ ಪ್ರಿಯತಮನಾದ ಭರತನನ್ನು ಯುವರಾಜನನ್ನಾಗಿ ಅಭಿಷೇಕಮಾಡಿ, ಪ್ರಜೆಗಳನ್ನು ಪರಿಪಾಲಿಸಿ, ಸೀತಾದೇವಿಯಿಂದ ಇಬ್ಬರು ಗಂಡುಮಕ್ಕಳನ್ನು ಪಡೆದು, ಯಜ್ಞದ ನಿಮಿತ್ತದಿಂದ ತನ್ನನ್ನೇ ಪೂಜಿಸಿಕೊಂಡು, ಸೀತಾದೇವಿಯ ಕಾರಣದಿಂದ ದೈತ್ಯರನ್ನು ಮೋಹಗೊಳಿಸಿ, ದೇವತಾಸಮೂಹಗಳಿಂದ ಪ್ರಾರ್ಥಿಸಲ್ಪಟ್ಟು, ಸಜ್ಜನರಿಂದ ಯುಕ್ತನಾಗಿ ತನ್ನ ಸ್ಥಾನವಾದ ವೈಕುಂಠ ಲೋಕಕ್ಕೆ ತೆರಳಿದ, ಹನುಮಂತದೇವರಿಗೆ ವರಪ್ರದಾನ ಮಾಡಿದ, ಶ‍್ರೀರಾಮನು ರಕ್ಷಿಸಲಿ.

ಈ ಅಧ್ಯಾಯದಲ್ಲಿ ೧೪೩ ಶ್ಲೋಕಗಳಿವೆ.

ಲಕ್ಷ್ಮಣನು ಯುವರಾಜ್ಯ ಪದವಿಯನ್ನು ಬೇಡವೆಂದು ಹೇಳಿ ಶ‍್ರೀರಾಮನ ಪಾದಸೇವೆಯನ್ನು ಮಾತ್ರ ಅಪೇಕ್ಷಿಸಿದುದು, ಭರತನು ಯುವರಾಜನಾಗುವುದು. ಶ‍್ರೀರಾಮನ ರಾಜ್ಯಭಾರಕ್ರಮದ ವಿವರಣೆ, ಇಡೀ ರಾಜ್ಯವು ಸ್ವರ್ಗಾದಿ ಲೋಕಗಳನ್ನೂ ಮೀರಿಸುವಂತೆ ಇದ್ದುದು, ಜನರ ಸೌಖ್ಯ, ಶ‍್ರೀರಾಮ-ಸೀತಾದೇವಿಯರ ಕ್ರೀಡೆ, ಕುಶ-ಲವರ ಜನನ, ಭರತನಿಂದ ಲವಣಾಸುರನ ಸಂಹಾರ, ಶ‍್ರೀರಾಮನು ಯಜ್ಞವನ್ನು ಆಚರಿಸುವುದು. ಜಂಘನೆಂಬ ಅಸುರನು ಶೂದ್ರನಾಗಿ ತಪಸ್ಸನ್ನು ಆಚರಿಸುವುದು, ಅದರ ಫಲವಾಗಿ ಬ್ರಾಹ್ಮಣನ ಪುತ್ರನ ಮರಣ, ಶ‍್ರೀರಾಮನಿಂದ ಅಸುರನ ಹನನ, ಬ್ರಾಹ್ಮಣನ ಪುತ್ರನನ್ನು ಬದುಕಿಸುವುದು, ಶ‍್ರೀರಾಮನು ಅಗಸ್ತ್ಯರಿಂದ ಮಾಲೆಯನ್ನು ಸ್ವೀಕರಿಸುವುದು, ಸುರಾಣಕರು ಭೂಮಿಯಲ್ಲಿ ಹುಟ್ಟಿ ಶ‍್ರೀರಾಮನನ್ನೂ ಸೀತಾದೇವಿಯನ್ನೂ ಅಗಲಿಸುವುದು, ಸೀತಾದೇವಿಯು ಇತರರಿಗೆ ಅದೃಶ್ಯಳಾಗಿ ೭೦೦ ವರ್ಷ ಶ‍್ರೀರಾಮನೊಡನೆ ವಿಹಾರಮಾಡಿದುದು, ರುದ್ರದೇವರು ಬಂದು ಶ‍್ರೀರಾಮನಿಗೆ ಅವತಾರಕಾರ್ಯ ಪೂರ್ತಿಯಾದ ಕಾರಣದಿಂದ ವೈಕುಂಠಕ್ಕೆ ತೆರಳಲು ವಿಜ್ಞಾಪಿಸಿಕೊಳ್ಳುವುದು, ದೂರ್ವಾಸಮುನಿಗಳಿಗೆ ಶ‍್ರೀರಾಮನಿಂದ ಭೋಜನ ವ್ಯವಸ್ಥೆ, ಲಕ್ಷ್ಮಣನು ತನ್ನ ಲೋಕಕ್ಕೆ ತೆರಳುವದು, ಶ‍್ರೀರಾಮನು ಸರ್ವರನ್ನೂ ವೈಕುಂಠಲೋಕಕ್ಕೆ ಆಹ್ವಾನಿಸಿ ಡಂಗುರ ಹೊಡೆಸುವುದು, ಕುಶನನ್ನು ರಾಜನನ್ನಾಗಿಯೂ, ಲವನನ್ನು ಯುವರಾಜನನ್ನಾಗಿಯೂ ಸ್ಥಾಪಿಸುವುದು, ಹನುಮಂತನನ್ನು ಆಲಂಗಿಸಿಕೊಳ್ಳುವುದು. ಹನುಮಂತ ದೇವರ ಸ್ತೋತ್ರ, ಸರ್ವಾಲಂಕಾರಭೂಷಿತನಾದ ಶ‍್ರೀರಾಮನು ಖಗಾದಿಗಳಿಂದ ಯುಕ್ತನಾಗಿ ಉತ್ತರ ದಿಕ್ಕಿಗೆ ಪ್ರಯಾಣ ಮಾಡುವುದು, ಶ‍್ರೀರಾಮನ ರೂಪ ವರ್ಣನೆ, ಸೀತಾದೇವಿಯು ಶ‍್ರೀ-ಹ್ರೀ ಎಂಬ ಎರಡು ರೂಪಗಳಿಂದ ಶ‍್ರೀರಾಮನಿಗೆ ಚಾಮರವನ್ನು ಬೀಸುವುದು, ಭರತ-ಶತ್ರುಘ್ನರು ತಮ್ಮ ಮೂಲರೂಪಗಳೊಂದಿಗೆ ಸೇರುವುದು, ಶ‍್ರೀರಾಮನ ಆಜ್ಞೆಯಂತೆ ಬ್ರಹ್ಮದೇವರು ಇತರರಿಗೆ ಉಚಿತವಾದ ಸ್ಥಾನಗಳನ್ನು ದಯಪಾಲಿಸುವುದು, ಶ‍್ರೀರಾಮನು ಸ್ವರ್ಗಲೋಕದಲ್ಲಿ ಒಂದು ರೂಪದಿಂದಲೂ, ಬ್ರಹ್ಮದೇವರ ಸಭೆಯಲ್ಲಿ ಒಂದು ರೂಪದಿಂದಲೂ, ಮೂರನೇ ರೂಪದಿಂದ ವೈಕುಂಠ ಲೋಕದಲ್ಲೂ ಇರುವುದು, ಎಲ್ಲ ದೇವತೆಗಳೂ ತಮ್ಮ ತಮ್ಮ ಸ್ಥಾನಗಳಿಗೆ ತೆರಳುವುದು, ಹನುಮಂತ ದೇವರು ಮೂಲರಾಮಾಯಣಾದಿ ಗ್ರಂಥಗಳನ್ನು ವ್ಯಾಖ್ಯಾನಮಾಡುತ್ತಾ ಶ‍್ರೀರಾಮನನ್ನು ಸ್ತುತಿಸುವುದು, ಗ್ರಂಥಕ್ಕೆ ಪ್ರಾಮಾಣ್ಯಸ್ಥಾಪನೆ, ನಿರ್ಣಯದ ಪ್ರಯೋಜನ, ವಿಪರೀತ ಜ್ಞಾನದಿಂದಾಗುವ ಅನರ್ಥ, ತಾತ್ಪರ್ಯ ನಿರ್ಣಯ ಗ್ರಂಥದ ಉಪಾದೇಯತ್ವ-ಈ ವಿಷಯಗಳು ಈ ಅಧ್ಯಾಯದಲ್ಲಿ ನಿರೂಪಿತವಾಗಿವೆ.


\section*{ಅಧ್ಯಾಯ\enginline{-}೧೦}

\begin{verse}
\textbf{ಕ್ಷೀರಾಬ್ದ್ಯುನ್ಮಥನಾದಿಕಾತ್ಮಚರಿತಂ ದೇವೈರ್ಗೃಣದ್ಭಿಸ್ತುತಃ }\\\textbf{ಸತ್‌ಜ್ಞಾನಾಯ ಪರಾಶರಾಖ್ಯ ಮುನಿನಾ ಯಸ್ಸತ್ಯವತ್ಯಾನುಭೂತ್~।।} \\\textbf{ವ್ಯಾಸತ್ವೇನ ವಿಧಾಯ ವೇದವಿವೃತಿಂ ಶಾಸ್ತ್ರಾಣಿ ಸರ್ವಾಣ್ಯಪಿ}\\\textbf{ಜ್ಞಾನಂ ಸತ್ಸು ನಿಧಾಯ ತದ್ಗ ತಕಲಿಂ ನಿಘ್ನನ್ ಸ ನೋಽವ್ಯಾದ್ಧರಿಃ~।।}
\end{verse}

\vskip -1pt

ಕ್ಷೀರಸಮುದ್ರ ಮಂಥನವೇ ಮೊದಲಾದ ತನ್ನ ಮಾಹಾತ್ಮ್ಯ ಪೂರ್ವಕವಾದ ಚರಿತ್ರೆಯಿಂದ ದೇವತೆಗಳಿಂದ ಸ್ತೋತ್ರಮಾಡಿಸಿಕೊಳ್ಳಲ್ಪಟ್ಟ, ಯಥಾರ್ಥ ತತ್ವಜ್ಞಾನವನ್ನು ಸಜ್ಜನರಲ್ಲಿ ಉಪದೇಶ ಮಾಡುವ ಕಾರಣದಿಂದ ಪರಾಶರ ಮುನಿಗಳಿಂದ ಸತ್ಯವತೀದೇವಿಯಲ್ಲಿ ವ್ಯಾಸಾವತಾರವನ್ನು ತಾಳಿ, ವೇದಗಳ ಅರ್ಥವನ್ನು ನಿರ್ಣಾಯಕವಾಗಿ ತಿಳಿಸುವ ಬ್ರಹ್ಮಸೂತ್ರಗಳನ್ನೂ ಇತರ ಶಾಸ್ತ್ರಗಳನ್ನೂ ರಚಿಸಿದ, ಸಜ್ಜನರಲ್ಲಿ ಯಥಾರ್ಥಜ್ಞಾನವನ್ನು ಸ್ಥಾಪಿಸಿ, ಅಯಥಾರ್ಥಜ್ಞಾನಕ್ಕೆ ಕಾರಣನಾದ ಕಲಿಯನ್ನು ನಿಗ್ರಹಿಸಿದ ಶ‍್ರೀಹರಿಯು ನಮ್ಮನ್ನು ಸಲಹಲಿ.

ಈ ಅಧ್ಯಾಯದಲ್ಲಿ ೮೮ ಶ್ಲೋಕಗಳಿವೆ.

ಇಪ್ಪತ್ತೆಂಟನೇ ದ್ವಾಪರಯುಗದಲ್ಲಿ ಬ್ರಹ್ಮ, ರುದ್ರ, ಇಂದ್ರ ಮುಂತಾದ ದೇವತೆಗಳು ಕ್ಷೀರಸಮುದ್ರಕ್ಕೆ ತೆರಳಿ ಪುಂಡರೀಕಾಕ್ಷನನ್ನು ಸ್ತೋತ್ರ ಮಾಡಿದ ಬಗೆ, ಕ್ಷೀರಸಮುದ್ರವನ್ನು ಹಿಂದೆ ಮಥನ ಮಾಡಿದಾಗ ಶ‍್ರೀ ಪುರುಷೋತ್ತಮನು ತೋರಿಸಿದ ಮಹಿಮೆ, ವಿಷವು ಉತ್ಪತ್ತಿ\-ಯಾದುದು, ಅದನ್ನು ವಾಯುದೇವರು ಮರ್ದಿಸಿ ಸ್ವಲ್ಪವನ್ನು ರುದ್ರದೇವರಿಗೆ ಕೊಟ್ಟಿದ್ದು, ರುದ್ರದೇವರಿಗೆ ಅದನ್ನು ಪಾನ ಮಾಡಿದುದರಿಂದ ಆದ ತೊಂದರೆ, ಕ್ಷೀರ ಸಮುದ್ರದಿಂದ ಉತ್ಪನ್ನವಾದ ಇತರ ವಸ್ತುಗಳ ವಿವರಣೆ, ಧನ್ವಂತರೀ ರೂಪದಿಂದ ಶ‍್ರೀಹರಿಯು ಪ್ರಾದುರ್ಭವಿಸುವುದು, ಮೋಹಿನೀ ರೂಪವನ್ನು ಪಡೆದು ದೇವತೆಗಳಿಗೆ ಮಾತ್ರ ಅಮೃತವನ್ನು ಹಂಚಿದುದು, ಐರಾವತ ಮುಂತಾದ ವಸ್ತುಗಳ ವಿತರಣೆ, ದೇವತೆಗಳು ಶ‍್ರೀಹರಿಯನ್ನು ಯಥಾರ್ಥ ತತ್ವಜ್ಞಾನವನ್ನು ಉಪದೇಶಿಸಲು ಪ್ರಾರ್ಥಿಸುವುದು, ಶ‍್ರೀಹರಿಯು ದೇವತೆಗಳಿಗೆ ಅಭಯವನ್ನು ನೀಡುವುದು, ಸತ್ಯವತೀ ದೇವಿಯ ಉತ್ಪತ್ತಿ, ಪರಾಶರ-ಸತ್ಯವತೀ ದೇವಿಯರ ನಿಮಿತ್ತದಿಂದ ಯಮುನಾ ನದಿಯಲ್ಲಿ ದೋಣಿಯಲ್ಲಿ ವೇದವ್ಯಾಸರಾಗಿ ಪ್ರಾದುರ್ಭವಿಸುವುದು, ವೇದವ್ಯಾಸದೇವರ ಪರಮ ಮಂಗಳವಾದ ರೂಪದ ವರ್ಣನೆ, ಮೇರು ಪರ್ವತದಲ್ಲಿ ಶ‍್ರೀವೇದವ್ಯಾಸರು ದೇವತೆಗಳಿಗೆ, ಮುನಿಗಳಿಗೆ, ತತ್ವೋಪದೇಶ ಮಾಡುವುದು, ಬ್ರಹ್ಮಸೂತ್ರಗಳ ರಚನೆ, ಮಹಾಭಾರತ ಗ್ರಂಥದ ರಚನೆ, ವೇದಗಳ ವಿಭಜನೆ, ಕೀಟಸ್ವರೂಪದಲ್ಲಿದ್ದ ರಾಜನ ವೃತ್ತಾಂತ, ಶುಕಾಚಾರ್ಯರ ಉತ್ಪತ್ತಿ, ಅವರಿಗೆ ಜ್ಞಾನೋಪದೇಶ, ನಾರದ, ರೋಮಹರ್ಷಣ, ಸನಕಸನಂದಾದಿಗಳು, ಭೃಗುವೇ ಮೊದಲಾದ ಋಷಿಗಳು, ಜೈಮಿನಿ ಋಷಿಗಳು, ಇವರನ್ನೆಲ್ಲ ದೇವತೆಗಳು, ಋಷಿಗಳು, ಮನುಷ್ಯೋತ್ತಮರಲ್ಲಿ ಜ್ಞಾನಪ್ರಚಾರಕ್ಕಾಗಿ ನಿಯಮಿಸುವುದು, ಪಾಶುಪತಾದಿ ಗ್ರಂಥಗಳಿಗೂ ವೈಷ್ಣವ ಪುರಾಣಗಳಿಗೂ ಇರುವ ಭೇದ-ಈ ವಿಷಯಗಳು ಈ ಅಧ್ಯಾಯದಲ್ಲಿ ನಿರೂಪಿತವಾಗಿವೆ.


\section*{ಅಧ್ಯಾಯ\enginline{-}೧೧}

\begin{verse}
\textbf{ಭೂಪಾ ಯತ್ರ ಪುರೂರವಪ್ರಭೃತಯೋ ಜಾತಾ ವಿಧೋರನ್ವಯೇ}\\\textbf{ಯದ್ವಾದ್ಯಾ ಭರತಾದಯಃ ಕುರುಮುಖಾಃ ಭೀಷ್ಮಾಂಬಿಕೇಯಾದಯಃ~।} \\\textbf{ಭೂಭಾರಕ್ಷಯಕಾಂಕ್ಷಿಭಿಃ ಸುರವರೈಃ ಅಭ್ಯರ್ಥಿತಃ ಶ‍್ರೀಪತಿಃ}\\\textbf{ತತ್ರಾವಿರ್ಭವಿತುಂ ಸದಾ ಸುರವರೈರಿಚ್ಛನ್ ಹರಿಃ ಪಾತು ಮಾಮ್~।।}
\end{verse}

ಭೂಭಾರವನ್ನು ನಾಶಮಾಡಲು ದೇವತೆಗಳಿಂದ ಪ್ರಾರ್ಥಿತನಾಗಿ, ಪುರೂರವ, ಯದು, ಭರತ, ಕುರು, ಭೀಷ್ಮ, ಧೃತರಾಷ್ಟ್ರ ಮುಂತಾದವರು ಉತ್ಪನ್ನರಾದ ಚಂದ್ರವಂಶದಲ್ಲಿ ಇತರ ದೇವತೆಗಳೊಂದಿಗೆ ತಾನೂ ಸಹ ಅವತರಿಸಲು ಇಚ್ಛಿಸಿದ ಶ‍್ರೀ ರಮಾವಲ್ಲಭನು ನನ್ನನು ಸಲಹಲಿ.

ಈ ಅಧ್ಯಾಯದಲ್ಲಿ ೨೩೭ ಶ್ಲೋಕಗಳಿವೆ.

ಪುರೂರವ ವಂಶದ ವಿವರಣೆ, ಯದುವಂಶದ ವಿಚಾರ, ಬಾಹ್ಲೀಕರಾಜನ ಪೂರ್ವ ವೃತ್ತಾಂತ, ವರುಣನು ಬ್ರಹ್ಮದೇವರ ಶಾಪದಿಂದ ಶಂತನು ಎಂಬ ಹೆಸರಿನಿಂದ ಹುಟ್ಟುವುದು, ದ್ಯುನಾಮಕ ವಸುವು ಭೀಷ್ಮನಾಗಿ ಅವತರಿಸುವುದು, ಶಂತನು-ಗಂಗಾದೇವಿಯ ವಿವಾಹ, ಗಂಗೆಯು ಶಂತನುವನ್ನು ಬಿಟ್ಟು ಹೋಗುವುದು, ಭೀಷ್ಮಾಚಾರ್ಯರ ವಿದ್ಯಾಭ್ಯಾಸ, ಕೃಪ-\-ಕೃಪಿಯರ ಉತ್ಪತ್ತಿ, ಬೃಹಸ್ಪತಿಯು ದ್ರೋಣನಾಗಿ ಹುಟ್ಟುವುದು, ದ್ರುಪದನ ಜನನ, ಭಾರದ್ವಾಜರಲ್ಲಿ ವಿದ್ಯಾಭ್ಯಾಸ, ದ್ರುಪದನು ದ್ರೋಣನಿಗೆ ಅರ್ಧರಾಜ್ಯವನ್ನು ಕೊಡುವುದಾಗಿ ಹೇಳುವುದು, ದ್ರೋಣನು ಹಸ್ತಿನಾಪುರದಲ್ಲಿ ಶಿಲೋಂಛವೃತ್ತಿಯಿಂದ ಜೀವನ ಮಾಡುವುದು, ಭೀಷ್ಮಾಚಾರ್ಯನ ಪ್ರತಿಜ್ಞೆ, ಶಂತನು-ಸತ್ಯವತಿಯರ ವಿವಾಹ, ಚಿತ್ರಾಂಗದ ಮತ್ತು ವಿಚಿತ್ರವೀರ್ಯರ ಜನನ, ಶಂತನುವಿನ ಮರಣ, ವಿಚಿತ್ರವೀರ್ಯನು ರಾಜನಾಗುವುದು, ಅಂಬಿಕಾ, ಅಂಬಾಲಿಕಾ ಎಂಬ ಕಾಶೀರಾಜನ ಪುತ್ರಿಯರೊಡನೆ ವಿಚಿತ್ರ ವೀರ್ಯನ ವಿವಾಹ, ಅಂಬೆಯು ರುದ್ರದೇವರನ್ನು ಕುರಿತು ತಪಸ್ಸು ಮಾಡಿ ಭೀಷ್ಮಾಚಾರ್ಯರ ಸಂಹಾರಕ್ಕೆ ರುದ್ರದೇವರಿಂದ ಹಾರವನ್ನು ಪಡೆದು ಅದನ್ನು ದ್ರುಪದನ ಮನೆಯ ಮುಂದೆ ಇಡುವುದು, ಶಿಖಂಡಿಯ ವೃತ್ತಾಂತ, ವಿಚಿತ್ರ ವೀರ್ಯನ ಮರಣ, ಅಂಬಿಕೆಗೆ ವೇದವ್ಯಾಸರು ಭಯಂಕರವಾದ ರೂಪವನ್ನು ತೋರಿಸಿದುದು, ಅದರಿಂದ ಅವಳು ಕಣ್ಣುಗಳನ್ನು ಮುಚ್ಚಿಕೊಂಡಕಾರಣ ಕುರುಡನಾದ ಧೃತರಾಷ್ಟ್ರನ ಜನನ, ಇದರಂತೆ ಅಂಬಾಲಿಕೆಯು ವೇದವ್ಯಾಸರ ಅದ್ಭುತ ರೂಪವನ್ನು ನೋಡಿ ಪಾಂಡು (ಬಿಳಿಯ) ವರ್ಣದವಳಾದಕಾರಣದಿಂದ ಮೈಯೆಲ್ಲ ಬಿಳುಪಾದ ಪಾಂಡುವಿನ ಜನನ, ಶೂದ್ರಳಾದ ಪರಿಚಾರಿಣಿಯಿಂದ ವಿದುರನ ಜನನ, ಭೀಷ್ಮಾಚಾರ್ಯರಿಂದ ಈ ಮೂರು ಜನರಿಗೂ ಸಂಸ್ಕಾರಾದಿಗಳು, ವಿದ್ಯಾಭ್ಯಾಸ, ಸಂಜಯನ ಉತ್ಪತ್ತಿ, ಧೃತರಾಷ್ಟ್ರ -\-ಗಾಂಧಾರಿ ವಿವಾಹ, ಕುಂತಿಯ ಸ್ವರೂಪ ವರ್ಣನೆ, ದುರ್ವಾಸರಿಂದ ಅವಳಿಗೆ ಮಂತ್ರೋಪದೇಶ, ಮಂತ್ರಬಲದಿಂದ ಸೂರ್ಯನು ಕುಂತಿಯಿಂದ ಅವತರಿಸುವುದು, ಆ ಮಗುವನ್ನು ಯಮುನಾನದಿಯಲ್ಲಿ ಬಿಡುವುದು, ಸೂತಪುತ್ರನಾದ, ಅಧಿರಥನಿಂದ ಆ ಮಗುವಿನ ರಕ್ಷಣೆ, ಪಾಂಡು-ಕುಂತೀ ವಿವಾಹ, ಶಲ್ಯನ ವೃತ್ತಾಂತ, ಮಾದ್ರಿ-ಪಾಂಡುವಿನ ವಿವಾಹ, ಪಾಂಡುರಾಜನು ರಾಜ್ಯವನ್ನು ಪರಿಪಾಲಿಸುತ್ತಿದ್ದುದು, ಸತ್ಯವತೀ-ಅಂಬಿಕಾ\-ಅಂಬಾಲಿಕೆಯರಿಗೆ ಸದ್ಗತಿ ದೊರೆತುದು, ಪಾಂಡುರಾಜನು ಪತ್ನಿಯರೊಂದಿಗೆ ವನಕ್ಕೆ ಹೋಗುವುದು, ಅಲ್ಲಿ ತಪಸ್ಸನ್ನು ಆಚರಿಸುವುದು, ಈ ಕಾಲದಲ್ಲಿ ಬ್ರಹ್ಮಾದಿ ದೇವತೆಗಳು ಶೇಷಶಾಯಿಯಾದ ಶ‍್ರೀಮನ್ನಾರಾಯಣನ ಮೊರೆ ಹೊಂದುವುದು, ದೇವಾಸುರರ ಯುದ್ಧದ ವರ್ಣನೆ, ಉಗ್ರಸೇನ, ಕಂಸನ ಉತ್ಪತ್ತಿ, ಜರಾಸಂಧನ ವೃತ್ತಾಂತ, ಶಿಶುಪಾಲ-ದಂತವಕ್ರರ ಪೂರ್ವಜನ್ಮಗಳ ಕಥೆ, ಕೀಚಕನ ಸ್ವರೂಪ ವರ್ಣನೆ, ಶ‍್ರೀಹರಿಯ ಅವತಾರಕ್ಕಾಗಿ ಬ್ರಹ್ಮಾದಿಗಳ ವಿಜ್ಞಾಪನೆ, ವಸುದೇವ-ದೇವಕಿಯರ ವರ್ಣನೆ, ಶ‍್ರೀಹರಿಯ ಆಶ್ವಾಸನೆ ಹಾಗೂ ದೇವತೆಗಳಿಗೆ ತಾವೂ ಸಹ ಅವತಾರ ಮಾಡಬೇಕೆಂಬ ಆಜ್ಞೆ, ಅದೇ ರೀತಿ ದೇವತೆಗಳು ಅವತಾರ ಮಾಡಿದುದು, ಯಾರು ದೇವತೆಗಳ ಅಂಶಭೂತರು ಯಾರು ದೈತ್ಯರ ಅಂಶಭೂತರು ಎಂದು ತಿಳಿದು ಕೊಳ್ಳುವ ಬಗೆ, ಈ ವಿಷಯಗಳು ಈ ಅಧ್ಯಾಯದಲ್ಲಿ ನಿರೂಪಿತವಾಗಿವೆ.

\vskip 1cm

\section*{ಅಧ್ಯಾಯ\enginline{-}೧೨}

\begin{verse}
\textbf{ದೇವಕ್ಕಾಂ ವಸುದೇವತೋಽಗ್ರಜಯುತೋಜಾತೋವ್ರಜಂ ಯೋ ಗತೋ}\\\textbf{ಬಾಲಘ್ನೀ ಶಕಟಾಕ್ಷಹಾ ಸ್ವಜನನೀಮಾನ್ಯಸ್ತೃಣಾವರ್ತಹಾ~।}\\\textbf{ಯತ್ಪೂರ್ವಂ ಪರತಶ್ಚ ಪಾಂಡುತನಯಾ ಯತ್ಸೇವನಂ ಜಜ್ಞಿರೇ}\\\textbf{ಕರ್ತುಂ ಧರ್ಮಮರುದ್ ವೃಷಾಶ್ವಿನ ಇಮಂ ನಂದಾತ್ಮಜಂ ನೌಮಿ ತಮ್~।।}
\end{verse}

ದೇವಕಿ-ವಸುದೇವರಲ್ಲಿ ಜನ್ಮವೆತ್ತಿ, ಗೋಕುಲಕ್ಕೆ ತೆರಳಿ, ಅಣ್ಣನಾದ ಬಲರಾಮನ ಜೊತೆಯಲ್ಲಿಯೇ ಇರುತ್ತಾ, ಶಿಶುಗಳನ್ನು ಕೊಲ್ಲುತ್ತಿದ್ದ ಪೂತನಿಯನ್ನೂ, ಶಕಟಾಸುರನನ್ನೂ, ತೃಣಾವರ್ತನನ್ನೂ ಸಂಹರಿಸಿದ, ತಾಯಿಯಾದ ಯಶೋದೆಯಿಂದ ಗೌರವಿಸಲ್ಪಟ್ಟ (ಬಾಯಲ್ಲಿ ಬ್ರಹ್ಮಾಂಡವನ್ನು ತೋರಿಸಿದ ಕಾರಣದಿಂದ), ಪಾಂಡುಪುತ್ರರಾಗಿ ಅವತಾರ ಮಾಡಿದ\break ಯಮಧರ್ಮ-ವಾಯು-ಇಂದ್ರ-ಅಶ್ವಿನೀದೇವತೆಗಳಿಂದ ಕೃಷ್ಣಾವತಾರಕ್ಕೆ ಪೂರ್ವದಲ್ಲಿಯೂ,\break ನಂತರವೂ ಸೇವಿಸಿಕೊಳ್ಳಲ್ಪಟ್ಟ ನಂದಕುಮಾರನನ್ನು ನಮಸ್ಕರಿಸುತ್ತೇನೆ.

ಈ ಅಧ್ಯಾಯದಲ್ಲಿ ೧೩೬ ಶ್ಲೋಕಗಳಿವೆ.

ದೇವಕಿಯ ಜನ್ಮ, ವಸುದೇವ-ದೇವಕಿಯರ ವಿವಾಹ, ದೇವಕಿಯ ಎಂಟನೆಯ ಗರ್ಭದಲ್ಲಿ ಹುಟ್ಟುವ ಮಗುವಿನಿಂದ ಕಂಸನ ಸಂಹಾರವೆಂದು ವಾಯುದೇವರು ಆಕಾಶವಾಣಿಯಿಂದ ನುಡಿಯುವುದು, ದೇವಕಿಯ ಸಂಹಾರಕ್ಕೆ ಕಂಸನ ಪ್ರಯತ್ನ, ಎಲ್ಲ ಮಕ್ಕಳನ್ನೂ ಹುಟ್ಟಿದ ಕೂಡಲೇ ಕಂಸನಿಗೆ ಅರ್ಪಿಸುವುದಾಗಿ ದೇವಕಿಯ ವಾಗ್ದಾನ, ವಸುದೇವನೊಡನೆ ಇತರ ಕನ್ನಿಕೆಯರ ವಿವಾಹ, ಪೌಂಡ್ರಕ ವಾಸುದೇವನ ವೃತ್ತಾಂತ, ಸೃಗಾಲವಾಸುದೇವನ ಜನ್ಮವೃತ್ತಾಂತ, ಕಂಸನು ದೇವಕಿಯ ಮಕ್ಕಳನ್ನು ಸಂಹರಿಸುವುದು, ಈ ಮಕ್ಕಳ ಪೂರ್ವಚರಿತ್ರೆ, ಕಲಿ-\-ಇಂದ್ರಜಿತ್ ಮೊದಲಾದ ಅಸುರರು ಧೃತರಾಷ್ಟ್ರನ ಪತ್ನಿಯಾದ ಗಾಂಧಾರಿ ಗರ್ಭದಲ್ಲಿ ಪ್ರವೇಶಮಾಡುವುದು, ಪಾಂಡುರಾಜನಿಂದ ಕುಂತೀದೇವಿಗೆ ಮಂತ್ರಗಳಿಂದ ಪುತ್ರರನ್ನು ಉತ್ಪಾದಿಸಲು ಆಜ್ಞೆ, ಯುಧಿಷ್ಠಿರನ ಜನನ, ಗಾಂಧಾರಿಯಿಂದ ತನ್ನ ಗರ್ಭಕ್ಕೆ ಪೆಟ್ಟು, ವೇದವ್ಯಾಸರ ಸಹಾಯದಿಂದ ಸುಯೋಧನನೇ ಮೊದಲಾದ ನೂರು ಗಂಡುಮಕ್ಕಳೂ, ದುಃಶಲಾ ಎಂಬ ಒಬ್ಬ ಹೆಣ್ಣು ಮಗಳೂ ಉತ್ಪನ್ನರಾಗುವುದು, ಕಲಿಯು ಸುಯೋಧನನಾಗಿ ಹುಟ್ಟಲು ಕಾರಣ, ಇತರ ಮಕ್ಕಳ ಸ್ವರೂಪ ವರ್ಣನೆ, ಕುಂತಿಯಿಂದ ಮುಖ್ಯಪ್ರಾಣರು ಭೀಮನಾಗಿ ಅವತಾರ ಮಾಡುವುದು, ಶೇಷದೇವರು ಪರಮಾತ್ಮನ ಆಜ್ಞೆಯಂತೆ ಬಲರಾಮನಾಗಿ ಹುಟ್ಟುವುದು, ಅದಾದ ಮೂರು ತಿಂಗಳಿನ ಮೇಲೆ ಶ‍್ರೀಹರಿಯು ಕೃಷ್ಣನಾಗಿ ಅವತರಿಸುವುದು, ವಸುದೇವ-ದೇವಕಿ\-ಯರಿಗೆ ಶ‍್ರೀಹರಿಯು ಸರ್ವಾಲಂಕಾರಭೂಷಣಗಳಿಂದ ದರ್ಶನಕೊಡುವುದು, ನಂದಗೋಪನ ಮನೆಗೆ ಶ‍್ರೀಕೃಷ್ಣನನ್ನು ಒಯ್ದು ಅಲ್ಲಿದ್ದ ದುರ್ಗಾದೇವಿಯನ್ನು ವಸುದೇವನು ತರುವುದು, ಕಂಸನು ತನ್ನನ್ನು ಸಂಹರಿಸಲು ಬಂದಾಗ ದುರ್ಗೆಯು ಕಂಸನ ಸಂಹಾರಕನು ಬೇರೆಡೆಯಲ್ಲಿ ಇರುತ್ತಾನೆಂದು ಹೇಳುವುದು, ಕಂಸನು ತನ್ನ ಮಂತ್ರಿಗಳ ಸಲಹೆಯಂತೆ ಬಾಲಕರ ಸಂಹಾರಕ್ಕೆ ಆಜ್ಞೆ ಮಾಡುವುದು, ನಂದಗೋಪ-ಯಶೋದೆಯರ\break ಸಂತೋಷ-ಸಡಗರ, ಪೂತನಿಯ ಸಂಹಾರ, ಶ‍್ರೀಕೃಷ್ಣನಿಗೆ ನಾಲ್ಕು ತಿಂಗಳು ಆದಾಗ ಶಕಟಾಸುರನ ಸಂಹಾರ, ಕುಂತಿಯಲ್ಲಿ ದೇವೇಂದ್ರನು ಅವತರಿಸುವುದು, ಉದ್ಧವನ ಜನನ, ಅಶ್ವತ್ಥಾಮನ ಉತ್ಪತ್ತಿ, ಶ‍್ರೀಕೃಷ್ಣನು ಯಶೋದೆಗೆ ತನ್ನ ಬಾಯಲ್ಲಿ ಸ್ಥಾವರ ಜಂಗಮಾತ್ಮಕವಾದ ನಿಖಿಲ ಬ್ರಹ್ಮಾಂಡವನ್ನೂ ತೋರಿಸಿ ತಾನು ಅಪ್ರಾಕೃತನೆಂದು ಪ್ರಸಿದ್ಧಪಡಿಸುವುದು, ತೃಣಾವರ್ತನ ಸಂಹಾರ, ಶ‍್ರೀಕೃಷ್ಣನು ನವನೀತ-ಕ್ಷೀರ ಮುಂತಾದುವುಗಳನ್ನು ಕದ್ದು ತಿನ್ನುವುದು, ಕುಂತೀದೇವಿಯು ತನ್ನ ಸವತಿಯಾದ ಮಾದ್ರಿದೇವಿಗೂ ಮಕ್ಕಳಾಗಲೆಂದು ಮಂತ್ರೋಪದೇಶ ಮಾಡುವುದು, ಅಶ್ವಿನೀದೇವತೆಗಳು ಅವಳಲ್ಲಿ ಅವತರಿಸುವುದು, ಯುಧಿಷ್ಠಿರಾದಿಗಳಲ್ಲಿ ವಾಯುದೇವರು ಸಮಾವಿಷ್ಟರಾಗಿರುವುದು, ಪಾಂಡವರ ಅಭಿವೃದ್ಧಿ ಈ ವಿಷಯಗಳು ಈ ಅಧ್ಯಾಯದಲ್ಲಿ ನಿರೂಪಿತವಾಗಿವೆ.


\section*{ಅಧ್ಯಾಯ\enginline{-}೧೩}

\begin{verse}
\textbf{ಸಂಸ್ಕಾರಾನ್ ಪ್ರಾಪ್ಯ ಗರ್ಗಾದ್ಬಹುಶಿಶುಚರಿತೈಃ ಪ್ರೀಣಯನ್ ಗೋಪಗೋಪೀಃ} \\\textbf{ವತ್ಸಾನ್ ಧೇನೂಶ್ಚ ರಕ್ಷನ್ನಹಿಪತಿದಮನೋ ಯಃ ಪಪೌ ಕಾನನಾಗ್ನಿಮ್~।}\\\textbf{ವಿಪ್ರಸ್ತ್ರೀ ಪ್ರೀತಿಕಾರೀ ಧೃತಧರಣಿಧರೋ ಗೊಪಿಕಾಭಿರ್ನಿಶಾಸು} \\\textbf{ಕ್ರೀಡನ್ಮಲ್ಲಾಂಶ್ಚ ಕಂಸಂ ನ್ಯಹನದುಪಗತೋಽವ್ಯಾತ್ಸ ಕೃಷ್ಣಃ ಪುರೀಂ ಸ್ವಾಮ್~।।}
\end{verse}

ಗರ್ಗಾಚಾರ್ಯರಿಂದ ನಾಮಕರಣಾದಿ ಸಂಸ್ಕಾರಗಳನ್ನು ಪಡೆದು, ತನ್ನ ಆಟಪಾಟಗಳಿಂದ ಗೋಪಾಲಕರನ್ನೂ ಗೋಪಿಕಾಸ್ತ್ರೀಯರನ್ನೂ ಸಂತೋಷಪಡಿಸಿ, ಹಸು-ಕರುಗಳನ್ನು ರಕ್ಷಿಸಿ, ಕಾಳಿಂಗಸರ್ಪನ ಹೆಡೆಯನ್ನು ತುಳಿದು, ಕಾಡುಗಿಚ್ಚನ್ನು ಪಾನಮಾಡಿ, ಬ್ರಾಹ್ಮಣಸ್ತ್ರೀಯರನ್ನು ಅನುಗ್ರಹಿಸಿ, ಗೋವರ್ಧನಪರ್ವತವನ್ನು ಎತ್ತಿ ಹಿಡಿದು, ರಾತ್ರಿಯಲ್ಲಿ ಗೋಪಸ್ತ್ರೀಯರೊಡನೆ ರಾಸಕ್ರೀಡೆಯನ್ನು ಆಡಿ, ತನ್ನ ನಗರವಾದ ಮಥುರೆಗೆ ಪ್ರಾಪ್ತನಾಗಿ, ಅಲ್ಲಿದ್ದ ಜಟ್ಟಿಗಳನ್ನೂ ಕಂಸನನ್ನೂ ಸಂಹರಿಸಿದ ಶ‍್ರೀಕೃಷ್ಣನು ನಮ್ಮನ್ನು ಸಲಹಲಿ.

ಈ ಅಧ್ಯಾಯದಲ್ಲಿ ೧೩೮ ಶ್ಲೋಕಗಳಿವೆ.

ಶ‍್ರೀಕೃಷ್ಣ ಬಲರಾಮರಿಗೆ ಗರ್ಗಾಚಾರ್ಯರು ಜಾತಕರ್ಮಾದಿ ಸಂಸ್ಕಾರಗಳನ್ನು ಮಾಡು\-ವುದು, ಶ‍್ರೀಕೃಷ್ಣನು ಮಣ್ಣನ್ನು ತಿಂದು ಇದಕ್ಕಾಗಿ ಶಿಕ್ಷಿಸಿದ ಯಶೋದೆಗೆ ಮೂಲ\-ಪ್ರಕೃತಿ, ಮಹದಾದಿ ತತ್ವಗಳು, ಮುಂತಾದ ತನ್ನ ಸ್ವರೂಪ ಶಕ್ತಿಯನ್ನು ಬಾಯಲ್ಲಿ ತೋರಿಸುವುದು, ಮೊಸರಿನ ಪಾತ್ರೆಯನ್ನು ಒಡೆಯುವುದು, ಬೆಣ್ಣೆಯನ್ನು ಕದ್ದು ತಿನ್ನುವುದು, ಯಶೋದೆಯು ಶ‍್ರೀಕೃಷ್ಣನನ್ನು ಕಟ್ಟುವುದು, ವೃಕ್ಷಗಳನ್ನು ನಾಶಮಾಡಿ ಕುಬೇರ ಪುತ್ರರಿಗೆ ಶಾಪವಿಮೋಚನೆಯನ್ನು ಮಾಡುವುದು, ಪೂತನಿ - ಶಕಟಾಸುರ - ತೃಣಾವರ್ತರ ವಧೆ, ವೃಂದಾವನಕ್ಕೆ ಪ್ರಯಾಣ, ವತ್ಸಾಸುರನ ಸಂಹಾರ, ಬಕನ ವಧೆ, ಕಾಲಿಂಗನ ಹೆಡೆಯಮೇಲೆ ತುಳಿಯುವುದು, ನಾಗಪತ್ನಿಯರಿಂದ ಸ್ತೋತ್ರ, ಉಗ್ರಾಸುರನ ಸಂಹಾರ, ನೀಲಾದೇವಿಯೊಡನೆ ವಿವಾಹ, ಬಲರಾಮನಿಂದ ಧೇನುಕಾಸುರನ ಮತ್ತು ಪ್ರಲಂಬಾಸುರನ ವಧೆ, ಶ‍್ರೀಕೃಷ್ಣನಿಂದ ಕಾಡಾಗ್ನಿಯ ಪಾನ, ಬ್ರಾಹ್ಮಣ ಸ್ತ್ರೀಯರು ವನದಲ್ಲಿ ಶ‍್ರೀಕೃಷ್ಣನಿಗೆ ಮೃಷ್ಟಾನ್ನವನ್ನು ಸಮರ್ಪಿಸಿ ಅನುಗ್ರಹೀತ\-ರಾಗುವುದು, ಗೋವರ್ಧನ ಗಿರಿಗೆ ಬದಲು ತನ್ನನ್ನೇ ಪೂಜಿಸಲು ಶ‍್ರೀಕೃಷ್ಣನು ಗೊಲ್ಲರಿಗೆ ಹೇಳುವುದು, ಇಂದ್ರನ ಕೋಪ, ಭಾರಿಮಳೆ, ಗೋವರ್ಧನಗಿರಿಯನ್ನು ಎತ್ತಿ ಶ‍್ರೀಕೃಷ್ಣನು ಗೊಲ್ಲರನ್ನು ರಕ್ಷಿಸುವುದು, ಗೋಪಿಕಾ ಸ್ತ್ರೀಯರೊಡನೆ ಶ‍್ರೀಕೃಷ್ಣನು ಕ್ರೀಡಿಸುವುದು, ಶ‍್ರೀಕೃಷ್ಣನಿಂದ ವೇಣುಗಾನ, ಶಂಖಚೂಡನ ಮತ್ತು ಅರಿಷ್ಟಾಸುರನ ಸಂಹಾರ, ಕೇಶಿನಾಮಕ ಅಸುರನ ಶರೀರವನ್ನು ಸೀಳುವುದು, ಅಕ್ರೂರನ ವೃತ್ತಾಂತ, ಬಲರಾಮ-ಕೃಷ್ಣರನ್ನು ಮಥುರೆಗೆ ಕರೆತರಲು ಕಂಸನಿಂದ ಅಕ್ರೂರನಿಗೆ ಆಜ್ಞೆ, ಶ‍್ರೀಕೃಷ್ಣನಿಂದ ಅಕ್ರೂರನಿಗೆ ಉಪಚಾರ, ಮಥುರೆಗೆ ಬಲರಾಮ-ಕೃಷ್ಣ-ಅಕ್ರೂರ ಇವರ ಪ್ರಯಾಣ, ಯಮುನಾನದಿಯ ನೀರಿನಲ್ಲಿ ಅಕ್ರೂರನಿಗೆ ಶ‍್ರೀಕೃಷ್ಣನು ತನ್ನ ಸ್ವರೂಪವನ್ನು ತೋರಿಸುವುದು, ಅಕ್ರೂರನ ಸ್ತುತಿ, ಮಥುರಾದಲ್ಲಿ ರಜಕನ ವಧೆ, ಗಂಧವನ್ನು ಅರ್ಪಿಸಿದ ತ್ರಿವಕ್ರೆಗೆ ಸುಂದರ ರೂಪವನ್ನು ಕೊಡುವುದು, ರುದ್ರದೇವರಿಂದ ಕೊಡಲ್ಪಟ್ಟ ಧನುಸ್ಸನ್ನು ಮುರಿಯುವುದು, ಕಂಸನ ಭಯ, ಕುವಲಯಾಪೀಡನೆಂಬ ಆನೆಯನ್ನು ಸಂಹರಿಸುವುದು, ಚಾಣೂರಮುಷ್ಟಿಕರೆಂಬ ಜಟ್ಟಿಗಳ ವಧೆ, ಕಂಸನ ತಮ್ಮಂದಿರನ್ನು ಬಲರಾಮನು ಸಂಹರಿಸುವುದು, ಮಂಚದ ಮೇಲಿನಿಂದ ಕಂಸನನ್ನು ಕೃಷ್ಣನು ಕೆಡವಿ ಸಂಹರಿಸುವುದು, ಮಾಹಾತ್ಮ ಜ್ಞಾನಪೂರ್ವಕವಾದ ಹರಿಭಕ್ತಿಯಿಂದಲೇ ಮುಕ್ತಿ, ಶ‍್ರೀಕೃಷ್ಣನ ಸ್ವರೂಪ ವರ್ಣನೆ ಈ ವಿಷಯಗಳು ಈ ಅಧ್ಯಾಯದಲ್ಲಿ ನಿರೂಪಿತವಾಗಿವೆ.


\section*{ಅಧ್ಯಾಯ\enginline{-}೧೪}

\begin{verse}
\textbf{ಪಿತ್ರೋರ್ಬಂಧಂ ನಿರಸ್ಯ ಕ್ಷತಿಪತಿಮಕರೋದುಗ್ರಸೇನಂ ಗುರೋರ್ಯಃ}\\\textbf{ಪುತ್ರಂ ಪ್ರಾದಾತ್ಪರೇತಂ ಯುಧಿ ವಿಜಿತಜರಾಸಂಧಪೂರ್ವಾರಿವರ್ಗಃ~।}\\\textbf{ಪಾರ್ಥಾನ್ ಪಿತ್ರಾವಿಹೀನಾನುಪಗತನಗರಾನ್ಯಸ್ತ್ವಜೋಽಪಾದ್ವಿಪದ್ಭ್ಯಃ}\\\textbf{ನಂದಾದೀನುದ್ಧವೋಕ್ತ್ಯಾ ಗತವಿರಹಶುಚಃ ಕಾರಯನ್ನೊಽವತಾನ್ಮಾಮ್~।।}
\end{verse}

ತನ್ನ ತಂದೆತಾಯಿಯರನ್ನು ಕಾರಾಗೃಹದಿಂದ ಬಿಡಿಸಿದ, ಉಗ್ರಸೇನನನ್ನು ರಾಜನನ್ನಾಗಿ ಸ್ಥಾಪಿಸಿದ, ಗುರುಗಳಾದ ಸಾಂದೀಪಿನೀ ಆಚಾರ್ಯರ ಮೃತಮಗನನ್ನು (ಗುರುದಕ್ಷಿಣೆ ರೂಪದಿಂದ) ಬದುಕಿಸಿ ತಂದ, ಜರಾಸಂಧನೇ ಮೊದಲಾಗಿ ಉಳ್ಳ ಶತ್ರುಸಮೂಹವನ್ನು ಪರಾಭವಗೊಳಿಸಿದ, ಪಿತೃಹೀನರಾದ ಪಾಂಡವರು ಹಸ್ತಿನಾಪುರಕ್ಕೆ ಬಂದಾಗ ಅವರನ್ನು ಅನೇಕ ಅಪಾಯಗಳಿಂದ ರಕ್ಷಿಸಿದ, ತಾನು ವೃಂದಾವನದಿಂದ ಮಥುರಾನಗರಕ್ಕೆ ಹೋದ ಕಾರಣದಿಂದ ದುಃಖತಪ್ತರಾದ ನಂದ ಗೋಪನೇ ಮೊದಲಾದವರನ್ನು ಉದ್ಧವನ ಮೂಲಕ ಸಂತೈಸಿದ ಶ‍್ರೀಕೃಷ್ಣನು ನನ್ನನ್ನು ಸಲಹಲಿ.

ಈ ಅಧ್ಯಾಯದಲ್ಲಿ ೧೧೨ ಶ್ಲೋಕಗಳಿವೆ.

ವಸುದೇವ - ದೇವಕಿಯರನ್ನು ಶೃಂಖಲೆಯಿಂದ ಬಿಡಿಸಿ, ಉಗ್ರಸೇನನನ್ನು ಮಥುರಾನಗರಕ್ಕೆ ರಾಜನನ್ನಾಗಿ ಮಾಡಿದುದು, ವಸುದೇವನಿಂದ ಗಾಯತ್ರಿ ಉಪದೇಶವನ್ನು ಪಡೆದುದು, ನಂದಗೋಪನು ವೃಂದಾವನಕ್ಕೆ ತೆರಳುವುದು, ಸಾಂದೀಪಿನೀ ಆಚಾರ್ಯರಲ್ಲಿ ವಿದ್ಯಾಭ್ಯಾಸ, ಗುರುದಕ್ಷಿಣೆಗಾಗಿ ಅವರ ಮೃತಮಗನನ್ನು ಬದುಕಿಸಿ ತಂದುಕೊಡುವುದು, ಮಥುರೆಯಲ್ಲಿ ವಾಸ, ಮಥುರಾನಗರದ ವೈಭವ ವರ್ಣನೆ, ಜರಾಸಂಧನು ಮಥುರಾನಗರಕ್ಕೆ ಗದೆಯನ್ನು ಎಸೆಯುವುದು, ಮಥುರಾ ನಗರಕ್ಕೆ ಜರಾಸಂಧನು ವಿಂದ, ವಿಂದನ, ಅನುವಿಂದ ಎಂಬುವರೊಡನೆ ಯುದ್ಧಕ್ಕೆ ಬರುವುದು, ಉಗ್ರಸೇನನು ಜರಾಸಂಧನನ್ನು ಸೈನ್ಯದೊಡನೆ ಎದುರಿಸುವುದು, ಶ‍್ರೀಕೃಷ್ಣನು ಜರಾಸಂಧನನ್ನೂ ಅವನ ಅನುಯಾಯಿಗಳನ್ನೂ ಸೋಲಿಸುವುದು, ಜರಾಸಂಧನೊಡನೆ ಬಲರಾಮನು ಭಯಂಕರವಾದ ಯುದ್ಧ ಮಾಡುವುದು, ಜರಾಸಂಧನ ಪರಾಭವ, ಪಾಂಡವರ ಅಭಿವೃದ್ಧಿ, ಪಾಂಡುರಾಜನ ಮರಣ, ಮಾದ್ರಿ ದೇವಿಯ ಸಹಗಮನ, ಕೌರವ-ಪಾಂಡವರು ಆಟವಾಡುವಾಗ ಯಾವಾಗಲೂ ಭೀಮನಿಗೇ ಜಯ, ಕೌರವರು ಭೀಮನಿಗೆ ವಿಷವನ್ನು ಕೊಡುವುದು, ಭೀಮನು ಅದನ್ನು ಜೀರ್ಣಮಾಡಿಕೊಂಡಿದ್ದು, ಭೀಮನನ್ನು ಹಗ್ಗದಿಂದ ಕಟ್ಟಿ ಗಂಗಾನದಿಯಲ್ಲಿ ಎಸೆಯುವುದು, ಸರ್ಪಗಳಿಂದ ಭೀಮನನ್ನು ಕಚ್ಚಿಸುವುದು, ಶಿಶುಪಾಲ-ದಂತವಕ್ರರ ವೃತ್ತಾಂತ, ಭೀಮಸೇನನು ಅನೇಕ ಶತ್ರುರಾಜರನ್ನು ಸೋಲಿಸುವುದು, ಧೃತರಾಷ್ಟ್ರನಿಗೆ ಅಕ್ರೂರನ ಉಪದೇಶ, ಶ‍್ರೀಕೃಷ್ಣನು ಉದ್ಧವನನ್ನು ನಂದಗೋಕುಲಕ್ಕೆ ಕಳಿಸುವುದು, ನಂದಗೋಪನನ್ನು ವಿದ್ಯಾಧರನೆಂಬ ಗಂಧರ್ವನು ಹೆಬ್ಬಾವಿನ ರೂಪದಿಂದ ಹಿಡಿಯುವುದು, ಗಂಧರ್ವನಿಂದ ಶ‍್ರೀಕೃಷ್ಣನ ಮಹಿಮೆಯ ವರ್ಣನೆ, ಉದ್ಧವನು ನಂದಗೋಪಾದಿಗಳಿಗೆ ಹಿತವಚನಗಳನ್ನು ನುಡಿದು ಪುನಃ ಮಥುರಾ ನಗರಕ್ಕೆ ಬರುವುದು – ಈ ವಿಷಯಗಳು ಈ ಅಧ್ಯಾಯದಲ್ಲಿ ನಿರೂಪಿತವಾಗಿವೆ.


\section*{ಅಧ್ಯಾಯ\enginline{-}೧೫}

\begin{verse}
\textbf{ಯಸ್ಮಾತ್ ವ್ಯಾಸಸ್ವರೂಪಾದಪಿ ವಿದಿತಸುವಿದ್ಯಾಮವಾಪುಃ ಪ್ರಮೋದಂ}\\\textbf{ಪಾರ್ಥಾನ್ ದ್ರೋಣಃ ಸುತಾರ್ಥಂ ಪ್ರತಿಗತಭ್ರಗುಪಃ }\\\textbf{ಯನ್ನಿಯತ್ಯಾರ್ಥಕಾಮಃ~।} \\\textbf{ತಸ್ಮಾದಾಪ್ತೋರು ವಿದ್ಯೋ ದ್ರುಪದಮುಪಗತೋಽನಾಪ್ತ ಕಾಮೋsಸ್ತ್ರವಿದ್ಯಾ} \\\textbf{ಶಿಷ್ಯೇಭ್ಯಃ ಕೌರವೇಭ್ಯೋ ರವಿಜನಿರಸನೋಽದಾತ್ಸನೋಽವ್ಯಾನ್ಮುರಾರಿಃ~।।}
\end{verse}

ಪಾಂಡವರು ಯಾವ ವೇದವ್ಯಾಸರೂಪದಿಂದಿದ್ದ ಪರಮಾತ್ಮನಿಂದ ಸಕಲ ಶ್ರೇಷ್ಠ ವಿದ್ಯೆಗಳನ್ನೂ ಪಡೆದು ಸಂತೋಷವನ್ನು ಹೊಂದಿದರೋ, ಪರಶುರಾಮರಿಂದ ನಮ್ಮ ಮಗನಿಗೋಸ್ಕರವೂ, ದ್ರವ್ಯಕ್ಕಾಗಿಯೂ ದ್ರೋಣಾಚಾರ್ಯರು ಸಕಲ ಶಸ್ತ್ರವಿದ್ಯೆಗಳನ್ನೂ ಕಲಿಯುವಂತೆ ಮಾಡಿದನೋ, ದ್ರುಪದ ರಾಜನಿಂದ ನಿರಾಶೆಯನ್ನು ಹೊಂದಿದ ದ್ರೋಣಾಚಾರ್ಯರನ್ನು ಕೌರವ-ಪಾಂಡವರಿಗೆ ಶಸ್ತ್ರಾಸ್ತ್ರ ವಿದ್ಯೆಯನ್ನು ಉಪದೇಶ ಮಾಡುವಂತೆ ಮಾಡಿದನೋ, ಕರ್ಣನ ಅಸ್ತ್ರಪ್ರೌಢಿಮೆಯನ್ನು ನಿರಸನ ಮಾಡಿದನೋ ಅಂತಹ ಮುರವೈರಿಯಾದ ಶ‍್ರೀಕೃಷ್ಣನು ನಮ್ಮನ್ನು ಸಲಹಲಿ.

\vskip 4pt

ಈ ಅಧ್ಯಾಯದಲ್ಲಿ ೬೬ ಶ್ಲೋಕಗಳಿವೆ.

\vskip 2pt

ತ್ರಿವಕ್ರೆ, ಭೀಮಸೇನನ ಸಾರಥಿಯಾದ ವಿಶೋಕ ಇವರ ಜನ್ಮವೃತ್ತಾಂತ, ಭೀಮಸೇನನು ಶ‍್ರೀವೇದವ್ಯಾಸರಿಂದ ಪರವಿದ್ಯೆಯನ್ನೂ ಜ್ಞಾನವಿಶೇಷವನ್ನೂ ಪಡೆದುದು, ಅಶ್ವತ್ಥಾಮರಿಗೆ ತಾಯಿಯು ಹಿಟ್ಟನ್ನು ನೀರಿನಲ್ಲಿ ಕಲಸಿ ಹಾಲೆಂದು ಕೊಡುವುದು, ಕೌರವರು ನಿಜವಾದ ಹಾಲನ್ನು ಕುಡಿಸಿದ ಮೇಲೆ ದ್ರೋಣರು ಗೋವನ್ನು ಸಂಪಾದಿಸಲು ಹೊರಟು ಪರಶುರಾಮರನ್ನು ಬೇಡುವುದು, ಪರಶುರಾಮರು ದ್ರೋಣರಿಗೆ ಗೋವನ್ನು ಬಿಟ್ಟು ಹನ್ನೆರಡು ವರ್ಷಕಾಲ ಜ್ಞಾನೋಪದೇಶವನ್ನು ಮಾಡುವುದು, ದ್ರೋಣರು ದ್ರುಪದನಬಳಿಗೆ ಹೋಗಿ ಅವಮಾನಿತರಾಗುವುದು, ದ್ರುಪದನ ಮೇಲೆ ಸೇಡು ತೀರಿಸಿಕೊಳ್ಳಲು ದ್ರೋಣರ ನಿರ್ಧಾರ, ಧರ್ಮರಾಜನ ಉಂಗುರವನ್ನೂ, ಚೆಂಡನ್ನೂ ಬಾವಿಯಿಂದ ಅಸ್ತ್ರವಿದ್ಯೆ ಸಹಾಯದಿಂದ ಮೇಲೆ ತರುವುದು, ಭೀಷ್ಮಾಚಾರ್ಯರು ದ್ರೋಣರನ್ನು ಕೌರವ-ಪಾಂಡವರಿಗೆ ಗುರುಗಳನ್ನಾಗಿ ನೇಮಿಸುವುದು, ಅರ್ಜುನನ ಪ್ರತಿಜ್ಞೆ, ಶಿಷ್ಯರೆಲ್ಲರೂ ದ್ರೋಣರಿಂದ ಸಕಲ ಶಸ್ತ್ರವಿದ್ಯೆಗಳನ್ನೂ ಕಲಿಯುವುದು, ಕರ್ಣ-ಏಕಲವ್ಯರಿಗೆ ಉಪದೇಶ ಮಾಡಲು ದ್ರೋಣರ ನಿರಾಕರಣೆ, ಕರ್ಣನಿಗೆ ಪರಶುರಾಮರಿಂದ ಉಪದೇಶ, ಕರ್ಣನ ತೊಡೆಯಮೇಲೆ ತಲೆ ಇಟ್ಟು ಮಲಗಿದ್ದ ಪರಶುರಾಮರ ನಿದ್ರಾಭಂಗ, ಪರಶುರಾಮರಿಂದ ಕರ್ಣನಿಗೆ ಶಾಪ, ಏಕಲವ್ಯನ ಶಸ್ತ್ರಾಭ್ಯಾಸ, ಅರ್ಜುನನ ಮೇಲಿನ ಪ್ರೀತಿಯಿಂದ ದ್ರೋಣರು ಏಕಲವ್ಯನ ಹಸ್ತಾಂಗುಷ್ಠವನ್ನು ಗುರುದಕ್ಷಿಣೆಯಾಗಿ ಪಡೆಯುವುದು ಈ ವಿಷಯಗಳು ಈ ಅಧ್ಯಾಯದಲ್ಲಿ ನಿರೂಪಿತವಾಗಿವೆ.

\vskip 2pt

\section*{ಅಧ್ಯಾಯ\enginline{-}೧೬}

\begin{verse}
\textbf{ಭೂಯಸ್ತ್ವಾಗತವಾಹವೇ ಸಹಜರಾಸಂಧ್ಯೆರ್ನೃಪೈರ್ನೀತಯೇ }\\\textbf{ಜ್ಞಾತ್ವಾಗಾತ್ಸಹಜಾನ್ವಿತೋಽತಿಗಹನಂ ಗೋಮಂತಮಾತ್ರಾಗತಾತ್~।}\\\textbf{ತಾರ್ಕ್ಷ್ಯಾಲಬ್ದ ಕಿರೀಟ ಉನ್ನತಗಿರೇರಾಪ್ಲುತ್ಯ ಜಿತ್ವಾ ರಿಪೂನ್}\\\textbf{ಹತ್ವಾ ಸ್ವೀಯಸೃಗಾಲಮಾತ್ಮನಗರೀಂ ಪ್ರಾಪ್ತಃ ಸ ನೋಽವ್ಯಾದ್ಧರಿಃ~।।}
\end{verse}

\vskip 2pt

ಜರಾಸಂಧನು ಯಾದವರೊಡನೆ ಯುದ್ಧ ಮಾಡಲು ಬಂದಿರುವನೆಂದು ತಿಳಿದು ಎತ್ತರವಾದ ಗೋಮಂತಕ ಪರ್ವತಶಿಖರವನ್ನು ಅಣ್ಣನಾದ ಬಲರಾಮನೊಡನೆ ಏರಿ, ಗರುಡನಿಂದ ತರಲ್ಪಟ್ಟ ತನ್ನ ಕಿರೀಟವನ್ನು ಸ್ವೀಕರಿಸಿ (ವೈಕುಂಠದಿಂದ ಬಲಿಯು ಕದ್ದು ಪಾತಾಳಲೋಕಕ್ಕೆ ತೆಗೆದುಕೊಂಡುಹೋದ ಕಿರೀಟ) ಜರಾಸಂಧಾದಿಗಳು ಪರ್ವತದ ಬಳಿ ಬಂದಾಗ ಗೋಮಂತಕಶಿಖರದಿಂದ ಕೆಳಗೆ ಹಾರಿ, ಜರಾಸಂಧನನ್ನೂ ಅವನ ಪರಿವಾರವರ್ಗದವರನ್ನೂ ಸೋಲಿಸಿ, ಬಂಧುವಾದ ಶೃಗಾಲವಾಸುದೇವನನ್ನು ಸಂಹರಿಸಿ ಮಥುರಾನಗರಕ್ಕೆ ಹಿಂತಿರುಗಿ ಬಂದ ಶ‍್ರೀಕೃಷ್ಣನು ನಮ್ಮನ್ನು ರಕ್ಷಿಸಲಿ.

\vskip 2pt

ಈ ಅಧ್ಯಾಯದಲ್ಲಿ ೩೩ ಶ್ಲೋಕಗಳಿವೆ.

\vskip 2pt

ಜರಾಸಂಧನು ಯಾದವರೊಡನೆ ಯುದ್ಧ ಮಾಡಲು ಸೈನ್ಯದಿಂದಲೂ ಇತರ ರಾಜರಿಂದಲೂ ಯುಕ್ತನಾಗಿ ಬರುವುದು, ಅಪರಿಮಿತ ಶಕ್ತಿಯುಕ್ತನಾದರೂ ಶ‍್ರೀಕೃಷ್ಣನು ಪಟ್ಟಣ\-ವನ್ನು ಬಿಟ್ಟು ಬಲರಾಮನಿಂದ ಸಹಿತನಾಗಿ ದಕ್ಷಿಣಕ್ಕೆ ಪ್ರಯಾಣ ಮಾಡಿದುದು, ಪರಶುರಾಮರಿಂದ ಗೋಮಂತ ಪರ್ವತದ ವೈಭವವನ್ನು ಕೇಳಿ ಅಲ್ಲಿಗೆ ಬಂದುದು, ದೇವತೆಗಳಿಂದಲೂ, ಋಷಿಗಳಿಂದಲೂ ಪೂಜೆಯನ್ನು ಹೊಂದಲು ಕ್ಷೀರಸಮುದ್ರ ಪ್ರದೇಶದಲ್ಲಿರುವವರನ್ನು ಅಮುಕ್ತ ಸ್ಥಾನಕ್ಕೆ ಕರೆತರುವುದು, ಬಲಿಯಲ್ಲಿ ಅಸುರಾವೇಶ, ಎಲ್ಲ ದೇವತೆಗಳೂ ಕಣ್ಣುಗಳನ್ನು ಮುಚ್ಚಿ ಕುಳಿತಿದ್ದಾಗ ಬಲಿಯು ಶ‍್ರೀಹರಿಯ ಕಿರೀಟವನ್ನು ತೆಗೆದುಕೊಂಡು ಪಾತಾಳಕ್ಕೆ ಓಡುವುದು, ಎಲ್ಲರೂ ನಗುವುದು, ಗರುಡನು ಹೋಗಿ ಬಲಿಯನ್ನು ಸೋಲಿಸಿ ಕಿರೀಟವನ್ನು ತರುವುದು, ಅದನ್ನು ಶ‍್ರೀಕೃಷ್ಣನ ತಲೆಯಲ್ಲಿಟ್ಟು ಸ್ತೋತ್ರ ಮಾಡುವುದು, ಶ‍್ರೀಕೃಷ್ಣ -\break ನಾರಾಯಣರ ಅಭೇದ ಪ್ರದರ್ಶನ, ಬಲರಾಮನು ತನ್ನ ಮೂರು ಪತ್ನಿಯರಿಂದ ವಿಹರಿಸುವುದು, ಜರಾಸಂಧನು ಗೋಮಂತಕ ಪರ್ವತಕ್ಕೆ ಬೆಂಕಿ ಹಚ್ಚುವುದು, ಶ‍್ರೀಕೃಷ್ಣನ ಪಾದತಾಡನದಿಂದ ಹೊರಟ ಜಲಧಾರೆಯು ಬೆಂಕಿಯನ್ನು ಆರಿಸಿದುದು, ಬಲರಾಮ ಕೃಷ್ಣರೊಡನೆ ಜರಾಸಂಧ ಮತ್ತು ಅವನ ಅನುಯಾಯಿಗಳ ಯುದ್ಧ, ಶ‍್ರೀಕೃಷ್ಣನು ಶತ್ರುಪಕ್ಷದ ಇಪ್ಪತ್ತಮೂರು ಅಕ್ಷೌಹಿಣೀ ಸೈನ್ಯವನ್ನು ನಾಶಪಡಿಸುವುದು, ಜರಾಸಂಧನು ಮೂರ್ಛೆ ಹೋಗುವುದು, ಜರಾಸಂಧನು ಬಲರಾಮನನ್ನು ಮೂರ್ಛಿತನನ್ನಾಗಿ ಮಾಡುವುದು, ಪುನಃ ಜರಾಸಂಧ-ಬಲರಾಮರ ಯುದ್ದ, ವಾಯುದೇವರು ಬಲರಾಮನಿಗೆ ಜರಾಸಂಧನ ಹಂತಕನು ಬೇರೊಬ್ಬನಿದ್ದಾನೆ ಎಂದು ತಿಳಿಸುವುದು, ಶ‍್ರೀಕೃಷ್ಣನಿಂದ ಜರಾಸಂಧನಿಗೆ ಹೊಡೆತ, ಜರಾಸಂಧನ ಪಲಾಯನ, ಶ‍್ರೀಕೃಷ್ಣನು ದಮಘೋಷನಿಂದ ಯುಕ್ತನಾಗಿ ಕರವೀರಪುರಕ್ಕೆ ಪ್ರಯಾಣಮಾಡುವುದು, ಶೃಗಾಲವಾಸುದೇವನು ಶ‍್ರೀಕೃಷ್ಣನೊಡನೆ ಯುದ್ಧಕ್ಕೆ ಬರುವುದು. ಶ‍್ರೀಕೃಷ್ಣನು ಅವನನ್ನು ಸೀಳಿ ಬಲರಾಮನೊಡನೆ ಮಥುರಾನಗರಕ್ಕೆ ಹಿಂತಿರುಗುವುದು - ಈ ವಿಷಯಗಳು ಈ ಅಧ್ಯಾಯದಲ್ಲಿ ನಿರೂಪಿತವಾಗಿವೆ.

\newpage

\section*{ಅಧ್ಯಾಯ\enginline{-}೧೭}

\begin{verse}
\textbf{ಭಗ್ನಾಶಾನ್ ನೃಪತೀನರೀನ್ ವ್ಯಥತಯೋ ಸ್ವರ್ಗಾಧಿಪಾಗ್ರ್ಯಾಸನೇ} \\\textbf{ಲಗ್ನೋ ಭೀಷ್ಮಕಸತ್ಕೃತೋಽಥ ಯವನಂ ಜಘ್ನೇ ಸತೀಮಾತ್ಮನಃ~।}\\\textbf{ನಿಘ್ನಾಂ ಯೋಽಕೃತ ರುಕ್ಕಿಣೀಂ ಸಮಜಯದ್ದು ರ್ಗರ್ವಿರುಕ್ಮ್ಯಾ ದಿಕಾನ್} \\\textbf{ವಿಘ್ನಂ ಸತ್ರಾಜಿದಾತ್ಮಜಾಪತಿರಸೌ ಮೇಘ್ನನ್ ಭವೇತ್ಸರ್ವದಾ~।।}
\end{verse}

ಶತ್ರುರಾಜರ ಆಶೋತ್ತರಗಳನ್ನು ಪುಡಿಪುಡಿಮಾಡಿ, ಇಂದ್ರನಿಂದ ಕಳಿಸಲ್ಪಟ್ಟ ಶ್ರೇಷ್ಠವಾದ ಆಸನವನ್ನು ಅಲಂಕರಿಸಿ, ಭೀಷ್ಮಕರಾಜನ ಆತಿಥ್ಯವನ್ನು ಸ್ವೀಕರಿಸಿ ಕಾಲಯವನನನ್ನು ಸಂಹರಿಸಿ, ರುಕ್ಷ್ಮಿಣೀದೇವಿಯನ್ನು ಪತ್ನಿಯನ್ನಾಗಿ ಸ್ವೀಕರಿಸಿ, ಮದೋನ್ಮತ್ತನಾದ ರುಕ್ಷ್ಮಿಯನ್ನೂ ಅವನ ಅನುಯಾಯಿಗಳನ್ನೂ ಪರಾಭವಗೊಳಿಸಿದ, ಸತ್ರಾಜಿತ್ ರಾಜನಪುತ್ರಿ ಸತ್ಯಭಾಮಾದೇವಿಯ ಪತಿಯಾದ ಶ‍್ರೀಕೃಷ್ಣನು ನಮ್ಮ ವಿಘ್ನಗಳನ್ನು ಪರಿಹರಿಸಲಿ.

ಈ ಅಧ್ಯಾಯದಲ್ಲಿ ೨೯೮ ಶ್ಲೋಕಗಳಿವೆ.

ರುಕ್ಷ್ಮಿಣೀದೇವಿಯನ್ನು ಶ‍್ರೀಕೃಷ್ಣನಿಗೆ ಕೊಡಲು ಸಿದ್ಧರಾದ ಭೀಷ್ಮಕ ಮುಂತಾದವರನ್ನು ರುಕ್ಮಿಯು ತಡೆದುದು, ಸ್ವಯಂವರಕ್ಕೆ ಶ‍್ರೀಕೃಷ್ಣನು ಗರುಡಾರೂಢನಾಗಿ ಬರುವುದು, ಜರಾಸಂಧನಿಂದ ಶ‍್ರೀಕೃಷ್ಣನ ಪರಾಕ್ರಮದ ಶ್ಲಾಘನೆ, ಶಿಶುಪಾಲ-ದಂತವಕ್ರರಿಂದ ಶ‍್ರೀಕೃಷ್ಣನ ಸರ್ವೋತ್ತಮತ್ವ ಪ್ರತಿಪಾದನೆ, ಜರಾಸಂಧಾದಿಗಳಿಂದ ಪಿತೂರಿ, ದೇವೇಂದ್ರನಿಂದ ಜರಾಸಂಧನಿಗೆ ಎಚ್ಚರಿಕೆ, ಇಂದ್ರನು ಶ‍್ರೀಕೃಷ್ಣನಿಗೆ ಶ್ರೇಷ್ಠವಾದ ಆಸನವನ್ನು ಕಳಿಸುವುದು, ಎಲ್ಲ ರಾಜರೂ ಶ‍್ರೀಕೃಷ್ಣನಿಗೆ ಅಭಿಷೇಕ ಮಾಡುವುದು, ಶ‍್ರೀಕೃಷ್ಣನಿಂದ ಭೀಷ್ಮಕರಾಜನಿಗೆ ಹಿತನುಡಿಗಳು, ತನ್ನ ವಿಶ್ವರೂಪವನ್ನು ಪ್ರಕಟಿಸಿ ಭೀಷ್ಮಕನಿಗೆ ತಾನು ಯಾರೆಂಬುದನ್ನು ತೋರಿಸುವುದು, ಜರಾಸಂಧನಿಗೆ ರುದ್ರದೇವರಲ್ಲಿದ್ದ ಭಕ್ತಿ, ಗರ್ಗಾಚಾರ್ಯರು ಕೃಷ್ಣನಲ್ಲಿ ವಿರೋಧ ಮಾಡುವುದು, ಕಾಲಯವನನ ಜನನ, ಜರಾಸಂಧನ ಅಸಹಾಯಕತೆ. ಸಾಲ್ವನು ಕಾಲಯವನನ ಬಳಿ ಹೋಗುವುದು, ಶ‍್ರೀಕೃಷ್ಣನಿಂದ ದ್ವಾರಕಾಪಟ್ಟಣ ನಿರ್ಮಾಣ, ಮಥುರಾ ಪಟ್ಟಣದಲ್ಲಿರುವವರನ್ನು ಕ್ಷಣಮಾತ್ರದಲ್ಲಿ ದ್ವಾರಕಾಕ್ಕೆ ಕಳಿಸುವುದು, ಘಟದಲ್ಲಿ ಸರ್ಪವನ್ನು ಇಟ್ಟು ಕೃಷ್ಣನು ಕಾಲಯವನನಿಗೆ ಕಳಿಸುವುದು, ಕಾಲಯವನನು ಘಟದಲ್ಲಿ ಇರುವೆಗಳನ್ನು ತುಂಬಿ ಕಳಿಸುವುದು, ಮುಚುಕುಂದನ ವೃತ್ತಾಂತ, ಮುಚುಕಂದನ ನೋಟದಿಂದ ಕಾಲಯವನನು ಭಸ್ಮಿ ಭೂತನಾಗುವುದು, ದೇವತೆಗಳನ್ನು ತೃಪ್ತಿ ಪಡಿಸುವುದು ಎಂದಿಗೂ ವ್ಯರ್ಥವಲ್ಲವೆಂಬ ಪ್ರಮೇಯ ನಿರೂಪಣೆ, ಬ್ರಾಹ್ಮಣನ ವಚನವನ್ನು ಕೇಳಿ ಕೃಷ್ಣನು ವಿದರ್ಭಕ್ಕೆ ಹೋಗುವುದು. ಎಲ್ಲ ರಾಜರೂ ನೋಡುತ್ತಿರುವಂತೆಯೇ ಕೃಷ್ಣನು ರುಕ್ಮಿಣಿಯನ್ನು ರಥದಲ್ಲಿ ಕೂಡಿಸಿಕೊಂಡು ಹೋಗುವುದು, ಬಲರಾಮನಿಂದ ಜರಾಸಂಧನಿಗೆ ಹೊಡೆತ, ಸಾತ್ಯಕಿ-ಶಿಶುಪಾಲರ ಯುದ್ಧ, ಏಕಲವ್ಯ ದಂತವಕ್ರಾದಿಗಳ ಪಲಾಯನ, ಕೃಷ್ಣನಿಂದ ರುಕ್ಮಿಗೆ ಅಪಮಾನ, ಕೃಷ್ಣ-\-ರುಕ್ಮಿಣಿಯರು ದ್ವಾರಕಾಕ್ಕೆ ಹೋಗುವುದು, ರೇವತಿ-ಬಲರಾಮರ ವಿವಾಹ, ಪ್ರದ್ಯುಮ್ನನ ಜನನ, ಶಂಬರಾಸುರ-ರತಿ ಇವರ ವೃತ್ತಾಂತ, ಪ್ರದ್ಯುಮ್ನ -ರತಿಯರ ಲಗ್ನ, ಪ್ರದ್ಯುಮ್ಮನಿಂದ ಶಂಬರಾಸುರನ ಸಂಹಾರ, ಸತ್ರಾಜಿತನ ಕಥೆ, ಸ್ಯಮಂತಕಮಣಿ ವ್ಯಾಜದಿಂದ ಶ‍್ರೀಕೃಷ್ಣನಿಗೆ ಅಪವಾದ, ಜಾಂಬವಂತನಿಗೆ ಶ‍್ರೀರಾಮರೂಪದಿಂದ ದರ್ಶನ ಕೊಡುವುದು, ಜಾಂಬವತಿಯನ್ನು ಪತ್ನಿಯನ್ನಾಗಿ ಸ್ವೀಕರಿಸುವುದು, ಸಾಕ್ಷಾತ್ ಲಕ್ಷ್ಮಿ ಸ್ವರೂಪಳಾದ ಸತ್ಯಭಾಮಾದೇವಿಯನ್ನು ಶ‍್ರೀಕೃಷ್ಣನು ಪತ್ನಿಯನ್ನಾಗಿ ಪಡೆಯುವುದು, ಹಂಸಡಿಭಿಕರ ವಿಚಾರ, ಶ‍್ರೀಕೃಷ್ಣನೊಡನೆ ಹಂಸಡಿಭೀಕರ ಯುದ್ಧ, ಬಲರಾಮನಿಂದ ಹಿಡಿಂಬಾಸುರನ ಸಂಹಾರ, ಹಂಸಡಿಭಿಕರ ನಾಶ, ಈ ವಿಷಯಗಳು ಈ ಅಧ್ಯಾಯದಲ್ಲಿ ನಿರೂಪಿತವಾಗಿವೆ.

\vspace{-.12cm}

\section*{ಅಧ್ಯಾಯ\enginline{-}೧೮}

\begin{verse}
\textbf{ಅಸ್ತ್ರೇಷ್ವಧಿಕೋಽರ್ಜುನೋಽಥ ಯದನುಕ್ರೋಶೇನ ಭೀಮಂ ವಿನಾ}\\\textbf{ಸದ್ಧರ್ಮೇ ನಿರತಂ ದದೌ ಸ್ವಗುರವೇ ಬಧ್ವಾ ನೃಪಂ ಪಾರ್ಷತಮ್~।}\\\textbf{ಪುತ್ರೌ ಸದ್ರುಪದೋsಪಿ ವಹ್ನಿವಿಬುಧಾತ್ ಸ್ತ್ರೀರೂಪಕಂ ಪ್ರಾಪ್ತವಾನ್}\\\textbf{ಇಷ್ಟಾಂ ಧರ್ಮಜ ಆಪ ರಾಜ್ಯ ಪದವೀಂ ಸಃ ಪ್ರೀಯತಾಂ ಮೇ ಹರಿಃ~।।}
\end{verse}

\vskip 2pt

ನಿರಂತರವೂ ಭಾಗವತಧರ್ಮದಲ್ಲಿಯೇ ಇರುವ ಭೀಮಸೇನನನ್ನು ಬಿಟ್ಟು ಉಳಿದ ವೀರರಲ್ಲಿ ಅರ್ಜುನನನ್ನು ಅಸ್ತ್ರವಿದ್ಯೆಯಲ್ಲಿ ಕೃಪೆಯಿಂದ ಅಗ್ರಗಣ್ಯನನ್ನಾಗಿ ಮಾಡಿದ, ಅರ್ಜುನನು ದ್ರುಪದರಾಜನನ್ನು ಕಟ್ಟಿ ದ್ರೋಣರ ಮುಂದೆ ನಿಲ್ಲಿಸುವಂತೆ ಮಾಡಿದ, ಆ ದ್ರುಪದನು ಅಗ್ನಿ ಮತ್ತು ಭಾರತೀದೇವಿಯರ ಅಂಶಭೂತರಾದ ಮಕ್ಕಳನ್ನು ಪಡೆಯುವಂತೆ ಮಾಡಿದ ಮತ್ತು ಧರ್ಮರಾಜನಿಗೆ ಇಷ್ಟವಾದ ರಾಜ್ಯ ಪದವಿಯನ್ನು ದೊರಕಿಸಿಕೊಟ್ಟ ಶ‍್ರೀಹರಿಯು ನನ್ನಲ್ಲಿ ಪ್ರೀತನಾಗಲಿ.

ಈ ಅಧ್ಯಾಯದಲ್ಲಿ ೧೯೯ ಶ್ಲೋಕಗಳಿವೆ.

ಅರ್ಜುನನು ಎಲ್ಲರಿಗಿಂತ ಶ್ರೇಷ್ಠನಾದ ವೀರನಾದುದು (ಭೀಮಸೇನನನ್ನು ಹೊರತಾಗಿ), ಭೀಮಸೇನನಿಗೆ ತನ್ನ ಪ್ರತಿಭೆಯಿಂದಲೇ ಅಸ್ತ್ರವಿದ್ಯೆಯು ತಿಳಿದಿರುವುದು, ಭೀಮಸೇನನ ವೈಷ್ಣವಧರ್ಮನಿಷ್ಠೆ, ಅವೈಷ್ಣವರಲ್ಲಿ ಭೀಮಸೇನನು ಮಾಡುತ್ತಿದ್ದ ದ್ವೇಷ, ಭೀಮಸೇನನು ಯಾರಲ್ಲಿಯೂ ಯಾವುದನ್ನೂ ಯಾವ ಕಾಲದಲ್ಲಿಯ ಬೇಡದೇ ಇರುತ್ತಿದ್ದುದು, ಇತರರು ಕೃಷ್ಣನ ವಿಷಯದಲ್ಲಿ ನಡೆದುಕೊಳ್ಳುತ್ತಿದ್ದ ಬಗೆ, ದ್ರೋಣಾಚಾರ್ಯರ ಶಿಷ್ಯರೆಲ್ಲರೂ ತಮ್ಮ ಅಸ್ತ್ರವಿದ್ಯೆಯನ್ನು ಪ್ರದರ್ಶನ ಮಾಡಿದುದು, ಅರ್ಜುನನ ಗುರಿಯ ವೈಖರಿ, ಕರ್ಣನು ಇನ್ನೂ ಶ್ರೇಷ್ಠ\-ನೆಂದು ತೋರಿಸುವುದು, ಕರ್ಣನಿಗೆ ಅಂಗದೇಶದ ರಾಜ್ಯಾಧಿಪತ್ಯವನ್ನು ದುರ್ಯೋಧನನು ಕೊಡುವುದು, ಭೀಮಸೇನ-ದುರ್ಯೊಧನರ ಗದಾ ಶಿಕ್ಷಣ ಪ್ರದರ್ಶನ, ಗುರುದಕ್ಷಿಣೆಗೋಸ್ಕರ ದ್ರೋಣರು ಪಾಂಚಾಲರಾಜನಾದ ದ್ರುಪದನನ್ನು ಕಟ್ಟಿ ತರಲು ಶಿಷ್ಯರಿಗೆ ಆಜ್ಞೆ ಮಾಡುವುದು, ಕೌರವರು ಕರ್ಣನಿಂದ ಸಹಿತರಾಗಿ ಪಾಂಚಾಲಕ್ಕೆ ಹೋಗಿ ದ್ರುಪದನಿಂದ ಅವಮಾನವನ್ನು ಹೊಂದುವುದು, ನಂತರ ಪಾಂಡವರು ಪಾಂಚಾಲಕ್ಕೆ ಹೋಗಿ ದ್ರುಪದನನ್ನು ಪರಾಭವಗೊಳಿಸುವುದು, ಅರ್ಜುನನು ದ್ರುಪದನನ್ನು ಕಟ್ಟಿ ದ್ರೋಣರ ಮುಂದೆ ತರುವುದು, ದ್ರುಪದನು ದ್ರೋಣರಿಗೆ ಅರ್ಧರಾಜ್ಯವನ್ನು ಕೊಡುವುದು, ದ್ರೋಣರನ್ನು ಸಂಹರಿಸತಕ್ಕ ಮಗನನ್ನೂ ಅರ್ಜುನನ್ನು ವರಿಸಲು ಯೋಗ್ಯಳಾದ ಮಗಳನ್ನೂ ಪಡೆಯಲು ದ್ರುಪದನು ಯಜ್ಞ ಮಾಡು\-ವುದು, ಧೃಷ್ಟದ್ಯುಮ್ನ-ದ್ರೌಪದೀ ಯಜ್ಞಕುಂಡದಿಂದ ಆವಿರ್ಭವಿಸುವುದು, ದ್ರೌಪದೀದೇವಿಯಲ್ಲಿ ಇತರ ದೇವತಾಸ್ತ್ರೀಯರ ಅಂಶಗಳು, ದ್ರೌಪದೀದೇವಿಯರ ಸ್ವರೂಪ ವರ್ಣನೆ, ದ್ರೌಪದೀದೇವಿಯರಲ್ಲಿ ಶಚೀ, ಶ್ಯಾಮಲಾ, ಪಾರ್ವತಿ ಇತ್ಯಾದಿ ದೇವತೆಗಳ ಅಂಶಗಳಿರಲು ಕಾರಣ, ದ್ರೌಪದೀದೇವಿಗೆ ಐದು ಜನ ಪತಿಗಳಿರಲು ಸನ್ನಿವೇಶ, ಧೃತರಾಷ್ಟ್ರನು ಯುಧಿಷ್ಠಿರನಿಗೆ ಯುವ ರಾಜ್ಯ ಪದವಿಯನ್ನು ಕೊಡುವುದು, ಭೀಮಾರ್ಜುನರ ದಿಗ್ವಿಜಯ ಈ ವಿಷಯಗಳು ಈ ಅಧ್ಯಾಯದಲ್ಲಿ ವಿವರಿಸಲ್ಪಟ್ಟಿವೆ.


\section*{ಅಧ್ಯಾಯ\enginline{-}೧೯}

\begin{verse}
\textbf{ಯತ್ಕಾರುಣ್ಯ ಬಲೇನ ಪಾಂಡುತನಯಾ ನಿಸ್ತೀರ್ಯ ನಾನಾಪದೋ}\\\textbf{ಭಿಕ್ಷಾನ್ನಾಶಿನ ಆಗಮಾಭ್ಯಾಸನಿನೋ ಹತ್ವಾ ಬಕಂ ದ್ರೌಪದೀಮ್~।}\\\textbf{ಉದ್ವಾಹ್ಯಾಖಿಲಭೂಪತೀನಪಿ ರಣೇ ಜಿತ್ವಾ ಗತಾಃ ಸ್ವಾಂ ಪುರೀಂ} \\\textbf{ಇಂದ್ರಪ್ರಸ್ಥಪುರೇಽವಸನ್ ಕೃತಧರಾ ರಕ್ಷಾಃ ಸ ನೋಽವ್ಯಾದ್ಧರಿಃ~।।}
\end{verse}

ಯಾವ ಶ‍್ರೀಹರಿಯ ಕೃಪಾಬಲದಿಂದ ಪಾಂಡವರು ತಮಗೆ ಒದಗಿದ ಅನೇಕ ಆಪತ್ತುಗಳನ್ನು ದಾಟಿ, ಏಕಚಕ್ರನಗರದಲ್ಲಿದ್ದಾಗ ಭಿಕ್ಷಾನ್ನದಿಂದಲೂ, ವೇದಾಭ್ಯಾಸದಿಂದಲೂ ಜೀವನ\-ವನ್ನು ನಡೆಸಿ, ಬಕನನ್ನು ಸಂಹರಿಸಿ, ದ್ರೌಪದಿಯನ್ನು ವಿವಾಹಮಾಡಿಕೊಂಡು, ಯುದ್ದದಲ್ಲಿ ಶತ್ರುರಾಜರನ್ನು ಸೋಲಿಸಿ, ತಮ್ಮ ಪಟ್ಟಣವಾದ ಇಂದ್ರಪ್ರಸ್ಥಪುರದಲ್ಲಿರುತ್ತಾ ತಮ್ಮ ರಾಜ್ಯವನ್ನು ಪಾಲಿಸಿದರೋ ಅಂತಹ ಶ‍್ರೀಹರಿಯು ನಮ್ಮನ್ನು ರಕ್ಷಿಸಲಿ.

ಈ ಅಧ್ಯಾಯದಲ್ಲಿ ೨೨೩ ಶ್ಲೋಕಗಳಿವೆ.

ದುರ್ಯೊಧನಾದಿಗಳು ಶಕುನಿಯ ಉಪದೇಶದಿಂದ ನಾಸ್ತಿಕ್ಯಧರ್ಮವನ್ನು ಸ್ವೀಕರಿಸುವುದು, ನಾಸ್ತಿಕ್ಯದ ಸ್ವರೂಪ ವರ್ಣನೆ, ಶಕುನಿಯ ಕಪಟವಿದ್ಯೆ, ದುರ್ಯೋಧನನು ಧೃತರಾಷ್ಟ್ರನಿಗೆ ದುರ್ಬೋಧನೆ ಮಾಡುವುದು, ಪಾಂಡವರನ್ನು ವಾರಣಾವತಕ್ಕೆ ಕಳಿಸಲು ಸಂಚು, ಭೀಮಸೇನನ ಪ್ರತಿಭಟನೆ, ನಂತರ ತೆರಳುವುದು, ಲಾಕ್ಷಾಗೃಹದಲ್ಲಿ ಪಾಂಡವರು ವಾಸಿಸುವಂತೆ ಮಾಡುವುದು, ಈ ವಿಷಯವನ್ನು ವಿದುರನು ಮ್ಲೇಂಛ ಭಾಷೆಯಲ್ಲಿ ಧರ್ಮರಾಜನಿಗೆ ತಿಳಿಸುವುದು, ಪುರೋಚನನ ಕಪಟವಿದ್ಯೆ, ವಿದುರನು ಲಾಕ್ಷಾಗೃಹದಿಂದ ಸುರಂಗಮಾರ್ಗವನ್ನು ಏರ್ಪಡಿಸುವುದು, ಭೀಮಸೇನನು ಪುರೋಚನ ಹಾಗೂ ಅವನ ಸಹೋದರಿ\-ಯನ್ನು ಸಂಹರಿಸಿ, ಲಾಕ್ಷಾಗೃಹದಿಂದ ಸುರಂಗಮಾರ್ಗದಿಂದ ಸಹೋದರರನ್ನೂ ತಾಯಿಯನ್ನೂ ಹೊತ್ತುಕೊಂಡು ಹೋಗಿ ಗಂಗಾನದಿಯನ್ನು ದಾಟಿ ಅರಣ್ಯವನ್ನು ಸೇರುವುದು, ದುರ್ಯೊಧನಾದಿಗಳು ಪಾಂಡವರು ಲಾಕ್ಷಾಗೃಹದಲ್ಲಿ ಮೃತರಾದರೆಂದು ಕಪಟ ದುಃಖವನ್ನು ವ್ಯಕ್ತಪಡಿಸಿ ತಿಲಾಂಜಲಿಯನ್ನು ಕೊಡುವುದು, ಹಿಡಿಂಬಿಯ ಪೂರ್ವ ವೃತ್ತಾಂತ, ಅವಳ ಸಹೋದರನಾದ ಹಿಡಿಂಬನು ಭೀಮಸೇನನಿಂದ ಮರಣಹೊಂದುವುದು, ಅರಣ್ಯದಲ್ಲಿ ವೇದವ್ಯಾಸರು ಪ್ರಾದುರ್ಭವಿಸಿ ಭೀಮಸೇನನಿಗೆ ಹಿಡಿಂಬಿಯನ್ನು ಮದುವೆಯಾಗಲು ಹೇಳುವುದು, ಘಟೋತ್ಕಚನ ಜನನ, ಪಾಂಡವರು ಏಕಚಕ್ರನಗರಕ್ಕೆ ಬ್ರಾಹ್ಮಣನ ವೇಷದಿಂದ ತೆರಳಿ ಒಬ್ಬ ಬ್ರಾಹ್ಮಣನ ಮನೆಯಲ್ಲಿ ವಾಸಮಾಡುವುದು, ಭೀಮಸೇನನು ವೈಶ್ಯನ ಮನೆಯಲ್ಲಿ ಹೂಂಕಾರದಿಂದ ಭಿಕ್ಷಾನ್ನವನ್ನು ತರುತ್ತಿದ್ದುದು, ಪಾಂಡವರು ವೇದಾಭ್ಯಾಸ ಮಾಡುವುದು, ನಂತರ ಭೀಮನನ್ನು ಮನೆಯಲ್ಲಿಯೇ ಇರಲು ಹೇಳಿ ಅರ್ಜುನಾದಿಗಳು ಭಿಕ್ಷೆಗೆ ಹೋಗುವುದು, ಆ ಮನೆಯ ಬ್ರಾಹ್ಮಣನ ರೋದನದ ಕಾರಣವನ್ನು ಕುಂತಿಯಿಂದ ತಿಳಿದು, ಬಕಾಸುರನನ್ನು ಸಂಹರಿಸಲು ಭೀಮಸೇನನು ಅನ್ನ ಭಕ್ಷಾದಿಗಳನ್ನು ತೆಗೆದುಕೊಂಡು ಗಾಡಿಯಲ್ಲಿ ಹೋಗುವುದು, ದಾರಿಯಲ್ಲಿ ಭೀಮಸೇನನೇ ಎಲ್ಲವನ್ನೂ ಭಕ್ಷಿಸಿ, ಬಕಾಸುರನೊಡನೆ ಯುದ್ಧ ಮಾಡಿ ಅವನನ್ನು ಸಂಹರಿಸುವುದು, ಬಕಾಸುರನ ಮೃತ ಶರೀರವನ್ನು ಏಕಚಕ್ರನಗರಕ್ಕೆ ತರುವುದು, ದ್ರೌಪದೀ ಸ್ವಯಂವರಕ್ಕೆ ಏಕಚಕ್ರ ನಗರದ ಬ್ರಾಹ್ಮಣರ ಸಂಗಡ ಪಾಂಡವರೂ ಬ್ರಾಹ್ಮಣ ವೇಷದಲ್ಲಿಯೇ ತೆರಳುವುದು, ಶ‍್ರೀಕೃಷ್ಣನ ಆಗಮನ, ಚಿತ್ರರಥನ ವೃತ್ತಾಂತ, ಪಾಂಡವರು ಧೌಮ್ಯರನ್ನು ಪುರೋಹಿತರನ್ನಾಗಿ ತೆಗೆದುಕೊಳ್ಳುವುದು, ಸ್ವಯಂವರದ ನಿಯಮವನ್ನು ಧೃಷ್ಟ\-ದ್ಯುಮ್ನನು ಸಭೆಯಲ್ಲಿ ವಿವರಿಸುವುದು, ಜರಾಸಂಧಾದಿ ವೀರರೆಲ್ಲರೂ ಮತ್ಸ್ಯಯಂತ್ರದ ಪರೀಕ್ಷೆಯಲ್ಲಿ ವಿಫಲರಾಗುವುದು, ಅರ್ಜುನನು ಅದನ್ನು ಭೇದಿಸುವುದು, ಭೀಮಸೇನನು ಅನೇಕ ಕ್ಷತ್ರಿಯರಾಜರನ್ನು ಹೊಡೆದೋಡಿಸುವುದು, ಕರ್ಣ-ಅರ್ಜುನರ ಧನುರ್ಯುದ್ದ, ಕರ್ಣನು ಹೊರಟುಹೋಗುವುದು, ಬ್ರಾಹ್ಮಣ ವೇಷದಲ್ಲಿದ್ದ ಪಾಂಡವರನ್ನು ಪರೀಕ್ಷಿಸಿ ಅವರು ಯಾರೆಂದು ಕಂಡುಹಿಡಿಯಲು ಧೃಷ್ಟದ್ಯುಮ್ನನ ಪ್ರಯತ್ನ, ಧರ್ಮರಾಜನು ದ್ರುಪದನಿಗೆ ನಿಜವಾದ ವಿಷಯವನ್ನು ಹೇಳುವುದು, ಐದುಜನರಿಗೂ ಸೇರಿ ದ್ರೌಪದಿಯನ್ನು ಕೊಡಲು ದ್ರುಪದನು ಒಪ್ಪದೇ ಇರುವುದು, ವೇದವ್ಯಾಸರು ಪ್ರಾದುರ್ಭವಿಸಿ ದ್ರುಪದನಿಗೆ ಹಿತೋಕ್ತಿಯನ್ನು ಹೇಳಿ, ಸ್ವರ್ಗದಲ್ಲಿರುವ ಪಾಂಡುರಾಜನ ಮೂಲರೂಪವನ್ನೂ, ಶಚಿ, ಶ್ಯಾಮಲಾ, ಉಷಾ ರೂಪಗಳಿಂದ ಯುಕ್ತಳಾದ ಭಾರತಿದೇವಿಯರನ್ನೂ ದ್ರುಪದನಿಗೆ ತೋರಿಸುವುದು, ಸಂತೋಷಗೊಂಡ ದ್ರುಪದನು ಪಾಂಡವರಿಗೆ ದ್ರೌಪದಿಯನ್ನು ಕೊಟ್ಟು ವಿವಾಹ ನಡೆಸುವುದು, ಕೃಷ್ಣನು ದ್ರೌಪದಿಗೂ, ಪಾಂಡವರಿಗೂ ಉಡುಗೊರೆಗಳನ್ನು ನೀಡುವುದು, ದುರ್ಯೊಧನಾದಿಗಳಿಗೂ-ದ್ರುಪದನಿಗೂ ಆದ ಯುದ್ಧ, ಪಾಂಡವರು ದುರ್ಯೊಧನಾದಿಗಳಿಗೆ ಪರಾಭವವನ್ನು ಉಂಟುಮಾಡುವುದು, ದುರ್ಯೋಧನನ ಸಂಕಟ, ವಿದುರನು ಧೃತರಾಷ್ಟ್ರ ಮತ್ತು ಅವನ ಮಕ್ಕಳಿಗೆ ಬುದ್ಧಿಯನ್ನು ಹೇಳುವುದು, ವಿದುರನು ಪಾಂಚಾಲನಗರದಿಂದ ಪಾಂಡವರನ್ನೂ, ದ್ರೌಪದಿಯನ್ನೂ, ಕುಂತಿದೇವಿಯನ್ನು ಹಸ್ತಿನಾವತೀ ನಗರಕ್ಕೆ ಕರೆದುಕೊಂಡುಬರುವುದು, ಪಾಂಡವರು ದ್ರೌಪದಿ ಯೊಡನೆ ಕ್ರೀಡಿಸುತ್ತಿದ್ದ ಕ್ರಮ, ದೇವತೆಗಳ ಭೋಗದ ವಿಲಕ್ಷಣಗಳು, ಕಾಶಿರಾಜನ ಮಗಳನ್ನು ದುರ್ಯೋಧನನು ಬಲಾತ್ಕಾರದಿಂದ ಒಯ್ಯುವುದು, ಜರಾ\-ಸಂಧ-ಕರ್ಣರ ಯುದ್ದ ನಂತರ ಅವರಿಬ್ಬರ ಸ್ನೇಹ, ದುರ್ಯೊಧನಾದಿಗಳು ಬೇರೆ ಬೇರೆ ಕನ್ನಿಕೆಯರನ್ನು ಲಗ್ನವಾಗುವುದು, ಕಲಿಂಗರಾಜನ ಮಗಳನ್ನು ಸುಯೋಧನನು ಬಲಾತ್ಕಾರವಾಗಿ ಹಿಡಿದುಕೊಳ್ಳುವುದು, ಕಲಿಂಗರಾಜನೊಡನೆ ಸುಯೋಧನಾಧಿಗಳು ಯುದ್ಧ ಮಾಡಿ ಪರಾಜಯ ಹೊಂದುವುದು, ಪಾಂಡವರು ತಮ್ಮ ಅರ್ಧರಾಜ್ಯವನ್ನು ಆಳಲು ಇಂದ್ರಪ್ರಸ್ಥ ನಗರಕ್ಕೆ ಹೋಗುವುದು, ಹಸ್ತಿನಾಪುರದ ಸಜ್ಜನರು ಪಾಂಡವರನ್ನು ಹಿಂಬಾಲಿಸಿ ಹೋಗುವುದು, ವಿಶ್ವಕರ್ಮನಿಂದ ಇಂದ್ರಪ್ರಸ್ಥಪುರವು ರತ್ನಮಯವಾಗಿಯೂ ಸುಂದರವಾಗಿಯೂ ನಿರ್ಮಾಣವಾಗುವುದು, ಹಸ್ತಿನಾಪುರದಲ್ಲಿ ಸುಯೋಧನನಿಗೆ ರಾಜ್ಯಾಭಿಷೇಕ, ಇಂದ್ರಪ್ರಸ್ಥಪುರದಲ್ಲಿ ಪಾಂಡವರು ಇರುತ್ತಾ ಧರ್ಮ ಮತ್ತು ನ್ಯಾಯದಿಂದ ರಾಜ್ಯವನ್ನು ಆಳುವುದು-ಈ ವಿಷಯಗಳು ಈ ಅಧ್ಯಾಯದಲ್ಲಿ ನಿರೂಪಿತವಾಗಿವೆ.


\section*{ಅಧ್ಯಾಯ\enginline{-}೨೦}

\begin{verse}
\textbf{ಯಃ ಪಾರ್ಥಾನ್ ಪರಿಪಾಲಯನ್ ಹರಿಪುರೇ ಸ್ತ್ರೀ ಪುತ್ರಸಂಪದ್ಯುತಾನ್}\\\textbf{ಸಂಹರ್ತಾ ಶತಧನ್ವನೋಽಷ್ಟಮಹಿಷೀಭರ್ತಾ ಸುರರ್ಷಿಸ್ತುತಃ~।}\\\textbf{ಹತ್ವಾ ಭೌಮಮಪಾಹರತ್ ಸುರತರುಂ ಬಹ್ವೀರುವಾಹಾಂಗನಾಃ }\\\textbf{ಪ್ರಾಯಚ್ಛದ್ಧರಿಸೂನವೇ ಸ್ವಸಹಜಾಂ ಪಾಯಾತ್ ಸ ನಃ ಕೇಶವಃ~।।}
\end{verse}

 ಸ್ತ್ರೀಯರು, ಪುತ್ರರು, ಸಂಪತ್ತು ಇವುಗಳಿಂದ ಯುಕ್ತರಾದ ಪಾಂಡವರನ್ನು ಇಂದ್ರಪ್ರಸ್ಥನಗರದಲ್ಲಿ ರಕ್ಷಿಸಿದ, ಶತಧನ್ವನನ್ನು ಸಂಹರಿಸಿ ಅಷ್ಟಮಹಿಷಿಯರಿಗೆ ಪತಿಯಾದ, ನಾರದರಿಂದ ಸ್ತುತಿಸಲ್ಪಟ್ಟ, ನರಕಾಸುರನನ್ನು ಕೊಂದು ಪಾರಿಜಾತವೃಕ್ಷವನ್ನು ಸ್ವರ್ಗಲೋಕದಿಂದ ತಂದ, ಅನೇಕ ಸ್ತ್ರೀಯರನ್ನು ವರಿಸಿದ, ಅರ್ಜುನನಿಗೆ ತನ್ನ ತಂಗಿಯಾದ ಸುಭದ್ರೆಯನ್ನು ಕೊಟ್ಟು ಲಗ್ನ ಮಾಡಿದ ಶ‍್ರೀಕೃಷ್ಣನು ನಮ್ಮನ್ನು ಸಲಹಲಿ.

ಈ ಅಧ್ಯಾಯದಲ್ಲಿ ೨೪೬ ಶ್ಲೋಕಗಳಿವೆ.

ಇಂದ್ರಪ್ರಸ್ಥ ನಗರದಲ್ಲಿ ಪಾಂಡವರು ರಾಜ್ಯವನ್ನು ಆಳುತ್ತಿದ್ದ ಕ್ರಮ, ಯಾರು ಯಾರು ಯಾವ ಯಾವ ಕೆಲಸ ಮಾಡುತ್ತಿದ್ದರು, ಶ‍್ರೀಕೃಷ್ಣನು ಪಾಂಡವರ ಬಳಿಯೇ ಇರುತ್ತಿದುದು, ಧರ್ಮರಾಜ-ಭೀಮಸೇನರು ಲಗ್ನ ಮಾಡಿ ಕೊಳ್ಳುವುದು, ಭೀಮಸೇನನು ವೇದಗಳ ಪ್ರಾಮಾಣ್ಯವನ್ನೂ, ವಿಷ್ಣು ಸರ್ವೊಮತ್ವವನ್ನೂ ಒತ್ತಿ ಪ್ರತಿಪಾದಿಸಿ ಪಾಶುಪತಾಗಮಗಳನ್ನು ಅಪ್ರಮಾಣವೆಂದು ಸಾರುವುದು, ಭೀಮಸೇನನು ಜರಾಸಂಧನನ್ನು ಗಂಗಾನದಿಯಲ್ಲಿ ಎಸೆಯುವುದು, ಶ‍್ರೀಕೃಷ್ಣನಿಂದ ಶತಧನ್ವನ ಸಂಹಾರ, ಬಲರಾಮನಲ್ಲಿ ದುರ್ಯೋಧನನ ಗದಾಭ್ಯಾಸ, ಸುಭದ್ರೆಯನ್ನು ದುರ್ಯೋಧನನಿಗೆ ಕೊಡುವುದಾಗಿ ಬಲರಾಮನ ವಾಗ್ದಾನ, ಸುಭದ್ರೆಯ ಸ್ವರೂಪ, ಅಕ್ರೂರನ ಬಳಿ ರತ್ನವು ಇರುವುದು, ಶ‍್ರೀಕೃಷ್ಣನು ಕಾಲಿಂದಿಯನ್ನು ಪತ್ನಿಯನ್ನಾಗಿ ಸ್ವೀಕರಿಸುವುದು, ನೀಲಾದೇವಿ, ಮಿತ್ರವಿಂದೆ, ಭದ್ರಾ-ಇವರೂ ಸಹ ಶ‍್ರೀಕೃಷ್ಣನನ್ನು ಮದುವೆಯಾಗುವುದು, ಬೃಹತ್ಸೇನನ ಮಗಳಾದ ಲಕ್ಷಣೆಯು ಶ‍್ರೀಕೃಷ್ಣನ ಪತ್ನಿಯಾದುದು, ಶ‍್ರೀ ಕೃಷ್ಣನಿಗೆ ಅರ್ಧನಾರಾಯಣ ಎಂಬ ಹೆಸರು ಬರಲು ಕಾರಣ, ರೈವತಾಚಲದಲ್ಲಿ ಶ‍್ರೀಕೃಷ್ಣನ ಮಹಿಮಾ ಪ್ರದರ್ಶನ, ನರಕಾಸುರನ ಸಂಹಾರ, ಲಕ್ಷ್ಮೀದೇವಿಯ ಆವೇಶವುಳ್ಳ ಹದಿನಾರು ಸಾವಿರದ ಒಂದು ನೂರು ಜನ ಕನ್ನಿಕೆಯರನ್ನು ಶ‍್ರೀ ಕೃಷ್ಣನು ಸ್ವೀಕರಿಸಿ ದ್ವಾರಕಾ ಪಟ್ಟಣಕ್ಕೆ ಕಳಿಸುವುದು, ಸ್ವರ್ಗಲೋಕದ ನಂದನವನದಲ್ಲಿ\break ಶ‍್ರೀಕೃಷ್ಣನು ಸತ್ಯಭಾಮೆಯೊಡನೆ ಕ್ರೀಡಿಸುವುದು, ಪಾರಿಜಾತ ವೃಕ್ಷವನ್ನು ದ್ವಾರಕಾಕ್ಕೆ ತರುವಾಗ ಇಂದ್ರಾದಿಗಳ ವಿರೋಧ, ಸತ್ಯಭಾಮಾದೇವಿಯರಿಂದ ಇಂದ್ರಾದಿಗಳಿಗೆ ಪರಾಭವ, ಶ‍್ರೀಕೃಷ್ಣನು ನರಕಾಸುರನ ಪಟ್ಟಣದಿಂದ ತರಲ್ಪಟ್ಟ ಕನ್ನಿಕೆಯರನ್ನು ಲಗ್ನವಾಗಿ ಒಬ್ಬೊಬ್ಬರಲ್ಲಿಯೂ ಹತ್ತು ಮಕ್ಕಳನ್ನು ಉತ್ಪಾದಿಸುವುದು, ನಾರದರು ಪಾಂಡವರಲ್ಲಿ ದ್ರೌಪದೀದೇವಿಯರೊಡನೆ\break ಕ್ರೀಡಿಸುವ ನಿಯಮವನ್ನು ಅನುಸರಿಸುವಂತೆ ಸೂಚಿಸುವುದು, ನಿಯಮದ ಉಲ್ಲಂಘನೆಗಾಗಿ ಅರ್ಜುನನು ತೀರ್ಥಯಾತ್ರೆಗಾಗಿ ತೆರಳುವುದು, ಅರ್ಜುನ-ಚಿತ್ರಾಂಗದಾ ಇವರ ವಿವಾಹ, ಬಭ್ರುವಾಹನನ ಜನನ, ಅರ್ಜುನನು ಪ್ರಭಾಸ ಕ್ಷೇತ್ರಕ್ಕೆ ಬರುವುದು, ಸುಭದ್ರೆಯನ್ನು ದುರ್ಯೋಧನನಿಗೆ ಕೊಡಲು ಸನ್ನಾಹ ಕೇಳಿ ಸನ್ಯಾಸವೇಷದಿಂದ ದ್ವಾರಕಾನಗರಕ್ಕೆ ಹೋಗುವುದು, ರೈವತ ಪರ್ವತದಲ್ಲಿ ಸನ್ಯಾಸಿವೇಷದ ಅರ್ಜುನನ ವಾಸ, ಸೇವೆಗೆ ಸುಭದ್ರೆಯನ್ನು ನೇಮಿಸುವುದು, ಸುಭದ್ರಾರ್ಜುನರ ವಿವಾಹ, ಸುಭದ್ರೆಯಿಂದ ವಿರೋಧಿಗಳ ಪರಾಭವ, ಶ‍್ರೀಕೃಷ್ಣನ ಹಿತೋಕ್ತಿ, ಶ‍್ರೀಕೃಷ್ಣನು ಪಾಂಡವರೊಡನೆ ಇಂದ್ರಪ್ರಸ್ಥಪುರದಲ್ಲಿ ವಾಸಿಸುವುದು, ಖಾಂಡವವನದಲ್ಲಿ ಕೃಷ್ಣ-ಸತ್ಯಭಾಮೆ ಹಾಗೂ ಅರ್ಜುನ-ಸುಭದ್ರೆಯರ ವಿಹಾರ, ಅಗ್ನಿಯು ಬ್ರಾಹ್ಮಣ ರೂಪದಲ್ಲಿ ಬಂದು ಬಲವೃದ್ಧಿಗಾಗಿ ಖಾಂಡವವನದ ಭಕ್ಷಣವನ್ನು ಬೇಡುವುದು, ಬ್ರಹ್ಮದೇವರು ಗಾಂಡೀವವನ್ನು ಅರ್ಜುನನಿಗೆ ನೀಡುವುದು, ಅಗ್ನಿಯಿಂದ ಖಾಂಡವವನ ದಹನ, ಅರ್ಜುನನಿಂದ ಅಶ್ವಸೇನನೆಂಬ ತಕ್ಷಕನ ಮಗನ ಪುಚ್ಛದ ಛೇದನ, ಅಶ್ವಸೇನನು ಅರ್ಜುನನ್ನು ಸಂಹರಿಸಲು ಕರ್ಣನ ಬತ್ತಳಿಕೆಯಲ್ಲಿ ಪ್ರವೇಶ, ಮಯಾಸುರನಿಂದ ಯುಧಿಷ್ಠಿರನಿಗೋಸ್ಕರ ವಿಚಿತ್ರ ಸಭಾ ನಿಮಾ೯ಣ ಈ ವಿಷಯಗಳು ಈ ಅಧ್ಯಾಯದಲ್ಲಿ\break ನಿರೂಪಿತವಾಗಿವೆ.


\section*{ಅಧ್ಯಾಯ\enginline{-}೨೧}

\begin{verse}
\textbf{ಪಾರ್ಥಾನ್ ಲಬ್ಧ ಸಭಾನ್ವಿಧಾಯ ಮಯತೋ ಪ್ರಾಪ್ತಃ ಪುರಂ ಸ್ವಾಂ ಗತಃ}\\\textbf{ಕ್ಷೇತ್ರಂ ಕೌರವಮರ್ಕಪರ್ವಣಿಪುರೀಂ ಸಂಪ್ರಾಪ್ಯ ಕರ್ತಾ ಕ್ರತೋಃ~।} \\\textbf{ಪಾಂಡೂನ್ ಪ್ರಾಪ್ಯ ಜರಾಸುತೇ ವಿನಿಹತೇ ತೈಃ ಕಾರಯಿತ್ವಾಽಧ್ವರಂ }\\\textbf{ಪ್ರಾಪ್ತಃ ಸ್ವಂ ಪುರಮಚ್ಯುತೋ ವಿಜಯತೇ ದ್ಯೂತೇ ಜಿತೈಶ್ಚ ಸ್ಮೃತಃ~।।}
\end{verse}

ಪಾಂಡವರಿಗೋಸ್ಕರ ಮಯನಿಂದ ಸಭಾಮಂದಿರವನ್ನು ನಿರ್ಮಾಣಮಾಡಿಸಿ ದ್ವಾರಕೆಗೆ ತೆರಳಿದ, ಸೂರ್ಯಗ್ರಹಣಕಾಲದಲ್ಲಿ ಕುರುಕ್ಷೇತ್ರಕ್ಕೆ ಬಂದು ಅಲ್ಲಿ ಯಜ್ಞವನ್ನು ಆಚರಿಸಿದ, ಪುನಃ ಪಾಂಡವರಲ್ಲಿ ಹೋಗಿ ಭೀಮಸೇನನಿಂದ ಜರಾಸಂಧನನ್ನು ಸಂಹಾರಮಾಡಿಸಿ ಪಾಂಡವ\-ರಿಂದ ರಾಜಸೂಯಯಾಗವನ್ನು ಮಾಡಿಸಿ, ತನ್ನ ಪಟ್ಟಣಕ್ಕೆ ತೆರಳಿದ, ಕಪಟದ್ಯೂತದಲ್ಲಿ ಪರಾ\-ಜಯಹೊಂದಿದ ಪಾಂಡವರಿಂದ ನಿತ್ಯದಲ್ಲಿಯೂ ಸ್ಮರಿಸಲ್ಪಡತಕ್ಕ ಶ‍್ರೀಅಚ್ಯುತನು ಉತ್ಕೃಷ್ಟನಾಗಿದ್ದಾನೆ.

ಈ ಅಧ್ಯಾಯದಲ್ಲಿ ೪೧೮ ಶ್ಲೋಕಗಳಿವೆ.

ಮಯಾಸುರನು ಶ‍್ರೀ ಕೃಷ್ಣನ ಆಜ್ಞೆಯಂತೆ ಪಾಂಡವರಿಗಾಗಿ ಸಭಾಮಂದಿರವನ್ನು ನಿರ್ಮಿಸುವುದು, ಮಯನು ಭೀಮಸೇನನಿಗೆ ವಾಯುದೇವರಿಂದ ಧರಿಸಲ್ಪಟ್ಟ ಗದೆಯನ್ನು ಕೊಡುವುದು, ಎರಡು ವರ್ಷಗಳ ಕಾಲ ಇಂದ್ರಪ್ರಸ್ಥಪುರದಲ್ಲಿ ಪಾಂಡವರಿಂದ ಪೂಜೆಗೊಂಡ ಶ‍್ರೀಕೃಷ್ಣನು ದ್ವಾರಕೆಗೆ ಹೋಗುವುದು, ಸೂರ್ಯಗ್ರಹಣ ಕಾಲದಲ್ಲಿ ಬಂಧುಜನರಿಂದಲೂ, ಭಕ್ತರಿಂದಲೂ ಕೂಡಿದ ಶ‍್ರೀಕೃಷ್ಣನು ಕುರುಕ್ಷೇತ್ರಕ್ಕೆ ಹೋಗುವುದು, ಪರಶುರಾಮ ವೇದವ್ಯಾಸ ಶ‍್ರೀಕೃಷ್ಣ ಈ ಮೂರು ರೂಪಗಳಿಂದ ಭಕ್ತರಿಗೆ ದರ್ಶನವನ್ನು ನೀಡಿ ವಸುದೇವನಿಂದ ಯಜ್ಞ ಮಾಡಿಸುವುದು, ಪುನಃ ಪಾಂಡವರಿಂದ ಯುಕ್ತನಾಗಿ ದ್ವಾರಕಿಗೆ ಬಂದು ಅಲ್ಲಿ ಒಂದು ದಿವಸದಲ್ಲಿಯೇ ಮುಗಿಯುವಂತಹ ಅಶ್ವಮೇಧಯಜ್ಞವನ್ನು ಆಚರಿಸುವುದು, ಭೀಮಾರ್ಜುನರೂ ಶ‍್ರೀಕೃಷ್ಣನ ಪುತ್ರರೂ ಕುದುರೆಯ ಹಿಂದೆ ಹೋಗಿ ಜರಾಸಂಧನೇ ಮುಂತಾದ ಅನೇಕರನ್ನು ಸೋಲಿಸುವುದು, ಶ‍್ರೀಕೃಷ್ಣನು ಯಜ್ಞದೀಕ್ಷಿತನಾಗಿದ್ದಾಗ ಒಬ್ಬ ಬ್ರಾಹ್ಮಣನು ಬಂದು ತನಗೆ ಹುಟ್ಟಿದ ಮಕ್ಕಳೆಲ್ಲರೂ ಹುಟ್ಟಿದ ಕೂಡಲೇ ಅದೃಶ್ಯರಾಗುತ್ತಾರೆಂದೂ, ತನ್ನ ಮಕ್ಕಳನ್ನು ರಕ್ಷಿಸಬೇಕೆಂದೂ ಶ‍್ರೀಕೃಷ್ಣನನ್ನು ಬೇಡುವುದು, ಅರ್ಜುನನು ಮುಂದೆ ಬಂದು ತಾನು ಆ ಮಕ್ಕಳನ್ನು ರಕ್ಷಿಸುವುದಾಗಿ ಪ್ರತಿಜ್ಞೆ ಮಾಡಿ ಬ್ರಾಹ್ಮಣನೊಡನೆ ಹೋಗುವುದು, ಅರ್ಜುನನು ಇದ್ದರೂ ಶಿಶುವು ಅದೃಶ್ಯವಾಗಲು ಅರ್ಜುನನು ಅಗ್ನಿ ಪ್ರವೇಶ ಮಾಡಲು ಸಿದ್ಧನಾಗುವುದು, ಆಗ ಶ‍್ರೀಕೃಷ್ಣನು ರಥದಲ್ಲಿ ಹೊರಟು ಏಳು ಸಮುದ್ರಗಳನ್ನು ದಾಟಿ ಆ ಮಕ್ಕಳನ್ನು ಹಿಡಿದು ತರುವುದು, ಆ ಮಕ್ಕಳ ಪೂರ್ವಜನ್ಮದ ವೃತ್ತಾಂತ, ಅರ್ಜುನನ ಗರ್ವನಾಶ, ಯಾಗದ ಪರಿಸಮಾಪ್ತಿ, ಶ‍್ರೀಕೃಷ್ಣನಿಂದ ದಂತವಕ್ರನ ಸಂಹಾರ, ಅರ್ಜುನನಿಗೆ ಶ‍್ರೀಕೃಷ್ಣನು ಬ್ರಹ್ಮಾಂಡದ ವಿವರಣೆಯನ್ನು ಬೋಧಿಸುವುದು, ಅಂಧಂತಮಸ್ಸಿನ ವರ್ಣನೆ, ಶ‍್ರೀಕೃಷ್ಣನ ಅಪ್ರಾಕೃತವಾದ ರೂಪವರ್ಣನೆ, ತನ್ನ ಸ್ವಾತಂತ್ರ್ಯದ ಪ್ರತಿಪಾದನೆ, ಪಾಂಡವರು ಇಂದ್ರಪ್ರಸ್ಥಕ್ಕೆ ವಾಪಸ್ಸು ಹೋಗುವುದು, ನಾರದರು ಪಾಂಡವರ ಬಳಿ ಬಂದು ಸ್ವರ್ಗಲೋಕಾದಿಗಳನ್ನು ವರ್ಣಿಸಿ ಹರಿಶ್ಚಂದ್ರನು ಇಂದ್ರನ ಸಭೆಯಲ್ಲಿ ಇರುವಂತೆ ಪಾಂಡುರಾಜನೂ ಇಂದ್ರನ ಸಭೆಯಲ್ಲಿ ಇರಲು ಪಾಂಡವರು ರಾಜಸೂಯಯಾಗವನ್ನು ಮಾಡಬೇಕೆಂದು ಬೋಧಿಸುವುದು, ಧರ್ಮರಾಜನು ಶ‍್ರೀಕೃಷ್ಣನನ್ನು ಬರಮಾಡಿಕೊಂಡು ಅವನ ಅಭಿಪ್ರಾಯವನ್ನು ಕೇಳುವುದು, ಶ‍್ರೀಕೃಷ್ಣನು ರಾಜಸೂಯಾಗದ ಮಹತ್ವವನ್ನೂ, ಪಾಂಡುರಾಜನಿಗೆ ಇಂದ್ರನ ಶಾಪ ವೃತ್ತಾಂತವನ್ನೂ ತಿಳಿಸಿ ಧರ್ಮರಾಜನಿಗೆ ರಾಜಸೂಯಯಾಗ ಮಾಡಲು ಹೇಳುವುದು, ರಾಜಸೂಯಯಾಗದ ಮುಖ್ಯ ಯೋಗ್ಯತೆಯು ಭೀಮಸೇನನಲ್ಲಿ ಮಾತ್ರವೇ ಎಂಬುದಾಗಿ ಶ‍್ರೀಕೃಷ್ಣನು ಹೇಳುವುದು, ಧರ್ಮರಾಜನು ಜರಾಸಂಧನೇ ಮುಂತಾದ ಪ್ರಬಲರಿಂದ ಭಯವನ್ನು ಹೊಂದುವುದು, ಭೀಮಸೇನನು ಸಮಾಧಾನ ಹೇಳುವುದು, ಜರಾಸಂಧನನ್ನು ಸಂಹರಿಸಲು ಭೀಮಸೇನನು ಮಾತ್ರವೇ ಸಮರ್ಥನೆಂದು ಶ‍್ರೀಕೃಷ್ಣನು ಸ್ಪಷ್ಟಪಡಿಸುವುದು, ಭೀಮಸೇನನು ಶ‍್ರೀಕೃಷ್ಣನ ಸರ್ವೊತ್ತಮತ್ವವನ್ನು ಒತ್ತಿ ಹೇಳುವುದು, ಶ‍್ರೀಕೃಷ್ಣ ಭೀಮಾರ್ಜುನರು ಮಗಧದೇಶಕ್ಕೆ ತೆರಳುವುದು, ಅಲ್ಲಿ ಲಿಂಗಾಕಾರವಾಗಿದ್ದ ಜರಾಸಂಧನಿಂದ ಮಾನ್ಯವಾದ ಪರ್ವತವನ್ನು ಕೆಡವಿಹಾಕುವುದು, ಭೇರಿಗಳನ್ನು ನಾಶ ಮಾಡುವುದು,\break ಬ್ರಾಹ್ಮಣ ವೇಷದಿಂದ ಜರಾಸಂಧನ ಅರಮನೆಯನ್ನು ಪ್ರವೇಶಿಸುವುದು, ಭೀಮಸೇನನ\break ಸಾಮರ್ಥ್ಯದ ವರ್ಣನೆ, ಭೀಮ-ಜರಾಸಂಧರ ಯುದ್ಧ, ವೇದ ವಚನಗಳಿಂದ ಭೀಮಸೇನನು ಜರಾಸಂಧನನ್ನು ಮೊದಲು ಪರಾಭವಗೊಳಿಸಿ ನಂತರ ಜರಾಸಂಧನ ದೇಹವನ್ನು ಸೀಳುವುದು, ಜರಾಸಂಧನ ಮಗನಾದ ಸಹದೇವನಿಂದ ಭೀಮಸೇನನಿಗೆ ರಥ ಪ್ರದಾನ, ವೇದವ್ಯಾಸರು ರಾಜಸೂಯದ ಅಂಗವಾಗಿ ಭೀಮಸೇನನನ್ನು ದಿಗ್ವಿಜಯಕ್ಕಾಗಿ ಕಳಿಸುವುದು, ಅರ್ಜುನನು ಉತ್ತರ ದಿಕ್ಕಿಗೆ ತೆರಳುವುದು, ನಕುಲಸಹದೇವರು ಪಶ್ಚಿಮ ದಕ್ಷಿಣ ದಿಕ್ಕುಗಳಿಗೆ ತೆರಳುವುದು, ಭೀಮಸೇನನ ದಿಗ್ವಿಜಯ ಯಾತ್ರೆಯ ವರ್ಣನೆ, ವಿಭೀಷಣನೂ ಸಹ ಯಾಗಕ್ಕೆ ಕಪ್ಪವನ್ನು ನೀಡುವುದು, ಅರ್ಜುನನ ಪರಾಕ್ರಮ ವರ್ಣನೆ, ರಾಜಸೂಯಯಾಗ ಪ್ರಾರಂಭ, ಯಾಗಕ್ಕೆ ಆಹ್ವಾನಿತರಾದ ದೇವತೆಗಳ ವಿವರಣೆ, ಯಾಗಸಭೆಯಲ್ಲಿ ತತ್ವ ನಿರ್ಣಯ ಚರ್ಚೆ, ಭೀಷ್ಮಾಚಾರ್ಯರಿಂದ ಶ‍್ರೀಕೃಷ್ಣನ ಸರ್ವೋತ್ತಮತ್ವದ ಪ್ರತಿಪಾದನೆ, ಶ‍್ರೀಕೃಷ್ಣನಿಗೆ ಅಗ್ರಪೂಜೆ, ಶಿಶುಪಾಲನಿಂದ ನಿಂದೆ, ಜಯವಿಜಯರಿಗೆ ಶಾಪದ ವೃತ್ತಾಂತ, ದುರ್ಯೊಧನಾದಿಗಳ ಕೋಪ, ಶ‍್ರೀಕೃಷ್ಣನು ಚಕ್ರದಿಂದ ಶಿಶುಪಾಲನ ಶಿರಸ್ಸನ್ನು ಕತ್ತರಿಸುವುದು, ರಾಜಸೂಯ\-ಯಾಗದ ವೈಭವ, ಅದರ ಪರಿಸಮಾಪ್ತಿ; ಸಭೆಯಲ್ಲಿ ಕೃಷ್ಣ, ಯುಧಿಷ್ಠಿರ ಮುಂತಾದವರು ಕುಳಿತಿರಲು ದುರ್ಯೋಧನನಿಗೆ ಆದ ಅವಮಾನ, ದ್ರೌಪದಿಯೇ ಮೊದಲಾದ ಸ್ತ್ರೀಯರು ನಗುವುದು, ಯುಧಿಷ್ಠಿರನು ದುರ್ಯೋಧನನಿಗೆ ಬೇರೆ ವಸ್ತ್ರಗಳನ್ನು ತರಿಸಿಕೊಡುವುದು, ದುರ್ಯೋಧನನ ಕೋಪ, ಶಕುನಿಯ ಕುತಂತ್ರ, ಪಾಂಡವರನ್ನು ದ್ಯೂತಕ್ಕೆ ಕರೆಸಲು ಧೃತರಾಷ್ಟ್ರನನ್ನು ಒಪ್ಪಿಸುವುದು, ವಿದುರನ ಹಿತೋಕ್ತಿ, ವಿದುರನು ಪಾಂಡವರನ್ನು ಕರೆತರುವುದು, ಧರ್ಮರಾಜನು ತನ್ನ ಪ್ರತಿಜ್ಞಾನುಸಾರವಾಗಿ ದ್ಯೂತಕ್ಕೆ ಒಪ್ಪುವುದು, ಧರ್ಮರಾಜನು ದ್ಯೂತದಲ್ಲಿ ಎಲ್ಲ ದ್ರವ್ಯವನ್ನೂ ಪಣವಾಗಿಟ್ಟು ಸೋಲುವುದು, ನಂತರ ತನ್ನ ತಮ್ಮಂದಿರನ್ನೂ ಪತ್ನಿಯಾದ ದ್ರೌಪದೀ ದೇವಿಯನ್ನೂ ಪಣವಾಗಿಟ್ಟು ಸೋಲುವುದು, ದುರ್ಯೊಧನಾದಿಗಳಿಂದ ದ್ರೌಪದೀ ದೇವಿಗೆ ಸಭೆಯಲ್ಲಿ ಅಪಮಾನ, ದ್ರೌಪದೀದೇವಿಯ ಕೋಪದ ಮಾತುಗಳು, ಭೀಷ್ಮಾಚಾರ್ಯರೇ ಮೊದಲಾದವರ ಮೌನ, ವಿದುರನ ಹಿತೋಕ್ತಿ, ಭೀಮಸೇನನ ಕೋಪ, ಕರ್ಣನ ಅಪಮಾನಕರವಾದ ನುಡಿಗಳು, ಭೀಮಸೇನನು ದುರ್ಯೊಧನನ ತೊಡೆಗಳನ್ನು ಗದೆಯಿಂದ ಸೀಳುವುದಾಗಿ ಪ್ರತಿಜ್ಞೆ ಮಾಡುವುದು, ಪಾಂಡವರ ವಸ್ತ್ರಗಳನ್ನು ದುರ್ಯೋಧನಾದಿಗಳು ಅಪಹರಿಸುವುದು, ದ್ರೌಪದೀದೇವಿಯರ ವಸ್ತ್ರಾಪಹರಣ ಕಾಲದಲ್ಲಿ ಶ‍್ರೀಕೃಷ್ಣನು ಮಾಡಿದ ಅನುಗ್ರಹ, ದ್ರೌಪದೀದೇವಿಯ ಪ್ರತಿಜ್ಞೆ, ಭೀಮಸೇನನು ದುರ್ಯೊಧನಾದಿಗಳನ್ನು ಸಂಹರಿಸಲು ಹೊರಡುವುದು, ಕೌರವರ ಭಯ, ವಿದುರನು ಧೃತರಾಷ್ಟ್ರನಿಗೆ ಕಪಟ\-ದ್ಯೂತದ ಫಲವನ್ನು ವಿವರಿಸುವುದು, ಕೌರವರಿಗೆ ಆದ ದುಶ್ಶಕುನಗಳು, ದ್ರೌಪದೀದೇವಿಯನ್ನು ಧೃತರಾಷ್ಟ್ರನು ಬಿಡುಗಡೆ ಮಾಡುವುದು ಮತ್ತು ದ್ರೌಪದೀದೇವಿಗೆ ವರಗಳನ್ನು ನೀಡುವುದು, ಪಾಂಡವರು ದ್ರೌಪದಿಯಿಂದ ಯುಕ್ತರಾಗಿ ಇಂದ್ರಪ್ರಸ್ಥಕ್ಕೆ ತೆರಳುವುದು, ಧೃತರಾಷ್ಟ್ರನು ಪಾಂಡವರನ್ನು ಪುನಃ ಬರಮಾಡಿಕೊಳ್ಳುವುದು, ಅರಣ್ಯವಾಸ-ಅಜ್ಞಾತವಾಸದ ಪಣವನ್ನಿಟ್ಟು ಪುನಃ ದ್ಯೂತವನ್ನಾಡಲು ಯುಧಿಷ್ಠಿರನನ್ನು ಒಪ್ಪಿಸುವುದು, ಕೌರವರು ದ್ರೌಪದಿಯನ್ನು ಮತ್ತೆ ಅವಮಾನ ಮಾಡುವುದು, ಕೌರವರೆಲ್ಲರನ್ನೂ ತಾನೊಬ್ಬನೆ ಸಂಹರಿಸುವುದಾಗಿ ಭೀಮಸೇನನು ಪ್ರತಿಜ್ಞೆ ಮಾಡುವುದು, ದುರ್ಯೊಧನಾದಿಗಳು ತಮ್ಮನ್ನು ರಕ್ಷಿಸುವಂತೆ ದ್ರೋಣರನ್ನು ಕೇಳಿಕೊಳ್ಳುವುದು, ಯುಧಿಷ್ಠಿರನು ದ್ಯೂತದಲ್ಲಿ ಪರಾಜಿತನಾಗಿ ಪತ್ನಿ ಹಾಗೂ ತಮ್ಮಂದಿರಿಂದ ಸಹಿತನಾಗಿ ವನಕ್ಕೆ ಹೊರಡುವುದು, ವಿದುರನು ಕುಂತಿಯನ್ನು ತಡೆಯುವುದು, ಪಾಂಡವರು ಕೌರವರ ವಿಷಯದಲ್ಲಿ ಪ್ರತ್ಯೇಕ ಪ್ರತ್ಯೇಕ ಪ್ರತಿಜ್ಞೆ ಮಾಡುವುದು, ಪುರೋಹಿತರಾದ ಧೌಮ್ಯರು ಪ್ರೇತಸಂಸ್ಕಾರ ಮಂತ್ರಗಳನ್ನು ಉಚ್ಚರಿಸುತ್ತಾ ಪಾಂಡವರ ಸಂಗಡ ವನಕ್ಕೆ ಹೊರಡುವುದು, ಪಾಂಡವರು ಭಾಗೀರಥಿ ನದಿಯ ದಡದಲ್ಲಿ ವಟವೃಕ್ಷದ ಕೆಳಗೆ ಕುಳಿತುಕೊಂಡು ಸರ್ವೆಶ್ವರನಾದ ಶ‍್ರೀಮನ್ನಾರಾಯಣನನ್ನು ಭಕ್ತಿಯಿಂದ ಸ್ಮರಿಸುವುದು ಈ ವಿವರಗಳು ಈ ಅಧ್ಯಾಯದಲ್ಲಿ ನಿರೂಪಿತವಾಗಿವೆ.


\section*{ಅಧ್ಯಾಯ\enginline{-}೨೨}

\begin{verse}
\textbf{ಪಾರ್ಥಾಃ ಯಾತಾಽರಣ್ಯಂ ನಿಹಿತನಿಶಿಚರಾಃ ಪ್ರೀಣಯಂತೋ ದ್ವಿಜೌಘಾನ್}\\\textbf{ವಾರ್ತಾಂ ಶ್ರುತ್ವಾ ಸ್ವಕೀಯಾಮುಗತಹರಿಣಾ ಮಾನಿತಾ ಸಿಂಧುರಾಜಮ್~।}\\\textbf{ಜಿತ್ವಾ ದುರ್ಯೊಧನಾದೀನ್ ಹರಿಹಯಪುರುಷವ್ರಾತಬದ್ಧಾನ್ವಿಮೋಚ್ಯ}\\\textbf{ಪ್ರಾಪ್ತಾ ಧರ್ಮಪ್ರಸಾದಂ ಯಮಥ ಮಧುರಿಪುಂ ತುಷ್ಟುವುಸ್ತಂ ಪ್ರಪದ್ಯೇ~।।}
\end{verse}

ಪಾಂಡವರು ವನಪ್ರವೇಶಮಾಡಿ ಅನೇಕ ರಾಕ್ಷಸರನ್ನು ಸಂಹರಿಸಿ, ಬ್ರಾಹ್ಮಣರ ಸಮೂಹವನ್ನು ತೃಪ್ತಿ ಪಡಿಸಿದ ವಿಷಯವನ್ನು ಕೇಳಿ, ಅವರ ಬಳಿ ಬಂದ ದುರ್ಯೊಧನನೇ ಮೊದಲಾದವರನ್ನು ಬಂಧಿಸಿದ ಗಂಧರ್ವರನ್ನು ಪಾಂಡವರಿಂದ ಸೋಲಿಸಿ, ಕೌರವರನ್ನು ಬಂಧನದಿಂದ ಬಿಡಿಸಿದ, ಯಕ್ಷರೂಪದಿಂದ ಧರ್ಮರಾಜನಿಗೆ ಧರ್ಮಸ್ವರೂಪವನ್ನು ಉಪದೇಶಿಸಿ ಪಾಂಡವರನ್ನು ಸಂತೋಷಪಡಿಸಿ ಅವರಿಂದ ಸನ್ಮಾನಿತನಾದ ಮಧುಸೂದನನನ್ನು ಶರಣು ಹೊಂದುತ್ತೇನೆ.

ಈ ಅಧ್ಯಾಯದಲ್ಲಿ ೪೫೯ ಶ್ಲೋಕಗಳಿವೆ.

ಅರಣ್ಯದಲ್ಲಿ ಭೀಮಸೇನನು ಕಿರ್ಮೀರನೆಂಬ ರಾಕ್ಷಸನನ್ನು ಸಂಹಾರ ಮಾಡಿದುದು, ತೊಂಭತ್ತೆಂಟು ಸಹಸ್ರ ಮುನಿಗಳೂ, ಗೃಹಸ್ಥರೂ ಇವರಿಂದ ಕೂಡಿಕೊಂಡು ವಾಸಮಾಡು\-ತ್ತಿದ್ದುದು, ಅವರ ಭೋಜನಾದಿಗಳಿಗಾಗಿ ಧರ್ಮರಾಜನು ಸೂರ್ಯಾಂತರ್ಯಾಮಿಯಾದ ನಾರಾಯಣನಿಂದ ಭಕ್ಷ ಭೋಜ್ಯವನ್ನು ಕೊಡತಕ್ಕ ಪಾತ್ರೆಯನ್ನು ಪಡೆದುದು, ನಿತ್ಯದಲ್ಲಿಯೂ ಹರಿಮಹಿಮೆಗಳನ್ನು ಶ್ರವಣಮಾಡುತ್ತಿದುದು, ವಿದುರನು ಪಾಂಡವರಲ್ಲಿ ಬಂದು ಪುನಃ ಧೃತರಾಷ್ಟ್ರನನ್ನು ಸೇರಿದುದು, ಧೃತರಾಷ್ಟ್ರನಿಗೆ ವೇದವ್ಯಾಸರ ಹಿತೋಕ್ತಿ, ಮೈತ್ರೇಯರು ದುರ್ಯೋಧನನಿಗೆ ಶಾಪ ಕೊಡುವುದು, ಕೃಷ್ಣ ದ್ರುಪದರು ಅರಣ್ಯಕ್ಕೆ ಬಂದು ಪಾಂಡವರ ಯೋಗಕ್ಷೇಮವನ್ನು ವಿಚಾರಿಸುವುದು, ಕೃಷ್ಣನು ದ್ರೌಪದಿಯನ್ನು ಸಮಾಧಾನ ಪಡಿಸುವುದು, ಸಾಲ್ವನ ವೃತ್ತಾಂತ, ಸಾಲ್ವನು ಪ್ರದ್ಯುಮ್ನನ ಜತೆ ಯುದ್ಧ ಮಾಡುವುದು, ಕೃಷ್ಣನು ಸಾಲ್ವನನ್ನು ಸಂಹರಿಸುವುದು, ದ್ರೌಪದಿಯನ್ನು ಬಿಟ್ಟು ಪಾಂಡವರ ಇತರ ಪತ್ನಿಯರು ಕುಂತಿಯಿಂದ ಯುಕ್ತರಾಗಿ ವಿದುರನ ಬಳಿ ಇರುತ್ತಿದುದು, ಕೃಷ್ಣನು ಸತ್ಯಭಾಮಾದೇವಿಯಿಂದ ಸಹಿತನಾಗಿ ಅರಣ್ಯದಿಂದ ದ್ವಾರಕಾನಗರಕ್ಕೆ ಹಿಂದಿರುಗುವುದು, ಧರ್ಮರಾಜನ ಜೀವನಕ್ರಮ, ಭೀಮಸೇನನು ದ್ರೌಪದಿಯನ್ನು ಧರ್ಮರಾಜನ ಬಳಿ ಕಳುಹಿಸಿ ಅಜ್ಞಾತವಾಸವಾದ ಮೇಲೆ ದುರ್ಯೊಧನನೊಡನೆ ಯುದ್ಧಮಾಡಿಯಾದರೂ ತನ್ನ ರಾಜ್ಯವನ್ನು ಪುನಃ ಪಡೆಯುವಂತೆ ವಾಗ್ದಾನ ಪಡೆಯಲು ಪ್ರಯತ್ನ ಪಡುವುದು, ದ್ರೌಪದಿ-ಧರ್ಮರಾಜರ ಸಂಭಾಷಣೆ, ಜೀವಕರ್ತೃತ್ವ ವಿಚಾರದಲ್ಲಿ ಚರ್ಚೆ ನಡೆದು ಧರ್ಮರಾಜನು ದ್ರೌಪದಿಯನ್ನು ಗದರಿಸುವುದು, ಭೀಮಸೇನನು ಬಂದು ಜೀವನ ಪ್ರಯತ್ನ, ಜೀವಕರ್ತೃತ್ವ ಮುಂತಾದ ವೇದಾಂತ ವಿಚಾರವನ್ನು ಧರ್ಮರಾಜನಿಗೆ ಉಪದೇಶ ಮಾಡಿ ಜೀವನ ಸ್ವಭಾವ ಮುಂತಾದ ನಿಮಿತ್ತಗಳನ್ನು ವಿವರಿಸುವುದು, ಶ‍್ರೀಹರಿಯ ಸ್ವಾತಂತ್ರ್ಯದ ಪ್ರಶಂಸೆ, ಚಾತುರ್ವಣ್ರಗಳ ಧರ್ಮಗಳು, ವಿಷ್ಣುವಿನಲ್ಲಿ ದೋಷರಾಹಿತ್ಯ ಗುಣಪೂರ್ಣತ್ವ ಸಮರ್ಥನೆ, ಧರ್ಮರಾಜನು ಹದಿಮೂರು ವರ್ಷಗಳ ನಂತರ ಯುದ್ಧ ಮಾಡಿ ರಾಜ್ಯವನ್ನು ಪಡೆಯಲು ಒಪ್ಪುವುದು, ವೇದವ್ಯಾಸರು ಧರ್ಮರಾಜನ ಮೂಲಕ ಅರ್ಜುನನಿಗೆ ಭೀಷ್ಮಾದಿಗಳನ್ನು ಗೆಲ್ಲಲು ಮಂತ್ರೋಪದೇಶ ಮಾಡುವುದು, ಅರ್ಜುನನು ಮಂತ್ರವನ್ನು ಪಡೆದು ಇಂದ್ರಕೀಲಕ ಪರ್ವತಕ್ಕೆ ತೆರಳಿ ರುದ್ರದೇವರನ್ನು ಕುರಿತು ತಪಸ್ಸನ್ನಾಚರಿಸಿ ಅವರಿಂದ ಪಾಶುಪತಾಸ್ತ್ರವನ್ನು ಪಡೆದುದು, ಅರ್ಜುನನು ಸ್ವರ್ಗಲೋಕದಲ್ಲಿ ಐದು ಸಂವತ್ಸರ ವಾಸಮಾಡುವುದು, ಊರ್ವಶಿಯಿಂದ ನಪುಂಸಕನಾಗೆಂದು ಅರ್ಜುನನಿಗೆ ಶಾಪಪ್ರಾಪ್ತಿ, ಅರ್ಜುನನು ಸ್ವರ್ಗಲೋಕದಲ್ಲಿ ಚಿತ್ರಸೇನನಿಂದ ಗಾಂಧರ್ವವೇದವನ್ನು ಪಡೆಯುವುದು, ಅಜ್ಞಜನರ ಮೋಹನಾರ್ಥವಾಗಿ ಶ‍್ರೀಕೃಷ್ಣನು ರುದ್ರದೇವರಿಗೆ ಪಾಶುಪತಾದಿ ಆಗಮಗಳನ್ನು ರಚಿಸುವಂತೆ ಆಜ್ಞೆ ಮಾಡುವುದು, ಶ‍್ರೀಕೃಷ್ಣನು ರುದ್ರದೇವರ ಪೂಜೆಗೋಸ್ಕರ ಬದರಿಕಾಶ್ರಮಕ್ಕೆ ಹೋಗುವುದು, ಅಲ್ಲಿ ಘಂಟಾಕರ್ಣ, ಕರ್ಣ ಎಂಬ ಇಬ್ಬರು ಪಿಶಾಚಿಗಳ ಶಾಪ ವಿಮೋಚನೆಯನ್ನು ಮಾಡುವುದು, ಅಲ್ಲಿಂದ ಕೃಷ್ಣನು ಕೈಲಾಸಕ್ಕೆ ತೆರಳಿ ರುದ್ರದೇವರನ್ನು ಕುರಿತು ತಪಸ್ಸು ಮಾಡುವಂತೆ ನಟಿಸುವುದು, ಗ್ರಹಗಳಿಗೆ ಆಜ್ಞೆ ಮಾಡಿ ಒಂದು ದಿನವನ್ನು ಹನ್ನೆರಡು ಸಂವತ್ಸರಗಳಿಗೆ ಸಮನಾಗಿ ಮಾಡುವುದು, ರುದ್ರದೇವರಿಂದ ಪುತ್ರನಿಗಾಗಿ ವರವನ್ನು ಪಡೆದಂತೆ ತೋರಿಸುವುದು, ರುದ್ರದೇವರು ಶ‍್ರೀ ಕೃಷ್ಣನನ್ನು ವಿಶೇಷವಾಗಿ ಸ್ತೋತ್ರ ಮಾಡುವುದು, ಪೌಂಡ್ರಕ ವಾಸುದೇವನು ದ್ವಾರಕಾನಗರದ ಮೇಲೆ ಹಲ್ಲೆ ಮಾಡುವುದು,\break ಬಲರಾಮ ಸಾತ್ಯಕಿಯರು ಅವನೊಡನೆ ಯುದ್ದ ಮಾಡುವುದು, ಬಲರಾಮನಿಂದ ಏಕವಲ್ಯನಿಗೆ ಪರಾಜಯ, ಶ‍್ರೀ ಕೃಷ್ಣನು ಪೌಂಡ್ರಕವಾಸುದೇವನನ್ನು ಸಂಹರಿಸುವುದು, ಅವನಿಗೆ ಸಹಾಯಮಾಡಿದ ಕಾಶಿರಾಜನನ್ನೂ ಸಂಹರಿಸಿ ಅಂಧಂತಮಸ್ಸಿಗೆ ಕಳಿಸುವುದು, ಪ್ರದ್ಯುಮ್ನನ ಜನನ, ಶ‍್ರೀಕೃಷ್ಣನೊಡನೆ ಪುನಃ ಯುದ್ಧಕ್ಕೆ ಬಂದ ಏಕಲವ್ಯನು ಮೃತನಾಗುವುದು, ಶ‍್ರೀಕೃಷ್ಣನ ಚಕ್ರದಿಂದ ಸುದಕ್ಷಿಣನ ಮರಣ, ಶ‍್ರೀಕೃಷ್ಣ-ರುಕ್ಷ್ಮಿಣಿಯರ ಸರಸ ಸಲ್ಲಾಪ, ಯಮುನೆಯು ಬಲರಾಮನನ್ನು ಸ್ತೋತ್ರಮಾಡುವುದು, ಬಲರಾಮನ ಸಾಮರ್ಥ್ಯ ವಿವರಣೆ, ಸಾಂಬ-ಲಕ್ಷಣಾ ಇವರ ವಿವಾಹ, ಬಾಣಾಸುರನ ಮಗಳಾದ ಉಷಾ ಶ‍್ರೀಕೃಷ್ಣನ ಪೌತ್ರನಾದ ಅನಿರುದ್ಧನನ್ನು ತನ್ನ ಸಖಿಯ ಮೂಲಕ ಕರೆಸಿಕೊಂಡು ಕ್ರೀಡಿಸುವುದು, ಬಾಣಾಸುರನು ಅನಿರುದ್ಧನನ್ನು ನಾಗಾಸ್ತ್ರದಿಂದ ಕಟ್ಟುವುದು, ರುದ್ರ-ಕೃಷ್ಣರ ಯುದ್ದ, ರುದ್ರನ ಪರಾಭವ, ಕೃಷ್ಣನು ಅನಿರುದ್ಧನನ್ನು ಬಿಡಿಸಿಕೊಂಡು ದ್ವಾರಕಿಗೆ ಬರುವುದು, ಲೋಮಶಮುನಿಯ ಆದೇಶದಂತೆ ಪಾಂಡವರು ಭಾರತಭೂಮಿಯಲ್ಲಿ ತೀರ್ಥಯಾತ್ರೆಯನ್ನು ಮಾಡಿದುದು, ಹಿಮವತ್ಪರ್ವತದಲ್ಲಿ ಪಾಂಡವರು ಘಟೋತ್ಕಚನ ಸಹಾಯದಿಂದ ಮಾಡಿದ ವ್ಯಾಪಾರಗಳು, ದ್ರೌಪದೀದೇವಿಯಿಂದ ಶ್ರೇಷ್ಠವಾದ ಪರಿಮಳಯುಕ್ತವಾದ ಕಮಲಪುಷ್ಪಗಳ ಬೇಡಿಕೆ, ಭೀಮಸೇನನು ತರಲು ಹೋಗುವ ಮಾರ್ಗದಲ್ಲಿ ತನ್ನದೇ ಆದ ಹನುಮಂತ ರೂಪವನ್ನು ನೋಡಿ ಕ್ರೀಡಿಸುವುದು, ಪುಪ್ಪಗಳನ್ನುಳ್ಳ ಸರೋವರದ ಬಳಿ ಕ್ರೋಧವಶರೆಂಬ ರಾಕ್ಷಸರು ಭೀಮಸೇನನನ್ನು ತಡೆಯುವುದು, ಭೀಮಸೇನನು ಅವರನ್ನು ಸಂಹರಿಸಿ, ವಿಷ್ಣು ಸರ್ವೋತ್ತಮತ್ವಾದಿ ಪ್ರಮೇಯಗಳನ್ನು ಸ್ಥಾಪಿಸಿ ಉಳಿದ ರಾಕ್ಷಸರನ್ನು ಓಡಿಸುವುದು, ಪುಷ್ಪಗ್ರಹಣ, ಬಹಳ ಹೊತ್ತಾದರೂ ತಿರುಗಿ ಭೀಮಸೇನನು ಬಾರದ ಕಾರಣ ಧರ್ಮರಾಜನು ದ್ರೌಪದೀ ಮೊದಲಾದವರಿಂದ ಸಹಿತನಾಗಿ ಸರೋವರದ ಬಳಿ ಬರುವುದು, ಭೀಮಸೇನನಿಗೆ ಮುಂದೆ ಹೋಗಬೇಡವೆಂದು ಹೇಳುವುದು, ದ್ರೌಪದಿಯ ಇಚ್ಛೆಯಂತೆ ಭೀಮಸೇನನು ಕುಬೇರನಿಂದ ಅಧಿಷ್ಠಿತವಾದ ಗಂಧಮಾದನ\break ಪರ್ವತಕ್ಕೆ ಪಂಚವರ್ಣಗಳ ಪುಷ್ಪಗಳನ್ನು ತರಲು ತೆರಳುವುದು, ಅಲ್ಲಿ ಅಡ್ಡಿ ಬಂದ ಮಣಿ\-ಮಂತನೇ ಮೊದಲಾದ ಅನೇಕ ರಾಕ್ಷಸರನ್ನು ಸಂಹರಿಸಿದುದು, ಮಣಿಮಂತನು ಕಲಿಯುಗದಲ್ಲಿ ಪುನಹ ಅವತರಿಸುವ ಸೂಚನೆ, ಕುಬೇರನು ಭೀಮಸೇನನ ಸಾಮರ್ಥ್ಯವನ್ನು ನೋಡಿ ಭಯಪಟ್ಟು ಧರ್ಮರಾಜಾದಿಗಳಿಗೆ ತನ್ನ ಅರಮನೆಯನ್ನೇ ಬಿಟ್ಟುಕೊಟ್ಟು ಅವರನ್ನು ಉಪಚರಿಸುವುದು, ನಾಲ್ಕು ವರ್ಷಗಳ ಕಾಲ ಧರ್ಮರಾಜಾದಿಗಳು ಅಲ್ಲಿ ವಾಸಿಸುವುದು, ಇಂದ್ರನು ಅರ್ಜುನನಿಂದ “ನಿವಾತಕವಚ" ರೆಂಬ ರಾಕ್ಷಸರನ್ನು ಸಂಹರಿಸುವುದೆಂಬ ಗುರುದಕ್ಷಿಣೆಯನ್ನು ಕೇಳುವುದು, ಅರ್ಜುನನು ಗಾಂಡೀವವನ್ನೂ, ದೇವತೆಗಳಿಂದ ಕೊಡಲ್ಪಟ್ಟ ಶಂಖವನ್ನೂ ತೆಗೆದುಕೊಂಡು ನಿವಾತಕವಚರನ್ನೂ ನಂತರ ಪೌಲೋಮಗಣ, ಕಾಲೇಯಗಣರೆಂಬ ರಾಕ್ಷಸರನ್ನು ಸಂಹರಿಸಿ ಇಂದ್ರನನ್ನು ಸಂತೋಷಪಡಿಸುವುದು, ಅರ್ಜುನನು ಕುಬೇರನ ಬಳಿ ಇದ್ದ ತನ್ನ ಸಹೋದರರೊಡನೆ ಬಂದು ಸೇರುವುದು, ಭೀಮಸೇನನು ಅಜಗರ ರೂಪದಲ್ಲಿದ್ದ ನಹುಷನನ್ನು ಭೇಟಿ ಮಾಡುವುದು, ನಹುಷನಿಗೆ ಶಾಪ ಬಂದ ಕಾರಣ, ನಹುಷನ\break ಪ್ರಶ್ನೆಗಳಿಗೆ ಭೀಮಸೇನನು ಶಕ್ತನಾಗಿದ್ದರೂ ಉತ್ತರ ಕೊಡದೆ ನಹುಷನ ವಶನಾದಂತೆ ನಟನೆ ಮಾಡಿದುದು, ನಹುಷನ ತಪಸ್ಸಿನ ಬಲವೆಲ್ಲವೂ ಭೀಮಸೇನನಲ್ಲಿ ಬರುವುದು, ಧರ್ಮರಾಜನು ಬಂದು ನಹುಷನ ಪ್ರಶ್ನೆ ಗಳಿಗೆ ಉತ್ತರ ಹೇಳುವುದು, ನಹುಷನ ಶಾಪವಿಮೋಚನೆ, ಪಾಂಡವರು ದೈತವನಕ್ಕೆ ಆಗಮಿಸುವುದು, ಶ‍್ರೀಕೃಷ್ಣನು ಸತ್ಯಭಾಮಾದೇವಿಯಿಂದ ಯುಕ್ತನಾಗಿ ಪಾಂಡವರ ಬಳಿ ಬರುವುದು, ದ್ರೌಪದಿಯು ಸ್ತ್ರೀ ಧರ್ಮವನ್ನು ನಿರೂಪಿಸುವುದು, ಜಯದ್ರಥನು ದ್ರೌಪದೀದೇವಿಯರನ್ನು ಕ್ರೀಡೆಗಾಗಿ ಬಲಾತ್ಕರಿಸಿ ಸೈಂಧವನ ರಥವನ್ನು ಏರುವುದು, ಪಾಂಡವರು ರಥವನ್ನು ಹಿಂಬಾಲಿಸಿ ಅವರ ಸೈನ್ಯವನ್ನು ನಾಶಮಾಡಿ, ಜಯದ್ರಥನನ್ನು ಕಟ್ಟಿ ದ್ರೌಪದಿಯ ಪಾದಗಳ ಮೇಲೆ ಕೆಡವುವುದು, ಜಯದ್ರಥನನ್ನು ಕ್ಷಮಿಸಿ ಕಳಿಸುವುದು, ಮಾರ್ಕಂಡೇಯ ಋಷಿಗಳ ಆಗಮನ ಹಾಗೂ ಅವರಿಂದ ರಾಮಾಯಣ ಕಥಾ ನಿರೂಪಣೆ; ಗುಹ್ಯ, ದರ್ಶನ, ಸಮಾಧಿ ಭಾಷೆಗಳ ಪರಿಚಯ; ದುರ್ಯೋಧನನು “ಪೌಂಡರೀಕ” ಎಂಬ ಯಾಗವನ್ನು ಮಾಡಿದುದು, ತಮ್ಮ ವೈಭವವನ್ನು ತೋರಿಸಲು ಕರ್ಣ, ದುರ್ಯೊಧನಾದಿಗಳು ದೈತವನಕ್ಕೆ ಪ್ರಾಪ್ತರಾಗಿ ತಮ್ಮ ಸಂಪತ್ತನ್ನು ಪಾಂಡವರ ಮುಂದೆ ಪ್ರದರ್ಶಿಸುವುದು, ಇಂದ್ರನಿಂದ ಕಳಿಸಲ್ಪಟ್ಟ ಚಿತ್ರಸೇನನೆಂಬ ಗಂಧರ್ವನು ಕರ್ಣನನ್ನು ಸೋಲಿಸಿ ದುರ್ಯೋಧನನನ್ನು ಕಟ್ಟಿ ಹಾಕುವುದು, ಕೌರವರ ಮಂತ್ರಿಗಳು ಪಾಂಡವರ ಬಳಿ ಮೊರೆ ಇಡುವುದು, `ಭೀಮಾರ್ಜುನರು ದುರ್ಯೋಧನನನ್ನು ಸೆರೆಯಿಂದ ಬಿಡಿಸುವುದು, ಧುರ್ಯೋಧನನಿಗೆ ಆದ ನಾಚಿಕೆ, ಶುಕ್ರಾಚಾರ್ಯರಿಂದ ನಿರ್ಮಿತವಾದ ಭೂತವಿಶೇಷವು ದುರ್ಯೋಧನನನ್ನು ರಾತ್ರಿಯಲ್ಲಿ ಪಾತಾಳಲೋಕಕ್ಕೆ ಕರೆದುಕೊಂಡು ಹೋಗಿ ಪಾಂಡವರ ಮೇಲೆ ಯುದ್ಧ ಮಾಡಲು ಹುರಿದುಂಬಿಸುವುದು, ಮಾರನೇ ಬೆಳಿಗ್ಗೆ ಕರ್ಣ-ದುಶ್ಯಾಸನರಿಂದ ಯುಕ್ತನಾಗಿ ದುರ್ಯೊಧನನು ಹಸ್ತಿನಾವತೀ ನಗರಕ್ಕೆ ಹೋಗುವುದು, ಇಂದ್ರನು ಕರ್ಣನ ಸಹಜವಾದ ಕರ್ಣಕುಂಡಲಗಳನ್ನು ಪಡೆಯುವ ಇಚ್ಛೆ, ಸೂರ್ಯನಿಂದ ಸ್ವಪ್ನದಲ್ಲಿ ಕರ್ಣನಿಗೆ ಕೊಡ ಬೇಡವೆಂಬ ಎಚ್ಚರಿಕೆ, ಬ್ರಾಹ್ಮಣ ವೇಶದಲ್ಲಿ ಬಂದ ಇಂದ್ರನಿಗೆ ಕರ್ಣನು ಕರ್ಣಕುಂಡಲಗಳನ್ನು ಕೊಡುವುದು, ಭೇಟಿಯ ಅವಸರದಲ್ಲಿ ಸರೋವರದ ನೀರನ್ನು ಪಾನಮಾಡಿ ಭೀಮಾದಿಗಳು ಬೀಳುವುದು, ಸಮರ್ಥರಾದರೂ ಭೀಮಾರ್ಜುನರು ಯಕ್ಷನಿಂದ ಕೇಳಲ್ಪಟ್ಟ ಪ್ರಶ್ನೆಗಳಿಗೆ ಉತ್ತರ ಹೇಳದಿರಲು ಕಾರಣ, ಧರ್ಮರಾಜನು ಉತ್ತರ ನೀಡುವುದು, ಸಂತುಷ್ಟನಾದ ಯಕ್ಷನು ಭೀಮಾದಿಗಳು ಏಳುವಂತೆ ಮಾಡಿದುದು, ಪಾಂಡವರು ಈ ರೀತಿ ಸಂತೋಷದಿಂದಲೂ, ಶ‍್ರೀಕೃಷ್ಣನ ನಿತ್ಯ ಸ್ಮರಣೆಯಿಂದಲೂ ಹನ್ನೆರಡು ವರ್ಷಗಳ ಕಾಲ ಅರಣ್ಯವಾಸವನ್ನು ಮುಗಿಸಿದ ವಿಚಾರಗಳು ಈ ಅಧ್ಯಾಯದಲ್ಲಿ ನಿರೂಪಿತವಾಗಿವೆ.


\section*{ಅಧ್ಯಾಯ\enginline{-}೨೩}

\begin{verse}
\textbf{ಅನ್ಯಂ ವೇಷಮುಪಾಗತಾಃ ಪೃಥಗಿತೋ ಗತ್ವಾ ವಿರಾಟಾಲಯಂ}\\\textbf{ತದ್ದೇಹಸ್ಥ ಹರೇರ್ನಿಷೇವಣಪರಾಮಲ್ಲಂ ತಥಾ ಕೀಚಕಾನ್~।}\\\textbf{ಹತ್ವಾ ಗೋಗ್ರಹಣೋದ್ಯ ತಾನಪಿ ಕುರೂನ್ ಜಿತ್ವಾ ವಿರಾಟಾರ್ಚಿತಾಃ }\\\textbf{ಪಾರ್ಥಾಃ ಸ್ವಾಂತಿಕಮಾಗತಂ ಯಮಜಿತಂ ಭೇಜುಸ್ತಮೀಡೇಽಚ್ಯುತಮ್~।।}
\end{verse}

ಅಜ್ಞಾತವಾಸಕ್ಕಾಗಿ ಪಾಂಡವರು ವೇಷಾಂತರದಿಂದ ವಿರಾಟರಾಜನ ಮನೆಯಲ್ಲಿರುತ್ತಾ ಅವನ ಅಂತರ್ಯಾಮಿಯಾದ ಶ‍್ರೀಹರಿಯನ್ನು ಸೇವೆ ಮಾಡಿ, ವಿರಾಟನ ಅಸ್ಥಾನಕ್ಕೆ ಬಂದಿದ್ದ ಜಟ್ಟಿಯನ್ನೂ ಕೀಚಕನನ್ನೂ ಸಂಹರಿಸಿ, ಕೌರವರು ವಿರಾಟನ ಗೋವುಗಳನ್ನು ಅಪಹರಿಸಿದಾಗ ಅವರನ್ನು ಸೋಲಿಸಿ ಗೋವುಗಳನ್ನು ಬಿಡಿಸಿ, ವಿರಾಟನ ಪೂಜೆಯನ್ನು ಸ್ವೀಕರಿಸಿ, ತಮ್ಮ ಬಳಿಗೆ ಬಂದ ಯಾವ ಶ‍್ರೀಕೃಷ್ಣನನ್ನು ಪೂಜಿಸಿದರೋ ಅಂತಹ ಅಚ್ಯುತನನ್ನು ಸ್ತುತಿಸುತ್ತೇನೆ.

ಈ ಅಧ್ಯಾಯದಲ್ಲಿ ೬೦ ಶ್ಲೋಕಗಳಿವೆ.

ಪಾಂಡವರು ಹನ್ನೆರಡು ವರ್ಷ ವನವಾಸವನ್ನು ಮುಗಿಸಿ ಒಂದು ವರ್ಷದ ಅಜ್ಞಾತವಾಸಕ್ಕಾಗಿ ವಿರಾಟರಾಜನ ಅರಮನೆಯಲ್ಲಿ ವೇಷಾಂತರದಿಂದ ಇದ್ದುದು, (ಧರ್ಮರಾಜನು ಯತಿವೇಷದಲ್ಲಿ, ಭೀಮಸೇನನು ಪಾಚಕನಾಗಿ, ಅರ್ಜುನನು ಷಂಡನಾಗಿ ನಾಟ್ಯಗಾಯನ ಕಲಿಸುವವನಾಗಿ, ನಕುಲನು ರಥಿಕನಾಗಿ, ಸಹದೇವನು ಗೋರಕ್ಷಕನಾಗಿ, ದ್ರೌಪದಿಯು ಗಂಧಲೇಪನಾದಿಗಳನ್ನು ಮಾಡುವವಳಾಗಿ) ತಮ್ಮ ಶಸ್ತ್ರಾಸ್ತ್ರಗಳನ್ನು ಶಮೀ ವೃಕ್ಷದಲ್ಲಿ ಇಟ್ಟ ವಿಷಯ, ಪಾಂಡವರು ಈ ವೇಷಗಳನ್ನು ಧರಿಸುವುದರ ಔಚಿತ್ಯ, ವಿರಾಟನ ಅರಮನೆಗೆ ಬಂದ ಒಬ್ಬ ಹೊಸಮಲ್ಲನನ್ನೂ ಅವನ ಸಾಮರ್ಥವನ್ನೂ ಕಂಡು ಆಸ್ಥಾನದ ಮಲ್ಲರೆಲ್ಲರೂ ಓಡಿ\-ಹೋದುದು, ವಿರಾಟನ ಚಿಂತೆ, ಯತಿಯ ವೇಷದ ಯುಧಿಷ್ಠಿರನು ಸಲಹೆ ಮಾಡಿದಂತೆ ಪಾಚಕನನ್ನು ಮಲ್ಲನ ಜತೆಯಲ್ಲಿ ಯುದ್ಧ ಮಾಡಲು ಹೇಳುವುದು, ಭೀಮಸೇನನಿಂದ ಮಲ್ಲನ ಸಂಹಾರ, ಹೀಗೆ ಹತ್ತು ತಿಂಗಳು ಕಳೆಯಲು ವಿರಾಟನ ಪತ್ನಿಯ ಸಹೋದರನಾದ ಕೀಚಕನು ದ್ರೌಪದಿಯನ್ನು ಕ್ರೀಡೆಗೋಸ್ಕರ ಆಹ್ವಾನಿಸುವುದು, ದ್ರೌಪದಿಯ ತಿರಸ್ಕಾರ, ಕೀಚಕನು ತನ್ನ ತಂಗಿಯಾದ ಸುದೇಷ್ಣೆಯನ್ನು ಬೇಡುವುದು, ಸುದೇಷ್ಣೆಯ ಆಜ್ಞೆಯಂತೆ ದ್ರೌಪದಿಯು ಕೀಚಕನ ಮನೆಯಿಂದ ಮದ್ಯವನ್ನು ತರಲು ಹೋದಾಗ ಕೀಚಕನು ಅವಳನ್ನು ಹಿಡಿಯುವುದು, ದ್ರೌಪದಿಯು ಬಿಡಿಸಿಕೊಂಡು ಓಡಿ ಹೋಗುವುದು, ದ್ರೌಪದಿಯು ಈ ವಿಚಾರವನ್ನು ಭೀಮಸೇನನಿಗೆ ತಿಳಿಸುವುದು, ಭೀಮಸೇನನು ರಾತ್ರಿ ಶೂನ್ಯಗೃಹದಲ್ಲಿ ಕೀಚಕನನ್ನು ಹೊಂದಿ ಬಾಹು ಯುದ್ಧದಲ್ಲಿ ಅವನನ್ನು ಸಂಹರಿಸಿದುದು, ನೂರಐದುಮಂದಿ ಉಪಕೀಚಕರನ್ನು ಸಂಹಾರ ಮಾಡಿದುದು, ಸುದೇಷ್ಣೆಯ ಭಯ, ದ್ರೌಪದಿಯಿಂದ ಹದಿಮೂರು ದಿವಸಗಳು ಮಾತ್ರ ರಕ್ಷಿಸು ಎಂಬ ಪ್ರಾರ್ಥನೆ, ಒಂದು ಸಂವತ್ಸರ ಈ ರೀತಿ ಕಳೆದಮೇಲೆ ದುರ್ಯೊಧನಾದಿಗಳು ವಿರಾಟನ ಮೇಲೆ ಹಲ್ಲೆಮಾಡಲು ಬರುವುದು, ಸುಶರ್ಮನು ವಿರಾಟನ ಗೋವುಗಳನ್ನು ಅಪಹರಿಸುವುದು, ಸುಶರ್ಮನ ಬಂಧನ, ಭೀಮಸೇನನು ಸುಶರ್ಮನ ಸೇನೆಯನ್ನು ಸಂಹರಿಸಿ ವಿರಾಟನನ್ನು ಬಿಡಿಸುವುದು, ಭೀಷ್ಮ ದ್ರೋಣ ಅಶ್ವತ್ಥಾಮಾದಿಗಳು ಪುನಃ ಬಂದು ವಿರಾಟನ ಗೋವುಗಳನ್ನು ಅಪಹರಿಸುವುದು, ಉತ್ತರಕುಮಾರನು ಯುದ್ಧಕ್ಕೆ ಹೋದಾಗ ಅರ್ಜುನನು ಸಾರಥಿಯಾಗುವುದು, ಯುದ್ಧ ಭೂಮಿಯಲ್ಲಿ ಉತ್ತರನ ಭಯ, ಶಮೀವೃಕ್ಷದಲ್ಲಿದ್ದ ಗಾಂಡೀವಾದಿ ಶಸ್ತ್ರಾಸ್ತ್ರಗಳನ್ನು ಪಡೆದು ಅರ್ಜುನನು ಯುದ್ಧ ಮಾಡಿ ಸಮ್ಮೋಹನಾಸ್ತ್ರದಿಂದ ಕೌರವರನ್ನು ಮೂರ್ಛೆಗೊಳಿಸಿ ಸೋಲಿಸಿದುದು, ವಿರಾಟನು ಯುಧಿಷ್ಠಿರನನ್ನು ದ್ಯೂತಪಾಶದಿಂದ ಹೊಡೆದುದು, ಮಾರನೇ ದಿವಸ ಪಾಂಡವರು ತಮ್ಮ ನಿಜರೂಪದಿಂದ ಪ್ರಕಟವಾದುದು, ಕೃಷ್ಣ ಬಲರಾಮರ ಆಗಮನ, ಅಭಿಮನ್ಯು -ಉತ್ತರೆಯರ ವಿವಾಹ, ದುರ್ಯೋಧನನು ಒಂದು ಸಂವತ್ಸರಕ್ಕೆ ಮುಂಚೆಯೇ ಪ್ರಕಟವಾದುದರಿಂದ ಪಾಂಡವರು ಮತ್ತೆ ಹನ್ನೆರಡು ವರ್ಷ ವನವಾಸಕ್ಕೆ ಹೋಗಬೇಕೆಂದು ದೂತನಿಂದ ಹೇಳಿಕಳಿಸುವುದು, ಚಾಂದ್ರಮಾನ ರೀತಿಯಿಂದ ಒಂದು ವರ್ಷ ಅಜ್ಞಾತವಾಸವು ಪೂರ್ಣವಾಗಿದೆ ಎಂದು ನಿರ್ಣಯಿಸುವುದು, ಶ‍್ರೀಕೃಷ್ಣನಿಂದ ಸಹಿತರಾದ ಪಾಂಡವರು ಉಪಪ್ಲಾವ್ಯ ನಗರದಲ್ಲಿ ಕೆಲವು ದಿವಸಗಳು ಆವಾಸಮಾಡುವುದು, ಈ ವಿಷಯಗಳು ಈ ಅಧ್ಯಾಯದಲ್ಲಿ ವರ್ಣಿಸಲ್ಪಟ್ಟಿವೆ.


\section*{ಅಧ್ಯಾಯ\enginline{-}೨೪}

\begin{verse}
\textbf{ಯತ್ಸಮ್ಮತ್ಯಾ ಪೃಷತತನುಜಪ್ರೇಷಿತಬ್ರಾಹ್ಮಣೋಕ್ತ್ಯಾ }\\\textbf{ರಾಜ್ಯಂ ನಾದಾದನುಜಜನಿತಸ್ಯಾಂಬಿಕೇಯೋsರ್ಜುನಸ್ಯ~।}\\\textbf{ಯಃ ಸಾಹಾಯ್ಯಂ ವ್ಯಧಿತನಗರೀಂ ಕೌರವಾಣಾಮವಾಪ್ತಃ }\\\textbf{ಸ್ವೋಕ್ತ್ಯೈ ಕೃಷ್ಣಸ್ತದನಭಿಮತೇಽವಾಪ್ತಪಾರ್ಥಃ ಸ ನೋಽವ್ಯಾತ್~।।}
\end{verse}

ಯಾವ ಶ‍್ರೀಕೃಷ್ಣನ ಸಮ್ಮತಿಯಿಂದ ದ್ರುಪದರಾಜನ ಮಗನಿಂದ ಕಳಿಸಲ್ಪಟ್ಟ ಬ್ರಾಹ್ಮಣನು ಧೃತರಾಷ್ಟ್ರನ ಬಳಿ ಹೋಗಿ ಪಾಂಡವರಿಗಾಗಿ ರಾಜ್ಯವನ್ನು ಕೇಳಿ ಧೃತರಾಷ್ಟ್ರ ನಿಂದ ತಿರಸ್ಕೃತನಾದನೋ, ನಂತರ ಯಾವ ಶ‍್ರೀಕೃಷ್ಣನು ಸ್ವಯಂ ಕೌರವರ ಬಳಿ ಸಂಧಾನಕ್ಕಾಗಿ ಹೋದಾಗ ಮಾತುಕತೆಗಳು ಫಲಿಸಲಿಲ್ಲವೋ, ಮತ್ತು ಯಾವ ಶ‍್ರೀಕೃಷ್ಣನು ಅರ್ಜುನನಿಗೆ ಸಹಾಯಮಾಡಲು ಪಾಂಡವರ ಬಳಿ ಪುನಃ ಬಂದನೋ ಅಂತಹ ಶ‍್ರೀಕೃಷ್ಣನು ನಮ್ಮನ್ನು ರಕ್ಷಿಸಲಿ.

ಈ ಅಧ್ಯಾಯದಲ್ಲಿ ೮೯ ಶ್ಲೋಕಗಳಿವೆ.

ಶ‍್ರೀಕೃಷ್ಣನ ಅನುಮತಿಯಿಂದ ದ್ರುಪದರಾಜನು ತನ್ನ ಪುರೋಹಿತನನ್ನು ಧೃತರಾಷ್ಟ್ರನ ಬಳಿ ಕೌರವಪಾಂಡವರ ವಿರೋಧಪರಿಹಾರಕ್ಕಾಗಿ ಕಳಿಸುವುದು, ಪುರೋಹಿತನು ಭೀಷ್ಮದ್ರೋಣರ ಸಹಿತನಾಗಿ ಇರುವ ಧೃತರಾಷ್ಟ್ರನಿಗೆ ಪಾಂಡವರ ಶಕ್ತಿ-ಸಾಮರ್ಥ್ಯಗಳನ್ನು ವಿವರಿಸುವುದು, ಭೀಮಸೇನನಿಂದ ಜಟಾಸುರನ ವಧೆ, ಅರ್ಜುನನು ನಿವಾತಕವಚಾದಿಗಳನ್ನು ಸಂಹರಿಸಿದ ವಿಚಾರವನ್ನು ಒತ್ತಿ ಹೇಳುವುದು, ಪಾಂಡವರಿಗೆ ಶ‍್ರೀಕೃಷ್ಣನು ಸಹಾಯಕನಾಗಿರುವುದು ಈ ವಿಷಯಗಳನ್ನು ತಿಳಿಸಿದರೂ ಸಹ ಧೃತರಾಷ್ಟ್ರನು ಪಾಂಡವರಿಗೆ ಅವರ ರಾಜ್ಯವನ್ನು ಹಿಂತಿರುಗಿಸಲು ಒಪ್ಪದೇ ಇದ್ದುದು, ಶ‍್ರೀಕೃಷ್ಣನು ದ್ವಾರಕೆಗೆ ತೆರಳುವುದು, ಅರ್ಜುನ-ದುರ್ಯೊಧನರು ಮುಂದೆ ನಡೆಯಬಹುದಾದ ಯುದ್ಧದಲ್ಲಿ ಶ‍್ರೀಕೃಷ್ಣನ ಸಹಾಯವನ್ನು ಬೇಡಲು ದ್ವಾರಕೆಗೆ ಹೋಗುವುದು, ತನ್ನ ಶಿರಸ್ಸಿನ ಬಳಿ ದುರ್ಯೋಧನನನ್ನೂ ಪಾದದ ಬಳಿ ಅರ್ಜುನನನ್ನೂ ಕಂಡ ಶ‍್ರೀಕೃಷ್ಣನು ಕುಶಲ ಪ್ರಶ್ನೆಯನ್ನು ಮಾಡಿದ ನಂತರ, ಶ‍್ರೀಕೃಷ್ಣನಿಂದ ಹತ್ತು ಲಕ್ಷಜನ ತನ್ನ ಮಕ್ಕಳ ಸಹಾಯವನ್ನು ದುರ್ಯೊಧನನು ಪಡೆಯುವುದು, ಅರ್ಜುನನು ಶ‍್ರೀಕೃಷ್ಣನೊಬ್ಬನೇ ತನ್ನ ಕಡೆಯಲ್ಲಿ ಇದ್ದರೆ ಸಾಕೆಂದು ಹೇಳುವುದು, ಶ‍್ರೀಕೃಷ್ಣನು ಪಾಂಡವರ ಬಳಿ ಬರುವುದು, ಉಪಪ್ಲಾವ್ಯ ನಗರದಲ್ಲಿ ನಡೆದ ಸಭೆಯ ವಿಚಾರ, ದುರ್ಯೋಧನನಿಗೆ ಹನ್ನೊಂದು ಅಕ್ಷೌಹಿಣೀ ಸೈನ್ಯದ ಜಮಾವಣೆ, ಪಾಂಡವರಿಗೆ ಏಳು ಅಕ್ಷೌಹಿಣೀ ಸೈನ್ಯದ ಜಮಾವಣೆ, ಎರಡು ಕಡೆಯಲ್ಲಿರುವ ವೀರರ ವಿಚಾರ, ದುರ್ಯೋಧನನು ಕಪಟದಿಂದ ಶಲ್ಯನನ್ನು ತನ್ನ ಕಡೆಗೆ ಸೇರಿಸಿಕೊಳ್ಳುವುದು, ಧೃತರಾಷ್ಟ್ರನು ಸಂಧಾನಕ್ಕಾಗಿ ಪಾಂಡವರ ಬಳಿ ಸಂಜಯನನ್ನು ಕಳಿಸುವುದು, ವಿದುರನು ಧೃತರಾಷ್ಟ್ರನಿಗೆ ಬುದ್ಧಿವಾದ ಹೇಳುವುದು, ದುರ್ಯೊಧನಾದಿಗಳ ಸಂಹಾರವನ್ನು ಧರ್ಮವೆಂದು ಪ್ರಸಿದ್ಧ ಪಡಿಸಲು ಶ‍್ರೀಕೃಷ್ಣನು ಭೀಮಸೇನನನ್ನು ಹುರಿದುಂಬಿಸುವುದು, ಶ‍್ರೀಕೃಷ್ಣನು ಮನುಷ್ಯನಂತೆ ನಟನೆ ಮಾಡುತ್ತಾ ನಕುಲನಿಂದ ಶಿಕ್ಷಿತನಾದಂತೆ ತೋರಿಸುವುದು, ಇತರ ಪಾಂಡವರ ಮನಸ್ಸನ್ನೂ ತಿಳಿದುಕೊಂಡು ಶ‍್ರೀಕೃಷ್ಣನು ಪರಶುರಾಮ-ವೇದವ್ಯಾಸರೂಪಗಳಿಂದ ಹಿಂಬಾಲಿಸಿಕೊಳ್ಳಲ್ಪಟ್ಟವನಾಗಿ ಹಸ್ತಿನಾಪುರಕ್ಕೆ ತೆರಳುವುದು, ಭೀಷ್ಮಾಚಾರ್ಯರು ಶ‍್ರೀಕೃಷ್ಣನಿಗೆ ಸನ್ಮಾನಮಾಡುವುದು, ದುರ್ಯೋಧನನ ಪೂಜೆಯನ್ನು ಸ್ವೀಕರಿಸದೆ ಶ‍್ರೀಕೃಷ್ಣನು ವಿದುರನ ಆತಿಥ್ಯವನ್ನು ಸ್ವೀಕರಿಸಿದುದು, ಸಭೆಯಲ್ಲಿ ಶ‍್ರೀಕೃಷ್ಣನು ಧೃತರಾಷ್ಟ್ರನಿಗೆ ಬುದ್ಧಿ ಹೇಳಿ ರಾಜ್ಯವನ್ನು ಪಾಂಡವರಿಗೆ ಕೊಡುವುದು ಶ್ರೇಯಸ್ಕರವೆಂದೂ ಕೊಡದಿದ್ದರೆ ವಿನಾಶವು ನಿಶ್ಚಯವೆಂದೂ ತಿಳಿಸುವುದು, ಶ‍್ರೀಕೃಷ್ಣನನ್ನು ಕಟ್ಟಬೇಕೆಂಬ ದುರ್ಯೊಧನನ ಕುತಂತ್ರ, ಶ‍್ರೀಕೃಷ್ಣನು ತನ್ನ ಸ್ವರೂಪವನ್ನು ಧೃತರಾಷ್ಟ್ರನಿಗೆ ಅವನ ಯೋಗ್ಯತೆಗೆ ತಕ್ಕಂತೆ ತೋರಿಸುವುದು, ಇದಕ್ಕಾಗಿ ಧೃತರಾಷ್ಟ್ರನಿಗೆ ದಿವ್ಯ ದೃಷ್ಟಿಯನ್ನು ನೀಡುವುದು, ದುರ್ಯೊಧನಾದಿಗಳು ಆ ರೂಪವನ್ನು ನೋಡಲಾರದೆ ಕಣ್ಣು ಮುಚ್ಚಿಕೊಳ್ಳುವುದು, ಶ‍್ರೀಕೃಷ್ಣನು ಧೃತರಾಷ್ಟ್ರನನ್ನು ಪುನಃ ಅಂಧನನ್ನಾಗಿ ಮಾಡಿ ದಿವ್ಯರೂಪವನ್ನು ತೊರೆದು ಸಾಮಾನ್ಯರೂಪದಿಂದ ಸಭೆಯಿಂದ ಹೊರಟು, ಕುಂತಿಯನ್ನು ನೋಡಿ ಕರ್ಣನಿಗೆ “ನೀನು ಕುಂತಿಯ ಮಗನು” ಎಂಬ ವಿಷಯವನ್ನು ತಿಳಿಸುವುದು, ನಂತರ ಶ‍್ರೀಕೃಷ್ಣನು ಅಶ್ವತ್ಥಾಮಾಚಾರ್ಯರನ್ನು ಕಂಡು ಮಾತನಾಡಿಸುವುದು, ಕುಂತೀ-ಕರ್ಣರ ಭೇಟಿ, ಕರ್ಣನು ಕುಂತೀದೇವಿಗೆ ಕೊಟ್ಟ ವಾಗ್ದಾನ, ನಂತರ ಕೌರವರೂ ಪಾಂಡವರೂ ತಮ್ಮ ತಮ್ಮ ಸೈನ್ಯಗಳೊಂದಿಗೆ ಕುರುಕ್ಷೇತ್ರಕ್ಕೆ ಬಂದು ಶಿಬಿರಗಳನ್ನು ಸ್ಥಾಪಿಸುವುದು ಈ ವಿಷಯಗಳು ಈ ಅಧ್ಯಾಯದಲ್ಲಿ ವಿವರಿಸಲ್ಪಟ್ಟಿವೆ.

\vspace{-.2cm}

\section*{ಅಧ್ಯಾಯ\enginline{-}೨೫}

\begin{verse}
\textbf{ಸೇನಾಂ ವೀಕ್ಷ್ಯ ರಣೋನ್ಮುಖೇ ಕರುಣಯಾ ಶಸ್ತ್ರೋಜ್ಝಿತಂ ಫಲ್ಗುನಂ}\\\textbf{ಸದ್ಗೀತಾಮುಪದಿಶ್ಯ ಕಾರ್ಮುಕಧರಂ ಚಕ್ರೇಽಸ್ಯ ಯಃ ಸಾರಥಿಃ~।}\\\textbf{ಅನ್ಯೋನ್ಯಂ ಕುರುಪಾಂಡವೈಶ್ಯ ಪೃತನಾಂ ಯೋಽಜೀಘನತ್ಸ್ಯಂದನಾತ್ }\\\textbf{ಯೋ ಭೀಷ್ಮಂ ನಿರಪಾತಯತ್ಸುತಶರೈಃ ಪಾಂಡೋಸ್ತಮೀಡೇಽಚ್ಯುತಮ್~।।}
\end{verse}

ಯುದ್ಧ ಭೂಮಿಯಲ್ಲಿ ಸೈನ್ಯವನ್ನು ನೋಡಿ ಕೃಪೆಯಿಂದ ಶಸ್ತ್ರಾಸ್ತ್ರಗಳನ್ನು ತೊರೆದು ಯುದ್ಧ ಮಾಡಲು ಒಪ್ಪದ ಅರ್ಜುನನಿಗೆ ದೋಷರಹಿತವಾದ ಗೀತೋಪದೇಶವನ್ನು ಮಾಡಿ ಅರ್ಜುನನ ಕೈಯಲ್ಲಿ ಗಾಂಡೀವವನ್ನು ಹಿಡಿಸಿ ತಾನು ಅವನಿಗೆ ಸಾರಥಿಯಾಗಿ, ಕೌರವರ-\-ಪಾಂಡವರ ಸೈನ್ಯಗಳನ್ನು ಯುದ್ಧದಲ್ಲಿ ನಾಶಮಾಡಿಸಿ, ಪಾಂಡುಸುತನಾದ ಪಾರ್ಥನ ಬಾಣಗಳಿಂದ ಭೀಷ್ಮಾಚಾರ್ಯರನ್ನು ಕೆಳಗೆ ಬೀಳುವಂತೆ ಮಾಡಿದ ಅಚ್ಯುತನನ್ನು ಸ್ತುತಿಸುತ್ತೇನೆ.

ಈ ಅಧ್ಯಾಯದಲ್ಲಿ ೧೪೪ ಶ್ಲೋಕಗಳಿವೆ.

ಯುದ್ಧ ಭೂಮಿಯಲ್ಲಿ ಭೀಷ್ಮ, ದ್ರೋಣ ಮುಂತಾದವರಿಂದ ಕೂಡಿದ ಕೌರವರ ಹಾಗೂ ಪಾಂಡವರ ಸೈನ್ಯಗಳನ್ನೂ ನೋಡಿದ ಅರ್ಜುನನು ವಿಷಾದದಿಂದ ಯುದ್ಧ ಮಾಡಲು ಒಪ್ಪದಿರುವುದು, ಶ‍್ರೀಕೃಷ್ಣನು ಪಾರ್ಥನಿಗೆ ಉಪದೇಶಮಾಡಿದ ಸರ್ವೊತ್ಕೃಷ್ಟವಾದ ಗೀತೆಯ ಸಾರಾಂಶ, ಅರ್ಜುನನಿಗೆ ವಿಶ್ವರೂಪವನ್ನು ಅವನ ಯೋಗ್ಯತೆಗೆ ತಕ್ಕಷ್ಟು ತೋರಿಸಿದುದು, ಅರ್ಜುನನು ಯುದ್ಧ ಮಾಡಲು ಒಪ್ಪಿದುದು, ಭೀಮಸೇನನ ಪರಾಕ್ರಮ ವರ್ಣನೆ, ದ್ವಂದ್ವಯುದ್ಧಗಳ ವಿವರಣೆ, ಭೀಷ್ಮಾಚಾರ್ಯರ ಸಾಮರ್ಥ್ಯ, ಹತ್ತು ದಿನಗಳ ಪರ್ಯಂತ ಅವರ ಸೇನಾಧಿಪತ್ಯ, ಕೌರವರ ಪರಾಭವ, ದುರ್ಯೊಧನನು ರಾತ್ರಿಯಲ್ಲಿ ಭೀಷ್ಮರನ್ನು ಕಂಡು ಚರ್ಚಿಸುವುದು; ಭೀಷ್ಮರು ಕೃಷ್ಣ ಮತ್ತು ಪಾಂಡವರು ಯಾರೆಂಬ ವಿಷಯವನ್ನು ದುರ್ಯೋಧನನಿಗೆ ಹೇಳಿ ಪಾಂಡವರನ್ನು ಸೋಲಿಸಲು ಸಾಧ್ಯವಿಲ್ಲ ಎಂದು ತಿಳಿಸುವುದು, ದ್ರೋಣ, ಧೃಷ್ಟದ್ಯುಮ್ನ, ಕೃತವರ್ಮ, ದುರ್ಯೊಧನ, ಕೃಪ ಇವರು ಪರಾಕ್ರಮದಿಂದ ಯುದ್ಧ ಮಾಡುವುದು, ಅರ್ಜುನ-\-ಭೀಷ್ಮರ ಭೀಕರ ಯುದ್ಧ, ಭೀಮಸೇನನಿಂದ ಕೌರವರ ಅನೇಕ ಜನರ ಸಂಹಾರ, ಘಟೋತ್ಕಚ-\-ಭಗದತ್ತರ ಯುದ್ಧ ವಿವರಣೆ, ದುರ್ಯೊಧನ-ಘಟೋತ್ಕಚನ ಯುದ್ದದ ವಿಚಾರ, ಅನೇಕ ಕೌರವ ಸೈನಿಕರ ನಾಶ, ದುರ್ಯೋಧನನು ರಾತ್ರಿಯಲ್ಲಿ ಭೀಷ್ಮರನ್ನು ನಿಂದಿಸುವುದು, ಭೀಷ್ಮರ ಪ್ರತಿಜ್ಞೆ, ದುರ್ಯೊಧನ-ಅಭಿಮನ್ಯುವಿನ ಯುದ್ಧ, ಯುಧಿಷ್ಠಿರನು ರಾತ್ರಿ ಶಿಬಿರದಲ್ಲಿ ಭೀಷ್ಮರನ್ನು ಭೇಟಿ ಮಾಡುವುದು, ಭೀಷ್ಮಾಚಾರ್ಯರು ತಮ್ಮನ್ನು ಸಂಹಾರ ಮಾಡಲು ಪಾಂಡವರು ಮಾಡಬೇಕಾದ ಉಪಾಯ, ಯುಧಿಷ್ಠಿರ-ನಕುಲ-ಸಹದೇವರು ಭೀಷ್ಮಾಚಾರ್ಯರನ್ನು ಹೊಂದುವುದು, ಅರ್ಜುನನಿಂದ ರಕ್ಷಿತನಾದ ಶಿಖಂಡಿಯು ಭೀಷ್ಮರನ್ನು ಎದುರಿಸಿ ಬಾಣ ಹೊಡೆಯುವುದು, ಸ್ತ್ರೀತ್ವದಿಂದ ಇರುವ ಶಿಖಂಡಿಯ ಮೇಲೆ ಭೀಷ್ಮರು ಬಾಣಪ್ರಯೋಗ ಮಾಡದಿರುವುದು, ಅರ್ಜುನ-ದುಶ್ಯಾಸನರ ಹೋರಾಟ, ಭೀಷ್ಮರು ಪಾಂಡವರ ಸೈನ್ಯದಲ್ಲಿ 25 ಸಹಸ್ರ ಸೈನಿಕರನ್ನು ನಾಶಮಾಡುವುದು, ಅರ್ಜುನನು ಶ‍್ರೀಕೃಷ್ಣನ ಸಲಹೆಯಂತೆ ಭೀಷ್ಮರ ಎಂಟು ಧನುಸ್ಸುಗಳನ್ನು ನಾಶ ಮಾಡುವುದು, ಅರ್ಜುನನೂ ಇತರ ಪಾಂಡವರೂ ಒಟ್ಟಾಗಿ ಸೂರ್ಯಕಿರಣಗಳಿಗೆ ಸದೃಶವಾದ ಬಾಣಗಳನ್ನು ಭೀಷ್ಮರ ಮೇಲೆ ಬಿಟ್ಟ ಕಾರಣದಿಂದ ಭೀಷ್ಮರ ಪತನ, ಪಾಂಡವ-ಕೌರವರೂ ಶಿಬಿರದಲ್ಲಿ ಭೀಷ್ಮರ ದರ್ಶನ ಮಾಡುವುದು, ಭೀಷ್ಮರ ಬಳಿ ಶ‍್ರೀಕೃಷ್ಣನ ಸಾನ್ನಿಧ್ಯ, ಅರ್ಜುನನು ವಾರಣಾಸ್ತ್ರದಿಂದ ಭೂಮಿಯಿಂದ ಪರಿಮಳ ಯುಕ್ತವಾದ ನೀರನ್ನು ಭೀಷ್ಮರ ಬಾಯಲ್ಲಿ ಧಾರೆಯಾಗಿ ಬೀಳುವಂತೆ ಮಾಡಿದುದು, ಭೀಷ್ಮರು ದುರ್ಯೋಧನನಿಗೆ ಹಿತೋಕ್ತಿಗಳನ್ನು ಹೇಳುವುದು, ಉತ್ತರಾಯಣವನ್ನು ನಿರೀಕ್ಷಿಸುತ್ತಾ ಭೀಷ್ಮರು ಮಲಗಿರುವುದು, ಈ ವೃತ್ತಾಂತವನ್ನು ಸಂಜಯನಿಂದ ಕೇಳಿದ ಧೃತರಾಷ್ಟ್ರನು ದುಃಖಿತನಾಗುವುದು-ಈ ವಿವರಗಳು ಈ ಅಧ್ಯಾಯದಲ್ಲಿ ನಿರೂಪಿತವಾಗಿವೆ.

\vspace{-.2cm}

\section*{ಅಧ್ಯಾಯ\enginline{-}೨೬}

\begin{verse}
\textbf{ದ್ರೋಣೇ ಯುಧ್ಯತಿ ಪಾಂಡವೈರ್ವಿನಿಹಿತಂ ಪ್ರಾಗ್ಜ್ಯೋತಿಷಂ ಪಾರ್ಥತಃ}\\\textbf{ಕೃತ್ವಾ ತಸ್ಯ ಸುತೇ ಹತೇ ನಿಶಿ ಶಿವಂ ನೀತ್ವಾಽರ್ಜುನಂ ಸೈಂಧವಮ್~।}\\\textbf{ತದ್ದತ್ತಾಸ್ತ್ರ ಬಲಾದಜೀಘನದತೋ ದ್ರೋಣೇ ಹತೇ ದ್ರೌಪದೇಃ}\\\textbf{ಯೋ ಭೀಮಂ ಚ ನಿಜಾಸ್ತ್ರನಮ್ರಮಕರೋತ್ತಂ ನೌಮಿ ನಾರಾಯಣಮ್~।।}
\end{verse}

ದ್ರೋಣರು ಪಾಂಡವರೊಡನೆ ಯುದ್ಧ ಮಾಡುತ್ತಿರುವಾಗ ಅರ್ಜುನನಿಂದ ಭಗದತ್ತನನ್ನು ಸಂಹಾರಮಾಡಿಸಿದ, ಅರ್ಜುನನ ಪುತ್ರನಾದ ಅಭಿಮನ್ಯುವು ಯುದ್ಧದಲ್ಲಿ ಮೃತನಾಗಲು ಅರ್ಜುನನನ್ನು ರಾತ್ರಿ ಸ್ವಪ್ನದಲ್ಲಿ ರುದ್ರದೇವರ ಬಳಿ ಕರೆದುಕೊಂಡು ಹೋಗಿ ಅವರಿಂದ ಅಸ್ತ್ರವನ್ನು ಕೊಡಿಸಿ, ಅರ್ಜುನನಿಂದ ಜಯದ್ರಥನನ್ನೂ, ಧೃಷ್ಟದ್ಯುಮ್ನನಿಂದ ದ್ರೋಣರನ್ನೂ ಸಂಹಾರಮಾಡಿಸಿದ ಮತ್ತು ನಾರಾಯಣಾಸ್ತ್ರಕ್ಕೆ ಭಿಮಸೇನನು ನಮಸ್ಕರಿಸುವಂತೆ ಮಾಡಿದ ಶ‍್ರೀಮನ್ನಾರಾಯಣನನ್ನು ನಮಿಸುತ್ತೇನೆ.

ಈ ಅಧ್ಯಾಯದಲ್ಲಿ ೩೨೧ ಶ್ಲೋಕಗಳಿವೆ.

ಭೀಷ್ಮರ ಪತನವಾಗಲು ದ್ರೋಣರು ಸೇನಾಪತಿತ್ವವನ್ನು ವಹಿಸಿಕೊಳ್ಳುವುದು, ಯುಧಿಷ್ಠಿರನನ್ನು ಹಿಡಿಯಲು ದ್ರೋಣರ ಪ್ರತಿಜ್ಞೆ, ಭೀಮ-ಶಲ್ಯರ ಗದಾಯುದ್ಧ, ಶಲ್ಯನ ಹಾಗೂ ಕೌರವರ ಸೈನ್ಯದ ಪಲಾಯನ, ದ್ರೋಣರು ಯುಧಿಷ್ಠಿರನನ್ನು ಹಿಡಿಯಲು ಅಸಮರ್ಥರಾಗಲು ದುರ್ಯೋಧನನು ಅವರನ್ನು ನಿಂದಿಸುವುದು, ಭೀಮಸೇನನ ಉಗ್ರವಾದ ಯುದ್ಧ, ಭಗದತ್ತ-\-ಅರ್ಜುನರ ಕಾಳಗ, ಶ‍್ರೀಕೃಷ್ಣನಿಂದ ಕೊಡಲ್ಪಟ್ಟ ವೈಷ್ಣವಾಸ್ತ್ರದಿಂದ ಅರ್ಜುನನು ಭಗದತ್ತನನ್ನೂ ಅವನ ಆನೆಯನ್ನೂ ಸಂಹರಿಸುವುದು, ಪುನಃ ದ್ರೋಣರ ಪ್ರತಿಜ್ಞೆ, ಮಾರನೇ ದಿವಸ ದ್ರೋಣರಿಂದ ಪದ್ಮವ್ಯೂಹದ ರಚನೆ, ಅದರಲ್ಲಿ ಅಭಿಮನ್ಯುವು ಪ್ರವೇಶಿಸಿ ಯುದ್ಧ ಮಾಡುವುದು, ಕರ್ಣಾದಿಗಳು ಅಭಿಮನ್ಯುವನ್ನು ಶಸ್ತ್ರಹೀನನನ್ನಾಗಿ ಮಾಡುವುದು,\break ಅಭಿಮನ್ಯುವು ದುಃಶಾಸನನ ಮಗನೊಂದಿಗೆ ಯುದ್ಧ ಮಾಡುವಾಗ ಇಬ್ಬರೂ ಮೃತರಾಗುವುದು, ಯುಧಿಷ್ಠಿರನ ದುಃಖ, ವೇದವ್ಯಾಸರ ಹಿತೋಕ್ತಿ, ಶ‍್ರೀಕೃಷ್ಣನು ಅರ್ಜುನನನ್ನು ರಾತ್ರಿ ಸ್ವಪ್ನದಲ್ಲಿ ರುದ್ರದೇವರ ಬಳಿ ಕರೆದುಕೊಂಡು ಹೋಗಿ ಜಯದ್ರಥನನ್ನು ಸಂಹರಿಸಲು ಸಾಧ್ಯವಾಗುವ ಅಸ್ತ್ರವನ್ನು ರುದ್ರದೇವರಿಂದ ಕೊಡಿಸುವುದು, ಸೈಂಧವನನ್ನು ಸಂಹರಿಸಿ ಅವನ ಶಿರಸ್ಸನ್ನು ಅವನ ತಂದೆಯ ತೊಡೆಯಲ್ಲಿರಿಸುವುದಾಗಿ ಅರ್ಜುನನು ಪ್ರತಿಜ್ಞೆ ಮಾಡುವುದು, ಮಾರನೇದಿನ ಯುದ್ಧದಲ್ಲಿ ಸೈಂಧವನನ್ನು ರಕ್ಷಿಸಲು ದ್ರೋಣರು ಸರ್ವಪ್ರಯತ್ನವನ್ನೂ ಮಾಡುತ್ತಾ ಸೈನ್ಯವನ್ನು ಶಕಟಾಬ್ಜ ಚಕ್ರವ್ಯೂಹದಲ್ಲಿ ನಿಲ್ಲಿಸುವುದು, ದ್ರೋಣ-ಅರ್ಜುನರ ಯುದ್ಧ ಅರ್ಜುನನು ಕೌರವರ ಸೈನ್ಯದಲ್ಲಿ ಅನೇಕ ಪರಾಕ್ರಮಶಾಲಿಗಳನ್ನು ಸಂಹರಿಸುವುದು, ಯುಧಿಷ್ಠಿರನನ್ನು ಹಿಡಿಯಲು ದ್ರೋಣರ ಮಹಾಪ್ರಯತ್ನ, ಭೀಮಸೇನನು ಅಲಂಬುಸ\break ರಾಕ್ಷಸನನ್ನು ಪೀಡಿಸುವುದು, ಘಟೋತ್ಕಚನು ಅಲಂಬುಸ ರಾಕ್ಷಸನನ್ನು ಸೀಳುವುದು,\break ದ್ರೋಣರು ಪಾಂಡವರ ಸೈನ್ಯದಲ್ಲಿ ನುಗ್ಗಿ ದ್ರುಪದನ ಮಗನಾದ ಸತ್ಯಜಿತುವಿನ ಶಿರಸ್ಸನ್ನು ಕತ್ತರಿಸಿ ಯುಧಿಷ್ಠಿರನನ್ನು ಆಯುಧ-ರಥ ರಹಿತನನ್ನಾಗಿ ಮಾಡುವುದು, ಸಾತ್ಯಕಿ-ಕೃತವರ್ಮರ ಭಯಂಕರ ಯುದ್ಧ, ಭೀಮಸೇನನು ದ್ರೋಣರ ರಥವನ್ನು ತನ್ನ ಗದಾ ಪ್ರಹಾರದಿಂದ ಪುಡಿ ಮಾಡುವುದು, ವಿಂದನೇ ಮೊದಲಾದ ಹನ್ನೆರಡುಮಂದಿ ಧೃತರಾಷ್ಟ್ರನ ಮಕ್ಕಳು ಭೀಮಸೇನನಿಂದ ಹತರಾಗುವುದು, ಕರ್ಣನು ಭೀಮಸೇನನೊಡನೆ ಯುದ್ಧ ಮಾಡಿ ಪರಾಭವಗೊಳ್ಳುವುದು, ಜಯದ್ರಥನ ವಧೆಗಾಗಿ ಶ‍್ರೀಕೃಷ್ಣನು ಕತ್ತಲೆಯನ್ನು ಸೃಷ್ಟಿಸುವುದು, ಕೌರವರ ಸೈನ್ಯವೆಲ್ಲವೂ ವಿಶ್ರಾಂತಿಯನ್ನು ಪಡೆಯುತ್ತಿರುವಾಗ ಅರ್ಜುನನು ಶ‍್ರೀಕೃಷ್ಣನ ಆಜ್ಞೆಯಂತೆ ಬಾಣಪ್ರಯೋಗದಿಂದ ಜಯದ್ರಥನ ಶಿರಸ್ಸನ್ನು ಕತ್ತರಿಸಿ ಅವನ ತಂದೆಯಾದ ವೃದ್ಧಕ್ಷತ್ರನ ತೊಡೆಯಲ್ಲಿ ಬೀಳಿಸುವುದು, ಶ‍್ರೀಕೃಷ್ಣನು ಕೂಡಲೆ ಕತ್ತಲೆಯನ್ನು ನಿವಾರಿಸುವುದು, ಪಾಂಡವರ ಕಡೆಯಲ್ಲಿ ಸಂತೋಷ, ಕೌರವರಿಗೆ ಅತ್ಯಂತ ದುಃಖ, ದುರ್ಯೋಧನನು ಕೋಪದಿಂದ ಪಾಂಡವರೊಡನೆ ಭೀಕರ ಯುದ್ಧವನ್ನು ಮಾಡುವುದು, ದ್ರೋಣರ ಪ್ರತಿಜ್ಞೆ, ಘಟೋತ್ಕಚ-\-ಅಶ್ವತ್ಥಾಮರ ಕಾಳಗ, ಅಲಾಯುಧನು ಘಟೋತ್ಕಚನಿಂದ ಸಾಯುವುದು, ಕರ್ಣನು ವಾಸವೀ ಶಕ್ತಿಯನ್ನು ಪ್ರಯೋಗಿಸಿ ಘಟೋತ್ಕಚನನ್ನು ಸಂಹರಿಸುವುದು, ಇದರಿಂದ ಕೌರವರ ಸಂತೋಷ, ಭೀಮಸೇನನು ಬಾಹ್ಲೀಕರಾಜನನ್ನು ಗದೆಯಿಂದ ಸಂಹರಿಸುವುದು, ರಾತ್ರಿಕಾಲದಲ್ಲಿ ದೀಪಗಳನ್ನು ಹಿಡಿದುಕೊಂಡು ಯುದ್ಧ ಮಾಡುವ ವಿವರಣೆ, ಸೂರ್ಯೋದಯಾನಂತರ ದ್ರೋಣರು ಇಪ್ಪತ್ತು ಸಹಸ್ರ ಪಾಂಡವರ ರಥಗಳನ್ನು ನಾಶಮಾಡುವುದು, ದ್ರೋಣರಿಂದ ವಿರಾಟ ಹಾಗೂ ದ್ರುಪದರಾಜರ ಸಂಹಾರ, ಭೀಮಸೇನನಿಂದ ರಕ್ಷಿತನಾದ ಧೃಷ್ಟದ್ಯುಮ್ನನು ದ್ರೋಣರನ್ನು ಎದುರಿಸುವುದು, ಯುಧಿಷ್ಠಿರನು “ಅಶ್ವತ್ಥಾಮಾ ಹತಃ” ಎಂಬುದಾಗಿ ಕೂಗಿದುದು, ಅದನ್ನು ಕೇಳಿ ದ್ರೋಣರು ಶಸ್ತ್ರ ಪರಿತ್ಯಾಗ ಮಾಡುವುದು, ಧೃಷ್ಟದ್ಯುಮ್ನನು ದ್ರೋಣರ ಶಿರಸ್ಸನ್ನು ಛೇದಿಸುವುದು, ಅಶ್ವತ್ಥಾಮಾಚಾರ್ಯರು ಕೋಪದಿಂದ ನಾರಾಯಣಾಸ್ತ್ರವನ್ನು ಪ್ರಯೋಗಿಸುವುದು, ಶ‍್ರೀಕೃಷ್ಣನ ಸೂಚನೆಯಂತೆ ಭೀಮಸೇನನ ವಿನಃ ಇತರ ಪಾಂಡವರು ಅಸ್ತ್ರಕ್ಕೆ ನಮಸ್ಕಾರ ಮಾಡಿ ರಕ್ಷಿತರಾದುದು, ಭೀಮಸೇನನು ಮಾತ್ರ ನಮಸ್ಕಾರ ಮಾಡದೇ ಇದ್ದರೂ ರಕ್ಷಿತನಾದುದು, ಅರ್ಜುನ-ಅಶ್ವತ್ಥಾಮರ ಘೋರ ಯುದ್ಧ, ಅಶ್ವತ್ಥಾಮರಿಗೆ ವೇದವ್ಯಾಸರ ಹಿತೋಕ್ತಿ, ಕೌರವರ ದುಃಖ, ಪಾಂಡವರು ಸಂತೋಷದಿಂದ ಶಿಬಿರಕ್ಕೆ ಹೋಗುವುದು ಈ ವಿಷಯಗಳು ಈ ಅಧ್ಯಾಯದಲ್ಲಿ ನಿರೂಪಿತವಾಗಿವೆ.


\section*{ಅಧ್ಯಾಯ\enginline{-}೨೭}

\begin{verse}
\textbf{ಯತ್ಸಾಮರ್ಥ್ಯಬಲೇನ ಸೂರ್ಯತನುಜೇ ಪಾರ್ಥೇನ ಯುದ್ಧೇ ಜಿತೇ}\\\textbf{ಪಶ್ಚಾತ್ ಶಲ್ಯಮವಾಪ್ಯ ಸಾರಥಿವರಂ ಧರ್ಮಾತ್ಮಜಂ ಸಾಯಕೈಃ~।}\\\textbf{ಶೀರ್ಣಾಂಗಂ ಕೃತವತ್ಯಮುಂ ಶಿಬಿರಗಂ ಪಾರ್ಥಂ ಚ ಮೃತ್ಯೋರಪಾ}-\\\textbf{ದ್ಯಃ ಪಾರ್ಥೇನ ಹತೇಽರ್ಕಜೇ ನೃಪನುತಃ ಪಾಯಾತ್ ಸ ನಃ ಕೇಶವಃ~।।}
\end{verse}

ಯಾವ ಶ‍್ರೀಕೃಷ್ಣನ ಸಾಮರ್ಥ್ಯ ಮತ್ತು ಅನುಗ್ರಹ ಬಲಗಳಿಂದ ಅರ್ಜುನನು ಕರ್ಣನನ್ನು ಯುದ್ಧದಲ್ಲಿ ಪರಾಭವಗೊಳಿಸಿದನೋ, ಕರ್ಣನು ಮತ್ತೆ ಶಲ್ಯನನ್ನು ಸಾರಥಿಯನ್ನಾಗಿ ಮಾಡಿಕೊಂಡು ಬಂದು ಧರ್ಮರಾಜನಿಗೆ ಶಸ್ತ್ರಾಸ್ತ್ರಗಳಿಂದ ದೇಹದಲ್ಲಿ ಅನೇಕ ಗಾಯಗಳನ್ನು ಮಾಡಿದಾಗ ಯಾವ ಶ‍್ರೀಕೃಷ್ಣನು ಅವನನ್ನು ಶಿಬಿರಕ್ಕೆ ಕಳಿಸಿದನೋ, ಮತ್ತು ಕರ್ಣನು ಅರ್ಜುನನನ್ನು ಸಂಹರಿಸಲು ನಾಗಾಸ್ತ್ರವನ್ನು ಪ್ರಯೋಗಿಸಿದಾಗ ಅರ್ಜುನನನ್ನು ರಕ್ಷಿಸಿ, ಅರ್ಜುನನಿಂದ ಕರ್ಣನನ್ನು ಸಂಹಾರ ಮಾಡಿಸಿದನೋ ಅಂತಹ ಕೇಶವನು ನಮ್ಮನು ಸಲಹಲಿ.

ಈ ಅಧ್ಯಾಯದಲ್ಲಿ ೧೯೨ ಶ್ಲೋಕಗಳಿವೆ.

ದ್ರೋಣರು ಹತರಾದ ನಂತರ ಕಣ೯ನು ಸೇನಾಪತಿಯಾಗುವುದು, ಭೀಮಸೇನನು ಕ್ಷೇಮಧೂರ್ತಿಯನ್ನು ಸಂಹರಿಸುವುದು, ಭೀಮಸೇನ-ಅಶ್ವತ್ಥಾಮರ ಘೋರ ಸಮರ, ಧರ್ಮರಾಜ-ದುರ್ಯೋಧನರ ಯುದ್ದ, ಸಾತ್ಯಕಿಯಿಂದ ವಿಂದಅನುವಿಂದರ ಹನನ, ಅರ್ಜುನನಿಂದ ಕರ್ಣನ ಪರಾಭವ, ದುರ್ಯೋಧನನು ಶಲ್ಯನನ್ನು ಕರ್ಣನ ಸಾರಥಿಯಾಗಿ ನೇಮಿಸುವುದು, ಕರ್ಣನು ಶ‍್ರೀಕೃಷ್ಣನನ್ನು ನಿಂದಿಸುವುದು, ಭೀಮಸೇನನು ಕರ್ಣನಿಗೆ ಬಾಣಗಳಿಂದ ಭಯಂಕರವಾದ ಗಾಯಗಳನ್ನು ಮಾಡುವುದು, ಅಶ್ವತ್ಥಾಮ-ಅರ್ಜುನರ ಭೀಕರ ಯುದ್ಧ, ಅಶ್ವತ್ಥಾಮನು ಅರ್ಜುನನನ್ನು ಬಾಣಗಳಿಂದ ಕಟ್ಟುವುದು, ಶ‍್ರೀಕೃಷ್ಣನ ಆಲಿಂಗನದಿಂದ ಬಂಧ ವಿಮೋಚನೆ, ಧೃಷ್ಟದ್ಯುಮ್ನ-ಅಶ್ವತ್ಥಾಮರ ಯುದ್ಧ, ಭೀಮಸೇನನಿಂದ ಕರ್ಣನ ಮಗನಾದ ಸುಷೇಣನು ಮೃತನಾಗುವುದು, ಕರ್ಣನು ಧರ್ಮರಾಜನಿಗೆ ತೀಕ್ಷ್ಣವಾದ ಬಾಣಗಳಿಂದ ಗಾಯಗಳನ್ನು ಮಾಡುವುದು, ಯುದ್ದದಲ್ಲಿ ಕೌರವರ ಕೈ ಮೇಲಾಗುವುದು, ಭೀಮಸೇನನ ಸಿಂಹನಾದ, ಕರ್ಣನು ಭಾರ್ಗವಾಸ್ತ್ರವನ್ನು ಪ್ರಯೋಗಿಸುವುದು, ಕೃಷ್ಣಾರ್ಜುನರು ಶಿಬಿರದಲ್ಲಿ ಗಾಯಗೊಂಡ ಧರ್ಮರಾಜನನ್ನು ನೋಡಲು ಹೋದಾಗ ಕರ್ಣನನ್ನು ಸಂಹಾರಮಾಡದೇ ಇದ್ದ ಕಾರಣದಿಂದ ಅರ್ಜುನನನ್ನು ನಿಂದಿಸುವುದು, ಅರ್ಜುನನು ಧರ್ಮರಾಜನನ್ನು ಸಂಹರಿಸಲು ಖಡ್ಗವನ್ನು ಸೆಳೆಯುವುದು, ಶ‍್ರೀಕೃಷ್ಣನು ಅರ್ಜುನನಿಗೆ ಧರ್ಮಾಧರ್ಮಗಳ ಉಪದೇಶಮಾಡಿ ಸೌಹಾರ್ದವನ್ನು ತರುವುದು, ಧರ್ಮರಾಜನಿಗೆ ಅಗೌರವವನ್ನು ಮಾಡಿದ ಕಾರಣ ಅರ್ಜುನನು ಆತ್ಮ ಪ್ರಶಂಸೆಯನ್ನು ಮಾಡಿಕೊಂಡು ಆತ್ಮಹತ್ಯೆಗೆ ಸಮನಾದ ಪರಿಹಾರವನ್ನು ಮಾಡಿಕೊಳ್ಳುವುದು, ಧರ್ಮರಾಜ ಮತ್ತು ಅರ್ಜುನರು ಶ‍್ರೀಕೃಷ್ಣನನ್ನು ಸ್ತುತಿಸುವುದು, ಶ‍್ರೀಕೃಷ್ಣನು ಅರ್ಜುನನಲ್ಲಿ ನರಾಂಶವನ್ನು ಅಭಿವೃದ್ಧಿಪಡಿಸಿ ಕರ್ಣನೊಡನೆ ಯುದ್ಧಮಾಡಲು ಸೈರ್ಯ-ಧೈರ್ಯಗಳನ್ನು ತುಂಬುವುದು, ಭೀಮಸೇನನು ದುಶ್ಯಾಸನನನ್ನು ಕೆಡವಿ, ಎದೆಯನ್ನು ಸೀಳಿ, ಅವನ ರಕ್ತಪಾನ ಮಾಡುವುದು, ರಕ್ತವು ದಂತಗಳ ಅಂತರ್ಭಾಗವನ್ನು ಪ್ರವೇಶಮಾಡದೆ ಇರುವುದು, ಈ ಕೃತ್ಯವನ್ನು ಭೀಮಸೇನನು ನರಸಿಂಹರೂಪಿ ಪರಮಾತ್ಮನಿಗೆ ಸಮರ್ಪಿಸುವುದು, ಭೀಮಸೇನನು ಕೌರವರನ್ನು ಯುದ್ದಕ್ಕಾಗಿ ಆಹ್ವಾನಿಸುವುದು, ಶತ್ರುಸೇನೆಯ ಮಧ್ಯದಲ್ಲಿ ಕುಣಿದಾಡುವುದು, ಇಪ್ಪತ್ತು ಮಂದಿ ಧೃತರಾಷ್ಟ್ರನ ಮಕ್ಕಳನ್ನು ಭೀಮಸೇನನು ಸಂಹರಿಸುವುದು, ಕರ್ಣಾರ್ಜುನರ ಭೀಕರ ಯುದ್ದ, ದೇವತೆಗಳು ಈ ದೃಶ್ಯವನ್ನು ನೋಡುವುದು, ಯುದ್ಧಭೂಮಿಯಲ್ಲಿ ಅಶ್ವತ್ಥಾಮರು ದುರ್ಯೋಧನನಿಗೆ ಬುದ್ಧಿ ಹೇಳುವುದು, ಕರ್ಣನು ಅರ್ಜುನನಲ್ಲಿ ಬ್ರಹ್ಮಾಸ್ತ್ರದಿಂದ ಕೂಡಿದ ನಾಗಾತ್ಮಕವಾದ ಬಾಣಶ್ರೇಷ್ಠವನ್ನು ಪ್ರಯೋಗಿಸುವುದು, ಶ‍್ರೀಕೃಷ್ಣನು ರಥವನ್ನು ಐದು ಅಂಗುಲದಷ್ಟು ಬಗ್ಗಿಸಿ ನಾಗಾಸ್ತ್ರವು ಅರ್ಜುನನ ಕಿರೀಟವನ್ನು ಪುಡಿಮಾಡುವಂತೆ ಮಾಡುವುದು, ಕರ್ಣನು ರಥವು ವಿಪ್ರಶಾಪದ ನಿಮಿತ್ತದಿಂದ ಭೂಮಿಯಲ್ಲಿ ಹೂತಿಕೊಳ್ಳುವುದು, ರಥವನ್ನು ಮೇಲಕ್ಕೆ ಎತ್ತುತ್ತಾ ಇರುವ ಸಮಯದಲ್ಲಿ ಶ‍್ರೀಕೃಷ್ಣನು ಅರ್ಜುನನನ್ನು ಬಾಣಪ್ರಯೋಗ ಮಾಡುವಂತೆ ಹೇಳುವುದು, ಅರ್ಜುನನು ರುದ್ರದೇವರಿಂದ ಕೊಡಲ್ಪಟ್ಟ ಅಂಜಲಿ ಎಂಬ ಬಾಣವನ್ನು ಪ್ರಯೋಗಿಸಿ ಕರ್ಣನ ಶಿರಸ್ಸನ್ನು ಛೇದಿಸುವುದು, ಕೌರವರ ಸೈನ್ಯದಲ್ಲಿ ಅಗಾಧ ದುಃಖ, ಪಾಂಡವರ ಕಡೆಯಲ್ಲಿ ಸಂತೋಷ, ಪಾಂಡವರು ಶ‍್ರೀಕೃಷ್ಣನನ್ನು ಸ್ತೋತ್ರಮಾಡುವುದು ಈ ವಿಷಯಗಳು ಈ ಅಧ್ಯಾಯದಲ್ಲಿ ವಿವರಿಸಲ್ಪಟ್ಟಿವೆ.


\section*{ಅಧ್ಯಾಯ\enginline{-}೨೮}

\begin{verse}
\textbf{ಶಲೈ ಧರ್ಮಸುತಾದ್ಧತೇ ಕುರುಬಲೇ ಪಾರ್ಥೈಃ ಸಮಸ್ತೇ ಹತೇ}\\\textbf{ಭೀಮೇನಾರ್ಜುನಸಂಯುತೇ ವಿನಿಹತೇ ದುರ್ಯೋಧನೇ ದ್ರೋಣಿನಾ~।}\\\textbf{ಸುಸ್ತಾನಾಂ ನಿಧನೇ ಕೃತೇ ನಿಶಿತತೋ ಮುಕ್ತ್ವಾ ವಿಧೇರಸ್ತ್ರತಃ}\\\textbf{ಪಾರ್ಥಾನ್ ರಾಜ್ಯ ಮಿತಾಂಶ್ಚ ತತ್ಸು ತಸುತಂ ಯೋಽಪಾತ್ಸನೋಽವ್ಯಾದ್ಧರಿಃ~।।}
\end{verse}

ಯಾವ ಶ‍್ರೀಕೃಷ್ಣನು ಧರ್ಮರಾಜನಿಂದ ಶಲ್ಯನು ಹತನಾಗುವಂತೆ ಮಾಡಿದನೋ, ಭೀಮಾರ್ಜುನರಿಂದ ಕೌರವರ ಸಮಸ್ತ ಸೈನ್ಯವನ್ನೂ ನಾಶಪಡಿಸಿದನೋ, ಭೀಮಸೇನನಿಂದ ದುರ್ಯೋಧನನನ್ನು ಸಂಹರಿಸಿದನೋ, ಅಶ್ವತ್ಥಾಮರಿಂದ ಉಪಪಾಂಡವರನ್ನು ನಿರ್ಮೂಲ ಮಾಡಿಸಿದನೋ, ಯಾವ ಶ‍್ರೀಹರಿಯು ಪಾಂಡವರಿಗೆ ಅವರ ರಾಜ್ಯವನ್ನು ದೊರಕಿಸಿಕೊಟ್ಟನೋ, ಯಾವ ಶ‍್ರೀಕೃಷ್ಣನು ಪರೀಕ್ಷಿತ ರಾಜನನ್ನು ಅಶ್ವತ್ಥಾಮಾಚಾರ್ಯರಿಂದ ರಕ್ಷಿಸಿದನೋ ಅಂತಹ ಶ‍್ರೀಹರಿಯು ನಮ್ಮನ್ನು ಸಲಹಲಿ.

ಈ ಅಧ್ಯಾಯದಲ್ಲಿ ೨೪೧ ಶ್ಲೋಕಗಳಿವೆ.

ದುರ್ಯೋಧನನು ಶಲ್ಯನನ್ನು ಸೇನಾಧಿಪತಿಯನ್ನಾಗಿ ನಿಯಮಿಸಿ ಯುದ್ಧಕ್ಕೆ ಕಳಿಸುವುದು; ಧರ್ಮರಾಜ-ಶಲ್ಯ, ಭೀಮಸೇನ-ಅಶ್ವತ್ಥಾಮರ ಘೋರವಾದ ಯುದ್ಧ, ನಂತರ ಅರ್ಜುನ-ಶಲ್ಯರ ಕಾಳಗ, ಇಬ್ಬರೂ ಅದ್ಭುತವಾದ ಅಸ್ತ್ರಗಳನ್ನು ಪ್ರಯೋಗಿಸುವುದು, ರಥಗಳ ನಾಶ, ಧರ್ಮರಾಜನು ಭೀಮಸೇನನಿಂದ ಶಿಥಿಲ ಗೊಳಿಸಲ್ಪಟ್ಟ ಶಲ್ಯನ ಮೇಲೆ ಬಾಣಪ್ರಯೋಗಮಾಡಿ ಸಂಹರಿಸುವುದು, ನಂತರ ಭೀಮಸೇಸನು ಉಳಿದ ದುರ್ಯೋಧನನ ತಮ್ಮಂದಿರನ್ನು ಸಂಹಾರ ಮಾಡುವುದು, ಸಹದೇವನು ಶಕುನಿಯನ್ನು ಸಂಹರಿಸುವುದು, ಅಶ್ವತ್ಥಾಮ, ಕೃತವರ್ಮ, ಕೃಪಾಚಾರ್ಯರು ಭೀಮಸೇನನಿಂದ ಪರಾಜಯವನ್ನು ಹೊಂದಿ ಅರಣ್ಯವನ್ನು ಪ್ರವೇಶ ಮಾಡುವುದು, ದುರ್ಯೋಧನನಿಗೂ ಪಾಂಡವರಿಗೂ ಆದ ಯುದ್ಧ, ದುರ್ಯೋಧನನ ರಥಾದಿಗಳನ್ನು ಭೀಮಸೇನನು ನಾಶಪಡಿಸಿದ ಮೇಲೆ ದುರ್ಯೋಧನನು ವೇದವ್ಯಾಸರ ಸರೋವರವನ್ನು ಪ್ರವೇಶಿಸುವುದು, ಭೀಮಾರ್ಜುನರು ಮತ್ತು ಇತರರಿಂದ ಎಷ್ಟೆಷ್ಟು ಸೈನ್ಯಗಳು ಹತವಾದುವು ಎಂಬ ವಿವರಣೆ, ದುರ್ಯೋಧನನು ಸರೋವರದಲ್ಲಿ ದುರ್ವಾಸರಿಂದ ಉಪದೇಶಮಾಡಲ್ಪಟ್ಟ ಮಂತ್ರಗಳನ್ನು ಜಪಿಸುವುದು, ಪಾಂಡವರು ವ್ಯಾಧರಿಂದ ದುರ್ಯೋಧನನು ಇರತಕ್ಕ ಸ್ಥಳವನ್ನು ಕಂಡುಹಿಡಿಯುವುದು, ಶ‍್ರೀಕೃಷ್ಣನಿಂದ ಸಹಿತರಾದ ಪಾಂಡವರನ್ನು ನೋಡಿ ಅಶ್ವತ್ಥಾಮ, ಕೃತವರ್ಮ, ಕೃಪಾಚಾರ್ಯರು ಓಡಿಹೋಗುವುದು, ದುರ್ಯೋಧನನು ಮತ್ತೆ ಜಲಪ್ರವೇಶ ಮಾಡುವುದು, ಶ‍್ರೀಕೃಷ್ಣನ ಆಜ್ಞೆಯಿಂದ ಧರ್ಮರಾಜನು ದುರ್ಯೋಧನನನ್ನು ನಿಷ್ಠುರವಾಕ್ಯಗಳಿಂದ ನಿಂದಿಸುವುದು, ದುರ್ಯೋಧನನು ಜಲದಿಂದ ಮೇಲೆ ಬಂದ ನಂತರ ಧರ್ಮರಾಜನು ಅವನನ್ನು ಯುದ್ಧಕ್ಕೆ ಕರೆಯುವುದು, ಯಾರೊಡನೆ ಬೇಕಾದರೂ ಯುದ್ಧ ಮಾಡಲು ದುರ್ಯೋಧನನಿಗೆ ಆಹ್ವಾನಕೊಡುವುದು, ಅವನು ಭೀಮಸೇನನೊಂದಿಗೆ ಗದೆಯಿಂದ ಯುದ್ಧ ಮಾಡುವುದಾಗಿ ಹೇಳುವುದು, ಶ‍್ರೀಕೃಷ್ಣನ ಆಜ್ಞೆಯಂತೆ ಭೀಮಸೇನನು ದುರ್ಯೋಧನನ ತೊಡೆಯನ್ನು ಸೀಳುವುದಾಗಿ ಮಾಡಿದ್ದ ಪ್ರತಿಜ್ಞೆಯನ್ನು ಘೋಷಿಸುವುದು, ಭೀಮಸೇನ-ದುರ್ಯೊಧನರ ಗದಾಯುದ್ಧ, ಬಲರಾಮನಿಂದ ತಡೆಯಲ್ಪಟ್ಟರೂ ಯುದ್ದವು ಮುಂದುವರೆಯುವುದು, ದುರ್ಯೋಧನನು ತಲೆಯನ್ನು ಕೆಳಗೆ ಮಾಡಿ ನಿಂತಾಗ ಭೀಮ\-ಸೇನನು ದುರ್ಯೋಧನನ ತೊಡೆಯನ್ನು ಗದೆಯಿಂದ ಸೀಳುವುದು, ಕಾಲಿನಿಂದ\break ದುರ್ಯೊಧನನ ಶಿರಸ್ಸಿನಲ್ಲಿ ತಾಡನ, ದುರ್ಯೋಧನನು ಮಾಡಿದ ನೀಚಕೃತ್ಯಗಳನ್ನು ಅವನಸ್ಮೃತಿಗೆ ತರುವುದು, ಬಲರಾಮನ ಆಕ್ಷೇಪಣೆ, ಶ‍್ರೀಕೃಷ್ಣನು ಸಮಾಧಾನಪಡಿಸಿ ಭೀಮಸೇನನು ಮಾಡಿದ ಕಾರ್ಯವು ಅಧರ್ಮವಲ್ಲವೆಂಬುದನ್ನು ಸ್ಥಿರಪಡಿಸುವುದು, ಬಲರಾಮನ ನಿರ್ಗಮನ, ಧರ್ಮರಾಜನ ಸಂದೇಹವನ್ನು ಶ‍್ರೀಕೃಷ್ಣನು ಪರಿಹರಿಸುವುದು, ಇಂತಹ ಕಾಲದಲ್ಲಿಯೂ ದುರ್ಯೋಧನನು ಶ‍್ರೀಕೃಷ್ಣನನ್ನು ನಿಂದಿಸುವುದು, ಶ‍್ರೀಕೃಷ್ಣನು ದುರ್ಯೋಧನನು ಮಾಡಿದ ಪಾಪಕೃತ್ಯಗಳನ್ನು ಅವನಿಗೆ ತೋರಿಸುವುದು ಭಗವಂತನ ನಿಂದೆಯನ್ನು ಮಾಡುತ್ತಾ ದುರ್ಯೋಧನನು ಪ್ರಾಣ ಬಿಡುವ ಸ್ಥಿತಿಯಲ್ಲಿರುವುದು, ದೇವತೆಗಳು ಅವನ ಮೇಲೆ ಪುಷ್ಪವೃಷ್ಟಿಯನ್ನು ಮಾಡುವುದು, ನಂತರ ಸಕಲರೂ ಶ‍್ರೀಕೃಷ್ಣನನ್ನು ಪೂಜಿಸುವುದು, ಸಂಜಯನಿಂದ ವಿಷಯವನ್ನು ಕೇಳಿದ ಧೃತರಾಷ್ಟ್ರನು ಮಗನ ಗತಿಗೆ ದುಃಖಿತನಾಗುವುದು, ಶ‍್ರೀ ಕೃಷ್ಣನ ಸಮಾಧಾನ, ನಾಯಿನರಿಗಳು ತಿನ್ನುತ್ತಾ ಇರುವ ಶ್ವಾಸಮಾತ್ರವನ್ನು ಬಿಡುತ್ತಿರುವ ದುರ್ಯೋಧನನ ಬಳಿ ಅಶ್ವತಾಮಾಚಾರ್ಯರು ಪ್ರಾಪ್ತರಾಗುವುದು, ದುರ್ಯೋಧನನು ಅಶ್ವತ್ಥಾಮಾಚಾರ್ಯರಿಗೆ ತನ್ನ ಪತ್ನಿಯಿಂದ ಸಂತಾನವನ್ನು ಉತ್ಪಾದಿಸಿ ಅದರ ಮೂಲಕ ಪಾಂಡವರನ್ನು ನಿರ್ನಾಮ ಮಾಡುವಂತೆ ಕೇಳಿಕೊಳ್ಳುವುದು, ಅಶ್ವತ್ಥಾಮರು ಇದಕ್ಕೆ ಒಪ್ಪುವುದು, ರಾತ್ರಿಯಲ್ಲಿ ಅಶ್ವತ್ಥಾಮರು ಪಾಂಡವರ ಶಿಬಿರಕ್ಕೆ ಬರುವುದು, ದ್ವಾರದಲ್ಲಿ ಶ‍್ರೀಕೃಷ್ಣನಿಂದ ಆವೃತವಾದ ಬಹು ಕೋಟಿ ಸ್ವರೂಪವುಳ್ಳ ತನ್ನದೇ ಆದ ರುದ್ರರೂಪವನ್ನು ಕಾಣುವುದು, ರುದ್ರ\-ದೇವರಿಗೂ ಅಶ್ವತ್ಥಾಮಾಚಾರ್ಯರಿಗೂ ಯುದ್ಧ, ಅಶ್ವತ್ಥಾಮರು ಶಿಬಿರ ಪ್ರವೇಶ ಮಾಡಿ\break ಧೃಷ್ಟದ್ಯುಮ್ನನನ್ನು ಸಂಹರಿಸುವುದು, ನಂತರ ದ್ರುಪದರಾಜನ ನಾಲ್ಕು ಮಕ್ಕಳನ್ನು ಸಂಹರಿಸುವುದು, ಶಿಬಿರದಿಂದ ಓಡಿ ಹೋಗುತ್ತಿದ್ದವರನ್ನು ದ್ವಾರದಲ್ಲಿ ಕೃಪ, ಕೃತವರ್ಮರು ಹನನ ಮಾಡುವುದು, ದೌಪದೀ ಪುತ್ರರ ಶಿರಸ್ಸುಗಳನ್ನು ಅಶ್ವತ್ಥಾಮರು ಇನ್ನೂ ಉಸಿರಾಡುತ್ತಿರುವ ದುರ್ಯೋಧನನಿಗೆ ತೋರಿಸುವುದು, ದುರ್ಯೊಧನನ ಮರಣ, ಅಶ್ವತ್ಥಾಮಾದಿಗಳು ಭೀಮಸೇನನ ಭಯದಿಂದ ಓಡಿ ಹೋಗುವುದು, ಭೀಮಸೇನನ ಮೇಲೆ ಅಶ್ವತ್ಥಾಮರಿಂದ ಬ್ರಹ್ಮಾಸ್ತ್ರ ಪ್ರಯೋಗ, ಶ‍್ರೀಕೃಷ್ಣನು ಬ್ರಹ್ಮಾಸ್ತ್ರದಿಂದ ಎಲ್ಲರನ್ನೂ ರಕ್ಷಿಸುವುದು, ಶ‍್ರೀಕೃಷ್ಣನ ಆಜ್ಞೆಯಂತೆ ಅಶ್ವತ್ಥಾಮರು ತಮ್ಮ ಶಿರಸ್ಸಿನಲ್ಲಿದ್ದ ಅಮೋಘವಾದ ಮಣಿಯನ್ನು ಭೀಮಸೇನನಿಗೆ ಕೊಡುವುದು, ಅಶ್ವತ್ಥಾಮರು ಉತ್ತರೆಯ ಗರ್ಭದಲ್ಲಿರತಕ್ಕ ಶಿಶುವಿನ ಮೇಲೆ ಅಸ್ತ್ರ ಪ್ರಯೋಗಮಾಡಲು ಪ್ರಯತ್ನಿಸುವುದು, ಶ‍್ರೀಕೃಷ್ಣನು ಶಿಶುವನ್ನು ರಕ್ಷಿಸಿ ಅಶ್ವತ್ಥಾಮರಿಗೆ ಶಾಪ ಕೊಡುವುದು, ನಂತರ ಪಾಂಡವರು ಹಸ್ತಿನಾವತೀಪುರಕ್ಕೆ ಬರುವುದು, ಧೃತರಾಷ್ಟ್ರನಿಗೆ ಪಾಂಡವರು ನಮಸ್ಕರಿಸುವಾಗ ಭೀಮಸೇನನ ಪ್ರತಿಮೆಯನ್ನು ಮುಂದಿಟ್ಟು ಅದು ಪುಡಿಯಾಗಲು ಧೃತರಾಷ್ಟ್ರನ ಮನಸ್ಸನ್ನು ವ್ಯಕ್ತಪಡಿಸುವುದು, ಗಾಂಧಾರಿಯ ಶೋಕ, ದುರ್ಯೋಧನನ ಸಂಹಾರವು ಅಧರ್ಮವಲ್ಲವೆಂದು ಗಾಂಧಾರಿಗೆ ತಿಳಿಸುವುದು, ಭೀಮಸೇನನು ದುಶ್ಯಾಸನನ ರಕ್ತಪಾನ ಮಾಡಿದ ಬಗ್ಗೆ ಇದ್ದ ಸಂಶಯ ನಿವಾರಣೆ, ಗಾಂಧಾರಿ ಮತ್ತು ಅವಳ ಸೊಸೆಯರು ರಣರಂಗಕ್ಕೆ ಶವಗಳನ್ನು ನೋಡಲು ಬರುವುದು, ಗಾಂಧಾರಿಗೆ ಶ‍್ರೀಕೃಷ್ಣನು ದಿವ್ಯ ದೃಷ್ಟಿಯನ್ನು ಕೊಡುವುದು, ಗಾಂಧಾರಿಯ ದುಃಖ, ಅವಳು ಶ‍್ರೀಕೃಷ್ಣನಿಗೆ ಶಾಪ ಕೊಡುವುದು, ಇದರಿಂದ ಗಾಂಧಾರಿಯ ತಪಸ್ಸು ನಾಶವಾಗುವುದು, ದ್ರೌಪದಿಗೆ ದುರ್ಯೊಧನಾದಿಗಳ ಶವಗಳನ್ನು ಶ‍್ರೀಕೃಷ್ಣನು ತೋರಿಸುವುದು, ಶವಗಳ ಸಂಸ್ಕಾರ, ಕರ್ಣನು ತನ್ನ ಅಗ್ರಜನೆಂದು ತಿಳಿದ ಧರ್ಮರಾಜನು ಕುಂತೀನಿಮಿತ್ತದಿಂದ ಸ್ತ್ರೀಯರಿಗೆಲ್ಲಾ ಶಾಪ ಕೊಡುವುದು, ನಾರದ-ಶ‍್ರೀಕೃಷ್ಣರು ಧರ್ಮರಾಜನನ್ನು ಸಮಾಧಾನಪಡಿಸುವುದು ಈ ವಿಷಯಗಳು ಈ ಅಧ್ಯಾಯದಲ್ಲಿ ನಿರೂಪಿತವಾಗಿವೆ.


\section*{ಅಧ್ಯಾಯ\enginline{-}೨೯}

\begin{verse}
\textbf{ಕೃಷ್ಣಾಭ್ಯಾಮಪಿ ಭೂಸುರೈಃ ನೃಪಸುತೋ ರಾಜ್ಯೇsಭಿಷಿಕ್ತೋ ದ್ವಿಜೈಃ}\\\textbf{ದಗ್ಧೇ ನಿಂದಿತಭಿಕ್ಷುಕೇ ಖಲತರೇ ಸ್ವಂ ವಿಪ್ರತೀಸಾರತಃ~।}\\\textbf{ರಾಜ್ಯಂ ತ್ಯಕ್ತುಮಥೋದ್ಯತೋ ವಚನತೋ ಯಸ್ಯಾಪ್ತಭೀಷ್ಮಾತ್ತತಃ }\\\textbf{ಶುಶ್ರಾವಾಖಿಲಧರ್ಮನಿರ್ಣಯಮದಃ ಕೃಷ್ಣ ದ್ವಯಂ ಧೀಮಹಿ~।।}
\end{verse}

ಶ‍್ರೀ ಕೃಷ್ಣ ಮತ್ತು ವೇದವ್ಯಾಸ ರೂಪಗಳಿಂದ ಧರ್ಮರಾಜನಿಗೆ ಬ್ರಾಹ್ಮಣರ ಸಹಕಾರ\-ದೊಂದಿಗೆ ರಾಜ್ಯಾಭಿಷೇಕವನ್ನು ನೆರವೇರಿಸಿದ, ಧರ್ಮಜನನ್ನು ನಿಂದಿಸಿದ ಯತಿವೇಷದಲ್ಲಿದ್ದ ಅಸುರನನ್ನು ಬ್ರಾಹ್ಮಣೋತ್ತಮರ ಶಾಪದಿಂದ ಸಂಹರಿಸಿದ, ಅಧರ್ಮದ ಶಂಕೆಯಿಂದ ರಾಜ್ಯವನ್ನು ಪರಿತ್ಯಾಗಮಾಡಲು ಸಿದ್ಧನಾದ ಧರ್ಮರಾಜನಿಗೆ ಅವನಿಗೆ ಅತ್ಯಂತ ಆಪ್ತರಾದ\break ಭೀಷ್ಮಾಚಾರ್ಯರಿಂದ ಅನೇಕ ಧರ್ಮಗಳನ್ನು ಉಪದೇಶ ಮಾಡಿಸಿದ ಶ‍್ರೀಕೃಷ್ಣನ ಎರಡು ರೂಪಗಳನ್ನು ಧ್ಯಾನಿಸುತ್ತೇನೆ.

ಈ ಅಧ್ಯಾಯದಲ್ಲಿ ೬೩ ಶ್ಲೋಕಗಳಿವೆ.

ಶ‍್ರೀಕೃಷ್ಣ ವೇದವ್ಯಾಸರು ಬ್ರಾಹ್ಮಣರ ಸಮೂಹದಿಂದ ಯುಕ್ತರಾಗಿ ಧರ್ಮ ರಾಜನನ್ನು ಸಾಮ್ರಾಜ್ಯದಲ್ಲಿಯೂ, ಭೀಮಸೇನನನ್ನು ಯುವರಾಜಪದವಿಯಲ್ಲಿಯೂ ಅಭಿಷೇಕ ಮಾಡಿದುದು, ಚಾರ್ವಾಕನೆಂಬ ದುಷ್ಟನು ಯತಿಯ ವೇಷದಲ್ಲಿ ಬಂದು ಧರ್ಮರಾಜನನ್ನು ನಿಂದಿಸಿದಾಗ ಬ್ರಾಹ್ಮಣರು ಶಾಪದಿಂದ ಅವನನ್ನು ಭಸ್ಮೀಭೂತನನ್ನಾಗಿ ಮಾಡುವುದು, ದ್ರೋಣ-\-ಕರ್ಣ-ದುರ್ಯೊಧನಾದಿಗಳ ಹನನದಿಂದ ತನಗೆ ಪಾಪವು ಉಂಟಾಗಿದೆ ಎಂಬ ಶಂಕೆಯಿಂದ ಧರ್ಮರಾಜನು ರಾಜ್ಯವನ್ನು ತ್ಯಜಿಸಲು ಮನಸ್ಸು ಮಾಡುವುದು, ದ್ರೌಪದಿ - ಭೀಮಾರ್ಜುನರು - ಶ‍್ರೀಕೃಷ್ಣ - ವೇದವ್ಯಾಸರು-ವಿಪ್ರೋತ್ತಮರು ಎಲ್ಲರೂ ಎಷ್ಟು ಬಗೆಯಾಗಿ ವಿವರಿಸಿದರೂ ಧರ್ಮರಾಜನ ಶಂಕೆಯು ನಿವೃತ್ತಿಯಾಗದೇ ಇರುವುದು, ಆಗ ಭೀಷ್ಮಾಚಾರ್ಯರನ್ನು ಕೇಳು ಎಂಬುದಾಗಿ ಧರ್ಮರಾಜನಿಗೆ ಸಲಹೆ ಕೊಡುವುದು, ಧರ್ಮರಾಜನು ಭೀಷ್ಮರನ್ನು ಪ್ರಶ್ನೆ ಮಾಡಲಾಗಿ ಶ‍್ರೀಕೃಷ್ಣನೇ ಭೀಷ್ಮರಲ್ಲಿ ನಿಂತು ಧರ್ಮೋಪದೇಶ ಮಾಡುವುದು, ವಿಷ್ಣು ಸರ್ವೋತ್ತಮತ್ವ ಜ್ಞಾನ, ರಮಾದಿ ಇತರ ದೇವತೆಗಳ ತಾರತಮ್ಯ ಜ್ಞಾನ, ವಿಷ್ಣು ಭಕ್ತರ ಬಗ್ಗೆ ಗೌರವ, ಬ್ರಾಹ್ಮಣಾದಿ ವರ್ಣದವರಿಗೆ ದಂಡನೆ ಮಾಡುವ ಕ್ರಮ, ಆಪತ್ಕಾಲದಲ್ಲಿ ಕ್ಷತ್ರಿಯರ ಧರ್ಮ, ಸಾಮಾದಿಗಳ ಉಪಯೋಗ, ವೈಶ್ಯರ ಧರ್ಮಗಳು, ಹರಿಭಕ್ತನು ಹೀನಜಾತಿಯವನಾದರೂ - ಪ್ರೀತಿಸಲ್ಪಡತಕ್ಕವನು, ತತ್ವೋಪದೇಶಮಾಡಲು ಅಧಿಕಾರಿಗಳು, ಸ್ತ್ರೀಯರ ಧರ್ಮ, ಋಷ್ಯಾದಿ ಸ್ತ್ರೀಸಂಗವನ್ನು ಇಚ್ಛಿ ಸಿದರೂ ಎಂತಹ ಪಾಪ ಎಂಬ ವಿವರಣೆ, ಸ್ತ್ರೀಯರಿಗೂ ಸಹ ಇದೇರೀತಿಯ ನಿಯಮ, ಉತ್ತಮ ಸ್ತ್ರೀಯರಿಗೆ ವೇದಾಧಿಕಾರ, ಸಾಮಾನ್ಯ ಸ್ತ್ರೀಯರಿಗೆ ಭಾರತಾದಿ ಗ್ರಂಥಗಳ ಶ್ರವಣ, ಶ‍್ರೀಹರಿನಾಮೋಚ್ಛಾರಣೆ, ರಾಮಕೃಷ್ಣಾದಿ ಅವತಾರಗಳಿಗೂ ಮೂಲರೂಪಕ್ಕೂ ಯಾವಾಗಲೂ ಅಭೇದವೇ, ಅವತಾರಗಳಲ್ಲಿಯೂ ಪರಸ್ಪರವಾಗಿ ಅಭೇದ, ಶ‍್ರೀಹರಿಯೊಬ್ಬನೇ ಸ್ವತಂತ್ರ ಅನ್ಯರಲ್ಲ, ಮುಕ್ತಿಗೆ ಮಾರ್ಗ, ಮುಂತಾದ ತತ್ವಗಳ ಉಪದೇಶ, ಧರ್ಮ-ಅರ್ಥಕಾಮ ಇವುಗಳಲ್ಲಿ ಯಾವುದು ಶ್ರೇಷ್ಠ ಎಂಬ ಬಗ್ಗೆ ನಿರೂಪಣೆ, ಭೀಮಸೇನನಿಂದ - “ಕಾಮ” ಶಬ್ದಕ್ಕೆ ಅರ್ಥ, ಕಾಮಕ್ಕೆ ಶ್ರೇಷ್ಠತೆಯ ಸಮರ್ಥನೆ, ಶ‍್ರೀಹರಿಯೇ ಸಕಲ ಸಜ್ಜನರಿಂದ ಕಾವ್ಯ, ಇಂತಹ ಉಪದೇಶದಿಂದ ಧರ್ಮರಾಜನಿಗೆ ಸಮಾಧಾನ, ಈ ವಿಷಯಗಳು ಈ ಅಧ್ಯಾಯದಲ್ಲಿ ನಿರೂಪಿತವಾಗಿವೆ.

\vspace{-.2cm}

\section*{ಅಧ್ಯಾಯ\enginline{-}೩೦}

\begin{verse}
\textbf{ಸ್ಮೃತ್ವಾ ಯಂ ದ್ಯುಸರಿತ್ಸುತೋ ವಸುರಭೂದ್ರಾಜಾ ಯದಾಶಾಸಿತೋ}\\\textbf{ನಿರ್ದುಃಖೋsಥ ಜುಗೋಪ ಧರ್ಮನಿರತೋ ಜಿತ್ವಾ ಸ್ವರಾಜ್ಯೇ ಕಲಿಮ್~।}\\\textbf{ಯಃ ಪಾರ್ಥಂ ಸಮಬೋಧಯನ್ಮೃತಶಿಶುಂ ಯೋsಜೀವಯತ್ಪಾಂಡವೈಃ}\\\textbf{ಯೋ ಯಜ್ಞಂ ಸಮಕಾರಯತ್‌ ಬಹುಧನೈಃ ಧ್ಯಾಯಾಮಿ ತಂ ಕೇಶವಮ್~।।}
\end{verse}

ಯಾವ ಶ‍್ರೀಕೃಷ್ಣನ ಅನುಗ್ರಹದಿಂದ ಗಂಗೆಯ ಮಗನಾದ ಭೀಷ್ಮರು ಶ‍್ರೀಕೃಷ್ಣನನ್ನು ಸ್ತುತಿಸುತ್ತಾ ವಸುಪದವಿಯನ್ನು ಪುನಃ ಪಡೆದರೋ, ಯಾರ ಕೃಪಾಬಲದಿಂದ ಧರ್ಮರಾಜನು ಧರ್ಮದಿಂದಲೂ ಸುಖ-ಸಂತೋಷದಿಂದಲೂ ರಾಜ್ಯವನ್ನು ಪರಿಪಾಲಿಸುತ್ತಾ ತನ್ನ ರಾಜ್ಯಕ್ಕೆ ಬಂದ ಕಲಿಯನ್ನು ನಿಗ್ರಹಿಸಿದನೋ ಯಾವ ಶ‍್ರೀಕೃಷ್ಣನು ಅರ್ಜುನನಿಗೆ ಪುನಃ ಗೀತೋಪದೇಶವನ್ನು ಮಾಡಿದನೋ, ಬ್ರಾಹ್ಮಣನಿಗೆ ಮೃತ ಶಿಶುವನ್ನು ಬದುಕಿಸಿ ತಂದಿತ್ತನೋ, ಯಾವ ಶ‍್ರೀಹರಿಯು ಪಾಂಡವರಿಂದ ವಿಪುಲಧನವೆಚ್ಚದಿಂದ ಶಾಸ್ತ್ರೋಕ್ತ ಪ್ರಕಾರವಾಗಿ ಯಜ್ಞವನ್ನು ಮಾಡಿಸಿದನೋ ಅಂತಹ ಕೇಶವನನ್ನು ಧ್ಯಾನಿಸುತ್ತೇನೆ.

ಈ ಅಧ್ಯಾಯದಲ್ಲಿ ೧೭೯ ಶ್ಲೋಕಗಳಿವೆ.

ಧರ್ಮರಾಜನಿಗೆ ಧರ್ಮೋಪದೇಶ ಮಾಡಿದ ನಂತರ ಭೀಷ್ಮರು ದೇಹತ್ಯಾಗ ಮಾಡಿ ವಸುಪದವಿಗೆ ಪುನಃ ಹೋಗುವುದು, ಧರ್ಮರಾಜನು ಅವರಿಗಾಗಿ ಉತ್ತರ ಕ್ರಿಯಾದಿಗಳನ್ನು ನೆರವೇರಿಸುವುದು, ಪಾಪರಹಿತವಾದ ಶತ್ರುನಾಶ ಕಾರ್ಯದಲ್ಲಿ ”ಪಾಪ” ಎಂಬ ಸಂಶಯದಿಂದ ಧರ್ಮರಾಜನು ದುಃಖಿತನಾಗುವುದು, ಶ‍್ರೀಕೃಷ್ಣನು ಅಶ್ವಮೇಧ ಯಜ್ಞವನ್ನು ಮಾಡಿ ಸಂಶಯ ನಿವಾರಣೆ ಮಾಡಿಕೋ ಎಂದು ಸೂಚಿಸುವುದು, ಧರ್ಮರಾಜನು ಅತ್ಯಂತ ಧರ್ಮದಿಂದ ರಾಜ್ಯವನ್ನು ಪರಿಪಾಲಿಸುವುದು, ಸ್ತ್ರೀಭೋಗಾದಿಗಳನ್ನು ತ್ಯಜಿಸಿ ಗೋವ್ರತಾದಿ\-ಗಳಿಂದ ಜೀವನವನ್ನು ನಡೆಸುವುದು, ನಿತ್ಯದಲ್ಲಿಯೂ ಸತ್ಪಾತ್ರರಲ್ಲಿ ದಾನ ಮಾಡುವುದು, ಭೀಮಸೇನನು ದ್ರೌಪದೀದೇವಿಯರೊಡನೆ ಕ್ರೀಡಿಸುತ್ತಾ ದುರ್ಯೊಧನನ ಅರಮನೆಯಲ್ಲಿರುವುದು, ದೌಪದೀದೇವಿಯು ಭೀಮಸೇನನನ್ನು ಸೇವಿಸುವುದು, ಭೀಮಸೇನನು ತನ್ನ ಇತರ ಇಪ್ಪತ್ತೊಂದು ಪತ್ನಿಯರೊಡನೆ ಭೋಗಿಸುವುದು, ಭೀಮಸೇನನು ಪ್ರತಿ ಗ್ರಾಮದಲ್ಲಿಯೂ ಭಾಗವತಧರ್ಮದಲ್ಲಿಯೇ ಆಸಕ್ತರಾದ ಐದು ಜನ ಬ್ರಾಹ್ಮಣರನ್ನು ಧರ್ಮಪ್ರಚಾರಕ್ಕಾಗಿ ನಿಯಮಿಸುವುದು, ರಾಜ್ಯದಲ್ಲಿ ಎಲ್ಲಿಯೂ ವಿಷ್ಣು ಭಕ್ತಿರಹಿತರಾದವರು ಇಲ್ಲದೇಹೋದುದು, ವರ್ಣಾಶ್ರಮಧರ್ಮಗಳಲ್ಲಿಯೇ ಪ್ರಜೆಗಳು ಆಸಕ್ತರಾದುದು, ಧರ್ಮಕಾರ್ಯಗಳಿಂದ ಯುಗ ಯುಗಗಳಲ್ಲಿ ಫಲದ ತಾರತಮ್ಯ, ಅರ್ಜುನನು ಆಶ್ರಿತರಾಜರಿಂದ ನಿರ್ದಿಷ್ಟವಾದ ಕಪ್ಪಗಳನ್ನು ಪಡೆಯುತ್ತಿದ್ದುದು, ಸುಭದ್ರಾ-ಚಿತ್ರಾಂಗದೆಯರೊಡನೆ ದುಶ್ಯಾಸನನ ಅರಮನೆಯಲ್ಲಿ ಅರ್ಜುನನು ಕ್ರೀಡಿಸಿದುದು, ನಕುಲನು ಭೃತ್ಯರಿಗೆ ನಿಯತವಾದ ವೇತನಗಳನ್ನು ಕೊಡುತ್ತಾ ದುರ್ಮುಖನಾಮಕ ಧೃತರಾಷ್ಟ್ರನ ಮಗನ ಅರಮನೆಯಲ್ಲಿ ಶಲ್ಯನ ಮಗಳಿಂದ ಕೂಡಿಕೊಂಡು ಇರುತ್ತಿದುದು, ಸಹದೇವನು ಜರಾಸಂಧನ ಮಗಳೊಡನೆ ದುರ್ಮರ್ಷಣನಾಮಕ ಧೃತರಾಷ್ಟ್ರನ ಮಗನ ಅರಮನೆಯಲ್ಲಿ ರಾಜನೀತಿ ಕುಶಲನಾಗಿ ವಾಸಮಾಡುತ್ತಿದುದು, ಕೃಪಾಚಾರ್ಯರು ಸೇನಾಧಿಪತಿಯಾದುದು, ಎಲ್ಲರೂ ಧೃತರಾಷ್ಟ್ರ-ಗಾಂಧಾರಿ\-ಯನ್ನು ಸೇವೆ ಮಾಡುತ್ತಿದುದು, ರಾಜ್ಯದಲ್ಲಿ ಸುಭಿಕ್ಷೆ, ಗೋವುಗಳು ಯಥೇಷ್ಟ ಹಾಲು ಕೊಡುತ್ತಿದುದು, ಪಾಂಡವರ ಕೀರ್ತಿಯು ದೇವಲೋಕಗಳಲ್ಲಿಯೂ ಹರಡಿದುದು, ಬಲಿನಾಮಕ ದೈತ್ಯನ ನಿಗ್ರಹ, ಕಲಿಗೂ ಧರ್ಮರಾಜನಿಗೂ ಸಂಭಾಷಣೆ, ಕಲಿಗೆ ಧರ್ಮರಾಜನ ಎಚ್ಚರಿಕೆ, ಕಲಿಯ ನಿರ್ಗಮನ, ಶ‍್ರೀಕೃಷ್ಣನು ಅರ್ಜುನನಿಗೆ ಪುನಃ ತತ್ವೋಪದೇಶ ಮಾಡುವುದು, ಹರಿಸರ್ವೋತ್ತಮತ್ವ ವಾಯುಜೀವೋತ್ತಮತ್ವಗಳನ್ನು ಉಪದೇಶಿಸುವುದು, ಶ‍್ರೀಹರಿಯ ಸ್ವಾತಂತ್ರ್ಯ, ಅನ್ಯರೆಲ್ಲರ ಪಾರತಂತ್ರ್ಯ, ಮುಕ್ತಿಯೋಗ್ಯನ-ತಮೋಯೋಗ್ಯನ ಲಕ್ಷಣಗಳು, ಉದಂಕ ಋಷಿಗಳಿಗೆ ಶ‍್ರೀಕೃಷ್ಣನ ಉಪದೇಶ, ಶ‍್ರೀಕೃಷ್ಣನು ದ್ವಾರಕಿಯಲ್ಲಿ ವಾಸಮಾಡುವುದು, ಧರ್ಮರಾಜನು ಅಶ್ವಮೇಧ ಯಜ್ಞವನ್ನು ಮಾಡಲು ನಿಶ್ಚಯಿಸಿ ದ್ರವ್ಯಕ್ಕಾಗಿ ಶ‍್ರೀಕೃಷ್ಣನನ್ನು ಯಾಚಿಸುವುದು, ಶ‍್ರೀಕೃಷ್ಣನ ಸಲಹೆಯಂತೆ ಪಾಂಡವರು ಪರಶುರಾಮದೇವರಿಂದ ವಿಪುಲವಾದ ದ್ರವ್ಯವನ್ನು ಪಡೆಯುವುದು, ಹಸ್ತಿನಾವತೀ ಪಟ್ಟಣಕ್ಕೆ ಶ‍್ರೀಕೃಷ್ಣನ ಆಗಮನ, ದಾರಿಯಲ್ಲಿ ಉದಂಕ ಋಷಿಗಳಿಗೆ ಅನುಗ್ರಹ, ಉತ್ತರಾದೇವಿಯು ಮೃತಶಿಶುವನ್ನು ಪ್ರಸವಿಸುವುದು, ಶ‍್ರೀಕೃಷ್ಣನು ಆ ಮಗುವನ್ನು (ಪರೀಕ್ಷಿತನನ್ನು) ಬದುಕಿಸುವುದು, ಪಾಂಡವರ ಸಂತೋಷ, ವೇದವ್ಯಾಸದೇವರ ನೇತೃತ್ವದಲ್ಲಿ ಧರ್ಮರಾಜನು ಅಶ್ವಮೇಧ ಯಜ್ಞವನ್ನು ಮಾಡುವುದು, ಅರ್ಜುನನು ಅಶ್ವದ ಹಿಂದೆ ಹೋಗುವುದು, ಬಭ್ರುವಾಹನ-ಅರ್ಜುನರ ಯುದ್ಧ, ಅರ್ಜುನನ ಮೂರ್ಛೆ, ಉಲೂಪಿಯಿಂದ ಅರ್ಜುನನು ಮೂರ್ಛೆಯಿಂದ ಏಳುವುದು, ಅರ್ಜುನನು ಮೂರ್ಛೆ ಹೊಂದಲು ಕಾರಣ, ಅರ್ಜುನನ ದಿಗ್ವಿಜಯ, ಧರ್ಮರಾಜನ ಸಂತೋಷ, ಪುರುಷರಲ್ಲಿ ಇರಬಹುದಾದ ದುರ್ಲಕ್ಷಣಗಳು, ಭೀಮಸೇನ-ದೌಪದೀದೇವಿ ಮಾತ್ರರೇ ದುರ್ಲಕ್ಷಣ ರಹಿತರು, ಯಜ್ಞಶಾಲೆಯಲ್ಲಿ ವೇದಾಂತ ಚರ್ಚೆ, ಯಥೇಷ್ಟದಾನ, ಭೋಜನ ವ್ಯವಸ್ಥೆ, ಯಜ್ಞಶಾಲೆಯ ವರ್ಣನೆ, ಹದಿನೈದು ವರ್ಷಗಳ ಕಾಲ ನಡೆದ ಯಜ್ಞದ ವೈಭವ, ಪಾಂಡವರಿಂದ ರಾಜ್ಯದಾನ, ಶ‍್ರೀಕೃಷ್ಣನ ಆಜ್ಞೆಯಂತೆ ಪುನಃ ಸ್ವೀಕಾರ, ಅವಭೃತ ಸ್ನಾನ, ಕ್ರೋಧನಾಮಕ ದೈತ್ಯನು ಮುಂಗುಸಿ ರೂಪದಿಂದ ಬಂದು ನಿಂದಿಸುವುದು, ಇವನು ಮುಂಗುಸಿ ರೂಪವನ್ನು ಪಡೆಯಲು ಕಾರಣ, ಕೊನೆಗೆ ಇವನು ಅಂಧಂತಮಸ್ಸನ್ನು ಕುರಿತು ಹೋಗುವುದು, ಜ್ಞಾನಯುಕ್ತನಾಗಿ ಮಾಡಿದ ಸತ್ಕರ್ಮಕ್ಕೆ ಫಲಾಧಿಕ್ಯವೆಂಬ ಪ್ರಮೇಯ ವರ್ಣನೆ, ಅವೈಷ್ಣವರಿಂದ ಮಾಡಲ್ಪಟ್ಟ ಸತ್ಕರ್ಮಕ್ಕೆ ಫಲವಿಲ್ಲವೆಂಬ ತತ್ವ; ದೇವತೆಗಳು, ಗಂಧರ್ವರು, ಮನುಷ್ಯರು ಇವರು ಮಾಡತಕ್ಕ ಕರ್ಮಗಳಲ್ಲಿ ಫಲತಾರತಮ್ಯ, ವಿಷ್ಣು ಪ್ರೀತಿ ಇದ್ದರೆ ಮಾತ್ರವೇ ಫಲ, ಶ‍್ರೀಕೃಷ್ಣನು ವರ್ಣಾಶ್ರಮ ಧರ್ಮಗಳನ್ನು ಉಪದೇಶ ಮಾಡಿದುದು, ಪಾಂಡವರ ಸಂತೋಷ, ಯಜ್ಞ ಸಮಾಪ್ತಿ ಈ ವಿಷಯಗಳು ಈ ಅಧ್ಯಾಯದಲ್ಲಿ ನಿರೂಪಿಸಲ್ಪಟ್ಟಿವೆ.

\vspace{-.2cm}

\section*{ಅಧ್ಯಾಯ\enginline{-}೩೧}

\begin{verse}
\textbf{ಯದ್ಯುಕ್ತಾಃ ಪಾಂಡುಪುತ್ರಾಃ ಕ್ಷಿತಿಮಥ ಜುಗುಪುಃ ಧರ್ಮರಾಜಸ್ತ್ವರಾವಾನ್} \\\textbf{ಧರ್ಮೇ ಯತ್ಪ್ರೀತಯೇಽಭೂತ್ಪವನಜವಚನೈಃ ಆಂಭಿಕೇಯಂ ವಿರಕ್ತಮ್~।}\\\textbf{ವ್ಯಾಸಾತ್ಮಾ ಯೋ ವನಸ್ಥಂ ತ್ವಕೃತನಿಜಮನೋಽಭೀಷ್ಟವಂತಂ ಗತಂ ಸ್ವಂ}\\\textbf{ನಾಥಂ ಪಾರ್ಥಾಃ ಸ್ಮರಂತೋ ಮುಮುದುರಪಿಪದಂ ಯಸ್ಯ ಕೃಷ್ಣಂ ತಮೀಡೇ~।}
\end{verse}

ಯಾವ ಶ‍್ರೀಕೃಷ್ಣನ ಅನುಗ್ರಹ ಬಲದಿಂದ ಪಾಂಡವರು ಧರ್ಮದಿಂದ ರಾಜ್ಯವನ್ನು ಆಳಿದರೋ, ಯಾವ ಶ‍್ರೀಕೃಷ್ಣನ ಪ್ರೀತಿಗಾಗಿ ಧರ್ಮರಾಜನು ಅನೇಕ ಧರ್ಮಕಾರ್ಯಗಳನ್ನು ನೆರವೇರಿಸಿದನೋ, ಯಾವ ಶ‍್ರೀಹರಿಯ ಇಚ್ಛೆಯಂತೆ ಭೀಮಸೇನನು ಧೃತರಾಷ್ಟ್ರನಿಗೆ ವೈರಾಗ್ಯವು ಉತ್ಪತ್ತಿಯಾಗುವಂತೆ ಮಾಡಿ ಅವನನ್ನು ಅರಣ್ಯಕ್ಕೆ ಕಳಿಸಿದನೋ, ಅರಣ್ಯದಲ್ಲಿದ್ದ ಧೃತರಾಷ್ಟ್ರನಿಗೆ ಯಾವ ಶ‍್ರೀಕೃಷ್ಣನು ಅನುಗ್ರಹಿಸಿ ಗಂಧರ್ವ ಪದವಿಯನ್ನು ದಯಪಾಲಿಸಿದನೋ, ಯಾವ ಶ‍್ರೀಕೃಷ್ಣನನ್ನು ಪಾಂಡವರು ಪರಮ ಭಕ್ತಿಯಿಂದ ಸ್ತುತಿಸಿ ಪೂಜಿಸಿ ಸಂತೋಷ ಪಡೆದರೋ ಅಂತಹ ಶ‍್ರೀಕೃಷ ನನ್ನು ಸ್ತೋತ್ರ ಮಾಡುತ್ತೇನೆ.

ಈ ಅಧ್ಯಾಯದಲ್ಲಿ ೭೭ ಶ್ಲೋಕಗಳಿವೆ.

ಧರ್ಮದಿಂದ ಪಾಂಡವರು ರಾಜ್ಯವಾಳುತ್ತಿರುವಾಗ ಒಂದು ರಾತ್ರಿ ಒಬ್ಬ ಬ್ರಾಹ್ಮಣನು ಯಾಗಮಾಡಲು ಯುಧಿಷ್ಠಿರನ ಬಳಿ ಬಂದು ದ್ರವ್ಯವನ್ನು ಯಾಚನೆ ಮಾಡುವುದು, ಧರ್ಮರಾಜನು ಬೆಳಿಗ್ಗೆ ಕೊಡುತ್ತೇನೆಂದು ಹೇಳುವುದು, ಬ್ರಾಹ್ಮಣನು ಕೂಡಲೇ ಭೀಮಸೇನನನ್ನು ಬೇಡುವುದು, ಭೀಮಸೇನನು ತನ್ನ ಕೈಯಲ್ಲಿದ್ದ ಸುವರ್ಣ ಕಂಕಣವನ್ನು ನೀಡಿ ಉತ್ಸವ ಭೇರಿ ತಾಡನ ಮಾಡಿಸುವುದು, ಶಬ್ದವನ್ನು ಕೇಳಿದ ಧರ್ಮರಾಜನು ಕಾರಣವನ್ನು ತಿಳಿಯಲು ದೂತನನ್ನು ಕಳಿಸುವುದು, ಅನಿತ್ಯವಾದ ದೇಹ ಉಳ್ಳವನಾದಾಗ್ಯೂ ಧರ್ಮರಾಜನು ನಿಶ್ಚಿತವಾದ ಆಯುಸ್ಸನ್ನು ಪಡೆದಿರುತ್ತೇನೆಂಬ ಅಭಿಪ್ರಾಯ ಹೊಂದಿರುವುದೇ ಭೇರಿ ತಾಡನಕ್ಕೆ ಕಾರಣವೆಂದು ಭೀಮಸೇನನು ತಿಳಿಸುವುದು, ಇದರಿಂದ ಧರ್ಮರಾಜನಿಗೆ ಧರ್ಮದ ವಿಷಯದಲ್ಲಿ ವಿಶೇಷವಾದ ಆಸಕ್ತಿ ಉತ್ಪನ್ನವಾಗುವುದು, ಧೃತರಾಷ್ಟ್ರನಿಗೆ ವೈರಾಗ್ಯ ಬರುವಂತೆ ಮಾಡಲು ಭೀಮಸೇನನು ತಿರಸ್ಕಾರ ಮಾತುಗಳನ್ನಾಡುವುದು, ವಿದುರನು ಭೀಮಸೇನನ ಇಂಗಿತವನ್ನು ತಿಳಿದು ಧೃತರಾಷ್ಟ್ರನಿಗೆ ಹಿತ ವಚನಗಳನ್ನಾಡಿ ಅವನಲ್ಲಿ ವಿರಕ್ತಿ ಹುಟ್ಟುವಂತೆ ಮಾಡುವುದು, ಧೃತರಾಷ್ಟ್ರನು ಉಪವಾಸ ಮಾಡುವುದು, ನಂತರ ಅರಣ್ಯಕ್ಕೆ ಹೋಗಲು ನಿರ್ಧರಿಸುವುದು, ಧೃತರಾಷ್ಟ್ರನು ತನ್ನ ಮಕ್ಕಳ ಶ್ರಾದ್ಧ ಮಾಡಲು ಧರ್ಮರಾಜನಿಂದ ದ್ರವ್ಯವನ್ನು ಪಡೆಯಲು ವಿದುರನನ್ನು ಕಳಿಸುವುದು, ಭೀಮಸೇನನು ವಿರೋಧಿಸುವುದು, ಆದರೂ ಧರ್ಮರಾಜನು ವಿದುರನಿಗೆ ದ್ರವ್ಯವನ್ನು ನೀಡುವುದು, ಧೃತರಾಷ್ಟ್ರನು ಶ್ರಾದ್ಧ ದಾನಾದಿಗಳನ್ನು ಮಾಡಿ ಪ್ರಜೆಗಳಿಂದ ಕ್ಷಮೆಯನ್ನು ಯಾಚಿಸುವುದು, ಪ್ರಜೆಗಳ ಮುಂದೆ ತನ್ನ ಮಕ್ಕಳ ತಪ್ಪುಗಳನ್ನು ಒಪ್ಪಿಕೊಳ್ಳುವುದು, ಸಂಜಯ ವಿದುರ ಗಾಂಧಾರಿಯಿಂದ ಯುಕ್ತನಾಗಿ ಅರಣ್ಯಕ್ಕೆ ತೀರ್ಥ ಯಾತ್ರಾಕ್ರಮದಿಂದ ತೆರಳುವುದು, ಬದರಿಕಾಶ್ರಮದಲ್ಲಿ ನಾರದರು ಬಂದು ಇನ್ನು ಮೂರು ಸಂವತ್ಸರಗಳಲ್ಲಿ ಗಂಧರ್ವ ಪದವಿ ಪ್ರಾಪ್ತವಾಗುತ್ತದೆಂದು ತಿಳಿಸುವುದು, ಧೃತರಾಷ್ಟ್ರನು ಉಗ್ರವಾದ ತಪಸ್ಸನ್ನು ಆಚರಿಸುವುದು, ವಿದುರನು ಯುಧಿಷ್ಠಿರನಲ್ಲಿ ಐಕ್ಯವನ್ನು ಹೊಂದುವುದು, ವೇದವ್ಯಾಸರು ಬರುವುದು, ಆಗ ಪಾಂಡವರೂ ಸಹ ಧೃತರಾಷ್ಟ್ರನನ್ನು ನೋಡಲು ಬಂದಿರುವುದು, ವೇದವ್ಯಾಸರು ಎಲ್ಲರ ಇಷ್ಟಾರ್ಥಗಳನ್ನೂ ಪೂರೈಸುವುದಾಗಿ ವಚನ ಕೊಡುವುದು, ಪಾಂಡವರು ಭಕ್ತಿ ಅಭಿವೃದ್ಧಿಯನ್ನು ಬೇಡುವುದು, ಕುಂತಿದೇವಿಯು ದೋಷಪರಿಹಾರವನ್ನು ಬೇಡುವುದು, ಧೃತರಾಷ್ಟ್ರನು ಮೃತರಾದ ತನ್ನ ಮಕ್ಕಳನ್ನು ನೋಡಬೇಕೆಂದು ಕೇಳುವುದು, ವೇದವ್ಯಾಸರು ಅವನಿಗೆ ದಿವ್ಯ ದೃಷ್ಟಿಯನ್ನು ದಯಪಾಲಿಸಿ ಪುತ್ರಾದಿಗಳನ್ನು ತೋರಿಸುವುದು, ಅವರೆಲ್ಲರೂ ಆ ರಾತ್ರಿ ಬದರಿಕಾಶ್ರಮದಲ್ಲಿಯೇ ವಾಸಮಾಡುವುದು, ಧೃತರಾಷ್ಟ್ರನ ತೃಪ್ತಿ, ಭಾರತಕಥೆಯನ್ನು ಶ್ರವಣ ಮಾಡುತ್ತಿದ್ದ ಜನಮೇಜಯನಿಗೆ ಇದರಿಂದ ಆಶ್ಚರ್ಯವಾಗುವುದು, ತಾನೂ ಸಹ ತಂದೆಯಾದ ಪರೀಕ್ಷಿತರಾಜನ ದರ್ಶನವನ್ನು ಇಚ್ಚಿಸುವುದು, ವೇದವ್ಯಾಸರು ಅವನ ಇಚ್ಛೆಯನ್ನು ಪೂರ್ಣಗೊಳಿಸುವುದು, ನಂತರ ಜನಮೇಜಯನಿಗೆ ಭಾರತಕಥೆಯಲ್ಲಿ ಆದರವು ಅಭಿವೃದ್ಧಿಯಾಗುವುದು, ಪಾಂಡವರು ಹಸ್ತಿನಾವತೀ ಪುರಕ್ಕೆ ವಾಪಸ್ಸು ಬರುವುದು, ಮೂರು ವರ್ಷಗಳಲ್ಲಿ ಅಗ್ನಿ ಸ್ಪರ್ಶದಿಂದ ಧೃತರಾಷ್ಟ್ರ, ಗಾಂಧಾರಿ, ಕುಂತಿ ಇವರ ಮರಣ ವಾರ್ತೆಯನ್ನು ಸಂಜಯನಿಂದ ಕೇಳುವುದು, ಅವರಿಗಾಗಿ ಪಾಂಡವರು ಶ್ರಾದ್ಧಾದಿ ಕರ್ಮಗಳನ್ನು ಆಚರಿಸುವುದು, ಸಂಜಯನು ಪುನಃ ವೇದವ್ಯಾಸರ ಸಮೀಪಕ್ಕೆ ಬಂದು ಅವರನ್ನು ಸೇರಿ ಸದ್ಗತಿಯನ್ನು ಹೊಂದುವುದು, ಪಾಂಡವರು ನಂತರ ಹದಿನೆಂಟು ಸಂವತ್ಸರಗಳು ಭೂಮಿಯನ್ನು ಧರ್ಮದಿಂದ ಪರಿಪಾಲನೆಯನ್ನು ಮಾಡಿದುದು, ಈ ವಿಷಯಗಳು ಈ ಅಧ್ಯಾಯದಲ್ಲಿ ನಿರೂಪಿತವಾಗಿವೆ.


\section*{ಅಧ್ಯಾಯ\enginline{-}೩೨}

\begin{verse}
\textbf{ಯೋ ಯಷ್ಟಾ ವಿಪ್ರಶಾಪಾದ್ಯದುಕುಲಮವಧೀದರ್ಥಿತೋಽಗಾತ್ ಸ್ವಲೋಕಂ}\\\textbf{ದೇವೈಃ ಭೈಷಿದ್ಯುಪೇತೋ ಯದನುನಿಜಪದಂ ಪಾಂಡವಾಽಪ್ಯವಾಪುಃ~।}\\\textbf{ದೈತ್ಯಾ ಯದ್ವೇಷತೋಽಂಧೇತಮಸಿ ನಿಪತಿತಾ ಬುದ್ದ ರೂಪೋsಭವದ್ಯಃ}\\\textbf{ಕಲ್ಕ್ಯಾತ್ಮಾಂತೇ ಕಲೇರ್ಯಃ ಕುಜನನಿಧನಕೃತ್ಪಾತು ಸೋಽಸ್ಮಾನ್ ಮುಕುಂದಃ~।।}
\end{verse}

ಹನ್ನೆರಡು ವರ್ಷ ಕಾಲ ಯಜ್ಞವನ್ನು ಆಚರಿಸಿ, ಬ್ರಾಹ್ಮಣಶಾಪಕ್ಕೆ ಮನ್ನಣೆ ಇತ್ತು ಯಾದವ\-ಕುಲವನ್ನು ನಾಶಮಾಡಿ, ತನ್ನ ಅವತಾರ ಕಾರ್ಯ ಮುಗಿದ ನಂತರ ಬ್ರಹ್ಮಾದಿ ದೇವತೆಗಳಿಂದ ಪ್ರಾರ್ಥಿತನಾಗಿ ರುಕ್ಷ್ಮಿಣೀ ಸತ್ಯಭಾಮಾ ರೂಪಗಳಿಂದ ಯುಕ್ತಳಾದ ರಮಾದೇವಿಯ ಸಹಿತ ವೈಕುಂಠ ಲೋಕಕ್ಕೆ ತೆರಳಿ, ಪಾಂಡವರನ್ನೂ ಸಹ ಮೂಲರೂಪಕ್ಕೆ ಕಳಿಸಿ, ಬುದ್ಧ ರೂಪದಿಂದ ವಿಷ್ಣು ದ್ವೇಷಿಗಳನ್ನು ಅಂಧಂತಮಸ್ಸಿಗೆ ಕಳಿಸಿದ, ಕಲಿಯುಗದ ಅಂತ್ಯ ಭಾಗದಲ್ಲಿ ಕಲ್ಕಿ ರೂಪವನ್ನು ಪಡೆದು ದುಷ್ಟರನ್ನು ನಿಗ್ರಹಿಸಲಿರುವ ಶ‍್ರೀಮುಕುಂದನು ನಮ್ಮನ್ನು ಸಲಹಲಿ.

ಈ ಅಧ್ಯಾಯದಲ್ಲಿ ೧೮೨ ಶ್ಲೋಕಗಳಿವೆ.

ಶ‍್ರೀಕೃಷ್ಣನು ಕುರುಕ್ಷೇತ್ರದಲ್ಲಿ ಹನ್ನೆರಡು ವರ್ಷ ಯಜ್ಞದೀಕ್ಷೆಯಲ್ಲಿ ಇದ್ದುದು, ಬ್ರಹ್ಮಾದಿ ದೇವತೆಗಳೂ, ಭೃಗುಋಷಿಗಳೇ ಮೊದಲಾದ ಋಷಿಗಳೂ, ಪಾಂಡವರೂ, ಇತರ ಭಕ್ತರೂ ಯಜ್ಞದಲ್ಲಿ ತಮ್ಮ ಸೇವೆಯನ್ನು ಮಾಡಿದುದು, ಚರ್ಚೆಯಿಂದ ತತ್ವ ನಿಶ್ಚಯವಾದುದು, ಯಾಗದ ವೈಭವ, ಅದರಲ್ಲಿ ಭಾಗವಹಿಸಿದವರಿಗೆ ಬಂದ ಫಲ, ಉದ್ದವನಿಗೆ ತತ್ವೋಪದೇಶ ಮಾಡಿದುದು, ನೂರಏಳು ವರ್ಷಗಳ ಕಾಲ ಪಾಂಡವರೊಡನೆ ಭೂಮಿಯನ್ನು ರಕ್ಷಿಸುದುದು, ಕಲಿಯುಗವು ಪ್ರಾರಂಭವಾದ ನಂತರವೂ ಕೃತಯುಗದಂತೆಯೇ ಪ್ರವರ್ತಿಸಿದುದು, ಯುಗಧರ್ಮಕ್ಕೆ ಅನುಗುಣವಾದ ಫಲಗಳು, ಬ್ರಹ್ಮಾದಿ ದೇವತೆಗಳು ಶ‍್ರೀಕೃಷ್ಣನನ್ನು ಅವತಾರ ಸಮಾಪ್ತಿ ಮಾಡಲು ಕೇಳಿಕೊಳ್ಳುವುದು, ಶ‍್ರೀಕೃಷ್ಣನು ಯಾದವರನ್ನು ಪ್ರಭಾಸ ಕ್ಷೇತ್ರಕ್ಕೆ ಕರೆದುಕೊಂಡು ಹೋಗಿ ಅವರಿಂದ ನಾನಾ ದಾನಗಳನ್ನು ಮಾಡಿಸಿದುದು, ಬಲರಾಮನು ತನ್ನ ಮೂಲರೂಪವನ್ನು ಪಡೆದುದು, ಇತರ ಯಾದವರು ವಿಪ್ರ ಶಾಪದಿಂದ ಮಧುಪಾನಮತ್ತರಾಗಿ ಪರಸ್ಪರವಾಗಿ ನಾಶವಾದುದು, ದಾರುಕನನ್ನು ಪಾಂಡವರ ಬಳಿ ಕಳಿಸಿದುದು, ಅಶ್ವತ್ಥವೃಕ್ಷದ ಕೆಳಗೆ ಕುಳಿತಿದ್ದ ಶ‍್ರೀಕೃಷ್ಣನ ಕೆಂಪಾದ ಪಾದಕ್ಕೆ ಜರಾಖ್ಯವ್ಯಾಧನು ಬಾಣದಿಂದ ಹೊಡೆದುದು, ವ್ಯಾಧನ ಶೋಕ, ಭೃಗುಋಷಿಗಳೇ ವ್ಯಾಧನಾಗಿ ಅವತರಿಸಲು ಕಾರಣ, ಶ‍್ರೀಕೃಷ್ಣನ ನಿರ್ಗಮನ, ತನ್ನ ಐದು ರೂಪಗಳ ವ್ಯವಸ್ಥೆ, ಅಜ್ಞಜನರ ಮೋಹನಾರ್ಥವಾಗಿ ಪಾರ್ಥಿವ ಶರೀರವನ್ನು ನಿರ್ಮಿಸಿ ಬಿಟ್ಟು ಹೋಗುವುದು, ಅರ್ಜುನನು ಅದನ್ನು ಸುಡುವುದು, ರುಕ್ಷ್ಮಿಣೀ\break ಸತ್ಯಭಾಮೆಯರೂ ಸಹ ಶ‍್ರೀಕೃಷ್ಣನ ಪಾರ್ಶ್ವಕ್ಕೆ ತೆರಳುವುದು, ಇತರ ಆರುಜನ ಮಹಿಷ್ಯೆ\-ಯರೂ ಸಹ ದೇಹತ್ಯಾಗ ಮಾಡುವುದು, ವಸುದೇವನು ಕಶ್ಯಪತ್ವವನ್ನು ಹೊಂದುವುದು, ಅರ್ಜುನನು ಸ್ತ್ರೀ ಬಾಲಕರನ್ನು ಕರೆದುಕೊಂಡು ದ್ವಾರಾವತಿಯಿಂದ ಹೋಗುವುದು, ದ್ವಾರಾ\-ವತಿಯು ಸಮುದ್ರದಲ್ಲಿ ಮುಳುಗುವುದು, ದಾರಿಯಲ್ಲಿ ಮ್ಲೇಂಛ ಜನರಿಂದ ಸ್ತ್ರೀಯರಿಗೆ ಬಾಧೆ, ಅರ್ಜುನನು ಸಾಮರ್ಥ್ಯಹೀನನಾಗುವುದು, ಅರ್ಜುನನು ಹೇಗೋ ಸ್ತ್ರೀಯರನ್ನೂ ಧನವನ್ನೂ ರಕ್ಷಿಸಿಕೊಂಡು ಪ್ರಯಾಣ ಮಾಡುವುದು, ವೇದವ್ಯಾಸರ ದರ್ಶನ, ಮ್ಲೇಂಛರಿಂದ ಅಪಹರಿಸಿಕೊಳ್ಳಲ್ಪಟ್ಟ ಸ್ತ್ರೀಯರು ಗೋವಿಂದೈಕಾದಶೀ ವ್ರತದಿಂದ ಶುದ್ದರಾಗುವುದು, ಅರ್ಜುನನು ಸಹೋದರರಿಗೆ ಎಲ್ಲ ಸಮಾಚಾರವನ್ನೂ ತಿಳಿಸುವುದು, ಪಾಂಡವರು ಪರೀಕ್ಷಿತನಿಗೆ ರಾಜ್ಯಾಭಿಷೇಕ ಮಾಡುವುದು, ಪಾಂಡವರು ಭೂಮಂಡಲವನ್ನು ಬಿಡಲು ಸಿದ್ಧರಾಗುವುದು.

ಪರೀಕ್ಷಿತನಿಗೆ ರಾಜಧರ್ಮಗಳನ್ನು ಉಪದೇಶಮಾಡಲು ಸುಭದ್ರಾ, ಯುಯುತ್ಸು ಮತ್ತು ಇತರರನ್ನು ಬಿಟ್ಟು ಪಾಂಡವರು ದ್ರೌಪದಿಯಿಂದಲೂ ನಾಯಿಯಿಂದಲೂ ಯುಕ್ತರಾಗಿ ಈಶಾನ್ಯ ದಿಕ್ಕಿನಲ್ಲಿ ಪ್ರಯಾಣ ಮಾಡುವುದು, ಏಳು ಸಮುದ್ರಗಳನ್ನು ದಾಟಿ ಹೋಗುವುದು, ಗಂಧ\-ಮಾದನ ಪರ್ವತಕ್ಕೆ ಪಾಂಡವರ ಆಗಮನ; ಬದರಿಕಾಶ್ರಮದಲ್ಲಿ ದೌಪದೀ, ಸಹದೇವ, ನಕುಲ, ಅರ್ಜುನ, ಭೀಮಸೇನ ಇವರ ದೇಹಗಳ ಪತನ, ಈ ವಿಷಯದಲ್ಲಿ (ದೇಹಪತನ ವಿಷಯದಲ್ಲಿ) ಧರ್ಮರಾಜನು ಕಾರಣಗಳನ್ನು ಹೇಳುತ್ತಾ ಒಬ್ಬೊಬ್ಬರಲ್ಲಿಯೂ ದೋಷಗಳನ್ನು ತೋರಿಸುವುದು, ಇಂತಹ ದೋಷಗಳ ಕಾರಣದಿಂದ ದೇಹತ್ಯಾಗವೆಂಬ ವಿಷಯವನ್ನು ಅಲ್ಲಗಳೆಯುವುದು, ಋಜುಗಣಸ್ಥರಾದ ಭೀಮಸೇನ-ದೌಪದೀದೇವಿಗೆ ದೋಷಾಭಾವ ಸಮರ್ಥನೆ, ಧರ್ಮರಾಜನಿಗೆ ದೇವತೆಗಳು ರಥವನ್ನು ಕಳಿಸುವುದು, ರಥದಲ್ಲಿ ಪ್ರಯಾಣ, ಸ್ವರ್ಗಲೋಕದಂತೆ ಇದ್ದ ಸ್ಥಾನದಲ್ಲಿ ಧರ್ಮರಾಜನು ದುರ್ಯೋಧನನನ್ನು ನೋಡುವುದು, ಧರ್ಮರಾಜನ ಆಶ್ಚರ್ಯ, ದೇವತೆಗಳು ಧರ್ಮರಾಜನನ್ನು ದುರ್ಗಂಧದಿಂದ ಕೂಡಿದ ಮಾರ್ಗದಲ್ಲಿ ಕರೆದು ಕೊಂಡು ಹೋಗುವುದು, ಅಲ್ಲಿ ತನ್ನ ತಮ್ಮಂದಿರ ಧ್ವನಿಯಂತೆ ಇರುವ ಧ್ವನಿಯನ್ನು ಕೇಳುವುದು, ಧರ್ಮರಾಜನ ದುಃಖ, ಇಂದ್ರನು ಧರ್ಮರಾಜನಿಗೆ ನಿಜವಾದ ವೃತ್ತಾಂತವನ್ನು ತಿಳಿಸುವುದು, ಧರ್ಮರಾಜನ ದಯೆಯ ದ್ಯೋತಕಗಳಾದ ಸನ್ನಿವೇಶ ವರ್ಣನೆ, ಧರ್ಮರಾಜನು ಶ‍್ರೀಕೃಷ್ಣನ ವಾಕ್ಯದಲ್ಲಿ ಸಂದೇಹ ಹೊಂದಿದ ಕಾರಣದಿಂದ ಇಂತಹ ದೃಶ್ಯದ ಅನುಭವ, ಪ್ರಾರಬ್ಧ ಕರ್ಮದ ಫಲಾನುಭವ, ಅಪರೋಕ್ಷ ಜ್ಞಾನಾನಂತರ ಕರ್ಮಫಲಗಳ ವಿವರಣೆ, ದೈತ್ಯರಿಗೆ (ವಿಷ್ಣು ದ್ವೇಷಿಗಳಿಗೆ) ನರಕವು ತಪ್ಪಿದ್ದಲ್ಲವೆಂಬ ಪ್ರಮೇಯ ವಿವರಣೆ, ದೇವತೆಗಳ ಮಧ್ಯದಲ್ಲಿ ಧರ್ಮರಾಜನು ದ್ರೌಪದಿಯನ್ನು ಕಂಡು ಸ್ಪರ್ಶಮಾಡಲು ಪ್ರಯತ್ನ ಪಡುವುದು, ದೇವೇಂದ್ರನು ಧರ್ಮರಾಜನನ್ನು ತಡೆಯುವುದು, ದೌಪದೀ ದೇವಿಯ ಸ್ವರೂಪ ವರ್ಣನೆ, ಶ‍್ರೀಕೃಷ್ಣನ ಸಂಗಡ ಪಾಂಡವರು ತಮ್ಮ ಮೂಲ ರೂಪದಿಂದ ತಮ್ಮ ಪತ್ನಿಯರೊಡನೆ ಸಂತೋಷವಾಗಿ ಕ್ರೀಡಿಸುವುದು. ನಾಲ್ಕು ಸಾವಿರ ಮುನ್ನೂರು ವರ್ಷಗಳಾದ ಮೇಲೆ ದುರ್ಯೊಧನನೇ ಮೊದಲಾದ ಅಸುರರು ಭೂಮಿಯಲ್ಲಿ ಅವತರಿಸಿ “ನಾನೇ ಈಶ್ವರನು, ಜಗತ್ತು ಮಿಥ್ಯ” ಮುಂತಾದ ಜ್ಞಾನವನ್ನು ಹರಡುವುದು, ಅಂಧಂತಮಸ್ಸಿನಲ್ಲಿ ದುಃಖಾನುಭವದಲ್ಲಿ ತಾರತಮ್ಯ, ಭೀಮಸೇನನು ಕಲಿಯುಗದಲ್ಲಿ ಬ್ರಾಹ್ಮಣರೂಪದಿಂದ ಅವತರಿಸಿ ಹರಿಯ ನಿಜ ಸ್ವರೂಪವನ್ನು ಸಜ್ಜನರಿಗೆ ತಿಳಿಸುವುದು, ಭಾರತೀದೇವಿಯರ ಪಾತ್ರ, ಭೀಮಸೇನ-ಭಾರತಿಯರಿಗೆ ದೊರೆಯುವ ಸದ್ಗತಿಯ ವರ್ಣನೆ, ಶ‍್ರೀಹರಿಯಲ್ಲಿ ಇವರು ಮಾಡುವ ಭಕ್ತಿಯ ಶ್ರೇಷ್ಠತೆ, ಇವುಗಳನ್ನು ಸಮರ್ಥನೆಮಾಡುವ ಶ್ರುತಿವಾಕ್ಯಗಳ ಉದಾಹರಣೆ ಪರೀಕ್ಷಿತನ ರಾಜ್ಯಭಾರದಲ್ಲಿ ಪ್ರಜೆಗಳಿಗೆ ಆಗುತ್ತಿದ್ದ ಸುಖ-\-ಸಂತೋಷಗಳು, ಕಲಿಯುಗದಲ್ಲಿ ಒಂದು ಸಾವಿರ ವರ್ಷಗಳು ಕಳೆಯಲು ಅಸುರರು ಅವತಾರ ಮಾಡಿದುದು, ಅವರೂ ಸಹ ವೇದ\-ಶಾಸ್ತ್ರಗಳಲ್ಲಿ ಹೇಳಲ್ಪಟ್ಟ ಸಂಪ್ರದಾಯವನ್ನೇ ಅನುಸರಿಸುತ್ತಾ ಸತ್ಕರ್ಮಾನುಷ್ಠರಾಗಿ ಇರುತ್ತಿದುದು, ಒಳ್ಳೆಯ ತತ್ವಜ್ಞಾನವನ್ನು ಪಡೆದುದು, ದೇವತೆಗಳು ಶ‍್ರೀಹರಿಯನ್ನು ಪ್ರಾರ್ಥಿಸಿದುದು, ಶ‍್ರೀಹರಿಯು ಶುದ್ಧೋದನ-ಗಯಾ ದಂಪತಿಗಳಲ್ಲಿ ಉತ್ಪನ್ನವಾದ ಶಿಶುವನ್ನು ದೂರದಲ್ಲಿ ಬಿಸುಟು ತಾನು ಶಿಶುರೂಪದಿಂದ ಅದೇ ಸ್ಥಳದಲ್ಲಿ ಮಲಗಿ, ಶುದ್ಧೋದನಾದಿಗಳಿಗೆ ಸತ್ಕರ್ಮಾನುಷ್ಠಾನದ ಬಗ್ಗೆ ಹಾಸ್ಯ ಮಾಡಿ ಮಾತನಾಡಿ ಆಶ್ಚರ್ಯವನ್ನು ಉಂಟುಮಾಡಿದುದು, ಅವರಿಗೆ ನಂಬಿಕೆ ಹುಟ್ಟಲು ದೇವತೆಗಳಿಂದಲೂ ವಿಷ್ಣುವಿ\-ನಿಂದಲೂ ಬಿಡಲ್ಪಟ್ಟ ಆಯುಧಗಳನ್ನು ನುಂಗಿ ತಾನು ಬುದ್ಧನೆಂದು ಹೇಳಿಕೊಂಡು ಬೌದ್ಧ ಧರ್ಮವನ್ನು ಉಪದೇಶಮಾಡಿದುದು, ಶುದ್ಧೋದನಾದಿಗಳು ಆ ಧರ್ಮವನ್ನು ಆಶ್ರಯ ಮಾಡಿದುದು, ಬೌದ್ಧ ಧರ್ಮದ ಉಪದೇಶ ವಾಕ್ಯಗಳ (ಕ್ಷಣಿಕ, ಶೂನ್ಯ ಇತ್ಯಾದಿ) ನಿಜವಾದ ಅಭಿಪ್ರಾಯವನ್ನು ದೇವತೆಗಳಿಗೆ ಶ‍್ರೀಹರಿಯು ವಿವರಿಸಿದುದು, ಪುನಃ ನಿಜವಾದ ಮಗುವನ್ನು ಆ ಸ್ಥಳದಲ್ಲಿ ತಂದಿಟ್ಟು ತಾನು ಅದೃಶ್ಯನಾದುದು.

ಮುಂದೆ ಅಸುರರು ಕುತರ್ಕಗಳಿಂದ ಜನರನ್ನು ಮೋಹಗೊಳಿಸಲು ಶ‍್ರೀಹರಿಯು ಶಂಫಲವೆಂಬ ಗ್ರಾಮದಲ್ಲಿ ವಿಷ್ಣುಯಶ ಎಂಬ ವಿಪ್ರನಲ್ಲಿ ಕಲ್ಕಿ ಅವತಾರವನ್ನು ಮಾಡಿ, ದುಷ್ಟರನ್ನೆಲ್ಲ ಸಂಹರಿಸಿ ಧರ್ಮ, ಜ್ಞಾನ, ವಿಷ್ಣುವಿನಲ್ಲಿ ಭಕ್ತಿ ಇವುಗಳನ್ನು ಸಜ್ಜನರಲ್ಲಿ ಸಂಸ್ಥಾಪಿ\-ಸುವನೆಂಬ ವಿಷಯವನ್ನು ನಿರೂಪಿಸಿ, ಗ್ರಂಥಕರ್ತಗಳಾದ ಶ‍್ರೀಮದಾನಂದತೀರ್ಥರು ಬದರಿಯಲ್ಲಿನ ಶ‍್ರೀನಾರಾಯಣನ ಆಜ್ಞೆಯಿಂದ ಈ ಗ್ರಂಥವನ್ನು ರಚಿಸಿರುವುದಾಗಿ ಹೇಳುತ್ತಾರೆ. ಈ ಗ್ರಂಥವು ಸಮಸ್ತ ಶಾಸ್ತ್ರಗಳ ನಿರ್ಣಯದಿಂದ ಕೂಡಿರುತ್ತದೆ ಎಂದೂ ಅದರಲ್ಲಿಯೂ ಭಾರತ ಗ್ರಂಥವನ್ನು ವಿಶೇಷವಾಗಿ ಅನುಸರಿಸುತ್ತದೆ ಎಂದೂ, ಇದರಿಂದ ಶ‍್ರೀಹರಿ ಪ್ರೀತಿಯಾಗಲಿ ಎಂದು ಹೇಳಿರುತ್ತಾರೆ. ಶ‍್ರೀವೇದವ್ಯಾಸರು ಬ್ರಹ್ಮಸೂತ್ರಗಳನ್ನು ರಚಿಸಲು ಕಾರಣ ಮತ್ತು ಆ ಸೂತ್ರಗಳಿಗೆ ಶ‍್ರೀಪೂರ್ಣಪ್ರಜ್ಞರು ಭಾಷ್ಯವನ್ನು ರಚಿಸಲು ಕಾರಣವನ್ನು ಹೇಳಿ ಬಳಿತ್ಥಾ ಸೂಕ್ತ ಪ್ರತಿಪಾದ್ಯರಾದ ಪೂರ್ಣಪ್ರಜ್ಞರು ತಮ್ಮ ಅವತಾರದಲ್ಲಿ ಮಾಡಿದ ಕಾರ್ಯವನ್ನು ಸಂಕ್ಷೇಪವಾಗಿ ವಿವರಿಸಿ, ಶ‍್ರೀಮನ್ಮಹಾಭಾರತ ತಾತ್ಪರ್ಯ ಗ್ರಂಥವು ಪರಮ ಪ್ರಮಾಣವೆಂದು ಸಜ್ಜನರಿಗೆ ತಿಳಿಸಿ, ಜೀವೋತ್ತಮರಾದ ತಾವು ತಮ್ಮ ಮೂರು ಅವತಾರಗಳಲ್ಲಿಯೂ ಶ‍್ರೀಹರಿಯ ಸೇವೆಯನ್ನು ಮಾಡಿದ ಬಗೆಯನ್ನು ಸಂಕ್ಷೇಪವಾಗಿ ತಿಳಿಸಿ, ಈ ಗ್ರಂಥದಿಂದ ಸದಾ ತಮ್ಮ ಬಗ್ಗೆ ಪ್ರೀತನಾಗಿರುವ ಶ‍್ರೀಹರಿಯು ಇನ್ನೂ ಹೆಚ್ಚು ಪ್ರೀತನಾಗಲಿ ಎಂದು ಪ್ರಾರ್ಥಿಸಿ ಅಂತಿಮವಾಗಿ ಶ‍್ರೀಹರಿಯ ಸರ್ವೊತ್ತಮತ್ವ, ಸಕಲಗುಣ ಪರಿಪೂರ್ಣತ್ವ, ಸಮಸ್ತದೋಷವಿವರ್ಜಿತತ್ವವನ್ನು ಸಾರಿ ಹೇಳಿ ಗ್ರಂಥವನ್ನು ಪರಿಸಮಾಪ್ತಿಗೊಳಿಸಿರುತ್ತಾರೆ.

\begin{verse}
ಮಹಾಭಾರತತಾತ್ಪರ್ಯನಿರ್ಣಯಾಶಯಸಂಗ್ರಹಃ~।\\ ರಾಘವೇಂದ್ರಯತಿನಾ ಕೃತಃ ಸಜ್ಜನಸಂವಿದೇ~।।
\end{verse}

ಸಜ್ಜನರಿಗೆ ಯಥಾರ್ಥಜ್ಞಾನವನ್ನು ಅಭಿವೃದ್ಧಿಗೊಳಿಸಲು ಶ‍್ರೀ ರಾಘವೇಂದ್ರ ಯತಿಗಳಿಂದ “ಶ‍್ರೀಮನ್ಮಹಾಭಾರತತಾತ್ಪರ್ಯನಿರ್ಣಯ ಭಾವಸಂಗ್ರಹ” ಎಂಬ ಈ ಗ್ರಂಥವು ರಚಿತವಾಗಿದೆ.

\begin{center}
\textbf{।। ಆಚಾರ್ಯಾಃ ಶ‍್ರೀಮದಾಚಾರ್ಯಾಃ ಸಂತು ಮೇ ಜನ್ಮ ಜನ್ಮನಿ~।।}
\end{center}

