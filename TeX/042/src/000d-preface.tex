
\chapter*{ಅನುವಾದಕನ ವಿಜ್ಞಾಪನೆ}

\vskip -.6cm

 ಶ‍್ರೀಮನ್ಮಧ್ವಸಿದ್ಧಾಂತ ಪ್ರಚಾರಕ್ಕಾಗಿಯೇ ಇರುವ ಮದ್ರಾಸಿನ ವೆಸ್ಟ್ ಮಾಂಬಲಂ ಶ‍್ರೀ ಮಧ್ವ ಸಮಾಜದಲ್ಲಿ ಶ‍್ರೀಮದಾನಂದತೀರ್ಥರಿಂದ ರಚಿತವಾದ “ಶ‍್ರೀಮನ್ಮಹಾಭಾರತತಾತ್ಪರ್ಯ\-ನಿರ್ಣಯ” ಗ್ರಂಥವನ್ನು ಪ್ರವಚನ ಮಾಡುವ ಸುಯೋಗ ನನಗೆ ಪ್ರಾಪ್ತವಾಯಿತು. ಶ‍್ರೀ ಜನಾರ್ದನ ಭಟೀಯ ವ್ಯಾಖ್ಯಾನದ ಆಧಾರದಮೇಲೆ ಮೂಲ ಗ್ರಂಥದ ಪ್ರತಿಯೊಂದು ಶ್ಲೋಕವನ್ನೂ ಓದಿ, ಅರ್ಥ ವಿವರಣೆಯನ್ನು ನನ್ನ ಯೋಗ್ಯತಾನುಸಾರ ಮಾಡಿ ಮಂಗಳವನ್ನು ಆಚರಿಸಲಾಯಿತು. ಮೂವತ್ತೆರಡು ಅಧ್ಯಾಯಗಳಿಂದ ಯುಕ್ತವಾದ ಈ ಗ್ರಂಥದ ಸಂಕ್ಷಿಪ್ತ ಪರಿಚಯವನ್ನು ಶ‍್ರೀ ಮಂತ್ರಾಲಯ ಪ್ರಭುಗಳು ಮೂವತ್ತೆರಡು ಶ್ಲೋಕಗಳಲ್ಲಿ ಮಾಡಿರುತ್ತಾರೆ. ಪ್ರತಿ ಅಧ್ಯಾಯದ ಪ್ರವಚನ ಮುಗಿದಮೇಲೆ ಶ‍್ರೀ ಗುರುರಾಜರಿಂದ ರಚಿತವಾದ “ ಶ‍್ರೀಮನ್ಮಹಾಭಾರತತಾತ್ಪರ್ಯನಿರ್ಣಯ. ಭಾವಸಂಗ್ರಹ ” ಎಂಬ ಗ್ರಂಥದಿಂದ ಆ ಅಧ್ಯಾಯಕ್ಕೆ ಸಂಬಂಧಪಟ್ಟ ಶ್ಲೋಕವನ್ನು ಓದಿ, ತಾತ್ಪರ್ಯವನ್ನು ವಿವರಿಸಿ, ಸಂಕ್ಷಿಪ್ತವಾಗಿ ವಿವರಿಸುವ ಶ‍್ರೀ ಗುರುರಾಜರ ಕೌಶಲ್ಯವನ್ನು ಶೋತೃಗಳಿಗೆ ತೋರಿಸಿದೆ.

ಶ‍್ರೀ ಗುರುರಾಜರ ಆ ಗ್ರಂಥವನ್ನು ನನ್ನ ಯೋಗ್ಯತಾನುಸಾರವಾಗಿ ಕನ್ನಡದಲ್ಲಿ ಅನುವಾದಮಾಡಿ ಸಜ್ಜನರ ಮುಂದೆ ಇಡುತ್ತಿದ್ದೇನೆ. ಈ ಕಾರ್ಯವು ಶ‍್ರೀ ಹರಿವಾಯುಗಳ ಹಾಗೂ ಶ‍್ರೀ ಗುರುರಾಜರ ಸೇವೆ ಎಂಬ ದೃಷ್ಟಿಯಿಂದ ಮಾಡಲಾಗಿದೆ. ಸಜ್ಜನರು ಆದರದಿಂದ ಇದನ್ನು ಸ್ವೀಕರಿಸಿ ನನ್ನನ್ನು ಆಶೀರ್ವದಿಸ ಬೇಕಾಗಿ ಪ್ರಾರ್ಥಿಸುತ್ತೇನೆ. ಈ ಸಂಕ್ಷಿಪ್ತ ಪರಿಚಯದಿಂದ ಭಕ್ತರು ಶ‍್ರೀಮದಾಚಾರ್ಯರ ಮೂಲಗ್ರಂಥವನ್ನು ಅಧ್ಯಯನ ಮಾಡಲು ಇಚ್ಛಿಸಿ\-ದರೆ, ನನ್ನ ಈ ಅಲ್ಪಸೇವೆಯು ಸಾರ್ಥಕವೆಂದು ಭಾವಿಸುತ್ತೇನೆ.

ಈ ಗ್ರಂಥವನ್ನು ಪ್ರಕಾಶಿಸುತ್ತಿರುವ ಶ‍್ರೀ ಮಧ್ವ ಸಮಾಜಕ್ಕೆ ನನ್ನ ಅನೇಕ ವಂದನೆಗಳು.

\begin{flushleft}
ವಿಭವನಾಮ ಸಂ।। ಅಕ್ಷಯ ತೃತೀಯ \\ 19-4-1988 \hfill ಸಜ್ಞನ ವಿಧೇಯ,\\\hfill \textbf{ಏರೀ ಸುಬ್ಬಣ್ಣಾಚಾರ್ಯ}
\end{flushleft}

\newpage

\begin{center}
ಶ‍್ರೀ ಸತ್ಯರ್ಮಾನುಷ್ಠಾನ ನಿರತರೂ, ಶ‍್ರೀ ಹರಿವಾಯುಗಳ ಅನುಗ್ರಹಕ್ಕೆ ಪಾತ್ರರೂ, ಸಚ್ಚಾ ಶ್ರವಣ ಪ್ರವಚನದಲ್ಲಿ ಆಸಕ್ತರೂ, ನನ್ನ ತೀರ್ಥರೂಪರೂ ಆದ
\end{center}

\begin{center}
\textbf{ದಿವಂಗತ ಶ‍್ರೀ ಏರೀ ರಾಮಾಚಾರ್ಯರ}
\end{center}

\begin{center}
\textbf{ಪಾದಕಮಲಗಳಲ್ಲಿ ಭಕ್ತಿಯಿಂದ ಅರ್ಪಿಸಿರುವ ಅಲ್ಪ ಕಾಣಿಕೆ}
\end{center}

