\sethyphenation{kannada}{
गुरुराजो
विजयते
श्री
श्री-ः
ಅಂಗ-ಗಳೇ
ಅಂಗದ
ಅಂಗದ-ನನ್ನು
ಅಂಗದೇಶದ
ಅಂಗಭೂತರು
ಅಂಗ-ವಾಗಿ
ಅಂಗುಲದಷ್ಟು
ಅಂಜನಾ-ದೇವಿಯ
ಅಂಜಲಿ
ಅಂತರ್ಭಾಗ-ವನ್ನು
ಅಂತರ್ಯಾಮಿ-ಯಾದ
ಅಂತಹ
ಅಂತಿಮ
ಅಂತಿಮ-ಗತಿ
ಅಂತಿಮ-ವಾಗಿ
ಅಂತ್ಯ
ಅಂತ್ಯವೇ
ಅಂಧಂತಮಸ್ಸನ್ನು
ಅಂಧಂತಮಸ್ಸಿಗೆ
ಅಂಧಂತಮಸ್ಸಿನ
ಅಂಧಂತಮಸ್ಸಿ-ನಲ್ಲಿ
ಅಂಧ-ನನ್ನಾಗಿ
ಅಂಬಾಲಿಕಾ
ಅಂಬಾಲಿಕೆಯು
ಅಂಬಿಕಾ
ಅಂಬಿಕೆಗೆ
ಅಂಬೆಯು
ಅಂಶ-ಗಳಿರಲು
ಅಂಶ-ಗಳು
ಅಂಶಭೂತ-ನಾದ
ಅಂಶಭೂತ-ರಾದ
ಅಂಶಭೂತರು
ಅಕಂಪನ
ಅಕ್ರೂರನ
ಅಕ್ರೂರ-ನಿಗೆ
ಅಕ್ಷಯ
ಅಕ್ಷಯ-ಕುಮಾರನು
ಅಕ್ಷಯ-ಕು-ಮಾರನೇ
ಅಕ್ಷಾದೀಶ್ವಿ-ನಿಹತ್ಯ
ಅಕ್ಷೌಹಿಣೀ
ಅಗಲಿ-ಸು-ವುದು
ಅಗಸ್ತರ
ಅಗಸ್ಯ-ರಿಂದ
ಅಗಾಧ
ಅಗುವ
ಅಗೌರವ-ವನ್ನು
ಅಗ್ನಿ
ಅಗ್ನಿ-ಯಿಂದ
ಅಗ್ನಿಯು
ಅಗ್ರಗಣ್ಯ-ನನ್ನಾಗಿ
ಅಗ್ರಜ-ನೆಂದು
ಅಗ್ರ-ಪೂಜೆ
ಅಚ್ಯುತ-ನನ್ನು
ಅಜಗರ
ಅಜ್ಜನರ
ಅಜ್ಞಾತ-ವಾಸಕ್ಕಾಗಿ
ಅಜ್ಞಾತ-ವಾಸ-ವಾದ
ಅಜ್ಞಾತ-ವಾಸವು
ಅಡ್ಡಿ
ಅಣ್ಣ-ನಾದ
ಅತಿಕಾಯನ
ಅತುಲ-ನೆಂಬ
ಅತ್ಯಂತ
ಅದನ್ನು
ಅದರ
ಅದ-ರಲ್ಲಿ
ಅದರಲ್ಲಿಯೂ
ಅದ-ರಿಂದ
ಅದಾದ
ಅದು
ಅದೃಶ್ಯನಾ-ದುದು
ಅದೃಶ್ಯರಾಗುತ್ತಾರೆಂದೂ
ಅದೃಶ್ಯಳಾಗಿ
ಅದೃಶ್ಯ-ವಾಗಲು
ಅದೇ
ಅದ್ಭುತ
ಅದ್ಭುತ-ವಾದ
ಅಧರ್ಮ
ಅಧರ್ಮದ
ಅಧರ್ಮ-ವಲ್ಲ-ವೆಂಬುದನ್ನು
ಅಧಿಕಾರಿ-ಗಳು
ಅಧಿರಥ-ನಿಂದ
ಅಧಿಷ್ಠಿತ-ವಾದ
ಅಧ್ಯಯನ
ಅಧ್ಯಾಯ
ಅಧ್ಯಾಯಕ್ಕೆ
ಅಧ್ಯಾಯ-ಗಳಿಂದ
ಅಧ್ಯಾಯದ
ಅಧ್ಯಾಯ-ದಲ್ಲಿ
ಅನರ್ಥ
ಅನಿತ್ಯ-ವಾದ
ಅನಿರುದ್ಧ
ಅನಿರುದ್ಧ-ನನ್ನು
ಅನು-ಗುಣ-ವಾದ
ಅನುಗ್ರಹ
ಅನುಗ್ರಹಕ್ಕೆ
ಅನುಗ್ರಹ-ದಿಂದ
ಅನುಗ್ರಹಾರ್ಥ-ವಾಗಿ
ಅನುಗ್ರಹಿಸಿ
ಅನುಗ್ರಹೀತರಾಗು-ವುದು
ಅನುಭವ
ಅನುಮತಿ-ಯಿಂದ
ಅನುಯಾಯಿ-ಗಳ
ಅನುಯಾಯಿ-ಗಳನ್ನೂ
ಅನು-ವಾದ
ಅನು-ವಾದ-ಕನ
ಅನು-ವಾದ-ಮಾಡಿ
ಅನು-ವಿಂದ
ಅನುಸರಿಸುತ್ತದೆ
ಅನುಸರಿ-ಸುತ್ತಾ
ಅನುಸರಿಸು-ವಂತೆ
ಅನು-ಸಾರ-ವಾಗಿ
ಅನೇಕ
ಅನೇಕ-ರನ್ನು
ಅನ್ನ
ಅನ್ನೋನ್ಯಂ
ಅನ್ಯಂ
ಅನ್ಯರಲ್ಲ
ಅನ್ಯರಿಗಲ್ಲ
ಅನ್ಯರೆಲ್ಲರ
ಅಪಮಾನ
ಅಪಮಾನ-ಕರ-ವಾದ
ಅಪರಿಮಿತ
ಅಪರಿಮಿತ-ವಾದ
ಅಪರೋಕ್ಷ
ಅಪ-ವಾದ
ಅಪ-ಹರಿ-ಸಿ-ಕೊಳ್ಳಲ್ಪಟ್ಟ
ಅಪ-ಹರಿ-ಸಿದಾಗ
ಅಪ-ಹರಿ-ಸು-ವುದು
ಅಪಾಯ-ಗಳಿಂದ
ಅಪೇಕ್ಷಿಸಿ-ದುದು
ಅಪ್ರಮಾಣ-ವೆಂದು
ಅಪ್ರಾಕೃತ-ಗಳೇ
ಅಪ್ರಾಕೃತ-ನೆಂದು
ಅಪ್ರಾಕೃತ-ವಾದ
ಅಭಯ-ವನ್ನು
ಅಭಿಪ್ರಾಯ
ಅಭಿಪ್ರಾಯ-ವನ್ನು
ಅಭಿ-ಮನ್ನುವು
ಅಭಿಮನ್ಯು
ಅಭಿಮನ್ಯು-ವನ್ನು
ಅಭಿಮನ್ಯುವು
ಅಭಿಮಾನಿ
ಅಭಿವೃದ್ಧಿ
ಅಭಿವೃದ್ಧಿ-ಗೊಳಿ-ಸಲು
ಅಭಿವೃದ್ಧಿ-ಯನ್ನು
ಅಭಿವೃದ್ಧಿ-ಯಾಗು-ವುದು
ಅಭಿಷೇಕ
ಅಭಿಷೇಕ-ಮಾಡಿ
ಅಭೀಷ್ಟ-ವನ್ನು
ಅಭೆದ
ಅಭೇದ
ಅಭೇದವೇ
ಅಭ್ಯರ್ಥಿತಃ
ಅಮುಕ್ತ
ಅಮೃತ-ವನ್ನು
ಅಮೋಘ-ವಾದ
ಅಯಥಾರ್ಥಜ್ಞಾನಕ್ಕೆ
ಅಯೋಧ್ಯಾ
ಅಯೋಧ್ಯಾ-ನಗರಕ್ಕೆ
ಅಯೋಧ್ಯೆಗೆ
ಅರಣ್ಯಕ್ಕೆ
ಅರಣ್ಯ-ದಲ್ಲಿ
ಅರಣ್ಯ-ದಲ್ಲಿದ್ದ
ಅರಣ್ಯ-ದಿಂದ
ಅರಣ್ಯ-ವನ್ನು
ಅರಣ್ಯ-ವಾಸ-ಅಜ್ಞಾತ-ವಾಸದ
ಅರಣ್ಯ-ವಾಸ-ವನ್ನು
ಅರ-ಮನೆಗೆ
ಅರ-ಮನೆ-ಯನ್ನು
ಅರ-ಮನೆಯನ್ನೇ
ಅರ-ಮನೆ-ಯಲ್ಲಿ
ಅರ-ಮನೆ-ಯಲ್ಲಿ-ರು-ವುದು
ಅರಿಷ್ಟಾ-ಸುರನ
ಅರ್ಜುನ
ಅರ್ಜುನ-ಅಶ್ವತ್ಥಾಮರ
ಅರ್ಜುನ-ಚಿತ್ರಾಂಗದಾ
ಅರ್ಜುನ-ದುರ್ಯೊಧನರು
ಅರ್ಜುನ-ದುಶ್ಯಾಸನರ
ಅರ್ಜುನನ
ಅರ್ಜುನ-ನನ್ನು
ಅರ್ಜುನ-ನನ್ನೂ
ಅರ್ಜುನ-ನಲ್ಲಿ
ಅರ್ಜುನ-ನಿಂದ
ಅರ್ಜುನ-ನಿಗೆ
ಅರ್ಜುನನು
ಅರ್ಜುನನೂ
ಅರ್ಜು-ನನ್ನು
ಅರ್ಜುನ-ಭೀಷ್ಮರ
ಅರ್ಜುನರು
ಅರ್ಜುನ-ಶಲ್ಯರ
ಅರ್ಜುನ-ಸುಭದ್ರೆ-ಯರ
ಅರ್ಜುನಾ-ದಿ-ಗಳು
ಅರ್ಥ
ಅರ್ಥ-ಗಳನ್ನು
ಅರ್ಥ-ವನ್ನು
ಅರ್ಥ-ವನ್ನೇ
ಅರ್ಧ
ಅರ್ಧ-ರಾಜ್ಯ-ವನ್ನು
ಅರ್ಪಿ-ಸಿದ
ಅರ್ಪಿಸಿ-ರುವ
ಅರ್ಪಿ-ಸು-ವು-ದಾಗಿ
ಅಲಂಕರಿಸಿ
ಅಲಂಬುಸ
ಅಲಾಯುಧನು
ಅಲ್ಪ
ಅಲ್ಪ-ಸೇವೆಯು
ಅಲ್ಲ-ಗಳೆಯು-ವುದು
ಅಲ್ಲಿ
ಅಲ್ಲಿಂದ
ಅಲ್ಲಿಗೆ
ಅಲ್ಲಿದ್ದ
ಅವಗೃತ
ಅವತರಿ-ಸಲು
ಅವತರಿಸಿ
ಅವತರಿ-ಸಿದ
ಅವತರಿಸುವ
ಅವತರಿಸು-ವುದು
ಅವ-ತಾರ
ಅವ-ತಾರ-ಕಾರ್ಯ
ಅವ-ತಾರಕ್ಕಾಗಿ
ಅವ-ತಾರ-ಗಳ
ಅವ-ತಾರ-ಗಳನ್ನು
ಅವ-ತಾರ-ಗಳಲ್ಲಿ
ಅವ-ತಾರ-ಗಳಲ್ಲಿಯೂ
ಅವ-ತಾರ-ಗಳಿಗೂ
ಅವ-ತಾರ-ಗಳು
ಅವ-ತಾರ-ದಲ್ಲಿ
ಅವ-ತಾರ-ವನ್ನು
ಅವನ
ಅವ-ನನ್ನು
ಅವ-ನಲ್ಲಿ
ಅವನಸ್ಕೃತಿಗೆ
ಅವ-ನಿಗೆ
ಅವನು
ಅವ-ನೊಡನೆ
ಅವಮಾನ
ಅವಮಾನ-ವನ್ನು
ಅವಮಾನಿತರಾಗು-ವುದು
ಅವರ
ಅವ-ರನ್ನು
ಅವ-ರಿಂದ
ಅವರಿ-ಗಾಗಿ
ಅವ-ರಿಗೆ
ಅವರಿಬ್ಬರ
ಅವರು
ಅವರೂ
ಅವರೆಲ್ಲರೂ
ಅವಳ
ಅವಳನ್ನು
ಅವಳಲ್ಲಿ
ಅವಳಿಗೆ
ಅವಳು
ಅವಶ್ಯಕತೆ
ಅವಸರ-ದಲ್ಲಿ
ಅವೆಲ್ಲ-ವನ್ನೂ
ಅವೈಷ್ಣವ-ರಲ್ಲಿ
ಅವೈಷ್ಣವ-ರಿಂದ
ಅಶೋಕ-ವನ-ದಲ್ಲಿ
ಅಶೋಕ-ವನ-ವನ್ನು
ಅಶ್ವತಾಮಾಚಾರ್ಯರು
ಅಶ್ವತ್ಥವೃಕ್ಷದ
ಅಶ್ವತ್ಥಾಮ
ಅಶ್ವತ್ಥಾಮ-ಅರ್ಜುನರ
ಅಶ್ವತ್ಥಾಮನ
ಅಶ್ವತ್ಥಾ-ಮನು
ಅಶ್ವತ್ಥಾಮ-ರಿಂದ
ಅಶ್ವತ್ಥಾಮ-ರಿಗೆ
ಅಶ್ವತ್ಥಾಮರು
ಅಶ್ವತ್ಥಾಮಾ
ಅಶ್ವತ್ಥಾಮಾ-ಚಾರರು
ಅಶ್ವತ್ಥಾಮಾ-ಚಾರೈರಿಗೂ
ಅಶ್ವತ್ಥಾಮಾ-ಚಾರ್ಯ-ರನ್ನು
ಅಶ್ವತ್ಥಾಮಾ-ಚಾರ್ಯ-ರಿಂದ
ಅಶ್ವತ್ಥಾಮಾ-ಚಾರ್ಯ-ರಿಗೆ
ಅಶ್ವತ್ಥಾಮಾ-ದಿ-ಗಳು
ಅಶ್ವದ
ಅಶ್ವಮೇಧ
ಅಶ್ವಮೇಧ-ಯಜ್ಞ-ವನ್ನು
ಅಶ್ವಸೇನನು
ಅಶ್ವಸೇನ-ನೆಂಬ
ಅಶ್ವಿನೀ
ಅಷ್ಟಕರ್ತತ್ವವು
ಅಷ್ಟಮಹಿಷಿಯ-ರಿಗೆ
ಅಸಂಖ್ಯಾತ-ವಾದ
ಅಸಮರ್ಥರಾಗಲು
ಅಸಹಾಯಕತೆ
ಅಸುರ
ಅಸುರನ
ಅಸುರ-ನನ್ನು
ಅಸುರನು
ಅಸುರರು
ಅಸುರಾವೇಶ
ಅಸ್ತ್ರ
ಅಸ್ತ್ರಕ್ಕೆ
ಅಸ್ತ್ರ-ಗಳನ್ನು
ಅಸ್ತ್ರ-ಗಳಿಂದ
ಅಸ್ತ್ರಪ್ರೌಢಿ-ಮೆ-ಯನ್ನು
ಅಸ್ತ್ರ-ಮಂತ್ರ-ಗಳನ್ನು
ಅಸ್ತ್ರ-ವನ್ನು
ಅಸ್ತ್ರ-ವಿದ್ಯೆ
ಅಸ್ತ್ರ-ವಿದ್ಯೆ-ಯನ್ನು
ಅಸ್ತ್ರ-ವಿದ್ಯೆ-ಯಲ್ಲಿ
ಅಸ್ತ್ರ-ವಿದ್ಯೆಯು
ಅಸ್ತ್ರೇಷ್ವಧಿಕೋಽರ್ಜುನೋಽಥ
ಅಸ್ಥಾನಕ್ಕೆ
ಅಹಲೈ-ಯನ್ನು
ಆ
ಆಂಭಿಕೇಯಂ
ಆಕಾಶವಾಣಿ-ಯಿಂದ
ಆಕ್ಷೇಪಣೆ
ಆಗ
ಆಗ-ಮ-ಗಳನ್ನು
ಆಗಮನ
ಆಗಮಾಭ್ಯಾಸ-ನಿನೋ
ಆಗಮಿ-ಸಿದ
ಆಗಮಿ-ಸು-ವುದು
ಆಗುತ್ತಿದ್ದ
ಆಗುವ
ಆಗ್ನಿಯು
ಆಚರಿಸಲಾಯಿತು
ಆಚರಿಸಿ
ಆಚರಿ-ಸಿದ
ಆಚರಿ-ಸು-ವುದು
ಆಚಾರ್ಯ
ಆಚಾರ್ಯರ
ಆಚಾರ್ಯ-ರಲ್ಲಿ
ಆಚಾರ್ಯಾಃ
ಆಜ್ಞಾನು-ಸಾರ-ವಾಗಿ
ಆಜ್ಞೆ
ಆಜ್ಞೆ-ಯಂತೆ
ಆಜ್ಞೆ-ಯನ್ನು
ಆಜ್ಞೆ-ಯಿಂದ
ಆಟಪಾಟ-ಗಳಿಂದ
ಆಟವಾಡುವಾಗ
ಆಡಿ
ಆತಿಥ್ಯ
ಆತಿಥ್ಯ-ವನ್ನು
ಆತ್ಮ
ಆತ್ಮ-ಹತ್ಯೆಗೆ
ಆದ
ಆದರ-ದಿಂದ
ಆದರವು
ಆದರೂ
ಆದಾಗ
ಆದೇಶ-ದಂತೆ
ಆದೌ
ಆಧಾಯಾಷ್ಯ
ಆಧಾರದ-ಮೇಲೆ
ಆನೆ-ಯನ್ನು
ಆನೆ-ಯನ್ನೂ
ಆಪ
ಆಪತ್ಕಾಲ-ದಲ್ಲಿ
ಆಪತ್ತು-ಗಳನ್ನು
ಆಪ್ತ-ರಾದ
ಆಯುಧ-ಗಳನ್ನು
ಆಯುಧ-ದಿಂದ
ಆಯುಧರಥ
ಆಯುಸ್ಸನ್ನು
ಆರಿಸಿ-ದುದು
ಆರುಜನ
ಆರ್
ಆಲಂಗಿಸಿಕೊಳ್ಳು
ಆಲಂಗಿಸಿಕೊಳ್ಳು-ವುದು
ಆಲಿಂಗನ-ದಿಂದ
ಆಳಲು
ಆಳಿ-ದರೋ
ಆಳುತ್ತಿದ್ದ
ಆಳು-ವುದುಈ
ಆವರಿ-ಸು-ವುದು
ಆವಾಸ-ಮಾಡುವುದು
ಆವಿರ್ಭವಿ-ಸು-ವುದು
ಆವೃತ-ವಾದ
ಆವೇಶ
ಆವೇಶ-ವುಳ್ಳ
ಆಶೀರ್ವದಿಸ
ಆಶೋತ್ತರ-ಗಳನ್ನು
ಆಶ್ಚರ್ಯ
ಆಶ್ಚರ್ಯ-ವನ್ನು
ಆಶ್ಚರ್ಯ-ವಾಗು-ವುದು
ಆಶ್ರಯ
ಆಶ್ರಿತ-ರಾಜ-ರಿಂದ
ಆಶ್ವಾಸನೆ
ಆಸಕ್ತ-ರಾದ
ಆಸಕ್ತರಾ-ದುದು
ಆಸಕ್ತರೂ
ಆಸಕ್ತಿ
ಆಸನ-ವನ್ನು
ಆಸುರೀ
ಆಸೆ-ಪಟ್ಟ-ವಳಂತೆ
ಆಸ್ಥಾನಕ್ಕೆ
ಆಸ್ಥಾನದ
ಆಹ್ವಾನ-ಕೊಡು-ವುದು
ಆಹ್ವಾನಿತ-ರಾದ
ಆಹ್ವಾನಿಸಿ
ಆಹ್ವಾನಿ-ಸು-ವುದು
ಇಂಗಿತ-ವನ್ನು
ಇಂತಹ
ಇಂದ್ರ
ಇಂದ್ರ-ಕೀಲಕ
ಇಂದ್ರ-ಜಿತು-ವಿನ
ಇಂದ್ರ-ಜಿತು-ವಿನಿಂದ
ಇಂದ್ರ-ಜಿತುವು
ಇಂದ್ರ-ಜಿತ್
ಇಂದ್ರ-ದೇ-ವರ
ಇಂದ್ರನ
ಇಂದ್ರ-ನನ್ನು
ಇಂದ್ರ-ನಿಂದ
ಇಂದ್ರ-ನಿಗೆ
ಇಂದ್ರನು
ಇಂದ್ರಪ್ರಸ್ಥ
ಇಂದ್ರಪ್ರಸ್ಥಕ್ಕೆ
ಇಂದ್ರಪ್ರಸ್ಥ-ನಗರ-ದಲ್ಲಿ
ಇಂದ್ರಪ್ರಸ್ಥ-ಪುರ-ದಲ್ಲಿ
ಇಂದ್ರಪ್ರಸ್ಥ-ಪುರ-ದಲ್ಲಿ-ರುತ್ತಾ
ಇಂದ್ರಪ್ರಸ್ಥ-ಪುರವು
ಇಂದ್ರಪ್ರಸ್ಥ-ಪುರೇ-ಽವಸನ್
ಇಂದ್ರಾದಿ-ಗಳ
ಇಂದ್ರಾದಿ-ಗಳಿಗೆ
ಇಂದ್ರಿಯ-ಗಳು
ಇಚ್ಚಿ-ಸು-ವುದು
ಇಚ್ಛಿ
ಇಚ್ಛಿ-ಸಿದ
ಇಚ್ಛಿ-ಸಿದರೆ
ಇಚ್ಛೆ
ಇಚ್ಛೆ-ಯಂತೆ
ಇಚ್ಛೆ-ಯನ್ನು
ಇಟ್ಟ
ಇಟ್ಟು
ಇಟ್ಟು-ಕೊಂಡು
ಇಡಿ-ಸು-ವುದು
ಇಡೀ
ಇಡುತ್ತಿದ್ದೇನೆ
ಇಡು-ವುದು
ಇತರ
ಇತರ-ರನ್ನು
ಇತರ-ರಿಂದ
ಇತರ-ರಿಗೆ
ಇತರರು
ಇತರರೂ
ಇತ್ತು
ಇತ್ಯಾದಿ
ಇದಕ್ಕಾಗಿ
ಇದಕ್ಕೆ
ಇದನ್ನು
ಇದರಂತೆ
ಇದ-ರಿಂದ
ಇದೇ-ರೀತಿಯ
ಇದ್ದ
ಇದ್ದರೂ
ಇದ್ದರೆ
ಇದ್ದುದು
ಇನ್ನು
ಇನ್ನೂ
ಇಪ್ಪತ್ತ-ಮೂರು
ಇಪ್ಪತ್ತು
ಇಪ್ಪತ್ತೆಂಟನೇ
ಇಪ್ಪತ್ತೊಂದು
ಇಬ್ಬರು
ಇಬ್ಬರೂ
ಇಮಂ
ಇರ
ಇರ-ತಕ್ಕ
ಇರಲು
ಇರುತ್ತಾ
ಇರುತ್ತಾ-ನೆಂದು
ಇರುತ್ತಿ-ದುದು
ಇರುತ್ತಿದ್ದುದು
ಇರುವ
ಇರು-ವಂತೆ
ಇರು-ವುದು
ಇರುವೆ-ಗಳನ್ನು
ಇಲ್ಲ
ಇಲ್ಲ-ದೇಹೋ-ದುದು
ಇಲ್ಲ-ವೆಂದು
ಇವನು
ಇವರ
ಇವರನ್ನೆಲ್ಲ
ಇವ-ರಲ್ಲಿ
ಇವ-ರಿಂದ
ಇವರಿಬ್ಬ-ರಲ್ಲಿ
ಇವರು
ಇವ-ರೊಡನೆ
ಇವು-ಗಳನ್ನು
ಇವು-ಗಳಲ್ಲಿ
ಇವು-ಗಳಿಂದ
ಇಷ್ಟ-ವನ್ನು
ಇಷ್ಟ-ವಾದ
ಇಷ್ಟಾಂ
ಇಷ್ಟಾರ್ಥ-ಗಳನ್ನೂ
ಈ
ಈಡೇರಿ-ಸಿದ
ಈಶಾನ್ಯ
ಈಶ್ವರನು
ಉಂಗುರ-ವನ್ನೂ
ಉಂಟಾಗಿದೆ
ಉಂಟು-ಮಾಡಿ-ದುದು
ಉಂಟು-ಮಾಡುವುದು
ಉಗ್ರ-ವಾದ
ಉಗ್ರಸೇನ
ಉಗ್ರಸೇನ-ನನ್ನು
ಉಗ್ರಸೇನನು
ಉಗ್ರಾ-ಸುರನ
ಉಚಿತ-ವಾದ
ಉಚ್ಚರಿ-ಸುತ್ತಾ
ಉಡುಗೊರೆ-ಗಳನ್ನು
ಉತ್ಕೃಷ್ಟ-ನಾಗಿದ್ದಾನೆ
ಉತ್ತಮ
ಉತ್ತಮತ್ವವೇ
ಉತ್ತಮರು
ಉತ್ತರ
ಉತ್ತರ-ಕುಮಾರನು
ಉತ್ತರನ
ಉತ್ತರಾ-ದೇವಿಯು
ಉತ್ತರಾಯಣ-ವನ್ನು
ಉತ್ತರೆಯ
ಉತ್ತರೆ-ಯರ
ಉತ್ಪತ್ತಿ
ಉತ್ಪತ್ತಿ-ಯಾಗು-ವಂತೆ
ಉತ್ಪತ್ತಿ-ಯಾ-ದುದು
ಉತ್ಪನ್ನರಾಗು-ವುದು
ಉತ್ಪನ್ನ-ರಾದ
ಉತ್ಪನ್ನವಾಗು-ವುದು
ಉತ್ಪನ್ನ-ವಾದ
ಉತ್ಪಾದಿ-ಸಲು
ಉತ್ಪಾದಿಸಿ
ಉತ್ಪಾದಿ-ಸು-ವುದು
ಉತ್ಸವ
ಉದಂಕ
ಉದರ-ದಲ್ಲಿ
ಉದ-ಹರಿಸಿ
ಉದಾಹರಣೆ
ಉದ್ದವನ
ಉದ್ದವ-ನಿಗೆ
ಉದ್ದವನು
ಉದ್ಧವನ
ಉದ್ಧವ-ನನ್ನು
ಉದ್ವಾಹ್ಯಾಖಿಲಭೂಪತೀನಪಿ
ಉನ್ನತಗಿರೇರಾಪ್ಲುತ್ಯ
ಉಪ
ಉಪಕೀಚಕ-ರನ್ನು
ಉಪ-ಗತಃ
ಉಪಚರಿ-ಸು-ವುದು
ಉಪಚಾರ
ಉಪದೇಶ
ಉಪದೇಶ-ದಿಂದ
ಉಪದೇಶ-ಮಾಡಲು
ಉಪದೇಶ-ಮಾಡಲ್ಪಟ್ಟ
ಉಪದೇಶ-ಮಾಡಿ
ಉಪದೇಶ-ಮಾಡಿದ
ಉಪದೇಶ-ಮಾಡಿ-ದುದು
ಉಪದೇಶ-ವನ್ನು
ಉಪದೇಶಿ
ಉಪದೇಶಿ-ಸಲು
ಉಪದೇಶಿಸಿ
ಉಪ-ಪಾದನೆ
ಉಪಪ್ಪಾವ
ಉಪಯೋಗ
ಉಪ-ವಾಸ
ಉಪಾದೇಯತ್ವಈ
ಉಪಾಯ
ಉಲೂಪಿ-ಯಿಂದ
ಉಲ್ಲಂಘನೆ-ಗಾಗಿ
ಉಳಿದ
ಉಳ್ಳ
ಉಳ್ಳ-ವ-ನಾದಾಗ್ಯೂ
ಉಷಾ
ಉಸಿರಾಡುತ್ತಿ-ರುವ
ಊರ್ವಶಿ-ಯಿಂದ
ಋಜುಗಣಸ್ಥ-ರಾದ
ಋಷಿ-ಗಳ
ಋಷಿ-ಗಳಿಂದಲೂ
ಋಷಿ-ಗಳಿಗೆ
ಋಷಿ-ಗಳು
ಋಷಿ-ಗಳೂ
ಋಷ್ಯಮೂಕ
ಋಷ್ಯಾದಿ
ಎ
ಎಂಟನೆಯ
ಎಂಟು
ಎಂತಹ
ಎಂದಿಗೂ
ಎಂದು
ಎಂದೂ
ಎಂಬ
ಎಂಬು-ದಾಗಿ
ಎಂಬುವ-ರೊಡನೆ
ಎಚ್ಚರಿಕೆ
ಎಚ್ಚರಿಸಿ
ಎತ್ತರ-ವಾದ
ಎತ್ತಿ
ಎತ್ತುತ್ತಾ
ಎದುರಿಸಿ
ಎದುರಿ-ಸು-ವುದು
ಎದೆ-ಯನ್ನು
ಎರಡನೇ
ಎರಡು
ಎಲ್ಲ
ಎಲ್ಲರ
ಎಲ್ಲ-ರನ್ನೂ
ಎಲ್ಲ-ರಿಗಿಂತ
ಎಲ್ಲರೂ
ಎಲ್ಲ-ವನ್ನೂ
ಎಲ್ಲಿಯೂ
ಎಳೆದು
ಎಷ್ಟು
ಎಷ್ಟೆಷ್ಟು
ಎಸೆಯು-ವುದು
ಏಕಚಕ್ರ
ಏಕಚಕ್ರ-ನಗರಕ್ಕೆ
ಏಕಚಕ್ರ-ನಗರ-ದಲ್ಲಿದ್ದಾಗ
ಏಕಲವ್ಯ
ಏಕಲವ್ಯನ
ಏಕಲವ್ಯನು
ಏಕವಲ್ಯ-ನಿಗೆ
ಏರಿ
ಏರೀ
ಏರು-ವುದು
ಏರ್ಪಡಿ-ಸು-ವುದು
ಏಳು
ಏಳು-ಮಂದಿ
ಏಳು-ವಂತೆ
ಏಳು-ವುದು
ಏಳು-ವುದೆಂಬ
ಐಕ್ಯ-ವನ್ನು
ಐದು
ಐದು-ಜನರಿಗೂ
ಐರಾವತ
ಒಂದಾಗಿ
ಒಂದು
ಒಂದೇ
ಒಂದೊಂದು
ಒಟ್ಟಾಗಿ
ಒಡೆಯು-ವುದು
ಒತ್ತಿ
ಒದಗಿದ
ಒಪ್ಪದ
ಒಪ್ಪದಿರು-ವುದು
ಒಪ್ಪದೇ
ಒಪ್ಪಿ
ಒಪ್ಪಿ-ದುದು
ಒಪ್ಪಿ-ಸು-ವುದು
ಒಪ್ಪು-ವುದು
ಒಬ್ಬ
ಒಬ್ಬೊಬ್ಬ
ಒಬ್ಬೊಬ್ಬ-ರಲ್ಲಿಯೂ
ಒಯ್ದು
ಒಯ್ಯಲ್ಪಟ್ಟ
ಒಯ್ಯು-ವುದು
ಒಳ-ಗೊಂಡಿದೆ
ಒಳ್ಳೆಯ
ಓಡಿ
ಓಡಿ-ಸು-ವುದು
ಓಡಿ-ಹೋಗು-ವುದು
ಓಡಿ-ಹೋ-ದುದು
ಓಡು-ವುದು
ಓದಿ
ಔಚಿತ್ಯ
ಔಷಧ-ಗಳನ್ನು
ಔಷಧ-ದಿಂದ
ಕಂಕಣ-ವನ್ನು
ಕಂಡ
ಕಂಡು
ಕಂಡು-ಹಿಡಿ-ಯಲು
ಕಂಡು-ಹಿಡಿಯು-ವುದು
ಕಂಸಂ
ಕಂಸನ
ಕಂಸ-ನನ್ನು
ಕಂಸ-ನನ್ನೂ
ಕಂಸ-ನಿಂದ
ಕಂಸ-ನಿಗೆ
ಕಂಸನು
ಕಚ್ಚಿ-ಸು-ವುದು
ಕಟ್ಟ
ಕಟ್ಟ-ಬೇಕೆಂಬ
ಕಟ್ಟಿ
ಕಟ್ಟು-ವುದು
ಕಡೆಗೆ
ಕಡೆ-ಯಲ್ಲಿ
ಕಡೆ-ಯಲ್ಲಿ-ರುವ
ಕಣನು
ಕಣ್ಣಿನಲ್ಲಿದ್ದ
ಕಣ್ಣು
ಕಣ್ಣು-ಗಳನ್ನು
ಕತ್ತರಿಸಿ
ಕತ್ತರಿ-ಸಿದ
ಕತ್ತರಿ-ಸು-ವುದು
ಕತ್ತಲೆ-ಯನ್ನು
ಕಥಾ
ಕಥೆ
ಕದನ
ಕದ್ದು
ಕದ್ರ-ದೇವಿಯ
ಕನ್ನಡ
ಕನ್ನಡ-ದಲ್ಲಿ
ಕನ್ನಿಕೆ-ಯರ
ಕನ್ನಿಕೆ-ಯ-ರನ್ನು
ಕಪಟ
ಕಪಟ-ದಿಂದ
ಕಪಟ-ವಿದ್ಯೆ
ಕಪಟವ್ಯೂತದ
ಕಪಟವ್ಯೂತದಲ್ಲಿ
ಕಪಿ
ಕಪಿ-ಗಳ
ಕಪಿ-ಗಳನ್ನು
ಕಪಿ-ಗಳಿಂದಲೂ
ಕಪಿ-ಗಳಿಗೆ
ಕಪಿ-ಗಳು
ಕಪಿ-ಗಳೂ
ಕಪಿ-ಗಳೊಡನೆ
ಕಪಿ-ಭಿರ್ಯುತೋ
ಕಪೀನಾಂ
ಕಪ್ಪ-ಗಳನ್ನು
ಕಪ್ಪ-ವನ್ನು
ಕಬಂಧನ
ಕಮಲಪುಷ್ಪ-ಗಳ
ಕರವೀರ-ಪುರಕ್ಕೆ
ಕರುಣಯಾ
ಕರೆ-ತರಲು
ಕರೆ-ತರು-ವುದು
ಕರೆದು
ಕರೆದು-ಕೊಂಡು
ಕರೆದು-ಕೊಂಡು-ಬರು-ವುದು
ಕರೆಯು-ವುದು
ಕರೆ-ಸಲು
ಕರೆಸಿ-ಕೊಂಡು
ಕರ್ಣ
ಕರ್ಣ-ಅರ್ಜುನರ
ಕರ್ಣ-ಏಕಲವ್ಯ-ರಿಗೆ
ಕರ್ಣ-ದುಶ್ಯಾಸನರಿಂದ
ಕರ್ಣನ
ಕರ್ಣ-ನನ್ನು
ಕರ್ಣ-ನಿಂದ
ಕರ್ಣ-ನಿಗೆ
ಕರ್ಣನು
ಕರ್ಣ-ನೊಡನೆ
ಕರ್ಣಾ-ದಿ-ಗಳು
ಕರ್ಣಾರ್ಜುನರ
ಕರ್ತಾ
ಕರ್ತುಂ
ಕರ್ಮ-ಗಳನ್ನು
ಕರ್ಮ-ಗಳಲ್ಲಿ
ಕರ್ಮದ
ಕರ್ಮಫಲ-ಗಳ
ಕಲಸಿ
ಕಲಿ
ಕಲಿಂಗ-ರಾಜನ
ಕಲಿಂಗ-ರಾಜ-ನೊಡನೆ
ಕಲಿ-ಇಂದ್ರ-ಜಿತ್
ಕಲಿಗೂ
ಕಲಿಗೆ
ಕಲಿಮ್
ಕಲಿಯ
ಕಲಿ-ಯನ್ನು
ಕಲಿಯು
ಕಲಿ-ಯು-ಗದ
ಕಲಿ-ಯುಗ-ದಲ್ಲಿ
ಕಲಿ-ಯು-ಗವು
ಕಲಿ-ಯು-ವಂತೆ
ಕಲಿ-ಯು-ವುದು
ಕಲಿ-ಸುವ-ವ-ನಾಗಿ
ಕಲೇರ್ಯಃ
ಕಲ್ಕಿ
ಕಲ್ಕ್ಯಾತ್ಮಾಂತೇ
ಕಲ್ಪ-ದಂತೆಯೇ
ಕಳಿ-ಸಲು
ಕಳಿ-ಸಲ್ಪಟ್ಟ
ಕಳಿಸಿ
ಕಳಿ-ಸಿದ
ಕಳಿಸಿ-ದನೋ
ಕಳಿಸಿ-ದುದು
ಕಳಿ-ಸು-ವುದು
ಕಳುಹಿಸಿ
ಕಳೆದ-ಮೇಲೆ
ಕಳೆ-ಯಲು
ಕಶ್ಯ
ಕಸಿಶ್ರೇಷ್ಠ-ರೊಂದಿಗೆ
ಕಾಕಾಕ್ಷಿಗಂ
ಕಾಗೆಯ
ಕಾಗೆ-ಯಾಗಿ
ಕಾಡಾಗ್ನಿಯ
ಕಾಡುಗಿಚ್ಚನ್ನು
ಕಾಣ-ಬಹುದು
ಕಾಣಿಕೆ
ಕಾಣು-ವುದು
ಕಾನನಾಗ್ನಿಮ್
ಕಾಮ
ಕಾಮಕ್ಕೆ
ಕಾಮೋತ್ರ-ವಿದ್ಯಾ
ಕಾರಣ
ಕಾರಣ-ಗಳನ್ನು
ಕಾರಣ-ದಿಂದ
ಕಾರಣ-ನಾದ
ಕಾರಣ-ವನ್ನು
ಕಾರಣ-ವೆಂದು
ಕಾರಯನ್ನೊಽವತಾನ್ಮಾಮ್
ಕಾರಯಿತ್ವಾಽಧ್ವರಂ
ಕಾರಾಗೃಹ-ದಿಂದ
ಕಾರ್ಮುಕಧರಂ
ಕಾರ್ಯ
ಕಾರ್ಯ-ದಲ್ಲಿ
ಕಾರ್ಯ-ವನ್ನು
ಕಾರ್ಯವು
ಕಾಲ
ಕಾಲ-ದಲ್ಲಿ
ಕಾಲ-ದಲ್ಲಿಯ
ಕಾಲ-ದಲ್ಲಿಯೂ
ಕಾಲ-ಯನನ-ನಿಗೆ
ಕಾಲ-ಯವನನ
ಕಾಲ-ಯವನ-ನನ್ನು
ಕಾಲ-ಯವನನು
ಕಾಲಾವಧಿ
ಕಾಲಿಂಗನ
ಕಾಲಿಂದಿ-ಯನ್ನು
ಕಾಲಿ-ನಿಂದ
ಕಾಲೇಯ
ಕಾಳಗ
ಕಾಳಿಂಗಸರ್ಪನ
ಕಾವ್ಯ
ಕಾಶಿ
ಕಾಶಿ-ರಾಜನ
ಕಾಶೀ-ರಾಜನ
ಕಿರೀಟ
ಕಿರೀಟ-ವನ್ನು
ಕಿರ್ಮೀರ-ನೆಂಬ
ಕಿವಿಮೂಗು-ಗಳ
ಕೀಚ-ಕನ
ಕೀಚಕ-ನನ್ನು
ಕೀಚಕ-ನನ್ನೂ
ಕೀಚ-ಕನು
ಕೀಚಕಾನ್
ಕೀಟಸ್ವ-ರೂಪ-ದಲ್ಲಿದ್ದ
ಕೀರ್ತಿಯು
ಕುಂಡಲ-ಗಳನ್ನು
ಕುಂತಿ
ಕುಂತಿ-ದೇವಿ-ಯನ್ನು
ಕುಂತಿ-ದೇವಿಯು
ಕುಂತಿಯ
ಕುಂತಿ-ಯನ್ನು
ಕುಂತಿ-ಯಲ್ಲಿ
ಕುಂತಿ-ಯಿಂದ
ಕುಂತೀ-ಕರ್ಣರ
ಕುಂತೀ-ದೇವಿಗೆ
ಕುಂತೀ-ದೇವಿಯು
ಕುಂತೀ-ನಿಮಿತ್ತ-ದಿಂದ
ಕುಂಭ
ಕುಂಭ-ಕರ್ಣ
ಕುಂಭ-ಕರ್ಣ-ನನ್ನು
ಕುಂಭ-ಕರ್ಣ-ನಿಗೆ
ಕುಜನ-ನಿಧನ-ಕೃತ್ಪಾತು
ಕುಡಿ-ಸಿದ
ಕುಣಿದಾಡು-ವುದು
ಕುತಂತ್ರ
ಕುತರ್ಕ-ಗಳಿಂದ
ಕುದುರೆಯ
ಕುಪ್ಪಯ್ಯ
ಕುಬೇರ
ಕುಬೇರನ
ಕುಬೇರ-ನಿಂದ
ಕುಬೇರನು
ಕುಮಾರ-ರಾಗಿ
ಕುರಂಗ-ನೆಂಬ
ಕುರಂಗಾಸುರ-ನನ್ನು
ಕುರಿತು
ಕುರು
ಕುರುಕ್ಷೇತ್ರಕ್ಕೆ
ಕುರುಕ್ಷೇತ್ರ-ದಲ್ಲಿ
ಕುರು-ಡ-ನಾದ
ಕುರು-ಪಾಂಡವೈಶ್ಯ
ಕುರು-ಬಲೇ
ಕುರು-ಮುಖಾಃ
ಕುರೂನ್
ಕುಳಿತಿದ್ದ
ಕುಳಿತಿದ್ದಾಗ
ಕುಳಿತಿರಲು
ಕುಳಿತು
ಕುಳಿತು-ಕೊಂಡು
ಕುವಲಯಾಪೀಡ-ನೆಂಬ
ಕುಶ-ನನ್ನು
ಕುಶಲ
ಕುಶಲ-ನಾಗಿ
ಕುಶಲ-ವರ
ಕೂಗಿ-ದುದು
ಕೂಡಲೆ
ಕೂಡಲೇ
ಕೂಡಿ-ಕೊಂಡು
ಕೂಡಿದ
ಕೂಡಿರುತ್ತದೆ
ಕೂಡಿಸಿ-ಕೊಂಡು
ಕೂರ್ಮಾವ-ತಾರ
ಕೃತಃ
ಕೃತಧರಾ
ಕೃತ-ಯುಗ-ದಂತೆಯೇ
ಕೃತವತ್ಯಮುಂ
ಕೃತ-ವರ್ಮ
ಕೃತ-ವರ್ಮರು
ಕೃತಾದಿ
ಕೃತಿ
ಕೃತಿ-ಜದಿತಿಸುತಾನಾರ್ಥಿತೋ
ಕೃತೇ
ಕೃತ್ಯ-ಗಳನ್ನು
ಕೃತ್ಯ-ವನ್ನು
ಕೃತ್ವಾ
ಕೃಪ
ಕೃಪ-ಕೃಪಿ-ಯರ
ಕೃಪಾಚಾರ್ಯರು
ಕೃಪಾ-ದೃಷ್ಟಿ-ಯಿಂದಲೇ
ಕೃಪಾ-ಬಲ-ದಿಂದ
ಕೃಪೆ-ಯಿಂದ
ಕೃಷ್ಣ
ಕೃಷ್ಣಂ
ಕೃಷ್ಣಃ
ಕೃಷ್ಣನ
ಕೃಷ್ಣ-ನನ್ನು
ಕೃಷ್ಣ-ನಲ್ಲಿ
ಕೃಷ್ಣ-ನಾಗಿ
ಕೃಷ್ಣ-ನಿಂದ
ಕೃಷ್ಣ-ನಿಗೆ
ಕೃಷ್ಣನು
ಕೃಷ್ಣ-ರುಕ್ಕಿಣಿ-ಯರು
ಕೃಷ್ಣ-ರೊಡನೆ
ಕೃಷ್ಣ-ಸತ್ಯಭಾಮೆ
ಕೃಷ್ಣಸ್ತದನಭಿಮತೇಽವಾಪ್ತಪಾರ್ಥಃ
ಕೃಷ್ಣಾಭ್ಯಾಮಪಿ
ಕೃಷ್ಣಾರ್ಜುನರು
ಕೃಷ್ಣಾವ-ತಾರಕ್ಕೆ
ಕೆಂಪಾದ
ಕೆಡವಿ
ಕೆಡವಿ-ಹಾಕು-ವುದು
ಕೆಡವು-ವುದು
ಕೆಲ-ವರು
ಕೆಲವು
ಕೆಲಸ
ಕೆಳಗೆ
ಕೇಳಲ್ಪಟ್ಟ
ಕೇಳಿ
ಕೇಳಿ-ಕೊಳ್ಳು-ವುದು
ಕೇಳಿದ
ಕೇಳು
ಕೇಳು-ವುದು
ಕೇವಲ
ಕೇಶವಃ
ಕೇಶವ-ನನ್ನು
ಕೇಶವನು
ಕೇಶವಮ್
ಕೇಶಿ-ನಾಮಕ
ಕೈ
ಕೈಯಲ್ಲಿ
ಕೈಯಲ್ಲಿದ್ದ
ಕೈಲಾಸಕ್ಕೆ
ಕೈಲಾಸ-ದಿಂದ
ಕೊಂಡು
ಕೊಂದ
ಕೊಂದು
ಕೊಟ್ಟ
ಕೊಟ್ಟಿದ್ದು
ಕೊಟ್ಟು
ಕೊಡ
ಕೊಡ-ತಕ್ಕ
ಕೊಡ-ದಿದ್ದರೆ
ಕೊಡದೆ
ಕೊಡಲು
ಕೊಡಲ್ಪಟ್ಟ
ಕೊಡಿಸಿ
ಕೊಡಿ-ಸು-ವುದು
ಕೊಡುತ್ತಾ
ಕೊಡುತ್ತಿ-ದುದು
ಕೊಡುತ್ತೇನೆಂದು
ಕೊಡುವ
ಕೊಡು-ವು-ದಾಗಿ
ಕೊಡು-ವುದು
ಕೊನೆಗೆ
ಕೊಲ್ಲುತ್ತಿದ್ದ
ಕೊಳ್ಳಲ್ಪಟ್ಟ
ಕೊಳ್ಳುತ್ತಿದ್ದ
ಕೊಳ್ಳುವ
ಕೊಳ್ಳು-ವುದು
ಕೋಟಿ
ಕೋಪ
ಕೋಪದ
ಕೋಪ-ದಿಂದ
ಕೌರವ
ಕೌರವ-ಪಾಂಡ-ವರ
ಕೌರವ-ಪಾಂಡ-ವರಿಗೆ
ಕೌರವ-ಪಾಂಡ-ವರು
ಕೌರವ-ಮರ್ಕಪರ್ವಣಿ-ಪುರೀಂ
ಕೌರ-ವರ
ಕೌರವ-ರನ್ನು
ಕೌರವ-ರ-ಪಾಂಡ-ವರ
ಕೌರವ-ರಿಗೆ
ಕೌರ-ವರು
ಕೌರ-ವರೂ
ಕೌರವ-ರೆಲ್ಲ-ರನ್ನೂ
ಕೌರವಾಣಾಮವಾಪ್ತಃ
ಕೌರವೇಭ್ಯೋ
ಕೌಶಲ್ಯ-ವನ್ನು
ಕೌಸಲ್ಯಯ
ಕ್ರತೂ
ಕ್ರತೋ
ಕ್ರಮ
ಕ್ರಮ-ವಾಗಿ
ಕ್ರಿಯಾತ್ಮಕ-ವಾಗಿ-ರುವ
ಕ್ರಿಯಾದಿ-ಗಳನ್ನು
ಕ್ರೀಡನ್ದೇವೈರಜಾದ್ಯೈರಗಣಿತಸು-ಗುಣೋ
ಕ್ರೀಡನ್ಮಲ್ಲಾಂಶ್ಚ
ಕ್ರೀಡಿಸಿ-ದುದು
ಕ್ರೀಡಿ-ಸುತ್ತಾ
ಕ್ರೀಡಿಸುತ್ತಿದ್ದ
ಕ್ರೀಡಿಸುವ
ಕ್ರೀಡಿಸು-ವುದು
ಕ್ರೀಡಿಸು-ವುದುಈ
ಕ್ರೀಡೆ
ಕ್ರೀಡೆ-ಗಾಗಿ
ಕ್ರೀಡೆ-ಗೋಸ್ಕರ
ಕ್ರೀಡೆ-ಯಿಂದ
ಕ್ರೋಧ
ಕ್ರೋಧ-ನಾಮಕ
ಕ್ಷಣದೊಳಗೆ
ಕ್ಷಣ-ಮಾತ್ರ-ದಲ್ಲಿ
ಕ್ಷಣಿಕ
ಕ್ಷತಿಪತಿಮಕರೋದುಗ್ರಸೇನಂ
ಕ್ಷತ್ರನ
ಕ್ಷತ್ರಿ-ಯರ
ಕ್ಷತ್ರಿಯ-ರಾಜ-ರನ್ನು
ಕ್ಷಮಿಸಿ
ಕ್ಷಮೆ-ಯನ್ನು
ಕ್ಷಿತಿಮಥ
ಕ್ಷೀರ
ಕ್ಷೀರ-ಸಮುದ್ರ
ಕ್ಷೀರ-ಸಮುದ್ರಕ್ಕೆ
ಕ್ಷೀರ-ಸಮುದ್ರ-ವನ್ನು
ಕ್ಷೀರಾಬ್ದ್ಯುನ್ಮಥ-ನಾದಿಕಾತ್ಮಚರಿತಂ
ಕ್ಷೇತ್ರಂ
ಕ್ಷೇತ್ರಕ್ಕೆ
ಕ್ಷೇಮಧೂರ್ತಿ-ಯನ್ನು
ಖಗಾದಿ-ಗಳಿಂದ
ಖಡ್ಗ-ವನ್ನು
ಖರದೂಷಣ-ರಿಂದ
ಖರದೂಷಣಾದಿ-ಗಳ
ಖರನೇ
ಖರಾದೀನ್ಖಲಾನ್
ಖಲತರೇ
ಖಾಂಡವವನ
ಖಾಂಡವವನದ
ಖಾಂಡವವನ-ದಲ್ಲಿ
ಗಂಗಾ-ನದಿ-ಯನ್ನು
ಗಂಗಾ-ನದಿ-ಯಲ್ಲಿ
ಗಂಗೆ
ಗಂಗೆ-ಯನ್ನು
ಗಂಗೆಯು
ಗಂಡು-ಮಕ್ಕಳನ್ನು
ಗಂಡು-ಮಕ್ಕಳೂ
ಗಂಧಮಾದನ
ಗಂಧರ್ವ
ಗಂಧರ್ವ-ನಿಂದ
ಗಂಧರ್ವನು
ಗಂಧರ್ವ-ರನ್ನು
ಗಂಧರ್ವರು
ಗಂಧಲೇಪ-ನಾದಿ-ಗಳನ್ನು
ಗಂಧ-ವನ್ನು
ಗಚ್ಛತಾ
ಗಣ-ರೆಂಬ
ಗತಂ
ಗತಃ
ಗತವಿರಹಶುಚಃ
ಗತಾಃ
ಗತಿಗೆ
ಗತೋ
ಗತ್ವಾ
ಗದ-ರಿ-ಸು-ವುದು
ಗದಾ
ಗದಾಭ್ಯಾಸ
ಗದಾ-ಯುದ್ಧ
ಗದೆ-ಯನ್ನು
ಗದೆ-ಯಿಂದ
ಗಮನಾಗಮನ-ಗಳಿಂದ
ಗರುಡನ
ಗರುಡ-ನಿಂದ
ಗರುಡನು
ಗರುಡಶೇಷರ
ಗರುಡಾರೂಢ-ನಾಗಿ
ಗರ್ಗಾಚಾರ್ಯ-ರಿಂದ
ಗರ್ಗಾಚಾರ್ಯರು
ಗರ್ಗಾದ್ಬಹು-ಶಿಶುಚರಿತೈಃ
ಗರ್ಭಕ್ಕೆ
ಗರ್ಭ-ದಲ್ಲಿ
ಗರ್ಭ-ದಲ್ಲಿ-ರ-ತಕ್ಕ
ಗರ್ವ-ನಾಶ
ಗಳನ್ನೂ
ಗಳಿಗೆ
ಗಾಂಡೀವ-ವನ್ನು
ಗಾಂಡೀವ-ವನ್ನೂ
ಗಾಂಡೀ-ವಾದಿ
ಗಾಂಧರ್ವ-ವೇದ-ವನ್ನು
ಗಾಂಧಾರಿ
ಗಾಂಧಾ-ರಿಗೆ
ಗಾಂಧಾರಿಯ
ಗಾಂಧಾರಿ-ಯಿಂದ
ಗಾಡಿ-ಯಲ್ಲಿ
ಗಾಯ-ಗಳನ್ನು
ಗಾಯ-ಗೊಂಡ
ಗಾಯತ್ರಿ
ಗಾಳಿ
ಗಿರಿಗೆ
ಗೀತೆಯ
ಗೀತೋಪದೇಶ-ವನ್ನು
ಗುಣ-ಪೂರ್ಣ-ಗಳೇ
ಗುಣ-ಪೂರ್ಣತ್ವ
ಗುರಿಯ
ಗುರು-ಗಳ
ಗುರು-ಗಳನ್ನಾಗಿ
ಗುರು-ಗಳಲ್ಲಿಯೂ
ಗುರು-ಗ-ಳಾದ
ಗುರು-ದಕ್ಷಿಣೆ
ಗುರು-ದಕ್ಷಿಣೆ-ಗಾಗಿ
ಗುರು-ದಕ್ಷಿಣೆ-ಗೋಸ್ಕರ
ಗುರು-ದಕ್ಷಿಣೆ-ಯನ್ನು
ಗುರು-ದಕ್ಷಿಣೆ-ಯಾಗಿ
ಗುರು-ರಾಜರ
ಗುರು-ರಾಜ-ರಿಂದ
ಗುರೋರ್ಯಃ
ಗುಹ-ನಿಂದ
ಗುಹೆಗೆ
ಗುಹ್ಯ
ಗುಹ್ಯ-ಭಾಷೆ-ಗಳ
ಗೃಹಸ್ಥರೂ
ಗೆಜೆಟೆಡ್
ಗೆಲ್ಲಲು
ಗೊಪಿಕಾಭಿರ್ನಿಶಾಸು
ಗೊಲ್ಲ-ರನ್ನು
ಗೊಲ್ಲ-ರಿಗೆ
ಗೊಳಿ-ಸಲ್ಪಟ್ಟ
ಗೋಕುಲಕ್ಕೆ
ಗೋಗ್ರಹಣೋದ್ಯ
ಗೋಪಗೋಪೀಃ
ಗೋಪನೇ
ಗೋಪತ್ರೀ-ಯ-ರೊಡನೆ
ಗೋಪಾಲಕ-ರನ್ನೂ
ಗೋಪಿಕಾ
ಗೋಪಿಕಾತ್ರೀ-ಯ-ರನ್ನೂ
ಗೋಮಂತ
ಗೋಮಂತ-ಕಶಿಖರ-ದಿಂದ
ಗೋಮಂತ-ಮಾತ್ರಾಗತಾತ್
ಗೋಮಾಂತಕ
ಗೋರಕ್ಷಕ-ನಾಗಿ
ಗೋವತಾದಿ-ಗಳಿಂದ
ಗೋವನ್ನು
ಗೋವರ್ಧನ
ಗೋವರ್ಧನ-ಗಿ-ರಿ-ಯನ್ನು
ಗೋವರ್ಧನ-ಪರ್ವತ-ವನ್ನು
ಗೋವಿಂದೈಕಾದಶೀ
ಗೋವು-ಗಳನ್ನು
ಗೋವು-ಗಳು
ಗೌತಮರ
ಗೌರವ
ಗೌರವಿ-ಸಲ್ಪಟ್ಟ
ಗ್ರಂಥಕರ್ತ-ಗ-ಳಾದ
ಗ್ರಂಥಕ್ಕೆ
ಗ್ರಂಥ-ಗಳ
ಗ್ರಂಥ-ಗಳನ್ನು
ಗ್ರಂಥ-ಗಳಿಗೂ
ಗ್ರಂಥದ
ಗ್ರಂಥ-ದಿಂದ
ಗ್ರಂಥ-ವನ್ನು
ಗ್ರಂಥವು
ಗ್ರಹ-ಗಳಿಗೆ
ಗ್ರಹಣ
ಗ್ರಾಮ-ದಲ್ಲಿ
ಗ್ರಾಮ-ದಲ್ಲಿಯೂ
ಘಂಟಾ-ಕರ್ಣ
ಘಟ-ದಲ್ಲಿ
ಘಟೋತ್ಕಚ-ಅಶ್ವತ್ಥಾಮರ
ಘಟೋತ್ಕಚನ
ಘಟೋತ್ಕಚನನ್ನು
ಘಟೋತ್ಕಚನಿಂದ
ಘಟೋತ್ಕಚನು
ಘಟೋತ್ಕಚಭ-ಗದತ್ತರ
ಘೋರ
ಘೋರ-ವಾದ
ಘೋಷಿ-ಸು-ವುದು
ಚ
ಚಂದ್ರವಂಶ-ದಲ್ಲಿ
ಚಕ್ರ-ದಿಂದ
ಚಕ್ರವ್ಯೂಹ-ದಲ್ಲಿ
ಚಕ್ರೇಽಸ್ಯ
ಚತುರ್ಮುಖ
ಚತುರ್ಮುಖಬ್ರಹ್ಮ-ದೇ-ವರೇ
ಚತುರ್ವಿಧ
ಚರಣೆಯ
ಚರಿತ್ರೆ-ಯಿಂದ
ಚರ್ಚಿ-ಸು-ವುದು
ಚರ್ಚೆ
ಚರ್ಚೆ-ಯಿಂದ
ಚಾಂದ್ರಮಾನ
ಚಾಣೂರಮುಷ್ಟಿಕ-ರೆಂಬ
ಚಾತುರ್ವಣ್ರ-ಗಳ
ಚಾಮರ-ವನ್ನು
ಚಾರ್ವಾಕ-ನೆಂಬ
ಚಿಂತೆ
ಚಿತ್ರಕೂಟಪರ್ವತ
ಚಿತ್ರರಥನ
ಚಿತ್ರಸೇನ-ನಿಂದ
ಚಿತ್ರಸೇನ-ನೆಂಬ
ಚಿತ್ರಾಂಗದ
ಚೂಡಾ-ಮಣಿ-ಯನ್ನು
ಚೆಂಡನ್ನೂ
ಚೆಟ್ಟ
ಛೇದನ
ಛೇದಿ-ಸು-ವುದು
ಜಂಗಮಾತ್ಮ-ಕ-ವಾದ
ಜಂಘ-ನೆಂಬ
ಜಗತಿ
ಜಗತ್
ಜಗತ್ತನ್ನು
ಜಗತ್ತು
ಜಘ್ನೇ
ಜಜ್ಞಿರೇ
ಜಟಾಯು-ವಿನ
ಜಟಾ-ಸುರನ
ಜಟ್ಟಿ-ಗಳ
ಜಟ್ಟಿ-ಗಳನ್ನೂ
ಜಟ್ಟಿ-ಯನ್ನೂ
ಜತೆ
ಜತೆ-ಗೂಡಿ-ಸಿದ
ಜತೆ-ಯಲ್ಲಿ
ಜನ
ಜನಕದು-ಹಿತರಿ
ಜನಕ-ರಾಜನ
ಜನನ
ಜನಮೇಜಯ-ನಿಗೆ
ಜನರ
ಜನ-ರನ್ನು
ಜನ-ರಿಂದ
ಜನರಿಗೂ
ಜನ-ರಿಗೆ
ಜನರು
ಜನಾರ್ದನ
ಜನಿ-ಸಿದ-ವರು
ಜನ್ಮ
ಜನ್ಮನಿ
ಜನ್ಮ-ವೃತ್ತಾಂತ
ಜನ್ಮ-ವೆತ್ತಿ
ಜಪಿ-ಸು-ವುದು
ಜಮಾವಣೆ
ಜಯ
ಜಯಂತನು
ಜಯತ್ಯಮಿತಸದ್
ಜಯದ್ರಥ
ಜಯದ್ರಥನ
ಜಯದ್ರಥ-ನನ್ನು
ಜಯದ್ರಥನು
ಜಯವಿಜ-ಯರ
ಜಯವಿಜಯ-ರಿಗೆ
ಜಯಾ
ಜಯಿಸಿ
ಜರಾಖ್ಯವ್ಯಾಧನು
ಜರಾಸಂಧ
ಜರಾಸಂಧ-ಕರ್ಣರ
ಜರಾಸಂಧನ
ಜರಾಸಂಧ-ನನ್ನು
ಜರಾಸಂಧ-ನನ್ನೂ
ಜರಾಸಂಧ-ನಿಂದ
ಜರಾಸಂಧ-ನಿಗೆ
ಜರಾಸಂಧನು
ಜರಾಸಂಧನೇ
ಜರಾಸಂಧ-ನೊಡನೆ
ಜರಾಸಂಧ-ಬಲ-ರಾಮರ
ಜರಾಸಂಧಾದಿ
ಜರಾಸಂಧಾದಿ-ಗಳಿಂದ
ಜರಾಸಂಧಾದಿ-ಗಳು
ಜರಾ-ಸುತೇ
ಜಲ-ದಿಂದ
ಜಲಧಾರೆಯು
ಜಲಪ್ರವೇಶ
ಜಾಂಬವಂತ
ಜಾಂಬವಂತ-ನಿಗೆ
ಜಾಂಬವತಿ-ಯನ್ನು
ಜಾತಕರ್ಮಾದಿ
ಜಾತಾ
ಜಾದಿನಿಬುಧಾನ್
ಜಾನಕಿ-ದೇವಿ-ಯನ್ನು
ಜಾನಕಿ-ದೇವಿ-ಯಿಂದ
ಜಾನಕೀಂ
ಜಿಂಕೆ-ಗಾಗಿ
ಜಿಂಕೆಯ
ಜಿತೇ
ಜಿತೈಶ್ಚ
ಜಿತ್ವಾ
ಜೀರ್ಣ-ಮಾಡಿ-ಕೊಂಡಿದ್ದು
ಜೀವ
ಜೀವ-ಕರ್ತೃತ್ವ
ಜೀವ-ಜಡಾತ್ಮ-ಕ-ವಾದ
ಜೀವನ
ಜೀವ-ನಕ್ರಮ
ಜೀವ-ನ-ವನ್ನು
ಜೀವರ
ಜೀವ-ರಲ್ಲಿ
ಜೀವೋತ್ತಮ-ರಾದ
ಜುಗುಪುಃ
ಜುಗೋಪ
ಜೇತಾ
ಜೈಮಿನಿ
ಜೊತೆ-ಯಲ್ಲಿ
ಜೊತೆ-ಯಲ್ಲಿಯೇ
ಜೋಡಿಸಿ
ಜ್ಞಾತ್ವಾಗಾತ್ಸಹಜಾನ್ವಿತೋಽತಿಗಹನಂ
ಜ್ಞಾನ
ಜ್ಞಾನಂ
ಜ್ಞಾನ-ದಿಂದಲೇ
ಜ್ಞಾನ-ದಿಂದಾಗುವ
ಜ್ಞಾನ-ಪೂರ್ವ-ಕ-ವಾದ
ಜ್ಞಾನಪ್ರಚಾರಕ್ಕಾಗಿ
ಜ್ಞಾನ-ಭಕ್ತಿ
ಜ್ಞಾನ-ಯುಕ್ತ-ನಾಗಿ
ಜ್ಞಾನ-ವನ್ನು
ಜ್ಞಾನ-ವಿಶೇಷ-ವನ್ನೂ
ಜ್ಞಾನ-ಸುಖಶಕ್ತಿ
ಜ್ಞಾನಾ-ನಂತರ
ಜ್ಞಾನಾ-ನಂದಾದಿ
ಜ್ಞಾನೋಪದೇಶ
ಜ್ಞಾನೋಪದೇಶ-ವನ್ನು
ಡಂಗುರ
ಡೌನ್ನರು
ಡ್ಯೂತಕ್ಕೆ
ಡ್ಯೂತ-ದಲ್ಲಿ
ತಂ
ತಂಗಿ-ಯಾದ
ತಂದ
ತಂದಿಟ್ಟು
ತಂದಿತ್ತನೋ
ತಂದು-ಕೊಡು-ವುದು
ತಂದೆ-ತಾಯಿಯ-ರನ್ನು
ತಂದೆಯ
ತಂದೆ-ಯಾದ
ತಕಲಿಂ
ತಕ್ಕಂತೆ
ತಕ್ಕಷ್ಟು
ತಕ್ಷ-ಕನ
ತಡೆ-ದುದು
ತಡೆಯಲ್ಪಟ್ಟರೂ
ತಡೆಯು-ವುದು
ತತ
ತತ್ತೋಪದೇಶ
ತತ್ತೋಪದೇಶ-ಮಾಡಲು
ತತ್ಪರ-ರಾಗಿ-ರುವ-ರುಇವೇ
ತತ್ರಾವಿರ್ಭವಿತುಂ
ತತ್ರೋಪದೇಶ
ತತ್ವ
ತತ್ವ-ಗಳ
ತತ್ವ-ಗಳು
ತತ್ವಜ್ಞಾನ-ವನ್ನು
ತತ್ವ-ವನ್ನು
ತತ್ವಾಭಿಮಾನಿ
ತತ್ಸು
ತಥಾ
ತದ್ಗ
ತದ್ಘಾವಮುಚ್ಚೈಃ
ತದ್ದತ್ತಾಸ್ತ್ರ
ತದ್ದೇಹಸ್ಥ
ತನಗೆ
ತನ್ನ
ತನ್ನದೇ
ತನ್ನನ್ನು
ತನ್ನನ್ನೇ
ತನ್ನಲ್ಲಿಯೇ
ತಪಸ್ಸನ್ನಾಚರಿಸಿ
ತಪಸ್ಸನ್ನು
ತಪಸ್ಸಿನ
ತಪಸ್ಸು
ತಪ್ಪಾಗಿ
ತಪ್ಪಿದ್ದಲ್ಲ-ವೆಂಬ
ತಪ್ಪು-ಗಳನ್ನು
ತಮಗೆ
ತಮೀಡೇ
ತಮ್
ತಮ್ಮ
ತಮ್ಮಂದಿರ
ತಮ್ಮಂದಿ-ರನ್ನು
ತಮ್ಮಂದಿ-ರನ್ನೂ
ತಮ್ಮಂದಿ-ರಿಂದ
ತಮ್ಮಂದಿ-ರೊಡನೆ
ತಮ್ಮ-ನಾದ
ತಮ್ಮನ್ನು
ತಮ್ಮಿಂದ
ತರಲು
ತರಲ್ಪಟ್ಟ
ತರಿಸಿ-ಕೊಡು-ವುದು
ತರುತ್ತಿದ್ದುದು
ತರುವಾಗ
ತರು-ವುದು
ತಲೆ
ತಲೆ-ಯನ್ನು
ತಲೆ-ಯಲ್ಲಿಟ್ಟು
ತಸುತಂ
ತಸ್ಮಾದಾಪ್ತೋರು
ತಸ್ಯ
ತಾಂ
ತಾಟಕಿ-ಯನ್ನು
ತಾಟಕಿಸು-ಬಾಹು-ಗಳನ್ನು
ತಾಡನ
ತಾಡನಕ್ಕೆ
ತಾತ್ಪರ್ಯ
ತಾತ್ಪರ್ಯ-ವನ್ನು
ತಾತ್ಪರ್ಯವು
ತಾನಪಿ
ತಾನು
ತಾನೂ
ತಾನೊಬ್ಬನೆ
ತಾಯಿಯ
ತಾಯಿ-ಯನ್ನೂ
ತಾಯಿ-ಯಾದ
ತಾಯಿಯು
ತಾರ-ತಮ್ಮ
ತಾರ-ತಮ್ಯ
ತಾರ-ತಮ್ಯಂ
ತಾರ-ತಮ್ಯ-ದಿಂದ
ತಾರ-ತಮ್ಯ-ವನ್ನೂ
ತಾರ್ಕ್ಷ್ಯಾಲಬ್ದ
ತಾಲಾನೃಪ್ತ
ತಾಳ-ವೃಕ್ಷ-ಗಳನ್ನು
ತಾಳಿ
ತಾವು
ತಾವೂ
ತಿಂಗಳಿನ
ತಿಂಗಳು
ತಿಂದು
ತಿನ್ನುತ್ತಾ
ತಿನ್ನು-ವುದು
ತಿರಸ್ಕಾರ
ತಿರಸ್ಕೃತ-ನಾದನೋ
ತಿರುಗಿ
ತಿಲಾಂಜಲಿ-ಯನ್ನು
ತಿಳಿದ
ತಿಳಿದಿರು-ವುದು
ತಿಳಿದು
ತಿಳಿದು-ಕೊಂಡು
ತಿಳಿ-ಯಲು
ತಿಳಿಯಲ್ಪಡಬೇಕಾದ
ತಿಳಿಸಿ
ತಿಳಿಸಿ-ದರೂ
ತಿಳಿಸುವ
ತಿಳಿಸು-ವುದು
ತೀಕ-ವಾದ
ತೀರಿಸಿಕೊಳ್ಳಲು
ತೀರ್ಣಾಬ್ಧಿಃ
ತೀರ್ಥ
ತೀರ್ಥ-ಯಾತ್ರೆ-ಗಾಗಿ
ತೀರ್ಥ-ಯಾತ್ರೆ-ಯನ್ನು
ತೀರ್ಥ-ರೂಪರೂ
ತೀರ್ಮಾನ-ವಾದ
ತುಂಬಿ
ತುಂಬು-ವುದು
ತುಳಿದು
ತುಳಿಯು-ವುದು
ತುಷ್ಟುವುಸ್ತಂ
ತೃಣಾ-ವರ್ತನ
ತೃಣಾ-ವರ್ತ-ನನ್ನೂ
ತೃಣಾ-ವರ್ತರ
ತೃತೀಯ
ತೃಪ್ತಿ
ತೆಗೆದು
ತೆಗೆದು-ಕೊಂಡು
ತೆಗೆದು-ಕೊಂಡು-ಹೋದ
ತೆಗೆದು-ಕೊಳ್ಳು-ವುದು
ತೆರಳಲು
ತೆರಳಿ
ತೆರಳಿದ
ತೆರಳುವದು
ತೆರಳು-ವುದು
ತೈಃ
ತೊಂದರೆ
ತೊಂಭತ್ತೆಂಟು
ತೊಡೆ-ಗಳನ್ನು
ತೊಡೆ-ಯನ್ನು
ತೊಡೆಯ-ಮೇಲೆ
ತೊಡೆ-ಯಲ್ಲಿ
ತೊಡೆ-ಯಲ್ಲಿ-ರಿ-ಸು-ವು-ದಾಗಿ
ತೊರೆದು
ತೋರಿ-ಸಲು
ತೋರಿಸಿ
ತೋರಿ-ಸಿದ
ತೋರಿಸಿ-ದುದು
ತೋರಿ-ಸಿದೆ
ತೋರಿ-ಸು-ವುದು
ತ್ಯಕ್ತುಮಥೋದ್ಯತೋ
ತ್ಯಕ್ತ್ವಾ
ತ್ಯಜಿ-ಸಲು
ತ್ಯಜಿಸಿ
ತ್ರಿವಕ್ರೆ
ತ್ರಿವಕ್ರೆಗೆ
ತ್ರಿವಿಧ-ರಾದ
ತ್ವಕೃತ-ನಿಜ-ಮನೋಽಭೀಷ್ಟವಂತಂ
ದಂಡನೆ
ದಂತ-ಗಳ
ದಂತವಕ್ತಾದಿ-ಗಳ
ದಂತವಕ್ರನ
ದಂತೆ
ದಂಪತಿ-ಗಳಲ್ಲಿ
ದಕ್ಷಿಣ
ದಕ್ಷಿಣಕ್ಕೆ
ದಕ್ಷಿಣ-ಮಾ-ಗತೋ
ದಕ್ಷಿಣ-ಸಮುದ್ರಕ್ಕೆ
ದಗ್ಧೇ
ದಗ್ಧ್ವಾ
ದಡ-ದಲ್ಲಿ
ದದೌ
ದಮಘೋಷ-ನಿಂದ
ದಯ
ದಯಪಾಲಿಸಿ
ದಯಪಾಲಿಸಿ-ದನೋ
ದಯ-ಪಾಲಿಸು-ವುದು
ದಯೆಯ
ದರ್ಶನ
ದರ್ಶನ-ಕೊಡು-ವುದು
ದರ್ಶನ-ಭಾಷೆ
ದರ್ಶನ-ವನ್ನು
ದಲ್ಲಿ
ದಲ್ಲಿಯೇ
ದವ-ರಿಗೆ
ದಶಕಂ
ದಶಮುಖಭ್ರಾತ್ರಿಷ್ಟತೋಽಭ್ತರ್ಥಿತೋ
ದಶರಥನ
ದಶರಥನು
ದಶರಥ-ರಾಜ
ದಶರಥ-ರಾಜನ
ದಶರಥಾತ್ಸಾಯಾತ್ಸ
ದಹನ
ದಾಟಲು
ದಾಟಿ
ದಾಟುವುದ-ರಲ್ಲಿ
ದಾನ
ದಾನ-ಗಳನ್ನು
ದಾನಾದಿ-ಗಳನ್ನು
ದಾರಿ-ಯಲ್ಲಿ
ದಾರುಕ
ದಿಕಾನ್
ದಿಕ್ಕನ್ನು
ದಿಕ್ಕಿಗೆ
ದಿಕ್ಕಿ-ನಲ್ಲಿ
ದಿಕ್ಕು-ಗಳಿಗೂ
ದಿಕ್ಕು-ಗಳಿಗೆ
ದಿಗ್ವಿಜಯ
ದಿಗ್ವಿಜಯಕ್ಕಾಗಿ
ದಿತಿಅದಿತಿ-ದೇವಿ-ಯ-ರಿಂದ
ದಿನ-ಗಳ
ದಿನದಿ-ನಕ್ಕೆ
ದಿನ-ವನ್ನು
ದಿವಂಗತ
ದಿವಸ
ದಿವಸ-ಗಳು
ದಿವ್ಯ
ದಿವ್ಯ-ರೂಪ-ವನ್ನು
ದಿಶಂ
ದೀಪ-ಗಳನ್ನು
ದುಂದುಭಿಯ
ದುಃಖ
ದುಃಖ-ಗಳನ್ನು
ದುಃಖ-ತಪ್ತ-ರಾದ
ದುಃಖ-ವನ್ನು
ದುಃಖಾನುಭವ-ದಲ್ಲಿ
ದುಃಖಿತನಾಗು-ವುದು
ದುಃಖಿತನಾಗು-ವುದುಈ
ದುಃಶಲಾ
ದುಃಶಾಸನನ
ದುಃಶಾಸನರ
ದುದು
ದುರ್ಗಂಧ-ದಿಂದ
ದುರ್ಗಾ-ದೇವಿ-ಯನ್ನು
ದುರ್ಗೆಯು
ದುರ್ಬೋಧನೆ
ದುರ್ಮಷ್ರಣ-ನಾಮಕ
ದುರ್ಮುಖ-ನಾಮಕ
ದುರ್ಯೊಧನ
ದುರ್ಯೊಧನ-ಅಭಿಮನ್ಯು-ವಿನ
ದುರ್ಯೊಧನ-ಘಟೋತ್ಕಚನ
ದುರ್ಯೊಧನನ
ದುರ್ಯೊಧನನು
ದುರ್ಯೊಧನನೇ
ದುರ್ಯೊಧನ-ನೊಡನೆ
ದುರ್ಯೊಧ-ನಾದಿ-ಗಳ
ದುರ್ಯೊಧ-ನಾದಿ-ಗಳನ್ನು
ದುರ್ಯೊಧ-ನಾದಿ-ಗಳಿಂದ
ದುರ್ಯೊಧ-ನಾದಿ-ಗಳಿಗೂದ್ರುಪದ-ನಿಗೂ
ದುರ್ಯೊಧ-ನಾದಿ-ಗಳಿಗೆ
ದುರ್ಯೊಧ-ನಾದಿ-ಗಳು
ದುರ್ಯೊಧ-ನಾದೀನ್
ದುರ್ಯೋಧನನ
ದುರ್ಯೋಧನ-ನನ್ನು
ದುರ್ಯೋಧನ-ನನ್ನೂ
ದುರ್ಯೋಧನ-ನಿಗೂ
ದುರ್ಯೋಧನ-ನಿಗೆ
ದುರ್ಯೋಧನನು
ದುರ್ಯೋಧನಾ-ದಿ-ಗಳು
ದುರ್ಯೋಧನೇ
ದುರ್ಲಕ್ಷಣ
ದುರ್ಲಕ್ಷಣ-ಗಳು
ದುರ್ವಾಸ-ರಿಂದ
ದುಶ್ಯಕುನ-ಗಳು
ದುಶ್ಯಾಸನನ
ದುಶ್ಯಾಸನ-ನನ್ನು
ದುಷ್ಟನು
ದುಷ್ಟ-ರನ್ನು
ದುಷ್ಟರನ್ನೆಲ್ಲ
ದೂತ-ನನ್ನು
ದೂತ-ನಿಂದ
ದೂತಪಾಶ-ದಿಂದ
ದೂರಕ್ಕೆ
ದೂರ-ದಲ್ಲಿ
ದೂರ್ವಾಸ-ಮುನಿ-ಗಳಿಗೆ
ದೃಶ್ಯದ
ದೃಶ್ಯ-ವನ್ನು
ದೃಷ್ಟಿ-ಯನ್ನು
ದೃಷ್ಟಿ-ಯಿಂದ
ದೇವಕಿ
ದೇವಕಿಯ
ದೇವಕಿ-ಯ-ರನ್ನು
ದೇವಕಿ-ವಸು-ದೇವ-ರಲ್ಲಿ
ದೇವಕ್ಕಾಂ
ದೇವತಾ
ದೇವತಾ-ಸಮೂಹ-ಗಳಿಂದ
ದೇವತಾತ್ರೀ-ಯರ
ದೇವತೆ-ಗಳ
ದೇವತೆ-ಗಳನ್ನು
ದೇವತೆ-ಗಳನ್ನೂ
ದೇವತೆ-ಗಳಲ್ಲಿ
ದೇವತೆ-ಗಳಲ್ಲಿ-ರುವ
ದೇವತೆ-ಗಳಿಂದ
ದೇವತೆ-ಗಳಿಂದಲೂ
ದೇವತೆ-ಗಳಿಗೆ
ದೇವತೆ-ಗಳು
ದೇವತೆ-ಗಳೂ
ದೇವತೆ-ಗಳೊಂದಿಗೆ
ದೇವತೆ-ಗಳೊಡನೆ
ದೇವರ
ದೇವ-ರನ್ನು
ದೇವ-ರಲ್ಲಿ
ದೇವ-ರಿಂದ
ದೇವರಿಗೂ
ದೇವರು
ದೇವ-ಲೋಕ-ಗಳಲ್ಲಿಯೂ
ದೇವಸಂಘೈಃ
ದೇವಾನ್ಸುರಾನುಕ್ರಮಾತ್
ದೇವಾಸುರರ
ದೇವಿಗೆ
ದೇವಿಯ
ದೇವಿ-ಯನ್ನೂ
ದೇವಿ-ಯರ
ದೇವಿಯು
ದೇವಿ-ಯೊಡನೆ
ದೇವೇಂದ್ರ-ನಿಂದ
ದೇವೇಂದ್ರನು
ದೇವೈಃ
ದೇವೈರ್ಗೃಣದ್ಭಿಸ್ತುತಃ
ದೇಹ
ದೇಹ-ಗಳ
ದೇಹತ್ಯಾಗ
ದೇಹತ್ಯಾಗ-ವೆಂಬ
ದೇಹ-ದಲ್ಲಿ
ದೇಹ-ದಲ್ಲಿ-ರುವ
ದೇಹ-ಪತನ
ದೇಹ-ವನ್ನು
ದೈತ-ವನಕ್ಕೆ
ದೈತ್ಯ
ದೈತ್ಯಂ
ದೈತ್ಯನ
ದೈತ್ಯನು
ದೈತ್ಯರ
ದೈತ್ಯ-ರನ್ನು
ದೈತ್ಯ-ರಲ್ಲಿ
ದೈತ್ಯ-ರಿಗೆ
ದೈತ್ಯಾ
ದೊರಕಿಸಿ-ಕೊಟ್ಟ
ದೊರಕಿಸಿ-ಕೊಟ್ಟನೋ
ದೊರೆತುದು
ದೊರೆ-ಯುವ
ದೋಣಿ-ಯಲ್ಲಿ
ದೋಷ-ಗಳ
ದೋಷ-ಗಳನ್ನು
ದೋಷ-ಪರಿ-ಹಾರ-ವನ್ನು
ದೋಷರ-ಹಿತ-ವಾದ
ದೋಷರಾ-ಹಿತ್ಯ
ದೋಷಾಭಾವ
ದೌಪದೀ
ದೌಪದೀ-ದೇವಿಯು
ದೌಮ್ಯ-ರನ್ನು
ದ್ಯಃ
ದ್ಯುನಾನಕ
ದ್ಯುಸರಿತ್ಸುತೋ
ದ್ಯೂತೇ
ದ್ಯೋತಕ-ಗ-ಳಾದ
ದ್ಯೋತಕ್ಕೆ
ದ್ಯೋತವನ್ನಾಡಲು
ದ್ರವ್ಯಕ್ಕಾಗಿ
ದ್ರವ್ಯಕ್ಕಾಗಿಯೂ
ದ್ರವ್ಯ-ವನ್ನು
ದ್ರವ್ಯ-ವನ್ನೂ
ದ್ರುಪದ
ದ್ರುಪದನ
ದ್ರುಪದ-ನನ್ನು
ದ್ರುಪದ-ನ-ಬಳಿಗೆ
ದ್ರುಪದ-ನಿಂದ
ದ್ರುಪದ-ನಿಗೆ
ದ್ರುಪದನು
ದ್ರುಪದ-ನು-ಪ-ಗತೋ-ಽನಾಪ್ತ
ದ್ರುಪದ-ರಾಜನ
ದ್ರುಪದ-ರಾಜ-ನನ್ನು
ದ್ರುಪದ-ರಾಜನು
ದ್ರುಪದ-ರಾಜರ
ದ್ರುಪದರು
ದ್ರೋಣ
ದ್ರೋಣಃ
ದ್ರೋಣ-ಅರ್ಜುನರ
ದ್ರೋಣ-ಕರ್ಣ-ದುರ್ಯೊಧ-ನಾದಿ-ಗಳ
ದ್ರೋಣ-ನಾಗಿ
ದ್ರೋಣ-ನಿಗೆ
ದ್ರೋಣನು
ದ್ರೋಣರ
ದ್ರೋಣ-ರನ್ನು
ದ್ರೋಣ-ರನ್ನೂ
ದ್ರೋಣ-ರಿಂದ
ದ್ರೋಣ-ರಿಗೆ
ದ್ರೋಣರು
ದ್ರೋಣಾಚಾರ್ಯರ
ದ್ರೋಣಾ-ಚಾರ್ಯ-ರನ್ನು
ದ್ರೋಣಾಚಾರ್ಯರು
ದ್ರೋಣಿನಾ
ದ್ರೋಣೇ
ದ್ರೌಪದಿ
ದ್ರೌಪದಿಗೂ
ದ್ರೌಪದಿಗೆ
ದ್ರೌಪದಿ-ಧರ್ಮ
ದ್ರೌಪದಿಯ
ದ್ರೌಪದಿ-ಯನ್ನು
ದ್ರೌಪದಿ-ಯನ್ನೂ
ದ್ರೌಪದಿ-ಯಿಂದ
ದ್ರೌಪದಿ-ಯಿಂದಲೂ
ದ್ರೌಪದಿಯು
ದ್ರೌಪದಿಯೇ
ದ್ರೌಪದೀ
ದ್ರೌಪದೀ-ದೇವಿಗೆ
ದ್ರೌಪದೀ-ದೇವಿಯ
ದ್ರೌಪದೀ-ದೇವಿ-ಯನ್ನು
ದ್ರೌಪದೀ-ದೇವಿ-ಯರ
ದ್ರೌಪದೀ-ದೇವಿ-ಯ-ರನ್ನು
ದ್ರೌಪದೀ-ದೇವಿ-ಯ-ರಲ್ಲಿ
ದ್ರೌಪದೀ-ದೇವಿ-ಯರು
ದ್ರೌಪದೀ-ದೇವಿ-ಯ-ರೊಡನೆ
ದ್ರೌಪದೀ-ದೇವಿ-ಯಲ್ಲಿ
ದ್ರೌಪದೀ-ದೇವಿ-ಯಿಂದ
ದ್ರೌಪದೀಮ್
ದ್ರೌಪದೇಃ
ದ್ವಂದ್ವ-ಯುದ್ಧ-ಗಳ
ದ್ವಯಂ
ದ್ವಾಪರ-ಯುಗ-ದಲ್ಲಿ
ದ್ವಾರಕಾ
ದ್ವಾರಕಾಕ್ಕೆ
ದ್ವಾರಕಾ-ನಗರಕ್ಕೆ
ದ್ವಾರಕಾ-ನಗರದ
ದ್ವಾರಕಾ-ಪಟ್ಟಣ
ದ್ವಾರಕಿಗೆ
ದ್ವಾರಕಿ-ಯಲ್ಲಿ
ದ್ವಾರಕೆಗೆ
ದ್ವಾರ-ದಲ್ಲಿ
ದ್ವಾರಾವತಿ-ಯಿಂದ
ದ್ವಾರಾವತಿಯು
ದ್ವಿಜೈ
ದ್ವಿಜೌಘಾನ್
ದ್ವಿಷದಸುಖಕರಃ
ದ್ವೇಷ
ದ್ವೇಷ-ಗಳು
ದ್ವೇಷಿ-ಗಳನ್ನು
ದ್ವೇಷಿ-ಗಳಿಗೆ
ದ್ವೇಷಿಸುವವ-ರಿಗೆ
ದ್ವೌ
ಧನ-ರಾಜನು
ಧನ-ವನ್ನೂ
ಧನುರ್ಯುದ್ದ
ಧನುಸ್ಸನ್ನು
ಧನುಸ್ಸಿನ
ಧನುಸ್ಸು-ಗಳನ್ನು
ಧನ್ವಂತರೀ
ಧರಿ-ಸಲ್ಪಟ್ಟ
ಧರಿಸಿ
ಧರಿಸಿ-ದುದೂ
ಧರಿಸುವುದರ
ಧರ್ಮ
ಧರ್ಮ-ಅರ್ಥ-ಕಾಮ
ಧರ್ಮ-ಕಾರ್ಯ-ಗಳನ್ನು
ಧರ್ಮ-ಕಾರ್ಯ-ಗಳಿಂದ
ಧರ್ಮ-ಗಳನ್ನು
ಧರ್ಮ-ಗಳು
ಧರ್ಮಜ
ಧರ್ಮ-ಜ-ನನ್ನು
ಧರ್ಮದ
ಧರ್ಮ-ದಿಂದ
ಧರ್ಮ-ದಿಂದಲೂ
ಧರ್ಮ-ನಿರತೋ
ಧರ್ಮಪ್ರಚಾರಕ್ಕಾಗಿ
ಧರ್ಮಪ್ರಸಾದಂ
ಧರ್ಮ-ಮರುದ್
ಧರ್ಮ-ರಾಜ
ಧರ್ಮ-ರಾಜ-ದುರ್ಯೋಧನರ
ಧರ್ಮ-ರಾಜನ
ಧರ್ಮ-ರಾಜ-ನನ್ನು
ಧರ್ಮ-ರಾಜ-ನಿಂದ
ಧರ್ಮ-ರಾಜ-ನಿಗೂ
ಧರ್ಮ-ರಾಜ-ನಿಗೆ
ಧರ್ಮ-ರಾಜನು
ಧರ್ಮ-ರಾಜ-ಭೀಮಸೇನರು
ಧರ್ಮ-ರಾಜ-ಶಲ್ಯ
ಧರ್ಮ-ರಾಜತ್ವರಾವಾನ್
ಧರ್ಮ-ರಾಜಾದಿ-ಗಳಿಗೆ
ಧರ್ಮ-ರಾಜಾ-ದಿ-ಗಳು
ಧರ್ಮ-ವನ್ನು
ಧರ್ಮ-ವೆಂದು
ಧರ್ಮ-ಸುತಾದ್ಧತೇ
ಧರ್ಮಸ್ವ-ರೂಪ-ವನ್ನು
ಧರ್ಮಾತ್ಮಜಂ
ಧರ್ಮಾ-ಧರ್ಮ-ಗಳ
ಧರ್ಮೇ
ಧರ್ಮೋಪದೇಶ
ಧಾರೆ-ಯಾಗಿ
ಧಿಕ್ಕರಿಸಿ
ಧೀಮಹಿ
ಧುರ್ಯೋಧನ-ನಿಗೆ
ಧೃತಧರಣಿಧರೋ
ಧೃತರಾಷ್ಟ್ರ
ಧೃತರಾಷ್ಟ್ರ-ಗಾಂಧಾರಿ-ಯನ್ನು
ಧೃತರಾಷ್ಟ್ರನ
ಧೃತರಾಷ್ಟ್ರ-ನನ್ನು
ಧೃತರಾಷ್ಟ್ರ-ನಿಗೆ
ಧೃತರಾಷ್ಟ್ರನು
ಧೃಷ್ಟದ್ಯುಮ್ಮ
ಧೃಷ್ಟದ್ಯುಮ್ಮ-ಅಶ್ವತ್ಥಾಮರ
ಧೃಷ್ಟದ್ಯುಮ್ಮದ್ರೌಪದೀ
ಧೃಷ್ಟದ್ಯುಮ್ಮನ
ಧೃಷ್ಟದ್ಯುಮ್ಮ-ನನ್ನು
ಧೃಷ್ಟದ್ಯುಮ್ಮನು
ಧೇನುಕಾ-ಸುರನ
ಧೇನೂಶ್ಚ
ಧೈಯನು
ಧ್ಯಾನಿಸುತ್ತೇನೆ
ಧ್ಯಾಯಾಮಿ
ಧ್ವಂಸ
ಧ್ವನಿ-ಯಂತೆ
ಧ್ವನಿ-ಯನ್ನು
ನಂತರ
ನಂತರವೂ
ನಂದ
ನಂದ-ಕುಮಾರ-ನನ್ನು
ನಂದ-ಗೋಕುಲಕ್ಕೆ
ನಂದ-ಗೋಪನ
ನಂದ-ಗೋಪ-ನನ್ನು
ನಂದ-ಗೋಪನು
ನಂದ-ಗೋಪಯಶೋದೆ-ಯರ
ನಂದ-ಗೋಪಾದಿ-ಗಳಿಗೆ
ನಂದ-ನ-ವನ-ದಲ್ಲಿ
ನಂದಾತ್ಮಜಂ
ನಂದಾದೀನುದ್ಧವೋಕ್ತ್ಯಾ
ನಂಬಿಕೆ
ನಃ
ನಕುಲ
ನಕುಲ-ನಿಂದ
ನಕುಲನು
ನಕುಲ-ಸಹ-ದೇ-ವರು
ನಗರಕ್ಕೆ
ನಗರದ
ನಗರ-ದಲ್ಲಿ
ನಗರ-ದಿಂದ
ನಗರ-ವಾದ
ನಗರೀಂ
ನಗು-ವುದು
ನಟನೆ
ನಟಿಸಿ
ನಟಿ-ಸಿದ
ನಟಿಸಿ-ದುದು
ನಟಿ-ಸು-ವುದು
ನಡೆದ
ನಡೆದು
ನಡೆಯ-ಬಹು-ದಾದ
ನಡೆಯುತಾಳೆಂಬ
ನಡೆಸಿ
ನಡೆ-ಸು-ವುದು
ನದಿಯ
ನದಿ-ಯಲ್ಲಿ
ನನಗೆ
ನನ್ನ
ನನ್ನನು
ನನ್ನನ್ನು
ನನ್ನಲ್ಲಿ
ನನ್ನು
ನನ್ನೂ
ನಪುಂಸ-ಕನಾಗೆಂದು
ನಮಸ್ಕರಿಸಿ
ನಮಸ್ಕರಿಸಿ-ದನೋ
ನಮಸ್ಕರಿಸುತ್ತೇನೆ
ನಮಸ್ಕರಿಸು-ವಂತೆ
ನಮಸ್ಕರಿಸುವಾಗ
ನಮಸ್ಕರಿ-ಸು-ವುದು
ನಮಸ್ಕಾರ
ನಮಿಸುತ್ತೇನೆ
ನಮ್ಮ
ನಮ್ಮನು
ನಮ್ಮನ್ನು
ನರಕವು
ನರಕಾದಿ
ನರಕಾ-ಸುರನ
ನರಕಾ-ಸುರ-ನನ್ನು
ನರಸಿಂಹ-ರೂಪಿ
ನರಸಿಂಹಾವ-ತಾರ
ನರಸಿಂಹೋಖಿಲಾಜ್ಞಾನ-ಮತಧಾಂತದಿವಾಕರಃ
ನರಾಂಶ-ವನ್ನು
ನರಿ-ಗಳು
ನವನೀತಕ್ಷೀರ
ನವವಿಧ-ವಾದ
ನಹುಷನ
ನಹುಷ-ನನ್ನು
ನಹುಷ-ನಿಗೆ
ನಾಗ-ಪತ್ನಿ
ನಾಗಾತ್ಮ-ಕ-ವಾದ
ನಾಗಾತ್ರ-ವನ್ನು
ನಾಗಾತ್ರವು
ನಾಗಾಸ್ವ-ದಿಂದ
ನಾಚಿಕೆ
ನಾಟ್ಯಗಾಯನ
ನಾಥಂ
ನಾದಾದನುಜಜನಿ-ತಸ್ಯಾಂಬಿಕೇಯೋರ್ಜುನಸ್ಯ
ನಾನಾ
ನಾನಾ-ಪದೋ
ನಾನೇ
ನಾಮ
ನಾಮ-ಕರಣಾದಿ
ನಾಯಿ
ನಾಯಿ-ಯಿಂದಲೂ
ನಾರದ
ನಾರದ-ರಿಂದ
ನಾರದರು
ನಾರದಶ್ರೀ-ಕೃಷ್ಣರು
ನಾರಾಯಣ
ನಾರಾಯಣ-ನಿಂದ
ನಾರಾಯಣಮ್
ನಾರಾಯಣರ
ನಾರಾಯಣಾತ್ರಕ್ಕೆ
ನಾರಾಯಣಾತ್ರ-ವನ್ನು
ನಾಲ್ಕು
ನಾಶ
ನಾಶ-ಪಡಿಸಿದ
ನಾಶ-ಪಡಿಸಿ-ದನೋ
ನಾಶ-ಪಡಿ-ಸು-ವುದು
ನಾಶ-ಮಾಡಲು
ನಾಶ-ಮಾಡಿ
ನಾಶ-ಮಾಡಿಸಿ
ನಾಶ-ಮಾಡುವುದು
ನಾಶ-ರ-ಹಿತನು
ನಾಶ-ವಾಗು-ವುದು
ನಾಶ-ವಾ-ದುದು
ನಾಸ್ತಿ-ಕ-ವಾದಕ್ಕೆ
ನಾಸ್ತಿಕ್ಯದ
ನಾಸ್ತಿಕ್ಯ-ಧರ್ಮ-ವನ್ನು
ನಿಂತಾಗ
ನಿಂತು
ನಿಂದ
ನಿಂದನೇ
ನಿಂದಿತಭಿಕ್ಷುಕೇ
ನಿಂದಿ-ಸಿದ
ನಿಂದಿ-ಸಿದಾಗ
ನಿಂದಿ-ಸು-ವುದು
ನಿಂದೆ
ನಿಂದೆ-ಯನ್ನು
ನಿಕುಂಭ
ನಿಖಿಲ
ನಿಗೆ
ನಿಗ್ರಹ
ನಿಗ್ರಹಿ-ಸಲಿ-ರುವ
ನಿಗ್ರಹಿ-ಸಿದ
ನಿಗ್ರಹಿ-ಸಿದನೋ
ನಿಘ್ನನ್
ನಿಘ್ನಾಂ
ನಿಜ
ನಿಜ-ರೂಪ-ದಿಂದ
ನಿಜ-ವಾದ
ನಿಜಾತ್ರ-ನಮ್ರಮಕರೋತ್ತಂ
ನಿತ್ಯ
ನಿತ್ಯ-ದಲ್ಲಿಯೂ
ನಿತ್ಯ-ವಾದ
ನಿದ್ದೆ-ಯಿಂದ
ನಿದ್ರಾಭಂಗ
ನಿಧನ
ನಿಧನೇ
ನಿಧಾಯ
ನಿಪತಿತಾ
ನಿಮಾಣ
ನಿಮಿತ್ತ-ಗಳನ್ನು
ನಿಮಿತ್ತ-ದಿಂದ
ನಿಯತ-ವಾದ
ನಿಯಮ
ನಿಯಮದ
ನಿಯಮ-ವನ್ನು
ನಿಯಮಿಸಿ
ನಿಯಮಿ-ಸು-ವುದು
ನಿಯುಕ್ತೋ
ನಿರಂತರವೂ
ನಿರತಂ
ನಿರತರೂ
ನಿರಪಾತಯತ್ಸುತಶರೈಃ
ನಿರಶನವ್ರತ
ನಿರಸನ
ನಿರಸ್ಯ
ನಿರಾಕರಣೆ
ನಿರಾಯಾಸ-ವಾಗಿ
ನಿರಾಶೆ-ಯನ್ನು
ನಿರೀಕ್ಷಿ-ಸುತ್ತಾ
ನಿರೂಪಣೆ
ನಿರೂಪಿ-ತ-ವಾಗಿದೆ
ನಿರೂಪಿ-ತ-ವಾಗಿವೆ
ನಿರೂಪಿ-ಸಲ್ಪಟ್ಟಿವೆ
ನಿರೂಪಿಸಿ
ನಿರೂಪಿ-ಸು-ವುದು
ನಿರ್ಗಮನ
ನಿರ್ಣಯ
ನಿರ್ಣಯಗ್ರಂಥ-ವನ್ನು
ನಿರ್ಣಯದ
ನಿರ್ಣಯ-ದಿಂದ
ನಿರ್ಣಯಿ-ಸು-ವುದು
ನಿರ್ಣಾಯಕ-ವಾಗಿ
ನಿರ್ದಿಷ್ಟ-ವಾದ
ನಿರ್ದುಃಖೋಥ
ನಿರ್ಧರಿ-ಸು-ವುದು
ನಿರ್ಧಾರ
ನಿರ್ನಾಮ
ನಿರ್ಮಾಣ
ನಿರ್ಮಾಣ-ಮಾಡಿಸಿ
ನಿರ್ಮಾಣ-ವಾಗು-ವುದು
ನಿರ್ಮಿತ-ವಾದ
ನಿರ್ಮಿಸಿ
ನಿರ್ಮಿ-ಸು-ವುದು
ನಿರ್ಮೂಲ
ನಿಲುಲಿತೇ
ನಿಲ್ಲಿಸು-ವಂತೆ
ನಿಲ್ಲಿ-ಸು-ವುದು
ನಿವಾತಕವಚ
ನಿವಾತಕವಚ-ರನ್ನೂ
ನಿವಾತಕವಚಾದಿ-ಗಳನ್ನು
ನಿವಾರಣೆ
ನಿವಾರಿಸು-ವಂತೆ
ನಿವಾರಿ-ಸು-ವುದು
ನಿವೃತ್ತಿ
ನಿವೃತ್ತಿ-ಯಾ-ಗದೇ
ನಿಶಾಮ್ಯ
ನಿಶಿ
ನಿಶಿ-ಚರೀ
ನಿಶಿ-ತತೋ
ನಿಶ್ಚಯವಾ-ದುದು
ನಿಶ್ಚಯವೆಂದೂ
ನಿಶ್ಚಯಿಸಿ
ನಿಶ್ಚಿತ-ವಾದ
ನಿಷ್ಟುರ-ವಾಕ್ಯ-ಗಳಿಂದ
ನಿಸ್ತೀರ್ಯ
ನಿಹತ್ಯ
ನಿಹಿತ-ನಿಶಿ-ಚರಾಃ
ನೀಚ-ಕೃತ್ಯ-ಗಳನ್ನು
ನೀಚೋಚ್ಚಭಾವೈಃ
ನೀಡಿ
ನೀಡು-ವುದು
ನೀತ್ವಾಽರ್ಜುನಂ
ನೀನು
ನೀರನ್ನು
ನೀರಿ-ನಲ್ಲಿ
ನೀಲ
ನೀಲ-ನಾಗಿ
ನೀಲಾ
ನೀಲಾ-ದೇವಿ
ನು
ನುಂಗಿ
ನುಗ್ಗಿ
ನುಡಿ-ಗಳು
ನುಡಿದು
ನುಡಿಯು-ವುದು
ನೂರ-ಏಳು
ನೂರ-ಐದು-ಮಂದಿ
ನೂರು
ನೃಗು-ಋಷಿ-ಗಳೇ
ನೃತ್ಯ-ರಿಗೆ
ನೃಪಂ
ನೃಪತೀನರೀನ್
ನೃಪನುತಃ
ನೃಪಸುತೋ
ನೆಂಬ
ನೆರವೇರಿ-ಸಿದ
ನೆರವೇರಿ-ಸಿದನೋ
ನೆರವೇರಿ-ಸು-ವುದು
ನೇತೃತ್ವ-ದಲ್ಲಿ
ನೇಮಿ-ಸು-ವುದು
ನೊಡನೆ
ನೋಟ-ದಿಂದ
ನೋಡಬೇಕೆಂದು
ನೋಡಲಾರದೆ
ನೋಡಲು
ನೋಡಿ
ನೋಡಿದ
ನೋಡಿ-ದುದು
ನೋಡುತ್ತಿ-ರು-ವಂತೆಯೇ
ನೋಡು-ವುದು
ನೋಽವ್ಯಾತ್
ನೋಽವ್ಯಾದ್ಧರಿಃ
ನೌಮಿ
ನ್ಯಹನದುಪ-ಗತೋ-ಽವ್ಯಾತ್ಸ
ನ್ಯಾಯ-ದಿಂದ
ಪಂಚ
ಪಂಚ-ವರ್ಣ-ಗಳ
ಪಟ್ಟ-ಣಕ್ಕೆ
ಪಟ್ಟ-ಣ-ದಲ್ಲಿ-ರುವ-ವ-ರನ್ನು
ಪಟ್ಟ-ಣ-ದಿಂದ
ಪಟ್ಟ-ಣ-ವನ್ನು
ಪಟ್ಟ-ಣ-ವಾದ
ಪಟ್ಟಾಭಿಷೇಕ
ಪಟ್ಟಾಭಿಷೇಕಕ್ಕೆ
ಪಡಿ-ಸಲು
ಪಡಿಸಿ
ಪಡಿಸಿದ
ಪಡಿ-ಸು-ವುದು
ಪಡು-ವುದು
ಪಡೆ
ಪಡೆ-ದರೋ
ಪಡೆ-ದಿರುತ್ತೆ
ಪಡೆದು
ಪಡೆ-ದುದು
ಪಡೆ-ದುದೂ
ಪಡೆ-ಯಲು
ಪಡೆ-ಯುತ್ತಿದ್ದುದು
ಪಡೆ-ಯುತ್ತಿ-ರುವಾಗ
ಪಡೆ-ಯುವ
ಪಡೆ-ಯು-ವಂತೆ
ಪಡೆ-ಯು-ವುದು
ಪಣವನ್ನಿಟ್ಟು
ಪಣ-ವಾಗಿಟ್ಟು
ಪತನ
ಪತನ-ವಾಗಲು
ಪತಿ-ಗಳಿರಲು
ಪತಿ-ಯಾದ
ಪತಿ-ಯೊಡನೆ
ಪತ್ನಿ
ಪತ್ನಿಯ
ಪತ್ನಿ-ಯನ್ನಾಗಿ
ಪತ್ನಿ-ಯರು
ಪತ್ನಿ-ಯ-ರೊಂದಿಗೆ
ಪತ್ನಿ-ಯ-ರೊಡನೆ
ಪತ್ನಿ-ಯಾದ
ಪತ್ನಿ-ಯಾ-ದುದು
ಪತ್ನಿ-ಯಿಂದ
ಪದಂ
ಪದವಿ
ಪದವಿ-ಗಳಿಗೆ
ಪದವಿ-ಯನ್ನು
ಪದವೀಂ
ಪದ್ಮವ್ಯೂಹದ
ಪಪೌ
ಪಯೋನಿಧಿಃ
ಪರತಶ್ಚ
ಪರಮ
ಪರಮ-ಪುರುಷತಾಂ
ಪರಮಾತ್ಮನ
ಪರಮಾತ್ಮ-ನಿಂದ
ಪರಮಾತ್ಮ-ನಿಗೆ
ಪರ-ವಿದ್ಯೆ-ಯನ್ನೂ
ಪರಶು-ರಾಮ
ಪರಶು-ರಾಮ-ದೇ-ವ-ರಿಂದ
ಪರಶು-ರಾಮರ
ಪರಶು-ರಾಮ-ರನ್ನು
ಪರಶು-ರಾಮ-ರಿಂದ
ಪರಶು-ರಾಮರು
ಪರಶು-ರಾಮ-ವೇದವ್ಯಾಸ-ರೂಪ-ಗಳಿಂದ
ಪರಶು-ರಾಮಾವ-ತಾರ
ಪರಸ್ಪರ-ವಾಗಿ
ಪರಾಕ್ರಮ
ಪರಾಕ್ರಮದ
ಪರಾಕ್ರಮ-ದಿಂದ
ಪರಾಕ್ರಮ-ಶಾಲಿ-ಗಳನ್ನು
ಪರಾಜಯ
ಪರಾಜ-ಯ-ವನ್ನು
ಪರಾಜ-ಯ-ಹೊಂದಿದ
ಪರಾಜಿತ-ನಾಗಿ
ಪರಾಧೀನ-ವಿಶೇಷಾ
ಪರಾಭವ
ಪರಾಭವ-ಗೊಳಿಸಿ
ಪರಾಭವ-ಗೊಳಿಸಿದ
ಪರಾಭವ-ಗೊಳಿಸಿ-ದನೋ
ಪರಾಭವ-ಗೊಳಿ-ಸು-ವುದು
ಪರಾಭವ-ಗೊಳ್ಳು-ವುದು
ಪರಾಭವ-ವನ್ನು
ಪರಾಶರ
ಪರಾಶರ-ಸತ್ಯವತೀ
ಪರಾಶರಾಖ್ಯ
ಪರಿಚಯ
ಪರಿಚಯ-ದಿಂದ
ಪರಿಚಯ-ವನ್ನು
ಪರಿಚಾರಿಣಿ-ಯಿಂದ
ಪರಿತ್ಯಜಿಸಿ
ಪರಿತ್ಯಾಗ
ಪರಿತ್ಯಾಗ-ಮಾಡಲು
ಪರಿಪಾಲನೆ-ಯನ್ನು
ಪರಿಪಾಲಯನ್
ಪರಿಪಾಲಿಸಿ
ಪರಿಪಾಲಿ-ಸುತ್ತಾ
ಪರಿಪಾಲಿಸುತ್ತಿದ್ದುದು
ಪರಿ-ಪಾಲಿಸು-ವುದು
ಪರಿಪೂರ್ಣತ್ವ
ಪರಿಪೂರ್ಣ-ನಾದ
ಪರಿಮಳ
ಪರಿಮಳ-ಯುಕ್ತ-ವಾದ
ಪರಿವಾರ
ಪರಿವಾರ-ವರ್ಗದ-ವ-ರನ್ನೂ
ಪರಿಶೀಲಿ-ಸಿದರೆ
ಪರಿ-ಸಮಾಪ್ತಿ
ಪರಿ-ಸಮಾಪ್ತಿ-ಗೊಳಿಸಿ-ರುತ್ತಾರೆ
ಪರಿ-ಹರಿ-ಸಲಿ
ಪರಿ-ಹರಿ-ಸು-ವುದು
ಪರಿ-ಹಾರ-ವನ್ನು
ಪರೀಕ್ಷಿತ
ಪರೀಕ್ಷಿತನ
ಪರೀಕ್ಷಿತ-ನನ್ನು
ಪರೀಕ್ಷಿತ-ನಿಗೆ
ಪರೀಕ್ಷಿಸಿ
ಪರೀಕ್ಷೆ-ಯಲ್ಲಿ
ಪರ್ಯಂತ
ಪರ್ವ-ತಕ್ಕೆ
ಪರ್ವತದ
ಪರ್ವತ-ದಲ್ಲಿ
ಪರ್ವತ-ವನ್ನು
ಪರ್ವತಶಿಖರ-ವನ್ನು
ಪಲಾಯನ
ಪಶ್ಚಾತ್
ಪಶ್ಚಿಮ
ಪಾಂಚಾಲಕ್ಕೆ
ಪಾಂಚಾಲ-ನಗರ-ದಿಂದ
ಪಾಂಚಾಲ-ರಾಜ-ನಾದ
ಪಾಂಡವ-ಕೌರ-ವರೂ
ಪಾಂಡ-ವರ
ಪಾಂಡ-ವ-ರನ್ನು
ಪಾಂಡ-ವರನ್ನೂ
ಪಾಂಡ-ವರಲ್ಲಿ
ಪಾಂಡ-ವರಿಂದ
ಪಾಂಡ-ವರಿ-ಗಾಗಿ
ಪಾಂಡ-ವರಿಗೂ
ಪಾಂಡ-ವರಿಗೆ
ಪಾಂಡ-ವರಿ-ಗೋಸ್ಕರ
ಪಾಂಡ-ವರು
ಪಾಂಡ-ವರೂ
ಪಾಂಡ-ವರೊಡನೆ
ಪಾಂಡವಾಽಪ್ಯವಾಪುಃ
ಪಾಂಡವೈರ್ವಿನಿ-ಹಿತಂ
ಪಾಂಡು
ಪಾಂಡು-ಕುಂತೀ
ಪಾಂಡು-ತನಯಾ
ಪಾಂಡು-ಪುತ್ರ-ರಾಗಿ
ಪಾಂಡು-ಪುತ್ರಾಃ
ಪಾಂಡು-ರಾಜನ
ಪಾಂಡು-ರಾಜ-ನಿಂದ
ಪಾಂಡು-ರಾಜ-ನಿಗೆ
ಪಾಂಡು-ರಾಜನು
ಪಾಂಡು-ರಾಜನೂ
ಪಾಂಡು-ವಿನ
ಪಾಂಡು-ಸುತ-ನಾದ
ಪಾಂಡೂನ್
ಪಾಂಡೋಸ್ತಮೀಡೇ-ಽಚ್ಯು-ತಮ್
ಪಾಚಕ-ನನ್ನು
ಪಾಚಕ-ನಾಗಿ
ಪಾತಾಳಕ್ಕೆ
ಪಾತಾಳ-ಲೋಕಕ್ಕೆ
ಪಾತು
ಪಾತ್ರ
ಪಾತ್ರರೂ
ಪಾತ್ರೆ-ಯನ್ನು
ಪಾದಕಮಲ-ಗಳಲ್ಲಿ
ಪಾದಕ್ಕೆ
ಪಾದ-ಗಳ
ಪಾದ-ತಾಡನ-ದಿಂದ
ಪಾದದ
ಪಾದ-ಸೇವೆ-ಯನ್ನು
ಪಾದುಕಾ
ಪಾನ
ಪಾನ-ಮಾಡಿ
ಪಾಪ
ಪಾಪ-ಕೃತ್ಯ-ಗಳನ್ನು
ಪಾಪ-ರ-ಹಿತ-ವಾದ
ಪಾಪವು
ಪಾಯಾತ್
ಪಾರತಂತ್ರ್ಯ
ಪಾರಿಜಾತ
ಪಾರಿಜಾತ-ವೃಕ್ಷ-ವನ್ನು
ಪಾರ್ಥಂ
ಪಾರ್ಥತಃ
ಪಾರ್ಥನ
ಪಾರ್ಥ-ನಿಗೆ
ಪಾರ್ಥಾಃ
ಪಾರ್ಥಾನ್
ಪಾರ್ಥಿವ
ಪಾರ್ಥೆನ
ಪಾರ್ಥೇನ
ಪಾರ್ಥೈಃ
ಪಾರ್ವತಿ
ಪಾರ್ಶಕ್ಕೆ
ಪಾರ್ಷ-ತಮ್
ಪಾಲಿ-ಸಿದರೋ
ಪಾಲಿಸುವ
ಪಾಶುಪತಾಗ-ಮ-ಗಳನ್ನು
ಪಾಶುಪತಾದಿ
ಪಾಶುಪತಾತ್ರ-ವನ್ನು
ಪಿತೂರಿ
ಪಿತೃಹೀನ-ರಾದ
ಪಿತ್ರಾವಿಹೀನಾನುಪಗತನಗರಾನ್ಯಸ್ತ್ವಜೋಽಪಾದ್ವಿಪದ್ಭ್ಯಃ
ಪಿತ್ರೋರ್ಬಂಧಂ
ಪಿಶಾಚಿ-ಗಳ
ಪೀಡಿ-ಸು-ವುದು
ಪುಂಡರೀಕಾಕ್ಷ-ನನ್ನು
ಪುಚ್ಛದ
ಪುಡಿ
ಪುಡಿ-ಪುಡಿ-ಮಾಡಿ
ಪುಡಿ-ಮಾಡುವಂತೆ
ಪುಡಿ-ಯಾಗಲು
ಪುತ್ರಂ
ಪುತ್ರನ
ಪುತ್ರ-ನನ್ನು
ಪುತ್ರ-ನಾದ
ಪುತ್ರನಿ-ಗಾಗಿ
ಪುತ್ರರ
ಪುತ್ರ-ರನ್ನು
ಪುತ್ರ-ರಿಗೆ
ಪುತ್ರರು
ಪುತ್ರರೂ
ಪುತ್ರಸಂಪದ್ಯುತಾನ್
ಪುತ್ರಾದಿ-ಗಳನ್ನು
ಪುತ್ರಿ-ಯ-ರೊಡನೆ
ಪುತ್ರೌ
ಪುನಃ
ಪುನರು-ಜೀವಿ-ಸು-ವುದು
ಪುನಹ
ಪುಪ್ಪ-ಗಳನ್ನುಳ್ಳ
ಪುರಂ
ಪುರಕ್ಕೆ
ಪುರಮಚ್ಯುತೋ
ಪುರಮಬ್ಬ
ಪುರಾಣ-ಗಳಿಗೂ
ಪುರೀಂ
ಪುರುಷ-ರಲ್ಲಿ
ಪುರುಷೋತ್ತ-ಮನು
ಪುರೂರವ
ಪುರೂರವಪ್ರಭೃತಯೋ
ಪುರೋಚನ
ಪುರೋಚನನ
ಪುರೋ-ಹಿತ-ನನ್ನು
ಪುರೋ-ಹಿತನು
ಪುರೋ-ಹಿತ-ರನ್ನಾಗಿ
ಪುರೋ-ಹಿತ-ರಾದ
ಪುಷ್ಪಕ
ಪುಷ್ಪ-ಗಳನ್ನು
ಪುಷ್ಪಗ್ರಹಣ
ಪುಷ್ಪಮಾಲಾರ್ಪಣೆ
ಪುಷ್ಪವೃಷ್ಟಿ-ಯನ್ನು
ಪೂಜಾ
ಪೂಜಿ-ಸಲು
ಪೂಜಿಸಿ
ಪೂಜಿಸಿ-ಕೊಂಡು
ಪೂಜಿ-ಸು-ವುದು
ಪೂಜೆ
ಪೂಜೆ-ಗೊಂಡ
ಪೂಜೆ-ಗೋಸ್ಕರ
ಪೂಜೆ-ಯನ್ನು
ಪೂತನಿ
ಪೂತನಿಯ
ಪೂರೈ-ಸು-ವು-ದಾಗಿ
ಪೂರ್ಣಗೊಳಿ-ಸು-ವುದು
ಪೂರ್ಣಪ್ರಜ್ಞರು
ಪೂರ್ಣ-ವಾಗಿದೆ
ಪೂರ್ತಿ-ಯಾದ
ಪೂರ್ವ
ಪೂರ್ವ-ಕಥೆ
ಪೂರ್ವ-ಕ-ವಾದ
ಪೂರ್ವ-ಚರಿತ್ರೆ
ಪೂರ್ವ-ಜನ್ಮ-ಗಳ
ಪೂರ್ವ-ಜನ್ಮದ
ಪೂರ್ವ-ದಲ್ಲಿ
ಪೂರ್ವ-ದಲ್ಲಿಯೂ
ಪೂರ್ವ-ವದಸ್ಯ
ಪೃತನಾಂ
ಪೃಥಗಿತೋ
ಪೃಷತತನುಜಪ್ರೇಷಿತಬ್ರಾಹ್ಮಣೋಕ್ತ್ಯಾ
ಪೆಟ್ಟಿ-ನಿಂದ
ಪೆಟ್ಟು
ಪೋಸ್ಟ್
ಪೌಂಡರೀಕ
ಪೌಂಡ್ರಕ
ಪೌಂಡ್ರಕ-ವಾಸುದೇವ-ನನ್ನು
ಪೌತ್ರ-ನಾದ
ಪೌಲೋಮಗಣ
ಪ್ರಕಟವಾ-ದುದ-ರಿಂದ
ಪ್ರಕಟವಾ-ದುದು
ಪ್ರಕಟಿಸಿ
ಪ್ರಕಾರ
ಪ್ರಕಾರ-ವಾಗಿ
ಪ್ರಕಾಶಕರು
ಪ್ರಕಾಶಿ-ಸುತ್ತಿ-ರುವ
ಪ್ರಚಾರಕ್ಕಾಗಿಯೇ
ಪ್ರಜೆ-ಗಳ
ಪ್ರಜೆ-ಗಳನ್ನು
ಪ್ರಜೆ-ಗಳಿಂದ
ಪ್ರಜೆ-ಗಳಿಗೆ
ಪ್ರಜೆ-ಗಳು
ಪ್ರತಿ
ಪ್ರತಿ-ಗತಭ್ರಗುಪಃ
ಪ್ರತಿಜ್ಞಾನು
ಪ್ರತಿಜ್ಞೆ
ಪ್ರತಿಜ್ಞೆ-ಯನ್ನು
ಪ್ರತಿ-ಪಾದನೆ
ಪ್ರತಿ-ಪಾದಿತ-ವಾಗಿ
ಪ್ರತಿ-ಪಾದಿಸಿ
ಪ್ರತಿ-ಪಾದ್ಯ-ರಾದ
ಪ್ರತಿ-ಭಟನೆ
ಪ್ರತಿ-ಭೆ-ಯಿಂದಲೇ
ಪ್ರತಿ-ಮೆ-ಯನ್ನು
ಪ್ರತಿ-ಯೊಂದು
ಪ್ರತ್ಯೇಕ
ಪ್ರದರ್ಶನ
ಪ್ರದರ್ಶಿ-ಸು-ವುದು
ಪ್ರದಾನ
ಪ್ರದಾನ-ಮಾಡಿದ
ಪ್ರದೇಶ-ದಲ್ಲಿ-ರುವ-ವ-ರನ್ನು
ಪ್ರದ್ಯುಮ್ನ
ಪ್ರದ್ಯುಮ್ನನ
ಪ್ರದ್ಯುಮ್ಮ
ಪ್ರದ್ಯುಮ್ಮನ
ಪ್ರದ್ಯುಮ್ಮ-ನಿಂದ
ಪ್ರಧಾನ
ಪ್ರಪದ್ಯೇ
ಪ್ರಬಲ-ರಿಂದ
ಪ್ರಭಾಸ
ಪ್ರಭು-ಗಳು
ಪ್ರಮಾಣ
ಪ್ರಮಾಣಗ್ರಂಥ-ಗಳು
ಪ್ರಮಾಣ-ವೆಂದು
ಪ್ರಮೇಯ
ಪ್ರಮೇಯಕ್ಕೆ
ಪ್ರಮೇಯ-ಗಳನ್ನು
ಪ್ರಮೇಯ-ಗಳು
ಪ್ರಮೋದಂ
ಪ್ರಯತ್ನ
ಪ್ರಯತ್ನಿ-ಸು-ವುದು
ಪ್ರಯಾಣ
ಪ್ರಯಾಣ-ಮಾಡುವುದು
ಪ್ರಯೋಗ
ಪ್ರಯೋಗ-ಮಾಡಲು
ಪ್ರಯೋಗಿಸಿ
ಪ್ರಯೋಗಿಸಿ-ದಾಗ
ಪ್ರಯೋಗಿಸು
ಪ್ರಯೋಗಿಸು-ವುದು
ಪ್ರಯೋಜನ
ಪ್ರಲಂಬಾ
ಪ್ರವಚನ
ಪ್ರವಚನ-ದಲ್ಲಿ
ಪ್ರವರ್ತಿಸಿ-ದುದು
ಪ್ರವಾಹಕ್ಕೆ
ಪ್ರವೇಶ
ಪ್ರವೇಶ-ಮಾಡದೆ
ಪ್ರವೇಶ-ಮಾಡಿ
ಪ್ರವೇಶ-ಮಾಡುವುದು
ಪ್ರವೇಶಿಸಿ
ಪ್ರವೇಶಿ-ಸು-ವುದು
ಪ್ರಶಂಸೆ
ಪ್ರಶಂಸೆ-ಯನ್ನು
ಪ್ರಶ್ನೆ
ಪ್ರಶ್ನೆ-ಗಳಿಗೆ
ಪ್ರಶ್ನೆ-ಯನ್ನು
ಪ್ರಸವ
ಪ್ರಸವಿ-ಸು-ವುದು
ಪ್ರಸಾದ-ದಿಂದಲೇ
ಪ್ರಸಿದ್ಧ
ಪ್ರಹಸ್ಯ
ಪ್ರಹಾರ-ಗಳು
ಪ್ರಹಾರ-ದಿಂದ
ಪ್ರಾಗ್ಜ್ಯೋತಿಷಂ
ಪ್ರಾಣ
ಪ್ರಾಣತ್ಯಾಗ
ಪ್ರಾಣಿ-ಗಳು
ಪ್ರಾದಾತ್ಪರೇತಂ
ಪ್ರಾದುರ್ಭವಿಸಿ
ಪ್ರಾದುರ್ಭವಿ-ಸು-ವುದು
ಪ್ರಾಪ್ತಃ
ಪ್ರಾಪ್ತ-ನಾಗಿ
ಪ್ರಾಪ್ತ-ರಾಗಿ
ಪ್ರಾಪ್ತರಾಗು-ವುದು
ಪ್ರಾಪ್ತವಾಗುತ್ತದೆಂದು
ಪ್ರಾಪ್ತವಾನ್
ಪ್ರಾಪ್ತವಾಯಿತು
ಪ್ರಾಪ್ತಾ
ಪ್ರಾಪ್ತೋ-ವಾಯು-ಸುತೇನ
ಪ್ರಾಪ್ತೋಽವತಾದ್ರಾಘವಃ
ಪ್ರಾಪ್ಯ
ಪ್ರಾಪ್ಯ-ಯಜ್ಚ್ನೈರ್ಯಜನ್
ಪ್ರಾಮಾಣ್ಯ-ವನ್ನೂ
ಪ್ರಾಮಾಣ್ಯಸ್ಥಾಪನೆ
ಪ್ರಾಯಚ್ಛದ್ಧರಿಸೂನವೇ
ಪ್ರಾರಂಭ
ಪ್ರಾರಂಭ-ವಾದ
ಪ್ರಾರಬ್ಧ
ಪ್ರಾರ್ಥನೆ
ಪ್ರಾರ್ಥಿತ-ನಾಗಿ
ಪ್ರಾರ್ಥಿ-ಸಲ್ಪಟ್ಟು
ಪ್ರಾರ್ಥಿಸಿ
ಪ್ರಾರ್ಥಿಸಿ-ದುದು
ಪ್ರಾರ್ಥಿಸುತ್ತೇನೆ
ಪ್ರಾರ್ಥಿ-ಸು-ವುದು
ಪ್ರಿಯತಮ-ನಾದ
ಪ್ರಿಯತಮ-ಭರತಂ
ಪ್ರೀಣಯಂತೋ
ಪ್ರೀಣಯನ್
ಪ್ರೀತನಾಗಲಿ
ಪ್ರೀತ-ನಾಗಿ-ರುವ
ಪ್ರೀತ-ನಾದ
ಪ್ರೀತಿ
ಪ್ರೀತಿ-ಕಾರೀ
ಪ್ರೀತಿ-ಗಾಗಿ
ಪ್ರೀತಿ-ಯಾಗಲಿ
ಪ್ರೀತಿ-ಯಿಂದ
ಪ್ರೀತಿ-ಸಲ್ಪಡ-ತಕ್ಕ-ವನು
ಪ್ರೀಯತಾಂ
ಪ್ರೇತ-ಸಂಸ್ಕಾರ
ಫಲ
ಫಲ-ಗಳು
ಫಲ-ತಾರ-ತಮ್ಯ
ಫಲದ
ಫಲವತ್ತಾಗಿ
ಫಲ-ವನ್ನು
ಫಲ-ವಾಗಿ
ಫಲವಿಲ್ಲ-ವೆಂಬ
ಫಲಾಧಿಕ್ಯ-ವೆಂಬ
ಫಲಾನುಭವ
ಫಲಿ-ಸಲಿಲ್ಲವೋ
ಫಲ್ಗುನಂ
ಬಂದ
ಬಂದನೋ
ಬಂದಾಗ
ಬಂದಿದ್ದ
ಬಂದಿ-ರುವ-ನೆಂದು
ಬಂದಿರು-ವುದು
ಬಂದು
ಬಂದುದು
ಬಂಧ
ಬಂಧನ
ಬಂಧ-ನ-ದಿಂದ
ಬಂಧಿ-ಸಿದ
ಬಂಧು-ಗ-ಳಾದ
ಬಂಧು-ಜನ-ರಿಂದಲೂ
ಬಂಧು-ವಾದ
ಬಕಂ
ಬಕನ
ಬಕ-ನನ್ನು
ಬಕಾ-ಸುರನ
ಬಕಾ-ಸುರ-ನನ್ನು
ಬಕಾ-ಸುರ-ನೊಡನೆ
ಬಗೆ
ಬಗೆ-ಯನ್ನು
ಬಗೆ-ಯಾಗಿ
ಬಗ್ಗಿಸಿ
ಬಗ್ಗೆ
ಬತ್ತಳಿಕೆ-ಯಲ್ಲಿ
ಬದರಿಕಾಶ್ರಮಕ್ಕೆ
ಬದರಿಕಾಶ್ರಮ-ದಲ್ಲಿ
ಬದರಿಕಾಶ್ರಮ-ದಲ್ಲಿಯೇ
ಬದರಿ-ಯಲ್ಲಿನ
ಬದಲು
ಬದುಕಿಸಿ
ಬದುಕಿ-ಸು-ವುದು
ಬಧ್ರಾ
ಬಧ್ವಾ
ಬಭ್ರುವಾಹನ-ಅರ್ಜುನರ
ಬಭ್ರುವಾ-ಹನನ
ಬಯ-ಸು-ವುದು
ಬರ-ಮಾಡಿ-ಕೊಂಡು
ಬರ-ಮಾಡಿ-ಕೊಳ್ಳು-ವುದು
ಬರಲು
ಬರುವ
ಬರು-ವಂತೆ
ಬರು-ವು-ದಾಗಿ
ಬರು-ವುದು
ಬಲ-ಗಳಿಂದ
ಬಲ-ದಿಂದ
ಬಲ-ರಾಮ
ಬಲ-ರಾಮ-ಕೃಷ್ಣ-ಅಕ್ರೂರ
ಬಲ-ರಾಮ-ಕೃಷ್ಣ-ರನ್ನು
ಬಲ-ರಾಮನ
ಬಲ-ರಾಮ-ನನ್ನು
ಬಲ-ರಾಮ-ನಲ್ಲಿ
ಬಲ-ರಾಮ-ನಾಗಿ
ಬಲ-ರಾಮ-ನಿಂದ
ಬಲ-ರಾಮ-ನಿಗೆ
ಬಲ-ರಾಮನು
ಬಲ-ರಾಮ-ನೊಡನೆ
ಬಲ-ರಾಮರ
ಬಲ-ರಾಮ-ರಿಗೆ
ಬಲವೃದ್ಧಿ-ಗಾಗಿ
ಬಲವೆಲ್ಲವೂ
ಬಲಾಡ್ಯನಾಗು-ವುದು
ಬಲಾತ್ಕರಿಸಿ
ಬಲಾತ್ಕಾರ-ದಿಂದ
ಬಲಾತ್ಕಾರ-ವಾಗಿ
ಬಲಾದಜೀಘನದತೋ
ಬಲಿ-ನಾಮಕ
ಬಲಿ-ಯನ್ನು
ಬಲಿ-ಯಲ್ಲಿ
ಬಲಿಯು
ಬಲೇನ
ಬಳಿ
ಬಳಿಗೆ
ಬಳಿತ್ತಾ-ಸೂಕ್ತ-ವನ್ನು
ಬಳಿತ್ಸಾ
ಬಳಿಯೇ
ಬಹಳ
ಬಹು
ಬಹು-ದಾದ
ಬಹು-ಧನೈಃ
ಬಹ್ವೀ-ರುವಾಹಾಂಗನಾಃ
ಬಾಘೀಕ-ರಾಜನ
ಬಾಣ
ಬಾಣ-ಗಳನ್ನು
ಬಾಣ-ಗಳಿಂದ
ಬಾಣದ
ಬಾಣ-ದಿಂದ
ಬಾಣಪ್ರಯೋಗ
ಬಾಣಪ್ರಯೋಗ-ದಿಂದ
ಬಾಣಪ್ರಯೋಗ-ಮಾಡಿ
ಬಾಣ-ವನ್ನು
ಬಾಣಶ್ರೇಷ್ಠ-ವನ್ನು
ಬಾಣಾ-ಸುರನ
ಬಾಣಾ-ಸುರನು
ಬಾಧೆ
ಬಾಯಲ್ಲಿ
ಬಾರದ
ಬಾಲಕರ
ಬಾಲಕ-ರನ್ನು
ಬಾಲಕಾಂಡದ
ಬಾಲಕ್ಕೆ
ಬಾಲಘ್ನೀ
ಬಾವಿ-ಯಿಂದ
ಬಾಹೀಕ-ರಾಜ-ನನ್ನು
ಬಾಹು
ಬಾಹು-ಗಳ
ಬಿಟ್ಟ
ಬಿಟ್ಟು
ಬಿಟ್ಟು-ಕೊಟ್ಟು
ಬಿಡಲು
ಬಿಡಲ್ಪಟ್ಟ
ಬಿಡಿಸಿ
ಬಿಡಿಸಿ-ಕೊಂಡು
ಬಿಡಿ-ಸಿದ
ಬಿಡಿ-ಸು-ವುದು
ಬಿಡುಗಡೆ
ಬಿಡುತ್ತಿ-ರುವ
ಬಿಡುವ
ಬಿಡು-ವುದು
ಬಿಳಿಯ
ಬಿಳುಪಾದ
ಬಿಸಾಡು-ವುದು
ಬಿಸುಟು
ಬೀಳಿ-ಸು-ವುದು
ಬೀಳು-ವಂತೆ
ಬೀಳು-ವುದು
ಬೀಸು-ವುದು
ಬುದ್ದ
ಬುದ್ದಿ
ಬುದ್ದಿ-ವಾದ
ಬುದ್ಧ
ಬುದ್ಧ-ನೆಂದು
ಬುದ್ಧಿ-ಯನ್ನು
ಬೃಗು-ಋಷಿ-ಗಳೇ
ಬೃಹತ್ತೇನನ
ಬೃಹಸ್ಪತಿಯು
ಬೆಂಕಿ
ಬೆಂಕಿ-ಯನ್ನು
ಬೆಣ್ಣೆ-ಯನ್ನು
ಬೆಳಿಗ್ಗೆ
ಬೆಳೆ-ಸಿದ
ಬೇಕಾಗಿ
ಬೇಕಾ-ದರೂ
ಬೇಡದೇ
ಬೇಡಲು
ಬೇಡ-ವೆಂದು
ಬೇಡ-ವೆಂಬ
ಬೇಡಿಕೆ
ಬೇಡು-ವುದು
ಬೇರೆ
ಬೇರೆ-ಡೆ-ಯಲ್ಲಿ
ಬೇರೊಬ್ಬನಿದ್ದಾನೆ
ಬೋಧಿ-ಸಲು
ಬೋಧಿ-ಸು-ವುದು
ಬೌದ್ಧ
ಬ್ರಹ್ಮ
ಬ್ರಹ್ಮ-ದೇ-ವರ
ಬ್ರಹ್ಮ-ದೇ-ವರು
ಬ್ರಹ್ಮ-ಪದವಿಯ
ಬ್ರಹ್ಮ-ಸೂತ್ರ-ಗಳ
ಬ್ರಹ್ಮ-ಸೂತ್ರ-ಗಳನ್ನು
ಬ್ರಹ್ಮ-ಸೂತ್ರ-ಗಳನ್ನೂ
ಬ್ರಹ್ಮಾಂಡಂ
ಬ್ರಹ್ಮಾಂಡದ
ಬ್ರಹ್ಮಾಂಡ-ದಲ್ಲಿ
ಬ್ರಹ್ಮಾಂಡ-ವನ್ನು
ಬ್ರಹ್ಮಾಂಡ-ವನ್ನೂ
ಬ್ರಹ್ಮಾದಿ
ಬ್ರಹ್ಮಾದಿ-ಗಳ
ಬ್ರಹ್ಮಾಸ್ತ್ರ
ಬ್ರಹ್ಮಾತ್ರ-ದಿಂದ
ಬ್ರಹ್ಮಾತ್ರ-ವನ್ನು
ಬ್ರಾಹ್ಮಣ
ಬ್ರಾಹ್ಮಣನ
ಬ್ರಾಹ್ಮಣ-ನಿಗೆ
ಬ್ರಾಹ್ಮಣನು
ಬ್ರಾಹ್ಮಣ-ನೊಡನೆ
ಬ್ರಾಹ್ಮಣರ
ಬ್ರಾಹ್ಮಣ-ರನ್ನು
ಬ್ರಾಹ್ಮಣರು
ಬ್ರಾಹ್ಮಣ-ರೂಪ-ದಿಂದ
ಬ್ರಾಹ್ಮಣ-ಶಾಪಕ್ಕೆ
ಬ್ರಾಹ್ಮಣತ್ರೀ-ಯ-ರನ್ನು
ಬ್ರಾಹ್ಮಣಾದಿ
ಬ್ರಾಹ್ಮಣೋತ್ತಮರ
ಭಂಕ್ತ್ವಾ-ವನಂ
ಭಕ್ತರ
ಭಕ್ತ-ರಿಂದಲೂ
ಭಕ್ತ-ರಿಗೆ
ಭಕ್ತರು
ಭಕ್ತರೂ
ಭಕ್ತಾ-ದಿ-ಗಳು
ಭಕ್ತಾದಿಶಬ್ದ
ಭಕ್ತಾನಾಂ
ಭಕ್ತಿ
ಭಕ್ತಿಗೆ
ಭಕ್ತಿಯ
ಭಕ್ತಿ-ಯಿಂದ
ಭಕ್ತಿ-ರ-ಹಿತ-ರಾದ-ವರು
ಭಕ್ಷ
ಭಕ್ಷ-ಣ-ನನ್ನು
ಭಕ್ಷಾದಿ-ಗಳನ್ನು
ಭಕ್ಷಿಸಿ
ಭಗದತ್ತ-ಅರ್ಜುನರ
ಭಗದತ್ತ-ನನ್ನು
ಭಗದತ್ತ-ನನ್ನೂ
ಭಗವಂತನ
ಭಗವದ್ಗೀ-ತೆಯ
ಭಗ್ನಗೊಳಿ-ಸು-ವುದು
ಭಗ್ನಾಶಾನ್
ಭಜೇ
ಭಟೀಯ
ಭದ್ರಾ-ಇ-ವರೂ
ಭಯ
ಭಯಂಕರ
ಭಯಂಕರ-ವಾದ
ಭಯ-ದಿಂದ
ಭಯ-ಪಟ್ಟು
ಭಯ-ವನ್ನು
ಭರತ
ಭರತ-ನನ್ನು
ಭರತ-ನಿಂದ
ಭರತ-ನಿಗೆ
ಭರತನು
ಭರತ-ಶತ್ರುಘ್ನರು
ಭರತಾದಯಃ
ಭವೇತ್ಸರ್ವದಾ
ಭಸ್ಮಿ
ಭಾಗ-ದಲ್ಲಿ
ಭಾಗವತ-ಧರ್ಮ-ದಲ್ಲಿಯೇ
ಭಾಗವಹಿಸಿ
ಭಾಗೀರಥಿ
ಭಾರತ
ಭಾರತ-ಕಥೆ-ಯನ್ನು
ಭಾರತ-ಕಥೆ-ಯಲ್ಲಿ
ಭಾರತದ
ಭಾರತ-ಭೂಮಿ-ಯಲ್ಲಿ
ಭಾರ-ತಸ್ಯ
ಭಾರತಾದಿ
ಭಾರತಾರ್ಥವು
ಭಾರತಿ-ದೇವಿ-ಯ-ರನ್ನೂ
ಭಾರತೀ-ದೇವಿ-ಯರ
ಭಾರದ್ವಾಜ-ರಲ್ಲಿ
ಭಾರಿಮಳೆ
ಭಾರ್ಗವಸ್ಯ
ಭಾರ್ಗವಾತ್ರ-ವನ್ನು
ಭಾವದ
ಭಾವಸಂಗ್ರಹ
ಭಾವಸಂಗ್ರಹಃ
ಭಾವಿಸುತ್ತೇನೆ
ಭಾಷೆ-ಗಳ
ಭಾಷೆ-ಯಲ್ಲಿ
ಭಾಷ್ಯ-ವನ್ನು
ಭಿಕ್ಷಾನ್ನ-ದಿಂದಲೂ
ಭಿಕ್ಷಾನ್ನ-ವನ್ನು
ಭಿಕ್ಷಾನ್ನಾಶಿನ
ಭಿಕ್ಷೆಗೆ
ಭಿಮಸೇನನು
ಭೀಕರ
ಭೀಮಂ
ಭೀಮ-ಜರಾಸಂಧರ
ಭೀಮ-ನನ್ನು
ಭೀಮ-ನಾಗಿ
ಭೀಮ-ನಿಗೆ
ಭೀಮನಿಗೇ
ಭೀಮನು
ಭೀಮ-ಶಲ್ಯರ
ಭೀಮಸೇನ
ಭೀಮಸೇನ-ಅಶ್ವತ್ಥಾಮರ
ಭೀಮಸೇನ-ದುರ್ಯೊಧನರ
ಭೀಮಸೇನ-ದೌಪದೀ-ದೇವಿ
ಭೀಮಸೇನ-ದೌಪದೀ-ದೇವಿಗೆ
ಭೀಮಸೇನನ
ಭೀಮಸೇನ-ನನ್ನು
ಭೀಮಸೇನ-ನಲ್ಲಿ
ಭೀಮಸೇನ-ನಿಂದ
ಭೀಮಸೇನ-ನಿಗೆ
ಭೀಮಸೇನನು
ಭೀಮಸೇನನೇ
ಭೀಮಸೇನ-ನೊಂದಿಗೆ
ಭೀಮಸೇನ-ನೊಡನೆ
ಭೀಮಸೇನ-ಭಾ-ರತಿಯ-ರಿಗೆ
ಭೀಮಸೇನ-ರ-ಅಶ್ವತ್ಥಾಮರ
ಭೀಮಸೇಸನು
ಭೀಮಾ-ದಿ-ಗಳು
ಭೀಮಾರ್ಜುನರ
ಭೀಮಾರ್ಜುನ-ರಿಂದ
ಭೀಮಾರ್ಜುನರು
ಭೀಮಾರ್ಜುನರೂ
ಭೀಮುಂ
ಭೀಮೇನಾರ್ಜುನಸಂಯುತೇ
ಭೀಷ್ಕರ
ಭೀಷ್ಕರು
ಭೀಷ್ಮ
ಭೀಷ್ಮಂ
ಭೀಷ್ಮಕ
ಭೀಷ್ಮ-ಕ-ನಿಗೆ
ಭೀಷ್ಮ-ಕ-ರಾಜನ
ಭೀಷ್ಮ-ಕ-ರಾಜ-ನಿಗೆ
ಭೀಷ್ಮ-ಕ-ಸತ್ಕೃತ್ತೋಽಥ
ಭೀಷ್ಮದ್ರೋಣರ
ಭೀಷ್ಮ-ನಾಗಿ
ಭೀಷ್ಮರ
ಭೀಷ್ಮ-ರನ್ನು
ಭೀಷ್ಮ-ರಲ್ಲಿ
ಭೀಷ್ಮರು
ಭೀಷ್ಮಾಂಬಿಕೇ-ಯಾದಯಃ
ಭೀಷ್ಮಾಚಾರರ
ಭೀಷ್ಮಾಚಾರೈರು
ಭೀಷ್ಮಾಚಾರ್ಯನ
ಭೀಷ್ಮಾಚಾರ್ಯರ
ಭೀಷ್ಮಾ-ಚಾರ್ಯ-ರನ್ನು
ಭೀಷ್ಮಾಚಾರ್ಯ-ರಿಂದ
ಭೀಷ್ಮಾಚಾರ್ಯರು
ಭೀಷ್ಮಾಚಾರ್ಯರೇ
ಭೀಷ್ಮಾದಿ-ಗಳನ್ನು
ಭುಜ-ಗಳ
ಭೂತ-ನನ್ನಾಗಿ
ಭೂತನಾಗು-ವುದು
ಭೂತ-ವಿಶೇಷವು
ಭೂಪಾ
ಭೂಭಾರಕ್ಷಯಕಾಂಕ್ಷಿಭಿಃ
ಭೂಭಾರ-ವನ್ನು
ಭೂಮಂಡಲ-ವನ್ನು
ಭೂಮಿ-ಯನ್ನು
ಭೂಮಿ-ಯಲ್ಲಿ
ಭೂಮಿ-ಯಿಂದ
ಭೂಯಸ್ತ್ವಾಗತವಾಹವೇ
ಭೂಸುರೈಃ
ಭೇಜುಸ್ತಮೀಡೇ-ಽಚ್ಯು-ತಮ್
ಭೇಟ
ಭೇಟಿ
ಭೇಟಿ-ಮಾಡಿದ
ಭೇಟಿಯ
ಭೇದಈ
ಭೇದ-ಗಳು
ಭೇದವೂ
ಭೇದಿ-ಸು-ವುದು
ಭೇರಿ
ಭೇರಿ-ಗಳನ್ನು
ಭೈಷಿದ್ಯುಸೇತೋ
ಭೋಗದ
ಭೋಗಿ-ಸು-ವುದು
ಭೋಜನ
ಭೋಜ-ನಾದಿ-ಗಳಿ-ಗಾಗಿ
ಭೋವನ್ನು
ಭೌಮಮಪಾಹರತ್
ಭ್ಯಾಸ-ದಿಂದಲೂ
ಭ್ರಗುವೇ
ಮಂಗಳ-ವನ್ನು
ಮಂಗಳ-ವಾದ
ಮಂಗಳಾ
ಮಂಚದ
ಮಂತ್ರ-ಗಳನ್ನು
ಮಂತ್ರ-ಗಳಿಂದ
ಮಂತ್ರ-ಬಲ-ದಿಂದ
ಮಂತ್ರ-ವನ್ನು
ಮಂತ್ರಾಲಯ
ಮಂತ್ರಿ
ಮಂತ್ರಿ-ಗಳ
ಮಂತ್ರಿ-ಗಳು
ಮಂತ್ರೋಪದೇಶ
ಮಂಥನವೇ
ಮಂಥರೆಯ
ಮಂದಿ
ಮಕ
ಮಕ್ಕಳ
ಮಕ್ಕಳನ್ನು
ಮಕ್ಕಳನ್ನೂ
ಮಕ್ಕಳಾಗಲೆಂದು
ಮಕ್ಕಳಿಗೆ
ಮಕ್ಕಳು
ಮಕ್ಕಳೆಲ್ಲರೂ
ಮಗಧದೇಶಕ್ಕೆ
ಮಗನ
ಮಗ-ನನ್ನೂ
ಮಗ-ನಾಗಿ
ಮಗ-ನಾದ
ಮಗ-ನಿಂದ
ಮಗ-ನಿ-ಗೋಸ್ಕರವೂ
ಮಗನು
ಮಗ-ನೊಂದಿಗೆ
ಮಗಳನ್ನು
ಮಗಳನ್ನೂ
ಮಗ-ಳಾದ
ಮಗಳಿಂದ
ಮಗಳೂ
ಮಗಳೊಡನೆ
ಮಗು-ವನ್ನು
ಮಗು-ವಿನ
ಮಗುವಿ-ನಿಂದ
ಮಣಿಮಂತನು
ಮಣಿಮಂತನೇ
ಮಣಿ-ಯನ್ನು
ಮಣ್ಣನ್ನು
ಮತಕ್ಕೆ
ಮತ್ತು
ಮತ್ತೆ
ಮತ್ತ್ವಯಂತ್ರದ
ಮತ್ಯಾದಿ
ಮತ್ಯಾವ-ತಾರ
ಮತ್ಸ್ಯಾದಿ-ರೂಪೋಽಭವನ್
ಮಥನ
ಮಥುರಾ
ಮಥುರಾ-ದಲ್ಲಿ
ಮಥುರಾ-ನಗರಕ್ಕೆ
ಮಥುರಾ-ನಗರದ
ಮಥುರೆಗೆ
ಮಥುರೆ-ಯಲ್ಲಿ
ಮದರಾಸು
ಮದುವೆ-ಯಾಗಲು
ಮದುವೆಯಾಗು-ವುದು
ಮದೋನ್ಮತ್ತ-ನಾದ
ಮದ್ಯ-ವನ್ನು
ಮದ್ರಾಸಿನ
ಮದ್ರಾಸ್
ಮಧು-ಪಾನ-ಮತ್ತ-ರಾಗಿ
ಮಧುರಿಪುಂ
ಮಧುಸೂದನ-ನನ್ನು
ಮಧ್ಯ-ದಲ್ಲಿ
ಮಧ್ವ
ಮಧ್ವ-ರಿಗೆ
ಮಧ್ವೋ
ಮನಸ್ಸನ್ನು
ಮನಸ್ಸನ್ನೂ
ಮನಸ್ಸು
ಮನು
ಮನುಷ್ಯನಂತೆ
ಮನುಷ್ಯರಂತೆ
ಮನುಷ್ಯರು
ಮನೆಗೆ
ಮನೆಯ
ಮನೆ-ಯಲ್ಲಿ
ಮನೆ-ಯಲ್ಲಿಯೇ
ಮನೆ-ಯಲ್ಲಿ-ರುತ್ತಾ
ಮನೆ-ಯಿಂದ
ಮನ್ನಣೆ
ಮಯತೋಪ್ರಾಪ್ತಃ
ಮಯ-ನಿಂದ
ಮಯನು
ಮಯ-ವಾಗಿಯೂ
ಮಯಾ-ಸುರನಿಂದ
ಮಯಾ-ಸುರನು
ಮರಣ
ಮರಣ-ಹೊಂದು-ವುದು
ಮರು-ಭೂಮಿ-ಯನ್ನು
ಮರ್ದಿಸಿ
ಮಲಗಿ
ಮಲಗಿದ್ದ
ಮಲಗಿ-ರು-ವುದು
ಮಲ್ಲನ
ಮಲ್ಲರೆಲ್ಲರೂ
ಮಹತ್ವ-ವನ್ನೂ
ಮಹ-ದಾದಿ
ಮಹಾಪಾರ್ಶ್ವ
ಮಹಾಪ್ರಯತ್ನ
ಮಹಾ-ಭಾರತ
ಮಹಾ-ಭಾರತ-ತಾತ್ಪರ್ಯ-ನಿರ್ಣಯಾಶಯಸಂಗ್ರಹಃ
ಮಹಾ-ಭಾರತವೇ
ಮಹಾ-ಲಕ್ಷ್ಮಿಯು
ಮಹಿಮಾ
ಮಹಿಮೆ
ಮಹಿಮೆಯ
ಮಹಿಷ್ಠೆ-ಯರೂ
ಮಹೇಂದ್ರ
ಮಹೋದರ
ಮಾಂಬಲಂ
ಮಾಂಬಳಂ
ಮಾಡ-ತಕ್ಕ
ಮಾಡದಿರು-ವುದು
ಮಾಡದೇ
ಮಾಡಬೇಕಾದ
ಮಾಡಬೇಕೆಂದು
ಮಾಡ-ಬೇಕೆಂಬ
ಮಾಡಲಾಗಿ
ಮಾಡಲಾಗಿದೆ
ಮಾಡಲು
ಮಾಡಲ್ಪಟ್ಟ
ಮಾಡಿ
ಮಾಡಿ-ಕೊಂಡದ್ದು
ಮಾಡಿ-ಕೊಂಡು
ಮಾಡಿ-ಕೊಳ್ಳು-ವುದು
ಮಾಡಿಕೋ
ಮಾಡಿದ
ಮಾಡಿ-ದಂತೆ
ಮಾಡಿ-ದನೋ
ಮಾಡಿ-ದಾಗ
ಮಾಡಿ-ದುದ-ರಿಂದ
ಮಾಡಿ-ದುದು
ಮಾಡಿ-ದುದೂ
ಮಾಡಿದ್ದ
ಮಾಡಿ-ರುತ್ತಾರೆ
ಮಾಡಿಸಿ
ಮಾಡಿ-ಸಿದ
ಮಾಡಿ-ಸಿ-ದನೋ
ಮಾಡಿ-ಸಿ-ದುದು
ಮಾಡಿ-ಸು-ವುದು
ಮಾಡುತ್ತಾ
ಮಾಡುತ್ತಿ-ದುದು
ಮಾಡುತ್ತಿದ್ದ
ಮಾಡುತ್ತಿದ್ದರು
ಮಾಡುತ್ತಿ-ರುವಾಗ
ಮಾಡುತ್ತೇನೆ
ಮಾಡುವ
ಮಾಡುವಂತೆ
ಮಾಡುವ-ವಳಾಗಿ
ಮಾಡುವಾಗ
ಮಾಡು-ವು-ದಾಗಿ
ಮಾಡುವುದು
ಮಾತನಾಡಿ
ಮಾತನಾಡಿ-ಸು-ವುದು
ಮಾತಿನಂತೆ
ಮಾತುಕತೆ-ಗಳು
ಮಾತು-ಗಳನ್ನಾಡು-ವುದು
ಮಾತು-ಗಳು
ಮಾತುಶ್ಚ
ಮಾತ್ರ
ಮಾತ್ರರೇ
ಮಾತ್ರವೇ
ಮಾದ್ರಿ
ಮಾದ್ರಿ-ದೇವಿಗೂ
ಮಾದ್ರಿ-ಪಾಂಡು-ವಿನ
ಮಾನಿತಾ
ಮಾನ್ಯ-ವಾದ
ಮಾಮ್
ಮಾಯಾ
ಮಾರನೇ
ಮಾರನೇ-ದಿನ
ಮಾರೀಚಂ
ಮಾರೀಚನ
ಮಾರೀಚ-ನನ್ನು
ಮಾರುತಿಯುಜಾ
ಮಾರ್ಕಂಡೇಯ
ಮಾರ್ಗ
ಮಾರ್ಗ-ದಲ್ಲಿ
ಮಾರ್ಜಾಲದಷ್ಟು
ಮಾರ್ತಾಂಡಿಂ
ಮಾಲಿ-ನಿಧನಂ
ಮಾಲೆ-ಯನ್ನು
ಮಾಸ್ಟರ್
ಮಾಹಾತ್ಮ
ಮಾಹಾತ್ಮ್ಯ
ಮಿತಾಂಶ್ಚ
ಮಿತ್ರ-ವಿಂದೆ
ಮಿಥಿಲಾ-ನಗರ-ದಲ್ಲಿ
ಮಿಥ್ಯ
ಮಿಶ್ರ-ಸೃಷ್ಟಿ
ಮುಂಗುಸಿ
ಮುಂಚೆಯೇ
ಮುಂತಾದ
ಮುಂತಾದ-ವ-ರನ್ನು
ಮುಂತಾದ-ವ-ರಿಂದ
ಮುಂತಾ-ದವ-ರಿಗೆ
ಮುಂತಾದ-ವರು
ಮುಂತಾದುವು-ಗಳನ್ನು
ಮುಂದಿಟ್ಟು
ಮುಂದು-ವರೆಯು-ವುದು
ಮುಂದೆ
ಮುಕುಂದಃ
ಮುಕ್ತಿ
ಮುಕ್ತಿಗೆ
ಮುಕ್ತಿ-ದಾತಾ
ಮುಕ್ತಿ-ಯೋಗ್ಯನತಮೋಯೋಗ್ಯನ
ಮುಕ್ತ್ವಾ
ಮುಖ-ದಲ್ಲಿ
ಮುಖ್ಯ
ಮುಖ್ಯಪ್ರಾಣರು
ಮುಗಿದ
ಮುಗಿದ-ಮೇಲೆ
ಮುಗಿ-ಯುವಂತಹ
ಮುಗಿಸಿ
ಮುಗಿ-ಸಿದ
ಮುಗ್ಧರಾಗು-ವುದು
ಮುಚುಕಂದನ
ಮುಚುಕುಂದನ
ಮುಚ್ಚಿ
ಮುಚ್ಚಿ-ಕೊಂಡ-ಕಾರಣ
ಮುಚ್ಚಿ-ಕೊಳ್ಳು-ವುದು
ಮುದಮಿತಾ
ಮುದ್ರೆ
ಮುದ್ರೆ-ಯುಂಗುರ-ವನ್ನು
ಮುನಿ-ಗಳಿಂದ
ಮುನಿ-ಗಳಿಗೆ
ಮುನಿ-ಗಳೂ
ಮುನಿನಾ
ಮುನ್ನೂರು
ಮುಮುದುರಪಿ-ಪದಂ
ಮುರವೈರಿ-ಯಾದ
ಮುರಿದು
ಮುರಿಯು-ವುದು
ಮುಳುಗು-ವುದು
ಮೂಗುಕಿವಿ-ಗಳನ್ನು
ಮೂರನೇ
ಮೂರು
ಮೂರ್ಛ
ಮೂರ್ಛ-ಗೊಳಿಸಿ
ಮೂರ್ಛ-ಯಿಂದ
ಮೂರ್ಛಿತ-ನನ್ನಾಗಿ
ಮೂರ್ಛಿತನಾಗು-ವುದು
ಮೂಲ
ಮೂಲಕ
ಮೂಲಗ್ರಂಥ-ಗಳಲ್ಲಿನ
ಮೂಲಗ್ರಂಥ-ವನ್ನು
ಮೂಲಪ್ರ-ಕೃತಿ
ಮೂಲ-ರಾಮಾಯಣಾದಿ
ಮೂಲ-ರೂಪಕ್ಕೂ
ಮೂಲ-ರೂಪಕ್ಕೆ
ಮೂಲ-ರೂಪ-ಗಳೊಂದಿಗೆ
ಮೂಲ-ರೂಪ-ವನ್ನು
ಮೂಲ-ರೂಪ-ವನ್ನೂ
ಮೂವತ್ತೆರಡು
ಮೃತ
ಮೃತ-ನಾಗಲು
ಮೃತ-ನಾಗು-ವುದು
ಮೃತ-ಮಗ-ನನ್ನು
ಮೃತ-ರಾಗು-ವುದು
ಮೃತ-ರಾದ
ಮೃತ-ರಾದ-ರೆಂದು
ಮೃತ-ಶಿಶು-ವನ್ನು
ಮೃತ-ಸಂಜೀ-ವಿನಿ
ಮೃತ್ಯೋರಪಾ
ಮೃಷ್ಟಾನ್ನ-ವನ್ನು
ಮೇ
ಮೇಂಛ-ರಿಂದ
ಮೇಘ್ನನ್
ಮೇರು
ಮೇಲಕ್ಕೆ
ಮೇಲಾಗು-ವುದು
ಮೇಲಿನ
ಮೇಲಿ-ನಿಂದ
ಮೇಲೆ
ಮೈಂಛ
ಮೈತ್ರೇ-ಯರು
ಮೈನಾಕಪರ್ವತವು
ಮೈಯೆಲ್ಲ
ಮೊದಲನೇ
ಮೊದಲಾಗಿ
ಮೊದಲಾದ
ಮೊದಲಾದ-ವರ
ಮೊದಲಾದ-ವ-ರನ್ನು
ಮೊದಲಾದ-ವ-ರಿಂದ
ಮೊದಲಿನಂತೆ
ಮೊದಲು
ಮೊರೆ
ಮೊಸರಿನ
ಮೋಕ್ಷ
ಮೋಕ್ಷ-ದಾತ
ಮೋಕ್ಷ-ವನ್ನು
ಮೋಹ
ಮೋಹ-ಗೊಳಿ-ಸಲು
ಮೋಹ-ಗೊಳಿಸಿ
ಮೋಹ-ಗೊಳಿಸುವ
ಮೋಹ-ನಾರ್ಥ-ವಾಗಿ
ಮೋಹ-ಯನ್ಮರ್ತ್ಯವೃತ್ತ್ಯಾ
ಮೋಹಿನೀ
ಮೌನ
ಮ್ಲೇಂಛ
ಯ
ಯಂ
ಯಃ
ಯಕ್ಷ-ನಿಂದ
ಯಕ್ಷನು
ಯಕ್ಷ-ರೂಪ-ದಿಂದ
ಯಜ್ಞ
ಯಜ್ಞಂ
ಯಜ್ಞ-ಕುಂಡ-ದಿಂದ
ಯಜ್ಞದ
ಯಜ್ಞ-ದಲ್ಲಿ
ಯಜ್ಞ-ದೀಕ್ಷಿತ-ನಾಗಿದ್ದಾಗ
ಯಜ್ಞ-ದೀಕ್ಷೆ-ಯಲ್ಲಿ
ಯಜ್ಞ-ವನ್ನು
ಯಜ್ಞ-ಶಾಲೆಯ
ಯಜ್ಞ-ಶಾಲೆಯಲ್ಲಿ
ಯತಿ
ಯತಿ-ಗಳಿಂದ
ಯತಿಯ
ಯತಿ-ವೇಷ-ದಲ್ಲಿ
ಯತಿ-ವೇಷ-ದಲ್ಲಿದ್ದ
ಯತ್ಕಾರುಣ್ಯ
ಯತ್ಪೂರ್ವಂ
ಯತ್ಪ್ರೀತಯೇಽಭೂತ್ಪವನಜ-ವಚನೈಃ
ಯತ್ರ
ಯತ್ಸಮ್ಮತ್ಯಾ
ಯತ್ಸೇ-ವನಂ
ಯಥಾಯೋಗ್ಯ-ವಾದ
ಯಥಾರ್ಥ
ಯಥಾರ್ಥಜ್ಞಾನ-ವನ್ನು
ಯಥೇಷ್ಟ
ಯಥೇಷ್ಟ-ದಾನ
ಯದನುಕ್ರೋಶೇನ
ಯದನು-ನಿಜ-ಪದಂ
ಯದಾಶಾಸಿತೋ
ಯದು
ಯದು-ವಂಶದ
ಯದ್ದರ್ಶನಾತ್ಸಜ್ಜನಾಃ
ಯದ್ಯುಕ್ತಾಃ
ಯದ್ವಾದ್ಯಾ
ಯದ್ವೃದ್ಧಿರ್ಜನ-ಮೋಹಿನೀ
ಯದ್ವೇಷ-ತೋಽಧೇತಮಸಿ
ಯನ್ನಿಯತ್ಯಾರ್ಥ-ಕಾಮಃ
ಯನ್ನೂ
ಯಮಜಿತಂ
ಯಮಥ
ಯಮ-ಧರ್ಮ-ವಾಯು-ಇಂದ್ರ-ಅಶ್ವಿನೀ-ದೇವತೆ-ಗಳಿಂದ
ಯಮನಮತ್ತಂ
ಯಮುನಾ
ಯಮುನಾ-ನದಿ
ಯಮುನಾ-ನದಿಯ
ಯಮುನೆಯು
ಯರಿಂದ
ಯಲ್ಲಿ
ಯವನಂ
ಯವ-ರಾಜ್ಯೇಽಭಿಉಷಿಚ್ಯಸ್ವೀಯಾನ್ರಕ್ಷನ್
ಯಶ
ಯಶೋದೆಗೆ
ಯಶೋದೆ-ಯಿಂದ
ಯಶೋದೆಯು
ಯಷ್ಟಾ
ಯಸ್ಮಾತ್
ಯಸ್ಯ
ಯಸ್ಯಾಪ್ತ-ಭೀಷ್ಮಾತ್ತತಃ
ಯಸ್ಸತ್ಯವತ್ಯಾನುಭೂತ್
ಯಾಗಕ್ಕೆ
ಯಾಗದ
ಯಾಗ-ಮಾಡಲು
ಯಾಗ-ವನ್ನು
ಯಾಗ-ಸಭೆ-ಯಲ್ಲಿ
ಯಾಚನೆ
ಯಾಚಿ-ಸು-ವುದು
ಯಾತಾಽರಣ್ಯಂ
ಯಾತ್ರಾಕ್ರಮ-ದಿಂದ
ಯಾತ್ರೆಯ
ಯಾದ-ವಕುಲ-ವನ್ನು
ಯಾದ-ವ-ರನ್ನು
ಯಾದ-ವರು
ಯಾದ-ವ-ರೊಡನೆ
ಯಾದುದು
ಯಾಮರ್ಥ್ಯ-ಬಲೇನ
ಯಾಮ್ಯಾಂ
ಯಾರ
ಯಾರಲ್ಲಿಯೂ
ಯಾರಿಗೂ
ಯಾರು
ಯಾರೆಂದು
ಯಾರೆಂಬ
ಯಾರೆಂಬುದನ್ನು
ಯಾರೊಡನೆ
ಯಾವ
ಯಾವವು
ಯಾವಾಗಲೂ
ಯಾವುದನ್ನೂ
ಯಾವುದು
ಯುಂಗುರ-ವನ್ನು
ಯುಕ್ತ-ನಾಗಿ
ಯುಕ್ತ-ರಾಗಿ
ಯುಕ್ತ-ರಾದ
ಯುಕ್ತ-ಳಾದ
ಯುಕ್ತ-ವಾದ
ಯುಗ
ಯುಗ-ಗಳ
ಯುಗ-ಗಳಲ್ಲಿ
ಯುಗ-ದಲ್ಲಿ
ಯುಗ-ಧರಕ್ಕೆ
ಯುತಃ
ಯುದ್ದ
ಯುದ್ದಕ್ಕಾಗಿ
ಯುದ್ದದ
ಯುದ್ದ-ದಲ್ಲಿ
ಯುದ್ದವು
ಯುದ್ದೇ
ಯುದ್ಧ
ಯುದ್ಧಕ್ಕೆ
ಯುದ್ಧದ
ಯುದ್ಧ-ದಲ್ಲಿ
ಯುದ್ಧ-ಭೂಮಿ-ಯಲ್ಲಿ
ಯುದ್ಧ-ಮಾಡಿ-ಯಾದರೂ
ಯುಧಿ
ಯುಧಿಷ್ಠಿರ
ಯುಧಿಷ್ಠಿರನ
ಯುಧಿಷ್ಠಿರ-ನಕುಲ-ಸಹ-ದೇ-ವರು
ಯುಧಿಷ್ಠಿರ-ನನ್ನು
ಯುಧಿಷ್ಠಿರ-ನಲ್ಲಿ
ಯುಧಿಷ್ಠಿರ-ನಿಗೆ
ಯುಧಿಷ್ಠಿರ-ನಿ-ಗೋಸ್ಕರ
ಯುಧಿಷ್ಠಿರನು
ಯುಧಿಷ್ಠಿರಾದಿ-ಗಳಲ್ಲಿ
ಯುಧ್ಯತಿ
ಯುಯುತ್ಸು
ಯುವ-ರಾಜ-ನನ್ನಾಗಿ
ಯುವ-ರಾಜ-ನನ್ನಾಗಿಯೂ
ಯುವ-ರಾಜ-ನಾಗು-ವುದು
ಯುವ-ರಾಜ-ಪದವಿ-ಯಲ್ಲಿಯೂ
ಯುವ-ರಾಜ್ಯ
ಯೂಪ-ನೇತ್ರ
ಯೇನ
ಯೇನರ್ಷಿಪ್ರಿಯಕಾರಿಣಾ
ಯೊಡನೆ
ಯೋ
ಯೋಗಕ್ಷೇಮ-ವನ್ನು
ಯೋಗ್ಯ-ತಾನು-ಸಾರ
ಯೋಗ್ಯ-ತಾನು-ಸಾರ-ವಾಗಿ
ಯೋಗ್ಯತೆ-ಗಳು
ಯೋಗ್ಯತೆಗೆ
ಯೋಗ್ಯ-ತೆಯು
ಯೋಗ್ಯ-ಳಾದ
ಯೋಚನೆ
ಯೋಜನ
ಯೋಜೀವ-ಯತ್ಪಾಂಡವೈಃ
ಯೋಹಲ್ಯಾಂ
ಯೋಽಕೃತ
ಯೋಽಗೋಽಭೂದ್ವಿಶ್ವಗರ್ಭಃ
ಯೋಽಜೀಘನತ್ಸ್ಯಂದನಾತ್
ಯೋಽಪಾತ್ಸ-ನೋಽವ್ಯಾದ್ಧರಿಃ
ರಕ್ತ-ಪಾನ
ರಕ್ತವು
ರಕ್ಷಣಾರ್ಥ-ವಾಗಿ
ರಕ್ಷಣೆ
ರಕ್ಷನ್ನಹಿಪತಿದಮನೋ
ರಕ್ಷಾಃ
ರಕ್ಷಿತಃ
ರಕ್ಷಿತ-ನಾದ
ರಕ್ಷಿತನಾ-ದುದು
ರಕ್ಷಿತರಾ-ದುದು
ರಕ್ಷಿಸಬೇಕೆಂದೂ
ರಕ್ಷಿ-ಸಲಿ
ರಕ್ಷಿ-ಸಲು
ರಕ್ಷಿಸಿ
ರಕ್ಷಿಸಿ-ಕೊಂಡು
ರಕ್ಷಿ-ಸಿದ
ರಕ್ಷಿಸಿ-ದನೋ
ರಕ್ಷಿಸಿ-ದುದು
ರಕ್ಷಿಸು
ರಕ್ಷಿಸು-ದುದು
ರಕ್ಷಿಸು-ವಂತೆ
ರಕ್ಷಿಸು-ವು-ದಾಗಿ
ರಕ್ಷಿಸು-ವುದು
ರಚನೆ
ರಚನೆಗೆ
ರಚಿತ-ವಾಗಿದೆ
ರಚಿತ-ವಾಗಿ-ರುವ
ರಚಿತ-ವಾದ
ರಚಿ-ಸಲು
ರಚಿ-ಸಿದ
ರಚಿಸಿ-ರು-ವು-ದಾಗಿ
ರಚಿಸು-ವಂತೆ
ರಜ-ಕನ
ರಜ-ನಿರಸನೋಽ-ದಾತ್ಸನೋಽವ್ಯಾನ್ಮುರಾರಿಃ
ರಣರಂಗಕ್ಕೆ
ರಣೇ
ರಣೋನ್ಮುಖೇ
ರತಿ-ಯರ
ರತ್ನ
ರತ್ನವು
ರಥ
ರಥ-ಗಳ
ರಥ-ಗಳನ್ನು
ರಥ-ದಲ್ಲಿ
ರಥ-ವನ್ನು
ರಥವು
ರಥಾದಿ-ಗಳನ್ನು
ರಥಿಕ-ನಾಗಿ
ರಮಾದಿ
ರಮಾ-ದೇವಿಯ
ರಮಾವಲ್ಲಭನು
ರಮೇಶಃ
ರಹಿತ-ನನ್ನಾಗಿ
ರಹಿತರು
ರಾಕ್ಷಸನ
ರಾಕ್ಷಸನ-ಗರ-ವಾದ
ರಾಕ್ಷಸ-ನನ್ನು
ರಾಕ್ಷಸ-ಪುರೀಂ
ರಾಕ್ಷಸರ
ರಾಕ್ಷಸ-ರನ್ನು
ರಾಕ್ಷಸರು
ರಾಕ್ಷಸಾನ್
ರಾಕ್ಷಸಿಯ
ರಾಕ್ಷಸಿ-ಯಾದ
ರಾಘವೇಂದ್ರ
ರಾಘವೇಂದ್ರ-ಯತಿನಾ
ರಾಜ-ಧರ್ಮ-ಗಳನ್ನು
ರಾಜನ
ರಾಜ-ನನ್ನಾಗಿ
ರಾಜ-ನನ್ನಾಗಿಯೂ
ರಾಜ-ನನ್ನು
ರಾಜ-ನನ್ನೂ
ರಾಜ-ನ-ಪುತ್ರಿ
ರಾಜ-ನಾಗು
ರಾಜ-ನಿಂದ
ರಾಜ-ನೀತಿ
ರಾಜನು
ರಾಜನೂ
ರಾಜರ
ರಾಜ-ರಿಂದಲೂ
ರಾಜರೂ
ರಾಜ-ಸೂಯದ
ರಾಜ-ಸೂಯಯಾಗ
ರಾಜ-ಸೂಯ-ಯಾ-ಗದ
ರಾಜ-ಸೂಯಯಾಗ-ವನ್ನು
ರಾಜ-ಸೂ-ಯಾ-ಗದ
ರಾಜ್ಯ
ರಾಜ್ಯಂ
ರಾಜ್ಯಕ್ಕೆ
ರಾಜ್ಯ-ದಲ್ಲಿ
ರಾಜ್ಯ-ದಾನ
ರಾಜ್ಯ-ಭಾರಕ್ರಮದ
ರಾಜ್ಯ-ಭಾರ-ದಲ್ಲಿ
ರಾಜ್ಯ-ವನ್ನು
ರಾಜ್ಯ-ವಾಳುತ್ತಿ-ರುವಾಗ
ರಾಜ್ಯ-ವಿತೊ
ರಾಜ್ಯವು
ರಾಜ್ಯಾಧಿಪತ್ಯ-ವನ್ನು
ರಾಜ್ಯಾಭಿಷೇಕ
ರಾಜ್ಯಾಭಿಷೇಕ-ವನ್ನು
ರಾಜ್ಯೇಭಿಷಿಕ್ತೋ
ರಾತ್ರಿ
ರಾತ್ರಿ-ಕಾಲ-ದಲ್ಲಿ
ರಾತ್ರಿ-ಯಲ್ಲಿ
ರಾಮ
ರಾಮಃ
ರಾಮ-ಕೃಷ್ಣಾದಿ
ರಾಮ-ಚಂದ್ರಂ
ರಾಮ-ದಯಿತಾಂ
ರಾಮನ
ರಾಮ-ನನ್ನು
ರಾಮ-ನಾಗಿ
ರಾಮ-ನಿಂದ
ರಾಮ-ನಿಗೆ
ರಾಮನು
ರಾಮ-ಲಕ್ಷ್ಮಣರು
ರಾಮ-ಸೀತಾ
ರಾಮಾಚಾರ್ಯರ
ರಾಮಾಯಣ
ರಾಮೋ-ಗತೋ-ಽವ್ಯಾತ್ಸ
ರಾಮೋಭೂದನುಜಾನ್ವಿತೋ
ರಾಮೋಽವತಾದ್ವಂದಿತಃ
ರಾರ್ಮಽವತಾತ್ಸೋಸ್ತು
ರಾವಣ
ರಾವಣ-ಕುಂಭ-ಕರ್ಣರ
ರಾವಣನ
ರಾವಣ-ನನ್ನು
ರಾವಣ-ನಿಂದ
ರಾವಣ-ನಿಗಾದ
ರಾವಣ-ನಿಗೆ
ರಾವಣನು
ರಾವಣನೇ
ರಾವಣ-ಹೃತಾಂ
ರಾಸಕ್ರೀಡೆ-ಯನ್ನು
ರಿ
ರಿಗೆ
ರಿಪೂನ್
ರೀತಿ
ರೀತಿ-ಯಿಂದ
ರುಕ್ಕಿಗೆ
ರುಕ್ಕಿಣಿ-ಯನ್ನು
ರುಕ್ಕಿಣೀಂ
ರುಕ್ಕಿಯು
ರುಕ್ಷ್ಮಿಣಿ
ರುಕ್ಷ್ಮಿಣೀ-ದೇವಿ-ಯನ್ನು
ರುಕ್ಷ್ಮಿ-ಯನ್ನೂ
ರುದ್ರ
ರುದ್ರ-ಕೃಷ್ಣರ
ರುದ್ರ-ದೇ-ವರ
ರುದ್ರ-ದೇ-ವ-ರನ್ನು
ರುದ್ರ-ದೇ-ವರಲ್ಲಿದ್ದ
ರುದ್ರ-ದೇ-ವ-ರಿಂದ
ರುದ್ರ-ದೇ-ವರಿಗೆ
ರುದ್ರ-ದೇ-ವರು
ರುದ್ರನ
ರುದ್ರ-ರೂಪ-ವನ್ನು
ರುವ
ರೂಪ
ರೂಪ-ಗಳ
ರೂಪ-ಗಳನ್ನು
ರೂಪ-ಗಳಿಂದ
ರೂಪ-ಚತುಷ್ಟಯಿಾಂ
ರೂಪದ
ರೂಪ-ದಲ್ಲಿ
ರೂಪ-ದಲ್ಲಿದ್ದ
ರೂಪ-ದಿಂದ
ರೂಪ-ದಿಂದಲೂ
ರೂಪ-ಲಾವಣ್ಯ-ಗಳಿಂದ
ರೂಪ-ವನ್ನು
ರೂಪ-ವರ್ಣನೆ
ರೂಪ-ಸೃಷ್ಟಿ
ರೂಪೋಭವದ್ಯಃ
ರೆಂಬ
ರೆಕ್ಕೆ-ಗಳ
ರೇವತಿ-ಬಲ-ರಾಮರ
ರೈವತ
ರೈವತಾಚಲ-ದಲ್ಲಿ
ರೋದನದ
ರೋಮಹರ್ಷಣ
ರ್ಗರ್ವಿರುಕ್ಮ್ಯಾ
ಲಂಕಾಂ
ಲಂಕಾ-ಪಟ್ಟ-ಣಕ್ಕೆ
ಲಂಕಾ-ಪಟ್ಟ-ಣ-ವನ್ನು
ಲಂಕೆಗೆ
ಲಂಕೆ-ಯಲ್ಲಿ
ಲಂಬ-ನಾಮಕ
ಲಕ್ಷಜನ
ಲಕ್ಷಣ
ಲಕ್ಷಣ-ಇಂದ್ರ-ಜಿತು
ಲಕ್ಷಣ-ಗಳು
ಲಕ್ಷಣ-ನಿಂದ
ಲಕ್ಷಣನು
ಲಕ್ಷಣೆಯು
ಲಕ್ಷ್ಮಣ
ಲಕ್ಷ್ಮಣನು
ಲಕ್ಷ್ಮಣನೂ
ಲಕ್ಷ್ಮಿ
ಲಕ್ಷ್ಮಿ-ದೇವಿಯು
ಲಕ್ಷ್ಮಿ-ಪತಿಯು
ಲಕ್ಷ್ಮೀಂ
ಲಕ್ಷ್ಮೀ-ದೇವಿಯ
ಲಕ್ಷ್ಮೀರಮಣ-ನಾದ
ಲಗ್ನ
ಲಗ್ನ-ವಾಗಿ
ಲಗ್ನ-ವಾಗು-ವುದು
ಲಗ್ನೋ
ಲಬ್ಧ
ಲವಣಾ-ಸುರನ
ಲವ-ನನ್ನು
ಲಾಕ್ಷಾಗೃಹ-ದಲ್ಲಿ
ಲಾಕ್ಷಾಗೃಹ-ದಿಂದ
ಲಿಂಗಾಕಾರ-ವಾಗಿದ್ದ
ಲೋಕ
ಲೋಕಕ್ಕೆ
ಲೋಕ-ದಲ್ಲೂ
ಲೋಮಶಮುನಿಯ
ವಂದನೆ-ಗಳು
ವಂದಿ-ಸು-ವುದು
ವಂಶದ
ವಕ್ತುಂ
ವಚನ
ವಚನ-ಗಳನ್ನಾಡಿ
ವಚನ-ಗಳಿಂದ
ವಚನತೋ
ವಚನ-ವನ್ನು
ವಟವೃಕ್ಷದ
ವತ್ಪರ್ವತ-ದಲ್ಲಿ
ವತ್ಸಾನ್
ವದತಾಂ
ವಧೆ
ವಧೆ-ಗಾಗಿ
ವಧೇ
ವನಂ
ವನಕ್ಕೆ
ವನ-ದಲ್ಲಿ
ವನಪ್ರವೇಶ-ಮಾಡಿ
ವನ-ವಾಸಕ್ಕೆ
ವನ-ವಾಸ-ವನ್ನು
ವನಸ್ಥಂ
ವನ್ನು
ವರ-ಗಳನ್ನು
ವರಪ್ರ-ದಾನ
ವರ-ವನ್ನು
ವರಿ-ಸಲು
ವರಿ-ಸಿದ
ವರು-ಣನು
ವರ್ಣ-ದವ-ರಿಗೆ
ವರ್ಣದವ-ಳಾದ-ಕಾರಣ-ದಿಂದ
ವರ್ಣನೆ
ವರ್ಣಾಶ್ರಮ
ವರ್ಣಾಶ್ರಮ-ಧರ್ಮ-ಗಳಲ್ಲಿಯೇ
ವರ್ಣಿ-ಸಲ್ಪಟ್ಟಿವೆ
ವರ್ಣಿಸಿ
ವರ್ತ್ಮನಿ
ವರ್ಷ
ವರ್ಷ-ಕಾಲ
ವರ್ಷ-ಗಳ
ವರ್ಷ-ಗಳಲ್ಲಿ
ವರ್ಷ-ಗ-ಳಾದ
ವರ್ಷ-ಗಳು
ವರ್ಷದ
ವಲ್ಲ-ವೆಂದು
ವಶನಾ-ದಂತೆ
ವಶ-ರೆಂಬ
ವಸುದೇವ
ವಸುದೇವ-ತೋಽಗ್ರಜಯುತೋಜಾತೋವ್ರಜಂ
ವಸುದೇವ-ದೇವಕಿ-ಯರ
ವಸುದೇವ-ದೇವಕಿ-ಯ-ರಿಗೆ
ವಸುದೇವ-ನಿಂದ
ವಸುದೇವನು
ವಸುದೇವ-ನೊಡನೆ
ವಸು-ಪದವಿಗೆ
ವಸು-ಪದವಿ-ಯನ್ನು
ವಸುರಭೂದ್ರಾಜಾ
ವಸುವು
ವಸ್ತು-ಗಳ
ವಸ್ತ್ರ-ಗಳನ್ನು
ವಸ್ತ್ರಾಪಹರಣ
ವಹಿಸಿ-ಕೊಳ್ಳು-ವುದು
ವಹ್ನಿವಿಬುಧಾತ್
ವಾಕ್ಯ-ಗಳ
ವಾಕ್ಯ-ಗಳನ್ನು
ವಾಕ್ಯ-ದಲ್ಲಿ
ವಾಗ್ತಾನ
ವಾಗ್ದಾನ
ವಾಚ್ಯತ್ವ
ವಾಪಸ್ಸು
ವಾಮನಾವ-ತಾರ
ವಾಯು
ವಾಯು-ಕುಮಾ-ರ-ನಾದ
ವಾಯು-ಜೀವೋತ್ತಮತ್ವ-ಗಳನ್ನು
ವಾಯು-ದೇ-ವರ
ವಾಯು-ದೇ-ವ-ರಿಂದ
ವಾಯು-ದೇ-ವರು
ವಾಯುಸ್ಪರ್ಶ-ದಿಂದ
ವಾಯೋರ್ಜಿವೋತ್ತಮುತ್ಯಾದಿಕಮಪಿ
ವಾರಣಾವ-ತಕ್ಕೆ
ವಾರಣಾತ್ರ-ದಿಂದ
ವಾರ್ತಾಂ
ವಾರ್ತೆ-ಯನ್ನು
ವಾಲಿನಃ
ವಾಲಿಯ
ವಾಲಿ-ಯನ್ನು
ವಾಲಿ-ಸುಗ್ರೀ-ವರ
ವಾಸ
ವಾಸ-ಮಾಡುತ್ತಿ-ದುದು
ವಾಸ-ಮಾಡುತ್ತಿದ್ದುದು
ವಾಸ-ಮಾಡುವುದು
ವಾಸವೀ
ವಾಸಿಸು-ವಂತೆ
ವಾಸಿ-ಸು-ವುದು
ವಾಸುದೇವ
ವಾಸುದೇವನ
ವಾಸುದೇವನು
ವಾಸುದೇವನೇ
ವಾಸುದೇ-ವಾದಿ
ವಾಸುರನ
ವಾಸ್ತಿ
ವಿಂದ
ವಿಂದ-ಅನು-ವಿಂದರ
ವಿಂದನ
ವಿಕರ್ಣ-ಘೋಣಖ-ಚರೀ-ಬಂಧೂನ್
ವಿಘ್ನಂ
ವಿಘ್ನ-ಗಳನ್ನು
ವಿಚಾರ
ವಿಚಾರ-ಗಳು
ವಿಚಾರ-ದಲ್ಲಿ
ವಿಚಾರ-ವನ್ನು
ವಿಚಾರಿ-ಸು-ವುದು
ವಿಚಿತ್ರ
ವಿಚಿತ್ರ-ವೀರ್ಯನು
ವಿಚಿತ್ರ-ವೀರ್ಯರ
ವಿಚಿನ್ವನ್ನಿವ
ವಿಜಯತೇ
ವಿಜಿತ-ಜರಾಸಂಧ-ಪೂರ್ವಾರಿ-ವರ್ಗಃ
ವಿಜ್ಞಾಪನೆ
ವಿಜ್ಞಾಪಿಸಿ-ಕೊಳ್ಳು-ವುದು
ವಿತರಣೆ
ವಿದರ್ಭಕ್ಕೆ
ವಿದಿತಸು-ವಿದ್ಯಾ-ಮವಾಪುಃ
ವಿದುರ
ವಿದುರನ
ವಿದುರ-ನನ್ನು
ವಿದುರ-ನಿಗೆ
ವಿದುರನು
ವಿದ್ಯಾ-ಧರ-ನೆಂಬ
ವಿದ್ಯಾಭ್ಯಾಸ
ವಿದ್ಯೆ-ಗಳನ್ನೂ
ವಿದ್ಯೆ-ಯನ್ನು
ವಿದ್ಯೋ
ವಿಧಾಯ
ವಿಧೇಯ
ವಿಧೇರತ್ರತಃ
ವಿಧೋರನ್ವಯೇ
ವಿನಃ
ವಿನಹ
ವಿನಾ
ವಿನಾ-ಶವು
ವಿನಿ-ಹತೇ
ವಿಪರೀತ
ವಿಪುಲಧನವೆಚ್ಚ-ದಿಂದ
ವಿಪುಲ-ವಾದ
ವಿಪ್ರ
ವಿಪ್ರ-ತೀ-ಸಾರತಃ
ವಿಪ್ರ-ನಲ್ಲಿ
ವಿಪ್ರ-ಶಾಪಾದ್ಯದು-ಕುಲಮವಧೀದರ್ಥಿತೋಽಗಾತ್
ವಿಪ್ರಸ್ತ್ರೀ
ವಿಫಲರಾಗು-ವುದು
ವಿಭಜನೆ
ವಿಭವ-ನಾಮ
ವಿಭಿದ್ಯ
ವಿಭೀಷಣ-ನಿಂದ
ವಿಭೀಷಣ-ನಿಗೆ
ವಿಭೀಷಣನೂ
ವಿಮಾನ-ದಲ್ಲಿ
ವಿಮೋಚನೆ
ವಿಮೋಚನೆ-ಯನ್ನು
ವಿರಕ್ತಮ್
ವಿರಕ್ತಿ
ವಿರಚಿತಃ
ವಿರಾಟ
ವಿರಾಟನ
ವಿರಾಟ-ನನ್ನು
ವಿರಾಟನು
ವಿರಾಟ-ರಾಜನ
ವಿರಾಟಾರ್ಚಿತಾಃ
ವಿರಾಟಾಲಯಂ
ವಿರೂಪ-ನೇತ್ರ
ವಿರೋಧ
ವಿರೋಧ-ಪರಿಹಾರಕ್ಕಾಗಿ
ವಿರೋಧಿ-ಗಳ
ವಿರೋಧಿ-ಸು-ವುದು
ವಿಲಕ್ಷಣ-ಗಳು
ವಿವರ-ಗಳು
ವಿವರಣೆ
ವಿವರ-ಣೆ-ಯನ್ನು
ವಿವರಿ-ಸಲ್ಪಟ್ಟಿವೆ
ವಿವರಿಸಿ
ವಿವರಿಸಿ-ದರೂ
ವಿವರಿಸಿ-ದುದು
ವಿವರಿಸುವ
ವಿವರಿಸು-ವುದು
ವಿವರ್ಜಿ-ತತ್ವ
ವಿವಾಹ
ವಿವಾಹ-ಮಾಡಿ-ಕೊಂಡು
ವಿಶೇಷ
ವಿಶೇಷ-ವಾಗಿ
ವಿಶೇಷ-ವಾದ
ವಿಶೋಕ
ವಿಶ್ರಾಂತಿ-ಯನ್ನು
ವಿಶ್ವಕರ್ಮ-ನಿಂದ
ವಿಶ್ವ-ರೂಪ-ವನ್ನು
ವಿಶ್ವಾಮಿತ್ರ
ವಿಶ್ವಾಮಿತ್ರರ
ವಿಶ್ವಾಮಿತ್ರ-ರಿಂದ
ವಿಶ್ವಾಮಿತ್ರ-ರೊಡನೆ
ವಿಷಯ
ವಿಷಯಕ್ಕೆ
ವಿಷಯ-ಗಳನ್ನು
ವಿಷಯ-ಗಳನ್ನೂ
ವಿಷಯ-ಗಳು
ವಿಷಯ-ದಲ್ಲಿ
ವಿಷಯ-ವನ್ನು
ವಿಷ-ವನ್ನು
ವಿಷವು
ವಿಷಾದ-ದಿಂದ
ವಿಷ್ಣು
ವಿಷ್ಣು-ವಿನ
ವಿಷ್ಣು-ವಿ-ನಲ್ಲಿ
ವಿಷ್ಣು-ವಿ-ನಿಂದಲೂ
ವಿಷ್ಣುವೇ
ವಿಷ್ಣೋಃ
ವಿಹರಿ-ಸು-ವುದು
ವಿಹಾರ
ವಿಹಾರ-ಮಾಡಿ-ದುದು
ವೀಕ್ಷ್ಯ
ವೀರನಾ-ದುದು
ವೀರರ
ವೀರ-ರಲ್ಲಿ
ವೀರರೆಲ್ಲರೂ
ವೀರ್ಯನ
ವುದು
ವೃಂದಾ-ವನಕ್ಕೆ
ವೃಂದಾವನ-ದಿಂದ
ವೃಕ್ಷ-ಗಳನ್ನು
ವೃಕ್ಷ-ದಲ್ಲಿ
ವೃಕ್ಷ-ವನ್ನು
ವೃತ್ತಾಂತ
ವೃತ್ತಾಂತ-ವನ್ನು
ವೃತ್ತಾಂತ-ವನ್ನೂ
ವೃದ್ದ
ವೃಷಾಶ್ವನ
ವೆಸ್ಟ್
ವೇಣುಗಾನ
ವೇತನ-ಗಳನ್ನು
ವೇದ
ವೇದ-ಗಳ
ವೇದ-ಗಳಲ್ಲಿ
ವೇದ-ಗಳಿಂದಲೇ
ವೇದಪ್ರತಿ-ಪಾದ್ಯತ್ವ
ವೇದ-ವಾಕ್ಯ-ಗಳ
ವೇದ-ವಿವೃತಿಂ
ವೇದವ್ಯಾಸ
ವೇದವ್ಯಾಸ-ದೇ-ವರ
ವೇದವ್ಯಾಸರ
ವೇದವ್ಯಾಸ-ರನ್ನು
ವೇದವ್ಯಾಸ-ರಾಗಿ
ವೇದವ್ಯಾಸರು
ವೇದವ್ಯಾಸ-ರು-ವಿಪ್ರೋತ್ತಮರು
ವೇದವ್ಯಾಸ-ರೂಪ-ದಿಂದಿದ್ದ
ವೇದವ್ಯಾಸಶ್ರೀ-ಕೃಷ್ಣ
ವೇದ-ಶಾತ್ರ-ಗಳಲ್ಲಿ
ವೇದಾ
ವೇದಾಂತ
ವೇದಾ-ಧಿಕಾರ
ವೇದಾಭ್ಯಾಸ
ವೇದೈರ್ವೆದ್ಯೋಽಸ್ತದೋಷೋಽಪ್ಯ
ವೇದೋತ್ಕೃಷ್ಟಸ್ಯ
ವೇಶ-ದಲ್ಲಿ
ವೇಷ-ಗಳನ್ನು
ವೇಷದ
ವೇಷ-ದಲ್ಲಿ
ವೇಷ-ದಲ್ಲಿದ್ದ
ವೇಷ-ದಲ್ಲಿಯೇ
ವೇಷದಿಂದ
ವೇಷಮುಪಾ-ಗತಾಃ
ವೇಷಾಂತರ-ದಿಂದ
ವೈಕುಂಠ
ವೈಕುಂಠಕ್ಕೆ
ವೈಕುಂಠ-ದಿಂದ
ವೈಕುಂಠ-ಲೋಕಕ್ಕೆ
ವೈಖರಿ
ವೈಭವ
ವೈಭವ-ವನ್ನು
ವೈರಾಗ್ಯ
ವೈರಾಗ್ಯ-ಗಳಲ್ಲಿ
ವೈರಾಗ್ಯವು
ವೈಶ್ಯನ
ವೈಶ್ಯರ
ವೈಷ್ಣವ
ವೈಷ್ಣವ-ಧರ್ಮ-ನಿಷ್ಠೆ
ವೈಷ್ಣವಾತ್ರ-ದಿಂದ
ವ್ಯಕ್ತ-ಪಡಿಸಿ
ವ್ಯಕ್ತ-ಪಡಿ-ಸು-ವುದು
ವ್ಯಥತಯೋ
ವ್ಯಧಾತ್
ವ್ಯಧಿತ-ನಗರೀಂ
ವ್ಯಧಿತಸುವಚಸಾಮುದ್ಧೃತಿಂ
ವ್ಯರ್ಥವಲ್ಲ-ವೆಂಬ
ವ್ಯವಸ್ಥೆ
ವ್ಯಸ್ಯ
ವ್ಯಾಖ್ಯಾನ
ವ್ಯಾಖ್ಯಾನದ
ವ್ಯಾಖ್ಯಾನ-ಮಾಡುತ್ತಾ
ವ್ಯಾಜ-ದಿಂದ
ವ್ಯಾಧನ
ವ್ಯಾಧ-ನಾಗಿ
ವ್ಯಾಧ-ರಿಂದ
ವ್ಯಾಪಾರ-ಗಳು
ವ್ಯಾಪಾರ-ದಂತೆಯೇ
ವ್ಯಾಸತ್ವೇನ
ವ್ಯಾಸ-ಮೀಡೇತಮೀಶಮ್
ವ್ಯಾಸರ
ವ್ಯಾಸರು
ವ್ಯಾಸಸ್ವ-ರೂಪಾದಪಿ
ವ್ಯಾಸಾತ್ಮಾ
ವ್ಯಾಸಾವ-ತಾರ-ವನ್ನು
ವ್ರತ-ದಿಂದ
ಶಂಕೆ-ಯಿಂದ
ಶಂಕೆಯು
ಶಂಖಚೂಡನ
ಶಂಖ-ವನ್ನೂ
ಶಂತನು
ಶಂತನು-ಗಂಗಾ-ದೇವಿಯ
ಶಂತನು-ವನ್ನು
ಶಂತನು-ವಿನ
ಶಂತನು-ಸತ್ಯವತಿ-ಯರ
ಶಂಫಲ-ವೆಂಬ
ಶಂಬರಾ-ಸುರನ
ಶಂಬರಾಸುರರತಿ
ಶಕಟಾಕ್ಷಹಾ
ಶಕಟಾಸುರ
ಶಕಟಾ-ಸುರನ
ಶಕಟಾಸುರ-ನನ್ನೂ
ಶಕಬಾಬ್ಬ
ಶಕುನಿಯ
ಶಕುನಿ-ಯನ್ನು
ಶಕುನಿಯು
ಶಕ್ತ-ನಾಗಿದ್ದರೂ
ಶಕ್ತಿ-ಯನ್ನು
ಶಕ್ತಿಯುಕ್ತ-ನಾದರೂ
ಶಕ್ತಿ-ಸಾಮರ್ಥ-ಗಳನ್ನು
ಶಕ್ಯಾಖ್ಯ
ಶಚಿ
ಶಚೀ
ಶತಧನ್ವನ
ಶತಧನ್ವ-ನನ್ನು
ಶತಧನ್ವನೋಽಷ್ಟಮಹಿಷೀಭರ್ತಾ
ಶತ್ರುಘ್ನರ
ಶತ್ರು-ನಾಶ
ಶತ್ರುಪಕ್ಷದ
ಶತ್ರು-ರಾಜರ
ಶತ್ರು-ರಾಜ-ರನ್ನು
ಶತ್ರು-ಸಮೂಹ-ವನ್ನು
ಶತ್ರುಸೇನೆಯ
ಶಪಥಂ
ಶಬರಿಯ
ಶಬ್ದಕ್ಕೆ
ಶಬ್ದ-ವನ್ನು
ಶಮಾ
ಶಮೀ-ವೃಕ್ಷ-ದಲ್ಲಿದ್ದ
ಶರಣು
ಶರಭಂಗರು
ಶರೀರ-ಗಳೂ
ಶರೀರ-ವನ್ನು
ಶರೀರವು
ಶಲೈ
ಶಲ್ಯನ
ಶಲ್ಯ-ನನ್ನು
ಶಲ್ಯನು
ಶಲ್ಯ-ಮ-ವಾಸ್ಯ
ಶವ-ಗಳ
ಶವ-ಗಳನ್ನು
ಶಸ್ತ್ರ
ಶಸ್ತ್ರ-ವಿದ್ಯೆ-ಗಳನ್ನೂ
ಶಸ್ತ್ರ-ಹೀನ-ನನ್ನಾಗಿ
ಶಸ್ತ್ರಾಭ್ಯಾಸ
ಶಸ್ತ್ರಾಸ್ತ್ರ
ಶಸ್ತ್ರಾತ್ರ-ಗಳನ್ನು
ಶಸ್ತ್ರಾತ್ರ-ಗಳಿಂದ
ಶಸ್ತ್ರೋಜ್ಝಿತಂ
ಶಾಂತಿ
ಶಾಪ
ಶಾಪದ
ಶಾಪ-ದಿಂದ
ಶಾಪಪ್ರಾಪ್ತಿ
ಶಾಪ-ವಿಮೋಚನೆ
ಶಾಪ-ವಿಮೋಚನೆ-ಯನ್ನು
ಶಾಸ್ತ್ರ-ಗಳ
ಶಾಸ್ತ್ರ-ಗಳನ್ನೂ
ಶಾಸ್ತ್ರದ
ಶಾಸ್ತ್ರಾಣಿ
ಶಾಸ್ಪೋಕ್ತ
ಶಿಂಶುಪಾವೃಕ್ಷದ
ಶಿಕ್ಷಣ
ಶಿಕ್ಷಿತನಾ-ದಂತೆ
ಶಿಕ್ಷಿ-ಸಿದ
ಶಿಖಂಡಿಯ
ಶಿಖಂಡಿಯು
ಶಿಥಿಲ
ಶಿಬಿರ
ಶಿಬಿರಕ್ಕೆ
ಶಿಬಿರಗಂ
ಶಿಬಿರ-ಗಳನ್ನು
ಶಿಬಿರ-ದಲ್ಲಿ
ಶಿಬಿರ-ದಿಂದ
ಶಿರಸ್ಸನ್ನು
ಶಿರಸ್ಸಿನ
ಶಿರಸ್ಸಿ-ನಲ್ಲಿ
ಶಿರಸ್ಸಿನಲ್ಲಿದ್ದ
ಶಿರಸ್ಸು-ಗಳನ್ನು
ಶಿರೋ-ರತ್ನ-ವನ್ನು
ಶಿಲೆ-ಯಾಗಿದ್ದ
ಶಿಲೋಂಛವೃತ್ತಿ-ಯಿಂದ
ಶಿವ
ಶಿವಂ
ಶಿವ-ಧನುಸ್ಸನ್ನು
ಶಿವ-ಧನುಸ್ಸಿನ
ಶಿಶು-ಗಳನ್ನು
ಶಿಶುಪಾಲದಂತವಕ್ರರ
ಶಿಶುಪಾಲದಂತವಕ್ರ-ರಿಂದ
ಶಿಶುಪಾಲನ
ಶಿಶುಪಾಲ-ನಿಂದ
ಶಿಶು-ರೂಪ-ದಿಂದ
ಶಿಶು-ವನ್ನು
ಶಿಶು-ವಿನ
ಶಿಶುವು
ಶಿಷೇಭ್ಯಃ
ಶಿಷ್ಯ-ರಿಗೆ
ಶಿಷ್ಯರೆಲ್ಲರೂ
ಶೀರ್ಣಾಂಗಂ
ಶುಕಾಚಾರ್ಯರ
ಶುಕ್ರಾಚಾರ್ಯ-ರಿಂದ
ಶುದ್ದರಾಗು-ವುದು
ಶುದ್ಧ
ಶುದ್ಧೋದನಗಯಾ
ಶುದ್ಧೋದ-ನಾದಿ-ಗಳಿಗೆ
ಶುದ್ಧೋದನಾ-ದಿ-ಗಳು
ಶುಭ-ಗುಣ-ಗಳಿಂದ
ಶುಶ್ರಾವಾಖಿಲ-ಧರ್ಮ-ನಿರ್ಣಯ-ವುದಃ
ಶೂದ್ರ-ನಾಗಿ
ಶೂದ್ರ-ಳಾದ
ಶೂನ್ಯ
ಶೂನ್ಯ-ಗೃಹ-ದಲ್ಲಿ
ಶೂರ್ಪಣಖಿಯ
ಶೃಂಖಲೆ-ಯಿಂದ
ಶೃಗಾಲ-ವಾಸುದೇವ-ನನ್ನು
ಶೃಗಾಲ-ವಾಸುದೇವನು
ಶೇಷ-ದೇ-ವರ
ಶೇಷ-ದೇ-ವರು
ಶೇಷ-ನೆಂದು
ಶೇಷಶಾಯಿ-ಯಾದ
ಶೇಷಾದಿ-ಗಳಿಗಿಂತ
ಶೋಕ
ಶೋತೃ-ಗಳಿಗೆ
ಶ್ಯಾಮಲಾ
ಶ್ರವಣ
ಶ್ರವಣ-ಮಾಡುತ್ತಿ-ದುದು
ಶ್ರಾದ್ಧ
ಶ್ರಾದ್ಧಾದಿ
ಶ್ರೀ
ಶ್ರೀಅಚ್ಯುತನು
ಶ್ರೀಕೃಷ
ಶ್ರೀಕೃಷ್ಣ
ಶ್ರೀಕೃಷ್ಣನ
ಶ್ರೀಕೃಷ್ಣ-ನನ್ನು
ಶ್ರೀಕೃಷ್ಣ-ನಿಂದ
ಶ್ರೀಕೃಷ್ಣ-ನಿಗೆ
ಶ್ರೀಕೃಷ್ಣನು
ಶ್ರೀಕೃಷ್ಣನೇ
ಶ್ರೀಕೃಷ್ಣ-ನೊಡನೆ
ಶ್ರೀಕೃಷ್ಣ-ನೊಬ್ಬನೇ
ಶ್ರೀಕೃಷ್ಣ-ರುಕ್ಷ್ಮಿಣಿ-ಯರ
ಶ್ರೀನಾರಾಯಣನ
ಶ್ರೀಪತಿಃ
ಶ್ರೀಪೂರ್ಣಪ್ರಜ್ಞರು
ಶ್ರೀಮದಾಚಾರ್ಯರ
ಶ್ರೀಮದಾಚಾರ್ಯರು
ಶ್ರೀಮದಾಚಾರ್ಯಾಃ
ಶ್ರೀಮ-ದಾನಂದ-ತೀರ್ಥ-ರಿಂದ
ಶ್ರೀಮ-ದಾನಂದ-ತೀರ್ಥರು
ಶ್ರೀಮಧ್ವ-ರಿಗೆ
ಶ್ರೀಮನ್ನಾರಾಯಣನ
ಶ್ರೀಮನ್ನಾರಾಯಣ-ನನ್ನು
ಶ್ರೀಮನ್ನಾರಾಯಣನೇ
ಶ್ರೀಮನ್ಮಧ್ವ-ಸಿದ್ಧಾಂತ
ಶ್ರೀಮನ್ಮಹಾ-ಭಾರತ
ಶ್ರೀಮನ್ಮಹಾ-ಭಾರತ-ತಾತ್ಪರ್ಯ-ನಿರ್ಣಯ
ಶ್ರೀಮನ್ಮಹಾ-ಭಾರತ-ತಾತ್ಪರ್ಯ-ನಿರ್ಣಯದ
ಶ್ರೀಮಹಾ-ಲಕ್ಷ್ಮಿಯೂ
ಶ್ರೀಮುಕುಂದನು
ಶ್ರೀರಾಮ-ಚಂದ್ರ-ನನ್ನು
ಶ್ರೀರಾಮ-ದೇ-ವರ
ಶ್ರೀರಾಮನ
ಶ್ರೀರಾಮ-ನನ್ನು
ಶ್ರೀರಾಮ-ನನ್ನೂ
ಶ್ರೀರಾಮ-ನಾಗಿ
ಶ್ರೀರಾಮ-ನಿಂದ
ಶ್ರೀರಾಮ-ನಿಗೆ
ಶ್ರೀರಾಮನು
ಶ್ರೀರಾಮ-ನೊಡನೆ
ಶ್ರೀರಾಮ-ರಾವಣರ
ಶ್ರೀರಾಮ-ರೂಪ-ದಿಂದ
ಶ್ರೀರಾಮ-ಲಕ್ಷಣರು
ಶ್ರೀರಾಮ-ಸೀತಾ-ದೇವಿ-ಯರ
ಶ್ರೀರಾಮ-ಸುಗ್ರೀ-ವರ
ಶ್ರೀವಾಯು-ದೇ-ವರ
ಶ್ರೀವಾಯು-ದೇ-ವರಿಗೆ
ಶ್ರೀವೇದವ್ಯಾಸ-ರಿಂದ
ಶ್ರೀವೇದವ್ಯಾಸರು
ಶ್ರೀಹನುಮಂತನು
ಶ್ರೀಹನುಮಾನನುಗ್ರಹಬಲಾತ್ತೀರ್ಣಾಂಬುಧಿರ್ಲಿಲಯಾ
ಶ್ರೀಹರಿ
ಶ್ರೀಹರಿ-ನಾಮೋಚ್ಛಾರಣೆ
ಶ್ರೀಹರಿಯ
ಶ್ರೀಹರಿ-ಯನ್ನು
ಶ್ರೀಹರಿ-ಯಲ್ಲಿ
ಶ್ರೀಹರಿ-ಯಿಂದ
ಶ್ರೀಹರಿಯು
ಶ್ರೀಹರಿಯೇ
ಶ್ರೀಹರಿ-ಯೊಬ್ಬನೇ
ಶ್ರೀಹ್ರೀ
ಶ್ರುತಿ-ವಾಕ್ಯ-ಗಳ
ಶ್ರುತ್ವಾ
ಶ್ರೇಯಸ್ಕರವೆಂದೂ
ಶ್ರೇಷ್ಠ
ಶ್ರೇಷ್ಠತೆ
ಶ್ರೇಷ್ಠ-ತೆಯ
ಶ್ರೇಷ್ಠ-ನಾದ
ಶ್ರೇಷ್ಠ-ರಿಂದ
ಶ್ರೇಷ್ಠ-ರು-ಗಳು
ಶ್ರೇಷ್ಠ-ರೆಂದು
ಶ್ರೇಷ್ಠ-ವಾದ
ಶ್ಲಾಘನೆ
ಶ್ಲೋಕ-ಗಳನ್ನು
ಶ್ಲೋಕ-ಗಳಲ್ಲಿ
ಶ್ಲೋಕ-ಗಳಿವೆ
ಶ್ಲೋಕದ
ಶ್ಲೋಕ-ವನ್ನು
ಶ್ಲೋಕ-ವನ್ನೂ
ಶ್ಲೋಕವು
ಶ್ವಾಸ-ನಾತ್ರ-ವನ್ನು
ಶ್ವೇತ-ವರಾಹ
ಷಂಡ-ನಾಗಿ
ಷೋತ್ತಮ-ರಲ್ಲಿ
ಸ
ಸಂ
ಸಂಕಟ
ಸಂಕರ್ಷಣ
ಸಂಕರ್ಷಣ-ಜಯಾ
ಸಂಕಲ್ಪ
ಸಂಕ್ಷಿಪ್ತ
ಸಂಕ್ಷಿಪ್ತ-ವಾಗಿ
ಸಂಕ್ಷೇಪ-ವಾಗಿ
ಸಂಖ್ಯೆ-ಯಲ್ಲಿ
ಸಂಗಡ
ಸಂಚು
ಸಂಜಯ
ಸಂಜಯನ
ಸಂಜಯ-ನನ್ನು
ಸಂಜಯ-ನಿಂದ
ಸಂಜಯನು
ಸಂತಾನ-ವನ್ನು
ಸಂತು
ಸಂತುಷ್ಟ-ನಾದ
ಸಂತೈ-ಸಿದ
ಸಂತೋಷ
ಸಂತೋಷ-ಗೊಂಡ
ಸಂತೋಷ-ದಿಂದ
ಸಂತೋಷ-ದಿಂದಲೂ
ಸಂತೋಷ-ಪಡಿಸಿ
ಸಂತೋಷ-ಪಡಿ-ಸು-ವುದು
ಸಂತೋಷ-ಪಡು-ವುದು
ಸಂತೋಷ-ವನ್ನು
ಸಂತೋಷ-ವಾಗಿ
ಸಂತೋಷ-ಸಡಗರ
ಸಂದೇಹ
ಸಂದೇಹ-ವನ್ನು
ಸಂಧಾನಕ್ಕಾಗಿ
ಸಂಧಾನಕ್ಕೆ
ಸಂಪತ್ತನ್ನು
ಸಂಪತ್ತು
ಸಂಪಾತಿಯ
ಸಂಪಾದಿ-ಸಲು
ಸಂಪ್ರದಾಯ-ವನ್ನೇ
ಸಂಪ್ರಾಪ್ಯ
ಸಂಬಂಧ-ಪಟ್ಟ
ಸಂಭಾಷಣೆ
ಸಂವತ್ಸರ
ಸಂವತ್ಸರಕ್ಕೆ
ಸಂವತ್ಸರ-ಗಳಲ್ಲಿ
ಸಂವತ್ಸರ-ಗಳಿಗೆ
ಸಂವತ್ಸರ-ಗಳು
ಸಂವಾದ
ಸಂಶಯ
ಸಂಶಯ-ದಿಂದ
ಸಂಸ್ಕಾರ
ಸಂಸ್ಕಾರ-ಗಳನ್ನು
ಸಂಸ್ಕಾರಾ-ದಿ-ಗಳು
ಸಂಸ್ಕಾರಾನ್
ಸಂಸ್ಥಾಪಿಸುವ-ನೆಂಬ
ಸಂಹರಿ-ಸ-ತಕ್ಕ
ಸಂಹರಿ-ಸಲು
ಸಂಹರಿಸಿ
ಸಂಹರಿ-ಸಿದ
ಸಂಹರಿ-ಸಿ-ದನೋ
ಸಂಹರಿ-ಸಿ-ದುದು
ಸಂಹರಿ-ಸು-ವು-ದಾಗಿ
ಸಂಹರಿ-ಸು-ವುದು
ಸಂಹರಿ-ಸು-ವುದೆಂಬ
ಸಂಹರ್ತಾ
ಸಂಹಾರ
ಸಂಹಾರ-ಕನು
ಸಂಹಾರಕ್ಕಾಗಿ
ಸಂಹಾರಕ್ಕೆ
ಸಂಹಾರ-ಮಾಡದೇ
ಸಂಹಾರ-ಮಾಡಿಸಿ
ಸಂಹಾರ-ಮಾಡಿ-ಸಿದ
ಸಂಹಾರ-ವನ್ನು
ಸಂಹಾರವು
ಸಂಹಾರ-ವೆಂದು
ಸಃ
ಸಕಲ
ಸಕಲ-ಗುಣ
ಸಕಲ-ದೋಷ-ವರ್ಜಿತ-ನಾದ
ಸಕಲರೂ
ಸಖಿತ್ವಮಾಪ್ಯ
ಸಖಿಯ
ಸಖ್ಯ
ಸಚ್ಚಾ
ಸಚ್ಛಾ
ಸಚ್ಛಾತ್ರ-ಗಳ
ಸಜ್ಜನರ
ಸಜ್ಜನ-ರಲ್ಲಿ
ಸಜ್ಜನ-ರಿಂದ
ಸಜ್ಜನ-ರಿಗೆ
ಸಜ್ಜನರು
ಸಜ್ಜನಸಂವಿದೇ
ಸಜ್ಞನ
ಸತೀಮಾತ್ಮನಃ
ಸತ್ಕರ್ಮಕ್ಕೆ
ಸತ್ಕರ್ಮ-ಗಳು
ಸತ್ಕರ್ಮಾನುಷ್ಠ-ರಾಗಿ
ಸತ್ಕರ್ಮಾನುಷ್ಠಾನದ
ಸತ್ಪಾತ್ರ-ರಲ್ಲಿ
ಸತ್ಯ-ಜಿತು-ವಿನ
ಸತ್ಯಭಾಮಾ
ಸತ್ಯಭಾಮಾ-ದೇವಿಯ
ಸತ್ಯಭಾಮಾ-ದೇವಿ-ಯನ್ನು
ಸತ್ಯಭಾಮಾ-ದೇವಿ-ಯ-ರಿಂದ
ಸತ್ಯಭಾಮಾ-ದೇವಿ-ಯಿಂದ
ಸತ್ಯಭಾಮೆ-ಯರೂ
ಸತ್ಯಭಾಮೆ-ಯೊಡನೆ
ಸತ್ಯರ್ಮಾನುಷ್ಠಾನ
ಸತ್ಯವತೀ
ಸತ್ಯವತೀ-ಅಂಬಿಕಾ-ಅಂಬಾಲಿಕೆಯ-ರಿಗೆ
ಸತ್ಯವತೀ-ದೇವಿ-ಯಲ್ಲಿ
ಸತ್ರಾಜಿತನ
ಸತ್ರಾ-ಜಿತ್
ಸತ್ರಾಜಿ-ದಾತ್ಮಜಾಪತಿರಸೌ
ಸತ್ವ-ವನ್ನು
ಸತ್ಸು
ಸತ್ಜ್ಞಾನಾಯ
ಸದಾ
ಸದಾ-ಗಮ-ಗಳ
ಸದೃಶ-ವಾದ
ಸದ್ಗತಿ
ಸದ್ಗತಿ-ಯನ್ನು
ಸದ್ಗೀತಾಮುಪದಿಶ್ಯ
ಸದ್ಗ್ರಂಥಾನಾಂ
ಸದ್ದತಿಯ
ಸದ್ಧತಿ-ಯನ್ನು
ಸದ್ಧರ್ಮೇ
ಸದ್ಭಿರ್ಯುಕ್ತೋ
ಸದ್ರು-ಪದೋಪಿ
ಸನಕಸ-ನಂದಾದಿ-ಗಳು
ಸನ್ನಾಹ
ಸನ್ನಿವೇಶ
ಸನ್ಮಾನ-ಮಾಡುವುದು
ಸನ್ಮಾನಿತ-ನಾದ
ಸನ್ಯಾಸ-ವೇಷದಿಂದ
ಸನ್ಯಾಸಿ-ವೇಷದ
ಸಪತಿಂ
ಸಭಾ
ಸಭಾನ್ವಿಧಾಯ
ಸಭಾ-ಮಂದಿ-ರ-ವನ್ನು
ಸಭೆ
ಸಭೆಯ
ಸಭೆ-ಯಲ್ಲಿ
ಸಭೆ-ಯಿಂದ
ಸಮಕಾರಯತ್
ಸಮಜಯದ್ದು
ಸಮ-ನಾಗಿ
ಸಮ-ನಾದ
ಸಮಬೋಧಯನ್ಮೃತ-ಶಿಶುಂ
ಸಮಯ-ದಲ್ಲಿ
ಸಮರ
ಸಮರ್ಥನೆ
ಸಮರ್ಥ-ನೆಂದು
ಸಮರ್ಥನೆ-ಮಾಡುವ
ಸಮರ್ಥ-ರಾದರೂ
ಸಮರ್ಪಿಸಿ
ಸಮರ್ಪಿ-ಸು-ವುದು
ಸಮಸ್ತ
ಸಮಸ್ತ-ದೋಷವಿ-ವರ್ಜಿ
ಸಮಸ್ತೆ
ಸಮಾಚಾರ-ವನ್ನೂ
ಸಮಾಜ
ಸಮಾಜಕ್ಕೆ
ಸಮಾಧಾನ
ಸಮಾಧಾನ-ಪಡಿಸಿ
ಸಮಾಧಾನ-ಪಡಿ-ಸು-ವುದು
ಸಮಾಧಿ
ಸಮಾಧಿ-ಭಾಷೆ
ಸಮಾನ
ಸಮಾಪ್ತಿ
ಸಮಾವಿಷ್ಟ-ರಾಗಿ-ರು-ವುದು
ಸಮೀಚೀನ
ಸಮೀಪಕ್ಕೆ
ಸಮುದಾಯದ
ಸಮುದ್ರ
ಸಮುದ್ರಕ್ಕೆ
ಸಮುದ್ರ-ಗಳನ್ನು
ಸಮುದ್ರದ
ಸಮುದ್ರ-ದಲ್ಲಿ
ಸಮುದ್ರ-ದಿಂದ
ಸಮುದ್ರ-ವನ್ನು
ಸಮೂಹ-ದಿಂದ
ಸಮೂಹ-ವನ್ನು
ಸಮೂಹೇ
ಸಮೇತ
ಸಮ್ಮತಿ-ಯಿಂದ
ಸಮ್ಮೋಹ-ನಾತ್ರ-ದಿಂದ
ಸರಸ
ಸರಿ-ಸಮರಿಲ್ಲ
ಸರೋ-ವರದ
ಸರೋ-ವರ-ದಲ್ಲಿ
ಸರೋ-ವರ-ವನ್ನು
ಸರ್ಪ-ಗಳಿಂದ
ಸರ್ಪ-ವನ್ನು
ಸರ್ಪಾತ್ರ-ದಿಂದ
ಸರ್ವಜ-ಗತೋ
ಸರ್ವದಾ
ಸರ್ವಪ್ರಯತ್ನ-ವನ್ನೂ
ಸರ್ವ-ರನ್ನೂ
ಸರ್ವ-ಶಾತ್ರ-ಗಳನ್ನೂ
ಸರ್ವಾಣ್ಯಪಿ
ಸರ್ವಾಲಂಕಾರಭೂಷಣ-ಗಳಿಂದ
ಸರ್ವಾಲಂಕಾರಭೂಷಿತ-ನಾದ
ಸರ್ವೆಶ್ವ-ರ-ನಾದ
ಸರ್ವೊತ್ತಮತ್ವ
ಸರ್ವೊತ್ತಮತ್ವ-ವನ್ನು
ಸರ್ವೊತ್ತಮತ್ವ-ವನ್ನೂ
ಸರ್ವೊತ್ಯಷ್ಟ-ವಾದ
ಸರ್ವೊಮತ್ವ-ವನ್ನೂ
ಸರ್ವೋತ್ತಮತ್ವ
ಸರ್ವೋತ್ತಮತ್ವದ
ಸರ್ವೋತ್ತಮತ್ವಾದಿ
ಸರ್ವೋತ್ತಮ-ನೆಂದು
ಸಲಹಲಿ
ಸಲಹೆ
ಸಲಹೆ-ಯಂತೆ
ಸಲ್ಲಾಪ
ಸವತಿ-ಯಾದ
ಸಹ
ಸಹಕಾರದೊಂದಿಗೆ
ಸಹಗಮನ
ಸಹ-ಜರಾಸಂಧ್ಯೆರ್ನೃಪೈರ್ನೀತಯೇ
ಸಹಜ-ವಾದ
ಸಹದೇವ
ಸಹದೇವ-ನಿಂದ
ಸಹದೇವನು
ಸಹಸ್ರ
ಸಹಾಯಕ-ನಾಗಿ-ರು-ವುದು
ಸಹಾಯಕ್ಕಾಗಿ
ಸಹಾಯ-ದಿಂದ
ಸಹಾಯ-ಮಾಡಲು
ಸಹಾಯ-ಮಾಡಿದ
ಸಹಾಯ-ವನ್ನು
ಸಹಿತ
ಸಹಿತ-ನಾಗಿ
ಸಹಿತ-ನಾದ
ಸಹಿತ-ರಾಗಿ
ಸಹಿತ-ರಾದ
ಸಹಿತ-ಳಾದ
ಸಹೋದ-ರ-ನಾದ
ಸಹೋದ-ರ-ರನ್ನೂ
ಸಹೋದ-ರ-ರಿಗೆ
ಸಹೋದ-ರ-ರೊಡನೆ
ಸಹೋದ-ರಿ-ಯನ್ನು
ಸಾಂದೀಪಿನೀ
ಸಾಂಬ-ಲಕ್ಷಣಾ
ಸಾಕಲ್ವೇನ
ಸಾಕೆಂದು
ಸಾಕ್ಷಾತ್
ಸಾಕ್ಷಿ-ಯಾಗಿ
ಸಾತ್ಯಕಿ-ಕೃತ-ವರ್ಮರ
ಸಾತ್ಯಕಿ-ಯರು
ಸಾತ್ಯಕಿ-ಯಿಂದ
ಸಾತ್ಯಕಿಶಿಶುಪಾಲರ
ಸಾಧ್ಯವಾಗುವ
ಸಾಧ್ಯವಿಲ್ಲ
ಸಾನ್ನಿಧ್ಯ
ಸಾಮರ್ಥ
ಸಾಮರ್ಥದ
ಸಾಮರ್ಥ-ವನ್ನೂ
ಸಾಮರ್ಥ-ಹೀನನಾಗು-ವುದು
ಸಾಮರ್ಥ್ಯ
ಸಾಮರ್ಥ್ಯ-ವನ್ನು
ಸಾಮಾದಿ-ಗಳ
ಸಾಮಾನ್ಯ
ಸಾಮಾನ್ಯ-ರೂಪ-ದಿಂದ
ಸಾಮ್ರಾಜ್ಯ
ಸಾಮ್ರಾಜ್ಯ-ದಲ್ಲಿಯೂ
ಸಾಯಕೈಃ
ಸಾಯು-ವುದು
ಸಾರಥಿಃ
ಸಾರ-ಥಿ-ಯನ್ನಾಗಿ
ಸಾರ-ಥಿ-ಯಾಗಿ
ಸಾರ-ಥಿಯಾಗು-ವುದು
ಸಾರ-ಥಿ-ಯಾದ
ಸಾರ-ಥಿ-ವರಂ
ಸಾರ-ವಾಗಿ
ಸಾರವು
ಸಾರಾಂಶ
ಸಾರಿ
ಸಾರುವ
ಸಾರು-ವುದು
ಸಾರ್ಥಕ-ವೆಂದು
ಸಾಲ್ವನ
ಸಾಲ್ವ-ನನ್ನು
ಸಾಲ್ವನು
ಸಾವಿರ
ಸಾವಿರದ
ಸಾಹಾಯ್ಯಂ
ಸಿಂಧುಂ
ಸಿಂಧು-ರಾಜಮ್
ಸಿಂಹ-ನಾದ
ಸಿಂಹ-ನಾದ-ದಿಂದ
ಸಿಂಹಿಕಾ
ಸಿಕ್ಕಿ-ದಂತೆ
ಸಿದರೂ
ಸಿದ್ಧನಾಗು-ವುದು
ಸಿದ್ಧ-ನಾದ
ಸಿದ್ಧರಾಗು-ವುದು
ಸಿದ್ಧ-ರಾದ
ಸೀತಾ
ಸೀತಾಂ
ಸೀತಾ-ಕೃತಿ-ಯನ್ನು
ಸೀತಾ-ಕೃತಿಯು
ಸೀತಾ-ದೇವಿಗೆ
ಸೀತಾ-ದೇವಿಯ
ಸೀತಾ-ದೇವಿ-ಯನ್ನು
ಸೀತಾ-ದೇವಿ-ಯನ್ನೂ
ಸೀತಾ-ದೇವಿ-ಯಾಗಿ
ಸೀತಾ-ದೇವಿ-ಯಿಂದ
ಸೀತಾ-ದೇವಿಯು
ಸೀತಾ-ದೇವಿ-ಯೊಡನೆ
ಸೀತಾನ್ವೇಷಣಮಿಚ್ಛತಾಬ್ಧಿತ-ರಣೇ
ಸೀತಾ-ಲಕ್ಷ್ಮಣ-ರಿಂದ
ಸೀತಾ-ಹೇತೋರ್ನಿ-ಮೋಹ್ಯ
ಸೀತೆ-ಯನ್ನು
ಸೀತೆ-ಯಿಂದ
ಸೀಳಿ
ಸೀಳಿ-ದುದು
ಸೀಳು-ವು-ದಾಗಿ
ಸೀಳು-ವುದು
ಸುಂದರ
ಸುಂದರ-ವಾಗಿಯೂ
ಸುಖನಿಧಿರಮಿತೈರ್ವಾಸುದೇ-ವಾದಿ-ರೂಪೈಃ
ಸುಖ-ಸಂತೋಷ-ಗಳು
ಸುಖ-ಸಂತೋಷ-ದಿಂದಲೂ
ಸುಖಾತ್ಮ-ನಾದ
ಸುಗ್ರೀವ
ಸುಗ್ರೀವ-ಕುಂಭ-ಕರ್ಣರ
ಸುಗ್ರೀವನ
ಸುಗ್ರೀವ-ನನ್ನು
ಸುಗ್ರೀವ-ನಾಗಿ
ಸುಗ್ರೀವ-ನಿಗೆ
ಸುಗ್ರೀವ-ನೊಡನೆ
ಸುಗ್ರೀವೇಣ
ಸುಟ್ಟು
ಸುಡು-ವುದು
ಸುತಾರ್ಥಂ
ಸುತೇ
ಸುತೌ
ಸುದಕ್ಷಿಣನ
ಸುದೇಷ್ಟೆಯ
ಸುದೇಷ್ಟೆ-ಯನ್ನು
ಸುಬ್ಬಣ್ಣಾಚಾರ್
ಸುಬ್ಬಣ್ಣಾಚಾರ್ಯ
ಸುಭದ್ರಾ
ಸುಭದ್ರಾ-ಚಿತ್ರಾಂಗದೆ-ಯ-ರೊಡನೆ
ಸುಭದ್ರಾರ್ಜುನರ
ಸುಭದ್ರೆಯ
ಸುಭದ್ರೆ-ಯನ್ನು
ಸುಭದ್ರೆ-ಯಿಂದ
ಸುಭಿಕ್ಷೆ
ಸುಯೋಗ
ಸುಯೋಧನ-ನಾಗಿ
ಸುಯೋಧನ-ನಿಗೆ
ಸುಯೋಧನನು
ಸುಯೋಧನನೇ
ಸುಯೋಧನಾಧಿ-ಗಳು
ಸುರಂಗ-ಮಾರ್ಗ-ದಿಂದ
ಸುರಂಗ-ಮಾರ್ಗ-ವನ್ನು
ಸುರಜನಮನೋ
ಸುರತರುಂ
ಸುರನ
ಸುರರ್ಷಿಸ್ತುತಃ
ಸುರ-ವರೈಃ
ಸುರ-ವರೈರಿಚ್ಛನ್
ಸುರಾಣಕರು
ಸುರಾಣಾಂ
ಸುವರ್ಣ
ಸುವರ್ಣ-ಮಯ-ವಾದ
ಸುವುದು
ಸುಶರ್ಮನ
ಸುಶರ್ಮನು
ಸುಷೇಣನು
ಸುಸ್ತಾನಾಂ
ಸೂಕ್ತ
ಸೂಕ್ಷ್ಮ-ವಾಗಿ
ಸೂಚನೆ
ಸೂಚನೆ-ಯಂತೆ
ಸೂಚಿ-ಸು-ವುದು
ಸೂತ-ಪುತ್ರ-ನಾದ
ಸೂತ್ರ-ಗಳ
ಸೂತ್ರ-ಗಳಿಗೆ
ಸೂರ್ಯ
ಸೂರ್ಯ-ಕಿರಣ-ಗಳಿಗೆ
ಸೂರ್ಯಗ್ರಹಣ-ಕಾಲ-ದಲ್ಲಿ
ಸೂರ್ಯ-ಜಯುಜಾ
ಸೂರ್ಯ-ತನುಜೇ
ಸೂರ್ಯನ
ಸೂರ್ಯ-ನಿಂದ
ಸೂರ್ಯನು
ಸೂರ್ಯ-ವಂಶಜ-ನಾದ
ಸೂರ್ಯಾಂತರ್ಯಾಮಿ-ಯಾದ
ಸೂರ್ಯೋದಯಾ
ಸೃಗಾಲ-ವಾಸುದೇವನ
ಸೃಜತಿ
ಸೃಜಿಸಿ
ಸೃಷ್ಟಾ
ಸೃಷ್ಟಾದಿ
ಸೃಷ್ಟಿ
ಸೃಷ್ಟಿ-ಗಳನ್ನು
ಸೃಷ್ಟಿಗೆ
ಸೃಷ್ಟಿಸಿ
ಸೃಷ್ಟಿ-ಸು-ವುದು
ಸೆರೆ-ಯಿಂದ
ಸೆಳೆಯು
ಸೇಡು
ಸೇತುಮವಾಪ್ಯ
ಸೇತುವೆ-ಯನ್ನು
ಸೇನಾಂ
ಸೇನಾಧಿಪತಿ
ಸೇನಾಧಿಪತಿ-ಯನ್ನಾಗಿ
ಸೇನಾಧಿಪತ್ಯ
ಸೇನಾನಿ-ಯನ್ನೂ
ಸೇನಾಪತಿತ್ವ-ವನ್ನು
ಸೇನಾಪತಿಯಾಗು-ವುದು
ಸೇನೆ-ಯನ್ನು
ಸೇರಿ
ಸೇರಿ-ದಂತೆ
ಸೇರಿ-ದುದು
ಸೇರಿಸಿ
ಸೇರಿ-ಸಿ-ಕೊಳ್ಳು-ವುದು
ಸೇರು-ವುದು
ಸೇವಿಸಿ-ಕೊಳ್ಳಲ್ಪಟ್ಟ
ಸೇವಿ-ಸು-ವುದು
ಸೇವೆ
ಸೇವೆಗೆ
ಸೇವೆ-ಯನ್ನು
ಸೇವೆ-ಯಿಂದ
ಸೈಂಧವನ
ಸೈಂಧವ-ನನ್ನು
ಸೈಂಧವಮ್
ಸೈನಿಕರ
ಸೈನಿಕ-ರನ್ನು
ಸೈನ್ಯ-ಗಳನ್ನು
ಸೈನ್ಯ-ಗಳನ್ನೂ
ಸೈನ್ಯ-ಗಳು
ಸೈನ್ಯ-ಗಳೊಂದಿಗೆ
ಸೈನ್ಯದ
ಸೈನ್ಯ-ದಲ್ಲಿ
ಸೈನ್ಯ-ದಿಂದ
ಸೈನ್ಯ-ದಿಂದಲೂ
ಸೈನ್ಯದೊಡನೆ
ಸೈನ್ಯ-ವನ್ನು
ಸೈನ್ಯ-ವನ್ನೂ
ಸೈನ್ಯವೆಲ್ಲವೂ
ಸೈನ್ಯೈಃ
ಸೈರ್ಯಧೈರ್ಯ-ಗಳನ್ನು
ಸೊಸೆ-ಯರು
ಸೋಲಿ-ಸಲು
ಸೋಲಿಸಿ
ಸೋಲಿಸಿ-ದರೋ
ಸೋಲಿಸಿ-ದುದು
ಸೋಲಿ-ಸು-ವುದು
ಸೋಲು-ವುದು
ಸೋಽಸ್ಮಾನ್
ಸೌಂದರ್ಯರಾಶಿ-ಯನ್ನು
ಸೌಖ್ಯ
ಸೌಹಾರ್ದ
ಸ್ಕೃತ್ವಾ
ಸ್ಟೋಕ್ತ್ಯೈ
ಸ್ತುತಿ
ಸ್ತುತಿ-ಸಲ್ಪಟ್ಟ
ಸ್ತುತಿಸಿ
ಸ್ತುತಿಸು
ಸ್ತುತಿ-ಸುತ್ತಾ
ಸ್ತುತಿ-ಸುತ್ತೇನೆ
ಸ್ತುತಿ-ಸು-ವುದು
ಸ್ತೋತ್ರ
ಸ್ತೋತ್ರ-ಮಾಡಿ-ಸಿ-ಕೊಳ್ಳಲ್ಪಟ್ಟ
ಸ್ತೋತ್ರ-ಮಾಡುವುದು
ಸ್ತ್ರ
ಸ್ತ್ರೀ
ಸ್ತ್ರೀತ್ವ-ದಿಂದ
ಸ್ತ್ರೀಬೋಗಾದಿ-ಗಳನ್ನು
ಸ್ತ್ರೀಯನ್ನಾಗಿ
ಸ್ತ್ರೀಯರ
ಸ್ತ್ರೀಯ-ರನ್ನು
ಸ್ತ್ರೀಯ-ರನ್ನೂ
ಸ್ತ್ರೀಯರಿಗೂ
ಸ್ತ್ರೀಯ-ರಿಗೆ
ಸ್ತ್ರೀಯ-ರಿಗೆಲ್ಲಾ
ಸ್ತ್ರೀಯರು
ಸ್ತ್ರೀಯ-ರೊಡನೆ
ಸ್ತ್ರೀರೂಪಕಂ
ಸ್ತ್ರೀಸಂಗ-ವನ್ನು
ಸ್ಥಳ-ದಲ್ಲಿ
ಸ್ಥಳ-ವನ್ನು
ಸ್ಥಾನಕ್ಕೆ
ಸ್ಥಾನ-ಗಳನ್ನು
ಸ್ಥಾನ-ಗಳಿಗೆ
ಸ್ಥಾನ-ದಲ್ಲಿ
ಸ್ಥಾನ-ವಾದ
ಸ್ಥಾಪಿಸಿ
ಸ್ಥಾಪಿ-ಸಿದ
ಸ್ಥಾಪಿ-ಸು-ವುದು
ಸ್ಥಾವರ
ಸ್ಥಿತಮ್
ಸ್ಥಿತಿ-ಯಲ್ಲಿ-ರು-ವುದು
ಸ್ಥಿರ-ಪಡಿ-ಸು-ವುದು
ಸ್ನಾನ
ಸ್ನೇಹ
ಸ್ನೇಹ-ವನ್ನು
ಸ್ಪರ್ಶ-ದಿಂದ
ಸ್ಪರ್ಶ-ಮಾಡಲು
ಸ್ಪಷ್ಟ-ಪಡಿ-ಸು-ವುದು
ಸ್ಮರಂತೋ
ಸ್ಮರಣೆ-ಯಿಂದಲೂ
ಸ್ಮರಿಸಲ್ಪಡ-ತಕ್ಕ
ಸ್ಮರಿ-ಸು-ವುದು
ಸ್ಮೃತಃ
ಸ್ಯಮಂತಕಮಣಿ
ಸ್ರಷ್ಟಾ
ಸ್ವಂ
ಸ್ವಕೀಯಾಮುಗತ-ಹರಿಣಾ
ಸ್ವಗತಭೇದ
ಸ್ವಗುರವೇ
ಸ್ವಜನನೀಮಾನ್ಯಸ್ತೃಣಾ-ವರ್ತಹಾ
ಸ್ವತಂತ್ರ
ಸ್ವಪ್ನ-ದಲ್ಲಿ
ಸ್ವಭಾವ
ಸ್ವಮ್
ಸ್ವಯಂ
ಸ್ವಯಂವರಕ್ಕೆ
ಸ್ವಯಂವರದ
ಸ್ವರಾಜ್ಯ
ಸ್ವರಾಜ್ಯೇ
ಸ್ವರೂಪ
ಸ್ವರೂಪ-ಳಾದ
ಸ್ವರೂಪ-ವನ್ನು
ಸ್ವರೂಪ-ವುಳ್ಳ
ಸ್ವರ್ಗ-ದಲ್ಲಿ-ರುವ
ಸ್ವರ್ಗ-ಲೋಕದ
ಸ್ವರ್ಗ-ಲೋಕ-ದಂತೆ
ಸ್ವರ್ಗ-ಲೋಕ-ದಲ್ಲಿ
ಸ್ವರ್ಗ-ಲೋಕ-ದಿಂದ
ಸ್ವರ್ಗ-ಲೋಕಾದಿ-ಗಳನ್ನು
ಸ್ವರ್ಗಾದಿ
ಸ್ವರ್ಗಾಧಿಪಾಗ್ರ್ಯಾಸನೇ
ಸ್ವಲೋಕಂ
ಸ್ವಲ್ಪ-ವನ್ನು
ಸ್ವಸಹಜಾಂ
ಸ್ವಾಂ
ಸ್ವಾಂತಿಕಮಾ-ಗತಂ
ಸ್ವಾಗತಿಸಿ
ಸ್ವಾತಂತ್ರ್ಯ
ಸ್ವಾತಂತ್ರ್ಯದ
ಸ್ವಾಮ್
ಸ್ವೀಕರಿಸದೆ
ಸ್ವೀಕರಿಸಿ
ಸ್ವೀಕರಿ-ಸಿದ
ಸ್ವೀಕರಿಸಿ-ದುದು
ಸ್ವೀಕರಿ-ಸು-ವುದು
ಸ್ವೀಕಾರ
ಸ್ವೀಟ್
ಸ್ವೀಯಸೃಗಾಲಮಾತ್ಮ-ನಗರೀಂ
ಹಂಚಿ-ದುದು
ಹಂತ-ಕನು
ಹಂಸಡಿಭಿಕರ
ಹಂಸಡಿ-ಭೀಕರ
ಹಗ್ಗ-ದಿಂದ
ಹಚ್ಚು-ವುದು
ಹತಃ
ಹತನಾಗು-ವಂತೆ
ಹತರಾಗು
ಹತ-ರಾದ
ಹತ-ವಾದುವು
ಹತೇ
ಹತೇ-ಽರ್ಕಜೇ
ಹತ್ತು
ಹತ್ವಾ
ಹತ್ವೇಂದ್ರಾರಿ-ಕುಂಭ-ಕರ್ಣ-ದಶಕಾದೀನ್
ಹದಿನಾರು
ಹದಿ-ನಾಲ್ಕು
ಹದಿನೆಂಟು
ಹದಿನೈದು
ಹದಿ-ಮೂರು
ಹನನ
ಹನನ-ದಿಂದ
ಹನುಮಂತ
ಹನುಮಂತ-ದೇ-ವ-ರಿಂದ
ಹನುಮಂತ-ದೇ-ವರಿಗೆ
ಹನುಮಂತ-ದೇ-ವರು
ಹನುಮಂತನ
ಹನುಮಂತ-ನನ್ನು
ಹನುಮಂತ-ನಾಗಿ
ಹನುಮಂತ-ನಿಂದ
ಹನುಮಂತನು
ಹನುಮಂತ-ನೊಡನೆ
ಹನೂಮದ್ವರದ
ಹನ್ತ್ರಾ
ಹನ್ನೆರಡು
ಹನ್ನೆರಡು-ಮಂದಿ
ಹನ್ನೊಂದು
ಹರಡಿ-ದುದು
ಹರಡು-ವುದು
ಹರತ್ಯಂತತಃ
ಹರದನುರ್ಭಂಕ್ತ್ವಾಽವಹಜ್ಞಾನಕೀಂ
ಹರಿ
ಹರಿಃ
ಹರಿಗೇ
ಹರಿ-ಪುರೇ
ಹರಿ-ಭಕ್ತನು
ಹರಿ-ಭಕ್ತಿ-ಯಿಂದಲೇ
ಹರಿ-ಮ-ಹಿಮೆ-ಗಳನ್ನು
ಹರಿಯ
ಹರಿಯು
ಹರಿ-ವಾಯು-ಗಳ
ಹರಿಶ್ಚಂದ್ರನು
ಹರಿ-ಸರ್ವೋತ್ತಮತ್ವ
ಹರಿ-ಹಯಪುರುಷವ್ರಾತಬದ್ಧಾನ್ವಿಮೋಚ್ಯ
ಹರೇರ್ನಿಷೇವಣಪ-ರಾಮಲ್ಲಂ
ಹರ್ಷ-ವನ್ನುಂಟು-ಮಾಡಿದ
ಹಲ್ಲೆ
ಹಲ್ಲೆ-ಮಾಡಲು
ಹಸುಕರು-ಗಳನ್ನು
ಹಸ್ತಸ್ಪರ್ಶ-ದಿಂದ
ಹಸ್ತಾಂಗುಷ್ಠ-ವನ್ನು
ಹಸ್ತಿನಾ-ಪುರಕ್ಕೆ
ಹಸ್ತಿನಾಪುರದ
ಹಸ್ತಿನಾಪುರ-ದಲ್ಲಿ
ಹಸ್ತಿನಾವತೀ
ಹಸ್ತಿನಾವತೀ-ಪುರಕ್ಕೆ
ಹಾಕಿ
ಹಾಕು-ವುದು
ಹಾಗೂ
ಹಾರ-ವನ್ನು
ಹಾರಿ
ಹಾರಿ-ದುದು
ಹಾರಿಸಿ
ಹಾಲನ್ನು
ಹಾಲು
ಹಾಲೆಂದು
ಹಾಸ್ಯ
ಹಿಂತಿರುಗಿ
ಹಿಂತಿರುಗಿದ
ಹಿಂತಿರುಗಿ-ಸಲು
ಹಿಂತಿರುಗುವಾಗ
ಹಿಂತಿರುಗು-ವುದು
ಹಿಂದಿನ
ಹಿಂದಿರುಗು-ವುದು
ಹಿಂದೆ
ಹಿಂಬಾಲಿಸಿ
ಹಿಂಬಾಲಿಸಿ-ಕೊಳ್ಳಲ್ಪಟ್ಟ-ವ-ನಾಗಿ
ಹಿಂಸೆ-ಕೊಡಲು
ಹಿಟ್ಟನ್ನು
ಹಿಡಿಂಬನು
ಹಿಡಿಂಬಾ-ಸುರನ
ಹಿಡಿಂಬಿಯ
ಹಿಡಿಂಬಿ-ಯನ್ನು
ಹಿಡಿದು
ಹಿಡಿದು-ಕೊಂಡು
ಹಿಡಿದು-ಕೊಳ್ಳು-ವುದು
ಹಿಡಿ-ಯಲು
ಹಿಡಿಯು-ವುದು
ಹಿಡಿಸಿ
ಹಿತ
ಹಿತ-ನುಡಿ-ಗಳು
ಹಿತ-ವಚನ-ಗಳನ್ನು
ಹಿತೋಕ್ತಿ
ಹಿತೋಕ್ತಿ-ಗಳನ್ನು
ಹಿತೋಕ್ತಿ-ಯನ್ನು
ಹಿಮ
ಹೀಗೆ
ಹೀನಜಾತಿಯವ-ನಾದರೂ
ಹುಟ್ಟಲು
ಹುಟ್ಟಿ
ಹುಟ್ಟಿದ
ಹುಟ್ಟುವ
ಹುಟ್ಟು-ವಂತೆ
ಹುಟ್ಟು-ವುದು
ಹುಡುಕುವ
ಹುಡುಕುವ-ವನಂತೆ
ಹುಡುಕುವುದಕ್ಕಾಗಿ
ಹುರಿದುಂಬಿ-ಸು-ವುದು
ಹೂಂಕಾರ-ದಿಂದ
ಹೂತಿ-ಕೊಳ್ಳು-ವುದು
ಹೆಚ್ಚು
ಹೆಡೆ-ಯನ್ನು
ಹೆಡೆಯ-ಮೇಲೆ
ಹೆಣ್ಣು
ಹೆದೆಯೇರಿಸು-ನಲ್ಲಿ
ಹೆಬ್ಬಾ-ವಿನ
ಹೆಸರಿ-ನಿಂದ
ಹೆಸರು
ಹೇಗೋ
ಹೇಳದಿರಲು
ಹೇಳಲ್ಪಟ್ಟ
ಹೇಳಲ್ಪಟ್ಟಿವೆ
ಹೇಳಿ
ಹೇಳಿ-ಕೊಂಡು
ಹೇಳಿ-ರುತ್ತಾರೆ
ಹೇಳುತ್ತಾ
ಹೇಳುತ್ತಾರೆ
ಹೇಳು-ವುದು
ಹೊಂದಲು
ಹೊಂದಿ
ಹೊಂದಿದ
ಹೊಂದಿ-ದಂತೆ
ಹೊಂದಿ-ದರೋ
ಹೊಂದಿ-ರುವುದೇ
ಹೊಂದುತ್ತಿ-ರುವುದನ್ನು
ಹೊಂದುತ್ತೇನೆ
ಹೊಂದು-ವುದು
ಹೊಗಳು-ವುದು
ಹೊಟ್ಟೆ-ಯಲ್ಲಿ
ಹೊಡೆತ
ಹೊಡೆತ-ದಿಂದ
ಹೊಡೆತ-ದಿಂದಲೇ
ಹೊಡೆ-ದುದು
ಹೊಡೆದೋಡಿ-ಸು-ವುದು
ಹೊಡೆಯು-ವುದು
ಹೊಡೆ-ಸು-ವುದು
ಹೊತ್ತಾ-ದರೂ
ಹೊತ್ತು-ಕೊಂಡು
ಹೊರಕ್ಕೆ
ಹೊರಟ
ಹೊರಟು
ಹೊರಟು-ಹೋಗು-ವುದು
ಹೊರಡು-ವುದು
ಹೊರತಾಗಿ
ಹೊಸಮಲ್ಲ-ನನ್ನೂ
ಹೋಗಬೇಕೆಂದು
ಹೋಗ-ಬೇಡ-ವೆಂದು
ಹೋಗಲು
ಹೋಗಿ
ಹೋಗುತ್ತಿದ್ದ-ವ-ರನ್ನು
ಹೋಗುವ
ಹೋಗು-ವುದು
ಹೋದ
ಹೋದಾಗ
ಹೋರಾಟ
}
