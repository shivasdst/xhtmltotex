\sethyphenation{kannada}{
ಅ
ಅಂಕಗಳ
ಅಂಕ-ಗ-ಳನ್ನು
ಅಂಕ-ಗ-ಳನ್ನೇ
ಅಂಕ-ಗ-ಳಿ-ಗೇನೇ
ಅಂಕ-ಗ-ಳಿಸಿ
ಅಂಕಗಳು
ಅಂಕಿ
ಅಂಕಿತ
ಅಂಕಿ-ತಪ್ರ-ದಾನ
ಅಂಕು
ಅಂಕುಡೊಂಕು
ಅಂಕು-ಡೊಂಕು-ಗ-ಳನ್ನು
ಅಂಕು-ಡೊಂಕು-ಗಳು
ಅಂಕುರಿತ
ಅಂಕು-ರಿ-ಸಿತು
ಅಂಕೆ
ಅಂಗ
ಅಂಗಚೇಷ್ಟೆ
ಅಂಗಡಿ
ಅಂಗಡಿಯ
ಅಂಗ-ಡಿ-ಯನ್ನು
ಅಂಗ-ಡಿ-ಯಲ್ಲಿ
ಅಂಗ-ಡಿ-ಯಲ್ಲಿದ್ದ
ಅಂಗ-ಡಿ-ಯಾತ
ಅಂಗ-ಡಿ-ಯಿಂದ
ಅಂಗ-ರ-ಚ-ನಾ-ಶಾಸ್ತ್ರದ
ಅಂಗ-ಲಕ್ಷ-ಣ-ಗ-ಳನ್ನು
ಅಂಗಲಾಚಿ
ಅಂಗವನ್ನು
ಅಂಗ-ವಿ-ಕಲ
ಅಂಗ-ವಿ-ಕ-ಲತೆ
ಅಂಗಾಂಗ-ಗಳ
ಅಂಗಾಂಗ-ಗ-ಳಾಗಿ
ಅಂಗಾಂಗ-ಗ-ಳಿಗೆ
ಅಂಗಾಂಗ-ಗಳು
ಅಂಗಾ-ಭಿ-ನ-ಯಕ್ಕಾಗಿ
ಅಂಗೀ-ಕ-ರಿ-ಸಿದ್ದಾರೆ
ಅಂಗೀ-ಕ-ರಿ-ಸಿದ
ಅಂಗುತ್ತ-ರ-ನಿ-ಕಾ-ಯ-ದಲ್ಲಿ
ಅಂಗೈ
ಅಂಚನ್ನು
ಅಂಚಿ-ನಲ್ಲಿ-ರುವ
ಅಂಚೆ
ಅಂಜದೆ
ಅಂಜಿಕೆ
ಅಂಜಿಕೆಯ
ಅಂಜಿ-ಕೆ-ಯನ್ನುಂಟು
ಅಂಜಿ-ಕೊಳ್ಳುತ್ತಿದ್ದಾನೆ
ಅಂಜುತ್ತಾನೆ
ಅಂಜುವ
ಅಂಜೈನಾ
ಅಂಟಿ
ಅಂಟಿ-ಕೊಂಡಿ-ರುವ
ಅಂಟಿಕೊಂಡ
ಅಂಟಿ-ಕೊಂಡಿತ್ತು
ಅಂಟಿ-ಕೊಂಡಿ-ದೆಯೇ
ಅಂಟಿ-ಕೊಂಡಿದ್ದರೂ
ಅಂಟಿ-ಕೊಂಡಿದ್ದರೆ
ಅಂಟಿಕೊಂಡು
ಅಂಟಿಕೊಂಡೆ
ಅಂಟಿ-ಕೊಳ್ಳು-ವಂತೆ
ಅಂಟಿ-ಸುತ್ತಾರೆ
ಅಂಟು-ಜಾಡ್ಯದ
ಅಂಡ್
ಅಂತ
ಅಂತಸ್ಸಂಪರ್ಕಿತ
ಅಂತಃ
ಅಂತಃಕ-ರ-ಣ-ದಲ್ಲಿ
ಅಂತಃಕ-ರ-ಣ-ಪೂರ್ವ-ಕ-ವಾಗಿ
ಅಂತಃಪು-ರ-ದೊ-ಳ-ಗಿ-ರಲಿ
ಅಂತಃಶಕ್ತಿ-ಗಳ
ಅಂತಃಶಕ್ತಿಯ
ಅಂತಕನ
ಅಂತರ
ಅಂತ-ರಂಗದ
ಅಂತ-ರಂಗ-ದಲ್ಲಿ
ಅಂತ-ರಂಗ-ದಲ್ಲಿ-ರುವ
ಅಂತ-ರಂಗ-ವನ್ನು
ಅಂತ-ರ-ರಾಷ್ಟ್ರೀಯ
ಅಂತ-ರ-ವನ್ನು
ಅಂತರವೇ
ಅಂತರಾತ್ಮ
ಅಂತ-ರಾತ್ಮನೋ
ಅಂತ-ರಾ-ಧಾ-ರಿತ
ಅಂತ-ರಾರ್ಥವೂ
ಅಂತ-ರಾರ್ಥವೇ
ಅಂತ-ರಾ-ಳ-ದಲ್ಲಿ
ಅಂತ-ರಾ-ಳ-ವನ್ನು
ಅಂತ-ರಿಕ್ಷ-ದಲ್ಲಿ
ಅಂತರ್
ಅಂತರ್ಜ-ಗತ್ತಿಗೆ
ಅಂತರ್ಜ-ಗತ್ತಿಗೆ
ಅಂತರ್ಜ-ಗತ್ತಿ-ಗೊಂದು
ಅಂತರ್ಜ-ಗತ್ತಿನ
ಅಂತರ್ಜ್ಞಾ-ನ-ದಿಂದ
ಅಂತರ್ದಷ್ಟಿಯೂ
ಅಂತರ್ದೃಷ್ಟಿ
ಅಂತರ್ದೃಷ್ಟಿ-ಇವು
ಅಂತರ್ದೃಷ್ಟಿ-ಯನ್ನು
ಅಂತರ್ದೃಷ್ಟಿ-ಯಿಂದ
ಅಂತರ್ದ್ವಂದ್ವ
ಅಂತರ್ದ್ವಂದ್ವ-ವನ್ನು
ಅಂತರ್ಮ-ನಸ್ಸಿನ
ಅಂತರ್ಮ-ನಸ್ಸಿ-ನಲ್ಲಿ
ಅಂತರ್ಮು-ಖತೆ
ಅಂತರ್ಮು-ಖ-ತೆ-ಯಿಂದ
ಅಂತರ್ಮು-ಖಿ-ಯಾ-ಗದೆ
ಅಂತರ್ಮು-ಖಿ-ಯಾ-ಗಲಿ
ಅಂತರ್ಮು-ಖಿ-ಯಾ-ಗ-ಲೇ-ಬೇ-ಕಾ-ಗು-ವುದು
ಅಂತರ್ಮು-ಖಿ-ಯಾಗಿ
ಅಂತರ್ಮು-ಖಿ-ಯಾ-ದದ್ದು
ಅಂತರ್ಯಾ-ಮಿ-ಯಾಗಿ
ಅಂತಶ್ಶಕ್ತಿ-ಯನ್ನು
ಅಂತಸ್ತು
ಅಂತಸ್ಸಂಬಂಧಿತ
ಅಂತಹ
ಅಂತ-ಹ-ದಾ-ಗಿತ್ತು
ಅಂತಹುದೇ
ಅಂತಾ-ರಾಷ್ಟ್ರೀಯ
ಅಂತಿಂಥ
ಅಂತಿಮ
ಅಂತಿ-ಮ-ಯಾತ್ರೆ-ಯಲ್ಲಿ
ಅಂತಿ-ಮ-ವಲ್ಲ
ಅಂತೂ
ಅಂತೆ
ಅಂತೆಯೆ
ಅಂತೆಯೇ
ಅಂತೋನಿ
ಅಂತ್ಯ
ಅಂತ್ಯ-ಗ-ಳನ್ನು
ಅಂತ್ಯದಲ್ಲಿ
ಅಂತ್ಯ-ದ-ವ-ರೆಗೂ
ಅಂತ್ಯವಲ್ಲ
ಅಂತ್ಯ-ವಾ-ಗು-ವು-ದಿಲ್ಲ
ಅಂತ್ಯ-ವಿ-ರದು
ಅಂತ್ಯ-ವಿ-ರ-ಲಿಲ್ಲ
ಅಂತ್ಯವೇ
ಅಂಥ
ಅಂಥ-ದಕ್ಕೊಂದು
ಅಂಥದನ್ನು
ಅಂಥದು
ಅಂಥದೆ
ಅಂಥದೇ
ಅಂಥದೇನೂ
ಅಂಥದೊಂದು
ಅಂಥವನ
ಅಂಥವನು
ಅಂಥ-ವ-ನೊಬ್ಬ
ಅಂಥವರ
ಅಂಥ-ವ-ರನ್ನು
ಅಂಥ-ವ-ರಲ್ಲಿ
ಅಂಥ-ವ-ರಿಗೆ
ಅಂಥವರು
ಅಂಥವರೂ
ಅಂದಆಯ
ಅಂದಮೇಲೆ
ಅಂದರು
ಅಂದರೆ
ಅಂದಿನ
ಅಂದಿನಿಂದ
ಅಂದು
ಅಂಧ
ಅಂಧ-ಕಾ-ರ-ಮ-ಯ-ವಾಗಿ
ಅಂಧ-ಕಾ-ರ-ಮ-ಯ-ವಾ-ಯಿತು
ಅಂಧ-ಕಾ-ರವೇ
ಅಂಧ-ಮ-ಹಿಳೆ
ಅಂಧಶ್ರದ್ಧೆ
ಅಂಧಶ್ರದ್ಧೆ-ಯನ್ನೂ
ಅಂಧಾ-ನು-ಕ-ರ-ಣೆ-ಯಲ್ಲಿದೆ
ಅಂಧೇ-ರಿ-ಯಲ್ಲಿ
ಅಂಬೆ-ಗಾ-ಲಿ-ಡು-ವು-ದಕ್ಕೆ
ಅಂಬೇಡ್ಕರ್
ಅಂಬೋಣ
ಅಂಬೋ-ಣಕ್ಕ-ನು-ಗು-ಣ-ವಾಗಿ
ಅಂಬೋ-ಣ-ವನ್ನು
ಅಂಬೋ-ಣ-ವಿದೆ
ಅಂಶ
ಅಂಶ-ಗ-ಳನ್ನಾ-ಗಿಯೇ
ಅಂಶ-ಗ-ಳನ್ನು
ಅಂಶ-ಗ-ಳನ್ನೂ
ಅಂಶ-ಗ-ಳಲ್ಲಿ
ಅಂಶ-ಗ-ಳಿಂದ
ಅಂಶ-ಗ-ಳಿ-ಗಿಂತಲೂ
ಅಂಶಗಳು
ಅಂಶ-ಗ-ಳೆಂದೂ
ಅಂಶದ
ಅಂಶವನ್ನು
ಅಂಶವಿದೆ
ಅಂಶವೂ
ಅಂಶ-ವೆಂದರೆ
ಅಕಳಂಕ
ಅಕಸ್ಮಾತ್
ಅಕಸ್ಮಾತ್ತಾಗಿ
ಅಕಾಡೆಮಿ
ಅಕೇಡಮಿ
ಅಕ್ಕ
ಅಕ್ಕತಂಗಿ
ಅಕ್ಕ-ತಂಗಿ-ಯ-ರಿಗೂ
ಅಕ್ಕ-ತಂಗಿ-ಯರು
ಅಕ್ಕನಾಗಿ
ಅಕ್ಕ-ಮ-ಹಾ-ದೇವಿ
ಅಕ್ಕಸಾಲಿ
ಅಕ್ಕ-ಸಾ-ಲಿ-ಗ-ನಿಗೆ
ಅಕ್ಕಿ
ಅಕ್ಕಿಯನ್ನು
ಅಕ್ಟೋಬರ್
ಅಕ್ಟೋ-ಬರ್ನಲ್ಲಿ
ಅಕ್ರಮ
ಅಕ್ರಮದ
ಅಕ್ರೂ-ರ-ನಾದ
ಅಕ್ಷದಲ್ಲಿ
ಅಕ್ಷಯ
ಅಕ್ಷ-ಯ-ವಾ-ಗು-ವಂತೆ
ಅಕ್ಷರ
ಅಕ್ಷ-ರ-ಗ-ಳನ್ನು
ಅಕ್ಷ-ರ-ಗ-ಳಲ್ಲಿ
ಅಕ್ಷ-ರ-ಗಳು
ಅಕ್ಷ-ರ-ದಿಂದ
ಅಕ್ಷ-ರ-ಮಾ-ಲೆಯ
ಅಕ್ಷರಶಃ
ಅಕ್ಷ-ರಸ್ಥ-ನಾ-ಗಲಿ
ಅಕ್ಷ-ರಾಭ್ಯಾಸ
ಅಕ್ಷ-ವೆಂದರೆ
ಅಖಂಡ
ಅಖಂಡ-ವಾಗಿ
ಅಖಿ-ಲಾಂತ-ರಾತ್ಮಾ
ಅಖಿ-ಲಾಂತ-ರಾತ್ಮ-ನಾದ
ಅಗತ್ಯ
ಅಗತ್ಯಕ್ಕಾಗಿ
ಅಗತ್ಯ-ವಲ್ಲವೆ
ಅಗತ್ಯ-ವಾ-ಗ-ಬ-ಹುದು
ಅಗತ್ಯ-ವಾಗಿ
ಅಗತ್ಯ-ವಾ-ಗಿತ್ತು
ಅಗತ್ಯ-ವಾದ
ಅಗತ್ಯ-ವಾ-ದಂತೆ
ಅಗತ್ಯ-ವಾ-ದಾಗ
ಅಗತ್ಯ-ವಿತ್ತು
ಅಗತ್ಯ-ವಿಲ್ಲ
ಅಗತ್ಯ-ವಿಲ್ಲ-ವೆಂದೂ
ಅಗಮ್ಯ
ಅಗರ್ಭ
ಅಗಲ
ಅಗ-ಲ-ಗ-ಳನ್ನು
ಅಗಲದ
ಅಗ-ಲಿ-ಕೆಯ
ಅಗ-ಲು-ವಿ-ಕೆಯ
ಅಗಸ
ಅಗಸನ
ಅಗ-ಸ-ನನ್ನು
ಅಗ-ಸ-ನಾದ
ಅಗ-ಸ-ನೆಂದೇ
ಅಗಸ್ಟೋ
ಅಗಾಧ
ಅಗಾಧತೆ
ಅಗಾ-ಧ-ವಾದ
ಅಗಾ-ಧ-ಸಾ-ಗ-ರ-ಗಳೂ
ಅಗೋ
ಅಗೋಚರ
ಅಗೋ-ಚ-ರ-ನಾದ
ಅಗೋ-ಚ-ರ-ವಾ-ಗಿವೆ
ಅಗೋ-ಚ-ರ-ವಾದ
ಅಗೋ-ಚ-ರ-ವಾ-ದು-ದಲ್ಲವೆ
ಅಗೋ-ಚ-ರ-ವಾ-ದುದು
ಅಗೋ-ಚ-ರವೇ
ಅಗೋ-ಚ-ರ-ವೇನೂ
ಅಗೌರವ
ಅಗೌ-ರ-ವ-ವೆಂದೇ
ಅಗ್ಗದ
ಅಗ್ನಿ-ಕುಂಡ-ದಲ್ಲಿ
ಅಗ್ನಿಯಲ್ಲಿ
ಅಗ್ನಿಯೂ
ಅಗ್ನಿಯೇ
ಅಗ್ರ
ಅಗ್ರ-ಗಣ್ಯ-ರಾ-ದ-ವರು
ಅಗ್ರ-ದೂ-ತ-ರಾಗಿ
ಅಗ್ರಪಟ್ಟ
ಅಗ್ರಮಾನ್ಯ
ಅಗ್ರಸ್ಥಾ-ನ-ವನ್ನು
ಅಚಲ
ಅಚ-ಲ-ವಾದ
ಅಚ-ಲ-ವಿಶ್ವಾ-ಸ-ವಿ-ಡು-ವುದು
ಅಚ-ಲಶ್ರದ್ಧೆ-ಯಿಂದ
ಅಚಿಂತ್ಯ
ಅಚ್ಚ
ಅಚ್ಚರಿ
ಅಚ್ಚ-ರಿ-ಗೊಂಡ
ಅಚ್ಚ-ರಿ-ಗೊ-ಳಿ-ಸಿತು
ಅಚ್ಚ-ರಿ-ಗೊ-ಳಿ-ಸಿದ
ಅಚ್ಚ-ರಿ-ಗೊ-ಳಿ-ಸು-ವಷ್ಟು
ಅಚ್ಚ-ರಿ-ಪಟ್ಟಿತು
ಅಚ್ಚ-ರಿ-ಪ-ಡುತ್ತಿದ್ದ
ಅಚ್ಚ-ರಿ-ಪ-ಡುತ್ತಿದ್ದರು
ಅಚ್ಚ-ರಿ-ಯಾ-ಗ-ದಿ-ರದು
ಅಚ್ಚ-ರಿ-ಯಿಂದ
ಅಚ್ಚ-ಳಿ-ಯದೆ
ಅಚ್ಚಿನ
ಅಚ್ಚಿಸಿದ
ಅಚ್ಚು
ಅಚ್ಚು-ಕಟ್ಟಾಗಿ
ಅಚ್ಚುಕಟ್ಟು
ಅಚ್ಚು-ಕಟ್ಟು-ತನ
ಅಚ್ಚು-ಕೂ-ಟ-ವನ್ನು
ಅಚ್ಚೊತ್ತಿ
ಅಚ್ಯು-ತ-ವಾ-ದುದು
ಅಜ-ಗ-ಜಾಂತರ
ಅಜ-ಮೀ-ರ-ದಿಂದ
ಅಜಾ-ಗ-ರೂ-ಕ-ತೆ-ಯಿಂದ
ಅಜಾ-ಗ-ರೂ-ಕ-ರಾಗಿ
ಅಜೀರ್ಣ
ಅಜ್ಜ
ಅಜ್ಜನಿಗೆ
ಅಜ್ಜಿ
ಅಜ್ಞ
ಅಜ್ಞ-ನಲ್ಲೂ-ಒಬ್ಬನು
ಅಜ್ಞರೂ
ಅಜ್ಞಾತ
ಅಜ್ಞಾ-ತ-ವಾಗಿ
ಅಜ್ಞಾ-ತ-ವಾ-ಸ-ದಿಂದ
ಅಜ್ಞಾ-ತ-ಶಕ್ತಿ
ಅಜ್ಞಾನ
ಅಜ್ಞಾ-ನ-ಇವು
ಅಜ್ಞಾನಕ್ಕೆ
ಅಜ್ಞಾನದ
ಅಜ್ಞಾ-ನ-ದಲ್ಲಿ
ಅಜ್ಞಾ-ನ-ದಿಂದ
ಅಜ್ಞಾ-ನ-ದಿಂದುಂಟಾದ
ಅಜ್ಞಾ-ನ-ವನ್ನಾ-ಗಲಿ
ಅಜ್ಞಾ-ನ-ವ-ಶ-ನಾ-ಗಿದ್ದು-ಕೊಂಡು
ಅಜ್ಞಾನವು
ಅಜ್ಞಾನವೇ
ಅಜ್ಞಾ-ನ-ವೊಂದೇ
ಅಜ್ಞಾನಿ
ಅಜ್ಞಾ-ನಿ-ಗಳ
ಅಜ್ಞಾ-ನಿ-ಗ-ಳಿಗೆ
ಅಜ್ಞಾ-ನಿ-ಯಾ-ಗಿಯೇ
ಅಟೆನ್ಶನ್
ಅಟ್ಟದ
ಅಟ್ಟಹಾಸ
ಅಟ್ಟ-ಹಾ-ಸಕ್ಕೆ
ಅಟ್ಟಿ-ಸಿ-ಕೊಂಡು
ಅಟ್ಲಾಂಟಿಕ್
ಅಟ್ಲಾಂಟಿಸ್
ಅಟ್ಲಾಂಟಿಸ್ಗೆ
ಅಟ್ಲಾಂಟಿಸ್ನ
ಅಟ್ಲಾಂಟಿಸ್ಪ್ರ-ಳ-ಯಕ್ಕೆ
ಅಡ
ಅಡ-ಕ-ವಾ-ಗಿದೆ
ಅಡಕೆಯ
ಅಡಗಿ
ಅಡ-ಗಿ-ಕೊಂಡಿದೆ
ಅಡ-ಗಿ-ಕೊಂಡಿ-ದೆ-ಎಂಬುದು
ಅಡ-ಗಿ-ಕೊಂಡಿವೆ
ಅಡಗಿತ್ತು
ಅಡಗಿದ
ಅಡ-ಗಿ-ದಂತೆ
ಅಡಗಿದೆ
ಅಡ-ಗಿ-ದೆ-ಯಲ್ಲವೆ
ಅಡ-ಗಿ-ದೆ-ಯಲ್ಲವೇ
ಅಡ-ಗಿದ್ದರೂ
ಅಡ-ಗಿ-ರುತ್ತದೆ
ಅಡ-ಗಿ-ರುತ್ತವೆ
ಅಡ-ಗಿ-ರುವ
ಅಡ-ಗಿ-ರು-ವಂತೆ
ಅಡ-ಗಿ-ರು-ವನೊ
ಅಡ-ಗಿ-ರು-ವು-ದನ್ನು
ಅಡ-ಗಿ-ರು-ವುದು
ಅಡಗಿವೆ
ಅಡಗಿಸಿ
ಅಡ-ಗಿ-ಸಿ-ಡು-ವುದು
ಅಡ-ಗಿ-ಸಿ-ಡೋಣ
ಅಡಚಣೆ
ಅಡ-ವಿ-ಯಲ್ಲಿನ
ಅಡಿ
ಅಡಿ-ಗಲ್ಲಿನ
ಅಡಿಗಲ್ಲು
ಅಡಿಗಳ
ಅಡಿ-ಗ-ಳಷ್ಟು
ಅಡಿಗೆ
ಅಡಿ-ಗೆ-ಮ-ನೆಗೆ
ಅಡಿಗೆಯ
ಅಡಿ-ಟಿಪ್ಪ-ಣಿ-ಯನ್ನು
ಅಡಿ-ಟಿಪ್ಪ-ಣಿ-ಯಲ್ಲಿ
ಅಡಿಪಾಯ
ಅಡಿ-ಪಾ-ಯ-ಗ-ಳಿಲ್ಲದೆ
ಅಡಿ-ಪಾ-ಯದ
ಅಡಿ-ಪಾ-ಯ-ದಲ್ಲಿ-ರುವ
ಅಡಿ-ಪಾ-ಯ-ವಿಲ್ಲದೇ
ಅಡಿಬಿದ್ದು
ಅಡಿ-ಯಾ-ಳಾಗಿ
ಅಡಿಯಾಳು
ಅಡಿಯಿಟ್ಟೆ
ಅಡಿ-ಯಿ-ಡು-ವು-ದಿಲ್ಲ
ಅಡುಗೆ
ಅಡುಗೆಯ
ಅಡು-ಗೆ-ಯನ್ನು
ಅಡೆತಡೆ
ಅಡ್ಡ-ಗೋ-ಡೆ-ಯನ್ನು
ಅಡ್ಡ-ದಾ-ರಿ-ಗ-ಳಿಂದ
ಅಡ್ಡ-ದಾ-ರಿ-ಗ-ಳೆ-ಯುತ್ತಾರೆ
ಅಡ್ಡ-ದಾ-ರಿ-ಯನ್ನು
ಅಡ್ಡ-ಮೋ-ರೆಯ
ಅಡ್ಡಿ
ಅಡ್ಡಿ-ಯಾ-ಗವು
ಅಡ್ಡಿಯಾಗಿ
ಅಡ್ಡಿ-ಯಾ-ಗು-ವಂತೆ
ಅಡ್ಡಿ-ಯಾ-ಯಿತು
ಅಣಕ
ಅಣಿ-ಯಾ-ಗು-ವಾಗ
ಅಣು
ಅಣು-ಕ-ಣ-ಗಳ
ಅಣು-ಕ-ಣ-ವನ್ನೂ
ಅಣು-ಕ-ಣವೂ
ಅಣು-ಪ-ರ-ಮಾ-ಣು-ಗ-ಳಿಂದ
ಅಣು-ಪ-ರ-ಮಾ-ಣು-ರಾ-ಕೆಟ್ಟು-ಗಳ
ಅಣು-ಬಾಂಬಿನ
ಅಣುಬಾಂಬು
ಅಣು-ಭೌ-ತ-ವಿಜ್ಞಾನಿ
ಅಣುವಿನ
ಅಣು-ವಿ-ನಲ್ಲಿ
ಅಣುಶಕ್ತಿ
ಅಣು-ಶಕ್ತಿಯ
ಅಣ್ಣ
ಅಣ್ಣ-ತಮ್ಮಂದಿ-ರಾ-ಗಿದ್ದರು
ಅಣ್ಣ-ತಮ್ಮಂದಿರು
ಅಣ್ಣನ
ಅಣ್ವಸ್ತ್ರ-ಗಳ
ಅಣ್ವಸ್ತ್ರ-ಗ-ಳನ್ನು
ಅತಂತ್ರ-ವಾ-ಗುತ್ತಿದೆ
ಅತಿ
ಅತಿಕ್ರ-ಮಿಸಿ
ಅತಿಥಿ
ಅತಿ-ಥಿ-ಗ-ಳನ್ನು
ಅತಿ-ಥಿ-ಗ-ಳನ್ನೂ
ಅತಿ-ಥಿ-ಗ-ಳಾಗಿ
ಅತಿ-ಥಿ-ಗ-ಳಿಗೆ
ಅತಿ-ಥಿ-ಗಳು
ಅತಿ-ದೂ-ರ-ದಲ್ಲೂ
ಅತಿ-ಮಾ-ನವ
ಅತಿ-ಮಾ-ನ-ಸಕ್ಷೇತ್ರ-ದಲ್ಲಿ
ಅತಿ-ಮಾ-ನುಷ
ಅತಿಯಾಗಿ
ಅತಿಯಾದ
ಅತಿ-ರಂಜಿತ
ಅತಿ-ರ-ಭ-ಸ-ದಿಂದ
ಅತಿ-ರೇ-ಕಕ್ಕೆ
ಅತಿ-ರೇ-ಕಕ್ಕೆ-ಳ-ಸಿದೆ
ಅತಿ-ರೇ-ಕದ
ಅತಿ-ವಿ-ರಳ
ಅತಿ-ಶ-ಯೋಕ್ತಿ
ಅತಿ-ಶ-ಯೋಕ್ತಿಯ
ಅತಿ-ಶ-ಯೋಕ್ತಿ-ಯಲ್ಲ
ಅತಿಶ್ರೇಷ್ಠ
ಅತಿಸೂಕ್ಷ್ಮ
ಅತಿ-ಹೆಚ್ಚಿನ
ಅತೀ
ಅತೀಂ
ಅತೀಂದ್ರಿಯ
ಅತೀಂದ್ರಿ-ಯಾ-ನು-ಭವ
ಅತೀಂದ್ರಿ-ಯಾ-ನು-ಭ-ವಕ್ಕೆ
ಅತೀಂದ್ರಿ-ಯಾ-ನು-ಭ-ವ-ಗಳ
ಅತೀಂದ್ರಿ-ಯಾ-ನು-ಭ-ವ-ಗ-ಳನ್ನು
ಅತೀಂದ್ರಿ-ಯಾ-ನು-ಭ-ವ-ವನ್ನು
ಅತೀಂದ್ರಿ-ಯಾ-ನು-ಭ-ವಿ-ಗಳ
ಅತೀಂದ್ರಿ-ಯಾ-ನು-ಭ-ವಿ-ಗಳು
ಅತೀಂದ್ರಿ-ಯಾ-ನು-ಭೂ-ತಿ-ಯನ್ನು
ಅತೀತ
ಅತೀ-ತ-ನಾ-ಗಿದ್ದಾನೆ
ಅತೀ-ತ-ನಾ-ಗು-ವು-ದಕ್ಕಾಗಿ
ಅತೀತನೋ
ಅತೀ-ತ-ವಾಗಿ
ಅತೀ-ತ-ವಾ-ಗಿದೆ
ಅತೀ-ತ-ವಾದ
ಅತೀತವೋ
ಅತೀವ
ಅತೃಪ್ತ-ನಾ-ದ-ವ-ನಿ-ರಲಿ
ಅತೃಪ್ತಿ
ಅತೃಪ್ತಿ-ಗ-ಳನ್ನು
ಅತ್ತ
ಅತ್ತಿ-ಗೆ-ಯ-ವರ
ಅತ್ತಿತ್ತ
ಅತ್ತಿ-ಹಣ್ಣಿನ
ಅತ್ತು
ಅತ್ತೆಯ
ಅತ್ತೆ-ಯಾ-ದ-ವಳು
ಅತ್ತೆಯು
ಅತ್ತೇಬಿಟ್ಟ
ಅತ್ಯಂತ
ಅತ್ಯಗತ್ಯ
ಅತ್ಯದ್ಭುತ
ಅತ್ಯದ್ಭು-ತ-ವಾಗಿ
ಅತ್ಯಧಿಕ
ಅತ್ಯ-ಭಿ-ಮಾನ
ಅತ್ಯಾಚಾರ
ಅತ್ಯಾ-ಚಾ-ರಕ್ಕೊ-ಳ-ಗಾ-ಗುತ್ತಾನೆ
ಅತ್ಯಾ-ಚಾ-ರ-ಗ-ಳನ್ನು
ಅತ್ಯಾ-ತು-ರ-ದಿಂದ
ಅತ್ಯಾ-ಧು-ನಿಕ
ಅತ್ಯಾ-ವಶ್ಯಕ
ಅತ್ಯಾ-ವಶ್ಯ-ಕ-ವಾದ
ಅತ್ಯಾಶ್ಚರ್ಯ-ಕ-ರ-ವಾದ
ಅತ್ಯಾಶ್ಚರ್ಯ-ವಾ-ಯಿತು
ಅತ್ಯಾಶ್ಚರ್ಯ-ವೆಂಬಂತೆ
ಅತ್ಯಾ-ಸಕ್ತ-ನಾ-ಗದೆ
ಅತ್ಯಾ-ಸಕ್ತಿ-ಯನ್ನು
ಅತ್ಯುತ್ತಮ
ಅತ್ಯುತ್ತ-ಮ-ವಾಗಿ
ಅತ್ಯುನ್ನತ
ಅಥವಾ
ಅದ
ಅದಕ್ಕಂಟಿ-ಕೊಂಡ
ಅದಕ್ಕಂಟಿ-ಕೊಳ್ಳದೆ
ಅದಕ್ಕ-ನು-ಗು-ಣ-ವಾಗಿ
ಅದಕ್ಕ-ನು-ಗು-ಣ-ವಾ-ಗಿಯೇ
ಅದಕ್ಕ-ನು-ಗು-ಣ-ವಾದ
ಅದಕ್ಕಾಗಿ
ಅದಕ್ಕಾ-ಗಿಯೇ
ಅದಕ್ಕಾ-ದರೊ
ಅದಕ್ಕಿಂತ
ಅದಕ್ಕಿಂತಲೂ
ಅದಕ್ಕಿತ್ತರೆ
ಅದಕ್ಕಿದೆ
ಅದಕ್ಕಿಲ್ಲ
ಅದಕ್ಕೂ
ಅದಕ್ಕೆ
ಅದಕ್ಕೆಂದೇ
ಅದಕ್ಕೇನು
ಅದಕ್ಕೊಂದು
ಅದಕ್ಷತೆ
ಅದನ್ನ-ರಿ-ಯಲೂ
ಅದನ್ನ-ವನು
ಅದನ್ನ-ವರು
ಅದನ್ನೀಗ
ಅದನ್ನು
ಅದನ್ನೂ
ಅದನ್ನೆಲ್ಲಾ
ಅದನ್ನೇ
ಅದನ್ನೇಕೆ
ಅದಮ್ಯ
ಅದಮ್ಯ-ವಾದ
ಅದರ
ಅದರಂತೆ
ಅದರಡಿ
ಅದರರ್ಥ
ಅದ-ರಲ್ಲ-ಡ-ಗಿ-ರುವ
ಅದರಲ್ಲಿ
ಅದ-ರಲ್ಲಿದ್ದುವು
ಅದ-ರಲ್ಲಿಲ್ಲ
ಅದ-ರಲ್ಲುಂಟಾ-ಗಿ-ರುವ
ಅದರಲ್ಲೂ
ಅದರಲ್ಲೇ
ಅದ-ರಲ್ಲೇನೂ
ಅದರಿಂದ
ಅದ-ರಿಂದಲೇ
ಅದ-ರಿಂದಾಗಿ
ಅದ-ರಿಂದಾ-ಗುವ
ಅದ-ರಿಂದೆಂಥ
ಅದ-ರೆ-ಡೆಗೇ
ಅದ-ರೊಂದಿಗೇ
ಅದ-ರೊ-ಳಗೆ
ಅದಲ್ಲ
ಅದಾಗಲೇ
ಅದಾಗಿತ್ತು
ಅದಿರನ್ನು
ಅದಿ-ರಿ-ನಿಂದ
ಅದಿ-ರು-ವುದು
ಅದು
ಅದುಡಾ
ಅದು-ದ-ರಿಂದ
ಅದುಮಿ
ಅದುಮಿಟ್ಟು
ಅದು-ವ-ರೆಗೂ
ಅದೂ
ಅದೃಷ್ಟ
ಅದೃಷ್ಟಕ್ಕೆ
ಅದೃಷ್ಟದ
ಅದೃಷ್ಟ-ದಲ್ಲಿ
ಅದೃಷ್ಟ-ದಿಂದ
ಅದೃಷ್ಟ-ವನ್ನು
ಅದೃಷ್ಟ-ವನ್ನೇ
ಅದೃಷ್ಟ-ವಲ್ಲದೆ
ಅದೃಷ್ಟ-ವ-ಶಾತ್
ಅದೃಷ್ಟವೂ
ಅದೃಷ್ಟ-ವೆಂದರೆ
ಅದೃಷ್ಟ-ವೆನ್ನುತ್ತೇವೆ
ಅದೆಂಥ
ಅದೆಲ್ಲ
ಅದೆಲ್ಲ-ವನ್ನೂ
ಅದೆಷ್ಟು
ಅದೆಷ್ಟೋ
ಅದೇ
ಅದೇಕೆ
ಅದೇ-ಕೆನ್ಸಿಂಗ್ಟನ್
ಅದೇನನ್ನೂ
ಅದೇನು
ಅದೇನೂ
ಅದೇನೇ
ಅದೇನೋ
ಅದೇ-ಮ-ನಸ್ಸೊಂದು
ಅದೊಂದು
ಅದೊಂದೆ
ಅದೊಂದೇ
ಅದೊಂದೇ-ಎಂಬುದು
ಅದೋ
ಅದ್ಭುತ
ಅದ್ಭು-ತ-ಗಳ
ಅದ್ಭು-ತ-ಗ-ಳನ್ನು
ಅದ್ಭು-ತ-ಗ-ಳನ್ನೆಲ್ಲ
ಅದ್ಭು-ತ-ಗ-ಳನ್ನೇ
ಅದ್ಭು-ತ-ಗಳು
ಅದ್ಭು-ತ-ಗ-ಳೆಷ್ಟೋ
ಅದ್ಭುತದ
ಅದ್ಭು-ತ-ಪಾತ್ರ
ಅದ್ಭು-ತ-ಯಂತ್ರ-ವನ್ನೋ
ಅದ್ಭು-ತ-ಯಂತ್ರ-ವಾದ
ಅದ್ಭು-ತ-ವನ್ನು
ಅದ್ಭು-ತ-ವಾಗಿ
ಅದ್ಭು-ತ-ವಾದ
ಅದ್ಭು-ತ-ವಾ-ದರೆ
ಅದ್ಭುತವು
ಅದ್ಭುತವೂ
ಅದ್ಭು-ತ-ವೆ-ನಿ-ಸವು
ಅದ್ಭು-ತ-ವೆ-ನಿ-ಸುವ
ಅದ್ಭು-ತ-ಶಕ್ತಿಯ
ಅದ್ಭು-ತಾ-ನಂದರ
ಅದ್ಭು-ತಾ-ನಂದ-ರನ್ನು
ಅದ್ಭು-ತಾ-ನಂದ-ರಿಗೆ
ಅದ್ಭು-ತಾ-ನಂದರು
ಅದ್ಭು-ತಾ-ನಂದರೇ
ಅದ್ರೋ-ಹ-ದಿಂದ
ಅದ್ವಿತೀಯ
ಅದ್ವಿ-ತೀ-ಯ-ರೆ-ನಿ-ಸಿ-ಕೊಂಡ
ಅದ್ವಿ-ತೀ-ಯ-ವಾ-ಗಿತ್ತು
ಅದ್ವೈತ
ಅಧಃಪ-ತನ
ಅಧಃಪ-ತ-ನಕ್ಕೆ
ಅಧಃಪ-ತ-ನ-ಗ-ಳನ್ನು
ಅಧಃಪ-ತ-ನದ
ಅಧಃಪ-ತ-ನ-ದೆ-ಡೆಗೆ
ಅಧ-ಮ-ನಿಗೆ
ಅಧಮರು
ಅಧರ್ಮ
ಅಧರ್ಮದ
ಅಧರ್ಮಿ-ಗಳ
ಅಧಾರ್ಮಿ-ಕರೂ
ಅಧಿಕ
ಅಧಿ-ಕ-ವಾಗಿ
ಅಧಿ-ಕ-ವಾ-ಗು-ವುದು
ಅಧಿ-ಕ-ವಾ-ಯಿತು
ಅಧಿಕಾರ
ಅಧಿ-ಕಾ-ರ-ದಲ್ಲಿ-ರು-ವ-ವರೂ
ಅಧಿ-ಕಾ-ರಸ್ಥಾ-ನ-ದಲ್ಲಿ-ರು-ವ-ವರು
ಅಧಿ-ಕಾ-ರಕ್ಕೆ
ಅಧಿ-ಕಾ-ರ-ಗ-ಳಿ-ಸು-ವುದು
ಅಧಿ-ಕಾ-ರ-ಗಳು
ಅಧಿ-ಕಾ-ರದ
ಅಧಿ-ಕಾ-ರ-ಯುತ
ಅಧಿ-ಕಾ-ರ-ಯು-ತವೂ
ಅಧಿ-ಕಾ-ರ-ವನ್ನು
ಅಧಿ-ಕಾ-ರ-ವಾ-ಣಿಯ
ಅಧಿ-ಕಾ-ರ-ವಾ-ಣಿ-ಯಿಂದ
ಅಧಿ-ಕಾ-ರಾ-ರೂ-ಢರೂ
ಅಧಿಕಾರಿ
ಅಧಿ-ಕಾ-ರಿ-ಗಳ
ಅಧಿ-ಕಾ-ರಿ-ಗ-ಳಾ-ಗಿಯೋ
ಅಧಿ-ಕಾ-ರಿ-ಗ-ಳಿಗೂ
ಅಧಿ-ಕಾ-ರಿ-ಗ-ಳಿಗೆ
ಅಧಿ-ಕಾ-ರಿ-ಗಳು
ಅಧಿ-ಕಾ-ರಿ-ಯನ್ನು
ಅಧಿ-ಕಾ-ರಿ-ಯೊಬ್ಬರು
ಅಧಿಕೃತ
ಅಧಿ-ಕೃ-ತ-ವಾದ
ಅಧಿಷ್ಠಾ-ತೃವೋ
ಅಧೀ-ನ-ದಲ್ಲಿ-ರಿ-ಸಿ-ಕೊಳ್ಳಲು
ಅಧೀ-ನ-ಗೊ-ಳಿ-ಸಿ-ದುದೇ
ಅಧೀ-ನ-ದಲ್ಲಿ-ರಿ-ಸಿ-ಕೊಂಡಿದ್ದಳು
ಅಧೀ-ನ-ದಲ್ಲಿ-ರಿ-ಸಿ-ಕೊಳ್ಳುವ
ಅಧೀನರು
ಅಧೀ-ನ-ವಲ್ಲ
ಅಧೀ-ನ-ವಾ-ಗದ
ಅಧೀ-ನ-ವಾಗಿ
ಅಧೀ-ನ-ವಿ-ರ-ದು-ನಮ್ಮ
ಅಧೋಗತಿ
ಅಧೋ-ಗ-ತಿ-ಯಲ್ಲದೆ
ಅಧ್ಯಕ್ಷ
ಅಧ್ಯಕ್ಷ-ರಾದ
ಅಧ್ಯಕ್ಷರು
ಅಧ್ಯಕ್ಷರೂ
ಅಧ್ಯಯನ
ಅಧ್ಯ-ಯ-ನಕ್ಕಾಗಿ
ಅಧ್ಯ-ಯ-ನಕ್ಕೆ
ಅಧ್ಯ-ಯ-ನ-ಗಳ
ಅಧ್ಯ-ಯ-ನ-ಗ-ಳಾ-ಗಿ-ರದೆ
ಅಧ್ಯ-ಯ-ನ-ಗಳು
ಅಧ್ಯ-ಯ-ನದ
ಅಧ್ಯ-ಯ-ನ-ದಲ್ಲಿ
ಅಧ್ಯ-ಯ-ನ-ದಲ್ಲೂ
ಅಧ್ಯ-ಯ-ನ-ದಿಂದ
ಅಧ್ಯ-ಯ-ನ-ಮಾಡಿ
ಅಧ್ಯ-ಯ-ನ-ವನ್ನು
ಅಧ್ಯ-ಯ-ನವು
ಅಧ್ಯಾತ್ಮ
ಅಧ್ಯಾತ್ಮ-ಗಳ
ಅಧ್ಯಾತ್ಮ-ಗ-ಳಲ್ಲಿ
ಅಧ್ಯಾತ್ಮದ
ಅಧ್ಯಾತ್ಮ-ರಾಜ್ಯ-ವನ್ನು
ಅಧ್ಯಾತ್ಮ-ಲೋ-ಕ-ದಲ್ಲಿ
ಅಧ್ಯಾತ್ಮ-ವನ್ನು
ಅಧ್ಯಾತ್ಮ-ವಾ-ದಿಯೂ
ಅಧ್ಯಾತ್ಮವೇ
ಅಧ್ಯಾತ್ಮ-ಶೀಲ
ಅಧ್ಯಾತ್ಮಿಕ
ಅಧ್ಯಾಪಕ
ಅಧ್ಯಾ-ಪ-ಕ-ನಾ-ಗಿದ್ದ
ಅಧ್ಯಾ-ಪ-ಕ-ನಾ-ದ-ವನು
ಅಧ್ಯಾ-ಪ-ಕರ
ಅಧ್ಯಾ-ಪ-ಕ-ರನ್ನು
ಅಧ್ಯಾ-ಪ-ಕ-ರನ್ನೂ
ಅಧ್ಯಾ-ಪ-ಕ-ರಲ್ಲ
ಅಧ್ಯಾ-ಪ-ಕ-ರಲ್ಲಿ
ಅಧ್ಯಾ-ಪ-ಕ-ರಿಗೆ
ಅಧ್ಯಾ-ಪ-ಕರು
ಅಧ್ಯಾ-ಪ-ಕ-ರು-ಗಳ
ಅಧ್ಯಾ-ಪ-ಕರೂ
ಅಧ್ಯಾ-ಪ-ಕ-ರೆಂದು
ಅಧ್ಯಾ-ಪ-ಕ-ರೊಂದಿಗೆ
ಅಧ್ಯಾ-ಪ-ಕ-ರೊಬ್ಬರು
ಅಧ್ಯಾ-ಪ-ಕ-ವಿದ್ಯಾರ್ಥಿ-ಗ-ಳಲ್ಲಿ-ರಲಿ
ಅಧ್ಯಾಪಕಿ
ಅಧ್ಯಾ-ಪ-ಕಿಯ
ಅಧ್ಯಾ-ಪ-ಕಿ-ಯನ್ನು
ಅಧ್ಯಾ-ಪ-ಕಿ-ಯಾಗಿ
ಅಧ್ಯಾ-ಪ-ಕಿ-ಯಾ-ಗಿದ್ದ
ಅಧ್ಯಾ-ಪ-ಕಿ-ಯಾ-ಗಿದ್ದಷ್ಟು
ಅಧ್ಯಾ-ಪ-ಕಿ-ಯಾ-ಗಿದ್ದು
ಅಧ್ಯಾ-ಪ-ಕಿ-ಯಾದ
ಅಧ್ಯಾಯ
ಅಧ್ಯಾ-ಯ-ದಲ್ಲಿ
ಅಧ್ಯಾ-ಯ-ದಲ್ಲಿದೆ
ಅಧ್ಯಾ-ಯ-ವೊಂದ-ರಲ್ಲಿ
ಅಧ್ಯಾಯಕ್ಕೇ
ಅಧ್ಯಾ-ಯ-ವನ್ನು
ಅನಂತ
ಅನಂತ-ಗಳ
ಅನಂತ-ಗ-ಳನ್ನೂ
ಅನಂತ-ಗಳು
ಅನಂತ-ತೆ-ಯನ್ನೂ
ಅನಂತನೂ
ಅನಂತರ
ಅನಂತ-ರವೂ
ಅನಂತ-ರವೇ
ಅನಂತ-ರಾವ್
ಅನಂತ-ವಾದ
ಅನಂತ-ವಾ-ದುದು
ಅನಂತವೇ
ಅನಂತಾತ್ಮನ
ಅನಂತಾತ್ಮನೇ
ಅನ-ಗತ್ಯ-ವಾ-ದು-ದು-ಕಾ-ರಣ
ಅನತಿ
ಅನನ್ಯ
ಅನನ್ಯ-ವಾ-ದವು
ಅನರ್ಘ್ಯ-ರತ್ನ-ವನ್ನು
ಅನರ್ಥ-ಕಾರಿ
ಅನ-ವ-ರತ
ಅನಾ-ಗ-ರಿಕ
ಅನಾ-ಗ-ರಿ-ಕತೆ
ಅನಾ-ಗ-ರಿ-ಕ-ರೆಂದರೆ
ಅನಾ-ಗ-ರಿ-ಕ-ರೆಂದಲ್ಲ
ಅನಾ-ಗ-ರೀ-ಕ-ತೆ-ಯಿಂದ
ಅನಾ-ಚಾ-ರ-ಗಳ
ಅನಾ-ಚಾ-ರ-ಗ-ಳನ್ನೂ
ಅನಾಥ
ಅನಾ-ಥ-ನಾದ
ಅನಾಥರ
ಅನಾ-ಥ-ಶಿ-ಶು-ಗಳ
ಅನಾ-ಥಾ-ಲಯ
ಅನಾ-ಥಾ-ಲ-ಯ-ಗ-ಳಲ್ಲಿ
ಅನಾ-ಥಾಶ್ರಮ
ಅನಾ-ದ-ರಕ್ಕೆ
ಅನಾ-ದ-ರ-ಣೀಯ
ಅನಾ-ದ-ರ-ಣೆ-ಯಲ್ಲಿ
ಅನಾ-ದ-ರ-ವನ್ನು
ಅನಾ-ದಿ-ಕಾ-ಲ-ದಿಂದಲೂ
ಅನಾ-ನು-ಕೂ-ಲ-ವಾ-ಯಿತು
ಅನಾರೋಗ್ಯ
ಅನಾ-ರೋಗ್ಯಕ್ಕೂ
ಅನಾ-ರೋಗ್ಯ-ಗದ
ಅನಾ-ವಶ್ಯಕ
ಅನಾ-ಸಕ್ತರೂ
ಅನಾ-ಸಕ್ತಿ-ಯಿಂದ
ಅನಾಹತ
ಅನಾಹುತ
ಅನಾ-ಹು-ತಕ್ಕೆ
ಅನಾ-ಹು-ತ-ಗ-ಳಿಗೆ
ಅನಾ-ಹು-ತ-ಗಳು
ಅನಾ-ಹು-ತ-ದಿಂದಷ್ಟೇ
ಅನಾ-ಹು-ತ-ವಾ-ಗುವ
ಅನಾ-ಹು-ತವೇ
ಅನಿ-ವಾರ್ಯ-ವಾದ
ಅನಿತ್ಯ-ವನ್ನು
ಅನಿ-ಯಂತ್ರಿತ
ಅನಿ-ಯ-ಮಿತ
ಅನಿ-ರೀಕ್ಷಿತ
ಅನಿ-ರೀಕ್ಷಿ-ತ-ವಾಗಿ
ಅನಿರ್
ಅನಿರ್ದೇಶ್ಯ-ವಪುಃ
ಅನಿವಾರ್ಯ
ಅನಿ-ವಾರ್ಯ-ವಾ-ದಾಗ
ಅನಿ-ವಾರ್ಯ-ದಿಂದ
ಅನಿ-ವಾರ್ಯ-ವನ್ನು
ಅನಿ-ವಾರ್ಯ-ವಾ-ಗ-ಬ-ಹುದು
ಅನಿ-ವಾರ್ಯ-ವಾಗಿ
ಅನಿ-ವಾರ್ಯ-ವಾ-ಗುತ್ತದೆ
ಅನಿ-ವಾರ್ಯ-ವಾದ
ಅನಿ-ವಾರ್ಯ-ವಾ-ದು-ದನ್ನು
ಅನಿ-ವಾರ್ಯವೂ
ಅನಿಶ್ಚಿತ
ಅನಿಶ್ಚಿ-ತತೆ
ಅನಿಶ್ಚಿ-ತ-ತೆ-ಇವು
ಅನಿಶ್ಚಿ-ತ-ತೆ-ಯಿಂದ
ಅನಿ-ಸ-ಬ-ಹುದು
ಅನಿಸಿಕೆ
ಅನಿ-ಸಿ-ಕೆ-ಗಳ
ಅನಿ-ಸಿ-ಕೆ-ಗ-ಳನ್ನು
ಅನಿ-ಸಿ-ಕೆ-ಗ-ಳನ್ನೂ
ಅನಿ-ಸಿ-ಕೆ-ಗ-ಳಿಗೆ
ಅನಿ-ಸಿ-ಕೆ-ಗಳು
ಅನಿ-ಸಿ-ಕೆ-ಗ-ಳು-ಇ-ವೆಲ್ಲ-ವನ್ನೂ
ಅನಿ-ಸಿ-ಕೆ-ಗಳೇ
ಅನಿ-ಸಿ-ಕೆ-ಯನ್ನು
ಅನು
ಅನುಕಂಪ
ಅನು-ಕಂಪ-ದಿಂದ
ಅನು-ಕಂಪ-ವನ್ನು
ಅನು-ಕಂಪವೇ
ಅನುಕಂಪೆ
ಅನು-ಕಂಪೆ-ಗಳೇ
ಅನು-ಕಂಪೆಯ
ಅನು-ಕಂಪೆ-ಯಿಂದ
ಅನು-ಕಂಪೆ-ಯಿಂದೊ-ಡ-ಗೂಡಿ
ಅನು-ಕಂಪೆಯೇ
ಅನು-ಕ-ರ-ಣೀ-ಯ-ವಲ್ಲವೆ
ಅನು-ಕ-ರ-ಣೀ-ಯವೇ
ಅನು-ಕ-ರಣೆ
ಅನು-ಕ-ರ-ಣೆಗೆ
ಅನು-ಕ-ರ-ಣೆಯ
ಅನು-ಕ-ರ-ಣೆ-ಯಿಂದ
ಅನು-ಕ-ರ-ಣೆ-ಯಿಂದಲೇ
ಅನು-ಕ-ರ-ಣೆ-ಯಿಂದ-ವರು
ಅನು-ಕ-ರ-ಣೆಯೂ
ಅನು-ಕ-ರಿಸ
ಅನು-ಕ-ರಿ-ಸ-ಹೊ-ರ-ಟರೆ
ಅನು-ಕ-ರಿ-ಸಲು
ಅನು-ಕ-ರಿ-ಸುವ
ಅನು-ಕ-ರಿ-ಸು-ವಂಥದು
ಅನು-ಕ-ರಿ-ಸು-ವುದು
ಅನು-ಕೂ-ಲ-ಗ-ಳಿ-ರ-ಬೇ-ಕೆಂದು
ಅನು-ಕೂ-ಲ-ತೆ-ಗ-ಳಿ-ರುವ
ಅನು-ಕೂ-ಲ-ತೆ-ಗ-ಳುಳ್ಳ
ಅನು-ಕೂ-ಲ-ತೆ-ಗಳೂ
ಅನು-ಕೂ-ಲ-ತೆಯ
ಅನು-ಕೂ-ಲ-ರಾಗಿ
ಅನು-ಕೂ-ಲ-ವಾ-ಗಿದ್ದರೆ
ಅನು-ಕೂ-ಲ-ವಾ-ಗುವ
ಅನು-ಕೂ-ಲ-ವಾ-ಗು-ವಂತೆ
ಅನು-ಕೂ-ಲ-ವಾದ
ಅನು-ಕೂ-ಲ-ವಾ-ಯಿತೋ
ಅನು-ಕೂ-ಲ-ವಿದ್ದೂ
ಅನು-ಕೂ-ಲ-ವಿ-ರದೆ
ಅನು-ಕೂ-ಲ-ವಿ-ರುವ
ಅನು-ಕೂ-ಲ-ವಿಲ್ಲ
ಅನುಕ್ಷ-ಣವೂ
ಅನು-ಗು-ಣ-ವಾಗಿ
ಅನು-ಗು-ಣ-ವಾ-ಗಿಯೆ
ಅನು-ಗು-ಣ-ವಾ-ಗಿಯೇ
ಅನು-ಗು-ಣ-ವಾ-ಗಿ-ರುತ್ತವೆ
ಅನು-ಗು-ಣ-ವಾದ
ಅನು-ಗು-ಣ-ವಾ-ದದ್ದು
ಅನುಗ್ರಹ
ಅನುಗ್ರ-ಹ-ದಿಂದ
ಅನುಗ್ರ-ಹಿ-ಸಿದ್ದಕ್ಕೆ
ಅನುಗ್ರ-ಹಿ-ಸುತ್ತಿ
ಅನುದಿನ
ಅನುಧ್ಯಾನ
ಅನುಪಮ
ಅನು-ಪ-ಮ-ವಾ-ದುದು
ಅನು-ಪಸ್ಥಿ-ತಿ-ಯಲ್ಲಿ
ಅನುಭವ
ಅನು-ಭ-ವ-ಇ-ವನ್ನು
ಅನು-ಭ-ವಈ
ಅನು-ಭ-ವಕ್ಕೂ
ಅನು-ಭ-ವಕ್ಕೆ
ಅನು-ಭ-ವ-ಗಮ್ಯ-ವಾದ
ಅನು-ಭ-ವ-ಗಳ
ಅನು-ಭ-ವ-ಗ-ಳನ್ನು
ಅನು-ಭ-ವ-ಗ-ಳನ್ನೂ
ಅನು-ಭ-ವ-ಗ-ಳನ್ನೆಲ್ಲ
ಅನು-ಭ-ವ-ಗ-ಳನ್ನೇ
ಅನು-ಭ-ವ-ಗ-ಳಲ್ಲ
ಅನು-ಭ-ವ-ಗ-ಳಲ್ಲಿ
ಅನು-ಭ-ವ-ಗ-ಳಿಂದ
ಅನು-ಭ-ವ-ಗ-ಳಿಗೂ
ಅನು-ಭ-ವ-ಗ-ಳಿಗೆ
ಅನು-ಭ-ವ-ಗ-ಳಿವೆ
ಅನು-ಭ-ವ-ಗಳು
ಅನು-ಭ-ವ-ಗಳೂ
ಅನು-ಭ-ವ-ಗ-ಳೆಲ್ಲ
ಅನು-ಭ-ವ-ಗ-ಳೆಲ್ಲಾ
ಅನು-ಭ-ವ-ಗ-ಳೇ-ನಾ-ದರೂ
ಅನು-ಭ-ವ-ಗ-ಳೇ-ನಿ-ರ-ಬ-ಹುದು
ಅನು-ಭ-ವ-ಗ-ಳೊಂದಿಗೆ
ಅನು-ಭ-ವ-ಗ-ಳೊ-ಡನೆ
ಅನು-ಭ-ವದ
ಅನು-ಭ-ವ-ದಲ್ಲಿ
ಅನು-ಭ-ವ-ದಿಂದ
ಅನು-ಭ-ವ-ರ-ಹಿತ
ಅನು-ಭ-ವ-ವನ್ನಾ-ಧ-ರಿ-ಸಿಯೇ
ಅನು-ಭ-ವ-ವನ್ನು
ಅನು-ಭ-ವ-ವನ್ನೂ
ಅನು-ಭ-ವ-ವನ್ನೋ
ಅನು-ಭ-ವ-ವಾ-ಗ-ಬೇಕು
ಅನು-ಭ-ವ-ವಾ-ಗಿತ್ತು
ಅನು-ಭ-ವ-ವಾ-ಗಿದ್ದರೆ
ಅನು-ಭ-ವ-ವಾ-ಗುತ್ತ-ದೆಂಬು-ದ-ರಲ್ಲಿ
ಅನು-ಭ-ವ-ವಾ-ಗು-ವುದು
ಅನು-ಭ-ವ-ವಾಣಿ
ಅನು-ಭ-ವ-ವಾ-ಣಿ-ಯನ್ನು
ಅನು-ಭ-ವ-ವಾ-ದಾ-ಗ-ಅ-ವು-ಗಳ
ಅನು-ಭ-ವ-ವಾ-ಯಿತು
ಅನು-ಭ-ವ-ವಿದೆ
ಅನು-ಭ-ವ-ವಿ-ರುವ
ಅನು-ಭ-ವ-ವಿಲ್ಲದ
ಅನು-ಭ-ವವೂ
ಅನು-ಭ-ವವೇ
ಅನು-ಭ-ವ-ವೊಂದು
ಅನು-ಭ-ವ-ಸಿದ್ಧ
ಅನು-ಭ-ವಿ-ಗಳ
ಅನು-ಭ-ವಿ-ಗ-ಳಾ-ಗಿದ್ದರೂ
ಅನು-ಭ-ವಿ-ಗ-ಳಾದ
ಅನು-ಭ-ವಿ-ಗಳು
ಅನು-ಭ-ವಿ-ಗಳೂ
ಅನು-ಭ-ವಿಯ
ಅನು-ಭ-ವಿ-ಯೊಬ್ಬನ
ಅನು-ಭ-ವಿ-ಯೊಬ್ಬರ
ಅನು-ಭ-ವಿ-ಸ-ತೊ-ಡ-ಗಿತು
ಅನು-ಭ-ವಿ-ಸದ
ಅನು-ಭ-ವಿ-ಸ-ಬಲ್ಲಳು
ಅನು-ಭ-ವಿ-ಸ-ಬಲ್ಲವು
ಅನು-ಭ-ವಿ-ಸ-ಬ-ಹು-ದಲ್ಲವೇ
ಅನು-ಭ-ವಿ-ಸ-ಬ-ಹು-ದಾದ
ಅನು-ಭ-ವಿ-ಸ-ಬ-ಹುದು
ಅನು-ಭ-ವಿ-ಸ-ಬೇ-ಕಲ್ಲ
ಅನು-ಭ-ವಿ-ಸ-ಬೇ-ಕಾ-ಗುತ್ತ-ದಲ್ಲವೇ
ಅನು-ಭ-ವಿ-ಸ-ಬೇ-ಕಾ-ಗುತ್ತ-ದೆಂದು
ಅನು-ಭ-ವಿ-ಸ-ಬೇ-ಕಾ-ಗು-ವುದು
ಅನು-ಭ-ವಿ-ಸ-ಬೇ-ಕಾ-ಗು-ವು-ದು-ಎಂದು
ಅನು-ಭ-ವಿ-ಸ-ಬೇ-ಕಾ-ಗು-ವುದೊ
ಅನು-ಭ-ವಿ-ಸ-ಬೇ-ಕಾದ
ಅನು-ಭ-ವಿ-ಸ-ಬೇ-ಕಾ-ಯಿತು
ಅನು-ಭ-ವಿ-ಸ-ಬೇಕೊ
ಅನು-ಭ-ವಿ-ಸಲು
ಅನು-ಭ-ವಿ-ಸಲೇ
ಅನು-ಭ-ವಿ-ಸ-ಲೇ-ಬೇಕು
ಅನು-ಭ-ವಿಸಿ
ಅನು-ಭ-ವಿ-ಸಿದ
ಅನು-ಭ-ವಿ-ಸಿ-ದಂತೆ
ಅನು-ಭ-ವಿ-ಸಿ-ದರೂ
ಅನು-ಭ-ವಿ-ಸಿ-ದರೆ
ಅನು-ಭ-ವಿ-ಸಿದೆ
ಅನು-ಭ-ವಿ-ಸಿದ್ದ
ಅನು-ಭ-ವಿ-ಸಿಯೇ
ಅನು-ಭ-ವಿ-ಸಿಲ್ಲವೋ
ಅನು-ಭ-ವಿಸು
ಅನು-ಭ-ವಿ-ಸುತ್ತಲೇ
ಅನು-ಭ-ವಿ-ಸುತ್ತಾನೆ
ಅನು-ಭ-ವಿ-ಸುತ್ತಾರೆ
ಅನು-ಭ-ವಿ-ಸುತ್ತಾಳೆ
ಅನು-ಭ-ವಿ-ಸುತ್ತಿದೆ
ಅನು-ಭ-ವಿ-ಸುತ್ತಿದ್ದ
ಅನು-ಭ-ವಿ-ಸುತ್ತಿದ್ದರು
ಅನು-ಭ-ವಿ-ಸುತ್ತಿದ್ದಾನೆ
ಅನು-ಭ-ವಿ-ಸುತ್ತಿದ್ದೇನೆ
ಅನು-ಭ-ವಿ-ಸುತ್ತಿ-ರು-ವ-ವ-ನಂತೆ
ಅನು-ಭ-ವಿ-ಸುವ
ಅನು-ಭ-ವಿ-ಸು-ವ-ವ-ನಲ್ಲಿ
ಅನು-ಭ-ವಿ-ಸು-ವು-ದನ್ನು
ಅನು-ಭ-ವಿ-ಸು-ವು-ದಿಲ್ಲ
ಅನು-ಭ-ವಿ-ಸು-ವುದು
ಅನು-ಭ-ವಿ-ಸುವೆ
ಅನು-ಭಾ-ವದ
ಅನು-ಭಾ-ವ-ದಿಂದ
ಅನುಭಾವಿ
ಅನು-ಭಾ-ವಿ-ಗಳ
ಅನು-ಭಾ-ವಿ-ಗ-ಳನ್ನೂ
ಅನು-ಭಾ-ವಿ-ಗ-ಳಾ-ಗ-ಬೇ-ಕಿಲ್ಲ
ಅನು-ಭಾ-ವಿ-ಗ-ಳಾದ
ಅನು-ಭಾ-ವಿ-ಗಳು
ಅನುಭಾವೀ
ಅನುಭೂತಿ
ಅನು-ಭೂ-ತಿ-ಗಳು
ಅನು-ಭೂ-ತಿ-ಯನ್ನು
ಅನು-ಭೂ-ತಿ-ಯನ್ನೇ
ಅನು-ಭೂ-ತಿ-ಯಲ್ಲಿ
ಅನು-ಭೂ-ತಿ-ಯಿಂದ
ಅನು-ಭೂ-ತಿಯೇ
ಅನುಮತಿ
ಅನು-ಮ-ತಿ-ಯನ್ನು
ಅನುಮಾನ
ಅನು-ಮಾ-ನಕ್ಕೆ
ಅನು-ಯಾ-ಯಿ-ಗಳ
ಅನು-ಯಾ-ಯಿ-ಗ-ಳನ್ನೂ
ಅನು-ಯಾ-ಯಿ-ಗ-ಳಲ್ಲಿ
ಅನು-ಯಾ-ಯಿ-ಗ-ಳಾದ
ಅನು-ಯಾ-ಯಿ-ಗ-ಳಿಗೆ
ಅನು-ಯಾ-ಯಿ-ಗಳು
ಅನು-ಯಾ-ಯಿ-ಗಳೇ
ಅನು-ರಕ್ತ-ರಾ-ದರೆ
ಅನು-ರಕ್ತಿ-ಯನ್ನೂ
ಅನು-ರ-ಣಿ-ತ-ವಾ-ಗಿತ್ತು
ಅನು-ರ-ಣಿ-ತ-ವಾ-ಗುವ
ಅನು-ರಾ-ಗಕ್ಕೆ
ಅನು-ರಾ-ಗದ
ಅನು-ರಾ-ಗ-ದಲ್ಲಿ
ಅನು-ವಂಶೀ-ಯ-ತೆ-ಯದೇ
ಅನು-ವಂಶೀ-ಯ-ತೆ-ಯನ್ನೂ
ಅನುವಾದ
ಅನು-ವಾ-ದಿತ
ಅನುವು
ಅನು-ವು-ಮಾ-ಡಿ-ಕೊಟ್ಟಿದ್ದಾರೆ
ಅನು-ಶಾ-ಸ-ನ-ವನ್ನು
ಅನುಷ್ಠಾನ
ಅನುಷ್ಠಾ-ನಕ್ಕೆ
ಅನುಷ್ಠಾ-ನ-ಗಳು
ಅನುಷ್ಠಾ-ನದ
ಅನುಷ್ಠಾ-ನ-ದಲ್ಲಿ
ಅನುಷ್ಠಾ-ನ-ದಿಂದ
ಅನುಷ್ಠಾ-ನ-ಯೋಗ್ಯ-ವಾದ
ಅನುಷ್ಠಾ-ನಾತ್ಮಕ
ಅನು-ಸಂಧಾ-ನಕ್ಕೂ
ಅನು-ಸ-ರಿ-ಸ-ದ-ವರೇ
ಅನು-ಸ-ರಿ-ಸದೆ
ಅನು-ಸ-ರಿ-ಸ-ಬೇ-ಕಾ-ಗುತ್ತ-ದೆಂಬುದು
ಅನು-ಸ-ರಿ-ಸ-ಬೇ-ಕಾದ
ಅನು-ಸ-ರಿ-ಸ-ಬೇಕು
ಅನು-ಸ-ರಿ-ಸ-ಬೇ-ಕೆಂಬ
ಅನು-ಸ-ರಿ-ಸ-ಬೇ-ಕೆಂಬು-ದನ್ನು
ಅನು-ಸ-ರಿ-ಸಲು
ಅನು-ಸ-ರಿಸಿ
ಅನು-ಸ-ರಿ-ಸಿದ
ಅನು-ಸ-ರಿ-ಸಿ-ದರೆ
ಅನು-ಸ-ರಿ-ಸಿ-ದಲ್ಲಿ
ಅನು-ಸ-ರಿ-ಸಿ-ದ-ವ-ರೆಲ್ಲ-ರಿಗೂ
ಅನು-ಸ-ರಿ-ಸಿ-ದಾಗ
ಅನು-ಸ-ರಿ-ಸಿದೆ
ಅನು-ಸ-ರಿ-ಸುತ್ತ
ಅನು-ಸ-ರಿ-ಸುತ್ತದೆ
ಅನು-ಸ-ರಿ-ಸುತ್ತವೆ
ಅನು-ಸ-ರಿ-ಸುತ್ತಾರೆ
ಅನು-ಸ-ರಿ-ಸುತ್ತಾ-ರೆಂಬುದು
ಅನು-ಸ-ರಿ-ಸುತ್ತಿದ್ದ
ಅನು-ಸ-ರಿ-ಸುತ್ತಿದ್ದಾ-ರೆಯೇ
ಅನು-ಸ-ರಿ-ಸುತ್ತಿಲ್ಲ
ಅನು-ಸ-ರಿ-ಸುವ
ಅನು-ಸ-ರಿ-ಸು-ವಂತಾ-ಗ-ಬೇ-ಕಲ್ಲವೆ
ಅನು-ಸ-ರಿ-ಸು-ವು-ದಕ್ಕಲ್ಲ
ಅನು-ಸ-ರಿ-ಸು-ವುದು
ಅನೃತ
ಅನೇಕ
ಅನೇ-ಕ-ರಿ-ಗಿಂತ
ಅನೇ-ಕ-ರಿಗೆ
ಅನೇಕರು
ಅನೇ-ಕಾ-ನೇಕ
ಅನೈ-ಕಾಗ್ರತೆ
ಅನೈಕ್ಯಕ್ಕೆ
ಅನೈಕ್ಯ-ಗಳ
ಅನೈಚ್ಛಿಕ
ಅನೈತಿಕ
ಅನೈ-ತಿ-ಕತ
ಅನೈ-ತಿ-ಕತೆ
ಅನೈ-ತಿ-ಕ-ತೆಗೆ
ಅನೈ-ತಿ-ಕ-ತೆ-ಯತ್ತ
ಅನ್ನ
ಅನ್ನ-ಕೊಟ್ಟರೇ
ಅನ್ನದಲ್ಲಿ
ಅನ್ನವನ್ನೂ
ಅನ್ನ-ವಿಲ್ಲದೆ
ಅನ್ನಾಭಾವ
ಅನ್ನಾಹಾರ
ಅನ್ನಾ-ಹಾ-ರ-ಗ-ಳಿಂದ
ಅನ್ನಿ-ಸ-ಬ-ಹುದು
ಅನ್ನಿ-ಸಿದ್ದುಂಟು
ಅನ್ನು
ಅನ್ನುತ್ತೀಯಾ
ಅನ್ನುವಂತೆ
ಅನ್ನು-ವು-ದೇಕೆ
ಅನ್ನೂ
ಅನ್ಯ
ಅನ್ಯಕ್ಷೇತ್ರ-ದಲ್ಲಿ
ಅನ್ಯತ್ರ
ಅನ್ಯಥಾ
ಅನ್ಯ-ಧರ್ಮ-ಗ-ಳಿಗೆ
ಅನ್ಯಮತ
ಅನ್ಯ-ಮ-ನಸ್ಕತೆ
ಅನ್ಯ-ಮ-ನಸ್ಕ-ನಾಗಿ
ಅನ್ಯ-ಮ-ನಸ್ಕ-ಳಾಗಿ
ಅನ್ಯ-ಮಾರ್ಗ-ವಿಲ್ಲ
ಅನ್ಯರ
ಅನ್ಯಾಯ
ಅನ್ಯಾ-ಯ-ಇವು
ಅನ್ಯಾಯದ
ಅನ್ಯಾ-ಯ-ವನ್ನು
ಅನ್ಯೋನ್ಯ
ಅನ್ಯೋನ್ಯ-ವಾ-ಗಿ-ರುತ್ತಾರೆ
ಅನ್ವ-ಯ-ವಾ-ಗುತ್ತದೆ
ಅನ್ವ-ಯಿ-ಸ-ಬ-ಹುದು
ಅನ್ವ-ಯಿ-ಸ-ಬೇಕು
ಅನ್ವ-ಯಿ-ಸುವ
ಅನ್ವ-ಯಿ-ಸು-ವಂಥ
ಅನ್ವ-ಯಿ-ಸು-ವು-ದ-ರಿಂದ
ಅನ್ವರ್ಥ-ನಾ-ಮ-ವನ್ನೇ
ಅನ್ವೇ-ಷ-ಕ-ರಿಗೆ
ಅನ್ವೇಷಣಾ
ಅನ್ವೇ-ಷ-ಣಾ-ತಜ್ಞ
ಅನ್ವೇ-ಷ-ಣಾ-ತಜ್ಞರ
ಅನ್ವೇ-ಷ-ಣಾ-ತಜ್ಞರು
ಅನ್ವೇ-ಷ-ಣಾ-ಸಕ್ತಿ
ಅನ್ವೇಷಣೆ
ಅನ್ವೇ-ಷ-ಣೆ-ಗಳ
ಅನ್ವೇ-ಷ-ಣೆ-ಗ-ಳನ್ನು
ಅನ್ವೇ-ಷ-ಣೆ-ಗ-ಳಿಂದ
ಅನ್ವೇ-ಷ-ಣೆ-ಗ-ಳಿಗೆ
ಅನ್ವೇ-ಷ-ಣೆ-ಗಳು
ಅನ್ವೇ-ಷ-ಣೆ-ಗಳೂ
ಅನ್ವೇ-ಷ-ಣೆಗೆ
ಅನ್ವೇ-ಷ-ಣೆಯ
ಅನ್ವೇ-ಷ-ಣೆ-ಯನ್ನು
ಅನ್ವೇ-ಷ-ಣೆ-ಯಲ್ಲೂ
ಅನ್ವೇ-ಷ-ಣೆ-ಯಿಂದ
ಅನ್ವೇ-ಷ-ಣೆಯು
ಅನ್ವೇ-ಷ-ಣೆಯೇ
ಅನ್ವೇಷಿಸ
ಅನ್ವೇ-ಷಿ-ಸಲು
ಅನ್ವೇಷಿಸಿ
ಅಪ
ಅಪಕಾರ
ಅಪ-ಕಾ-ರದ
ಅಪಕೀರ್ತಿ
ಅಪಕ್ವ
ಅಪಕ್ವ-ವಾ-ಗಿದ್ದ
ಅಪಘಾತ
ಅಪ-ಘಾ-ತಕ್ಕೀ-ಡಾಗಿ
ಅಪ-ಘಾ-ತ-ದಲ್ಲಿ
ಅಪ-ಘಾ-ತ-ದಿಂದ
ಅಪ-ಘಾ-ತ-ವೊಂದ-ರಲ್ಲಿ
ಅಪ-ಜ-ಯದ
ಅಪ-ಜ-ಯ-ವಿಲ್ಲ
ಅಪ-ನಂಬಿಕೆ
ಅಪ-ನಂಬಿ-ಕೆ-ಗ-ಳಿಂದ
ಅಪ-ನಂಬಿ-ಕೆಯ
ಅಪ-ನಂಬಿ-ಕೆ-ಯನ್ನು
ಅಪ-ನಂಬಿ-ಕೆ-ಯಿಂದ
ಅಪಪ್ರ-ಚಾರ
ಅಪಮಾನ
ಅಪ-ಮಾ-ನಕ್ಕೀ-ಡಾ-ಗು-ವಂತಹ
ಅಪ-ಮಾ-ನಕ್ಕೆ
ಅಪ-ಮಾ-ನ-ಗಳು
ಅಪ-ಮಾ-ನದ
ಅಪ-ಮಾ-ನ-ದಿಂದ
ಅಪ-ಮಾ-ನ-ವನ್ನು
ಅಪ-ಮಾ-ನ-ಸೂ-ಚಕ
ಅಪ-ಮೃತ್ಯು-ವಿಗೆ
ಅಪರ
ಅಪರಾಧ
ಅಪ-ರಾ-ಧ-ವೆ-ನಿ-ಸಿ-ಬಿ-ಡುತ್ತಿತ್ತು
ಅಪ-ರಾ-ಧಕ್ಕ-ನು-ಗು-ಣ-ವಾಗಿ
ಅಪ-ರಾ-ಧಕ್ಕೆ
ಅಪ-ರಾ-ಧ-ಗ-ಳಲ್ಲಿ
ಅಪ-ರಾ-ಧ-ಗ-ಳಿ-ಗಾಗಿ
ಅಪ-ರಾ-ಧ-ಗಳು
ಅಪ-ರಾ-ಧ-ವಾ-ಗುತ್ತದೆ
ಅಪ-ರಾ-ಧಿ-ಗ-ಳನ್ನು
ಅಪರಾಹ್ನ
ಅಪ-ರಿಗ್ರಹ
ಅಪ-ರಿಗ್ರ-ಹ-ಗ-ಳೆಂಬ
ಅಪ-ರಿ-ಮಿ-ತ-ಶಕ್ತಿ-ಯನ್ನು
ಅಪ-ರಿ-ವರ್ತ-ನೀಯ
ಅಪ-ರಿ-ಹಾರ್ಯ
ಅಪ-ರಿ-ಹಾರ್ಯ-ಗ-ಳನ್ನು
ಅಪ-ರಿ-ಹಾರ್ಯವೆ
ಅಪರೂಪ
ಅಪ-ರೂ-ಪ-ವಾಗಿ
ಅಪವಾದ
ಅಪ-ವಾ-ದ-ಗಳ
ಅಪ-ವಾ-ದ-ಗ-ಳಿಗೆ
ಅಪ-ವಾ-ದ-ವಿ-ರ-ಬ-ಹುದು
ಅಪ-ವಾ-ದ-ವಿ-ರ-ಲಿಲ್ಲ
ಅಪವಿತ್ರ
ಅಪಸ್ವ-ರ-ದಲ್ಲಿ
ಅಪಸ್ವ-ರ-ವಿಲ್ಲ
ಅಪ-ಹ-ರಿ-ಸಿದ
ಅಪ-ಹ-ರಿ-ಸಿ-ದರು
ಅಪ-ಹ-ರಿ-ಸಿ-ದ-ವ-ರಂತೆ
ಅಪ-ಹ-ರಿ-ಸಿ-ಬಿಟ್ಟ
ಅಪಹಾಸ್ಯ
ಅಪ-ಹಾಸ್ಯಕ್ಕೆ
ಅಪ-ಹಾಸ್ಯಕ್ಕೊಂದು
ಅಪಾಯ
ಅಪಾ-ಯ-ಕಾ-ರಿಯೂ
ಅಪಾ-ಯ-ಕರ
ಅಪಾ-ಯ-ಕಾರಿ
ಅಪಾ-ಯ-ಕಾ-ರಿ-ಯಾಗಿ
ಅಪಾಯದ
ಅಪಾ-ಯ-ದಿಂದ
ಅಪಾ-ಯ-ವನ್ನು
ಅಪಾ-ಯ-ವಾ-ಗುತ್ತದೆ
ಅಪಾ-ಯ-ವಾದ
ಅಪಾಯವೂ
ಅಪಾಯವೇ
ಅಪಾರ
ಅಪಾ-ರ-ಮಟ್ಟದ
ಅಪಾ-ರ-ಮಟ್ಟ-ದಲ್ಲಿ
ಅಪಾ-ರ-ವಾದ
ಅಪಾ-ರ-ವಾ-ದುದು
ಅಪಾ-ರ-ಶಕ್ತಿ
ಅಪಾ-ರ-ಶಕ್ತಿ-ಯನ್ನು
ಅಪೂರ್ಣ
ಅಪೂರ್ಣ-ತೆ-ಯನ್ನು
ಅಪೂರ್ಣನೂ
ಅಪೂರ್ಣ-ನೆಂದೂ
ಅಪೂರ್ವ
ಅಪೂರ್ವ-ವಾದ
ಅಪೂರ್ವ-ವಾ-ದವು
ಅಪೂರ್ವ-ಶಕ್ತಿ
ಅಪೇಕ್ಷಿ-ಗಳೇ
ಅಪೇಕ್ಷಿ-ಸದ
ಅಪೇಕ್ಷಿ-ಸದೆ
ಅಪೇಕ್ಷಿ-ಸಿದ
ಅಪೇಕ್ಷಿ-ಸಿ-ದರು
ಅಪೇಕ್ಷಿ-ಸಿದ್ದೇ
ಅಪೇಕ್ಷಿ-ಸುವ
ಅಪೌ-ರು-ಷೇ-ಯವೂ
ಅಪ್ಪ
ಅಪ್ಪಟ
ಅಪ್ಪಣೆ
ಅಪ್ಪಳಿಸಿ
ಅಪ್ಪ-ಳಿ-ಸಿ-ದರು
ಅಪ್ಪ-ಳಿ-ಸಿದ್ದರ
ಅಪ್ಪಾ
ಅಪ್ಪಿಕೊಂಡ
ಅಪ್ರ
ಅಪ್ರತಿಮ
ಅಪ್ರತ್ಯಕ್ಷ-ವಾಗಿ
ಅಪ್ರ-ಯೋ-ಜಕ
ಅಪ್ರಾ-ಮಾ-ಣಿಕ
ಅಪ್ರಾ-ಮಾ-ಣಿ-ಕ-ತೆ-ಯಾ-ಗುತ್ತದೆ
ಅಬ್ರಹಾಂ
ಅಭಯ
ಅಭ-ಯ-ಕರ
ಅಭ-ಯ-ದಾ-ತ-ನಾದ
ಅಭ-ಯ-ವ-ಚ-ನ-ವೀ-ಯುತ್ತ
ಅಭ-ಯ-ವನ್ನಿತ್ತು
ಅಭಾವ
ಅಭಾ-ವಕ್ಕಿಂತಲೂ
ಅಭಾವಕ್ಕೆ
ಅಭಾವದ
ಅಭಾ-ವ-ದಿಂದ
ಅಭಾ-ವ-ವನ್ನು
ಅಭಾ-ವ-ವಿ-ರ-ಬ-ಹುದು
ಅಭಾವವೂ
ಅಭಾ-ವ-ವೆಂದೇ
ಅಭಾವವೇ
ಅಭಿ-ನಂದ-ನೆ-ಗ-ಳನ್ನು
ಅಭಿನಯ
ಅಭಿನೇತ್ರಿ
ಅಭಿಪ್ರಾಯ
ಅಭಿಪ್ರಾ-ಯ-ಪಟ್ಟಿದ್ದಾರೆ
ಅಭಿಪ್ರಾ-ಯ-ಗ-ಳನ್ನು
ಅಭಿಪ್ರಾ-ಯ-ಗ-ಳನ್ನೂ
ಅಭಿಪ್ರಾ-ಯ-ಗ-ಳನ್ನೊಪ್ಪ-ದ-ವರ
ಅಭಿಪ್ರಾ-ಯ-ಗ-ಳಿ-ಗಿಂತಲೂ
ಅಭಿಪ್ರಾ-ಯ-ಗ-ಳಿಲ್ಲ
ಅಭಿಪ್ರಾ-ಯ-ಗ-ಳೇನು
ಅಭಿಪ್ರಾ-ಯದ
ಅಭಿಪ್ರಾ-ಯ-ದಂತೆ
ಅಭಿಪ್ರಾ-ಯ-ದಲ್ಲಿ
ಅಭಿಪ್ರಾ-ಯ-ವನ್ನಾ-ಗಲಿ
ಅಭಿಪ್ರಾ-ಯ-ವನ್ನು
ಅಭಿಪ್ರಾ-ಯ-ವಾದ
ಅಭಿಪ್ರಾ-ಯ-ವಿದು
ಅಭಿಪ್ರಾ-ಯವೇ
ಅಭಿ-ಮಂತ್ರಿಸಿ
ಅಭಿಮತ
ಅಭಿಮಾನ
ಅಭಿ-ಮಾ-ನ-ಗ-ಳನ್ನಾ-ದರೂ
ಅಭಿ-ಮಾ-ನ-ಗಳು
ಅಭಿ-ಮಾ-ನದ
ಅಭಿ-ಮಾ-ನ-ದಿಂದ
ಅಭಿ-ಮಾ-ನ-ವನ್ನು
ಅಭಿ-ಮಾ-ನವು
ಅಭಿಮಾನಿ
ಅಭಿ-ಮಾ-ನಿ-ಗ-ಳಾ-ಗಿದ್ದರು
ಅಭಿ-ಮಾ-ನಿ-ಗಳೂ
ಅಭಿರುಚಿ
ಅಭಿ-ರು-ಚಿ-ಗ-ಳನ್ನೂ
ಅಭಿ-ರು-ಚಿಗೆ
ಅಭಿ-ರು-ಚಿಯ
ಅಭಿ-ರು-ಚಿ-ಯನ್ನು
ಅಭಿ-ರು-ಚಿ-ಯನ್ನುಂಟು-ಮಾ-ಡು-ವು-ದರ
ಅಭಿ-ರು-ಚಿ-ಯಾ-ಗಲಿ
ಅಭಿ-ರು-ಚಿ-ಯುಳ್ಳ-ವರು
ಅಭಿಲಾಷೆ
ಅಭಿ-ವರ್ಧ-ನೆಯ
ಅಭಿವೃದ್ಧಿ
ಅಭಿ-ವೃದ್ಧಿ-ಗಳು
ಅಭಿ-ವೃದ್ಧಿಗೂ
ಅಭಿ-ವೃದ್ಧಿಗೆ
ಅಭಿ-ವೃದ್ಧಿಯ
ಅಭಿ-ವೃದ್ಧಿ-ಯನ್ನು
ಅಭಿ-ವೃದ್ಧಿ-ಯಾ-ಗದು
ಅಭಿ-ವೃದ್ಧಿಯೂ
ಅಭಿ-ವೃದ್ಧಿ-ಯೆ-ಡೆಗೆ
ಅಭಿ-ವೃದ್ಧಿಯೇ
ಅಭಿ-ವೃದ್ಧಿ-ಶೀಲ
ಅಭಿವ್ಯಕ್ತ-ವಾ-ಗ-ದಿದ್ದರೂ
ಅಭಿವ್ಯಕ್ತ-ವಾ-ಗು-ವುದು
ಅಭಿವ್ಯಕ್ತಿ
ಅಭಿವ್ಯಕ್ತಿ-ಗ-ಳಷ್ಟೇ
ಅಭಿವ್ಯಕ್ತಿಗೆ
ಅಭಿವ್ಯಕ್ತಿ-ಗೊಳ್ಳುತ್ತದೆ
ಅಭಿವ್ಯಕ್ತಿಯ
ಅಭಿವ್ಯಕ್ತಿ-ಯನ್ನು
ಅಭಿವ್ಯಕ್ತಿ-ಯನ್ನೇ
ಅಭಿವ್ಯಕ್ತಿ-ಯಾ-ದರೆ
ಅಭಿವ್ಯಕ್ತಿ-ಯಿಂದಲೇ
ಅಭಿವ್ಯಕ್ತಿ-ಯುಂಟಾ-ದಾ-ಗಲೇ
ಅಭಿವ್ಯಕ್ತಿಯೂ
ಅಭಿವ್ಯಕ್ತಿಯೇ
ಅಭೀಪ್ಸೆ
ಅಭೀಪ್ಸೆ-ಗ-ಳಿಂದ
ಅಭೀಪ್ಸೆ-ಗಳೇ
ಅಭೀಪ್ಸೆಯ
ಅಭೂ-ತ-ಪೂರ್ವ
ಅಭೇದ್ಯ
ಅಭೇದ್ಯ-ವಾದ
ಅಭೌತಿಕ
ಅಭೌ-ತಿ-ಕ-ವಾದ
ಅಭ್ಯರ್ಥಿ
ಅಭ್ಯ-ಸಿ-ಸ-ದಿದ್ದರೂ
ಅಭ್ಯ-ಸಿ-ಸ-ಬೇಕು
ಅಭ್ಯ-ಸಿ-ಸಲು
ಅಭ್ಯಸಿಸಿ
ಅಭ್ಯ-ಸಿ-ಸಿದ
ಅಭ್ಯ-ಸಿ-ಸಿದ್ದರು
ಅಭ್ಯ-ಸಿ-ಸು-ವುದು
ಅಭ್ಯಾ-ಗ-ತ-ರನ್ನು
ಅಭ್ಯಾರೋಹ
ಅಭ್ಯಾಸ
ಅಭ್ಯಾ-ಸ-ವಾಗಿ
ಅಭ್ಯಾಸಕ್ಕೆ
ಅಭ್ಯಾ-ಸಕ್ಕೆ-ಳೆ-ಸೀತು
ಅಭ್ಯಾ-ಸ-ಗಳ
ಅಭ್ಯಾ-ಸ-ಗ-ಳನ್ನು
ಅಭ್ಯಾ-ಸ-ಗ-ಳಲ್ಲಿ
ಅಭ್ಯಾ-ಸ-ಗ-ಳಿಂದ
ಅಭ್ಯಾ-ಸ-ಗ-ಳಿ-ಗಿಂತಲೂ
ಅಭ್ಯಾ-ಸ-ಗಳು
ಅಭ್ಯಾ-ಸ-ಗಳೂ
ಅಭ್ಯಾ-ಸ-ಗಳೆ
ಅಭ್ಯಾ-ಸ-ಗಳೇ
ಅಭ್ಯಾ-ಸ-ತ-ರ-ಬೇತಿ
ಅಭ್ಯಾಸದ
ಅಭ್ಯಾ-ಸ-ದಿಂದ
ಅಭ್ಯಾ-ಸ-ದೊಂದಿಗೆ
ಅಭ್ಯಾ-ಸ-ಮಾಡಿ
ಅಭ್ಯಾ-ಸ-ವನ್ನು
ಅಭ್ಯಾ-ಸ-ವ-ಶ-ವಾ-ಗುತ್ತದೆ
ಅಭ್ಯಾ-ಸ-ವಾ-ಗಿ-ಬಿಟ್ಟಿದೆ
ಅಭ್ಯಾ-ಸ-ವಾ-ಗು-ವುದು
ಅಭ್ಯಾ-ಸ-ವಾ-ದು-ದನ್ನು
ಅಭ್ಯಾ-ಸ-ವಿದೆ
ಅಭ್ಯಾ-ಸ-ವಿ-ರು-ವ-ವ-ರನ್ನು
ಅಭ್ಯಾಸವು
ಅಭ್ಯಾಸವೂ
ಅಭ್ಯಾ-ಸ-ವೆಂದರೆ
ಅಭ್ಯಾ-ಸ-ವೇನೋ
ಅಭ್ಯಾ-ಸ-ಶೀ-ಲ-ರಾದ
ಅಭ್ಯಾ-ಸ-ಸಿದ್ಧಿ-ಗಾಗಿ
ಅಭ್ಯುದಯ
ಅಭ್ಯು-ದ-ಯ-ಇವು
ಅಭ್ಯು-ದ-ಯಕ್ಕಾಗಿ
ಅಭ್ಯು-ದ-ಯಕ್ಕೂ
ಅಭ್ಯು-ದ-ಯಕ್ಕೆ
ಅಭ್ಯು-ದ-ಯಕ್ಕೇನು
ಅಭ್ಯು-ದ-ಯ-ಗಳ
ಅಭ್ಯು-ದ-ಯದ
ಅಭ್ಯು-ದ-ಯ-ವನ್ನು
ಅಭ್ಯು-ದ-ಯ-ವಾ-ಗ-ಬೇಕು
ಅಭ್ಯು-ದ-ಯ-ವಾ-ಗ-ಬೇ-ಕೆಂಬ
ಅಭ್ಯು-ದ-ಯ-ವಿಲ್ಲ
ಅಭ್ಯು-ದ-ಯಾ-ಕಾಂಕ್ಷಿ-ಗಳು
ಅಮಂಗ-ಲಕ್ಕೆ
ಅಮಂಗ-ಳ-ಎಂದು
ಅಮರ
ಅಮ-ರಜ್ಯೋತಿ
ಅಮ-ರ-ತತ್ವ
ಅಮರತ್ವ
ಅಮ-ರ-ನಾದ
ಅಮ-ರ-ವಾದ
ಅಮಲನ್ನು
ಅಮಲಿನ
ಅಮ-ಲಿ-ನಲ್ಲಿ
ಅಮ-ಲಿ-ನಲ್ಲಿದ್ದ
ಅಮಲು
ಅಮಾನುಷ
ಅಮಾ-ವಾಸ್ಯೆಯ
ಅಮಿತ
ಅಮೂರ್ತ
ಅಮೂರ್ತ-ರೂ-ಪ-ದಲ್ಲಿ
ಅಮೂರ್ತ-ಭಾ-ವನೆ
ಅಮೂಲ್ಯ
ಅಮೂಲ್ಯ-ವಲ್ಲವೇ
ಅಮೂಲ್ಯ-ವಾ-ಗಿತ್ತು
ಅಮೂಲ್ಯ-ವಾ-ಗಿದೆ
ಅಮೂಲ್ಯ-ವಾದ
ಅಮೃತ
ಅಮೃತಂ
ಅಮೃ-ತ-ಕುಂಭ-ವಿ-ರುತ್ತ
ಅಮೃತತ್ವ
ಅಮೃತದ
ಅಮೃ-ತ-ರ-ಸ-ವನ್ನು
ಅಮೃ-ತ-ವನ್ನಾ-ಗಿ-ಸ-ಬ-ಹುದು
ಅಮೃ-ತ-ವನ್ನು
ಅಮೃ-ತ-ವನ್ನೂ
ಅಮೆ-ರಿ-ಕದ
ಅಮೆ-ರಿ-ಕ-ದಲ್ಲಿ
ಅಮೇರಿಕ
ಅಮೇ-ರಿ-ಕಕ್ಕೆ
ಅಮೇ-ರಿ-ಕದ
ಅಮೇ-ರಿ-ಕ-ದಲ್ಲಿ
ಅಮೇ-ರಿ-ಕ-ದಲ್ಲೇ
ಅಮೇ-ರಿ-ಕ-ದಿಂದ
ಅಮೇ-ರಿ-ಕನ್
ಅಮೇ-ರಿ-ಕನ್ನರ
ಅಮೇ-ರಿ-ಕನ್ನ-ರಿಗೆ
ಅಮೇರಿಕಾ
ಅಮೇ-ರಿ-ಕಾದ
ಅಮೇ-ರಿ-ಕಾ-ದಲ್ಲಿ
ಅಮೇ-ರಿ-ಕಾ-ದೇ-ಶದ
ಅಮೋಘ
ಅಮ್ಮ
ಅಮ್ಮಾ
ಅಯಮಾರಾ
ಅಯಸ್ಕಾಂತದ
ಅಯಸ್ಕಾಂತವು
ಅಯಾ-ಚಿ-ತ-ವಾಗಿ
ಅಯಾನ್
ಅಯುಕ್ತ-ವಾ-ಗಿಯೋ
ಅಯೋಗ್ಯ
ಅಯೋಗ್ಯನೇ
ಅಯೋಗ್ಯ-ರಾ-ಗ-ಬಲ್ಲೆವು
ಅಯ್ಯಂಗಾ-ರರು
ಅಯ್ಯಾ
ಅಯ್ಯೊ
ಅಯ್ಯೋ
ಅರ-ಗಿ-ಸಿ-ಕೊಂಡ-ವಾಗ್ಬಾ-ಣ-ಗ-ಳಲ್ಲಿ
ಅರ-ಗಿ-ಸಿ-ಕೊಂಡಿ-ರುತ್ತಾರೆ
ಅರ-ಗಿ-ಸಿ-ಕೊಂಡು
ಅರ-ಗಿ-ಸಿ-ಕೊಳ್ಳದೆ
ಅರ-ಗಿ-ಸಿ-ಕೊಳ್ಳಿ
ಅರಚುತ್ತ
ಅರಣ್ಯ
ಅರಣ್ಯಕ್ಕೆ
ಅರಣ್ಯ-ಗ-ಳಲ್ಲಿ
ಅರಣ್ಯದ
ಅರಣ್ಯ-ದಲ್ಲಿ
ಅರಣ್ಯ-ದಿಂದ
ಅರಣ್ಯ-ರೋ-ದ-ನ-ವಾ-ಗು-ವುದು
ಅರಬರ
ಅರ-ಬ-ರಿಗೆ
ಅರಮನೆ
ಅರ-ಮ-ನೆಗೆ
ಅರ-ಮ-ನೆಯ
ಅರ-ಮ-ನೆ-ಯಲ್ಲಿ
ಅರಳಲಿ
ಅರಳಲು
ಅರಳಿ
ಅರಳಿತು
ಅರಳಿದ
ಅರ-ಳಿ-ದಂತೆ
ಅರಳಿಸಿ
ಅರ-ಳಿ-ಸಿತು
ಅರ-ಳಿ-ಸುವ
ಅರ-ಳುತ್ತದೆ
ಅರ-ಳು-ವಂತೆ
ಅರ-ಳು-ವಂಥ
ಅರ-ವತ್ತ-ಮೂರು
ಅರವತ್ತು
ಅರ-ವಿಂದ-ರಾ-ಗಲಿ
ಅರ-ವಿಂದರು
ಅರ-ಸ-ನಿಂದ
ಅರ-ಸ-ನಿಗೆ
ಅರ-ಸಿ-ದಾಗ
ಅರ-ಸು-ಪುತ್ರ
ಅರಿತ
ಅರಿತರೂ
ಅರಿತಾಗ
ಅರಿ-ತಿದ್ದಾ-ನೆಯೇ
ಅರಿ-ತಿದ್ದೆನು
ಅರಿ-ತಿ-ರದ
ಅರಿ-ತಿ-ರ-ಬೇಕು
ಅರಿತು
ಅರಿ-ತು-ಕೊಂಡ-ರೇನೇ
ಅರಿ-ತು-ಕೊಂಡು
ಅರಿ-ತು-ಕೊಳ್ಳದೆ
ಅರಿ-ತು-ಕೊಳ್ಳದೇ
ಅರಿ-ತು-ಕೊಳ್ಳ-ಬ-ಹುದು
ಅರಿ-ತು-ಕೊಳ್ಳ-ಬೇಕು
ಅರಿ-ತು-ಕೊಳ್ಳ-ಬೇ-ಕೆಂಬ
ಅರಿ-ತು-ಕೊಳ್ಳುತ್ತವೆ
ಅರಿ-ತು-ಕೊಳ್ಳುತ್ತಾನೆ
ಅರಿ-ತು-ಕೊಳ್ಳುತ್ತಿದ್ದೇನೆ
ಅರಿ-ತು-ಕೊಳ್ಳುವ
ಅರಿ-ತು-ಕೊಳ್ಳು-ವು-ದಕ್ಕೆ
ಅರಿತೆ
ಅರಿಯದ
ಅರಿ-ಯ-ದವ
ಅರಿ-ಯ-ದ-ವರು
ಅರಿಯದೆ
ಅರಿಯದೇ
ಅರಿ-ಯ-ಬಲ್ಲರು
ಅರಿ-ಯ-ಬೇ-ಕಾದ
ಅರಿಯರು
ಅರಿ-ಯ-ಲ-ಶಕ್ಯ-ವಾದ
ಅರಿ-ಯ-ಲಾ-ರದ
ಅರಿ-ಯ-ಲಾ-ರದು
ಅರಿಯಲು
ಅರಿಯುವ
ಅರಿ-ಯು-ವಿಕೆ
ಅರಿ-ಯು-ವು-ದಕ್ಕೆ
ಅರಿ-ಯು-ವುದು
ಅರಿಯೋಣ
ಅರಿವನ್ನು
ಅರಿ-ವ-ಳಿ-ಕೆಯ
ಅರಿ-ವ-ಳಿ-ಕೆ-ಯನ್ನುಂಟು-ಮಾ-ಡು-ವು-ದ-ರಲ್ಲಿ
ಅರಿವಾ
ಅರಿವಾಗಿ
ಅರಿ-ವಾ-ಗಿತ್ತು
ಅರಿ-ವಾ-ಗಿ-ರ-ಬ-ಹುದು
ಅರಿ-ವಾ-ಗುತ್ತದೆ
ಅರಿ-ವಾ-ಗುತ್ತ-ಲಿದೆ
ಅರಿ-ವಾ-ಗುತ್ತಿಲ್ಲ
ಅರಿ-ವಾ-ಗು-ವುದು
ಅರಿ-ವಾ-ಗು-ವು-ದೆಂತು
ಅರಿ-ವಾ-ಯಿತು
ಅರಿವಿಗೂ
ಅರಿವಿಗೆ
ಅರಿವಿನ
ಅರಿ-ವಿ-ನಿಂದ
ಅರಿ-ವಿ-ರವು
ಅರಿ-ವಿ-ರ-ವು-ಗಳ
ಅರಿ-ವಿಲ್ಲದೆ
ಅರಿ-ವಿಲ್ಲದೇ
ಅರಿವು
ಅರಿ-ವುಂಟಾಗಿ
ಅರಿ-ವು-ಕಲ್ಪನೆ
ಅರಿವೂ
ಅರಿವೇ
ಅರಿ-ಷಡ್ವರ್ಗ-ಗ-ಳಲ್ಲಿ
ಅರಿ-ಷಡ್ವರ್ಗ-ಗ-ಳಿಗೆ
ಅರು-ಚುತ್ತೀಯೆ
ಅರು-ಣೋ-ದಯ
ಅರು-ಣೋ-ದ-ಯ-ದಲ್ಲಿ
ಅರು-ಣೋ-ದ-ಯ-ವಾದ
ಅರುವತ್ತ
ಅರುವತ್ತು
ಅರುಹಿದ್ದೆ
ಅರು-ಹುತ್ತಿದ್ದ
ಅರೆ-ಗ-ಳಲ್ಲಿ
ಅರೆ-ಗ-ಳಿ-ಗೆಯೂ
ಅರೆ-ನಿದ್ರೆ-ಯಲ್ಲಿ-ರು-ವಾಗ
ಅರೆಬೆಂದ
ಅರೆ-ಹುಚ್ಚ-ನಂತೆ
ಅರೇ
ಅರ್ಜಿ
ಅರ್ಜುನ
ಅರ್ಜುನನ
ಅರ್ಜು-ನ-ನಿಗೆ
ಅರ್ಜೆಂಟ್
ಅರ್ಥ
ಅರ್ಥ-ಎಂಬು-ದನ್ನು
ಅರ್ಥ-ಕಾ-ಮ-ಗಳೇ
ಅರ್ಥಕ್ಕಿಂತಲೂ
ಅರ್ಥ-ಗರ್ಭಿತ
ಅರ್ಥದ
ಅರ್ಥದಲ್ಲಿ
ಅರ್ಥಪೂರ್ಣ
ಅರ್ಥ-ಪೂರ್ಣ-ವಾಗಿ
ಅರ್ಥ-ಪೂರ್ಣ-ವಾ-ಗಿದೆ
ಅರ್ಥ-ಪೂರ್ಣ-ವಾ-ಗಿವೆ
ಅರ್ಥ-ಪೂರ್ಣ-ವಾ-ಗುತ್ತದೆ
ಅರ್ಥ-ಪೂರ್ಣ-ವಾ-ಗುತ್ತವೆ
ಅರ್ಥ-ಪೂರ್ಣ-ವಾದ
ಅರ್ಥ-ಪೂರ್ಣ-ವಾ-ದೀತು
ಅರ್ಥ-ಪೂರ್ಣ-ವೆಂದು
ಅರ್ಥಮಾಡಿ
ಅರ್ಥ-ಮಾ-ಡಿ-ಕೊಂಡಿ-ರುವ
ಅರ್ಥ-ಮಾ-ಡಿ-ಕೊಂಡಿಲ್ಲ-ವೆನ್ನ-ಬ-ಹುದು
ಅರ್ಥ-ಮಾ-ಡಿ-ಕೊಂಡು
ಅರ್ಥ-ಮಾ-ಡಿ-ಕೊಳ್ಳದೆ
ಅರ್ಥ-ಮಾ-ಡಿ-ಕೊಳ್ಳ-ಬ-ಹುದು
ಅರ್ಥ-ಮಾ-ಡಿ-ಕೊಳ್ಳ-ಬೇಕು
ಅರ್ಥ-ಮಾ-ಡಿ-ಕೊಳ್ಳಲು
ಅರ್ಥ-ಮಾ-ಡಿ-ಕೊಳ್ಳುವ
ಅರ್ಥ-ಮಾ-ಡಿ-ಕೊಳ್ಳು-ವ-ವ-ರಾರು
ಅರ್ಥ-ಮಾ-ಡಿ-ಕೊಳ್ಳು-ವು-ದಾ-ದರೂ
ಅರ್ಥ-ವತ್ತಾಗಿ
ಅರ್ಥವನ್ನು
ಅರ್ಥ-ವನ್ನುಂಟು-ಮಾ-ಡು-ವಂತೆ
ಅರ್ಥವಲ್ಲ
ಅರ್ಥ-ವಾಖ್ಯಾನ
ಅರ್ಥ-ವಾ-ಗದ
ಅರ್ಥ-ವಾ-ಗದೆ
ಅರ್ಥ-ವಾ-ಗದೇ
ಅರ್ಥ-ವಾ-ಗಿ-ರ-ಲಿಲ್ಲ
ಅರ್ಥ-ವಾ-ಗುತ್ತ-ದಷ್ಟೆ
ಅರ್ಥ-ವಾ-ಗುತ್ತದೆ
ಅರ್ಥ-ವಾ-ಗುತ್ತ-ದೆಂದೆ-ಣಿ-ಸು-ವುದು
ಅರ್ಥ-ವಾ-ಗುತ್ತಾ
ಅರ್ಥ-ವಾ-ಗುತ್ತಿಲ್ಲ
ಅರ್ಥ-ವಾ-ಗುವ
ಅರ್ಥ-ವಾ-ಗು-ವು-ದಿಲ್ಲವೆ
ಅರ್ಥ-ವಾ-ಗು-ವುದು
ಅರ್ಥ-ವಾ-ದರೂ
ಅರ್ಥ-ವಾ-ದರೆ
ಅರ್ಥ-ವಾ-ದೀತು
ಅರ್ಥ-ವಾ-ದೀತೇ
ಅರ್ಥ-ವಾ-ಯಿತು
ಅರ್ಥ-ವಾ-ಯಿತೆ
ಅರ್ಥ-ವಿ-ದೆಯೇ
ಅರ್ಥ-ವಿ-ರಲಿ
ಅರ್ಥ-ವಿ-ಸಿ-ಕೊಂಡು
ಅರ್ಥವುಂಟು
ಅರ್ಥವೂ
ಅರ್ಥವೆಂದು
ಅರ್ಥವೇ
ಅರ್ಥವೇನು
ಅರ್ಥವ್ಯಾಪ್ತಿ-ಯನ್ನು
ಅರ್ಥವ್ಯಾಪ್ತಿ-ಯನ್ನೂ
ಅರ್ಥಶಾಸ್ತ್ರ
ಅರ್ಥ-ಶಾಸ್ತ್ರಕ್ಕೆ
ಅರ್ಥ-ಶಾಸ್ತ್ರಜ್ಞರು
ಅರ್ಥ-ಶಾಸ್ತ್ರಜ್ಞರೂ
ಅರ್ಥ-ಶಾಸ್ತ್ರದ
ಅರ್ಥಹೀನ
ಅರ್ಥ-ಹೀ-ನ-ತೆ-ಗಳು
ಅರ್ಥ-ಹೀ-ನ-ವಷ್ಟೆ
ಅರ್ಥ-ಹೀ-ನ-ವೆಂದು
ಅರ್ಥ-ಹೀ-ನ-ವೆನ್ನುವ
ಅರ್ಥಾತ್
ಅರ್ಥೈ-ಸ-ಬ-ಹುದು
ಅರ್ಥೈ-ಸಿ-ಕೊಳ್ಳ
ಅರ್ಥೈ-ಸಿ-ಕೊಳ್ಳ-ಬಲ್ಲ
ಅರ್ಥೈ-ಸಿ-ಕೊಳ್ಳ-ಬ-ಹುದು
ಅರ್ಥೈ-ಸಿ-ಕೊಳ್ಳ-ಬೇಕು
ಅರ್ಥೈ-ಸಿ-ಕೊಳ್ಳಲು
ಅರ್ಥೈಸುವ
ಅರ್ಧ
ಅರ್ಧಕ್ಕಿ-ಳಿ-ಸ-ಬ-ಹುದು
ಅರ್ಧಕ್ಕಿ-ಳಿ-ಸಿ-ದಾಗ
ಅರ್ಧಕ್ಕೆ
ಅರ್ಧಗಂಟೆ
ಅರ್ಧ-ಗಂಟೆ-ಗೊಂದು
ಅರ್ಧ-ಘಂಟೆ-ಯಲ್ಲೇ
ಅರ್ಧ-ಜಾ-ಗೃ-ತ-ರಾ-ಗಿ-ರುತ್ತೇವೆ
ಅರ್ಧಜ್ಞಾ-ನಿ-ಗಳ
ಅರ್ಧಭಾಗ
ಅರ್ಧ-ಶ-ತ-ಮಾ-ನದ
ಅರ್ಧ-ಸತ್ತಂತಾಗಿ
ಅರ್ಧ-ಸ-ಮಯ
ಅರ್ಧಾಂಶಕ್ಕಿಂತ
ಅರ್ನಾಲ್ಡ್
ಅರ್ಪಣೆ
ಅರ್ಪಿಸಿದ
ಅರ್ಪಿ-ಸಿ-ದರು
ಅರ್ಪಿ-ಸಿ-ದರೊ
ಅರ್ಪಿ-ಸುತ್ತೇನೆ
ಅರ್ಪಿಸುವ
ಅರ್ಪಿಸುವೆ
ಅರ್ಹ
ಅರ್ಹತೆ
ಅರ್ಹತೆಗೆ
ಅರ್ಹ-ತೆ-ಯನ್ನು
ಅರ್ಹನಾದ
ಅರ್ಹರು
ಅರ್ಹ-ವಾ-ಗಿದೆ
ಅರ್ಹವಾದ
ಅರ್ಹಿಗೊ
ಅರ್ಹಿಗೋನ
ಅರ್ಹಿ-ಗೋ-ನಲ್ಲಿ
ಅಲಂಕ-ರಿಸಿ
ಅಲಂಕಾರ
ಅಲಂಘ್ಯ
ಅಲಕ್ಷಿ-ಸ-ಬೇಕು
ಅಲಕ್ಷಿ-ಸ-ಬೇ-ಡಿ-ನಿಷ್ಠೆ-ಯಿಂದ
ಅಲಕ್ಷಿ-ಸಿ-ರ-ಬ-ಹುದು
ಅಲಕ್ಷಿ-ಸುವ
ಅಲಕ್ಷಿ-ಸು-ವು-ದಿಲ್ಲ
ಅಲಕ್ಷಿ-ಸು-ವುದು
ಅಲಕ್ಷ್ಯ
ಅಲಕ್ಷ್ಯ-ದಿಂದ
ಅಲಕ್ಷ್ಯ-ರಾ-ಗುತ್ತೇವೆ
ಅಲ-ಗು-ಗ-ಳಂತೆ
ಅಲ-ಗು-ಗಳು
ಅಲರ್ಜಿ
ಅಲಾಸ್ಕಾದ
ಅಲು-ಗಾ-ಟಕ್ಕೆ
ಅಲುಗಾಡ
ಅಲು-ಗಾ-ಡದ
ಅಲು-ಗಾ-ಡದು
ಅಲು-ಗಾ-ಡದೆ
ಅಲು-ಗಾ-ಡದೇ
ಅಲು-ಗಾ-ಡಿ-ಸಲು
ಅಲು-ಗಾ-ಡಿ-ಸುತ್ತದೆ
ಅಲು-ಗಾ-ಡಿ-ಸು-ವಂತಿಲ್ಲ
ಅಲು-ಗಾ-ಡಿ-ಸು-ವಾ-ಗಿನ
ಅಲು-ಗು-ವಿ-ಕೆ-ಯಿಂದ
ಅಲೆ
ಅಲೆ-ಅ-ಲೆ-ಯಾಗಿ
ಅಲೆಕ್ಸಿಸ್
ಅಲೆಗಳ
ಅಲೆ-ಗ-ಳಂತೆ
ಅಲೆಗಳು
ಅಲೆ-ತ-ಗ-ಳನ್ನೂ
ಅಲೆದ
ಅಲೆ-ದಾ-ಡುತ್ತಿದ್ದ
ಅಲೆದು
ಅಲೆಯಂತೆ
ಅಲೆ-ಯಾ-ದರೋ
ಅಲೆಯುತ್ತ
ಅಲೆ-ಯುತ್ತಿ-ರು-ವಾಗ
ಅಲೆಯುವ
ಅಲೆ-ಯು-ವು-ದಕ್ಕೆ
ಅಲೌಕಿಕ
ಅಲೌ-ಕಿ-ಕ-ವಾದ
ಅಲ್
ಅಲ್ಪ
ಅಲ್ಪ-ಕಾ-ಲದ
ಅಲ್ಪತೆ
ಅಲ್ಪನನ್ನು
ಅಲ್ಪಪ್ರ-ಮಾ-ಣ-ದಲ್ಲಿ
ಅಲ್ಪ-ಬೆ-ಲೆಗೆ
ಅಲ್ಪ-ಭಾ-ಗ-ವನ್ನು
ಅಲ್ಪ-ಭಾ-ವ-ದಿಂದ
ಅಲ್ಪ-ಮಾ-ತು-ಗ-ಳಿಂದ
ಅಲ್ಪರೂ
ಅಲ್ಪವಾಗಿ
ಅಲ್ಪ-ಶಕ್ತಿಯ
ಅಲ್ಪ-ಸಂಖ್ಯಾ-ತ-ರಾ-ಗಲು
ಅಲ್ಪ-ಸಂಖ್ಯೆಯ
ಅಲ್ಪಸ್ವಲ್ಪ
ಅಲ್ಪಸ್ವಲ್ಪ-ವಾ-ದರೂ
ಅಲ್ಪಾ-ಹಾ-ರಿ-ಯಾ-ಗಿದ್ದು-ಕೊಂಡು
ಅಲ್ಫೆನ್ಸೊ
ಅಲ್ಬಾಮಾ
ಅಲ್ಬಾಮಾಳ
ಅಲ್ಲ
ಅಲ್ಲ-ಎಂದರು
ಅಲ್ಲ-ಗ-ಳೆದ
ಅಲ್ಲ-ಗ-ಳೆದು
ಅಲ್ಲ-ಗ-ಳೆ-ಯ-ಬ-ಹುದು
ಅಲ್ಲ-ಗ-ಳೆ-ಯ-ಲಾ-ಗದು
ಅಲ್ಲ-ಗ-ಳೆ-ಯ-ಲಾರ
ಅಲ್ಲ-ಗ-ಳೆ-ಯ-ಲಾ-ರರು
ಅಲ್ಲ-ಗ-ಳೆ-ಯ-ಲಿಲ್ಲ
ಅಲ್ಲ-ಗ-ಳೆ-ಯಲು
ಅಲ್ಲ-ಗ-ಳೆ-ಯುತ್ತ-ವಷ್ಟೆ
ಅಲ್ಲ-ಗ-ಳೆ-ಯುವ
ಅಲ್ಲ-ಗ-ಳೆ-ಯು-ವಂತಿಲ್ಲ
ಅಲ್ಲ-ಗ-ಳೆ-ಯು-ವಂತಿಲ್ಲ-ವಾ-ದರೂ
ಅಲ್ಲ-ಗ-ಳೆ-ಯು-ವು-ದ-ರಿಂದ
ಅಲ್ಲ-ಗ-ಳೆ-ಯು-ವುದು
ಅಲ್ಲ-ಗ-ಳೆ-ಯು-ವು-ದುಂಟು
ಅಲ್ಲ-ಗ-ಳೆ-ಯು-ವು-ದೆಂದರೆ
ಅಲ್ಲ-ಗೌ-ತಮ
ಅಲ್ಲದ
ಅಲ್ಲದೆ
ಅಲ್ಲದೇ
ಅಲ್ಲ-ಮಪ್ರ-ಭು-ವಿನ
ಅಲ್ಲಲ್ಲಿ
ಅಲ್ಲಲ್ಲೇ
ಅಲ್ಲವಷ್ಟೆ
ಅಲ್ಲವೆ
ಅಲ್ಲವೆಂದೂ
ಅಲ್ಲ-ವೆಂಬಂತೆ
ಅಲ್ಲವೇ
ಅಲ್ಲಸಾನಿ
ಅಲ್ಲಾ
ಅಲ್ಲಾಡಿಸಿ
ಅಲ್ಲಿ
ಅಲ್ಲಿಂದ
ಅಲ್ಲಿಂದಲೇ
ಅಲ್ಲಿಂದೀ-ಚೆಗೆ
ಅಲ್ಲಿಗೂ
ಅಲ್ಲಿಗೆ
ಅಲ್ಲಿಗೇ
ಅಲ್ಲಿದ್ದ
ಅಲ್ಲಿದ್ದರು
ಅಲ್ಲಿದ್ದಾನೆ
ಅಲ್ಲಿದ್ದು-ಕೊಂಡಿ-ರುವ
ಅಲ್ಲಿನ
ಅಲ್ಲಿಯ
ಅಲ್ಲಿ-ಯ-ವ-ರೆಗೂ
ಅಲ್ಲಿ-ಯ-ವ-ರೆಗೆ
ಅಲ್ಲಿಯೂ
ಅಲ್ಲಿಯೆ
ಅಲ್ಲಿಯೇ
ಅಲ್ಲಿ-ರ-ಬ-ಹು-ದೆಂಬ
ಅಲ್ಲಿ-ರುತ್ತಿದ್ದ
ಅಲ್ಲಿರುವ
ಅಲ್ಲಿ-ರು-ವುದು
ಅಲ್ಲಿಹನು
ಅಲ್ಲೂ
ಅಲ್ಲೆ
ಅಲ್ಲೆಲ್ಲ
ಅಲ್ಲೆಲ್ಲೂ
ಅಲ್ಲೇ
ಅಲ್ಲೊಂದು
ಅಲ್ಲೊಬ್ಬ
ಅಲ್ಸೇಷನ್
ಅಳತೆ
ಅಳ-ತೆ-ಗೋಲು
ಅಳತೆಯೇ
ಅಳ-ತೊ-ಡ-ಗಿದ
ಅಳಬೇಡ
ಅಳಲನ್ನು
ಅಳಲು
ಅಳ-ಲೇ-ಕಾ-ಯಿಯೇ
ಅಳ-ವ-ಡದ
ಅಳ-ವ-ಡಿ-ಸಿ-ಕೊಂಡರೆ
ಅಳ-ವ-ಡಿ-ಸಿ-ಕೊಂಡಿದ್ದೇನೆ
ಅಳ-ವ-ಡಿ-ಸಿ-ಕೊಳ್ಳ-ದಿದ್ದರೆ
ಅಳ-ವಾ-ದಂತೆಲ್ಲ
ಅಳಿದರೂ
ಅಳಿ-ದು-ಳಿದ
ಅಳಿಯದು
ಅಳಿಯಲಿ
ಅಳಿ-ಯು-ವುದು
ಅಳಿ-ಲು-ಸೇ-ವೆ-ಯನ್ನು
ಅಳಿ-ಸ-ಲಾ-ಗದ್ದು
ಅಳಿಸಿದ
ಅಳು
ಅಳುತ್ತ
ಅಳುತ್ತಾನೆ
ಅಳುತ್ತಿದ್ದಾಳೆ
ಅಳುತ್ತಿದ್ದೆ
ಅಳುತ್ತಿ-ರುವ
ಅಳುತ್ತಿ-ರು-ವು-ದನ್ನೂ
ಅಳುಮೂತಿ
ಅಳುಮೋರೆ
ಅಳು-ವಂತಿದೆ
ಅಳು-ವ-ವರು
ಅಳುವಿಗೆ
ಅಳುವುದು
ಅಳುವೇ
ಅಳೆಯದೇ
ಅಳೆ-ಯ-ಲಾ-ಗದ
ಅಳೆ-ಯ-ಲಾ-ರೆವು
ಅಳೆಯಲೂ
ಅಳೆ-ಯುತ್ತಾರೆ
ಅಳೆಯುವ
ಅಳೆ-ಯು-ವು-ದಕ್ಕಿಂತಲೂ
ಅಳೆ-ಯು-ವುದು
ಅವಕಾಶ
ಅವ-ಕಾ-ಶ-ಇ-ವು-ಗ-ಳನ್ನು
ಅವ-ಕಾ-ಶ-ಕೊ-ಡದೆ
ಅವ-ಕಾ-ಶ-ಗಳ
ಅವ-ಕಾ-ಶ-ಗ-ಳನ್ನು
ಅವ-ಕಾ-ಶ-ಗ-ಳಿಗೆ
ಅವ-ಕಾ-ಶ-ಗ-ಳಿವೆ
ಅವ-ಕಾ-ಶ-ಗಳು
ಅವ-ಕಾ-ಶ-ವನ್ನು
ಅವ-ಕಾ-ಶ-ವನ್ನೂ
ಅವ-ಕಾ-ಶ-ವಾ-ಗದ
ಅವ-ಕಾ-ಶ-ವಾ-ಗದೆ
ಅವ-ಕಾ-ಶ-ವಾ-ಗುತ್ತದೆ
ಅವ-ಕಾ-ಶ-ವಿತ್ತು
ಅವ-ಕಾ-ಶ-ವಿದೆ
ಅವ-ಕಾ-ಶ-ವಿದ್ದರೂ
ಅವ-ಕಾ-ಶ-ವಿ-ರ-ಬೇ-ಕಾ-ದುದು
ಅವ-ಕಾ-ಶ-ವಿ-ರ-ಲಿಲ್ಲ
ಅವ-ಕಾ-ಶ-ವಿಲ್ಲ
ಅವ-ಕಾ-ಶ-ವಿಲ್ಲ-ದಿದ್ದರೂ
ಅವ-ಕಾ-ಶ-ವಿಲ್ಲ-ದಿ-ರ-ಬ-ಹುದು
ಅವ-ಕಾ-ಶ-ವಿಲ್ಲ-ವೆಂಬ
ಅವ-ಕಾ-ಶವೇ
ಅವ-ಗ-ಣ-ನೆಗೆ
ಅವ-ಗಾ-ಹ-ನೆಗೆ
ಅವ-ಘ-ಡ-ಗ-ಳಲ್ಲಿ
ಅವ-ಘ-ಡ-ಗಳು
ಅವ-ಘ-ಡ-ವೊಂದ-ರಲ್ಲಿ
ಅವಜ್ಞೆ
ಅವ-ತ-ರ-ಣಿ-ಕೆ-ಗಳ
ಅವ-ತ-ರಿ-ಸಿದ್ದ
ಅವ-ತ-ರಿ-ಸಿದ್ದಾರೆ
ಅವತಾರ
ಅವ-ತಾ-ರ-ಗಳು
ಅವ-ತಾ-ರ-ವಾ-ಯಿತು
ಅವಧಾನ
ಅವಧಿಗೆ
ಅವನ
ಅವನಂಥ
ಅವ-ನ-ತ-ವಾಗಿ
ಅವನತಿ
ಅವ-ನ-ತಿಗೆ
ಅವ-ನ-ತಿಯ
ಅವ-ನ-ತಿ-ಯತ್ತ
ಅವ-ನ-ದಾ-ಗುತ್ತದೆ
ಅವನದು
ಅವನದೇ
ಅವನನ್ನು
ಅವನನ್ನೇ
ಅವನಲ್ಲಿ
ಅವ-ನಲ್ಲಿಗೆ
ಅವನಲ್ಲೇ
ಅವನಷ್ಟು
ಅವ-ನಾ-ಡಿದ
ಅವನಿಂದ
ಅವ-ನಿಂದಲೇ
ಅವ-ನಿಂದಾಗಿ
ಅವ-ನಿ-ಗ-ರಿ-ವಿಲ್ಲ-ದೆಯೇ
ಅವ-ನಿ-ಗರ್ಥ-ವಾ-ಗದೇ
ಅವ-ನಿ-ಗಾಗಿ
ಅವ-ನಿ-ಗಾ-ಯಿತು
ಅವ-ನಿ-ಗಿದೆ
ಅವ-ನಿ-ಗಿ-ರದು
ಅವ-ನಿ-ಗಿ-ರ-ಲಿಲ್ಲ
ಅವ-ನಿ-ಗಿ-ರುತ್ತಿ-ರ-ಲಿಲ್ಲ
ಅವ-ನಿ-ಗಿಲ್ಲ
ಅವ-ನಿ-ಗುಂಟಾ-ಗಿದೆ
ಅವನಿಗೆ
ಅವನಿಗೇ
ಅವ-ನಿ-ಗೊಂದು
ಅವ-ನಿ-ಗೊ-ದಗಿ
ಅವ-ನಿದ್ದಾನೆ
ಅವನಿನ್ನೂ
ಅವನು
ಅವನೂ
ಅವ-ನೆಂಥ-ವ-ನೆಂಬು-ದನ್ನು
ಅವನೆಂದ
ಅವ-ನೆಂದಂತೆ-ಯ-ಶಸ್ಸಿನ
ಅವ-ನೆ-ಡೆಗೇ
ಅವನೇ
ಅವನೇಕೆ
ಅವನೇನೂ
ಅವನೇನೋ
ಅವ-ನೊಂದಿಗೆ
ಅವನೊಂದು
ಅವ-ನೊ-ಡನೆ
ಅವನೊಬ್ಬ
ಅವ-ನೊಬ್ಬನೇ
ಅವನೊಮ್ಮೆ
ಅವನ್ನು
ಅವನ್ನೂ
ಅವನ್ನೆಲ್ಲ
ಅವ-ಮಾ-ನ-ಪ-ಡಿ-ಸಿ-ಕೊಂಡಂತೆ
ಅವ-ಮಾ-ನ-ವನ್ನು
ಅವ-ಮಾ-ನ-ವನ್ನೂ
ಅವ-ಮಾ-ನಿ-ಸ-ಬೇಡ
ಅವಯವ
ಅವ-ಯ-ವ-ಗ-ಳಂತೆ
ಅವರ
ಅವರಂತೆ
ಅವರಂಥ
ಅವರದು
ಅವ-ರದ್ದಲ್ಲ
ಅವರದ್ದು
ಅವರನ್ನು
ಅವರನ್ನೂ
ಅವ-ರನ್ನೆಂತು
ಅವ-ರನ್ನೆಲ್ಲ
ಅವರನ್ನೇ
ಅವರಲ್ಲ
ಅವರಲ್ಲಿ
ಅವ-ರಲ್ಲಿತ್ತು
ಅವ-ರಲ್ಲಿ-ರುವ
ಅವ-ರಲ್ಲಿ-ರು-ವು-ದೆಲ್ಲ
ಅವರಲ್ಲೇ
ಅವ-ರಲ್ಲೊಬ್ಬ
ಅವ-ರಲ್ಲೊಬ್ಬರು
ಅವ-ರ-ವರ
ಅವರಷ್ಟೇ
ಅವ-ರಾ-ಗಿದ್ದರು
ಅವ-ರಾ-ಗಿದ್ದರೆ
ಅವ-ರಾ-ಗಿ-ರ-ಲಿಲ್ಲ
ಅವ-ರಾ-ಗುತ್ತಾರೆ
ಅವ-ರಾ-ದರೂ
ಅವರಿಂದ
ಅವ-ರಿ-ಗಾಗಿ
ಅವ-ರಿ-ಗಿಂತ
ಅವ-ರಿ-ಗಿದೆ
ಅವ-ರಿ-ಗಿಬ್ಬರೇ
ಅವರಿಗೂ
ಅವರಿಗೆ
ಅವ-ರಿ-ಗೆಲ್ಲ
ಅವರಿಗೇ
ಅವ-ರಿ-ಗೊಂದು
ಅವ-ರಿ-ಗೊಬ್ಬ
ಅವ-ರಿದ್ದಿದ್ದರೆ
ಅವರಿನ್ನೂ
ಅವ-ರಿಬ್ಬರ
ಅವ-ರಿಬ್ಬರೂ
ಅವ-ರಿಬ್ಬರೇ
ಅವ-ರಿ-ರುವ
ಅವ-ರಿ-ವರ
ಅವ-ರಿ-ವ-ರಿಗೆ
ಅವ-ರಿ-ವರು
ಅವರೀಗ
ಅವರು
ಅವ-ರು-ಗಳು
ಅವರೂ
ಅವ-ರೆಂದಂತೆ
ಅವ-ರೆಂದರು
ಅವರೆಂದೂ
ಅವ-ರೆ-ಡೆಗೆ
ಅವ-ರೆ-ಣಿ-ಕೆ-ಯಂತೆ
ಅವ-ರೆ-ದು-ರಿಗೆ
ಅವ-ರೆ-ದುರು
ಅವ-ರೆನ್ನುತ್ತಾರೆ
ಅವ-ರೆನ್ನು-ವಂತೆ
ಅವರೆಲ್ಲ
ಅವ-ರೆಲ್ಲರ
ಅವ-ರೆಲ್ಲ-ರಿಗೂ
ಅವ-ರೆಲ್ಲರೂ
ಅವರೆಲ್ಲಾ
ಅವರೇ
ಅವ-ರೇ-ನಾ-ದರೂ
ಅವರೇನು
ಅವರೇನೂ
ಅವ-ರೊಂದಿಗೆ
ಅವ-ರೊ-ಡನೆ
ಅವ-ರೊಬ್ಬರೇ
ಅವ-ಲಂಬಿ-ಸ-ಬೇಕು
ಅವ-ಲಂಬಿಸಿ
ಅವ-ಲಂಬಿ-ಸಿದೆ
ಅವ-ಲಂಬಿ-ಸಿದ್ದು
ಅವ-ಲಂಬಿ-ಸಿಲ್ಲ
ಅವ-ಲೋ-ಕಿಸಿ
ಅವ-ಲೋ-ಕಿ-ಸುತ್ತದೆ
ಅವಳ
ಅವ-ಳ-ಡೆಗೆ
ಅವಳದು
ಅವಳನ್ನು
ಅವಳಲ್ಲಿ
ಅವಳಲ್ಲೂ
ಅವಳಾಗಿ
ಅವ-ಳಾ-ಗಿಯೇ
ಅವಳಿ
ಅವಳಿಗೆ
ಅವಳು
ಅವಳೇನೂ
ಅವ-ಶೇ-ಷ-ಗ-ಳಾದ
ಅವಶ್ಯ
ಅವಶ್ಯಂ
ಅವಶ್ಯ-ವಾಗಿ
ಅವಶ್ಯ-ವಾದ
ಅವಶ್ಯವೂ
ಅವಶ್ಯ-ವೆ-ನಿ-ಸುವ
ಅವಶ್ಯವೇ
ಅವಸರ
ಅವ-ಸ-ರ-ದಲ್ಲಿ
ಅವ-ಸ-ರ-ವಾಗಿ
ಅವಸಾದ
ಅವ-ಸಾ-ದ-ಗ-ಳನ್ನು
ಅವ-ಸಾ-ದ-ವನ್ನು
ಅವಸಾನ
ಅವಸ್ಥಾತ್ರಯ
ಅವಸ್ಥೆ
ಅವಸ್ಥೆ-ಯನ್ನು
ಅವಸ್ಥೆ-ಯಲ್ಲಿ
ಅವಸ್ಥೆ-ಯಲ್ಲಿ-ರು-ವಾಗ
ಅವಸ್ಥೆ-ಯಿಂದ
ಅವಸ್ಥೆಯೇ
ಅವ-ಹೇ-ಳನ
ಅವ-ಹೇ-ಳ-ನ-ಮಾ-ಡಿ-ದರು
ಅವಾಕ್ಕಾದೆ
ಅವಾಚ್ಯ
ಅವಿ-ಚಾ-ರಿತ
ಅವಿಜ್ಞೇ-ಯ-ನಾ-ಗಿದ್ದಾನೆ
ಅವಿದ್ಯಾ-ವಂತ-ರಿಗೆ
ಅವಿದ್ಯಾ-ವಂತರೂ
ಅವಿದ್ಯಾ-ವಂತ-ರೆ-ನಿ-ಸಿ-ಕೊಂಡ
ಅವಿ-ಧೇ-ಯತೆ
ಅವಿನಾಶಿ
ಅವಿಭಾಜ್ಯ
ಅವಿರತ
ಅವಿ-ರೋ-ಧವೂ
ಅವಿ-ವೇ-ಕ-ದಿಂದ
ಅವಿಶ್ರಾಂತ
ಅವಿಶ್ವಾಸ
ಅವು
ಅವು-ಗ-ಳಿಲ್ಲದೆ
ಅವುಗಳ
ಅವು-ಗ-ಳದೇ
ಅವು-ಗ-ಳನ್ನು
ಅವು-ಗ-ಳನ್ನೆ-ಣಿ-ಸದೆ
ಅವು-ಗ-ಳನ್ನೆಲ್ಲ
ಅವು-ಗ-ಳನ್ನೆಲ್ಲಾ
ಅವು-ಗ-ಳಲ್ಲಿ
ಅವು-ಗ-ಳಲ್ಲಿನ
ಅವು-ಗ-ಳಲ್ಲಿ-ರುವ
ಅವು-ಗ-ಳಲ್ಲೊಂದಷ್ಟೆ
ಅವು-ಗ-ಳಲ್ಲೊಂದು
ಅವು-ಗ-ಳಿಂದ
ಅವು-ಗ-ಳಿಂದಾ-ಗಿಯೇ
ಅವು-ಗ-ಳಿಂದೇನು
ಅವು-ಗ-ಳಿ-ಗಲ್ಲ
ಅವು-ಗ-ಳಿ-ಗುಂಟಾದ
ಅವು-ಗ-ಳಿಗೆ
ಅವು-ಗ-ಳಿ-ಗೇನು
ಅವುಗಳು
ಅವು-ಗ-ಳೆಲ್ಲ
ಅವು-ಗ-ಳೆಲ್ಲಾ
ಅವುಗಳೇ
ಅವು-ಗ-ಳೊ-ಡನೆ
ಅವೆಲ್ಲ
ಅವೆಲ್ಲಕ್ಕೂ
ಅವೆಲ್ಲ-ವನ್ನೂ
ಅವೆಲ್ಲ-ವು-ಗಳ
ಅವೆಲ್ಲವೂ
ಅವೆಲ್ಲಾ
ಅವೇ
ಅವೈಜ್ಞಾ-ನಿಕ
ಅವ್ಯಕ್ತ
ಅವ್ಯಕ್ತಸ್ಥಿತಿ
ಅವ್ಯ-ಯ-ವಾದ
ಅವ್ಯ-ವಸ್ಥಿತ
ಅವ್ಯವಸ್ಥೆ
ಅವ್ಯ-ವಸ್ಥೆ-ಗೊ-ಳ-ಗಾದ
ಅವ್ಯ-ವಸ್ಥೆಯ
ಅವ್ಯ-ವಸ್ಥೆ-ಯಿಂದ
ಅವ್ಯಾ-ವ-ಹಾ-ರಿಕ
ಅಶಕ್ತ
ಅಶಕ್ತ-ನನ್ನು
ಅಶಕ್ತ-ನಾ-ಗಿದ್ದು-ದ-ರಿಂದ
ಅಶಕ್ತ-ನಾ-ದರೆ
ಅಶಕ್ತ-ನೆಂದೂ
ಅಶಕ್ತ-ರೆಂದು-ಕೊಳ್ಳ-ಬಲ್ಲೆವೆ
ಅಶಕ್ಯ-ವಾದ
ಅಶನಕ್ಕೆ
ಅಶಾಂತಿ
ಅಶಾಂತಿ-ಗ-ಳಿಂದ
ಅಶಾಂತಿ-ಯಿಂದ
ಅಶಾಂತಿ-ಯುಂಟಾ-ದರೆ
ಅಶಿ-ಸು-ವನೆ
ಅಶಿಸ್ತಿನ
ಅಶಿಸ್ತು
ಅಶುದ್ಧ-ರಾ-ಗು-ವಿರಿ
ಅಶುದ್ಧ-ರೆಂದು
ಅಶು-ಭ-ವನ್ನು
ಅಶೋಕ
ಅಶೋ-ಕ-ಚಕ್ರ
ಅಶ್ರದ್ಧೆ
ಅಶ್ರದ್ಧೆ-ಗಳು
ಅಶ್ರದ್ಧೆಯೇ
ಅಶ್ರು-ಜ-ಲ-ದಿಂದ
ಅಶ್ಲೀಲ
ಅಷ್ಟ-ರ-ವ-ರೆಗೂ
ಅಷ್ಟ-ರ-ವ-ರೆಗೆ
ಅಷ್ಟಷ್ಟೂ
ಅಷ್ಟಾಗಿ
ಅಷ್ಟಾದರೂ
ಅಷ್ಟು
ಅಷ್ಟೆ
ಅಷ್ಟೇ
ಅಷ್ಟೇಕೆ
ಅಷ್ಟೇನೂ
ಅಷ್ಟೊಂದು
ಅಸಂಖ್ಯ
ಅಸಂಖ್ಯ-ಜಾ-ತಿಯ
ಅಸಂಖ್ಯಾತ
ಅಸಂಗತ
ಅಸಂಗ-ತ-ವಾಗಿ
ಅಸಂತುಷ್ಟ-ನಾಗಿ
ಅಸಂಬದ್ಧ
ಅಸಂಬದ್ಧ-ವಾಗಿ
ಅಸಂಬದ್ಧ-ವಾ-ಗಿ-ಯಾ-ದರೂ
ಅಸಂಬದ್ಧ-ವಾ-ಗುತ್ತದೆ
ಅಸಂಸ್ಕೃತ
ಅಸಡ್ಡೆ
ಅಸಡ್ಡೆ-ಯಿಂದ
ಅಸತೋ
ಅಸತ್ಯ-ವನ್ನು
ಅಸತ್ಯ-ವನ್ನೂ
ಅಸತ್ಯವು
ಅಸದೃಶ
ಅಸ-ದೃ-ಶ-ವಾ-ಗಿತ್ತು
ಅಸಭ್ಯ
ಅಸ-ಮರ್ಥ-ನಾ-ಗಿಯೇ
ಅಸ-ಮರ್ಥ-ರನ್ನು
ಅಸ-ಮರ್ಥ-ರಾದ
ಅಸ-ಮರ್ಥ-ರಾ-ದರು
ಅಸ-ಮರ್ಥ-ರಾ-ದುದು
ಅಸ-ಮರ್ಥ-ಳಾ-ಗಿದ್ದಾಳೆ
ಅಸ-ಮರ್ಥ-ಳಾ-ದಳು
ಅಸ-ಮರ್ಥ-ವಾ-ಗಿ-ರು-ವು-ದ-ರಿಂದ
ಅಸ-ಮರ್ಥ-ವಾ-ಗಿವೆ
ಅಸ-ಮರ್ಥ-ವಾ-ದುದು
ಅಸ-ಮರ್ಥ-ವಾ-ದುವು
ಅಸ-ಮಾ-ಧಾನ
ಅಸ-ಮಾ-ಧಾ-ನ-ಗಳ
ಅಸ-ಮಾ-ಧಾ-ನ-ಗ-ಳನ್ನು
ಅಸ-ಮಾ-ಧಾ-ನ-ಗಳು
ಅಸ-ಮಾ-ಧಾ-ನದ
ಅಸ-ಮಾ-ಧಾ-ನ-ವನ್ನು
ಅಸ-ಮಾ-ನ-ತೆಯ
ಅಸ-ಹ-ನೀಯ
ಅಸ-ಹ-ನೀ-ಯ-ವಾ-ಗುತ್ತವೆ
ಅಸಹನೆ
ಅಸಹಾಯ
ಅಸ-ಹಾ-ಯಕ
ಅಸ-ಹಾ-ಯ-ಕತೆ
ಅಸ-ಹಾ-ಯ-ಕ-ತೆಈ
ಅಸ-ಹಾ-ಯ-ಕ-ತೆಗೆ
ಅಸ-ಹಾ-ಯ-ಕ-ತೆ-ಯನ್ನು
ಅಸ-ಹಾ-ಯ-ಕ-ತೆ-ಯಾಗಿ
ಅಸ-ಹಾ-ಯ-ಕ-ನಾಗಿ
ಅಸ-ಹಾ-ಯ-ಕ-ನಾ-ಗಿ-ಬಿ-ಡುತ್ತಾನೆ
ಅಸ-ಹಾ-ಯ-ಕನೇ
ಅಸ-ಹಾ-ಯ-ಕ-ಳಾಗಿ
ಅಸ-ಹಾ-ಯ-ಕಸ್ಥಿ-ತಿ-ಗಳು
ಅಸ-ಹಾ-ಯ-ನಾಗಿ
ಅಸ-ಹಾ-ಯಸ್ಥಿ-ತಿ-ಯಿಂದ
ಅಸ-ಹಿಷ್ಣುತೆ
ಅಸಹ್ಯ
ಅಸಹ್ಯ-ವೆ-ನಿ-ಸ-ಬ-ಹು-ದಾದ
ಅಸಹ್ಯ-ಕ-ರವೂ
ಅಸಹ್ಯ-ಪ-ಡು-ವೆಯೋ
ಅಸಹ್ಯ-ಭಾ-ವ-ನೆ-ಯನ್ನು
ಅಸಹ್ಯ-ವಾ-ಗಿತ್ತು
ಅಸಹ್ಯ-ವಾ-ಗು-ವುದು
ಅಸಾ-ಧಾ-ರಣ
ಅಸಾಧುವೂ
ಅಸಾಧ್ಯ
ಅಸಾಧ್ಯ-ಇಂಥ
ಅಸಾಧ್ಯ-ಎನ್ನುವ
ಅಸಾಧ್ಯದ
ಅಸಾಧ್ಯ-ವಲ್ಲ
ಅಸಾಧ್ಯ-ವಾ-ಗಿದೆ
ಅಸಾಧ್ಯ-ವಾದ
ಅಸಾಧ್ಯ-ವಾ-ದರೂ
ಅಸಾಧ್ಯ-ವಾ-ದಾಗ
ಅಸಾಧ್ಯ-ವಾ-ದು-ದನ್ನು
ಅಸಾಧ್ಯ-ವಾ-ದುದು
ಅಸಾಧ್ಯ-ವಾ-ವುದೂ
ಅಸಾಧ್ಯ-ವಿಲ್ಲ
ಅಸಾಧ್ಯವೂ
ಅಸಾಧ್ಯ-ವೆಂದು
ಅಸಾಧ್ಯ-ವೆಂದೂ
ಅಸಾಧ್ಯ-ವೆ-ನಿ-ಸಿದ
ಅಸಾಧ್ಯ-ವೆ-ನಿ-ಸು-ವುದು
ಅಸಾಧ್ಯ-ವೇ-ನಲ್ಲ
ಅಸಾಮರ್ಥ್ಯ
ಅಸಾಮಾನ್ಯ
ಅಸಾ-ಮಾನ್ಯ-ವೆ-ನಿ-ಸುವ
ಅಸಾ-ಮಾನ್ಯತೆ
ಅಸಾ-ಮಾನ್ಯ-ರಾಗಿ
ಅಸೀಮ
ಅಸೀ-ಮ-ವಾದ
ಅಸು
ಅಸುಖಿ
ಅಸು-ನೀ-ಗಿ-ದರು
ಅಸು-ಹಿಂಡುತ್ತಿವೆ
ಅಸೂಯಾ
ಅಸೂ-ಯಾ-ಪರ
ಅಸೂಯೆ
ಅಸೂ-ಯೆ-ಗಳ
ಅಸೂ-ಯೆ-ಗ-ಳಿಂದಾಗಿ
ಅಸೂ-ಯೆ-ಗ-ಳಿಗೆ
ಅಸೂ-ಯೆ-ಗೊಂಡು
ಅಸೌಖ್ಯ-ದಿಂದ
ಅಸ್ಖಲಿತ
ಅಸ್ಟಿ-ಯೋ-ಪ-ಥಿಜ್ಞ
ಅಸ್ತಮಾ
ಅಸ್ತ-ಮಾ-ರೋ-ಗ-ದಿಂದ
ಅಸ್ತ-ಮಿ-ತ-ರಾ-ದಾಗ
ಅಸ್ತ-ಮಿ-ಸಿ-ದಾಗ
ಅಸ್ತ-ಮಿ-ಸುತ್ತಿವೆ
ಅಸ್ತವ್ಯಸ್ತ
ಅಸ್ತವ್ಯಸ್ತ-ವಾಗಿ
ಅಸ್ತವ್ಯಸ್ತ-ವಾ-ಗು-ವುದು
ಅಸ್ತವ್ಯಸ್ತವೊ
ಅಸ್ತಿತ್ವ
ಅಸ್ತಿತ್ವ-ಅ-ದೃಷ್ಟಕ್ಕೆ
ಅಸ್ತಿತ್ವ-ಇ-ವು-ಗ-ಳನ್ನು
ಅಸ್ತಿತ್ವಕ್ಕಾಗಿ
ಅಸ್ತಿತ್ವಕ್ಕಿ-ರುವ
ಅಸ್ತಿತ್ವಕ್ಕೂ
ಅಸ್ತಿತ್ವಕ್ಕೆ
ಅಸ್ತಿತ್ವ-ಗಳ
ಅಸ್ತಿತ್ವದ
ಅಸ್ತಿತ್ವ-ದಲ್ಲಿ
ಅಸ್ತಿತ್ವ-ವನ್ನು
ಅಸ್ತಿತ್ವ-ವನ್ನೇ
ಅಸ್ತಿತ್ವ-ವಾ-ದದ
ಅಸ್ತಿತ್ವ-ವಾದಿ
ಅಸ್ತಿತ್ವ-ವಿದೆ
ಅಸ್ತಿತ್ವ-ವಿಲ್ಲ
ಅಸ್ತಿತ್ವವೇ
ಅಸ್ತಿ-ಪಾಸ್ತಿ-ಗಳ
ಅಸ್ತಿ-ಭಾ-ರದ
ಅಸ್ತಿ-ಭಾ-ರ-ವಿಲ್ಲದ
ಅಸ್ತೇಯ
ಅಸ್ತ್ರ
ಅಸ್ತ್ರಕ್ಕೂ
ಅಸ್ತ್ರಗಳ
ಅಸ್ತ್ರಗಳು
ಅಸ್ತ್ರ-ದೊಂದಿಗೆ
ಅಸ್ತ್ರವಾಗಿ
ಅಸ್ಥಾಯಿಯೋ
ಅಸ್ಪಷ್ಟ
ಅಸ್ಪಷ್ಟ-ವಾಗಿ
ಅಸ್ಪಷ್ಟ-ವಾ-ಗಿಯೂ
ಅಸ್ಪಷ್ಟ-ವಾದ
ಅಸ್ವಸ್ಥತೆ
ಅಸ್ವಸ್ಥ-ತೆ-ಯಿಂದ
ಅಸ್ವಸ್ಥ-ನಾದ
ಅಸ್ವಾ-ಭಾ-ವಿ-ಕ-ವಲ್ಲ
ಅಸ್ವೀಕಾರ
ಅಸ್ಸಿಸಿಯ
ಅಹಂ
ಅಹಂಅನ್ನು
ಅಹಂಕಾರ
ಅಹಂಕಾ-ರತ್ಯಾ-ಗ-ಇ-ವು-ಗ-ಳೆಲ್ಲ
ಅಹಂಕಾ-ರದ
ಅಹಂಕಾ-ರ-ದಿಂದ
ಅಹಂಕಾ-ರ-ದಿಂದಲೇ
ಅಹಂಕಾ-ರ-ವನ್ನು
ಅಹಂಕಾ-ರ-ವಲ್ಲ
ಅಹಂಕಾ-ರವು
ಅಹಂಕಾ-ರವೇ
ಅಹಂಕಾ-ರಿ-ಗ-ಳಿ-ಗಿಂತ
ಅಹಂಗಳು
ಅಹಂಗೆ
ಅಹಂನ
ಅಹಂನ್ನು
ಅಹಂಪ್ರಜ್ಞೆ
ಅಹ-ಮಿ-ಕೆಯ
ಅಹರ್ನಿಶಿ
ಅಹಿಂಸೆ
ಅಹಿಂಸೆಯ
ಅಹಿಂಸೆ-ಯಲ್ಲಿ
ಅಹಿಂಸೆಯೂ
ಅಹಿರ್ಬುಧ್ನ್ಯ-ಸಂಹಿ-ತೆ-ಯಲ್ಲಿ
ಅಹುದು
ಅಹೋ
ಆ
ಆಂ
ಆಂಗ್ಲ
ಆಂಗ್ಲನೊಬ್ಬ
ಆಂಗ್ಲ-ಭಾ-ಷೆಯ
ಆಂಗ್ಲ-ಭಾ-ಷೆ-ಯಲ್ಲಿ
ಆಂಗ್ಲ-ಭಾ-ಷೆ-ಯಲ್ಲಿದೆ
ಆಂಗ್ಲರಿಗೂ
ಆಂಗ್ಲರಿಗೆ
ಆಂಗ್ಲರು
ಆಂಜನೇಯ
ಆಂಜ-ನೇ-ಯಸ್ವಾ-ಮಿಗೆ
ಆಂತ
ಆಂತರಿಕ
ಆಂತ-ರಿ-ಕ-ಶಕ್ತಿ
ಆಂತ-ರಿ-ಕ-ಶಾಂತಿ-ಯನ್ನೂ
ಆಂತರ್ಯ
ಆಂತರ್ಯದ
ಆಂತರ್ಯ-ದಲ್ಲಿ
ಆಂತರ್ಯ-ದಲ್ಲಿನ
ಆಂತರ್ಯ-ದಲ್ಲೂ
ಆಂತರ್ಯ-ದಲ್ಲೇ
ಆಂಶಿಕ
ಆಂಶಿ-ಕ-ದೃಷ್ಟಿ-ಯಿಂದ
ಆಕರ್ಷಕ
ಆಕರ್ಷ-ಕವೂ
ಆಕರ್ಷಣೆ
ಆಕರ್ಷ-ಣೆ-ಗಳು
ಆಕರ್ಷ-ಣೆಗೆ
ಆಕರ್ಷ-ಣೆಗೇ
ಆಕರ್ಷ-ಣೆ-ಯಿಂದ
ಆಕರ್ಷಿ-ತ-ನಾ-ಗ-ಲಿಲ್ಲ
ಆಕರ್ಷಿ-ತ-ರಾ-ಗುತ್ತಿದ್ದರೆ
ಆಕರ್ಷಿ-ತ-ರಾ-ಗುತ್ತಿದ್ದಾರೆ
ಆಕರ್ಷಿ-ತ-ರಾ-ಗುವ
ಆಕರ್ಷಿಸಿ
ಆಕರ್ಷಿ-ಸಿದೆ
ಆಕರ್ಷಿ-ಸುತ್ತಿದ್ದ
ಆಕಸ್ಮಾತ್
ಆಕಸ್ಮಿಕ
ಆಕಸ್ಮಿ-ಕ-ಗ-ಳಲ್ಲ-ಅವು
ಆಕಸ್ಮಿ-ಕ-ಗಳು
ಆಕಸ್ಮಿ-ಕದ
ಆಕಸ್ಮಿ-ಕ-ವಲ್ಲ
ಆಕಸ್ಮಿ-ಕ-ವಾಗಿ
ಆಕಸ್ಮಿ-ಕ-ವಾದ
ಆಕಸ್ಮಿ-ಕವೇ
ಆಕಾಂಕ್ಷೆ
ಆಕಾಂಕ್ಷೆ-ಗಳ
ಆಕಾಂಕ್ಷೆ-ಗ-ಳನ್ನಿಟ್ಟು-ಕೊಂಡು
ಆಕಾಂಕ್ಷೆ-ಗ-ಳನ್ನು
ಆಕಾಂಕ್ಷೆ-ಗಳು
ಆಕಾಂಕ್ಷೆ-ಯನ್ನದು
ಆಕಾಂಕ್ಷೆ-ಯನ್ನು
ಆಕಾಂಕ್ಷೆಯೂ
ಆಕಾರ
ಆಕಾರಕ್ಕೇ
ಆಕಾ-ರ-ಗ-ಳನ್ನುಂಟು
ಆಕಾ-ರ-ಗ-ಳಲ್ಲಿ
ಆಕಾ-ರ-ಗ-ಳಿಂದ
ಆಕಾ-ರ-ದಲ್ಲಿ
ಆಕಾ-ರ-ವನ್ನು
ಆಕಾರವೂ
ಆಕಾಶ
ಆಕಾ-ಶ-ಕಾ-ಯ-ಗಳ
ಆಕಾಶದ
ಆಕಾ-ಶ-ದಂತೆ
ಆಕಾ-ಶ-ದಲ್ಲಿ
ಆಕಾ-ಶ-ದಿಂದ
ಆಕಾ-ಶ-ಭಾ-ವ-ದಲ್ಲಿ
ಆಕಾ-ಶ-ಯಾ-ನದ
ಆಕಾ-ಶ-ವನ್ನು
ಆಕಾ-ಶ-ವಾ-ಣಿಯ
ಆಕೃ-ತಿ-ಗಳ
ಆಕೃ-ತಿ-ಯಲ್ಲಿ
ಆಕೆ
ಆಕೆಗೆ
ಆಕೆಯ
ಆಕೆಯನ್ನು
ಆಕೆಯಿಂದ
ಆಕೆಯು
ಆಕೆಯೇ
ಆಕ್ರಂದ-ನ-ಗಳ
ಆಕ್ರಮಣ
ಆಕ್ರ-ಮ-ಣ-ಗ-ಳಿಂದಲ್ಲ
ಆಕ್ರ-ಮ-ಣ-ವನ್ನು
ಆಕ್ರ-ಮ-ಣ-ವಾ-ಯಿತು
ಆಕ್ರಮಿಸಿ
ಆಕ್ಸ್ಫರ್ಡಿನ
ಆಕ್ಸ್ಫರ್ಡ್ನಂಥ
ಆಗ
ಆಗಂತುಕ
ಆಗತಾನೇ
ಆಗದಂತೆ
ಆಗ-ದ-ವನ
ಆಗ-ದಿದ್ದರೂ
ಆಗ-ದಿದ್ದರೆ
ಆಗದು
ಆಗದೆ
ಆಗದೆಂದು
ಆಗದೇ
ಆಗಬಲ್ಲ
ಆಗ-ಬಲ್ಲಿರಿ
ಆಗ-ಬ-ಹುದು
ಆಗ-ಬಾ-ರದು
ಆಗ-ಬೇ-ಕಾ-ಗಿದೆ
ಆಗ-ಬೇ-ಕಾದ
ಆಗ-ಬೇ-ಕಾ-ದರೆ
ಆಗ-ಬೇ-ಕಾ-ದುದು
ಆಗಬೇಕು
ಆಗ-ಬೇ-ಕೆಂದಿದ್ದರೆ
ಆಗ-ಬೇ-ಕೆಂದಿ-ರು-ವಿರಾ
ಆಗ-ಬೇ-ಕೆಂದಿ-ರು-ವಿರಾ
ಆಗ-ಬೇ-ಕೆಂದಿಲ್ಲ
ಆಗ-ಬೇ-ಕೆಂದು
ಆಗ-ಮ-ನ-ಗಳ
ಆಗರ
ಆಗ-ರ-ಆ-ಕಾಶ
ಆಗರರೂ
ಆಗ-ರ-ವಾದ
ಆಗರವೇ
ಆಗಲಮ್ಮ
ಆಗ-ಲಾ-ರದು
ಆಗಲಿ
ಆಗಲಿಲ್ಲ
ಆಗಲು
ಆಗಲೆ
ಆಗಲೇ
ಆಗ-ಸ-ದಲ್ಲಿ
ಆಗ-ಸ-ವನ್ನು
ಆಗಸ್ಟ್
ಆಗಾಗ
ಆಗಾಗ್ಗೆ
ಆಗಿ
ಆಗಿಂದಾಗ್ಗೆ
ಆಗಿತ್ತು
ಆಗಿತ್ತು-ಎಂಬು-ದನ್ನು
ಆಗಿದೆ
ಆಗಿ-ದೆ-ಯಲ್ಲ
ಆಗಿದ್ದ
ಆಗಿದ್ದ-ನಲ್ಲವೆ
ಆಗಿದ್ದರು
ಆಗಿದ್ದರೂ
ಆಗಿದ್ದರೆ
ಆಗಿದ್ದ-ರೆಂದು
ಆಗಿದ್ದಳು
ಆಗಿದ್ದಾನೆ
ಆಗಿದ್ದಾ-ನೆ-ಭ-ಗ-ವದ್ಗೀತಾ
ಆಗಿದ್ದೀಯೇ
ಆಗಿದ್ದು
ಆಗಿದ್ದುವೇ
ಆಗಿನ
ಆಗಿ-ಬಿಟ್ಟಿದೆ
ಆಗಿ-ಬಿ-ಡುತ್ತದೆ
ಆಗಿಯೇ
ಆಗಿರ
ಆಗಿ-ರ-ದಿದ್ದ
ಆಗಿ-ರ-ಬ-ಹುದು
ಆಗಿ-ರ-ಬೇಕು
ಆಗಿ-ರ-ಲಿಲ್ಲ
ಆಗಿರಲು
ಆಗಿ-ರುತ್ತದೆ
ಆಗಿ-ರುತ್ತಾರೆ
ಆಗಿರುವ
ಆಗಿ-ರು-ವನೋ
ಆಗಿ-ರು-ವು-ದ-ರಿಂದ
ಆಗಿ-ರು-ವುದು
ಆಗಿಲ್ಲ
ಆಗಿವೆ
ಆಗಿಸಲು
ಆಗಿ-ಹೋ-ಗಿ-ರು-ವರು
ಆಗೀಗ
ಆಗುತ್ತ
ಆಗುತ್ತದೆ
ಆಗುತ್ತ-ಲಿದೆ
ಆಗುತ್ತ-ಲಿದ್ದಾರೆ
ಆಗುತ್ತಾನೆ
ಆಗುತ್ತಾರೆ
ಆಗುತ್ತಾ-ರೆಂದು
ಆಗುತ್ತಿತ್ತು
ಆಗುತ್ತಿ-ದೆಯೇ
ಆಗುತ್ತಿದ್ದೀರಿ
ಆಗುತ್ತಿ-ರ-ಲಿಲ್ಲ
ಆಗುತ್ತಿ-ರುತ್ತದೆ
ಆಗುತ್ತಿ-ರುವ
ಆಗುತ್ತಿಲ್ಲ
ಆಗುತ್ತಿಲ್ಲ-ವೆಂದು
ಆಗುತ್ತೀರಿ
ಆಗುತ್ತೇನೆ
ಆಗುವ
ಆಗು-ವಂಥ-ವು-ಗ-ಳಲ್ಲ
ಆಗುವಷ್ಟು
ಆಗುವಿರಿ
ಆಗು-ವು-ದ-ರಲ್ಲಿ
ಆಗು-ವು-ದಿಲ್ಲ
ಆಗು-ವು-ದಿಲ್ಲ-ವೆಂದ
ಆಗುವುದು
ಆಗು-ವು-ದೆಂದರೆ
ಆಗುವುದೇ
ಆಗು-ಹೋ-ಗು-ಗಳ
ಆಗು-ಹೋ-ಗು-ಗ-ಳನ್ನು
ಆಘಾತ
ಆಘಾ-ತ-ವನ್ನುಂಟು-ಮಾ-ಡಿದ
ಆಘಾ-ತ-ಗಳ
ಆಘಾ-ತ-ಗ-ಳನ್ನು
ಆಘಾ-ತ-ಗೊಂಡು
ಆಘಾ-ತ-ಗೊ-ಳಿ-ಸು-ವಂತೆ
ಆಘಾತದ
ಆಘಾ-ತ-ದಿಂದ
ಆಘಾ-ತ-ವನ್ನು
ಆಘಾ-ತ-ವನ್ನುಂಟು-ಮಾ-ಡಿದೆ
ಆಘಾ-ತ-ವನ್ನುಂಟು-ಮಾ-ಡು-ವ-ವರು
ಆಘಾ-ತ-ವನ್ನೂ
ಆಘಾ-ತ-ವಾ-ಗ-ದಂತೆ
ಆಘಾ-ತ-ವಾ-ಗುತ್ತವೆ
ಆಘಾ-ತ-ವಾ-ಗು-ವು-ದಿಲ್ಲ
ಆಘಾ-ತ-ವಾ-ಗು-ವು-ದೆಂಬು-ದನ್ನು
ಆಘಾ-ತ-ವಾದ
ಆಘಾ-ತ-ವಾ-ದರೆ
ಆಘಾ-ತ-ವಾ-ದಾಗ
ಆಘಾತವೇ
ಆಚ-ರ-ಣೆಗೆ
ಆಚ-ರ-ಣೆ-ಯಲ್ಲಿ
ಆಚ-ರ-ಣೆ-ಯಿಂದ
ಆಚ-ರಿ-ಸ-ಬೇ-ಕೆಂಬ
ಆಚರಿಸಿ
ಆಚ-ರಿ-ಸುತ್ತಾರೋ
ಆಚ-ರಿ-ಸು-ವಂತೆ
ಆಚಾರ
ಆಚಾ-ರ-ಗಳು
ಆಚಾ-ರ-ವಿ-ಚಾ-ರ-ಗ-ಳಲ್ಲಿ
ಆಚಾ-ರ-ವಿಲ್ಲ
ಆಚಾರ್ಯ-ಪು-ರು-ಷರು
ಆಚಾರ್ಯ-ಪು-ರು-ಷರೂ
ಆಚಾರ್ಯರು
ಆಚೀಚಿನ
ಆಚೆ
ಆಚೆಗೂ
ಆಚೆಗೆ
ಆಚ್ಛಾ-ದ-ನ-ಗ-ಳನ್ನು
ಆಜನ್ಮ-ದುಃಖಿ-ಗಳು
ಆಜೀವನ
ಆಜ್ಞಾ-ಪಿ-ಸಿ-ದರು
ಆಜ್ಞಾ-ಪಿ-ಸಿದ್ದರು
ಆಜ್ಞಾಪಿಸು
ಆಜ್ಞಾ-ಪಿ-ಸುತ್ತಾರೆ
ಆಜ್ಞೆ
ಆಜ್ಞೆಗೆ
ಆಜ್ಞೆಯನ್ನು
ಆಟ
ಆಟಂ
ಆಟಂಬಾಂಬನ್ನೂ
ಆಟಕ್ಕಾಗಿ
ಆಟಕ್ಕೆ
ಆಟ-ಗ-ಳಾ-ಗಲಿ
ಆಟಗಾರ
ಆಟ-ಗಾ-ರ-ನಾ-ಗಿದ್ದ-ನೆಂದೂ
ಆಟ-ಗಾ-ರ-ನೊಬ್ಬ
ಆಟ-ಗಾ-ರರ
ಆಟ-ಗಾ-ರ-ರಿ-ರಲಿ
ಆಟ-ಗಾ-ರರು
ಆಟದ
ಆಟದಲ್ಲಿ
ಆಟದಿಂದ
ಆಟ-ಪಾ-ಠ-ಗ-ಳಲ್ಲಿ
ಆಟವನ್ನು
ಆಟ-ವನ್ನೇನೋ
ಆಟ-ವಾ-ಗಿತ್ತು
ಆಟ-ವಾ-ಡುತ್ತಿದ್ದಾಗ
ಆಟ-ವಾ-ಡು-ವು-ದಿಲ್ಲ
ಆಟಿಕೆಯ
ಆಟೋ-ಟ-ಗ-ಳಲ್ಲೂ
ಆಟೋಟದ
ಆಡಂಬ-ರದ
ಆಡ-ತೊ-ಡ-ಗಿತು
ಆಡದೆ
ಆಡಬೇಕು
ಆಡಲು
ಆಡಳಿತ
ಆಡ-ಳಿ-ತದ
ಆಡ-ಳಿ-ತ-ದಲ್ಲಿ
ಆಡ-ಳಿ-ತ-ವನ್ನು
ಆಡಿ-ಕೊಳ್ಳು-ವ-ವರು
ಆಡಿದ
ಆಡಿಲ್ಲ
ಆಡಿಸಿ
ಆಡಿಸಿದ
ಆಡು
ಆಡುತ್ತ
ಆಡುತ್ತಾ
ಆಡುತ್ತಿದ್ದ
ಆಡುತ್ತಿದ್ದಾ-ರೆಂದೆ-ಣಿಸಿ
ಆಡುತ್ತಿ-ರು-ವಾ-ಗಲೇ
ಆಡುತ್ತಿ-ರು-ವುದು
ಆತ
ಆತಂಕ
ಆತಂಕ-ಕಾ-ರಿ-ಯಾದ
ಆತಂಕ-ಗ-ಳಲ್ಲೇ
ಆತಂಕ-ರೋ-ಗ-ರು-ಜಿ-ನ-ಗಳ
ಆತಂಕ-ವಿಲ್ಲದೇ
ಆತಂಕವೂ
ಆತನ
ಆತನಂದ
ಆತನನ್ನು
ಆತನನ್ನೇ
ಆತನಲ್ಲಿ
ಆತ-ನಲ್ಲಿದ್ದ
ಆತನಲ್ಲೇ
ಆತನಿಂದ
ಆತ-ನಿಂದಲೇ
ಆತ-ನಿಂದಾಗಿ
ಆತ-ನಿ-ಗಾದ
ಆತ-ನಿ-ಗಿಂತ
ಆತನಿಗೆ
ಆತ-ನಿದ್ದಾನೆ
ಆತ-ನಿದ್ದಾ-ನೆಂಬುದು
ಆತನು
ಆತನೂ
ಆತನೆಂದ
ಆತನೆಲ್ಲಿ
ಆತನೇ
ಆತನೊಂದು
ಆತ-ನೊಬ್ಬನೇ
ಆತ-ನೊ-ಲಿ-ದರೆ
ಆತುರ
ಆತು-ರ-ಗ-ಳಿಗೂ
ಆತು-ರ-ತೆ-ಯಿಂದ
ಆತುರದ
ಆತು-ರ-ದಲ್ಲಿ
ಆತು-ರ-ದ-ವರು
ಆತು-ರ-ದಿಂದ
ಆತ್ತ
ಆತ್ಮ
ಆತ್ಮ-ವಿಶ್ವಾ-ಸಕ್ಕೆ
ಆತ್ಮ-ಇ-ವು-ಗ-ಳಲ್ಲಿನ
ಆತ್ಮಕ್ಕೆ
ಆತ್ಮ-ಗ-ಳಿವೆ
ಆತ್ಮಗಳು
ಆತ್ಮಗುಣ
ಆತ್ಮ-ಗೌ-ರವ
ಆತ್ಮ-ಗೌ-ರ-ವಕ್ಕೆ
ಆತ್ಮ-ಗೌ-ರ-ವದ
ಆತ್ಮ-ಗೌ-ರ-ವ-ವನ್ನು
ಆತ್ಮಗ್ಲಾ-ನಿ-ಯನ್ನು
ಆತ್ಮ-ಘಾ-ತಕ
ಆತ್ಮ-ಘಾ-ತುಕ
ಆತ್ಮಜ್ಞಾನ
ಆತ್ಮ-ತತ್ತ್ವ-ವನ್ನು
ಆತ್ಮ-ತತ್ತ್ವವೇ
ಆತ್ಮತೃಪ್ತಿ
ಆತ್ಮ-ತೃಪ್ತಿ-ಪಟ್ಟು
ಆತ್ಮ-ತೃಪ್ತಿಯ
ಆತ್ಮದ
ಆತ್ಮನ
ಆತ್ಮನನ್ನು
ಆತ್ಮನಲ್ಲಿ
ಆತ್ಮ-ನಲ್ಲಿ-ಡುವ
ಆತ್ಮ-ನಿಕ್ಷೇ-ಪ-ಕಾರ್ಪಣ್ಯೇ
ಆತ್ಮ-ನಿ-ರು-ವನು
ಆತ್ಮಪ್ರ-ಶಂಸೆ-ಗಳ
ಆತ್ಮ-ಭಾ-ವನೆ
ಆತ್ಮ-ಮ-ತ-ವನ್ನು
ಆತ್ಮ-ರಕ್ಷಣೆ
ಆತ್ಮ-ರಕ್ಷ-ಣೆಗೆ
ಆತ್ಮರೂಪಿ
ಆತ್ಮವನ್ನು
ಆತ್ಮವಾದ
ಆತ್ಮ-ವಿ-ಕಾ-ಸ-ವನ್ನೂ
ಆತ್ಮವಿದೆ
ಆತ್ಮ-ವಿ-ಮರ್ಶೆ-ಗ-ಳಿಂದ
ಆತ್ಮ-ವಿ-ಮರ್ಶೆಯ
ಆತ್ಮ-ವಿಶ್ವಾಸ
ಆತ್ಮ-ವಿಶ್ವಾ-ಸಕ್ಕೆ
ಆತ್ಮ-ವಿಶ್ವಾ-ಸ-ತನ್ನಲ್ಲಿ
ಆತ್ಮ-ವಿಶ್ವಾ-ಸದ
ಆತ್ಮ-ವಿಶ್ವಾ-ಸ-ದಿಂದ
ಆತ್ಮ-ವಿಶ್ವಾ-ಸ-ವನ್ನಾ-ಗಲೀ
ಆತ್ಮ-ವಿಶ್ವಾ-ಸ-ವನ್ನು
ಆತ್ಮ-ವಿಶ್ವಾ-ಸ-ವೆಂದರೆ
ಆತ್ಮ-ವಿಶ್ವಾ-ಸವೇ
ಆತ್ಮ-ವಿಶ್ವಾ-ಸ-ಹೀ-ನತೆ
ಆತ್ಮ-ವಿಶ್ವಾ-ಸ-ಹೀ-ನ-ನಾಗಿ
ಆತ್ಮವು
ಆತ್ಮವೆಂದು
ಆತ್ಮವೇ
ಆತ್ಮವೊಂದು
ಆತ್ಮ-ಶಕ್ತಿ-ಇ-ವು-ಗಳ
ಆತ್ಮ-ಶಕ್ತಿಯ
ಆತ್ಮ-ಶಕ್ತಿ-ಯನ್ನೂ
ಆತ್ಮಶ್ರದ್ಧೆ
ಆತ್ಮಶ್ರದ್ಧೆಗೂ
ಆತ್ಮಶ್ರದ್ಧೆಯ
ಆತ್ಮಶ್ರದ್ಧೆ-ಯನ್ನು
ಆತ್ಮಶ್ರದ್ಧೆ-ಯನ್ನುಂಟು-ಮಾಡಿ
ಆತ್ಮಶ್ರದ್ಧೆ-ಯಿಂದ
ಆತ್ಮಶ್ರದ್ಧೆಯೇ
ಆತ್ಮಶ್ಲಾ-ಘ-ನೆ-ಯನ್ನೆಂದೂ
ಆತ್ಮ-ಸಂಯಮ
ಆತ್ಮ-ಸಂಯ-ಮಈ
ಆತ್ಮ-ಸ-ಮರ್ಪಣ
ಆತ್ಮ-ಸಾಕ್ಷಾತ್ಕಾರ
ಆತ್ಮ-ಸಾಕ್ಷಾತ್ಕಾ-ರದ
ಆತ್ಮಹತ್ಯೆ
ಆತ್ಮ-ಹತ್ಯೆಯ
ಆತ್ಮ-ಹತ್ಯೆ-ಯಂಥ
ಆತ್ಮ-ಹತ್ಯೆ-ಯನ್ನು
ಆತ್ಮಾನಂ
ಆತ್ಮಾನು
ಆತ್ಮಾ-ನು-ಭೂ-ತಿ-ಯನ್ನು
ಆತ್ಮಾನ್ವೇ-ಷ-ಣೆ-ಯಲ್ಲಿ
ಆತ್ಮಾ-ಭಿ-ಮಾನ
ಆತ್ಮಾ-ಭಿ-ಮಾ-ನಕ್ಕೆ
ಆತ್ಮಾ-ಭಿ-ಮು-ಖಿ-ಯಾಗಿ
ಆತ್ಮೀಯ
ಆತ್ಮೀಯತೆ
ಆತ್ಮೀ-ಯ-ತೆ-ಯಿಂದ
ಆತ್ಮೀ-ಯ-ನೊಬ್ಬ
ಆತ್ಮೀ-ಯ-ರಾ-ಗಲಿ
ಆತ್ಮೋದ್ಧಾರ
ಆತ್ಯಂತಿಕ
ಆತ್ಯುತ್ತಮ
ಆದ
ಆದನಾತ
ಆದಂತೆ
ಆದದ್ದು
ಆದದ್ದೇ
ಆದದ್ದೇನು
ಆದರ
ಆದ-ರ-ಗ-ಳನ್ನು
ಆದ-ರ-ಣೀ-ಯ-ರೆಂದು
ಆದ-ರ-ಣೆಯೂ
ಆದ-ರ-ದಿಂದ
ಆದ-ರಿ-ಸು-ವನೋ
ಆದರು
ಆದರೂ
ಆದರೆ
ಆದರ್ಶ
ಆದರ್ಶ-ಜೀ-ವನ
ಆದರ್ಶಕ್ಕಾಗಿ
ಆದರ್ಶಕ್ಕೆ
ಆದರ್ಶ-ಗಳ
ಆದರ್ಶ-ಗ-ಳನ್ನು
ಆದರ್ಶ-ಗ-ಳಲ್ಲವೆ
ಆದರ್ಶ-ಗಳು
ಆದರ್ಶದ
ಆದರ್ಶ-ದಲ್ಲಿ
ಆದರ್ಶ-ದಿಂದ
ಆದರ್ಶ-ಪಥ
ಆದರ್ಶ-ವನ್ನು
ಆದರ್ಶ-ವನ್ನೂ
ಆದರ್ಶ-ವನ್ನೊ
ಆದರ್ಶ-ವಾ-ಗ-ಬೇಕು
ಆದರ್ಶ-ವಿದ್ದ-ರೇನೆ
ಆದರ್ಶವೂ
ಆದರ್ಶ-ವೆಂದರೆ
ಆದಳು
ಆದ-ವ-ನೊಬ್ಬ
ಆದಷ್ಟು
ಆದಷ್ಟೂ
ಆದಾಗ್ಯೂ
ಆದಾಯ
ಆದಾಯಕ್ಕೆ
ಆದಾ-ಯ-ವನ್ನು
ಆದಾ-ಯ-ವನ್ನೇ
ಆದಿ
ಆದಿಕವಿ
ಆದಿ-ಭಾ-ಗ-ದಿಂದಲೇ
ಆದಿಮ
ಆದಿಮೂಲ
ಆದಿ-ಯಿಂದಲೇ
ಆದೀತೇ
ಆದುದ
ಆದು-ದ-ರಿಂದ
ಆದು-ದ-ರಿಂದಲೆ
ಆದು-ದ-ರಿಂದಲೇ
ಆದುದು
ಆದೆ
ಆದೇಶ
ಆದೇ-ಶಕ್ಕೊ-ಳ-ಪಟ್ಟು
ಆದೇ-ಶ-ಗಳು
ಆದೇ-ಶ-ದಂತೆ
ಆದೇ-ಶ-ವನ್ನು
ಆದ್ದರಿಂದ
ಆದ್ದ-ರಿಂದಲೇ
ಆದ್ಯಂತ-ರ-ಹಿ-ತ-ವಾದ
ಆದ್ಯತೆ
ಆಧರಿಸಿ
ಆಧ-ರಿ-ಸಿ-ಕೊಂಡಿಲ್ಲ-ವಷ್ಟೆ
ಆಧ-ರಿ-ಸಿದ
ಆಧ-ರಿ-ಸಿದೆ
ಆಧ-ರಿ-ಸಿ-ರ-ಬೇಕು
ಆಧ-ರಿ-ಸಿವೆ
ಆಧಾರ
ಆಧಾ-ರ-ಗಳ
ಆಧಾ-ರ-ಗ-ಳಿಂದ
ಆಧಾರದ
ಆಧಾ-ರ-ದಿಂದ
ಆಧಾ-ರ-ದಿಂದಲೇ
ಆಧಾ-ರ-ವನ್ನು
ಆಧಾ-ರ-ವಾಗಿ
ಆಧಾ-ರ-ವಾದ
ಆಧಾ-ರ-ವಿ-ರದೆ
ಆಧಾರವೂ
ಆಧಾ-ರ-ವೇನು
ಆಧಾ-ರಸ್ತಂಭ-ಗ-ಳಾದ
ಆಧಾ-ರಸ್ತಂಭ-ಗ-ಳೆಂದರೂ
ಆಧಿಕ್ಯ
ಆಧುನಿಕ
ಆಧು-ನಿ-ಕ-ತೆಯ
ಆಧು-ನಿ-ಕರ
ಆಧು-ನಿ-ಕ-ರಲ್ಲಿ
ಆಧು-ನಿ-ಕರೇ
ಆಧು-ನೀ-ಕ-ರ-ಣ-ಗಳ
ಆಧ್ಯಾತ್ಮಿಕ
ಆಧ್ಯಾತ್ಮಿ-ಕತೆ
ಆಧ್ಯಾತ್ಮಿ-ಕ-ತೆ-ಗಿಂತಲೂ
ಆಧ್ಯಾತ್ಮಿ-ಕ-ತೆಯ
ಆಧ್ಯಾತ್ಮಿ-ಕ-ಪ-ಥ-ದಲ್ಲಿ
ಆಧ್ಯಾತ್ಮಿ-ಕ-ವಾಗಿ
ಆಧ್ಯಾತ್ಮಿ-ಕಸ್ಥಿತಿ
ಆನಂತ-ರದ
ಆನಂದ
ಆನಂದ-ಇ-ವು-ಗಳ
ಆನಂದ-ಗಳ
ಆನಂದ-ಗ-ಳನ್ನು
ಆನಂದದ
ಆನಂದ-ದಿಂದ
ಆನಂದ-ದಿಂದಲೇ
ಆನಂದ-ದೊಂದಿಗೆ
ಆನಂದ-ಪ-ಡುತ್ತಿದ್ದ
ಆನಂದ-ಪೂ-ರಿತ
ಆನಂದಪ್ರಾಪ್ತಿ
ಆನಂದ-ಭ-ರಿ-ತರೂ
ಆನಂದ-ವನ್ನಾ-ಗಲೀ
ಆನಂದ-ವನ್ನು
ಆನಂದ-ವನ್ನುಂಟು-ಮಾ-ಡು-ವುದು
ಆನಂದ-ವಿದೆ
ಆನಂದ-ವುಂಟಾ-ಗುತ್ತದೆ
ಆನಂದವೂ
ಆನಂದವೇ
ಆನಂದಸ್ವಾ-ಮಿ-ಯಾ-ಗಿ-ರುವ
ಆನಂದಾ-ತಿ-ರೇ-ಕ-ದಿಂದ
ಆನಂದಿ-ಸ-ಬೇ-ಕು-ದುಃಖಿ-ಯಾ-ಗ-ಬಾ-ರದು
ಆನಂದಿ-ಸ-ಬಲ್ಲಿರಾ
ಆನಂದಿ-ಸಲು
ಆನಂದಿ-ಸು-ವು-ದಿಲ್ಲ
ಆನರ್ಸ್
ಆನಿಕೆ
ಆನಿ-ಕೆ-ಗ-ಳನ್ನೊ-ದ-ಗಿ-ಸಿ-ದರೆ
ಆನು-ಕೂಲ್ಯಸ್ಯ
ಆನು-ಮಾ-ನಿಕ
ಆನು-ವಂಶೀ-ಯತೆ
ಆನೆ
ಆನೆಯನ್ನು
ಆನೆಯೊಂದು
ಆಪತ್ತಿ-ನಿಂದ
ಆಪತ್ತು
ಆಪ-ರೇ-ಶನ್ನು-ಗ-ಳನ್ನು
ಆಪ-ರೇ-ಷನ್
ಆಪಾದನೆ
ಆಪಾ-ದ-ನೆಯೇ
ಆಪ್ತ
ಆಪ್ತತೆ
ಆಪ್ತನೂ
ಆಪ್ತ-ರನ್ನಾಗಿ
ಆಪ್ತರನ್ನು
ಆಪ್ತರು
ಆಪ್ತರೆಂದು
ಆಪ್ತವಾಕ್ಯ
ಆಪ್ತ-ವಾಕ್ಯ-ಗ-ಳನ್ನು
ಆಪ್ತೇಷ್ಟ-ರಿಗೆ
ಆಪ್ಯಾ-ಯ-ಮಾನ
ಆಪ್ಯಾ-ಯ-ಮಾ-ನ-ವೆ-ನಿ-ಸಿ-ಯಾವೇ
ಆಪ್ಯಾ-ಯ-ಮಾ-ನ-ವೆ-ನಿ-ಸೀತೇ
ಆಫೀಸನ್ನು
ಆಫೀ-ಸ-ರನ್ನಾಗಿ
ಆಫೀ-ಸ-ರನ್ನು
ಆಫೀ-ಸ-ರ-ರೊಂದಿಗೆ
ಆಫೀ-ಸ-ರ-ರೊ-ಡನೆ
ಆಫೀಸರ್
ಆಫೀಸಿ
ಆಫೀಸಿಗೆ
ಆಫೀಸಿನ
ಆಫೀ-ಸಿ-ನಲ್ಲಿ
ಆಫೀ-ಸಿ-ನಲ್ಲಿದ್ದ
ಆಫೀ-ಸಿ-ನಲ್ಲಿಯೂ
ಆಫೀ-ಸಿ-ನಲ್ಲಿ-ರು-ವಂತೆ
ಆಫೀ-ಸಿ-ನಿಂದ
ಆಫೀಸು
ಆಫ್
ಆಫ್ರಿಕದ
ಆಫ್ರಿ-ಕ-ದ-ವರು
ಆಫ್ರಿಕಾ
ಆಫ್ರಿಕಾದ
ಆಬಾಲ
ಆಭ-ರ-ಣ-ಭೂ-ಷಿ-ತೆ-ಯಾಗಿ
ಆಭಾಸ
ಆಭಾ-ಸ-ವಾ-ಗ-ಬ-ಹು-ದಲ್ಲವೇ
ಆಮಂತ್ರಣ
ಆಮಂತ್ರ-ಣದ
ಆಮಂತ್ರ-ಣ-ವನ್ನಿತ್ತ
ಆಮಂತ್ರಿ-ಸಿ-ಕೊಳ್ಳುತ್ತಾನೆ
ಆಮಂತ್ರಿ-ಸಿದ
ಆಮಂತ್ರಿ-ಸಿ-ದಂತೆ
ಆಮದು
ಆಮೂಲಾಗ್ರ
ಆಮೂ-ಲಾಗ್ರ-ವಾಗಿ
ಆಮೇಲೆ
ಆಮ್ಲ-ಜ-ನ-ಕ-ವನ್ನು
ಆಯಾ
ಆಯಾನ್
ಆಯಾಮ
ಆಯಾ-ಮ-ಗ-ಳನ್ನು
ಆಯಾ-ಮ-ವಿದೆ
ಆಯಾಯ
ಆಯಾಸ
ಆಯಾ-ಸ-ಗೊಂಡು
ಆಯಾ-ಸ-ದಿಂದ
ಆಯಾ-ಸ-ವಾಗಿ
ಆಯಿತು
ಆಯು-ಧ-ಗ-ಳಿಂದ
ಆಯು-ರಾ-ರೋಗ್ಯ
ಆಯುಷ್ಯ-ವನ್ನು
ಆಯುಸ್ಸಿ-ನಲ್ಲಿ
ಆಯುಸ್ಸು
ಆಯುಸ್ಸೂ
ಆಯ್ಕೆ
ಆಯ್ಕೆ-ಮಾ-ಡಿ-ಕೊಂಡ
ಆರ
ಆರಂಭ
ಆರಂಭ-ವಾ-ಯಿತು
ಆರಂಭಕ್ಕಾಗಿ
ಆರಂಭದ
ಆರಂಭ-ದಲ್ಲಿ
ಆರಂಭ-ವಾ-ಗಿದೆ
ಆರಂಭ-ವಾ-ಗುತ್ತದೆ
ಆರಂಭ-ವಾ-ಗುತ್ತಿತ್ತು
ಆರಂಭ-ವಾ-ಗು-ವು-ದಕ್ಕೆ
ಆರಂಭ-ವಾ-ಗು-ವುವು
ಆರಂಭ-ವಾ-ದದ್ದು
ಆರಂಭ-ವಾ-ಯಿತು
ಆರಂಭ-ಶೂ-ರ-ರಾದ
ಆರಂಭಿಸಿ
ಆರಂಭಿ-ಸಿ-ದರೆ
ಆರಂಭಿ-ಸಿದ
ಆರಂಭಿ-ಸಿ-ದರು
ಆರಂಭಿ-ಸಿ-ದವು
ಆರಂಭಿ-ಸಿ-ದಾಗ
ಆರಂಭಿ-ಸಿ-ದಿರಿ
ಆರಂಭಿ-ಸಿಲ್ಲವೇ
ಆರಂಭಿ-ಸುತ್ತದೆ
ಆರಂಭಿ-ಸುತ್ತಾರೆ
ಆರಕ್ಷ-ಕರ
ಆರ-ಡಿ-ಯಷ್ಟು
ಆರತಿ
ಆರ-ನೆ-ಯದೇ
ಆರನೇ
ಆರ-ರ-ವ-ರೆಗೆ
ಆರವು
ಆರಾ-ಧ-ಕ-ರಾದ
ಆರಾಮ
ಆರಿಸಿ
ಆರಿಸಿಕೊ
ಆರಿ-ಸಿ-ಕೊಂಡ
ಆರಿ-ಸಿ-ಕೊಳ್ಳಲಿ
ಆರಿ-ಸಿ-ಕೊಳ್ಳ-ಲಿಲ್ಲ
ಆರಿ-ಸಿ-ಕೊಳ್ಳಿ
ಆರಿ-ಸಿ-ಕೊಳ್ಳುತ್ತಿದ್ದವು
ಆರಿ-ಸಿ-ಕೊಳ್ಳುವ
ಆರಿಸಿದ
ಆರಿ-ಹೋ-ದಾಗ
ಆರು
ಆರುನೂರು
ಆರು-ಬಿತ್ತಿ-ದಂತೆ
ಆರೇಳು
ಆರೈಕೆ
ಆರೈ-ಕೆ-ಗೊ-ಳ-ಗಾದ
ಆರೈ-ಕೆ-ಯನ್ನು
ಆರೈ-ಕೆ-ಯನ್ನೂ
ಆರೋಗ್ಯ
ಆರೋಗ್ಯ-ಕರ
ಆರೋಗ್ಯ-ಕ-ರ-ವಾದ
ಆರೋಗ್ಯ-ಕಾರಿ
ಆರೋಗ್ಯಕ್ಕೆ
ಆರೋಗ್ಯದ
ಆರೋಗ್ಯ-ದಿಂದ
ಆರೋಗ್ಯ-ದೊಂದಿಗೆ
ಆರೋಗ್ಯ-ರಕ್ಷಣೆ
ಆರೋಗ್ಯ-ಲಾ-ಭಕ್ಕಾಗಿ
ಆರೋಗ್ಯ-ವಂತ
ಆರೋಗ್ಯ-ವಂತನ
ಆರೋಗ್ಯ-ವಂತ-ನಾದ
ಆರೋಗ್ಯ-ವಂತ-ರಾ-ಗು-ವಂತೆ
ಆರೋಗ್ಯ-ವಂತರೂ
ಆರೋಗ್ಯ-ವನ್ನಾ-ಗಲೀ
ಆರೋಗ್ಯ-ವನ್ನು
ಆರೋಗ್ಯ-ವರ್ಧಕ
ಆರೋಗ್ಯ-ವಾ-ಗಿದ್ದೇನೆ
ಆರೋಗ್ಯ-ವಾದ
ಆರೋಗ್ಯ-ವಿ-ರುವ
ಆರೋಗ್ಯವೂ
ಆರೋ-ಪಿ-ಸುತ್ತಾರೆ
ಆರ್
ಆರ್ಕಿ-ಮಿ-ಡಿಸ್
ಆರ್ಕಿ-ಮಿ-ಡೀಸ್ನು
ಆರ್ಜನೆ
ಆರ್ಜಿಸಲು
ಆರ್ಜಿಸಿದ
ಆರ್ಟ್
ಆರ್ಡರ್
ಆರ್ತನಾದ
ಆರ್ತ-ನಾ-ದಕ್ಕೆ
ಆರ್ತ-ನಾ-ದ-ವನ್ನೂ
ಆರ್ತ-ರಾ-ದರು
ಆರ್ತರಿಗೆ
ಆರ್ಥರ್
ಆರ್ಥಿಕ
ಆರ್ಥಿ-ಕ-ವಾಗಿ
ಆರ್ಥಿ-ಕ-ವಾ-ಗಿಯೂ
ಆರ್ಥಿ-ಕಸ್ಥಿತಿ
ಆರ್ನಾಲ್ಡ್
ಆರ್ಷೇಯ
ಆಲಂಬ-ನೆ-ಗ-ಳಲ್ಲಿ
ಆಲಂಬ-ನೆ-ಗಳು
ಆಲದ
ಆಲ-ಯ-ದಲ್ಲಿ
ಆಲ-ಸಿ-ಗಳೂ
ಆಲಸ್ಯ
ಆಲಸ್ಯಕ್ಕೆ
ಆಲಸ್ಯದ
ಆಲಸ್ಯ-ವನ್ನೇ
ಆಲಾಪ
ಆಲಾ-ಪ-ವಲ್ಲ
ಆಲಾ-ಪ-ವಿಲ್ಲ
ಆಲಿ-ಸ-ಬ-ಹುದು
ಆಲಿಸಿ
ಆಲಿ-ಸಿ-ದರು
ಆಲಿ-ಸುತ್ತಿದ್ದರು
ಆಲೋ-ಚ-ನೆ-ಯಿಂದ
ಆಲ್ಡಸ್
ಆಲ್ಫ್ರೆಡ್
ಆಲ್ಬರ್ಟ್
ಆಲ್ವ
ಆಲ್ವಾರಿಸ್
ಆಳ
ಆಳ-ಅಂತ-ರಾ-ಳ-ದಲ್ಲಿ
ಆಳ-ಆ-ಳಕ್ಕೆ
ಆಳಕ್ಕಿ-ಳಿದು
ಆಳಕ್ಕೂ
ಆಳಕ್ಕೆ
ಆಳ-ಚಿಂತ-ನೆಯ
ಆಳದ
ಆಳ-ದಲ್ಲಂತೂ
ಆಳ-ದಲ್ಲ-ಡ-ಗಿ-ರುವ
ಆಳದಲ್ಲಿ
ಆಳ-ದಲ್ಲಿಯೇ
ಆಳ-ದಲ್ಲಿ-ರುವ
ಆಳ-ದಲ್ಲಿವೆ
ಆಳದಲ್ಲೇ
ಆಳದಿಂದ
ಆಳ-ದಿಂದಲೂ
ಆಳ-ನಿ-ರಾ-ಳ-ವನ್ನು
ಆಳನ್ನು
ಆಳಪ್ರ-ದೇ-ಶಕ್ಕೆ
ಆಳಲನ್ನು
ಆಳವನ್ನು
ಆಳವನ್ನೂ
ಆಳವಾಗಿ
ಆಳವಾದ
ಆಳ-ವಾ-ದಂತೆ
ಆಳವೂ
ಆಳಾಗದೆ
ಆಳಿದ
ಆಳು-ಕಾ-ಳು-ಗ-ಳಿಗೆ
ಆಳುತ್ತಿದ್ದ
ಆಳುತ್ತಿ-ರುತ್ತದೆ
ಆಳುತ್ತಿವೆ
ಆಳುವ
ಆಳು-ವ-ವ-ನೊಬ್ಬ
ಆಳು-ವ-ವರು
ಆಳುವುದು
ಆಳ್ವಿಕೆಯ
ಆಳ್ವಿ-ಕೆ-ಯಲ್ಲಿ
ಆಳ್ವಿ-ಕೆ-ಯಿಂದ
ಆವ-ರ-ಣ-ಗಳ
ಆವ-ರ-ಣ-ದಲ್ಲಿ
ಆವ-ರ-ಣ-ವಿದ್ದಿ-ರ-ಲಿಲ್ಲ
ಆವ-ರಿ-ಸ-ತೊ-ಡ-ಗಿತು
ಆವರಿಸಿ
ಆವ-ರಿ-ಸಿ-ಕೊಂಡಿ-ರುವ
ಆವ-ರಿ-ಸಿ-ಕೊಳ್ಳ-ಬೇಕು
ಆವ-ರಿ-ಸಿತ್ತು
ಆವ-ರಿ-ಸಿ-ದಂತಾ-ಗುತ್ತದೆ
ಆವ-ರಿ-ಸಿ-ದಾಗ
ಆವ-ರಿ-ಸುತ್ತದೆ
ಆವ-ರಿ-ಸುತ್ತಿತ್ತು
ಆವರ್ತ-ನದ
ಆವರ್ತನೆ
ಆವಶ್ಯಕ
ಆವಶ್ಯ-ಕತೆ
ಆವಶ್ಯ-ಕ-ತೆ-ಗಳ
ಆವಶ್ಯ-ಕ-ತೆ-ಗ-ಳನ್ನು
ಆವಶ್ಯ-ಕ-ತೆ-ಗ-ಳನ್ನೂ
ಆವಶ್ಯ-ಕ-ತೆ-ಗ-ಳಲ್ಲಿ
ಆವಶ್ಯ-ಕ-ತೆ-ಗಳೂ
ಆವಶ್ಯ-ಕ-ತೆಯ
ಆವಶ್ಯ-ಕ-ತೆ-ಯನ್ನು
ಆವಶ್ಯ-ಕ-ತೆ-ಯನ್ನೂ
ಆವಶ್ಯ-ಕ-ತೆ-ಯನ್ನೇ
ಆವಶ್ಯ-ಕ-ವಲ್ಲ
ಆವಶ್ಯ-ಕ-ವಲ್ಲವೆ
ಆವಶ್ಯ-ಕ-ವಾ-ಗಿದೆ
ಆವಶ್ಯ-ಕ-ವಾ-ಗಿ-ರ-ಲಿಲ್ಲ
ಆವಶ್ಯ-ಕ-ವಾದ
ಆವಶ್ಯ-ಕ-ವಾ-ದರೂ
ಆವಶ್ಯ-ಕ-ವಾ-ದುದು
ಆವಶ್ಯ-ಕ-ವಿ-ರ-ಬ-ಹುದು
ಆವಶ್ಯ-ಕವೂ
ಆವಶ್ಯ-ಕ-ವೆಂದು
ಆವಶ್ಯ-ಕವೇ
ಆವಾಸ
ಆವಾಸಿಕ
ಆವಾಹನೆ
ಆವಾ-ಹ-ನೆಯ
ಆವಾ-ಹ-ನೆ-ಯನ್ನು
ಆವಾ-ಹಿ-ತ-ನಾ-ಗಿದ್ದಾನೆ
ಆವಾ-ಹಿ-ತ-ನಾ-ಗು-ವಂಥ
ಆವಾ-ಹಿ-ತ-ನಾದ
ಆವಾ-ಹಿ-ತ-ವಾ-ಗಿ-ರುತ್ತಿತ್ತು
ಆವಾ-ಹಿ-ತ-ವಾ-ದಾಗ
ಆವಿ-ಯಾ-ಗಿ-ಸುತ್ತವೆ
ಆವಿ-ಯಾ-ದರೆ
ಆವಿರ್ಭ-ವಿ-ಸುವ
ಆವಿರ್ಭಾವ
ಆವಿರ್ಭಾ-ವಕ್ಕೇ
ಆವಿರ್ಭಾ-ವ-ಗಳು
ಆವಿರ್ಭಾ-ವ-ವಾ-ಗುವ
ಆವಿರ್ಭಾ-ವವೇ
ಆವಿಷ್ಕ-ರಿ-ಸಿದ್ದೇನೆ
ಆವಿಷ್ಕಾರ
ಆವಿಷ್ಕಾ-ರಕ್ಕೆ
ಆವಿಷ್ಕಾ-ರ-ಗಳ
ಆವಿಷ್ಕಾ-ರ-ಗಳು
ಆವಿಷ್ಕಾ-ರ-ದಿಂದ
ಆವಿಷ್ಕಾ-ರ-ವನ್ನು
ಆವೃ-ತ-ರಾಗಿ
ಆವೃತ್ತಿ-ಯನ್ನು
ಆವೃತ್ತಿ-ಯಾಗಿ
ಆವೇಗ
ಆವೇಶ
ಆಶಾ
ಆಶಾ-ದಾ-ಯ-ಕ-ವಾ-ಗಿ-ವೆಯೇ
ಆಶಾ-ದಾ-ಯ-ಕ-ವಾ-ಗಿ-ದೆಯೇ
ಆಶಾಭಂಗ
ಆಶಾ-ಭಂಗ-ವಾ-ಗು-ವು-ದಿಲ್ಲವೇ
ಆಶಾ-ಭಾ-ವ-ನೆ-ಯಿಂದ
ಆಶಾವಾದ
ಆಶಾ-ವಾ-ದಿ-ಗ-ಳಾದ
ಆಶಾ-ವಾ-ದಿ-ಗಳು
ಆಶಾ-ವಾ-ದಿ-ಯಾ-ಗ-ಬೇಕು
ಆಶಾ-ವಾ-ದಿ-ಯಾಗಿ
ಆಶಿ-ಸ-ದಿ-ರು-ವು-ದನ್ನು
ಆಶಿ-ಸ-ಬ-ಹುದು
ಆಶಿ-ಸ-ಲಾ-ರನೇ
ಆಶಿಸಿ
ಆಶಿ-ಸಿದ್ದೇನೆ
ಆಶಿ-ಸುತ್ತಾನೆ
ಆಶಿ-ಸುತ್ತೇವೆ
ಆಶಿ-ಸು-ವುದು
ಆಶೀರ್ವ-ಚ-ನ-ದಿಂದ
ಆಶೀರ್ವಾದ
ಆಶೀರ್ವಾ-ದ-ಗ-ಳನ್ನು
ಆಶೀರ್ವಾ-ದ-ಮಾಡು
ಆಶೆ
ಆಶೋತ್ತ-ರ-ಗ-ಳನ್ನು
ಆಶೋತ್ತ-ರ-ಗಳ
ಆಶ್ಚರ್ಯ
ಆಶ್ಚರ್ಯ-ವಾ-ಯಿತು
ಆಶ್ಚರ್ಯ-ಕರ
ಆಶ್ಚರ್ಯ-ಕ-ರ-ವಾಗಿ
ಆಶ್ಚರ್ಯ-ಕ-ರ-ವಾದ
ಆಶ್ಚರ್ಯ-ಕ-ರವೂ
ಆಶ್ಚರ್ಯ-ಗ-ಳಿಂದ
ಆಶ್ಚರ್ಯ-ಚ-ಕಿ-ತ-ರಾಗಿ
ಆಶ್ಚರ್ಯ-ಚ-ಕಿ-ತ-ರಾ-ಗುತ್ತೀ-ರಿ-ನಿ-ಮಗೆ
ಆಶ್ಚರ್ಯದ
ಆಶ್ಚರ್ಯ-ದಾ-ಯ-ಕವೂ
ಆಶ್ಚರ್ಯ-ದಿಂದ
ಆಶ್ಚರ್ಯ-ವಲ್ಲ
ಆಶ್ಚರ್ಯ-ವಾ-ದರೂ
ಆಶ್ಚರ್ಯ-ವಾ-ಯಿತು
ಆಶ್ಚರ್ಯ-ವಿಲ್ಲ
ಆಶ್ಚರ್ಯ-ವೆಂದರೆ
ಆಶ್ಚರ್ಯ-ವೆ-ನಿ-ಸದೆ
ಆಶ್ಚರ್ಯ-ವೆ-ನಿ-ಸಿ-ದರೂ
ಆಶ್ಚರ್ಯ-ವೆ-ನಿ-ಸುವ
ಆಶ್ಚರ್ಯ-ವೇ-ನಲ್ಲ
ಆಶ್ಚರ್ಯ-ವೇ-ನಿದೆ
ಆಶ್ಚರ್ಯ-ಸದಾ
ಆಶ್ರಮ
ಆಶ್ರಮಕ್ಕೆ
ಆಶ್ರಮದ
ಆಶ್ರ-ಮ-ದಲ್ಲಿ
ಆಶ್ರ-ಮ-ದಲ್ಲಿದ್ದಾಗ
ಆಶ್ರ-ಮ-ದಿಂದ
ಆಶ್ರಯ
ಆಶ್ರಯನೋ
ಆಶ್ರ-ಯ-ವಾದ
ಆಶ್ರ-ಯಸ್ಥಾ-ನ-ವಾದ
ಆಶ್ರಯಿಸಿ
ಆಶ್ರ-ಯಿ-ಸು-ವಿ-ರೇನು
ಆಶ್ವಾಸನೆ
ಆಷ್ಟನ್
ಆಸಕ್ತ-ನಾಗಿ
ಆಸಕ್ತ-ರಾ-ಗ-ಬೇಕು
ಆಸಕ್ತ-ರಾ-ಗಿದ್ದಾರೆ
ಆಸಕ್ತ-ರಾ-ಗುವ
ಆಸಕ್ತ-ರಾ-ದರು
ಆಸಕ್ತರು
ಆಸಕ್ತರೂ
ಆಸಕ್ತಿ
ಆಸಕ್ತಿ-ಗ-ಳನ್ನು
ಆಸಕ್ತಿ-ಗ-ಳನ್ನೂ
ಆಸಕ್ತಿ-ಗ-ಳಲ್ಲಿ
ಆಸಕ್ತಿ-ಗಳು
ಆಸಕ್ತಿ-ಯನ್ನು
ಆಸಕ್ತಿ-ಯಿಂದ
ಆಸಕ್ತಿ-ಯಿದ್ದ
ಆಸನ
ಆಸ-ನ-ಗ-ಳನ್ನೂ
ಆಸ-ನ-ದಲ್ಲೇ
ಆಸ-ನ-ವನ್ನು
ಆಸರೆ
ಆಸ-ರೆ-ಯಲ್ಲೇ
ಆಸ-ರೆ-ಯಾ-ಗ-ಬಲ್ಲರು
ಆಸು-ಪಾ-ಸಿನ
ಆಸುರೀ
ಆಸೆ
ಆಸೆ-ಗ-ಳನ್ನು
ಆಸೆ-ಗ-ಳಿಂದ
ಆಸೆಗೆ
ಆಸೆ-ಪ-ಡೆದ
ಆಸೆ-ಯನ್ನಷ್ಟೇ
ಆಸೆಯಿಂದ
ಆಸೆ-ಯಿಂದಲೇ
ಆಸೆಯು
ಆಸೆಯೂ
ಆಸೆಯೇ
ಆಸ್ಕರ್
ಆಸ್ಟ್ರೇಲಿಯಾ
ಆಸ್ಟ್ರೇ-ಲಿ-ಯಾ-ದ-ವರು
ಆಸ್ತಮಾ
ಆಸ್ತಮಾದ
ಆಸ್ತಿ
ಆಸ್ತಿಕನ
ಆಸ್ತಿಕರ
ಆಸ್ತಿ-ಕ-ರನ್ನೇ
ಆಸ್ತಿಕರು
ಆಸ್ತಿಪಾಸ್ತಿ
ಆಸ್ತಿ-ಪಾಸ್ತಿ-ಗ-ಳಿಂದ
ಆಸ್ತಿ-ಪಾಸ್ತಿ-ಗ-ಳಿಲ್ಲ
ಆಸ್ತಿಯನ್ನು
ಆಸ್ತಿ-ಯನ್ನೆಲ್ಲ
ಆಸ್ಥಾನಕ್ಕೆ
ಆಸ್ಥಾನದ
ಆಸ್ಥಾ-ನ-ದಲ್ಲಿ
ಆಸ್ಥಾ-ನ-ದಲ್ಲಿದ್ದ
ಆಸ್ಥಾ-ನ-ವೈದ್ಯ-ನಾ-ಗಿದ್ದ
ಆಸ್ಥಾ-ನಿ-ಕರ
ಆಸ್ಥೆ
ಆಸ್ಥೆಯನ್ನು
ಆಸ್ಥೆಯಿಂದ
ಆಸ್ಪತ್ರೆ
ಆಸ್ಪತ್ರೆ-ಗ-ಳನ್ನು
ಆಸ್ಪತ್ರೆ-ಗ-ಳಲ್ಲಿ
ಆಸ್ಪತ್ರೆ-ಗ-ಳಿ-ಗಾಗಿ
ಆಸ್ಪತ್ರೆ-ಗ-ಳಿವೆ
ಆಸ್ಪತ್ರೆಗೆ
ಆಸ್ಪತ್ರೆಯ
ಆಸ್ಪತ್ರೆ-ಯನ್ನು
ಆಸ್ಪತ್ರೆ-ಯಲ್ಲಿ
ಆಸ್ಪತ್ರೆ-ಯಲ್ಲಿದ್ದರು
ಆಸ್ಪ-ದ-ವಿದ್ದರೂ
ಆಸ್ವಾದನೆ
ಆಸ್ವಾ-ದ-ನೆ-ಯನ್ನು
ಆಹಾ
ಆಹಾರ
ಆಹಾ-ರ-ಇ-ವನ್ನು
ಆಹಾ-ರಕ್ಕಾಗಿ
ಆಹಾ-ರ-ಗಳ
ಆಹಾರದ
ಆಹಾ-ರ-ದಿಂದ
ಆಹಾ-ರ-ನೀ-ರು-ಗಳ
ಆಹಾ-ರ-ನೀ-ರು-ಗ-ಳಿಲ್ಲದೆ
ಆಹಾ-ರ-ವನ್ನು
ಆಹಾ-ರ-ವಾಗಿ
ಆಹಾ-ರ-ವಾ-ದರೆ
ಆಹಾ-ರ-ವಿ-ರ-ಲಿಲ್ಲ
ಆಹಾ-ರ-ವಿಲ್ಲ-ದಂತೆ
ಆಹಾ-ರ-ವಿಲ್ಲದೆ
ಆಹಾ-ರ-ಸತ್ವ-ವಾಗಿ
ಆಹಾ-ರಾ-ಭಾ-ವ-ದಿಂದಲೆ
ಆಹು-ತಿ-ಯಾ-ಗುತ್ತಿವೆ
ಆಹ್ವಾ-ನ-ಗ-ಳನ್ನಿತ್ತರೂ
ಆಹ್ವಾನದ
ಆಹ್ವಾ-ನ-ವೀ-ಯುತ್ತದೆ
ಆಹ್ವಾ-ನಿ-ತ-ರಾಗಿ
ಆಹ್ವಾ-ನಿ-ಸಲೂ
ಆಹ್ವಾ-ನಿ-ಸುತ್ತಾ
ಇ
ಇಂಕ್ವಾ-ಯ-ರಿಸ್
ಇಂಗಿತ
ಇಂಗಿ-ತ-ವಿದ್ದರೂ
ಇಂಗ್ಲಿಷಿನ
ಇಂಗ್ಲೀ-ಷಿ-ನಲ್ಲಿ
ಇಂಗ್ಲೀ-ಷಿ-ನಲ್ಲೇ
ಇಂಗ್ಲೀಷ್
ಇಂಗ್ಲೆಂಡಿಗೆ
ಇಂಗ್ಲೆಂಡಿನ
ಇಂಗ್ಲೆಂಡಿ-ನಲ್ಲಿ
ಇಂಗ್ಲೆಂಡಿ-ನಲ್ಲಿ-ರುವ
ಇಂಗ್ಲೆಂಡಿ-ನ-ವರು
ಇಂಗ್ಲೆಂಡು
ಇಂಗ್ಲೆಂಡ್
ಇಂಗ್ಲೆಂಡ್ನ
ಇಂಚು
ಇಂಜಿ-ನಿ-ಯ-ರಿಂಗ್
ಇಂಜಿ-ನಿ-ಯರ್
ಇಂಜಿನ್ನನ್ನು
ಇಂಜೆಕ್ಶನ್
ಇಂಟು
ಇಂಡಿಯದ
ಇಂಡಿ-ಯ-ನರ
ಇಂಡಿಯನ್
ಇಂಡಿಯನ್ಸ್
ಇಂಡಿಯಾ
ಇಂಡಿ-ಯಾ-ದಲ್ಲಿ
ಇಂಡಿ-ಯಾ-ದಲ್ಲಿದೆ
ಇಂಡಿಯಾನಾ
ಇಂತಹ
ಇಂತ-ಹ-ದ-ರಲ್ಲಿ
ಇಂತಿಂಥ
ಇಂತಿದೆ
ಇಂತಿ-ರು-ವಾಗ
ಇಂತಿವೆ
ಇಂತಿ-ವೆ-ದೇಶ
ಇಂತಿ-ವೆ-ಬುದ್ಧಿ-ವಂತರು
ಇಂತೆಂದರು
ಇಂಥ
ಇಂಥದು
ಇಂಥದ್ದಾ-ಗಿತ್ತು
ಇಂಥ-ವ-ನನ್ನು
ಇಂಥವನು
ಇಂಥವರ
ಇಂಥ-ವ-ರನ್ನು
ಇಂಥ-ವ-ರಲ್ಲಿ
ಇಂಥವರು
ಇಂಥವರೂ
ಇಂಥವರೇ
ಇಂದಿಂದೇ
ಇಂದಿಗೂ
ಇಂದಿಗೆ
ಇಂದಿನ
ಇಂದಿ-ನ-ವರು
ಇಂದಿ-ನ-ವ-ರೆಗೂ
ಇಂದಿ-ನ-ವ-ರೆಗೆ
ಇಂದಿ-ನಿಂದಲೇ
ಇಂದೀಗ
ಇಂದು
ಇಂದೂ
ಇಂದೆನಗೆ
ಇಂದೇ
ಇಂದೋ
ಇಂದ್ರಜಾಲ
ಇಂದ್ರ-ಜಾ-ಲ-ಗ-ಳನ್ನು
ಇಂದ್ರ-ಜಾ-ಲ-ವಲ್ಲ
ಇಂದ್ರಿಯ
ಇಂದ್ರಿ-ಯ-ಪ-ಟುತ್ವ
ಇಂದ್ರಿ-ಯ-ಗಳ
ಇಂದ್ರಿ-ಯ-ಗ-ಳನ್ನು
ಇಂದ್ರಿ-ಯ-ಗ-ಳನ್ನೂ
ಇಂದ್ರಿ-ಯ-ಗ-ಳಿಂದ
ಇಂದ್ರಿ-ಯ-ಗ-ಳಿಗೆ
ಇಂದ್ರಿ-ಯ-ಗಳು
ಇಂದ್ರಿ-ಯ-ಗ-ಳೆಲ್ಲ
ಇಂದ್ರಿ-ಯ-ಗೋ-ಚ-ರ-ವಾದ
ಇಂದ್ರಿ-ಯ-ಚ-ಪ-ಲ-ತೆ-ಯನ್ನು
ಇಂದ್ರಿ-ಯ-ನಿಗ್ರಹ
ಇಂದ್ರಿ-ಯ-ಪ-ರಾ-ಯ-ಣತೆ
ಇಂದ್ರಿ-ಯ-ಪ-ರಾ-ಯ-ಣ-ನಾಗಿ
ಇಂದ್ರಿ-ಯ-ಭೋ-ಗ-ವನ್ನೂ
ಇಂದ್ರಿ-ಯ-ಭೋ-ಗವೇ
ಇಂದ್ರಿ-ಯ-ಲೋ-ಲು-ಪ-ತೆ-ಗ-ಳಿ-ಗಾಗಿ
ಇಂದ್ರಿ-ಯ-ವನ್ನು
ಇಂದ್ರಿ-ಯ-ಸು-ಖವೇ
ಇಂದ್ರಿ-ಯಾ-ತೀತ
ಇಂದ್ರಿ-ಯಾ-ತೀತ
ಇಂದ್ರಿ-ಯಾ-ತೀ-ತ-ವಾದ
ಇಂಬು-ಗೊ-ಡದೆ
ಇಂಬು-ಕೊ-ಡ-ದಂತೆ
ಇಂಬು-ಕೊ-ಡುತ್ತ
ಇಂಬು-ಕೊ-ಡುವ
ಇಕ್ಕಟ್ಟಾದ
ಇಕ್ಕ-ಳ-ದಿಂದ
ಇಕ್ಕಳು
ಇಕ್ಕೆ-ಲ-ಗ-ಳಲ್ಲಿ
ಇಕ್ಕೆ-ಲ-ಗ-ಳಲ್ಲಿ-ರುವ
ಇಕ್ಷ್ವಾ-ಕು-ವಿಗೆ
ಇಗರ್ಜಿ
ಇಗ್ನೇ-ಶಿ-ಯಸ್
ಇಚ್ಛಾಕ್ರಿ-ಯಾಜ್ಞಾ-ನಾತ್ಮಕ
ಇಚ್ಛಾ-ಧೀ-ನ-ವಾಗಿ
ಇಚ್ಛಾಪತ್ರ
ಇಚ್ಛಾಶಕ್ತಿ
ಇಚ್ಛಾ-ಶಕ್ತಿಗೂ
ಇಚ್ಛಾ-ಶಕ್ತಿಗೆ
ಇಚ್ಛಾ-ಶಕ್ತಿಯ
ಇಚ್ಛಾ-ಶಕ್ತಿ-ಯನ್ನು
ಇಚ್ಛಾ-ಶಕ್ತಿ-ಯಿಂದಲೇ
ಇಚ್ಛಾ-ಶಕ್ತಿ-ಯು-ನ-ಡೆ-ಯಲು
ಇಚ್ಛಿಸದೆ
ಇಚ್ಛಿ-ಸಿ-ರ-ಬ-ಹುದು
ಇಚ್ಛಿ-ಸುತ್ತಿ-ರು-ವಂತೆಯೇ
ಇಚ್ಛಿ-ಸುತ್ತೇನೆ
ಇಚ್ಛಿ-ಸುತ್ತೇವೆ
ಇಚ್ಛಿ-ಸು-ವ-ವ-ರಲ್ಲಿ
ಇಚ್ಛಿ-ಸು-ವಿರಾ
ಇಚ್ಛಿ-ಸು-ವು-ದಿಲ್ಲ
ಇಚ್ಛೆ
ಇಚ್ಛೆಇವು
ಇಚ್ಛೆ-ಎಂಬುದು
ಇಚ್ಛೆಎಲ್ಲ
ಇಚ್ಛೆ-ಗ-ನು-ಗು-ಣ-ವಾ-ಗಿಯೇ
ಇಚ್ಛೆಗಳ
ಇಚ್ಛೆಗೆ
ಇಚ್ಛೆಯ
ಇಚ್ಛೆಯನ್ನು
ಇಚ್ಛೆಯಿಂದ
ಇಚ್ಛೆಯೇ
ಇಟ-ಲಿ-ಯನ್ನು
ಇಟಾ-ಲಿ-ಯನ್
ಇಟೆಲಿಗೆ
ಇಟೆಲಿಯ
ಇಟೆ-ಲಿ-ಯಲ್ಲಿ
ಇಟ್ಟಂತಾ-ಯಿ-ತಲ್ಲವೇ
ಇಟ್ಟರೆ
ಇಟ್ಟಾಗ
ಇಟ್ಟಿಗೆ
ಇಟ್ಟಿ-ಗೆ-ಗ-ಳಿಂದ
ಇಟ್ಟಿ-ಗೆ-ಗ-ಳಿಂದಲೇ
ಇಟ್ಟಿ-ಗೆ-ಗಳು
ಇಟ್ಟಿಗೆಯೂ
ಇಟ್ಟಿದ್ದ
ಇಟ್ಟಿದ್ದರು
ಇಟ್ಟಿದ್ದು
ಇಟ್ಟಿದ್ದೇನೆ
ಇಟ್ಟಿ-ರ-ಬೇಕು
ಇಟ್ಟಿರು
ಇಟ್ಟಿರುವ
ಇಟ್ಟು
ಇಟ್ಟು-ಕೊಂಡ-ವ-ರಲ್ಲ
ಇಟ್ಟು-ಕೊಂಡಿದೆ
ಇಟ್ಟು-ಕೊಂಡಿದ್ದ
ಇಟ್ಟು-ಕೊಂಡಿಲ್ಲ
ಇಟ್ಟು-ಕೊಂಡಿವೆ
ಇಟ್ಟುಕೊಂಡು
ಇಟ್ಟು-ಕೊಳ್ಳದೆ
ಇಟ್ಟು-ಹೋ-ಗಿ-ರುವ
ಇಡ-ಬಾ-ರದು
ಇಡ-ಬೇ-ಕಾದ
ಇಡಬೇಕು
ಇಡಲಾರ
ಇಡಿ
ಇಡಿಯ
ಇಡೀ
ಇಡು
ಇಡುತ್ತಲೇ
ಇಡುತ್ತಾ-ನೆಂಬು-ದಕ್ಕೆ
ಇಡುತ್ತಿದ್ದಾಗ
ಇಡು-ವು-ದನ್ನಾ-ಗಲೀ
ಇಡೋಣ
ಇಣುಕಿ
ಇಣು-ಕಿ-ದ-ನೆಂದರೆ
ಇಣು-ಕು-ನೋಟ
ಇತರ
ಇತ-ರ-ರೊ-ಡನೆ
ಇತರರ
ಇತ-ರ-ರದು
ಇತ-ರ-ರನ್ನು
ಇತ-ರ-ರನ್ನೂ
ಇತ-ರ-ರನ್ನೇ
ಇತ-ರ-ರನ್ನೋ
ಇತ-ರ-ರಲ್ಲಿ
ಇತ-ರ-ರಲ್ಲಿ-ರುವ
ಇತ-ರ-ರಲ್ಲೂ
ಇತ-ರ-ರಾ-ಗ-ಬೇಡಿ
ಇತ-ರ-ರಾ-ಗಲೀ
ಇತ-ರ-ರಿಂದ
ಇತ-ರ-ರಿ-ಗಾ-ಗಲೀ
ಇತ-ರ-ರಿ-ಗಾಗಿ
ಇತ-ರ-ರಿಗೂ
ಇತ-ರ-ರಿಗೆ
ಇತರರು
ಇತರರೂ
ಇತ-ರ-ರೆಲ್ಲ
ಇತರರೇ
ಇತ-ರ-ರೇನೋ
ಇತ-ರ-ರೊಂದಿಗೆ
ಇತಿಮಿತಿ
ಇತಿ-ಮಿ-ತಿಯೇ
ಇತಿಹಾಸ
ಇತಿ-ಹಾ-ಸ-ಅತಿ
ಇತಿ-ಹಾ-ಸ-ಕಾರ
ಇತಿ-ಹಾ-ಸ-ಕಾ-ರರು
ಇತಿ-ಹಾ-ಸಜ್ಞ
ಇತಿ-ಹಾ-ಸದ
ಇತಿ-ಹಾ-ಸ-ದಲ್ಲಿ
ಇತಿ-ಹಾ-ಸ-ದಲ್ಲೆ
ಇತಿ-ಹಾ-ಸ-ದಲ್ಲೆಲ್ಲಾ
ಇತಿ-ಹಾ-ಸ-ದಲ್ಲೇ
ಇತಿ-ಹಾ-ಸ-ದುದ್ದಕ್ಕೂ
ಇತಿ-ಹಾ-ಸ-ವನ್ನು
ಇತಿ-ಹಾ-ಸ-ವನ್ನೆಲ್ಲ
ಇತಿ-ಹಾ-ಸವು
ಇತ್ತ
ಇತ್ತಿದ್ದರು
ಇತ್ತೀಚಿನ
ಇತ್ತೀ-ಚಿ-ನದು
ಇತ್ತೀಚೆಗೆ
ಇತ್ತು
ಇತ್ತೆಂದರೆ
ಇತ್ತೇ
ಇತ್ಯರ್ಥ-ವಾ-ದಂತೆ
ಇತ್ಯಾದಿ
ಇದ
ಇದಕೆ
ಇದಕ್ಕಾಗಿ
ಇದಕ್ಕಾ-ಗಿಯೇ
ಇದಕ್ಕಿಂತ
ಇದಕ್ಕಿಂತಲೂ
ಇದಕ್ಕೂ
ಇದಕ್ಕೆ
ಇದಕ್ಕೆಲ್ಲ
ಇದಕ್ಕೇನು
ಇದನ್ನು
ಇದನ್ನೆಲ್ಲ
ಇದನ್ನೇ
ಇದರ
ಇದ-ರಂತೆಯೇ
ಇದರರ್ಥ
ಇದರಲ್ಲಿ
ಇದರಿಂದ
ಇದ-ರೊಂದಿಗೆ
ಇದಲ್ಲ
ಇದಲ್ಲದೆ
ಇದಾಗಿದೆ
ಇದಾದರೂ
ಇದಾದೀತು
ಇದಿ-ರಿ-ನಲ್ಲಿ
ಇದಿ-ರಿ-ಸಲು
ಇದಿ-ರಿ-ಸುವ
ಇದಿರು
ಇದಿ-ರು-ಬ-ರುತ್ತದೆ
ಇದಿರೇನೇ
ಇದೀಗ
ಇದು
ಇದು-ವ-ರೆಗೂ
ಇದು-ವ-ರೆಗೆ
ಇದೂ
ಇದೆ
ಇದೆ-ಯಲ್ಲವೇ
ಇದೆಂಥ
ಇದೆಕತೆ
ಇದೆ-ಯಲ್ಲವೆ
ಇದೆಯೆ
ಇದೆ-ಯೆಂದಾ-ದರೆ
ಇದೆಯೆಂದು
ಇದೆಯೇ
ಇದೆಲ್ಲ
ಇದೆಲ್ಲ-ದರ
ಇದೆಲ್ಲವು
ಇದೆಲ್ಲವೂ
ಇದೆಲ್ಲಾ
ಇದೇ
ಇದೇ-ನಾಶ್ಚರ್ಯ
ಇದೇನು
ಇದೊ
ಇದೊಂದಂತಲ್ಲ
ಇದೊಂದು
ಇದೊಂದೇ
ಇದೋ
ಇದ್ದ
ಇದ್ದಂತೆ
ಇದ್ದಕ್ಕಿದ್ದಂತೆ
ಇದ್ದರು
ಇದ್ದರೂ
ಇದ್ದರೆ
ಇದ್ದರೋ
ಇದ್ದಲಾಗಿ
ಇದ್ದಲ್ಲಿ
ಇದ್ದಳು
ಇದ್ದವನು
ಇದ್ದವರು
ಇದ್ದವು
ಇದ್ದಾಗ
ಇದ್ದಾನೆ
ಇದ್ದಾ-ನೆಂಬುದು
ಇದ್ದಾ-ನೆ-ಆತ
ಇದ್ದಾ-ನೆ-ಭ-ಗ-ವಾನ್
ಇದ್ದಾನೆಯೆ
ಇದ್ದಾರೆ
ಇದ್ದಾಳೆ
ಇದ್ದಿ-ರ-ಬ-ಹುದು
ಇದ್ದೀಯೆ
ಇದ್ದು
ಇದ್ದು-ಕೊಂಡಿತ್ತು
ಇದ್ದು-ಕೊಂಡಿದೆ
ಇದ್ದು-ಕೊಂಡಿ-ರುವ
ಇದ್ದುಕೊಂಡು
ಇದ್ದು-ದಕ್ಕಾಗಿ
ಇದ್ದುದನ್ನು
ಇದ್ದು-ದ-ರಲ್ಲಿ
ಇದ್ದು-ದ-ರಲ್ಲೇ
ಇದ್ದು-ದ-ರಿಂದ
ಇದ್ದುದಾಗಿ
ಇದ್ದು-ದಾ-ದರೆ
ಇದ್ದು-ಬಿ-ಡ-ಬ-ಹು-ದೆಂಬ
ಇದ್ದೇ
ಇದ್ದೇನೆ
ಇದ್ದೇವೆ
ಇದ್ರಿಸಿ
ಇನ್ನರ್ಧ
ಇನ್ನಷ್ಟು
ಇನ್ನಾದರೂ
ಇನ್ನಾರು
ಇನ್ನಾರೂ
ಇನ್ನಾವ
ಇನ್ನಿತರ
ಇನ್ನಿಬ್ಬರು
ಇನ್ನಿಲ್ಲ
ಇನ್ನಿಷ್ಟು
ಇನ್ನು
ಇನ್ನೂ
ಇನ್ನೂರ
ಇನ್ನೂ-ರ-ಮೂ-ವತ್ತ-ರ-ವ-ರೆಗೂ
ಇನ್ನೂ-ರ-ಮೂ-ವತ್ತ-ರ-ವ-ರೆಗೆ
ಇನ್ನೂರಲ್ಲಿ
ಇನ್ನೂರು
ಇನ್ನೆರಡು
ಇನ್ನೆಲ್ಲಿ
ಇನ್ನೇ-ನಾ-ದರೂ
ಇನ್ನೇನು
ಇನ್ನೊಂದನ್ನು
ಇನ್ನೊಂದ-ರಂತಿಲ್ಲ
ಇನ್ನೊಂದ-ರಲ್ಲಿ
ಇನ್ನೊಂದಿಲ್ಲ
ಇನ್ನೊಂದು
ಇನ್ನೊಂದೆಡೆ
ಇನ್ನೊಬ್ಬ
ಇನ್ನೊಬ್ಬನು
ಇನ್ನೊಬ್ಬನೂ
ಇನ್ನೊಬ್ಬರ
ಇನ್ನೊಬ್ಬ-ರನ್ನು
ಇನ್ನೊಬ್ಬ-ರಿಂದ
ಇನ್ನೊಬ್ಬ-ರಿಗೆ
ಇನ್ನೊಬ್ಬರು
ಇನ್ನೊಬ್ಬ-ರೆ-ದುರು
ಇನ್ನೊಬ್ಬಳು
ಇನ್ನೊಬ್ಬಾ-ಕೆಗೆ
ಇನ್ನೊಬ್ಬಾ-ತತ
ಇನ್ನೊಮ್ಮೆ
ಇನ್ಶೂರೆನ್ಸ್
ಇಪ್ಪ
ಇಪ್ಪತ್ತ
ಇಪ್ಪತ್ತಕ್ಕೂ
ಇಪ್ಪತ್ತ-ನಾಲ್ಕನೇ
ಇಪ್ಪತ್ತ-ನಾಲ್ಕು
ಇಪ್ಪತ್ತನೇ
ಇಪ್ಪತ್ತ-ಮೂರು
ಇಪ್ಪತ್ತ-ರಿಂದ
ಇಪ್ಪತ್ತ-ರೊ-ಳಗೇ
ಇಪ್ಪತ್ತು
ಇಪ್ಪತ್ತು-ಕೋಟಿ
ಇಪ್ಪತ್ತೆಂಟು
ಇಪ್ಪತ್ತೆ-ರಡು
ಇಪ್ಪತ್ತೇಳು
ಇಪ್ಪತ್ತೈ-ದನೆ
ಇಪ್ಪತ್ತೈದು
ಇಪ್ಪತ್ತೊಂದನೇ
ಇಪ್ಪತ್ತೊಂದು
ಇಬ್ಬರ
ಇಬ್ಬರಲ್ಲಿ
ಇಬ್ಬರಲ್ಲೂ
ಇಬ್ಬರಿಗೂ
ಇಬ್ಬರು
ಇಬ್ಬರೂ
ಇಬ್ಬರೇ
ಇಬ್ಬರೋ
ಇಬ್ಭಾ-ಗ-ವಾ-ಯಿತು
ಇಮೇಜಿಗೆ
ಇಮೇಜ್
ಇಮ್ಮಡಿ
ಇಮ್ಮ-ಡಿ-ಯಾ-ಗ-ಲಿದೆ
ಇಮ್ಮಾರ್ಟಲ್
ಇಯಾ-ಗೋ-ವಿ-ನಂತೆ
ಇರದ
ಇರ-ದಿದ್ದರೂ
ಇರದು
ಇರದೆ
ಇರಬಲ್ಲ
ಇರ-ಬಲ್ಲೆ-ವೇನು
ಇರ-ಬ-ಹುದು
ಇರ-ಬ-ಹುದೆ
ಇರ-ಬ-ಹು-ದೆಂದು
ಇರ-ಬ-ಹುದೇ
ಇರ-ಬೇ-ಕಾ-ಗುತ್ತಿತ್ತು
ಇರ-ಬೇ-ಕಾದ
ಇರ-ಬೇ-ಕಾ-ದಂತಹ
ಇರ-ಬೇ-ಕಾ-ದರೆ
ಇರ-ಬೇ-ಕಾ-ದುದು
ಇರಬೇಕು
ಇರಬೇಕೆ
ಇರ-ಬೇ-ಕೆಂದು
ಇರಲಾರ
ಇರ-ಲಾ-ರರು
ಇರ-ಲಾ-ರವು
ಇರಲಿ
ಇರಲಿಈ
ಇರಲಿಲ್ಲ
ಇರಲು
ಇರಲೆಂದು
ಇರ-ಲೇ-ಬೇಕೋ
ಇರವಿನ
ಇರಿ-ಟೇ-ಶನ್
ಇರಿದು
ಇರಿಸಿ
ಇರಿ-ಸಿ-ಕೊಂಡ
ಇರಿ-ಸಿ-ಕೊಂಡಂಥ
ಇರಿ-ಸಿ-ಕೊಂಡಿದೆ
ಇರಿ-ಸಿ-ಕೊಂಡು
ಇರಿ-ಸಿ-ಕೊಳ್ಳಿ
ಇರಿಸಿತು
ಇರಿಸಿದ
ಇರಿ-ಸಿದ್ದೀಯ
ಇರು
ಇರು-ವಂತಾ-ಗಿದೆ
ಇರುತ್ತದೆ
ಇರುತ್ತವೆ
ಇರುತ್ತಾನೆ
ಇರುತ್ತಾರೆ
ಇರುತ್ತಿದ್ದರು
ಇರುತ್ತಿದ್ದಳು
ಇರುತ್ತಿದ್ದೆ
ಇರುತ್ತಿ-ರ-ಲಾಗಿ
ಇರುತ್ತಿ-ರ-ಲಿಲ್ಲ
ಇರುತ್ತೆ
ಇರುತ್ತೇ-ವೆಯೆ
ಇರುಳು
ಇರುವ
ಇರುವಂತೆ
ಇರುವನೆ
ಇರುವನೇ
ಇರುವರು
ಇರುವಲ್ಲಿ
ಇರು-ವಲ್ಲಿಗೆ
ಇರು-ವಲ್ಲಿಯೇ
ಇರುವಳು
ಇರು-ವ-ವನು
ಇರು-ವ-ವನೂ
ಇರು-ವ-ವರ
ಇರು-ವ-ವ-ರಿಗೆ
ಇರು-ವಾ-ಗಲೂ
ಇರುವಿಕೆ
ಇರು-ವಿ-ಕೆಗೆ
ಇರು-ವಿ-ಕೆ-ನಂಬಿಕೆ
ಇರು-ವಿ-ಕೆಯ
ಇರು-ವಿ-ಕೆ-ಯಲ್ಲಿ
ಇರು-ವಿ-ಕೆಯೇ
ಇರು-ವು-ದಾ-ದರೂ
ಇರು-ವು-ದಿಲ್ಲ
ಇರು-ವು-ದನ್ನು
ಇರು-ವು-ದ-ರಿಂದ
ಇರು-ವು-ದ-ರಿಂದಲೇ
ಇರು-ವು-ದಾ-ದರೆ
ಇರು-ವು-ದಿಲ್ಲ
ಇರುವುದು
ಇರು-ವು-ದೊಂದೇ
ಇರುವುದೋ
ಇರುವೆ
ಇರು-ವೆ-ಗಳ
ಇರು-ವೆ-ಗ-ಳದ್ದೇ
ಇರು-ವೆ-ಗ-ಳಿಗೆ
ಇರುವೆಯ
ಇರು-ವೆ-ಯಪ್ಪ-ನ-ವ-ರನ್ನು
ಇರು-ವೆ-ಯಾ-ದರೋ
ಇಲಾಖೆ
ಇಲಾಖೆಯ
ಇಲಾ-ಖೆ-ಯಲ್ಲಿ
ಇಲಾ-ಖೆ-ಯ-ವರು
ಇಲಿಯು
ಇಲೆ
ಇಲೆಕ್ಟ್ರಾ-ನನ್ನೂ
ಇಲೆಕ್ಟ್ರಾ-ನಿಕ್ಸ್
ಇಲೆಕ್ಟ್ರಾನು
ಇಲೆಕ್ಟ್ರಾ-ನು-ಗ-ಳಿಂದ
ಇಲೆಕ್ಟ್ರಾ-ನು-ಗ-ಳೆಂಬ
ಇಲೆಕ್ಟ್ರಾನ್
ಇಲೆಕ್ಟ್ರಿಕ್
ಇಲ್ಲ
ಇಲ್ಲ-ಎಂಬು-ದನ್ನು
ಇಲ್ಲದ
ಇಲ್ಲದಂತೆ
ಇಲ್ಲದಲ್ಲಿ
ಇಲ್ಲ-ದ-ವನು
ಇಲ್ಲ-ದ-ವರು
ಇಲ್ಲದಾಗ
ಇಲ್ಲ-ದಾ-ತನೇ
ಇಲ್ಲದಿದ್ದ
ಇಲ್ಲ-ದಿದ್ದರೂ
ಇಲ್ಲ-ದಿದ್ದರೆ
ಇಲ್ಲ-ದಿದ್ದಲ್ಲಿ
ಇಲ್ಲ-ದಿದ್ದಾಗ
ಇಲ್ಲ-ದಿ-ರಲಿ
ಇಲ್ಲ-ದಿ-ರು-ವು-ದನ್ನು
ಇಲ್ಲ-ದಿ-ರು-ವುದು
ಇಲ್ಲದಿಲ್ಲ
ಇಲ್ಲದೆ
ಇಲ್ಲದೇ
ಇಲ್ಲದ್ದಾ-ಗಿತ್ತು
ಇಲ್ಲವಲ್ಲ
ಇಲ್ಲವಷ್ಟೆ
ಇಲ್ಲವಾಗಿ
ಇಲ್ಲ-ವಾ-ಗಿದೆ
ಇಲ್ಲ-ವಾ-ಗಿ-ಸುತ್ತದೆ
ಇಲ್ಲ-ವಾ-ಗುತ್ತಿದ್ದವು
ಇಲ್ಲ-ವಾ-ಗು-ವವೇ
ಇಲ್ಲ-ವಾ-ದರೂ
ಇಲ್ಲ-ವಾ-ದರೆ
ಇಲ್ಲ-ವಾ-ದಲ್ಲಿ
ಇಲ್ಲ-ವಾ-ದಾಗ
ಇಲ್ಲವೆ
ಇಲ್ಲ-ವೆಂದರೂ
ಇಲ್ಲ-ವೆಂದಲ್ಲ
ಇಲ್ಲವೆಂದು
ಇಲ್ಲವೆಂದೂ
ಇಲ್ಲವೆಂಬ
ಇಲ್ಲ-ವೆಂಬುದು
ಇಲ್ಲ-ವೆ-ನಿ-ಸು-ವುದು
ಇಲ್ಲವೆನ್ನಿ
ಇಲ್ಲ-ವೆನ್ನು-ವಂತಾ-ಗಿದೆ
ಇಲ್ಲವೇ
ಇಲ್ಲವೋ
ಇಲ್ಲಿ
ಇಲ್ಲಿಂದ
ಇಲ್ಲಿಗೆ
ಇಲ್ಲಿದೆ
ಇಲ್ಲಿ-ದೆ-ಯಲ್ಲವೆ
ಇಲ್ಲಿದ್ದು
ಇಲ್ಲಿನ
ಇಲ್ಲಿಯ
ಇಲ್ಲಿ-ಯ-ವ-ರೆಗೂ
ಇಲ್ಲಿ-ಯ-ವ-ರೆಗೆ
ಇಲ್ಲಿಯೇ
ಇಲ್ಲಿರುವ
ಇಲ್ಲಿಲ್ಲ-ವೆಂಬೀ
ಇಲ್ಲಿವೆ
ಇಲ್ಲೂ
ಇಲ್ಲೆಲ್ಲಾ
ಇಲ್ಲೇ
ಇಲ್ಲೊಂದು
ಇಲ್ಲೊಬ್ಬ
ಇಳಿದ
ಇಳಿದು
ಇಳಿದೆ
ಇಳಿ-ಮು-ಖ-ವಾ-ಗುತ್ತ-ಲಿತ್ತು
ಇಳಿ-ಯ-ಬಲ್ಲವು
ಇಳಿ-ಯ-ಬಾ-ರ-ದೆಂದು
ಇಳಿ-ಯ-ಲಾ-ರದ
ಇಳಿಯಿತು
ಇಳಿಯುತ್ತ
ಇಳಿ-ಯುತ್ತಿದ್ದಾಗ
ಇಳಿ-ಯು-ವಂತೆ
ಇಳಿ-ವ-ಯಸ್ಸಿನ
ಇಳಿ-ವ-ಯಸ್ಸಿ-ನಲ್ಲೂ
ಇಳಿ-ಸಿದ್ದರು
ಇಳೆಯಲ್ಲಿ
ಇಳೆಯೊಳು
ಇವನ
ಇವನನ್ನು
ಇವ-ನಿ-ಗಿಂತ
ಇವನಿಗೆ
ಇವನು
ಇವನೂ
ಇವನೆಂಥ
ಇವನೇನೋ
ಇವ-ನೊ-ಳಗೆ
ಇವನ್ನು
ಇವನ್ನೂ
ಇವನ್ನೆಲ್ಲ
ಇವರ
ಇವರದೇ
ಇವರನ್ನು
ಇವರಲ್ಲಿ
ಇವರಿಂದ
ಇವರಿಗೂ
ಇವರಿಗೆ
ಇವ-ರಿ-ಗೊಬ್ಬ
ಇವ-ರಿಬ್ಬರ
ಇವರು
ಇವ-ರೆ-ಡೆಗೆ
ಇವ-ರೆಲ್ಲರ
ಇವ-ರೆಲ್ಲ-ರಿಗೂ
ಇವ-ರೆಲ್ಲರೂ
ಇವ-ರೊ-ಡನೆ
ಇವಳೇಕೆ
ಇವಿಷ್ಟು
ಇವು
ಇವುಗಳ
ಇವು-ಗ-ಳನ್ನು
ಇವು-ಗ-ಳನ್ನೂ
ಇವು-ಗ-ಳನ್ನೆಲ್ಲ
ಇವು-ಗ-ಳಲ್ಲಿ
ಇವು-ಗ-ಳಿಂದ
ಇವು-ಗ-ಳಿ-ಗಾಗಿ
ಇವು-ಗ-ಳಿಗೆ
ಇವುಗಳು
ಇವು-ಗ-ಳೆಲ್ಲ
ಇವು-ಗ-ಳೊ-ಡ-ನೆಯೇ
ಇವೂ
ಇವೆ
ಇವೆಅಂಥ
ಇವೆಯೇ
ಇವೆರಡೂ
ಇವೆಲ್ಲ
ಇವೆಲ್ಲಕ್ಕೂ
ಇವೆಲ್ಲ-ವು-ಗಳ
ಇವೆಲ್ಲ-ವು-ಗ-ಳಿಂದ
ಇವೆಲ್ಲವೂ
ಇವೇ
ಇಷ್ಟ
ಇಷ್ಟಕ್ಕೂ
ಇಷ್ಟ-ದ-ರು-ಶನ
ಇಷ್ಟ-ದೇ-ವರ
ಇಷ್ಟ-ದೇ-ವ-ರಲ್ಲಿ
ಇಷ್ಟ-ನಾ-ಮದ
ಇಷ್ಟ-ಪಟ್ಟರೆ
ಇಷ್ಟ-ಪ-ಡ-ಲಿಲ್ಲ
ಇಷ್ಟ-ಪ-ಡು-ವು-ದಿಲ್ಲ
ಇಷ್ಟಬಂದ
ಇಷ್ಟ-ರ-ವ-ರೆಗೆ
ಇಷ್ಟವಾದ
ಇಷ್ಟವಿಲ್ಲ
ಇಷ್ಟ-ವಿಲ್ಲ-ದಿದ್ದರೂ
ಇಷ್ಟಾದರೂ
ಇಷ್ಟಾರ್ಥ
ಇಷ್ಟಿದ್ದರೆ
ಇಷ್ಟು
ಇಷ್ಟು-ಹೀ-ಗೆಂದು
ಇಷ್ಟೆ
ಇಷ್ಟೆಯೇ
ಇಷ್ಟೆಲ್ಲ
ಇಷ್ಟೇ
ಇಷ್ಟೊಂದು
ಇಸವಿ
ಇಸವಿಯ
ಇಸ-ವಿ-ಯಲ್ಲಿ
ಇಸ-ವಿ-ಯ-ವ-ರೆಗೂ
ಇಸ-ವಿ-ಯಿಂದ
ಇಸ್ಪೀಟ್
ಇಸ್ರೇ-ಲಿ-ಗಳೂ
ಇಸ್ರೇಲ್
ಇಸ್ರೇಲ್ನ-ವರ
ಇಸ್ಲಾಂ
ಇಹ
ಇಹದಲ್ಲಿ
ಇಹಪರ
ಇಹ-ಲೋ-ಕ-ದಲ್ಲೂ
ಇಹ-ಲೋ-ಕ-ದಿಂದ
ಇಹ-ಲೋ-ಕ-ವನ್ನು
ಇಹ-ಲೋ-ಕವೇ
ಇಹುದು
ಈ
ಈಕೆ
ಈಕೆಯ
ಈಕೆಯು
ಈಕ್ಲೇಸ್
ಈಗ
ಈಗ-ಲಾ-ದರೂ
ಈಗಲೂ
ಈಗಲೇ
ಈಗಾಗಲೆ
ಈಗಾಗಲೇ
ಈಗಿಂದೀಗ
ಈಗಿಂದೀ-ಗಲೇ
ಈಗಿನ
ಈಗಿರುವ
ಈಗೀಗ
ಈಗೇಕೆ
ಈಚಿನಿಂದ
ಈಚೆಗೆ
ಈಜಬಲ್ಲ
ಈಜಲು
ಈಜಿಪ್ಟಿಗೆ
ಈಜಿಪ್ಟಿನ
ಈಜಿಪ್ಟ್
ಈಜಿಪ್ಟ್ನಲ್ಲಿ
ಈಜು
ಈಜುತ್ತಿದ್ದೆ
ಈಜುವ
ಈಡಾ-ಗ-ಬೇ-ಕಾ-ಗಿದೆ
ಈಡು-ಮಾ-ಡುತ್ತವೆ
ಈತ
ಈತತಲೆ
ಈತನ
ಈತನಿಗೆ
ಈತನು
ಈರ್ಷ್ಯೆ-ಗ-ಳನ್ನು
ಈವ-ರೆ-ಗಿನ
ಈವರೆಗೂ
ಈವರೆಗೆ
ಈಶ
ಈಶ್ವರ
ಈಸಬೇಕು
ಉ
ಉಂಟಯ್ಯ
ಉಂಟಾ
ಉಂಟಾಗುವ
ಉಂಟಾ-ಗ-ದಿದ್ದರೆ
ಉಂಟಾಗದು
ಉಂಟಾ-ಗ-ಬ-ಹುದು
ಉಂಟಾ-ಗ-ಬ-ಹು-ದೆಂಬು-ದನ್ನು
ಉಂಟಾ-ಗ-ಬೇಕು
ಉಂಟಾ-ಗ-ಬೇ-ಕೆಂದಿಲ್ಲ
ಉಂಟಾ-ಗ-ಲಿಲ್ಲ
ಉಂಟಾಗಲು
ಉಂಟಾ-ಗ-ಲೆಂದು
ಉಂಟಾಗಿ
ಉಂಟಾಗಿತ್ತು
ಉಂಟಾಗಿದೆ
ಉಂಟಾ-ಗಿ-ದೆಯೆ
ಉಂಟಾ-ಗಿ-ಬಿ-ಡುತ್ತದೆ
ಉಂಟಾ-ಗಿ-ರಲಿ
ಉಂಟಾ-ಗಿ-ರುವ
ಉಂಟಾ-ಗುತ್ತದೆ
ಉಂಟಾ-ಗುತ್ತವೆ
ಉಂಟಾಗುವ
ಉಂಟಾ-ಗು-ವಂಥದು
ಉಂಟಾ-ಗು-ವ-ವು-ಗಳು
ಉಂಟಾ-ಗು-ವು-ದ-ರಿಂದ
ಉಂಟಾ-ಗು-ವುದು
ಉಂಟಾ-ಗು-ವುವು
ಉಂಟಾದ
ಉಂಟಾದರೆ
ಉಂಟಾ-ದ-ವರ
ಉಂಟಾದಾಗ
ಉಂಟಾದೀತು
ಉಂಟಾ-ದೀ-ತೆಂದು
ಉಂಟಾ-ದು-ದಲ್ಲ
ಉಂಟಾದುದು
ಉಂಟಾ-ದು-ದೆಂದು
ಉಂಟಾ-ಯಿ-ತಲ್ಲದೆ
ಉಂಟಾಯಿತು
ಉಂಟು
ಉಂಟು-ಮಾ-ಡುವ
ಉಂಟು-ಮಾ-ಡ-ದಂತೆ
ಉಂಟು-ಮಾ-ಡದೆ
ಉಂಟು-ಮಾ-ಡ-ಬ-ಹು-ದಾ-ದರೂ
ಉಂಟು-ಮಾ-ಡಲು
ಉಂಟುಮಾಡಿ
ಉಂಟು-ಮಾ-ಡಿತು
ಉಂಟು-ಮಾ-ಡಿದ
ಉಂಟು-ಮಾ-ಡಿ-ಬಿಟ್ಟಿತ್ತು
ಉಂಟು-ಮಾ-ಡುತ್ತದೆ
ಉಂಟು-ಮಾ-ಡುತ್ತ-ದೆಂದು
ಉಂಟು-ಮಾ-ಡುತ್ತಾನೆ
ಉಂಟು-ಮಾ-ಡುವ
ಉಂಟು-ಮಾ-ಡು-ವಂಥ
ಉಂಟು-ಮಾ-ಡು-ವು-ದಕ್ಕೆ
ಉಂಟೆ
ಉಂಡ
ಉಂಡವನು
ಉಂಡಿದ್ದರೆ
ಉಂಡು
ಉಕ್ಕ-ಲಿಪ್ರೀ-ತಿ-ಗಾಗಿ
ಉಕ್ಕಿ
ಉಕ್ಕಿತು
ಉಕ್ಕಿನ
ಉಕ್ಕಿಬಂತು
ಉಕ್ಕಿ-ಸ-ಲೆಂದೇ
ಉಕ್ಕಿಸುವ
ಉಕ್ಕಿ-ಹ-ರಿ-ಯುವ
ಉಕ್ಕು
ಉಕ್ಕುತ್ತವೆ
ಉಕ್ಕುತ್ತಿದೆ
ಉಕ್ತವಾದ
ಉಕ್ತಿ
ಉಕ್ತಿಯಂತೆ
ಉಕ್ತಿಯನ್ನು
ಉಗಮ
ಉಗ-ಮಸ್ಥಾನ
ಉಗಿಶಕ್ತಿ
ಉಗುರು
ಉಗು-ರು-ಗಳ
ಉಗ್ಗಿನ
ಉಗ್ಗು
ಉಗ್ಗುತ್ತಿದ್ದ
ಉಗ್ಗುತ್ತಿ-ರ-ಲಿಲ್ಲ
ಉಗ್ಗು-ದ-ನಿಯ
ಉಗ್ರ
ಉಗ್ರವಾಗಿ
ಉಗ್ರ-ವಾ-ಯಿತು
ಉಗ್ರಾಣ
ಉಚಿತ
ಉಚಿ-ತ-ವಲ್ಲ
ಉಚಿ-ತ-ವಲ್ಲ-ಅಷ್ಟು
ಉಚಿ-ತ-ವಲ್ಲವೇ
ಉಚಿ-ತ-ವಾದ
ಉಚ್ಚ-ಕಂಠ-ದಿಂದ
ಉಚ್ಚ-ಮಟ್ಟದ
ಉಚ್ಚರಿಸ
ಉಚ್ಚ-ರಿ-ಸ-ತೊ-ಡ-ಗಿದ
ಉಚ್ಚ-ರಿ-ಸ-ತೊ-ಡ-ಗಿದ್ದರು
ಉಚ್ಚ-ರಿ-ಸ-ಬಲ್ಲಿರಾ
ಉಚ್ಚರಿಸಿ
ಉಚ್ಚ-ರಿ-ಸಿದ
ಉಚ್ಚ-ರಿ-ಸಿ-ದರು
ಉಚ್ಚ-ರಿ-ಸಿ-ದಾ-ಗಲೂ
ಉಚ್ಚ-ರಿ-ಸಿ-ದಾ-ಗಲೇ
ಉಚ್ಚ-ರಿ-ಸುತ್ತಿದ್ದಾ-ಳೆಂದೇ
ಉಚ್ಚ-ರಿ-ಸುತ್ತ
ಉಚ್ಚ-ರಿ-ಸುತ್ತಲೇ
ಉಚ್ಚ-ರಿ-ಸು-ವು-ದಲ್ಲ
ಉಚ್ಚ-ರಿ-ಸು-ವು-ದಷ್ಟೇ
ಉಚ್ಚ-ವರ್ಗದ
ಉಚ್ಚ-ವರ್ಗ-ದ-ವ-ರಿ-ಗಿಂತ
ಉಚ್ಚಶ್ರೇ-ಣಿ-ಯಲ್ಲಿ
ಉಚ್ಚಸ್ಥಿ-ತಿ-ಯನ್ನು
ಉಚ್ಚಾಟನೆ
ಉಚ್ಚಾರ
ಉಚ್ಚಾ-ರ-ಣೆಯ
ಉಜ್ಜಿ
ಉಜ್ಜುವಂತೆ
ಉಜ್ವಲ
ಉಜ್ವಲತೆ
ಉಡಲು
ಉಡು-ಗಿ-ಹೋ-ಯಿತು
ಉಡು-ಗೆ-ಯನ್ನು
ಉಡುಪ
ಉಡುಪನ್ನು
ಉಣಲುಂಟು
ಉಣಿಸನ್ನು
ಉಣಿಸಿದ
ಉಣಿಸುವ
ಉಣ್ಣ-ಬೇ-ಕಾ-ಗಿದೆ
ಉಣ್ಣ-ಬೇ-ಕಾ-ಗು-ವುದು
ಉಣ್ಣಲೇ
ಉಣ್ಣ-ಲೇ-ಬೇ-ಕಾ-ಗುತ್ತದೆ
ಉಣ್ಣ-ಲೇ-ಬೇಕು
ಉಣ್ಣು-ವ-ವನು
ಉತ್ಕೃಷ್ಟ
ಉತ್ಕೃಷ್ಟ-ವಾ-ದುದು
ಉತ್ತಮ
ಉತ್ತ-ಮ-ಗೊ-ಳಿ-ಸು-ವು-ದೆಂದರೆ
ಉತ್ತಮರ
ಉತ್ತ-ಮ-ವಾ-ಗ-ಬೇ-ಕಾ-ದರೆ
ಉತ್ತ-ಮ-ವಾ-ಗ-ಲಿಲ್ಲ
ಉತ್ತ-ಮ-ವಾಗಿ
ಉತ್ತ-ಮ-ವಾ-ಗಿತ್ತು
ಉತ್ತ-ಮ-ವಾದ
ಉತ್ತರ
ಉತ್ತ-ರ-ಗ-ಳನ್ನು
ಉತ್ತರದ
ಉತ್ತ-ರ-ದಿಂದ
ಉತ್ತ-ರ-ಭಾ-ರ-ತ-ದಲ್ಲಿ
ಉತ್ತ-ರ-ವನ್ನು
ಉತ್ತ-ರ-ವನ್ನೂ
ಉತ್ತ-ರ-ವಾ-ಗ-ಲಾ-ರ-ದಷ್ಟೆ
ಉತ್ತ-ರ-ವಾಗಿ
ಉತ್ತ-ರ-ವಿದೆ
ಉತ್ತರವೇ
ಉತ್ತ-ರಾ-ಧಿ-ಕಾ-ರಿ-ಯೊಬ್ಬನ
ಉತ್ತರಿಸ
ಉತ್ತ-ರಿ-ಸಲು
ಉತ್ತ-ರಿ-ಸಿದ
ಉತ್ತ-ರಿ-ಸಿ-ದರು
ಉತ್ತ-ರಿ-ಸಿ-ದ-ರು-ಅ-ವರ
ಉತ್ತ-ರಿ-ಸಿದೆ
ಉತ್ತ-ರಿ-ಸಿದ್ದ
ಉತ್ತ-ರಿ-ಸುತ್ತಾರೆ
ಉತ್ತ-ರಿ-ಸುತ್ತಿದ್ದಾಗ
ಉತ್ತ-ರಿ-ಸುವ
ಉತ್ತ-ರೋತ್ತರ
ಉತ್ತೀರ್ಣ-ನಾಗಿ
ಉತ್ತೀರ್ಣ-ನಾದ
ಉತ್ತೀರ್ಣ-ರಾ-ದರೂ
ಉತ್ತುಂಗ
ಉತ್ತೇಜಕ
ಉತ್ತೇ-ಜ-ನಈ
ಉತ್ಥಾನದ
ಉತ್ಪತ್ತಿ
ಉತ್ಪತ್ತಿಗೆ
ಉತ್ಪತ್ತಿ-ದಾ-ಯ-ಕ-ವಲ್ಲದ
ಉತ್ಪತ್ತಿಯ
ಉತ್ಪನ್ನ
ಉತ್ಪನ್ನ-ವಾ-ಗ-ದಿದ್ದರೆ
ಉತ್ಪಾದಕ
ಉತ್ಪಾ-ದ-ನೆ-ಯನ್ನು
ಉತ್ಪಾ-ದ-ನೆ-ಯಿಂದ
ಉತ್ಪಾ-ದಿ-ಸ-ಲಾ-ರವು
ಉತ್ಪಾ-ದಿ-ಸಿತು
ಉತ್ಪ್ರೇಕ್ಷೆ
ಉತ್ಸಾಹ
ಉತ್ಸಾ-ಹ-ಗ-ಳನ್ನು
ಉತ್ಸಾ-ಹ-ಗ-ಳಿಂದ
ಉತ್ಸಾ-ಹ-ಗ-ಳಿಲ್ಲದ
ಉತ್ಸಾ-ಹ-ಗ-ಳುಂಟಾ-ದರೆ
ಉತ್ಸಾ-ಹ-ಗಳೇ
ಉತ್ಸಾಹದ
ಉತ್ಸಾ-ಹ-ದಿಂದ
ಉತ್ಸಾ-ಹ-ದಿಂದಲೇ
ಉತ್ಸಾ-ಹ-ದೊಂದಿಗೆ
ಉತ್ಸಾ-ಹ-ಪೂ-ರಿತ
ಉತ್ಸಾ-ಹ-ಯುತ
ಉತ್ಸಾ-ಹ-ವಿಲ್ಲ
ಉತ್ಸಾ-ಹ-ವಿಲ್ಲದೆ
ಉತ್ಸಾ-ಹ-ಹೀ-ನ-ನಾ-ಗುವ
ಉತ್ಸಾಹಿ
ಉತ್ಸಾ-ಹಿ-ಗ-ಳಾಗಿ
ಉತ್ಸಾ-ಹಿ-ಗಳು
ಉತ್ಸಾ-ಹಿ-ತ-ರಾಗಿ
ಉತ್ಸಾಹೀ
ಉದಯ
ಉದಯಕ್ಕೆ
ಉದ-ಯ-ರ-ವಿಯ
ಉದ-ಯಿ-ಸ-ದಿದ್ದರೆ
ಉದ-ಯಿ-ಸಿ-ದರು
ಉದ-ಯಿ-ಸುತ್ತವೆ
ಉದಾ
ಉದಾತ್ತ
ಉದಾತ್ತ-ಗೊ-ಳಿ-ಸ-ಬಲ್ಲ
ಉದಾತ್ತ-ಗೊ-ಳಿ-ಸ-ಬ-ಹುದು
ಉದಾತ್ತ-ವಾ-ಗಿ-ರ-ಬ-ಹುದು
ಉದಾತ್ತೀ-ಕ-ರ-ಣಕ್ಕೆ
ಉದಾರ
ಉದಾರತೆ
ಉದಾ-ರ-ದೃಷ್ಟಿ
ಉದಾ-ರ-ಮ-ನಸ್ಕರು
ಉದಾ-ರ-ಮ-ನಸ್ಸಿ-ನಿಂದ
ಉದಾ-ರ-ವಾಗಿ
ಉದಾ-ರ-ವಾ-ಗಿದ್ದರೂ
ಉದಾ-ರ-ವಾದಿ
ಉದಾರಿ
ಉದಾ-ರಿ-ಗಳು
ಉದಾ-ಹ-ರಣೆ
ಉದಾ-ಹ-ರ-ಣೆ-ಸ-ಹಿತ
ಉದಾ-ಹ-ರ-ಣೆ-ಗಳ
ಉದಾ-ಹ-ರ-ಣೆ-ಗ-ಳಲ್ಲವೆ
ಉದಾ-ಹ-ರ-ಣೆ-ಗ-ಳಲ್ಲಿ
ಉದಾ-ಹ-ರ-ಣೆ-ಗ-ಳಿವೆ
ಉದಾ-ಹ-ರ-ಣೆ-ಗಳು
ಉದಾ-ಹ-ರ-ಣೆ-ಗಳೇ
ಉದಾ-ಹ-ರ-ಣೆಗೆ
ಉದಾ-ಹ-ರ-ಣೆ-ಯಂತಿತ್ತು
ಉದಾ-ಹ-ರ-ಣೆ-ಯಂತೆ
ಉದಾ-ಹ-ರ-ಣೆ-ಯನ್ನಾ-ದರೂ
ಉದಾ-ಹ-ರ-ಣೆ-ಯನ್ನು
ಉದಾ-ಹ-ರ-ಣೆ-ಯಲ್ಲ
ಉದಾ-ಹ-ರ-ಣೆ-ಯಲ್ಲಿ
ಉದಾ-ಹ-ರ-ಣೆ-ಯಾ-ದರು
ಉದಾ-ಹ-ರ-ಣೆ-ಯಿಂದ
ಉದಿ-ಸ-ದಿದ್ದರೆ
ಉದಿಸದು
ಉದಿ-ಸ-ಬೇಕು
ಉದಿ-ಸ-ಲೇ-ಬೇಕು
ಉದಿಸಿ
ಉದಿ-ಸಿ-ತಂತೆ
ಉದಿಸಿತು
ಉದಿಸಿದ
ಉದಿ-ಸಿ-ದ-ನೆಂದರೆ
ಉದಿ-ಸಿ-ದುದೇ
ಉದಿ-ಸಿ-ರ-ಬೇಕು
ಉದಿ-ಸುತ್ತದೆ
ಉದಿ-ಸುತ್ತವೆ
ಉದಿಸುವ
ಉದಿ-ಸು-ವುದು
ಉದುರಿದ
ಉದು-ರಿ-ದರೂ
ಉದುರಿಸಿ
ಉದ್ಗ-ರಿ-ಸಿತು
ಉದ್ಗ-ರಿ-ಸಿದ
ಉದ್ಗ-ರಿ-ಸಿ-ದರು
ಉದ್ಗ-ರಿ-ಸಿ-ದಳು
ಉದ್ಗ-ರಿ-ಸಿದ್ದರು
ಉದ್ಗ-ರಿ-ಸಿದ್ದುಂಟು
ಉದ್ಗ-ರಿ-ಸುತ್ತಾರೆ
ಉದ್ಗಾರ
ಉದ್ಗಾ-ರ-ಗಳ
ಉದ್ಗಾ-ರ-ಗಳು
ಉದ್ಗಾ-ರ-ವಲ್ಲ
ಉದ್ಗ್ರಂಥ
ಉದ್ಗ್ರಂಥ-ಗ-ಳನ್ನು
ಉದ್ಗ್ರಂಥ-ವನ್ನು
ಉದ್ಘೋ-ಷಿ-ಸಿ-ದರು
ಉದ್ದ
ಉದ್ದಕ್ಕೂ
ಉದ್ದ-ಗ-ಲಕ್ಕೂ
ಉದ್ದನೆಯ
ಉದ್ದ-ವಾ-ಗಿದ್ದು
ಉದ್ದವಾದ
ಉದ್ದೀಪಿಸಿ
ಉದ್ದೇಶ
ಉದ್ದೇ-ಶ-ಗಳ
ಉದ್ದೇ-ಶ-ಗ-ಳನ್ನು
ಉದ್ದೇ-ಶ-ಗ-ಳಿಲ್ಲ
ಉದ್ದೇ-ಶ-ಗ-ಳಿಲ್ಲದೆ
ಉದ್ದೇ-ಶ-ಗ-ಳಿವೆ
ಉದ್ದೇ-ಶ-ಗ-ಳಿ-ವೆ-ಇದು
ಉದ್ದೇ-ಶ-ಗ-ಳೇನು
ಉದ್ದೇಶದ
ಉದ್ದೇ-ಶ-ದಿಂದ
ಉದ್ದೇ-ಶ-ದಿಂದಾ-ಗಲಿ
ಉದ್ದೇ-ಶ-ಪೂ-ರಿ-ತವೂ
ಉದ್ದೇ-ಶ-ಪೂರ್ವ-ಕ-ವಾಗಿ
ಉದ್ದೇ-ಶ-ಪೂರ್ವ-ಕ-ವಾದ
ಉದ್ದೇ-ಶ-ಬೇಕು
ಉದ್ದೇ-ಶ-ವನ್ನು
ಉದ್ದೇ-ಶ-ವಲ್ಲ
ಉದ್ದೇ-ಶ-ವಾ-ದರೆ
ಉದ್ದೇ-ಶ-ವಿ-ರುತ್ತದೆ
ಉದ್ದೇ-ಶ-ವಿಲ್ಲದ
ಉದ್ದೇಶಿಸಿ
ಉದ್ದೇ-ಶಿ-ಸಿ-ರು-ವುದು
ಉದ್ಧರಿಸ
ಉದ್ಧ-ರಿ-ಸ-ಬೇಕು
ಉದ್ಧ-ರಿ-ಸಲಿ
ಉದ್ಧ-ರಿ-ಸಲು
ಉದ್ಧರಿಸಿ
ಉದ್ಧ-ರಿ-ಸಿದ
ಉದ್ಧ-ರಿ-ಸುವ
ಉದ್ಧ-ವ-ನನ್ನು
ಉದ್ಧಾರ
ಉದ್ಧಾ-ರಕ್ಕಾಗಿ
ಉದ್ಧಾರಕ್ಕೆ
ಉದ್ಧಾರದ
ಉದ್ಧಾ-ರ-ವಾ-ಗ-ಬೇ-ಕೆಂಬ
ಉದ್ಧಾ-ರ-ವಾ-ಗುತ್ತಿತ್ತು
ಉದ್ಧಾ-ರ-ವಾ-ಗು-ವುದು
ಉದ್ಧಾ-ರ-ವಾದ
ಉದ್ಧಾ-ರ-ವಾ-ಯಿ-ತೆ-ನಿ-ಸುತ್ತದೆ
ಉದ್ಭವಿಸಿ
ಉದ್ಭ-ವಿ-ಸಿತು
ಉದ್ಭ-ವಿ-ಸಿದ
ಉದ್ಭ-ವಿ-ಸಿ-ದು-ದಲ್ಲ
ಉದ್ಭವಿಸು
ಉದ್ಭ-ವಿ-ಸುತ್ತದೆ
ಉದ್ಭ-ವಿ-ಸುತ್ತವೆ
ಉದ್ಭ-ವಿ-ಸುವ
ಉದ್ಯಮ
ಉದ್ಯ-ಮ-ಗ-ಳಲ್ಲಿ
ಉದ್ಯಮದ
ಉದ್ಯ-ಮ-ದಿಂದ
ಉದ್ಯ-ಮ-ಶೀಲ
ಉದ್ಯ-ಮ-ಶೀ-ಲತೆ
ಉದ್ಯ-ಮ-ಶೀ-ಲ-ತೆ-ಗಳ
ಉದ್ಯ-ಮ-ಶೀ-ಲ-ತೆ-ಗ-ಳನ್ನು
ಉದ್ಯ-ಮ-ಶೀ-ಲ-ನಾದ
ಉದ್ಯ-ಮ-ಶೀ-ಲ-ರಾ-ದರು
ಉದ್ಯ-ಮ-ಶೀ-ಲರು
ಉದ್ಯ-ಮಿ-ಗಳೂ
ಉದ್ಯಾ-ನ-ಗ-ಳಲ್ಲಿ-ರಲಿ
ಉದ್ಯುಕ್ತ-ರಾ-ಗಿದ್ದರೂ
ಉದ್ಯೋಗ
ಉದ್ಯೋ-ಗ-ಗ-ಳಿ-ಗಿಂತಲೂ
ಉದ್ಯೋ-ಗ-ದಲ್ಲಿ-ರು-ವಾಗ
ಉದ್ಯೋ-ಗ-ವೆಂದರೆ
ಉದ್ಯೋ-ಗಕ್ಕಾಗಿ
ಉದ್ಯೋಗಕ್ಕೆ
ಉದ್ಯೋ-ಗ-ಗ-ಳಲ್ಲಿ-ರುವ
ಉದ್ಯೋ-ಗ-ಗಳು
ಉದ್ಯೋಗದ
ಉದ್ಯೋ-ಗ-ದಲ್ಲಿ
ಉದ್ಯೋ-ಗ-ದಲ್ಲಿ-ರುವ
ಉದ್ಯೋ-ಗ-ದಲ್ಲೇ
ಉದ್ಯೋ-ಗ-ವನ್ನು
ಉದ್ಯೋಗವು
ಉದ್ಯೋಗವೂ
ಉದ್ಯೋ-ಗ-ವೆಂದಿಲ್ಲ
ಉದ್ಯೋಗವೇ
ಉದ್ಯೋಗಸ್ಥ
ಉದ್ಯೋ-ಗಸ್ಥರ
ಉದ್ಯೋ-ಗಸ್ಥ-ರಾ-ದೊ-ಡನೆ
ಉದ್ಯೋ-ಗಾ-ವ-ಕಾ-ಶ-ವಿ-ರ-ಲಿಲ್ಲ
ಉದ್ರಿಕ್ತ
ಉದ್ರೇ-ಕಿ-ಸುತ್ತದೆ
ಉದ್ವಿಗ್ನ-ತೆ-ಗಳು
ಉದ್ವಿಗ್ನ-ತೆಯ
ಉದ್ವಿಗ್ನ-ನಾದ
ಉದ್ವಿಗ್ನ-ರಾ-ಗ-ಬೇಡಿ
ಉದ್ವಿಗ್ನವೂ
ಉದ್ವೇಗ
ಉದ್ವೇ-ಗಕ್ಕಾ-ಗಿಯೇ
ಉದ್ವೇಗಕ್ಕೆ
ಉದ್ವೇ-ಗ-ಗ-ಳನ್ನು
ಉದ್ವೇ-ಗ-ಗ-ಳನ್ನುಂಟು-ಮಾ-ಡುವ
ಉದ್ವೇ-ಗ-ಗ-ಳಿ-ರ-ಲಿಲ್ಲ
ಉದ್ವೇ-ಗ-ಗ-ಳಿಲ್ಲದೇ
ಉದ್ವೇ-ಗ-ಗಳು
ಉದ್ವೇಗದ
ಉದ್ವೇ-ಗ-ದಿಂದ
ಉದ್ವೇ-ಗ-ಭ-ರಿ-ತ-ನಾ-ದ-ವನ
ಉದ್ವೇ-ಗ-ವನ್ನು
ಉದ್ವೇ-ಗ-ವನ್ನುಂಟು
ಉನ್ನತ
ಉನ್ನ-ತ-ಮಟ್ಟದ
ಉನ್ನ-ತ-ಮಟ್ಟ-ವನ್ನು
ಉನ್ನ-ತ-ವಾಗಿ
ಉನ್ನ-ತ-ವಾದ
ಉನ್ನ-ತ-ವಾ-ದುದು
ಉನ್ನ-ತಸ್ತ-ರ-ಗ-ಳನ್ನು
ಉನ್ನ-ತಸ್ತ-ರ-ಗ-ಳಿ-ಗೇ-ರಿಸಿ
ಉನ್ನ-ತಸ್ಥಾ-ನ-ದಲ್ಲಿ-ರು-ವ-ವ-ರಿಗೂ
ಉನ್ನತಿ
ಉನ್ನ-ತಿ-ಗಾಗಿ
ಉನ್ನತಿಗೂ
ಉನ್ನತಿಗೆ
ಉನ್ನ-ತಿ-ಗೇ-ರಲು
ಉನ್ನ-ತಿ-ಗೇರಿ
ಉನ್ನ-ತಿ-ಗೇ-ರಿಸಿ
ಉನ್ನತಿಯ
ಉನ್ನ-ತಿ-ಯನ್ನು
ಉನ್ನ-ತಿ-ಯಾ-ಗ-ದೆಂದು
ಉನ್ನತಿಯೂ
ಉನ್ಮತ್ತ-ನನ್ನಾ-ಗಿ-ಸುವ
ಉನ್ಮತ್ತ-ನಾಗಿ
ಉನ್ಮತ್ತ-ರನ್ನಾ-ಗಿ-ಸಿ-ದೆಯೆ
ಉನ್ಮಾ-ದ-ಗ-ಳೆಲ್ಲ
ಉನ್ಮಾದದ
ಉನ್ಮಾ-ದ-ದಲ್ಲಿ
ಉಪ
ಉಪ-ಯೋ-ಗಿಸಿ
ಉಪಕಥೆ
ಉಪ-ಕ-ರಣ
ಉಪ-ಕ-ರ-ಣ-ಗಳ
ಉಪ-ಕ-ರ-ಣ-ಗ-ಳನ್ನು
ಉಪ-ಕ-ರ-ಣ-ಗ-ಳಿಂದಾ-ಗುವ
ಉಪ-ಕ-ರ-ಣ-ಗಳು
ಉಪ-ಕ-ರ-ಣದ
ಉಪ-ಕ-ರ-ಣ-ದಿಂದ
ಉಪ-ಕ-ರ-ಣ-ವನ್ನು
ಉಪ-ಕ-ರಿ-ಸಿ-ದ-ವರು
ಉಪಕಾರ
ಉಪ-ಕಾ-ರಕ
ಉಪ-ಕಾ-ರ-ಕ-ವಾ-ಗಿ-ರು-ವು-ದ-ರಿಂದ
ಉಪ-ಕಾ-ರ-ಕ-ವಾದ
ಉಪ-ಕಾ-ರ-ಗ-ಳಿಂದೇನು
ಉಪ-ಕಾ-ರ-ವನ್ನು
ಉಪ-ಕಾ-ರ-ವಲ್ಲ
ಉಪ-ಕಾ-ರ-ವಾ-ಗುತ್ತದೆ
ಉಪ-ಕಾ-ರ-ವಾ-ಗುತ್ತ-ದೆನ್ನುತ್ತಾ-ರ-ವರು
ಉಪ-ಕಾ-ರ-ವಾ-ಗು-ವುದು
ಉಪ-ಕಾ-ರ-ವಾದ
ಉಪ-ಕಾ-ರ-ವಾ-ಯಿತು
ಉಪಕಾರಿ
ಉಪ-ಕಾ-ರಿ-ಯಾ-ಗ-ಲಿಲ್ಲ
ಉಪ-ಕಾ-ರಿ-ಯಾದ
ಉಪ-ಕಾ-ರಿ-ಯಾ-ದದ್ದು
ಉಪ-ಕಾ-ರಿ-ಯಾ-ದರೂ
ಉಪ-ಕೃ-ತ-ರಾ-ದ-ವರ
ಉಪ-ಚ-ರಿಸಿ
ಉಪಚಾರ
ಉಪ-ಚಾ-ರಕ್ಕೆ
ಉಪ-ಚಾ-ರ-ಗ-ಳಲ್ಲಿ
ಉಪ-ಚಾ-ರದ
ಉಪ-ಚಾ-ರ-ದಲ್ಲಿ
ಉಪ-ಚಾ-ರವೇ
ಉಪಟಳ
ಉಪ-ಟ-ಳ-ದಿಂದ
ಉಪದೇಶ
ಉಪ-ದೇ-ಶಕ್ಕೆ
ಉಪ-ದೇ-ಶ-ಗಳ
ಉಪ-ದೇ-ಶ-ಗ-ಳನ್ನು
ಉಪ-ದೇ-ಶ-ಗ-ಳಲ್ಲಿನ
ಉಪ-ದೇ-ಶ-ಗಳು
ಉಪ-ದೇ-ಶ-ಗ-ಳೊಂದಿಗೆ
ಉಪ-ದೇ-ಶ-ದಲ್ಲಿ
ಉಪ-ದೇ-ಶ-ವನ್ನ-ನು-ಸ-ರಿಸಿ
ಉಪ-ದೇ-ಶ-ವನ್ನು
ಉಪ-ದೇ-ಶ-ವಾ-ದರೂ
ಉಪ-ದೇ-ಶ-ವಿತ್ತಿದ್ದೀರಿ
ಉಪ-ದೇ-ಶಿಸ
ಉಪ-ದೇ-ಶಿ-ಸ-ಬ-ಹುದು
ಉಪ-ದೇ-ಶಿ-ಸಿದ
ಉಪ-ದೇ-ಶಿ-ಸು-ವಂತಿತ್ತು
ಉಪ-ನಿ-ಷತ್
ಉಪ-ನಿ-ಷತ್ತಿನ
ಉಪ-ನಿ-ಷತ್ತಿ-ನಲ್ಲಿ
ಉಪ-ನಿ-ಷತ್ತು
ಉಪ-ನಿ-ಷತ್ತು-ಗಳ
ಉಪ-ನಿ-ಷತ್ತು-ಗ-ಳಲ್ಲಿ
ಉಪನ್ಯಾಸ
ಉಪನ್ಯಾ-ಸ-ಗ-ಳನ್ನು
ಉಪನ್ಯಾ-ಸ-ಗ-ಳನ್ನೂ
ಉಪನ್ಯಾ-ಸ-ದಲ್ಲಿ
ಉಪನ್ಯಾ-ಸ-ವನ್ನು
ಉಪ-ಪಠ್ಯ-ವನ್ನಾ-ಗಿ-ಯಾ-ದರೂ
ಉಪ-ಫ-ಲ-ಗ-ಳಲ್ಲ
ಉಪಯುಕ್ತ
ಉಪ-ಯುಕ್ತ-ತೆಯ
ಉಪ-ಯುಕ್ತ-ರಾ-ದರು
ಉಪ-ಯುಕ್ತ-ವಲ್ಲವೆ
ಉಪ-ಯುಕ್ತ-ವಾ-ಗುವ
ಉಪ-ಯುಕ್ತ-ವಾ-ಗು-ವಂಥ
ಉಪ-ಯುಕ್ತ-ವಾದ
ಉಪ-ಯುಕ್ತವೂ
ಉಪ-ಯುಕ್ತ-ವೆ-ನಿ-ಸುತ್ತವೆ
ಉಪಯೋಗ
ಉಪ-ಯೋ-ಗಕ್ಕೆ
ಉಪ-ಯೋ-ಗ-ಗ-ಳನ್ನು
ಉಪ-ಯೋ-ಗ-ದಿಂದಲೇ
ಉಪ-ಯೋ-ಗ-ವನ್ನು
ಉಪ-ಯೋ-ಗ-ವಾ-ಗುವ
ಉಪ-ಯೋ-ಗ-ವಾ-ಗು-ವುದು
ಉಪ-ಯೋ-ಗ-ವಿದೆ
ಉಪ-ಯೋ-ಗವೂ
ಉಪ-ಯೋ-ಗಿ-ಸಿದ್ದರು
ಉಪ-ಯೋ-ಗಿ-ಯಾ-ಗು-ವುದು
ಉಪ-ಯೋ-ಗಿ-ಸದೆ
ಉಪ-ಯೋ-ಗಿ-ಸ-ಬ-ಹು-ದೆಂದು
ಉಪ-ಯೋ-ಗಿ-ಸ-ಬೇಕು
ಉಪ-ಯೋ-ಗಿ-ಸ-ಬೇ-ಕೆಂದಾ-ದಲ್ಲಿ
ಉಪ-ಯೋ-ಗಿ-ಸ-ಬೇ-ಕೆಂದು
ಉಪ-ಯೋ-ಗಿ-ಸ-ಬೇಡ
ಉಪ-ಯೋ-ಗಿ-ಸ-ಲಾ-ಗುತ್ತಿತ್ತು
ಉಪ-ಯೋ-ಗಿ-ಸ-ಲಾ-ರ-ರಲ್ಲ
ಉಪ-ಯೋ-ಗಿ-ಸಲು
ಉಪ-ಯೋ-ಗಿಸಿ
ಉಪ-ಯೋ-ಗಿ-ಸಿ-ಕೊಂಡು
ಉಪ-ಯೋ-ಗಿ-ಸಿ-ಕೊಂಡೆ
ಉಪ-ಯೋ-ಗಿ-ಸಿ-ಕೊಳ್ಳ-ದಿದ್ದರೆ
ಉಪ-ಯೋ-ಗಿ-ಸಿ-ಕೊಳ್ಳ-ಬ-ಹು-ದೆಂಬು-ದರ
ಉಪ-ಯೋ-ಗಿ-ಸಿ-ಕೊಳ್ಳ-ಬೇಕು
ಉಪ-ಯೋ-ಗಿ-ಸಿ-ಕೊಳ್ಳುತ್ತಿದ್ದೀರಾ
ಉಪ-ಯೋ-ಗಿ-ಸಿಕೋ
ಉಪ-ಯೋ-ಗಿ-ಸಿದ
ಉಪ-ಯೋ-ಗಿ-ಸಿ-ದಂತೆ
ಉಪ-ಯೋ-ಗಿ-ಸಿ-ದರು
ಉಪ-ಯೋ-ಗಿ-ಸಿ-ದರೂ
ಉಪ-ಯೋ-ಗಿ-ಸಿದೆ
ಉಪ-ಯೋ-ಗಿ-ಸಿದ್ದ
ಉಪ-ಯೋ-ಗಿ-ಸಿದ್ದಾ-ರೆಯೇ
ಉಪ-ಯೋ-ಗಿ-ಸಿದ್ದಿ
ಉಪ-ಯೋ-ಗಿ-ಸು-ವರೆ
ಉಪ-ಯೋ-ಗಿ-ಸುತ್ತಿತ್ತು
ಉಪ-ಯೋ-ಗಿ-ಸುತ್ತಿದ್ದ
ಉಪ-ಯೋ-ಗಿ-ಸುತ್ತಿದ್ದಾರೆ
ಉಪ-ಯೋ-ಗಿ-ಸುತ್ತಿದ್ದೇ-ವೆಯೇ
ಉಪ-ಯೋ-ಗಿ-ಸುತ್ತಿ-ರ-ಬ-ಹುದು
ಉಪ-ಯೋ-ಗಿ-ಸುತ್ತಿ-ರು-ವನೇ
ಉಪ-ಯೋ-ಗಿ-ಸುತ್ತೇವೆ
ಉಪ-ಯೋ-ಗಿ-ಸುವ
ಉಪ-ಯೋ-ಗಿ-ಸು-ವನೆ
ಉಪ-ಯೋ-ಗಿ-ಸು-ವಾಗ
ಉಪ-ಯೋ-ಗಿ-ಸೋಣ
ಉಪ-ಲಕ್ಷ-ಣ-ವಾಗಿ
ಉಪವಾಸ
ಉಪ-ವಾ-ಸ-ದಿಂದ
ಉಪ-ವಾ-ಸವೇ
ಉಪ-ಶ-ಮನ
ಉಪಾ-ಧಿ-ಗ-ಳಿವೆ
ಉಪಾ-ಧಿ-ಗಳು
ಉಪಾಯ
ಉಪಾಯಕ್ಕೆ
ಉಪಾ-ಯ-ಗ-ಳನ್ನೇ
ಉಪಾ-ಯ-ಗ-ಳಿಂದ
ಉಪಾ-ಯ-ಗಳು
ಉಪಾ-ಯ-ಗಳೂ
ಉಪಾ-ಯ-ವನ್ನು
ಉಪಾ-ಯ-ವಾ-ವುದೂ
ಉಪಾಯವೂ
ಉಪಾಯವೇ
ಉಪಾ-ಯ-ವೇನೂ
ಉಪಾಸನಾ
ಉಪಾಸನೆ
ಉಪಾ-ಸ-ನೆ-ಗಳು
ಉಪಾ-ಸ-ನೆ-ಯಲ್ಲಿ
ಉಪಾಹಾರ
ಉಪಾ-ಹಾ-ರದ
ಉಪೇಕ್ಷಾ-ದೃಷ್ಟಿ-ಯಿಂದ
ಉಪ್ಪಿಲ್ಲ
ಉಪ್ಪು-ನೀ-ರಿನ
ಉಮರ್
ಉಯಿಲನ್ನು
ಉರಿಯನ್ನು
ಉರಿ-ಯುತ್ತಾರೆ
ಉರಿ-ಯುತ್ತಿದ್ದ
ಉರಿ-ಯುತ್ತಿ-ರುತ್ತದೆ
ಉರಿ-ಯುತ್ತಿ-ರುತ್ತಾನೆ
ಉರಿ-ಯುತ್ತಿ-ರುತ್ತಾರೆ
ಉರಿಯುವ
ಉರಿ-ಸ-ಬಾ-ರದು
ಉರಿಸಲು
ಉರಿ-ಸುತ್ತಾರೆ
ಉರುಳದೇ
ಉರು-ಳಾ-ಟ-ದಲ್ಲಿ
ಉರು-ಳಿ-ದವು
ಉರುಳಿಸ
ಉರು-ಳಿ-ಸು-ವ-ವ-ರಂತೆ
ಉರುಳು
ಉಲ್ಬ-ಣ-ಗೊಂಡು
ಉಲ್ಬ-ಣ-ಗೊಳ್ಳುವ
ಉಲ್ಬ-ಣ-ಗೊಳ್ಳು-ವುದು
ಉಲ್ಬಣತೆ
ಉಲ್ಬ-ಣ-ವಾ-ಗು-ವುದು
ಉಲ್ಬಣಿಸಿ
ಉಲ್ಬ-ಣಿ-ಸಿ-ದಾಗ
ಉಲ್ಬ-ಣಿ-ಸು-ವುದು
ಉಲ್ಲಂಘ-ನೆಯ
ಉಲ್ಲಾಸ
ಉಲ್ಲಾಸದ
ಉಲ್ಲಾ-ಸ-ದಲ್ಲಿ-ರು-ವಾಗ
ಉಲ್ಲೇ-ಖ-ವಿದೆ
ಉಲ್ಲೇ-ಖಿ-ತ-ವಾ-ಗಿ-ರುವ
ಉಲ್ಲೇ-ಖಿ-ಸ-ಬ-ಹು-ದಾ-ಗಿ-ದೆ-ಕ-ಳೆದ
ಉಲ್ಲೇ-ಖಿ-ಸ-ಲಾ-ಗಿದೆ
ಉಲ್ಲೇ-ಖಿ-ಸಲು
ಉಲ್ಲೇ-ಖಿ-ಸಿದ
ಉಳಿ
ಉಳಿತಾಯ
ಉಳಿ-ತಾ-ಯ-ವಾ-ಗುತ್ತದೆ
ಉಳಿದ
ಉಳಿದರು
ಉಳಿ-ದ-ವ-ರಿಂದ
ಉಳಿ-ದ-ವ-ರಿಗೂ
ಉಳಿ-ದ-ವರು
ಉಳಿ-ದ-ವರೂ
ಉಳಿ-ದ-ವು-ಗಳು
ಉಳಿದಾಳೆ
ಉಳಿ-ದಿದ್ದಳು
ಉಳಿ-ದಿ-ರ-ಬಲ್ಲುದು
ಉಳಿ-ದಿ-ರುತ್ತದೆ
ಉಳಿ-ದಿ-ರುತ್ತವೆ
ಉಳಿದೀತೆ
ಉಳಿದು
ಉಳಿ-ದು-ದನ್ನು
ಉಳಿ-ದು-ಕೊಂಡಿದೆ
ಉಳಿ-ದು-ಕೊಂಡಿ-ರುತ್ತವೆ
ಉಳಿ-ದು-ಕೊಳ್ಳುತ್ತದೆ
ಉಳಿ-ದು-ದೆಲ್ಲ-ವನ್ನೂ
ಉಳಿ-ದು-ಬಿ-ಡುತ್ತದೆ
ಉಳಿ-ದು-ಬಿ-ಡುತ್ತವೆ
ಉಳಿದೆಲ್ಲ
ಉಳಿಯದು
ಉಳಿಯದೇ
ಉಳಿ-ಯ-ಬೇ-ಕಾ-ಗುತ್ತ-ದೆಂದು
ಉಳಿ-ಯ-ಬೇ-ಕಾ-ದರೆ
ಉಳಿ-ಯ-ಬೇಕು
ಉಳಿ-ಯ-ಬೇ-ಕೆಂಬ
ಉಳಿ-ಯ-ಬೇಕೋ
ಉಳಿಯಿತು
ಉಳಿ-ಯುತ್ತದೆ
ಉಳಿ-ಯುತ್ತ-ದೆಂಬ
ಉಳಿ-ಯುತ್ತಿತ್ತು
ಉಳಿಯುವ
ಉಳಿ-ಯು-ವಂತೆ
ಉಳಿ-ಯು-ವಂಥದು
ಉಳಿ-ಯು-ವುದು
ಉಳಿಸಲು
ಉಳಿಸಿ
ಉಳಿ-ಸಿ-ಕೊಂಡದ್ದೆ
ಉಳಿ-ಸಿ-ಕೊಂಡರೇ
ಉಳಿ-ಸಿ-ಕೊಂಡಲ್ಲಿ
ಉಳಿ-ಸಿ-ಕೊಂಡಿದ್ದಾರೆ
ಉಳಿ-ಸಿ-ಕೊಂಡಿ-ರುತ್ತವೆ
ಉಳಿ-ಸಿ-ಕೊಂಡು
ಉಳಿ-ಸಿ-ಕೊಳ್ಳ-ಬೇಕು
ಉಳಿ-ಸಿ-ಕೊಳ್ಳಲು
ಉಳಿ-ಸಿ-ಕೊಳ್ಳುತ್ತಾನೆ
ಉಳಿ-ಸಿ-ಕೊಳ್ಳುತ್ತಿ-ರ-ಲಿಲ್ಲ
ಉಳಿ-ಸಿ-ಕೊಳ್ಳುವ
ಉಳಿ-ಸಿ-ಕೊಳ್ಳು-ವುದು
ಉಳಿ-ಸಿ-ದು-ದನ್ನು
ಉಳಿ-ಸಿದ್ದರು
ಉಳಿಸುವ
ಉಳುಮೆ
ಉಳ್ಳ
ಉಳ್ಳ-ವ-ರಾ-ಗಿದ್ದಾ-ರೆ-ಎನ್ನುತ್ತಾರೆ
ಉಳ್ಳವರು
ಉಷ್ಟ್ರ
ಉಷ್ಣದ
ಉಸಿ-ರಾ-ಗಲಿ
ಉಸಿರಾಗಿ
ಉಸಿರಾಟ
ಉಸಿ-ರಾ-ಟ-ದಲ್ಲಿ
ಉಸಿ-ರಾ-ಡಲು
ಉಸಿ-ರಾ-ಡಿದ
ಉಸಿ-ರಾ-ಡು-ವಂತಾ-ಗ-ಬೇಕು
ಉಸಿ-ರಾ-ಡು-ವಾಗ
ಉಸಿ-ರಾ-ಡು-ವು-ದಕ್ಕೆ
ಉಸಿ-ರಾ-ಯಿತು
ಉಸಿರು
ಉಸಿ-ರು-ಕಟ್ಟು-ವಂಥ
ಉಸಿ-ರೆ-ಳೆ-ಯುವ
ಉಸು-ರ-ಬೇಕು
ಉಸುರಿದ
ಉಸು-ರಿ-ದ-ನಂತೆ
ಊ
ಊಟ
ಊಟಕ್ಕೂ
ಊಟಕ್ಕೆ
ಊಟ-ತಿಂಡಿ-ಗೆಂದು
ಊಟದ
ಊಟದಲ್ಲಿ
ಊಟ-ಮಾ-ಡಿದೆ
ಊಟವನ್ನು
ಊಟವಾದ
ಊಟೆ
ಊಟೋ-ಪ-ಚಾರ
ಊದಿ
ಊದಿದರು
ಊರ
ಊರನ್ನು
ಊರಲ್ಲಿ
ಊರಲ್ಲಿದ್ದಾರೆ
ಊರಿಂದ
ಊರಿಗೆ
ಊರಿಗೇ
ಊರಿದ
ಊರಿನ
ಊರಿನಲ್ಲಿ
ಊರಿನಿಂದ
ಊರು
ಊರುಕೇರಿ
ಊರು-ಗ-ಳನ್ನು
ಊರು-ಗ-ಳಲ್ಲಿದ್ದ
ಊರು-ಗ-ಳಿಗೆ
ಊರು-ಗೋ-ಲನ್ನು
ಊರು-ಗೋ-ಲು-ಗಳ
ಊರೇ
ಊರ್ಜಿ-ತ-ಗೊ-ಳಿ-ಸಿ-ದರೆ
ಊಳಿಗ
ಊಹನೆ
ಊಹಾತೀತ
ಊಹಿ-ಸ-ಬ-ಹುದು
ಊಹಿಸಲೂ
ಊಹಿ-ಸಿ-ಕೊಳ್ಳ-ಬ-ಹುದು
ಊಹಿ-ಸಿ-ಕೊಳ್ಳಿ
ಊಹಿ-ಸಿ-ಕೊಳ್ಳು-ವುದು
ಊಹಿ-ಸಿ-ದಳು
ಊಹಿ-ಸಿ-ರ-ಬ-ಹುದು
ಊಹಿ-ಸು-ವಂತೆಯೂ
ಊಹೆಗೆ
ಋಗ್ವೇದ
ಋಣ
ಋಣದ
ಋಷಿಗಳ
ಋಷಿಗಳು
ಋಷಿಗಳೂ
ಋಷಿಗಳೆ
ಋಷಿಗೆ
ಋಷಿ-ಜೀ-ವ-ನ-ವನ್ನು
ಋಷಿ-ಮು-ನಿ-ಗ-ಳಿಗೆ
ಋಷಿ-ಮು-ನಿ-ಗಳು
ಋಷಿ-ಮು-ನಿ-ಯೋ-ಗಿ-ಗಳು
ಋಷಿಯ
ಎ
ಎಂ
ಎಂಜಿ-ನಿ-ಯರ್
ಎಂಜೆಲೋ
ಎಂಟನೇ
ಎಂಟಿಟಿ
ಎಂಟು
ಎಂಟು-ದೇ-ವ-ರಿ-ರು-ವನೇ
ಎಂಟುನೂರು
ಎಂಟುಬಾರಿ
ಎಂಟೂವರೆ
ಎಂಡ್
ಎಂತಹ
ಎಂತಾ-ಗ-ಬ-ಹುದು
ಎಂತಾದೀತು
ಎಂತಿ-ರ-ಬೇಕು
ಎಂತಿವೆ
ಎಂತು
ಎಂಥ
ಎಂಥದು
ಎಂಥದೇ
ಎಂಥದೋ
ಎಂಥವು
ಎಂಥಾ
ಎಂಥೆಂಥ
ಎಂದ
ಎಂದಂತಾಯ್ತು
ಎಂದಂತೆ
ಎಂದದ್ದು
ಎಂದನಂತೆ
ಎಂದನಾತ
ಎಂದನು
ಎಂದರಂತೆ
ಎಂದ-ರ-ವರು
ಎಂದರು
ಎಂದರುಆ
ಎಂದರೂ
ಎಂದರೆ
ಎಂದರ್ಥ
ಎಂದರ್ಥ-ವಲ್ಲ
ಎಂದಲ್ಲ
ಎಂದಲ್ಲ-ನಮ್ಮ
ಎಂದಲ್ಲಿ
ಎಂದಳಂತೆ
ಎಂದ-ಳ-ವಳು
ಎಂದಳಾಕೆ
ಎಂದಳು
ಎಂದವನ
ಎಂದವನು
ಎಂದವರು
ಎಂದವರೂ
ಎಂದವಳು
ಎಂದಷ್ಟೆ
ಎಂದಷ್ಟೇ
ಎಂದಾ
ಎಂದಾಕೆ
ಎಂದಾಗ
ಎಂದಾಗದೆ
ಎಂದಾ-ಗುತ್ತದೆ
ಎಂದಾ-ಗುತ್ತ-ದೆಯೇ
ಎಂದಾತ
ಎಂದಾ-ದ-ಮೇಲೆ
ಎಂದಾದರೂ
ಎಂದಾದರೆ
ಎಂದಾ-ದ-ರೊಂದು
ಎಂದಾದಲ್ಲಿ
ಎಂದಾ-ನಲ್ಲವೆ
ಎಂದಾ-ಯಿ-ತಲ್ಲವೇ
ಎಂದಾಯಿತು
ಎಂದಿಗೂ
ಎಂದಿಟ್ಟುಕೊ
ಎಂದಿತು
ಎಂದಿತ್ತು
ಎಂದಿದ್ದ-ನಂತೆ
ಎಂದಿದ್ದರು
ಎಂದಿದ್ದರೆ
ಎಂದಿದ್ದಾನೆ
ಎಂದಿದ್ದಾರೆ
ಎಂದಿದ್ದಾ-ಳಾಕೆ
ಎಂದಿದ್ದು-ಕೊಂಡರೆ
ಎಂದಿನಂತೆ
ಎಂದಿನಿಂದ
ಎಂದಿ-ರ-ಬ-ಹುದು
ಎಂದಿ-ರ-ಬೇ-ಕಲ್ಲವೆ
ಎಂದು
ಎಂದುಕೊಂಡ
ಎಂದು-ಕೊಂಡಿ-ರ-ಬೇಕು
ಎಂದು-ಕೊಂಡಿ-ರುತ್ತಾನೆ
ಎಂದುಕೊಂಡು
ಎಂದು-ಕೊಳ್ಳ-ಬ-ಹುದು
ಎಂದು-ಕೊಳ್ಳುತ್ತ-ದೆ-ಯಲ್ಲ-ಅದೇ
ಎಂದು-ಕೊಳ್ಳುತ್ತೇ-ನಲ್ಲ
ಎಂದು-ಕೊಳ್ಳುವ
ಎಂದು-ಕೊಳ್ಳು-ವಂಥದು
ಎಂದುತ್ತ-ರಿ-ಸಿದೆ
ಎಂದುದಕ್ಕೆ
ಎಂದುದೂ
ಎಂದು-ವಾ-ದಿ-ಸ-ಬ-ಹು-ದಾ-ದರೂ
ಎಂದು-ವಾ-ದಿ-ಸುವ
ಎಂದೂ
ಎಂದೆ
ಎಂದೆಂದಿಗೂ
ಎಂದೆಂದೂ
ಎಂದೆಣಿಸಿ
ಎಂದೆ-ಣಿ-ಸಿ-ಕೊಳ್ಳಿ
ಎಂದೆ-ನಿ-ಸುತ್ತದೆ
ಎಂದೆಲ್ಲ
ಎಂದೇ
ಎಂದೇಕೆ
ಎಂದೇನೂ
ಎಂದೇನೋ
ಎಂದೊಡನೆ
ಎಂದೋ
ಎಂಬ
ಎಂಬಂತೆ
ಎಂಬಂತೆಯೇ
ಎಂಬಂಥ
ಎಂಬತ್ತ-ರಷ್ಟು
ಎಂಬತ್ತಾ-ರನೇ
ಎಂಬತ್ತು
ಎಂಬತ್ತೆ-ರಡು
ಎಂಬತ್ತೈದು
ಎಂಬರ್ಥದ
ಎಂಬರ್ಥ-ದಲ್ಲೇ
ಎಂಬಲ್ಲಿ
ಎಂಬಲ್ಲಿನ
ಎಂಬ-ವ-ರಿಂದ
ಎಂಬವು
ಎಂಬಾಕೆ
ಎಂಬಾತ
ಎಂಬಿ-ಬಿ-ಎಸ್
ಎಂಬಿ-ವು-ಗಳೇ
ಎಂಬೀ
ಎಂಬು
ಎಂಬುದನ್ನು
ಎಂಬುದಕ್ಕೆ
ಎಂಬು-ದಕ್ಕೊಂದು
ಎಂಬುದನು
ಎಂಬು-ದನ್ನಲ್ಲ
ಎಂಬುದನ್ನು
ಎಂಬುದನ್ನೂ
ಎಂಬುದನ್ನೇ
ಎಂಬುದರ
ಎಂಬು-ದ-ರಲ್ಲಿ
ಎಂಬುದಾಗಿ
ಎಂಬುದು
ಎಂಬುದೂ
ಎಂಬುದೆ
ಎಂಬುದೇ
ಎಂಬುದೇನೋ
ಎಂಬುವರು
ಎಂಬೆಲ್ಲ
ಎಂಭತ್ತ
ಎಂಭತ್ತನೇ
ಎಂಭತ್ತ-ಮೂ-ರನೇ
ಎಕ-ರೆ-ಗ-ಳನ್ನು
ಎಕ-ರೆ-ಗ-ಳಷ್ಟು
ಎಕೆ-ಡ-ಮಿಯ
ಎಕ್ಸ್
ಎಕ್ಸ್ಟೆಂಡ್
ಎಕ್ಸ್ಪೊ
ಎಕ್ಸ್ರೇ
ಎಕ್ಸ್ರೇ-ಗ-ಳನ್ನು
ಎಕ್ಸ್ರೇನಲ್ಲಿ
ಎಚ್
ಎಚ್ಚರ
ಎಚ್ಚ-ರ-ಗೊಂಡ
ಎಚ್ಚ-ರ-ಗೊಂಡು
ಎಚ್ಚ-ರ-ಗೊ-ಳಿ-ಸ-ಬ-ಹು-ದಾದ
ಎಚ್ಚ-ರ-ಗೊ-ಳಿ-ಸಲು
ಎಚ್ಚ-ರ-ಗೊ-ಳಿ-ಸುತ್ತದೆ
ಎಚ್ಚ-ರ-ಗೊ-ಳಿ-ಸುತ್ತವೆ
ಎಚ್ಚ-ರ-ಗೊಳ್ಳದ
ಎಚ್ಚರದ
ಎಚ್ಚ-ರ-ದಲ್ಲೂ
ಎಚ್ಚ-ರ-ದಿಂದ
ಎಚ್ಚ-ರ-ದಿಂದಿ-ರು-ವಾಗ
ಎಚ್ಚ-ರ-ವಾ-ಗು-ವಂತೆ
ಎಚ್ಚ-ರ-ವಾ-ದಾಗ
ಎಚ್ಚ-ರ-ವಾ-ಯಿತು
ಎಚ್ಚರಿಕೆ
ಎಚ್ಚ-ರಿ-ಕೆಯ
ಎಚ್ಚ-ರಿ-ಕೆ-ಯನ್ನು
ಎಚ್ಚ-ರಿ-ಕೆ-ಯಾಗಿ
ಎಚ್ಚ-ರಿ-ಕೆ-ಯಿಂದ
ಎಚ್ಚ-ರಿ-ಕೆಯೇ
ಎಚ್ಚ-ರಿ-ಸ-ಬ-ಹುದು
ಎಚ್ಚ-ರಿ-ಸ-ಬೇಕು
ಎಚ್ಚ-ರಿ-ಸಲು
ಎಚ್ಚರಿಸಿ
ಎಚ್ಚ-ರಿ-ಸಿದ್ದ
ಎಚ್ಚ-ರಿ-ಸಿದ್ದರೂ
ಎಚ್ಚ-ರಿ-ಸುತ್ತದೆ
ಎಚ್ಚ-ರಿ-ಸುವ
ಎಚ್ಚ-ರಿ-ಸು-ವು-ದುಂಟು
ಎಚ್ಚೆತ್ತ
ಎಚ್ಚೆತ್ತು
ಎಟಕುವ
ಎಡ-ಕತ್ತಿ-ನಲ್ಲಿ
ಎಡರು
ಎಡಿಸನ್
ಎಡಿ-ಸನ್ನನ್ನು
ಎಡೆ
ಎಡೆ-ಬಿ-ಡದೆ
ಎಡೆಗೆ
ಎಡೆ-ಬಿ-ಡದೆ
ಎಡೆಯಿಲ್ಲ
ಎಡೆ-ಯಿಲ್ಲದೆ
ಎಡೆಯೆಲ್ಲ
ಎಡ್ಗರ್
ಎಡ್ಗರ್ಕೇಸೀ
ಎಡ್ಡಿ
ಎಡ್ಡಿಯನ್ನು
ಎಡ್ಲೆ
ಎಡ್ವರ್ಡ್
ಎಣಿ-ಸ-ಲಾ-ಗ-ದಷ್ಟು
ಎಣಿಸಿ
ಎಣಿ-ಸಿ-ದ-ರಾ-ಗದು
ಎಣಿ-ಸಿ-ದಾ-ಗ-ಲೆಲ್ಲ
ಎಣಿಸುತ್ತ
ಎಣಿಸುವ
ಎಣಿ-ಸು-ವು-ದಿಲ್ಲ
ಎಣೆ
ಎಣ್ಣೆಯ
ಎಣ್ಣೆಯಾಗಿ
ಎತ್ತ
ಎತ್ತ-ಣಿಂದಲೋ
ಎತ್ತರ
ಎತ್ತರಕ್ಕೆ
ಎತ್ತರದ
ಎತ್ತ-ರ-ದಲ್ಲಿ
ಎತ್ತ-ರ-ದಲ್ಲಿದ್ದರೂ
ಎತ್ತ-ರ-ದಲ್ಲಿದ್ದ-ವರು
ಎತ್ತ-ರ-ದಲ್ಲಿ-ರು-ವ-ವ-ರೆಂದು
ಎತ್ತ-ರ-ದಿಂದ
ಎತ್ತ-ರ-ಬ-ದು-ಕಿನ
ಎತ್ತ-ರ-ವನ್ನು
ಎತ್ತ-ರ-ವನ್ನೂ
ಎತ್ತ-ರ-ವಾದ
ಎತ್ತ-ರ-ವಿ-ರುವ
ಎತ್ತ-ರ-ವೆಷ್ಟು
ಎತ್ತ-ಲಾ-ರ-ದಾದ
ಎತ್ತಲು
ಎತ್ತಿ
ಎತ್ತಿಕಟ್ಟಿ
ಎತ್ತಿ-ಕೊಳ್ಳಲು
ಎತ್ತಿತಂದು
ಎತ್ತಿತೋರಿ
ಎತ್ತಿನ
ಎತ್ತಿ-ಬಿ-ಡ-ಬಲ್ಲೆ
ಎತ್ತಿ-ಹಿ-ಡಿದು
ಎತ್ತಿಹುದು
ಎತ್ತು-ಗ-ಳಿಂದ-ಕಾ-ಯ-ವೆಂಬ
ಎತ್ತುತ್ತಿದ್ದ
ಎತ್ತುತ್ತೇವೆ
ಎತ್ತುವ
ಎತ್ತು-ವ-ವರು
ಎತ್ತುವುದೇ
ಎದುರಲ್ಲಿ
ಎದು-ರಲ್ಲಿಟ್ಟು
ಎದುರಲ್ಲೇ
ಎದುರಾಗಿ
ಎದು-ರಾ-ಗಿ-ರುವ
ಎದು-ರಾ-ಗುತ್ತವೆ
ಎದು-ರಾ-ಗುವ
ಎದು-ರಾ-ಡ-ಬೇಡ
ಎದು-ರಾ-ದರೆ
ಎದು-ರಾ-ಳಿಯ
ಎದು-ರಿ-ಗಿಟ್ಟು-ಕೊಂಡೆ
ಎದು-ರಿ-ಗಿದ್ದ
ಎದುರಿಗೂ
ಎದುರಿಗೆ
ಎದು-ರಿ-ನಲ್ಲಿ
ಎದು-ರಿ-ಸದೆ
ಎದು-ರಿ-ಸ-ಬಲ್ಲ
ಎದು-ರಿ-ಸ-ಬಲ್ಲವು
ಎದು-ರಿ-ಸ-ಬಲ್ಲೆ
ಎದು-ರಿ-ಸ-ಬೇ-ಕಲ್ಲದೇ
ಎದು-ರಿ-ಸ-ಬೇ-ಕಾ-ಗುತ್ತಿತ್ತು
ಎದು-ರಿ-ಸ-ಬೇ-ಕಾ-ಗು-ವುದು
ಎದು-ರಿ-ಸ-ಬೇ-ಕಾದ
ಎದು-ರಿ-ಸ-ಬೇ-ಕಾ-ಯಿತು
ಎದು-ರಿ-ಸ-ಬೇ-ಕಿತ್ತು
ಎದು-ರಿ-ಸ-ಬೇಕು
ಎದು-ರಿ-ಸ-ಬೇ-ಕೆಂಬು-ದನ್ನು
ಎದು-ರಿ-ಸಲು
ಎದುರಿಸಿ
ಎದು-ರಿ-ಸಿದ
ಎದು-ರಿ-ಸಿ-ದರು
ಎದು-ರಿ-ಸಿ-ದರೆ
ಎದು-ರಿ-ಸುತ್ತ
ಎದು-ರಿ-ಸುತ್ತಾನೆ
ಎದು-ರಿ-ಸುತ್ತಾರೆ
ಎದು-ರಿ-ಸುತ್ತಿದ್ದೇ-ವೆಯೇ
ಎದು-ರಿ-ಸುವ
ಎದು-ರಿ-ಸು-ವಲ್ಲಿ
ಎದು-ರಿ-ಸು-ವಾಗ
ಎದು-ರಿ-ಸು-ವು-ದ-ರಲ್ಲಿ
ಎದು-ರಿ-ಸು-ವುದು
ಎದುರು
ಎದು-ರು-ಗಡೆ
ಎದು-ರು-ಗ-ಡೆ-ಯಿಂದ
ಎದು-ರು-ಗೊಂಡು
ಎದೆ
ಎದೆ-ಗಾ-ರಿಕೆ
ಎದೆ-ಗಾ-ರಿ-ಕೆ-ಯಿಂದ
ಎದೆ-ಗು-ದಿಯೇ
ಎದೆ-ಗೂ-ಡಿನ
ಎದೆಗೆ
ಎದೆನೋವು
ಎದೆಯೂ
ಎದೆಯೊಡ್ಡಿ
ಎದ್ದಾಗ
ಎದ್ದಿತ್ತು
ಎದ್ದಿದ್ದೇನೆ
ಎದ್ದು
ಎದ್ದು-ಕಾ-ಣುತ್ತಿದೆ
ಎದ್ದು-ಕು-ಳಿತು
ಎದ್ದುನಿಂತ
ಎದ್ದುನಿಂತು
ಎದ್ದು-ನಿಲ್ಲು-ವುದು
ಎದ್ದುಬಿದ್ದು
ಎದ್ದೇಳಿ
ಎದ್ದೇಳು
ಎನಿ-ಸ-ದಿದ್ದರೂ
ಎನಿಸದೇ
ಎನಿ-ಸಿ-ಕೊಂಡ
ಎನಿ-ಸಿ-ಕೊಂಡ-ವರು
ಎನಿ-ಸಿ-ಕೊಂಡು
ಎನಿ-ಸಿ-ಕೊಳ್ಳುವ
ಎನಿಸಿದ
ಎನಿ-ಸಿ-ದರೂ
ಎನಿ-ಸುತ್ತದೆ
ಎನಿಸುವ
ಎನ್ನ
ಎನ್ನ-ತೊ-ಡ-ಗಿದ
ಎನ್ನದೆ
ಎನ್ನನೆ
ಎನ್ನನೇ
ಎನ್ನ-ಬಲ್ಲೆವೆ
ಎನ್ನಬಹು
ಎನ್ನ-ಬ-ಹುದು
ಎನ್ನ-ಬ-ಹುದೆ
ಎನ್ನ-ಬೇ-ಕಾ-ಗುತ್ತದೆ
ಎನ್ನ-ಬೇ-ಕಾ-ಗು-ವುದು
ಎನ್ನಬೇಡಿ
ಎನ್ನ-ಲಾ-ಗದು
ಎನ್ನ-ಲಾ-ಗಿದೆ
ಎನ್ನ-ಲಾ-ದೀತೆ
ಎನ್ನಲಾರ
ಎನ್ನಲಾರೆ
ಎನ್ನಲು
ಎನ್ನಲೂ
ಎನ್ನಿ-ಸ-ದಿ-ರದು
ಎನ್ನಿ-ಸ-ಬ-ಹುದು
ಎನ್ನಿ-ಸಿ-ಕೊಂಡ
ಎನ್ನಿ-ಸಿ-ಕೊಂಡ-ವ-ನಾತ
ಎನ್ನಿ-ಸಿ-ಕೊಳ್ಳುವ
ಎನ್ನಿ-ಸಿ-ಕೊಳ್ಳು-ವು-ದ-ರಲ್ಲಿ
ಎನ್ನಿಸಿತು
ಎನ್ನಿ-ಸುತ್ತದೆ
ಎನ್ನಿ-ಸುತ್ತಿತ್ತು
ಎನ್ನಿ-ಸುತ್ತಿದೆ
ಎನ್ನಿಸುವ
ಎನ್ನುತ್ತ
ಎನ್ನುತ್ತದೆ
ಎನ್ನುತ್ತಾ
ಎನ್ನುತ್ತಾ-ನಲ್ಲವೆ
ಎನ್ನುತ್ತಾನೆ
ಎನ್ನುತ್ತಾ-ರ-ವರು
ಎನ್ನುತ್ತಾ-ರಷ್ಟೆ
ಎನ್ನುತ್ತಾರೆ
ಎನ್ನುತ್ತಿದ್ದ-ರ-ವರು
ಎನ್ನುತ್ತಿದ್ದರು
ಎನ್ನುತ್ತಿದ್ದಾರೆ
ಎನ್ನುತ್ತಿ-ರ-ಬ-ಹುದು
ಎನ್ನುತ್ತಿ-ರುತ್ತಾನೆ
ಎನ್ನುತ್ತೀ-ರಷ್ಟೆ
ಎನ್ನುತ್ತೀರಿ
ಎನ್ನುತ್ತೇನೆ
ಎನ್ನುತ್ತೇವೆ
ಎನ್ನುವ
ಎನ್ನು-ವಂತಿಲ್ಲ
ಎನ್ನುವಂತೆ
ಎನ್ನುವಂಥ
ಎನ್ನು-ವಂಥದು
ಎನ್ನು-ವ-ರಲ್ಲವೇ
ಎನ್ನು-ವ-ವ-ರಿಗೆ
ಎನ್ನು-ವ-ವ-ರಿದ್ದಾರೆ
ಎನ್ನು-ವ-ವರು
ಎನ್ನು-ವಷ್ಟ-ರಲ್ಲೇ
ಎನ್ನುವಷ್ಟು
ಎನ್ನುವಿರಾ
ಎನ್ನು-ವು-ದನ್ನು
ಎನ್ನು-ವು-ದಕ್ಕಿಂತಲೂ
ಎನ್ನು-ವು-ದಕ್ಕೆ
ಎನ್ನು-ವು-ದನ್ನು
ಎನ್ನು-ವು-ದರ
ಎನ್ನು-ವು-ದಿಲ್ಲ
ಎನ್ನುವುದು
ಎನ್ನು-ವು-ದುಂಟು
ಎನ್ನೋಣ
ಎಪ್ಪತ್ತ-ರಷ್ಟು
ಎಪ್ಪತ್ತಾರು
ಎಪ್ಪತ್ತು
ಎಪ್ಪತ್ತು-ಸಾ-ವಿರ
ಎಪ್ಪತ್ತೇ-ಳರ
ಎಪ್ಪತ್ತೊಂಬತ್ತು
ಎಫ್
ಎಬೇಕಸ್
ಎಬ್ಬಿ-ಸ-ಬ-ಹುದು
ಎಬ್ಬಿಸಿ
ಎಬ್ಬಿ-ಸು-ವುದು
ಎಮರ್ಸನ್
ಎಮೆಸಾನ್
ಎಮ್ಮೆ
ಎಮ್ಮೆಯ
ಎರ-ಗು-ವುವು
ಎರಡಕ್ಕೆ
ಎರಡನೆ
ಎರ-ಡ-ನೆಯ
ಎರ-ಡ-ನೆ-ಯ-ದಕ್ಕೆ
ಎರ-ಡ-ನೆ-ಯದು
ಎರ-ಡ-ನೆ-ಯದೇ
ಎರ-ಡ-ನೆ-ಯ-ವನು
ಎರಡನೇ
ಎರಡನ್ನೂ
ಎರಡರ
ಎರ-ಡ-ರಷ್ಟು
ಎರಡಲ್ಲ
ಎರಡು
ಎರ-ಡು-ನಿಮ್ಮಲ್ಲಿದೆ
ಎರ-ಡು-ಸಂಶಯ
ಎರಡೂ
ಎರ-ಡೆ-ರಡು
ಎರಡೇ
ಎರಿಕ್
ಎರಿಜ್ಜೋ
ಎರೆ-ದಿ-ರ-ಬ-ಹುದು
ಎರೆ-ಹು-ಳು-ವಿ-ನೋ-ಪಾ-ದಿ-ಯಲ್ಲಿ
ಎಲಿ-ಜ-ಬೆತ್
ಎಲಿ-ಯಟ್ಟ-ರನ್ನು
ಎಲುಬಿನ
ಎಲುಬು
ಎಲೆ
ಎಲೆ-ಅ-ಡಿ-ಕೆ-ಯನ್ನು
ಎಲೆಕ್ಟ್ರಾ-ನು-ಇ-ವೆಲ್ಲವೂ
ಎಲೆಕ್ಟ್ರಿಕ್
ಎಲೆ-ಗ-ಳನ್ನು
ಎಲೆ-ಗ-ಳಲ್ಲಿ
ಎಲೆ-ಗ-ಳೆಲ್ಲ
ಎಲೆಗಳೇ
ಎಲೆ-ಮ-ರೆಯ
ಎಲೆಯೂ
ಎಲ್
ಎಲ್ಲ
ಎಲ್ಲ-ರೆ-ಡೆಗೂ
ಎಲ್ಲ-ಎಂದರೆ
ಎಲ್ಲಕ್ಕಿಂತ
ಎಲ್ಲಕ್ಕೂ
ಎಲ್ಲ-ತೆ-ರ-ನಾದ
ಎಲ್ಲದಕ್ಕೂ
ಎಲ್ಲದರ
ಎಲ್ಲ-ದ-ರಲ್ಲೂ
ಎಲ್ಲ-ಧರ್ಮ-ಗಳ
ಎಲ್ಲರ
ಎಲ್ಲರನು
ಎಲ್ಲರನ್ನು
ಎಲ್ಲರನ್ನೂ
ಎಲ್ಲ-ರಲ್ಲಿದೆ
ಎಲ್ಲರಲ್ಲೂ
ಎಲ್ಲ-ರಿಂದಲೂ
ಎಲ್ಲ-ರಿ-ಗಾಗಿ
ಎಲ್ಲ-ರಿ-ಗಿಂತ
ಎಲ್ಲ-ರಿ-ಗಿಂತಲೂ
ಎಲ್ಲರಿಗೂ
ಎಲ್ಲರೂ
ಎಲ್ಲ-ರೆ-ಡೆ-ಗಿನ
ಎಲ್ಲ-ರೆ-ಡೆಗೂ
ಎಲ್ಲ-ರೆ-ದುರೂ
ಎಲ್ಲ-ರೊಂದಿಗೆ
ಎಲ್ಲ-ರೊ-ಡನೆ
ಎಲ್ಲವನ್ನು
ಎಲ್ಲವನ್ನೂ
ಎಲ್ಲ-ವು-ಗಳ
ಎಲ್ಲವೂ
ಎಲ್ಲಾ
ಎಲ್ಲಾದರೂ
ಎಲ್ಲಿ
ಎಲ್ಲಿಂದ
ಎಲ್ಲಿಂದಲೋ
ಎಲ್ಲಿಗೆ
ಎಲ್ಲಿಗೋ
ಎಲ್ಲಿದೆ
ಎಲ್ಲಿದ್ದಾರೆ
ಎಲ್ಲಿದ್ದೀರಿ
ಎಲ್ಲಿ-ಯ-ತ-ನಕ
ಎಲ್ಲಿ-ಯ-ವ-ರೆಗೆ
ಎಲ್ಲಿ-ಯಾ-ದರೂ
ಎಲ್ಲಿ-ಯಾ-ದಾರೂ
ಎಲ್ಲಿಯೋ
ಎಲ್ಲಿರುವೆ
ಎಲ್ಲಿವೆ
ಎಲ್ಲೆ
ಎಲ್ಲೆಂದ-ರಲ್ಲಿ
ಎಲ್ಲೆಡೆ
ಎಲ್ಲೆ-ಡೆ-ಗ-ಳಿಂದಲೂ
ಎಲ್ಲೆಡೆಗೂ
ಎಲ್ಲೆ-ಡೆ-ಯಲ್ಲೂ
ಎಲ್ಲೆಡೆಯೂ
ಎಲ್ಲೆಲ್ಲಿ
ಎಲ್ಲೆಲ್ಲಿಯು
ಎಲ್ಲೆಲ್ಲೂ
ಎಲ್ಲೆಲ್ಲೊ
ಎಲ್ಲೆಲ್ಲೋ
ಎಲ್ಲೇ
ಎಲ್ಲೋ
ಎಳ-ವೆ-ಯಲ್ಲೇ
ಎಳೆ-ಕೂ-ಸನ್ನು
ಎಳೆದರೆ
ಎಳೆ-ದಲ್ಲಿಗೆ
ಎಳೆದು
ಎಳೆ-ದು-ಕೊಂಡಿದ್ದ
ಎಳೆ-ದು-ಕೊಂಡು
ಎಳೆ-ದು-ಕೊಳ್ಳುತ್ತಾನೆ
ಎಳೆಯ
ಎಳೆ-ಯ-ಬಲ್ಲರು
ಎಳೆ-ಯ-ರನ್ನು
ಎಳೆ-ಯ-ರಲ್ಲಿ
ಎಳೆ-ಯ-ರಿಗೆ
ಎಳೆಯರು
ಎಳೆ-ಯಲ್ಪಟ್ಟ
ಎಳೆಯುತ್ತ
ಎಳೆ-ಯುತ್ತದೆ
ಎಳೆ-ಯುತ್ತವೆ
ಎಳೆಯುತ್ತಾ
ಎಳೆ-ಯುತ್ತಿದ್ದಾರೆ
ಎಳೆ-ಯುತ್ತಿಲ್ಲ
ಎಳೆಯುವ
ಎಳೆ-ಯು-ವಂತೆ
ಎಳೆ-ಯು-ವಂಥ
ಎಳೆಸೀತು
ಎಳ್ಳಷ್ಟೂ
ಎಷ್ಟಕ್ಕೂ
ಎಷ್ಟರ
ಎಷ್ಟ-ರ-ಮಟ್ಟಿಗೆ
ಎಷ್ಟು
ಎಷ್ಟೆಂದರೆ
ಎಷ್ಟೆಲ್ಲ
ಎಷ್ಟೆಷ್ಟೋ
ಎಷ್ಟೇ
ಎಷ್ಟೊಂದನ್ನು
ಎಷ್ಟೊಂದು
ಎಷ್ಟೋ
ಎಷ್ಟೋಬಾರಿ
ಎಸ-ಗಿ-ದ-ವ-ನನ್ನು
ಎಸ-ಗು-ವರು
ಎಸೆದ
ಎಸೆದಂತೆ
ಎಸೆ-ದಿದ್ದರೆ
ಎಸೆದು
ಎಸೆ-ಯ-ಬಲ್ಲರೇ
ಎಸೆಯಲು
ಎಸೆಯಲೂ
ಎಸೆ-ಯುತ್ತಾರೆ
ಎಸೆ-ಯುತ್ತಿದ್ದ
ಎಸೆ-ಯು-ವಂತೆ
ಎಸ್
ಎಸ್ಎ-ಸ್ಎಲ್ಸಿ-ಯಲ್ಲಿ
ಎಸ್ಟೇ-ಟಿ-ನಲ್ಲಿ
ಎಸ್ಪೆರಿನ್
ಏಕ
ಏಕ-ಕಾ-ಲ-ದಲ್ಲಿ
ಏಕತಾ
ಏಕ-ತಾ-ನ-ತೆಯ
ಏಕತೆ
ಏಕತೆಯ
ಏಕ-ತೆ-ಯನ್ನು
ಏಕ-ತೆ-ಯನ್ನುಂಟು-ಮಾ-ಡಲು
ಏಕ-ತೆ-ಯನ್ನೂ
ಏಕತೆಯು
ಏಕತೆಯೇ
ಏಕನಿಷ್ಠೆ
ಏಕಪ್ಪಾ
ಏಕಪ್ರ-ಕಾರ
ಏಕಪ್ರ-ಕಾ-ರ-ವಾಗಿ
ಏಕಮಾತ್ರ
ಏಕ-ಮಾತ್ರ-ಪುತ್ರ-ನಾ-ಗಿದ್ದೆ
ಏಕ-ಮು-ಖ-ವಾ-ದು-ದಲ್ಲ
ಏಕ-ಮು-ಖ-ವಾಗಿ
ಏಕಮೇವ
ಏಕ-ಮೇ-ವಾದ್ವಿ-ತೀ-ಯ-ನ-ವನು
ಏಕ-ಮೇ-ವಾದ್ವಿ-ತೀ-ಯ-ವಾದ
ಏಕವೂ
ಏಕಾಂಗಿ
ಏಕಾಂಗಿ-ಯಾಗಿ
ಏಕಾಂತ
ಏಕಾಂತ-ದಲ್ಲಿ
ಏಕಾಂತಪ್ರಿ-ಯತೆ
ಏಕಾಂತ-ಭಾ-ವ-ದಿಂದ
ಏಕಾಂತ-ವನ್ನು
ಏಕಾ-ಕಿ-ಯಾಗಿ
ಏಕಾಗ್ರ-ಚಿತ್ತ
ಏಕಾಗ್ರ-ಚಿತ್ತ-ದಿಂದ
ಏಕಾಗ್ರ-ಚಿತ್ತ-ನಾಗಿ
ಏಕಾಗ್ರತೆ
ಏಕಾಗ್ರ-ತೆ-ಗ-ಳನ್ನು
ಏಕಾಗ್ರ-ತೆ-ಗ-ಳಿಂದಲೇ
ಏಕಾಗ್ರ-ತೆ-ಗಳು
ಏಕಾಗ್ರ-ತೆಯ
ಏಕಾಗ್ರ-ತೆ-ಯಿಂದ
ಏಕಿಷ್ಟು
ಏಕೀ-ಭ-ವಿ-ಸಿ-ದಾಗ
ಏಕೆ
ಏಕೆಂದರೆ
ಏಕೆ-ಸಂಪೂರ್ಣ
ಏಕೈಕ
ಏಕೈ-ಕ-ಮತ
ಏಕೋ
ಏಜಂಟ-ರು-ಗಳು
ಏಜೆಂಟ-ರನ್ನು
ಏಜೆಂಟರು
ಏಜೆಂಟ್
ಏಟನ್ನು
ಏಟಾಗಿ
ಏಟಿನ
ಏಟಿನಿಂದ
ಏಟು
ಏಣಿ
ಏತ
ಏತಕ್ಕಾಗಿ
ಏನ-ಡ-ಗಿದೆ
ಏನನ್ನಾ
ಏನನ್ನಾ-ದರೂ
ಏನನ್ನು
ಏನನ್ನೂ
ಏನನ್ನೋ
ಏನರ್ಥ
ಏನಾ-ಗ-ಬ-ಹು-ದೆಂಬು-ದನ್ನು
ಏನಾಗಿತ್ತು
ಏನಾಗಿದೆ
ಏನಾ-ಗಿ-ರು-ವನೋ
ಏನಾ-ಗಿ-ರು-ವೆವೋ
ಏನಾ-ಗುತ್ತದೆ
ಏನಾದರೂ
ಏನಾ-ದ-ರೊಂದಿಷ್ಟು
ಏನಾ-ದ-ರೊಂದು
ಏನಾದೀತು
ಏನಾ-ದೀ-ತೆಂದು
ಏನಾಶ್ಚರ್ಯ
ಏನಿದು
ಏನಿದೆ
ಏನಿದ್ದರೂ
ಏನು
ಏನು-ಎಂಬು-ದನ್ನು
ಏನುಬೇಕು
ಏನೂ
ಏನೆಂದರೂ
ಏನೆಂದು
ಏನೆಂದೆ
ಏನೆಂಬ
ಏನೆಂಬು-ದನ್ನು
ಏನೆಂಬು-ದನ್ನು
ಏನೆನ್ನು-ವರು
ಏನೆಲ್ಲ
ಏನೆಲ್ಲಾ
ಏನೇ
ಏನೇನು
ಏನೇನೂ
ಏನೇನೋ
ಏನೋ
ಏನ್ಶಿಯಂಟ್
ಏರಬಲ್ಲ
ಏರ-ಬಲ್ಲರು
ಏರ-ಬ-ಹು-ದಾದ
ಏರ-ಬ-ಹುದು
ಏರ-ಬೇ-ಕಾ-ದರೆ
ಏರಬೇಕು
ಏರಲು
ಏರಿ
ಏರಿದ
ಏರಿದರು
ಏರಿದೆ
ಏರಿದ್ದ
ಏರಿಯನ್
ಏರಿ-ರುತ್ತದೆ
ಏರಿ-ಳಿ-ತ-ಗ-ಳನ್ನು
ಏರಿ-ಳಿ-ತ-ಗ-ಳಿಂದ
ಏರಿ-ಳಿ-ತ-ಗ-ಳೊಂದಿಗೆ
ಏರಿಸಲು
ಏರಿಸಿ
ಏರಿಸಿತು
ಏರಿ-ಸಿ-ದರೆ
ಏರಿ-ಸಿದ್ದರು
ಏರಿ-ಸಿದ್ದವು
ಏರಿಸುತ್ತ
ಏರಿಸುವ
ಏರಿ-ಸು-ವು-ದಾ-ದರೂ
ಏರಿ-ಸು-ವು-ದೆಂಬುದು
ಏರುತ್ತದೆ
ಏರುತ್ತ-ಲಿದೆ
ಏರುತ್ತಿ-ರುವ
ಏರುಪೇರು
ಏರು-ಪೇ-ರು-ಗ-ಳನ್ನೂ
ಏರು-ಪೇ-ರು-ಗ-ಳಿಂದ
ಏರು-ಪೇ-ರು-ಗ-ಳಿಲ್ಲದೆ
ಏರು-ಪೇ-ರು-ಗಳು
ಏರು-ವು-ದುಂಟು
ಏರೊತ್ತಡ
ಏರೊತ್ತ-ಡವೇ
ಏರ್ಪ-ಡಿ-ಸಿದ
ಏರ್ಪಾ-ಡಾ-ಗ-ಬೇ-ಕೆಂದಾಗ
ಏಳದಂತೆ
ಏಳ-ನೆ-ಯದು
ಏಳನೇ
ಏಳಿ
ಏಳಿಗೆ
ಏಳಿಗೆಗೆ
ಏಳಿಗೆಯ
ಏಳು
ಏಳುನೂರು
ಏಳುಪ್ರಾರ್ಥ-ನೆ-ಯಿಂದ
ಏಳುವ
ಏಳು-ವಂತಾ-ಗಲಿ
ಏಳು-ವಂತಿಲ್ಲ
ಏಳುವಾಗ
ಏಳುವಿರಿ
ಏಳು-ವು-ದೆಂದೂ
ಏಳೆಂಟು
ಏಳ್ಗೆ
ಏಳ್ಗೆಗಾಗಿ
ಏಳ್ಗೆಗೆ
ಏಷ್ಯಾ-ದ-ವ-ರು-ಎಲ್ಲರೂ
ಏಸು
ಏಸುಕ್ರಿಸ್ತ
ಏಸುಕ್ರಿಸ್ತನ
ಏಸುಕ್ರಿಸ್ತನು
ಏಸುವಿನ
ಏಸುವು
ಐಕ-ಮತ್ಯ-ದಿಂದ
ಐಕ-ಮತ್ಯ-ವನ್ನು
ಐಗಳ
ಐಗಳಾಗಿ
ಐತಿ-ಹಾ-ಸಿಕ
ಐದ-ನೆ-ಯದೇ
ಐದನೇ
ಐದು
ಐದು-ಅದ್ಭು-ತ-ಘ-ಟ-ನೆ-ಗಳು
ಐನೂ-ರ-ರಷ್ಟು
ಐನೂರು
ಐನ್
ಐನ್ಸ್ಟೀನರ
ಐನ್ಸ್ಟೀ-ನ-ರೆಂದದ್ದು-ಎಲ್ಲ
ಐನ್ಸ್ಟೀನ್
ಐನ್ಸ್ಟೀನ್ರ
ಐನ್ಸ್ಟೈನ್
ಐಬು
ಐರ್ಲೆಂಡಿ-ನಲ್ಲಿದ್ದಾಗ
ಐರ್ಲೆಂಡಿ-ನಿಂದ
ಐವತ್ತರ
ಐವತ್ತ-ರಷ್ಟು
ಐವತ್ತು
ಐವತ್ತೈದು
ಐವತ್ತೊಂಬತ್ತು
ಐವನೋವಾ
ಐಶ್ವರ್ಯ
ಐಶ್ವರ್ಯ-ವಂತ-ನಿದ್ದು
ಐಶ್ವರ್ಯ-ವನ್ನೆಲ್ಲ
ಐಸಾಕ್
ಐಸೆಂಕ್
ಐಹಿಕ
ಒಂಟಿಯಾಗಿ
ಒಂದಂತೂ
ಒಂದಂಶ
ಒಂದಂಶ-ವನ್ನು
ಒಂದಂಶ-ವಾ-ದರೂ
ಒಂದಂಶ-ವಿದ್ದರೂ
ಒಂದಕ್ಕೆ
ಒಂದಕ್ಷರ
ಒಂದಗುಳು
ಒಂದನ್ನು
ಒಂದರ
ಒಂದರಂತೆ
ಒಂದರಲ್ಲಿ
ಒಂದರಲ್ಲೇ
ಒಂದರಷ್ಟು
ಒಂದ-ರಿಂದ-ಲಾ-ದರೂ
ಒಂದರೆಡು
ಒಂದರ್ಧ
ಒಂದಲ್ಲ
ಒಂದಲ್ಲಾ
ಒಂದಲ್ಲೊಂದು
ಒಂದಷ್ಟು
ಒಂದಾ
ಒಂದಾಗಿ
ಒಂದಾಗಿಲ್ಲ
ಒಂದಾ-ಗಿ-ಸಿದೆ
ಒಂದಾ-ಗುತ್ತವೆ
ಒಂದಾ-ಗುತ್ತೀರಿ
ಒಂದಾದ
ಒಂದಾನೊಂದು
ಒಂದಿದೆ
ಒಂದಿ-ದೆ-ಅದೇ
ಒಂದಿಲ್ಲ
ಒಂದಿಲ್ಲೊಂದು
ಒಂದಿಷ್ಟು
ಒಂದು
ಒಂದು-ಒ-ಳಿ-ತಿನ
ಒಂದುಗೂಡಿ
ಒಂದು-ಗೂ-ಡಿ-ಸುತ್ತದೆ
ಒಂದು-ಗೂ-ಡಿ-ಸುವ
ಒಂದು-ಗೂ-ಡು-ವುವು
ಒಂದುದಿನ
ಒಂದುಪ್ರ-ಯತ್ನ-ದಿಂದ
ಒಂದು-ಭ-ಗ-ವಂತನ
ಒಂದೂ
ಒಂದೂವರೆ
ಒಂದೆ
ಒಂದೆಡೆ
ಒಂದೆ-ಡೆ-ಯಲ್ಲಿ
ಒಂದೆ-ರ-ಡನ್ನು
ಒಂದೆರಡು
ಒಂದೆರೆಡು
ಒಂದೇ
ಒಂದೇಕರ್ಮ
ಒಂದೇ-ನೋ-ಡು-ವ-ವರ
ಒಂದೇ-ವರ್ತ-ಮಾನ
ಒಂದೇ-ಸ-ಮನೆ
ಒಂದೊಂದು
ಒಂದೊಂದೇ
ಒಂದೋ
ಒಂಬತ್ತನೇ
ಒಂಬತ್ತರ
ಒಂಬತ್ತು
ಒಂಬೈನೂರು
ಒಂಭತ್ತ-ನೆಯ
ಒಂಭತ್ತನೇ
ಒಕ್ಕೊ-ರ-ಲಿ-ನಿಂದ
ಒಕ್ಕೊರಳ
ಒಗಟನ್ನೋ
ಒಗಟೇ
ಒಗೆದುಕೊ
ಒಗೆ-ದು-ಕೊಳ್ಳುತ್ತಾರೆ
ಒಗೆಯಲು
ಒಗೆ-ಯುತ್ತಿದ್ದ
ಒಗ್ಗಟ್ಟಾಗಿ
ಒಗ್ಗಟ್ಟಿನ
ಒಗ್ಗಟ್ಟಿ-ನಿಂದ
ಒಗ್ಗಟ್ಟಿ-ನಿಂದಲೇ
ಒಗ್ಗಟ್ಟು
ಒಗ್ಗದ
ಒಗ್ಗಿ
ಒಗ್ಗಿ-ಕೊಂಡಿದ್ದ
ಒಗ್ಗುವಂತೆ
ಒಟ್ಟಾಗಿ
ಒಟ್ಟಾ-ಗಿ-ರಿ-ಸಿದ
ಒಟ್ಟಿಗೆ
ಒಟ್ಟಿನಲ್ಲಿ
ಒಟ್ಟು
ಒಟ್ಟುಸಂಖ್ಯೆ
ಒಡಕನ್ನು
ಒಡಕು
ಒಡನಾಟ
ಒಡ-ನಾ-ಡಿ-ಗಳು
ಒಡನೆಯೇ
ಒಡಲನ್ನು
ಒಡಲಲ್ಲಿ
ಒಡ-ವೆ-ಗ-ಳಿಲ್ಲ
ಒಡ-ವೆ-ಗ-ಳಂತೆ
ಒಡ-ವೆ-ಗ-ಳನ್ನು
ಒಡ-ವೆ-ವಸ್ತ್ರ-ಗ-ಳಲ್ಲಿಲ್ಲ
ಒಡೆತನ
ಒಡೆದರು
ಒಡೆದು
ಒಡೆ-ಯ-ನಾ-ಗಿ-ರು-ವಷ್ಟು
ಒಡೆ-ಯುತ್ತದೆ
ಒಡ್ಡಿ
ಒಡ್ಡುತ್ತದೆ
ಒಡ್ಡುತ್ತವೆ
ಒಣ
ಒಣ-ಗ-ತೊ-ಡ-ಗಿತು
ಒಣಗಿದ
ಒಣಜಂಭ
ಒಣರೊಟ್ಟಿ
ಒಣ-ಹೆಮ್ಮೆ-ಯಿಂದ
ಒತ್ತಡ
ಒತ್ತ-ಡ-ದಿಂದಲೂ
ಒತ್ತ-ಡಕ್ಕೊ-ಳ-ಗಾ-ಗಿವೆ
ಒತ್ತ-ಡ-ಗ-ಳನ್ನು
ಒತ್ತ-ಡ-ಗ-ಳಿಂದ
ಒತ್ತ-ಡ-ಗ-ಳಿಗೆ
ಒತ್ತ-ಡ-ಗ-ಳಿ-ಗೊ-ಳ-ಗಾದ
ಒತ್ತ-ಡ-ದಿಂದ
ಒತ್ತ-ಡ-ದಿಂದಾಗಿ
ಒತ್ತ-ಡ-ಯಾರೋ
ಒತ್ತ-ಡ-ವಿ-ರ-ಲಿಲ್ಲ
ಒತ್ತಡವೂ
ಒತ್ತ-ಡ-ಸುತ್ತ-ಲಿಂದಲೂ
ಒತ್ತ-ರಿ-ಸಿ-ಬಂದಾಗ
ಒತ್ತಾಯ
ಒತ್ತಾಯಕ್ಕೂ
ಒತ್ತಾಯಕ್ಕೆ
ಒತ್ತಾಯದ
ಒತ್ತಾ-ಯ-ದಿಂದ
ಒತ್ತಾ-ಯಿ-ಸಿದೆ
ಒತ್ತಾಸೆ
ಒತ್ತಾಸೆಯ
ಒತ್ತಿ
ಒತ್ತಿ-ದು-ದ-ರಿಂದಲೇ
ಒತ್ತಿ-ದೊ-ಡನೆ
ಒತ್ತುತ್ತವೆ
ಒಥೆಲೋ
ಒದಗಿತು
ಒದಗಿದ
ಒದ-ಗಿ-ದರೆ
ಒದ-ಗಿ-ಬಂದಾಗ
ಒದ-ಗಿ-ರುವ
ಒದ-ಗಿ-ಸ-ಬ-ಹು-ದಾ-ದರೂ
ಒದ-ಗಿ-ಸ-ಲಾ-ಗಿತ್ತು
ಒದ-ಗಿ-ಸ-ಲಾ-ರರು
ಒದ-ಗಿ-ಸಿ-ದಂತೆ
ಒದ-ಗಿ-ಸಿ-ದರೂ
ಒದ-ಗಿ-ಸಿ-ಯಾವು
ಒದ-ಗಿ-ಸಿ-ರು-ವು-ದಿಲ್ಲ-ವೆಂಬುದು
ಒದ-ಗಿ-ಸುತ್ತವೆ
ಒದ-ಗಿ-ಸು-ವ-ವ-ರನ್ನು
ಒದ-ಗಿ-ಸು-ವು-ದಿಲ್ಲ
ಒದ-ಗೀ-ತು-ಎಂಬು-ದನ್ನು
ಒದು
ಒದೆ
ಒದೆತ
ಒದೆಯಲು
ಒದ್ದಾಟ
ಒದ್ದಾ-ಟ-ಪ-ಡುತ್ತಿದ್ದ
ಒದ್ದಾ-ಟ-ವಲ್ಲ
ಒದ್ದಾ-ಟ-ವಿಲ್ಲದೆ
ಒದ್ದಾ-ಡುತ್ತಿದ್ದೇವೆ
ಒದ್ದಾಡುವ
ಒದ್ದೆ
ಒದ್ದೆ-ಮಾ-ಡಿ-ಕೊಳ್ಳುತ್ತೇನೆ
ಒದ್ದೆ-ಯಾ-ದವು
ಒಪ್ಪಂದ
ಒಪ್ಪದ
ಒಪ್ಪ-ದ-ವನು
ಒಪ್ಪ-ದ-ವರ
ಒಪ್ಪ-ದಿದ್ದ-ರೂ-ಮ-ರ-ಣಾ-ನಂತರ
ಒಪ್ಪ-ದಿದ್ದರೆ
ಒಪ್ಪ-ಬೇ-ಕಾ-ಯಿತು
ಒಪ್ಪಲಾರ
ಒಪ್ಪಲಿ
ಒಪ್ಪಲಿಲ್ಲ
ಒಪ್ಪಲು
ಒಪ್ಪಲೇ
ಒಪ್ಪ-ಲೇ-ಬೇ-ಕಾದ
ಒಪ್ಪಿಕೊ
ಒಪ್ಪಿಕೊಂಡ
ಒಪ್ಪಿಕೊಂಡು
ಒಪ್ಪಿಕೊಂಡೆ
ಒಪ್ಪಿಕೊಳ್ಳ
ಒಪ್ಪಿ-ಕೊಳ್ಳುತ್ತಾನೆ
ಒಪ್ಪಿ-ಕೊಳ್ಳುವ
ಒಪ್ಪಿಗೆ
ಒಪ್ಪಿದರೆ
ಒಪ್ಪಿಲ್ಲ-ವೆಂದೇ
ಒಪ್ಪಿ-ಸ-ಬೇಕು
ಒಪ್ಪಿ-ಸ-ಬೇ-ಕೆಂಬುದು
ಒಪ್ಪಿ-ಸಿ-ಬಿ-ಡುತ್ತಿದ್ದ
ಒಪ್ಪುತ್ತಾನೆ
ಒಪ್ಪುತ್ತಿಲ್ಲ
ಒಪ್ಪುವ
ಒಪ್ಪುವಂತೆ
ಒಪ್ಪು-ವ-ವರೇ
ಒಪ್ಪು-ವು-ದಿಲ್ಲ
ಒಪ್ಪುವುದು
ಒಬ್ಬ
ಒಬ್ಬನ
ಒಬ್ಬನನ್ನು
ಒಬ್ಬನಾಗಿ
ಒಬ್ಬನಿಗೆ
ಒಬ್ಬನು
ಒಬ್ಬನೂ
ಒಬ್ಬನೇ
ಒಬ್ಬ-ನೇ-ಎಲ್ಲರೂ
ಒಬ್ಬರ
ಒಬ್ಬರಂತೆ
ಒಬ್ಬರನ್ನು
ಒಬ್ಬರನ್ನೂ
ಒಬ್ಬ-ರನ್ನೊಬ್ಬರು
ಒಬ್ಬರಲ್ಲ
ಒಬ್ಬರಾದ
ಒಬ್ಬರಿಲ್ಲ
ಒಬ್ಬರು
ಒಬ್ಬರೋ
ಒಬ್ಬಳಾಗಿ
ಒಬ್ಬಳೇ
ಒಬ್ಬಾಕೆ
ಒಬ್ಬಾಕೆಯ
ಒಬ್ಬಾ-ಕೆ-ಯನ್ನು
ಒಬ್ಬಾತ
ಒಬ್ಬಾತನ
ಒಬ್ಬಾ-ತ-ನನ್ನು
ಒಬ್ಬಿಬ್ಬರು
ಒಬ್ಬೊಬ್ಬರು
ಒಬ್ಬೊಬ್ಬರೂ
ಒಬ್ಬೊಬ್ಬರೇ
ಒಬ್ಸೆಶ್ಶನ್
ಒಮ್ಮ-ತ-ವನ್ನು
ಒಮ್ಮ-ತ-ವನ್ನುಂಟು-ಮಾ-ಡು-ವಂಥ
ಒಮ್ಮ-ನಸ್ಸಿನ
ಒಮ್ಮ-ನಸ್ಸಿ-ನಿಂದ
ಒಮ್ಮಲೇ
ಒಮ್ಮಿಂದೊಮ್ಮೆಗೆ
ಒಮ್ಮಿಂದೊಮ್ಮೆಗೇ
ಒಮ್ಮು-ಖ-ವಾಗಿ
ಒಮ್ಮೆ
ಒಮ್ಮೆಗೇ
ಒಮ್ಮೆಯೂ
ಒಮ್ಮೊಮ್ಮೆ
ಒಯ್ಯುತ್ತದೆ
ಒಯ್ಯುತ್ತಾ-ರಷ್ಟೆ
ಒರಟಾಗಿ
ಒರ-ಟಾ-ಗುತ್ತ
ಒರಟಾದ
ಒರಿಯಾ
ಒರಿಸ್ಸಾ
ಒರಿಸ್ಸಾದ
ಒರೆ
ಒರೆ-ಗಲ್ಲಿಗೆ
ಒರೆ-ಗಲ್ಲಿ-ನಲ್ಲಿ
ಒರೆಗಲ್ಲು
ಒರೆಯಿಂದ
ಒಲವನ್ನೇ
ಒಲವಿನ
ಒಲವು
ಒಲವೇ
ಒಲಿದ
ಒಲಿ-ಯು-ವರು
ಒಲಿ-ವನ್ಸು-ಲ-ಭನೊ
ಒಲಿಸಲು
ಒಲಿ-ಸ-ಲು-ಬ-ಹುದು
ಒಲಿ-ಸಿ-ಕೊಳ್ಳುತ್ತೇನೆ
ಒಲುಮೆ
ಒಲ್ಮೆ-ಇ-ವ-ರಿಬ್ಬರ
ಒಲ್ಲ
ಒಲ್ಲದ
ಒಲ್ಲೆ-ನೆನ್ನದೆ
ಒಳ
ಒಳಗಡೆ
ಒಳ-ಗ-ಡೆಯ
ಒಳಗಣ
ಒಳ-ಗಾ-ಗದ
ಒಳ-ಗಾ-ಗುತ್ತೇವೆ
ಒಳ-ಗಾ-ಗು-ವರು
ಒಳಗಾದ
ಒಳ-ಗಾ-ದ-ವನು
ಒಳಗಿನ
ಒಳ-ಗಿ-ನಿಂದಲೇ
ಒಳ-ಗಿ-ರುವ
ಒಳಗುಟ್ಟು
ಒಳ-ಗುಟ್ಟೇನು
ಒಳಗೂ
ಒಳಗೆ
ಒಳಗೊಂಡ
ಒಳ-ಗೊಂಡಂತಿದೆ
ಒಳ-ಗೊಂಡಿದೆ
ಒಳಗೊಂಡು
ಒಳ-ಗೊ-ಳಗೆ
ಒಳ-ಗೊ-ಳಗೇ
ಒಳ-ಜ-ಗಳ
ಒಳ-ಜ-ಗ-ಳ-ಗ-ಳನ್ನು
ಒಳ-ತೋ-ಟಿಯ
ಒಳದಾರಿ
ಒಳ-ದಾ-ರಿ-ಗಳೇ
ಒಳ-ನುಗ್ಗು-ವುದೋ
ಒಳ-ಪಟ್ಟಾಗ
ಒಳ-ಪ-ಡದ
ಒಳ-ಪ-ಡಿಸಿ
ಒಳಭಾಗ
ಒಳ-ಮ-ನಸ್ಸು
ಒಳ-ರ-ಚ-ನೆ-ಯನ್ನು
ಒಳ-ಹೊ-ರ-ಗು-ಗ-ಳನ್ನು
ಒಳ-ಹೊ-ರಗೆ
ಒಳಿತನ್ನು
ಒಳಿತನ್ನೂ
ಒಳಿತನ್ನೇ
ಒಳಿತಲ್ಲ
ಒಳಿ-ತಲ್ಲವೇ
ಒಳಿ-ತಾ-ಗಲು
ಒಳಿ-ತಾ-ಗು-ವುದು
ಒಳಿ-ತಾ-ಗು-ವು-ದೆಂದು
ಒಳಿ-ತಾ-ದು-ದೆಲ್ಲ-ವನ್ನೂ
ಒಳಿ-ತಾ-ಯಿತು
ಒಳಿ-ತಿ-ಗಾಗಿ
ಒಳಿತು
ಒಳಿತೇ
ಒಳಿತೋ
ಒಳ್ಳೆಯ
ಒಳ್ಳೆ-ಯ-ತನ
ಒಳ್ಳೆ-ಯ-ತ-ನ-ದಲ್ಲಿ
ಒಳ್ಳೆ-ಯ-ತ-ನ-ವನ್ನ-ವ-ಲಂಬಿ-ಸದೆ
ಒಳ್ಳೆ-ಯ-ತ-ನ-ವನ್ನು
ಒಳ್ಳೆ-ಯ-ದನ್ನು
ಒಳ್ಳೆ-ಯ-ದನ್ನುಂಟು-ಮಾ-ಡು-ವುದೋ
ಒಳ್ಳೆ-ಯ-ದನ್ನೂ
ಒಳ್ಳೆ-ಯ-ದನ್ನೇ
ಒಳ್ಳೆ-ಯ-ದಾಗಿ
ಒಳ್ಳೆ-ಯ-ದಾ-ಗಿದ್ದರೆ
ಒಳ್ಳೆ-ಯ-ದಾ-ಗಿ-ಬಿಟ್ಟಿತು
ಒಳ್ಳೆ-ಯ-ದಾ-ಗಿಯೇ
ಒಳ್ಳೆ-ಯ-ದಾ-ಗುತ್ತೆ
ಒಳ್ಳೆ-ಯ-ದಿದೆ
ಒಳ್ಳೆಯದು
ಒಳ್ಳೆ-ಯ-ವ-ನನ್ನಾಗಿ
ಒಳ್ಳೆ-ಯ-ವ-ನಾ-ಗ-ಬೇ-ಕೆಂಬ
ಒಳ್ಳೆ-ಯ-ವ-ನಾ-ಗಲಿ
ಒಳ್ಳೆ-ಯ-ವ-ನಾ-ಗಿ-ಬಿ-ಡುತ್ತಾ-ನೆಂದು
ಒಳ್ಳೆ-ಯ-ವನು
ಒಳ್ಳೆ-ಯ-ವ-ನೆಂದು
ಒಳ್ಳೆ-ಯ-ವನೇ
ಒಳ್ಳೆ-ಯ-ವ-ರಾ-ಗಲು
ಒಳ್ಳೆ-ಯ-ವ-ರಾ-ದರೆ
ಒಳ್ಳೆ-ಯ-ವ-ರಿಗೆ
ಒಳ್ಳೆ-ಯ-ವರು
ಒಳ್ಳೆ-ಯ-ವರೂ
ಒಳ್ಳೆ-ಯ-ವ-ರೆ-ಆ-ದರೆ
ಒಳ್ಳೆ-ಯ-ವರೇ
ಒಸ್ಟ್ರೆಂಡರ್
ಒಸ್ಲರ್
ಒಹೈಯೋ
ಓ
ಓಗೊಟ್ಟ
ಓಗೊಟ್ಟು
ಓಗೊಟ್ಟೆ-ನಷ್ಟೆ
ಓಗೊ-ಡುತ್ತಾನೆ
ಓಟ
ಓಟ-ಕೀ-ಳುತ್ತ-ದೆ-ಎಂದೂ
ಓಟದ
ಓಟು
ಓಟು-ಗ-ಳಿಸಿ
ಓಡ-ತೊ-ಡ-ಗಿ-ದರು
ಓಡಬೇಡಿ
ಓಡ-ಲಾ-ರಂಭಿ-ಸುತ್ತದೆ
ಓಡಲು
ಓಡಾಟ
ಓಡಾಡ
ಓಡಾ-ಡ-ದಿದ್ದ-ವ-ರನ್ನು
ಓಡಾಡಿ
ಓಡಾಡಿತು
ಓಡಾಡಿದ
ಓಡಾ-ಡಿ-ದಳು
ಓಡಾ-ಡುತ್ತಾರೆ
ಓಡಿ
ಓಡಿದ
ಓಡಿದರೆ
ಓಡಿದ್ದಾ-ದರೂ
ಓಡಿ-ಬ-ರ-ಲಾ-ರಂಭಿ-ಸಿತು
ಓಡಿ-ಬ-ರುತ್ತಿ-ರುವ
ಓಡಿ-ಸ-ಬೇ-ಕೆನ್ನ-ತೊ-ಡ-ಗಿ-ದರು
ಓಡಿಸಲು
ಓಡಿ-ಸ-ಲೇ-ಬೇಕು
ಓಡಿ-ಸುತ್ತಿದೆ
ಓಡಿಸುವ
ಓಡಿ-ಹೋ-ಗ-ದಂತೆ
ಓಡಿ-ಹೋ-ಗಲು
ಓಡುತ್ತ
ಓಡುತ್ತದೆ
ಓಡುತ್ತ-ಲಿದೆ
ಓಡುತ್ತಿತ್ತು
ಓಡುತ್ತಿದ್ದಾನೆ
ಓಡುತ್ತಿದ್ದಾ-ರಲ್ಲ
ಓಡುತ್ತಿದ್ದೆ
ಓಡುತ್ತಿ-ರುವ
ಓಡುವ
ಓಡು-ವಂತಿದೆ
ಓಡು-ವು-ದಕ್ಕಿಂತ
ಓಡು-ವು-ದ-ರಿಂದಲ್ಲ
ಓಡುವುದೇ
ಓಡೋಡಿ
ಓದ
ಓದ-ತೊ-ಡ-ಗಿದ
ಓದದೇ
ಓದಬಲ್ಲ
ಓದ-ಬ-ಹುದು
ಓದ-ಬೇ-ಕಾದ
ಓದಬೇಕು
ಓದಲು
ಓದಿ
ಓದಿಕೊಂಡ
ಓದಿ-ಕೊಂಡರೆ
ಓದಿ-ಕೊಂಡ-ವರು
ಓದಿ-ಕೊಂಡ-ವಳು
ಓದಿಕೊಂಡು
ಓದಿದ
ಓದಿದರೆ
ಓದಿ-ದ-ವನು
ಓದಿ-ದ-ವ-ರಿಗೆ
ಓದಿದಾಗ
ಓದಿದೆ
ಓದಿದ್ದ
ಓದಿದ್ದ-ನಷ್ಟೆ
ಓದಿದ್ದರೂ
ಓದಿದ್ದೆಲ್ಲ
ಓದಿನಲ್ಲಿ
ಓದಿ-ರುತ್ತೀರಿ
ಓದು
ಓದುಗರ
ಓದು-ಗ-ರನ್ನು
ಓದು-ಗ-ರಲ್ಲಿ
ಓದು-ಗ-ರಿ-ಗೊಂದು
ಓದುಗರು
ಓದುತ್ತಿ
ಓದುತ್ತಿದ್ದ
ಓದುತ್ತಿದ್ದರೆ
ಓದುತ್ತಿದ್ದಿ
ಓದುತ್ತಿ-ರುವ
ಓದುತ್ತಿ-ರು-ವಾಗ
ಓದುತ್ತೇವೆ
ಓದುತ್ತೋ-ದುತ್ತ
ಓದು-ಬ-ರಹ
ಓದು-ಬ-ರ-ಹದ
ಓದುವ
ಓದು-ವ-ವರ
ಓದುವಾಗ
ಓದು-ವು-ದನ್ನು
ಓದುವುದೇ
ಓರಿ-ಯೆಂಟೆಡ್
ಓರೆ-ಕೋ-ರೆ-ಯಲ್ಲಿದೆ
ಓರ್ವ
ಓಹ್
ಔಚಿತ್ಯ
ಔಚಿತ್ಯ-ಪೂರ್ಣ
ಔತ-ಣ-ದಲ್ಲಿ
ಔದಾರ್ಯ
ಔದಾರ್ಯದ
ಔದಾರ್ಯ-ದಿಂದ
ಔದ್ಯಮಿಕ
ಔನ್ನತ್ಯ
ಔನ್ನತ್ಯಕ್ಕೂ
ಔರಂಗ-ಜೇ-ಬನ
ಔಷಧ
ಔಷ-ಧಕ್ಕಿಂತಲೂ
ಔಷ-ಧ-ಗಳ
ಔಷ-ಧ-ಗ-ಳನ್ನು
ಔಷ-ಧ-ಗ-ಳಾ-ಗಲೀ
ಔಷ-ಧ-ಗ-ಳಿಂದಲೋ
ಔಷ-ಧ-ಗ-ಳಿಗೆ
ಔಷ-ಧ-ಗಳು
ಔಷ-ಧ-ಗಳೂ
ಔಷ-ಧ-ಪಥ್ಯ-ಗ-ಳನ್ನು
ಔಷ-ಧ-ವನ್ನು
ಔಷ-ಧ-ವನ್ನೂ
ಔಷ-ಧ-ವಲ್ಲ
ಔಷಧವೂ
ಔಷ-ಧ-ವೆಲ್ಲಿದೆ
ಔಷಧವೇ
ಔಷ-ಧ-ವೇ-ನನ್ನೂ
ಔಷ-ಧ-ವೇನೂ
ಔಷ-ಧ-ಸೂ-ಚಿ-ಯನ್ನು
ಔಷ-ಧ-ಸೇ-ವನೆ
ಔಷ-ಧಾ-ಲ-ಯ-ಗಳೂ
ಔಷಧಿ
ಔಷ-ಧಿ-ಗ-ಳನ್ನು
ಔಷಧೋಪ
ಔಷ-ಧೋ-ಪ-ಕ-ರ-ಣ-ಗ-ಳನ್ನಾ-ಗಲೀ
ಔಷ-ಧೋ-ಪ-ಚಾರ
ಔಷ-ಧೋ-ಪ-ಚಾ-ರ-ಗ-ಳನ್ನು
ಔಷ-ಧೋ-ಪ-ಚಾ-ರ-ಗ-ಳಿಂದ
ಔಷ-ಧೋ-ಪ-ಚಾ-ರದ
ಔಷ-ಧೋ-ಪ-ಚಾ-ರ-ವನ್ನು
ಕಂಕ-ಣ-ಬದ್ಧ-ರಾ-ದ-ವರೇ
ಕಂಗಾಲಾ
ಕಂಗಾಲಾಗಿ
ಕಂಗಾ-ಲಾ-ಗಿದ್ದಾರೆ
ಕಂಗಾ-ಲಾ-ಗಿದ್ದೀ-ರೆಂಬುದು
ಕಂಗಾ-ಲಾ-ಗುತ್ತಾರೆ
ಕಂಗಾ-ಲಾ-ಗುತ್ತಿದ್ದೇನೆ
ಕಂಗಾಲಾದ
ಕಂಗೆಟ್ಟ
ಕಂಗೆಟ್ಟು
ಕಂಗೆ-ಡ-ದಂತೆ
ಕಂಗೆಡದೆ
ಕಂಗೆ-ಡಿ-ಸ-ಲಾ-ರವು
ಕಂಗೆ-ಡಿ-ಸುತ್ತದೆ
ಕಂಗೆ-ಡಿ-ಸುತ್ತವೆ
ಕಂಗೆ-ಡುತ್ತಾನೆ
ಕಂಗೆಡುವ
ಕಂಗೊ-ಳಿ-ಸಿದ
ಕಂಗೊ-ಳಿ-ಸಿ-ದರು
ಕಂಗೊ-ಳಿ-ಸುತ್ತಾನೆ
ಕಂಗೊ-ಳಿ-ಸುತ್ತಿದ್ದಾ-ನೆಂದು
ಕಂಗೊ-ಳಿ-ಸುವ
ಕಂಗೊ-ಳಿ-ಸು-ವ-ವ-ರೆಲ್ಲರ
ಕಂಟ-ಕ-ಗ-ಳನ್ನು
ಕಂಟ-ಕ-ಗಳೂ
ಕಂಟ-ಕ-ಗಳೇ
ಕಂಟ-ಕಪ್ರಾ-ಯ-ರಾ-ಗಿದ್ದು
ಕಂಠ
ಕಂಠಪಾಠ
ಕಂಠವನ್ನು
ಕಂಠ-ಶೋ-ಷಣೆ
ಕಂಠಸ್ಥ-ವಾ-ಗಿ-ಸಿ-ಕೊಂಡು
ಕಂಡ
ಕಂಡಂತೆ
ಕಂಡಂದಿ-ನಿಂದ
ಕಂಡಕ್ಟ-ರು-ಗ-ಳನ್ನು
ಕಂಡದ್ದನ್ನು
ಕಂಡದ್ದು
ಕಂಡ-ರಿ-ಯದ
ಕಂಡ-ರಿ-ಯ-ದಿದ್ದರೆ
ಕಂಡರು
ಕಂಡರೂ
ಕಂಡರೆ
ಕಂಡಳು
ಕಂಡ-ವ-ನಾ-ದರೂ
ಕಂಡವನು
ಕಂಡ-ವ-ರಲ್ಲ
ಕಂಡ-ವ-ರಿಲ್ಲ
ಕಂಡವರು
ಕಂಡವು
ಕಂಡಷ್ಟೂ
ಕಂಡಾಗ
ಕಂಡಾ-ಗ-ಲೆಲ್ಲ
ಕಂಡಾಗಲೇ
ಕಂಡಿತು
ಕಂಡಿದೆ
ಕಂಡಿದ್ದರು
ಕಂಡಿದ್ದರೆ
ಕಂಡಿದ್ದ-ರೆ-ಅ-ವನು
ಕಂಡಿದ್ದೇನೆ
ಕಂಡಿರದ
ಕಂಡಿ-ರ-ಬೇಕು
ಕಂಡಿಲ್ಲ
ಕಂಡು
ಕಂಡು-ಬ-ರುತ್ತಿ-ದೆಯೇ
ಕಂಡು-ಹಿ-ಡಿ-ಯ-ಲಾ-ರಿರಿ
ಕಂಡು-ಹಿ-ಡಿ-ಯಲು
ಕಂಡು-ಹಿ-ಡಿ-ಯು-ವಂತಿಲ್ಲ
ಕಂಡು-ಕೇ-ಳಿದ
ಕಂಡುಕೊಂಡ
ಕಂಡು-ಕೊಂಡಂತಾ-ಗು-ವುದು
ಕಂಡು-ಕೊಂಡರು
ಕಂಡು-ಕೊಂಡರೆ
ಕಂಡು-ಕೊಂಡಿದ್ದರು
ಕಂಡು-ಕೊಂಡಿದ್ದಾನೆ
ಕಂಡು-ಕೊಂಡಿದ್ದಾರೆ
ಕಂಡು-ಕೊಂಡಿದ್ದಾ-ರೆ-ಕಂಡು-ಕೊಳ್ಳುತ್ತಿದ್ದಾರೆ
ಕಂಡು-ಕೊಂಡಿದ್ದೇನೆ
ಕಂಡು-ಕೊಂಡಿ-ರ-ಲಿಲ್ಲ
ಕಂಡುಕೊಂಡು
ಕಂಡುಕೊಳ್ಳ
ಕಂಡು-ಕೊಳ್ಳ-ಬ-ಹುದು
ಕಂಡು-ಕೊಳ್ಳ-ಬೇಕು
ಕಂಡು-ಕೊಳ್ಳ-ಲಾ-ರದೆ
ಕಂಡು-ಕೊಳ್ಳಲೂ
ಕಂಡುಕೊಳ್ಳಿ
ಕಂಡು-ಕೊಳ್ಳುವ
ಕಂಡು-ದಾ-ದರೂ
ಕಂಡುದು
ಕಂಡುಬಂತು
ಕಂಡುಬಂದ
ಕಂಡು-ಬಂದರೂ
ಕಂಡು-ಬಂದರೆ
ಕಂಡು-ಬಂದ-ವು-ಆತ
ಕಂಡು-ಬಂದಿತು
ಕಂಡು-ಬಂದಿದೆ
ಕಂಡು-ಬಂದಿವೆ
ಕಂಡುಬಂದು
ಕಂಡು-ಬ-ರ-ದಿದ್ದರೆ
ಕಂಡು-ಬ-ರ-ಲಿಲ್ಲ
ಕಂಡು-ಬ-ರುತ್ತದೆ
ಕಂಡು-ಬ-ರುತ್ತ-ಲಿದೆ
ಕಂಡು-ಬ-ರುತ್ತವೆ
ಕಂಡು-ಬ-ರುತ್ತಿದೆ
ಕಂಡು-ಬ-ರುತ್ತಿ-ದೆ-ಯಲ್ಲ
ಕಂಡು-ಬ-ರುತ್ತಿದ್ದಾರೆ
ಕಂಡು-ಬ-ರುವ
ಕಂಡು-ಬ-ರು-ವು-ದ-ರಲ್ಲಿ
ಕಂಡು-ಹಿ-ಡಿದ
ಕಂಡು-ಹಿ-ಡಿ-ದರು
ಕಂಡು-ಹಿ-ಡಿ-ದಾಗ
ಕಂಡು-ಹಿ-ಡಿ-ದಿದ್ದರೂ
ಕಂಡು-ಹಿ-ಡಿ-ದಿದ್ದ-ರೆ-ಗೂ-ಟ-ದಲ್ಲಿ
ಕಂಡು-ಹಿ-ಡಿ-ದಿದ್ದೇನೆ
ಕಂಡು-ಹಿ-ಡಿ-ದಿದ್ದೇವೆ
ಕಂಡು-ಹಿ-ಡಿ-ದಿ-ರು-ವಿರಾ
ಕಂಡು-ಹಿ-ಡಿ-ದಿಲ್ಲ
ಕಂಡು-ಹಿ-ಡಿದು
ಕಂಡು-ಹಿ-ಡಿದೆ
ಕಂಡು-ಹಿ-ಡಿ-ಯ-ಬೇಕು
ಕಂಡು-ಹಿ-ಡಿ-ಯ-ಲಾ-ಗದೆ
ಕಂಡು-ಹಿ-ಡಿ-ಯ-ಲಾ-ಗಿದೆ
ಕಂಡು-ಹಿ-ಡಿ-ಯಲು
ಕಂಡು-ಹಿ-ಡಿ-ಯುತ್ತಾ-ರೆನ್ನಿ
ಕಂಡು-ಹಿ-ಡಿ-ಯುತ್ತಿದ್ದಾರೆ
ಕಂಡು-ಹಿ-ಡಿ-ಯುವ
ಕಂಡು-ಹಿ-ಡಿ-ಯು-ವಂತಿಲ್ಲ
ಕಂಡು-ಹಿ-ಡಿ-ಯು-ವ-ವ-ರೆಗೂ
ಕಂಡು-ಹಿ-ಡಿ-ಯು-ವಾಗ
ಕಂಡೂ
ಕಂಡೆ
ಕಂಡೆನೆಂದೂ
ಕಂಡೆ-ವೆಂತಲೂ
ಕಂಡೇ
ಕಂಡೊಡನೆ
ಕಂಡ್ಯ
ಕಂತು-ಗ-ಳಲ್ಲಿ
ಕಂತೆ
ಕಂತೆಯ
ಕಂದ
ಕಂದ-ಕ-ದಲ್ಲಿ
ಕಂದನ
ಕಂದನನ್ನು
ಕಂದು
ಕಂದು-ಗು-ಲಾಬಿ
ಕಂಪನಿಯ
ಕಂಪನ್ನು
ಕಂಪೆ-ನಿ-ಗ-ಳಿಂದ
ಕಂಪೆನಿಯ
ಕಂಪೆ-ನಿ-ಯ-ವರು
ಕಂಪ್ಯೂ-ಟ-ರನ್ನು
ಕಂಪ್ಯೂ-ಟ-ರನ್ನೋ
ಕಂಪ್ಯೂ-ಟ-ರಿಗೆ
ಕಂಪ್ಯೂ-ಟ-ರಿನ
ಕಂಪ್ಯೂ-ಟ-ರೀ-ಕ-ರ-ಣದ
ಕಂಪ್ಯೂಟರ್
ಕಂಬಗಳ
ಕಂಬನಿ
ಕಂಬ-ನಿ-ಗ-ರೆದ
ಕಂಬ-ನಿ-ಗ-ರೆ-ಯುತ್ತಿದ್ದ-ರ-ವರು
ಕಂಬ-ನಿ-ದುಂಬಿ
ಕಂಬ-ನಿ-ಯನ್ನು
ಕಂಬ-ನಿ-ಯಿಂದ
ಕಂಬಿಗೆ
ಕಂಭಗಳು
ಕಗ್ಗ
ಕಗ್ಗಂಟನ್ನು
ಕಗ್ಗಂಟು
ಕಗ್ಗತ್ತಲ
ಕಗ್ಗವಲ್ಲ
ಕಗ್ಗಾ-ಡಿ-ನಲ್ಲಿ-ರುವ
ಕಗ್ಗಾ-ಡು-ಗ-ಳಲ್ಲಿ
ಕಚ್ಚ
ಕಚ್ಚಬೇಡ
ಕಚ್ಚಲು
ಕಚ್ಚಾ-ಡುತ್ತಿದ್ದೇವೆ
ಕಚ್ಚಾ-ವಸ್ತು-ವಿ-ನಿಂದ
ಕಚ್ಚಿ
ಕಚ್ಚಿ-ಕೊಲ್ಲು-ವಂತೆ
ಕಚ್ಚಿದಾಗ
ಕಚ್ಚಿದ್ದಿ-ರ-ಬೇಕು
ಕಚ್ಚಿಸಿ
ಕಚ್ಚು-ವು-ದಿಲ್ಲ
ಕಛೇರಿ
ಕಛೇ-ರಿ-ಗ-ಳಲ್ಲಿ
ಕಛೇರಿಗೆ
ಕಛೇ-ರಿ-ಯನ್ನು
ಕಛೇ-ರಿ-ಯಲ್ಲಿ
ಕಟಿ-ಬಂಧ-ನ-ಗೈ-ನಿಲ್ಲು
ಕಟಿ-ಬದ್ಧ-ರಾ-ಗಿದ್ದರು
ಕಟು
ಕಟುಕನ
ಕಟುಟೀಕೆ
ಕಟು-ನು-ಡಿ-ಗ-ಳಿಂದ
ಕಟು-ನು-ಡಿ-ಯನ್ನಾ-ಡ-ಲಿಲ್ಲ
ಕಟು-ಮ-ನೋ-ಭಾವ
ಕಟು-ವಿ-ಮರ್ಶೆಗೆ
ಕಟ್ಟ
ಕಟ್ಟ-ಕ-ಡೆಗೆ
ಕಟ್ಟಡ
ಕಟ್ಟಡದ
ಕಟ್ಟ-ಡ-ವನ್ನು
ಕಟ್ಟಡವು
ಕಟ್ಟ-ಬ-ಹುದು
ಕಟ್ಟ-ಬೇ-ಕಾ-ದದ್ದು
ಕಟ್ಟ-ಲಾ-ದೀತೇ
ಕಟ್ಟಲು
ಕಟ್ಟಲ್ಪಟ್ಟ
ಕಟ್ಟಲ್ಪಟ್ಟಿಲ್ಲ
ಕಟ್ಟಾ-ಗಿ-ರು-ವು-ದೆಂದರೆ
ಕಟ್ಟಿ
ಕಟ್ಟಿಕೊಂಡು
ಕಟ್ಟಿ-ಕೊಳ್ಳ-ಬೇ-ಕಾ-ಗು-ವುದು
ಕಟ್ಟಿಟ್ಟದ್ದು
ಕಟ್ಟಿದ
ಕಟ್ಟಿ-ದಂತಾ-ಗ-ಬ-ಹು-ದು-ಎಂಬು-ದನ್ನು
ಕಟ್ಟಿದ್ದರು
ಕಟ್ಟಿ-ನಿಂದಾಗಿ
ಕಟ್ಟಿ-ಯಾ-ದರೂ
ಕಟ್ಟಿವೆ
ಕಟ್ಟಿಸಲು
ಕಟ್ಟಿ-ಸಿದ್ದೇನೆ
ಕಟ್ಟಿ-ಹಾ-ಕಿದ್ದರು
ಕಟ್ಟು
ಕಟ್ಟು-ಪಾ-ಡು-ಗ-ಳಿಂದ
ಕಟ್ಟು-ಕ-ತೆ-ಯಲ್ಲ
ಕಟ್ಟು-ಕ-ಥೆ-ಯಂತೆ
ಕಟ್ಟುತ್ತ
ಕಟ್ಟುತ್ತೇನೆ
ಕಟ್ಟು-ನಿಟ್ಟಾಗಿ
ಕಟ್ಟು-ನಿಟ್ಟಾದ
ಕಟ್ಟು-ನಿಟ್ಟಿಗೆ
ಕಟ್ಟು-ನಿಟ್ಟಿ-ನಂತೆ
ಕಟ್ಟು-ಪಾ-ಡು-ಗಳು
ಕಟ್ಟುಬಿದ್ದ
ಕಟ್ಟುಬಿದ್ದು
ಕಟ್ಟುವಂತೆ
ಕಟ್ಟುವುದು
ಕಟ್ಟು-ಹಾ-ಕ-ಬ-ಹುದು
ಕಟ್ಟೆ
ಕಟ್ಟೆಯಲ್ಲಿ
ಕಠಿಣ
ಕಠಿ-ಣ-ದೆ-ಡೆಗೆ
ಕಠಿ-ಣ-ರೋ-ಗ-ದಿಂದ
ಕಠಿ-ಣ-ವಾ-ಗಿದ್ದವು
ಕಠಿ-ಣ-ವಾದ
ಕಠಿ-ಣೋಕ್ತಿ-ಗ-ಳನ್ನೂ
ಕಠೋ-ರ-ವಾಗಿ
ಕಡ-ತ-ಗ-ಳನ್ನು
ಕಡಮೆ
ಕಡ-ಲ-ಕ-ರೆ-ಯನ್ನು
ಕಡ-ಲಾ-ಗಿತ್ತು
ಕಡ-ಲಿ-ನಲ್ಲಿ
ಕಡಲು
ಕಡಿದರೆ
ಕಡಿದು
ಕಡಿಮೆ
ಕಡಿ-ಮೆ-ಯಾ-ಗುತ್ತದೆ
ಕಡಿ-ಮೆ-ಯಾ-ದು-ದನ್ನು
ಕಡಿ-ಮೆ-ಮಾ-ಡಿ-ಕೊಳ್ಳ-ಬಲ್ಲ
ಕಡಿ-ಮೆ-ಮಾ-ಡಿ-ಕೊಳ್ಳ-ಬೇ-ಕಾ-ಗು-ವುದು
ಕಡಿ-ಮೆ-ಯಾಗಿ
ಕಡಿ-ಮೆ-ಯಾ-ಗಿ-ದೆ-ಯೆಂದು
ಕಡಿ-ಮೆ-ಯಾ-ಗುತ್ತದೆ
ಕಡಿ-ಮೆ-ಯಾ-ಗುತ್ತವೆ
ಕಡಿ-ಮೆ-ಯಾ-ಗುತ್ತಾ
ಕಡಿ-ಮೆ-ಯಾ-ಗುವ
ಕಡಿ-ಮೆ-ಯಾ-ಗು-ವಂತೆ
ಕಡಿ-ಮೆ-ಯಾ-ಗು-ವುದು
ಕಡಿ-ಮೆ-ಯಾದ
ಕಡಿ-ಮೆ-ಯಾ-ದರೂ
ಕಡಿ-ಮೆ-ಯಾ-ದಷ್ಟೂ
ಕಡಿ-ಮೆ-ಯಾ-ದಾಗ
ಕಡಿ-ಮೆ-ಯಾ-ದಾ-ಗಲೂ
ಕಡಿ-ಮೆ-ಯಾ-ಯಿತು
ಕಡಿ-ಯಿ-ತಂತೆ
ಕಡಿಯಿತು
ಕಡಿಯುವ
ಕಡಿ-ಯು-ವು-ದಿಲ್ಲ
ಕಡೆ
ಕಡೆ-ಗ-ಣಿಸಿ
ಕಡೆ-ಗ-ಣಿ-ಸಿ-ದಂತಾ-ಗ-ದೆ-ಕೇಸೀ
ಕಡೆ-ಗ-ಣಿ-ಸುವ
ಕಡೆ-ಗ-ಳಲ್ಲಿ
ಕಡೆ-ಗ-ಳಲ್ಲೂ
ಕಡೆಗೂ
ಕಡೆಗೆ
ಕಡೆಗೇ
ಕಡೆಗೇನೂ
ಕಡೆದಂಥ
ಕಡೆದು
ಕಡೆಯಲಿ
ಕಡೆಯಲ್ಲಿ
ಕಡೆ-ಯಲ್ಲಿದ್ದು
ಕಡೆಯಿಂದ
ಕಡ್ಡಿ
ಕಡ್ಡಿಯ
ಕಡ್ಡಿಯನ್ನು
ಕಡ್ಡಿಯೂ
ಕಣ-ಕ-ಣ-ದಲ್ಲಿಯೂ
ಕಣ-ಕ-ಣ-ದಲ್ಲೂ
ಕಣಗಳ
ಕಣ-ಗ-ಳಲ್ಲಿ-ರುವ
ಕಣ-ಗ-ಳಿವೆ
ಕಣಗಳು
ಕಣ-ಗ-ಳೆಂಬ
ಕಣಗಳೇ
ಕಣಯ್ಯ
ಕಣವೂ
ಕಣ್ಣನ್ನು
ಕಣ್ಣಾರೆ
ಕಣ್ಣಿಗೂ
ಕಣ್ಣಿಗೆ
ಕಣ್ಣಿಟ್ಟು
ಕಣ್ಣಿನ
ಕಣ್ಣಿನಲ್ಲಿ
ಕಣ್ಣಿನಿಂದ
ಕಣ್ಣೀರ
ಕಣ್ಣೀರಿಟ್ಟು
ಕಣ್ಣೀ-ರಿ-ಡುತ್ತಾನೆ
ಕಣ್ಣೀರಿನ
ಕಣ್ಣೀರು
ಕಣ್ಣು
ಕಣ್ಣು-ಕೋ-ರೈ-ಸು-ವಂಥದು
ಕಣ್ಣುಗಳ
ಕಣ್ಣು-ಗ-ಳನ್ನು
ಕಣ್ಣು-ಗ-ಳಿಂದ
ಕಣ್ಣುಗಳು
ಕಣ್ಣುಗಳೂ
ಕಣ್ಣು-ತಪ್ಪಿಸಿ
ಕಣ್ಣುಬಿಟ್ಟು
ಕಣ್ಣುಮುಚ್ಚಿ
ಕಣ್ಣೆದುರು
ಕಣ್ತುಂಬಿ
ಕಣ್ತೆ-ರೆ-ದಾಗ
ಕಣ್ತೆರೆದು
ಕಣ್ಮ-ಣಿ-ಯಾಗಿ
ಕಣ್ಮರೆ
ಕಣ್ಮ-ರೆ-ಯಾಗಿ
ಕಣ್ಮ-ರೆ-ಯಾ-ಗಿದೆ
ಕಣ್ಮ-ರೆ-ಯಾ-ಗುತ್ತದೆ
ಕಣ್ಮ-ರೆ-ಯಾ-ಗುತ್ತವೆ
ಕಣ್ಮ-ರೆ-ಯಾ-ಗುತ್ತಾರೆ
ಕಣ್ಮ-ರೆ-ಯಾ-ಗು-ವ-ವರು
ಕಣ್ಮ-ರೆ-ಯಾದ
ಕಣ್ಮ-ರೆ-ಯಾ-ದಳು
ಕಣ್ಮ-ರೆ-ಯಾ-ಯಿತು
ಕಣ್ಮುಂದೆ
ಕಣ್ಮುಚ್ಚಿ
ಕತೆ
ಕತೆ-ಕಾ-ದಂಬ-ರಿ-ಗ-ಳಲ್ಲಿ
ಕತೆ-ಕಾ-ದಂಬ-ರಿ-ಗಳು
ಕತೆ-ಗ-ಳಲ್ಲಿ
ಕತೆಗಳು
ಕತೆಗೆ
ಕತೆಯ
ಕತೆಯಂತೆ
ಕತೆಯನ್ನು
ಕತೆಯಲ್ಲಿ
ಕತೆ-ಯಿ-ದು-ಒಮ್ಮೆ
ಕತೆಯೂ
ಕತೆಯೇ
ಕತೆ-ಯೊಂದಿದೆ
ಕತ್ತರಿ
ಕತ್ತರಿಯ
ಕತ್ತ-ರಿ-ಯನ್ನೂ
ಕತ್ತ-ರಿ-ಸಲು
ಕತ್ತ-ರಿ-ಸಲೂ
ಕತ್ತರಿಸಿ
ಕತ್ತ-ರಿ-ಸುವ
ಕತ್ತ-ರಿ-ಸು-ವಂತೆ
ಕತ್ತಲನ್ನು
ಕತ್ತಲನ್ನೆ
ಕತ್ತಲಲ್ಲಿ
ಕತ್ತ-ಲಲ್ಲಿದ್ದಾಗ
ಕತ್ತ-ಲಾ-ಗದೆ
ಕತ್ತ-ಲಾ-ಗಿದೆ
ಕತ್ತ-ಲಾ-ದಾಗ
ಕತ್ತಲಿನ
ಕತ್ತಲು
ಕತ್ತಲೆ
ಕತ್ತಲೆಗೆ
ಕತ್ತ-ಲೆ-ಡೆಗೇ
ಕತ್ತ-ಲೆ-ಯನ್ನು
ಕತ್ತ-ಲೆ-ಯಲ್ಲಿ
ಕತ್ತ-ಲೆ-ಯಲ್ಲಿದ್ದು
ಕತ್ತ-ಲೆ-ಯಿಂದ
ಕತ್ತ-ಲೆ-ಯೆ-ಡೆಗೆ
ಕತ್ತಿ
ಕತ್ತಿನ
ಕತ್ತಿನಿಂದ
ಕತ್ತೆ
ಕತ್ತೆ-ಗ-ಳನ್ನು
ಕಥನ
ಕಥ-ನ-ಕರ್ತೃ-ಗಳು
ಕಥೆ
ಕಥೆ-ಗ-ಳಲ್ಲಿ
ಕಥೆಗಳು
ಕಥೆಯ
ಕಥೆಯಲ್ಲ
ಕದಲದೇ
ಕದಲಿಲ್ಲ
ಕದ್ದಿದ್ದ
ಕದ್ದು
ಕದ್ದು-ಮುಚ್ಚಿ-ಯಾ-ದರೂ
ಕನ-ಕ-ದಾಸ
ಕನಸನ್ನು
ಕನಸಲ್ಲ
ಕನಸಿನ
ಕನ-ಸಿ-ನಲ್ಲಿ
ಕನ-ಸಿ-ನಲ್ಲೂ
ಕನ-ಸಿ-ನಲ್ಲೇ
ಕನಸು
ಕನ-ಸು-ಗ-ಳನ್ನು
ಕನಿಕರ
ಕನಿ-ಕ-ರಕ್ಕೆ
ಕನಿ-ಕ-ರ-ಪ-ಡುತ್ತೇನೆ
ಕನಿಷ್ಠ
ಕನಿಷ್ಠ-ಪಕ್ಷ
ಕನಿಷ್ಠ-ಮಟ್ಟದ
ಕನಿಷ್ಠ-ವಾ-ದು-ದಲ್ಲ
ಕನ್ನಡ
ಕನ್ನ-ಡ-ಕದ
ಕನ್ನಡದ
ಕನ್ನ-ಡ-ದಲ್ಲಿ
ಕನ್ನ-ಡಿ-ಗರೇ
ಕನ್ನ-ಡಿ-ಯಲ್ಲಿ
ಕನ್ನಡಿಯೂ
ಕನ್ನ-ಡಿ-ಯೊ-ಳ-ಗಿನ
ಕನ್ಫ್ಯೂ-ಶಿ-ಯಸ್
ಕಪ-ಟ-ಗ-ಳಿಂದ
ಕಪಾಲಕ್ಕೆ
ಕಪಿ
ಕಪಿಗಳು
ಕಪಿಗೆ
ಕಪಿ-ಮುಷ್ಠಿ-ಯಿಂದ
ಕಪಿಯು
ಕಪೋ-ಲ-ಗಳು
ಕಪ್ಪಾ-ಗಿದ್ದವು
ಕಪ್ಪಾದ
ಕಪ್ಪು
ಕಪ್ಪು-ಕನ್ನ-ಡಕ
ಕಬ-ಳಿ-ಸುತ್ತಿದ್ದರು
ಕಬ-ಳಿ-ಸುವ
ಕಬ-ಳಿ-ಸು-ವುದು
ಕಬ್ಬಿಣ
ಕಬ್ಬಿಣದ
ಕಬ್ಬಿ-ಣ-ದಿಂದ
ಕಬ್ಬಿ-ಣ-ದಿಂದಲ್ಲ
ಕಮಲ
ಕಮಲಾ
ಕಮೋಡಿಗೆ
ಕಮ್ಯುನಿಸ್ಟ್
ಕರಗತ
ಕರ-ಗ-ತ-ಮಾ-ಡಿ-ಕೊಂಡ
ಕರ-ಗ-ತ-ವಾ-ಗುತ್ತ-ದೆಂಬು-ದನ್ನು
ಕರಗಿ
ಕರಗಿದ
ಕರಗಿಯೇ
ಕರಗುತ್ತ
ಕರಗುವ
ಕರ-ಗು-ವುದು
ಕರಟಿ
ಕರಡಿ
ಕರಡಿಯ
ಕರ-ಡಿ-ಯನ್ನು
ಕರಡು
ಕರಣಿಕ
ಕರವಸ್ತ್ರ
ಕರಾರು
ಕರಾ-ರು-ವಾಕ್ಕಾಗಿ
ಕರಾಳ
ಕರಿ
ಕರಿಯನೂ
ಕರಿಯನೇ
ಕರು
ಕರು-ಣಾ-ಕರ
ಕರು-ಣಾ-ಮಯಿ
ಕರು-ಣಾ-ಳು-ವಿ-ನಲ್ಲಿ-ರು-ವಂತೆಯೇ
ಕರು-ಣಾ-ಸಾ-ಗರ
ಕರುಣಿಸಿ
ಕರು-ಣಿ-ಸಿ-ದ-ನೆಂದು
ಕರು-ಣಿ-ಸಿದ್ದಾನೆ
ಕರುಣೆ
ಕರು-ಣೆ-ದೋ-ರ-ದಿದ್ದರೆ
ಕರು-ಣೆ-ದೋ-ರ-ಬೇಕು
ಕರು-ಣೆ-ಗ-ಳನ್ನು
ಕರು-ಣೆ-ಗ-ಳನ್ನೂ
ಕರು-ಣೆ-ಗ-ಳಿಂದ
ಕರು-ಣೆ-ಗ-ಳಿಂದಲೇ
ಕರು-ಣೆ-ಗಳು
ಕರು-ಣೆ-ದೋರು
ಕರುಣೆಯ
ಕರು-ಣೆ-ಯಲ್ಲಿ
ಕರು-ಣೆ-ಯಿಂದ
ಕರು-ಬ-ಬೇ-ಕಿಲ್ಲ
ಕರು-ಬುತ್ತಾರೆ
ಕರುಮ
ಕರುಳನ್ನು
ಕರುಳಲ್ಲಿ
ಕರು-ಳಿ-ನಲ್ಲಿ
ಕರುಳು
ಕರುವನ್ನು
ಕರೆ
ಕರೆ-ತ-ರುತ್ತಿದ್ದರು
ಕರೆ-ದೊಯ್ದರು
ಕರೆಂಟ್ನಂಥ
ಕರೆ-ಕೊ-ಡುತ್ತಾನೆ
ಕರೆಗಂಟೆ
ಕರೆಗೆ
ಕರೆ-ತಂದರು
ಕರೆ-ತಂದಾಗ
ಕರೆತಂದು
ಕರೆ-ತಂದು-ದಲ್ಲ
ಕರೆ-ತ-ರಲು
ಕರೆ-ತ-ರು-ವಂತೆ
ಕರೆದ
ಕರೆದರು
ಕರೆದರೂ
ಕರೆದರೆ
ಕರೆದಿದೆ
ಕರೆ-ದಿಲ್ಲ-ವೆಂದಾ-ಗಲಿ
ಕರೆದು
ಕರೆ-ದು-ಕೊಂಡ
ಕರೆ-ದು-ಕೊಂಡು
ಕರೆ-ದೊಯ್ದರು
ಕರೆ-ದೊಯ್ದಾಗ
ಕರೆದೊಯ್ದು
ಕರೆ-ದೊಯ್ಯ-ಬಲ್ಲೆ
ಕರೆ-ದೊಯ್ಯ-ಲಾ-ಗಿತ್ತು
ಕರೆ-ದೊಯ್ಯುತ್ತಿದ್ದಳು
ಕರೆ-ದೊಯ್ಯುತ್ತಿದ್ದಳು
ಕರೆ-ದೊಯ್ಯುತ್ತೇನೆ
ಕರೆ-ದೊಯ್ಯು-ವಳು
ಕರೆ-ನೀ-ಡಿ-ದರು
ಕರೆ-ಯ-ತೊ-ಡ-ಗ-ಬ-ಹುದು
ಕರೆಯನ್ನು
ಕರೆ-ಯ-ಬ-ಹು-ದಾ-ದಂತೆ
ಕರೆ-ಯ-ಬ-ಹುದು
ಕರೆ-ಯ-ಬ-ಹುದೋ
ಕರೆ-ಯ-ಲಾ-ಗಿದೆ
ಕರೆ-ಯ-ಲಾ-ಗುವ
ಕರೆ-ಯ-ಲಾ-ಗು-ವು-ದಿಲ್ಲ
ಕರೆ-ಯ-ಲಾದ
ಕರೆ-ಯಲ್ಪ-ಡುವ
ಕರೆ-ಯಿತ್ತರು
ಕರೆ-ಯುತ್ತ-ಲಿದ್ದ
ಕರೆ-ಯುತ್ತಾನೆ
ಕರೆ-ಯುತ್ತಾರೆ
ಕರೆ-ಯುತ್ತಿದ್ದರು
ಕರೆ-ಯುತ್ತಿದ್ದಾನೆ
ಕರೆ-ಯುತ್ತಿ-ರು-ವಾಗ
ಕರೆ-ಯುತ್ತೇವೋ
ಕರೆಯುವ
ಕರೆ-ಯು-ವಂತೆ
ಕರೆ-ಯು-ವಿಯೋ
ಕರೆ-ಯು-ವಿರಿ
ಕರೆ-ಯು-ವು-ದ-ರಲ್ಲಿ
ಕರೆ-ಯು-ವುದು
ಕರೆಯೋಣ
ಕರೆಸಿ
ಕರೆ-ಸಿ-ಕೊಳ್ಳ-ಬೇಕು
ಕರೆಸಿದ್ದ
ಕರ್ಕಶ
ಕರ್ಕ-ಶ-ವಾಗಿ
ಕರ್ಣನ
ಕರ್ತವ್ಯ
ಕರ್ತವ್ಯ-ಗ-ಳನ್ನು
ಕರ್ತವ್ಯ-ಗ-ಳೊಂದಿಗೆ
ಕರ್ತವ್ಯದ
ಕರ್ತವ್ಯ-ದಲ್ಲಿ
ಕರ್ತವ್ಯ-ನಿಷ್ಠರೂ
ಕರ್ತವ್ಯ-ನಿಷ್ಠ-ರೆಂದು
ಕರ್ತವ್ಯ-ನಿಷ್ಠೆ
ಕರ್ತವ್ಯ-ನಿಷ್ಠೆಗೆ
ಕರ್ತವ್ಯ-ನಿಷ್ಠೆಯ
ಕರ್ತವ್ಯ-ನಿಷ್ಠೆಯೇ
ಕರ್ತವ್ಯ-ಪಾ-ಲ-ನೆಯ
ಕರ್ತವ್ಯಭ್ರಷ್ಟ-ನಾ-ಗು-ವೆನೊ
ಕರ್ತವ್ಯ-ವನ್ನು
ಕರ್ತವ್ಯ-ವಾ-ಗ-ಬೇಕು
ಕರ್ತವ್ಯ-ವಾ-ಗಿ-ರುತ್ತದೆ
ಕರ್ತವ್ಯ-ವಿ-ದೆಯೋ
ಕರ್ತವ್ಯವೂ
ಕರ್ತೃ
ಕರ್ತೃತ್ವ
ಕರ್ತೃತ್ವ-ಶಾಲಿ
ಕರ್ನಲ್
ಕರ್ನಾಟಕ
ಕರ್ನಾ-ಟ-ಕದ
ಕರ್ನಾ-ಟ-ಕ-ದಲ್ಲಿ
ಕರ್ಮ
ಕರ್ಮಕ್ಕ-ನು-ಗು-ಣ-ವಾಗಿ
ಕರ್ಮಕ್ಕೆ
ಕರ್ಮಗಳ
ಕರ್ಮ-ಗ-ಳನ್ನು
ಕರ್ಮ-ಗ-ಳಲ್ಲಿ
ಕರ್ಮ-ಗ-ಳಿಂದಾ-ಗಿಯೆ
ಕರ್ಮ-ಗ-ಳಿಗೆ
ಕರ್ಮಗಳೇ
ಕರ್ಮ-ಜನ್ಮಾಂತರ
ಕರ್ಮ-ಜನ್ಮಾಂತ-ರ-ವಾದ
ಕರ್ಮ-ಜನ್ಮಾಂತ-ರ-ವಾ-ದ-ವನ್ನು
ಕರ್ಮದ
ಕರ್ಮ-ದಿಂದಲೇ
ಕರ್ಮ-ದಿಂದಾಗಿ
ಕರ್ಮ-ದಿಂದಾದ
ಕರ್ಮ-ದೋ-ಷ-ದಿಂದ
ಕರ್ಮ-ದೋ-ಷ-ವಾ-ಗು-ವುದು
ಕರ್ಮ-ನಿ-ಯಂತ್ರಿ-ತ-ವಲ್ಲ
ಕರ್ಮ-ನಿ-ಯಮ
ಕರ್ಮ-ನಿ-ಯ-ಮ-ವನ್ನು
ಕರ್ಮ-ನಿ-ಯ-ಮವೂ
ಕರ್ಮ-ಪು-ಣ-ಶೇ-ಷಂಗ-ಳು-ಳಿ-ಯ-ದಿರೆ
ಕರ್ಮಫಲ
ಕರ್ಮ-ಫ-ಲ-ಗ-ಳನ್ನು
ಕರ್ಮ-ಫ-ಲ-ವನ್ನು
ಕರ್ಮ-ಬಂಧ-ನ-ದಿಂದ
ಕರ್ಮ-ಯೋ-ಗ-ವನ್ನು
ಕರ್ಮ-ಯೋ-ಗಿಯ
ಕರ್ಮ-ಯೋ-ಗಿಯೂ
ಕರ್ಮವನ್ನು
ಕರ್ಮವಾದ
ಕರ್ಮ-ವಾ-ದ-ಇ-ವು-ಗಳ
ಕರ್ಮ-ವಾ-ದ-ವನ್ನು
ಕರ್ಮವಿದೆ
ಕರ್ಮ-ವೀ-ರರ
ಕರ್ಮ-ವೀ-ರ-ರನ್ನಾ-ಗಿ-ಸಿತು
ಕರ್ಮ-ವೀ-ರರು
ಕರ್ಮವು
ಕರ್ಮವೂ
ಕರ್ಮ-ವೆಂದರೆ
ಕರ್ಮ-ವೆನ್ನುವ
ಕರ್ಮ-ಶೀ-ಲತೆ
ಕರ್ಮ-ಸಂಬಂಧ-ವಾದ
ಕರ್ಮ-ಸಿದ್ಧಾಂತ
ಕರ್ಮ-ಸಿದ್ಧಾಂತಕ್ಕೆ
ಕರ್ಮ-ಸಿದ್ಧಾಂತದ
ಕರ್ಮ-ಸಿದ್ಧಾಂತ-ವನ್ನು
ಕರ್ಮ-ಸಿದ್ಧಾಂತ-ವೆಂದೊ-ಡನೆ
ಕರ್ಮ-ಹೀ-ನ-ತೆ-ಯಲ್ಲ
ಕರ್ಮಾಧ್ಯಕ್ಷಃ
ಕರ್ಮೇಂದ್ರಿ-ಯ-ಗ-ಳಿಂದ
ಕರ್ರ-ಗಿದ್ದರೂ
ಕಲಕಿತು
ಕಲಹ
ಕಲಾಕೃತಿ
ಕಲಾ-ಕೃ-ತಿಯ
ಕಲಾ-ಕೇಂದ್ರಕ್ಕೆ
ಕಲಾ-ನೈ-ಪುಣ್ಯ-ವನ್ನು
ಕಲಾ-ಪ-ಗ-ಳನ್ನಾ-ಗಲೀ
ಕಲಾಪ್ರೇಮಿ
ಕಲಾವಿದ
ಕಲಾ-ವಿ-ದ-ನಾ-ಗಲು
ಕಲಾ-ವಿ-ದ-ನೊಬ್ಬ
ಕಲಿ
ಕಲಿಕೆ
ಕಲಿ-ಕೆ-ಗಳು
ಕಲಿಕೆಗೆ
ಕಲಿಕೆಯ
ಕಲಿ-ಕೆ-ಯಲ್ಲಿ
ಕಲಿತ
ಕಲಿ-ತದ್ದನ್ನು
ಕಲಿ-ತ-ರಾ-ಯಿತು
ಕಲಿತರೆ
ಕಲಿತಲ್ಲಿ
ಕಲಿತಳು
ಕಲಿ-ತ-ವನ
ಕಲಿ-ತ-ವರು
ಕಲಿ-ತಾ-ಗಲೇ
ಕಲಿ-ತಿದ್ದಾನೆ
ಕಲಿ-ತಿ-ರ-ಬೇ-ಕಲ್ಲವೆ
ಕಲಿತಿರಿ
ಕಲಿತಿಲ್ಲ
ಕಲಿತು
ಕಲಿ-ತು-ಕೊಂಡ
ಕಲಿ-ತು-ಬಿ-ಡುತ್ತದೆ
ಕಲಿತೆ
ಕಲಿತೆನೋ
ಕಲಿಯದ
ಕಲಿ-ಯ-ದ-ವರು
ಕಲಿ-ಯ-ದಿದ್ದರೆ
ಕಲಿಯದೆ
ಕಲಿ-ಯ-ಬಲ್ಲರು
ಕಲಿ-ಯ-ಬಲ್ಲೆವು
ಕಲಿ-ಯ-ಬ-ಹುದು
ಕಲಿ-ಯ-ಬಾ-ರ-ದೆಂಬ
ಕಲಿ-ಯ-ಬೇ-ಕಷ್ಟೆ
ಕಲಿ-ಯ-ಬೇ-ಕಾ-ಗ-ಬ-ಹುದು
ಕಲಿ-ಯ-ಬೇ-ಕಾ-ದುದು
ಕಲಿ-ಯ-ಬೇಕು
ಕಲಿ-ಯ-ಬೇ-ಕೆಂಬ
ಕಲಿ-ಯ-ಬೇ-ಕೆಂಬುದೇ
ಕಲಿ-ಯ-ಲಾ-ರದು
ಕಲಿ-ಯ-ಲಾ-ರೆವು
ಕಲಿಯಲಿ
ಕಲಿಯಲು
ಕಲಿ-ಯ-ಲೇ-ಬೇಕು
ಕಲಿಯಿರಿ
ಕಲಿ-ಯಿ-ರಿಯ
ಕಲಿ-ಯಿ-ರಿ-ಯಲ್ಲಿ
ಕಲಿಯು
ಕಲಿಯುತ್ತ
ಕಲಿ-ಯುತ್ತದೆ
ಕಲಿ-ಯುತ್ತಲೇ
ಕಲಿ-ಯುತ್ತಿದ್ದ
ಕಲಿ-ಯುತ್ತಿದ್ದಾಗ
ಕಲಿಯುವ
ಕಲಿ-ಯು-ವಂತಿ-ರ-ಲಿಲ್ಲ
ಕಲಿ-ಯು-ವಂತೆ
ಕಲಿ-ಯು-ವನು
ಕಲಿ-ಯು-ವಲ್ಲೂ
ಕಲಿ-ಯು-ವ-ವರು
ಕಲಿ-ಯು-ವಾತ
ಕಲಿ-ಯು-ವು-ದಕ್ಕೆ
ಕಲಿ-ಯು-ವು-ದ-ರಲ್ಲಿ
ಕಲಿ-ಯು-ವು-ದಿಲ್ಲ
ಕಲಿ-ಯು-ವುದು
ಕಲಿ-ಯು-ವು-ದೆಂತು
ಕಲಿ-ಯು-ವುದೇ
ಕಲಿಸಲು
ಕಲಿಸಿ
ಕಲಿ-ಸಿ-ಕೊ-ಡುತ್ತದೆ
ಕಲಿಸಿತು
ಕಲಿಸಿದ
ಕಲಿಸು
ಕಲಿ-ಸುತ್ತಿದ್ದರು
ಕಲಿ-ಸುತ್ತಿದ್ದಾನೆ
ಕಲಿ-ಸುತ್ತಿ-ರು-ವಂತೆ
ಕಲಿ-ಸುತ್ತೇನೆ
ಕಲಿ-ಸು-ವುದು
ಕಲು-ಷಿ-ತ-ಗೊ-ಳಿ-ಸ-ಲಿಕ್ಕಲ್ಲ
ಕಲೆ
ಕಲೆಇವು
ಕಲೆ-ಇ-ವು-ಗ-ಳನ್ನು
ಕಲೆಕ್ಟರ್ನೊ-ಡನೆ
ಕಲೆ-ಗ-ಳಲ್ಲಿ
ಕಲೆತು
ಕಲೆಯ
ಕಲೆಯನ್ನು
ಕಲೆಯನ್ನೂ
ಕಲೆಯನ್ನೋ
ಕಲೆಯಲ್ಲಿ
ಕಲೆಯೋ
ಕಲೆ-ಹಾ-ಕಿದ್ದಾರೆ
ಕಲ್ಕತ್ತದ
ಕಲ್ಕತ್ತ-ದಲ್ಲಿ
ಕಲ್ಕತ್ತೆಯ
ಕಲ್ಕತ್ತೆ-ಯಿಂದ
ಕಲ್ಚರ್
ಕಲ್ಪ-ನಾತ್ಮ-ಕ-ವಾದ
ಕಲ್ಪ-ನಾ-ವಿ-ಲಾಸ
ಕಲ್ಪನೆ
ಕಲ್ಪ-ನೆ-ಗಳ
ಕಲ್ಪ-ನೆ-ಗ-ಳನ್ನು
ಕಲ್ಪ-ನೆ-ಗ-ಳಿಂದ
ಕಲ್ಪ-ನೆ-ಗ-ಳಿಂದಲೇ
ಕಲ್ಪ-ನೆ-ಗ-ಳಿಗೆ
ಕಲ್ಪ-ನೆ-ಗ-ಳಿಗೇ
ಕಲ್ಪ-ನೆ-ಗ-ಳಿವೆ
ಕಲ್ಪ-ನೆ-ಗಳು
ಕಲ್ಪ-ನೆ-ಗಳೂ
ಕಲ್ಪ-ನೆ-ಗಿಂತ
ಕಲ್ಪನೆಗೂ
ಕಲ್ಪನೆಗೆ
ಕಲ್ಪನೆಗೇ
ಕಲ್ಪ-ನೆ-ನೀನು
ಕಲ್ಪನೆಯ
ಕಲ್ಪ-ನೆ-ಯನ್ನು
ಕಲ್ಪ-ನೆ-ಯಲ್ಲ
ಕಲ್ಪ-ನೆ-ಯಲ್ಲ-ವೆಂದೂ
ಕಲ್ಪ-ನೆ-ಯಲ್ಲಿ
ಕಲ್ಪ-ನೆ-ಯಾ-ಗು-ವು-ದಲ್ಲವೆ
ಕಲ್ಪ-ನೆ-ಯಿಂದ
ಕಲ್ಪನೆಯು
ಕಲ್ಪನೆಯೂ
ಕಲ್ಪನೆಯೇ
ಕಲ್ಪ-ನೆ-ಯೊಂದಿಗೆ
ಕಲ್ಪಿತ
ಕಲ್ಪಿ-ತ-ವಾ-ಗಿದ್ದರೂ
ಕಲ್ಪಿಸಲು
ಕಲ್ಪಿ-ಸಿ-ಕೊಂಡ
ಕಲ್ಪಿ-ಸಿ-ಕೊಂಡಿ
ಕಲ್ಪಿ-ಸಿ-ಕೊಂಡಿದ್ದ
ಕಲ್ಪಿ-ಸಿ-ಕೊಂಡಿ-ರುತ್ತಾನೆ
ಕಲ್ಪಿ-ಸಿ-ಕೊಂಡಿ-ರುತ್ತೇ-ವೆಯೋ
ಕಲ್ಪಿ-ಸಿ-ಕೊಂಡು
ಕಲ್ಪಿ-ಸಿ-ಕೊಳ್ಳ
ಕಲ್ಪಿ-ಸಿ-ಕೊಳ್ಳ-ಬ-ಹುದು
ಕಲ್ಪಿ-ಸಿ-ಕೊಳ್ಳಲು
ಕಲ್ಪಿ-ಸಿ-ಕೊಳ್ಳಿ
ಕಲ್ಪಿ-ಸಿ-ಕೊಳ್ಳುತ್ತ
ಕಲ್ಪಿ-ಸಿ-ಕೊಳ್ಳುವ
ಕಲ್ಪಿ-ಸಿ-ಕೊಳ್ಳು-ವಂಥ
ಕಲ್ಪಿ-ಸಿ-ಕೊಳ್ಳು-ವುದೇ
ಕಲ್ಪಿಸಿದ್ದು
ಕಲ್ಪಿಸುತ್ತ
ಕಲ್ಪಿಸುವ
ಕಲ್ಯಾಣ
ಕಲ್ಯಾ-ಣಕ್ಕಾಗಿ
ಕಲ್ಯಾಣಕ್ಕೂ
ಕಲ್ಯಾಣಕ್ಕೆ
ಕಲ್ಯಾ-ಣ-ಚಿಂತ-ನೆಗೆ
ಕಲ್ಯಾಣದ
ಕಲ್ಯಾ-ಣ-ವನ್ನುಂಟು-ಮಾ-ಡುವ
ಕಲ್ಯಾ-ಣಾ-ಗ-ಮ-ಗ-ಳನ್ನು
ಕಲ್ಲನ್ನು
ಕಲ್ಲಲ್ಲ
ಕಲ್ಲಾ-ಗಿ-ರು-ವು-ದ-ರಿಂದ
ಕಲ್ಲಿಗೆ
ಕಲ್ಲಿನ
ಕಲ್ಲು
ಕಲ್ಲು-ಗ-ಳನ್ನಿಟ್ಟು
ಕಲ್ಲು-ಗ-ಳಿಂದ
ಕಲ್ಲು-ಗುಂಡನ್ನು
ಕಲ್ಲು-ಮುಳ್ಳು-ಗ-ಳನ್ನು
ಕಲ್ಲು-ಮುಳ್ಳು-ಗಳು
ಕಲ್ಲು-ಹಾ-ಕಿ-ಕೊಳ್ಳುತ್ತೇವೆ
ಕಲ್ಲೆಣ್ಣೆಯ
ಕಲ್ಲೊಂದನ್ನು
ಕಳ
ಕಳಂಕ
ಕಳಂಕ-ಮಯ
ಕಳಕಳಿ
ಕಳ-ಕ-ಳಿಯ
ಕಳ-ಕ-ಳಿ-ಯಿಂದ
ಕಳಚಿದ್ದು
ಕಳ-ಚಿ-ಹೋ-ಗಿದೆ
ಕಳಪೆ
ಕಳ-ಪೆ-ಯಲ್ಲ
ಕಳವಳ
ಕಳ-ವ-ಳದ
ಕಳವು
ಕಳ-ವು-ಕಳ್ಳ-ಸಾ-ಗಣೆ
ಕಳಿಂಗ
ಕಳಿಸಿ
ಕಳಿ-ಸಿ-ದರು
ಕಳು-ವಾ-ಗಿಲ್ಲ
ಕಳುಹಿಸಿ
ಕಳು-ಹಿ-ಸಿದ್ದಾ-ರೆಂದೂ
ಕಳು-ಹಿ-ಸಿದ
ಕಳು-ಹಿ-ಸಿ-ದರು
ಕಳು-ಹಿ-ಸುತ್ತಾರೆ
ಕಳು-ಹಿ-ಸುತ್ತಿದ್ದ
ಕಳು-ಹಿ-ಸುತ್ತಿದ್ದರು
ಕಳು-ಹಿ-ಸುತ್ತಿದ್ದೀರಿ
ಕಳೆ-ಗುಂದಿದ
ಕಳೆ-ಗುಂದಿದ್ದವು
ಕಳೆದ
ಕಳೆ-ದಂತಾ-ಯಿ-ತಲ್ಲ
ಕಳೆದಂತೆ
ಕಳೆದರು
ಕಳೆದರೆ
ಕಳೆ-ದ-ವರು
ಕಳೆದವು
ಕಳೆದಿತ್ತು
ಕಳೆ-ದಿದ್ದಳು
ಕಳೆ-ದಿದ್ದೇನೆ
ಕಳೆ-ದಿ-ರುವಿ
ಕಳೆ-ದಿ-ರೆಂದರೆ
ಕಳೆ-ದಿ-ರೆಂಬುದು
ಕಳೆದು
ಕಳೆ-ದು-ಕೊಂಡ
ಕಳೆ-ದು-ಕೊಂಡಂತೆ
ಕಳೆ-ದು-ಕೊಂಡರು
ಕಳೆ-ದು-ಕೊಂಡರೂ
ಕಳೆ-ದು-ಕೊಂಡರೆ
ಕಳೆ-ದು-ಕೊಂಡ-ವ-ನಲ್ಲ
ಕಳೆ-ದು-ಕೊಂಡ-ವ-ರಾ-ಗಿದ್ದರು
ಕಳೆ-ದು-ಕೊಂಡಾಗ
ಕಳೆ-ದು-ಕೊಂಡಿದ್ದಾರೆ
ಕಳೆ-ದು-ಕೊಂಡಿದ್ದೇನೆ
ಕಳೆ-ದು-ಕೊಂಡಿ-ರುತ್ತದೆ
ಕಳೆ-ದು-ಕೊಂಡಿ-ರುವ
ಕಳೆ-ದು-ಕೊಂಡಿಲ್ಲ
ಕಳೆ-ದು-ಕೊಂಡು
ಕಳೆ-ದು-ಕೊಂಡೆ-ವಲ್ಲ
ಕಳೆ-ದು-ಕೊಳ್ಳ
ಕಳೆ-ದು-ಕೊಳ್ಳ-ಬ-ಹುದು
ಕಳೆ-ದು-ಕೊಳ್ಳ-ಬೇ-ಕಾ-ಗು-ವುದು
ಕಳೆ-ದು-ಕೊಳ್ಳ-ಬೇಡ
ಕಳೆ-ದು-ಕೊಳ್ಳ-ಲಾ-ರರು
ಕಳೆ-ದು-ಕೊಳ್ಳುತ್ತಾರೆ
ಕಳೆ-ದು-ಕೊಳ್ಳುವ
ಕಳೆ-ದು-ಕೊಳ್ಳು-ವರು
ಕಳೆ-ದು-ಹೋ-ಗಿದೆ
ಕಳೆ-ದು-ಹೋದ
ಕಳೆ-ದು-ಹೋ-ದವು
ಕಳೆದೆ
ಕಳೆಯದೆ
ಕಳೆ-ಯ-ಲಿಲ್ಲ
ಕಳೆಯಲು
ಕಳೆಯಿತು
ಕಳೆ-ಯುತ್ತದೆ
ಕಳೆ-ಯುತ್ತಿದ್ದ
ಕಳೆ-ಯುತ್ತಿದ್ದರು
ಕಳೆ-ಯುತ್ತಿದ್ದೀಯೆ
ಕಳೆ-ಯುತ್ತಿದ್ದೆ
ಕಳೆ-ಯುತ್ತೀ-ರೆಂಬು-ದನ್ನು
ಕಳೆ-ಯುತ್ತೇನೆ
ಕಳೆ-ಯು-ವರು
ಕಳ್ಳ
ಕಳ್ಳ-ಕಾ-ಕರ
ಕಳ್ಳತನ
ಕಳ್ಳ-ತ-ನ-ಗಳು
ಕಳ್ಳ-ತ-ನದ
ಕಳ್ಳನೂ
ಕಳ್ಳರ
ಕಳ್ಳರು
ಕಳ್ಳವ್ಯಾ-ಪಾರಿ
ಕವ-ನ-ದಲ್ಲಿ
ಕವಲು
ಕವ-ಲು-ಗ-ಳಷ್ಟೆ
ಕವಾ-ಯ-ತಿನ
ಕವಿ
ಕವಿ-ಕಲ್ಪನೆ
ಕವಿ-ಕಲ್ಪ-ನೆಯೂ
ಕವಿ-ಗ-ಳಾದ
ಕವಿಗಳೂ
ಕವಿಗೆ
ಕವಿತೆಯ
ಕವಿ-ತೆ-ಯಲ್ಲಿ
ಕವಿದ
ಕವಿ-ದು-ಕೊಂಡಿದೆ
ಕವಿಯ
ಕವಿಯಾಗಿ
ಕವಿಯಿತು
ಕವಿ-ಯುತ್ತಿದ್ದರೂ
ಕವಿ-ವಾಕ್ಯ-ಗ-ಳಲ್ಲಿ
ಕವಿ-ವಾಕ್ಯದ
ಕವಿವಾಣಿ
ಕಶ್ಮ-ಲ-ಗಳು
ಕಷ್ಟ
ಕಷ್ಟ-ಕಂಟ-ಕ-ಗ-ಳನ್ನು
ಕಷ್ಟ-ಕಂಟ-ಕ-ಗ-ಳನ್ನೂ
ಕಷ್ಟ-ಕಂಟ-ಕ-ಗ-ಳಲ್ಲಿ
ಕಷ್ಟ-ಕಂಟ-ಕ-ಗಳು
ಕಷ್ಟ-ಕಂಟ-ಕ-ಗಳೇ
ಕಷ್ಟ-ಕ-ರ-ವಾ-ದದ್ದು
ಕಷ್ಟ-ಕಾ-ಲ-ದಲ್ಲಿ
ಕಷ್ಟ-ಕಾ-ಲ-ದಲ್ಲೂ
ಕಷ್ಟ-ಕೊ-ಡು-ವು-ದಿಲ್ಲ
ಕಷ್ಟ-ಕೋ-ಟ-ಲೆ-ಗಳ
ಕಷ್ಟಕ್ಕೆ
ಕಷ್ಟಗಳ
ಕಷ್ಟ-ಗ-ಳನ್ನು
ಕಷ್ಟ-ಗ-ಳನ್ನೂ
ಕಷ್ಟ-ಗ-ಳನ್ನೆಲ್ಲ
ಕಷ್ಟ-ಗ-ಳನ್ನೇ
ಕಷ್ಟ-ಗ-ಳಲ್ಲಿ
ಕಷ್ಟಗಳು
ಕಷ್ಟಗಳೂ
ಕಷ್ಟ-ಗ-ಳೆಂಥವು
ಕಷ್ಟದ
ಕಷ್ಟದಲ್ಲಿ
ಕಷ್ಟ-ದಲ್ಲಿಯೂ
ಕಷ್ಟದಿಂದ
ಕಷ್ಟನಷ್ಟ
ಕಷ್ಟ-ನಷ್ಟ-ಗಳ
ಕಷ್ಟ-ನಷ್ಟ-ಗ-ಳನ್ನು
ಕಷ್ಟ-ನಷ್ಟ-ಗ-ಳಿಗೂ
ಕಷ್ಟ-ಪಟ್ಟಾ-ದರೂ
ಕಷ್ಟಪಟ್ಟು
ಕಷ್ಟ-ಪ-ಡ-ಬೇ-ಕಾ-ಗುತ್ತದೆ
ಕಷ್ಟ-ಪ-ಡ-ಬೇ-ಕಾ-ಯಿತು
ಕಷ್ಟ-ಪ-ಡುತ್ತಾರೆ
ಕಷ್ಟ-ಪ-ಡುತ್ತಿದ್ದ
ಕಷ್ಟ-ಪ-ಡುತ್ತಿದ್ದಾ-ರೆಂದು
ಕಷ್ಟ-ಪ-ರಂಪ-ರೆ-ಗ-ಳಿಂದ
ಕಷ್ಟ-ಪ-ರಂಪ-ರೆಯೇ
ಕಷ್ಟವನ್ನು
ಕಷ್ಟವನ್ನೂ
ಕಷ್ಟವಲ್ಲ
ಕಷ್ಟ-ವಾ-ಗದು
ಕಷ್ಟ-ವಾ-ಗ-ಬ-ಹುದು
ಕಷ್ಟ-ವಾ-ಗ-ಲಿಲ್ಲ
ಕಷ್ಟ-ವಾ-ಗು-ವುದೇ
ಕಷ್ಟ-ವಾ-ದರೂ
ಕಷ್ಟ-ವಾ-ಯಿತು
ಕಷ್ಟ-ವಾ-ಯಿ-ತೆಂದು
ಕಷ್ಟವಿತ್ತು
ಕಷ್ಟವಿಲ್ಲ
ಕಷ್ಟ-ವಿಲ್ಲದೇ
ಕಷ್ಟವೂ
ಕಷ್ಟವೆಂಬ
ಕಷ್ಟ-ವೆ-ನಿ-ಸಿ-ದರೂ
ಕಷ್ಟ-ವೆ-ನಿ-ಸುತ್ತದೆ
ಕಷ್ಟವೇ
ಕಷ್ಟವೇನೂ
ಕಷ್ಟ-ಸಂಕಟ
ಕಷ್ಟ-ಸಂಕ-ಟ-ಗ-ಳನ್ನು
ಕಷ್ಟ-ಸಂಕ-ಟ-ಗಳು
ಕಷ್ಟ-ಸ-ಹಿಷ್ಣು-ಗ-ಳಾಗಿ
ಕಷ್ಟ-ಸಾಧ್ಯ-ವಾ-ಗು-ವು-ದಕ್ಕೆ
ಕಷ್ಟ-ಸಾಧ್ಯವೂ
ಕಷ್ಟ-ಸಾಧ್ಯವೇ
ಕಸ
ಕಸ-ಕೊ-ಳೆ-ಯನ್ನು
ಕಸಬು
ಕಸ-ರತ್ತಲ್ಲ
ಕಸಿ-ವಿ-ಸಿ-ಗೊಳ್ಳುತ್ತದೆ
ಕಸು-ಬಾ-ಗಿದ್ದ
ಕಹಳೆ
ಕಹ-ಳೆ-ಯೂದಿ
ಕಹಿ
ಕಹಿ-ಫ-ಲ-ವನ್ನು
ಕಹಿ-ಭಾ-ವನೆ
ಕಹಿಯೂ
ಕಾ
ಕಾಂಗ್ರೆಸ್
ಕಾಂಗ್ರೆಸ್ಸಿ-ನಲ್ಲಿ
ಕಾಂತ-ಶಕ್ತಿ-ಯಂತೂ
ಕಾಂತಿ
ಕಾಂತಿವಂತ
ಕಾಂಪೆನ್ಸೇ-ಷನ್
ಕಾಕ-ತಾ-ಳೀಯ
ಕಾಗದ
ಕಾಗದದ
ಕಾಗ-ದ-ಪತ್ರ-ಗ-ಳನ್ನು
ಕಾಗ-ದ-ವನ್ನೇ
ಕಾಗೆ
ಕಾಟ
ಕಾಟದಿಂದ
ಕಾಟವನ್ನು
ಕಾಟ-ವಾ-ಗಿ-ರ-ಬ-ಹುದೇ
ಕಾಟ-ವಿ-ರಲಿ
ಕಾಡ-ತೊ-ಡ-ಗಿತು
ಕಾಡ-ತೊ-ಡ-ಗಿದ್ದಾನೆ
ಕಾಡದೆ
ಕಾಡನ್ನು
ಕಾಡ-ಬೇ-ಕಷ್ಟೆ
ಕಾಡಲು
ಕಾಡಿ
ಕಾಡಿಗೆ
ಕಾಡಿತು
ಕಾಡಿನ
ಕಾಡಿನಲ್ಲಿ
ಕಾಡಿ-ರ-ಲಿಲ್ಲ
ಕಾಡು
ಕಾಡುತ್ತಲೇ
ಕಾಡುತ್ತವೆ
ಕಾಡುತ್ತಿದ್ದ
ಕಾಡುತ್ತಿದ್ದವು
ಕಾಡುತ್ತಿ-ರುವ
ಕಾಡುತ್ತೇನೆ
ಕಾಡು-ದಾ-ರಿ-ಯಲ್ಲಿ
ಕಾಡುಪ್ರಾ-ಣಿ-ಗಳೂ
ಕಾಣ-ದ-ವಳು
ಕಾಣ-ಬಲ್ಲೆವು
ಕಾಣ-ತೊ-ಡ-ಗಿ-ದವು
ಕಾಣದ
ಕಾಣದಂತೆ
ಕಾಣದಾಗಿ
ಕಾಣ-ದಾ-ಗಿ-ದೆಯೆ
ಕಾಣ-ದಾ-ಗುತ್ತದೆ
ಕಾಣ-ದಿದ್ದರೆ
ಕಾಣ-ದಿದ್ದಾಗ
ಕಾಣ-ದಿ-ರದು
ಕಾಣ-ದಿ-ರಲು
ಕಾಣದು
ಕಾಣದೆ
ಕಾಣದೇ
ಕಾಣ-ಬ-ಯ-ಸುತ್ತಾ-ರೆಯೆ
ಕಾಣ-ಬ-ರುವ
ಕಾಣ-ಬಲ್ಲ-ವ-ರಿಗೆ
ಕಾಣ-ಬ-ಹು-ದಲ್ಲವೇ
ಕಾಣ-ಬ-ಹು-ದಾ-ದರೂ
ಕಾಣ-ಬ-ಹುದು
ಕಾಣ-ಬ-ಹುದೆ
ಕಾಣ-ಬ-ಹು-ದೆಂಬು-ದನ್ನು
ಕಾಣ-ಬೇ-ಕಾ-ದರೆ
ಕಾಣ-ಬೇ-ಕೆಂಬು-ದಾಗಿ
ಕಾಣ-ಲಾ-ಗ-ಲಿಲ್ಲ
ಕಾಣ-ಲಾ-ರ-ದವ
ಕಾಣ-ಲಾ-ರ-ದಾ-ಗಿದ್ದಾರೆ
ಕಾಣ-ಲಾ-ರದು
ಕಾಣ-ಲಾ-ರದ್ದ-ರಿಂದಲ್ಲವೇ
ಕಾಣ-ಲಾ-ರಿರಿ
ಕಾಣಲಿ
ಕಾಣಲಿಲ್ಲ
ಕಾಣಲು
ಕಾಣ-ಸಿ-ಕೊಂಡರು
ಕಾಣ-ಸಿ-ಕೊಳ್ಳು-ವನು
ಕಾಣ-ಸಿ-ಗದ
ಕಾಣ-ಸಿ-ಗುವ
ಕಾಣಿಕೆ
ಕಾಣಿ-ಕೆ-ಗ-ಳನ್ನು
ಕಾಣಿ-ಕೆ-ಡಬ್ಬಿಗೆ
ಕಾಣಿ-ಕೆ-ಯನ್ನಿತ್ತ
ಕಾಣಿ-ಕೆ-ಯಾಗಿ
ಕಾಣಿ-ಕೆ-ಯೊಂದಿದೆ
ಕಾಣಿಸ
ಕಾಣಿ-ಸ-ದಿದ್ದರೂ
ಕಾಣಿ-ಸ-ತೊ-ಡ-ಗಿವೆ
ಕಾಣಿಸದ
ಕಾಣಿ-ಸ-ದಂತೆ
ಕಾಣಿಸದು
ಕಾಣಿಸದೇ
ಕಾಣಿಸನು
ಕಾಣಿ-ಸ-ಲಿಲ್ಲ
ಕಾಣಿಸಿ
ಕಾಣಿ-ಸಿ-ಕೊಳ್ಳುತ್ತಾರೆ
ಕಾಣಿ-ಸಿ-ಕೊಂಡ
ಕಾಣಿ-ಸಿ-ಕೊಂಡರು
ಕಾಣಿ-ಸಿ-ಕೊಂಡವು
ಕಾಣಿ-ಸಿ-ಕೊಂಡಿತು
ಕಾಣಿ-ಸಿ-ಕೊಂಡಿತ್ತು
ಕಾಣಿ-ಸಿ-ಕೊಂಡಿದ್ದಾರೆ
ಕಾಣಿ-ಸಿ-ಕೊಂಡಿ-ರ-ಬ-ಹುದು
ಕಾಣಿ-ಸಿ-ಕೊಂಡು
ಕಾಣಿ-ಸಿ-ಕೊಳ್ಳ-ದಿದ್ದರೂ
ಕಾಣಿ-ಸಿ-ಕೊಳ್ಳದೆ
ಕಾಣಿ-ಸಿ-ಕೊಳ್ಳ-ಬಲ್ಲ
ಕಾಣಿ-ಸಿ-ಕೊಳ್ಳ-ಬ-ಹುದು
ಕಾಣಿ-ಸಿ-ಕೊಳ್ಳುತ್ತದೆ
ಕಾಣಿ-ಸಿ-ಕೊಳ್ಳುತ್ತವೆ
ಕಾಣಿ-ಸಿ-ಕೊಳ್ಳುತ್ತಿದ್ದ
ಕಾಣಿ-ಸಿ-ಕೊಳ್ಳು-ವನು
ಕಾಣಿ-ಸಿ-ತಂತೆ
ಕಾಣಿಸಿತು
ಕಾಣಿಸಿದ
ಕಾಣಿ-ಸಿ-ದರೂ
ಕಾಣಿ-ಸುತ್ತದೆ
ಕಾಣಿ-ಸುತ್ತಾನೆ
ಕಾಣಿ-ಸುತ್ತಿತ್ತು
ಕಾಣಿ-ಸುತ್ತಿ-ದೆ-ರಿ-ಪೇ-ರಿ-ಯಾದ
ಕಾಣಿ-ಸುತ್ತಿದ್ದ
ಕಾಣಿ-ಸುತ್ತಿದ್ದವು
ಕಾಣಿ-ಸುತ್ತಿಲ್ಲ-ವಲ್ಲ
ಕಾಣಿಸುವ
ಕಾಣಿ-ಸು-ವಂಥ-ವ-ನಲ್ಲ
ಕಾಣಿ-ಸು-ವು-ದಿಲ್ಲ
ಕಾಣಿ-ಸು-ವು-ದಿಲ್ಲವೇ
ಕಾಣಿ-ಸು-ವುದು
ಕಾಣಿ-ಸು-ವು-ದುಂಟು
ಕಾಣು-ವಂತಾ-ಗಲಿ
ಕಾಣುತ್ತ
ಕಾಣುತ್ತದೆ
ಕಾಣುತ್ತಲೇ
ಕಾಣುತ್ತಾ
ಕಾಣುತ್ತಾನೆ
ಕಾಣುತ್ತಾರೆ
ಕಾಣುತ್ತಾಳೆ
ಕಾಣುತ್ತಿತ್ತು
ಕಾಣುತ್ತಿದ್ದ
ಕಾಣುತ್ತಿದ್ದೇನೆ
ಕಾಣುತ್ತಿದ್ದೇವೆ
ಕಾಣುತ್ತಿ-ರುವ
ಕಾಣುತ್ತಿ-ರು-ವಷ್ಟ-ರಲ್ಲೇ
ಕಾಣುತ್ತಿಲ್ಲ
ಕಾಣುತ್ತೇವೆ
ಕಾಣುತ್ತೇ-ವೆಯೇ
ಕಾಣುತ್ತೇವೋ
ಕಾಣುವ
ಕಾಣುವಂತೆ
ಕಾಣುವಂಥ
ಕಾಣುವಾಗ
ಕಾಣು-ವು-ದಂತೂ
ಕಾಣು-ವು-ದಿಲ್ಲ
ಕಾಣು-ವು-ದಿಲ್ಲ-ವಲ್ಲ
ಕಾಣುವುದು
ಕಾಣುವುವು
ಕಾಣುವೆವು
ಕಾಣೆಎಂದು
ಕಾಣೆ-ಯಾ-ಗಿ-ದೆ-ಕ-ರೆಂಟ್
ಕಾಣೆಯಾದ
ಕಾಣೆವು
ಕಾತರ
ಕಾತರತೆ
ಕಾತ-ರ-ತೆ-ಯಿಂದ
ಕಾತ-ರ-ನಾ-ಗಿದ್ದರೆ
ಕಾತ-ರ-ನಾ-ಗಿ-ರುವೆ
ಕಾತ-ರಿ-ಸದೆ
ಕಾತ-ರಿ-ಸಿದ
ಕಾತ-ರಿ-ಸುತ್ತಾ
ಕಾತ-ರಿ-ಸುವ
ಕಾದ
ಕಾದಂಬ-ರಿ-ಗಳು
ಕಾದಂಬ-ರಿ-ಗಾ-ರ-ನಾ-ಗಲು
ಕಾದಾ-ಡುತ್ತಾರೆ
ಕಾದಾ-ಡು-ವು-ದನ್ನು
ಕಾದಿತ್ತು
ಕಾದಿತ್ತೆಂಬುದು
ಕಾದಿದೆ
ಕಾದಿದೆಯೋ
ಕಾದಿದ್ದ
ಕಾದಿ-ರುತ್ತದೆ
ಕಾದು
ಕಾದು-ಕು-ಳಿ-ತಿದ್ದಾರೆ
ಕಾನೂನಿನ
ಕಾನೂನು
ಕಾನೂ-ನು-ಗ-ಳನ್ನು
ಕಾನೂ-ನು-ಬದ್ಧ
ಕಾನ್ವೆಂಟಿ-ನಲ್ಲಿ
ಕಾನ್ವೆಂಟೊಂದ-ರಲ್ಲಿ
ಕಾಪಾ-ಡ-ಬೇಕು
ಕಾಪಾಡಲು
ಕಾಪಾ-ಡಿ-ಕೊಳ್ಳಲು
ಕಾಪಾ-ಡಿ-ಕೊಳ್ಳುವ
ಕಾಪಾ-ಡಿ-ಕೊಳ್ಳು-ವುದು
ಕಾಪಾಡಿದ
ಕಾಪಾ-ಡಿದ್ದವು
ಕಾಪಾಡು
ಕಾಪಾ-ಡುತ್ತಾ-ನೆಂಬುದು
ಕಾಪಾಡುವ
ಕಾಪಿ
ಕಾಪಿ-ಪುಸ್ತಕ
ಕಾಪಿ-ಹೊ-ಡೆ-ಯುವ
ಕಾಪ್ರಾ
ಕಾಪ್ರಾರ
ಕಾಫಿ
ಕಾಫಿಗೆ
ಕಾಮ
ಕಾಮಪ್ರ-ಚೋ-ದಕ
ಕಾಮಕ್ಕೇ
ಕಾಮ-ತೃ-ಷೆ-ಯನ್ನು
ಕಾಮತ್
ಕಾಮದ
ಕಾಮಧೇನು
ಕಾಮನ
ಕಾಮ-ನೆ-ಗ-ಳನ್ನು
ಕಾಮ-ನೆ-ಗ-ಳಿಂದ
ಕಾಮಪ್ರ-ಚೋ-ದಕ
ಕಾಮವನ್ನೇ
ಕಾಮ-ವಿ-ರದು
ಕಾಮವೇ
ಕಾಮಾದಿ
ಕಾಮಾ-ದಿ-ದೋ-ಷ-ರ-ಹಿತಂ
ಕಾಮಾಸಕ್ತಿ
ಕಾಮಾ-ಸಕ್ತಿಯು
ಕಾಮೇಚ್ಛೆಯ
ಕಾಯಕ
ಕಾಯಕದ
ಕಾಯಕಲ್ಪ
ಕಾಯ-ಕ-ವನ್ನು
ಕಾಯದೆ
ಕಾಯಬೇಕಾ
ಕಾಯ-ಬೇ-ಕಾಗಿ
ಕಾಯ-ಬೇ-ಕಾ-ಗುತ್ತದೆ
ಕಾಯಿದೆ
ಕಾಯಿ-ದೆ-ಗ-ಳನ್ನೂ
ಕಾಯಿಯಂತೆ
ಕಾಯಿ-ಯಲ್ಲಿದ್ದ
ಕಾಯಿ-ಯಾ-ಗಿದ್ದಾಗ
ಕಾಯಿರಿ
ಕಾಯಿಲೆ
ಕಾಯಿ-ಲೆ-ಗ-ಳಲ್ಲಿ
ಕಾಯಿ-ಲೆ-ಯನ್ನು
ಕಾಯಿ-ಲೆ-ಯಾ-ದರೆ
ಕಾಯಿ-ಲೆ-ಯಿಂದ
ಕಾಯಿಸಿ
ಕಾಯಿ-ಸಿ-ದರು
ಕಾಯಿಸುವ
ಕಾಯು
ಕಾಯುತ್ತಿದ್ದ
ಕಾಯುತ್ತಿದ್ದರು
ಕಾಯುತ್ತ
ಕಾಯುತ್ತಾ
ಕಾಯುತ್ತಿದ್ದ
ಕಾಯುತ್ತಿದ್ದರು
ಕಾಯುವಂಥ
ಕಾಯು-ವ-ವನೂ
ಕಾಯ್ದುಕೊಂಡು
ಕಾಯ್ದು-ಕೊಳ್ಳಲು
ಕಾಯ್ದು-ಕೊಳ್ಳುವ
ಕಾರ
ಕಾರಂತ
ಕಾರಂತನೋ
ಕಾರಂತರ
ಕಾರಂತ-ರನ್ನು
ಕಾರಂತರು
ಕಾರ-ಕೂ-ನರ
ಕಾರಣ
ಕಾರ-ಣ-ಗ-ಳಿ-ರ-ಬೇ-ಕೆಂದು
ಕಾರ-ಣ-ಕೊಡಿ
ಕಾರ-ಣಕ್ಕಾಗಿ
ಕಾರ-ಣ-ಗ-ಳನ್ನು
ಕಾರ-ಣ-ಗ-ಳಲ್ಲ
ಕಾರ-ಣ-ಗ-ಳಲ್ಲಿ
ಕಾರ-ಣ-ಗ-ಳಾ-ಗಿ-ರ-ಲಿಲ್ಲ
ಕಾರ-ಣ-ಗ-ಳಿಂದ
ಕಾರ-ಣ-ಗ-ಳಿ-ಗಾಗಿ
ಕಾರ-ಣ-ಗ-ಳಿ-ರ-ಬ-ಹುದು
ಕಾರ-ಣ-ಗಳು
ಕಾರ-ಣ-ಗಳೂ
ಕಾರ-ಣ-ಗ-ಳೇನು
ಕಾರಣದ
ಕಾರ-ಣ-ದಲ್ಲಿ
ಕಾರ-ಣ-ದಿಂದ
ಕಾರ-ಣ-ದಿಂದಲೂ
ಕಾರ-ಣ-ನಾ-ಗುತ್ತಾ-ನೆಂಬ
ಕಾರಣಫಿ
ಕಾರ-ಣ-ರಾ-ಗುತ್ತಾರೆ
ಕಾರ-ಣ-ರಾದ
ಕಾರ-ಣ-ರಾ-ದರು
ಕಾರಣರು
ಕಾರ-ಣ-ರೆಂದು
ಕಾರ-ಣ-ರೆಂಬು-ದನ್ನು
ಕಾರ-ಣ-ವಂತೂ
ಕಾರ-ಣ-ವನ್ನು
ಕಾರ-ಣ-ವನ್ನೂ
ಕಾರ-ಣ-ವಲ್ಲ
ಕಾರ-ಣ-ವಲ್ಲವೆ
ಕಾರ-ಣ-ವಾ-ಗ-ಬಲ್ಲಂಥ-ವು-ಗಳು
ಕಾರ-ಣ-ವಾ-ಗ-ಬಲ್ಲವು
ಕಾರ-ಣ-ವಾ-ಗ-ಬ-ಹು-ದಾದ
ಕಾರ-ಣ-ವಾ-ಗ-ಬ-ಹುದು
ಕಾರ-ಣ-ವಾ-ಗ-ಬಾ-ರದು
ಕಾರ-ಣ-ವಾ-ಗಿತ್ತು
ಕಾರ-ಣ-ವಾ-ಗಿಲ್ಲ
ಕಾರ-ಣ-ವಾ-ಗಿವೆ
ಕಾರ-ಣ-ವಾಗು
ಕಾರ-ಣ-ವಾ-ಗುತ್ತದೆ
ಕಾರ-ಣ-ವಾ-ಗುತ್ತವೆ
ಕಾರ-ಣ-ವಾ-ಗುತ್ತಿ-ವೆ-ಯಲ್ಲ
ಕಾರ-ಣ-ವಾ-ಗುವ
ಕಾರ-ಣ-ವಾ-ಗು-ವಂತೆ
ಕಾರ-ಣ-ವಾ-ಗು-ವಂಥ
ಕಾರ-ಣ-ವಾ-ಗು-ವುದು
ಕಾರ-ಣ-ವಾ-ಗು-ವುವು
ಕಾರ-ಣ-ವಾದ
ಕಾರ-ಣ-ವಾ-ದದ್ದು
ಕಾರ-ಣ-ವಾ-ದರೂ
ಕಾರ-ಣ-ವಾ-ದರೆ
ಕಾರ-ಣ-ವಾ-ದವು
ಕಾರ-ಣ-ವಾ-ದೀತು
ಕಾರ-ಣ-ವಾ-ದು-ದ-ರಿಂದ
ಕಾರ-ಣ-ವಾ-ದೆ-ವಲ್ಲ
ಕಾರ-ಣ-ವಾದ್ದ-ರಿಂದ
ಕಾರ-ಣ-ವಾ-ಯಿತು
ಕಾರ-ಣ-ವಾರಿ
ಕಾರ-ಣ-ವಾ-ರಿ-ಯಲ್ಲ
ಕಾರ-ಣ-ವಿ-ರ-ಬೇಕು
ಕಾರ-ಣ-ವಿ-ರಲಿ
ಕಾರ-ಣ-ವಿ-ರು-ವಂತೆ
ಕಾರ-ಣ-ವಿಲ್ಲದೆ
ಕಾರ-ಣ-ವಿಲ್ಲದೇ
ಕಾರ-ಣ-ವಿಷ್ಟೆ
ಕಾರಣವು
ಕಾರಣವೂ
ಕಾರ-ಣ-ವೆಂದ
ಕಾರ-ಣ-ವೆಂದರೆ
ಕಾರ-ಣ-ವೆಂದಾ-ದರೆ
ಕಾರ-ಣ-ವೆಂದಾ-ಯಿ-ತಲ್ಲವೇ
ಕಾರ-ಣ-ವೆಂದಾ-ಯಿತು
ಕಾರ-ಣ-ವೆಂದು
ಕಾರ-ಣ-ವೆಂಬುದು
ಕಾರ-ಣ-ವೆನ್ನಲು
ಕಾರ-ಣ-ವೆಲ್ಲಿ-ರ-ಬೇಕು
ಕಾರಣವೇ
ಕಾರ-ಣ-ವೇ-ನಿ-ರ-ಬ-ಹುದು
ಕಾರ-ಣ-ವೇನು
ಕಾರ-ಣ-ವೇ-ನೆಂದರೆ
ಕಾರ-ಣ-ವೇ-ನೆಂಬು-ದನ್ನು
ಕಾರ-ಣ-ವೊಂದು
ಕಾರ-ಣ-ಸು-ಮಾರು
ಕಾರ-ಣಾಂತ-ರ-ಗ-ಳಿಂದ
ಕಾರ-ಣೀ-ಭೂ-ತ-ರಾ-ದ-ವರು
ಕಾರನ್ನು
ಕಾರಾ-ಗೃ-ಹ-ದಿಂದ
ಕಾರಿಗೆ
ಕಾರಿನ
ಕಾರಿ-ನೊ-ಳ-ಗಿದ್ದ
ಕಾರು
ಕಾರುಗಳ
ಕಾರು-ಣಿ-ಕನೂ
ಕಾರುಣ್ಯ
ಕಾರುತ್ತವೆ
ಕಾರ್ಖಾ-ನೆ-ಗಳ
ಕಾರ್ಖಾ-ನೆ-ಗ-ಳಲ್ಲಿ
ಕಾರ್ಖಾನೆಯ
ಕಾರ್ಖಾ-ನೆ-ಯಾ-ಗುತ್ತಾನೆ
ಕಾರ್ಡ್
ಕಾರ್ನೆಗೀ
ಕಾರ್ಪಣ್ಯ-ಗ-ಳನ್ನು
ಕಾರ್ಮಿಕರ
ಕಾರ್ಮಿಕರು
ಕಾರ್ಮಿಕ್
ಕಾರ್ಮೋಡ
ಕಾರ್ಮೋ-ಡ-ಗಳು
ಕಾರ್ಯ
ಕಾರ್ಯಒಂದೇ
ಕಾರ್ಯ-ಕರ್ತ-ರೊಂದಿಗೆ
ಕಾರ್ಯಕರ್ತೆ
ಕಾರ್ಯ-ಕಾ-ರಣ
ಕಾರ್ಯ-ಕಾ-ರ-ಣ-ಗಳ
ಕಾರ್ಯ-ಕಾ-ರ-ಣ-ಗ-ಳನ್ನು
ಕಾರ್ಯ-ಕಾ-ರ-ಣದ
ಕಾರ್ಯಕ್ಕಾಗಿ
ಕಾರ್ಯಕ್ಕಿ-ಳಿ-ಸ-ದುದೇ
ಕಾರ್ಯಕ್ಕೂ
ಕಾರ್ಯಕ್ಕೆ
ಕಾರ್ಯಕ್ಕೆಂದು
ಕಾರ್ಯಕ್ರಮ
ಕಾರ್ಯಕ್ರ-ಮ-ಗಳ
ಕಾರ್ಯಕ್ರ-ಮ-ಗ-ಳನ್ನು
ಕಾರ್ಯಕ್ರ-ಮ-ದಲ್ಲಿ
ಕಾರ್ಯಕ್ಷೇತ್ರ-ದಲ್ಲಿ
ಕಾರ್ಯಗತ
ಕಾರ್ಯ-ಗ-ತ-ಗೊ-ಳಿ-ಸುತ್ತದೆ
ಕಾರ್ಯ-ಗ-ತ-ವಾ-ಗಲೂ
ಕಾರ್ಯಗಳ
ಕಾರ್ಯ-ಗ-ಳನ್ನು
ಕಾರ್ಯ-ಗ-ಳನ್ನೂ
ಕಾರ್ಯ-ಗ-ಳಲ್ಲಿ
ಕಾರ್ಯ-ಗ-ಳಿಗೆ
ಕಾರ್ಯ-ಗ-ಳಿಲ್ಲದೇ
ಕಾರ್ಯಗಳು
ಕಾರ್ಯಗಳೂ
ಕಾರ್ಯ-ಗ-ಳೆಲ್ಲ
ಕಾರ್ಯತಃ
ಕಾರ್ಯ-ತತ್ಪ-ರತೆ
ಕಾರ್ಯದ
ಕಾರ್ಯ-ದಕ್ಷತೆ
ಕಾರ್ಯ-ದಕ್ಷ-ತೆಗೆ
ಕಾರ್ಯ-ದಕ್ಷ-ತೆಯ
ಕಾರ್ಯ-ದರ್ಶಿ-ನಿ-ಯೊಬ್ಬರ
ಕಾರ್ಯದಲ್ಲಿ
ಕಾರ್ಯ-ದಲ್ಲಿಯೇ
ಕಾರ್ಯದಲ್ಲೂ
ಕಾರ್ಯದಲ್ಲೇ
ಕಾರ್ಯ-ನಿ-ರ-ತ-ನಾಗಿ
ಕಾರ್ಯ-ನಿ-ರ-ತ-ನಾ-ದರೆ
ಕಾರ್ಯ-ನಿ-ರ-ತ-ರಾ-ಗ-ಬೇಕು
ಕಾರ್ಯ-ನಿ-ರ-ತ-ರಾ-ಗಿ-ರು-ವುದು
ಕಾರ್ಯ-ನಿರ್ವ-ಹಣೆ
ಕಾರ್ಯ-ನಿರ್ವ-ಹಿ-ಸುತ್ತಾರೆ
ಕಾರ್ಯ-ನಿರ್ವಾ-ಹ-ಕರು
ಕಾರ್ಯ-ನಿಷ್ಠೆ-ಯನ್ನೂ
ಕಾರ್ಯನೀತಿ
ಕಾರ್ಯಪ್ರ-ವೃತ್ತ-ರನ್ನಾಗಿ
ಕಾರ್ಯ-ಭಾ-ರ-ತ-ಲೆ-ಭಾರ
ಕಾರ್ಯಮಗ್ನ
ಕಾರ್ಯ-ಮಗ್ನ-ರಾಗಿ
ಕಾರ್ಯ-ಮಾ-ಡುತ್ತಾ
ಕಾರ್ಯ-ರೂ-ಪಕ್ಕೂ
ಕಾರ್ಯ-ರೂ-ಪಕ್ಕೆ
ಕಾರ್ಯವನ್ನು
ಕಾರ್ಯವನ್ನೂ
ಕಾರ್ಯವನ್ನೇ
ಕಾರ್ಯ-ವಾ-ಗಿತ್ತು
ಕಾರ್ಯ-ವಿ-ಧಾನ
ಕಾರ್ಯ-ವಿ-ಧಾ-ನ-ಗ-ಳಲ್ಲಿ
ಕಾರ್ಯವಿಲ್ಲ
ಕಾರ್ಯವು
ಕಾರ್ಯವೂ
ಕಾರ್ಯ-ವೆ-ಸ-ಗಿದ್ದೆ
ಕಾರ್ಯವೇ
ಕಾರ್ಯ-ವೈ-ಶಿಷ್ಟ್ಯ
ಕಾರ್ಯ-ಶಕ್ತಿ-ಗ-ಳನ್ನು
ಕಾರ್ಯ-ಶಕ್ತಿಗೆ
ಕಾರ್ಯ-ಶಕ್ತಿಯ
ಕಾರ್ಯ-ಶಕ್ತಿ-ಯನ್ನೂ
ಕಾರ್ಯ-ಸಾ-ಮರ್ಥ್ಯ
ಕಾರ್ಯ-ಸಾ-ಮರ್ಥ್ಯ-ವನ್ನು
ಕಾರ್ಯ-ಸಾ-ಮರ್ಥ್ಯ-ವನ್ನೂ
ಕಾರ್ಯಾಗಾರ
ಕಾರ್ಯಾರ್ಥ-ವಾಗಿ
ಕಾರ್ಯೋತ್ಸಾಹ
ಕಾರ್ಯೋತ್ಸಾ-ಹ-ಇ-ವು-ಗಳ
ಕಾರ್ಯೋನ್ಮು-ಖ-ರಾಗಿ
ಕಾರ್ಯೋನ್ಮು-ಖ-ರಾ-ದಲ್ಲಿ
ಕಾರ್ಲೈಲನ
ಕಾರ್ಲೈಲ್
ಕಾರ್ಸ್ನಿ-ಕೋ-ಲಾೖವ್
ಕಾಲ
ಕಾಲ-ಕ-ಳೆ-ಯದೆ
ಕಾಲ-ಕ-ಳೆ-ಯ-ಬ-ಹುದು
ಕಾಲ-ಕ-ಳೆ-ಯಲು
ಕಾಲ-ಕ-ಳೆ-ಯು-ವು-ದಕ್ಕಿಂತ
ಕಾಲಕ್ಕೂ
ಕಾಲಕ್ಕೆ
ಕಾಲಕ್ಷೇಪ
ಕಾಲ-ಗ-ತಿ-ಯಿಂದ
ಕಾಲಗಳ
ಕಾಲ-ಗ-ಳನ್ನು
ಕಾಲ-ಗ-ಳಲ್ಲಿ
ಕಾಲ-ಗ-ಳಲ್ಲೂ
ಕಾಲ-ಗ-ಳಿಗೂ
ಕಾಲ-ಚಕ್ರದ
ಕಾಲಡಿಯ
ಕಾಲದ
ಕಾಲದಲ್ಲಿ
ಕಾಲ-ದಲ್ಲಿದ್ದು
ಕಾಲದಲ್ಲೂ
ಕಾಲದಲ್ಲೇ
ಕಾಲದಿಂದ
ಕಾಲ-ದಿಂದಲೂ
ಕಾಲ-ದೇ-ಶ-ಗಳು
ಕಾಲದ್ದಾ-ದರೂ
ಕಾಲ-ಧರ್ಮ-ದಿಂದ
ಕಾಲನ
ಕಾಲನು
ಕಾಲನ್ನು
ಕಾಲನ್ನೇ
ಕಾಲಭೇದ
ಕಾಲಮುಂದೆ
ಕಾಲಮೇಲೆ
ಕಾಲ-ಯಾ-ಪನೆ
ಕಾಲವನ್ನು
ಕಾಲ-ವಾ-ದರೂ
ಕಾಲ-ವಿ-ದೆ-ಯೆಂದು
ಕಾಲವಿದ್ದು
ಕಾಲ-ಹ-ರಣ
ಕಾಲ-ಹ-ರ-ಣಕ್ಕೆ
ಕಾಲ-ಹ-ರ-ಣ-ವಾ-ಯಿ-ತೆಂದು
ಕಾಲಾಂತ-ರ-ದಲ್ಲಿ
ಕಾಲಾ-ಧೀ-ನ-ವೆನ್ನುತ್ತ
ಕಾಲಾ-ವ-ಕಾಶ
ಕಾಲಾ-ವ-ಕಾ-ಶ-ಬೇಕು
ಕಾಲಾ-ವ-ಧಿಯ
ಕಾಲಾ-ವ-ಧಿ-ಯನ್ನು
ಕಾಲಿಕವೂ
ಕಾಲಿಗೆ
ಕಾಲಿನ
ಕಾಲಿನಿಂದ
ಕಾಲು
ಕಾಲು-ಕೆ-ರೆ-ಯು-ವು-ದನ್ನು
ಕಾಲು-ಗ-ಳನ್ನು
ಕಾಲುಗಳು
ಕಾಲುಗಳೂ
ಕಾಲುಗಳೇ
ಕಾಲುಜಾರಿ
ಕಾಲು-ಜಾ-ರಿ-ದರೆ
ಕಾಲು-ದಾ-ರಿ-ಯನ್ನು
ಕಾಲು-ಬೆ-ರ-ಳಿಗೆ
ಕಾಲುವೆ
ಕಾಲು-ವೆ-ಯನ್ನೂ
ಕಾಲೇ
ಕಾಲೇಜಿಗೆ
ಕಾಲೇಜಿನ
ಕಾಲೇ-ಜಿ-ನಲ್ಲಿ
ಕಾಲೇಜು
ಕಾಲೇ-ಜು-ಗ-ಳಲ್ಲಿ
ಕಾಲೇ-ಜೊಂದ-ರಲ್ಲಿ
ಕಾಲ್ಚೆಂಡಿ-ನಂತೆ
ಕಾಲ್ಪನಿಕ
ಕಾಳ
ಕಾಳಜಿ
ಕಾಳ-ಜಿ-ಗಳು
ಕಾಳಜಿಯ
ಕಾಳ-ಜಿ-ಯನ್ನು
ಕಾಳ-ಜಿ-ವ-ಹಿ-ಸು-ವ-ವರೇ
ಕಾಳ-ಸಂತೆ-ಗಳು
ಕಾಳ-ಸಂತೆಯ
ಕಾಳಿದಾಸ
ಕಾಳೀ
ಕಾಳೀಪದ
ಕಾಳೀ-ಪ-ದ-ನಲ್ಲಿ
ಕಾಳು
ಕಾಳು-ಗ-ಳನ್ನು
ಕಾವಲು
ಕಾವ-ಲು-ಗಾರ
ಕಾವಿ
ಕಾವಿನಿಂದ
ಕಾವು
ಕಾವೇ
ಕಾವ್ಯ-ಗ-ಳದ್ದು
ಕಾವ್ಯವನ್ನು
ಕಾಶಿಗೆ
ಕಾಶಿಯ
ಕಾಶಿಯಲ್ಲಿ
ಕಾಸನ್ನಿಟ್ಟು-ಕೊಳ್ಳದ
ಕಾಸನ್ನು
ಕಾಸಿನ
ಕಾಸು
ಕಿಂಕರಗೆ
ಕಿಂಕರ್ತವ್ಯ
ಕಿಂಕರ್ತವ್ಯ-ಮೂ-ಢ-ರನ್ನಾಗಿ
ಕಿಂಕರ್ತವ್ಯ-ಮೂ-ಢ-ರಾಗಿ
ಕಿಂಚಿತ್ತಾ-ದರೂ
ಕಿಂಚಿತ್ತೂ
ಕಿಂಡಿಗಳೇ
ಕಿಕ್ಕಿರಿದು
ಕಿಚ್ಚಲ್ಲ
ಕಿಚ್ಚು
ಕಿಟಕಿ
ಕಿಟ-ಕಿ-ಬಾ-ಗಿ-ಲು-ಗ-ಳನ್ನು
ಕಿಟಕಿಯ
ಕಿಟ-ಕಿ-ಯಲ್ಲಿ
ಕಿಟಿಕಿಯ
ಕಿಡಿ
ಕಿಡಿ-ಕಿ-ಡಿ-ಯಾಗಿ
ಕಿಡಿ-ಗ-ಳಲ್ಲವೆ
ಕಿಡಿಗಳು
ಕಿಡಿ-ಗೇ-ಡಿ-ಗಳೂ
ಕಿಡಿ-ಗೇ-ಡಿ-ತ-ನ-ಇ-ವೆಲ್ಲವೂ
ಕಿಡಿಯನ್ನು
ಕಿಡಿ-ಯಾ-ದು-ದ-ರಿಂದ
ಕಿತಾ-ಪ-ತಿ-ಮಾಡಿ
ಕಿತ್ತಳೆ
ಕಿತ್ತು
ಕಿತ್ತು-ಕೊಟ್ಟರೆ
ಕಿತ್ತು-ಕೊಳ್ಳ-ಬ-ಹು-ದೆಂಬ
ಕಿತ್ತು-ಬ-ರು-ವಂತೆ
ಕಿತ್ತು-ಹಾ-ಕಿದ
ಕಿತ್ತು-ಹೋ-ಯಿತು
ಕಿತ್ತೆಸೆದು
ಕಿತ್ತೆ-ಸೆ-ದುದು
ಕಿತ್ತೆಸೆಯ
ಕಿತ್ತೆ-ಸೆ-ಯದೆ
ಕಿತ್ತೆ-ಸೆ-ಯ-ಬ-ಹುದು
ಕಿತ್ತೆ-ಸೆ-ಯ-ಬ-ಹುದೇ
ಕಿತ್ತೆ-ಸೆ-ಯ-ಬೇಕು
ಕಿತ್ತೆ-ಸೆ-ಯುತ್ತೇನೆ
ಕಿರ-ಣ-ಗಳು
ಕಿರಾ-ತಾರ್ಜು-ನೀ-ಯ-ದಂಥ
ಕಿರಿಕಿರಿ
ಕಿರಿ-ಚಾ-ಡುತ್ತ
ಕಿರಿ-ಚಿ-ಕೊಂಡ
ಕಿರಿ-ಚಿ-ಕೊಂಡಾಗ
ಕಿರಿ-ಚುತ್ತಿ-ರುತ್ತಾ-ನಪ್ಪ
ಕಿರಿ-ದು-ಅದು
ಕಿರಿಯ
ಕಿರಿಯರ
ಕಿರಿಯರು
ಕಿರಿ-ಯ-ವಳು
ಕಿರಿ-ಯು-ವಂತೆ
ಕಿರೀಟದ
ಕಿರೀ-ಟ-ದಲ್ಲಿ
ಕಿರೀ-ಟ-ವೊಂದನ್ನು
ಕಿರು-ಕು-ಳ-ಗ-ಳನ್ನೂ
ಕಿರು-ಕು-ಳ-ವನ್ನು
ಕಿರು-ಚುತ್ತಾರೆ
ಕಿರು-ಚುತ್ತಿ-ರುತ್ತಾನೆ
ಕಿರು-ದಾ-ರಿ-ಯಲ್ಲಿ
ಕಿರು-ನ-ಗೆ-ಯನ್ನು
ಕಿರ್ಕೆಗಾರ್ಡ್ನ
ಕಿಲೋ-ಮೀ-ಟರ್
ಕಿಲೋಗ್ರಾಂ
ಕಿಲೋ-ಮೀ-ಟರ್
ಕಿವಿ
ಕಿವಿಗಳು
ಕಿವಿಗೂ
ಕಿವಿಗೆ
ಕಿವಿಗೊಟ್ಟು
ಕಿವಿ-ಗೊ-ಡ-ಲಿಲ್ಲ
ಕಿವಿಯ
ಕಿವಿಯಲ್ಲಿ
ಕಿವುಡ
ಕಿವುಡನ
ಕಿವು-ಡ-ನಂತೆಯೇ
ಕಿವುಡಾಗಿ
ಕಿವು-ಡಾ-ಗಿದ್ದು
ಕಿಸೆಯ
ಕಿಸೆಯಿಂದ
ಕೀನ್ಸೆ
ಕೀಯವಾಗಿ
ಕೀರ್ತಿ
ಕೀರ್ತಿಗಾಗಿ
ಕೀರ್ತಿ-ವಂತ-ರಾಗಿ
ಕೀರ್ತಿ-ಶಾ-ಲಿ-ಗ-ಳಾ-ಗಿದ್ದರು
ಕೀರ್ತಿ-ಶಾ-ಲಿ-ಯಾದ
ಕೀರ್ತಿಸಿ
ಕೀರ್ಲಿ
ಕೀರ್ಲಿಯನ್
ಕೀಲಿಕೈ
ಕೀಲಿ-ಕೈ-ಯಂತಿದೆ
ಕೀಲುಗಳ
ಕೀಳರಿಮೆ
ಕೀಳ-ರಿ-ಮೆ-ಗಳು
ಕೀಳ-ರಿ-ಮೆಯ
ಕೀಳ-ರಿ-ಮೆ-ಯನ್ನು
ಕೀಳ-ರಿ-ಮೆ-ಯಿಂದ
ಕೀಳಲು
ಕೀಳಲ್ಲ
ಕೀಳಾಗಿ
ಕೀಳು
ಕೀಳು-ತ-ರದ
ಕುಂಟ
ಕುಂಟರು
ಕುಂಟು
ಕುಂಠಿ-ತ-ವಾಗಿ
ಕುಂಡಲಿನೀ
ಕುಂದದ
ಕುಂದದಂತೆ
ಕುಂದಾ-ಗ-ದಂತೆ
ಕುಂದಾ-ಪು-ರದ
ಕುಂದಾ-ಪು-ರ-ದಲ್ಲಿ
ಕುಂದು
ಕುಂದು-ಕೊ-ರ-ತೆ-ಗ-ಳಿ-ರ-ಲಿಲ್ಲ
ಕುಂದು-ಕೊ-ರ-ತೆ-ಗಳು
ಕುಂದು-ಗ-ಳನ್ನು
ಕುಂದು-ತ-ರುವ
ಕುಂಭ-ಕೋ-ಣ-ದಲ್ಲಿ
ಕುಕರ್ಮ-ಗಳ
ಕುಕರ್ಮದ
ಕುಕೃತ್ಯ
ಕುಕ್ಕಿದರು
ಕುಖ್ಯಾ-ತ-ರಾದ
ಕುಗ್ಗಿಸಿ
ಕುಗ್ಗಿ-ಸಿ-ಕೊಳ್ಳು-ವು-ದಕ್ಕಾಗಿ
ಕುಗ್ಗು-ನು-ಡಿಯ
ಕುಚೋದ್ಯದ
ಕುಟೀರ
ಕುಟೀ-ರ-ದಲ್ಲಿ
ಕುಟೀ-ರ-ದಲ್ಲಿ-ರುವ
ಕುಟುಂಬ
ಕುಟುಂಬಕ್ಕೂ
ಕುಟುಂಬಕ್ಕೆ
ಕುಟುಂಬ-ಗಳ
ಕುಟುಂಬ-ಗ-ಳಿಂದ
ಕುಟುಂಬದ
ಕುಟುಂಬ-ದಲ್ಲಿ
ಕುಟುಂಬ-ದ-ವ-ರನ್ನು
ಕುಟುಂಬ-ದ-ವರು
ಕುಟುಂಬ-ದಿಂದ
ಕುಟುಂಬ-ನಾಶ
ಕುಟುಂಬ-ರಕ್ಷ-ಣೆ-ಗಾಗಿ
ಕುಟುಂಬವೇ
ಕುಟ್ಟಿ
ಕುಠಾ-ರಪ್ರಾ-ಯ-ನಾ-ಗುತ್ತಾ-ನೆಂದು
ಕುಠಾ-ರಾ-ಘಾತ
ಕುಠಾ-ರಾ-ಘಾ-ತ-ವಾ-ಗು-ವಂತಹ
ಕುಠಾ-ರಾ-ಘಾ-ತ-ವಾಗಿ
ಕುಠಾ-ರಾ-ಘಾ-ತ-ವಾ-ಗಿ-ದೆ-ಯಲ್ಲ
ಕುಠಾ-ರಾ-ಘಾ-ತ-ವಾ-ಗು-ವಂಥ
ಕುಡಿತ
ಕುಡಿತದ
ಕುಡಿತವೂ
ಕುಡಿದ
ಕುಡಿದು
ಕುಡಿಯಲು
ಕುಡಿ-ಯುತ್ತಾರೆ
ಕುಡಿ-ಯುತ್ತಿದ್ದೆ
ಕುಡಿ-ಯು-ವುದು
ಕುಡಿ-ಯು-ವುದೇ
ಕುಡಿಸಿ
ಕುಡುಕನೂ
ಕುಣಿ-ಕೆ-ಯಲಿ
ಕುಣಿ-ದಾ-ಡಿ-ಯಾರು
ಕುಣಿ-ದಾ-ಡುತ್ತದೆ
ಕುಣಿದು
ಕುಣಿ-ಯ-ಬೇ-ಕಾ-ಗುತ್ತದೆ
ಕುಣಿ-ಯ-ವು-ದೆ-ಅಂದರೆ
ಕುಣಿ-ಯುತ್ತಿದ್ದೆ
ಕುಣಿ-ಯು-ವಂತೆ
ಕುಣಿ-ಯು-ವುದು
ಕುಣಿವ
ಕುಣಿ-ಸುತ್ತದೆ
ಕುಣಿಸುವ
ಕುತಂತ್ರ-ಗ-ಳಲ್ಲಿ
ಕುತರ್ಕ
ಕುತೂ
ಕುತೂಹಲ
ಕುತೂ-ಹ-ಲ-ಕಾರಿ
ಕುತೂ-ಹ-ಲ-ಕಾ-ರಿ-ಯಾ-ಗಿತ್ತು
ಕುತೂ-ಹ-ಲ-ಕಾ-ರಿ-ಯಾ-ಗಿದೆ
ಕುತೂ-ಹ-ಲ-ಕಾ-ರಿ-ಯಾದ
ಕುತೂ-ಹ-ಲ-ಕಾ-ರಿಯೂ
ಕುತೂ-ಹ-ಲ-ಗೊಂಡ
ಕುತೂ-ಹ-ಲ-ದಿಂದ
ಕುತೂ-ಹ-ಲ-ದಿಂದಲೇ
ಕುತೂ-ಹ-ಲ-ವನ್ನು
ಕುತೂ-ಹ-ಲ-ವಾ-ಗಿತ್ತು
ಕುತೂ-ಹ-ಲ-ವಿಲ್ಲ
ಕುತೂ-ಹ-ಲಿ-ಗ-ಳಾದ
ಕುತೂ-ಹ-ಲಿ-ಗ-ಳಾ-ಗಿದ್ದರು
ಕುತೂ-ಹ-ಲಿ-ಯಾ-ಯಿತು
ಕುತ್ತಿಗೆಗೆ
ಕುದಿ-ಯುತ್ತಿ-ರುತ್ತದೆ
ಕುದಿ-ಯುತ್ತಿ-ರುತ್ತಾನೆ
ಕುದಿವರು
ಕುದುರಲಿ
ಕುದುರಲೇ
ಕುದು-ರೆ-ಗಳ
ಕುದು-ರೆ-ಗಳು
ಕುದುರೆಗೆ
ಕುದು-ರೆ-ಯನ್ನು
ಕುಪಿ-ತ-ರಾಗಿ
ಕುಪಿ-ತ-ಳಾಗಿ
ಕುಪ್ಪಳಿಸಿ
ಕುಪ್ರ-ಸಿದ್ಧ-ವಾದ
ಕುಬೇರನ
ಕುಬ್ಲೆರ್
ಕುಬ್ಲೇರ್
ಕುಮಾರನು
ಕುಮಾರಿ
ಕುರಾನನ್ನೂ
ಕುರಾನ್
ಕುರಿ
ಕುರಿ-ತಾ-ಗಿಯೂ
ಕುರಿಗಳ
ಕುರಿ-ಗ-ಳಂತೆ
ಕುರಿ-ಗ-ಳಂತೆಯೇ
ಕುರಿಗಳು
ಕುರಿತ
ಕುರಿತಾಗಿ
ಕುರಿತಾದ
ಕುರಿ-ತಾ-ದದ್ದು
ಕುರಿ-ತಾ-ದುದು
ಕುರಿ-ತಾ-ದು-ವಲ್ಲ
ಕುರಿತು
ಕುರಿತೂ
ಕುರಿತೆ
ಕುರಿಮಂದೆ
ಕುರಿ-ಮಂದೆಯ
ಕುರಿ-ಮಂದೆ-ಯನ್ನು
ಕುರಿ-ಮಂದೆ-ಯಲ್ಲಿ
ಕುರಿಮರಿ
ಕುರಿಯಂತೆ
ಕುರಿ-ಯಂತೆಯೇ
ಕುರಿಯನ್ನು
ಕುರಿಯನ್ನೇ
ಕುರಿ-ಯಲ್ಲ-ವೆಂಬ
ಕುರಿ-ಯಾ-ಗಿ-ಸಿದ್ದು
ಕುರಿಯಾದ
ಕುರಿಯೆಂದು
ಕುರಿಯೆಂದೇ
ಕುರಿಹುಲಿ
ಕುರಿ-ಹು-ಲಿಗೆ
ಕುರಿ-ಹು-ಲಿ-ಯನ್ನು
ಕುರು
ಕುರುಡ
ಕುರು-ಡ-ನಾ-ಗಿದ್ದೆ
ಕುರು-ಡ-ನಾಗು
ಕುರು-ಡ-ನಾದ
ಕುರುಡರು
ಕುರು-ಡಾ-ಗಲಿ
ಕುರುಡಾಗಿ
ಕುರು-ಡಾ-ಗುತ್ತಾರೆ
ಕುರು-ಡಾ-ಗು-ವಂತೆ
ಕುರು-ಡಾ-ದವು
ಕುರುಡಿ
ಕುರು-ಡಿ-ಯನ್ನು
ಕುರುಡು
ಕುರೂ-ಪಿ-ಯಲ್ಲೂ
ಕುರ್ಚಿ
ಕುರ್ಚಿಯ
ಕುರ್ಚಿಯಲ್ಲಿ
ಕುಲ
ಕುಲಕ್ಕೆ
ಕುಲ-ಗೆ-ಡಿ-ಸಿ-ತೆಂಬು-ದರ
ಕುಲಗೋತ್ರ
ಕುಲದಲ್ಲಿ
ಕುಲ-ದ-ವ-ರಾ-ದರೂ
ಕುಲುಮೆ
ಕುಲು-ಮೆ-ಯಲ್ಲಿ
ಕುಳಿತ
ಕುಳಿ-ತ-ರಾ-ಯಿತು
ಕುಳಿತರು
ಕುಳಿ-ತಲ್ಲಿಗೆ
ಕುಳಿತಾಗ
ಕುಳಿತಿತ್ತು
ಕುಳಿತಿದೆ
ಕುಳಿತಿದ್ದ
ಕುಳಿ-ತಿದ್ದರು
ಕುಳಿ-ತಿದ್ದರೆ
ಕುಳಿ-ತಿದ್ದಾಗ
ಕುಳಿ-ತಿದ್ದಾರೆ
ಕುಳಿ-ತಿದ್ದು-ದ-ರಿಂದ
ಕುಳಿ-ತಿ-ರ-ದಿದ್ದರೆ
ಕುಳಿ-ತಿ-ರ-ಬೇ-ಕೆಂದಲ್ಲ
ಕುಳಿ-ತಿ-ರ-ಲಿಲ್ಲ
ಕುಳಿ-ತಿ-ರಲು
ಕುಳಿ-ತಿ-ರಲೂ
ಕುಳಿ-ತಿ-ರುತ್ತಾರೆ
ಕುಳಿ-ತಿ-ರುತ್ತಿದ್ದ
ಕುಳಿತು
ಕುಳಿ-ತು-ಕೊಂಡ
ಕುಳಿ-ತು-ಕೊಂಡಿದ್ದಾಗ
ಕುಳಿ-ತು-ಕೊಂಡು
ಕುಳಿ-ತು-ಕೊಳ್ಳದೇ
ಕುಳಿ-ತು-ಕೊಳ್ಳ-ಲಾ-ರನೋ
ಕುಳಿ-ತು-ಕೊಳ್ಳ-ಲಿಲ್ಲ
ಕುಳಿ-ತು-ಕೊಳ್ಳು-ವು-ದಕ್ಕೆ
ಕುಳಿ-ತು-ಕೊಳ್ಳೋ-ಣ-ವೆಂದು
ಕುಳಿ-ತು-ಬಿ-ಡುತ್ತಾನೆ
ಕುಳಿತೇ
ಕುಳ್ಳ-ನಾ-ಗಿ-ರು-ವು-ದ-ರಿಂದ
ಕುಳ್ಳಿ-ರಿ-ಸಿ-ಕೊಂಡು
ಕುವೆಂಪು
ಕುಶ-ಲ-ಕ-ಲೆ-ಗ-ಳಾ-ಗಲಿ
ಕುಶಲತೆ
ಕುಶ-ಲ-ತೆ-ಯನ್ನು
ಕುಶಾ-ಲ-ಚಂದ-ರನ್ನು
ಕುಶಾಲ್ಚಂದರು
ಕುಷ್ಠ-ರೋ-ಗಕ್ಕೆ
ಕುಷ್ಠ-ರೋ-ಗ-ದಿಂದ
ಕುಸಂಸ್ಕಾ-ರ-ಗ-ಳಿಂದ
ಕುಸಿತ
ಕುಸಿತಕ್ಕೆ
ಕುಸಿ-ತ-ಗ-ಳನ್ನು
ಕುಸಿ-ತ-ದಿಂದ
ಕುಸಿ-ತ-ವನ್ನು
ಕುಸಿ-ತ-ವನ್ನೂ
ಕುಸಿದಿಲ್ಲ
ಕುಸಿದು
ಕುಸಿ-ದು-ಬಿದ್ದ
ಕುಸಿ-ದು-ಬಿದ್ದು
ಕುಸು-ಮ-ಗ-ಳತ್ತ
ಕುಸ್ತಿಯ
ಕುಹಕ
ಕುಹಕದ
ಕುಹ-ಕ-ವನ್ನು
ಕುಹ-ಕಿ-ಗಳ
ಕೂಗ-ಲಾ-ರಂಭಿ-ಸಿದ
ಕೂಗಾಟ
ಕೂಗಾ-ಡಿ-ದರು
ಕೂಗಾಡುತ್ತ
ಕೂಗಾಡುತ್ತಾ
ಕೂಗಾ-ಡು-ವುದು
ಕೂಗಿ
ಕೂಗಿಕೊಂಡ
ಕೂಗಿ-ಕೊಂಡರು
ಕೂಗಿ-ಕೊಂಡರೆ
ಕೂಗಿ-ಕೊಳ್ಳಲೂ
ಕೂಗಿದಂತೆ
ಕೂಗಿದಾಗ
ಕೂಗಿನ
ಕೂಗಿ-ಸಿ-ದ-ನಂತೆ
ಕೂಗು
ಕೂಗು-ಗ-ಳನ್ನೂ
ಕೂಗುತ್ತ
ಕೂಗುತ್ತಿತ್ತು
ಕೂಗುವ
ಕೂಗೊಂದು-ಯು-ರೇಕಾ
ಕೂಟ
ಕೂಟದಲ್ಲಿ
ಕೂಟ-ವೊಂದ-ರಲ್ಲಿ
ಕೂಡ
ಕೂಡದು
ಕೂಡದೆಂದೂ
ಕೂಡ-ಬೇ-ಕಿಲ್ಲ
ಕೂಡಲೆ
ಕೂಡಲೇ
ಕೂಡಾ
ಕೂಡಿ
ಕೂಡಿಕೊಂಡ
ಕೂಡಿಕೊಂಡು
ಕೂಡಿಟ್ಟರೆ
ಕೂಡಿಡುವ
ಕೂಡಿ-ಡು-ವುದೂ
ಕೂಡಿದ
ಕೂಡಿದೆ
ಕೂಡಿದ್ದ
ಕೂಡಿದ್ದಾರೆ
ಕೂಡಿದ್ದು
ಕೂಡಿಬಂದು
ಕೂಡಿಯೇ
ಕೂಡಿ-ರ-ಬೇಕು
ಕೂಡಿ-ರುತ್ತದೆ
ಕೂಡಿ-ರುತ್ತ-ದೆಂಬುದು
ಕೂಡಿರುವ
ಕೂಡಿವೆ
ಕೂಡಿ-ವೆ-ಎಂಬುದು
ಕೂಡಿಸಿ
ಕೂಡುತ್ತದೆ
ಕೂಡು-ವು-ದಿಲ್ಲ
ಕೂಡೆ
ಕೂತರು
ಕೂತರೆ
ಕೂತಿ-ರ-ಬೇಕೆ
ಕೂತು
ಕೂತು-ಕೊಳ್ಳು-ವಿ-ರಾ-ಎಂದರು
ಕೂಪಕ್ಕೆ
ಕೂಲಂಕ-ಷ-ವಾಗಿ
ಕೂಲಿ
ಕೂಸಾ-ಗಿ-ರು-ವಾಗ
ಕೂಸೆ
ಕೃತಕ
ಕೃತ-ಕ-ತೆಯ
ಕೃತ-ಕ-ತೆಯೂ
ಕೃತಕೃತ್ಯ
ಕೃತ-ಕೃತ್ಯ-ತೆಯ
ಕೃತ-ಕೃತ್ಯ-ತೆ-ಯನ್ನು
ಕೃತ-ಕೃತ್ಯ-ತೆ-ಯನ್ನುಂಟು-ಮಾ-ಡುವ
ಕೃತಘ್ನ-ತೆ-ಯನ್ನು
ಕೃತಘ್ನ-ನಾ-ಗ-ಬೇಡ
ಕೃತಜ್ಞತಾ
ಕೃತಜ್ಞತೆ
ಕೃತಜ್ಞ-ತೆ-ಗ-ಳನ್ನು
ಕೃತಜ್ಞ-ತೆಯ
ಕೃತಜ್ಞ-ತೆ-ಯನ್ನೂ
ಕೃತಜ್ಞ-ತೆ-ಯಿಂದ
ಕೃತಜ್ಞ-ನಾ-ಗಿದ್ದೇನೆ
ಕೃತಾರ್ಥ
ಕೃತಿ
ಕೃತಿ-ಗ-ಳಲ್ಲಿ
ಕೃತಿ-ಗ-ಳಿಂದ
ಕೃತಿ-ಗಿ-ಳಿ-ಸಲು
ಕೃತಿಯ
ಕೃತಿಯನ್ನು
ಕೃತಿಯಲ್ಲಿ
ಕೃತಿಯು
ಕೃತಿಶಃ
ಕೃತ್ಯ
ಕೃತ್ಯಕ್ಕಾಗಿ
ಕೃತ್ಯಗಳ
ಕೃತ್ಯ-ಗ-ಳನ್ನೆ-ಸ-ಗುತ್ತಾರೆ
ಕೃತ್ಯ-ಗ-ಳಿಂದ
ಕೃತ್ಯ-ಗ-ಳಿ-ಗೆ-ಳ-ಸ-ಬಲ್ಲದು
ಕೃತ್ಯವನ್ನು
ಕೃಪ-ಯಾ-ವಾರೇ
ಕೃಪಾ
ಕೃಪಾ-ಪ-ರ-ವ-ಶ-ನಾಗಿ
ಕೃಪಾಬಿಂದು
ಕೃಪಾ-ಲಾ-ಭಕ್ಕೆ
ಕೃಪಾವಾರಿ
ಕೃಪೆ
ಕೃಪೆಗಾಗಿ
ಕೃಪೆಗೆ
ಕೃಪೆಗೇ
ಕೃಪೆದೋರಿ
ಕೃಪೆದೋರು
ಕೃಪೆ-ದೋ-ರುತ್ತಾನೆ
ಕೃಪೆಯ
ಕೃಪೆಯನ್ನು
ಕೃಪೆಯಿಂದ
ಕೃಪೆಯಿಟ್ಟು
ಕೃಪೆ-ಯಿಲ್ಲದೆ
ಕೃಪೆಯೆಂಬ
ಕೃಶವಾಗಿ
ಕೃಷಿ
ಕೃಷ್ಣ
ಕೃಷ್ಣ-ದೇ-ವ-ರಾ-ಯನು
ಕೃಷ್ಣನನ್ನು
ಕೃಷ್ಣಯ್ಯರ್
ಕೃಷ್ಣರ
ಕೆ
ಕೆಂಗಣ್ಣಿಗೆ
ಕೆಂಗಿಚ್ಚು
ಕೆಂಟ-ಕಿ-ಯಲ್ಲಿ
ಕೆಂಟುಕಿಯ
ಕೆಂಡ
ಕೆಂಪ-ಗಾ-ಗುತ್ತದೆ
ಕೆಂಪಗೆ
ಕೆಂಪ-ಡ-ರ-ತೊ-ಡ-ಗಿತ್ತು
ಕೆಂಪಾಗಿ
ಕೆಂಪು
ಕೆಚ್ಚನ್ನು
ಕೆಚ್ಚ-ಲಿ-ನಿಂದ
ಕೆಚ್ಚಿನಿಂದ
ಕೆಚ್ಚು
ಕೆಚ್ಚೆದೆಯ
ಕೆಚ್ಚೆ-ದೆ-ಯಿಂದ
ಕೆಜಿನಕ್ಸಿ
ಕೆಟ್ಟ
ಕೆಟ್ಟಂತೆ
ಕೆಟ್ಟದು
ಕೆಟ್ಟದ್ದನ್ನು
ಕೆಟ್ಟದ್ದಾ-ಗಲು
ಕೆಟ್ಟದ್ದಾ-ದರೆ
ಕೆಟ್ಟದ್ದಿ-ರ-ಬೇಕು
ಕೆಟ್ಟದ್ದು
ಕೆಟ್ಟ-ರೀ-ತಿಯ
ಕೆಟ್ಟ-ರೀ-ತಿ-ಯಲ್ಲಿ
ಕೆಟ್ಟರೋಗ
ಕೆಟ್ಟವ
ಕೆಟ್ಟ-ವ-ನಾ-ದರೆ
ಕೆಟ್ಟವನೇ
ಕೆಟ್ಟ-ಹೆ-ಸ-ರನ್ನು
ಕೆಟ್ಟು
ಕೆಟ್ಟು-ಹೋ-ಗಿದೆ
ಕೆಟ್ಟೆವು
ಕೆಡ-ಕನ್ನುಂಟು-ಮಾ-ಡಿದ
ಕೆಡಕು
ಕೆಡ-ಕು-ಗಳ
ಕೆಡ-ಕು-ಗ-ಳನ್ನುಂಟು
ಕೆಡ-ಕು-ಗ-ಳನ್ನೂ
ಕೆಡವಿತು
ಕೆಡಹಿ
ಕೆಡ-ಹಿ-ಬಿ-ಡೋ-ಣ-ವೆಂದೂ
ಕೆಡಿ-ಸ-ಬಲ್ಲ-ದೆನ್ನು-ವುದು
ಕೆಡಿ-ಸ-ಬಲ್ಲ-ರೆಂಬು-ದನ್ನು
ಕೆಡಿಸಿ
ಕೆಡಿ-ಸಿ-ಕೊಂಡು
ಕೆಡಿ-ಸಿ-ಕೊಳ್ಳ-ಬೇ-ಕಾ-ಗಿಲ್ಲ
ಕೆಡಿ-ಸಿ-ಕೊಳ್ಳಲಿ
ಕೆಡಿಸುವ
ಕೆಡಿ-ಸು-ವು-ದುಂಟು
ಕೆಡುಕನ್ನು
ಕೆಡು-ಕನ್ನುಂಟು-ಮಾಡಿ
ಕೆಡು-ಕನ್ನುಂಟು-ಮಾ-ಡುವ
ಕೆಡುಕನ್ನೂ
ಕೆಡುಕಲ್ಲ
ಕೆಡು-ಕಾ-ಗದ
ಕೆಡು-ಕಾ-ಗುತ್ತದೆ
ಕೆಡು-ಕಾ-ಗು-ವ-ದಾ-ರಿ-ಯನ್ನು
ಕೆಡು-ಕಾ-ದುದು
ಕೆಡುಕಿಗೆ
ಕೆಡುಕಿನ
ಕೆಡುಕಿಲ್ಲ
ಕೆಡುಕು
ಕೆಡು-ಕು-ಗಳ
ಕೆಡು-ಕು-ಗ-ಳನ್ನು
ಕೆಡು-ಕು-ಗ-ಳನ್ನೋ
ಕೆಡುಕೂ
ಕೆಡು-ಕೆಂಬು-ದನ್ನು
ಕೆಡುಕೇ
ಕೆಡುಕೋ
ಕೆಡುತ್ತ
ಕೆಡುತ್ತದೆ
ಕೆಡುವ
ಕೆಣಕುವ
ಕೆತ್ತನೆಯ
ಕೆತ್ತಲ್ಪಟ್ಟ
ಕೆಥೊಲಿಕ್
ಕೆಥೋಲಿಕ್
ಕೆದಕಿ
ಕೆನಡಾದ
ಕೆನ್ನೆತ್
ಕೆನ್ನೆಯ
ಕೆನ್ಸಿಂಗ್ಟನ್
ಕೆಮ್ಮು
ಕೆಯ
ಕೆರಲ್
ಕೆರಳಿ
ಕೆರಳಿದ
ಕೆರಳಿಸಿ
ಕೆರ-ಳಿ-ಸುತ್ತ
ಕೆರ-ಳಿ-ಸುವ
ಕೆರ-ಳು-ವಂತೆ
ಕೆರೆಯ
ಕೆರೆಯಲ್ಲಿ
ಕೆರೆಯುವ
ಕೆರೆಲರ
ಕೆರೆ-ಲ-ರೆಂದಂತೆ
ಕೆರೆಲ್
ಕೆರೆಲ್ರ
ಕೆಲ
ಕೆಲಕಾಲ
ಕೆಲಕ್ಷ-ಣ-ಗ-ಳಲ್ಲೇ
ಕೆಲ-ದಿ-ನ-ಗಳ
ಕೆಲ-ವಂಶ-ಗ-ಳನ್ನು
ಕೆಲ-ವನ್ನಾ-ದರೂ
ಕೆಲವನ್ನು
ಕೆಲವರ
ಕೆಲ-ವ-ರದು
ಕೆಲ-ವ-ರನ್ನು
ಕೆಲ-ವ-ರಲ್ಲಿ
ಕೆಲ-ವ-ರಾ-ದರೂ
ಕೆಲ-ವ-ರಿ-ಗಾ-ದರೂ
ಕೆಲ-ವ-ರಿಗೆ
ಕೆಲವರು
ಕೆಲ-ವ-ರೇನೋ
ಕೆಲವು
ಕೆಲವೆ
ಕೆಲವೆಡೆ
ಕೆಲವೇ
ಕೆಲವೊಂದು
ಕೆಲವೊಮ್ಮೆ
ಕೆಲ-ವೊಮ್ಮೆ-ಯಾ-ದರೂ
ಕೆಲಸ
ಕೆಲ-ಸ-ವಾ-ಗು-ವುದೇ
ಕೆಲ-ಸ-ಇವೇ
ಕೆಲ-ಸ-ಕಾರ್ಯ
ಕೆಲ-ಸ-ಕಾರ್ಯ-ಗಳ
ಕೆಲ-ಸ-ಕಾರ್ಯ-ಗ-ಳನ್ನೆಲ್ಲ
ಕೆಲ-ಸಕ್ಕಾಗಿ
ಕೆಲಸಕ್ಕೆ
ಕೆಲ-ಸಕ್ಕೆ-ಒಂದು
ಕೆಲ-ಸ-ಗಳ
ಕೆಲ-ಸ-ಗ-ಳನ್ನ-ವರು
ಕೆಲ-ಸ-ಗ-ಳನ್ನು
ಕೆಲ-ಸ-ಗ-ಳನ್ನೂ
ಕೆಲ-ಸ-ಗ-ಳನ್ನೆಲ್ಲಾ
ಕೆಲ-ಸ-ಗ-ಳನ್ನೇ
ಕೆಲ-ಸ-ಗ-ಳಲ್ಲಿ
ಕೆಲ-ಸ-ಗ-ಳಲ್ಲಿ-ರುವ
ಕೆಲ-ಸ-ಗ-ಳಲ್ಲೂ
ಕೆಲ-ಸ-ಗ-ಳಿಂದ
ಕೆಲ-ಸ-ಗ-ಳಿಗೆ
ಕೆಲ-ಸ-ಗ-ಳಿವೆ
ಕೆಲ-ಸ-ಗಳು
ಕೆಲ-ಸ-ಗಳೂ
ಕೆಲ-ಸ-ಗ-ಳೆಂದರೆ
ಕೆಲ-ಸ-ಗಳ್ಳ-ರಾ-ಗಲು
ಕೆಲ-ಸ-ಗಳ್ಳರು
ಕೆಲ-ಸ-ಗಳ್ಳರೂ
ಕೆಲ-ಸ-ಗಾರ
ಕೆಲ-ಸ-ಗಾ-ರ-ನಾದ
ಕೆಲ-ಸ-ಗಾ-ರ-ನೆ-ನಿ-ಸಿ-ಕೊಳ್ಳುತ್ತಾನೆ
ಕೆಲ-ಸ-ಗಾ-ರ-ನೊಬ್ಬ
ಕೆಲ-ಸ-ಗಾ-ರರ
ಕೆಲ-ಸ-ಗಾ-ರ-ರನ್ನು
ಕೆಲ-ಸ-ಗಾ-ರರು
ಕೆಲ-ಸ-ಗಾ-ರ-ರೆಂದೂ
ಕೆಲಸದ
ಕೆಲ-ಸ-ದಲ್ಲಿ
ಕೆಲ-ಸ-ದಲ್ಲೂ
ಕೆಲ-ಸ-ದಲ್ಲೆಲ್ಲ
ಕೆಲ-ಸ-ದಲ್ಲೇ
ಕೆಲ-ಸ-ದಿಂದ
ಕೆಲ-ಸ-ದೊಂದಿಗೆ
ಕೆಲ-ಸ-ಮಾಡಿ
ಕೆಲ-ಸ-ಮಾ-ಡುತ್ತಿದ್ದ
ಕೆಲ-ಸ-ಮಾ-ಡುತ್ತಿ-ರುವ
ಕೆಲ-ಸ-ಮಾ-ಡುವ
ಕೆಲ-ಸ-ಮಾ-ಡು-ವುದು
ಕೆಲ-ಸ-ವನ್ನಾ-ದರೂ
ಕೆಲ-ಸ-ವನ್ನು
ಕೆಲ-ಸ-ವನ್ನೂ
ಕೆಲ-ಸ-ವನ್ನೇ
ಕೆಲ-ಸ-ವನ್ನೋ
ಕೆಲ-ಸ-ವಲ್ಲ
ಕೆಲ-ಸ-ವಾ-ಗ-ಬಾ-ರದು
ಕೆಲ-ಸ-ವಾ-ಗ-ಬೇಕು
ಕೆಲ-ಸ-ವಾ-ಗ-ಲಿಲ್ಲ
ಕೆಲ-ಸ-ವಾ-ಗಿತ್ತು
ಕೆಲ-ಸ-ವಾ-ಗಿದ್ದರೂ
ಕೆಲ-ಸ-ವಾ-ಗಿ-ರ-ಲಿಲ್ಲ
ಕೆಲ-ಸ-ವಾ-ದರೂ
ಕೆಲ-ಸ-ವಾ-ದರೋ
ಕೆಲ-ಸ-ವಿತ್ತು
ಕೆಲ-ಸ-ವಿಲ್ಲ
ಕೆಲಸವು
ಕೆಲಸವೂ
ಕೆಲಸವೆ
ಕೆಲ-ಸ-ವೆಂಬು-ದೊಂದು
ಕೆಲ-ಸ-ವೆಲ್ಲ
ಕೆಲಸವೇ
ಕೆಲ-ಸ-ವೊಂದನ್ನು
ಕೆಲ್ವಿನ್
ಕೆಲ್ಸಿ
ಕೆಳಕ್ಕಿ-ಳಿ-ಸಿ-ದರು
ಕೆಳಕ್ಕು-ರು-ಳಿದ
ಕೆಳಕ್ಕೆ
ಕೆಳಕ್ಕೆ-ಳೆ-ಯು-ವುದು
ಕೆಳಗಡೆ
ಕೆಳ-ಗಿದ್ದೇ-ವೆಂದು
ಕೆಳಗಿನ
ಕೆಳ-ಗಿ-ನ-ವನೆ
ಕೆಳ-ಗಿ-ನ-ವ-ರನ್ನು
ಕೆಳ-ಗಿ-ನ-ವರು
ಕೆಳ-ಗಿ-ರು-ವ-ವರ
ಕೆಳ-ಗಿ-ರು-ವ-ವ-ರನ್ನು
ಕೆಳ-ಗಿ-ಳಿದ
ಕೆಳ-ಗಿ-ಳಿದು
ಕೆಳಗೆ
ಕೆಳಗೇ
ಕೆಳ-ಭಾ-ಗ-ದಲ್ಲಿ
ಕೆಳ-ಭಾ-ಗ-ವನ್ನೆಲ್ಲಾ
ಕೆಳ-ಮಟ್ಟದ
ಕೆಳ-ಮಟ್ಟ-ದ-ವ-ರಾ-ಗ-ಬ-ಹುದು
ಕೆಳ-ಮು-ಖ-ವಾ-ಗಿದ್ದ
ಕೆಳ-ವರ್ಗ-ದ-ವ-ರಿಗೆ
ಕೆಸ-ರಿ-ನಲ್ಲಿ
ಕೇಂದ್ರ
ಕೇಂದ್ರಗಳ
ಕೇಂದ್ರ-ಗ-ಳಾಗಿ
ಕೇಂದ್ರ-ಗ-ಳಿಂದ
ಕೇಂದ್ರಗಳು
ಕೇಂದ್ರಗಳೂ
ಕೇಂದ್ರದ
ಕೇಂದ್ರ-ಬಿಂದು-ವಿಲ್ಲವೆ
ಕೇಂದ್ರ-ಬಿಂದುವೇ
ಕೇಂದ್ರ-ವಾ-ಗಿದೆ
ಕೇಂದ್ರ-ವಾ-ಗುತ್ತಾನೆ
ಕೇಂದ್ರವಾದ
ಕೇಕೆ
ಕೇಕೆಯನ್ನು
ಕೇಡನ್ನು
ಕೇಡಿನಿಂದ
ಕೇಡು
ಕೇಡುಗಾಲ
ಕೇಬಲ್
ಕೇರ್
ಕೇಳ
ಕೇಳದ
ಕೇಳ-ದ-ವರು
ಕೇಳದೆ
ಕೇಳದೇ
ಕೇಳ-ಬ-ಯ-ಸು-ವು-ದಿಲ್ಲ
ಕೇಳಬಲ್ಲ
ಕೇಳ-ಬ-ಹುದು
ಕೇಳ-ಬೇ-ಕಾ-ಗುತ್ತದೆ
ಕೇಳ-ಬೇ-ಕಾ-ಗು-ವುದು
ಕೇಳ-ಬೇ-ಕಾ-ಯಿತು
ಕೇಳಬೇಕು
ಕೇಳ-ಬೇ-ಕೆಂದಿದ್ದೇನೆ
ಕೇಳ-ಬೇ-ಕೆಂದೂ
ಕೇಳಬೇಡ
ಕೇಳಲಾಗಿ
ಕೇಳ-ಲಾ-ಗಿದೆ
ಕೇಳ-ಲಾ-ಗುತ್ತಿದೆ
ಕೇಳ-ಲಾ-ಯಿತು
ಕೇಳಲಿ
ಕೇಳಲಿಕ್ಕೆ
ಕೇಳಲಿಲ್ಲ
ಕೇಳಲು
ಕೇಳಲೇ
ಕೇಳಿ
ಕೇಳಿದ್ದೀರಾ
ಕೇಳಿ-ಕೆ-ಯಾ-ಗಲಿ
ಕೇಳಿಕೊಂಡ
ಕೇಳಿ-ಕೊಂಡರು
ಕೇಳಿ-ಕೊಂಡರೆ
ಕೇಳಿ-ಕೊಂಡಳು
ಕೇಳಿ-ಕೊಂಡಾಗ
ಕೇಳಿ-ಕೊಂಡಿತು
ಕೇಳಿ-ಕೊಂಡಿದ್ದಾರೆ
ಕೇಳಿ-ಕೊಳ್ಳ-ಬೇಕು
ಕೇಳಿ-ಕೊಳ್ಳುತ್ತಿದ್ದ
ಕೇಳಿ-ಕೊಳ್ಳು-ವುದೂ
ಕೇಳಿತು
ಕೇಳಿದ
ಕೇಳಿ-ದಂತಾಗಿ
ಕೇಳಿದಂತೆ
ಕೇಳಿದಕ್ಕೆ
ಕೇಳಿ-ದ-ಭ-ಗ-ವಾನ್
ಕೇಳಿ-ದ-ರಂತೆ
ಕೇಳಿದರು
ಕೇಳಿದರೂ
ಕೇಳಿದರೆ
ಕೇಳಿ-ದ-ಳಂತೆ
ಕೇಳಿದಳು
ಕೇಳಿ-ದ-ವ-ರಲ್ಲ
ಕೇಳಿ-ದ-ವ-ಳಾ-ಗಿ-ರ-ಲಿಲ್ಲ
ಕೇಳಿದಾ
ಕೇಳಿದಾಗ
ಕೇಳಿದೆ
ಕೇಳಿದೆಈ
ಕೇಳಿ-ದೊ-ಡ-ನೆಯೇ
ಕೇಳಿದ್ದ
ಕೇಳಿದ್ದೀರಾ
ಕೇಳಿದ್ದೀರಿ
ಕೇಳಿದ್ದು
ಕೇಳಿದ್ದೆ
ಕೇಳಿದ್ದೇನೆ
ಕೇಳಿದ್ದೇವೆ
ಕೇಳಿಬಂತು
ಕೇಳಿಯೇ
ಕೇಳಿಯೋ
ಕೇಳಿರದ
ಕೇಳಿ-ರ-ಬ-ಹುದು
ಕೇಳಿ-ರ-ಲಿಲ್ಲ
ಕೇಳಿರಿ
ಕೇಳಿ-ರುತ್ತೀರಿ
ಕೇಳಿ-ರು-ವು-ದಾಗಿ
ಕೇಳಿರುವೆ
ಕೇಳಿಲ್ಲ
ಕೇಳಿಲ್ಲಿ
ಕೇಳಿ-ಸ-ದ-ವ-ರಂತೆ
ಕೇಳಿ-ಸ-ದಿ-ರು-ವುದೆ
ಕೇಳಿಸದೆ
ಕೇಳಿ-ಸ-ಬೇಕು
ಕೇಳಿ-ಸಿ-ಕೊಂಡರೂ
ಕೇಳಿ-ಸಿ-ಕೊಂಡಿದ್ದಳು
ಕೇಳಿಸಿತು
ಕೇಳಿಸಿಯೇ
ಕೇಳಿ-ಸುತ್ತದೆ
ಕೇಳಿ-ಸುತ್ತಿದೆ
ಕೇಳಿ-ಸುತ್ತಿದ್ದುವು
ಕೇಳಿ-ಸು-ವಂತೆ
ಕೇಳು
ಕೇಳು-ಕೆ-ಲವು
ಕೇಳು-ಗ-ರಿಗೆ
ಕೇಳುತ್ತ
ಕೇಳುತ್ತದೆ
ಕೇಳುತ್ತಲೇ
ಕೇಳುತ್ತಾನೆ
ಕೇಳುತ್ತಿದ್ದ
ಕೇಳುತ್ತಿ-ರು-ವಾಗ
ಕೇಳುವ
ಕೇಳು-ವಂತಾ-ಗಲಿ
ಕೇಳುವನ್
ಕೇಳು-ವು-ದಕ್ಕಾಗಿ
ಕೇಳು-ವು-ದಕ್ಕೂ
ಕೇಳು-ವು-ದಿಲ್ಲ-ವೆಂದು
ಕೇಳುವುದು
ಕೇವಲ
ಕೇವಲೋ
ಕೇಶ-ವ-ಚಂದ್ರ
ಕೇಸಿಯ
ಕೇಸೀ
ಕೇಸೀಗೆ
ಕೇಸೀಯ
ಕೇಸೀಯನ್ನು
ಕೇಸೀಯಲ್ಲಿ
ಕೇಸೀಯಿಂದ
ಕೇಸೀಯು
ಕೇಸೀಯೇ
ಕೇಸೀ-ಯೊ-ಡನೆ
ಕೇಸು-ಗ-ಳನ್ನು
ಕೈ
ಕೈಂಕರ್ಯ
ಕೈಕಟ್ಟಿ
ಕೈಕಟ್ಟಿ-ಕೊಂಡು
ಕೈಕ-ವ-ಚ-ವನ್ನು
ಕೈಕಾಲು
ಕೈಕಾ-ಲು-ಗಳ
ಕೈಕೊಂಡರೆ
ಕೈಗನ್ನಡಿ
ಕೈಗಳ
ಕೈಗಳನ್ನು
ಕೈಗಳಿಂದ
ಕೈಗಳೇ
ಕೈಗಾ-ರಿ-ಕೆ-ಗ-ಳನ್ನು
ಕೈಗಾ-ರಿ-ಕೆಯ
ಕೈಗೂ-ಡ-ದಿದ್ದಾಗ
ಕೈಗೂಡದು
ಕೈಗೂ-ಡುತ್ತ-ದೆಂಬುದು
ಕೈಗೆ
ಕೈಗೆ-ಟು-ಕದ
ಕೈಗೆ-ಲ-ಸ-ವನ್ನು
ಕೈಗೊಂಡ
ಕೈಗೊಂಡದ್ದಿಲ್ಲ
ಕೈಗೊಂಡರು
ಕೈಗೊಂಡರೂ
ಕೈಗೊಂಡ-ವ-ರಲ್ಲ
ಕೈಗೊಂಡಿದ್ದಾರೆ
ಕೈಗೊಂಡು
ಕೈಗೊಂಬೆ
ಕೈಗೊಳ್ಳ-ದಿ-ರ-ಲಾರ
ಕೈಗೊಳ್ಳ-ಬಾ-ರದು
ಕೈಗೊಳ್ಳ-ಬೇ-ಕಾ-ಗಿದೆ
ಕೈಗೊಳ್ಳ-ಬೇ-ಕೆಂದು
ಕೈಗೊಳ್ಳ-ಲಿಲ್ಲ
ಕೈಗೊಳ್ಳಲು
ಕೈಗೊಳ್ಳಿ-ರಿ-ಅದು
ಕೈಗೊಳ್ಳುತ್ತಿದ್ದರು
ಕೈಗೊಳ್ಳುವ
ಕೈಗೊಳ್ಳು-ವು-ದಕ್ಕಿಂತ
ಕೈಗೊಳ್ಳು-ವು-ದಕ್ಕೆ
ಕೈಗೊಳ್ಳೋಣ
ಕೈಚ-ಳ-ಕ-ದಿಂದ
ಕೈದಿ-ಗ-ಳಿ-ಗಾಗಿ
ಕೈದಿಗಳು
ಕೈದಿ-ಯಾ-ಗಿದ್ದ
ಕೈದೀ-ವಿ-ಗೆ-ಯಿಂದು
ಕೈಬಿಟ್ಟ
ಕೈಬಿಟ್ಟರು
ಕೈಬಿಟ್ಟಿತು
ಕೈಬಿಡನು
ಕೈಬಿ-ಡ-ಬೇ-ಕಾ-ಯಿತು
ಕೈಬಿ-ಡ-ಬೇ-ಕಿಲ್ಲ
ಕೈಬಿ-ಡು-ವು-ದಿಲ್ಲ-ವೆಂಬ
ಕೈಬೆ-ರ-ಳಣಿ
ಕೈಬೆ-ರ-ಳೆ-ಣಿ-ಕೆ-ಯಲ್ಲಿ
ಕೈಮುಚ್ಚಿ-ಕೊಂಡು
ಕೈಯ
ಕೈಯಲ್ಲಿ
ಕೈಯಲ್ಲಿದೆ
ಕೈಯಲ್ಲಿದ್ದ
ಕೈಯಲ್ಲಿ-ರುವ
ಕೈಯಲ್ಲಿವೆ
ಕೈಯಲ್ಲೇ
ಕೈಯಿಂದ
ಕೈಯಿಂದಲೇ
ಕೈಯಿಟ್ಟು
ಕೈಯೆಡೆಗೆ
ಕೈಯೊಡ್ಡಿ
ಕೈಲಾದ
ಕೈವಾಡದ
ಕೈಸೇರಿದೆ
ಕೈಹಾ-ಕುತ್ತಿ-ರ-ಲಿಲ್ಲ
ಕೈಹಿಡಿದು
ಕೊ
ಕೊಂಕು
ಕೊಂಕುನುಡಿ
ಕೊಂಕು-ಗ-ಳು-ಇವು
ಕೊಂಚ
ಕೊಂಚಕಾಲ
ಕೊಂಚ-ವಾ-ದರೂ
ಕೊಂಚವೂ
ಕೊಂಡ
ಕೊಂಡಂತಾ-ಗು-ವುದು
ಕೊಂಡರು
ಕೊಂಡರೂ
ಕೊಂಡರೆ
ಕೊಂಡಲ್ಲಿ
ಕೊಂಡವರ
ಕೊಂಡವರು
ಕೊಂಡ-ವ-ರು-ಇ-ವ-ರು-ಗಳ
ಕೊಂಡವರೇ
ಕೊಂಡವು
ಕೊಂಡ-ವು-ನೀ-ವೇಕೆ
ಕೊಂಡಾಗ
ಕೊಂಡಾಡ
ಕೊಂಡಾ-ಡಿದ್ದಾರೆ
ಕೊಂಡಾ-ಡುತ್ತವೆ
ಕೊಂಡಾ-ಡುತ್ತಾನೆ
ಕೊಂಡಾ-ಡುತ್ತಾರೆ
ಕೊಂಡಿದೆ
ಕೊಂಡಿದ್ದ
ಕೊಂಡಿದ್ದರು
ಕೊಂಡಿದ್ದಾನೆ
ಕೊಂಡಿದ್ದಾರೆ
ಕೊಂಡಿದ್ದೇವೆ
ಕೊಂಡಿಯೇ
ಕೊಂಡಿ-ರುತ್ತವೆ
ಕೊಂಡಿರುವ
ಕೊಂಡಿ-ರು-ವುದು
ಕೊಂಡಿಲ್ಲ
ಕೊಂಡು
ಕೊಂಡು-ಕೊಳ್ಳುತ್ತಾರೆ
ಕೊಂಡುಕೊಂಡ
ಕೊಂಡುಕೊಂಡು
ಕೊಂಡುಕೊಂಡೆ
ಕೊಂಡು-ಕೊಳ್ಳ-ಬೇಡಿ
ಕೊಂಡು-ಕೊಳ್ಳಲು
ಕೊಂಡು-ಕೊಳ್ಳು-ವಾಗ
ಕೊಂಡುಬಂದ
ಕೊಂಡುಹೋಗಿ
ಕೊಂಡೇ
ಕೊಂಡೊಯ್ದ
ಕೊಂಡೊಯ್ದವು
ಕೊಂಡೊಯ್ದಾಗ
ಕೊಂಡೊಯ್ದು
ಕೊಂಡೊಯ್ಯ
ಕೊಂಡೊಯ್ಯ-ಬೇ-ಕೆಂದೂ
ಕೊಂಡೊಯ್ಯ-ಲಾ-ಗಿದೆ
ಕೊಂಡೊಯ್ಯ-ಲಾ-ರವು
ಕೊಂಡೊಯ್ಯಲು
ಕೊಂಡೊಯ್ಯುತ್ತ
ಕೊಂಡೊಯ್ಯುತ್ತವೆ
ಕೊಂಡೊಯ್ಯುತ್ತಾರೆ
ಕೊಂಡೊಯ್ಯುತ್ತಾ-ರೆಂಬು-ದನ್ನು
ಕೊಂಡೊಯ್ಯುತ್ತಿದೆ
ಕೊಂಡೊಯ್ಯುವ
ಕೊಂಡೊಯ್ಯು-ವುದೇ
ಕೊಂದರೂ
ಕೊಂದು
ಕೊಂದೆವು
ಕೊಂದೇ
ಕೊಂಪೆಯಲ್ಲಿ
ಕೊಂಬೆಗಳು
ಕೊಂಬೆಯನ್ನು
ಕೊಚ್ಚಿ
ಕೊಚ್ಚಿಕೊಂಡು
ಕೊಚ್ಚಿ-ಹೋ-ಗಿದೆ
ಕೊಟ್ಟ
ಕೊಟ್ಟಂತೆ
ಕೊಟ್ಟದ್ದನ್ನು
ಕೊಟ್ಟರು
ಕೊಟ್ಟರೂ
ಕೊಟ್ಟರೆ
ಕೊಟ್ಟಲ್ಲಿ
ಕೊಟ್ಟ-ವ-ರ-ವೆಂದು
ಕೊಟ್ಟಾಗ
ಕೊಟ್ಟಿದೆ
ಕೊಟ್ಟಿದ್ದ
ಕೊಟ್ಟಿದ್ದರೆ
ಕೊಟ್ಟಿದ್ದಳೋ
ಕೊಟ್ಟಿದ್ದಾನೆ
ಕೊಟ್ಟಿದ್ದಾರೆ
ಕೊಟ್ಟಿದ್ದೆ
ಕೊಟ್ಟಿ-ರ-ಲಿಲ್ಲ
ಕೊಟ್ಟಿರುವ
ಕೊಟ್ಟಿರುವು
ಕೊಟ್ಟಿ-ರು-ವು-ದರ
ಕೊಟ್ಟಿ-ರು-ವುದು
ಕೊಟ್ಟು
ಕೊಟ್ಟೆ
ಕೊಠ-ಡಿ-ಯಲ್ಲಿ
ಕೊಡ
ಕೊಡಗು
ಕೊಡದ
ಕೊಡ-ದಿದ್ದರೆ
ಕೊಡದು
ಕೊಡದೆ
ಕೊಡದೇ
ಕೊಡ-ಬಲ್ಲುದೆ
ಕೊಡಬಲ್ಲೆ
ಕೊಡ-ಬ-ಹುದು
ಕೊಡ-ಬಾ-ರದು
ಕೊಡ-ಬೇ-ಕಾದ
ಕೊಡಬೇಕು
ಕೊಡ-ಬೇ-ಕೆಂದು
ಕೊಡ-ಬೇ-ಕೆಂಬುದು
ಕೊಡ-ಮಾ-ಡಿದ
ಕೊಡ-ಲಾ-ಗಿದೆ
ಕೊಡ-ಲಾ-ಗುತ್ತದೆ
ಕೊಡಲಿ
ಕೊಡಲಿಲ್ಲ
ಕೊಡಲು
ಕೊಡಲೂ
ಕೊಡಲ್ಪಟ್ಟ
ಕೊಡಲ್ಪ-ಡುತ್ತದೆ
ಕೊಡ-ಹೊ-ರಡ
ಕೊಡಿ
ಕೊಡಿಸಿದ
ಕೊಡಿ-ಸು-ವಿರಾ
ಕೊಡು
ಕೊಡುತ್ತಿದ್ದರು
ಕೊಡುಗೆ
ಕೊಡು-ಗೆ-ಗ-ಳೊಂದಿಗೆ
ಕೊಡು-ಗೆ-ಯನ್ನು
ಕೊಡು-ಗೆ-ಯಾ-ಗ-ಬಲ್ಲದು
ಕೊಡು-ಗೆ-ಯಾಗಿ
ಕೊಡುತ್ತ
ಕೊಡುತ್ತದೆ
ಕೊಡುತ್ತಾನೆ
ಕೊಡುತ್ತಾರೆ
ಕೊಡುತ್ತಿದ್ದ
ಕೊಡುತ್ತಿದ್ದಂತೆ
ಕೊಡುತ್ತಿದ್ದಳು
ಕೊಡುತ್ತಿ-ರುತ್ತೇವೆ
ಕೊಡುತ್ತಿ-ರುವ
ಕೊಡುತ್ತೇನೆ
ಕೊಡುವ
ಕೊಡು-ವ-ರೆಂದು
ಕೊಡುವರೇ
ಕೊಡು-ವ-ವನೋ
ಕೊಡು-ವು-ದನ್ನು
ಕೊಡು-ವು-ದರ
ಕೊಡು-ವು-ದ-ರಿಂದ
ಕೊಡು-ವು-ದಿಲ್ಲ
ಕೊಡುವುದು
ಕೊಡು-ವು-ದೆಂದರೇ
ಕೊಡೆಯನ್ನು
ಕೊನ-ರು-ವುದು
ಕೊನೆ
ಕೊನೆ-ಗಾ-ಲ-ದಲ್ಲಿ
ಕೊನೆ-ಗಾ-ಲ-ವನ್ನು
ಕೊನೆಗೂ
ಕೊನೆಗೆ
ಕೊನೆಗೇನೂ
ಕೊನೆಗೊಂಡು
ಕೊನೆ-ಗೊಳ್ಳುವ
ಕೊನೆ-ಗೊಳ್ಳು-ವಂತೆ
ಕೊನೆ-ಗೊಳ್ಳು-ವುದು
ಕೊನೆಯ
ಕೊನೆ-ಯ-ಗ-ಳಿಗೆ
ಕೊನೆ-ಯ-ದಲ್ಲ
ಕೊನೆ-ಯ-ದಾಗಿ
ಕೊನೆ-ಯ-ದಿನ
ಕೊನೆಯದು
ಕೊನೆಯದೇ
ಕೊನೆ-ಯ-ಪಕ್ಷ
ಕೊನೆಯಲ್ಲಿ
ಕೊನೆ-ಯ-ವ-ರೆಗೆ
ಕೊಬ್ಬಿದ
ಕೊಬ್ಬಿ-ನಿಂದಾಗಿ
ಕೊರ
ಕೊರಗಿ
ಕೊರ-ಗಿ-ದರೆ
ಕೊರಡನ್ನು
ಕೊರ-ಡು-ಗ-ಳಲ್ಲ
ಕೊರಡೂ
ಕೊರತೆ
ಕೊರ-ತೆ-ಗ-ಳಿಗೆ
ಕೊರ-ತೆ-ಯಿಂದಾಗಿ
ಕೊರ-ತೆ-ಯಿದೆ
ಕೊರ-ತೆ-ಯಿಲ್ಲ
ಕೊರತೆಯೂ
ಕೊರತೆಯೇ
ಕೊರಳನು
ಕೊರಳನ್ನು
ಕೊರಳಿರಿ
ಕೊರಳು
ಕೊರಾನ್
ಕೊರೆ-ತ-ವನ್ನು
ಕೊಲ
ಕೊಲಂಬ-ಸನ
ಕೊಲೆ
ಕೊಲೆ-ಗ-ಡು-ಕ-ತನ
ಕೊಲೆ-ಗ-ಡು-ಕ-ನಂತೆ
ಕೊಲೆ-ಗ-ಡು-ಕ-ನಾಗಿ
ಕೊಲೆ-ಗ-ಡು-ಕನೂ
ಕೊಲೆ-ಗ-ಡು-ಕರ
ಕೊಲೆ-ಗ-ಡು-ಕ-ರನ್ನಾಗಿ
ಕೊಲೆ-ಗ-ಡು-ಕರು
ಕೊಲೆಗಳ
ಕೊಲೆ-ಗ-ಳನ್ನು
ಕೊಲೆ-ಗೈ-ಯುತ್ತಿ-ರುವ
ಕೊಲೆ-ಮಾ-ಡು-ವಂತಹ
ಕೊಲೆಯನ್ನು
ಕೊಲ್ಲ
ಕೊಲ್ಲ-ಕೂ-ಡದು
ಕೊಲ್ಲ-ದಿ-ರು-ವುದು
ಕೊಲ್ಲಲು
ಕೊಲ್ಲುತ್ತಾನೆ
ಕೊಲ್ಲುವ
ಕೊಲ್ಲುವಂತೆ
ಕೊಳ
ಕೊಳ-ಕಿ-ನಲ್ಲಿ
ಕೊಳಕು
ಕೊಳಚೆ
ಕೊಳ-ಚೆಪ್ರ-ದೇಶ
ಕೊಳಚೆಯ
ಕೊಳದ
ಕೊಳದಲ್ಲಿ
ಕೊಳ-ವೊಂದಕ್ಕೆ
ಕೊಳೆ
ಕೊಳೆಎಂದು
ಕೊಳೆ-ಗ-ಳನ್ನೆಲ್ಲ
ಕೊಳೆತ
ಕೊಳೆಯನ್ನು
ಕೊಳೆಯಲು
ಕೊಳೆಯಿಂದ
ಕೊಳ್ಳ-ಬೇ-ಕಾ-ಯಿತು
ಕೊಳ್ಳಬೇಕು
ಕೊಳ್ಳ-ಲಾ-ಗುತ್ತದೆ
ಕೊಳ್ಳಲು
ಕೊಳ್ಳಿ
ಕೊಳ್ಳುತ್ತ
ಕೊಳ್ಳುತ್ತದೆ
ಕೊಳ್ಳುತ್ತವೆ
ಕೊಳ್ಳುತ್ತಾನೆ
ಕೊಳ್ಳುತ್ತಾರೆ
ಕೊಳ್ಳುತ್ತಿದ್ದೆ
ಕೊಳ್ಳುವ
ಕೊಳ್ಳುವಾಗ
ಕೊಳ್ಳು-ವು-ದಕ್ಕೆ
ಕೊಳ್ಳೆ
ಕೊಳ್ಳೆಯ
ಕೋಟ
ಕೋಟ-ಲೆ-ಯಲ್ಲಿ
ಕೋಟಿ
ಕೋಟಿ-ಗಟ್ಟಲೆ
ಕೋಟಿಗೂ
ಕೋಟಿನ
ಕೋಟೆ
ಕೋಟ್ಯಂತರ
ಕೋಣೆ
ಕೋಣೆ-ಗ-ಳನ್ನೆಲ್ಲ
ಕೋಣೆಗೆ
ಕೋಣೆಯ
ಕೋಣೆಯನ್ನು
ಕೋಣೆಯಲ್ಲಿ
ಕೋಣೆ-ಯಲ್ಲಿದ್ದ
ಕೋಣೆ-ಯಲ್ಲಿ-ರಲಿ
ಕೋಣೆ-ಯಲ್ಲಿ-ರುವ
ಕೋಣೆ-ಯಲ್ಲಿ-ರು-ವ-ವ-ರಿಗೆ
ಕೋಣೆಯಲ್ಲೂ
ಕೋಣೆಯಲ್ಲೇ
ಕೋಣೆಯಿಂದ
ಕೋಣೆಯು
ಕೋಣೆ-ಯೊಂದ-ರಲ್ಲಿ
ಕೋಣೆ-ಯೊ-ಳಗೆ
ಕೋತಿಗಳ
ಕೋತಿ-ಗ-ಳಂತೆ
ಕೋತಿ-ಗ-ಳನ್ನು
ಕೋತಿಗಳು
ಕೋತಿಯೊಂದು
ಕೋನ
ಕೋಪ
ಕೋಪ-ಗ-ಳಿಂದ
ಕೋಪ-ಗೊಂಡರೆ
ಕೋಪತಾಪ
ಕೋಪ-ತಾ-ಪ-ಗ-ಳನ್ನು
ಕೋಪದ
ಕೋಪದಿಂದ
ಕೋಪವನ್ನು
ಕೋಪ-ವನ್ನು-ಳಿ-ಸಿ-ಕೊಳ್ಳುತ್ತಾರೆ
ಕೋಪಿಷ್ಠ
ಕೋಪಿಷ್ಠ-ರಿಂದ
ಕೋಪಿಷ್ಠರೂ
ಕೋಪಿ-ಸಿ-ಕೊಳ್ಳ-ಲೂ-ಬ-ಹುದು
ಕೋಪಿ-ಸಿ-ಕೊಳ್ಳುತ್ತಿದ್ದಾರೆ
ಕೋಪೋದ್ರಿಕ್ತ-ರಾಗಿ
ಕೋಮಲ
ಕೋಮ-ಲ-ವಾದ
ಕೋರುತ್ತ
ಕೋರೈಸುವ
ಕೋರೈ-ಸು-ವಂತಿದ್ದರೂ
ಕೋಲಾ
ಕೋಲಾಹಕ್ಕೆ
ಕೋಲಾಹಲ
ಕೋಲ್ರಿಜ್
ಕೋಲ್ರಿಡ್ಜ್
ಕೋವಿಗಳ
ಕೋಶವನ್ನು
ಕೋಶಾ-ಧಿ-ಕಾ-ರಿ-ಯಾ-ಗಿದ್ದ-ನೆಂದೂ
ಕೌಟಿಲ್ಯ
ಕೌಟುಂಬಿಕ
ಕ್ಕಾಗ-ದಿದ್ದರೆ
ಕ್ಕಾಗಿ
ಕ್ಕೆಂದರೆ
ಕ್ಕೆಳ-ಸು-ವುವು
ಕ್ಕೆಸೆದು
ಕ್ಕೊಳ-ಗಾ-ದ-ವ-ರನ್ನು
ಕ್ಟ್ರಾನ್
ಕ್ಯಾಂಪಿನಲ್ಲಿ
ಕ್ಯಾಟರಾಕ್ಟ್
ಕ್ಯಾಥೋಲಿಕ್
ಕ್ಯಾನ್ಸರ್
ಕ್ಯಾಮರಾ
ಕ್ಯಾರಿ-ಯ-ರಿ-ನಲ್ಲಿ
ಕ್ಯೂನಲ್ಲಿ
ಕ್ಯೂಬಾನ್
ಕ್ಯೂರಿ
ಕ್ಯೇನನ್
ಕ್ಯೋಂ
ಕ್ರಂದನ
ಕ್ರಂದ-ನ-ವನ್ನು
ಕ್ರಮ
ಕ್ರಮ-ಇ-ವು-ಗ-ಳನ್ನು
ಕ್ರಮಕ್ಕಿಂತ
ಕ್ರಮಕ್ರ-ಮ-ವಾಗಿ
ಕ್ರಮಗಳ
ಕ್ರಮ-ಗ-ಳನ್ನು
ಕ್ರಮಗಳು
ಕ್ರಮ-ಗ-ಳೆಲ್ಲ
ಕ್ರಮ-ಗ-ಳೇನು
ಕ್ರಮದ
ಕ್ರಮದಂತೆ
ಕ್ರಮದಲ್ಲಿ
ಕ್ರಮದಿಂದ
ಕ್ರಮ-ನಿ-ಯಮ
ಕ್ರಮಬದ್ಧ
ಕ್ರಮ-ಬದ್ಧತೆ
ಕ್ರಮ-ಬದ್ಧ-ತೆಯ
ಕ್ರಮ-ಬದ್ಧ-ವಾಗಿ
ಕ್ರಮ-ವನ್ನ-ನು-ಸ-ರಿ-ಸುವ
ಕ್ರಮವನ್ನು
ಕ್ರಮ-ವ-ರಿತು
ಕ್ರಮವಾಗಿ
ಕ್ರಮವಾದ
ಕ್ರಮ-ವಾ-ಯಿತು
ಕ್ರಮ-ವೊಂದನ್ನು
ಕ್ರಮ-ಹೀ-ನತೆ
ಕ್ರಮ-ಹೀ-ನ-ತೆಯೂ
ಕ್ರಮಿ-ಸ-ಬಲ್ಲೆ-ವೇನು
ಕ್ರಮೇಣ
ಕ್ರಯ-ದ-ವು-ಗ-ಳಿಂದ
ಕ್ರಸ್ನೋದರ್
ಕ್ರಾಂತಿ
ಕ್ರಾಂತಿ-ಕಾ-ರಕ
ಕ್ರಾಂತಿಕಾರಿ
ಕ್ರಾಂತಿ-ಕಾ-ರಿ-ಗಳ
ಕ್ರಾಂತಿ-ಕಾ-ರಿ-ಗಳು
ಕ್ರಾಂತಿ-ಕಾ-ರಿ-ಗ-ಳೆ-ನಿ-ಸಿ-ಕೊಂಡು
ಕ್ರಾಂತಿಮಾಡಿ
ಕ್ರಾಂತಿಯ
ಕ್ರಾಂತಿಯನ್ನು
ಕ್ರಾಂತಿಯನ್ನೇ
ಕ್ರಾಂತಿ-ಯಾ-ಗದೆ
ಕ್ರಿ
ಕ್ರಿಕೆಟ್
ಕ್ರಿಪೂ
ಕ್ರಿಮಿಕೀಟ
ಕ್ರಿಯಾ
ಕ್ರಿಯಾ-ಕ-ಲಾ-ಪ-ಗಳು
ಕ್ರಿಯಾ-ಶಕ್ತಿ-ಯನ್ನೂ
ಕ್ರಿಯೆ
ಕ್ರಿಯೆಗಳ
ಕ್ರಿಯೆ-ಗ-ಳಂತೆ
ಕ್ರಿಯೆ-ಗ-ಳನ್ನಾ-ಗಲಿ
ಕ್ರಿಯೆ-ಗ-ಳನ್ನು
ಕ್ರಿಯೆ-ಗ-ಳನ್ನೂ
ಕ್ರಿಯೆ-ಗ-ಳಿಗೂ
ಕ್ರಿಯೆ-ಗ-ಳಿ-ಗೆಲ್ಲ
ಕ್ರಿಯೆಗಳು
ಕ್ರಿಯೆಗಳೂ
ಕ್ರಿಯೆಗೂ
ಕ್ರಿಯೆಗೆ
ಕ್ರಿಯೆಪ್ರ-ತಿಕ್ರಿ-ಯೆ-ಗಳು
ಕ್ರಿಯೆಪ್ರ-ತಿಕ್ರಿ-ಯೆಯ
ಕ್ರಿಯೆ-ಯಿಂದಲೇ
ಕ್ರಿಯೆಯೂ
ಕ್ರಿಯೋನ್ಮುಖ
ಕ್ರಿಶ್ಚಿಯನ್
ಕ್ರಿಸ್ಟೋಫರ್
ಕ್ರಿಸ್ತ
ಕ್ರಿಸ್ತನ
ಕ್ರಿಸ್ತನು
ಕ್ರಿಸ್ತವಾಣಿ
ಕ್ರಿಸ್ತಶಕ
ಕ್ರೀಡಾ
ಕ್ರೀಡಾಂಗಣ
ಕ್ರೀಡಾಂಗ-ಣಕ್ಕೆ
ಕ್ರೀಡಾಂಗ-ಣ-ದಲ್ಲಿ
ಕ್ರೀಡಾಂಗ-ಣ-ದಿಂದ
ಕ್ರೀಡಾ-ಪ-ಟು-ವಾ-ಗಿದ್ದ
ಕ್ರೀಡಾ-ಪತ್ರಿಕೆ
ಕ್ರುದ್ಧರಾಗಿ
ಕ್ರೂರ
ಕ್ರೂರ-ತ-ನಕ್ಕೆ-ಳ-ಸದು
ಕ್ರೂರತೆ
ಕ್ರೂರ-ದೃಷ್ಟಿಗೆ
ಕ್ರೂರ-ದೃಷ್ಟಿ-ಯನ್ನೂ
ಕ್ರೂರವಾಗಿ
ಕ್ರೂರಿ
ಕ್ರೂರಿಗಳೂ
ಕ್ರೂರಿಯನ್ನು
ಕ್ರೂರಿಯೂ
ಕ್ರೆಸ್ಸಿ
ಕ್ರೈಸ್ತ
ಕ್ರೈಸ್ತ-ಧರ್ಮ-ಗು-ರು-ಗಳು
ಕ್ರೈಸ್ತ-ಧರ್ಮಾ-ಧಿ-ಕಾರಿ
ಕ್ರೈಸ್ತ-ಧರ್ಮಾ-ಧಿ-ಕಾ-ರಿ-ಗ-ಳಲ್ಲಿ
ಕ್ರೈಸ್ತ-ಮ-ತದ
ಕ್ರೈಸ್ತ-ಮ-ತ-ದಲ್ಲಿ
ಕ್ರೈಸ್ತ-ಮ-ತೀಯ
ಕ್ರೈಸ್ತರ
ಕ್ರೈಸ್ತ-ಸಂಪ್ರ-ದಾ-ಯದ
ಕ್ರೈಸ್ತ-ಸಂಪ್ರ-ದಾ-ಯ-ದಲ್ಲಿ
ಕ್ರೋಧ
ಕ್ರೋಧದ
ಕ್ರೋಧಾದ್ಭ-ವತಿ
ಕ್ರೋಧೋನ್ಮತ್ತ-ನಾ-ದಾಗ
ಕ್ರೋಶದಲ್ಲೂ
ಕ್ರೌರ್ಯ
ಕ್ರೌರ್ಯಕ್ಕೆ
ಕ್ರೌರ್ಯವನ್ನೂ
ಕ್ರ್ಯಾನ್ಸಿ
ಕ್ರ್ಯಾನ್ಸೀ
ಕ್ಲಬ್ನ
ಕ್ಲಬ್ಬನ್ನು
ಕ್ಲಬ್ಬಿಗೆ
ಕ್ಲಬ್ಬಿನಲ್ಲಿ
ಕ್ಲಬ್ಬೂ
ಕ್ಲಾಸನ್ನು
ಕ್ಲಾಸಿಗೆ
ಕ್ಲಾಸಿನ
ಕ್ಲಾಸಿನಲ್ಲಿ
ಕ್ಲಾಸಿ-ನಲ್ಲಿದ್ದ
ಕ್ಲಾಸು-ಗ-ಳಲ್ಲಿ
ಕ್ಲಾಸು-ಗ-ಳಲ್ಲೂ
ಕ್ಲಿನಿಕಲ್
ಕ್ಲಿನಿಕ್ನ
ಕ್ಲಿನಿಕ್ನಲ್ಲೇ
ಕ್ಲಿಷ್ಟ-ವಾ-ಗುತ್ತದೆ
ಕ್ಲೆಮೆಂಟ್
ಕ್ಲೇರ್
ಕ್ಲೇಶ
ಕ್ಲೇಶ-ಕ-ರ-ವಾದ
ಕ್ಲೇಶ-ಗ-ಳನ್ನು
ಕ್ಲೇಶದಿಂದ
ಕ್ವಾಂಟಂ
ಕ್ಷಣ
ಕ್ಷಣಕಾಲ
ಕ್ಷಣ-ಕಾ-ಲ-ವಿ-ರುತ್ತದೆ
ಕ್ಷಣಕ್ಷ-ಣದ
ಕ್ಷಣಕ್ಷ-ಣವೂ
ಕ್ಷಣ-ಗ-ಳಲ್ಲಿ
ಕ್ಷಣ-ಗ-ಳಲ್ಲೇ
ಕ್ಷಣದಲ್ಲಿ
ಕ್ಷಣದಲ್ಲೆ
ಕ್ಷಣ-ಮಾತ್ರ-ದಲ್ಲಾ-ಗ-ಬೇಕು
ಕ್ಷಣ-ವಾ-ದರೂ
ಕ್ಷಣವೂ
ಕ್ಷಣವೇ
ಕ್ಷಣಾರ್ಧ-ದಲ್ಲಿ
ಕ್ಷಣಿಕ
ಕ್ಷಣಿ-ಕ-ವಾದ
ಕ್ಷಣಿ-ಕ-ವೆಂದು
ಕ್ಷಮಾ-ಯಾ-ಚನೆ
ಕ್ಷಮಾ-ಯಾ-ಚನೆ
ಕ್ಷಮಾ-ಶೀ-ಲ-ರಾ-ಗುತ್ತಾರೆ
ಕ್ಷಮಿ-ಸ-ಲಾ-ರದು
ಕ್ಷಮಿಸಿ
ಕ್ಷಮಿ-ಸಿದ್ದಳು
ಕ್ಷಮಿ-ಸಿ-ಯಾನು
ಕ್ಷಮಿಸು
ಕ್ಷಮಿಸುತ್ತ
ಕ್ಷಮೆ
ಕ್ಷಮೆ-ಇ-ವು-ಗ-ಳನ್ನು
ಕ್ಷಮ್ಯವಾದ
ಕ್ಷಯ
ಕ್ಷಯದಿಂದ
ಕ್ಷಯರೋಗ
ಕ್ಷಯ-ರೋ-ಗದ
ಕ್ಷಯ-ರೋ-ಗ-ದಿಂದ
ಕ್ಷಯ-ರೋ-ಗವು
ಕ್ಷಾತ್ರ-ತೇ-ಜವು
ಕ್ಷಾತ್ರಶಕ್ತಿ
ಕ್ಷೀಣ
ಕ್ಷುದ್ರ
ಕ್ಷುದ್ರತನ
ಕ್ಷುಲ್ಲಕ
ಕ್ಷೇತ್ರ
ಕ್ಷೇತ್ರಗಳ
ಕ್ಷೇತ್ರ-ಗ-ಳನ್ನು
ಕ್ಷೇತ್ರ-ಗ-ಳಲ್ಲಿ
ಕ್ಷೇತ್ರ-ಗ-ಳಲ್ಲಿನ
ಕ್ಷೇತ್ರ-ಗ-ಳಲ್ಲಿಯೂ
ಕ್ಷೇತ್ರ-ಗ-ಳಲ್ಲೂ
ಕ್ಷೇತ್ರ-ಗ-ಳಿಗೂ
ಕ್ಷೇತ್ರದ
ಕ್ಷೇತ್ರದಂತೆ
ಕ್ಷೇತ್ರದಲ್ಲಿ
ಕ್ಷೇತ್ರ-ದಲ್ಲಿನ
ಕ್ಷೇತ್ರದಲ್ಲೂ
ಕ್ಷೇತ್ರದಲ್ಲೇ
ಕ್ಷೇತ್ರವನ್ನು
ಕ್ಷೇತ್ರ-ವಾ-ಗಲಿ
ಕ್ಷೇಮ
ಕ್ಷೇಮಕ್ಕಾಗಿ
ಕ್ಷೇಮಕ್ಕೆ
ಕ್ಷೇಮ-ಚಿಂತ-ನೆ-ಯಿಂದ
ಕ್ಷೇಮ-ದಿಂದಿದ್ದೇನೆ
ಕ್ಷೇಮ-ಪಾ-ಲ-ಕರೂ
ಕ್ಷೇಮ-ವಿ-ದೆಯೆ
ಕ್ಷೇಮವೇ
ಕ್ಷೋಭೆ-ಯನ್ನುಂಟು-ಮಾ-ಡಿದೆ
ಕ್ಷೌರ
ಕ್ಷೌರಿಕ
ಖಂಡಗಳ
ಖಂಡದ
ಖಂಡ-ನೆ-ಯನ್ನು
ಖಂಡಿತ
ಖಂಡಿ-ತ-ವಾಗಿ
ಖಂಡಿ-ತ-ವಾ-ಗಿಯೂ
ಖಂಡಿಸ
ಖಂಡಿಸದೆ
ಖಂಡಿ-ಸ-ಬೇ-ಕಾ-ಗುತ್ತದೆ
ಖಂಡಿಸಿದ
ಖಂಡಿ-ಸು-ವುದೇ
ಖಗೋ-ಳಾ-ದಿ-ಶಾಸ್ತ್ರ-ಗಳು
ಖಚಿತ
ಖಚಿ-ತ-ಪ-ಡಿ-ಸಿ-ರು-ವು-ದ-ರಿಂದ
ಖಚಿ-ತ-ವಾ-ಗ-ಬೇಕು
ಖಚಿ-ತ-ವಾಗಿ
ಖಚಿ-ತ-ವಾ-ಗುತ್ತದೆ
ಖಚಿ-ತ-ವಾದ
ಖಚಿ-ತ-ವಾ-ದರೆ
ಖಚಿ-ತ-ವಾ-ದು-ದಕ್ಕೆ
ಖಚಿ-ತ-ವೆ-ನಿ-ಸುತ್ತದೆ
ಖಜಾನೆ
ಖಡ್ಗ
ಖಡ್ಗವೇ
ಖಭೌತ
ಖಯಾಲಿ
ಖಯ್ಯಾಮ್
ಖರ್ಚಾಗಿತ್ತು
ಖರ್ಚಿಗೆ
ಖರ್ಚಿನ
ಖರ್ಚು
ಖರ್ಚು-ಗ-ಳೇನೂ
ಖರ್ಚುಮಾಡಿ
ಖರ್ಚು-ಮಾ-ಡುವ
ಖರ್ಚುವೆಚ್ಚ
ಖರ್ಚೇನೂ
ಖಲೀಲ್
ಖಾತ್ರಿ
ಖಾತ್ರಿ-ಯಾ-ಗಿ-ರ-ಬೇಕು
ಖಾರ
ಖಾರ-ವಾ-ಗಿಯೇ
ಖಾಸಗಿ
ಖಾಸ-ಗಿ-ಯ-ವ-ರೆಂದು
ಖಾಸ-ಗಿ-ಯಾ-ಗಿಯೇ
ಖಾಸಗೀ
ಖಿನ್ನತೆ
ಖಿನ್ನ-ನಾ-ಗುತ್ತಿದ್ದ
ಖಿನ್ನ-ರಾ-ಗಿ-ಬಿ-ಡುವ
ಖುದ್ದಾಗಿ
ಖುಷಿ-ಯಾ-ಗಿತ್ತು
ಖೇಲನದ
ಖೋಟಾತನ
ಖ್ಯಾತ
ಖ್ಯಾತಿ
ಖ್ಯಾತಿಯ
ಗಂಗೆಯ
ಗಂಟಲ
ಗಂಟಲು
ಗಂಟ-ಲು-ಬೇ-ನೆಗೆ
ಗಂಟಿನಂತೆ
ಗಂಟು
ಗಂಟು-ತಿಂದ-ವ-ರಂತೆ
ಗಂಟೆ
ಗಂಟೆ-ಗ-ಳಿಗೂ
ಗಂಟೆ-ಗಂಟೆಗೂ
ಗಂಟೆ-ಗಟ್ಟಲೆ
ಗಂಟೆಗಳ
ಗಂಟೆ-ಗ-ಳಲ್ಲಿ
ಗಂಟೆ-ಗ-ಳಾ-ಗಿದ್ದುವು
ಗಂಟೆ-ಗ-ಳಿ-ಗಿಂತಲೂ
ಗಂಟೆ-ಗ-ಳಿಗೂ
ಗಂಟೆ-ಗ-ಳಿ-ಗೊಮ್ಮೆ
ಗಂಟೆ-ಗ-ಳಿ-ವೆ-ಯಷ್ಟೆ
ಗಂಟೆಗೆ
ಗಂಟೆಗೇನೇ
ಗಂಟೆಯ
ಗಂಟೆಯಲ್ಲಿ
ಗಂಟೆಯಾಗಿ
ಗಂಟೆ-ಯಾ-ದರೂ
ಗಂಟೆಯಿಂದ
ಗಂಡ
ಗಂಡಂದಿರ
ಗಂಡಂದಿ-ರನ್ನು
ಗಂಡಂದಿರು
ಗಂಡನ
ಗಂಡನನ್ನು
ಗಂಡನಾಗಿ
ಗಂಡ-ನಿ-ಗಾ-ಗಿಯೂ
ಗಂಡನಿಗೆ
ಗಂಡನೇ
ಗಂಡಯೋಗ
ಗಂಡಸರು
ಗಂಡಸರೂ
ಗಂಡಸಾಗಿ
ಗಂಡ-ಸಾ-ಗಿದ್ದು-ದಾಗಿ
ಗಂಡಸು
ಗಂಡ-ಹೆಂಡಿರ
ಗಂಡ-ಹೆಂಡಿ-ರಲ್ಲಿ-ರಲಿ
ಗಂಡಾಂತರ
ಗಂಡಾಂತ-ರಕ್ಕೊಡ್ಡಿ
ಗಂಡಾಂತ-ರ-ಗ-ಳನ್ನು
ಗಂಡು
ಗಂಡು-ಹೆಣ್ಣು-ಗಳ
ಗಂತವ್ಯ-ವನ್ನು
ಗಂತವ್ಯಸ್ಥಾ-ನಕ್ಕೆ
ಗಂತಿ-ಗ-ಳನ್ನೂ
ಗಂಧ
ಗಂಧಗಾಳಿ
ಗಂಧದ
ಗಂಧದಂತೆ
ಗಂಭೀರ
ಗಂಭೀ-ರಧ್ಯಾ-ನ-ದಲ್ಲಿ
ಗಂಭೀ-ರ-ವಾಗಿ
ಗಂಭೀ-ರ-ವಾ-ಗಿದ್ದಳು
ಗಂಭೀ-ರ-ವಾ-ಗುತ್ತಾನೆ
ಗಂಭೀ-ರ-ವಾದ
ಗಗನ
ಗಗ-ನ-ಕು-ಸು-ಮವೆ
ಗಗ-ನ-ದಲ್ಲಿ
ಗಗ-ನ-ಯಾನ
ಗಜಗಳ
ಗಜೇಂದ್ರ-ನಿಗೆ
ಗಟ್ಟಿ-ಯಂತಲ್ಲ
ಗಟ್ಟಿಯಾಗಿ
ಗಟ್ಟಿಯಿದೆ
ಗಟ್ಟಿಯೇ
ಗಡ-ಸಾ-ಗುತ್ತದೆ
ಗಡಿ
ಗಡಿ-ಬಿ-ಡಿಯ
ಗಡೀ-ಪಾ-ರಾ-ದರೂ
ಗಡೆ
ಗಡ್ಡ
ಗಡ್ಡ-ಧಾ-ರಿ-ಗ-ಳೆಲ್ಲ
ಗಡ್ಡವನ್ನು
ಗಡ್ಡೆ-ಗ-ಳನ್ನೂ
ಗಣನಾಥ
ಗಣನೆಗೆ
ಗಣನೆಯ
ಗಣಿ
ಗಣಿಗಳ
ಗಣಿತ
ಗಣಿತಜ್ಞ
ಗಣಿ-ತಜ್ಞ-ತೆ-ಯಾ-ಗಲಿ
ಗಣಿ-ತಜ್ಞರು
ಗಣಿ-ತಜ್ಞಾ-ನದ
ಗಣಿತದ
ಗಣಿ-ತ-ದಲ್ಲಿ
ಗಣಿ-ತ-ವನ್ನು
ಗಣಿ-ತ-ಶಾಸ್ತ್ರದ
ಗಣಿ-ತ-ಸೂತ್ರ
ಗಣಿಯಾಗಿ
ಗಣಿ-ಯಾ-ಗಿ-ರುವ
ಗಣಿ-ಯಾ-ಗುತ್ತಾನೆ
ಗಣ್ಯ
ಗಣ್ಯನೇ
ಗಣ್ಯ-ರೊಬ್ಬರು
ಗತ-ಕಾ-ಲದ
ಗತ-ಜೀ-ವ-ನ-ದಲ್ಲಿ
ಗತಾ-ನು-ಗ-ತಿ-ಕ-ವಾಗಿ
ಗತಿ
ಗತಿಯ
ಗತಿಯನ್ನು
ಗತಿ-ಯಾ-ಯಿತು
ಗತಿಯಿಲ್ಲ
ಗತಿಯೇ
ಗತಿ-ವೈ-ಚಿತ್ರ್ಯ
ಗತಿಸಿದ
ಗತಿ-ಸಿದ್ದ-ನೆಂದು
ಗತಿ-ಸಿ-ಹೋದ
ಗತ್ಯಂತ-ರ-ವಿಲ್ಲ
ಗದ
ಗದ-ಗಿ-ನಲ್ಲಿ
ಗದರಿಕೆ
ಗದ-ರಿ-ಕೆ-ಗಳು
ಗದ-ರಿ-ಸ-ಬಲ್ಲರು
ಗದ-ರಿ-ಸದೆ
ಗದರಿಸಿ
ಗದ-ರಿ-ಸಿತು
ಗದ-ರಿ-ಸಿದ
ಗದ-ರಿ-ಸಿ-ದರೂ
ಗದ-ರಿ-ಸಿ-ದರೆ
ಗದ-ರಿ-ಸುವ
ಗದ-ರಿ-ಸು-ವು-ದುಂಟು
ಗದು
ಗದ್ಗ-ದಿ-ತ-ನಾಗಿ
ಗದ್ದಲ
ಗದ್ದ-ಲ-ಗಳ
ಗಬ್ಬದಲಿ
ಗಬ್ಬ-ವಾ-ಗಿ-ರುವ
ಗಭೀರ
ಗಮನ
ಗಮನಕ್ಕೆ
ಗಮನಕ್ಕೇ
ಗಮ-ನ-ದಲ್ಲಿಟ್ಟು-ಕೊಂಡಿ-ರು-ವು-ದಿಲ್ಲ
ಗಮ-ನ-ದಲ್ಲಿಟ್ಟು-ಕೊಂಡು
ಗಮ-ನ-ದಲ್ಲಿ-ರಿ-ಸಿ-ಕೊಂಡು
ಗಮ-ನ-ವನ್ನು
ಗಮ-ನ-ವನ್ನೇ
ಗಮ-ನ-ವಿಟ್ಟು
ಗಮ-ನ-ವಿತ್ತರೂ
ಗಮ-ನ-ವಿತ್ತರೆ
ಗಮ-ನ-ವಿತ್ತಿ-ರ-ಲಿಲ್ಲ
ಗಮ-ನ-ವಿತ್ತಿಲ್ಲ
ಗಮ-ನ-ವಿತ್ತು
ಗಮ-ನ-ವಿ-ರ-ಬೇ-ಕಾ-ದುದು
ಗಮ-ನ-ವಿ-ರಿ-ಸದೆ
ಗಮ-ನ-ವೀಯ
ಗಮ-ನ-ವೀ-ಯದೆ
ಗಮ-ನ-ವೀ-ಯ-ಬಲ್ಲರು
ಗಮ-ನ-ವೀ-ಯ-ಬೇ-ಕಾ-ಗಿಲ್ಲ
ಗಮ-ನ-ವೀ-ಯ-ಬೇಕು
ಗಮ-ನ-ವೀ-ಯ-ಲಿಲ್ಲ
ಗಮ-ನ-ವೀ-ಯುತ್ತ
ಗಮನವೂ
ಗಮನವೇ
ಗಮನಾರ್ಹ
ಗಮ-ನಾರ್ಹ-ವಾ-ದುದು
ಗಮ-ನಿ-ಸಿಲ್ಲ-ವಲ್ಲ
ಗಮ-ನಿ-ಸ-ತಕ್ಕ
ಗಮ-ನಿ-ಸ-ದಂತಿರಿ
ಗಮ-ನಿ-ಸದೆ
ಗಮ-ನಿ-ಸ-ಬೇ-ಕಾದ
ಗಮ-ನಿ-ಸ-ಬೇಕು
ಗಮ-ನಿ-ಸಲು
ಗಮ-ನಿ-ಸಲೇ
ಗಮನಿಸಿ
ಗಮ-ನಿ-ಸಿ-ದಂತಿಲ್ಲ
ಗಮ-ನಿ-ಸಿ-ದರೂ
ಗಮ-ನಿ-ಸಿ-ದಾಗ
ಗಮ-ನಿ-ಸಿದ್ದೀರಿ
ಗಮ-ನಿ-ಸುತ್ತಿದ್ದರು
ಗಮ-ನಿ-ಸುತ್ತಿದ್ದೆ
ಗಮ-ನಿ-ಸುತ್ತಿಲ್ಲ
ಗಮ-ನಿ-ಸುತ್ತೇನೆ
ಗಮಯ
ಗಮ್ಯವಾದ
ಗಯಟೆ
ಗಯಟೇ
ಗಯ್ಯಾಳಿ
ಗರ-ಡಿ-ಯಲ್ಲಿ
ಗರಿ-ಗೆ-ದ-ರಲು
ಗರಿ-ಗೆ-ದ-ರಿ-ದು-ದನ್ನು
ಗರಿ-ಗೆ-ದ-ರು-ವು-ದನ್ನು
ಗರಿ-ಮೆ-ಗಳ
ಗರ್
ಗರ್ಜನೆ
ಗರ್ಜ-ನೆ-ಯನ್ನು
ಗರ್ಜಿಸಿ
ಗರ್ದಭ
ಗರ್ಭ
ಗರ್ಭದಲ್ಲಿ
ಗರ್ಭ-ದಲ್ಲಿರು
ಗರ್ಭದಿಂದ
ಗರ್ಭ-ಧಾ-ರ-ಣೆಯ
ಗರ್ಭ-ವಾ-ಗಿದ್ದ
ಗರ್ಭವಾಸ
ಗರ್ಭ-ವಾ-ಸದ
ಗರ್ಭಿ-ಣಿ-ಯ-ರಾ-ದಾಗ
ಗರ್ಲ್ಫ್ರೆಂಡ್
ಗಲಾಟೆ
ಗಲಿಬಿಲಿ
ಗಲಿ-ಬಿ-ಲಿ-ಗಳು
ಗಲಿ-ಬಿ-ಲಿಗೆ
ಗಲ್ಲದಲ್ಲಿ
ಗಲ್ಲಿಗೇರಿ
ಗಲ್ಲಿ-ಯಲ್ಲಿನ
ಗಲ್ಲುಗಳು
ಗಳಂತೆ
ಗಳನ್ನು
ಗಳನ್ನೆಲ್ಲ
ಗಳಲ್ಲಿಯೂ
ಗಳ-ಹುತ್ತ-ದೆಂಬುದು
ಗಳಿಂದ
ಗಳಿದ್ದಂತೆ
ಗಳಿವೆ
ಗಳಿ-ಸ-ಬಲ್ಲಿರಿ
ಗಳಿಸಲು
ಗಳಿಸಿ
ಗಳಿ-ಸಿ-ಕೊಂಡಿದ್ದೀರಾ
ಗಳಿ-ಸಿ-ಕೊಳ್ಳಲು
ಗಳಿಸಿದ
ಗಳಿ-ಸಿ-ದರು
ಗಳಿ-ಸಿ-ದರೂ
ಗಳಿ-ಸಿ-ದರೆ
ಗಳಿ-ಸಿದ್ದಾ-ನಂತೆ
ಗಳಿ-ಸಿದ್ದೇನೆ
ಗಳಿಸುತ್ತ
ಗಳೂ
ಗಳೊಬ್ಬರು
ಗಹನ
ಗಹನತೆ
ಗಾಂಧಿ
ಗಾಂಧೀಜಿ
ಗಾಂಧೀ-ಜಿ-ಇ-ವ-ರನ್ನು
ಗಾಂಧೀ-ಜಿ-ಯನ್ನು
ಗಾಂಧೀ-ಜಿ-ಯ-ವರ
ಗಾಂಧೀ-ಜಿ-ಯ-ವ-ರನ್ನು
ಗಾಂಧೀಜೀ
ಗಾಂಧೀ-ಜೀ-ಯ-ವರು
ಗಾಂಭೀರ್ಯ
ಗಾಂಭೀರ್ಯದ
ಗಾಂಭೀರ್ಯ-ದಿಂದ
ಗಾಗಿ
ಗಾಜಿನ
ಗಾಡಿ
ಗಾಡಿಗೆ
ಗಾಡಿಯನ್ನು
ಗಾಡಿಯಲ್ಲಿ
ಗಾಡ್
ಗಾಢ-ಚಿಂತ-ನೆ-ಯಲ್ಲಿ
ಗಾಢ-ಚಿಂತ-ನೆ-ಯಿಂದ
ಗಾಢ-ನಿದ್ರೆಯ
ಗಾಢ-ನಿದ್ರೆ-ಯಲ್ಲಿ
ಗಾಢ-ನಿದ್ರೆ-ಯಲ್ಲೇ
ಗಾಢವಾಗಿ
ಗಾಢವಾದ
ಗಾಢಾಂಧ-ಕಾರ
ಗಾತ್ರ
ಗಾತ್ರ-ಗ-ಳಿಗೇ
ಗಾತ್ರದ
ಗಾದರೂ
ಗಾದೆ
ಗಾದೆಯ
ಗಾದೆ-ಯೊಂದನ್ನು
ಗಾನ-ಕ-ಲಾ-ಪ-ರಿ-ಣತ
ಗಾನ-ಗೋಷ್ಠಿ-ಯನ್ನು
ಗಾಬರಿ
ಗಾಬ-ರಿ-ಗ-ಳನ್ನು
ಗಾಬ-ರಿ-ಗೊಳ್ಳದೇ
ಗಾಬ-ರಿ-ಯಾಗಿ
ಗಾಯಕ
ಗಾಯ-ಕ-ನಿಗೆ
ಗಾಯಕನು
ಗಾಯ-ಗೊಂಡರು
ಗಾಯಗೊಂಡು
ಗಾಯ-ಗೊಳ್ಳು-ವ-ವರು
ಗಾಯತ್ರಿ
ಗಾಯತ್ರೀ
ಗಾಯದಿಂದ
ಗಾಯವನ್ನು
ಗಾಯವನ್ನೂ
ಗಾರರು
ಗಾರುಡಿ
ಗಾಳಿ
ಗಾಳಿ-ಗ-ಳಂತೆ
ಗಾಳಿಗಳು
ಗಾಳಿಗೆ
ಗಾಳಿ-ಗೋ-ಪು-ರದ
ಗಾಳಿನೀರು
ಗಾಳಿಯ
ಗಾಳಿಯನ್ನು
ಗಾಳಿಯಲ್ಲಿ
ಗಾಳಿಯಿದೆ
ಗಾಳಿ-ಯೊ-ಡನೆ
ಗಾವಲಿನ
ಗಿಂತ
ಗಿಟ್ಟಿ-ಸ-ಬಲ್ಲರು
ಗಿಟ್ಟಿ-ಸಿ-ಕೊಂಡಿ-ತಂತೆ
ಗಿಟ್ಟು-ವು-ದಿಲ್ಲ-ವಂತೆ
ಗಿಡ-ಗ-ಳನ್ನು
ಗಿಡ-ಗ-ಳಿಗೂ
ಗಿಡದ
ಗಿಡ-ಮ-ರ-ಗ-ಳಲ್ಲಿ
ಗಿಡವಲ್ಲ
ಗಿದೆ
ಗಿದ್ದ
ಗಿದ್ದು
ಗಿನ
ಗಿಮಿಕ್ಸ್ನಿಂದ
ಗಿರ-ಗಿ-ರನೆ
ಗಿರಾ-ಕಿ-ಗ-ಳೆ-ದು-ರಿಗೆ
ಗಿರಾ-ಕಿ-ಗ-ಳೊ-ಡನೆ
ಗಿರಿ-ಧ-ರ-ಲಾಲ್
ಗಿರಿಪ್ರ-ದಕ್ಷಿಣೆ
ಗಿರಿಯ
ಗೀತ
ಗೀತಾ
ಗೀತೋಕ್ತಿಗೆ
ಗೀತೋ-ಪ-ದೇ-ಶ-ವಾ-ಯಿತು
ಗೀನಾ
ಗೀಳು
ಗುಂಗಿನಲ್ಲೇ
ಗುಂಡ
ಗುಂಡಣ್ಣ
ಗುಂಡಿ
ಗುಂಡಿನ
ಗುಂಡಿ-ನೇ-ಟಿಗೆ
ಗುಂಡು
ಗುಂಡು-ಗ-ಳಿಂದ
ಗುಂಡು-ಗ-ಳಿಗೆ
ಗುಂಡುಸೂಜಿ
ಗುಂಡು-ಸೂ-ಜಿ-ಯನ್ನು
ಗುಂಡು-ಸೂ-ಜಿಯು
ಗುಂಡು-ಹಾ-ರಿ-ಸಿ-ಕೊಂಡು
ಗುಂಡೂರಾಯ
ಗುಂಡೊಂದನ್ನು
ಗುಂಪನ್ನು
ಗುಂಪಿಗೆ
ಗುಂಪಿನ
ಗುಂಪಿನಲ್ಲಿ
ಗುಂಪಿ-ನಲ್ಲಿದ್ದ
ಗುಂಪಿನಲ್ಲೂ
ಗುಂಪಿ-ನೊಂದಿಗೆ
ಗುಂಪು
ಗುಂಪು-ಗ-ಳಲ್ಲಿ
ಗುಂಪು-ಗ-ಳಾಗಿ
ಗುಂಪು-ಗ-ಳಿವೆ
ಗುಂಪುಗಳು
ಗುಂಪು-ಗ-ಳೊ-ಳಗೆ
ಗುಂಪು-ಗೂ-ಡಿ-ಕೊಂಡು
ಗುಂಪು-ಗೂ-ಡು-ವುದು
ಗುಂಪೊಂದು
ಗುಜ-ರಾ-ತಿನ
ಗುಜ್ಜಾಗುವ
ಗುಜ್ಜು
ಗುಟ್ಟಾಗಿ
ಗುಟ್ಟು
ಗುಟ್ಟೂ
ಗುಟ್ಟೆಲ್ಲ
ಗುಟ್ಟೇ-ನೆಂದೊಮ್ಮೆ
ಗುಡಾಣ
ಗುಡಿ
ಗುಡಿಗಳೇ
ಗುಡಿಯಲ್ಲಿ
ಗುಡಿಸದೇ
ಗುಡಿ-ಸ-ಲಲ್ಲಿ-ರು-ವ-ವನೂ
ಗುಡಿ-ಸ-ಲಿ-ನಲ್ಲಿದೆ
ಗುಡಿಸಿ
ಗುಡು-ಗಿ-ದರು
ಗುಡ್ಡದ
ಗುಣ
ಗುಣ-ಪ-ಡಿ-ಸಿದೆ
ಗುಣ-ವಂತರೂ
ಗುಣ-ಕರ್ಮ-ಗ-ಳನ್ನೂ
ಗುಣಗಳ
ಗುಣ-ಗ-ಳನ್ನು
ಗುಣ-ಗ-ಳನ್ನೂ
ಗುಣ-ಗ-ಳನ್ನೇ
ಗುಣ-ಗ-ಳಿಂದ
ಗುಣ-ಗ-ಳಿದ್ದರೂ
ಗುಣ-ಗ-ಳಿವೆ
ಗುಣಗಳು
ಗುಣ-ಗ-ಳು-ಶೌಚ
ಗುಣ-ಗ-ಳೆಲ್ಲವೂ
ಗುಣ-ಗಾತ್ರಕ್ಕಿಂತ
ಗುಣಗಾನ
ಗುಣ-ಗು-ಣಿ-ಸುತ್ತ
ಗುಣಗ್ರಾ-ಹಿ-ಗ-ಳೆಲ್ಲರೂ
ಗುಣ-ಧರ್ಮ-ಗಳ
ಗುಣ-ಧರ್ಮ-ಗ-ಳೊಂದಿಗೆ
ಗುಣ-ಧರ್ಮ-ವನ್ನು
ಗುಣ-ನ-ಡ-ತೆ-ಗ-ಳಲ್ಲಿಲ್ಲ
ಗುಣ-ಪ-ಡಿ-ಸ-ಬಲ್ಲದು
ಗುಣ-ಪ-ಡಿ-ಸ-ಬ-ಹುದು
ಗುಣ-ಪ-ಡಿ-ಸ-ಲಾ-ಗ-ಲಿಲ್ಲ
ಗುಣ-ಪ-ಡಿ-ಸಲು
ಗುಣ-ಪ-ಡಿ-ಸಲೂ
ಗುಣ-ಪ-ಡಿ-ಸಿದ
ಗುಣ-ಪ-ಡಿ-ಸುವ
ಗುಣ-ಪ-ಡಿ-ಸು-ವ-ವ-ರನ್ನೂ
ಗುಣಮಟ್ಟ
ಗುಣ-ಮಟ್ಟವೂ
ಗುಣ-ಮು-ಖ-ರಾ-ಗುತ್ತಿದ್ದರು
ಗುಣ-ಮು-ಖ-ರಾ-ದರು
ಗುಣ-ವಂತ-ರನ್ನು
ಗುಣವನ್ನು
ಗುಣ-ವಾ-ಗದು
ಗುಣವಾಗಿ
ಗುಣ-ವಾ-ಗುತ್ತವೆ
ಗುಣವಾದ
ಗುಣ-ವಾ-ದು-ದನ್ನು
ಗುಣವೇ
ಗುಣ-ವೈ-ಭ-ವ-ವನ್ನು
ಗುಣ-ವೈ-ಶಿಷ್ಟ್ಯ-ಗ-ಳನ್ನು
ಗುಣಾ-ವ-ಗು-ಣ-ಗ-ಳಿಂದಾಗಿ
ಗುಣಿಸಿದ
ಗುಣಿ-ಸುತ್ತಿದ್ದೆ
ಗುತ್ತ
ಗುದ್ದಾಟ
ಗುದ್ದಾಡಿ
ಗುದ್ದಾಡಿಯೇ
ಗುದ್ದು
ಗುಪ್ತ
ಗುಪ್ತದ್ವೇ-ಷವು
ಗುಪ್ತನಿಧಿ
ಗುಪ್ತ-ನಿ-ಧಿ-ಯನ್ನು
ಗುಪ್ತ-ರೂ-ಪ-ದಿಂದ
ಗುಪ್ತ-ರೂ-ಪ-ವಾ-ಗಿ-ಡು-ವ-ವನು
ಗುಪ್ತವಾಗಿ
ಗುಪ್ತ-ವಾ-ಗಿಯೋ
ಗುಪ್ತ-ವಿ-ಚಾ-ರ-ಗ-ಳನ್ನು
ಗುಬ್ಬಚ್ಚಿ-ಗಳು
ಗುರಿ
ಗುರಿ-ಯಾ-ದ-ವ-ರನ್ನು
ಗುರಿಗಳ
ಗುರಿ-ಗೊಯ್ಯುವು
ಗುರಿ-ಪ-ಡಿ-ಸೋಣ
ಗುರಿಯ
ಗುರಿಯತ್ತ
ಗುರಿಯನ್ನು
ಗುರಿಯನ್ನೂ
ಗುರಿಯನ್ನೇ
ಗುರಿ-ಯನ್ನೇನೋ
ಗುರಿಯನ್ನೊ
ಗುರಿಯಲ್ಲಿ
ಗುರಿ-ಯಲ್ಲಿನ
ಗುರಿಯಷ್ಟೇ
ಗುರಿ-ಯಾ-ಗಲು
ಗುರಿ-ಯಾ-ಗಿತ್ತು
ಗುರಿ-ಯಾ-ಗಿದೆ
ಗುರಿ-ಯಾ-ಗಿದ್ದ
ಗುರಿ-ಯಾ-ಗುತ್ತಾನೆ
ಗುರಿ-ಯಾ-ಗುವ
ಗುರಿ-ಯಾ-ಗು-ವುದು
ಗುರಿ-ಯಾ-ದರೆ
ಗುರಿಯಿಂದ
ಗುರಿಯು
ಗುರಿ-ಯೆ-ಡೆಗೆ
ಗುರಿಯೇ
ಗುರಿ-ಸೇ-ರು-ವ-ವ-ರೆಗೂ
ಗುರು
ಗುರು-ಮ-ನೆ-ಗ-ಳಲ್ಲಿ-ರು-ವ-ವನೂ
ಗುರುಗಳ
ಗುರುಗಳು
ಗುರು-ಗ-ಳೆಲ್ಲ
ಗುರು-ತಿ-ಸ-ದ-ವರು
ಗುರು-ತಿ-ಸ-ಬಲ್ಲದು
ಗುರು-ತಿ-ಸ-ಬಲ್ಲಳೇ
ಗುರು-ತಿ-ಸ-ಬಲ್ಲ-ವ-ನಾ-ಗಿದ್ದ
ಗುರು-ತಿ-ಸ-ಬಲ್ಲುದು
ಗುರು-ತಿ-ಸ-ಬ-ಹು-ದಾದ
ಗುರು-ತಿ-ಸ-ಬ-ಹು-ದಿತ್ತು
ಗುರು-ತಿ-ಸ-ಬ-ಹುದು
ಗುರು-ತಿ-ಸ-ಬೇಕು
ಗುರು-ತಿ-ಸ-ಲಾ-ರಳು
ಗುರು-ತಿ-ಸಲು
ಗುರು-ತಿ-ಸಲೇ
ಗುರುತಿಸಿ
ಗುರು-ತಿ-ಸಿದ
ಗುರು-ತಿ-ಸಿ-ದರು
ಗುರು-ತಿ-ಸಿದೆ
ಗುರು-ತಿ-ಸಿದ್ದರು
ಗುರು-ತಿ-ಸುವ
ಗುರುತು
ಗುರು-ತು-ಗ-ಳನ್ನು
ಗುರುತ್ವಾ-ಕರ್ಷಣ
ಗುರು-ಭಕ್ತಿ-ಯಲ್ಲಿ
ಗುರುವಾಗಿ
ಗುರು-ವಾ-ಗಿದ್ದರೂ
ಗುರುವಿಗೆ
ಗುರುವಿನ
ಗುರು-ಶಕ್ತಿ-ಯಿಂದ
ಗುರು-ಹಿ-ರಿ-ಯರ
ಗುರು-ಹಿ-ರಿ-ಯ-ರಲ್ಲಿ
ಗುರು-ಹಿ-ರಿ-ಯ-ರಿಗೆ
ಗುರು-ಹಿ-ರಿ-ಯರು
ಗುರೂಜಿ
ಗುರೂ-ಪ-ದೇ-ಶ-ವನ್ನು
ಗುರ್ಗು-ರಾ-ಯತೇ
ಗುಲಾಬಿ
ಗುಲಾ-ಬಿ-ಹೂ-ವನ್ನು
ಗುಲಾ-ಬಿ-ಹೂ-ವಿನ
ಗುಲಾಮ
ಗುಲಾ-ಮ-ಗಿ-ರಿಗೆ
ಗುಲಾ-ಮ-ಗಿ-ರಿಯ
ಗುಲಾ-ಮ-ಗಿ-ರಿ-ಯಲ್ಲಿ
ಗುಲಾ-ಮ-ಗಿ-ರಿ-ಯಲ್ಲಿಟ್ಟು
ಗುಲಾ-ಮ-ಗಿ-ರಿ-ಯಿಂದ
ಗುಲಾ-ಮ-ನಾ-ಗ-ಲಿಚ್ಛಿ-ಸು-ವು-ದಿಲ್ಲ
ಗುಲಾ-ಮ-ನಾ-ಗುತ್ತಾನೆ
ಗುಲಾಮರ
ಗುಲಾ-ಮ-ರಂತೆ
ಗುಲಾ-ಮ-ರಾ-ಗಿ-ರಲು
ಗುಲಾ-ಮ-ರಿಗೆ
ಗುಲಾಮರು
ಗುಲ್ಲೆಬ್ಬಿ-ಸಿ-ದಳು
ಗುಳಿ-ಕ-ಕಾಲ
ಗುಳಿಗೆ
ಗುಳಿ-ಗೆ-ಗ-ಳನ್ನು
ಗುಳ್ಳೆ-ಗ-ಳಂತೆ
ಗುಳ್ಳೆಯಂತೆ
ಗುವ
ಗುಹೆ
ಗುಹೆಯೇ
ಗುಹ್ಯ-ರೋ-ಗ-ಗಳು
ಗುಹ್ಯ-ರೋ-ಗದ
ಗೂಂಡಾ-ಗ-ಳಿಂದ
ಗೂಟ
ಗೂಟಕ್ಕೆ
ಗೂಟ-ಗ-ಳನ್ನು
ಗೂಟದ
ಗೂಟವನ್ನು
ಗೂಡದಂತೆ
ಗೂಡಿ-ನಂತೆಯೇ
ಗೂಡಿನಂಥ
ಗೂಡಿ-ಸಿ-ದರು
ಗೂಡು-ಕಟ್ಟಿ-ಕೊಳ್ಳು-ವಂತೆ
ಗೂಡುವುದು
ಗೂಢಃ
ಗೂಢತತ್ವ
ಗೂಢವಾದ
ಗೂರಲು
ಗೂಳಿ
ಗೂಳಿಯನ್ನು
ಗೂಳಿಯು
ಗೂಳಿಯೊಂದು
ಗೃಹ
ಗೃಹ-ಕೃತ್ಯದ
ಗೃಹಕ್ಕೆ
ಗೃಹಗಳ
ಗೃಹದ
ಗೃಹಸ್ಥ
ಗೃಹಸ್ಥ-ರನ್ನೂ
ಗೃಹಸ್ಥ-ರಿಗೂ
ಗೃಹಸ್ಥರೂ
ಗೃಹಿಣಿ
ಗೃಹಿಣಿಯ
ಗೆ
ಗೆಟ್ಟ
ಗೆಣಸನ್ನು
ಗೆದ್ದ
ಗೆದ್ದಂತೆ
ಗೆದ್ದರೂ
ಗೆದ್ದಲು
ಗೆದ್ದವಳು
ಗೆದ್ದು
ಗೆದ್ದುಕೊಂಡ
ಗೆದ್ದುಬಿಟ್ಟ
ಗೆಲವಿನ
ಗೆಲವು
ಗೆಲಿಲಿಯೋ
ಗೆಲುವನ್ನು
ಗೆಲುವಿನ
ಗೆಲು-ವೆನ್ನದೆ
ಗೆಲ್ಲ
ಗೆಲ್ಲ-ಬ-ಹುದು
ಗೆಲ್ಲ-ಬೇ-ಕೆಂದು
ಗೆಲ್ಲಲು
ಗೆಲ್ಲಿ
ಗೆಲ್ಲಿರಿ
ಗೆಲ್ಲು
ಗೆಲ್ಲುವುದು
ಗೆಳ-ತಿ-ಯಾ-ಗಲಿ
ಗೆಳೆಯ
ಗೆಳೆಯನ
ಗೆಳೆಯರ
ಗೆಳೆ-ಯ-ರನ್ನು
ಗೆಳೆಯರಾ
ಗೆಳೆಯರು
ಗೆಳೆ-ಯ-ರೊಂದಿಗೆ
ಗೇಟಿನಿಂದ
ಗೇರುಎಲ್ಲ
ಗೇಲಿ
ಗೈಯಲು
ಗೈಯುವ
ಗೈರು-ಹಾ-ಜ-ರಾ-ಗಲು
ಗೈರು-ಹಾ-ಜ-ರಾ-ಗಿದ್ದರೆ
ಗೊಂಡ
ಗೊಂದಲ
ಗೊಂದಲಕ್ಕೆ
ಗೊಂದ-ಲಕ್ಕೊ-ಳ-ಗಾ-ಗಿದ್ದಾರೆ
ಗೊಂದ-ಲ-ಗಳ
ಗೊಂದ-ಲ-ಗ-ಳನ್ನು
ಗೊಂದ-ಲ-ಗ-ಳಿಂದ
ಗೊಂದ-ಲ-ಗ-ಳಿಗೆ
ಗೊಂದ-ಲ-ಗ-ಳಿ-ಗೇನು
ಗೊಂದ-ಲ-ಗ-ಳಿಲ್ಲದೆ
ಗೊಂದ-ಲ-ಗಳು
ಗೊಂದಲದ
ಗೊಂದ-ಲ-ದಲ್ಲಿ
ಗೊಂದ-ಲ-ವನ್ನು
ಗೊಂದ-ಲ-ವೆ-ನಿ-ಸು-ವುದು
ಗೊಂದ-ಲ-ವೆಬ್ಬಿಸಿ
ಗೊಂದಾ-ವ-ಳೇ-ಕರ್
ಗೊಂದು
ಗೊಂದು-ಗ-ಳಲ್ಲಿ-ರುವ
ಗೊಡ್ಡು
ಗೊಡ್ಡುಕಂತೆ
ಗೊಣಗಾಟ
ಗೊಣ-ಗುಟ್ಟುತ್ತ
ಗೊಣಗುತ್ತ
ಗೊಣ-ಗುತ್ತಲೇ
ಗೊತ್ತಾಗ
ಗೊತ್ತಾ-ಗ-ಲಿಲ್ಲ
ಗೊತ್ತಾ-ಗಿ-ರ-ಬ-ಹು-ದಲ್ಲವೇ
ಗೊತ್ತಾ-ಗಿ-ರುವ
ಗೊತ್ತಾ-ಗುತ್ತದೆ
ಗೊತ್ತಾ-ಗು-ವುದು
ಗೊತ್ತಾದ
ಗೊತ್ತಾಯಿತು
ಗೊತ್ತಿತ್ತು
ಗೊತ್ತಿತ್ತು-ಮಂತ್ರ
ಗೊತ್ತಿದೆ
ಗೊತ್ತಿದ್ದ
ಗೊತ್ತಿದ್ದರೂ
ಗೊತ್ತಿದ್ದರೆ
ಗೊತ್ತಿದ್ದೂ
ಗೊತ್ತಿ-ರ-ದಂತೆ
ಗೊತ್ತಿರಲಿ
ಗೊತ್ತಿರುವ
ಗೊತ್ತಿ-ರು-ವುದು
ಗೊತ್ತಿಲ್ಲ
ಗೊತ್ತಿಲ್ಲದ
ಗೊತ್ತಿಲ್ಲ-ವಲ್ಲ
ಗೊತ್ತು
ಗೊತ್ತು-ಪ-ಡಿ-ಸಿ-ಕೊಂಡ
ಗೊತ್ತು-ಪ-ಡಿ-ಸಿ-ಕೊಂಡ
ಗೊತ್ತು-ಪ-ಡಿ-ಸಿ-ಕೊಂಡಿದ್ದರು
ಗೊತ್ತು-ಪ-ಡಿ-ಸಿ-ಕೊಂಡು
ಗೊತ್ತು-ಪ-ಡಿ-ಸಿದ
ಗೊತ್ತೆ
ಗೊತ್ತೇ
ಗೊತ್ತೇ-ನು-ನಮ್ಮ
ಗೊಬ್ಬರ
ಗೊಬ್ಬ-ರ-ವನ್ನು
ಗೊರಕೆ
ಗೊರಲಿನ
ಗೊರೂರು
ಗೊಲ್ಲ
ಗೊಲ್ಲನ
ಗೊಲ್ಲರ
ಗೊಳ-ಗಾ-ದ-ವ-ರಲ್ಲೆಲ್ಲ
ಗೊಳ-ಪ-ಡಿಸಿ
ಗೊಳ-ಪ-ಡಿ-ಸಿದೆ
ಗೊಳಿಸಲು
ಗೊಳಿಸಿ
ಗೊಳಿ-ಸುತ್ತಾನೆ
ಗೊಳಿಸುವ
ಗೊಳಿ-ಸು-ವಂತೆ
ಗೊಳಿ-ಸು-ವುದು
ಗೊಳೋ
ಗೊಳ್ಳಲು
ಗೊಳ್ಳೆಂದು
ಗೋಂದಾವಳಿ
ಗೋಗರೆದ
ಗೋಗ-ರೆ-ಯುತ್ತ
ಗೋಗ-ರೆ-ಯುತ್ತಿದ್ದರೆ
ಗೋಗ-ರೆ-ಯುತ್ತಿದ್ದೆ
ಗೋಚ-ರ-ವಾ-ಗುತ್ತಿತ್ತು
ಗೋಚ-ರ-ವಲ್ಲ
ಗೋಚ-ರ-ವಾ-ಗದು
ಗೋಚ-ರ-ವಾ-ಗುತ್ತದೆ
ಗೋಚ-ರ-ವಾ-ಗುತ್ತ-ದೆ-ಯಷ್ಟೆ
ಗೋಚ-ರ-ವಾ-ಗುವ
ಗೋಚ-ರ-ವಾ-ಗು-ವು-ದಕ್ಕಾಗಿ
ಗೋಚ-ರ-ವಾದ
ಗೋಚ-ರ-ವಾ-ಯಿತು
ಗೋಚ-ರಿ-ಸದ
ಗೋಚ-ರಿ-ಸದು
ಗೋಚ-ರಿ-ಸದೇ
ಗೋಚರಿಸಿ
ಗೋಚ-ರಿ-ಸಿತು
ಗೋಚ-ರಿ-ಸಿತೋ
ಗೋಚ-ರಿ-ಸಿದ
ಗೋಚ-ರಿ-ಸಿ-ದರೂ
ಗೋಚ-ರಿ-ಸಿದೆ
ಗೋಚ-ರಿ-ಸುತ್ತದೆ
ಗೋಚ-ರಿ-ಸುವ
ಗೋಚ-ರಿ-ಸು-ವು-ದಿಲ್ಲ
ಗೋಚ-ರಿ-ಸು-ವುದು
ಗೋಚ-ರಿ-ಸು-ವುದೇ
ಗೋಜಿಗೆ
ಗೋಜಿಗೇ
ಗೋಟುವಾದ್ಯ
ಗೋಡೆ
ಗೋಡೆಗೆ
ಗೋಡೆಯ
ಗೋಡೆಯನ್ನು
ಗೋತ್ರ
ಗೋತ್ರ-ಗ-ಳನ್ನು
ಗೋಪಾ-ಲ-ದಾ-ಸರ
ಗೋಪಾ-ಲ-ದಾ-ಸ-ರಲ್ಲಿಗೆ
ಗೋಪಾ-ಲ-ದಾ-ಸರು
ಗೋಪಾಲನ
ಗೋಪು-ರ-ದಂತೆ
ಗೋಪೂ-ಜೆ-ಯನ್ನು
ಗೋಪ್ತೃತ್ವ-ವ-ರಣಂ
ಗೋಮು-ಖವ್ಯಾಘ್ರ-ರಂತೆ
ಗೋಳನ್ನು
ಗೋಳನ್ನೂ
ಗೋಳ-ಬಾ-ಳಲಿ
ಗೋಳಿಡುತ್ತ
ಗೋಳು
ಗೋವನ್ನು
ಗೋವಾದಲ್ಲಿ
ಗೋಶಾ-ಲೆ-ಯಲ್ಲಿ-ರಲಿ
ಗೌಜು
ಗೌಜು-ಗೊಂದಲ
ಗೌಣಸ್ಥಾ-ನ-ವಿ-ರಲಿ
ಗೌತಮ
ಗೌತಮನು
ಗೌತ-ಮ-ಬದ್ಧ
ಗೌತ-ಮ-ಬುದ್ಧ
ಗೌರವ
ಗೌರವಕ್ಕೆ
ಗೌರವಕ್ಕೇ
ಗೌರ-ವ-ಗ-ಳನ್ನು
ಗೌರ-ವ-ಗ-ಳಿಂದ
ಗೌರ-ವ-ಗ-ಳಿಂದಲೇ
ಗೌರ-ವ-ಗ-ಳಿಗೆ
ಗೌರ-ವ-ಗಳು
ಗೌರ-ವ-ಗಳೂ
ಗೌರವದ
ಗೌರ-ವ-ದಿಂದ
ಗೌರ-ವ-ಪೂರ್ವ-ಕ-ವಾದ
ಗೌರ-ವ-ಮಿಶ್ರಿತ
ಗೌರ-ವ-ವನ್ನು
ಗೌರ-ವ-ವನ್ನೂ
ಗೌರ-ವ-ವಲ್ಲವೇ
ಗೌರ-ವ-ವಿದ್ದರೂ
ಗೌರ-ವಸ್ಥಾನ
ಗೌರ-ವಸ್ಥಾ-ನ-ವನ್ನು
ಗೌರ-ವಾ-ದ-ರ-ಗ-ಳನ್ನು
ಗೌರ-ವಾ-ದ-ರ-ಗ-ಳಿಂದ
ಗೌರ-ವಾ-ದ-ರ-ಗಳು
ಗೌರವಾರ್ಹ
ಗೌರ-ವಿ-ಸದ
ಗೌರ-ವಿ-ಸ-ದಿದ್ದರೆ
ಗೌರ-ವಿ-ಸ-ಬ-ಹುದು
ಗೌರವಿಸಿ
ಗೌರ-ವಿ-ಸಿತು
ಗೌರ-ವಿ-ಸಿದ
ಗೌರ-ವಿ-ಸಿ-ದರೆ
ಗೌರವಿಸು
ಗೌರ-ವಿ-ಸುವ
ಗೌರೀ-ಶಂಕ-ರ-ದಷ್ಟು
ಗ್ರಂಥ
ಗ್ರಂಥಗ್ರಂಥ-ಕರ್ತರ
ಗ್ರಂಥ-ಕರ್ತನ
ಗ್ರಂಥ-ಕರ್ತ-ನಾಗಿ
ಗ್ರಂಥ-ಕರ್ತ-ನಿದ್ದಾನೆ
ಗ್ರಂಥ-ಕರ್ತ-ನೊಬ್ಬ-ನಿ-ರು-ವ-ನೆಂಬು-ದಕ್ಕೆ
ಗ್ರಂಥ-ಕರ್ತರ
ಗ್ರಂಥ-ಕರ್ತರು
ಗ್ರಂಥ-ಕರ್ತ-ರೊಬ್ಬರು
ಗ್ರಂಥಗಳ
ಗ್ರಂಥ-ಗ-ಳನ್ನು
ಗ್ರಂಥ-ಗ-ಳನ್ನೂ
ಗ್ರಂಥ-ಗ-ಳನ್ನೇನೋ
ಗ್ರಂಥ-ಗ-ಳಲ್ಲಿ
ಗ್ರಂಥ-ಗ-ಳಾ-ಗಲಿ
ಗ್ರಂಥ-ಗ-ಳಾ-ಗುತ್ತ-ವೆಯೇ
ಗ್ರಂಥ-ಗ-ಳಿಂದ
ಗ್ರಂಥ-ಗ-ಳಿಗೆ
ಗ್ರಂಥಗಳು
ಗ್ರಂಥಗಳೂ
ಗ್ರಂಥ-ಗ-ಳೆಲ್ಲ
ಗ್ರಂಥದ
ಗ್ರಂಥದಲ್ಲಿ
ಗ್ರಂಥ-ಪಾ-ಲ-ರನ್ನು
ಗ್ರಂಥ-ರ-ಚ-ನೆಗೆ
ಗ್ರಂಥ-ರ-ಚ-ನೆ-ಯಲ್ಲಿ
ಗ್ರಂಥವನ್ನು
ಗ್ರಂಥವನ್ನೇ
ಗ್ರಂಥವಾದ
ಗ್ರಂಥವು
ಗ್ರಂಥಾಲಯ
ಗ್ರಂಥಿಗಳ
ಗ್ರಂಥಿಗಳ
ಗ್ರಂಥಿ-ಗ-ಳನ್ನು
ಗ್ರಂಥಿ-ಗ-ಳನ್ನೂ
ಗ್ರಹ
ಗ್ರಹಗತಿ
ಗ್ರಹ-ಗ-ಳಿಗೆ
ಗ್ರಹಚಾರ
ಗ್ರಹಣ
ಗ್ರಹಿ-ಸ-ಕೂ-ಡದು
ಗ್ರಹಿ-ಸ-ದಿದ್ದರೆ
ಗ್ರಹಿ-ಸ-ದಿದ್ದಾಗ
ಗ್ರಹಿ-ಸ-ಬಲ್ಲ-ವರು
ಗ್ರಹಿ-ಸ-ಬ-ಹುದು
ಗ್ರಹಿ-ಸ-ಲ-ಸಾಧ್ಯ-ವಾದ
ಗ್ರಹಿಸಿ
ಗ್ರಹಿ-ಸಿ-ಕೊಂಡು
ಗ್ರಹಿ-ಸಿ-ದರೆ
ಗ್ರಹಿ-ಸಿ-ದ-ವನ
ಗ್ರಹಿ-ಸುತ್ತೇವೆ
ಗ್ರಹಿಸುವ
ಗ್ರಹಿ-ಸು-ವುದು
ಗ್ರಾಫಿನ
ಗ್ರಾಮ-ಗ-ಳಲ್ಲಿ
ಗ್ರಾಮದ
ಗ್ರಾಮ-ದ-ವನು
ಗ್ರಾಮಿನಷ್ಟು
ಗ್ರಾಮೀಣ
ಗ್ರಾಹ-ಕ-ಫ-ಲ-ಕ-ದಂತೆ
ಗ್ರಾಹ್ಯ-ವಾ-ಗುತ್ತ-ವೆಂದು
ಗ್ರೀಕ್
ಗ್ರೀಸ್
ಗ್ರೆಹ್ಯಾಮ್
ಗ್ರೇಟ್
ಗ್ರ್ಯಾಜು-ಯಲ್ನೆಸ್
ಗ್ಲಾಡ್ಮನ್
ಗ್ಲಾಡ್ಸ್ಟೋನ್
ಗ್ಲೇಡಿಸ್
ಘಂಟಾ-ಘೋ-ಷ-ವಾಗಿ
ಘಂಟೆಗಳ
ಘಂಟೆ-ಗ-ಳಷ್ಟು
ಘಂಟೆಯ
ಘಂಟೆಯಲ್ಲಿ
ಘಟ-ಕ-ನಾ-ಗುತ್ತಿದ್ದಾ-ನೆಯೆ
ಘಟ-ನಾ-ವಳಿ
ಘಟ-ನಾ-ವ-ಳಿ-ಗಳ
ಘಟ-ನಾ-ವ-ಳಿ-ಗ-ಳನ್ನು
ಘಟನೆ
ಘಟ-ನೆ-ಗಳ
ಘಟ-ನೆ-ಗ-ಳನ್ನು
ಘಟ-ನೆ-ಗ-ಳನ್ನೂ
ಘಟ-ನೆ-ಗ-ಳನ್ನೇ
ಘಟ-ನೆ-ಗ-ಳಲ್ಲಿ
ಘಟ-ನೆ-ಗ-ಳಿಗೆ
ಘಟ-ನೆ-ಗ-ಳಿ-ಗೊಂದು
ಘಟ-ನೆ-ಗ-ಳಿದ್ದರೂ
ಘಟ-ನೆ-ಗ-ಳಿ-ರು-ವುದು
ಘಟ-ನೆ-ಗ-ಳಿವೆ
ಘಟ-ನೆ-ಗಳು
ಘಟ-ನೆ-ಗ-ಳು-ಇಲ್ಲಿ
ಘಟ-ನೆ-ಗಳೂ
ಘಟ-ನೆ-ಗಳೇ
ಘಟನೆಗೆ
ಘಟನೆಗೇ
ಘಟನೆಯ
ಘಟ-ನೆ-ಯನ್ನು
ಘಟ-ನೆ-ಯಲ್ಲಿ
ಘಟ-ನೆ-ಯಾ-ದರೋ
ಘಟ-ನೆ-ಯಾ-ದು-ದ-ರಿಂದ
ಘಟ-ನೆ-ಯಿಂದ
ಘಟ-ನೆ-ಯಿ-ದು-ಎಂಟು
ಘಟನೆಯೂ
ಘಟ-ನೆ-ಯೊಂದನ್ನು
ಘಟ-ನೆ-ಯೊಂದು
ಘಟಿ-ತ-ವಾದ
ಘಟಿ-ಸ-ಲಿಲ್ಲ-ಎಂಬು-ದನ್ನು
ಘಟಿಸಿ
ಘಟಿಸಿದ
ಘಟ್ಟ
ಘನತೆ
ಘನ-ಪ-ದಾರ್ಥ-ವಾ-ಗಿತ್ತೆಂದು
ಘನೀ-ಭೂ-ತ-ವಾ-ಗು-ವುದು
ಘರ್ಷಣೆ
ಘರ್ಷ-ಣೆ-ಗ-ಳನ್ನೂ
ಘರ್ಷಣೆಗೆ
ಘರ್ಷಣೆಯ
ಘಳಿ-ಗೆ-ಗಳೂ
ಘಾಟಿನಲ್ಲಿ
ಘಾತು-ಕ-ವಾ-ಗುತ್ತ-ದೆಂಬುದು
ಘೋರ
ಘೋರ-ಕೃತ್ಯ-ದಿಂದ
ಘೋರಾಂಧ-ಕಾ-ರ-ವಿದೆ
ಘೋಷ-ಣೆ-ಗ-ಳಾದ
ಘೋಷಣೆಗೆ
ಘೋಷವನ್ನು
ಘೋಷಿ-ಸಲ್ಪ-ಡು-ವುದು
ಘೋಷ್
ಚ
ಚಂಚಲ
ಚಂಚಲತೆ
ಚಂಚ-ಲ-ತೆ-ಯನ್ನು
ಚಂಚ-ಲ-ನಾಗಿ
ಚಂಚ-ಲ-ರನ್ನಾಗಿ
ಚಂಚ-ಲ-ವಾ-ಗು-ವು-ದುಂಟು
ಚಂಚ-ಲ-ವಾದ
ಚಂದ-ನೇಶ್ವರ
ಚಂದ್ರ
ಚಂದ್ರ-ಗುಪ್ತನ
ಚಂದ್ರಗ್ರ-ಹದ
ಚಂದ್ರಗ್ರ-ಹ-ವನ್ನು
ಚಂದ್ರನೂ
ಚಂದ್ರನೇ
ಚಂದ್ರಬಿಂಬ
ಚಂದ್ರಭಾಗಾ
ಚಂದ್ರ-ಶೇ-ಖರ
ಚಕ-ಮ-ಕಿಯ
ಚಕಿ-ತ-ರಾ-ದರು
ಚಕಿ-ತ-ಗೊ-ಳಿ-ಸಿದ
ಚಕಿ-ತ-ಗೊ-ಳಿ-ಸು-ವಂಥ
ಚಕಿ-ತ-ನಾಗಿ
ಚಕಿ-ತ-ನಾ-ಗಿದ್ದೆ
ಚಕಿ-ತ-ನಾ-ಗುತ್ತಾನೆ
ಚಕಿ-ತ-ನಾದ
ಚಕಿ-ತ-ನಾ-ದರೆ
ಚಕಿತನೂ
ಚಕಿ-ತ-ರಾ-ಗ-ಬ-ಹುದು
ಚಕಿ-ತ-ರಾಗಿ
ಚಕಿ-ತ-ರಾ-ಗು-ವುದು
ಚಕಿ-ತ-ರಾದ
ಚಕಿ-ತ-ರಾ-ದರು
ಚಕಿ-ತ-ರಾ-ದರೂ
ಚಕಿತರೂ
ಚಕ್ರ
ಚಕ್ರಕ್ಕೆ
ಚಕ್ರಗಳು
ಚಕ್ರದ
ಚಕ್ರದಲ್ಲಿ
ಚಕ್ರ-ಬಂಧಸ್ಪರ್ಧೆ-ಯಿಂದ
ಚಕ್ರಬಡ್ಡಿ
ಚಕ್ರ-ವ-ನಂತ
ಚಕ್ರವನ್ನು
ಚಕ್ರವರ್ತಿ
ಚಕ್ರ-ವರ್ತಿಯ
ಚಕ್ರ-ವರ್ತಿ-ಯನ್ನು
ಚಕ್ರ-ವರ್ತಿ-ಯಾದ
ಚಕ್ರಾ-ಧಿ-ಪತ್ಯ
ಚಕ್ರಾ-ಧಿ-ಪತ್ಯದ
ಚಕ್ರಾ-ಧಿ-ಪತ್ಯ-ವನ್ನು
ಚಚ್ಚಿ
ಚಟ
ಚಟಕ್ಕೆ
ಚಟ-ಗ-ಳನ್ನು
ಚಟದ
ಚಟದಿಂದ
ಚಟವನ್ನು
ಚಟ-ವಾ-ಗಲಿ
ಚಟವಾಗಿ
ಚಟ-ವಾ-ಗಿ-ರು-ವುದು
ಚಟಾಕಿ
ಚಟಾ-ಕಿ-ಗ-ಳಿಂದ
ಚಟಾ-ಕಿ-ಯನ್ನು
ಚಟು-ವ-ಟಿಕೆ
ಚಟು-ವ-ಟಿ-ಕೆ-ಗಳ
ಚಟು-ವ-ಟಿ-ಕೆ-ಗ-ಳನ್ನು
ಚಟು-ವ-ಟಿ-ಕೆ-ಗ-ಳನ್ನೂ
ಚಟು-ವ-ಟಿ-ಕೆ-ಗ-ಳನ್ನೆಲ್ಲ
ಚಟು-ವ-ಟಿ-ಕೆ-ಗ-ಳಲ್ಲಿ
ಚಟು-ವ-ಟಿ-ಕೆ-ಗ-ಳಿಂದ
ಚಟು-ವ-ಟಿ-ಕೆ-ಗ-ಳಿಗೂ
ಚಟು-ವ-ಟಿ-ಕೆ-ಗ-ಳಿಗೆ
ಚಟು-ವ-ಟಿ-ಕೆ-ಗಳು
ಚಟು-ವ-ಟಿ-ಕೆ-ಗಳೂ
ಚಟು-ವ-ಟಿ-ಕೆ-ಗ-ಳೆಲ್ಲ
ಚಟು-ವ-ಟಿ-ಕೆಯ
ಚಟು-ವ-ಟಿ-ಕೆ-ಯನ್ನು
ಚಟು-ವ-ಟಿ-ಕೆ-ಯಾ-ಗಲಿ
ಚಟು-ವ-ಟಿ-ಕೆ-ಯಾಗಿ
ಚಟು-ವ-ಟಿ-ಕೆ-ಯಿಂದ
ಚಟು-ವ-ಟಿ-ಕೆ-ಯಿಂದಿದ್ದಾಗ
ಚಟು-ವ-ಟಿ-ಕೆ-ಯಿಂದಿ-ರುತ್ತಾರೆ
ಚಡ-ಪ-ಡಿಕೆ
ಚಡ-ಪ-ಡಿ-ಕೆ-ಇ-ವು-ಗಳು
ಚಡ-ಪ-ಡಿ-ಸಿದ
ಚಡ-ಪ-ಡಿ-ಸುತ್ತಿದ್ದ
ಚಡ-ಪ-ಡಿ-ಸು-ವರು
ಚತುರ
ಚತು-ರ-ನಾ-ಗಿದ್ದರೂ
ಚತುರೋಕ್ತಿ
ಚತುರ್ಭುಜ
ಚಪರಾಸಿ
ಚಪ-ರಾ-ಸಿನಾ
ಚಪಲ
ಚಪ-ಲ-ತೆ-ಯಿಂದಾ-ಗಲೀ
ಚಪ-ಲ-ತೆ-ಗ-ಳಿಂದ
ಚಪ-ಲ-ತೆ-ಗ-ಳಿಗೆ
ಚಪ್ಪಟೆ
ಚಪ್ಪ-ರಿ-ಸಿತು
ಚಪ್ಪ-ರಿ-ಸುತ್ತ
ಚಪ್ಪಾಳೆ
ಚಮ-ಚೆ-ಗ-ಳನ್ನು
ಚಮತ್ಕಾ-ರಕ್ಕೇ
ಚಮತ್ಕಾ-ರದ
ಚಮತ್ಕಾ-ರ-ವನ್ನು
ಚರಂಡಿಗೆ
ಚರಂಡಿಯ
ಚರಂಡಿ-ಯಲ್ಲಿ
ಚರ-ಣ-ಗ-ಳಲ್ಲಿ
ಚರ-ಣ-ವನ್ನು
ಚರಮ
ಚರ-ಮ-ಸಂದೇಶ
ಚರ-ಮ-ಸೀಮೆ
ಚರವಾದ
ಚರಾಚರ
ಚರಾ-ಚ-ರ-ಭೂ-ತ-ಗಳ
ಚರಿ-ತ-ರಾದ
ಚರಿತ್ರೆ
ಚರಿತ್ರೆಯ
ಚರಿತ್ರೆ-ಯಲ್ಲಿ
ಚರ್ಚಿಗೆ
ಚರ್ಚಿನಲ್ಲಿ
ಚರ್ಚಿಲ್ರು
ಚರ್ಚಿಸಿ
ಚರ್ಚಿ-ಸಿದ್ದಾರೆ
ಚರ್ಚಿಸುತ್ತ
ಚರ್ಚಿ-ಸುತ್ತಿದ್ದಾಗ
ಚರ್ಚಿ-ಸು-ವುದು
ಚರ್ಚು-ಗ-ಳನ್ನು
ಚರ್ಚು-ಗ-ಳಲ್ಲಿ
ಚರ್ಚೆ
ಚರ್ಚೆಯನ್ನು
ಚರ್ಚೆಯಲ್ಲಿ
ಚರ್ಚೆಯಿಂದ
ಚರ್ಚೊಂದನ್ನು
ಚರ್ಚೊಂದರ
ಚರ್ಚೊಂದ-ರಲ್ಲಿ
ಚರ್ಚ್
ಚರ್ಮ
ಚರ್ಮವನ್ನು
ಚಲ
ಚಲಚ್ಚಿತ್ರ
ಚಲಚ್ಚಿತ್ರ-ಗ-ಳಲ್ಲಿ
ಚಲಚ್ಚಿತ್ರ-ಗಳು
ಚಲತಿ
ಚಲ-ನ-ಚಿತ್ರ
ಚಲ-ನ-ಚಿತ್ರ-ದಲ್ಲಿ
ಚಲ-ನ-ವ-ಲ-ನ-ಗ-ಳನ್ನು
ಚಲ-ನ-ವ-ಲ-ನ-ಗ-ಳಲ್ಲಿ
ಚಲ-ನ-ವ-ಲ-ನ-ವೆಲ್ಲ
ಚಲ-ನ-ಹೀ-ನ-ವಾ-ಗಿ-ಬಿ-ಡುತ್ತದೆ
ಚಲನೆ
ಚಲನೆಗೆ
ಚಲನೆಯ
ಚಲಾ-ಯಿ-ಸ-ತೊ-ಡ-ಗುತ್ತಾರೆ
ಚಲಾ-ಯಿ-ಸಲೂ
ಚಲಾ-ಯಿ-ಸುವ
ಚಲಿಸಿ
ಚಲಿ-ಸುತ್ತವೆ
ಚಲಿ-ಸುತ್ತಿ-ರುವ
ಚಲಿ-ಸುತ್ತಿಲ್ಲ
ಚಲಿ-ಸು-ವಂತೆ
ಚಲಿ-ಸು-ವುದು
ಚಳ-ವ-ಳಿಗೆ
ಚಳ-ವ-ಳಿಯ
ಚಳಿ-ಗಾ-ಲದ
ಚಳಿ-ಗಾ-ಲ-ದಲ್ಲಿ
ಚಳಿ-ಗಾ-ಲ-ವಾ-ಗಿದ್ದರೂ
ಚಳುವಳಿ
ಚಾಂಚಲ್ಯ-ವಾ-ಗಲಿ
ಚಾಂಪಿಯನ್
ಚಾಕ-ಚಕ್ಯ-ತೆ-ಯಿಂದ
ಚಾಕರಿ
ಚಾಕ-ರಿ-ಗಾಗಿ
ಚಾಕರಿಯ
ಚಾಕು
ಚಾಕು-ವಿ-ನಿಂದ
ಚಾಕು-ವಿ-ನಿಂದಲೇ
ಚಾಚುವಂತೆ
ಚಾಚೂ
ಚಾಡಿ
ಚಾತುರ್ಯ
ಚಾತುರ್ಯಕ್ಕೆ
ಚಾತುರ್ಯ-ದಿಂದ
ಚಾನ್ಸ್
ಚಾಪಲ್ಯ-ದಿಂದ
ಚಾಪಲ್ಯ-ವನ್ನೂ
ಚಾರ
ಚಾರಿತ್ರಿಕ
ಚಾರಿತ್ರ್ಯ
ಚಾರಿತ್ರ್ಯಕ್ಕೂ
ಚಾರಿತ್ರ್ಯ-ಗಳ
ಚಾರಿತ್ರ್ಯದ
ಚಾರಿತ್ರ್ಯ-ದಿಂದ
ಚಾರಿತ್ರ್ಯ-ಬಲ
ಚಾರಿತ್ರ್ಯ-ವಂತ
ಚಾರಿತ್ರ್ಯ-ವಂತ-ರನ್ನಾಗಿ
ಚಾರಿತ್ರ್ಯ-ವನ್ನು
ಚಾರಿತ್ರ್ಯ-ವನ್ನೇ
ಚಾರಿತ್ರ್ಯ-ಶುದ್ಧಿಯ
ಚಾರಿತ್ರ್ಯ-ಹ-ನನ
ಚಾರು-ವಾ-ಕರ
ಚಾರ್ಡಿನ್
ಚಾರ್ಲ್ಸ್
ಚಾಲಕ
ಚಾಲ-ಕ-ನಾ-ಗ-ಬೇ-ಕೆಂದು
ಚಾಲನ
ಚಾಲನೆ
ಚಾಳಿ
ಚಿಂಕ್ರೋಭ
ಚಿಂಕ್ರೋ-ಭಕ್ಕಿದೆ
ಚಿಂಕ್ರೋಭಕ್ಕೆ
ಚಿಂಕ್ರೋ-ಭ-ಗಳ
ಚಿಂಕ್ರೋ-ಭ-ಗ-ಳಿಂದ
ಚಿಂಕ್ರೋ-ಭ-ಗಳು
ಚಿಂಕ್ರೋಭದ
ಚಿಂಕ್ರೋ-ಭ-ದಿಂದ
ಚಿಂಕ್ರೋ-ಭ-ದಿಂದುಂಟಾ-ಗುವ
ಚಿಂಕ್ರೋ-ಭ-ವನ್ನು
ಚಿಂತಕರ
ಚಿಂತನ
ಚಿಂತ-ನ-ಮಂಥನ
ಚಿಂತ-ನ-ಮಂಥನ
ಚಿಂತ-ನ-ಮಂಥ-ನ-ಗ-ಳಿಂದ
ಚಿಂತ-ನ-ಶೀಲ
ಚಿಂತ-ನ-ಶೀ-ಲ-ನನ್ನಾ-ಗಿಯೂ
ಚಿಂತ-ನ-ಶೀ-ಲ-ನಾ-ಗುತ್ತಾನೆ
ಚಿಂತ-ನ-ಶೀ-ಲ-ರನ್ನಾಗಿ
ಚಿಂತ-ನ-ಶೀ-ಲವ್ಯಕ್ತಿ-ಗಳ
ಚಿಂತನೆ
ಚಿಂತ-ನೆ-ಗಳ
ಚಿಂತ-ನೆ-ಗಳು
ಚಿಂತ-ನೆ-ಗಳೂ
ಚಿಂತ-ನೆ-ಗಾಗಿ
ಚಿಂತನೆಗೆ
ಚಿಂತನೆಯ
ಚಿಂತ-ನೆ-ಯನ್ನು
ಚಿಂತ-ನೆ-ಯಲ್ಲೇ
ಚಿಂತನೆಯೆ
ಚಿಂತಾಕ್ರಾಂತನೂ
ಚಿಂತಾಕ್ರೋಭ
ಚಿಂತಾಕ್ರೋಶ
ಚಿಂತಾಕ್ರೋ-ಶ-ಭ-ಯೋದ್ವೇ-ಗ-ಗ-ಳಿಂದ
ಚಿಂತಾಕ್ರೋ-ಶ-ಭ-ಯೋದ್ವೇ-ಗ-ಗ-ಳಿಂದಲೇ
ಚಿಂತಾಕ್ರೋ-ಶ-ಭ-ಯೋದ್ವೇ-ಗ-ಗಳು
ಚಿಂತಾಕ್ರೋ-ಶ-ಭ-ಯೋದ್ವೇ-ಗವು
ಚಿಂತಾ-ಜ-ನ-ಕ-ವಾ-ಗಿದೆ
ಚಿಂತಾ-ಮ-ಣಿ-ಯನ್ನು
ಚಿಂತಾ-ಶೀ-ಲ-ರಾ-ಗುತ್ತಾರೆ
ಚಿಂತಿ-ತ-ರಾ-ಗ-ಬೇ-ಕಿಲ್ಲ
ಚಿಂತಿ-ತ-ರಾ-ದರು
ಚಿಂತಿಸ
ಚಿಂತಿ-ಸ-ತೊ-ಡ-ಗು-ವುದು
ಚಿಂತಿ-ಸ-ದಿರು
ಚಿಂತಿಸದೆ
ಚಿಂತಿ-ಸ-ಬೇ-ಕಿ-ರ-ಲಿಲ್ಲ
ಚಿಂತಿ-ಸ-ಬೇಡ
ಚಿಂತಿಸಲಿ
ಚಿಂತಿಸಿ
ಚಿಂತಿಸಿದ
ಚಿಂತಿ-ಸಿ-ದರೆ
ಚಿಂತಿ-ಸಿ-ದಾಗ
ಚಿಂತಿ-ಸಿ-ದಾ-ಗ-ಲೆಲ್ಲ
ಚಿಂತಿಸಿಲ್ಲ
ಚಿಂತಿಸುತ್ತ
ಚಿಂತಿ-ಸುತ್ತಿದ್ದರೆ
ಚಿಂತಿ-ಸುತ್ತಿದ್ದೆ
ಚಿಂತಿಸುವ
ಚಿಂತಿ-ಸು-ವು-ದಿಲ್ಲ
ಚಿಂತಿ-ಸು-ವುದೇ
ಚಿಂತೆ
ಚಿಂತೆಗಳ
ಚಿಂತೆ-ಗ-ಳಿಲ್ಲ
ಚಿಂತೆಗಿಂದೇ
ಚಿಂತೆ-ಗೀ-ಡು-ಮಾ-ಡಿ-ದಂತೆ
ಚಿಂತೆಗೊಂದು
ಚಿಂತೆ-ಗೊ-ಳ-ಗಾದ
ಚಿಂತೆ-ಭ-ಯೋದ್ವೇ-ಗ-ಗಳ
ಚಿಂತೆಯ
ಚಿಂತೆಯನ್ನು
ಚಿಂತೆಯಲ್ಲಿ
ಚಿಂತೆ-ಯಾ-ಗಿತ್ತು
ಚಿಂತೆಯಿಂದ
ಚಿಂತೆಯು
ಚಿಕಾಗೊ
ಚಿಕಾಗೋ
ಚಿಕಿತ್ಸ-ಕ-ಳಾದ
ಚಿಕಿತ್ಸಾ
ಚಿಕಿತ್ಸಾ-ಲ-ಯಕ್ಕೆ
ಚಿಕಿತ್ಸಾ-ಲ-ಯದ
ಚಿಕಿತ್ಸಾ-ಲ-ಯ-ದಲ್ಲಿದ್ದಾಗ
ಚಿಕಿತ್ಸಾ-ವಿ-ಧಾನ
ಚಿಕಿತ್ಸಾ-ವಿ-ಧಾ-ನ-ಗಳು
ಚಿಕಿತ್ಸಾ-ವಿ-ಧಾ-ನ-ವನ್ನೂ
ಚಿಕಿತ್ಸೆ
ಚಿಕಿತ್ಸೆ-ಗಳೂ
ಚಿಕಿತ್ಸೆ-ಗಾಗಿ
ಚಿಕಿತ್ಸೆಗೆ
ಚಿಕಿತ್ಸೆ-ಗೊ-ಳ-ಪ-ಡಿ-ಸುವ
ಚಿಕಿತ್ಸೆಯ
ಚಿಕಿತ್ಸೆ-ಯನ್ನು
ಚಿಕ್ಕ
ಚಿಕ್ಕಂದಿ-ನಲ್ಲಿ
ಚಿಕ್ಕಂದಿ-ನಲ್ಲಿಯೇ
ಚಿಕ್ಕಂದಿ-ನಲ್ಲೇ
ಚಿಕ್ಕಂದಿ-ನಿಂದಲೇ
ಚಿಕ್ಕಚಿಕ್ಕ
ಚಿಕ್ಕ-ತಂದೆಯ
ಚಿಕ್ಕದು
ಚಿಕ್ಕಪ್ಪ
ಚಿಕ್ಕ-ಮಕ್ಕ-ಳಲ್ಲಿಯೂ
ಚಿಕ್ಕ-ಮಟ್ಟ-ದಿಂದ
ಚಿಕ್ಕ-ವ-ರನ್ನು
ಚಿಕ್ಕ-ವ-ರಾ-ಗಿದ್ದಾಗ
ಚಿಕ್ಕವರು
ಚಿಕ್ಕ-ವ-ಳಾ-ಗಿದ್ದಾಗ
ಚಿಕ್ಕಾಸನ್ನೂ
ಚಿಗುರಲು
ಚಿಗುರಿ
ಚಿಗುರು
ಚಿಗು-ರು-ವಂತೆ
ಚಿಗುರೊಡೆ
ಚಿಗು-ರೊ-ಡೆದು
ಚಿತೆಯಿಂದ
ಚಿತೆಯು
ಚಿತ್
ಚಿತ್ತ
ಚಿತ್ತದಿಂದ
ಚಿತ್ತನೂ
ಚಿತ್ತ-ರಂಜನ
ಚಿತ್ತರಾಗಿ
ಚಿತ್ತವುಳ್ಳ
ಚಿತ್ತವೃತ್ತಿ
ಚಿತ್ತಶುದ್ಧಿ
ಚಿತ್ತ-ಶುದ್ಧಿ-ಯಿಂದ
ಚಿತ್ತ-ಶೋ-ಧ-ನೆ-ಯಿಂದ
ಚಿತ್ರ
ಚಿತ್ರಕಲೆ
ಚಿತ್ರಕೂಟ
ಚಿತ್ರಕ್ಕ-ನು-ಗು-ಣ-ವಾಗಿ
ಚಿತ್ರಕ್ಕೆ
ಚಿತ್ರಗಳ
ಚಿತ್ರ-ಗ-ಳನ್ನು
ಚಿತ್ರ-ಗ-ಳಲ್ಲಿ
ಚಿತ್ರ-ಗ-ಳಿಂದ
ಚಿತ್ರ-ಗ-ಳಿಗೆ
ಚಿತ್ರಗಳು
ಚಿತ್ರಗಳೇ
ಚಿತ್ರಗುಪ್ತ
ಚಿತ್ರ-ಗುಪ್ತನ
ಚಿತ್ರ-ಗುಪ್ತ-ನಿದ್ದಾನೆ
ಚಿತ್ರಣ
ಚಿತ್ರ-ಣ-ವನ್ನೂ
ಚಿತ್ರದ
ಚಿತ್ರದಲ್ಲಿ
ಚಿತ್ರದಿಂದ
ಚಿತ್ರ-ದೊಂದಿಗೆ
ಚಿತ್ರ-ರೂ-ಪ-ದಲ್ಲಿ
ಚಿತ್ರವನ್ನು
ಚಿತ್ರವನ್ನೂ
ಚಿತ್ರ-ವನ್ನೇಕೆ
ಚಿತ್ರ-ವಿ-ಚಿತ್ರ-ರೂ-ಪ-ವನ್ನು
ಚಿತ್ರವು
ಚಿತ್ರವೇ
ಚಿತ್ರವೊಂದು
ಚಿತ್ರ-ಸೌ-ಧ-ಗ-ಳನ್ನು
ಚಿತ್ರಿಸಿ
ಚಿತ್ರಿ-ಸಿ-ಕೊಂಡಿ-ರುತ್ತೇವೋ
ಚಿತ್ರಿ-ಸಿ-ಕೊಂಡು
ಚಿತ್ರಿ-ಸಿ-ಕೊಳ್ಳ-ಬೇಕು
ಚಿತ್ರಿಸಿದ
ಚಿತ್ರಿ-ಸಿ-ರು-ವು-ದನ್ನು
ಚಿತ್ರಿಸುತ್ತ
ಚಿತ್ರಿ-ಸುತ್ತಿದ್ದಳು
ಚಿತ್ರಿಸುವ
ಚಿತ್ರಿ-ಸು-ವಲ್ಲಿ
ಚಿತ್ರೀ-ಕ-ರಿಸಿ
ಚಿನಿವಾರ
ಚಿನ್ನ
ಚಿನ್ನದ
ಚಿನ್ನ-ದಂತಾ-ಗು-ವುದು
ಚಿನ್ನ-ವಲ್ಲದೇ
ಚಿನ್ನಾ-ಭ-ರಣ
ಚಿನ್ಮಯ
ಚಿಮ್ಮಿಸಿ
ಚಿಮ್ಮುವ
ಚಿರಂತ-ನ-ವಾದ
ಚಿರ-ಜೀ-ವ-ನದ
ಚಿರ-ಪ-ರಿ-ಚಿ-ತ-ನಂತೆ
ಚಿರಸ್ಥಾ-ಯಿ-ಯಾ-ಗ-ಬಲ್ಲದು
ಚಿಲುಮೆ
ಚಿಲು-ಮೆ-ಯಂತೆ
ಚಿಲು-ಮೆ-ಯಾ-ಗಿ-ರ-ಬೇ-ಕಾ-ದರೆ
ಚಿಲು-ಮೆ-ಯಾದ
ಚಿಲ್ಲರೆ
ಚಿಲ್ಲ-ರೆ-ಯಾ-ಗು-ವು-ದೇನೋ
ಚಿಹ್ನೆ
ಚೀಕ್
ಚೀಟಿಯನ್ನು
ಚೀಟಿಯನ್ನೂ
ಚೀತ್ಕಾರ
ಚೀನ
ಚೀನಕ್ಕಾ-ಗಲೀ
ಚೀನಾ-ದೇ-ಶದ
ಚೀನಾ-ದೇ-ಶ-ದಿಂದ
ಚೀರಾ-ಟ-ಗಳ
ಚೀರಿ-ಕೊಳ್ಳುತ್ತಿದ್ದೆ
ಚೀಲ
ಚೀಲ-ದಲ್ಲಿಟ್ಟಿದ್ದ
ಚೀಲ-ದೊ-ಳಗೇ
ಚೀಲವನ್ನು
ಚುಕ್ಕಾಣಿ
ಚುಕ್ಕೆಗಳೂ
ಚುಚ್ಚದಂತೆ
ಚುಚ್ಚಿ
ಚುಚ್ಚಿ-ಕೊಂಡರೆ
ಚುಚ್ಚಿದ
ಚುಚ್ಚಿ-ದಂತಾದ
ಚುಚ್ಚಿದರು
ಚುಚ್ಚಿದರೂ
ಚುಚ್ಚುತ್ತಿತ್ತು
ಚುನಾವಣೆ
ಚುನಾ-ವ-ಣೆ-ಗ-ಳು-ಅ-ದ-ರಿಂದಾ-ಗುವ
ಚುನಾ-ವ-ಣೆಯ
ಚುನಾ-ವ-ಣೆ-ಯಲ್ಲಿ
ಚುರುಕಾಗಿ
ಚುರುಕು
ಚುರು-ಕು-ಗೊ-ಳಿಸಿ
ಚುರು-ಕು-ಗೊ-ಳಿಸು
ಚುರು-ಕು-ಗೊ-ಳಿ-ಸು-ವು-ದೆಂದರೆ
ಚುರು-ಕು-ತ-ನ-ದಿಂದ
ಚುರು-ಕು-ಬುದ್ಧಿಯ
ಚುರು-ಕು-ಬುದ್ಧಿ-ಯ-ವಳೂ
ಚೂಪಾ-ಗಿ-ಸುವ
ಚೂರನ್ನು
ಚೂರಾಗಿ
ಚೂರಾ-ಗುತ್ತಿದೆ
ಚೂರಿನ
ಚೆಂಡನ್ನು
ಚೆಂಡಿನಂತೆ
ಚೆಂಡು-ಗ-ಳನ್ನು
ಚೆಂಡುಗಳು
ಚೆಂಬರ್ಲಿನ್
ಚೆಕ್
ಚೆಕ್ಕು
ಚೆಕ್ಕುಇಲ್ಲಿ
ಚೆದುರಿ
ಚೆದುರಿತು
ಚೆನ್ನ
ಚೆನ್ನ-ಮಲ್ಲಿ-ಕಾರ್ಜುನ
ಚೆನ್ನಾಗಿ
ಚೆನ್ನಾಗಿದೆ
ಚೆನ್ನಾಗಿಯೂ
ಚೆನ್ನಾಗಿಯೆ
ಚೆನ್ನಾಗಿಯೇ
ಚೆನ್ನಾ-ಗಿ-ರು-ವಾಗ
ಚೆನ್ನಾಗಿಲ್ಲ
ಚೆನ್ನಾ-ಗಿಲ್ಲ-ವೆಂದಾ-ದರೆ
ಚೆಲು-ವೆ-ಯಾದ
ಚೆಲ್ಲಾಟಕ್ಕೆ
ಚೆಲ್ಲಾ-ಪಿಲ್ಲಿ-ಯಾಗಿ
ಚೆಲ್ಲಿದ
ಚೇಂಬರ್ಲಿನ್
ಚೇಂಬರ್ಲಿನ್ಗೆ
ಚೇತನ
ಚೇತ-ನ-ಗಳೇ
ಚೇತ-ನ-ಗೊ-ಳಿ-ಸ-ಲಾ-ರ-ದಷ್ಟು
ಚೇತ-ನ-ದಲ್ಲಿ
ಚೇತ-ನ-ವನ್ನು
ಚೇತನವು
ಚೇತರಿಕೆ
ಚೇತ-ರಿ-ಸಿ-ಕೊಂಡ
ಚೇತ-ರಿ-ಸಿ-ಕೊಳ್ಳು-ವ-ವ-ರಿ-ರ-ಬ-ಹುದು
ಚೇತ-ರಿ-ಸು-ವಂತೆ
ಚೇತಾ
ಚೇಷ್ಟೆ
ಚೈತನ್ಯ
ಚೈತನ್ಯ-ಮಯ
ಚೈತನ್ಯದ
ಚೈತನ್ಯ-ದಗ್ನಿ
ಚೈತನ್ಯ-ಮಯ
ಚೈತನ್ಯ-ಮ-ಯನು
ಚೈತನ್ಯರು
ಚೈತನ್ಯ-ವನ್ನು
ಚೈತನ್ಯ-ವಿದೆ
ಚೈತನ್ಯವು
ಚೈತನ್ಯ-ಶ-ರೀ-ರ-ವನ್ನು
ಚೈತನ್ಯ-ಶಾ-ಲಿ-ಯಾದ
ಚೈನಾ
ಚೈನು-ಗ-ಳನ್ನು
ಚೊಕ್ಕ
ಚೊಕ್ಕ-ಟ-ತ-ನ-ಗ-ಳನ್ನು
ಚೊಕ್ಕತನ
ಚೌಕಟ್ಟಿ-ನಲ್ಲಿ
ಚ್ಚಿತ್ರ
ಚ್ಯುತರಾದ
ಚ್ಯುತಿ
ಛಡಿ
ಛತ್ರ-ಗ-ಳನ್ನು
ಛತ್ರಪತಿ
ಛಲ
ಛಲಇಂಥ
ಛಲದಿಂದ
ಛಲ-ದೊಂದಿಗೆ
ಛಲವಿದೆ
ಛಲವೇ
ಛಾತಿ
ಛಾಯಾಗ್ರ-ಹ-ಣದ
ಛಾಯಾಚಿತ್ರ
ಛಾಯಾ-ಚಿತ್ರಕ್ಕಾಗಿ
ಛಾಯಾ-ಚಿತ್ರಕ್ಕೆ
ಛಾಯೆ
ಛಾವಣಿ
ಛಿದ್ರವಾದ
ಛಿನ್ನ-ಭಿನ್ನ-ವಾದ
ಛೀ
ಛೆ
ಛೇದಿಸಿ
ಜಂಘಾ-ಬ-ಲವೇ
ಜಂಜಾ-ಟ-ದಲ್ಲಿ
ಜಂತು
ಜಂತುಗಳ
ಜಂತು-ಗ-ಳಿಗೆ
ಜಂಭ-ಗ-ಳಿಲ್ಲ-ದಿದ್ದರೆ
ಜಂಭವನ್ನು
ಜಂಭವನ್ನೂ
ಜಖಂ
ಜಗಕ್ಕೆ
ಜಗ-ಜ-ಗಿ-ಸುತ್ತವೆ
ಜಗತ್ತನ್ನು
ಜಗತ್ತನ್ನೆ
ಜಗತ್ತನ್ನೇ
ಜಗತ್ತಿಗೆ
ಜಗತ್ತಿಗೇ
ಜಗತ್ತಿನ
ಜಗತ್ತಿ-ನಲ್ಲಾ-ಗಲಿ
ಜಗತ್ತಿ-ನಲ್ಲಿ
ಜಗತ್ತಿ-ನಲ್ಲೂ
ಜಗತ್ತಿ-ನಲ್ಲೆಲ್ಲ
ಜಗತ್ತಿ-ನಲ್ಲೇ
ಜಗತ್ತಿ-ನಾದ್ಯಂತ
ಜಗತ್ತಿ-ನಿಂದ
ಜಗತ್ತಿ-ನೊ-ಡನೆ
ಜಗತ್ತು
ಜಗತ್ತೆಲ್ಲ
ಜಗತ್ತೇ
ಜಗತ್ತೊಂದನ್ನು
ಜಗತ್ಪ್ರ-ಸಿದ್ಧ
ಜಗದ
ಜಗದಲ್ಲಿ
ಜಗ-ದಾತ್ಮಾ-ನಂದ
ಜಗ-ದಾತ್ಮಾ-ನಂದರ
ಜಗ-ದಾತ್ಮಾ-ನಂದರು
ಜಗದಾದಿ
ಜಗ-ದಾ-ದಿ-ಕಾ-ರ-ಣ-ವನ್ನು
ಜಗ-ದೊಳ್ಪಿಗೆ
ಜಗದ್ವಿಖ್ಯಾತ
ಜಗನ್ನಾಥ
ಜಗನ್ನಾ-ಥ-ದಾ-ಸರ
ಜಗನ್ನಾ-ಥ-ದಾ-ಸ-ರಾ-ದರು
ಜಗನ್ನಾ-ಥನ
ಜಗನ್ನಿ-ಯಾ-ಮಕ
ಜಗನ್ನಿ-ಯಾ-ಮ-ಕ-ಇ-ವು-ಗಳ
ಜಗಲಿ
ಜಗಲಿಯ
ಜಗಳ
ಜಗ-ಳ-ವಾ-ದಾ-ಗಲೂ
ಜಗ-ಳಕ್ಕಾಗಿ
ಜಗ-ಳ-ಗಂಟರು
ಜಗ-ಳ-ದಿಂದುಂಟಾ-ಗಿದ್ದ
ಜಗ-ಳ-ವನ್ನು
ಜಗ-ಳ-ವಾಗಿ
ಜಗ-ಳ-ವಾ-ಡುತ್ತ
ಜಗ-ಳ-ವಾ-ದಾಗ
ಜಗಳಾಡಿ
ಜಗ-ಳಾ-ಡುತ್ತಲೇ
ಜಗ-ಳಾ-ಡುತ್ತಿದ್ದಾರೆ
ಜಗುಲಿ
ಜಗ್ಗದ
ಜಟಿಲ
ಜಟಿ-ಲ-ವಾ-ಗಿದೆ
ಜಟಿ-ಲ-ವಾದ
ಜಟಿಲವೂ
ಜಟ್ಟಿಗಳು
ಜಡ
ಜಡತೆ
ಜಡರಾಗ
ಜಡವನು
ಜಡ-ವಸ್ತು-ವಲ್ಲ
ಜಡ-ವಸ್ತು-ವೆಂದೂ
ಜಡ-ವಾ-ಗಿಯೇ
ಜಡವಾದ
ಜಡ-ವಾ-ದ-ದಲ್ಲಿನ
ಜಡ-ವಾ-ದದ
ಜಡ-ವಾ-ದ-ದೇ-ಹಾತ್ಮ-ವಾದ
ಜಡ-ಸೃಷ್ಟಿಯ
ಜನ
ಜನಪ್ರಿ-ಯತೆ
ಜನ-ಕಂಡರೆ
ಜನ-ಕ-ನಾದ
ಜನಕನು
ಜನ-ಕ-ನೆ-ನಿ-ಸಿದ
ಜನ-ಕ-ವಾ-ಗಿ-ರ-ದಿದ್ದರೂ
ಜನ-ಕೋ-ಟಿಯ
ಜನಕ್ಕೆ
ಜನಗಳ
ಜನ-ಗ-ಳಂತೆ
ಜನಗಳೆ
ಜನಗಳೇ
ಜನ-ಜೀ-ವನ
ಜನ-ಜೀ-ವ-ನ-ದಲ್ಲಿನ
ಜನ-ಜೀ-ವ-ನ-ವನ್ನು
ಜನತೆ
ಜನತೆಗೆ
ಜನತೆಯ
ಜನ-ತೆ-ಯಿಂದ
ಜನ-ದಟ್ಟ-ಣಿಯ
ಜನನ
ಜನನಂ
ಜನ-ನ-ಕಾಲ
ಜನ-ನ-ಕಾ-ಲ-ದಿಂದಲೇ
ಜನನದ
ಜನ-ನ-ವ-ನ-ರಿಯೆ
ಜನನವೂ
ಜನ-ನಾ-ಯ-ಕರೂ
ಜನನಾಶ
ಜನ-ನಾ-ಶಕ್ಕೆ
ಜನ-ನಿ-ಬಿಡ
ಜನಪ್ರಿಯ
ಜನಪ್ರಿ-ಯತೆ
ಜನಪ್ರಿ-ಯ-ತೆಯ
ಜನಪ್ರಿ-ಯ-ತೆ-ಯನ್ನು
ಜನಪ್ರಿ-ಯ-ರಾ-ಗ-ಬೇ-ಕೆಂದಿ-ರು-ವಿರಾ
ಜನ-ಮ-ನ-ಮಾ-ಲಿನ್ಯ-ವೃದ್ಧಿ-ಗಾಗಿ
ಜನ-ಮ-ನಕ್ಕೆ
ಜನ-ಮ-ನ-ದಲ್ಲಿ
ಜನ-ಮಾ-ನ-ಸಕ್ಕೆ
ಜನ-ಮಾ-ನ-ಸ-ದಲ್ಲಿ
ಜನರ
ಜನರನ್ನು
ಜನ-ರನ್ನು-ಳಿ-ಸಲು
ಜನರನ್ನೂ
ಜನರಲ್
ಜನರಲ್ಲ
ಜನರಲ್ಲಿ
ಜನರಲ್ಲೂ
ಜನರಿಂದ
ಜನ-ರಿಂದಲೇ
ಜನ-ರಿ-ಗಾಗಿ
ಜನ-ರಿ-ಗಿಂತ
ಜನರಿಗೂ
ಜನರಿಗೆ
ಜನ-ರಿದ್ದರೂ
ಜನರು
ಜನರೂ
ಜನರೆಲ್ಲ
ಜನರೇ
ಜನರೇನೋ
ಜನ-ರೊಂದಿಗೆ
ಜನ-ರೊ-ಡನೆ
ಜನವರಿ
ಜನಸಂಖ್ಯೆ
ಜನ-ಸಂಖ್ಯೆ-ಯಲ್ಲಿ
ಜನ-ಸಂಖ್ಯೆ-ಯಿಂದ
ಜನ-ಸಂಪರ್ಕ
ಜನ-ಸ-ಮೂಹ
ಜನ-ಸ-ಮೂ-ಹದ
ಜನ-ಸಾ-ಮಾನ್ಯರ
ಜನ-ಸಾ-ಮಾನ್ಯರು
ಜನಸ್ತೋ-ಮ-ವನ್ನು
ಜನ-ಹೃ-ದ-ಯದ
ಜನಾಂಗ
ಜನಾಂಗ-ದೆ-ಡೆಗೆ
ಜನಾಂಗಕ್ಕೆ
ಜನಾಂಗ-ಗಳ
ಜನಾಂಗ-ಗ-ಳನ್ನೆಲ್ಲ
ಜನಾಂಗ-ಗ-ಳಲ್ಲಿ
ಜನಾಂಗ-ಗ-ಳಲ್ಲೂ
ಜನಾಂಗ-ಗ-ಳಿಗೂ
ಜನಾಂಗ-ಗ-ಳಿಗೆ
ಜನಾಂಗ-ಗಳೂ
ಜನಾಂಗದ
ಜನಾಂಗ-ದಲ್ಲಿ
ಜನಾಂಗ-ದಲ್ಲೂ
ಜನಾಂಗ-ದ-ವ-ರನ್ನು
ಜನಾಂಗ-ಭೇ-ದದ
ಜನಾಂಗ-ವನ್ನಾಗಿ
ಜನಾಂಗ-ವನ್ನು
ಜನಾಂಗ-ವನ್ನೇ
ಜನಾಂಗ-ವಾಗಿ
ಜನಾಂಗವೇ
ಜನಾಃ
ಜನಾ-ದ-ರಣೆ
ಜನಾ-ದ-ರ-ಣೆ-ಯನ್ನೂ
ಜನಿಸಿ
ಜನಿಸಿತು
ಜನಿಸಿದ
ಜನಿ-ಸಿ-ದಳು
ಜನಿ-ಸಿ-ದ-ವನು
ಜನಿ-ಸಿ-ದ-ವರು
ಜನಿ-ಸಿ-ದಾಗ
ಜನಿ-ಸಿ-ದುದು
ಜನಿಸಿದೆ
ಜನಿ-ಸಿದ್ದರು
ಜನಿ-ಸಿದ್ದಾರೆ
ಜನಿ-ಸಿದ್ದಾಳೆ
ಜನಿಸಿದ್ದು
ಜನಿ-ಸಿದ್ದು-ದಾಗಿ
ಜನಿ-ಸುತ್ತದೆ
ಜನಿ-ಸುತ್ತಾರೆ
ಜನಿ-ಸು-ವುದು
ಜನಿ-ಸು-ವುದೂ
ಜನುಮ
ಜನುಮದ
ಜನ್ನು
ಜನ್ಮ
ಜನ್ಮಗಳ
ಜನ್ಮ-ಗ-ಳನ್ನು
ಜನ್ಮ-ಗ-ಳಲ್ಲಿ
ಜನ್ಮ-ಗ-ಳಲ್ಲೂ
ಜನ್ಮಗಳೇ
ಜನ್ಮ-ಜನ್ಮಾಂತ-ರದ
ಜನ್ಮತಃ
ಜನ್ಮ-ತಾ-ಳ-ಬ-ಹುದು
ಜನ್ಮದ
ಜನ್ಮ-ದಲ್ಲಾ-ಗಲೀ
ಜನ್ಮದಲ್ಲಿ
ಜನ್ಮದಲ್ಲೂ
ಜನ್ಮದಲ್ಲೇ
ಜನ್ಮ-ವಾ-ಗದು
ಜನ್ಮ-ವೆತ್ತಿ-ದನು
ಜನ್ಮ-ವೆತ್ತಿದ್ದು
ಜನ್ಮವೇ
ಜನ್ಮ-ವೊಂದ-ರಲ್ಲಿ-ಅಟ್ಲಾಂಟಿಸ್
ಜನ್ಮಸ್ಥಾ-ನ-ವಾದ
ಜನ್ಮಹೀನ
ಜನ್ಮಾಂತರ
ಜನ್ಮಾಂತ-ರ-ಇ-ವು-ಗ-ಳೆಲ್ಲಾ
ಜನ್ಮಾಂತ-ರಕ್ಕೆ
ಜನ್ಮಾಂತ-ರ-ಗಳ
ಜನ್ಮಾಂತ-ರದ
ಜನ್ಮಾಂತ-ರ-ದಲ್ಲಿ
ಜನ್ಮಾಂತ-ರ-ವಾ-ದ-ಗ-ಳನ್ನವು
ಜನ್ಮಾಂತ-ರ-ವಾ-ದದ
ಜನ್ಮಾಂತ-ರ-ವಾ-ದ-ದಲ್ಲಿ
ಜನ್ಮಾಂತ-ರ-ವಾ-ದ-ವನ್ನು
ಜಪ
ಜಪಂಗ-ಳಯ್ಯ
ಜಪದಲ್ಲಿ
ಜಪಮಾಲೆ
ಜಪಾನಿನ
ಜಪಾ-ನಿ-ನಲ್ಲಿ
ಜಪಾನೀ
ಜಪಾ-ನೀ-ಯರು
ಜಪಾ-ನೀ-ಯರೂ
ಜಪಾನ್
ಜಪಾನ್ದೇ-ಶದ
ಜಪಾನ್ದೇ-ಶ-ದಲ್ಲಿ
ಜಪಿ-ಸಿ-ದರೆ
ಜಪಿಸು
ಜಪಿಸುತ್ತ
ಜಪಿ-ಸುತ್ತಾರೆ
ಜಮೀ-ನಿ-ನಲ್ಲಿ
ಜಯ
ಜಯತೇ
ಜಯದ
ಜಯ-ನ-ಗರ
ಜಯಭೇರಿ
ಜಯವನ್ನು
ಜಯವು
ಜಯ-ಶಾ-ಲಿ-ಗ-ಳಾದ
ಜಯ-ಶಾ-ಲಿ-ಗ-ಳಾ-ದ-ರಲ್ಲದೆ
ಜಯ-ಶೀ-ಲ-ರಾ-ಗುತ್ತಾರೆ
ಜಯಸಿತು
ಜಯಿಸ
ಜಯಿ-ಸ-ಬ-ಹುದು
ಜಯಿಸಲೂ
ಜಯಿಸಿ
ಜಯಿ-ಸಿ-ದರೆ
ಜಯಿ-ಸಿ-ದ-ವ-ರನ್ನು
ಜಯಿ-ಸುತ್ತದೆ
ಜಯಿ-ಸು-ವುದು
ಜರುಗ
ಜರೆದ
ಜರ್ಜ-ರಿ-ತ-ರಾ-ಗಿದ್ದಾರೋ
ಜರ್ನಲ್
ಜರ್ಮ-ನಿ-ಗಾ-ಗಲೀ
ಜರ್ಮ-ನಿ-ಯಲ್ಲಿ
ಜರ್ಮನ್
ಜಲ-ಕ-ಣ-ಗಳ
ಜಲ-ಕ-ಣ-ಗ-ಳಾಗಿ
ಜಲ-ಚ-ರ-ಗಳ
ಜಲದಂತೆ
ಜಲದಲ್ಲಿ
ಜಲ-ರಾ-ಶಿಯ
ಜಲ-ರಾ-ಶಿ-ಯಲ್ಲಿ
ಜಲಾಂತರ್ನೌಕೆ
ಜವದಿ
ಜವನಿಕೆ
ಜವಾಬ್ದಾ-ರ-ನಲ್ಲ
ಜವಾಬ್ದಾ-ರ-ನೆಂಬ
ಜವಾಬ್ದಾರಿ
ಜವಾಬ್ದಾ-ರಿಯ
ಜವಾಬ್ದಾ-ರಿ-ಯನ್ನು
ಜವಾಬ್ದಾ-ರಿ-ಯಿಂದ
ಜವಾಹರ
ಜವಾ-ಹ-ರ-ಲಾ-ಲರ
ಜವಾ-ಹ-ರ-ಲಾ-ಲ-ರನ್ನು
ಜವಾ-ಹ-ರ-ಲಾ-ಲ-ರಿಗೆ
ಜವಾ-ಹ-ರ-ಲಾ-ಲರು
ಜವಾ-ಹ-ರ-ಲಾಲ್
ಜವುಳಿ
ಜಸ-ಬೀ-ರನ
ಜಸ-ಬೀ-ರ-ನನ್ನು
ಜಸ-ಬೀ-ರ-ನಿಗೆ
ಜಸ-ಬೀ-ರನು
ಜಸಬೀರ್
ಜಹರ್
ಜಾಂಬ-ವಂತ-ರನ್ನು
ಜಾಗ
ಜಾಗಕ್ಕೆ
ಜಾಗ-ಗ-ಳಿಗೂ
ಜಾಗತಿಕ
ಜಾಗದ
ಜಾಗದಲ್ಲಿ
ಜಾಗ-ದಲ್ಲಿತ್ತು
ಜಾಗದಲ್ಲೇ
ಜಾಗದಿಂದ
ಜಾಗ-ದಿಂದಲೇ
ಜಾಗರಣ
ಜಾಗ-ರ-ಣ-ಇವು
ಜಾಗ-ರ-ಣ-ವಾ-ಗಲು
ಜಾಗರಣೆ
ಜಾಗ-ರೂ-ಕತೆ
ಜಾಗ-ರೂ-ಕ-ತೆಯ
ಜಾಗ-ರೂ-ಕ-ತೆ-ಯಿಂದ
ಜಾಗ-ರೂ-ಕ-ರನ್ನಾಗಿ
ಜಾಗ-ರೂ-ಕ-ರಾ-ಗ-ದಿದ್ದರೆ
ಜಾಗ-ರೂ-ಕ-ರಾ-ಗಿದ್ದಾ-ರೆಯೇ
ಜಾಗ-ರೂ-ಕ-ರಾ-ಗಿದ್ದು-ಕೊಂಡು
ಜಾಗ-ರೂ-ಕ-ರಾ-ಗಿ-ರ-ಬೇಕು
ಜಾಗ-ರೂ-ಕ-ರಾ-ಗುತ್ತೀರಾ
ಜಾಗ-ರೂ-ಕ-ರಾ-ಗುತ್ತೇವೆ
ಜಾಗ-ರೂ-ಕ-ರಾ-ಗುತ್ತೇ-ವೆಯೆ
ಜಾಗ-ರೂ-ಕರೂ
ಜಾಗ-ರೂ-ಕ-ವಾ-ಗಿದ್ದು
ಜಾಗ-ರೂ-ಕ-ವಾ-ಗಿ-ರುತ್ತದೆ
ಜಾಗವನ್ನು
ಜಾಗೃ-ತ-ಗೊಳಿ
ಜಾಗೃ-ತ-ನಾಗಿ
ಜಾಗೃ-ತ-ವಾ-ಗಿ-ರುತ್ತವೆ
ಜಾಗೃ-ತ-ವಾದ
ಜಾಗೃತಿ
ಜಾಗೃ-ತಿ-ಯನ್ನುಂಟು-ಮಾಡು
ಜಾಗೃ-ತಿ-ಯಾ-ಗ-ಬೇ-ಕಾ-ದರೆ
ಜಾಗ್ರತ
ಜಾಗ್ರ-ತ-ಗೊ-ಳಿಸಿ
ಜಾಗ್ರ-ತ-ಗೊ-ಳಿ-ಸು-ವುದೇ
ಜಾಗ್ರ-ತ-ನಾಗು
ಜಾಗ್ರ-ತ-ರಾ-ಗ-ಬೇಕು
ಜಾಗ್ರ-ತ-ರಾಗಿ
ಜಾಗ್ರ-ತ-ರಾ-ಗಿ-ರ-ಬೇ-ಕೆನ್ನು-ವು-ದಕ್ಕೆ
ಜಾಗ್ರ-ತ-ವಾಗಿ
ಜಾಗ್ರ-ತಾ-ವಸ್ಥೆಗೆ
ಜಾಗ್ರ-ತಾ-ವಸ್ಥೆ-ಯಲ್ಲಿ
ಜಾಗ್ರತೆ
ಜಾಗ್ರ-ತೆ-ಯಾ-ಗಿ-ರು-ವು-ದೊ-ಳಿತು
ಜಾಠರ
ಜಾಠರು
ಜಾಡನ್ನೇ
ಜಾಡ-ಮಾ-ಲಿಶ್ರೀ-ರಾ-ಮ-ಕೃಷ್ಣರ
ಜಾಡಿನಲ್ಲಿ
ಜಾಡಿನಿಂದ
ಜಾಡೇ
ಜಾಡ್ಯ
ಜಾಡ್ಯ-ಗ-ಳಿಂದ
ಜಾಣ
ಜಾತಕದ
ಜಾತ-ಕ-ವನ್ನು
ಜಾತಿ
ಜಾತಿ-ಗ-ಳಲ್ಲಿ
ಜಾತಿ-ಗ-ಳಲ್ಲೂ
ಜಾತಿ-ಗ-ಳಾಗಿ
ಜಾತಿ-ಗ-ಳಿಗೆ
ಜಾತಿಗೆ
ಜಾತಿ-ಜಾ-ತಿ-ಗ-ಳೊ-ಳ-ಗಿನ
ಜಾತಿಮತ
ಜಾತಿ-ಮ-ತ-ಕು-ಲ-ಗೋತ್ರ
ಜಾತಿ-ಮ-ತ-ಕು-ಲ-ಗೋತ್ರಕ್ಕೆ
ಜಾತಿ-ಮ-ತ-ಗಳ
ಜಾತಿ-ಮ-ತ-ಗ-ಳೊ-ಡನೆ
ಜಾತಿ-ಮ-ತ-ದ-ವರ
ಜಾತಿಯ
ಜಾತಿಯನ್ನು
ಜಾತಿಯಲ್ಲಿ
ಜಾತಿ-ಯ-ವ-ರಿಗೂ
ಜಾತಿ-ಯ-ವ-ರೆಂಬು-ದನ್ನು
ಜಾತಿ-ಯ-ವ-ಳಾ-ಗಿದ್ದಳು
ಜಾತಿಯವು
ಜಾತಿಯೂ
ಜಾತಿಯೇ
ಜಾತೀಯ
ಜಾತೀ-ಯ-ತೆಯ
ಜಾತೀ-ಯ-ತೆ-ಯನ್ನು
ಜಾತ್ಯ-ತೀ-ತ-ರಾಗಿ
ಜಾತ್ರೆ
ಜಾನ್
ಜಾನ್ಫಾಕ್ಸ್
ಜಾನ್ಫಾರ್ಕ್
ಜಾನ್ಸನರ
ಜಾನ್ಸ-ನ-ರಿಗೆ
ಜಾನ್ಸನ್
ಜಾಫ್ರೀ
ಜಾಮೀನು
ಜಾಯಮಾನ
ಜಾಯಸ್ವಮ್ರಿ-ಯಸ್ವ
ಜಾರಿ
ಜಾರಿಗೆ
ಜಾರಿಬಿದ್ದ
ಜಾರೀಶ್ವರ
ಜಾರುತ್ತಿದ್ದೇನೆ
ಜಾರುತ್ತೇವೆ
ಜಾರ್ಜ್
ಜಾಲ
ಜಾಲ-ಗ-ಳಿಂದ
ಜಾಲದಂತೆ
ಜಾಲದಲ್ಲಿ
ಜಾಲದಿಂದ
ಜಾಲ-ವ-ನಂತ
ಜಾಲವನ್ನು
ಜಾವ
ಜಾಸ್ತಿ
ಜಾಸ್ತಿ-ಯಾ-ದರೂ
ಜಾಸ್ತಿ-ಯಾ-ದರೆ
ಜಾಹಿರಾತು
ಜಾಹಿ-ರಾ-ತುಈ
ಜಾಹೀ-ರಾ-ತಿನ
ಜಿ
ಜಿಂಕೆ
ಜಿಂಕೆಯು
ಜಿಗಿದು
ಜಿಪು-ಣ-ತ-ನದ
ಜಿಪು-ಣ-ತ-ನ-ವಿಲ್ಲ
ಜಿಲ್ಲೆ
ಜಿಲ್ಲೆಯ
ಜೀನಾ
ಜೀನ್
ಜೀಬ್ರಾನ್
ಜೀರ್ಣ-ರ-ಸ-ಗಳ
ಜೀರ್ಣ-ಶಕ್ತಿಯ
ಜೀರ್ಣಾಂಗ
ಜೀರ್ಣಾಂಗ-ಗಳ
ಜೀರ್ಣಾಂಗದ
ಜೀರ್ಣಾಂಗವ್ಯೂ-ಹವೇ
ಜೀರ್ಣಿ-ಸಿ-ಕೊಂಡರು
ಜೀರ್ಣಿ-ಸಿ-ಕೊಂಡು
ಜೀರ್ಣಿ-ಸಿ-ಕೊಳ್ಳಲೂ
ಜೀವ
ಜೀವ-ಜೀ-ವ-ನ-ವನ್ನೇ
ಜೀವಶಿವ
ಜೀವಂತ
ಜೀವಂತ-ರಾ-ಗಿ-ರು-ವ-ವ-ರಲ್ಲದೆ
ಜೀವಂತ-ವಾಗಿ
ಜೀವಂತ-ವಾ-ಗಿದೆ
ಜೀವಂತ-ವಾ-ಗಿದ್ದಾಗ
ಜೀವಂತ-ವಾ-ಗಿದ್ದು
ಜೀವಂತ-ವಾ-ಗಿಯೇ
ಜೀವಂತ-ವಾ-ಗಿ-ರುವ
ಜೀವಂತ-ವಾ-ಗಿ-ಸ-ಬ-ಹು-ದೆಂಬು-ದನ್ನು
ಜೀವಕೆ
ಜೀವ-ಕೋ-ಶ-ಗಳು
ಜೀವ-ಕೋ-ಶ-ದಲ್ಲಿನ
ಜೀವಕ್ಕಿಂತ
ಜೀವಕ್ಕೆ
ಜೀವ-ಗ-ಳನ್ನು
ಜೀವ-ಜಂತು-ಗ-ಳಲ್ಲೂ
ಜೀವ-ಜಂತು-ಗ-ಳಿಗೆ
ಜೀವ-ಜೀ-ವ-ನದ
ಜೀವ-ಜೀ-ವ-ನ-ವನ್ನು
ಜೀವತತ್ತ್ವ
ಜೀವ-ತತ್ತ್ವದ
ಜೀವ-ತತ್ತ್ವ-ವನ್ನು
ಜೀವ-ತತ್ವವು
ಜೀವದಯೆ
ಜೀವದಲ್ಲಿ
ಜೀವದಾನ
ಜೀವನ
ಜೀವ-ನಕ್ಕಾಗಿ
ಜೀವನಕ್ಕೆ
ಜೀವ-ನಕ್ಕೊಂದು
ಜೀವ-ನ-ಗಳ
ಜೀವ-ನ-ಗ-ಳನ್ನೂ
ಜೀವ-ನ-ಗ-ಳಲ್ಲಿ
ಜೀವ-ನ-ಗಳು
ಜೀವ-ನ-ಚ-ರಿತ್ರೆಯ
ಜೀವನದ
ಜೀವ-ನ-ದಲ್ಲಾ-ಗಲೀ
ಜೀವ-ನ-ದಲ್ಲಿ
ಜೀವ-ನ-ದಲ್ಲಿ-ರುವ
ಜೀವ-ನ-ದಲ್ಲೂ
ಜೀವ-ನ-ದಾಯಿ
ಜೀವ-ನ-ದಾ-ಯಿ-ಯಾದ
ಜೀವ-ನ-ದಿಂದ
ಜೀವ-ನ-ದುದ್ದಕ್ಕೂ
ಜೀವ-ನ-ಪೂರ್ತಿ
ಜೀವ-ನ-ಮಟ್ಟ-ವನ್ನು
ಜೀವ-ನ-ಯಾತ್ರೆ
ಜೀವ-ನ-ವನ್ನಾ-ಗಲೀ
ಜೀವ-ನ-ವನ್ನು
ಜೀವ-ನ-ವನ್ನೂ
ಜೀವ-ನ-ವನ್ನೆಲ್ಲಾ
ಜೀವ-ನ-ವನ್ನೇ
ಜೀವ-ನ-ವಲ್ಲ
ಜೀವ-ನ-ವಾ-ಗದೇ
ಜೀವ-ನ-ವಿದೆ
ಜೀವ-ನ-ವಿ-ದೆಯೇ
ಜೀವನವು
ಜೀವನವೂ
ಜೀವನವೆ
ಜೀವ-ನ-ವೆಂಬುದು
ಜೀವ-ನ-ವೆಲ್ಲ
ಜೀವನವೇ
ಜೀವ-ನ-ಸಾ-ಧನೆ
ಜೀವ-ನಾ-ಕಾಂಕ್ಷೆ-ಇವು
ಜೀವ-ನಾ-ದರ್ಶ-ಗ-ಳನ್ನು
ಜೀವ-ನಾ-ದರ್ಶ-ಗ-ಳನ್ನೂ
ಜೀವ-ನಾ-ದರ್ಶ-ವನ್ನು
ಜೀವ-ನಾ-ನು-ಭ-ವ-ಗ-ಳಿಂದ
ಜೀವ-ನಾ-ನು-ಭ-ವ-ದಿಂದ
ಜೀವ-ನಾ-ವ-ಲೋ-ಕನ
ಜೀವನು
ಜೀವ-ನೋ-ಪಾ-ಯಕ್ಕಾಗಿ
ಜೀವ-ನೋ-ಪಾ-ಯಕ್ಕೆ
ಜೀವ-ಮಾ-ನ-ವೆಲ್ಲ
ಜೀವ-ರ-ಸ-ದಂತಿದ್ದ
ಜೀವವನ್ನು
ಜೀವವಿಮಾ
ಜೀವ-ವಿ-ರು-ವ-ವು-ಗಳು
ಜೀವವೇ
ಜೀವಶಾಸ್ತ್ರ
ಜೀವ-ಶಾಸ್ತ್ರಜ್ಞ
ಜೀವ-ಶಾಸ್ತ್ರ-ದಲ್ಲಿನ
ಜೀವಶಿವ
ಜೀವಸತ್ವ
ಜೀವಾತ್ಮ
ಜೀವಾತ್ಮದ
ಜೀವಾತ್ಮನ
ಜೀವಾತ್ಮ-ನನ್ನು
ಜೀವಾತ್ಮನು
ಜೀವಾತ್ಮ-ನೆಂದು
ಜೀವಾತ್ಮ-ನೊಬ್ಬ
ಜೀವಾತ್ಮರು
ಜೀವಾತ್ಮವೂ
ಜೀವಾಳ
ಜೀವಿ
ಜೀವಿ-ಕೆ-ಗಾಗಿ
ಜೀವಿಗಳ
ಜೀವಿ-ಗ-ಳನ್ನು
ಜೀವಿ-ಗ-ಳಲ್ಲಿ
ಜೀವಿ-ಗ-ಳಲ್ಲೂ
ಜೀವಿ-ಗ-ಳಿಂದ
ಜೀವಿ-ಗ-ಳಿಗೆ
ಜೀವಿಗಳು
ಜೀವಿಗಳೂ
ಜೀವಿ-ಗ-ಳೆ-ಡೆಗೆ
ಜೀವಿ-ಗ-ಳೆ-ನಿ-ಸಿ-ಕೊಂಡ-ವ-ರಲ್ಲಿ
ಜೀವಿಗೆ
ಜೀವಿತಕ್ಕೆ
ಜೀವಿತದ
ಜೀವಿ-ತ-ದಲ್ಲಿ
ಜೀವಿಯ
ಜೀವಿಯನ್ನು
ಜೀವಿ-ಯಲ್ಲಿ-ರುವ
ಜೀವಿಯಲ್ಲೂ
ಜೀವಿ-ಯಾ-ಗಲಿ
ಜೀವಿಯು
ಜೀವಿಯೂ
ಜೀವಿಸಿದ್ದ
ಜೀವಿ-ಸಿದ್ದಾಗ
ಜೀವಿ-ಸಿ-ರು-ವ-ವರು
ಜುಗುಪ್ಸೆ
ಜುಗುಪ್ಸೆ-ಯನ್ನುಂಟು-ಮಾಡಿ
ಜುಲೈ
ಜುಲ್ಮಾನೆ
ಜುಹು-ನಲ್ಲಿದ್ದ
ಜುಹುವಿಗೆ
ಜೂಜಿನ
ಜೂಜು
ಜೂನ್
ಜೂಲಿಯಸ್
ಜೆ
ಜೆಮ್ಶೆಡ್ಪು-ರದ
ಜೇಡನ
ಜೇಡರ
ಜೇನಾದೀತು
ಜೇನು
ಜೇಬಿನಲ್ಲಿ
ಜೇಬಿ-ನಲ್ಲಿ-ರುವ
ಜೇಮ್ಸ್
ಜೈನ
ಜೈಲಿಗೆ
ಜೈಲಿನಲ್ಲಿ
ಜೈಲಿ-ನಲ್ಲಿ-ರುವ
ಜೈಲಿನಲ್ಲೇ
ಜೈಲು
ಜೈಲುವಾಸ
ಜೈವ
ಜೈವಸ್ರೋತ
ಜೈವಿಕ
ಜೈಸಬೇಕು
ಜೊತೆ
ಜೊತೆಯಲ್ಲೇ
ಜೊತೆಗಿದ್ದು
ಜೊತೆಗೂಡಿ
ಜೊತೆ-ಗೂ-ಡಿ-ಸು-ವುದು
ಜೊತೆಗೆ
ಜೊತೆಗೇ
ಜೊತೆ-ಜೊ-ತೆಗೆ
ಜೊತೆ-ಜೊ-ತೆಗೇ
ಜೊತೆಯ
ಜೊತೆಯಲ್ಲಿ
ಜೊತೆಯಲ್ಲೇ
ಜೊತೆ-ಯಾ-ಗಿಯೇ
ಜೋಕೆ
ಜೋಗದಲ್ಲಿ
ಜೋಡಿಸಿ
ಜೋಡಿ-ಸಿ-ಡುತ್ತದೆ
ಜೋಡಿ-ಸುತ್ತಿದ್ದೇನೆ
ಜೋಡಿಸುವ
ಜೋತಾ-ಡುತ್ತಿತ್ತು
ಜೋತಾ-ಡು-ವು-ದನ್ನೂ
ಜೋನಾಥನ್
ಜೋರಾಗಿ
ಜೋಲು-ಮು-ಖ-ದಿಂದ
ಜೋಲುಮೋರೆ
ಜೋಳದ
ಜೋಸೆಫ್
ಜ್ಞಾನ
ಜ್ಞಾನ-ಉಳ್ಳ-ವ-ರಾ-ಗಿದ್ದರು
ಜ್ಞಾನ-ಕಿಂಡಿ-ಯನ್ನು
ಜ್ಞಾನಕ್ಕೆ
ಜ್ಞಾನಗಳ
ಜ್ಞಾನದ
ಜ್ಞಾನ-ದಂತ-ವೊಂದು
ಜ್ಞಾನದಲ್ಲಿ
ಜ್ಞಾನದಾಹ
ಜ್ಞಾನ-ದಿಂದಲೆ
ಜ್ಞಾನ-ದೀ-ಪ-ವನ್ನು
ಜ್ಞಾನ-ನಿ-ಧಿ-ಯಾ-ಗಿದ್ದ
ಜ್ಞಾನಪೀಠ
ಜ್ಞಾನಪ್ರ-ಸಾ-ರದ
ಜ್ಞಾನಮಯ
ಜ್ಞಾನ-ಮಾರ್ಗಾ-ವ-ಲಂಬಿ-ಗಳು
ಜ್ಞಾನಮ್
ಜ್ಞಾನ-ಯಾತ್ರೆ-ಯನ್ನು
ಜ್ಞಾನರಾಶಿ
ಜ್ಞಾನವನ್ನು
ಜ್ಞಾನವನ್ನೂ
ಜ್ಞಾನ-ವಾ-ಗಿದೆ
ಜ್ಞಾನವಿದೆ
ಜ್ಞಾನ-ವಿಲ್ಲದ
ಜ್ಞಾನವೂ
ಜ್ಞಾನವೇ
ಜ್ಞಾನ-ಸಂಗ್ರ-ಹವು
ಜ್ಞಾನ-ಸಂಪನ್ನ-ರಾದ
ಜ್ಞಾನಾ-ಭಾ-ವದ
ಜ್ಞಾನಿ
ಜ್ಞಾನಿಗಳ
ಜ್ಞಾನಿಗಳು
ಜ್ಞಾನಿಗೆ
ಜ್ಞಾನಿಯ
ಜ್ಞಾನೇಂದ್ರಿಯ
ಜ್ಞಾಪಕ
ಜ್ಞಾಪಕಕ್ಕೆ
ಜ್ಞಾಪ-ಕ-ದಲ್ಲೇ
ಜ್ಞಾಪಿ-ಸಿ-ಕೊಳ್ಳಿ
ಜ್ಯೋತಿ
ಜ್ಯೋತಿಯನ್ನು
ಜ್ಯೋತಿರ್ಗ-ಮಯ
ಜ್ಯೋತಿಷ್ಕರು
ಜ್ಯೋತಿಷ್ಯ
ಜ್ವರ
ಜ್ವರದಿಂದ
ಜ್ವರವೂ
ಜ್ವಲಂತ
ಜ್ವಲ-ನಾತ್ಮಕ
ಜ್ವಾಲಾ-ಮು-ಖಿಯ
ಜ್ವಾಲೆ
ಜ್ವಾಲೆ-ಗ-ಳಂತೆ
ಜ್ವಾಲೆಗಳು
ಜ್ವಾಲೆಯ
ಝಾಡಿಸಿ
ಟನ್
ಟರ್ಮ್
ಟಸ್ಕಗೀ
ಟಾಗೂರ್
ಟಾನಿಕ್
ಟಾನಿಕ್ಕಿ-ಗಿಂತಲೂ
ಟಾನಿಕ್ಕು-ಗಳಿ
ಟಾಯ್ನಬೀ
ಟಾರ್ಚು
ಟಾರ್ಚು-ಧಾ-ರಿ-ಯಾ-ಗಿ-ರುತ್ತಿದ್ದೆ
ಟಾವೊ
ಟಾವೋ
ಟಿ
ಟಿಕೆಟ್
ಟಿಟಾನಿಕ್
ಟಿಪ್ಪಣಿಯ
ಟೀಕಾ-ಕಾ-ರರ
ಟೀಕಾಸ್ತ್ರಕ್ಕೆ
ಟೀಕಿಸಿ
ಟೀಕಿಸಿದ
ಟೀಕಿ-ಸಿ-ದರೆ
ಟೀಕಿ-ಸಿ-ಯಾರೇ
ಟೀಕಿಸುತ್ತಾ
ಟೀಕಿ-ಸುತ್ತಿದ್ದ
ಟೀಕಿಸುವ
ಟೀಕಿ-ಸು-ವ-ವರು
ಟೀಕಿ-ಸು-ವಾಗ
ಟೀಕಿ-ಸು-ವಾತ
ಟೀಕಿ-ಸು-ವುದು
ಟೀಕಿ-ಸು-ವು-ದೆಂದರೆ
ಟೀಕೆ
ಟೀಕೆಗಳ
ಟೀಕೆ-ಗ-ಳನ್ನು
ಟೀಕೆ-ಗ-ಳಲ್ಲಿ
ಟೀಕೆ-ಗ-ಳಿಂದ
ಟೀಕೆಗಳು
ಟೀಕೆಗೆ
ಟೀಕೆ-ಮಾ-ಡು-ವ-ವ-ರನ್ನು
ಟೀಕೆಯ
ಟೀಕೆಯನ್ನು
ಟೀಕೆಯಾಗಿ
ಟೀಕೆ-ಯಾ-ದರೆ
ಟೀಕೆಯಿಂದ
ಟೆಟಾನಿಕ್
ಟೆನ್ನಿಸ್
ಟೆಲಿಗ್ರಾಂ
ಟೆಲಿಗ್ರಾಫ್
ಟೆಲಿಫೋನ್
ಟೆಲಿ-ವಿ-ಷನ್
ಟೆಲಿ-ವಿ-ಷನ್ಗ-ಳಲ್ಲಿ
ಟೆಲಿ-ವಿ-ಷನ್ನಿನ
ಟೆಲೆಸ್ಕೋಪ್
ಟೆಸ್ಲಾ
ಟೇಪ್
ಟೇಪ್ರೆ-ಕಾರ್ಡ-ರಿ-ನಲ್ಲಿ
ಟೈಂಸ್
ಟೈಗರ್
ಟೈಪಿಸ್ಟ್
ಟೈಪು
ಟೈಪ್
ಟೈಫಾಯಿಡ್
ಟೈಮ್
ಟೈಮ್ಸ್
ಟೊಯ್ನಬಿ
ಟೊಯ್ನಬೀ
ಟೋಕಿಯೋ
ಟೋಪಿ
ಟೋಪಿಯೂ
ಟೋಪೇಕಾದ
ಟೋಪೇ-ಕಾ-ದಲ್ಲಿ-ರುವ
ಟೋಯ್ನಬೀ
ಟ್ಯಾಕ್ಸ್
ಟ್ಯೂಬನ್ನು
ಟ್ಯೂಬಿನ
ಟ್ಯೂಬಿನಲ್ಲಿ
ಟ್ಯೂಬಿನಿಂದ
ಟ್ರಾನ್ಸ್ಪರ್ಸ-ನಲ್
ಟ್ರೈನ್
ಟ್ವೆಮ್ಲೋ
ಟ್ವೆಮ್ಲೋರ
ಟ್ಸಿವೊಲ್ಕೊವ್ಸ್ಕಿ
ಡಂಕನ್
ಡಂಕನ್ನ
ಡಬ್
ಡಬ್ಬ-ಗ-ಳನ್ನು
ಡಬ್ಬಿಯಲ್ಲಿ
ಡಯಲ್ಮಾಡಿ
ಡವ್ಸನ್
ಡಾ
ಡಾಕ್ಟರನ್ನು
ಡಾಕ್ಟರರ
ಡಾಕ್ಟ-ರ-ರನ್ನು
ಡಾಕ್ಟರರು
ಡಾಕ್ಟರರೂ
ಡಾಕ್ಟ-ರ-ರೊಂದಿಗೆ
ಡಾಕ್ಟ-ರ-ರೊಬ್ಬರು
ಡಾಕ್ಟರಲ್ಲ
ಡಾಕ್ಟರಾಗಿ
ಡಾಕ್ಟರಿಗೆ
ಡಾಕ್ಟರು
ಡಾಕ್ಟ-ರು-ಗಳ
ಡಾಕ್ಟ-ರು-ಗ-ಳನ್ನು
ಡಾಕ್ಟ-ರು-ಗ-ಳಲ್ಲಿ
ಡಾಕ್ಟ-ರು-ಗಳು
ಡಾಕ್ಟ-ರೇ-ಟು-ಗ-ಳನ್ನು
ಡಾಕ್ಟ-ರೊಬ್ಬರು
ಡಾಕ್ಟರ್
ಡಾರ್ವಿನ್
ಡಾರ್ವಿನ್ನ
ಡಾಲ-ರು-ಗಳ
ಡಾಲ-ರು-ಗ-ಳನ್ನು
ಡಾಲರ್
ಡಾಲರ್ಗ-ಳಲ್ಲಿ
ಡಿ
ಡಿಕ್ಕಿ
ಡಿಗ್ರಿ
ಡಿಗ್ರಿಗಳ
ಡಿಮೋಸ್ತ-ನೀಸ್
ಡಿಮೋಸ್ತ-ನೀಸ್ನ
ಡಿವಿಜಿ
ಡಿಸ್ನೀ
ಡೀಸೆಲ್
ಡೆನ್ನಿಸ್
ಡೆಮೊನ್ಸ್ಟ್ರೇ-ಟರ್
ಡೆವಿ-ಡೊ-ವಿಚ್
ಡೆಹ-ರಾ-ಡೂ-ನಿನ
ಡೆಹ-ರಾ-ಡೂ-ನಿ-ನಲ್ಲಿ
ಡೇಟನ್ನಿನ
ಡೇನಿಸ್
ಡೇಲ್
ಡೇವಿಡ್
ಡೇವಿ-ಸ್ಎಲ್ಲರೂ
ಡೈಜಸ್ಟ್
ಡೈಜೆಸ್ಟ್
ಡೈಜೆಸ್ಟ್ನಂಥ
ಡೈಜೆಸ್ಟ್ಮೇ
ಡೈವರ್
ಡೊಂಕುಗಳು
ಡೊನ್ನೆಲಿ
ಡೋಲಾ-ಯ-ಮಾ-ನ-ವಾ-ದರೂ
ಡ್ಯೂಕ್
ಡ್ರಾಯರ್
ಡ್ರೈವ-ರ-ನಿ-ಗಾಗಿ
ಢಿಕ್ಕಿ
ತಂಗಮ್ಮನ
ತಂಗಮ್ಮ-ನನ್ನು
ತಂಗಮ್ಮ-ನಿಗೆ
ತಂಗಿ
ತಂಗಿದ್ದರು
ತಂಗಿದ್ದಾ-ನೆಯೇ
ತಂಗಿದ್ದೆ
ತಂಗಿಯ
ತಂಗಿಯನ್ನು
ತಂಗಿಯನ್ನೊ
ತಂಗಿ-ಯ-ರಾ-ಗಿದ್ದು
ತಂಡ
ತಂಡಗಳು
ತಂಡದ
ತಂಡ-ದಲ್ಲಿದ್ದ
ತಂಡ-ದಲ್ಲಿದ್ದರು
ತಂಡವನ್ನು
ತಂಡ-ವೆಂದರೆ
ತಂತಿ
ತಂತಿಗಳು
ತಂತಿಯನ್ನು
ತಂತಿಯಲ್ಲಿ
ತಂತು-ಗ-ಳ-ನಂತ
ತಂತ್ರ
ತಂತ್ರ-ಗ-ಳನ್ನು
ತಂತ್ರ-ಗ-ಳನ್ನೂ
ತಂತ್ರ-ಗಾ-ರಿಕೆ
ತಂತ್ರಜ್ಞಾ-ನ-ಗ-ಳಲ್ಲಿಯೂ
ತಂತ್ರದಿಂದ
ತಂತ್ರವನ್ನು
ತಂತ್ರವೆಂದು
ತಂತ್ರೋದ್ಯಮ
ತಂದ
ತಂದಂತಲ್ಲ
ತಂದದ್ದು
ತಂದರು
ತಂದಳು
ತಂದಾಗ
ತಂದಾರು
ತಂದಿತು
ತಂದಿ-ತೆಂಬು-ದನ್ನು
ತಂದಿದೆಯೆ
ತಂದಿರಿ
ತಂದಿ-ರು-ವುದು
ತಂದಿಲ್ಲ
ತಂದೀತು
ತಂದೀತೆಂದು
ತಂದೀ-ತೆಂಬು-ದನ್ನು
ತಂದು
ತಂದು-ಕೊಳ್ಳು-ವಿರಿ
ತಂದು-ಕೊಂಡದ್ದೇ
ತಂದು-ಕೊಂಡರು
ತಂದು-ಕೊಂಡಳು
ತಂದು-ಕೊಂಡಿದ್ದೆ
ತಂದು-ಕೊಂಡಿಲ್ಲ-ವಲ್ಲ
ತಂದುಕೊಂಡು
ತಂದು-ಕೊಂಡುವ
ತಂದುಕೊಟ್ಟು
ತಂದು-ಕೊ-ಡ-ಬಲ್ಲುದು
ತಂದು-ಕೊ-ಡಲು
ತಂದು-ಕೊ-ಡುವ
ತಂದು-ಕೊ-ಡು-ವಂತೆ
ತಂದು-ಕೊಳ್ಳ-ಬ-ಹುದು
ತಂದು-ಕೊಳ್ಳ-ಬೇಕು
ತಂದು-ಕೊಳ್ಳ-ಬೇ-ಕೆಂದು
ತಂದುಕೊಳ್ಳಿ
ತಂದು-ಕೊಳ್ಳುತ್ತೇ-ವೆಂಬುದು
ತಂದು-ಕೊಳ್ಳು-ವು-ದಿಲ್ಲ
ತಂದೆ
ತಂದೆಗೂ
ತಂದೆಗೆ
ತಂದೆ-ತಾ-ಯಂದಿರ
ತಂದೆ-ತಾ-ಯಂದಿ-ರಿಗೆ
ತಂದೆ-ತಾ-ಯಂದಿರು
ತಂದೆತಾಯಿ
ತಂದೆ-ತಾ-ಯಿ-ಗಳ
ತಂದೆ-ತಾ-ಯಿ-ಗ-ಳನ್ನು
ತಂದೆ-ತಾ-ಯಿ-ಗ-ಳನ್ನೇ
ತಂದೆ-ತಾ-ಯಿ-ಗ-ಳಲ್ಲಿ
ತಂದೆ-ತಾ-ಯಿ-ಗ-ಳಾ-ದರೋ
ತಂದೆ-ತಾ-ಯಿ-ಗ-ಳಿಂದ
ತಂದೆ-ತಾ-ಯಿ-ಗ-ಳಿಗೆ
ತಂದೆ-ತಾ-ಯಿ-ಗ-ಳಿಬ್ಬರೂ
ತಂದೆ-ತಾ-ಯಿ-ಗಳು
ತಂದೆ-ತಾ-ಯಿ-ಗಳೂ
ತಂದೆ-ಮಕ್ಕಳು
ತಂದೆಯ
ತಂದೆಯಂತೆ
ತಂದೆಯದು
ತಂದೆಯನ್ನು
ತಂದೆಯನ್ನೇ
ತಂದೆಯಷ್ಟು
ತಂದೆ-ಯಾ-ದರೂ
ತಂದೆಯು
ತಂದೆಯೇ
ತಂದೆ-ಯೊಂದಿಗೆ
ತಂದೇ
ತಂದೊಡ್ಡಿದ್ದಳು
ತಂದೊಡ್ಡುತ್ತವೆ
ತಂದೊಪ್ಪಿ-ಸಿದ
ತಕ್ಕ
ತಕ್ಕಂತೆ
ತಕ್ಕಡಿಯ
ತಕ್ಕಷ್ಟು
ತಕ್ಕುದಾದ
ತಗ-ಲ-ಬಲ್ಲುದು
ತಗ-ಲಿ-ದಂತಾ-ಗುತ್ತದೆ
ತಗ-ಲಿದ್ದರೂ
ತಗು-ಲಿ-ಕೊಂಡಿದೆ
ತಗುಲಿದ
ತಜ್ಞ
ತಜ್ಞನ
ತಜ್ಞನನ್ನು
ತಜ್ಞ-ನಾ-ಗಲೀ
ತಜ್ಞನು
ತಜ್ಞನೂ
ತಜ್ಞ-ನೆ-ನಿ-ಸಿದ
ತಜ್ಞನೊಬ್ಬ
ತಜ್ಞ-ನೊಬ್ಬನು
ತಜ್ಞರ
ತಜ್ಞರನ್ನು
ತಜ್ಞರನ್ನೂ
ತಜ್ಞರಲ್ಲಿ
ತಜ್ಞ-ರಾ-ದ-ವರು
ತಜ್ಞರಿಂದ
ತಜ್ಞ-ರಿಂದಲೂ
ತಜ್ಞರಿಗೆ
ತಜ್ಞರು
ತಜ್ಞ-ರು-ಗ-ಳಲ್ಲಿ
ತಜ್ಞ-ರು-ಗ-ಳಿಗೆ
ತಜ್ಞರೂ
ತಟ್ಟದು
ತಟ್ಟನೆ
ತಟ್ಟನೇ
ತಟ್ಟ-ಬಲ್ಲದು
ತಟ್ಟಿ
ತಟ್ಟು
ತಟ್ಟುತ್ತಾನೆ
ತಟ್ಟುವುದು
ತಟ್ಟೆ
ತಟ್ಟೆಯಲ್ಲಿ
ತಡ
ತಡ-ಕಾ-ಡು-ವ-ವರು
ತಡ-ವಾ-ಗಿ-ಯಾ-ದರೂ
ತಡ-ವಾ-ದದ್ದಿಲ್ಲ
ತಡ-ವಾ-ದರೂ
ತಡವಿಲ್ಲ
ತಡೆ
ತಡೆ-ಗಟ್ಟಲು
ತಡೆ-ಗ-ಳನ್ನಿಟ್ಟರೂ
ತಡೆ-ಗ-ಳನ್ನು
ತಡೆ-ಗ-ಳನ್ನೊಡ್ಡುವ
ತಡೆ-ಗ-ಳಿಲ್ಲದ
ತಡೆಗಳು
ತಡೆ-ತೊಂದ-ರೆ-ಗ-ಳನ್ನು
ತಡೆ-ದೀ-ತೆಂದು
ತಡೆದು
ತಡೆ-ದು-ಕೊಂಡಲ್ಲಿ
ತಡೆ-ದು-ಕೊಂಡು
ತಡೆ-ದು-ಕೊಳ್ಳು-ವುದು
ತಡೆಯದೆ
ತಡೆಯನ್ನು
ತಡೆ-ಯನ್ನೊಡ್ಡು-ವು-ದಿಲ್ಲ
ತಡೆ-ಯ-ಬಲ್ಲ
ತಡೆ-ಯ-ಬಲ್ಲೆ-ವಾ-ದರೆ
ತಡೆ-ಯ-ಬಲ್ಲೆವೇ
ತಡೆ-ಯ-ಬ-ಹುದು
ತಡೆ-ಯ-ಲಾ-ರದ
ತಡೆ-ಯ-ಲಾ-ರದೆ
ತಡೆ-ಯ-ಲಾ-ರದೇ
ತಡೆಯಲು
ತಡೆ-ಯಾ-ಗ-ದಂತೆ
ತಡೆಯಾಗಿ
ತಡೆ-ಯಿಲ್ಲದೆ
ತಡೆ-ಯಿಲ್ಲದೇ
ತಡೆ-ಯುತ್ತ-ದೆ-ಎಂಬು-ದನ್ನು
ತಡೆ-ಯುತ್ತಿಲ್ಲ
ತಡೆ-ಯು-ವುದು
ತಡೆ-ಹಿ-ಡಿ-ಯದೆ
ತಡೆ-ಹಿ-ಡಿ-ಯುವ
ತಡೆ-ಹಿ-ಡಿ-ಯು-ವು-ದಿಲ್ಲ
ತಣಿ-ಸ-ಲಾ-ರನೇ
ತಣಿಸಲು
ತಣಿ-ಸುತ್ತಿ-ರ-ಲಿಲ್ಲ
ತಣಿ-ಸು-ವಲ್ಲಿ
ತಣ್ಣಗೆ
ತತ್ಕಾಲದ
ತತ್ಕ್ಷಣವೇ
ತತ್ತರಿಸಿ
ತತ್ತ-ರಿ-ಸಿ-ದರು
ತತ್ತ-ರಿ-ಸಿ-ಹೋದ
ತತ್ತ-ರಿ-ಸುತ್ತದೆ
ತತ್ತ-ರಿ-ಸುತ್ತಾನೆ
ತತ್ತ-ರಿ-ಸುವ
ತತ್ತ-ರಿ-ಸು-ವ-ವ-ರಿದ್ದಾರೆ
ತತ್ತ-ರಿ-ಸು-ವು-ದಾ-ಗಲಿ
ತತ್ತ್ವ
ತತ್ತ್ವ-ಇ-ದನ್ನು
ತತ್ತ್ವಕ್ಕೆ
ತತ್ತ್ವಗಳ
ತತ್ತ್ವ-ಗ-ಳನ್ನು
ತತ್ತ್ವ-ಗ-ಳಲ್ಲಿ
ತತ್ತ್ವ-ಗ-ಳಿವೆ
ತತ್ತ್ವಗಳು
ತತ್ತ್ವ-ಚಿಂತನೆ
ತತ್ತ್ವಜ್ಞಾನ
ತತ್ತ್ವಜ್ಞಾ-ನವೋ
ತತ್ತ್ವಜ್ಞಾನಿ
ತತ್ತ್ವದ
ತತ್ತ್ವ-ದನ್ವಯ
ತತ್ತ್ವಮಸಿ
ತತ್ತ್ವವನ್ನು
ತತ್ತ್ವವಿದೆ
ತತ್ತ್ವವು
ತತ್ತ್ವವೂ
ತತ್ತ್ವವೆಂದೂ
ತತ್ತ್ವ-ವೆಂದೆಂದೂ
ತತ್ತ್ವವೇ
ತತ್ತ್ವ-ವೇತ್ತ-ನಾದ
ತತ್ತ್ವಶಃ
ತತ್ತ್ವಶಾಸ್ತ್ರ
ತತ್ತ್ವ-ಶಾಸ್ತ್ರಕ್ಕೆ
ತತ್ತ್ವ-ಶಾಸ್ತ್ರಜ್ಞರು
ತತ್ತ್ವ-ಶಾಸ್ತ್ರದ
ತತ್ತ್ವ-ಶಾಸ್ತ್ರ-ದಲ್ಲಿ
ತತ್ಪ-ರಿ-ಣಾ-ಮ-ವಾಗಿ
ತತ್ಫ-ಲ-ವಾಗಿ
ತತ್ವ
ತತ್ವಕ್ಕೆ
ತತ್ವ-ಗ-ಳನ್ನು
ತತ್ವಗಳು
ತತ್ವಜ್ಞಾನ
ತತ್ವದ
ತತ್ವ-ದಲ್ಲ-ಡ-ಗಿದೆ
ತತ್ವ-ದಿಂದಲೇ
ತತ್ವವನ್ನು
ತತ್ವ-ವೊಂದನ್ನು
ತತ್ಸಂಬಂಧ-ವಾದ
ತತ್ಸದೃಶ
ತತ್ಸಮಾನ
ತತ್ಸ-ಮಾಪ್ನೋತಿ
ತಥಾ
ತಥ್ಯ
ತಥ್ಯಗಳ
ತಥ್ಯ-ಗ-ಳನ್ನು
ತಥ್ಯ-ಗ-ಳನ್ನೊ-ಳ-ಗೊಂಡ
ತಥ್ಯ-ಗ-ಳಾ-ಗಿದ್ದವು
ತಥ್ಯಗಳು
ತಥ್ಯ-ಭ-ಗ-ವಂತ-ನಲ್ಲಿ
ತಥ್ಯ-ವಿ-ದೆಯೇ
ತದ-ನು-ಗು-ಣ-ವಾಗಿ
ತದೇಕ
ತದೇ-ಕ-ಚಿತ್ತ-ದಿಂದ
ತದ್ವಾ
ತದ್ವಿ-ರುದ್ಧ-ವಾದ
ತನ
ತನಕ
ತನಕವೂ
ತನ-ಗ-ರಿ-ವಿಲ್ಲದೆ
ತನ-ಗಾ-ಗ-ದ-ವ-ರನ್ನೂ
ತನಗಾದ
ತನಗಿಂತ
ತನ-ಗಿಂತಲೂ
ತನಗೂ
ತನಗೆ
ತನ-ಗೆ-ರ-ಡ-ನೆ-ಯ-ದಿಲ್ಲದ
ತನಗೇ
ತನಗೇನೂ
ತನು
ತನುಮನ
ತನು-ಮ-ನ-ಗ-ಳಲ್ಲಿ
ತನ್ನ
ತನ್ನಂತೆ
ತನ್ನಂತೆಯೇ
ತನ್ನ-ತ-ನ-ವನ್ನು
ತನ್ನದನ್ನೂ
ತನ್ನದಾಗಿ
ತನ್ನದು
ತನ್ನದೆಂಬ
ತನ್ನದೇ
ತನ್ನ-ದೇ-ನಾ-ದರೂ
ತನ್ನನ್ನ-ರಿ-ತು-ಕೊಳ್ಳ-ದಿ-ರುವ
ತನ್ನನ್ನು
ತನ್ನನ್ನೇ
ತನ್ನಲ್ಲ-ಡ-ಗಿದೆ
ತನ್ನಲ್ಲ-ಡ-ಗಿ-ದೆಯೋ
ತನ್ನಲ್ಲಿ
ತನ್ನಲ್ಲಿಯೇ
ತನ್ನಲ್ಲಿ-ರುವ
ತನ್ನಲ್ಲಿಲ್ಲ
ತನ್ನಲ್ಲಿವೆ
ತನ್ನಲ್ಲೇ
ತನ್ನ-ವ-ನೊಬ್ಬ-ನನ್ನು
ತನ್ನವರ
ತನ್ನಾತ್ಮವ
ತನ್ನಿ
ತನ್ನಿಂದ
ತನ್ನಿಂದಲೇ
ತನ್ನಿಂದಾದ
ತನ್ನಿಂದಾ-ದೀತೆ
ತನ್ನಿಚ್ಛೆ-ಯಂತೆ
ತನ್ನೆಡೆಗೆ
ತನ್ನೆಡೆಗೇ
ತನ್ನೆಲ್ಲ
ತನ್ನೊಂದಿಗೆ
ತನ್ನೊ-ಳ-ಗಿನ
ತನ್ಮ-ಯ-ರಾ-ಗಿದ್ದರು
ತನ್ಮೂಲಕ
ತಪಂಗ-ಳಯ್ಯಾ
ತಪಸ್ವಿ
ತಪಸ್ವಿ-ಗ-ಳನ್ನು
ತಪಸ್ವಿ-ಗಳು
ತಪಸ್ವಿಗೆ
ತಪಸ್ವಿ-ನಿಯ
ತಪಸ್ವಿಯ
ತಪಸ್ವಿ-ಯನ್ನು
ತಪಸ್ಸನ್ನಾ-ಚ-ರಿಸಿ
ತಪಸ್ಸನ್ನಾ-ಚ-ರಿ-ಸುತ್ತಿದ್ದರು
ತಪಸ್ಸಾ-ಗುತ್ತದೆ
ತಪಸ್ಸಿನ
ತಪಸ್ಸಿ-ನಂತೆ
ತಪಸ್ಸಿ-ನಲ್ಲೇ
ತಪಸ್ಸು
ತಪಸ್ಸು-ತ-ಪಸ್ವಿ-ಗಳ
ತಪಸ್ಸೆಂದು
ತಪೋನಿಷ್ಠ
ತಪೋಮಯ
ತಪೋವನ
ತಪ್ತ-ಳಾ-ಗಿದ್ದಳು
ತಪ್ಪದೆ
ತಪ್ಪದೇ
ತಪ್ಪನ್ನು
ತಪ್ಪ-ಬಾ-ರದು
ತಪ್ಪಲ್ಲ
ತಪ್ಪಾ
ತಪ್ಪಾ-ಗ-ದಂತೆ
ತಪ್ಪಾಗದು
ತಪ್ಪಾಗಿ
ತಪ್ಪಾ-ಗಿ-ರ-ಲಿಲ್ಲ
ತಪ್ಪಾ-ಗುತ್ತ-ದೆಯೋ
ತಪ್ಪಿ
ತಪ್ಪಿಗಾಗಿ
ತಪ್ಪಿಗೆ
ತಪ್ಪಿತಸ್ಥ
ತಪ್ಪಿ-ತಸ್ಥ-ರನ್ನು
ತಪ್ಪಿದರೆ
ತಪ್ಪಿದ್ದಲ್ಲ
ತಪ್ಪಿನ
ತಪ್ಪಿನಿಂದ
ತಪ್ಪಿ-ಬೀ-ಳುವ
ತಪ್ಪಿಲ್ಲ
ತಪ್ಪಿಲ್ಲದೆ
ತಪ್ಪಿಲ್ಲದೇ
ತಪ್ಪಿಸ
ತಪ್ಪಿಸದೆ
ತಪ್ಪಿ-ಸ-ಬ-ಹುದು
ತಪ್ಪಿ-ಸ-ಬೇ-ಕೆಂಬುದು
ತಪ್ಪಿಸಲು
ತಪ್ಪಿಸಿ
ತಪ್ಪಿ-ಸಿ-ಕೊಂಡಂತೆ
ತಪ್ಪಿ-ಸಿ-ಕೊಂಡ-ವರು
ತಪ್ಪಿ-ಸಿ-ಕೊಂಡು
ತಪ್ಪಿ-ಸಿ-ಕೊಳ್ಳ-ಬ-ಹು-ದೆಂಬು-ದನ್ನು
ತಪ್ಪಿ-ಸಿ-ಕೊಳ್ಳ-ಬ-ಹುದು
ತಪ್ಪಿ-ಸಿ-ಕೊಳ್ಳ-ಬೇ-ಕಾ-ದರೆ
ತಪ್ಪಿ-ಸಿ-ಕೊಳ್ಳ-ಲಾ-ಗದ
ತಪ್ಪಿ-ಸಿ-ಕೊಳ್ಳ-ಲಾ-ರದೆ
ತಪ್ಪಿ-ಸಿ-ಕೊಳ್ಳ-ಲಾ-ರದೇ
ತಪ್ಪಿ-ಸಿ-ಕೊಳ್ಳಲು
ತಪ್ಪಿ-ಸಿ-ಕೊಳ್ಳು-ವಂತಿಲ್ಲ
ತಪ್ಪಿ-ಸಿ-ಕೊಳ್ಳುವ
ತಪ್ಪಿ-ಸಿ-ಕೊಳ್ಳು-ವಂತಿಲ್ಲ
ತಪ್ಪಿಸಿತು
ತಪ್ಪಿ-ಸು-ವು-ದಕ್ಕೆ
ತಪ್ಪು
ತಪ್ಪುಗಳ
ತಪ್ಪು-ಗ-ಳನ್ನು
ತಪ್ಪು-ಗ-ಳನ್ನೂ
ತಪ್ಪು-ಗ-ಳನ್ನೆಲ್ಲ
ತಪ್ಪು-ಗ-ಳನ್ನೆ-ಸ-ಗು-ವರು
ತಪ್ಪು-ಗ-ಳಾ-ಗಿದ್ದರೆ
ತಪ್ಪು-ಗ-ಳಾ-ಗಿವೆ
ತಪ್ಪು-ಗ-ಳಾ-ದವು
ತಪ್ಪು-ಗ-ಳಾ-ದಾಗ
ತಪ್ಪು-ಗ-ಳಿಂದ
ತಪ್ಪುಗಳೂ
ತಪ್ಪು-ತಿ-ಳಿ-ದು-ಕೊಳ್ಳುತ್ತಾನೆ
ತಪ್ಪುತ್ತದೆ
ತಪ್ಪುತ್ತಿದೆ
ತಪ್ಪು-ದಾ-ರಿ-ಯಲ್ಲಿ
ತಪ್ಪುವ
ತಪ್ಪು-ವು-ದೆಂತು
ತಪ್ಪುವುದೇ
ತಪ್ಪೂ
ತಪ್ಪೊಪ್ಪಿ-ಕೊಂಡು
ತಬ್ಬಲಿ
ತಬ್ಬ-ಲಿ-ಯಾದ
ತಬ್ಬ-ಲಿ-ಯಾ-ಯಿತು
ತಬ್ಬಿ
ತಬ್ಬಿ-ಕೊಂಡರು
ತಬ್ಬಿಕೊಂಡು
ತಬ್ಬಿ-ಹಿ-ಡಿದ
ತಮ-ಗ-ರಿ-ವಿಲ್ಲದೆ
ತಮ-ಗಾ-ಗಿ-ರುವ
ತಮ-ಗಾ-ಗುತ್ತಿ-ರುವ
ತಮಗಾದ
ತಮಗಿಂತ
ತಮಗೂ
ತಮಗೆ
ತಮಭ್ಯರ್ಚ್ಯ
ತಮಸೋ
ತಮಸ್ಸನ್ನು
ತಮಾಷೆ
ತಮೋ-ಗು-ಣ-ವನ್ನು
ತಮ್ಮ
ತಮ್ಮಂದಿರೂ
ತಮ್ಮ-ತ-ನ-ವನ್ನು
ತಮ್ಮತಮ್ಮ
ತಮ್ಮ-ದಾ-ಗಿ-ಸಿ-ಕೊಳ್ಳಲು
ತಮ್ಮದು
ತಮ್ಮದೇ
ತಮ್ಮನು
ತಮ್ಮನ್ನು
ತಮ್ಮಲ್ಲಿ
ತಮ್ಮಲ್ಲಿ-ರುವ
ತಮ್ಮಷ್ಟು
ತಮ್ಮಿಂದ
ತಮ್ಮಿಂದಾದ
ತಮ್ಮೆಡೆಗೆ
ತಮ್ಮೊಡನೆ
ತಮ್ಮೊ-ಳ-ಗಿರ್ದ
ತಮ್ಮೊಳಗೆ
ತಯಾರಾಗಿ
ತಯಾ-ರಾ-ಗಿದ್ದರೂ
ತಯಾರಾದ
ತಯಾರಿ
ತಯಾ-ರಿ-ಕೆ-ಯನ್ನು
ತಯಾ-ರಿ-ಸ-ಬ-ಹುದು
ತಯಾ-ರಿ-ಸಲು
ತಯಾ-ರಿ-ಸಲ್ಪ-ಡುತ್ತಿದ್ದ
ತಯಾರಿಸಿ
ತಯಾ-ರಿ-ಸಿದ
ತಯಾ-ರಿ-ಸಿದ್ದ
ತಯಾ-ರಿ-ಸಿದ್ದರು
ತಯಾ-ರಿ-ಸಿದ್ದಾರೆ
ತಯಾ-ರಿ-ಸುತ್ತಿದ್ದ
ತಯಾ-ರಿ-ಸುತ್ತಿದ್ದರು
ತಯಾ-ರಿ-ಸು-ವ-ವರೇ
ತಯಾ-ರು-ಮಾ-ಡುತ್ತಾರೆ
ತರಂಗ
ತರಂಗ-ಗ-ಳನ್ನು
ತರಂಗ-ಗಳು
ತರಂಗದ
ತರಂಗ-ದೂ-ರಕ್ಕೆ
ತರಂಗ-ವೊಂದನ್ನು
ತರಕಾರಿ
ತರ-ಕಾ-ರಿ-ಯನ್ನು
ತರ-ಗ-ತಿ-ಗ-ಳಿಂದಲೇ
ತರ-ಗ-ತಿಯ
ತರ-ಗ-ತಿ-ಯಲ್ಲಿ
ತರ-ಗ-ತಿ-ಯಲ್ಲಿ-ರು-ವಾಗ
ತರ-ಗ-ತಿ-ಯ-ವ-ರೆಗೆ
ತರ-ಗ-ತಿ-ಯಷ್ಟ-ರದು
ತರ-ಗ-ತಿ-ಯಿಂದ
ತರ-ಗೆ-ಲೆ-ಗ-ಳಂತೆ
ತರದ
ತರಬಲ್ಲ
ತರ-ಬಲ್ಲದು
ತರ-ಬಲ್ಲರು
ತರ-ಬ-ಹು-ದಾ-ದರೂ
ತರ-ಬ-ಹುದು
ತರ-ಬೇ-ಕಾ-ಗಿ-ರುವ
ತರ-ಬೇ-ಕೆಂದು
ತರ-ಬೇ-ತನ್ನು
ತರ-ಬೇ-ತನ್ನೂ
ತರಬೇತಿ
ತರ-ಬೇ-ತಿ-ಗ-ಳನ್ನು
ತರ-ಬೇ-ತಿ-ಗ-ಳಿಂದ
ತರ-ಬೇ-ತಿ-ಗಾಗಿ
ತರ-ಬೇ-ತಿಯ
ತರ-ಬೇ-ತಿ-ಯನ್ನು
ತರ-ಬೇ-ತಿ-ಯನ್ನೂ
ತರ-ಬೇ-ತಿಯೂ
ತರಬೇತು
ತರಲು
ತರಹ
ತರಹದ
ತರಿದು
ತರುಣ
ತರುಣನ
ತರು-ಣ-ನಾಗಿ
ತರು-ಣ-ನೊಬ್ಬ
ತರುಣರ
ತರುಣರು
ತರು-ಣ-ರೆಲ್ಲರೂ
ತರು-ಣ-ರೊಬ್ಬರು
ತರುಣಿ
ತರುಣಿಗೆ
ತರು-ಣಿ-ಯನ್ನು
ತರು-ಣಿ-ಯೊಬ್ಬಳು
ತರು-ಣಿ-ಯೋರ್ವಳ
ತರುತ್ತದೆ
ತರುತ್ತ-ಲಿದ್ದೇನೆ
ತರುತ್ತಿದ್ದಾರೆ
ತರುತ್ತಿ-ರ-ಬ-ಹು-ದೆಂದು
ತರುತ್ತಿ-ರ-ಲಿಲ್ಲ
ತರುವ
ತರುವಂಥ
ತರು-ವ-ವರು
ತರು-ವು-ದಕ್ಕಾಗಿ
ತರುವುದು
ತರು-ವು-ದೆಲ್ಲಿಂದ
ತರುವುದೇ
ತರುವೃಕ್ಷ
ತರ್ಕ
ತರ್ಕ-ಕು-ತರ್ಕ-ಗ-ಳನ್ನು
ತರ್ಕದ
ತರ್ಕದಿಂದ
ತರ್ಕ-ಬದ್ಧ-ವಾಗಿ
ತರ್ಕ-ಬದ್ಧ-ವಾ-ಗಿದೆ
ತರ್ಕ-ಯುಕ್ತಿ-ಗಳ
ತರ್ಕ-ವಿ-ತರ್ಕ-ಗಳು
ತರ್ಕಶಕ್ತಿ
ತಲದಲ್ಲಿ
ತಲ-ಪ-ಬೇ-ಕೆಂದಾ-ಗದು
ತಲಪಲು
ತಲಪಿ
ತಲಸ್ಪರ್ಶಿ-ಯಾಗಿ
ತಲಸ್ಪರ್ಶಿಯೂ
ತಲಾಂತ-ರ-ದಿಂದ
ತಲು-ಪ-ಬ-ಹುದು
ತಲು-ಪ-ಬೇ-ಕಿದ್ದರೆ
ತಲುಪಲು
ತಲುಪಿತು
ತಲುಪಿದ
ತಲು-ಪಿ-ದಾಗ
ತಲುಪಿವೆ
ತಲು-ಪುತ್ತವೆ
ತಲು-ಪುತ್ತಾನೆ
ತಲುಪುವ
ತಲು-ಪು-ವಂತೆ
ತಲು-ಪು-ವಾಗ
ತಲೆ
ತಲೆ-ಕೆ-ಳ-ಗಾಗಿ
ತಲೆಕೊಟ್ಟು
ತಲೆಗೆ
ತಲೆಗೊಂದು
ತಲೆಚಿಟ್ಟು
ತಲೆ-ತ-ಲಾಂತ-ರ-ದಿಂದ
ತಲೆ-ತಿ-ರುಗು
ತಲೆ-ತೂ-ಗಿದ
ತಲೆ-ದಿಂಬಾಗಿ
ತಲೆದಿಂಬು
ತಲೆ-ದೂ-ಗುತ್ತಿದ್ದ-ರಂತೆ
ತಲೆ-ದೂ-ಗು-ವಂತೆಯೂ
ತಲೆ-ದೋ-ರುತ್ತದೆ
ತಲೆ-ನೋ-ವಾಗಿ
ತಲೆ-ನೋ-ವಿನ
ತಲೆನೋವು
ತಲೆನೋವೇ
ತಲೆಬಾಗಿ
ತಲೆ-ಬಾ-ಗಿ-ದರು
ತಲೆ-ಬಾ-ಗಿ-ದರೂ
ತಲೆ-ಬಾ-ಗು-ವಂತೆ
ತಲೆಬಿಸಿ
ತಲೆ-ಬು-ಡ-ವಿಲ್ಲದ
ತಲೆ-ಬು-ರು-ಡೆ-ಯನ್ನು
ತಲೆ-ಬು-ರು-ಡೆ-ಯಿಂದ
ತಲೆಭಾರ
ತಲೆ-ಮಾ-ರಿನ
ತಲೆ-ಮಾ-ರಿ-ನ-ವರು
ತಲೆ-ಮಾ-ರು-ಗ-ಳಲ್ಲಿ
ತಲೆಯ
ತಲೆಯನ್ನು
ತಲೆ-ಯ-ಮೇಲೆ
ತಲೆಯಲ್ಲಿ
ತಲೆ-ಯಿ-ರಿಸಿ
ತಲೆಯೇ
ತಲೆ-ಯೊ-ಳಗೆ
ತಲೆ-ಸುತ್ತಿದ್ದು
ತಲೆಸುತ್ತು
ತಲೆ-ಹಾ-ಕುತ್ತಿ-ರ-ಲಿಲ್ಲ
ತಲ್ಲ-ಣ-ಗೊ-ಳಿ-ಸ-ಲೂ-ಬ-ಹುದು
ತಲ್ಲ-ಣ-ಗೊ-ಳಿ-ಸಿದ್ದವು
ತಲ್ಲ-ಣಿ-ಸ-ದಿರು
ತಲ್ಲ-ಣಿ-ಸುತ್ತದೆ
ತಲ್ಲೀ-ನ-ರಾದ
ತಳಕ್ಕೆ
ತಳದಲ್ಲಿ
ತಳ-ಪಾ-ಯದ
ತಳಮಳ
ತಳ-ಮ-ಳ-ದಿಂದ
ತಳ-ಮ-ಳ-ವನ್ನುಂಟು-ಮಾಡಿ
ತಳವನ್ನು
ತಳಹದಿ
ತಳ-ಹ-ದಿಯೆ
ತಳ-ಹ-ದಿಯೇ
ತಳೆದ
ತಳೆದರೆ
ತಳೆದಿಲ್ಲ
ತಳೆದು
ತಳೆ-ಯುತ್ತದೆ
ತಳೆ-ಯುತ್ತಾನೆ
ತಳೆಯುವು
ತಳ್ಳಲು
ತಳ್ಳಿ
ತಳ್ಳಿದ
ತಳ್ಳಿ-ದಂತಾ-ಗುತ್ತದೆ
ತಳ್ಳಿದೆವು
ತಳ್ಳಿದ್ದಾರೆ
ತಳ್ಳಿನನ್ನ
ತಳ್ಳಿ-ಹಾ-ಕ-ಲಿಲ್ಲ
ತಳ್ಳುತ್ತ
ತಳ್ಳುತ್ತದೆ
ತಳ್ಳುತ್ತಿದೆ
ತಳ್ಳುತ್ತಿ-ವೆ-ಯಲ್ಲ
ತಳ್ಳುವ
ತಳ್ಳು-ವಂತಾ-ದರೆ
ತಳ್ಳುವಂಥ
ತಳ್ಳುವರು
ತವಕ
ತವಕಿಸಿ
ತವ-ಕಿ-ಸು-ವುದು
ತವರಾಗಿ
ತವ-ರಾ-ಗುತ್ತಾನೆ
ತವರಿಗೆ
ತವರು
ತವ-ರು-ಮನೆ
ತಾ
ತಾಂಡವ
ತಾಂಡವಕ್ಕೆ
ತಾಂಡ-ವ-ವಾ-ಡುತ್ತದೆ
ತಾಂಡ-ವ-ವಾ-ಡುತ್ತಿ-ರು-ವುದು
ತಾಂಡ-ವ-ವಾ-ಡುತ್ತಿವೆ
ತಾಂಡ-ವ-ವಾ-ಡುವ
ತಾಂತ್ರಿಕ
ತಾಂತ್ರಿ-ಕ-ತಜ್ಞನೇ
ತಾಂತ್ರಿ-ಕ-ತೆಯ
ತಾಕ-ಲಾ-ಟ-ಗ-ಳಿಂದ
ತಾಕ-ಲಾ-ಟ-ದಲ್ಲಿ
ತಾಗಿ
ತಾಗಿದಾಗ
ತಾಗಿ-ಸಿ-ದಾಗ
ತಾಣ
ತಾಣದಲ್ಲಿ
ತಾಣ-ವಾ-ಗುತ್ತದೆ
ತಾಣವಾದ
ತಾತ್ಕಾಲಿಕ
ತಾತ್ತ್ವಿಕ
ತಾತ್ತ್ವಿಕನ
ತಾತ್ತ್ವಿಕರ
ತಾತ್ಪರ್ಯ
ತಾತ್ವಿಕ
ತಾದಾತ್ಮ್ಯ
ತಾದಾತ್ಮ್ಯ-ಗೊಂಡಾಗ
ತಾದಾತ್ಮ್ಯ-ದಿಂದಲೇ
ತಾದಾತ್ಮ್ಯ-ವನ್ನು
ತಾದಾತ್ಮ್ಯವೂ
ತಾನ-ರಿ-ಯುವ
ತಾನ-ವ-ಳನ್ನು
ತಾನಾಗಿ
ತಾನಾಗಿಯೇ
ತಾನಿರುವ
ತಾನಿ-ರು-ವೆ-ನೆಂದೂ
ತಾನಿಲ್ಲ-ದಿದ್ದರೆ
ತಾನಿಷ್ಟು
ತಾನು
ತಾನೂ
ತಾನೆ
ತಾನೆ-ಸ-ಗ-ಬೇ-ಕೆಂದು
ತಾನೇ
ತಾನೇ-ತಾ-ನಾಗಿ
ತಾನೇನು
ತಾನೇನೂ
ತಾನೊಂದು
ತಾನೊಬ್ಬ
ತಾನೊಬ್ಬನೇ
ತಾಪ
ತಾಪಗಳ
ತಾಪಗಳು
ತಾಪತ್ರ-ಯ-ಗಳ
ತಾಪತ್ರ-ಯ-ಗ-ಳಲ್ಲಿ
ತಾಪತ್ರ-ಯ-ಗ-ಳಿಂದ
ತಾಪತ್ರ-ಯ-ಗಳು
ತಾಪತ್ರ-ಯ-ವಿನ್ನೂ
ತಾಪತ್ರ-ಯ-ವಿಲ್ಲ
ತಾಪತ್ರ-ಯ-ವಿಲ್ಲ-ವೆಂದು
ತಾಪದಿಂದ
ತಾಪ-ಸಿ-ಯಾಗಿ
ತಾಮಸ
ತಾಮಸಿಕ
ತಾಯಂದಿರ
ತಾಯಂದಿ-ರಿಂದ
ತಾಯಂದಿ-ರಿಗೆ
ತಾಯಂದಿರು
ತಾಯಂದಿರೂ
ತಾಯಂದಿರೆ
ತಾಯಂದಿ-ರೊ-ಡನೆ
ತಾಯಿ
ತಾಯಿಗಳ
ತಾಯಿ-ಗ-ಳಿಂದ
ತಾಯಿ-ಗ-ಳಿಗೆ
ತಾಯಿ-ಗ-ಳಿ-ಗೇಕೆ
ತಾಯಿಗಳು
ತಾಯಿಗೂ
ತಾಯಿಗೆ
ತಾಯಿತಂದೆ
ತಾಯಿ-ತಂದೆ-ಗಳ
ತಾಯಿ-ಬೇ-ರಿ-ನಿಂದ
ತಾಯಿಮಗ
ತಾಯಿಮಗು
ತಾಯಿಯ
ತಾಯಿ-ಯಂತೆಯೇ
ತಾಯಿಯನ್ನು
ತಾಯಿಯಲ್ಲಿ
ತಾಯಿಯಾದ
ತಾಯಿ-ಯಾ-ದರೋ
ತಾಯಿ-ಯಾ-ದ-ವ-ಳಿಗೆ
ತಾಯಿಯಿಂದ
ತಾಯಿ-ಯಿಲ್ಲದ
ತಾಯಿಯು
ತಾಯಿಯೆಂದು
ತಾಯಿ-ಯೊ-ಡನೆ
ತಾಯಿ-ಯೊಬ್ಬರು
ತಾಯಿ-ಯೊಬ್ಬಳ
ತಾಯಿ-ಯೊಬ್ಬಳು
ತಾಯೇ
ತಾಯ್ತನ
ತಾಯ್ತನದ
ತಾಯ್ನಾ-ಡಿ-ನಿಂದ
ತಾರ-ಕ-ಮಂತ್ರ
ತಾರ-ಕ-ವಾಗು
ತಾರ-ಕ-ವಾ-ಗುತ್ತಿದ್ದು-ದನ್ನು
ತಾರತಮ್ಯ
ತಾರ-ತಮ್ಯದ
ತಾರ-ತಮ್ಯ-ವಿಲ್ಲದೆ
ತಾರ-ತಮ್ಯವೇ
ತಾರಾ-ಚಂದರು
ತಾರಿ-ಕೆ-ಗ-ಳಂತೆ
ತಾರೀಕಿನ
ತಾರೀಖು
ತಾರುಣ್ಯ
ತಾರುಣ್ಯ-ದಲ್ಲಾ-ಗಲೀ
ತಾರುಣ್ಯ-ದಿಂದಲೇ
ತಾರೆ
ತಾರ್ಕಿ-ಕ-ವಾಗಿ
ತಾಳಕ್ಕೆ
ತಾಳ-ದ-ವನು
ತಾಳಬೇಕು
ತಾಳ-ಲ-ಯದ
ತಾಳಲಾರ
ತಾಳ-ಲಾ-ರದೆ
ತಾಳ-ಲಾ-ರದೇ
ತಾಳಿ
ತಾಳಿದ
ತಾಳಿದಂತೆ
ತಾಳಿದವ
ತಾಳಿದೆ
ತಾಳಿ-ಬಾ-ಳಲು
ತಾಳು
ತಾಳುತ್ತವೆ
ತಾಳ್ಮೆ
ತಾಳ್ಮೆಗಳ
ತಾಳ್ಮೆ-ಗ-ಳಿಗೇ
ತಾಳ್ಮೆಗೆ
ತಾಳ್ಮೆಗೆಟ್ಟ
ತಾಳ್ಮೆ-ಗೆಟ್ಟರೆ
ತಾಳ್ಮೆಗೆಟ್ಟು
ತಾಳ್ಮೆ-ಗೆ-ಡದೆ
ತಾಳ್ಮೆ-ಗೆ-ಡ-ಬೇಡ
ತಾಳ್ಮೆಯ
ತಾಳ್ಮೆಯನ್ನು
ತಾಳ್ಮೆಯಿಂದ
ತಾಳ್ಮೆ-ಯಿಂದಲೇ
ತಾಳ್ಮೆಯೇ
ತಾವು
ತಾವೂ
ತಾವೆಂಥ
ತಾವೆಲ್ಲ
ತಾವೇ
ತಾವೇನು
ತಾವೊ
ತಿಂಗಳ
ತಿಂಗಳಲ್ಲಿ
ತಿಂಗಳಲ್ಲೇ
ತಿಂಗಳಿಗೆ
ತಿಂಗ-ಳಿ-ಗೊಮ್ಮೆ
ತಿಂಗಳು
ತಿಂಗ-ಳು-ಗಳ
ತಿಂಗ-ಳು-ಗ-ಳನ್ನು
ತಿಂಗ-ಳು-ಗ-ಳಲ್ಲಿ
ತಿಂಗ-ಳು-ಗ-ಳಲ್ಲಿಯೇ
ತಿಂಗ-ಳು-ಗ-ಳಲ್ಲೆ
ತಿಂಗ-ಳು-ಗ-ಳಲ್ಲೇ
ತಿಂಗ-ಳು-ಗ-ಳ-ವ-ರೆಗೂ
ತಿಂಗ-ಳು-ಗ-ಳಿಂದ
ತಿಂಗ-ಳು-ಗ-ಳಿದ್ದರು
ತಿಂಗಳೂ
ತಿಂಗ-ಳೊ-ಳಗೆ
ತಿಂಡಿ
ತಿಂಡಿ-ಗ-ಳಲ್ಲಿ
ತಿಂಡಿಗಾಗಿ
ತಿಂಡಿ-ತಿ-ನಿ-ಸನ್ನು
ತಿಂಡಿ-ತೀರ್ಥ-ಗ-ಳನ್ನೀ-ಯುವ
ತಿಂಡಿಯ
ತಿಂಡಿ-ಯಂಗ-ಡಿ-ಗ-ಳನ್ನು
ತಿಂಡಿಯನ್ನು
ತಿಂದ
ತಿಂದವರು
ತಿಂದಿತೇ
ತಿಂದು
ತಿಂದುಕೊಂಡು
ತಿಂದೇ
ತಿಂದೇ-ಬಿಟ್ಟರು
ತಿದ್ದ-ಲಾ-ಗಿದೆ
ತಿದ್ದಲು
ತಿದ್ದಿ
ತಿದ್ದಿ-ಕೊಳ್ಳ-ಬ-ಹುದು
ತಿದ್ದಿ-ಕೊಳ್ಳ-ಬೇ-ಕಾ-ದರೆ
ತಿದ್ದಿ-ಕೊಳ್ಳಲು
ತಿದ್ದಿ-ಕೊಳ್ಳು-ವು-ದಕ್ಕೆ
ತಿದ್ದಿ-ಬೆ-ಳೆ-ಸಲು
ತಿದ್ದುತ್ತ
ತಿದ್ದು-ಪ-ಡಿ-ಗ-ಳನ್ನು
ತಿದ್ದು-ಪ-ಡಿಗೆ
ತಿದ್ದುವ
ತಿದ್ದುವಂತೆ
ತಿದ್ದುವಲ್ಲಿ
ತಿದ್ದು-ವ-ವರು
ತಿದ್ದುವುದು
ತಿನಿಸು
ತಿನ್ನದೆ
ತಿನ್ನ-ಲಾ-ಗದ
ತಿನ್ನ-ಲಾ-ರದ
ತಿನ್ನಲು
ತಿನ್ನಿಸಿ
ತಿನ್ನಿ-ಸುತ್ತಿದ್ದರು
ತಿನ್ನಿ-ಸುತ್ತಿದ್ದಾ-ರೆಂದು
ತಿನ್ನುತ್ತಾರೆ
ತಿನ್ನುತ್ತಿ-ದೆಯೋ
ತಿನ್ನುತ್ತಿದ್ದೀಯೇ
ತಿನ್ನುತ್ತಿ-ರು-ವು-ದ-ರಿಂದ
ತಿನ್ನುತ್ತಿ-ವೆ-ಯಲ್ಲ
ತಿನ್ನುವ
ತಿನ್ನುವಿಯೋ
ತಿನ್ನುವುದು
ತಿಪ್ಪೇಸ್ವಾಮಿ
ತಿರಸ್ಕ-ರಿ-ಸದೇ
ತಿರಸ್ಕ-ರಿ-ಸ-ಬಲ್ಲರು
ತಿರಸ್ಕ-ರಿ-ಸ-ಬ-ಹುದು
ತಿರಸ್ಕ-ರಿ-ಸಿ-ದರೆ
ತಿರಸ್ಕ-ರಿ-ಸಿ-ದ-ವರ
ತಿರಸ್ಕ-ರಿ-ಸಿ-ದಷ್ಟು
ತಿರಸ್ಕ-ರಿ-ಸುತ್ತಿದ್ದ
ತಿರಸ್ಕ-ರಿ-ಸುತ್ತೇನೆ
ತಿರಸ್ಕ-ರಿ-ಸು-ವರು
ತಿರಸ್ಕ-ರಿ-ಸು-ವ-ವರು
ತಿರಸ್ಕಾರ
ತಿರಸ್ಕಾ-ರಕ್ಕೊ-ಳ-ಗಾದ
ತಿರಸ್ಕಾ-ರ-ಗ-ಳಿಗೆ
ತಿರಸ್ಕಾ-ರ-ಗಳು
ತಿರಸ್ಕಾ-ರದ
ತಿರಸ್ಕಾ-ರ-ದಿಂದ
ತಿರಸ್ಕಾ-ರ-ಪೂ-ರಿತ
ತಿರಸ್ಕಾ-ರವು
ತಿರಸ್ಕೃತ
ತಿರಸ್ಕೃ-ತ-ನಾದ
ತಿರಸ್ಕೃ-ತರ
ತಿರು
ತಿರುಕನ
ತಿರು-ಗ-ಲಾ-ರಂಭಿ-ಸಿತು
ತಿರುಗಲು
ತಿರು-ಗಾ-ಟ-ಎಲ್ಲಕ್ಕೂ
ತಿರು-ಗಾ-ಡ-ಬೇ-ಕಿತ್ತು
ತಿರು-ಗಾ-ಡಲು
ತಿರು-ಗಾ-ಡು-ವಷ್ಟು
ತಿರುಗಿ
ತಿರುಗಿತು
ತಿರು-ಗಿ-ದಾಗ
ತಿರು-ಗಿ-ಬಂದು
ತಿರು-ಗಿ-ಸ-ಬೇಕು
ತಿರುಗಿಸಿ
ತಿರು-ಗಿ-ಸಿ-ದಾಗ
ತಿರು-ಗಿ-ಸುತ್ತ
ತಿರು-ಗಿ-ಸುತ್ತದೆ
ತಿರು-ಗಿ-ಸು-ವುದು
ತಿರುಗು
ತಿರು-ಗುತ್ತಿದ್ದರು
ತಿರು-ಗುತ್ತಿದ್ದರೂ
ತಿರು-ಗುತ್ತಿ-ರುವ
ತಿರು-ಗುತ್ತಿ-ರು-ವಂತೆ
ತಿರು-ಗುತ್ತಿವೆ
ತಿರು-ಗು-ಬಾ-ಣ-ವಾಗಿ
ತಿರು-ಗು-ವಲ್ಲಿ
ತಿರು-ಗು-ವು-ದುಂಟು
ತಿರು-ತಿ-ರುಗಿ
ತಿರುಳನ್ನು
ತಿರುಳು
ತಿರು-ವಣ್ಣಾ-ಮ-ಲೆಯ
ತಿರುವನ್ನು
ತಿರುವು
ತಿಲಾಂಜ-ಲಿ-ಯಿತ್ತು
ತಿಳಿ
ತಿಳಿ-ವ-ಳಿ-ಕೆ-ಯನ್ನೇ
ತಿಳಿಗೇಡಿ
ತಿಳಿದ
ತಿಳಿ-ದಂತಿದೆ
ತಿಳಿ-ದಂತಿಲ್ಲ
ತಿಳಿ-ದದ್ದುಂಟು
ತಿಳಿದರೂ
ತಿಳಿದರೆ
ತಿಳಿದಳು
ತಿಳಿ-ದ-ವನು
ತಿಳಿ-ದ-ವ-ರಿಗೆ
ತಿಳಿ-ದ-ವರು
ತಿಳಿ-ದ-ವ-ರುಂಟು
ತಿಳಿದಾಗ
ತಿಳಿದಾತ
ತಿಳಿದಿತ್ತು
ತಿಳಿದಿದೆ
ತಿಳಿ-ದಿ-ದೆಯೇ
ತಿಳಿದಿದ್ದ
ತಿಳಿ-ದಿದ್ದರು
ತಿಳಿ-ದಿದ್ದರೋ
ತಿಳಿ-ದಿದ್ದೇವೆ
ತಿಳಿ-ದಿ-ರ-ದಿದ್ದ
ತಿಳಿ-ದಿ-ರ-ಬೇಕು
ತಿಳಿ-ದಿ-ರಲಿ
ತಿಳಿ-ದಿ-ರ-ಲಿಲ್ಲ
ತಿಳಿ-ದಿ-ರಲು
ತಿಳಿ-ದಿ-ರು-ವು-ದ-ರಿಂದ
ತಿಳಿ-ದಿ-ರು-ವು-ದಿಲ್ಲ
ತಿಳಿ-ದಿ-ರು-ವುದು
ತಿಳಿದಿಲ್ಲ
ತಿಳಿ-ದಿಲ್ಲ-ವೆಂದೇ
ತಿಳಿ-ದಿಲ್ಲವೇ
ತಿಳಿ-ದೀ-ತೆಂಬ
ತಿಳಿದು
ತಿಳಿ-ದು-ಕೊಂಡ-ವರು
ತಿಳಿ-ದು-ಕೊಂಡಿದ್ದರು
ತಿಳಿ-ದು-ಕೊಳ್ಳು-ವು-ದಿಲ್ಲ
ತಿಳಿ-ದು-ಕೊಳ್ಳು-ವುದೇ
ತಿಳಿ-ದು-ಕೊಂಡ
ತಿಳಿ-ದು-ಕೊಂಡಂತಾ-ಗು-ವುದು
ತಿಳಿ-ದು-ಕೊಂಡಂತಿದೆ
ತಿಳಿ-ದು-ಕೊಂಡಂತೆ
ತಿಳಿ-ದು-ಕೊಂಡದ್ದನ್ನು
ತಿಳಿ-ದು-ಕೊಂಡದ್ದು
ತಿಳಿ-ದು-ಕೊಂಡರು
ತಿಳಿ-ದು-ಕೊಂಡರೂ
ತಿಳಿ-ದು-ಕೊಂಡರೆ
ತಿಳಿ-ದು-ಕೊಂಡ-ರೇನೇ
ತಿಳಿ-ದು-ಕೊಂಡಲ್ಲಿ
ತಿಳಿ-ದು-ಕೊಂಡ-ವ-ನಿಗೂ
ತಿಳಿ-ದು-ಕೊಂಡ-ವನು
ತಿಳಿ-ದು-ಕೊಂಡ-ವ-ರಲ್ಲಿ
ತಿಳಿ-ದು-ಕೊಂಡ-ವ-ರಲ್ಲೂ
ತಿಳಿ-ದು-ಕೊಂಡ-ವರು
ತಿಳಿ-ದು-ಕೊಂಡಾಗ
ತಿಳಿ-ದು-ಕೊಂಡಿ
ತಿಳಿ-ದು-ಕೊಂಡಿತು
ತಿಳಿ-ದು-ಕೊಂಡಿದ್ದ
ತಿಳಿ-ದು-ಕೊಂಡಿದ್ದನೊ
ತಿಳಿ-ದು-ಕೊಂಡಿದ್ದರು
ತಿಳಿ-ದು-ಕೊಂಡಿದ್ದರೂ
ತಿಳಿ-ದು-ಕೊಂಡಿದ್ದರೆ
ತಿಳಿ-ದು-ಕೊಂಡಿದ್ದಾ-ಗಲೂ
ತಿಳಿ-ದು-ಕೊಂಡಿದ್ದಾರೆ
ತಿಳಿ-ದು-ಕೊಂಡಿದ್ದಾಳೆ
ತಿಳಿ-ದು-ಕೊಂಡಿದ್ದೇನೆ
ತಿಳಿ-ದು-ಕೊಂಡಿ-ರ-ಬ-ಹುದು
ತಿಳಿ-ದು-ಕೊಂಡಿಲ್ಲ
ತಿಳಿ-ದು-ಕೊಂಡು
ತಿಳಿ-ದು-ಕೊಂಡೆ
ತಿಳಿ-ದು-ಕೊಳ್ಳದ
ತಿಳಿ-ದು-ಕೊಳ್ಳದೆ
ತಿಳಿ-ದು-ಕೊಳ್ಳದೇ
ತಿಳಿ-ದು-ಕೊಳ್ಳ-ಬಲ್ಲ
ತಿಳಿ-ದು-ಕೊಳ್ಳ-ಬಲ್ಲರು
ತಿಳಿ-ದು-ಕೊಳ್ಳ-ಬಹು
ತಿಳಿ-ದು-ಕೊಳ್ಳ-ಬ-ಹುದು
ತಿಳಿ-ದು-ಕೊಳ್ಳ-ಬೇಕು
ತಿಳಿ-ದು-ಕೊಳ್ಳ-ಬೇ-ಕು-ಅಜ್ಞಾ-ನಿ-ಯಾ-ಗಿ-ರ-ಬಾ-ರದು
ತಿಳಿ-ದು-ಕೊಳ್ಳ-ಲಾರ
ತಿಳಿ-ದು-ಕೊಳ್ಳಲಿ
ತಿಳಿ-ದು-ಕೊಳ್ಳಲು
ತಿಳಿ-ದು-ಕೊಳ್ಳ-ಲೇ-ಬೇ-ಕಾ-ದಂಥ
ತಿಳಿ-ದು-ಕೊಳ್ಳಿ
ತಿಳಿ-ದು-ಕೊಳ್ಳುತ್ತಾರೆ
ತಿಳಿ-ದು-ಕೊಳ್ಳುತ್ತಿದ್ದಾರೆ
ತಿಳಿ-ದು-ಕೊಳ್ಳುತ್ತಿಲ್ಲ
ತಿಳಿ-ದು-ಕೊಳ್ಳುತ್ತೇನೆ
ತಿಳಿ-ದು-ಕೊಳ್ಳು-ವಂತಾ-ದರೆ
ತಿಳಿ-ದು-ಕೊಳ್ಳು-ವಂತೆ
ತಿಳಿ-ದು-ಕೊಳ್ಳು-ವ-ವರೂ
ತಿಳಿ-ದು-ಕೊಳ್ಳು-ವು-ದ-ರಿಂದ
ತಿಳಿ-ದು-ಕೊಳ್ಳು-ವುದು
ತಿಳಿ-ದು-ಕೊಳ್ಳು-ವುದೂ
ತಿಳಿ-ದು-ಕೊಳ್ಳು-ವುದೇ
ತಿಳಿ-ದು-ದಿಷ್ಟು-ಅ-ವಳು
ತಿಳಿ-ದು-ಬಂತು
ತಿಳಿ-ದು-ಬಂದ
ತಿಳಿ-ದು-ಬಂದರೆ
ತಿಳಿ-ದು-ಬಂದಿದೆ
ತಿಳಿ-ದು-ಬ-ರು-ವುದು
ತಿಳಿ-ದು-ಬಿ-ಡುತ್ತದೆ
ತಿಳಿದೂ
ತಿಳಿದೊ
ತಿಳಿದೋ
ತಿಳಿಯ
ತಿಳಿ-ಯ-ಲಾದ
ತಿಳಿ-ಯ-ಗೊ-ಡ-ಲಿಲ್ಲ
ತಿಳಿ-ಯ-ತೊ-ಡ-ಗಿದ್ದಾರೆ
ತಿಳಿಯದ
ತಿಳಿ-ಯ-ದಂತೆ
ತಿಳಿ-ಯ-ದ-ವ-ರಂತೆ
ತಿಳಿ-ಯ-ದಾ-ದರು
ತಿಳಿ-ಯ-ದಿದ್ದ
ತಿಳಿ-ಯ-ದಿದ್ದರೂ
ತಿಳಿ-ಯ-ದಿದ್ದರೆ
ತಿಳಿ-ಯ-ದಿದ್ದಲ್ಲಿ
ತಿಳಿ-ಯ-ದಿದ್ದುದು
ತಿಳಿ-ಯ-ದಿ-ರದು
ತಿಳಿ-ಯ-ದಿ-ರ-ಬ-ಹುದು
ತಿಳಿ-ಯ-ದಿ-ರಲಿ
ತಿಳಿಯದು
ತಿಳಿಯದೆ
ತಿಳಿ-ಯ-ದೆಂದು
ತಿಳಿ-ಯ-ದೆಂದೂ
ತಿಳಿ-ಯ-ದೆ-ಅಲ್ಲಿನ
ತಿಳಿ-ಯ-ದೆಯೊ
ತಿಳಿ-ಯ-ದೆಯೋ
ತಿಳಿಯದೇ
ತಿಳಿ-ಯದ್ದನ್ನು
ತಿಳಿಯದ್ದು
ತಿಳಿ-ಯ-ಪ-ಡಿ-ಸಿ-ದರು
ತಿಳಿ-ಯ-ಬ-ಯ-ಸಿ-ದರು
ತಿಳಿ-ಯ-ಬಲ್ಲ
ತಿಳಿ-ಯ-ಬಲ್ಲದು
ತಿಳಿ-ಯ-ಬಲ್ಲರು
ತಿಳಿ-ಯ-ಬಲ್ಲವು
ತಿಳಿ-ಯ-ಬ-ಹುದು
ತಿಳಿ-ಯ-ಬಾ-ರದು
ತಿಳಿ-ಯ-ಬಾ-ರ-ದೆಂದಲ್ಲ
ತಿಳಿ-ಯ-ಬೇ-ಕಾ-ಗಿದೆ
ತಿಳಿ-ಯ-ಬೇ-ಕಾ-ಗು-ವುದು
ತಿಳಿ-ಯ-ಬೇ-ಕಾದ
ತಿಳಿ-ಯ-ಬೇ-ಕಾ-ದರೆ
ತಿಳಿ-ಯ-ಬೇ-ಕಿತ್ತು
ತಿಳಿ-ಯ-ಬೇಕು
ತಿಳಿ-ಯ-ಬೇ-ಕೆನ್ನುವ
ತಿಳಿ-ಯ-ಬೇಡ
ತಿಳಿಯರೇ
ತಿಳಿ-ಯ-ಲಾ-ಗದ
ತಿಳಿ-ಯ-ಲಾ-ಗದು
ತಿಳಿ-ಯ-ಲಾ-ಗ-ಲಿಲ್ಲ-ವೇಕೆ
ತಿಳಿ-ಯ-ಲಾ-ಗಿದೆ
ತಿಳಿ-ಯ-ಲಾದ
ತಿಳಿ-ಯ-ಲಾ-ರದ
ತಿಳಿ-ಯ-ಲಾ-ರೆವು
ತಿಳಿಯಲು
ತಿಳಿ-ಯ-ಲೇ-ಬೇಕು
ತಿಳಿ-ಯ-ಲೋ-ಸು-ಗವೇ
ತಿಳಿಯಿತು
ತಿಳಿಯಿರಿ
ತಿಳಿ-ಯು-ವಷ್ಟು
ತಿಳಿ-ಯು-ವು-ದಿಲ್ಲ
ತಿಳಿ-ಯುತ್ತದೆ
ತಿಳಿ-ಯುತ್ತಾನೋ
ತಿಳಿ-ಯುತ್ತಾ-ರಂತೆ
ತಿಳಿ-ಯುತ್ತಾರೆ
ತಿಳಿ-ಯುತ್ತಿದೆ
ತಿಳಿ-ಯುತ್ತಿದ್ದರು
ತಿಳಿ-ಯುತ್ತಿ-ರುತ್ತದೆ
ತಿಳಿ-ಯುತ್ತೇನೆ
ತಿಳಿಯುವ
ತಿಳಿ-ಯು-ವಂತಾ-ಗಿದೆ
ತಿಳಿ-ಯು-ವಂತಾ-ದರೆ
ತಿಳಿ-ಯು-ವಂತಿಲ್ಲ
ತಿಳಿ-ಯು-ವಷ್ಟು
ತಿಳಿ-ಯು-ವು-ದಕ್ಕಾಗಿ
ತಿಳಿ-ಯು-ವು-ದಕ್ಕೆ
ತಿಳಿ-ಯು-ವು-ದಿಲ್ಲ
ತಿಳಿ-ಯು-ವು-ದಿಷ್ಟು
ತಿಳಿ-ಯು-ವುದು
ತಿಳಿ-ಯು-ವು-ದುಂಟು
ತಿಳಿಯೆ
ತಿಳಿವನ್ನು
ತಿಳಿ-ವ-ಳಿಕೆ
ತಿಳಿ-ವ-ಳಿ-ಕೆಗೆ
ತಿಳಿ-ವ-ಳಿ-ಕೆ-ಗೊ-ಳ-ಗಾ-ಗ-ಲಿಲ್ಲ
ತಿಳಿ-ವ-ಳಿ-ಕೆ-ಯಂತೆ
ತಿಳಿ-ವ-ಳಿ-ಕೆ-ಯನ್ನು
ತಿಳಿ-ವ-ಳಿ-ಕೆ-ಯಾ-ಗು-ವ-ವ-ರೆಗೂ
ತಿಳಿ-ವ-ಳಿ-ಕೆ-ಯಿಂದ
ತಿಳಿ-ವ-ಳಿ-ಕೆ-ಯಿಂದಾಗಿ
ತಿಳಿ-ವ-ಳಿ-ಕೆಯೂ
ತಿಳಿ-ವಿ-ನಿಂದ
ತಿಳಿವು
ತಿಳಿವೂ
ತಿಳಿಸ
ತಿಳಿಸದೆ
ತಿಳಿ-ಸ-ಬಲ್ಲ
ತಿಳಿ-ಸ-ಬಲ್ಲಿ-ರಾ-ಎಂದು
ತಿಳಿ-ಸ-ಬ-ಹು-ದೆಂದು
ತಿಳಿ-ಸ-ಬೇ-ಕಿತ್ತು
ತಿಳಿ-ಸ-ಬೇ-ಕಿಲ್ಲ
ತಿಳಿ-ಸ-ಬೇಕು
ತಿಳಿ-ಸ-ಲಾ-ಗಿದೆ
ತಿಳಿ-ಸ-ಲಾ-ರನೇ
ತಿಳಿ-ಸ-ಲಿಲ್ಲ
ತಿಳಿಸಲು
ತಿಳಿಸಲೇ
ತಿಳಿಸಿ
ತಿಳಿ-ಸಿ-ಕೊಟ್ಟು
ತಿಳಿ-ಸಿ-ಕೊಟ್ಟ
ತಿಳಿ-ಸಿ-ಕೊಟ್ಟರು
ತಿಳಿ-ಸಿ-ಕೊಟ್ಟರೂ
ತಿಳಿ-ಸಿ-ಕೊಟ್ಟರೆ
ತಿಳಿ-ಸಿ-ಕೊಟ್ಟಿದ್ದರೆ
ತಿಳಿ-ಸಿ-ಕೊಟ್ಟಿದ್ದಾರೆ
ತಿಳಿ-ಸಿ-ಕೊಟ್ಟು
ತಿಳಿ-ಸಿ-ಕೊ-ಡ-ಬೇ-ಕಲ್ಲವೇ
ತಿಳಿ-ಸಿ-ಕೊ-ಡ-ಬೇಕು
ತಿಳಿ-ಸಿ-ಕೊ-ಡಲು
ತಿಳಿ-ಸಿ-ಕೊಡಿ
ತಿಳಿ-ಸಿ-ಕೊಡು
ತಿಳಿ-ಸಿ-ಕೊ-ಡುತ್ತದೆ
ತಿಳಿ-ಸಿ-ಕೊ-ಡುತ್ತಾನೆ
ತಿಳಿ-ಸಿ-ಕೊ-ಡುತ್ತಾರೆ
ತಿಳಿ-ಸಿ-ಕೊ-ಡುವ
ತಿಳಿ-ಸಿ-ಕೊ-ಡು-ವ-ವ-ರಾರು
ತಿಳಿ-ಸಿ-ಕೊ-ಡು-ವು-ದಿ-ರಲಿ
ತಿಳಿಸಿದ
ತಿಳಿ-ಸಿ-ದಂತೆಯೇ
ತಿಳಿ-ಸಿ-ದ-ರಂತೆ
ತಿಳಿ-ಸಿ-ದರು
ತಿಳಿ-ಸಿ-ದ-ರು-ಒಮ್ಮೆ
ತಿಳಿ-ಸಿ-ದರೂ
ತಿಳಿ-ಸಿ-ದರೆ
ತಿಳಿ-ಸಿ-ದಳು
ತಿಳಿ-ಸಿ-ದಾಗ
ತಿಳಿ-ಸಿ-ದು-ದಲ್ಲದೆ
ತಿಳಿಸಿದೆ
ತಿಳಿಸಿದ್ದ
ತಿಳಿ-ಸಿದ್ದರು
ತಿಳಿ-ಸಿದ್ದಲ್ಲದೆ
ತಿಳಿ-ಸಿದ್ದಾರೆ
ತಿಳಿ-ಸಿದ್ದಿಷ್ಟು
ತಿಳಿ-ಸಿ-ರ-ಬ-ಹು-ದು-ಎನ್ನುವ
ತಿಳಿ-ಸಿ-ರ-ಲಿಲ್ಲ
ತಿಳಿ-ಸಿ-ರು-ವುದು
ತಿಳಿ-ಸಿ-ರುವೆ
ತಿಳಿ-ಸಿಲ್ಲವೇ
ತಿಳಿಸೀತು
ತಿಳಿಸೀತೇ
ತಿಳಿಸು
ತಿಳಿ-ಸು-ವುದೂ
ತಿಳಿಸುತ್ತ
ತಿಳಿ-ಸುತ್ತದೆ
ತಿಳಿ-ಸುತ್ತಾನೆ
ತಿಳಿ-ಸುತ್ತಾರೆ
ತಿಳಿ-ಸುತ್ತಿ-ದೆ-ಯಾ-ದರೂ
ತಿಳಿ-ಸುತ್ತಿದ್ದ
ತಿಳಿ-ಸುತ್ತಿದ್ದರು
ತಿಳಿ-ಸುತ್ತೇನೆ
ತಿಳಿಸುವ
ತಿಳಿ-ಸು-ವಂತೆ
ತಿಳಿ-ಸು-ವ-ವ-ರಾರು
ತಿಳಿ-ಸು-ವ-ವ-ರೆಗೆ
ತಿಳಿ-ಸು-ವು-ದಿಲ್ಲ
ತಿಳಿ-ಸು-ವುದು
ತಿಳಿ-ಸು-ವುದೂ
ತಿಳು-ವ-ಳಿ-ಕೆ-ಯಿಂದ
ತಿವಿ-ಯುತ್ತಿ-ರು-ವಂತೆ
ತೀಕ್ಷ್ಣ
ತೀಕ್ಷ್ಣತೆ
ತೀಕ್ಷ್ಣ-ಮೇ-ಧಾ-ಸಂಪನ್ನ
ತೀಡಿ
ತೀಡಿ-ದ-ವ-ರ-ವರು
ತೀತ
ತೀರ
ತೀರದ
ತೀರದಲ್ಲಿ
ತೀರ-ಬೇ-ಕು-ಇದೆ
ತೀರಾ
ತೀರಿ
ತೀರಿಕೊಂಡ
ತೀರಿ-ಕೊಂಡದ್ದು
ತೀರಿ-ಕೊಂಡ-ನಂತೆ
ತೀರಿ-ಕೊಂಡರು
ತೀರಿ-ಕೊಂಡವು
ತೀರಿ-ಕೊಂಡಿತು
ತೀರಿ-ಕೊಂಡಿದ್ದ
ತೀರಿಕೊಂಡು
ತೀರಿ-ಕೊಂಡೆ-ನೆಂದು
ತೀರಿ-ಕೊಳ್ಳು-ವು-ದಕ್ಕೆ
ತೀರಿ-ಸ-ಬ-ಹುದು
ತೀರಿಸಲು
ತೀರಿ-ಸ-ಲೇ-ಬೇ-ಕೆಂದು
ತೀರಿ-ಸಿ-ಕೊಳ್ಳಿರಿ
ತೀರಿ-ಸುತ್ತಾರೆ
ತೀರಿ-ಸು-ವುದ
ತೀರಿ-ಹೋ-ಗುವ
ತೀರುತ್ತದೆ
ತೀರುತ್ತಿದ್ದ
ತೀರುವುದು
ತೀರು-ವು-ದೆಂಬ
ತೀರ್ಥ
ತೀರ್ಥಪ್ರ-ಸಾ-ದ-ಗ-ಳನ್ನು
ತೀರ್ಥಯಾತ್ರೆ
ತೀರ್ಥ-ಯಾತ್ರೆ-ಗ-ಳನ್ನೂ
ತೀರ್ಥ-ರೂ-ಪರ
ತೀರ್ಥಸ್ಥಾ-ನ-ವಾದ
ತೀರ್ಪು-ಗಾ-ರ-ರಾಗಿ
ತೀರ್ಮಾನ
ತೀರ್ಮಾನಕ್ಕೆ
ತೀರ್ಮಾ-ನ-ಗ-ಳನ್ನು
ತೀರ್ಮಾ-ನಿ-ಸ-ಬಾ-ರದು
ತೀವ್ರ
ತೀವ್ರ-ಗ-ತಿ-ಯಿಂದ
ತೀವ್ರ-ಗೊ-ಳಿ-ಸ-ಬಾ-ರದು
ತೀವ್ರ-ತ-ರದ
ತೀವ್ರ-ತ-ರ-ವಾಗಿ
ತೀವ್ರ-ತ-ರ-ವಾದ
ತೀವ್ರತೆ
ತೀವ್ರ-ತೆ-ಯನ್ನು
ತೀವ್ರತೆಯೂ
ತೀವ್ರ-ನಿಷ್ಠೆ-ಯಿಂದ
ತೀವ್ರವಾಗಿ
ತೀವ್ರ-ವಾ-ಗಿತ್ತು
ತೀವ್ರ-ವಾ-ಗಿತ್ತೆಂದರೆ
ತೀವ್ರ-ವಾ-ಗಿದ್ದರೆ
ತೀವ್ರ-ವಾ-ಗುತ್ತದೆ
ತೀವ್ರ-ವಾ-ಗುತ್ತ-ಲಿದೆ
ತೀವ್ರವಾದ
ತೀವ್ರ-ವಾ-ದಂತೆ
ತೀವ್ರ-ವಾ-ದಾಗ
ತೀವ್ರ-ವಾ-ದಾ-ಗಲೆ
ತೀವ್ರ-ವಾ-ದಿ-ಗ-ಳಾದ
ತೀವ್ರ-ವಾ-ಯಿತು
ತೀವ್ರ-ವೇ-ಗ-ದಿಂದ
ತುಂಟ
ತುಂಟ-ಕು-ದು-ರೆ-ಯಾ-ಗ-ಬಲ್ಲದು
ತುಂಟತನ
ತುಂಟ-ತ-ನ-ಗ-ಳನ್ನು
ತುಂಟ-ತ-ನ-ದಿಂದ
ತುಂಟರ
ತುಂಟರಲ್ಲಿ
ತುಂಟರು
ತುಂಟಾಟ
ತುಂಡ-ರಿ-ಸದ
ತುಂಡ-ರಿ-ಸಿದ್ದಿ-ದೆಯೇ
ತುಂಡಿ
ತುಂಡಿನಿಂದ
ತುಂಬ
ತುಂಬ-ತೊ-ಡ-ಗಿ-ದವು
ತುಂಬಾ
ತುಂಬಿ
ತುಂಬಿ-ಕೊಂಡಿದ್ದರು
ತುಂಬಿಕೊಂಡ
ತುಂಬಿ-ಕೊಂಡಿವೆ
ತುಂಬಿಕೊಂಡು
ತುಂಬಿ-ಕೊಳ್ಳ-ಲಾ-ಗು-ವುದೇ
ತುಂಬಿ-ಕೊಳ್ಳುತ್ತಿದ್ದೇ-ನಲ್ಲ
ತುಂಬಿತು
ತುಂಬಿ-ತು-ಳು-ಕುತ್ತಿದ್ದವು
ತುಂಬಿತ್ತು
ತುಂಬಿದ
ತುಂಬಿದಂತೆ
ತುಂಬಿ-ದ-ವರು
ತುಂಬಿದಿರಿ
ತುಂಬಿದೆ
ತುಂಬಿದ್ದರೂ
ತುಂಬಿ-ಬ-ರುತ್ತ-ಲಿ-ದೆ-ಅಯ್ಯೋ
ತುಂಬಿ-ರ-ಬೇಕು
ತುಂಬಿ-ರುತ್ತದೆ
ತುಂಬಿ-ರುತ್ತವೆ
ತುಂಬಿರುವ
ತುಂಬಿಸಲು
ತುಂಬಿ-ಸಿ-ಕೊಂಡು
ತುಂಬಿಸಿದ
ತುಂಬು
ತುಂಬುತ್ತಿತ್ತು
ತುಂಬು-ವು-ದಕ್ಕೆ
ತುಂಬು-ವು-ದಲ್ಲವೇ
ತುಚ್ಛವಾಗಿ
ತುಚ್ಛವೂ
ತುಚ್ಛ-ವೆ-ನಿ-ಸೀತು
ತುಟಿ
ತುಟಿಗಳ
ತುಟಿ-ಗ-ಳನ್ನು
ತುಟಿ-ಗ-ಳಲ್ಲಿ
ತುಡಿಯಿತು
ತುಣಕನ್ನೊ
ತುಣುಕನ್ನು
ತುಣು-ಕಿ-ನಲ್ಲಿ
ತುಣುಕು
ತುಣು-ಕು-ಗ-ಳನ್ನು
ತುಣು-ಕು-ಗಳು
ತುಣುಕೂ
ತುಣು-ಕೊಂದಕ್ಕಾಗಿ
ತುತ್ತ-ತು-ದಿ-ಯನ್ನು
ತುತ್ತಾಗ
ತುತ್ತಾ-ಗ-ಬ-ಹುದು
ತುತ್ತಾ-ಗ-ಬ-ಹು-ದೆಂದು
ತುತ್ತಾ-ಗ-ಬೇ-ಕೆಂಬ
ತುತ್ತಾಗಿ
ತುತ್ತಾ-ಗಿದ್ದು-ದ-ರಿಂದ
ತುತ್ತಾ-ಗು-ವುದು
ತುತ್ತಾದ
ತುತ್ತಿಗೂ
ತುದಿ
ತುದಿಗೆ
ತುದಿಯನ್ನು
ತುಪ್ಪ
ತುಮುಲ
ತುಮು-ಲ-ವನ್ನು
ತುಮ್ಹೇಂ
ತುರಿ-ಕೆ-ಯನ್ನುಂಟು-ಮಾ-ಡುವ
ತುರಿ-ಸಿ-ಕೊಂಡಂತೆ
ತುರುಕಲು
ತುರ್ತು
ತುಲ-ನಾತ್ಮಕ
ತುಲ-ಸೀ-ಕಟ್ಟೆಯ
ತುಲ-ಸೀ-ದಾ-ಸರ
ತುಳಿತ
ತುಳಿ-ತಕ್ಕೊ-ಳ-ಗಾದ
ತುಳಿ-ತಕ್ಕೊ-ಳ-ಗಾ-ದ-ವನು
ತುಳಿ-ತಕ್ಕೊ-ಳ-ಗಾ-ದ-ವರ
ತುಳಿ-ತಕ್ಕೊ-ಳ-ಗಾ-ದ-ವ-ರಿಗೆ
ತುಳಿ-ತಕ್ಕೊ-ಳ-ಗಾ-ದ-ವರು
ತುಳಿದು
ತುಳಿ-ದು-ಬಿಟ್ಟೆ
ತುಳಿ-ಯ-ಬೇ-ಕೆನ್ನುವ
ತುಳಿಯಲು
ತುಳಿ-ಯು-ವುದು
ತುಳುಕಿ
ತುಳು-ಕುತ್ತಿದೆ
ತುಷ್ಟಿ
ತುಷ್ಟಿ-ಗೊ-ಳಿ-ಸ-ಬೇ-ಕೆನ್ನುವ
ತುಸು
ತೂಕ
ತೂಕದ
ತೂಕ-ವನ್ನಷ್ಟೇ
ತೂಗ-ಹಾ-ಕಿದ್ದರು
ತೂಗು
ತೂತುಗಳು
ತೂರಿ
ತೂರಿಕೊಂಡು
ತೂರುತ್ತಾರೆ
ತೃಣದ
ತೃಣಮಪಿ
ತೃಪ್ತನಾಗಿ
ತೃಪ್ತ-ರಾ-ಗದೆ
ತೃಪ್ತಿ
ತೃಪ್ತಿ-ಗ-ಳನ್ನೂ
ತೃಪ್ತಿಗಳು
ತೃಪ್ತಿಗಾಗಿ
ತೃಪ್ತಿ-ಗಾ-ಗಿಯೇ
ತೃಪ್ತಿ-ಪ-ಡಿಸಿ
ತೃಪ್ತಿ-ಪ-ಡಿ-ಸು-ವು-ದೆಂದೇ
ತೃಪ್ತಿಯ
ತೃಪ್ತಿಯನ್ನು
ತೃಪ್ತಿಯನ್ನೂ
ತೃಪ್ತಿಯಿಲ್ಲ
ತೃಷ್ಣೆಗಳ
ತೆಗ-ಳಿ-ದರೆ
ತೆಗಳುವ
ತೆಗೆದ
ತೆಗೆದಂತೆ
ತೆಗೆದರೆ
ತೆಗೆದಾಗ
ತೆಗೆ-ದಿದ್ದರು
ತೆಗೆ-ದಿ-ರಿಸ
ತೆಗೆ-ದಿ-ರಿ-ಸಿ-ಕೊಂಡು
ತೆಗೆ-ದಿ-ರಿ-ಸಿ-ದರೆ
ತೆಗೆದು
ತೆಗೆ-ದು-ಕೊಂಡದ್ದಕ್ಕೆ
ತೆಗೆ-ದು-ಕೊಂಡರು
ತೆಗೆ-ದು-ಕೊಂಡರೆ
ತೆಗೆ-ದು-ಕೊಂಡ-ವರು
ತೆಗೆ-ದು-ಕೊಂಡು
ತೆಗೆ-ದು-ಕೊಂಡೆ
ತೆಗೆ-ದು-ಕೊಳ್ಳದೆ
ತೆಗೆ-ದು-ಕೊಳ್ಳದೇ
ತೆಗೆ-ದು-ಕೊಳ್ಳ-ಬೇ-ಕಾದ
ತೆಗೆ-ದು-ಕೊಳ್ಳಲು
ತೆಗೆ-ದು-ಕೊಳ್ಳುತ್ತಿದ್ದ
ತೆಗೆ-ದು-ಕೊಳ್ಳು-ವುದು
ತೆಗೆ-ದು-ಹಾ-ಕಲು
ತೆಗೆ-ದು-ಹಾಕಿ
ತೆಗೆ-ದು-ಹಾ-ಕಿ-ಬಿಡಿ
ತೆಗೆ-ದೆ-ಸೆ-ಯಲು
ತೆಗೆಯದೆ
ತೆಗೆಯಲು
ತೆಗೆ-ಯುತ್ತಾರೆ
ತೆಗೆಸುತ್ತಾ
ತೆತ್ತಾಗ
ತೆಪ್ಪಗಾದ
ತೆರದ
ತೆರನ
ತೆರನದು
ತೆರನಾಗಿ
ತೆರ-ನಾ-ಗಿ-ರು-ವನು
ತೆರನಾದ
ತೆರ-ಬೇ-ಕಾದ
ತೆರಳಿ
ತೆರ-ಳಿ-ದಾಗ
ತೆರ-ಳಿ-ಯಾ-ದರೂ
ತೆರಳುತ್ತಾ
ತೆರಳುವ
ತೆರಿ-ಗೆ-ಯಿಂದ
ತೆರುವ
ತೆರೆ
ತೆರೆಗಳು
ತೆರೆದ
ತೆರೆ-ದಂತಾ-ಗುತ್ತದೆ
ತೆರೆದರೆ
ತೆರೆ-ದಿದ್ದರೂ
ತೆರೆದು
ತೆರೆಯ
ತೆರೆ-ಯ-ಬಲ್ಲದು
ತೆರೆ-ಯಲ್ಪಟ್ಟಾಗ
ತೆರೆ-ಯಲ್ಪ-ಡುತ್ತದೆ
ತೆರೆಯುವ
ತೆರೆ-ಯು-ವ-ನೆಂದು
ತೆರೆ-ಯು-ವುದು
ತೆಳ್ಳ-ಗಿದ್ದರೆ
ತೇಗಿನಲ್ಲಿ
ತೇಜಸ್ಸಿದೆ
ತೇಜೋಬಿಂದು
ತೇಜೋಮಯ
ತೇದ
ತೇನ
ತೇಪೆ
ತೇಯುವ
ತೇಯ್ದು
ತೇರೇ
ತೇರ್ಗ-ಡೆ-ಯಾ-ಗು-ವುದು
ತೇಲ
ತೇಲಾ-ಡು-ವ-ವರು
ತೇಲಿ
ತೇಲಿಕೊಂಡು
ತೇಲಿ-ಸಿ-ಬಿಟ್ಟರು
ತೇಲುತ್ತ
ತೇಲುತ್ತಿದ್ದ
ತೇಲುತ್ತಿದ್ದರು
ತೇಲುತ್ತಿದ್ದರೋ
ತೇಲುತ್ತಿದ್ದೇವೆ
ತೇಲುತ್ತಿ-ರುತ್ತಾನೆ
ತೇಲುತ್ತಿ-ರುವ
ತೇಲುತ್ತಿ-ರು-ವ-ವ-ರಿಗೆ
ತೇಲುವಂತೆ
ತೇಲು-ವಿ-ಕೆಯ
ತೈಲ-ಧಾ-ರೆ-ಯಂತೆ
ತೊಂದರೆ
ತೊಂದ-ರೆ-ಯಾ-ಗದೇ
ತೊಂದ-ರೆ-ಗ-ಳನ್ನು
ತೊಂದ-ರೆ-ಗ-ಳನ್ನೂ
ತೊಂದ-ರೆ-ಗ-ಳಾ-ವುವೂ
ತೊಂದ-ರೆ-ಗ-ಳಿಂದ
ತೊಂದ-ರೆ-ಗ-ಳಿಂದಲೂ
ತೊಂದ-ರೆ-ಗ-ಳಿ-ರ-ದಿದ್ದು-ದ-ರಿಂದ
ತೊಂದ-ರೆ-ಗ-ಳಿಲ್ಲದೆ
ತೊಂದ-ರೆ-ಗಳು
ತೊಂದ-ರೆ-ಗಳೇ
ತೊಂದ-ರೆ-ಗ-ಳೇ-ನಾ-ದರೂ
ತೊಂದರೆಗೆ
ತೊಂದ-ರೆ-ಗೊ-ಳ-ಗಾ-ಗ-ಬೇ-ಕಾ-ಗು-ವು-ದಷ್ಟೆ
ತೊಂದ-ರೆ-ಗೊ-ಳ-ಗಾ-ಗುತ್ತೇವೆ
ತೊಂದ-ರೆ-ಗೊ-ಳ-ಗಾದ
ತೊಂದ-ರೆ-ಗೊ-ಳ-ಗಾ-ದರೂ
ತೊಂದರೆಯ
ತೊಂದ-ರೆ-ಯನ್ನು
ತೊಂದ-ರೆ-ಯನ್ನುಂಟು
ತೊಂದ-ರೆ-ಯನ್ನೂ
ತೊಂದ-ರೆ-ಯಾ-ಗ-ದಂತೆ
ತೊಂದ-ರೆ-ಯಾ-ಗು-ವುದು
ತೊಂದರೆಯೆ
ತೊಂಬತ್ತು
ತೊಂಬತ್ತು-ಸಾ-ವಿರ
ತೊಂಬತ್ತೊಂಬತ್ತು
ತೊಂಬತ್ನಾಲ್ಕು
ತೊಟ್ಟಿ-ಯಿಂದಲೇ
ತೊಟ್ಟಿ-ಯೊ-ಳಕ್ಕೆ
ತೊಡ-ಕಾ-ಗದೆ
ತೊಡಕು
ತೊಡ-ಕು-ಗ-ಳನ್ನು
ತೊಡ-ಕು-ಗ-ಳಾ-ಗಿ-ರ-ಬ-ಹುದು
ತೊಡ-ಕು-ಗ-ಳಿಂದಲೇ
ತೊಡ-ಗ-ಬೇಕು
ತೊಡಗಿ
ತೊಡಗಿತು
ತೊಡಗಿತ್ತು
ತೊಡಗಿದ
ತೊಡ-ಗಿ-ದ-ನಾತ
ತೊಡ-ಗಿ-ದಂದಿ-ನಿಂದ
ತೊಡ-ಗಿ-ದ-ನೆನ್ನಿ
ತೊಡ-ಗಿ-ದರು
ತೊಡ-ಗಿ-ದವು
ತೊಡ-ಗಿ-ದಾಗ
ತೊಡಗಿದೆ
ತೊಡಗಿದ್ದ
ತೊಡ-ಗಿದ್ದರು
ತೊಡ-ಗಿ-ಸ-ಲಾರ
ತೊಡಗು
ತೊಡ-ಗುತ್ತದೆ
ತೊಡ-ಗುತ್ತವೆ
ತೊಡ-ಗುತ್ತಾ-ನಷ್ಟೆ
ತೊಡ-ಗುತ್ತಾನೆ
ತೊಡ-ಗುತ್ತಾರೆ
ತೊಡ-ಗುತ್ತಿದ್ದ
ತೊಡಗುವು
ತೊಡ-ಗು-ವುದು
ತೊಡ-ರು-ಗ-ಳನ್ನು
ತೊಡ-ರು-ಗಳು
ತೊಡಿಸಿ
ತೊಡೆ-ದು-ಹಾ-ಕುವ
ತೊದಲು
ತೊರೆದರೆ
ತೊರೆ-ದ-ವರು
ತೊರೆದಷ್ಟೂ
ತೊರೆದಿಲ್ಲ
ತೊರೆದು
ತೊರೆಯ
ತೊರೆಯನ್ನು
ತೊರೆ-ಯ-ಬಾ-ರದು
ತೊರೆ-ಯ-ಬೇಡಿ
ತೊರೆ-ಯುತ್ತಾರೆ
ತೊರೆ-ಯು-ವಂತೆ
ತೊರೆ-ಯು-ವುದೇ
ತೊಲ-ಗ-ಬೇ-ಕಾ-ಯಿತು
ತೊಲ-ಗ-ಲಾ-ರ-ವೆಂಬುದು
ತೊಲ-ಗ-ಲಿಲ್ಲ
ತೊಲ-ಗಿ-ದರೂ
ತೊಲ-ಗಿ-ಸಲು
ತೊಲ-ಗುತ್ತದೆ
ತೊಳ-ಲಾ-ಡುವ
ತೊಳ-ಲಾ-ಡುತ್ತಿದ್ದಳು
ತೊಳ-ಲು-ವ-ವರು
ತೊಳೆದು
ತೊಳೆ-ದು-ಕೊಳ್ಳೋಣ
ತೊಳೆ-ಯ-ಬೇಕು
ತೊಳೆ-ಯುತ್ತಿ-ರುವ
ತೋಚದೆ
ತೋಚಿದಂತೆ
ತೋಟದ
ತೋಟದಲ್ಲಿ
ತೋಡ
ತೋಡಿ
ತೋಡಿಕೊಂಡ
ತೋಡಿ-ಕೊಂಡರು
ತೋಡಿ-ಕೊಂಡಳು
ತೋಡಿ-ಕೊಂಡಿದ್ದ
ತೋಡಿಕೊಂಡು
ತೋಡಿ-ಕೊಳ್ಳಲು
ತೋಡಿ-ಕೊಳ್ಳು-ವುದು
ತೋಡಿದ
ತೋಡಿ-ಸಿದ್ದೇನೆ
ತೋತಾಪುರಿ
ತೋತಾ-ಪು-ರಿ-ಯನ್ನೇ
ತೋತಾ-ಪು-ರಿ-ಯ-ವರು
ತೋಯದೆ
ತೋಯಿಸದ
ತೋಯಿಸಿದ
ತೋರ
ತೋರ-ದಿದ್ದಾಗ
ತೋರ-ದಿ-ರದು
ತೋರದೆ
ತೋರಬಲ್ಲ
ತೋರ-ಬಲ್ಲರೇ
ತೋರ-ಬ-ಹುದು
ತೋರಮ್ಮಾ
ತೋರಲಿಲ್ಲ
ತೋರಲು
ತೋರವು
ತೋರಿ
ತೋರಿಸುವ
ತೋರಿ-ಕೆ-ಗ-ಳಾ-ಗಿ-ರು-ವು-ದ-ರಲ್ಲಿ
ತೋರಿಕೆಗೆ
ತೋರಿಕೆಯ
ತೋರಿ-ಕೆ-ಯನ್ನು
ತೋರಿ-ಕೆ-ಯನ್ನೇ
ತೋರಿಕೆಯೇ
ತೋರಿಕೊಂಡು
ತೋರಿತು
ತೋರಿದ
ತೋರಿದರು
ತೋರಿದರೂ
ತೋರಿದಿರಿ
ತೋರಿದ್ದ-ರಿಂದ
ತೋರಿದ್ದರೆ
ತೋರಿದ್ದಾನೆ
ತೋರಿ-ಬ-ರ-ಲಿಲ್ಲ
ತೋರಿ-ಬ-ರು-ವಲ್ಲೂ
ತೋರಿ-ಸ-ದಿದ್ದರೆ
ತೋರಿ-ಸ-ಬೇ-ಕಾ-ದಲ್ಲಿ
ತೋರಿ-ಸ-ಬೇಕು
ತೋರಿ-ಸ-ಬೇಡಿ
ತೋರಿಸಲು
ತೋರಿಸಿ
ತೋರಿ-ಸಿ-ದುದು
ತೋರಿ-ಸಿ-ಕೊಂಡು
ತೋರಿ-ಸಿ-ಕೊಟ್ಟ
ತೋರಿ-ಸಿ-ಕೊಟ್ಟರೆ
ತೋರಿ-ಸಿ-ಕೊಟ್ಟಾಗ
ತೋರಿ-ಸಿ-ಕೊಟ್ಟಿ-ರುವ
ತೋರಿ-ಸಿ-ಕೊ-ಡುತ್ತದೆ
ತೋರಿ-ಸಿ-ಕೊ-ಡುವ
ತೋರಿ-ಸಿ-ಕೊ-ಡು-ವು-ದಕ್ಕಿಂತ
ತೋರಿ-ಸಿ-ಕೊ-ಡು-ವುದು
ತೋರಿ-ಸಿ-ಕೊಳ್ಳದೆ
ತೋರಿ-ಸಿ-ಕೊಳ್ಳದೇ
ತೋರಿ-ಸಿ-ಕೊಳ್ಳುತ್ತಾನೆ
ತೋರಿಸಿತು
ತೋರಿಸಿದ
ತೋರಿ-ಸಿ-ದಂತೆ
ತೋರಿ-ಸಿ-ದರು
ತೋರಿ-ಸಿ-ದರೂ
ತೋರಿ-ಸಿ-ದರೆ
ತೋರಿ-ಸಿ-ದಳು
ತೋರಿ-ಸಿ-ದ-ವನು
ತೋರಿ-ಸಿ-ದಾಗ
ತೋರಿ-ಸಿ-ದು-ದಕ್ಕೆ
ತೋರಿಸಿದೆ
ತೋರಿ-ಸಿದ್ದ-ಳೆಂದು
ತೋರಿ-ಸಿದ್ದಾರೆ
ತೋರಿಸುತ್ತ
ತೋರಿ-ಸುತ್ತ-ದಷ್ಟೆ
ತೋರಿ-ಸುತ್ತದೆ
ತೋರಿ-ಸುತ್ತಾರೆ
ತೋರಿ-ಸುತ್ತಿತ್ತು
ತೋರಿ-ಸುತ್ತಿದ್ದಾರೆ
ತೋರಿ-ಸುತ್ತೇನೆ
ತೋರಿಸುವ
ತೋರಿ-ಸು-ವಂತೆ
ತೋರಿ-ಸು-ವಂತೆಯೂ
ತೋರಿ-ಸು-ವಂತೆಯೇ
ತೋರಿ-ಸು-ವು-ದಿ-ರಲಿ
ತೋರಿ-ಸು-ವುದು
ತೋರಿಸೆಂದು
ತೋರು
ತೋರುತಿರೆ
ತೋರುತ್ತದೆ
ತೋರುತ್ತಾನೆ
ತೋರುತ್ತಿದೆ
ತೋರುತ್ತಿದ್ದ
ತೋರುತ್ತಿದ್ದವು
ತೋರುತ್ತಿ-ರ-ಬ-ಹುದು
ತೋರುತ್ತಿ-ರುವ
ತೋರುವ
ತೋರುವಂತೆ
ತೋರು-ವ-ವ-ರಿಗೆ
ತೋರು-ವು-ದನ್ನು
ತೋರುವುದು
ತೋರು-ವು-ದೆಂದರೆ
ತೋರ್ಪ-ಡಿ-ಸುತ್ತಾ
ತೋರ್ಪ-ಡಿ-ಸುವ
ತೋರ್ಪ-ಡಿ-ಸು-ವ-ವರೂ
ತೋಳನ್ನು
ತೋಳಿಗೆ
ತೌರಿಗೆ
ತ್ಕಾರದ
ತ್ತದೆಯೇ
ತ್ತರು
ತ್ತವೆ
ತ್ತವೆಯೇ
ತ್ತಾನೆ
ತ್ತಿತ್ತು
ತ್ತಿದೆ
ತ್ತಿದ್ದ
ತ್ತಿದ್ದರು
ತ್ತಿದ್ದರೂ
ತ್ತಿದ್ದರೆ
ತ್ತಿದ್ದಾನೆ
ತ್ತಿದ್ದುದನ್ನು
ತ್ತಿದ್ದೆ
ತ್ತಿದ್ದೇನೆ
ತ್ತಿದ್ದೇವೆ
ತ್ತಿರು
ತ್ತಿರುವಂತೆ
ತ್ತಿರುವಾಗ
ತ್ತಿರು-ವಾ-ಗಲೇ
ತ್ತಿವೆ
ತ್ತೀರಿ
ತ್ತೆಂಟಕ್ಕೆ
ತ್ತೇವೆ
ತ್ತೇವೋ
ತ್ತೈದು
ತ್ತೊಂದನೇ
ತ್ಮಕ
ತ್ಯಜಿ-ಸ-ಬೇಕು
ತ್ಯಜಿಸಲು
ತ್ಯಜಿಸಿ
ತ್ಯಜಿಸಿದ
ತ್ಯಜಿಸುವ
ತ್ಯಜಿ-ಸು-ವಂತೆ
ತ್ಯಜಿ-ಸು-ವುದು
ತ್ಯಾಗ
ತ್ಯಾಗಕ್ಕೂ
ತ್ಯಾಗ-ಗ-ಳಿಂದ
ತ್ಯಾಗದ
ತ್ಯಾಗದಿಂದ
ತ್ಯಾಗ-ಪ-ಥ-ವನ್ನು
ತ್ಯಾಗಮಯ
ತ್ಯಾಗವನ್ನು
ತ್ಯಾಗವು
ತ್ಯಾಗಿಗಳ
ತ್ಯಾಗಿಯ
ತ್ಯಾಗಿ-ಯ-ವರ
ತ್ಯಾಗಿಯು
ತ್ಯಾಗೀ
ತ್ರಾಣ-ವಿ-ರದ
ತ್ರಿಕ-ರ-ಣ-ಪೂರ್ವ-ಕ-ವಾದ
ತ್ರಿಕ-ರ-ಣ-ಸಾಂಗತ್ಯ
ತ್ರಿಕಾ-ಲ-ದಲ್ಲೂ
ತ್ರಿಕಾ-ಲಾ-ಬಾ-ಧಿ-ತ-ವಾದ
ತ್ರಿಕೋನ
ತ್ರಿವಿಧ
ತ್ರಿವೇಂಡ್ರಂನ
ತ್ರೈಮಾಸಿಕ
ತ್ವರಿ-ತ-ವಾಗಿ
ಥಟ್ಟನೆ
ಥಟ್ಟನೇ
ಥಳಕು
ಥಳ-ಥ-ಳನೆ
ಥಾಮಸ್
ಥೂ
ಥೆರೇಸಾ
ದ
ದಂಗಾ-ಗಿದ್ದಾರೆ
ದಂಗು-ಬ-ಡಿ-ಸಿತ್ತು
ದಂಗು-ಬ-ಡಿ-ಸುತ್ತವೆ
ದಂಡ-ನೆ-ಯನ್ನು
ದಂಡ-ಭ-ಯ-ದಿಂದ
ದಂಡವನ್ನು
ದಂಡಿ-ಸಲ್ಪಟ್ಟ-ವನೂ
ದಂಡಿ-ಸುತ್ತಾನೆ
ದಂಡೆತ್ತಿ
ದಂತ-ಕ-ಥೆ-ಗ-ಳಲ್ಲ
ದಂತ-ವೈದ್ಯರು
ದಂತಾಯಿತೆ
ದಂತೂ
ದಂತೆ
ದಂಪ-ತಿ-ಗಳ
ದಂಪ-ತಿ-ಗ-ಳಲ್ಲಿ
ದಂಪ-ತಿ-ಗ-ಳಾ-ಗಲಿ
ದಂಪ-ತಿ-ಗ-ಳಿಗೆ
ದಂಪ-ತಿ-ಗ-ಳಿದ್ದರು
ದಂಪ-ತಿ-ಗಳು
ದಂಷ್ಟ್ರಾಶ್ರಿತ
ದಕ್ಕಾಗಿ
ದಕ್ಕೀತೇ
ದಕ್ಕೆ
ದಕ್ಷ
ದಕ್ಷತೆ
ದಕ್ಷ-ತೆ-ಇವು
ದಕ್ಷ-ತೆ-ಗ-ಳನ್ನು
ದಕ್ಷ-ತೆ-ಗ-ಳಿಗೆ
ದಕ್ಷತೆಯ
ದಕ್ಷ-ತೆ-ಯನ್ನು
ದಕ್ಷ-ತೆ-ಯಿಂದ
ದಕ್ಷ-ನಲ್ಲದ
ದಕ್ಷ-ನಾ-ಗಿದ್ದ
ದಕ್ಷ-ನಾ-ಗಿ-ರ-ಬೇ-ಕೆಂಬುದು
ದಕ್ಷ-ನಾ-ಗುತ್ತದೆ
ದಕ್ಷನೇನೂ
ದಕ್ಷ-ರಾ-ದ-ವರೇ
ದಕ್ಷರೂ
ದಕ್ಷಳಾದ
ದಕ್ಷಳೂ
ದಕ್ಷಿಣ
ದಕ್ಷಿ-ಣ-ಕನ್ನಡ
ದಕ್ಷಿ-ಣೇಶ್ವ-ರಕ್ಕೆ
ದಕ್ಷಿ-ಣೇಶ್ವ-ರದ
ದಕ್ಷಿ-ಣೇಶ್ವ-ರ-ದಲ್ಲಿ
ದಟ್ಟಡವಿ
ದಟ್ಟವಾಗಿ
ದಟ್ಟವಾದ
ದಟ್ಟೈ-ಸಿ-ದಾಗ
ದಡಕ್ಕೆ
ದಡ್ಡ
ದಡ್ಡ-ನಾ-ಗಿ-ರ-ಲಿಲ್ಲ
ದಡ್ಡನು
ದಡ್ಡನೂ
ದಡ್ಡರೂ
ದಡ್ಡರೆಂದು
ದಣಿದು
ದಣಿ-ಯು-ವಂತೆ
ದಣಿ-ವಿಲ್ಲದ
ದತ್ತ
ದತ್ತಾತ್ರೇಯ
ದತ್ತು
ದತ್ತುಸ್ವೀ-ಕಾರ
ದತ್ತುಸ್ವೀ-ಕಾ-ರಕ್ಕೆ
ದನ
ದನ-ಕ-ರು-ಗಳ
ದನ-ಕ-ರು-ಗ-ಳು-ಮುಂತಾಗಿ
ದನ-ಗ-ಳೇ-ಮುಂತಾದ
ದನಿ
ದನಿಯನ್ನು
ದನಿಯಿಂದ
ದನಿಯು
ದನೆಯೂ
ದನ್ನು
ದನ್ನೂ
ದಪ್ಪದ
ದಪ್ಪವಾಗಿ
ದಬ್ಬಾ
ದಬ್ಬಾಳಿಕೆ
ದಬ್ಬಾ-ಳಿ-ಕೆಗೆ
ದಬ್ಬಾ-ಳಿ-ಕೆ-ಗೊ-ಳ-ಗಾ-ದ-ವರ
ದಬ್ಬಾ-ಳಿ-ಕೆ-ಯನ್ನು
ದಬ್ಬಾ-ಳಿ-ಕೆ-ಯಲ್ಲಿಟ್ಟು
ದಬ್ಬಾ-ಳಿ-ಕೆ-ಯಿಂದ
ದಮ-ನ-ಕಾ-ರ-ರಿಗೂ
ದಮ-ನಕ್ಕೊ-ಳ-ಗಾ-ಗಿ-ರುವ
ದಮ-ನಕ್ಕೊ-ಳ-ಗಾ-ದ-ವ-ರಿಗೂ
ದಯನೀಯ
ದಯ-ಪಾ-ಲಿಸು
ದಯಮಾಡಿ
ದಯವಿಟ್ಟು
ದಯವಿಲ್ಲ
ದಯಾ-ಮ-ಯನೂ
ದಯಾ-ದಾಕ್ಷಿಣ್ಯ-ವಿಲ್ಲ
ದಯಾನಿಧಿ
ದಯಾ-ಪ-ರ-ರಾ-ಗಲು
ದಯಾ-ಭಾ-ವ-ವನ್ನು
ದಯಾಮಯ
ದಯಾ-ಮ-ಯ-ನಾದ
ದಯಾ-ಳು-ಗ-ಳಾದ
ದಯಾ-ಶೀ-ಲ-ರಾ-ಗುತ್ತಾರೆ
ದಯೆ
ದಯೆಯ
ದಯೆಯನ್ನು
ದಯೆಯಿಂದ
ದರಿದ್ರ
ದರಿದ್ರ-ನಾಗಿ
ದರಿದ್ರ-ರಾದ
ದರು
ದರೂ
ದರೆ
ದರೋಡೆ
ದರೋ-ಡೆ-ಕಾ-ರನು
ದರೋ-ಡೆ-ಕೋ-ರರ
ದರೋ-ಡೆ-ಗಳು
ದರೋ-ಡೆ-ಯಾ-ಗದೇ
ದರ್ಜೆ
ದರ್ಜೆಯ
ದರ್ಜೆಯಲ್ಲಿ
ದರ್ಪ
ದರ್ಪ-ಮ-ತಾಂಧ-ತೆಯ
ದರ್ಪವನ್ನು
ದರ್ಶ-ಕ-ವನ್ನು
ದರ್ಶನ
ದರ್ಶ-ನ-ಗ-ಳನ್ನು
ದರ್ಶ-ನ-ದಿಂದ
ದರ್ಶನವ
ದರ್ಶ-ನ-ವನ್ನು
ದರ್ಶ-ನ-ವಾ-ಗದೆ
ದರ್ಶ-ನ-ವಾ-ಗ-ಬೇ-ಕಾ-ಗಿ-ರುವ
ದರ್ಶ-ನ-ವಾ-ಗ-ಲಿಲ್ಲ-ವೆಂದು
ದರ್ಶ-ನ-ವಾ-ಗುತ್ತಲೇ
ದರ್ಶ-ನ-ವಾ-ಗು-ವಂಥ
ದರ್ಶ-ನ-ಶಾಸ್ತ್ರ-ವನ್ನು
ದರ್ಶಿ-ಸ-ಬ-ಹುದು
ದಲಿತ
ದಲಿತರ
ದಲಿ-ತ-ರಿಗೆ
ದಲಿತರೂ
ದಲೇ
ದಲ್ಲಂತೂ
ದಲ್ಲವೇ
ದಲ್ಲಾಗಲೀ
ದಲ್ಲಿ
ದಲ್ಲೂ
ದಲ್ಲೇ
ದಳ-ಗ-ಳಿಂದಾ-ಗಲೀ
ದಳ್ಳುರಿ
ದಳ್ಳು-ರಿ-ಯಿಂದ
ದವಡೆಗೆ
ದವ-ಡೆ-ಯನ್ನು
ದವ-ಡೆ-ಯಲ್ಲಿ
ದಶ-ಕ-ಗ-ಳಲ್ಲಿ
ದಶ-ಕ-ಗ-ಳಲ್ಲೇ
ದಶ-ಕ-ಗ-ಳಿಂದ
ದಶ-ಕ-ಗ-ಳಿಂದಲೂ
ದಶ-ಕ-ದಲ್ಲಿ
ದಶ-ಕ-ದ-ವ-ರೆಗೂ
ದಶ-ದಿಕ್ಕು-ಗಳೂ
ದಶ-ದಿ-ಶೆಗೂ
ದಶ-ಮು-ಖ-ನಾ-ಗುತ್ತಾನೆ
ದಶೆ
ದಹಿ-ಸುತ್ತಿದೆ
ದಾಂಡಿಗ
ದಾಂತೆ
ದಾಖ-ಲಾ-ಗಿ-ರುತ್ತದೆ
ದಾಖ-ಲಾ-ಗಿ-ರು-ವು-ದ-ರಿಂದಲೇ
ದಾಖ-ಲಾ-ಗಿವೆ
ದಾಖ-ಲಾ-ಗುತ್ತ-ಲಿ-ರುತ್ತವೆ
ದಾಖ-ಲಿ-ಸಿದೆ
ದಾಖ-ಲಿ-ಸುತ್ತ-ಲಿದೆ
ದಾಖಲೆ
ದಾಖ-ಲೆ-ಗಳ
ದಾಖ-ಲೆ-ಗ-ಳನ್ನು
ದಾಖ-ಲೆ-ಗ-ಳಲ್ಲಿ
ದಾಖ-ಲೆ-ಗ-ಳಲ್ಲಿದೆ
ದಾಖ-ಲೆ-ಗಳು
ದಾಖ-ಲೆ-ಗಳೂ
ದಾಖ-ಲೆ-ಗ-ಳೊಂದಿಗೆ
ದಾಗ
ದಾಟ-ಬ-ಹು-ದೆಂದು
ದಾಟ-ಲಾ-ರನು
ದಾಟ-ಲಾ-ರರು
ದಾಟಲು
ದಾಟಿ
ದಾಟಿದ
ದಾಟಿಯೇ
ದಾಟಿ-ರ-ದಿದ್ದರೂ
ದಾಟಿ-ರ-ಬೇಕು
ದಾಟುತ್ತಿ-ರುವ
ದಾಟುವ
ದಾಟುವುದು
ದಾದ
ದಾದರೂ
ದಾದರೆ
ದಾದಿ
ದಾದಿಯಂತೇ
ದಾದಿಯರ
ದಾದಿಯು
ದಾದಿ-ಯೊಬ್ಬಳು
ದಾನ
ದಾನಕ್ಕೆ
ದಾನ-ಗ-ಳಿಂದ
ದಾನ-ದಿಂದಲೂ
ದಾನಪತ್ರ
ದಾನ-ಬುದ್ಧಿ-ಮುಂತಾ-ದ-ವು-ಗಳು
ದಾನ-ಮಾ-ಡಲು
ದಾನವನ್ನು
ದಾನವನ್ನೂ
ದಾನವರು
ದಾನವಾಗಿ
ದಾನವೆಂದು
ದಾನ-ಶೀ-ಲ-ರಾ-ಗುತ್ತಾರೆ
ದಾನಿ-ತ-ನ-ದಿಂದ
ದಾರಿ
ದಾರಿ-ಕಾ-ಣದೆ
ದಾರಿ-ಗ-ನೊಬ್ಬ
ದಾರಿ-ಗ-ರನ್ನು
ದಾರಿಗೆ
ದಾರಿ-ತಪ್ಪಿಸಿ
ದಾರಿ-ದೀ-ಪ-ವಾ-ಗ-ಬಲ್ಲವು
ದಾರಿದ್ರ್ಯ
ದಾರಿದ್ರ್ಯ-ಗ-ಳಿಂದ
ದಾರಿದ್ರ್ಯದ
ದಾರಿಯ
ದಾರಿಯನ್ನು
ದಾರಿಯನ್ನೇ
ದಾರಿಯಲ್ಲಿ
ದಾರಿ-ಯಲ್ಲಿಯೇ
ದಾರಿಯಲ್ಲೇ
ದಾರಿಯಿಂದ
ದಾರಿಯಿಲ್ಲ
ದಾರಿಯು
ದಾರಿಯೂ
ದಾರಿಯೆಂದು
ದಾರಿ-ಹಿ-ಡಿ-ಯು-ವೆನೋ
ದಾರುಣ
ದಾರುಣತೆ
ದಾರುಣವೆ
ದಾರು-ಣಸ್ಥಿ-ತಿಯ
ದಾರ್ಢ್ಯ
ದಾರ್ಶನಿಕ
ದಾರ್ಶ-ನಿ-ಕ-ನೊಬ್ಬನ
ದಾರ್ಶ-ನಿ-ಕರು
ದಾರ್ಶ-ನಿ-ಕರೂ
ದಾಳ
ದಾಳಿ-ಕಾ-ರರೂ
ದಾವಾ-ನ-ಲದ
ದಾವಾ-ನ-ಲ-ವನ್ನು
ದಾಸ
ದಾಸತ್ವಕ್ಕೆ
ದಾಸ-ನನ್ನಾಗಿ
ದಾಸನಲಿ
ದಾಸ-ನಾ-ಗ-ದಂತೆ
ದಾಸ-ನಾ-ಗ-ಬಾ-ರದು
ದಾಸರ
ದಾಸರನ್ನು
ದಾಸ-ರಾ-ದರೆ
ದಾಸರು
ದಾಸರೂ
ದಾಸವಾಣಿ
ದಾಸ-ವಾ-ಣಿ-ಯಲ್ಲೂ
ದಾಸಿಯರು
ದಾಸೋಹ
ದಾಸ್ತಾನು
ದಾಸ್ಯ
ದಾಸ್ಯಕ್ಕೊ-ಳ-ಗಾದ
ದಾಸ್ಯದಿಂದ
ದಾಸ್ಯ-ಭಾ-ವ-ದಿಂದ
ದಾಹದಿಂದ
ದಿ
ದಿಂಡರು
ದಿಂದ
ದಿಕ್ಕನ್ನು
ದಿಕ್ಕನ್ನೇ
ದಿಕ್ಕಿಗೆ
ದಿಕ್ಕಿನ
ದಿಕ್ಕಿನತ್ತ
ದಿಕ್ಕಿನಲ್ಲಿ
ದಿಕ್ಕು
ದಿಕ್ಕು-ಕಾ-ಣದ
ದಿಕ್ಕುಗಳ
ದಿಕ್ಕು-ತೋ-ಚ-ದಂತಾಗಿ
ದಿಕ್ಕೆಟ್ಟರೆ
ದಿಕ್ಕೆಟ್ಟು
ದಿಕ್ಕೆಲ್ಲ
ದಿಕ್ಕೇ
ದಿಕ್ತ-ಟ-ಗಳು
ದಿಕ್ಸೂಚಿಗೆ
ದಿಗಂತ
ದಿಗಂತ-ದಲ್ಲಿ
ದಿಗಿಲು
ದಿಗಿ-ಲು-ಗಳು
ದಿಗ್ದರ್ಶಿ-ಸಿದ
ದಿಗ್ಭ್ರಾಂತ-ನಾದೆ
ದಿಟ
ದಿಟ-ವಾ-ಯಿತು
ದಿಟವೇ
ದಿಟ್ಟ
ದಿಟ್ಟತನ
ದಿಟ್ಟ-ತ-ನ-ವನ್ನು
ದಿಟ್ಟಿಸಿ
ದಿಟ್ಟಿಸುತ್ತ
ದಿಟ್ಟಿ-ಸುತ್ತಿದೆ
ದಿಣ್ಣೆಯ
ದಿತ್ತು
ದಿನ
ದಿನಂಪ್ರತಿ
ದಿನಕ್ಕೆ
ದಿನಗಳ
ದಿನ-ಗ-ಳನ್ನು
ದಿನ-ಗ-ಳಲ್ಲಾ-ಗುವ
ದಿನ-ಗ-ಳಲ್ಲಿ
ದಿನ-ಗ-ಳಲ್ಲೂ
ದಿನ-ಗ-ಳಲ್ಲೇ
ದಿನ-ಗ-ಳ-ವ-ರೆಗೂ
ದಿನ-ಗ-ಳ-ವ-ರೆಗೆ
ದಿನ-ಗ-ಳಾ-ಗಿ-ವೆ-ಯಷ್ಟೆ
ದಿನ-ಗ-ಳಿಂದ
ದಿನ-ಗ-ಳಿ-ಗೊಮ್ಮೆ
ದಿನಗಳು
ದಿನಗಳೂ
ದಿನಗಳೇ
ದಿನ-ಚ-ರಿಯ
ದಿನ-ಚ-ರಿ-ಯನ್ನೂ
ದಿನದ
ದಿನದಲ್ಲಿ
ದಿನದಿಂದ
ದಿನ-ದಿಂದಲೇ
ದಿನ-ದಿ-ನದ
ದಿನ-ದಿ-ನವೂ
ದಿನನಿತ್ಯ
ದಿನ-ನಿತ್ಯದ
ದಿನ-ನಿತ್ಯವೂ
ದಿನವನ್ನು
ದಿನವನ್ನೂ
ದಿನ-ವಾ-ದರೂ
ದಿನವಿಡೀ
ದಿನವೂ
ದಿನವೆಲ್ಲ
ದಿನಾಂಕ
ದಿನಾಂಕದ
ದಿನೇ
ದಿನೇಶ
ದಿನೇಶನ
ದಿನೇಶನು
ದಿರದು
ದಿರುವುದೇ
ದಿಲ್ಖುಶ್
ದಿಲ್ಲ
ದಿಲ್ಲಿಗೆ
ದಿಲ್ಲಿಯ
ದಿವ-ಸ-ಗಳ
ದಿವ-ಸ-ಗ-ಳಿಂದ
ದಿವ್ಯ
ದಿವ್ಯಜ್ಞಾ-ನ-ವನ್ನೂ
ದಿವ್ಯತೆ
ದಿವ್ಯತೆಯ
ದಿವ್ಯ-ತೆ-ಯನ್ನು
ದಿವ್ಯ-ತೆ-ಯನ್ನೇರ
ದಿವ್ಯ-ತೆ-ಯಲ್ಲಿ
ದಿವ್ಯ-ತೆ-ಯಲ್ಲಿ-ಡುವ
ದಿವ್ಯ-ತೆ-ಯಲ್ಲಿನ
ದಿವ್ಯತ್ವ-ದಲ್ಲಿ
ದಿವ್ಯ-ದರ್ಶ-ನ-ಗಳು
ದಿವ್ಯದಾರಿ
ದಿವ್ಯದೃಷ್ಟಿ
ದಿವ್ಯನಿಧಿ
ದಿವ್ಯ-ನಿ-ಧಿ-ಯಾದ
ದಿವ್ಯ-ಪು-ರು-ಷ-ರನ್ನು
ದಿವ್ಯಪ್ರೀತಿ
ದಿವ್ಯಪ್ರೀ-ತಿಯ
ದಿವ್ಯ-ಭಾ-ವಾ-ಪನ್ನ-ರಾ-ದರು
ದಿವ್ಯವಾಣಿ
ದಿವ್ಯ-ವಾ-ಣಿ-ಯಲ್ಲಿ
ದಿವ್ಯವಾದ
ದಿವ್ಯ-ವಾ-ದು-ದನ್ನು
ದಿವ್ಯಶಕ್ತಿ
ದಿವ್ಯ-ಶಕ್ತಿ-ಯನ್ನು
ದಿವ್ಯ-ಶಕ್ತಿ-ಯಿಂದ
ದಿವ್ಯಶಾಂತಿ
ದಿವ್ಯ-ಸಾನ್ನಿಧ್ಯ
ದಿವ್ಯಸ್ಥಿ-ತಿ-ಗೇ-ರಲೂ
ದಿವ್ಯಾ-ನಂದದ
ದಿವ್ಯಾಗ್ನಿ-ಯಲ್ಲಿ
ದಿವ್ಯಾ-ನಂದ-ವನ್ನು
ದಿವ್ಯಾನು
ದಿವ್ಯೌಷಧ
ದಿವ್ಯೌ-ಷ-ಧಿ-ಯಾ-ಗ-ಬಲ್ಲುದು
ದಿಸೆಯಲ್ಲಿ
ದೀಕ್ಷಾ-ಬದ್ಧ-ರಾಗು
ದೀತು
ದೀನ
ದೀನ-ತೆ-ಯಿಂದ
ದೀನ-ದ-ಲಿ-ತರ
ದೀನ-ದ-ಲಿ-ತ-ರನ್ನು
ದೀನ-ದ-ಲಿ-ತ-ರಿಗೆ
ದೀನ-ದುರ್ಬಲ
ದೀನನಾಗಿ
ದೀನರಲ್ಲಿ
ದೀನರಾಗಿ
ದೀನಳಾಗಿ
ದೀನ-ಹೀ-ನ-ತೆ-ಗಳೇ
ದೀಪಗಳು
ದೀಪ-ಗ-ಳೆಲ್ಲ
ದೀಪದ
ದೀಪಹಾಕಿ
ದೀರ್ಘ
ದೀರ್ಘ-ಕಾ-ಯದ
ದೀರ್ಘಕಾಲ
ದೀರ್ಘ-ಕಾ-ಲದ
ದೀರ್ಘ-ಕಾ-ಲ-ದಿಂದ
ದೀರ್ಘ-ಕಾ-ಲ-ವಿದ್ದ
ದೀರ್ಘವಾಗಿ
ದೀರ್ಘ-ವಾ-ಗಿ-ದೆಯೇ
ದೀರ್ಘ-ವಾ-ಗಿಲ್ಲ
ದೀರ್ಘವಾದ
ದೀರ್ಘಸೇವಾ
ದೀರ್ಘಾ-ಯು-ಗ-ಳಾ-ಗ-ಬಲ್ಲ-ರುಈ
ದೀರ್ಘಾ-ಯು-ವಾ-ಗಲು
ದೀರ್ಘಾವಧಿ
ದೀರ್ಘಾ-ವ-ಧಿಯ
ದುಂಟು
ದುಂದು-ಗಾ-ರ-ರಲ್ಲ
ದುಂಬಿ-ಗ-ಳನ್ನು
ದುಂಬುತ್ತಾನೆ
ದುಃಖ
ದುಃಖ-ದುಮ್ಮಾ-ನ-ಗ-ಳನ್ನು
ದುಃಖಇವು
ದುಃಖಕ್ಕೀಡು
ದುಃಖಕ್ಕೂ
ದುಃಖಕ್ಕೆ
ದುಃಖಕ್ಲೇ-ಶ-ಗ-ಳಿಂದ
ದುಃಖಗಳ
ದುಃಖ-ಗ-ಳನ್ನು
ದುಃಖ-ಗ-ಳಿಗೂ
ದುಃಖ-ಗ-ಳಿಗೆ
ದುಃಖ-ಗ-ಳಿ-ಗೇನು
ದುಃಖ-ಗ-ಳೇನು
ದುಃಖತಪ್ತ
ದುಃಖ-ತಪ್ತ-ರಾದ
ದುಃಖ-ತಪ್ತ-ವಾ-ಗು-ವುದು
ದುಃಖದ
ದುಃಖ-ದಲ್ಲಿದ್ದ
ದುಃಖ-ದಾ-ಯಕ
ದುಃಖದಿಂದ
ದುಃಖ-ದು-ಗು-ಡ-ಗ-ಳನ್ನೆಲ್ಲ
ದುಃಖ-ದು-ರಂತ-ಗ-ಳನ್ನು
ದುಃಖ-ದೌ-ರಾತ್ಮ್ಯ-ಗ-ಳನ್ನು
ದುಃಖ-ನಿ-ವೃತ್ತಿ-ಯನ್ನೂ
ದುಃಖ-ಪ-ಡುವ
ದುಃಖ-ಭಾ-ಗಿ-ಯಾಗಿ
ದುಃಖಮಯ
ದುಃಖ-ಮ-ಯ-ವಾ-ಗಲಿ
ದುಃಖ-ರೂ-ಪ-ವಾದ
ದುಃಖ-ವನ್ನಾ-ಗಲಿ
ದುಃಖವನ್ನು
ದುಃಖ-ವಾ-ಗಲೀ
ದುಃಖ-ವಾ-ಗಿತ್ತೆಂದರೆ
ದುಃಖ-ವಾ-ಗುತ್ತದೆ
ದುಃಖ-ವಾ-ಗು-ವು-ದಿಲ್ಲವೆ
ದುಃಖವು
ದುಃಖವೇ
ದುಃಖವೊಂದೇ
ದುಃಖ-ಶೋ-ಕವೂ
ದುಃಖ-ಸಂಕ-ಟ-ಗ-ಳನ್ನು
ದುಃಖ-ಸಂಕ-ಟ-ಗಳೂ
ದುಃಖಿ
ದುಃಖಿಗಳು
ದುಃಖಿಗಳೇ
ದುಃಖಿತ
ದುಃಖಿ-ತ-ನಲ್ಲೂ
ದುಃಖಿ-ತ-ನಾ-ಗಿದ್ದಾನೆ
ದುಃಖಿ-ತ-ನಾ-ದರೂ
ದುಃಖಿ-ತ-ರಾ-ಗು-ವರು
ದುಃಖಿ-ತ-ರಾದ
ದುಃಖಿ-ತ-ರಾ-ದ-ವ-ರೆಷ್ಟೋ
ದುಃಖಿತರೇ
ದುಃಖಿ-ತ-ಳಾಗಿ
ದುಃಖಿ-ತ-ಳಾ-ದಳು
ದುಃಸ್ಥಿತಿ
ದುಃಸ್ಥಿತಿಗೆ
ದುಃಸ್ಥಿತಿಯ
ದುಃಸ್ಥಿ-ತಿ-ಯನ್ನು
ದುಃಸ್ಥಿ-ತಿ-ಯಲ್ಲಿ
ದುಃಸ್ಥಿ-ತಿ-ಯಿಂದ
ದುಗು-ಡ-ವನ್ನು
ದುಡಿ
ದುಡಿದ
ದುಡಿ-ದದ್ದಕ್ಕಾಗಿ
ದುಡಿದರು
ದುಡಿದರೆ
ದುಡಿ-ದ-ವ-ರಿದ್ದಾ-ರೆಂದೂ
ದುಡಿದಿದ್ದ
ದುಡಿ-ದಿದ್ದರು
ದುಡಿದು
ದುಡಿ-ದು-ದ-ರಿಂದ
ದುಡಿದೆನು
ದುಡಿಮೆ
ದುಡಿಮೆಯ
ದುಡಿ-ಮೆ-ಯಲ್ಲಿ
ದುಡಿ-ಮೆ-ಯಿಂದ
ದುಡಿ-ಮೆ-ಯಿಂದಾಗಿ
ದುಡಿ-ಮೆ-ಯಿಲ್ಲದ
ದುಡಿ-ಯ-ತೊ-ಡ-ಗಿದೆ
ದುಡಿ-ಯ-ದಿದ್ದರೆ
ದುಡಿಯದೆ
ದುಡಿ-ಯ-ಬಲ್ಲ
ದುಡಿ-ಯ-ಬಲ್ಲರು
ದುಡಿ-ಯ-ಬೇ-ಕಿಲ್ಲ
ದುಡಿ-ಯ-ಬೇಕು
ದುಡಿ-ಯ-ಬೇಡ
ದುಡಿಯಲಿ
ದುಡಿ-ಯ-ಲಿಲ್ಲ
ದುಡಿಯಲು
ದುಡಿ-ಯ-ಲೇ-ಬೇಕು
ದುಡಿಯು
ದುಡಿಯುತ್ತ
ದುಡಿ-ಯುತ್ತಿದ್ದ
ದುಡಿ-ಯುತ್ತಿದ್ದರು
ದುಡಿ-ಯುತ್ತಿದ್ದರೂ
ದುಡಿ-ಯುತ್ತಿದ್ದಾರೆ
ದುಡಿ-ಯುತ್ತಿದ್ದಾ-ರೆಂಬು-ದಕ್ಕೊಂದು
ದುಡಿ-ಯುತ್ತಿದ್ದು-ದ-ರಿಂದ
ದುಡಿ-ಯುತ್ತಿದ್ದೆ
ದುಡಿ-ಯುತ್ತಿದ್ದೇನೆ
ದುಡಿ-ಯುತ್ತಿ-ರುವ
ದುಡಿಯುವ
ದುಡಿ-ಯು-ವರು
ದುಡಿ-ಯು-ವರೋ
ದುಡಿ-ಯು-ವ-ವ-ನಾ-ಗಿದ್ದು
ದುಡಿ-ಯು-ವ-ವರ
ದುಡಿ-ಯು-ವ-ವ-ರನ್ನು
ದುಡಿ-ಯು-ವ-ವ-ರಿಗೂ
ದುಡಿ-ಯು-ವ-ವ-ರಿಗೇ
ದುಡಿ-ಯು-ವ-ವರು
ದುಡಿ-ಯು-ವ-ವರೂ
ದುಡಿ-ಯು-ವಾಗ
ದುಡಿ-ಯು-ವು-ದಾ-ಗಲಿ
ದುಡಿ-ಸಿ-ಕೊಳ್ಳ
ದುಡು-ಕ-ಬೇಡ
ದುಡು-ಕಿ-ದರೆ
ದುಡ್ಡು
ದುಡ್ಡು-ಮಾ-ಡುವ
ದುದನ್ನು
ದುದಾದರೆ
ದುದು
ದುದೇ
ದುಮ್ಮಾ-ನ-ಗಳ
ದುಮ್ಮಾ-ನ-ಗ-ಳನ್ನು
ದುಮ್ಮಾ-ನ-ಗ-ಳನ್ನೂ
ದುಮ್ಮಾ-ನ-ಗ-ಳಿಗೂ
ದುರಂತ
ದುರಂತಕ್ಕೂ
ದುರಂತಕ್ಕೆ
ದುರಂತಕ್ಕೆ-ಳೆ-ಯುವ
ದುರಂತಕ್ಕೆ-ಳೆ-ಸುತ್ತದೆ
ದುರಂತಕ್ಕೆ-ಳೆ-ಸುವ
ದುರಂತ-ಗಳು
ದುರಂತ-ಗಳೇ
ದುರಂತದ
ದುರಂತ-ದಂಚಿ-ನಲ್ಲಿ
ದುರಂತ-ದತ್ತ
ದುರಂತ-ದಲ್ಲಿ
ದುರಂತ-ದೆಡೆ
ದುರಂತ-ವನ್ನು
ದುರಂತ-ವಾ-ಗ-ದಿ-ರಲಿ
ದುರದೃಷ್ಟ
ದುರ-ದೃಷ್ಟದ
ದುರ-ದೃಷ್ಟ-ದಿಂದ
ದುರ-ದೃಷ್ಟ-ವನ್ನು
ದುರ-ದೃಷ್ಟ-ವ-ಶ-ದಿಂದ
ದುರ-ದೃಷ್ಟ-ವೆಂದರೆ
ದುರ-ಭಿ-ಮಾನ
ದುರಭ್ಯಾಸ
ದುರಭ್ಯಾ-ಸಕ್ಕೂ
ದುರಭ್ಯಾ-ಸ-ಗಳ
ದುರಭ್ಯಾ-ಸ-ಗ-ಳೆಂದು
ದುರಭ್ಯಾ-ಸ-ಗ-ಳೇನು
ದುರಭ್ಯಾ-ಸ-ದಿಂದ
ದುರಭ್ಯಾ-ಸ-ವನ್ನು
ದುರವಸ್ಥೆ
ದುರ-ವಸ್ಥೆಯ
ದುರ-ಹಂಕಾರ
ದುರ-ಹಂಕಾ-ರದ
ದುರ-ಹಂಕಾ-ರಿ-ಯಾ-ಗಿದ್ದ
ದುರಾಚಾರೀ
ದುರಾತ್ಮರು
ದುರಾಶೆಯ
ದುರಾ-ಶೆ-ಯಿಂದ
ದುರಾ-ಸಕ್ತಿ-ಗ-ಳನ್ನೂ
ದುರಾಸೆ
ದುರಾ-ಸೆ-ಗಳೂ
ದುರಾಸೆಯ
ದುರಿತ
ದುರಿ-ತ-ಗ-ಳನ್ನು
ದುರಿ-ತ-ಗಳು
ದುರಿ-ತ-ಗ-ಳೆಲ್ಲ
ದುರಿ-ತ-ವನ್ನು
ದುರುಗಪ್ಪ
ದುರು-ಗುಟ್ಟಿ-ಕೊಂಡೇ
ದುರುದ್ದೇ-ಶದ
ದುರು-ಪ-ಯೋಗ
ದುರು-ಪ-ಯೋ-ಗ-ಇವು
ದುರು-ಪ-ಯೋ-ಗ-ದಿಂದ
ದುರು-ಪ-ಯೋ-ಗ-ಪ-ಡಿ-ಸಿ-ಕೊಂಡರು
ದುರು-ಪ-ಯೋ-ಗ-ಪ-ಡಿ-ಸು-ವ-ವರೂ
ದುರು-ಪ-ಯೋ-ಗ-ಮಾ-ಡಲು
ದುರು-ಪ-ಯೋ-ಗ-ವನ್ನು
ದುರು-ಪ-ಯೋ-ಗ-ವಾ-ಗ-ಬ-ಹು-ದೆಂಬ
ದುರು-ಪ-ಯೋ-ಗ-ವಾ-ಗುತ್ತ
ದುರ್
ದುರ್ಗಂಧ-ವನ್ನು
ದುರ್ಗ-ಣ-ಗಳ
ದುರ್ಗತಿ
ದುರ್ಗ-ತಿ-ಯನ್ನು
ದುರ್ಗಮ
ದುರ್ಗಾದಾಸ
ದುರ್ಗಾ-ಮಾ-ತೆಯ
ದುರ್ಗುಣ
ದುರ್ಗು-ಣ-ಗಳು
ದುರ್ಗು-ಣ-ಗ-ಳೆಲ್ಲವೂ
ದುರ್ಘ-ಟ-ನೆ-ಗ-ಳನ್ನು
ದುರ್ಘ-ಟ-ನೆ-ಗ-ಳನ್ನೇ
ದುರ್ಘ-ಟ-ನೆ-ಗ-ಳಲ್ಲ
ದುರ್ಘ-ಟ-ನೆ-ಗಳು
ದುರ್ಘ-ಟ-ನೆಗೆ
ದುರ್ಘ-ಟ-ನೆ-ಯನ್ನು
ದುರ್ಘ-ಟ-ನೆ-ಯಿಂದ
ದುರ್ಜನರ
ದುರ್ಜ-ನ-ರಿ-ರು-ವಂತೆ
ದುರ್ದೆ-ಶೆ-ಯುಂಟಾ-ಗುತ್ತದೆ
ದುರ್ದೈವ
ದುರ್ದೈವದ
ದುರ್ದೈವೀ
ದುರ್ನಡತೆ
ದುರ್ನ-ಡ-ತೆ-ಗ-ಳಿಂದ
ದುರ್ನ-ಡ-ತೆ-ಗಳು
ದುರ್ನ-ಡ-ತೆಯ
ದುರ್ಬಲ
ದುರ್ಬ-ಲ-ಗೊ-ಳಿ-ಸುವ
ದುರ್ಬ-ಲ-ಗೊ-ಳಿಸಿ
ದುರ್ಬ-ಲ-ಗೊ-ಳಿ-ಸಿತ್ತು
ದುರ್ಬ-ಲ-ಗೊ-ಳಿ-ಸುತ್ತ
ದುರ್ಬ-ಲ-ಗೊ-ಳಿ-ಸುತ್ತದೆ
ದುರ್ಬ-ಲ-ಗೊ-ಳಿ-ಸುತ್ತಿದೆ
ದುರ್ಬ-ಲ-ಗೊ-ಳಿ-ಸುವ
ದುರ್ಬ-ಲ-ಗೊ-ಳಿ-ಸು-ವುದು
ದುರ್ಬಲತೆ
ದುರ್ಬ-ಲ-ತೆ-ಗ-ಳನ್ನು
ದುರ್ಬ-ಲ-ತೆ-ಗ-ಳಿಂದ
ದುರ್ಬ-ಲ-ತೆ-ಚಿಂತೆ
ದುರ್ಬ-ಲ-ತೆ-ಯನ್ನಲ್ಲ
ದುರ್ಬ-ಲ-ತೆ-ಯನ್ನು
ದುರ್ಬ-ಲ-ತೆ-ಯಿಂದ
ದುರ್ಬ-ಲ-ತೆಯೇ
ದುರ್ಬ-ಲ-ನಾ-ಗ-ಬೇಡ
ದುರ್ಬ-ಲ-ನಾಗಿ
ದುರ್ಬ-ಲ-ನಾ-ಗಿದ್ದ
ದುರ್ಬ-ಲ-ನಾ-ಗುತ್ತಿದ್ದೇನೆ
ದುರ್ಬ-ಲ-ನಾ-ದ-ನೆಂದರ್ಥವೇ
ದುರ್ಬ-ಲ-ನಿಗೆ
ದುರ್ಬಲನು
ದುರ್ಬಲನೂ
ದುರ್ಬ-ಲ-ನೆಂದೂ
ದುರ್ಬ-ಲ-ರನ್ನಾಗಿ
ದುರ್ಬ-ಲ-ರನ್ನಾ-ಗಿ-ಸಲು
ದುರ್ಬ-ಲ-ರನ್ನು
ದುರ್ಬ-ಲ-ರನ್ನೇ
ದುರ್ಬ-ಲ-ರಲ್ಲಿ
ದುರ್ಬ-ಲ-ರಾಗಿ
ದುರ್ಬ-ಲ-ರಾ-ಗು-ವರು
ದುರ್ಬಲರೂ
ದುರ್ಬ-ಲ-ರೆಂದೂ
ದುರ್ಬ-ಲ-ರೆಂದೇ
ದುರ್ಬಲರೇ
ದುರ್ಬಲರೋ
ದುರ್ಬ-ಲ-ವಾ-ಗಲೂ
ದುರ್ಬ-ಲ-ವಾಗಿ
ದುರ್ಬ-ಲ-ವಾ-ಗುತ್ತ
ದುರ್ಬ-ಲ-ವಾ-ಗುತ್ತವೆ
ದುರ್ಬ-ಲ-ವಾ-ಗುತ್ತಿತ್ತು
ದುರ್ಬ-ಲ-ವಾ-ಗು-ವಂತೆ
ದುರ್ಬ-ಲ-ವಾ-ದು-ವೆಂದರ್ಥ-ವಲ್ಲ
ದುರ್ಬಲವೂ
ದುರ್ಬೀ-ಜ-ಗ-ಳನ್ನು
ದುರ್ಬೀ-ಜ-ವನ್ನು
ದುರ್ಭರ
ದುರ್ಭಾ-ವ-ನೆ-ಗಳು
ದುರ್ಭಾ-ವ-ನೆ-ಗಳೂ
ದುರ್ಭಾ-ವ-ನೆ-ಯನ್ನು
ದುರ್ಮ-ರ-ಣದ
ದುರ್ಮಾರ್ಗ-ದಲ್ಲಿ
ದುರ್ಮಾರ್ಗ-ವನ್ನು
ದುರ್ಯೋಧನ
ದುರ್ಯೋ-ಧ-ನ-ನಿಂದ
ದುರ್ಯೋ-ಧ-ನರು
ದುರ್ಯೋ-ಧ-ನ-ರು-ಗಳ
ದುರ್ಲಭ
ದುರ್ಲ-ಭ-ವಾದ
ದುರ್ವರ್ತನೆ
ದುರ್ವರ್ತ-ನೆ-ಗ-ಳನ್ನೂ
ದುರ್ವಾಸನೆ
ದುರ್ವ್ಯ-ವ-ಹಾ-ರ-ವನ್ನೂ
ದುರ್ವ್ಯ-ಸ-ನಕ್ಕೆ
ದುಶ್ಚ-ಟ-ಗ-ಳನ್ನೂ
ದುಶ್ಚ-ಟ-ಗ-ಳಿಗೆ
ದುಷ್ಕರ್ಮಕ್ಕಾಗಿ
ದುಷ್ಕರ್ಮ-ಗಳ
ದುಷ್ಕರ್ಮ-ಗ-ಳಿಗೂ
ದುಷ್ಕರ್ಮ-ಗ-ಳಿಗೆ
ದುಷ್ಕರ್ಮದ
ದುಷ್ಕರ್ಮ-ಫ-ಲ-ವೆಂದೇ
ದುಷ್ಕಾರ್ಯಕ್ಕೆ
ದುಷ್ಕೃತ್ಯ
ದುಷ್ಟ
ದುಷ್ಟ-ಕರ್ಮ-ಗಳು
ದುಷ್ಟ-ಕೂ-ಟಕ್ಕೆ
ದುಷ್ಟಕ್ರಿ-ಯೆಯ
ದುಷ್ಟಚಟ
ದುಷ್ಟತನ
ದುಷ್ಟ-ತ-ನದ
ದುಷ್ಟ-ತ-ನ-ವನ್ನು
ದುಷ್ಟ-ದೃಷ್ಟಿ-ಯಿಂದ
ದುಷ್ಟನಲ್ಲೂ
ದುಷ್ಟನಿಗೆ
ದುಷ್ಟರ
ದುಷ್ಟರಾಗಿ
ದುಷ್ಟ-ರಾ-ದರು
ದುಷ್ಟರಿಂದ
ದುಷ್ಟರು
ದುಷ್ಟರೂ
ದುಷ್ಟ-ವಾ-ಗಿದ್ದರೆ
ದುಷ್ಟವ್ಯಕ್ತಿ-ಗಳು
ದುಷ್ಟ-ಶಬ್ದ-ಗಳು
ದುಷ್ಟಸ್ವ-ಭಾ-ವ-ವನ್ನು
ದುಷ್ಟ-ಹಂಚಿ-ಕೆ-ಗಳು
ದುಷ್ಪ-ರಿ-ಣಾಮ
ದುಷ್ಪ-ರಿ-ಣಾ-ಮ-ಗಳ
ದುಷ್ಪ-ರಿ-ಣಾ-ಮದ
ದುಷ್ಪ-ರಿ-ಣಾ-ಮ-ವನ್ನ-ರಿ-ತಾಗ
ದುಷ್ಪ-ರಿ-ಣಾ-ಮ-ವನ್ನು
ದುಸ್ಥಿತಿಯ
ದುಸ್ಸಂಗ-ದಿಂದ
ದೂಡಿದರೇ
ದೂತರನ್ನು
ದೂತರಿಗೆ
ದೂತರು
ದೂರ
ದೂರ-ವಾ-ಗಿತ್ತಲ್ಲದೆ
ದೂರ-ವಿದ್ದೇನೆ
ದೂರಕ್ಕೆ
ದೂರಕ್ಕೆ-ಸೆದು
ದೂರಕ್ಕೆ-ಸೆ-ಯ-ಬೇಕು
ದೂರಕ್ಕೆ-ಸೆ-ಯ-ಬೇ-ಕೆಂದು
ದೂರಕ್ಕೆ-ಸೆ-ಯ-ಲ-ರಿ-ಯದ
ದೂರಕ್ಕೆ-ಸೆ-ಯುವ
ದೂರಕ್ಕೆ-ಸೆ-ಯು-ವುದು
ದೂರಕ್ಕೋ-ಡಿ-ಸಲು
ದೂರಕ್ಕೋ-ಡಿ-ಸಿದ್ದವು
ದೂರಕ್ಕೋ-ಡಿಸು
ದೂರಕ್ಕೋ-ಡು-ವುವು
ದೂರ-ಗೊ-ಳಿ-ಸಲು
ದೂರ-ಗೊ-ಳಿಸಿ
ದೂರ-ಗೊ-ಳಿ-ಸು-ಹೀ-ಗೆಂದು
ದೂರದ
ದೂರ-ದರ್ಶ-ಕ-ಗಳ
ದೂರ-ದರ್ಶ-ಕದ
ದೂರ-ದರ್ಶನ
ದೂರದರ್ಶಿ
ದೂರ-ದರ್ಶಿತ್ವದ
ದೂರದಲ್ಲಿ
ದೂರ-ದಲ್ಲಿತ್ತು
ದೂರ-ದಲ್ಲಿದೆ
ದೂರ-ದಲ್ಲಿದ್ದ
ದೂರ-ದಲ್ಲಿ-ರುವ
ದೂರ-ದ-ವ-ರೆಗೆ
ದೂರ-ದೂ-ರ-ದಿಂದ
ದೂರದೃಷ್ಟಿ
ದೂರ-ದೃಷ್ಟಿಯ
ದೂರ-ದೃಷ್ಟಿ-ಯೊಂದಿಗೆ
ದೂರಬೇಡ
ದೂರ-ಮ-ನಸ್ಪರ್ಶ
ದೂರ-ಮಾ-ಡ-ಬಲ್ಲ
ದೂರ-ಮಾ-ಡ-ಬೇ-ಕೆಂದು
ದೂರ-ಮಾ-ಡಲು
ದೂರ-ಮಾ-ಡಲೂ
ದೂರಮಾಡಿ
ದೂರ-ಮಾ-ಡಿ-ದರು
ದೂರ-ಮಾ-ಡಿದ್ದಾರೆ
ದೂರ-ಮಾ-ಡುವ
ದೂರ-ಮಾ-ಡು-ವಲ್ಲಿ
ದೂರ-ಮಾ-ಡು-ವ-ವನು
ದೂರ-ಮಾ-ಡು-ವು-ದಕ್ಕಾಗಿ
ದೂರ-ವಾ-ಗದ
ದೂರ-ವಾ-ಗ-ಬ-ಹುದು
ದೂರ-ವಾ-ಗ-ಬೇ-ಕೆಂದು
ದೂರ-ವಾ-ಗ-ಲಿಲ್ಲ
ದೂರ-ವಾ-ಗಲು
ದೂರವಾಗಿ
ದೂರ-ವಾ-ಗಿತ್ತು
ದೂರ-ವಾ-ಗಿ-ರ-ಲಿಲ್ಲ
ದೂರ-ವಾ-ಗುತ್ತದೆ
ದೂರ-ವಾ-ಗುತ್ತವೆ
ದೂರ-ವಾ-ಗುತ್ತಿ-ದೆಯೇ
ದೂರ-ವಾ-ಗುತ್ತಿದ್ದವು
ದೂರ-ವಾ-ಗುತ್ತಿ-ರು-ವಾ-ಗಲೇ
ದೂರ-ವಾ-ಗುತ್ತೇ-ನೆಂಬುದು
ದೂರ-ವಾ-ಗುವ
ದೂರ-ವಾ-ಗು-ವಂತಿಲ್ಲ
ದೂರ-ವಾ-ಗು-ವು-ದಕ್ಕೆ
ದೂರ-ವಾ-ಗು-ವು-ದನ್ನು
ದೂರ-ವಾ-ಗು-ವುದು
ದೂರ-ವಾ-ಗು-ವುವು
ದೂರ-ವಾ-ಣಿಯ
ದೂರ-ವಾ-ದರು
ದೂರ-ವಾ-ದವು
ದೂರ-ವಾ-ಯಿತು
ದೂರ-ವಾ-ಯಿ-ತೆಂದರೆ
ದೂರವಿಟ್ಟು
ದೂರ-ವಿದ್ದರೂ
ದೂರ-ವಿ-ರಲಿ
ದೂರ-ವಿ-ರಲು
ದೂರ-ವಿ-ರುವ
ದೂರವಿಲ್ಲ
ದೂರ-ವು-ದಿಲ್ಲ
ದೂರವೆ
ದೂರ-ಸ-ರಿ-ದರೆ
ದೂರ-ಸ-ರಿದು
ದೂರಾ-ಗುತ್ತದೆ
ದೂರಿಕೊಂಡ
ದೂರಿತ್ತ-ನೆಂದು
ದೂರಿದ
ದೂರಿ-ದ-ನಂತೆ
ದೂರೀ-ಕ-ರಿ-ಸ-ಬ-ಹುದು
ದೂರೀ-ಕ-ರಿ-ಸಲ್ಪಟ್ಟು
ದೂರು-ಇತ್ಯಾ-ದಿ-ಗ-ಳನ್ನು
ದೂರುತ್ತ
ದೂರುತ್ತಿ-ರ-ಲಿಲ್ಲ
ದೂಷಣೆ
ದೂಷಣೆಯ
ದೃಢ
ದೃಢ-ಗೊ-ಳಿ-ಸುವ
ದೃಢ-ಪ-ಡಿ-ಸಿ-ಕೊಂಡಿದ್ದಾರೆ
ದೃಢ-ಗೊ-ಳಿ-ಸುತ್ತಾರೆ
ದೃಢ-ಚಿತ್ತರೂ
ದೃಢತೆ
ದೃಢ-ತೆ-ಯನ್ನು
ದೃಢ-ತೆ-ಯಿಂದ
ದೃಢ-ನಂಬಿಕೆ
ದೃಢ-ನಂಬಿ-ಕೆಯ
ದೃಢ-ನಂಬಿ-ಕೆ-ಯನ್ನಿ-ರಿಸಿ
ದೃಢ-ನಂಬಿ-ಕೆ-ಯಲ್ಲವೇ
ದೃಢ-ನಂಬಿ-ಕೆ-ಯಿಂದ
ದೃಢ-ನಂಬಿ-ಕೆಯೂ
ದೃಢ-ನಿರ್ಧಾರ
ದೃಢ-ನಿರ್ಧಾ-ರವೇ
ದೃಢ-ನಿ-ಲು-ಮೆ-ಯನ್ನು
ದೃಢ-ನಿಶ್ಚಯ
ದೃಢ-ನಿಶ್ಚ-ಯ-ಇವೇ
ದೃಢ-ನಿಶ್ಚ-ಯ-ಮಾಡು
ದೃಢನಿಷ್ಠೆ
ದೃಢ-ನಿಷ್ಠೆ-ಯಿಂದ
ದೃಢ-ಪಟ್ಟಿದೆ
ದೃಢ-ಪ-ಡಿಸಿ
ದೃಢ-ಪ-ಡಿ-ಸಿ-ಕೊಂಡ
ದೃಢ-ಪ-ಡಿ-ಸಿ-ಕೊಂಡರೆ
ದೃಢ-ಪ-ಡಿ-ಸಿ-ಕೊಂಡ-ವರು
ದೃಢ-ಪ-ಡಿ-ಸಿ-ಕೊಂಡಿ-ರ-ಬ-ಹುದು
ದೃಢ-ಪ-ಡಿ-ಸಿ-ಕೊಂಡು
ದೃಢ-ಪ-ಡಿ-ಸಿ-ಕೊಂಡೆ-ವೆಂದರೆ
ದೃಢ-ಪ-ಡಿ-ಸಿ-ಕೊಳ್ಳ-ಬ-ಹುದು
ದೃಢ-ಪ-ಡಿ-ಸುವ
ದೃಢಪ್ರಜ್ಞೆ
ದೃಢಪ್ರ-ತಿಷ್ಠ-ನಾ-ದರೆ
ದೃಢಪ್ರ-ತಿಷ್ಠೆ
ದೃಢ-ಬ-ಲ-ದಿಂದ
ದೃಢ-ಭಕ್ತಿ-ಯಿಂದ
ದೃಢ-ಮ-ನಸ್ಕ-ಳಾ-ಗಿದ್ದು-ಕೊಂಡು
ದೃಢ-ಮ-ನಸ್ಸು
ದೃಢವಾಗಿ
ದೃಢ-ವಾ-ಗಿತ್ತು
ದೃಢ-ವಾ-ಗಿದೆ
ದೃಢ-ವಾ-ಗುತ್ತಲೇ
ದೃಢ-ವಾ-ಗು-ವಂತೆ
ದೃಢ-ವಾ-ಗು-ವ-ವ-ರೆಗೂ
ದೃಢ-ವಾ-ಗು-ವುದು
ದೃಢವಾದ
ದೃಢ-ವಾ-ದರೆ
ದೃಢ-ವಾ-ಯಿತು
ದೃಢ-ವಿಶ್ವಾಸ
ದೃಢ-ವಿಶ್ವಾ-ಸದ
ದೃಢ-ವಿಶ್ವಾ-ಸ-ದಿಂದ
ದೃಢ-ವಿಶ್ವಾ-ಸ-ವಾ-ಗಿತ್ತು
ದೃಢ-ವಿಶ್ವಾ-ಸ-ವಿ-ರಿ-ಸಿದ
ದೃಢ-ವಿಶ್ವಾ-ಸ-ವಿ-ರುತ್ತದೆ
ದೃಢ-ವಿಶ್ವಾ-ಸವು
ದೃಢ-ವಿಶ್ವಾ-ಸವೇ
ದೃಢವೂ
ದೃಢಶ್ರದ್ಧೆ-ಯನ್ನು
ದೃಢಶ್ರದ್ಧೆ-ಯಿಂದ
ದೃಢಶ್ರದ್ಧೆಯು
ದೃಢ-ಸಂಕಲ್ಪ
ದೃಢ-ಸಂಕಲ್ಪ-ದಿಂದ
ದೃಢಸ್ವ-ರ-ದಲ್ಲಿ
ದೃಢ-ಹೃ-ದ-ಯವೂ
ದೃಢೀ-ಕ-ರಿ-ಸುತ್ತವೆ
ದೃಢೀ-ಕ-ರಿ-ಸುತ್ತೇನೆ
ದೃಶ್ಯ
ದೃಶ್ಯ-ಗ-ಳನ್ನೆಲ್ಲ
ದೃಶ್ಯ-ಗ-ಳನ್ನೇ
ದೃಶ್ಯಗಳು
ದೃಶ್ಯವನ್ನು
ದೃಶ್ಯವೂ
ದೃಷ್ಟಿ
ದೃಷ್ಟಿಕೋನ
ದೃಷ್ಟಿ-ಕೋ-ನಕ್ಕೂ
ದೃಷ್ಟಿ-ಕೋ-ನಕ್ಕೆ
ದೃಷ್ಟಿ-ಕೋ-ನ-ಗ-ಳಿಂದ
ದೃಷ್ಟಿ-ಕೋ-ನ-ಗ-ಳಿಂದಲೇ
ದೃಷ್ಟಿ-ಕೋ-ನದ
ದೃಷ್ಟಿ-ಕೋ-ನ-ದಿಂದ
ದೃಷ್ಟಿ-ಕೋ-ನ-ವನ್ನು
ದೃಷ್ಟಿ-ಕೋ-ನ-ವನ್ನೇ
ದೃಷ್ಟಿ-ಕೋ-ನ-ವಿದೆ
ದೃಷ್ಟಿ-ಕೋ-ನವೂ
ದೃಷ್ಟಿ-ಕೋ-ನ-ವೆಂದಿಟ್ಟು-ಕೊಂಡರೆ
ದೃಷ್ಟಿ-ಕೋ-ನ-ವೆನ್ನ-ಬ-ಹುದು
ದೃಷ್ಟಿ-ಕೋ-ನ-ವೊಂದೇ
ದೃಷ್ಟಿ-ಗ-ಳಿಂದ
ದೃಷ್ಟಿಗಳು
ದೃಷ್ಟಿಗೆ
ದೃಷ್ಟಿ-ಗೋ-ಚ-ರ-ವಾ-ಗು-ವು-ದಿಲ್ಲ
ದೃಷ್ಟಿ-ಗೋ-ಚ-ರ-ವಾ-ದಾಗ
ದೃಷ್ಟಿದೋಷ
ದೃಷ್ಟಿ-ಪಾ-ಟವ
ದೃಷ್ಟಿಯ
ದೃಷ್ಟಿ-ಯನ್ನಿಟ್ಟು-ಕೊಂಡು
ದೃಷ್ಟಿಯನ್ನು
ದೃಷ್ಟಿಯಲ್ಲಿ
ದೃಷ್ಟಿಯಿಂದ
ದೃಷ್ಟಿ-ಯಿಂದಲೂ
ದೃಷ್ಟಿ-ಯಿಂದಲೇ
ದೃಷ್ಟಿಯು
ದೃಷ್ಟಿ-ಶಕ್ತಿ-ಯುಳ್ಳ-ವ-ರಿಗೆ
ದೃಷ್ಟಿ-ಸೃಷ್ಟಿ-ಯಾ-ಯಿತು
ದೃಷ್ಟ್ವಾ
ದೆಂದಿದ್ದರೆ
ದೆಂದು
ದೆಂಬುದೇ
ದೆನೇ
ದೆವ್ವದ
ದೆವ್ವವಾಗಿ
ದೆಸೆಯಿಂದ
ದೆಸೆ-ಯಿಂದಲೇ
ದೆಹಲಿಯ
ದೆಹ-ಲಿ-ಯಲ್ಲಿದ್ದ
ದೇ
ದೇಣಿಗೆ
ದೇಣಿ-ಗೆ-ಯನ್ನು
ದೇಣಿಗೆಯೂ
ದೇತಕ್ಕೆ
ದೇದೀಪ್ಯ-ಮಾನ
ದೇವ
ದೇವಃ
ದೇವತಾ
ದೇವ-ತೆ-ಗಳ
ದೇವ-ತೆ-ಗ-ಳಲ್ಲಿ
ದೇವ-ತೆ-ಗ-ಳಿರಾ
ದೇವ-ತೆ-ಗಳು
ದೇವ-ತೆ-ಗಳೂ
ದೇವ-ತೆ-ಯಾಗಿ
ದೇವತ್ವಕ್ಕೆ
ದೇವತ್ವದ
ದೇವತ್ವ-ವನ್ನೂ
ದೇವ-ದತ್ತ-ವಾಗಿ
ದೇವ-ದೇ-ವ-ತೆ-ಗ-ಳಾಗಿ
ದೇವ-ನಾಂಪ್ರಿ-ಯನು
ದೇವನು
ದೇವನೇ
ದೇವ-ನೊಬ್ಬ-ನಿ-ರು-ವ-ನೆಂದು
ದೇವಪ್ಪ
ದೇವಪ್ಪನ
ದೇವ-ಮಂದಿ-ರ-ವನ್ನು
ದೇವ-ಮಾ-ನ-ವ-ರಲ್ಲಿ
ದೇವ-ಮಾ-ನ-ವರು
ದೇವರ
ದೇವರತ್ತ
ದೇವರದ್ದು
ದೇವ-ರ-ನಾಮ
ದೇವರನ್ನು
ದೇವ-ರಲ್ಲದೆ
ದೇವ-ರಲ್ಲದೇ
ದೇವ-ರಲ್ಲವೇ
ದೇವರಲ್ಲಿ
ದೇವ-ರಲ್ಲಿನ
ದೇವ-ರಲ್ಲಿಲ್ಲ
ದೇವ-ರಾ-ಗಲಿ
ದೇವ-ರಾ-ಗಿದ್ದ
ದೇವ-ರಾ-ಗುತ್ತಾನೆ
ದೇವರಾಣೆ
ದೇವರಿಂದ
ದೇವ-ರಿ-ಗಾಗಿ
ದೇವ-ರಿ-ಗಿಂತ
ದೇವರಿಗೆ
ದೇವರಿಗೇ
ದೇವ-ರಿದ್ದಾನೆ
ದೇವ-ರಿದ್ದಾ-ನೆಯೆ
ದೇವ-ರಿ-ರು-ವನೇ
ದೇವರು
ದೇವ-ರು-ಗ-ಳಲ್ಲಿ
ದೇವ-ರು-ಗ-ಳಲ್ಲಿನ
ದೇವರೂ
ದೇವರೆ
ದೇವ-ರೆಂದರು
ದೇವ-ರೆಂದರೆ
ದೇವ-ರೆ-ಡೆ-ಗಿನ
ದೇವ-ರೆ-ಡೆಗೆ
ದೇವರೇ
ದೇವ-ರೊ-ಡನೆ
ದೇವ-ರೊಡ್ಡುವ
ದೇವಶಿಲ್ಪ
ದೇವಸ್ಥಾನ
ದೇವಸ್ಥಾ-ನದ
ದೇವಸ್ಥಾ-ನ-ದಲ್ಲಿ
ದೇವಾ
ದೇವಾತ್ಮ-ನಾ-ಗುತ್ತಾನೆ
ದೇವಾ-ನಾಂಪ್ರಿಯ
ದೇವಾ-ನಾಂಪ್ರಿ-ಯನ
ದೇವಾ-ಲ-ಯಕ್ಕೂ
ದೇವಾ-ಲ-ಯ-ಗ-ಳಲ್ಲಿ
ದೇವಾ-ಲ-ಯ-ಗ-ಳಿಗೆ
ದೇವಾ-ಲ-ಯದ
ದೇವಾ-ಲ-ಯ-ದಲ್ಲಿ
ದೇವಾ-ಲ-ಯ-ವನ್ನು
ದೇವಿ
ದೇವೀ
ದೇಶ
ದೇಶಕಾಲ
ದೇಶ-ಕಾ-ಲಕ್ಕೆ
ದೇಶ-ಕಾ-ಲ-ಗಳ
ದೇಶ-ಕಾ-ಲ-ಗ-ಳಿಂದ
ದೇಶ-ಕಾ-ಲದ
ದೇಶ-ಕಾ-ಲಾ-ತೀ-ತ-ನಾಗಿ
ದೇಶಕ್ಕಾಗಿ
ದೇಶಕ್ಕೆ
ದೇಶಕ್ಕೆಂಥ
ದೇಶಕ್ಕೊಂದು
ದೇಶಗಳ
ದೇಶ-ಗ-ಳನ್ನು
ದೇಶ-ಗ-ಳಲ್ಲಿ
ದೇಶ-ಗ-ಳಲ್ಲಿದ್ದು
ದೇಶ-ಗ-ಳಲ್ಲಿನ
ದೇಶ-ಗ-ಳಲ್ಲೂ
ದೇಶ-ಗ-ಳಿ-ಗಿಂತ
ದೇಶ-ಗ-ಳಿಗೂ
ದೇಶ-ಗ-ಳಿಗೆ
ದೇಶದ
ದೇಶದಲ್ಲಿ
ದೇಶ-ದಲ್ಲಿನ
ದೇಶದಲ್ಲೇ
ದೇಶ-ದ-ವರು
ದೇಶದ್ರೋ-ಹ-ವಾ-ಗುತ್ತ-ದೆಂದು
ದೇಶದ್ರೋ-ಹಿ-ಗಳು
ದೇಶಪ್ರೇಮ
ದೇಶಪ್ರೇ-ಮ-ಇ-ವು-ಗ-ಳನ್ನು
ದೇಶಪ್ರೇ-ಮದ
ದೇಶಪ್ರೇಮಿ
ದೇಶಪ್ರೇ-ಮಿ-ಗಳ
ದೇಶಪ್ರೇ-ಮಿ-ಗಳು
ದೇಶಪ್ರೇ-ಮಿ-ಗಳೂ
ದೇಶಭಕ್ತ
ದೇಶ-ಭಕ್ತ-ರಾ-ಗು-ವು-ದಕ್ಕೆ
ದೇಶವನ್ನು
ದೇಶವಾದ
ದೇಶ-ವಾ-ಸಿ-ಗ-ಳಲ್ಲೂ
ದೇಶವು
ದೇಶವೂ
ದೇಶವೆ
ದೇಶ-ವೊಂದನ್ನು
ದೇಶೀಯ
ದೇಹ
ದೇಹಇವು
ದೇಹಕ್ಕಿಂತ
ದೇಹಕ್ಕೂ
ದೇಹಕ್ಕೆ
ದೇಹಕ್ಕೇ
ದೇಹ-ಗ-ಳಲ್ಲಿ
ದೇಹ-ಗ-ಳಲ್ಲಿದ್ದ-ನೆಂದಾ-ದರೆ
ದೇಹ-ಗ-ಳಿಂದ
ದೇಹ-ಗ-ಳಿ-ವೆಯೇ
ದೇಹತ್ಯಾಗ
ದೇಹತ್ಯಾ-ಗದ
ದೇಹತ್ಯಾ-ಗ-ವನ್ನೂ
ದೇಹದ
ದೇಹದಲ್ಲಿ
ದೇಹ-ದಲ್ಲಿದ್ದೇವೆ
ದೇಹದಿಂದ
ದೇಹ-ದೊಂದಿಗೆ
ದೇಹ-ದೊಂದಿಗೇ
ದೇಹ-ದೊ-ಳ-ಗಿನ
ದೇಹ-ಧಾ-ರಿ-ಗ-ಳಾದ
ದೇಹ-ಧಾ-ರಿ-ಯಾಗಿ
ದೇಹ-ಧಾ-ರಿ-ಯಾದ
ದೇಹ-ಧಾ-ರಿ-ಯಾ-ದಾಗ
ದೇಹ-ಪಂಜ-ರ-ದಿಂದ
ದೇಹ-ಪ-ರಿ-ಶುದ್ಧ-ತೆ-ಯಲ್ಲಿ
ದೇಹ-ಭಾ-ರ-ದಿಂದ
ದೇಹವನ್ನು
ದೇಹವನ್ನೂ
ದೇಹ-ವನ್ನೆಲ್ಲ
ದೇಹವನ್ನೇ
ದೇಹವಲ್ಲ
ದೇಹ-ವಲ್ಲದೇ
ದೇಹ-ವಿ-ದೆ-ಎನ್ನುವ
ದೇಹ-ವಿಲ್ಲ-ದಿದ್ದರೂ
ದೇಹವು
ದೇಹ-ವೆನ್ನುವ
ದೇಹ-ವೆನ್ನು-ವುದು
ದೇಹವೇ
ದೇಹಸ್ಥಿತಿ
ದೇಹಾ
ದೇಹಾತೀತ
ದೇಹಾ-ತೀ-ತ-ವಾ-ದುದು
ದೇಹಾಯಾಸ
ದೇಹಾರೋಗ್ಯ
ದೇಹಾ-ಲಸ್ಯದ
ದೇಹಿ
ದೇಹೇಂದ್ರಿಯ
ದೇಹೇಂದ್ರಿ-ಯ-ಗ-ಳಿಂದ
ದೇಹೇಂದ್ರಿ-ಯ-ಗಳು
ದೈತ್ಯ-ಶಕ್ತಿಯ
ದೈನಂದಿನ
ದೈನಿಕ
ದೈನ್ಯ
ದೈನ್ಯದಿಂದ
ದೈನ್ಯಪ್ರ-ದರ್ಶನ
ದೈನ್ಯಾ-ವಸ್ಥೆ-ಗಳ
ದೈವ
ದೈವ-ಧಾರ್ಮಿಕ
ದೈವಕೃಪೆ
ದೈವ-ಕೃ-ಪೆಯ
ದೈವಕ್ಕೆ
ದೈವತ್ವ
ದೈವತ್ವದ
ದೈವತ್ವ-ವ-ನೇ-ರಲು
ದೈವತ್ವ-ವನ್ನು
ದೈವತ್ವ-ವಿದೆ
ದೈವ-ದತ್ತ-ವಾದ
ದೈವ-ಭಕ್ತನ
ದೈವ-ಭಕ್ತ-ರಿಗೂ
ದೈವಭಕ್ತಿ
ದೈವ-ಭಕ್ತಿ-ನಮ್ಮ
ದೈವ-ಭಕ್ತಿ-ಯನ್ನ-ವ-ಲಂಬಿ-ಸಿಯೇ
ದೈವ-ಭೀ-ತಿ-ಯುಳ್ಳ-ವ-ರಾಗಿ
ದೈವವನ್ನು
ದೈವವೇ
ದೈವವ್ಯ-ವಸ್ಥೆ
ದೈವಿಕ
ದೈವೀ
ದೈವೀಕೃಪೆ
ದೈವೀ-ಕೃ-ಪೆಯ
ದೈವೀ-ಕೃ-ಪೆ-ಯನ್ನು
ದೈವೀ-ಕೃ-ಪೆ-ಯಿಂದ
ದೈವೀಗುಣ
ದೈವೀ-ಗು-ಣ-ಗ-ಳಿಂದ
ದೈವೀ-ಚೇ-ತನ
ದೈವೀ-ಪಿ-ಪಾ-ಸೆ-ಗಳು
ದೈವೀಪ್ರಜ್ಞೆ
ದೈವೀಪ್ರೀತಿ
ದೈವೀಪ್ರೀ-ತಿಯ
ದೈವೀಪ್ರೀ-ತಿ-ಯನ್ನು
ದೈವೀಪ್ರೀ-ತಿ-ಯನ್ನೋ
ದೈವೀಪ್ರೇ-ಮ-ವನ್ನು
ದೈವೀ-ವಾ-ಣಿ-ಯಲ್ಲಿ
ದೈವೀಶಕ್ತಿ
ದೈವೀ-ಶಕ್ತಿ-ಯನ್ನು
ದೈವೀ-ಶಕ್ತಿ-ಯಿಂದ
ದೈವೀಸೃಷ್ಟಿ
ದೈವೀಸ್ವ-ಭಾ-ವ-ವನ್ನು
ದೈವೇಚ್ಛೆಗೆ
ದೈಹಿಕ
ದೈಹಿ-ಕ-ವಾಗಿ
ದೈಹಿ-ಕ-ವಾ-ಗಿಯೂ
ದೈಹಿ-ಕ-ಶಕ್ತಿ
ದೊಂದಿಗೆ
ದೊಡ್ಡ
ದೊಡ್ಡತನ
ದೊಡ್ಡ-ತ-ನ-ದಿಂದ
ದೊಡ್ಡದಾಗಿ
ದೊಡ್ಡ-ದಾ-ಗು-ವು-ದಿಲ್ಲ
ದೊಡ್ಡದಾದ
ದೊಡ್ಡದು
ದೊಡ್ಡದೊಂದು
ದೊಡ್ಡಮನೆ
ದೊಡ್ಡ-ವ-ರಾ-ದರೆ
ದೊಡ್ಡ-ವ-ನನ್ನಾ-ಗಿ-ಸ-ಬ-ಹುದು
ದೊಡ್ಡ-ವ-ನಾದ
ದೊಡ್ಡ-ವ-ರನ್ನಾಗಿ
ದೊಡ್ಡ-ವ-ರಾಗಿ
ದೊಡ್ಡ-ವ-ರಾ-ಗಿ-ರ-ಲಿಲ್ಲ
ದೊಡ್ಡವರು
ದೊಡ್ಡವಳು
ದೊಡ್ಡ-ಸಂಗತಿ
ದೊಡ್ಡ-ಸ-ಮಸ್ಯೆ-ಗ-ಳಾಗಿ
ದೊರ-ಕ-ಬ-ಹು-ದಾದ
ದೊರ-ಕ-ಬೇ-ಕಲ್ಲವೆ
ದೊರ-ಕ-ತೊ-ಡ-ಗಿ-ದವು
ದೊರಕದು
ದೊರ-ಕ-ಬ-ಹು-ದಲ್ಲವೇ
ದೊರ-ಕ-ಬ-ಹುದೋ
ದೊರಕಿತು
ದೊರಕಿತ್ತು
ದೊರ-ಕಿ-ದರೆ
ದೊರ-ಕಿ-ದಾಗ
ದೊರ-ಕಿ-ದಾ-ಗ-ಲೆಲ್ಲ
ದೊರ-ಕಿ-ರು-ವು-ದನ್ನು
ದೊರ-ಕುತ್ತದೆ
ದೊರ-ಕುತ್ತವೆ
ದೊರಕುವ
ದೊರ-ಕು-ವಂತೆ
ದೊರ-ಕು-ವ-ವ-ರೆಗೂ
ದೊರಕುವು
ದೊರ-ಕು-ವು-ದ-ರಲ್ಲಿ
ದೊರ-ಕು-ವು-ದಲ್ಲದೇ
ದೊರ-ಕು-ವುದು
ದೊರ-ಕು-ವುದೆ
ದೊರ-ತೀ-ತೆಂದು
ದೊರತೀತೇ
ದೊರತೇ
ದೊರೆ
ದೊರೆಗಳ
ದೊರೆಗಳು
ದೊರೆಗೆ
ದೊರೆತ
ದೊರೆ-ತಂತಾ-ಗು-ವುದು
ದೊರೆತರೆ
ದೊರೆತವು
ದೊರೆತಾಗ
ದೊರೆತಾನೇ
ದೊರೆತಿತ್ತು
ದೊರೆತಿದೆ
ದೊರೆ-ತಿ-ರ-ಲಿಲ್ಲ
ದೊರೆತು
ದೊರೆ-ತು-ದನ್ನು
ದೊರೆ-ತು-ದ-ರಿಂದ
ದೊರೆತೇ
ದೊರೆ-ತೊ-ಡನೆ
ದೊರೆಯದ
ದೊರೆ-ಯ-ದಂತೆ
ದೊರೆ-ಯ-ದಾ-ಗಲೂ
ದೊರೆ-ಯ-ದಿದ್ದ
ದೊರೆಯದು
ದೊರೆ-ಯ-ಬ-ಹು-ದಾದ
ದೊರೆ-ಯ-ಬೇ-ಕಾ-ದರೆ
ದೊರೆ-ಯ-ಬೇಕು
ದೊರೆ-ಯ-ಲಿಲ್ಲ
ದೊರೆ-ಯ-ಲಿಲ್ಲ-ವೆಂದು
ದೊರೆಯಲೇ
ದೊರೆಯಿತು
ದೊರೆ-ಯುತ್ತದೆ
ದೊರೆ-ಯುತ್ತ-ದೆನ್ನುವ
ದೊರೆ-ಯುತ್ತ-ದೆ-ಹೀಗೆ
ದೊರೆ-ಯುತ್ತವೆ
ದೊರೆ-ಯುತ್ತ-ವೆಯೇ
ದೊರೆ-ಯುತ್ತಿದೆ
ದೊರೆ-ಯುತ್ತಿದ್ದ
ದೊರೆಯುವ
ದೊರೆ-ಯು-ವಂತೆ
ದೊರೆ-ಯು-ವು-ದಿಲ್ಲ
ದೊರೆ-ಯು-ವುದು
ದೊರೆ-ಯು-ವುದೇ
ದೋಚಿ
ದೋಣಿ-ಇ-ವು-ಗ-ಳಂತೆ
ದೋಣಿಯಂತೆ
ದೋಣಿಯನ್ನು
ದೋಷ
ದೋಷ-ಗ-ಳನ್ನು
ದೋಷ-ಗ-ಳನ್ನೂ
ದೋಷ-ಗ-ಳನ್ನೋ
ದೋಷ-ಗ-ಳಿಂದ
ದೋಷ-ಗ-ಳಿಂದಲೇ
ದೋಷ-ಗ-ಳಿದ್ದರೂ
ದೋಷಗಳು
ದೋಷ-ದೌರ್ಬಲ್ಯ-ಗ-ಳನ್ನು
ದೋಷ-ಪೂ-ರಿತ
ದೋಷ-ಮುಕ್ತ-ರನ್ನಾಗಿ
ದೋಷಯುಕ್ತ
ದೋಷ-ಯುಕ್ತವೂ
ದೋಷ-ರ-ಹಿತ
ದೋಷ-ರ-ಹಿ-ತ-ರಲ್ಲ
ದೋಷವನ್ನು
ದೋಷವನ್ನೂ
ದೋಷವಲ್ಲ
ದೋಷವಿದೆ
ದೋಷವೂ
ದೋಷವೇ
ದೋಷವೋ
ದೋಷಾ
ದೋಷಾ-ರೋ-ಪಣೆ
ದೋಷಾ-ರೋ-ಪಣೆ
ದೋಷಾ-ರೋ-ಪ-ಣೆ-ಗ-ಳಿಂದ
ದೌರಾತ್ಮ್ಯ-ಗ-ಳಿಗೆ
ದೌರ್ಜನ್ಯ-ಗಳ
ದೌರ್ಜನ್ಯದ
ದೌರ್ಜನ್ಯ-ವನ್ನು
ದೌರ್ಬಲ್ಯ
ದೌರ್ಬಲ್ಯ-ಅತಿ
ದೌರ್ಬಲ್ಯಕ್ಕೆ
ದೌರ್ಬಲ್ಯ-ಗಳ
ದೌರ್ಬಲ್ಯ-ಗ-ಳನ್ನು
ದೌರ್ಬಲ್ಯ-ಗ-ಳಿಂದ
ದೌರ್ಬಲ್ಯ-ಗ-ಳಿಗೂ
ದೌರ್ಬಲ್ಯ-ಗಳು
ದೌರ್ಬಲ್ಯದ
ದೌರ್ಬಲ್ಯ-ದಿಂದ
ದೌರ್ಬಲ್ಯ-ದಿಂದಾಗಿ
ದೌರ್ಬಲ್ಯ-ವನ್ನು
ದೌರ್ಬಲ್ಯ-ವಾಗಿ
ದೌರ್ಬಲ್ಯ-ವೆಂದು
ದೌರ್ಬಲ್ಯವೇ
ದೌರ್ಭಾಗ್ಯ
ದೌರ್ಭಾಗ್ಯಕ್ಕೆ
ದೌರ್ಭಾಗ್ಯ-ವನ್ನುಂಟು-ಮಾ-ಡಿದೆ
ದೌರ್ಭಾಗ್ಯ-ವಾಗಿ
ದೌರ್ಭಾಗ್ಯ-ಶಾ-ಲಿ-ಯನ್ನು
ದೌರ್ಮ-ನಸ್ಯ-ಗ-ಳನ್ನು
ದ್ದರೆ
ದ್ದಳೂ
ದ್ದಾಗ
ದ್ದೀಯೆ
ದ್ರವ
ದ್ರವಾಂಶ
ದ್ರವ್ಯ
ದ್ರವ್ಯ-ಗ-ಳನ್ನು
ದ್ರವ್ಯಗಳು
ದ್ರವ್ಯ-ಗ-ಳು-ಇ-ವು-ಗಳ
ದ್ರವ್ಯ-ಗು-ಣ-ವನ್ನು
ದ್ರವ್ಯ-ಗು-ಣ-ವಿಲ್ಲದ
ದ್ರವ್ಯದಿಂದ
ದ್ರವ್ಯರಾಶಿ
ದ್ರವ್ಯ-ರಾ-ಶಿ-ಯಲ್ಲಿ
ದ್ರವ್ಯ-ರಾ-ಶಿ-ಯಲ್ಲಿ-ರುವ
ದ್ರಷ್ಟಾ
ದ್ರಷ್ಟಾರ
ದ್ರಷ್ಟಾರನ
ದ್ರಷ್ಟಾ-ರ-ರನ್ನು
ದ್ರಷ್ಟಾರರು
ದ್ರಾಕ್ಷಿ
ದ್ರಾವ-ಣ-ವನ್ನು
ದ್ರಿಯ
ದ್ರುತ-ಗ-ತಿಯ
ದ್ರುತ-ಗ-ತಿ-ಯಿಂದ
ದ್ರೋಹ
ದ್ರೋಹಗಳ
ದ್ರೌಪದಿಗೆ
ದ್ರೌಪದಿಯ
ದ್ರೌಪ-ದಿ-ಯನ್ನು
ದ್ವಂದ್ವಗಳ
ದ್ವಂದ್ವ-ಗ-ಳನ್ನು
ದ್ವಂದ್ವ-ಗ-ಳಿಂದ
ದ್ವಿ
ದ್ವಿತೀ-ಯ-ರೆ-ನಿ-ಸದೆ
ದ್ವಿಷ್ನಿಂದ
ದ್ವೀಪ
ದ್ವೀಪದಲ್ಲಿ
ದ್ವೀಪವು
ದ್ವೇಷ
ದ್ವೇಷಇವು
ದ್ವೇಷ-ಇ-ವು-ಗ-ಳಿಂದ
ದ್ವೇಷ-ಕಾ-ರುತ್ತ
ದ್ವೇಷಕ್ಕಿಂತ
ದ್ವೇಷಕ್ಕೆ
ದ್ವೇಷ-ಗ-ಳನ್ನು
ದ್ವೇಷ-ಗ-ಳನ್ನೂ
ದ್ವೇಷ-ಗ-ಳಿಂದ
ದ್ವೇಷಗಳು
ದ್ವೇಷದ
ದ್ವೇಷದಿಂದ
ದ್ವೇಷ-ದೃಷ್ಟಿ-ಯನ್ನು
ದ್ವೇಷ-ದೌರ್ಬಲ್ಯ-ಗಳ
ದ್ವೇಷ-ಪೂ-ರಿತ
ದ್ವೇಷಪ್ಪ-ನ-ವರ
ದ್ವೇಷಭಾವ
ದ್ವೇಷ-ಭಾ-ವನೆ
ದ್ವೇಷ-ಭಾ-ವ-ನೆ-ಗಳು
ದ್ವೇಷ-ಭಾ-ವ-ನೆ-ಯನ್ನು
ದ್ವೇಷ-ಭಾ-ವ-ನೆಯೂ
ದ್ವೇಷ-ಭಾ-ವ-ವನ್ನೂ
ದ್ವೇಷ-ಮಾ-ಡು-ವ-ವ-ನನ್ನು
ದ್ವೇಷ-ಮಾ-ಡು-ವುದು
ದ್ವೇಷ-ಮೂ-ಲ-ವಾದ
ದ್ವೇಷವನ್ನು
ದ್ವೇಷವನ್ನೂ
ದ್ವೇಷವಿದೆ
ದ್ವೇಷ-ವಿ-ರು-ವುದೋ
ದ್ವೇಷವು
ದ್ವೇಷವೂ
ದ್ವೇಷವೆಂಬ
ದ್ವೇಷವೇ
ದ್ವೇಷ-ಸಿದ್ಧಾಂತ
ದ್ವೇಷಾಸೂಯೆ
ದ್ವೇಷಾ-ಸೂ-ಯೆ-ಗ-ಳನ್ನು
ದ್ವೇಷಾ-ಸೂ-ಯೆ-ಯನ್ನು
ದ್ವೇಷಾಸೂಯೆ
ದ್ವೇಷಾ-ಸೂ-ಯೆ-ಗ-ಳನ್ನು
ದ್ವೇಷಾ-ಸೂ-ಯೆ-ಗಳ
ದ್ವೇಷಾ-ಸೂ-ಯೆ-ಗ-ಳನ್ನು
ದ್ವೇಷಾ-ಸೂ-ಯೆಯ
ದ್ವೇಷಾ-ಸೂ-ಯೆ-ಯನ್ನು
ದ್ವೇಷಿಗಳ
ದ್ವೇಷಿ-ಸ-ತೊ-ಡ-ಗಿದ
ದ್ವೇಷಿ-ಸ-ತೊ-ಡ-ಗಿ-ದಾಗ
ದ್ವೇಷಿಸದೆ
ದ್ವೇಷಿಸದೇ
ದ್ವೇಷಿ-ಸ-ಬಲ್ಲನೆ
ದ್ವೇಷಿ-ಸ-ಬಲ್ಲರು
ದ್ವೇಷಿ-ಸ-ಲಾರ
ದ್ವೇಷಿ-ಸ-ಲಾ-ರಿರಿ
ದ್ವೇಷಿಸಲು
ದ್ವೇಷಿಸಿ
ದ್ವೇಷಿ-ಸಿ-ಕೊಳ್ಳುವ
ದ್ವೇಷಿ-ಸಿ-ಕೊಳ್ಳು-ವಂತಾ-ಗ-ಬೇಕೇ
ದ್ವೇಷಿ-ಸಿ-ದಂತೆಯೆ
ದ್ವೇಷಿ-ಸಿ-ದಂತೆಯೇ
ದ್ವೇಷಿ-ಸಿ-ದರೆ
ದ್ವೇಷಿ-ಸುತ್ತೀರಾ
ದ್ವೇಷಿ-ಸುತ್ತೀರೋ
ದ್ವೇಷಿ-ಸುತ್ತೇ-ವೆಯೆ
ದ್ವೇಷಿ-ಸುತ್ತೇ-ವೇನು
ದ್ವೇಷಿ-ಸು-ವಂತೆ
ದ್ವೇಷಿ-ಸು-ವ-ವರು
ದ್ವೇಷಿ-ಸು-ವುದು
ಧಕ್ಕೆ
ಧಗ-ಧ-ಗಿ-ಸಿತು
ಧಗ-ಧ-ಗಿ-ಸಿದೆ
ಧಡಾರ್
ಧನ
ಧನ-ಕ-ನ-ಕಾ-ದಿ-ಗಳು
ಧನ-ಕು-ಬೇ-ರನ
ಧನ-ದಾ-ಹಿ-ಗ-ಳಾದ
ಧನವಂತ
ಧನ-ಸಂಗ್ರಹ
ಧನಾರ್ಜ-ನೆ-ಗಾಗಿ
ಧನ್ಯತೆ
ಧನ್ಯತೆಯ
ಧನ್ಯ-ರಾ-ಗು-ವಿರಿ
ಧನ್ಯ-ರಾ-ದೆ-ವೆಂದು
ಧನ್ಯರು
ಧನ್ಯವಾದ
ಧನ್ಯ-ವಾ-ದ-ಗ-ಳನ್ನು
ಧರಿ-ಸ-ಬೇಕು
ಧರಿಸಲು
ಧರಿಸಿ
ಧರಿಸಿದ
ಧರಿ-ಸಿ-ದಂತೆ
ಧರಿ-ಸಿ-ದಾಗ
ಧರಿ-ಸಿದ್ದಕ್ಕಾಗಿ
ಧರಿ-ಸುತ್ತಿದ್ದಿರಿ
ಧರಿಸುವ
ಧರ್ಮ
ಧರ್ಮಮೂರ್ತಿ
ಧರ್ಮ-ಇ-ವು-ಗ-ಳೆಲ್ಲ
ಧರ್ಮ-ಕಾ-ಯ-ವೆಂದೂ
ಧರ್ಮ-ಕಾರ್ಯ-ಗ-ಳಲ್ಲೂ
ಧರ್ಮಕ್ಕೆ
ಧರ್ಮಕ್ಕೋ
ಧರ್ಮಗಳ
ಧರ್ಮ-ಗ-ಳಲ್ಲಿ
ಧರ್ಮ-ಗ-ಳಲ್ಲಿ-ಕರ್ಮ
ಧರ್ಮ-ಗ-ಳಲ್ಲೂ
ಧರ್ಮ-ಗ-ಳಿಗೆ
ಧರ್ಮ-ಗ-ಳಿವೆ
ಧರ್ಮಗಳು
ಧರ್ಮಗಳೂ
ಧರ್ಮ-ಗ-ಳೆಲ್ಲವೂ
ಧರ್ಮಗಳೇ
ಧರ್ಮ-ಗ-ಳೊ-ಳಗೆ
ಧರ್ಮಗುರು
ಧರ್ಮ-ಗು-ರು-ಗ-ಳೊಬ್ಬರು
ಧರ್ಮ-ಗು-ರು-ಗಳು
ಧರ್ಮಗ್ರಂಥ-ಗ-ಳಲ್ಲಿ
ಧರ್ಮಗ್ರಂಥ-ಗ-ಳಲ್ಲೂ
ಧರ್ಮಗ್ರಂಥ-ಗ-ಳಾ-ಗಲಿ
ಧರ್ಮಗ್ರಂಥ-ಗಳು
ಧರ್ಮಗ್ರಂಥ-ಗಳೂ
ಧರ್ಮಗ್ಲಾ-ನಿಗೆ
ಧರ್ಮಗ್ಲಾ-ನಿಯ
ಧರ್ಮ-ಛತ್ರ-ವನ್ನಾಗಿ
ಧರ್ಮತತ್ವ
ಧರ್ಮದ
ಧರ್ಮ-ದಲ್ಲ-ಡ-ಗಿ-ರುವ
ಧರ್ಮದಲ್ಲಿ
ಧರ್ಮ-ದ-ವರು
ಧರ್ಮ-ದೇ-ವರು
ಧರ್ಮ-ನಾ-ಶಕ
ಧರ್ಮನಿಂದೆ
ಧರ್ಮ-ನಿ-ಯ-ಮ-ಗ-ಳನ್ನು
ಧರ್ಮ-ನಿ-ರ-ಪೇಕ್ಷ
ಧರ್ಮ-ನಿ-ರ-ಪೇಕ್ಷತೆ
ಧರ್ಮ-ನಿಷ್ಠೆಯ
ಧರ್ಮಪ್ರ-ಚಾ-ರ-ಕಾರ್ಯ
ಧರ್ಮ-ಬೋ-ಧ-ಕರೂ
ಧರ್ಮ-ಭೀ-ರು-ಗ-ಳಾದ
ಧರ್ಮಭೂಮಿ
ಧರ್ಮ-ಮೂ-ಲದ
ಧರ್ಮ-ಮೂ-ಲ-ದಲ್ಲಿ
ಧರ್ಮ-ರಾ-ಜನು
ಧರ್ಮಲಾಭ
ಧರ್ಮವನ್ನು
ಧರ್ಮವನ್ನೂ
ಧರ್ಮ-ವಾ-ಗದು
ಧರ್ಮ-ವಿ-ರೋಧ
ಧರ್ಮ-ವಿ-ರೋ-ಧ-ವಲ್ಲದ
ಧರ್ಮ-ವಿ-ರೋಧೀ
ಧರ್ಮ-ವಿಲ್ಲದ
ಧರ್ಮ-ವಿಲ್ಲದೆ
ಧರ್ಮವು
ಧರ್ಮವೂ
ಧರ್ಮವೃಕ್ಷ
ಧರ್ಮ-ವೃಕ್ಷವು
ಧರ್ಮವೇ
ಧರ್ಮವ್ಯಾಧ
ಧರ್ಮಶ್ರದ್ಧೆ
ಧರ್ಮಶ್ರದ್ಧೆಯ
ಧರ್ಮಶ್ರದ್ಧೆ-ಯನ್ನೇ
ಧರ್ಮ-ಸಂದೇ-ಶದ
ಧರ್ಮ-ಸಂಬಂಧಿ
ಧರ್ಮ-ಸಂರಕ್ಷಣ
ಧರ್ಮ-ಸಂಸ್ಕೃ-ತಿ-ಗಳ
ಧರ್ಮ-ಸಂಸ್ಥೆಗೋ
ಧರ್ಮ-ಸ-ಮನ್ವಯ
ಧರ್ಮ-ಸಾ-ಧನೆ
ಧರ್ಮಾತ್ಮ-ರು-ಮ-ಹಾತ್ಮರು
ಧರ್ಮಾ-ಚ-ರಣೆ
ಧರ್ಮಾ-ಚ-ರ-ಣೆಯ
ಧರ್ಮಾತ್ಮ-ನಾಗಿ
ಧರ್ಮಾ-ನು-ಯಾ-ಯಿ-ಗ-ಳಲ್ಲಿ
ಧರ್ಮಾ-ನು-ಯಾ-ಯಿ-ಗ-ಳಲ್ಲಿ
ಧರ್ಮಾ-ನು-ಯಾ-ಯಿ-ಗಳು
ಧರ್ಮಾ-ನುಷ್ಠಾ-ನ-ದಲ್ಲಿ
ಧರ್ಮೀಯರು
ಧರ್ಮೋ-ಪ-ದೇ-ಶ-ಕ-ನಾ-ಗ-ಬೇ-ಕೆಂಬ
ಧರ್ಮೋ-ಪ-ದೇ-ಶ-ಕ-ನಾ-ಗಿದ್ದ
ಧರ್ಮೋ-ಪ-ದೇ-ಶ-ಕ-ನಾ-ಗುವ
ಧಾಟಿ
ಧಾಟಿಯಿಂದ
ಧಾನ್ಯದ
ಧಾರಗಳ
ಧಾರ-ಣೆ-ಗಳ
ಧಾರಾ-ಕಾ-ರ-ವಾಗಿ
ಧಾರಾ-ಪಾತ್ರೆ-ಯನ್ನು
ಧಾರಾ-ಳ-ವಾಗಿ
ಧಾರಾ-ಳಿ-ಗಳು
ಧಾರೆ-ಧಾ-ರೆ-ಯಾಗಿ
ಧಾರೆಯಾಗಿ
ಧಾರೆಯೆರೆ
ಧಾರ್ಮಿಕ
ಧಾರ್ಮಿ-ಕ-ನಾ-ಗಿದ್ದೂ
ಧಾರ್ಮಿ-ಕ-ನಾ-ಗದೆ
ಧಾರ್ಮಿ-ಕ-ನಾ-ದಾನೆ
ಧಾರ್ಮಿ-ಕ-ನೆ-ನಿ-ಸ-ಲಾರ
ಧಾರ್ಮಿಕರ
ಧಾರ್ಮಿ-ಕ-ರನ್ನು
ಧಾರ್ಮಿ-ಕ-ರಲ್ಲಿ
ಧಾರ್ಮಿ-ಕ-ರಾ-ದರೆ
ಧಾರ್ಮಿ-ಕ-ರಿಗೂ
ಧಾರ್ಮಿ-ಕ-ರಿಗೆ
ಧಾರ್ಮಿಕರು
ಧಾರ್ಮಿಕರೂ
ಧಾರ್ಮಿ-ಕ-ರೆ-ನಿ-ಸಿ-ಕೊಂಡ-ವ-ರಿಂದಲೇ
ಧಾರ್ಮಿ-ಕ-ರೆನ್ನಿ-ಸಿ-ಕೊಂಡ-ವರು
ಧಾರ್ಮಿ-ಕ-ರೆಲ್ಲರೂ
ಧಾರ್ಮಿ-ಕ-ಳಾ-ಗಿದ್ದೇನೆ
ಧಾರ್ಮಿ-ಕ-ವಾಗಿ
ಧಾರ್ಷ್ಟ್ಯ-ಇ-ವು-ಗಳ
ಧಾಳಿ-ಗೊ-ಳ-ಗಾ-ಗಿವೆ
ಧಾವಿ-ಸ-ದಿ-ರ-ಬೇ-ಕಾ-ದರೆ
ಧಾವಿ-ಸ-ಬಲ್ಲಿರಾ
ಧಾವಿಸಲು
ಧಾವಿ-ಸಿ-ಬಂದರು
ಧಾವಿ-ಸಿ-ದರು
ಧಾವಿ-ಸುತ್ತಿದೆ
ಧಾವಿ-ಸುತ್ತಿ-ರುವ
ಧಾವಿ-ಸುತ್ತಿ-ರು-ವನೆ
ಧಾವಿ-ಸು-ವು-ದರ
ಧಿಸಿ
ಧೀಮಂತ
ಧೀಮಂತರು
ಧೀಮಂತರೂ
ಧೀಮಂತಿ-ಕೆಯ
ಧೀರ
ಧೀರನಾದ
ಧೀರ-ನಾ-ಯಕ
ಧೀರರ
ಧೀರರು
ಧೀರರೂ
ಧೀಶಕ್ತಿ
ಧುನಿ
ಧುನಿಯ
ಧುರೀಣರು
ಧುರೀಣರೂ
ಧೂಮಕೇತು
ಧೂಮಪಾನ
ಧೂಳನ್ನು
ಧೂಳಿನ
ಧೃತಿಗೆಟ್ಟು
ಧೃತಿ-ಗೆ-ಡದೆ
ಧೃತಿ-ಗೆ-ಡದೇ
ಧೃತಿ-ಗೆ-ಡ-ಬಾ-ರದು
ಧೃತಿ-ಗೆ-ಡ-ಲಿಲ್ಲ
ಧೈರ್ಯ
ಧೈರ್ಯ-ಇ-ವು-ಗಳ
ಧೈರ್ಯಗಳ
ಧೈರ್ಯ-ಗೆ-ಡ-ದಿ-ರುವ
ಧೈರ್ಯದ
ಧೈರ್ಯದಿಂದ
ಧೈರ್ಯ-ದಿಂದಲೇ
ಧೈರ್ಯಬಂತು
ಧೈರ್ಯ-ಬ-ರು-ವುದು
ಧೈರ್ಯ-ಮಾ-ಡಿದ
ಧೈರ್ಯ-ಮಾ-ಡಿದ್ದುಂಟೇ
ಧೈರ್ಯವನ್ನು
ಧೈರ್ಯ-ವಾ-ಗಿಯೇ
ಧೈರ್ಯ-ವಾ-ಗಿ-ರು-ವುದೇ
ಧೈರ್ಯ-ವಿ-ರ-ಲಿಲ್ಲ
ಧೈರ್ಯ-ವೆಂಥದು
ಧೈರ್ಯ-ಶಾ-ಲಿ-ಗಳು
ಧೈರ್ಯಸ್ಥರು
ಧೈರ್ಯಸ್ಥೈರ್ಯ-ದಿಂದ
ಧೈರ್ಯೋತ್ಸಾ-ಹ-ಗಳು
ಧೋಬಿ
ಧೋಬಿಗೆ
ಧೋರಣೆ
ಧೋರ-ಣೆ-ಯನ್ನಲ್ಲ
ಧೋರ-ಣೆ-ಯನ್ನು
ಧ್ಯಾನ
ಧ್ಯಾನ-ಕಾ-ಲ-ದಲ್ಲಿ
ಧ್ಯಾನಕ್ಕೆ
ಧ್ಯಾನ-ಗ-ಳಲ್ಲಿ
ಧ್ಯಾನದ
ಧ್ಯಾನದಲ್ಲಿ
ಧ್ಯಾನ-ದಿಂದೆದ್ದು
ಧ್ಯಾನ-ನಿ-ರ-ತ-ರಾದ
ಧ್ಯಾನ-ಬ-ಲ-ದಿಂದ
ಧ್ಯಾನ-ಮಗ್ನ-ನಾದೆ
ಧ್ಯಾನ-ಮಗ್ನ-ರಾ-ದರು
ಧ್ಯಾನಮಾಡಿ
ಧ್ಯಾನವೇ
ಧ್ಯಾನ-ಶೀ-ಲ-ರಾ-ಗುತ್ತಾರೆ
ಧ್ಯಾನ-ಸಿದ್ಧರ
ಧ್ಯಾನಸ್ಥ-ರಾಗಿ
ಧ್ಯಾನಾಭ್ಯಾಸ
ಧ್ಯಾನಿ
ಧ್ಯಾನಿಸಲು
ಧ್ಯಾನಿ-ಸುತ್ತಾರೆ
ಧ್ಯಾನಿ-ಸು-ವು-ದಾ-ಗಲಿ
ಧ್ಯೇಯ
ಧ್ಯೇಯದ
ಧ್ಯೇಯದತ್ತ
ಧ್ಯೇಯನಿಷ್ಠೆ
ಧ್ಯೇಯವಾದ
ಧ್ಯೇಯವಾದಿ
ಧ್ಯೇಯೋದ್ದೇಶ
ಧ್ಯೇಯೋದ್ದೇ-ಶ-ಗ-ಳನ್ನು
ಧ್ಯೇಯೋದ್ದೇ-ಶ-ಗಳು
ಧ್ರುವ
ಧ್ರುವ-ಸತ್ಯ-ವೆಂದು
ಧ್ವಂಸ
ಧ್ವಂಸ-ವಾ-ಯಿತು
ಧ್ವಜದ
ಧ್ವಜಸ್ತಂಭ-ವನ್ನು
ಧ್ವನಿ
ಧ್ವನಿತ
ಧ್ವನಿ-ತಂತು-ಗಳ
ಧ್ವನಿ-ತ-ವಾ-ಗಿದೆ
ಧ್ವನಿ-ಮೊ-ಳ-ಗಿತು
ಧ್ವನಿಯನ್ನು
ಧ್ವನಿಯಲ್ಲಿ
ಧ್ವನಿಯಿಂದ
ಧ್ವನಿಯೇ
ಧ್ವನಿ-ಸಿದ್ದಾರೆ
ಧ್ವನಿ-ಸುತ್ತದೆ
ಧ್ವನಿ-ಸುತ್ತ-ವಲ್ಲವೆ
ಧ್ವನಿ-ಸುತ್ತಾರೆ
ಧ್ವನಿಸುವ
ಧ್ವನಿ-ಸು-ವ-ವರು
ಧ್ವನಿ-ಸು-ವುದು
ನ
ನಂಟು
ನಂತರ
ನಂತರದ
ನಂತರವೂ
ನಂತರವೋ
ನಂತಿ-ಹು-ದು-ಎಂದು
ನಂತೆ
ನಂದದ
ನಂದದಂತೆ
ನಂದ-ನ-ವ-ನ-ವನ್ನಾ-ಗಿ-ಸಿದ್ದಾರೆ
ನಂದ-ನ-ವ-ನ-ವಾ-ಗುತ್ತಿತ್ತು
ನಂದ-ನ-ವ-ನ-ವಾ-ಗು-ವುದು
ನಂದಿ
ನಂದಿ-ಸಿ-ರುವ
ನಂಬ-ಬ-ಹು-ದಾದ
ನಂಬ-ದಾ-ದರು
ನಂಬದೆಯೇ
ನಂಬ-ಬ-ಹುದು
ನಂಬ-ಬೇ-ಕಿದ್ದರೆ
ನಂಬಬೇಕು
ನಂಬರಿನ
ನಂಬ-ಲರ್ಹ-ವಾದ
ನಂಬ-ಲ-ಸಾಧ್ಯ-ವಾದ
ನಂಬಲಾರ
ನಂಬ-ಲಾ-ರಳು
ನಂಬಲಿಲ್ಲ
ನಂಬಲು
ನಂಬಿ
ನಂಬಿಕೆ
ನಂಬಿ-ಕೆ-ಯನ್ನಿಟ್ಟು
ನಂಬಿ-ಕೆ-ಯಿಂದಲೇ
ನಂಬಿ-ಕೆ-ಯುಂಟಾದ
ನಂಬಿ-ಕೆ-ಗ-ಳನ್ನು
ನಂಬಿ-ಕೆ-ಗ-ಳಿಂದ
ನಂಬಿ-ಕೆ-ಗಳು
ನಂಬಿಕೆಗೆ
ನಂಬಿಕೆಯ
ನಂಬಿ-ಕೆ-ಯನ್ನಿಟ್ಟಲ್ಲ
ನಂಬಿ-ಕೆ-ಯನ್ನಿಟ್ಟಿದ್ದ
ನಂಬಿ-ಕೆ-ಯನ್ನಿಡಿ
ನಂಬಿ-ಕೆ-ಯನ್ನು
ನಂಬಿ-ಕೆ-ಯನ್ನುಂಟು
ನಂಬಿ-ಕೆ-ಯನ್ನು-ಮುಖ್ಯ-ವಾಗಿ
ನಂಬಿ-ಕೆ-ಯನ್ನೂ
ನಂಬಿ-ಕೆ-ಯನ್ನೂ-ಗ-ಳಿ-ಸದು
ನಂಬಿ-ಕೆ-ಯನ್ನೇ
ನಂಬಿ-ಕೆ-ಯಲ್ಲ
ನಂಬಿ-ಕೆ-ಯಲ್ಲಿ
ನಂಬಿ-ಕೆ-ಯಿಂದ
ನಂಬಿ-ಕೆ-ಯಿಂದಲೇ
ನಂಬಿ-ಕೆ-ಯಿಂದೇಕೆ
ನಂಬಿಕೆಯು
ನಂಬಿಕೆಯೂ
ನಂಬಿಕೆಯೆ
ನಂಬಿಕೆಯೇ
ನಂಬಿ-ಕೊಂಡಿದ್ದರು
ನಂಬಿಕೊಂಡು
ನಂಬಿದ
ನಂಬಿ-ದಂತಿದೆ
ನಂಬಿದರೂ
ನಂಬಿ-ದ-ವರು
ನಂಬಿದುದೇ
ನಂಬಿದ್ದ
ನಂಬಿದ್ದರು
ನಂಬಿದ್ದಾನೆ
ನಂಬಿದ್ದಾರೆ
ನಂಬಿದ್ದಾ-ರೆ-ಹಿಂದಿನ
ನಂಬಿದ್ದೇವೆ
ನಂಬಿ-ಧರ್ಮವು
ನಂಬಿರಿ
ನಂಬುತ್ತ
ನಂಬುತ್ತಾರೆ
ನಂಬುತ್ತೇವೋ
ನಂಬುವ
ನಂಬು-ವಂತಾ-ಗಿತ್ತು
ನಂಬು-ವ-ವರು
ನಂಬುವಿರಾ
ನಂಬು-ವು-ದಲ್ಲ
ನಂಬು-ವು-ದಾ-ದರೆ
ನಂಬು-ವು-ದಿಲ್ಲ
ನಂಬುವುದು
ನಕಾ-ಶೆ-ಗ-ಳಿ-ಗ-ನು-ಗು-ಣ-ವಾ-ಗಿಯೇ
ನಕ್ಕನು
ನಕ್ಕರೆ
ನಕ್ಕಳು
ನಕ್ಕಿತು
ನಕ್ಕು
ನಕ್ಕು-ನ-ಗಿ-ಸುತ್ತಾಳೆ
ನಕ್ಕು-ನ-ಲಿ-ಯು-ವು-ದನ್ನು
ನಕ್ಷತ್ರ
ನಕ್ಷತ್ರ-ಗಳ
ನಕ್ಷತ್ರ-ಗಳು
ನಕ್ಷತ್ರದ
ನಕ್ಷತ್ರ-ದಂತೆ
ನಕ್ಷತ್ರ-ವನ್ನೂ
ನಗಣ್ಯ
ನಗರಕ್ಕೆ
ನಗ-ರ-ಗ-ಳಿಂದ
ನಗ-ರ-ಗಳು
ನಗರದ
ನಗ-ರ-ದಲ್ಲಿ
ನಗ-ರ-ದಿಂದ
ನಗ-ರ-ವಾ-ಸಿ-ಗಳು
ನಗಲು
ನಗಿ-ಸಿದ್ದೇನೆ
ನಗಿ-ಸು-ವುದು
ನಗು
ನಗುತ್ತ
ನಗುತ್ತದೆ
ನಗುತ್ತಲೇ
ನಗುತ್ತಾ
ನಗುತ್ತಾನೆ
ನಗುತ್ತಿ
ನಗುತ್ತೀರಿ
ನಗು-ನ-ಗುತ್ತ
ನಗು-ಮು-ಖದ
ನಗು-ಮು-ಖ-ದಿಂದ
ನಗು-ವನ್ನಾ-ಗಲೀ
ನಗುವನ್ನು
ನಗುವಿದೆ
ನಗು-ವಿ-ನಿಂದ
ನಗುವೂ
ನಗೆ-ಗೀ-ಡಾ-ಗುತ್ತದೆ
ನಗೆ-ಗೀ-ಡಾ-ಗುವ
ನಗೆಗೇಡು
ನಗ್ನನಾಗಿ
ನಜ-ರೇ-ತಿನ
ನಜ್ಜು
ನಜ್ಜು-ಗುಜ್ಜಾ-ಗುತ್ತೇವೆ
ನಟ-ನ-ಟಿ-ಯ-ರನ್ನು
ನಟ-ನಾ-ಗುತ್ತೇನೆ
ನಟ-ನೊಬ್ಬ-ನನ್ನು
ನಟಿಸುತ್ತಾ
ನಟಿ-ಸುತ್ತಾನೆ
ನಟಿ-ಸು-ವ-ವ-ರಿ-ಗಲ್ಲ
ನಟಿ-ಸು-ವ-ವರು
ನಟಿ-ಸು-ವುದು
ನಡತೆ
ನಡತೆಗೆ
ನಡತೆಯ
ನಡ-ತೆ-ಯನ್ನಾ-ದರೂ
ನಡ-ತೆ-ಯನ್ನು
ನಡ-ತೆ-ಯನ್ನೂ
ನಡ-ತೆ-ಯಲ್ಲಿ
ನಡ-ತೆ-ಯಿಂದ
ನಡದೇ
ನಡ-ವ-ಳಿಕೆ
ನಡ-ವ-ಳಿ-ಕೆ-ಗ-ಳಲ್ಲಿ
ನಡ-ವ-ಳಿ-ಕೆ-ಗ-ಳೆಲ್ಲ
ನಡ-ವ-ಳಿ-ಕೆಗೆ
ನಡ-ವ-ಳಿ-ಕೆ-ಯನ್ನು
ನಡ-ವ-ಳಿ-ಕೆ-ಯಲ್ಲಿ
ನಡುಕ
ನಡು-ಗಿ-ದಳು
ನಡುಗುತ್ತ
ನಡು-ಗುತ್ತಾಳೆ
ನಡು-ಗುತ್ತಿದ್ದೆ
ನಡು-ಬೀ-ದಿ-ಯಲ್ಲಿ
ನಡುವಣ
ನಡುವೆ
ನಡುವೆಯೂ
ನಡೆ
ನಡೆಇವು
ನಡೆ-ಇ-ವು-ಗ-ಳನ್ನು
ನಡೆದ
ನಡೆ-ದಂತಾ-ಗು-ವು-ದಿಲ್ಲ
ನಡೆ-ದಂತಾ-ಯಿತು
ನಡೆ-ದಂತೆಯೆ
ನಡೆ-ದಂತೆಲ್ಲ
ನಡೆ-ದಂಥ-ವು-ಗಳೇ
ನಡೆ-ದದ್ದಿಲ್ಲ
ನಡೆದದ್ದು
ನಡೆದದ್ದೇ
ನಡೆದರು
ನಡೆದರೂ
ನಡೆದರೆ
ನಡೆದವು
ನಡೆದಾಗ
ನಡೆ-ದಾ-ಡುತ್ತಿದ್ದ
ನಡೆ-ದಾ-ಡುವ
ನಡೆದಿದೆ
ನಡೆದಿದ್ದ
ನಡೆ-ದಿದ್ದರೆ
ನಡೆ-ದಿ-ರ-ಬೇ-ಕಲ್ಲವೇ
ನಡೆ-ದಿ-ರ-ಬೇಕು
ನಡೆ-ದಿ-ರುತ್ತದೆ
ನಡೆ-ದಿ-ರು-ವಂಥದು
ನಡೆ-ದಿ-ರು-ವಂಥ-ವು-ಗಳೇ
ನಡೆದಿವೆ
ನಡೆದು
ನಡೆ-ದು-ಕೊಂಡ
ನಡೆ-ದು-ಕೊಂಡ
ನಡೆ-ದು-ಕೊಂಡರು
ನಡೆ-ದು-ಕೊಂಡರೆ
ನಡೆ-ದು-ಕೊಂಡಳು
ನಡೆ-ದು-ಕೊಂಡಾ-ದರೂ
ನಡೆ-ದು-ಕೊಂಡಿತು
ನಡೆ-ದು-ಕೊಂಡು
ನಡೆ-ದು-ಕೊಂಡೇ
ನಡೆ-ದು-ಕೊಳ್ಳ-ಬಲ್ಲ
ನಡೆ-ದು-ಕೊಳ್ಳ-ಬ-ಹುದು
ನಡೆ-ದು-ಕೊಳ್ಳ-ಬೇಕು
ನಡೆ-ದು-ಕೊಳ್ಳ-ಬೇ-ಕೆಂದು
ನಡೆ-ದು-ಕೊಳ್ಳಲು
ನಡೆ-ದು-ಕೊಳ್ಳುತ್ತಾನೆ
ನಡೆ-ದು-ಕೊಳ್ಳುತ್ತಾ-ನೆಂಬ
ನಡೆ-ದು-ಕೊಳ್ಳುತ್ತಾರೆ
ನಡೆ-ದು-ಕೊಳ್ಳುವ
ನಡೆ-ದು-ಕೊಳ್ಳು-ವರೋ
ನಡೆ-ದು-ಕೊಳ್ಳು-ವಲ್ಲಿ
ನಡೆ-ದು-ಕೊಳ್ಳು-ವು-ದಕ್ಕೂ
ನಡೆ-ದು-ಕೊಳ್ಳು-ವುದು
ನಡೆ-ದು-ಕೊಳ್ಳು-ವೆ-ನೆಂದು
ನಡೆ-ದು-ದನ್ನು
ನಡೆದುದು
ನಡೆ-ದು-ಬಂದ
ನಡೆ-ದು-ಬಿ-ಡುತ್ತಿದ್ದೆ
ನಡೆದೂ
ನಡೆದೇ
ನಡೆನುಡಿ
ನಡೆ-ನು-ಡಿ-ಗ-ಳನ್ನು
ನಡೆ-ನು-ಡಿ-ಯಲ್ಲಿ
ನಡೆಯ
ನಡೆ-ಯ-ದಂತೆ
ನಡೆ-ಯ-ಬಲ್ಲ
ನಡೆ-ಯ-ಬಲ್ಲೆ
ನಡೆ-ಯ-ಬ-ಹು-ದಾದ
ನಡೆ-ಯ-ಬೇ-ಕಷ್ಟೆ
ನಡೆ-ಯ-ಬೇ-ಕಾದ
ನಡೆ-ಯ-ಬೇ-ಕಾ-ದ-ವರು
ನಡೆ-ಯ-ಬೇಕು
ನಡೆ-ಯ-ಬೇ-ಕೆಂದು
ನಡೆ-ಯ-ಬೇ-ಕೆನ್ನುವ
ನಡೆ-ಯ-ಲಾ-ಗುತ್ತಿ-ರ-ಲಿಲ್ಲ
ನಡೆ-ಯ-ಲಾ-ರವು
ನಡೆ-ಯ-ಲಿಲ್ಲ
ನಡೆಯಲು
ನಡೆಯಿಂದ
ನಡೆ-ಯಿ-ತಾ-ದರೂ
ನಡೆಯಿತು
ನಡೆ-ಯಿ-ಸ-ಬ-ಹುದು
ನಡೆ-ಯಿ-ಸ-ಬೇ-ಕೆಂದು
ನಡೆ-ಯಿ-ಸಲು
ನಡೆಯಿಸಿ
ನಡೆ-ಯಿ-ಸಿ-ಕೊಳ್ಳುತ್ತೇವೋ
ನಡೆ-ಯಿ-ಸಿದ
ನಡೆ-ಯಿ-ಸಿ-ದರೂ
ನಡೆ-ಯಿ-ಸಿ-ದಾಗ
ನಡೆ-ಯಿ-ಸಿ-ದಿರಿ
ನಡೆ-ಯಿ-ಸಿದೆ
ನಡೆ-ಯಿ-ಸಿದ್ದ
ನಡೆ-ಯಿ-ಸಿದ್ದಳು
ನಡೆ-ಯಿ-ಸಿದ್ದಾರೆ
ನಡೆಯಿಸು
ನಡೆ-ಯಿ-ಸುತ್ತ
ನಡೆ-ಯಿ-ಸುತ್ತಿದ್ದ
ನಡೆ-ಯಿ-ಸುತ್ತಿದ್ದಾರೆ
ನಡೆ-ಯಿ-ಸುತ್ತಿ-ರುತ್ತಾರೆ
ನಡೆ-ಯಿ-ಸುತ್ತಿವೆ
ನಡೆ-ಯಿ-ಸುವ
ನಡೆ-ಯಿ-ಸು-ವ-ವರು
ನಡೆ-ಯಿ-ಸು-ವು-ದ-ರಲ್ಲಿ
ನಡೆ-ಯಿ-ಸು-ವು-ದ-ರಿಂದ
ನಡೆ-ಯಿ-ಸು-ವುದು
ನಡೆ-ಯಿ-ಸು-ವು-ದುಂಟು
ನಡೆಯು
ನಡೆಯುತ್ತ
ನಡೆ-ಯುತ್ತದೆ
ನಡೆ-ಯುತ್ತವೆ
ನಡೆ-ಯುತ್ತಾ-ನಷ್ಟೆ
ನಡೆ-ಯುತ್ತಿತ್ತು
ನಡೆ-ಯುತ್ತಿದೆ
ನಡೆ-ಯುತ್ತಿದ್ದರು
ನಡೆ-ಯುತ್ತಿ-ರುತ್ತವೆ
ನಡೆ-ಯುತ್ತಿ-ರುವ
ನಡೆ-ಯುತ್ತಿ-ರು-ವ-ವ-ನಿಗೆ
ನಡೆ-ಯುತ್ತಿಲ್ಲ
ನಡೆ-ಯುತ್ತಿವೆ
ನಡೆ-ಯುತ್ತೇನೆ
ನಡೆ-ಯುತ್ತೇವೆ
ನಡೆಯುವ
ನಡೆ-ಯು-ವಂತೆ
ನಡೆ-ಯು-ವಾಗ
ನಡೆ-ಯು-ವು-ದಲ್ಲ
ನಡೆ-ಯು-ವು-ದಷ್ಟೇ
ನಡೆ-ಯು-ವು-ದಿಲ್ಲ
ನಡೆ-ಯು-ವುದು
ನಡೆ-ಯು-ವು-ದುಂಟು
ನಡೆ-ಯು-ವುದೇ
ನಡೆವ
ನಡೆ-ಸ-ದ-ವ-ರಾ-ಗಿದ್ದರು
ನಡೆ-ಸ-ಲಾ-ರ-ದ-ವರೇ
ನಡೆ-ಸ-ಲಾ-ರದು
ನಡೆಸಲು
ನಡೆಸಿ
ನಡೆ-ಸಿ-ಕೊಳ್ಳುವ
ನಡೆಸಿದ
ನಡೆ-ಸಿ-ದರು
ನಡೆಸಿದೆ
ನಡೆಸಿದ್ದ
ನಡೆ-ಸಿದ್ದರು
ನಡೆ-ಸಿದ್ದರೆ
ನಡೆ-ಸಿದ್ದಾರೆ
ನಡೆ-ಸಿದ್ದಿ-ದೆಯೇ
ನಡೆ-ಸಿ-ರುತ್ತದೆ
ನಡೆಸು
ನಡೆ-ಸುತ್ತ-ದಲ್ಲವೇ
ನಡೆ-ಸುತ್ತಿದ್ದ
ನಡೆ-ಸುತ್ತಿದ್ದರು
ನಡೆ-ಸುತ್ತಿದ್ದಾರೆ
ನಡೆ-ಸುತ್ತಿ-ರುತ್ತದೆ
ನಡೆ-ಸುತ್ತಿ-ರುವ
ನಡೆಸುವ
ನಡೆ-ಸು-ವಷ್ಟು
ನಡೆ-ಸು-ವು-ದಕ್ಕಾಗಿ
ನಡೆ-ಸು-ವು-ದ-ರಲ್ಲಿ
ನಡೆ-ಸು-ವುದು
ನಡೆ-ಸು-ವುದೇ
ನತದೃಷ್ಟ
ನದಿ
ನದಿಗಳೂ
ನದಿ-ತೀ-ರಕ್ಕೆ
ನದಿಯ
ನದಿಯಲ್ಲಿ
ನದಿಯಿಂದ
ನನ-ಗನ್ನಿ-ಸಿತು
ನನ-ಗನ್ನಿ-ಸಿತ್ತು
ನನ-ಗನ್ನಿ-ಸಿದೆ
ನನ-ಗನ್ನಿ-ಸುತ್ತಿದೆ
ನನ-ಗ-ರಿ-ಯ-ದಂತೆ
ನನ-ಗಾ-ಗಲೀ
ನನಗಾಗಿ
ನನ-ಗಾ-ಗಿಯೇ
ನನ-ಗಾ-ಗುತ್ತಿತ್ತು
ನನ-ಗಾ-ಗು-ವು-ದಿಲ್ಲ
ನನ-ಗಾ-ದರೋ
ನನಗಿಂತ
ನನಗಿತ್ತು
ನನಗಿದು
ನನಗಿದೆ
ನನ-ಗಿ-ರುವ
ನನಗಿಲ್ಲ
ನನ-ಗಿಲ್ಲ-ವಾ-ಗಿದೆ
ನನಗೂ
ನನಗೆ
ನನಗೇ
ನನಗೇಕೆ
ನನ-ಗೇ-ತಕ್ಕೆ
ನನ-ಗೇ-ನಾ-ದರೂ
ನನಗೇನು
ನನಗೇನೂ
ನನಗೇನೋ
ನನ-ಗೊಡ್ಡಿದ
ನನ-ಗೊಬ್ಬ-ನಿಗೇ
ನನಗೋ
ನನಸಾದ
ನನಸು
ನನ್ನ
ನನ್ನಂತೆಯೇ
ನನ್ನಂಥ
ನನ್ನಂಥ-ವರ
ನನ್ನ-ದಾ-ಗಿತ್ತು
ನನ್ನದು
ನನ್ನದೇ
ನನ್ನ-ನ-ಳಿಸು
ನನ್ನನ್ನು
ನನ್ನನ್ನೂ
ನನ್ನನ್ನೇ
ನನ್ನನ್ನೇಕೆ
ನನ್ನಲ್ಲಿ
ನನ್ನಲ್ಲಿದೆ
ನನ್ನಲ್ಲುಂಟಾ-ಯಿತು
ನನ್ನ-ವ-ರನ್ನು
ನನ್ನಷ್ಟು
ನನ್ನಾಗಿ
ನನ್ನಿಂದ
ನನ್ನಿಂದಾ-ದೀತೇ
ನನ್ನೆದೆಯ
ನನ್ನೊಡನೆ
ನಮಗದು
ನಮ-ಗಾ-ಗುತ್ತದೆ
ನಮ-ಗಾ-ಗುತ್ತಿ-ರು-ವಾಗ
ನಮ-ಗಾ-ಗು-ವು-ದಿಲ್ಲ
ನಮ-ಗಾ-ಗು-ವುದು
ನಮಗಿಂತ
ನಮ-ಗಿಂತಲೂ
ನಮ-ಗಿಂದಿಗೆ
ನಮಗಿಂದು
ನಮಗಿದೆ
ನಮ-ಗಿ-ದೆಯೆ
ನಮ-ಗಿದ್ದರೆ
ನಮ-ಗಿ-ರ-ಬೇ-ಕಾ-ಗುತ್ತದೆ
ನಮ-ಗಿ-ರುವ
ನಮಗಿಲ್ಲಿ
ನಮಗೂ
ನಮಗೆ
ನಮಗೆಲ್ಲ
ನಮ-ಗೆಲ್ಲ-ರಿಗೂ
ನಮಗೇ
ನಮಗೇನೂ
ನಮಗೊಂದು
ನಮ-ನ-ಗ-ಳಿಂದ
ನಮಸ್ಕ-ರಿಸಿ
ನಮಸ್ಕಾರ
ನಮಸ್ಕಾ-ರವೆ
ನಮಿಸಿ
ನಮ್ಮ
ನಮ್ಮಲ್ಲಿ-ರುವ
ನಮ್ಮಂತೆಯೇ
ನಮ್ಮಂಥ-ವ-ರಿಗೆ
ನಮ್ಮತನ
ನಮ್ಮ-ದಾ-ಗದು
ನಮ್ಮದಾಗಿ
ನಮ್ಮ-ದಾ-ಗಿ-ಸಿ-ಕೊಂಡು
ನಮ್ಮದು
ನಮ್ಮದೇ
ನಮ್ಮ-ನಮ್ಮಲ್ಲೇ
ನಮ್ಮನ್ನದು
ನಮ್ಮನ್ನು
ನಮ್ಮನ್ನೂ
ನಮ್ಮನ್ನೆಲ್ಲ
ನಮ್ಮ-ಪಾ-ಲಿಗೆ
ನಮ್ಮ-ಮ-ನಸ್ಸಿನ
ನಮ್ಮರಾಷ್ಟ್ರ
ನಮ್ಮಲ್ಲಿ
ನಮ್ಮಲ್ಲಿಗೆ
ನಮ್ಮಲ್ಲಿಟ್ಟ
ನಮ್ಮಲ್ಲಿದೆ
ನಮ್ಮಲ್ಲಿದ್ದರೆ
ನಮ್ಮಲ್ಲಿನ
ನಮ್ಮಲ್ಲಿ-ರುವ
ನಮ್ಮಲ್ಲಿವೆ
ನಮ್ಮಲ್ಲೂ
ನಮ್ಮಲ್ಲೇ
ನಮ್ಮವನೇ
ನಮ್ಮಷ್ಟೇ
ನಮ್ಮ-ಸ-ಮಾಜ
ನಮ್ಮಾ-ಯುಸ್ಸಿನ
ನಮ್ಮಿಂದ
ನಮ್ಮಿಂದಾ-ಗದ
ನಮ್ಮಿಂದಾದ
ನಮ್ಮೀ
ನಮ್ಮೆ-ದು-ರಿಗೆ
ನಮ್ಮೆಲ್ಲರ
ನಮ್ಮೆಲ್ಲ-ರಲ್ಲೂ
ನಮ್ಮೊಂದಿ-ಗಿದ್ದಾರೆ
ನಮ್ಮೊ-ಳ-ಗಿ-ರುವ
ನಮ್ಮೊಳಗೆ
ನಮ್ರನಾಗಿ
ನಮ್ರನಾದ
ನಯ
ನಯ-ನಾ-ಜೂ-ಕು-ಗ-ಳನ್ನು
ನಯವಾದ
ನಯ-ವಿ-ನಯ
ನರ
ನರ-ದೌರ್ಬಲ್ಯ
ನರಕ
ನರ-ಕಕ್ಕಿಂತ
ನರ-ಕಕ್ಕಿಂತಲೂ
ನರಕಕ್ಕೆ
ನರಕದ
ನರ-ಕ-ದಲ್ಲಿ
ನರ-ಕ-ಯಾ-ತನೆ
ನರ-ಕ-ವನ್ನು
ನರ-ಕ-ಸ-ದೃ-ಶ-ವಾಗಿ
ನರ-ಕ-ಸಿದ್ಧಾಂತ-ದೊಂದಿಗೆ
ನರಗಳ
ನರ-ಗ-ಳನ್ನು
ನರ-ತ-ನದ
ನರದ
ನರ-ನಾ-ರಿ-ಯರ
ನರ-ನಾ-ರಿ-ಯ-ರಲ್ಲಿ
ನರ-ನಾ-ರಿ-ಯರು
ನರ-ಭಕ್ಷಕ
ನರ-ಭಕ್ಷ-ಕರು
ನರ-ಮಂಡಲ
ನರ-ಮಂಡ-ಲ-ಗಳ
ನರ-ಮಂಡ-ಲ-ಗ-ಳಿಂದ
ನರ-ಮಂಡ-ಲದ
ನರ-ಮಂಡ-ಲ-ವನ್ನು
ನರ-ಮಂಡ-ಲ-ವಾ-ಗಲಿ
ನರರ
ನರರನು
ನರರಲ್ಲಿ
ನರಳದೆ
ನರ-ಳ-ಬೇ-ಕಾ-ಗುತ್ತದೆ
ನರ-ಳ-ಬೇ-ಕಾ-ಯಿತು
ನರ-ಳ-ಬೇ-ಕೆಂದಲ್ಲ
ನರಳಾಟ
ನರ-ಳಾ-ಟಕ್ಕೆ
ನರ-ಳಾ-ಟ-ಗಳ
ನರ-ಳಾ-ಟ-ಗ-ಳಲ್ಲಿ
ನರ-ಳಾ-ಟ-ಗ-ಳಿಗೆ
ನರ-ಳಾ-ಟ-ಗಳು
ನರ-ಳಾ-ಟ-ಗಳೂ
ನರ-ಳಾ-ಟದ
ನರ-ಳಾ-ಟ-ದಿಂದ
ನರ-ಳಾ-ಟ-ವನ್ನು
ನರ-ಳಾ-ಟ-ವನ್ನೂ
ನರ-ಳಾ-ಡಿದ
ನರ-ಳಾ-ಡಿದೆ
ನರ-ಳಾ-ಡುತ್ತಿದ್ದ
ನರಳಿ
ನರಳಿದ
ನರ-ಳಿ-ದರು
ನರ-ಳಿ-ದಾಗ
ನರ-ಳುತ್ತಿ-ರು-ವು-ದನ್ನೂ
ನರಳುತ್ತ
ನರ-ಳುತ್ತದೆ
ನರ-ಳುತ್ತಾನೆ
ನರ-ಳುತ್ತಿದೆ
ನರ-ಳುತ್ತಿದ್ದ
ನರ-ಳುತ್ತಿದ್ದರು
ನರ-ಳುತ್ತಿದ್ದರೂ
ನರ-ಳುತ್ತಿದ್ದರೆ
ನರ-ಳುತ್ತಿದ್ದಳು
ನರ-ಳುತ್ತಿದ್ದಾನೆ
ನರ-ಳುತ್ತಿದ್ದಾರೆ
ನರ-ಳುತ್ತಿದ್ದು-ದಲ್ಲ
ನರ-ಳುತ್ತಿದ್ದೆ
ನರ-ಳುತ್ತಿದ್ದೇವೆ
ನರ-ಳುತ್ತಿ-ರುವ
ನರಳುವ
ನರ-ಳು-ವಂತಾಗಿ
ನರ-ಳು-ವಂತಾ-ಯಿತು
ನರ-ಳು-ವಂತೆಯೇ
ನರ-ಳು-ವರು
ನರ-ಳು-ವ-ವ-ನೆ-ದುರೇ
ನರ-ಳು-ವ-ವ-ರನ್ನು
ನರ-ಳು-ವ-ವ-ರಷ್ಟೇ
ನರ-ಳು-ವಾಗ
ನರ-ಳು-ವಾತ
ನರವ್ಯೂಹ
ನರವ್ಯೂ-ಹ-ಅ-ದೊಂದೇ
ನರವ್ಯೂ-ಹ-ಗಳ
ನರವ್ಯೂ-ಹ-ಗ-ಳಲ್ಲಿ
ನರವ್ಯೂ-ಹ-ಗ-ಳಿಂದ
ನರವ್ಯೂ-ಹವು
ನರ-ಸಿಂಹ-ನನ್ನು
ನರ-ಸಿಂಹನ್
ನರ-ಸೋ-ಬ-ವಾ-ಡಿಯ
ನರಹರಿ
ನರಿ-ನಾ-ಯಿ-ಗಳು
ನರಿಯ
ನರಿ-ಯಾ-ಗುತ್ತಾನೆ
ನರ್ತಿ-ಸುತ್ತಿದ್ದಳು
ನರ್ಸರಿಗೆ
ನಲವತ್ತು
ನಲ-ವತ್ತೆಂಟು
ನಲ-ವತ್ತೈದು
ನಲ-ವತ್ಮೂರು
ನಲಿದರೆ
ನಲಿ-ದಾ-ಡುತ್ತಿದ್ದೇನೆ
ನಲಿದು
ನಲಿ-ದೊ-ಲಿದು
ನಲಿವನ್
ನಲಿವಿಗೆ
ನಲಿವಿನ
ನಲಿ-ವಿ-ನಿಂದ
ನಲಿವು
ನಲಿ-ವು-ಗ-ಳನ್ನು
ನಲಿ-ವು-ಗ-ಳಿಗೆ
ನಲ್ಮೆ
ನಲ್ಮೆಯಿಂದ
ನಲ್ಲ
ನಲ್ಲಿ
ನಲ್ಲಿಯ
ನಲ್ಲಿಯನ್ನು
ನಲ್ಲಿಯೇ
ನಲ್ವತ್ತು
ನಲ್ವತ್ತೈದು
ನವ
ನವ-ಚೇ-ತ-ನ-ವನ್ನೂ
ನವಜಾತ
ನವ-ನಾ-ಗ-ರಿ-ಕ-ತೆಯು
ನವಶಕ್ತಿ
ನವಿ-ರೇ-ಳುವ
ನವಿ-ಲು-ಗಳ
ನವಿ-ಲು-ಗ-ಳನ್ನು
ನವೆಂಬರ್
ನವೋ-ದ-ಯ-ದಲ್ಲಿ
ನವ್ಯರ
ನಶಿಸಿ
ನಶ್ವ-ರ-ವಾದ
ನಷ್ಟ
ನಷ್ಟಗಳ
ನಷ್ಟದ
ನಷ್ಟಪ್ರಾ-ಯ-ವನ್ನಾಗಿ
ನಷ್ಟ-ಮಾ-ಡಿದ್ದ
ನಷ್ಟ-ವಾ-ಗ-ಲಿಲ್ಲ
ನಷ್ಟ-ವಾ-ಯಿತು
ನಸುಕಿಗೆ
ನಾ
ನಾಂದಿ
ನಾಂದಿ-ಯಾ-ಗುತ್ತದೆ
ನಾಂದಿ-ಯಾ-ಯಿತು
ನಾಕ
ನಾಗಪ್ಪಯ್ಯ
ನಾಗ-ರ-ಹಾ-ವಿನ
ನಾಗ-ರ-ಹಾವು
ನಾಗ-ರ-ಹಾ-ವೊಂದು
ನಾಗರಿಕ
ನಾಗ-ರಿ-ಕತೆ
ನಾಗ-ರಿ-ಕ-ತೆ-ಗಳ
ನಾಗ-ರಿ-ಕ-ತೆ-ಗ-ಳಲ್ಲಿ
ನಾಗ-ರಿ-ಕ-ತೆ-ಗಳು
ನಾಗ-ರಿ-ಕ-ತೆಯ
ನಾಗ-ರಿ-ಕ-ತೆ-ಯಿಂದ
ನಾಗ-ರಿ-ಕ-ತೆಯೇ
ನಾಗ-ರಿ-ಕ-ರನ್ನಾಗಿ
ನಾಗ-ರಿ-ಕ-ರನ್ನಾ-ಗಿ-ಸಲು
ನಾಗ-ರಿ-ಕ-ರಾ-ಗಿ-ಬಿ-ಡು-ವರೇ
ನಾಗ-ರಿ-ಕ-ರಿದ್ದೇವೆ
ನಾಗ-ರಿ-ಕರು
ನಾಗ-ರಿ-ಕ-ರೆ-ನಿ-ಸಿ-ಕೊಂಡ
ನಾಗ-ರೀ-ಕ-ತೆಯ
ನಾಚಿ-ಕೆ-ಗೇ-ಡಿನ
ನಾಚಿಕೆಯ
ನಾಚಿ-ಕೆ-ಯಾ-ಯಿತು
ನಾಚಿಕೊಂಡು
ನಾಚುವ
ನಾಚುವಂತೆ
ನಾಜೂಕಾಗಿ
ನಾಜೂಕು
ನಾಜೂ-ಕು-ತನ
ನಾಟಕ
ನಾಟ-ಕ-ಗಳ
ನಾಟ-ಕ-ಗಳು
ನಾಟ-ಕ-ಗಾರ್ತಿ-ಯಾ-ಗಿದ್ದಳು
ನಾಟ-ಕ-ಗೃ-ಹ-ಗ-ಳಲ್ಲಿಯೂ
ನಾಟ-ಕ-ದಲ್ಲಿ
ನಾಟ-ಕ-ನೃತ್ಯ-ಗಾ-ರರೂ
ನಾಟ-ಕಪ್ರ-ದರ್ಶ-ನ-ಗ-ಳನ್ನೂ
ನಾಟ-ಕ-ವನ್ನು
ನಾಟ-ಕ-ವಲ್ಲ
ನಾಟಕವೇ
ನಾಟ-ಕೀ-ಯ-ವಾ-ಗ-ದಂತೆ
ನಾಟ-ಕೀ-ಯವೋ
ನಾಟುತ್ತಿ-ರ-ಲಿಲ್ಲ
ನಾಡಿ
ನಾಡಿನ
ನಾಡಿನಲ್ಲಿ
ನಾಡಿ-ನ-ವರು
ನಾಡಿ-ಬ-ಡಿತ
ನಾಡಿಯ
ನಾಡುವ
ನಾಡು-ವು-ದೆಂದರೆ
ನಾಣ್ಯ-ವಾ-ಗಿ-ರ-ಬ-ಹುದು
ನಾತ್ರ
ನಾದ
ನಾದ-ಮಾ-ಧುರ್ಯ-ದಿಂದ
ನಾನರಿಯೆ
ನಾನ-ವ-ನಿಗೆ
ನಾನಾ
ನಾನಾಗಿ
ನಾನಾದ್ದ-ರಿಂದ
ನಾನಾ-ರೀ-ತಿಯ
ನಾನಾ-ರೀ-ತಿ-ಯಲ್ಲಿ
ನಾನಾ-ರೂ-ಪ-ಗಳು
ನಾನಾವಿಧ
ನಾನಾ-ವಿ-ಧದ
ನಾನಾ-ವಿ-ಧ-ವಾದ
ನಾನಿದ್ದೆ
ನಾನಿನ್ನೇನು
ನಾನಿ-ರುತ್ತೇನೆ
ನಾನಿರುವ
ನಾನಿರುವೆ
ನಾನಿಲ್ಲಿ
ನಾನೀ
ನಾನೀಗ
ನಾನು
ನಾನುನನ್ನ
ನಾನುವಿಗೆ
ನಾನುವಿನ
ನಾನು-ವಿ-ನಲ್ಲಿ
ನಾನು-ವಿ-ವಿಧ
ನಾನೂ
ನಾನೂರು
ನಾನೃ-ತಮ್ಸತ್ಯ-ವೊಂದೇ
ನಾನೆಂಥ
ನಾನೆಷ್ಟು
ನಾನೇ
ನಾನೇಕೆ
ನಾನೇನು
ನಾನೇನೂ
ನಾನೊಂದು
ನಾನೊಬ್ಬ
ನಾನೊಬ್ಬನೇ
ನಾನೋ
ನಾಪಾಲ್ಕೀ-ವಾ-ಲರು
ನಾಮ
ನಾಮಜಪ
ನಾಮದಲ್ಲಿ
ನಾಮ-ರೂ-ಪ-ಗ-ಳಾಗಿ
ನಾಮವನ್ನು
ನಾಮ-ವೊಂದನ್ನು
ನಾಮಸ್ಮ-ರಣೆ
ನಾಮಸ್ಮ-ರ-ಣೆ-ಯನ್ನಂತೂ
ನಾಮಸ್ಮ-ರ-ಣೆ-ಯನ್ನು
ನಾಮೋಚ್ಚಾ-ರಣೆ
ನಾಮ್
ನಾಯಕ
ನಾಯ-ಕ-ರಾಗಿ
ನಾಯ-ಕ-ರಾ-ದರೂ
ನಾಯಕರು
ನಾಯರ್
ನಾಯಿ
ನಾಯಿ-ಗ-ಳಲ್ಲಿ
ನಾಯಿ-ಗ-ಳಿಗೆ
ನಾಯಿ-ಗ-ಳಿ-ರುವ
ನಾಯಿಗಳು
ನಾಯಿಯ
ನಾಯಿಯನ್ನು
ನಾಯಿಯೊಂದು
ನಾರಾ-ಯ-ಣನ
ನಾಲಗೆ
ನಾಲಗೆಯ
ನಾಲ-ಗೆ-ಯನ್ನು
ನಾಲ-ಗೆ-ಯಿಂದ
ನಾಲಿಗೆಯ
ನಾಲಿ-ಗೆ-ಯಲ್ಲಿ
ನಾಲ್ಕನೆಯ
ನಾಲ್ಕ-ನೆ-ಯ-ದಾದ
ನಾಲ್ಕ-ನೆ-ಯದೇ
ನಾಲ್ಕನೇ
ನಾಲ್ಕರಲ್ಲಿ
ನಾಲ್ಕಾರು
ನಾಲ್ಕು
ನಾಲ್ಕುಪ್ರೀ-ತಿಯ
ನಾಲ್ಕು-ಸಾ-ವಿರ
ನಾಲ್ಕೇ
ನಾಲ್ಕೈದು
ನಾಲ್ವರಲ್ಲಿ
ನಾಲ್ವರು
ನಾಳಿನ
ನಾಳೆ
ನಾಳೆಗದು
ನಾಳೆಗಳ
ನಾಳೆಯನ್ನು
ನಾಳೆಯಿಂದ
ನಾಳೆಯೋ
ನಾವದನ್ನು
ನಾವಾಗಲು
ನಾವಾ-ಗು-ವುದು
ನಾವಾ-ಗುತ್ತೇವೆ
ನಾವಾ-ಗು-ವಂತೆ
ನಾವಿಂದು
ನಾವಿಕರ
ನಾವಿಟ್ಟ
ನಾವಿನ್ನೂ
ನಾವಿಬ್ಬರೂ
ನಾವಿಲ್ಲಿ
ನಾವು
ನಾವೂ
ನಾವೆ-ಣಿ-ಸಿ-ದಷ್ಟು
ನಾವೆಲ್ಲ
ನಾವೆಲ್ಲರೂ
ನಾವೇ
ನಾವೇಕೆ
ನಾವೇನು
ನಾವೊಂದು
ನಾಶ
ನಾಶಕ
ನಾಶ-ಕಾ-ರಕ
ನಾಶ-ಕಾರ್ಯ-ದಲ್ಲಿ
ನಾಶಕ್ಕೂ
ನಾಶಕ್ಕೆ
ನಾಶಕ್ಕೇ
ನಾಶ-ಗೈ-ಯುತ್ತ
ನಾಶ-ಗೈ-ಯುವ
ನಾಶ-ಗೊ-ಳಿ-ಸಲು
ನಾಶ-ಗೊ-ಳಿ-ಸುತ್ತದೆ
ನಾಶ-ಗೊ-ಳಿ-ಸುವ
ನಾಶ-ಗೊ-ಳಿ-ಸು-ವುದು
ನಾಶ-ದೊಂದಿಗೆ
ನಾಶ-ಮಾ-ಡ-ಬಲ್ಲುದು
ನಾಶ-ಮಾ-ಡದೆ
ನಾಶ-ಮಾ-ಡ-ಬೇ-ಕಾ-ಗಿಲ್ಲ
ನಾಶಮಾಡಿ
ನಾಶ-ಮಾ-ಡಿದ
ನಾಶ-ಮಾ-ಡಿದೆ
ನಾಶ-ಮಾ-ಡಿ-ಬಿ-ಡು-ವು-ದುಂಟು
ನಾಶ-ಮಾ-ಡುತ್ತ
ನಾಶ-ಮಾ-ಡುತ್ತದೆ
ನಾಶ-ಮಾ-ಡುವ
ನಾಶ-ಮಾ-ಡು-ವುವು
ನಾಶ-ರ-ಹಿತ
ನಾಶ-ರ-ಹಿ-ತ-ವಾದ
ನಾಶ-ವನ್ನಾ-ಗಲಿ
ನಾಶವನ್ನು
ನಾಶವಲ್ಲ
ನಾಶ-ವಾ-ಗದ
ನಾಶ-ವಾ-ಗದು
ನಾಶ-ವಾ-ಗದೇ
ನಾಶ-ವಾ-ಗ-ಬೇ-ಕೆಂಬ
ನಾಶ-ವಾ-ಗಲು
ನಾಶ-ವಾ-ಗಿದ್ದರೂ
ನಾಶ-ವಾ-ಗಿ-ಬಿ-ಡುತ್ತದೆ
ನಾಶ-ವಾ-ಗಿಲ್ಲ
ನಾಶ-ವಾ-ಗುತ್ತ-ವಷ್ಟೆ
ನಾಶ-ವಾ-ಗು-ವು-ದಲ್ಲದೇ
ನಾಶ-ವಾ-ಗು-ವು-ದಿಲ್ಲ
ನಾಶ-ವಾ-ಗು-ವು-ದಿಲ್ಲವೇ
ನಾಶ-ವಾ-ಗು-ವುದು
ನಾಶ-ವಾ-ಗು-ವು-ವೆಂದು
ನಾಶ-ವಾ-ದರು
ನಾಶ-ವಾ-ದರೂ
ನಾಶ-ವಾ-ದವು
ನಾಶ-ವಾ-ದಾಗ
ನಾಶ-ವಾ-ದೀತು
ನಾಶ-ವಾ-ಯಿತು
ನಾಶವಿಲ್ಲ
ನಾಶವೇ
ನಾಸಿಕ್ನಲ್ಲಿ
ನಾಸ್ತಿಕ
ನಾಸ್ತಿಕನ
ನಾಸ್ತಿಕನೂ
ನಾಸ್ತಿ-ಕ-ರಿಗೆ
ನಾಹಂ
ನಿಂತ
ನಿಂತಂತಿದೆ
ನಿಂತಂತೆ
ನಿಂತಂತೆಯೂ
ನಿಂತರೂ
ನಿಂತರೆ
ನಿಂತಲ್ಲೇ
ನಿಂತ-ವ-ನಲ್ಲ
ನಿಂತವನು
ನಿಂತಾಗ
ನಿಂತಿತ್ತು
ನಿಂತಿದೆ
ನಿಂತಿದ್ದರೂ
ನಿಂತಿದ್ದವು
ನಿಂತಿದ್ದು
ನಿಂತಿರಲು
ನಿಂತಿ-ರುತ್ತಿದ್ದರು
ನಿಂತಿರುವ
ನಿಂತಿ-ರು-ವು-ದ-ರಿಂದ
ನಿಂತಿಲ್ಲ
ನಿಂತಿವೆ
ನಿಂತು
ನಿಂತುಕೊಂಡ
ನಿಂತು-ಕೊಂಡಿದ್ದ
ನಿಂತು-ಕೊಂಡಿದ್ದೆ
ನಿಂತು-ಕೊಂಡಿ-ರುವ
ನಿಂತು-ಕೊಳ್ಳುತ್ತೇವೆ
ನಿಂತುದನ್ನೂ
ನಿಂತು-ಹೋ-ಗುತ್ತಿತ್ತು
ನಿಂತು-ಹೋ-ಯಿತು
ನಿಂತೆ
ನಿಂತೊ-ಡ-ನೆಯೇ
ನಿಂದ
ನಿಂದಕರ
ನಿಂದಕರು
ನಿಂದ-ನೀ-ಯ-ವಾಗಿ
ನಿಂದನೆ
ನಿಂದನೆಯ
ನಿಂದಲೂ
ನಿಂದಾಗಿ
ನಿಂದಿಸದೆ
ನಿಂದಿ-ಸ-ಬಾ-ರದು
ನಿಂದಿಸಲಿ
ನಿಂದಿ-ಸಲ್ಪ-ಡುವೆ
ನಿಂದಿ-ಸ-ಹೊ-ರ-ಟರೆ
ನಿಂದಿಸಿ
ನಿಂದಿಸಿದ
ನಿಂದಿಸುತ್ತ
ನಿಂದಿ-ಸುತ್ತಾರೆ
ನಿಂದಿ-ಸುತ್ತಿದ್ದ
ನಿಂದಿಸುವ
ನಿಂದಿ-ಸು-ವು-ದನ್ನೂ
ನಿಂದಿ-ಸು-ವುದು
ನಿಂದೆ
ನಿಂದೆ-ಗ-ಳಿಂದ
ನಿಂದೆ-ಗ-ಳಿ-ಗಿಂತಲೂ
ನಿಂದೆಯ
ನಿಂದೆ-ಯನ್ನಾ-ಗಲಿ
ನಿಂದೆಯನ್ನು
ನಿಃಸ್ವಾರ್ಥ
ನಿಃಸ್ವಾರ್ಥ-ತೆಯೇ
ನಿಃಸ್ವಾರ್ಥತೆ
ನಿಃಸ್ವಾರ್ಥ-ತೆ-ಗಳು
ನಿಃಸ್ವಾರ್ಥ-ತೆಯ
ನಿಃಸ್ವಾರ್ಥ-ತೆಯೇ
ನಿಃಸ್ವಾರ್ಥಪ್ರೀತಿ
ನಿಃಸ್ವಾರ್ಥಪ್ರೀ-ತಿಯ
ನಿಃಸ್ವಾರ್ಥಪ್ರೇಮ
ನಿಕಟ
ನಿಕ-ಟ-ವಾ-ಗಿದ್ದು
ನಿಕ-ಟ-ವಾ-ಗಿ-ರುವ
ನಿಕಷ
ನಿಕಷಕ್ಕೆ
ನಿಕೃಷ್ಟ
ನಿಕೃಷ್ಟ-ಮಟ್ಟ-ದಲ್ಲಿಯೇ
ನಿಕೃಷ್ಟ-ವಾಗಿ
ನಿಕೇ-ತ-ನ-ಗ-ಳನ್ನು
ನಿಕೊಲಸ್
ನಿಕೊಲಾಯ್
ನಿಕೋಲಾಯ್
ನಿಖ-ರ-ವಾಗಿ
ನಿಖ-ರ-ವಾದ
ನಿಗದಿತ
ನಿಗ-ದಿ-ಮಾ-ಡಿ-ದೆ-ಯೆಂದೂ
ನಿಗಮದ
ನಿಗೂ
ನಿಗೂಢ
ನಿಗೂಢತೆ
ನಿಗೂ-ಢಪ್ರಶ್ನೆ-ಗ-ಳಿಗೂ
ನಿಗ್ರಹ
ನಿಗ್ರಹದ
ನಿಗ್ರ-ಹ-ವನ್ನು
ನಿಗ್ರ-ಹ-ವಿಲ್ಲ-ದ-ವರೂ
ನಿಗ್ರ-ಹಿ-ಸಲು
ನಿಗ್ರಹಿಸಿ
ನಿಗ್ರ-ಹಿ-ಸುತ್ತಾನೆ
ನಿಚ್ಚ
ನಿಚ್ಚ-ಣಿ-ಕೆ-ಯನ್ನೇ-ರಲು
ನಿಚ್ಚಳ
ನಿಚ್ಚ-ಳ-ವಾಗಿ
ನಿಚ್ಚ-ಳ-ವಾ-ಗು-ವುದು
ನಿಚ್ಚ-ಳ-ವಾ-ದರೆ
ನಿಚ್ಚಳವೋ
ನಿಜ
ನಿಜಕ್ಕೂ
ನಿಜದ
ನಿಜಪ್ರೀ-ತಿಯ
ನಿಜವನ್ನು
ನಿಜವಾಗಿ
ನಿಜ-ವಾ-ಗಿತ್ತು
ನಿಜ-ವಾ-ಗಿಯೂ
ನಿಜ-ವಾ-ಗಿವೆ
ನಿಜವಾದ
ನಿಜವೆ
ನಿಜವೆಂದು
ನಿಜವೆಂಬ
ನಿಜವೇ
ನಿಟ್ಟಿನ
ನಿಟ್ಟಿನಲ್ಲಿ
ನಿಟ್ಟಿನಲ್ಲೇ
ನಿಟ್ಟುಸಿರು
ನಿಟ್ಟು-ಸಿ-ರು-ಬಿಟ್ಟ
ನಿಟ್ಟು-ಸಿ-ರು-ಬಿಟ್ಟು
ನಿಟ್ಟು-ಸಿ-ರು-ಬಿ-ಡುತ್ತೀರಿ
ನಿತ್ಯ
ನಿತ್ಯಕರ್ಮ
ನಿತ್ಯ-ಕರ್ಮ-ಗ-ಳನ್ನು
ನಿತ್ಯತೆ
ನಿತ್ಯತೆಯ
ನಿತ್ಯತ್ವ
ನಿತ್ಯದ
ನಿತ್ಯ-ನಿ-ಯ-ಮಿತ
ನಿತ್ಯ-ನೂ-ತ-ನ-ವಾ-ದುದು
ನಿತ್ಯವಾದ
ನಿತ್ಯವಿಧಿ
ನಿತ್ಯವು
ನಿತ್ಯವೂ
ನಿತ್ಯಶ್ರದ್ಧೆ-ಯಿಂದ
ನಿತ್ಯ-ಸಂಬಂಧ-ವನ್ನು
ನಿತ್ಯ-ಸತ್ಯ-ವನ್ನು
ನಿತ್ಯ-ಸತ್ಯ-ವಾಗಿ
ನಿತ್ಯ-ಸತ್ಯ-ವೊಂದನ್ನು
ನಿತ್ಯಸ್ಫೂರ್ತಿಯ
ನಿತ್ಯೋತ್ಸವ
ನಿದರ್ಶನ
ನಿದರ್ಶ-ನ-ಗಳ
ನಿದರ್ಶ-ನ-ಗ-ಳನ್ನು
ನಿದರ್ಶ-ನ-ಗ-ಳಲ್ಲಿ
ನಿದರ್ಶ-ನ-ಗ-ಳಿವು
ನಿದರ್ಶ-ನ-ಗ-ಳಿವೆ
ನಿದರ್ಶ-ನ-ಗಳು
ನಿದರ್ಶ-ನ-ದಿಂದಲೂ
ನಿದರ್ಶ-ನ-ರಾಗಿ
ನಿದರ್ಶ-ನ-ವನ್ನಿತ್ತಿದ್ದಾರೆ
ನಿದರ್ಶ-ನ-ವನ್ನು
ನಿದರ್ಶ-ನ-ವಲ್ಲವೇ
ನಿದರ್ಶ-ನ-ವಿದೆ
ನಿದ್ದೆ
ನಿದ್ದೆಯನ್ನು
ನಿದ್ದೆ-ಯಿಂದೇ-ಳು-ವಾ-ಗಲೇ
ನಿದ್ರಾ-ಜ-ಯಿ-ಗ-ಳಾಗಿ
ನಿದ್ರಾನಾಶ
ನಿದ್ರಾ-ಮಾ-ನ-ವ-ರಿದ್ದಾರೆ
ನಿದ್ರಾ-ಸು-ಖ-ದಿಂದ
ನಿದ್ರಾ-ಹೀ-ನತೆ
ನಿದ್ರಾ-ಹೀ-ನ-ತೆ-ಗಳು
ನಿದ್ರಿ-ಪು-ದಲ್ಲಿ
ನಿದ್ರಿಸ
ನಿದ್ರಿ-ಸ-ಬಲ್ಲಳೆ
ನಿದ್ರಿಸಲು
ನಿದ್ರಿ-ಸುತ್ತಿದೆ
ನಿದ್ರಿ-ಸುತ್ತಿದ್ದ
ನಿದ್ರಿ-ಸುತ್ತಿದ್ದಳು
ನಿದ್ರಿ-ಸುತ್ತಿ-ರು-ವಾಗ
ನಿದ್ರಿಸುವ
ನಿದ್ರಿ-ಸು-ವಷ್ಟು
ನಿದ್ರಿ-ಸು-ವಾಗ
ನಿದ್ರಿ-ಸು-ವು-ದಕ್ಕೆ
ನಿದ್ರೆ
ನಿದ್ರೆಗಳು
ನಿದ್ರೆಗೆ
ನಿದ್ರೆ-ಗೊ-ಳ-ಗಾ-ದ-ವ-ರಲ್ಲಿ
ನಿದ್ರೆ-ಗೊ-ಳ-ಪ-ಡಿಸಿ
ನಿದ್ರೆಯ
ನಿದ್ರೆಯನ್ನು
ನಿದ್ರೆಯಲ್ಲಿ
ನಿದ್ರೆಯಲ್ಲೂ
ನಿದ್ರೆಯಿಂದ
ನಿದ್ರೆ-ಯಿಂದೆದ್ದು
ನಿಧಾ-ನ-ವಾಗಿ
ನಿಧಾ-ನ-ವಾ-ದರೂ
ನಿಧಿ
ನಿಧಿಯನ್ನು
ನಿಧಿ-ಯಲ್ಲವೇ
ನಿಧಿ-ಯಾ-ಗ-ಬಲ್ಲುದು
ನಿಧಿ-ಯಾ-ಗಿದ್ದರು
ನಿಧಿ-ಯಾ-ಗುತ್ತಾನೆ
ನಿಧಿಯೇ
ನಿನಗದು
ನಿನ-ಗರ್ಪಿತ
ನಿನಗೀಗ
ನಿನಗೆ
ನಿನಗೇನು
ನಿನಗೇನೂ
ನಿನ್ನ
ನಿನ್ನಂತಹ
ನಿನ್ನನ್ನು
ನಿನ್ನನ್ನೇ
ನಿನ್ನನ್ನೇನು
ನಿನ್ನಲ್ಲಿ
ನಿನ್ನಲ್ಲಿಯೂ
ನಿನ್ನಲ್ಲಿ-ವೆಯೆ
ನಿನ್ನ-ವ-ನು-ನಿನ್ನ-ವನೇ
ನಿನ್ನಷ್ಟು
ನಿನ್ನಿಂದ
ನಿನ್ನೆ
ನಿನ್ನೆಡೆಗೆ
ನಿನ್ನೆಯ
ನಿನ್ನೆ-ಯ-ದಲ್ಲ
ನಿನ್ನೆಲ್ಲ
ನಿನ್ನೊಳಗೆ
ನಿನ್ನೊಳಗೇ
ನಿನ್ನೊಳ್ಪಿಗೆ
ನಿಪು-ಣ-ರಾಗಿ
ನಿಬಂಧ-ನೆ-ಗ-ಳನ್ನು
ನಿಮ-ಗನ್ನಿ-ಸ-ದಿ-ರದು
ನಿಮ-ಗಾ-ದರೂ
ನಿಮಗಾಗಿ
ನಿಮ-ಗಾ-ಗಿ-ರ-ಬೇ-ಕಲ್ಲವೆ
ನಿಮ-ಗಾ-ಗು-ವುದು
ನಿಮ-ಗಾ-ದರೆ
ನಿಮ-ಗಿ-ದೆಯೆ
ನಿಮ-ಗಿ-ರ-ಬೇ-ಕಾ-ಗುತ್ತದೆ
ನಿಮಗೀಗ
ನಿಮಗೂ
ನಿಮಗೆ
ನಿಮ-ಗೆ-ದು-ರಾಗಿ
ನಿಮಗೇಕೆ
ನಿಮ-ಗೇ-ನಾ-ದರೂ
ನಿಮಗೇನು
ನಿಮಗೊಂದು
ನಿಮಿತ್ತ
ನಿಮಿತ್ತ-ವಾ-ಗಲು
ನಿಮಿಷ
ನಿಮಿ-ಷ-ಗಳ
ನಿಮಿ-ಷ-ಗ-ಳನ್ನಾ-ದರೂ
ನಿಮಿ-ಷ-ಗ-ಳಲ್ಲಿ
ನಿಮಿ-ಷ-ಗ-ಳಲ್ಲೇ
ನಿಮಿಷದ
ನಿಮಿ-ಷ-ದಲ್ಲಿ
ನಿಮಿ-ಷ-ದಲ್ಲೇ
ನಿಮ್ನ
ನಿಮ್ಮ
ನಿಮ್ಮ-ದಾ-ಗ-ಬಲ್ಲದು
ನಿಮ್ಮದಾಗಿ
ನಿಮ್ಮ-ದಾ-ಗಿ-ಸಿ-ಕೊಂಡಲ್ಲಿ
ನಿಮ್ಮದೇ
ನಿಮ್ಮನ್ನು
ನಿಮ್ಮ-ಪು-ರಾ-ಣ-ಗ-ಳಲ್ಲಿ
ನಿಮ್ಮ-ಲಿಲ್ಲದ
ನಿಮ್ಮಲ್ಲಿ
ನಿಮ್ಮಲ್ಲಿನ
ನಿಮ್ಮಲ್ಲಿ-ರುವ
ನಿಮ್ಮಲ್ಲಿ-ರು-ವು-ದಾ-ದರೆ
ನಿಮ್ಮಲ್ಲೂ
ನಿಮ್ಮಿಂದ
ನಿಮ್ಮಿಬ್ಬರ
ನಿಮ್ಮೆಡೆಗೆ
ನಿಮ್ಮೆಲ್ಲ-ರನ್ನೂ
ನಿಯಂತ್ರಣ
ನಿಯಂತ್ರ-ಣಕ್ಕೆ
ನಿಯಂತ್ರ-ಣದ
ನಿಯಂತ್ರ-ಣ-ದಲ್ಲಿಟ್ಟು-ಕೊಂಡೆ
ನಿಯಂತ್ರ-ಣ-ದಲ್ಲಿದ್ದು
ನಿಯಂತ್ರ-ಣ-ದಿಂದ
ನಿಯಂತ್ರಿ-ಸುವ
ನಿಯಂತ್ರಿ-ಸ-ದಂತೆ
ನಿಯಂತ್ರಿ-ಸ-ಬಲ್ಲರು
ನಿಯಂತ್ರಿ-ಸ-ಬ-ಹು-ದೆಂಬುದು
ನಿಯಂತ್ರಿ-ಸಲು
ನಿಯಂತ್ರಿ-ಸಲ್ಪಟ್ಟಿದೆ
ನಿಯಂತ್ರಿಸಿ
ನಿಯಂತ್ರಿ-ಸಿ-ಕೊಳ್ಳ-ಲಾ-ರದೆ
ನಿಯಂತ್ರಿ-ಸಿ-ಕೊಳ್ಳುವ
ನಿಯಂತ್ರಿ-ಸಿದ್ದಾ-ನೆಂದು
ನಿಯಂತ್ರಿಸು
ನಿಯಂತ್ರಿ-ಸುತ್ತವೆ
ನಿಯಂತ್ರಿ-ಸುತ್ತಿ-ರುತ್ತ-ದೆ-ಯೆಂದು
ನಿಯಂತ್ರಿ-ಸುವ
ನಿಯಂತ್ರಿ-ಸು-ವುದು
ನಿಯಂತ್ರಿ-ಸು-ವುದೂ
ನಿಯ-ತ-ಕಾ-ಲಿ-ಕೆ-ಗ-ಳಲ್ಲಿ
ನಿಯತಾಂಕ
ನಿಯ-ತಾಂಕದ
ನಿಯ-ತಾಂಕ-ವಾದ
ನಿಯಮ
ನಿಯ-ಮ-ಅ-ವರ
ನಿಯ-ಮಕ್ಕ-ನು-ಗು-ಣ-ವಾಗಿ
ನಿಯ-ಮಕ್ಕಿದೆ
ನಿಯಮಕ್ಕೂ
ನಿಯಮಕ್ಕೆ
ನಿಯ-ಮ-ಗಳ
ನಿಯ-ಮ-ಗ-ಳನ್ನ-ನು-ಸ-ರಿಸಿ
ನಿಯ-ಮ-ಗ-ಳನ್ನು
ನಿಯ-ಮ-ಗ-ಳನ್ನುಲ್ಲಂಘಿ-ಸದೆ
ನಿಯ-ಮ-ಗ-ಳನ್ನೂ
ನಿಯ-ಮ-ಗ-ಳಲ್ಲಿ
ನಿಯ-ಮ-ಗ-ಳಲ್ಲಿಯೇ
ನಿಯ-ಮ-ಗ-ಳಲ್ಲಿ-ರುವ
ನಿಯ-ಮ-ಗ-ಳಾ-ಗಲಿ
ನಿಯ-ಮ-ಗ-ಳಾದ
ನಿಯ-ಮ-ಗ-ಳಿಂದ
ನಿಯ-ಮ-ಗ-ಳಿ-ಗ-ನು-ಗು-ಣ-ವಾಗಿ
ನಿಯ-ಮ-ಗ-ಳಿಗೆ
ನಿಯ-ಮ-ಗ-ಳಿಲ್ಲದ
ನಿಯ-ಮ-ಗಳು
ನಿಯ-ಮ-ಗ-ಳೆಂದು
ನಿಯ-ಮ-ಗ-ಳೇ-ಯೋ-ಗ-ಶಾಸ್ತ್ರ-ದಲ್ಲಿ
ನಿಯಮದ
ನಿಯ-ಮ-ದಂತೆ
ನಿಯ-ಮ-ನಿಷ್ಠೆ
ನಿಯ-ಮ-ಪಾ-ಲನೆ
ನಿಯ-ಮ-ಪಾ-ಲ-ನೆ-ಯಲ್ಲಿ
ನಿಯ-ಮ-ಪಾ-ಲ-ನೆ-ಯಿಂದ
ನಿಯ-ಮ-ಪೂರ್ವ-ಕ-ವಾಗಿ
ನಿಯ-ಮ-ಭಂಗ-ದಿಂದ
ನಿಯ-ಮ-ರ-ಹಿ-ತ-ವಾಗಿ
ನಿಯ-ಮ-ವನ್ನ-ನು-ಸ-ರಿಸಿ
ನಿಯ-ಮ-ವನ್ನು
ನಿಯ-ಮ-ವಲ್ಲವೆ
ನಿಯ-ಮ-ವಾ-ಗಿದೆ
ನಿಯ-ಮ-ವಿದೆ
ನಿಯ-ಮ-ವಿ-ರ-ಲಿಲ್ಲ
ನಿಯಮವು
ನಿಯಮವೆಂ
ನಿಯಮವೇ
ನಿಯಮವೊ
ನಿಯ-ಮ-ಸಾ-ಪೇಕ್ಷ
ನಿಯ-ಮಾ-ನು-ಸಾ-ರ-ವಾಗಿ
ನಿಯಮಿತ
ನಿಯ-ಮಿ-ತ-ವಾಗಿ
ನಿಯ-ಮಿ-ತ-ವಾದ
ನಿಯಾಮಕ
ನಿಯಾ-ಮ-ಕ-ನನ್ನು
ನಿಯಾ-ಮ-ಕನು
ನಿಯಾ-ಮ-ಕನೂ
ನಿಯಾ-ಮ-ಕರ
ನಿಯೋಗದ
ನಿಯೋಜಿತ
ನಿರಂಕು-ಶಪ್ರ-ಭು-ಗ-ಳಾಗಿ
ನಿರಂಜನ
ನಿರಂತರ
ನಿರಂತ-ರ-ವಾಗಿ
ನಿರಂತ-ರ-ವಾದ
ನಿರಂತ-ರವೂ
ನಿರಕ್ಷ-ರಿ-ಗ-ಳಾದ
ನಿರಕ್ಷ-ರಿ-ಯ-ರಾದ
ನಿರಕ್ಷ-ರಿ-ಯಾ-ಗಲಿ
ನಿರ-ತ-ನಾದ
ನಿರ-ತ-ನಾದೆ
ನಿರ-ತ-ರಾಗಿ
ನಿರ-ತ-ರಾದ
ನಿರ-ತ-ರಾ-ದ-ವರು
ನಿರ-ತ-ರಾ-ದ-ವ-ರೆಂದು
ನಿರ-ತ-ರಿಗೆ
ನಿರತರೂ
ನಿರರ್ಗ-ಳ-ವಾಗಿ
ನಿರರ್ಥಕ
ನಿರರ್ಥ-ಕ-ವಾ-ಗಿಯೋ
ನಿರರ್ಥ-ಕ-ವೆಂದು
ನಿರ-ಹಂಕಾರ
ನಿರ-ಹಂಕಾ-ರಿ-ಗಳೂ
ನಿರಾ-ಕ-ರಿ-ಸಲಿ
ನಿರಾ-ಕ-ರಿ-ಸಿದ
ನಿರಾ-ಕ-ರಿ-ಸು-ವ-ವರು
ನಿರಾಕಾರ
ನಿರಾ-ಕಾ-ರ-ನೆಂದು
ನಿರಾ-ಕಾ-ರಿ-ಯೆಂದೂ
ನಿರಾ-ಡಂಬ-ರಿ-ಗಳೂ
ನಿರಾತಂಕ
ನಿರಾ-ತಂಕ-ವಾಗಿ
ನಿರಾ-ಶ-ನಾ-ಗುತ್ತಿದ್ದೆ
ನಿರಾ-ಶ-ರಾ-ಗದೆ
ನಿರಾ-ಶಾ-ದಾ-ಯಕ
ನಿರಾ-ಶಾ-ದಾ-ಯ-ಕ-ವಾ-ದರೂ
ನಿರಾ-ಶಾ-ಮ-ನೋ-ಭಾವ
ನಿರಾ-ಶಾ-ವಾದ
ನಿರಾ-ಶಾ-ವಾ-ದಕ್ಕೆ-ಳ-ಸದ
ನಿರಾ-ಶಾ-ವಾ-ದಿ-ಗ-ಳಾಗಿ
ನಿರಾ-ಶಾ-ವಾ-ದಿ-ಗ-ಳಾ-ಗಿ-ರ-ಬ-ಹುದು
ನಿರಾ-ಶಾ-ವಾ-ದಿ-ಗಳೂ
ನಿರಾಶೆಯ
ನಿರಾಸೆ
ನಿರಾ-ಸೆ-ಗ-ಳನ್ನು
ನಿರಾಸೆಯ
ನಿರೀಕ್ಷಿ-ಸ-ಬ-ಹುದು
ನಿರೀಕ್ಷಿಸಿ
ನಿರೀಕ್ಷಿ-ಸಿದ
ನಿರೀಕ್ಷಿ-ಸಿದ್ದಕ್ಕಿಂತ
ನಿರೀಕ್ಷಿ-ಸುತ್ತ
ನಿರೀಕ್ಷಿ-ಸುತ್ತವೆ
ನಿರೀಕ್ಷಿ-ಸುತ್ತಿದೆ
ನಿರೀಕ್ಷಿ-ಸುತ್ತಿದ್ದಾರೆ
ನಿರೀಕ್ಷಿ-ಸುತ್ತಿ-ರು-ವರು
ನಿರೀಕ್ಷಿ-ಸು-ವ-ವ-ರಿದ್ದಾರೆ
ನಿರೀಕ್ಷಿ-ಸು-ವ-ವರೂ
ನಿರೀಕ್ಷಿ-ಸು-ವುದು
ನಿರೀಕ್ಷಿ-ಸು-ವು-ದುಂಟು
ನಿರೀಕ್ಷೆ
ನಿರೀಕ್ಷೆಗೆ
ನಿರೀಕ್ಷೆಯ
ನಿರೀಕ್ಷೆ-ಯಂತೆ
ನಿರೀಕ್ಷೆ-ಯಲ್ಲಿ
ನಿರುತ್ಸಾಹ
ನಿರುತ್ಸಾ-ಹ-ಗ-ಳಿಂದ
ನಿರುತ್ಸಾ-ಹದ
ನಿರುತ್ಸಾ-ಹ-ದಾ-ಯಕ
ನಿರುತ್ಸಾ-ಹಿ-ಗಳೂ
ನಿರುತ್ಸಾ-ಹಿಯೂ
ನಿರು-ಪದ್ರ-ವಿ-ಗಳೂ
ನಿರು-ಪ-ಯುಕ್ತ
ನಿರೂಪಣೆ
ನಿರೂ-ಪ-ಣೆ-ಗಳು
ನಿರೂ-ಪ-ಣೆಯ
ನಿರೂಪಿಸಿ
ನಿರೂ-ಪಿ-ಸಿದ್ದಲ್ಲದೆ
ನಿರೋ-ಗಿ-ಗ-ಳಾ-ಗ-ಬೇ-ಕಿತ್ತು-ಇ-ದಕ್ಕೆ
ನಿರೋಧಕ
ನಿರೋ-ಧ-ಶಕ್ತಿ
ನಿರ್ಗ-ತಿ-ಕ-ನಾ-ಗಿದ್ದ
ನಿರ್ಗ-ತಿ-ಕ-ರಾಗಿ
ನಿರ್ಗ-ಮಿ-ಸುತ್ತಾನೆ
ನಿರ್ಗು-ಣ-ನಿ-ರಾ-ಕಾ-ರದ
ನಿರ್ಗುಣಶ್ಚ
ನಿರ್ಜನ
ನಿರ್ಜೀ-ವ-ವಾ-ದುವು
ನಿರ್ಣಯ
ನಿರ್ಣಯಕ್ಕೆ
ನಿರ್ಣ-ಯಿ-ಸ-ಬಾ-ರದು
ನಿರ್ಣ-ಯಿ-ಸು-ವು-ದ-ರಲ್ಲಿ
ನಿರ್ಣ-ಯಿ-ಸು-ವುದು
ನಿರ್ಣೀ-ತ-ವಾದ
ನಿರ್ದಯತೆ
ನಿರ್ದ-ಯ-ವಾಗಿ
ನಿರ್ದಾಕ್ಷಿಣ್ಯ-ವಾಗಿ
ನಿರ್ದಿಷ್ಟ
ನಿರ್ದಿಷ್ಟ-ವಾಗಿ
ನಿರ್ದಿಷ್ಟ-ವಾದ
ನಿರ್ದಿಷ್ಟಾ-ವ-ಧಿ-ಯಲ್ಲಿ
ನಿರ್ದೇ-ಶ-ದಲ್ಲಿ
ನಿರ್ದೇಶನ
ನಿರ್ದೇ-ಶ-ನಕ್ಕೆ
ನಿರ್ದೇ-ಶ-ನ-ದಂತೆಯೇ
ನಿರ್ದೇ-ಶ-ನ-ದಲ್ಲಿ
ನಿರ್ದೇ-ಶಾ-ನು-ಸಾರ
ನಿರ್ದೇ-ಶಿ-ತ-ವಾ-ಗದೆ
ನಿರ್ದೇ-ಶಿ-ತವೊ
ನಿರ್ದೇ-ಶಿ-ಸಿ-ದರು
ನಿರ್ದೇ-ಶಿ-ಸಿದ್ದರು
ನಿರ್ದೇ-ಶಿ-ಸುತ್ತಾರೆ
ನಿರ್ದೇ-ಶಿ-ಸುತ್ತಿದ್ದಾರೆ
ನಿರ್ದೇ-ಶಿ-ಸುತ್ತೇವೊ
ನಿರ್ದೋ-ಷ-ವಾದ
ನಿರ್ಧ-ರಿ-ಸ-ಬ-ಹುದು
ನಿರ್ಧ-ರಿ-ಸಿ-ಕೊಂಡಿ-ರ-ಬ-ಹುದು
ನಿರ್ಧ-ರಿ-ಸಿ-ಕೊಳ್ಳ-ಬೇಕು
ನಿರ್ಧ-ರಿ-ಸಿ-ಕೊಳ್ಳಿ
ನಿರ್ಧ-ರಿ-ಸಿ-ದ-ನಾ-ದ-ರೆ-ಎಂದರೆ
ನಿರ್ಧ-ರಿ-ಸಿ-ದರು
ನಿರ್ಧ-ರಿ-ಸಿ-ದು-ದನ್ನು
ನಿರ್ಧ-ರಿ-ಸಿದೆ
ನಿರ್ಧ-ರಿ-ಸುತ್ತಾರೆ
ನಿರ್ಧ-ರಿ-ಸು-ವಾಗ
ನಿರ್ಧಾರ
ನಿರ್ಧಾರಕ್ಕೂ
ನಿರ್ಧಾರಕ್ಕೆ
ನಿರ್ಧಾ-ರಾತ್ಮಕ
ನಿರ್ನಾಮ
ನಿರ್ನಾ-ಮ-ವಾ-ಗುತ್ತಿದ್ದ
ನಿರ್ನಾ-ಮ-ವಾ-ಗು-ವಾಗ
ನಿರ್ಬಂಧ
ನಿರ್ಬಂಧಕ
ನಿರ್ಬಂಧ-ವಿ-ರ-ಲಿಲ್ಲ
ನಿರ್ಬಂಧಿ-ಸುತ್ತದೆ
ನಿರ್ಭ-ಯ-ದಿಂದ
ನಿರ್ಭ-ಯ-ವಾಗಿ
ನಿರ್ಭರತೆ
ನಿರ್ಭ-ರ-ವಾದ
ನಿರ್ಭರಾಂ
ನಿರ್ಭೀತ
ನಿರ್ಭೀ-ತ-ರಾ-ಗ-ಬಲ್ಲೆವು
ನಿರ್ಭೀ-ತ-ರಾ-ಗುತ್ತೇವೆ
ನಿರ್ಭೀತವೂ
ನಿರ್ಭೀತಿ
ನಿರ್ಭೀತಿಯ
ನಿರ್ಭೀ-ತಿ-ಯಿಂದ
ನಿರ್ಮಲ
ನಿರ್ಮಾಣ
ನಿರ್ಮಾ-ಣ-ಕಾರ್ಯ-ದಲ್ಲಿ
ನಿರ್ಮಾ-ಣಕ್ಕಾಗಿ
ನಿರ್ಮಾಣಕ್ಕೆ
ನಿರ್ಮಾಣದ
ನಿರ್ಮಾ-ಣ-ದಲ್ಲಾ-ಗಲೀ
ನಿರ್ಮಾ-ಣ-ದಲ್ಲಿ
ನಿರ್ಮಾ-ಣ-ದಲ್ಲೂ
ನಿರ್ಮಾ-ಣ-ವನ್ನೂ
ನಿರ್ಮಾ-ಣ-ವಾ-ಗ-ದಿದ್ದರೆ
ನಿರ್ಮಾ-ಣ-ವಾ-ಗುತ್ತ-ದೆಂಬು-ದನ್ನು
ನಿರ್ಮಾ-ಣ-ವಾ-ಗುತ್ತಿದೆ
ನಿರ್ಮಾ-ಣ-ವಾ-ಗು-ವು-ದಲ್ಲವೆ
ನಿರ್ಮಾ-ಣ-ವಾ-ಗು-ವುದು
ನಿರ್ಮಾತೃ
ನಿರ್ಮಾ-ತೃ-ಗ-ಳಾದ
ನಿರ್ಮಾ-ಪ-ಕರೂ
ನಿರ್ಮಿ-ತ-ವಲ್ಲ
ನಿರ್ಮಿ-ತ-ವಾ-ಗುತ್ತದೆ
ನಿರ್ಮಿ-ಸ-ಬಲ್ಲದು
ನಿರ್ಮಿ-ಸ-ಬ-ಹುದು
ನಿರ್ಮಿಸಲು
ನಿರ್ಮಿಸಿ
ನಿರ್ಮಿ-ಸಿ-ಕೊಳ್ಳು-ವುದು
ನಿರ್ಮಿ-ಸಿ-ಕೊಂಡ
ನಿರ್ಮಿ-ಸಿ-ಕೊಂಡರು
ನಿರ್ಮಿ-ಸಿ-ಕೊಂಡು
ನಿರ್ಮಿ-ಸಿ-ಕೊಳ್ಳುತ್ತ-ದೆ-ರೇಷ್ಮೆ-ಹುಳು
ನಿರ್ಮಿ-ಸಿ-ಕೊಳ್ಳುತ್ತಾರೆ
ನಿರ್ಮಿ-ಸಿ-ಕೊಳ್ಳುವ
ನಿರ್ಮಿ-ಸಿ-ಕೊಳ್ಳು-ವ-ವರು
ನಿರ್ಮಿ-ಸಿ-ಕೊಳ್ಳು-ವ-ವರೂ
ನಿರ್ಮಿಸಿದ
ನಿರ್ಮಿ-ಸಿ-ದರೂ
ನಿರ್ಮಿ-ಸುತ್ತಿದ್ದೆ
ನಿರ್ಮಿ-ಸುತ್ತಿದ್ದೇವೆ
ನಿರ್ಮಿಸುವ
ನಿರ್ಮೂಲನ
ನಿರ್ಮೂ-ಲ-ಮಾ-ಡು-ವ-ಕ-ಲ-ಬೆ-ರ-ಕೆಯ
ನಿರ್ಮೂಲಿಸಿ
ನಿರ್ಯಾತ
ನಿರ್ಲಕ್ಷಿ-ಸಿದ್ದಾರೆ
ನಿರ್ಲಕ್ಷ್ಯ
ನಿರ್ಲಕ್ಷ್ಯ-ದಿಂದಾಗಿ
ನಿರ್ಲಕ್ಷ್ಯ-ದಿಂದಿದ್ದು
ನಿರ್ಲಜ್ಜ-ನಾಗಿ
ನಿರ್ಲಿಪ್ತ-ರಾಗಿ
ನಿರ್ಲಿಪ್ತತೆ
ನಿರ್ಲಿಪ್ತ-ನಾಗಿ
ನಿರ್ಲಿಪ್ತ-ರಾಗಿ
ನಿರ್ವಂಚ-ನೆ-ಯಿಂದ
ನಿರ್ವಹಣೆ
ನಿರ್ವ-ಹ-ಣೆ-ಗ-ಳನ್ನು
ನಿರ್ವ-ಹ-ಣೆ-ಗ-ಳಿಂದ
ನಿರ್ವ-ಹ-ಣೆಯ
ನಿರ್ವ-ಹ-ಣೆ-ಯನ್ನೂ
ನಿರ್ವ-ಹಿ-ಸಿ-ದರು
ನಿರ್ವ-ಹಿ-ಸ-ದಿದ್ದಲ್ಲಿ
ನಿರ್ವ-ಹಿ-ಸದೇ
ನಿರ್ವ-ಹಿ-ಸ-ಬೇ-ಕಾ-ಗುತ್ತಿತ್ತು
ನಿರ್ವ-ಹಿ-ಸ-ಬೇ-ಕಾದ
ನಿರ್ವ-ಹಿ-ಸ-ಬೇ-ಕೆಂಬುದು
ನಿರ್ವ-ಹಿ-ಸಲು
ನಿರ್ವ-ಹಿ-ಸ-ಲೇ-ಬೇ-ಕಿತ್ತು
ನಿರ್ವಹಿಸಿ
ನಿರ್ವ-ಹಿ-ಸಿ-ದ-ರಂತೆ
ನಿರ್ವ-ಹಿ-ಸುತ್ತ
ನಿರ್ವ-ಹಿ-ಸುತ್ತದೆ
ನಿರ್ವ-ಹಿ-ಸುವ
ನಿರ್ವ-ಹಿ-ಸು-ವಲ್ಲಿ
ನಿರ್ವ-ಹಿ-ಸು-ವಲ್ಲಿಗೇ
ನಿರ್ವ-ಹಿ-ಸು-ವು-ದ-ರಲ್ಲಿದೆ
ನಿರ್ವ-ಹಿ-ಸು-ವುದು
ನಿರ್ವ-ಹಿ-ಸು-ವು-ದೆಂದು
ನಿರ್ವಾತ
ನಿರ್ವಿ-ಕಾ-ರ-ನಾ-ಗಿದ್ದು-ಕೊಂಡು
ನಿರ್ವಿವಾದ
ನಿರ್ವೀರ್ಯ
ನಿರ್ವೀರ್ಯ-ನನ್ನಾಗಿ
ನಿರ್ವೀರ್ಯ-ನಾ-ಗ-ಬೇಡ
ನಿಲ-ಲ-ರಿ-ಯನ್
ನಿಲು-ಕ-ದಂಥದ್ದು
ನಿಲುಕದ
ನಿಲು-ಕ-ದ-ವು-ಗ-ಳಲ್ಲ
ನಿಲುಕದ್ದು
ನಿಲುಕವ
ನಿಲು-ಕುತ್ತವೆ
ನಿಲು-ಕು-ವಂತೆ
ನಿಲು-ಕು-ವಂಥ
ನಿಲು-ಕು-ವಷ್ಟೇ
ನಿಲು-ಕು-ವುದೂ
ನಿಲು-ಗನ್ನ-ಡಿಯ
ನಿಲು-ಮೆ-ಯನ್ನು
ನಿಲುವನ್
ನಿಲು-ವಾ-ಗಿತ್ತು
ನಿಲುವು
ನಿಲುವೂ
ನಿಲ್ದಾಣದ
ನಿಲ್ದಾ-ಣ-ದಲ್ಲಿ
ನಿಲ್ಲದು
ನಿಲ್ಲಬಲ್ಲ
ನಿಲ್ಲ-ಬಲ್ಲೆ-ವೇನು
ನಿಲ್ಲಬೇಡ
ನಿಲ್ಲಲು
ನಿಲ್ಲಲೂ
ನಿಲ್ಲಿ
ನಿಲ್ಲಿಸದೇ
ನಿಲ್ಲಿ-ಸ-ಲಾ-ರಿರಿ
ನಿಲ್ಲಿಸಲು
ನಿಲ್ಲಿಸಲೇ
ನಿಲ್ಲಿಸಿ
ನಿಲ್ಲಿ-ಸಿ-ಕೊಂಡು
ನಿಲ್ಲಿಸಿದ
ನಿಲ್ಲಿ-ಸಿ-ದರು
ನಿಲ್ಲಿಸಿದೆ
ನಿಲ್ಲಿ-ಸಿದ್ದಾರೆ
ನಿಲ್ಲಿ-ಸುತ್ತದೆ
ನಿಲ್ಲಿ-ಸುತ್ತವೆ
ನಿಲ್ಲಿಸುವ
ನಿಲ್ಲಿ-ಸು-ವಂತೆ
ನಿಲ್ಲಿ-ಸು-ವುದು
ನಿಲ್ಲುತ್ತಾನೆ
ನಿಲ್ಲುತ್ತಾರೆ
ನಿಲ್ಲುವ
ನಿಲ್ಲು-ವಂತಾ-ಗಲಿ
ನಿಲ್ಲು-ವಂತಾ-ಗುತ್ತದೆ
ನಿಲ್ಲುವಂತೆ
ನಿಲ್ಲುವವು
ನಿಲ್ಲು-ವು-ದಿಲ್ಲವೇ
ನಿಲ್ಲುವುದು
ನಿವಾರಣೆ
ನಿವಾ-ರ-ಣೆಯ
ನಿವಾ-ರ-ಣೆ-ಯಾ-ಗಿತ್ತು
ನಿವಾ-ರ-ಣೆ-ಯಾ-ಗುತ್ತವೆ
ನಿವಾ-ರ-ಣೆ-ಯಾ-ಗು-ವು-ದರ
ನಿವಾ-ರ-ಣೆ-ಯಾ-ದಂತೆ
ನಿವಾ-ರಿ-ಸಲು
ನಿವಾ-ರಿ-ಸುವ
ನಿವಾ-ರಿ-ಸು-ವಂತಿಲ್ಲ
ನಿವಾ-ರಿ-ಸು-ವಲ್ಲಿ
ನಿವಾ-ರಿ-ಸು-ವು-ದಕ್ಕೆ
ನಿವಾ-ಸಿ-ಗಳ
ನಿವಾ-ಸಿ-ಗಳು
ನಿವಾ-ಸಿ-ಯಾದ
ನಿವೃತ್ತ
ನಿವೃತ್ತ-ಳಾ-ಗಿದ್ದು-ದ-ರಿಂದ
ನಿವೃತ್ತಿ
ನಿವೇ-ದ-ನೆ-ಗ-ಳಿಂದ
ನಿವೇದಿತಾ
ನಿವೇ-ದಿ-ಸಿ-ಕೊಂಡಾಗ
ನಿವೇ-ದಿ-ಸಿ-ಕೊಳ್ಳು-ವುದು
ನಿಶ್ಚಯ
ನಿಶ್ಚ-ಯಜ್ಞಾ-ನವೇ
ನಿಶ್ಚ-ಯ-ದಿಂದ
ನಿಶ್ಚ-ಯ-ವಾಗಿ
ನಿಶ್ಚ-ಯ-ವೆಂದು
ನಿಶ್ಚ-ಯಿ-ಸಿದ
ನಿಶ್ಚ-ಯಿ-ಸಿ-ದರು
ನಿಶ್ಚಿಂತ-ರಾಗಿ
ನಿಶ್ಚಿಂತ-ರಾ-ಗುತ್ತಿದ್ದರು
ನಿಶ್ಚಿಂತೆ
ನಿಶ್ಚಿಂತೆಯ
ನಿಶ್ಚಿಂತೆ-ಯಿಂದ
ನಿಶ್ಚಿತ
ನಿಶ್ಚಿ-ತ-ತೆ-ಯಿಂದ
ನಿಶ್ಚಿ-ತ-ಪಥ
ನಿಶ್ಚಿ-ತ-ವಾಗಿ
ನಿಶ್ಚಿ-ತ-ವಾ-ಗಿ-ರಿ-ಸಿ-ಕೊಳ್ಳಿ
ನಿಶ್ಚಿ-ತ-ವಾ-ದರೆ
ನಿಶ್ಚಿತವೂ
ನಿಶ್ಶಕ್ತನು
ನಿಶ್ಶಕ್ತ-ವಾ-ದವು
ನಿಷೇ-ಧ-ಮಯ
ನಿಷೇಧಾ
ನಿಷೇ-ಧಾತ್ಮಕ
ನಿಷೇ-ಧಾತ್ಮ-ಕ-ವಲ್ಲದ
ನಿಷೇ-ಧಾತ್ಮ-ಕ-ವಾಗಿ
ನಿಷೇ-ಧಾತ್ಮ-ಕ-ವಾದ
ನಿಷೇ-ಧಾತ್ಮ-ಕವೊ
ನಿಷ್ಕಪಟ
ನಿಷ್ಕ-ಪ-ಟಿ-ಯಾಗಿ
ನಿಷ್ಕ-ರು-ಣೆಯ
ನಿಷ್ಕ-ರು-ಣೆ-ಯಿಂದ
ನಿಷ್ಕರ್ಷೆ
ನಿಷ್ಕಲ್ಮಷ
ನಿಷ್ಕಾ-ಮ-ಕರ್ಮ
ನಿಷ್ಕಾ-ಮ-ಕರ್ಮ-ದಿಂದ
ನಿಷ್ಕೃಷ್ಟ
ನಿಷ್ಠಾಯುಕ್ತ
ನಿಷ್ಠಾವಂತ
ನಿಷ್ಠುರದ
ನಿಷ್ಠೆ
ನಿಷ್ಠೆ-ಯಿಲ್ಲದೇ
ನಿಷ್ಠೆ-ಗ-ಳಿಂದ
ನಿಷ್ಠೆಗೆ
ನಿಷ್ಠೆಯ
ನಿಷ್ಠೆಯಿಂದ
ನಿಷ್ಣಾ-ತ-ರಾದ
ನಿಷ್ಪಕ್ಷ-ಪಾತ
ನಿಷ್ಪಕ್ಷ-ಪಾ-ತ-ವಾಗಿ
ನಿಷ್ಪಕ್ಷವೂ
ನಿಷ್ಪ್ರ-ಯೋ-ಜಕ
ನಿಷ್ಪ್ರ-ಯೋ-ಜ-ಕನೇ
ನಿಷ್ಪ್ರ-ಯೋ-ಜ-ಕ-ವೆಂದು
ನಿಷ್ಫಲ
ನಿಷ್ಫ-ಲ-ವಲ್ಲವೆ
ನಿಷ್ಫ-ಲ-ವಲ್ಲವೇ
ನಿಸರ್ಗ
ನಿಸರ್ಗದ
ನಿಸರ್ಗ-ದೆಡೆ
ನಿಸರ್ಗ-ವನ್ನು
ನಿಸ್ತ-ರಂಗ-ಗೊ-ಳಿ-ಸಿದೆ
ನಿಸ್ತೇ-ಜ-ವಾ-ದಾಗ
ನಿಸ್ವಾರ್ಥ
ನಿಸ್ವಾರ್ಥತೆ
ನಿಸ್ವಾರ್ಥ-ತೆಯ
ನಿಸ್ವಾರ್ಥಪ್ರೀ-ತಿಯ
ನಿಸ್ಸಂ
ನಿಸ್ಸಂಕೋ-ಚ-ವಾಗಿ
ನಿಸ್ಸಂದಿಗ್ಧ-ವಾದ
ನಿಸ್ಸಂಶ-ಯ-ವಾಗಿ
ನಿಸ್ಸಾರ
ನಿಸ್ಸೀಮ
ನೀ
ನೀಗಲು
ನೀಗಿದ
ನೀಗಿ-ಸ-ಲಾ-ರವು
ನೀಗುವ
ನೀಗ್ರೊ
ನೀಗ್ರೊ-ಗ-ಳನ್ನೂ
ನೀಗ್ರೋ
ನೀಗ್ರೋಗಳು
ನೀಚ
ನೀಚತೆ
ನೀಚತೆಯ
ನೀಚ-ಮಟ್ಟದ
ನೀಚರಲ್ಲಿ
ನೀಚರಿಗೂ
ನೀಚರು
ನೀಚವೂ
ನೀಚಾ-ತಿ-ನೀ-ಚ-ರೆಂದು
ನೀಡ
ನೀಡದ
ನೀಡ-ದ-ವನು
ನೀಡ-ದಿದ್ದರೆ
ನೀಡದೆ
ನೀಡಬಲ್ಲ
ನೀಡ-ಬಲ್ಲದು
ನೀಡ-ಬಲ್ಲರು
ನೀಡ-ಬಲ್ಲವು
ನೀಡ-ಬಲ್ಲಿರಾ
ನೀಡ-ಬಲ್ಲುದು
ನೀಡ-ಬಲ್ಲು-ದೆಂಬು-ದಕ್ಕೆ
ನೀಡ-ಬ-ಹು-ದಾದ
ನೀಡ-ಬ-ಹುದು
ನೀಡ-ಬಾ-ರ-ದೆಂದು
ನೀಡಬೇಕು
ನೀಡ-ಬೇ-ಕೆಂದು
ನೀಡ-ಲಾ-ಗ-ದಿ-ರು-ವುದು
ನೀಡ-ಲಾ-ಗ-ಲಿಲ್ಲ-ವೆಂದು
ನೀಡ-ಲಾ-ಗುತ್ತಿದೆ
ನೀಡಲಾದ
ನೀಡಲಾಪ
ನೀಡ-ಲಾ-ಪರು
ನೀಡ-ಲಾ-ರದು
ನೀಡ-ಲಾ-ರರು
ನೀಡ-ಲಾ-ರವು
ನೀಡಲು
ನೀಡಿ
ನೀಡಿತು
ನೀಡಿತ್ತು
ನೀಡಿದ
ನೀಡಿ-ದಂತಾ-ಗುತ್ತದೆ
ನೀಡಿದಂತೆ
ನೀಡಿದರು
ನೀಡಿದರೂ
ನೀಡಿದರೆ
ನೀಡಿ-ದ-ವರು
ನೀಡಿದಾಗ
ನೀಡಿ-ದು-ದಲ್ಲ
ನೀಡಿದುದೇ
ನೀಡಿದೆ
ನೀಡಿದೆಯೇ
ನೀಡಿದ್ದ
ನೀಡಿದ್ದರು
ನೀಡಿದ್ದಳು
ನೀಡಿದ್ದಾನೆ
ನೀಡಿದ್ದಾರೆ
ನೀಡಿದ್ದಾ-ರೆ-ವಶ್ಯ-ಸುಪ್ತಿ-ಗೊ-ಳ-ಪ-ಡಿಸಿ
ನೀಡಿದ್ದು
ನೀಡಿಯಾರು
ನೀಡಿ-ರ-ಲಿಲ್ಲ
ನೀಡಿ-ರು-ವರೋ
ನೀಡಿ-ರು-ವುದೂ
ನೀಡೀತು
ನೀಡೀತೆ
ನೀಡು
ನೀಡುತ್ತ
ನೀಡುತ್ತದೆ
ನೀಡುತ್ತ-ದೆಂದು
ನೀಡುತ್ತಾ
ನೀಡುತ್ತಾನೆ
ನೀಡುತ್ತಾರೆ
ನೀಡುತ್ತಿತ್ತು
ನೀಡುತ್ತಿದೆ
ನೀಡುತ್ತಿದ್ದ
ನೀಡುತ್ತಿದ್ದರು
ನೀಡುತ್ತಿದ್ದರೂ
ನೀಡುತ್ತಿದ್ದಾರೆ
ನೀಡುತ್ತಿದ್ದೆ
ನೀಡುತ್ತಿದ್ದೇನೆ
ನೀಡುತ್ತಿ-ರುವ
ನೀಡುವ
ನೀಡು-ವಂತಾ-ಗಿತ್ತು
ನೀಡು-ವಂಥದು
ನೀಡುವರು
ನೀಡುವಲ್ಲಿ
ನೀಡು-ವ-ವನು
ನೀಡು-ವ-ವನೇ
ನೀಡು-ವ-ವರು
ನೀಡು-ವ-ವರೂ
ನೀಡು-ವು-ದನ್ನು
ನೀಡು-ವು-ದ-ರಲ್ಲಲ್ಲ
ನೀಡು-ವು-ದ-ರಲ್ಲಿ
ನೀಡು-ವು-ದಾ-ದರೂ
ನೀಡು-ವು-ದಿಲ್ಲ
ನೀಡುವುದು
ನೀಡುವುದೂ
ನೀಡುವುವು
ನೀಡುವೆ
ನೀಡೋಣ
ನೀಡೌ
ನೀತಿ
ನೀತಿ-ಗ-ಳಲ್ಲಿ
ನೀತಿ-ನಿ-ಯ-ಮಕ್ಕೆ
ನೀತಿ-ನಿ-ಯ-ಮ-ಗ-ಳಲ್ಲಿ
ನೀತಿ-ನಿಷ್ಠ-ನಲ್ಲೂ
ನೀತಿ-ಪ-ರ-ತೆ-ಯನ್ನಾ-ಗಲೀ
ನೀತಿ-ಬಾ-ಹಿರ
ನೀತಿಯ
ನೀನಂದು-ಕೊಂಡಿ-ರುವ
ನೀನಲ್ಲದೇ
ನೀನಾಗು
ನೀನಾ-ಗು-ವುದು
ನೀನಿನ್ನೂ
ನೀನು
ನೀನು-ಮಂಕು-ತಿಮ್ಮ
ನೀನೂ
ನೀನೆ
ನೀನೇ
ನೀನೇಕೆ
ನೀನೇನು
ನೀನೇನೂ
ನೀರ-ಡಿ-ಕೆ-ಗ-ಳನ್ನು
ನೀರ-ಡಿ-ಕೆ-ಗ-ಳಾ-ಗುತ್ತಲೇ
ನೀರ-ನಿಕ್ಕು-ವಂದದಿ
ನೀರನ್ನು
ನೀರನ್ನೂ
ನೀರನ್ನೇ
ನೀರಲ್ಲದೆ
ನೀರವ
ನೀರ-ಸ-ವೆ-ನಿ-ಸಿ-ದರೂ
ನೀರಿ-ಗಿ-ಳಿ-ಯ-ಲೇ-ಬೇಕು
ನೀರಿಗಾಗಿ
ನೀರಿಗೆ
ನೀರಿನ
ನೀರಿ-ನಂತಹ
ನೀರಿನಂತೆ
ನೀರಿನಡಿ
ನೀರಿನಲ್ಲಿ
ನೀರಿನಿಂದ
ನೀರು
ನೀರೂ-ರು-ವು-ದಿಲ್ಲವೇ
ನೀರೆರೆದು
ನೀರೊ-ಳ-ಗಿರ್ದೂ
ನೀಲ
ನೀಲಿ
ನೀಲಿ-ನ-ಕಾಶೆ
ನೀಳ್ಗ-ತೆ-ಯಲ್ಲಿ
ನೀವದಕ್ಕೆ
ನೀವಾ-ಗ-ಲಾ-ರಿರಿ
ನೀವಾದರೋ
ನೀವಾರು
ನೀವಿಡುವ
ನೀವಿನ್ನೂ
ನೀವಿ-ರಿ-ಸಿ-ಕೊಳ್ಳ-ಬ-ಹು-ದಾದ
ನೀವು
ನೀವು-ಗ-ಳೆಲ್ಲ
ನೀವೂ
ನೀವೆಂಥ
ನೀವೆಷ್ಟು
ನೀವೆಷ್ಟೇ
ನೀವೇ
ನೀವೇಕೆ
ನೀಸ್
ನೀಹಾರಿಕೆ
ನೀಹಾ-ರಿ-ಕೆ-ಗಳ
ನೀಹಾ-ರಿ-ಕೆ-ಗ-ಳಲ್ಲಿ
ನೀಹಾ-ರಿ-ಕೆ-ಗಳು
ನುಂಗ-ಲಾ-ಗದ
ನುಂಗ-ಲಾ-ರದ
ನುಂಗ-ಲೇ-ಬೇ-ಕಾ-ಗಿದೆ
ನುಂಗಿ
ನುಂಗಿಕೊಂಡ
ನುಂಗಿ-ಕೊಂಡಂತಿತ್ತು
ನುಂಗಿಕೊಂಡು
ನುಂಗಿ-ಕೊಳ್ಳ-ಬ-ಹುದು
ನುಂಗಿದ
ನುಂಗಿ-ದ-ವಳು
ನುಂಗುತ್ತಾರೆ
ನುಂಗುವ
ನುಗ್ಗ-ಲಾ-ರದ
ನುಗ್ಗಲು
ನುಗ್ಗಿ
ನುಗ್ಗಿತು
ನುಗ್ಗುವುವು
ನುಚ್ಚು-ನೂ-ರಾ-ಯಿತು
ನುಡಿ
ನುಡಿ-ಗ-ಳನ್ನು
ನುಡಿ-ಗ-ಳನ್ನೂ
ನುಡಿ-ಗ-ಳಲ್ಲಲ್ಲ
ನುಡಿ-ಗ-ಳಿಲ್ಲಿವೆ
ನುಡಿಗಳು
ನುಡಿಗಳೂ
ನುಡಿದ
ನುಡಿದಂತೆ
ನುಡಿದರು
ನುಡಿ-ದ-ವ-ಳಲ್ಲ
ನುಡಿ-ದಿದ್ದರು
ನುಡಿ-ಯನ್ನಿಲ್ಲಿ
ನುಡಿಯನ್ನು
ನುಡಿಯಲು
ನುಡಿ-ಯುತ್ತದೆ
ನುಡಿ-ಯುತ್ತಾರೆ
ನುಡಿಯೂ
ನುಡಿಯೊಂದು
ನುಡಿಸಲು
ನುಡಿ-ಸುತ್ತಿದ್ದಾಗ
ನುಣು-ಚಿ-ಕೊಳ್ಳುವ
ನುರಿತ
ನೂಕಿ
ನೂಕು
ನೂಕು-ನುಗ್ಗಲು
ನೂತನ
ನೂತ-ನ-ವಿ-ದೆ-ನಿಪ
ನೂರ
ನೂರಕ್ಕೂ
ನೂರಕ್ಕೆ
ನೂರ-ತೊಂಬತ್ತು
ನೂರ-ಮೂ-ವತ್ತ-ರಿಂದ
ನೂರರಲ್ಲಿ
ನೂರರಷ್ಟು
ನೂರಲ್ಲ
ನೂರಾರು
ನೂರಾ-ರು-ಕೋಟಿ
ನೂರು
ನೂರು-ಕೇ-ಸು-ಗಳ
ನೂರುಪಟ್ಟು
ನೂರೆಂಬತ್ತ-ರಿಂದ
ನೂರೆಂಬತ್ತು
ನೂರೈವತ್ತು
ನೂಲಿನಂತೆ
ನೃತ್ಯ
ನೃತ್ಯದ
ನೆ
ನೆಂಟ-ರಿಷ್ಟರೂ
ನೆಂಟರೆಲ್ಲ
ನೆಂದಾದರೆ
ನೆಂದು
ನೆಂಬ
ನೆಕ್ಕಿ
ನೆಗೆತಕ್ಕೆ
ನೆಗೆಯಲು
ನೆಗೆ-ಯುತ್ತಿದ್ದೆ
ನೆಗೆಯುವ
ನೆಗೆ-ಯು-ವಂತೆ
ನೆಚ್ಚಿ
ನೆಚ್ಚಿಕೊಂಡ
ನೆಚ್ಚಿಕೊಂಡು
ನೆಚ್ಚಿ-ಕೊಳ್ಳದೆ
ನೆಚ್ಚಿದ್ದ-ರಿಂದ
ನೆಚ್ಚಿನ
ನೆಚ್ಚುವ
ನೆಟ್ಟ
ನೆಟ್ಟಗಿಲ್ಲ
ನೆಟ್ಟಗೆ
ನೆಟ್ಟ-ದೃಷ್ಟಿ-ಯಿಂದ
ನೆಟ್ಟಿದ್ದರೂ
ನೆಟ್ಟು
ನೆಡಿ-ಸಿದ್ದೇನೆ
ನೆಡುವಾಗ
ನೆತ್ತರು
ನೆತ್ತಿಗೇರಿ
ನೆನ-ಕೆ-ಗ-ಳನ್ನು
ನೆನಪನ್ನು
ನೆನ-ಪಾ-ಗುತ್ತದೆ
ನೆನ-ಪಾ-ಗು-ವು-ದಾ-ದರೂ
ನೆನಪಿಗೆ
ನೆನ-ಪಿ-ಡ-ಬೇ-ಕು-ಅ-ನು-ಕ-ರಣೆ
ನೆನ-ಪಿ-ಡಲು
ನೆನಪಿಡಿ
ನೆನಪಿತ್ತು
ನೆನಪಿದೆ
ನೆನ-ಪಿ-ದೆಯೆ
ನೆನ-ಪಿ-ನಲ್ಲಿಟ್ಟಿ-ರ-ಬೇಕು
ನೆನ-ಪಿ-ನಲ್ಲಿ-ಡ-ಬೇ-ಕಾದ
ನೆನ-ಪಿ-ನಲ್ಲಿ-ಡ-ಬೇಕು
ನೆನ-ಪಿ-ನಲ್ಲಿ-ರಿ-ಸಿ-ಕೊಂಡು
ನೆನ-ಪಿ-ರಲಿ
ನೆನ-ಪಿ-ರ-ಲಿಲ್ಲ
ನೆನ-ಪಿ-ಸಲು
ನೆನ-ಪಿ-ಸಿ-ಕೊಂಡು
ನೆನ-ಪಿ-ಸಿ-ಕೊಳ್ಳ
ನೆನ-ಪಿ-ಸಿ-ಕೊಳ್ಳ-ಬಲ್ಲ
ನೆನ-ಪಿ-ಸಿ-ಕೊಳ್ಳ-ಬ-ಹುದು
ನೆನ-ಪಿ-ಸಿ-ಕೊಳ್ಳಿ
ನೆನಪು
ನೆನ-ಪು-ಇ-ವು-ಗಳ
ನೆನ-ಪು-ಗಳ
ನೆನ-ಪು-ಗಳು
ನೆನ-ಪು-ಮಾ-ಡಿ-ಕೊಳ್ಳುವಿ
ನೆನಪೇ
ನೆನ-ಸಿ-ಕೊಳ್ಳ-ಬೇಡ
ನೆನೆ
ನೆನೆದು
ನೆನೆ-ನೆ-ನೆದು
ನೆನೆಯ
ನೆನೆಯದ
ನೆನೆಯುವ
ನೆನೆ-ಸಿ-ಕೊಂಡು
ನೆನೆ-ಸಿ-ಕೊಳ್ಳಿ
ನೆನೆ-ಸಿ-ಕೊಳ್ಳು-ವಿ-ಯೇನು
ನೆಪದಲ್ಲಿ
ನೆಪೋ-ಲಿ-ಯನ್
ನೆಮ್ಮದಿ
ನೆಮ್ಮ-ದಿ-ಗ-ಳಿಗೆ
ನೆಮ್ಮದಿಗೆ
ನೆಮ್ಮ-ದಿ-ಯನ್ನು
ನೆಮ್ಮ-ದಿ-ಯನ್ನುಂಟು-ಮಾ-ಡುತ್ತಿತ್ತು
ನೆಮ್ಮ-ದಿ-ಯನ್ನೂ
ನೆಮ್ಮ-ದಿ-ಯಾ-ಗಲಿ
ನೆಮ್ಮ-ದಿ-ಯಾ-ದರೂ
ನೆಯ
ನೆರಳಾಗಿ
ನೆರ-ಳಿ-ನಂತೆ
ನೆರ-ಳಿ-ನಂಥ-ದೊಂದು
ನೆರ-ಳಿ-ನಲ್ಲಿ
ನೆರ-ಳಿ-ನಲ್ಲೇ
ನೆರ-ಳಿ-ರಲಿ
ನೆರಳು
ನೆರ-ಳು-ಬೆ-ಳಕು
ನೆರವನ್ನು
ನೆರ-ವಾ-ಗ-ತೊ-ಡ-ಗಿದ
ನೆರ-ವಾ-ಗ-ಬಲ್ಲರು
ನೆರ-ವಾ-ಗ-ಲಿಲ್ಲ
ನೆರವಾಗಿ
ನೆರ-ವಾ-ಗುತ್ತದೆ
ನೆರ-ವಾ-ಗುತ್ತಾರೆ
ನೆರ-ವಾ-ಗುವ
ನೆರ-ವಾ-ಗು-ವ-ವನೇ
ನೆರವಾದ
ನೆರ-ವಾ-ದುದು
ನೆರ-ವಿ-ಗಾ-ಗಿ-ರುವ
ನೆರವಿಗೆ
ನೆರವಿನ
ನೆರ-ವಿ-ನಿಂದ
ನೆರವು
ನೆರ-ವೇ-ರ-ಲೇ-ಬೇಕು
ನೆರ-ವೇ-ರಿಕೆ
ನೆರ-ವೇ-ರಿ-ದಂತೆ
ನೆರ-ವೇ-ರಿಯೇ
ನೆರ-ವೇ-ರಿ-ಸದೆ
ನೆರ-ವೇ-ರಿ-ಸ-ಬೇ-ಕೆಂಬುದು
ನೆರ-ವೇ-ರಿ-ಸಲು
ನೆರ-ವೇ-ರಿಸಿ
ನೆರ-ವೇ-ರಿ-ಸುತ್ತವೆ
ನೆರ-ವೇ-ರಿ-ಸುತ್ತಿದ್ದ
ನೆರ-ವೇ-ರಿ-ಸುತ್ತಿದ್ದರು
ನೆರ-ವೇ-ರಿ-ಸುವ
ನೆರ-ವೇ-ರುತ್ತದೆ
ನೆರ-ವೇ-ರುತ್ತವೆ
ನೆರ-ವೇ-ರುತ್ತ-ವೆಯೇ
ನೆರೆ
ನೆರೆ-ಕೆ-ರೆ-ಯಲ್ಲೂ
ನೆರೆ-ಕೆ-ರೆ-ಯ-ವ-ರನ್ನೂ
ನೆರೆ-ಕೆ-ರೆ-ಯ-ವ-ರಲ್ಲೂ
ನೆರೆದ
ನೆರೆ-ನಿ-ಪು-ಣ-ನೆಂಬುದು
ನೆರೆ-ಮ-ನೆಯ
ನೆರೆಯ
ನೆರೆ-ಹೊ-ರೆ-ಯ-ವ-ರನ್ನು
ನೆರೆ-ಹೊ-ರೆಯ
ನೆರೆ-ಹೊ-ರೆ-ಯಲ್ಲೂ
ನೆರೆ-ಹೊ-ರೆ-ಯ-ವರು
ನೆರೆ-ಹೊ-ರೆ-ಯ-ವ-ರೊಂದಿಗೆ
ನೆಲ-ಅ-ವ-ರಿಗೆ
ನೆಲಕ್ಕೆ
ನೆಲ-ಜ-ಲಾ-ಗ-ಸ-ಗ-ಳಲ್ಲಿ
ನೆಲದ
ನೆಲ-ದ-ಮೇಲೆ
ನೆಲದಲ್ಲಿ
ನೆಲ-ದಲ್ಲಿಟ್ಟು
ನೆಲದಿಂದ
ನೆಲ-ಮಾ-ಳಿ-ಗೆಯ
ನೆಲ-ಮಾ-ಳಿ-ಗೆ-ಯಲ್ಲಿ
ನೆಲವನ್ನು
ನೆಲವನ್ನೇ
ನೆಲಸದು
ನೆಲಸಮ
ನೆಲ-ಸ-ಲಾ-ರಂಭಿ-ಸಿದ
ನೆಲ-ಸಿ-ದರು
ನೆಲ-ಸಿದ್ದರು
ನೆಲ-ಸಿ-ರುತ್ತದೆ
ನೆಲಸುವ
ನೆಲೆ
ನೆಲೆಗಟ್ಟು
ನೆಲೆಗೊಂಡು
ನೆಲೆನಿಂತ
ನೆಲೆನಿಂತು
ನೆಲೆ-ನಿಲ್ಲ-ಬೇಕು
ನೆಲೆ-ನಿಲ್ಲಲೂ
ನೆಲೆ-ಬೀ-ಡಾದ
ನೆಲೆಯನ್ನು
ನೆಲೆಯಲ್ಲಿ
ನೆಲೆಯಾಗಿ
ನೆಲೆಯಾದ
ನೆಲೆಯೂ
ನೆಲೆಸಲು
ನೆಲೆಸಿ
ನೆಲೆಸಿದ
ನೆಲೆಸಿದ್ದ
ನೆಲೆ-ಸಿದ್ದರು
ನೆಲೆಸು
ನೆಲೆ-ಸುತ್ತದೆ
ನೆವದಿ
ನೆವದಿಂದ
ನೆಹರು
ನೆಹರೂ
ನೇ
ನೇಗಿ-ಲಿ-ನಿಂದ
ನೇಣು
ನೇತೃತ್ವ-ದಲ್ಲಿ
ನೇಮ
ನೇಮ-ಕ-ಗೊಂಡು
ನೇಮ-ಕ-ಗೊಂಡಾಗ
ನೇಮ-ಕ-ಗೊಂಡಿದ್ದಾರೆ
ನೇಯ್ದರು
ನೇರ
ನೇರಕ್ಕೆ
ನೇರ-ದಾ-ರಿ-ಗಿಂತ
ನೇರಳೆ
ನೇರವಾಗಿ
ನೇವ-ರಿ-ಸಿ-ದಳು
ನೇಸರು
ನೈಚ್ಯಾ-ನು-ಸಂಧಾ-ನ-ದಿಂದ
ನೈಚ್ಯಾ-ನು-ಸಂಧಿಯ
ನೈಜ
ನೈಜತೆಯ
ನೈಜ-ಭಕ್ತಿ-ಯಿಂದ
ನೈಜಸ್ಥಿ-ತಿ-ಯನ್ನು
ನೈಜಸ್ಥಿ-ತಿ-ಯಲ್ಲಿ
ನೈಜಸ್ವ-ಭಾವ
ನೈಜಸ್ವ-ಭಾ-ವ-ಗಳ
ನೈಜಸ್ವ-ಭಾ-ವದ
ನೈಜಸ್ವ-ರೂಪ
ನೈಜಸ್ವ-ರೂ-ಪ-ಗ-ಳನ್ನು
ನೈಜಸ್ವ-ರೂ-ಪದ
ನೈಜಸ್ವ-ರೂ-ಪ-ವನ್ನು
ನೈಜಸ್ವ-ರೂ-ಪ-ವಾದ
ನೈತಿಕ
ನೈತಿಕತೆ
ನೈತಿ-ಕ-ನಿಷ್ಠೆ-ಯಿಂದ
ನೈತಿ-ಕ-ನಿಷ್ಠೆ-ಯಿಲ್ಲದ
ನೈತಿ-ಕ-ಮಟ್ಟದ
ನೈತಿ-ಕ-ವಾಗಿ
ನೈತಿ-ಕ-ಶಕ್ತಿ
ನೈಪುಣ್ಯ
ನೈರ್ಮಲ್ಯ
ನೈವೇದ್ಯ
ನೈಸರ್ಗಿಕ
ನೊಂದ
ನೊಂದವನು
ನೊಂದಿಗೆ
ನೊಂದಿದ್ದಳು
ನೊಂದು
ನೊಂದುಕೊಂಡ
ನೊಂದು-ಕೊಂಡಿದ್ದರೂ
ನೊಂದು-ಕೊಳ್ಳದೆ
ನೊಂದೆ
ನೊಣ
ನೊಣೆ-ಯು-ವಂತೆ
ನೊಬೆಲ್
ನೊಯ್
ನೋಟಕ್ಕೆ
ನೋಟ-ದಿಂದಲೆ
ನೋಟು
ನೋಡ
ನೋಡ-ತೊ-ಡ-ಗಿ-ದರು
ನೋಡದ
ನೋಡದಂತೆ
ನೋಡ-ದಿದ್ದರೆ
ನೋಡದೆ
ನೋಡದೇ
ನೋಡಬಲ್ಲ
ನೋಡ-ಬಲ್ಲ-ವ-ನಾ-ಗಿದ್ದ
ನೋಡ-ಬ-ಹು-ದಾ-ಗಿದೆ
ನೋಡ-ಬ-ಹುದು
ನೋಡ-ಬೇ-ಕಾ-ಗಿದೆ
ನೋಡ-ಬೇ-ಕಾ-ಯಿತು
ನೋಡಬೇಕು
ನೋಡ-ಬೇ-ಕೆಂಬ
ನೋಡ-ಬೇ-ಡವೇ
ನೋಡಬೇಡಿ
ನೋಡಯ್ಯಾ
ನೋಡ-ಲಾ-ಗದು
ನೋಡ-ಲಾ-ಗಿದೆ
ನೋಡಲಿ
ನೋಡಲಿಲ್ಲ
ನೋಡಲು
ನೋಡಿ
ನೋಡಿ-ಕೊಂಡರು
ನೋಡಿ-ಕೊಳ್ಳುತ್ತಾನೆ
ನೋಡಿಕೊಂಡ
ನೋಡಿ-ಕೊಂಡರು
ನೋಡಿ-ಕೊಂಡರೇ
ನೋಡಿ-ಕೊಂಡಲ್ಲ
ನೋಡಿ-ಕೊಂಡಲ್ಲಿ
ನೋಡಿ-ಕೊಂಡಿದ್ದೆ
ನೋಡಿ-ಕೊಂಡಿ-ರು-ವಾಗ
ನೋಡಿಕೊಂಡು
ನೋಡಿ-ಕೊಂಡೆನು
ನೋಡಿ-ಕೊಳ್ಳ-ದಿದ್ದರೆ
ನೋಡಿ-ಕೊಳ್ಳ-ಬಲ್ಲ
ನೋಡಿ-ಕೊಳ್ಳ-ಬೇ-ಕಾ-ಗಿತ್ತು
ನೋಡಿ-ಕೊಳ್ಳ-ಬೇಕು
ನೋಡಿ-ಕೊಳ್ಳ-ಬೇಡಿ
ನೋಡಿ-ಕೊಳ್ಳ-ಲಾ-ರದ
ನೋಡಿ-ಕೊಳ್ಳ-ಲಿಲ್ಲ
ನೋಡಿ-ಕೊಳ್ಳಲು
ನೋಡಿಕೊಳ್ಳಿ
ನೋಡಿಕೊಳ್ಳು
ನೋಡಿ-ಕೊಳ್ಳುತ್ತದೆ
ನೋಡಿ-ಕೊಳ್ಳುತ್ತಾರೆ
ನೋಡಿ-ಕೊಳ್ಳುತ್ತಿದ್ದ
ನೋಡಿ-ಕೊಳ್ಳುತ್ತಿದ್ದಾರೆ
ನೋಡಿ-ಕೊಳ್ಳುತ್ತಿದ್ದೆನು
ನೋಡಿ-ಕೊಳ್ಳುತ್ತಿದ್ದೇನೆ
ನೋಡಿ-ಕೊಳ್ಳುತ್ತೇನೆ
ನೋಡಿ-ಕೊಳ್ಳುವ
ನೋಡಿ-ಕೊಳ್ಳು-ವ-ವ-ರಿಗೆ
ನೋಡಿ-ಕೊಳ್ಳು-ವಷ್ಟು
ನೋಡಿ-ಕೊಳ್ಳು-ವುದು
ನೋಡಿಕೋ
ನೋಡಿತು
ನೋಡಿದ
ನೋಡಿದಂತೆ
ನೋಡಿದರು
ನೋಡಿದರೂ
ನೋಡಿದರೆ
ನೋಡಿ-ದ-ವನು
ನೋಡಿದಾಗ
ನೋಡಿ-ದಾ-ಗಲೂ
ನೋಡಿ-ದಾ-ಗ-ಲೆಲ್ಲ
ನೋಡಿದಿರಾ
ನೋಡಿದೆ
ನೋಡಿದೆಯಾ
ನೋಡಿ-ದೊ-ಡ-ನೆಯೇ
ನೋಡಿದ್ದ
ನೋಡಿದ್ದರು
ನೋಡಿದ್ದ-ರೆ-ಭ-ಯದ
ನೋಡಿದ್ದಿ
ನೋಡಿದ್ದೀರಾ
ನೋಡಿದ್ದೆ
ನೋಡಿದ್ದೇನೆ
ನೋಡಿ-ಬಂದಾಗ
ನೋಡಿ-ಯಾ-ಯಿತು
ನೋಡಿಯೇ
ನೋಡಿ-ರ-ಬ-ಹುದು
ನೋಡಿರಿ
ನೋಡಿರುವ
ನೋಡು
ನೋಡುತ್ತ
ನೋಡುತ್ತ-ಲಿದ್ದ
ನೋಡುತ್ತ-ಲಿದ್ದರು
ನೋಡುತ್ತ-ಲಿದ್ದೇನೆ
ನೋಡುತ್ತಲೇ
ನೋಡುತ್ತಾ
ನೋಡುತ್ತಾನೆ
ನೋಡುತ್ತಾ-ರಂತೆ
ನೋಡುತ್ತಾರೆ
ನೋಡುತ್ತಿತ್ತು
ನೋಡುತ್ತಿದ್ದ
ನೋಡುತ್ತಿದ್ದಂತೆಯೇ
ನೋಡುತ್ತಿದ್ದರು
ನೋಡುತ್ತಿದ್ದರೂ
ನೋಡುತ್ತಿದ್ದಾನೆ
ನೋಡುತ್ತಿದ್ದೆ
ನೋಡುತ್ತಿ-ರ-ಲಿಲ್ಲ
ನೋಡುತ್ತಿ-ರುತ್ತಾಳೆ
ನೋಡುತ್ತೀರಿ
ನೋಡುತ್ತೇನೆ
ನೋಡುತ್ತೇ-ನೆಂದು
ನೋಡುತ್ತೇವೆ
ನೋಡುವ
ನೋಡು-ವಂತೆಯೇ
ನೋಡುವರೋ
ನೋಡು-ವ-ವನೋ
ನೋಡು-ವ-ವ-ರಿ-ಗಿಂತ
ನೋಡು-ವ-ವ-ರಿಗೆ
ನೋಡುವಾಗ
ನೋಡು-ವು-ದನ್ನೂ
ನೋಡು-ವು-ದ-ರಲ್ಲಲ್ಲ
ನೋಡು-ವು-ದ-ರಿಂದ
ನೋಡು-ವು-ದ-ರಿಂದಲೇ
ನೋಡು-ವು-ದಿಲ್ಲ
ನೋಡುವುದು
ನೋಡೋ
ನೋಡೋಣ
ನೋಡೋ-ಣ-ವೆಂದು
ನೋಪಾಯಕ್ಕೆ
ನೋಬೆಲ್
ನೋಯಿಸದೇ
ನೋಯಿಸಲು
ನೋಯಿ-ಸಿ-ದ-ವ-ನಾ-ಗಿದ್ದ
ನೋಯಿಸಿದ್ದಿ
ನೋಯಿಸಿವೆ
ನೋಯಿ-ಸುತ್ತಿಲ್ಲ
ನೋಯಿಸುವ
ನೋವನ್ನು
ನೋವನ್ನುಂಟು
ನೋವನ್ನುಂಟು-ಮಾಡಿ
ನೋವನ್ನುಂಟು-ಮಾ-ಡಿ-ದ-ವರ
ನೋವನ್ನೂ
ನೋವಾ-ಗ-ದಂತೆ
ನೋವಾ-ಗುತ್ತದೆ
ನೋವಾ-ಗುತ್ತಿದೆ
ನೋವಿಗೆ
ನೋವಿನ
ನೋವಿನಿಂದ
ನೋವಿ-ರು-ವು-ದಿಲ್ಲ-ವೆಂದು
ನೋವಿಲ್ಲದ
ನೋವು
ನೋವುಂಟಾ-ಯಿತು
ನೋವು-ಗ-ಳನ್ನು
ನೋವು-ನ-ಲಿವು
ನೋವು-ಸಂಕಟ
ನೋವೇ
ನೌಕರ
ನೌಕ-ರ-ನಾ-ಗಿದ್ದು
ನೌಕರರ
ನೌಕ-ರ-ರನ್ನು
ನೌಕ-ರ-ರಿಗೆ
ನೌಕ-ರ-ರಿ-ಗೇನೋ
ನೌಕ-ರ-ರಿದ್ದರು
ನೌಕರರು
ನೌಕರಿಯ
ನೌಕ-ರಿ-ಯಲ್ಲಿ-ರುವ
ನೌಕಾಬಲ
ನೌಕೆ
ನೌಕೆ-ಗ-ಳಲ್ಲಿ
ನ್ನಾಗಲೀ
ನ್ನೀಯುವ
ನ್ನೀಯು-ವು-ದುಂಟು
ನ್ನೆಸ-ಗಿ-ದ-ವರು
ನ್ನೋಣ
ನ್ಯಾ
ನ್ಯಾಯ
ನ್ಯಾಯ-ಪ-ರನೂ
ನ್ಯಾಯನಿಷ್ಠ
ನ್ಯಾಯ-ಪ-ರತೆ
ನ್ಯಾಯ-ಪ-ರಾ-ಯ-ಣತೆ
ನ್ಯಾಯಪ್ರಿ-ಯ-ರಂತೆ
ನ್ಯಾಯಬದ್ಧ
ನ್ಯಾಯ-ವಂತ-ರಾಗಿ
ನ್ಯಾಯವಾದಿ
ನ್ಯಾಯ-ವಾ-ದಿಗೆ
ನ್ಯಾಯ-ವಿ-ಧಾ-ನ-ಗ-ಳಲ್ಲಿ
ನ್ಯಾಯವೇ
ನ್ಯಾಯಶಾಸ್ತ್ರ
ನ್ಯಾಯಾ-ಲ-ಯದ
ನ್ಯಾಯಾಂಗ
ನ್ಯಾಯಾ-ಧಿ-ಕಾ-ರಿ-ಗಳ
ನ್ಯಾಯಾನ್ಯಾ-ಯ-ಗಳು
ನ್ಯಾಯಾ-ಲ-ಯ-ದಲ್ಲಿ
ನ್ಯೂ
ನ್ಯೂಟನ್
ನ್ಯೂಟನ್ನ
ನ್ಯೂಟ್ರಾ-ನು-ಗ-ಳೆಂಬ
ನ್ಯೂಟ್ರಾನ್
ನ್ಯೂನತೆ
ನ್ಯೂನ-ತೆ-ಗ-ಳನ್ನು
ನ್ಯೂನ-ತೆ-ಗ-ಳಿಗೆ
ನ್ಯೂನತೆಗೆ
ನ್ಯೂನಾ-ತಿ-ರೇ-ಕ-ಗ-ಳನ್ನು
ನ್ಯೂಮಾವ್
ನ್ಯೂಯಾರ್ಕ್
ನ್ಯೂಯಾರ್ಕ್ನ
ನ್ಯೂಯಾರ್ಕ್ನಲ್ಲಿದ್ದಾಗ
ನ್ವಯ
ನ್ಹಾಸಾ-ದಲ್ಲಿ-ರುವ
ಪಂಗಡ
ಪಂಗ-ಡ-ಗ-ಳಾ-ಗು-ವುದು
ಪಂಗ-ಡ-ಗ-ಳೊ-ಳ-ಗಿನ
ಪಂಗ-ಡ-ಗ-ಳಂತೆ
ಪಂಗ-ಡ-ಗ-ಳಲ್ಲಿ
ಪಂಗ-ಡ-ಗ-ಳಾ-ಗುತ್ತಿವೆ
ಪಂಗ-ಡ-ಗ-ಳಿವೆ
ಪಂಗ-ಡ-ಗ-ಳೊಂದಿಗೆ
ಪಂಗ-ಡ-ಗ-ಳೊ-ಳಗೆ
ಪಂಗಡದ
ಪಂಗ-ಡ-ವನ್ನು
ಪಂಚ-ಭೂ-ತ-ಗ-ಳನ್ನು
ಪಂಚವಟಿ
ಪಂಚ-ವ-ಟಿಯ
ಪಂಚ-ವ-ಟಿ-ಯಿಂದ
ಪಂಚ-ವಾರ್ಷಿಕ
ಪಂಚ-ಶೀ-ಲ-ಗ-ಳನ್ನು
ಪಂಚಾ-ಮೃ-ತ-ವೆಂದು
ಪಂಚೀ-ಕ-ರ-ಣವೇ
ಪಂಚೇಂದ್ರಿ-ಯ-ಗಳ
ಪಂಜರ
ಪಂಜ-ರ-ಗ-ಳನ್ನು
ಪಂಜ-ರ-ದಲ್ಲಿ
ಪಂಜ-ರ-ದಿಂದ
ಪಂಡಿತ
ಪಂಡಿತಂ
ಪಂಡಿತಃ
ಪಂಡಿತನು
ಪಂಡಿ-ತ-ರಿ-ಗಂದೇ
ಪಂಡಿತರು
ಪಂಢ-ರ-ಪು-ರಕ್ಕೆ
ಪಂಥ
ಪಂಥಕ್ಕೆ
ಪಂಥ-ಗ-ಳನ್ನು
ಪಂಥ-ಗ-ಳಲ್ಲಿ
ಪಂಥ-ಗ-ಳ-ವರೇ
ಪಂಥ-ಗ-ಳಿಗೂ
ಪಂಥಗಳು
ಪಂಥಾಹ್ವಾನ
ಪಂಥಾಹ್ವಾ-ನ-ವಾ-ಯಿತು
ಪಂದ್ಯ
ಪಂದ್ಯದಲ್ಲಿ
ಪಂದ್ಯಾಟಕ್ಕೆ
ಪಂಪಿಸುವ
ಪಂಪಿ-ಸು-ವಂತೆ
ಪಕ್ಕದಲ್ಲಿ
ಪಕ್ಕವಾದ್ಯ
ಪಕ್ಕಿ
ಪಕ್ವತೆ
ಪಕ್ವಫಲ
ಪಕ್ವವಾಗಿ
ಪಕ್ಷ
ಪಕ್ಷಕ್ಕೆ
ಪಕ್ಷ-ಗ-ಳ-ವರೂ
ಪಕ್ಷಪಾತ
ಪಕ್ಷ-ಪಾ-ತಿ-ಯಲ್ಲ
ಪಕ್ಷವನ್ನು
ಪಕ್ಷ-ವೊಂದನ್ನು
ಪಕ್ಷಸ್ವಾರ್ಥ-ದಿಂದ
ಪಕ್ಷಿ-ಗ-ಳನ್ನು
ಪಕ್ಷಿ-ಗ-ಳನ್ನೂ
ಪಕ್ಷಿ-ಗ-ಳಲ್ಲಿ
ಪಕ್ಷಿಗಳೂ
ಪಕ್ಷಿ-ಗ-ಳೆಲ್ಲ-ವನ್ನೂ
ಪಕ್ಷಿಯು
ಪಗಡೆಯ
ಪಟ-ಲ-ದಲ್ಲಿ
ಪಟುಗಳು
ಪಟೇಲರು
ಪಟ್ಟ
ಪಟ್ಟಣ
ಪಟ್ಟಣಕ್ಕೆ
ಪಟ್ಟ-ಣ-ಗ-ಳಲ್ಲಿ
ಪಟ್ಟ-ಣ-ಗಳು
ಪಟ್ಟಣದ
ಪಟ್ಟ-ಣ-ದಲ್ಲಿ
ಪಟ್ಟ-ಣ-ದಿಂದ
ಪಟ್ಟಭದ್ರ
ಪಟ್ಟಾಗಿ
ಪಟ್ಟಿ
ಪಟ್ಟಿಗಳ
ಪಟ್ಟಿದ್ದ
ಪಟ್ಟಿದ್ದಿ
ಪಟ್ಟಿಯನ್ನು
ಪಟ್ಟು-ಕೊಳ್ಳದೆ
ಪಠಿಸಿದ
ಪಠಿ-ಸುತ್ತವೆ
ಪಠಿ-ಸುತ್ತಾಳೆ
ಪಠಿ-ಸು-ವಂತಾ-ಗ-ಬೇ-ಕಾ-ದರೆ
ಪಠಿ-ಸು-ವಂತೆ
ಪಠಿ-ಸು-ವು-ದಲ್ಲವೆ
ಪಠಿ-ಸು-ವುದು
ಪಠ್ಯ
ಪಠ್ಯೇತರ
ಪಡದೆ
ಪಡದೇ
ಪಡ-ಬ-ಹುದು
ಪಡ-ಬಾ-ರದ
ಪಡ-ಬೇ-ಕಾ-ಗು-ವುದು
ಪಡ-ಬೇ-ಕಿಲ್ಲ
ಪಡಿ-ಸ-ಬ-ಹುದು
ಪಡಿಸಲು
ಪಡಿ-ಸಿ-ಕೊಳ್ಳ-ಬೇಕು
ಪಡಿ-ಸಿ-ಕೊಳ್ಳು-ವು-ದರ
ಪಡಿ-ಸಿದ್ದರು
ಪಡಿ-ಸು-ವುದು
ಪಡುಕೋಣೆ
ಪಡುತ್ತಾನೆ
ಪಡುತ್ತಾರೆ
ಪಡುವ
ಪಡೆ
ಪಡೆಯುವ
ಪಡೆದ
ಪಡೆ-ದಂತಾ-ಗು-ವು-ದ-ರಿಂದ
ಪಡೆ-ದಂದಿ-ನಿಂದ
ಪಡೆ-ದದ್ದಲ್ಲ
ಪಡೆ-ದ-ರಾ-ದರೆ
ಪಡೆದರು
ಪಡೆದರೂ
ಪಡೆದರೆ
ಪಡೆದರೇ
ಪಡೆದಲ್ಲಿ
ಪಡೆ-ದ-ವ-ನಲ್ಲ
ಪಡೆ-ದ-ವನು
ಪಡೆ-ದ-ವನೇ
ಪಡೆ-ದ-ವರ
ಪಡೆ-ದ-ವ-ರನ್ನು
ಪಡೆ-ದ-ವ-ರಾಗಿ
ಪಡೆ-ದ-ವರು
ಪಡೆ-ದ-ವ-ರೆಂದಾ-ಗಲಿ
ಪಡೆದವೆ
ಪಡೆದಾಗ
ಪಡೆದಾರು
ಪಡೆದಿದೆ
ಪಡೆದಿದ್ದ
ಪಡೆ-ದಿದ್ದ-ನೆಂಬುದು
ಪಡೆ-ದಿದ್ದರು
ಪಡೆ-ದಿದ್ದರೂ
ಪಡೆ-ದಿದ್ದರೆ
ಪಡೆ-ದಿದ್ದಾರೆ
ಪಡೆ-ದಿದ್ದಾ-ರೆಯೇ
ಪಡೆ-ದಿದ್ದೇನೆ
ಪಡೆ-ದಿದ್ದೇವೆ
ಪಡೆ-ದಿ-ರಿ-ಸಿ-ಕೊಂಡ
ಪಡೆದಿವೆ
ಪಡೆದು
ಪಡೆ-ದು-ಕೊಂಡ
ಪಡೆ-ದು-ಕೊಂಡ-ವ-ರಲ್ಲ
ಪಡೆ-ದು-ಕೊಂಡಿದ್ದಾ-ನೆನ್ನಿ
ಪಡೆ-ದು-ಕೊಂಡಿದ್ದೆ
ಪಡೆ-ದು-ಕೊಂಡಿದ್ದೇ-ವೆಂದು
ಪಡೆ-ದು-ಕೊಂಡು
ಪಡೆ-ದು-ಕೊಳ್ಳು-ವು-ದ-ರಲ್ಲಿ
ಪಡೆ-ದು-ದನ್ನು
ಪಡೆ-ದು-ದಲ್ಲ
ಪಡೆದುದು
ಪಡೆ-ದು-ಹೋ-ಗುತ್ತಿದ್ದವು
ಪಡೆದೂ
ಪಡೆದೆವು
ಪಡೆದೇ
ಪಡೆಯ
ಪಡೆ-ಯ-ತೊ-ಡ-ಗಿದ್ದರು
ಪಡೆಯದ
ಪಡೆ-ಯ-ದಿದ್ದರೂ
ಪಡೆಯದು
ಪಡೆಯದೆ
ಪಡೆ-ಯ-ಬಲ್ಲ
ಪಡೆ-ಯ-ಬಲ್ಲರು
ಪಡೆ-ಯ-ಬಲ್ಲವು
ಪಡೆ-ಯ-ಬ-ಹು-ದಾದ
ಪಡೆ-ಯ-ಬ-ಹುದು
ಪಡೆ-ಯ-ಬ-ಹು-ದೆಂದು
ಪಡೆ-ಯ-ಬ-ಹು-ದೆಂದೂ
ಪಡೆ-ಯ-ಬ-ಹು-ದೆಂಬು-ದನ್ನು
ಪಡೆ-ಯ-ಬ-ಹು-ದೆಂಬುದು
ಪಡೆ-ಯ-ಬೇ-ಕಾ-ಗುತ್ತದೆ
ಪಡೆ-ಯ-ಬೇ-ಕಾ-ದದ್ದು
ಪಡೆ-ಯ-ಬೇ-ಕಾ-ದರೆ
ಪಡೆ-ಯ-ಬೇಕು
ಪಡೆ-ಯ-ಬೇ-ಕೆಂಬ
ಪಡೆ-ಯ-ಲಾ-ರದೆ
ಪಡೆ-ಯ-ಲಾ-ರರು
ಪಡೆ-ಯ-ಲಾ-ರೆವು
ಪಡೆ-ಯ-ಲಿಚ್ಛಿ-ಸು-ವ-ವನು
ಪಡೆಯಲು
ಪಡೆ-ಯ-ಲೇ-ಬೇ-ಕಾ-ಗುತ್ತ-ದಷ್ಟೆ
ಪಡೆ-ಯಲ್ಲಿದ್ದ
ಪಡೆಯಿತು
ಪಡೆಯಿರಿ
ಪಡೆಯು
ಪಡೆಯುತ್ತ
ಪಡೆ-ಯುತ್ತದೆ
ಪಡೆ-ಯುತ್ತಲೇ
ಪಡೆ-ಯುತ್ತಾನೆ
ಪಡೆ-ಯುತ್ತಾ-ನೆ-ಸೋ-ಮಾ-ರಿ-ಯಾಗಿ
ಪಡೆ-ಯುತ್ತಾರೆ
ಪಡೆ-ಯುತ್ತಾ-ರೆ-ಇಂಥ-ವರು
ಪಡೆಯುತ್ತಿ
ಪಡೆ-ಯುತ್ತಿದ್ದರು
ಪಡೆ-ಯುತ್ತಿದ್ದಾರೆ
ಪಡೆ-ಯುತ್ತಿದ್ದೆ
ಪಡೆ-ಯುತ್ತಿ-ರ-ಬ-ಹುದು
ಪಡೆಯುವ
ಪಡೆ-ಯು-ವಂತಾ-ಗ-ಬೇಕು
ಪಡೆ-ಯು-ವಂತಾ-ದರೆ
ಪಡೆ-ಯು-ವಂತೆ
ಪಡೆ-ಯು-ವನು
ಪಡೆ-ಯು-ವ-ನೆಂದು
ಪಡೆ-ಯು-ವರು
ಪಡೆ-ಯು-ವ-ರೆಂದು
ಪಡೆ-ಯು-ವ-ವ-ನನ್ನು
ಪಡೆ-ಯು-ವ-ವರೂ
ಪಡೆ-ಯು-ವ-ವ-ರೆಗೂ
ಪಡೆ-ಯು-ವಿರಾ
ಪಡೆಯುವು
ಪಡೆ-ಯು-ವು-ದಕ್ಕಾಗಿ
ಪಡೆ-ಯು-ವು-ದಕ್ಕಾ-ಗಿಯೂ
ಪಡೆ-ಯು-ವು-ದಕ್ಕಾ-ಗಿಯೆ
ಪಡೆ-ಯು-ವು-ದಕ್ಕೆ
ಪಡೆ-ಯು-ವುದು
ಪಡೆ-ಯು-ವುದೇ
ಪಡೆ-ಯು-ವುವು
ಪಡೆಯೇ
ಪಣತೊಟ್ಟ
ಪಣ-ನೀ-ಡಿ-ದರು
ಪಣ-ವಾ-ಗಿಟ್ಟು
ಪತಂಜಲಿ
ಪತಂಜ-ಲಿ-ಮ-ಹರ್ಷಿ
ಪತನ
ಪತನಕ್ಕೆ
ಪತ-ನ-ದತ್ತ
ಪತ-ನ-ದಿಂದ
ಪತಿ
ಪತಿಗಾಗಿ
ಪತಿಗೆ
ಪತಿ-ಪತ್ನಿ-ಯ-ರಿಬ್ಬ-ರಲ್ಲೂ
ಪತಿ-ಪ-ರಾ-ಯ-ಣೆ-ಯಾದ
ಪತಿಯ
ಪತಿಯನ್ನು
ಪತಿಯು
ಪತಿಯೂ
ಪತಿಯೇ
ಪತಿ-ವಿ-ಯೋ-ಗದ
ಪತಿಸೇವಾ
ಪತ್ತೆದಾರ
ಪತ್ತೆ-ಯಾ-ಗ-ಲಿಲ್ಲ
ಪತ್ತೆ-ಯಾ-ಗು-ವುದು
ಪತ್ತೆ-ಹಚ್ಚ-ಲಾ-ಗಿದೆ
ಪತ್ತೆ-ಹಚ್ಚಲು
ಪತ್ತೆ-ಹಚ್ಚಿದ್ದಾರೆ
ಪತ್ತೇದಾರ
ಪತ್ನಿ
ಪತ್ನಿಯ
ಪತ್ನಿಯನ್ನು
ಪತ್ನಿಯರ
ಪತ್ನಿಯಲ್ಲಿ
ಪತ್ನಿಯು
ಪತ್ನಿಯೇ
ಪತ್ನಿ-ಯೊಂದಿಗೆ
ಪತ್ರ
ಪತ್ರಗಳೂ
ಪತ್ರದ
ಪತ್ರದಂತೆ
ಪತ್ರ-ಬ-ರೆದ
ಪತ್ರವನ್ನು
ಪತ್ರವ್ಯ-ವ-ಹಾರ
ಪತ್ರಿಕಾ
ಪತ್ರಿ-ಕಾ-ಕರ್ತರೂ
ಪತ್ರಿಕೆ
ಪತ್ರಿ-ಕೆ-ಗ-ಳಲ್ಲಿ
ಪತ್ರಿ-ಕೆ-ಗ-ಳ-ವರ
ಪತ್ರಿ-ಕೆ-ಗಳು
ಪತ್ರಿ-ಕೆ-ಗಾಗಿ
ಪತ್ರಿಕೆಯ
ಪತ್ರಿ-ಕೆ-ಯನ್ನು
ಪತ್ರಿ-ಕೆ-ಯಲ್ಲಿ
ಪತ್ರಿ-ಕೆ-ಯಲ್ಲೂ
ಪತ್ರಿ-ಕೋದ್ಯಮಿ
ಪಥ
ಪಥಕ್ಕೂ
ಪಥಕ್ಕೆ
ಪಥ-ಗ-ಳಿಂದ
ಪಥಗಳು
ಪಥದಲ್ಲಿ
ಪಥ-ದಲ್ಲಿದೆ
ಪಥದಿಂದ
ಪಥಪ್ರ-ದರ್ಶ-ಕ-ಳಾದ
ಪಥವನ್ನು
ಪಥ-ವಾ-ಗದು
ಪಥ್ಯ
ಪಥ್ಯ-ವೆ-ನಿ-ಸದು
ಪದ
ಪದ-ಗ-ಳನ್ನು
ಪದ-ಗ-ಳನ್ನೂ
ಪದಗಳು
ಪದ-ತ-ಲ-ದಲ್ಲಿ
ಪದ-ತ-ಲದಿ
ಪದದಿಂದ
ಪದ-ರ-ಗ-ಳನ್ನು
ಪದ-ರ-ಗ-ಳಲ್ಲಿ
ಪದ-ರ-ಗ-ಳಲ್ಲೇ
ಪದ-ರು-ಗ-ಳಲ್ಲಿ
ಪದವಿ
ಪದ-ವಿ-ಗ-ಳನ್ನು
ಪದ-ವಿ-ಗ-ಳನ್ನೂ
ಪದ-ವೀ-ಧರ
ಪದ-ವೀ-ಧ-ರ-ನನ್ನಾಗಿ
ಪದ-ವೀ-ಧ-ರ-ನಾಗಿ
ಪದ-ವೀ-ಧ-ರನೂ
ಪದ-ವೀ-ಧ-ರ-ರಿಗೆ
ಪದ-ವೀ-ಧ-ರರು
ಪದ-ವೀ-ಧರೆ
ಪದ-ವೀ-ಧ-ರೆ-ಯಾ-ದರು
ಪದವೇ
ಪದಾ-ಧಿ-ಕಾ-ರಿ-ಗ-ಳಾಗಿ
ಪದಾರ್ಥ
ಪದಾರ್ಥ-ಗಳ
ಪದಾರ್ಥ-ಗ-ಳನ್ನು
ಪದಾರ್ಥ-ಗ-ಳಾಗಿ
ಪದೇ
ಪದ್ಧತಿ
ಪದ್ಧತಿಈ
ಪದ್ಧತಿಗೆ
ಪದ್ಧತಿಯ
ಪದ್ಧ-ತಿ-ಯನ್ನೇ
ಪದ್ಧ-ತಿ-ಯಲ್ಲಿ
ಪದ್ಧ-ತಿ-ಯಿಂದ
ಪದ್ಧತಿಯೂ
ಪಯಣ
ಪಯ-ಣಿ-ಸಿ-ದಂತೆ
ಪಯ-ಣಿ-ಸಿದೆ
ಪಯಣಿಸು
ಪಯ-ಣಿ-ಸುವ
ಪಯ-ಣಿ-ಸು-ವು-ದಕ್ಕಾ-ಗಿದೆ
ಪಯ-ಣಿ-ಸು-ವೆ-ಯಲ್ಲವೆ
ಪಯೋಷ್ಣೀ
ಪರಂಜ್ಯೋ-ತಿಯ
ಪರಂಜ್ಯೋ-ತಿಯೇ
ಪರಂಪ-ರಾ-ಗತ
ಪರಂಪರೆ
ಪರಂಪ-ರೆ-ಗಳ
ಪರಂಪ-ರೆ-ಗ-ಳನ್ನೇ
ಪರಂಪ-ರೆ-ಗ-ಳಿಂದ
ಪರಂಪ-ರೆಗೆ
ಪರಂಪ-ರೆಯ
ಪರಂಪ-ರೆ-ಯನ್ನು
ಪರಂಪ-ರೆ-ಯನ್ನೇ
ಪರಂಪ-ರೆ-ಯಲ್ಲಿ
ಪರಂಪ-ರೆ-ಯಾಗಿ
ಪರಂಪ-ರೆ-ಯಿಂದ
ಪರಂಪ-ರೆಯು
ಪರಕಾಯ
ಪರಕಾಲ
ಪರ-ಕೀ-ಯರ
ಪರ-ಚಿ-ಬಿಟ್ಟಾರು
ಪರತತ್ತ್ವ
ಪರ-ತತ್ತ್ವದ
ಪರ-ತತ್ತ್ವವು
ಪರ-ತತ್ತ್ವವೇ
ಪರತೆ
ಪರದಲ್ಲಿ
ಪರದುಃಖ
ಪರದೆ
ಪರದೆಯ
ಪರ-ದೆ-ಯನ್ನು
ಪರ-ದೇ-ಶಿ-ಗ-ಳಾ-ಗ-ಬಾ-ರದು
ಪರನಾದ
ಪರನಿಂದೆ
ಪರ-ನಿಂದೆಯ
ಪರ-ಪು-ರುಷ
ಪರ-ಬೊಮ್ಮಂಗೆ-ಮಂಕು-ತಿಮ್ಮ
ಪರಮ
ಪರ-ಮ-ಕರ್ತವ್ಯ-ನಿಷ್ಠೆ-ಯಿಂದ
ಪರ-ಮ-ಗುರಿ
ಪರ-ಮಜ್ಞಾ-ನಿ-ಗ-ಳಿಗೆ
ಪರಮತ
ಪರ-ಮ-ತಕ್ಕೂ
ಪರ-ಮ-ತಕ್ಕೆ
ಪರ-ಮ-ತ-ದ-ವ-ರನ್ನು
ಪರ-ಮ-ತ-ವನ್ನು
ಪರ-ಮ-ಪ-ವಿತ್ರ-ವಾದ
ಪರ-ಮ-ಪೂಜ್ಯ-ರಾದ
ಪರ-ಮಪ್ರಿಯ
ಪರ-ಮ-ಭಕ್ತ-ನಾದ
ಪರ-ಮ-ಭಕ್ತೆ-ಯಾದ
ಪರ-ಮ-ಭಾಗ್ಯ-ಶಾ-ಲಿ-ಗ-ಳಾ-ದರು
ಪರ-ಮ-ಮಂತ್ರ
ಪರ-ಮ-ವೈ-ಭ-ವ-ವನ್ನು
ಪರ-ಮ-ಶಾಸ್ತ್ರಕ್ಕಿಂತ
ಪರ-ಮಶ್ರೇಷ್ಠ
ಪರ-ಮ-ಸತ್ಯ
ಪರ-ಮ-ಸತ್ಯದ
ಪರ-ಮ-ಸತ್ಯಸ್ವ-ರೂ-ಪ-ನಾದ
ಪರ-ಮ-ಸಾಧ್ವಿ
ಪರ-ಮ-ಸೂಕ್ಷ್ಮ
ಪರ-ಮ-ಸೂಕ್ಷ್ಮ-ವಾ-ದು-ದ-ರಿಂದ
ಪರ-ಮ-ಹಂಸ
ಪರ-ಮ-ಹಂಸರ
ಪರ-ಮ-ಹಂಸ-ರನ್ನು
ಪರ-ಮ-ಹಂಸ-ರಲ್ಲಿಗೇ
ಪರ-ಮ-ಹಂಸರು
ಪರ-ಮ-ಹಂಸರೂ
ಪರ-ಮ-ಹಂಸ-ರೆಂದರು
ಪರ-ಮ-ಹಂಸ-ರೆನ್ನುತ್ತಿದ್ದ-ರು-ಕು-ರಿ-ಯಂತೆ
ಪರ-ಮ-ಹಂಸರೇ
ಪರಮಾಣು
ಪರ-ಮಾ-ಣು-ಗ-ಳನ್ನು
ಪರ-ಮಾ-ಣು-ಗಳು
ಪರ-ಮಾ-ಣು-ಗ-ಳೆಂಬ
ಪರ-ಮಾ-ಣು-ಲೀ-ಲೆ-ಯನ್ನು
ಪರ-ಮಾ-ಣು-ವಿನ
ಪರ-ಮಾ-ಣು-ಶಕ್ತಿಯ
ಪರಮಾತ್ಮ
ಪರ-ಮಾತ್ಮನ
ಪರ-ಮಾತ್ಮ-ನಲ್ಲಿ
ಪರ-ಮಾತ್ಮ-ನಲ್ಲಿ-ರುವ
ಪರ-ಮಾತ್ಮನು
ಪರ-ಮಾ-ದ-ರದಿ
ಪರ-ಮಾದ್ಭುತ
ಪರ-ಮಾದ್ಭು-ತ-ದಿಂದಲೇ
ಪರ-ಮಾದ್ಭು-ತ-ವಾದ
ಪರಮಾನ
ಪರ-ಮಾಪ್ತ-ನಾ-ಗಿಯೂ
ಪರ-ಮಾಪ್ತ-ನಾದ
ಪರಮಾರ್ಥ
ಪರ-ಮಾರ್ಥ-ಇ-ವು-ಗ-ಳನ್ನು
ಪರ-ಮಾರ್ಥಕ್ಕೂ
ಪರ-ಮಾರ್ಥ-ತತ್ತ್ವದ
ಪರ-ಮಾರ್ಥ-ದಲ್ಲಿನ
ಪರ-ಮಾರ್ಥ-ವನ್ನು
ಪರ-ಮಾ-ವಧಿ
ಪರ-ಮಾ-ವ-ಧಿ-ಗೇ-ರುತ್ತದೆ
ಪರ-ಮಾ-ವ-ಧಿ-ಯನ್ನು
ಪರ-ಮಾಶ್ಚರ್ಯ
ಪರ-ಮಾಶ್ಚರ್ಯ-ವೆ-ನಿ-ಸುತ್ತದೆ
ಪರಮಾಸ್ತ್ರ
ಪರ-ಮೋದ್ದೇ-ಶದ
ಪರ-ಮೌ-ಷಧ
ಪರರ
ಪರ-ರಿ-ಗಿಂತ
ಪರರಿಗೆ
ಪರ-ಲೋ-ಕದ
ಪರ-ಲೋ-ಕ-ದಲ್ಲೂ
ಪರ-ವ-ಶ-ನಾಗಿ
ಪರ-ವಾ-ಗಿಲ್ಲ
ಪರಸ್ತ್ರೀ-ಯ-ರಲ್ಲಿ
ಪರಸ್ಪರ
ಪರಸ್ಪ-ರರ
ಪರ-ಹಿ-ತಕ್ಕಾಗಿ
ಪರ-ಹಿ-ತ-ವನ್ನು
ಪರ-ಹಿ-ತಾ-ಸಕ್ತ-ರಾದ
ಪರಾಕಾಷ್ಠೆ
ಪರಾ-ಕಾಷ್ಠೆ-ಗೇ-ರಿ-ದರು
ಪರಾ-ಕಾಷ್ಠೆ-ಯನ್ನೂ
ಪರಾ-ಕಾಷ್ಠೆ-ಯನ್ನೇ-ರ-ಬೇಕು
ಪರಾ-ಕಾಷ್ಠೆ-ಯನ್ನೇರಿ
ಪರಾ-ಕಾಷ್ಠೆ-ಯನ್ನೇ-ರಿದ
ಪರಾಕ್ರಮ
ಪರಾಕ್ರ-ಮ-ಗಳೂ
ಪರಾಧೀನ
ಪರಾ-ಧೀ-ನ-ವಾ-ಗಿ-ರುವ
ಪರಾ-ನು-ಕ-ರ-ಣ-ಶೀ-ಲ-ತೆಯೂ
ಪರಾ-ನು-ಕ-ರಣೆ
ಪರಾ-ನು-ಕ-ರ-ಣೆ-ಯಿಂದ
ಪರಾ-ನು-ಕ-ರ-ಣೆಯೇ
ಪರಾಮನ
ಪರಾ-ಮರ್ಶಿಸಿ
ಪರಾ-ಮಾ-ನ-ಸ-ಶಾಸ್ತ್ರ
ಪರಾ-ಮಾ-ನ-ಸ-ಶಾಸ್ತ್ರ-ದಲ್ಲಿ
ಪರಾ-ಯ-ಣತೆ
ಪರಾ-ಯ-ಣ-ರಾಗಿ
ಪರಾ-ಯ-ಣೆ-ಯಾದ
ಪರಾ-ವ-ಲಂಬಿ
ಪರಾ-ವ-ಲಂಬಿ-ಗಳೂ
ಪರಿ
ಪರಿಪೂರ್ಣ
ಪರಿ-ಕ-ರ-ಗಳೇ
ಪರಿ-ಗ-ಣ-ನೆಗೆ
ಪರಿ-ಗ-ಣಿ-ತ-ನಾ-ಗಿದ್ದ
ಪರಿ-ಗ-ಣಿ-ತ-ರಾಗಿ
ಪರಿ-ಗ-ಣಿ-ತ-ರಾ-ಗಿ-ರುವ
ಪರಿ-ಗ-ಣಿ-ತ-ರಾ-ಗುವ
ಪರಿ-ಗ-ಣಿ-ತ-ವಾಗಿ
ಪರಿ-ಗ-ಣಿ-ತ-ವಾದ
ಪರಿ-ಗ-ಣಿ-ಸಲ್ಪಟ್ಟ
ಪರಿ-ಗ-ಣಿಸಿ
ಪರಿ-ಗ-ಣಿ-ಸುತ್ತಾ-ರೆಪ್ರಿಜೊ
ಪರಿಗ್ರ-ಹಿ-ಸು-ವುದು
ಪರಿಗ್ರಾ-ಹಿ-ಯಾ-ಗು-ವಂತೆ
ಪರಿಗ್ರಾಹ್ಯವೂ
ಪರಿಚಯ
ಪರಿ-ಚ-ಯ-ದಿಂದ
ಪರಿ-ಚ-ಯ-ವನ್ನು
ಪರಿ-ಚ-ಯ-ವನ್ನು-ಮುಖ್ಯ-ವಾಗಿ
ಪರಿ-ಚ-ಯ-ವಾ-ಗ-ದಂತೆ
ಪರಿ-ಚ-ಯ-ವಾ-ಗದು
ಪರಿ-ಚ-ಯ-ವಾ-ದ-ವ-ರಿಗೆ
ಪರಿ-ಚ-ಯ-ವಾ-ದಾಗ
ಪರಿ-ಚ-ಯ-ವಿತ್ತೇ
ಪರಿ-ಚ-ಯ-ವಿದ್ದರೂ
ಪರಿ-ಚ-ಯ-ವಿಲ್ಲದ
ಪರಿ-ಚ-ಯವೂ
ಪರಿ-ಚ-ಯವೇ
ಪರಿ-ಚ-ಯಿ-ಸುತ್ತಾನೆ
ಪರಿ-ಚ-ರಿಸಿ
ಪರಿ-ಚ-ರಿ-ಸಿದ್ದೇನೆ
ಪರಿಚರ್ಯೆ
ಪರಿಚಿತ
ಪರಿ-ಚಿ-ತರ
ಪರಿ-ಚಿ-ತರು
ಪರಿ-ಚಿ-ತ-ರೊಬ್ಬರ
ಪರಿ-ಚಿ-ತ-ರೊಬ್ಬ-ರೊ-ಡನೆ
ಪರಿ-ಚಿ-ತ-ವಾ-ಗಿತ್ತೊ
ಪರಿ-ಚಿ-ತ-ವಿಲ್ಲದ
ಪರಿಜ್ಞಾ-ನ-ವಿಲ್ಲದೆ
ಪರಿಣತ
ಪರಿ-ಣ-ತರ
ಪರಿ-ಣ-ತ-ರಾ-ಗಿದ್ದರು
ಪರಿ-ಣ-ತರು
ಪರಿಣತಿ
ಪರಿ-ಣ-ತಿಯ
ಪರಿ-ಣ-ತಿ-ಯನ್ನು
ಪರಿ-ಣ-ತಿ-ಯಲ್ಲಿ
ಪರಿ-ಣ-ಮಿ-ಸ-ಬೇ-ಕಾ-ದರೆ
ಪರಿ-ಣ-ಮಿಸಿ
ಪರಿ-ಣ-ಮಿ-ಸಿತು
ಪರಿ-ಣ-ಮಿ-ಸುತ್ತದೆ
ಪರಿ-ಣ-ಮಿ-ಸುತ್ತಿದೆ
ಪರಿ-ಣ-ಮಿ-ಸುತ್ತಿವೆ
ಪರಿ-ಣ-ಮಿ-ಸು-ವುವು
ಪರಿಣಾಮ
ಪರಿ-ಣಾ-ಮ-ವನ್ನುಂಟು
ಪರಿ-ಣಾ-ಮ-ವೇ-ನಾ-ದೀತು
ಪರಿ-ಣಾ-ಮ-ಎ-ದು-ರಾ-ಳಿಯ
ಪರಿ-ಣಾ-ಮ-ಕಾರಿ
ಪರಿ-ಣಾ-ಮ-ಕಾ-ರಿ-ಯಾಗಿ
ಪರಿ-ಣಾ-ಮ-ಕಾ-ರಿ-ಯಾ-ಗಿ-ರುತ್ತದೆ
ಪರಿ-ಣಾ-ಮ-ಕಾ-ರಿ-ಯಾ-ಗುತ್ತದೆ
ಪರಿ-ಣಾ-ಮ-ಕಾ-ರಿ-ಯಾ-ಗುತ್ತವೆ
ಪರಿ-ಣಾ-ಮ-ಕಾ-ರಿ-ಯಾದ
ಪರಿ-ಣಾ-ಮಕ್ಕೆ
ಪರಿ-ಣಾ-ಮ-ಗಳ
ಪರಿ-ಣಾ-ಮ-ಗ-ಳನ್ನು
ಪರಿ-ಣಾ-ಮ-ಗ-ಳನ್ನೂ
ಪರಿ-ಣಾ-ಮ-ಗಳು
ಪರಿ-ಣಾ-ಮ-ದಲ್ಲಿ
ಪರಿ-ಣಾ-ಮದ್ದಾ-ದರೂ
ಪರಿ-ಣಾ-ಮಪ್ರ-ತಿಕ್ರಿ-ಯೆ-ಗಳು
ಪರಿ-ಣಾ-ಮ-ವನ್ನು
ಪರಿ-ಣಾ-ಮ-ವನ್ನುಂಟು
ಪರಿ-ಣಾ-ಮ-ವನ್ನುಂಟು-ಮಾ-ಡಿತು
ಪರಿ-ಣಾ-ಮ-ವನ್ನುಂಟು-ಮಾ-ಡುವ
ಪರಿ-ಣಾ-ಮ-ವನ್ನೂ
ಪರಿ-ಣಾ-ಮ-ವನ್ನೇ
ಪರಿ-ಣಾ-ಮ-ವಾಗಿ
ಪರಿ-ಣಾ-ಮ-ವಾ-ದೀತು
ಪರಿ-ಣಾ-ಮ-ವಾ-ದೀ-ತೆಂಬು-ದನ್ನು
ಪರಿ-ಣಾ-ಮವು
ಪರಿತಪಿ
ಪರಿ-ತ-ಪಿ-ಸ-ಬೇ-ಕಾ-ಗುತ್ತದೆ
ಪರಿ-ತ-ಪಿ-ಸಿದ
ಪರಿ-ತ-ಪಿ-ಸುತ್ತಿದ್ದರು
ಪರಿ-ತ-ಪಿ-ಸುತ್ತಿದ್ದೆ
ಪರಿತಾಪ
ಪರಿ-ತಾ-ಪ-ಪ-ಡು-ವು-ದುಂಟು
ಪರಿಧಿ
ಪರಿಧಿಗೆ
ಪರಿಧಿಯ
ಪರಿ-ಧಿ-ಯನ್ನು
ಪರಿ-ಧಿ-ಯಲ್ಲಿ
ಪರಿ-ಧಿ-ಯಲ್ಲೇ
ಪರಿ-ಧಿ-ಯೊ-ಳಗೆ
ಪರಿ-ಧಿ-ಯೊ-ಳಗೇ
ಪರಿಪಕ್ವ
ಪರಿ-ಪಕ್ವ-ವಾದ
ಪರಿಪಾಠ
ಪರಿ-ಪಾ-ಠ-ದಿಂದ
ಪರಿ-ಪಾ-ಲನೆ
ಪರಿ-ಪಾ-ಲಿ-ಸ-ಬೇ-ಕಾದ
ಪರಿ-ಪಾ-ಲಿಸಿ
ಪರಿಪೂರ್ಣ
ಪರಿ-ಪೂರ್ಣ-ಗೊ-ಳಿ-ಸುವ
ಪರಿ-ಪೂರ್ಣತೆ
ಪರಿ-ಪೂರ್ಣ-ತೆಗೆ
ಪರಿ-ಪೂರ್ಣ-ತೆಯ
ಪರಿ-ಪೂರ್ಣ-ತೆ-ಯನ್ನು
ಪರಿ-ಪೂರ್ಣ-ತೆ-ಯಲ್ಲಿ
ಪರಿ-ಪೂರ್ಣ-ತೆ-ಯೆ-ಡೆಗೆ
ಪರಿ-ಪೂರ್ಣ-ನಾ-ಗಿದ್ದಾನೆ
ಪರಿ-ಪೂರ್ಣ-ರಲ್ಲ-ವೆಂಬು-ದೇನೋ
ಪರಿ-ಪೂರ್ಣ-ವಲ್ಲ-ಆಂಶಿಕ
ಪರಿ-ಪೂರ್ಣ-ವಾಗಿ
ಪರಿ-ಪೂರ್ಣ-ವಾದ
ಪರಿ-ಪೋ-ಷಣೆ
ಪರಿ-ಭಾ-ವಿ-ಸಿ-ದಾಗ
ಪರಿಭ್ರ-ಮಿ-ಸು-ವಂತೆ
ಪರಿ-ಮ-ಳವು
ಪರಿ-ಮ-ಳಿ-ಸಿ-ದಂತೆ
ಪರಿಮಾಣ
ಪರಿ-ಮಾ-ಣದ
ಪರಿಮಿತ
ಪರಿ-ಮಿ-ತ-ಗೊ-ಳಿಸಿ
ಪರಿ-ಮಿ-ತ-ಗೊ-ಳಿ-ಸಿ-ದೆ-ಯಾ-ದರೂ
ಪರಿ-ಮಿ-ತ-ಗೊ-ಳಿ-ಸುತ್ತದೆ
ಪರಿ-ಮಿ-ತ-ಗೊ-ಳಿ-ಸು-ವುದು
ಪರಿಮಿತಿ
ಪರಿ-ಮಿ-ತಿ-ಗ-ಳನ್ನು
ಪರಿ-ಮಿ-ತಿ-ಗ-ಳಾ-ಗಲಿ
ಪರಿ-ಮಿ-ತಿ-ಗ-ಳಾದ
ಪರಿ-ಮಿ-ತಿ-ಗ-ಳಿಂದ
ಪರಿ-ಮಿ-ತಿ-ಗಳು
ಪರಿ-ಮಿ-ತಿ-ಗೊ-ಳಿ-ಸಿ-ಕೊಳ್ಳುತ್ತದೆ
ಪರಿ-ಮಿ-ತಿಯ
ಪರಿ-ಮಿ-ತಿ-ಯನ್ನು
ಪರಿ-ಮಿ-ತಿ-ಯಿಂದ
ಪರಿ-ಮಿ-ತಿ-ಯಿಂದುಂಟಾದ
ಪರಿ-ಮಿ-ತಿ-ಯಿದ್ದು
ಪರಿ-ವರ್ತ-ನ-ಶೀ-ಲವೂ
ಪರಿ-ವರ್ತನೆ
ಪರಿ-ವರ್ತ-ನೆ-ಗೊಂಡ
ಪರಿ-ವರ್ತ-ನೆಯ
ಪರಿ-ವರ್ತ-ನೆ-ಯನ್ನೂ
ಪರಿ-ವರ್ತ-ನೆ-ಯಾ-ಗದೇ
ಪರಿ-ವರ್ತ-ನೆ-ಯಾದ
ಪರಿ-ವರ್ತ-ನೆ-ಯಾ-ದು-ದನ್ನು
ಪರಿ-ವರ್ತ-ನೆ-ಯಿಲ್ಲದೆ
ಪರಿವರ್ತಿ
ಪರಿ-ವರ್ತಿ-ತ-ನಾಗ
ಪರಿ-ವರ್ತಿ-ತ-ರಾಗಿ
ಪರಿ-ವರ್ತಿ-ತ-ರಾದ
ಪರಿ-ವರ್ತಿ-ತ-ವಾ-ದರೆ
ಪರಿ-ವರ್ತಿ-ಸ-ಬ-ಹುದು
ಪರಿ-ವರ್ತಿ-ಸ-ಬ-ಹು-ದೆಂಬು-ದನ್ನು
ಪರಿ-ವರ್ತಿ-ಸಲು
ಪರಿ-ವರ್ತಿಸಿ
ಪರಿ-ವರ್ತಿ-ಸಿ-ಕೊಳ್ಳ-ಬ-ಹುದು
ಪರಿ-ವರ್ತಿ-ಸಿದ
ಪರಿ-ವರ್ತಿ-ಸಿ-ದರೆ
ಪರಿ-ವಾ-ರ-ದ-ವರೂ
ಪರಿವೆಯೇ
ಪರಿವ್ರಾ-ಜಕ
ಪರಿವ್ರಾ-ಜ-ಕ-ರಾಗಿ
ಪರಿ-ಶೀ-ಲ-ಕ-ನಿಗೆ
ಪರಿ-ಶೀ-ಲ-ಕರೂ
ಪರಿ-ಶೀ-ಲನೆ
ಪರಿ-ಶೀ-ಲ-ನೆ-ಇ-ವು-ಗ-ಳಿಂದ
ಪರಿ-ಶೀ-ಲ-ನೆ-ಗಳ
ಪರಿ-ಶೀ-ಲ-ನೆ-ಗ-ಳನ್ನೂ
ಪರಿ-ಶೀ-ಲ-ನೆ-ಗ-ಳಿಂದ
ಪರಿ-ಶೀ-ಲ-ನೆ-ಗಾಗಿ
ಪರಿ-ಶೀ-ಲ-ನೆ-ಯಂತೆ
ಪರಿ-ಶೀ-ಲಿ-ಸ-ಬಲ್ಲ
ಪರಿ-ಶೀ-ಲಿ-ಸ-ಬ-ಹುದು
ಪರಿ-ಶೀ-ಲಿ-ಸ-ಬಾರ
ಪರಿ-ಶೀ-ಲಿ-ಸ-ಬೇಕು
ಪರಿ-ಶೀ-ಲಿ-ಸಲಿ
ಪರಿ-ಶೀ-ಲಿ-ಸಲು
ಪರಿ-ಶೀ-ಲಿ-ಸಲೇ
ಪರಿ-ಶೀ-ಲಿ-ಸಲ್ಪಟ್ಟ
ಪರಿ-ಶೀ-ಲಿಸಿ
ಪರಿ-ಶೀ-ಲಿ-ಸಿದ
ಪರಿ-ಶೀ-ಲಿ-ಸಿ-ದರು
ಪರಿ-ಶೀ-ಲಿ-ಸಿ-ದರೆ
ಪರಿ-ಶೀ-ಲಿ-ಸಿ-ದ-ವನು
ಪರಿ-ಶೀ-ಲಿ-ಸಿ-ದಾಗ
ಪರಿ-ಶೀ-ಲಿ-ಸಿದೆ
ಪರಿ-ಶೀ-ಲಿ-ಸಿ-ದೆವು
ಪರಿ-ಶೀ-ಲಿ-ಸಿದ್ದರು
ಪರಿ-ಶೀ-ಲಿ-ಸಿದ್ದಾರೆ
ಪರಿ-ಶೀ-ಲಿ-ಸಿದ್ದೀರಾ
ಪರಿ-ಶೀ-ಲಿ-ಸುವ
ಪರಿ-ಶೀ-ಲಿ-ಸು-ವು-ದಿಲ್ಲ
ಪರಿ-ಶೀ-ಲಿ-ಸೋಣ
ಪರಿ-ಶುದ್ದಪ್ರೀ-ತಿ-ಯನ್ನು
ಪರಿಶುದ್ಧ
ಪರಿ-ಶುದ್ಧ-ಗೊ-ಳಿ-ಸುತ್ತಾನೆ
ಪರಿ-ಶುದ್ಧ-ಗೊ-ಳಿ-ಸು-ವು-ದಕ್ಕಾ-ಗಿಯೇ
ಪರಿ-ಶುದ್ಧ-ತೆ-ಗಾಗಿ
ಪರಿ-ಶುದ್ಧ-ತೆಗೆ
ಪರಿ-ಶುದ್ಧಪ್ರೇಮ
ಪರಿ-ಶುದ್ಧ-ರಾದ
ಪರಿ-ಶುದ್ಧ-ರೆಂದು
ಪರಿ-ಶುದ್ಧರೇ
ಪರಿ-ಶುದ್ಧ-ವಾ-ಗಿದೆ
ಪರಿ-ಶುದ್ಧ-ವಾ-ಗುತ್ತಾ
ಪರಿ-ಶುದ್ಧ-ವಾ-ಗು-ವುದು
ಪರಿ-ಶೋ-ಧನೆ
ಪರಿ-ಶೋ-ಧ-ನೆ-ಯಾ-ಗ-ಲಿಲ್ಲ
ಪರಿ-ಶೋ-ಧಿ-ಸಿ-ದಾಗ
ಪರಿಶ್ರಮ
ಪರಿಶ್ರ-ಮ-ಇವು
ಪರಿಶ್ರ-ಮಕ್ಕೆ
ಪರಿಶ್ರ-ಮಕ್ಕೊಂದು
ಪರಿಶ್ರ-ಮದ
ಪರಿಶ್ರ-ಮ-ದಿಂದ
ಪರಿಶ್ರ-ಮ-ಪ-ಡು-ವ-ವರು
ಪರಿಶ್ರ-ಮ-ವಿಲ್ಲದೆ
ಪರಿಷ್ಕ-ರಿ-ಸಲು
ಪರಿಷ್ಕೃತ
ಪರಿಷ್ಕೃ-ತ-ವಾದ
ಪರಿಸರ
ಪರಿ-ಸ-ರ-ಇವು
ಪರಿ-ಸ-ರ-ಇ-ವು-ಗಳ
ಪರಿ-ಸ-ರಕ್ಕೆ
ಪರಿ-ಸ-ರ-ಗಳ
ಪರಿ-ಸ-ರ-ಗ-ಳಲ್ಲಿ
ಪರಿ-ಸ-ರ-ಗ-ಳಿಂದ
ಪರಿ-ಸ-ರದ
ಪರಿ-ಸ-ರ-ದಲ್ಲಿ
ಪರಿ-ಸ-ರ-ದಿಂದ
ಪರಿ-ಸ-ರ-ವನ್ನೇ
ಪರಿ-ಸ-ರವು
ಪರಿ-ಸ-ರ-ಹೀಗೆ
ಪರಿಸ್ಥಿತಿ
ಪರಿಸ್ಥಿ-ತಿ-ಯಲ್ಲೂ
ಪರಿಸ್ಥಿ-ತಿ-ಗಳ
ಪರಿಸ್ಥಿ-ತಿ-ಗ-ಳಲ್ಲಿ
ಪರಿಸ್ಥಿ-ತಿ-ಗ-ಳಲ್ಲೂ
ಪರಿಸ್ಥಿ-ತಿ-ಗ-ಳಿಂದ
ಪರಿಸ್ಥಿ-ತಿಗೆ
ಪರಿಸ್ಥಿ-ತಿಯ
ಪರಿಸ್ಥಿ-ತಿ-ಯನ್ನು
ಪರಿಸ್ಥಿ-ತಿ-ಯನ್ನೂ
ಪರಿಸ್ಥಿ-ತಿ-ಯನ್ನೇ
ಪರಿಸ್ಥಿ-ತಿ-ಯಲ್ಲಿ
ಪರಿಸ್ಥಿ-ತಿ-ಯಲ್ಲಿದ್ದರೂ
ಪರಿಸ್ಥಿ-ತಿ-ಯಲ್ಲಿದ್ದಾಗ
ಪರಿಸ್ಥಿ-ತಿ-ಯಲ್ಲಿ-ರಲಿ
ಪರಿಸ್ಥಿ-ತಿ-ಯಲ್ಲೇ
ಪರಿಸ್ಥಿ-ತಿ-ಯಿಂದ
ಪರಿಸ್ಥಿ-ತಿ-ಯುಂಟಾ-ಗ-ಬ-ಹುದು
ಪರಿಸ್ಥಿ-ತಿ-ಯುಂಟಾಗಿ
ಪರಿಸ್ಥಿ-ತಿ-ಯೊಂದಿಗೆ
ಪರಿ-ಹ-ರಿ-ಸ-ಬಲ್ಲ
ಪರಿ-ಹ-ರಿ-ಸ-ಬಲ್ಲರು
ಪರಿ-ಹ-ರಿ-ಸ-ಬ-ಹುದು
ಪರಿ-ಹ-ರಿ-ಸ-ಲಾ-ಗ-ದು-ದನ್ನು
ಪರಿ-ಹ-ರಿ-ಸ-ಲಾ-ಗ-ದೆಂದು
ಪರಿ-ಹ-ರಿ-ಸ-ಲಾ-ಗು-ವು-ದಿಲ್ಲ
ಪರಿ-ಹ-ರಿ-ಸಲು
ಪರಿ-ಹ-ರಿಸಿ
ಪರಿ-ಹ-ರಿ-ಸಿ-ಕೊಳ್ಳ-ಬೇ-ಕಲ್ಲವೇ
ಪರಿ-ಹ-ರಿ-ಸಿ-ಕೊಳ್ಳಲು
ಪರಿ-ಹ-ರಿ-ಸಿ-ಕೊಳ್ಳು-ವು-ದ-ರಲ್ಲೇ
ಪರಿ-ಹ-ರಿ-ಸುತ್ತದೆ
ಪರಿ-ಹ-ರಿ-ಸು-ವು-ದಿಲ್ಲ
ಪರಿ-ಹ-ರಿ-ಸು-ವುದೂ
ಪರಿಹಾರ
ಪರಿ-ಹಾ-ರ-ದಲ್ಲಿ
ಪರಿ-ಹಾ-ರಕ್ಕಾಗಿ
ಪರಿ-ಹಾ-ರ-ಗಳು
ಪರಿ-ಹಾ-ರ-ವನ್ನು
ಪರಿ-ಹಾ-ರ-ವಾ-ಗ-ದಿದ್ದರೂ
ಪರಿ-ಹಾ-ರ-ವಾಗಿ
ಪರಿ-ಹಾ-ರ-ವಾ-ಗುತ್ತದೆ
ಪರಿ-ಹಾ-ರ-ವಾ-ಯಿತು
ಪರಿ-ಹಾ-ರ-ವಿದೆ
ಪರಿ-ಹಾ-ರ-ವಿ-ದೆಯೆ
ಪರಿ-ಹಾ-ರವೇ
ಪರಿ-ಹಾ-ರ-ವೇ-ನಾ-ದರೂ
ಪರಿ-ಹಾ-ರ-ವೇ-ನೆಂಬು-ದನ್ನು
ಪರಿ-ಹಾ-ರೋ-ಪಾಯ
ಪರಿ-ಹಾ-ರೋ-ಪಾ-ಯವೇ
ಪರಿಹಾಸ್ಯ
ಪರೀಕ್ಷ-ಕ-ರಲ್ಲಿ
ಪರೀಕ್ಷ-ಣ-ಗ-ಳಿಂದ
ಪರೀಕ್ಷ-ಣೆ-ಗೊ-ಳ-ಪ-ಡ-ಲಾ-ರದ
ಪರೀಕ್ಷ-ಣೆ-ಗ-ಳಿಗೆ
ಪರೀಕ್ಷ-ಣೆ-ಗಳು
ಪರೀಕ್ಷ-ಣೆ-ಗ-ಳೇನು
ಪರೀಕ್ಷಾ
ಪರೀಕ್ಷಾ-ಧಿ-ಕಾ-ರಿ-ಗ-ಳನ್ನು
ಪರೀಕ್ಷಾ-ಭ-ಯ-ದಿಂದ
ಪರೀಕ್ಷಾ-ಭ-ವ-ನ-ದಿಂದ
ಪರೀಕ್ಷಿ-ತ-ನಾಗಿ
ಪರೀಕ್ಷಿ-ಸಲು
ಪರೀಕ್ಷಿಸಿ
ಪರೀಕ್ಷಿ-ಸಿದ
ಪರೀಕ್ಷಿ-ಸಿ-ದಾಗ
ಪರೀಕ್ಷಿಸು
ಪರೀಕ್ಷಿ-ಸು-ವೆ-ನೆಂದು-ಕೊಂಡರು
ಪರೀಕ್ಷೆ
ಪರೀಕ್ಷೆ-ಗ-ಳನ್ನು
ಪರೀಕ್ಷೆ-ಗ-ಳಲ್ಲಿ
ಪರೀಕ್ಷೆ-ಗ-ಳಿಂದ
ಪರೀಕ್ಷೆ-ಗಳು
ಪರೀಕ್ಷೆ-ಗಾಗಿ
ಪರೀಕ್ಷೆಗೆ
ಪರೀಕ್ಷೆ-ಗೊ-ಳ-ಪ-ಡಿ-ಸಿ-ದರು
ಪರೀಕ್ಷೆಯ
ಪರೀಕ್ಷೆ-ಯಲ್ಲಿ
ಪರೀಕ್ಷೆ-ಯಾ-ಗಿ-ರುತ್ತದೆ
ಪರೆ
ಪರೋಕ್ಷ-ದಲ್ಲಿ
ಪರೋಕ್ಷ-ವಾಗಿ
ಪರೋಕ್ಷ-ವಾ-ಗಿಯೋ
ಪರೋ-ಪ-ಕಾ-ರ-ದಲ್ಲೇ
ಪರೋ-ಪ-ಕಾರ
ಪರೋ-ಪ-ಕಾ-ರ-ಬುದ್ಧಿ
ಪರೋ-ಪ-ಕಾ-ರಿ-ಗಳೂ
ಪರೋ-ಪ-ಜೀ-ವಿ-ಯಾ-ಗುತ್ತಿದ್ದಾ-ನಲ್ಲ
ಪರ್ಯ-ವ-ಸಾನ
ಪರ್ಯ-ವ-ಸಾ-ನ-ವಾ-ಗುವ
ಪರ್ಯ-ವ-ಸಾ-ನ-ವಾ-ಗು-ವು-ದಿಲ್ಲ
ಪರ್ಯ-ವ-ಸಾ-ನ-ವಾ-ಗು-ವುದು
ಪರ್ವತ
ಪರ್ವ-ತ-ಗಳೂ
ಪರ್ವತದ
ಪರ್ವ-ತ-ವಾ-ಸಿ-ಗ-ಳಾದ
ಪರ್ವತವೇ
ಪರ್ವ-ತ-ಶಿ-ಖ-ರ-ವಾ-ಗಿತ್ತು
ಪರ್ವ-ತಾಗ್ರಕ್ಕೆ
ಪರ್ವ-ತಾಗ್ರ-ದಲ್ಲಿನ
ಪರ್ವವನ್ನೆ
ಪರ್ಶಿಯನ್
ಪರ್ಶಿಯಾ
ಪರ್ಷಿಯನ್
ಪರ್ಸೆಂಟ್
ಪಲಾಯನ
ಪಲಾ-ಯ-ನ-ವಾ-ದಿ-ಗ-ಳೆನ್ನ-ಬೇಕೆ
ಪಲಾ-ಯ-ನ-ಸೂತ್ರ
ಪಲ್ಲಕ್ಕಿಗೆ
ಪಲ್ಲಕ್ಕಿ-ಯಲ್ಲಿ-ರಲಿ
ಪಲ್ಲಕ್ಕಿ-ಯಲ್ಲಿ
ಪಲ್ಲವಿತ
ಪಲ್ಲ-ವಿ-ಯನ್ನು
ಪಳ-ಗಿ-ದರೆ
ಪಳಗಿದ್ದು
ಪಳ-ಗಿ-ಸಿ-ದರು
ಪಳ-ಗಿ-ಸುವ
ಪಳೆ-ಯು-ಳಿಕೆ
ಪವಾಡ
ಪವಾಡದ
ಪವಾ-ಡ-ದಂಥ
ಪವಾ-ಡ-ಪು-ರು-ಷ-ನಿಂದ
ಪವಾ-ಡ-ವನ್ನು
ಪವಾ-ಡ-ವೆ-ನಿ-ಸಿ-ದರೂ
ಪವಾ-ಡ-ಶಕ್ತಿ-ಯಾ-ಗಿ-ರ-ಲಿಲ್ಲ
ಪವಾ-ಡ-ಸ-ದೃಶ
ಪವಿತ್ರ
ಪವಿತ್ರ-ಗೊ-ಳಿ-ಸು-ವುದು
ಪವಿತ್ರಗ್ರಂಥ
ಪವಿತ್ರಗ್ರಂಥ-ಗ-ಳಲ್ಲೂ
ಪವಿತ್ರಗ್ರಂಥ-ಗ-ಳಿಗೆ
ಪವಿತ್ರತೆ
ಪವಿತ್ರ-ತೆಯ
ಪವಿತ್ರ-ತೆಯೂ
ಪವಿತ್ರ-ಭಾ-ವ-ನೆ-ಯಿಂದ
ಪವಿತ್ರವೂ
ಪಶು
ಪಶುಪಕ್ಷಿ
ಪಶು-ಪಕ್ಷಿ-ಗಳ
ಪಶು-ಪಕ್ಷಿ-ಗಳೂ
ಪಶು-ವಿ-ಗಿಂತಲೂ
ಪಶು-ಸ-ದೃ-ಶ-ರಾ-ಗಿ-ರು-ವರು
ಪಶುಸ್ವ-ಭಾವಿ
ಪಶ್ಚಾತ್ತಾಪ
ಪಶ್ಚಾತ್ತಾ-ಪಕ್ಕೆ
ಪಶ್ಚಾತ್ತಾ-ಪದ
ಪಶ್ಚಾತ್ತಾ-ಪ-ದಿಂದ
ಪಶ್ಚಾತ್ತಾ-ಪ-ಪಟ್ಟು
ಪಶ್ಚಾತ್ತಾ-ಪ-ಪ-ಡದೆ
ಪಶ್ಚಿಮ
ಪಶ್ಚಿಮದ
ಪಶ್ಚಿ-ಮ-ದಲ್ಲಿ
ಪಶ್ಚಿ-ಮ-ದೆಡೆ
ಪಶ್ಚಿ-ಮ-ದೇಶ
ಪಶ್ಚಿ-ಮ-ದೇ-ಶ-ಗಳ
ಪಶ್ಚಿ-ಮ-ದೇ-ಶದ
ಪಶ್ಚಿ-ಮ-ದೇ-ಶ-ದಲ್ಲಿ
ಪಶ್ಚಿ-ಮ-ವನ್ನು
ಪಸ-ರಿ-ಸಲು
ಪಾಂಡಿತ್ಯಕ್ಕೊಂದು
ಪಾಂಡಿತ್ಯದ
ಪಾಂಡಿತ್ಯ-ದಿಂದ
ಪಾಂಡಿತ್ಯ-ಪೂರ್ಣ
ಪಾಂಡಿತ್ಯ-ವನ್ನು
ಪಾಂಡುರಂಗ
ಪಾಕ-ಶಾ-ಲೆಯ
ಪಾಕ-ಶಾ-ಲೆ-ಯಲ್ಲಿ
ಪಾಚಿಯನ್ನು
ಪಾಠ
ಪಾಠಕ್ಕಾಗಿ
ಪಾಠ-ಗ-ಳನ್ನು
ಪಾಠ-ಗ-ಳಲ್ಲೊಂದು
ಪಾಠಗಳು
ಪಾಠ-ಪಟ್ಟಿ-ಯಲ್ಲಿ
ಪಾಠಪ್ರ-ವ-ಚನ
ಪಾಠಪ್ರ-ವ-ಚ-ನ-ಗ-ಳನ್ನು
ಪಾಠವನ್ನು
ಪಾಠವನ್ನೂ
ಪಾಠವನ್ನೇ
ಪಾಡಲು
ಪಾಡಾದರೆ
ಪಾಡಿಗಾಗಿ
ಪಾಡಿಗೆ
ಪಾಡಿಗೇ
ಪಾಡು
ಪಾಡೇನು
ಪಾತ-ಕ-ಗ-ಳಿಲ್ಲ
ಪಾತ-ಕಿ-ಗ-ಳನ್ನು
ಪಾತಿವ್ರತ್ಯಕ್ಕೆ
ಪಾತಿವ್ರತ್ಯದ
ಪಾತ್ರ
ಪಾತ್ರ-ಗ-ಳನ್ನು
ಪಾತ್ರದ
ಪಾತ್ರನಾಗಿ
ಪಾತ್ರನಾದ
ಪಾತ್ರ-ನಾ-ದ-ವನ
ಪಾತ್ರನು
ಪಾತ್ರನೆ
ಪಾತ್ರರ
ಪಾತ್ರ-ರಾ-ಗು-ವರು
ಪಾತ್ರ-ರಾ-ದರು
ಪಾತ್ರರೇ
ಪಾತ್ರವನ್ನು
ಪಾತ್ರ-ವ-ಹಿ-ಸಿದ್ದರೂ
ಪಾತ್ರ-ವ-ಹಿ-ಸುತ್ತದೆ
ಪಾತ್ರ-ವ-ಹಿ-ಸುತ್ತವೆ
ಪಾತ್ರ-ವಾ-ಗುತ್ತಿ-ರು-ವುದು
ಪಾತ್ರವೇನು
ಪಾತ್ರೆ
ಪಾತ್ರೆಗೆ
ಪಾತ್ರೆಯನ್ನು
ಪಾದಕ್ಕೆ
ಪಾದ-ಗ-ಳಲ್ಲಿ
ಪಾದದ
ಪಾದದಲ್ಲಿ
ಪಾದದಿಂದ
ಪಾದರಕ್ಷೆ
ಪಾದ-ರಕ್ಷೆ-ಗ-ಳನ್ನು
ಪಾದಸ್ಪರ್ಶ-ವನ್ನು
ಪಾದಾ-ರ-ವಿಂದ-ವಲ್ಲದೆ
ಪಾದು-ಕೆ-ಗ-ಳನ್ನು
ಪಾದ್ರಿ
ಪಾದ್ರಿ-ಗ-ಳಿಂದ
ಪಾದ್ರಿಗಳು
ಪಾದ್ರಿ-ಗ-ಳು-ಇ-ವ-ರೆಲ್ಲ-ರಿಗೆ
ಪಾದ್ರಿಗೆ
ಪಾದ್ರಿ-ಯ-ವರು
ಪಾದ್ರಿ-ಯಾ-ಗು-ವಂತೆ
ಪಾದ್ರಿಯು
ಪಾದ್ರಿ-ಯೊಬ್ಬರು
ಪಾನ
ಪಾನಪೀಠ
ಪಾನೀ
ಪಾನೀಯ
ಪಾನೀ-ಯ-ಗ-ಳಾ-ಗಲೀ
ಪಾಪ
ಪಾಪ-ಕರ್ಮವೂ
ಪಾಪ-ಕಾರ್ಯ-ವನ್ನೂ
ಪಾಪ-ಕಾರ್ಯ-ವಾ-ಗುತ್ತದೆ
ಪಾಪಕೃತ್ಯ
ಪಾಪ-ಕೃತ್ಯ-ಗ-ಳನ್ನು
ಪಾಪ-ಕೃತ್ಯ-ಗ-ಳಿಂದ
ಪಾಪ-ಕೃತ್ಯ-ಗ-ಳಿಗೂ
ಪಾಪಕ್ಕೆ
ಪಾಪದ
ಪಾಪ-ಪ-ರಾ-ಯಣ
ಪಾಪ-ಪುಣ್ಯದ
ಪಾಪಪ್ರಜ್ಞೆ
ಪಾಪ-ಶೀ-ಲ-ರಾ-ಗುತ್ತಾರೆ
ಪಾಪಿ
ಪಾಪಿಗಳು
ಪಾಪಿ-ಗ-ಳೆಂದೂ
ಪಾಪಿಯೆಂದೂ
ಪಾಮ-ರ-ರನ್ನು
ಪಾಮರರು
ಪಾರ-ತಂತ್ರ್ಯದ
ಪಾರ-ತಂತ್ರ್ಯ-ದಿಂದ
ಪಾರ-ಮಾರ್ಥಿಕ
ಪಾರ-ವಿ-ದೆಯೇ
ಪಾರಾಗಲು
ಪಾರಾಗ
ಪಾರಾ-ಗ-ದಿದ್ದರೆ
ಪಾರಾ-ಗ-ಬಲ್ಲನು
ಪಾರಾ-ಗ-ಬ-ಹುದು
ಪಾರಾ-ಗ-ಬ-ಹು-ದೆಂಬು-ದನ್ನು
ಪಾರಾಗಲು
ಪಾರಾಗಿ
ಪಾರಾ-ಗಿದ್ದಾರೆ
ಪಾರಾ-ಗುತ್ತದೆ
ಪಾರಾ-ಗುತ್ತಾನೆ
ಪಾರಾ-ಗುತ್ತೇವೆ
ಪಾರಾಗುವ
ಪಾರಾ-ಗು-ವು-ದಕ್ಕಾಗಿ
ಪಾರಾ-ಗು-ವು-ದಕ್ಕೆ
ಪಾರಾ-ಗು-ವುದು
ಪಾರಾದ
ಪಾರಾ-ದ-ವ-ರಿದ್ದಾರೆ
ಪಾರಾಯಣ
ಪಾರಾ-ಯ-ಣ-ಗ-ಳಿಂದಲೂ
ಪಾರಾ-ಯ-ಣ-ವನ್ನು
ಪಾರಾ-ವಾ-ರ-ವಿಲ್ಲದ
ಪಾರಿ-ತೋ-ಷಕ
ಪಾರಿ-ತೋ-ಷ-ಕ-ವನ್ನು
ಪಾರಿ-ಭಾ-ಷಿಕ
ಪಾರು
ಪಾರುಪತ್ಯ
ಪಾರು-ಮಾ-ಡಲು
ಪಾರು-ಮಾ-ಡಿದ್ದೇನೆ
ಪಾರುಮಾಡು
ಪಾರು-ಮಾ-ಡೆಂದು
ಪಾರ್ಕಿಗೆ
ಪಾರ್ಕಿನಲ್ಲಿ
ಪಾರ್ಲಿ-ಮೆಂಟಿ-ನಲ್ಲಿ
ಪಾರ್ಲಿಮೆಂಟು
ಪಾರ್ವ
ಪಾರ್ಶ್ವ-ವಾ-ಯು-ವಿ-ನಿಂದ
ಪಾಲ
ಪಾಲನೆ
ಪಾಲ-ನೆ-ಗ-ಳಿಂದ
ಪಾಲ-ನೆ-ಯಿಂದ
ಪಾಲಾಗಿ
ಪಾಲಿ-ಗಿ-ದೆಯೇ
ಪಾಲಿ-ಗಿ-ರಲಿ
ಪಾಲಿಗೂ
ಪಾಲಿಗೆ
ಪಾಲಿನ
ಪಾಲಿಶ್
ಪಾಲಿಸದ
ಪಾಲಿ-ಸ-ದಿದ್ದರೆ
ಪಾಲಿ-ಸ-ಬೇಕು
ಪಾಲಿಸಲು
ಪಾಲಿಸಿ
ಪಾಲಿ-ಸಿ-ದರೆ
ಪಾಲಿ-ಸಿ-ದಳು
ಪಾಲಿಸುತ್ತ
ಪಾಲಿ-ಸುತ್ತ-ಲಿ-ರು-ವುದೇ
ಪಾಲಿಸುವ
ಪಾಲಿ-ಸು-ವ-ವರು
ಪಾಲಿ-ಸು-ವುದೇ
ಪಾಲು
ಪಾಲು-ಗಾ-ರ-ರೆಂದೂ
ಪಾಲು-ಗೊಳ್ಳುವ
ಪಾಲು-ದಾ-ರ-ರಾ-ಗಿದ್ದಾರೆ
ಪಾಲು-ವಿ-ಷದ
ಪಾಲೂ
ಪಾಲೆಂದಿಟ್ಟು-ಕೊಂಡರೆ
ಪಾಲೋ
ಪಾಲ್
ಪಾಲ್ಗೊಂಡಿದ್ದೇನೆ
ಪಾವಿತ್ರ್ಯ
ಪಾವಿತ್ರ್ಯಕ್ಕೆ
ಪಾವಿತ್ರ್ಯ-ವನ್ನ-ನು-ಸ-ರಿಸಿ
ಪಾವೆಲ್
ಪಾವ್ಲೋವ್
ಪಾವ್ಲೋವ್ನ
ಪಾಶ
ಪಾಶಕ್ಕೊ-ಳ-ಗಾದ
ಪಾಶದ
ಪಾಶವಿಕ
ಪಾಶ-ವಿ-ಕವೂ
ಪಾಶವು
ಪಾಶ್ಚಾತ್ಯ
ಪಾಶ್ಚಾತ್ಯ-ದೇ-ಶ-ಗ-ಳಲ್ಲಿನ
ಪಾಸಾ
ಪಾಸಾಗದ
ಪಾಸಾ-ಗ-ಬಲ್ಲರು
ಪಾಸಾಗಿ
ಪಾಸಾ-ಗಿ-ರ-ಲಿಲ್ಲ
ಪಾಸಾ-ಗು-ವು-ದು-ಇ-ವು-ಗಳ
ಪಾಸಾದ
ಪಾಸಾದದ್ದು
ಪಾಸಾದರು
ಪಾಸು
ಪಾಸುಮಾಡಿ
ಪಾಸು-ಮಾ-ಡಿ-ಸಲು
ಪಾಸ್ಕಲನ
ಪಾಸ್ಕಲನ್ಣ
ಪಾಹಿ
ಪಿಕಾ-ಸಿ-ಯಿಂದ
ಪಿಚ್
ಪಿಟೀಲಿನ
ಪಿಟೀಲು
ಪಿಟ್
ಪಿಟ್ರಿಮ್
ಪಿಡಬ್ಲ್ಯುಡಿ
ಪಿಡುಗನ್ನು
ಪಿಡು-ಗಿ-ನಿಂದ
ಪಿತ-ನೆ-ನಿ-ಸಿದ
ಪಿತಾಮಹ
ಪಿತೃ
ಪಿತೃ-ವಿ-ಯೋ-ಗದ
ಪಿತ್ರಾರ್ಜಿತ
ಪಿಪಾ-ಸು-ವಿಗೆ
ಪಿಪಾ-ಸೆ-ಗಳ
ಪಿಯಾನೊ
ಪಿಯುಸಿ
ಪಿರ-ಮಿಡ್ಡು-ಗ-ಳಲ್ಲಿ-ರುವ
ಪಿಳ್ಳೆ
ಪಿಶಾಚಿ
ಪಿಶಾ-ಚಿ-ಗಳ
ಪಿಶಾಚಿಗೂ
ಪಿಶಾಚಿಗೆ
ಪಿಶಾಚಿಯ
ಪಿಶಾ-ಚಿ-ಯನ್ನು
ಪಿಶಾ-ಚಿ-ಯಾದೆ
ಪಿಷ್ಟಾಂಶವು
ಪಿಸ್ತೂಲನ್ನು
ಪಿಸ್ತೂಲು
ಪೀಠಿಕೆ
ಪೀಠಿ-ಕೆ-ಯಲ್ಲಿ
ಪೀಡಿತ
ಪೀಡಿ-ತ-ರಾಗಿ
ಪೀಡಿ-ತ-ರಾದ
ಪೀಡಿ-ತ-ರೆಂದರೆ
ಪೀಡಿ-ತ-ಳಾ-ದಳು
ಪೀಡಿ-ಸ-ಲಿಲ್ಲವೇ
ಪೀಡಿಸಿತು
ಪೀಡಿ-ಸಿ-ದ-ವನು
ಪೀಡಿ-ಸಿ-ದ-ವ-ರಿದ್ದರು
ಪೀಡಿ-ಸುತ್ತಾ-ರೆಂದು
ಪೀಡಿ-ಸು-ವು-ದಕ್ಕಾಗಿ
ಪೀಡಿ-ಸು-ವು-ದ-ರಿಂದ
ಪೀಡೆ
ಪೀಡೆ-ಗ-ಳನ್ನೇ
ಪೀಲ್
ಪೀಳಿಗೆ
ಪೀಳಿಗೆಗೆ
ಪೀಳಿ-ಗೆ-ಯಲ್ಲಿ
ಪು
ಪುಂಖಾ-ನು-ಪುಂಖ-ವಾಗಿ
ಪುಜು-ಜೀ-ವಿ-ಗಳು
ಪುಜು-ಮಾರ್ಗ-ಗ-ಳಿಂದ
ಪುಟ
ಪುಟಗಳ
ಪುಟ-ಗ-ಳಲ್ಲಿ
ಪುಟ-ಗ-ಳಷ್ಟು
ಪುಟ-ಗೊಟ್ಟೀತು
ಪುಟ-ಗೊಳ್ಳಲು
ಪುಟ-ಗೊಳ್ಳ-ಲೆಂದು
ಪುಟ-ಗೊಳ್ಳುತ್ತದೆ
ಪುಟ-ವಿಟ್ಟಂತಾ-ಗು-ವುದು
ಪುಟಿ-ದೇ-ಳುತ್ತದೆ
ಪುಟಿ-ದೇ-ಳುವ
ಪುಟ್ಟ
ಪುಟ್ಟ-ಕಾ-ಣಿಕೆ
ಪುಟ್ಟ-ಕಾ-ಲು-ಗ-ಳಿಂದ
ಪುಟ್ಟದೇಶ
ಪುಟ್ಟಪುಟ್ಟ
ಪುಟ್ಟಶಿಶು
ಪುಣದಿಂದ
ಪುಣಾ-ನು-ಬಂಧ-ದಿಂದ
ಪುಣೆಯ
ಪುಣ್ಯ
ಪುಣ್ಯ-ಕಟ್ಟಿ-ಕೊಂಡು
ಪುಣ್ಯ-ಕ-ತೆ-ಗಳು
ಪುಣ್ಯ-ಕಾರ್ಯ-ದಿಂದೊ-ದ-ಗುವ
ಪುಣ್ಯಕ್ಷೇತ್ರ-ಗ-ಳನ್ನು
ಪುಣ್ಯವಂತ
ಪುಣ್ಯ-ವಿ-ಶೇ-ಷ-ದಿಂದ
ಪುಣ್ಯ-ಸಂಪಾ-ದಿ-ಸಿ-ಕೊಂಡಳು
ಪುಣ್ಯಸ್ಮೃ-ತಿಯ
ಪುತ
ಪುತು-ಭೇ-ದ-ಗ-ಳನ್ನುಂಟು-ಮಾಡಿ
ಪುತು-ಭೇ-ದ-ಗಳು
ಪುತ್ರ
ಪುತ್ರ-ನಾ-ಗಿದ್ದು
ಪುತ್ರನಿಂದ
ಪುತ್ರರತ್ನ
ಪುತ್ರರಲ್ಲಿ
ಪುತ್ರರೂ
ಪುತ್ರ-ಶೋ-ಕ-ದಿಂದ
ಪುತ್ರಿ
ಪುತ್ರಿಯ
ಪುತ್ರಿಯರೂ
ಪುನಃ
ಪುನಃಪ್ರ-ಕಾ-ಶಕ್ಕೆ
ಪುನಃಪ್ರ-ತಿಷ್ಠಾ-ಪ-ನೆಯೇ
ಪುನಃಪ್ರ-ತಿಷ್ಠೆ-ಯಾ-ಗದೆ
ಪುನರಪಿ
ಪುನ-ರಾ-ವ-ತಾರ
ಪುನ-ರಾ-ವರ್ತಿತ
ಪುನ-ರಾ-ವರ್ತಿ-ತ-ವಾ-ದರೆ
ಪುನ-ರುಚ್ಚ-ರಿ-ಸ-ಬೇ-ಕಾ-ಗಿತ್ತು
ಪುನ-ರುಚ್ಚ-ರಿ-ಸಿ-ದರು
ಪುನ-ರುಚ್ಚ-ರಿ-ಸುವ
ಪುನ-ರುಜ್ಜೀ-ವ-ನ-ಗೈ-ದರು
ಪುನ-ರುತ್ಥಾನ
ಪುನ-ರುತ್ಥಾ-ನದ
ಪುನ-ರುತ್ಥಾ-ನ-ವಾ-ಗ-ಬೇಕು
ಪುನರ್
ಪುನರ್ಜ-ನನ
ಪುನರ್ಜನ್ಮ
ಪುನರ್ಜನ್ಮಕ್ಕೊಂದು
ಪುನರ್ಜನ್ಮದ
ಪುನರ್ಜನ್ಮ-ವನ್ನು
ಪುನರ್ಜನ್ಮ-ವಲ್ಲದೇ
ಪುನರ್ಜನ್ಮ-ವಾದ
ಪುನರ್ಜನ್ಮ-ವಾ-ದ-ವನ್ನು
ಪುನರ್ಜನ್ಮ-ವೆಂದರೆ
ಪುನರ್ಜನ್ಮ-ವೆತ್ತಿ
ಪುನರ್ನಿರ್ಮಾಣ
ಪುನರ್ನಿರ್ಮಾ-ಣಕ್ಕಾಗಿ
ಪುನಾ-ರ-ಚ-ನೆ-ಚಂಚ-ಲ-ವಾದ
ಪುರಂದ-ರ-ದಾ-ಸರು
ಪುರ-ಸೊತ್ತಿದ್ದಾಗ
ಪುರಸ್ಕಾ-ರ-ಗ-ಳೆಲ್ಲವೂ
ಪುರಾ-ಣ-ಗ-ಳಲ್ಲಿ
ಪುರಾ-ಣ-ಗ-ಳಲ್ಲಿ-ರುವ
ಪುರಾ-ಣ-ಗಳು
ಪುರಾಣದ
ಪುರಾ-ಣ-ಯು-ಗಕ್ಕೆ
ಪುರಾತನ
ಪುರಾ-ತ-ನರು
ಪುರಾ-ವೆ-ಗಳ
ಪುರಾ-ವೆ-ಯನ್ನೊ-ದ-ಗಿ-ಸುತ್ತದೆ
ಪುರಾ-ವೆ-ಯಾ-ದೀತು
ಪುರಾವೆಯೂ
ಪುರೀಕ್ಷೇತ್ರ-ದಲ್ಲಿ
ಪುರುಷ
ಪುರು-ಷ-ಇ-ವ-ರಲ್ಲಿ
ಪುರು-ಷ-ಕಾ-ರ-ವನ್ನು
ಪುರು-ಷ-ಕಾ-ರ-ವನ್ನೂ
ಪುರು-ಷ-ನಾ-ಗಿದ್ದ
ಪುರು-ಷ-ನಾ-ಗಿದ್ದು-ದ-ರಿಂದ
ಪುರುಷನೂ
ಪುರುಷನೇ
ಪುರುಷರ
ಪುರು-ಷ-ರಲ್ಲಿ
ಪುರುಷರು
ಪುರುಷೋ
ಪುರು-ಸೊತ್ತಿಲ್ಲ
ಪುಲಿ-ಕೇ-ಶಿಯ
ಪುಷಿ-ಗ-ಳಿಗೆ
ಪುಷಿಗಳು
ಪುಷಿಗಳೂ
ಪುಷಿ-ಪ-ರಂಪರೆ
ಪುಷ್ಟ-ನಾ-ದ-ವನೇ
ಪುಷ್ಟವಾದ
ಪುಷ್ಟಿ
ಪುಷ್ಟಿಕರ
ಪುಷ್ಟಿಗಳ
ಪುಷ್ಟೀ-ಕ-ರಿ-ಸುತ್ತವೆ
ಪುಷ್ಪ-ಗ-ಳೆಂದಾ-ದರೂ
ಪುಷ್ಪ-ಗ-ಳೆ-ಡೆಗೆ
ಪುಷ್ಪಿ-ತ-ವಾಗ
ಪುಸ್ತಕ
ಪುಸ್ತ-ಕ-ಗಳ
ಪುಸ್ತ-ಕ-ಗ-ಳನ್ನು
ಪುಸ್ತ-ಕ-ಗ-ಳನ್ನೆಲ್ಲ
ಪುಸ್ತ-ಕ-ಗ-ಳಿಂದ
ಪುಸ್ತ-ಕ-ಗ-ಳಿವೆ
ಪುಸ್ತ-ಕ-ಗಳು
ಪುಸ್ತ-ಕ-ಗ-ಳುಂಟಾ-ಗುವ
ಪುಸ್ತ-ಕ-ಗಳೇ
ಪುಸ್ತಕದ
ಪುಸ್ತ-ಕ-ದಲ್ಲಿ
ಪುಸ್ತ-ಕ-ದಿಂದ
ಪುಸ್ತ-ಕಪ್ರೇ-ಮಿ-ಗಳು
ಪುಸ್ತ-ಕ-ವನ್ನಂತೂ
ಪುಸ್ತ-ಕ-ವನ್ನಿಡಿ
ಪುಸ್ತ-ಕ-ವನ್ನು
ಪುಸ್ತ-ಕ-ವನ್ನೂ
ಪುಸ್ತಕವು
ಪುಸ್ತಕವೂ
ಪುಸ್ತ-ಕ-ವೆಂದು
ಪುಸ್ತ-ಕ-ವೊಂದನ್ನೇ
ಪುಸ್ತ-ಕಾ-ಲಯ
ಪುಸ್ತ-ಕಾ-ಲ-ಯದ
ಪುಸ್ತ-ಕಾ-ಲ-ಯ-ದಿಂದ
ಪುಸ್ತ-ಕಾ-ಲ-ಯ-ವಿದೆ
ಪುಹಾರಿಚ್
ಪೂ
ಪೂಜಕರ
ಪೂಜನೀಯ
ಪೂಜಾ
ಪೂಜಾಂತ್ಯದ
ಪೂಜಾರ್ಹ-ರಾದ
ಪೂಜಾ-ವಿ-ಧಾ-ನ-ಗ-ಳಲ್ಲಿ
ಪೂಜಾಸ್ಥಾ-ನ-ಗಳು
ಪೂಜಿ-ಸ-ಬ-ಹುದು
ಪೂಜಿಸಲು
ಪೂಜಿ-ಸ-ಲೇ-ಬೇಕು
ಪೂಜಿ-ಸಲ್ಪ-ಡುವ
ಪೂಜಿಸಿ
ಪೂಜಿಸಿದ
ಪೂಜಿಸುತ್ತಾ
ಪೂಜಿ-ಸುತ್ತಾರೆ
ಪೂಜಿ-ಸುತ್ತಿ-ರು-ವನು
ಪೂಜಿ-ಸು-ವರು
ಪೂಜಿ-ಸು-ವುದು
ಪೂಜಿಸೋಣ
ಪೂಜೆ
ಪೂಜೆ-ಗ-ಳಿಂದಲೂ
ಪೂಜೆಗೆ
ಪೂಜೆಯ
ಪೂಜೆಯನ್ನೂ
ಪೂಜ್ಯ
ಪೂಜ್ಯರೆ
ಪೂರಕ
ಪೂರ-ಕ-ವಾಗಿ
ಪೂರ-ಕ-ವಾದ
ಪೂರ-ಣ-ವಾ-ಗು-ವುದು
ಪೂರೈ-ಕೆ-ಗಾಗಿ
ಪೂರೈಕೆಗೆ
ಪೂರೈ-ಕೆ-ಯಾ-ಗದೆ
ಪೂರೈ-ಸ-ಬೇ-ಕಾ-ದರೆ
ಪೂರೈಸಲು
ಪೂರೈಸಲೂ
ಪೂರೈಸಿ
ಪೂರೈ-ಸಿ-ಕೊಳ್ಳಲು
ಪೂರೈ-ಸು-ವಂತಿಲ್ಲ
ಪೂರೈ-ಸು-ವ-ವನೇ
ಪೂರ್ಣ
ಪೂರ್ಣ-ಕಾ-ಲ-ವನ್ನು
ಪೂರ್ಣ-ಗೊ-ಳಿ-ಸ-ಬ-ಹುದು
ಪೂರ್ಣ-ಗೊ-ಳಿ-ಸಿದ
ಪೂರ್ಣ-ಗೊ-ಳಿ-ಸು-ವು-ದರ
ಪೂರ್ಣತೆ
ಪೂರ್ಣತೆಯ
ಪೂರ್ಣ-ತೆ-ಯನ್ನು
ಪೂರ್ಣ-ತೆ-ಯನ್ನೂ
ಪೂರ್ಣ-ದೃಷ್ಟಿಯ
ಪೂರ್ಣ-ದೃಷ್ಟಿ-ಯಿಂದ
ಪೂರ್ಣ-ನಂಬಿಕೆ
ಪೂರ್ಣ-ನಿ-ವಾ-ರಣೆ
ಪೂರ್ಣ-ಮ-ನಸ್ಸಿ-ನಿಂದ
ಪೂರ್ಣ-ರೀ-ತಿ-ಯಿಂದ
ಪೂರ್ಣವಲ್ಲ
ಪೂರ್ಣವಾಗಿ
ಪೂರ್ಣವಾದ
ಪೂರ್ಣ-ವಿ-ಕ-ಸ-ನ-ದಲ್ಲಿ
ಪೂರ್ಣ-ವಿ-ಧೇ-ಯ-ಳಾ-ಗಿ-ರು-ವುದೇ
ಪೂರ್ಣ-ವಿ-ರಾಮ
ಪೂರ್ಣವೂ
ಪೂರ್ಣ-ಶಕ್ತಿಯು
ಪೂರ್ಣಶ್ರದ್ಧೆ-ಯಿಂದ
ಪೂರ್ಣಸತ್ಯ
ಪೂರ್ಣ-ಸತ್ಯ-ವಲ್ಲ
ಪೂರ್ಣ-ಸಿದ್ಧಿ-ಯನ್ನು
ಪೂರ್ತಿಯಾಗಿ
ಪೂರ್ವ
ಪೂರ್ವಜರು
ಪೂರ್ವಕ
ಪೂರ್ವ-ಕರ್ಮ-ಗ-ಳನ್ನು
ಪೂರ್ವ-ಕರ್ಮ-ಗ-ಳಿಂದ
ಪೂರ್ವ-ಕ-ವಾಗಿ
ಪೂರ್ವ-ಕ-ವಾದ
ಪೂರ್ವ-ಕಾ-ಲ-ದಲ್ಲಿ
ಪೂರ್ವ-ಜನ್ಮ-ಗಳ
ಪೂರ್ವ-ಜನ್ಮದ
ಪೂರ್ವ-ಜನ್ಮ-ದಲ್ಲಿ
ಪೂರ್ವ-ಜನ್ಮಸ್ಮ-ರಣೆ
ಪೂರ್ವಜರು
ಪೂರ್ವದ
ಪೂರ್ವದಲ್ಲಿ
ಪೂರ್ವ-ದೇ-ಶ-ಗಳ
ಪೂರ್ವ-ನಿಶ್ಚಿತ
ಪೂರ್ವ-ಪು-ರು-ಷರು
ಪೂರ್ವ-ಯೋ-ಜಿತ
ಪೂರ್ವಸ್ಥಿ-ತಿಗೇ
ಪೂರ್ವಾಗ್ರಹ
ಪೂರ್ವಾಗ್ರ-ಹ-ಗ-ಳನ್ನೂ
ಪೂರ್ವಾಗ್ರ-ಹ-ಗಳು
ಪೂರ್ವಾಗ್ರ-ಹ-ದೂ-ಷಿ-ತರ
ಪೂರ್ವಾಗ್ರ-ಹ-ವಿ-ರುವ
ಪೂರ್ವಾಗ್ರ-ಹ-ವಿಲ್ಲದೆ
ಪೂರ್ವಾ-ಜಿ-ತದ
ಪೂರ್ವಾಪರ
ಪೂರ್ವಿಕರ
ಪೂರ್ವೋತ್ತ-ರ-ಗ-ಳನ್ನೆಲ್ಲ
ಪೃಥಕ್ಕ-ರ-ಣ-ಪ-ರಿಷ್ಕ-ರ-ಣ-ಗ-ಳಿಂದ
ಪೃಥಿವಿಗೆ
ಪೆಟ್ಟನ್ನೂ
ಪೆಟ್ರೋಲ್
ಪೆದ್ದನ
ಪೆದ್ದನನ್ನು
ಪೆನ್
ಪೆನ್ಫೀಲ್ಡನ
ಪೆನ್ಫೀಲ್ಡ್
ಪೆನ್ಸಿಲನ್ನು
ಪೆನ್ಸಿಲನ್ನೂ
ಪೆನ್ಸಿ-ಲು-ಗಳ
ಪೆನ್ಸಿ-ಲು-ಗ-ಳನ್ನು
ಪೆನ್ಸಿಲ್
ಪೆರಾ-ಸಲ್ಸಸ್
ಪೆರು
ಪೆರು-ವಿ-ನಲ್ಲಿ
ಪೇಚಾಟ
ಪೇಟೆ
ಪೇಟೆಗೆ
ಪೇಟೆಯ
ಪೇಟೆ-ಯಂತಿ-ರುವ
ಪೇಟೆಯಲ್ಲಿ
ಪೇಟೆಯಿಂದ
ಪೇತ
ಪೇಪರು
ಪೇರಿ-ಸುತ್ತಲೇ
ಪೇಳ್ವ
ಪೇಸ್ಟನ್ನು
ಪೈಕಿ
ಪೈಪ್
ಪೈಶಾಚಿಕ
ಪೈಸೆ-ಗ-ಳ-ವ-ರೆಗೆ
ಪೈಸೆಯಿಂದ
ಪೊಗ-ಳಿ-ಕೆಗೆ
ಪೊಟ್ಟ-ಣ-ಗ-ಳನ್ನು
ಪೊನ್ನಂಪೇಟೆ
ಪೊರೆ
ಪೊಳ್ಳು
ಪೋಟೋ
ಪೋಟೋದಲ್ಲಿ
ಪೋಪರ
ಪೋಪ್
ಪೋರರೇ
ಪೋರ್ಚು-ಗೀ-ಸರು
ಪೋರ್ಚು-ಗೀ-ಸರೂ
ಪೋರ್ಚುಗೀಸ್
ಪೋಲೀ-ಸ-ರಿಗೂ
ಪೋಲೀಸರ
ಪೋಲೀ-ಸ-ರಿಂದ
ಪೋಲೀಸರು
ಪೋಲೀಸು
ಪೋಲೀ-ಸು-ನಾಯಿ
ಪೋಲೀಸ್
ಪೋಲೆಂಡನ್ನೂ
ಪೋಷಕ
ಪೋಷಕರ
ಪೋಷಕರು
ಪೋಷ-ಕ-ವಾದ
ಪೋಷಕವೂ
ಪೋಷ-ಣ-ಗ-ಳನ್ನೂ
ಪೋಷಣೆ
ಪೋಷ-ಣೆ-ಗ-ಳಲ್ಲಿ
ಪೋಷಣೆಯ
ಪೋಷಾಕನ್ನು
ಪೋಷಿ-ತ-ವಾಗಿ
ಪೋಷಿ-ಸ-ಲಾ-ಗದು
ಪೋಷಿಸಿ
ಪೋಷಿ-ಸುತ್ತಿ-ದೆ-ಯಲ್ಲ
ಪೋಷಿಸುವ
ಪೌಂಡಿಗೆ
ಪೌಂಡು
ಪೌಂಡ್
ಪೌರಪ್ರಜ್ಞೆ
ಪೌರ-ಶಿಕ್ಷಣ
ಪೌರಸ್ತ್ಯ
ಪೌರು-ಷ-ವಾ-ದಿ-ಗ-ಳಿದ್ದಾರೆ
ಪೌರುಷವೋ
ಪೌಲೋ
ಪೌಷ್ಟಿಕ
ಪ್ಯಾಟನ
ಪ್ಯಾಟನು
ಪ್ಯಾಟ್
ಪ್ಯಾಟ್ನನ್ನು
ಪ್ಯಾರಿಸ್ಸಿನ
ಪ್ಯಾರ್
ಪ್ಯಾಲಾ
ಪ್ಯಾಲೇಸ್ತೀ-ನಿನ
ಪ್ಯಾಶನ್
ಪ್ಯಾಶ್ಚರ್
ಪ್ಯೂರಿ-ಟನ್ನರ
ಪ್ರಕಟ
ಪ್ರಕ-ಟ-ವಾ-ಯಿತು
ಪ್ರಕ-ಟ-ಗೊಂಡ
ಪ್ರಕ-ಟ-ಗೊಳ್ಳಲು
ಪ್ರಕ-ಟ-ಣೆ-ಯಾದ
ಪ್ರಕಟನ
ಪ್ರಕ-ಟ-ವಾ-ಗದ
ಪ್ರಕ-ಟ-ವಾ-ಗ-ದಿದ್ದಿ-ರ-ಬ-ಹುದು
ಪ್ರಕ-ಟ-ವಾ-ಗಲು
ಪ್ರಕ-ಟ-ವಾ-ಗಿತ್ತು
ಪ್ರಕ-ಟ-ವಾ-ಗಿದೆ
ಪ್ರಕ-ಟ-ವಾ-ಗಿವೆ
ಪ್ರಕ-ಟ-ವಾ-ಗುತ್ತದೆ
ಪ್ರಕ-ಟ-ವಾ-ಗುತ್ತವೆ
ಪ್ರಕ-ಟ-ವಾ-ಗುತ್ತಿತ್ತು
ಪ್ರಕ-ಟ-ವಾ-ಗುತ್ತಿದೆ
ಪ್ರಕ-ಟ-ವಾ-ಗುವ
ಪ್ರಕ-ಟ-ವಾ-ಗು-ವು-ದನ್ನು
ಪ್ರಕ-ಟ-ವಾ-ಗು-ವುದು
ಪ್ರಕ-ಟ-ವಾದ
ಪ್ರಕ-ಟ-ವಾ-ದಂತಾ-ಯಿತೆ
ಪ್ರಕ-ಟ-ವಾ-ದಾಗ
ಪ್ರಕ-ಟ-ವಾ-ಯಿತು
ಪ್ರಕ-ಟಿ-ಸ-ಬ-ಹು-ದಾದ
ಪ್ರಕ-ಟಿ-ಸ-ಲಾ-ಗಿದೆ
ಪ್ರಕ-ಟಿ-ಸ-ಲಾ-ಗಿದ್ದು
ಪ್ರಕ-ಟಿ-ಸಲು
ಪ್ರಕ-ಟಿ-ಸಿತು
ಪ್ರಕ-ಟಿ-ಸಿದ
ಪ್ರಕ-ಟಿ-ಸಿ-ದರು
ಪ್ರಕ-ಟಿ-ಸಿ-ದ-ವರು
ಪ್ರಕ-ಟಿ-ಸಿದ್ದರು
ಪ್ರಕ-ಟಿ-ಸಿದ್ದಾರೆ
ಪ್ರಕ-ಟಿ-ಸುತ್ತಲೇ
ಪ್ರಕ-ಟಿ-ಸುವ
ಪ್ರಕ-ಟಿ-ಸು-ವು-ದಕ್ಕೆ
ಪ್ರಕ-ಟಿ-ಸು-ವು-ದಿಲ್ಲ
ಪ್ರಕ-ರ-ಣಕ್ಕೆ
ಪ್ರಕ-ರ-ಣ-ಗ-ಳಲ್ಲೂ
ಪ್ರಕಾರ
ಪ್ರಕಾ-ರ-ಗ-ಳಲ್ಲಿಯೂ
ಪ್ರಕಾಶ
ಪ್ರಕಾ-ಶ-ಕರ
ಪ್ರಕಾ-ಶ-ಕ-ರಿಗೆ
ಪ್ರಕಾ-ಶ-ಕರು
ಪ್ರಕಾಶನ
ಪ್ರಕಾ-ಶ-ನದ
ಪ್ರಕಾ-ಶ-ವಾ-ಗು-ವಂತೆ
ಪ್ರಕಾ-ಶ-ವಾ-ಗು-ವುವು
ಪ್ರಕಾ-ಶಿ-ತ-ವಾ-ಗುವ
ಪ್ರಕಾಶ್
ಪ್ರಕೃತ
ಪ್ರಕೃತಿ
ಪ್ರಕೃತಿಗೆ
ಪ್ರಕೃತಿಗೇ
ಪ್ರಕೃ-ತಿಪ್ರೇಮ
ಪ್ರಕೃತಿಯ
ಪ್ರಕೃ-ತಿ-ಯನ್ನು
ಪ್ರಕೃ-ತಿ-ಯಲ್ಲಿ
ಪ್ರಕೃತಿಯು
ಪ್ರಕೃ-ತಿ-ಯೆಲ್ಲ
ಪ್ರಕೃ-ತಿ-ರ-ಹಸ್ಯ-ಗ-ಳನ್ನು
ಪ್ರಕೃ-ತಿ-ಶಕ್ತಿ-ಗಳ
ಪ್ರಕ್ರಿ-ಯೆ-ಗ-ಳಂತೆ
ಪ್ರಖ್ಯಾತ
ಪ್ರಖ್ಯಾ-ತ-ಳಾದ
ಪ್ರಗತಿ
ಪ್ರಗ-ತಿ-ಕಾರ್ಯದ
ಪ್ರಗ-ತಿಕ್ಷಿಪ್ರ-ತೆ-ಯನ್ನು
ಪ್ರಗ-ತಿ-ಗ-ಳಿಂದ
ಪ್ರಗ-ತಿ-ಗಳೂ
ಪ್ರಗತಿಗೂ
ಪ್ರಗತಿಗೆ
ಪ್ರಗ-ತಿ-ಪ-ಥ-ದಲ್ಲಿ
ಪ್ರಗತಿಯ
ಪ್ರಗ-ತಿ-ಯತ್ತ
ಪ್ರಗ-ತಿ-ಯನ್ನು
ಪ್ರಗ-ತಿ-ಯಲ್ಲ
ಪ್ರಗ-ತಿ-ಯಾ-ಗ-ಬೇಕು
ಪ್ರಗ-ತಿ-ಯಾಗಿ
ಪ್ರಗ-ತಿ-ಯಾದ
ಪ್ರಗ-ತಿ-ಯಿಂದ
ಪ್ರಗತಿಯು
ಪ್ರಗತಿಯೂ
ಪ್ರಗ-ತಿ-ಶೀಲ
ಪ್ರಚಂಡ
ಪ್ರಚ-ಲಿ-ತ-ವಾ-ದದ್ದೆ
ಪ್ರಚಾರ
ಪ್ರಚಾ-ರಕ್ಕಾಗಿ
ಪ್ರಚಾ-ರ-ಗಳು
ಪ್ರಚಾ-ರ-ದಿಂದ
ಪ್ರಚಾ-ರ-ವಾ-ಗಿ-ರುವ
ಪ್ರಚಾ-ರ-ವಾ-ಗುತ್ತಿ-ರುವ
ಪ್ರಚಾ-ರ-ವಾ-ಗುತ್ತಿ-ರು-ವುದು
ಪ್ರಚಾರವೇ
ಪ್ರಚೋ-ದ-ನೆಗೆ
ಪ್ರಚೋ-ದ-ನೆ-ಯಿಂದ
ಪ್ರಚೋ-ದಿ-ಸ-ಲಾ-ಯಿತು
ಪ್ರಚ್ಛನ್ನ
ಪ್ರಜಾ-ಜ-ನರ
ಪ್ರಜಾಪ್ರ-ಭುತ್ವ
ಪ್ರಜಾಪ್ರ-ಭುತ್ವ-ವನ್ನು-ಳಿಸಿ
ಪ್ರಜಾಪ್ರ-ಭುತ್ವಕ್ರ-ಮದ
ಪ್ರಜಾಪ್ರ-ಭುತ್ವದ
ಪ್ರಜಾಪ್ರ-ಭುತ್ವ-ದಲ್ಲಿ
ಪ್ರಜಾಪ್ರ-ಭುತ್ವ-ವೆಂದರೆ
ಪ್ರಜಾ-ಸತ್ತಾತ್ಮಕ
ಪ್ರಜೆಗಳ
ಪ್ರಜೆ-ಗ-ಳನ್ನಾ-ಗಿಸ
ಪ್ರಜೆ-ಗ-ಳಾ-ಗಲು
ಪ್ರಜೆ-ಗ-ಳಾಗಿ
ಪ್ರಜೆ-ಗ-ಳಾದ
ಪ್ರಜೆಗಳು
ಪ್ರಜೆಯ
ಪ್ರಜೆ-ಯಾ-ಗಿದ್ದಾ-ನೆಂದು
ಪ್ರಜ್ಞಾ-ಪೂರ್ವಕ
ಪ್ರಜ್ಞಾನ
ಪ್ರಜ್ಞಾ-ಪೂರ್ವಕ
ಪ್ರಜ್ಞಾ-ಪೂರ್ವ-ಕ-ವಾಗಿ
ಪ್ರಜ್ಞೆ
ಪ್ರಜ್ಞೆಗಳ
ಪ್ರಜ್ಞೆಗೆ
ಪ್ರಜ್ಞೆತಪ್ಪಿ
ಪ್ರಜ್ಞೆಯ
ಪ್ರಜ್ಞೆಯನ್ನು
ಪ್ರಜ್ಞೆಯನ್ನೂ
ಪ್ರಜ್ಞೆ-ಯನ್ನೊ-ಳ-ಗೊಂಡ
ಪ್ರಜ್ಞೆಯಲ್ಲಿ
ಪ್ರಜ್ಞೆಯಿಂದ
ಪ್ರಜ್ಞೆ-ಯಿಂದಲೇ
ಪ್ರಜ್ಞೆಯು
ಪ್ರಜ್ಞೆ-ಯುಳ್ಳ-ವರು
ಪ್ರಜ್ವ-ಲ-ಗೊ-ಳಿ-ಸುತ್ತಲೇ
ಪ್ರಣಶ್ಯತಿ
ಪ್ರಣಾಮ
ಪ್ರಣಾ-ಮ-ಗ-ಳನ್ನು
ಪ್ರಣಿಧಾನ
ಪ್ರತಿ
ಪ್ರತಿ-ಗ-ಮನ
ಪ್ರತಿಯೊಬ್ಬ
ಪ್ರತಿ-ಯೊಬ್ಬ-ನಿಗೂ
ಪ್ರತಿಕೂಲ
ಪ್ರತಿ-ಕೂ-ಲ-ವಾದ
ಪ್ರತಿ-ಕೂ-ಲಸ್ಯ
ಪ್ರತಿ-ಕೃ-ತಿ-ಯಂತೆಯೇ
ಪ್ರತಿಕ್ರಿಯೆ
ಪ್ರತಿಕ್ರಿ-ಯೆ-ಇವು
ಪ್ರತಿಕ್ರಿ-ಯೆ-ಗಳ
ಪ್ರತಿಕ್ರಿ-ಯೆ-ಗ-ಳನ್ನು
ಪ್ರತಿಕ್ರಿ-ಯೆ-ಗ-ಳನ್ನೂ
ಪ್ರತಿಕ್ರಿ-ಯೆ-ಗಳು
ಪ್ರತಿಕ್ರಿ-ಯೆ-ಗ-ಳು-ಇ-ವೆಲ್ಲ-ವು-ಗ-ಳಿಂದ
ಪ್ರತಿಕ್ರಿ-ಯೆ-ಗಳೆ
ಪ್ರತಿಕ್ರಿ-ಯೆ-ಟೀಕೆ
ಪ್ರತಿಕ್ರಿ-ಯೆಯ
ಪ್ರತಿಕ್ರಿ-ಯೆ-ಯನ್ನು
ಪ್ರತಿಕ್ರಿ-ಯೆ-ಯನ್ನೂ
ಪ್ರತಿಕ್ರಿ-ಯೆ-ಯಾಗಿ
ಪ್ರತಿಕ್ರಿ-ಯೆ-ಯುಂಟಾಗಿ
ಪ್ರತಿಕ್ಷ-ಣವೂ
ಪ್ರತಿ-ಗ-ಮನ
ಪ್ರತಿಗಳು
ಪ್ರತಿಗಾಮಿ
ಪ್ರತಿಜ್ಞಾ
ಪ್ರತಿಜ್ಞೆ
ಪ್ರತಿಜ್ಞೆ-ಯನ್ನೇ
ಪ್ರತಿದಿನ
ಪ್ರತಿ-ದಿ-ನವೂ
ಪ್ರತಿಧ್ವನಿ
ಪ್ರತಿಧ್ವ-ನಿ-ಸಿ-ದರು
ಪ್ರತಿನಿತ್ಯ
ಪ್ರತಿ-ನಿತ್ಯವೂ
ಪ್ರತಿನಿಧಿ
ಪ್ರತಿ-ನಿ-ಧಿ-ಗ-ಳೊ-ಡನೆ
ಪ್ರತಿ-ನಿ-ಧಿ-ಗ-ಳಾ-ಗುತ್ತಾ-ರೆನ್ನ-ಲಾ-ಗದು
ಪ್ರತಿ-ನಿ-ಧಿ-ಗ-ಳಾ-ಗುವ
ಪ್ರತಿ-ನಿ-ಧಿ-ಗ-ಳಾದ
ಪ್ರತಿ-ನಿ-ಧಿ-ಗಳು
ಪ್ರತಿ-ನಿ-ಧಿ-ಗಳೂ
ಪ್ರತಿ-ನಿ-ಧಿ-ಗಳೇ
ಪ್ರತಿ-ನಿ-ಧಿಯ
ಪ್ರತಿ-ನಿ-ಧಿ-ಯಾಗಿ
ಪ್ರತಿ-ನಿ-ಧಿ-ಸುತ್ತಾನೆ
ಪ್ರತಿ-ಪಾ-ದ-ಕರೇ
ಪ್ರತಿ-ಪಾ-ದಿ-ಸಲು
ಪ್ರತಿ-ಪಾ-ದಿ-ಸಲ್ಪ-ಡುವ
ಪ್ರತಿ-ಪಾ-ದಿ-ಸಿ-ದರು
ಪ್ರತಿಫಲ
ಪ್ರತಿ-ಫ-ಲ-ಒಂದು
ಪ್ರತಿ-ಫ-ಲ-ವನ್ನ-ಪೇಕ್ಷಿ-ಸದೇ
ಪ್ರತಿ-ಫ-ಲ-ವನ್ನು
ಪ್ರತಿ-ಫ-ಲ-ವಾಗಿ
ಪ್ರತಿ-ಫ-ಲಾ-ಪೇಕ್ಷೆ
ಪ್ರತಿಬಾರಿ
ಪ್ರತಿ-ಬಾ-ರಿಯೂ
ಪ್ರತಿಬಿಂಬ
ಪ್ರತಿ-ಬೈ-ಗಳು
ಪ್ರತಿ-ಭ-ಟ-ನೆ-ಗಳ
ಪ್ರತಿ-ಭಾನ್ವಿತ
ಪ್ರತಿ-ಭಾ-ವಂತ
ಪ್ರತಿ-ಭಾ-ಶಾಲಿ
ಪ್ರತಿ-ಭಾ-ಶಾ-ಲಿ-ಗ-ಳಾ-ಗಿದ್ದರೂ
ಪ್ರತಿ-ಭಾ-ಶಾ-ಲಿ-ಗ-ಳಲ್ಲದ
ಪ್ರತಿ-ಭಾ-ಶಾ-ಲಿ-ಗ-ಳಲ್ಲಿ
ಪ್ರತಿ-ಭಾ-ಶಾ-ಲಿ-ಗಳೇ
ಪ್ರತಿಭೆ
ಪ್ರತಿಭೆಗೂ
ಪ್ರತಿಭೆಯ
ಪ್ರತಿ-ಭೆ-ಯನ್ನು
ಪ್ರತಿ-ಭೆ-ಯಲ್ಲಿ
ಪ್ರತಿಮೂರ್ತಿ
ಪ್ರತಿಯನ್ನು
ಪ್ರತಿಯಾಗಿ
ಪ್ರತಿ-ಯೊಂದನ್ನು
ಪ್ರತಿ-ಯೊಂದನ್ನೂ
ಪ್ರತಿ-ಯೊಂದ-ರಲ್ಲೂ
ಪ್ರತಿಯೊಂದು
ಪ್ರತಿಯೊಬ್ಬ
ಪ್ರತಿ-ಯೊಬ್ಬ-ನಲ್ಲೂ
ಪ್ರತಿ-ಯೊಬ್ಬನ
ಪ್ರತಿ-ಯೊಬ್ಬ-ನನ್ನೂ
ಪ್ರತಿ-ಯೊಬ್ಬ-ನಲ್ಲಿ
ಪ್ರತಿ-ಯೊಬ್ಬ-ನಲ್ಲಿಯೂ
ಪ್ರತಿ-ಯೊಬ್ಬ-ನಲ್ಲೂ
ಪ್ರತಿ-ಯೊಬ್ಬ-ನಿಗೂ
ಪ್ರತಿ-ಯೊಬ್ಬನೂ
ಪ್ರತಿ-ಯೊಬ್ಬರ
ಪ್ರತಿ-ಯೊಬ್ಬ-ರನ್ನೂ
ಪ್ರತಿ-ಯೊಬ್ಬ-ರಲ್ಲೂ
ಪ್ರತಿ-ಯೊಬ್ಬರೂ
ಪ್ರತಿ-ವರ್ಷವೂ
ಪ್ರತಿಷ್ಠಾ-ಪನೆ
ಪ್ರತಿಷ್ಠಾ-ಪ-ನೆಯ
ಪ್ರತಿಷ್ಠಾ-ಪಿ-ಸ-ಬೇಕು
ಪ್ರತಿಷ್ಠಾ-ಪಿ-ಸಲು
ಪ್ರತಿಷ್ಠಾ-ಪಿ-ಸಿ-ದರೆ
ಪ್ರತಿಷ್ಠಿ-ತ-ವಾ-ಗಿದೆ
ಪ್ರತಿಷ್ಠೆಯ
ಪ್ರತಿಸ್ಪಂದಿ-ಸುತ್ತಿದ್ದರೆ
ಪ್ರತಿಸ್ಪರ್ಧಿ
ಪ್ರತಿಸ್ಪರ್ಧಿ-ಗ-ಳೆಂದು
ಪ್ರತಿಸ್ಪರ್ಧಿ-ಗ-ಳೆನ್ನುವ
ಪ್ರತಿಹಿಂಸೆ
ಪ್ರತೀ-ಕ-ರಾ-ಗೋಣ
ಪ್ರತೀ-ಕ-ವಾಗಿ
ಪ್ರತೀ-ಕ-ವಾದ
ಪ್ರತೀ-ಕ-ವಾ-ದಾಗ
ಪ್ರತೀ-ಕ-ವೆಂದರೆ
ಪ್ರತೀಕವೇ
ಪ್ರತೀಕಾರ
ಪ್ರತೀ-ಕಾ-ರ-ಗಳು
ಪ್ರತ್ಯಕ್ಷ
ಪ್ರತ್ಯಕ್ಷದ
ಪ್ರತ್ಯಕ್ಷ-ನಾಗಿ
ಪ್ರತ್ಯಕ್ಷಪ್ರ-ಮಾಣ
ಪ್ರತ್ಯಕ್ಷ-ವಾಗಿ
ಪ್ರತ್ಯಕ್ಷ-ವಾ-ಗಿಯೋ
ಪ್ರತ್ಯಕ್ಷ-ವಾ-ಗುತ್ತದೆ
ಪ್ರತ್ಯಕ್ಷ-ವಾ-ಗುತ್ತವೆ
ಪ್ರತ್ಯಕ್ಷ-ವಾ-ಗು-ವುವೋ
ಪ್ರತ್ಯಕ್ಷ-ವೆನ್ನುತ್ತಾರೆ
ಪ್ರತ್ಯಕ್ಷ-ಸಾಕ್ಷಿಗೆ
ಪ್ರತ್ಯಕ್ಷಾ-ನು-ಭ-ವ-ಗಳೂ
ಪ್ರತ್ಯಕ್ಷೀ-ಕ-ರಿ-ಸಲು
ಪ್ರತ್ಯ-ಯ-ಗಳು
ಪ್ರತ್ಯೇಕ
ಪ್ರತ್ಯೇ-ಕ-ವಾಗಿ
ಪ್ರತ್ಯೇ-ಕ-ವಾದ
ಪ್ರಥಮ
ಪ್ರಥ-ಮ-ದರ್ಜೆ-ಯ-ವ-ರಂತೆ
ಪ್ರಥ-ಮ-ದಲ್ಲೇ
ಪ್ರಥ-ಮ-ದಿಂದಲೇ
ಪ್ರಥ-ಮ-ಬಾರಿ
ಪ್ರಥ-ಮಶ್ರೇ-ಣಿ-ಯಲ್ಲಿ
ಪ್ರದರ್ಶನ
ಪ್ರದರ್ಶ-ನ-ಗ-ಳನ್ನೇರ್ಪ-ಡಿಸಿ
ಪ್ರದರ್ಶ-ನಕ್ಕೆ
ಪ್ರದರ್ಶ-ನ-ಗ-ಳನ್ನು
ಪ್ರದರ್ಶ-ನದ
ಪ್ರದರ್ಶ-ನ-ದಲ್ಲಿ
ಪ್ರದರ್ಶ-ನ-ವನ್ನಿತ್ತ
ಪ್ರದರ್ಶ-ನ-ವನ್ನು
ಪ್ರದರ್ಶಿತ
ಪ್ರದರ್ಶಿ-ತ-ವಾ-ಗು-ವು-ದಕ್ಕಿಂತಲೂ
ಪ್ರದರ್ಶಿ-ಸ-ಬೇ-ಕಾ-ಗುತ್ತದೆ
ಪ್ರದರ್ಶಿ-ಸ-ಬ-ಹುದು
ಪ್ರದರ್ಶಿ-ಸಲು
ಪ್ರದರ್ಶಿಸಿ
ಪ್ರದರ್ಶಿ-ಸಿ-ದಂತಾ-ದರೂ
ಪ್ರದರ್ಶಿ-ಸಿ-ದಂತಾ-ಯಿತೇ
ಪ್ರದರ್ಶಿ-ಸುತ್ತಾರೆ
ಪ್ರದರ್ಶಿ-ಸುತ್ತ
ಪ್ರದಾ-ನ-ಮಾಡಿ
ಪ್ರದೇಶ
ಪ್ರದೇಶಕ್ಕೆ
ಪ್ರದೇಶದ
ಪ್ರದೇ-ಶ-ದಂತೆ
ಪ್ರದೇ-ಶ-ದಲ್ಲಿ
ಪ್ರದೇ-ಶ-ದಲ್ಲಿದ್ದ
ಪ್ರದೇ-ಶ-ದಿಂದ
ಪ್ರದೇ-ಶ-ವನ್ನು
ಪ್ರಧಾನ
ಪ್ರಧಾ-ನ-ಪಾತ್ರ
ಪ್ರಧಾ-ನ-ಮಂತ್ರಿ-ಯಾ-ಗಿದ್ದ
ಪ್ರಧಾ-ನ-ವಾ-ಗಿ-ರುವ
ಪ್ರಧಾನಿ
ಪ್ರಧಾ-ನಿ-ಯಾದ
ಪ್ರಧಾ-ನಿ-ಯಾ-ದದ್ದು
ಪ್ರಪಂಚ
ಪ್ರಪಂಚ-ಅಟ್ಲಾಂಟಿಸ್
ಪ್ರಪಂಚಕ್ಕೆ
ಪ್ರಪಂಚದ
ಪ್ರಪಂಚ-ದಲ್ಲಿ
ಪ್ರಪಂಚ-ದಲ್ಲಿಲ್ಲ
ಪ್ರಪಂಚ-ವನ್ನು
ಪ್ರಪಂಚ-ವನ್ನೂ
ಪ್ರಪಾ-ತ-ದೆ-ಡೆಗೆ
ಪ್ರಬಂಧ-ಗ-ಳಿವೆ
ಪ್ರಬಂಧ-ಗಳು
ಪ್ರಬಲ
ಪ್ರಬ-ಲ-ವಾಗಿ
ಪ್ರಬ-ಲ-ವಾ-ಗಿಯೇ
ಪ್ರಬ-ಲ-ವಾದ
ಪ್ರಬ-ಲ-ವಾ-ದವು
ಪ್ರಬುದ್ಧ
ಪ್ರಬುದ್ಧತೆ
ಪ್ರಬುದ್ಧ-ವಾಗಿ
ಪ್ರಬುದ್ಧಾ-ನಂದಜೀ
ಪ್ರಭಾವ
ಪ್ರಭಾ-ವ-ಗ-ಳನ್ನು
ಪ್ರಭಾ-ವ-ಗ-ಳಿಗೆ
ಪ್ರಭಾ-ವ-ಗೊ-ಳಿಸಿ
ಪ್ರಭಾ-ವ-ಗೊ-ಳಿ-ಸು-ವು-ದೆಂದು
ಪ್ರಭಾವದ
ಪ್ರಭಾ-ವ-ದಿಂದ
ಪ್ರಭಾ-ವ-ದಿಂದಲ್ಲವೆ
ಪ್ರಭಾ-ವ-ಲಯ
ಪ್ರಭಾ-ವ-ಲ-ಯದ
ಪ್ರಭಾ-ವ-ಲ-ಯ-ದಿಂದ
ಪ್ರಭಾ-ವ-ಲ-ಯವು
ಪ್ರಭಾ-ವ-ವನ್ನು
ಪ್ರಭಾ-ವ-ವನ್ನುಂಟು-ಮಾಡಿ
ಪ್ರಭಾ-ವ-ವನ್ನೂ
ಪ್ರಭಾ-ವ-ಶಾಲಿ
ಪ್ರಭಾ-ವ-ಶಾ-ಲಿ-ಯಾದ
ಪ್ರಭಾ-ವಿ-ತ-ರಾಗಿ
ಪ್ರಭಾ-ವಿ-ತ-ನಾಗಿ
ಪ್ರಭಾ-ವಿ-ತ-ನಾ-ಗಿದ್ದೆ
ಪ್ರಭಾ-ವಿ-ತ-ರಾ-ಗುತ್ತಾರೆ
ಪ್ರಭಾ-ವಿ-ತ-ರೆಂಬುದು
ಪ್ರಭಾ-ವಿ-ತ-ವಾ-ಗ-ದಿ-ರಲು
ಪ್ರಭಾ-ವಿ-ತ-ವಾ-ಗಿವೆ
ಪ್ರಭಾವೀ
ಪ್ರಭು
ಪ್ರಭುಗಳ
ಪ್ರಭು-ಶಕ್ತಿ-ಗಿಂತಲೂ
ಪ್ರಭೃ-ತಿ-ಗಳು
ಪ್ರಭೆಯ
ಪ್ರಭೇ-ದ-ಗಳು
ಪ್ರಮತ್ತ-ರಾಗಿ
ಪ್ರಮಾಣ
ಪ್ರಮಾಣದ
ಪ್ರಮಾ-ಣ-ದಲ್ಲಿ
ಪ್ರಮಾ-ಣ-ಪತ್ರ-ಗಳು
ಪ್ರಮಾ-ಣ-ಪತ್ರ-ಗಳೂ
ಪ್ರಮಾ-ಣ-ವನ್ನು
ಪ್ರಮಾಣವೂ
ಪ್ರಮಾ-ದ-ವನ್ನುಂಟು-ಮಾ-ಡ-ಬಲ್ಲವು
ಪ್ರಮುಖ
ಪ್ರಮುಖನೂ
ಪ್ರಮು-ಖ-ರಲ್ಲಿ
ಪ್ರಮು-ಖ-ರಾದ
ಪ್ರಮುಖರು
ಪ್ರಮುಖರೂ
ಪ್ರಮು-ಖ-ವಾದ
ಪ್ರಮೇ-ಯ-ಗ-ಳನ್ನು
ಪ್ರಮೇಯವೇ
ಪ್ರಯಚ್ಛ
ಪ್ರಯತ್ನ
ಪ್ರಯತ್ನ-ಪಟ್ಟರೂ
ಪ್ರಯತ್ನಕ್ಕೆ
ಪ್ರಯತ್ನ-ಗ-ಳನ್ನು
ಪ್ರಯತ್ನ-ಗ-ಳಿಂದ
ಪ್ರಯತ್ನ-ಗ-ಳಿಂದಲೂ
ಪ್ರಯತ್ನ-ಗ-ಳಿಲ್ಲ-ದಿದ್ದರೆ
ಪ್ರಯತ್ನ-ಗಳು
ಪ್ರಯತ್ನ-ಗಳೂ
ಪ್ರಯತ್ನ-ಗ-ಳೆಲ್ಲ
ಪ್ರಯತ್ನದ
ಪ್ರಯತ್ನ-ದಂತೆಯೇ
ಪ್ರಯತ್ನ-ದಲ್ಲಿ
ಪ್ರಯತ್ನ-ದಿಂದ
ಪ್ರಯತ್ನ-ದಿಂದೇನು
ಪ್ರಯತ್ನ-ಪೂರ್ವ-ಕ-ವಾಗಿ
ಪ್ರಯತ್ನಪ್ರಾರ್ಥನೆ
ಪ್ರಯತ್ನ-ಮಾ-ಡಿ-ದರೂ
ಪ್ರಯತ್ನ-ವನ್ನು
ಪ್ರಯತ್ನ-ವನ್ನೇ
ಪ್ರಯತ್ನ-ವಿಲ್ಲದೆ
ಪ್ರಯತ್ನವು
ಪ್ರಯತ್ನವೂ
ಪ್ರಯತ್ನ-ವೆಂಬ
ಪ್ರಯತ್ನ-ವೆಲ್ಲ-ವನ್ನೂ
ಪ್ರಯತ್ನವೇ
ಪ್ರಯತ್ನ-ವೇಕೆ
ಪ್ರಯತ್ನ-ಶೀಲ
ಪ್ರಯತ್ನ-ಶೀ-ಲತೆ
ಪ್ರಯತ್ನ-ಶೀ-ಲ-ರಾ-ದು-ದೇ-ಆ-ಗಿದೆ
ಪ್ರಯತ್ನ-ಸು-ವು-ದ-ರಿಂದಷ್ಟೆ
ಪ್ರಯತ್ನಿ
ಪ್ರಯತ್ನಿ-ಸಿ-ದರೂ
ಪ್ರಯತ್ನಿ-ಸಿದ್ದಳು
ಪ್ರಯತ್ನಿಸ
ಪ್ರಯತ್ನಿ-ಸ-ದಿದ್ದರೆ
ಪ್ರಯತ್ನಿಸಿ
ಪ್ರಯತ್ನಿ-ಸಿ-ದನು
ಪ್ರಯತ್ನಿ-ಸಿ-ದರೂ
ಪ್ರಯತ್ನಿ-ಸಿ-ದರೆ
ಪ್ರಯತ್ನಿ-ಸಿದೆ
ಪ್ರಯತ್ನಿ-ಸಿ-ದೆ-ಸಾಧ್ಯ-ವಾ-ಯಿತು
ಪ್ರಯತ್ನಿ-ಸಿದ್ದಾ-ದರೆ
ಪ್ರಯತ್ನಿ-ಸಿದ್ದೀರಿ
ಪ್ರಯತ್ನಿ-ಸಿಯೂ
ಪ್ರಯತ್ನಿ-ಸುತ್ತಾರೆ
ಪ್ರಯತ್ನಿ-ಸುತ್ತಿ-ರುತ್ತಾನೆ
ಪ್ರಯತ್ನಿ-ಸುತ್ತಿ-ರು-ವ-ವ-ರೆಂಬು-ದನ್ನು
ಪ್ರಯತ್ನಿ-ಸುತ್ತೇವೆ
ಪ್ರಯತ್ನಿ-ಸು-ವು-ದ-ರಿಂದ
ಪ್ರಯತ್ನಿ-ಸು-ವುದು
ಪ್ರಯಾಣ
ಪ್ರಯಾಣದ
ಪ್ರಯಾ-ಣ-ದಲ್ಲಿ-ರು-ವಾಗ
ಪ್ರಯಾ-ಣ-ಮಾ-ಡು-ವಂತೆ
ಪ್ರಯುಕ್ತ
ಪ್ರಯೋಗ
ಪ್ರಯೋ-ಗ-ಅ-ನು-ಭ-ವ-ಪು-ರಸ್ಸ-ರ-ವಾಗಿ
ಪ್ರಯೋ-ಗಕ್ಕೊ-ಳ-ಪಟ್ಟ
ಪ್ರಯೋ-ಗ-ಗಳ
ಪ್ರಯೋ-ಗ-ಗ-ಳನ್ನು
ಪ್ರಯೋ-ಗ-ಗ-ಳಲ್ಲಿ
ಪ್ರಯೋ-ಗ-ಗ-ಳಿಂದ
ಪ್ರಯೋ-ಗ-ಗಳು
ಪ್ರಯೋಗದ
ಪ್ರಯೋ-ಗ-ದಲ್ಲಿ
ಪ್ರಯೋ-ಗ-ದಿಂದ
ಪ್ರಯೋ-ಗ-ಪ-ರಿ-ಶೀ-ಲ-ನೆ-ಗ-ಳನ್ನೊ-ಳ-ಗೊಂಡ
ಪ್ರಯೋ-ಗ-ವನ್ನು
ಪ್ರಯೋ-ಗ-ವನ್ನೇ
ಪ್ರಯೋ-ಗ-ಶಾಲೆ
ಪ್ರಯೋ-ಗ-ಶಾ-ಲೆಯ
ಪ್ರಯೋ-ಗ-ಶಾ-ಲೆ-ಯಲ್ಲಿ
ಪ್ರಯೋ-ಗ-ಶಾ-ಲೆ-ಯಿಂದ
ಪ್ರಯೋ-ಗ-ಶಾಸ್ತ್ರದ
ಪ್ರಯೋ-ಗಾತ್ಮಕ
ಪ್ರಯೋ-ಗಾತ್ಮ-ಕ-ವಾಗಿ
ಪ್ರಯೋ-ಗಾ-ಲ-ಯಕ್ಕೆ
ಪ್ರಯೋ-ಗಾ-ಲ-ಯದ
ಪ್ರಯೋ-ಗಿ-ಸದೆ
ಪ್ರಯೋಗಿಸಿ
ಪ್ರಯೋ-ಗಿ-ಸಿ-ದ-ವರೇ
ಪ್ರಯೋಜನ
ಪ್ರಯೋ-ಜ-ನ-ವಾ-ಗ-ಲಿಲ್ಲ
ಪ್ರಯೋ-ಜ-ನ-ವಿಲ್ಲ-ದ-ವನು
ಪ್ರಯೋ-ಜ-ನ-ಇವು
ಪ್ರಯೋ-ಜ-ನ-ಕಾ-ರಿ-ಯಾ-ಗ-ಲಾ-ರದು
ಪ್ರಯೋ-ಜ-ನ-ಕಾ-ರಿ-ಯಾದ
ಪ್ರಯೋ-ಜ-ನ-ವನ್ನು
ಪ್ರಯೋ-ಜ-ನ-ವಾ-ಗ-ದಿ-ರು-ವುದು
ಪ್ರಯೋ-ಜ-ನ-ವಾ-ಗ-ಲಾ-ರದು
ಪ್ರಯೋ-ಜ-ನ-ವಾ-ಗ-ಲಿಲ್ಲ
ಪ್ರಯೋ-ಜ-ನ-ವಾ-ಗಿ-ರ-ಲಿಲ್ಲ
ಪ್ರಯೋ-ಜ-ನ-ವಿಲ್ಲ
ಪ್ರಯೋ-ಜ-ನ-ವೇನು
ಪ್ರಲಾಪ
ಪ್ರಲೋ-ಭ-ನೆಯ
ಪ್ರಳಯಕ್ಕೆ
ಪ್ರಳ-ಯ-ವಾಗಿ
ಪ್ರವ-ಚ-ನ-ಕಾ-ರರ
ಪ್ರವ-ಚ-ನ-ಗ-ಳಾ-ಗ-ದಿ-ರ-ಬ-ಹುದು
ಪ್ರವ-ಚ-ನ-ದಲ್ಲಿ
ಪ್ರವರ್ತಕ
ಪ್ರವರ್ತ-ಕನ
ಪ್ರವರ್ತ-ಕ-ರನ್ನು
ಪ್ರವರ್ತ-ಕರು
ಪ್ರವರ್ತ-ಕ-ವಾ-ಗ-ಬಲ್ಲ
ಪ್ರವರ್ತ-ಕ-ವಾ-ಗ-ಬಲ್ಲದು
ಪ್ರವರ್ತಿ-ಸು-ತಿದ್ದಾನೆ
ಪ್ರವ-ಹಿ-ಸ-ಲಾ-ರಂಭಿಸಿ
ಪ್ರವಾದಿ
ಪ್ರವಾ-ದಿ-ಗಳು
ಪ್ರವಾ-ದಿ-ಗಳೂ
ಪ್ರವಾದಿಯ
ಪ್ರವಾ-ದಿ-ಯಲ್ಲಿ
ಪ್ರವಾಸ
ಪ್ರವಾಸಕ್ಕೆ
ಪ್ರವಾ-ಸ-ದಿಂದ
ಪ್ರವಾ-ಸಾ-ನು-ಭ-ವ-ವನ್ನು
ಪ್ರವಾ-ಸಿ-ಗಳ
ಪ್ರವಾ-ಸಿ-ಗಳೂ
ಪ್ರವಾಸೀ
ಪ್ರವಾಹ
ಪ್ರವಾಹದ
ಪ್ರವಾ-ಹ-ದಲ್ಲಿ
ಪ್ರವಾ-ಹ-ದಲ್ಲೂ
ಪ್ರವಾ-ಹ-ದಲ್ಲೇ
ಪ್ರವಾ-ಹ-ದಿಂದ
ಪ್ರವಾ-ಹ-ವನ್ನು
ಪ್ರವಾ-ಹ-ವನ್ನೇ
ಪ್ರವಾ-ಹಾ-ಕಾ-ರ-ವಾಗಿ
ಪ್ರವಾ-ಹಾ-ಕಾ-ರ-ವಾದ
ಪ್ರವೀಣ
ಪ್ರವೃತ್ತಿ
ಪ್ರವೃತ್ತಿ-ಗಳ
ಪ್ರವೃತ್ತಿ-ಗ-ಳನ್ನು
ಪ್ರವೃತ್ತಿ-ಗ-ಳಿಂದ
ಪ್ರವೃತ್ತಿ-ಗ-ಳಿಗೆ
ಪ್ರವೃತ್ತಿ-ಗಳು
ಪ್ರವೃತ್ತಿಯ
ಪ್ರವೃತ್ತಿ-ಯನ್ನು
ಪ್ರವೃತ್ತಿ-ಯನ್ನೂ
ಪ್ರವೃತ್ತಿ-ಯಲ್ಲಿ
ಪ್ರವೃತ್ತಿ-ಯ-ವರು
ಪ್ರವೃತ್ತಿ-ಯಾ-ಗದೇ
ಪ್ರವೃತ್ತಿ-ಯುಳ್ಳ
ಪ್ರವೃತ್ತಿ-ಯುಳ್ಳ-ವರೂ
ಪ್ರವೃತ್ತಿಯೂ
ಪ್ರವೃತ್ತಿ-ಯೊಂದು
ಪ್ರವೇಶ
ಪ್ರವೇಶದ
ಪ್ರವೇ-ಶ-ವಿ-ರದ
ಪ್ರವೇ-ಶಿ-ಸ-ದಂತೆ
ಪ್ರವೇ-ಶಿ-ಸ-ಬಾ-ರದೇ
ಪ್ರವೇ-ಶಿ-ಸ-ಬಿ-ಡು-ವುದು
ಪ್ರವೇಶಿಸಿ
ಪ್ರವೇ-ಶಿ-ಸಿತು
ಪ್ರವೇ-ಶಿ-ಸಿತ್ತು
ಪ್ರವೇ-ಶಿ-ಸಿ-ದರು
ಪ್ರವೇ-ಶಿ-ಸಿ-ದರೂ
ಪ್ರವೇ-ಶಿ-ಸಿ-ದರೆ
ಪ್ರವೇ-ಶಿ-ಸಿ-ದಾಗ
ಪ್ರವೇ-ಶಿ-ಸಿದೆ
ಪ್ರವೇ-ಶಿ-ಸಿಯೇ
ಪ್ರವೇ-ಶಿ-ಸಿ-ರ-ಬ-ಹುದು
ಪ್ರವೇ-ಶಿ-ಸಿ-ರುತ್ತದೆ
ಪ್ರವೇ-ಶಿ-ಸಿ-ರುವ
ಪ್ರವೇ-ಶಿ-ಸುತ್ತದೆ
ಪ್ರವೇ-ಶಿ-ಸುತ್ತವೆ
ಪ್ರವೇ-ಶಿ-ಸುತ್ತಾನೆ
ಪ್ರವೇ-ಶಿ-ಸುವ
ಪ್ರವೇ-ಶಿ-ಸು-ವ-ವ-ರೆ-ಗಿನ
ಪ್ರವೇ-ಶಿ-ಸು-ವು-ದಕ್ಕಿಂತ
ಪ್ರವೇ-ಶಿ-ಸು-ವು-ದಕ್ಕೆ
ಪ್ರವೇ-ಶಿ-ಸು-ವುದು
ಪ್ರಶಂಸಿ-ಸ-ಬೇಕು
ಪ್ರಶಂಸಿ-ಸುತ್ತಾನೆ
ಪ್ರಶಂಸಿ-ಸು-ವು-ದುಈ
ಪ್ರಶಂಸೆ
ಪ್ರಶಂಸೆ-ಗ-ಳನ್ನೂ
ಪ್ರಶಂಸೆಯ
ಪ್ರಶಸ್ತ-ವಾದ
ಪ್ರಶಸ್ತಿ
ಪ್ರಶಸ್ತಿ-ಗ-ಳನ್ನೂ
ಪ್ರಶಾಂತ
ಪ್ರಶ್ನ-ಪತ್ರಿ-ಕೆ-ಗ-ಳನ್ನು
ಪ್ರಶ್ನ-ಪತ್ರಿ-ಕೆ-ಯನ್ನು
ಪ್ರಶ್ನಿ
ಪ್ರಶ್ನಿ-ಸ-ತೊ-ಡ-ಗಿ-ದರು
ಪ್ರಶ್ನಿ-ಸ-ಬ-ಹುದು
ಪ್ರಶ್ನಿ-ಸ-ಲಾ-ಯಿತು
ಪ್ರಶ್ನಿಸಿ
ಪ್ರಶ್ನಿಸಿದ
ಪ್ರಶ್ನಿ-ಸಿ-ದರು
ಪ್ರಶ್ನಿ-ಸಿ-ದ-ರು-ಏನು
ಪ್ರಶ್ನಿ-ಸಿ-ದ-ರು-ನಿಮ್ಮ
ಪ್ರಶ್ನಿ-ಸಿ-ದರೆ
ಪ್ರಶ್ನಿ-ಸಿ-ದಳು
ಪ್ರಶ್ನಿ-ಸಿ-ದಾಗ
ಪ್ರಶ್ನಿಸಿದೆ
ಪ್ರಶ್ನಿ-ಸಿದ್ದರು
ಪ್ರಶ್ನೆ
ಪ್ರಶ್ನೆಗಳ
ಪ್ರಶ್ನೆ-ಗ-ಳನ್ನು
ಪ್ರಶ್ನೆ-ಗ-ಳಿಗೆ
ಪ್ರಶ್ನೆ-ಗ-ಳಿ-ಗೆಲ್ಲ
ಪ್ರಶ್ನೆ-ಗ-ಳಿ-ವೆಯೆ
ಪ್ರಶ್ನೆಗಳು
ಪ್ರಶ್ನೆ-ಗುತ್ತ-ರ-ವಾಗಿ
ಪ್ರಶ್ನೆಗೆ
ಪ್ರಶ್ನೆಯ
ಪ್ರಶ್ನೆಯನ್ನು
ಪ್ರಶ್ನೆಯೂ
ಪ್ರಶ್ನೆಯೇ
ಪ್ರಶ್ನೆ-ಸೂರ್ಯ-ಚಂದ್ರರೂ
ಪ್ರಸಂಗ
ಪ್ರಸಂಗ-ಗ-ಳನ್ನು
ಪ್ರಸಂಗ-ಗಳೂ
ಪ್ರಸಂಗ-ದಿಂದ
ಪ್ರಸಂಗ-ವನ್ನವು
ಪ್ರಸಂಗವೇ
ಪ್ರಸಕ್ತ
ಪ್ರಸನ್ನತೆ
ಪ್ರಸನ್ನ-ವಾ-ಗಿಯೇ
ಪ್ರಸನ್ನ-ವಾದ
ಪ್ರಸವ
ಪ್ರಸ-ವಿ-ಸಿದ
ಪ್ರಸಾರ
ಪ್ರಸಾ-ರ-ಇವು
ಪ್ರಸಾ-ರಕ್ಕಾ-ಗಿ-ರುವ
ಪ್ರಸಾ-ರ-ಗೊ-ಳಿ-ಸ-ಬೇ-ಕೆಂದು
ಪ್ರಸಾ-ರ-ದಲ್ಲಿ
ಪ್ರಸಾ-ರ-ದಿಂದ
ಪ್ರಸಾ-ರ-ಮಾ-ಡಲು
ಪ್ರಸಾ-ರ-ಮಾ-ಡುತ್ತಿ-ರುವ
ಪ್ರಸಾ-ರ-ವಾಗಿ
ಪ್ರಸಾ-ರ-ವಾ-ಗಿಯೇ
ಪ್ರಸಾ-ರ-ವಾ-ಗುತ್ತ-ಲಿದೆ
ಪ್ರಸಾ-ರ-ವಾ-ಗುತ್ತಿದೆ
ಪ್ರಸಾ-ರ-ವಾ-ಗುತ್ತಿ-ರ-ಬ-ಹುದು
ಪ್ರಸಾ-ರ-ವಾ-ಗುತ್ತಿ-ವೆ-ಎಂಬು-ದನ್ನೆಲ್ಲ
ಪ್ರಸಿದ್ಧ
ಪ್ರಸಿದ್ಧ-ನಾ-ಗಿದ್ದ
ಪ್ರಸಿದ್ಧ-ನಾದ
ಪ್ರಸಿದ್ಧಬ್ಯಾಂಕ-ರರೂ
ಪ್ರಸಿದ್ಧ-ರಾ-ಗಿದ್ದಾರೆ
ಪ್ರಸಿದ್ಧ-ರಾದ
ಪ್ರಸಿದ್ಧ-ವಾ-ಗಿದೆ
ಪ್ರಸಿದ್ಧ-ವಾದ
ಪ್ರಸಿದ್ಧಿ
ಪ್ರಸಿದ್ಧಿ-ಗ-ಳನ್ನು
ಪ್ರಸಿದ್ಧಿಗೆ
ಪ್ರಸಿದ್ಧಿ-ಯನ್ನು
ಪ್ರಸಿದ್ಧಿ-ಯಾ-ಗು-ವು-ದಿಲ್ಲ
ಪ್ರಸ್ತಾಪವೂ
ಪ್ರಸ್ತಾ-ಪಿ-ಸ-ದಿದ್ದರೂ
ಪ್ರಸ್ತಾ-ಪಿ-ಸಿದ್ದೇನೆ
ಪ್ರಸ್ತಾ-ಪಿ-ಸಿಲ್ಲ
ಪ್ರಸ್ತಾ-ಪಿ-ಸುತ್ತ
ಪ್ರಸ್ತಾ-ಪಿ-ಸುತ್ತಿದ್ದರು
ಪ್ರಸ್ತುತ
ಪ್ರಸ್ತು-ತ-ಪ-ಡಿ-ಸಿದ
ಪ್ರಸ್ತು-ತ-ಪ-ಡಿ-ಸಿದ್ದೇನೆ
ಪ್ರಹ್ಲಾದ
ಪ್ರಾಂಜ-ಲ-ವಾಗಿ
ಪ್ರಾಂತ
ಪ್ರಾಂತ-ಗ-ಳಲ್ಲಿ
ಪ್ರಾಂತದ
ಪ್ರಾಂತದಲ್ಲಿ
ಪ್ರಾಂತ್ಯದ
ಪ್ರಾಕೃ-ತ-ರು-ಎಂದರೆ
ಪ್ರಾಕೃತಿಕ
ಪ್ರಾಕ್ಟೀಸ್
ಪ್ರಾಕ್ತನ
ಪ್ರಾಚೀನ
ಪ್ರಾಚೀ-ನ-ಕಾ-ಲ-ದಲ್ಲೂ
ಪ್ರಾಚೀನತೆ
ಪ್ರಾಜ್ಞ
ಪ್ರಾಜ್ಞನಲ್ಲೂ
ಪ್ರಾಜ್ಞ-ರಾ-ಗುತ್ತೇವೆ
ಪ್ರಾಣ
ಪ್ರಾಣದಾನ
ಪ್ರಾಣ-ಧಾ-ರಣೆ
ಪ್ರಾಣಪಕ್ಷಿ
ಪ್ರಾಣ-ಪಕ್ಷಿಯು
ಪ್ರಾಣಪ್ರ-ತಿಷ್ಠೆ
ಪ್ರಾಣಬಿಟ್ಟ
ಪ್ರಾಣ-ಬಿಟ್ಟರು
ಪ್ರಾಣ-ರಕ್ಷ-ಣೆ-ಗಾಗಿ
ಪ್ರಾಣವನ್ನು
ಪ್ರಾಣವಾಯು
ಪ್ರಾಣವೇ
ಪ್ರಾಣಾ
ಪ್ರಾಣಾಂತಿಕ
ಪ್ರಾಣಾ-ಪಾ-ಯ-ದಿಂದ
ಪ್ರಾಣಾ-ಪಾ-ಯವೂ
ಪ್ರಾಣಿ
ಪ್ರಾಣಿ-ಯಾ-ಗಲೀ
ಪ್ರಾಣಿಗಳ
ಪ್ರಾಣಿ-ಗ-ಳನ್ನು
ಪ್ರಾಣಿ-ಗ-ಳಲ್ಲಾ-ಗಲೀ
ಪ್ರಾಣಿ-ಗ-ಳಲ್ಲಿ
ಪ್ರಾಣಿ-ಗ-ಳಲ್ಲೂ
ಪ್ರಾಣಿ-ಗ-ಳಿಗೂ
ಪ್ರಾಣಿ-ಗ-ಳಿಗೆ
ಪ್ರಾಣಿಗಳು
ಪ್ರಾಣಿಗಳೇ
ಪ್ರಾಣಿ-ಗ-ಳೊ-ಡನೆ
ಪ್ರಾಣಿ-ಗಿಂತಲೂ
ಪ್ರಾಣಿ-ಗೃ-ಹ-ಗ-ಳಿವೆ
ಪ್ರಾಣಿ-ಗೃ-ಹ-ಗಳೂ
ಪ್ರಾಣಿಗೆ
ಪ್ರಾಣಿಯ
ಪ್ರಾಣಿಯನ್ನು
ಪ್ರಾಣಿ-ಯಾ-ಗುತ್ತಾನೆ
ಪ್ರಾಣಿ-ಶಾಸ್ತ್ರದ
ಪ್ರಾಣಿಹಿತ
ಪ್ರಾಣೋತ್ಕ್ರ-ಮಣ
ಪ್ರಾತಃಕಾಲ
ಪ್ರಾತಃಸ್ಮ-ರ-ಣೀ-ಯ-ನಾದ
ಪ್ರಾಥಮಿಕ
ಪ್ರಾದುರ್ಭಾ-ವಕ್ಕೆ
ಪ್ರಾಧಾನ್ಯ
ಪ್ರಾಧಾನ್ಯ-ವನ್ನು
ಪ್ರಾಧಾನ್ಯ-ವಿತ್ತು
ಪ್ರಾಧ್ಯಾಪಕ
ಪ್ರಾಧ್ಯಾ-ಪ-ಕ-ರಾಗಿ
ಪ್ರಾಧ್ಯಾ-ಪ-ಕ-ರಾ-ಗಿದ್ದರು
ಪ್ರಾಧ್ಯಾ-ಪ-ಕರು
ಪ್ರಾಧ್ಯಾ-ಪ-ಕ-ರು-ಗಳು
ಪ್ರಾಧ್ಯಾ-ಪ-ಕರೇ
ಪ್ರಾಧ್ಯಾ-ಪ-ಕ-ರೊಬ್ಬರು
ಪ್ರಾಪಂಚಿಕ
ಪ್ರಾಪಂಚಿ-ಕ-ನಾಗಿ
ಪ್ರಾಪಂಚಿ-ಕ-ವಾದ
ಪ್ರಾಪ್ತ-ವಾ-ಯಿತು
ಪ್ರಾಪ್ತ-ವಾ-ಗ-ತೊ-ಡ-ಗಿತು
ಪ್ರಾಪ್ತ-ವಾ-ಗಿದೆ
ಪ್ರಾಪ್ತ-ವಾ-ಗು-ವುದು
ಪ್ರಾಪ್ತವಾದ
ಪ್ರಾಪ್ತ-ವಾ-ದು-ದಲ್ಲ-ಎಂಬು-ದನ್ನು
ಪ್ರಾಪ್ತಿಗೆ
ಪ್ರಾಮಾಣಿಕ
ಪ್ರಾಮಾ-ಣಿ-ಕತೆ
ಪ್ರಾಮಾ-ಣಿ-ಕ-ತೆಯ
ಪ್ರಾಮಾ-ಣಿ-ಕ-ತೆ-ಯನ್ನು
ಪ್ರಾಮಾ-ಣಿ-ಕ-ತೆ-ಯಿಂದ
ಪ್ರಾಮಾ-ಣಿ-ಕ-ನಲ್ಲ
ಪ್ರಾಮಾ-ಣಿ-ಕ-ನಾಗಿ
ಪ್ರಾಮಾ-ಣಿ-ಕ-ನಾ-ಗಿ-ರುತ್ತಾನೆ
ಪ್ರಾಮಾ-ಣಿ-ಕ-ನಾಗು
ಪ್ರಾಮಾ-ಣಿ-ಕ-ರಾಗಿ
ಪ್ರಾಮಾ-ಣಿ-ಕರೂ
ಪ್ರಾಮಾ-ಣಿ-ಕ-ವಾಗಿ
ಪ್ರಾಮಾ-ಣಿ-ಕ-ವಾ-ಗಿದ್ದರೆ
ಪ್ರಾಮಾ-ಣಿ-ಕ-ವಾದ
ಪ್ರಾಮುಖ್ಯ
ಪ್ರಾಯಃ
ಪ್ರಾಯಕ್ಕೆ
ಪ್ರಾಯದ
ಪ್ರಾಯಪ್ರ-ಬುದ್ಧ-ನಿಗೆ
ಪ್ರಾಯಶಃ
ಪ್ರಾಯಶ್ಚಿತ್ತ-ವನ್ನು
ಪ್ರಾಯಸ್ಥ
ಪ್ರಾಯೋಗಿಕ
ಪ್ರಾರಂಭ-ವಾ-ಯಿತು
ಪ್ರಾರಂಭ-ಗೊಂಡು
ಪ್ರಾರಂಭದ
ಪ್ರಾರಂಭ-ದಲ್ಲಿ
ಪ್ರಾರಂಭ-ದಲ್ಲೇ
ಪ್ರಾರಂಭ-ದಿಂದಲೇ
ಪ್ರಾರಂಭ-ವಾ-ಗಿತ್ತು
ಪ್ರಾರಂಭ-ವಾ-ಗಿದೆ
ಪ್ರಾರಂಭ-ವಾ-ಗುತ್ತದೆ
ಪ್ರಾರಂಭ-ವಾ-ದದ್ದು
ಪ್ರಾರಂಭ-ವಾ-ಯಿತು
ಪ್ರಾರಂಭಿಸ
ಪ್ರಾರಂಭಿ-ಸ-ಬ-ಹುದು
ಪ್ರಾರಂಭಿ-ಸ-ಬೇ-ಕಾ-ಗಿದೆ
ಪ್ರಾರಂಭಿ-ಸ-ಬೇಕು
ಪ್ರಾರಂಭಿ-ಸ-ಬೇಡಿ
ಪ್ರಾರಂಭಿಸಿ
ಪ್ರಾರಂಭಿ-ಸಿತು
ಪ್ರಾರಂಭಿ-ಸಿದ
ಪ್ರಾರಂಭಿ-ಸಿ-ದರು
ಪ್ರಾರಂಭಿ-ಸಿ-ದ-ರೇನು
ಪ್ರಾರಂಭಿ-ಸಿ-ದಳು
ಪ್ರಾರಂಭಿ-ಸಿ-ದ-ವಳು
ಪ್ರಾರಂಭಿ-ಸಿ-ದೊ-ಡನೆ
ಪ್ರಾರಂಭಿ-ಸಿದ್ದ
ಪ್ರಾರಂಭಿ-ಸುತ್ತವೆ
ಪ್ರಾರಂಭಿ-ಸುತ್ತಾರೆ
ಪ್ರಾರಂಭಿ-ಸುತ್ತಿದ್ದೀರಿ
ಪ್ರಾರಂಭಿ-ಸು-ವಾ-ಗಲೇ
ಪ್ರಾರಂಭಿ-ಸು-ವು-ದಾ-ದಲ್ಲಿ
ಪ್ರಾರಂಭಿ-ಸು-ವುದು
ಪ್ರಾರಬ್ಧ
ಪ್ರಾರಬ್ಧ-ವನ್ನು
ಪ್ರಾರ್ಥನಾ
ಪ್ರಾರ್ಥ-ನಾ-ಧರ್ಮಕ್ಕೆ
ಪ್ರಾರ್ಥ-ನಾ-ಶೀಲ
ಪ್ರಾರ್ಥ-ನಾ-ಶೀ-ಲರೂ
ಪ್ರಾರ್ಥನೆ
ಪ್ರಾರ್ಥ-ನೆ-ಇ-ವು-ಗ-ಳಿಂದ
ಪ್ರಾರ್ಥ-ನೆ-ಗ-ಳನ್ನು
ಪ್ರಾರ್ಥ-ನೆ-ಗ-ಳಲ್ಲಿ
ಪ್ರಾರ್ಥ-ನೆ-ಗ-ಳಿಂದ
ಪ್ರಾರ್ಥ-ನೆ-ಗ-ಳಿ-ಗಾಗಿ
ಪ್ರಾರ್ಥ-ನೆ-ಗಳು
ಪ್ರಾರ್ಥನೆಗೆ
ಪ್ರಾರ್ಥನೆಯ
ಪ್ರಾರ್ಥ-ನೆ-ಯನ್ನು
ಪ್ರಾರ್ಥ-ನೆ-ಯನ್ನೂ
ಪ್ರಾರ್ಥ-ನೆ-ಯನ್ನೆ
ಪ್ರಾರ್ಥ-ನೆ-ಯಲ್ಲಿ
ಪ್ರಾರ್ಥ-ನೆ-ಯಲ್ಲಿದೆ
ಪ್ರಾರ್ಥ-ನೆ-ಯಿಂದ
ಪ್ರಾರ್ಥ-ನೆ-ಯಿಂದಲೇ
ಪ್ರಾರ್ಥ-ನೆ-ಯಿಂದಾಗಿ
ಪ್ರಾರ್ಥನೆಯು
ಪ್ರಾರ್ಥನೆಯೇ
ಪ್ರಾರ್ಥ-ನೆ-ಯೊಂದಿಗೆ
ಪ್ರಾರ್ಥ-ನೆ-ಯೊಂದೇ
ಪ್ರಾರ್ಥ-ನೆಸ್ವಲ್ಪ-ವಾ-ದರೂ
ಪ್ರಾರ್ಥಿ-ಸ-ತೊ-ಡ-ಗಿ-ದಳು
ಪ್ರಾರ್ಥಿ-ಸ-ಬಲ್ಲನು
ಪ್ರಾರ್ಥಿ-ಸ-ಬಲ್ಲಿರಾ
ಪ್ರಾರ್ಥಿ-ಸ-ಬಲ್ಲೆಯಾ
ಪ್ರಾರ್ಥಿ-ಸ-ಬೇಕು
ಪ್ರಾರ್ಥಿ-ಸ-ಲಾರೆ
ಪ್ರಾರ್ಥಿಸಲು
ಪ್ರಾರ್ಥಿಸಿ
ಪ್ರಾರ್ಥಿ-ಸಿ-ಕೊಳ್ಳ-ಬೇಕು
ಪ್ರಾರ್ಥಿಸಿದ
ಪ್ರಾರ್ಥಿ-ಸಿ-ದರು
ಪ್ರಾರ್ಥಿ-ಸಿ-ದರೂ
ಪ್ರಾರ್ಥಿ-ಸಿ-ದರೆ
ಪ್ರಾರ್ಥಿ-ಸಿ-ದ-ರೆ-ತೀವ್ರ
ಪ್ರಾರ್ಥಿ-ಸಿ-ದಳು
ಪ್ರಾರ್ಥಿ-ಸಿ-ದಾ-ಗ-ಲೆಲ್ಲ
ಪ್ರಾರ್ಥಿಸಿದೆ
ಪ್ರಾರ್ಥಿಸಿದ್ದು
ಪ್ರಾರ್ಥಿಸಿಯೂ
ಪ್ರಾರ್ಥಿಸು
ಪ್ರಾರ್ಥಿಸುತ್ತ
ಪ್ರಾರ್ಥಿ-ಸುತ್ತಾರೆ
ಪ್ರಾರ್ಥಿ-ಸುತ್ತಿದ್ದ
ಪ್ರಾರ್ಥಿ-ಸುತ್ತೇವೆ
ಪ್ರಾರ್ಥಿಸುವ
ಪ್ರಾರ್ಥಿ-ಸು-ವಂತಾ-ದರೆ
ಪ್ರಾರ್ಥಿ-ಸು-ವ-ವನು
ಪ್ರಾರ್ಥಿ-ಸು-ವಾ-ಗ-ಲೆಲ್ಲ
ಪ್ರಾರ್ಥಿ-ಸು-ವಾ-ಗಿ-ನಷ್ಟೇ
ಪ್ರಾರ್ಥಿ-ಸು-ವುದು
ಪ್ರಾಶಸ್ತ್ಯ-ವೀ-ಯ-ಲಿಲ್ಲ
ಪ್ರಾಸಬದ್ಧ
ಪ್ರಿಯ
ಪ್ರಿಯದರ್ಶಿ
ಪ್ರಿಯನು
ಪ್ರಿಯವಾದ
ಪ್ರಿಯಸ್ವ-ಭಾವಿ
ಪ್ರಿಯಸ್ವ-ಭಾ-ವಿಯೂ
ಪ್ರಿಸೀ-ಡಿ-ಯಮ್
ಪ್ರೀತಿ
ಪ್ರೀತಿ-ಸು-ವನು
ಪ್ರೀತಿಇವು
ಪ್ರೀತಿ-ಗ-ಳನ್ನು
ಪ್ರೀತಿಗಾಗಿ
ಪ್ರೀತಿ-ಗಾ-ಗಿಯೇ
ಪ್ರೀತಿಗಿದೆ
ಪ್ರೀತಿಗೂ
ಪ್ರೀತಿಗೆ
ಪ್ರೀತಿ-ಗೆಂತಹ
ಪ್ರೀತಿಗೇ
ಪ್ರೀತಿಪಾತ್ರ
ಪ್ರೀತಿ-ಪಾತ್ರ-ನಿ-ಗಾಗಿ
ಪ್ರೀತಿ-ಪಾತ್ರರ
ಪ್ರೀತಿ-ಪಾತ್ರ-ವಾದ
ಪ್ರೀತಿಯ
ಪ್ರೀತಿ-ಯನ್ನಲ್ಲ
ಪ್ರೀತಿ-ಯನ್ನಾ-ದರೂ
ಪ್ರೀತಿಯನ್ನು
ಪ್ರೀತಿಯನ್ನೂ
ಪ್ರೀತಿಯನ್ನೋ
ಪ್ರೀತಿ-ಯಲ್ಲಿದೆ
ಪ್ರೀತಿಯಿಂದ
ಪ್ರೀತಿಯು
ಪ್ರೀತಿಯೂ
ಪ್ರೀತಿಯೆ
ಪ್ರೀತಿಯೆಂಬ
ಪ್ರೀತಿ-ಯೆಂಬುದು
ಪ್ರೀತಿಯೇ
ಪ್ರೀತಿ-ಯೊಂದನ್ನು
ಪ್ರೀತಿಯೊಂದೇ
ಪ್ರೀತಿ-ವಾತ್ಸಲ್ಯ
ಪ್ರೀತಿ-ವಿಶ್ವಾ-ಸ-ದಿಂದ
ಪ್ರೀತಿ-ವಿಶ್ವಾ-ಸ-ದಿಂದಲೇ
ಪ್ರೀತಿ-ಸದ್ಭಾ-ವನೆ
ಪ್ರೀತಿ-ಸ-ಬಲ್ಲ-ವನೇ
ಪ್ರೀತಿ-ಸ-ಬೇಕು
ಪ್ರೀತಿ-ಸ-ಲಾ-ರ-ಳಾ-ದರೆ
ಪ್ರೀತಿಸಲು
ಪ್ರೀತಿ-ಸಲ್ಪ-ಡು-ವು-ದು-ನ-ವ-ಜಾ-ತ-ಶಿ-ಶು-ಗಳು
ಪ್ರೀತಿ-ಸಲ್ಪ-ಡುವ
ಪ್ರೀತಿ-ಸಲ್ಪ-ಡು-ವು-ದು-ಒಂದು
ಪ್ರೀತಿಸಿ
ಪ್ರೀತಿ-ಸಿ-ದನೇ
ಪ್ರೀತಿ-ಸಿ-ದ-ವ-ನಲ್ಲ
ಪ್ರೀತಿಸಿರಿ
ಪ್ರೀತಿ-ಸುತ್ತಾನೆ
ಪ್ರೀತಿ-ಸುತ್ತಿದ್ದರೂ
ಪ್ರೀತಿ-ಸುತ್ತಿದ್ದೀಯಾ
ಪ್ರೀತಿ-ಸುತ್ತಿದ್ದೇನೆ
ಪ್ರೀತಿ-ಸುತ್ತಿ-ಹನೋ
ಪ್ರೀತಿ-ಸುತ್ತೇನೆ
ಪ್ರೀತಿಸುವ
ಪ್ರೀತಿ-ಸು-ವ-ನ-ವನು
ಪ್ರೀತಿ-ಸು-ವ-ವ-ನಿಂದ
ಪ್ರೀತಿ-ಸು-ವ-ವರು
ಪ್ರೀತಿ-ಸು-ವಿ-ರೇನು
ಪ್ರೀತಿ-ಸು-ವಿರೊ
ಪ್ರೀತಿ-ಸು-ವು-ದ-ರಲ್ಲಿ
ಪ್ರೀತಿ-ಸು-ವುದು
ಪ್ರೀತಿ-ಸು-ವು-ದೆಂದರೆ
ಪ್ರೀತ್ಯಾ-ದ-ರ-ಗ-ಳನ್ನು
ಪ್ರೀತ್ಯಾ-ದ-ರ-ಗ-ಳಿಂದ
ಪ್ರೆಸ್
ಪ್ರೆಸ್ಸನ್ನು
ಪ್ರೆಸ್ಸನ್ನೂ
ಪ್ರೇಕ್ಷಕ
ಪ್ರೇಕ್ಷಕನ
ಪ್ರೇತ
ಪ್ರೇತಗಳ
ಪ್ರೇತ-ಗ-ಳಲ್ಲೂ
ಪ್ರೇತ-ಗ-ಳಿದ್ದವು
ಪ್ರೇತ-ಗ-ಳಿ-ರ-ಬ-ಹುದು
ಪ್ರೇತಗಳು
ಪ್ರೇತದ
ಪ್ರೇತವಾಗಿ
ಪ್ರೇಮ
ಪ್ರೇಮ-ಗ-ಳಾ-ದರೋ
ಪ್ರೇಮದ
ಪ್ರೇಮದಲ್ಲಿ
ಪ್ರೇಮದಿಂದ
ಪ್ರೇಮ-ದಿಂದಲೇ
ಪ್ರೇಮ-ದಿಂದು-ದಿ-ಸಿದ
ಪ್ರೇಮಮಯ
ಪ್ರೇಮ-ಮ-ಯ-ನಾದ
ಪ್ರೇಮವಲ್ಲ
ಪ್ರೇಮ-ವೆಂದೆ-ಣಿಸಿ
ಪ್ರೇಮವೇ
ಪ್ರೇಮವೊಂದೇ
ಪ್ರೇಮಸ್ವ-ರೂಪ
ಪ್ರೇಮಾಶ್ರು
ಪ್ರೇಮಾಶ್ರು-ವನ್ನು
ಪ್ರೇಮಾಶ್ರು-ವಿ-ನಿಂದ
ಪ್ರೇಮಿ-ಗ-ಳಲ್ಲಿ-ರಲಿ
ಪ್ರೇಮಿಗಳು
ಪ್ರೇಮಿಯಾದ
ಪ್ರೇರಕ
ಪ್ರೇರ-ಕ-ವಾ-ಗಿವೆ
ಪ್ರೇರ-ಕ-ವಾ-ಗುತ್ತದೆ
ಪ್ರೇರ-ಕ-ವಾ-ಗುವ
ಪ್ರೇರ-ಕ-ವಾ-ಗು-ವಂಥ
ಪ್ರೇರ-ಕ-ವಾದ
ಪ್ರೇರ-ಕ-ವಾ-ದರೆ
ಪ್ರೇರಕವೂ
ಪ್ರೇರ-ಕ-ಶಕ್ತಿ
ಪ್ರೇರ-ಕ-ಶಕ್ತಿ-ಯಾ-ಗುತ್ತದೆ
ಪ್ರೇರ-ಕ-ಶಕ್ತಿಯು
ಪ್ರೇರಣೆ
ಪ್ರೇರಣೆಗೆ
ಪ್ರೇರ-ಣೆ-ಯಂತೆ
ಪ್ರೇರ-ಣೆ-ಯನ್ನೀ-ಯುತ್ತಿದೆ
ಪ್ರೇರ-ಣೆ-ಯನ್ನು
ಪ್ರೇರ-ಣೆ-ಯಾಗಿ
ಪ್ರೇರ-ಣೆ-ಯಿಂದ
ಪ್ರೇರಿ-ತ-ನಾಗಿ
ಪ್ರೇರಿ-ತ-ರಾಗಿ
ಪ್ರೇರಿ-ತ-ರಾದ
ಪ್ರೇರಿ-ತ-ವಲ್ಲ
ಪ್ರೇರಿ-ತ-ವಾ-ಗಿ-ರ-ಬೇಕು
ಪ್ರೇರಿ-ಸ-ದಿದ್ದರೂ
ಪ್ರೇರಿ-ಸ-ದಿದ್ದಲ್ಲಿ
ಪ್ರೇರಿ-ಸ-ಲಾ-ರರು
ಪ್ರೇರಿಸಲು
ಪ್ರೇರಿಸಿ
ಪ್ರೇರಿಸಿತು
ಪ್ರೇರಿಸಿದ
ಪ್ರೇರಿಸಿದೆ
ಪ್ರೇರಿ-ಸಿದ್ದರು
ಪ್ರೇರಿ-ಸಿ-ರ-ಬ-ಹುದು
ಪ್ರೇರಿಸಿವೆ
ಪ್ರೇರಿ-ಸುತ್ತದೆ
ಪ್ರೇರಿಸುವ
ಪ್ರೇರಿ-ಸು-ವಂತಿ-ರ-ಬೇಕು
ಪ್ರೇರೇ-ಪಿ-ಸಿ-ದರು
ಪ್ರೇರೇ-ಪಿ-ಸೀತು
ಪ್ರೇಷಣೆಯ
ಪ್ರೊ
ಪ್ರೊಟೆಸ್ಟೆಂಟ್
ಪ್ರೊಫೆಸರ್
ಪ್ರೋಟಾನು
ಪ್ರೋಟಾನ್
ಪ್ರೋತ್ಸಾಹ
ಪ್ರೋತ್ಸಾಹಕ್ಕೂ
ಪ್ರೋತ್ಸಾ-ಹ-ಗಳ
ಪ್ರೋತ್ಸಾ-ಹ-ಗ-ಳಿಂದ
ಪ್ರೋತ್ಸಾಹದ
ಪ್ರೋತ್ಸಾ-ಹ-ವನ್ನು
ಪ್ರೋತ್ಸಾ-ಹಿ-ಸ-ಬಲ್ಲಿರಾ
ಪ್ರೋತ್ಸಾ-ಹಿ-ಸಿ-ದರು
ಪ್ರೋತ್ಸಾ-ಹಿ-ಸಿ-ದರೆ
ಪ್ರೋತ್ಸಾ-ಹಿ-ಸುತ್ತಿದ್ದರು
ಪ್ರೌಢ-ಶಾ-ಲೆ-ಯಲ್ಲಿ-ರಲಿ
ಪ್ರ್ಯಾಕ್ಟೀಸ್
ಪ್ಲಂ
ಪ್ಲಾಟಿನಸ್
ಪ್ಲಾಸ್ಟಿಕ್
ಪ್ಲಾಸ್ಟಿಕ್ನ
ಪ್ಲಾಸ್ಮಾಬಾಡಿ
ಪ್ಲುಟಾರ್ಕ್
ಪ್ಲುಟೋ-ನಿ-ಯಂನಿಂದ
ಪ್ಲೇಗ್ನ
ಪ್ಲೇಟೋ
ಪ್ಲೇಟ್ಲೆಟ್ಗಳೂ
ಪ್ಲೋರಿಡಾ
ಫರ್
ಫರ್ಲಾಂಗು
ಫರ್ಸ್ಟ್
ಫಲ
ಫಲಕ
ಫಲಕದ
ಫಲಕಾರಿ
ಫಲ-ಕಾ-ರಿ-ಯಾ-ಗ-ದಾಗ
ಫಲ-ಕಾ-ರಿ-ಯಾ-ಗ-ಬಲ್ಲುದು
ಫಲ-ಕಾ-ರಿ-ಯಾ-ಗಲು
ಫಲ-ಕಾ-ರಿ-ಯಾ-ಗು-ವುದು
ಫಲ-ಕೊ-ಡು-ವಂತೆ
ಫಲಕ್ಕಾಗಿ
ಫಲಗಳ
ಫಲ-ಗ-ಳನ್ನು
ಫಲ-ಗ-ಳಲ್ಲದೇ
ಫಲಗಳು
ಫಲ-ಗ-ಳೆಲ್ಲ
ಫಲಗಳೇ
ಫಲದ
ಫಲ-ದಾ-ಯ-ಕ-ವಾ-ದು-ದನ್ನು
ಫಲದಿಂದ
ಫಲಪ್ರ-ದ-ವಾ-ಗ-ದಿದ್ದಾಗ
ಫಲಪ್ರ-ದ-ವಾ-ಗುತ್ತದೆ
ಫಲ-ಭ-ರಿ-ತ-ವಾ-ಗು-ವು-ದಲ್ಲ
ಫಲ-ರೂ-ಪ-ವಾಗಿ
ಫಲವನ್ನು
ಫಲವಾಗಿ
ಫಲ-ವಾ-ಗಿಯೇ
ಫಲವಾದ
ಫಲವು
ಫಲವೂ
ಫಲವೇ
ಫಲಿತಾಂಶ
ಫಲಿ-ತಾಂಶ-ವನ್ನೂ
ಫಲಿ-ಸು-ವುದು
ಫಸ್ಟ್ಕ್ಲಾಸ್
ಫಾದರ್
ಫಾಶಿ-ಯಾ-ದರೂ
ಫಿಟ್ಟಿಂಗ್
ಫಿರಂಗಿ
ಫಿಲಿಪ್
ಫಿಲೋ-ಕಾ-ಲಿಯಾ
ಫೀಡ್
ಫೀಲ್ಡ್
ಫುಟ್ಬಾಲ್
ಫುಲ್ಲರ್
ಫೇಲಾ
ಫೇಲಾದ
ಫೇಲಾದಾಗ
ಫೇಲಾದುದು
ಫೋಟೋ
ಫೋಟೋ-ಗ-ಳನ್ನೂ
ಫೋಟೋಗ್ರ-ಫಿಯ
ಫೋಟೋಗ್ರಾ-ಫರ್
ಫೋನನ್ನು
ಫೋನಿನಲ್ಲಿ
ಫೋನು
ಫೋನುಮಾಡಿ
ಫೋನ್
ಫೌಸ್ಟ್
ಫ್ಯಾನು
ಫ್ಯಾನ್
ಫ್ಯಾಷನ್
ಫ್ಯಾಷನ್ನಿ-ನಿಂದ
ಫ್ರಾಂಕ್ಫರ್ಟ್
ಫ್ರಾಂಕ್ಲಿನ್
ಫ್ರಾನ್ಸಿನ
ಫ್ರಾನ್ಸಿನಲ್ಲಿ
ಫ್ರಾನ್ಸಿಸ್
ಫ್ರಾನ್ಸಿಸ್ಕೋ
ಫ್ರಾನ್ಸ್ಈ
ಫ್ರಾಯಿಡ್
ಫ್ರಾಯ್ಡನ
ಫ್ರಾಯ್ಡ್
ಫ್ರಾೖಡ್
ಫ್ರಿಜೋ
ಫ್ರಿಟ್ಸ್
ಫ್ರಿಟ್ಸ್ನ
ಫ್ರೆಂಚ್
ಫ್ರೋಂ
ಫ್ರೋಮ್
ಫ್ಲೆನಾಗನ್
ಫ್ಲೆನಾಗನ್ರ
ಫ್ಲೋರಿಡಾ
ಫ್ಲೋರಿ-ಡಾಕ್ಕಿಂತ
ಫ್ಲೋರಿ-ಡಾ-ದಲ್ಲಿ-ರುವ
ಬಂಗ-ಲೆ-ಗಾಗಿ
ಬಂಗ-ಲೆ-ಯನ್ನು
ಬಂಗ-ಲೆ-ಯನ್ನೂ
ಬಂಗ-ಲೆ-ಯಲ್ಲಿದ್ದ
ಬಂಗ-ಲೆ-ಯೊಂದರ
ಬಂಗ-ಲೆ-ಯೊಂದು
ಬಂಗಾರದ
ಬಂಗಾಲ
ಬಂಗಾ-ಲಿ-ಯಲ್ಲಿ
ಬಂಗಾಳದ
ಬಂಗಾ-ಳ-ದಲ್ಲಿ
ಬಂಜೆಯಾದ
ಬಂಡಿ-ಯನ್ನೆ-ಳೆ-ಯುವ
ಬಂಡುಗಾರ
ಬಂಡೆಯ
ಬಂತಲ್ಲ
ಬಂತು
ಬಂತುಆತ
ಬಂತೆಂದು
ಬಂದ
ಬಂದಂತಾ-ಯಿತು
ಬಂದಂತಿಲ್ಲ-ವೆಂಬುದು
ಬಂದಂತೆ
ಬಂದಂತೆಲ್ಲ
ಬಂದಂಥವು
ಬಂದದ್ದ-ರಿಂದ
ಬಂದದ್ದು
ಬಂದನಂತೆ
ಬಂದಮೇಲೆ
ಬಂದರಷ್ಟೆ
ಬಂದರು
ಬಂದರೂ
ಬಂದರೆ
ಬಂದಲ್ಲಿ
ಬಂದಳು
ಬಂದ-ವ-ನಲ್ಲ
ಬಂದ-ವ-ನಾತ
ಬಂದವನು
ಬಂದವನೇ
ಬಂದ-ವ-ರಂತೆ
ಬಂದ-ವ-ರಿಗೆ
ಬಂದವರು
ಬಂದ-ವ-ರೆಂಬುದು
ಬಂದ-ವ-ರೆಲ್ಲ
ಬಂದವಲ್ಲ
ಬಂದವು
ಬಂದಾಗ
ಬಂದಾ-ಗ-ಲಂತೂ
ಬಂದಾಗಲೂ
ಬಂದಾ-ಗ-ಲೆಲ್ಲ
ಬಂದಿ-ತಲ್ಲವೆ
ಬಂದಿತು
ಬಂದಿತ್ತು
ಬಂದಿತ್ತೆಂದು
ಬಂದಿದೆ
ಬಂದಿದ್ದ
ಬಂದಿದ್ದರು
ಬಂದಿದ್ದರೂ
ಬಂದಿದ್ದಳು
ಬಂದಿದ್ದಾಗ
ಬಂದಿದ್ದಾನೆ
ಬಂದಿದ್ದಾರೆ
ಬಂದಿದ್ದಾ-ಳೆಂದು
ಬಂದಿದ್ದೀಯೆ
ಬಂದಿದ್ದೀರಿ
ಬಂದಿದ್ದೆ
ಬಂದಿದ್ದೇನೆ
ಬಂದಿದ್ದೇವೆ
ಬಂದಿರ
ಬಂದಿ-ರ-ಬ-ಹುದು
ಬಂದಿ-ರ-ಬೇಕು
ಬಂದಿರಲಿ
ಬಂದಿ-ರುತ್ತದೆ
ಬಂದಿ-ರುತ್ತಾರೆ
ಬಂದಿರುವ
ಬಂದಿಲ್ಲ
ಬಂದಿವೆ
ಬಂದೀತು
ಬಂದೀತೆಂಬ
ಬಂದು
ಬಂದುಆಕೆ
ಬಂದುದರ
ಬಂದು-ದ-ರಿಂದ
ಬಂದು-ದಾ-ದರೂ
ಬಂದುದು
ಬಂದುದೇ
ಬಂದು-ಬಿಟ್ಟಿದೆ
ಬಂದು-ಬಿ-ಡುತ್ತದೆ
ಬಂದು-ಬಿ-ಡುತ್ತ-ದೆಂಬ
ಬಂದು-ಬಿ-ಡುತ್ತ-ದೆಯೇ
ಬಂದು-ಬಿ-ಡುತ್ತವೆ
ಬಂದು-ಬಿ-ಡುತ್ತಾನೆ
ಬಂದು-ಬಿ-ಡು-ವು-ದಿಲ್ಲ
ಬಂದೆ
ಬಂದೆನೆಂದೂ
ಬಂದೇ
ಬಂದೊಡನೆ
ಬಂದೊ-ಡ-ನೆಯೇ
ಬಂದೊ-ದ-ಗ-ದಿದ್ದಲ್ಲಿ
ಬಂದೊ-ದ-ಗಿತು
ಬಂದೊ-ದ-ಗಿದ
ಬಂದೊ-ದ-ಗಿ-ದರೂ
ಬಂದೊ-ದ-ಗಿ-ದಾಗ
ಬಂದೊ-ದ-ಗುತ್ತವೆ
ಬಂದೊ-ದ-ಗುವ
ಬಂದೋ-ಪಾಧ್ಯಾ-ಯರು
ಬಂಧನ
ಬಂಧ-ನ-ಇ-ವು-ಗಳ
ಬಂಧನಕ್ಕೂ
ಬಂಧ-ನಕ್ಕೊ-ಳ-ಗಾದ
ಬಂಧ-ನ-ಗ-ಳಿಂದ
ಬಂಧ-ನ-ಗಳೂ
ಬಂಧ-ನ-ದಲ್ಲಿ
ಬಂಧ-ನ-ವನ್ನು
ಬಂಧ-ನ-ವನ್ನೂ
ಬಂಧನವು
ಬಂಧ-ನ-ವೆ-ನಿ-ಸಿ-ದರೂ
ಬಂಧನವೇ
ಬಂಧ-ಮುಕ್ತ-ಭಾ-ವ-ನೆಯ
ಬಂಧಿ-ತ-ನ-ವನು
ಬಂಧಿ-ತ-ನಾ-ಗಿದ್ದಾನೆ
ಬಂಧಿ-ತ-ನಾದ
ಬಂಧಿ-ತ-ವಾ-ಗದ
ಬಂಧಿ-ಯಾ-ದಂತೆ
ಬಂಧಿ-ಸಿ-ಕೊಂಡು
ಬಂಧು
ಬಂಧುಗಳ
ಬಂಧು-ಗ-ಳನ್ನು
ಬಂಧು-ಗ-ಳಿಗೂ
ಬಂಧು-ಗ-ಳಿಗೆ
ಬಂಧುಗಳು
ಬಂಧು-ಗ-ಳೆಲ್ಲ
ಬಂಧುತ್ವದ
ಬಂಧು-ಬ-ಳ-ಗ-ದ-ವ-ರಿಂದ
ಬಂಧು-ಬ-ಳ-ಗ-ದ-ವ-ರೆಲ್ಲರ
ಬಂಧು-ಬಾಂಧ-ವರ
ಬಂಧು-ಬಾಂಧ-ವ-ರಿಂದ
ಬಂಧು-ಬಾಂಧ-ವ-ರಿಗೂ
ಬಂಧು-ಬಾಂಧ-ವ-ರಿ-ರುವ
ಬಂಧು-ಬಾಂಧ-ವರು
ಬಂಧು-ಬಾಂಧ-ವ-ರೊ-ಡನೆ
ಬಂಧುವೊಬ್ಬ
ಬಂಧು-ವೊಬ್ಬರು
ಬಂಪದಾ
ಬಂಪಾದಾಕ್ಕೆ
ಬಕಧ್ಯಾನ
ಬಗೆ
ಬಗೆಗಿನ
ಬಗೆಗೂ
ಬಗೆಗೆ
ಬಗೆ-ಗೆ-ಮುಂದಿ-ರಿ-ಸು-ವು-ದ-ರಿಂದ
ಬಗೆಗೇ
ಬಗೆದರು
ಬಗೆದರೆ
ಬಗೆ-ಬ-ಗೆ-ಯಾಗಿ
ಬಗೆಯ
ಬಗೆಯದು
ಬಗೆಯನ್ನು
ಬಗೆ-ಯಿಂದಲೇ
ಬಗೆ-ಯು-ವ-ವ-ರನ್ನು
ಬಗೆ-ಹ-ರಿ-ಸ-ಬಲ್ಲುದು
ಬಗೆ-ಹ-ರಿ-ಸ-ಲಾ-ರದ್ದೆಂದು
ಬಗೆ-ಹ-ರಿ-ಸುವ
ಬಗ್ಗದ
ಬಗ್ಗಿ-ದ-ವ-ನಿಗೆ
ಬಗ್ಗಿ-ಸ-ಬ-ಹುದು
ಬಗ್ಗಿಸಿ
ಬಗ್ಗಿ-ಸು-ವುದು
ಬಗ್ಗುವಂತೆ
ಬಗ್ಗೆ
ಬಗ್ಗೆಯೂ
ಬಗ್ಲರ್
ಬಚಾ-ವಾ-ಗಲು
ಬಚ್ಚಲಿನ
ಬಚ್ಚಲು
ಬಚ್ಚಿಟ್ಟ-ರಂತೆ
ಬಚ್ಚಿ-ಡ-ಬಾ-ರದು
ಬಚ್ಚಿ-ಡುತ್ತಾನೆ
ಬಜಾರಿಗೆ
ಬಜೆಟ್
ಬಟ್ಟೆ
ಬಟ್ಟೆ-ಗ-ಳನ್ನೇ
ಬಟ್ಟೆ-ಗ-ಳಿಂದ
ಬಟ್ಟೆಗಳು
ಬಟ್ಟೆಗೂ
ಬಟ್ಟೆ-ಬ-ರೆ-ಗ-ಳಾಗಿ
ಬಟ್ಟೆ-ಬ-ರೆ-ಗಳು
ಬಟ್ಟೆಯನ್ನು
ಬಟ್ಟೆಯಿಲ್ಲ
ಬಡ
ಬಡಗಿ
ಬಡ-ಜ-ನರ
ಬಡತನ
ಬಡ-ತ-ನಕ್ಕೂ
ಬಡ-ತ-ನ-ಗ-ಳನ್ನು
ಬಡ-ತ-ನದ
ಬಡ-ತ-ನ-ವನ್ನು
ಬಡ-ತ-ನ-ವಿದ್ದರೂ
ಬಡ-ಪಾ-ಯಿಯ
ಬಡರೈತ
ಬಡವ
ಬಡ-ವ-ನಲ್ಲಿ
ಬಡ-ವ-ನಾ-ಗಲಿ
ಬಡ-ವ-ನಾ-ಗಿದ್ದರೂ
ಬಡವರ
ಬಡ-ವ-ರಲ್ಲಿ
ಬಡ-ವ-ರಿಗೂ
ಬಡ-ವ-ರಿಗೆ
ಬಡವರು
ಬಡಾ-ವ-ಣೆ-ಯಲ್ಲಿ
ಬಡಿತ
ಬಡಿತದ
ಬಡಿ-ತ-ವನ್ನು
ಬಡಿ-ದಟ್ಟುವ
ಬಡಿ-ದಪ್ಪ-ಳಿ-ಸಿ-ದಾಗ
ಬಡಿದಾಗ
ಬಡಿದಾಟ
ಬಡಿದು
ಬಡಿ-ದು-ಕೊಳ್ಳು-ವುದು
ಬಡಿದೂ
ಬಡಿ-ದೆಬ್ಬಿ-ಸ-ಬೇ-ಕಾ-ದರೆ
ಬಡಿ-ದೆಬ್ಬಿ-ಸುವ
ಬಡಿ-ದೆಬ್ಬಿ-ಸು-ವುದು
ಬಡಿ-ದೋ-ಡಿ-ಸುವ
ಬಡಿಯಿತು
ಬಡುಕ
ಬಡ್ಡಿ
ಬಡ್ತಿ-ಗ-ಳಿಂದಲೇ
ಬಡ್ತಿಯಾಗಿ
ಬಣ್ಣ
ಬಣ್ಣಕ್ಕೆ
ಬಣ್ಣ-ಗ-ಳನ್ನು
ಬಣ್ಣ-ಗ-ಳಲ್ಲಿ
ಬಣ್ಣ-ಗ-ಳಿಂದ
ಬಣ್ಣದ
ಬಣ್ಣ-ದೊಂದಿಗೆ
ಬಣ್ಣದ್ದಾಗಿ
ಬಣ್ಣವೂ
ಬತ್ತದ
ಬತ್ತ-ಲಾ-ದರೆ
ಬತ್ತಲು
ಬತ್ತ-ಲೆ-ಯಾಗಿ
ಬತ್ತಿ
ಬತ್ತಿಯನ್ನು
ಬತ್ತಿಸಿತು
ಬತ್ತುತ್ತ
ಬದ-ನೇ-ಕಾ-ಯಿ-ಗಳು
ಬದ-ಲಾ-ಗ-ಬೇ-ಕಿತ್ತು
ಬದ-ಲಾ-ಗ-ಬೇಕು
ಬದ-ಲಾ-ಗ-ಲೇ-ಬೇಕು
ಬದಲಾಗಿ
ಬದ-ಲಾ-ಗುತ್ತದೆ
ಬದ-ಲಾ-ಗುತ್ತವೆ
ಬದ-ಲಾ-ಗು-ವುದು
ಬದ-ಲಾ-ಗು-ವು-ದೆಂಬ
ಬದ-ಲಾ-ದರೂ
ಬದ-ಲಾ-ಯಿತು
ಬದ-ಲಾ-ಯಿ-ಸ-ಬಲ್ಲ
ಬದ-ಲಾ-ಯಿ-ಸ-ಬ-ಹು-ದಾದ
ಬದ-ಲಾ-ಯಿಸಿ
ಬದ-ಲಾ-ಯಿ-ಸಿ-ಕೊಳ್ಳಲು
ಬದ-ಲಾ-ಯಿ-ಸಿತು
ಬದ-ಲಾ-ಯಿ-ಸುತ್ತದೆ
ಬದ-ಲಾ-ಯಿ-ಸುತ್ತಿ-ರುತ್ತವೆ
ಬದ-ಲಾ-ವಣೆ
ಬದ-ಲಾ-ವ-ಣೆ-ಗ-ಳನ್ನುಂಟು
ಬದ-ಲಾ-ವ-ಣೆ-ಗ-ಳನ್ನುಂಟು-ಮಾ-ಡುತ್ತವೆ
ಬದ-ಲಾ-ವ-ಣೆ-ಗಳ
ಬದ-ಲಾ-ವ-ಣೆ-ಗ-ಳನ್ನು
ಬದ-ಲಾ-ವ-ಣೆ-ಗ-ಳನ್ನೂ
ಬದ-ಲಾ-ವ-ಣೆ-ಗ-ಳಿಂದ
ಬದ-ಲಾ-ವ-ಣೆ-ಗಳು
ಬದ-ಲಾ-ವ-ಣೆ-ಗಳೂ
ಬದ-ಲಾ-ವ-ಣೆಗೆ
ಬದ-ಲಾ-ವ-ಣೆ-ಯನ್ನಾ-ಗಲಿ
ಬದ-ಲಾ-ವ-ಣೆ-ಯನ್ನು
ಬದ-ಲಾ-ವ-ಣೆ-ಯಾ-ಗದೇ
ಬದ-ಲಾ-ವ-ಣೆ-ಯಾ-ಗ-ಬ-ಹುದು
ಬದ-ಲಾ-ವ-ಣೆ-ಯಾ-ಗುತ್ತಿ-ರುತ್ತದೆ
ಬದ-ಲಾ-ವ-ಣೆ-ಯಾ-ದರೆ
ಬದ-ಲಾ-ವ-ಣೆ-ಯಿಂದ
ಬದಲಿಗೆ
ಬದ-ಲಿ-ಸದೇ
ಬದ-ಲಿ-ಸ-ಬಲ್ಲ
ಬದ-ಲಿ-ಸ-ಬಲ್ಲವು
ಬದ-ಲಿ-ಸ-ಬಲ್ಲುದು
ಬದ-ಲಿ-ಸ-ಬ-ಹುದು
ಬದ-ಲಿ-ಸ-ಲಾ-ಗದ
ಬದ-ಲಿ-ಸಲು
ಬದ-ಲಿ-ಸಿ-ಕೊಳ್ಳಲು
ಬದ-ಲಿ-ಸಿತು
ಬದ-ಲಿ-ಸಿದ
ಬದ-ಲಿ-ಸಿ-ದರೆ
ಬದ-ಲಿ-ಸಿ-ದಾಗ
ಬದ-ಲಿ-ಸಿದೆ
ಬದ-ಲಿ-ಸಿದ್ದಾರೆ
ಬದ-ಲಿ-ಸೀ-ತಲ್ಲವೆ
ಬದ-ಲಿ-ಸೀತು
ಬದ-ಲಿ-ಸುತ್ತದೆ
ಬದ-ಲಿ-ಸುವ
ಬದ-ಲಿ-ಸು-ವಂಥ
ಬದ-ಲಿ-ಸು-ವುದು
ಬದಲು
ಬದಿಗಿಟ್ಟು
ಬದಿಯಲ್ಲಿ
ಬದಿಯಲ್ಲೇ
ಬದು-ಕ-ಗೊ-ಡುವ
ಬದುಕನ್ನು
ಬದುಕನ್ನೇ
ಬದು-ಕ-ಬ-ಹುದು
ಬದು-ಕ-ಬಾ-ರ-ದೆಂದೇ
ಬದು-ಕ-ಬೇ-ಕಾ-ಗಿತ್ತು
ಬದು-ಕ-ಬೇ-ಕಾ-ದರೆ
ಬದು-ಕ-ಬೇ-ಕು-ನಾ-ಶ-ವಾ-ಗ-ಬಾ-ರದು
ಬದು-ಕ-ಲಾ-ರೆವು
ಬದುಕಲು
ಬದು-ಕಾ-ದರೂ
ಬದುಕಿ
ಬದು-ಕಿ-ಕೊಂಡಿದ್ದಾರೆ
ಬದು-ಕಿ-ಕೊಂಡಿ-ರುತ್ತಾ-ರೆಂಬುದು
ಬದು-ಕಿ-ಕೊಂಡಿ-ರು-ವುದು
ಬದು-ಕಿ-ಕೊಳ್ಳುತ್ತಿದ್ದ
ಬದು-ಕಿ-ಗಾಗಿ
ಬದುಕಿಗೂ
ಬದುಕಿಗೆ
ಬದು-ಕಿ-ಗೇನು
ಬದು-ಕಿ-ಗೊಂದು
ಬದುಕಿದ
ಬದು-ಕಿ-ದಂತಾ-ಗುತ್ತದೆ
ಬದು-ಕಿ-ದ-ರಾ-ಯಿತು
ಬದು-ಕಿ-ದಳು
ಬದು-ಕಿ-ದ-ವ-ನಾ-ಗಿದ್ದ
ಬದು-ಕಿ-ದ-ವರು
ಬದುಕಿದೆ
ಬದುಕಿದ್ದ
ಬದು-ಕಿದ್ದರೂ
ಬದು-ಕಿದ್ದಳು
ಬದು-ಕಿದ್ದಾ-ದರೆ
ಬದುಕಿನ
ಬದು-ಕಿ-ನಲ್ಲಿ
ಬದು-ಕಿ-ನಲ್ಲೂ
ಬದು-ಕಿ-ನಲ್ಲೇ
ಬದು-ಕಿ-ನಿಂದ
ಬದು-ಕಿ-ನುದ್ದಕ್ಕೂ
ಬದು-ಕಿ-ನೊ-ಳ-ಗುಟ್ಟ
ಬದು-ಕಿ-ರ-ಬೇ-ಕೆಂದು
ಬದು-ಕಿ-ರ-ಲಾರ
ಬದು-ಕಿ-ರುವ
ಬದು-ಕಿ-ರು-ವ-ವ-ನಿಗೂ
ಬದು-ಕಿ-ಸಿ-ದಿರಿ
ಬದು-ಕಿ-ಸಿದೆ
ಬದುಕು
ಬದು-ಕುಗ್ರಂಥ-ಕರ್ತರ
ಬದು-ಕುತ್ತಾನೆ
ಬದುಕುವ
ಬದು-ಕು-ವಂತೆ
ಬದುಕೇ
ಬದ್ಧ-ಕಂಕ-ಣ-ವಾ-ದಂತಿದೆ
ಬದ್ಧ-ತೆ-ಯನ್ನು
ಬದ್ಧನಾದ
ಬದ್ಧನೂ
ಬದ್ಧನೆಂದೂ
ಬದ್ಧರಾಗಿ
ಬದ್ಧವಾಗಿ
ಬದ್ಧ-ವಾ-ಗಿ-ರು-ವು-ದೆಂಬುದೇ
ಬದ್ಧ-ವಾ-ಗಿ-ಸಿ-ಕೊಂಡು
ಬನಾರಸ್
ಬನಾ-ರಸ್ನಲ್ಲಿ
ಬನಾ-ರಸ್ನಲ್ಲಿದ್ದಾರೆ
ಬನಿಯಾ
ಬಯ
ಬಯಕೆ
ಬಯ-ಕೆ-ಇವು
ಬಯ-ಕೆ-ಗಳ
ಬಯ-ಕೆ-ಗ-ಳನ್ನು
ಬಯ-ಕೆ-ಗ-ಳಿಲ್ಲ
ಬಯಕೆಯ
ಬಯ-ಲಾ-ಗಿ-ದೆ-ಯಲ್ಲವೆ
ಬಯ-ಲಿ-ಗೆ-ಳೆ-ದಿದ್ದೆ
ಬಯ-ಲಿ-ಗೆ-ಳೆ-ಯಲು
ಬಯ-ಲಿ-ನಲ್ಲಿ
ಬಯಲು
ಬಯ-ಲೊ-ಳಗೆ
ಬಯ-ಲೋ-ಜಿ-ಕಲ್
ಬಯಸದೆ
ಬಯಸದೇ
ಬಯ-ಸಿ-ದವ
ಬಯ-ಸಿದ್ದನ್ನು
ಬಯಸಿದ್ದು
ಬಯ-ಸಿ-ರುವ
ಬಯ-ಸುತ್ತಾನೋ
ಬಯ-ಸುತ್ತೇನೆ
ಬಯಸುವ
ಬಯ-ಸು-ವ-ವರು
ಬಯ-ಸು-ವುದು
ಬಯ್ದು
ಬಯ್ಯುವುದೇ
ಬರ
ಬರ-ಕೂ-ಡ-ದೆಂದು
ಬರಗಾಲ
ಬರಡೂ
ಬರತಕ್ಕ
ಬರ-ತಕ್ಕ-ವಲ್ಲ
ಬರ-ತೊ-ಡ-ಗಿತು
ಬರದ
ಬರದಂತೆ
ಬರ-ಬ-ರುತ್ತ
ಬರ-ಬ-ಹು-ದಾ-ಗಿದೆ
ಬರ-ಬ-ಹು-ದಾದ
ಬರ-ಬ-ಹುದು
ಬರ-ಬ-ಹು-ದು-ಇದು
ಬರ-ಬ-ಹುದೆ
ಬರ-ಬ-ಹು-ದೆಂದು
ಬರ-ಬ-ಹು-ದೆಂಬು-ದನ್ನು
ಬರ-ಬಾ-ರದು
ಬರ-ಬೇ-ಕಾಗಿ
ಬರ-ಬೇ-ಕಾ-ಗುತ್ತದೆ
ಬರ-ಬೇ-ಕಾ-ದರೆ
ಬರ-ಬೇ-ಕಿತ್ತೆಂದು
ಬರಬೇಕು
ಬರಮಾಡಿ
ಬರ-ಮಾ-ಡಿ-ಕೊಂಡರು
ಬರ-ಮಾ-ಡಿ-ಕೊಳ್ಳುತ್ತಿದ್ದವು
ಬರ-ಮಾ-ಡಿ-ಕೊಳ್ಳು-ವ-ನೆಂಬು-ದನ್ನು
ಬರ-ಲಾ-ಗುತ್ತಿಲ್ಲ
ಬರ-ಲಾ-ಗುತ್ತಿಲ್ಲ-ವೆಂದು
ಬರ-ಲಾ-ಗು-ವು-ದಿಲ್ಲ
ಬರ-ಲಾ-ರ-ದೆಂಬು-ದನ್ನು
ಬರಲಾರೆ
ಬರಲಿ
ಬರ-ಲಿ-ರುವ
ಬರಲಿಲ್ಲ
ಬರ-ಲಿಲ್ಲ-ವಲ್ಲ
ಬರ-ಲಿಲ್ಲ-ವೆಂದು
ಬರ-ಲಿಲ್ಲವೇ
ಬರಲು
ಬರಲೂ
ಬರ-ವ-ಣಿ-ಗೆಯ
ಬರವನ್ನು
ಬರ-ವನ್ನೆ-ದು-ರಿ-ಸಲು
ಬರಹ
ಬರಹಕ್ಕೆ
ಬರ-ಹ-ಗ-ಳನ್ನು
ಬರ-ಹ-ಗ-ಳಲ್ಲಿ
ಬರ-ಹ-ಗ-ಳಿಂದ
ಬರ-ಹ-ಗಳು
ಬರ-ಹ-ಗಾರ
ಬರ-ಹ-ಗಾ-ರನ
ಬರ-ಹ-ಗಾ-ರನು
ಬರ-ಹ-ಗಾ-ರ-ರಾದ
ಬರ-ಹ-ಗಾ-ರರು
ಬರ-ಹ-ಗಾ-ರರೂ
ಬರಹದ
ಬರ-ಹ-ದಲ್ಲಿ
ಬರ-ಹ-ಬಲ್ಲ
ಬರ-ಹ-ವನ್ನು
ಬರಹವೂ
ಬರಿದಾ
ಬರಿ-ದಾ-ಗದ
ಬರಿ-ದಾ-ದದ್ದಿ-ದೆಯೆ
ಬರಿದೆ
ಬರಿ-ನೆ-ಲ-ದಲ್ಲಿ
ಬರಿಯ
ಬರಿಸಿತ್ತು
ಬರಿಸುವ
ಬರೀ
ಬರು
ಬರುತ್ತ
ಬರುತ್ತ-ದಷ್ಟೆ
ಬರುತ್ತದೆ
ಬರುತ್ತ-ದೆಂದು
ಬರುತ್ತದೋ
ಬರುತ್ತ-ಲಿದೆ
ಬರುತ್ತ-ಲಿದ್ದರೆ
ಬರುತ್ತಲೇ
ಬರುತ್ತವೆ
ಬರುತ್ತ-ವೆಂದು
ಬರುತ್ತಾನೆ
ಬರುತ್ತಾ-ನೆಂಬುದು
ಬರುತ್ತಾ-ರಷ್ಟೆ
ಬರುತ್ತಾರೆ
ಬರುತ್ತಿ
ಬರುತ್ತಿತ್ತು
ಬರುತ್ತಿದೆ
ಬರುತ್ತಿ-ದೆ-ಯೆಂದು
ಬರುತ್ತಿದ್ದ
ಬರುತ್ತಿದ್ದಂತೆಯೇ
ಬರುತ್ತಿದ್ದರು
ಬರುತ್ತಿದ್ದ-ರುಈ
ಬರುತ್ತಿದ್ದರೆ
ಬರುತ್ತಿದ್ದವು
ಬರುತ್ತಿದ್ದಾಗ
ಬರುತ್ತಿದ್ದಾನೆ
ಬರುತ್ತಿ-ರ-ಲಿಲ್ಲ
ಬರುತ್ತಿ-ರುವ
ಬರುತ್ತಿ-ರು-ವಷ್ಟ-ರಲ್ಲೇ
ಬರುತ್ತಿ-ರು-ವುದು
ಬರುತ್ತಿವೆ
ಬರುತ್ತೇನೆ
ಬರುತ್ತೇ-ನೆಂದು
ಬರುವ
ಬರು-ವಂತಾ-ಗಲು
ಬರು-ವಂತಾ-ಗಲೂ
ಬರು-ವಂತಾ-ಯಿತು
ಬರು-ವಂತಿಲ್ಲ
ಬರು-ವಂತಿಲ್ಲ-ವೆನ್ನುತ್ತಾನೆ
ಬರುವಂತೆ
ಬರುವಂಥ
ಬರು-ವಂಥ-ವು-ಗಳೇ
ಬರು-ವ-ವ-ರೆಗೆ
ಬರು-ವಷ್ಟಕ್ಕೆ
ಬರುವಷ್ಟು
ಬರುವಾಗ
ಬರುವು
ಬರು-ವು-ದನ್ನು
ಬರು-ವು-ದಿಲ್ಲ
ಬರು-ವು-ದಿಲ್ಲ-ವೆಂದಲ್ಲಿ
ಬರು-ವು-ದಿಲ್ಲ-ವೆಂದು
ಬರು-ವು-ದಿಲ್ಲವೇ
ಬರುವುದು
ಬರು-ವು-ದುಂಟು
ಬರು-ವು-ದು-ಆಗ
ಬರು-ವು-ದೆಲ್ಲಿಂದ
ಬರುವುದೇ
ಬರುವುವು
ಬರೆ
ಬರೆದ
ಬರೆದದ್ದು
ಬರೆದರು
ಬರೆದರೆ
ಬರೆ-ದ-ವರು
ಬರೆ-ದಿಟ್ಟರು
ಬರೆ-ದಿ-ಡುತ್ತಾನೆ
ಬರೆದಿದ್ದ
ಬರೆ-ದಿದ್ದರು
ಬರೆ-ದಿದ್ದಾನೆ
ಬರೆ-ದಿದ್ದಾರೆ
ಬರೆ-ದಿ-ರಿ-ಸಿ-ಕೊಂಡ
ಬರೆ-ದಿ-ರಿ-ಸಿ-ಕೊಂಡದ್ದು
ಬರೆ-ದಿ-ರಿ-ಸಿ-ಕೊಂಡಿದ್ದ
ಬರೆದು
ಬರೆ-ದು-ಕೊ-ಡುತ್ತಿದ್ದ
ಬರೆ-ದು-ದಲ್ಲ
ಬರೆ-ಯ-ಲಾ-ಯಿತು
ಬರೆಯದೇ
ಬರೆ-ಯ-ಲಾ-ಗು-ವುದು
ಬರೆಯಲು
ಬರೆಯಿರಿ
ಬರೆ-ಯುತ್ತಾ-ರೆಂದು
ಬರೆ-ಯುತ್ತಿ-ರ-ಲಿಲ್ಲ
ಬರೆ-ಯು-ವಾಗ
ಬರೆಸಿದ
ಬರೇ
ಬರ್ಟ್ರಾಂಡ್
ಬರ್ಡ್
ಬರ್ನಾಡ್
ಬರ್ನಾರ್ಡ್ಷಾ
ಬರ್ನೆಟ್
ಬರ್ಬರ
ಬಲ
ಬಲ-ಸಂವರ್ಧ-ನೆ-ಯಾ-ಗ-ಬೇ-ಕೆನ್ನು-ವ-ವನು
ಬಲ-ಇ-ವು-ಗ-ಳನ್ನು
ಬಲ-ಇ-ವು-ಗ-ಳಿಂದ
ಬಲ-ಕಿ-ಸೆ-ಯಿಂದ
ಬಲಗಳ
ಬಲ-ಗೈ-ಯಾಗಿ
ಬಲ-ಗೊ-ಳಿ-ಸಲು
ಬಲ-ಗೊ-ಳಿ-ಸುವ
ಬಲದ
ಬಲದಿಂದ
ಬಲ-ದಿಂದಲೇ
ಬಲ-ದಿಂದಲ್ಲ
ಬಲ-ದೆ-ದು-ರಲ್ಲಿ
ಬಲ-ಪ-ಡಿ-ಸಿ-ಕೊಳ್ಳುತ್ತ
ಬಲ-ಪ-ಡಿ-ಸುತ್ತ
ಬಲ-ಪಾರ್ಶ್ವ-ದಲ್ಲಿ
ಬಲ-ಯು-ತ-ರಾಗಿ
ಬಲ-ಯು-ತವೂ
ಬಲರಾಮ
ಬಲ-ರಾ-ಮನ
ಬಲವಂತ
ಬಲ-ವಂತ-ವಾಗಿ
ಬಲ-ವತ್ತ-ರ-ವಾಗಿ
ಬಲ-ವತ್ತ-ರ-ವಾ-ಗು-ವುದು
ಬಲವನ್ನು
ಬಲ-ವನ್ನು-ಪ-ಯೋ-ಗಿಸಿ
ಬಲವನ್ನೂ
ಬಲವಾಗಿ
ಬಲ-ವಾ-ಗಿ-ರು-ವಾಗ
ಬಲ-ವಾ-ಗುತ್ತದೆ
ಬಲವಾದ
ಬಲ-ವಾ-ದಂತೆ
ಬಲ-ವಿಲ್ಲ-ದಿದ್ದರೂ
ಬಲವುಳ್ಳ
ಬಲ-ಶಾ-ಲಿ-ಗ-ಳಾ-ಗು-ವಿರಿ
ಬಲ-ಶಾ-ಲಿ-ಗ-ಳಿ-ರಲಿ
ಬಲ-ಶಾ-ಲಿ-ಗಳೂ
ಬಲ-ಶಾ-ಲಿ-ಗ-ಳೆಂದು
ಬಲ-ಸಂವರ್ಧನೆ
ಬಲ-ಸಂವರ್ಧ-ನೆಗೂ
ಬಲ-ಸೋ-ರಿನ
ಬಲಸೋರ್
ಬಲ-ಹೀ-ನ-ರಿಗೆ
ಬಲಾಢ್ಯ-ರಾಗಿ
ಬಲಾಢ್ಯ-ರಿ-ಗಿಂತ
ಬಲಾತ್ಕಾರ
ಬಲಾತ್ಕಾ-ರದ
ಬಲಾತ್ಕಾ-ರ-ದಿಂದ
ಬಲಿ
ಬಲಿ-ಕೊ-ಡುತ್ತಾರೆ
ಬಲಿ-ಕೊ-ಡು-ವು-ದಿಲ್ಲ
ಬಲಿಟಕ್
ಬಲಿತ
ಬಲಿತು
ಬಲಿ-ತು-ಬಿ-ಡುತ್ತವೆ
ಬಲಿದಾನ
ಬಲಿ-ಪ-ಶು-ವಾಗಿ
ಬಲಿಬಿದ್ದ
ಬಲಿ-ಬಿದ್ದರೆ
ಬಲಿ-ಬಿದ್ದ-ವರ
ಬಲಿಬಿದ್ದು
ಬಲಿ-ಬೀ-ಳು-ವಂತೆ
ಬಲಿ-ಯಲ್ಲವೆ
ಬಲಿಯಾಗಿ
ಬಲಿ-ಯಾ-ಗಿದ್ದಾನೆ
ಬಲಿ-ಯಾ-ಗು-ವುವೆಂ
ಬಲಿಯಾದ
ಬಲಿ-ಯಾ-ದ-ವನು
ಬಲಿ-ಯಾ-ದ-ವರು
ಬಲಿಯಿತು
ಬಲಿ-ಯು-ವು-ದಕ್ಕೆ
ಬಲಿಷ್ಠ
ಬಲಿಷ್ಠ-ನಿಗೂ
ಬಲಿಷ್ಠ-ನಿಗೆ
ಬಲಿಷ್ಠ-ರನ್ನಲ್ಲ
ಬಲಿಷ್ಠ-ರನ್ನಾಗಿ
ಬಲಿಷ್ಠ-ರಾಗಿ
ಬಲಿಷ್ಠ-ವಾಗಿ
ಬಲಿಷ್ಠ-ವಾ-ಗಿದೆ
ಬಲು
ಬಲೆಗೆ
ಬಲೆಯ
ಬಲೆಯಲ್ಲಿ
ಬಲ್ಬಿನ
ಬಲ್ಬು
ಬಲ್ಬೊಂದು
ಬಲ್ಲ
ಬಲ್ಲದು
ಬಲ್ಲನೇ
ಬಲ್ಲರು
ಬಲ್ಲ-ವ-ನಾ-ಗಿದ್ದ-ನೆಂದು
ಬಲ್ಲ-ವ-ನಾದ್ದ-ರಿಂದ
ಬಲ್ಲವನು
ಬಲ್ಲ-ವ-ರಲ್ಲ
ಬಲ್ಲವರು
ಬಲ್ಲವರೇ
ಬಲ್ಲಿರಾ
ಬಲ್ಲಿರಿ
ಬಲ್ಲೆವು
ಬಲ್ಲೆವೆ
ಬಲ್ಲೆವೇನು
ಬಳಕೆ
ಬಳಕೆಯ
ಬಳ-ಕೆ-ಯಲ್ಲಿ-ರುವ
ಬಳ-ಲ-ಬೇ-ಕಾ-ಗಿತ್ತು
ಬಳ-ಲಾ-ಡು-ವುದು
ಬಳಲಿ
ಬಳ-ಲಿ-ದ-ವ-ರಿಗೆ
ಬಳಲಿಕೆ
ಬಳ-ಲಿ-ಕೆ-ಯಾ-ಗುತ್ತದೆ
ಬಳಲಿಸಿ
ಬಳ-ಲುತ್ತಿದ್ದಾನೆ
ಬಳ-ಲುತ್ತಿದ್ದಾರೆ
ಬಳ-ಲುತ್ತಿ-ರ-ಬ-ಹು-ದೆಂದು
ಬಳ-ಲುತ್ತಿ-ರುವ
ಬಳಲುವ
ಬಳ-ಲು-ವಂತಾ-ಗಿದೆ
ಬಳ-ಸ-ಲಾ-ಗುವ
ಬಳಸಿ
ಬಳ-ಸಿ-ಕೊಂಡಿದ್ದೇನೆ
ಬಳ-ಸಿ-ದರು
ಬಳ-ಸಿ-ದರೂ
ಬಳ-ಸಿ-ನೋ-ಡಿದೆ
ಬಳ-ಸುತ್ತಾ-ರೆಂಬು-ದನ್ನು
ಬಳಸುವ
ಬಳ-ಸು-ವನೆ
ಬಳಿ
ಬಳಿಕ
ಬಳಿ-ಕ-ವಾ-ಗಲಿ
ಬಳಿಕವೆ
ಬಳಿಕವೋ
ಬಳಿಗೆ
ಬಳಿ-ಯುತ್ತಾರೆ
ಬಳಿಯೇ
ಬಳುವಳಿ
ಬಳು-ವ-ಳಿ-ಯಾಗಿ
ಬಳ್ಳಿ-ಬಳ್ಳಿ-ಯಾಗಿ
ಬವಣೆ
ಬವಣೆಗೆ
ಬವ-ಣೆ-ಯಿಂದ
ಬಸ-ವಣ್ಣ-ನ-ವರ
ಬಸ-ವಣ್ಣ-ನ-ವರು
ಬಸಿಯಲು
ಬಸಿ-ರಿ-ನಲಿ
ಬಸುವಿನ
ಬಸ್
ಬಸ್ಸನ್ನೇ-ರಿದ
ಬಸ್ಸಿಗಾಗಿ
ಬಸ್ಸಿನ
ಬಸ್ಸಿನಲ್ಲಿ
ಬಸ್ಸಿನಲ್ಲೇ
ಬಸ್ಸು
ಬಸ್ಸೇ
ಬಸ್ಸೊಂದ-ರಲ್ಲಿ
ಬಸ್ಸೊಂದು
ಬಹಳ
ಬಹ-ಳ-ಕಾ-ಲ-ದಿಂದ
ಬಹ-ಳ-ವಾಗಿ
ಬಹಳಷ್ಟು
ಬಹಿರ್ವ್ಯಾ-ಪಾ-ರ-ಗ-ಳನ್ನು
ಬಹಿಷ್ಕ-ರಿ-ಸ-ಬೇ-ಕೆಂದಿದ್ದರು
ಬಹಿಷ್ಕ-ರಿ-ಸುತ್ತವೆ
ಬಹು
ಬಹು-ಮಟ್ಟಿಗೆ
ಬಹುಕಾಲ
ಬಹು-ಕಾ-ಲದ
ಬಹು-ಕಾ-ಲ-ದಿಂದ
ಬಹುಜನ
ಬಹು-ಜ-ನರ
ಬಹು-ಜ-ನರು
ಬಹು-ಜ-ನ-ಸು-ಖಕ್ಕೆ
ಬಹು-ಜ-ನ-ಹಿತ
ಬಹು-ತ-ಪ-ದಿಂದ
ಬಹುದಾದ
ಬಹುದಿನ
ಬಹುದು
ಬಹುದೂರ
ಬಹುದೆಂದು
ಬಹು-ದೆಂಬು-ದನ್ನು
ಬಹುದೊಡ್ಡ
ಬಹುದೋ
ಬಹುಪಾಲು
ಬಹುಬೇಗ
ಬಹು-ಬೇ-ಗನೆ
ಬಹು-ಬೇ-ಗನೇ
ಬಹುಭಾಗ
ಬಹು-ಭಾ-ಷಾಜ್ಞಾನ
ಬಹು-ಭಾ-ಷಾ-ತಜ್ಞ-ರಷ್ಟೇ
ಬಹು-ಭಾ-ಷೆ-ಗ-ಳನ್ನು
ಬಹುಮಂದಿ
ಬಹು-ಮಂದಿಯ
ಬಹು-ಮಟ್ಟಿಗೆ
ಬಹುಮಾನ
ಬಹು-ಮಾ-ನದ
ಬಹು-ಮಾ-ನ-ವನ್ನು
ಬಹು-ಮಾ-ನ-ವಾಗಿ
ಬಹು-ಮಾ-ನವೂ
ಬಹು-ಮು-ಖ-ಗಳು
ಬಹುರೂಪಿ
ಬಹುವಾಗಿ
ಬಹು-ವಿಘ್ನ-ಗಳು
ಬಹು-ವಿ-ಧ-ಯೋ-ಜ-ನೆ-ಗ-ಳನ್ನು
ಬಹುಶಃ
ಬಹುಶ್ರು-ತ-ರಾ-ಗಲಿ
ಬಹು-ಸಂಖ್ಯೆಯ
ಬಹು-ಸಂಖ್ಯೆ-ಯಲ್ಲಿದ್ದಾರೆ
ಬಾ
ಬಾಂಧವರು
ಬಾಂಧ-ವ-ರೊ-ಳ-ಗಿನ
ಬಾಂಧ-ವ-ಸ-ಮು-ದಾ-ಯ-ಗಳ
ಬಾಂಧವ್ಯ
ಬಾಂಧವ್ಯ-ಗಳ
ಬಾಂಧವ್ಯದ
ಬಾಂಬಿ-ನಿಂದಾ-ಗಲೀ
ಬಾಂಬಿಗಿಂತ
ಬಾಂಬುಗಳು
ಬಾಂಬ್
ಬಾಕತನ
ಬಾಗಿ
ಬಾಗಿಲನ್ನು
ಬಾಗಿಲಿನ
ಬಾಗಿ-ಲಿ-ನಲ್ಲಿ
ಬಾಗಿಲು
ಬಾಗಿ-ಲು-ಗ-ಳನ್ನು
ಬಾಗಿ-ಲು-ಗ-ಳಿಗೆ
ಬಾಗಿ-ಲು-ಗಳು
ಬಾಗಿ-ಲು-ಗ-ಳೆಲ್ಲ
ಬಾಗು-ವ-ವ-ರಲ್ಲ
ಬಾಗ್
ಬಾಗ್ಬ-ಜಾ-ರಿನ
ಬಾಗ್ಬ-ಜಾ-ರಿ-ನಲ್ಲಿ
ಬಾಡಿ
ಬಾಡುತ್ತಿ-ರುತ್ತಾನೆ
ಬಾಣಗಳು
ಬಾಧಕ
ಬಾಧ-ಕ-ಗ-ಳಾದ
ಬಾಧ-ಕ-ವಾ-ಗುತ್ತವೆ
ಬಾಧ-ಕ-ವಿಲ್ಲ
ಬಾಧಕವೂ
ಬಾಧಿ-ಸ-ದಿದ್ದರೂ
ಬಾಧಿ-ಸುತ್ತದೆ
ಬಾಧಿ-ಸುತ್ತಿತ್ತು
ಬಾಧಿ-ಸು-ವು-ದಿಲ್ಲ
ಬಾಧೆ
ಬಾಧೆಯಿಂದ
ಬಾಧ್ಯ-ತೆ-ಗಳ
ಬಾಧ್ಯ-ತೆ-ಗ-ಳನ್ನು
ಬಾನು-ಬು-ವಿ-ಗಳ
ಬಾಬಾರು
ಬಾಬು
ಬಾಯ-ಬಿ-ಡು-ವಂಥ
ಬಾಯಲ್ಲಿ
ಬಾಯಾರಿ
ಬಾಯಾರಿಕೆ
ಬಾಯಿ
ಬಾಯಿಂದ
ಬಾಯಿಂದಲೇ
ಬಾಯಿಪಾಠ
ಬಾಯಿಬಿಟ್ಟು
ಬಾಯಿ-ಮಾ-ತಿನ
ಬಾಯಿಯನ್ನು
ಬಾಯಿಯಲ್ಲಿ
ಬಾಯಿಯಿಂದ
ಬಾಯ್ಬಿಟ್ಟಿದೆ
ಬಾಯ್ಬಿಟ್ಟು
ಬಾಯ್ಬಿಡದ
ಬಾರದ
ಬಾರ-ದಂತಾ-ಗ-ಬೇಕಾ
ಬಾರದಂತೆ
ಬಾರದವ
ಬಾರ-ದ-ವನು
ಬಾರ-ದ-ವರು
ಬಾರ-ದ-ವ-ರೆಂದು
ಬಾರ-ದಿ-ರ-ಬ-ಹುದು
ಬಾರದು
ಬಾರದೇ
ಬಾರಮ್ಮಾ
ಬಾರಿ
ಬಾರಿಗೆ
ಬಾರಿಯೂ
ಬಾರಿಯೇ
ಬಾರಿ-ಸ-ತೊ-ಡ-ಗಿ-ತೆಂದರೆ
ಬಾರಿ-ಸಿ-ದರೂ
ಬಾರಿ-ಸಿ-ದರೆ
ಬಾರಿ-ಸಿದ್ದಾರೆ
ಬಾರಿಸಿಯೇ
ಬಾರಿ-ಸು-ವ-ವರು
ಬಾರಿ-ಸು-ವಾಗ
ಬಾರಿ-ಸು-ವು-ದ-ರಲ್ಲಿ
ಬಾರಿ-ಸು-ವು-ದಾ-ಗಲಿ
ಬಾಲ
ಬಾಲಕ
ಬಾಲ-ಕ-ನೊಬ್ಬ
ಬಾಲಕನ
ಬಾಲ-ಕ-ನಂತೆ
ಬಾಲ-ಕ-ನನ್ನು
ಬಾಲ-ಕ-ನನ್ನೇ
ಬಾಲ-ಕ-ನಾ-ಗಿದ್ದ
ಬಾಲ-ಕ-ನಾ-ಗಿದ್ದಾಗ
ಬಾಲ-ಕ-ನಿಗೆ
ಬಾಲಕನು
ಬಾಲಕನೂ
ಬಾಲ-ಕ-ನೊಬ್ಬ
ಬಾಲ-ಕ-ನೊಬ್ಬನ
ಬಾಲ-ಕ-ನೊಬ್ಬ-ನನ್ನು
ಬಾಲ-ಕ-ನೊಬ್ಬ-ನಿಗೆ
ಬಾಲಕರು
ಬಾಲ-ಕಾಶ್ರಮ
ಬಾಲಗ್ರಹ
ಬಾಲ-ಶಿಕ್ಷ-ಣ-ತಜ್ಞ
ಬಾಲ-ಸ-ಹಜ
ಬಾಲಿಶ
ಬಾಲೇಶ್ವರ
ಬಾಲೇಶ್ವ-ರ-ನನ್ನು
ಬಾಲೇಶ್ವ-ರ-ನಿಗೇ
ಬಾಲ್
ಬಾಲ್ಟಿಮೋರ್
ಬಾಲ್ಯ
ಬಾಲ್ಯ-ದಿಂದಲೇ
ಬಾಲ್ಯ-ಕಾ-ಲ-ದಲ್ಲೇ
ಬಾಲ್ಯ-ಗ-ಳಲ್ಲಿ
ಬಾಲ್ಯದ
ಬಾಲ್ಯದಲ್ಲಿ
ಬಾಲ್ಯ-ದಲ್ಲಿಯೇ
ಬಾಲ್ಯದಲ್ಲೇ
ಬಾಲ್ಯದಿಂದ
ಬಾಲ್ಯ-ದಿಂದಲೂ
ಬಾಲ್ಯ-ದಿಂದಲೇ
ಬಾಲ್ಯ-ದಿಂದ-ಲೇ-ತಂದೆ
ಬಾಲ್ಯಾರಭ್ಯ
ಬಾಳ
ಬಾಳನ್ನು
ಬಾಳ-ಬುತ್ತಿಯ
ಬಾಳಬೇಕು
ಬಾಳಲು
ಬಾಳ-ಸಂಗಾ-ತಿ-ಯೊಂದಿಗೆ
ಬಾಳಿ
ಬಾಳಿ-ಕೊಂಡಿ-ರುವ
ಬಾಳಿಕೊಂಡು
ಬಾಳಿಗೂ
ಬಾಳಿಗೆ
ಬಾಳಿಗೊಂದು
ಬಾಳಿದ
ಬಾಳಿ-ದಂಥ-ವರ
ಬಾಳಿದರೆ
ಬಾಳಿದ್ದಾರೆ
ಬಾಳಿದ್ದು
ಬಾಳಿನ
ಬಾಳಿನಲ್ಲಿ
ಬಾಳಿ-ಬ-ದು-ಕು-ವಂತೆ
ಬಾಳಿಯಾನು
ಬಾಳೀತು
ಬಾಳು
ಬಾಳುತ್ತಿದ್ದು-ದನ್ನು
ಬಾಳುತ್ತಿದ್ದೆ
ಬಾಳುವ
ಬಾಳು-ವಂತಾ-ಗಲಿ
ಬಾಳು-ವ-ನೆಂದು
ಬಾಳುವುದು
ಬಾಳು-ವೆ-ಯನ್ನ-ನು-ಭ-ವಿ-ಸುತ್ತಿ-ರುವ
ಬಾಳ್ವೆಗೆ
ಬಾಳ್ವೆಯ
ಬಾಳ್ವೆಯಲ್ಲಿ
ಬಾವಿ
ಬಾವಿ-ಕಪ್ಪೆ-ತನ
ಬಾವಿ-ಗ-ಳನ್ನು
ಬಾವಿಗೆ
ಬಾವಿಯ
ಬಾವಿಯಲ್ಲಿ
ಬಾವಿಯಿಂದ
ಬಾಶಮ್
ಬಾಸ್ಕೆಟ್
ಬಾಸ್ಟನ್
ಬಾಸ್ಟನ್ನಿನ
ಬಾಸ್ವೆಲ್
ಬಾಸ್ವೆಲ್ನಿಗೆ
ಬಾಹ್ಯ
ಬಾಹ್ಯ-ಜ-ಗತ್ತನ್ನು
ಬಾಹ್ಯ-ರೂ-ಪ-ದಿಂದ
ಬಾಹ್ಯ-ವಸ್ತು-ಗಳ
ಬಾಹ್ಯ-ವಸ್ತು-ಗಳು
ಬಾಹ್ಯಶೌಚ
ಬಾಹ್ಯಾ-ಚ-ರ-ಣೆ-ಗ-ಳಿಂದಲೇ
ಬಿ
ಬಿಂದು
ಬಿಂದು-ಗ-ಳನ್ನು
ಬಿಂದು-ವಿ-ನಲ್ಲಿ
ಬಿಂದು-ವಿ-ನಲ್ಲಿ-ರುವ
ಬಿಂದು-ವಿ-ನಷ್ಟು
ಬಿಂಬ-ಗ-ಳನ್ನು
ಬಿಂಬವನ್ನು
ಬಿಂಬಿ-ಸುತ್ತಾರೆ
ಬಿಂಬಿ-ಸು-ವು-ದುಂಟು
ಬಿಕನಾಸಿ
ಬಿಕ್ಕಟ್ಟಿನ
ಬಿಕ್ಕಿಬಿಕ್ಕಿ
ಬಿಗಿ
ಬಿಗಿ-ಗೊ-ಳಿ-ಸಲು
ಬಿಗಿ-ಗೊ-ಳಿಸಿ
ಬಿಗಿದು
ಬಿಗಿ-ಯಾ-ಗಿಯೇ
ಬಿಗಿ-ಯುತ್ತಾರೆ
ಬಿಗಿ-ಹಿ-ಡಿ-ತ-ಗ-ಳಿಂದ
ಬಿಗಿ-ಹಿ-ಡಿ-ತ-ದಿಂದ
ಬಿಗುವನ್ನು
ಬಿಗು-ವಾ-ಗಿಯೇ
ಬಿಚ್ಚಿ
ಬಿಚ್ಚಿದ
ಬಿಚ್ಚಿದಾಗ
ಬಿಜ್ಜೆಯ
ಬಿಟ್ಟ
ಬಿಟ್ಟದ್ದು
ಬಿಟ್ಟರೂ
ಬಿಟ್ಟರೆ
ಬಿಟ್ಟ-ರೆ-ಗ-ಳಿಗೆ
ಬಿಟ್ಟಳು
ಬಿಟ್ಟ-ವ-ರುಂಟು
ಬಿಟ್ಟ-ವ-ಳಲ್ಲ
ಬಿಟ್ಟಾಗ
ಬಿಟ್ಟಾರೆಯೇ
ಬಿಟ್ಟಿ
ಬಿಟ್ಟಿತು
ಬಿಟ್ಟಿದೆ
ಬಿಟ್ಟಿದೆಯೊ
ಬಿಟ್ಟಿದ್ದರೂ
ಬಿಟ್ಟಿದ್ದಾನೆ
ಬಿಟ್ಟಿದ್ದಾರೆ
ಬಿಟ್ಟಿರಲು
ಬಿಟ್ಟಿ-ರು-ವು-ದಿಲ್ಲ
ಬಿಟ್ಟಿಲ್ಲ
ಬಿಟ್ಟು
ಬಿಟ್ಟು-ಕೊ-ಡ-ಲಿಲ್ಲ
ಬಿಟ್ಟುಕೊಡು
ಬಿಟ್ಟು-ಕೊ-ಡುತ್ತೀರಿ
ಬಿಟ್ಟುಬಂದೆ
ಬಿಟ್ಟು-ಬಿಟ್ಟರೆ
ಬಿಟ್ಟುಬಿಟ್ಟು
ಬಿಟ್ಟು-ಬಿ-ಡಲು
ಬಿಟ್ಟುಬಿಡು
ಬಿಟ್ಟು-ಬಿ-ಡುತ್ತವೆ
ಬಿಟ್ಟೇನು
ಬಿಟ್ಟೇ-ಳು-ವುದು
ಬಿಡ-ಕೂ-ಡ-ದೆಂಬುದು
ಬಿಡದ
ಬಿಡದಂತೆ
ಬಿಡದಂಥ
ಬಿಡದಿ
ಬಿಡ-ದಿದ್ದರೆ
ಬಿಡದು
ಬಿಡದೆ
ಬಿಡದೇ
ಬಿಡ-ಬ-ಹುದು
ಬಿಡ-ಬಾ-ರದು
ಬಿಡಬೇಕು
ಬಿಡಬೇಡಿ
ಬಿಡ-ಲಾ-ಗು-ವು-ದಿಲ್ಲ
ಬಿಡ-ಲಾ-ರವು
ಬಿಡಲಾರೆ
ಬಿಡಲಿ
ಬಿಡಲಿಲ್ಲ
ಬಿಡ-ಲಿಲ್ಲ-ವೆಂದೆ-ಣಿಸಿ
ಬಿಡಲು
ಬಿಡಲ್ಪಟ್ಟ
ಬಿಡಾ-ರ-ವಿದ್ದಾಗ
ಬಿಡಿ
ಬಿಡಿ-ಬಿ-ಡಿ-ಯಾಗಿ
ಬಿಡಿ-ಸ-ಲಾ-ಗದ
ಬಿಡಿ-ಸ-ಲಾ-ಗದ
ಬಿಡಿಸಲು
ಬಿಡಿಸಿ
ಬಿಡಿ-ಸಿ-ಕೊಳ್ಳು-ವುದು
ಬಿಡಿ-ಸಿ-ಕೊಂಡು
ಬಿಡಿ-ಸಿ-ಕೊಳ್ಳ
ಬಿಡಿ-ಸಿ-ಕೊಳ್ಳ-ಬ-ಹುದು
ಬಿಡಿ-ಸಿ-ಕೊಳ್ಳ-ಬ-ಹು-ದೆಂಬು-ದನ್ನು
ಬಿಡಿ-ಸಿ-ಕೊಳ್ಳ-ಲಾ-ಗದ
ಬಿಡಿ-ಸಿ-ಕೊಳ್ಳ-ಲಾ-ರರು
ಬಿಡಿ-ಸಿ-ಕೊಳ್ಳಲು
ಬಿಡಿ-ಸಿ-ಕೊಳ್ಳಲೂ
ಬಿಡಿ-ಸಿ-ಕೊಳ್ಳುವ
ಬಿಡಿ-ಸಿ-ಕೊಳ್ಳು-ವುದು
ಬಿಡಿ-ಸಿ-ದಂತಾ-ಗುತ್ತದೆ
ಬಿಡಿ-ಸಿ-ದರೆ
ಬಿಡಿ-ಸಿದ್ದರು
ಬಿಡಿ-ಸುತ್ತಾನೆ
ಬಿಡಿ-ಸುತ್ತಾರೆ
ಬಿಡಿಸುವ
ಬಿಡಿ-ಸು-ವತ್ತ
ಬಿಡು
ಬಿಡುಗಡೆ
ಬಿಡು-ಗ-ಡೆ-ಯಾ-ಗಲು
ಬಿಡು-ಗ-ಡೆ-ಗೊ-ಳಿ-ಸುವ
ಬಿಡು-ಗ-ಡೆ-ಯನ್ನು
ಬಿಡು-ಗ-ಡೆ-ಯಾ-ಗ-ಬೇ-ಡವೇ
ಬಿಡು-ಗ-ಡೆ-ಯಾ-ಗಲು
ಬಿಡುತ್ತ
ಬಿಡುತ್ತದೆ
ಬಿಡುತ್ತ-ದೆಯೇ
ಬಿಡುತ್ತಲೇ
ಬಿಡುತ್ತವೆ
ಬಿಡುತ್ತಾ
ಬಿಡುತ್ತಾನೆ
ಬಿಡುತ್ತಾ-ನೆಂದು
ಬಿಡುತ್ತಾರೆ
ಬಿಡುತ್ತಿದ್ದ
ಬಿಡುತ್ತಿದ್ದರು
ಬಿಡುತ್ತಿ-ರ-ಲಿಲ್ಲ
ಬಿಡುತ್ತೇನೆ
ಬಿಡುತ್ತೇವೆ
ಬಿಡುತ್ತೇ-ವೆಂದು
ಬಿಡುವ
ಬಿಡು-ವಂತಾ-ಯಿತು
ಬಿಡು-ವ-ವ-ರಲ್ಲ
ಬಿಡು-ವ-ವ-ರೆಗೂ
ಬಿಡು-ವು-ದಕ್ಕೆ
ಬಿಡು-ವು-ದಿಲ್ಲ
ಬಿಡುವುದು
ಬಿಡೆನು
ಬಿತ್ತನ್ನು
ಬಿತ್ತ-ಬಲ್ಲರು
ಬಿತ್ತ-ರಿ-ಸುತ್ತಾರೆ
ಬಿತ್ತ-ರಿ-ಸ-ಬೇ-ಕಾದ
ಬಿತ್ತ-ರಿ-ಸಿದ
ಬಿತ್ತ-ರಿ-ಸಿ-ದಿ-ರಾ-ದರೆ
ಬಿತ್ತಲು
ಬಿತ್ತಾಗಿ
ಬಿತ್ತಿ
ಬಿತ್ತಿದ
ಬಿತ್ತಿ-ದಂತಾ-ಗುತ್ತ-ದಲ್ಲವೆ
ಬಿತ್ತಿ-ದಂತಾ-ಗುತ್ತದೆ
ಬಿತ್ತಿದಂತೆ
ಬಿತ್ತಿದರೆ
ಬಿತ್ತಿದಾಗ
ಬಿತ್ತಿದ್ದಾರೆ
ಬಿತ್ತಿಯೇ
ಬಿತ್ತು
ಬಿತ್ತುತ್ತ
ಬಿತ್ತುತ್ತಾನೋ
ಬಿತ್ತುತ್ತಾರೆ
ಬಿಥೋವನ್
ಬಿದ್ದ
ಬಿದ್ದಂತೆ
ಬಿದ್ದರು
ಬಿದ್ದಲ್ಲಿಂದ
ಬಿದ್ದಳು
ಬಿದ್ದ-ವ-ನಲ್ಲ
ಬಿದ್ದವನು
ಬಿದ್ದಾಗ
ಬಿದ್ದಿತೇ
ಬಿದ್ದಿದೆ
ಬಿದ್ದಿದ್ದ-ವರು
ಬಿದ್ದಿದ್ದವು
ಬಿದ್ದಿರಲು
ಬಿದ್ದಿರುವ
ಬಿದ್ದೀರಿ
ಬಿದ್ದು
ಬಿದ್ದು-ಕೊಂಡಿತು
ಬಿದ್ದು-ಕೊಂಡಿ-ರು-ವು-ದನ್ನು
ಬಿದ್ದುಕೊಂಡು
ಬಿದ್ದು-ಕೊಳ್ಳುವ
ಬಿದ್ದುದನ್ನು
ಬಿದ್ದು-ದ-ರಿಂದ
ಬಿದ್ದು-ದಾ-ದರೂ
ಬಿದ್ದು-ಬಿಟ್ಟಿತು
ಬಿದ್ದು-ಬಿ-ಡ-ಬ-ಹುದು
ಬಿದ್ದು-ಬಿ-ಡ-ಬ-ಹು-ದೆಂಬ
ಬಿದ್ದು-ಬಿ-ಡುತ್ತದೆ
ಬಿದ್ದುಹೋದ
ಬಿದ್ದೆ
ಬಿನದ
ಬಿನ್ನ-ವಿ-ಸಿ-ಕೊಂಡ
ಬಿಭೇ-ಮಿ-ನಾನು
ಬಿಮಿನೀ
ಬಿಮ್ಮುಗಳು
ಬಿಯಾಂಡ್
ಬಿರುಕಿನ
ಬಿರು-ಕಿ-ನಿಂದ
ಬಿರುಕು
ಬಿರು-ಕು-ಗೊಳಿ
ಬಿರುಗಾಳಿ
ಬಿರು-ಗಾ-ಳಿ-ಗೊಡ್ಡಿದ
ಬಿರು-ದಾಂಕಿ-ತನ
ಬಿರು-ದು-ಗ-ಳನ್ನು
ಬಿರುದೂ
ಬಿರುನುಡಿ
ಬಿರು-ನು-ಡಿಯ
ಬಿರು-ನು-ಡಿ-ಯನ್ನೂ
ಬಿರುಸಾಗಿ
ಬಿರು-ಸಾ-ಗಿಯೇ
ಬಿರುಸಿಗೆ
ಬಿರು-ಸಿ-ನಿಂದಲೇ
ಬಿಲ್ಲು-ಗ-ಳಾ-ದರೆ
ಬಿಲ್ಲೆತ್ತಿದೆ
ಬಿಲ್ವ-ಪತ್ರೆ-ಯಲ್ಲಿ
ಬಿಳಿ
ಬಿಳಿ-ಜಾ-ತಿಯ
ಬಿಳಿಯ
ಬಿಳಿಯನೂ
ಬಿಳಿಯನೇ
ಬಿಳಿಯರ
ಬಿಳಿಯರು
ಬಿಳಿರಕ್ತ
ಬಿಸಾ-ಡಿ-ಬಿಟ್ಟ
ಬಿಸಿ
ಬಿಸಿಬಿಸಿ
ಬಿಸಿ-ರಕ್ತ-ವನ್ನು
ಬಿಸಿ-ಲಿ-ನಲ್ಲಿ
ಬಿಸಿಲು
ಬಿಸಿಲೇರಿ
ಬೀಗ
ಬೀಗಕ್ಕೆ
ಬೀಗಿ-ದ-ವ-ರಿಗೆ
ಬೀಗಿ-ಸಿ-ಕೊಳ್ಳುತ್ತಾನೆ
ಬೀಚಿನ
ಬೀಜ
ಬೀಜ-ಕ-ಣ-ಗಳ
ಬೀಜ-ಕೇಂದ್ರಕ್ಕೂ
ಬೀಜ-ಕೇಂದ್ರ-ವಿದೆ
ಬೀಜ-ಗ-ಣಿತ
ಬೀಜ-ಗ-ಳನ್ನು
ಬೀಜ-ಗ-ಳಿಂದ
ಬೀಜದಲ್ಲಿ
ಬೀಜದಿಂದ
ಬೀಜವನ್ನು
ಬೀಜವನ್ನೇ
ಬೀಜ-ವಿಲ್ಲದೇ
ಬೀಡು
ಬೀದಿ
ಬೀದಿಗೊಬ್ಬ
ಬೀದಿ-ಪಾ-ಲಾ-ಗುತ್ತಾರೆ
ಬೀದಿಯ
ಬೀದಿಯಲ್ಲಿ
ಬೀದಿಯಲ್ಲೇ
ಬೀದಿ-ಹೋ-ಕ-ರನ್ನು
ಬೀರ-ಬಲ್ಲವು
ಬೀರ-ಬ-ಹುದು
ಬೀರಲು
ಬೀರಿ-ಕೊಂಡಂತಾ-ಗಿದೆ
ಬೀರಿತು
ಬೀರಿತ್ತು
ಬೀರಿದ
ಬೀರಿದೆ
ಬೀರಿದ್ದ
ಬೀರಿಯೇ
ಬೀರುತ್ತ
ಬೀರುತ್ತದೆ
ಬೀರುತ್ತ-ದೆಂಬು-ದನ್ನು
ಬೀರುತ್ತ-ದೆಂಬುದು
ಬೀರುತ್ತವೆ
ಬೀರುತ್ತಾನೆ
ಬೀರುತ್ತಿ-ರು-ವುದು
ಬೀರುವ
ಬೀರುವನು
ಬೀರುವನ್ನೇ
ಬೀರುವುದು
ಬೀಳದಂತೆ
ಬೀಳ-ದಂದದಿ
ಬೀಳ-ದಿ-ರಲಿ
ಬೀಳ-ದಿ-ರ-ಲೆಂದು
ಬೀಳದು
ಬೀಳ-ಬೇ-ಕಾ-ಗ-ಬ-ಹುದು
ಬೀಳ-ಲಿ-ರುವ
ಬೀಳಿ-ಸಿ-ಕೊಂಡರು
ಬೀಳಿ-ಸು-ವುದು
ಬೀಳುತ್ತದೆ
ಬೀಳುತ್ತಲೇ
ಬೀಳುತ್ತ-ವಷ್ಟೆ
ಬೀಳುತ್ತಿದ್ದ
ಬೀಳುತ್ತಿದ್ದೆ
ಬೀಳುತ್ತಿ-ರು-ವುದು
ಬೀಳುವ
ಬೀಳುವಂತೆ
ಬೀಳು-ವಂಥಾದ್ದಲ್ಲ
ಬೀಳು-ವ-ವ-ರೆ-ಗಿನ
ಬೀಳುವಾಗ
ಬೀಳುವು
ಬೀಳು-ವು-ದ-ರಿಂದ
ಬೀಳುವುದು
ಬೀಳುವುದೊ
ಬೀಳ್ಕೊಂಡೆ
ಬೀಳ್ಕೊಳ್ಳುವ
ಬೀಸ
ಬೀಸ-ತೊ-ಡ-ಗಿತ್ತು
ಬೀಸಿ
ಬೀಸಿದರೂ
ಬೀಸಿದಾಗ
ಬೀಸುತ್ತಲೇ
ಬೀಸುವ
ಬುಗ್ಗೆ
ಬುಗ್ಗೆಯನ್ನು
ಬುಟ್ಟಿಯಲ್ಲಿ
ಬುಡ
ಬುತ್ತಿಯ
ಬುತ್ತಿಯಾಗಿ
ಬುದು
ಬುದ್ಧ
ಬುದ್ಧ-ಗು-ರು-ವಿ-ನಿಂದ
ಬುದ್ಧನ
ಬುದ್ಧ-ನಾ-ಗು-ವು-ದಕ್ಕೆ
ಬುದ್ಧ-ನಿ-ಗೇನು
ಬುದ್ಧನು
ಬುದ್ಧರು
ಬುದ್ಧಿ
ಬುದ್ಧಿ-ವಾ-ದ-ವನ್ನು
ಬುದ್ಧಿಗಳ
ಬುದ್ಧಿಗಳು
ಬುದ್ಧಿಗಳೇ
ಬುದ್ಧಿಗೆ
ಬುದ್ಧಿಜೀವಿ
ಬುದ್ಧಿ-ಜೀ-ವಿ-ಗಳು
ಬುದ್ಧಿ-ಜೀ-ವಿ-ಗ-ಳೆ-ನಿ-ಸಿ-ಕೊಂಡ
ಬುದ್ಧಿ-ಜೀ-ವಿ-ಗ-ಳೆ-ನಿ-ಸಿ-ಕೊಂಡ-ವರೂ
ಬುದ್ಧಿ-ಜೀ-ವಿ-ಗ-ಳೆನ್ನಿ-ಸಿ-ಕೊಂಡ-ವರು
ಬುದ್ಧಿ-ಜೀ-ವಿ-ಯಾ-ಗಲೀ
ಬುದ್ಧಿ-ತಿ-ಳಿ-ಯು-ವು-ದಕ್ಕೂ
ಬುದ್ಧಿ-ನಾ-ಶಾತ್
ಬುದ್ಧಿ-ಪೂರ್ವಕ
ಬುದ್ಧಿ-ಬ-ಲ-ವನ್ನು
ಬುದ್ಧಿಭ್ರ-ಮಣೆ
ಬುದ್ಧಿಭ್ರ-ಮ-ಣೆ-ಯಾದ
ಬುದ್ಧಿಭ್ರ-ಮ-ಣೆ-ಯಿಂದಾಗಿ
ಬುದ್ಧಿಭ್ರಮೆ
ಬುದ್ಧಿ-ಮಾಂದ್ಯ-ಅ-ವ-ರೆಲ್ಲ
ಬುದ್ಧಿಯ
ಬುದ್ಧಿಯನ್ನು
ಬುದ್ಧಿ-ಯನ್ನು-ಪ-ಯೋ-ಗಿಸಿ
ಬುದ್ಧಿಯನ್ನೇ
ಬುದ್ಧಿ-ಯ-ವಳು
ಬುದ್ಧಿಯಿಂದ
ಬುದ್ಧಿ-ಯಿಂದಲ್ಲ-ಅಲ್ಲೆಲ್ಲ
ಬುದ್ಧಿವಂತ
ಬುದ್ಧಿ-ವಂತ-ನಲ್ಲ
ಬುದ್ಧಿ-ವಂತ-ನಾದ
ಬುದ್ಧಿ-ವಂತ-ನೆ-ನಿಸಿ
ಬುದ್ಧಿ-ವಂತ-ರನ್ನೇನೊ
ಬುದ್ಧಿ-ವಂತ-ರಾದ
ಬುದ್ಧಿ-ವಂತ-ರಾ-ದರೂ
ಬುದ್ಧಿ-ವಂತ-ರಿಗೆ
ಬುದ್ಧಿ-ವಂತ-ರಿ-ರಲಿ
ಬುದ್ಧಿ-ವಂತರು
ಬುದ್ಧಿ-ವಂತರೂ
ಬುದ್ಧಿ-ವಂತ-ರೆಂದು
ಬುದ್ಧಿ-ವಂತ-ರೆ-ನಿಸಿ
ಬುದ್ಧಿ-ವಂತ-ರೆ-ನಿ-ಸಿ-ಕೊಂಡ
ಬುದ್ಧಿ-ವಂತ-ರೆ-ನಿ-ಸಿ-ಕೊಂಡ-ವ-ರಲ್ಲಿ
ಬುದ್ಧಿ-ವಂತರೇ
ಬುದ್ಧಿ-ವಂತಳು
ಬುದ್ಧಿ-ವಂತಿಕೆ
ಬುದ್ಧಿ-ವಂತಿ-ಕೆಯ
ಬುದ್ಧಿವಾದ
ಬುದ್ಧಿಶಕ್ತಿ
ಬುದ್ಧಿ-ಶಕ್ತಿಗೂ
ಬುದ್ಧಿ-ಶಕ್ತಿಯ
ಬುದ್ಧಿ-ಶಕ್ತಿ-ಯನ್ನಾ-ಗಲೀ
ಬುದ್ಧಿ-ಶಕ್ತಿ-ಯನ್ನು
ಬುದ್ಧಿ-ಶಕ್ತಿಯು
ಬುದ್ಧಿ-ಶಕ್ತಿ-ಯುಳ್ಳ
ಬುದ್ಧಿ-ಶಾ-ಲಿ-ಗ-ಳಾ-ದ-ವರು
ಬುದ್ಧಿ-ಶಾ-ಲಿ-ಗಳು
ಬುದ್ಧಿ-ಶಾ-ಲಿ-ಗ-ಳೇನೂ
ಬುದ್ಧಿ-ಶಾ-ಲಿ-ಯಾದ
ಬುದ್ಧಿ-ಸಾ-ಮರ್ಥ್ಯ-ವನ್ನೂ
ಬುದ್ಧಿಹೀನ
ಬುದ್ಧಿ-ಹೀ-ನ-ರಾ-ಗಿದ್ದವು
ಬುನಾದಿ
ಬುನಾದಿಯ
ಬುಸು-ಗುಟ್ಟ-ಬೇಕು
ಬುಸು-ಗುಟ್ಟ-ಬೇಡ
ಬುಸುಗುಟ್ಟಿ
ಬುಸುಗುಟ್ಟು
ಬುಸ್
ಬೂಕ-ರ-ನಿಗೆ
ಬೂಕರ್
ಬೂಟ್ಸಿನ
ಬೂತ್
ಬೂದಿ
ಬೂಮರ್ಯಾಂಗ್
ಬೃಹತ್
ಬೃಹತ್ತಾದ
ಬೃಹತ್ದರ್ಶನ
ಬೃಹ-ದಾ-ಕಾ-ರದ
ಬೃಹ-ದಾ-ಕಾ-ರ-ವಾ-ಗಿಯೂ
ಬೃಹ-ದಾ-ರಣ್ಯ-ಕೋ-ಪ-ನಿ-ಷತ್ತಿನ
ಬೆಂಕಿ
ಬೆಂಕಿಗೆ
ಬೆಂಕಿಯ
ಬೆಂಕಿಯನ್ನು
ಬೆಂಕಿಯಲ್ಲಿ
ಬೆಂಗ-ಳೂ-ರಿನ
ಬೆಂಗ-ಳೂ-ರಿ-ನಲ್ಲಿ
ಬೆಂಗಳೂರು
ಬೆಂಗಾವಲು
ಬೆಂಚಿನ
ಬೆಂಜಮಿನ್
ಬೆಂಡಾ-ಗು-ವರು
ಬೆಂದ
ಬೆಂದಾಗ
ಬೆಂದು
ಬೆಂದೆ
ಬೆಂಬತ್ತಿ
ಬೆಂಬಲ
ಬೆಂಬ-ಲ-ವಿ-ರುವ
ಬೆಂಬ-ಲ-ವಿಲ್ಲ
ಬೆಂಬ-ಲ-ವಿಲ್ಲದೆ
ಬೆಂಬಿಡದ
ಬೆಂಬಿಡದೆ
ಬೆಕ್ಕಿನ
ಬೆಕ್ಕು
ಬೆಕ್ಕೊಂದು
ಬೆಚ್ಚಿದರೆ
ಬೆಟ್ಟ-ಗ-ಳನ್ನು
ಬೆಟ್ಟುಮಾಡಿ
ಬೆಡ್ಸೋರ್ಗ-ಳಾ-ಗಿವೆ
ಬೆಣಚು
ಬೆತ್
ಬೆತ್ತದ
ಬೆದ-ಕಿ-ದರೆ
ಬೆದರದೆ
ಬೆದರಿಕೆ
ಬೆದ-ರಿ-ದರೆ
ಬೆದರಿಸಿ
ಬೆದ-ರು-ಗಣ್ಣಿ-ನಿಂದ
ಬೆದ-ರು-ಗಣ್ಣು-ಗ-ಳಿಂದಲೇ
ಬೆದ-ರು-ವುದು
ಬೆನ್ನಟ್ಟಿ-ದರೆ
ಬೆನ್ನನ್ನು
ಬೆನ್ನು
ಬೆನ್ನು-ಮೂ-ಳೆಯ
ಬೆನ್ನು-ಹತ್ತ-ದಂತೆ
ಬೆನ್ನೆಲುಬು
ಬೆನ್ಸನ್
ಬೆಪ್ಪು
ಬೆಪ್ಪು-ತಕ್ಕ-ಡಿ-ಯನ್ನು
ಬೆರ-ಗಾ-ಗಿದ್ದ
ಬೆರ-ಗು-ಗಣ್ಣು-ಗ-ಳಿಂದ
ಬೆರ-ಗು-ಗೊ-ಳಿ-ಸಿದ
ಬೆರ-ಗು-ಗೊ-ಳಿ-ಸಿದ್ದಲ್ಲದೆ
ಬೆರ-ಗು-ಗೊ-ಳಿ-ಸುವ
ಬೆರ-ಳಿ-ಡ-ಬೇಕು
ಬೆರ-ಳಿ-ನಿಂದ
ಬೆರ-ಳು-ಗ-ಳನ್ನು
ಬೆರ-ಳು-ಗಳು
ಬೆರ-ಳೆ-ಣಿ-ಕೆ-ಯಷ್ಟಿ-ರ-ಬ-ಹುದು
ಬೆರೆತು
ಬೆರೆ-ಯುತ್ತಿದ್ದರು
ಬೆರೆ-ಯುತ್ತಿ-ರ-ಲಿಲ್ಲ
ಬೆರೆ-ಯು-ವಾಗ
ಬೆರೆಸಿ
ಬೆಲೆ
ಬೆಲೆ-ಬಾ-ಳುವ
ಬೆಲೆಯನ್ನು
ಬೆಲೆ-ಯಾ-ಗಿದೆ
ಬೆಲ್ಜಿಯಂ
ಬೆಳಕ
ಬೆಳಕನ್ನು
ಬೆಳಕನ್ನೂ
ಬೆಳ-ಕಾ-ಗ-ಬ-ಹುದು
ಬೆಳಕಾದ
ಬೆಳ-ಕಾ-ಯಿ-ತಲ್ಲವೇ
ಬೆಳ-ಕಾ-ಯಿತು
ಬೆಳಕಿಗೆ
ಬೆಳಕಿನ
ಬೆಳ-ಕಿ-ನಲ್ಲಿ
ಬೆಳ-ಕಿ-ನಷ್ಟು
ಬೆಳ-ಕಿ-ನಿಂದ
ಬೆಳ-ಕಿ-ನೆ-ಡೆಗೆ
ಬೆಳ-ಕಿ-ನೆ-ಡೆಗೇ
ಬೆಳ-ಕಿ-ರು-ವಲ್ಲಿ
ಬೆಳಕು
ಬೆಳ-ಕು-ಇವು
ಬೆಳ-ಕು-ಕಂಡು
ಬೆಳ-ಕು-ಗಳ
ಬೆಳ-ಕು-ಗ-ಳಿಗೂ
ಬೆಳ-ಕು-ಗಳು
ಬೆಳಕೂ
ಬೆಳಕೇ
ಬೆಳಗದ
ಬೆಳ-ಗ-ಬೇಕು
ಬೆಳಗಲು
ಬೆಳ-ಗಾ-ಗುತ್ತಲೇ
ಬೆಳ-ಗಾ-ಗು-ವುದು
ಬೆಳ-ಗಾ-ದರೆ
ಬೆಳಗಿ
ಬೆಳಗಿತು
ಬೆಳಗಿದ
ಬೆಳಗಿನ
ಬೆಳ-ಗಿ-ಸಲಿ
ಬೆಳ-ಗಿ-ಸುತ್ತ
ಬೆಳಗು
ಬೆಳಗುತ್ತ
ಬೆಳ-ಗುತ್ತದೆ
ಬೆಳ-ಗುತ್ತಿದೆ
ಬೆಳ-ಗುತ್ತಿ-ರುವ
ಬೆಳಗುವ
ಬೆಳ-ಗು-ವಂಥದು
ಬೆಳ-ಗು-ವನು
ಬೆಳ-ಗು-ವ-ವನು
ಬೆಳ-ಗು-ವಿಕೆ
ಬೆಳ-ಗು-ವು-ದ-ರಲ್ಲಿ
ಬೆಳ-ಗು-ವುದು
ಬೆಳಗ್ಗೆ
ಬೆಳದು
ಬೆಳ-ವ-ಣಿಗೆ
ಬೆಳ-ವ-ಣಿ-ಗೆ-ಗದು
ಬೆಳ-ವ-ಣಿ-ಗೆಗೆ
ಬೆಳ-ವ-ಣಿ-ಗೆಯ
ಬೆಳ-ವ-ಣಿ-ಗೆ-ಯನ್ನು
ಬೆಳ-ವ-ಣಿ-ಗೆ-ಯಾ-ಗದೆ
ಬೆಳ-ವ-ಣಿ-ಗೆ-ಯಾ-ಗು-ವು-ದಕ್ಕೆ
ಬೆಳ-ವ-ಣಿ-ಗೆ-ಯಿಂದ
ಬೆಳ-ವ-ಣಿ-ಗೆಯು
ಬೆಳ-ವ-ಣಿ-ಗೆ-ಯೊಂದಿಗೆ
ಬೆಳ-ಸಿ-ಕೊಂಡು
ಬೆಳ-ಸಿ-ಕೊಳ್ಳ-ಬೇಕು
ಬೆಳ-ಸಿ-ಕೊಳ್ಳುತ್ತವೆ
ಬೆಳಸಿನ
ಬೆಳಿಗ್ಗೆ
ಬೆಳೆ
ಬೆಳೆ-ಯುತ್ತ-ಲಿದೆ
ಬೆಳೆದ
ಬೆಳೆ-ದ-ವನು
ಬೆಳೆ-ದ-ವ-ರಿಗೆ
ಬೆಳೆದಿದ್ದ
ಬೆಳೆ-ದಿ-ರುತ್ತದೆ
ಬೆಳೆ-ದಿಲ್ಲ-ವೇಕೆ
ಬೆಳೆದು
ಬೆಳೆ-ದು-ಬಂದಿತ್ತು
ಬೆಳೆದೆ
ಬೆಳೆಯ
ಬೆಳೆ-ಯ-ದಂತೆ
ಬೆಳೆ-ಯ-ದಿದ್ದರೆ
ಬೆಳೆಯದೆ
ಬೆಳೆಯದೇ
ಬೆಳೆಯನ್ನು
ಬೆಳೆ-ಯ-ಬ-ಹುದು
ಬೆಳೆ-ಯ-ಬೇ-ಕಿದ್ದರೆ
ಬೆಳೆ-ಯ-ಲಾ-ಗದ
ಬೆಳೆಯಲಿ
ಬೆಳೆಯಲು
ಬೆಳೆಯಿತು
ಬೆಳೆ-ಯಿ-ಸಲು
ಬೆಳೆಯಿಸಿ
ಬೆಳೆ-ಯಿ-ಸಿ-ಕೊಂಡರೆ
ಬೆಳೆ-ಯಿ-ಸುತ್ತೆಂಬುದು
ಬೆಳೆಯುತ್ತ
ಬೆಳೆ-ಯುತ್ತದೆ
ಬೆಳೆ-ಯುತ್ತವೆ
ಬೆಳೆ-ಯುತ್ತಾನೆ
ಬೆಳೆ-ಯುತ್ತಾರೆ
ಬೆಳೆ-ಯುತ್ತಿದೆ
ಬೆಳೆ-ಯುತ್ತಿದ್ದರು
ಬೆಳೆ-ಯುತ್ತಿದ್ದಾರೆ
ಬೆಳೆ-ಯುತ್ತಿ-ರುವ
ಬೆಳೆಯುವ
ಬೆಳೆ-ಯು-ವಂತೆ
ಬೆಳೆ-ಸ-ಲಾ-ಗುತ್ತಿದೆ
ಬೆಳೆಸಲು
ಬೆಳೆಸಿ
ಬೆಳೆ-ಸಿ-ಕೊಂಡ
ಬೆಳೆ-ಸಿ-ಕೊಂಡಂಥ
ಬೆಳೆ-ಸಿ-ಕೊಂಡ-ತುಂಬ
ಬೆಳೆ-ಸಿ-ಕೊಂಡ-ಮೇಲೆ
ಬೆಳೆ-ಸಿ-ಕೊಂಡರೆ
ಬೆಳೆ-ಸಿ-ಕೊಂಡ-ವ-ರಿಗೆ
ಬೆಳೆ-ಸಿ-ಕೊಂಡಿದ್ದಳು
ಬೆಳೆ-ಸಿ-ಕೊಂಡಿದ್ದಾರೆ
ಬೆಳೆ-ಸಿ-ಕೊಂಡಿದ್ದಾ-ರೆಯೇ
ಬೆಳೆ-ಸಿ-ಕೊಂಡಿ-ರುವ
ಬೆಳೆ-ಸಿ-ಕೊಂಡು
ಬೆಳೆ-ಸಿ-ಕೊಂಡೆ
ಬೆಳೆ-ಸಿ-ಕೊಳ್ಳ-ದ-ವರೇ
ಬೆಳೆ-ಸಿ-ಕೊಳ್ಳ-ಬ-ಹುದು
ಬೆಳೆ-ಸಿ-ಕೊಳ್ಳ-ಬೇ-ಕಾದ
ಬೆಳೆ-ಸಿ-ಕೊಳ್ಳ-ಬೇಕು
ಬೆಳೆ-ಸಿ-ಕೊಳ್ಳ-ಲಾ-ರದು
ಬೆಳೆ-ಸಿ-ಕೊಳ್ಳಲಿ
ಬೆಳೆ-ಸಿ-ಕೊಳ್ಳಲು
ಬೆಳೆ-ಸಿ-ಕೊಳ್ಳಿ
ಬೆಳೆ-ಸಿ-ಕೊಳ್ಳುತ್ತದೆ
ಬೆಳೆ-ಸಿ-ಕೊಳ್ಳುತ್ತಾನೆ
ಬೆಳೆ-ಸಿ-ಕೊಳ್ಳುತ್ತಾರೆ
ಬೆಳೆ-ಸಿ-ಕೊಳ್ಳುತ್ತಿದ್ದೇ-ವೆಯೇ
ಬೆಳೆ-ಸಿ-ಕೊಳ್ಳುವ
ಬೆಳೆ-ಸಿ-ಕೊಳ್ಳು-ವನು
ಬೆಳೆ-ಸಿ-ಕೊಳ್ಳು-ವು-ದ-ರಿಂದಷ್ಟೇ
ಬೆಳೆ-ಸಿ-ದ-ವ-ರೆಲ್ಲ
ಬೆಳೆ-ಸಿದ್ದೇನೆ
ಬೆಳೆ-ಸು-ವಂತಾ-ಯಿತು
ಬೆಳೆ-ಸುತ್ತ-ದೆಯೇ
ಬೆಳೆ-ಸುತ್ತಾರೆ
ಬೆಳೆ-ಸುತ್ತಿದ್ದೇ-ವೆಯೇ
ಬೆಳೆಸುವ
ಬೆಳೆ-ಸು-ವಂತಾ-ದರೆ
ಬೆಳೆ-ಸು-ವ-ವರು
ಬೆಳೆ-ಸು-ವು-ದಕ್ಕೆ
ಬೆಳ್ಳಂಬೆ-ಳ-ಗಾ-ಗು-ವಾ-ಗಲೆ
ಬೆವತು
ಬೆವರಿನ
ಬೆವ-ರಿ-ಳಿಸಿ
ಬೆವ-ರುತ್ತದೆ
ಬೆವೆತು
ಬೆಸುಗೆ
ಬೆಸೆಯುವ
ಬೇಂದ್ರೆ
ಬೇಂದ್ರೆ-ಯ-ವರು
ಬೇಕನ್
ಬೇಕಪ್ಪಾ
ಬೇಕಲ್ಲ
ಬೇಕಲ್ಲವೆ
ಬೇಕಷ್ಟು
ಬೇಕಷ್ಟೆ
ಬೇಕಾ
ಬೇಕಾ-ಗ-ಬ-ಹು-ದೆಂಬು-ದನ್ನು
ಬೇಕಾಗಿದೆ
ಬೇಕಾ-ಗಿ-ರು-ವು-ದಕ್ಕೆ
ಬೇಕಾ-ಗಿ-ರು-ವುದು
ಬೇಕಾಗಿಲ್ಲ
ಬೇಕಾ-ಗುತ್ತದೆ
ಬೇಕಾ-ಗುತ್ತ-ದೆಂದು
ಬೇಕಾ-ಗುತ್ತಿದ್ದವು
ಬೇಕಾಗುವ
ಬೇಕಾ-ಗು-ವುದು
ಬೇಕಾದ
ಬೇಕಾದದ್ದೆ
ಬೇಕಾದರೂ
ಬೇಕಾದರೆ
ಬೇಕಾದಲ್ಲಿ
ಬೇಕಾ-ದ-ವ-ನಲ್ಲ
ಬೇಕಾ-ದ-ವರು
ಬೇಕಾದವು
ಬೇಕಾ-ದಷ್ಟನ್ನು
ಬೇಕಾ-ದಷ್ಟಿವೆ
ಬೇಕಾದಷ್ಟು
ಬೇಕಾದಾಗ
ಬೇಕಾ-ದು-ದನ್ನು
ಬೇಕಾದುದು
ಬೇಕಾದುದೇ
ಬೇಕಾಯಿತು
ಬೇಕಿತ್ತು
ಬೇಕಿದ್ದರೆ
ಬೇಕಿ-ರ-ಲಿಲ್ಲ
ಬೇಕಿಲ್ಲ
ಬೇಕೀ
ಬೇಕು
ಬೇಕು-ಒಟ್ಟಿ-ನಲ್ಲಿ
ಬೇಕುಗಳ
ಬೇಕು-ಗ-ಳನ್ನು
ಬೇಕು-ಗ-ಳಾ-ಗ-ದಿ-ರಲಿ
ಬೇಕು-ಗ-ಳಾ-ಗಲಿ
ಬೇಕು-ಗ-ಳಿಗೆ
ಬೇಕುವಿಗೆ
ಬೇಕುವಿನ
ಬೇಕೆ
ಬೇಕೆಂತಲೇ
ಬೇಕೆಂದಾಗ
ಬೇಕೆಂದಿದ್ದರೆ
ಬೇಕೆಂದಿ-ರು-ವ-ವರ
ಬೇಕೆಂದು
ಬೇಕೆಂಬು-ದನ್ನು
ಬೇಕೆಂಬುದು
ಬೇಕೆಂಬುದೇ
ಬೇಕೆ-ನಿ-ಸಿದ
ಬೇಕೆ-ನಿ-ಸಿ-ದರೂ
ಬೇಕೆನ್ನುತ್ತಾನೆ
ಬೇಕೇ
ಬೇಕೇನು
ಬೇಕೇ-ಬೇ-ಕಾ-ಗುತ್ತದೆ
ಬೇಕೋ
ಬೇಗ
ಬೇಗನೆ
ಬೇಗನೇ
ಬೇಗ-ಬೇ-ಗನೇ
ಬೇಗೆಯಲ್ಲಿ
ಬೇಜಾರು
ಬೇಟೆ
ಬೇಟೆ-ಗಾ-ರ-ನಿಂದ
ಬೇಟೆಯಲ್ಲಿ
ಬೇಟೆ-ಯಲ್ಲಿನ
ಬೇಟೆಯಾಡಿ
ಬೇಡ
ಬೇಡ-ಲಾ-ಗು-ವು-ದೆಂದು
ಬೇಡಎಂದು
ಬೇಡ-ಗ-ಳು-ಇ-ವು-ಗ-ಳೆಲ್ಲ
ಬೇಡದ
ಬೇಡಮ್ಮ
ಬೇಡ-ವಾ-ಗಿ-ರು-ವು-ದನ್ನು
ಬೇಡವಾದ
ಬೇಡ-ವಾ-ದ-ವನು
ಬೇಡ-ವಾ-ದು-ದನ್ನು
ಬೇಡ-ವಾ-ದು-ದರ
ಬೇಡ-ವೆಂದರೂ
ಬೇಡ-ವೆಂದರೆ
ಬೇಡ-ವೆಂದಲ್ಲ
ಬೇಡವೆಂದು
ಬೇಡವೇ
ಬೇಡಿ
ಬೇಡಿಕೆ
ಬೇಡಿ-ಕೆ-ಗ-ಳನ್ನು
ಬೇಡಿ-ಕೆ-ಗ-ಳಿಗೆ
ಬೇಡಿ-ಕೆ-ಗಳು
ಬೇಡಿಕೆಯು
ಬೇಡಿಕೊಂಡ
ಬೇಡಿ-ಕೊಂಡಳು
ಬೇಡಿ-ಕೊಳ್ಳುತ್ತಿವೆ
ಬೇಡಿ-ಕೊಳ್ಳು-ವಂತೆ
ಬೇಡಿದರೆ
ಬೇಡಿದೆ
ಬೇಡಿದ್ದನ್ನು
ಬೇಡಿ-ಯಾ-ದರೂ
ಬೇಡಿಯಿಂದ
ಬೇಡುತ್ತ
ಬೇಡುತ್ತಾರೆ
ಬೇಡುತ್ತಿದ್ದೇನೆ
ಬೇಡುವ
ಬೇಡು-ವ-ವ-ರಿಗೆ
ಬೇತಾಳ
ಬೇನೆ
ಬೇನೆ-ಗ-ಳನ್ನು
ಬೇನೆ-ಗ-ಳಿಗೇ
ಬೇನೆಗೆ
ಬೇನೆಯ
ಬೇಯಿಸಿ
ಬೇಯಿ-ಸಿ-ಕೊಳ್ಳು-ವು-ದ-ರಲ್ಲೇ
ಬೇಯು-ವು-ದುಂಟೆ
ಬೇಯೋದಕ್ಕೆ
ಬೇರಲ್ಲ
ಬೇರಾರಿಗೂ
ಬೇರಾರೂ
ಬೇರಾವ
ಬೇರಾವುದೂ
ಬೇರಿಗೆ
ಬೇರಿಲ್ಲ
ಬೇರು
ಬೇರು-ಗ-ಳಿಲ್ಲದ
ಬೇರುಬಿಟ್ಟ
ಬೇರು-ಬಿಟ್ಟರೆ
ಬೇರು-ಬಿಟ್ಟಿದ್ದ
ಬೇರುಬಿಟ್ಟು
ಬೇರೂರಿತ್ತು
ಬೇರೂರಿದ
ಬೇರೂ-ರಿ-ರುವ
ಬೇರೂ-ರು-ವಂತೆ
ಬೇರೆ
ಬೇರೆ-ಯಾ-ಗು-ವು-ದನ್ನು
ಬೇರೆಡೆ
ಬೇರೆಡೆಗೆ
ಬೇರೆಬೇರೆ
ಬೇರೆ-ಬೇ-ರೆ-ಯಾ-ಯಿತು
ಬೇರೆಯಲ್ಲ
ಬೇರೆ-ಯ-ವ-ರಿಂದ
ಬೇರೆ-ಯ-ವ-ರಿಗೆ
ಬೇರೆ-ಯ-ವರು
ಬೇರೆ-ಯಾ-ಗ-ಬ-ಹುದು
ಬೇರೆಯಾಗಿ
ಬೇರೆ-ಯಾ-ಗಿ-ರ-ಲಿಲ್ಲ
ಬೇರೆ-ಯಾ-ದರೂ
ಬೇರೆಯೇ
ಬೇರೆಲ್ಲ
ಬೇರೆಲ್ಲಿ-ಯಾ-ದರೂ
ಬೇರೆಲ್ಲೂ
ಬೇರೇ-ನಾ-ದರೂ
ಬೇರೇನೂ
ಬೇರೊಂದಿಲ್ಲ
ಬೇರೊಂದು
ಬೇರ್ಪ-ಡಿ-ಸಲು
ಬೇರ್ಪಡಿಸಿ
ಬೇರ್ಪ-ಡಿ-ಸುತ್ತದೆ
ಬೇಲಿಯನ್ನು
ಬೇಲೂರು
ಬೇಳೆ
ಬೇಳೆಕಾಳು
ಬೇಳೆ-ಕಾ-ಳು-ಗಳ
ಬೇಳೆ-ಬೇ-ಯಿ-ಸಿ-ಕೊಳ್ಳಲು
ಬೇವನ್ನು
ಬೇವು
ಬೇಸಗೆಯ
ಬೇಸತ್ತು
ಬೇಸರ
ಬೇಸ-ರ-ದಲ್ಲಿ-ರುತ್ತಾನೆ
ಬೇಸ-ರ-ಗೊಳ್ಳದೆ
ಬೇಸ-ರ-ವಾ-ಗ-ದಂತೆ
ಬೇಸ-ರ-ವಾ-ಗು-ವಂತೆ
ಬೇಸ-ರ-ವಿಲ್ಲ
ಬೇಸಾ-ಯ-ಗಾ-ರ-ರನ್ನು
ಬೇಸಾ-ಯ-ಗಾ-ರರು
ಬೇಹಾರ
ಬೇಹಾರ್
ಬೈಗಳ
ಬೈಗಳಿಗೆ
ಬೈಗಳು
ಬೈಗುಳು
ಬೈದು
ಬೈದು-ಭಂಗಿ-ಸು-ವು-ದಾ-ಗಲೀ
ಬೈಬಲ್
ಬೈಬಲ್ಲನ್ನು
ಬೈಯ-ಬ-ಹುದು
ಬೈಯಲು
ಬೈಯುತ್ತಾರೋ
ಬೈಯುತ್ತಿದ್ದ
ಬೈಯುತ್ತೀ-ರೆಂದು
ಬೈಯು-ವ-ವರು
ಬೈಯ್ಯುತ್ತಾಳೆ
ಬೈಯ್ಯುವ
ಬೈಯ್ಯು-ವು-ದ-ರಲ್ಲಿ
ಬೈರನ್
ಬೈರಾಗಿ
ಬೈರಾಗಿಗೆ
ಬೈರಾ-ಗಿ-ಯೊಬ್ಬ
ಬೊಂತೆ
ಬೊಂಬಾಯಿಗೆ
ಬೊಂಬಾಯಿಯ
ಬೊಂಬಾ-ಯಿ-ಯಲ್ಲಿ
ಬೊಂಬಾ-ಯಿ-ಯಿಂದ
ಬೊಕ್ಕೆ
ಬೊಕ್ಕೆಯಂತೆ
ಬೊಕ್ಕೆಯಲ್ಲಿ
ಬೊಕ್ಕೆಯು
ಬೊಗ-ಳುತ್ತದೆ
ಬೊಗ-ಸೆ-ಯಿಂದ
ಬೊಜ್ಜು
ಬೊಮ್ಮಗಿಹ
ಬೋಗಿ-ಗ-ಳಂತೆ
ಬೋಧನೆ
ಬೋಧ-ನೆ-ಇ-ವಾ-ವುವೂ
ಬೋಧ-ನೆ-ಗ-ಳಲ್ಲಿ
ಬೋಧ-ನೆ-ಗ-ಳಾ-ಗಲಿ
ಬೋಧ-ನೆ-ಗಳು
ಬೋಧನೆಗೇ
ಬೋಧನೆಯ
ಬೋಧ-ನೆ-ಯನ್ನು
ಬೋಧ-ನೆ-ಯಲ್ಲ
ಬೋಧ-ನೆ-ಯಲ್ಲೂ
ಬೋಧ-ನೆ-ಯಿಂದ
ಬೋಧ-ನೆ-ಯಿಂದಲೇ
ಬೋಧನೆಯು
ಬೋಧನೆಯೇ
ಬೋಧಿ-ಸ-ಬೇಕು
ಬೋಧಿ-ಸ-ಬೇಡಿ
ಬೋಧಿ-ಸ-ಲಾ-ಗಿದೆ
ಬೋಧಿಸಲು
ಬೋಧಿ-ಸ-ಹೊ-ರ-ಟಂತೆ
ಬೋಧಿ-ಸ-ಹೊ-ರ-ಡ-ದಿದ್ದರೂ
ಬೋಧಿಸಿ
ಬೋಧಿಸಿದ
ಬೋಧಿ-ಸಿ-ದಂತಹ
ಬೋಧಿ-ಸಿ-ದರು
ಬೋಧಿಸಿದ್ದ
ಬೋಧಿ-ಸಿದ್ದರು
ಬೋಧಿ-ಸಿದ್ದಾರೆ
ಬೋಧಿ-ಸುತ್ತದೆ
ಬೋಧಿ-ಸುತ್ತವೆ
ಬೋಧಿಸುತ್ತಾ
ಬೋಧಿ-ಸುತ್ತಾರೆ
ಬೋಧಿ-ಸುತ್ತಾ-ರೆಂದು
ಬೋಧಿ-ಸುತ್ತಿದ್ದರು
ಬೋಧಿ-ಸುತ್ತೇನೆ
ಬೋಧಿಸುವ
ಬೋಧಿ-ಸು-ವು-ದಕ್ಕಾ-ಗಿಯೇ
ಬೋಧಿ-ಸು-ವುದು
ಬೋಧೆ
ಬೋಧೆಯಲ್ಲಿ
ಬೋರು
ಬೋರ್
ಬೋರ್ಡನ್ನು
ಬೋಳಾಗಿ
ಬೋಳಿಸಿ
ಬೋಸ್
ಬೋಸ್ಟನ್ನಿಗೆ
ಬೋಸ್ಟನ್ನಿ-ನಲ್ಲಿ
ಬೌದ್ಧ
ಬೌದ್ಧಧರ್ಮ
ಬೌದ್ಧ-ಮ-ತ-ವನ್ನು
ಬೌದ್ಧಿಕ
ಬೌದ್ಧಿ-ಕ-ಮಟ್ಟ-ದಲ್ಲಿ
ಬೌದ್ಧಿ-ಕ-ವಾಗಿ
ಬೌದ್ಧಿ-ಕ-ವಾದ
ಬ್ಯಾ
ಬ್ಯಾಂಕಿಗೆ
ಬ್ಯಾಂಕಿನ
ಬ್ಯಾಂಕಿನಿಂದ
ಬ್ಯಾಂಕ್
ಬ್ಯಾಂಡೇಜನ್ನು
ಬ್ಯಾಂಡೇಜ್
ಬ್ಯಾಂಡ್
ಬ್ಯಾಡ್ಮಿಂಟನ್
ಬ್ಯಾಲೆನ್ಸ್
ಬ್ಯಾಲೆನ್ಸ್ಶೀಟ್
ಬ್ರಂಟನ್
ಬ್ರದರ್
ಬ್ರಶ್ಶನ್ನು
ಬ್ರಶ್ಶಿನಿಂದ
ಬ್ರಹ್ಮ
ಬ್ರಹ್ಮ-ಚೈ-ತನ್ಯ
ಬ್ರಹ್ಮಚರ್ಯ
ಬ್ರಹ್ಮ-ಚರ್ಯ-ದಲ್ಲಿ
ಬ್ರಹ್ಮಚಾರಿ
ಬ್ರಹ್ಮ-ಚೈ-ತನ್ಯ
ಬ್ರಹ್ಮ-ಚೈ-ತನ್ಯರ
ಬ್ರಹ್ಮ-ಚೈ-ತನ್ಯ-ರಿಂದ
ಬ್ರಹ್ಮ-ಚೈ-ತನ್ಯರು
ಬ್ರಹ್ಮ-ನುದ್ಯಾ-ನ-ವನ
ಬ್ರಹ್ಮಾ-ನಂದ-ರನ್ನು
ಬ್ರಹ್ಮಾ-ನಂದರು
ಬ್ರಹ್ಮಾಂಡದ
ಬ್ರಹ್ಮಾ-ನಂದ-ದಲ್ಲಿ
ಬ್ರಹ್ಮಾ-ನಂದರ
ಬ್ರಹ್ಮಾ-ನಂದ-ರಿಗೆ
ಬ್ರಹ್ಮಾ-ನಂದರು
ಬ್ರಹ್ಮಾ-ನಂದ-ರೆಂದು
ಬ್ರಾಹ್ಮಣ
ಬ್ರಾಹ್ಮ-ಣ-ಜಾ-ತಿಗೆ
ಬ್ರಾಹ್ಮ-ಣ-ಜಾ-ತಿಯ
ಬ್ರಾಹ್ಮ-ಣತ್ವದ
ಬ್ರಾಹ್ಮ-ಣ-ರಲ್ಲಾ-ಗಲೀ
ಬ್ರಾಹ್ಮ-ಣ-ರಿಗೆ
ಬ್ರಾಹ್ಮಣರು
ಬ್ರಾಹ್ಮಣಿಯ
ಬ್ರಾಹ್ಮ-ಣೇ-ತ-ರ-ರಲ್ಲಾ-ಗಲೀ
ಬ್ರಾಹ್ಮ-ಣೇ-ತ-ರರೂ
ಬ್ರಾಹ್ಮ-ಣೇ-ತ-ರರೇ
ಬ್ರಿಟನ್ನಿನ
ಬ್ರಿಟನ್ನಿ-ನಲ್ಲಿ
ಬ್ರಿಟಿಷರ
ಬ್ರಿಟಿ-ಷ-ರನ್ನು
ಬ್ರಿಟಿಷರು
ಬ್ರಿಟಿಷ್
ಬ್ರೂನೋ
ಬ್ರೆಜಿಲ್ಲಿನ
ಬ್ರೇಕ್
ಬ್ರೈಲ್
ಬ್ರೌನಿಂಗ್
ಬ್ಲಾಂಟನ್ರ
ಬ್ಲೆಂಡ್
ಬ್ಲೇಡಿನಿಂದ
ಬ್ಲ್ಯಾಕ್
ಭಂಗ
ಭಂಗ-ಗೊ-ಳಿ-ಸು-ವುದು
ಭಂಗ-ವಾ-ಗ-ಬ-ಹುದು
ಭಂಗ-ವುಂಟಾ-ಗಿಯೇ
ಭಂಗಿ-ಗ-ಳಲ್ಲಿ
ಭಂಗಿಸಿ
ಭಂಗಿ-ಸಿ-ದರೆ
ಭಂಗಿಸುವ
ಭಂಡಾ-ರ-ದಿಂದ
ಭಂಡಾ-ರ-ವನ್ನು
ಭಕ್ತ
ಭಕ್ತ-ಜ-ನರು
ಭಕ್ತನ
ಭಕ್ತ-ನಿ-ಗಾಗಿ
ಭಕ್ತನಿಗೆ
ಭಕ್ತನೂ
ಭಕ್ತ-ನೆ-ನಿ-ಸಿ-ಕೊಂಡ-ವನು
ಭಕ್ತನೊಬ್ಬ
ಭಕ್ತ-ನೊಬ್ಬ-ನಿಗೆ
ಭಕ್ತ-ಪ-ರಿ-ವೃ-ತ-ರಾಗಿ
ಭಕ್ತರ
ಭಕ್ತರನ್ನು
ಭಕ್ತರನ್ನೂ
ಭಕ್ತರಾಗಿ
ಭಕ್ತರು
ಭಕ್ತ-ರೊಂದಿಗೆ
ಭಕ್ತ-ರೊಬ್ಬರು
ಭಕ್ತ-ವತ್ಸಲ
ಭಕ್ತ-ಸಾ-ಧ-ಕರ
ಭಕ್ತಿ
ಭಕ್ತಿಂ
ಭಕ್ತಿಗಳ
ಭಕ್ತಿ-ಗ-ಳಿಂದ
ಭಕ್ತಿಗೆ
ಭಕ್ತಿ-ಪ-ರ-ವ-ಶ-ರಾದ
ಭಕ್ತಿಭಾವ
ಭಕ್ತಿ-ಭಾ-ವದ
ಭಕ್ತಿಯ
ಭಕ್ತಿಯನ್ನು
ಭಕ್ತಿಯನ್ನೂ
ಭಕ್ತಿಯಿಂದ
ಭಕ್ತಿ-ಯಿದ್ದು-ದ-ರಿಂದ
ಭಕ್ತಿ-ಯುಂಟಾ-ಗುತ್ತದೆ
ಭಕ್ತಿ-ರ-ಹಿತ
ಭಕ್ತಿ-ಶಾಸ್ತ್ರದ
ಭಕ್ತಿಶ್ರದ್ಧೆ-ಗ-ಳಿಂದ
ಭಕ್ತ್ಯಾ-ದ-ರ-ಗ-ಳಿಂದ
ಭಕ್ಷಿಸಿ
ಭಕ್ಷ್ಯ-ಭೋಜ್ಯ-ಗ-ಳನ್ನಿಟ್ಟಾಗ
ಭಗ-ವ-ದ-ಭಿ-ಮು-ಖ-ವಾಗಿ
ಭಗವಂತ
ಭಗ-ವಂತನ
ಭಗ-ವಂತ-ನನ್ನು
ಭಗ-ವಂತ-ನನ್ನೇ
ಭಗ-ವಂತ-ನಲ್ಲಿ
ಭಗ-ವಂತ-ನಲ್ಲಿ-ಡುವ
ಭಗ-ವಂತ-ನಿಂದ
ಭಗ-ವಂತ-ನಿ-ಗಾಗಿ
ಭಗ-ವಂತ-ನಿಗೆ
ಭಗ-ವಂತ-ನಿದ್ದಾನೆ
ಭಗ-ವಂತ-ನಿ-ರು-ವು-ದನ್ನು
ಭಗ-ವಂತನು
ಭಗ-ವಂತನೂ
ಭಗ-ವಂತ-ನೆ-ಡೆಗೆ
ಭಗ-ವಂತನೇ
ಭಗ-ವಂತ-ನೊ-ಡನೆ
ಭಗವತ್
ಭಗ-ವತ್ಪ್ರೇ-ಮದ
ಭಗ-ವತ್ಸಾಕ್ಷಾತ್ಕಾ-ರದ
ಭಗ-ವತ್ಸಾನ್ನಿಧ್ಯದ
ಭಗ-ವತ್ಸ್ವ-ರೂ-ಪ-ವನ್ನು
ಭಗ-ವ-ದಿಚ್ಛೆ-ಯಿಂದ
ಭಗವದ್
ಭಗ-ವದ್ಗೀತಾ
ಭಗ-ವದ್ಗೀತೆ
ಭಗ-ವದ್ಗೀ-ತೆಯ
ಭಗ-ವದ್ಗೀ-ತೆ-ಯಲ್ಲಿ
ಭಗ-ವದ್ಭಕ್ತನು
ಭಗ-ವದ್ಭಕ್ತರ
ಭಗ-ವದ್ಭಕ್ತ-ರೆಂದು
ಭಗ-ವದ್ಭಕ್ತಿ
ಭಗ-ವದ್ಭಕ್ತಿ-ಯನ್ನು
ಭಗ-ವದ್ಭಾ-ವನೆ
ಭಗ-ವದ್ಭಾ-ವ-ನೆ-ಗ-ಳನ್ನೂ
ಭಗ-ವದ್ವಿ-ಚಾ-ರ-ವನ್ನು
ಭಗ-ವದ್ವಿ-ಚಾ-ರ-ವಾಗಿ
ಭಗ-ವನ್ನಾ-ಮ-ದಲ್ಲಿ
ಭಗವಾನ್
ಭಗೀರಥ
ಭಜನೆ
ಭಜನೆಗೆ
ಭಜ-ನೆ-ಮಾ-ಡಿ-ಧನ್ಯ-ನಾ-ದೆ-ನೆಂದು
ಭತ್ತವನ್ನು
ಭದ್ರ
ಭದ್ರ-ವಾ-ಗಿ-ದೆಯೇ
ಭದ್ರತೆ
ಭದ್ರನೆಲೆ
ಭದ್ರ-ಪ-ಡಿ-ಸಿ-ಕೊಳ್ಳಿ-ಎಂದಾರೆ
ಭದ್ರವಾಗಿ
ಭಯ
ಭಯ-ಭೀ-ತ-ರಾ-ಗ-ಬಲ್ಲೆವೆ
ಭಯಂಕರ
ಭಯಂಕ-ರನು
ಭಯಂಕ-ರ-ವಾದ
ಭಯಇವು
ಭಯ-ಇ-ವು-ಗಳೇ
ಭಯಕೃತ್
ಭಯಕ್ಕೆ
ಭಯಕ್ಕೊ-ಳ-ಗಾ-ದಾಗ
ಭಯ-ಗ-ಳನ್ನು
ಭಯ-ಗ-ಳಿಗೂ
ಭಯ-ಗ-ಳಿಗೆ
ಭಯಗಳು
ಭಯಗ್ರಸ್ತ
ಭಯಗ್ರಸ್ತ-ನಾ-ಗುತ್ತಾನೆ
ಭಯಗ್ರಸ್ತ-ನಾದ
ಭಯಗ್ರಸ್ತ-ರಾ-ಗು-ವುದು
ಭಯಗ್ರಸ್ತ-ರಾದ
ಭಯಗ್ರಸ್ತ-ಳಾ-ಗಿದ್ದಳು
ಭಯಗ್ರಸ್ತ-ವಾಗಿ
ಭಯಗ್ರಸ್ತ-ವಾ-ಗು-ವು-ದಾ-ಗಲಿ
ಭಯದ
ಭಯದಿಂದ
ಭಯ-ದಿಂದಲೇ
ಭಯ-ದೊಂದಿಗೆ
ಭಯ-ನಾ-ಶನಃ
ಭಯ-ಪ-ಡ-ಬೇ-ಕಿಲ್ಲ
ಭಯ-ಪ-ಡ-ಬೇಕು
ಭಯ-ಪ-ಡ-ಲಿಲ್ಲ
ಭಯ-ಪ-ಡುತ್ತ
ಭಯ-ಪ-ಡುತ್ತದೆ
ಭಯ-ಪ-ಡುತ್ತಿದ್ದರು
ಭಯ-ಪ-ಡುತ್ತಿದ್ದೆ
ಭಯ-ಪ-ಡುವ
ಭಯ-ಪ-ಡು-ವುದು
ಭಯ-ಪ-ಡು-ವು-ದೆಂದರೆ
ಭಯಯಾರೋ
ಭಯವನ್ನು
ಭಯ-ವನ್ನುಂಟು-ಮಾ-ಡುತ್ತವೆ
ಭಯ-ವನ್ನುಂಟು-ಮಾ-ಡುವ
ಭಯವಲ್ಲ
ಭಯ-ವಿಲ್ಲದೇ
ಭಯ-ವಿಹ್ವ-ಲ-ನಾಗಿ
ಭಯ-ವಿಹ್ವ-ಲ-ರಾ-ಗ-ದಿ-ರಲು
ಭಯ-ವಿಹ್ವ-ಲ-ರಾಗಿ
ಭಯ-ವಿಹ್ವ-ಲ-ವಾಗಿ
ಭಯ-ವಿಹ್ವ-ಲ-ವಾ-ಗು-ವುದು
ಭಯವು
ಭಯವೂ
ಭಯ-ವೆಗ್ಗ-ಳಿ-ಸುತ್ತಿದೆ
ಭಯವೇ
ಭಯ-ಸಂಕ-ಟ-ಗ-ಳಿಂದ
ಭಯಾನಕ
ಭಯಾ-ನ-ಕ-ವೆ-ನಿ-ಸು-ವಷ್ಟು
ಭಯಾ-ನ-ಕ-ವಲ್ಲವೆ
ಭಯಾ-ನ-ಕ-ವಾದ
ಭಯಾ-ನ-ಕ-ವಾ-ದದ್ದಾ-ದರೂ
ಭಯಾ-ನ-ಕ-ವಾ-ದು-ದನ್ನು
ಭಯಾ-ನ-ಕ-ವಾ-ದುದು
ಭಯಾ-ನ-ಕ-ವೆ-ನಿ-ಸುವ
ಭಯಾ-ನ-ಕ-ವೆ-ನಿ-ಸು-ವಷ್ಟು
ಭಯಾರ್ತ-ರಾದ
ಭಯೋದ್ವೇ-ಗ-ಗಳ
ಭಯೋದ್ವೇ-ಗ-ಗ-ಳನ್ನು
ಭಯೋದ್ವೇ-ಗ-ಗ-ಳಿಂದ
ಭಯೋದ್ವೇ-ಗ-ಗಳು
ಭರಣ
ಭರಣೆ
ಭರತ
ಭರ-ತ-ಖಂಡ
ಭರ-ತ-ಖಂಡದ
ಭರ-ತ-ಖಂಡ-ದಲ್ಲಿ
ಭರ-ತ-ಖಂಡ-ವನ್ನು
ಭರ-ತ-ವರ್ಷ
ಭರ-ತ-ವರ್ಷದ
ಭರ-ತ-ವರ್ಷ-ದಲ್ಲಿ
ಭರವಸೆ
ಭರ-ವ-ಸೆ-ಎಂದೆಂದೂ
ಭರ-ವ-ಸೆ-ಗ-ಳನ್ನಿ-ಡ-ಬ-ಹುದೋ
ಭರ-ವ-ಸೆ-ಗ-ಳನ್ನು
ಭರ-ವ-ಸೆಗೆ
ಭರ-ವ-ಸೆಯ
ಭರ-ವ-ಸೆ-ಯನ್ನು
ಭರ-ವ-ಸೆ-ಯಿತ್ತರೆ
ಭರ-ವ-ಸೆ-ಯೆಲ್ಲ
ಭರಿ-ಸ-ಲಾ-ಗದ
ಭರಿ-ಸ-ಲಾ-ರದು
ಭರ್ಜ-ರಿ-ಯಾದ
ಭರ್ತಿ
ಭರ್ತೃ-ಹ-ರಿಯ
ಭರ್ತ್ಸನೆ
ಭರ್ತ್ಸ-ನೆ-ಗ-ಳಿಂದ
ಭರ್ತ್ಸ-ನೆ-ಗ-ಳಿಗೆ
ಭವಂತು
ಭವ-ಗ-ಳಿಂದ
ಭವ-ತಾ-ರಿ-ಣಿಯ
ಭವದ
ಭವಾನ್
ಭವಿತವ್ಯ
ಭವಿ-ತವ್ಯದ
ಭವಿಷ್ಯ
ಭವಿಷ್ಯ-ದಲ್ಲೂ
ಭವಿಷ್ಯ-ಗಳ
ಭವಿಷ್ಯ-ಗ-ಳನ್ನು
ಭವಿಷ್ಯ-ಚಿಂತನೆ
ಭವಿಷ್ಯಜ್ಞ-ರಿಂದ
ಭವಿಷ್ಯತ್ಕಾ-ಲ-ಗಳ
ಭವಿಷ್ಯತ್ತಿನ
ಭವಿಷ್ಯದ
ಭವಿಷ್ಯ-ದರ್ಶನ
ಭವಿಷ್ಯ-ದಲ್ಲಿ
ಭವಿಷ್ಯ-ದೆಡೆ
ಭವಿಷ್ಯ-ವನ್ನು
ಭವಿಷ್ಯ-ವಾಣಿ
ಭವಿಷ್ಯ-ವಾ-ದಿ-ಗ-ಳನ್ನೂ
ಭವಿಷ್ಯ-ವಿದೆ
ಭವಿಷ್ಯ-ವಿ-ದೆಯೆ
ಭವಿಷ್ಯ-ವಿನ್ನೂ
ಭವಿಷ್ಯ-ವಿಲ್ಲ
ಭವಿಷ್ಯವು
ಭವ್ಯ
ಭವ್ಯ-ಗ-ಳೆಲ್ಲ
ಭವ್ಯತೆ
ಭವ್ಯ-ವಾ-ಗ-ಬೇ-ಕೆಂಬ
ಭವ್ಯವಾದ
ಭಸ್ಮಾ-ಸು-ರನೇ
ಭಾಗ
ಭಾಗಕ್ಕೂ
ಭಾಗಕ್ಕೆ
ಭಾಗಗಳ
ಭಾಗ-ಗ-ಳನ್ನು
ಭಾಗ-ಗ-ಳಲ್ಲಿ
ಭಾಗ-ಗ-ಳಲ್ಲಿಯೂ
ಭಾಗ-ಗ-ಳಲ್ಲೂ
ಭಾಗ-ಗ-ಳಿಂದ
ಭಾಗ-ಗ-ಳಿಂದಲೂ
ಭಾಗಗಳು
ಭಾಗ-ಗ-ಳೆಲ್ಲ
ಭಾಗಗಳೇ
ಭಾಗದಲ್ಲಿ
ಭಾಗದಲ್ಲೇ
ಭಾಗ-ವ-ತ-ರಾ-ದ-ವರು
ಭಾಗ-ವ-ತಾಗ್ರೇ-ಸ-ರ-ರಾದ
ಭಾಗ-ವ-ತೋತ್ತ-ಮರೂ
ಭಾಗ-ವನ್ನದು
ಭಾಗವನ್ನು
ಭಾಗ-ವ-ಹಿ-ಸಲು
ಭಾಗ-ವ-ಹಿಸಿ
ಭಾಗ-ವ-ಹಿ-ಸಿದ
ಭಾಗ-ವ-ಹಿ-ಸಿ-ದ-ವ-ರಲ್ಲಿ
ಭಾಗವಾಗಿ
ಭಾಗವು
ಭಾಗಶಃ
ಭಾಗಿ-ಗ-ಳಾ-ಗುತ್ತ-ಲಿದ್ದಾರೆ
ಭಾಗಿ-ಗ-ಳಾ-ಗುತ್ತೇವೆ
ಭಾಗಿ-ಯಾ-ಗು-ವುದು
ಭಾಗ್ಯ
ಭಾಗ್ಯ-ಗ-ಳಿಂದ
ಭಾಗ್ಯ-ವಂತ-ರಾ-ದರೂ
ಭಾಗ್ಯವೇ
ಭಾಗ್ಯ-ಶಾ-ಲಿ-ಗಳೂ
ಭಾಗ್ಯ-ಸೂತ್ರ-ವನ್ನು
ಭಾಗ್ಯೋ-ದ-ಯದ
ಭಾನು-ವಾ-ರದ
ಭಾನು-ವಿ-ನು-ದ-ಯಕ್ಕೆ
ಭಾಯಾ
ಭಾಯಿ
ಭಾರ
ಭಾರತ
ಭಾರತಕ್ಕೂ
ಭಾರತಕ್ಕೆ
ಭಾರ-ತ-ಚೀನಾ
ಭಾರತದ
ಭಾರ-ತ-ದಂಥ
ಭಾರ-ತ-ದಲ್ಲಂತೂ
ಭಾರ-ತ-ದಲ್ಲಿ
ಭಾರ-ತ-ದಲ್ಲಿದ್ದೂ
ಭಾರ-ತ-ದಲ್ಲಿ-ರುವ
ಭಾರ-ತ-ದಲ್ಲೂ
ಭಾರ-ತ-ದಲ್ಲೇ
ಭಾರ-ತ-ದಿಂದ
ಭಾರ-ತ-ದಿಂದಲೇ
ಭಾರ-ತ-ದೆಡೆ
ಭಾರ-ತ-ದೇ-ಶಕ್ಕೆ
ಭಾರ-ತ-ವನ್ನು
ಭಾರ-ತ-ವನ್ನೇ
ಭಾರ-ತ-ವಾ-ಸಿ-ಗಳು
ಭಾರತವು
ಭಾರತೀಯ
ಭಾರ-ತೀ-ಯ-ನನ್ನು
ಭಾರ-ತೀ-ಯನೂ
ಭಾರ-ತೀ-ಯರ
ಭಾರ-ತೀ-ಯ-ರಲ್ಲಿ-ರುವ
ಭಾರ-ತೀ-ಯರು
ಭಾರ-ತೀ-ಯರೂ
ಭಾರ-ತೀ-ಯ-ರೊಬ್ಬರು
ಭಾರದ
ಭಾರದಿಂದ
ಭಾರವನ್ನು
ಭಾರವನ್ನೂ
ಭಾರ-ವಾ-ಗುತ್ತಿದ್ದಾರೆ
ಭಾರವಾದ
ಭಾರವಿ
ಭಾರವಿಗೆ
ಭಾರವಿಯ
ಭಾರವು
ಭಾರ-ವೆ-ನಿ-ಸಿದೆ
ಭಾರವೇ
ಭಾರ-ಸು-ಖಕ್ಕೆ
ಭಾರೀ
ಭಾರ್ಯೆಯರು
ಭಾವ
ಭಾವ-ಗ-ಳಿಂದ
ಭಾವ-ಗ-ಳಿಗೆ
ಭಾವ-ಚಿತ್ರ-ಗಳು
ಭಾವ-ಚಿತ್ರಗ್ರಾ-ಹಿ-ಯಾದ
ಭಾವ-ಚಿತ್ರದ
ಭಾವ-ಚಿತ್ರ-ವನ್ನು
ಭಾವ-ಜೀ-ವನ
ಭಾವದ
ಭಾವ-ದ-ವರು
ಭಾವದಿಂದ
ಭಾವ-ದಿಂದಲೇ
ಭಾವ-ದೊಂದಿಗೆ
ಭಾವನಾ
ಭಾವ-ನಾತ್ಮಕ
ಭಾವ-ನಾತ್ಮ-ಕವೂ
ಭಾವ-ನಾ-ವೈ-ಪ-ರೀತ್ಯ-ಗ-ಳನ್ನು
ಭಾವ-ನಾ-ಶಕ್ತಿ-ಗಳ
ಭಾವನೆ
ಭಾವ-ನೆ-ಗಳ
ಭಾವ-ನೆ-ಗ-ಳನ್ನು
ಭಾವ-ನೆ-ಗ-ಳನ್ನೂ
ಭಾವ-ನೆ-ಗ-ಳಲ್ಲಿ
ಭಾವ-ನೆ-ಗ-ಳಾ-ಗಲಿ
ಭಾವ-ನೆ-ಗ-ಳಿಂದ
ಭಾವ-ನೆ-ಗ-ಳಿಗೆ
ಭಾವ-ನೆ-ಗ-ಳಿಗೇ
ಭಾವ-ನೆ-ಗ-ಳಿವೆ
ಭಾವ-ನೆ-ಗಳು
ಭಾವ-ನೆ-ಗಳೂ
ಭಾವ-ನೆ-ಗ-ಳೆಲ್ಲಾ
ಭಾವ-ನೆ-ಗಳೇ
ಭಾವನೆಗೆ
ಭಾವನೆಯ
ಭಾವ-ನೆ-ಯನ್ನ-ನು-ಸ-ರಿಸಿ
ಭಾವ-ನೆ-ಯನ್ನಲ್ಲ
ಭಾವ-ನೆ-ಯನ್ನು
ಭಾವ-ನೆ-ಯನ್ನುಂಟು-ಮಾಡಿ
ಭಾವ-ನೆ-ಯನ್ನುಂಟು-ಮಾ-ಡು-ವುದೋ
ಭಾವ-ನೆ-ಯನ್ನೇ
ಭಾವ-ನೆ-ಯಾ-ಗಿದೆ
ಭಾವ-ನೆ-ಯಾ-ಗಿ-ದೆ-ಐನ್ಸ್ಟೀನ್
ಭಾವನೆಯಿಂ
ಭಾವ-ನೆ-ಯಿಂದ
ಭಾವ-ನೆ-ಯಿಂದಲೇ
ಭಾವ-ನೆ-ಯಿಂದಲ್ಲ
ಭಾವ-ನೆ-ಯಿಂದಿ-ರ-ಬೇಕು
ಭಾವನೆಯು
ಭಾವ-ನೆ-ಯುಂಟಾ-ಗು-ವುದು
ಭಾವನೆಯೂ
ಭಾವನೆಯೇ
ಭಾವ-ನೆ-ಯೊಂದಿಗೆ
ಭಾವ-ಪ-ರ-ವ-ಶತೆ
ಭಾವ-ಪ-ರ-ವ-ಶ-ತೆ-ಯಲ್ಲ
ಭಾವ-ಪೂರ್ಣ-ವಾಗಿ
ಭಾವಪ್ರೇ-ಷಕ
ಭಾವಪ್ರೇ-ಷಣ
ಭಾವಪ್ರೇ-ಷ-ಣೆ-ಯಿಂದಲೋ
ಭಾವವನ್ನು
ಭಾವ-ವನ್ನುಂಟು
ಭಾವವನ್ನೂ
ಭಾವವಲ್ಲ
ಭಾವ-ವಿಹ್ವ-ಲ-ತೆ-ಯನ್ನೇ
ಭಾವವು
ಭಾವ-ವೆಂದಾಯ್ತು
ಭಾವವೆಂದು
ಭಾವಾಂತ-ರ-ವನ್ನು
ಭಾವಿ-ಸ-ಲಿಲ್ಲ
ಭಾವಿಸಿ
ಭಾವಿ-ಸಿ-ದಲ್ಲಿ
ಭಾವಿ-ಸಿದ್ದೆವು
ಭಾವಿ-ಸಿ-ರು-ವು-ದಕ್ಕಿಂತ
ಭಾವಿಸು
ಭಾವಿ-ಸುತ್ತಾನೆ
ಭಾವಿ-ಸುತ್ತಾರೆ
ಭಾವಿ-ಸುತ್ತಿದ್ದರು
ಭಾವಿ-ಸುತ್ತಿದ್ದರೆ
ಭಾವಿಸುವ
ಭಾವಿ-ಸು-ವಿರಾ
ಭಾವಿ-ಸು-ವುದೇ
ಭಾವೀ-ಜೀ-ವ-ನ-ದಲ್ಲಿ
ಭಾವುಕತೆ
ಭಾವೋದ್ವೇಗ
ಭಾವೋದ್ವೇ-ಗ-ಗ-ಳಿಗೆ
ಭಾವೋದ್ವೇ-ಗ-ದಿಂದ
ಭಾವೋದ್ವೇ-ಗ-ವಿಲ್ಲದೆ
ಭಾವೋದ್ವೇ-ಗವೂ
ಭಾಷಣ
ಭಾಷ-ಣ-ಕಾ-ರ-ಸ-ಭಿ-ಕ-ರಲ್ಲಿ-ರಲಿ
ಭಾಷ-ಣ-ಮಾಡಿ
ಭಾಷ-ಣ-ಮಾ-ಡಿದ
ಭಾಷ-ಣ-ವನ್ನು
ಭಾಷ-ಣೋ-ಪ-ಯೋಗಿ
ಭಾಷಾಂತ-ರಿ-ಸುವ
ಭಾಷಾ-ತಜ್ಞ-ರಿಗೆ
ಭಾಷಾ-ವಿಜ್ಞಾ-ನಕ್ಕೆ
ಭಾಷಾ-ವಿ-ವಾದ
ಭಾಷೆ
ಭಾಷೆ-ಗ-ಳನ್ನು
ಭಾಷೆ-ಗ-ಳಲ್ಲಿ
ಭಾಷೆ-ಗ-ಳಲ್ಲೂ
ಭಾಷೆ-ಗ-ಳಿದ್ದಿ-ರ-ಬ-ಹುದು
ಭಾಷೆ-ಗ-ಳಿ-ರ-ಬ-ಹುದು
ಭಾಷೆಯ
ಭಾಷೆಯದ್ದು
ಭಾಷೆಯನ್ನು
ಭಾಷೆಯನ್ನೇ
ಭಾಷೆಯಲ್ಲ
ಭಾಷೆಯಲ್ಲಿ
ಭಾಷೆ-ಯಲ್ಲಿದೆ
ಭಾಷೆ-ಯಲ್ಲಿನ
ಭಾಷೆ-ಯಲ್ಲಿ-ರುವ
ಭಾಷೆಯಲ್ಲೇ
ಭಾಷೆಯಾದ
ಭಾಸ-ವಾ-ಗ-ಬೇಕೆ
ಭಾಸ-ವಾ-ಗುತ್ತದೆ
ಭಾಸ-ವಾ-ಗುತ್ತಿತ್ತು
ಭಾಸ-ವಾ-ಗುತ್ತಿದೆ
ಭಾಸ-ವಾ-ಗುವ
ಭಾಸ-ವಾ-ದಂತೆ
ಭಾಸ-ವಾ-ದರೂ
ಭಾಸ-ವಾ-ದರೆ
ಭಿಕಾ-ರಿ-ಯಂತೆ
ಭಿಕ್ಷಾ
ಭಿಕ್ಷಾ-ಪಾತ್ರೆ-ಯನ್ನು
ಭಿಕ್ಷು-ಕ-ರನ್ನಾಗಿ
ಭಿಕ್ಷುಗಳೂ
ಭಿಕ್ಷುಗಳೇ
ಭಿಕ್ಷೆ
ಭಿನ್ನತೆ
ಭಿನ್ನಭಾವ
ಭಿನ್ನ-ವಾ-ಗಿದೆ
ಭಿನ್ನ-ವಾ-ಗಿದ್ದರೂ
ಭಿನ್ನವಾದ
ಭಿನ್ನ-ವಾ-ದುದು
ಭಿನ್ನಾ-ಭಿಪ್ರಾಯ
ಭೀಕರ
ಭೀಕ-ರ-ವಾದ
ಭೀತ-ರಾ-ದ-ವರು
ಭೀತರೂ
ಭೀತಿ
ಭೀತಿ-ಗ-ಳನ್ನು
ಭೀತಿ-ಮೂ-ಲ-ವಲ್ಲ
ಭೀತಿ-ಮೂ-ಲ-ವಿ-ರ-ಬ-ಹುದು
ಭೀತಿಯ
ಭೀತಿಯನ್ನು
ಭೀತಿ-ಯನ್ನುಂಟು-ಮಾ-ಡಿಯೇ
ಭೀತಿಯನ್ನೇ
ಭೀತಿಯಿಂದ
ಭೀತಿ-ಯಿಂದಲ್ಲವೇ
ಭೀತಿಯೇ
ಭೀಮ-ರಾ-ಯರ
ಭೀಷಣ
ಭೀಷ್ಮ
ಭೀಷ್ಮರ
ಭುಗಿ-ಭು-ಗಿ-ಲೆನೆ
ಭುಗಿ-ಲೇ-ಳುತ್ತವೆ
ಭುಜ
ಭುಜ-ಗ-ಳನ್ನು
ಭೂ
ಭೂಕಂಪನ
ಭೂಕಂಪ-ವಾಗಿ
ಭೂಖಂಡ-ದಲ್ಲಿನ
ಭೂಗರ್ಭದ
ಭೂತ
ಭೂತ-ಕನ್ನ-ಡಿ-ಯಿಂದ
ಭೂತ-ಕಾ-ಲದ
ಭೂತ-ಗ-ಣ-ಗ-ಳನ್ನು
ಭೂತ-ಗ-ಣದ
ಭೂತ-ಗ-ಣ-ದೊ-ಡನೆ
ಭೂತಗಳ
ಭೂತ-ಗ-ಳನ್ನು
ಭೂತ-ಗ-ಳಲ್ಲೂ
ಭೂತ-ಗ-ಳಿ-ಗೆಷ್ಟು
ಭೂತ-ಪಿ-ಶಾ-ಚಿ-ಗಳ
ಭೂತ-ಪಿ-ಶಾ-ಚಿ-ಗ-ಳಿಗೆ
ಭೂತಪ್ರೇ-ತ-ಗಳ
ಭೂತಪ್ರೇ-ತ-ಗ-ಳಿಗೆ
ಭೂತ-ಬಾ-ಧೆ-ಯಿದೆ
ಭೂತ-ಭ-ವಿಷ್ಯ-ಗಳ
ಭೂತವನ್ನು
ಭೂತ-ವಲ್ಲದೆ
ಭೂತಿಯ
ಭೂತಿ-ಯಾ-ಗದೇ
ಭೂಪ್ರ-ದೇ-ಶದ
ಭೂಮಂಡ-ಲದ
ಭೂಮಧ್ಯ
ಭೂಮ-ರೂ-ಪವೇ
ಭೂಮಿ
ಭೂಮಿ-ಕೆ-ಯಲ್ಲಿ
ಭೂಮಿತಾಯಿ
ಭೂಮಿಯ
ಭೂಮಿಯನ್ನು
ಭೂಮಿಯಲ್ಲಿ
ಭೂಮಿಯಲ್ಲೇ
ಭೂಮಿ-ಯಾ-ದರೋ
ಭೂಮಿಯೂ
ಭೂಮ್ಯಾ-ಕರ್ಷಣ
ಭೂಮ್ಯಾ-ಕಾ-ಶ-ಗ-ಳಲ್ಲಿ-ರುವ
ಭೂಲೋಕದ
ಭೂಲೋ-ಕ-ದಲ್ಲೇ
ಭೂಲೋ-ಕ-ದಿಂದ
ಭೂಲೋ-ಕ-ವನ್ನೇ
ಭೂವಿ-ವ-ರ-ಣೆ-ಗಾರ
ಭೂಷಣ
ಭೂಷ-ಣ-ರಾ-ಗಿದ್ದಾ-ರೆಯೇ
ಭೆಟ್ಟಿಯಾಗಿ
ಭೆಟ್ಟಿಯಾದ
ಭೇಟಿ
ಭೇಟಿ-ಮಾ-ಡಲು
ಭೇಟಿಮಾಡಿ
ಭೇಟಿಯಾಗಿ
ಭೇಟಿ-ಯಾ-ಗು-ವೆ-ನೆಂದು
ಭೇಟಿ-ಯಾ-ದರು
ಭೇಟಿ-ಯಾ-ದಳು
ಭೇಟಿ-ಯಾ-ದಾಗ
ಭೇಟಿಯಾದೆ
ಭೇದ
ಭೇದದ
ಭೇದ-ಪೈ-ಪೋ-ಟಿಯ
ಭೇದ-ರ-ಹಿತ
ಭೇದ-ವಿ-ರು-ವುದು
ಭೇದ-ವಿಲ್ಲದೆ
ಭೇದ-ವಿಲ್ಲದೇ
ಭೇದವೂ
ಭೇದ-ವೆಲ್ಲಿದೆ
ಭೇದಿಸಿ
ಭೇದಿ-ಸಿ-ದಳು
ಭೇದಿ-ಸಿದ್ದಾನೆ
ಭೇದಿ-ಸಿದ್ದಾ-ನೆಯೇ
ಭೈರವ
ಭೈರಾಗಿ
ಭೋಕ್ತೃತ್ವ
ಭೋಗ
ಭೋಗವೇ
ಭೋಗಾ-ಕಾಂಕ್ಷೆ-ಯನ್ನು
ಭೋಗ್ಯ-ವಸ್ತು-ಗಳು
ಭೋಜನ
ಭೋಜ-ನ-ವಾದ
ಭೌಗೋಲಿಕ
ಭೌತ
ಭೌತ-ಜ-ಗತ್ತಿಗೆ
ಭೌತದ್ರವ್ಯ-ಗಳ
ಭೌತದ್ರವ್ಯ-ಗ-ಳಿಂದಲೇ
ಭೌತಪ್ರ-ಪಂಚದ
ಭೌತ-ವಸ್ತು-ಗಳ
ಭೌತ-ವಾ-ದದ
ಭೌತ-ವಾ-ದಿ-ಗ-ಳಿಗೆ
ಭೌತ-ವಿಜ್ಞಾನ
ಭೌತ-ವಿಜ್ಞಾ-ನದ
ಭೌತ-ವಿಜ್ಞಾ-ನ-ದಲ್ಲಿ
ಭೌತ-ಶಾಸ್ತ್ರದ
ಭೌತ-ಶಾಸ್ತ್ರ-ದಲ್ಲಿ
ಭೌತಿಕ
ಭೌತಿ-ಕ-ಕಾ-ರ-ಣ-ಗ-ಳನ್ನೂ
ಭೌತಿ-ಕ-ವಲ್ಲದ
ಭೌತಿ-ಕ-ವಸ್ತು
ಭೌತಿಕವೇ
ಭೌತಿ-ಕಸ್ತ-ರ-ದಲ್ಲಿ
ಭ್ರಮ-ಣೆ-ಗೊಂಡ-ವ-ಳಂತೆ
ಭ್ರಮಾತ್ಮಕ
ಭ್ರಮಿಸಿ
ಭ್ರಮಿ-ಸುತ್ತಾನೆ
ಭ್ರಮೆ
ಭ್ರಮೆಯಿಂದ
ಭ್ರಷ್ಟತೆಯ
ಭ್ರಷ್ಟ-ರಾ-ದರು
ಭ್ರಷ್ಟರೂ
ಭ್ರಷ್ಟಾಚಾರ
ಭ್ರಷ್ಟಾ-ಚಾ-ರ-ಗಳ
ಭ್ರಷ್ಟಾ-ಚಾ-ರ-ಗಳು
ಭ್ರಾಂತನಾಗಿ
ಭ್ರಾಂತ-ರಾ-ಗು-ವಿ-ರೇನು
ಭ್ರಾಂತ-ರಿದ್ದಾರೆ
ಭ್ರಾಂತಿ
ಭ್ರಾಂತಿ-ದರ್ಶನ
ಭ್ರಾಮಕ
ಭ್ರೂಣ
ಭ್ರೂಮಧ್ಯ-ದಲ್ಲೂ
ಮಂಕು-ತಿಮ್ಮನ
ಮಂಕು-ತಿಮ್ಮ-ನಿ-ಗನ್ನಿ-ಸಿ-ದಂತೆ
ಮಂಗಗಳ
ಮಂಗಲ
ಮಂಗ-ಲಕ್ಕಾಗಿ
ಮಂಗ-ಲ-ಮಯ
ಮಂಗ-ಲ-ರೂ-ಪ-ವನ್ನು
ಮಂಗ-ಲ-ವಾ-ಗು-ವುದು
ಮಂಗ-ಳ-ಕ-ರ-ವಾದ
ಮಂಗ-ಳ-ವಾ-ಗು-ವುದು
ಮಂಗ-ಳಾ-ರ-ತಿ-ಯಾಯ್ತು
ಮಂಚದ
ಮಂಚದಲ್ಲಿ
ಮಂಜಿನಂತೆ
ಮಂಜು
ಮಂಜುಗಡ್ಡೆ
ಮಂಡಲದ
ಮಂಡಲಿ
ಮಂಡ-ಲಿ-ಯನ್ನು
ಮಂಡ-ಲಿ-ಯಲ್ಲ
ಮಂತ್ರ
ಮಂತ್ರ-ಜ-ಪ-ವನ್ನು
ಮಂತ್ರದ
ಮಂತ್ರದಲ್ಲಿ
ಮಂತ್ರ-ದೀಕ್ಷೆ-ಯನ್ನಿತ್ತು
ಮಂತ್ರ-ದೀಕ್ಷೆ-ಯನ್ನು
ಮಂತ್ರಮಾಟ
ಮಂತ್ರವಾಗಿ
ಮಂತ್ರವೆಂಬ
ಮಂತ್ರಿ
ಮಂತ್ರಿಗಳು
ಮಂತ್ರಿ-ಗ-ಳೊಬ್ಬ-ರನ್ನು
ಮಂತ್ರಿ-ಯಾ-ಗಿದ್ದ
ಮಂತ್ರಿಸಿ
ಮಂತ್ರೋ-ಪ-ದೇಶ
ಮಂತ್ರೋ-ಪ-ದೇಶ
ಮಂತ್ರೋ-ಪ-ದೇ-ಶ-ವನ್ನು
ಮಂಥನ
ಮಂಥ-ನ-ಗಳು
ಮಂದಿ
ಮಂದಿಗೆ
ಮಂದಿಯ
ಮಂದಿಯನ್ನು
ಮಂದಿಯನ್ನೇ
ಮಂದಿಯಲ್ಲಿ
ಮಂದಿ-ಯಾ-ದರೂ
ಮಂದಿ-ಯೊ-ಡನೆ
ಮಂದಿರ
ಮಂದಿ-ರ-ಗ-ಳೆಂದು
ಮಂದಿರದ
ಮಂದಿ-ರ-ದ-ವರು
ಮಂದೆ-ಯಲ್ಲಿದ್ದು-ಕೊಂಡು
ಮಕ್ಕಳ
ಮಕ್ಕಳಂತೆ
ಮಕ್ಕ-ಳಂತೆಯೇ
ಮಕ್ಕಳದೇ
ಮಕ್ಕಳದ್ದು
ಮಕ್ಕಳನ್ನು
ಮಕ್ಕಳನ್ನೂ
ಮಕ್ಕಳಲ್ಲ
ಮಕ್ಕಳಲ್ಲಿ
ಮಕ್ಕ-ಳಾ-ಗಿದ್ದಾಗ
ಮಕ್ಕ-ಳಾ-ಗಿ-ರು-ವಾಗ
ಮಕ್ಕ-ಳಾ-ಟಿಕೆ
ಮಕ್ಕಳಿ
ಮಕ್ಕಳಿಂದ
ಮಕ್ಕ-ಳಿ-ಗಾ-ಗಲೀ
ಮಕ್ಕ-ಳಿ-ಗಾಗಿ
ಮಕ್ಕ-ಳಿ-ಗಿಂತ
ಮಕ್ಕಳಿಗೂ
ಮಕ್ಕಳಿಗೆ
ಮಕ್ಕ-ಳಿಬ್ಬರು
ಮಕ್ಕ-ಳಿ-ರುವ
ಮಕ್ಕ-ಳಿಲ್ಲದ
ಮಕ್ಕಳು
ಮಕ್ಕಳೂ
ಮಕ್ಕಳೆಂದೂ
ಮಕ್ಕ-ಳೆ-ಡೆಗೆ
ಮಕ್ಕಳೆಲ್ಲ
ಮಕ್ಕಳೇ
ಮಕ್ಕ-ಳೊಂದಿಗೆ
ಮಕ್ಕ-ಳೊ-ಡನೆ
ಮಗ
ಮಗಚಲು
ಮಗದೊಬ್ಬ
ಮಗ-ಧಪ್ರಾಂತದ
ಮಗನ
ಮಗ-ನನ್ನಾ-ಗಿ-ಸಿದ್ದು
ಮಗನನ್ನು
ಮಗನಲ್ಲ
ಮಗನಾಗಿ
ಮಗ-ನಾ-ಗಿದ್ದ
ಮಗ-ನಾ-ಗಿಯೇ
ಮಗ-ನಿ-ಗಂತೂ
ಮಗನಿಗೂ
ಮಗನೆಂದು
ಮಗನೆಂದೂ
ಮಗ-ನೊ-ಡನೆ
ಮಗಳ
ಮಗಳನ್ನು
ಮಗಳಾಗಿ
ಮಗ-ಳಾ-ಗಿದ್ದು
ಮಗಳು
ಮಗಳೇ
ಮಗ-ಳೊಂದಿಗೆ
ಮಗು
ಮಗುಚಿ
ಮಗುವನ್ನು
ಮಗುವನ್ನೂ
ಮಗುವನ್ನೋ
ಮಗುವಾಗಿ
ಮಗು-ವಾ-ಗಿದ್ದಾಗ
ಮಗು-ವಾ-ಗಿಯೇ
ಮಗು-ವಾ-ದರೋ
ಮಗುವಿಗೆ
ಮಗುವಿನ
ಮಗು-ವಿ-ನಂತೆಯೇ
ಮಗು-ವಿ-ನಂಥ
ಮಗು-ವಿ-ನಲ್ಲಿ
ಮಗು-ವಿ-ನೊಂದಿಗೆ
ಮಗುವು
ಮಗುವೇ
ಮಗೂ
ಮಗ್ಗಲು
ಮಗ್ಗುಲಲ್ಲೇ
ಮಗ್ನನಾಗಿ
ಮಗ್ನ-ನಾ-ಗಿದ್ದ
ಮಗ್ನ-ನಾ-ಗಿದ್ದು
ಮಗ್ನ-ರಾ-ಗ-ಬೇಕು
ಮಗ್ನ-ರಾ-ಗಿದ್ದ
ಮಗ್ನ-ರಾ-ಗಿ-ರುತ್ತಿದ್ದರು
ಮಗ್ನ-ರಾ-ಗಿ-ರುವ
ಮಗ್ನ-ರಾ-ಗುತ್ತಿದ್ದರು
ಮಗ್ನರಾದ
ಮಗ್ನ-ರಾ-ದುದು
ಮಗ್ನ-ವಾ-ಗು-ವಂಥ
ಮಚ್ಚೆ-ಗ-ಳನ್ನು
ಮಚ್ಚೆಗಳು
ಮಜಲನ್ನು
ಮಜಲು
ಮಜ-ಲು-ಗ-ಳನ್ನು
ಮಟ್ಟ
ಮಟ್ಟಕ್ಕಿ-ಳಿ-ದಾಗ
ಮಟ್ಟಕ್ಕಿ-ಳಿದು
ಮಟ್ಟಕ್ಕೆ
ಮಟ್ಟಕ್ಕೆ-ಳೆ-ಯುತ್ತಾರೆ
ಮಟ್ಟಕ್ಕೆ-ಳೆ-ಯುವ
ಮಟ್ಟಕ್ಕೇ-ರಿ-ದರು
ಮಟ್ಟಕ್ಕೇ-ರುತ್ತದೆ
ಮಟ್ಟದ
ಮಟ್ಟದಲ್ಲಿ
ಮಟ್ಟದಲ್ಲೂ
ಮಟ್ಟ-ದ-ವ-ರೆ-ಗಿನ
ಮಟ್ಟದ್ದಾ-ಗ-ಬ-ಹು-ದೆಂಬು-ದನ್ನು
ಮಟ್ಟದ್ದಾ-ಗಿದೆ
ಮಟ್ಟದ್ದು
ಮಟ್ಟವನ್ನು
ಮಟ್ಟ-ವನ್ನೇ-ರದೆ
ಮಟ್ಟಿ-ಗಾ-ದರೂ
ಮಟ್ಟಿಗೂ
ಮಟ್ಟಿಗೆ
ಮಠ-ದಲ್ಲಿದ್ದಾಗ
ಮಠವನ್ನು
ಮಠ-ವನ್ನೇನೋ
ಮಡಕೆಯ
ಮಡದಿ
ಮಡಿ
ಮಡಿದ
ಮಡಿ-ದ-ರಂತೆ
ಮಡಿದರೆ
ಮಡಿ-ದ-ವರ
ಮಡಿ-ದ-ವರು
ಮಡಿ-ದಿದ್ದರು
ಮಡಿ-ದಿದ್ದಾರೆ
ಮಡಿ-ಮೈ-ಲಿ-ಗೆಯ
ಮಡಿ-ಯ-ಬೇ-ಕೆಂಬ
ಮಡಿ-ಯು-ವ-ವರ
ಮಡಿ-ಯು-ವ-ವರು
ಮಡಿಲಲ್ಲಿ
ಮಡಿ-ಲಲ್ಲಿದ್ದಾನೆ
ಮಡಿ-ಲಾ-ಗಿತ್ತು
ಮಡಿಲು
ಮಡಿಲೇ
ಮಡಿವಾಳ
ಮಡುವಿಗೆ
ಮಡು-ವಿ-ನಲ್ಲಿ
ಮಣಿದು
ಮಣಿ-ಯು-ವಂತೆಯೂ
ಮಣ್ಣಾ-ಗು-ವು-ದನ್ನು
ಮಣ್ಣಿನ
ಮಣ್ಣಿನಲ್ಲಿ
ಮಣ್ಣು
ಮಣ್ಣೆ-ರ-ಚುವ
ಮತ
ಮತಇವು
ಮತ-ಇ-ವು-ಗಳು
ಮತಕ್ಕೆ
ಮತಕ್ಕೇ
ಮತಗಳ
ಮತ-ಗ-ಳಲ್ಲಿ
ಮತ-ಗ-ಳಿಗೆ
ಮತದಲ್ಲಿ
ಮತ-ದ-ವ-ರನ್ನೂ
ಮತ-ದ-ವರು
ಮತ-ದ-ವರೂ
ಮತ-ದಾ-ರ-ರನ್ನು
ಮತ-ದಾ-ರ-ರಲ್ಲಿ
ಮತ-ಪಂಡಿ-ತರ
ಮತಪಥ
ಮತಭೇದ
ಮತವನ್ನು
ಮತ-ವಿ-ಭಿನ್ನ-ರು-ಚಿಯ
ಮತವೂ
ಮತಾಂತರ
ಮತಾಂಧತೆ
ಮತಾಂಧ-ತೆ-ಇವು
ಮತಾಂಧ-ತೆ-ಯನ್ನು
ಮತಾಂಧ-ತೆ-ಯಿಂದ
ಮತಾಂಧ-ನಾ-ಗುವ
ಮತಾಂಧರು
ಮತಾಂಧ-ಳಲ್ಲದ
ಮತಾ-ಚಾ-ರ-ಗಳು
ಮತಾ-ನು-ಯಾ-ಯಿ-ಯಾಗಿ
ಮತಿ
ಮತಿಭ್ರಮೆ
ಮತಿ-ವಂತನೆ
ಮತಿ-ಹೀ-ನ-ತೆ-ಗ-ಳನ್ನು
ಮತಿ-ಹೀ-ನ-ತೆ-ಗಳೇ
ಮತಿ-ಹೀ-ನ-ತೆಯ
ಮತೀಯ
ಮತ್ತ-ನನ್ನಾ-ಗಿ-ಸಿ-ಮಾ-ಯೆಯ
ಮತ್ತ-ನಾ-ಗದೆ
ಮತ್ತರ್ಧ
ಮತ್ತಷ್ಟನ್ನು
ಮತ್ತಷ್ಟು
ಮತ್ತಿತರ
ಮತ್ತು
ಮತ್ತು-ಬ-ರಿ-ಸುವ
ಮತ್ತು-ಮಂಪರು
ಮತ್ತೂ
ಮತ್ತೆ
ಮತ್ತೆಮತ್ತೆ
ಮತ್ತೇ-ನಾ-ಯಿ-ತೆಂದು
ಮತ್ತೇನು
ಮತ್ತೊಂದ-ರಿಂದ
ಮತ್ತೊಂದು
ಮತ್ತೊಬ್ಬ
ಮತ್ತೊಬ್ಬ-ನಿಲ್ಲ
ಮತ್ತೊಮ್ಮೆ
ಮತ್ಸರ
ಮತ್ಸರಕ್ಕೆ
ಮತ್ಸ-ರ-ಗ-ಳಿಂದ
ಮತ್ಸ-ರಗ್ರಸ್ತ
ಮತ್ಸ-ರಗ್ರಸ್ತ-ರಾಗಿ
ಮತ್ಸರದ
ಮತ್ಸ-ರ-ದಿಂದ
ಮತ್ಸ-ರ-ಪ-ಡು-ವಂಥ
ಮತ್ಸರಮ್ಮ
ಮತ್ಸ-ರ-ರಾ-ಹಿತ್ಯ
ಮತ್ಸ-ರ-ವನ್ನು
ಮತ್ಸ-ರ-ವಿಲ್ಲ-ದ-ವರೇ
ಮತ್ಸ-ರಿ-ಗಳು
ಮಥಿತಾರ್ಥ
ಮಥಿಸಿ
ಮದ
ಮದ-ಗ-ಳನ್ನು
ಮದ-ರಾ-ಸಿನ
ಮದರಾಸು
ಮದರ್
ಮದ-ವೇ-ರಿದ
ಮದುವೆ
ಮದು-ವೆ-ಯಾ-ಗ-ಲಿಚ್ಛಿ-ಸಿದ್ದ
ಮದು-ವೆ-ಗೊಂದು
ಮದುವೆಯ
ಮದು-ವೆ-ಯಾ-ಗದೇ
ಮದು-ವೆ-ಯಾ-ಗ-ಬ-ಹುದು
ಮದು-ವೆ-ಯಾ-ಗ-ಬ-ಹುದೆ
ಮದು-ವೆ-ಯಾ-ಗಲು
ಮದು-ವೆ-ಯಾ-ಗಿತ್ತು
ಮದು-ವೆ-ಯಾ-ಗಿಲ್ಲ
ಮದು-ವೆ-ಯಾ-ಗುತ್ತೇನೆ
ಮದು-ವೆ-ಯಾ-ಗುತ್ತೇ-ನೆಂದಾಗ
ಮದು-ವೆ-ಯಾದ
ಮದು-ವೆ-ಯಾ-ದರೂ
ಮದು-ವೆ-ಯಾ-ದರೆ
ಮದು-ವೆ-ಯಾದೆ
ಮದ್ದನ್ನು
ಮದ್ದ-ರೆ-ಯ-ಬೇ-ಕೆಂಬು-ದನ್ನು
ಮದ್ದಿನ
ಮದ್ದು
ಮದ್ದು-ಗುಂಡು-ಗ-ಳಿ-ಗಾಗಿ
ಮದ್ಯ
ಮದ್ಯ-ಕು-ಡಿದು
ಮದ್ಯದ
ಮದ್ಯ-ಪಾ-ನದ
ಮದ್ಯ-ಪಾ-ನ-ಮತ್ತ
ಮದ್ಯವನ್ನು
ಮದ್ಯವನ್ನೇ
ಮದ್ಯವಲ್ಲ
ಮದ್ಯ-ವಾ-ಗಲಿ
ಮದ್ಯ-ಸೇ-ವನೆ
ಮಧುರ
ಮಧ್ಯ
ಮಧ್ಯದ
ಮಧ್ಯದಲಿ
ಮಧ್ಯದಲ್ಲಿ
ಮಧ್ಯದಲ್ಲೇ
ಮಧ್ಯ-ಭಾ-ಗಕ್ಕೆ
ಮಧ್ಯ-ಭಾ-ಗದ
ಮಧ್ಯ-ಭಾ-ಗ-ದಲ್ಲಿ
ಮಧ್ಯ-ಭಾ-ಗ-ವನ್ನು
ಮಧ್ಯ-ಮ-ರದು
ಮಧ್ಯ-ಮ-ವರ್ಗ-ದ-ವರು
ಮಧ್ಯ-ಮ-ವರ್ಗೀ-ಯರು
ಮಧ್ಯರಾತ್ರಿ
ಮಧ್ಯ-ರಾತ್ರಿಯ
ಮಧ್ಯಸ್ಥಿ-ಕೆ-ಯಲ್ಲೆ
ಮಧ್ಯಾಹ್ನ
ಮಧ್ಯಾಹ್ನದ
ಮಧ್ಯಾಹ್ನ-ದ-ವ-ರೆಗೂ
ಮಧ್ಯೆ
ಮನ
ಮನಃಪ್ರ-ವೃತ್ತಿಯು
ಮನಃಶಾಂತಿ
ಮನಃಶಾಸ್ತ್ರಜ್ಞರೂ
ಮನಃಸ್ಥಿ-ತಿ-ಯಿಂದ
ಮನಃಸ್ಥೈರ್ಯ
ಮನಃಸ್ಥೈರ್ಯ-ವನ್ನು
ಮನಃಸ್ವಾಸ್ಥ್ಯ-ವಾ-ಗಲಿ
ಮನ-ಕ-ರ-ಗಿ-ದ-ವ-ರಾಗಿ
ಮನ-ಕ-ರ-ಗಿ-ಸುವ
ಮನ-ಕ-ರ-ಗುವ
ಮನಗಂಡ
ಮನ-ಗಂಡಾ-ಗಲೇ
ಮನಗಂಡು
ಮನಗಂಡೇ
ಮನಗಳ
ಮನಗಳು
ಮನ-ಗಾ-ಣ-ಬ-ಹುದು
ಮನ-ಗಾ-ಣ-ಬೇ-ಕಾ-ಗಿದೆ
ಮನ-ಗಾ-ಣಲಿ
ಮನ-ಗಾ-ಣಿ-ಸಿತು
ಮನ-ಗಾ-ಣುತ್ತಿದೆ
ಮನ-ಗಾ-ಣುತ್ತಿದ್ದಾರೆ
ಮನ-ಗಾ-ಣುತ್ತೇವೆ
ಮನಗೊಟ್ಟು
ಮನ-ದಂತ-ರಾ-ಳ-ದಲ್ಲಿ
ಮನ-ದಟ್ಟಾ-ಗಿತ್ತು
ಮನದಲ್ಲಿ
ಮನ-ದಲ್ಲೆದ್ದ
ಮನ-ದ-ಳ-ಲನ್ನು
ಮನ-ದೊ-ಳಗೆ
ಮನನ
ಮನನೀಯ
ಮನ-ನೀ-ಯ-ವಾ-ಗಿದೆ
ಮನ-ನೋ-ಯಿ-ಸದೇ
ಮನ-ನೋ-ಯಿಸಿ
ಮನ-ನೋ-ಯು-ವಂತೆ
ಮನ-ಬಂದಂತೆ
ಮನಬಿಚ್ಚಿ
ಮನಮಾಡು
ಮನ-ಮುಟ್ಟು-ವಂತೆ
ಮನ-ರಂಜ-ನೆ-ಗಾಗಿ
ಮನವ
ಮನ-ವ-ರಿಕೆ
ಮನ-ವ-ರಿ-ಕೆ-ಯಾ-ದಾಗ
ಮನ-ವ-ರಿ-ಕೆ-ಯಾ-ಗ-ಬೇ-ಕಾ-ಗಿದೆ
ಮನ-ವ-ರಿ-ಕೆ-ಯಾ-ಗುತ್ತಿದೆ
ಮನ-ವ-ರಿ-ಕೆ-ಯಾ-ಗು-ವ-ವ-ರೆಗೆ
ಮನ-ವ-ರಿ-ಕೆ-ಯಾ-ಗು-ವುದು
ಮನ-ವ-ರಿ-ಕೆ-ಯಾದ
ಮನವೆ
ಮನವೆಂದೂ
ಮನವೇ
ಮನಶ್ಚಾಂಚಲ್ಯ-ವನ್ನು
ಮನಶ್ಶಾಂತಿ
ಮನಶ್ಶಾಂತಿಯ
ಮನಶ್ಶಾಂತಿ-ಯನ್ನು
ಮನಶ್ಶಾಂತಿಯೇ
ಮನಶ್ಶಾಸ್ತ್ರಜ್ಞ-ರನ್ನು
ಮನಶ್ಶಾಸ್ತ್ರಜ್ಞೆ
ಮನಶ್ಶಾಸ್ತ್ರವು
ಮನಶ್ಶಾಸ್ತ್ರೀ-ಯ-ವಾಗಿ
ಮನಸಾ
ಮನಸಾರೆ
ಮನ-ಸಿ-ನಲ್ಲಿಯೂ
ಮನ-ಸೆ-ಳೆದ
ಮನ-ಸೆ-ಳೆ-ಯುವ
ಮನ-ಸೋ-ತ-ವರು
ಮನಸ್ಕ-ರಾ-ಗಿದ್ದು
ಮನಸ್ತತ್ವ
ಮನಸ್ತಾಪ
ಮನಸ್ಥೈರ್ಯ
ಮನಸ್ವೀ
ಮನಸ್ಸನ್ನಲ್ಲಿಗೆ
ಮನಸ್ಸನ್ನು
ಮನಸ್ಸನ್ನೂ
ಮನಸ್ಸನ್ನೆಂದೂ
ಮನಸ್ಸನ್ನೇ-ರಿ-ಸ-ಬೇಕು
ಮನಸ್ಸಾಕ್ಷಿಗೆ
ಮನಸ್ಸಾ-ದರೂ
ಮನಸ್ಸಿ
ಮನಸ್ಸಿ-ನೊ-ಳಗೆ
ಮನಸ್ಸಿ-ಗಿದೆ
ಮನಸ್ಸಿಗೂ
ಮನಸ್ಸಿಗೆ
ಮನಸ್ಸಿ-ಗೊಂದು
ಮನಸ್ಸಿದೆ
ಮನಸ್ಸಿದ್ದರೆ
ಮನಸ್ಸಿನ
ಮನಸ್ಸಿ-ನ-ಮೇಲೆ
ಮನಸ್ಸಿ-ನಲ್ಲಿ
ಮನಸ್ಸಿ-ನಲ್ಲಿಯೇ
ಮನಸ್ಸಿ-ನಲ್ಲಿ-ರುತ್ತವೆ
ಮನಸ್ಸಿ-ನಲ್ಲಿ-ರುವ
ಮನಸ್ಸಿ-ನಲ್ಲೂ
ಮನಸ್ಸಿ-ನಲ್ಲೇ
ಮನಸ್ಸಿ-ನ-ವ-ನೆಂದು
ಮನಸ್ಸಿ-ನಾ-ಳಕ್ಕೆ
ಮನಸ್ಸಿ-ನಿಂದ
ಮನಸ್ಸಿ-ನಿಂದಲೇ
ಮನಸ್ಸಿ-ನೊ-ಳಗೆ
ಮನಸ್ಸಿನ್ನೂ
ಮನಸ್ಸಿಲ್ಲದ
ಮನಸ್ಸು
ಮನಸ್ಸು-ಇ-ವು-ಗ-ಳನ್ನು
ಮನಸ್ಸು-ಗಳ
ಮನಸ್ಸು-ಗ-ಳನ್ನು
ಮನಸ್ಸು-ಗ-ಳಲ್ಲಿ
ಮನಸ್ಸು-ಗ-ಳಿಂದ
ಮನಸ್ಸು-ಗಳು
ಮನಸ್ಸು-ದೇ-ಹೇಂದ್ರಿ-ಯ-ಗಳು
ಮನಸ್ಸೆಂದರೆ
ಮನಸ್ಸೆಂಬ
ಮನಸ್ಸೇ
ಮನಸ್ಸೇನು
ಮನೀ-ಷಿ-ಗ-ಳಿದ್ದರು
ಮನೀ-ಷಿ-ಗಳು
ಮನು-ಕು-ಲ-ವನ್ನು
ಮನು-ಕು-ಲಕ್ಕೆ
ಮನು-ಕು-ಲದ
ಮನು-ಕು-ಲವೇ
ಮನುಜ
ಮನುಜತೆ
ಮನುಜರು
ಮನುವಿಗೆ
ಮನುವು
ಮನುಷ್ಯ
ಮನುಷ್ಯ-ಕುಲ
ಮನುಷ್ಯ-ಕು-ಲಕ್ಕೆ
ಮನುಷ್ಯ-ಕು-ಲದ
ಮನುಷ್ಯ-ಕು-ಲ-ವನ್ನೇ
ಮನುಷ್ಯ-ಕೃತ
ಮನುಷ್ಯ-ಚಾ-ರಿತ್ರ್ಯ
ಮನುಷ್ಯತ್ವದ
ಮನುಷ್ಯತ್ವವು
ಮನುಷ್ಯ-ದೇ-ಹ-ದಿಂದ
ಮನುಷ್ಯನ
ಮನುಷ್ಯ-ನನ್ನು
ಮನುಷ್ಯ-ನನ್ನೇ
ಮನುಷ್ಯ-ನಲ್ಲ-ಡ-ಗಿ-ರುವ
ಮನುಷ್ಯ-ನಲ್ಲಿ
ಮನುಷ್ಯ-ನಲ್ಲಿದೆ
ಮನುಷ್ಯ-ನಲ್ಲೂ
ಮನುಷ್ಯ-ನಾ-ಗಲೀ
ಮನುಷ್ಯ-ನಾಗಿ
ಮನುಷ್ಯ-ನಿಗೂ
ಮನುಷ್ಯ-ನಿಗೆ
ಮನುಷ್ಯನು
ಮನುಷ್ಯನೂ
ಮನುಷ್ಯನೇ
ಮನುಷ್ಯ-ನೊಬ್ಬ
ಮನುಷ್ಯ-ಮ-ನುಷ್ಯ-ರೊ-ಳಗೆ
ಮನುಷ್ಯ-ಮ-ನುಷ್ಯ-ರನ್ನು
ಮನುಷ್ಯ-ಮ-ನುಷ್ಯ-ರೊ-ಳ-ಗಿನ
ಮನುಷ್ಯರ
ಮನುಷ್ಯ-ರನ್ನು
ಮನುಷ್ಯ-ರಲ್ಲಿ
ಮನುಷ್ಯ-ರಲ್ಲಿ-ರುವ
ಮನುಷ್ಯ-ರಲ್ಲೂ
ಮನುಷ್ಯ-ರಾ-ಗಲೂ
ಮನುಷ್ಯ-ರಾ-ಗುತ್ತಿದ್ದಾ-ರೆಯೆ
ಮನುಷ್ಯ-ರಿಗೆ
ಮನುಷ್ಯರು
ಮನುಷ್ಯರೂ
ಮನುಷ್ಯ-ರೆಂದರೆ
ಮನುಷ್ಯ-ರೆಂದು
ಮನುಷ್ಯ-ರೆಲ್ಲ-ರಿಗೂ
ಮನುಷ್ಯ-ರೆಲ್ಲರೂ
ಮನುಷ್ಯರೇ
ಮನುಷ್ಯ-ರೊಂದಿಗೆ
ಮನುಷ್ಯ-ರೊ-ಳ-ಗಣ
ಮನುಷ್ಯ-ಶ-ರೀ-ರದ
ಮನುಷ್ಯ-ಶ-ರೀ-ರ-ದಿಂದ
ಮನೆ
ಮನೆ-ಗ-ಳಲ್ಲಿ
ಮನೆ-ಗ-ಳಿಗೂ
ಮನೆ-ಗ-ಳಿಗೆ
ಮನೆಗೂ
ಮನೆಗೆ
ಮನೆ-ಗೆ-ಲಸ
ಮನೆ-ಗೆ-ಲ-ಸಕ್ಕೂ
ಮನೆ-ಗೆ-ಲ-ಸ-ವೆಲ್ಲ
ಮನೆಗೇ
ಮನೆಗೋ
ಮನೆತನ
ಮನೆ-ತ-ನದ
ಮನೆ-ತ-ನ-ವನ್ನು
ಮನೆ-ಮಂದಿ-ಗೆಲ್ಲಾ
ಮನೆ-ಮಂದಿ-ಯಲ್ಲೇ
ಮನೆ-ಮಾ-ಡಲು
ಮನೆ-ಮಾ-ಡಿ-ಕೊಂಡಿದ್ದ
ಮನೆ-ಮಾ-ರು-ಗ-ಳನ್ನು
ಮನೆಯ
ಮನೆ-ಯಂಗ-ಳಕ್ಕೂ
ಮನೆಯನ್ನು
ಮನೆಯನ್ನೂ
ಮನೆಯನ್ನೇ
ಮನೆಯಲ್ಲಿ
ಮನೆ-ಯಲ್ಲಿದ್ದ
ಮನೆ-ಯಲ್ಲಿದ್ದು-ಕೊಂಡೇ
ಮನೆ-ಯಲ್ಲಿಪ್ರೀತಿ
ಮನೆ-ಯಲ್ಲಿ-ರಲು
ಮನೆ-ಯಲ್ಲಿ-ರು-ವ-ವರು
ಮನೆಯಲ್ಲೂ
ಮನೆಯಲ್ಲೇ
ಮನೆ-ಯ-ವ-ರಿಗೆ
ಮನೆ-ಯ-ವರು
ಮನೆ-ಯ-ವ-ರೆಲ್ಲರ
ಮನೆ-ಯ-ವ-ರೆಲ್ಲರೂ
ಮನೆ-ಯ-ವ-ರೊ-ಡನೆ
ಮನೆಯಿಂದ
ಮನೆಯೇ
ಮನೆ-ಯೊ-ಡ-ತಿ-ಯಾ-ಗು-ವುದು
ಮನೆ-ಯೊ-ಡೆ-ಯ-ನನ್ನು
ಮನೆ-ಯೊ-ಳ-ಗಣ
ಮನೆ-ಯೊ-ಳಗೆ
ಮನೆ-ವಾರ್ತೆಯ
ಮನೇಲಲ್ವೆ
ಮನೋ
ಮನೋ-ಭಾ-ವ-ನೆ-ಗಿಂತಲೂ
ಮನೋ-ವಿಜ್ಞಾ-ನಾತ್ಮ-ಕವೂ
ಮನೋ-ವಿಜ್ಞಾ-ನದ
ಮನೋ-ವಿಜ್ಞಾನಿ
ಮನೋವೃತ್ತಿ
ಮನೋ-ವೃತ್ತಿಯ
ಮನೋ-ಚಾಂಚಲ್ಯ-ವನ್ನು
ಮನೋ-ದುರ್ಬ-ಲತೆ
ಮನೋ-ದೈ-ಹಿಕ
ಮನೋಧರ್ಮ
ಮನೋ-ಬ-ಲವೇ
ಮನೋಭಾವ
ಮನೋ-ಭಾ-ವ-ವನ್ನಾ-ದರೂ
ಮನೋ-ಭಾ-ವ-ವನ್ನು
ಮನೋ-ಭಾ-ವಕ್ಕೆ
ಮನೋ-ಭಾ-ವ-ಗ-ಳನ್ನು
ಮನೋ-ಭಾ-ವ-ತನ್ಮೂ-ಲಕ
ಮನೋ-ಭಾ-ವದ
ಮನೋ-ಭಾ-ವ-ದ-ವನು
ಮನೋ-ಭಾ-ವ-ದ-ವರು
ಮನೋ-ಭಾ-ವ-ದ-ವ-ರೆಂದು
ಮನೋ-ಭಾ-ವ-ದ-ವ-ರೆನ್ನಿ-ಸಿ-ಕೊಂಡ
ಮನೋ-ಭಾ-ವ-ದಿಂದ
ಮನೋ-ಭಾ-ವನೆ
ಮನೋ-ಭಾ-ವ-ನೆ-ಗಳು
ಮನೋ-ಭಾ-ವ-ನೆ-ಯನ್ನು
ಮನೋ-ಭಾ-ವ-ನೆ-ಯಿಂದ
ಮನೋ-ಭಾ-ವ-ವನ್ನು
ಮನೋ-ಭಾ-ವ-ವಿ-ರುತ್ತದೆ
ಮನೋ-ಭಾ-ವ-ವಿಲ್ಲದ
ಮನೋ-ಭಾ-ವವೇ
ಮನೋ-ಮಂಡ-ಲದ
ಮನೋ-ರಂಜನಾ
ಮನೋ-ರಂಜ-ನೆಯ
ಮನೋ-ರಂಜ-ನೆ-ಯನ್ನು
ಮನೋರೋಗ
ಮನೋ-ರೋ-ಗ-ಗಳು
ಮನೋ-ರೋ-ಗ-ತಜ್ಞ-ರನ್ನೂ
ಮನೋ-ರೋ-ಗ-ತಜ್ಞ-ರಲ್ಲಿಗೆ
ಮನೋ-ರೋ-ಗ-ತಜ್ಞರು
ಮನೋ-ರೋ-ಗ-ತಜ್ಞ-ರು-ಅ-ಲೆಗ್ಸಾಂಡರ್
ಮನೋ-ರೋ-ಗ-ತಜ್ಞರೂ
ಮನೋ-ರೋ-ಗ-ತಜ್ಞೆ-ಯಾದ
ಮನೋ-ರೋ-ಗಿ-ಗ-ಳನ್ನು
ಮನೋಲೀಲೆ
ಮನೋ-ವಿ-ಕಾ-ರ-ಗ-ಳನ್ನು
ಮನೋ-ವಿ-ಕಾ-ರ-ಗ-ಳಿಗೆ
ಮನೋ-ವಿಜ್ಞಾನ
ಮನೋ-ವಿಜ್ಞಾ-ನ-ಇ-ವು-ಗಳ
ಮನೋ-ವಿಜ್ಞಾ-ನದ
ಮನೋ-ವಿಜ್ಞಾನಿ
ಮನೋ-ವಿಜ್ಞಾ-ನಿ-ಗ-ಳನ್ನೂ
ಮನೋ-ವಿಜ್ಞಾ-ನಿ-ಗ-ಳಲ್ಲಿ
ಮನೋ-ವಿಜ್ಞಾ-ನಿ-ಗಳು
ಮನೋ-ವಿಜ್ಞಾ-ನಿ-ಗಳೂ
ಮನೋ-ವಿಜ್ಞಾ-ನಿಯ
ಮನೋ-ವಿಜ್ಞಾ-ನಿ-ಯೊಬ್ಬರು
ಮನೋ-ವಿಶ್ಲೇ-ಷಣ
ಮನೋ-ವಿಶ್ಲೇ-ಷ-ಣೆ-ಗಿಂತಲೂ
ಮನೋವೃತ್ತಿ
ಮನೋ-ವೃತ್ತಿಗೆ
ಮನೋ-ವೃತ್ತಿಯ
ಮನೋ-ವೃತ್ತಿ-ಯನ್ನು
ಮನೋ-ವೃತ್ತಿ-ಯನ್ನೂ
ಮನೋ-ವೃತ್ತಿ-ಯನ್ನೇ
ಮನೋ-ವೃತ್ತಿ-ಯ-ವನೂ
ಮನೋ-ವೃತ್ತಿ-ಯ-ವ-ರಿಗೆ
ಮನೋ-ವೃತ್ತಿ-ಯ-ವ-ರೆಂದು
ಮನೋ-ವೃತ್ತಿಯೇ
ಮನೋ-ವೈಜ್ಞಾ-ನಿಕ
ಮನೋವ್ಯಾಧಿ
ಮನ್
ಮನ್ಮ-ಥ-ರನ್ನಾಗಿ
ಮಮ-ಕಾ-ರ-ಗ-ಳನ್ನು
ಮಮ-ಕಾ-ರ-ಗ-ಳಿಂದ
ಮಮ-ಗ-ಳನ್ನು
ಮಮ-ಗ-ಳನ್ನೂ
ಮಮ-ತಾ-ಮಯಿ
ಮಮತೆ
ಮಮತೆಯ
ಮರ
ಮರಗಟ್ಟಿ
ಮರಗಳ
ಮರ-ಗ-ಳನ್ನು
ಮರ-ಗ-ಳೇ-ಳ-ದಿರೆ
ಮರಣ
ಮರ-ಣ-ದೊಂದಿಗೆ
ಮರಣಂ
ಮರ-ಣ-ಕಾ-ಲ-ದಲ್ಲಿ
ಮರಣಕ್ಕೂ
ಮರಣಕ್ಕೆ
ಮರಣದ
ಮರ-ಣ-ದಂಡ-ನೆಯ
ಮರ-ಣ-ದಿಂದುಂಟಾ-ಗುವ
ಮರ-ಣ-ದೊಂದಿಗೆ
ಮರ-ಣಪ್ರಾ-ಯ-ವಾದ
ಮರ-ಣ-ವನ್ನು
ಮರ-ಣ-ವನ್ನೂ
ಮರ-ಣ-ಶಯ್ಯೆ-ಯಲ್ಲಿದ್ದಾಗ
ಮರ-ಣ-ಹೊಂದು-ವಂತೆ
ಮರ-ಣಾಂತಿಕ
ಮರ-ಣಾಂತಿ-ಕ-ವಾದ
ಮರ-ಣಾ-ತೀತ
ಮರ-ಣಾ-ನಂತ-ರದ
ಮರದ
ಮರದಂತೆ
ಮರ-ದ-ಡಿಯ
ಮರ-ದ-ಡಿ-ಯಲ್ಲಿ
ಮರದಲ್ಲಿ
ಮರದಿಂದ
ಮರಳಿ
ಮರಳಿದ
ಮರ-ಳಿ-ನಲ್ಲಿ
ಮರಳು
ಮರ-ಳುತ್ತದೆ
ಮರ-ಳುತ್ತಾನೆ
ಮರಳುವ
ಮರವನ್ನು
ಮರ-ವನ್ನೇರಿ
ಮರವು
ಮರವೆ
ಮರ-ವೆ-ಯಾ-ದರೂ
ಮರವೇ
ಮರಿ
ಮರಿಯನ್ನು
ಮರೀಚಿಕೆ
ಮರೀ-ಚಿ-ಕೆ-ಯನ್ನು
ಮರು
ಮರುಕ
ಮರುಕದ
ಮರು-ಕ-ದಿಂದ
ಮರು-ಕ-ಳಿ-ಸ-ಬ-ಹು-ದೆಂಬ
ಮರು-ಕ-ಳಿ-ಸಿ-ದರೂ
ಮರು-ಕ-ಳಿ-ಸುವ
ಮರು-ಕ-ವನ್ನು
ಮರುಕವೇ
ಮರುಕ್ಷ-ಣವೇ
ಮರುಗಿ
ಮರು-ಗುತ್ತಿದ್ದಾರೆ
ಮರು-ಗುತ್ತೇನೆ
ಮರುಗುವ
ಮರುದನಿ
ಮರುದಿನ
ಮರು-ಪಾ-ವ-ತಿ-ಯಾ-ಗದೇ
ಮರು-ಭೂ-ಮಿಯ
ಮರು-ಭೂ-ಮಿ-ಯನ್ನು
ಮರು-ಭೂ-ಮಿ-ಯಲ್ಲಿ
ಮರು-ಳಾ-ಗ-ಬೇಡಿ
ಮರೆ
ಮರೆತ
ಮರೆತರು
ಮರೆತರೆ
ಮರೆತರೇ
ಮರೆ-ತ-ಳು-ವು-ದೇಕೆ
ಮರೆತವು
ಮರೆ-ತಿದ್ದಾನೆ
ಮರೆತಿದ್ದ
ಮರೆ-ತಿದ್ದೇವೆ
ಮರೆ-ತಿ-ರ-ಬ-ಹು-ದಾದ
ಮರೆ-ತಿ-ರ-ಬ-ಹುದು
ಮರೆ-ತಿ-ರು-ವಿರಾ
ಮರೆತು
ಮರೆತುದು
ಮರೆತುದೇ
ಮರೆ-ತು-ಬಿಟ್ಟರೂ
ಮರೆ-ತು-ಬಿಟ್ಟೆ-ನಲ್ಲ
ಮರೆ-ತು-ಬಿಡು
ಮರೆ-ತು-ಹೋ-ಗು-ವುದೋ
ಮರೆ-ತು-ಹೋದ
ಮರೆ-ಮಾ-ಚಲು
ಮರೆಮಾಡಿ
ಮರೆ-ಯ-ಬೇಡ
ಮರೆ-ಯ-ತಕ್ಕ
ಮರೆ-ಯ-ತಕ್ಕದ್ದಲ್ಲ
ಮರೆ-ಯ-ತೊ-ಡ-ಗಿ-ರು-ವುದು
ಮರೆ-ಯ-ದಿ-ರ-ಬೇಕು
ಮರೆ-ಯ-ದಿರಿ
ಮರೆ-ಯ-ದಿ-ರೋಣ
ಮರೆಯದೆ
ಮರೆ-ಯ-ಬಾ-ರದು
ಮರೆ-ಯ-ಬೇ-ಕಾ-ಗಿದೆ
ಮರೆ-ಯ-ಬೇಡ
ಮರೆ-ಯ-ಬೇಡಿ
ಮರೆ-ಯ-ಬೇ-ಡಿ-ಟೀ-ಕಾ-ಸು-ರ-ರನ್ನು
ಮರೆ-ಯ-ಲಾ-ಗದ
ಮರೆ-ಯ-ಲಾ-ಗದು
ಮರೆ-ಯ-ಲಾರೆ
ಮರೆ-ಯ-ಲಿಲ್ಲ
ಮರೆಯಲು
ಮರೆ-ಯ-ಲೆತ್ನಿ-ಸಿ-ದಂತಲ್ಲವೆ
ಮರೆಯಲ್ಲಿ
ಮರೆ-ಯಾ-ಗಿ-ಸಿದೆ
ಮರೆ-ಯಾ-ದಂತೆ
ಮರೆ-ಯಿ-ಸಿದೆ
ಮರೆಯು
ಮರೆ-ಯು-ವಂತಿಲ್ಲ
ಮರೆಯುತ್ತ
ಮರೆ-ಯುತ್ತಾನೆ
ಮರೆ-ಯುತ್ತಾರೆ
ಮರೆ-ಯುತ್ತಿದ್ದಾನೆ
ಮರೆ-ಯುತ್ತೀರಾ
ಮರೆ-ಯು-ವಂತಿಲ್ಲ
ಮರೆ-ಯು-ವು-ದಿಲ್ಲ
ಮರೆ-ಯು-ವುದು
ಮರೆವು
ಮರೆವೆಯೂ
ಮರೆ-ಸಿ-ಕೊಂಡು
ಮರ್ತ್ಯನಲ್ಲ
ಮರ್ತ್ಯನೆಂದೂ
ಮರ್ಮ
ಮರ್ಮವನ್ನು
ಮರ್ಮವಿದು
ಮರ್ಮಾ-ಥ-ವನ್ನೂ
ಮರ್ಯಾದೆ
ಮರ್ಯಾ-ದೆ-ಯಿಂದಲೇ
ಮಲಗಿ
ಮಲ-ಗಿ-ಕೊಂಡ
ಮಲ-ಗಿ-ಕೊಂಡಂತಿ-ರು-ವು-ದನ್ನು
ಮಲ-ಗಿ-ಕೊಂಡಿದ್ದ
ಮಲ-ಗಿ-ಕೊಂಡಿದ್ದಾಗ
ಮಲ-ಗಿ-ಕೊಂಡಿ-ರು-ವೆ-ನೆಂದೂ
ಮಲ-ಗಿ-ಕೊಂಡೇ
ಮಲಗಿದ
ಮಲ-ಗಿ-ದಾಗ
ಮಲ-ಗಿ-ದೊ-ಡ-ನೆಯೇ
ಮಲಗಿದ್ದ
ಮಲ-ಗಿದ್ದರು
ಮಲ-ಗಿದ್ದರೂ
ಮಲ-ಗಿದ್ದಲ್ಲಿಂದ
ಮಲ-ಗಿದ್ದ-ವನು
ಮಲ-ಗಿದ್ದಾರೆ
ಮಲ-ಗಿದ್ದು-ದನ್ನೂ
ಮಲ-ಗಿದ್ದೇನೆ
ಮಲ-ಗಿ-ರುವ
ಮಲ-ಗಿ-ಸಿ-ಕೊಂಡರೆ
ಮಲ-ಗಿ-ಸಿ-ಕೊಳ್ಳುತ್ತಿದ್ದ
ಮಲ-ಗಿ-ಸಿ-ಕೊಳ್ಳು-ವಂತೆ
ಮಲ-ಗಿ-ಸಿದ್ದರು
ಮಲಗುವ
ಮಲತಾಯಿ
ಮಲವನ್ನು
ಮಲ-ವಿ-ಸರ್ಜನೆ
ಮಲ-ವಿ-ಸರ್ಜ-ನೆ-ಯನ್ನು
ಮಲಿ-ನ-ವಾ-ಗ-ಲಿಲ್ಲ
ಮಲಿ-ನ-ವಾ-ಗಿದ್ದರೆ
ಮಲಿ-ನ-ವಾ-ಗಿಲ್ಲ
ಮಲೆಯಾಳಂ
ಮಳೆ
ಮಳೆ-ಗ-ರೆದು
ಮಳೆ-ಗ-ರೆ-ಯುತ್ತ
ಮಳೆ-ಗ-ರೆ-ಯು-ವ-ವರೇ
ಮಳೆಗಾಲ
ಮಳೆಗಾಳಿ
ಮಳೆಯ
ಮಳೆಯನ್ನು
ಮಳೆಯಲ್ಲಿ
ಮಳೆಯಾಗಿ
ಮಳೆಯು
ಮಳೆಯೂ
ಮಸಿ
ಮಸಿ-ಕು-ಡಿಕೆ
ಮಸಿಯಂತೆ
ಮಸೀದಿ
ಮಸೀ-ದಿ-ಯನ್ನೂ
ಮಸುಕು
ಮಹಡಿಯ
ಮಹ-ಡಿ-ಯಿಂದ
ಮಹ-ಡಿ-ಯೊಂದ-ರಲ್ಲಿ
ಮಹತ್
ಮಹತ್ಕಾರ್ಯ
ಮಹತ್ಕಾರ್ಯ-ಗಳು
ಮಹತ್ಕಾರ್ಯದ
ಮಹತ್ಕಾರ್ಯ-ವನ್ನು
ಮಹತ್ಕಾರ್ಯವೂ
ಮಹತ್ತನ್ನು
ಮಹತ್ತ-ರ-ವಾ-ಗಿದೆ
ಮಹತ್ತಾದ
ಮಹತ್ತಾ-ದು-ದನ್ನು
ಮಹತ್ತಾ-ದು-ದೇ-ನನ್ನಾ-ದರೂ
ಮಹತ್ತಿ-ಗಿಂತಲೂ
ಮಹತ್ತ್ವ-ವನ್ನು
ಮಹತ್ತ್ವ-ವಿದೆ
ಮಹತ್ಫಲ
ಮಹತ್ವ
ಮಹತ್ವ-ಇದು
ಮಹತ್ವ-ಗ-ಳನ್ನು
ಮಹತ್ವದ
ಮಹತ್ವದ್ದು
ಮಹತ್ವದ್ದು-ಅ-ದನ್ನು
ಮಹತ್ವ-ವನ್ನು
ಮಹತ್ವ-ವನ್ನೂ
ಮಹತ್ವಾ-ಕಾಂಕ್ಷೆ
ಮಹತ್ವಾ-ಕಾಂಕ್ಷೆ-ಗ-ಳಿಲ್ಲ-ನಿರ್ದಿಷ್ಟ
ಮಹತ್ವಾ-ಕಾಂಕ್ಷೆ-ಯನ್ನು
ಮಹತ್ವಿ-ಕೆ-ಯನ್ನೇ-ರಿ-ದರು
ಮಹತ್ವಿ-ಕೆ-ಗೇ-ರಿ-ಸಲು
ಮಹತ್ವಿ-ಕೆಯ
ಮಹ-ನೀ-ಯರ
ಮಹ-ನೀ-ಯರು
ಮಹ-ನೀ-ಯ-ರು-ಗಳ
ಮಹ-ನೀ-ಯರೂ
ಮಹ-ನೀ-ಯ-ರೊಬ್ಬರು
ಮಹಮ್ಮ-ದೀಯ
ಮಹರ್ಷಿ-ಗ-ಳನ್ನು
ಮಹರ್ಷಿ-ಗ-ಳಾ-ಗಲಿ
ಮಹರ್ಷಿ-ಗಳು
ಮಹಾ
ಮಹಾ-ನಾ-ಗ-ರಿ-ಕ-ತೆ-ಗಳು
ಮಹಾ-ಭಾ-ರತ
ಮಹಾ-ಯುದ್ಧ-ದ-ವ-ರೆ-ಗಿನ
ಮಹಾ-ಸಂದೇ-ಶದ
ಮಹಾ-ಕಾರ್ಯ-ವನ್ನು
ಮಹಾಕಾವ್ಯ
ಮಹಾ-ಕಾವ್ಯ-ಗ-ಳನ್ನೂ
ಮಹಾ-ಕಾವ್ಯ-ವನ್ನು
ಮಹಾ-ಗ-ಣಿ-ತಜ್ಞ
ಮಹಾ-ಗ-ಣಿ-ತಜ್ಞ-ರಾದ
ಮಹಾಗುಣ
ಮಹಾ-ಗು-ಣ-ದೊಂದಿಗೆ
ಮಹಾ-ಘ-ಟನೆ
ಮಹಾಘನ
ಮಹಾ-ಘ-ನದ
ಮಹಾ-ಘ-ನ-ವ-ರಿ-ಯರು
ಮಹಾತತ್ತ್ವ
ಮಹಾ-ತತ್ತ್ವ-ವೇತ್ತ-ನೆಂದು
ಮಹಾತ್ಮ
ಮಹಾತ್ಮನ
ಮಹಾತ್ಮ-ನನ್ನಾ-ಗಿ-ಸಿತು
ಮಹಾತ್ಮ-ನನ್ನು
ಮಹಾತ್ಮನು
ಮಹಾತ್ಮನೂ
ಮಹಾತ್ಮನೋ
ಮಹಾತ್ಮರ
ಮಹಾತ್ಮ-ರಾ-ಗಲು
ಮಹಾತ್ಮ-ರಾರೂ
ಮಹಾತ್ಮ-ರಿಂದ
ಮಹಾತ್ಮರು
ಮಹಾತ್ಮರೂ
ಮಹಾತ್ಮ-ರೆಂದರೆ
ಮಹಾತ್ಮ-ರೆನ್ನಿ-ಸಿ-ಕೊಂಡ-ವರೂ
ಮಹಾತ್ಮ-ರೆನ್ನಿ-ಸಿ-ಕೊಳ್ಳಲು
ಮಹಾತ್ಮ-ರೊಬ್ಬರ
ಮಹಾತ್ಮಾ
ಮಹಾತ್ಮಾ-ಗಾಂಧೀಜಿ
ಮಹಾತ್ಮಾ-ಗಾಂಧೀಜಿ
ಮಹಾತ್ಮ್ಯೆ-ಗ-ಳಿಗೆ
ಮಹಾತ್ಯಾಗ
ಮಹಾದೇವ
ಮಹಾದೋಷ
ಮಹಾ-ದೋ-ಷ-ವೆಂದು
ಮಹಾದ್ಭುತ
ಮಹಾಧರ್ಮ
ಮಹಾ-ಧರ್ಮ-ಗ-ಳಿಗೆ
ಮಹಾ-ಧರ್ಮ-ಗಳು
ಮಹಾ-ಧರ್ಮ-ಗಳೇ
ಮಹಾ-ನ-ಗರ
ಮಹಾ-ನಾ-ಗ-ರಿ-ಕ-ತೆಯ
ಮಹಾ-ನಾ-ಗ-ರಿ-ಕ-ತೆಯೇ
ಮಹಾ-ನಾ-ಯಕ
ಮಹಾ-ನು-ಭಾ-ವನ
ಮಹಾ-ನು-ಭಾ-ವ-ನೆನ್ನು-ವಂತೆ
ಮಹಾ-ನು-ಭಾ-ವರ
ಮಹಾ-ನು-ಭಾ-ವರು
ಮಹಾ-ನು-ಭಾ-ವರೂ
ಮಹಾನ್
ಮಹಾಪಾಠ
ಮಹಾ-ಪಾ-ಪಿ-ಗ-ಳೆನ್ನಿ-ಸಿ-ಕೊಂಡ-ವ-ರಿಗೂ
ಮಹಾ-ಪು-ರುಷ
ಮಹಾ-ಪು-ರು-ಷರ
ಮಹಾ-ಪು-ರು-ಷ-ರಲ್ಲಿ
ಮಹಾ-ಪು-ರು-ಷ-ರಿಗೇ
ಮಹಾ-ಪು-ರು-ಷರು
ಮಹಾ-ಪೂ-ರಕ್ಕೆ
ಮಹಾಪ್ರ-ತಿಭೆ
ಮಹಾಪ್ರ-ಬಂಧ-ವನ್ನು
ಮಹಾಪ್ರಾಧ್ಯಾ-ಪ-ಕ-ನಾಗಿ
ಮಹಾಪ್ರಾಧ್ಯಾ-ಪ-ಕ-ರಾ-ದರು
ಮಹಾಪ್ರೇ-ರ-ಣೆ-ಯನ್ನು
ಮಹಾ-ಬ-ಲವು
ಮಹಾಬುದ್ಧಿ
ಮಹಾ-ಭಾ-ರತ
ಮಹಾ-ಭಾ-ರ-ತದ
ಮಹಾ-ಭಾ-ರ-ತ-ದಲ್ಲಿ
ಮಹಾ-ಭಾ-ವ-ನೆ-ಗಳ
ಮಹಾ-ಮ-ಡಿ-ವಂತರು
ಮಹಾಮುನಿ
ಮಹಾ-ಮೇ-ಧಾ-ವಿ-ಗ-ಳೆಂದಾ-ಗಲೀ
ಮಹಾ-ಮೇ-ಧಾ-ವಿಯೂ
ಮಹಾ-ಯುದ್ಧ-ಗ-ಳಿಂದ
ಮಹಾ-ಯುದ್ಧದ
ಮಹಾ-ಯುದ್ಧ-ದಲ್ಲಿ
ಮಹಾ-ರಾ-ಜರ
ಮಹಾ-ರಾ-ಜ-ರನ್ನು
ಮಹಾ-ರಾ-ಜ-ರಿಗೆ
ಮಹಾ-ರಾ-ಜರು
ಮಹಾರಾಜ್
ಮಹಾರಾಯ
ಮಹಾ-ರಾಷ್ಟ್ರ-ಗ-ಳಿಗೆ
ಮಹಾ-ರಾಷ್ಟ್ರದ
ಮಹಾ-ರಾಷ್ಟ್ರ-ವಾ-ಗಲು
ಮಹಾವಾಕ್ಯ
ಮಹಾ-ವಾಕ್ಯ-ಗಳು
ಮಹಾ-ವಾಕ್ಯ-ವನ್ನು
ಮಹಾವಾಗ್ಮಿ
ಮಹಾ-ವಿಜ್ಞಾನಿ
ಮಹಾ-ವಿಜ್ಞಾ-ನಿ-ಗಳು
ಮಹಾ-ವಿದ್ವಾಂಸರ
ಮಹಾ-ವಿದ್ವಾಂಸ-ರಾದ
ಮಹಾವೀರ
ಮಹಾ-ವೀ-ರ-ನಾದ
ಮಹಾವೈರಿ
ಮಹಾವ್ಯಕ್ತಿ-ಗಳೇ
ಮಹಾಶಕ್ತಿ
ಮಹಾ-ಶಕ್ತಿ-ಯನ್ನು
ಮಹಾಶತ್ರು
ಮಹಾಶಯ
ಮಹಾ-ಶ-ಯನು
ಮಹಾ-ಶ-ಯರು
ಮಹಾ-ಶ-ಯರೆ
ಮಹಾ-ಶ-ಯರೇ
ಮಹಾ-ಶೂ-ರ-ನೆ-ನಿ-ಸಿದ್ದರೂ
ಮಹಾ-ಸಂಘದ
ಮಹಾ-ಸಂದೇ-ಶ-ವನ್ನು
ಮಹಾ-ಸಂಶೋ-ಧನೆ
ಮಹಾಸತ್ಯ
ಮಹಾ-ಸತ್ಯದ
ಮಹಾ-ಸತ್ಯ-ವನ್ನು
ಮಹಾಸಭೆ
ಮಹಾ-ಸರ್ಜನ್
ಮಹಾ-ಸಾ-ಧು-ವೊಬ್ಬ
ಮಹಾ-ಸುಳ್ಳು-ಗಾ-ರ-ರೆಂಬುದು
ಮಹಿಮ
ಮಹಿಮರ
ಮಹಿ-ಮಾ-ತಿ-ಶ-ಯ-ಗ-ಳನ್ನು
ಮಹಿ-ಮಾ-ತಿ-ಶ-ಯದ
ಮಹಿ-ಮಾ-ತಿ-ಶ-ಯ-ವನ್ನು
ಮಹಿ-ಮಾನ್ವಿ-ತ-ವಾ-ದು-ದೆಂದು
ಮಹಿ-ಮಾ-ಮಯ
ಮಹಿಮೆ
ಮಹಿ-ಮೆ-ಗ-ಳನ್ನು
ಮಹಿಮೆಗೆ
ಮಹಿಮೆಯ
ಮಹಿ-ಮೆ-ಯನ್ನ-ರಿ-ಯದೇ
ಮಹಿ-ಮೆ-ಯನ್ನಲ್ಲ
ಮಹಿ-ಮೆ-ಯನ್ನು
ಮಹಿ-ಮೆ-ಯನ್ನೂ
ಮಹಿ-ಮೆ-ಯೆ-ಡೆಗೆ
ಮಹಿರ್ಷಿ
ಮಹಿಳಾ
ಮಹಿಳೆ
ಮಹಿಳೆಗೆ
ಮಹಿಳೆಗೇ
ಮಹಿಳೆಯ
ಮಹಿ-ಳೆ-ಯನ್ನು
ಮಹಿ-ಳೆ-ಯರು
ಮಹಿ-ಳೆ-ಯೊಬ್ಬರು
ಮಹಿ-ಳೆ-ಯೊಬ್ಬಳ
ಮಹಿ-ಳೆ-ಯೊಬ್ಬಳು
ಮಹೋ-ದ-ಯರು
ಮಹೋದ್ದೇ-ಶ-ವಾ-ಗಿ-ರುವ
ಮಹೋನ್ನತ
ಮಾ
ಮಾಂತ್ರಿಕ
ಮಾಂತ್ರಿಕತೆ
ಮಾಂತ್ರಿ-ಕ-ನಾ-ಗಿ-ರ-ಲಿಲ್ಲ
ಮಾಂಸ
ಮಾಂಸಖಂಡ
ಮಾಂಸ-ಖಂಡ-ಗಳು
ಮಾಟಗಳ
ಮಾಡ
ಮಾಡ-ಬ-ಹು-ದೆಂಬ
ಮಾಡ-ಬೇ-ಕಾ-ಗು-ವುದೊ
ಮಾಡ-ಬೇ-ಕಾ-ದ-ವರು
ಮಾಡ-ಲಾ-ಗ-ದಿದ್ದರೆ
ಮಾಡ-ಲೆ-ಳ-ಸುವ
ಮಾಡ-ಕೂ-ಡದು
ಮಾಡ-ತೊ-ಡ-ಗಿದೆ
ಮಾಡ-ತೊ-ಡ-ಗುತ್ತಾನೆ
ಮಾಡದ
ಮಾಡ-ದಂತಾ-ಗಲಿ
ಮಾಡದಂತೆ
ಮಾಡ-ದ-ವರು
ಮಾಡ-ದ-ವ-ರೆಗೆ
ಮಾಡ-ದಿದ್ದರೆ
ಮಾಡ-ದಿದ್ದಾಗ
ಮಾಡ-ದಿ-ರ-ಲಾರ
ಮಾಡ-ದಿ-ರ-ಲಾ-ರರು
ಮಾಡ-ದಿ-ರಲು
ಮಾಡ-ದಿ-ರು-ವುದು
ಮಾಡದು
ಮಾಡದೆ
ಮಾಡದೇ
ಮಾಡನ್ನು
ಮಾಡಬಲ್ಲ
ಮಾಡ-ಬಲ್ಲಂಥ-ವರು
ಮಾಡ-ಬಲ್ಲದು
ಮಾಡ-ಬಲ್ಲದೋ
ಮಾಡ-ಬಲ್ಲರು
ಮಾಡ-ಬಲ್ಲಿರಾ
ಮಾಡ-ಬಲ್ಲಿರಿ
ಮಾಡ-ಬಲ್ಲುದು
ಮಾಡ-ಬಲ್ಲೆ-ಯಾ-ದರೆ
ಮಾಡ-ಬ-ಹು-ದಾದ
ಮಾಡ-ಬ-ಹು-ದಾ-ದರೂ
ಮಾಡ-ಬ-ಹು-ದಿತ್ತೇ
ಮಾಡ-ಬ-ಹುದು
ಮಾಡ-ಬ-ಹು-ದು-ವೃತ್ತಾ-ಕಾರ
ಮಾಡ-ಬ-ಹು-ದೆಂದು
ಮಾಡ-ಬ-ಹು-ದೆಂಬು-ದನ್ನು
ಮಾಡಬಾರ
ಮಾಡ-ಬಾ-ರದ
ಮಾಡ-ಬಾ-ರದು
ಮಾಡ-ಬಾ-ರ-ದೆಂದಲ್ಲ
ಮಾಡ-ಬೇ-ಕಲ್ಲ
ಮಾಡ-ಬೇ-ಕಲ್ಲವೇ
ಮಾಡ-ಬೇ-ಕಲ್ಲಾ
ಮಾಡ-ಬೇ-ಕಾಗಿ
ಮಾಡ-ಬೇ-ಕಾ-ಗಿದೆ
ಮಾಡ-ಬೇ-ಕಾ-ಗಿದ್ದ
ಮಾಡ-ಬೇ-ಕಾ-ಗಿ-ರ-ಲಿಲ್ಲ
ಮಾಡ-ಬೇ-ಕಾ-ಗಿ-ರುವ
ಮಾಡ-ಬೇ-ಕಾದ
ಮಾಡ-ಬೇ-ಕಾ-ದರೆ
ಮಾಡ-ಬೇ-ಕಾ-ದ-ವರು
ಮಾಡ-ಬೇ-ಕಾ-ದಾಗ
ಮಾಡ-ಬೇ-ಕಾ-ದು-ದನ್ನು
ಮಾಡ-ಬೇ-ಕಾ-ಯಿತು
ಮಾಡ-ಬೇ-ಕಿದ್ದ
ಮಾಡ-ಬೇ-ಕಿಲ್ಲ
ಮಾಡಬೇಕು
ಮಾಡಬೇಕೆ
ಮಾಡ-ಬೇ-ಕೆಂದಾಗ
ಮಾಡ-ಬೇ-ಕೆಂದು
ಮಾಡ-ಬೇ-ಕೆಂದೇ
ಮಾಡ-ಬೇ-ಕೆಂಬ
ಮಾಡ-ಬೇ-ಕೆಂಬು-ದನ್ನು
ಮಾಡ-ಬೇ-ಕೆಂಬುದು
ಮಾಡಬೇಡ
ಮಾಡಬೇಡಿ
ಮಾಡ-ಬೇ-ಡಿರಿ
ಮಾಡಯ್ಯ
ಮಾಡರು
ಮಾಡ-ಲ-ಸಾಧ್ಯ-ವಾ-ದು-ದನ್ನು
ಮಾಡ-ಲಾ-ಗ-ದಿದ್ದರೆ
ಮಾಡ-ಲಾ-ಗಿದೆ
ಮಾಡ-ಲಾ-ಗು-ವು-ದಿಲ್ಲ
ಮಾಡ-ಲಾ-ಯಿತು
ಮಾಡಲಾರ
ಮಾಡ-ಲಾ-ರಂಭಿ-ಸುತ್ತದೆ
ಮಾಡ-ಲಾ-ರ-ದಾ-ದಳು
ಮಾಡ-ಲಾ-ರನೇ
ಮಾಡ-ಲಾ-ರರು
ಮಾಡಲಾರೆ
ಮಾಡ-ಲಾ-ರೆವು
ಮಾಡಲಿ
ಮಾಡ-ಲಿಚ್ಛಿ-ಸು-ವ-ವನು
ಮಾಡಲಿಲ್ಲ
ಮಾಡಲು
ಮಾಡಲೂ
ಮಾಡಲೇ
ಮಾಡ-ಲೇ-ಬೇ-ಕಾದ
ಮಾಡ-ಲೇ-ಬೇ-ಕಾ-ದಂಥ
ಮಾಡ-ಲೇ-ಬೇ-ಕೆಂದು
ಮಾಡಲ್ಪಟ್ಟ
ಮಾಡ-ಹೊ-ರ-ಟ-ವ-ನೆ-ನಿರ್ಭೀ-ತಿಯ
ಮಾಡ-ಹೊ-ರ-ಟ-ವ-ರಿಗೆ
ಮಾಡ-ಹೊ-ರಟು
ಮಾಡ-ಹೊ-ರ-ಡು-ವುದು
ಮಾಡ-ಹೋ-ಗ-ಬೇಡಿ
ಮಾಡಿ
ಮಾಡಿ-ಕೊಂಡಂತಾ-ಗು-ವುದು
ಮಾಡಿ-ಕೊಂಡ-ವ-ರಿದ್ದಾರೆ
ಮಾಡಿದ್ದಳು
ಮಾಡಿಅದು
ಮಾಡಿಕೊಂಡ
ಮಾಡಿ-ಕೊಂಡಂತೆ
ಮಾಡಿ-ಕೊಂಡದ್ದು
ಮಾಡಿ-ಕೊಂಡ-ನಂತೆ
ಮಾಡಿ-ಕೊಂಡರು
ಮಾಡಿ-ಕೊಂಡರೂ
ಮಾಡಿ-ಕೊಂಡರೆ
ಮಾಡಿ-ಕೊಂಡಿದ್ದ-ನೆಂದು
ಮಾಡಿ-ಕೊಂಡಿದ್ದರು
ಮಾಡಿ-ಕೊಂಡಿದ್ದಾರೆ
ಮಾಡಿ-ಕೊಂಡಿದ್ದೀರಿ
ಮಾಡಿ-ಕೊಂಡಿದ್ದೇ-ವೆನ್ನುವ
ಮಾಡಿ-ಕೊಂಡಿ-ರುತ್ತದೆ
ಮಾಡಿ-ಕೊಂಡಿ-ರುತ್ತೇವೆ
ಮಾಡಿ-ಕೊಂಡಿಲ್ಲ
ಮಾಡಿಕೊಂಡು
ಮಾಡಿಕೊಟ್ಟ
ಮಾಡಿ-ಕೊಟ್ಟಿತು
ಮಾಡಿ-ಕೊ-ಡಲು
ಮಾಡಿ-ಕೊ-ಡುತ್ತದೆ
ಮಾಡಿ-ಕೊ-ಡುತ್ತವೆ
ಮಾಡಿ-ಕೊ-ಡುವ
ಮಾಡಿ-ಕೊ-ಡೆಂದು
ಮಾಡಿ-ಕೊಳ್ಳ-ಬಲ್ಲರು
ಮಾಡಿ-ಕೊಳ್ಳ-ಬ-ಹುದು
ಮಾಡಿ-ಕೊಳ್ಳ-ಬೇಕು
ಮಾಡಿ-ಕೊಳ್ಳಲಿ
ಮಾಡಿ-ಕೊಳ್ಳಲು
ಮಾಡಿಕೊಳ್ಳಿ
ಮಾಡಿ-ಕೊಳ್ಳುತ್ತದೆ
ಮಾಡಿ-ಕೊಳ್ಳುತ್ತಾನೆ
ಮಾಡಿ-ಕೊಳ್ಳುತ್ತಾರೆ
ಮಾಡಿ-ಕೊಳ್ಳುತ್ತಿ
ಮಾಡಿ-ಕೊಳ್ಳುತ್ತಿತ್ತು
ಮಾಡಿ-ಕೊಳ್ಳುತ್ತಿದ್ದ
ಮಾಡಿ-ಕೊಳ್ಳುತ್ತಿದ್ದ-ನೆಂದೂ
ಮಾಡಿ-ಕೊಳ್ಳುತ್ತಿದ್ದರು
ಮಾಡಿ-ಕೊಳ್ಳುತ್ತಿದ್ದ-ವೆಂದು
ಮಾಡಿ-ಕೊಳ್ಳುತ್ತಿಲ್ಲ
ಮಾಡಿ-ಕೊಳ್ಳುತ್ತೇನೆ
ಮಾಡಿ-ಕೊಳ್ಳುತ್ತೇವೆ
ಮಾಡಿ-ಕೊಳ್ಳುವ
ಮಾಡಿ-ಕೊಳ್ಳು-ವಂತೆ
ಮಾಡಿ-ಕೊಳ್ಳು-ವಿರಿ
ಮಾಡಿ-ಕೊಳ್ಳು-ವು-ದುಂಟು
ಮಾಡಿ-ಕೊಳ್ಳು-ವುದೇ
ಮಾಡಿತಷ್ಟೇ
ಮಾಡಿತು
ಮಾಡಿತ್ತು
ಮಾಡಿತ್ತೋ
ಮಾಡಿದ
ಮಾಡಿ-ದಂತಾ-ಗುತ್ತದೆ
ಮಾಡಿದಂಥ
ಮಾಡಿದನೆ
ಮಾಡಿದರು
ಮಾಡಿ-ದ-ರು-ತಾಂತ್ರಿಕ
ಮಾಡಿದರೂ
ಮಾಡಿದರೆ
ಮಾಡಿ-ದ-ರೇನು
ಮಾಡಿದಲ್ಲಿ
ಮಾಡಿದಳು
ಮಾಡಿದಳೇ
ಮಾಡಿ-ದ-ವ-ನಲ್ಲ
ಮಾಡಿ-ದ-ವ-ನಾ-ದರೂ
ಮಾಡಿ-ದ-ವ-ನಿಗೆ
ಮಾಡಿ-ದ-ವನು
ಮಾಡಿ-ದ-ವ-ನೊಬ್ಬ-ನಿಗೆ
ಮಾಡಿ-ದ-ವ-ರನ್ನು
ಮಾಡಿ-ದ-ವ-ರಲ್ಲ
ಮಾಡಿ-ದ-ವರು
ಮಾಡಿ-ದ-ವ-ಳಲ್ಲ
ಮಾಡಿದವು
ಮಾಡಿದಷ್ಟೇ
ಮಾಡಿದಾಗ
ಮಾಡಿದಿ
ಮಾಡಿ-ದಿ-ರಾ-ದರೆ
ಮಾಡಿದಿರಿ
ಮಾಡಿ-ದು-ದ-ರಲ್ಲಿ
ಮಾಡಿ-ದು-ದ-ರಿಂದ
ಮಾಡಿದುದು
ಮಾಡಿದುದೇ
ಮಾಡಿದುವು
ಮಾಡಿದೆ
ಮಾಡಿದೆನೆ
ಮಾಡಿ-ದೆ-ನೆಂದು
ಮಾಡಿ-ದೆ-ನೆಂಬ
ಮಾಡಿದೆಯೆ
ಮಾಡಿದ್ದ
ಮಾಡಿದ್ದಕ್ಕೆ
ಮಾಡಿದ್ದಕ್ಕೆಲ್ಲಾ
ಮಾಡಿದ್ದನ್ನು
ಮಾಡಿದ್ದರು
ಮಾಡಿದ್ದರೆ
ಮಾಡಿದ್ದ-ರೆಂದೇ
ಮಾಡಿದ್ದಳು
ಮಾಡಿದ್ದಾ-ದರೆ
ಮಾಡಿದ್ದಾನೆ
ಮಾಡಿದ್ದಾರೆ
ಮಾಡಿದ್ದಾ-ರೆಂದು
ಮಾಡಿದ್ದಾ-ರೆಂಬುದು
ಮಾಡಿದ್ದಾ-ರೇನೋ
ಮಾಡಿದ್ದಿಲ್ಲ
ಮಾಡಿದ್ದೀ-ಯೆಂದಾ-ಗಲಿ
ಮಾಡಿದ್ದೀರಿ
ಮಾಡಿದ್ದು
ಮಾಡಿದ್ದುಣ್ಣೋ
ಮಾಡಿದ್ದೆ
ಮಾಡಿದ್ದೆಂದು
ಮಾಡಿದ್ದೇವೆ
ಮಾಡಿನ
ಮಾಡಿ-ನ-ಡಿ-ಯಲ್ಲಿ
ಮಾಡಿನಿಂದ
ಮಾಡಿ-ಬಿ-ಡುತ್ತದೆ
ಮಾಡಿ-ಬಿ-ಡು-ವೆನೊ
ಮಾಡಿ-ಬಿ-ಡು-ವೆನೋ
ಮಾಡಿ-ಯಾ-ರೆಂಬು-ದನ್ನು
ಮಾಡಿ-ಯಾ-ರೆಂದು
ಮಾಡಿಯಾವು
ಮಾಡಿ-ಯೇ-ಬಿಟ್ಟಿತು
ಮಾಡಿ-ರ-ದಂಥ
ಮಾಡಿ-ರ-ದಿದ್ದ
ಮಾಡಿ-ರ-ಬ-ಹುದು
ಮಾಡಿರಲಿ
ಮಾಡಿರಿ
ಮಾಡಿ-ರುತ್ತಾನೆ
ಮಾಡಿರುವ
ಮಾಡಿ-ರು-ವುದೂ
ಮಾಡಿರುವೆ
ಮಾಡಿವೆ
ಮಾಡಿ-ಸ-ಬೇ-ಕಾ-ಯಿತು
ಮಾಡಿಸಲು
ಮಾಡಿಸಿ
ಮಾಡಿ-ಸಿ-ಕೊಂಡರೂ
ಮಾಡಿ-ಸಿ-ಕೊಂಡು
ಮಾಡಿ-ಸಿ-ಕೊಂಡು-ದಲ್ಲದೇ
ಮಾಡಿ-ಸಿ-ಕೊಳ್ಳುವ
ಮಾಡಿಸಿದ
ಮಾಡಿ-ಸಿ-ದರು
ಮಾಡಿ-ಸಿ-ದರೆ
ಮಾಡಿ-ಸಿದ್ದರೆ
ಮಾಡಿಸು
ಮಾಡಿ-ಸುತ್ತದೆ
ಮಾಡಿ-ಸುತ್ತಾರೆ
ಮಾಡಿಸುವ
ಮಾಡಿ-ಸು-ವುದು
ಮಾಡೀತು
ಮಾಡು
ಮಾಡುತ್ತಾರೆ
ಮಾಡುತ್ತಿದ್ದಾ-ರಲ್ಲ
ಮಾಡುತ್ತಿದ್ದು-ದನ್ನು
ಮಾಡುತ್ತಿ-ರಲಿ
ಮಾಡು-ವಂತಿಲ್ಲ
ಮಾಡುವಂಥ
ಮಾಡುತ್ತ
ಮಾಡುತ್ತದೆ
ಮಾಡುತ್ತ-ಲಿ-ರುತ್ತದೆ
ಮಾಡುತ್ತಲೂ
ಮಾಡುತ್ತಲೇ
ಮಾಡುತ್ತವೆ
ಮಾಡುತ್ತಾ
ಮಾಡುತ್ತಾನೆ
ಮಾಡುತ್ತಾ-ರಲ್ಲ
ಮಾಡುತ್ತಾ-ರಲ್ಲದೆ
ಮಾಡುತ್ತಾ-ರಲ್ಲವೇ
ಮಾಡುತ್ತಾರೆ
ಮಾಡುತ್ತಾ-ರೆಂದಲ್ಲ
ಮಾಡುತ್ತಾ-ರೆಂದು
ಮಾಡುತ್ತಾ-ರೆಂದು-ಕೊಂಡು
ಮಾಡುತ್ತಾ-ರೆಂಬುದು
ಮಾಡುತ್ತಾ-ರೆನ್ನು-ವಂತಿಲ್ಲ
ಮಾಡುತ್ತಾ-ರೆ-ಷೇಕ್ಸ್ಪಿ-ಯ-ರನ
ಮಾಡುತ್ತಾರೋ
ಮಾಡುತ್ತಾಳೆ
ಮಾಡುತ್ತಿದೆ
ಮಾಡುತ್ತಿದ್ದ
ಮಾಡುತ್ತಿದ್ದರು
ಮಾಡುತ್ತಿದ್ದರೂ
ಮಾಡುತ್ತಿದ್ದರೆ
ಮಾಡುತ್ತಿದ್ದಳು
ಮಾಡುತ್ತಿದ್ದ-ವರು
ಮಾಡುತ್ತಿದ್ದಾನೆ
ಮಾಡುತ್ತಿದ್ದಾರೆ
ಮಾಡುತ್ತಿದ್ದಾಳೆ
ಮಾಡುತ್ತಿದ್ದೀ-ಯಮ್ಮಾ
ಮಾಡುತ್ತಿದ್ದೀರಿ
ಮಾಡುತ್ತಿದ್ದು
ಮಾಡುತ್ತಿದ್ದುದು
ಮಾಡುತ್ತಿದ್ದೆ
ಮಾಡುತ್ತಿದ್ದೇನೆ
ಮಾಡುತ್ತಿದ್ದೇ-ವಾ-ದರೂ
ಮಾಡುತ್ತಿದ್ದೇವೆ
ಮಾಡುತ್ತಿದ್ದೇ-ವೆಂಬು-ದನ್ನು
ಮಾಡುತ್ತಿ-ರಲಿ
ಮಾಡುತ್ತಿ-ರ-ಲಿಲ್ಲ
ಮಾಡುತ್ತಿ-ರುತ್ತಾರೆ
ಮಾಡುತ್ತಿ-ರುತ್ತೇವೆ
ಮಾಡುತ್ತಿ-ರುವ
ಮಾಡುತ್ತಿ-ರು-ವರೆ
ಮಾಡುತ್ತಿ-ರು-ವಾಗ
ಮಾಡುತ್ತಿ-ರು-ವಾ-ಗಲೇ
ಮಾಡುತ್ತಿ-ರು-ವು-ದಕ್ಕೆ
ಮಾಡುತ್ತಿ-ರು-ವು-ದನ್ನೂ
ಮಾಡುತ್ತಿ-ರು-ವುದು
ಮಾಡುತ್ತಿಲ್ಲ
ಮಾಡುತ್ತಿವೆ
ಮಾಡುತ್ತೀಯಾ
ಮಾಡುತ್ತೀರಾ
ಮಾಡುತ್ತೀರಿ
ಮಾಡುತ್ತೇನೆ
ಮಾಡುತ್ತೇ-ನೆಂದರೆ
ಮಾಡುತ್ತೇವೆ
ಮಾಡುತ್ತೇ-ವೆಂಬುದು
ಮಾಡುವ
ಮಾಡು-ವಂತಹ
ಮಾಡು-ವಂತಾಗ
ಮಾಡು-ವಂತಾ-ಗ-ಬೇಕು
ಮಾಡು-ವಂತಾ-ಗ-ಬೇ-ಕೆಂಬುದು
ಮಾಡು-ವಂತಾ-ದರೆ
ಮಾಡುವಂತೆ
ಮಾಡುವನು
ಮಾಡುವನೆ
ಮಾಡುವರು
ಮಾಡುವರೋ
ಮಾಡುವಲ್ಲಿ
ಮಾಡುವಳು
ಮಾಡು-ವ-ವನ
ಮಾಡು-ವ-ವ-ನಲ್ಲ
ಮಾಡು-ವ-ವನು
ಮಾಡು-ವ-ವನೂ
ಮಾಡು-ವ-ವನೇ
ಮಾಡು-ವ-ವರ
ಮಾಡು-ವ-ವ-ರನ್ನು
ಮಾಡು-ವ-ವ-ರಾರು
ಮಾಡು-ವ-ವರು
ಮಾಡು-ವ-ವ-ರೆಗೂ
ಮಾಡುವಾಗ
ಮಾಡು-ವಾ-ಗಲೂ
ಮಾಡುವಿರಿ
ಮಾಡು-ವು-ದಲ್ಲದೆ
ಮಾಡು-ವು-ದಾ-ಗಲೀ
ಮಾಡು-ವು-ದಿಲ್ಲವೇ
ಮಾಡು-ವು-ದಕ್ಕಾಗಿ
ಮಾಡು-ವು-ದಕ್ಕಿಂತ
ಮಾಡು-ವು-ದಕ್ಕೂ
ಮಾಡು-ವು-ದಕ್ಕೆ
ಮಾಡು-ವು-ದನ್ನು
ಮಾಡು-ವು-ದರ
ಮಾಡು-ವು-ದ-ರಲ್ಲಿ
ಮಾಡು-ವು-ದ-ರಲ್ಲೇ
ಮಾಡು-ವು-ದ-ರಿಂದ
ಮಾಡು-ವು-ದಲ್ಲದೆ
ಮಾಡು-ವು-ದಲ್ಲದೇ
ಮಾಡು-ವು-ದಾ-ಗಲಿ
ಮಾಡು-ವು-ದಿ-ರಲಿ
ಮಾಡು-ವು-ದಿಲ್ಲ
ಮಾಡು-ವು-ದಿಲ್ಲ-ವೆಂದು
ಮಾಡು-ವು-ದಿಷ್ಟೆ-ಜಾಗ್ರತ
ಮಾಡುವುದು
ಮಾಡುವುದೂ
ಮಾಡು-ವು-ದೆಂದರೆ
ಮಾಡು-ವು-ದೆಂದು
ಮಾಡುವುದೇ
ಮಾಡು-ವು-ದೇನು
ಮಾಡುವುದೊ
ಮಾಡು-ವು-ದೊಂದೇ
ಮಾಡು-ವು-ದೊ-ಳಿ-ತಲ್ಲವೇ
ಮಾಡುವುವು
ಮಾಡುವೆ
ಮಾಡೆಂದು
ಮಾಡೋಣ
ಮಾಡೋ-ಣ-ವೆಂದು
ಮಾತ
ಮಾತನಾಡ
ಮಾತ-ನಾ-ಡ-ತೊ-ಡ-ಗಿದ
ಮಾತ-ನಾ-ಡ-ತೊ-ಡ-ಗಿ-ದರು
ಮಾತ-ನಾ-ಡ-ತೊ-ಡ-ಗಿ-ದಳು
ಮಾತ-ನಾ-ಡ-ಬಲ್ಲರು
ಮಾತ-ನಾ-ಡ-ಬೇ-ಕಾ-ಗಿದೆ
ಮಾತ-ನಾ-ಡ-ಬೇಕು
ಮಾತ-ನಾ-ಡ-ಲಾರ
ಮಾತ-ನಾ-ಡ-ಲಾ-ರೆವು
ಮಾತ-ನಾ-ಡಲಿ
ಮಾತ-ನಾ-ಡಲು
ಮಾತ-ನಾ-ಡಲೂ
ಮಾತ-ನಾ-ಡ-ಹೊ-ರ-ಟಾಗ
ಮಾತನಾಡಿ
ಮಾತ-ನಾ-ಡಿದ
ಮಾತ-ನಾ-ಡಿ-ದಂತೆ
ಮಾತ-ನಾ-ಡಿ-ದರೆ
ಮಾತ-ನಾ-ಡಿ-ದ-ವ-ರಲ್ಲ
ಮಾತ-ನಾ-ಡಿ-ದಾಗ
ಮಾತ-ನಾ-ಡಿ-ದಿರಿ
ಮಾತ-ನಾ-ಡಿ-ದುದು
ಮಾತ-ನಾ-ಡಿದ್ದಕ್ಕೆ
ಮಾತ-ನಾ-ಡಿದ್ದಿಲ್ಲ
ಮಾತ-ನಾ-ಡಿದ್ದೆವು
ಮಾತ-ನಾ-ಡಿದ್ದೇನೆ
ಮಾತ-ನಾ-ಡಿ-ಸದೇ
ಮಾತ-ನಾ-ಡಿಸಿ
ಮಾತ-ನಾ-ಡಿ-ಸಿದೆ
ಮಾತ-ನಾ-ಡಿ-ಸಿದ್ದ-ರಿಂದಲೇ
ಮಾತನಾಡು
ಮಾತ-ನಾ-ಡುತ್ತ
ಮಾತ-ನಾ-ಡುತ್ತಾನೆ
ಮಾತ-ನಾ-ಡುತ್ತಾ-ರಷ್ಟೆ
ಮಾತ-ನಾ-ಡುತ್ತಿ
ಮಾತ-ನಾ-ಡುತ್ತಿದ್ದ
ಮಾತ-ನಾ-ಡುತ್ತಿದ್ದಾಗ
ಮಾತ-ನಾ-ಡುತ್ತಿದ್ದಾ-ಗಲೇ
ಮಾತ-ನಾ-ಡುತ್ತಿದ್ದಾ-ಳೆಂದು
ಮಾತ-ನಾ-ಡುತ್ತಿದ್ದೆ
ಮಾತ-ನಾ-ಡುತ್ತಿ-ರುತ್ತಾನೆ
ಮಾತ-ನಾ-ಡುತ್ತಿ-ರು-ವುದು
ಮಾತ-ನಾ-ಡುವ
ಮಾತ-ನಾ-ಡು-ವಂತೆ
ಮಾತ-ನಾ-ಡು-ವಾಗ
ಮಾತ-ನಾ-ಡು-ವು-ದನ್ನು
ಮಾತ-ನಾ-ಡು-ವು-ದಿಲ್ಲ
ಮಾತ-ನಾ-ಡು-ವುದು
ಮಾತ-ನಾ-ಡು-ವು-ದುಂಟು-ನಿನ್ನ
ಮಾತನ್ನಾ-ಡ-ಲಾ-ರರು
ಮಾತನ್ನು
ಮಾತನ್ನೂ
ಮಾತನ್ನೇ
ಮಾತನ್ನೋ
ಮಾತಲ್ಲ
ಮಾತಾಗಿಯೇ
ಮಾತಾ-ಡ-ತೊ-ಡ-ಗಿದ್ದಾರೆ
ಮಾತಾಡಲು
ಮಾತಾ-ಡಿ-ಸುತ್ತಾಳೆ
ಮಾತಾ-ಡಿ-ದರು
ಮಾತಾ-ಡಿದ್ದಿಲ್ಲ
ಮಾತಾ-ಡಿ-ಬಿ-ಡುತ್ತೇನೆ
ಮಾತಾಡುತ್ತ
ಮಾತಾ-ಡುತ್ತಾರೆ
ಮಾತಾ-ಡುತ್ತಿದ್ದ-ತನ್ನ
ಮಾತಾ-ಡು-ವ-ವರ
ಮಾತಾಡೋದು
ಮಾತಾ-ನಾ-ಡಿ-ದಾಗ
ಮಾತಾ-ಪಿ-ತೃ-ಗಳ
ಮಾತಾ-ಪಿ-ತೃ-ಗ-ಳಿಗೆ
ಮಾತಾ-ಪಿ-ತೃ-ಗಳು
ಮಾತಿಗೂ
ಮಾತಿಗೆ
ಮಾತಿದು
ಮಾತಿದೆ
ಮಾತಿನ
ಮಾತಿನಂತೆ
ಮಾತಿನಲ್ಲಿ
ಮಾತಿನಲ್ಲೇ
ಮಾತಿನಿಂದ
ಮಾತಿ-ನಿಂದಲೇ
ಮಾತಿಲ್ಲಿ
ಮಾತು
ಮಾತುಅದೂ
ಮಾತು-ಇಪ್ಪತ್ತೈದು
ಮಾತುಕತೆ
ಮಾತು-ಕ-ತೆ-ಗ-ಳನ್ನೂ
ಮಾತು-ಕ-ತೆ-ಯಾಡಿ
ಮಾತು-ಕ-ತೆ-ಯಾ-ಡುತ್ತ
ಮಾತು-ಕ-ತೆ-ಯಾ-ಡುತ್ತಿದ್ದರು
ಮಾತು-ಕ-ತೆ-ಯೆಲ್ಲ
ಮಾತುಗಳ
ಮಾತು-ಗ-ಳನ್ನಾಡಿ
ಮಾತು-ಗ-ಳನ್ನಾ-ಡುತ್ತ
ಮಾತು-ಗ-ಳನ್ನು
ಮಾತು-ಗ-ಳನ್ನೂ
ಮಾತು-ಗ-ಳನ್ನೆ
ಮಾತು-ಗ-ಳನ್ನೇ
ಮಾತು-ಗ-ಳಲ್ಲ
ಮಾತು-ಗ-ಳಲ್ಲಾ-ದರೂ
ಮಾತು-ಗ-ಳಲ್ಲಿ
ಮಾತು-ಗ-ಳಲ್ಲೇ
ಮಾತು-ಗ-ಳಿಂದ
ಮಾತು-ಗ-ಳಿ-ಗಾಗಿ
ಮಾತು-ಗ-ಳಿಲ್ಲ
ಮಾತು-ಗ-ಳಿವು
ಮಾತುಗಳು
ಮಾತುಗಳೂ
ಮಾತು-ಗ-ಳೆಲ್ಲ
ಮಾತುಗಳೇ
ಮಾತು-ಗಾ-ರ-ರನ್ನೂ
ಮಾತು-ಗಾ-ರಿ-ಕೆ-ಯನ್ನು
ಮಾತು-ಗಾ-ರಿ-ಕೆ-ಯಾ-ಗಲಿ
ಮಾತು-ಮಾ-ತಿಗೆ
ಮಾತೃ
ಮಾತೃತ್ವದ
ಮಾತೃ-ದೇ-ವೋ-ಭವ
ಮಾತೃ-ಭಾ-ವದ
ಮಾತೃ-ಭೂ-ಮಿ-ಯನ್ನು
ಮಾತೃ-ಮೂರ್ತಿ-ಯನ್ನು
ಮಾತೃ-ಮೂರ್ತಿಯೇ
ಮಾತೃಸ್ಥಾ-ನದ
ಮಾತೃ-ಹೃ-ದಯ
ಮಾತೆ
ಮಾತೆತ್ತಿ
ಮಾತೆಯ
ಮಾತೆಯಂತೆ
ಮಾತೆಯನ್ನೇ
ಮಾತೆಯರ
ಮಾತೆಯಾಗಿ
ಮಾತೆ-ಯೊಬ್ಬಳ
ಮಾತೇ
ಮಾತೇನಲ್ಲ
ಮಾತೊಂದಿದೆ
ಮಾತ್ರ
ಮಾತ್ರಕ್ಕೆ
ಮಾತ್ರಕ್ಕೇ
ಮಾತ್ರದಿಂದ
ಮಾತ್ರ-ದಿಂದಲೆ
ಮಾತ್ರ-ದಿಂದಲೇ
ಮಾತ್ರ-ದಿಂದಲೋ
ಮಾತ್ರನೇ
ಮಾತ್ರವಲ್ಲ
ಮಾತ್ರ-ವಲ್ಲದೆ
ಮಾತ್ರ-ವಲ್ಲವೇ
ಮಾತ್ರವಾಗಿ
ಮಾತ್ರ-ವಿದ್ದಲ್ಲಿ
ಮಾತ್ರವೆಂದು
ಮಾತ್ರವೇ
ಮಾತ್ರಹೇಗೆ
ಮಾತ್ರೆ
ಮಾತ್ರೆಗಳ
ಮಾತ್ಸರ್ಯ
ಮಾತ್ಸರ್ಯ-ಗ-ಳಿಲ್ಲ-ದಿ-ರುತ್ತಿದ್ದರೆ
ಮಾತ್ಸರ್ಯ-ಗ-ಳೆಂಬ
ಮಾತ್ಸರ್ಯದ
ಮಾದಕ
ಮಾದ-ಕದ್ರವ್ಯ
ಮಾದ-ಕದ್ರವ್ಯ-ಗಳೇ
ಮಾದ-ಕ-ವಸ್ತು-ಗಳ
ಮಾದ-ರಿ-ಗ-ಳನ್ನು
ಮಾದ-ರಿ-ಗಳು
ಮಾದರಿಯ
ಮಾದ-ರಿ-ಯನ್ನ-ನು-ಸ-ರಿ-ಸಿಯೇ
ಮಾದ-ರಿ-ಯಾ-ಗ-ಬಲ್ಲ
ಮಾಧವ
ಮಾಧುರ್ಯ
ಮಾಧ್ಯಮವೇ
ಮಾಧ್ಯಮಿಕ
ಮಾನ-ಭಂಗ-ವಾ-ಗದೆ
ಮಾನವ
ಮಾನ-ವ-ರಿದ್ದಾರೆ
ಮಾನ-ವ-ಕಲ್ಯಾ-ಣದ
ಮಾನ-ವ-ಕು-ಲದ
ಮಾನ-ವ-ಜ-ನಾಂಗಕ್ಕೆ
ಮಾನ-ವ-ಜ-ನಾಂಗದ
ಮಾನ-ವ-ಜೀವಿ
ಮಾನ-ವ-ತಾ-ವಾದಿ
ಮಾನವತೆ
ಮಾನ-ವ-ತೆ-ಇಂಥ
ಮಾನ-ವ-ತೆಯೂ
ಮಾನವನ
ಮಾನ-ವ-ನನ್ನಾಗಿ
ಮಾನ-ವ-ನನ್ನು
ಮಾನ-ವ-ನಾಗಿ
ಮಾನ-ವ-ನಿಂದ
ಮಾನ-ವ-ನಿಗೆ
ಮಾನವನು
ಮಾನವನೂ
ಮಾನ-ವ-ನೆಂದರೆ
ಮಾನ-ವ-ನೆಂದೇ
ಮಾನವರ
ಮಾನ-ವ-ರಲ್ಲಿ
ಮಾನ-ವ-ರಾಕ್ಷ-ಸ-ರೆಂದು
ಮಾನ-ವ-ರಾ-ಗ-ಬೇಕು
ಮಾನ-ವ-ರಿಗೆ
ಮಾನ-ವ-ರಿದ್ದಾರೆ
ಮಾನವರು
ಮಾನ-ವ-ರೆಂಬ
ಮಾನ-ವ-ಳಾಗಿ
ಮಾನವಾಗಿ
ಮಾನವೀಯ
ಮಾನ-ವೀ-ಯ-ತೆಯ
ಮಾನ-ವೀ-ಯ-ತೆ-ಯಪ್ರಾ-ಮಾ-ಣಿ-ಕ-ತೆಯ
ಮಾನವೂ
ಮಾನಸಂ
ಮಾನ-ಸ-ದಲ್ಲಿ
ಮಾನಸಿಕ
ಮಾನ-ಸಿ-ಕ-ತೊಂದ-ರೆ-ಸೈ-ಕ-ಲಾ-ಜಿ-ಕಲ್
ಮಾನ-ಸಿ-ಕ-ದಾರ್ಢ್ಯ
ಮಾನ-ಸಿ-ಕ-ಮೂ-ಲ-ವಾ-ದದ್ದೆಂದು
ಮಾನ-ಸಿ-ಕ-ವಾಗಿ
ಮಾನ-ಸಿ-ಕ-ವಾ-ಗಿಯೂ
ಮಾನ-ಸಿ-ಕ-ವಾದ
ಮಾನುಷವೂ
ಮಾನ್ಯ-ತೆ-ಇ-ವೆಲ್ಲ
ಮಾನ್ಯನೂ
ಮಾಯ-ವಾ-ಗುತ್ತ-ದೆ-ಎಂಬುದು
ಮಾಯ-ವಾ-ಯಿತು
ಮಾಯ-ವಾ-ಗ-ತೊ-ಡಗಿ
ಮಾಯ-ವಾ-ಗ-ತೊ-ಡ-ಗುತ್ತದೆ
ಮಾಯ-ವಾ-ಗದೆ
ಮಾಯ-ವಾ-ಗಲೇ
ಮಾಯವಾಗಿ
ಮಾಯ-ವಾ-ಗಿವೆ
ಮಾಯ-ವಾ-ಗುತ್ತದೆ
ಮಾಯ-ವಾ-ಗುತ್ತವೆ
ಮಾಯ-ವಾ-ಗುವ
ಮಾಯ-ವಾ-ಗು-ವಾಗ
ಮಾಯ-ವಾ-ಗು-ವು-ದನ್ನು
ಮಾಯ-ವಾ-ಗು-ವು-ದ-ರಲ್ಲಿ
ಮಾಯ-ವಾ-ಗು-ವು-ದಲ್ಲವೇ
ಮಾಯ-ವಾ-ಗು-ವುದು
ಮಾಯ-ವಾ-ಗು-ವು-ದೆಂಬು-ದನ್ನು
ಮಾಯ-ವಾ-ಗು-ವುದೋ
ಮಾಯವಾದ
ಮಾಯ-ವಾ-ದಂತೆ
ಮಾಯ-ವಾ-ದದ್ದಿ-ದೆಯೆ
ಮಾಯ-ವಾ-ಯಿತು
ಮಾಯಾ
ಮಾಯಾ-ಶಕ್ತಿ-ಯಿಂದ
ಮಾಯಾ-ಲೋ-ಕ-ವೆಂದು
ಮಾಯೆಯ
ಮಾಯೆಯತ್ತ
ಮಾರಕ
ಮಾರ-ಕಪ್ರಾ-ಯ-ವಾಗಿ
ಮಾರ-ಕ-ವಾ-ಗುತ್ತ-ಲಿದೆ
ಮಾರ-ಕ-ಶಕ್ತಿ
ಮಾರ-ಕ-ಶಕ್ತಿಯ
ಮಾರ-ಕಾಸ್ತ್ರ-ಗಳ
ಮಾರ-ಕಾಸ್ತ್ರ-ಗ-ಳಿಂದ
ಮಾರನೇ
ಮಾರಲು
ಮಾರಾ-ಟಕ್ಕಾ-ಗಿಯೂ
ಮಾರಾ-ಟ-ದಿಂದ
ಮಾರಿ
ಮಾರಿದರು
ಮಾರಿದಾಗ
ಮಾರಿಸನ್
ಮಾರೀಚತೆ
ಮಾರು-ಕಟ್ಟೆಗೆ
ಮಾರು-ಕಟ್ಟೆಯ
ಮಾರು-ಕಟ್ಟೆ-ಯಲ್ಲೂ
ಮಾರುತಿ
ಮಾರು-ತಿ-ದೇ-ವರ
ಮಾರುತ್ತೇನೆ
ಮಾರುಬಿಟ್ಟು
ಮಾರುವ
ಮಾರು-ವಂತಾ-ಯಿತು
ಮಾರ್ಕೋಪೊಲೋ
ಮಾರ್ಕ್ಸ್
ಮಾರ್ಗ
ಮಾರ್ಗ-ದರ್ಶ-ಕರ
ಮಾರ್ಗ-ದರ್ಶನ
ಮಾರ್ಗ-ಗ-ಳಂತೆ
ಮಾರ್ಗ-ಗ-ಳನ್ನು
ಮಾರ್ಗ-ಗ-ಳಲ್ಲಿ
ಮಾರ್ಗದ
ಮಾರ್ಗ-ದರ್ಶಕ
ಮಾರ್ಗ-ದರ್ಶ-ಕ-ರೆಂದರೆ
ಮಾರ್ಗ-ದರ್ಶ-ಕ-ನಾಗಿ
ಮಾರ್ಗ-ದರ್ಶ-ಕರ
ಮಾರ್ಗ-ದರ್ಶ-ಕ-ರಾ-ಗಿದ್ದರು
ಮಾರ್ಗ-ದರ್ಶ-ಕ-ರಾ-ಗಿಲ್ಲ-ದಿ-ರು-ವುದು
ಮಾರ್ಗ-ದರ್ಶನ
ಮಾರ್ಗ-ದರ್ಶ-ನದ
ಮಾರ್ಗ-ದರ್ಶ-ನ-ದಲ್ಲಿ
ಮಾರ್ಗ-ದರ್ಶ-ನ-ವನ್ನು
ಮಾರ್ಗ-ದರ್ಶ-ನ-ವಾ-ಗ-ಬಲ್ಲ
ಮಾರ್ಗದಲ್ಲಿ
ಮಾರ್ಗದಿಂದ
ಮಾರ್ಗ-ದಿಂದಲೇ
ಮಾರ್ಗವನ್ನು
ಮಾರ್ಗವಾಗಿ
ಮಾರ್ಗ-ವಾ-ವುದು
ಮಾರ್ಗವಿದ್ದೇ
ಮಾರ್ಗವೇ
ಮಾರ್ಗವೇನು
ಮಾರ್ಗವೊಂದೇ
ಮಾರ್ಚ್
ಮಾರ್ದನಿ
ಮಾರ್ದ-ನಿ-ಗೊಂಡಿತು
ಮಾರ್ದ-ನಿ-ಸುತ್ತಿ-ದೆ-ಯಷ್ಟೇ
ಮಾರ್ಪ-ಡ-ಬಾ-ರದು
ಮಾರ್ಪಾಡಿನ
ಮಾರ್ಫಿನ್
ಮಾರ್ಮಿಕ
ಮಾರ್ಮಿ-ಕ-ವಾ-ಗಿಯೂ
ಮಾಲಿಕರೂ
ಮಾಲಿನ್ಯ
ಮಾಲಿನ್ಯದ
ಮಾಲೀ-ಕ-ನೌ-ಕ-ರ-ರೊ-ಳ-ಗಿ-ರಲಿ
ಮಾಲ್ಟ್ಸ್
ಮಾಳಿಗೆ
ಮಾಳಿಗೆಯ
ಮಾವನ
ಮಾವನ್ನು
ಮಾವಿನ
ಮಾವಿ-ನ-ಕಾಯಿ
ಮಾವಿ-ನ-ತೋ-ಪು-ಗ-ಳನ್ನು
ಮಾವಿ-ನ-ಮ-ರದ
ಮಾವಿ-ನ-ಹಣ್ಣನ್ನು
ಮಾವು
ಮಾಸ
ಮಾಸ-ಪತ್ರಿಕೆ
ಮಾಸ-ಪತ್ರಿ-ಕೆ-ಯಲ್ಲಿ
ಮಾಸ-ಪತ್ರಿ-ಕೆ-ಯಲ್ಲೂ
ಮಾಸ-ಪತ್ರಿ-ಕೆ-ಯೊಂದ-ರಲ್ಲಿ
ಮಾಸು-ವು-ದಿಲ್ಲ
ಮಾಸುವುದು
ಮಾಸ್ಕೋ
ಮಾಸ್ತಿ
ಮಾಹಾತ್ಮ್ಯ
ಮಾಹಾತ್ಮ್ಯ-ಗಳ
ಮಾಹಾತ್ಮ್ಯ-ಗ-ಳನ್ನು
ಮಾಹಾತ್ಮ್ಯ-ವನ್ನು
ಮಾಹಾತ್ಮ್ಯ-ವನ್ನೂ
ಮಾಹಾತ್ಮ್ಯೆ
ಮಾಹಾತ್ಮ್ಯೆ-ಗ-ಳನ್ನು
ಮಾಹಾತ್ಮ್ಯೆ-ಯನ್ನು
ಮಾಹಿ-ತಿ-ಗ-ಳನ್ನು
ಮಿಂಚಿ
ಮಿಂಚಿನಂತೆ
ಮಿಂಚಿಹೋದ
ಮಿಂದು
ಮಿಕ್ಕದ್ದನ್ನು
ಮಿಕ್ಕಿ
ಮಿಗಿಲಾಗಿ
ಮಿಗಿ-ಲಾ-ಗಿ-ರುವ
ಮಿಗಿಲಾದ
ಮಿಗುತ್ತ-ದೆ-ಯೆಂದರೆ
ಮಿಚ್ಚೆಲ್
ಮಿಡಿ-ತ-ದೊಂದಿಗೆ
ಮಿಡಿಯುವ
ಮಿಡ್ನಾ-ಪು-ರ-ಜಿಲ್ಲೆಯ
ಮಿತ-ಭಾ-ಷಿ-ಗ-ಳನ್ನೂ
ಮಿತವಾದ
ಮಿತವ್ಯ-ಯಕ್ಕೆ
ಮಿತವ್ಯ-ಯಿಯ
ಮಿತಾಹಾರ
ಮಿತಿ
ಮಿತಿ-ಗ-ಳಿಂದ
ಮಿತಿಗೆ
ಮಿತಿಮೀರಿ
ಮಿತಿ-ಮೀ-ರಿದ
ಮಿತಿ-ಯಿ-ರ-ಬೇಕು
ಮಿತಿಯಿಲ್ಲ
ಮಿತ್ಥಂ
ಮಿತ್ರ
ಮಿತ್ರನೊಬ್ಬ
ಮಿತ್ರರ
ಮಿತ್ರರಿಗೆ
ಮಿತ್ರರು
ಮಿತ್ರ-ರು-ಇತ್ಯಾ-ದಿ-ಗ-ಳನ್ನು
ಮಿತ್ರರೂ
ಮಿತ್ರರೆಲ್ಲ
ಮಿತ್ರ-ರೊಬ್ಬರು
ಮಿಥ್ಯ
ಮಿಥ್ಯತ್ವ-ಗ-ಳನ್ನು
ಮಿಥ್ಯಾ-ರೋ-ಪ-ಗ-ಳನ್ನೂ
ಮಿಥ್ಯೆ
ಮಿದುಳನ್ನು
ಮಿದುಳನ್ನೇ
ಮಿದು-ಳಿ-ಗಿಂತ
ಮಿದುಳಿಗೆ
ಮಿದುಳಿನ
ಮಿದು-ಳಿ-ನಲ್ಲಿ
ಮಿದುಳಿಲ್ಲ
ಮಿದುಳು
ಮಿನಿ-ಟು-ಗ-ಳಂತೆ
ಮಿನು-ಗ-ಲಿಲ್ಲ
ಮಿನುಗುವ
ಮಿಯಾವ್
ಮಿರರ್
ಮಿಲಾನ್
ಮಿಲಿಟರಿ
ಮಿಲಿ-ತ-ವಾ-ಗಿದ್ದವು
ಮಿಲಿ-ಮೀ-ಟರ್
ಮಿಲಿಯ
ಮಿಲಿ-ಯಕ್ಕಿಂತ
ಮಿಲಿ-ಯ-ಗಟ್ಟಲೆ
ಮಿಲಿ-ಯ-ದಲ್ಲಿ
ಮಿಲ್
ಮಿಲ್ಟನ್
ಮಿಳಿತ
ಮಿಶ್ರ-ಣ-ವಿ-ರ-ಬ-ಹು-ದೆಂಬ
ಮಿಶ್ರ-ಣ-ದಿಂದ
ಮಿಶ್ರ-ಣ-ವಿ-ದೆಯೇ
ಮಿಶ್ರಿತ
ಮಿಷನ್ನಿನ
ಮೀಟಿಂಗಿ-ನಲ್ಲಿ
ಮೀನುಗಳ
ಮೀಮಾಂಸೆ
ಮೀರತ್
ಮೀರಿ
ಮೀರಿತು
ಮೀರಿದ
ಮೀರಿದರೆ
ಮೀರಿ-ದು-ದ-ರಿಂದ
ಮೀರಿದುದು
ಮೀರಿನಿಂತ
ಮೀರಿ-ಸ-ಬಲ್ಲ
ಮೀರಿಸಿದ
ಮೀರಿಸುವ
ಮೀರಿ-ಸು-ವಲ್ಲಿ
ಮೀರಿಹೋಗಿ
ಮೀರಿ-ಹೋ-ಗುವ
ಮೀಸ-ಲಾ-ಗಿ-ಡುತ್ತಿದ್ದೆ
ಮೀಸಲಲ್ಲ
ಮೀಸ-ಲಾ-ಗದೇ
ಮೀಸ-ಲಾ-ಗಿಡು
ಮೀಸಲಾದ
ಮೀಸ-ಲಾ-ದು-ದಲ್ಲ
ಮೀಸ-ಲಿಟ್ಟಿದ್ದರೆ
ಮೀಸ-ಲಿ-ಡುತ್ತೇನೆ
ಮೀಸೆ
ಮೀಸೆಗಳ
ಮೀಸೆಯ
ಮುಂಚಿ-ತ-ವಾಗಿ
ಮುಂಚಿನ
ಮುಂಜಾನೆ
ಮುಂಜಾ-ನೆ-ಯಿಂದ
ಮುಂತಾದ
ಮುಂತಾ-ದ-ವರು
ಮುಂತಾ-ದ-ವು-ಗ-ಳಿಂದುಂಟಾದ
ಮುಂತಾ-ದ-ವು-ಗ-ಳನ್ನು
ಮುಂದಕ್ಕಿ-ರಿ-ಸಿ-ಕೊಳ್ಳುತ್ತೀರಾ
ಮುಂದಕ್ಕೆ
ಮುಂದಾ-ಗ-ಬೇ-ಕಾ-ಯಿತು
ಮುಂದಾಗಿ
ಮುಂದಾದ
ಮುಂದಾ-ಲೋ-ಚನೆ
ಮುಂದಾ-ಲೋ-ಚ-ನೆಗೆ
ಮುಂದಾ-ಳು-ಗಳು
ಮುಂದಾ-ಳು-ಗಳೂ
ಮುಂದಿಟ್ಟು
ಮುಂದಿದ್ದಾರೆ
ಮುಂದಿದ್ದಾ-ರೆಂದೂ
ಮುಂದಿನ
ಮುಂದಿ-ರಿ-ಸಿ-ಕೊಂಡಿವೆ
ಮುಂದಿ-ರುತ್ತೇನೆ
ಮುಂದಿರುವ
ಮುಂದಿರ್ಪು-ದುಚ್ಚ
ಮುಂದಿಲ್ಲ
ಮುಂದು
ಮುಂದು-ವ-ರಿ-ದರೂ
ಮುಂದು-ವ-ರಿ-ದ-ವ-ರಲ್ಲಿ
ಮುಂದುವರಿ
ಮುಂದು-ವ-ರಿದ
ಮುಂದು-ವ-ರಿ-ದ-ವರ
ಮುಂದು-ವ-ರಿ-ದ-ವರು
ಮುಂದು-ವ-ರಿ-ದ-ವ-ರೆಲ್ಲರೂ
ಮುಂದು-ವ-ರಿ-ದಿತ್ತು
ಮುಂದು-ವ-ರಿ-ದಿದ್ದರು
ಮುಂದು-ವ-ರಿ-ದಿದ್ದರೆ
ಮುಂದು-ವ-ರಿದು
ಮುಂದು-ವ-ರಿ-ಯ-ತೊ-ಡ-ಗಿದ
ಮುಂದು-ವ-ರಿ-ಯ-ದ-ವರು
ಮುಂದು-ವ-ರಿ-ಯ-ಬೇಕು
ಮುಂದು-ವ-ರಿ-ಯ-ಲಾ-ಗದ
ಮುಂದು-ವ-ರಿ-ಯಲು
ಮುಂದು-ವ-ರಿ-ಯ-ಲೇ-ಬೇಕು
ಮುಂದು-ವ-ರಿ-ಯಿತು
ಮುಂದು-ವ-ರಿ-ಯಿರಿ
ಮುಂದು-ವ-ರಿ-ಯುತ್ತ
ಮುಂದು-ವ-ರಿ-ಯುತ್ತದೆ
ಮುಂದು-ವ-ರಿ-ಯುತ್ತ-ಲಿದೆ
ಮುಂದು-ವ-ರಿ-ಯುತ್ತಲೇ
ಮುಂದು-ವ-ರಿ-ಯುತ್ತಾನೆ
ಮುಂದು-ವ-ರಿ-ಯುತ್ತಿದೆ
ಮುಂದು-ವ-ರಿ-ಯುತ್ತಿ-ರುತ್ತಾನೆ
ಮುಂದು-ವ-ರಿ-ಯು-ವಂತಾ-ಗಿ-ದೆ-ಯಲ್ಲ
ಮುಂದು-ವ-ರಿ-ಯು-ವಾಗ
ಮುಂದು-ವ-ರಿ-ಯು-ವುದು
ಮುಂದು-ವ-ರಿ-ಸ-ಬೇಕು
ಮುಂದು-ವ-ರಿ-ಸ-ಲಾ-ರ-ದಾ-ದರು
ಮುಂದು-ವ-ರಿ-ಸಲಿ
ಮುಂದು-ವ-ರಿ-ಸಲು
ಮುಂದು-ವ-ರಿ-ಸಿ-ಕೊಂಡು
ಮುಂದು-ವ-ರಿ-ಸಿ-ದರು
ಮುಂದು-ವ-ರಿ-ಸಿ-ದರೆ
ಮುಂದು-ವ-ರಿಸು
ಮುಂದು-ವ-ರಿ-ಸುತ್ತ
ಮುಂದು-ವ-ರಿ-ಸುತ್ತಲೇ
ಮುಂದು-ವ-ರಿ-ಸುತ್ತಿದೆ
ಮುಂದೂಡಲೂ
ಮುಂದೂ-ಡ-ಬಲ್ಲರು
ಮುಂದೂ-ಡಲ್ಪ-ಡುತ್ತಿದ್ದೇನೆ
ಮುಂದೆ
ಮುಂದೆಯೂ
ಮುಂದೊಂದು
ಮುಕ್ಕಾಲು
ಮುಕ್ತ-ಕಂಠ-ದಿಂದ
ಮುಕ್ತ-ನಾ-ಗ-ಬಲ್ಲ
ಮುಕ್ತ-ನಾ-ಗುತ್ತಾನೆ
ಮುಕ್ತಮಾರ್ಗ
ಮುಕ್ತ-ರನ್ನಾ-ಗಿ-ಸುತ್ತದೆ
ಮುಕ್ತ-ರಾ-ಗಿದ್ದಾರೆ
ಮುಕ್ತ-ರಾ-ಗಿದ್ದಾ-ರೆಯೇ
ಮುಕ್ತ-ರಾ-ಗುವ
ಮುಕ್ತ-ರಾ-ಗು-ವುದು
ಮುಕ್ತರಾದ
ಮುಕ್ತ-ರಾ-ದರು
ಮುಕ್ತ-ವಾ-ಗುತ್ತದೆ
ಮುಕ್ತಸ್ವ-ಭಾ-ವ-ವನ್ನು
ಮುಕ್ತಾಯ
ಮುಕ್ತಿ
ಮುಕ್ತಿ-ಇ-ವು-ಗ-ಳನ್ನು
ಮುಕ್ತಿಯ
ಮುಖ
ಮುಖ-ಭಾ-ವಪ್ರ-ಕಾ-ಶದ
ಮುಖಂಡ
ಮುಖಂಡ-ರಾ-ಗಿಯೋ
ಮುಖಂಡನ
ಮುಖಂಡ-ನಾ-ಗು-ವ-ವನು
ಮುಖಂಡ-ನಿಗೂ
ಮುಖಂಡ-ನಿಗೆ
ಮುಖಂಡನು
ಮುಖಂಡ-ನೆ-ನಿ-ಸಿ-ಕೊಂಡ
ಮುಖಂಡ-ನೊಬ್ಬನ
ಮುಖಂಡರ
ಮುಖಂಡ-ರಾ-ಗಲು
ಮುಖಂಡ-ರಾ-ಗುವ
ಮುಖಂಡ-ರಾ-ಗು-ವ-ವ-ರಲ್ಲಿ
ಮುಖಂಡ-ರಾ-ದರೋ
ಮುಖಂಡ-ರಿಂದ
ಮುಖಂಡ-ರಿಂದಲೇ
ಮುಖಂಡ-ರಿಗೂ
ಮುಖಂಡರು
ಮುಖಂಡ-ರು-ಗ-ಳಲ್ಲಿ
ಮುಖಂಡ-ರು-ಗ-ಳಿಂದ
ಮುಖಂಡ-ರು-ಗಳು
ಮುಖಂಡರೂ
ಮುಖಕ್ಕೆ
ಮುಖಗಳ
ಮುಖ-ಗ-ಳನ್ನು
ಮುಖ-ಗ-ಳಾದ
ಮುಖಗಳು
ಮುಖದ
ಮುಖದಲ್ಲಿ
ಮುಖದಲ್ಲೂ
ಮುಖ-ಭಂಗ-ಗೊಂಡಂತಾಗಿ
ಮುಖ-ಭಾ-ವ-ವನ್ನೂ
ಮುಖ-ಮಾರ್ಜನ
ಮುಖ-ಮುದ್ರೆ-ಇವು
ಮುಖವನ್ನು
ಮುಖವನ್ನೂ
ಮುಖವನ್ನೇ
ಮುಖವಲ್ಲ
ಮುಖವಾಗಿ
ಮುಖ-ವಾ-ಗಿಯೇ
ಮುಖ-ವಾ-ಡ-ವನ್ನು
ಮುಖವಿದೆ
ಮುಖ್ಯ
ಮುಖ್ಯ-ಕಾ-ರ-ಣ-ಗ-ಳೆನ್ನುತ್ತಾರೆ
ಮುಖ್ಯನ್ಯಾ-ಯಾ-ಧೀ-ಶರು
ಮುಖ್ಯವಲ್ಲ
ಮುಖ್ಯವಾಗಿ
ಮುಖ್ಯ-ವಾ-ಗಿ-ಕ-ಳೆದ
ಮುಖ್ಯವಾದ
ಮುಖ್ಯ-ವಾ-ದು-ದ-ರಿಂದ
ಮುಖ್ಯ-ವಾ-ದು-ದೆಂಬುದು
ಮುಖ್ಯ-ಸಂಗತಿ
ಮುಖ್ಯಸ್ಥ-ನಾ-ಗುವ
ಮುಖ್ಯಸ್ಥ-ರನ್ನೊಮ್ಮೆ
ಮುಖ್ಯಸ್ಥ-ರಾದ
ಮುಖ್ಯಸ್ಥರು
ಮುಖ್ಯಸ್ಥರೂ
ಮುಖ್ಯಸ್ಥ-ರೊಂದಿಗೆ
ಮುಖ್ಯಾಂಶ-ದಲ್ಲಿ
ಮುಖ್ಯಾಂಶ-ವನ್ನು
ಮುಖ್ಯಾಧ್ಯಾ-ಪಕಿ
ಮುಖ್ಯಾಧ್ಯಾ-ಪ-ಕಿಯ
ಮುಗಿಸುವ
ಮುಗಿದ
ಮುಗಿದರೆ
ಮುಗಿ-ದಿದ್ದರೆ
ಮುಗಿ-ದಿಲ್ಲವೇ
ಮುಗಿದು
ಮುಗಿ-ದು-ಹೋ-ಗುತ್ತದೆ
ಮುಗಿ-ಯ-ಲಿಲ್ಲ-ವೆನ್ನುವ
ಮುಗಿಯಿತು
ಮುಗಿ-ಯಿ-ತೆಂದು
ಮುಗಿ-ಯು-ವು-ದ-ರೊ-ಳ-ಗಾ-ಗಿಯೇ
ಮುಗಿ-ಲೆತ್ತ-ರಕ್ಕೇರಿ
ಮುಗಿ-ಸ-ಬಲ್ಲ
ಮುಗಿಸಲು
ಮುಗಿಸಿ
ಮುಗಿಸಿದ
ಮುಗಿ-ಸಿ-ದರೂ
ಮುಗಿ-ಸಿ-ದಾಗ
ಮುಗಿಸಿದ್ದ
ಮುಗಿ-ಸಿದ್ದರೆ
ಮುಗಿ-ಸಿದ್ದಾನೆ
ಮುಗಿ-ಸಿ-ಬಿ-ಡುತ್ತೇನೆ
ಮುಗಿಸಿಯೂ
ಮುಗಿ-ಸುತ್ತಾರೆ
ಮುಗಿಸುವ
ಮುಗಿ-ಸು-ವು-ದ-ರೊ-ಳಗೆ
ಮುಗು-ಳು-ನ-ಗೆ-ಯನ್ನು
ಮುಗುಳ್ನ-ಗುತ್ತ
ಮುಗ್ಗಟ್ಟೆಂದು
ಮುಗ್ಗಟ್ಟೊಂದಿಲ್ಲ-ದಿದ್ದರೆ
ಮುಗ್ಗ-ರಿ-ಸದೆ
ಮುಗ್ಗರಿಸಿ
ಮುಗ್ಧ
ಮುಗ್ಧ-ಗೊ-ಳಿ-ಸುತ್ತಿದ್ದವು
ಮುಗ್ಧನಾದೆ
ಮುಗ್ಧರಾಗಿ
ಮುಚ್ಚ-ಲಾ-ಗಿತ್ತು
ಮುಚ್ಚ-ಳ-ವನ್ನು
ಮುಚ್ಚಿ
ಮುಚ್ಚಿಕೊಂಡು
ಮುಚ್ಚಿ-ಕೊಳ್ಳುತ್ತದೆ
ಮುಚ್ಚಿ-ಗೆ-ಯ-ವ-ರೆಗೂ
ಮುಚ್ಚಿಡಲು
ಮುಚ್ಚಿಡುವ
ಮುಚ್ಚಿದ
ಮುಚ್ಚಿ-ಬಿ-ಡುತ್ತದೆ
ಮುಚ್ಚಿ-ರ-ಬೇಕು
ಮುಚ್ಚಿವೆ
ಮುಚ್ಚಿಸಲು
ಮುಚ್ಚಿ-ಸ-ಹೊ-ರ-ಡು-ವು-ದಾ-ಗಲೀ
ಮುಚ್ಚು
ಮುಚ್ಚುಮರೆ
ಮುಚ್ಚು-ಮ-ರೆ-ಯಿಲ್ಲದ
ಮುಚ್ಚುವ
ಮುಚ್ಚುವಂತೆ
ಮುಜಪ್ಫರ್
ಮುಟ್ಟ-ಬಲ್ಲದು
ಮುಟ್ಟ-ಲಾ-ಗುತ್ತಿದೆ
ಮುಟ್ಟಲಿ
ಮುಟ್ಟಲಿಲ್ಲ
ಮುಟ್ಟಿ
ಮುಟ್ಟಿತು
ಮುಟ್ಟಿದ
ಮುಟ್ಟಿ-ದಾ-ಗಲೂ
ಮುಟ್ಟಿದ್ದು
ಮುಟ್ಟಿದ್ದೆಂದು
ಮುಟ್ಟಿನೋಡಿ
ಮುಟ್ಟಿಯೇ
ಮುಟ್ಟಿ-ಸಿ-ದರೆ
ಮುಟ್ಟಿ-ಸುತ್ತದೆ
ಮುಟ್ಟುಗೋಲು
ಮುಟ್ಟುತ್ತದೆ
ಮುಟ್ಟುತ್ತಲೇ
ಮುಟ್ಟುವ
ಮುಟ್ಟು-ವು-ದಕ್ಕೆ
ಮುಟ್ಟು-ವು-ದಿಲ್ಲ
ಮುಡಿ-ಪಾ-ಗಿಟ್ಟು
ಮುತ್ತಲೂ
ಮುತ್ತಿ
ಮುತ್ತಿಹಾಕಿ
ಮುತ್ತಿಕೊಂಡು
ಮುತ್ತುತ್ತಿದ್ದರು
ಮುತ್ತು-ಗ-ಳಿವು
ಮುತ್ತು-ರತ್ನ-ಗ-ಳಿ-ರು-ವಂತೆ
ಮುತ್ತು-ರತ್ನ-ಗಳು
ಮುದುಕ
ಮುದು-ಕ-ನನ್ನು
ಮುದು-ಕಿ-ಯನ್ನು
ಮುದು-ಕಿ-ಯಾದ
ಮುದು-ಡ-ಲಿಲ್ಲ
ಮುದು-ಡಿ-ಕೊಂಡದ್ದಿ-ದೆಯೇ
ಮುದು-ಡಿ-ಕೊಂಡಿದ್ದೇನೆ
ಮುದ್ದಿನಿಂದ
ಮುದ್ದು
ಮುದ್ದೆಗಳೇ
ಮುದ್ದೆ-ಯಾ-ಗುತ್ತಾನೆ
ಮುದ್ರಕರು
ಮುದ್ರಣ
ಮುದ್ರ-ಣ-ಗ-ಳನ್ನು
ಮುದ್ರ-ಣ-ದಲ್ಲಿ
ಮುದ್ರ-ಣ-ವಾಗಿ
ಮುದ್ರ-ಣೋದ್ಯಮಿ
ಮುದ್ರಿ-ತ-ವಾಗಿ
ಮುದ್ರಿ-ತ-ವಾ-ಗಿದ್ದರೆ
ಮುದ್ರಿ-ಸಿದ್ದೇವೆ
ಮುದ್ರಿಸುವ
ಮುದ್ರೆ-ಗ-ಳನ್ನು
ಮುದ್ರೆಯನ್ನು
ಮುನಿ
ಮುನಿ-ಯುತ್ತಾನೆ
ಮುನಿ-ವೃತ್ತಿ-ಯನ್ನೂ
ಮುನ್ನ
ಮುನ್ನಡಿ
ಮುನ್ನಡೆ
ಮುನ್ನ-ಡೆ-ದ-ವರು
ಮುನ್ನಡೆಗೆ
ಮುನ್ನಡೆದ
ಮುನ್ನ-ಡೆ-ದರು
ಮುನ್ನ-ಡೆ-ದರೆ
ಮುನ್ನ-ಡೆ-ದಲ್ಲಿ
ಮುನ್ನ-ಡೆ-ದ-ವರ
ಮುನ್ನ-ಡೆ-ದ-ವರು
ಮುನ್ನ-ಡೆ-ದ-ವರೇ
ಮುನ್ನ-ಡೆ-ದಿದ್ದವು
ಮುನ್ನಡೆದು
ಮುನ್ನಡೆಯ
ಮುನ್ನ-ಡೆ-ಯ-ದಿದ್ದ-ವರು
ಮುನ್ನ-ಡೆ-ಯ-ಬ-ಯ-ಸು-ವ-ವರು
ಮುನ್ನ-ಡೆ-ಯ-ಬಲ್ಲೆಯಾ
ಮುನ್ನ-ಡೆ-ಯ-ಬಲ್ಲೆವೇ
ಮುನ್ನ-ಡೆ-ಯ-ಬ-ಹು-ದಾ-ದಂತೆ
ಮುನ್ನ-ಡೆ-ಯ-ಬ-ಹುದು
ಮುನ್ನ-ಡೆ-ಯ-ಬೇ-ಕಾ-ದರೆ
ಮುನ್ನ-ಡೆ-ಯ-ಬೇಕು
ಮುನ್ನ-ಡೆ-ಯ-ಬೇ-ಕೆಂಬ
ಮುನ್ನ-ಡೆ-ಯ-ಬೇ-ಕೆಂಬು-ದನ್ನು
ಮುನ್ನ-ಡೆ-ಯ-ಬೇ-ಕೆನ್ನು-ವ-ವ-ರಿಗೆ
ಮುನ್ನ-ಡೆ-ಯಲು
ಮುನ್ನ-ಡೆ-ಯಿ-ಸಿತು
ಮುನ್ನ-ಡೆ-ಯಿಸು
ಮುನ್ನಡೆಯು
ಮುನ್ನ-ಡೆ-ಯುತ್ತಾ
ಮುನ್ನ-ಡೆ-ಯುತ್ತಾರೆ
ಮುನ್ನ-ಡೆ-ಯುತ್ತಿದ್ದಂತೆ
ಮುನ್ನ-ಡೆ-ಯುತ್ತಿದ್ದೇವೆ
ಮುನ್ನ-ಡೆ-ಯುವ
ಮುನ್ನ-ಡೆ-ಯು-ವಂತೆ
ಮುನ್ನ-ಡೆ-ಯು-ವ-ವ-ನಿಗೆ
ಮುನ್ನ-ಡೆ-ಯು-ವು-ದಷ್ಟೇ
ಮುನ್ನ-ಡೆ-ಯು-ವು-ದಿ-ರಲಿ
ಮುನ್ನಡೆವ
ಮುನ್ನ-ಡೆ-ಸ-ಬೇ-ಕೆಂದು
ಮುನ್ನ-ಡೆ-ಸಿ-ದ-ವರೂ
ಮುನ್ನು
ಮುನ್ನುಗ್ಗ-ಲೇ-ಬೇಕು
ಮುನ್ನುಗ್ಗಿ
ಮುನ್ನುಗ್ಗಿದ್ದ
ಮುನ್ನುಗ್ಗುತ್ತಾನೆ
ಮುನ್ನುಗ್ಗುತ್ತಿದ್ದರು
ಮುನ್ನುಡಿ
ಮುನ್ನೂ
ಮುನ್ನೂರು
ಮುನ್ನೆಚ್ಚ-ರಿಕೆ
ಮುನ್ನೋಟ
ಮುನ್ಸಿ-ಪಾ-ಲಿ-ಟಿಗೆ
ಮುನ್ಸೂಚನೆ
ಮುನ್ಸ್ಟರ್ನಲ್ಲಿನ
ಮುಪ್ಪು-ರಿ-ಗೊಂಡ
ಮುಪ್ಪೇರು
ಮುಯ್ಯಿ
ಮುಯ್ಯಿಗೆ
ಮುರಾರೇ
ಮುರಿದಿದ್ದ
ಮುರಿದು
ಮುರಿ-ದು-ಕೊಳ್ಳು-ವುದು
ಮುರಿಯ
ಮುರಿಯಲು
ಮುರಿ-ಯು-ವಂಥ
ಮುರಿ-ಯು-ವರು
ಮುರಿ-ಯು-ವ-ವರು
ಮುರುಕು
ಮುಳು-ಗ-ಡೆ-ಯಾ-ದಾಗ
ಮುಳುಗಲು
ಮುಳುಗಿ
ಮುಳುಗಿತು
ಮುಳುಗಿದ
ಮುಳು-ಗಿ-ದಂತೆಲ್ಲ
ಮುಳು-ಗಿ-ದರು
ಮುಳು-ಗಿ-ದ-ವನು
ಮುಳು-ಗಿ-ದಾಗ
ಮುಳುಗಿದೆ
ಮುಳು-ಗಿದ್ದರು
ಮುಳು-ಗಿದ್ದರೂ
ಮುಳು-ಗಿದ್ದಿ-ರ-ಬ-ಹುದು
ಮುಳುಗಿದ್ದು
ಮುಳು-ಗಿ-ಬಿ-ಡುವ
ಮುಳು-ಗಿ-ರುತ್ತಾ-ನೆ-ಆ-ದರೆ
ಮುಳು-ಗಿ-ರುತ್ತಿದ್ದರು
ಮುಳು-ಗಿ-ಸುತ್ತಿದ್ದೀಯೆ
ಮುಳು-ಗಿ-ಹೋ-ಗಿದ್ದರು
ಮುಳು-ಗುತ್ತಾನೆ
ಮುಳು-ಗುತ್ತಾಳೆ
ಮುಳು-ಗುತ್ತಿ-ರುವ
ಮುಳು-ಗುತ್ತೇವೆ
ಮುಳುಗುವ
ಮುಳು-ಗು-ವಂತಾ-ದಾಗ
ಮುಳು-ಗು-ವಾಗ
ಮುಳುಗೆದ್ದ
ಮುಳುವಾಗಿ
ಮುಳ್ಳನ್ನು
ಮುಳ್ಳಿನ
ಮುಳ್ಳಿನಿಂದ
ಮುಳ್ಳು
ಮುವ್ವತ್ತು
ಮುಷ್ಕರ
ಮುಷ್ಕ-ರ-ಗಳ
ಮುಷ್ಕರದ
ಮುಷ್ಕ-ರ-ನಿ-ರತ
ಮುಷ್ಕ-ರ-ಹೂಡಿ
ಮುಸಲೋನಿ
ಮುಸಲ್ಮಾನ
ಮುಸಲ್ಮಾ-ನ-ರಲ್ಲೂ
ಮುಸು-ಕಿ-ನಲ್ಲಿ
ಮುಸು-ಕಿ-ರುವ
ಮುಸುಕಿವೆ
ಮುಸು-ಕುತ್ತವೆ
ಮುಸೊಲಿನಿ
ಮುಸ್ಲಿಂ
ಮೂಕ
ಮೂಕ-ನಾ-ಗುತ್ತಿದ್ದ
ಮೂಕ-ಜಂತು-ಗ-ಳಿಗೆ
ಮೂಕ-ವಿಸ್ಮಿ-ತ-ರಾಗಿ
ಮೂಕ-ವಿಸ್ಮಿ-ತ-ರಾ-ದರು
ಮೂಕ-ವೇ-ದನೆ
ಮೂಗಿನ
ಮೂಗು
ಮೂಗುದಾರ
ಮೂಟೆ
ಮೂಟೆಗಳು
ಮೂಟೆಯ
ಮೂಡ-ಣ-ಬಾ-ನಿ-ನಲ್ಲಿ
ಮೂಡ-ದಿದ್ದರೆ
ಮೂಡದೆ
ಮೂಡ-ಲಾ-ರದು
ಮೂಡಲಿ
ಮೂಡಲಿಲ್ಲ
ಮೂಡಿ
ಮೂಡಿದಾಗ
ಮೂಡಿ-ಬ-ರುತ್ತದೆ
ಮೂಡಿರ
ಮೂಡಿ-ಸಿ-ಕೊಂಡರೆ
ಮೂಡಿಸಿದ
ಮೂಡಿ-ಸಿ-ದರು
ಮೂಡಿಸುತ್ತ
ಮೂಡಿ-ಸುತ್ತವೆ
ಮೂಡಿಸುವ
ಮೂಡೀತು
ಮೂಡುತ್ತದೆ
ಮೂಡುತ್ತವೆ
ಮೂಡುತ್ತಿದೆ
ಮೂಡುವ
ಮೂಡುವುದು
ಮೂಢ
ಮೂಢ-ನಂಬಿ-ಕೆ-ಗ-ಳನ್ನು-ಳಿದು
ಮೂಢನಂತೆ
ಮೂಢ-ನಂಬಿಕೆ
ಮೂಢ-ನಂಬಿ-ಕೆ-ಗಳ
ಮೂಢ-ನಂಬಿ-ಕೆ-ಗ-ಳನ್ನು
ಮೂಢ-ನಂಬಿ-ಕೆ-ಗ-ಳನ್ನೇ
ಮೂಢ-ನಂಬಿ-ಕೆ-ಗ-ಳಿಂದ
ಮೂಢ-ನಂಬಿ-ಕೆ-ಗ-ಳಿಗೆ
ಮೂಢ-ನಂಬಿ-ಕೆ-ಗಳು
ಮೂಢ-ನಂಬಿ-ಕೆ-ಗಿಂತಲೂ
ಮೂಢ-ನಂಬಿ-ಕೆಯ
ಮೂಢ-ನಂಬಿ-ಕೆ-ಯನ್ನು
ಮೂಢ-ನಂಬಿ-ಕೆ-ಯನ್ನೂ
ಮೂಢ-ನಂಬಿ-ಕೆ-ಯಲ್ಲ
ಮೂಢ-ನಂಬಿ-ಕೆ-ಯಲ್ಲದೆ
ಮೂಢ-ನಂಬಿ-ಕೆ-ಯೆಂದು
ಮೂಢ-ನಂಬಿ-ಕೆಯೇ
ಮೂಢ-ನಾ-ಗುತ್ತಾನೆ
ಮೂಢನೆಂದು
ಮೂಢರೂ
ಮೂಢ-ವಲ್ಲದ
ಮೂತಿಗಳ
ಮೂತ್ರ
ಮೂತ್ರದ
ಮೂರನೆ
ಮೂರನೆಯ
ಮೂರ-ನೆ-ಯದೆ
ಮೂರ-ನೆ-ಯದೇ
ಮೂರ-ನೆ-ಯ-ವನು
ಮೂರನೇ
ಮೂರರಲ್ಲಿ
ಮೂರು
ಮೂರು-ಚಿಂತೆಯ
ಮೂರುತಾವು
ಮೂರು-ದಿ-ನ-ಗಳ
ಮೂರು-ಮಂದಿಯ
ಮೂರುವರೆ
ಮೂರು-ಸಾ-ವಿರ
ಮೂರೂವರೆ
ಮೂರೂ-ವ-ರೆ-ಲೀ-ಟರ್
ಮೂರೇ
ಮೂರ್ಖ
ಮೂರ್ಖ-ನಾ-ಗಿ-ರುತ್ತಾನೆ
ಮೂರ್ಖ-ಕೆ-ಲಸ
ಮೂರ್ಖತನ
ಮೂರ್ಖ-ತ-ನ-ವಲ್ಲವೇ
ಮೂರ್ಖ-ತ-ನವೇ
ಮೂರ್ಖತೆ
ಮೂರ್ಖತೆಯ
ಮೂರ್ಖ-ತೆ-ಯನ್ನು
ಮೂರ್ಖನೂ
ಮೂರ್ಛೆ
ಮೂರ್ತ-ರೂ-ಪ-ವಾಗಿ
ಮೂರ್ತಿ
ಮೂರ್ತಿ-ಗ-ಳನ್ನು
ಮೂರ್ತಿಗಳು
ಮೂರ್ತಿಪೂಜಾ
ಮೂರ್ತಿ-ಪೂ-ಜೆಯ
ಮೂರ್ತಿ-ಭಂಜನೆ
ಮೂರ್ತಿಯ
ಮೂರ್ತಿಯನ್ನು
ಮೂರ್ತಿಯಲ್ಲಿ
ಮೂರ್ತಿ-ಯೊಂದನ್ನು
ಮೂರ್ತಿವಂತ
ಮೂರ್ತಿ-ವತ್ತಾದ
ಮೂರ್ತಿ-ವೆತ್ತಂತಿದ್ದರೆ
ಮೂಲ
ಮೂಲದ್ರವ್ಯ-ಗ-ಳಲ್ಲವೇ
ಮೂಲಕ
ಮೂಲ-ಕ-ವಲ್ಲವೇ
ಮೂಲಕವೂ
ಮೂಲಕವೆ
ಮೂಲಕವೇ
ಮೂಲ-ಕಾ-ರಣ
ಮೂಲ-ಕಾ-ರ-ಣ-ಗ-ಳಾದ
ಮೂಲ-ಕಾ-ರ-ಣ-ವನ್ನು
ಮೂಲ-ಕಾ-ರ-ಣ-ವೇ-ನಿ-ರ-ಬ-ಹುದು
ಮೂಲಕ್ಕೆ
ಮೂಲಕ್ಕೇ
ಮೂಲಗಳ
ಮೂಲ-ಗ-ಳಿವೆ
ಮೂಲ-ಘ-ಟ-ಕ-ಗ-ಳಾದ
ಮೂಲ-ಘ-ಟ-ಕ-ಗಳೂ
ಮೂಲತಃ
ಮೂಲ-ತತ್ತ್ವ-ಗ-ಳಾ-ಗಲಿ
ಮೂಲ-ತತ್ತ್ವದ
ಮೂಲ-ತತ್ತ್ವ-ವನ್ನು
ಮೂಲ-ತತ್ತ್ವ-ವನ್ನೂ
ಮೂಲ-ತತ್ವ-ಗಳ
ಮೂಲ-ತತ್ವ-ಗಳು
ಮೂಲದ
ಮೂಲದಲ್ಲಿ
ಮೂಲ-ದ-ವು-ಗ-ಳೆಂಬು-ದನ್ನು
ಮೂಲದಿಂದ
ಮೂಲ-ದಿಂದಲೇ
ಮೂಲದ್ರವ್ಯ-ಗ-ಳಿಂದ
ಮೂಲದ್ರವ್ಯ-ಗಳು
ಮೂಲಧಾತು
ಮೂಲ-ನಿ-ವಾ-ಸಿ-ಗಳ
ಮೂಲ-ನಿ-ವಾ-ಸಿ-ಗ-ಳದ್ದಲ್ಲ
ಮೂಲ-ನಿ-ವಾ-ಸಿ-ಗ-ಳಾದ
ಮೂಲ-ಭಾ-ವ-ನೆ-ಗಳು
ಮೂಲ-ಭಾ-ವ-ನೆಗೆ
ಮೂಲಭೂತ
ಮೂಲಮಂತ್ರ
ಮೂಲ-ಮಂತ್ರ-ವಾ-ಯಿತು
ಮೂಲವನ್ನು
ಮೂಲವನ್ನೂ
ಮೂಲ-ವಾ-ಗುತ್ತದೆ
ಮೂಲವಾದ
ಮೂಲ-ವಾ-ದರೋ
ಮೂಲವಿದೆ
ಮೂಲ-ವಿ-ರು-ವುದು
ಮೂಲ-ವಿ-ರು-ವುದೂ
ಮೂಲವೆಂದು
ಮೂಲವೆಲ್ಲಿ
ಮೂಲವೇನು
ಮೂಲಸೂತ್ರ
ಮೂಲ-ಸೂತ್ರ-ವಿಲ್ಲಿ-ದೆ-ಯಲ್ಲವೇ
ಮೂಲಸ್ರೋತ
ಮೂಲಸ್ರೋ-ತ-ದೆ-ಡೆಗೇ
ಮೂಲಸ್ರೋ-ತ-ವಾದ
ಮೂಲಸ್ರೋ-ತ-ವಿದೆ
ಮೂಲಸ್ವ-ಭಾ-ವದ
ಮೂಲಸ್ವ-ರೂಪ
ಮೂಲಾ-ಧಾ-ರ-ದಲ್ಲಿ
ಮೂಲೆ
ಮೂಲೆಗೆ
ಮೂಲೆ-ಯಲ್ಲಲ್ಲ
ಮೂಲೆಯಲ್ಲಿ
ಮೂಲೆಯಲ್ಲೇ
ಮೂಲೆ-ಯೊಂದ-ರಲ್ಲಿ
ಮೂಳೆ
ಮೂಳೆ-ಗ-ಳನ್ನು
ಮೂವ
ಮೂವತ್ತ-ನಾಲ್ಕು
ಮೂವತ್ತ-ಮೂರು
ಮೂವತ್ತು
ಮೂವತ್ತೇಳು
ಮೂವತ್ತೈದು
ಮೂವರ
ಮೂಸಂಬಿ
ಮೂಸಿದಾಗ
ಮೂಸು-ವು-ದರ
ಮೂಸೆಯಲ್ಲಿ
ಮೃತರಾದ
ಮೃತನ
ಮೃತನಾದ
ಮೃತ-ನಾ-ದುದೇ
ಮೃತಪಟ್ಟ
ಮೃತವ್ಯಕ್ತಿ-ಯನ್ನು
ಮೃತ್ಯುಂಜಯ
ಮೃತ್ಯು-ದೇ-ವ-ತೆ-ಯಾದ
ಮೃತ್ಯು-ಪಾ-ಶ-ವನ್ನು
ಮೃತ್ಯು-ಮು-ಖ-ದಲ್ಲೂ
ಮೃತ್ಯುವನ್ನು
ಮೃತ್ಯು-ವ-ಶ-ನಾ-ದಾಗ
ಮೃತ್ಯು-ಶಯ್ಯೆಯ
ಮೃತ್ಯುಸ್ಥಿ-ತಿ-ಯಿಂದ
ಮೃತ್ಯುಹೀನ
ಮೃತ್ಯೋರ್ಮಾ
ಮೃದು
ಮೃದುಲ
ಮೃದುಲಳ
ಮೃದು-ಲ-ಳನ್ನು
ಮೃದುಲಾ
ಮೃದು-ವಾ-ಗಿಯೂ
ಮೃದುಸ್ಪರ್ಶ-ಗಳ
ಮೃದುಸ್ವ-ಭಾ-ವ-ದ-ವರು
ಮೃಷ್ಟಾನ್ನದ
ಮೆಂಟಲ್
ಮೆಂಬರಿಕೆ
ಮೆಕಾಲೆ
ಮೆಗಾಸ್ತ
ಮೆಚ್ಚ-ದಿದ್ದುದು
ಮೆಚ್ಚ-ಲಾ-ರರು
ಮೆಚ್ಚಲು
ಮೆಚ್ಚಿ
ಮೆಚ್ಚಿಕೆ
ಮೆಚ್ಚಿ-ಕೊಂಡಿದ್ದ
ಮೆಚ್ಚಿ-ಕೊಳ್ಳ-ತೊ-ಡ-ಗಿದ್ದರು
ಮೆಚ್ಚಿಗೆಯ
ಮೆಚ್ಚಿ-ಗೆ-ಯನ್ನು
ಮೆಚ್ಚಿದರೆ
ಮೆಚ್ಚಿ-ದ-ವ-ಳನ್ನು
ಮೆಚ್ಚಿದಾಗ
ಮೆಚ್ಚಿ-ದಾ-ಗಲೂ
ಮೆಚ್ಚಿ-ರ-ದಿದ್ದರೂ
ಮೆಚ್ಚಿಸುವ
ಮೆಚ್ಚು
ಮೆಚ್ಚುಗೆ
ಮೆಚ್ಚು-ಗೆ-ಗ-ಳಿ-ಸಿದ
ಮೆಚ್ಚು-ಗೆ-ಗಾಗಿ
ಮೆಚ್ಚುಗೆಯ
ಮೆಚ್ಚು-ಗೆ-ಯನ್ನು
ಮೆಚ್ಚುತ್ತಿದ್ದೇ-ನೆಂದು
ಮೆಚ್ಚುತ್ತಿಲ್ಲ
ಮೆಚ್ಚುತ್ತೇನೆ
ಮೆಚ್ಚು-ವಂತಿಲ್ಲ
ಮೆಚ್ಚು-ವು-ದೆಂದರೆ
ಮೆಟರ್ಲಿಂಕ್
ಮೆಟ್ಟಲನ್ನು
ಮೆಟ್ಟಲಾಗಿ
ಮೆಟ್ಟಲಿನ
ಮೆಟ್ಟಲು
ಮೆಟ್ಟ-ಲು-ಗ-ಳನ್ನು
ಮೆಟ್ಟ-ಲು-ಗ-ಳಲ್ಲಿ-ರು-ವ-ವ-ರಿಗೆ
ಮೆಟ್ಟಿ
ಮೆಟ್ಟಿಕೊಂಡೇ
ಮೆಟ್ಟಿನಿಂತು
ಮೆಟ್ಟಿಯೇ
ಮೆಟ್ಟಿ-ಲು-ಗ-ಳಾಗಿ
ಮೆಡಿಕಲ್
ಮೆಡಿ-ಟ-ರೇ-ನಿ-ಯನ್
ಮೆಡಿ-ಸನ್ನಲ್ಲಿ
ಮೆತ್ತಗೆ
ಮೆದು-ಳಿ-ಗಿಂತ
ಮೆದುಳಿಗೆ
ಮೆದುಳಿನ
ಮೆನ್ನಿಂಗರ್
ಮೆರ-ವ-ಣಿ-ಗೆ-ಯಲ್ಲಿ
ಮೆರುಗು
ಮೆರೆದಾಡಿ
ಮೆರೆ-ಯು-ವ-ವರು
ಮೆರೆಸಲು
ಮೆರೆಸಿದ
ಮೆರೆಸಿದ್ದ
ಮೆರೆಸು
ಮೆಲಕು
ಮೆಲು-ಕಾ-ಡ-ಬ-ಹುದು
ಮೆಲು-ಕಾ-ಡುತ್ತ
ಮೆಲು-ಕಾ-ಡುತ್ತಿದ್ದರೆ
ಮೆಲು-ಕಾ-ಡು-ವು-ದುಂಟು
ಮೆಲುಕು
ಮೆಲೆ
ಮೆಲ್ಲ
ಮೆಲ್ಲನೆ
ಮೆಲ್ಲನೇ
ಮೆಲ್ಲ-ಮೆಲ್ಲನೆ
ಮೆಲ್ಲ-ಮೆಲ್ಲ-ನೆ-ಯಿಂದಲೇ
ಮೆಲ್ಲ-ಮೆಲ್ಲನೇ
ಮೆಸ್ಮರನ
ಮೆಸ್ಸಿಂಗನ
ಮೆಸ್ಸಿಂಗ-ನನ್ನು
ಮೆಸ್ಸಿಂಗ್
ಮೇ
ಮೇಕೆಯನ್ನು
ಮೇಜನ್ನು
ಮೇಜಿನ
ಮೇಜು
ಮೇಡು-ಗ-ಳಲ್ಲಿ
ಮೇಣದ
ಮೇದು
ಮೇಧಾವಿ
ಮೇಧಾ-ವಿ-ಗ-ಳಾ-ಗ-ದಿದ್ದರೂ
ಮೇಧಾ-ವಿ-ಗ-ಳಾ-ಗಿ-ರ-ಲಿಲ್ಲ
ಮೇಧಾ-ವಿ-ಗ-ಳಾದ
ಮೇಧಾ-ವಿ-ಗ-ಳಿಗೆ
ಮೇಧಾ-ವಿ-ಗಳು
ಮೇಧಾ-ವಿ-ಗಳೂ
ಮೇಧಾ-ವಿ-ಯನ್ನಾಗಿ
ಮೇಧಾ-ವಿ-ಯಾ-ಗಿ-ರ-ಲಿಲ್ಲ
ಮೇಯುತ್ತಿದ್ದಾಗ
ಮೇಯೋ
ಮೇರು
ಮೇರೆ
ಮೇರೆಗೆ
ಮೇರೆಯನ್ನು
ಮೇಲಕ್ಕೆ
ಮೇಲಕ್ಕೆತ್ತ-ಬಲ್ಲದು
ಮೇಲಕ್ಕೆತ್ತ-ಬೇ-ಕಾ-ದರೆ
ಮೇಲಕ್ಕೆತ್ತಲು
ಮೇಲಕ್ಕೆತ್ತಿ
ಮೇಲಕ್ಕೆತ್ತಿತು
ಮೇಲಕ್ಕೆದ್ದೆ
ಮೇಲಕ್ಕೆ-ಳೆ-ದಂತೆ
ಮೇಲಕ್ಕೆ-ಳೆ-ದಾಗ
ಮೇಲಕ್ಕೆ-ಸೆ-ಯುತ್ತಲೂ
ಮೇಲಕ್ಕೇ-ರಲು
ಮೇಲಕ್ಕೇರಿ
ಮೇಲಕ್ಕೇ-ರಿದ
ಮೇಲಕ್ಕೇ-ರಿ-ಸುವ
ಮೇಲಕ್ಕೇ-ರುತ್ತ
ಮೇಲಕ್ಕೇ-ಳ-ಲಾ-ರದೆ
ಮೇಲಕ್ಕೇ-ಳು-ವಂತೆ
ಮೇಲಕ್ಕೊಯ್ಯ-ಬೇಕು
ಮೇಲಣ
ಮೇಲ-ಧಿ-ಕಾರಿ
ಮೇಲ-ಧಿ-ಕಾ-ರಿ-ಗಳ
ಮೇಲ-ಧಿ-ಕಾ-ರಿ-ಗ-ಳಿಗೆ
ಮೇಲ-ಧಿ-ಕಾ-ರಿ-ಗಳು
ಮೇಲ-ಧಿ-ಕಾ-ರಿಗೆ
ಮೇಲರಿಮೆ
ಮೇಲಲ್ಲವೆ
ಮೇಲಾಗಿ
ಮೇಲಾಗುವ
ಮೇಲಾ-ಗು-ವುದು
ಮೇಲಿನಂತೆ
ಮೇಲಿಟ್ಟು-ಕೊಂಡಾಗ
ಮೇಲಿತ್ತು
ಮೇಲಿದ್ದ
ಮೇಲಿದ್ದಾಗ
ಮೇಲಿನ
ಮೇಲಿನಂತೆ
ಮೇಲಿ-ನ-ವನೆ
ಮೇಲಿ-ನ-ವರ
ಮೇಲಿ-ನ-ವರು
ಮೇಲಿನಿಂದ
ಮೇಲಿರಿಸಿ
ಮೇಲಿ-ರಿ-ಸಿದ್ದ
ಮೇಲಿರುವ
ಮೇಲಿ-ರು-ವ-ವರ
ಮೇಲಿ-ರು-ವ-ವ-ರನ್ನು
ಮೇಲಿಲ್ಲ-ವೆಂದು
ಮೇಲು
ಮೇಲು-ವರ್ಗ-ದ-ವರೇ
ಮೇಲುಕೀಳು
ಮೇಲು-ಕೀ-ಳೆನ್ನದೇ
ಮೇಲುಗೈ
ಮೇಲು-ವರ್ಗ-ದ-ವ-ರೆ-ನಿ-ಸಿ-ಕೊಂಡ-ವರು
ಮೇಲು-ವರ್ಗ-ದ-ವ-ರೆನ್ನಿಸಿ
ಮೇಲುಸಿರು
ಮೇಲೂ
ಮೇಲೆ
ಮೇಲೆಕ್ಕೆ-ಸೆದು
ಮೇಲೆತ್ತ
ಮೇಲೆತ್ತಲು
ಮೇಲೆತ್ತಿ
ಮೇಲೆತ್ತಿ-ಕೊಳ್ಳುವ
ಮೇಲೆತ್ತುವ
ಮೇಲೆತ್ತು-ವಂತೆ
ಮೇಲೆದ್ದು
ಮೇಲೆಮೇಲೆ
ಮೇಲೆಯೂ
ಮೇಲೆಯೇ
ಮೇಲೆ-ರ-ಗ-ಬ-ಹು-ದೆಂದು
ಮೇಲೆ-ಳೆ-ಯು-ವಂತೆ
ಮೇಲೇ
ಮೇಲೇಕೆ
ಮೇಲೇ-ನಾ-ಗುತ್ತದೆ
ಮೇಲೇರ
ಮೇಲೇರದ
ಮೇಲೇ-ರ-ಬಲ್ಲೆ-ವೇನು
ಮೇಲೇ-ರ-ಬೇ-ಕಾ-ದರೆ
ಮೇಲೇ-ರ-ಬೇಕು
ಮೇಲೇ-ರ-ಬೇ-ಕೆಂಬ
ಮೇಲೇರಲು
ಮೇಲೇ-ರ-ಲೇ-ಬೇಕು
ಮೇಲೇರಿ
ಮೇಲೇ-ರಿದ್ದಾನೆ
ಮೇಲೇರುತ್ತ
ಮೇಲೇರುವ
ಮೇಲೇ-ರು-ವಂತೆ
ಮೇಲೇ-ಳ-ದಂತೆ
ಮೇಲೇ-ಳ-ಬ-ಹು-ದಲ್ಲವೆ
ಮೇಲೇ-ಳ-ಬೇ-ಕೆಂಬ
ಮೇಲೇಳಲು
ಮೇಲೇಳಲೇ
ಮೇಲೇ-ಳುತ್ತದೆ
ಮೇಲೇ-ಳುತ್ತ-ಲಿ-ರು-ವಾ-ಗಲೇ
ಮೇಲೇ-ಳು-ವಂತಾ-ಗ-ಬೇಕು
ಮೇಲೇ-ಳು-ವಂಥ
ಮೇಲೊಂದ-ರಂತೆ
ಮೇಲೊಂದು
ಮೇಲ್ಕಂಡ
ಮೇಲ್ಜಾ-ತಿ-ಯ-ವ-ರಾಗಿ
ಮೇಲ್ತನಿಖೆ
ಮೇಲ್ಭಾಗಕ್ಕೆ
ಮೇಲ್ಭಾ-ಗ-ದಲ್ಲಿ
ಮೇಲ್ಮಟ್ಟ-ದಲ್ಲಿದ್ದೇನೆ
ಮೇಲ್ಮಟ್ಟದ್ದು
ಮೇಲ್ಮೆಗಾಗಿ
ಮೇಲ್ಮೆಗೆ
ಮೇಲ್ಮೆಯನ್ನು
ಮೇಲ್ವಿ-ಚಾ-ರ-ಕ-ರಿಂದ
ಮೇಲ್ವಿ-ಚಾ-ರಣೆ
ಮೇಲ್ವಿ-ಚಾ-ರ-ಣೆ-ಯನ್ನು
ಮೇವಾದ್ವಿ-ತೀಯ
ಮೇಷ್ಟರಿಗೆ
ಮೇಷ್ಟರು
ಮೇಸ್ತ್ರಿ
ಮೈ
ಮೈಕೇಲ್
ಮೈಕೈ
ಮೈಕೊಡಹಿ
ಮೈಕ್ರೋಸ್ಕೋ-ಪನ್ನು
ಮೈಕ್ರೋಸ್ಕೋಪ್
ಮೈಗಳ್ಳ
ಮೈಗಳ್ಳರೂ
ಮೈಗೂಡಲು
ಮೈಗೂಡಿಸಿ
ಮೈಗೂ-ಡಿ-ಸಿ-ಕೊಂಡಲ್ಲಿ
ಮೈಗೂ-ಡಿ-ಸಿ-ಕೊಂಡಿ-ರುತ್ತಾರೆ
ಮೈಗೂ-ಡಿ-ಸಿ-ಕೊಂಡ
ಮೈಗೂ-ಡಿ-ಸಿ-ಕೊಂಡು
ಮೈಗೂ-ಡು-ವುದು
ಮೈಗೆ
ಮೈತ್ರೇಯ
ಮೈತ್ರೇಯರು
ಮೈದ-ಳೆ-ದಿ-ರುತ್ತವೆ
ಮೈದ-ಳೆ-ಯುತ್ತಿದೆ
ಮೈದಾ-ನ-ದಲ್ಲಿ
ಮೈದೋರಿ
ಮೈದೋ-ರುತ್ತಾನೆ
ಮೈಬಣ್ಣ-ಇ-ವು-ಗ-ಳೆಲ್ಲ
ಮೈಮರೆತ
ಮೈಮರೆತು
ಮೈಮ-ರೆ-ಯುತ್ತಾರೆ
ಮೈಮ-ರೆ-ಯುತ್ತಾಳೆ
ಮೈಮ-ರೆ-ಯುತ್ತೇನೆ
ಮೈಮ-ರೆ-ಯು-ವಂತೆ
ಮೈಮೇಲೆ
ಮೈಮೇಲೇರಿ
ಮೈಯಲ್ಲಿ
ಮೈಲಿ
ಮೈಲಿಗಳ
ಮೈಲಿ-ಗೆ-ಯಾ-ಯಿತೊ
ಮೈಲಿನ
ಮೈಲಿಯ
ಮೈಲು
ಮೈಲುಗಲ್ಲು
ಮೈಲುಗಳ
ಮೈಲು-ವೇ-ಗ-ದಿಂದ
ಮೈವಡೆದ
ಮೈಸೂರಿನ
ಮೈಸೂರು
ಮೊಂಟೇಗ್
ಮೊಂಡುವಾದ
ಮೊಟ-ಕು-ಗೊ-ಳಿ-ಸ-ದಂತೆ
ಮೊಟ-ಕು-ಗೊ-ಳಿ-ಸಲು
ಮೊಟ-ಕು-ಗೊ-ಳಿ-ಸಿ-ಕೊಂಡಂತೆ
ಮೊಟ-ಕು-ಗೊ-ಳಿ-ಸುತ್ತದೆ
ಮೊಟ-ಕು-ಗೊ-ಳಿ-ಸುವ
ಮೊಟ-ಕು-ಗೊ-ಳಿ-ಸು-ವುದು
ಮೊಟ್ಟ
ಮೊತ್ತ
ಮೊತ್ತದಿಂದ
ಮೊತ್ತ-ಮೊ-ದಲು
ಮೊತ್ತವನ್ನು
ಮೊತ್ತವಲ್ಲ
ಮೊತ್ತವೇ
ಮೊದ
ಮೊದ-ಮೊ-ದಲು
ಮೊದಲ
ಮೊದಲಂತೂ
ಮೊದಲನೆ
ಮೊದ-ಲ-ನೆಯ
ಮೊದ-ಲ-ನೆ-ಯ-ದಾಗಿ
ಮೊದ-ಲ-ನೆ-ಯದು
ಮೊದ-ಲ-ನೆ-ಯದೇ
ಮೊದ-ಲ-ನೆ-ಯ-ವನು
ಮೊದಲನೇ
ಮೊದಲಾಗಿ
ಮೊದಲಾದ
ಮೊದ-ಲಾ-ದ-ವರು
ಮೊದ-ಲಾ-ದ-ವ-ರೊ-ಡನೆ
ಮೊದ-ಲಾ-ದ-ವು-ಗ-ಳಲ್ಲಿ
ಮೊದ-ಲಾ-ದ-ವು-ಗ-ಳಿಗೆ
ಮೊದ-ಲಾ-ದ-ವು-ಗಳು
ಮೊದ-ಲಾ-ದು-ವು-ಗ-ಳನ್ನು
ಮೊದಲಿಗೆ
ಮೊದಲಿನ
ಮೊದ-ಲಿ-ನಂತಾ-ಗಲಿ
ಮೊದ-ಲಿ-ನಷ್ಟೆ
ಮೊದ-ಲಿ-ನಿಂದ
ಮೊದ-ಲಿ-ನಿಂದಲೂ
ಮೊದ-ಲಿ-ನಿಂದಲೇ
ಮೊದಲಿಲ್ಲ
ಮೊದಲು
ಮೊದಲೆ
ಮೊದಲೇ
ಮೊನಚಾದ
ಮೊನೆಯಷ್ಟು
ಮೊನೆಯಿಂದ
ಮೊನ್ನೆ
ಮೊಮ್ಮಕ್ಕಳ
ಮೊಮ್ಮಕ್ಕ-ಳನ್ನು
ಮೊಮ್ಮಗಳು
ಮೊರಿಸನ್
ಮೊರೆ
ಮೊರೆಗೆ
ಮೊರೆದಲ್ಲಿ
ಮೊರೆದಾಗ
ಮೊರೆಯ
ಮೊರೆ-ಯ-ಬ-ಹುದು
ಮೊರೆಯಲು
ಮೊರೆಯಿಟ್ಟು
ಮೊರೆ-ಯಿ-ಡುತ್ತಿದ್ದೆ
ಮೊರೆ-ಯುತ್ತಾನೆ
ಮೊರೆ-ಯು-ವೆನು
ಮೊರೆಯೆ
ಮೊರೆಹೊಕ್ಕು
ಮೊಲೆ-ಯೂ-ಡಿ-ಸದ
ಮೊಲೆ-ಹಾ-ಲನ್ನು
ಮೊಲೆಹಾಲು
ಮೊಳ-ಕೆ-ಯೊ-ಡೆ-ದೀತೇ
ಮೊಳಗಿ
ಮೊಳಗಿತು
ಮೊಳ-ಗು-ವಂತೆ
ಮೊಳೆ-ಗ-ಳನ್ನು
ಮೊಳೆಗಳು
ಮೊಳೆತು
ಮೊಳೆ-ಯಿ-ಸಿತು
ಮೊಹಂತಿ
ಮೊಹಂತಿಗೆ
ಮೋಂಟ್ರೀಲಿನ
ಮೋಕ್ಷಕ್ಕೂ
ಮೋಕ್ಷ-ವಾ-ಗಿಲ್ಲ
ಮೋಕ್ಷಾಕಾಂಕ್ಷೆ
ಮೋಜಿನ
ಮೋಜೆ-ನಿ-ಸುತ್ತದೆ
ಮೋಟರ್
ಮೋಟಾ
ಮೋಟಾರಡಿ
ಮೋಟಾರ್
ಮೋಡದ
ಮೋಡವಾಗಿ
ಮೋಡವು
ಮೋಡಿ
ಮೋಡಿಗೆ
ಮೋಡಿಯಿಂದ
ಮೋಡಿ-ಯಿಂದಾ-ಗಲೀ
ಮೋಸ
ಮೋಸಕ್ಕೆ
ಮೋಸಗಳು
ಮೋಸ-ಗಾ-ರ-ರೆಂದು
ಮೋಸ-ಗಾ-ರಿ-ಕೆ-ಯಾ-ಗುತ್ತದೆ
ಮೋಸ-ಗೊ-ಳಿಸ
ಮೋಸ-ಗೊ-ಳಿ-ಸ-ಬಲ್ಲದು
ಮೋಸ-ಗೊ-ಳಿ-ಸ-ಬ-ಹುದು
ಮೋಸ-ಗೊ-ಳಿ-ಸಲು
ಮೋಸ-ಗೊ-ಳಿ-ಸುವ
ಮೋಸದ
ಮೋಸದಿಂದ
ಮೋಸ-ಮಾ-ಡಿ-ಕೊಳ್ಳುವ
ಮೋಸ-ವಿ-ರ-ಬ-ಹು-ದೆಂಬ
ಮೋಸವೂ
ಮೋಹ
ಮೋಹಕ
ಮೋಹಕವೂ
ಮೋಹವನ್ನು
ಮೋಹ-ವಿಲ್ಲದ
ಮೋಹವೂ
ಮೋಹಿ-ತ-ರಾಗಿ
ಮೌಢ್ಯ
ಮೌಢ್ಯಗಳ
ಮೌನ
ಮೌನ-ವಾ-ಗಿಯೇ
ಮೌನ-ದಿಂದಿ-ರು-ವ-ವ-ರನ್ನೂ
ಮೌನವನ್ನು
ಮೌನವಾಗಿ
ಮೌನ-ವಾ-ಗಿದ್ದು
ಮೌನ-ವಾ-ಗಿದ್ದು-ಕೊಂಡು
ಮೌರಿಸ್
ಮೌಲಿ-ಕ-ವಾದ
ಮೌಲಿಕವೂ
ಮೌಲ್ಯ
ಮೌಲ್ಯಗಳ
ಮೌಲ್ಯ-ಗ-ಳನ್ನಾ-ದರೂ
ಮೌಲ್ಯ-ಗ-ಳನ್ನು
ಮೌಲ್ಯ-ಗ-ಳನ್ನೂ
ಮೌಲ್ಯ-ಗ-ಳನ್ನೆಲ್ಲ
ಮೌಲ್ಯ-ಗ-ಳಲ್ಲಿ
ಮೌಲ್ಯ-ಗ-ಳಿಗೂ
ಮೌಲ್ಯ-ಗ-ಳಿಗೆ
ಮೌಲ್ಯಗಳು
ಮೌಲ್ಯದ
ಮೌಲ್ಯಯುತ
ಮ್ಯಾಕ್ಸ್ಮುಲ್ಲರ್
ಮ್ಯಾಕ್ಸ್ವೆಲ್
ಮ್ಯಾಗಜಿನ್
ಮ್ಯಾಜಿಕ್
ಮ್ಯಾಜಿಕ್ಕಣ್ಣು-ಕಟ್ಟುವ
ಮ್ಯಾನೇ-ಜ-ರ-ನಾ-ಗಿದ್ದು
ಮ್ಯಾನೇ-ಜ-ರ-ನೊ-ಡನೆ
ಮ್ಯಾನೇ-ಜ-ರಿಗೆ
ಮ್ಯಾನೇಜರ್
ಮ್ಯಾರಿನರ್
ಯಂಗ-ಡಿ-ಯಿಂದ
ಯಂತ್ರ
ಯಂತ್ರಗಳ
ಯಂತ್ರ-ಗ-ಳನ್ನು
ಯಂತ್ರದ
ಯಂತ್ರದಂತೆ
ಯಂತ್ರದಲ್ಲೂ
ಯಂತ್ರದಿಂದ
ಯಂತ್ರ-ಮಂತ್ರ-ಗ-ಳಿಂದಾ-ದರೂ
ಯಂತ್ರವತ್
ಯಂತ್ರವನ್ನು
ಯಂತ್ರ-ವಾ-ಗುವ
ಯಂತ್ರವಾದ
ಯಂತ್ರ-ವಾ-ದರು
ಯಂತ್ರ-ಶಾಸ್ತ್ರದ
ಯಂತ್ರಾ-ಗಾ-ರ-ದಿಂದಲೂ
ಯಃಕಶ್ಚಿತ್
ಯಕೃತ್
ಯಕ್ಷಗಾನ
ಯಕ್ಷಪ್ರಶ್ನೆ-ಯಾ-ಗಿತ್ತು
ಯಕ್ಷಿಣಿ
ಯಕ್ಷಿ-ಣಿ-ಗಾರ
ಯಕ್ಷಿ-ಣಿ-ಗಾ-ರನ
ಯಜಮಾನ
ಯಜ-ಮಾ-ನನ
ಯಜ-ಮಾ-ನ-ನದ್ದಾ-ದರೆ
ಯಜ-ಮಾ-ನ-ನನ್ನು
ಯಜ-ಮಾ-ನ-ನಿ-ಗಲ್ಲವೇ
ಯಜ-ಮಾ-ನ-ನಿಗೆ
ಯಜ-ಮಾ-ನನೂ
ಯಜ-ಮಾ-ನರ
ಯಜ-ಮಾ-ನ-ರಿಗೆ
ಯಜ-ಮಾ-ನರು
ಯತಿ-ಯ-ದಾ-ಗಲಿ
ಯತಿ-ವರ್ಯರು
ಯತ್
ಯತ್ಕ-ರೋತ್ಯ-ಶುಭಂ
ಯತ್ನ
ಯತ್ನಕ್ಕೆ
ಯತ್ನದಲ್ಲಿ
ಯತ್ನ-ಮಾ-ಡ-ದಿ-ರು-ವುದು
ಯತ್ನವೂ
ಯತ್ನವೇ
ಯತ್ನಿ
ಯತ್ನಿಸ
ಯತ್ನಿ-ಸ-ತೊ-ಡ-ಗುತ್ತಾನೆ
ಯತ್ನಿ-ಸ-ದಿದ್ದರೆ
ಯತ್ನಿ-ಸ-ದಿ-ರು-ವುದು
ಯತ್ನಿಸದೆ
ಯತ್ನಿಸದೇ
ಯತ್ನಿ-ಸ-ಬೇಕು
ಯತ್ನಿ-ಸ-ಲಾ-ಗಿದೆ
ಯತ್ನಿಸಲಿ
ಯತ್ನಿ-ಸ-ಲಿಲ್ಲ
ಯತ್ನಿಸಲು
ಯತ್ನಿಸಿ
ಯತ್ನಿಸಿದ
ಯತ್ನಿ-ಸಿ-ದಂತಾ-ಗದೆ
ಯತ್ನಿ-ಸಿ-ದರು
ಯತ್ನಿ-ಸಿ-ದರೂ
ಯತ್ನಿ-ಸಿ-ದರೆ
ಯತ್ನಿ-ಸಿ-ದಳು
ಯತ್ನಿ-ಸಿ-ದ-ವರ
ಯತ್ನಿ-ಸಿ-ದ-ವ-ರಲ್ಲಿ
ಯತ್ನಿ-ಸಿ-ದಾಗ
ಯತ್ನಿ-ಸಿ-ದುದೇ
ಯತ್ನಿಸಿದೆ
ಯತ್ನಿ-ಸಿದ್ದರೂ
ಯತ್ನಿ-ಸಿದ್ದಳು
ಯತ್ನಿ-ಸಿದ್ದೇನೆ
ಯತ್ನಿಸು
ಯತ್ನಿಸುತ್ತ
ಯತ್ನಿ-ಸುತ್ತ-ಲಿದ್ದಾರೆ
ಯತ್ನಿ-ಸುತ್ತಲೇ
ಯತ್ನಿ-ಸುತ್ತಾ-ನಷ್ಟೆ
ಯತ್ನಿ-ಸುತ್ತಾನೆ
ಯತ್ನಿ-ಸುತ್ತಾರೆ
ಯತ್ನಿ-ಸುತ್ತಾ-ರೆಯೆ
ಯತ್ನಿ-ಸುತ್ತಿದ್ದರು
ಯತ್ನಿ-ಸುತ್ತಿದ್ದೀರಿ
ಯತ್ನಿ-ಸುತ್ತಿದ್ದುದು
ಯತ್ನಿ-ಸುತ್ತಿ-ರುವ
ಯತ್ನಿ-ಸುತ್ತಿ-ರು-ವಂತಿದೆ
ಯತ್ನಿ-ಸುತ್ತಿಲ್ಲ-ವಲ್ಲ
ಯತ್ನಿ-ಸುತ್ತಿವೆ
ಯತ್ನಿ-ಸುತ್ತೇವೆ
ಯತ್ನಿಸುವ
ಯತ್ನಿ-ಸು-ವ-ವನ
ಯತ್ನಿ-ಸು-ವ-ವ-ರಿಲ್ಲ-ದಿಲ್ಲ
ಯತ್ನಿ-ಸು-ವ-ವ-ರೆಷ್ಟು
ಯತ್ನಿ-ಸು-ವಾಗ
ಯತ್ನಿ-ಸು-ವು-ದುಂಟು
ಯತ್ನಿಸೋಣ
ಯಥಾರ್ಥ
ಯಥಾರ್ಥತೆ
ಯಥಾರ್ಥತೆ
ಯಥಾರ್ಥ-ತೆ-ಗಿಂತಲೂ
ಯಥಾರ್ಥ-ತೆಗೆ
ಯಥಾರ್ಥ-ತೆಯ
ಯಥಾರ್ಥ-ಬಿಂಬಗ್ರಾಹಿ
ಯಥಾರ್ಥ-ವನ್ನು
ಯಥಾರ್ಥ-ವಲ್ಲ-ವೆಂದು
ಯಥಾರ್ಥ-ವಾಗಿ
ಯಥಾರ್ಥ-ವಾ-ಗಿ-ರಲು
ಯಥಾರ್ಥ-ವಾದ
ಯಥಾರ್ಥ-ವಾ-ದವು
ಯಥಾರ್ಥವೂ
ಯಥಾರ್ಥ-ವೇ-ನೆಂಬು-ದನ್ನು
ಯಥಾವತ್
ಯಥಾ-ವತ್ತಾಗಿ
ಯಥಾಸ್ಥಿತಿ
ಯಥೇಷ್ಟ-ವಾಗಿ
ಯದಾಗಿ
ಯದಿ
ಯದು
ಯದ್ವಾ
ಯನ್
ಯನ್ನಿಟ್ಟಿದ್ದಾರೆ
ಯನ್ನು
ಯನ್ನೂ
ಯನ್ನೇ
ಯಮ
ಯಮಧರ್ಮ
ಯಮನ
ಯಮನು
ಯಮ-ಭಾ-ರ-ವಾಗಿ
ಯಮ-ಯಾ-ತ-ನೆ-ಯನ್ನು
ಯಮ-ಯಾ-ತ-ನೆ-ಯುಂಟಾ-ಗುತ್ತಿತ್ತು
ಯಮರಾಯ
ಯರವಾಡ
ಯರು
ಯಲು
ಯಲ್ಲವೇ
ಯಲ್ಲಿ
ಯವರು
ಯಶಸ್ವಿ
ಯಶಸ್ವಿ-ಗ-ಳಾ-ಗ-ಬಲ್ಲೆವೆ
ಯಶಸ್ವಿ-ಗ-ಳಾ-ಗ-ಲೇ-ಬೇಕು
ಯಶಸ್ವಿ-ಗ-ಳಾ-ಗುವ
ಯಶಸ್ವಿ-ಗ-ಳಾದ
ಯಶಸ್ವಿ-ಗ-ಳಾ-ದ-ವ-ರನ್ನು
ಯಶಸ್ವಿ-ಗ-ಳಾ-ದು-ದಲ್ಲದೇ
ಯಶಸ್ವಿ-ಯಾಗ
ಯಶಸ್ವಿ-ಯಾ-ಗ-ದಿದ್ದರೂ
ಯಶಸ್ವಿ-ಯಾ-ಗದು
ಯಶಸ್ವಿ-ಯಾ-ಗ-ಬೇ-ಕಾ-ದರೆ
ಯಶಸ್ವಿ-ಯಾ-ಗ-ಬೇ-ಕೆಂಬ
ಯಶಸ್ವಿ-ಯಾ-ಗ-ಲಾ-ರೆವು
ಯಶಸ್ವಿ-ಯಾಗಿ
ಯಶಸ್ವಿ-ಯಾ-ಗಿ-ರು-ವುದು
ಯಶಸ್ವಿ-ಯಾ-ಗುವಿ
ಯಶಸ್ವಿ-ಯಾ-ಗು-ವುದು
ಯಶಸ್ವಿ-ಯಾದ
ಯಶಸ್ವಿ-ಯಾ-ದರೂ
ಯಶಸ್ವಿ-ಯಾ-ದೆವು
ಯಶಸ್ವಿ-ಯಾ-ಯಿತು
ಯಶಸ್ವಿಯೂ
ಯಶಸ್ವೀ
ಯಶಸ್ಸನ್ನು
ಯಶಸ್ಸಿ-ಗಾಗಿ
ಯಶಸ್ಸಿಗೂ
ಯಶಸ್ಸಿಗೆ
ಯಶಸ್ಸಿನ
ಯಶಸ್ಸಿ-ನತ್ತ
ಯಶಸ್ಸು
ಯಶಸ್ಸು-ಗಳ
ಯಶಸ್ಸು-ಗಳು
ಯಶಸ್ಸೂ
ಯಹೂ-ದಿ-ಗಳ
ಯಹೂ-ದಿ-ಗ-ಳಿಗೆ
ಯಾಂತ್ರಿಕ
ಯಾಂತ್ರಿಕತೆ
ಯಾಂತ್ರಿ-ಕ-ವಾದ
ಯಾಕಾಗಿ
ಯಾಕೆ
ಯಾಗ-ಬೇ-ಕಾ-ಗಿದೆ
ಯಾಗಿ
ಯಾಗಿತ್ತು
ಯಾಗಿದೆ
ಯಾಗಿದ್ದರೆ
ಯಾಗಿಯೂ
ಯಾಗು-ವು-ದುಂಟು
ಯಾಗುವೆ
ಯಾಚಿ-ಸು-ವು-ದಕ್ಕಾಗಿ
ಯಾಚಿಸದ
ಯಾಚಿ-ಸ-ಲಿಲ್ಲ
ಯಾಚಿ-ಸು-ವಂತೆ
ಯಾಜ್ಞ-ವಲ್ಕ್ಯ-ರನ್ನು
ಯಾಜ್ಞ-ವಲ್ಕ್ಯರು
ಯಾತನೆ
ಯಾತ-ನೆ-ಗ-ಳನ್ನು
ಯಾತನೆಯ
ಯಾತ-ನೆ-ಯನ್ನು
ಯಾತ್ರಿ-ಕ-ನಾಗಿ
ಯಾತ್ರಿ-ಕ-ನೊಬ್ಬ
ಯಾತ್ರಿಕರು
ಯಾತ್ರಿಕರೂ
ಯಾತ್ರೆ
ಯಾತ್ರೆಗೆ
ಯಾತ್ರೆಯನ್ನೇ
ಯಾತ್ರೆಯಲ್ಲಿ
ಯಾತ್ರೆಯೂ
ಯಾದ
ಯಾದಂತೆ
ಯಾದಂಥ-ವು-ಗಳು
ಯಾದರೂ
ಯಾದವ
ಯಾದ-ವ-ಗಿರಿ
ಯಾದವರ
ಯಾದವು
ಯಾದಾಗ
ಯಾದೀತೆ
ಯಾದು-ದ-ರಿಂದ
ಯಾನ
ಯಾನೆ
ಯಾಮದ
ಯಾಯಿತೋ
ಯಾರ
ಯಾರದೋ
ಯಾರದ್ದೇ
ಯಾರನ್ನಾ-ದರೂ
ಯಾರನ್ನು
ಯಾರನ್ನೂ
ಯಾರನ್ನೇ
ಯಾರಪ್ಪಾ
ಯಾರಯ್ಯ
ಯಾರಲ್ಲಾ-ದರೂ
ಯಾರಲ್ಲಿ
ಯಾರಾದರೂ
ಯಾರಾ-ದ-ರೊಬ್ಬರು
ಯಾರಿಂದ
ಯಾರಿಂದ-ಲಾ-ದರೂ
ಯಾರಿಂದಲೂ
ಯಾರಿಂದಲೋ
ಯಾರಿ-ಗಾ-ದರೂ
ಯಾರಿಗಿಲ್ಲ
ಯಾರಿಗೂ
ಯಾರಿಗೆ
ಯಾರು
ಯಾರೂ
ಯಾರೆಂಬು-ದನ್ನು
ಯಾರೇ
ಯಾರೇನು
ಯಾರೊಬ್ಬರೂ
ಯಾರೋ
ಯಾರ್ಯಾರ
ಯಾವ
ಯಾವತ್ತೂ
ಯಾವನು
ಯಾವನೂ
ಯಾವಾಗ
ಯಾವಾಗಲೂ
ಯಾವಾ-ಗಿ-ನಿಂದ
ಯಾವು
ಯಾವು-ದಕ್ಕಾಗಿ
ಯಾವುದಕ್ಕೂ
ಯಾವು-ದನ್ನಾ-ದರೂ
ಯಾವುದನ್ನು
ಯಾವುದನ್ನೂ
ಯಾವುದಯ್ಯಾ
ಯಾವು-ದಾ-ದರೂ
ಯಾವು-ದಾ-ದ-ರೊಂದು
ಯಾವುದು
ಯಾವುದೂ
ಯಾವುದೆಂದು
ಯಾವು-ದೆಂಬು-ದನ್ನು
ಯಾವುದೇ
ಯಾವುದೊ
ಯಾವುದೋ
ಯಾವುವು
ಯಾವೆಲ್ಲ
ಯಿಂದ
ಯಿಂದೊ-ಡ-ಗೂಡಿ
ಯಿತು
ಯು
ಯುಕ್ತ
ಯುಕ್ತ-ವಾ-ಗಿಯೋ
ಯುಕ್ತವಾದ
ಯುಕ್ತವೆಂದು
ಯುಕ್ತವೋ
ಯುಕ್ತಾಯುಕ್ತ
ಯುಕ್ತಿ
ಯುಕ್ತಿ-ಗ-ಳನ್ನೆ
ಯುಕ್ತಿ-ಗ-ಳಿಂದಲೂ
ಯುಕ್ತಿಗಳೇ
ಯುಕ್ತಿ-ತರ್ಕ-ಗ-ಳನ್ನೂ
ಯುಕ್ತಿ-ಪೂರ್ವಕ
ಯುಕ್ತಿಯು
ಯುಕ್ತಿ-ಯುಕ್ತ-ವಾಗಿ
ಯುಕ್ತಿ-ಯುಕ್ತ-ವಾದ
ಯುಕ್ತಿ-ಯುಕ್ತವೂ
ಯುಕ್ತಿ-ಸಂಗ-ತವೂ
ಯುಗ
ಯುಗಕ್ಕೆ
ಯುಗ-ಗ-ಳಷ್ಟು
ಯುಗದ
ಯುಗದತಿ
ಯುಗದಲ್ಲಿ
ಯುಗದಲ್ಲೂ
ಯುಗ-ದೊಳ್ಪಿಗೆ
ಯುತ್ತದೆ
ಯುತ್ತಿವೆ
ಯುದ್ಧ
ಯುದ್ಧಕ್ಕೆ
ಯುದ್ಧ-ಗ-ಳಲ್ಲಿ
ಯುದ್ಧಗಳು
ಯುದ್ಧದ
ಯುದ್ಧದಲ್ಲಿ
ಯುದ್ಧ-ಮಾ-ಡು-ವಾಗ
ಯುದ್ಧ-ರಂಗ-ವಾ-ಗಿದೆ
ಯುದ್ಧವನ್ನು
ಯುದ್ಧ-ವಿಲ್ಲ-ದಿ-ರು-ವುದು
ಯುದ್ಧ-ವೊಂದ-ರಲ್ಲಿ
ಯುಧಿಷ್ಠಿ-ರನ
ಯುನಿ-ಟೇ-ರಿ-ಯನ್
ಯುರೇಕಾ
ಯುರೋ-ಪಿ-ನ-ವರ
ಯುರೋಪು
ಯುವ
ಯುವಕ
ಯುವ-ಕ-ರನ್ನು
ಯುವಕನ
ಯುವ-ಕ-ನನ್ನು
ಯುವ-ಕ-ನನ್ನೂ
ಯುವ-ಕ-ನನ್ನೋ
ಯುವ-ಕ-ನಲ್ಲಿ
ಯುವ-ಕ-ನಾತ
ಯುವ-ಕ-ನಾ-ದರೊ
ಯುವ-ಕ-ನಾ-ದರೋ
ಯುವ-ಕ-ನಿಗೆ
ಯುವಕನು
ಯುವ-ಕ-ನೊಬ್ಬ
ಯುವ-ಕ-ನೊಬ್ಬನ
ಯುವ-ಕ-ನೊಬ್ಬ-ನಿಗೆ
ಯುವಕರ
ಯುವ-ಕ-ರನ್ನು
ಯುವ-ಕ-ರನ್ನೂ
ಯುವ-ಕ-ರಲ್ಲಿ
ಯುವ-ಕ-ರಿಗೆ
ಯುವ-ಕ-ರಿ-ಗೊಂದು
ಯುವಕರು
ಯುವಕರೂ
ಯುವಜನ
ಯುವ-ಜ-ನರ
ಯುವ-ಜ-ನ-ರಿಂದ
ಯುವ-ಜ-ನಾಂಗ-ದಲ್ಲಿ
ಯುವ-ತಿ-ಯ-ರಲ್ಲಿ
ಯುವ-ತಿ-ಯ-ರಿಗೆ
ಯುವ-ತಿ-ಯಾದ
ಯುವ-ಪೀ-ಳಿಗೆ
ಯುವ-ವಿದ್ಯಾರ್ಥಿ-ಗ-ಳಿಗೆ
ಯುವು-ದ-ರಿಂದ
ಯೂಂಗರ
ಯೂಂಗ್
ಯೂನಿಯನ್
ಯೂನಿ-ಯನ್ನು-ಗಳ
ಯೂರೋಪಿನ
ಯೂರೋ-ಪಿ-ನಲ್ಲಿ
ಯೂರೋಪು
ಯೆಂಬ
ಯೇಲ್
ಯೇಸುಕ್ರಿಸ್ತನ
ಯೇಸುಭಕ್ತ
ಯೇಸುವು
ಯೊಂದನ್ನು
ಯೊಂದು
ಯೊಬ್ಬ
ಯೊಬ್ಬನಿಗೆ
ಯೊಳಗಣ
ಯೋಗ
ಯೋಗದಲ್ಲಿ
ಯೋಗ-ದಿಂದಲೋ
ಯೋಗವನ್ನು
ಯೋಗ-ವಾ-ಸಿಷ್ಠದ
ಯೋಗ-ವಾ-ಸಿಷ್ಠ-ದಲ್ಲಿ
ಯೋಗವು
ಯೋಗ-ವೇ-ದಾಂತ-ಗ-ಳಾ-ಗಲಿ
ಯೋಗ-ಶಕ್ತಿ-ಯಿಂದ
ಯೋಗಶಾಸ್ತ್ರ
ಯೋಗ-ಶಾಸ್ತ್ರಕ್ಕೆ
ಯೋಗ-ಶಾಸ್ತ್ರ-ದಂಥ
ಯೋಗ-ಶಾಸ್ತ್ರ-ದಲ್ಲಿ
ಯೋಗ-ಶಾಸ್ತ್ರ-ವನ್ನು
ಯೋಗ-ಸೂತ್ರ-ಕಾ-ರ-ರಾದ
ಯೋಗ-ಸೂತ್ರ-ದಲ್ಲಿ
ಯೋಗಾ-ನಂದ-ರಿಗೆ
ಯೋಗಾ-ನಂದರು
ಯೋಗಾ-ನಂದರೂ
ಯೋಗಾಭ್ಯಾ-ಸ-ಗಳ
ಯೋಗಾಭ್ಯಾ-ಸ-ದಿಂದ
ಯೋಗಾಸನ
ಯೋಗಿ-ಗ-ಳಿಗೆ
ಯೋಗಿಯಂತೆ
ಯೋಗಿ-ಸಿ-ಕೊಳ್ಳು-ವಾಗ
ಯೋಗಿಸಿದ
ಯೋಗಿ-ಸುತ್ತಿದ್ದಂತೆ
ಯೋಗ್ಯ
ಯೋಗ್ಯತೆ
ಯೋಗ್ಯತೆಗೆ
ಯೋಗ್ಯತೆಯ
ಯೋಗ್ಯ-ತೆ-ಯನ್ನು
ಯೋಗ್ಯ-ತೆ-ಯಾ-ಗಲಿ
ಯೋಗ್ಯನಾದ
ಯೋಗ್ಯ-ರೀ-ತಿ-ಯಿಂದ
ಯೋಗ್ಯರು
ಯೋಗ್ಯ-ವಾ-ಗು-ವಂತೆ
ಯೋಗ್ಯವಾದ
ಯೋಚನಾ
ಯೋಚ-ನಾ-ಲ-ಹರಿ
ಯೋಚ-ನಾ-ವಿ-ಧಾನ
ಯೋಚ-ನಾ-ಶೀ-ಲ-ರಾದ
ಯೋಚನೆ
ಯೋಚ-ನೆ-ಗಳ
ಯೋಚ-ನೆ-ಗ-ಳನ್ನು
ಯೋಚ-ನೆ-ಗ-ಳನ್ನೂ
ಯೋಚ-ನೆ-ಗ-ಳನ್ನೆಲ್ಲ
ಯೋಚ-ನೆ-ಗ-ಳನ್ನೇ
ಯೋಚ-ನೆ-ಗ-ಳಲ್ಲಿ
ಯೋಚ-ನೆ-ಗ-ಳಿಂದ
ಯೋಚ-ನೆ-ಗ-ಳಿಂದಲೇ
ಯೋಚ-ನೆ-ಗ-ಳಿಗೆ
ಯೋಚ-ನೆ-ಗಳು
ಯೋಚ-ನೆ-ಗಳೂ
ಯೋಚ-ನೆ-ಗ-ಳೆಡೆ
ಯೋಚ-ನೆ-ಗ-ಳೆ-ಡೆಗೆ
ಯೋಚ-ನೆ-ಗ-ಳೆಲ್ಲ
ಯೋಚ-ನೆ-ಗಳೇ
ಯೋಚನೆಯ
ಯೋಚ-ನೆ-ಯನ್ನು
ಯೋಚ-ನೆ-ಯಿಂದ
ಯೋಚನೆಯು
ಯೋಚನೆಯೇ
ಯೋಚಿ-ಸ-ತೊ-ಡ-ಗಿದ
ಯೋಚಿ-ಸ-ತೊ-ಡ-ಗಿ-ದರು
ಯೋಚಿಸದ
ಯೋಚಿ-ಸ-ದಿದ್ದರೂ
ಯೋಚಿ-ಸ-ದಿದ್ದರೆ
ಯೋಚಿಸದೆ
ಯೋಚಿಸದೇ
ಯೋಚಿ-ಸ-ಬಲ್ಲನೇ
ಯೋಚಿ-ಸ-ಬಲ್ಲರು
ಯೋಚಿ-ಸ-ಬಲ್ಲರೇ
ಯೋಚಿ-ಸ-ಬ-ಹುದು
ಯೋಚಿ-ಸ-ಬೇ-ಕಾದ
ಯೋಚಿ-ಸ-ಬೇ-ಡವೇ
ಯೋಚಿ-ಸ-ಲಾರ
ಯೋಚಿಸಲು
ಯೋಚಿಸಲೂ
ಯೋಚಿಸಿ
ಯೋಚಿ-ಸಿ-ಕೊಂಡ
ಯೋಚಿ-ಸಿ-ಕೊಂಡಿದ್ದ
ಯೋಚಿ-ಸಿ-ಕೊಂಡಿದ್ದರು
ಯೋಚಿಸಿದ
ಯೋಚಿ-ಸಿ-ದಂತಿಲ್ಲ
ಯೋಚಿ-ಸಿ-ದಂತೆ
ಯೋಚಿ-ಸಿ-ದನೊ
ಯೋಚಿ-ಸಿ-ದರು
ಯೋಚಿ-ಸಿ-ದರೂ
ಯೋಚಿ-ಸಿ-ದರೆ
ಯೋಚಿ-ಸಿ-ದಾಗ
ಯೋಚಿಸಿದೆ
ಯೋಚಿ-ಸಿದ್ದ-ಳು-ಎಂದರೆ
ಯೋಚಿ-ಸಿದ್ದಿದೆ
ಯೋಚಿ-ಸಿದ್ದೀರಾ
ಯೋಚಿಸಿದ್ದೂ
ಯೋಚಿ-ಸಿ-ರ-ಬ-ಹುದು
ಯೋಚಿ-ಸಿ-ರ-ಲಿಲ್ಲ
ಯೋಚಿಸು
ಯೋಚಿಸುತ್ತ
ಯೋಚಿ-ಸುತ್ತ-ಲಿದ್ದಾರೆ
ಯೋಚಿ-ಸುತ್ತಿ-ದೆಯೇ
ಯೋಚಿ-ಸುತ್ತಿದ್ದ
ಯೋಚಿ-ಸುತ್ತಿದ್ದರೆ
ಯೋಚಿ-ಸುತ್ತಿದ್ದಳು
ಯೋಚಿ-ಸುತ್ತಿದ್ದು-ದಲ್ಲ
ಯೋಚಿ-ಸುತ್ತಿಲ್ಲ
ಯೋಚಿ-ಸುತ್ತೀರಿ
ಯೋಚಿಸುವ
ಯೋಚಿ-ಸು-ವ-ವ-ರಾರು
ಯೋಚಿ-ಸು-ವ-ವ-ರಿದ್ದಾರೆ
ಯೋಚಿ-ಸು-ವಿರೋ
ಯೋಚಿ-ಸು-ವು-ದಿಲ್ಲ
ಯೋಚಿ-ಸು-ವುದು
ಯೋಜ-ಕ-ನೊಬ್ಬ
ಯೋಜನೆ
ಯೋಜ-ನೆ-ಗಳ
ಯೋಜ-ನೆ-ಗ-ಳನ್ನ-ನು-ಸ-ರಿ-ಸುತ್ತ
ಯೋಜ-ನೆ-ಗ-ಳನ್ನು
ಯೋಜ-ನೆ-ಗ-ಳಲ್ಲಿ
ಯೋಜ-ನೆ-ಗ-ಳಿಂದ
ಯೋಜ-ನೆ-ಗಳು
ಯೋಜ-ನೆ-ಗಳೂ
ಯೋಜ-ನೆ-ಗ-ಳೇನು
ಯೋಜನೆಗೆ
ಯೋಜ-ನೆಪ್ಲಾ-ನಿಂಗ್ಆ-ಧು-ನಿಕ
ಯೋಜನೆಯ
ಯೋಜ-ನೆ-ಯಂತೆ
ಯೋಜ-ನೆ-ಯನ್ನು
ಯೋಜ-ನೆ-ಯಾ-ಗಿತ್ತು
ಯೋಜ-ನೆ-ಯಿಂದ
ಯೋಜ-ನೆ-ಯೊಂದು
ಯೋಜಿಸಿ
ಯೋಜಿ-ಸಿ-ಕೊಂಡ
ಯೋಜಿಸಿದ
ಯೋಧರು
ಯೌಗಿಕ
ಯೌವನದ
ಯೌವ-ನ-ದಲ್ಲಿ
ಯೌವ-ನ-ದಲ್ಲೇ
ಯೌವ-ನಾ-ವಸ್ಥೆ-ಗ-ಳಲ್ಲಿ
ರ
ರಂಗ
ರಂಗ-ಗ-ಳಲ್ಲಿ
ರಂಗ-ಗ-ಳಲ್ಲೂ
ರಂಗನ
ರಂಗನೆಡೆ
ರಂಗಸ್ಥ-ಳ-ದಲ್ಲಿಯೂ
ರಂಗೋಲಿ
ರಂಜ-ಕ-ವಾಗಿ
ರಂಜಿಪ
ರಂದು
ರಂಧ್ರ
ರಂಪಾಟ
ರಕ್ತ
ರಕ್ತ-ಕ-ಣ-ಗಳು
ರಕ್ತ-ಕ-ಣ-ಗಳೂ
ರಕ್ತಕ್ರಾಂತಿ
ರಕ್ತಜ್ವಾಲೆ
ರಕ್ತದ
ರಕ್ತದಲ್ಲಿ
ರಕ್ತದಲ್ಲೇ
ರಕ್ತದಿಂದ
ರಕ್ತನಾಳ
ರಕ್ತ-ನಾ-ಳದ
ರಕ್ತಪರಿ
ರಕ್ತ-ಪ-ರಿ-ಚ-ಲನೆ
ರಕ್ತ-ಪಿ-ಪಾಸು
ರಕ್ತ-ಪಿ-ಪಾ-ಸು-ಗ-ಳಾ-ಗಿಯೇ
ರಕ್ತಪ್ರ-ವಾಹ
ರಕ್ತಪ್ರ-ವಾ-ಹ-ದಲ್ಲಿನ
ರಕ್ತಮಾಂಸ
ರಕ್ತ-ಮಾಂಸ-ಗ-ಳಿಂದ
ರಕ್ತ-ಮಾಂಸ-ಮಜ್ಜೆ
ರಕ್ತ-ರ-ಹಿತ
ರಕ್ತವನ್ನು
ರಕ್ತವು
ರಕ್ತವೇ
ರಕ್ತ-ಶುದ್ಧಿ-ಯಾ-ದಂತೆ
ರಕ್ತ-ಸಂಚಾರ
ರಕ್ತಸ್ರಾವ
ರಕ್ತಸ್ರಾ-ವ-ದಿಂದ
ರಕ್ತ-ಹೀ-ನ-ತೆ-ಯಿಂದ
ರಕ್ತ-ಹೆಪ್ಪು-ಗಟ್ಟು-ವಂಥ
ರಕ್ಷ-ಕ-ನಾ-ಗಿದ್ದ
ರಕ್ಷ-ಕ-ನೆಂಬ
ರಕ್ಷಕರು
ರಕ್ಷಣಾ
ರಕ್ಷ-ಣಾ-ಧಿ-ಕಾ-ರಿ-ಗಳ
ರಕ್ಷಣೆ
ರಕ್ಷ-ಣೆ-ಇ-ವು-ಗ-ಳಿಗೆ
ರಕ್ಷ-ಣೆ-ಗಾಗಿ
ರಕ್ಷ-ಣೆ-ಗಾ-ಗಿಯೇ
ರಕ್ಷಣೆಗೆ
ರಕ್ಷ-ಣೆ-ಗೆಂದು
ರಕ್ಷಣೆಯ
ರಕ್ಷ-ಣೆ-ಯನ್ನು
ರಕ್ಷ-ಣೆ-ಯನ್ನೂ
ರಕ್ಷ-ಣೋ-ಪಾಯ
ರಕ್ಷಿಷ್ಯ-ತೀತಿ
ರಕ್ಷಿ-ಸ-ಬಲ್ಲ
ರಕ್ಷಿ-ಸ-ಲಿಲ್ಲ-ವೆಂದು
ರಕ್ಷಿಸಲು
ರಕ್ಷಿಸಿ
ರಕ್ಷಿ-ಸಿ-ಕೊಂಡು
ರಕ್ಷಿ-ಸಿ-ಕೊಳ್ಳ-ಬೇಕು
ರಕ್ಷಿ-ಸಿ-ಕೊಳ್ಳ-ಲಾ-ರನೋ
ರಕ್ಷಿ-ಸಿ-ಕೊಳ್ಳು-ವುದು
ರಕ್ಷಿ-ಸಿ-ಕೊಳ್ಳು-ವು-ದೆಂದರೆ
ರಕ್ಷಿ-ಸಿ-ದರೆ
ರಕ್ಷಿ-ಸಿ-ದಿರಿ
ರಕ್ಷಿ-ಸಿ-ರ-ಬ-ಹುದು
ರಕ್ಷಿ-ಸುತ್ತದೆ
ರಕ್ಷಿ-ಸುತ್ತವೆ
ರಕ್ಷಿ-ಸುತ್ತಿ-ರುವ
ರಕ್ಷಿಸುವ
ರಕ್ಷಿ-ಸು-ವು-ದಕ್ಕಾಗಿ
ರಘು-ಪ-ತಿಯೇ
ರಘುಪತೇ
ರಘು-ಪುಂಗವ
ರಘು-ವಂಶದ
ರಚ-ನಾತ್ಮಕ
ರಚ-ನಾತ್ಮ-ಕತೆ
ರಚ-ನಾತ್ಮ-ಕ-ವಾಗಿ
ರಚ-ನಾತ್ಮ-ಕ-ವಾದ
ರಚ-ನಾತ್ಮ-ಕವೂ
ರಚ-ನಾತ್ಮ-ಕವೊ
ರಚ-ನಾ-ಸಾ-ಮರ್ಥ್ಯ-ವನ್ನು
ರಚನೆ
ರಚನೆಯ
ರಚ-ನೆ-ಯಲ್ಲಾ-ಗಲೀ
ರಚ-ನೆ-ಯಲ್ಲಿ
ರಚ-ನೆ-ಯಾ-ಗ-ಬೇಕು
ರಚ-ನೆ-ಯಾ-ಗಿದೆ
ರಚಿ-ತ-ವಾ-ಗಿದೆ
ರಚಿ-ತ-ವಾದ
ರಚಿ-ತ-ವಾ-ದ-ವು-ಗಳು
ರಚಿ-ಸ-ಬ-ಹುದು
ರಚಿಸಲಿ
ರಚಿಸಲು
ರಚಿಸಿ
ರಚಿಸಿತು
ರಚಿಸಿತ್ತು
ರಚಿಸಿದ
ರಚಿ-ಸಿ-ದರು
ರಚಿ-ಸಿ-ದ-ವನು
ರಜಪೂತ
ರಜಸ್ತಮೋ
ರಜಾ
ರಜೆ
ರಜೆಯ
ರಜೆಯಲ್ಲಿ
ರಜೆಯಿಂದ
ರಟ್ಟಿನ
ರಡಿ
ರಡು
ರಣ
ರಣ-ಮೋ-ದದ
ರಣ-ರಂಗ-ದಲ್ಲಿ
ರಣ-ರೋ-ಷದ
ರಣ-ಸನ್ನಾಹ
ರಣಿ-ತ-ವಾ-ಗುತ್ತಿತ್ತು
ರಣೆಯ
ರತರಿಗೆ
ರತ್ನ-ಖ-ಚಿತ
ರತ್ನದ
ರಥದ
ರಥವನ್ನು
ರಥೋತ್ಸವ
ರದ
ರದ್ದು
ರನ್ನು
ರಫ್ತು
ರಭ-ಸ-ದಿಂದ
ರಭ-ಸ-ವಾಗಿ
ರಮಣ
ರಮ-ಣ-ಮ-ಹರ್ಷಿ-ಗಳು
ರಮ-ಣ-ಮ-ಹರ್ಷಿ-ಗ-ಳೆನ್ನು-ವಂತೆ
ರಮ-ಣಾಶ್ರ-ಮದ
ರರು
ರಲ್ಲಿ
ರಲ್ಲೇ
ರವರೆಗೆ
ರವಿದತ್ತ
ರವೀಂದ್ರ-ನಾಥ
ರಷ್
ರಷ್ಟು
ರಷ್ಯದ
ರಷ್ಯದಲ್ಲಿ
ರಷ್ಯ-ದ-ವರು
ರಷ್ಯ-ನ-ರಿಗೆ
ರಷ್ಯನ್
ರಷ್ಯಾ
ರಷ್ಯಾಕ್ಕೆ
ರಷ್ಯಾದ
ರಷ್ಯಾದಲ್ಲಿ
ರಷ್ಯಾ-ದೇ-ಶದ
ರಸ-ಋ-ಷಿ-ಯಾಗಿ
ರಸ-ಭ-ರಿತ
ರಸಾ-ತ-ಳಕ್ಕೆ
ರಸಾಯನ
ರಸಿಕ
ರಸಿ-ಕ-ಲಾಲ
ರಸೂಲ್ಪು-ರಕ್ಕಿಂತ
ರಸೂಲ್ಪು-ರಕ್ಕೆ
ರಸೂಲ್ಪು-ರ-ದಲ್ಲಿ
ರಸೆಲ್
ರಸೆಲ್ಲರು
ರಸ್ಕಿನ್
ರಸ್ತೆ
ರಸ್ತೆಗೆ
ರಸ್ತೆಯ
ರಸ್ತೆಯಲ್ಲಿ
ರಸ್ಸೆಲ್
ರಸ್ಸೆಲ್ಲರ
ರಹತೇ
ರಹಸ್ಯ
ರಹಸ್ಯ-ಗ-ಳನ್ನು
ರಹಸ್ಯ-ಗಳು
ರಹಸ್ಯ-ಗ-ಳೇನು
ರಹಸ್ಯದ
ರಹಸ್ಯ-ಭೇ-ದ-ನೆಯ
ರಹಸ್ಯ-ವನ್ನು
ರಹಸ್ಯ-ವನ್ನೂ
ರಹಸ್ಯ-ವಾಗಿ
ರಹಸ್ಯ-ವಿ-ರು-ವುದು
ರಹಸ್ಯವೂ
ರಹಸ್ಯ-ವೆಂಥದು
ರಹಸ್ಯ-ವೇನು
ರಹಸ್ಯ-ವೇ-ನೆಂದು
ರಹಸ್ಯಾತ್ಮ-ಕವೂ
ರಹಸ್ಯೋದ್ಘಾ-ಟನೆ
ರಹಸ್ಯೋದ್ಘಾ-ಟ-ನೆ-ಯನ್ನು
ರಹಿತ
ರಾ
ರಾಕೆಟ್
ರಾಕೆಟ್ಟು
ರಾಕ್ಷಸೀ
ರಾಖಾಲ
ರಾಖಾಲನ
ರಾಖಾ-ಲ-ನಿಗೆ
ರಾಖಾ-ಲ-ನಿದ್ದಾನೆ
ರಾಖಾಲರ
ರಾಗ
ರಾಗ-ಗ-ಳನ್ನು
ರಾಗ-ಗ-ಳಿಗೆ
ರಾಗದ
ರಾಗದೇ
ರಾಗದ್ವೇಷ
ರಾಗದ್ವೇ-ಷ-ಗಳ
ರಾಗದ್ವೇ-ಷ-ಗ-ಳನ್ನು
ರಾಗದ್ವೇ-ಷ-ಗ-ಳನ್ನೂ
ರಾಗದ್ವೇ-ಷ-ಗಳು
ರಾಗದ್ವೇ-ಷ-ಗ-ಳು-ಇ-ವು-ಗ-ಳೆಲ್ಲ
ರಾಗದ್ವೇ-ಷ-ಗ-ಳು-ನಿತ್ಯವೂ
ರಾಗವನ್ನು
ರಾಗಿ
ರಾಗು-ವು-ದುಂಟು
ರಾಜ
ರಾಜಕೀಯ
ರಾಜ-ನೀ-ತಿಜ್ಞರೂ
ರಾಜ-ಮಾರ್ಗ-ವಾ-ಯಿತು
ರಾಜ-ಕಾ-ರಣಿ
ರಾಜ-ಕಾ-ರ-ಣಿ-ಗಳು
ರಾಜಕೀಯ
ರಾಜ-ಕೀ-ಯ-ಇ-ವು-ಗ-ಳೆಲ್ಲ
ರಾಜ-ಕೀ-ಯದ
ರಾಜ-ಕೀ-ಯ-ದಲ್ಲೂ
ರಾಜ-ಕೀ-ಯ-ವಾಗಿ
ರಾಜ-ಕೀ-ಯ-ವೇನು
ರಾಜ-ಕೀ-ಯವೋ
ರಾಜ-ಕೀ-ಯ-ಶಾಸ್ತ್ರದ
ರಾಜ-ಕೀ-ಯಸ್ಥ
ರಾಜ-ಕೀ-ಯಸ್ಥರು
ರಾಜ-ಕೀ-ಯಸ್ಥರೂ
ರಾಜ-ಗೋ-ಪಾ-ಲಾ-ಚಾರಿ
ರಾಜಣ್ಣ
ರಾಜ-ಧಾ-ನಿ-ಯಲ್ಲೇ
ರಾಜನ
ರಾಜನನ್ನು
ರಾಜನೀತಿ
ರಾಜನು
ರಾಜಪುತ್ರ
ರಾಜ-ಪುತ್ರ-ನಂತೆ
ರಾಜ-ಪುತ್ರ-ನನ್ನು
ರಾಜ-ಪುತ್ರ-ನೆಂದು
ರಾಜ-ಪುತ್ರನೇ
ರಾಜ-ಭೋ-ಗ-ಗ-ಳನ್ನಾ-ಗಲೀ
ರಾಜಮಾರ್ಗ
ರಾಜ-ಮಾರ್ಗ-ಗ-ಳನ್ನು
ರಾಜರ
ರಾಜರು
ರಾಜ-ರು-ಗಳ
ರಾಜರ್ಷಿ-ಗ-ಳಿಗೆ
ರಾಜಾಜಿ
ರಾಜಾಜ್ಞೆ-ಯನ್ನು
ರಾಜಾಶ್ರ-ಯ-ವಿ-ರುತ್ತಿತ್ತು
ರಾಜೀನಾಮೆ
ರಾಜ್ಯ
ರಾಜ್ಯಕ್ಕೆ
ರಾಜ್ಯದ
ರಾಜ್ಯದಲ್ಲಿ
ರಾಜ್ಯ-ದಲ್ಲಿನ
ರಾಜ್ಯವನ್ನು
ರಾಜ್ಯ-ವಾ-ಗ-ಬೇಕು
ರಾಜ್ಯ-ವೊಂದರ
ರಾಜ್ಯವ್ಯ-ವಸ್ಥೆ-ಯನ್ನು
ರಾಜ್ಯಾಂಗ
ರಾಠೋಡ
ರಾತ್ರಿ
ರಾತ್ರಿಕ್ಲಬ್ಬು-ಗಳ
ರಾತ್ರಿಯಲ್ಲಿ
ರಾತ್ರಿ-ಯೆನ್ನದೆ
ರಾತ್ರಿಯೆಲ್ಲ
ರಾತ್ರಿಹೊತ್ತು
ರಾದ್ಧಾಂತ
ರಾಬರ್ಟ್
ರಾಮ
ರಾಮಕೃಷ್ಣ
ರಾಮ-ಕೃಷ್ಣರ
ರಾಮ-ಕೃಷ್ಣರು
ರಾಮ-ಕೃಷ್ಣಾ-ನಂದ
ರಾಮ-ತೀರ್ಥ-ರನ್ನು
ರಾಮ-ತೀರ್ಥರು
ರಾಮನ
ರಾಮ-ನಿದ್ದಲ್ಲಿ
ರಾಮ-ಬಾ-ಣದ
ರಾಮ-ಭಾ-ವ-ನೆಯು
ರಾಮ-ರಕ್ಷಾಸ್ತೋತ್ರ
ರಾಮರಾಜ್ಯ
ರಾಮ-ರಾಜ್ಯದ
ರಾಮಾ-ನು-ಜನ್
ರಾಯಭಾರಿ
ರಾರಾ-ಜಿ-ಸುತ್ತಾರೆ
ರಾವಣ
ರಾವ್
ರಾಶಿ
ರಾಷ್ಟ್ರ
ರಾಷ್ಟ್ರಂಗಳು
ರಾಷ್ಟ್ರಕ್ಕಾಗಿ
ರಾಷ್ಟ್ರಕ್ಕೆ
ರಾಷ್ಟ್ರಗಳ
ರಾಷ್ಟ್ರ-ಗ-ಳಲ್ಲಿ
ರಾಷ್ಟ್ರ-ಗ-ಳಿಗೆ
ರಾಷ್ಟ್ರಗಳು
ರಾಷ್ಟ್ರಗಳೂ
ರಾಷ್ಟ್ರಗಳೇ
ರಾಷ್ಟ್ರದ
ರಾಷ್ಟ್ರ-ದಲ್ಲಿ-ರ-ಬ-ಹು-ದು-ಸುಪ್ತ-ನಿದ್ರೆಯ
ರಾಷ್ಟ್ರದೀಪ
ರಾಷ್ಟ್ರಧ್ವ-ಜ-ದಲ್ಲಿ
ರಾಷ್ಟ್ರ-ಪ-ತಿ-ಗಳೇ
ರಾಷ್ಟ್ರಪಿತ
ರಾಷ್ಟ್ರಪ್ರೇ-ಮಿ-ಗಳೂ
ರಾಷ್ಟ್ರ-ರ-ಚ-ನೆಯ
ರಾಷ್ಟ್ರವನ್ನು
ರಾಷ್ಟ್ರವನ್ನೂ
ರಾಷ್ಟ್ರವಾಗಿ
ರಾಷ್ಟ್ರ-ವಾದ್ದ-ರಿಂದ
ರಾಷ್ಟ್ರವು
ರಾಷ್ಟ್ರವೂ
ರಾಷ್ಟ್ರವ್ಯಾಪಿ
ರಾಷ್ಟ್ರ-ಹಿ-ತಕ್ಕಾಗಿ
ರಾಷ್ಟ್ರಾ-ಭಿ-ಮಾ-ನ-ವನ್ನುಂಟು-ಮಾಡಿ
ರಾಷ್ಟ್ರಾ-ಭಿ-ಮಾ-ನಿ-ಗ-ಳ-ಗಾಗಿ
ರಾಷ್ಟ್ರೀಯ
ರಾಷ್ಟ್ರೀ-ಯ-ತಾ-ಭಾ-ವ-ನೆ-ಯಿಂದ
ರಾಷ್ಟ್ರೀಯತೆ
ರಾಸಾ-ಯ-ನಿಕ
ರಾಹುಕಾಲ
ರಿಂದ
ರಿಂದಲೂ
ರಿಂದಲೇ
ರಿಕ
ರಿಕನ್ಸ್ಟ್ರಕ್ಶನ್
ರಿಗೆ
ರಿಪೇರಿ
ರಿಪೇ-ರಿ-ಮಾ-ಡುವ
ರಿಯೋ-ಡಿ-ಜೆ-ನಿರೊ
ರಿಲಿಜನ್
ರಿಲೇಕ್ಸ್
ರಿಲೇ-ಷನ್ಷಿಪ್
ರಿಲ್ಯಾಕ್ಸ್
ರಿಸರ್ವ್
ರಿಸೀವರ್
ರೀಡರ್ಸ್
ರೀತಿ
ರೀತಿಗಳ
ರೀತಿ-ಗ-ಳಲ್ಲಿ
ರೀತಿ-ಗ-ಳಿಂದ
ರೀತಿನೀತಿ
ರೀತಿ-ನೀ-ತಿ-ಗ-ಳನ್ನೂ
ರೀತಿ-ನೀ-ತಿ-ಗಳು
ರೀತಿಯ
ರೀತಿಯದು
ರೀತಿಯನ್ನು
ರೀತಿಯಲ್ಲಿ
ರೀತಿಯಲ್ಲೂ
ರೀತಿಯಲ್ಲೇ
ರೀತಿ-ಯ-ವು-ಗ-ಳಿದ್ದರೂ
ರೀತಿಯಾಗಿ
ರೀತಿ-ಯಾ-ಗಿ-ರು-ವುದು
ರೀತಿಯಿಂದ
ರೀತಿ-ಯಿಂದಲೂ
ರೀನೆ
ರುಚಿ
ರುಚಿಕೊಟ್ಟು
ರುಚಿಯ
ರುಚಿಯನ್ನು
ರುಚಿಯೂ
ರುಚಿ-ಸು-ವು-ದಿಲ್ಲ
ರುಜಿ-ನ-ಗ-ಳಿಂದ
ರುಜು
ರುದ್ರ
ರುಬ್ಬುವುದೇ
ರುಯ್ಸ್ಬ್ರೋಕ್
ರುವ
ರೂಢ
ರೂಢಮೂಲ
ರೂಢ-ಮೂ-ಲ-ವಾ-ಗಲು
ರೂಢ-ಮೂ-ಲ-ವಾ-ಗಿತ್ತು
ರೂಢ-ಮೂ-ಲ-ವಾ-ಗುತ್ತಿದೆ
ರೂಢ-ಮೂ-ಲ-ವಾದ
ರೂಢ-ಮೂ-ಲ-ವಾ-ದರೆ
ರೂಢಿ
ರೂಢಿ-ಗ-ಳನ್ನು
ರೂಢಿಯಾಗಿ
ರೂಢಿಸಿ
ರೂಢಿ-ಸಿ-ಕೊಳ್ಳುವ
ರೂಢಿ-ಸಿ-ಕೊಂಡ
ರೂಢಿ-ಸಿ-ಕೊಂಡರೆ
ರೂಢಿ-ಸಿ-ಕೊಂಡಲ್ಲಿ
ರೂಢಿ-ಸಿ-ಕೊಂಡಿದ್ದ
ರೂಢಿ-ಸಿ-ಕೊಂಡಿದ್ದರು
ರೂಢಿ-ಸಿ-ಕೊಂಡಿದ್ದೇವೆ
ರೂಢಿ-ಸಿ-ಕೊಂಡು
ರೂಢಿ-ಸಿ-ಕೊಳ್ಳ-ಬೇ-ಕಾದ
ರೂಢಿ-ಸಿ-ಕೊಳ್ಳ-ಬೇಕು
ರೂಢಿ-ಸಿ-ಕೊಳ್ಳ-ಬೇ-ಕೆಂಬು-ದನ್ನು
ರೂಢಿ-ಸಿ-ಕೊಳ್ಳ-ಬೇ-ಕೆಂಬುದು
ರೂಢಿ-ಸಿ-ಕೊಳ್ಳಲು
ರೂಢಿ-ಸಿ-ಕೊಳ್ಳಿ
ರೂಢಿ-ಸಿ-ಕೊಳ್ಳುತ್ತಿದ್ದಾರೆ
ರೂಢಿ-ಸಿ-ಕೊಳ್ಳುವ
ರೂಢಿ-ಸಿ-ಕೊಳ್ಳು-ವರೇ
ರೂಪ
ರೂಪಕ್ಕೆ
ರೂಪಗಳು
ರೂಪ-ತಾ-ಳುತ್ತವೆ
ರೂಪ-ತಾ-ಳುತ್ತಿತ್ತು
ರೂಪದ
ರೂಪದಲ್ಲಿ
ರೂಪದ್ದು
ರೂಪರೇಖೆ
ರೂಪವನ್ನು
ರೂಪವಾಗಿ
ರೂಪ-ವಿಲ್ಲ-ದಿದ್ದರೂ
ರೂಪವೆ
ರೂಪ-ವೆಂದರೆ
ರೂಪವೇ
ರೂಪಾಂತ-ರ-ಗಳು
ರೂಪಾಂತ-ರಿ-ತ-ವಾ-ಗಿವೆ
ರೂಪಾಯಿ
ರೂಪಾ-ಯಿ-ಗಳ
ರೂಪಾ-ಯಿ-ಗ-ಳನ್ನು
ರೂಪಾ-ಯಿ-ಗ-ಳನ್ನೂ
ರೂಪಾ-ಯಿ-ಗ-ಳಲ್ಲಿ
ರೂಪಾ-ಯಿ-ಗ-ಳಷ್ಟು
ರೂಪಾ-ಯಿ-ಗಳು
ರೂಪಿ
ರೂಪಿ-ತ-ವಾ-ಗುತ್ತದೆ
ರೂಪಿ-ತ-ವಾ-ಗುತ್ತಿ-ರುತ್ತ-ದೆ-ಎನ್ನು-ವುದು
ರೂಪಿ-ತ-ವಾ-ಗು-ವುದು
ರೂಪಿನಲ್ಲಿ
ರೂಪಿಯೂ
ರೂಪಿ-ಸ-ತಕ್ಕ
ರೂಪಿಸಲು
ರೂಪಿಸಿ
ರೂಪಿ-ಸಿ-ಕೊಂಡ
ರೂಪಿ-ಸಿ-ಕೊಂಡರೆ
ರೂಪಿ-ಸಿ-ಕೊಂಡಿ-ರ-ಬ-ಹುದು
ರೂಪಿ-ಸಿ-ಕೊಳ್ಳ-ಬೇಕು
ರೂಪಿ-ಸಿ-ಕೊಳ್ಳ-ಬೇಕೆ
ರೂಪಿ-ಸಿ-ಕೊಳ್ಳುತ್ತೀರಿ
ರೂಪಿ-ಸಿ-ಕೊಳ್ಳುವ
ರೂಪಿ-ಸಿ-ಕೊಳ್ಳು-ವಂತೆ
ರೂಪಿಸಿದ
ರೂಪಿ-ಸುತ್ತವೆ
ರೂಪಿ-ಸುತ್ತಿದ್ದೀರಿ
ರೂಪಿಸುವ
ರೂಪಿ-ಸು-ವಲ್ಲಿ
ರೂಪಿ-ಸು-ವ-ವ-ರಿಲ್ಲ
ರೂಪಿ-ಸು-ವು-ದ-ರಲ್ಲಿ
ರೂಪಿ-ಸು-ವುದು
ರೂಪು
ರೂಪುಗೊಂಡ
ರೂಪು-ಗೊಂಡಿದೆ
ರೂಪು-ಗೊಳ್ಳ-ಬ-ಹು-ದೆಂಬು-ದನ್ನು
ರೂಪು-ಗೊಳ್ಳಲು
ರೂಪು-ಗೊಳ್ಳಲೂ
ರೂಪು-ರೇ-ಖೆ-ಗಳ
ರೂಬ-ಲು-ಗ-ಳನ್ನು
ರೂರ್ಕೆ
ರೂವಾರಿ
ರೂವಾ-ರಿ-ಗಳು
ರೆಂಚ್
ರೆಂಬುದು
ರೆಕಾರ್ಡರ್
ರೆಕ್ಕೆ
ರೆಡ್
ರೆಡ್ನಿಕ್
ರೆನ್ನಿ-ಸಿ-ಕೊಂಡ-ವರ
ರೆಫ್ರಿ-ಜ-ರೇ-ಟರ್
ರೆಲವೆಂಟೆ
ರೆಸಿ-ಡೆನ್ಸಿ-ಯಲ್
ರೇಖೆ
ರೇಖೆ-ಗ-ಳನ್ನು
ರೇಖೆಗಳೂ
ರೇಗನ್ನರು
ರೇಗಾ-ಡು-ವುದು
ರೇಗಿ
ರೇಗೂ
ರೇಡಿ-ಯಂನಂತೆ
ರೇಡಿಯಮ್
ರೇಡಿಯೊ
ರೇಡಿಯೋ
ರೇಡಿ-ಯೋ-ವನ್ನು
ರೇಶ್ಮೆ
ರೇಷ್ಮೆಯನ್ನು
ರೈತನೊಬ್ಬ
ರೈತರು
ರೈಲಿನ
ರೈಲಿನಲ್ಲಿ
ರೈಲು
ರೈಲ್ವೇ
ರೈಲ್ವೇಸ್ಟೇ-ಷನ್ನಿ-ನಿಂದ
ರೊಂದಿಗೆ
ರೊಚ್ಚನ್ನು
ರೊಚ್ಚಿ
ರೊಚ್ಚಿಗೆ
ರೊಚ್ಚಿ-ಗೇ-ಳು-ವಂತೆ
ರೊಟ್ಟಿ
ರೊಟ್ಟಿ-ಗ-ಳನ್ನೂ
ರೊಬ್ಬರು
ರೋಗ
ರೋಗಕ್ಕಿಂತಲೂ
ರೋಗಕ್ಕೂ
ರೋಗಕ್ಕೆ
ರೋಗಕ್ಲೇ-ಶ-ಗ-ಳಿಂದ
ರೋಗಗಳ
ರೋಗ-ಗ-ಳನ್ನು
ರೋಗ-ಗ-ಳನ್ನೂ
ರೋಗ-ಗ-ಳಲ್ಲಿ
ರೋಗ-ಗ-ಳಿಂದ
ರೋಗ-ಗ-ಳಿಂದಲೂ
ರೋಗ-ಗ-ಳಿಗೂ
ರೋಗ-ಗ-ಳಿಗೆ
ರೋಗಗಳು
ರೋಗಗಳೂ
ರೋಗಗ್ರಸ್ತ
ರೋಗಗ್ರಸ್ತ-ರಾಗಿ
ರೋಗಗ್ರಸ್ತ-ವಾಗಿ
ರೋಗದ
ರೋಗದಿಂದ
ರೋಗ-ದಿಂದಲೇ
ರೋಗ-ನಿ-ದಾ-ನಕ್ಕೂ
ರೋಗ-ನಿರ್ಮೂ-ಲ-ನ-ಶಕ್ತಿ-ಯನ್ನು
ರೋಗ-ಪೀ-ಡಿ-ತ-ನಾಗಿ
ರೋಗ-ಪೀ-ಡಿ-ತ-ರಾದ
ರೋಗ-ಮ-ನುಷ್ಯರು
ರೋಗ-ಮುಕ್ತ-ರನ್ನಾಗಿ
ರೋಗ-ಮುಕ್ತ-ರಾ-ಗಲು
ರೋಗ-ಮುಕ್ತ-ರಾ-ದ-ವರು
ರೋಗ-ಮುಕ್ತ-ಳಾ-ದೇ-ನೆಂದು
ರೋಗ-ರು-ಜಿನ
ರೋಗ-ರು-ಜಿ-ನ-ಗಳ
ರೋಗ-ರು-ಜಿ-ನ-ಗ-ಳನ್ನು
ರೋಗ-ರು-ಜಿ-ನ-ಗ-ಳಿಂದ
ರೋಗ-ರು-ಜಿ-ನ-ಗಳು
ರೋಗವನ್ನು
ರೋಗ-ವಿ-ಧಾ-ನವೂ
ರೋಗವೂ
ರೋಗವೆಂದು
ರೋಗವೆಲ್ಲ
ರೋಗವೇ
ರೋಗಾ-ಣು-ಗಳ
ರೋಗಾ-ಣು-ಗ-ಳಿಂದ
ರೋಗಾಸ್ತ್ರ-ಗ-ಳನ್ನು
ರೋಗಿ
ರೋಗಿಗಳ
ರೋಗಿ-ಗ-ಳನ್ನು
ರೋಗಿ-ಗ-ಳಲ್ಲಿ
ರೋಗಿ-ಗ-ಳಲ್ಲೂ
ರೋಗಿ-ಗ-ಳಿಂದ
ರೋಗಿ-ಗ-ಳಿಗೆ
ರೋಗಿಗಳು
ರೋಗಿಗಳೂ
ರೋಗಿ-ಗ-ಳೆಲ್ಲರೂ
ರೋಗಿಗೆ
ರೋಗಿಯ
ರೋಗಿಯನ್ನು
ರೋಗಿಯನ್ನೂ
ರೋಗಿಯಾಗಿ
ರೋಗಿಯು
ರೋಗಿಯೊಬ್ಬ
ರೋಗಿಷ್ಠನೂ
ರೋಗಿಷ್ಠರೂ
ರೋಚ-ಕ-ವಾ-ಗಿದೆ
ರೋಜರ್
ರೋಜಿನ
ರೋದ-ನ-ವಾ-ಗುತ್ತದೆ
ರೋಪಣೆಯ
ರೋಮ
ರೋಮಾಂಚ-ಕಾರಿ
ರೋಮಾಂಚ-ಕಾ-ರಿ-ಯಾದ
ರೋಮಾಂಚ-ನ-ಗೊಂಡ
ರೋಮಾಂಚ-ನ-ವಾಗ
ರೋಮಾಂಚಿ-ತ-ನಾದ
ರೋಮಿನ
ರೋಮಿನಲ್ಲಿ
ರೋಮ್
ರೋಷದ
ರೋಷವನ್ನು
ರೋಷಾ-ವೇ-ಶ-ದಿಂದ
ರೋಸಿ
ರೋಸಿತು
ರೋಸಿ-ಹೋ-ಗುತ್ತದೆ
ರೋಸ್
ರೋಸ್ವೈದ್ಯರು
ರ್ಯಾಂಕ್
ರ್ಯಾಗಿಂಗ್
ರ್ಹೆಟರಿಕ್
ರ್ಹೈನ್
ಲಂಗರು
ಲಂಗು-ಲ-ಗಾ-ಮಿಲ್ಲದ
ಲಂಗು-ಲ-ಗಾ-ಮಿಲ್ಲದೆ
ಲಂಚ
ಲಂಚಕೋರ
ಲಂಚ-ಕೋ-ರ-ತನ
ಲಂಚ-ಕೋ-ರ-ತ-ನ-ವನ್ನು
ಲಂಚಪ್ರ-ಲೋ-ಭ-ನೆಯ
ಲಂಡನ್
ಲಂಡನ್ನಿಗೆ
ಲಂಡನ್ನಿನ
ಲಂಪಟತೆ
ಲಂಪ-ಟ-ತೆಯ
ಲಕ್ಷ
ಲಕ್ಷಕ್ಕಿಂತ
ಲಕ್ಷಕ್ಕೂ
ಲಕ್ಷಕ್ಕೇ-ರಿದೆ
ಲಕ್ಷಕ್ಕೊಬ್ಬ-ನಿಗೂ
ಲಕ್ಷ-ಗಟ್ಟಲೆ
ಲಕ್ಷಣ
ಲಕ್ಷ-ಣ-ಗಳ
ಲಕ್ಷ-ಣ-ಗ-ಳನ್ನು
ಲಕ್ಷ-ಣ-ಗ-ಳಷ್ಟೆ
ಲಕ್ಷ-ಣ-ಗ-ಳಿಂದ
ಲಕ್ಷ-ಣ-ಗ-ಳಿ-ವೆಯೇ
ಲಕ್ಷ-ಣ-ಗಳು
ಲಕ್ಷ-ಣ-ಗಳೂ
ಲಕ್ಷ-ಣ-ಗ-ಳೆಲ್ಲ
ಲಕ್ಷಣದ
ಲಕ್ಷ-ಣ-ವಾಗಿ
ಲಕ್ಷ-ಣ-ವಾ-ದರೆ
ಲಕ್ಷ-ಣ-ವಿಲ್ಲ
ಲಕ್ಷಣವೂ
ಲಕ್ಷಣವೆ
ಲಕ್ಷ-ಣ-ವೆಂದು
ಲಕ್ಷ-ಣ-ವೆನ್ನು-ವಂತಾ-ಗಿದೆ
ಲಕ್ಷದ
ಲಕ್ಷ-ದ-ವ-ರೆಗೂ
ಲಕ್ಷಮಂದಿ
ಲಕ್ಷಲಕ್ಷ
ಲಕ್ಷಾಂತರ
ಲಕ್ಷ್ಮಮ್ಮ
ಲಕ್ಷ್ಮೀ
ಲಕ್ಷ್ಮೀ-ದೇ-ವಮ್ಮ
ಲಕ್ಷ್ಯ
ಲಕ್ಷ್ಯ-ಇ-ವು-ಗ-ಳಲ್ಲಿ
ಲಕ್ಷ್ಯ-ವೆಂದ-ರಿತು
ಲಗೌರಿ
ಲಘುವಾಗಿ
ಲನ್ನು
ಲಪ-ಟಾ-ಯಿಸಿ
ಲಬ್
ಲಬ್ಧ
ಲಭತೇ
ಲಭಿ-ಸ-ಲಿಲ್ಲ
ಲಭಿಸಿ
ಲಭಿಸಿತು
ಲಭಿಸಿಯೇ
ಲಭಿ-ಸುತ್ತವೆ
ಲಭಿ-ಸುತ್ತ-ವೆಯೇ
ಲಭಿಸುವ
ಲಭಿ-ಸು-ವುದು
ಲಭ್ಯ
ಲಭ್ಯನು
ಲಭ್ಯ-ವಾ-ಗವು
ಲಭ್ಯ-ವಾ-ಗಿ-ರು-ವು-ದಾ-ದರೂ
ಲಭ್ಯ-ವಾ-ಗುತ್ತದೆ
ಲಭ್ಯ-ವಾ-ಗುವ
ಲಭ್ಯ-ವಾ-ಗು-ವು-ದೆಂಬು-ದನ್ನು
ಲಭ್ಯವಾದ
ಲಭ್ಯ-ವಾ-ದರೂ
ಲಭ್ಯ-ವಾ-ದು-ದರ
ಲಭ್ಯ-ವಾ-ದೊ-ಡ-ನೆಯೇ
ಲಭ್ಯವಿದ್ದ
ಲಭ್ಯವಿಲ್ಲ
ಲಯ-ಇ-ವು-ಗ-ಳಿಗೆ
ಲಯ-ಕರ್ತೃ-ವಾದ
ಲಯ-ಗ-ಳನ್ನು
ಲಯ-ಬದ್ಧತೆ
ಲಯ-ಬದ್ಧ-ತೆ-ಯನ್ನು
ಲಯ-ಬದ್ಧ-ವಾಗಿ
ಲಲಿತಾ
ಲಾ
ಲಾಂಗ್
ಲಾಂಛ-ನ-ವಾಗಿ
ಲಾಗಿತ್ತು
ಲಾಗುತ್ತಿಲ್ಲ
ಲಾಟರಿ
ಲಾಟರಿಯ
ಲಾಭ
ಲಾಭ-ಗ-ಳಿಸಿ
ಲಾಭದ
ಲಾಭ-ದಾ-ಯ-ಕ-ವೆಂಬ
ಲಾಭವಾದ
ಲಾಭ-ವಿದ್ದರೂ
ಲಾಯ-ಗ-ಳಲ್ಲಿದ್ದಾಗ
ಲಾಯರ್
ಲಾಯಿತು
ಲಾಯಿ-ತು-ಎಂದು
ಲಾರ
ಲಾರಂಭಿ-ಸಿತು
ಲಾರರು
ಲಾರಿಗೆ
ಲಾರೆಂಟಳ
ಲಾರೆಂಟ್
ಲಾರೆಂಟ್ಳ
ಲಾರೆನ್ಸ್
ಲಾರೆನ್ಸ್ನ
ಲಾರೆನ್ಸ್ನದು
ಲಾರೆವು
ಲಾರ್ಸನ್
ಲಾಲನೆ
ಲಾಲ-ನೆ-ಪಾ-ಲ-ನೆಯ
ಲಾಲರ
ಲಾಲ-ಸೆ-ಯನ್ನು
ಲಾಲಿಸಿ
ಲಾವಾ
ಲಾಸ್ಟ್
ಲಿಂಕನ್
ಲಿಂಕ್
ಲಿಂಗ
ಲಿಂಗ-ಭೇ-ದ-ವಾ-ಗುತ್ತ-ದೆಂದು
ಲಿಂಬೆಹಣ್ಣು
ಲಿಖಿತ
ಲಿಚಿ
ಲಿದೆ
ಲಿನ್
ಲಿಮಾ
ಲಿಮಾದ
ಲಿಮಾ-ದಲ್ಲಿದ್ದ
ಲಿಯೋನಾರ್ಡೊ
ಲಿಲ್ಲ
ಲಿಸಿ
ಲಿಸ್ಲಿ
ಲೀಟರಿಗೂ
ಲೀಟರ್
ಲೀಡರ್
ಲೀಡರ್ಗಳ
ಲೀಡರ್ಗಳು
ಲೀನ
ಲೀನರಾಗಿ
ಲೀನ-ವಾ-ಗು-ವಂಥದು
ಲೀನ-ವಾ-ಗುತ್ತವೆ
ಲೀಯೋನಾರ್ಡೋ
ಲೀಲಾ-ಜಾ-ಲ-ವಾಗಿ
ಲೀಲಾಪ್ರ-ಸಂಗ
ಲೀಲಾ-ವಿ-ನೋದ
ಲೀಲೆ
ಲೀಲೆಯನ್ನು
ಲೀಲೆಯನ್ನೂ
ಲೂಟಿ
ಲೂಯಿಯ
ಲೂಸ್
ಲೆಕ್ಕ
ಲೆಕ್ಕಕ್ಕೂ
ಲೆಕ್ಕ-ಗ-ಳನ್ನು
ಲೆಕ್ಕ-ಗ-ಳನ್ನೂ
ಲೆಕ್ಕಗಳು
ಲೆಕ್ಕದ
ಲೆಕ್ಕದಲ್ಲಿ
ಲೆಕ್ಕ-ಬಾ-ರದು
ಲೆಕ್ಕ-ವನ್ನಿ-ಡಲು
ಲೆಕ್ಕ-ವನ್ನಿ-ಡುತ್ತಾನೆ
ಲೆಕ್ಕವನ್ನು
ಲೆಕ್ಕಾಚಾರ
ಲೆಕ್ಕಿಸದೆ
ಲೆಕ್ಕಿ-ಸು-ವು-ದಿಲ್ಲ
ಲೆಟರ್
ಲೆಫ್ಟಿ-ನೆಂಟಾಗಿ
ಲೆರಿಂಜೈ-ಟಿಸ್
ಲೆವೋ
ಲೇಕೋಮ್
ಲೇಕ್ರೊನ್
ಲೇಖಕ
ಲೇಖಕನ
ಲೇಖ-ಕ-ನಿಗೆ
ಲೇಖ-ಕ-ನೊಬ್ಬ
ಲೇಖ-ಕ-ರಾದ
ಲೇಖಕರು
ಲೇಖ-ಕ-ರೊ-ಡನೆ
ಲೇಖ-ಕಿ-ಯರು
ಲೇಖ-ಕಿ-ಯಾಗಿ
ಲೇಖ-ಕಿ-ಯೊಬ್ಬಳ
ಲೇಖನ
ಲೇಖ-ನ-ಗಳ
ಲೇಖ-ನ-ಗ-ಳನ್ನು
ಲೇಖನದ
ಲೇಖ-ನ-ದಲ್ಲಿ
ಲೇಖ-ನ-ಮಾ-ಲೆಯ
ಲೇಖ-ನ-ವನ್ನು
ಲೇಖನವೇ
ಲೇಖ-ನಿ-ಯನ್ನು
ಲೇಲೆ
ಲೇಸನ್ನು
ಲೇಸ-ಬ-ಯ-ಸು-ವುದು
ಲೇಸು
ಲೈಂಗಿಕ
ಲೈಂಗಿ-ಕ-ತೆಗೆ
ಲೈಂಗಿ-ಕ-ತೆಯ
ಲೈಂಗಿ-ಕ-ತೆ-ಯಲ್ಲಿ
ಲೈಂಗಿ-ಕ-ತೆ-ಯೊಂದಿಗೆ
ಲೈನ್
ಲೈನ್ನ
ಲೊಜನೋವ್
ಲೋಕ
ಲೋಕಕ್ಕೆ
ಲೋಕಗಳ
ಲೋಕ-ಗ-ಳನ್ನು
ಲೋಕ-ಗ-ಳಲ್ಲಿ
ಲೋಕದ
ಲೋಕದಲ್ಲಿ
ಲೋಕದಿಂದ
ಲೋಕವನ್ನು
ಲೋಕವನ್ನೇ
ಲೋಕಾಂತರ
ಲೋಕಾ-ಭಿ-ರಾ-ಮ-ವಾಗಿ
ಲೋಚನಕ್ಕೆ
ಲೋಟ
ಲೋಭ
ಲೋಭವು
ಲೋಹಗಳ
ಲೌಕಿಕ
ಲೌಕಿ-ಕ-ಇ-ವು-ಗಳ
ಲ್ಪಟ್ಟ
ಲ್ಯಾಂಡಿಗೆ
ಲ್ಯಾಟಿನ್
ಲ್ಯಾಮರ್ಸ್
ಲ್ಯಾಮರ್ಸ್ಗೆ
ಲ್ಯಾಮರ್ಸ್ನ
ಲ್ಲಿ
ಳಿಕೆಯ
ವಂಚಕರ
ವಂಚನೆ
ವಂಚ-ನೆ-ಗ-ಳನ್ನು
ವಂಚ-ನೆ-ಗ-ಳಿಂದ
ವಂಚನೆಯ
ವಂಚಿ-ತ-ರಾಗಿ
ವಂಚಿ-ತ-ರಾ-ಗುತ್ತಾ-ರಷ್ಟೆ
ವಂಚಿ-ತ-ರಾದ
ವಂಚಿ-ತ-ಳಾದ
ವಂಚಿಸಲು
ವಂಚಿಸಿ
ವಂತನೆ
ವಂತರು
ವಂತ-ರೆ-ನಿ-ಸಿ-ಕೊಂಡ
ವಂತಾಗಲಿ
ವಂತಿಕೆ
ವಂತಿಕೆಯ
ವಂತೆ
ವಂಥ
ವಂಥ-ವು-ಗಳು
ವಂದನೆ
ವಂದ-ನೆ-ಗಳು
ವಂದಿಸಿ
ವಂದಿ-ಸಿ-ದರು
ವಂಶ
ವಂಶ-ಗ-ತ-ವಾಗಿ
ವಂಶಜರು
ವಂಶ-ನಾ-ಶದ
ವಂಶ-ಪ-ರಂಪ-ರೆಯ
ವಂಶ-ಪಾ-ರಂಪರ್ಯ-ವೆಂಬ
ವಂಶ-ವಾ-ಹ-ಕ-ಗಳ
ವಂಶ-ವಾ-ಹ-ಕ-ಗಳೂ
ವಂಶ-ವಾ-ಹಿ-ಗಳ
ವಂಶ-ವಾ-ಹಿ-ಗ-ಳಾಗಿ
ವಂಶಸ್ಥರು
ವಂಶಾ-ಭಿ-ವೃದ್ಧಿಯ
ವಚನ
ವಚ-ನ-ದಂತೆ
ವಚ-ನ-ವಿತ್ತ
ವಚನವೇ
ವಚ-ನಾ-ಮೃ-ತ-ವನ್ನು
ವಚನೀಯ
ವಜ್ರೋ-ಪ-ಮ-ವಾದ
ವಠಾ-ರ-ದಲ್ಲಿ
ವದನ
ವದಾಮಿ
ವನಪರ್ವ
ವನು
ವನ್ನು
ವನ್ನೂ
ವನ್ನೋ
ವಯಸ್ಕ
ವಯಸ್ಕನ
ವಯಸ್ಕ-ನೆಂದೇ
ವಯಸ್ಕ-ನೊಬ್ಬ
ವಯಸ್ಕರ
ವಯಸ್ಕರೂ
ವಯಸ್ಸಾ-ಗುತ್ತ
ವಯಸ್ಸಾದ
ವಯಸ್ಸಾ-ದಂತೆಲ್ಲ
ವಯಸ್ಸಾ-ದರೂ
ವಯಸ್ಸಾ-ದ-ವರ
ವಯಸ್ಸಿಗೂ
ವಯಸ್ಸಿಗೆ
ವಯಸ್ಸಿಗೇ
ವಯಸ್ಸಿನ
ವಯಸ್ಸಿ-ನಲ್ಲಾ-ದರೆ
ವಯಸ್ಸಿ-ನಲ್ಲಿ
ವಯಸ್ಸಿ-ನಲ್ಲೇ
ವಯಸ್ಸಿ-ನ-ವ-ನಿದ್ದಾಗ
ವಯಸ್ಸಿ-ನ-ವ-ಳಂತೆ
ವಯಸ್ಸಿ-ನಿಂದಲೇ
ವಯಸ್ಸಿ-ನೊ-ಳಗೆ
ವಯಸ್ಸು
ವಯೋವೃದ್ಧ
ವರ
ವರಗಳು
ವರ-ದ-ಮುದ್ರೆ-ಗಳ
ವರದಿ
ವರದಿ
ವರ-ದಿ-ಗಳು
ವರ-ದಿ-ಗಾ-ರರು
ವರದಿಯ
ವರ-ದಿ-ಯಂತೆ
ವರ-ದಿ-ಯನ್ನು
ವರ-ದಿ-ಯಲ್ಲಿ
ವರ-ದಿ-ಯಾ-ಗಿದೆ
ವರ-ವಾ-ಗುತ್ತದೆ
ವರವಿತ್ತು
ವರಾಂಡ-ದಲ್ಲಿ
ವರಿತು
ವರಿದ
ವರಿ-ಸಿದ್ದರೆ
ವರುಷ
ವರುಷಕ್ಕೆ
ವರು-ಷ-ಗಳ
ವರು-ಷ-ವನ್ನು
ವರೂ
ವರೆ
ವರೆಗಿನ
ವರೆಗೂ
ವರೆಗೆ
ವರೋ
ವರ್ಗ
ವರ್ಗಕ್ಕೆ
ವರ್ಗದ
ವರ್ಗದಲ್ಲಿ
ವರ್ಗ-ದ-ವರು
ವರ್ಗ-ದ-ವರೂ
ವರ್ಗದಷ್ಟು
ವರ್ಗದಿಂದ
ವರ್ಗ-ದೊ-ಡನೆ
ವರ್ಗ-ಭೇ-ದ-ವಿಲ್ಲ
ವರ್ಗ-ವಾ-ಯಿತು
ವರ್ಗ-ಶತ್ರುಕ್ಲಾ-ಸ್ಎ-ನಿ-ಮಿ-ವಾಗಿ
ವರ್ಗಾ-ವ-ಣೆಯ
ವರ್ಗಾವಣೆ
ವರ್ಜನಂ
ವರ್ಜೀನಿಯಾ
ವರ್ಣ
ವರ್ಣ-ಗ-ಳನ್ನು
ವರ್ಣ-ಗ-ಳನ್ನೂ
ವರ್ಣ-ಗ-ಳಲ್ಲಿ
ವರ್ಣಗಳು
ವರ್ಣನೆ
ವರ್ಣ-ನೆ-ಗ-ಳಲ್ಲಿ
ವರ್ಣ-ನೆ-ಗ-ಳಿಂದ
ವರ್ಣ-ನೆ-ಯಲ್ಲ
ವರ್ಣ-ವೈ-ವಿಧ್ಯ-ಗ-ಳನ್ನು
ವರ್ಣಿ-ಸ-ಬಲ್ಲ
ವರ್ಣಿಸಲು
ವರ್ಣಿಸಿದ
ವರ್ಣಿ-ಸಿ-ದಂತೆಯೇ
ವರ್ಣಿ-ಸಿ-ದರು
ವರ್ಣಿ-ಸಿದ್ದಾನೆ
ವರ್ಣಿ-ಸಿದ್ದಾರೆ
ವರ್ಣಿಸುವ
ವರ್ಣಿ-ಸು-ವ-ವರು
ವರ್ತ
ವರ್ತಕ
ವರ್ತನೆ
ವರ್ತ-ನೆ-ಗಳ
ವರ್ತ-ನೆ-ಗ-ಳನ್ನು
ವರ್ತ-ನೆ-ಗ-ಳನ್ನೂ
ವರ್ತ-ನೆ-ಗ-ಳಲ್ಲಿ
ವರ್ತ-ನೆ-ಗ-ಳಿಂದ
ವರ್ತ-ನೆ-ಗಳು
ವರ್ತ-ನೆ-ಗ-ಳುಂಟಾ-ಗುತ್ತವೆ
ವರ್ತನೆಗೆ
ವರ್ತನೆಯ
ವರ್ತ-ನೆ-ಯಂತೂ
ವರ್ತ-ನೆ-ಯನ್ನು
ವರ್ತ-ನೆ-ಯಲ್ಲಿ
ವರ್ತ-ನೆ-ಯಿಂದ
ವರ್ತನೆಯು
ವರ್ತಮಾನ
ವರ್ತ-ಮಾ-ನದ
ವರ್ತ-ಮಾ-ನ-ದಲ್ಲಿ
ವರ್ತ-ಮಾ-ನ-ದಲ್ಲೇ
ವರ್ತ-ಮಾ-ನ-ವನ್ನು
ವರ್ತಿ
ವರ್ತಿ-ಸ-ತೊ-ಡ-ಗಿದ
ವರ್ತಿ-ಸ-ತೊ-ಡ-ಗಿ-ದಳು
ವರ್ತಿಸತ್ತ
ವರ್ತಿ-ಸ-ಬಲ್ಲದು
ವರ್ತಿ-ಸ-ಬಲ್ಲೆ
ವರ್ತಿ-ಸ-ಬ-ಹುದು
ವರ್ತಿ-ಸ-ಬಾ-ರದು
ವರ್ತಿ-ಸ-ಲಾ-ರದು
ವರ್ತಿಸಲು
ವರ್ತಿಸಿ
ವರ್ತಿಸಿತು
ವರ್ತಿಸಿದ
ವರ್ತಿ-ಸಿ-ದಲ್ಲಿ
ವರ್ತಿ-ಸಿ-ದ-ವರು
ವರ್ತಿ-ಸಿ-ದಾಗ
ವರ್ತಿ-ಸಿ-ದಾ-ಗಲೂ
ವರ್ತಿ-ಸಿ-ದಿರಿ
ವರ್ತಿ-ಸಿ-ದೆ-ಇವೇ
ವರ್ತಿಸಿದ್ದ
ವರ್ತಿ-ಸಿದ್ದರೂ
ವರ್ತಿ-ಸುತ್ತದೆ
ವರ್ತಿ-ಸುತ್ತಾ-ನಷ್ಟೆ
ವರ್ತಿ-ಸುತ್ತಾನೆ
ವರ್ತಿ-ಸುತ್ತಾರೆ
ವರ್ತಿ-ಸುತ್ತಾ-ರೆಯೆ
ವರ್ತಿ-ಸುತ್ತಿದ್ದ
ವರ್ತಿ-ಸುತ್ತೇವೆ
ವರ್ತಿಸುವ
ವರ್ತಿ-ಸು-ವಂತಾ-ಯಿತು
ವರ್ತಿ-ಸು-ವಂತೆ
ವರ್ತಿ-ಸು-ವ-ವರ
ವರ್ತಿ-ಸು-ವ-ವರು
ವರ್ತಿ-ಸು-ವುದು
ವರ್ತಿ-ಸು-ವುವೋ
ವರ್ಧಿ-ಸುತ್ತದೆ
ವರ್ಧಿ-ಸುತ್ತ-ಲಿವೆ
ವರ್ಧಿಸುವ
ವರ್ವಾರಾ
ವರ್ಷ
ವರ್ಷಕ್ಕೆ
ವರ್ಷಕ್ಕೊಂದು
ವರ್ಷಕ್ಕೊಂದೆ-ರಡು
ವರ್ಷ-ಗಟ್ಟಲೆ
ವರ್ಷಗಳ
ವರ್ಷ-ಗ-ಳಲ್ಲ
ವರ್ಷ-ಗ-ಳಲ್ಲಿ
ವರ್ಷ-ಗ-ಳ-ವ-ರೆಗೆ
ವರ್ಷ-ಗ-ಳಷ್ಟಾ-ದರೂ
ವರ್ಷ-ಗ-ಳಷ್ಟು
ವರ್ಷ-ಗ-ಳಾ-ಗುತ್ತಲೇ
ವರ್ಷ-ಗ-ಳಾ-ಗುತ್ತವೆ
ವರ್ಷ-ಗ-ಳಾ-ಗು-ವಾಗ
ವರ್ಷ-ಗ-ಳಾ-ದರೂ
ವರ್ಷ-ಗ-ಳಾ-ದವು
ವರ್ಷ-ಗ-ಳಿಂದ
ವರ್ಷ-ಗ-ಳಿಂದೀ-ಚೆಗೆ
ವರ್ಷ-ಗ-ಳಿ-ಗೊಮ್ಮೆ
ವರ್ಷಗಳು
ವರ್ಷಗಳೇ
ವರ್ಷದ
ವರ್ಷದಲ್ಲಿ
ವರ್ಷದಲ್ಲೇ
ವರ್ಷ-ದ-ವ-ರಿದ್ದಾಗ
ವರ್ಷ-ದ-ವ-ಳಿದ್ದಾಗ
ವರ್ಷದಿಂದ
ವರ್ಷವಷ್ಟೇ
ವರ್ಷವೇ
ವರ್ಷ-ವೊಂದಕ್ಕೆ
ವರ್ಷ-ವೊಂದ-ರಲ್ಲಿ
ವರ್ಷಾಂತ್ಯದ
ವಲ-ಸೆ-ಬಂದಿದ್ದುದೇ
ವಲ್ಲ-ಭ-ಬಾಯಿ
ವವನು
ವಶದಲ್ಲಿ
ವಶ-ದಲ್ಲಿಟ್ಟು
ವಶ-ಪ-ಡಿ-ಸಿ-ಕೊಳ್ಳಲು
ವಶ-ಮಾ-ಡಿ-ಕೊಳ್ಳುವ
ವಶ-ವರ್ತಿ-ಗ-ಳಾ-ಗು-ವುವು
ವಶ-ವಾ-ಯಿತು
ವಶೀ-ಕ-ರಣ
ವಶೀ-ಕ-ರ-ಣದ
ವಶ್ಯಸುಪ್ತಿ
ವಶ್ಯ-ಸುಪ್ತಿ-ಗೊ-ಳ-ಪ-ಡಿ-ಸ-ಬೇ-ಕಾದ
ವಶ್ಯ-ಸುಪ್ತಿಗೆ
ವಶ್ಯ-ಸುಪ್ತಿ-ಗೊ-ಳ-ಗಾದ
ವಶ್ಯ-ಸುಪ್ತಿ-ಗೊ-ಳ-ಗಾ-ದಾಗ
ವಶ್ಯ-ಸುಪ್ತಿ-ಗೊ-ಳ-ಪಟ್ಟ
ವಶ್ಯ-ಸುಪ್ತಿ-ಗೊ-ಳ-ಪಟ್ಟಾಗ
ವಶ್ಯ-ಸುಪ್ತಿ-ಗೊ-ಳ-ಪ-ಡಿಸಿ
ವಶ್ಯ-ಸುಪ್ತಿಯ
ವಶ್ಯ-ಸುಪ್ತಿ-ಯಂಥ
ವಶ್ಯ-ಸುಪ್ತಿ-ಯಲ್ಲಿ
ವಶ್ಯ-ಸುಪ್ತಿ-ಯಲ್ಲಿದ್ದ
ವಶ್ಯ-ಸುಪ್ತಿ-ಯಿಂದ
ವಷ್ಟು
ವಸಡಿಗೆ
ವಸ-ಡಿ-ನಿಂದ
ವಸತಿಯ
ವಸತಿಯೇ
ವಸಿಷ್ಠ-ಮ-ಹರ್ಷಿ-ಗಳು
ವಸ್ತು
ವಸ್ತುಗಳ
ವಸ್ತು-ಗ-ಳನ್ನು
ವಸ್ತು-ಗ-ಳನ್ನೂ
ವಸ್ತು-ಗ-ಳನ್ನೇ
ವಸ್ತು-ಗ-ಳಲ್ಲಿ
ವಸ್ತು-ಗ-ಳಲ್ಲೂ
ವಸ್ತು-ಗ-ಳಾಗಿ
ವಸ್ತು-ಗ-ಳಿಂದ
ವಸ್ತು-ಗ-ಳಿಗೆ
ವಸ್ತುಗಳು
ವಸ್ತುಗಳೂ
ವಸ್ತು-ಗ-ಳೆ-ಡೆಗೆ
ವಸ್ತು-ಗ-ಳೆಲ್ಲ
ವಸ್ತುಪ್ರ-ಪಂಚದ
ವಸ್ತು-ವನ್ನಾಗಿ
ವಸ್ತು-ವನ್ನಾ-ಗಿ-ಸುವ
ವಸ್ತುವನ್ನು
ವಸ್ತುವನ್ನೂ
ವಸ್ತು-ವಾ-ಗಲೀ
ವಸ್ತುವಾಗಿ
ವಸ್ತು-ವಾ-ಗಿಯೂ
ವಸ್ತು-ವಿಜ್ಞಾ-ನದ
ವಸ್ತುವಿನ
ವಸ್ತು-ವಿ-ನಲ್ಲಿ
ವಸ್ತು-ವಿ-ನಲ್ಲಿಲ್ಲ
ವಸ್ತು-ವಿ-ನಲ್ಲೇ
ವಸ್ತುವು
ವಸ್ತುವೂ
ವಸ್ತುವೆಂಬ
ವಸ್ತುವ್ಯಾ-ಮೋ-ಹವೂ
ವಸ್ತು-ಸಾ-ಗ-ರ-ದಲ್ಲಿ
ವಸ್ತುಸ್ಥಿತಿ
ವಸ್ತುಸ್ಥಿ-ತಿಗೂ
ವಸ್ತುಸ್ಥಿ-ತಿಯ
ವಸ್ತುಸ್ಥಿ-ತಿ-ಯನ್ನು
ವಹಿ-ಸ-ಬೇಕು
ವಹಿಸಿ
ವಹಿ-ಸಿ-ಕೊಂಡ
ವಹಿ-ಸಿ-ಕೊಂಡಿತು
ವಹಿ-ಸಿ-ಕೊಟ್ಟಿದೆ
ವಹಿಸಿದ
ವಹಿ-ಸಿದ್ದೇನೆ
ವಹಿ-ಸುತ್ತದೆ
ವಹಿ-ಸುತ್ತವೆ
ವಹಿ-ಸುತ್ತಾನೆ
ವಹಿ-ಸುತ್ತಿದ್ದೆ
ವಹಿಸುವ
ವಾ
ವಾಂತಿ
ವಾಂತಿ-ಯಾ-ಗುತ್ತಿತ್ತು
ವಾಂಪಾಯರ್
ವಾಂಬಾಕ್
ವಾಕಿಂಗ್
ವಾಕ್
ವಾಕ್ಕಾಗಿ
ವಾಕ್ಚಾತುರ್ಯ
ವಾಕ್ಯ
ವಾಕ್ಯ-ಗ-ಳನ್ನು
ವಾಕ್ಯ-ಗ-ಳನ್ನೂ
ವಾಕ್ಯ-ಗ-ಳಲ್ಲಿ
ವಾಕ್ಯ-ಗ-ಳಾ-ವುವು
ವಾಕ್ಯ-ಗ-ಳಿಂದ
ವಾಕ್ಯಗಳೂ
ವಾಕ್ಯ-ಗ-ಳೊಂದಿಗೆ
ವಾಕ್ಯದ
ವಾಕ್ಯದಲ್ಲಿ
ವಾಕ್ಯ-ಪುಂಜ-ವನ್ನೋ
ವಾಕ್ಯವನ್ನು
ವಾಕ್ಶಕ್ತಿ
ವಾಗದಂತೆ
ವಾಗ-ಬ-ಹು-ದಾ-ದಂಥ
ವಾಗಲಿ
ವಾಗಲೇ
ವಾಗುತ್ತಿ-ರ-ಲಿಲ್ಲ
ವಾಗುತ್ತಿಲ್ಲ
ವಾಗುವುದು
ವಾಗ್ಗೇ-ಯ-ಕಾರ
ವಾಗ್ದೇವಿ
ವಾಗ್ಮಿ-ಗ-ಳಿ-ರಲಿ
ವಾಗ್ಮಿ-ಯಾ-ಗಲು
ವಾಗ್ಮಿ-ಯೊಬ್ಬನ
ವಾಗ್ಯುದ್ಧ
ವಾಗ್ವೈ-ಖ-ರಿಯ
ವಾಙ್ಮಯ
ವಾಚು
ವಾಚ್ಯಾರ್ಥ-ದಿಂದ
ವಾಟರ್
ವಾಟ್
ವಾಡಿಕೆ
ವಾಡಿ-ಕೆ-ಯಲ್ಲಿಲ್ಲದ
ವಾಣಿ
ವಾಣಿ-ಗ-ಳಿಂದ
ವಾಣಿಜ್ಯೋದ್ಯಮ
ವಾಣಿಯ
ವಾಣಿಯನ್ನು
ವಾಣಿಯಲ್ಲಿ
ವಾಣಿ-ಯಲ್ಲಿ-ಡುವ
ವಾತಾ-ರ-ವಣ
ವಾತಾ-ವ-ರಣ
ವಾತಾ-ವ-ರ-ಣ-ಇವು
ವಾತಾ-ವ-ರ-ಣಕ್ಕೆ
ವಾತಾ-ವ-ರ-ಣ-ಗ-ಳಿದ್ದರೂ
ವಾತಾ-ವ-ರ-ಣ-ಜ-ನ-ಗ-ಳೊ-ಡನೆ
ವಾತಾ-ವ-ರ-ಣದ
ವಾತಾ-ವ-ರ-ಣ-ದಂತೆ
ವಾತಾ-ವ-ರ-ಣ-ದಲ್ಲಿ
ವಾತಾ-ವ-ರ-ಣ-ದಲ್ಲಿಡಿ
ವಾತಾ-ವ-ರ-ಣ-ದಲ್ಲಿ-ಡು-ವುದೆ
ವಾತಾ-ವ-ರ-ಣ-ದೆ-ಡೆಗೆ
ವಾತಾ-ವ-ರ-ಣ-ವನ್ನು
ವಾತಾ-ವ-ರ-ಣ-ವನ್ನೂ
ವಾತಾ-ವ-ರ-ಣ-ವಿದ್ದಂತೆ
ವಾತಾ-ವ-ರ-ಣ-ವಿದ್ದರೂ
ವಾತಾ-ವ-ರ-ಣವೂ
ವಾತಾ-ವ-ರಣ್
ವಾತ್ಸಲ್ಯ
ವಾತ್ಸಲ್ಯ-ಗಳ
ವಾತ್ಸಲ್ಯದ
ವಾತ್ಸಲ್ಯ-ದಿಂದ
ವಾತ್ಸಲ್ಯ-ಪೂ-ರಿತ
ವಾದ
ವಾದಕ್ಕೆ
ವಾದದ
ವಾದ-ದಕ್ಷತೆ
ವಾದದಲ್ಲಿ
ವಾದ-ನ-ದಲ್ಲಿ
ವಾದ-ನ-ಪ-ಟು-ಗಳ
ವಾದಪ್ರಿಯ
ವಾದರೂ
ವಾದ-ವಾ-ಗುತ್ತ
ವಾದವೂ
ವಾದಿರಾಜ
ವಾದಿ-ಸ-ಬ-ಹುದು
ವಾದೀತು
ವಾದುದು
ವಾದ್ಯ
ವಾದ್ಯದ
ವಾಮಮಾರ್ಗ
ವಾಯವ್ಯ-ದಲ್ಲಿ-ರುವ
ವಾಯಿತು
ವಾಯು
ವಾಯು-ಇ-ವೆಲ್ಲ-ವು-ಗಳ
ವಾರ
ವಾರಕ್ಕೊಮ್ಮೆ
ವಾರಗಳ
ವಾರ-ಗ-ಳಲ್ಲಿ
ವಾರ-ಗ-ಳಲ್ಲೇ
ವಾರದ
ವಾರದಲ್ಲಿ
ವಾರ-ದ-ವ-ರೆಗೆ
ವಾರ-ದಿಂದಲೇ
ವಾರ-ಪತ್ರಿ-ಕೆ-ಯಲ್ಲಿ
ವಾರಾ-ಣ-ಸಿಯ
ವಾರಿಧಿ
ವಾರ್ಡಿನಲ್ಲಿ
ವಾರ್ಡಿನಲ್ಲೇ
ವಾರ್ತೆ
ವಾರ್ತೆ-ಗ-ಳನ್ನು
ವಾರ್ತೆ-ಗ-ಳಲ್ಲಿ
ವಾರ್ತೆಯನ್ನು
ವಾರ್ಧಕ್ಯ-ದಲ್ಲಿ
ವಾರ್ಫೇರ್
ವಾಲಿಕೊಂಡು
ವಾಲು-ವ-ವರು
ವಾಲ್ಟ್
ವಾಲ್ಮೀಕಿ
ವಾಶಿಂಗ್ಟನ್ನದು
ವಾಷಿಂಗ್ಟನ್
ವಾಸ
ವಾಸಮಾಡು
ವಾಸದ
ವಾಸನೆ
ವಾಸ-ನೆ-ಗಳ
ವಾಸ-ನೆ-ಗ-ಳಿಂದ
ವಾಸ-ಮಾ-ಡ-ಬೇ-ಕೆಂದು
ವಾಸ-ಮಾ-ಡುತ್ತಾನೆ
ವಾಸ-ವಾ-ಗಿತ್ತು
ವಾಸ-ವಾ-ಗಿದ್ದ
ವಾಸ-ವಾ-ಗಿ-ರುವ
ವಾಸಸ್ಥಾನ
ವಾಸಿ
ವಾಸಿ-ಮಾ-ಡ-ಬ-ಹುದು
ವಾಸಿ-ಯಾ-ಗ-ಲಿಲ್ಲ
ವಾಸಿ-ಯಾ-ಗುತ್ತದೆ
ವಾಸಿ-ಯಾ-ಗು-ವುದು
ವಾಸಿಯೂ
ವಾಸಿ-ಸ-ತೊ-ಡ-ಗಿ-ದರು
ವಾಸಿ-ಸ-ಬೇ-ಕಾ-ಗಿದ್ದ
ವಾಸಿಸಿದ
ವಾಸಿಸುತ್ತ
ವಾಸಿ-ಸುತ್ತಾರೆ
ವಾಸಿ-ಸುತ್ತಿದ್ದ
ವಾಸಿ-ಸುತ್ತಿದ್ದೇನೆ
ವಾಸಿ-ಸುತ್ತಿ-ರುವ
ವಾಸಿಸುವ
ವಾಸುದೇವ
ವಾಸು-ದೇ-ವ-ಪುರ
ವಾಸು-ದೇ-ವಾ-ಚಾರ್ಯರು
ವಾಸ್ತವ
ವಾಸ್ತ-ವ-ತೆಯ
ವಾಸ್ತವವೂ
ವಾಸ್ತವಿಕ
ವಾಸ್ತ-ವಿ-ಕ-ವಾಗಿ
ವಾಸ್ತು-ಶಿಲ್ಪ-ಗ-ಳಾ-ಗಲಿ
ವಾಸ್ತು-ಶಿಲ್ಪಕ್ಕೆ
ವಾಹನ
ವಾಹ-ನಕ್ಕೆ-ದು-ರಾಗಿ
ವಾಹ-ನ-ಗಳ
ವಾಹ-ನ-ಗ-ಳನ್ನು
ವಾಹ-ನ-ವಿದೆ
ವಾಹ-ನ-ವೇರಿ
ವಿ
ವಿಂಗ-ಡ-ವಾಗಿ
ವಿಂಗ-ಡ-ವಾ-ಗಿ-ರುವ
ವಿಂಗ-ಡ-ವಾ-ಗುವ
ವಿಂಗ-ಡ-ವಾ-ಗು-ವುದು
ವಿಂಗಡಿ
ವಿಂಗಡಿಸಿ
ವಿಂಗ-ಡಿ-ಸಿದ
ವಿಂಗ-ಡಿ-ಸುತ್ತದೆ
ವಿಂಚಿಯ
ವಿಂಟರ್
ವಿಕಟ
ವಿಕಸನ
ವಿಕ-ಸಿ-ತ-ವಾ-ಗುವ
ವಿಕಾರ
ವಿಕಾ-ರ-ಮಾ-ಡಿ-ಕೊಂಡು
ವಿಕಾ-ರ-ವಾ-ಗದೆ
ವಿಕಾರವೂ
ವಿಕಾಸ
ವಿಕಾಸಕ್ಕೆ
ವಿಕಾ-ಸ-ಗ-ಳನ್ನ-ವ-ಲಂಬಿಸಿ
ವಿಕಾ-ಸ-ಗೊ-ಳಿ-ಸಲು
ವಿಕಾಸದ
ವಿಕಾ-ಸ-ಪ-ಥ-ದಲ್ಲಿ
ವಿಕಾ-ಸ-ವನ್ನೂ
ವಿಕಾ-ಸ-ವಾ-ದದ
ವಿಕಾ-ಸ-ವಾ-ದವು
ವಿಕಾ-ಸ-ವೆನ್ನು-ವುದು
ವಿಕಾಸವೇ
ವಿಕಾ-ಸ-ಹೊಂದಿ
ವಿಕಿ-ರ-ಣದ
ವಿಕಿ-ರ-ಣ-ಶೀ-ಲದ್ದು
ವಿಕೋಪಕ್ಕೆ
ವಿಕೋ-ಪ-ದಿಂದ
ವಿಕ್ಟರ್
ವಿಕ್ರಮ
ವಿಖ್ಯಾತ
ವಿಖ್ಯಾ-ತ-ರಾ-ದರು
ವಿಖ್ಯಾತರೂ
ವಿಗ್ರ-ಹ-ದಲ್ಲೇ
ವಿಘ್ನ
ವಿಘ್ನ-ಗ-ಳನ್ನು
ವಿಘ್ನ-ಗ-ಳಾದ
ವಿಘ್ನ-ಗ-ಳಿಗೆ
ವಿಘ್ನ-ವಿ-ರೋ-ಧ-ಗಳ
ವಿಚಕ್ಷ-ಣ-ತೆ-ಯನ್ನು
ವಿಚ-ಲಿ-ತ-ನಾ-ಗದೆ
ವಿಚ-ಲಿ-ತ-ನಾ-ಗು-ವು-ದಾ-ಗಲೀ
ವಿಚ-ಲಿ-ತ-ನಾದ
ವಿಚ-ಲಿ-ತ-ರಾ-ಗ-ದಂತೆ
ವಿಚ-ಲಿ-ತ-ರಾ-ಗದೆ
ವಿಚಾರ
ವಿಚಾ-ರಕ್ಕಿಂತ
ವಿಚಾರಕ್ಕೆ
ವಿಚಾ-ರ-ಗಳ
ವಿಚಾ-ರ-ಗ-ಳನ್ನು
ವಿಚಾ-ರ-ಗ-ಳನ್ನೂ
ವಿಚಾ-ರ-ಗ-ಳನ್ನೊ-ಳ-ಗೊಂಡ
ವಿಚಾ-ರ-ಗ-ಳಲ್ಲಿ
ವಿಚಾ-ರ-ಗ-ಳಲ್ಲೇ
ವಿಚಾ-ರ-ಗ-ಳಿಂದ
ವಿಚಾ-ರ-ಗ-ಳಿ-ಗಿಂತಲೂ
ವಿಚಾ-ರ-ಗ-ಳಿಗೂ
ವಿಚಾ-ರ-ಗ-ಳಿಗೆ
ವಿಚಾ-ರ-ಗಳು
ವಿಚಾ-ರ-ಗ-ಳೆಲ್ಲ
ವಿಚಾ-ರ-ಣೆಯ
ವಿಚಾರದ
ವಿಚಾ-ರ-ದಲ್ಲಿ
ವಿಚಾ-ರ-ಧಾ-ರೆಯ
ವಿಚಾ-ರ-ಧಾ-ರೆ-ಯನ್ನು
ವಿಚಾ-ರ-ಧಾ-ರೆ-ಯಿಂದ
ವಿಚಾ-ರಪ್ರಿ-ಯರೂ
ವಿಚಾ-ರ-ಯುಕ್ತಿ
ವಿಚಾ-ರ-ಲ-ಹ-ರಿಗೆ
ವಿಚಾ-ರ-ವಂತ
ವಿಚಾ-ರ-ವಂತರ
ವಿಚಾ-ರ-ವಂತ-ರನ್ನೂ
ವಿಚಾ-ರ-ವಂತ-ರಲ್ಲಿ
ವಿಚಾ-ರ-ವಂತ-ರಾರೂ
ವಿಚಾ-ರ-ವಂತ-ರಿಗೂ
ವಿಚಾ-ರ-ವಂತ-ರಿಗೆ
ವಿಚಾ-ರ-ವಂತರು
ವಿಚಾ-ರ-ವಂತರೂ
ವಿಚಾ-ರ-ವಂತ-ರೆ-ನಿ-ಸಿ-ಕೊಂಡ-ವರು
ವಿಚಾ-ರ-ವಂತ-ರೆಲ್ಲರೂ
ವಿಚಾ-ರ-ವಂತ-ರೊಬ್ಬರು
ವಿಚಾ-ರ-ವಂತೂ
ವಿಚಾ-ರ-ವನ್ನು
ವಿಚಾ-ರ-ವನ್ನೂ
ವಿಚಾ-ರ-ವನ್ನೇ
ವಿಚಾ-ರ-ವಲ್ಲವೆ
ವಿಚಾ-ರ-ವಾ-ಗಲಿ
ವಿಚಾ-ರ-ವಾಗಿ
ವಿಚಾ-ರ-ವಾ-ಗಿದೆ
ವಿಚಾ-ರ-ವಾ-ಗಿಯೇ
ವಿಚಾ-ರ-ವಾದ
ವಿಚಾ-ರ-ವಾ-ದಕ್ಕೆ
ವಿಚಾ-ರ-ವಾ-ದದ
ವಿಚಾ-ರ-ವಾದಿ
ವಿಚಾ-ರ-ವಾ-ದಿ-ಗ-ಳಿಗೆ
ವಿಚಾ-ರ-ವಾ-ದಿ-ಗಳು
ವಿಚಾ-ರ-ವಾ-ದಿ-ಗಳೂ
ವಿಚಾ-ರ-ವಾ-ದು-ದ-ರಿಂದ
ವಿಚಾ-ರ-ವಾ-ಹಿ-ನಿಯ
ವಿಚಾ-ರ-ವಿದೆ
ವಿಚಾ-ರ-ವಿ-ಧಾನ
ವಿಚಾ-ರ-ವಿ-ಮರ್ಶೆಯ
ವಿಚಾರವೂ
ವಿಚಾರವೆ
ವಿಚಾ-ರ-ವೆಂಬ
ವಿಚಾರವೇ
ವಿಚಾ-ರ-ಶಕ್ತಿ-ಯನ್ನೂ
ವಿಚಾ-ರ-ಶಕ್ತಿಯೂ
ವಿಚಾ-ರ-ಶೀ-ಲ-ರೆಲ್ಲ
ವಿಚಾ-ರ-ಸಂಗ್ರಹ
ವಿಚಾ-ರ-ಹೀ-ನರೂ
ವಿಚಾ-ರಿ-ಸ-ಲಾಗಿ
ವಿಚಾರಿಸಿ
ವಿಚಾ-ರಿ-ಸಿ-ದರು
ವಿಚಾ-ರಿ-ಸಿ-ದಾಗ
ವಿಚಾ-ರಿ-ಸಿದ್ದರು
ವಿಚಾ-ರಿ-ಸಿದ್ದೆ
ವಿಚಾ-ರಿ-ಸು-ವ-ವ-ರಿಗೆ
ವಿಚಾ-ರಿ-ಸು-ವ-ವ-ರಿಲ್ಲ
ವಿಚಿತ್ರ
ವಿಚಿತ್ರ-ವಾಗಿ
ವಿಚಿತ್ರ-ವಾದ
ವಿಚಿತ್ರ-ವಾ-ದರೂ
ವಿಚಿತ್ರವೂ
ವಿಚ್ಛಿದ್ರ-ಕಾ-ರಕ
ವಿಚ್ಛೇದ
ವಿಚ್ಛೇದಕ್ಕೆ
ವಿಚ್ಛೇದದ
ವಿಚ್ಛೇ-ದ-ನಕ್ಕೂ
ವಿಚ್ಛೇ-ದ-ವಾಗಿ
ವಿಚ್ಛೇದವೂ
ವಿಚ್ಯು-ತಿ-ಗೊ-ಳಿ-ಸ-ಬೇಡ
ವಿಜಯ
ವಿಜಯಕ್ಕೆ
ವಿಜ-ಯ-ಗಳ
ವಿಜ-ಯ-ಗ-ಳನ್ನು
ವಿಜ-ಯ-ಗಳು
ವಿಜಯದ
ವಿಜ-ಯ-ದಾ-ಸ-ರನ್ನೂ
ವಿಜ-ಯ-ದಾ-ಸರು
ವಿಜ-ಯ-ದಾ-ಸ-ರೆ-ಡೆಗೆ
ವಿಜ-ಯ-ಪ-ತಾ-ಕೆ-ಯನ್ನು
ವಿಜ-ಯ-ವನ್ನು
ವಿಜ-ಯ-ವನ್ನೂ
ವಿಜಯವೂ
ವಿಜ-ಯಿ-ಗ-ಳಾ-ಗು-ವು-ದಿಲ್ಲ
ವಿಜ-ಯಿ-ಯಾದ
ವಿಜೃಂಭಣೆ
ವಿಜೃಂಭಿಸ
ವಿಜೃಂಭಿ-ಸುವ
ವಿಜೇತ
ವಿಜೇ-ತ-ರಾದ
ವಿಜೇತರು
ವಿಜ್ಞಾನ
ವಿಜ್ಞಾ-ನ-ಇ-ವು-ಗ-ಳನ್ನು
ವಿಜ್ಞಾನಕ್ಕೆ
ವಿಜ್ಞಾ-ನ-ಗಳ
ವಿಜ್ಞಾನದ
ವಿಜ್ಞಾ-ನ-ದಲ್ಲಿ
ವಿಜ್ಞಾ-ನ-ನಿಷ್ಠ-ನಾ-ಗಿದ್ದು-ಕೊಂಡು
ವಿಜ್ಞಾ-ನ-ಯು-ಗದ
ವಿಜ್ಞಾ-ನ-ಯು-ಗ-ದಲ್ಲಿ
ವಿಜ್ಞಾ-ನ-ವನ್ನು
ವಿಜ್ಞಾ-ನ-ವಿಂದು
ವಿಜ್ಞಾ-ನ-ವಿನ್ನೂ
ವಿಜ್ಞಾ-ನ-ವಿಲ್ಲದ
ವಿಜ್ಞಾನವು
ವಿಜ್ಞಾ-ನ-ವೆಂದು
ವಿಜ್ಞಾ-ನ-ಶಾಸ್ತ್ರ-ಗ-ಳನ್ನು
ವಿಜ್ಞಾ-ನ-ಶಾಸ್ತ್ರ-ವಾ-ಗಲಿ
ವಿಜ್ಞಾ-ನ-ಶಾಸ್ತ್ರವೇ
ವಿಜ್ಞಾ-ನಾ-ನಂದಜಿ
ವಿಜ್ಞಾನಿ
ವಿಜ್ಞಾ-ನಿ-ಗಳ
ವಿಜ್ಞಾ-ನಿ-ಗ-ಳನ್ನೂ
ವಿಜ್ಞಾ-ನಿ-ಗ-ಳಲ್ಲಿ
ವಿಜ್ಞಾ-ನಿ-ಗ-ಳಾ-ಗ-ಬೇ-ಕಿಲ್ಲ
ವಿಜ್ಞಾ-ನಿ-ಗ-ಳಿಂದಲೇ
ವಿಜ್ಞಾ-ನಿ-ಗ-ಳಿಗೂ
ವಿಜ್ಞಾ-ನಿ-ಗ-ಳಿಗೆ
ವಿಜ್ಞಾ-ನಿ-ಗಳು
ವಿಜ್ಞಾ-ನಿ-ಗಳೂ
ವಿಜ್ಞಾ-ನಿ-ಗಳೇ
ವಿಜ್ಞಾನಿಗೂ
ವಿಜ್ಞಾನಿಯ
ವಿಜ್ಞಾ-ನಿ-ಯಾ-ಗಲಿ
ವಿಜ್ಞಾ-ನಿ-ಯಾಗಿ
ವಿಜ್ಞಾ-ನಿ-ಯಾತ
ವಿಜ್ಞಾನಿಯು
ವಿಜ್ಞಾನಿಯೂ
ವಿಟಮಿನ್
ವಿಟ್ಟಿ-ರು-ವಿ-ರೆಂದು
ವಿಟ್ಟು
ವಿಟ್ಠಲ
ವಿಠಲನ
ವಿಡಂಬ-ನೆಯ
ವಿಡಿಯೋ
ವಿಡೀ
ವಿತರ್ಕ-ಗ-ಳನ್ನು
ವಿತ್ತ
ವಿತ್ತ-ಇ-ವು-ಗ-ಳನ್ನೆಲ್ಲ
ವಿತ್ತದಲ್ಲಿ
ವಿದುರ
ವಿದುರನ
ವಿದೆ
ವಿದೇ-ಶ-ಗ-ಳಲ್ಲಿ
ವಿದೇ-ಶ-ದಲ್ಲಿ
ವಿದೇಶಿ
ವಿದೇಶೀ
ವಿದೇ-ಶೀ-ಯರ
ವಿದೇ-ಶೀ-ಯ-ರನ್ನು
ವಿದೇ-ಹ-ರಾ-ಜ-ನಾದ
ವಿದ್ದು-ದ-ರಿಂದ
ವಿದ್ಧಿ
ವಿದ್ಯಾ
ವಿದ್ಯಾ-ವಂತರೂ
ವಿದ್ಯಾಬುದ್ಧಿ
ವಿದ್ಯಾಭ್ಯಾಸ
ವಿದ್ಯಾಭ್ಯಾ-ಸಕ್ಕಾಗಿ
ವಿದ್ಯಾಭ್ಯಾ-ಸಕ್ಕೆ
ವಿದ್ಯಾಭ್ಯಾ-ಸಕ್ಕೆಂದು
ವಿದ್ಯಾಭ್ಯಾ-ಸದ
ವಿದ್ಯಾಭ್ಯಾ-ಸ-ದಲ್ಲಿ
ವಿದ್ಯಾಭ್ಯಾ-ಸ-ವನ್ನು
ವಿದ್ಯಾಭ್ಯಾ-ಸವು
ವಿದ್ಯಾಭ್ಯಾ-ಸ-ವೆಂದೇ
ವಿದ್ಯಾರ್ಥಿ
ವಿದ್ಯಾರ್ಥಿ-ಗ-ಳಿ-ಗಿಂತಲೂ
ವಿದ್ಯಾರ್ಥಿ-ಗಳ
ವಿದ್ಯಾರ್ಥಿ-ಗ-ಳನ್ನು
ವಿದ್ಯಾರ್ಥಿ-ಗ-ಳನ್ನುದ್ದೇ-ಶಿಸಿ
ವಿದ್ಯಾರ್ಥಿ-ಗ-ಳನ್ನೂ
ವಿದ್ಯಾರ್ಥಿ-ಗ-ಳಲ್ಲಾ-ದರೂ
ವಿದ್ಯಾರ್ಥಿ-ಗ-ಳಲ್ಲಿ
ವಿದ್ಯಾರ್ಥಿ-ಗ-ಳಿಗೆ
ವಿದ್ಯಾರ್ಥಿ-ಗ-ಳಿದ್ದಾರೆ
ವಿದ್ಯಾರ್ಥಿ-ಗಳು
ವಿದ್ಯಾರ್ಥಿ-ಗಳೂ
ವಿದ್ಯಾರ್ಥಿ-ಗ-ಳೆಲ್ಲ
ವಿದ್ಯಾರ್ಥಿ-ಗಳೇ
ವಿದ್ಯಾರ್ಥಿ-ಗ-ಳೊಂದಿಗೆ
ವಿದ್ಯಾರ್ಥಿಗೆ
ವಿದ್ಯಾರ್ಥಿ-ಜೀ-ವ-ನ-ದಲ್ಲಿ
ವಿದ್ಯಾರ್ಥಿ-ನಿ-ಯರೂ
ವಿದ್ಯಾರ್ಥಿಯ
ವಿದ್ಯಾರ್ಥಿ-ಯನ್ನು
ವಿದ್ಯಾರ್ಥಿ-ಯನ್ನೂ
ವಿದ್ಯಾರ್ಥಿ-ಯಲ್ಲಾದ
ವಿದ್ಯಾರ್ಥಿ-ಯಾ-ಗಿದ್ದಾಗ
ವಿದ್ಯಾರ್ಥಿ-ಯಾ-ಗಿದ್ದಿ
ವಿದ್ಯಾರ್ಥಿಯು
ವಿದ್ಯಾರ್ಥಿಯೂ
ವಿದ್ಯಾರ್ಥಿ-ಯೊಬ್ಬ
ವಿದ್ಯಾವಂತ
ವಿದ್ಯಾ-ವಂತ-ನೊಬ್ಬ
ವಿದ್ಯಾ-ವಂತರ
ವಿದ್ಯಾ-ವಂತ-ರನ್ನು
ವಿದ್ಯಾ-ವಂತ-ರಲ್ಲಿ
ವಿದ್ಯಾ-ವಂತ-ರಾ-ದ-ವ-ರನ್ನು
ವಿದ್ಯಾ-ವಂತ-ರಿ-ಗಿಂತ
ವಿದ್ಯಾ-ವಂತ-ರಿಗೂ
ವಿದ್ಯಾ-ವಂತ-ರಿಗೆ
ವಿದ್ಯಾ-ವಂತರು
ವಿದ್ಯಾ-ವಂತರೂ
ವಿದ್ಯಾ-ವಂತ-ರೆ-ನಿ-ಸಿ-ಕೊಂಡ
ವಿದ್ಯಾ-ವಂತ-ರೆ-ನಿ-ಸಿ-ಕೊಂಡ-ವ-ರಲ್ಲಿ
ವಿದ್ಯಾ-ವಂತ-ರೆ-ನಿ-ಸಿ-ಕೊಂಡ-ವರೂ
ವಿದ್ಯಾ-ವಂತ-ರೆನ್ನಿ-ಸಿ-ಕೊಂಡ-ವರು
ವಿದ್ಯಾ-ವಂತಿ-ಕೆ-ಗಳು
ವಿದ್ಯಾ-ಸಂಪನ್ನ-ರಿಗೆ
ವಿದ್ಯಾ-ಸಂಸ್ಥೆ-ಗಳು
ವಿದ್ಯಾ-ಸಂಸ್ಥೆಯ
ವಿದ್ಯಾಸಕ್ತಿ
ವಿದ್ಯುಚ್ಛಕ್ತಿ
ವಿದ್ಯುಚ್ಛಕ್ತಿ-ಗಿಂತಲೂ
ವಿದ್ಯುತ್
ವಿದ್ಯುತ್ತಾಗಿ
ವಿದ್ಯುತ್ತಿನ
ವಿದ್ಯುತ್ತಿ-ನಂಥ
ವಿದ್ಯುತ್ತಿ-ನಿಂದಾಗಿ
ವಿದ್ಯುತ್ಪ್ರ-ವಾಹ
ವಿದ್ಯುತ್ರಾಜ್ಯದ
ವಿದ್ಯು-ದಾ-ಘಾ-ತ-ವಾ-ಯಿತು
ವಿದ್ಯುದ್ದೀ-ಪದ
ವಿದ್ಯುದ್ಯಂತ್ರ-ಗ-ಳನ್ನು
ವಿದ್ಯುದ್ವೇ-ಗ-ದಿಂದ
ವಿದ್ಯೆ
ವಿದ್ಯೆ-ಗ-ಳಿ-ಸಿದ
ವಿದ್ಯೆಗಳು
ವಿದ್ಯೆಯ
ವಿದ್ಯೆಯನ್ನು
ವಿದ್ಯೆಯನ್ನೂ
ವಿದ್ಯೆಯಲ್ಲೂ
ವಿದ್ಯೆಯಿಂದ
ವಿದ್ಯೆಯಿಲ್ಲ
ವಿದ್ಯೆಯು
ವಿದ್ಯೆಯೇ
ವಿದ್ವತ್ತನ್ನೂ
ವಿದ್ವತ್ತಿದೆ
ವಿದ್ವತ್ಪೂರ್ಣ
ವಿದ್ವಾಂಸ
ವಿದ್ವಾಂಸರ
ವಿದ್ವಾಂಸ-ರನ್ನು
ವಿದ್ವಾಂಸ-ರಲ್ಲ
ವಿದ್ವಾಂಸ-ರಾದ
ವಿದ್ವಾಂಸ-ರಿಗೂ
ವಿದ್ವಾಂಸರು
ವಿದ್ವಾಂಸ-ರು-ಗ-ಳಿಗೆ
ವಿದ್ವಾಂಸರೂ
ವಿದ್ವೇಷ
ವಿದ್ವೇ-ಷ-ಗಳು
ವಿದ್ವೇ-ಷ-ಭಾ-ವನೆ
ವಿಧ
ವಿಧ-ಗ-ಳಲ್ಲಿ
ವಿಧ-ಗ-ಳಾಗಿ
ವಿಧ-ಗ-ಳಿಂದಲೂ
ವಿಧ-ಗ-ಳಿ-ರ-ಬ-ಹುದು
ವಿಧದ
ವಿಧ-ದಿಂದಲೂ
ವಿಧವಾಗಿ
ವಿಧವಾದ
ವಿಧ-ವೆ-ಯರ
ವಿಧಾನ
ವಿಧಾ-ನ-ಗಳ
ವಿಧಾ-ನ-ಗ-ಳನ್ನು
ವಿಧಾ-ನ-ಗ-ಳನ್ನೂ
ವಿಧಾ-ನ-ಗ-ಳನ್ನೆಲ್ಲ
ವಿಧಾ-ನ-ಗ-ಳಾ-ಗಲಿ
ವಿಧಾ-ನ-ಗ-ಳಿಂದ
ವಿಧಾ-ನ-ಗ-ಳಿಂದಾ-ಗಲೀ
ವಿಧಾ-ನ-ಗ-ಳಿವೆ
ವಿಧಾ-ನ-ಗಳು
ವಿಧಾ-ನ-ಗಳೂ
ವಿಧಾ-ನ-ಗ-ಳೊಂದಿಗೆ
ವಿಧಾನದ
ವಿಧಾ-ನ-ದಂತಲ್ಲವೇ
ವಿಧಾ-ನ-ದಲ್ಲಿ
ವಿಧಾ-ನ-ದಿಂದ
ವಿಧಾ-ನ-ದಿಂದಲೇ
ವಿಧಾ-ನ-ವಂತೂ
ವಿಧಾ-ನ-ವನ್ನು
ವಿಧಾ-ನ-ವನ್ನು-ಜೀವ
ವಿಧಾ-ನ-ವನ್ನೂ
ವಿಧಾ-ನ-ವಲ್ಲವೇ
ವಿಧಾ-ನ-ವಾ-ದರೂ
ವಿಧಾ-ನ-ವಾ-ದರೋ
ವಿಧಾ-ನ-ವಿದೆ
ವಿಧಾನವು
ವಿಧಾ-ನ-ವೆನ್ನ-ಬ-ಹುದು
ವಿಧಾ-ನ-ವೆನ್ನು-ವಂತಿಲ್ಲ-ವಷ್ಟೆ
ವಿಧಾ-ನ-ಸ-ಭೆಯ
ವಿಧಾಯಕ
ವಿಧಿ
ವಿಧಿಯ
ವಿಧಿಯನ್ನು
ವಿಧಿಯು
ವಿಧಿಯೇ
ವಿಧಿಯೋ
ವಿಧಿವಶ
ವಿಧಿವಾದ
ವಿಧಿ-ವಾ-ದಿ-ಗ-ಳಿದ್ದಾರೆ
ವಿಧಿ-ವಿ-ಧಾ-ನ-ಗ-ಳನ್ನು
ವಿಧಿ-ವಿ-ಧಾ-ನ-ಗ-ಳಲ್ಲಿ
ವಿಧಿ-ಸಲ್ಪಟ್ಟ
ವಿಧಿ-ಸುತ್ತಾರೆ
ವಿಧಿ-ಸುತ್ತಾ-ರೆಂದು
ವಿಧಿ-ಸು-ವುದು
ವಿಧೇಯ
ವಿಧೇಯತೆ
ವಿಧೇ-ಯ-ತೆ-ಯಿಂದ
ವಿಧೇ-ಯ-ತೆ-ಯಿಂದಲೇ
ವಿಧೇ-ಯ-ತೆಯೇ
ವಿಧೇ-ಯ-ನಾ-ಗಿದ್ದೆ
ವಿಧೇ-ಯ-ನಾ-ಗಿ-ರುತ್ತಾ-ನಲ್ಲವೇ
ವಿಧೇ-ಯ-ನಾ-ದರೆ
ವಿಧೇಯರೂ
ವಿಧೇ-ಯ-ರೆಂದೂ
ವಿಧ್ವಂಸಕ
ವಿನಮ್ರ
ವಿನಮ್ರ-ನಾಗಿ
ವಿನಮ್ರ-ಭಾ-ವ-ದಿಂದ
ವಿನಮ್ರರೋ
ವಿನಯ
ವಿನ-ಯ-ಗ-ಳನ್ನು-ಳಿಸಿ
ವಿನ-ಯ-ವನ್ನು
ವಿನ-ಯ-ಸಂಪನ್ನ-ರಾದ
ವಿನಾ
ವಿನಾ-ಕಾ-ರ-ಣ-ವಾ-ಗಿ-ಯಾ-ಗಲಿ
ವಿನಾ-ಶ-ಕಾರ್ಯ-ದಲ್ಲಿ
ವಿನಾ-ಶಕ್ಕಾ-ಗಿಯೇ
ವಿನಾ-ಶಕ್ಕಿಂತಲೂ
ವಿನಾ-ಶ-ದೆ-ಡೆಗೆ
ವಿನಾ-ಶ-ವಾ-ಗ-ಬೇಕು
ವಿನಾ-ಶ-ಶೀ-ಲವೂ
ವಿನಿಮಯ
ವಿನಿ-ಮ-ಯ-ವಿ-ರು-ವಂತೆ
ವಿನಿ-ಯೋ-ಗ-ವಾ-ಗುತ್ತದೆ
ವಿನಿ-ಯೋ-ಗ-ವಾ-ಗುವ
ವಿನಿ-ಯೋ-ಗಿ-ಸಿದೆ
ವಿನಿ-ಯೋ-ಗಿ-ಸುತ್ತಿದೆ
ವಿನೋದ
ವಿನ್ಸ್ಟನ್
ವಿಪತ್ತನ್ನು
ವಿಪತ್ತಿ-ಗೀ-ಡಾಗಿ
ವಿಪತ್ತು-ಗ-ಳನ್ನು
ವಿಪರೀತ
ವಿಪ-ರೀ-ತ-ವಿತ್ತು
ವಿಪರ್ಯಾಸ
ವಿಪಶ್ಯ-ನ-ಗ-ಳಿಂದ
ವಿಪುಲ
ವಿಪ್ಲವದ
ವಿಫ-ಲ-ಳಾ-ಗಿದ್ದಳು
ವಿಫ-ಲ-ಗೊ-ಳಿ-ಸುತ್ತಾರೆ
ವಿಫ-ಲ-ನಾಗಿ
ವಿಫ-ಲ-ನಾ-ಗಿದ್ದೇನೆ
ವಿಫ-ಲ-ನಾ-ಗುತ್ತಾನೆ
ವಿಫ-ಲ-ನಾದ
ವಿಫ-ಲ-ಯತ್ನ
ವಿಫ-ಲ-ರಾ-ದರು
ವಿಫ-ಲ-ರಾ-ದರೂ
ವಿಭಾಗಕ್ಕೆ
ವಿಭಾ-ಗ-ಗಳೂ
ವಿಭಾಗದ
ವಿಭಾ-ಗ-ದಲ್ಲಿ
ವಿಭಾ-ಗ-ವನ್ನು
ವಿಭಿನ್ನ
ವಿಭಿನ್ನ-ರು-ಚಿಯ
ವಿಭಿನ್ನಸ್ತರ
ವಿಭೂತಿ
ವಿಭೂ-ತಿ-ಬಾಬು
ವಿಭ್ರಮಃ
ವಿಮರ್ಶ-ಕನ
ವಿಮರ್ಶ-ಕ-ನಾ-ಗಿದ್ದ
ವಿಮರ್ಶ-ಕ-ರಾ-ಗು-ವುದು
ವಿಮರ್ಶಾ
ವಿಮರ್ಶಾತ್ಮಕ
ವಿಮರ್ಶಾತ್ಮ-ಕ-ವಾಗಿ
ವಿಮರ್ಶಾ-ವಿ-ಧಾ-ನ-ಗ-ಳಲ್ಲಿ
ವಿಮರ್ಶಿಸ
ವಿಮರ್ಶಿಸಿ
ವಿಮರ್ಶಿ-ಸಿ-ಕೊಳ್ಳಿ
ವಿಮರ್ಶಿ-ಸುವ
ವಿಮರ್ಶಿ-ಸು-ವುದು
ವಿಮರ್ಶೆ
ವಿಮರ್ಶೆ-ಗ-ಳಿಂದ
ವಿಮರ್ಶೆ-ಗ-ಳಿ-ಗಿಂತಲೂ
ವಿಮರ್ಶೆ-ಗಳು
ವಿಮರ್ಶೆಯ
ವಿಮರ್ಶೆ-ಯಾ-ಗದೆ
ವಿಮರ್ಶೆ-ಯಾ-ಗು-ವುದೇ
ವಿಮರ್ಶೆಯು
ವಿಮಾ
ವಿಮಾನ
ವಿಮಾನದ
ವಿಮುಕ್ತಿ
ವಿಮು-ಖ-ರಾಗಿ
ವಿಮೆ
ವಿಯೆಟ್ನಾಮ್
ವಿಯೋಗ
ವಿಯೋ-ಗ-ಇ-ವು-ಗ-ಳನ್ನು
ವಿರಕ್ತಿ
ವಿರ-ಮಿ-ಸುತ್ತೇನೆ
ವಿರಳ
ವಿರ-ಳ-ವಾ-ದರೂ
ವಿರಸ
ವಿರ-ಸ-ವನ್ನ-ನು-ಭ-ವಿ-ಸ-ಬೇ-ಕಾ-ದೀತು
ವಿರ-ಹಿ-ತ-ವಾದ
ವಿರಾ-ಜಿ-ಸುತ್ತದೆ
ವಿರಾಟ್
ವಿರಾಮ
ವಿರಾ-ಮ-ವನ್ನು
ವಿರುದ್ಧ-ವಾಗಿ
ವಿರುದ್ಧ-ವಾದ
ವಿರು-ವ-ವ-ರಿಗೆ
ವಿರು-ವ-ವರು
ವಿರೋಧ
ವಿರೋ-ಧ-ವಾ-ಗಿ-ರ-ಲಿಲ್ಲ
ವಿರೋ-ಧ-ಗಳ
ವಿರೋಧದ
ವಿರೋ-ಧ-ಪಕ್ಷ-ಗ-ಳ-ವರೂ
ವಿರೋ-ಧ-ವನ್ನು
ವಿರೋ-ಧ-ವನ್ನೂ
ವಿರೋ-ಧ-ವಲ್ಲ
ವಿರೋ-ಧ-ವಾ-ಗ-ದಂತೆ
ವಿರೋ-ಧ-ವಾಗಿ
ವಿರೋ-ಧ-ವಾ-ಗು-ವಂತೆಯೂ
ವಿರೋ-ಧ-ವಾದ
ವಿರೋ-ಧ-ವಾ-ದು-ದೆಂದು
ವಿರೋ-ಧ-ವೆಂದು
ವಿರೋಧವೋ
ವಿರೋ-ಧಾ-ಭಾಸ
ವಿರೋ-ಧಾ-ಭಾಸ
ವಿರೋಧಿ
ವಿರೋ-ಧಿ-ಗಳ
ವಿರೋ-ಧಿ-ಗಳು
ವಿರೋಧಿಸ
ವಿರೋ-ಧಿ-ಸದೆ
ವಿರೋ-ಧಿ-ಸ-ಬಾ-ರ-ದೆಂದೂ
ವಿರೋ-ಧಿ-ಸ-ಬೇ-ಕಾ-ಗುತ್ತದೆ
ವಿರೋ-ಧಿ-ಸ-ಲಾ-ಯಿ-ತೆಂದು
ವಿರೋ-ಧಿ-ಸ-ಲಾ-ರ-ನಷ್ಟೆ
ವಿರೋ-ಧಿ-ಸ-ಲಿಲ್ಲ
ವಿರೋ-ಧಿ-ಸಿ-ದರು
ವಿರೋ-ಧಿ-ಸಿ-ದರೂ
ವಿರೋ-ಧಿ-ಸಿದ್ದಿಲ್ಲ
ವಿರೋ-ಧಿ-ಸು-ವಂತಿಲ್ಲ
ವಿರೋ-ಧಿ-ಸುತ್ತ
ವಿರೋ-ಧಿ-ಸುತ್ತದೆ
ವಿರೋ-ಧಿ-ಸುತ್ತಾನೆ
ವಿರೋ-ಧಿ-ಸುತ್ತಾರೆ
ವಿರೋ-ಧಿ-ಸುತ್ತಿ-ರುತ್ತದೆ
ವಿರೋ-ಧಿ-ಸು-ವುದು
ವಿರೋ-ಧೀ-ಶಕ್ತಿ-ಗಳ
ವಿಲಕ್ಷ-ಣವೂ
ವಿಲಾ
ವಿಲಾಯಿತಿ
ವಿಲಾಸ
ವಿಲಾಸದ
ವಿಲಿಯಂ
ವಿಲಿಯಮ್
ವಿಲಿವಿಲಿ
ವಿಲ್ಸನ್
ವಿಳಂಬ-ವಾ-ಗುತ್ತದೆ
ವಿಳಾ-ಸ-ಗ-ಳನ್ನು
ವಿಳಾ-ಸ-ಗ-ಳನ್ನೂ
ವಿಳಾ-ಸ-ರ-ಹಿತ
ವಿಳಾ-ಸ-ವನ್ನೂ
ವಿವರ
ವಿವ-ರ-ಗ-ಳನ್ನು
ವಿವ-ರ-ಗ-ಳನ್ನೂ
ವಿವ-ರ-ಗ-ಳನ್ನೆಲ್ಲ
ವಿವ-ರ-ಗಳು
ವಿವ-ರ-ಗ-ಳೆಲ್ಲ
ವಿವ-ರ-ಗ-ಳೆಲ್ಲವೂ
ವಿವರಣೆ
ವಿವ-ರ-ಣೆ-ಗಳ
ವಿವ-ರ-ಣೆ-ಗ-ಳತ್ತ
ವಿವ-ರ-ಣೆ-ಗ-ಳನ್ನು
ವಿವ-ರ-ಣೆ-ಗಳೂ
ವಿವ-ರ-ಣೆಗೆ
ವಿವ-ರ-ಣೆ-ಗೆಲ್ಲ
ವಿವ-ರ-ಣೆ-ಯನ್ನಿತ್ತಿದ್ದಾರೆ
ವಿವ-ರ-ಣೆ-ಯನ್ನು
ವಿವ-ರ-ಣೆ-ಯನ್ನೂ
ವಿವ-ರ-ಣೆ-ಯಲ್ಲೂ
ವಿವ-ರ-ಣೆ-ಯಾ-ದರೋ
ವಿವ-ರ-ಣೆ-ಯಾ-ಯಿತೇ
ವಿವ-ರ-ಣೆಯೂ
ವಿವ-ರ-ಣೆ-ಯೊಂದಿಗೆ
ವಿವ-ರ-ವನ್ನು
ವಿವ-ರ-ವಾಗಿ
ವಿವ-ರ-ವಾದ
ವಿವ-ರಿ-ಸ-ಬಲ್ಲ
ವಿವ-ರಿ-ಸ-ಬ-ಹುದು
ವಿವ-ರಿ-ಸ-ಬೇ-ಕಿಲ್ಲ
ವಿವ-ರಿ-ಸ-ಬೇ-ಕಿಲ್ಲ-ವಷ್ಟೆ
ವಿವ-ರಿ-ಸ-ಲ-ಸಾಧ್ಯ-ವಾ-ದುದು
ವಿವ-ರಿ-ಸ-ಲಾ-ಗದ
ವಿವ-ರಿ-ಸಲು
ವಿವ-ರಿ-ಸಲ್ಪಟ್ಟ
ವಿವರಿಸಿ
ವಿವ-ರಿ-ಸಿದ
ವಿವ-ರಿ-ಸಿ-ದರು
ವಿವ-ರಿ-ಸಿದೆ
ವಿವ-ರಿ-ಸಿದ್ದ
ವಿವ-ರಿ-ಸಿದ್ದರು
ವಿವ-ರಿ-ಸಿದ್ದಾರೆ
ವಿವ-ರಿ-ಸುತ್ತಾನೆ
ವಿವ-ರಿ-ಸುತ್ತಾರೆ
ವಿವ-ರಿ-ಸುತ್ತಿದ್ದ
ವಿವ-ರಿ-ಸುತ್ತಿದ್ದೆ
ವಿವ-ರಿ-ಸುವ
ವಿವ-ರಿ-ಸು-ವಂತೆ
ವಿವ-ರಿ-ಸು-ವು-ದಕ್ಕಿಂತ
ವಿವ-ರಿ-ಸು-ವುದೇ
ವಿವರ್ಣ-ವಾ-ದು-ದನ್ನು
ವಿವಾಹ
ವಿವಾಹದ
ವಿವಾ-ಹ-ವಾಗಿ
ವಿವಾ-ಹ-ವಾ-ಯಿತು
ವಿವಾ-ಹ-ವಿಚ್ಛೇದ
ವಿವಾ-ಹಿ-ತ-ರಲ್ಲಿ
ವಿವಾ-ಹಿ-ತ-ರಾದ
ವಿವಾ-ಹಿ-ತ-ರಿ-ಗಿಂತ
ವಿವಿಧ
ವಿವಿ-ಧ-ತೆ-ಯಲ್ಲಿ
ವಿವಿ-ಧಾ-ವ-ಧಿಯ
ವಿವೇಕ
ವಿವೇ-ಕ-ಗ-ಳನ್ನು
ವಿವೇಕದ
ವಿವೇ-ಕ-ಯುತ
ವಿವೇ-ಕ-ವನ್ನು
ವಿವೇ-ಕ-ವನ್ನುಂಟು
ವಿವೇ-ಕ-ವಾ-ಣಿ-ಯನ್ನು
ವಿವೇ-ಕಾ-ನಂದ
ವಿವೇ-ಕಾ-ನಂದರ
ವಿವೇ-ಕಾ-ನಂದ-ರದು
ವಿವೇ-ಕಾ-ನಂದ-ರನ್ನು
ವಿವೇ-ಕಾ-ನಂದ-ರಿಗೆ
ವಿವೇ-ಕಾ-ನಂದರು
ವಿವೇ-ಕಾ-ನಂದ-ರೆಂದರು
ವಿವೇ-ಕಿ-ಗಳೂ
ವಿವೇಚನಾ
ವಿವೇ-ಚ-ನಾ-ಶಕ್ತಿ
ವಿವೇ-ಚ-ನಾ-ಸಾ-ಮರ್ಥ್ಯ
ವಿವೇಚನೆ
ವಿವೇ-ಚಿ-ಸು-ವುದು
ವಿಶ-ದ-ವಾಗಿ
ವಿಶಾಲ
ವಿಶಾ-ಲ-ವಾಗಿ
ವಿಶಾ-ಲ-ವಾದ
ವಿಶಾ-ಲ-ವಾ-ದಂತೆ
ವಿಶಾ-ಲ-ವಾ-ದಂತೆಲ್ಲ
ವಿಶಿಷ್ಟ
ವಿಶಿಷ್ಟ-ಲೇ-ಖನ
ವಿಶಿಷ್ಟ-ವೆಂದು
ವಿಶಿಷ್ಟ-ಶಕ್ತಿ
ವಿಶೇಷ
ವಿಶೇಷಣ
ವಿಶೇಷತೆ
ವಿಶೇ-ಷ-ವಾಗಿ
ವಿಶೇ-ಷ-ವಾದ
ವಿಶೇ-ಷ-ವೆಂದರೆ
ವಿಶ್ರ-ಮಿ-ಸದೆ
ವಿಶ್ರ-ಮಿ-ಸುತ್ತಿದ್ದರು
ವಿಶ್ರಾಂತ
ವಿಶ್ರಾಂತಿ
ವಿಶ್ರಾಂತಿಯ
ವಿಶ್ರಾಂತಿ-ಯನ್ನು
ವಿಶ್ರಾಮ
ವಿಶ್ರಾ-ಮ-ಭಾ-ವ-ನೆಯ
ವಿಶ್ರಾ-ಮ-ಶಿ-ಥಿ-ಲೀ-ಕ-ರ-ಣ-ನಿದ್ರಾಸ್ಥಿ-ತಿಯ
ವಿಶ್ಲೇ-ಷ-ಣದ
ವಿಶ್ಲೇಷಣಾ
ವಿಶ್ಲೇ-ಷ-ಣಾತ್ಮಕ
ವಿಶ್ಲೇಷಣೆ
ವಿಶ್ಲೇ-ಷ-ಣೆ-ಗಳೂ
ವಿಶ್ಲೇ-ಷ-ಣೆಗೆ
ವಿಶ್ಲೇ-ಷ-ಣೆಯ
ವಿಶ್ಲೇ-ಷ-ಣೆ-ಯನ್ನು
ವಿಶ್ಲೇ-ಷ-ಣೆ-ಯಲ್ಲಾ-ಗಲೀ
ವಿಶ್ಲೇ-ಷ-ಣೆಯೇ
ವಿಶ್ಲೇಷಿಸಿ
ವಿಶ್ಲೇ-ಷಿ-ಸಿದ
ವಿಶ್ಲೇ-ಷಿ-ಸಿ-ದರೆ
ವಿಶ್ಲೇ-ಷಿ-ಸುತ್ತಾರೆ
ವಿಶ್ವ
ವಿಶ್ವ-ನಿ-ಯಮ
ವಿಶ್ವ-ವಿದ್ಯಾ-ನಿ-ಲ-ಯದ
ವಿಶ್ವ-ವಿದ್ಯಾ-ಲ-ಯ-ದಲ್ಲಿ-ರಲಿ
ವಿಶ್ವ-ಕರ್ಮ-ರಾ-ಗಿದ್ದಾ-ರೆಯೇ
ವಿಶ್ವದ
ವಿಶ್ವದಲ್ಲಿ
ವಿಶ್ವ-ದಲ್ಲಿ-ರುವ
ವಿಶ್ವ-ದೆಲ್ಲರ
ವಿಶ್ವ-ಧರ್ಮ-ಸಮ್ಮೇ-ಳ-ನವು
ವಿಶ್ವ-ನಾ-ಥ-ದೇ-ವಾ-ಲ-ಯ-ದಲ್ಲಿ
ವಿಶ್ವ-ನಾ-ಥನ
ವಿಶ್ವ-ನಾ-ಥ-ಮಂದಿ-ರಕ್ಕೆ
ವಿಶ್ವ-ನಿ-ಯಮ
ವಿಶ್ವ-ನಿ-ಯ-ಮ-ವೆಂದು
ವಿಶ್ವ-ನಿ-ಯಾ-ಮಕ
ವಿಶ್ವ-ನಿ-ಯಾ-ಮ-ಕನ
ವಿಶ್ವ-ನಿ-ಯಾ-ಮ-ಕ-ನಲ್ಲಿ
ವಿಶ್ವ-ನಿ-ಯಾ-ಮ-ಕ-ನಿಗೂ
ವಿಶ್ವ-ನಿ-ಯಾ-ಮ-ಕ-ನಿದ್ದಾನೆ
ವಿಶ್ವಪ್ರಜ್ಞೆ
ವಿಶ್ವಪ್ರಜ್ಞೆ-ಯಲ್ಲಿ
ವಿಶ್ವಬ್ರಹ್ಮಾಂಡ
ವಿಶ್ವಬ್ರಹ್ಮಾಂಡ-ಗಳ
ವಿಶ್ವಬ್ರಹ್ಮಾಂಡದ
ವಿಶ್ವಬ್ರಹ್ಮಾಂಡ-ದಲ್ಲಿ
ವಿಶ್ವಬ್ರಹ್ಮಾಂಡ-ವನ್ನು
ವಿಶ್ವಬ್ರಹ್ಮಾಂಡ-ವೆಲ್ಲ
ವಿಶ್ವಭ್ರಾ-ತೃತ್ವದ
ವಿಶ್ವ-ಮಟ್ಟದ
ವಿಶ್ವ-ಮಾ-ನವ
ವಿಶ್ವರೂಪ
ವಿಶ್ವವನ್ನು
ವಿಶ್ವವನ್ನೇ
ವಿಶ್ವ-ವಿ-ಜ-ಯಿ-ಗ-ಳಾ-ಗು-ವಂತೆ
ವಿಶ್ವ-ವಿದ್ಯಾ-ನಿ-ಲ-ಯದ
ವಿಶ್ವ-ವಿದ್ಯಾ-ಲಯ
ವಿಶ್ವ-ವಿದ್ಯಾ-ಲ-ಯ-ಗಳ
ವಿಶ್ವ-ವಿದ್ಯಾ-ಲ-ಯ-ಗ-ಳಲ್ಲಿ
ವಿಶ್ವ-ವಿದ್ಯಾ-ಲ-ಯ-ಗ-ಳಿದ್ದರೆ
ವಿಶ್ವ-ವಿದ್ಯಾ-ಲ-ಯ-ಗಳು
ವಿಶ್ವ-ವಿದ್ಯಾ-ಲ-ಯದ
ವಿಶ್ವ-ವಿದ್ಯಾ-ಲ-ಯ-ದಲ್ಲಿ
ವಿಶ್ವ-ವಿದ್ಯಾ-ಲ-ಯ-ದಲ್ಲೂ
ವಿಶ್ವ-ವಿದ್ಯಾ-ಲ-ಯ-ವಿತ್ತು
ವಿಶ್ವ-ವಿ-ಶಾಲ
ವಿಶ್ವವು
ವಿಶ್ವವೆಂಬ
ವಿಶ್ವವೆಲ್ಲ
ವಿಶ್ವವೇ
ವಿಶ್ವವ್ಯಾ-ಪಕ
ವಿಶ್ವವ್ಯಾ-ಪಿಯೂ
ವಿಶ್ವ-ಶಾಂತಿ-ಯನ್ನು
ವಿಶ್ವ-ಸಂಸ್ಥೆಯ
ವಿಶ್ವ-ಸ-ನೀ-ಯ-ವಾದ
ವಿಶ್ವ-ಸೃಷ್ಟಿ-ಯಾ-ದಂದಿ-ನಿಂದಲೂ
ವಿಶ್ವಾದ್ಯಂತ
ವಿಶ್ವಾಸ
ವಿಶ್ವಾ-ಸ-ಇ-ವು-ಗಳ
ವಿಶ್ವಾಸಕ್ಕೆ
ವಿಶ್ವಾ-ಸ-ಗಳ
ವಿಶ್ವಾ-ಸ-ಗ-ಳನ್ನು
ವಿಶ್ವಾ-ಸ-ಗ-ಳಿಂದ
ವಿಶ್ವಾಸದ
ವಿಶ್ವಾ-ಸ-ದಿಂದ
ವಿಶ್ವಾ-ಸ-ಬೇಕು
ವಿಶ್ವಾ-ಸ-ವನ್ನಿಟ್ಟು
ವಿಶ್ವಾ-ಸ-ವನ್ನು
ವಿಶ್ವಾ-ಸ-ವನ್ನೂ
ವಿಶ್ವಾ-ಸ-ವಿಟ್ಟಿದ್ದಾ-ದರೆ
ವಿಶ್ವಾ-ಸ-ವಿಟ್ಟಿದ್ದಾರೆ
ವಿಶ್ವಾ-ಸ-ವಿಟ್ಟು
ವಿಶ್ವಾ-ಸ-ವಿ-ಡದೆ
ವಿಶ್ವಾ-ಸ-ವಿಡಿ
ವಿಶ್ವಾ-ಸ-ವಿತ್ತು
ವಿಶ್ವಾ-ಸ-ವಿದೆ
ವಿಶ್ವಾ-ಸ-ವಿದ್ದು-ದಷ್ಟೇ
ವಿಶ್ವಾ-ಸ-ವಿ-ರ-ದಿದ್ದರೆ
ವಿಶ್ವಾ-ಸ-ವಿ-ರದು
ವಿಶ್ವಾ-ಸ-ವಿ-ರಲಿ
ವಿಶ್ವಾ-ಸ-ವಿ-ರಿ-ಸ-ದಿದ್ದ-ವರೂ
ವಿಶ್ವಾ-ಸ-ವಿ-ರು-ವ-ವ-ರೆಲ್ಲ
ವಿಶ್ವಾ-ಸ-ವಿಲ್ಲದ
ವಿಶ್ವಾ-ಸ-ವಿಲ್ಲ-ದ-ವರು
ವಿಶ್ವಾ-ಸ-ವಿಲ್ಲ-ದಿದ್ದ-ವರು
ವಿಶ್ವಾ-ಸ-ವಿಲ್ಲದೆ
ವಿಶ್ವಾಸವು
ವಿಶ್ವಾಸವೂ
ವಿಶ್ವಾಸವೇ
ವಿಶ್ವಾಸಾರ್ಹ
ವಿಶ್ವಾ-ಸಾರ್ಹ-ತೆಕ್ರಿ-ಡಿ-ಬಿ-ಲಿ-ಡಿ-ಯನ್ನು
ವಿಶ್ವಾಸೋ
ವಿಶ್ವೇಶ್ವ-ರಯ್ಯ-ನ-ವರು
ವಿಷ
ವಿಷ-ಗ-ಳಿ-ಗೆ-ಯಲ್ಲಿ
ವಿಷಗಳು
ವಿಷಚಕ್ರ
ವಿಷ-ಜಂತು-ಗಳ
ವಿಷಣ್ಣ
ವಿಷದ
ವಿಷದಿಂದ
ವಿಷಪ್ರಾ-ಯ-ವಾಗಿ
ವಿಷಮ
ವಿಷಮತೆ
ವಿಷಮಯ
ವಿಷಯ
ವಿಷ-ಯ-ಗಳ
ವಿಷ-ಯ-ಗ-ಳನ್ನು
ವಿಷ-ಯ-ಗ-ಳನ್ನೂ
ವಿಷ-ಯ-ಗ-ಳಲ್ಲಿ
ವಿಷ-ಯ-ಗ-ಳಲ್ಲೂ
ವಿಷ-ಯ-ಗ-ಳಿಂದ
ವಿಷ-ಯ-ಗ-ಳಿಗೂ
ವಿಷ-ಯ-ಗ-ಳಿಗೆ
ವಿಷ-ಯ-ಗಳು
ವಿಷ-ಯ-ಗಳೂ
ವಿಷ-ಯ-ಗ-ಳೆಂದು
ವಿಷಯದ
ವಿಷ-ಯ-ದಲ್ಲಂತೂ
ವಿಷ-ಯ-ದಲ್ಲಿ
ವಿಷ-ಯ-ದಲ್ಲೂ
ವಿಷ-ಯ-ಮೂಲ
ವಿಷ-ಯ-ಲಂಪ-ಟತ್ವ-ಇ-ವು-ಗಳ
ವಿಷ-ಯ-ಲಂಪ-ಟನೂ
ವಿಷ-ಯ-ವನ್ನು
ವಿಷ-ಯ-ವನ್ನೇ
ವಿಷ-ಯ-ವನ್ನೋ
ವಿಷ-ಯ-ವಲ್ಲ
ವಿಷ-ಯ-ವಲ್ಲವೆ
ವಿಷ-ಯ-ವಸ್ತು-ಗಳ
ವಿಷ-ಯ-ವಸ್ತು-ಗ-ಳಲ್ಲಿ
ವಿಷ-ಯ-ವಸ್ತು-ಗ-ಳಿಂದ
ವಿಷ-ಯ-ವಾ-ಗಲಿ
ವಿಷ-ಯ-ವಾಗಿ
ವಿಷ-ಯ-ವಾ-ಗಿತ್ತು
ವಿಷ-ಯ-ವಾ-ಗಿ-ರ-ಲಿಲ್ಲ
ವಿಷ-ಯ-ವಾ-ಗಿ-ರು-ವು-ದಿಲ್ಲ
ವಿಷ-ಯ-ವಾದ
ವಿಷ-ಯ-ವಾ-ದು-ದ-ರಿಂದ
ವಿಷಯವೇ
ವಿಷ-ಯಾಗ್ನಿಯ
ವಿಷ-ಯಾ-ಸಕ್ತಿ
ವಿಷ-ಯುಕ್ತ-ವಾ-ಗುತ್ತವೆ
ವಿಷವನ್ನು
ವಿಷವನ್ನೂ
ವಿಷ-ವರ್ತುಲ
ವಿಷ-ವರ್ತು-ಲ-ದಿಂದ
ವಿಷವೇಕೆ
ವಿಷ-ವೇ-ರಿ-ದಂತೆ
ವಿಷಸರ್ಪ
ವಿಷ-ಸರ್ಪ-ವೊಂದಿದೆ
ವಿಷ್ಣು-ಸ-ಹಸ್ರ-ನಾ-ಮ-ದಲ್ಲಿ
ವಿಷ್ಣು-ಸ-ಹಸ್ರ-ನಾ-ಮವೇ
ವಿಸರ್ಜ-ನೆ-ಯಾ-ಗಿ-ರ-ಲೇ-ಬೇಕು
ವಿಸ್ತ-ರಿ-ಸ-ಬೇ-ಕೆಂದಿದ್ದೇನೆ
ವಿಸ್ತ-ರಿ-ಸಿ-ತು-ನಾನು
ವಿಸ್ತ-ರಿ-ಸಿ-ದಾಗ
ವಿಸ್ತ-ರಿ-ಸುತ್ತ-ಲಿದೆ
ವಿಸ್ತಾರ
ವಿಸ್ತಾ-ರ-ವಾಗಿ
ವಿಸ್ತಾರವೂ
ವಿಸ್ತೃತ
ವಿಸ್ತೃ-ತ-ವಾಗಿ
ವಿಸ್ತೃತವೂ
ವಿಸ್ಮ-ಯ-ಕರ
ವಿಸ್ಮ-ಯ-ಕಾ-ರಕ
ವಿಸ್ಮ-ಯ-ಗೊ-ಳಿ-ಸಿದ
ವಿಸ್ಮ-ಯ-ವೆ-ನಿ-ಸಿದ
ವಿಹ-ರಿ-ಸುತ್ತಿದ್ದ
ವಿಹಾರ
ವಿಹಾರಕ್ಕೆ
ವಿಹೇದಿಗೂ
ವಿಹೇದಿಗೆ
ವಿಹೇದಿಯ
ವಿಹೇ-ದಿ-ಯಲ್ಲಿ
ವಿಹೇ-ದಿ-ಯಲ್ಲಿದ್ದಾರೆ
ವಿಹೇ-ದಿ-ಯಿಂದ
ವಿಹೇದೀ
ವಿಹ್ವ-ಲ-ನಾ-ಗುತ್ತಾನೆ
ವೀಕ್ಷಣೆ
ವೀಕ್ಷ-ಣೆ-ಗಾಗಿ
ವೀಕ್ಷಿ-ಸ-ಬಲ್ಲ-ವ-ಳಾ-ಗಿದ್ದಳು
ವೀಕ್ಷಿಸಲು
ವೀಕ್ಷಿಸಿ
ವೀಕ್ಷಿಸಿದ
ವೀಕ್ಷಿ-ಸಿ-ದ-ವರು
ವೀಕ್ಷಿ-ಸುತ್ತಿದ್ದ
ವೀಕ್ಷಿ-ಸುತ್ತಿದ್ದರು
ವೀಕ್ಷಿಸುವ
ವೀಕ್ಷಿ-ಸು-ವಂತೆ
ವೀರ
ವೀರನಂತೆ
ವೀರರ
ವೀರರಾದ
ವೀರರು
ವೀರ-ವೇ-ದಾಂತದ
ವುಡ್
ವುಡ್ಡೆ
ವುಡ್ವರ್ಡಿಗೆ
ವುಡ್ವರ್ಡ್
ವುದನ್ನು
ವುದರಲ್ಲೇ
ವುದು
ವೂಲ್ಗರ್
ವೂಲ್ಫ್
ವೃಕ್ಷ
ವೃಕ್ಷಗಳ
ವೃಕ್ಷಗಳೂ
ವೃಕ್ಷದ
ವೃಕ್ಷವಿಲ್ಲ
ವೃತ್ತ
ವೃತ್ತದ
ವೃತ್ತವೇ
ವೃತ್ತಾಂತ-ಗ-ಳನ್ನು
ವೃತ್ತಾಂತ-ವನ್ನೂ
ವೃತ್ತಿ
ವೃತ್ತಿ-ಗ-ಳನ್ನೆಲ್ಲ
ವೃತ್ತಿ-ಗ-ಳಲ್ಲಿ
ವೃತ್ತಿ-ಗ-ಳಿಗೆ
ವೃತ್ತಿಯ
ವೃತ್ತಿಯಲ್ಲಿ
ವೃತ್ತಿ-ಯ-ವ-ರೆ-ನಿ-ಸಿ-ಕೊಂಡ
ವೃತ್ತಿಯಿಂದ
ವೃದ್ಧ
ವೃದ್ಧ-ನೊಬ್ಬನು
ವೃದ್ಧನನ್ನು
ವೃದ್ಧ-ನಾ-ಗಿದ್ದ
ವೃದ್ಧರಿಗೂ
ವೃದ್ಧರು
ವೃದ್ಧಾಪ್ಯ
ವೃದ್ಧಾಪ್ಯ-ದಲ್ಲೂ
ವೃದ್ಧಾಪ್ಯ-ವನ್ನು
ವೃದ್ಧಾಪ್ಯವು
ವೃದ್ಧಿ
ವೃದ್ಧಿಕ್ಷ-ಯ-ಗ-ಳಿಲ್ಲದೆ
ವೃದ್ಧಿ-ಗೊ-ಳಿ-ಸಿ-ಕೊಳ್ಳಲು
ವೃದ್ಧಿ-ಯಾ-ಗ-ಲೇ-ಬೇಕು
ವೃದ್ಧಿ-ಯಾ-ಗುತ್ತ
ವೃದ್ಧಿ-ಯಾ-ಗುತ್ತದೆ
ವೃದ್ಧಿ-ಯಾ-ಗುತ್ತವೆ
ವೃದ್ಧಿ-ಯಾ-ದರೂ
ವೃದ್ಧಿ-ಯಾ-ದರೆ
ವೃದ್ಧಿ-ಯಾ-ಯಿತು
ವೃದ್ಧಿಸದೆ
ವೃದ್ಧಿ-ಸ-ಬೇ-ಕಾ-ದರೆ
ವೃದ್ಧಿಸಲೂ
ವೃದ್ಧಿಸಿ
ವೃದ್ಧಿ-ಸಿ-ಕೊಂಡು
ವೃದ್ಧಿ-ಸಿ-ಕೊಳ್ಳುತ್ತ
ವೃದ್ಧಿ-ಸುತ್ತದೆ
ವೃದ್ಧಿಸುವ
ವೃದ್ಧಿ-ಸು-ವಂಥದ್ದು
ವೃದ್ಧಿ-ಸು-ವುದು
ವೃಷ್ಟಿ
ವೃಷ್ಟಿ-ಯಾ-ಯಿತು
ವೆಂಕಟೇಶ
ವೆಂದರೆ
ವೆಂದೂ
ವೆಂದೆ-ಣಿ-ಸಿದ
ವೆಂದೇ
ವೆಚ್ಚ
ವೆಚ್ಚಗಳು
ವೆಚ್ಚದಲ್ಲಿ
ವೆಚ್ಚವನ್ನು
ವೆಚ್ಚ-ವಾ-ಗುತ್ತಿದೆ
ವೆಟರನ್ಸ್
ವೆನಿಸಿತು
ವೆನಿ-ಸೀ-ತು-ಅದೇ
ವೆನಿಸುವ
ವೆಬ್
ವೆರಿಗುಡ್
ವೆಲಿಸಿವ್ರ
ವೆಲ್ಸ್
ವೆಸ-ಗುತ್ತಾರೆ
ವೆಸಿಲಿವ್
ವೇಗ
ವೇಗಕ್ಕೂ
ವೇಗ-ಗ-ಳನ್ನು
ವೇಗದ
ವೇಗದಲ್ಲಿ
ವೇಗದಿಂದ
ವೇಗ-ದಿಂದಲೇ
ವೇಗವನ್ನು
ವೇಗ-ವನ್ನೆಂದಿಗೂ
ವೇಗವಾಗಿ
ವೇಗವು
ವೇದ
ವೇದಕ
ವೇದ-ನೆ-ಗ-ಳನ್ನು
ವೇದ-ಶಾಸ್ತ್ರವೇ
ವೇದಾಂತ
ವೇದಾಂತ-ಕೇ-ಸರಿ
ವೇದಾಂತ-ತತ್ತ್ವ
ವೇದಾಂತದ
ವೇದಾಂತವು
ವೇದಿಕೆಗೆ
ವೇದಿಕೆಯ
ವೇದಿ-ಕೆ-ಯಲ್ಲಿ-ನೀ-ಡುವ
ವೇದಿ-ಕೆ-ಯಿಂದ
ವೇದೋ-ಪ-ನಿ-ಷತ್ತು-ಗ-ಳನ್ನೂ
ವೇದ್ಯ-ವಾ-ಗುತ್ತ-ದೆಂದೂ
ವೇನು
ವೇಲ್ಯೂ
ವೇಳೆ
ವೇಳೆಗೆ
ವೇಳೆಯಲ್ಲಿ
ವೇಳೆಯಲ್ಲೂ
ವೇಶ್ಯೆಯನ್ನು
ವೇಶ್ಯೆ-ಯೊಬ್ಬಳ
ವೇಷ
ವೇಷಗಳು
ವೇಷದಲ್ಲಿ
ವೇಷ-ಭೂ-ಷಣ
ವೇಷ-ಭೂ-ಷ-ಣ-ಗ-ಳನ್ನು
ವೇಷ-ಭೂ-ಷ-ಣ-ಗ-ಳಲ್ಲಿ
ವೇಷ-ಭೂ-ಷ-ಣ-ಗ-ಳಲ್ಲೇ
ವೈಖರಿ
ವೈಖ-ರಿ-ಇ-ವು-ಗ-ಳನ್ನೆಲ್ಲ
ವೈಚಾರಿಕ
ವೈಚಾ-ರಿ-ಕತೆ
ವೈಚಾ-ರಿ-ಕ-ತೆ-ಗಳ
ವೈಚಾ-ರಿ-ಕ-ತೆಯ
ವೈಚಾ-ರಿ-ಕರು
ವೈಚಿತ್ರ್ಯ
ವೈಚಿತ್ರ್ಯ-ಗಳು
ವೈಚಿತ್ರ್ಯದ
ವೈಚಿತ್ರ್ಯ-ವನ್ನು
ವೈಜ್ಞಾನಿ
ವೈಜ್ಞಾನಿಕ
ವೈಜ್ಞಾ-ನಿ-ಕತೆ
ವೈಜ್ಞಾ-ನಿ-ಕ-ಯು-ಗದ
ವೈಜ್ಞಾ-ನಿ-ಕ-ರಿಗೂ
ವೈಜ್ಞಾ-ನಿ-ಕ-ವಾಗಿ
ವೈಜ್ಞಾ-ನಿ-ಕ-ವಾದ
ವೈದಿಕ
ವೈದಿ-ಕ-ಯು-ಗ-ದಿಂದಲೂ
ವೈದುಷ್ಯ-ಗಳು
ವೈದ್ಯ
ವೈದ್ಯಕೀಯ
ವೈದ್ಯನ
ವೈದ್ಯನಾಗಿ
ವೈದ್ಯ-ನಾ-ಗಿದ್ದ
ವೈದ್ಯ-ನಾ-ಗಿದ್ದು
ವೈದ್ಯ-ಮಿತ್ರ-ರಲ್ಲಿಗೆ
ವೈದ್ಯರ
ವೈದ್ಯರನ್ನು
ವೈದ್ಯರಿಗೆ
ವೈದ್ಯರು
ವೈದ್ಯ-ರು-ಗ-ಳನ್ನು
ವೈದ್ಯ-ರು-ಗ-ಳಿಗೆ
ವೈದ್ಯ-ರು-ಗಳು
ವೈದ್ಯ-ರು-ಗಳೂ
ವೈದ್ಯ-ರು-ಗ-ಳೆಲ್ಲ
ವೈದ್ಯರೂ
ವೈದ್ಯರೆಲ್ಲ
ವೈದ್ಯ-ರೊಬ್ಬ-ರನ್ನು
ವೈದ್ಯ-ರೊಬ್ಬರು
ವೈದ್ಯ-ವಿ-ಭಾ-ಗದ
ವೈದ್ಯ-ಶಾಸ್ತ್ರಜ್ಞರು
ವೈದ್ಯೆ
ವೈಧವ್ಯ
ವೈಪ-ರೀತ್ಯ-ಗಳು
ವೈಪ-ರೀತ್ಯ-ದಿಂದ
ವೈಪ-ರೀತ್ಯ-ವಲ್ಲವೇ
ವೈಭವ
ವೈಭ-ವ-ಯುತ
ವೈಭ-ವ-ಗಳ
ವೈಭ-ವ-ಗ-ಳನ್ನು
ವೈಭ-ವ-ಗ-ಳನ್ನೂ
ವೈಭವದ
ವೈಭ-ವ-ವನ್ನು
ವೈಭವೋ
ವೈಮನಸ್ಯ
ವೈಮ-ನಸ್ಯ-ವಲ್ಲ
ವೈಯಕ್ತಿಕ
ವೈಯಕ್ತಿ-ಕ-ವಾಗಿ
ವೈಯರಿಂಗ್
ವೈರ
ವೈರವನ್ನು
ವೈರ-ವಿದ್ವೇಷ
ವೈರಾಗ್ಯ
ವೈರಾಗ್ಯ-ಗ-ಳೊಂದಿಗೆ
ವೈರಾಗ್ಯ-ದಲ್ಲೂ
ವೈರಿಗಳು
ವೈರಿಯನ್ನೂ
ವೈರ್ನ
ವೈಲ್ಡರ್
ವೈಲ್ಡ್
ವೈವಾಹಿಕ
ವೈವಿಧ್ಯ
ವೈವಿಧ್ಯಕ್ಕೆ
ವೈವಿಧ್ಯಕ್ಕೇನು
ವೈವಿಧ್ಯ-ಗ-ಳಿ-ಗಂತೂ
ವೈವಿಧ್ಯ-ಗ-ಳಿ-ಗೇನು
ವೈವಿಧ್ಯ-ಗ-ಳಿವೆ
ವೈವಿಧ್ಯ-ಗಳು
ವೈವಿಧ್ಯ-ಗ-ಳೇನು
ವೈವಿಧ್ಯದ
ವೈವಿಧ್ಯ-ಪೂರ್ಣ
ವೈವಿಧ್ಯ-ಪೂರ್ಣವೂ
ವೈವಿಧ್ಯ-ವನ್ನು
ವೈವಿಧ್ಯ-ವನ್ನೋ
ವೈವಿಧ್ಯ-ವಿದೆ
ವೈವಿಧ್ಯ-ವಿ-ದೆ-ಪು-ರಾಣ
ವೈವಿಧ್ಯವು
ವೈವಿಧ್ಯವೇ
ವೈಶಾಲ್ಯ-ಗ-ಳಿಂದ
ವೈಶಾಲ್ಯ-ವನ್ನು
ವೈಶಿಷ್ಟ್ಯ
ವೈಶಿಷ್ಟ್ಯಕ್ಕ-ನು-ಗು-ಣ-ವಾಗಿ
ವೈಶಿಷ್ಟ್ಯ-ಗ-ಳನ್ನು
ವೈಶಿಷ್ಟ್ಯ-ಗ-ಳನ್ನೇ
ವೈಶಿಷ್ಟ್ಯ-ಗ-ಳಿ-ರುತ್ತವೆ
ವೈಶಿಷ್ಟ್ಯ-ಗ-ಳಿ-ರು-ವಂತೆ
ವೈಶಿಷ್ಟ್ಯ-ಗಳೂ
ವೈಶಿಷ್ಟ್ಯ-ಗ-ಳೇನು
ವೈಶಿಷ್ಟ್ಯದ
ವೈಶಿಷ್ಟ್ಯ-ದಿಂದ
ವೈಶಿಷ್ಟ್ಯ-ದೊಂದಿಗೆ
ವೈಶಿಷ್ಟ್ಯ-ರ-ಹಿ-ತ-ರಾ-ಗಿ-ರುತ್ತಾರೆ
ವೈಶಿಷ್ಟ್ಯ-ವನ್ನು
ವೈಶಿಷ್ಟ್ಯ-ವಿದೆ
ವೈಶಿಷ್ಟ್ಯವೂ
ವೈಶಿಷ್ಟ್ಯ-ವೆಂದೇ
ವೈಶಿಷ್ಟ್ಯವೋ
ವೈಷಮ್ಯದ
ವೈಸ್ಛಾನ್ಸ-ಲರ್
ವೈಸ್ರಾಯ್ಗೆ
ವೊಮ್ಮೆ
ವೊರ್ಸೆಲಸ್
ವ್ಯಂಗ್ಯ
ವ್ಯಂಗ್ಯಚಿತ್ರ
ವ್ಯಂಗ್ಯ-ಟೀ-ಕೆ-ಗ-ಳನ್ನೂ
ವ್ಯಂಗ್ಯೋಕ್ತಿ-ಗಳ
ವ್ಯಕ್ತ
ವ್ಯಕ್ತ-ಗೊ-ಳಿ-ಸಿ-ದಾಗ
ವ್ಯಕ್ತ-ಗೊ-ಳಿ-ಸ-ದಂತೆ
ವ್ಯಕ್ತ-ಗೊ-ಳಿ-ಸದೆ
ವ್ಯಕ್ತ-ಗೊ-ಳಿ-ಸಲು
ವ್ಯಕ್ತ-ಗೊ-ಳಿಸಿ
ವ್ಯಕ್ತ-ಗೊ-ಳಿ-ಸಿ-ದವು
ವ್ಯಕ್ತ-ಗೊ-ಳಿ-ಸು-ವ-ವ-ನಲ್ಲೂ
ವ್ಯಕ್ತ-ಗೊ-ಳಿ-ಸುತ್ತಾ-ರಷ್ಟೆ
ವ್ಯಕ್ತ-ಗೊ-ಳಿ-ಸುತ್ತಿಲ್ಲ
ವ್ಯಕ್ತ-ಗೊ-ಳಿ-ಸುತ್ತೇವೆ
ವ್ಯಕ್ತ-ಗೊ-ಳಿ-ಸುವ
ವ್ಯಕ್ತ-ಗೊ-ಳಿ-ಸು-ವುದು
ವ್ಯಕ್ತ-ದೇ-ವ-ತೆಯ
ವ್ಯಕ್ತ-ನಲ್ಲದ
ವ್ಯಕ್ತ-ನಾ-ಗ-ಬಲ್ಲ
ವ್ಯಕ್ತ-ನಾ-ಗಿದ್ದಾ-ನೆ-ಜೀ-ವಾತ್ಮನು
ವ್ಯಕ್ತ-ನಾ-ಗಿ-ರು-ವಂತ್
ವ್ಯಕ್ತ-ಪ-ಡಿ-ಸ-ತೊ-ಡ-ಗುತ್ತದೆ
ವ್ಯಕ್ತ-ಪ-ಡಿ-ಸದೇ
ವ್ಯಕ್ತ-ಪ-ಡಿ-ಸಲು
ವ್ಯಕ್ತ-ಪ-ಡಿ-ಸ-ಲೇ-ಬೇಕು
ವ್ಯಕ್ತ-ಪ-ಡಿಸಿ
ವ್ಯಕ್ತ-ಪ-ಡಿ-ಸಿ-ದರು
ವ್ಯಕ್ತ-ಪ-ಡಿ-ಸಿ-ದರೂ
ವ್ಯಕ್ತ-ಪ-ಡಿ-ಸಿ-ದ-ರೆಂದು
ವ್ಯಕ್ತ-ಪ-ಡಿ-ಸಿ-ದಾ-ಗಲೇ
ವ್ಯಕ್ತ-ಪ-ಡಿ-ಸಿದ್ದರು
ವ್ಯಕ್ತ-ಪ-ಡಿ-ಸುತ್ತದೆ
ವ್ಯಕ್ತ-ಪ-ಡಿ-ಸುತ್ತಾ
ವ್ಯಕ್ತ-ಪ-ಡಿ-ಸುತ್ತಿದ್ದರು
ವ್ಯಕ್ತ-ಪ-ಡಿ-ಸುತ್ತಿದ್ದೆ
ವ್ಯಕ್ತ-ಪ-ಡಿ-ಸುತ್ತಿ-ರುವ
ವ್ಯಕ್ತ-ಪ-ಡಿ-ಸು-ವಾ-ಗಲೂ
ವ್ಯಕ್ತ-ಪ-ಡಿ-ಸು-ವು-ದ-ರಲ್ಲಿ
ವ್ಯಕ್ತ-ವಾ-ಗ-ದಿದ್ದರೂ
ವ್ಯಕ್ತ-ವಾ-ಗ-ದಿದ್ದಲ್ಲಿ
ವ್ಯಕ್ತ-ವಾ-ಗ-ದಿ-ರದು
ವ್ಯಕ್ತ-ವಾ-ಗ-ದಿ-ರ-ಬ-ಹುದು
ವ್ಯಕ್ತ-ವಾ-ಗದು
ವ್ಯಕ್ತ-ವಾ-ಗದೇ
ವ್ಯಕ್ತ-ವಾ-ಗ-ಬ-ಹು-ದಾದ
ವ್ಯಕ್ತ-ವಾ-ಗ-ಬ-ಹುದು
ವ್ಯಕ್ತ-ವಾ-ಗಲು
ವ್ಯಕ್ತವಾಗಿ
ವ್ಯಕ್ತ-ವಾ-ಗಿದೆ
ವ್ಯಕ್ತ-ವಾ-ಗಿಯೋ
ವ್ಯಕ್ತ-ವಾ-ಗಿ-ರುವ
ವ್ಯಕ್ತ-ವಾ-ಗುತ್ತದೆ
ವ್ಯಕ್ತ-ವಾ-ಗುತ್ತಿತ್ತು
ವ್ಯಕ್ತ-ವಾ-ಗುತ್ತಿದೆ
ವ್ಯಕ್ತ-ವಾ-ಗುತ್ತಿದ್ದರೆ
ವ್ಯಕ್ತ-ವಾ-ಗುವ
ವ್ಯಕ್ತ-ವಾ-ಗು-ವು-ದನ್ನು
ವ್ಯಕ್ತ-ವಾ-ಗು-ವು-ದಿಲ್ಲ
ವ್ಯಕ್ತ-ವಾ-ಗು-ವುದು
ವ್ಯಕ್ತವಾದ
ವ್ಯಕ್ತ-ವಾ-ದರೂ
ವ್ಯಕ್ತ-ವಾ-ದರೆ
ವ್ಯಕ್ತ-ವಾ-ದಾಗ
ವ್ಯಕ್ತ-ವಾ-ಯಿತು
ವ್ಯಕ್ತಸ್ಥಿತಿ
ವ್ಯಕ್ತಿ
ವ್ಯಕ್ತಿ-ಗ-ಳಿದ್ದಾರೆ
ವ್ಯಕ್ತಿ-ಗ-ಳಿಲ್ಲ-ದಿದ್ದರೆ
ವ್ಯಕ್ತಿಯಲ್ಲೂ
ವ್ಯಕ್ತಿಗಳ
ವ್ಯಕ್ತಿ-ಗ-ಳದ್ದಲ್ಲ
ವ್ಯಕ್ತಿ-ಗ-ಳನ್ನು
ವ್ಯಕ್ತಿ-ಗ-ಳನ್ನೂ
ವ್ಯಕ್ತಿ-ಗ-ಳನ್ನೇ
ವ್ಯಕ್ತಿ-ಗ-ಳಲ್ಲಿ
ವ್ಯಕ್ತಿ-ಗ-ಳಲ್ಲಿತ್ತು
ವ್ಯಕ್ತಿ-ಗ-ಳಲ್ಲಿ-ರ-ಲಿಲ್ಲ
ವ್ಯಕ್ತಿ-ಗ-ಳಲ್ಲೊಬ್ಬ-ರಾದ
ವ್ಯಕ್ತಿ-ಗ-ಳಾ-ಗ-ಬೇಕು
ವ್ಯಕ್ತಿ-ಗ-ಳಾ-ಗ-ಲಾ-ರರು
ವ್ಯಕ್ತಿ-ಗ-ಳಾಗಿ
ವ್ಯಕ್ತಿ-ಗ-ಳಾ-ಗುತ್ತಾರೆ
ವ್ಯಕ್ತಿ-ಗ-ಳಿಂದ
ವ್ಯಕ್ತಿ-ಗ-ಳಿ-ಗಿಂತಲೂ
ವ್ಯಕ್ತಿ-ಗ-ಳಿಗೆ
ವ್ಯಕ್ತಿ-ಗ-ಳಿ-ರು-ವುದು
ವ್ಯಕ್ತಿಗಳು
ವ್ಯಕ್ತಿಗಳೂ
ವ್ಯಕ್ತಿ-ಗ-ಳೆಂದು
ವ್ಯಕ್ತಿ-ಗ-ಳೆಂದೇ
ವ್ಯಕ್ತಿ-ಗ-ಳೆ-ಡೆಗೆ
ವ್ಯಕ್ತಿ-ಗ-ಳೆಲ್ಲ
ವ್ಯಕ್ತಿಗಳೇ
ವ್ಯಕ್ತಿ-ಗ-ಳೊಂದಿಗೆ
ವ್ಯಕ್ತಿ-ಗ-ಳೊ-ಳ-ಗಣ
ವ್ಯಕ್ತಿ-ಗ-ಳೊ-ಳಗೆ
ವ್ಯಕ್ತಿ-ಗಾ-ದರೂ
ವ್ಯಕ್ತಿಗೂ
ವ್ಯಕ್ತಿಗೆ
ವ್ಯಕ್ತಿ-ಗೌ-ರವ
ವ್ಯಕ್ತಿ-ಗೌ-ರ-ವಕ್ಕೆ
ವ್ಯಕ್ತಿ-ಜೀ-ವ-ನದ
ವ್ಯಕ್ತಿತ್ತ್ವ-ದಿಂದ
ವ್ಯಕ್ತಿತ್ವ
ವ್ಯಕ್ತಿತ್ವಕ್ಕೆ
ವ್ಯಕ್ತಿತ್ವ-ಗ-ಳಿಂದಾಗಿ
ವ್ಯಕ್ತಿತ್ವ-ಗಳು
ವ್ಯಕ್ತಿತ್ವದ
ವ್ಯಕ್ತಿತ್ವ-ವನ್ನು
ವ್ಯಕ್ತಿತ್ವ-ವನ್ನೇ
ವ್ಯಕ್ತಿತ್ವವೂ
ವ್ಯಕ್ತಿಯ
ವ್ಯಕ್ತಿಯನ್ನು
ವ್ಯಕ್ತಿಯನ್ನೂ
ವ್ಯಕ್ತಿಯಲ್ಲ
ವ್ಯಕ್ತಿಯಲ್ಲಿ
ವ್ಯಕ್ತಿ-ಯಲ್ಲಿ-ರುವ
ವ್ಯಕ್ತಿಯಲ್ಲೂ
ವ್ಯಕ್ತಿಯಷ್ಟು
ವ್ಯಕ್ತಿ-ಯಾ-ಗಲು
ವ್ಯಕ್ತಿಯಾಗಿ
ವ್ಯಕ್ತಿ-ಯಾ-ಗಿದ್ದರು
ವ್ಯಕ್ತಿ-ಯಾ-ಗಿದ್ದೆ-ನೆಂಬುದು
ವ್ಯಕ್ತಿ-ಯಾ-ಗಿಯೂ
ವ್ಯಕ್ತಿ-ಯಾ-ಗಿ-ರುತ್ತಿ-ರ-ಲಿಲ್ಲ
ವ್ಯಕ್ತಿಯಾದ
ವ್ಯಕ್ತಿಯು
ವ್ಯಕ್ತಿಯೂ
ವ್ಯಕ್ತಿಯೇ
ವ್ಯಕ್ತಿ-ಯೊಂದಿಗೆ
ವ್ಯಕ್ತಿ-ಯೊ-ಡನೆ
ವ್ಯಕ್ತಿಯೊಬ್ಬ
ವ್ಯಕ್ತಿ-ಯೊಬ್ಬ-ನಲ್ಲಿ
ವ್ಯಕ್ತಿ-ಯೊಬ್ಬನ
ವ್ಯಕ್ತಿ-ಯೊಬ್ಬ-ನನ್ನು
ವ್ಯಕ್ತಿ-ಯೊಬ್ಬ-ನಿಗೆ
ವ್ಯಕ್ತಿ-ಯೊಬ್ಬನು
ವ್ಯಕ್ತಿ-ಯೊಬ್ಬನೇ
ವ್ಯಕ್ತಿ-ವೈ-ಶಿಷ್ಟ್ಯದ
ವ್ಯಕ್ತಿವ್ಯಕ್ತಿ-ಗ-ಳೊ-ಳ-ಗಿನ
ವ್ಯಕ್ತಿವ್ಯಕ್ತಿ-ಗ-ಳಲ್ಲಿ
ವ್ಯಕ್ತಿವ್ಯಕ್ತಿ-ಯೊ-ಳ-ಗಿನ
ವ್ಯಕ್ತಿಸ್ವಾ-ತಂತ್ರ್ಯಕ್ಕೆ
ವ್ಯಕ್ತಿಸ್ವಾರ್ಥ
ವ್ಯಗ್ರ
ವ್ಯಗ್ರತೆ
ವ್ಯಗ್ರ-ತೆ-ಗಳ
ವ್ಯಗ್ರತೆಯ
ವ್ಯಗ್ರ-ತೆ-ಯನ್ನು
ವ್ಯಗ್ರನಾದ
ವ್ಯಗ್ರರಾಗಿ
ವ್ಯತಿ-ರಿಕ್ತ-ವಾ-ಗಿದ್ದರೂ
ವ್ಯತ್ಯಾಸ
ವ್ಯತ್ಯಾಸಕ್ಕೆ
ವ್ಯತ್ಯಾ-ಸ-ಗ-ಳಿ-ರ-ಬ-ಹು-ದಾ-ದರೂ
ವ್ಯತ್ಯಾಸದ
ವ್ಯತ್ಯಾ-ಸ-ದಿಂದ
ವ್ಯತ್ಯಾ-ಸ-ವಾ-ಗು-ವುದು
ವ್ಯತ್ಯಾ-ಸ-ವಿದೆ
ವ್ಯತ್ಯಾ-ಸ-ವಿ-ರ-ಲಿಲ್ಲ
ವ್ಯಥಿತ
ವ್ಯಥಿ-ತ-ನಾಗಿ
ವ್ಯಥಿ-ತ-ನಾ-ಗಿದ್ದಾನೆ
ವ್ಯಥೆ
ವ್ಯಥೆಗಳೂ
ವ್ಯಥೆಯನ್ನು
ವ್ಯಥೆ-ಯಲ್ಲಿದ್ದಾ-ಗಲೇ
ವ್ಯಥೆಯಿಂದ
ವ್ಯಭಿ-ಚಾ-ರ-ದಲ್ಲಿ
ವ್ಯಯ
ವ್ಯಯ-ಮಾ-ಡುತ್ತಿದ್ದಾರೆ
ವ್ಯಯ-ಮಾ-ಡುತ್ತಿವೆ
ವ್ಯಯ-ಮಾ-ಡು-ವುದೂ
ವ್ಯಯ-ವಾ-ಗುತ್ತದೆ
ವ್ಯಯಿಸಲು
ವ್ಯಯಿ-ಸ-ಬ-ಹುದು
ವ್ಯಯಿಸಿ
ವ್ಯಯಿ-ಸಿ-ಕೊಂಡು
ವ್ಯಯಿ-ಸಿ-ದರೆ
ವ್ಯಯಿ-ಸುತ್ತಾ-ರೆ-ಮಕ್ಕಳು
ವ್ಯರ್ಥ
ವ್ಯರ್ಥಪ್ರ-ಯಾ-ಸ-ಕ-ರವೂ
ವ್ಯರ್ಥ-ವಾ-ಗ-ದಂತೆ
ವ್ಯರ್ಥ-ವಾ-ಗದು
ವ್ಯರ್ಥವಾಗಿ
ವ್ಯರ್ಥ-ವಾ-ಗುತ್ತದೆ
ವ್ಯರ್ಥ-ವಾ-ದವು
ವ್ಯರ್ಥ-ವಾ-ಯಿ-ತಲ್ಲಾ
ವ್ಯರ್ಥವೆಂದು
ವ್ಯರ್ಥವೇ
ವ್ಯರ್ಥ-ಹ-ರ-ಟೆ-ಯಲ್ಲಿ
ವ್ಯರ್ಥಾಲಾಪ
ವ್ಯವಸಾಯ
ವ್ಯವ-ಸಾ-ಯಕ್ಕೆ
ವ್ಯವ-ಸಾ-ಯ-ದಿಂದ
ವ್ಯವಸ್ಥಿತ
ವ್ಯವಸ್ಥಿ-ತ-ವಾಗಿ
ವ್ಯವಸ್ಥಿ-ತ-ವಾದ
ವ್ಯವಸ್ಥೆ
ವ್ಯವಸ್ಥೆ-ಗಳ
ವ್ಯವಸ್ಥೆ-ಗಾಗಿ
ವ್ಯವಸ್ಥೆ-ಗಾ-ಗಿಯೂ
ವ್ಯವಸ್ಥೆಗೆ
ವ್ಯವಸ್ಥೆ-ನಮ್ಮ
ವ್ಯವಸ್ಥೆಯ
ವ್ಯವಸ್ಥೆ-ಯನ್ನು
ವ್ಯವಸ್ಥೆ-ಯನ್ನೂ
ವ್ಯವಸ್ಥೆ-ಯಲ್ಲಾ-ಗಲಿ
ವ್ಯವಸ್ಥೆ-ಯಲ್ಲಿ-ರಲಿ
ವ್ಯವಸ್ಥೆಯು
ವ್ಯವಸ್ಥೆಯೂ
ವ್ಯವ-ಹ-ರಿ-ಸ-ತೊ-ಡ-ಗಿ-ದರು
ವ್ಯವ-ಹ-ರಿ-ಸ-ಬೇಕು
ವ್ಯವ-ಹ-ರಿ-ಸಿ-ದರು
ವ್ಯವ-ಹ-ರಿ-ಸುತ್ತಾ
ವ್ಯವಹಾರ
ವ್ಯವ-ಹಾ-ರ-ಇವೇ
ವ್ಯವ-ಹಾ-ರ-ಕು-ಶ-ಲನು
ವ್ಯವ-ಹಾ-ರ-ಗ-ಳನ್ನೂ
ವ್ಯವ-ಹಾ-ರ-ಗ-ಳಲ್ಲಿ
ವ್ಯವ-ಹಾ-ರ-ಗ-ಳಲ್ಲೂ
ವ್ಯವ-ಹಾ-ರ-ಗ-ಳಿಂದ
ವ್ಯವ-ಹಾ-ರ-ಗಳೂ
ವ್ಯವ-ಹಾ-ರ-ಗ-ಳೆಲ್ಲ
ವ್ಯವ-ಹಾ-ರಜ್ಞಾ-ನ-ವಿಲ್ಲದ
ವ್ಯವ-ಹಾ-ರ-ದಲ್ಲಿ
ವ್ಯವ-ಹಾ-ರ-ದಲ್ಲಿದ್ದ-ವ-ನಾ-ಗಿದ್ದ
ವ್ಯವ-ಹಾ-ರ-ದಲ್ಲಿನ
ವ್ಯವ-ಹಾ-ರ-ದಿಂದ
ವ್ಯವ-ಹಾ-ರ-ವನ್ನು
ವ್ಯವ-ಹಾ-ರಸ್ಥ
ವ್ಯಸನಕ್ಕೆ
ವ್ಯಾಕುಲ
ವ್ಯಾಕುಲತೆ
ವ್ಯಾಕು-ಲ-ತೆ-ಗ-ಳಷ್ಟೇ
ವ್ಯಾಕು-ಲ-ತೆ-ಗ-ಳಿಂದ
ವ್ಯಾಕು-ಲ-ತೆ-ಗಳು
ವ್ಯಾಕು-ಲ-ತೆಯ
ವ್ಯಾಕು-ಲ-ತೆ-ಯನ್ನು
ವ್ಯಾಕು-ಲ-ತೆ-ಯಿಂದ
ವ್ಯಾಕು-ಲ-ತೆಯೂ
ವ್ಯಾಕು-ಲ-ತೆಯೇ
ವ್ಯಾಕು-ಲ-ನಾಗಿ
ವ್ಯಾಖ್ಯಾ-ನ-ವನ್ನು
ವ್ಯಾಧಿ
ವ್ಯಾಧಿಗಳ
ವ್ಯಾಧಿಗೆ
ವ್ಯಾಧಿಗ್ರಸ್ತ-ನಾ-ಗ-ಬೇ-ಕಿಲ್ಲ
ವ್ಯಾಧಿಗ್ರಸ್ತ-ರಾ-ಗು-ವುದು
ವ್ಯಾಧಿಯ
ವ್ಯಾಪಕ
ವ್ಯಾಪ-ಕ-ಜಾಲ
ವ್ಯಾಪಕವೂ
ವ್ಯಾಪ-ಕ-ಶಕ್ತಿ-ಯನ್ನೂ
ವ್ಯಾಪಾರ
ವ್ಯಾಪಾ-ರ-ಗ-ಳಲ್ಲಿ
ವ್ಯಾಪಾ-ರ-ಗ-ಳಿ-ಗಿಂತ
ವ್ಯಾಪಾ-ರ-ದಲ್ಲಿ
ವ್ಯಾಪಾ-ರ-ವನ್ನು
ವ್ಯಾಪಾ-ರಸ್ಥರ
ವ್ಯಾಪಾರಿ
ವ್ಯಾಪಾ-ರಿ-ಗಳ
ವ್ಯಾಪಾ-ರಿ-ಗ-ಳಾ-ದ-ವರು
ವ್ಯಾಪಾ-ರಿ-ಗ-ಳಿಂದ
ವ್ಯಾಪಾ-ರಿ-ಗಳು
ವ್ಯಾಪಾ-ರಿ-ಗ-ಳೊಬ್ಬರು
ವ್ಯಾಪಾ-ರಿ-ಗಿ-ರಾ-ಕಿ-ಗ-ಳಲ್ಲಿ-ರ-ಲಿ-ಎಲ್ಲೇ
ವ್ಯಾಪಾ-ರಿ-ಯಾಗಿ
ವ್ಯಾಪಾರಿಯೂ
ವ್ಯಾಪಾರೀ
ವ್ಯಾಪಾ-ರೋದ್ಯ-ಮಿ-ಗಳು
ವ್ಯಾಪಿ-ಯಾ-ಗಿ-ರುವ
ವ್ಯಾಪಿಯೂ
ವ್ಯಾಪಿಸಿ
ವ್ಯಾಪಿ-ಸಿ-ಕೊಂಡಿದೆ
ವ್ಯಾಪಿ-ಸಿ-ಕೊಂಡಿದ್ದವು
ವ್ಯಾಪಿಸಿದೆ
ವ್ಯಾಪಿ-ಸಿ-ದೆಯೆ
ವ್ಯಾಪಿ-ಸಿ-ಬಿಟ್ಟಿದೆ
ವ್ಯಾಪಿ-ಸಿ-ಬಿ-ಡುತ್ತದೆ
ವ್ಯಾಪಿ-ಸಿ-ಬಿ-ಡು-ವುದು
ವ್ಯಾಪ್ತಿಯಲ್ಲಿ
ವ್ಯಾಪ್ತಿಯುಳ್ಳ
ವ್ಯಾಮೋಹ
ವ್ಯಾಮೋಹಕ್ಕೆ
ವ್ಯಾಯಾಮ
ವ್ಯಾಯಾ-ಮ-ಗ-ಳನ್ನಾ-ಗಲಿ
ವ್ಯಾಯಾ-ಮ-ಗ-ಳನ್ನೂ
ವ್ಯಾಯಾಮದ
ವ್ಯಾಲೆಂಟೈನ್
ವ್ಯಾವ-ಹಾ-ರಿಕ
ವ್ಯಾವ-ಹಾ-ರಿಕ
ವ್ಯಾವ-ಹಾ-ರಿ-ಕ-ತೆ-ಇ-ವು-ಗಳ
ವ್ಯಾವ-ಹಾ-ರಿ-ಕ-ವಾಗಿ
ವ್ಯಾಸಂಗ
ವ್ಯುತ್ಪತ್ತಿ
ವ್ಯೂಹ
ವ್ಯೂಹದ
ವ್ರತ
ವ್ರತದಂತೆ
ವ್ಹಿಟ್ಮನ್
ಶ
ಶಂಕ-ರ-ಲಾ-ಲನ
ಶಂಕ-ರ-ಲಾ-ಲರೂ
ಶಂಕ-ರ-ಲಾಲ್
ಶಂಕ-ರಾ-ಚಾರ್ಯ-ರಿ-ಗೇನು
ಶಕ್ತ
ಶಕ್ತನಾಗಿ
ಶಕ್ತ-ನಾ-ದರೆ
ಶಕ್ತನಿಗೆ
ಶಕ್ತನೋ
ಶಕ್ತರಾಗಿ
ಶಕ್ತರಾದ
ಶಕ್ತವಾಗಿ
ಶಕ್ತಿ
ಶಕ್ತಿಯೊಂದು
ಶಕ್ತಿಆ
ಶಕ್ತಿ-ಕೇಂದ್ರ-ವನ್ನು
ಶಕ್ತಿಕೊಡು
ಶಕ್ತಿಗಳ
ಶಕ್ತಿ-ಗ-ಳನ್ನಾತ
ಶಕ್ತಿ-ಗ-ಳನ್ನು
ಶಕ್ತಿ-ಗ-ಳನ್ನೂ
ಶಕ್ತಿ-ಗ-ಳನ್ನೆಲ್ಲಾ
ಶಕ್ತಿ-ಗ-ಳಿಗೂ
ಶಕ್ತಿಗಳು
ಶಕ್ತಿಗಳೂ
ಶಕ್ತಿಗಳೇ
ಶಕ್ತಿಗಾಗಿ
ಶಕ್ತಿಗೂ
ಶಕ್ತಿಗೆ
ಶಕ್ತಿ-ದೇ-ವ-ರಿದ್ದಾನೆ
ಶಕ್ತಿ-ದೇ-ಹದ
ಶಕ್ತಿ-ಧಾರ್ಮಿಕ
ಶಕ್ತಿಬೇಕು
ಶಕ್ತಿಮೂಲ
ಶಕ್ತಿ-ಮೂ-ಲ-ಗ-ಳನ್ನು
ಶಕ್ತಿಯ
ಶಕ್ತಿ-ಯನ್ನಾ-ಗಲೀ
ಶಕ್ತಿ-ಯನ್ನೀವ
ಶಕ್ತಿಯನ್ನು
ಶಕ್ತಿ-ಯನ್ನುಂಟು-ಮಾ-ಡು-ವುದು
ಶಕ್ತಿಯನ್ನೂ
ಶಕ್ತಿ-ಯನ್ನೆಲ್ಲ
ಶಕ್ತಿಯನ್ನೇ
ಶಕ್ತಿ-ಯಲ್ಲಲ್ಲ
ಶಕ್ತಿಯಲ್ಲಿ
ಶಕ್ತಿ-ಯಲ್ಲಿಟ್ಟ
ಶಕ್ತಿಯಾಗಿ
ಶಕ್ತಿ-ಯಾ-ಗಿಯೂ
ಶಕ್ತಿ-ಯಾ-ಗುತ್ತವೆ
ಶಕ್ತಿಯಾದ
ಶಕ್ತಿಯಿಂದ
ಶಕ್ತಿಯು
ಶಕ್ತಿಯುಳ್ಳ
ಶಕ್ತಿ-ಯುಳ್ಳ-ವ-ರನ್ನೂ
ಶಕ್ತಿ-ಯುಳ್ಳವು
ಶಕ್ತಿಯೂ
ಶಕ್ತಿಯೆ
ಶಕ್ತಿಯೆಲ್ಲ
ಶಕ್ತಿಯೇ
ಶಕ್ತಿ-ಯೊಂದಿಗೆ
ಶಕ್ತಿ-ಯೊಂದಿದೆ
ಶಕ್ತಿ-ಯೊ-ಡನೆ
ಶಕ್ತಿ-ವಿ-ಶಿಷ್ಟ-ವಾದ
ಶಕ್ತಿವೃದ್ಧಿ
ಶಕ್ತಿ-ಶ-ರೀ-ರವು
ಶಕ್ತಿಶಾಲಿ
ಶಕ್ತಿ-ಶಾ-ಲಿ-ಯಾ-ಗುತ್ತದೆ
ಶಕ್ತಿ-ಶಾ-ಲಿ-ಯಾದ
ಶಕ್ತಿ-ಸಾ-ಗರ
ಶಕ್ತಿ-ಸಾಧ್ಯ-ತೆ-ಗ-ಳನ್ನ-ರಿ-ಯದೆ
ಶಕ್ತಿ-ಸಾ-ಮರ್ಥ್ಯದ
ಶಕ್ತಿಸ್ವ-ರೂ-ಪ-ಗ-ಳನ್ನು
ಶಕ್ತಿ-ಹೀ-ನರು
ಶಕ್ತಿ-ಹೀ-ನ-ವಾಗಿ
ಶಕ್ಯವಲ್ಲ
ಶತಃಸಿದ್ಧ
ಶತಪಥ
ಶತಮಾನ
ಶತ-ಮಾ-ನಕ್ಕೆ
ಶತ-ಮಾ-ನ-ಗಳ
ಶತ-ಮಾ-ನ-ಗ-ಳಲ್ಲೆಲ್ಲ
ಶತ-ಮಾ-ನ-ಗ-ಳಲ್ಲಿ
ಶತ-ಮಾ-ನ-ಗ-ಳಿಂದ
ಶತ-ಮಾ-ನ-ಗಳು
ಶತ-ಮಾ-ನ-ಗಳೇ
ಶತ-ಮಾ-ನದ
ಶತ-ಮಾ-ನ-ದಲ್ಲಿ
ಶತ-ಮಾ-ನ-ದಲ್ಲೇ
ಶತ-ಮಾ-ನ-ದಿಂದ
ಶತ-ಶ-ತ-ಮಾ-ನ-ಗಳ
ಶತ-ಶ-ತ-ಮಾ-ನ-ಗ-ಳಿಂದ
ಶತ್ರು
ಶತ್ರುಗಳ
ಶತ್ರು-ಗ-ಳನ್ನು
ಶತ್ರು-ಗ-ಳನ್ನೋ
ಶತ್ರು-ಗ-ಳಾ-ಗ-ಬಲ್ಲಿರಿ
ಶತ್ರು-ಗ-ಳಿಗೆ
ಶತ್ರುಪಕ್ಷ
ಶತ್ರುವಿಗೆ
ಶತ್ರುವಿನ
ಶಪಥ
ಶಪಿ-ಸ-ತೊ-ಡ-ಗಿದ
ಶಪಿ-ಸಿದ್ದಾರೆ
ಶಪಿ-ಸುತ್ತಾರೆ
ಶಬ್ದ
ಶಬ್ದಕೂಡ
ಶಬ್ದಕೋಶ
ಶಬ್ದ-ಕೋ-ಶ-ದಲ್ಲಿ
ಶಬ್ದಕ್ಕೆ
ಶಬ್ದಗಳ
ಶಬ್ದ-ಗ-ಳನ್ನಾ-ಡು-ವು-ದಿಲ್ಲ
ಶಬ್ದ-ಗ-ಳನ್ನು
ಶಬ್ದ-ಗ-ಳನ್ನೋ
ಶಬ್ದ-ಗ-ಳಲ್ಲಿ
ಶಬ್ದ-ಗ-ಳಲ್ಲೇ
ಶಬ್ದ-ಗ-ಳಿಂದ
ಶಬ್ದ-ಗ-ಳಿಗೆ
ಶಬ್ದಗಳು
ಶಬ್ದ-ಚ-ಮತ್ಕಾರ
ಶಬ್ದ-ತ-ರಂಗ-ಗ-ಳಿಂದ
ಶಬ್ದದ
ಶಬ್ದ-ದಿಂದಲೇ
ಶಬ್ದಪ್ರ-ಮಾಣ
ಶಬ್ದಪ್ರ-ಮಾ-ಣ-ವನ್ನೇ
ಶಬ್ದವನ್ನು
ಶಬ್ದವನ್ನೂ
ಶಬ್ದವಿದೆ
ಶಬ್ದವು
ಶಬ್ದವೂ
ಶಬ್ದವೇ
ಶಬ್ದಶಃ
ಶಮನ
ಶಮ-ನ-ಗೊ-ಳಿ-ಸುವ
ಶಮ-ನ-ವಾ-ಗುತ್ತದೆ
ಶಮ-ನ-ವಾದ
ಶಯವಾಗಿ
ಶರ-ಣ-ರನ್ನು
ಶರ-ಣಾ-ಗ-ತ-ನಾ-ಗಿದ್ದಾನೆ
ಶರ-ಣಾ-ಗ-ತ-ರಾಗಿ
ಶರ-ಣಾ-ಗ-ತ-ರಾ-ಗುತ್ತೇವೆ
ಶರ-ಣಾ-ಗ-ತ-ರಾ-ದ-ವರು
ಶರ-ಣಾ-ಗತಿಃ
ಶರ-ಣಾ-ಗ-ತಿ-ಗಳು
ಶರ-ಣಾ-ಗ-ತಿ-ಭಾವ
ಶರ-ಣಾ-ಗ-ತಿಯ
ಶರ-ಣಾ-ಗದೆ
ಶರ-ಣಾ-ಗ-ಬೇಕು
ಶರಣಾಗಿ
ಶರ-ಣಾ-ಗುವ
ಶರ-ಣಾ-ದರೆ
ಶರಣೀತ
ಶರಣು
ಶರಾ-ಬಿ-ಗಾಗಿ
ಶರಾಬು
ಶರೀರ
ಶರೀ-ರ-ಗ-ಳಾಗಿ
ಶರೀರದ
ಶರೀ-ರ-ದಲ್ಲಿ
ಶರೀ-ರ-ದಲ್ಲಿ-ತಂದೆ
ಶರೀ-ರ-ದಲ್ಲಿದ್ದ
ಶರೀ-ರ-ದಲ್ಲಿದ್ದು-ಕೊಂಡಿದ್ದರೂ
ಶರೀ-ರ-ದಲ್ಲಿ-ರುವ
ಶರೀ-ರ-ದಲ್ಲಿ-ರು-ವಾಗ
ಶರೀ-ರ-ದಿಂದ
ಶರೀ-ರ-ಧಾರಿ
ಶರೀ-ರ-ಧಾ-ರಿಯ
ಶರೀ-ರ-ಧಾ-ರಿ-ಯಾಗಿ
ಶರೀ-ರ-ವನ್ನು
ಶರೀ-ರ-ವನ್ನೂ
ಶರೀ-ರ-ವನ್ನೇ
ಶರೀ-ರ-ವಲ್ಲ
ಶರೀ-ರ-ವಿಜ್ಞಾನ
ಶರೀ-ರ-ವಿದೆ
ಶರೀರವು
ಶರೀರವೂ
ಶರೀರವೇ
ಶರೀ-ರ-ವೇನೋ
ಶರೀ-ರ-ಶಾಸ್ತ್ರಜ್ಞರೂ
ಶರೀ-ರಾದ್ಯಂತ
ಶರೀ-ರಾ-ರೋಗ್ಯಕ್ಕೆ
ಶರೀ-ರೇಂದ್ರಿ-ಯ-ಗ-ಳಿಂದ
ಶರ್ಮಾ
ಶಲ-ಭಾ-ಸನ
ಶವಕ್ಕೆ
ಶವಗಳು
ಶವದಲ್ಲಿ
ಶವವನ್ನು
ಶವವನ್ನೇ
ಶವ-ವಾ-ಗಿ-ರು-ವು-ದನ್ನು
ಶವ-ಸಂಸ್ಕಾರ
ಶಸ್ತ್ರ
ಶಸ್ತ್ರಗಳ
ಶಸ್ತ್ರ-ಚಿ-ಕಿತ್ಸೆ
ಶಸ್ತ್ರ-ಚಿ-ಕಿತ್ಸೆ-ಗ-ಳಿಂದ
ಶಸ್ತ್ರ-ಚಿ-ಕಿತ್ಸೆ-ಗಳು
ಶಸ್ತ್ರ-ಚಿ-ಕಿತ್ಸೆ-ಗಳೂ
ಶಸ್ತ್ರ-ಚಿ-ಕಿತ್ಸೆಗೆ
ಶಸ್ತ್ರ-ಚಿ-ಕಿತ್ಸೆಯ
ಶಸ್ತ್ರ-ಚಿ-ಕಿತ್ಸೆ-ಯನ್ನು
ಶಸ್ತ್ರ-ಚಿ-ಕಿತ್ಸೆ-ಯಲ್ಲಿ
ಶಸ್ತ್ರ-ಚಿ-ಕಿತ್ಸೆಯು
ಶಸ್ತ್ರ-ವೈದ್ಯರು
ಶಸ್ತ್ರಾಸ್ತ್ರಕ್ಕೆ
ಶಸ್ತ್ರಾಸ್ತ್ರ-ಗಳ
ಶಸ್ತ್ರಾಸ್ತ್ರ-ಗ-ಳನ್ನು
ಶಾಂತ
ಶಾಂತ-ಗೊ-ಳಿ-ಸ-ಬೇಕು
ಶಾಂತ-ಚಿತ್ತ-ನಾ-ಗಿ-ರ-ಬಲ್ಲ
ಶಾಂತ-ಚಿತ್ತ-ರಾಗಿ
ಶಾಂತ-ಚಿತ್ತ-ರಾ-ಗಿದ್ದು-ಕೊಂಡು
ಶಾಂತ-ಚಿತ್ತ-ರಾ-ಗಿರಿ
ಶಾಂತ-ಚಿತ್ತ-ರಾ-ಗಿ-ರುವ
ಶಾಂತತೆ
ಶಾಂತ-ತೆ-ಇವು
ಶಾಂತ-ನಾ-ಗಿದ್ದ
ಶಾಂತ-ಭಾ-ವ-ದಿಂದ
ಶಾಂತ-ಮ-ನಸ್ಕರೂ
ಶಾಂತಮ್ಮ
ಶಾಂತ-ವಾ-ಗಿಟ್ಟು-ಕೊಳ್ಳ-ಬೇ-ಕೆಂದು
ಶಾಂತ-ವಾ-ಗಿತ್ತು
ಶಾಂತ-ವಾ-ಯಿತು
ಶಾಂತವೂ
ಶಾಂತ-ಸಾ-ಗ-ರ-ದಲ್ಲಿನ
ಶಾಂತಾ
ಶಾಂತಿ
ಶಾಂತಿ-ಇ-ವು-ಗ-ಳನ್ನು
ಶಾಂತಿ-ಗ-ಳನ್ನು
ಶಾಂತಿ-ಗ-ಳಿಂದ
ಶಾಂತಿಗಾಗಿ
ಶಾಂತಿಗೆ
ಶಾಂತಿಪ್ರಿ-ಯವೂ
ಶಾಂತಿಯ
ಶಾಂತಿ-ಯನ್ನಾ-ಗಲೀ
ಶಾಂತಿಯನ್ನು
ಶಾಂತಿಯುತ
ಶಾಂತಿ-ಯೆ-ಡೆಗೆ
ಶಾಂತಿ-ಸು-ಭಿಕ್ಷೆ-ಗಳ
ಶಾಖ-ದಿಂದಲೇ
ಶಾಖೆ-ಗ-ಳಿವೆ
ಶಾಪ
ಶಾಪಗ್ರಸ್ತ-ರಾ-ಗುವ
ಶಾಮಣ್ಣ
ಶಾರ-ದಾ-ದೇ-ವಿ-ಯ-ವರ
ಶಾರ-ದಾ-ನಂದರ
ಶಾರ-ದಾ-ನಂದ-ರನ್ನು
ಶಾರ-ದಾ-ನಂದರು
ಶಾರ-ದಾಶ್ರಮ
ಶಾರೀರಿಕ
ಶಾಲಾ
ಶಾಲಾ-ಕಾ-ಲೇ-ಜು-ಗಳ
ಶಾಲಾ-ಕಾ-ಲೇ-ಜು-ಗ-ಳಿಂದ
ಶಾಲಾ-ಜೀ-ವ-ನದ
ಶಾಲಿನ
ಶಾಲಿಯಾದ
ಶಾಲು
ಶಾಲೆ
ಶಾಲೆ-ಗ-ಳನ್ನು
ಶಾಲೆ-ಗ-ಳಲ್ಲಿ
ಶಾಲೆಗಳು
ಶಾಲೆಗಳೆ
ಶಾಲೆಗೆ
ಶಾಲೆಯ
ಶಾಲೆಯನ್ನು
ಶಾಲೆಯಲ್ಲಿ
ಶಾಲೆ-ಯಲ್ಲಿ-ರಲಿ
ಶಾಲೆಯಲ್ಲೂ
ಶಾಲೆಯಲ್ಲೇ
ಶಾಶ್ವತ
ಶಾಶ್ವ-ತ-ಗೊ-ಳಿ-ಸಿದ
ಶಾಶ್ವ-ತ-ದೊ-ಳಿ-ಹು-ದೆಂತು
ಶಾಶ್ವ-ತ-ವಾಗಿ
ಶಾಶ್ವ-ತ-ವಾ-ಗಿದ್ದು
ಶಾಶ್ವ-ತ-ವಾದ
ಶಾಶ್ವ-ತ-ಸತ್ಯ-ವನ್ನು
ಶಾಸನ
ಶಾಸ-ನ-ಗ-ಳನ್ನು
ಶಾಸ-ನ-ಗ-ಳಿಂದಾ-ಗ-ತಕ್ಕದ್ದಲ್ಲ
ಶಾಸ-ನ-ದಲ್ಲಿ
ಶಾಸ್ತಿ-ಮಾ-ಡಲು
ಶಾಸ್ತ್ರ
ಶಾಸ್ತ್ರಗಳ
ಶಾಸ್ತ್ರ-ಗ-ಳಲ್ಲಿ
ಶಾಸ್ತ್ರ-ಗ-ಳಾ-ಗಲಿ
ಶಾಸ್ತ್ರಗಳು
ಶಾಸ್ತ್ರಗ್ರಂಥ-ಗಳ
ಶಾಸ್ತ್ರ-ಚಿಂತನೆ
ಶಾಸ್ತ್ರಜ್ಞ-ನಾ-ಗಿತ್ತು
ಶಾಸ್ತ್ರಜ್ಞರೂ
ಶಾಸ್ತ್ರಜ್ಞಾನ
ಶಾಸ್ತ್ರದ
ಶಾಸ್ತ್ರದಲ್ಲಿ
ಶಾಸ್ತ್ರವನ್ನು
ಶಾಸ್ತ್ರಿ
ಶಾಸ್ತ್ರಿ-ಕೆ-ಯನ್ನು
ಶಾಸ್ತ್ರೀಯ
ಶಾಸ್ತ್ರೋಕ್ತ
ಶಿಕ್ಷಕರು
ಶಿಕ್ಷಕರೂ
ಶಿಕ್ಷಣ
ಶಿಕ್ಷ-ಣ-ಇವು
ಶಿಕ್ಷ-ಣಕ್ಕಾಗಿ
ಶಿಕ್ಷಣಕ್ಕೂ
ಶಿಕ್ಷಣಕ್ಕೆ
ಶಿಕ್ಷ-ಣ-ಗ-ಳನ್ನು
ಶಿಕ್ಷ-ಣ-ಗಳು
ಶಿಕ್ಷ-ಣ-ಗಳೂ
ಶಿಕ್ಷ-ಣ-ತಜ್ಞ
ಶಿಕ್ಷ-ಣ-ತಜ್ಞ-ರನ್ನು
ಶಿಕ್ಷ-ಣ-ತಜ್ಞರು
ಶಿಕ್ಷಣದ
ಶಿಕ್ಷ-ಣ-ದಲ್ಲಿ
ಶಿಕ್ಷ-ಣ-ದ-ವ-ರೆಗೆ
ಶಿಕ್ಷ-ಣ-ದಿಂದ
ಶಿಕ್ಷ-ಣ-ದೊಂದಿಗೆ
ಶಿಕ್ಷ-ಣ-ಪದ್ಧತಿ
ಶಿಕ್ಷ-ಣ-ವನ್ನು
ಶಿಕ್ಷ-ಣ-ವನ್ನೂ
ಶಿಕ್ಷ-ಣ-ವಿ-ಧಾ-ನವು
ಶಿಕ್ಷ-ಣ-ವೆಂದರೆ
ಶಿಕ್ಷ-ಣ-ವೇತ್ತರೂ
ಶಿಕ್ಷ-ಣಾ-ಲ-ಯ-ದಲ್ಲಿ
ಶಿಕ್ಷಾರೂಪಿ
ಶಿಕ್ಷಾರೂಪೀ
ಶಿಕ್ಷಾ-ವ-ಧಿಯ
ಶಿಕ್ಷಿತ
ಶಿಕ್ಷಿ-ಸ-ಬಾ-ರ-ದೆಂದೂ
ಶಿಕ್ಷಿ-ಸ-ಬೇ-ಕಾ-ಗ-ಬ-ಹು-ದಾ-ದರೂ
ಶಿಕ್ಷಿ-ಸಿ-ಕೊಂಡಂತೆ
ಶಿಕ್ಷಿ-ಸು-ವ-ವರು
ಶಿಕ್ಷೆ
ಶಿಕ್ಷೆಗೆ
ಶಿಕ್ಷೆಯ
ಶಿಕ್ಷೆಯನ್ನು
ಶಿಖರ
ಶಿಖ-ರ-ಗ-ಳಂತಿದ್ದಾರೆ
ಶಿಖ-ರ-ದಲ್ಲಿ
ಶಿಖ-ರ-ವನ್ನೇ-ರಲು
ಶಿಖ-ರ-ವನ್ನೇರಿ
ಶಿಖ-ರ-ವನ್ನೇ-ರಿ-ದ-ವರು
ಶಿಖ-ರ-ವನ್ನೇ-ರಿ-ದು-ದನ್ನು
ಶಿಥಿ-ಲ-ಗೊ-ಳಿ-ಸಿ-ದರು
ಶಿಥಿ-ಲ-ವಾ-ಗುತ್ತಿದೆ
ಶಿಫಾರಸು
ಶಿರಚ್ಛೇ-ದನ
ಶಿರಸಾ
ಶಿರೋ-ನಾ-ಮೆ-ಗಳು
ಶಿರೋ-ನಾ-ಮೆಯ
ಶಿರೋ-ನಾ-ಮೆ-ಯಲ್ಲಿ
ಶಿಲಾ-ಧಾ-ರಕ್ಕೆ
ಶಿಲಾ-ಫ-ಲ-ಕ-ವೊಂದು
ಶಿಲು-ಬೆ-ಗೇ-ರಿ-ಸಿದ
ಶಿಲ್ಪ
ಶಿಲ್ಪ-ಕ-ಲಾ-ವಿದ
ಶಿಲ್ಪಿ-ಗ-ಳಾಗ
ಶಿವ
ಶಿವನ
ಶಿವನನ್ನು
ಶಿವನಿಗೆ
ಶಿವನು
ಶಿವರಾಂ
ಶಿವರಾತ್ರಿ
ಶಿವ-ರಾತ್ರಿಯ
ಶಿವರಾಮ
ಶಿವ-ಲಿಂಗದ
ಶಿವ-ಶಕ್ತಿ-ಗಳ
ಶಿವ-ಶಕ್ತಿ-ಯರ
ಶಿವಸ್ವ-ರೂ-ಪ-ರಾ-ಗ-ಬೇಕು
ಶಿವಾಜಿ
ಶಿವಾ-ನಂದ-ರನ್ನು
ಶಿವಾ-ನಂದರು
ಶಿಶು
ಶಿಶುಗಳ
ಶಿಶು-ಗ-ಳನ್ನು
ಶಿಶು-ಗ-ಳಲ್ಲಿ
ಶಿಶು-ಗ-ಳಿಗೆ
ಶಿಶುಗಳು
ಶಿಶುವನ್ನು
ಶಿಶು-ವಾ-ಗಿ-ರು-ವ-ವ-ರೆ-ಗಿನ
ಶಿಶು-ವಾ-ಗಿ-ರು-ವಾಗ
ಶಿಶು-ವಾ-ಗಿ-ರು-ವಾ-ಗಲೇ
ಶಿಶುವಿಗೆ
ಶಿಶುವಿನ
ಶಿಶು-ವಿ-ಯೋ-ಗದ
ಶಿಶುವು
ಶಿಶು-ವೆ-ನಿ-ಸಿದ
ಶಿಶು-ಸಂಗೋ-ಪನಾ
ಶಿಷ್ಟಾ-ಚಾ-ರ-ಗಳು
ಶಿಷ್ಯ
ಶಿಷ್ಯನನ್ನು
ಶಿಷ್ಯರ
ಶಿಷ್ಯರಲ್ಲಿ
ಶಿಷ್ಯ-ರಲ್ಲೊಬ್ಬ-ರಾದ
ಶಿಷ್ಯರಾದ
ಶಿಷ್ಯರಿಗೂ
ಶಿಷ್ಯರಿಗೆ
ಶಿಷ್ಯರು
ಶಿಷ್ಯರೂ
ಶಿಷ್ಯಾಗ್ರ-ಣಿ-ಗ-ಳಲ್ಲಿ
ಶಿಸ್ತಿಗೂ
ಶಿಸ್ತು
ಶಿಸ್ತುಗೇಡಿ
ಶಿಸ್ತು-ಗೇ-ಡಿ-ಗಳೂ
ಶೀಘ್ರ
ಶೀಘ್ರ-ಕೋ-ಪಿ-ಗ-ಳಾಗಿ
ಶೀಘ್ರದಲ್ಲೇ
ಶೀಘ್ರ-ಲಿ-ಪಿ-ಕಾರ್ತಿ
ಶೀಘ್ರವಾಗಿ
ಶೀತೋಷ್ಣ-ಗಳ
ಶೀರ್ಷಾಸನ
ಶೀರ್ಷಿಕೆ
ಶೀಲ
ಶೀಲಕ್ಕಿತ್ತ
ಶೀಲತೆ
ಶೀಲದ
ಶೀಲ-ನಿರ್ಮಾ-ಣದ
ಶೀಲವಂತ
ಶೀಲವನ್ನು
ಶೀಲವು
ಶೀಲವೂ
ಶೀಲ-ಸಂವರ್ಧ-ನೆಯ
ಶೀಲಾ
ಶೀಲಿ-ಸಿ-ದರೆ
ಶೀಲೋತ್ಕರ್ಷಕ್ಕೆ
ಶೀಲೋತ್ಕರ್ಷದ
ಶೀಲೋತ್ಕರ್ಷ-ದಿಂದ
ಶುಕ್ಲರು
ಶುಚಿ
ಶುಚಿ-ಗೊ-ಳಿ-ಸಲು
ಶುಚಿ-ಗೊ-ಳಿ-ಸುವ
ಶುಚಿತ್ವ-ಗಳ
ಶುಚಿತ್ವದ
ಶುಚಿ-ಯಾ-ಗಿಟ್ಟು-ಕೊಳ್ಳಿ
ಶುಚಿಯಾದ
ಶುದ್ಧ
ಶುದ್ಧವಾಗಿ
ಶುದ್ಧ-ವಾ-ಗಿದ್ದರೆ
ಶುದ್ಧಿ-ಯಿಂದಲೇ
ಶುದ್ಧೀ-ಕ-ರಣ
ಶುಭ
ಶುಭಂ
ಶುಭ-ಕ-ರ-ವಾದ
ಶುಭ-ಕಾ-ಮ-ನೆ-ಯಿಂದ
ಶುಭ-ಕಾರ್ಯ-ವನ್ನೂ
ಶುಭಕ್ಕಾಗಿ
ಶುಭ-ಚಿಂತ-ನೆ-ಯನ್ನೇ
ಶುಭ-ಚಿಂತ-ನೆ-ಯಿಂದ
ಶುಭ-ದಿ-ನ-ಗ-ಳನ್ನು
ಶುಭ-ನಿ-ರೀಕ್ಷೆ
ಶುಭ-ಯೋ-ಚನೆ
ಶುಭವನ್ನು
ಶುಭ-ವನ್ನುಂಟು-ಮಾ-ಡು-ವುದೆ
ಶುಭವನ್ನೂ
ಶುಭವನ್ನೇ
ಶುಭ-ವಾ-ಗಿಯೆ
ಶುಭ-ವಾ-ಗುತ್ತದೆ
ಶುಭ-ವಾ-ಗು-ವುದು
ಶುಭ-ಸಂಕಲ್ಪ-ದಿಂದ
ಶುಭಾಕಾಂಕ್ಷಿ
ಶುಭಾಕಾಂಕ್ಷೆ
ಶುಭಾಶಯ
ಶುಭೇಚ್ಛೆ
ಶುಭ್ರ-ವಾ-ಗದೇ
ಶುರು-ಮಾ-ಡಿ-ದಾಗ
ಶುರು-ಮಾ-ಡುತ್ತಾರೆ
ಶುರು-ವಾ-ಯಿತು
ಶುಶ್ರೂ-ಷಿ-ಸು-ವು-ದಕ್ಕಾಗಿ
ಶುಶ್ರೂಷೆ
ಶುಶ್ರೂ-ಷೆ-ಯನ್ನು
ಶುಶ್ರೂ-ಷೆ-ಯಲ್ಲಿ
ಶುಶ್ರೂಷೆಯೂ
ಶೂನ್ಯತೆ
ಶೂನ್ಯ-ಸಂಪಾ-ದ-ನೆ-ಯಲ್ಲ-ಅಲ್ಲ-ಮ-ನದು
ಶೂನ್ಯ-ಸಂಪಾ-ದ-ನೆ-ಯಾ-ದೀತು
ಶೃಂಖ-ಲೆ-ಯಿಂದ
ಶೃಂಗಾರ
ಶೃಂಗಾ-ರ-ವನ್ನು
ಶೆಟ್ಟಿಯೋ
ಶೇಕಡಾ
ಶೇಖರಣೆ
ಶೇಖ-ರ-ವಾ-ಗಿತ್ತು
ಶೇಖ-ರ-ವಾದ
ಶೇಖ-ರಿ-ಸಲು
ಶೈಕ್ಷಣಿಕ
ಶೈತ್ಯೋ-ಪ-ಚಾರ
ಶೈಲಿಗಾಗಿ
ಶೈಲಿಯಲ್ಲಿ
ಶೈಶವ
ಶೈಶವದ
ಶೈಶ-ವ-ದಲ್ಲಿ
ಶೈಶ-ವ-ದಿಂದ
ಶೈಶ-ವ-ದಿಂದಲೇ
ಶೋಕ
ಶೋಕವನ್ನು
ಶೋಕಾ-ಕು-ಲನೂ
ಶೋಕಿ-ಸು-ವು-ದಿಲ್ಲ
ಶೋಚನೀಯ
ಶೋಧನೆ
ಶೋಧ-ನೆ-ಗ-ಳನ್ನು
ಶೋಧ-ನೆ-ಗಳು
ಶೋಧನೆಗೆ
ಶೋಧನೆಯ
ಶೋಧನೆಯೂ
ಶೋಧಿ-ಸ-ಬೇ-ಕಾ-ಯಿತು
ಶೋಧಿ-ಸ-ಬೇಕು
ಶೋಧಿ-ಸಲ್ಪ-ಡದೆ
ಶೋಧಿಸಿ
ಶೋಧಿ-ಸಿ-ಕೊಳ್ಳ-ಬೇ-ಕಾ-ಯಿತು
ಶೋಧಿ-ಸಿ-ದಾಗ
ಶೋಧಿ-ಸಿ-ದಾ-ಗಲೂ
ಶೋಧಿಸುತ್ತಾ
ಶೋಧಿಸುವ
ಶೋಧಿ-ಸು-ವುದು
ಶೋಭಾ-ರಾ-ಮ-ನಾಗಿ
ಶೋಭಾ-ರಾ-ಮ-ನಿ-ಗಾಗಿ
ಶೋಭಾ-ರಾ-ಮನು
ಶೋಭಾರಾಮ್
ಶೋಭಿ
ಶೋಭಿ-ಸ-ತೊ-ಡ-ಗು-ವುದು
ಶೋಭಿಸುವ
ಶೋಷಕನೂ
ಶೋಷಕರ
ಶೋಷ-ಕ-ರಿಗೂ
ಶೋಷ-ಕ-ರಿಲ್ಲ-ವೆಂದಾ-ಗಲಿ
ಶೋಷ-ಕ-ಶೋ-ಷ-ಣೆ-ಗಳ
ಶೋಷಣೆ
ಶೋಷ-ಣೆ-ಗಾಗಿ
ಶೋಷ-ಣೆ-ಗೊ-ಳ-ಗಾದ
ಶೋಷಣೆಯ
ಶೋಷಣೆಯೇ
ಶೋಷಿತರ
ಶೋಷಿ-ತ-ರಿಗೆ
ಶೋಷಿತರು
ಶೋಷಿ-ತ-ರೆಂದಾ-ಗಲಿ
ಶೋಷಿಸದೆ
ಶೋಷಿಸಲೂ
ಶೋಷಿಸಿ
ಶೋಷಿ-ಸು-ವು-ದಕ್ಕಾಗಿ
ಶೌಚ
ಶೌಚಕ್ಕೆ
ಶೌಚ-ಗೃ-ಹಕ್ಕೆ
ಶೌಚವಿಲ್ಲ
ಶೌರ್ಯ
ಶ್ಮಶಾನ
ಶ್ಮಶಾ-ನ-ಗ-ಳಲ್ಲಿ
ಶ್ರದ್ಧಾ
ಶ್ರದ್ಧಾಕೇಂದ್ರ
ಶ್ರದ್ಧಾ-ಕೇಂದ್ರ-ಗ-ಳಲ್ಲ
ಶ್ರದ್ಧಾ-ಕೇಂದ್ರ-ವಾ-ದ-ಧರ್ಮ
ಶ್ರದ್ಧಾ-ಕೇಂದ್ರ-ವಿದೆ
ಶ್ರದ್ಧಾ-ಕೇಂದ್ರವೇ
ಶ್ರದ್ಧಾನ್ವಿತ
ಶ್ರದ್ಧಾನ್ವಿ-ತ-ರಿಂದ
ಶ್ರದ್ಧಾ-ಬಿಂದು-ಗ-ಳನ್ನು
ಶ್ರದ್ಧಾಭಕ್ತಿ
ಶ್ರದ್ಧಾ-ಭಕ್ತಿ-ಗ-ಳಿಂದ
ಶ್ರದ್ಧಾ-ಭಕ್ತಿ-ಗ-ಳಿಲ್ಲದೇ
ಶ್ರದ್ಧಾ-ಭಕ್ತಿ-ಗಳು
ಶ್ರದ್ಧಾ-ಭಕ್ತಿ-ಯಿಂದ
ಶ್ರದ್ಧಾ-ಭಕ್ತಿ-ಯೆಂಬ
ಶ್ರದ್ಧಾ-ಭಕ್ತಿ-ಸ-ಹಿತ
ಶ್ರದ್ಧಾ-ರ-ಹಿತ
ಶ್ರದ್ಧಾರ್ಘ್ಯ-ವನ್ನು
ಶ್ರದ್ಧಾವಂತ
ಶ್ರದ್ಧಾವಾನ್
ಶ್ರದ್ಧಾ-ಶೀ-ಲ-ರಾ-ಗುತ್ತಾರೆ
ಶ್ರದ್ಧೆ
ಶ್ರದ್ಧೆ-ಇ-ವು-ಗಳೇ
ಶ್ರದ್ಧೆಗಳ
ಶ್ರದ್ಧೆ-ಗ-ಳಿಂದ
ಶ್ರದ್ಧೆ-ಗ-ಳಿ-ಗ-ನು-ಗು-ಣ-ವಾಗಿ
ಶ್ರದ್ಧೆಗಳು
ಶ್ರದ್ಧೆಗೂ
ಶ್ರದ್ಧೆಗೆ
ಶ್ರದ್ಧೆ-ಪ-ರ-ಮಾರ್ಥ-ದಲ್ಲಿನ
ಶ್ರದ್ಧೆ-ಭಕ್ತಿ-ಗ-ಳಿಂದ
ಶ್ರದ್ಧೆಯ
ಶ್ರದ್ಧೆ-ಯನ್ನ-ನು-ಸ-ರಿಸಿ
ಶ್ರದ್ಧೆ-ಯನ್ನಿಟ್ಟಿದ್ದರು
ಶ್ರದ್ಧೆ-ಯನ್ನಿ-ಡಲು
ಶ್ರದ್ಧೆಯನ್ನು
ಶ್ರದ್ಧೆಯಿಂದ
ಶ್ರದ್ಧೆ-ಯಿಂದಲೇ
ಶ್ರದ್ಧೆಯಿಡು
ಶ್ರದ್ಧೆಯಿಲ್ಲ
ಶ್ರದ್ಧೆಯು
ಶ್ರದ್ಧೆ-ಯುಂಟಾ-ಗು-ವಂತೆ
ಶ್ರದ್ಧೆಯೇ
ಶ್ರದ್ಧೆಯೊಂದೇ
ಶ್ರಮ
ಶ್ರಮಕ್ಕೆ
ಶ್ರಮದ
ಶ್ರಮ-ದಾ-ಯ-ಕ-ವಾ-ಗು-ವು-ದರ
ಶ್ರಮದಿಂದ
ಶ್ರಮ-ದಿಂದಲೇ
ಶ್ರಮ-ಪಟ್ಟರೂ
ಶ್ರಮಪಟ್ಟು
ಶ್ರಮ-ಪ-ಡ-ಲಿಲ್ಲ
ಶ್ರಮ-ಭಾ-ರ-ವನ್ನು
ಶ್ರಮವನ್ನು
ಶ್ರಮವನ್ನೇ
ಶ್ರಮ-ವ-ಹಿಸಿ
ಶ್ರಮವಿದ್ದೇ
ಶ್ರಮಶೀಲ
ಶ್ರಮ-ಸಾಧ್ಯ-ವಾದ
ಶ್ರಮಿ-ಸ-ಬೇ-ಕಾ-ಗಿದೆ
ಶ್ರಮಿ-ಸ-ಬೇಕು
ಶ್ರಮಿ-ಸ-ಬೇಡಿ
ಶ್ರಮಿಸಿ
ಶ್ರಮಿಸಿದ
ಶ್ರಮಿ-ಸಿ-ದರೂ
ಶ್ರಮಿ-ಸಿ-ದರೆ
ಶ್ರಮಿ-ಸಿ-ದ-ವರು
ಶ್ರಮಿ-ಸುತ್ತಾರೆ
ಶ್ರಮಿ-ಸು-ವುದೇ
ಶ್ರವ-ಣಾ-ತೀತ
ಶ್ರವ-ಣೇಂದ್ರಿ-ಯಕ್ಕೆ
ಶ್ರವ-ಣೇಂದ್ರಿ-ಯ-ಗ-ಳಾ-ಗಲಿ
ಶ್ರಾವ-ಣ-ಮಾ-ಸ-ದಲ್ಲಿ
ಶ್ರೀ
ಶ್ರೀಸಾ-ಮಾನ್ಯರ
ಶ್ರೀಕೃಷ್ಣ
ಶ್ರೀಕೃಷ್ಣ-ಚೈ-ತನ್ಯರು
ಶ್ರೀಕೃಷ್ಣನ
ಶ್ರೀಗು-ರು-ವಿ-ನಲ್ಲಿ
ಶ್ರೀಚಂದ-ನೇಶ್ವರ
ಶ್ರೀನಿವಾಸ
ಶ್ರೀನಿ-ವಾ-ಸಪ್ಪ
ಶ್ರೀನಿ-ವಾ-ಸಪ್ಪ-ನ-ವರ
ಶ್ರೀನಿ-ವಾ-ಸಪ್ಪ-ನ-ವ-ರಿಗೆ
ಶ್ರೀನಿ-ವಾ-ಸಪ್ಪ-ನ-ವರು
ಶ್ರೀನಿ-ವಾ-ಸ-ರಿಂದ
ಶ್ರೀಬ್ರಹ್ಮ-ಚೈ-ತನ್ಯ
ಶ್ರೀಮಂತ
ಶ್ರೀಮಂತನ
ಶ್ರೀಮಂತ-ನಾಗಿ
ಶ್ರೀಮಂತ-ನಾ-ಗಿದ್ದರೂ
ಶ್ರೀಮಂತ-ನಾ-ಗಿ-ರಲಿ
ಶ್ರೀಮಂತ-ನಿ-ಗಿಂತ
ಶ್ರೀಮಂತರ
ಶ್ರೀಮಂತರೇ
ಶ್ರೀಮಂತ-ರೊಬ್ಬರ
ಶ್ರೀಮಂತಿಕೆ
ಶ್ರೀಮಂತಿ-ಕೆಯ
ಶ್ರೀಮತಿ
ಶ್ರೀಮತಿಯ
ಶ್ರೀಮದ್ಗಾಂಭೀರ್ಯದ
ಶ್ರೀಮಹಾ
ಶ್ರೀಮಾತೆ
ಶ್ರೀಮಾತೆಯ
ಶ್ರೀಮಾ-ತೆ-ಯನ್ನು
ಶ್ರೀಮು-ಖ-ದಿಂದ
ಶ್ರೀಯುತ
ಶ್ರೀಯುತರು
ಶ್ರೀರಾಮ
ಶ್ರೀರಾ-ಮ-ಕೃಷ್ಣ
ಶ್ರೀರಾ-ಮ-ಕೃಷ್ಣ
ಶ್ರೀರಾ-ಮ-ಕೃಷ್ಣರ
ಶ್ರೀರಾ-ಮ-ಕೃಷ್ಣ-ರನ್ನು
ಶ್ರೀರಾ-ಮ-ಕೃಷ್ಣ-ರಲ್ಲಿ
ಶ್ರೀರಾ-ಮ-ಕೃಷ್ಣ-ರಿಗೆ
ಶ್ರೀರಾ-ಮ-ಕೃಷ್ಣರು
ಶ್ರೀರಾ-ಮ-ಕೃಷ್ಣ-ರೆಂದರು
ಶ್ರೀರಾ-ಮ-ಕೃಷ್ಣ-ರೆಂದಿದ್ದ-ರು-ಸಾ-ವೆಂದರೆ
ಶ್ರೀರಾ-ಮ-ಕೃಷ್ಣ-ರೆನ್ನುತ್ತಿದ್ದರು
ಶ್ರೀರಾ-ಮ-ದಾ-ಸರ
ಶ್ರೀರಾಮನ
ಶ್ರೀರಾ-ಮ-ನದು
ಶ್ರೀರಾ-ಮ-ನನ್ನು
ಶ್ರೀರಾ-ಮ-ನಾ-ಮಸ್ಮ-ರ-ಣೆ-ಯನ್ನು
ಶ್ರೀರಾ-ಮ-ನಿಗೆ
ಶ್ರೀರಾಮನು
ಶ್ರೀವಿ-ಜ-ಯ-ದಾ-ಸ-ರನ್ನು
ಶ್ರೀಶಾ-ರ-ದಾ-ದೇ-ವಿ-ಯನ್ನು
ಶ್ರೀಶಾ-ರ-ದಾ-ದೇ-ವಿ-ಯ-ವರು
ಶ್ರೀಸಾಮಾನ್ಯ
ಶ್ರೀಸಾ-ಮಾನ್ಯನ
ಶ್ರೀಸಾ-ಮಾನ್ಯರ
ಶ್ರೀಸಾ-ಮಾನ್ಯ-ರನ್ನು
ಶ್ರೀಸಾ-ಮಾನ್ಯ-ರಿಗೂ
ಶ್ರೀಸಾ-ಮಾನ್ಯ-ರಿಗೆ
ಶ್ರೀಸಾ-ಮಾನ್ಯರೂ
ಶ್ರೀಹರ್ಷ
ಶ್ರುತ-ಪ-ಡಿ-ಸಿ-ದರು
ಶ್ರುತಿ
ಶ್ರುತಿ-ಗೂ-ಡಿ-ಸುವ
ಶ್ರುತಿ-ಗೂ-ಡಿ-ಸು-ವು-ದ-ರಲ್ಲಿ
ಶ್ರುತಿ-ಗೊ-ಳಿ-ಸ-ಬ-ಹು-ದಾ-ದರೆ
ಶ್ರುತಿ-ಗೊ-ಳಿ-ಸು-ವಂತೆ
ಶ್ರೇಣಿಯಲ್ಲಿ
ಶ್ರೇಣಿ-ಯಲ್ಲಿ-ರುವ
ಶ್ರೇಯಸ್ಕರ
ಶ್ರೇಯಸ್ಸನ್ನುಂಟು
ಶ್ರೇಯಸ್ಸನ್ನೂ
ಶ್ರೇಯಸ್ಸಾ-ಗುತ್ತದೆ
ಶ್ರೇಯಸ್ಸಿಗೆ
ಶ್ರೇಯಸ್ಸಿನ
ಶ್ರೇಯಸ್ಸು
ಶ್ರೇಷ್ಠ
ಶ್ರೇಷ್ಠತೆ
ಶ್ರೇಷ್ಠತೆಯ
ಶ್ರೇಷ್ಠ-ತೆ-ಯನ್ನು
ಶ್ರೇಷ್ಠರು
ಶ್ರೇಷ್ಠರೆಂದು
ಶ್ರೇಷ್ಠವಾದ
ಶ್ರೇಷ್ಠ-ವಾ-ದುದು
ಶ್ರೋಡರ್
ಶ್ರೋತ್ರ
ಶ್ಲಾಘ-ನೆ-ಗ-ಳನ್ನೂ
ಶ್ಲಾಘ್ಯವೇ
ಶ್ಲೋಕ
ಶ್ಲೋಕ-ಗ-ಳನ್ನೂ
ಶ್ಲೋಕವನ್ನೊ
ಶ್ಲೋಕ-ವೊಂದ-ರಲ್ಲಿ
ಶ್ವಾನವತ್
ಶ್ವಾಸ-ಕೋ-ಶ-ಗಳ
ಶ್ವಾಸ-ಕೋ-ಶದ
ಶ್ವಾಸೋಚ್ಛ್ವಾಸ
ಶ್ವಾಸೋಚ್ಛ್ವಾಸ
ಶ್ವಾಸೋಚ್ಛ್ವಾ-ಸಕ್ರಿಯೆ
ಶ್ವಾಸೋಚ್ಛ್ವಾ-ಸ-ಗೈ-ಯುತ್ತಾ
ಶ್ವೇ
ಶ್ವೇತಾಶ್ವ-ತರ
ಶ್ಶಾಸ್ತ್ರದ
ಷಡ್ಭು-ಜ-ಗ-ಳನ್ನೂ
ಷಡ್ವಿಧಾ
ಷರತ್ತು-ಗಳು
ಷಿಂಡ್ಲರ್
ಷೆಡ್ಡಿನಲ್ಲಿ
ಷೆಲ್
ಷೋಕಿ
ಷೋಪೆನ್ಹಾ-ವರ್
ಷೋಫ-ನೀ-ಯರ್
ಷೋಫೆ-ನೀ-ರನ
ಸಂ
ಸಂಕಟ
ಸಂಕಟಕ್ಕೂ
ಸಂಕಟಕ್ಕೆ
ಸಂಕ-ಟಕ್ಕೊ-ಳ-ಗಾ-ಗುತ್ತ-ವೆಂದು
ಸಂಕ-ಟ-ಗಳ
ಸಂಕ-ಟ-ಗ-ಳನ್ನು
ಸಂಕ-ಟ-ಗ-ಳಿಂದ
ಸಂಕ-ಟ-ಗ-ಳಿಗೂ
ಸಂಕ-ಟ-ಗ-ಳಿಗೆ
ಸಂಕ-ಟ-ಗ-ಳಿವೆ
ಸಂಕ-ಟ-ಗಳು
ಸಂಕ-ಟ-ಗ-ಳೇಕೆ
ಸಂಕ-ಟಗ್ರಸ್ತ-ರಾದ
ಸಂಕ-ಟಗ್ರಸ್ತ-ವಾಗಿ
ಸಂಕ-ಟಗ್ರಸ್ತ-ವಾ-ಗಿದೆ
ಸಂಕ-ಟಗ್ರಸ್ತ-ವಾ-ಗುತ್ತದೆ
ಸಂಕ-ಟ-ದಲ್ಲಿ
ಸಂಕ-ಟ-ದಿಂದ
ಸಂಕ-ಟ-ಪಟ್ಟು
ಸಂಕ-ಟ-ಪ-ಡುತ್ತಿದ್ದಳು
ಸಂಕ-ಟ-ಮಯ
ಸಂಕ-ಟ-ವನ್ನು
ಸಂಕ-ಟ-ವನ್ನೂ
ಸಂಕ-ಟ-ವನ್ನೋ
ಸಂಕ-ಟ-ವಾ-ಗುತ್ತದೆ
ಸಂಕ-ಟ-ವಾ-ಯಿತು
ಸಂಕ-ಟ-ವುಂಟಾ-ಗುತ್ತದೆ
ಸಂಕಟವೇ
ಸಂಕಲ್ಪ
ಸಂಕಲ್ಪಃ
ಸಂಕಲ್ಪ-ಇ-ವು-ಗ-ಳಿಂದ
ಸಂಕಲ್ಪ-ಗಳೇ
ಸಂಕಲ್ಪ-ದಿಂದ
ಸಂಕಲ್ಪ-ದಿಂದಲೇ
ಸಂಕಲ್ಪ-ಮಾಡಿ
ಸಂಕಲ್ಪ-ವನ್ನು
ಸಂಕಲ್ಪ-ವಾ-ಗಿತ್ತು
ಸಂಕಲ್ಪಿ-ಸಿದ
ಸಂಕಲ್ಪಿ-ಸಿ-ದರು
ಸಂಕಲ್ಪಿ-ಸಿದ್ದ
ಸಂಕಲ್ಪಿ-ಸಿದ್ದರು
ಸಂಕಲ್ಪಿ-ಸಿ-ರ-ದಿದ್ದರೂ
ಸಂಕಲ್ಪಿ-ಸಿ-ರುವ
ಸಂಕಷ್ಟ-ಗ-ಳನ್ನು
ಸಂಕಷ್ಟ-ಗ-ಳಿಂದ
ಸಂಕಷ್ಟ-ಗ-ಳೆಲ್ಲ
ಸಂಕೀರ್ಣ
ಸಂಕೀರ್ಣ-ತೆ-ಯನ್ನಾ-ಗಲೀ
ಸಂಕೀರ್ಣ-ತೆ-ಯನ್ನು
ಸಂಕೀರ್ಣ-ಯಂತ್ರ-ದಲ್ಲಿ
ಸಂಕೀರ್ಣ-ಯಂತ್ರ-ವನ್ನು
ಸಂಕೀರ್ಣವೂ
ಸಂಕುಚಿತ
ಸಂಕು-ಚಿ-ತ-ಗೊ-ಳಿ-ಸು-ವುದು
ಸಂಕು-ಚಿ-ತ-ಗೊಳ್ಳ-ಬ-ಹುದು
ಸಂಕು-ಚಿ-ತತೆ
ಸಂಕು-ಚಿ-ತ-ತೆಗೆ
ಸಂಕು-ಚಿ-ತ-ತೆ-ಯಿಂದ
ಸಂಕೇತ
ಸಂಕೇ-ತ-ಗಳ
ಸಂಕೇ-ತ-ಗ-ಳಿವೆ
ಸಂಕೇ-ತ-ಗಳು
ಸಂಕೇ-ತ-ವಾ-ಗ-ಬೇ-ಕಾ-ಗುತ್ತದೆ
ಸಂಕೇ-ತ-ವಾಗಿ
ಸಂಕೇತವೇ
ಸಂಕೇತವೋ
ಸಂಕೋಚ
ಸಂಕೋ-ಚ-ಪ-ಡ-ಬೇಕು
ಸಂಕೋ-ಚ-ಪ-ಡ-ಬೇಡ
ಸಂಕೋ-ಚ-ಬೇಡ
ಸಂಕೋ-ಚ-ವೆ-ನಿ-ಸ-ಬ-ಹುದು
ಸಂಕೋಲೆ
ಸಂಕೋ-ಲೆ-ಯಿಂದ
ಸಂಕ್ರಾಂತಿ
ಸಂಕ್ಷಿಪ್ತ
ಸಂಕ್ಷಿಪ್ತ-ರೂಪ
ಸಂಕ್ಷಿಪ್ತ-ವಾಗಿ
ಸಂಕ್ಷೇ-ಪ-ವಾಗಿ
ಸಂಖ್ಯಾ
ಸಂಖ್ಯೆ
ಸಂಖ್ಯೆಗಳ
ಸಂಖ್ಯೆಗೂ
ಸಂಖ್ಯೆಯ
ಸಂಖ್ಯೆಯಲ್ಲಿ
ಸಂಖ್ಯೆಯಿಂದ
ಸಂಗ-ಡಿ-ಗರು
ಸಂಗ-ತ-ವಲ್ಲದ
ಸಂಗ-ತ-ವಾ-ಗಿದೆ
ಸಂಗ-ತ-ವಾ-ಗಿವೆ
ಸಂಗ-ತ-ವೆಂದಾ-ದರೆ
ಸಂಗ-ತ-ವೆಂದು
ಸಂಗತಿ
ಸಂಗ-ತಿ-ಗಳ
ಸಂಗ-ತಿ-ಗ-ಳನ್ನು
ಸಂಗ-ತಿ-ಗ-ಳಲ್ಲ
ಸಂಗ-ತಿ-ಗ-ಳಲ್ಲಿ
ಸಂಗ-ತಿ-ಗ-ಳಲ್ಲೂ
ಸಂಗ-ತಿ-ಗ-ಳಿಗೂ
ಸಂಗ-ತಿ-ಗ-ಳಿವು
ಸಂಗ-ತಿ-ಗಳು
ಸಂಗ-ತಿ-ಗಳೂ
ಸಂಗ-ತಿ-ಯನ್ನಂತೂ
ಸಂಗ-ತಿ-ಯನ್ನು
ಸಂಗ-ತಿ-ಯನ್ನೂ
ಸಂಗ-ತಿ-ಯಲ್ಲ
ಸಂಗ-ತಿ-ಯಲ್ಲಿ
ಸಂಗ-ತಿ-ಯಲ್ಲೂ
ಸಂಗ-ತಿ-ಯಾಗಿ
ಸಂಗ-ತಿ-ಯಾ-ಗು-ವುದು
ಸಂಗತಿಯೂ
ಸಂಗ-ತಿ-ಯೆಂದು
ಸಂಗತಿಯೇ
ಸಂಗ-ತಿ-ಸಂತೆ-ಯಲ್ಲಿ
ಸಂಗದಲ್ಲಿ
ಸಂಗವದು
ಸಂಗಾತಿ
ಸಂಗಾ-ತಿ-ಯೊಬ್ಬ
ಸಂಗೀತ
ಸಂಗೀ-ತ-ಇವು
ಸಂಗೀತಕ್ಕೆ
ಸಂಗೀ-ತ-ಗ-ಳಾ-ಗಲಿ
ಸಂಗೀ-ತ-ಗಾರ
ಸಂಗೀ-ತ-ಗಾ-ರ-ರನ್ನು
ಸಂಗೀ-ತ-ತಜ್ಞ-ರಿ-ರಲಿ
ಸಂಗೀತದ
ಸಂಗೀ-ತ-ವನ್ನು
ಸಂಗೀ-ತಾಭ್ಯಾ-ಸಿ-ಗಳು
ಸಂಗೀ-ತಾ-ಸಕ್ತಿ
ಸಂಗ್ರಹ
ಸಂಗ್ರ-ಹ-ಕಾರ್ಯ-ದಲ್ಲಿ
ಸಂಗ್ರಹಕ್ಕೆ
ಸಂಗ್ರ-ಹ-ಗ-ಳಲ್ಲಿ
ಸಂಗ್ರ-ಹ-ಗೈ-ಯುವ
ಸಂಗ್ರ-ಹ-ದಂತಲ್ಲ
ಸಂಗ್ರ-ಹ-ದಿಂದಾ-ಗಲೀ
ಸಂಗ್ರ-ಹ-ಮಾ-ಡಿಟ್ಟು-ಕೊಳ್ಳು-ವುದು
ಸಂಗ್ರ-ಹ-ಮಾ-ಡಿ-ಡ-ಬ-ಹುದು
ಸಂಗ್ರ-ಹ-ವಲ್ಲ
ಸಂಗ್ರ-ಹ-ವಾ-ಗದೊ
ಸಂಗ್ರ-ಹ-ವಾ-ಗಿ-ರುವ
ಸಂಗ್ರ-ಹ-ವಾ-ಗುತ್ತ
ಸಂಗ್ರ-ಹ-ವಾದ
ಸಂಗ್ರಹವೂ
ಸಂಗ್ರ-ಹಿ-ಸದೆ
ಸಂಗ್ರ-ಹಿ-ಸ-ಬ-ಹುದು
ಸಂಗ್ರ-ಹಿ-ಸ-ಲಾ-ಗಿದೆ
ಸಂಗ್ರಹಿಸಿ
ಸಂಗ್ರ-ಹಿ-ಸಿಟ್ಟಿದ್ದಾರೆ
ಸಂಗ್ರ-ಹಿ-ಸಿಟ್ಟು-ಕೊಂಡಿವೆ
ಸಂಗ್ರ-ಹಿ-ಸಿದ
ಸಂಗ್ರ-ಹಿ-ಸಿ-ದಂತೆಯೇ
ಸಂಗ್ರ-ಹಿ-ಸಿದ್ದಾರೆ
ಸಂಗ್ರ-ಹಿ-ಸಿದ್ದೇನೆ
ಸಂಗ್ರ-ಹಿ-ಸಿ-ಬಿ-ಡುತ್ತದೆ
ಸಂಗ್ರ-ಹಿ-ಸಿ-ರ-ಬ-ಹುದು
ಸಂಗ್ರ-ಹಿ-ಸಿ-ರುತ್ತದೆ
ಸಂಗ್ರ-ಹಿ-ಸುತ್ತ
ಸಂಗ್ರ-ಹಿ-ಸುತ್ತದೆ
ಸಂಗ್ರ-ಹಿ-ಸುತ್ತ-ಲಿದೆ
ಸಂಗ್ರ-ಹಿ-ಸುತ್ತಿದ್ದಾರೆ
ಸಂಗ್ರ-ಹಿ-ಸುತ್ತಿ-ರುವ
ಸಂಗ್ರ-ಹಿ-ಸು-ವು-ದಕ್ಕಾಗಿ
ಸಂಗ್ರ-ಹಿ-ಸು-ವು-ದಲ್ಲ
ಸಂಗ್ರ-ಹಿ-ಸು-ವುದೇ
ಸಂಗ್ರಾ-ಮ-ವನ್ನು
ಸಂಘ
ಸಂಘ-ಜೀ-ವಿ-ಯಾ-ಗಲಿ
ಸಂಘ-ಟ-ನೆ-ಗ-ಳಲ್ಲಿ
ಸಂಘ-ಟ-ನೆಯ
ಸಂಘಟಿತ
ಸಂಘ-ಟಿ-ತ-ರಾಗಿ
ಸಂಘದ
ಸಂಘರ್ಷ
ಸಂಘರ್ಷ-ದಲ್ಲಿ
ಸಂಘರ್ಷವೂ
ಸಂಘವನ್ನು
ಸಂಘ-ಸಂಸ್ಥೆ-ಗಳು
ಸಂಚಕಾರ
ಸಂಚಯ
ಸಂಚ-ಯ-ನಕ್ಕಾಗಿ
ಸಂಚ-ಯ-ವಾ-ಗು-ವುದು
ಸಂಚ-ಯ-ವಿದೆ
ಸಂಚ-ರಿ-ಸ-ಬೇ-ಕಾ-ಗಿ-ರ-ಲಿಲ್ಲ
ಸಂಚರಿಸಿ
ಸಂಚ-ರಿ-ಸಿ-ದರೂ
ಸಂಚ-ರಿ-ಸುತ್ತಿತ್ತು
ಸಂಚ-ರಿ-ಸುತ್ತಿದ್ದ
ಸಂಚ-ರಿ-ಸುತ್ತೇನೆ
ಸಂಚ-ರಿ-ಸುವ
ಸಂಚ-ರಿ-ಸು-ವಾಗ
ಸಂಚಾರ
ಸಂಚಾರದ
ಸಂಚಾ-ರ-ವನ್ನು
ಸಂಚಾ-ರ-ಶಕ್ತಿಯೂ
ಸಂಚಾರಿ
ಸಂಚಾರೀ
ಸಂಚಿ-ಕೆ-ಯನ್ನೂ
ಸಂಚಿ-ಕೆ-ಯಲ್ಲಿ
ಸಂಚಿಗೆ
ಸಂಚಿತ
ಸಂಚು
ಸಂಚು-ಎನ್ನುವ
ಸಂಜಾತ
ಸಂಜೀ-ವಿ-ನಿಯ
ಸಂಜೀ-ವಿ-ನಿ-ಯಾದ
ಸಂಜೆ
ಸಂಜೆಗೆ
ಸಂಜೆಯ
ಸಂಜೆಯಲ್ಲಿ
ಸಂಜೆ-ಯ-ವ-ರೆಗೂ
ಸಂಜೆ-ಯ-ವ-ರೆಗೆ
ಸಂಜೆ-ಯ-ಹೊತ್ತು
ಸಂಜೆಯಾದ
ಸಂಜೆ-ಯಾ-ಯಿತು
ಸಂಡೇ
ಸಂತ
ಸಂತ-ತಿ-ಯೊ-ಡನೆ
ಸಂತನಲ್ಲೂ
ಸಂತನೊಬ್ಬ
ಸಂತನೋ
ಸಂತಪ್ತ-ನಾದ
ಸಂತಪ್ತ-ನಾ-ದಾಗ
ಸಂತ-ಮ-ಹಾತ್ಮರ
ಸಂತರ
ಸಂತರನ್ನು
ಸಂತರಿಗೇ
ಸಂತರು
ಸಂತ-ರೆಲ್ಲರೂ
ಸಂತ-ರೊಬ್ಬ-ರಿಗೆ
ಸಂತ-ರೊಬ್ಬರು
ಸಂತಶ್ರೇಷ್ಠ-ನನ್ನಾಗಿ
ಸಂತಶ್ರೇಷ್ಠ-ರಲ್ಲೊಬ್ಬ-ರಾಗಿ
ಸಂತಸ
ಸಂತಸದ
ಸಂತ-ಸ-ದಿಂದ
ಸಂತಾನ
ಸಂತಾ-ನ-ವತ್ಸ-ಲೆ-ಯಾದ
ಸಂತಾ-ನ-ವನ್ನು
ಸಂತಾ-ನ-ವಿ-ಹೀ-ನ-ರಾ-ಗು-ವು-ದಕ್ಕೆ
ಸಂತಾ-ನ-ವೃದ್ಧಿ-ಗಾಗಿ
ಸಂತಾ-ಪ-ವನ್ನು
ಸಂತಾ-ಪ-ವನ್ನೂ
ಸಂತುಷ್ಟ-ನಲ್ಲೂ
ಸಂತುಷ್ಟ-ನಾ-ದಂತೆ
ಸಂತುಷ್ಟ-ರಾಗಿ
ಸಂತುಷ್ಟ-ರಾ-ಗಿ-ರಲು
ಸಂತುಷ್ಟ-ರಾ-ಗು-ವು-ದರ
ಸಂತುಷ್ಟಳು
ಸಂತೃಪ್ತ-ರಾ-ಗಿ-ರು-ವ-ವರೂ
ಸಂತೃಪ್ತಿ
ಸಂತೃಪ್ತಿ-ಗಳ
ಸಂತೃಪ್ತಿ-ಗಳು
ಸಂತೃಪ್ತಿಯ
ಸಂತೃಪ್ತಿ-ಯನ್ನು
ಸಂತೃಪ್ತಿ-ಯಾ-ಗ-ಬೇಕು
ಸಂತೃಪ್ತಿಯೇ
ಸಂತೆಯ
ಸಂತೋಷ
ಸಂತೋ-ಷ-ದಿಂದಿ-ರ-ಬಲ್ಲರು
ಸಂತೋ-ಷ-ವಾ-ಗಿ-ರುತ್ತೀರಿ
ಸಂತೋ-ಷಕ್ಕಾಗಿ
ಸಂತೋಷಕ್ಕೆ
ಸಂತೋ-ಷ-ಗ-ಳನ್ನುಂಟು
ಸಂತೋಷದ
ಸಂತೋ-ಷ-ದಾ-ಯ-ಕವೂ
ಸಂತೋ-ಷ-ದಿಂದ
ಸಂತೋ-ಷ-ದಿಂದಲೇ
ಸಂತೋ-ಷ-ಪಟ್ಟು
ಸಂತೋ-ಷ-ಪ-ಡು-ವು-ದನ್ನು
ಸಂತೋ-ಷ-ವನ್ನು
ಸಂತೋ-ಷ-ವನ್ನುಂಟು-ಮಾ-ಡುವ
ಸಂತೋ-ಷ-ವಾ-ಗಿ-ರದೆ
ಸಂತೋ-ಷ-ವಾ-ದರೂ
ಸಂತೋ-ಷ-ವಾ-ಯಿತು
ಸಂತೋಷವೂ
ಸಂತ್ರಸ್ತರ
ಸಂದ-ಣಿ-ಯಿದ್ದು-ದ-ರಿಂದ
ಸಂದರ್ಭ
ಸಂದರ್ಭಕ್ಕೆ
ಸಂದರ್ಭ-ಗ-ಳನ್ನು
ಸಂದರ್ಭ-ಗ-ಳಲ್ಲಿ
ಸಂದರ್ಭ-ಗ-ಳಲ್ಲೂ-ಜ-ನರು
ಸಂದರ್ಭ-ಗಳು
ಸಂದರ್ಭ-ದಲ್ಲಿ
ಸಂದರ್ಭ-ದಲ್ಲೂ
ಸಂದರ್ಭ-ವನ್ನು
ಸಂದರ್ಭ-ವನ್ನೂ
ಸಂದರ್ಭ-ವಿಲ್ಲ-ವೆಂದು
ಸಂದರ್ಭ-ವೊ-ದ-ಗಿ-ದಾಗ
ಸಂದರ್ಶಕ
ಸಂದರ್ಶ-ಕ-ನೊಬ್ಬ
ಸಂದರ್ಶ-ಕನೇ
ಸಂದರ್ಶ-ಕರು
ಸಂದರ್ಶನ
ಸಂದರ್ಶಿಸಿ
ಸಂದರ್ಶಿ-ಸಿ-ದಾಗ
ಸಂದರ್ಶಿ-ಸುತ್ತೇವೆ
ಸಂದಿಗ್ಧ
ಸಂದು
ಸಂದು-ಗ-ಳಲ್ಲಿ
ಸಂದು-ಹೋ-ದವು
ಸಂದೇಶ
ಸಂದೇ-ಶ-ಗಳ
ಸಂದೇ-ಶ-ಗ-ಳನ್ನು
ಸಂದೇ-ಶ-ಗ-ಳನ್ನೂ
ಸಂದೇ-ಶ-ದಿಂದಲೇ
ಸಂದೇ-ಶ-ವನ್ನು
ಸಂದೇ-ಶ-ವಾ-ಯಿತು
ಸಂದೇ-ಶ-ವಿದೆ
ಸಂದೇಶವು
ಸಂದೇಹ
ಸಂದೇ-ಹಕ್ಕೆ-ಡೆ-ಗೊ-ಡದ
ಸಂದೇ-ಹ-ಗ-ಳನ್ನು
ಸಂದೇ-ಹ-ಗಳು
ಸಂದೇ-ಹ-ಗಳೆ
ಸಂದೇ-ಹ-ಪ-ಡ-ಬೇಡ
ಸಂದೇ-ಹ-ಪ-ಡು-ವು-ದಿಲ್ಲ
ಸಂದೇ-ಹ-ರ-ಹಿ-ತ-ರಾ-ಗು-ವೆವು
ಸಂದೇ-ಹ-ವಾ-ದಿ-ಗ-ಳಿಗೆ
ಸಂದೇ-ಹ-ವಿಲ್ಲ
ಸಂದೇ-ಹ-ವಿಲ್ಲ-ವೆನ್ನುತ್ತಾನೆ
ಸಂದೇಹವೇ
ಸಂಧಿಸಲು
ಸಂಧಿಸಿ
ಸಂಧಿ-ಸಿ-ದಾಗ
ಸಂಧಿಸುವ
ಸಂಧ್ಯೆ
ಸಂನ್ಯಾಸಿ
ಸಂನ್ಯಾ-ಸಿ-ಗ-ಳನ್ನೂ
ಸಂನ್ಯಾ-ಸಿ-ಮ-ಠ-ವನ್ನು
ಸಂನ್ಯಾ-ಸಿ-ಯಾ-ಗಿದ್ದೆ
ಸಂನ್ಯಾಸಿಯೋ
ಸಂಪತ್ತನ್ನು
ಸಂಪತ್ತನ್ನೂ
ಸಂಪತ್ತನ್ನೆಲ್ಲ
ಸಂಪತ್ತಿ-ಗಿಂತಲೂ
ಸಂಪತ್ತಿ-ಗೇನು
ಸಂಪತ್ತಿದೆ
ಸಂಪತ್ತಿನ
ಸಂಪತ್ತಿ-ನಿಂದ
ಸಂಪತ್ತಿ-ನಿಂದಲೇ
ಸಂಪತ್ತಿ-ರುವ
ಸಂಪತ್ತು
ಸಂಪತ್ತೆಷ್ಟೋ
ಸಂಪನ್ನ-ನಾದ
ಸಂಪನ್ನರ
ಸಂಪನ್ನ-ರಾದ
ಸಂಪನ್ನ-ರೆ-ನಿ-ಸಿ-ಕೊಂಡ-ವ-ರಿಗೇ
ಸಂಪನ್ಮೂ-ಲ-ಗಳ
ಸಂಪನ್ಮೂ-ಲ-ಗ-ಳುಳ್ಳ
ಸಂಪರ್ಕ
ಸಂಪರ್ಕಕ್ಕೆ
ಸಂಪರ್ಕ-ದಿಂದ
ಸಂಪರ್ಕ-ದಿಂದಾಗಿ
ಸಂಪರ್ಕವೇ
ಸಂಪರ್ಕಿ-ಸಿ-ದರು
ಸಂಪಾ
ಸಂಪಾ-ದ-ಕ-ರೊಬ್ಬರು
ಸಂಪಾದನೆ
ಸಂಪಾ-ದ-ನೆಯ
ಸಂಪಾ-ದಿ-ಸ-ಬೇ-ಕೆಂಬುದು
ಸಂಪಾ-ದಿ-ಸಲು
ಸಂಪಾದಿಸಿ
ಸಂಪಾ-ದಿ-ಸಿ-ಕೊಂಡು
ಸಂಪಾ-ದಿ-ಸಿಕೋ
ಸಂಪಾ-ದಿ-ಸಿದ
ಸಂಪಾ-ದಿ-ಸಿ-ದರು
ಸಂಪಾ-ದಿ-ಸಿ-ದರೆ
ಸಂಪಾ-ದಿ-ಸಿ-ದೆ-ಯಲ್ಲವೆ
ಸಂಪಾ-ದಿ-ಸಿದ್ದ
ಸಂಪಾ-ದಿ-ಸಿದ್ದರು
ಸಂಪಾ-ದಿ-ಸಿದ್ದರೆ
ಸಂಪಾ-ದಿ-ಸಿದ್ದಾ-ರೆಯೆ
ಸಂಪಾ-ದಿ-ಸಿ-ರ-ಬೇಕು
ಸಂಪಾ-ದಿ-ಸುತ್ತಿದ್ದಾರೆ
ಸಂಪಾ-ದಿ-ಸು-ವುದು
ಸಂಪೂರ್ಣ
ಸಂಪೂರ್ಣ-ವಾಗಿ
ಸಂಪ್ರದಾಯ
ಸಂಪ್ರ-ದಾ-ಯಕ್ಕೆ
ಸಂಪ್ರ-ದಾ-ಯ-ಗ-ಳಲ್ಲಿ
ಸಂಪ್ರ-ದಾ-ಯ-ಗ-ಳಿಗೆ
ಸಂಪ್ರ-ದಾ-ಯದ
ಸಂಪ್ರ-ದಾ-ಯ-ದಲ್ಲಿ
ಸಂಪ್ರ-ದಾ-ಯ-ವನ್ನೂ
ಸಂಪ್ರ-ದಾ-ಯ-ವಾ-ದಿ-ಗಳ
ಸಂಪ್ರ-ದಾ-ಯಸ್ಥ
ಸಂಬಂಧ
ಸಂಬಂಧ-ದಿಂದಲೇ
ಸಂಬಂಧಕ್ಕೆ
ಸಂಬಂಧ-ಗಳ
ಸಂಬಂಧ-ಗ-ಳದ್ದು
ಸಂಬಂಧ-ಗ-ಳನ್ನು
ಸಂಬಂಧ-ಗ-ಳಲ್ಲಿ
ಸಂಬಂಧ-ಗಳು
ಸಂಬಂಧದ
ಸಂಬಂಧ-ದಂತೆ
ಸಂಬಂಧ-ದಲ್ಲಿ
ಸಂಬಂಧ-ದಿಂದ
ಸಂಬಂಧ-ದಿಂದಲೇ
ಸಂಬಂಧ-ಪಟ್ಟ
ಸಂಬಂಧ-ವನ್ನು
ಸಂಬಂಧ-ವನ್ನೂ
ಸಂಬಂಧ-ವನ್ನೇ
ಸಂಬಂಧ-ವಾಗಿ
ಸಂಬಂಧ-ವಾದ
ಸಂಬಂಧ-ವಿದೆ
ಸಂಬಂಧ-ವಿಲ್ಲ
ಸಂಬಂಧ-ವಿಲ್ಲದೆ
ಸಂಬಂಧವು
ಸಂಬಂಧವೂ
ಸಂಬಂಧ-ವೆಂಬುದು
ಸಂಬಂಧ-ವೇ-ನಾ-ದರೂ
ಸಂಬಂಧ-ವೇನು
ಸಂಬಂಧಿ
ಸಂಬಂಧಿ-ಕ-ನೊಬ್ಬ-ನಿದ್ದು
ಸಂಬಂಧಿ-ಗಳ
ಸಂಬಂಧಿ-ಗಳು
ಸಂಬಂಧಿತ
ಸಂಬಂಧಿ-ಯಾದ
ಸಂಬಂಧಿ-ಸದ
ಸಂಬಂಧಿಸಿ
ಸಂಬಂಧಿ-ಸಿದ
ಸಂಬಂಧಿ-ಸಿ-ದ-ವರ
ಸಂಬಂಧಿ-ಸಿ-ದವು
ಸಂಬಂಧಿ-ಸಿ-ದ-ವು-ಗಳು
ಸಂಬಂಧಿ-ಸಿದ್ದಾ-ದರೆ
ಸಂಬಂಧಿ-ಸಿದ್ದು
ಸಂಬಂಧಿ-ಸಿ-ರ-ದಿದ್ದರೂ
ಸಂಬಳ
ಸಂಬ-ಳ-ದಿಂದ
ಸಂಬ-ಳ-ವನ್ನು
ಸಂಬೋ
ಸಂಬೋ-ಧಿ-ಸ-ಲಾ-ರಂಭಿ-ಸಿ-ದರೆ
ಸಂಬೋಧಿಸಿ
ಸಂಭ-ವ-ನೀ-ಯವೋ
ಸಂಭ-ವ-ವಿತ್ತು
ಸಂಭ-ವ-ವಿದೆ
ಸಂಭ-ವ-ವಿ-ದೆಯೆ
ಸಂಭ-ವ-ವಿ-ದೆಯೇ
ಸಂಭವವೇ
ಸಂಭವವೋ
ಸಂಭವಿಸಿ
ಸಂಭ-ವಿ-ಸಿ-ದಾ-ಗ-ಲೆಲ್ಲಾ
ಸಂಭಾವ್ಯ
ಸಂಭಾ-ಷ-ಣಾ-ಚ-ತು-ರರು
ಸಂಭಾಷಣೆ
ಸಂಭಾ-ಷಿ-ಸಲು
ಸಂಭ್ರ-ಮ-ದಿಂದ
ಸಂಮೋಹಃ
ಸಂಮೋಹಾತ್
ಸಂಮೋ-ಹಿ-ನಿ-ವಿದ್ಯೆಯ
ಸಂಮೋಹಿನೀ
ಸಂಯಮ
ಸಂಯ-ಮ-ಗಳ
ಸಂಯ-ಮ-ಗ-ಳನ್ನು
ಸಂಯ-ಮ-ಗ-ಳಿಲ್ಲದ
ಸಂಯ-ಮ-ಗ-ಳಿಲ್ಲ-ದ-ವನ
ಸಂಯಮದ
ಸಂಯ-ಮ-ದಿಂದ
ಸಂಯ-ಮ-ದಿಂದಿ-ರಲು
ಸಂಯ-ಮ-ಯುಕ್ತ
ಸಂಯ-ಮ-ವಿಲ್ಲದೇ
ಸಂಯ-ಮಿ-ಗಳ
ಸಂಯುಕ್ತ
ಸಂರಕ್ಷ-ಣ-ಶೀಲ
ಸಂವರ್ಧ-ನೆಗೆ
ಸಂವರ್ಧ-ನೆಯ
ಸಂವಾ-ದ-ದಲ್ಲಿ
ಸಂವಿ-ಧಾ-ನ-ವಿ-ರ-ದಿದ್ದರೂ
ಸಂವೇ-ದ-ನಾ-ಶೀ-ಲ-ರಾ-ಗಿದ್ದೀರಾ
ಸಂವೇದನೆ
ಸಂವೇ-ದ-ನೆ-ಗಳೂ
ಸಂಶಯ
ಸಂಶ-ಯ-ದಿಂದಲೋ
ಸಂಶಯಃ
ಸಂಶ-ಯಕ್ಕೆ-ಡೆ-ಕೊ-ಡ-ದಂತಹ
ಸಂಶ-ಯ-ಗಳ
ಸಂಶ-ಯ-ಗ-ಳನ್ನು
ಸಂಶ-ಯ-ಗ-ಳನ್ನೇ
ಸಂಶ-ಯ-ಗ-ಳಿಂದ
ಸಂಶ-ಯ-ಗ-ಳಿಗೂ
ಸಂಶ-ಯ-ಗ-ಳಿಗೆ
ಸಂಶ-ಯ-ಗಳು
ಸಂಶ-ಯ-ಗಳೂ
ಸಂಶ-ಯಗ್ರಸ್ತ-ತೆಯೇ
ಸಂಶ-ಯಗ್ರಸ್ತ-ನಾದ
ಸಂಶ-ಯಗ್ರಸ್ತ-ರಾ-ಗ-ಬೇ-ಕಾ-ಗಿಲ್ಲ
ಸಂಶಯದ
ಸಂಶ-ಯ-ದಿಂದ
ಸಂಶ-ಯ-ದೃಷ್ಟಿಗೆ
ಸಂಶ-ಯ-ಪಿ-ಶಾಚಿ
ಸಂಶ-ಯ-ರ-ಹಿತ
ಸಂಶ-ಯ-ರ-ಹಿ-ತ-ನಾದ
ಸಂಶ-ಯ-ರ-ಹಿ-ತ-ರಾ-ಗುತ್ತೇವೆ
ಸಂಶ-ಯ-ರ-ಹಿ-ತ-ವಾದ
ಸಂಶ-ಯ-ರಾಕ್ಷ-ಸ-ನೊಮ್ಮೆ
ಸಂಶ-ಯ-ವನ್ನು
ಸಂಶ-ಯ-ವಾ-ಗಿತ್ತು
ಸಂಶ-ಯ-ವಾ-ದಿಯ
ಸಂಶ-ಯ-ವಿ-ದೆಯೇ
ಸಂಶ-ಯ-ವಿಲ್ಲ
ಸಂಶ-ಯ-ವಿಲ್ಲದೆ
ಸಂಶ-ಯ-ವಿಲ್ಲದೇ
ಸಂಶಯವೆ
ಸಂಶೋಧಕ
ಸಂಶೋ-ಧ-ಕನ
ಸಂಶೋ-ಧ-ಕ-ನನ್ನಾಗಿ
ಸಂಶೋ-ಧ-ಕರ
ಸಂಶೋ-ಧ-ಕ-ರಾದ
ಸಂಶೋ-ಧ-ಕರು
ಸಂಶೋ-ಧ-ಕರೂ
ಸಂಶೋಧನಾ
ಸಂಶೋ-ಧ-ನಾತ್ಮಕ
ಸಂಶೋ-ಧ-ನಾ-ನಿ-ರ-ತ-ರಾದ
ಸಂಶೋ-ಧ-ನಾ-ನಿ-ರ-ತರು
ಸಂಶೋ-ಧ-ನಾ-ಲ-ಯ-ಗಳ
ಸಂಶೋಧನೆ
ಸಂಶೋ-ಧ-ನೆ-ಯಲ್ಲಿ-ರುವ
ಸಂಶೋ-ಧ-ನೆ-ಗಳ
ಸಂಶೋ-ಧ-ನೆ-ಗ-ಳನ್ನು
ಸಂಶೋ-ಧ-ನೆ-ಗ-ಳಲ್ಲಿ
ಸಂಶೋ-ಧ-ನೆ-ಗ-ಳಲ್ಲಿ-ವಿದ್ಯುಚ್ಛಕ್ತಿ
ಸಂಶೋ-ಧ-ನೆ-ಗ-ಳಿಂದ
ಸಂಶೋ-ಧ-ನೆ-ಗ-ಳಿಂದಾದ
ಸಂಶೋ-ಧ-ನೆ-ಗ-ಳಿ-ಗಾಗಿ
ಸಂಶೋ-ಧ-ನೆ-ಗ-ಳಿ-ಗಿಂತ
ಸಂಶೋ-ಧ-ನೆ-ಗ-ಳಿಗೆ
ಸಂಶೋ-ಧ-ನೆ-ಗಳು
ಸಂಶೋ-ಧ-ನೆ-ಗಳೇ
ಸಂಶೋ-ಧ-ನೆ-ಗಾಗಿ
ಸಂಶೋ-ಧ-ನೆಗೆ
ಸಂಶೋ-ಧ-ನೆಯ
ಸಂಶೋ-ಧ-ನೆ-ಯನ್ನು
ಸಂಶೋ-ಧ-ನೆ-ಯಲ್ಲಿ
ಸಂಶೋ-ಧ-ನೆ-ಯಿಂದ
ಸಂಶೋ-ಧಿ-ಸ-ತೊ-ಡ-ಗು-ವುದೊ
ಸಂಶೋ-ಧಿ-ಸ-ಲಾ-ರದ
ಸಂಶೋ-ಧಿ-ಸಿದ
ಸಂಸಾರ
ಸಂಸಾರದ
ಸಂಸಾ-ರ-ದಲ್ಲಿ
ಸಂಸಾ-ರ-ದಲ್ಲಿದ್ದರೂ
ಸಂಸಾ-ರ-ದಲ್ಲಿ-ರು-ವು-ದಕ್ಕೆ
ಸಂಸಾ-ರ-ದಲ್ಲೇ
ಸಂಸಾ-ರ-ವಂದಿ-ಗ-ರಿಗೆ
ಸಂಸಾ-ರ-ವನ್ನು
ಸಂಸಾರವೇ
ಸಂಸಾ-ರ-ವೇ-ನೆಂದು
ಸಂಸಾ-ರ-ಸಾ-ಗ-ರ-ವನ್ನು
ಸಂಸಾ-ರಿ-ಕರು
ಸಂಸಾ-ರಿ-ಗ-ಳಾದ
ಸಂಸಾ-ರಿ-ಗ-ಳಿಗೂ
ಸಂಸಾರಿಯೋ
ಸಂಸ್ಕ-ರಿ-ಸುವ
ಸಂಸ್ಕಾರ
ಸಂಸ್ಕಾ-ರ-ವನ್ನುಂಟು
ಸಂಸ್ಕಾ-ರ-ಗಳ
ಸಂಸ್ಕಾ-ರ-ಗ-ಳನ್ನು
ಸಂಸ್ಕಾ-ರ-ಗ-ಳನ್ನೂ
ಸಂಸ್ಕಾ-ರ-ಗ-ಳಿಂದ
ಸಂಸ್ಕಾ-ರ-ಗಳು
ಸಂಸ್ಕಾ-ರ-ಗಳೂ
ಸಂಸ್ಕಾ-ರ-ಗ-ಳೆಲ್ಲ
ಸಂಸ್ಕಾ-ರ-ಗಳೇ
ಸಂಸ್ಕಾರದ
ಸಂಸ್ಕಾ-ರ-ದಿಂದ
ಸಂಸ್ಕಾ-ರ-ವನ್ನು
ಸಂಸ್ಕಾ-ರ-ವನ್ನುಂಟು
ಸಂಸ್ಕೃತ
ಸಂಸ್ಕೃತದ
ಸಂಸ್ಕೃತಿ
ಸಂಸ್ಕೃ-ತಿ-ಗಳ
ಸಂಸ್ಕೃ-ತಿ-ಗ-ಳನ್ನೂ
ಸಂಸ್ಕೃ-ತಿ-ಗ-ಳಿಂದ
ಸಂಸ್ಕೃತಿಯ
ಸಂಸ್ಕೃ-ತಿ-ಯನ್ನು
ಸಂಸ್ಕೃ-ತಿ-ಯನ್ನೂ
ಸಂಸ್ಕೃ-ತಿ-ಯಲ್ಲಿ
ಸಂಸ್ಕೃ-ತಿ-ವೃಕ್ಷ
ಸಂಸ್ಥಾನದ
ಸಂಸ್ಥೆ
ಸಂಸ್ಥೆ-ಗ-ಳನ್ನು
ಸಂಸ್ಥೆ-ಗ-ಳಲ್ಲಿ
ಸಂಸ್ಥೆ-ಗ-ಳಿವೆ
ಸಂಸ್ಥೆಗಳು
ಸಂಸ್ಥೆಗಳೂ
ಸಂಸ್ಥೆಯ
ಸಂಸ್ಥೆಯನ್ನು
ಸಂಸ್ಥೆಯನ್ನೂ
ಸಂಸ್ಥೆಯಲ್ಲಿ
ಸಂಸ್ಥೆ-ಯಲ್ಲಿದ್ದ
ಸಂಸ್ಥೆಯು
ಸಂಸ್ಥೆಯೇ
ಸಂಸ್ಥೆಯೊಂದು
ಸಂಸ್ಪರ್ಶದ
ಸಂಸ್ಪರ್ಶ-ದಲ್ಲೇ
ಸಂಸ್ಪರ್ಶ-ದಿಂದ
ಸಂಸ್ಪರ್ಶ-ವನ್ನು
ಸಕಲ
ಸಕ-ಲ-ಭ-ಯ-ತಲ್ಲ-ಣ-ವನ್ನು
ಸಕ-ಲ-ವಿಧ
ಸಕಾ-ಲ-ದಲ್ಲಿ
ಸಕ್ಕರೆ
ಸಕ್ಕರೆಯ
ಸಕ್ಕ-ರೆ-ಯಾಗಿ
ಸಖನೆಂದು
ಸಖೇ-ದಾಶ್ಚರ್ಯ-ಕರ
ಸಖ್ಯವನ್ನೂ
ಸಗು-ಣ-ಸಾ-ಕಾ-ರದ
ಸಚಿತ್ರ
ಸಚಿತ್ರ-ವಾಗಿ
ಸಚ್ಚ-ರಿ-ತೆಯ
ಸಚ್ಚಿಂತನೆ
ಸಚ್ಚಿ-ದಾ-ನಂದ
ಸಚ್ಚಿ-ದಾ-ನಂದ-ದಿಂದಲೇ
ಸಚ್ಚಿ-ದಾ-ನಂದದ
ಸಚ್ಚಿ-ದಾ-ನಂದ-ವಾ-ಗಿದ್ದೆನು
ಸಚ್ಚಿ-ದಾ-ನಂದವೇ
ಸಜ್ಜನ
ಸಜ್ಜ-ನ-ನನ್ನಾಗಿ
ಸಜ್ಜನರ
ಸಜ್ಜ-ನ-ರನ್ನು
ಸಜ್ಜ-ನ-ರಾ-ಗಿ-ರು-ವುದು
ಸಜ್ಜ-ನ-ರಿಗೂ
ಸಜ್ಜನರೂ
ಸಜ್ಜ-ನಿ-ಕೆ-ಗ-ಳನ್ನು
ಸಜ್ಜ-ನಿ-ಕೆ-ಯನ್ನಾ-ದರೂ
ಸಜ್ಜ-ನಿ-ಕೆ-ಯಿಂದ
ಸಜ್ಜಾ-ಗಿ-ರುತ್ತಾಳೆ
ಸಜ್ಜು-ಗೊ-ಳಿ-ಸಲು
ಸಜ್ಜು-ಗೊ-ಳಿ-ಸುತ್ತಿ-ರು-ವಷ್ಟ-ರಲ್ಲಿ
ಸಡಿಲಾಗಿ
ಸಣ್ಣ
ಸಣ್ಣಪುಟ್ಟ
ಸಣ್ಣ-ಪುಟ್ಟದ್ದೆಂದು
ಸಣ್ಣ-ಪುಟ್ಟದ್ದೆಂಬ
ಸಣ್ಣಸಣ್ಣ
ಸತತ
ಸತ-ತ-ವಾಗಿ
ಸತ-ತ-ವಾದ
ಸತಾಯಿಸಿ
ಸತಿ
ಸತಿಗೂ
ಸತಿಗೆ
ಸತಿ-ಪ-ತಿ-ಗ-ಳಲ್ಲಿ
ಸತ್
ಸತ್ಕ-ರಿ-ಸಲು
ಸತ್ಕ-ರಿ-ಸಿದ
ಸತ್ಕ-ರಿ-ಸುವ
ಸತ್ಕರ್ಮ
ಸತ್ಕರ್ಮಕ್ಕೆ
ಸತ್ಕರ್ಮ-ಗ-ಳಿಗೆ
ಸತ್ಕರ್ಮ-ಗ-ಳು-ಇ-ವು-ಗ-ಳನ್ನು
ಸತ್ಕರ್ಮ-ದಲ್ಲೇ
ಸತ್ಕರ್ಮ-ದಿಂದಲೇ
ಸತ್ಕರ್ಮ-ನಿ-ರ-ತ-ರಿಗೂ
ಸತ್ಕರ್ಮ-ವೆ-ನಿ-ಸುತ್ತದೆ
ಸತ್ಕರ್ಮಾ-ಚ-ರಣೆ
ಸತ್ಕಾರ
ಸತ್ಕಾ-ರ-ವನ್ನೂ
ಸತ್ಕಾರ್ಯ-ಗ-ಳಲ್ಲಿ
ಸತ್ಕಾರ್ಯ-ಗ-ಳಿಗೆ
ಸತ್ಕಾರ್ಯದ
ಸತ್ಕಾರ್ಯ-ವನ್ನು
ಸತ್ಚಿತ್
ಸತ್ತ
ಸತ್ತಂತೆ
ಸತ್ತದ್ದಲ್ಲ
ಸತ್ತದ್ದು
ಸತ್ತಮ
ಸತ್ತಮೇಲೆ
ಸತ್ತ-ರಾ-ಯಿತು
ಸತ್ತರೂ
ಸತ್ತರೆ
ಸತ್ತ-ವ-ನಿಗೂ
ಸತ್ತವರ
ಸತ್ತ-ವ-ರನ್ನು
ಸತ್ತಾಗ
ಸತ್ತಿತೆಂದು
ಸತ್ತಿದೆ
ಸತ್ತಿದ್ದಾಳೆ
ಸತ್ತಿ-ಹು-ದೀಗ
ಸತ್ತು
ಸತ್ತುದನ್ನು
ಸತ್ತು-ಬಿದ್ದಂತೆ
ಸತ್ತು-ಹೋ-ಗಿತ್ತು
ಸತ್ತು-ಹೋ-ಗಿದ್ದ
ಸತ್ತು-ಹೋ-ಗಿದ್ದೆ
ಸತ್ತು-ಹೋ-ಗಿವೆ
ಸತ್ತುಹೋದ
ಸತ್ತು-ಹೋ-ದ-ವರು
ಸತ್ತು-ಹೋ-ದ-ವಳೇ
ಸತ್ತುಹೋದೆ
ಸತ್ತು-ಹೋ-ಯಿತು
ಸತ್ತೆ
ಸತ್ತೇ
ಸತ್ತೇಹೋದ
ಸತ್ತ್ವ
ಸತ್ತ್ವದ
ಸತ್ತ್ವವೆಲ್ಲ
ಸತ್ತ್ವ-ಶಕ್ತಿ-ಗ-ಳನ್ನು
ಸತ್ಪ-ರಿ-ಣಾಮ
ಸತ್ಪ-ರಿ-ಣಾ-ಮ-ವನ್ನು
ಸತ್ಪು-ರು-ಷ-ರಿಗೆ
ಸತ್ಪು-ರು-ಷರು
ಸತ್ಪ್ರ-ಕಾ-ಶಾ-ನಂದಜೀ
ಸತ್ಪ್ರ-ಜೆ-ಗ-ಳಾಗಿ
ಸತ್ಪ್ರ-ಭಾ-ವ-ವನ್ನು
ಸತ್ಪ್ರ-ವೃತ್ತಿಗೆ
ಸತ್ಫ-ಲ-ಗ-ಳನ್ನು
ಸತ್ಫ-ಲ-ವನ್ನು
ಸತ್ಯ
ಸತ್ಯಂ
ಸತ್ಯಕ್ಕೆ
ಸತ್ಯ-ಗ-ಳನ್ನು
ಸತ್ಯ-ಗ-ಳನ್ನೂ
ಸತ್ಯ-ಗ-ಳಿವು
ಸತ್ಯ-ಗ-ಳೆಂದು
ಸತ್ಯ-ಘ-ಟ-ನೆ-ಗಳು
ಸತ್ಯಜ್ಞಾ-ನ-ಗಳ
ಸತ್ಯ-ತೆ-ಯನ್ನು
ಸತ್ಯತ್ವ
ಸತ್ಯದ
ಸತ್ಯ-ದರ್ಶನ
ಸತ್ಯದಲ್ಲಿ
ಸತ್ಯ-ದೂ-ರ-ವಾ-ಗಿದ್ದರೂ
ಸತ್ಯ-ಧರ್ಮ-ಗ-ಳನ್ನು
ಸತ್ಯ-ನಿಷ್ಠ-ರಾಗಿ
ಸತ್ಯ-ನಿಷ್ಠ-ರಾದ
ಸತ್ಯ-ನಿಷ್ಠರೂ
ಸತ್ಯನಿಷ್ಠೆ
ಸತ್ಯ-ಪ-ಥ-ದಲ್ಲಿ
ಸತ್ಯಪ್ರಿ-ಯತೆ
ಸತ್ಯಭಾಮೆ
ಸತ್ಯ-ಭಾ-ಮೆಯ
ಸತ್ಯ-ಮಾತ್ರವೇ
ಸತ್ಯಮೇವ
ಸತ್ಯವಂತ
ಸತ್ಯ-ವಂತನು
ಸತ್ಯ-ವಂತರೂ
ಸತ್ಯವತಿ
ಸತ್ಯ-ವ-ತಿಗೆ
ಸತ್ಯವನ್ನು
ಸತ್ಯವನ್ನೂ
ಸತ್ಯ-ವನ್ನೆಂದಿಗೂ
ಸತ್ಯವನ್ನೇ
ಸತ್ಯವಲ್ಲ
ಸತ್ಯವಷ್ಟೆ
ಸತ್ಯ-ವಾ-ಗ-ಬ-ಹು-ದೆಂದು
ಸತ್ಯವಾಗಿ
ಸತ್ಯ-ವಾ-ಗಿದ್ದರೆ
ಸತ್ಯ-ವಾ-ಗಿದ್ದಲ್ಲಿ
ಸತ್ಯ-ವಾ-ಗಿದ್ದವು
ಸತ್ಯ-ವಾ-ಗಿಯೇ
ಸತ್ಯ-ವಾ-ಗಿ-ರು-ವಾಗ
ಸತ್ಯ-ವಾ-ಗಿವೆ
ಸತ್ಯ-ವಾ-ಗು-ವುದು
ಸತ್ಯವಾದ
ಸತ್ಯ-ವಾ-ಯಿತೇ
ಸತ್ಯವು
ಸತ್ಯವೂ
ಸತ್ಯವೆ
ಸತ್ಯ-ವೆಂದರೆ
ಸತ್ಯವೆಂದು
ಸತ್ಯವೆಂದೇ
ಸತ್ಯವೇ
ಸತ್ಯ-ವೇ-ನೆಂದು
ಸತ್ಯ-ವೊಂದನ್ನು
ಸತ್ಯವೊಂದು
ಸತ್ಯವೋ
ಸತ್ಯ-ಶೀ-ಲರು
ಸತ್ಯ-ಸಂಗತಿ
ಸತ್ಯ-ಸಂಧರೂ
ಸತ್ಯ-ಸಾಕ್ಷಾತ್ಕಾ-ರಕ್ಕಾಗಿ
ಸತ್ಯ-ಸಾಕ್ಷಾತ್ಕಾ-ರದ
ಸತ್ಯ-ಸಾ-ಯಿ-ಬಾ-ಬಾರ
ಸತ್ಯಸ್ಥಿ-ತಿ-ಯನ್ನು
ಸತ್ಯಸ್ಯ
ಸತ್ಯಾಂಶ
ಸತ್ಯಾಂಶ-ವೇ-ನೆಂದು
ಸತ್ಯಾಂಶ-ವನ್ನು
ಸತ್ಯಾಗ್ರಹ
ಸತ್ಯಾಗ್ರ-ಹ-ಗಳ
ಸತ್ಯಾನ್ವೇ-ಷಣೆ
ಸತ್ಯಾನ್ವೇ-ಷ-ಣೆ-ಗಾಗಿ
ಸತ್ಯಾನ್ವೇ-ಷ-ಣೆ-ಗಿಂತಲೂ
ಸತ್ಯಾನ್ವೇ-ಷ-ಣೆಗೆ
ಸತ್ಯಾನ್ವೇ-ಷ-ಣೆಯ
ಸತ್ಯಾನ್ವೇ-ಷ-ಣೆ-ಯಿಂದಲೇ
ಸತ್ಯಾನ್ವೇ-ಷಿ-ಗಳ
ಸತ್ಯಾನ್ವೇ-ಷಿ-ಗ-ಳನ್ನು
ಸತ್ಯಾನ್ವೇ-ಷಿ-ಗಳು
ಸತ್ಯಾನ್ವೇ-ಷಿ-ಗ-ಳೆನ್ನದೆ
ಸತ್ಯಾನ್ವೇ-ಷಿಗೂ
ಸತ್ಯಾನ್ವೇ-ಷಿಗೆ
ಸತ್ಯಾನ್ವೇ-ಷಿಯ
ಸತ್ವ
ಸತ್ವ-ಪ-ರೀಕ್ಷೆ
ಸತ್ವ-ಶಕ್ತಿ-ಗ-ಳನ್ನು
ಸತ್ವ-ಶಕ್ತಿ-ಗ-ಳನ್ನೂ
ಸತ್ವ-ಶಕ್ತಿ-ಗ-ಳಿಂದಲೇ
ಸತ್ವ-ಶಕ್ತಿ-ಗ-ಳೇನು
ಸತ್ವಸ್ಫೂರ್ತಿ-ಗ-ಳಿಂದ
ಸತ್ವಾ-ದಿ-ಗು-ಣ-ಗ-ಳಿಗೆ
ಸತ್ಸಂಗ
ಸತ್ಸಂಗ-ವಿಲ್ಲ
ಸತ್ಸಂಸ್ಕಾ-ರ-ಗ-ಳನ್ನು
ಸದ-ವ-ಕಾಶ
ಸದ-ವ-ಕಾ-ಶ-ವನ್ನು
ಸದಸ್ಯ
ಸದಸ್ಯರ
ಸದಸ್ಯ-ರಿ-ಗಾಗಿ
ಸದಸ್ಯರು
ಸದಸ್ಯರೂ
ಸದಸ್ಯ-ರೆಲ್ಲರೂ
ಸದಸ್ಯ-ರೊಬ್ಬರು
ಸದಾ
ಸದಾಚಾರ
ಸದಾಚಾರ
ಸದಾ-ಲೋ-ಚನೆ
ಸದಾ-ಶ-ಯ-ಗ-ಳನ್ನು
ಸದುಕ್ತಿ
ಸದುದ್ದೇಶ
ಸದುದ್ದೇ-ಶ-ಗ-ಳನ್ನು
ಸದುದ್ದೇ-ಶ-ದಿಂದಲೇ
ಸದು-ಪ-ಯೋಗ
ಸದು-ಪ-ಯೋ-ಗ-ಗೊ-ಳಿ-ಸ-ಬೇಕು
ಸದು-ಪ-ಯೋ-ಗ-ವಾ-ಗ-ಬೇಕು
ಸದೃ-ಶ-ವಾದ
ಸದೆ
ಸದೆ-ಬ-ಡಿದು
ಸದ್ಗತಿ
ಸದ್ಗಮಯ
ಸದ್ಗುಣ
ಸದ್ಗು-ಣ-ಗಳ
ಸದ್ಗು-ಣ-ಗ-ಳನ್ನು
ಸದ್ಗು-ಣ-ಗ-ಳನ್ನೂ
ಸದ್ಗು-ಣ-ಗ-ಳಾದ
ಸದ್ಗು-ಣ-ಗಳು
ಸದ್ಗು-ಣ-ಗಳೂ
ಸದ್ಗು-ಣ-ಗ-ಳೊಂದಿಗೆ
ಸದ್ಗುಣದ
ಸದ್ಗು-ಣ-ವಂತ-ನಿ-ವನು
ಸದ್ಗು-ಣ-ವನ್ನು
ಸದ್ಗೃ-ಹಸ್ಥ-ನಾಗಿ
ಸದ್ಗೃಹಸ್ಥೆ
ಸದ್ಗ್ರಂಥ-ಗಳ
ಸದ್ದನ್ನು
ಸದ್ದಾದರೆ
ಸದ್ದಿಲ್ಲದೆ
ಸದ್ದು
ಸದ್ದು-ಗದ್ದ-ಲ-ದಿಂದ
ಸದ್ದು-ಬ-ರು-ವಂತೆ
ಸದ್ಭಾವನೆ
ಸದ್ಭಾ-ವ-ನೆ-ಗ-ಳನ್ನು
ಸದ್ಭಾ-ವ-ನೆ-ಗ-ಳನ್ನೇ
ಸದ್ಭಾ-ವ-ನೆ-ಗ-ಳಿಂದ
ಸದ್ಭಾ-ವ-ನೆ-ಗಳು
ಸದ್ಭಾ-ವ-ನೆ-ಯಿಂದ
ಸದ್ಯ
ಸದ್ಯಕ್ಕೆ
ಸದ್ಯದ
ಸದ್ವಿ-ಚಾ-ರ-ಗಳ
ಸದ್ವಿ-ಚಾ-ರ-ಗ-ಳನ್ನು
ಸದ್ವಿ-ನಿ-ಯೋ-ಗವೇ
ಸನಾತನ
ಸನಾ-ತ-ನ-ಧರ್ಮದ
ಸನಾತನಿ
ಸನಿಹಕ್ಕೆ
ಸನಿ-ಹ-ದಲ್ಲಿದ್ದು
ಸನಿ-ಹ-ದಲ್ಲೇ
ಸನ್
ಸನ್ನಾಹಕ್ಕೆ
ಸನ್ನಿ-ಧಿ-ಯಲ್ಲಿ
ಸನ್ನಿ-ಪಾ-ತ-ದಿಂದ
ಸನ್ನಿವೇಶ
ಸನ್ನಿ-ವೇ-ಶಕ್ಕೆ
ಸನ್ನಿ-ವೇ-ಶ-ಗ-ಳನ್ನು
ಸನ್ನಿ-ವೇ-ಶ-ಗ-ಳಲ್ಲಿ
ಸನ್ನಿ-ವೇ-ಶ-ಗ-ಳಲ್ಲೂ
ಸನ್ನಿ-ವೇ-ಶ-ಗ-ಳಿಂದ
ಸನ್ನಿ-ವೇ-ಶ-ಗ-ಳಿವೆ
ಸನ್ನಿ-ವೇ-ಶ-ಗಳು
ಸನ್ನಿ-ವೇ-ಶ-ಗಳೂ
ಸನ್ನಿ-ವೇ-ಶದ
ಸನ್ನಿ-ವೇ-ಶ-ದಲ್ಲಿ
ಸನ್ನಿ-ವೇ-ಶ-ದಲ್ಲೇ
ಸನ್ನಿ-ವೇ-ಶ-ವನ್ನು
ಸನ್ನಿ-ವೇ-ಶ-ವನ್ನೂ
ಸನ್ನಿ-ವೇ-ಶ-ವಾ-ಗಿತ್ತು
ಸನ್ನಿ-ಹಿ-ತ-ವಾ-ಗಿದೆ
ಸನ್ನೆಕೋಲು
ಸನ್ನೆಗೋಲು
ಸನ್ನೆಯ
ಸನ್ಮತಿ
ಸನ್ಮಾರ್ಗ-ವನ್ನು
ಸಪ್ತ
ಸಪ್ಪರ್
ಸಪ್ಪ-ಳ-ವನ್ನೂ
ಸಫಲ
ಸಫ-ಲ-ವಾ-ಗ-ಬಲ್ಲದು
ಸಫ-ಲ-ವಾ-ಗುತ್ತ-ದೆಯೇ
ಸಫ-ಲ-ವಾ-ಯಿತು
ಸಫಲತೆ
ಸಫ-ಲ-ತೆಯ
ಸಫ-ಲ-ತೆ-ಯನ್ನು
ಸಫ-ಲ-ರಾ-ಗ-ಬಲ್ಲರು
ಸಫ-ಲ-ರಾ-ಗ-ಲಿಲ್ಲ
ಸಫ-ಲ-ರಾ-ದರೂ
ಸಫ-ಲ-ವಾ-ಗ-ಲಿಲ್ಲ
ಸಬಕೋ
ಸಬ-ರ-ಮತಿ
ಸಭಾಂಗ-ಣ-ದಲ್ಲಿ
ಸಭಾ-ಕಂಪ-ದಿಂದ
ಸಭಿಕರು
ಸಭೆ
ಸಭೆಗೆ
ಸಭೆಯ
ಸಭೆಯನ್ನು
ಸಭೆಯಲ್ಲಿ
ಸಭೆ-ಯಲ್ಲಿದ್ದ
ಸಭೆ-ಯಲ್ಲಿದ್ದುದು
ಸಭೆ-ಯೊಂದ-ರಲ್ಲಿ
ಸಭ್ಯ-ಜೀ-ವನ
ಸಭ್ಯತೆ
ಸಭ್ಯ-ತೆ-ಗ-ಳಿಂದ
ಸಭ್ಯತೆಯ
ಸಭ್ಯ-ರಾ-ಗಿದ್ದರು
ಸಭ್ಯರಾದ
ಸಭ್ಯರೂ
ಸಮ
ಸಮಂಜಸ
ಸಮಂಜ-ಸ-ವಾದ
ಸಮಕ್ಕೆ
ಸಮಗ್ರ
ಸಮಗ್ರ-ವಾಗಿ
ಸಮಗ್ರ-ಹಿತ
ಸಮ-ಚಿತ್ತತೆ
ಸಮತೆಯ
ಸಮತೋಲ
ಸಮ-ತೋ-ಲನ
ಸಮ-ತೋ-ಲ-ನ-ವನ್ನು
ಸಮ-ತೋ-ಲ-ನ-ವನ್ನೇ
ಸಮ-ತೋ-ಲ-ವನ್ನಾ-ಗಲೀ
ಸಮ-ತೋ-ಲ-ವನ್ನೂ
ಸಮತ್ವ-ವನ್ನು
ಸಮನಾಗಿ
ಸಮನಾದ
ಸಮ-ನಾ-ದುದು
ಸಮನಾರು
ಸಮನೆ
ಸಮನ್ವಯ
ಸಮನ್ವ-ಯ-ಗೊ-ಳಿಸಿ
ಸಮನ್ವ-ಯ-ಗೊಳ್ಳ-ತೊ-ಡ-ಗಿ-ದಾಗ
ಸಮನ್ವ-ಯದ
ಸಮನ್ವ-ಯ-ದೃಷ್ಟಿ
ಸಮನ್ವ-ಯವೇ
ಸಮಯ
ಸಮಯಕ್ಕೆ
ಸಮಯದ
ಸಮ-ಯ-ದಲ್ಲಿ
ಸಮ-ಯ-ದಲ್ಲೂ
ಸಮ-ಯ-ದಲ್ಲೇ
ಸಮ-ಯ-ವನ್ನು
ಸಮ-ಯ-ವನ್ನು-ದಿ-ನದ
ಸಮ-ಯ-ವಿದೆ
ಸಮ-ಯ-ವಿಲ್ಲ
ಸಮಯವು
ಸಮಯವೂ
ಸಮ-ಯ-ವೆಷ್ಟು
ಸಮ-ರ-ವನ್ನು
ಸಮರಸ
ಸಮ-ರ-ಸದ
ಸಮರ್ಥ
ಸಮರ್ಥ-ರಾ-ಗಿದ್ದಾ-ರೆಯೇ
ಸಮರ್ಥ-ವಾ-ಗುತ್ತದೆ
ಸಮರ್ಥತೆ
ಸಮರ್ಥ-ನಲ್ಲವೇ
ಸಮರ್ಥ-ನಾ-ಗ-ಬಲ್ಲ
ಸಮರ್ಥ-ನಾ-ಗ-ಬ-ಹುದು
ಸಮರ್ಥ-ನಾ-ಗ-ಲಿಲ್ಲ
ಸಮರ್ಥ-ನಾ-ಗಿದ್ದ
ಸಮರ್ಥ-ನಾ-ಗುತ್ತಾನೆ
ಸಮರ್ಥ-ನಾದ
ಸಮರ್ಥ-ನಾ-ದರೆ
ಸಮರ್ಥ-ನಾ-ದಾನು
ಸಮರ್ಥ-ನೆ-ಯ-ವಾ-ದವೂ
ಸಮರ್ಥ-ರಾ-ಗಿದ್ದರು
ಸಮರ್ಥ-ರಾ-ಗಿದ್ದರೆ
ಸಮರ್ಥ-ರಾ-ಗಿದ್ದಾರೆ
ಸಮರ್ಥ-ರಾ-ಗಿದ್ದಾ-ರೆಯೇ
ಸಮರ್ಥ-ರಾ-ಗುತ್ತಿ-ರ-ಲಿಲ್ಲ
ಸಮರ್ಥ-ರಾ-ಗುತ್ತಿಲ್ಲ
ಸಮರ್ಥ-ರಾ-ಗು-ವುದೂ
ಸಮರ್ಥ-ರಾ-ದರು
ಸಮರ್ಥ-ರಾ-ದುದು
ಸಮರ್ಥ-ರಾ-ಮ-ದಾ-ಸ-ರಿಗೆ
ಸಮರ್ಥ-ಳಾ-ಗಿದ್ದಳು
ಸಮರ್ಥ-ಳಾ-ದು-ದ-ರಿಂದಲೇ
ಸಮರ್ಥ-ವಾಗಿ
ಸಮರ್ಥ-ವಾ-ಗಿಲ್ಲ
ಸಮರ್ಥಿ-ಸಲ್ಪಟ್ಟಾಗ
ಸಮರ್ಥಿ-ಸಿ-ಕೊಂಡು
ಸಮರ್ಥಿ-ಸಿ-ಕೊಳ್ಳುವ
ಸಮರ್ಥಿ-ಸಿದ
ಸಮರ್ಥಿ-ಸುತ್ತಾರೆ
ಸಮರ್ಪಕ
ಸಮರ್ಪ-ಕ-ವಾಗಿ
ಸಮರ್ಪ-ಕ-ವಾ-ಗಿಯೂ
ಸಮರ್ಪ-ಕ-ವಾ-ಯಿ-ತೆಂದರೆ
ಸಮರ್ಪಣ
ಸಮರ್ಪ-ಣ-ಭಾವ
ಸಮರ್ಪ-ಣ-ಭಾ-ವ-ವನ್ನೂ
ಸಮರ್ಪ-ಣಾ-ಭಾ-ವದ
ಸಮರ್ಪಣೆ
ಸಮರ್ಪ-ಣೆ-ಗಳು
ಸಮರ್ಪ-ಣೆಯ
ಸಮರ್ಪ-ಣೆ-ಯಲ್ಲಿ
ಸಮರ್ಪಿ-ಸಲು
ಸಮರ್ಪಿ-ಸಿ-ಕೊಂಡು
ಸಮ-ವಾ-ಗು-ವಷ್ಟು
ಸಮವಾದ
ಸಮ-ವಾ-ಯವೇ
ಸಮಷ್ಟಿ
ಸಮಷ್ಟಿ-ಯಾಗಿ
ಸಮ-ಸ-ಮ-ಯ-ದಲ್ಲೇ
ಸಮಸ್ತ
ಸಮಸ್ತ-ವಿಶ್ವದ
ಸಮಸ್ಯೆ
ಸಮಸ್ಯೆ-ಗ-ಳಿಗೆ
ಸಮಸ್ಯೆ-ಗಳೇ
ಸಮಸ್ಯೆ-ಗಳ
ಸಮಸ್ಯೆ-ಗ-ಳನ್ನು
ಸಮಸ್ಯೆ-ಗ-ಳನ್ನೋ
ಸಮಸ್ಯೆ-ಗ-ಳಲ್ಲಿ
ಸಮಸ್ಯೆ-ಗ-ಳಲ್ಲೂ
ಸಮಸ್ಯೆ-ಗ-ಳಲ್ಲೊಂದು
ಸಮಸ್ಯೆ-ಗ-ಳಿಂದ
ಸಮಸ್ಯೆ-ಗ-ಳಿಗೂ
ಸಮಸ್ಯೆ-ಗ-ಳಿಗೆ
ಸಮಸ್ಯೆ-ಗಳು
ಸಮಸ್ಯೆ-ಗಳೂ
ಸಮಸ್ಯೆ-ಗಳೆ
ಸಮಸ್ಯೆ-ಗ-ಳೆಲ್ಲ
ಸಮಸ್ಯೆಗೆ
ಸಮಸ್ಯೆ-ಗೊಂದು
ಸಮಸ್ಯೆ-ಮಾ-ನ-ವೀಯ
ಸಮಸ್ಯೆಯ
ಸಮಸ್ಯೆ-ಯನ್ನು
ಸಮಸ್ಯೆ-ಯನ್ನೂ
ಸಮಸ್ಯೆ-ಯನ್ನೇ
ಸಮಸ್ಯೆ-ಯಲ್ಲ
ಸಮಸ್ಯೆ-ಯಾ-ಗಿತ್ತು
ಸಮಸ್ಯೆ-ಯಾ-ಗಿ-ರ-ಬ-ಹುದು
ಸಮಸ್ಯೆ-ಯಾ-ಗು-ವ-ವರು
ಸಮಸ್ಯೆ-ಯಿಂದ
ಸಮಸ್ಯೆಯು
ಸಮಸ್ಯೆಯೂ
ಸಮಸ್ಯೆಯೇ
ಸಮಸ್ಯೆ-ಯೇನೋ
ಸಮಸ್ಯೆ-ಯೊಂದನ್ನು
ಸಮಾಚಾರ
ಸಮಾ-ಚಾ-ರ-ವನ್ನು
ಸಮಾಜ
ಸಮಾ-ಜ-ವಾ-ಗಲಿ
ಸಮಾಜಈ
ಸಮಾ-ಜ-ಕಂಟ-ಕ-ರಾ-ಗುತ್ತಾರೆ
ಸಮಾ-ಜಕ್ಕಾಗಿ
ಸಮಾಜಕ್ಕೂ
ಸಮಾಜಕ್ಕೆ
ಸಮಾ-ಜ-ಘಾ-ತುಕ
ಸಮಾ-ಜ-ಜೀ-ವಿ-ಯಾದ
ಸಮಾಜದ
ಸಮಾ-ಜ-ದಲ್ಲಿ
ಸಮಾ-ಜ-ದಲ್ಲೇ
ಸಮಾ-ಜ-ಬಾಂಧ-ವರು
ಸಮಾ-ಜ-ವನ್ನು
ಸಮಾ-ಜ-ವನ್ನೂ
ಸಮಾ-ಜ-ವಾದ
ಸಮಾಜವು
ಸಮಾಜವೂ
ಸಮಾ-ಜ-ಶಾಸ್ತ್ರ
ಸಮಾ-ಜ-ಶಾಸ್ತ್ರ-ಗಳ
ಸಮಾ-ಜ-ಶಾಸ್ತ್ರಜ್ಞ
ಸಮಾ-ಜ-ಶಾಸ್ತ್ರಜ್ಞ-ರಾದ
ಸಮಾ-ಜ-ಶಾಸ್ತ್ರಜ್ಞರು
ಸಮಾ-ಜ-ಶಾಸ್ತ್ರದ
ಸಮಾ-ಜ-ಸೇ-ವ-ಕರ
ಸಮಾ-ಜ-ಹಿ-ತ-ಚಿಂತ-ನೆಯ
ಸಮಾಧಾನ
ಸಮಾ-ಧಾ-ನ-ಪ-ಡಿ-ಸಿ-ದಳು
ಸಮಾ-ಧಾ-ನ-ಗಳ
ಸಮಾ-ಧಾ-ನ-ಗಳು
ಸಮಾ-ಧಾ-ನ-ಗೊ-ಳಿ-ಸಲು
ಸಮಾ-ಧಾ-ನ-ಗೊ-ಳಿ-ಸಿದ
ಸಮಾ-ಧಾ-ನದ
ಸಮಾ-ಧಾ-ನ-ದಲ್ಲಿ-ರು-ವಾಗ
ಸಮಾ-ಧಾ-ನ-ಪ-ಡಿ-ಸಲು
ಸಮಾ-ಧಾ-ನ-ಪ-ಡಿ-ಸಿದ
ಸಮಾ-ಧಾ-ನ-ಮಾ-ಡಿ-ದರೆ
ಸಮಾ-ಧಾ-ನ-ವನ್ನೂ
ಸಮಾ-ಧಾ-ನ-ವಾಗಿ
ಸಮಾ-ಧಾ-ನ-ವಿದೆ
ಸಮಾಧಿಸ್ಥ
ಸಮಾ-ಧಿಸ್ಥ-ನಾ-ಗುತ್ತೇನೆ
ಸಮಾನ
ಸಮಾನತೆ
ಸಮಾ-ನ-ತೆ-ಯನ್ನು
ಸಮಾನರು
ಸಮಾ-ನ-ಳೆಂದು
ಸಮಾ-ನ-ವಲ್ಲ
ಸಮಾ-ನ-ವಾಗಿ
ಸಮಾ-ನ-ವಾದ
ಸಮಾ-ನ-ಶೀಲ
ಸಮಾ-ನಾರ್ಥಕ
ಸಮಾರಂಭ
ಸಮಾ-ರಾ-ಧ-ನೆ-ಯನ್ನೇ
ಸಮಾ-ಲೋ-ಚಿ-ಸುವ
ಸಮೀ-ಕ-ರ-ಣ-ವನ್ನು
ಸಮೀ-ಕ-ರಿಸಿ
ಸಮೀಕ್ಷೆ
ಸಮೀಕ್ಷೆಯ
ಸಮೀಪ
ಸಮೀಪಕ್ಕೆ
ಸಮೀಪದ
ಸಮೀ-ಪ-ದಲ್ಲಿ
ಸಮೀ-ಪ-ದಲ್ಲಿದ್ದ
ಸಮೀ-ಪ-ದಲ್ಲಿದ್ದ-ವ-ರೆಲ್ಲ
ಸಮೀ-ಪ-ದಲ್ಲಿದ್ದೀಯೆ
ಸಮೀ-ಪ-ದಲ್ಲಿದ್ದು
ಸಮೀ-ಪ-ದಲ್ಲಿ-ರುವ
ಸಮೀ-ಪ-ದಲ್ಲೂ
ಸಮೀ-ಪ-ದಲ್ಲೇ
ಸಮೀ-ಪ-ದ-ವನು
ಸಮೀಪನು
ಸಮೀ-ಪ-ವಾ-ಗಿದೆ
ಸಮೀ-ಪ-ವಿ-ರಿ-ಸ-ಬೇಕು
ಸಮೀ-ಪ-ವಿ-ರುವ
ಸಮೀಪವೇ
ಸಮೀಪಿಸ
ಸಮೀ-ಪಿ-ಸ-ಬ-ಹುದು
ಸಮೀಪಿಸಿ
ಸಮೀ-ಪಿ-ಸಿತು
ಸಮೀ-ಪಿ-ಸಿದ
ಸಮೀ-ಪಿ-ಸಿ-ದಂತೆ
ಸಮೀ-ಪಿ-ಸಿ-ದಂತೆಲ್ಲ
ಸಮೀ-ಪಿ-ಸಿ-ದರೂ
ಸಮೀ-ಪಿ-ಸಿ-ದರೆ
ಸಮೀ-ಪಿ-ಸಿ-ದ-ವ-ರೊ-ಡನೆ
ಸಮೀ-ಪಿ-ಸಿ-ದವು
ಸಮೀ-ಪಿ-ಸಿ-ದಾಗ
ಸಮೀಪಿಸು
ಸಮೀ-ಪಿ-ಸುತ್ತದೆ
ಸಮೀ-ಪಿ-ಸುತ್ತ-ದೆನ್ನಲು
ಸಮೀ-ಪಿ-ಸುತ್ತಿ-ದೆ-ಯೆಂದೂ
ಸಮೀ-ಪಿ-ಸುತ್ತಿದ್ದಾರೆ
ಸಮೀ-ಪಿ-ಸುತ್ತಿದ್ದೇವೆ
ಸಮೀ-ಪಿ-ಸು-ವಂತೆ
ಸಮೀ-ಸಿ-ದಾಗ
ಸಮು-ದಾ-ಯದ
ಸಮುದ್ರ
ಸಮುದ್ರ-ಇವು
ಸಮುದ್ರಕ್ಕಿ-ಳಿ-ಯಲೇ
ಸಮುದ್ರಕ್ಕೆ
ಸಮುದ್ರ-ಗಳೂ
ಸಮುದ್ರದ
ಸಮುದ್ರ-ದಲ್ಲಿ
ಸಮುದ್ರ-ವನ್ನು
ಸಮುದ್ರವೇ
ಸಮುನ್ನತಿ
ಸಮುನ್ನ-ತಿಯ
ಸಮುನ್ನ-ತಿ-ಯನ್ನು
ಸಮೃದ್ಧಿ-ಯತ್ತ
ಸಮೃದ್ಧಿ-ಯನ್ನು
ಸಮ್ಮ-ತ-ವಾದ
ಸಮ್ಮತವೂ
ಸಮ್ಮ-ತಿ-ಯನ್ನು
ಸಮ್ಮ-ತಿ-ಸ-ಲಿಲ್ಲ
ಸಮ್ಮ-ತಿ-ಸುತ್ತಾರೆ
ಸಮ್ಮೇ-ಳ-ನ-ದಲ್ಲಿ
ಸಮ್ಯೋನ್
ಸರಕಾರ
ಸರ-ಕಾ-ರಕ್ಕೆ
ಸರ-ಕಾ-ರದ
ಸರ-ಕಾ-ರ-ದ-ವರು
ಸರ-ಕಾ-ರ-ದಿಂದ
ಸರ-ಕಾ-ರ-ವಾ-ಗಲಿ
ಸರ-ಕಾ-ರವು
ಸರಕಾರಿ
ಸರಕಾರೀ
ಸರಕಿನ
ಸರಕೆಂದು
ಸರ-ಕೆಲ್ಲವೂ
ಸರ-ದಾ-ರ-ನಾ-ಗು-ವ-ವನು
ಸರದಿ
ಸರ-ಬ-ರಾ-ಜನ್ನು
ಸರ-ಬ-ರಾ-ಜಾ-ಗ-ಬಲ್ಲದೋ
ಸರ-ಮಾ-ಲೆ-ಯನ್ನು
ಸರಳ
ಸರ-ಳ-ತೆ-ಯಿಂದೊ-ಡ-ಗೂಡಿ
ಸರಳನ್ನು
ಸರಳರೂ
ಸರ-ಳ-ವಾಗಿ
ಸರ-ಳ-ವಾ-ಗಿದ್ದರೂ
ಸರ-ಳ-ವಾ-ಗಿವೆ
ಸರ-ಳ-ವಾದ
ಸರ-ಳಾಂತಃಕ-ರ-ಣದ
ಸರ-ಳಾಂತಃಕ-ರ-ಣ-ದಿಂದ
ಸರಸತೆ
ಸರ-ಸ-ತೆ-ಇವು
ಸರ-ಸ-ತೆ-ಗಳು
ಸರ-ಸ-ಬೇಡ
ಸರ-ಸ-ರನೆ
ಸರಸ್ವ-ತಿಯ
ಸರಾ-ಗ-ವಾಗಿ
ಸರಾಸರಿ
ಸರಿ
ಸರಿದ
ಸರಿದಾರಿ
ಸರಿ-ದಾ-ರಿಗೆ
ಸರಿದು
ಸರಿ-ದೂ-ಗಲಿ
ಸರಿ-ಪ-ಡಿ-ಸದೇ
ಸರಿ-ಪ-ಡಿ-ಸಲು
ಸರಿ-ಪ-ಡಿ-ಸಿ-ಕೊಂಡ
ಸರಿ-ಪ-ಡಿ-ಸಿ-ಕೊಂಡಂತೆ
ಸರಿ-ಪ-ಡಿ-ಸಿ-ಕೊಂಡರೆ
ಸರಿ-ಪ-ಡಿ-ಸಿ-ಕೊಂಡು
ಸರಿ-ಪ-ಡಿ-ಸಿ-ಕೊಳ್ಳಲು
ಸರಿ-ಪ-ಡಿ-ಸಿ-ಕೊಳ್ಳು-ವು-ದಕ್ಕೆ
ಸರಿ-ಪ-ಡಿಸು
ಸರಿ-ಪ-ಡಿ-ಸುತ್ತೇ-ವೆಂದು
ಸರಿ-ಪ-ಡಿ-ಸುವ
ಸರಿ-ಪ-ಡಿ-ಸು-ವು-ದನ್ನು
ಸರಿ-ಪ-ಡಿ-ಸೆಂದು
ಸರಿ-ಯ-ತೊ-ಡ-ಗಿತು
ಸರಿಯಲ್ಲ
ಸರಿ-ಯಲ್ಲ-ವೆಂದು
ಸರಿಯಾಗಿ
ಸರಿ-ಯಾ-ಗಿಟ್ಟು-ಕೊಂಡರೆ
ಸರಿ-ಯಾ-ಗಿಟ್ಟು-ಕೊಳ್ಳ-ಬೇಕು
ಸರಿ-ಯಾ-ಗಿಟ್ಟು-ಕೊಳ್ಳಲು
ಸರಿ-ಯಾ-ಗಿತ್ತು
ಸರಿ-ಯಾ-ಗಿದೆ
ಸರಿ-ಯಾ-ಗಿದ್ದಂತೆ
ಸರಿ-ಯಾ-ಗಿದ್ದರೆ
ಸರಿ-ಯಾ-ಗಿಯೇ
ಸರಿ-ಯಾ-ಗಿ-ರುವ
ಸರಿ-ಯಾ-ಗುತ್ತದೆ
ಸರಿ-ಯಾ-ಗುತ್ತಿದೆ
ಸರಿ-ಯಾ-ಗು-ವಳು
ಸರಿ-ಯಾ-ಗು-ವುದೆ
ಸರಿಯಾದ
ಸರಿ-ಯಾ-ದರೆ
ಸರಿ-ಯಾ-ದಲ್ಲಿ
ಸರಿ-ಯಾ-ದು-ದನ್ನು
ಸರಿ-ಯಾ-ದು-ದಲ್ಲ
ಸರಿ-ಯಾ-ದುದೆ
ಸರಿ-ಯಾ-ಯಿ-ತೆಂದರೆ
ಸರಿ-ಯುತ್ತಾರೆ
ಸರಿ-ಯುತ್ತಿದೆ
ಸರಿ-ಯು-ದಾ-ಹ-ರಣೆ
ಸರಿ-ಯು-ವಂತೆ
ಸರಿಯೆಂದು
ಸರಿಸಿ
ಸರಿ-ಹೋ-ಗಿದೆ
ಸರಿ-ಹೋ-ಗು-ವುದು
ಸರೋವರ
ಸರೋ-ವ-ರದ
ಸರೋ-ವ-ರ-ದಲ್ಲಿ
ಸರ್
ಸರ್ಕಸ್
ಸರ್ಕಸ್ಸನ್ನು
ಸರ್ಕಸ್ಸಿನ
ಸರ್ಕಸ್ಸಿ-ನಲ್ಲಿ
ಸರ್ಕಾರ
ಸರ್ಕಾ-ರ-ದ-ವರು
ಸರ್ಜನ್
ಸರ್ಪವು
ಸರ್ರ್
ಸರ್ವ
ಸರ್ವ-ಜ-ನಗ್ರಾ-ಹಿ-ಯಾ-ಗುವ
ಸರ್ವಭೂತ
ಸರ್ವ-ಶಕ್ತ-ನಿದ್ದಾನೆ
ಸರ್ವ-ಕಾ-ಲ-ಗ-ಳಲ್ಲಿಯೂ
ಸರ್ವಗ್ರಾ-ಹಿ-ಯಾದ
ಸರ್ವಜನ
ಸರ್ವ-ಜ-ನಗ್ರಾ-ಹಿ-ಯಾ-ಗು-ವಂತೆ
ಸರ್ವ-ಜ-ನಗ್ರಾ-ಹಿ-ಯಾ-ದಂತೆ
ಸರ್ವ-ಜ-ನಾ-ದ-ರ-ಣೀ-ಯ-ರಾದ
ಸರ್ವಜ್ಞ
ಸರ್ವಜ್ಞತೆ
ಸರ್ವಜ್ಞ-ನಾದ
ಸರ್ವಜ್ಞನೂ
ಸರ್ವಜ್ಞವೂ
ಸರ್ವಜ್ಞಾ-ನವೂ
ಸರ್ವ-ತೋ-ಮುಖ
ಸರ್ವ-ತೋ-ಮು-ಖ-ವಾ-ದದ್ದು
ಸರ್ವ-ತೋ-ಮು-ಖ-ವಾಗಿ
ಸರ್ವ-ತೋ-ಮು-ಖ-ವಾದ
ಸರ್ವ-ತೋ-ಮು-ಖ-ವಾ-ದುದು
ಸರ್ವತ್ರ
ಸರ್ವತ್ರವೂ
ಸರ್ವತ್ರ-ವೃದ್ಧಿ-ಯಾ-ಗು-ವಂತೆ
ಸರ್ವಥಾ
ಸರ್ವದಾ
ಸರ್ವದುಃಖ
ಸರ್ವಧರ್ಮ
ಸರ್ವ-ಧರ್ಮ-ಸಮ್ಮೇ-ಳ-ನ-ದಲ್ಲಿ
ಸರ್ವನಾಶ
ಸರ್ವ-ನಾ-ಶಕ
ಸರ್ವ-ನಾ-ಶಕ್ಕೆ
ಸರ್ವ-ನಾ-ಶದ
ಸರ್ವ-ನಾ-ಶ-ದೆ-ಡೆಗೇ
ಸರ್ವ-ನಾ-ಶ-ವಾ-ಗುತ್ತದೆ
ಸರ್ವ-ನಾ-ಶ-ವಾ-ಗು-ವುದೆ
ಸರ್ವ-ನಾ-ಶ-ವಾ-ಗು-ವು-ದೆಂದೂ
ಸರ್ವಪ್ರ-ಕಾ-ರ-ಗ-ಳಲ್ಲೂ
ಸರ್ವಪ್ರ-ಕಾ-ರ-ವಾಗಿ
ಸರ್ವಪ್ರ-ಯತ್ನ
ಸರ್ವಪ್ರ-ಯತ್ನ-ದಿಂದ
ಸರ್ವಬಂಧ
ಸರ್ವ-ಬಂಧ-ನ-ಗ-ಳಿಂದ
ಸರ್ವ-ಬಂಧ-ಮುಕ್ತ-ನಾದ
ಸರ್ವ-ಭೂ-ತ-ಗಳ
ಸರ್ವ-ಭೂ-ತ-ಗ-ಳನ್ನೂ
ಸರ್ವ-ಭೂ-ತ-ಗ-ಳಲ್ಲಿಯೂ
ಸರ್ವ-ಭೂ-ತ-ಹಿ-ತೈ-ಷಿ-ಗ-ಳಾದ
ಸರ್ವ-ಭೂ-ತಾಂತ-ರಾತ್ಮಾ
ಸರ್ವ-ಭೂ-ತಾ-ಧಿ-ವಾಸಃ
ಸರ್ವ-ಭೂ-ತೇಷು
ಸರ್ವಮಾನ್ಯ
ಸರ್ವರ
ಸರ್ವರಿಗೂ
ಸರ್ವ-ರೆ-ಡೆಗೆ
ಸರ್ವವನ್ನೂ
ಸರ್ವ-ವಿ-ಧ-ದಲ್ಲಿಯೂ
ಸರ್ವವೂ
ಸರ್ವವೇದ್ಯ
ಸರ್ವವ್ಯಾಪಿ
ಸರ್ವವ್ಯಾ-ಪಿ-ಯಾ-ಗಿ-ರುವ
ಸರ್ವವ್ಯಾ-ಪಿತ್ವ
ಸರ್ವವ್ಯಾ-ಪಿತ್ವ-ವನ್ನು
ಸರ್ವವ್ಯಾ-ಪಿ-ಯಾಗಿ
ಸರ್ವವ್ಯಾ-ಪಿ-ಯಾ-ಗಿ-ಬಿಟ್ಟಿದೆ
ಸರ್ವವ್ಯಾ-ಪಿ-ಯಾ-ಗಿ-ರುವ
ಸರ್ವವ್ಯಾ-ಪಿ-ಯಾದ
ಸರ್ವವ್ಯಾ-ಪಿಯೂ
ಸರ್ವವ್ಯಾ-ಪಿಯೋ
ಸರ್ವವ್ಯಾಪೀ
ಸರ್ವಶಕ್ತ
ಸರ್ವ-ಶಕ್ತತೆ
ಸರ್ವ-ಶಕ್ತ-ತೆ-ಯಲ್ಲಿ
ಸರ್ವ-ಶಕ್ತನ
ಸರ್ವ-ಶಕ್ತ-ನಲ್ಲವೆ
ಸರ್ವ-ಶಕ್ತ-ನಾದ
ಸರ್ವ-ಶಕ್ತನು
ಸರ್ವ-ಶಕ್ತನೂ
ಸರ್ವ-ಶಕ್ತನೇ
ಸರ್ವ-ಶಕ್ತರು
ಸರ್ವ-ಶಕ್ತ-ವಾದ
ಸರ್ವಶಕ್ತಿ
ಸರ್ವ-ಶಕ್ತಿ-ಯುಳ್ಳದ್ದಾ-ಗಿದ್ದರೆ
ಸರ್ವ-ಸ-ಮ-ತೆಯ
ಸರ್ವ-ಸಮ್ಮತ
ಸರ್ವ-ಸಮ್ಮ-ತ-ವಾದ
ಸರ್ವ-ಸಾ-ಧಾ-ರ-ಣ-ರಲ್ಲಿ
ಸರ್ವ-ಸಾ-ಮಾನ್ಯ
ಸರ್ವಸ್ವ
ಸರ್ವಸ್ವ-ವನ್ನು
ಸರ್ವಸ್ವ-ವನ್ನೂ
ಸರ್ವಸ್ವ-ವನ್ನೇ
ಸರ್ವಸ್ವ-ವಲ್ಲ
ಸರ್ವಸ್ವ-ವೆಂದು
ಸರ್ವಸ್ವ-ವೆನ್ನುವ
ಸರ್ವಸ್ವವೇ
ಸರ್ವಾಂಗವೂ
ಸರ್ವಾಂಗ-ಸುಂದ-ರ-ವನ್ನಾಗಿ
ಸರ್ವಾಂಗಾ-ಸನ
ಸರ್ವಾಂಗೀಣ
ಸರ್ವಾ-ಧಿ-ಕಾ-ರಿ-ಯಂತೆ
ಸರ್ವಾ-ಧಿ-ಕಾರಿ
ಸರ್ವೇ
ಸರ್ವೋಚ್ಚ
ಸಲ
ಸಲ-ಕ-ರ-ಣೆ-ಗ-ಳಾ-ಗಲಿ
ಸಲ-ಕ-ರ-ಣೆ-ಗಳ
ಸಲ-ಕ-ರ-ಣೆ-ಗ-ಳನ್ನು
ಸಲ-ಕ-ರ-ಣೆ-ಗ-ಳಿಂದ
ಸಲ-ಕ-ರ-ಣೆ-ಗ-ಳೊಂದಿಗೆ
ಸಲಹಿ
ಸಲಹಿದ
ಸಲ-ಹಿ-ದರು
ಸಲ-ಹಿ-ದವು
ಸಲ-ಹುತ್ತಾನೆ
ಸಲ-ಹು-ವನು
ಸಲ-ಹು-ವ-ವನೂ
ಸಲ-ಹು-ವುದು
ಸಲಹೆ
ಸಲ-ಹೆ-ಗ-ಳನ್ನಿತ್ತು
ಸಲ-ಹೆ-ಗ-ಳನ್ನು
ಸಲ-ಹೆ-ಗ-ಳನ್ನೂ
ಸಲ-ಹೆ-ಗಾಗಿ
ಸಲ-ಹೆ-ಯಂತೆ
ಸಲ-ಹೆ-ಯಂತೆಯೆ
ಸಲ-ಹೆ-ಯನ್ನಾ-ದರೂ
ಸಲ-ಹೆ-ಯನ್ನು
ಸಲಹೊ
ಸಲಾಮು
ಸಲಿಂಗ-ರ-ತಿಯ
ಸಲಿ-ಗೆ-ಗ-ಳಿಂದ
ಸಲಿ-ಲ-ದಿಂದ
ಸಲೀ-ಸಲ್ಲವೇ
ಸಲೀಸಾಗಿ
ಸಲು
ಸಲುಗೆ
ಸಲುಗೆಯ
ಸಲು-ಗೆ-ಯಾಗಿ
ಸಲು-ಗೆ-ಯಿಂದ
ಸಲು-ಸಿದ್ಧನೊ
ಸಲ್ಲ
ಸಲ್ಲ-ತಕ್ಕು-ದಕ್ಕಿಂತಲೂ
ಸಲ್ಲದು
ಸಲ್ಲ-ಬೇ-ಕಾ-ದುದು
ಸಲ್ಲಿ-ಸ-ಬಲ್ಲದು
ಸಲ್ಲಿ-ಸ-ಬೇ-ಕೆಂದು
ಸಲ್ಲಿಸಲಿ
ಸಲ್ಲಿಸಲು
ಸಲ್ಲಿಸಿ
ಸಲ್ಲಿಸಿದ
ಸಲ್ಲಿ-ಸಿ-ದಂತಾ-ಗು-ವುದು
ಸಲ್ಲಿ-ಸಿ-ದರು
ಸಲ್ಲಿ-ಸಿ-ದರೂ
ಸಲ್ಲಿ-ಸಿ-ದರೆ
ಸಲ್ಲಿ-ಸುತ್ತಿದ್ದೇನೆ
ಸಲ್ಲಿ-ಸುತ್ತಿ-ರ-ಬೇ-ಕೆನ್ನು-ವಾ-ಗಲೇ
ಸಲ್ಲಿ-ಸುತ್ತೇನೆ
ಸಲ್ಲಿಸುವ
ಸಲ್ಲಿ-ಸು-ವಿರಾ
ಸಲ್ಲುತ್ತದೆ
ಸಲ್ಲುವುದು
ಸಲ್ಲುವುದೋ
ಸವಕಲು
ಸವರಿ
ಸವ-ರಿ-ಕೊಂಡರು
ಸವ-ರಿ-ದರು
ಸವ-ಲತ್ತು-ಗ-ಳನ್ನು
ಸವಾರ
ಸವಾ-ರ-ನನ್ನು
ಸವಾರಿ
ಸವಾಲನ್ನು
ಸವಾಲಾಗಿ
ಸವಾ-ಲಾ-ಗಿತ್ತು
ಸವಾ-ಲಾ-ಗಿದ್ದ
ಸವಾಲಾದ
ಸವಾಲು
ಸವಿ
ಸವಿದಂತೆ
ಸವಿದಾಗ
ಸವಿ-ನು-ಡಿ-ಯಲ್ಲಿ
ಸವಿ-ನೆ-ನ-ಪಾಗಿ
ಸವಿ-ನೆ-ನ-ಪಿ-ಗಾಗಿ
ಸವಿಯುತ್ತ
ಸವೆದೆ
ಸವೆಯಿಸಿ
ಸಶಕ್ತ-ವನ್ನಾ-ಗಿ-ಸುವ
ಸಸಿ
ಸಸಿಗೆ
ಸಸಿಯ
ಸಸಿಯನ್ನು
ಸಸ್ಪೆನ್ಸ್
ಸಸ್ಯ
ಸಸ್ಯಗಳೇ
ಸಸ್ಯ-ಜೀ-ವ-ನದ
ಸಸ್ಯರಾಶಿ
ಸಸ್ಯವೇ
ಸಹ
ಸಹ-ಕ-ರಿ-ಸ-ಲಾ-ರ-ದಾ-ದಾಗ
ಸಹ-ಕ-ರಿ-ಸಿತು
ಸಹ-ಕ-ರಿ-ಸಿ-ದರೆ
ಸಹ-ಕ-ರಿ-ಸಿದ್ದರ
ಸಹಕಾರ
ಸಹ-ಕಾ-ರಕ್ಕಿಂತ
ಸಹ-ಕಾ-ರ-ಗ-ಳನ್ನು
ಸಹ-ಕಾ-ರ-ಗ-ಳಿಂದ
ಸಹ-ಕಾ-ರ-ಗ-ಳಿಂದೊ-ಡ-ಗೂ-ಡಿದ
ಸಹ-ಕಾ-ರದ
ಸಹ-ಕಾ-ರ-ದಲ್ಲೂ
ಸಹ-ಕಾ-ರ-ದಿಂದ
ಸಹ-ಕಾ-ರ-ವಿಲ್ಲದೆ
ಸಹಕಾರಿ
ಸಹ-ಕಾ-ರಿ-ಯಾ-ಗ-ಬಲ್ಲುದು
ಸಹ-ಕಾ-ರಿ-ಯಾ-ಗುತ್ತದೆ
ಸಹ-ಕಾ-ರಿ-ಯಾ-ಗುತ್ತವೆ
ಸಹ-ಕಾ-ರಿ-ಯಾ-ಗು-ವುದು
ಸಹ-ಕಾ-ರಿ-ಯಾ-ದೀತು
ಸಹ-ಕಾ-ರಿ-ಯಾ-ಯಿ-ತೆಂಬು-ದನ್ನು
ಸಹಜ
ಸಹ-ಜ-ವಾ-ದುದೇ
ಸಹ-ಜ-ಗು-ಣ-ವಾ-ಗ-ದ-ವರು
ಸಹ-ಜ-ವಲ್ಲವೇ
ಸಹ-ಜ-ವಾಗಿ
ಸಹ-ಜ-ವಾ-ಗಿದೆ
ಸಹ-ಜ-ವಾ-ಗಿಯೇ
ಸಹ-ಜ-ವಾದ
ಸಹಜವೇ
ಸಹಜವೋ
ಸಹ-ಜಸ್ವ-ಭಾ-ವ-ವನ್ನಾಗಿ
ಸಹ-ಧರ್ಮಿ-ಣಿ-ಯಾಗಿ
ಸಹ-ನ-ಶೀ-ಲತೆ
ಸಹ-ನ-ಶೀ-ಲ-ತೆಗೆ
ಸಹ-ನ-ಶೀ-ಲ-ತೆಯ
ಸಹ-ನೀ-ಯ-ವಾ-ಗು-ವಂತೆ
ಸಹನೆ
ಸಹ-ನೆ-ಗ-ಳಿಂದ
ಸಹನೆಗೂ
ಸಹನೆಯ
ಸಹ-ನೆ-ಯಂತೆಯೇ
ಸಹ-ನೆ-ಯದ್ದೆ
ಸಹ-ನೆ-ಯನ್ನು
ಸಹ-ನೆ-ಯಿಂದ
ಸಹ-ನೆ-ಸಂಯ-ಮ-ಗಳ
ಸಹಪಾಠಿ
ಸಹ-ಬಾಳ್ವೆ-ಯನ್ನುಂಟು-ಮಾ-ಡಲೂ
ಸಹವಾಸ
ಸಹ-ವಾ-ಸ-ದಲ್ಲಿ
ಸಹ-ವಾ-ಸ-ದಿಂದ
ಸಹ-ವಾ-ಸ-ವನ್ನು
ಸಹ-ವಿದ್ಯಾರ್ಥಿ-ಗ-ಳು-ಇ-ವರು
ಸಹಸ್ರ
ಸಹಸ್ರ-ನಾಮ
ಸಹಸ್ರ-ವರ್ಷ-ಗಳ
ಸಹಸ್ರ-ಸ-ಹಸ್ರ
ಸಹಸ್ರಾರು
ಸಹಾ-ನು-ಭೂ-ತಿಯೂ
ಸಹಾ-ನು-ಭೂತಿ
ಸಹಾ-ನು-ಭೂ-ತಿ-ಗ-ಳನ್ನು
ಸಹಾ-ನು-ಭೂ-ತಿ-ಯನ್ನು
ಸಹಾ-ನು-ಭೂ-ತಿ-ಯಿಂದ
ಸಹಾಯ
ಸಹಾಯಕ
ಸಹಾ-ಯ-ಕ-ನಾಗಿ
ಸಹಾ-ಯ-ಕರ
ಸಹಾ-ಯ-ಕರು
ಸಹಾ-ಯ-ಕರೂ
ಸಹಾ-ಯ-ಕ-ರೊಂದಿಗೆ
ಸಹಾ-ಯ-ಕ-ವಾ-ಗ-ಬ-ಹುದು
ಸಹಾ-ಯ-ಕ-ವಾ-ಗುವ
ಸಹಾ-ಯ-ಕ-ವಾ-ಗು-ವಂತೆಯೂ
ಸಹಾ-ಯ-ಕ-ವಾ-ಗು-ವುದು
ಸಹಾ-ಯ-ಕ-ವಾದ
ಸಹಾ-ಯಕ್ಕಾಗಿ
ಸಹಾ-ಯಕ್ಕಿಲ್ಲ
ಸಹಾಯಕ್ಕೆ
ಸಹಾ-ಯ-ಗೈ-ಯುವ
ಸಹಾ-ಯ-ಗೈ-ಯು-ವುದು
ಸಹಾಯದ
ಸಹಾ-ಯ-ದಿಂದ
ಸಹಾ-ಯ-ಮಾ-ಡ-ಬೇ-ಕೆಂದು
ಸಹಾ-ಯ-ಮಾಡಿ
ಸಹಾ-ಯ-ಮಾ-ಡಿ-ಕೊಳ್ಳ-ಬಲ್ಲರೋ
ಸಹಾ-ಯ-ಮಾ-ಡುತ್ತದೆ
ಸಹಾ-ಯ-ಮಾ-ಡುವ
ಸಹಾ-ಯ-ವನ್ನಿತ್ತಿವೆ
ಸಹಾ-ಯ-ವನ್ನು
ಸಹಾ-ಯ-ವನ್ನೂ
ಸಹಾ-ಯ-ವಿಲ್ಲದೆ
ಸಹಾ-ಯ-ವಿಲ್ಲದೇ
ಸಹಾಯವು
ಸಹಿತ
ಸಹಿಷ್ಣುತೆ
ಸಹಿಸ
ಸಹಿ-ಸ-ಲ-ಸಾಧ್ಯ-ವಾದ
ಸಹಿ-ಸ-ಲಾ-ಗ-ಲಿಲ್ಲ
ಸಹಿ-ಸ-ಲಾ-ಗುತ್ತಿಲ್ಲ
ಸಹಿ-ಸ-ಲಾ-ರದೇ
ಸಹಿ-ಸ-ಲಾ-ರರು
ಸಹಿಸಿ
ಸಹಿ-ಸಿ-ಕೊಳ್ಳಲು
ಸಹಿ-ಸಿ-ಕೊಳ್ಳುವ
ಸಹಿ-ಸಿ-ಕೊಳ್ಳು-ವನು
ಸಹಿಸಿಕೋ
ಸಹಿಸಿದೆ
ಸಹಿಸುವ
ಸಹಿ-ಸು-ವಾಗ
ಸಹೃದಯ
ಸಹೃ-ದ-ಯ-ತೆ-ಗಳು
ಸಹೃ-ದ-ಯ-ತೆ-ಯನ್ನು
ಸಹೋದರ
ಸಹೋ-ದ-ರ-ರನ್ನು
ಸಹೋ-ದ-ರ-ರನ್ನೂ
ಸಹೋ-ದ-ರತೆ
ಸಹೋ-ದ-ರನ
ಸಹೋ-ದ-ರ-ನಿಗೆ
ಸಹೋ-ದ-ರನು
ಸಹೋ-ದ-ರನೇ
ಸಹೋ-ದ-ರ-ನೊಂದಿಗೆ
ಸಹೋ-ದ-ರ-ನೊ-ಡನೆ
ಸಹೋ-ದ-ರ-ಭಾವ
ಸಹೋ-ದ-ರರ
ಸಹೋ-ದ-ರ-ರನ್ನು
ಸಹೋ-ದ-ರ-ರೊ-ಡನೆ
ಸಹೋ-ದ-ರ-ರೊಬ್ಬರು
ಸಹೋ-ದ-ರ-ರೊ-ಳಗೆ
ಸಹೋ-ದ-ರಿಯ
ಸಹೋದ್ಯೋಗಿ
ಸಹೋದ್ಯೋ-ಗಿ-ಗಳ
ಸಹೋದ್ಯೋ-ಗಿ-ಗ-ಳಾ-ಗಲೀ
ಸಹೋದ್ಯೋ-ಗಿ-ಗ-ಳೆ-ದು-ರಿಗೆ
ಸಹೋದ್ಯೋ-ಗಿ-ಗ-ಳೊ-ಡನೆ
ಸಹ್ಯವಲ್ಲ
ಸಹ್ಯ-ವಲ್ಲದ
ಸಾಂಕೇತಿಕ
ಸಾಂಕ್ರಾಮಿಕ
ಸಾಂಕ್ರಾ-ಮಿ-ಕ-ರೋ-ಗವೂ
ಸಾಂಖ್ಯರು
ಸಾಂಗತ್ಯ
ಸಾಂಘಿಕ
ಸಾಂತ-ಗೊ-ಳಿ-ಸುತ್ತ-ದೆ-ಯಲ್ಲವೇ
ಸಾಂತತೆ
ಸಾಂತನೆಂದು
ಸಾಂತ್ವನ
ಸಾಂತ್ವ-ನ-ಗೊ-ಳಿ-ಸುತ್ತದೆ
ಸಾಂತ್ವನದ
ಸಾಂತ್ವ-ನ-ದಾ-ಯಿ-ಯಾದ
ಸಾಂತ್ವನವೆ
ಸಾಂದ್ರತೆಯ
ಸಾಂದ್ರ-ತೆ-ಯನ್ನು
ಸಾಂಬಾರಿಗೆ
ಸಾಂಸಾರಿಕ
ಸಾಂಸ್ಕೃತಿಕ
ಸಾಂಸ್ಕೃ-ತಿ-ಕ-ವಾಗಿ
ಸಾಂಸ್ಕೃ-ತಿ-ಕ-ವಾ-ಗಿಯೂ
ಸಾಕಷ್ಟಿದ್ದರೂ
ಸಾಕಷ್ಟು
ಸಾಕಷ್ಟು-ದೊ-ರ-ತಿವೆ
ಸಾಕಾ-ಗ-ಲಿಲ್ಲ
ಸಾಕಾಗವು
ಸಾಕಾಗಿದೆ
ಸಾಕಾ-ಗುತ್ತದೆ
ಸಾಕಾ-ಗುತ್ತ-ದೆ-ಇದು
ಸಾಕಾ-ಗುತ್ತಿಲ್ಲ
ಸಾಕಾ-ಗು-ವಷ್ಟು
ಸಾಕಾಯಿತು
ಸಾಕಾರ
ಸಾಕಾ-ರ-ನೆಂದು
ಸಾಕಾ-ರ-ಮೂರ್ತಿ
ಸಾಕಾ-ರಿ-ಯೆಂದೂ
ಸಾಕಿ
ಸಾಕಿದ
ಸಾಕಿ-ಸ-ಲ-ಹಿದ
ಸಾಕು
ಸಾಕು-ತಂದೆ-ಯಾದ
ಸಾಕುನಾಯಿ
ಸಾಕು-ನಾ-ಯಿ-ಯಲ್ಲ-ದಿದ್ದರೂ
ಸಾಕು-ಬೇ-ಕಾ-ಯಿತು
ಸಾಕುಮಗ
ಸಾಕು-ಸಂತೋ-ಷ-ವಾ-ಗಿ-ರ-ಬ-ಹುದು
ಸಾಕೇ
ಸಾಕ್ಷ-ರ-ರನ್ನಾಗಿ
ಸಾಕ್ಷಾ
ಸಾಕ್ಷಾತ್
ಸಾಕ್ಷಾತ್ಕ-ರಿ-ಸ-ಬ-ಹುದೆ
ಸಾಕ್ಷಾತ್ಕ-ರಿ-ಸಿ-ಕೊಂಡ-ವರ
ಸಾಕ್ಷಾತ್ಕ-ರಿ-ಸಿ-ಕೊಂಡ-ವ-ರಾ-ದು-ದ-ರಿಂದ
ಸಾಕ್ಷಾತ್ಕ-ರಿ-ಸಿ-ಕೊಂಡ-ವ-ರೆಂದು
ಸಾಕ್ಷಾತ್ಕ-ರಿ-ಸಿ-ಕೊಳ್ಳ-ಬ-ಹುದು
ಸಾಕ್ಷಾತ್ಕ-ರಿ-ಸಿ-ದಾಗ
ಸಾಕ್ಷಾತ್ಕಾರ
ಸಾಕ್ಷಾತ್ಕಾ-ರಕ್ಕಾಗಿ
ಸಾಕ್ಷಾತ್ಕಾ-ರಕ್ಕಾ-ಗಿದೆ
ಸಾಕ್ಷಾತ್ಕಾ-ರಕ್ಕೆ
ಸಾಕ್ಷಾತ್ಕಾ-ರ-ದಿಂದ
ಸಾಕ್ಷಾತ್ಕಾ-ರ-ವನ್ನು
ಸಾಕ್ಷಾತ್ಕಾ-ರ-ವಾ-ಗ-ದಿದ್ದರೆ
ಸಾಕ್ಷಾತ್ಕಾ-ರ-ವಾ-ಗುತ್ತಲೇ
ಸಾಕ್ಷಾತ್ಕಾ-ರ-ವೆಂಬ
ಸಾಕ್ಷಿ
ಸಾಕ್ಷಿ-ಗ-ಳಾ-ಗಿವೆ
ಸಾಕ್ಷಿ-ಗ-ಳಿಂದ
ಸಾಕ್ಷಿಗಳು
ಸಾಕ್ಷಿಚಿತ್ರ
ಸಾಕ್ಷಿಯ
ಸಾಕ್ಷಿಯಾಗಿ
ಸಾಕ್ಷಿ-ಯಾ-ಗಿವೆ
ಸಾಕ್ಷಿ-ಯಾ-ದರು
ಸಾಕ್ಷಿಯೋ
ಸಾಕ್ಷೀ
ಸಾಕ್ಷ್ಯ
ಸಾಕ್ಷ್ಯ-ಗ-ಳನ್ನು
ಸಾಕ್ಷ್ಯ-ಗ-ಳಿದ್ದರೂ
ಸಾಕ್ಷ್ಯವನ್ನು
ಸಾಕ್ಷ್ಯಾ
ಸಾಕ್ಷ್ಯಾಧಾರ
ಸಾಕ್ಷ್ಯಾ-ಧಾ-ರ-ಗ-ಳನ್ನು
ಸಾಕ್ಷ್ಯಾ-ಧಾ-ರ-ಗ-ಳಿಂದ
ಸಾಕ್ಷ್ಯಾ-ಧಾ-ರ-ಗಳು
ಸಾಕ್ಷ್ಯಾ-ಧಾ-ರ-ಗ-ಳೊಂದಿಗೆ
ಸಾಗಬೇಕು
ಸಾಗರ
ಸಾಗರದ
ಸಾಗ-ರ-ದಲ್ಲಿ
ಸಾಗ-ರ-ದಾ-ಳ-ದಲ್ಲಿ
ಸಾಗ-ರ-ದಿಂದಲೇ
ಸಾಗರರೂ
ಸಾಗ-ರ-ವನ್ನು
ಸಾಗರವೇ
ಸಾಗಲಿಲ್ಲ
ಸಾಗಿದರು
ಸಾಗಿದಾಗ
ಸಾಗಿದೆ
ಸಾಗಿಸಲು
ಸಾಗಿ-ಸಲ್ಪ-ಡುವ
ಸಾಗಿಸಿ
ಸಾಗಿಸುತ್ತ
ಸಾಗಿ-ಸುತ್ತಿ-ರುವ
ಸಾಗುತ್ತ-ಲಿದೆ
ಸಾಗುತ್ತಿವೆ
ಸಾಗುವ
ಸಾಗುವಂತೆ
ಸಾಗುವುದು
ಸಾತ್ತ್ವಿಕ
ಸಾತ್ವಿಕ
ಸಾತ್ವಿ-ಕಪ್ರ-ವೃತ್ತಿ-ಯನ್ನು
ಸಾದಾ
ಸಾದ್ಯ
ಸಾಧಕ
ಸಾಧ-ಕ-ಗ-ಳನ್ನಾಗಿ
ಸಾಧಕನ
ಸಾಧ-ಕ-ನಲ್ಲಿ
ಸಾಧ-ಕ-ನಿ-ಗದು
ಸಾಧಕನು
ಸಾಧಕರ
ಸಾಧ-ಕ-ರನ್ನು
ಸಾಧಕರು
ಸಾಧ-ಕ-ರೆಲ್ಲರ
ಸಾಧ-ಕ-ವಲ್ಲ
ಸಾಧ-ಕ-ವಾ-ಗ-ಬ-ಹು-ದಾದ
ಸಾಧ-ಕ-ವಾ-ಗುತ್ತದೆ
ಸಾಧನ
ಸಾಧ-ನ-ಗಳ
ಸಾಧ-ನ-ಗಳು
ಸಾಧ-ನ-ವನ್ನು
ಸಾಧ-ನ-ವಾಗಿ
ಸಾಧ-ನ-ವಾ-ಗಿ-ರದೆ
ಸಾಧ-ನ-ಸೌ-ಕರ್ಯ-ಗ-ಳಲ್ಲಿಯೂ
ಸಾಧ-ನಾ-ಹೀ-ನ-ರೆಂದು
ಸಾಧನೆ
ಸಾಧ-ನೆ-ಯಲ್ಲಿಯೂ
ಸಾಧ-ನೆ-ಗ-ಳನ್ನು
ಸಾಧ-ನೆ-ಗ-ಳಲ್ಲಿ
ಸಾಧ-ನೆ-ಗ-ಳಿಂದ
ಸಾಧ-ನೆ-ಗ-ಳಿಂದಲೇ
ಸಾಧ-ನೆ-ಗಳು
ಸಾಧ-ನೆ-ಗಳೂ
ಸಾಧ-ನೆ-ಗಾಗಿ
ಸಾಧನೆಗೆ
ಸಾಧನೆಯ
ಸಾಧ-ನೆ-ಯನ್ನು
ಸಾಧ-ನೆ-ಯಲ್ಲ
ಸಾಧ-ನೆ-ಯಲ್ಲಿ
ಸಾಧ-ನೆ-ಯಲ್ಲಿನ
ಸಾಧ-ನೆ-ಯಲ್ಲೂ
ಸಾಧ-ನೆ-ಯಾಗಿ
ಸಾಧ-ನೆ-ಯಾ-ಗುತ್ತದೆ
ಸಾಧ-ನೆ-ಯಿಂದ
ಸಾಧ-ನೆ-ಯಿಲ್ಲದೆ
ಸಾಧನೆಯೇ
ಸಾಧ-ನೆ-ಯೊಂದನು
ಸಾಧ-ನೆ-ಸಿದ್ಧಿ-ಗಳು
ಸಾಧಾರಣ
ಸಾಧಾ-ರ-ಣ-ವಾ-ದುದು
ಸಾಧಾ-ರ-ವಾಗಿ
ಸಾಧಿ-ಕೆ-ಯ-ರಿಗೆ
ಸಾಧಿ-ತ-ವಾ-ಗು-ವು-ದಿಲ್ಲ-ವೆಂಬು-ದನ್ನು
ಸಾಧಿ-ಸ-ಬಲ್ಲ
ಸಾಧಿ-ಸ-ಬಲ್ಲ-ರೆಂದು
ಸಾಧಿ-ಸ-ಬಲ್ಲಿರಿ
ಸಾಧಿ-ಸ-ಬಲ್ಲೆ
ಸಾಧಿ-ಸ-ಬ-ಹು-ದಾ-ಗಿದ್ದ
ಸಾಧಿ-ಸ-ಬ-ಹು-ದಾದ
ಸಾಧಿ-ಸ-ಬ-ಹುದು
ಸಾಧಿ-ಸ-ಬ-ಹು-ದೆಂಬು-ದಕ್ಕೆ
ಸಾಧಿ-ಸ-ಬೇ-ಕಾದ
ಸಾಧಿ-ಸ-ಬೇಕು
ಸಾಧಿ-ಸ-ಬೇ-ಕೆಂಬ
ಸಾಧಿ-ಸ-ಲಾ-ಗದೆ
ಸಾಧಿ-ಸ-ಲಾ-ರೆವು
ಸಾಧಿಸಲು
ಸಾಧಿ-ಸ-ಹೊ-ರಟ
ಸಾಧಿಸಿ
ಸಾಧಿ-ಸಿ-ಕೊಳ್ಳುತ್ತ
ಸಾಧಿ-ಸಿ-ಕೊಳ್ಳು-ವು-ದ-ರಲ್ಲಿ
ಸಾಧಿಸಿತ್ತು
ಸಾಧಿಸಿದ
ಸಾಧಿ-ಸಿ-ದಂತಾ-ಗ-ಲಿಲ್ಲ
ಸಾಧಿ-ಸಿ-ದಂತಾ-ಯಿತು
ಸಾಧಿ-ಸಿ-ದಂತಾ-ಯಿತೇ
ಸಾಧಿ-ಸಿ-ದರು
ಸಾಧಿ-ಸಿ-ದ-ವ-ರೆಲ್ಲರೂ
ಸಾಧಿ-ಸಿ-ದಾಗ
ಸಾಧಿ-ಸಿ-ದು-ದಕ್ಕಿಂತ
ಸಾಧಿಸಿದೆ
ಸಾಧಿ-ಸಿದ್ದರು
ಸಾಧಿ-ಸಿ-ರ-ಬ-ಹು-ದಾ-ಗಿತ್ತು
ಸಾಧಿ-ಸಿ-ರು-ವಿರಾ
ಸಾಧಿಸಿಲ್ಲ
ಸಾಧಿ-ಸುತ್ತದೆ
ಸಾಧಿಸುತ್ತಾ
ಸಾಧಿ-ಸುತ್ತಾನೆ
ಸಾಧಿಸುವ
ಸಾಧಿ-ಸು-ವತ್ತ
ಸಾಧಿ-ಸು-ವು-ದಕ್ಕಿಂತ
ಸಾಧು
ಸಾಧು-ವಾ-ದದ್ದು
ಸಾಧುಗಳು
ಸಾಧು-ಗ-ಳೊಬ್ಬರು
ಸಾಧು-ಗ-ಳೊಮ್ಮೆ
ಸಾಧು-ಮ-ಹಾತ್ಮ-ರಿಂದ
ಸಾಧು-ವಾ-ದದ್ದು
ಸಾಧು-ವಾ-ದುದು
ಸಾಧುವೇ
ಸಾಧುವೊಬ್ಬ
ಸಾಧು-ವೊಬ್ಬರು
ಸಾಧು-ಸಂನ್ಯಾ-ಸಿಯೂ
ಸಾಧ್ಯ
ಸಾಧ್ಯ-ವಿ-ದೆಯೇ
ಸಾಧ್ಯ-ವಿಲ್ಲ-ವಷ್ಟೆ
ಸಾಧ್ಯಎಂದ
ಸಾಧ್ಯ-ಖಂಡಿತ
ಸಾಧ್ಯತೆ
ಸಾಧ್ಯ-ತೆ-ಇ-ವು-ಗ-ಳನ್ನು
ಸಾಧ್ಯ-ತೆ-ಗ-ಳನ್ನು
ಸಾಧ್ಯ-ತೆ-ಗ-ಳನ್ನೂ
ಸಾಧ್ಯತೆಯ
ಸಾಧ್ಯ-ತೆ-ಯನ್ನೂ
ಸಾಧ್ಯ-ತೆ-ಯಾ-ದರೂ
ಸಾಧ್ಯ-ತೆ-ಯಿದೆ
ಸಾಧ್ಯವಲ್ಲ
ಸಾಧ್ಯ-ವಾ-ಗದ
ಸಾಧ್ಯ-ವಾ-ಗ-ದಾಗ
ಸಾಧ್ಯ-ವಾ-ಗದು
ಸಾಧ್ಯ-ವಾ-ಗದೆ
ಸಾಧ್ಯ-ವಾ-ಗದೇ
ಸಾಧ್ಯ-ವಾ-ಗ-ಬ-ಹುದು
ಸಾಧ್ಯ-ವಾ-ಗ-ಬೇ-ಕಾ-ದರೆ
ಸಾಧ್ಯ-ವಾ-ಗ-ಬೇ-ಕೆಂದರೆ
ಸಾಧ್ಯ-ವಾ-ಗ-ಲಿಲ್ಲ
ಸಾಧ್ಯ-ವಾ-ಗಲು
ಸಾಧ್ಯವಾಗಿ
ಸಾಧ್ಯ-ವಾ-ಗಿಯೇ
ಸಾಧ್ಯ-ವಾ-ಗಿಲ್ಲ
ಸಾಧ್ಯ-ವಾ-ಗಿವೆ
ಸಾಧ್ಯ-ವಾ-ಗುತ್ತದೆ
ಸಾಧ್ಯ-ವಾ-ಗುತ್ತ-ದೆಂಬು-ದನ್ನು
ಸಾಧ್ಯ-ವಾ-ಗುತ್ತಿ-ದೆಯೆ
ಸಾಧ್ಯ-ವಾ-ಗುತ್ತಿ-ದೆಯೇ
ಸಾಧ್ಯ-ವಾ-ಗುತ್ತಿಲ್ಲ
ಸಾಧ್ಯ-ವಾ-ಗುವ
ಸಾಧ್ಯ-ವಾ-ಗು-ವಂತೆ
ಸಾಧ್ಯ-ವಾ-ಗು-ವ-ವ-ರೆಗೂ
ಸಾಧ್ಯ-ವಾ-ಗು-ವು-ದಷ್ಟೆ
ಸಾಧ್ಯ-ವಾ-ಗು-ವುದು
ಸಾಧ್ಯ-ವಾ-ಗು-ವುದೆ
ಸಾಧ್ಯ-ವಾ-ಗು-ವು-ದೆಂದು
ಸಾಧ್ಯ-ವಾ-ಗು-ವುದೇ
ಸಾಧ್ಯವಾದ
ಸಾಧ್ಯ-ವಾ-ದರೂ
ಸಾಧ್ಯ-ವಾ-ದರೆ
ಸಾಧ್ಯ-ವಾ-ದಷ್ಟು
ಸಾಧ್ಯ-ವಾ-ದಾಗ
ಸಾಧ್ಯ-ವಾ-ದೀತೇ
ಸಾಧ್ಯ-ವಾ-ದು-ದಕ್ಕಿಂತ
ಸಾಧ್ಯ-ವಾ-ದು-ದನ್ನೆಲ್ಲ
ಸಾಧ್ಯ-ವಾ-ದು-ದ-ರಿಂದ
ಸಾಧ್ಯ-ವಾ-ದುದು
ಸಾಧ್ಯ-ವಾ-ಯಿತು
ಸಾಧ್ಯ-ವಾ-ಯಿತೆ
ಸಾಧ್ಯವಿದೆ
ಸಾಧ್ಯ-ವಿ-ದೆಯೇ
ಸಾಧ್ಯ-ವಿದ್ದರೂ
ಸಾಧ್ಯ-ವಿದ್ದರೆ
ಸಾಧ್ಯ-ವಿದ್ದ-ವರು
ಸಾಧ್ಯ-ವಿದ್ದಷ್ಟು
ಸಾಧ್ಯ-ವಿದ್ದಷ್ಟೂ
ಸಾಧ್ಯ-ವಿ-ರದ
ಸಾಧ್ಯ-ವಿ-ರ-ದಂಥ
ಸಾಧ್ಯ-ವಿ-ರದು
ಸಾಧ್ಯ-ವಿ-ರಲಿ
ಸಾಧ್ಯ-ವಿ-ರ-ಲಿಲ್ಲ
ಸಾಧ್ಯವಿಲ್ಲ
ಸಾಧ್ಯ-ವಿಲ್ಲ-ಎಂಬು-ದನ್ನು
ಸಾಧ್ಯ-ವಿಲ್ಲಮ್ಮಾ
ಸಾಧ್ಯ-ವಿಲ್ಲ-ವಷ್ಟೆ
ಸಾಧ್ಯ-ವಿಲ್ಲ-ವಾ-ಗಿದೆ
ಸಾಧ್ಯ-ವಿಲ್ಲವೆ
ಸಾಧ್ಯ-ವಿಲ್ಲ-ವೆಂಬ
ಸಾಧ್ಯ-ವಿಲ್ಲವೇ
ಸಾಧ್ಯವೂ
ಸಾಧ್ಯವೆ
ಸಾಧ್ಯ-ವೆಂದಲ್ಲ
ಸಾಧ್ಯವೆಂದು
ಸಾಧ್ಯವೇ
ಸಾಧ್ವೀ
ಸಾನಿಧ್ಯ
ಸಾನೇ
ಸಾನ್ನಿಧ್ಯ
ಸಾನ್ನಿಧ್ಯಕ್ಕಾಗಿ
ಸಾನ್ನಿಧ್ಯಕ್ಕೆ
ಸಾನ್ನಿಧ್ಯದ
ಸಾನ್ನಿಧ್ಯ-ದಲ್ಲಿ
ಸಾನ್ನಿಧ್ಯ-ದಲ್ಲೇ
ಸಾನ್ನಿಧ್ಯ-ದಿಂದ
ಸಾನ್ನಿಧ್ಯ-ಬೋಧೆ
ಸಾನ್ನಿಧ್ಯ-ವನ್ನು
ಸಾನ್ನಿಧ್ಯ-ವಿದೆ
ಸಾನ್ನಿಧ್ಯ-ವಿ-ರ-ಬೇಕು
ಸಾನ್ನಿಧ್ಯವೆ
ಸಾಪೇಕ್ಷ
ಸಾಪೇಕ್ಷ-ತಾ-ವಾದ
ಸಾಪೇಕ್ಷ-ತಾ-ಸಿದ್ಧಾಂತ-ವನ್ನು
ಸಾಪೇಕ್ಷ-ವಾ-ದದ
ಸಾಫಲ್ಯಕ್ಕೆ
ಸಾಫಲ್ಯವು
ಸಾಬೀ-ತಾ-ಗ-ಬ-ಹುದೆ
ಸಾಬೀ-ತಾ-ಗಿದೆ
ಸಾಬೀತಾದ
ಸಾಬೀ-ತಾ-ಯಿತು
ಸಾಬೀ-ತು-ಪ-ಡಿ-ಸಿ-ದರು
ಸಾಬೀ-ತು-ಪ-ಡಿ-ಸುತ್ತಾರೆ
ಸಾಮಂಜಸ್ಯ
ಸಾಮಂಜಸ್ಯ-ವಿ-ರುತ್ತದೆ
ಸಾಮಗ್ರಿ
ಸಾಮಗ್ರಿ-ಗ-ಳೊಂದಿಗೆ
ಸಾಮಗ್ರಿ-ಗಳ
ಸಾಮಗ್ರಿ-ಗ-ಳನ್ನು
ಸಾಮಗ್ರಿ-ಗ-ಳಿವೆ
ಸಾಮಗ್ರಿ-ಗಳು
ಸಾಮಗ್ರಿ-ಗ-ಳು-ಕಾರ್ಯ-ರೂ-ಪಕ್ಕೆ
ಸಾಮಗ್ರಿಯ
ಸಾಮಗ್ರಿ-ಯಂತ್ರ
ಸಾಮರಸ್ಯ
ಸಾಮ-ರಸ್ಯದ
ಸಾಮ-ರಸ್ಯ-ದಿಂದ
ಸಾಮ-ರಸ್ಯ-ವನ್ನುಂಟು
ಸಾಮ-ರಸ್ಯ-ವನ್ನೇರ್ಪ-ಡಿಸಿ
ಸಾಮ-ರಸ್ಯವು
ಸಾಮರ್ಥ್ಯ
ಸಾಮರ್ಥ್ಯಕ್ಕೆ
ಸಾಮರ್ಥ್ಯಕ್ಕೊಂದು
ಸಾಮರ್ಥ್ಯ-ಗಳ
ಸಾಮರ್ಥ್ಯ-ಗ-ಳನ್ನು
ಸಾಮರ್ಥ್ಯ-ಗ-ಳಲ್ಲಿ
ಸಾಮರ್ಥ್ಯ-ಗ-ಳಿಂದ
ಸಾಮರ್ಥ್ಯ-ಗ-ಳಿ-ಗ-ನು-ಗು-ಣ-ವಾಗಿ
ಸಾಮರ್ಥ್ಯ-ಗ-ಳಿಗೆ
ಸಾಮರ್ಥ್ಯ-ಗಳೂ
ಸಾಮರ್ಥ್ಯ-ಗ-ಳೇನು
ಸಾಮರ್ಥ್ಯ-ಗ-ಳೊಂದಿಗೆ
ಸಾಮರ್ಥ್ಯದ
ಸಾಮರ್ಥ್ಯ-ದಲ್ಲಿ
ಸಾಮರ್ಥ್ಯ-ದಲ್ಲೇ
ಸಾಮರ್ಥ್ಯ-ದಿಂದ
ಸಾಮರ್ಥ್ಯ-ವನ್ನೀ-ಯುವ
ಸಾಮರ್ಥ್ಯ-ವನ್ನು
ಸಾಮರ್ಥ್ಯ-ವಿದ್ದರೂ
ಸಾಮರ್ಥ್ಯ-ವಿ-ರ-ಲಿಲ್ಲ
ಸಾಮರ್ಥ್ಯ-ವಿ-ರುತ್ತದೆ
ಸಾಮರ್ಥ್ಯ-ವಿ-ರುವ
ಸಾಮರ್ಥ್ಯ-ವಿಲ್ಲ-ದ-ವರು
ಸಾಮರ್ಥ್ಯ-ವಿಲ್ಲ-ದಿದ್ದರೂ
ಸಾಮವೆಂದು
ಸಾಮಾಜಿಕ
ಸಾಮಾ-ಜಿ-ಕ-ಆರ್ಥಿ-ಕ-ಕೃ-ಷಿ-ಸಂಬಂಧ-ವಾದ
ಸಾಮಾ-ಜಿ-ಕ-ರಲ್ಲಿ
ಸಾಮಾ-ಜಿ-ಕ-ವಾಗಿ
ಸಾಮಾ-ಜಿ-ಕ-ವಾದ
ಸಾಮಾನನ್ನು
ಸಾಮಾನಿನ
ಸಾಮಾನು
ಸಾಮಾ-ನು-ಗ-ಳನ್ನು
ಸಾಮಾನೂ
ಸಾಮಾನ್ಯ
ಸಾಮಾನ್ಯ-ನಲ್ಲೂ
ಸಾಮಾನ್ಯನು
ಸಾಮಾನ್ಯನೂ
ಸಾಮಾನ್ಯನೇ
ಸಾಮಾನ್ಯರ
ಸಾಮಾನ್ಯ-ರಂತೆ
ಸಾಮಾನ್ಯ-ರಲ್ಲಿ
ಸಾಮಾನ್ಯ-ರಲ್ಲೇ
ಸಾಮಾನ್ಯ-ರಷ್ಟೇ
ಸಾಮಾನ್ಯ-ರಾದ
ಸಾಮಾನ್ಯ-ರಿಗೆ
ಸಾಮಾನ್ಯರೂ
ಸಾಮಾನ್ಯ-ರೆಂದು
ಸಾಮಾನ್ಯ-ವಲ್ಲ
ಸಾಮಾನ್ಯ-ವಾಗಿ
ಸಾಮಾನ್ಯ-ವಾದ
ಸಾಮಾನ್ಯ-ವೆಂದು
ಸಾಮಾನ್ಯ-ವೆ-ನಿ-ಸುವ
ಸಾಮಾನ್ಯವ್ಯಕ್ತಿ-ಯಲ್ಲಿ
ಸಾಮೀಪ್ಯವೇ
ಸಾಮೂಹಿಕ
ಸಾಮ್ಯ
ಸಾಮ್ಯವು
ಸಾಮ್ಯವೂ
ಸಾಮ್ಯ-ಹೋ-ಲಿ-ಕೆ-ಗ-ಳನ್ನು
ಸಾಮ್ರಾಜ್ಯ-ದಲ್ಲಿ
ಸಾಮ್ರಾಜ್ಯ-ಶಾಹಿ
ಸಾಮ್ರಾಜ್ಯ-ಶಾ-ಹಿ-ಗಳು
ಸಾಯ-ಬೇ-ಕಾ-ಗಿತ್ತು
ಸಾಯ-ಬೇ-ಕಾದ
ಸಾಯಬೇಕು
ಸಾಯ-ಲಾ-ರಿರಿ
ಸಾಯಲಿಲ್ಲ
ಸಾಯಿಸದೇ
ಸಾಯುತ್ತಾನೆ
ಸಾಯುತ್ತಾ-ನೆಂಬು-ದನ್ನು
ಸಾಯುತ್ತಾರೆ
ಸಾಯುತ್ತಾ-ರೆಂದು
ಸಾಯುತ್ತಿ-ರು-ವು-ದನ್ನು
ಸಾಯುವ
ಸಾಯುವಂತೆ
ಸಾಯು-ವ-ವ-ರೆಗೂ
ಸಾಯುವಾಗ
ಸಾಯುವಿರಿ
ಸಾಯು-ವು-ದಿಲ್ಲ
ಸಾಯುವುದು
ಸಾರ
ಸಾರ-ಗರ್ಭಿ-ತ-ವಾಗಿ
ಸಾರ-ನಾ-ಥದ
ಸಾರ-ರೂ-ಪ-ದಲ್ಲಿ
ಸಾರಲ್ಪ-ಡುವ
ಸಾರವನ್ನೂ
ಸಾರ-ವೃದ್ಧಿ-ಯಾ-ಗ-ಬೇಕು
ಸಾರವೆಲ್ಲ
ಸಾರವೇ
ಸಾರಾಂಶ-ವನ್ನು
ಸಾರಿ
ಸಾರಿಗೆ
ಸಾರಿತು
ಸಾರಿತ್ತು
ಸಾರಿದ
ಸಾರಿದರು
ಸಾರಿದೆ
ಸಾರಿದ್ದರು
ಸಾರಿದ್ದ-ವು-ಉ-ಪ-ನಿ-ಷತ್ತು
ಸಾರಿದ್ದಾನೆ
ಸಾರಿದ್ದಾರೆ
ಸಾರಿಸಿ
ಸಾರುತ್ತ
ಸಾರುತ್ತದೆ
ಸಾರುತ್ತವೆ
ಸಾರುತ್ತ-ವೆಯೋ
ಸಾರುತ್ತಾರೆ
ಸಾರುತ್ತಿದೆ
ಸಾರುತ್ತಿದ್ದಾರೆ
ಸಾರುತ್ತಿವೆ
ಸಾರುತ್ತೇ-ನೆ-ಅದು
ಸಾರುವ
ಸಾರು-ವು-ದಿಲ್ಲ
ಸಾರ್
ಸಾರ್ಥಕ
ಸಾರ್ಥ-ಕ-ಗೊ-ಳಿ-ಸುವ
ಸಾರ್ಥ-ಕ-ಜೀ-ವನ
ಸಾರ್ಥಕತೆ
ಸಾರ್ಥ-ಕ-ತೆಗೆ
ಸಾರ್ಥ-ಕ-ತೆ-ಯನ್ನೀ-ಯ-ಬಲ್ಲುದು
ಸಾರ್ಥ-ಕ-ವಾ-ಗ-ಬೇ-ಕಾ-ದರೆ
ಸಾರ್ಥ-ಕ-ವಾ-ಗಿತ್ತು
ಸಾರ್ಥ-ಕ-ವಾ-ಗಿಯೋ
ಸಾರ್ಥಕ್ಯ
ಸಾರ್ವ
ಸಾರ್ವ-ಕಾ-ಲಿಕ
ಸಾರ್ವ-ಕಾ-ಲೀ-ಕ-ವಾದ
ಸಾರ್ವ-ಜ-ನಿಕ
ಸಾರ್ವ-ಜ-ನಿ-ಕರ
ಸಾರ್ವತ್ರಿಕ
ಸಾರ್ವತ್ರಿ-ಕ-ವಾಗಿ
ಸಾರ್ವತ್ರಿ-ಕವೂ
ಸಾರ್ವಭೌಮ
ಸಾಲ
ಸಾಲಂಕೃತ
ಸಾಲಕ್ಕೆ
ಸಾಲಗಳ
ಸಾಲ-ಗಾ-ರ-ನಾ-ಗಿ-ಬಿ-ಡುವ
ಸಾಲದ
ಸಾಲದಿಂದ
ಸಾಲದು
ಸಾಲದ್ದಕ್ಕೆ
ಸಾಲ-ಪ-ಡೆದ
ಸಾಲವನ್ನು
ಸಾಲ-ವಾ-ದರೂ
ಸಾಲ-ವಿದ್ದರೆ
ಸಾಲಾಗಿ
ಸಾಲಿಗ್ಮ್ಯಾನ್
ಸಾಲಿಗ್ರಾ-ಮದ
ಸಾಲಿನಂತೆ
ಸಾಲಿನಲ್ಲಿ
ಸಾಲು
ಸಾಲುಗಳು
ಸಾವ-ಕಾ-ಶ-ವಾಗಿ
ಸಾವ-ಧಾ-ನ-ದಿಂದಲೇ
ಸಾವನ್ನಪ್ಪಿ
ಸಾವನ್ನಪ್ಪಿ-ದ-ನಂತೆ
ಸಾವನ್ನಪ್ಪಿ-ದರೆ
ಸಾವನ್ನಪ್ಪುತ್ತಾ-ರೆ-ಅ-ದೆಷ್ಟು
ಸಾವನ್ನಪ್ಪುವ
ಸಾವನ್ನು
ಸಾವ-ರಿ-ಸಿ-ಕೊಂಡ
ಸಾವಾಗಿ
ಸಾವಿಗಾಗಿ
ಸಾವಿಗಿಂತ
ಸಾವಿ-ಗಿಂತಲೂ
ಸಾವಿನ
ಸಾವಿನಿಂದ
ಸಾವಿ-ನೊಂದಿಗೆ
ಸಾವಿ-ನೊ-ಡನೆ
ಸಾವಿರ
ಸಾವಿ-ರ-ಕೋಟಿ
ಸಾವಿ-ರಕ್ಕಿಂತ
ಸಾವಿರಕ್ಕೂ
ಸಾವಿರದ
ಸಾವಿ-ರ-ವ-ರು-ಷ-ಗಳ
ಸಾವಿರಾರು
ಸಾವಿಲ್ಲದ
ಸಾವು
ಸಾವುಇವು
ಸಾವುಗಳ
ಸಾವು-ಗ-ಳಿ-ಗಿಂತ
ಸಾವುಗಳು
ಸಾಷ್ಟಾಂಗ
ಸಾಷ್ಟಾಂಗ-ನ-ಮನ
ಸಾಷ್ಟಾಂಗ-ವೆ-ರಗಿ
ಸಾಹಸ
ಸಾಹ-ಸ-ಕತೆ
ಸಾಹಸಕ್ಕೆ
ಸಾಹ-ಸ-ಗಳು
ಸಾಹಸದ
ಸಾಹ-ಸ-ದಲ್ಲಿ
ಸಾಹ-ಸ-ದಿಂದ
ಸಾಹ-ಸಪ್ರ-ವೃತ್ತಿ-ಯನ್ನು
ಸಾಹ-ಸ-ವನ್ನು
ಸಾಹಸಿ
ಸಾಹ-ಸಿ-ಗಳು
ಸಾಹ-ಸಿ-ಗ-ಳು-ಇ-ವ-ರೆಲ್ಲರ
ಸಾಹಸೀ
ಸಾಹಿತಿ
ಸಾಹಿತ್ಯ
ಸಾಹಿತ್ಯ-ಕೃ-ತಿ-ಗಳ
ಸಾಹಿತ್ಯಪ್ರಜ್ಞೆ-ಇವು
ಸಾಹಿತ್ಯವೋ
ಸಿ
ಸಿಂಗ-ಪು-ರ-ದಲ್ಲಿದ್ದರು
ಸಿಂಗ-ಪು-ರ-ದಿಂದ
ಸಿಂಡ-ರಿ-ಸಿ-ಕೊಂಡು
ಸಿಂಧು
ಸಿಂಧುವನ್ನೂ
ಸಿಂಧು-ವಿ-ನಲ್ಲಿ
ಸಿಂಹ
ಸಿಂಹದ
ಸಿಂಹಪಾಲು
ಸಿಂಹ-ವಾ-ಗಲು
ಸಿಂಹ-ಸ-ದೃಶ
ಸಿಂಹಾ-ವ-ಲೋ-ಕನ
ಸಿಂಹಾ-ವ-ಲೋ-ಕ-ನದ
ಸಿಕ್ಕ
ಸಿಕ್ಕಾಗ
ಸಿಕ್ಕಾಪಟ್ಟೆ
ಸಿಕ್ಕಿ
ಸಿಕ್ಕಿಕೊಂಡ
ಸಿಕ್ಕಿ-ಕೊಂಡಂತಾ-ಗಿದೆ
ಸಿಕ್ಕಿ-ಕೊಂಡರೆ
ಸಿಕ್ಕಿ-ಕೊಂಡಿದ್ದರೆ
ಸಿಕ್ಕಿ-ಕೊಂಡಿದ್ದಾ-ನೆಯೇ
ಸಿಕ್ಕಿ-ಕೊಂಡಿದ್ದೇನೆ
ಸಿಕ್ಕಿ-ಕೊಂಡಿದ್ದೇವೆ
ಸಿಕ್ಕಿಕೊಂಡು
ಸಿಕ್ಕಿ-ಕೊಂಡು-ದನ್ನು
ಸಿಕ್ಕಿ-ಕೊಳ್ಳದೆ
ಸಿಕ್ಕಿ-ಕೊಳ್ಳುತ್ತವೆ
ಸಿಕ್ಕಿ-ಕೊಳ್ಳುತ್ತಾನೆ
ಸಿಕ್ಕಿ-ಕೊಳ್ಳುತ್ತಾರೆ
ಸಿಕ್ಕಿತು
ಸಿಕ್ಕಿತ್ತು
ಸಿಕ್ಕಿದ
ಸಿಕ್ಕಿದರೆ
ಸಿಕ್ಕಿ-ದ-ವರು
ಸಿಕ್ಕಿದಾಗ
ಸಿಕ್ಕಿದೆ
ಸಿಕ್ಕಿಬಿದ್ದ
ಸಿಕ್ಕಿ-ಬಿದ್ದಾಗ
ಸಿಕ್ಕಿಯೇ
ಸಿಕ್ಕಿ-ರ-ಬೇಕು
ಸಿಕ್ಕಿ-ರ-ಲಿಲ್ಲ
ಸಿಕ್ಕಿರುವ
ಸಿಕ್ಕಿಲ್ಲ
ಸಿಕ್ಕಿಹುದು
ಸಿಕ್ಕೀತು
ಸಿಕ್ಕೀತೆಂದು
ಸಿಕ್ಕೀತೇ
ಸಿಕ್ಕುತ್ತಿದ್ದ-ವೆಂದು
ಸಿಕ್ಕುವ
ಸಿಕ್ಕುವುವು
ಸಿಕ್ಸರ್
ಸಿಗದಷ್ಟು
ಸಿಗ-ದಾ-ಯಿ-ತೆಂಬ-ವ-ರಿಗೆ
ಸಿಗದು
ಸಿಗ-ಬಲ್ಲ-ದೆಂದರ್ಥ
ಸಿಗ-ಬ-ಹು-ದು-ಎಂದ
ಸಿಗ-ಬ-ಹುದೇ
ಸಿಗ-ಬೇ-ಕೆಂದು
ಸಿಗಬೇಕೇ
ಸಿಗ-ರೇ-ಟಿನ
ಸಿಗರೇಟು
ಸಿಗ-ರೇ-ಟು-ಗ-ಳಿ-ಗಾಗಿ
ಸಿಗ-ರೇ-ಟು-ಗಳ
ಸಿಗಲಿಲ್ಲ
ಸಿಗ-ಲಿಲ್ಲ-ವೆಂದು
ಸಿಗಲು
ಸಿಗುತ್ತದೆ
ಸಿಗುತ್ತ-ದೆಯೇ
ಸಿಗುತ್ತವೆ
ಸಿಗುತ್ತ-ವೆಂದಾಯ್ತು
ಸಿಗುತ್ತಿಲ್ಲ
ಸಿಗುತ್ತೇ-ನೆಂದು
ಸಿಗುವ
ಸಿಗು-ವಂತಾ-ಗ-ಬೇಕು
ಸಿಗುವಂತೆ
ಸಿಗುವಷ್ಟು
ಸಿಗು-ವು-ದ-ರಲ್ಲಿ
ಸಿಗು-ವು-ದಿಲ್ಲ
ಸಿಗುವುದು
ಸಿಗುವುದೂ
ಸಿಗ್ಮಂಡ್
ಸಿಟ್ಟನ್ನು
ಸಿಟ್ಟಾ-ಗ-ಲಿಲ್ಲ
ಸಿಟ್ಟಾಗಿ
ಸಿಟ್ಟಾ-ಗುತ್ತಾರೆ
ಸಿಟ್ಟಿಗೆ
ಸಿಟ್ಟಿ-ಗೆದ್ದಾ-ಗಲೇ
ಸಿಟ್ಟಿಗೆದ್ದು
ಸಿಟ್ಟಿ-ಗೇ-ಳುತ್ತಾರೆ
ಸಿಟ್ಟಿ-ಗೇ-ಳುತ್ತೇವೆ
ಸಿಟ್ಟಿ-ಗೇ-ಳು-ವಂತೆ
ಸಿಟ್ಟಿನ
ಸಿಟ್ಟಿನಿಂದ
ಸಿಟ್ಟು
ಸಿಟ್ಟು-ಗಾ-ರರ
ಸಿಟ್ಟು-ಗಾ-ರ-ರಲ್ಲಿ
ಸಿಡಿದ
ಸಿಡಿದರು
ಸಿಡಿದು
ಸಿಡಿದೆದ್ದು
ಸಿಡಿ-ದೇ-ಳುತ್ತದೆ
ಸಿಡಿ-ಮಿ-ಡಿ-ಗೊಂಡು
ಸಿಡಿ-ಮಿ-ಡಿ-ಗೊಳ್ಳು-ವುದು
ಸಿಡಿ-ಯ-ಲಿಲ್ಲ
ಸಿಡಿ-ಯುತ್ತದೆ
ಸಿಡಿ-ಲ-ನು-ಡಿ-ಯಲ್ಲಿ
ಸಿಡಿಲು
ಸಿಡುಕಿನ
ಸಿಡು-ಕಿ-ನಿಂದ
ಸಿಡುಕು
ಸಿಡುಬು
ಸಿತು
ಸಿದ
ಸಿದೆ
ಸಿದ್ದ
ಸಿದ್ದಳಿಲ್ಲ
ಸಿದ್ಧ
ಸಿದ್ಧ-ಗೊ-ಳಿ-ಸಿ-ದಂತಾ-ಗುತ್ತ-ದೆಂಬು-ದನ್ನು
ಸಿದ್ಧತೆ
ಸಿದ್ಧ-ತೆ-ಗ-ಳಿಲ್ಲದೆ
ಸಿದ್ಧ-ತೆ-ಯನ್ನಾ-ಗಲೀ
ಸಿದ್ಧ-ತೆ-ಯನ್ನು
ಸಿದ್ಧ-ತೆ-ಯನ್ನೂ
ಸಿದ್ಧತೆಯೂ
ಸಿದ್ಧ-ನಾ-ಗುತ್ತಿದ್ದ
ಸಿದ್ಧನಾದ
ಸಿದ್ಧನಾದೆ
ಸಿದ್ಧ-ನಿ-ರ-ಲಿಲ್ಲ
ಸಿದ್ಧನೇ
ಸಿದ್ಧ-ಪ-ಡಿಸಿ
ಸಿದ್ಧ-ಪ-ಡಿ-ಸಿದ
ಸಿದ್ಧ-ಪ-ಡಿ-ಸು-ವಂತೆ
ಸಿದ್ಧ-ಪು-ರು-ಷರ
ಸಿದ್ಧ-ಪು-ರು-ಷ-ರಾದ
ಸಿದ್ಧ-ಪು-ರು-ಷರು
ಸಿದ್ಧಮಾಡಿ
ಸಿದ್ಧ-ಯೋ-ಗಿ-ಗ-ಳಿಗೆ
ಸಿದ್ಧರಾಗಿ
ಸಿದ್ಧರಾದ
ಸಿದ್ಧ-ರಾ-ದರು
ಸಿದ್ಧ-ರಿ-ರ-ಬೇಕು
ಸಿದ್ಧ-ರಿ-ರ-ಲಿಲ್ಲ
ಸಿದ್ಧರಿಲ್ಲ
ಸಿದ್ಧ-ರೆಂದಾ-ದರೆ
ಸಿದ್ಧ-ಳಾ-ಗಿದ್ದೇನೆ
ಸಿದ್ಧ-ವಸ್ತು-ವನ್ನು
ಸಿದ್ಧ-ವಾ-ಗಿದೆ
ಸಿದ್ಧ-ವಾ-ಗಿ-ದೆ-ಎಂದು
ಸಿದ್ಧ-ವಾ-ಗಿ-ದೆ-ಯೆ-ಎಂದು
ಸಿದ್ಧ-ವಾ-ಗಿದ್ದರೆ
ಸಿದ್ಧ-ವಾ-ಗಿ-ರ-ಬ-ಹುದು
ಸಿದ್ಧ-ವಾ-ಗಿ-ರುತ್ತದೆ
ಸಿದ್ಧ-ವಾ-ಗಿ-ರುತ್ತವೆ
ಸಿದ್ಧ-ವಾ-ಗು-ವುದು
ಸಿದ್ಧ-ಹಸ್ತ-ನಾದ
ಸಿದ್ಧ-ಹಸ್ತನೂ
ಸಿದ್ಧಾಂತ
ಸಿದ್ಧಾಂತಕ್ಕೆ
ಸಿದ್ಧಾಂತ-ಗಳ
ಸಿದ್ಧಾಂತ-ಗ-ಳನ್ನು
ಸಿದ್ಧಾಂತ-ಗ-ಳನ್ನೂ
ಸಿದ್ಧಾಂತ-ಗ-ಳನ್ನೇ
ಸಿದ್ಧಾಂತ-ಗ-ಳಲ್ಲಿ
ಸಿದ್ಧಾಂತ-ಗ-ಳಿಂದ
ಸಿದ್ಧಾಂತ-ಗ-ಳಿಗೆ
ಸಿದ್ಧಾಂತ-ಗಳು
ಸಿದ್ಧಾಂತದ
ಸಿದ್ಧಾಂತ-ವನ್ನ-ರಿ-ತರೆ
ಸಿದ್ಧಾಂತ-ವನ್ನು
ಸಿದ್ಧಾಂತ-ವಾ-ಗಿಯೇ
ಸಿದ್ಧಾಂತ-ವಾ-ದಿ-ಗಳ
ಸಿದ್ಧಾಂತವು
ಸಿದ್ಧಾಂತ-ವೆಂದು
ಸಿದ್ಧಿ
ಸಿದ್ಧಿಗಳ
ಸಿದ್ಧಿ-ಗ-ಳನ್ನಾ-ಗಲೀ
ಸಿದ್ಧಿ-ಗ-ಳನ್ನು
ಸಿದ್ಧಿ-ಗ-ಳನ್ನೂ
ಸಿದ್ಧಿ-ಗ-ಳನ್ನೆಲ್ಲ
ಸಿದ್ಧಿ-ಗ-ಳಿಗೆ
ಸಿದ್ಧಿಗಳು
ಸಿದ್ಧಿಗಾಗಿ
ಸಿದ್ಧಿಗೆ
ಸಿದ್ಧಿಯ
ಸಿದ್ಧಿಯನ್ನು
ಸಿದ್ಧಿಯಲ್ಲಿ
ಸಿದ್ಧಿ-ಯಾ-ಗಿದ್ದರೆ
ಸಿದ್ಧಿಯಾದ
ಸಿದ್ಧಿ-ಯಾ-ಯಿತು
ಸಿದ್ಧಿಯು
ಸಿದ್ಧಿಯೇ
ಸಿದ್ಧಿಸಿದೆ
ಸಿದ್ಧಿ-ಸು-ವುದು
ಸಿದ್ಧೌಷಧ
ಸಿನಿಕತೆ
ಸಿನಿ-ಕ-ತೆ-ಯನ್ನು
ಸಿನಿಮಾ
ಸಿನಿಮೀಯ
ಸಿನೆಮಾ
ಸಿನೆ-ಮಾ-ಗ-ಳಲ್ಲಿ
ಸಿನೆ-ಮಾ-ಗ-ಳ-ವ-ರೆಗೂ
ಸಿಪಾ-ಯಿ-ಗ-ಳಿ-ಗೊಪ್ಪಿ-ಸುವ
ಸಿಬ್ಬಂದಿ
ಸಿಬ್ಬಂದಿ-ವರ್ಗ-ದ-ವ-ರನ್ನು
ಸಿಮಶೀನ್
ಸಿಮೆಂಟಿ-ನಿಂದ
ಸಿಯರ್
ಸಿರ-ದಾ-ರನೂ
ಸಿರಿ
ಸಿಲುಕದ
ಸಿಲುಕದು
ಸಿಲುಕಿ
ಸಿಲು-ಕಿ-ಕೊಂಡಿದ್ದರೂ
ಸಿಲು-ಕಿ-ಕೊಳ್ಳುತ್ತಾರೆ
ಸಿಲುಕಿತು
ಸಿಲುಕಿದ
ಸಿಲು-ಕಿ-ದ-ವ-ರನ್ನು
ಸಿಲು-ಕಿ-ದ-ವ-ರಿಗೆ
ಸಿಲು-ಕಿ-ದಾಗ
ಸಿಲುಕಿದೆ
ಸಿಲು-ಕಿ-ರುವ
ಸಿಲು-ಕಿ-ಸಿವೆ
ಸಿಲು-ಕಿ-ಸುತ್ತವೆ
ಸಿಲ್ಕ್ಬಟ್ಟೆ-ಗ-ಳನ್ನು
ಸಿಸಿಲಿ
ಸಿಸ್ಟ-ರಿ-ಗೊಂದು
ಸಿಸ್ಟರ್
ಸಿಹಿ
ಸೀಟಿಗಾಗಿ
ಸೀಟು
ಸೀನುತ್ತಿದ್ದಳು
ಸೀಮಾ-ರ-ಹಿ-ತ-ವಾಗಿ
ಸೀಮಾ-ರ-ಹಿ-ತ-ವಾದ
ಸೀಮಿತ
ಸೀಮಿ-ತ-ಗೊ-ಳಿ-ಸದೆ
ಸೀಮಿ-ತ-ಗೊ-ಳಿಸಿ
ಸೀಮಿ-ತ-ಗೊ-ಳಿ-ಸುವ
ಸೀಮಿ-ತ-ಗೊ-ಳಿ-ಸು-ವು-ದನ್ನು
ಸೀಮಿ-ತ-ವಲ್ಲ
ಸೀಮಿ-ತ-ವಾ-ಗಲಿ
ಸೀಮಿ-ತ-ವಾ-ಗಿದ್ದ
ಸೀಮಿ-ತ-ವಾ-ಗಿ-ಯಾ-ದರೂ
ಸೀಮಿ-ತ-ವಾ-ಗುತ್ತದೆ
ಸೀಮಿ-ತ-ವಾ-ದು-ದಲ್ಲ
ಸೀಮಿ-ತ-ವೆಂದು
ಸೀಮಿತವೇ
ಸೀಮೆ
ಸೀಮೆಯನ್ನು
ಸೀಮೆ-ಸುಣ್ಣ-ದಿಂದ
ಸೀರಿಯಸ್
ಸೀರೆ
ಸೀಳು-ತು-ಟಿಯ
ಸೀಸರ್
ಸುಂದರ
ಸುಂದ-ರ-ಕಾಯ
ಸುಂದ-ರ-ವಾಗಿ
ಸುಂದ-ರ-ವಾದ
ಸುಂದ-ರ-ವಾ-ದೀತು
ಸುಂದ-ರಿ-ಯನ್ನು
ಸುಂದ-ರಿ-ಯಾದ
ಸುಂದರಿಯೂ
ಸುಂದ-ರಿ-ಯೊಬ್ಬಳು
ಸುಖ
ಸುಖ-ಸೌ-ಕರ್ಯ-ಗಳ
ಸುಖಕ್ಕಾಗಿ
ಸುಖಕ್ಕೂ
ಸುಖಕ್ಕೆ
ಸುಖಕ್ಕೇ
ಸುಖ-ಗ-ಳನ್ನೂ
ಸುಖ-ಜೀ-ವ-ನದ
ಸುಖದ
ಸುಖ-ದಾ-ಸೆಗೆ
ಸುಖ-ದುಃಖ-ಗಳ
ಸುಖ-ದುಃಖ-ಗ-ಳನ್ನು
ಸುಖ-ದುಃಖ-ಗ-ಳನ್ನೂ
ಸುಖ-ದುಃಖ-ಗ-ಳನ್ನೇ
ಸುಖ-ದುಃಖ-ಗ-ಳಲ್ಲಿ
ಸುಖ-ದುಃಖ-ಗ-ಳಾ-ಗಲಿ
ಸುಖ-ದುಃಖ-ಗ-ಳಿಗೂ
ಸುಖ-ದುಃಖ-ಗ-ಳಿಗೆ
ಸುಖ-ದುಃಖ-ಗ-ಳಿ-ಗೇನು
ಸುಖ-ದುಃಖ-ಗ-ಳಿವೆ
ಸುಖ-ದುಃಖ-ಗಳು
ಸುಖ-ದುಃಖ-ಗ-ಳೆಲ್ಲ-ವು-ಗಳ
ಸುಖ-ದುಃಖ-ಗಳ್ನು
ಸುಖ-ಭೋ-ಗ-ಗ-ಳನ್ನೂ
ಸುಖ-ಭೋ-ಗ-ವನ್ನು
ಸುಖ-ಭೋ-ಗಾ-ಸಕ್ತಿ-ಯನ್ನು
ಸುಖಮಯ
ಸುಖ-ಮ-ಯ-ವಾಗಿ
ಸುಖ-ಮ-ಯವೂ
ಸುಖ-ಲೋ-ಲು-ಪತೆ
ಸುಖವನ್ನು
ಸುಖ-ವಾ-ಗಲೀ
ಸುಖವಾಗಿ
ಸುಖ-ವಾ-ಗಿದ್ದ
ಸುಖ-ವಾ-ಗಿ-ರು-ವು-ದ-ರಲ್ಲೇ
ಸುಖ-ವಾ-ಗು-ವಂತೆ
ಸುಖವಿಲ್ಲ
ಸುಖವು
ಸುಖವೇ
ಸುಖ-ಶಾಂತಿ-ಗಳು
ಸುಖ-ಶಾಂತಿಯ
ಸುಖ-ಸಂತೋ-ಷ-ಗಳು
ಸುಖ-ಸಾ-ಧ-ನ-ಗ-ಳನ್ನೂ
ಸುಖ-ಸೌ-ಕರ್ಯ-ಗಳ
ಸುಖ-ಸೌ-ಕರ್ಯ-ಗ-ಳನ್ನು
ಸುಖ-ಸೌ-ಕರ್ಯ-ಗ-ಳನ್ನೂ
ಸುಖ-ಸೌ-ಕರ್ಯ-ಗ-ಳನ್ನೇ
ಸುಖ-ಸೌ-ಕರ್ಯ-ಗ-ಳಲ್ಲಿ
ಸುಖ-ಸೌ-ಕರ್ಯ-ಗ-ಳೆಲ್ಲ
ಸುಖಸ್ವಪ್ನ-ಗ-ಳನ್ನು
ಸುಖಿ-ಗ-ಳಲ್ಲ
ಸುಖಿ-ಗ-ಳಾ-ದರೆ
ಸುಖಿಗಳು
ಸುಖಿನೋ
ಸುಖಿ-ಯಾ-ಗಿ-ರುತ್ತಾನೆ
ಸುಖಿ-ಯಾ-ಗು-ವರೇ
ಸುಖಿ-ಯಾ-ಗು-ವು-ದಕ್ಕೂ
ಸುಖಿಸಲು
ಸುಖಿಸುವ
ಸುಖೋಮ್ಲಿಂಸ್ಕಿ
ಸುಗಂಧ
ಸುಗ-ಮ-ವಾ-ಗ-ಬೇಕು
ಸುಗ-ಮ-ವಾಗಿ
ಸುಗ-ಮ-ವಾ-ಗಿ-ಬಿ-ಡು-ವುದು
ಸುಗ-ಮ-ವಾ-ದೀತೆ
ಸುಗುಣ
ಸುಜಾತಾ
ಸುಟ್ಟ
ಸುಟ್ಟರೆ
ಸುಟ್ಟು
ಸುಟ್ಟುಹೋದ
ಸುಡ-ತೊ-ಡ-ಗುತ್ತದೆ
ಸುಡದ
ಸುಡ-ಬಲ್ಲುದು
ಸುಡು
ಸುಡುತ್ತದೆ
ಸುಡುವಂತೆ
ಸುಡು-ವು-ದಲ್ಲದೆ
ಸುಡು-ವು-ದಿಲ್ಲ
ಸುಣ್ಣ-ದ-ಕಲ್ಲು
ಸುಣ್ಣವಾಗಿ
ಸುತ್ತ
ಸುತ್ತಣ
ಸುತ್ತ-ತೊ-ಡಗಿ
ಸುತ್ತಮುತ್ತ
ಸುತ್ತ-ಮುತ್ತಲ
ಸುತ್ತ-ಮುತ್ತ-ಲಿನ
ಸುತ್ತ-ಮುತ್ತ-ಲಿ-ರುವ
ಸುತ್ತ-ಮುತ್ತ-ಲಿ-ರು-ವ-ವ-ರಿಗೆ
ಸುತ್ತ-ಮುತ್ತಲೂ
ಸುತ್ತಲಿನ
ಸುತ್ತಲೂ
ಸುತ್ತ-ವ-ರಿ-ದಿದೆ
ಸುತ್ತಾ-ಡುತ್ತಿದ್ದೀಯೆ
ಸುತ್ತಾ-ಡು-ವು-ದನ್ನು
ಸುತ್ತಿ
ಸುತ್ತಿಕೊಂಡು
ಸುತ್ತು-ಗ-ಳನ್ನು
ಸುತ್ತುತ್ತಿದೆ
ಸುತ್ತುತ್ತಿದ್ದೇವೆ
ಸುತ್ತುತ್ತಿ-ರು-ವು-ದನ್ನು
ಸುತ್ತುತ್ತಿವೆ
ಸುತ್ತು-ಬ-ಳಸಿ
ಸುತ್ತು-ಮುತ್ತಲ
ಸುತ್ತು-ಮುತ್ತ-ಲಿನ
ಸುತ್ತುವ
ಸುತ್ತು-ವ-ರಿದ
ಸುತ್ತು-ವ-ರಿ-ದರು
ಸುತ್ತು-ವ-ರಿ-ದಿ-ರುತ್ತವೆ
ಸುತ್ತುವಾಗ
ಸುದಿನ
ಸುದೂರ
ಸುದೂ-ರ-ಭ-ವಿಷ್ಯ-ವನ್ನು
ಸುದೈ-ವ-ದಿಂದ
ಸುದ್ದಿ
ಸುದ್ದಿ-ಗಿಂತಲೂ
ಸುದ್ದಿಯೇ
ಸುಧಾ
ಸುಧಾರಕ
ಸುಧಾ-ರ-ಕರೂ
ಸುಧಾರಣಾ
ಸುಧಾ-ರ-ಣಾ-ಕಾರ್ಯ-ವನ್ನು
ಸುಧಾ-ರ-ಣಾ-ವಾ-ದಿ-ಗಳು
ಸುಧಾರಣೆ
ಸುಧಾ-ರ-ಣೆ-ಗ-ಳನ್ನು
ಸುಧಾ-ರ-ಣೆ-ಗ-ಳಾ-ಗ-ಬೇಕು
ಸುಧಾ-ರ-ಣೆಯ
ಸುಧಾ-ರ-ಣೆ-ಯಿಂದ
ಸುಧಾರಿತ
ಸುಧಾ-ರಿ-ಸಲು
ಸುಧಾ-ರಿ-ಸಿತು
ಸುಧಾ-ರಿ-ಸಿದ
ಸುಧಾ-ರಿ-ಸಿ-ದಳು
ಸುಧಾ-ರಿ-ಸಿದ್ದಾರೆ
ಸುಧಾ-ರಿ-ಸು-ವುದು
ಸುಪೀ-ರಿ-ಯಾ-ರಿಟಿ
ಸುಪ್ತ
ಸುಪ್ತನಿದ್ರಾ
ಸುಪ್ತ-ನಿದ್ರೆಗೆ
ಸುಪ್ತ-ನಿದ್ರೆ-ಗೊ-ಳ-ಗಾದ
ಸುಪ್ತ-ನಿದ್ರೆ-ಗೊ-ಳ-ಪ-ಡಿ-ಸಿ-ದರೆ
ಸುಪ್ತ-ನಿದ್ರೆಯ
ಸುಪ್ತ-ನಿದ್ರೆ-ಯಲ್ಲಿ
ಸುಪ್ತ-ನಿದ್ರೆ-ಯಲ್ಲಿದ್ದು-ಕೊಂಡು
ಸುಪ್ತ-ನಿದ್ರೆ-ಯಲ್ಲಿದ್ದು-ಕೊಂಡೇ
ಸುಪ್ತ-ನಿದ್ರೆ-ಯಲ್ಲಿ-ರು-ವಾಗ
ಸುಪ್ತ-ನಿದ್ರೆ-ಯಲ್ಲಿ-ರು-ವಾ-ಗಲೇ
ಸುಪ್ತಪ್ರಜ್ಞೆಯ
ಸುಪ್ತಪ್ರಜ್ಞೆ-ಯನ್ನು
ಸುಪ್ತ-ಮ-ನದ
ಸುಪ್ತ-ಮ-ನಸ್ಸನ್ನು
ಸುಪ್ತ-ಮ-ನಸ್ಸಿನ
ಸುಪ್ತ-ಮ-ನಸ್ಸಿ-ನಲ್ಲಿ
ಸುಪ್ತ-ಮ-ನಸ್ಸು
ಸುಪ್ತ-ವಾ-ಗಿ-ರುವ
ಸುಪ್ತವಾಣಿ
ಸುಪ್ತ-ಶಕ್ತಿ-ಗ-ಳನ್ನು
ಸುಪ್ತ-ಶಕ್ತಿ-ಯನ್ನು
ಸುಪ್ತಸ್ಥ-ತಿಗೆ
ಸುಪ್ತಸ್ಥಿ-ತಿಗೆ
ಸುಪ್ತಸ್ಥಿ-ತಿ-ಯಲ್ಲಿ
ಸುಪ್ತಸ್ಥಿ-ತಿ-ಯಲ್ಲಿದ್ದ
ಸುಪ್ತಸ್ಥಿ-ತಿ-ಯಲ್ಲಿದ್ದಾಗ
ಸುಪ್ತಸ್ಥಿ-ತಿ-ಯಲ್ಲಿದ್ದು-ಕೊಂಡು
ಸುಪ್ತಾ-ವಸ್ಥೆಗೆ
ಸುಪ್ತಾ-ವಸ್ಥೆಯ
ಸುಪ್ತಾ-ವಸ್ಥೆ-ಯಲ್ಲಿ
ಸುಪ್ತಾ-ವಸ್ಥೆ-ಯಲ್ಲಿನ
ಸುಪ್ತಾ-ವಸ್ಥೆ-ಯಲ್ಲಿ-ರುವ
ಸುಪ್ತಾ-ವಸ್ಥೆ-ಯಲ್ಲಿ-ರು-ವುದು
ಸುಪ್ತಾ-ವಸ್ಥೆ-ಯಲ್ಲೇ
ಸುಪ್ತಿ
ಸುಪ್ತಿ-ಗೊ-ಳ-ಗಾ-ದ-ವನ
ಸುಪ್ತಿಯ
ಸುಪ್ತಿ-ಶಾಸ್ತ್ರಜ್ಞ
ಸುಪ್ತಿ-ಸೂ-ಚನೆ
ಸುಪ್ತ್ಯಾ-ವಾ-ಹನೆ
ಸುಪ್ತ್ಯಾ-ವಾ-ಹ-ನೆ-ಯನ್ನು
ಸುಪ್ತ್ಯಾ-ವಾ-ಹ-ನೆ-ಯಲ್ಲಿ
ಸುಪ್ತ್ಯಾ-ವಾ-ಹ-ನೆ-ಯಿಂದ
ಸುಪ್ರ-ನಿದ್ರೆ-ಯಿಂದ
ಸುಪ್ರಸಿದ್ಧ
ಸುಪ್ರೀಂಕೋರ್ಟಿನ
ಸುಭಾಷಿತ
ಸುಮಾರಾಗಿ
ಸುಮಾರಿಗೆ
ಸುಮಾರು
ಸುಮ್ಮ
ಸುಮ್ಮನಾಗಿ
ಸುಮ್ಮ-ನಾ-ಗುತ್ತಾ-ನೆಯೇ
ಸುಮ್ಮ-ನಾ-ದರೂ
ಸುಮ್ಮ-ನಿದ್ದಿರಿ
ಸುಮ್ಮನಿದ್ದು
ಸುಮ್ಮನೆ
ಸುಮ್ಮನೇ
ಸುಯೋಗ್ಯ
ಸುರಂಗ
ಸುರಂಗ-ವೊಂದ-ರಲ್ಲಿ
ಸುರಕ್ಷಿತ
ಸುರಕ್ಷಿ-ತ-ತೆಯ
ಸುರಕ್ಷಿ-ತ-ವಾ-ಗಿಟ್ಟು-ಕೊಳ್ಳು-ವು-ದೆಂದು
ಸುರಕ್ಷಿ-ತ-ವಾ-ಗಿ-ರ-ಬಲ್ಲು-ದೆಂದೇ
ಸುರಕ್ಷಿ-ತ-ವಾ-ಗಿ-ರು-ವು-ದನ್ನು
ಸುರಕ್ಷೆ
ಸುರಕ್ಷೆ-ಯನ್ನು
ಸುರಿದ
ಸುರಿದರೂ
ಸುರಿ-ಮ-ಳೆ-ಯನ್ನು
ಸುರಿಯ
ಸುರಿ-ಯ-ತೊ-ಡ-ಗಿತು
ಸುರಿಯಲು
ಸುರಿ-ಯಲ್ಪ-ಡುವ
ಸುರಿಯಿತು
ಸುರಿ-ಯುತ್ತದೆ
ಸುರಿ-ಯುತ್ತಿದೆ
ಸುರಿ-ಯುತ್ತಿದ್ದಾರೆ
ಸುರಿಯುವ
ಸುರಿ-ಯು-ವಂತೆ
ಸುರಿಸಿ
ಸುರಿ-ಸಿ-ದರೋ
ಸುರುವಿದ
ಸುಲಭ
ಸುಲ-ಭ-ಕಾರ್ಯ-ವೆಂದು
ಸುಲಭದ
ಸುಲ-ಭ-ದಲ್ಲಿ
ಸುಲ-ಭ-ದಿಂದ
ಸುಲ-ಭ-ವಲ್ಲ
ಸುಲ-ಭ-ವಾಗಿ
ಸುಲ-ಭ-ವಾ-ಗಿದ್ದು
ಸುಲ-ಭ-ವಾ-ಗಿ-ರ-ಲಿಲ್ಲ
ಸುಲ-ಭ-ವಾ-ಗು-ವಂತೆ
ಸುಲ-ಭ-ವಾದ
ಸುಲ-ಭ-ವಾ-ದು-ದನ್ನೇ
ಸುಲಭವೆ
ಸುಲ-ಭ-ವೆಂದು
ಸುಲ-ಭ-ವೆನ್ನುತ್ತಾನೆ
ಸುಲಭವೋ
ಸುಲ-ಭ-ಸಾಧ್ಯ
ಸುಲ-ಭ-ಸಾಧ್ಯ-ವಲ್ಲ
ಸುಲ-ಲಿ-ತ-ವಾಗಿ
ಸುಲಿಗೆ
ಸುಲಿ-ಗೆ-ಗಳ
ಸುಲಿ-ಗೆ-ಯಿಂದ
ಸುಲಿಯಿತು
ಸುಳಿದು
ಸುಳಿಯದು
ಸುಳಿ-ಯ-ಲಾ-ರವು
ಸುಳಿಯಲ್ಲಿ
ಸುಳಿವಿಲ್ಲ
ಸುಳಿವೂ
ಸುಳಿವೇ
ಸುಳ್ಳನ್ನೂ
ಸುಳ್ಳನ್ನೇ
ಸುಳ್ಳಲ್ಲ
ಸುಳ್ಳಾಗಲು
ಸುಳ್ಳಾಗಿತ್ತು
ಸುಳ್ಳಾ-ಡಿ-ದ-ನೆಂದು
ಸುಳ್ಳಾದರೂ
ಸುಳ್ಳು
ಸುಳ್ಳುಗಾರ
ಸುವ
ಸುವರ್ಣ
ಸುವರ್ಣಾ-ವ-ಕಾಶ
ಸುವಾ-ಸ-ನೆಯೂ
ಸುವು-ದ-ರಲ್ಲಿ
ಸುವು-ದ-ರಿಂದ
ಸುವ್ಯ-ವಸ್ಥಿತ
ಸುವ್ಯ-ವಸ್ಥಿ-ತವೂ
ಸುವ್ಯ-ವಸ್ಥೆಗೆ
ಸುಶಿಕ್ಷಿತ
ಸುಶಿಕ್ಷಿ-ತ-ರನ್ನಾಗಿ
ಸುಸಂಗತ
ಸುಸಂಗ-ತ-ವಾಗಿ
ಸುಸಂಘ-ಟಿತ
ಸುಸಂಸ್ಕೃತ
ಸುಸಂಸ್ಕೃ-ತ-ರಿ-ಗಿಂತ
ಸುಸಂಸ್ಕೃ-ತರೂ
ಸುಸಂಸ್ಕೃ-ತರೇ
ಸುಸಂಸ್ಕೃ-ತವೂ
ಸುಸಜ್ಜಿತ
ಸುಸಜ್ಜಿ-ತ-ರಾಗಿ
ಸುಸ್ತಾ-ಗಿ-ರ-ಬೇ-ಕೆಂದು
ಸುಸ್ತಾದವು
ಸುಸ್ತು
ಸುಸ್ಥಿತಿ
ಸುಸ್ಥಿತಿಗೂ
ಸುಸ್ಥಿತಿಗೆ
ಸುಸ್ಪಷ್ಟ-ವಾಗಿ
ಸುಸ್ಪಷ್ಟವೂ
ಸೂಕ್ತ
ಸೂಕ್ತ-ನಿರ್ವ-ಹ-ಣೆಯ
ಸೂಕ್ತವಾದ
ಸೂಕ್ಷ್ಮ
ಸೂಕ್ಷ್ಮ-ಕ-ಣ-ಗ-ಳಿವೆ
ಸೂಕ್ಷ್ಮ-ಚಿತ್ರಗ್ರ-ಹಣ
ಸೂಕ್ಷ್ಮ-ದರ್ಶಕ
ಸೂಕ್ಷ್ಮ-ದರ್ಶ-ಕದ
ಸೂಕ್ಷ್ಮದರ್ಶಿ
ಸೂಕ್ಷ್ಮದೇಹ
ಸೂಕ್ಷ್ಮ-ದೇ-ಹಕ್ಕೆ
ಸೂಕ್ಷ್ಮ-ನಾ-ಗಿ-ರು-ವು-ದ-ರಿಂದ
ಸೂಕ್ಷ್ಮ-ನಿ-ಯ-ಮ-ವನ್ನ-ನು-ಸ-ರಿಸಿ
ಸೂಕ್ಷ್ಮ-ನಿ-ಯ-ಮ-ವನ್ನು
ಸೂಕ್ಷ್ಮ-ನಿ-ಯ-ಮ-ವಿದೆ
ಸೂಕ್ಷ್ಮ-ಪ-ರಿ-ಚಯ
ಸೂಕ್ಷ್ಮಪ್ರಾ-ಕೃ-ತಿಕ
ಸೂಕ್ಷ್ಮರೂಪ
ಸೂಕ್ಷ್ಮ-ರೂ-ಪ-ದಲ್ಲಿ
ಸೂಕ್ಷ್ಮವಾಗಿ
ಸೂಕ್ಷ್ಮ-ವಾ-ಗಿದ್ದುದು
ಸೂಕ್ಷ್ಮವಾದ
ಸೂಕ್ಷ್ಮ-ವಾ-ದು-ದ-ರಿಂದ
ಸೂಕ್ಷ್ಮ-ವಾ-ದುದು
ಸೂಕ್ಷ್ಮ-ವಿ-ಷ-ಯ-ಗ-ಳನ್ನು
ಸೂಕ್ಷ್ಮವೂ
ಸೂಕ್ಷ್ಮ-ಶ-ರೀರ
ಸೂಕ್ಷ್ಮ-ಶ-ರೀ-ರ-ದಲ್ಲಿ
ಸೂಕ್ಷ್ಮ-ಶ-ರೀ-ರ-ವೆಂದು
ಸೂಕ್ಷ್ಮಸ್ಥಿ-ತಿ-ಯಲ್ಲಿ-ರುವ
ಸೂಕ್ಷ್ಮಸ್ಪಂದನ
ಸೂಕ್ಷ್ಮಾಣು
ಸೂಕ್ಷ್ಮಾ-ತಿ-ಸೂಕ್ಷ್ಮ
ಸೂಕ್ಷ್ಮಾ-ತಿ-ಸೂಕ್ಷ್ಮ-ಕ-ಣ-ಗಳ
ಸೂಕ್ಷ್ಮಾ-ತಿ-ಸೂಕ್ಷ್ಮ-ನಾಗಿ
ಸೂಕ್ಷ್ಮಾ-ತಿ-ಸೂಕ್ಷ್ಮ-ರೂ-ಪ-ದಿಂದಿ-ರುವ
ಸೂಕ್ಷ್ಮಾ-ತಿ-ಸೂಕ್ಷ್ಮ-ವಾ-ಗಿದೆ
ಸೂಕ್ಷ್ಮಾ-ತಿ-ಸೂಕ್ಷ್ಮ-ವಾ-ಗಿದ್ದು-ಕೊಂಡು
ಸೂಕ್ಷ್ಮಾ-ತಿ-ಸೂಕ್ಷ್ಮ-ವಾದ
ಸೂಕ್ಷ್ಮಾ-ತಿ-ಸೂಕ್ಷ್ಮಸ್ಥಿ-ತಿ-ಯನ್ನು
ಸೂಚನೆ
ಸೂಚ-ನೆ-ಕೊಡಿ
ಸೂಚ-ನೆ-ಗಳ
ಸೂಚ-ನೆ-ಗ-ಳನ್ನಿತ್ತರೆ
ಸೂಚ-ನೆ-ಗ-ಳನ್ನಿತ್ತು
ಸೂಚ-ನೆ-ಗ-ಳನ್ನು
ಸೂಚ-ನೆ-ಗ-ಳನ್ನೂ
ಸೂಚ-ನೆ-ಗಳು
ಸೂಚನೆಗೆ
ಸೂಚನೆಯ
ಸೂಚ-ನೆ-ಯಂತೆಯೇ
ಸೂಚ-ನೆ-ಯನ್ನ-ನು-ಸ-ರಿಸಿ
ಸೂಚ-ನೆ-ಯನ್ನಾ-ಗಲಿ
ಸೂಚ-ನೆ-ಯನ್ನು
ಸೂಚ-ನೆ-ಯಾ-ಗಲಿ
ಸೂಚ-ನೆ-ಯಿಂದ
ಸೂಚಿ-ತ-ವಾ-ಗಿದೆ
ಸೂಚಿ-ತ-ವಾ-ಗುತ್ತದೆ
ಸೂಚಿ-ತ-ವಾ-ಗು-ವುದು
ಸೂಚಿಸದೆ
ಸೂಚಿಸಲು
ಸೂಚಿ-ಸಲ್ಪಟ್ಟ
ಸೂಚಿ-ಸಲ್ಪಟ್ಟಿದೆ
ಸೂಚಿಸಿ
ಸೂಚಿಸಿದ
ಸೂಚಿ-ಸಿ-ದರು
ಸೂಚಿ-ಸಿ-ದರೆ
ಸೂಚಿಸಿದೆ
ಸೂಚಿಸಿದ್ದ
ಸೂಚಿ-ಸಿದ್ದರು
ಸೂಚಿ-ಸಿದ್ದಾರೆ
ಸೂಚಿ-ಸಿ-ರು-ವಂಥ
ಸೂಚಿ-ಸುತ್ತದೆ
ಸೂಚಿ-ಸುತ್ತ-ದೆ-ಎಂದು
ಸೂಚಿ-ಸುತ್ತ-ದೆ-ಯಲ್ಲವೆ
ಸೂಚಿ-ಸುತ್ತವೆ
ಸೂಚಿ-ಸುತ್ತಾನೆ
ಸೂಚಿ-ಸುತ್ತಿದ್ದ
ಸೂಚಿ-ಸುತ್ತಿದ್ದಾನೆ
ಸೂಚಿ-ಸುತ್ತಿದ್ದೆ
ಸೂಚಿಸುವ
ಸೂಚಿ-ಸು-ವಂತೆಯೂ
ಸೂಚಿ-ಸು-ವಂಥದು
ಸೂಚಿ-ಸು-ವುದು
ಸೂಚ್ಯವಾಗಿ
ಸೂಚ್ಯ-ವಾ-ಗಿಯೋ
ಸೂಜಿ
ಸೂಜಿಯನ್ನು
ಸೂಜಿಯು
ಸೂತ್ರ
ಸೂತ್ರಕ್ಕಿಂತ
ಸೂತ್ರ-ಗ-ಳನ್ನು
ಸೂತ್ರ-ಗ-ಳಿವು
ಸೂತ್ರದ
ಸೂತ್ರದಂತೆ
ಸೂತ್ರದಲ್ಲಿ
ಸೂತ್ರ-ದಿಂದೆ-ಳೆ-ದಂತೆ
ಸೂತ್ರಧಾರ
ಸೂತ್ರ-ಧಾ-ರ-ನನ್ನೇ
ಸೂತ್ರ-ಧಾ-ರ-ನಾದ
ಸೂತ್ರವನ್ನು
ಸೂತ್ರವನ್ನೊ
ಸೂತ್ರವಿದೆ
ಸೂತ್ರ-ವಿಲ್ಲದೆ
ಸೂತ್ರವು
ಸೂತ್ರವೂ
ಸೂತ್ರ-ವೊಂದರ
ಸೂಪರ್
ಸೂರೆ-ಗೈ-ಯಲು
ಸೂರೆ-ಗೊಳ್ಳಲು
ಸೂರೆ-ಗೊಳ್ಳುವ
ಸೂರ್ಯ
ಸೂರ್ಯ-ಕಿ-ರ-ಣ-ಗಳ
ಸೂರ್ಯ-ಕಿ-ರ-ಣ-ಗ-ಳನ್ನು
ಸೂರ್ಯ-ಕಿ-ರ-ಣ-ಗ-ಳಿಂದ
ಸೂರ್ಯ-ಕಿ-ರ-ಣ-ಗಳು
ಸೂರ್ಯಗ್ರ-ಹಣ
ಸೂರ್ಯಚಂದ್ರ
ಸೂರ್ಯ-ಚಂದ್ರ-ರಲ್ಲಿ
ಸೂರ್ಯ-ದೇ-ವನ
ಸೂರ್ಯನ
ಸೂರ್ಯನನ್ನು
ಸೂರ್ಯನನ್ನೇ
ಸೂರ್ಯ-ನಲ್ಲವೇ
ಸೂರ್ಯನಿಗೆ
ಸೂರ್ಯನು
ಸೂರ್ಯನೂ
ಸೂರ್ಯನೇ
ಸೂರ್ಯ-ಮಂಡ-ಲ-ವನ್ನು
ಸೂರ್ಯಾಸ್ತ-ಮಾ-ನ-ಗ-ಳಿಂದ
ಸೂರ್ಯೋದಯ
ಸೂರ್ಯೋ-ದ-ಯಕ್ಕೆ
ಸೂರ್ಯೋ-ದ-ಯ-ವಾ-ಗಿಲ್ಲ
ಸೂರ್ಯೋ-ದ-ಯ-ವಾ-ದು-ದನ್ನು
ಸೂಸ-ತೊ-ಡ-ಗುತ್ತವೆ
ಸೂಸುವಂತೆ
ಸೃಜ-ನ-ಶೀಲ
ಸೃಜ-ನಾತ್ಮಕ
ಸೃಜಿ-ಸ-ಬಲ್ಲ
ಸೃಜಿ-ಸಿ-ಹನೋ
ಸೃಷ್ಟಿ
ಸೃಷ್ಟಿ-ಕರ್ತನ
ಸೃಷ್ಟಿಯ
ಸೃಷ್ಟಿ-ಯ-ಲಿ-ಮಂಕು-ತಿಮ್ಮ
ಸೃಷ್ಟಿಯಲ್ಲಿ
ಸೃಷ್ಟಿ-ಯಲ್ಲಿ-ರುವ
ಸೃಷ್ಟಿಯಲ್ಲೂ
ಸೃಷ್ಟಿ-ಯಾ-ದಂದಿ-ನಿಂದ
ಸೃಷ್ಟಿ-ಸ-ಬಲ್ಲದು
ಸೃಷ್ಟಿ-ಸ-ಬ-ಹುದು
ಸೃಷ್ಟಿಸಿ
ಸೃಷ್ಟಿ-ಸಿ-ಕೊಂಡಿದೆ
ಸೃಷ್ಟಿ-ಸಿ-ಕೊಂಡು
ಸೃಷ್ಟಿ-ಸಿ-ಕೊಳ್ಳ-ಬೇಕು
ಸೃಷ್ಟಿ-ಸಿ-ಕೊಳ್ಳು-ವುದು
ಸೃಷ್ಟಿಸಿದ
ಸೃಷ್ಟಿ-ಸಿದ್ದರು
ಸೃಷ್ಟಿಸೀತು
ಸೃಷ್ಟಿ-ಸುತ್ತಾರೆ
ಸೃಷ್ಟಿಸುವ
ಸೃಷ್ಟಿಸ್ಥಿತಿ
ಸೆಕೆಂಡಿಗೆ
ಸೆಟೆದು
ಸೆಟೆ-ದು-ಕೊಂಡಿಲ್ಲ
ಸೆಟೆ-ದು-ಕೊಂಡು
ಸೆಟ್ಟಿ
ಸೆಣಸು
ಸೆಪ್ಟೆಂಬ-ರ-ದಲ್ಲಿ
ಸೆಪ್ಟೆಂಬರ್
ಸೆಮಿಸ್ಟರ್
ಸೆರೆ
ಸೆರೆ-ಮ-ನೆ-ಯಲ್ಲಿದ್ದಾಗ
ಸೆರೆ-ಸಿಕ್ಕಿ-ದಾಗ
ಸೆರ್ಮಿನಾರ
ಸೆರ್ಮಿನಾರಾ
ಸೆಲೆ
ಸೆಲ್ಫ್
ಸೆಲ್ಮ
ಸೆಲ್ಮಾಕ್ಕೆ
ಸೆಲ್ಯೂಟ್
ಸೆಳೆತ
ಸೆಳೆ-ತ-ಇ-ವು-ಗ-ಳಲ್ಲಿ
ಸೆಳೆತಕ್ಕೆ
ಸೆಳೆ-ತ-ಗಳು
ಸೆಳೆ-ತ-ದಿಂದ
ಸೆಳೆ-ತ-ಸು-ಖಕ್ಕಾಗಿ
ಸೆಳೆದ
ಸೆಳೆದಾಗ
ಸೆಳೆ-ದಿದ್ದ-ನಾತ
ಸೆಳೆ-ದು-ಕೊಳ್ಳಲು
ಸೆಳೆ-ಯ-ಲಾ-ರವು
ಸೆಳೆ-ಯ-ಲಿಲ್ಲ
ಸೆಳೆಯಲು
ಸೆಳೆ-ಯಲ್ಪ-ಡುತ್ತವೆ
ಸೆಳೆಯಿತು
ಸೆಳೆ-ಯುತ್ತ-ದೆಂಬುದು
ಸೆಳೆ-ಯುತ್ತವೆ
ಸೆಳೆ-ಯುತ್ತಿವೆ
ಸೆಳೆ-ಯು-ವುದು
ಸೆಳೆ-ಯು-ವುವು
ಸೇಡಿನ
ಸೇಡು
ಸೇತುವೆಯ
ಸೇತು-ವೆ-ಯಂತೆ
ಸೇದಲು
ಸೇದಿ
ಸೇದುವ
ಸೇನರ
ಸೇನೆಯಲ್ಲಿ
ಸೇಬು
ಸೇರದ
ಸೇರ-ಬ-ಹುದು
ಸೇರ-ಬ-ಹು-ದೆಂಬು-ದನ್ನು
ಸೇರಲು
ಸೇರಿ
ಸೇರಿ-ಕೊಂಡಾಗ
ಸೇರಿ-ಕೊಂಡಿವೆ
ಸೇರಿಕೊಂಡ
ಸೇರಿ-ಕೊಂಡಂಥ-ವನು
ಸೇರಿ-ಕೊಂಡಾಗ
ಸೇರಿ-ಕೊಂಡಿದ್ದ
ಸೇರಿ-ಕೊಂಡಿದ್ದರೆ
ಸೇರಿ-ಕೊಂಡಿದ್ದು-ದನ್ನು
ಸೇರಿ-ಕೊಂಡಿ-ರುತ್ತದೆ
ಸೇರಿ-ಕೊಂಡಿ-ರುವ
ಸೇರಿ-ಕೊಂಡಿವೆ
ಸೇರಿ-ಕೊಂಡಿ-ವೆ-ಯೆಂಬು-ದೇನೊ
ಸೇರಿಕೊಂಡು
ಸೇರಿ-ಕೊಳ್ಳುತ್ತಾನೆ
ಸೇರಿ-ಕೊಳ್ಳುತ್ತಿದ್ದ
ಸೇರಿತು
ಸೇರಿದ
ಸೇರಿದಂತೆ
ಸೇರಿದರು
ಸೇರಿದರೆ
ಸೇರಿ-ದ-ವ-ನಲ್ಲ
ಸೇರಿ-ದ-ವನು
ಸೇರಿ-ದ-ವ-ರನ್ನು
ಸೇರಿ-ದ-ವ-ರಾ-ಗಲಿ
ಸೇರಿ-ದ-ವ-ರಿಗೆ
ಸೇರಿ-ದ-ವರು
ಸೇರಿ-ದ-ವರೆ
ಸೇರಿ-ದ-ವ-ರೆಂದರೆ
ಸೇರಿ-ದ-ವ-ರೆಂದು
ಸೇರಿದಾಗ
ಸೇರಿದೆ
ಸೇರಿದ್ದ
ಸೇರಿದ್ದರು
ಸೇರಿದ್ದಲ್ಲ
ಸೇರಿದ್ದಾ-ರೆಂಬು-ದನ್ನು
ಸೇರಿವೆ
ಸೇರಿ-ಸ-ಲಾ-ಗಿದೆ
ಸೇರಿ-ಸ-ಲಾ-ಯಿತು
ಸೇರಿಸಲಿ
ಸೇರಿಸಲು
ಸೇರಿಸಿ
ಸೇರಿ-ಸಿ-ಕೊಂಡಿದ್ದು-ದರ
ಸೇರಿ-ಸಿ-ಕೊಂಡು
ಸೇರಿ-ಸಿ-ಕೊಳ್ಳದೆ
ಸೇರಿ-ಸಿ-ಕೊಳ್ಳುವ
ಸೇರಿಸಿದ
ಸೇರಿ-ಸಿ-ದರು
ಸೇರಿಸಿದ್ದ
ಸೇರಿ-ಸಿ-ಬಿಟ್ಟರು
ಸೇರಿಸಿಲ್ಲ
ಸೇರಿಸುತ್ತಾ
ಸೇರಿ-ಸುತ್ತಿ-ರ-ಲಿಲ್ಲ
ಸೇರಿಸುವ
ಸೇರಿ-ಸು-ವ-ವಳು
ಸೇರಿ-ಸು-ವುದು
ಸೇರು
ಸೇರುತ್ತದೆ
ಸೇರುತ್ತಾ-ನೆನ್ನಿ
ಸೇರುತ್ತಾರೆ
ಸೇರುತ್ತೇವೆ
ಸೇರುವ
ಸೇರುವಂತೆ
ಸೇರು-ವ-ವ-ರೆಗೂ
ಸೇರು-ವ-ವ-ರೆಗೆ
ಸೇರು-ವು-ದಂತೂ
ಸೇರು-ವು-ದಕ್ಕೆ
ಸೇರು-ವು-ದಿಲ್ಲ
ಸೇಲ್ಸ್ಮ್ಯಾನ್
ಸೇವಕ
ಸೇವ-ಕ-ನನ್ನಾಗಿ
ಸೇವ-ಕ-ನನ್ನು
ಸೇವಕನೂ
ಸೇವಕನೇ
ಸೇವ-ಕ-ರಲ್ಲ
ಸೇವ-ಕ-ರಾ-ಗ-ಬೇಡಿ
ಸೇವ-ಕ-ರಿಗೆ
ಸೇವಕರು
ಸೇವನೆ
ಸೇವನೆಗೆ
ಸೇವನೆಯ
ಸೇವ-ನೆ-ಯಿಂದ
ಸೇವ-ನೆ-ಯಿಲ್ಲದೆ
ಸೇವಾ
ಸೇವಾಕಾರ್ಯ
ಸೇವಾ-ಕಾರ್ಯ-ವನ್ನು
ಸೇವಾಕ್ಷೇತ್ರ-ದಲ್ಲಿ
ಸೇವಾಕ್ಷೇತ್ರ-ವನ್ನು
ಸೇವಾ-ದೃಷ್ಟಿ-ಯನ್ನು
ಸೇವಾ-ಧರ್ಮದ
ಸೇವಾ-ಪ-ರಾ-ಯ-ಣತೆ
ಸೇವಾ-ಪ-ರಾ-ಯ-ಣ-ರಾಗಿ
ಸೇವಾ-ಪ-ರಾ-ಯ-ಣೆ-ಯಾದ
ಸೇವಾಭಾವ
ಸೇವಾ-ಮ-ನೋ-ಭಾವ
ಸೇವಾ-ಮ-ನೋ-ಭಾ-ವ-ಇವೇ
ಸೇವಾ-ವೃತ್ತಿ-ಯಲ್ಲಿ
ಸೇವಿಂಗ್ಸ್
ಸೇವಿ-ಸ-ದಿದ್ದ
ಸೇವಿ-ಸ-ಬೇಕು
ಸೇವಿ-ಸ-ಹೊ-ರ-ಟನೋ
ಸೇವಿಸಿ
ಸೇವಿ-ಸಿ-ದಾಗ
ಸೇವಿ-ಸಿದ್ದ-ನಂತೆ
ಸೇವಿಸಿಯೂ
ಸೇವಿಸಿಯೇ
ಸೇವಿಸು
ಸೇವಿ-ಸು-ವ-ವರು
ಸೇವಿ-ಸುತ್ತವೆ
ಸೇವಿ-ಸುತ್ತಾರೆ
ಸೇವಿ-ಸುತ್ತಿದ್ದರೂ
ಸೇವಿಸುವ
ಸೇವಿ-ಸು-ವಾಗ
ಸೇವೆ
ಸೇವೆಇವು
ಸೇವೆ-ಇ-ವೆಲ್ಲ
ಸೇವೆಎಂಬ
ಸೇವೆಗಳ
ಸೇವೆ-ಗ-ಳನ್ನು
ಸೇವೆ-ಗ-ಳಲ್ಲೇ
ಸೇವೆ-ಗ-ಳಿಂದ
ಸೇವೆ-ಗ-ಳಿಂದಲ್ಲವೆ
ಸೇವೆ-ಗ-ಳಿಗೆ
ಸೇವೆಗಾಗಿ
ಸೇವೆಗೂ
ಸೇವೆಗೆ
ಸೇವೆಯ
ಸೇವೆ-ಯನ್ನಾ-ದರೂ
ಸೇವೆಯನ್ನು
ಸೇವೆಯನ್ನೂ
ಸೇವೆಯಲ್ಲ
ಸೇವೆ-ಯಾ-ಗುತ್ತದೆ
ಸೇವೆ-ಯಾ-ಗು-ವುದು
ಸೇವೆಯೇ
ಸೇವ್ಯನನ್ನು
ಸೇವ್ಯರ
ಸೈ
ಸೈಂಟಿ-ಫಿ-ಕಲೀ
ಸೈಂಟಿಫಿಕ್
ಸೈಕ-ಲನ್ನೇರಿ
ಸೈಕ-ಲಾ-ಜಿ-ಕಲ್
ಸೈಕಲ್
ಸೈಕಲ್ಲಿನ
ಸೈಕಲ್ಲಿ-ನಲ್ಲಿ-ರುವ
ಸೈಕಾಲಜಿ
ಸೈಕಾ-ಲ-ಜಿಯ
ಸೈಕಿಯಾಟ್ರಿ
ಸೈಕಿ-ಯಾಟ್ರಿಸ್ಟ್
ಸೈಕೀ
ಸೈಕೊ-ಲಾ-ಜಿ-ಕಲ್
ಸೈಕ್
ಸೈನಿಕ
ಸೈನಿ-ಕ-ನನ್ನು
ಸೈನಿಕನು
ಸೈನಿ-ಕ-ನೊಬ್ಬ
ಸೈನಿ-ಕ-ರಿಗೆ
ಸೈನ್ಯ
ಸೈನ್ಯಕ್ಕೆ
ಸೈನ್ಯದಲ್ಲಿ
ಸೈನ್ಯವು
ಸೈನ್ಯಾ-ಧಿ-ಕಾ-ರಿ-ಗಳು
ಸೈನ್ಸಸ್
ಸೈನ್ಸಸ್ನ
ಸೈನ್ಸ್
ಸೈರಿ-ಸಿ-ಕೊಂಡು
ಸೈರಿ-ಸಿ-ಕೊಳ್ಳು-ವು-ದನ್ನು
ಸೊಂಟ
ಸೊಂಟಕ್ಕೆ
ಸೊಂಟದ
ಸೊಂಡಿ-ಲಿ-ನಿಂದ
ಸೊಂಯ್
ಸೊಕ್ಕು
ಸೊಗ-ವಿ-ರ-ಲ-ದನು
ಸೊಗಸಾದ
ಸೊಟ್ಟ-ಗಾ-ಗಿದ್ದ
ಸೊತ್ತು
ಸೊತ್ತೂ
ಸೊಪ್ಪು-ಹಾ-ಕು-ವ-ವ-ರಲ್ಲೂ
ಸೊಬ-ಗಿ-ನೊಂದಿಗೆ
ಸೊರಗಿ
ಸೊರಗಿದೆ
ಸೊರೊಕಿನ್
ಸೊರೊ-ಕಿನ್ನರ
ಸೊರೊ-ಕಿನ್ನರು
ಸೊರೊಕಿನ್ರ
ಸೊರೋಕಿನ್
ಸೊಲ್ಲಿದು
ಸೊಲ್ಲು
ಸೊಲ್ಲು-ಗ-ಳಲ್ಲಿದೆ
ಸೊಳ್ಳೆಗೂ
ಸೊಸೆ
ಸೊಸೆಗೆ
ಸೊಸೆಯ
ಸೊಸೆ-ಯಾ-ಗಿ-ರು-ವಾಗ
ಸೊಸೈಟಿ
ಸೋಂಕು
ಸೋಗಲ್ಲ
ಸೋಗಿನಲ್ಲಿ
ಸೋತಂತೆ
ಸೋತರೂ
ಸೋತವರು
ಸೋತಾಗ
ಸೋತಿದ್ದಾ-ರೆಂದರೆ
ಸೋತು
ಸೋತೆನೆಂದು
ಸೋದರ
ಸೋದರತೆ
ಸೋದ-ರಿ-ಯರ
ಸೋಪಾನ
ಸೋಪಾ-ನಕ್ರ-ಮ-ದಿಂದ
ಸೋಪಾ-ನ-ಗಳು
ಸೋಪಾನವೇ
ಸೋಮವಾರ
ಸೋಮಾರಿ
ಸೋಮಾ-ರಿ-ಗಳು
ಸೋಮಾ-ರಿ-ತನ
ಸೋಮಾ-ರಿ-ಯಾಗಿ
ಸೋಮಾರಿಯೂ
ಸೋಲನ್ನು
ಸೋಲ-ಬ-ಹು-ದೆಂಬ
ಸೋಲಲಿಲ್ಲ
ಸೋಲಲ್ಲ
ಸೋಲವುದು
ಸೋಲಿನ
ಸೋಲಿನಿಂದ
ಸೋಲಿ-ಸ-ಬ-ಹು-ದೆಂದು
ಸೋಲಿಸಲು
ಸೋಲಿಸಿ
ಸೋಲು
ಸೋಲುಈ
ಸೋಲು-ಗ-ಳಿಂದ
ಸೋಲುತ್ತಿ-ರುವ
ಸೋವಿ
ಸೋವಿಯತ್
ಸೌಂದರ್ಯ
ಸೌಂದರ್ಯ-ದಿಂದ
ಸೌಂದರ್ಯ-ವನ್ನು
ಸೌಕರ್ಯ
ಸೌಕರ್ಯ-ಗಳ
ಸೌಕರ್ಯ-ಗ-ಳನ್ನು
ಸೌಕರ್ಯ-ಗ-ಳಲ್ಲಿ
ಸೌಕರ್ಯ-ಗ-ಳಾ-ವುವೂ
ಸೌಕರ್ಯ-ಗ-ಳಿಗೆ
ಸೌಕರ್ಯ-ಗಳೂ
ಸೌಖ್ಯ
ಸೌಖ್ಯವನ್ನು
ಸೌಜನ್ಯ
ಸೌಜನ್ಯಕ್ಕಾ-ದರೂ
ಸೌಜನ್ಯಕ್ಕಿಂತ
ಸೌಜನ್ಯದ
ಸೌಜನ್ಯ-ವನ್ನು
ಸೌಜನ್ಯವೇ
ಸೌಧ
ಸೌಧ-ಗ-ಳನ್ನು
ಸೌಧದ
ಸೌಧವನ್ನು
ಸೌಧ-ವೇ-ರು-ವ-ವನು
ಸೌಭಾಗ್ಯ
ಸೌಮಾವ್
ಸೌರಭ
ಸೌರಶಕ್ತಿ
ಸೌರ-ಶಕ್ತಿ-ಇ-ವು-ಗ-ಳನ್ನು
ಸೌಲಭ್ಯ-ಗ-ಳನ್ನು
ಸೌಲಭ್ಯ-ಗ-ಳನ್ನೂ
ಸೌಲಭ್ಯ-ಗ-ಳಿಂದ
ಸೌಲಭ್ಯ-ಗ-ಳಿ-ರುವ
ಸೌಷ್ಠವಕ್ಕೆ
ಸೌಹಾರ್ದ
ಸೌಹಾರ್ದ-ಗ-ಳನ್ನು
ಸೌಹಾರ್ದ-ತೆ-ಗಳ
ಸೌಹಾರ್ದ-ತೆ-ಗ-ಳನ್ನು
ಸೌಹಾರ್ದ-ಭಾ-ವನೆ
ಸೌಹಾರ್ದ-ವನ್ನು
ಸ್ಕೂಲ್
ಸ್ಟಾಲಿನ್
ಸ್ಟಿವನ್ಸನ್
ಸ್ಟೀನ್
ಸ್ಟೀಲಿನ
ಸ್ಟೀವನ್ಸನ್
ಸ್ಟೂಆರ್ಟ್
ಸ್ಟೂಯರ್ಟ್
ಸ್ಟೂವರ್ಟ್
ಸ್ಟೇಷನ್
ಸ್ಟೇಷನ್ನಲ್ಲಿ
ಸ್ಟೇಷನ್ನಿಂದ
ಸ್ಟೋನ್
ಸ್ಟ್ಯಾಂಪ್ರಹಿತ
ಸ್ಟ್ಯಾಲಿನ್ನಿಂದ
ಸ್ತಂಭವನ್ನು
ಸ್ತಂಭಿ-ತ-ನಾದ
ಸ್ತಂಭಿ-ತ-ರಾ-ದರು
ಸ್ತನಗಳ
ಸ್ತಬ್ಧ
ಸ್ತಬ್ಧ-ಗೊ-ಳಿ-ಸಿದೆ
ಸ್ತಬ್ಧ-ರಾ-ಗುತ್ತಾರೆ
ಸ್ತಬ್ಧ-ವಾ-ದಂತೆ
ಸ್ತಬ್ಧ-ವಾ-ದದ್ದಿ-ದೆಯೆ
ಸ್ತರಕ್ಕೆ
ಸ್ತರಗಳ
ಸ್ತರ-ಗ-ಳಲ್ಲ-ಡ-ಗಿ-ರುವ
ಸ್ತರ-ಗ-ಳಲ್ಲಿ
ಸ್ತರ-ಗ-ಳಲ್ಲಿ-ರುವ
ಸ್ತರ-ಗ-ಳಲ್ಲೂ
ಸ್ತರ-ಗ-ಳಿಂದ
ಸ್ತರ-ಗ-ಳಿಗೆ
ಸ್ತರ-ದಲ್ಲಷ್ಟೇ
ಸ್ತರದಲ್ಲಿ
ಸ್ತಿಮಿತ
ಸ್ತುತಿ
ಸ್ತುತಿಯನ್ನು
ಸ್ತುತಿ-ಸಿದ್ದಾರೆ
ಸ್ತುತಿಸು
ಸ್ತೋತ್ರ
ಸ್ತೋತ್ರಗಳ
ಸ್ತೋತ್ರಪಾಠ
ಸ್ತೋತ್ರ-ಪಾ-ರಾ-ಯಣ
ಸ್ತ್ರೀ
ಸ್ತ್ರೀಪುರುಷ
ಸ್ತ್ರೀಪು-ರು-ಷರ
ಸ್ತ್ರೀಪು-ರು-ಷ-ರನ್ನು
ಸ್ತ್ರೀಯ
ಸ್ತ್ರೀಯರ
ಸ್ತ್ರೀಯರನ್ನು
ಸ್ತ್ರೀಯರಲ್ಲಿ
ಸ್ತ್ರೀಯರಿಗೆ
ಸ್ತ್ರೀಯರು
ಸ್ತ್ರೀಸ್ವಾ-ತಂತ್ರ್ಯ-ಇದು
ಸ್ಥಗಿ-ತ-ಗೊ-ಳಿ-ಸಲು
ಸ್ಥಗಿ-ತ-ವಾ-ಗ-ದಿದ್ದರೂ
ಸ್ಥಗಿ-ತ-ವಾ-ಗುತ್ತವೆ
ಸ್ಥಗಿ-ತ-ವಾ-ಗು-ವುದು
ಸ್ಥಳ
ಸ್ಥಳಕ್ಕೆ
ಸ್ಥಳದ
ಸ್ಥಳದಿಂದ
ಸ್ಥಳವಿಲ್ಲ
ಸ್ಥಾನ
ಸ್ಥಾನಕ್ಕಾ-ಗಿಯೂ
ಸ್ಥಾನಕ್ಕೆ
ಸ್ಥಾನ-ಗ-ಳನ್ನು
ಸ್ಥಾನ-ಗ-ಳಲ್ಲಿ
ಸ್ಥಾನ-ಗ-ಳಿಗೂ
ಸ್ಥಾನ-ಗ-ಳಿಗೆ
ಸ್ಥಾನ-ಗ-ಳಿ-ಸಿ-ಕೊಂಡಿದ್ದೀರಿ
ಸ್ಥಾನಗಳೂ
ಸ್ಥಾನ-ಗ-ಳೆಲ್ಲ
ಸ್ಥಾನಚ್ಯು-ತಿಯ
ಸ್ಥಾನದ
ಸ್ಥಾನದಲ್ಲಿ
ಸ್ಥಾನ-ದಲ್ಲಿದ್ದ
ಸ್ಥಾನ-ದಲ್ಲಿದ್ದರೆ
ಸ್ಥಾನ-ದಲ್ಲಿದ್ದ-ವರೂ
ಸ್ಥಾನ-ದಲ್ಲಿದ್ದಾ-ನಲ್ಲ
ಸ್ಥಾನ-ದಲ್ಲಿ-ರಲಿ
ಸ್ಥಾನ-ದಲ್ಲಿ-ರಿಸಿ
ಸ್ಥಾನ-ದಲ್ಲಿ-ರುವ
ಸ್ಥಾನ-ದಲ್ಲಿ-ರು-ವ-ವರ
ಸ್ಥಾನ-ದಲ್ಲಿ-ರು-ವ-ವರು
ಸ್ಥಾನದಲ್ಲೇ
ಸ್ಥಾನದಿಂದ
ಸ್ಥಾನಮಾನ
ಸ್ಥಾನ-ಮಾ-ನ-ಇವು
ಸ್ಥಾನ-ಮಾ-ನ-ಗ-ಳನ್ನು
ಸ್ಥಾನ-ಮಾ-ನ-ಗ-ಳಲ್ಲಿ
ಸ್ಥಾನ-ಮಾ-ನ-ಗ-ಳಿಂದ
ಸ್ಥಾನವನ್ನು
ಸ್ಥಾನವನ್ನೇ
ಸ್ಥಾನವಾದ
ಸ್ಥಾನವಿತ್ತು
ಸ್ಥಾನ-ವಿ-ರದು
ಸ್ಥಾನ-ವಿಲ್ಲ-ದಿದ್ದರೆ
ಸ್ಥಾನ-ವೀ-ಯ-ದಿದ್ದ
ಸ್ಥಾನಾ-ಪನ್ನರೂ
ಸ್ಥಾಪ-ಕತ್ವ-ಇವು
ಸ್ಥಾಪಿ-ಸ-ಲಾ-ರನೇ
ಸ್ಥಾಪಿಸಲು
ಸ್ಥಾಪಿಸಿ
ಸ್ಥಾಪಿ-ಸಿ-ಕೊಳ್ಳುತ್ತಿದ್ದಾರೆ
ಸ್ಥಾಪಿ-ಸಿ-ಕೊಳ್ಳು-ವು-ದಕ್ಕೆ
ಸ್ಥಾಪಿಸಿದ
ಸ್ಥಾಪಿ-ಸಿ-ದರು
ಸ್ಥಾಪಿ-ಸಿ-ದರೆ
ಸ್ಥಾಯಿ-ಯಾ-ಗು-ವು-ವೆಂಬು-ದನ್ನು
ಸ್ಥಾಯಿತ್ವ
ಸ್ಥಾಯಿಯಲ್ಲ
ಸ್ಥಾಯಿಯಾಗಿ
ಸ್ಥಾಯಿ-ಯಾ-ಗಿದೆ
ಸ್ಥಾಯಿಯೋ
ಸ್ಥಾಯೀ
ಸ್ಥಿತಿ
ಸ್ಥಿತಿಗತಿ
ಸ್ಥಿತಿ-ಗ-ತಿ-ಗ-ಳನ್ನು
ಸ್ಥಿತಿ-ಗ-ತಿ-ಗ-ಳನ್ನೂ
ಸ್ಥಿತಿ-ಗ-ತಿಯ
ಸ್ಥಿತಿ-ಗ-ತಿ-ಯಲ್ಲಿ
ಸ್ಥಿತಿಗಳ
ಸ್ಥಿತಿ-ಗ-ಳನ್ನು
ಸ್ಥಿತಿ-ಗ-ಳಿಗೆ
ಸ್ಥಿತಿ-ಗಿ-ಳಿ-ದ-ವರೂ
ಸ್ಥಿತಿಗೂ
ಸ್ಥಿತಿಗೆ
ಸ್ಥಿತಿ-ಗೇ-ರಲು
ಸ್ಥಿತಿ-ಗೇ-ರಿ-ದಾಗ
ಸ್ಥಿತಿ-ಗೊಯ್ಯುವ
ಸ್ಥಿತಿಯ
ಸ್ಥಿತಿಯನ್ನು
ಸ್ಥಿತಿ-ಯನ್ನೇ-ರು-ವುದು
ಸ್ಥಿತಿಯಲ್ಲಿ
ಸ್ಥಿತಿ-ಯಲ್ಲಿದ್ದ
ಸ್ಥಿತಿ-ಯಲ್ಲಿದ್ದರೂ
ಸ್ಥಿತಿ-ಯಲ್ಲಿದ್ದಾಗ
ಸ್ಥಿತಿ-ಯಲ್ಲಿದ್ದಾನೆ
ಸ್ಥಿತಿ-ಯಲ್ಲಿ-ರಿ-ಸಲು
ಸ್ಥಿತಿ-ಯಲ್ಲಿ-ರುತ್ತಾರೆ
ಸ್ಥಿತಿ-ಯಲ್ಲಿ-ರುವ
ಸ್ಥಿತಿ-ಯಲ್ಲಿಲ್ಲ
ಸ್ಥಿತಿಯಲ್ಲೂ
ಸ್ಥಿತಿಯಲ್ಲೇ
ಸ್ಥಿತಿಯಿಂದ
ಸ್ಥಿತಿಯೂ
ಸ್ಥಿತಿ-ಯೊಂದಿಗೆ
ಸ್ಥಿತ್ಯಂತರ
ಸ್ಥಿರ-ಚಿತ್ತ-ರಾಗಿ
ಸ್ಥಿರತೆ
ಸ್ಥಿರವಾಗಿ
ಸ್ಥಿರ-ವಾ-ಗಿ-ರು-ವು-ದೆಂದಲ್ಲ
ಸ್ಥಿರವಾದ
ಸ್ಥಿರ-ವಾ-ದರೆ
ಸ್ಥಿರವೂ
ಸ್ಥಿರೀ-ಕೃ-ತ-ವಾ-ದುದು
ಸ್ಥೂಲ
ಸ್ಥೂಲದೇಹ
ಸ್ಥೂಲ-ದೇ-ಹ-ದಿಂದ
ಸ್ಥೂಲ-ದೇ-ಹವು
ಸ್ಥೂಲವಾಗಿ
ಸ್ಥೂಲ-ಸೂಕ್ಷ್ಮ-ಕಾ-ರ-ಣ-ಗ-ಳನ್ನು
ಸ್ಥೈರ್ಯ
ಸ್ಥೈರ್ಯ-ಇ-ವು-ಗಳ
ಸ್ಥೈರ್ಯ-ಗ-ಳನ್ನು
ಸ್ಥೈರ್ಯವನ್ನು
ಸ್ಥೈರ್ಯವನ್ನೂ
ಸ್ಥೈರ್ಯವೇ
ಸ್ನಾತಕರು
ಸ್ನಾನ
ಸ್ನಾನದ
ಸ್ನಾನ-ಮಾ-ಡುತ್ತಿದ್ದೆ
ಸ್ನಾನ-ಮಾ-ಡುತ್ತಿ-ರುವ
ಸ್ನಾನವು
ಸ್ನಾನ-ಶೀ-ಲ-ರಾ-ಗುತ್ತಾರೆ
ಸ್ನಾಯು
ಸ್ನಾಯುಗಳು
ಸ್ನೇಹ
ಸ್ನೇಹ-ಪ-ರತೆ
ಸ್ನೇಹ-ಪ-ರಾ-ಯ-ಣೆ-ಯಾದ
ಸ್ನೇಹ-ಯುಕ್ತ-ಬಾಳ್ವೆ
ಸ್ನೇಹವು
ಸ್ನೇಹಿತ
ಸ್ನೇಹಿತನ
ಸ್ನೇಹಿ-ತ-ನಂತೆ
ಸ್ನೇಹಿ-ತ-ನಿಗೆ
ಸ್ನೇಹಿತರ
ಸ್ನೇಹಿ-ತ-ರದ್ದಿ-ರಲಿ
ಸ್ನೇಹಿ-ತ-ರಲ್ಲಿ
ಸ್ನೇಹಿ-ತ-ರಾದ
ಸ್ನೇಹಿ-ತ-ರಾ-ದರೆ
ಸ್ನೇಹಿ-ತ-ರಿಗೆ
ಸ್ನೇಹಿತರು
ಸ್ನೇಹಿತರೂ
ಸ್ನೇಹಿತರೆ
ಸ್ನೇಹಿ-ತ-ರೆಲ್ಲ
ಸ್ನೇಹಿತರೇ
ಸ್ನೇಹಿ-ತ-ರೊ-ಡನೆ
ಸ್ನೇಹಿತೆಗೆ
ಸ್ನೇಹಿತೆಯು
ಸ್ಪಂದನ
ಸ್ಪಂದನಕ್ಕೆ
ಸ್ಪಂದನದ
ಸ್ಪಂದ-ನ-ದೊಂದಿಗೆ
ಸ್ಪಂದ-ನ-ವನ್ನು
ಸ್ಪಂದಿಸಿತ್ತು
ಸ್ಪಂದಿ-ಸುತ್ತದೆ
ಸ್ಪಂದಿ-ಸುತ್ತಿ-ದೆಯೇ
ಸ್ಪಂದಿಸುವ
ಸ್ಪರ್ಧಾ
ಸ್ಪರ್ಧಾಕ್ರ-ಮಕ್ಕಿಂತ
ಸ್ಪರ್ಧಾಕ್ರ-ಮದ
ಸ್ಪರ್ಧಾ-ಮ-ನೋ-ಭಾ-ವ-ಇ-ವು-ಗ-ಳಿಂದ
ಸ್ಪರ್ಧಾಮಯ
ಸ್ಪರ್ಧಿ-ಗ-ಳೆಂದು
ಸ್ಪರ್ಧಿಯಾಗಿ
ಸ್ಪರ್ಧಿಸಿ
ಸ್ಪರ್ಧೆ
ಸ್ಪರ್ಧೆ-ಗ-ಳಲ್ಲಿ
ಸ್ಪರ್ಧೆಯ
ಸ್ಪರ್ಧೆಯಲ್ಲಿ
ಸ್ಪರ್ಧೆ-ಯಾ-ಗಿ-ರ-ಲಿಲ್ಲ
ಸ್ಪರ್ಧೆಯಿಂದ
ಸ್ಪರ್ಧೆ-ಯಿಂದುಂಟಾ-ಗುವ
ಸ್ಪರ್ಧೆಯೂ
ಸ್ಪರ್ಧೆಯೇ
ಸ್ಪರ್ಧೆ-ಯೊಂದ-ರಲ್ಲಿ
ಸ್ಪರ್ಶ
ಸ್ಪರ್ಶದಿಂದ
ಸ್ಪರ್ಶ-ದಿಂದಲೇ
ಸ್ಪರ್ಶ-ಮಾತ್ರ-ದಿಂದ
ಸ್ಪರ್ಶಿಸಿ
ಸ್ಪರ್ಶಿ-ಸಿ-ದರು
ಸ್ಪರ್ಶಿ-ಸಿ-ದವು
ಸ್ಪರ್ಶಿಸಿಯೇ
ಸ್ಪರ್ಶಿ-ಸುತ್ತದೆ
ಸ್ಪಲ್ಪ
ಸ್ಪಷ್ಟ
ಸ್ಪಷ್ಟ-ವಾ-ಗಿ-ರ-ಬ-ಹು-ದಾದ
ಸ್ಪಷ್ಟ-ಪ-ಡಿ-ಸ-ಬೇಕು
ಸ್ಪಷ್ಟ-ಪ-ಡಿ-ಸ-ಲಾ-ಗಿದೆ
ಸ್ಪಷ್ಟ-ಪ-ಡಿ-ಸಿ-ಕೊಳ್ಳ-ಬೇಕು
ಸ್ಪಷ್ಟ-ಪ-ಡಿ-ಸಿದ್ದರು
ಸ್ಪಷ್ಟ-ಪ-ಡಿ-ಸುತ್ತವೆ
ಸ್ಪಷ್ಟ-ವಲ್ಲವೆ
ಸ್ಪಷ್ಟ-ವಾ-ಗ-ದಿ-ರದು
ಸ್ಪಷ್ಟ-ವಾ-ಗ-ಬ-ಹುದು
ಸ್ಪಷ್ಟವಾಗಿ
ಸ್ಪಷ್ಟ-ವಾ-ಗಿಯೂ
ಸ್ಪಷ್ಟ-ವಾ-ಗಿಯೇ
ಸ್ಪಷ್ಟ-ವಾ-ಗಿ-ರ-ಬೇಕು
ಸ್ಪಷ್ಟ-ವಾ-ಗಿ-ರುವ
ಸ್ಪಷ್ಟ-ವಾ-ಗುತ್ತದೆ
ಸ್ಪಷ್ಟ-ವಾ-ಗುತ್ತ-ಲಿದೆ
ಸ್ಪಷ್ಟ-ವಾ-ಗುತ್ತಾ
ಸ್ಪಷ್ಟ-ವಾ-ಗು-ವುದು
ಸ್ಪಷ್ಟವಾದ
ಸ್ಪಷ್ಟ-ವಾ-ಯಿತು
ಸ್ಪಷ್ಟವೂ
ಸ್ಪಷ್ಟೀ-ಕ-ರಿ-ಸ-ಲಾ-ಗಿ-ದೆ-ಗ-ಮ-ನಿಸಿ
ಸ್ಪಾಂಜನ್ನು
ಸ್ಪಾಂಜನ್ನೂ
ಸ್ಪಾಂಜಿನ
ಸ್ಪಾಂಟೇ-ನಿ-ಯಸ್
ಸ್ಪಿಟ್ಸ್
ಸ್ಪೀಟ್ಸ್
ಸ್ಪೃಹಾ
ಸ್ಪೆಯಿನಿನ
ಸ್ಪೇನಿಷ್
ಸ್ಫುಟವಾಗಿ
ಸ್ಫೂರ್ತಿ
ಸ್ಫೂರ್ತಿ-ದಾ-ಯಕ
ಸ್ಫೂರ್ತಿ-ಗ-ಳನ್ನು
ಸ್ಫೂರ್ತಿ-ಗ-ಳಿಗೆ
ಸ್ಫೂರ್ತಿಗಳು
ಸ್ಫೂರ್ತಿಗಾಗಿ
ಸ್ಫೂರ್ತಿ-ದಾ-ಯ-ಕ-ವಾ-ಗಿತ್ತು
ಸ್ಫೂರ್ತಿ-ಪ-ಡೆದ
ಸ್ಫೂರ್ತಿಯ
ಸ್ಫೂರ್ತಿಯನ್ನು
ಸ್ಫೂರ್ತಿಯನ್ನೂ
ಸ್ಫೂರ್ತಿಯಾಗಿ
ಸ್ಫೂರ್ತಿಯಿಂದ
ಸ್ಫೂರ್ತಿಯೇ
ಸ್ಫೋಟಗೊಂಡು
ಸ್ಫೋಟದ
ಸ್ಫೋಟ-ವಾ-ಗದೆ
ಸ್ಫೋಟ-ವಾ-ಗದೇ
ಸ್ಮರಣೀಯ
ಸ್ಮರಣೆ
ಸ್ಮರ-ಣೆ-ಗೈಯ್ಯುತ್ತಾ
ಸ್ಮರಣೆಯ
ಸ್ಮರ-ಣೆ-ಯನ್ನು
ಸ್ಮರ-ಣೆ-ಯಲ್ಲದೇ
ಸ್ಮರಣೆಯು
ಸ್ಮರ-ಣೆ-ಯುಳ್ಳ
ಸ್ಮರಿ-ಸ-ಬ-ಹುದು
ಸ್ಮರಿಸಿ
ಸ್ಮರಿ-ಸಿ-ಕೊಂಡರೆ
ಸ್ಮರಿ-ಸಿ-ಕೊಂಡು
ಸ್ಮರಿ-ಸಿ-ಕೊಳ್ಳ-ಬ-ಹುದು
ಸ್ಮರಿ-ಸಿ-ಕೊಳ್ಳು-ವುದು
ಸ್ಮರಿ-ಸಿ-ದಾ-ಗ-ಲೆಲ್ಲ
ಸ್ಮರಿಸು
ಸ್ಮರಿಸುತ್ತ
ಸ್ಮರಿ-ಸುತ್ತಾರೆ
ಸ್ಮರಿ-ಸು-ವರು
ಸ್ಮಶಾನಕ್ಕೆ
ಸ್ಮಶಾನದ
ಸ್ಮಶಾ-ನ-ದೊ-ಳಗೆ
ಸ್ಮಶಾ-ನ-ವನ್ನು
ಸ್ಮಾಯ್ಲ್ಸ್
ಸ್ಮಾೖಲ್ಸ್
ಸ್ಮೃತಿ
ಸ್ಮೃತಿ-ಗ-ಳನ್ನು
ಸ್ಮೃತಿ-ಗ-ಳನ್ನೂ
ಸ್ಮೃತಿಗಳು
ಸ್ಮೃತಿಚಿತ್ರ
ಸ್ಮೃತಿ-ಚಿತ್ರವು
ಸ್ಮೃತಿಭ್ರಂಶಾದ್ಬುದ್ಧಿ-ನಾಶೋ
ಸ್ಮೃತಿಯ
ಸ್ಮೃತಿಯನ್ನು
ಸ್ಮೃತಿಯನ್ನೂ
ಸ್ಮೃತಿಯಲ್ಲಿ
ಸ್ಮೃತಿಯೇ
ಸ್ಮೃತಿ-ಯೊಂದಿಗೆ
ಸ್ಮೃತಿ-ಶಕ್ತಿ-ಗಳು
ಸ್ಮೃತಿ-ಶಕ್ತಿಯ
ಸ್ಮೃತಿ-ಶಕ್ತಿಯು
ಸ್ಮೃತಿ-ಸಂಕೇ-ತ-ಗಳು
ಸ್ಮೈಲಿ
ಸ್ಯಾನ್
ಸ್ಯಾಮ್ಯುಯೆಲ್
ಸ್ರುಜ್
ಸ್ಲೀಮ್ನ
ಸ್ಲೋನ್
ಸ್ವ
ಸ್ವಂತ
ಸ್ವಂತಿಕೆ
ಸ್ವಕರ್ಮಣಾ
ಸ್ವಚ್ಛ
ಸ್ವಚ್ಛಂದ
ಸ್ವಚ್ಛಂದ-ವರ್ತನೆ
ಸ್ವಚ್ಛವಾಗಿ
ಸ್ವಜ-ನಾ-ಭಿ-ಮಾನ
ಸ್ವತಂತ್ರ
ಸ್ವತಂತ್ರ-ರಾ-ಗುತ್ತೇವೆ
ಸ್ವತಂತ್ರ-ರಾ-ಗುತ್ತೇವೆ
ಸ್ವತಂತ್ರ-ವಾಗಿ
ಸ್ವತಂತ್ರ-ವಾ-ಗಿಯೂ
ಸ್ವತಂತ್ರ-ವಾದ
ಸ್ವತಃ
ಸ್ವತ್ತಾಗಿದ್ದ
ಸ್ವತ್ತು
ಸ್ವತ್ವ
ಸ್ವತ್ವವನ್ನೂ
ಸ್ವದೇ-ಶಪ್ರೇ-ಮ-ದಿಂದ
ಸ್ವದೇ-ಶಾ-ಭಿ-ಮಾ-ನ-ಗ-ಳನ್ನುಂಟು
ಸ್ವದೇ-ಶಾ-ಭಿ-ಮಾ-ನ-ಗ-ಳನ್ನುಂಟು-ಮಾ-ಡುವ
ಸ್ವದೇಶೀ
ಸ್ವದೇ-ಶೀ-ಯರೇ
ಸ್ವದೋಷ
ಸ್ವನಾಶದ
ಸ್ವನಿ-ಯಂತ್ರಣ
ಸ್ವನಿರ್ದೇ-ಶನಾ
ಸ್ವಪ್ನ
ಸ್ವಪ್ನ-ಗ-ಳಲ್ಲಿ
ಸ್ವಪ್ನಗಳು
ಸ್ವಪ್ನತನ್ನ
ಸ್ವಪ್ನದ
ಸ್ವಪ್ನದಲ್ಲಿ
ಸ್ವಪ್ನದಲ್ಲೂ
ಸ್ವಪ್ನವನ್ನು
ಸ್ವಪ್ನವೆಂದು
ಸ್ವಪ್ನಾ-ವಸ್ಥೆ-ಯಲ್ಲಿ
ಸ್ವಪ್ರಯತ್ನ
ಸ್ವಪ್ರ-ಯತ್ನದ
ಸ್ವಪ್ರ-ಯತ್ನ-ದಲ್ಲಿ
ಸ್ವಪ್ರ-ಯತ್ನ-ದಿಂದ
ಸ್ವಪ್ರ-ಯತ್ನ-ದಿಂದಲೇ
ಸ್ವಪ್ರ-ಯತ್ನ-ವನ್ನು
ಸ್ವಪ್ರ-ಯತ್ನ-ವನ್ನೂ
ಸ್ವಭಾವ
ಸ್ವಭಾವಕ್ಕೆ
ಸ್ವಭಾ-ವ-ಗಳ
ಸ್ವಭಾ-ವ-ಗ-ಳನ್ನು
ಸ್ವಭಾ-ವ-ಗ-ಳಲ್ಲಿ
ಸ್ವಭಾವತಃ
ಸ್ವಭಾವದ
ಸ್ವಭಾ-ವ-ದ-ವನೂ
ಸ್ವಭಾ-ವ-ದ-ವನೆ
ಸ್ವಭಾ-ವ-ದ-ವರು
ಸ್ವಭಾ-ವ-ವನ್ನಾಗಿ
ಸ್ವಭಾ-ವ-ವನ್ನು
ಸ್ವಭಾ-ವ-ವನ್ನೂ
ಸ್ವಭಾ-ವ-ವಾ-ಗಿತ್ತು
ಸ್ವಭಾ-ವ-ವುಳ್ಳ-ವರು
ಸ್ವಭಾ-ವ-ವೆಂದು
ಸ್ವಭಾವವೇ
ಸ್ವಭಾವವೋ
ಸ್ವಭಾ-ವ-ಸಿದ್ಧ-ವಾದ
ಸ್ವಭಾ-ವಿ-ಗಳೂ
ಸ್ವಮತ
ಸ್ವಮ-ತಾ-ಭಿ-ಮಾನ
ಸ್ವಮಿ
ಸ್ವಯಂ
ಸ್ವಯಂಚಾ-ಲಿತ
ಸ್ವಯಂವೇದ್ಯ-ವಾ-ಗುತ್ತದೆ
ಸ್ವರ
ಸ್ವರದಲ್ಲಿ
ಸ್ವರಾಜ್ಯ
ಸ್ವರೂಪ
ಸ್ವರೂ-ಪ-ಇ-ವೆಲ್ಲ-ವನ್ನೂ
ಸ್ವರೂ-ಪ-ಗ-ಳನ್ನು
ಸ್ವರೂಪತಃ
ಸ್ವರೂಪದ
ಸ್ವರೂ-ಪ-ದಿಂದ
ಸ್ವರೂ-ಪ-ದೊಂದಿಗೆ
ಸ್ವರೂ-ಪ-ವನ್ನು
ಸ್ವರೂ-ಪ-ವನ್ನೂ
ಸ್ವರೂ-ಪ-ವಲ್ಲ
ಸ್ವರೂ-ಪ-ವಾದ
ಸ್ವರೂಪವೂ
ಸ್ವರೂ-ಪ-ವೇನು
ಸ್ವರ್ಗ
ಸ್ವರ್ಗಕ್ಕೆ
ಸ್ವರ್ಗಕ್ಕೇರಿ
ಸ್ವರ್ಗದ
ಸ್ವರ್ಗವನ್ನು
ಸ್ವರ್ಗ-ವೇ-ರ-ಲಾರೆ
ಸ್ವಲವೂ
ಸ್ವಲಾಭ
ಸ್ವಲಾ-ಭ-ಚಿಂತನೆ
ಸ್ವಲ್ಪ
ಸ್ವಲ್ಪಕಾಲ
ಸ್ವಲ್ಪ-ಕಾ-ಲ-ದಲ್ಲೇ
ಸ್ವಲ್ಪ-ಮಟ್ಟಿ-ಗಾ-ದರೂ
ಸ್ವಲ್ಪ-ಮಟ್ಟಿಗೆ
ಸ್ವಲ್ಪ-ಮಟ್ಟಿನ
ಸ್ವಲ್ಪ-ವನ್ನಾ-ದರೂ
ಸ್ವಲ್ಪ-ವಾ-ದರೂ
ಸ್ವಲ್ಪವೂ
ಸ್ವಲ್ಪಹೊತ್ತು
ಸ್ವವಿ-ಮರ್ಶೆಯ
ಸ್ವವ್ಯಕ್ತಿತ್ವ
ಸ್ವವ್ಯಕ್ತಿತ್ವ-ಚಿತ್ರ
ಸ್ವವ್ಯಕ್ತಿತ್ವ-ಚಿತ್ರಕ್ಕ-ನು-ಗು-ಣ-ವಾಗಿ
ಸ್ವವ್ಯಕ್ತಿತ್ವದ
ಸ್ವಶಕ್ತಿ-ಯಿಂದ
ಸ್ವಸಹಾಯ
ಸ್ವಸ್ಥ
ಸ್ವಸ್ಥನಲ್ಲ
ಸ್ವಸ್ಥ-ನಾ-ಗು-ವು-ದಕ್ಕಾಗಿ
ಸ್ವಸ್ಥನಾದ
ಸ್ವಸ್ಥ-ರಾ-ಗು-ವು-ದಕ್ಕಾ-ಗಿದೆ
ಸ್ವಹಿತ
ಸ್ವಾಗತ
ಸ್ವಾಗತಕ್ಕೆ
ಸ್ವಾಗ-ತಿ-ಸ-ದಿದ್ದರೂ
ಸ್ವಾಗತಿಸಿ
ಸ್ವಾಗ-ತಿ-ಸಿದ
ಸ್ವಾತಂತ್ರ್ಯ
ಸ್ವಾತಂತ್ರ್ಯಕ್ಕಾಗಿ
ಸ್ವಾತಂತ್ರ್ಯಕ್ಕೆ
ಸ್ವಾತಂತ್ರ್ಯ-ಗ-ಳನ್ನೂ
ಸ್ವಾತಂತ್ರ್ಯದ
ಸ್ವಾತಂತ್ರ್ಯ-ಪೂರ್ವ-ದಲ್ಲಿ
ಸ್ವಾತಂತ್ರ್ಯಪ್ರಿ-ಯತೆ
ಸ್ವಾತಂತ್ರ್ಯಪ್ರಿ-ಯ-ತೆಯು
ಸ್ವಾತಂತ್ರ್ಯ-ವನ್ನು
ಸ್ವಾತಂತ್ರ್ಯ-ವಿದೆ
ಸ್ವಾತಂತ್ರ್ಯ-ವಿ-ದೆಯೇ
ಸ್ವಾತಂತ್ರ್ಯ-ವಿಲ್ಲ-ವೆಂದಲ್ಲ
ಸ್ವಾತಂತ್ರ್ಯವು
ಸ್ವಾತಂತ್ರ್ಯವೂ
ಸ್ವಾತಂತ್ರ್ಯ-ವೆಂದರೆ
ಸ್ವಾತಂತ್ರ್ಯಾ-ನಂತರ
ಸ್ವಾತಂತ್ರ್ಯಾ-ನಂತ-ರದ
ಸ್ವಾಧೀ-ನ-ವಿಲ್ಲ-ವೆಂಬು-ದನ್ನು
ಸ್ವಾಧ್ಯಾಯ
ಸ್ವಾನು
ಸ್ವಾನುಭವ
ಸ್ವಾನು-ಭ-ವ-ದಿಂದ
ಸ್ವಾನು-ಭ-ವ-ವನ್ನೂ
ಸ್ವಾಭಾವಿಕ
ಸ್ವಾಭಾವಿಕ
ಸ್ವಾಭಾ-ವಿ-ಕತೆ
ಸ್ವಾಭಾ-ವಿ-ಕ-ವಲ್ಲವೇ
ಸ್ವಾಭಾ-ವಿ-ಕ-ವಷ್ಟೆ
ಸ್ವಾಭಾ-ವಿ-ಕ-ವಾಗಿ
ಸ್ವಾಭಾ-ವಿ-ಕ-ವಾ-ಗು-ವುದೋ
ಸ್ವಾಭಾ-ವಿ-ಕ-ವಾದ
ಸ್ವಾಭಾ-ವಿ-ಕ-ವಾ-ದು-ದನ್ನು
ಸ್ವಾಭಾ-ವಿ-ಕವೇ
ಸ್ವಾಭಿಮಾನ
ಸ್ವಾಭಿ-ಮಾ-ನವೂ
ಸ್ವಾಮಿ
ಸ್ವಾಮಿ-ಗ-ಳೊಬ್ಬರು
ಸ್ವಾಮಿ-ಗ-ಳೊಬ್ಬ-ರಿಗೆ
ಸ್ವಾಮಿಗೆ
ಸ್ವಾಮಿತ್ವ-ವಿತ್ತೆನ್ನು-ವು-ದಕ್ಕೆ
ಸ್ವಾಮಿಯ
ಸ್ವಾಮೀಜಿ
ಸ್ವಾಮೀ-ಜಿ-ಯ-ವರ
ಸ್ವಾಮೀ-ಜಿ-ಯ-ವ-ರನ್ನು
ಸ್ವಾಮೀ-ಜಿ-ಯ-ವರು
ಸ್ವಾಮೀ-ಜಿ-ಯೊ-ಡ-ನೆ-ನಿ-ಮಗೆ
ಸ್ವಾಮೀ-ಜೀ-ಯ-ವ-ರನ್ನು
ಸ್ವಾಮೀ-ಜೀ-ಯೊಬ್ಬರ
ಸ್ವಾರ್ಗಿಕ
ಸ್ವಾರ್ಥ
ಸ್ವಾರ್ಥ-ಮು-ಖದ
ಸ್ವಾರ್ಥಕ್ಕಾಗಿ
ಸ್ವಾರ್ಥಕ್ಕೆ
ಸ್ವಾರ್ಥಕ್ಕೇ
ಸ್ವಾರ್ಥಗಳೇ
ಸ್ವಾರ್ಥತೆ
ಸ್ವಾರ್ಥ-ತೆ-ಯಿಂದ
ಸ್ವಾರ್ಥತ್ಯಾಗ
ಸ್ವಾರ್ಥತ್ಯಾಗಿ
ಸ್ವಾರ್ಥದ
ಸ್ವಾರ್ಥದಿಂದ
ಸ್ವಾರ್ಥಪರ
ಸ್ವಾರ್ಥ-ಪ-ರತೆ
ಸ್ವಾರ್ಥ-ಪ-ರಾ-ಯಣ
ಸ್ವಾರ್ಥ-ಪ-ರಾ-ಯ-ಣತೆ
ಸ್ವಾರ್ಥ-ಪ-ರಾ-ಯ-ಣ-ರಾ-ಗುತ್ತಿದ್ದಾರೆ
ಸ್ವಾರ್ಥ-ರ-ಹಿತ
ಸ್ವಾರ್ಥವನ್ನು
ಸ್ವಾರ್ಥವೂ
ಸ್ವಾರ್ಥವೇ
ಸ್ವಾರ್ಥ-ಸಾ-ಧ-ನೆ-ಗಾಗಿ
ಸ್ವಾರ್ಥ-ಸಾ-ಧ-ನೆಗೆ
ಸ್ವಾರ್ಥ-ಸಾ-ಧ-ನೆ-ಯಾ-ಗುವ
ಸ್ವಾರ್ಥ-ಸು-ಖವೇ
ಸ್ವಾರ್ಥಾ-ಭಿ-ಲಾ-ಷೆ-ಯಿಲ್ಲದೆ
ಸ್ವಾರ್ಥಿ
ಸ್ವಾರ್ಥಿಗಳ
ಸ್ವಾರ್ಥಿಗಿಂತ
ಸ್ವಾರ್ಥಿ-ಯಾ-ಗುತ್ತಿದ್ದೀಯೆ
ಸ್ವಾರ್ಥಿ-ಯಾ-ದಂತೆ
ಸ್ವಾರ್ಥಿ-ಯಾ-ದರೆ
ಸ್ವಾವ-ಲಂಬನೆ
ಸ್ವಾವಲಂಬಿ
ಸ್ವಾವ-ಲಂಬಿ-ಯಾ-ಗುತ್ತಿದ್ದಾ-ನೆಯೆ
ಸ್ವಾಸ್ಥ್ಯ
ಸ್ವಾಸ್ಥ್ಯ-ರಕ್ಷಣೆ
ಸ್ವಾಸ್ಥ್ಯವನ್ನು
ಸ್ವಾಸ್ಥ್ಯವನ್ನೂ
ಸ್ವಿಸ್
ಸ್ವೀಕರಿ
ಸ್ವೀಕ-ರಿ-ಸದೇ
ಸ್ವೀಕ-ರಿ-ಸಲು
ಸ್ವೀಕರಿಸಿ
ಸ್ವೀಕ-ರಿ-ಸಿತು
ಸ್ವೀಕ-ರಿ-ಸಿತ್ತು
ಸ್ವೀಕ-ರಿ-ಸಿ-ದರೆ
ಸ್ವೀಕ-ರಿ-ಸಿ-ದ-ವನೇ
ಸ್ವೀಕ-ರಿ-ಸಿದ್ದ-ರಿಂದ
ಸ್ವೀಕ-ರಿ-ಸಿದ್ದರು
ಸ್ವೀಕ-ರಿ-ಸುತ್ತದೆ
ಸ್ವೀಕ-ರಿ-ಸುತ್ತಿದ್ದಾಗ
ಸ್ವೀಕ-ರಿ-ಸುತ್ತಿದ್ದೀರಿ
ಸ್ವೀಕ-ರಿ-ಸುತ್ತಿ-ರುತ್ತಾರೆ
ಸ್ವೀಕ-ರಿ-ಸುತ್ತೀರಿ
ಸ್ವೀಕ-ರಿ-ಸುತ್ತೇ-ವಷ್ಟೆ
ಸ್ವೀಕ-ರಿ-ಸು-ವ-ರೆಂದು
ಸ್ವೀಕ-ರಿ-ಸು-ವು-ದ-ರಿಂದಷ್ಟೇ
ಸ್ವೀಕ-ರಿ-ಸು-ವು-ದಿಲ್ಲ
ಸ್ವೀಕ-ರಿ-ಸು-ವುದು
ಸ್ವೀಕ-ರಿ-ಸು-ವುದೋ
ಸ್ವೀಕ-ರಿ-ಸು-ವುವು
ಸ್ವೀಕಾರ
ಸ್ವೀಕಾ-ರಕ್ಕೆಂದು
ಸ್ವೀಕೃ-ತ-ವಾಗಿ
ಸ್ವೀಕೃ-ತ-ವಾದ
ಸ್ವೀವನ್
ಸ್ವೀವನ್ಸನ್
ಸ್ವೇಚ್ಛಾ
ಸ್ವೇಚ್ಛಾಚಾರ
ಸ್ವೇಚ್ಛಾ-ಚಾ-ರಕ್ಕೆ-ಳ-ಸಿ-ತು-ಮುಖ್ಯ-ವಾಗಿ
ಸ್ವೇಚ್ಛಾ-ಚಾ-ರಕ್ಕೆ-ಳ-ಸಿದ
ಸ್ವೇಚ್ಛಾ-ಚಾ-ರಕ್ಕೆ-ಳೆ-ಸಿದೆ
ಸ್ವೇಚ್ಛಾ-ಚಾ-ರದ
ಸ್ವೇಚ್ಛಾ-ಚಾ-ರ-ದಿಂದ
ಸ್ವೇಟ್ಸ್
ಸ್ವೈರತೆ
ಸ್ವೈರತೆಯ
ಸ್ವೈರವೃತ್ತಿ
ಸ್ವೈರ-ವೃತ್ತಿ-ಗಳ
ಸ್ವೈರ-ವೃತ್ತಿಯೇ
ಸ್ವೈರಾಚಾರಿ
ಹಂಗಿಸಿದ
ಹಂಗಿಸುವ
ಹಂಗಿ-ಸು-ವುದು
ಹಂಚಬಲ್ಲ
ಹಂಚಬೇಕು
ಹಂಚಲೂ
ಹಂಚಿ
ಹಂಚಿಕೊಂಡು
ಹಂಚಿ-ಕೊಳ್ಳು-ವಿಕೆ
ಹಂಚಿ-ಹೋ-ಗಿದ್ದರು
ಹಂಚುತ್ತಿದ್ದ
ಹಂಚು-ವ-ವರು
ಹಂಟರ್
ಹಂತ
ಹಂತಕ್ಕೆ
ಹಂತ-ಗ-ಳಲ್ಲಿ
ಹಂತ-ಗ-ಳಲ್ಲಿ-ರು-ವ-ವ-ರಿಗೆ
ಹಂತದಲ್ಲಿ
ಹಂತ-ದಲ್ಲಿದ್ದರೂ
ಹಂತ-ದಲ್ಲಿದ್ದು-ಕೊಂಡಿ-ರುವ
ಹಂತ-ದಲ್ಲಿ-ರು-ವ-ವರು
ಹಂತದಲ್ಲೇ
ಹಂತ-ದಿಂದಲೇ
ಹಂತವನ್ನು
ಹಂತವಾಗಿ
ಹಂತ-ಹಂತ-ವಾಗಿ
ಹಂದಿ
ಹಂದಿ-ಗ-ಳನ್ನು
ಹಂದಿ-ಗ-ಳಿಗೆ
ಹಂಪಲು
ಹಂಬ
ಹಂಬಲ
ಹಂಬ-ಲ-ಇ-ವು-ಗ-ಳನ್ನುಂಟು
ಹಂಬಲಕ್ಕೆ
ಹಂಬ-ಲಕ್ಕೊಂದು
ಹಂಬ-ಲ-ಗ-ಳಿಲ್ಲದ
ಹಂಬ-ಲ-ಗಳು
ಹಂಬಲದ
ಹಂಬ-ಲ-ದಿಂದ
ಹಂಬ-ಲ-ದಿಂದಲೇ
ಹಂಬ-ಲ-ದಿಂದೇನೂ
ಹಂಬ-ಲ-ದೊಂದಿಗೆ
ಹಂಬ-ಲ-ವನ್ನಾ-ದರೂ
ಹಂಬ-ಲ-ವನ್ನು
ಹಂಬ-ಲ-ವನ್ನುಂಟು-ಮಾ-ಡುವ
ಹಂಬ-ಲ-ವನ್ನೂ
ಹಂಬ-ಲ-ವಿತ್ತು
ಹಂಬ-ಲ-ವಿದೆ
ಹಂಬ-ಲ-ವಿ-ರಿ-ಸಿ-ಕೊಂಡಿದ್ದ
ಹಂಬ-ಲ-ವಿ-ರುವ
ಹಂಬ-ಲ-ವಿ-ರು-ವ-ವ-ರಿಗೆ
ಹಂಬ-ಲ-ವಿಲ್ಲದ
ಹಂಬ-ಲ-ವಿಲ್ಲದೇ
ಹಂಬ-ಲ-ವುಂಟಾ-ಗು-ವುದು
ಹಂಬಲವೂ
ಹಂಬ-ಲ-ವೆಂದಾಗ
ಹಂಬಲವೇ
ಹಂಬ-ಲ-ವೇನೋ
ಹಂಬ-ಲ-ವೊಂದೇ
ಹಂಬ-ಲಿ-ಸದೇ
ಹಂಬ-ಲಿ-ಸ-ಲಾರ
ಹಂಬ-ಲಿ-ಸಿದ್ದ
ಹಂಬ-ಲಿ-ಸುತ್ತಾ
ಹಂಬ-ಲಿ-ಸುತ್ತೇವೆ
ಹಂಬ-ಲಿ-ಸುವ
ಹಕ್ಕನ್ನು
ಹಕ್ಕನ್ನೂ
ಹಕ್ಕಿ-ಗ-ಳನ್ನು
ಹಕ್ಕಿದೆಯೇ
ಹಕ್ಕಿನ
ಹಕ್ಕಿಯೂ
ಹಕ್ಕು
ಹಕ್ಕು-ಗ-ಳಿ-ಗಾಗಿ
ಹಕ್ಕುಗಳೂ
ಹಕ್ಸಲೀಯ
ಹಕ್ಸ್ಲೀ
ಹಗ-ಲಿ-ನಲ್ಲೇ
ಹಗ-ಲಿ-ರುಳು
ಹಗ-ಲಿ-ರುಳೂ
ಹಗಲು
ಹಗ-ಲು-ಗ-ನ-ಸು-ಗಳು
ಹಗ-ಲು-ರಾತ್ರಿ
ಹಗ-ಲೂ-ರಾತ್ರಿ
ಹಗ್ಗ
ಹಗ್ಗದಿಂದ
ಹಚ್ಚಿ
ಹಚ್ಚಿ-ಕೊಳ್ಳುವ
ಹಚ್ಚುವ
ಹಟ
ಹಟತೊಟ್ಟು
ಹಟವೂ
ಹಠ
ಹಠ-ಗ-ಳನ್ನು
ಹಠ-ಮಾ-ರಿ-ತನ
ಹಠ-ಯೋ-ಗದ
ಹಠ-ಯೋ-ಗಿಯ
ಹಠ-ಯೋ-ಗಿ-ಯೊಬ್ಬ
ಹಠ-ಹಿ-ಡಿದು
ಹಠಾತ್
ಹಠಾತ್ತಾದ
ಹಡಗನ್ನು
ಹಡಗಿನ
ಹಡ-ಗಿ-ನಲ್ಲಿ
ಹಡ-ಗಿ-ನಲ್ಲಿದ್ದ
ಹಡಗು
ಹಡ-ಗು-ಗಳ
ಹಡಗೊಂದು
ಹಣ
ಹಣಕ್ಕಾಗಿ
ಹಣಕ್ಕೆ
ಹಣ-ಗ-ಳಿ-ಸಲು
ಹಣ-ಗ-ಳಿ-ಸುವ
ಹಣದ
ಹಣದಿಂದ
ಹಣವನ್ನು
ಹಣವನ್ನೂ
ಹಣವು
ಹಣ-ಸಂಗ್ರ-ಹವೇ
ಹಣಹಾಕಿ
ಹಣೆಯ
ಹಣ್ಣನ್ನು
ಹಣ್ಣಾದಾಗ
ಹಣ್ಣಿಗಾಗಿ
ಹಣ್ಣಿನ
ಹಣ್ಣು
ಹಣ್ಣು-ಗ-ಳನ್ನು
ಹಣ್ಣುಗಳು
ಹಣ್ಣು-ಹಂಪ-ಲು-ಇ-ವನ್ನು
ಹತ-ಭಾ-ಗಿ-ನಿಗೆ
ಹತ-ರಾ-ದ-ವರು
ಹತಾಶ
ಹತಾ-ಶ-ನಾಗಿ
ಹತಾ-ಶ-ರನ್ನಾಗಿ
ಹತಾ-ಶ-ರಾ-ಗದೆ
ಹತಾ-ಶ-ರಾಗಿ
ಹತಾ-ಶ-ರಾ-ಗುತ್ತಾರೆ
ಹತಾ-ಶ-ರಾದ
ಹತಾ-ಶ-ಳಾ-ಗಿದ್ದೇನೆ
ಹತಾಶೆ
ಹತಾ-ಶೆ-ಯಿಂದ
ಹತಾ-ಶೆ-ಯಿಂದಲೇ
ಹತೋ-ಟಿ-ಯಲ್ಲಿ
ಹತ್ತನೇ
ಹತ್ತರಲ್ಲಿ
ಹತ್ತರಷ್ಟು
ಹತ್ತರಿಂದ
ಹತ್ತಲ್ಲ
ಹತ್ತಾ-ಗಿದ್ದರೂ
ಹತ್ತಾರು
ಹತ್ತಿ
ಹತ್ತಿಕ್ಕಲು
ಹತ್ತಿಕ್ಕಿ-ದಂತಾ-ಗುತ್ತದೆ
ಹತ್ತಿದ
ಹತ್ತಿರ
ಹತ್ತಿರಕ್ಕೆ
ಹತ್ತಿರದ
ಹತ್ತಿ-ರ-ದಲ್ಲೇ
ಹತ್ತಿ-ರ-ದಿಂದ
ಹತ್ತಿ-ರ-ಬಿಟ್ಟಾಗ
ಹತ್ತಿ-ರ-ವಾ-ಗಿ-ರುತ್ತದೆ
ಹತ್ತಿ-ರ-ವಿಟ್ಟು-ಕೊಳ್ಳ-ಬ-ಹುದು
ಹತ್ತಿ-ರ-ವಿದ್ದ
ಹತ್ತಿರವೂ
ಹತ್ತಿರವೇ
ಹತ್ತು
ಹತ್ತುಪಾಲು
ಹತ್ತುಲಕ್ಷ
ಹತ್ತು-ವರ್ಷ-ಗಳ
ಹತ್ತು-ವರ್ಷ-ಗ-ಳಲ್ಲಿ
ಹತ್ತು-ಸಾ-ವಿರ
ಹತ್ತೂವರೆ
ಹತ್ತೂ-ವ-ರೆಗೆ
ಹತ್ತೂ-ವ-ರೆ-ಯ-ವ-ರೆಗೂ
ಹತ್ತೊಂಬತ್ತನೆ
ಹತ್ತೊಂಬತ್ತನೇ
ಹತ್ತೊಂಬತ್ತು
ಹತ್ತೊಂಭತ್ತನೇ
ಹತ್ಯಾ-ಕಾಂಡದ
ಹತ್ಯಾ-ಕಾಂಡ-ಹಿಂಸೆ
ಹತ್ಯೆ
ಹತ್ಯೆಯ
ಹತ್ಯೆಯನ್ನೂ
ಹದ-ಗೆಟ್ಟಿದೆ
ಹದ-ಗೆ-ಡಿ-ಸೀತು
ಹದನ
ಹದವಾಗಿ
ಹದಿನೈದು
ಹದಿ-ನಾ-ರನೇ
ಹದಿನಾರು
ಹದಿ-ನಾಲ್ಕನೆ
ಹದಿ-ನಾಲ್ಕ-ನೆಯ
ಹದಿ-ನಾಲ್ಕನೇ
ಹದಿನಾಲ್ಕು
ಹದಿ-ನೆಂಟನೇ
ಹದಿ-ನೆಂಟ-ರಿಂದ
ಹದಿನೆಂಟು
ಹದಿ-ನೆಂಟು-ವರ್ಷ
ಹದಿ-ನೇ-ಳನೆ
ಹದಿನೇಳು
ಹದಿ-ನೈ-ದನೇ
ಹದಿನೈದು
ಹದಿ-ಮೂ-ರನೆ
ಹದಿಮೂರು
ಹದಿ-ಹ-ರೆ-ಯದ
ಹದ್ದು-ಬಸ್ತಿ-ನಲ್ಲಿ
ಹನನ
ಹನಿ
ಹನಿ-ಯನ್ನಾ-ಗಲೀ
ಹನಿ-ಯೊ-ಡೆದು
ಹನಿ-ಹ-ನಿ-ಯಾಗಿ
ಹನ್ನೆ-ರ-ಡ-ರ-ವ-ರೆಗೆ
ಹನ್ನೆರಡು
ಹನ್ನೆ-ರ-ಡು-ಪಟ್ಟು
ಹನ್ನೊಂದನೇ
ಹನ್ನೊಂದು
ಹಬ್ಬದಂದು
ಹಬ್ಬರ್ಡ್
ಹಬ್ಬ-ಹ-ರಿ-ದಿ-ನ-ಗ-ಳಲ್ಲೂ
ಹಬ್ಬಿ
ಹಬ್ಬಿ-ಕೊಂಡಿದೆ
ಹಬ್ಬಿದ
ಹಬ್ಬಿ-ದ-ರೆಂಬುದು
ಹಬ್ಬಿಸುವ
ಹಮ್ಮು
ಹಯ-ನ-ಹುದು
ಹರ
ಹರಕೆ
ಹರ-ಕೆ-ಯನ್ನೂ
ಹರ-ಗೌ-ರಿ-ಯರ
ಹರಗೌರೀ
ಹರಟೆ
ಹರ-ಟೆ-ಮಲ್ಲ-ತ-ನ-ವಾಗಿ
ಹರ-ಡ-ದಿ-ರು-ವುದೇ
ಹರ-ಡ-ಬೇ-ಕಾ-ದುದು
ಹರಡಿತು
ಹರ-ಡಿ-ಸಿದೆ
ಹರಣ
ಹರ-ಣೆ-ಗ-ಳಿವೆ
ಹರನನ್ನು
ಹರನೇ
ಹರ-ಳು-ಗ-ಳಿಂದ
ಹರ-ಸಿ-ದರು
ಹರ-ಸಿ-ದಾಗ
ಹರಿ
ಹರಿಕಾರ
ಹರಿಜನ
ಹರಿದ
ಹರಿದರೆ
ಹರಿದಾಗ
ಹರಿ-ದಾ-ಸರು
ಹರಿ-ದಿದ್ದಲ್ಲಿ
ಹರಿ-ದಿ-ರ-ಲಿಲ್ಲ
ಹರಿದಿಲ್ಲ
ಹರಿದು
ಹರಿನಾಮ
ಹರಿ-ಯ-ಗೊಟ್ಟಾಗ
ಹರಿ-ಯ-ಗೊ-ಡ-ಬೇಕು
ಹರಿ-ಯ-ಗೊ-ಡುತ್ತಾರೆ
ಹರಿ-ಯ-ಗೊ-ಡು-ವು-ದಿಲ್ಲ
ಹರಿ-ಯ-ತೊ-ಡ-ಗುತ್ತದೆ
ಹರಿ-ಯ-ಬಿ-ಡುತ್ತಾರೆ
ಹರಿ-ಯ-ಬಿ-ಡುವ
ಹರಿ-ಯ-ಬೇ-ಕಲ್ಲವೇ
ಹರಿ-ಯ-ಲಿಲ್ಲ
ಹರಿಯಲು
ಹರಿಯಿ
ಹರಿಯಿತು
ಹರಿ-ಯಿ-ಸ-ಬೇಕು
ಹರಿ-ಯಿ-ಸಲು
ಹರಿಯಿಸಿ
ಹರಿ-ಯಿ-ಸಿತು
ಹರಿ-ಯಿ-ಸಿ-ದರೆ
ಹರಿ-ಯಿ-ಸಿ-ದ-ವನು
ಹರಿ-ಯಿ-ಸಿದೆ
ಹರಿ-ಯಿ-ಸುತ್ತ
ಹರಿ-ಯಿ-ಸು-ವರು
ಹರಿ-ಯಿ-ಸು-ವು-ದರ
ಹರಿ-ಯುತ್ತಿದ್ದರೆ
ಹರಿ-ಯುತ್ತದೆ
ಹರಿ-ಯುತ್ತವೆ
ಹರಿ-ಯುತ್ತಿದೆ
ಹರಿ-ಯುತ್ತಿ-ರುತ್ತದೆ
ಹರಿ-ಯುತ್ತಿ-ರುವ
ಹರಿಯುವ
ಹರಿ-ಯು-ವಂತಾ-ದರೆ
ಹರಿ-ಯು-ವಂತೆ
ಹರಿ-ಯು-ವುದು
ಹರಿ-ಯು-ವುದೇ
ಹರಿ-ಯು-ವುವು
ಹರಿ-ಸ-ಬೇಕು
ಹರಿಸಿ
ಹರಿ-ಸುತ್ತಿದ್ದಾರೆ
ಹರಿಸುವ
ಹರಿ-ಸು-ವಲ್ಲಿ
ಹರಿ-ಸು-ವುದು
ಹರುಕು
ಹರೆಯ
ಹರ್ಬಟ್
ಹರ್ಮನ್
ಹರ್ಷಾ-ನಂದಜೀ
ಹಲಗೆಯ
ಹಲ-ಗೆ-ಯನ್ನು
ಹಲಬುತ್ತ
ಹಲವರ
ಹಲ-ವ-ರನ್ನು
ಹಲ-ವ-ರಲ್ಲಿ
ಹಲ-ವ-ರಿಗೆ
ಹಲ-ವ-ರಿ-ರ-ಬ-ಹುದು
ಹಲವರು
ಹಲವಾರು
ಹಲವು
ಹಲವುಳ್ಳ
ಹಲಸಿನ
ಹಲಸು
ಹಲ್ಮ-ಸೆ-ದಿದೆ
ಹಲ್ಲನ್ನು
ಹಲ್ಲಿನ
ಹಲ್ಲು
ಹಲ್ಲುಗಳ
ಹಲ್ಲುಗಳೂ
ಹಲ್ಲುಜ್ಜುತ್ತಿದ್ದರು
ಹಲ್ಲುಜ್ಜುವ
ಹಲ್ಲೆ-ಗೊ-ಳ-ಗಾ-ಗುತ್ತಾರೆ
ಹಳದಿ
ಹಳಸಿದ
ಹಳಿ-ದ-ನಂತೆ
ಹಳಿದು
ಹಳಿ-ದು-ಕೊಂಡರೂ
ಹಳಿಯುತ್ತಾ
ಹಳಿ-ಯುತ್ತಾನೋ
ಹಳಿ-ಯು-ವರೋ
ಹಳೆ
ಹಳೆಯ
ಹಳೇ-ಕಾ-ಲ-ದಿಂದ
ಹಳ್ಳ
ಹಳ್ಳಿಗ
ಹಳ್ಳಿ-ಗ-ನೊಬ್ಬ
ಹಳ್ಳಿಗರ
ಹಳ್ಳಿಗಳ
ಹಳ್ಳಿ-ಗ-ಳಲ್ಲಿ
ಹಳ್ಳಿ-ಗಾ-ಡಿನ
ಹಳ್ಳಿಗೆ
ಹಳ್ಳಿಯ
ಹಳ್ಳಿಯಲ್ಲಿ
ಹಳ್ಳಿ-ಯಾ-ಗಿದ್ದು
ಹಳ್ಳಿಯಿಂದ
ಹಳ್ಳಿ-ಯೊಂದ-ರಲ್ಲಿ
ಹವ-ಣಿ-ಸಿ-ದ-ವರೂ
ಹವ-ಣಿ-ಸುತ್ತಿ-ರುವ
ಹವ-ಣಿ-ಸು-ವು-ದುಂಟು
ಹವೆ
ಹವ್ಯಾಸ
ಹವ್ಯಾ-ಸ-ಗ-ಳನ್ನು
ಹವ್ಯಾ-ಸ-ವಾ-ಗಲಿ
ಹಸ-ನಾ-ಗಲು
ಹಸ-ನಾ-ಗಿ-ಸುವ
ಹಸ-ನಾ-ಗು-ವುದು
ಹಸ-ನಾ-ಗು-ವು-ದು-ಪು-ಟಕ್ಕಿಟ್ಟ
ಹಸ-ನು-ಗೊ-ಳಿ-ಸಿ-ಕೊಳ್ಳ-ಬೇಕು
ಹಸಿದ
ಹಸಿ-ದಿದ್ದರೂ
ಹಸಿದು
ಹಸಿವನ್ನು
ಹಸಿ-ವಿ-ನಿಂದ
ಹಸಿವು
ಹಸಿವೆ
ಹಸಿ-ವೆ-ಯನ್ನೇ
ಹಸಿ-ವೆ-ಯಿಂದ
ಹಸು
ಹಸು-ಗೂ-ಸಿಗೆ
ಹಸು-ರಾ-ಗಿದ್ದ
ಹಸು-ರಾ-ಗಿ-ರುತ್ತದೆ
ಹಸುವಿಗೆ
ಹಸು-ವಿ-ನಂತೆ
ಹಸು-ವಿ-ನೆ-ದುರು
ಹಸ್ತ
ಹಸ್ತಗಳ
ಹಸ್ತ-ದಿಂದಲೇ
ಹಸ್ತಪ್ರ-ತಿ-ಯನ್ನು
ಹಸ್ತವನ್ನು
ಹಸ್ತಿ-ನಾ-ಪು-ರ-ವನ್ನು
ಹಾಂ
ಹಾಕದೆ
ಹಾಕಬಲ್ಲೆ
ಹಾಕ-ಬ-ಹು-ದಿತ್ತು
ಹಾಕಬೇಕೋ
ಹಾಕಲು
ಹಾಕಿ
ಹಾಕಿಕೊಂಡ
ಹಾಕಿ-ಕೊಂಡಂತೆ
ಹಾಕಿ-ಕೊಂಡ-ವ-ನಲ್ಲ
ಹಾಕಿ-ಕೊಂಡಿದ್ದರು
ಹಾಕಿ-ಕೊಂಡಿದ್ದರೂ
ಹಾಕಿಕೊಂಡು
ಹಾಕಿ-ಕೊಳ್ಳ-ಬಾ-ರ-ದೆಂದಲ್ಲ
ಹಾಕಿ-ಕೊಳ್ಳ-ಬೇ-ಡವೇ
ಹಾಕಿ-ಕೊಳ್ಳಲು
ಹಾಕಿ-ಕೊಳ್ಳು-ವಂತೆ
ಹಾಕಿ-ಕೊಳ್ಳು-ವು-ದಿಲ್ಲ
ಹಾಕಿತು
ಹಾಕಿದ
ಹಾಕಿ-ದಂತಲ್ಲವೇ
ಹಾಕಿದಂತೆ
ಹಾಕಿದರು
ಹಾಕಿದರೆ
ಹಾಕಿ-ದೊ-ಡನೆ
ಹಾಕಿದ್ದ
ಹಾಕಿದ್ದರು
ಹಾಕಿದ್ದಾಗ
ಹಾಕಿದ್ದಾ-ದರೂ
ಹಾಕಿದ್ದೇ
ಹಾಕಿ-ಬಿ-ಡುತ್ತಾನೆ
ಹಾಕಿ-ರ-ಬ-ಹುದು
ಹಾಕಿ-ರು-ವಾಗ
ಹಾಕಿಲ್ಲ
ಹಾಕುತ್ತ
ಹಾಕುತ್ತಾ
ಹಾಕುತ್ತಾರೆ
ಹಾಕುತ್ತಿದ್ದರು
ಹಾಕುತ್ತಿ-ರ-ಲಿಲ್ಲ
ಹಾಕುತ್ತೀರಿ
ಹಾಕುತ್ತೇನೆ
ಹಾಕುವ
ಹಾಕುವಂತೆ
ಹಾಕು-ವ-ವರು
ಹಾಕುವಾಗ
ಹಾಕುವುದು
ಹಾಗಲ್ಲದೆ
ಹಾಗಾ-ಗ-ದಂತೆ
ಹಾಗಾಗಿ
ಹಾಗಾಗಿಯೇ
ಹಾಗಾ-ಗಿ-ರ-ಲಿಲ್ಲ
ಹಾಗಾ-ಗು-ವು-ದಿಲ್ಲ
ಹಾಗಾದರೆ
ಹಾಗಾದಲ್ಲಿ
ಹಾಗಾದಾಗ
ಹಾಗಾಯಿತು
ಹಾಗಿದೆ
ಹಾಗಿದ್ದರೆ
ಹಾಗಿದ್ದು-ದ-ರಿಂದಲೇ
ಹಾಗಿ-ರು-ವಾಗ
ಹಾಗಿಲ್ಲ
ಹಾಗಿಲ್ಲದೆ
ಹಾಗಿಲ್ಲ-ವೆಂದು
ಹಾಗೂ
ಹಾಗೆ
ಹಾಗೆಂದ-ರೇ-ನೆಂದು
ಹಾಗೆಂದಾ-ದರೆ
ಹಾಗೆಂದು
ಹಾಗೆಯೇ
ಹಾಗೇ
ಹಾಜರಾದ
ಹಾಜ-ರು-ಪ-ಡಿ-ಸು-ವಂತೆ
ಹಾಡ-ಬ-ಹುದು
ಹಾಡಿ
ಹಾಡಿ-ಕೊಳ್ಳುತ್ತಿದ್ದಾ-ಗಲೇ
ಹಾಡಿ-ಕೊಳ್ಳು-ವುದು
ಹಾಡಿದ
ಹಾಡು-ಗ-ಳನ್ನು
ಹಾಡು-ಗ-ಳನ್ನೂ
ಹಾಡುಗಳು
ಹಾಡುತ್ತವೆ
ಹಾಡುತ್ತಿ-ರು-ವಾ-ಗಲೇ
ಹಾಡುತ್ತಿ-ರುತ್ತಾರೆ
ಹಾಡುವ
ಹಾತೊ-ರೆ-ಯುತ್ತಿದೆ
ಹಾತೊ-ರೆ-ದರು
ಹಾತೊ-ರೆ-ದಿದ್ದನೋ
ಹಾತೊರೆದು
ಹಾತೊ-ರೆ-ಯುತ್ತದೆ
ಹಾತೊ-ರೆ-ಯುತ್ತಲೂ
ಹಾತೊ-ರೆ-ಯುತ್ತಿದೆ
ಹಾತೊ-ರೆ-ಯುತ್ತಿದ್ದೆವು
ಹಾತೊ-ರೆ-ಯುತ್ತಿದ್ದೇನೆ
ಹಾತೊ-ರೆ-ಯುವ
ಹಾತೊ-ರೆ-ಯು-ವಂತೆ
ಹಾತೊ-ರೆ-ಯು-ವುದೋ
ಹಾದಿ
ಹಾದಿಯಲ್ಲಿ
ಹಾದಿಯಲ್ಲೇ
ಹಾನಿ
ಹಾನಿಕರ
ಹಾನಿ-ಕಾ-ರಕ
ಹಾನಿಗೆ
ಹಾನಿ-ಗೊ-ಳ-ಗಾದ
ಹಾನಿಯನ್ನು
ಹಾನಿಯನ್ನೂ
ಹಾನಿ-ಯಾ-ಗ-ಲಿಲ್ಲ-ವಾ-ದರೂ
ಹಾನಿ-ಯಾ-ಗುತ್ತ-ದೆಂಬು-ದನ್ನು
ಹಾಪ್ಕಿನ್ಸ್
ಹಾಫ್ಕಿನ್ಸ್ವಿಲ್ಲೆ-ಯಲ್ಲಿ
ಹಾಬ್ಸ್
ಹಾಯಾಗಿ
ಹಾಯಾ-ಗಿ-ರುತ್ತೇನೆ
ಹಾಯಿ-ಬಿ-ಡಿ-ಸಿ-ದಂತೆ
ಹಾಯಿಯನ್ನೂ
ಹಾಯ್ದಿ-ರ-ಲಿಲ್ಲ
ಹಾರದೇ
ಹಾರಲು
ಹಾರಾಟ
ಹಾರಾಡಿ
ಹಾರಾ-ಡಿ-ದ-ನೆಂದರೆ
ಹಾರಿ
ಹಾರಿಕೆಯ
ಹಾರಿದ
ಹಾರಿಬಂದು
ಹಾರಿ-ಯೇ-ಹೋ-ಗುತ್ತ-ದೆ-ಎಂಬಂತೆ
ಹಾರಿಸಲು
ಹಾರಿಸಿ
ಹಾರಿ-ಸಿ-ಕೊಂಡು
ಹಾರಿಸಿದ
ಹಾರಿ-ಸಿ-ದರೂ
ಹಾರಿ-ಸುತ್ತಿದ್ದರು
ಹಾರಿಹೋಗಿ
ಹಾರಿ-ಹೋ-ದ-ರೇನು
ಹಾರುವ
ಹಾರೈಕೆ
ಹಾರೈಕೆಯೇ
ಹಾರೈಸಿದ
ಹಾರೈಸುತ್ತ
ಹಾರೈ-ಸುತ್ತಾನೆ
ಹಾರೈ-ಸುತ್ತಿ-ರುತ್ತಾರೆ
ಹಾರೈಸುವ
ಹಾರ್ಟ್
ಹಾರ್ಟ್ಗೆ
ಹಾರ್ಮೋ-ನಿ-ಯಮ್
ಹಾರ್ವರ್ಡ್
ಹಾಲನ್ನು
ಹಾಲಿಗೆ
ಹಾಲಿನಲ್ಲಿ
ಹಾಲು
ಹಾಲ್ಡೇನ್
ಹಾಳಾ-ಗುತ್ತವೆ
ಹಾಳಾ-ಗು-ವುದು
ಹಾಳಾದ
ಹಾಳಾದನು
ಹಾಳಾಯಿತು
ಹಾಳು
ಹಾಳು-ಗ-ಡೆ-ಹು-ವು-ದಿಲ್ಲ
ಹಾಳು-ಗೆ-ಡವಿ
ಹಾಳು-ಗೆ-ಡಹಿ
ಹಾಳು-ಗೆ-ಡ-ಹು-ವುದು
ಹಾಳುಮಾಡಿ
ಹಾಳು-ಮಾ-ಡಿ-ದೆ-ಯಲ್ಲಾ
ಹಾಳು-ಮಾ-ಡಿ-ಬಿ-ಡುತ್ತವೆ
ಹಾಳು-ಮಾ-ಡುತ್ತವೆ
ಹಾಳೆಗಳ
ಹಾಳೆಗಳು
ಹಾವನ್ನು
ಹಾವಭಾವ
ಹಾವ-ಭಾ-ವ-ಗ-ಳನ್ನು
ಹಾವಳಿ
ಹಾವ-ಳಿ-ಯಿಂದ
ಹಾವಿಗೆ
ಹಾವಿನ
ಹಾವಿನಿಂದ
ಹಾವಿ-ನೊ-ಡನೆ
ಹಾವು
ಹಾವು-ಗ-ಳನ್ನು
ಹಾವುಗಳು
ಹಾವೇ
ಹಾಸಿ
ಹಾಸಿಗೆ
ಹಾಸಿ-ಗೆ-ಯಲ್ಲಿ
ಹಾಸಿ-ಗೆ-ಯಲ್ಲೇ
ಹಾಸು
ಹಾಸು-ಹೊಕ್ಕಾ-ಗಿ-ರು-ವುವು
ಹಾಸ್ಟೆಲಿನ
ಹಾಸ್ಟೆ-ಲಿ-ನಲ್ಲಿ
ಹಾಸ್ಟೆ-ಲು-ಗಳ
ಹಾಸ್ಟೆಲ್
ಹಾಸ್ಯ
ಹಾಸ್ಯಕ್ಕೂ
ಹಾಸ್ಯ-ಗಾ-ರ-ರಂತೆ
ಹಾಸ್ಯದ
ಹಾಸ್ಯದೃಷ್ಟಿ
ಹಾಸ್ಯಪ್ರಜ್ಞೆ
ಹಿಂಗಡೆ
ಹಿಂಗು-ವು-ದಿಲ್ಲ
ಹಿಂಜ-ರಿ-ಯ-ತೊ-ಡ-ಗಿ-ದವು
ಹಿಂಜ-ರಿ-ಯನು
ಹಿಂಜ-ರಿ-ಯ-ಬಾ-ರದು
ಹಿಂಜ-ರಿ-ಯ-ಲಿಲ್ಲ
ಹಿಂಜ-ರಿ-ಯಳು
ಹಿಂಜ-ರಿ-ಯುತ್ತಾರೆ
ಹಿಂಜ-ರಿ-ಯುತ್ತಿ-ರ-ಲಿಲ್ಲ
ಹಿಂಜ-ರಿ-ಯುತ್ತೀರಾ
ಹಿಂಜ-ರಿ-ಯು-ವು-ದಿಲ್ಲ
ಹಿಂಡಿ-ದಂತಾ-ಗುತ್ತಿತ್ತು
ಹಿಂಡಿ-ದಂತಾ-ಯಿತು
ಹಿಂಡಿದಂತೆ
ಹಿಂಡುವ
ಹಿಂತಿರುಗಿ
ಹಿಂದಕ್ಕೆ
ಹಿಂದಣ
ಹಿಂದಾ-ಗ-ಲಿಲ್ಲ
ಹಿಂದಾದುಕ್ಕೆ
ಹಿಂದಾ-ದು-ದಕ್ಕೆ
ಹಿಂದಿದೆ
ಹಿಂದಿದ್ದಾರೆ
ಹಿಂದಿನ
ಹಿಂದಿನಂತೆ
ಹಿಂದಿ-ನ-ದನ್ನು
ಹಿಂದಿನಷ್ಟು
ಹಿಂದಿನಿಂದ
ಹಿಂದಿ-ನಿಂದಲೂ
ಹಿಂದಿ-ರು-ಗಿದೆ
ಹಿಂದಿ-ರು-ಗದು
ಹಿಂದಿ-ರು-ಗ-ಬೇ-ಕಾ-ಯಿತು
ಹಿಂದಿ-ರು-ಗ-ಬೇ-ಕೆಂದು
ಹಿಂದಿ-ರು-ಗಲು
ಹಿಂದಿರುಗಿ
ಹಿಂದಿ-ರು-ಗಿತು
ಹಿಂದಿ-ರು-ಗಿದ
ಹಿಂದಿ-ರು-ಗಿ-ದರು
ಹಿಂದಿ-ರು-ಗಿ-ದರೆ
ಹಿಂದಿ-ರು-ಗಿ-ದಳು
ಹಿಂದಿ-ರು-ಗಿ-ದ-ವರು
ಹಿಂದಿ-ರು-ಗಿ-ದಾಗ
ಹಿಂದಿ-ರು-ಗಿ-ದೆ-ನೆಂದೂ
ಹಿಂದಿ-ರು-ಗಿದ್ದರು
ಹಿಂದಿ-ರು-ಗಿ-ಸ-ಬೇಕು
ಹಿಂದಿ-ರು-ಗಿ-ಸಿದ
ಹಿಂದಿ-ರು-ಗಿ-ಸಿ-ದರೆ
ಹಿಂದಿ-ರು-ಗಿ-ಸಿದ್ದೇ-ವೆಂದು-ಕೊಂಡು
ಹಿಂದಿ-ರು-ಗಿ-ಸುತ್ತದೆ
ಹಿಂದಿ-ರು-ಗಿ-ಸುತ್ತಾನೆ
ಹಿಂದಿ-ರು-ಗಿ-ಸುತ್ತಿದ್ದರು
ಹಿಂದಿ-ರು-ಗಿ-ಸುತ್ತೇನೆ
ಹಿಂದಿ-ರು-ಗಿ-ಸು-ವು-ದಾ-ದರೂ
ಹಿಂದಿ-ರು-ಗಿ-ಸು-ವು-ದಿಲ್ಲ
ಹಿಂದಿ-ರು-ಗುತ್ತಿದ್ದರು
ಹಿಂದಿ-ರು-ಗುತ್ತಿದ್ದವು
ಹಿಂದಿ-ರು-ಗುತ್ತಿದ್ದೆ
ಹಿಂದಿ-ರು-ಗುತ್ತೇವೆ
ಹಿಂದಿ-ರು-ಗುವ
ಹಿಂದಿ-ರು-ಗು-ವಾಗ
ಹಿಂದಿ-ರು-ಗು-ವು-ದಕ್ಕೆ
ಹಿಂದಿ-ರು-ಗು-ವೆ-ನೆಂದೂ
ಹಿಂದೀ
ಹಿಂದು
ಹಿಂದುಗಳ
ಹಿಂದು-ಗ-ಳಿಗೆ
ಹಿಂದುಗಳು
ಹಿಂದುಮುಂದು
ಹಿಂದುಳಿದ
ಹಿಂದು-ಳಿ-ದ-ವ-ರನ್ನು
ಹಿಂದು-ಳಿ-ದ-ವ-ರಲ್ಲಿ
ಹಿಂದು-ಳಿ-ದ-ವ-ರಿಗೆ
ಹಿಂದು-ಳಿ-ದ-ವರು
ಹಿಂದು-ಳಿ-ದ-ವ-ರೆಲ್ಲರೂ
ಹಿಂದು-ಳಿ-ಯುತ್ತಾರೆ
ಹಿಂದು-ಳಿ-ಯು-ವಿಕೆ
ಹಿಂದೂ
ಹಿಂದೂ-ಗ-ಳಲ್ಲಿ
ಹಿಂದೂಧರ್ಮ
ಹಿಂದೂ-ಧರ್ಮದ
ಹಿಂದೂ-ಧರ್ಮ-ದಲ್ಲಿ
ಹಿಂದೂಸ್ಥಾ-ನದ
ಹಿಂದೆ
ಹಿಂದೆಂದಿ-ಗಿಂತಲೂ
ಹಿಂದೆ-ಗೆ-ಯುತ್ತಿದ್ದರು
ಹಿಂದೆಯೂ
ಹಿಂದೆಯೆ
ಹಿಂದೆಯೇ
ಹಿಂದೇಟು
ಹಿಂದೊಮ್ಮೆ
ಹಿಂಬಾ-ಲಿ-ಸಲ್ಪಟ್ಟು
ಹಿಂಬಾಲಿಸಿ
ಹಿಂಬಾ-ಲಿ-ಸಿ-ದಂತಾ-ಗುತ್ತದೆ
ಹಿಂಬಾ-ಲಿ-ಸಿ-ದೆ-ನಲ್ಲ
ಹಿಂಬಾ-ಲಿ-ಸು-ವ-ವನ
ಹಿಂಬಾ-ಲಿ-ಸುತ್ತ
ಹಿಂಬಾ-ಲಿ-ಸುತ್ತದೆ
ಹಿಂಬಾ-ಲಿ-ಸುತ್ತ-ಲಿದೆ
ಹಿಂಬಾ-ಲಿ-ಸುತ್ತಾ
ಹಿಂಬಾ-ಲಿ-ಸುತ್ತಾ-ರಷ್ಟೆ
ಹಿಂಬಾ-ಲಿ-ಸುತ್ತಿದ್ದು-ದನ್ನು
ಹಿಂಬಾ-ಲಿ-ಸು-ವಂತೆ
ಹಿಂಬಾ-ಲಿ-ಸು-ವುದು
ಹಿಂಭಾ-ರ-ವಾಗಿ
ಹಿಂಸಾ
ಹಿಂಸಾ-ಕೃತ್ಯಕ್ಕಿ-ಳಿ-ಯಲೂ
ಹಿಂಸಾ-ಕೃತ್ಯ-ಗ-ಳನ್ನು
ಹಿಂಸಾ-ಚ-ರಣೆ
ಹಿಂಸಾತುರ
ಹಿಂಸಾತ್ಮ-ಕ-ವಾಗಿ
ಹಿಂಸಾಪ್ರ-ವೃತ್ತಿ
ಹಿಂಸಾರತಿ
ಹಿಂಸಾ-ರ-ತಿ-ಇ-ವು-ಗಳ
ಹಿಂಸಿ-ಸ-ತೊ-ಡ-ಗಿದ್ದಾಳೆ
ಹಿಂಸಿಸಿ
ಹಿಂಸಿ-ಸಿ-ಕೊಂಡಂತೆ
ಹಿಂಸಿ-ಸಿ-ದಾಗ
ಹಿಂಸಿ-ಸು-ವುದೇ
ಹಿಂಸೆ
ಹಿಂಸೆ-ಗ-ಳನ್ನು
ಹಿಂಸೆ-ಗ-ಳಲ್ಲಿ
ಹಿಂಸೆಗೆ
ಹಿಂಸೆ-ಗೊ-ಳ-ಗಾ-ಗುವ
ಹಿಂಸೆ-ಮಾ-ಡುತ್ತ
ಹಿಂಸೆಯ
ಹಿಂಸೆಯನ್ನು
ಹಿಂಸೆಯಿಂದ
ಹಿಂಸ್ರಪ್ರಾ-ಣಿ-ಗ-ಳನ್ನು
ಹಿಗ್ಗಿ-ದು-ದನ್ನು
ಹಿಟ್ಲರನ
ಹಿಟ್ಲರ್
ಹಿಡ
ಹಿಡಿ
ಹಿಡಿಯಲು
ಹಿಡಿತಕ್ಕೆ
ಹಿಡಿ-ತ-ಗ-ಳಿಂದ
ಹಿಡಿ-ತ-ದಿಂದ
ಹಿಡಿದ
ಹಿಡಿ-ದಂತಾ-ಗು-ವಂತೆ
ಹಿಡಿ-ದಂತಾ-ಯಿತು
ಹಿಡಿದಂತೆ
ಹಿಡಿದರೆ
ಹಿಡಿ-ದ-ರೇನು
ಹಿಡಿ-ದ-ವ-ರನ್ನು
ಹಿಡಿ-ದ-ವರು
ಹಿಡಿ-ದ-ವರೇ
ಹಿಡಿದಾಶೆ
ಹಿಡಿ-ದಿಟ್ಟಿ-ರುತ್ತದೆ
ಹಿಡಿ-ದಿ-ಡ-ಲಾ-ಗಿದೆ
ಹಿಡಿದಿತ್ತೇ
ಹಿಡಿದಿದ್ದ
ಹಿಡಿ-ದಿದ್ದಾರೆ
ಹಿಡಿ-ದಿದ್ದೇನೆ
ಹಿಡಿದು
ಹಿಡಿ-ದು-ಕೊಂಡ
ಹಿಡಿ-ದು-ಕೊಂಡಾ
ಹಿಡಿ-ದು-ಕೊಂಡಿದೆ
ಹಿಡಿ-ದು-ಕೊಂಡು
ಹಿಡಿ-ದು-ಕೊಂಡೇ
ಹಿಡಿ-ದು-ಕೊಳ್ಳ-ಬೇಕು
ಹಿಡಿ-ದು-ಕೊಳ್ಳುತ್ತದೆ
ಹಿಡಿ-ದು-ಕೊಳ್ಳುತ್ತವೆ
ಹಿಡಿ-ದು-ತಂದು
ಹಿಡಿದೆ
ಹಿಡಿ-ದೆ-ಳೆ-ದ-ವ-ರಾ-ರೆಂದು
ಹಿಡಿಯದೇ
ಹಿಡಿ-ಯ-ಬಂದವು
ಹಿಡಿ-ಯ-ಬೇಕು
ಹಿಡಿ-ಯ-ಲಾ-ಗದ
ಹಿಡಿ-ಯ-ಲಾ-ಗ-ಲಿಲ್ಲ
ಹಿಡಿ-ಯ-ಲಾ-ರದು
ಹಿಡಿಯಲು
ಹಿಡಿಯಿತು
ಹಿಡಿ-ಯುತ್ತದೆ
ಹಿಡಿ-ಯುತ್ತಾ-ನೆಂಬ
ಹಿಡಿ-ಯುತ್ತಾರೆ
ಹಿಡಿಯುವ
ಹಿಡಿ-ಯು-ವಲ್ಲಿ
ಹಿಡಿ-ಯು-ವು-ದಕ್ಕೆ
ಹಿಡಿ-ಯು-ವು-ದಿಲ್ಲ
ಹಿಡಿ-ಯು-ವು-ದೆಂದರೆ
ಹಿಡಿಸದ
ಹಿಡಿ-ಸ-ದಿದ್ದಲ್ಲಿ
ಹಿಡಿ-ಸ-ದಿ-ರ-ಬ-ಹುದು
ಹಿಡಿಸದೆ
ಹಿಡಿಸದ್ದು
ಹಿಡಿ-ಸ-ಬ-ಹುದು
ಹಿಡಿ-ಸ-ಲಿಲ್ಲ
ಹಿಡಿಸಿತೋ
ಹಿಡಿಸಿದೆ
ಹಿಡಿ-ಸಿ-ರ-ಲಿಲ್ಲ
ಹಿಡಿ-ಸು-ವು-ದಿಲ್ಲ
ಹಿತ
ಹಿತಕರ
ಹಿತ-ಕಾ-ರಿ-ಯಾದ
ಹಿತಕ್ಕಾಗಿ
ಹಿತಕ್ಕೂ
ಹಿತಕ್ಕೆ
ಹಿತ-ಚಿಂತನೆ
ಹಿತ-ಚಿಂತ-ನೆಗೆ
ಹಿತ-ಚಿಂತ-ನೆಯ
ಹಿತ-ಚಿಂತ-ನೆ-ಯನ್ನು
ಹಿತ-ಚಿಂತ-ನೆ-ಯನ್ನೂ
ಹಿತ-ಚಿಂತ-ನೆ-ಯಲ್ಲಿ
ಹಿತದ
ಹಿತದಲ್ಲಿ
ಹಿತ-ದೃಷ್ಟಿ-ಯಿಂದ-ಲಾ-ದರೂ
ಹಿತ-ದೃಷ್ಟಿಯ
ಹಿತ-ದೃಷ್ಟಿ-ಯಿಂದ
ಹಿತ-ನು-ಡಿಯ
ಹಿತ-ರಕ್ಷ-ಣೆಯ
ಹಿತರು
ಹಿತ-ವ-ಚನ
ಹಿತವನ್ನು
ಹಿತವನ್ನೂ
ಹಿತವಾಗಿ
ಹಿತ-ವೆ-ನಿ-ಸುತ್ತದೆ
ಹಿತ-ಸು-ಖ-ಗ-ಳನ್ನು
ಹಿತಾ-ಕಾಂಕ್ಷೆ-ಯಿಂದ
ಹಿತಾಭ್ಯು-ದ-ಯ-ಗ-ಳನ್ನೇ
ಹಿತಾಯ
ಹಿತಾ-ಸಕ್ತಿಯ
ಹಿತೋ-ಪ-ದೇ-ಶ-ದಲ್ಲಿ
ಹಿನ್ನಡೆ
ಹಿನ್ನಡೆಗೆ
ಹಿನ್ನಲೆ
ಹಿನ್ನ-ಲೆ-ಯನ್ನು
ಹಿನ್ನೆಲೆ
ಹಿನ್ನೆಲೆಯ
ಹಿನ್ನೆ-ಲೆ-ಯನ್ನು
ಹಿನ್ನೆ-ಲೆ-ಯನ್ನೂ
ಹಿನ್ನೆ-ಲೆ-ಯಲ್ಲಿ
ಹಿನ್ನೆ-ಲೆ-ಯಲ್ಲಿನ
ಹಿನ್ನೆ-ಲೆ-ಯಲ್ಲಿ-ರುವ
ಹಿನ್ನೆ-ಲೆ-ಯಲ್ಲೂ
ಹಿನ್ನೆ-ಲೆ-ಯಿಂದ
ಹಿನ್ನೆ-ಲೆ-ಯಿಂದಲೆ
ಹಿನ್ನೆ-ಲೆ-ಯಿಲ್ಲದೆ
ಹಿನ್ನೆ-ಲೆ-ಯುಳ್ಳ
ಹಿಪಾಕ್ರಸಿ
ಹಿಪ್ನಾಸಿಸ್
ಹಿಪ್ನಾ-ಸಿ-ಸ್ಇ-ದರ
ಹಿಮ-ಕ-ರಡಿ
ಹಿಮ-ಕ-ರ-ಡಿಯ
ಹಿಮ-ಗಡ್ಡೆ-ಯಲ್ಲಿ
ಹಿಮವತ್
ಹಿಮಾಲಯ
ಹಿಮಾ-ಲ-ಯದ
ಹಿಮ್ಮು-ಖ-ವಾಗಿ
ಹಿರಣ್ಯ-ಕ-ಶಿಪು
ಹಿರಿ-ಕಿ-ರಿ-ಯ-ರೆಲ್ಲ-ರಿಗೂ
ಹಿರಿ-ತ-ನದ
ಹಿರಿದಾದ
ಹಿರಿ-ದಾ-ದುದು
ಹಿರಿಮೆ
ಹಿರಿ-ಮೆ-ಗ-ರಿ-ಮೆ-ಗಳ
ಹಿರಿ-ಮೆ-ಗ-ರಿ-ಮೆಯೇ
ಹಿರಿ-ಮೆ-ಯನ್ನು
ಹಿರಿಯ
ಹಿರಿಯಕ್ಕ
ಹಿರಿ-ಯಕ್ಕನ
ಹಿರಿ-ಯಣ್ಣ-ನನ್ನೊ
ಹಿರಿಯರ
ಹಿರಿ-ಯ-ರನ್ನು
ಹಿರಿ-ಯ-ರನ್ನೂ
ಹಿರಿ-ಯ-ರಿಂದ
ಹಿರಿ-ಯ-ರಿಗೆ
ಹಿರಿಯರು
ಹಿರಿಯರೂ
ಹಿರಿ-ಯ-ರೊ-ಡನೆ
ಹಿರಿ-ಯ-ರೊಬ್ಬರ
ಹಿರಿ-ಯ-ರೊಬ್ಬ-ರನ್ನು
ಹಿರಿ-ಯ-ರೊಬ್ಬರು
ಹಿರಿ-ಯ-ವನು
ಹಿರಿಹಿರಿ
ಹಿರೇ
ಹಿರೋ-ಶಿ-ಮಾದ
ಹಿರೋ-ಷಿ-ಮ-ವನ್ನು
ಹೀಗಾಗಿ
ಹೀಗಾಗಿದೆ
ಹೀಗಾ-ಗುತ್ತಿದೆ
ಹೀಗಾಯಿತು
ಹೀಗಿತ್ತು
ಹೀಗಿದೆ
ಹೀಗಿ-ದೆ-ಸಿ-ರಾಕ್ಯೂಸ್ನ
ಹೀಗಿದ್ದರೂ
ಹೀಗಿ-ರ-ಬ-ಹುದು
ಹೀಗಿರುತ್ತ
ಹೀಗೂ
ಹೀಗೆ
ಹೀಗೆಂ
ಹೀಗೆಂದ
ಹೀಗೆಂದ-ನನ್ನಲ್ಲಿ-ರು-ವು-ದೆಂದು
ಹೀಗೆಂದರು
ಹೀಗೆಂದ-ರು-ನನ್ನ
ಹೀಗೆಂದಿತು
ಹೀಗೆಂದಿದ್ದರು
ಹೀಗೆಂದಿದ್ದಾನೆ
ಹೀಗೆಂದಿದ್ದಾರೆ
ಹೀಗೆಂದು
ಹೀಗೆನ್ನ-ಬ-ಹುದು
ಹೀಗೆನ್ನುತ್ತದೆ
ಹೀಗೆನ್ನುತ್ತಾರೆ
ಹೀಗೆಯೇ
ಹೀನ
ಹೀನತೆ
ಹೀನತೆಗೂ
ಹೀನ-ಮಟ್ಟದ
ಹೀನವೂ
ಹೀನವೆಂದು
ಹೀನಸ್ಥಿ-ತಿ-ಯಲ್ಲಿ
ಹೀನಾಯ
ಹೀನಾ-ಯ-ಮಾ-ಡ-ಬೇ-ಕೆಂಬ
ಹೀನಾ-ಯ-ವಾಗಿ
ಹೀನೈ-ಸಿದ್ದೆಯೋ
ಹೀಯಾ-ಳಿ-ಸಲು
ಹೀಯಾ-ಳಿ-ಸು-ವು-ದ-ರಿಂದಲೂ
ಹೀರಿ
ಹೀರಿದರೆ
ಹೀರಿಯೇ
ಹೀರುತ್ತೇನೆ
ಹೀರುತ್ತೇ-ನೆಂದರೆ
ಹೀರುವ
ಹೀರುವುದು
ಹೀರುವುದೇ
ಹೀರೋ
ಹುಂಜ-ಗ-ಳಿಗೆ
ಹುಚ್ಚ
ಹುಚ್ಚ-ನಂತಾಗಿ
ಹುಚ್ಚರೂ
ಹುಚ್ಚಾ-ಟ-ವೆಂದು
ಹುಚ್ಚಿನ
ಹುಚ್ಚು
ಹುಚ್ಚುತನ
ಹುಚ್ಚು-ತ-ನವೇ
ಹುಚ್ಚುನಾಯಿ
ಹುಚ್ಚು-ನಾ-ಯಿ-ಯೊಂದು
ಹುಚ್ಚು-ವಾ-ದ-ವೆಂದು
ಹುಚ್ಚು-ಹವ್ಯಾಸ
ಹುಚ್ಚೆದ್ದು
ಹುಟ್ಟದೇ
ಹುಟ್ಟನ್ನು
ಹುಟ್ಟಿ
ಹುಟ್ಟಿ-ಕೊಂಡಿ-ವೆಯೆ
ಹುಟ್ಟಿ-ಕೊಂಡಿ-ವೆಯೇ
ಹುಟ್ಟಿತು
ಹುಟ್ಟಿದ
ಹುಟ್ಟಿ-ದಂದಿ-ನಿಂದ
ಹುಟ್ಟಿ-ದ-ಗ-ಳಿಗೆ
ಹುಟ್ಟಿ-ದ-ನಿ-ವನು
ಹುಟ್ಟಿ-ದ-ವರು
ಹುಟ್ಟಿ-ದಾ-ಗಿ-ನಿಂದಲೇ
ಹುಟ್ಟಿ-ದಾ-ರಭ್ಯ
ಹುಟ್ಟಿದ್ದು
ಹುಟ್ಟಿನ
ಹುಟ್ಟಿನಿಂದ
ಹುಟ್ಟಿ-ನಿಂದಲೇ
ಹುಟ್ಟಿಬಂದು
ಹುಟ್ಟಿರಲು
ಹುಟ್ಟಿಲ್ಲ
ಹುಟ್ಟಿ-ಸ-ಲಿಲ್ಲ
ಹುಟ್ಟಿಸಲು
ಹುಟ್ಟಿಸಿ
ಹುಟ್ಟಿ-ಸುತ್ತದೆ
ಹುಟ್ಟಿ-ಸುತ್ತಾನೆ
ಹುಟ್ಟಿ-ಸುತ್ತಿತ್ತು
ಹುಟ್ಟು
ಹುಟ್ಟು-ಕಿ-ವು-ಡಾ-ಗಿ-ರುವ
ಹುಟ್ಟು-ಗಟ್ಟಿದ
ಹುಟ್ಟು-ಗಟ್ಟಿ-ದಂತಿದೆ
ಹುಟ್ಟುಗುಣ
ಹುಟ್ಟು-ಗು-ಣವೆ
ಹುಟ್ಟುತ್ತಲೇ
ಹುಟ್ಟುವಂತೆ
ಹುಟ್ಟು-ವಾ-ಗಲೇ
ಹುಟ್ಟುವುದು
ಹುಟ್ಟು-ಹಾ-ಕು-ವ-ವನು
ಹುಡು-ಕ-ಬೇ-ಕಲ್ಲವೇ
ಹುಡು-ಕ-ಬೇಕು
ಹುಡುಕಲು
ಹುಡುಕಾಟ
ಹುಡು-ಕಾ-ಡ-ತೊ-ಡ-ಗಿ-ದರು
ಹುಡುಕಾಡಿ
ಹುಡು-ಕಾ-ಡಿದ
ಹುಡು-ಕಾ-ಡುತ್ತಿ-ರುತ್ತಾನೆ
ಹುಡುಕಿ
ಹುಡು-ಕಿ-ಕೊಂಡು
ಹುಡುಕಿದ
ಹುಡು-ಕಿ-ದರು
ಹುಡು-ಕಿ-ದರೆ
ಹುಡುಕು
ಹುಡು-ಕುತ್ತವೆ
ಹುಡುಕುತ್ತಾ
ಹುಡು-ಕುತ್ತಾನೆ
ಹುಡು-ಕುತ್ತಿದ್ದೆ
ಹುಡುಕುವ
ಹುಡು-ಕು-ವಲ್ಲಿ
ಹುಡು-ಕು-ವು-ದ-ರಲ್ಲಿ
ಹುಡು-ಕು-ವುದು
ಹುಡುಗ
ಹುಡುಗನ
ಹುಡು-ಗ-ನ-ದೇನೋ
ಹುಡು-ಗ-ನನ್ನು
ಹುಡು-ಗ-ನಲ್ಲ
ಹುಡು-ಗ-ನಲ್ಲಿ
ಹುಡು-ಗ-ನ-ವನು
ಹುಡು-ಗ-ನಾ-ಗಿದ್ದಾ-ಗಿನ
ಹುಡು-ಗ-ನಾತ
ಹುಡು-ಗ-ನಾದ
ಹುಡು-ಗ-ನಿಗೆ
ಹುಡುಗನು
ಹುಡುಗನೂ
ಹುಡು-ಗ-ನೆಂಬು-ವನು
ಹುಡುಗನೇ
ಹುಡು-ಗ-ನೊ-ಡನೆ
ಹುಡು-ಗ-ನೊಬ್ಬ
ಹುಡು-ಗ-ನೊಬ್ಬ-ನನ್ನು
ಹುಡುಗರ
ಹುಡು-ಗ-ರನ್ನು
ಹುಡು-ಗ-ರನ್ನೂ
ಹುಡು-ಗ-ರಿಗೆ
ಹುಡುಗರು
ಹುಡು-ಗ-ರೆಲ್ಲರೂ
ಹುಡು-ಗಾ-ಟಿ-ಕೆಯ
ಹುಡುಗಿ
ಹುಡುಗಿಯ
ಹುಡು-ಗಿ-ಯನ್ನು
ಹುಡು-ಗಿ-ಯನ್ನೇ
ಹುಡು-ಗಿ-ಯ-ರನ್ನು
ಹುಡು-ಗಿ-ಯರು
ಹುಡುಗಿಯೂ
ಹುಡು-ಗಿ-ಯೊಂದಿಗೆ
ಹುಡು-ಗಿ-ಯೊ-ಡನೆ
ಹುಡು-ಗಿ-ಯೊಬ್ಬಳ
ಹುಡು-ಗಿ-ಯೊಬ್ಬಳು
ಹುಣ್ಣಾ-ಗು-ವು-ದು-ತಿಳಿ
ಹುಣ್ಣು
ಹುಣ್ಣು-ಗ-ಳನ್ನು
ಹುಣ್ಣು-ಗ-ಳನ್ನೂ
ಹುತ್ತದಿಂದ
ಹುದು
ಹುದು-ಗಿ-ರುವ
ಹುದು-ಗಿ-ಕೊಂಡಿತ್ತೋ
ಹುದು-ಗಿ-ಕೊಂಡಿದೆ
ಹುದುಗಿದ
ಹುದುಗಿದೆ
ಹುದು-ಗಿ-ದೆ-ಯೆಂದೂ
ಹುದುಗಿದ್ದ
ಹುದುಗಿದ್ದು
ಹುದು-ಗಿ-ರುತ್ತದೆ
ಹುದು-ಗಿ-ರುತ್ತವೆ
ಹುದು-ಗಿ-ರುತ್ತ-ವೆ-ಆ-ಳ-ವಾದ
ಹುದು-ಗಿ-ರುವ
ಹುದುಗಿವೆ
ಹುದು-ಗಿ-ಸೋಣ
ಹುದು-ರೊ-ಳಗೆ
ಹುದ್ದೆ
ಹುದ್ದೆ-ಗ-ಳನ್ನು
ಹುದ್ದೆ-ಗ-ಳಿ-ಗಾಗಿ
ಹುದ್ದೆ-ಗಾ-ಗಿಯೂ
ಹುದ್ದೆಯ
ಹುದ್ದೆ-ಯಲ್ಲಿದ್ದ
ಹುದ್ದೆಯೇ
ಹುಬ್ಬು-ಗಂಟಿಕ್ಕುತ್ತೇವೆ
ಹುಬ್ಬೇ-ರಿ-ಸ-ಬ-ಹುದು
ಹುಮ್ಮಸ್ಸು
ಹುರಿ-ದುಂಬಿ-ಸುತ್ತಾನೆ
ಹುರಿಯ
ಹುರಿ-ಹಗ್ಗ-ವೆಂದ
ಹುರುಪು
ಹುರುಪುಳ್ಳ
ಹುರು-ಳಿಲ್ಲದ್ದು
ಹುರು-ಳಿಲ್ಲ-ವೆಂದಾ-ದರೆ
ಹುಲಿ
ಹುಲಿಗೂ
ಹುಲಿಗೆ
ಹುಲಿಮರಿ
ಹುಲಿಯ
ಹುಲಿಯಂತೆ
ಹುಲಿ-ಯಂತೆಯೆ
ಹುಲಿ-ಯನ್ನಂತೂ
ಹುಲಿಯನ್ನು
ಹುಲಿ-ಯಾ-ಗಿದ್ದರೂ
ಹುಲಿ-ಯಾ-ಗಿಯೇ
ಹುಲಿಯಿಂದ
ಹುಲಿಯೂ
ಹುಲಿಯೆಂದು
ಹುಲಿಯೇ
ಹುಲಿಯೊಂದು
ಹುಲುಸಾಗಿ
ಹುಲ್ಲು
ಹುಲ್ಲು-ಗಾ-ವ-ಲಿ-ನಲ್ಲಿ
ಹುಲ್ಲು-ಗಾ-ವ-ಲಿ-ನಲ್ಲಿದ್ದ
ಹುಲ್ಲು-ಗಾ-ವಲು
ಹುಲ್ಲೆಗಳು
ಹುಳಿ
ಹುಳು
ಹುಳುಕನ್ನು
ಹುಳು-ಹೆಜ್ಜೆಯ
ಹುಸಿಯ
ಹೂ
ಹೂಡಿ
ಹೂಡಿದ
ಹೂಡಿದಿರಿ
ಹೂಡಿದ್ದಾರೆ
ಹೂಡುವ
ಹೂತಿಡೋಣ
ಹೂದೋ-ಟ-ದಲ್ಲಿ
ಹೂವಿ-ನ-ಸು-ವಾ-ಸ-ನೆಯೂ
ಹೂವಿನಿಂದ
ಹೂವು
ಹೃತ್ಕ್ರಿಯೆ
ಹೃತ್ಕ್ರಿ-ಯೆ-ಯನ್ನು
ಹೃತ್ಪೂರ್ವಕ
ಹೃತ್ಪೂರ್ವ-ಕ-ವಾಗಿ
ಹೃದಯ
ಹೃದ-ಯಂಗ-ಮ-ವಾಗಿ
ಹೃದ-ಯ-ಕ-ಮ-ಲವೂ
ಹೃದಯಕ್ಕೆ
ಹೃದ-ಯ-ಗಳ
ಹೃದ-ಯ-ಗ-ಳನ್ನು
ಹೃದ-ಯ-ಗ-ಳನ್ನೇ
ಹೃದ-ಯ-ಗ-ಳಿವೆ
ಹೃದ-ಯ-ಚ-ಲ-ನೆ-ಯಲ್ಲಿ
ಹೃದಯದ
ಹೃದ-ಯ-ದಂತೆಯೇ
ಹೃದ-ಯ-ದಲಿ
ಹೃದ-ಯ-ದಲ್ಲಿ
ಹೃದ-ಯ-ದಲ್ಲಿ-ರುವ
ಹೃದ-ಯ-ದಲ್ಲಿ-ರು-ವು-ದಿಲ್ಲ
ಹೃದ-ಯ-ದಲ್ಲೂ
ಹೃದ-ಯ-ದಲ್ಲೇ
ಹೃದ-ಯ-ದಿಂದ
ಹೃದ-ಯ-ಪದ್ಮ-ದ-ಲ-ದಲಿ
ಹೃದ-ಯ-ಪ-ರಿ-ಶುದ್ಧಿಯೇ
ಹೃದ-ಯ-ಬೇ-ನೆ-ಯಿಂದ
ಹೃದ-ಯ-ವಂತ-ನಾ-ದರೆ
ಹೃದ-ಯ-ವಂತರೂ
ಹೃದ-ಯ-ವನು
ಹೃದ-ಯ-ವನ್ನು
ಹೃದ-ಯ-ವಿದ್ರಾ-ವಕ
ಹೃದ-ಯ-ವಿ-ರುತ್ತದೆ
ಹೃದಯವು
ಹೃದ-ಯ-ವುಳ್ಳ
ಹೃದಯವೂ
ಹೃದಯವೇ
ಹೃದ-ಯ-ಶುದ್ಧಿ
ಹೃದ-ಯ-ಶುದ್ಧಿ-ಯಾ-ಗದೆ
ಹೃದ-ಯ-ಸಂಪನ್ನತೆ
ಹೃದ-ಯ-ಹೀ-ನ-ತೆ-ಇವೇ
ಹೃದ-ಯಾಂತ-ರಾ-ಳದ
ಹೃದ-ಯಾಂತ-ರಾ-ಳ-ದಲ್ಲಿ
ಹೃದ-ಯಾ-ಘಾ-ತ-ವಾಗಿ
ಹೃದ-ಯಿ-ಗಳ
ಹೃದ-ಯಿ-ಗ-ಳಾದ
ಹೃದ-ಯಿ-ಗಳು
ಹೃದ-ಯೇಸ್ಮ-ದಿಯೇ
ಹೃನ್ಮನ
ಹೃನ್ಮ-ನ-ಗ-ಳನ್ನು
ಹೃನ್ಮ-ನ-ಗಳು
ಹೃಷಿ-ಕೇ-ಶದ
ಹೆಂಗಸನ್ನು
ಹೆಂಗ-ಸ-ರನ್ನು
ಹೆಂಗ-ಸ-ರನ್ನೂ
ಹೆಂಗ-ಸ-ರಲ್ಲಿ
ಹೆಂಗಸರು
ಹೆಂಗಸರೂ
ಹೆಂಗಸಾಗಿ
ಹೆಂಗಸಿನ
ಹೆಂಗಸು
ಹೆಂಗಸೂ
ಹೆಂಗ-ಸೊಬ್ಬಳು
ಹೆಂಡತಿ
ಹೆಂಡತಿಗೆ
ಹೆಂಡತಿಯ
ಹೆಂಡ-ತಿ-ಯನ್ನು
ಹೆಂಡ-ತಿ-ಯಾಗಿ
ಹೆಂಡ-ತಿ-ಯಾ-ಗು-ವ-ವಳು
ಹೆಂಡ-ತಿ-ಯೊ-ಡನೆ
ಹೆಂಡಿರು
ಹೆಗಲ
ಹೆಗ-ಲ-ಮೇ-ಲೇ-ರಿ-ಸಿ-ಕೊಂಡು
ಹೆಗಲಿಗೆ
ಹೆಗ-ಲು-ಕೊಟ್ಟು
ಹೆಗ್ಗ-ಳಿ-ಕೆ-ಯನ್ನು
ಹೆಗ್ಗು-ರು-ತಲ್ಲವೇ
ಹೆಚ್
ಹೆಚ್ಚಲಿ
ಹೆಚ್ಚಲು
ಹೆಚ್ಚಾಗಲು
ಹೆಚ್ಚಾಗಿ
ಹೆಚ್ಚಾಗಿದೆ
ಹೆಚ್ಚಾ-ಗಿ-ರು-ವು-ದನ್ನು
ಹೆಚ್ಚಾಯಿತು
ಹೆಚ್ಚಿ
ಹೆಚ್ಚಿ-ನ-ವರು
ಹೆಚ್ಚಿತು
ಹೆಚ್ಚಿದಂತೆ
ಹೆಚ್ಚಿ-ದಂತೆಲ್ಲ
ಹೆಚ್ಚಿದಷ್ಟೂ
ಹೆಚ್ಚಿದೆ
ಹೆಚ್ಚಿ-ದೆ-ಇದು
ಹೆಚ್ಚಿದೆಯೇ
ಹೆಚ್ಚಿದ್ದಲ್ಲಿ
ಹೆಚ್ಚಿನ
ಹೆಚ್ಚಿ-ನಂಶ-ವನ್ನು
ಹೆಚ್ಚಿ-ನ-ದಲ್ಲ
ಹೆಚ್ಚಿನದು
ಹೆಚ್ಚಿ-ನ-ದೇನೂ
ಹೆಚ್ಚಿ-ನ-ವ-ನಲ್ಲ
ಹೆಚ್ಚಿ-ನ-ವರ
ಹೆಚ್ಚಿ-ನ-ವ-ರಿಗೆ
ಹೆಚ್ಚಿ-ನ-ವರು
ಹೆಚ್ಚಿ-ನ-ವು-ಗ-ಳನ್ನು
ಹೆಚ್ಚಿ-ರ-ಬೇಕು
ಹೆಚ್ಚಿ-ರು-ವುದು
ಹೆಚ್ಚಿ-ಸ-ಬಲ್ಲದು
ಹೆಚ್ಚಿ-ಸ-ಬಲ್ಲವೇ
ಹೆಚ್ಚಿಸಿ
ಹೆಚ್ಚಿ-ಸಿ-ಕೊಂಡು
ಹೆಚ್ಚಿ-ಸಿ-ಕೊಳ್ಳದೇ
ಹೆಚ್ಚಿ-ಸಿ-ಕೊಳ್ಳ-ಬ-ಹುದು
ಹೆಚ್ಚಿ-ಸಿ-ಕೊಳ್ಳ-ಬೇ-ಕಲ್ಲವೇ
ಹೆಚ್ಚಿ-ಸಿ-ಕೊಳ್ಳ-ಬೇಕು
ಹೆಚ್ಚಿ-ಸಿ-ದಂತೆ
ಹೆಚ್ಚಿ-ಸಿ-ದವು
ಹೆಚ್ಚಿಸಿದೆ
ಹೆಚ್ಚಿ-ಸಿದ್ದಾನೆ
ಹೆಚ್ಚಿ-ಸುತ್ತ-ಲಿವೆ
ಹೆಚ್ಚಿ-ಸುತ್ತಿದ್ದಾರೆ
ಹೆಚ್ಚಿಸುವ
ಹೆಚ್ಚಿ-ಸು-ವ-ವರೇ
ಹೆಚ್ಚಿ-ಸು-ವು-ದಕ್ಕಾಗಿ
ಹೆಚ್ಚು
ಹೆಚ್ಚು-ಕ-ಡಿಮೆ
ಹೆಚ್ಚುಕಾಲ
ಹೆಚ್ಚು-ಗಾ-ರಿಕೆ
ಹೆಚ್ಚುತ್ತ
ಹೆಚ್ಚುತ್ತದೆ
ಹೆಚ್ಚುತ್ತ-ಲಿ-ರುತ್ತದೆ
ಹೆಚ್ಚುತ್ತ-ಲಿವೆ
ಹೆಚ್ಚುತ್ತಲೇ
ಹೆಚ್ಚುತ್ತವೆ
ಹೆಚ್ಚುತ್ತಿತ್ತು
ಹೆಚ್ಚುತ್ತಿದೆ
ಹೆಚ್ಚುತ್ತಿ-ದೆ-ಯಲ್ಲ
ಹೆಚ್ಚುತ್ತಿ-ರು-ವಾಗ
ಹೆಚ್ಚುಮಂದಿ
ಹೆಚ್ಚುವಂತೆ
ಹೆಚ್ಚು-ಹೆಚ್ಚಾಗಿ
ಹೆಚ್ಚುಹೆಚ್ಚು
ಹೆಚ್ಚೆಂದರೆ
ಹೆಚ್ಚೇ-ನಲ್ಲ-ಒಂದೆ-ರಡು
ಹೆಚ್ಚೇನೂ
ಹೆಜ್ಜೆ
ಹೆಜ್ಜೆಯ
ಹೆಜ್ಜೆಯಿಂದ
ಹೆಜ್ಜೆ-ಯಿಂದಲೇ
ಹೆಜ್ಜೆಯಿಟ್ಟು
ಹೆಜ್ಜೆ-ಯಿ-ಡುತ್ತ
ಹೆಜ್ಜೆಯೇ
ಹೆಜ್ಜೆ-ಹೆಜ್ಜೆಗೂ
ಹೆಜ್ಜೆ-ಹೆಜ್ಜೆ-ಯಾಗಿ
ಹೆಜ್ಜೇನು
ಹೆಡೆ
ಹೆಡೆಬಿಚ್ಚಿ
ಹೆಡೆ-ಯ-ಡಿ-ಯಲ್ಲಿ
ಹೆಡೆಯನ್ನು
ಹೆಡೆಯನ್ನೂ
ಹೆಡೆ-ಯಾ-ಡ-ತೊ-ಡ-ಗಿತು
ಹೆಣ-ಗ-ಬೇ-ಕಿಲ್ಲ
ಹೆಣ-ಗಾ-ಡುತ್ತಿದ್ದೇವೆ
ಹೆಣ-ಗುತ್ತದೆ
ಹೆಣ-ಗುತ್ತಿದ್ದ
ಹೆಣೆದು
ಹೆಣೆ-ದು-ಕೊಳ್ಳುತ್ತಾ
ಹೆಣೆ-ಯುತ್ತಿ-ರುತ್ತೇವೆ
ಹೆಣ್ಣು
ಹೆಣ್ಣು-ಗಂಡು-ಗಳ
ಹೆಣ್ಣುಗಳ
ಹೆಣ್ಣು-ಮಕ್ಕಳ
ಹೆಣ್ಣು-ಮಕ್ಕಳು
ಹೆಣ್ಣುಮಗು
ಹೆತ್ತ-ಕ-ರು-ಳನ್ನು
ಹೆತ್ತ-ಕ-ರು-ಳಿಗೆ
ಹೆತ್ತವರ
ಹೆತ್ತವರು
ಹೆತ್ತವರೂ
ಹೆತ್ತು
ಹೆತ್ತುಹೊತ್ತು
ಹೆದ-ರ-ಕೂ-ಡದು
ಹೆದ-ರ-ದಿರು
ಹೆದರದೇ
ಹೆದ-ರ-ಬೇ-ಕಾ-ಗಿಲ್ಲ
ಹೆದ-ರ-ಬೇ-ಕಾದ
ಹೆದ-ರ-ಬೇ-ಕಿಲ್ಲ
ಹೆದ-ರ-ಬೇಡ
ಹೆದರಿ
ಹೆದರಿಕೆ
ಹೆದ-ರಿ-ಕೆಗೆ
ಹೆದ-ರಿ-ಕೆಯ
ಹೆದ-ರಿ-ಕೆ-ಯಿಂದ
ಹೆದ-ರಿ-ಕೆಯು
ಹೆದ-ರಿ-ಕೊಂಡು
ಹೆದ-ರಿ-ಕೊಂಡೇ
ಹೆದ-ರಿ-ಕೊಳ್ಳು-ವಂತೆ
ಹೆದ-ರಿ-ಕೊಳ್ಳು-ವುದು
ಹೆದ-ರಿ-ದರೂ
ಹೆದ-ರಿ-ದರೆ
ಹೆದರಿದೆ
ಹೆದ-ರಿ-ಸ-ಬಾ-ರ-ದಾ-ಗಿತ್ತು
ಹೆದ-ರಿ-ಸಲು
ಹೆದರಿಸಿ
ಹೆದ-ರಿ-ಸು-ವು-ದಿಲ್ಲ
ಹೆದ-ರುತ್ತಲೆ
ಹೆದ-ರುತ್ತಿದ್ದರೆ
ಹೆದ-ರುತ್ತಿಲ್ಲ
ಹೆದ-ರು-ವಂತಾ-ಗಿದೆ
ಹೆದ-ರು-ವಿ-ರಾ-ದರೆ
ಹೆದ-ರು-ವು-ದ-ರಿಂದ
ಹೆದ-ರು-ವು-ದಾ-ಗಲಿ
ಹೆದ-ರು-ವು-ದಿಲ್ಲ
ಹೆದ-ರು-ವುದು
ಹೆದ-ರು-ವು-ದೇಕೆ
ಹೆದ್ದಾರಿ
ಹೆದ್ದೆ-ರೆ-ಗ-ಳಲ್ಲಿ
ಹೆಪ್ಪು-ಗಟ್ಟಿದ
ಹೆಬ್ಬಾ-ಗಿ-ಲಿಗೆ
ಹೆಬ್ಬು-ಲಿ-ಯಂತೆ
ಹೆಮ್ಮೆ
ಹೆಮ್ಮೆಗೆ
ಹೆರ-ತ-ನದ
ಹೆರಾಲ್ಡ್
ಹೆರಿಗೆಯ
ಹೆರಿ-ಗೆ-ಯಾ-ಗು-ವಾಗ
ಹೆರಿ-ಗೆ-ಯಾ-ದೊ-ಡ-ನೆಯೆ
ಹೆರಿ-ಗೆ-ಯಾ-ದೊ-ಡ-ನೆಯೇ
ಹೆಲನ್
ಹೆಸರನ್ನು
ಹೆಸರನ್ನೂ
ಹೆಸರಲ್ಲಿ
ಹೆಸರಾಂತ
ಹೆಸರಾದ
ಹೆಸ-ರಾ-ದರೋ
ಹೆಸರಿಗೆ
ಹೆಸರಿಡಿ
ಹೆಸರಿನ
ಹೆಸ-ರಿ-ನಲ್ಲಿ
ಹೆಸ-ರಿ-ನಲ್ಲೇ
ಹೆಸ-ರಿ-ನಿಂದ
ಹೆಸ-ರಿ-ಸಲಿ
ಹೆಸ-ರಿ-ಸಲು
ಹೆಸರಿಸಿ
ಹೆಸ-ರಿ-ಸುತ್ತಾನೆ
ಹೆಸ-ರಿ-ಸು-ವುದು
ಹೆಸರು
ಹೆಸ-ರು-ಗ-ಳನ್ನು
ಹೆಸ-ರು-ಗ-ಳನ್ನೂ
ಹೆಸ-ರು-ಗ-ಳಿಂದ
ಹೆಸ-ರು-ಗ-ಳಿ-ಸಿ-ದ-ವನು
ಹೆಸ-ರು-ಗಳು
ಹೆಸ-ರು-ವಾ-ಸಿ-ಯಾ-ಗಿದ್ದ
ಹೆಸರೂ
ಹೆಸರೇ
ಹೇ
ಹೇಗನ್ನಿ-ಸ-ಬ-ಹುದು
ಹೇಗಾದರೂ
ಹೇಗಿತ್ತು
ಹೇಗಿದೆ
ಹೇಗಿ-ರ-ಬ-ಹುದು
ಹೇಗಿ-ರುತ್ತದೆ
ಹೇಗಿ-ರುತ್ತಾ-ನಪ್ಪಾ
ಹೇಗಿ-ರು-ವು-ದೆಂಬು-ದರ
ಹೇಗೂ
ಹೇಗೆ
ಹೇಗೆಂದರೆ
ಹೇಗೆ-ಎಂಬುದು
ಹೇಗೆ-ನಿ-ಸಿತು
ಹೇಗೆನ್ನು-ವಿರಾ
ಹೇಗೇ
ಹೇಗೊ
ಹೇಗೋ
ಹೇಡಿಗಳು
ಹೇಡಿಯಂತೆ
ಹೇಡಿ-ಯನ್ನಾಗಿ
ಹೇಡ್
ಹೇಯ
ಹೇಯ-ಕೃತ್ಯ-ವನ್ನೆ-ಸಗಿ
ಹೇಯವಾದ
ಹೇಯವೂ
ಹೇರಳ
ಹೇರ-ಳ-ವಾ-ಗಿಯೇ
ಹೇರಿ
ಹೇರಿದರೆ
ಹೇರುವ
ಹೇಳ
ಹೇಳ-ತಕ್ಕದ್ದು-ಸರ್ವ-ಮ-ತ-ದ-ವ-ರಲ್ಲೂ
ಹೇಳ-ತೀ-ರದು
ಹೇಳ-ತೊ-ಡ-ಗಿದ
ಹೇಳದ
ಹೇಳ-ದ-ವರು
ಹೇಳ-ದಿದ್ದರೂ
ಹೇಳದೆ
ಹೇಳದೇ
ಹೇಳಬಲ್ಲ
ಹೇಳ-ಬಲ್ಲ-ನಾ-ದರೆ
ಹೇಳ-ಬಲ್ಲನೇ
ಹೇಳ-ಬಲ್ಲ-ವ-ನಾ-ಗಿದ್ದ
ಹೇಳ-ಬಲ್ಲ-ವ-ರಾ-ಗಿದ್ದರು
ಹೇಳ-ಬಲ್ಲೆವೇ
ಹೇಳ-ಬ-ಹು-ದಾದ
ಹೇಳ-ಬ-ಹುದು
ಹೇಳ-ಬ-ಹು-ದೆಂದು
ಹೇಳ-ಬ-ಹುದೋ
ಹೇಳ-ಬಾ-ರದು
ಹೇಳ-ಬಾ-ರ-ದು-ಎಂಬ
ಹೇಳ-ಬೇ-ಕಾ-ಗಿದೆ
ಹೇಳ-ಬೇ-ಕಾ-ಗುತ್ತದೆ
ಹೇಳ-ಬೇ-ಕಾ-ದು-ದಿಲ್ಲ
ಹೇಳ-ಬೇ-ಕಿಲ್ಲ
ಹೇಳಬೇಕು
ಹೇಳ-ಬೇ-ಕು-ಎಂಬು-ದನ್ನು
ಹೇಳಬೇಕೆ
ಹೇಳ-ಬೇ-ಕೆಂದಿದ್ದರೆ
ಹೇಳ-ಬೇ-ಕೆಂದು
ಹೇಳ-ಬೇ-ಕೆಂದೂ
ಹೇಳಬೇಕೇ
ಹೇಳಬೇಡಿ
ಹೇಳ-ಲಾ-ಗ-ದಿದ್ದರೂ
ಹೇಳ-ಲಾ-ಗಿದೆ
ಹೇಳಲಾದ
ಹೇಳಲಾರ
ಹೇಳ-ಲಾ-ರದ
ಹೇಳ-ಲಾ-ರ-ದಷ್ಟು
ಹೇಳ-ಲಾ-ರ-ನಾ-ದರೂ
ಹೇಳ-ಲಾ-ರರು
ಹೇಳ-ಲಾ-ರ-ರೆಂದು
ಹೇಳಲಾರೆ
ಹೇಳಲಿ
ಹೇಳಲಿಲ್ಲ
ಹೇಳ-ಲಿಲ್ಲ-ವಲ್ಲ
ಹೇಳಲು
ಹೇಳಲ್ಪಟ್ಟ
ಹೇಳ-ಹೆ-ಸ-ರಿಲ್ಲ-ದಂತೆ
ಹೇಳ-ಹೊ-ರಟ
ಹೇಳ-ಹೊ-ರ-ಟದ್ದು
ಹೇಳ-ಹೊ-ರ-ಟಿದ್ದ-ನಷ್ಟೆ
ಹೇಳ-ಹೊ-ರ-ಟಿಲ್ಲ
ಹೇಳಿ
ಹೇಳಿಕೆ
ಹೇಳಿ-ಕೆ-ಅಟ್ಲಾಂಟಿಸ್ನ
ಹೇಳಿ-ಕೆ-ಗಳ
ಹೇಳಿ-ಕೆ-ಗ-ಳನ್ನು
ಹೇಳಿ-ಕೆ-ಗ-ಳನ್ನೇ
ಹೇಳಿ-ಕೆ-ಗ-ಳಲ್ಲಿ
ಹೇಳಿ-ಕೆ-ಗ-ಳಲ್ಲಿನ
ಹೇಳಿ-ಕೆ-ಗ-ಳಿ-ಗಾಗಿ
ಹೇಳಿ-ಕೆ-ಗ-ಳಿಗೂ
ಹೇಳಿ-ಕೆ-ಗ-ಳಿಗೆ
ಹೇಳಿ-ಕೆ-ಗಳು
ಹೇಳಿ-ಕೆ-ಗ-ಳೊಂದಿಗೆ
ಹೇಳಿಕೆಯ
ಹೇಳಿ-ಕೆ-ಯಂತೆ
ಹೇಳಿ-ಕೆ-ಯನ್ನಿ
ಹೇಳಿ-ಕೆ-ಯನ್ನು
ಹೇಳಿ-ಕೆ-ಯನ್ನೂ
ಹೇಳಿ-ಕೆ-ಯಲ್ಲಿ
ಹೇಳಿ-ಕೆ-ಯಲ್ಲಿದ್ದಂತೆ
ಹೇಳಿ-ಕೆ-ಯಿಂದ
ಹೇಳಿಕೆಯು
ಹೇಳಿಕೊಂಡ
ಹೇಳಿ-ಕೊಂಡ-ನೆನ್ನಿ
ಹೇಳಿ-ಕೊಂಡರು
ಹೇಳಿ-ಕೊಂಡಳು
ಹೇಳಿಕೊಂಡು
ಹೇಳಿಕೊಟ್ಟ
ಹೇಳಿ-ಕೊಟ್ಟರು
ಹೇಳಿ-ಕೊ-ಡದೇ
ಹೇಳಿ-ಕೊ-ಡ-ಬ-ಹುದು
ಹೇಳಿ-ಕೊ-ಡ-ಲಿಲ್ಲ
ಹೇಳಿ-ಕೊ-ಡಲು
ಹೇಳಿಕೊಡು
ಹೇಳಿ-ಕೊ-ಡುತ್ತೀಯಾ
ಹೇಳಿ-ಕೊ-ಡುತ್ತೀರಾ
ಹೇಳಿ-ಕೊ-ಡುತ್ತೀ-ರಾ-ಎಂದು
ಹೇಳಿ-ಕೊ-ಡುತ್ತೇನೆ
ಹೇಳಿ-ಕೊ-ಡು-ವರು
ಹೇಳಿ-ಕೊಳ್ಳದೆ
ಹೇಳಿ-ಕೊಳ್ಳ-ಲಾ-ದರೂ
ಹೇಳಿ-ಕೊಳ್ಳಲು
ಹೇಳಿ-ಕೊಳ್ಳಲೂ
ಹೇಳಿ-ಕೊಳ್ಳುತ್ತಿದ್ದ
ಹೇಳಿ-ಕೊಳ್ಳುತ್ತಿದ್ದಳು
ಹೇಳಿ-ಕೊಳ್ಳುವ
ಹೇಳಿ-ಕೊಳ್ಳು-ವಂತಹ
ಹೇಳಿ-ಕೊಳ್ಳು-ವಂತಿ-ರ-ಲಿಲ್ಲ
ಹೇಳಿತು
ಹೇಳಿದ
ಹೇಳಿ-ದಂತಾ-ಯಿತು
ಹೇಳಿದಂತೆ
ಹೇಳಿದಈ
ಹೇಳಿ-ದ-ನಂತೆ
ಹೇಳಿದರು
ಹೇಳಿದರೂ
ಹೇಳಿದರೆ
ಹೇಳಿ-ದ-ರೆಂದರೆ
ಹೇಳಿ-ದ-ರೆನ್ನಿ
ಹೇಳಿದರೇ
ಹೇಳಿದಳು
ಹೇಳಿ-ದ-ವನ
ಹೇಳಿ-ದ-ವ-ರಾರು
ಹೇಳಿ-ದ-ವರು
ಹೇಳಿದಾಗ
ಹೇಳಿ-ದು-ದನ್ನು
ಹೇಳಿ-ದು-ದರ
ಹೇಳಿದೆ
ಹೇಳಿ-ದೆ-ಅಟ್ಟಕ್ಕೇ-ರಿದ
ಹೇಳಿದೆನೇ
ಹೇಳಿದ್ದ
ಹೇಳಿದ್ದಕ್ಕೂ
ಹೇಳಿದ್ದಕ್ಕೆ
ಹೇಳಿದ್ದನ್ನು
ಹೇಳಿದ್ದ-ರಲ್ಲವೇ
ಹೇಳಿದ್ದ-ರ-ವರು
ಹೇಳಿದ್ದರು
ಹೇಳಿದ್ದ-ರು-ಭ-ಗ-ವಂತನ
ಹೇಳಿದ್ದರೆ
ಹೇಳಿದ್ದಾನೆ
ಹೇಳಿದ್ದಾರೆ
ಹೇಳಿದ್ದಿಷ್ಟೆ
ಹೇಳಿದ್ದು
ಹೇಳಿದ್ದುಂಟು
ಹೇಳಿದ್ದೆ
ಹೇಳಿದ್ದೇನು
ಹೇಳಿ-ಬಿಟ್ಟರು
ಹೇಳಿ-ಬಿ-ಡು-ವುದು
ಹೇಳಿಯಾನು
ಹೇಳಿಯಾರು
ಹೇಳಿ-ಯೇ-ಬಿಟ್ಟರು
ಹೇಳಿರದ
ಹೇಳಿ-ರ-ದಿದ್ದರೂ
ಹೇಳಿರುವ
ಹೇಳಿ-ರು-ವಂತೆ
ಹೇಳಿಲ್ಲ
ಹೇಳಿ-ಸ-ಬೇ-ಕೆಂಬುದು
ಹೇಳಿ-ಸಿ-ಕೊಳ್ಳಲು
ಹೇಳಿ-ಸಿ-ದರು
ಹೇಳೀತು
ಹೇಳು
ಹೇಳುತ್ತ
ಹೇಳುತ್ತದೆ
ಹೇಳುತ್ತ-ದೆಂಬು-ದನ್ನು
ಹೇಳುತ್ತ-ಲಿದ್ದೇನೆ
ಹೇಳುತ್ತಲೇ
ಹೇಳುತ್ತವೆ
ಹೇಳುತ್ತಾನೆ
ಹೇಳುತ್ತಾ-ನೆಂದರೆ
ಹೇಳುತ್ತಾ-ನೆ-ಮ-ನುಷ್ಯ
ಹೇಳುತ್ತಾರೆ
ಹೇಳುತ್ತಾ-ರೆಂಬು-ದರ
ಹೇಳುತ್ತಾ-ರೆ-ಎನ್ನು-ವು-ದರ
ಹೇಳುತ್ತಾ-ರೇನು
ಹೇಳುತ್ತಾಳೆ
ಹೇಳುತ್ತಿತ್ತು
ಹೇಳುತ್ತಿದ್ದ
ಹೇಳುತ್ತಿದ್ದಂತೆ
ಹೇಳುತ್ತಿದ್ದರು
ಹೇಳುತ್ತಿದ್ದಳು
ಹೇಳುತ್ತಿದ್ದ-ಹೀ-ರುತ್ತೇನೆ
ಹೇಳುತ್ತಿದ್ದಾರೆ
ಹೇಳುತ್ತಿದ್ದಾ-ರೆಯೆ
ಹೇಳುತ್ತಿದ್ದೀರಿ
ಹೇಳುತ್ತಿದ್ದೇ-ನಷ್ಟೆ
ಹೇಳುತ್ತಿದ್ದೇನೆ
ಹೇಳುತ್ತಿ-ರ-ಬೇ-ಕೆಂದು
ಹೇಳುತ್ತಿ-ರ-ಲಿಲ್ಲ
ಹೇಳುತ್ತಿ-ರುತ್ತಾನೆ
ಹೇಳುತ್ತಿ-ರುತ್ತಾರೆ
ಹೇಳುತ್ತಿಲ್ಲ
ಹೇಳುತ್ತಿ-ವೆ-ಯೆಂಬು-ದನ್ನು
ಹೇಳುತ್ತೀರಿ
ಹೇಳುತ್ತೇನೆ
ಹೇಳುವ
ಹೇಳು-ವಂತಾ-ಗಿದೆ
ಹೇಳು-ವಂತಾ-ಗು-ವುದೋ
ಹೇಳು-ವಂತಿಲ್ಲ
ಹೇಳುವಂತೆ
ಹೇಳು-ವಂಥ-ದಲ್ಲ
ಹೇಳುವನು
ಹೇಳು-ವ-ವನು
ಹೇಳು-ವ-ವ-ರಲ್ಲ
ಹೇಳು-ವ-ವರು
ಹೇಳು-ವ-ವ-ರೆಗೆ
ಹೇಳುವಾಗ
ಹೇಳುವು
ಹೇಳು-ವು-ದಕ್ಕೆ
ಹೇಳು-ವು-ದನ್ನು
ಹೇಳು-ವು-ದ-ರಲ್ಲಿ
ಹೇಳು-ವು-ದ-ರಿಂದ
ಹೇಳು-ವು-ದಾ-ದರೆ
ಹೇಳು-ವು-ದಿಲ್ಲ
ಹೇಳುವುದು
ಹೇಳು-ವು-ದುಂಟು
ಹೇಳುವುದೂ
ಹೇಳುವುದೇ
ಹೇಳೆನದೋ
ಹೇಳೋಣ
ಹೇವ-ರಿ-ಕೆ-ಯನ್ನುಂಟು-ಮಾ-ಡು-ವಂಥ
ಹೇಸದ
ಹೈಂ
ಹೈಟ್
ಹೈಡ್ರೋಜನ್
ಹೈಡ್ರೋ-ಫೋ-ಬಿಯಾ
ಹೈದನೂ
ಹೈಸ್ಕೂಲು
ಹೈಸ್ಕೂ-ಲು-ಗ-ಳಲ್ಲಿ
ಹೈಸ್ಕೂಲ್
ಹೊಂಚು
ಹೊಂದದ
ಹೊಂದ-ದಿ-ರು-ವುದು
ಹೊಂದದೇ
ಹೊಂದ-ಬಲ್ಲರು
ಹೊಂದ-ಬೇ-ಕಾ-ಯಿತು
ಹೊಂದ-ಬೇ-ಕೆಂದು
ಹೊಂದ-ಲಾ-ರರು
ಹೊಂದಲಿ
ಹೊಂದಲು
ಹೊಂದಾಣಿಕೆ
ಹೊಂದಾ-ಣಿ-ಕೆಯ
ಹೊಂದಿ
ಹೊಂದಿ-ಕೊಂಡಿವೆ
ಹೊಂದಿ-ಕೆ-ಯಾ-ಗುತ್ತಿದ್ದವು
ಹೊಂದಿಕೊಂಡ
ಹೊಂದಿ-ಕೊಂಡಿದೆ
ಹೊಂದಿ-ಕೊಂಡಿದ್ದ
ಹೊಂದಿ-ಕೊಂಡಿದ್ದರೂ
ಹೊಂದಿ-ಕೊಂಡಿದ್ದರೆ
ಹೊಂದಿ-ಕೊಂಡಿದ್ದಾನೆ
ಹೊಂದಿ-ಕೊಂಡಿ-ರುತ್ತದೆ
ಹೊಂದಿ-ಕೊಂಡಿ-ರುತ್ತವೆ
ಹೊಂದಿ-ಕೊಂಡಿಲ್ಲ
ಹೊಂದಿಕೊಂಡು
ಹೊಂದಿಕೊಂಡೇ
ಹೊಂದಿ-ಕೊಳ್ಳ-ದಾ-ದಾಗ
ಹೊಂದಿ-ಕೊಳ್ಳ-ಲಾ-ಗದ
ಹೊಂದಿ-ಕೊಳ್ಳಲು
ಹೊಂದಿ-ಕೊಳ್ಳುತ್ತ
ಹೊಂದಿ-ಕೊಳ್ಳುತ್ತಾ
ಹೊಂದಿ-ಕೊಳ್ಳುತ್ತಾನೆ
ಹೊಂದಿ-ಕೊಳ್ಳುವ
ಹೊಂದಿ-ಕೊಳ್ಳು-ವಂತೆ
ಹೊಂದಿತು
ಹೊಂದಿದ
ಹೊಂದಿದನು
ಹೊಂದಿದರು
ಹೊಂದಿ-ದ-ವ-ರಲ್ಲಿ
ಹೊಂದಿ-ದ-ವ-ರಾ-ಗಿಯೇ
ಹೊಂದಿ-ದ-ವರು
ಹೊಂದಿದಾಗ
ಹೊಂದಿ-ದಿ-ರೆಂಬುದು
ಹೊಂದಿ-ದು-ದನ್ನು
ಹೊಂದಿ-ದು-ದಾ-ಗಿಯೂ
ಹೊಂದಿದುದು
ಹೊಂದಿದೆ
ಹೊಂದಿದ್ದ
ಹೊಂದಿದ್ದರೂ
ಹೊಂದಿದ್ದಾಗಿ
ಹೊಂದಿದ್ದಾನೆ
ಹೊಂದಿದ್ದಾರೆ
ಹೊಂದಿದ್ದಾ-ರೆಯೇ
ಹೊಂದಿದ್ದೂ
ಹೊಂದಿರಲಿ
ಹೊಂದಿ-ರುತ್ತದೆ
ಹೊಂದಿರುವ
ಹೊಂದಿಲ್ಲ
ಹೊಂದಿವೆ
ಹೊಂದಿ-ಸಿ-ಕೊಂಡಿ-ರ-ಬ-ಹುದು
ಹೊಂದಿ-ಸಿ-ಕೊಳ್ಳು-ವುದೇ
ಹೊಂದೀತೆಂಬ
ಹೊಂದುತ್ತದೆ
ಹೊಂದುತ್ತ-ಲಿದೆ
ಹೊಂದುತ್ತವೆ
ಹೊಂದುತ್ತಾನೆ
ಹೊಂದುತ್ತಾರೆ
ಹೊಂದುತ್ತಿದ್ದರು
ಹೊಂದುವ
ಹೊಂದು-ವ-ನೆಂಬುದು
ಹೊಂದುವಾಗ
ಹೊಂದುವಿಕೆ
ಹೊಂದು-ವು-ದಕ್ಕಾಗಿ
ಹೊಂದು-ವು-ದಕ್ಕಿಂತ
ಹೊಂದು-ವು-ದಿಲ್ಲ
ಹೊಂದುವುದು
ಹೊಕ್ಕಾ-ಗಿ-ರುತ್ತವೆ
ಹೊಕ್ಕಾಗಿವೆ
ಹೊಕ್ಕಿಹುದು
ಹೊಕ್ಕು
ಹೊಗಳದ
ಹೊಗ-ಳ-ಬ-ಹುದು
ಹೊಗ-ಳ-ಬಾ-ರ-ದೆಂದು
ಹೊಗ-ಳ-ಬೇಕು
ಹೊಗ-ಳ-ಲಾ-ಗುತ್ತಿದೆ
ಹೊಗ-ಳಲ್ಪಟ್ಟ
ಹೊಗಳಿ
ಹೊಗಳಿಕೆ
ಹೊಗ-ಳಿ-ಕೆಗೆ
ಹೊಗ-ಳಿ-ಕೆಯ
ಹೊಗ-ಳಿ-ಕೆ-ಯನ್ನು
ಹೊಗ-ಳಿ-ಕೆಯೇ
ಹೊಗ-ಳಿ-ದರೂ
ಹೊಗ-ಳಿ-ದರೆ
ಹೊಗ-ಳಿ-ರು-ವು-ದುಂಟು
ಹೊಗ-ಳುತ್ತಾರೆ
ಹೊಗ-ಳು-ವ-ವರು
ಹೊಗ-ಳು-ವ-ವರೆ
ಹೊಗ-ಳು-ವ-ಹಾ-ಗಿಲ್ಲ
ಹೊಗ-ಳು-ವುದು
ಹೊಗೆಬತ್ತಿ
ಹೊಗೆ-ಬತ್ತಿ-ಯನ್ನು
ಹೊಗೆಯ
ಹೊಗೆಯನ್ನು
ಹೊಟ್ಟನ್ನು
ಹೊಟ್ಟೆ
ಹೊಟ್ಟೆಕಿಚ್ಚು
ಹೊಟ್ಟೆ-ನೋ-ವಿ-ನಿಂದ
ಹೊಟ್ಟೆ-ಪಾ-ಡಿನ
ಹೊಟ್ಟೆಬಟ್ಟೆ
ಹೊಟ್ಟೆ-ಬಾ-ಕ-ತ-ನದ
ಹೊಟ್ಟೆ-ಬಾ-ಕ-ತ-ನ-ವೆಂಬ
ಹೊಟ್ಟೆಯ
ಹೊಟ್ಟೆಯಲ್ಲಿ
ಹೊಟ್ಟೆಯಿಂದ
ಹೊಟ್ಟೆ-ಸಂಕಟ
ಹೊಟ್ಟೆ-ಹುಣ್ಣಿನ
ಹೊಟ್ಟೆ-ಹೊ-ರೆ-ದು-ಕೊಳ್ಳಲು
ಹೊಡಿ
ಹೊಡೆತ
ಹೊಡೆ-ತ-ಬ-ಡಿ-ತ-ಗಳ
ಹೊಡೆ-ತ-ವನ್ನು
ಹೊಡೆದಂತೆ
ಹೊಡೆದರೂ
ಹೊಡೆ-ದ-ವ-ನಲ್ಲ
ಹೊಡೆದಾಟ
ಹೊಡೆ-ದಾ-ಡಿ-ಕೊಳ್ಳುತ್ತಿದ್ದಾ-ರಲ್ಲ
ಹೊಡೆ-ದಾ-ಡುವ
ಹೊಡೆದು
ಹೊಡೆ-ದು-ಕೊಂಡು
ಹೊಡೆ-ದು-ದನ್ನೂ
ಹೊಡೆ-ದು-ಬ-ಡಿದು
ಹೊಡೆ-ದೇ-ಬಿಟ್ಟ
ಹೊಡೆ-ದೋ-ಡಿ-ಸಲು
ಹೊಡೆ-ದೋ-ಡಿ-ಸಿತು
ಹೊಡೆಯದೆ
ಹೊಡೆಯದೇ
ಹೊಡೆಯಲು
ಹೊಡೆಯುತ್ತ
ಹೊಡೆ-ಯುತ್ತಿದ್ದಂತೆ
ಹೊಡೆಯುವ
ಹೊಡೆ-ಯು-ವಂತೆ
ಹೊಡೆ-ಸುತ್ತೇವೆ
ಹೊಣೆ
ಹೊಣೆ-ಗಾ-ರಿಕೆ
ಹೊಣೆ-ಗಾ-ರಿ-ಕೆಗೆ
ಹೊಣೆ-ಗಾ-ರಿ-ಕೆಯ
ಹೊಣೆ-ಗೇ-ಡಿ-ತ-ನವೇ
ಹೊಣೆಯನ್ನು
ಹೊಣೆ-ಯಾ-ಗುತ್ತೇನೆ
ಹೊಣೆಯೂ
ಹೊತ್ತವನ
ಹೊತ್ತಾಗಿದೆ
ಹೊತ್ತಾಯಿತು
ಹೊತ್ತಿ-ಕೊಂಡಿತು
ಹೊತ್ತಿಗೆ
ಹೊತ್ತಿಗೇ
ಹೊತ್ತಿಗೇನೇ
ಹೊತ್ತಿದ್ದಳು
ಹೊತ್ತಿದ್ದು
ಹೊತ್ತಿನ
ಹೊತ್ತಿನಲ್ಲಿ
ಹೊತ್ತಿನಲ್ಲೂ
ಹೊತ್ತಿನಲ್ಲೆ
ಹೊತ್ತಿನಲ್ಲೇ
ಹೊತ್ತಿಸಿ
ಹೊತ್ತಿಸಿತು
ಹೊತ್ತಿಸಿದ
ಹೊತ್ತಿ-ಸುತ್ತಾರೆ
ಹೊತ್ತಿ-ಸು-ವುದು
ಹೊತ್ತು
ಹೊತ್ತು-ಕೊಂಡಿದ್ದ
ಹೊತ್ತುಕೊಂಡು
ಹೊತ್ತುಬೇಕು
ಹೊತ್ತುಹಾಕಿ
ಹೊದಿ-ಕೆ-ಯಲ್ಲಿ
ಹೊದೆ-ದು-ಕೊಂಡ
ಹೊದೆ-ಯು-ವಷ್ಟು
ಹೊನಲನ್ನೆ
ಹೊನಲನ್ನೇ
ಹೊಮ್ಮ-ಬ-ಹುದು
ಹೊಮ್ಮುತ್ತಿದೆ
ಹೊಮ್ಮುವ
ಹೊರ
ಹೊರ-ಜ-ಗತ್ತಿನ
ಹೊರ-ಕ-ವಚ
ಹೊರಕ್ಕೆ
ಹೊರಗಡೆ
ಹೊರ-ಗ-ಡೆಯ
ಹೊರ-ಗ-ಡೆ-ಯಿಂದ
ಹೊರಗಣ
ಹೊರಗನ್ನು
ಹೊರಗಿನ
ಹೊರ-ಗಿ-ನಿಂದ
ಹೊರಗೂ
ಹೊರಗೆ
ಹೊರ-ಗೆ-ಡ-ಹಿ-ತುಈ
ಹೊರ-ಗೆ-ಳೆ-ಯಲು
ಹೊರ-ಗೊ-ಳಗೆ
ಹೊರ-ಚಾ-ಚು-ವಂತೆ
ಹೊರ-ಚಿಮ್ಮುತ್ತದೆ
ಹೊರ-ಚೆಲ್ಲಿತು
ಹೊರ-ಚೆಲ್ಲುವ
ಹೊರ-ಜ-ಗತ್ತಿನ
ಹೊರ-ಜ-ಗತ್ತಿ-ನಲ್ಲೇ
ಹೊರ-ಜ-ಗತ್ತಿ-ನಿಂದ
ಹೊರ-ಜ-ಗತ್ತಿ-ನೆ-ಡೆಗೆ
ಹೊರಟ
ಹೊರ-ಟ-ನಲ್ಲ-ಇ-ವ-ನೆಂಥ
ಹೊರಟರೂ
ಹೊರಟರೆ
ಹೊರ-ಟ-ವನ
ಹೊರ-ಟ-ವ-ರಿಗೆ
ಹೊರ-ಟ-ವರು
ಹೊರಟಾಗ
ಹೊರಟಿತು
ಹೊರ-ಟಿದ್ದರು
ಹೊರ-ಟಿದ್ದೇನೆ
ಹೊರಟಿಲ್ಲ
ಹೊರಟು
ಹೊರ-ಟು-ನಿಂತಾಗ
ಹೊರ-ಟು-ಹೋದ
ಹೊರ-ಟು-ಹೋ-ದರು
ಹೊರ-ಟು-ಹೋ-ಯಿತು
ಹೊರಟೆ
ಹೊರಟೇ
ಹೊರಡದ
ಹೊರಡದು
ಹೊರ-ಡ-ಬೇಕು
ಹೊರಡಲು
ಹೊರಡಿ
ಹೊರ-ಡುತ್ತದೆ
ಹೊರ-ಡುತ್ತಾನೆ
ಹೊರ-ಡುತ್ತಾರೆ
ಹೊರಡುವ
ಹೊರ-ಡು-ವಾಗ
ಹೊರ-ಡು-ವು-ದಕ್ಕೆ
ಹೊರ-ತ-ನದ
ಹೊರ-ತ-ರ-ಲಾ-ಯಿತು
ಹೊರ-ತ-ರುತ್ತದೆ
ಹೊರತಲ್ಲ
ಹೊರ-ತಳ್ಳ-ಬೇಕು
ಹೊರ-ತಳ್ಳುವ
ಹೊರತಾಗಿ
ಹೊರ-ತಾ-ಗಿಲ್ಲ
ಹೊರತು
ಹೊರ-ತೆ-ಗೆದು
ಹೊರ-ದೂ-ಡಿ-ತಲ್ಲವೇ
ಹೊರ-ದೇ-ಶ-ಗ-ಳಿಗೆ
ಹೊರ-ದೇ-ಶ-ದಲ್ಲಿ
ಹೊರ-ದೇ-ಶ-ದಿಂದ
ಹೊರ-ನ-ಡೆ-ಯುತ್ತಾರೆ
ಹೊರ-ಬ-ರುತ್ತಿತ್ತು
ಹೊರಬಿತ್ತು
ಹೊರ-ಬಿದ್ದ-ರಂತೆ
ಹೊರ-ಭಾ-ಗಕ್ಕೆ
ಹೊರ-ಭಾ-ಗ-ದಲ್ಲಿ
ಹೊರ-ಮ-ನಸ್ಸಿ-ನಲ್ಲಿ
ಹೊರ-ಮ-ನಸ್ಸು
ಹೊರ-ಮು-ಖ-ವಾಗಿ
ಹೊರರೂಪ
ಹೊರ-ಲಾ-ರದ
ಹೊರಲು
ಹೊರಳಾಡಿ
ಹೊರ-ಳಾ-ಡಿ-ದರು
ಹೊರ-ಳಾ-ಡು-ವು-ದನ್ನು
ಹೊರ-ವ-ಲ-ಯದ
ಹೊರ-ವ-ಲ-ಯ-ದಲ್ಲಿ-ರುವ
ಹೊರ-ಸೂ-ಸುತ್ತಾ
ಹೊರ-ಸೂ-ಸುತ್ತಿ-ರು-ವಂತೆ
ಹೊರ-ಸೂ-ಸುವ
ಹೊರ-ಸೂ-ಸು-ವಂತೆ
ಹೊರ-ಹಾ-ಕಿತ್ತು
ಹೊರ-ಹೊಮ್ಮಿದ
ಹೊರ-ಹೊಮ್ಮಿ-ದುದು
ಹೊರ-ಹೊಮ್ಮಿದೆ
ಹೊರ-ಹೊಮ್ಮುವ
ಹೊರ-ಹೊಮ್ಮು-ವಂತೆ
ಹೊರ-ಹೊಮ್ಮು-ವು-ದುಂಟು
ಹೊರ-ಹೊ-ರಟು
ಹೊರಿಸದೇ
ಹೊರಿಸಿ
ಹೊರಿ-ಸಿ-ದರೆ
ಹೊರಿ-ಸುತ್ತಾರೆ
ಹೊರು
ಹೊರುವಂತೆ
ಹೊರೆ
ಹೊರೆಯ
ಹೊರೆಯನ್ನು
ಹೊರೆ-ಯಲ್ಲದ
ಹೊರೆ-ಯ-ವರ
ಹೊರೆ-ಯಾ-ಗು-ವುದೂ
ಹೊರೆ-ಯಾ-ದಂತಾಗಿ
ಹೊರೆಯು
ಹೊರೆಯುವ
ಹೊಲ
ಹೊಲ-ಗ-ಳಲ್ಲಿ
ಹೊಲದ
ಹೊಲದಲ್ಲಿ
ಹೊಲಬನ್ನೇ
ಹೊಲಮನೆ
ಹೊಲಸನ್ನು
ಹೊಲಸಿನ
ಹೊಲಸು
ಹೊಲ-ಸು-ಮಟ್ಟ-ದಲ್ಲಿ
ಹೊಲಿ-ಗೆ-ಗಾ-ರರು
ಹೊಲಿಗೆಯ
ಹೊಲಿದು
ಹೊಳಹೂ
ಹೊಳೆ
ಹೊಳೆಯಲ್ಲಿ
ಹೊಳೆ-ಯುತ್ತವೆ
ಹೊಳೆ-ಯುತ್ತಿದೆ
ಹೊಳೆಯುವ
ಹೊಸ
ಹೊಸಚಟ
ಹೊಸಜನ್ಮ
ಹೊಸತನ
ಹೊಸ-ತ-ನದ
ಹೊಸತಾದ
ಹೊಸ-ದ-ರಲ್ಲಿ
ಹೊಸ-ದ-ರಲ್ಲೇ
ಹೊಸ-ದಲ್ಲ-ವಾ-ದರೂ
ಹೊಸದಾಗಿ
ಹೊಸ-ಬೆ-ಳಕು
ಹೊಸ-ಯು-ಗಕ್ಕೆ
ಹೊಸರೂಪ
ಹೊಸ-ವ-ರುಷ
ಹೊಸಹೊಸ
ಹೊಸ್ತಿಲು
ಹೋಗ-ದಿದ್ದರೆ
ಹೋಗದು
ಹೋಗ-ಬಲ್ಲ-ದೆನ್ನಲು
ಹೋಗ-ಬ-ಹುದು
ಹೋಗ-ಬಾ-ರದು
ಹೋಗ-ಬೇ-ಕಲ್ಲ
ಹೋಗ-ಬೇ-ಕಲ್ಲಾ
ಹೋಗಬೇಕಾ
ಹೋಗ-ಬೇ-ಕಾದ
ಹೋಗ-ಬೇ-ಕಾ-ಯಿತು
ಹೋಗಬೇಕು
ಹೋಗ-ಬೇ-ಕೆಂದು
ಹೋಗಬೇಕೇ
ಹೋಗಬೇಡ
ಹೋಗಬೇಡಿ
ಹೋಗ-ಲಾ-ಡಿ-ಸ-ಲಾ-ರದು
ಹೋಗ-ಲಾ-ಡಿ-ಸಲು
ಹೋಗ-ಲಾ-ರರು
ಹೋಗಲಿ
ಹೋಗ-ಲಿದ್ವೇಷ
ಹೋಗಲಿಲ್ಲ
ಹೋಗಲು
ಹೋಗಿ
ಹೋಗಿ-ಬಿ-ಡುತ್ತಿದ್ದ
ಹೋಗಿತ್ತು
ಹೋಗಿದೆ
ಹೋಗಿದ್ದ
ಹೋಗಿದ್ದರು
ಹೋಗಿದ್ದರೂ
ಹೋಗಿದ್ದಳು
ಹೋಗಿದ್ದಾಗ
ಹೋಗಿದ್ದಾ-ನೆಂದು
ಹೋಗಿದ್ದಾ-ನೆ-ಎಂದು
ಹೋಗಿದ್ದಾರೆ
ಹೋಗಿದ್ದಾರೋ
ಹೋಗಿದ್ದೀಯೆ
ಹೋಗಿದ್ದೆ
ಹೋಗಿದ್ದೇನೆ
ಹೋಗಿ-ಬ-ರುತ್ತಿದ್ದರು
ಹೋಗಿ-ರ-ಲಿಲ್ಲ
ಹೋಗಿ-ರ-ಲಿಲ್ಲ-ವೆಂದೂ
ಹೋಗಿರುವ
ಹೋಗಿಲ್ಲ
ಹೋಗಿವೆ
ಹೋಗು
ಹೋಗುತ್ತ
ಹೋಗುತ್ತದೆ
ಹೋಗುತ್ತ-ದೆಯೊ
ಹೋಗುತ್ತವೆ
ಹೋಗುತ್ತಾ-ನೆಂಬ
ಹೋಗುತ್ತಾ-ನೆಂಬು-ದನ್ನು
ಹೋಗುತ್ತಾ-ರಲ್ಲ
ಹೋಗುತ್ತಾರೆ
ಹೋಗುತ್ತಿತ್ತು
ಹೋಗುತ್ತಿದೆ
ಹೋಗುತ್ತಿದ್ದ
ಹೋಗುತ್ತಿದ್ದಂತೆ
ಹೋಗುತ್ತಿದ್ದರು
ಹೋಗುತ್ತಿದ್ದಳು
ಹೋಗುತ್ತಿದ್ದಾಗ
ಹೋಗುತ್ತಿದ್ದಾ-ರೆಂಬುದು
ಹೋಗುತ್ತಿದ್ದೀರಿ
ಹೋಗುತ್ತಿದ್ದೇವೆ
ಹೋಗುತ್ತಿ-ರುವ
ಹೋಗುತ್ತಿ-ರು-ವ-ವರು
ಹೋಗುತ್ತಿ-ರು-ವು-ದನ್ನು
ಹೋಗುತ್ತಿವೆ
ಹೋಗುತ್ತೆ
ಹೋಗುತ್ತೇನೆ
ಹೋಗುವ
ಹೋಗುವಂತೆ
ಹೋಗು-ವ-ರೆಂದೇನೂ
ಹೋಗು-ವ-ವರು
ಹೋಗು-ವ-ವ-ರೆಗೂ
ಹೋಗುವಾಗ
ಹೋಗುವು
ಹೋಗು-ವು-ದಕ್ಕಿಂತ
ಹೋಗು-ವು-ದನ್ನು
ಹೋಗು-ವು-ದ-ರಿಂದ
ಹೋಗು-ವು-ದ-ರಿಂದೇನು
ಹೋಗು-ವು-ದ-ರೊ-ಳಗೇ
ಹೋಗು-ವು-ದಲ್ಲದೇ
ಹೋಗು-ವು-ದಾ-ದರೆ
ಹೋಗು-ವು-ದಿಲ್ಲ
ಹೋಗುವುದು
ಹೋಗು-ವು-ದು-ಎಂದರು
ಹೋಗು-ವು-ದೆಂದು
ಹೋಗು-ವು-ದೆಂಬ
ಹೋಗು-ವು-ದೆಲ್ಲಿಗೆ
ಹೋಗುವುದೇ
ಹೋಗೋ-ಣ-ವೆ-ನಿ-ಸುತ್ತದೆ
ಹೋಗೋ-ಣ-ವೆ-ನಿ-ಸುತ್ತಿತ್ತು
ಹೋಟೆ-ಲಿ-ನಲ್ಲಿ
ಹೋಟೆಲ್
ಹೋಟೇಲು
ಹೋದ
ಹೋದಂತಾ-ಯಿತು
ಹೋದಂತೆ
ಹೋದಂದಿ-ನಿಂದ
ಹೋದದ್ದಂತೂ
ಹೋದನೆಂದೇ
ಹೋದರು
ಹೋದರೂ
ಹೋದರೆ
ಹೋದರೇ
ಹೋದಲ್ಲೆಲ್ಲಾ
ಹೋದಳು
ಹೋದ-ವ-ನಲ್ಲ
ಹೋದವನು
ಹೋದವು
ಹೋದಾಗ
ಹೋದಾ-ಗ-ಲಂತೂ
ಹೋದಾಗಲೂ
ಹೋದೀತು
ಹೋದುದನ್ನು
ಹೋದೆ
ಹೋಯಿತು
ಹೋರಾಟ
ಹೋರಾ-ಟ-ಇ-ವು-ಗಳ
ಹೋರಾ-ಟಕ್ಕಿಂತ
ಹೋರಾಟಕ್ಕೂ
ಹೋರಾಟಕ್ಕೆ
ಹೋರಾ-ಟ-ಗಳು
ಹೋರಾ-ಟ-ಗ-ಳೆಲ್ಲವೂ
ಹೋರಾ-ಟ-ಗಾ-ರನೂ
ಹೋರಾ-ಟ-ಗಾ-ರರ
ಹೋರಾಟದ
ಹೋರಾ-ಟ-ದಲ್ಲಿ
ಹೋರಾ-ಟ-ವಲ್ಲ
ಹೋರಾ-ಟ-ವಾ-ಗಲಿ
ಹೋರಾಟವೇ
ಹೋರಾ-ಟ-ವೇಕೆ
ಹೋರಾ-ಡ-ದಿ-ರು-ವುದೂ
ಹೋರಾ-ಡ-ಬೇ-ಕಷ್ಟೆ
ಹೋರಾ-ಡ-ಬೇ-ಕಾ-ಗು-ವುದು
ಹೋರಾ-ಡ-ಬೇಕು
ಹೋರಾ-ಡ-ಲೇ-ಬೇಕು
ಹೋರಾಡಿ
ಹೋರಾಡಿದ
ಹೋರಾ-ಡಿ-ದರು
ಹೋರಾ-ಡಿ-ದರೂ
ಹೋರಾ-ಡಿದ್ದನೋ
ಹೋರಾ-ಡಿದ್ದರೂ
ಹೋರಾ-ಡಿದ್ದೇನೆ
ಹೋರಾಡು
ಹೋರಾ-ಡುತ್ತಾರೆ
ಹೋರಾ-ಡುತ್ತಿ-ರು-ವನೊ
ಹೋರಾ-ಡು-ವುದು
ಹೋಲಿಕೆ
ಹೋಲಿಕೆಗೇ
ಹೋಲಿಕೆಯೂ
ಹೋಲಿ-ಸ-ಬ-ಹುದು
ಹೋಲಿಸಿ
ಹೋಲಿ-ಸಿ-ಕೊಂಡಾಗ
ಹೋಲಿ-ಸಿ-ಕೊಂಡು
ಹೋಲಿ-ಸಿ-ಕೊಳ್ಳುತ್ತ
ಹೋಲಿ-ಸಿ-ದರೆ
ಹೋಲಿ-ಸಿ-ದಾಗ
ಹೋಲಿಸಿದೆ
ಹೋಲಿ-ಸಿದ್ದಾರೆ
ಹೋಲಿ-ಸಿ-ನೋಡಿ
ಹೋಲಿ-ಸುತ್ತಾನೆ
ಹೋಲುತ್ತದೆ
ಹೌದಪ್ಪರ
ಹೌದಾದರೆ
ಹೌದು
ಹೌದೆಂಬಂತೆ
ಹೌದೆ-ನಿ-ಸಿತು
ಹೌಹಾರಿದ
ಹೌಹಾರುವ
ಹ್ಯಾನ್ಸ್ಸಿಲೀ
ಹ್ಯಾಪರಾಗಿ
ಹ್ಯಾಬಿಟ್
ಹ್ಯಾರಿಯೆಟ್
ಹ್ಯಾರಿಯೆಟ್ರ
ಹ್ಯಾರಿ-ಯೆಟ್ರ-ವ-ರಂತೆ
ಹ್ಯುಮ್ಯಾ-ನಿ-ಟಿ-ಯಲ್ಲಿ
ಹ್ಯುವೆನ್ತ್ಸಾಂಗನು
ಹ್ಯೂಗೊ
ಹ್ರಾಸಕ್ಕೆ
ಹ್ರಾಸ-ಗೊ-ಳಿ-ಸುತ್ತಿತ್ತು
ಹ್ಹ
}
