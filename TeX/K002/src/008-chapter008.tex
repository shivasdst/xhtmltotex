
\chapter{ದೇವರಿರುವನೇ?}

\indentsecionsintoc

{\centering

\begin{tabular}{@{}l@{}}
ಯಾರು ದೇವರು? \\
ಯಾರ ದೇವರು? \\
ಯಾವ ದೇವರು? \\
ಎಷ್ಟು ಮಂದಿ ದೇವರು? \\
ದೇವರಿರುವುದೆಲ್ಲಿ? \\
ದೇವರ ಹೆಸರಲ್ಲಿ ಜಗಳವೇಕೆ? \\
ಆಕಾಶ ನೋಡಲು ನೂಕಾಟವೇಕೆ? \\
\end{tabular}

}\smallskip

ಧರ್ಮಗಳು ಮನುಷ್ಯರನ್ನು ದ್ವೇಷ, ಹಿಂಸೆಯ ನರಕಕ್ಕೆ ತಳ್ಳಲು ಹುಟ್ಟಿಕೊಂಡಿವೆಯೇ? ದುರ್ಬಲರನ್ನು ಶೋಷಿಸುವುದಕ್ಕಾಗಿ ಹುಟ್ಟಿಕೊಂಡಿವೆಯೆ? ಮತಾಂಧತೆಯಿಂದ ಅವು ಮಾನವನನ್ನು ಉನ್ಮತ್ತನನ್ನಾಗಿಸುವ ಮಾದಕದ್ರವ್ಯಗಳೇ?

ಅಜ್ಞಾನವೇ ಎಲ್ಲ ದುಃಖಗಳಿಗೂ ಮೂಲ. ಸಂಕುಚಿತತೆ, ಮತಾಂಧತೆಯನ್ನು ಮನುಷ್ಯನೊಬ್ಬ ಬೆಳೆಸಿಕೊಂಡರೆ ತನ್ನನ್ನೇ ತಾನು ವ್ಯರ್ಥವಾಗಿ ಶಿಕ್ಷಿಸಿಕೊಂಡಂತೆ, ವ್ಯರ್ಥವಾಗಿ ಹಿಂಸಿಸಿಕೊಂಡಂತೆ ಅಲ್ಲವೇ?

ಕೆಲವು ವರ್ಷಗಳ ಹಿಂದೆ ಮಿಚ್ಚೆಲ್ ಎಂಬಾತ ಚಂದ್ರಗ್ರಹವನ್ನು ತಲಪಿ ಅಲ್ಲಿ ನಿಂತು\break ಅಲ್ಲಿಂದಲೇ ಭೂಮಿಯನ್ನು ಕುರಿತು ಹೇಳಿದ ಮಾತನ್ನು ಓದಿ ಚಕಿತನಾಗಿದ್ದೆ. ಆತನಿಗೆ ನಮ್ಮ ಆವಾಸ ಸ್ಥಾನವಾದ, ಅಸಂಖ್ಯ ಜೀವ ಜಂತುಗಳಿಗೆ ಆಶ್ರಯಸ್ಥಾನವಾದ ಈ ಭೂಮಿ ಆ ಎತ್ತರದಿಂದ ಒಂದು ಕಾಲ್ಚೆಂಡಿನಂತೆ ಕಾಣಿಸಿತಂತೆ! ಎಂಥ ವಿಸ್ಮಯಕರ ಅನುಭವ! ಈ ಭೂಮಿಯಾದರೋ ತನ್ನ ಅಕ್ಷದಲ್ಲಿ ಅತ್ಯಂತ ವೇಗದಿಂದಲೇ ಒಂದು ಗಂಟೆಯಲ್ಲಿ ಒಂದು ಸಾವಿರಕ್ಕಿಂತ ಹೆಚ್ಚು ಮೈಲಿ ಸುತ್ತುತ್ತಿದೆ. ಅದು ಕೆಲವು ಸುತ್ತುಗಳನ್ನು ಸುತ್ತುವಾಗ ಲಕ್ಷಗಟ್ಟಲೆ ಜನರು ಈ ಜಗತ್ತಿನಲ್ಲಿ ಕಾಣಿಸಿ\-ಕೊಳ್ಳುತ್ತಾರೆ. ಮತ್ತೆ ಕೆಲವು ಸುತ್ತುಗಳನ್ನು ಸುತ್ತುವಾಗ ಜನರು ತಮ್ಮ ತಮ್ಮ ಪಾತ್ರಗಳನ್ನು ಧರಿಸಿ ಕುಣಿದು ಕುಪ್ಪಳಿಸಿ ನಕ್ಕು ನಲಿದು ನೊಂದು ಬೆಂದು ಕಣ್ಮರೆಯಾಗುತ್ತಾರೆ.

ಮಂಕುತಿಮ್ಮನಿಗನ್ನಿಸಿದಂತೆ ನಿಮಗೂ ಅನ್ನಿಸಬಹುದು.

{\centering

\begin{longtable}[c]{@{}l@{}}
ಏನು ಭೈರವಲೀಲೆ ಈ ವಿಶ್ವವಿಭ್ರಮಣೆ \\
ಏನು ಭೂತಗ್ರಾಮ ನರ್ತನೋನ್ಮಾದ! \\
ಏನಗ್ನಿಗೊಳಗಳು ಏನಂತರಾಳಗಳು \\
ಏನು ವಿಸ್ಮಯ ಸೃಷ್ಟಿ?– \\
ಒಗಟೆಯೇನೀ ಸೃಷ್ಟಿ ಬಾಳಿನರ್ಥವದೇನು? \\
ಬಗೆದು ಬಿಡಿಸುವರಾರು ಸೋಜಿಗವನಿದನು? \\
\end{longtable}

}\smallskip

ಮಿಚ್ಚೆಲ್ ಪಡೆದ ಅನುಭವ ಕಟ್ಟುಕತೆಯಲ್ಲ. ಕವಿಕಲ್ಪನೆಯೂ ಅಲ್ಲ. ಈ ಅನುಭವಗಮ್ಯವಾದ ಸತ್ಯವನ್ನು ಯಾರೂ ಅಲ್ಲಗಳೆಯಲಾರರು. ಹಾಗೆ ಅಲ್ಲಗಳೆಯಲು ಸಾಧ್ಯವಿಲ್ಲವಷ್ಟೆ. ಧಾರ್ಮಿಕ ಮತಾಂಧರು ವಿಜ್ಞಾನವನ್ನು, ವೈಜ್ಞಾನಿಕ ಸಂಶೋಧನೆಗಳನ್ನು ಎಷ್ಟೇ ನಿಂದಿಸಲಿ ನಿರಾಕರಿ\-ಸಲಿ ಆದರೆ ಅವು ತೋರಿಸಿಕೊಡುವ ಬೆಳಕಿನಲ್ಲಿ ನಮ್ಮ ನಂಬಿಕೆಗಳನ್ನು ಪರಿಶೀಲಿಸಿ ಕೊಂಡಲ್ಲಿ ನಾವು ಬಲವಾಗಿ ಅಪ್ಪಿಕೊಂಡ ‘ಬಾವಿಕಪ್ಪೆತನ’ ಸಂಕುಚಿತತೆ ಮಾಯವಾಗುವುದರಲ್ಲಿ ಸಂಶಯವಿದೆಯೇ? ವಿಜ್ಞಾನವು ಇಂದು ಅಗಾಧವಾದ ಈ ವಿಶ್ವಬ್ರಹ್ಮಾಂಡದ ಹಿನ್ನೆಲೆಯಲ್ಲಿ ಪರಮಾದ್ಭುತವಾದ ಒಂದು ದಿವ್ಯಶಕ್ತಿಯನ್ನು ಕುರಿತು ನಿರ್ಭಯವಾಗಿ ಸಾರುತ್ತಿದೆ ಎಂದರೆ ತಪ್ಪಿಲ್ಲ. ಆ ಶಕ್ತಿ–ಆ ದೇವರು ಗುಡಿ ಮಸೀದಿ ಚರ್ಚುಗಳಲ್ಲಿ ಮಾತ್ರ ತಂಗಿದ್ದಾನೆಯೇ? ಯಾವುದೇ ಒಂದು ಪಥ ಪಂಥಗಳಲ್ಲಿ ಮಾತ್ರ ಸಿಕ್ಕಿಕೊಂಡಿದ್ದಾನೆಯೇ? ಪೂಜಾಸ್ಥಾನಗಳು ಆತನನ್ನು ಪ್ರಾರ್ಥಿಸಲು ಧ್ಯಾನಿಸಲು ಪೂಜಿಸಲು ಸದವಕಾಶ ಕಲ್ಪಿಸುವ ಒಂದು ವ್ಯವಸ್ಥೆ ಮಾತ್ರವಲ್ಲವೇ? ಬೇರೆ ಮತಿ ಬೇರೆ ಮತ ಎಂಬುದು ಅವನಿಗರ್ಥವಾಗದೇ? ಅದು ಅವನದೇ ವಿಧಾನವಲ್ಲವೇ? ಮನುಷ್ಯ ಹೃದಯದ ಭಾವನೆಗಳನ್ನು ಅರಿಯಲಾರದ ಅಸಹಾಯಕನೇ ಅವನು! ಕೆಲವು ವರ್ಷಗಳ ಹಿಂದೆ ನ್ಯೂಯಾರ್ಕ್ ಅಕೇಡಮಿ ಆಫ್ ಸೈನ್ಸಸ್ ಅಧ್ಯಕ್ಷರಾದ ವಿಜ್ಞಾನಿ ಎ. ಕ್ರೆಸ್ಸಿ ಮಾರಿಸನ್ ರೀಡರ್ಸ್ ಡೈಜೆಸ್ಟ್ ಮಾಸಪತ್ರಿಕೆಯಲ್ಲಿ ‘ವಿಜ್ಞಾನಿ ದೇವರಲ್ಲಿ ಏಕೆ ವಿಶ್ವಾಸವನ್ನು ಇಡುತ್ತಾನೆಂಬುದಕ್ಕೆ ಏಳು ಕಾರಣಗಳು’ ಎಂಬ ಲೇಖನವನ್ನು ಬರೆದಿದ್ದರು.\footnote{\engfoot{Seven Reasons why a scientist believes in God; by A. Cressy Morrison.}} ವಿಜ್ಞಾನದ ವಿವಿಧ ಕ್ಷೇತ್ರಗಳಲ್ಲಿ ನಡೆದ ಅಸಂಖ್ಯ ಶೋಧನೆಗಳನ್ನು ಗಮನದಲ್ಲಿಟ್ಟುಕೊಂಡು ಎತ್ತರದಲ್ಲಿ ನಿಂತು ಸುತ್ತಲೂ ನೋಡಿದಂತೆ ಸಂದೇಹಕ್ಕೆಡೆಗೊಡದ ವೈಜ್ಞಾನಿಕ ವೈಚಾರಿಕ ವಿಧಾನದಿಂದಲೇ ಅವರು ಸರ್ವತ್ರ ಸರ್ವವ್ಯಾಪಿಯಾದ ಅಚಿಂತ್ಯ ಶಕ್ತಿಯಾದ ಪರಮಾತ್ಮನ ಅಸ್ತಿತ್ವದಲ್ಲಿ ನಮ್ಮ ನಂಬಿಕೆಯನ್ನು ದೃಢಗೊಳಿಸುತ್ತಾರೆ. ‘ನಮ್ಮ ಅರಿವು ನುಗ್ಗಲಾರದ ಸಂಶೋಧಿಸಲಾರದ ಆ ಪರಮಾದ್ಭುತದಿಂದಲೇ ಈ ವಿಸ್ಮಯವೆನಿಸಿದ ವಿಶ್ವಬ್ರಹ್ಮಾಂಡ, ಅದರ ಹೊಳೆಯುವ ಸೌಂದರ್ಯ ವ್ಯಕ್ತವಾಗುತ್ತಿದೆ \footnote{\engfoot{‘What is impenetrable to us exists manifesting as this wonderful universe and radiant beauty.’}}ಎಂಬ ಐನ್​ಸ್ಟೀನ್ ವಾಕ್ಯ ಎಷ್ಟೊಂದು ಸಂಗತವಾಗಿದೆ. ಮೊರಿಸನ್ ತಮ್ಮ ಲೇಖನದಲ್ಲಿ ಹೇಳಿದ ಮಾತು ಇಂತಿದೆ. ‘ಡಾರ್ವಿನ್ ನಂತರ ಈ ೧೦೦ ವರ್ಷಗಳಲ್ಲಿ ನಾವು ಅತ್ಯಾಶ್ಚರ್ಯಕರವಾದ ಸಂಶೋಧನೆಗಳನ್ನು ಮಾಡಿದ್ದೇವೆ. ವೈಜ್ಞಾನಿಕವಾದ ವಿನಮ್ರ ಭಾವನೆಯಿಂದಲೇ, ಬರಿಯ ನಂಬಿಕೆಯಲ್ಲ, ಅರಿವನ್ನು ಆಧರಿಸಿದ ಮೂಢವಲ್ಲದ ನಂಬಿಕೆಯ ಪ್ರಕಾರ ಭಗವತ್ ಸಾನ್ನಿಧ್ಯದ ಅನುಭವವನ್ನು ಸಮೀಪಿಸುತ್ತಿದ್ದೇವೆ!’ ವಿಜ್ಞಾನಿ ಹೇಳುವ ಈ ಭಗವತ್ಸ್ವರೂಪವನ್ನು ಕುರಿತ ಹೇಳಿಕೆಯನ್ನು ವಿಚಾರವಂತರಾರೂ ಅಲ್ಲಗಳೆಯಲು ಸಾಧ್ಯವಿಲ್ಲ. ಆದರೆ ನಮ್ಮ ಸಂಕುಚಿತ ಕಲ್ಪನೆಗಳಿಗೆ ದೀರ್ಘಕಾಲದಿಂದ ತುತ್ತಾದ ನಮಗೆ ಅದರಿಂದ ಬಿಡಿಸಿ\-ಕೊಳ್ಳುವುದು ಕಷ್ಟವೆನಿಸಿದರೂ ಕ್ರಮೇಣ ಸಾಧ್ಯ ಎಂಬುದನ್ನು ನೆನಪಿನಲ್ಲಿಡಬೇಕು.

ವಸ್ತುವಿನ ಸೂಕ್ಷ್ಮಾತಿಸೂಕ್ಷ್ಮ ಸ್ವರೂಪದ ಬಗೆಗೂ ವಿಜ್ಞಾನದ ಸಂಶೋಧನೆ ಅಚ್ಚರಿ ಎನಿಸುವ ವಿಷಯಗಳನ್ನು ನಮಗೆ ತಿಳಿಸುತ್ತದೆ. ಹೆಚ್ಚಿನ ವಿವರಣೆಗಳ ಆವಶ್ಯಕತೆ ಇಲ್ಲ. ಆಟಂ ಅಥವಾ ಅಣು ಎಂದು ಕರೆಯಲ್ಪಡುವ ವಸ್ತುವಿನ ಗಾತ್ರ ಎಷ್ಟು ಗೊತ್ತೇ? ಒಂದು ಮಿಲಿಮೀಟರ್ ಪರಿಮಾಣದ ವಸ್ತುವನ್ನು ಮಿಲಿಯ ಎಂದರೆ ಹತ್ತುಲಕ್ಷ ಭಾಗ ಮಾಡಿ ಅದರಲ್ಲಿ ಒಂದು ಭಾಗವನ್ನು ತೆಗೆದುಕೊಂಡರೆ ಎಷ್ಟೋ ಅಷ್ಟು! ಈ ಅಣುವಿನಲ್ಲಿ ಇಲೆಕ್ಟ್ರಾನ್ ಪ್ರೋಟಾನ್ ನ್ಯೂಟ್ರಾನ್ ಎಂಬ ಸೂಕ್ಷ್ಮಾತಿಸೂಕ್ಷ್ಮ ಶಕ್ತಿಯ ಕಣಗಳಿವೆ. ವಿದ್ಯುತ್ತಿನಂಥ ಈ ಶಕ್ತಿಯ ಕಣಗಳೆಂಬ ಇಟ್ಟಿಗೆಗಳಿಂದಲೇ ಈ ವಿಶ್ವಬ್ರಹ್ಮಾಂಡದ ರಚನೆಯಾಗಿದೆ ಎಂದರೆ ನಂಬುವಿರಾ? ಈ ವಿಶ್ವಬ್ರಹ್ಮಾಂಡ ಹಾಗೂ ಅಸಂಖ್ಯ ನಾಮರೂಪಗಳಾಗಿ ವಿಂಗಡವಾಗಿರುವ ಜೀವ ಜಂತು ಆಕಾಶ ನೀರು ವಾಯು–ಇವೆಲ್ಲವುಗಳ ಮೂಲದಲ್ಲಿ ಒಂದೇ ಶಕ್ತಿ! ಆ ಶಕ್ತಿಯ ಆವಿರ್ಭಾವ, ಅರ್ಥಪೂರ್ಣ ಆವಿರ್ಭಾವ ಈ ಜಗತ್ತು. ಆ ಶಕ್ತಿಯ ಅಭಿವ್ಯಕ್ತಿ ಹಾಗೂ ಅರ್ಥಪೂರ್ಣವಾಗಿ ವರ್ತಿಸುವಂತೆ ಮಾಡುವ ಸೂಕ್ತನಿರ್ವಹಣೆಯ ಹಿನ್ನೆಲೆಯಲ್ಲಿ ಅಸೀಮ ಅನಂತ ಪ್ರಜ್ಞೆ ಅಥವಾ ಚೈತನ್ಯ ಅಡಗಿದೆ, ಸರ್ವವ್ಯಾಪಿಯಾದ ಸರ್ವ\-ಶಕ್ತನಿದ್ದಾನೆ ಎಂಬುದನ್ನು ವಿಜ್ಞಾನಿಗಳ ಈ ಕೆಳಗಿನ ವಾಕ್ಯಗಳೊಂದಿಗೆ ಹೋಲಿಸಿನೋಡಿ.

‘ನಮ್ಮ ದುರ್ಬಲವೂ ದೋಷಯುಕ್ತವೂ ಆದ ಮನಸ್ಸಿಗೆ ಪರಿಶೀಲಿಸಲು ಸಾಧ್ಯವಾಗುವ ಅತ್ಯಂತ ಸಾಮಾನ್ಯ ಹಾಗೂ ಸಣ್ಣಪುಟ್ಟ ಸಂಗತಿಗಳಲ್ಲೂ ತನ್ನಿಂದ ತಾನೇ ಪ್ರಕಾಶಿತವಾಗುವ ಆ ಸೀಮಾರಹಿತವಾದ ಅತ್ಯಂತ ಶ್ರೇಷ್ಠ ಚೇತನದಲ್ಲಿ ಗೌರವಪೂರ್ವಕವಾದ ಭಾವನೆಯೇ ನನ್ನ\break ಧಾರ್ಮಿಕ ದೃಷ್ಟಿ. ಅರಿಯಲು ಅಶಕ್ಯವಾದ ಈ ವಿಶ್ವದಲ್ಲಿ ವ್ಯಕ್ತವಾಗುವ ಶ್ರೇಷ್ಠ ವಿವೇಚನಾ ಶಕ್ತಿಯಲ್ಲಿ (ಎಂದರೆ ಸಮಸ್ತವಿಶ್ವದ ಹಿನ್ನೆಲೆಯಲ್ಲಿ ಸೂಕ್ಷ್ಮಾತಿಸೂಕ್ಷ್ಮವಾಗಿದ್ದುಕೊಂಡು ಸರ್ವವ್ಯಾಪಿ\-ಯಾಗಿರುವ ವಿಶ್ವನಿಯಾಮಕ ಶಕ್ತಿಯಲ್ಲಿ ಅಥವಾ ದೇವರಲ್ಲಿ) ಆಳವೂ ಭಾವನಾತ್ಮಕವೂ ಆದ ದೃಢವಿಶ್ವಾಸವೇ ದೇವರನ್ನು ಕುರಿತ ನನ್ನ ಭಾವನೆಯಾಗಿದೆ.’–ಐನ್​ಸ್ಟೀನ್

‘ವಿಶ್ವದ ಮೂಲಭೂತ ಏಕತೆಯ ಕುರಿತ ಸಿದ್ಧಾಂತ ಕೇವಲ ಅನುಭಾವಿಗಳ ದೃಢವಾದ ಅನುಭೂತಿ ಮಾತ್ರವಲ್ಲ ಅದು ಆಧುನಿಕ ಯುಗದ ಭೌತಶಾಸ್ತ್ರದ ಅಧ್ಯಯನದಿಂದ ತಿಳಿದು ಬರುವ ಪ್ರಮುಖವಾದ ಸಂಗತಿಯೂ ಹೌದು. ಪೌರಸ್ತ್ಯ ತಾತ್ತ್ವಿಕ ಪರಂಪರೆಯು ಭೇದರಹಿತ ಏಕಮೇವಾದ್ವಿತೀಯವಾದ ಪರಮಸತ್ಯದ ನೈಜಸ್ವರೂಪವನ್ನು ತಿಳಿಸುತ್ತ ಅದು ಸರ್ವವ್ಯಾಪಿ\-\break ಯಾಗಿದೆ ಎಂದು ಹೇಳುತ್ತದೆ. ಹಿಂದೂಧರ್ಮದಲ್ಲಿ ಅದನ್ನು ‘ಬ್ರಹ್ಮ’ ಎಂದೂ ಬೌದ್ಧ ಧರ್ಮದಲ್ಲಿ ‘ಧರ್ಮಕಾಯ’ವೆಂದೂ ಚೀನಾದೇಶದ ತಾವೊ ಮತದಲ್ಲಿ ‘ತಾವೊ’ ಎಂದೂ ಪರಿಗಣಿಸುತ್ತಾರೆ.’–ಪ್ರಿಜೊ ಕಾಪ್ರಾ.

\section*{ನಾನು ಯುವಕರಿಗೊಂದು ಕತೆ ಹೇಳಿದೆ}

\addsectiontoTOC{ನಾನು ಯುವಕರಿಗೊಂದು ಕತೆ ಹೇಳಿದೆ}

ಇರುವೆಗಳದ್ದೇ ಒಂದು ವಿಶ್ವವಿದ್ಯಾಲಯವಿತ್ತು.\ ಆ ವಿಶ್ವವಿದ್ಯಾಲಯದ ವೈಸ್​ಛಾನ್ಸಲರ್\break ಸಂಶೋಧನಾ ವಿಭಾಗದ ಮುಖ್ಯಸ್ಥರಾದ ಪ್ರೊ.\ ಇರುವೆಯಪ್ಪನವರನ್ನು ಕರೆದು ಮನುಷ್ಯರು ನಡೆಸುವ ಪ್ರೆಸ್ಸನ್ನೂ ಪುಸ್ತಕಗಳುಂಟಾಗುವ ವಿಧಾನವನ್ನೂ ಕುರಿತು ಸಂಶೋಧನೆ ಮಾಡುವಂತೆ\break ಹೇಳಿ ಒಂದು ಮಹಾಪ್ರಬಂಧವನ್ನು ಸಿದ್ಧಪಡಿಸುವಂತೆ ಆಜ್ಞಾಪಿಸಿದರು. ಆ ಇರುವೆಯಾದರೋ ಮೂರು ತಿಂಗಳುಗಳ ಕಾಲ ತನ್ನ ಸಹಾಯಕರೊಂದಿಗೆ ಪ್ರೆಸ್ಸನ್ನು ತಾವೇ ನಿರ್ಮಿಸಿಕೊಂಡ ಟೆಲೆ\-ಸ್ಕೋಪ್ ಮೈಕ್ರೋಸ್ಕೋಪ್ ಮೂಲಕ ಆಮೂಲಾಗ್ರವಾಗಿ ಪರಿಶೀಲಿಸಿ ಮಹಾಪ್ರಬಂಧವನ್ನು ರಚಿಸಿತು. ‘ನಾನು ಮತ್ತು ನನ್ನ ಸಹಾಯಕರು ಹಗಲೂರಾತ್ರಿ ಕಣ್ಣಿನ ಮೇಲೆ ಕಣ್ಣಿಟ್ಟು ಅಚ್ಚುಕೂಟವನ್ನು ಪರಿಶೀಲಿಸಿದೆವು. ಕೆಲವು ಬಿಳಿಯ ಹಾಳೆಗಳ ಮೇಲೆ ಕಬ್ಬಿಣದ ಮೊಳೆಗಳು ಕ್ರಮಬದ್ಧವಾಗಿ ಲಯಬದ್ಧವಾಗಿ ಬೀಳುವುದರಿಂದ ಗ್ರಂಥಗಳಾಗುತ್ತವೆಯೇ ಹೊರತು ಗ್ರಂಥಕರ್ತನೊಬ್ಬ\-ನಿರುವ\-ನೆಂಬುದಕ್ಕೆ ನನಗೆ ಯಾವ ಪುರಾವೆಯೂ ಸಿಗಲಿಲ್ಲ. ಗ್ರಂಥಕರ್ತನಿದ್ದಾನೆ ಎಂಬುದು ದೀರ್ಘಕಾಲದಿಂದ ಗಟ್ಟಿಯಾಗಿ ನೆಲೆನಿಂತ ಮೂಢನಂಬಿಕೆ. ಹೆಚ್ಚುಹೆಚ್ಚು ಪರಿಶೀಲನೆ ಪ್ರಯೋಗ ಅನ್ವೇಷಣೆಗಳು ನಡೆದಂತೆಲ್ಲ ಈ ಮೂಢನಂಬಿಕೆ ಮಾಯವಾಗುವುದು’ ಎಂದು ಅದು ತನ್ನ ವೈಜ್ಞಾನಿಕ ಅನಿಸಿಕೆಯನ್ನು ಸ್ಪಷ್ಟವಾಗಿ ಸಾರಿ ಹಿರಿಯ ಬುದ್ಧಿಜೀವಿ ಇರುವೆಗಳಿಗೆ ಸಿಗುವ ಬಹುಮಾನವನ್ನು ತಾನೂ ಗಿಟ್ಟಿಸಿಕೊಂಡಿತಂತೆ.

ಈಗ ಪುಸ್ತಕವನ್ನು ಒಂದು ಹಸುವಿನೆದುರು ಇಟ್ಟರೆ ಅದು ಅದನ್ನು ತಿನ್ನಲು ಸಾಧ್ಯವೇ ಎಂದು ನೆಕ್ಕಿ ನೋಡಿ ಅಸಾಧ್ಯವೆಂದು ತಿಳಿದ ಬಳಿಕ ಅದರ ಕಡೆಗೆ ಗಮನವನ್ನೇ ಕೊಡದು. ಹಸುವಿಗೆ ಪುಸ್ತಕವು ತಿನ್ನಲಾಗದ ವಸ್ತುವೆಂಬ ಜ್ಞಾನವಾಗಿದೆ. ಓದು ಬರಹದ ಗಂಧಗಾಳಿ ಇಲ್ಲದ ಪುಸ್ತಕದ ಬಗೆಗೆ ಏನನ್ನೂ ಕಂಡು ಕೇಳಿ ಅರಿಯದ ಆಫ್ರಿಕಾದ ಕಗ್ಗಾಡಿನಲ್ಲಿರುವ ವ್ಯಕ್ತಿಯ ಎದುರು ಪುಸ್ತಕವನ್ನಿಡಿ. ಅವನು ಅದನ್ನು ಮುಟ್ಟಿನೋಡಿ ತನ್ನ ಹತ್ತಿರವಿಟ್ಟುಕೊಳ್ಳಬಹುದು. ದೂರಕ್ಕೆ ಎಸೆಯಲೂ ಬಹುದು. ಒಂದು ವೇಳೆ ಪುಸ್ತಕವನ್ನು ತೆರೆದು ನೋಡಿದಾಗ ಅಲ್ಲಿರುವ ಅಕ್ಷರಗಳು ಅವನ ಪಾಲಿಗೆ ಕಪ್ಪು ಇರುವೆಗಳ ಸಾಲಿನಂತೆ ಕಾಣಬಹುದು. ಅವನ ತಿಳಿವಳಿಕೆಗೆ ಮೀರಿದ ವಸ್ತು ಅದು. ಆದರೂ ಪುಸ್ತಕದ ಬಗೆಗೆ ಒಂದು ತೆರನಾದ ಜ್ಞಾನ ಅವನಿಗುಂಟಾಗಿದೆ. ಈಗ ಪುಸ್ತಕವನ್ನು ಓದಿ ಅರ್ಥೈಸಿಕೊಳ್ಳಬಲ್ಲ ವ್ಯಕ್ತಿಯ ಎದುರು ಪುಸ್ತಕವನ್ನಿಡಿ. ‘ಓಹ್​! ಒಳ್ಳೆಯ ಪುಸ್ತಕವೆಂದು ತೋರುತ್ತದೆ. ನನಗೆ ಈ ಭಾಷೆ ತಿಳಿಯದು. ಏನು ಮಾಡಲಿ?’ ಎನ್ನುತ್ತಾನೆ. ಭಾಷೆಯನ್ನು ತಿಳಿದ ವ್ಯಕ್ತಿ ಆ ಗ್ರಂಥವನ್ನು ಓದುತ್ತೋದುತ್ತ ಕೆಲವೊಮ್ಮೆ ಗಂಭೀರವಾಗುತ್ತಾನೆ. ಕೆಲವೊಮ್ಮೆ ಕಂಬನಿ ದುಂಬುತ್ತಾನೆ. ಕೆಲವೊಮ್ಮೆ ನಗುತ್ತಾನೆ. ಕೆಲವೊಮ್ಮೆ ‘ಆಹಾ’ ‘ಓಹ್​’ ಎನ್ನುತ್ತಾನೆ. ಎಷ್ಟೊಂದು ಅರ್ಥಪೂರ್ಣವಾಗಿದೆ ಬರಹ ಎಂದು ಅಚ್ಚರಿಯಿಂದ ಪ್ರಶಂಸಿಸುತ್ತಾನೆ.

ಪುಸ್ತಕವು ಬರಹಗಾರನ ಅಭಿಪ್ರಾಯದ ಸುವ್ಯವಸ್ಥಿತ ಅಭಿವ್ಯಕ್ತಿ ಎಂಬುದು ಸ್ಪಷ್ಟ.

\newpage

ನಿಜ, ಗ್ರಂಥ ರಚನೆ ಮಾಡಿದವನು ಸೀಮಿತ ಶಕ್ತಿಯ ವ್ಯಕ್ತಿ. ಆದರೆ ವಿಶ್ವವೆಂಬ ಕಾವ್ಯವನ್ನು ರಚಿಸಿದವನು ಸರ್ವಜ್ಞನೂ ಸರ್ವಶಕ್ತನೂ ಸರ್ವವ್ಯಾಪಿಯೂ ಆದ ನಿಯಾಮಕನು. ಅವನನ್ನೇ ನಾವು ದೇವರು ಎನ್ನುವುದು. ಪ್ರತಿಯೊಂದು ಅಣುಕಣವನ್ನೂ ಅರ್ಥಪೂರ್ಣವಾಗಿ ವರ್ತಿಸುವಂತೆ ಮಾಡಿ ಈ ಪರಮಾದ್ಭುತ ವಿಶ್ವದ ಸೃಷ್ಟಿ ಸ್ಥಿತಿ ಮತ್ತು ಲಯಗಳನ್ನು ನಿರಂತರವಾಗಿ ನಡೆಯುವಂತೆ ಮಾಡಿದ್ದಾನೆ.

ಮೇಲಿನ ಮಾತುಗಳಲ್ಲಿ ದೇವರನ್ನು ಕುರಿತ ಸಂಕುಚಿತ ಭಾವನೆಗಳ ಬಿಗಿಹಿಡಿತದಿಂದ ನಮ್ಮನ್ನು ನಾವು ಹೇಗೆ ಬಿಡಿಸಿಕೊಳ್ಳಬಹುದೆಂಬುದನ್ನು ಕಾಣಬಹುದು. ಜೊತೆಜೊತೆಗೆ ದುರಭಿಮಾನ, ಪರಮತ ಅಸಹಿಷ್ಣುತೆ ದ್ವೇಷ ಮೌಢ್ಯಗಳ ಕಾರಾಗೃಹದಿಂದ ನಮ್ಮನ್ನು ನಾವು ಹೇಗೆ ತಪ್ಪಿಸಿಕೊಳ್ಳ\-ಬಹುದೆಂಬುದನ್ನು ಊಹಿಸಬಹುದು.

ಪರಮಾರ್ಥ ಸತ್ಯವು ಯಾವುದೇ ಒಂದು ಆವಿರ್ಭಾವಕ್ಕೇ ಸೀಮಿತವಲ್ಲ. ಅದು ಏಕಮುಖ\-ವಾದುದಲ್ಲ, ಅದು ಸರ್ವತೋಮುಖವಾದುದು. ಒಂದು ಅವತಾರ ಅಥವಾ ಒಬ್ಬ ಪ್ರವಾದಿಯ ಬೋಧನೆಗೇ ಸೀಮಿತವಲ್ಲ ಎಂಬ ಉದಾರದೃಷ್ಟಿ ಬೆಳೆಯದೇ ಧಾರ್ಮಿಕರಲ್ಲಿ ಪರಸ್ಪರ ದ್ವೇಷ ಜಗಳ ಬಡಿದಾಟ ನಿಲ್ಲದು, ಶಾಂತಿ ನೆಲಸದು. ವಿಜ್ಞಾನಕ್ಕೆ ಸಮ್ಮತವಾದ ಎಲ್ಲರೂ ಒಪ್ಪಲೇ ಬೇಕಾದ ಈ ವಿಶ್ವವಿಶಾಲ ಭಾವನೆಯನ್ನು ಭಾರತೀಯ ಪುಷಿಪರಂಪರೆ ಹಿಂದೆ ಸಾರಿತ್ತು. ಚಿಕಾಗೋ ಸರ್ವಧರ್ಮ ಸಮ್ಮೇಳನದಲ್ಲಿ ಸ್ವಾಮಿ ವಿವೇಕಾನಂದರು, ೧೯ನೇ ಶತಮಾನದ ಅಂತ್ಯ ಭಾಗದಲ್ಲಿ, ದಿಕ್ತಟಗಳು ಮೊಳಗುವಂತೆ ಇನ್ನೊಮ್ಮೆ ಈ ಉದಾರ ಭಾವನೆಯನ್ನೇ ಉದ್ಘೋಷಿಸಿದರು. ಮುಂದೆಯೂ ಭಾರತದಿಂದಲೇ ಈ ಭಾವ ಘೋಷಿಸಲ್ಪಡುವುದು. ಸಾವಕಾಶವಾಗಿ ಜಗತ್ತಿನ ಎಲ್ಲ ಜನಾಂಗಗಳೂ ಅದನ್ನು ಅನಿವಾರ್ಯವಾಗಿ ಸ್ವೀಕರಿಸುವುವು ಎಂಬುದರಲ್ಲಿ ಸಂದೇಹವಿಲ್ಲ. ಈ ಉದಾರ ಭಾವನೆಯನ್ನು ಬೋಧಿಸುವುದಕ್ಕಾಗಿಯೇ ಭಗವಂತ ಭರತಖಂಡವನ್ನು ರಕ್ಷಿಸಿ ತಲೆ ಎತ್ತಿ ನಿಲ್ಲುವಂತೆ ಮಾಡುವನು. ೧೮೯೩ರಲ್ಲಿ ಅಮೆರಿಕದಲ್ಲಿ ನಡೆದ ಚಿಕಾಗೋ ಸರ್ವಧರ್ಮ ಸಮ್ಮೇಳನದಲ್ಲಿ ಸ್ವಾಮೀಜಿ ಹೇಳಿದ ಮಾತುಗಳು ಇವು.

‘ಜಗತ್ತಿಗೆ ವಿಶ್ವಧರ್ಮಸಮ್ಮೇಳನವು ಏನಾದರೂ ತೋರಿದ್ದರೆ ಅದು ಇದು. ದೈವೀಪ್ರಜ್ಞೆ, ಪವಿತ್ರತೆ ದಯೆ ದಾನಬುದ್ಧಿ–ಮುಂತಾದವುಗಳು ಜಗತ್ತಿನ ಯಾವ ಒಂದು ಧರ್ಮಸಂಸ್ಥೆಗೋ ಧರ್ಮಕ್ಕೋ ಮೀಸಲಾದುದಲ್ಲ. ಪ್ರತಿಯೊಂದು ಧರ್ಮವೂ ಕೂಡ, ಉನ್ನತ ಮಟ್ಟದ ಉದಾತ್ತ ಚರಿತರಾದ ಸ್ತ್ರೀಪುರುಷರನ್ನು ಕೊಟ್ಟಿರುವುದು. ಇಷ್ಟೊಂದು ಸಾಕ್ಷ್ಯಾಧಾರಗಳು ಇರುವಾಗಲೂ ಯಾರಾದರೂ ತಮ್ಮ ಧರ್ಮ ಒಂದೇ ಬಾಳುವುದು, ಇತರ ಧರ್ಮಗಳೆಲ್ಲವೂ ನಾಶವಾಗುವುವೆಂದು ತಿಳಿದುಕೊಂಡರೆ ಅವರ ಅಜ್ಞಾನ ಮತಾಂಧತೆ ಮತಿಹೀನತೆಗಳನ್ನು ಕಂಡು ನಾನು ತೀವ್ರವಾಗಿ ಮರುಗುತ್ತೇನೆ, ಕನಿಕರಪಡುತ್ತೇನೆ. ಪ್ರತಿಯೊಂದು ಧರ್ಮದ ಧ್ವಜದ ಮೇಲೂ ವಿಭಿನ್ನ ಧರ್ಮಾನುಯಾಯಿಗಳು ಎಷ್ಟೇ ವಿರೋಧಿಸಿದರೂ ‘ಹೋರಾಟವಲ್ಲ, ಸಹಾಯ’, ‘ನಾಶವಲ್ಲ ಸ್ವೀಕಾರ’, ‘ವೈಮನಸ್ಯವಲ್ಲ, ಶಾಂತಿ ಮತ್ತು ಸಮನ್ವಯ’ ಎಂಬುದನ್ನು ಬೇಗ ಬರೆಯಲಾಗುವುದು.


\section*{ವೈಜ್ಞಾನಿಕ ಸಂಶೋಧನೆಗಳ ಬೆಳಕಿನಲ್ಲಿ}

\addsectiontoTOC{ವೈಜ್ಞಾನಿಕ ಸಂಶೋಧನೆಗಳ ಬೆಳಕಿನಲ್ಲಿ}

ವಿಜ್ಞಾನದ ಪರಮಾದ್ಭುತವಾದ ನೂತನ ಸಂಶೋಧನೆಗಳ ಬೆಳಕಿನಲ್ಲಿ ಇಡಿಯ ಜಗತ್ತೇ ಒಂದು ಎಂಬುದು ಸ್ಪಷ್ಟವಾಗಿರುವ ಸನ್ನಿವೇಶದಲ್ಲಿ ಜಗತ್ತಿನ ನಿಯಾಮಕನೂ ಒಬ್ಬನೇ,ಎಲ್ಲರೂ ದೇವರ ಮಕ್ಕಳೇ ಎಂಬ ಶ್ರೇಷ್ಠ ಸತ್ಯವನ್ನು ಎಲ್ಲ ಧಾರ್ಮಿಕ ಮುಖಂಡರೂ ಪ್ರಸಾರ ಮಾಡ ಬೇಕಾದ ಕಾಲ ಸನ್ನಿಹಿತವಾಗಿದೆ. ಭಾಷೆಗಳಲ್ಲಿ ವೈವಿಧ್ಯವಿದೆ. ಜಗತ್ತಿನಲ್ಲಿ ನಾಲ್ಕೈದು ಸಾವಿರ ಭಾಷೆಗಳಿರಬಹುದು. ಆ ಭಾಷೆಯ ಶಬ್ದವು ಸೂಚಿಸುವ ವಸ್ತುವಿನಲ್ಲಿಲ್ಲ. ಭಗವದ್ಭಾವನೆ ಮತ್ತು ಪೂಜಾವಿಧಾನಗಳಲ್ಲಿ ವ್ಯತ್ಯಾಸವಿದೆ. ಪೂಜಿಸಲ್ಪಡುವ ತತ್ತ್ವ ಒಂದೇ. ಅತಿರೇಕದ ಸ್ವಮತಾಭಿಮಾನ, ಧಾರ್ಮಿಕ ಮತಾಂಧತೆ ಅನ್ಯಮತ ದ್ವೇಷ, ತನ್ನ ಮತ ಮಾತ್ರ ಸತ್ಯ, ಅದು ಮಾತ್ರ ಉಳಿಯಬೇಕು, ಉಳಿದವರು ನಾಶವಾಗಬೇಕೆಂಬ ಆಂತರಿಕ ಬಯಕೆ–ಇವು ಈ ಸುಂದರ ಜಗತ್ತನ್ನು ಬಹುಕಾಲದಿಂದ ಆವರಿಸಿಕೊಂಡಿರುವ ವಿಚಾರ ಇತಿಹಾಸ ದಾಖಲಿಸಿದೆ. ದಾಖಲಿಸುತ್ತಲಿದೆ. ಜಗತ್ತಿನಲ್ಲಿ ಹಿಂಸೆಯ ನರಕಕ್ಕೆ ತಳ್ಳುವ, ರಕ್ತದ ಕಾಲುವೆ ಹರಿಸುವ, ಮಾನವತೆ ಮತ್ತು ಸಂಸ್ಕೃತಿಯನ್ನು ನಾಶಗೊಳಿಸುವ, ಬದುಕನ್ನು ನಿರಾಶೆಯ ಕೂಪಕ್ಕೆ ತಳ್ಳುವ ಧರ್ಮವಿರೋಧೀ ಈ ಕಾರ್ಯಕ್ರಮ ಧಾರ್ಮಿಕರೆನಿಸಿಕೊಂಡವರಿಂದಲೇ ಆರಂಭವಾಯಿತು ಎನ್ನುವ ಸಂಗತಿ ಎಂಥ ದುರಂತ!\break ಇದೀಗ ಅಂಥ ಚಟುವಟಿಕೆಗಳಿಗೆ ಪೂರ್ಣವಿರಾಮ ಹಾಕುವಂತೆ ಎಲ್ಲ ಧಾರ್ಮಿಕ ಪಂಗಡಗಳಲ್ಲಿ ಇರುವ ಉದಾರ ಭಾವನೆಯ ಧಾರ್ಮಿಕ ಮುಖಂಡರು ವಿಶೇಷ ರೀತಿಯಿಂದ ಶ್ರಮಿಸಬೇಕಾಗಿದೆ. ಭಗವಂತನನ್ನು ಸರ್ವಜ್ಞ ಸರ್ವಶಕ್ತ ಸರ್ವವ್ಯಾಪೀ ಎನ್ನುತ್ತ ಅವನ ಮನಸ್ಸಿಗೆ ನೋವನ್ನುಂಟು\-ಮಾಡಿ ಶಾಪಗ್ರಸ್ತರಾಗುವ ದುಷ್ಕಾರ್ಯಕ್ಕೆ ನಿಜವಾದ ಧಾರ್ಮಿಕ ಮುಖಂಡರು ಪ್ರೇರಣೆ ನೀಡಲು ಸಾಧ್ಯವೇ? ಎಲ್ಲ ಧರ್ಮಗಳೂ ಸಾರುವ ತಿರುಳು ಒಂದೇ ಎಂಬುದನ್ನು ಮತ್ತೆ ಮತ್ತೆ ನೆನಪಿಗೆ\break ತಂದುಕೊಂಡು ಸಮರಸದ ಭಾವನೆಯನ್ನು ಹರಡಬೇಕಾದುದು ಸಾಮಾಜಿಕ ಪ್ರಜ್ಞೆ ಉಳ್ಳ\break ವಿದ್ಯಾವಂತರ ಕರ್ತವ್ಯವಾಗಬೇಕು.

ನಾವು ದೇವರನ್ನು ಪ್ರೇಮಸ್ವರೂಪ ಎನ್ನುತ್ತೇವೆ. ಅವನು ನಮ್ಮನ್ನು ಎಂದರೆ ಯಾವುದೇ ಒಂದು ಪಥಕ್ಕೆ ಸೇರಿದವರನ್ನು ಮಾತ್ರ ಪ್ರೀತಿಸಿದನೇ? ಅವನು ತನ್ನ ಸ್ವರೂಪವನ್ನೂ ಮಹಿಮೆಯನ್ನೂ ಸತ್ಯವನ್ನೂ ಬೇರಾವ ರೀತಿಯಲ್ಲೂ ತಿಳಿಸಲು ಸಮರ್ಥನಲ್ಲವೇ? ತಿಳಿಸಿಲ್ಲವೇ? ಅವರ ವರ ಶಕ್ತಿ ಸಾಮರ್ಥ್ಯಗಳಿಗನುಗುಣವಾಗಿ ಅವರ ಹಿತದೃಷ್ಟಿಯಿಂದ ತಿಳಿಸಲಾರನೇ? ತಿಳಿಸಲೇ ಇಲ್ಲವೇ? ಈ ಪ್ರಶ್ನೆಗಳನ್ನು ನಾವು ಕೇಳಿಕೊಳ್ಳಬೇಕು. ಸರಿಯಾದ ಸರ್ವಸಮ್ಮತವಾದ ಉತ್ತರವನ್ನೂ ಹುಡುಕಬೇಕು. ನಮ್ಮ ಸಂಕುಚಿತ ದೃಷ್ಟಿ ಬದಲಾಗಬೇಕು. ಆ ದೃಷ್ಟಿ ಎಲ್ಲರೂ ಒಪ್ಪಲೇ ಬೇಕಾದ ಪೂರ್ಣ ಸತ್ಯವನ್ನು ಆಧರಿಸಿರಬೇಕು. ಅಜ್ಞಾನ ಮತ್ತು ಸಂಕುಚಿತತೆಯಿಂದ ದುರಭಿಮಾನ ದುರಹಂಕಾರ ಮತಾಂಧತೆ–ಇವು ಬೆಳೆಯುತ್ತವೆ. ಈ ಗುಣಗಳನ್ನು ಯಾವ ನಿಜವಾದ ಧರ್ಮವೂ ಎತ್ತಿ ಹಿಡಿಯಲಾರದು.

\newpage

ಪ್ರಪಂಚದ ಎಲ್ಲ ಧರ್ಮಗಳ ಪವಿತ್ರಗ್ರಂಥಗಳಲ್ಲೂ ಪರಮಾತ್ಮನ ಸರ್ವಶಕ್ತತೆ, ಸರ್ವಜ್ಞತೆ ಮತ್ತು ಸರ್ವವ್ಯಾಪಿತ್ವವನ್ನು ಕುರಿತು ಸ್ಪಷ್ಟವಾಗಿಯೇ ಸಾರಿದ್ದಾರೆ. ಆ ಮಾತುಗಳನ್ನು ಮನನ ಮಾಡಿದರೆ ನಮ್ಮ ಸಂಕುಚಿತತೆ ಮತ್ತು ಜಗಳಕ್ಕಾಗಿ ಕಾಲು ಕೆರೆಯುವ ಪ್ರವೃತ್ತಿ ಕಡಿಮೆಯಾಗುವ ಸಂಭವವಿದೆ.

‘ಬುದ್ಧಿಯನ್ನು ಚುರುಕುಗೊಳಿಸಿ ನೋಡು, ವಿಶ್ವವೆಲ್ಲ ಅವನಿಂದ ತುಂಬಿದೆ’ (ತೇಜೋಬಿಂದು ಉಪನಿಷತ್ ೧.೨೯)

ಇಲ್ಲಿ ಬುದ್ಧಿಯನ್ನು ಚುರುಕುಗೊಳಿಸುವುದೆಂದರೆ ತೆರೆದ ಹಾಗೂ ಶುದ್ಧ ಮನಸ್ಸಿನಿಂದ ಈ ಪರಮಾದ್ಭುತ ಜಗತ್ತನ್ನು ಪರಿಶೀಲಿಸಿ ಚಿಂತಿಸಿ ನೋಡು ಎಂದರ್ಥ. ಹಾಗೆ ಪರಿಶೀಲಿಸಿ ಪರಿಭಾವಿಸಿದಾಗ ಎಲ್ಲೆಲ್ಲೂ ಆತನಿದ್ದಾನೆಂಬುದು ತಿಳಿದುಬರುವುದು.

‘ಯಾರ ಮಹಿಮೆಯನ್ನು ಹಿಮವತ್ ಪರ್ವತಗಳೂ ಅಗಾಧಸಾಗರಗಳೂ ನದಿಗಳೂ ದಶದಿಕ್ಕುಗಳೂ ಸಾರುತ್ತವೆಯೋ ಆತನನ್ನು ನಮ್ಮ ಪರಿಶುದ್ಧ ನಿವೇದನೆಗಳಿಂದ ಪೂಜಿಸೋಣ’ (ಋಗ್ವೇದ ೧೦–೧೨೧–೩)

‘ಸಪ್ತ ಸಮುದ್ರಗಳೂ ಭೂಮಿಯೂ ಅಲ್ಲಿದ್ದುಕೊಂಡಿರುವ ಎಲ್ಲವೂ ಅವನನ್ನು ಹೊಗಳಿ ಹಾಡುತ್ತವೆ. ಅವನನ್ನು ಹಾಡಿ ಹೊಗಳದ ಯಾವ ವಸ್ತುವೂ ಇಲ್ಲ. ಆದರೆ ಅವು ಮಾಡುವ ಸ್ತುತಿಯನ್ನು ನೀವು ತಿಳಿದುಕೊಳ್ಳುತ್ತಿಲ್ಲ.’ (ಅಲ್ ಕೊರಾನ್, ೧೭.೩೩)

‘ಭಗವಂತನು ಚೈತನ್ಯಮಯನು, ಅವನು ಎಲ್ಲ ವಸ್ತುಗಳಲ್ಲೂ ವ್ಯಕ್ತನಾಗಿದ್ದಾನೆ–ಜೀವಾತ್ಮನು ಶರೀರದಲ್ಲಿ ವ್ಯಕ್ತನಾಗಿರುವಂತ್ ಕ್ರೈಸ್ತರ ಪ್ರಸಿದ್ಧ ಪವಿತ್ರಗ್ರಂಥ (ಫಿಲೋಕಾಲಿಯಾ, ಸಂ.\ ೧.೩೩೭)

ಫಿಲೋಕಾಲಿಯಾ ಗ್ರಂಥವು ಭಗವಂತನ ಮಹಿಮೆಯನ್ನು ಅತ್ಯಂತ ಸ್ಪಷ್ಟವಾಗಿ ಸಾರುವ ಭೂಮ್ಯಾಕಾಶಗಳಲ್ಲಿರುವ ಸಂಗತಿಗಳನ್ನು ಪಟ್ಟಿ ಮಾಡಿ ತಿಳಿಸುತ್ತದೆ:

‘ಸೂರ್ಯಚಂದ್ರ ನಕ್ಷತ್ರಗಳು, ಅವುಗಳ ಚಲನೆ ಮತ್ತು ಸ್ಥಿರತೆ, ಪುತುಭೇದಗಳು, ಮಳೆಗಾಳಿ, ಅಸಂಖ್ಯಜಾತಿಯ ಪ್ರಾಣಿಗಳು, ಸಸ್ಯರಾಶಿ, ಅವುಗಳ ನಾನಾರೂಪಗಳು, ವರ್ಣಗಳು, ಅವುಗಳಲ್ಲಿರುವ ಸಾಂಗತ್ಯ, ಸೌಂದರ್ಯ, ಸಾಮರಸ್ಯ, ಲಯಬದ್ಧತೆ, ಪ್ರಯೋಜನ–ಇವು ಭಗವಂತನ ಅಸ್ತಿತ್ವವನ್ನು ಸಾರುತ್ತವೆ.’ (ಫಿಲೋಕಾಲಿಯಾ ಸಂ.\ ೨.೨೭೯)

‘ಭಗವಂತನೇ ಸಮಸ್ತ ಜಗತ್ತಿನ ಉತ್ಪತ್ತಿಗೆ ಕಾರಣ. ಆತನಿಂದಲೇ ಜಗತ್ತು ನಡೆಯುತ್ತಿದೆ.’ (ಭಗವದ್​ಗೀತೆ. ೧೦.೮)

‘ಕಾರ್ಯವಿಧಾನಗಳಲ್ಲಿ ವೈವಿಧ್ಯವಿದೆ. ಆದರೆ ಆ ಒಬ್ಬ ದೇವನೇ ಎಲ್ಲರಲ್ಲೂ ಪ್ರವರ್ತಿಸು\-ತಿದ್ದಾನೆ’ (ಬೈಬಲ್. ೧ ಕೊ. ೧೨.೬)

‘ಭಗವಂತನ ಇಚ್ಛೆಯೇ ಎಲ್ಲವನ್ನೂ ಚಲಿಸುವಂತೆ ಮಾಡುತ್ತದೆ. ಎಲ್ಲ ಅಸ್ತಿತ್ವಕ್ಕೂ ಸ್ಥಿತಿಗೂ ಅದೇ ಕಾರಣ.'(ಫಿ.\ ಕಾ.\ ೨.\ ೨೭೯)

‘ಅವನಿಗೆ ತಿಳಿಯದೆ ಒಂದು ಎಲೆಯೂ ಬೀಳದು’ (ಅಲ್ ಕೊರಾನ್ ೬.೫೯)

ಪವಿತ್ರತೆ ಮತ್ತು ಅಂತರ್ದೃಷ್ಟಿಯನ್ನು ಪಡೆದುಕೊಂಡ ವ್ಯಕ್ತಿ ಮಾತ್ರ ಪ್ರಕೃತಿಯ ಚಟುವಟಿಕೆಗಳನ್ನು ಸರಿಯಾಗಿ ಸಮಗ್ರವಾಗಿ ಪರಿಶೀಲಿಸಬಲ್ಲ. ಮಾತ್ರವಲ್ಲ ಪ್ರಕೃತಿಯ ಪ್ರತಿಯೊಂದು ಅಂಶವೂ ಅಣುಕಣವೂ ಹೇಗೆ ಜೀವಂತವಾಗಿದೆ ಎಂಬುದನ್ನು ಪ್ರತ್ಯಕ್ಷ ನೋಡಬಲ್ಲ. ಇಡಿಯ ವಿಶ್ವವು ಸರ್ವಜ್ಞನಾದ ವಿಶ್ವನಿಯಾಮಕನ ಪ್ರಜ್ಞೆಯಿಂದ ತುಂಬಿ ತುಳುಕುತ್ತಿದೆ ಎಂಬುದು ಅವನ ಪಾಲಿಗೆ ಪ್ರತ್ಯಕ್ಷ ಅನುಭವದ ಸಂಗತಿಯಾಗುವುದು. - ಸ್ವಮಿ ರಾಮಕೃಷ್ಣಾನಂದ 

\vskip 1.2pt

‘ಕಣ್ಣು ತೆರೆದಿದ್ದರೂ ಸರ್ವಭೂತಗಳಲ್ಲಿಯೂ ಭಗವಂತನಿರುವುದನ್ನು ನೋಡಿದ್ದೇನೆ. ಮನುಷ್ಯರಲ್ಲಿ ಪ್ರಾಣಿಗಳಲ್ಲಿ ಗಿಡಮರಗಳಲ್ಲಿ ಸೂರ್ಯಚಂದ್ರರಲ್ಲಿ ನೆಲದಲ್ಲಿ ಜಲದಲ್ಲಿ ಸರ್ವ\-ಭೂತ\-\break ಗಳಲ್ಲಿಯೂ ಆತನೇ ಇದ್ದಾನೆ’–ಭಗವಾನ್ ಶ‍್ರೀರಾಮಕೃಷ್ಣ.

\vskip 1.2pt

ಅಲ್ಲಿಹನು ಇಲ್ಲಿಲ್ಲವೆಂಬೀ ಸೊಲ್ಲು ಸಲ್ಲದು ಒಳಹೊರಗೆ ನೀನಲ್ಲದೇ ಅನ್ಯತ್ರ ಬೇರಿಲ್ಲ ಎಂಬುದನು ಬಲ್ಲರು ಇಳೆಯೊಳು ಭಾಗವತರಾದವರು - ಕನಕದಾಸ 

\vskip 1.2pt

ಪರಮಸೂಕ್ಷ್ಮವಾದುದರಿಂದ ಆ ಪರತತ್ತ್ವವು ಅಗೋಚರ. ಅದು ದೂರದಲ್ಲಿದೆ ಎಂದು ಭಾಸವಾದರೂ ಹತ್ತಿರದಲ್ಲೇ ಇದೆ.

\vskip 1.2pt

‘ಅವನು ಎಲ್ಲ ಚರಾಚರಭೂತಗಳ ಹೊರಗೊಳಗೆ ಪರಿಪೂರ್ಣನಾಗಿದ್ದಾನೆ, ಮತ್ತು ಚರಾಚರ ರೂಪಿಯೂ ಸಹ ಆಗಿದ್ದಾನೆ. ಅವನು ಸೂಕ್ಷ್ಮನಾಗಿರುವುದರಿಂದ ಅವಿಜ್ಞೇಯನಾಗಿದ್ದಾನೆ. ಅತಿ ಸಮೀಪದಲ್ಲೂ ಅತಿದೂರದಲ್ಲೂ ಇರುವವನು ಅವನೇ ಆಗಿದ್ದಾನೆ’–ಭಗವದ್ಗೀತಾ. ೧೩.೧೫

ಭಗವಂತನು ಎಲ್ಲೆಲ್ಲೂ ಇದ್ದಾನೆ. ಎಲ್ಲ ಜೀವಜಂತುಗಳಲ್ಲೂ ಇದ್ದಾನೆ ಎಂದಾಗ, ನಮಗೆ ಆತನಲ್ಲಿ ನಿಜವಾದ ಶ್ರದ್ಧೆ ಪ್ರೀತಿ ಇರುವುದಾದರೆ, ಎಲ್ಲ ಜೀವ ಜಂತುಗಳ ಮೇಲೂ ನಮ್ಮ ಪ್ರೀತಿ ಕಾಳಜಿಗಳು ಉಕ್ಕಿ ಹರಿಯಬೇಕಲ್ಲವೇ?

ನಿಜವಾದ ಪ್ರೀತಿ ವ್ಯಕ್ತವಾಗುವ ಬಗೆ ಹೇಗೆ? ತ್ಯಾಗ ಸೇವೆಗಳ ಮೂಲಕವಲ್ಲವೇ? ಪ್ರೀತಿ ಪಾತ್ರರ ಮೆಚ್ಚುಗೆಗಾಗಿ ಅವರ ಸಂತೋಷಕ್ಕಾಗಿ ಎಂಥ ತ್ಯಾಗಕ್ಕೂ ಸೇವೆಗೂ ನಾವು ಸಿದ್ಧರೆಂದಾದರೆ ನಮ್ಮಲ್ಲಿ ನಿಜವಾದ ಪ್ರೀತಿ ಉಕ್ಕುತ್ತಿದೆ ಎಂದಾಯಿತಲ್ಲವೇ? ಇಂದು ಧಾರ್ಮಿಕರು ಎನಿಸಿಕೊಂಡವರು ಎಲ್ಲರೆಡೆಗೂ ಆ ಪ್ರೀತಿಯನ್ನು ಹರಿಯಿಸಲು ಸಮರ್ಥರಾಗಿದ್ದಾರೆಯೇ?

ಪರಸ್ಪರ ದ್ವೇಷಿಸಲು ನಮ್ಮಲ್ಲಿ ಬೇಕಷ್ಟು ಧರ್ಮಗಳಿವೆ ಆದರೆ ಪ್ರೀತಿಸಲು ಇಲ್ಲ.

ಧರ್ಮ ಎಂಬುದು ದೇವರತ್ತ ಮನುಷ್ಯನನ್ನು ಕೊಂಡೊಯ್ಯುವ ದಿವ್ಯದಾರಿ. ದೇವರಿಗೆ ಪ್ರಿಯವಾದ ಶ್ರದ್ಧೆ, ತಾಳ್ಮೆ, ಕ್ಷಮೆ, ಪ್ರೀತಿ ಕರ್ತವ್ಯನಿಷ್ಠೆಯೇ ಮೊದಲಾದ ಸದ್ಗುಣಗಳನ್ನು ಬೆಳೆಸಿ ಪೋಷಿಸುವ ನಿಯಮಗಳ ಪಾಲನೆ.

ನಿಸ್ವಾರ್ಥ ಪ್ರೀತಿ ಶಾಂತಿ ಸೌಹಾರ್ದ ಸಾಮರಸ್ಯ ಮತ್ತು ಸಮನ್ವಯವೇ ಮೊದಲಾದ\break ಭಾವನೆಗಳನ್ನು ಸಮಾಜದಲ್ಲಿ ಬಿತ್ತರಿಸಬೇಕಾದ ಧಾರ್ಮಿಕ ಮತಪಥ ಪಂಥಗಳು ದ್ವೇಷಾ\-ಸೂಯೆ ಹಿಂಸೆಗಳನ್ನು ಹೆಚ್ಚಿಸಿ ರಕ್ತಪಿಪಾಸುಗಳಾಗಿಯೇ ಮುಂದುವರಿಯುವಂತಾಗಿದೆಯಲ್ಲ! ಮನು\-ಕುಲವನ್ನು ನಿರಾಸೆಯ ಕೂಪಕ್ಕೆ ತಳ್ಳುತ್ತಿವೆಯಲ್ಲ. ಇದೆಂಥ ಅಣಕ? ವಿರೋಧಾಭಾಸ? ಈ ದುರಂತಕ್ಕೆ ಶಾಶ್ವತ ಪರಿಹಾರವೇನಾದರೂ ಇದೆಯೇ?

ಇದೆ.

ಗಾಡ್ ಅಲ್ಲಾ ಬ್ರಹ್ಮ ಎಂಬೆಲ್ಲ ಹೆಸರುಗಳು ಬೇರೆ ಬೇರೆಯಾದರೂ ಸರ್ವಶಕ್ತನಾದ ಭಗವಂತನು ಒಬ್ಬನೇ. ಎಲ್ಲರೂ ಆತನ ಮಕ್ಕಳೇ. ಕೆಲವರು ಸ್ವಲ್ಪ ಮುಂದಿದ್ದಾರೆ. ಇನ್ನು ಕೆಲವರು ಸ್ವಲ್ಪ ಹಿಂದಿದ್ದಾರೆ. ನಾವೆಲ್ಲರೂ ಪ್ರೀತಿವಿಶ್ವಾಸದಿಂದ ಬದುಕಲು ಕಲಿಯಬೇಕೆಂಬುದೇ ಆತನ ಇಚ್ಛೆ. ಪ್ರೇಮಮಯನಾದ ಭಗವಂತನ ಪ್ರತಿನಿಧಿಗಳಾದ ಬೆಳಕು ಗಾಳಿಗಳು ಎಲ್ಲರ ಮನೆಯಂಗಳಕ್ಕೂ ಹರಿಯುತ್ತವೆ. ಎಲ್ಲೆಡೆಗೂ ಹಬ್ಬಿ ಎಲ್ಲರನ್ನೂ ತಬ್ಬಿ ಅವು ನಿಸ್ವಾರ್ಥಪ್ರೀತಿಯ ಪಾಠವನ್ನೇ ಬೋಧಿಸುತ್ತವೆ. ನಾನಾ ಆಕಾರಗಳಿಂದ ಗೋಚರಿಸಿದರೂ ಆತ ಒಬ್ಬನೇ. ಎಲ್ಲರೂ ಬೇರೆ ಬೇರೆ ಪಥಗಳಿಂದ ಅವನನ್ನು ಸಮೀಪಿಸುತ್ತಿದ್ದಾರೆ ಎಂಬ ಸತ್ಯ ಎಲ್ಲರಿಗೂ ಮನವರಿಕೆ ಯಾಗಬೇಕಾಗಿದೆ. ಅಸಂಖ್ಯ ದೇವದೇವತೆಗಳಾಗಿ ಕಂಡರೂ ಆತನೊಬ್ಬನೇ.

ಪೂಜಾವಿಧಾನಗಳಲ್ಲಿ ವ್ಯತ್ಯಾಸವಿದೆ. ಒಂದೇ ತತ್ತ್ವವನ್ನು ಕೆಲವರು ನಿರಾಕಾರನೆಂದು ಪೂಜಿಸುತ್ತಾರೆ. ಕೆಲವರು ಸಾಕಾರನೆಂದು ಪೂಜಿಸುತ್ತಾರೆ. ನಿಜ ನೀರಿಗೆ ಯಾವ ಆಕಾರವೂ ಇಲ್ಲ. ಧ್ರುವ ಪ್ರದೇಶದಲ್ಲಿ ನಿರಂತರವಾಗಿ ಇದ್ದುಕೊಂಡಿರುವ ಮಂಜು ನಾನಾ ಆಕಾರಗಳಲ್ಲಿ ವ್ಯಕ್ತವಾಗುವುದು. ಮಂಜುಗಡ್ಡೆ ನೀರಲ್ಲದೆ ಬೇರೇನೂ ಅಲ್ಲ. ನೀರು ಆವಿಯಾದರೆ ಅದರ ಅಸ್ತಿತ್ವವೇ ಗೋಚರಿಸದು. ಸತ್ಯದ ಬೆಳಕಿನಲ್ಲಿ ಈ ವೈವಿಧ್ಯ ಹಾಗೂ ವೈವಿಧ್ಯದ ಹಿನ್ನೆಲೆಯಲ್ಲಿರುವ ಏಕತೆಯನ್ನು ಮೊದಲಿಗೆ ಎಲ್ಲ ಧಾರ್ಮಿಕ ಮುಖಂಡರೂ ಸ್ವೀಕರಿಸಿ ಎಲ್ಲರ ಅಭ್ಯುದಯಕ್ಕೂ ಕಾರಣವಾದ ಎಲ್ಲರ ವ್ಯಕ್ತಿತ್ವವೂ ಅರಳುವಂಥ ಸಾರ್ವತ್ರಿಕ ನೈತಿಕ ನಿಯಮಗಳನ್ನು ಬೋಧಿಸಬೇಕು. ಇದನ್ನು ಮಾಡದೇ ಧಾರ್ಮಿಕರು ಶಾಂತಿಗಾಗಿ ಮಾಡುವ ಪ್ರಯತ್ನ ಯಶಸ್ವಿಯಾಗದು. ಜನರ ನಂಬಿಕೆಯನ್ನೂಗಳಿಸದು. ಧಾರ್ಮಿಕ ಮುಖಂಡರು ಜಗಳಕ್ಕಾಗಿ ಕಾಲುಕೆರೆಯುವುದನ್ನು\break ನಿಲ್ಲಿಸದೇ ಜಗತ್ತಿನಲ್ಲಿ ಶಾಂತಿ ಸಾಧ್ಯವಾದೀತೇ?

ಹೇಗೂ ಇರಲಿ.

ಯಾವ ಮತಪಥ ಪಂಥಗಳವರೇ ಆಗಲಿ, ದೇವರಲ್ಲಿ ವಿಶ್ವಾಸವಿರುವವರೆಲ್ಲ ಶಾಂತಿ ಸೌಹಾರ್ದ ಸಮನ್ವಯ ಸದ್ಭಾವನೆಗಳು ಸರ್ವತ್ರವೃದ್ಧಿಯಾಗುವಂತೆ ಭಗವಂತನಲ್ಲಿ ಪ್ರಾರ್ಥಿಸಬೇಕು.

ಎಲ್ಲರೂ ಎಲ್ಲರಿಗಾಗಿ ಪ್ರಾರ್ಥಿಸುವಂತಾದರೆ ಜಗತ್ತಿನಲ್ಲಿ ಶಾಂತಿ ಸೌಹಾರ್ದ ಭಾವನೆಗಳು\break ಉಕ್ಕಿ ಹರಿಯುವುವು. ನೀವೂ ನಿಮ್ಮ ಪಾಲಿನ ಪ್ರಾರ್ಥನೆಯನ್ನು ಸಲ್ಲಿಸುವಿರಾ?

\begin{center}
\textbf{ಈಶ್ವರ ಅಲ್ಲಾ ತೇರೇ ನಾಮ್, ಸಬಕೋ ಸನ್ಮತಿ ದೇ ಭಗವಾನ್}
\end{center}

\chapterend

\addtocontents{toc}{\protect\par\egroup}

