
\chapter*{ಪ್ರಕಾಶಕರ ನುಡಿ}

ಜಗತ್ತಿನ ಸಹಸ್ರಾರು ವರ್ಷಗಳ ಇತಿಹಾಸವು ಮಾನವನಿಗೆ ತೋರಿಸಿಕೊಟ್ಟಿರುವ ಒಂದು ನಿಶ್ಚಿತ ಸತ್ಯವೆಂದರೆ ಬದುಕುವ ಕಲೆ ಅತ್ಯಂತ ಕಷ್ಟಕರವಾದದ್ದು ಎಂಬುದು. ಉಸಿರಾಡುವುದಕ್ಕೆ ತೊಡಗಿದಂದಿನಿಂದ ಮನುಷ್ಯ ಜೀವಿಯು ತಾಯಿ, ತಂದೆ, ಅಣ್ಣತಮ್ಮಂದಿರು, ಅಕ್ಕತಂಗಿಯರು, ಬಂಧುಬಾಂಧವರು, ತನ್ನ ಸುತ್ತಣ ಪರಿಸರ–ಹೀಗೆ ಅನೇಕ ಜನರ, ಅನೇಕ ಪರಿಸ್ಥಿತಿಗಳ ಬಗೆ ಬಗೆಯ ಬಾಂಧವ್ಯಗಳ ಜಟಿಲ ಜಾಲದಲ್ಲಿ ತನ್ನನ್ನು ತಾನು ಹೆಣೆದುಕೊಳ್ಳುತ್ತಾ ಹೋಗುವುದು ಅನಿವಾರ್ಯವಾಗುತ್ತದೆ. ಈ ಎಲ್ಲದರ ನಡುವೆ ತಾನು ತೊಂದರೆಗೆ ಸಿಕ್ಕಿಕೊಳ್ಳದೆ, ತನ್ನ ಸಂಪರ್ಕಕ್ಕೆ ಬರುವ ಇತರರಿಗೆ ತೊಂದರೆ ಕೊಡದೆ, ಸಾಧ್ಯವಾದರೆ ಅವರಿಗೆ ಕಿಂಚಿತ್ತಾದರೂ ನೆರವು ನೀಡುತ್ತಾ ಬಾಳುವುದು ಈ ಜಗತ್ತು ಎಲ್ಲರಿಗೂ ಸಹನೀಯವಾಗುವಂತೆ ಮಾಡುವ ದಾರಿ ಎಂದು ತೋರುತ್ತದೆ.

ಸ್ವಾಮಿ ಜಗದಾತ್ಮಾನಂದರು ಅಂಥದೊಂದು ದಾರಿಯನ್ನು ತೋರಿಸಿದ್ದಾರೆ. ಶ‍್ರೀರಾಮಕೃಷ್ಣ ಮಹಾಸಂಘದ ಆವರಣದಲ್ಲಿ ಅರಳಿದ ಅವರ ಚೇತನವು ಅಲ್ಲಿ ಅವರಿಗೆ ದೊರೆತ ಅನುಭವವನ್ನೂ, ಅವರ ಅತ್ಯಂತ ಆಳವೂ ವಿಸ್ತಾರವೂ ಆದ ಅಧ್ಯಯನವು ನೀಡಿದ ಹೊರೆಯಲ್ಲದ ವಿದ್ವತ್ತನ್ನೂ ಹದವಾಗಿ ಬೆರೆಸಿ, ಎರಡು ಭಾಗಗಳಲ್ಲಿ “ಬದುಕಲು ಕಲಿಯಿರಿ” ಎಂಬ ಗ್ರಂಥವನ್ನು ರಚಿಸಿ ಮೊದಲನೆಯ ಭಾಗವನ್ನು ೧೯೮೧ರಲ್ಲಿ ಮೊದಲ ಸಲ ಪ್ರಕಟಿಸಿದರು. ಇದರ ಎರಡನೆಯ ಭಾಗವು ೧೯೮೬ರಲ್ಲಿ ಪ್ರಕಟವಾಯಿತು. ಇದುವರೆಗೆ ಈ ಎರಡೂ ಭಾಗಗಳು ಸೇರಿದಂತೆ ಒಟ್ಟು ಒಂದು ಲಕ್ಷಕ್ಕೂ ಮೀರಿ ಪ್ರತಿಗಳು ಓದುಗರನ್ನು ತಲುಪಿವೆ ಎಂದರೆ ಅದರ ಜನಪ್ರಿಯತೆಯನ್ನು ಊಹಿಸಿಕೊಳ್ಳುವುದು ಸುಲಭ. ಕನ್ನಡ ಪ್ರಕಾಶನ ಪ್ರಪಂಚದಲ್ಲಿ ದಾಖಲೆಗಳನ್ನು ಸ್ಥಾಪಿಸಿದ ಕೃತಿಗಳಲ್ಲಿ ಇದು ಒಂದು ಎಂಬ, ಕ್ಷಮ್ಯವಾದ ಹೆಮ್ಮೆಗೆ ಈ ಕೃತಿಯು ಅರ್ಹವಾಗಿದೆ.

ಸ್ವಾಮಿ ಜಗದಾತ್ಮಾನಂದರ ಪುಸ್ತಕದ ಒಂದು ವಿಶಿಷ್ಟ ಲಕ್ಷಣದ ಕಡೆಗೆ ಓದುಗರ ಗಮನವನ್ನು ಸೆಳೆಯುವುದು ಅಗತ್ಯ. ಬದುಕನ್ನು ಹಸನುಗೊಳಿಸಿಕೊಳ್ಳಬೇಕು, ಬದುಕುವ ಕಲೆಯನ್ನು ಕರಗತ ಮಾಡಿಕೊಳ್ಳಬೇಕು ಎನ್ನುವ ಸಾಧಕರ ಮೇಲೆ ಉಪದೇಶಗಳ ಹೊರೆಯನ್ನು ಹೊರಿಸಿದರೆ ಆಗುವ ಪ್ರಯೋಜನ ಕಡಮೆ ಎಂಬುದನ್ನು ಅರ್ಥಮಾಡಿಕೊಂಡಿರುವ ಲೇಖಕರು ಪ್ರತಿಯೊಬ್ಬರೂ ತಮ್ಮ ಅಂತರಂಗವನ್ನು, ಮನಶ್ಶಾಸ್ತ್ರೀಯವಾಗಿ ವಿಶ್ಲೇಷಿಸಿ ನೋಡಬೇಕು ಎಂಬುದನ್ನು ಒತ್ತಿ ಹೇಳಿದ್ದಾರೆ. ನಮ್ಮ ಮನಸ್ಸಿನ ಪದರಗಳನ್ನು ಪ್ರಾಮಾಣಿಕವಾಗಿ ನಾವೇ ಬಿಡಿಸಿ ನೋಡಿಕೊಳ್ಳದಿದ್ದರೆ ಲೌಕಿಕ ಜಗತ್ತಿನಲ್ಲಾಗಲಿ, ಆಧ್ಯಾತ್ಮಿಕ ಜಗತ್ತಿನಲ್ಲಾಗಲಿ ನಾವು ಏನನ್ನೂ ಸಾಧಿಸಲಾರೆವು. ಈ ಸತ್ಯವನ್ನು ಮನಗಂಡು ಓದುಗರು ಈ ಕೃತಿಯನ್ನು ಆ ದೃಷ್ಟಿಯಿಂದ ನೋಡಿ ಬದುಕಿನ ಕಲೆಯಲ್ಲಿ ಪರಿಣತಿಯನ್ನು ಸಾಧಿಸುವತ್ತ ಹೆಜ್ಜೆ ಹಾಕುತ್ತಾರೆ ಎಂದು ಆಶಿಸುತ್ತೇವೆ.

೧೯೮೭ ಮತ್ತು ೨೦೦೨ರಲ್ಲಿ ‘ಬದುಕಲು ಕಲಿಯಿರಿ’ಯ ಎರಡೂ ಭಾಗಗಳನ್ನು ಸಂಯುಕ್ತ ಆವೃತ್ತಿಯಾಗಿ ಪ್ರಕಟಿಸಲಾಗಿದ್ದು, ಓದುಗರು ಅದನ್ನು ಆಸಕ್ತಿಯಿಂದ ಸ್ವೀಕರಿಸಿದ್ದರು. ಇದೀಗ ಪುನಃ ಸಂಯುಕ್ತ ಆವೃತ್ತಿಯನ್ನು ಓದಲು ಸುಲಭವಾಗುವಂತೆ ಆಕಾರವನ್ನು ಬದಲಾಯಿಸಿ, ‘ದೇವರಿರುವನೇ’ ಎಂಬ ಒಂದು ಹೊಸ ಅಧ್ಯಾಯವನ್ನು ಮುದ್ರಿಸಿದ್ದೇವೆ. ಈ ಆವೃತ್ತಿಯನ್ನು ಸಹೃದಯ ಓದುಗರು ಆದರದಿಂದ ಸ್ವೀಕರಿಸುವರೆಂದು ನಂಬಿದ್ದೇವೆ.

\bigskip

\noindent ಶ‍್ರೀ ರಾಮಕೃಷ್ಣ ಆಶ್ರಮ\hfill\textbf{ಅಧ್ಯಕ್ಷರು}

\noindent ಯಾದವಗಿರಿ, ಮೈಸೂರು 570 020

\newpage\thispagestyle{empty}


\begin{center}
{\LARGE \textbf{ಅರ್ಪಣೆ}}
\end{center}

\begin{center}
ಬಾನುಬುವಿಗಳ ತಬ್ಬಿಹಿಡಿದ ವೀರ ವ್ಯಕ್ತಿತ್ವ;\\ನರರಲ್ಲಿ ಹರನನ್ನು ಕಂಡ ನವ ವೇದಾಂತ ತತ್ತ್ವ.
\end{center}

\begin{center}
ನೋಟದಿಂದಲೆ ನವಿರೇಳುವ ಧೀರ ನಿಲುವು;\\ಹೃದಯದಲಿ ತುಂಬಿದೆ ವಿಶ್ವದೆಲ್ಲರ ನೋವು ನಲಿವು.
\end{center}

\begin{center}
ಸನಾತನ ಹಿಂದೂಧರ್ಮದ ಸಾಕಾರಮೂರ್ತಿ;\\ದಶದಿಶೆಗೂ ಹರಡಿಸಿದೆ ಭಾರತದ ಕೀರ್ತಿ.
\end{center}

\begin{center}
ದೀನದಲಿತರ ಉದ್ಧಾರ ನೀ ಕಂಡ ಕನಸು;\\ಬದುಕು ಬೆಳಗುವ ತಂತ್ರ ನಿನ್ನಿಂದ ನನಸು.
\end{center}

\begin{center}
ವಿಶ್ವಮಾನವ ನೀನು ವೇದಾಂತಕೇಸರಿ;\\ಜಡತೆ ಬಡಿದೋಡಿಸುವ ಗುರು ನೀನೆ ನರಹರಿ.
\end{center}

\begin{center}
ಶ‍್ರೀರಾಮಕೃಷ್ಣರು ಕಡೆದಂಥ ದೇವಶಿಲ್ಪ;\\‘ಏಳಿ, ಎದ್ದೇಳಿ’ ಕಹಳೆಯೂದಿ ಸವೆದೆ ಕಾಯಕಲ್ಪ.
\end{center}

\begin{center}
ಪ್ರಭು ನಿನ್ನ ಪದತಲದಿ ಶರಣು ಶರಣೀತ;\\ಬಾಳ ಬೆಳಗುವ ಕೈದೀವಿಗೆಯಿಂದು ನಿನಗರ್ಪಿತ.
\end{center}

\begin{center}
ಬಾಳಬುತ್ತಿಯ ಬಗೆಗೆ ಅರಿಯದವ ನಾನು;\\ತಿಳಿಸಿರುವೆ ಅಮೃತತ್ವ ಬಳಿ ಬಂದು ನೀನು.
\end{center}

\begin{center}
‘ಏಳಿ, ಎದ್ದೇಳಿ’ ಕಹಳೆ ಊದಿ ಸಾರಿದೆ ನೀನು;\\ಸೋಲು ಗೆಲುವೆನ್ನದೆ ಅಡಿಯಿಟ್ಟೆ ನಾನು.
\end{center}

\begin{center}
ನಾಳೆ ನಿನ್ನೆಯ ಚಿಂತೆ ಮರೆಯಿಸಿದೆ ನೀನು;\\ಗೋಳಬಾಳಲಿ ಬೆಳಕ ಹರಿಯಿಸಿದೆ ನೀನು.
\end{center}

\begin{center}
ಹೇಳಿ ತಿಳಿಸಿದೆ ಬದುಕಿನೊಳಗುಟ್ಟ ನೀನು;\\ಬೀಳದಂದದಿ ಸಲಹೊ ಮೊರೆಯುವೆನು ನಾನು.
\end{center}

\begin{center}
ಮೇಲುಕೀಳೆನ್ನದೇ ಉದ್ಧರಿಸಿ ನೀನು;\\ಹೇಳಿ ಬರೆಸಿದ ಕೃತಿಯ ಅರ್ಪಿಸುವೆ ನಾನು.
\end{center}


\chapter*{ಮುನ್ನುಡಿ}

\centerline{{\Large\bfseries ೧}}

ಸಾಮಾನ್ಯ ಶಿಕ್ಷಣ ಪದ್ಧತಿಯಲ್ಲಿ ದೊರೆಯದ, ಆದರೆ ಬದುಕಿಗೆ ಉಪಯುಕ್ತವಾಗುವ ಆತ್ಮ ವಿಶ್ವಾಸ ಉತ್ಸಾಹಗಳನ್ನು ಉಕ್ಕಿಸುವ ಸಂಗತಿಗಳನ್ನು ಯುವಕರ ಅಭಿರುಚಿ ಕೆರಳುವಂತೆ ಹೇಗೆ ತಿಳಿಸಬಹುದೆಂದು ನಾನು ಯೋಚಿಸಿದ್ದಿದೆ. ಶಾಲಾಕಾಲೇಜುಗಳ ವಿದ್ಯಾರ್ಥಿಗಳನ್ನುದ್ದೇಶಿಸಿ ಮಾತ ನಾಡುವ ಅವಕಾಶ ಬಂದಾಗಲಂತೂ ಯಾವ ವಿಷಯವನ್ನೇ ಆಗಲಿ, ಹಿರಿಕಿರಿಯರೆಲ್ಲರಿಗೂ ಮನಮುಟ್ಟುವಂತೆ ತಿಳಿಸಿಕೊಡಲು ‘ಪರಮಶಾಸ್ತ್ರಕ್ಕಿಂತ ಸರಿಯುದಾಹರಣೆ’ ಎಂಬ ಕವಿವಾಕ್ಯದ ಸತ್ಯ ಸಾಮರ್ಥ್ಯಗಳ ಅರಿವಾಗಿತ್ತು. ಇಂದಿನ ನಮ್ಮ ಶಿಕ್ಷಣ ವಿಧಾನವು ವಿಚಾರ ಸಂಗ್ರಹಕ್ಕೆ ವಿಶೇಷ ಪ್ರಾಧಾನ್ಯವಿತ್ತು ಶೀಲವನ್ನು ರೂಪಿಸುವಲ್ಲಿ ಅಸಮರ್ಥವಾಗಿರುವುದರಿಂದ ಬದುಕಿನ ಸಮಸ್ಯೆಗಳನ್ನು ಧೈರ್ಯ, ದಿಟ್ಟತನ ಮತ್ತು ಪುಜುಮಾರ್ಗಗಳಿಂದ ಎದುರಿಸಲು ಬೇಕಾಗುವ ಬಲ ಯುವಕರ ಪಾಲಿಗೆ ಇಲ್ಲವಾಗಿದೆ. ಅಧ್ಯಯನದಿಂದ ಬುದ್ಧಿಗೆ ತರಬೇತಿ ಸಿಗುತ್ತದೆ. ಆದರೆ ಸಂಕಲ್ಪ ಮತ್ತು ಭಾವನಾಶಕ್ತಿಗಳ ಸಂಸ್ಕಾರ ಮತ್ತು ನಿಯಂತ್ರಣಕ್ಕೆ ತರಬೇತಿ ಬೇಡವೇ? ಈ ಸಮಸ್ಯೆಯನ್ನು ನಮ್ಮ ಧೀಮಂತ ವರ್ಗ ಸರಿಯಾಗಿ ಅರ್ಥಮಾಡಿಕೊಂಡು ರಚನಾತ್ಮಕವಾಗಿ ಏನು ಮಾಡಬಹುದೆಂಬುದನ್ನು ಯೋಚಿಸಿದಂತಿಲ್ಲ! ಇರಲಿ. ‘ಸಚ್ಚರಿತೆಯ ಒಂದು ಕಾರ್ಯ ಒಂದು ಟನ್ ಉಪದೇಶಕ್ಕೆ ಸಮ.’ ಯಾರ ಬದುಕಿನಲ್ಲೇ ಆಗಲಿ, ಕೆಲವು ಅನುಭವಗಳೂ, ಒಳ್ಳೆಯ ಘಟನೆಗಳೂ ವಿಶೇಷ ಅಭಿರುಚಿಯನ್ನುಂಟುಮಾಡುವುದರ ಜೊತೆಗೆ ಚಾರಿತ್ರ್ಯ ನಿರ್ಮಾಣಕ್ಕೆ ಪ್ರೇರಕವಾಗುವಂಥ ಪ್ರಸಂಗಗಳೂ ಹೌದು. ಅಂಥ ಪ್ರಸಂಗಗಳನ್ನು ಆಧರಿಸಿ ಯುವಕರಿಗೆ ಸ್ಫೂರ್ತಿ ಮತ್ತು ಮಾರ್ಗದರ್ಶನ ನೀಡಬಲ್ಲ ಮನಸೆಳೆಯುವ ಅನೇಕ ಗ್ರಂಥಗಳು ಪಶ್ಚಿಮದೇಶದಲ್ಲಿ ಪ್ರಕಟವಾಗಿವೆ. ಸ್ಯಾಮ್ಯುಯೆಲ್ ಸ್ಮಾಯ್ಲ್ಸ್ ಅವರ ಗ್ರಂಥಗಳನ್ನು ಸ್ವಾತಂತ್ರ್ಯ ಪೂರ್ವದಲ್ಲಿ ನಮ್ಮ ದೇಶದ ವಿದ್ಯಾವಂತರು ಓದಿ ಸ್ಫೂರ್ತಿಪಡೆದ ವಿಚಾರ ಕೇಳಿದ್ದೆ. ಸ್ಮಾಯ್ಲ್ಸ್ ಅವರ \enginline{self-help–‘}ಸ್ವಸಹಾಯ’ ಎನ್ನುವ ಬರಹ ಸುಮಾರು ೧೮೫೮ರಲ್ಲಿ ಪ್ರಕಟವಾಗಿತ್ತು. ಅಲ್ಲಿಂದೀಚೆಗೆ ಅದು ನೂರಾರು ಮುದ್ರಣಗಳನ್ನು ಕಂಡಿದೆ. ಮಾತ್ರವಲ್ಲ, ಯೂರೋಪಿನ ಹೆಚ್ಚುಕಡಿಮೆ ಎಲ್ಲ ಭಾಷೆಗಳಲ್ಲೂ ಬೆಳಕನ್ನು ಕಂಡಿದೆ. ಕನ್ನಡದಲ್ಲಿ ಆ ಮಾದರಿಯ ಒಂದು ಬರಹವೂ ಇದುವರೆಗೂ ಬಂದಂತಿಲ್ಲವೆಂಬುದು ಸಖೇದಾಶ್ಚರ್ಯಕರ ವಿಷಯವಲ್ಲವೆ?

ನಮ್ಮಲ್ಲಿ ದುರ್ಬಲರನ್ನು ಬೈಯ್ಯುವ ಮೂಲಕ, ನಿಂದೆಯ ನುಡಿ ಮತ್ತು ತಿರಸ್ಕಾರ\break ಪ್ರದರ್ಶನದ ಮೂಲಕ ತಿದ್ದಲು ಹೊರಡುವ ಮೇಧಾವಿಗಳಿಗೆ ಕೊರತೆ ಇಲ್ಲ. ವ್ಯಕ್ತಿಯ ಆತ್ಮಗೌರವಕ್ಕೆ ಆಘಾತವಾದರೆ ಅದರಿಂದ ಆತನ ಬದುಕಿಗೆ ಅತೀವ ಹಾನಿಯಾಗುತ್ತದೆಂಬುದನ್ನು ಧೀಮಂತರು ಸರಿಯಾಗಿ ತಿಳಿದಂತಿಲ್ಲ; ಕಳಕಳಿಯಿಂದ ಆ ಬಗ್ಗೆ ಏನು ಮಾಡಬಹುದೆಂಬುದನ್ನು ಯೋಚಿಸಿದಂತಿಲ್ಲ. ಕೆಲವೊಂದು ಸಂದರ್ಭಗಳಲ್ಲಿ ತಪ್ಪುದಾರಿಯಲ್ಲಿ ನಡೆಯುವ ಕೆಲವರನ್ನು ಬೈದು ಭಂಗಿಸಿ ಶಿಕ್ಷಿಸಬೇಕಾಗಬಹುದಾದರೂ ಎಲ್ಲರ ಸಮಸ್ಯೆಗಳಿಗೂ ಅದೊಂದು ಉತ್ತರವಾಗಲಾರದಷ್ಟೆ! ಕಟುಟೀಕೆ, ವಿಮರ್ಶೆ, ನಿಂದೆಗಳಿಗಿಂತಲೂ ಯುವಕರಿಗೆ ತಮ್ಮನ್ನು ತಿದ್ದಿಕೊಳ್ಳಲು ಬೇಕಾಗಿರುವುದು ಯಥಾರ್ಥ ಆದರ್ಶಗಳು, ಮಾದರಿಗಳು ಮತ್ತು ಯೋಗ್ಯ ಮಾರ್ಗದರ್ಶನ. ‘ಬದುಕಲು ಕಲಿಯಿರಿ’ಯಲ್ಲಿ ಯುವಕರ ಈ ಸಮಸ್ಯೆಗಳಿಗೆ ಉತ್ತರ ನೀಡಲು ಯತ್ನಿಸಲಾಗಿದೆ ಎಂಬುದನ್ನು ಕಾಣಬಹುದು. ಒಂದು ರಾಷ್ಟ್ರದ ಉತ್ಸಾಹ, ಶಕ್ತಿ ಮತ್ತು ಭರವಸೆಯ ಪ್ರತೀಕವೇ ಯುವಕರು. ಅವರ ಅದಮ್ಯ ಶಕ್ತಿಯನ್ನು ಯೋಗ್ಯ ಮತ್ತು ಉಪಕಾರಕ ಮಾರ್ಗಗಳಲ್ಲಿ ಹರಿಯುವಂತೆ ಮಾಡ\-ಲಾಗದಿದ್ದರೆ ನಮ್ಮ ಎಲ್ಲ ಯೋಜನೆಗಳೂ ನಿಷ್ಫಲವಲ್ಲವೇ? ಇಂದು ಸಮಾಜದ ಹಣದಿಂದ, ರಾಷ್ಟ್ರದ ಪುಣದಿಂದ ವಿದ್ಯೆಯನ್ನು ಪಡೆದ ತರುಣರ ಮನಸ್ಸು ಎತ್ತ ಸಾಗಿದೆ? ಕಾಯಕದ ಮೇಲಣ ಅವರ ಗೌರವ, ಶ್ರದ್ಧೆಗಳು ಎಂತಿವೆ? ಮಕ್ಕಳು ಓದುಬರಹ ಕಲಿತು ಪರೀಕ್ಷೆ ಪಾಸುಮಾಡಿ ಧನ ಸಂಪಾದನೆ ಮಾಡಿದರೆ ಶಿಕ್ಷಕರು ಮತ್ತು ರಕ್ಷಕರು ತಾವು ಧನ್ಯರಾದೆವೆಂದು ತಿಳಿಯುವಂತಾಗಿದೆ. ಮಕ್ಕಳು ನಿಜವಾಗಿ ಉದಾತ್ತ ಗುಣಗಳ ಆರ್ಜನೆ ಮಾಡಿ ಯೋಗ್ಯ ಮನುಷ್ಯರಾಗುತ್ತಿದ್ದಾರೆಯೆ ಎಂಬುದನ್ನು ಯಾರೂ ಗಣನೆಗೆ ತರುವ ಲಕ್ಷಣ ಕಾಣುತ್ತಿಲ್ಲ. ವಿದ್ಯಾವಂತ ತರುಣರು ದೇಶಕ್ಕೊಂದು ಭೂಷಣರಾಗಿದ್ದಾರೆಯೇ? ರಾಷ್ಟ್ರವನ್ನು ಸರ್ವಾಂಗಸುಂದರವನ್ನಾಗಿ ಮಾಡಬಲ್ಲ ವಿಶ್ವಕರ್ಮ\-ರಾಗಿದ್ದಾರೆಯೇ? ತಮಗೆ ಸಿಕ್ಕಿದ ವಿದ್ಯೆಯಿಂದ ಅವರು ತಮ್ಮನ್ನು ರಕ್ಷಿಸಿಕೊಂಡು ದೇಶವನ್ನು ಕಾಪಾಡಲು ಸಮರ್ಥ\-ರಾಗಿದ್ದಾರೆಯೇ? ಶತಶತಮಾನಗಳಿಂದ ದಮನಕ್ಕೊಳಗಾಗಿರುವ ಲಕ್ಷಲಕ್ಷ ಹಳ್ಳಿಗಳಲ್ಲಿ ಜೀವ ಹಿಡಿದುಕೊಂಡು ಬಾಳನ್ನು ಹೇಗೋ ಸಾಗಿಸುತ್ತಿರುವ ಕೋಟ್ಯಂತರ ಬಡವರ ಬೆವರಿನ ಫಲದಿಂದ ವಿದ್ಯಾಭ್ಯಾಸ, ತಮ್ಮ ಉದ್ಯೋಗ, ಮತ್ತಿತರ ಸುಖಸೌಕರ್ಯಗಳನ್ನು ಪಡೆದ ಯುವಕರು ಸ್ವಲ್ಪವಾದರೂ ಆ ಬಡಜನರ ಹಿತಚಿಂತನೆಯನ್ನು ಮಾಡುವ ಸೌಜನ್ಯ, ಸಹೃದಯತೆಯನ್ನು ಪಡೆದಿದ್ದಾರೆಯೇ? ಬೇಡ, ತಾವು ಆರಿಸಿಕೊಂಡ ಕ್ಷೇತ್ರದಲ್ಲಿ ಉನ್ನತಮಟ್ಟದ ಸಿದ್ಧಿಯನ್ನು ಪಡೆಯುವ ಹಂಬಲವನ್ನಾದರೂ ಬೆಳೆಸಿಕೊಂಡಿದ್ದಾರೆಯೇ? ಕರ್ತವ್ಯನಿಷ್ಠೆಯ ಮನೋಭಾವ\-ವನ್ನಾದರೂ ಅವರು ಕಲಿತ ವಿದ್ಯೆ ಅವರಿಗೆ ನೀಡಿದೆಯೇ? ದೈಹಿಕವಾಗಿ ದುರ್ಬಲರೂ, ಸುಖ\-ಸೌಕರ್ಯಗಳ ದಾಸರೂ, ಮೈಗಳ್ಳರೂ, ಪರಾವಲಂಬಿಗಳೂ ಆದ ಯುವಜನರಿಂದ ದೇಶವು ಏನನ್ನು ನಿರೀಕ್ಷಿಸಬಹುದು? ದುಡಿಮೆಯಲ್ಲಿ ಪ್ರೀತಿ ಎಂಬುದು ಸ್ವಾಭಾವಿಕವಾದ ಒಂದು ಗುಣವಾಗಿ ಯುವಜನಾಂಗದಲ್ಲಿ ಬೆಳೆದಿಲ್ಲವೇಕೆ? ತಪ್ಪಿಸಿಕೊಳ್ಳಲಾರದೆ ಮಾಡಲೇಬೇಕಾದಂಥ ಪರಿಸ್ಥಿತಿ ಒದಗಿಬಂದಾಗ ಒತ್ತಾಯಕ್ಕೆ ಬಲಿಬಿದ್ದು ಗೊಣಗುತ್ತ ಅಸ್ತವ್ಯಸ್ತವಾಗಿ ಕೆಲಸಮಾಡುವ ಪ್ರವೃತ್ತಿ ಅವರಿಗೆ ಭೂಷಣ ತರುವ ಮನೋವೃತ್ತಿಯೇ? ನೀಚಮಟ್ಟದ ಸ್ವಾರ್ಥತೆಯಿಂದ ಸರ್ವನಾಶದ ಪ್ರಪಾತದೆಡೆಗೆ ಓಡುತ್ತಿರುವ ಸಮಾಜವನ್ನು ತಡೆಯಬಲ್ಲ ಯಾವ ಒಂದು ಸೂತ್ರವೂ ಇಲ್ಲವೆ?

\newpage

ರಷ್ಯಾ ದೇಶದ ಶಿಕ್ಷಣತಜ್ಞ ಸುಖೋಮ್ಲಿಂಸ್ಕಿ ಒಂದು ಮಾತು ಹೇಳುತ್ತಾರೆ: ‘ಆತ್ಮವಿಶ್ವಾಸ ಹೀನತೆ, ಅಶ್ರದ್ಧೆ, ಅಪನಂಬಿಕೆಗಳಿಂದ ನೇರವಾಗಿ ಹೊರಹೊಮ್ಮುವ ಫಲವೇ ಜಡತೆ ಮತ್ತು ಆಲಸ್ಯದ ಮನೋಭಾವ–ತನ್ಮೂಲಕ ಅವನತಿ.’\footnote{ ಸುಲೋಮಿಲ್ಂಸಿಲ್ಯ ಇನೊಲ್ಂದು ನುಡಿಯೂ ಲಮನಾಹಲ್:

\engfoot{Love of Work is a moral quality that can only be fostered within the collective. The more powerful the collective’s respect for work, the more effective the moulding of each school child.}} ನಮ್ಮಲ್ಲಿ, ‘ಮನುಷ್ಯನನ್ನು ಸಿಂಹವಾಗಲು ಬಿಡದಿದ್ದರೆ ಅವನು ನರಿಯಾಗುತ್ತಾನೆ’ ಎಂದಂತೆ ಇದು. ಮಕ್ಕಳ ಬಗೆಗೆ ಆತ ಹೇಳಿದ ಮಾತು ಇದಾದರೂ ನಮ್ಮ ದೇಶದ ಮೀಸೆ ಬಂದ ಮಕ್ಕಳಿಗೂ ಈ ರೋಗ ತಗುಲಿಕೊಂಡಿದೆ ಎಂಬುದು ಸತ್ಯ. ಇದಕ್ಕೆ ಒಂದು ಕಾರಣ ರಾಷ್ಟ್ರವು ದೀರ್ಘಕಾಲದಿಂದ ಗುಲಾಮಗಿರಿಯಲ್ಲಿ ಬಾಳಿದ್ದು. ಇನ್ನೊಂದು ಕಾರಣ, ಎತ್ತರದ ಸ್ಥಾನದಲ್ಲಿದ್ದವರೂ, ಧೀಮಂತರೂ ಕೆಳಗಿನವರನ್ನು ಮೇಲೆತ್ತಲು ಪ್ರೀತಿಯಿಂದ ಪ್ರಾಮಾಣಿಕವಾಗಿ ಯತ್ನಿಸಲು ಅಸಮರ್ಥರಾದುದು. ‘ಮನೆಗೆ ಬೆಂಕಿ ಬಿದ್ದಾಗ ನೀರಿಗಾಗಿ ಬಾವಿ ತೋಡಿದ’ ಎಂದಂತೆ ಸುದೂರಭವಿಷ್ಯವನ್ನು ನೋಡಲು ತ್ರಾಣವಿರದ ಮುಖಂಡರು ಸದ್ಯದ ಸಮಸ್ಯೆಗಳನ್ನು ಹೇಗೋ ಪರಿಹರಿಸಿಕೊಳ್ಳುವುದರಲ್ಲೇ ಮಗ್ನರಾದುದು ಮತ್ತೊಂದು ಕಾರಣ.

ಈ ಸಮಸ್ಯೆಗೆ ಉತ್ತರವಿದೆ. ಕಳೆದುಕೊಂಡ ಆತ್ಮಗೌರವವನ್ನು ಪುನಃ ಪ್ರತಿಷ್ಠಾಪಿಸಿದರೆ, ಅದಕ್ಕೆ ಪ್ರತಿಷ್ಠಾಪಿಸಲು ಸರಿಯಾದ ಮಾರ್ಗವನ್ನು ತೋರಿಸಿಕೊಟ್ಟರೆ, ವ್ಯಕ್ತಿಯೊಂದಿಗೆ ಸಮಾಜವೂ ಎದ್ದುನಿಲ್ಲುವುದು. ಆತ್ಮಗೌರವದ ಪ್ರತಿಷ್ಠಾಪನೆಯ ರಹಸ್ಯ ಇಂತಿದೆ: ತನ್ನಲ್ಲಿರುವ ಅಪಾರ ಶಕ್ತಿಯಲ್ಲಿ ವಿಶ್ವಾಸ ಉಂಟಾದಾಗ, ತನ್ನ ವ್ಯಕ್ತಿತ್ವ ಹಾಗೂ ಭವಿಷ್ಯದ ನಿರ್ಮಾತೃ ತಾನೇ ಎಂಬುದನ್ನು ತಿಳಿದುಕೊಂಡಾಗ ಪ್ರತಿಯೊಬ್ಬನೂ ತನ್ನ ಏಳಿಗೆಗೆ ಪ್ರಾಮಾಣಿಕನಾಗಿ ದುಡಿಯಲು ಯತ್ನಿಸುತ್ತಾನೆ. ನೆಲದ ಮೇಲೆ ಬಿದ್ದಾಗ ನೆಲವನ್ನೇ ಆಧರಿಸಿ ಮೇಲೇರಿ ನಿಂತಂತೆ ಎದ್ದು ನಿಲ್ಲುತ್ತಾನೆ. ಈ ದಿಸೆಯಲ್ಲಿ ‘ಬದುಕಲು ಕಲಿಯಿರಿ’ಯಲ್ಲಿ ನಡೆಯಿಸಿದ ಚಿಂತನ ಮಂಥನಗಳು ಆಬಾಲ ವೃದ್ಧರಿಗೂ ಗ್ರಾಹ್ಯವಾಗುತ್ತವೆಂದು ಆಶಿಸಿದ್ದೇನೆ.

ಸ್ವಾತಂತ್ರ್ಯಪೂರ್ವದಲ್ಲಿ ದೇಶಕ್ಕಾಗಿ ದುಡಿದು ಮಡಿದ ಬಹುಮಂದಿ ಧುರೀಣರು ಭಾರತವು ಜಗತ್ತಿಗೆ ನೀಡುವ ಒಂದು ವಿಶಿಷ್ಟ ಆಧ್ಯಾತ್ಮಿಕ ಸಂದೇಶವಿದೆ ಎಂದು ನಂಬಿದ್ದರು. ಮಹಾತ್ಮಾ ಗಾಂಧೀಜೀ ಆ ಸಂದೇಶವನ್ನು ಬದುಕಿನಲ್ಲಿ ಅನುಷ್ಠಾನ ಮಾಡಿ ಜನರನ್ನು ರಾಷ್ಟ್ರಹಿತಕ್ಕಾಗಿ ದುಡಿ ಯಲು ಅಂದು ಪ್ರೇರೇಪಿಸಿದರು. ತನ್ನ ವೈಶಿಷ್ಟ್ಯ ಹಾಗೂ ಸತ್ವಶಕ್ತಿಗಳನ್ನು ಉಳಿಸಿಕೊಂಡು ಜನರನ್ನು ಮೇಲೆತ್ತುವ ಹಂಬಲದಿಂದಲೇ ಸ್ವಾತಂತ್ರ್ಯದ ಹೋರಾಟ ಆರಂಭವಾದದ್ದು. ಸ್ವಾತಂತ್ರ್ಯಕ್ಕಾಗಿ ನಡೆದ ಆ ಹೋರಾಟದ ಅರುಣೋದಯದಲ್ಲಿ ಸ್ವಾಮಿ ವಿವೇಕಾನಂದರು, ಋಷಿಗಳು ಕಂಡ ಮತ್ತು ಶಾಶ್ವತ ನಿಯಮಗಳ ಮೇಲೆ ನಿಂತ ಧರ್ಮತತ್ವ ಯಾರ ಮತೀಯ ಭಾವನೆಗಳನ್ನೂ ನೋಯಿಸದೇ ಎಲ್ಲರಿಗೂ ಉಪಕಾರಕವಾಗಿರುವುದರಿಂದ, ಭಾರತೀಯರು ಆ ತತ್ವಗಳನ್ನು ಅರಿತು ಅರಗಿಸಿಕೊಂಡು ಸದ್ಭಾವನೆಯಿಂದ ಜಗತ್ತನ್ನೆ ಗೆಲ್ಲಬಹುದು ಎಂದು ಕರೆ ನೀಡಿದ್ದರು. ಪ್ರತಿಯೊಬ್ಬನಲ್ಲೂ ಅಡಗಿರುವ ದಿವ್ಯತೆ, ಸಮಗ್ರ ವಿಶ್ವದ ಆಧ್ಯಾತ್ಮಿಕ ಏಕತೆ, ಸರ್ವಧರ್ಮ ಸಮ ನ್ವಯ ದೃಷ್ಟಿ ಮತ್ತು ಜೀವ-ಶಿವ ಸೇವೆ–ಇವು ಬದುಕಿನ ಮೂಲಭೂತ ಪ್ರಶ್ನೆಗಳಿಗೆ ಕಂಡುಕೊಂಡ ನಿಜವಾದ ಉತ್ತರವನ್ನು ಆಧರಿಸಿ ನಿಂತ ಜೀವನ ದರ್ಶನ. ಆದುದರಿಂದಲೇ ಅದು ಸಾರ್ವ ಕಾಲಿಕವೂ, ಸಾರ್ವತ್ರಿಕವೂ ಆಗಿದೆ. ಆ ಮಹಾಭಾವನೆಗಳ ಪ್ರಭಾವ ಒಂದಲ್ಲ ಒಂದು ದಿನ ಸರ್ವತ್ರ ಪ್ರಸಾರವಾಗಿಯೇ ತೀರುತ್ತದೆ ಎಂದು ಅವರು ಸಾರಿದ್ದರು. ವಿಖ್ಯಾತ ಇತಿಹಾಸಜ್ಞ ಟಾಯ್ನಬೀ ಈ ಶತಮಾನದ ಮಧ್ಯಭಾಗದಲ್ಲಿ ಒಂದು ಭವಿಷ್ಯವನ್ನು ನುಡಿದರು. ‘ಇಪ್ಪ ತ್ತೊಂದನೇ ಶತಮಾನದಲ್ಲಿ ಭಾರತ ತನ್ನನ್ನು ಜಯಿಸಿದವರನ್ನು ಜಯಿಸುತ್ತದೆ. ಆದರೆ ರಾಜ ಕೀಯವಾಗಿ ಅಲ್ಲ’; ವಿವೇಕಾನಂದರು ಹೇಳಿರುವಂತೆ ಆಧ್ಯಾತ್ಮಿಕವಾಗಿ.\footnote{\engfoot{India will conquer her conquerers?culturally, not politically}\hfill\engfoot{ —Arnold Toynbee}} ಈ ಆಧ್ಯಾತ್ಮಿಕ ಭಾವನೆಗಳನ್ನು ವಿಜ್ಞಾನದ ನೂತನ ಸಂಶೋಧನೆಗಳ ಬೆಳಕಿನಲ್ಲಿ ಪರಿಶೀಲಿಸಿ, ಅದು ಸತ್ಯವೆಂದು ದೃಢಪಡಿಸಿಕೊಂಡರೆ ನಮ್ಮ ಅರಿವಿಗೆ ಇನ್ನೊಂದು ಪ್ರಬಲ ಆಯಾಮ ದೊರೆತಂತಾಗುವುದು. ಮಾತ್ರವಲ್ಲ, ಶೀಲಸಂವರ್ಧನೆಯ ಮೂಲತತ್ತ್ವವನ್ನೂ ಕಂಡುಕೊಂಡಂತಾಗುವುದು. ಈ ದಿಸೆಯಲ್ಲಿ ನಡೆದ ಚಿಂತನೆ ಮೌಲಿಕವೂ, ಸರ್ವಜನ ಪರಿಗ್ರಾಹ್ಯವೂ ಆಗುವುದರಲ್ಲಿ ಲೇಖಕನಿಗೆ ಸಂದೇಹವಿಲ್ಲ.

ನಮ್ಮ ರಾಷ್ಟ್ರ ಗುಡಿಸಲಿನಲ್ಲಿದೆ. ನಮ್ಮ ಜನರು ಮೇಲೇಳುವಂತಾಗಬೇಕು. ಶತಮಾನಗಳ ದಾಸ್ಯದಿಂದ ಅವರು ತಾವು ಕಳೆದುಕೊಂಡ ವ್ಯಕ್ತಿತ್ವವನ್ನು ಪುನಃ ಪಡೆಯುವಂತಾಗಬೇಕು. ಈ ನಿಟ್ಟಿನಲ್ಲಿ ವಿದ್ಯಾವಂತರ ಪಾತ್ರ ಇನ್ನೂ ಹೆಚ್ಚಿನದು. ಜಗತ್ತಿನ ದುಃಖ ಸಂಕಟಗಳಿಗೆ ಭೌತಿಕ ಸಹಾಯ ಒಂದೇ ಔಷಧವಲ್ಲ. ಪ್ರತಿಯೊಂದು ಮನೆಯನ್ನೂ ಧರ್ಮಛತ್ರವನ್ನಾಗಿ ಮಾಡಬಹುದು. ಎಲ್ಲೆಲ್ಲೂ ಆಸ್ಪತ್ರೆಗಳನ್ನು ಕಟ್ಟಬಹುದು. ಆದರೆ ಮನುಷ್ಯನ ದೃಷ್ಟಿಕೋನ ಬದಲಿಸದೇ, ಸ್ವಭಾವ ಪರಿವರ್ತನೆಯಾಗದೇ, ಚಾರಿತ್ರ್ಯ ಶುಭ್ರವಾಗದೇ, ದುಃಖ ಸಂಕಟಗಳು, ನೋವು ನರಳಾಟಗಳು ಎಂದೂ ದೂರವಾಗುವಂತಿಲ್ಲ. ಮನೆಯ ಮಾಡನ್ನು ಮಾತ್ರ ಏರಿಸಿದರೆ ಸಾಲದು. ಮನುಷ್ಯರ ಮನಸ್ಸನ್ನು ಏರಿಸುವ ಕೆಲಸವೂ ನಡೆಯಬೇಕು. ‘ಬದುಕಲು ಕಲಿಯಿರಿ’ ಈ ದಿಕ್ಕಿನಲ್ಲಿ ಒಂದು ವಿನಮ್ರ ವಿಧಾಯಕ ಯತ್ನ.

ಓದುಗರಿಗೊಂದು ಸೂಚನೆ. ಗ್ರಂಥದಲ್ಲಿ ಪ್ರಸ್ತುತಪಡಿಸಿದ ವಿಷಯಗಳು ಒಂದು ಬಾರಿ ಓದಿದಾಗ ಅರ್ಥವಾದರೂ ಓದಿದ ಉಪಯುಕ್ತ ವಿಚಾರಗಳು ಮನಸ್ಸಿನಲ್ಲಿ ನೆಲೆನಿಲ್ಲಲೂ, ಕಾರ್ಯ ರೂಪಕ್ಕೆ ಬರುವಂತಾಗಲೂ ಪುನಃ ಪುನಃ ಕೆಲವೊಂದು ಭಾಗಗಳನ್ನು ಓದುವ ಆವಶ್ಯಕತೆ ಇದೆ. ಮುಖ್ಯವಾಗಿ ಯುವಕರು ಇದನ್ನು ಗಮನಿಸಬೇಕು.

\newpage

\medskip\centerline{{\Large\bfseries ೨}}

ಸದ್ಯದ ಜಗತ್ತಿನ ಸಾಮಾಜಿಕ ನೈತಿಕ ಪರಿಸ್ಥಿತಿಯನ್ನು ಸಮಾಲೋಚಿಸುವ ಯಾವ ವ್ಯಕ್ತಿ ಗಾದರೂ ಮಾನವಜನಾಂಗದ ಭವಿಷ್ಯ ಅಂಧಕಾರಮಯವಾಗಿ ಕಾಣದಿರದು. ನಾಗರಿಕ, ಮುಂದು ವರಿದ, ಬಲಿಷ್ಠ ಎನಿಸಿಕೊಂಡ ರಾಷ್ಟ್ರಗಳೇ ಜಗತ್ತನ್ನು ಕ್ಷಣಾರ್ಧದಲ್ಲಿ ನಾಶಗೈಯುವ, ಹಿಂದೆ ಎಂದೂ ಕಂಡಿರದ, ಭಯಾನಕ ಶಸ್ತ್ರಗಳ ಸಂಗ್ರಹಕಾರ್ಯದಲ್ಲಿ ನಾ ಮುಂದೆ, ತಾ ಮುಂದೆ ಎಂದು ಸ್ಪರ್ಧಿಸಿ ಇವುಗಳಿಗಾಗಿ ಕೋಟಿಗಟ್ಟಲೆ ಡಾಲರ್ ವ್ಯಯಮಾಡುತ್ತಿವೆ. ‘ಭೂಮಿಯ ಭವಿತವ್ಯ’ ಎನ್ನುವ ತಮ್ಮ ಒಂದು ಗ್ರಂಥದಲ್ಲಿ ಜೋನಾಥನ್ ಷೆಲ್ ಮಾನವ ಜನಾಂಗ ಎಂಥ ದುರಂತದಂಚಿನಲ್ಲಿ ನಿಂತಿದೆ ಎಂಬುದನ್ನು ತಿಳಿಸಿಕೊಟ್ಟಿದ್ದಾರೆ. ಹಲವಾರು ವರ್ಷಗಳ ಹಿಂದೆ ಹಿರೋಷಿಮವನ್ನು ನಾಶಮಾಡಿದ ಅಸ್ತ್ರದೊಂದಿಗೆ ಅತ್ಯಾಧುನಿಕ ಶಸ್ತ್ರಾಸ್ತ್ರಗಳನ್ನು ಹೋಲಿಸಿದರೆ, ಮಾರಕಶಕ್ತಿಯ ದೃಷ್ಟಿಯಿಂದ ಅದು ಈಗಿನ ಶಸ್ತ್ರಾಸ್ತ್ರಗಳ ಮಾರಕಶಕ್ತಿಯ ಮಿಲಿಯದಲ್ಲಿ ಒಂದು ಪಾಲೂ ಇಲ್ಲ! ಇಷ್ಟಾದರೂ ರಾಷ್ಟ್ರಗಳು ಅಣ್ವಸ್ತ್ರಗಳನ್ನು ಸಂಗ್ರಹಿಸುತ್ತ ಪೇರಿಸುತ್ತಲೇ ಇವೆ. ಇದಕ್ಕೆ ವಿರೋಧವಾಗಿ ನಾಗರಿಕ ಜನರ ಪ್ರತಿಕ್ರಿಯೆ ಏನೇನೂ ಸಾಲದು ಎಂಬುದಕ್ಕೆ ಜನರಲ್ಲಿ ಸ್ವಹಿತ, ಸಾಮಾಜಿಕ ಹಿತಚಿಂತನೆಯ ಪ್ರವೃತ್ತಿ ಸತ್ತಿದೆ ಎಂದೇ ಅರ್ಥವೆಂದು ಷೆಲ್ ಖಾರವಾಗಿಯೇ ನುಡಿಯುತ್ತಾರೆ. ಜೀವಿಸಿರುವವರು ಸತ್ತವರನ್ನು ಕಂಡು ಮತ್ಸರಪಡುವಂಥ ಘೋರ ದುರಂತಕ್ಕೆ ಕಾರಣವಾಗುವ ಮನುಷ್ಯನ ದುಷ್ಟ ಪ್ರವೃತ್ತಿ ಹೆಡೆ ಎತ್ತಿ ನಿಂತಿದೆ. ಸಕಲ ಮಾನವ ಜನಾಂಗದ ಕಲ್ಯಾಣಕ್ಕಾಗಿ, ರಚನಾತ್ಮಕ ವಿಧಾನದಿಂದ, ಮೌನವಾಗಿ, ಮನುಷ್ಯನ ಸತ್​ಪ್ರವೃತ್ತಿಗೆ ಸಾಕ್ಷಿಯಾಗಿ, ದುಡಿಯುವ ಸಂಘಟಿತ ಸಂಸ್ಥೆಗಳೂ ಇಲ್ಲದಿಲ್ಲ. ಹಸಿದ ಜನರಿಗೆ ಅನ್ನ ನೀಡುವ, ರೋಗರುಜಿನಗಳನ್ನು ದೂರ ಮಾಡುವ, ಕಾರ್ಮಿಕರ ಹಕ್ಕು ಬಾಧ್ಯತೆಗಳನ್ನು ರಕ್ಷಿಸುವ, ನಾನಾ ರೀತಿಯ ನೈಸರ್ಗಿಕ ಹಾಗೂ ಮನುಷ್ಯಕೃತ ಹಾನಿಗೊಳಗಾದ ಸಂತ್ರಸ್ತರ ನೆರವಿಗಾಗಿರುವ, ಸಂಸ್ಕೃತಿ, ಕಲೆ, ವಿದ್ಯೆಯ ಪ್ರಸಾರಕ್ಕಾಗಿರುವ ವಿಶ್ವಸಂಸ್ಥೆಯ ವಿವಿಧ ಶಾಖೆಗಳಿವೆ, ಇತರ ಖಾಸಗೀ ಸಂಸ್ಥೆಗಳೂ ಇವೆ. ಆದರೆ ರಾಕ್ಷಸೀ ಪ್ರವೃತ್ತಿ, ವಿಜ್ಞಾನ ನೀಡುವ ಆಧುನಿಕ ಸಲಕರಣೆಗಳನ್ನು ಉಪಯೋಗಿಸಿಕೊಂಡು ಸಾತ್ವಿಕಪ್ರವೃತ್ತಿಯನ್ನು ನುಂಗಿ ನೊಣೆಯುವಂತೆ ತೋರುತ್ತಿದೆ! ಒಂದು ಕಡೆ ಉತ್ಪತ್ತಿದಾಯಕವಲ್ಲದ ಕೇವಲ ಜನನಾಶಕ್ಕೆ ಬಳಸಲಾಗುವ ಶಸ್ತ್ರಾಸ್ತ್ರಕ್ಕೆ ಕೋಟಿಗಟ್ಟಲೆ ಹಣ ವ್ಯಯಮಾಡುತ್ತಿದ್ದಾರೆ;\break ಇನ್ನೊಂದೆಡೆ ಮಾರಕ ರೋಗಗಳ ಮೂಲಕ ಜೀವ-ಜೀವನವನ್ನೇ ನಿರ್ಮೂಲಮಾಡುವ–ಕಲಬೆರ\-ಕೆಯ ಆಹಾರ ಸಾಮಗ್ರಿ, ಮನಸ್ಸನ್ನು ದುರ್ಬಲ ಹಾಗೂ ವಿಕಾರ ಗೊಳಿಸುವ ಅಶ್ಲೀಲ ಗ್ರಂಥಗಳು, ಅಶ್ಲೀಲ ಚಲಚ್ಚಿತ್ರಗಳು, ಮತ್ತು ಬರಿಸುವ ವಿವಿಧ ಮಾದಕ ದ್ರವ್ಯಗಳು–ಇವುಗಳ ಮಾರಾಟದಿಂದ ಧನದಾಹಿಗಳಾದ ವ್ಯಾಪಾರೋದ್ಯಮಿಗಳು ಕೋಟಿಗಟ್ಟಲೆ ಹಣ ಸಂಪಾದಿಸುತ್ತಿದ್ದಾರೆ! ಸ್ವಾರ್ಥ\-ಮುಖದ ಅಗ್ನಿಕುಂಡದಲ್ಲಿ ಕೌಟುಂಬಿಕ ಮೌಲ್ಯಗಳು ಆಹುತಿಯಾಗುತ್ತಿವೆ. ನಾಳಿನ ಪ್ರಜೆಗಳಾದ ಯುವಕರು ದುಷ್ಟ ಪ್ರಭಾವದ ಸೆಳೆತಕ್ಕೆ ಸಿಕ್ಕಿ ಸರ್ವತ್ರ ರೋಗಗ್ರಸ್ತ ಸಮಾಜ ನಿರ್ಮಾಣವಾಗುತ್ತಿದೆ.

\newpage

ನೈತಿಕ ಆಧ್ಯಾತ್ಮಿಕ ಮೌಲ್ಯಗಳಲ್ಲಿ ಅಸಡ್ಡೆ, ಇಂದ್ರಿಯಸುಖವೇ ಸರ್ವಸ್ವವೆನ್ನುವ ಜಡವಾದ\-ದಲ್ಲಿನ ಆಸಕ್ತಿ, ಅದಕ್ಕೆ ಪೋಷಕವಾದ ವೈಜ್ಞಾನಿಕ ವಿಚಾರವಿಧಾನ, ವಿಜ್ಞಾನ ನೀಡಿದ ಸಲಕರಣೆಗಳ ದುರುಪಯೋಗ–ಇವು ಮನುಷ್ಯನನ್ನು ಎಂಥ ದುರಂತಕ್ಕೆ ಸಿಲುಕಿಸಿವೆ ಎಂಬುದನ್ನು ತಿಳಿದು ಕೊಳ್ಳಲು ಕಷ್ಟವಿಲ್ಲ. ಜಗತ್ತನ್ನು ಜಯಿಸಿ ತನ್ನ ಆತ್ಮವನ್ನು ಕಳೆದುಕೊಂಡರೆ ಏನು ಮಾಡಿದ ಹಾಗಾಯಿತು? ಕಂಪ್ಯೂಟರನ್ನೋ, ಆಕಾಶಯಾನದ ಅದ್ಭುತಯಂತ್ರವನ್ನೋ ನಿಯಂತ್ರಿಸಲು ಕಲಿತ ವ್ಯಕ್ತಿ ತನ್ನ ಮನಸ್ಸನ್ನು ನಿಯಂತ್ರಿಸಲು ಕಲಿಯದಿದ್ದರೆ ಏನು ಸಾಧಿಸಿದಂತಾಯಿತು? ‘ಜಡವಾದ ಹಾಗೂ ಅದರಿಂದ ಉಂಟಾಗುವ ದುಃಖದುರಂತಗಳನ್ನು ಜಡವಾದದ ವಿಧಾನದಿಂದ ಪರಿಹರಿಸಲು ಸಾಧ್ಯವಿಲ್ಲ. ಪಶ್ಚಿಮದೇಶ ಜ್ವಾಲಾಮುಖಿಯ ಮೇಲೆ ಕುಳಿತಿದೆ. ಅಧ್ಯಾತ್ಮದ ಅಮೃತ ಸಲಿಲದಿಂದ ಜಡವಾದದ ಬೆಂಕಿಯನ್ನು ಶಾಂತಗೊಳಿಸಬೇಕು. ಇಲ್ಲವಾದರೆ ದುರಂತ ಕಟ್ಟಿಟ್ಟದ್ದು’ ಎಂದು ಕಳೆದ ಶತಮಾನದಲ್ಲಿ ಸ್ವಾಮಿ ವಿವೇಕಾನಂದರು ಎಚ್ಚರಿಕೆ ನೀಡಿದ್ದರು. ಅವರು ಎಚ್ಚರಿಕೆ ನೀಡಿದ ಕೆಲವು ದಶಕಗಳಲ್ಲೇ ಎರಡು ಮಹಾ ಯುದ್ಧಗಳು ಸ್ಫೋಟಗೊಂಡು, ಅಸಂಖ್ಯ ಜನನಾಶ ಕಾರ್ಯ ನಡೆಯಿತು ಎಂಬುದು ಐತಿಹಾಸಿಕ ಸತ್ಯ.

ಅಮೇರಿಕದಲ್ಲಿ ನೀಡಿದ ‘ಧರ್ಮಸಾಧನೆ’ ಎಂಬ ಒಂದು ಉಪನ್ಯಾಸದಲ್ಲಿ ಆತ್ಮಾಭಿಮುಖಿಯಾಗಿ ಆಂತರಿಕ ಆನಂದವನ್ನು ಪಡೆಯುವ ಧ್ಯೇಯೋದ್ದೇಶ, ಆ ದಿಸೆಯಲ್ಲಿ ಪ್ರಾಮಾಣಿಕ ಪ್ರಯತ್ನ ಇಲ್ಲದೇ ಮನುಷ್ಯಕುಲ ಶಾಂತಿ ಸಮಾಧಾನ ಪಡೆಯದು ಎಂಬುದನ್ನು ಸ್ವಾಮೀಜಿ ಹೀಗೆಂದರು: “ನೀವು ಎರಡು ಸಹಸ್ರ ಆರೋಗ್ಯ ನಿಕೇತನಗಳನ್ನು ನಿರ್ಮಿಸಬಹುದು. ಆತ್ಮಾನು ಭೂತಿಯಾಗದೇ ಅವುಗಳಿಂದ ಏನು ಬಂದ ಹಾಗಾಯಿತು? ಸಾಮಾನ್ಯ ನಾಯಿಯೊಂದು ತನ್ನ ಸಾಮಾನ್ಯ ಅನುಭವಗಳೊಂದಿಗೆ ಮರಣಹೊಂದುವಂತೆ ನೀವೂ ಸಾಯುವಿರಿ. ನಾಯಿ ತನ್ನ ಮರಣಕಾಲದಲ್ಲಿ ಕೂಗಾಡುವುದು, ಭಯವಿಹ್ವಲವಾಗುವುದು, ದುಃಖತಪ್ತವಾಗುವುದು. ಏಕೆಂದರೆ ಅದು ತನ್ನನ್ನು ಜಡವಸ್ತುವೆಂದೂ, ತನ್ನ ಸರ್ವನಾಶವಾಗುವುದೆಂದೂ ತಿಳಿಯುವುದು. ಸ್ಥಾಯಿತ್ವ ಅಥವಾ ನಿತ್ಯತ್ವ ಎನ್ನುವುದು ನಿಮ್ಮ ಆಂತರ್ಯದಲ್ಲೇ ಇದೆ; ಹೊರಗಡೆ ಇಲ್ಲ.\break ಆಂತರ್ಯದ ಆಳದಲ್ಲಿಯೇ ಅಪರಿವರ್ತನೀಯ ಅಸೀಮ ಆನಂದವಿದೆ. ಧ್ಯಾನವೇ ಅದನ್ನು ತಲುಪುವ ಮಾರ್ಗ. ಪ್ರಾರ್ಥನೆ, ಕ್ರಿಯಾಕಲಾಪಗಳು, ನಾನಾ ವಿಧವಾದ ಉಪಾಸನೆಗಳು– ಧ್ಯಾನಕ್ಕೆ ತರಬೇತಿ ಮಾತ್ರ. ಧ್ಯಾನಬಲದಿಂದ ನಾವು ನಮ್ಮ ಶರೀರದಿಂದ ಬೇರೆಯಾಗಬಹುದು; ಆತ್ಮನನ್ನು ತಿಳಿಯಬಹುದು. ನಮ್ಮ ಜನ್ಮಹೀನ ಮೃತ್ಯುಹೀನ ಸ್ವರೂಪವನ್ನು ದರ್ಶಿಸಬಹುದು. ಆಗ ಯಾವ ದುಃಖಶೋಕವೂ ಉಳಿಯದು. ಈ ಭೂಮಿಯಲ್ಲಿ ಮತ್ತೆ ಜನ್ಮವಾಗದು. ಜೀವಾತ್ಮ ವಿಕಾಸದ ಚಕ್ರದಲ್ಲಿ ಸಿಲುಕದು. ಆಗ ತನ್ನ ನಿತ್ಯ ಶುದ್ಧ ಬುದ್ಧ ಮುಕ್ತಸ್ವಭಾವವನ್ನು ಮಾನವ ಅರಿತುಕೊಳ್ಳುತ್ತಾನೆ.”

ಅಂತರ್ಮುಖಿಯಾಗಿ ಆತ್ಮತತ್ತ್ವವನ್ನು ತಿಳಿದುಕೊಳ್ಳಲು ಪ್ರಾಮಾಣಿಕ ಪ್ರಯತ್ನ ಮಾಡದವರೆಗೆ ಮನುಷ್ಯನ ಬೆಳವಣಿಗೆ ಪೂರ್ಣವಲ್ಲ. ವಿಜ್ಞಾನ ಬಾಹ್ಯಜಗತ್ತನ್ನು ವಿಶೇಷ ರೀತಿಯಿಂದ ಅಧ್ಯಯನ ಮಾಡುತ್ತಿದೆ, ಮನುಷ್ಯನ ಬಗೆಗೂ ಸಾಕಷ್ಟು ವಿಷಯಗಳನ್ನು ಸಂಗ್ರಹಿಸುತ್ತಲಿದೆ. ಶಾಶ್ವತ ಮೌಲ್ಯಗಳ ಅನುಸಂಧಾನಕ್ಕೂ ಅದು ಪ್ರೇರಣೆಯನ್ನು ನೀಡಬಹುದು. ಆದರೆ ಆಧ್ಯಾತ್ಮಿಕ ಅನುಭೂತಿ ಅದರ ಪಾಲಿಗೆ ನಿಲುಕದ್ದು. ಅದನ್ನು ಬೇರೆ ಮಾರ್ಗದಿಂದಲೇ ಕಂಡುಕೊಳ್ಳಬೇಕು. ನಿಷ್ಕಾಮಕರ್ಮದಿಂದ, ಚಿತ್ತಶೋಧನೆಯಿಂದ, ಚಿತ್ತಶುದ್ಧಿಯಿಂದ, ಪ್ರಾರ್ಥನೆ ಉಪಾಸನೆ ಧ್ಯಾನ ವಿಧಾನಗಳಿಂದ, ಇವುಗಳಿಗೆ ಆಧಾರವಾದ ನೈತಿಕ ನಿಯಮಗಳ ಪಾಲನೆಯಿಂದ, ಅದನ್ನು ಪಡೆಯಬೇಕು. ಬೇರೆ ದಾರಿ ಇಲ್ಲ. ಅತ್ಯಂತ ಪ್ರಾಚೀನ ಕಾಲದಲ್ಲಿ ಋಷಿಮುನಿಯೋಗಿಗಳು ಈ ಬಗ್ಗೆ ಸಾಕಷ್ಟು ಸಂಶೋಧನೆಗಳನ್ನು ಕೈಗೊಂಡು ಮನಸ್ಸಿನ ಪರಿಮಿತಿಯನ್ನು ದಾಟಿ, ಪರಮಾರ್ಥವನ್ನು ಪಡೆಯುವ ಪರಿಪೂರ್ಣ ವಿಧಾನವನ್ನು ಕಂಡುಕೊಂಡಿದ್ದರು. ಅಧ್ಯಾತ್ಮ ಹಾಗೂ ಲೌಕಿಕ–ಇವುಗಳ ಸಂಬಂಧದಲ್ಲಿ ಸಮತೋಲನ ತಪ್ಪಿದರೆ ಮನುಷ್ಯನ ಶಾಂತಿ, ನೆಮ್ಮದಿಗಳಿಗೆ ಭಂಗವುಂಟಾಗಿಯೇ ತೀರುವುದು.

ಪಶ್ಚಿಮದ ಜಡವಾದದ ಬಿರುಗಾಳಿ ಭಾರತಕ್ಕೂ ಕಳೆದ ಶತಮಾನದ ಆದಿಯಿಂದಲೇ ಬೀಸ ತೊಡಗಿತ್ತು. ದೀರ್ಘಕಾಲದ ದಾಸ್ಯದಿಂದ ಅರ್ಥಹೀನ ಪರಾನುಕರಣಶೀಲತೆಯೂ ಭಾರತದಲ್ಲಿ ಮನೆಮಾಡಲು ಪ್ರಾರಂಭವಾಗಿತ್ತು. ರಾಷ್ಟ್ರಕ್ಕೆ ಸ್ವಾತಂತ್ರ್ಯವನ್ನು ತಂದುಕೊಡಲು ಹೋರಾಡಿದ ಧುರೀಣರು ನಮ್ಮ ‘ಸ್ವ’ತ್ವ ವೈಶಿಷ್ಟ್ಯಗಳನ್ನು ಉಳಿಸಿಕೊಳ್ಳಲು, ಸ್ವದೇಶೀ ಭಾವನೆಗಳಿಂದ ಪ್ರಭಾವಿತ\-ರಾಗಿ ದುಡಿದರು. ಇಂದು ಆಡಳಿತದಲ್ಲಿ ಪ್ರಜಾಪ್ರಭುತ್ವ ವಿಧಾನವನ್ನು ಅಂಗೀಕರಿಸಿದ ಭಾರತದಂಥ ಹಿರಿಯ ರಾಷ್ಟ್ರ ತನ್ನ ಅಭ್ಯುದಯದ ಪಥದಲ್ಲಿ ಎದುರಿಸಬೇಕಾದ ಸಮಸ್ಯೆಗಳು ಹಲವು ಎಂಬುದನ್ನು ಮನಗಾಣುತ್ತಿದೆ. ಜನರನ್ನು ಸಾಕ್ಷರರನ್ನಾಗಿ ಮಾಡುವ ಕಾರ್ಯವೇ ಎಷ್ಟು ಕಠಿಣ ಎಂಬುದು ಇದೀಗ ಮನವರಿಕೆಯಾಗುತ್ತಿದೆ. ಜನಮಾನಸದಲ್ಲಿ ಉದಾತ್ತ ವಿಚಾರಗಳನ್ನು ಬಿತ್ತಿ, ಉನ್ನತವಾದ ಜೀವನವನ್ನು ನಡೆಯಿಸಲು ತೀವ್ರ ಹಂಬಲವನ್ನುಂಟುಮಾಡುವ ನೈತಿಕ ಸಮುನ್ನತಿಯನ್ನು ಕುರಿತು ಪ್ರಚಾರ ಮತ್ತು ತರಬೇತಿ ಇನ್ನೂ ಕಠಿಣ ಕಾರ್ಯ ಎಂಬುದನ್ನು ಬೇರೆ ಹೇಳಬೇಕಿಲ್ಲ. ಸ್ವಾತಂತ್ರ್ಯಪೂರ್ವದಲ್ಲಿ ಸುಸಂಸ್ಕೃತರೂ, ವಿದ್ಯಾವಂತರೂ, ಉನ್ನತ ಸ್ಥಾನಾಪನ್ನರೂ, ಸ್ವದೇಶಪ್ರೇಮದಿಂದ ಪ್ರೇರಿತರಾಗಿ ಬ್ರಿಟಿಷರ ಗುಂಡಿನೇಟಿಗೆ ಬೆದರದೆ ತಮ್ಮ ಹೃದಯದ ಬಿಸಿರಕ್ತವನ್ನು ಬಸಿಯಲು ಸಿದ್ಧರಾದರು; ಮನೆಮಾರುಗಳನ್ನು ಕಳೆದುಕೊಂಡು ಜೈಲು ವಾಸ ಮಾಡಿದರು. ಸ್ವಾತಂತ್ರ್ಯ ಹೋರಾಟದ ಕಾಲದಲ್ಲಿ ಅನುಷ್ಠಾನ ಮಾಡಿದ ತ್ಯಾಗ ಮತ್ತು ಸೇವೆಯ ಆದರ್ಶಗಳನ್ನು ನಂತರದ ಮುಖಂಡರೂ, ಜನರೂ, ಬಹು ಬೇಗನೇ ಮರೆಯುತ್ತ ಬಂದರು. ಆ ಉನ್ನತ ಆದರ್ಶಗಳನ್ನು ಮರೆತರೆ ಪರಿಸ್ಥಿತಿ ಏನಾಗಬಹುದೆಂಬುದನ್ನು ಅರುವತ್ತ ಐದು ವರ್ಷಗಳ ಹಿಂದೆ ಬ್ರಿಟಿಷ್ ಸರಕಾರದ ಕೈದಿಯಾಗಿದ್ದ ಶ‍್ರೀ ಚಕ್ರವರ್ತಿ ರಾಜಗೋಪಾಲಾಚಾರಿ ಹೀಗೆಂದು ಬರೆದರು: “ನಮಗೆ ಸ್ವಾತಂತ್ರ್ಯ ಲಭ್ಯವಾದೊಡನೆಯೇ ಚುನಾವಣೆಗಳು–ಅದರಿಂದಾಗುವ ಭ್ರಷ್ಟಾಚಾರ,\break ಅಧಿಕಾರ ಮತ್ತು ಸಂಪತ್ತಿನ ಬಲದಿಂದ ದಬ್ಬಾಳಿಕೆ, ಆಡಳಿತದ ಅಸಾಮರ್ಥ್ಯ ಇವೆಲ್ಲವೂ ಸೇರಿ ಜನಜೀವನವನ್ನು ನರಕಸದೃಶವಾಗಿ ಮಾಡುವುವು. ಜನರು ತಮಗೆ ಹಿಂದೆ ದೊರೆಯುತ್ತಿದ್ದ ನ್ಯಾಯಬದ್ಧ, ಸಮರ್ಥ, ಪ್ರಾಮಾಣಿಕ ಹಾಗೂ ಶಾಂತಿಯುತ ಆಡಳಿತವನ್ನು ದುಃಖದಿಂದ ಸ್ಮರಿಸುವರು. ಅಗೌರವ ಹಾಗೂ ಪಾರತಂತ್ರ್ಯದಿಂದ ಬಿಡುಗಡೆಯನ್ನು ಪಡೆದಿದ್ದೇವೆ ಎಂಬುದು ಮಾತ್ರ ನಮ್ಮ ಪಾಲಿನ ಲಾಭ. ವಿಶಾಲ ಮನೋಭಾವದ ವಿಶ್ವಮಟ್ಟದ ಶಿಕ್ಷಣ ಪದ್ಧತಿಯಿಂದ ಮಾತ್ರ ನಾವು ಒಳಿತನ್ನು ಆಶಿಸಬಹುದು. ಈ ಶಿಕ್ಷಣದಿಂದ ನಾಗರಿಕರು ಚಿಕ್ಕಂದಿನಿಂದಲೇ ಸದಾ\-ಚಾರ, ದೈವಭಕ್ತಿ, ಪ್ರೇಮವೇ ಮೊದಲಾದ ಮೌಲ್ಯಗಳನ್ನು ಬೆಳಸಿಕೊಂಡು ವಿಶ್ವಶಾಂತಿಯನ್ನು ಸ್ಥಾಪಿಸಲು ಸಾದ್ಯ. ಇಲ್ಲದಿದ್ದರೆ ಘೋರ ಅನ್ಯಾಯ ಹಾಗೂ ಹಣದ ದುರುಪಯೋಗ ನಡೆಯುವುದು. ಎಲ್ಲರೂ ಪರಸ್ಪರ ಪ್ರೀತಿಸುವುದರಲ್ಲಿ ಆನಂದವನ್ನು ಕಂಡುಕೊಂಡರೆ ಮತ್ತು ದೈವಭೀತಿಯುಳ್ಳವರಾಗಿ ನ್ಯಾಯವಂತರಾಗಿ ಬಾಳಿದರೆ, ಈ ಪ್ರಪಂಚ ಎಷ್ಟು ಸುಖಮಯ\-\break ವಾದೀತು! ಇತರ ದೇಶಗಳಿಗಿಂತ ಭಾರತದಲ್ಲಿ ಇಂತಹ ತತ್ತ್ವಗಳನ್ನು ಆಚರಣೆಗೆ ತರುವಂಥ ಶಕ್ತಿ ಹೆಚ್ಚು ಇದೆ.”

ಚಿಕ್ಕಂದಿನಲ್ಲಿಯೇ ಸದಾಚಾರ, ದೈವಭಕ್ತಿ, ಪ್ರೇಮವೇ ಮೊದಲಾದ ಮೌಲ್ಯಗಳನ್ನು ಬೆಳೆಯಿಸಿ ಕೊಂಡು ವಿಶ್ವಶಾಂತಿಯನ್ನು ಸ್ಥಾಪಿಸಲು ಸಾಧ್ಯ ಎಂದು ಹಿಂದಿನ ತಲೆಮಾರಿನ ಹಿರಿಯರು ಕನಸು ಕಂಡಿದ್ದರು. ಆದರೆ ಇಂದಿನ ಪರಿಸ್ಥಿತಿ ಈ ಭಾವನೆಗಳ ಬೆಳವಣಿಗೆಗೆ ಆಶಾದಾಯಕವಾಗಿದೆಯೇ?

ಪ್ರತಿಯೊಬ್ಬ ಭಾರತೀಯನೂ ತನ್ನ ಭವಿಷ್ಯದ ಬಗ್ಗೆ ಯೋಚಿಸಬಲ್ಲನೇ ಹೊರತು ಸಮಗ್ರ ರಾಷ್ಟ್ರದ ಭವಿಷ್ಯಚಿಂತನೆ ಮಾಡಲಾರ. ತನ್ನ ಕುಟುಂಬ, ಮಕ್ಕಳಿಗೆ ಎಷ್ಟು ಆಸ್ತಿಪಾಸ್ತಿ, ಹಣ ಸಂಗ್ರಹಮಾಡಿಡಬಹುದು ಎಂಬ ಬಗ್ಗೆ ವಹಿಸುವ ವಿಶೇಷ ಕಾಳಜಿಯ ಹತ್ತರಲ್ಲಿ ಒಂದು ಪಾಲು ದೇಶದ ಹಿತಚಿಂತನೆಯಲ್ಲಿ ತೊಡಗಿಸಲಾರ. ಮುಂದಿನ ಜನಾಂಗಕ್ಕೆ ಯಾವ ತೆರನಾದ ರಾಜಕೀಯ ವಿಧಾನ ಬೇಕು? ಎಂಥ ನ್ಯಾಯಾಂಗ ರಚನೆಯಾಗಬೇಕು? ಎಂಥ ಚಾರಿತ್ರ್ಯವಂತ ರಾಜಕೀಯಸ್ಥರು ಬೇಕು?–ಒಟ್ಟಿನಲ್ಲಿ ಎಂಥ ದೇಶವೊಂದನ್ನು ಅವರ ಪಾಲಿಗೆ ನಾವು ಬಿಟ್ಟು ಹೋಗುತ್ತಿದ್ದೇವೆ ಎಂಬುದನ್ನು ಯೋಚಿಸಲಾರ. ದೇಶದ ಪರಿಸ್ಥಿತಿ ಡೋಲಾಯಮಾನವಾದರೂ, ತನ್ನ ಪೀಳಿಗೆ ಹೇಗಾದರೂ ಸುರಕ್ಷಿತವಾಗಿರಬಲ್ಲುದೆಂದೇ ಅವರಲ್ಲಿ ಪ್ರತಿಯೊಬ್ಬರ ನಂಬಿಕೆ. ಈ ಸ್ವಾರ್ಥತೆ ಮತ್ತು ಅನೈಕ್ಯಗಳ ದುರುಪಯೋಗವನ್ನು ಪಡೆದು ತಮ್ಮ ಕಾರ್ಯ ಸಾಧಿಸಹೊರಟ ವಿರೋಧೀಶಕ್ತಿಗಳ ಪರಿಚಯವೂ ಅವನಿಗಿರದು.

ವಿದ್ಯಾಭ್ಯಾಸ ಮಾಡುತ್ತಿರುವ ಆಧುನಿಕ ಯುವಕರ ಧ್ಯೇಯೋದ್ದೇಶಗಳು ಆಶಾದಾಯಕ\-ವಾಗಿವೆಯೇ? ಎಷ್ಟೋ ಕಡೆಗಳಲ್ಲಿ ವಿದ್ಯಾರ್ಥಿಗಳು ಮುಷ್ಕರಹೂಡಿ ಪರೀಕ್ಷೆಗಳನ್ನು ಬಹಳವಾಗಿ ಕಡಿಮೆ ಮಾಡಿಕೊಂಡಿದ್ದಾರೆ. ಕಾಪಿಹೊಡೆಯುವ ಹಕ್ಕನ್ನೂ ಸ್ಥಾಪಿಸಿಕೊಳ್ಳುತ್ತಿದ್ದಾರೆ. ಪರೀಕ್ಷೆಗೆ ಮೊದಲೇ ಪ್ರಶ್ನಪತ್ರಿಕೆಗಳನ್ನು ಪಡೆಯುತ್ತಿದ್ದಾರೆ. ಲಂಚಪ್ರಲೋಭನೆಯ ಬಲದಿಂದ ಓದದೇ, ಬರೆಯದೇ ಶೇಕಡಾ ತೊಂಬತ್ತು ಅಂಕಗಳನ್ನು ಪಡೆಯುವ ವಿಕ್ರಮ ವಿಸ್ತರಿಸುತ್ತಲಿದೆ. ಇಂಥವರೇ ಮುಂದೆ ಮುಖಂಡ\-ರಾಗಿಯೋ, ಅಧಿಕಾರಿಗಳಾಗಿಯೋ ದೇಶಕ್ಕೆಂಥ ಅಪಘಾತ ಮಾಡಿಯಾ\-ರೆಂಬು\-ದನ್ನು ಯೋಚಿಸುವವರಾರು? ಬಹುಮಂದಿ ತಂದೆತಾಯಂದಿರು ತಮ್ಮ ಮಕ್ಕಳ ವಿದ್ಯಾಭ್ಯಾಸಕ್ಕಾಗಿ ಬಹಳ ಕಷ್ಟಪಡುತ್ತಾರೆ; ಸಾಕಷ್ಟು ಹಣವನ್ನೂ ವ್ಯಯಿಸುತ್ತಾರೆ—ಮಕ್ಕಳು ಹೇಗಾದರೂ ಹೋರಾಡಿ ತಮ್ಮ ಮುಂದಿನ ಜೀವನವನ್ನು ಸುಖಮಯವಾಗಿ ಮಾಡಿಕೊಳ್ಳಲಿ ಎಂಬ ಉದ್ದೇಶದಿಂದ. ಆದರೆ ಯುವಕರು ಆಧುನಿಕ ಸುಖಭೋಗಾಸಕ್ತಿಯನ್ನು ಕೆರಳಿಸುವ ಅಗ್ಗದ ಪ್ರಚಾರ ಪ್ರಲೋಭನೆಯ ಸೆಳೆತಕ್ಕೆ ಸಿಕ್ಕಿ, ದುಡಿಮೆಯ ಮಹಿಮೆಯನ್ನರಿಯದೇ ಯೌವನದ ಮಹತ್ವ ಪೂರ್ಣಕಾಲವನ್ನು ಅವ್ಯವಸ್ಥಿತ ರೀತಿಯಲ್ಲಿ ಕಳೆದು, ತಮ್ಮ ದುರ್ಬಲ ವ್ಯಕ್ತಿತ್ತ್ವದಿಂದ ಕುಟುಂಬಕ್ಕೂ, ಸಮಾಜಕ್ಕೂ ಭಾರವಾಗುತ್ತಿದ್ದಾರೆ. ಅಲ್ಪ ಸ್ವಲ್ಪ ವಿದ್ಯೆ ಕಲಿತ ಹಳ್ಳಿಯ ಯುವಕ, ಹಳ್ಳಿಯ ಯಾವ ಕೆಲಸದಲ್ಲೂ ಆಸ್ಥೆ ತಾಳಲಾರ. ಕಡಿಮೆ ಶ್ರಮ, ಹೆಚ್ಚು ಸಂಬಳ ಅವನ ಆದರ್ಶ. ಉದ್ಯೋಗದ ಸುರಕ್ಷೆ, ಮನೋರಂಜನೆಯ ಅನುಕೂಲತೆಗಳುಳ್ಳ ನಗರದ ಕಡೆಗೆ ಅವನ ಓಟ. ಇನ್ನು ತಾಂತ್ರಿಕ ವಿದ್ಯೆಗಳಿಸಿದ ಪರಿಣತರು ಭಾರತವನ್ನು ಬಿಟ್ಟು ಅಮೇರಿಕಕ್ಕೆ ಎಂದು ಹೋಗುವುದೆಂದು ಕಾದುಕುಳಿತಿದ್ದಾರೆ. ತಮ್ಮ ಸುತ್ತಮುತ್ತಲ ಜನರಿಗೆ ಅನುಕೂಲರಾಗಿ ದುಡಿದು ಸಾರ್ಥಕ್ಯ ಕಾಣುವ ವಾತಾವರಣ, ತರಬೇತಿ, ಶಿಕ್ಷಣಗಳು ಅವರಿಗೆ ಸಿಗುತ್ತಿಲ್ಲ. ಸರಸ್ವತಿಯ ಮಂದಿರಗಳೆಂದು ಕರೆಯಲ್ಪಡುವ ವಿದ್ಯಾಸಂಸ್ಥೆಗಳು, ವಿಶ್ವವಿದ್ಯಾಲಯಗಳು ರಾಜಕೀಯದ ರಾಗದ್ವೇಷ, ಭ್ರಷ್ಟಾಚಾರಗಳ ಕೇಂದ್ರಗಳಾಗಿ ಪರಿಣಮಿಸುತ್ತಿವೆ. ಉದ್ಯೋಗದ ಬೇಟೆಯಲ್ಲಿನ ಸ್ಪರ್ಧೆಯಿಂದ ವಿಭಿನ್ನ ಜಾತಿಗಳಲ್ಲಿ ದ್ವೇಷಾಸೂಯೆ ವರ್ಧಿಸುತ್ತಲಿವೆ.

ಈ ಪರಿಸ್ಥಿತಿಯ ಅರಿವು ವಿದ್ಯಾವಂತರನ್ನು ಎಚ್ಚರಿಸಿ, ಯಾವುದಾದರೂ ರಚನಾತ್ಮಕ ಕಾರ್ಯಕ್ಕೆ ಪ್ರೇರಿಸುವ ಸಂಭವವಿದೆಯೇ? ಕಳೆದ ಶತಮಾನದಲ್ಲೇ ವಿದ್ಯಾವಂತರಿಗೆ ಸ್ವಾಮಿ ವಿವೇಕಾನಂದರು ಎಚ್ಚರಿಕೆ ನೀಡಿದರು: “ತುಳಿತಕ್ಕೊಳಗಾದ ಕೋಟಿಗಟ್ಟಲೆ ಬಡಜನರ ಹೃದಯದ ರಕ್ತದಿಂದ, ಅವರ ಶ್ರಮದ ಫಲದಿಂದ, ವಿದ್ಯಾಭ್ಯಾಸವನ್ನು ಪಡೆದು ಸುಖಸೌಕರ್ಯಗಳಲ್ಲಿ ತೇಲಾಡುವವರು ಆ ಜನರ ಒಳಿತನ್ನು ಕುರಿತು ಸ್ವಲ್ಪವಾದರೂ ಯೋಚಿಸದಿದ್ದರೆ ಅಂಥವರನ್ನು ದೇಶದ್ರೋಹಿಗಳು ಎನ್ನುತ್ತೇನೆ.”

‘ಸ್ತ್ರೀಯರ ಉದ್ಧಾರ, ಜನಸಮೂಹದ ಜಾಗರಣ–ಇವು ಮೊದಲು ಆಗಬೇಕು. ಆಗ ಮಾತ್ರ ಭಾರತದೇಶಕ್ಕೆ ಏನಾದರೂ ಒಳಿತಾಗಲು ಸಾಧ್ಯ.’

‘ಜೀವನದುದ್ದಕ್ಕೂ ಸಮಾಜಹಿತಚಿಂತನೆಯ ಕಾರ್ಯಮಾಡುತ್ತಾ ಬಂದ ಅಥವಾ ಆ ಕಾರ್ಯ ಮಾಡಲು ಯತ್ನಿಸುತ್ತಿರುವ ನಾನು ನಿಮಗೊಂದು ವಿಚಾರ ಹೇಳುತ್ತೇನೆ. ನೀವು ಅಧ್ಯಾತ್ಮಶೀಲ ರಾಗದೇ ಭಾರತದ ಪುನರುತ್ಥಾನ ಸಾಧ್ಯವಿಲ್ಲ. ಧರ್ಮ, ಧರ್ಮ ಒಂದೇ ಭಾರತದ ಜೀವನ ಸರ್ವಸ್ವ. ಎಂದು ಧರ್ಮ ಮಾಯವಾಗುವುದೋ ಅಂದು ಭಾರತ ಸಾಯುವುದು. ಆದುದರಿಂದ ಪ್ರತಿಯೊಂದು ಪ್ರಗತಿಕಾರ್ಯದ ಮೊದಲು ಧಾರ್ಮಿಕ ಪುನರುತ್ಥಾನ ಭಾರತದಲ್ಲಿ ಆವಶ್ಯಕ.’

‘ಇತರ ವಿಚಾರಗಳು ಆವಶ್ಯಕವಲ್ಲ ಎಂದು ನಾನು ಹೇಳುತ್ತಿಲ್ಲ.... ನಮ್ಮ ಜೀವನದ ರಕ್ತವೇ ಆಧ್ಯಾತ್ಮಿಕತೆ. ಅದು ಸ್ವಚ್ಛವಾಗಿ ಹರಿದರೆ, ಶುದ್ಧವಾಗಿ ಶಕ್ತವಾಗಿ, ಸತ್ವಸ್ಫೂರ್ತಿಗಳಿಂದ ಕೂಡಿ ಹರಿದರೆ, ಎಲ್ಲವೂ ಸರಿಹೋಗುವುದು. ರಾಜಕೀಯ, ಸಾಮಾಜಿಕ, ಇತರ ಭೌತಿಕವೇ ಮೊದಲಾದವುಗಳಿಗೆ ಸಂಬಂಧಿಸಿದ ತಪ್ಪು ತಡೆಗಳು ದೇಶದ ಬಡತನದ ಸಮಸ್ಯೆ ಕೂಡ, ಜೀವನ ರಕ್ತ ಶುದ್ಧವಾಗಿದ್ದರೆ ಸರಿಯಾಗುತ್ತದೆ.’

‘ಪ್ರತಿಯೊಂದು ವ್ಯಕ್ತಿಯನ್ನೂ, ಪ್ರತಿಯೊಂದು ರಾಷ್ಟ್ರವನ್ನೂ ಮಹತ್ವಿಕೆಗೇರಿಸಲು ಮೂರು ಸಂಗತಿಗಳು ಬೇಕು. ಒಂದು–ಒಳಿತಿನ ಶಕ್ತಿಯಲ್ಲಿ ಅಚಲ ವಿಶ್ವಾಸ; ಎರಡು–ಸಂಶಯ, ಅಸೂಯೆಗಳ ಅಭಾವ ಮತ್ತು ಮೂರು–ತಾವು ಒಳ್ಳೆಯವರಾಗಲು ಯತ್ನಿಸುತ್ತ ಇತರರಿಗೆ ಒಳಿತನ್ನು ಮಾಡ\-ಲೆಳಸುವ, ಎಲ್ಲರಿಗೂ ಸಹಾಯಮಾಡುವ ಪ್ರವೃತ್ತಿ.’

‘ಇನ್ನೊಂದು ಮಹತ್ವದ ಪಾಠವನ್ನು ನೆನಪಿಡಬೇಕು–ಅನುಕರಣೆ ನಾಗರಿಕತೆ ಅಲ್ಲ. ಹೇಡಿಯಂತೆ ಅನುಕರಿಸುವುದು ಎಂದೂ ಪ್ರಗತಿಗೆ ಸಾಧಕವಲ್ಲ. ಅದು ಭಯಾನಕ ಹೀನಮಟ್ಟದ ಚಿಹ್ನೆ. ನಿಜವಾಗಿಯೂ ನಾವು ಇತರರಿಂದ ಕಲಿಯಬೇಕಾದುದು ಎಷ್ಟೋ ಇದೆ. ಕಲಿಯಲು ಒಪ್ಪದವನು ಸತ್ತಂತೆ. ಒಳಿತಾದುದೆಲ್ಲವನ್ನೂ ಇತರರಿಂದ ಕಲಿಯಬೇಕು. ಆದರೆ ನಿಮ್ಮದೇ ಆದ ರೀತಿಯಲ್ಲಿ ಅದನ್ನು ಸ್ವೀಕರಿಸಿ ಅರಗಿಸಿಕೊಳ್ಳಿ. ಇತರರಾಗಬೇಡಿ.’

‘ಒಂದು ಲಕ್ಷಮಂದಿ ನರನಾರಿಯರು, ಪವಿತ್ರತೆಯ ಸ್ಫೂರ್ತಿಯಿಂದ ಉತ್ಸಾಹಿತರಾಗಿ\break ಭಗವಂತನಲ್ಲಿ ನಿತ್ಯಶ್ರದ್ಧೆಯಿಂದ ಬಲಯುತರಾಗಿ ತುಳಿತಕ್ಕೊಳಗಾದ ದರಿದ್ರರಾದ ಮತ್ತು ಕೆಳಗೆ ಬಿದ್ದ ಜನರಿಗಾಗಿ ಅನುಕಂಪೆಯಿಂದೊಡಗೂಡಿ, ಸಿಂಹಸದೃಶ ಧೈರ್ಯವನ್ನು ಪಡೆದವರಾಗಿ\break ಮುಕ್ತಿಯ ಸಂದೇಶವನ್ನು ಸಹಾಯದ ಸಂದೇಶವನ್ನು ಸಾಮಾಜಿಕ ಜಾಗರಣ ಮತ್ತು ಉತ್ಥಾನದ ಸಂದೇಶವನ್ನು ದೇಶದ ಉದ್ದಗಲಕ್ಕೂ ಬೋಧಿಸುತ್ತಾ ಹೋಗಬೇಕು.’

‘ಎಂದು ನೂರಾರು ಉದಾರ ಹೃದಯಿಗಳಾದ ನರನಾರಿಯರು ಪ್ರಪಂಚದ ಭೋಗ ಮತ್ತು ಸುಖಸೌಕರ್ಯಗಳ ಲಾಲಸೆಯನ್ನು ತೊರೆದು, ದಾರಿದ್ರ್ಯ ಮತ್ತು ಅಜ್ಞಾನದಲ್ಲಿ ಮೆಲ್ಲಮೆಲ್ಲನೆ ಆಳಆಳಕ್ಕೆ ಮುಳುಗುತ್ತಿರುವ ಕೋಟಿ ಕೋಟಿ ಭಾರತೀಯರ ಮೇಲ್ಮೆಗಾಗಿ ಶಕ್ತಿ ಮೀರಿ ದುಡಿಯು ವರೋ, ಅಂದು ನಮ್ಮ ದೇಶ ಉದ್ಧಾರವಾಗುವುದು.’\footnote{ಭವ್ಯ ಭಾರತ ನಿರ್ಮಾಣ, ಶ‍್ರೀ ರಾಮಕೃಷ್ಣ ಆಶ್ರಮ, ಮೈಸೂರು}

ಮೇಲಿನ ಭಾವನೆಗಳನ್ನು ಈ ಸದ್ಯದ ವಿದ್ಯಾವಂತ ಸಮಾಜ ಸ್ವೀಕರಿಸಿ ಅವನ್ನು ಕಾರ್ಯರೂಪಕ್ಕೆ ತರುವ ಸಂಭವವಿದೆಯೇ? ದೇಶದ ಉದ್ಧಾರ, ತ್ಯಾಗ, ಸೇವೆ–ಇವೆಲ್ಲ ಭಾವನಾತ್ಮಕ ಉದ್ಗಾರಗಳು, ಭಾಷಣೋಪಯೋಗಿ ಸಾಮಗ್ರಿಗಳು–ಕಾರ್ಯರೂಪಕ್ಕೆ ಬರುವ ಸಂಗತಿಗಳಲ್ಲ ಎಂದು ಅವರು ಹೇಳದೇ ಬಿಟ್ಟಾರೆಯೇ? ಮೌಲ್ಯಗಳನ್ನು ಒಪ್ಪುವುದು ಕ್ರಾಂತಿಕಾರಕ ಬದಲಾವಣೆಗೆ ವಿರೋಧವಾದ ಪ್ರವೃತ್ತಿಯಾಗದೇ? ಅವು ಇಂದಿನ ಜೀವನದಲ್ಲಿ ‘ರೆಲವೆಂಟೆ’? ಅವುಗಳಿಗೆ ‘ರ್ಹೆಟರಿಕ್ ವೇಲ್ಯೂ’ ಮಾತ್ರ ಇರಬಹುದೇ?

ವಿದ್ಯಾವಂತರೆನ್ನಿಸಿಕೊಂಡವರು ಒಪ್ಪಲಿ, ಬಿಡಲಿ, ವಿದ್ವಾಂಸರ ಅಭಿಪ್ರಾಯ ಬೇರೆಯೇ ಇದೆ. ಆಧುನಿಕ ಸಮಾಜಶಾಸ್ತ್ರಜ್ಞರು, ಧರ್ಮವೇ ಸಂಸ್ಕೃತಿಯ ಅಡಿಗಲ್ಲು ಎಂಬುದನ್ನು ಸ್ಪಷ್ಟವಾಗಿ ಸಾರುತ್ತಾರೆ. ಸಮಾಜದಲ್ಲಿ ಹಿಂದೆ ಧರ್ಮವು ಪ್ರಮುಖ ಪಾತ್ರವಹಿಸಿದ್ದರೂ, ಇಂದಿನ ವೈಜ್ಞಾನಿಕ ಯುಗದಲ್ಲಿ ಅದೊಂದು ಪಳೆಯುಳಿಕೆ ಎನ್ನುವ ಕೆಲವು ರಾಜಕೀಯ ಸಿದ್ಧಾಂತವಾದಿಗಳ ಅಭಿಪ್ರಾಯ ಸರಿಯಾದುದಲ್ಲ ಎಂದು ಹೇಳುತ್ತಾರೆ. ಕ್ರಿಸ್ಟೋಫರ್ ಡವ್​ಸನ್ ತಮ್ಮ ‘ಇಂಕ್ವಾಯರಿಸ್ ಇಂಟು ರಿಲಿಜನ್ ಅಂಡ್ ಕಲ್ಚರ್​’ ಎನ್ನುವ ಗ್ರಂಥದಲ್ಲಿ ಹೇಳಿದ ಈ ಮಾತು ಗಮನಾರ್ಹ: “ಪ್ರಪಂಚದ ಮಹಾ\-ನಾಗರಿ\-ಕತೆಗಳು ಮಹಾಧರ್ಮಗಳಿಗೆ ಕಾರಣಗಳಲ್ಲ. ಧರ್ಮಗಳು ನಾಗರೀಕತೆಯ ಉಪಫಲಗಳಲ್ಲ. ಮಹಾಧರ್ಮಗಳೇ ಮಹಾನಾಗರಿಕತೆಯ ಅಡಿಗಲ್ಲು. ಯಾವುದೇ ಒಂದು ಸಮಾಜ ಧರ್ಮವನ್ನು ಕಳೆದುಕೊಂಡರೆ, ತನ್ನ ಸಂಸ್ಕೃತಿಯನ್ನೂ ಕಳೆದುಕೊಂಡಂತೆ.” ಸೊರೋಕಿನ್ ಹಾಗೂ ಟೊಯ್ನಬೀ ಕೂಡಾ ಈ ಅಭಿಪ್ರಾಯವನ್ನು ಸಮ್ಮತಿಸುತ್ತಾರೆ.\footnote{\engfoot{Religion is not a matter of personal sentiment that has nothing to do with the objective realities of society but is, on the contrary, the heart of social life and the root of every living culture. We are just beginning to understand how intimately and profoundly the vitality of any society is bound up in its religion. It is the religious impulse which supplies the cohesive force which unifies the society and the culture. The great civilizations of the world do not produce the great religions as a kind of cultural byproduct; in a very real sense, the religions are the foundations on which the great civilizations rest. A society which has lost its religion becomes sooner or later a society which has lost its culture.?Christopher Dowson: Inquiries into Religion and Culture, Sheed and Ward}}

ಸಾಮಾನ್ಯರಿಗೆ ಇಷ್ಟು ತಿಳಿದುಕೊಳ್ಳಲು ಕಷ್ಟವಾಗದು: ಪ್ರತಿಯೊಂದು ನಾಗರಿಕತೆಯ ಹುಟ್ಟು ಹಾಗೂ ಬೆಳವಣಿಗೆಯ ಮೂಲದಲ್ಲಿ ಒಂದು ಶ್ರದ್ಧಾಕೇಂದ್ರವಿದೆ. ಪುನರ್​ನಿರ್ಮಾಣ ಎಂದರೆ ಮನುಷ್ಯನ ಹೃದಯದಲ್ಲಿ ಒಮ್ಮತವನ್ನುಂಟುಮಾಡುವಂಥ ಭಾವದ ನಿರ್ಮಾಣ. ನಾಗರಿಕತೆ, ಶಾಂತಿ, ಪ್ರಗತಿ, ಅಭ್ಯುದಯ–ಇವು ಮನುಷ್ಯರಲ್ಲಿ ಪರಸ್ಪರ ಸಹಕಾರವಿಲ್ಲದೆ ಸಾಧ್ಯವಿಲ್ಲ. ಸಹಕಾರ ಎನ್ನುವುದು ಎಲ್ಲರನ್ನೂ ಒಂದುಗೂಡಿಸುವ ಸೂತ್ರವಿಲ್ಲದೆ ಸಾಧ್ಯವಾಗದು. ಎಲ್ಲರೂ ನಂಬ\-ಬಹು\-ದಾದ ಈ ಶ್ರದ್ಧಾಕೇಂದ್ರವೇ ಈ ಸಹಕಾರದ ಸೂತ್ರ. ಶ್ರದ್ಧೆಯ ಕೇಂದ್ರಬಿಂದುವೇ ಆತ್ಮ ಅಥವಾ ಪರಮಾತ್ಮ. ಆ ಪರಮಾತ್ಮನ ಅಸ್ತಿತ್ವವನ್ನು ಅನುಭವಿಸಿ, ಆ ಅನುಭವವನ್ನು ಜೀವನ ಮತ್ತು ಚಾರಿತ್ರ್ಯದ ಮೂಲಕ ತೋರಿಸಿಕೊಟ್ಟರೆ, ಪರಮಾತ್ಮನಲ್ಲಿರುವ ಶ್ರದ್ಧೆಯನ್ನು ಪುನಃ ಪ್ರತಿಷ್ಠಾಪನೆ ಮಾಡಿದಂತಾಗುತ್ತದೆ. ಉನ್ನತ ಆದರ್ಶಕ್ಕಾಗಿ ಒಂದುಗೂಡಿ ದುಡಿಯುವ ಪ್ರೇರಣೆ ಆಗ ಪ್ರತಿ\-ಯೊಬ್ಬನಿಗೂ ಲಭ್ಯವಾಗುತ್ತದೆ. ನಮ್ಮ ದೇಶದಲ್ಲಿ ಪ್ರತಿಯೊಂದು ಸಾಮಾಜಿಕ ಪುನರುತ್ಥಾನದ ಹಿನ್ನೆಲೆಯಲ್ಲಿ ಆಧ್ಯಾತ್ಮಿಕ ವೀರರ ದೇಣಿಗೆ ಅಪಾರವಾದುದು ಎಂಬುದನ್ನು ಇತಿಹಾಸ ಪ್ರಜ್ಞೆಯುಳ್ಳವರು ತಿಳಿಯಬಲ್ಲರು.

ನಾವು ಬದುಕು ಎಂದು ತಿಳಿದುಕೊಂಡದ್ದು ವೃತ್ತದ ಅರ್ಧಭಾಗ ಮಾತ್ರ. ಇನ್ನರ್ಧ ಭಾಗ ಬದುಕಿನ ಆಚೆಗೆ, ದೇಹಕ್ಕೆ ಅತೀತವಾದ ಅಸ್ತಿತ್ವದಲ್ಲಿ ಅಡಗಿಕೊಂಡಿದೆ. ಇದರ ಬಗೆಗೆ ‘ಇದ ಮಿತ್ಥಂ’ ಎಂದು ನಿಶ್ಚಿತವಾಗಿ ಸಾಮಾನ್ಯರಾದ ನಮಗೇನೂ ತಿಳಿಯುವಂತಿಲ್ಲ. ನದಿ ಎಂದು ಹೇಳಿದಾಗ, ನದಿಯ ಉಗಮಸ್ಥಾನ, ಅದರ ಪಾತ್ರ, ಅದರ ಇಕ್ಕೆಲಗಳಲ್ಲಿರುವ ಕಟ್ಟಡ ಪಟ್ಟಣಗಳು, ಅದು ಸೇರುವ ಸಮುದ್ರ–ಇವು ನಮ್ಮ ಅರಿವಿಗೆ ನಿಲುಕುತ್ತವೆ. ಆದರೆ ಇದು ನದಿಯ ಪೂರ್ಣ\break ಸ್ವರೂಪವಲ್ಲ.

ನದಿಯ ಸ್ವರೂಪದ ಇನ್ನೊಂದು ಮುಖವಿದೆ. ಅದು ಸೂಕ್ಷ್ಮವಾದುದು: ಸೂರ್ಯಕಿರಣಗಳು ಸಮುದ್ರದ ನೀರನ್ನು ಆವಿಯಾಗಿಸುತ್ತವೆ; ಗಾಳಿ ಅದನ್ನು ಪರ್ವತಾಗ್ರಕ್ಕೆ ಒಯ್ಯುತ್ತದೆ; ಅಲ್ಲಿ ಮಳೆಯಾಗಿ ಸುರಿದ ನೀರು ನದಿಯಲ್ಲಿ ಹರಿದು ತಿರುಗಿ ಸಮುದ್ರವನ್ನು ಸೇರುತ್ತದೆ. ಅಂತೆಯೇ ನಮ್ಮ ಜೀವನ ಪ್ರವಾಹದಲ್ಲೂ ಕಾಣುವ, ಕಾಣದ ಶಕ್ತಿಗಳು ಕೆಲಸ ಮಾಡುತ್ತಿವೆ. ಹುಟ್ಟಿ ಮೈವಡೆದ ವ್ಯಕ್ತಿ, ಮರಣದ ಜವನಿಕೆ ಬೀಳುವವರೆಗಿನ ಜೀವನವನ್ನೇ ಸರ್ವಸ್ವ ಎಂದುಕೊಂಡಿರುತ್ತಾನೆ. ಆದರೆ ದೇಹವನ್ನು ಧರಿಸಿದ ಚೇತನ ದೇಹದೊಂದಿಗೇ ನಾಶವಾಗುವುದಿಲ್ಲ ಎಂಬುದು ಕೆಲವರ ಪಾಲಿಗೆ ನಿಚ್ಚಳವಾಗಿ ತಿಳಿಯುತ್ತದೆ. ಈ ವಿಚಾರವನ್ನು ಸಾಕ್ಷಾತ್ಕರಿಸಿಕೊಂಡವರ ಪಾಲಿಗೆ, ದೇಹಾತೀತ ಅಸ್ತಿತ್ವ ಹಾಗೂ ಜೀವಾತ್ಮನ ಇರುವಿಕೆ–ನಂಬಿಕೆ ಅಥವಾ ಕಲ್ಪನೆಯಲ್ಲ; ಅನುಭವ ಗಮ್ಯವಾದ ಸತ್ಯ. ಆರು ಸಹಸ್ರ ವರ್ಷಗಳ ಹಿಂದೆಯೇ ಮನುಷ್ಯನು ಕೇವಲ ಶರೀರವಲ್ಲ, ಶರೀರಧಾರಿ ಎಂಬ ಸತ್ಯ ಭಾರತೀಯ ಋಷಿಮುನಿಗಳಿಗೆ ತಿಳಿದಿತ್ತು. ಒಂದು ನಿಯಮವನ್ನನುಸರಿಸಿ ಜೀವಿ ಈ ಜಗತ್ತಿಗೆ ಬಂದು ಹೋಗುತ್ತಾನೆಂಬುದನ್ನು ಆ ದ್ರಷ್ಟಾರರು ತಿಳಿದಿದ್ದರು. ವಿಜ್ಞಾನದ ನೂತನ ಸಂಶೋಧನೆಗಳ ಬೆಳಕಿನಲ್ಲಿ ಈ ವಿಚಾರಗಳ ಚಿಂತನ-ಮಂಥನ ಈ ಗ್ರಂಥದಲ್ಲಿ ನಡೆದಿದೆ. ಕರ್ಮಸಿದ್ಧಾಂತವೆಂದೊಡನೆ ಮೂಢನಂಬಿಕೆ, ಅವೈಜ್ಞಾನಿಕ, ಗೊಡ್ಡುಕಂತೆ ಎಂದು ಮೂಗು ಮುರಿಯುವವರು ನಮ್ಮಲ್ಲಿ\break ಹೆಚ್ಚುತ್ತಲೇ ಇದ್ದಾರೆ. ಹೆಚ್ಚು ದನಿ ಏರಿಸಿ ಈ ಸಿದ್ಧಾಂತಗಳ ಬಗೆಗೆ ಯಾರು ಬೈಯುತ್ತಾರೋ, ಅವರು ಹೆಚ್ಚು ಬುದ್ಧಿವಂತರೆಂದು ಪರಿಗಣಿತರಾಗುವ ದುಷ್ಟ ರೂಢಿ ಬಂದುಬಿಟ್ಟಿದೆ! ಯಾವುದೇ ವಿಚಾರವನ್ನೂ ಆಮೂಲಾಗ್ರವಾಗಿ ಅಧ್ಯಯನ ಮಾಡದೇ ವಿಮರ್ಶಿಸುವುದು, ಟೀಕಿಸುವುದು ವೈಜ್ಞಾನಿಕ ವಿಧಾನವಂತೂ ಅಲ್ಲ. ಇಲ್ಲಿ ನಡೆಯಿಸಿದ ವಿಚಾರ ವಿಮರ್ಶೆಗಳು ನಿದ್ರೆಯನ್ನು ನಟಿಸುವವರಿಗಲ್ಲ. ನಂಬುವವರು ತಮ್ಮ ನಂಬಿಕೆಗಳನ್ನು ಸಾಕ್ಷ್ಯಾ ಧಾರಗಳ ಬೆಳಕಿನಲ್ಲಿ ಪರಿಶೀಲಿಸಿ ದೃಢಪಡಿಸಿಕೊಳ್ಳಬಹುದು; ಕುಹಕಿಗಳ ಕುತರ್ಕ ಜಾಲದಿಂದ ಸಂಶಯಗ್ರಸ್ತರಾಗಬೇಕಾಗಿಲ್ಲ ಎಂದಷ್ಟೇ ಹೇಳ\break ಬಯಸುತ್ತೇನೆ.

ಕಷ್ಟ, ನೋವು, ನರಳಾಟಗಳಲ್ಲಿ ಸಿಲುಕಿದವರನ್ನು ಕಂಡು ಅವರಿಗೆ ತಮ್ಮಿಂದಾದ ಸಹಾಯ ಮಾಡುವುದು ಸತ್ಕರ್ಮವೆನಿಸುತ್ತದೆ. ‘ಕಷ್ಟ ಸಂಕಟಗಳು ಕರ್ಮದೋಷದಿಂದ ಬರುತ್ತವೆ, ನಾವೇನು ಸಹಾಯ ಮಾಡಲು ಸಾಧ್ಯ? ಮಾಡಿದ್ದುಣ್ಣೋ ಮಹಾರಾಯ!’ ಎಂದು ಹಂಗಿಸುವ, ಕಡೆಗಣಿಸುವ ದೃಷ್ಟಿಕೋನ, ಪ್ರವೃತ್ತಿಯನ್ನು ಕರ್ಮಸಿದ್ಧಾಂತ ಎಂದೂ ಎತ್ತಿ ಹಿಡಿಯುವುದಿಲ್ಲ. ಕಷ್ಟದಲ್ಲಿ ತೊಳ\-ಲಾಡುವ ಇತರರನ್ನು ಕಂಡು, ಕೈಲಾದ ಸಹಾಯ ಮಾಡದಿರುವುದು ದೋಷವೂ ಪಾಪಕರ್ಮವೂ ಆಗುತ್ತದೆ. ಕಷ್ಟದಲ್ಲಿ ತೊಳಲುವವರು ತಮ್ಮ ಸದ್ಯದ ದುಃಸ್ಥಿತಿಗೆ ಧೃತಿಗೆಟ್ಟು ಇತರರನ್ನೇ ದೂರುತ್ತ ಕಾಲಕಳೆಯುವುದಕ್ಕಿಂತ, ಪರಿಸ್ಥಿತಿಯ ಒತ್ತಡದಿಂದ ಪಾರಾಗಲು, ಭಗವಂತನಲ್ಲಿ ವಿಶ್ವಾಸವಿಟ್ಟು, ಪ್ರಾಮಾಣಿಕರಾಗಿ ಹೋರಾಡಲೇಬೇಕು. ಹಾಗೆ ಹೋರಾಡದಿರುವುದೂ ಕರ್ಮದೋಷವಾಗುವುದು. ವಿಜ್ಞಾನನಿಷ್ಠನಾಗಿದ್ದುಕೊಂಡು ಜಡವಾದದ ಬೆನ್ನುಹತ್ತದಂತೆ ಮಾಡಲು, ಧಾರ್ಮಿಕ\-ನಾಗಿದ್ದೂ ಮತಾಂಧತೆ, ಮೂಢನಂಬಿಕೆಗಳ ದಾಸನಾಗದಂತೆ ಮಾಡಲು, ಜೀವನದಲ್ಲಿ ಬರುವ ಕಷ್ಟಕಂಟಕಗಳನ್ನು ಸಹಿಸುವ ಸಾಮರ್ಥ್ಯವನ್ನು ನೀಡಿ, ಇತ್ತ ಸಾಹಸಪ್ರವೃತ್ತಿಯನ್ನು ಉಂಟು\-ಮಾಡುವ ಪರಿಪಕ್ವ ದೃಷ್ಟಿಯನ್ನು, (ಸರಿಯಾಗಿ ತಿಳಿದುಕೊಂಡರೆ) ‘ಕರ್ಮ’ವೆನ್ನುವ ಈ ವಿಶ್ವ\-ನಿಯಮ ನೀಡುತ್ತದೆಂದು ಕೇಸೀ ದಾಖಲೆಗಳು ಅಸಂಖ್ಯ ಪುರಾವೆಗಳ ಮೂಲಕ ದೃಢೀಕರಿಸುತ್ತವೆ. ಹಿಂದೆ ವಿಜ್ಞಾನದ ಬೆಳವಣಿಗೆಗೆ ವಿರೋಧ ತಂದ ಮತೀಯ ಶಕ್ತಿಗಳೇ, ಪಶ್ಚಿಮದಲ್ಲಿ ಕೇಸೀ ವಿಚಾರಧಾರೆಯನ್ನು ಹತ್ತಿಕ್ಕಲು ನಾನಾರೀತಿಯಲ್ಲಿ ಯತ್ನಿಸುತ್ತಿವೆ\footnote{ಆದಲೆ ಅಂಥ ಪ್ರಯತ್ನ ವಿಫಲವಾಗಲೇಬೇಕು. \engfoot{‘You can threaten to torture a man for discovering the truth; you can call him a fool and try to laugh him out of the court; but that does not alter the truth.’ ‘It lies not in your power’ said Andrew Melville, ‘to hang or exile the truth. Truth may be attacked, delayed, suppressed, mocked at, but time brings in its revenges and in the end truth prevails. A man must have a care that he is not fighting against the truth.’? The Daily Study Bible: The Gospel of Mark by William Barclay, Page 102; Published by Theological Publications in Inida, Malleswarm West, Bangalore.}} ಎಂದು ತಿಳಿದುಕೊಂಡರೆ ಈ ಬಗೆಯ ಅಧ್ಯಯನಕ್ಕೆ ಇರುವ ತೊಂದರೆಗಳನ್ನು ತಿಳಿದುಕೊಂಡಂತಾಗುವುದು.

ಕರ್ಮಸಿದ್ಧಾಂತವನ್ನು ತಪ್ಪಾಗಿ ತಿಳಿದುಕೊಂಡು, ಅದನ್ನು ವಿಧಿವಾದ, ಆಲಸ್ಯಕ್ಕೆ ಕಾರಣವಾದ ಸಿದ್ಧಾಂತ ಎಂದು ಟೀಕಿಸುವವರು ನಮ್ಮಲ್ಲಿ ಸರ್ವತ್ರ ಕಂಡುಬರುತ್ತಿದ್ದಾರೆ. (ಪ್ರತಿಯೊಂದು ಸಿದ್ಧಾಂತವನ್ನು ತಪ್ಪು ತಿಳಿದುಕೊಳ್ಳುವವರೂ, ಸ್ವಾರ್ಥಕ್ಕಾಗಿ ದುರುಪಯೋಗಪಡಿಸುವವರೂ ಇದ್ದೇ ಇರುತ್ತಾರೆ. ಸಿದ್ಧಾಂತದ ಸತ್ಯತ್ವ ಮಿಥ್ಯತ್ವಗಳನ್ನು ಅಂಥ ದುರ್ಬಲ ಉದಾಹರಣೆಗಳ ಅಧ್ಯಯನದ ಮೂಲಕ ನಿರ್ಣಯಿಸುವುದು ಮೂರ್ಖತನವಲ್ಲವೇ?) ತನಗೆ ಅಯಾಚಿತವಾಗಿ ಪ್ರಾಪ್ತವಾದ ಸುಪ್ತಾವಸ್ಥೆಯಲ್ಲಿನ ಅದ್ವಿತೀಯ ಅನುಭವಗಳಿಂದ ಎಡ್ಗರ್ ಕೇಸೀ ಈ ಬಗ್ಗೆ ಚೆಲ್ಲಿದ ಬೆಳಕು ನಿಜವಾಗಿಯೂ ಬದುಕಿಗೊಂದು ಹೊಸ ಆಯಾಮ, ಹೊಸ ತಿರುವು ನೀಡುವುದರಲ್ಲಿ ಸಂದೇಹವಿಲ್ಲ. ಸುಮಾರು ೨೫ ವರ್ಷಗಳ ಹಿಂದೆ ಜೀನಾ ಸೆರ್ಮಿನಾರ ಅವರ \enginline{Many Mansions} ಗ್ರಂಥವನ್ನು ಕಂಡಂದಿನಿಂದ ಇದುವರೆಗೂ ಈ ಬಗ್ಗೆ ದಾಖಲೆಗಳನ್ನು ಸಂಗ್ರಹಿಸಿ, ಚಿಂತನ-ಮಂಥನ ನಡೆಸಿ, ಈ ಸಂಬಂಧವಾದ ವಿಚಾರಗಳನ್ನು ಪ್ರಸ್ತುತಪಡಿಸಿದ್ದೇನೆ. ಮನುಷ್ಯರ ಸುಖದುಃಖಗಳ ಮೂಲವನ್ನು, ಅದನ್ನು ನಿಯಂತ್ರಿಸುವ ನಿಯಮವನ್ನು, ತಿಳಿದುಕೊಳ್ಳದೇ, ಜೀವನದ ಅರ್ಥ ಉದ್ದೇಶಗಳನ್ನು ತಿಳಿಯುವಂತಿಲ್ಲ. ಆ ಅರ್ಥ ಉದ್ದೇಶಗಳನ್ನು ಅರಿಯದ ಜೀವನ ಜೀವನವೇ? ಅದು ‘ಜಾಯಸ್ವ-ಮ್ರಿಯಸ್ವ’ ಗುಂಪಿಗೆ ಸೇರಿದ ಜೀವನವಾಗದೇ?

ವಿಜ್ಞಾನದ ಪರಿಮಿತಿ, ಮೌಲ್ಯದ ಮಹತ್ವ, ನಿಸ್ವಾರ್ಥ ಪ್ರೇಮದ ಶಕ್ತಿ, ಮನಶ್ಶಾಂತಿಯ ಮಾರ್ಗ, ಚಾರಿತ್ರ್ಯಬಲ ಸಂವರ್ಧನೆಯ ವಿಧಾನ, ನೀತಿಯ ನೆಲೆಗಟ್ಟು, ಸುಖದುಃಖಗಳ ಮರ್ಮ, ಪವಾಡ ಮತ್ತು ಅದ್ಭುತ ಘಟನೆಗಳ ಹಿನ್ನೆಲೆ, ಪ್ರಾರ್ಥನೆ, ಉಪಾಸನೆ, ಧ್ಯಾನ ವಿಧಾನಗಳ ರಹಸ್ಯ, ವ್ಯಕ್ತಿಯ ಅಭ್ಯುದಯ ಹಾಗೂ ಸಮಾಜ ಕಲ್ಯಾಣದ ಒಳಗುಟ್ಟು, ದೇವರು, ಧರ್ಮಗಳ ಮೂಲ ಸ್ವರೂಪ–ಇವೆಲ್ಲವನ್ನೂ ಈ ಗ್ರಂಥದಲ್ಲಿ ಚರ್ಚಿಸಿ, ಆಧುನಿಕರ ಕೆಲವು ಸಂಶಯಗಳಿಗೆ ಉತ್ತರ ಕೊಡುವ ವಿನಮ್ರ ಪ್ರಯತ್ನ ಮಾಡಲಾಗಿದೆ.

ಈ ಗ್ರಂಥರಚನೆಯಲ್ಲಿ ಕನ್ನಡ ಹಾಗೂ ಆಂಗ್ಲಭಾಷೆಯ ಹಲವಾರು ಗ್ರಂಥಗಳಿಂದ\break ಪ್ರಯೋಜನ ಪಡೆದಿದ್ದೇನೆ. ಆಯಾಯ ಸಂದರ್ಭಗಳಲ್ಲಿ ಗ್ರಂಥ-ಗ್ರಂಥಕರ್ತರ ಹೆಸರುಗಳನ್ನು ಅಲ್ಲಲ್ಲೇ ಸೂಚಿಸಲು ಯತ್ನಿಸಿದ್ದೇನೆ. ವಿವಿಧ ಕ್ಷೇತ್ರಗಳಲ್ಲಿ ದುಡಿಯುತ್ತಿರುವ ಹಿರಿಯರ ಕಿರಿಯರ ಜೀವನ ಅನುಭವಗಳನ್ನು ಕೇಳಿ ತಿಳಿದುಕೊಂಡದ್ದನ್ನು ಈ ಬರಹದಲ್ಲಿ ಬಳಸಿಕೊಂಡಿದ್ದೇನೆ.\break ಅನೇಕ ಮಹನೀಯರ ಸದ್ಗ್ರಂಥಗಳ ಸದ್ವಿಚಾರಗಳನ್ನು ಅಳವಡಿಸಿಕೊಂಡಿದ್ದೇನೆ. ಅವರೆಲ್ಲರಿಗೂ\break ಹೃತ್ಪೂರ್ವಕ ಕೃತಜ್ಞತೆಗಳನ್ನು ಅರ್ಪಿಸುತ್ತೇನೆ.

ಸದ್ವಿಚಾರಗಳ ಪ್ರಸಾರದಲ್ಲಿ ವಿಶೇಷ ಆಸಕ್ತಿ ತಳೆದ, ಅಧ್ಯಾತ್ಮಶೀಲ ಡಾ.\ ಎ.\ ಚಂದ್ರಶೇಖರ ಉಡುಪ ಇವರು ತಮ್ಮ ‘ವಿವೇಕ ಪ್ರಕಾಶನ’ದ ಮೂಲಕ ಈ ಬರಹವನ್ನು ಪ್ರಕಟಿಸಿದ್ದರು. ಇದೀಗ ಆಶ್ರಮದಿಂದ ಪ್ರಕಟಿಸಲು ಅನುವುಮಾಡಿಕೊಟ್ಟಿದ್ದಾರೆ. ಅವರಿಗೆ ವಿಶೇಷ ಕೃತಜ್ಞತೆ. ಮೈಸೂರಿನ ಶ‍್ರೀ ಎಂ.\ ಹೆಚ್.\ ತಿಪ್ಪೇಸ್ವಾಮಿ, ಇವರು ಶ್ರದ್ಧಾಭಕ್ತಿಯಿಂದ ಈ ಗ್ರಂಥದ ಹಸ್ತಪ್ರತಿಯನ್ನು ತಯಾರಿಸಿದ್ದಾರೆ. ಇವರೆಲ್ಲರಿಗೂ ವಿಶೇಷ ನೆನಕೆಗಳನ್ನು ಸಲ್ಲಿಸುತ್ತೇನೆ.

ಸ್ಯಾನ್ ಫ್ರಾನ್ಸಿಸ್ಕೋ ವೇದಾಂತ ಕೇಂದ್ರದ ಮುಖ್ಯಸ್ಥರಾದ ಪೂಜನೀಯ ಸ್ವಾಮಿ\break ಪ್ರಬುದ್ಧಾನಂದಜೀ ಹಲವು ಅಮೂಲ್ಯ ಗ್ರಂಥಗಳನ್ನು ಕಳಿಸಿ ನೀಡಿದ ಪ್ರೋತ್ಸಾಹವನ್ನು ಎಂದೂ ಮರೆಯಲಾರೆ.

ಈ ನೂತನ ಮುದ್ರಣದಲ್ಲಿ ಕೆಲವೊಂದು ಹೊಸ ವಿಚಾರಗಳನ್ನು ಸೇರಿಸಲಾಗಿದೆ. ಸಣ್ಣ ಪುಟ್ಟ ಅಚ್ಚಿನ ದೋಷಗಳನ್ನೂ ತಿದ್ದಲಾಗಿದೆ. ೧೯೮೬ರಲ್ಲಿ ಸಾಲಿಗ್ರಾಮದ ವಿವೇಕ ಪ್ರಕಾಶನದ ಮೂಲಕ ಪ್ರಕಟವಾದ ಈ ಗ್ರಂಥವನ್ನು ಪೂಜ್ಯ ಸ್ವಾಮಿ ಹರ್ಷಾನಂದಜೀ ೧೯೯೦ರಲ್ಲಿ ಮೊದಲ ಬಾರಿಗೆ ಬೆಂಗಳೂರು ಶ‍್ರೀರಾಮಕೃಷ್ಣ ಆಶ್ರಮದಿಂದ ಪ್ರಕಟಿಸಿದರು. ಅಲ್ಲಿ ಎರಡು ಮುದ್ರಣಗಳನ್ನು ಕಂಡ ಈ ಗ್ರಂಥ ಇದೀಗ ಕರ್ನಾಟಕದಲ್ಲಿ ರಾಮಕೃಷ್ಣ ಸಂಘದ ಪ್ರಮುಖ ಪ್ರಕಟನ ಕೇಂದ್ರವಾದ ಮೈಸೂರಿನ ಶ‍್ರೀರಾಮಕೃಷ್ಣ ಆಶ್ರಮದಿಂದ ಪ್ರಕಟವಾಗುತ್ತಿದೆ.

\newpage

ನಾಡಿನ ಜನರು ಪರಸ್ಪರ ಪ್ರೀತಿ ವಿಶ್ವಾಸಗಳಿಂದ ಬಾಳುವಂತಾಗಲಿ, ಶುಭದಿನಗಳನ್ನು ಕಾಣು\-ವಂತಾಗಲಿ ಎಂದು ಭಗವಂತನನ್ನು ಪ್ರಾರ್ಥಿಸುತ್ತ ವಿರಮಿಸುತ್ತೇನೆ.

\bigskip

\noindent ಶ‍್ರೀರಾಮಕೃಷ್ಣ ಶಾರದಾಶ್ರಮ\hfill \textbf{ಸ್ವಾಮಿ ಜಗದಾತ್ಮಾನಂದ}

\noindent ಪೊನ್ನಂಪೇಟೆ, ಕೊಡಗು ಜಿಲ್ಲೆ ೫೭೨೧೬

