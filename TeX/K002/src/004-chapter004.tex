
\chapter{ಪ್ರೀತಿಯ ಪ್ರಚಂಡ ಪ್ರಭಾವ}

\indentsecionsintoc

\begin{itemize}
\itemsep=5pt
\item ಜಗತ್ತೇ ಪರಿಶುದ್ಧ ಪ್ರೀತಿಗಾಗಿ ಆತುರದಿಂದ ಹಾತೊರೆಯುತ್ತಿದೆ. ಪ್ರತಿಫಲವನ್ನಪೇಕ್ಷಿಸದೇ ನಾವದನ್ನು ಹಂಚಬೇಕು. ಪ್ರೀತಿಗೆ ಎಂದಿಗೂ ಅಪಜಯವಿಲ್ಲ. ಇಂದೋ, ನಾಳೆಯೋ, ಶತಮಾನಗಳ ನಂತರವೋ ಸತ್ಯಮಾತ್ರವೇ ಜಯಿಸುವುದು. ಪ್ರೀತಿಯು ಜಯವನ್ನು ತಂದೇ ತರುವುದು. ನೀವು ನಿಮ್ಮ ಸಹೋದರರನ್ನು ಪ್ರೀತಿಸುವಿರೇನು? ಪ್ರೀತಿಯ ಸರ್ವಶಕ್ತತೆಯಲ್ಲಿ ನಂಬಿಕೆ ಇಡಿ. ನಿಮ್ಮಲ್ಲಿ ಪರಿಶುದ್ಧ ಪ್ರೀತಿ ಇದೆಯೆಂದಾದರೆ ನೀವೇ ಸರ್ವಶಕ್ತರು.\hfill –ಸ್ವಾಮಿ ವಿವೇಕಾನಂದ

 \item ಸದೃಶವಾದ ಯೋಚನೆಗಳು ತತ್​ಸದೃಶ ಯೋಚನೆಗಳನ್ನೇ ಆಕರ್ಷಿಸಿ, ಅದ್ಭುತ ಪ್ರಭಾವ ಬೀರುತ್ತವೆ. ಅಂತೆಯೇ ಪ್ರೀತಿ ಕೂಡ.\hfill–ಸುಭಾಷಿತ

 \item ಯಥಾರ್ಥ ಪ್ರೀತಿಯು ಪ್ರಚಂಡ ರಚನಾತ್ಮಕ ಚಟುವಟಿಕೆಯಾಗಿ ಪ್ರಕಟವಾಗುತ್ತದೆ, ಪ್ರೀತಿಯು (ಪ್ರೀತಿಸಲ್ಪಡುವ ವ್ಯಕ್ತಿಯ ಬಗ್ಗೆ) ಗಮನ, ಗೌರವ, ಜವಾಬ್ದಾರಿ ಮತ್ತು ಜ್ಞಾನಗಳ ಮೂಲಕ ವ್ಯಕ್ತವಾಗುತ್ತದೆ. ಒಬ್ಬ ವ್ಯಕ್ತಿಯನ್ನು ಪ್ರೀತಿಸುವುದೆಂದರೆ ಅವನ ಬೆಳವಣಿಗೆಯ ಬಗ್ಗೆ, ಸರ್ವತೋಮುಖ ಪ್ರಗತಿಯ ಬಗ್ಗೆ ತಾನು ಜವಾಬ್ದಾರನೆಂಬ ಭಾವನೆಯನ್ನು ತಳೆದು ಆತನನ್ನು ಗಮನವಿತ್ತು ನೋಡಿಕೊಳ್ಳುವುದು ಎಂದು ಸೂಚಿತವಾಗುವುದು.\hfill\hbox\bgroup–ಎರಿಕ್ ಫ್ರೋಮ್​\egroup

 \item \enginline{Like attracts like, begets like, like becomes like, So also love.}

 \item \enginline{As soon as a person can truly learn to love the disagreeable or restrictive person with whom he is involved, he will be released from the bondage.\general{\hfill}–Gina Cerminara}

 \item \enginline{The Genuine love is an expression of productiveness and implies care, respect, responsibility and knowledge. It is an active striving for the growth and happiness of the loved persons, rooted in one’s own capacity to love.\general{\hfill}–Eric Fromm}

 \item \enginline{Love never comes untill there is freedom. There is no true love possible in the slave... You may read all books in the Universe but this love is not to be had by the power of speech, not by the highest intellect, not by the study of various sciences. He who desires God will get Love, unto him God gives Himself. Thus love is mutual and reflective and without it we cannot get bliss.\general{\hfill}–Swami Vivekananda}

\end{itemize}


\section*{ಪ್ರೀತಿ ಮಾಡುವ ಮೋಡಿ}

\addsectiontoTOC{ಪ್ರೀತಿ ಮಾಡುವ ಮೋಡಿ}

ಪ್ರೀತಿ ಇದ್ದಲ್ಲಿ ಭೀತಿ ಇರದು! ‘ಪರಿಶುದ್ಧ ಪರಿಪೂರ್ಣ ಪ್ರೀತಿಯು ಭೀತಿಯನ್ನು ದೂರಕ್ಕೆಸೆಯುವುದು’ ಎಂಬುದು ಕ್ರಿಸ್ತವಾಣಿ.

\vskip 1.5pt

ಪರಿಶುದ್ಧ ನಿಃಸ್ವಾರ್ಥ ಪ್ರೀತಿ ಒಂದು ದೈವೀಶಕ್ತಿ. ಶ್ರದ್ಧೆ, ಭರವಸೆ ಮತ್ತು ಪ್ರೀತಿ ಇವುಗಳಲ್ಲಿ ಪ್ರೀತಿಯೇ ಅತ್ಯಂತ ಮಹಿಮಾನ್ವಿತವಾದುದೆಂದು ಕ್ರಿಸ್ತನ ಶಿಷ್ಯ ಸಂತ ಪಾಲ ಹೇಳಿದ.

\vskip 1.5pt

ಮನುಷ್ಯಕುಲವನ್ನೇ ಮೇಲಕ್ಕೆತ್ತಲು ಬೇಕಾಗುವ ಸ್ಫೂರ್ತಿಯನ್ನೂ, ಕ್ರಿಯಾಶಕ್ತಿಯನ್ನೂ ಈ ಪ್ರೀತಿ ನೀಡಬಲ್ಲದು.

\vskip 1.5pt

ಆಯುರಾರೋಗ್ಯ ಕಾಂತಿ ಶಾಂತಿಗಳನ್ನು ನೀಡಬಲ್ಲ ಶಕ್ತಿ ಈ ಪ್ರೀತಿಯಲ್ಲಿದೆ. ಎಂಥ ನರಕದ ಆಳದಿಂದಲೂ ವ್ಯಕ್ತಿಯನ್ನು ಎತ್ತಿ ಉದ್ಧರಿಸುವ ಶಕ್ತಿ ಈ ಪರಿಶುದ್ಧ ಪ್ರೀತಿಗಿದೆ ಎಂದರೆ, ಪರಿಶುದ್ಧ ಪ್ರೀತಿಯ ಸಂಸ್ಪರ್ಶದಿಂದ ವ್ಯಕ್ತಿ ಹಾಗೂ ಸಮುದಾಯದ ಸ್ವಭಾವ ಪರಿವರ್ತನೆ ಮಾತ್ರವಲ್ಲ, ಸರ್ವತೋಮುಖ ಅಭ್ಯುದಯ ಸಾಧ್ಯವಾಗುತ್ತದೆ.

\vskip 1.5pt

ಮನುಷ್ಯರು ಈ ಪ್ರೀತಿಯ ಸಂಸ್ಪರ್ಶದ ಅಭಾವದಿಂದ ಕೊರಗಿ, ಸೊರಗಿ ಕಂಗೆಟ್ಟು\break ಹೋಗುತ್ತಾರೆ.

\vskip 1.5pt

ಶೈಶವ, ಬಾಲ್ಯಗಳಲ್ಲಿ ತಂದೆತಾಯಂದಿರ ಪರಿಶುದ್ಧ ಪ್ರೀತಿಯನ್ನು ಕಂಡರಿಯದ ಮಕ್ಕಳು ದುಷ್ಟರೂ, ಭ್ರಷ್ಟರೂ, ಕ್ರೂರಿಗಳೂ ಆಗುತ್ತಾರೆಂದು ತಜ್ಞರು ಹೇಳುತ್ತಾರೆ.

\vskip 1.5pt

‘ಶೈಶವದಿಂದಲೇ ಮಕ್ಕಳನ್ನು ಭಯಗ್ರಸ್ತ ವಾತಾವರಣದಲ್ಲಿಡಿ. ಸೇಡಿನ ಮನೋಭಾವ, ಹಿಂಸಾಪ್ರವೃತ್ತಿ, ದುಷ್ಟತನ, ಕ್ರೂರತೆ, ಹೃದಯಹೀನತೆ–ಇವೇ ಮೊದಲಾದ ಗುಣಗಳು ಅವರಲ್ಲಿ ಹುಲುಸಾಗಿ ಬೆಳೆಯುತ್ತವೆ. ಮಕ್ಕಳನ್ನು ಶೈಶವದಿಂದಲೇ ಭಯಗ್ರಸ್ತ ವಾತಾವರಣದಲ್ಲಿಡುವುದೆ ಅವರನ್ನು ಕೊಲೆಗಡುಕರನ್ನಾಗಿ ಮಾಡುವ ನಿಶ್ಚಿತಪಥ’ ಎಂದು ತಜ್ಞನೊಬ್ಬನು ಹೇಳುತ್ತಾನೆ.

\vskip 1.5pt

ಮಕ್ಕಳಿಗೆ ಪರಿಶುದ್ಧ ಪ್ರೀತಿಯನ್ನು ತೋರುವುದೆಂದರೆ ಮಕ್ಕಳನ್ನು ಮನಬಂದಂತೆ ಬಿಡುವ ಸಲುಗೆ ಎಂದರ್ಥವಲ್ಲ. ಚಿಕ್ಕವರನ್ನು ತಿದ್ದುವ ವೇಳೆ ಅವರು ತಪ್ಪು ಮಾಡಿದಾಗ ಗದರಿಕೆ ಮತ್ತು ಶಿಕ್ಷೆ, ಒಳಿತನ್ನು ಸಾಧಿಸಿದಾಗ ಪ್ರಶಂಸೆ ಹಾಗೂ ಬಹುಮಾನ ನೀಡುವ ವಿಧಾನ ಸರ್ವಸಮ್ಮತ ಪದ್ಧತಿ ಎಂಬುದು ನಿರ್ವಿವಾದ. ಶಿಕ್ಷೆ ಗದರಿಕೆಗಳು ತಮ್ಮ ಒಳಿತಿಗಾಗಿ ಎಂಬುದು ಮನವರಿಕೆ\-ಯಾದಾಗ, ಶಿಕ್ಷೆಯನ್ನು ಅಪರಾಧಕ್ಕನುಗುಣವಾಗಿ ನೀಡಿದಾಗ ಮಕ್ಕಳು ಸಾಮಾನ್ಯವಾಗಿ ತಪ್ಪು ತಿಳಿದು\-ಕೊಳ್ಳುವುದಿಲ್ಲ. ತಪ್ಪು ತಿಳಿದುಕೊಂಡರೂ ಶಿಕ್ಷೆ ಅನಿವಾರ್ಯ. ಮದ್ದು ಮಾಡಬೇಕಾದಾಗ ಮುದ್ದು ಮಾಡಿದರೇನು ಪ್ರಯೋಜನ?

\smallskip


\section*{ಸೋತು ಗೆದ್ದವಳು}

\vskip 2pt\addsectiontoTOC{ಸೋತು ಗೆದ್ದವಳು}

ಸುಶಿಕ್ಷಿತ ದಂಪತಿಗಳಿದ್ದರು. ಇಬ್ಬರೂ ಒಂದೇ ಕಾಲೇಜಿನಲ್ಲಿ ಪ್ರಾಧ್ಯಾಪಕರಾಗಿದ್ದರು. ಇಬ್ಬರಲ್ಲೂ ಯಾವಾಗಲೂ ಮತಭೇದ ಪ್ರಕಟವಾಗುತ್ತಿತ್ತು. ವಾದವಾಗುತ್ತ ಒಂದೊಂದು ಸಾರಿ ವಿಕೋಪಕ್ಕೆ ಹೋಗುತ್ತಿತ್ತು. ಕೆಲವು ದಿವಸಗಳ ನಂತರ ಗಂಡನಿಗೆ ಬೇರೆ ಊರಿಗೆ ವರ್ಗವಾಯಿತು. ಇಬ್ಬರೂ ಆ ಊರಿಗೆ ಹೋಗಿ ನೆಲಸಿದರು. ಕೆಲವು ದಿನಗಳು ಸುಮಾರಾಗಿ ಕಳೆದವು. ಒಂದು ದಿನ ಜಗಳ ಪ್ರಾರಂಭವಾಯಿತು. ಗಂಡನ ಸಿಟ್ಟಿನ ಎಲ್ಲೆ ಮೀರಿತು. ಹೆಂಡತಿಯ ಕಪಾಲಕ್ಕೆ ಆತ ಹೊಡೆದೇಬಿಟ್ಟ. ಹೆಂಡತಿಗೆ ಅದನ್ನು ಸಹಿಸಲಾಗಲಿಲ್ಲ. ಅದೇ ರಾತ್ರಿ ಆಕೆಯು ತೌರಿಗೆ ಹೊರಟು ಹೋದಳು.

ಹೀಗೆಯೇ ಕೆಲವು ದಿನಗಳು ಕಳೆದವು. ಗಂಡನಿಗೆ ಟೈಫಾಯಿಡ್ ಜ್ವರ ಬರುತ್ತಿದೆ ಎಂಬ ಸಮಾಚಾರ ಆಕೆಗೆ ತಿಳಿಯಿತು. ಆಗ ‘ಏನು ಮಾಡುವುದು?’ ಎಂಬ ನಿರ್ಣಯಕ್ಕೆ ಬರಲು ಆಕೆಗೆ ಆಗಲಿಲ್ಲ. ಆಗ ಶ‍್ರೀಬ್ರಹ್ಮಚೈತನ್ಯ ಮಹಾರಾಜರನ್ನು ಭೇಟಿಯಾದಳು. ಆಗ ಮಹಾರಾಜರು ಹೀಗೆಂದರು: ‘ಎರಡು ಚಕ್ರದ ವಾಹನವಿದೆ ಎಂದು ತಿಳಿಯಿರಿ. ಅದರಲ್ಲಿ ಒಂದು ಚಕ್ರ ಒಂದು ಕಡೆಗೆ, ಇನ್ನೊಂದು ಚಕ್ರ ಇನ್ನೊಂದು ಕಡೆ ಎಳೆದರೆ ಆ ಗಾಡಿಗೆ ದುರ್ದೆಶೆಯುಂಟಾಗುತ್ತದೆ. ಸ್ತ್ರೀ ಮತ್ತು ಪುರುಷ–ಇವರಲ್ಲಿ ಭೇದವಿರುವುದು ಸ್ವಾಭಾವಿಕ. ಆದರೆ ಪ್ರಪಂಚದ ಆಟವನ್ನು ಸರಿಯಾಗಿ ಆಡಲು ಈ ಭೇದದ ಆವಶ್ಯಕತೆ ಇದೆ. ಪುರುಷರಲ್ಲಿ ಶೌರ್ಯ ಮತ್ತು ಔದಾರ್ಯ ಮೂರ್ತಿವೆತ್ತಂತಿದ್ದರೆ ಸ್ತ್ರೀಯರಲ್ಲಿ ಸಹನಶೀಲತೆ ಮತ್ತು ಅನನ್ಯ ಶರಣಾಗತಿಗಳು ಮೈದಳೆದಿರುತ್ತವೆ. ಸಂಸಾರದಲ್ಲಿ ಸೋಲವುದು ಎಂಬುದು ಸ್ತ್ರೀಯ ಸಹನಶೀಲತೆಗೆ ಪರೀಕ್ಷೆಯಾಗಿರುತ್ತದೆ.\break ದೇವರು ನಿರ್ಮಾಣ ಮಾಡಿರುವ ಈ ಜಗತ್ತು ಒಂದು ದೊಡ್ಡ ನಾಟಕವೇ ಸರಿ. ಅದರಲ್ಲಿ ಯಾರ ಭಾಗಕ್ಕೆ ಯಾವ ಪಾತ್ರ ಬರುತ್ತದೋ, ಅದನ್ನು ಅವರು ಉತ್ತಮ ರೀತಿಯಲ್ಲಿ ನಡೆಸುವುದೇ ಅವರ ಕರ್ತವ್ಯವಾಗಿರುತ್ತದೆ. ಯಾರಿಗಾದರೂ ಕೀಳುತರದ ಕೆಲಸ ಮಾಡಬೇಕಾಗಿ ಬಂದರೂ, ಅದನ್ನೂ ಅವರು ಉತ್ತಮವಾಗಿ ಮಾಡಬೇಕು. ಆಗ ದೊಡ್ಡ ಕೆಲಸ ಮಾಡಿದಷ್ಟೇ ಇದಕ್ಕೂ ಶ್ರೇಯಸ್ಸು ದೊರೆಯುತ್ತದೆ. ಪುತ್ರರತ್ನ ಹೊಂದಿ, ಮಾತೃಸ್ಥಾನದ ಮರ್ಯಾದೆ ಸಿಕ್ಕುವ ಅಧಿಕಾರ ನಿಮ್ಮದೇ ಆಗಿದೆ. ಅದು ಯಜಮಾನರಿಗೆ ಎಂದಿಗೂ ದೊರೆಯುವುದಿಲ್ಲ. ಆದ್ದರಿಂದ ನಿಮ್ಮ “ಹಕ್ಕು” ಎಂಬ ವಿಷಯವನ್ನು ಮನಸ್ಸಿನಿಂದ ತೆಗೆದುಹಾಕಿಬಿಡಿ. ಇಬ್ಬರೂ ಒಬ್ಬರನ್ನೊಬ್ಬರು ಪ್ರೀತಿಯಿಂದ ನೋಡಿಕೊಂಡು ಸುಖವಾಗಿರುವುದರಲ್ಲೇ ಸಮಾಧಾನವಿದೆ.’ ಆಗ ಆಕೆಯು ‘ನಾನು ಏನು ಮಾಡಲಿ?’ ಎಂದಾಗ ಮಹಾರಾಜರು ‘ಈಗ ಹೊರಡುವ ಮೊದಲನೇ ಗಾಡಿಯಲ್ಲಿ ನಿಮ್ಮ ಯಜಮಾನರು ಇರುವಲ್ಲಿಗೆ ಹೋಗಿ. ನಿಮ್ಮಿಬ್ಬರ ಮಧ್ಯೆ ಹಿಂದೆ ಏನೂ ನಡೆದೇ ಇಲ್ಲವೆಂಬ ಭಾವನೆಯಿಂದ ಅವರ ಶುಶ್ರೂಷೆಯನ್ನು ಮಾಡಿ. ಪ್ರತಿದಿನವೂ ಬೆಳಿಗ್ಗೆ ಅವರಿಗೆ ನಮಸ್ಕಾರ ಮಾಡಿ. ಎಲ್ಲಾ ಸರಿ ಹೋಗುತ್ತದೆ’ ಎಂದರು.

ಅಂತೆಯೇ ಆಕೆಯು ಪತಿಯ ಊರಿಗೆ ತೆರಳಿ, ಮಹಾರಾಜರು ಹೇಳಿದಂತೆ ನಡೆದುಕೊಳ್ಳಲು ಪ್ರಾರಂಭಿಸಿದಳು. ಕೆಲವು ದಿನಗಳಲ್ಲಿ ಪತಿಯ ಜ್ವರ ಇಳಿಯಿತು. ಒಂದು ದಿನ ಬೆಳಿಗ್ಗೆ ಎದ್ದಾಗ ಈಕೆಯು ಪತಿಗೆ ನಮಸ್ಕಾರ ಮಾಡುತ್ತಿದ್ದಳು. ಆಗ ಆತನಿಗೆ ಅಳುವೇ ಬಂದು ಬಿಟ್ಟಿತು. ಆತನೆಂದ ‘ನನ್ನ ದೇಹಸ್ಥಿತಿ ಸರಿಯಾಗುತ್ತಿದೆ, ಆದರೆ ಪಶ್ಚಾತ್ತಾಪ ನನಗೆ ಬಹಳವಾಗಿ ಆಗಿದೆ. ನನ್ನ ಈ ತರದ ನಡತೆಯಿಂದ ಬೇಸರಗೊಳ್ಳದೆ, ನನ್ನನ್ನು ಬಿಡದೆ, ನನಗಾಗಿ ನೀನು ಓಡಿ ಬಂದೆ. ನಿನಗೆ ಯಾರೋ ಮಹಾತ್ಮರ ದರ್ಶನ ಸಿಕ್ಕಿರಬೇಕು. ಇದರಲ್ಲಿ ಸಂದೇಹವಿಲ್ಲ. ಇನ್ನು ಮುಂದೆ ನನ್ನಿಂದ ತಪ್ಪಾಗದಂತೆ ನಡೆದುಕೊಳ್ಳುವೆನೆಂದು ನಾನು ವಚನ ಕೊಡುತ್ತೇನೆ.’ ಕೆಲವು ಕಾಲದ ನಂತರ, ಆ ದಂಪತಿಗಳಿಗೆ ಒಂದು ಗಂಡು ಮಗು ಜನಿಸಿತು. ಆ ಮಗುವಿನೊಂದಿಗೆ ಆ ದಂಪತಿಗಳು ಶ‍್ರೀ ಮಹಾರಾಜರ ದರ್ಶನ ಪಡೆದು, ಬಾಳಿನ ಗೆಲುವಿನ ದಾರಿ ತೋರಿಸಿದುದಕ್ಕೆ ಕೃತಜ್ಞತೆ ಅರ್ಪಿಸಿದರು.

~\\[-2\baselineskip]


\section*{ಒಳಗುಟ್ಟೇನು?}

\vskip -2pt\addsectiontoTOC{ಒಳಗುಟ್ಟೇನು?}

ಸಮಾಜಶಾಸ್ತ್ರದ ಪ್ರಾಧ್ಯಾಪಕರೊಬ್ಬರು ತಮ್ಮ ವಿದ್ಯಾರ್ಥಿಗಳಿಗೆ ಕೊಳಚೆ ಪ್ರದೇಶದ ಎಳೆಯರನ್ನು ಕುರಿತು ಅಧ್ಯಯನ ನಡೆಸಲು ಆದೇಶ ನೀಡಿದ್ದರು. ಇದು ನಡೆದುದು ಅಮೇರಿಕದ ಬಾಲ್ಟಿಮೋರ್ ಎಂಬ ಪಟ್ಟಣದಲ್ಲಿ. ವಿದ್ಯಾರ್ಥಿಗಳು ಆ ಪ್ರದೇಶಕ್ಕೆ ಹೋಗಿ, ಸುಮಾರು ಇನ್ನೂರು ಮಂದಿ ಎಳೆಯರನ್ನು ಭೇಟಿ ಮಾಡಿ ಅವರ ಬಗೆಗೆ ಅನೇಕ ಮಾಹಿತಿಗಳನ್ನು ಕಲೆ ಹಾಕಿದರು. ಆ ಮಕ್ಕಳು ವಾಸಿಸುತ್ತಿದ್ದ ವಾತಾವರಣ ಭಯಾನಕವೆನಿಸುವಷ್ಟು ಅಸಹ್ಯವಾಗಿತ್ತು. ಅವರು ಸುಯೋಗ್ಯ ಪ್ರಜೆಗಳಾಗಿ ಬದುಕಲು ಬೇಕಾಗುವ ಯಾವ ನಡತೆಯನ್ನೂ ಅಲ್ಲಿ ಕಲಿಯುವಂತಿರಲಿಲ್ಲ. ಆ ಮಕ್ಕಳಲ್ಲಿ ಶೇಕಡಾ ತೊಂಬತ್ತು ಮಂದಿಯಾದರೂ ದುಷ್ಟರಾಗಿ ಬೆಳೆದು ಜೈಲುವಾಸ ಮಾಡುತ್ತಾರೆಂದು ಅಧ್ಯಯನ ತಂಡದ ವಿದ್ಯಾರ್ಥಿಗಳು ಭವಿಷ್ಯ ನುಡಿದಿದ್ದರು.

ಇಪ್ಪತ್ತೈದು ವರ್ಷಗಳ ಬಳಿಕ ಆ ಪ್ರಾಧ್ಯಾಪಕರೇ ಇನ್ನೊಂದು ಅಧ್ಯಯನ ತಂಡವನ್ನು ಅದೇ ಜಾಗಕ್ಕೆ ಕಳುಹಿಸಿ, ಹಿಂದೆ ತನ್ನ ಹಳೆಯ ವಿದ್ಯಾರ್ಥಿಗಳು ನುಡಿದ ಭವಿಷ್ಯವಾಣಿ ಸತ್ಯವಾಯಿತೇ? ಎಂಬುದನ್ನು ತಿಳಿಯಲು ಕುತೂಹಲಿಗಳಾಗಿದ್ದರು. ಈ ಬಾರಿ ವಿದ್ಯಾರ್ಥಿಗಳು ಹಿಂದೆ ಭೇಟಿ ಮಾಡಿದ್ದ ಇನ್ನೂರಲ್ಲಿ ನೂರತೊಂಬತ್ತು ಮಂದಿಯನ್ನು ಕಂಡು ಮಾತನಾಡಿ, ಅವರ ಸ್ಥಿತಿಗಳ ಬಗ್ಗೆ ಪರಿಚಯ ಮಾಡಿಕೊಳ್ಳಲು ಸಮರ್ಥರಾದರು. ಹಿಂದಿನ ಅಧ್ಯಯನ ತಂಡದ ಭವಿಷ್ಯವಾಣಿ ಸುಳ್ಳಾಗಿತ್ತು. ಅವರೆಲ್ಲರೂ ಸಭ್ಯರಾಗಿದ್ದರು. ಅವರ ಪರಿವರ್ತನೆಯ ಒಳಗುಟ್ಟೇನು?


\section*{ಪರಿವರ್ತನೆಯ ಹರಿಕಾರ}

\vskip -2pt\addsectiontoTOC{ಪರಿವರ್ತ\-ನೆಯ ಹರಿಕಾರ}

{\parfillskip=0pt ವಿದ್ಯಾರ್ಥಿಗಳ ಭವಿಷ್ಯವಾಣಿ ಈ ರೀತಿ ಸುಳ್ಳಾಗಲು ಕಾರಣ ಏನೆಂಬುದನ್ನು ತಿಳಿಯಲು ಹೊಸ\par}\newpage\noindent ಅಧ್ಯಯನ ತಂಡ ಕುತೂಹಲಿಯಾಯಿತು. ಆ ಕೊಳಚೆ ಪ್ರದೇಶದ ಶಾಲೆಯಲ್ಲಿ ಕಲಿತ ವಿದ್ಯಾರ್ಥಿಗಳೆಲ್ಲ ಅಧ್ಯಾಪಕಿ ಕುಮಾರಿ ಶೀಲಾ ರೂರ್​ಕೆ ಎಂಬವರಿಂದ ಬಹಳಷ್ಟು ಪ್ರಭಾವಿತರೆಂಬುದು ತಿಳಿಯಿತು. ಶೀಲಾ ರೂರ್​ಕೆ ನಿವೃತ್ತಳಾಗಿದ್ದುದರಿಂದ ಅವರನ್ನು ಹುಡುಕುವುದು ಶ್ರಮಸಾಧ್ಯವಾದ ಕೆಲಸವಾಗಿದ್ದರೂ, ಅಧ್ಯಯನ ತಂಡ ಕೊನೆಗೂ ಅವರನ್ನು ಭೇಟಿ ಮಾಡಿಯೇಬಿಟ್ಟಿತು. ‘ಕೊಳಚೆ ಪ್ರದೇಶದ ಮಕ್ಕಳಲ್ಲಿ ಹೇಗೆ ನೀವು ಸ್ಫೂರ್ತಿಯನ್ನು ತುಂಬಿದಿರಿ? ಅವರ ಬದುಕು ಉನ್ನತಿಯ ಪಥದಲ್ಲಿ ಮುನ್ನಡೆಯುವಂತೆ ಏನು ಮಾಡಿದಿರಿ? ಅವರನ್ನು ಸಭ್ಯರಾದ ನಾಗರಿಕರನ್ನಾಗಿಸಲು ಯಾವ ಉಪಾಯ ಹೂಡಿದಿರಿ?’ ಎಂದು ಅಧ್ಯಯನ ತಂಡ ಪ್ರಶ್ನಿಸಿದಾಗ ಅವರೆಂದರು ‘ಅಂಥ ಉಪಾಯವಾವುದೂ ನನಗೆ ತಿಳಿದಿಲ್ಲ. ನಾನು ಶಾಲೆಯಲ್ಲಿ ಅಧ್ಯಾಪಕಿಯಾಗಿದ್ದಷ್ಟು ಕಾಲ ಪ್ರತಿಯೊಬ್ಬ ವಿದ್ಯಾರ್ಥಿಯನ್ನೂ ಪ್ರೀತ್ಯಾದರಗಳಿಂದ ನೋಡಿಕೊಂಡಿದ್ದೆ’ ಎಂದರು.

ಇದೇನು ಮಹಾ? ಅತ್ಯಂತ ಸಾಮಾನ್ಯ ವಿಚಾರ ಎನ್ನಿಸಬಹುದು ನಿಮಗೆ. ಹಾಗಿದ್ದರೆ ನೀವು ಪ್ರೀತಿಯ ಬಗೆಗೆ ಸರಿಯಾಗಿ ತಿಳಿದುಕೊಂಡಿಲ್ಲ ಎಂದೇ ಅರ್ಥ. ತಿಳಿದುಕೊಂಡಿದ್ದರೂ ತೀರ ಪರಿಮಿತಿಯ ಅಗ್ಗದ ಭಾವನಾತ್ಮಕ ಅಭಿವ್ಯಕ್ತಿಯನ್ನೇ ಪ್ರೀತಿ ಎಂದುಕೊಂಡಿರಬೇಕು.

ಪ್ರೀತಿ ಪರಿವರ್ತನೆಯ ಪ್ರವರ್ತಕವಾಗಬಲ್ಲದು. ಪ್ರೀತಿಯನ್ನು ಪಡೆದವನೇ ಪ್ರೀತಿಯ ಮಹಿಮೆಯನ್ನು ತಿಳಿಯಬಲ್ಲ. ಆತನೇ ಪ್ರೀತಿಯನ್ನು ಇತರರಿಗೂ ನೀಡಬಲ್ಲ.

ಪ್ರೀತಿಸುವವನಿಂದ ಪ್ರೀತಿಸಲ್ಪಡುವ ವ್ಯಕ್ತಿ ಆತ್ಮಶ್ರದ್ಧೆಯ ಅನರ್ಘ್ಯರತ್ನವನ್ನು ಪಡೆಯುತ್ತಾನೆ. ತನ್ನತನವನ್ನು ಉಳಿಸಿಕೊಳ್ಳುತ್ತಾನೆ. ವ್ಯಕ್ತಿತ್ವವನ್ನು ಬೆಳೆಸಿಕೊಳ್ಳುತ್ತಾನೆ.

ಹಿರಿಯರ ಪರಿಶುದ್ಧ ಪ್ರೀತಿ, ಉನ್ನತಮಟ್ಟದ ನೈತಿಕ ನಿರೀಕ್ಷೆ, ಕಿರಿಯರ ಆಂತರ್ಯದಲ್ಲಿನ ಸುಪ್ತಶಕ್ತಿಯನ್ನು ಸರಿಯಾದ ಪಥದಲ್ಲಿ ಹರಿಯುವಂತೆ ಮಾಡುತ್ತದೆ.

ಪ್ರೀತಿಯನ್ನು ಸರಿಯಾದ ರೀತಿಯಲ್ಲಿ ಪಡೆದು ಪುಷ್ಟನಾದವನೇ ಇತರರಿಗೂ ಪ್ರೀತಿಯನ್ನು ಹಂಚಬಲ್ಲ. ಶಿಶುವಾಗಿರುವಾಗ ನಾವೇ ಮಾತನಾಡಲಾರೆವು. ‘ಮಗು ಮಾತನಾಡುವುದಿಲ್ಲ, ನಾನೇಕೆ ಮಾತನಾಡಬೇಕು?’ ಎಂದು ತಾಯಿ ಹೇಳುವುದಿಲ್ಲ. ಮುದ್ದಿನಿಂದ ಎಷ್ಟೊಂದು ಮಾತಾಡಿ\-ಸುತ್ತಾಳೆ! ಮಮತೆಯ ಮಳೆಗರೆಯುತ್ತ ನಕ್ಕುನಗಿಸುತ್ತಾಳೆ. ಹೌದು, ಪ್ರೀತಿಯ ಭಾಷೆ ನಗು. ಹಾಗೆ ಮಾತನಾಡಿಸಿದ್ದರಿಂದಲೇ ಮಗು ಮಾತು ಕಲಿಯುತ್ತದೆ. ಮತ್ತೆ ಮಾತನಾಡಲೂ ಸಮರ್ಥ\-ವಾಗುತ್ತದೆ.

ಮಕ್ಕಳಾಗಿರುವಾಗ ಪ್ರೀತಿಯನ್ನು ಪಡೆಯುವ ಸ್ಥಿತಿಯಲ್ಲಿ ಇರುತ್ತೇವೆಯೆ ಹೊರತು ಪ್ರಜ್ಞಾ\-ಪೂರ್ವಕ ಪ್ರೀತಿಯನ್ನು ನೀಡುವುದರಲ್ಲಲ್ಲ.

ಇತರರು, ಎಂದರೆ ಮನೆಯಲ್ಲಿ ತಾಯಿತಂದೆ ಮತ್ತು ಹಿರಿಯರು, ಶಾಲೆಯಲ್ಲಿ ಅಧ್ಯಾಪಕರು ಮತ್ತು ಸಹವಿದ್ಯಾರ್ಥಿಗಳು–ಇವರು ನಮ್ಮಲ್ಲಿಟ್ಟ ಶ್ರದ್ಧೆ ಪ್ರೋತ್ಸಾಹಗಳಿಂದ ನಾವು ಬೆಳೆಯ ಬೇಕಷ್ಟೆ.

ನಮ್ಮ ವೈಶಿಷ್ಟ್ಯಗಳನ್ನು ಗುರುತಿಸಿ, ಗುಣಗಳನ್ನು ಗ್ರಹಿಸಿ, ತಪ್ಪುಗಳನ್ನು ಕ್ಷಮಿಸಿ, ತಿದ್ದಿ ನಮ್ಮ\-ಲ್ಲಿರುವ ಒಳಿತನ್ನು ಎತ್ತಿ ಹೇಳಿ, ನಮ್ಮ ವ್ಯಕ್ತಿತ್ವ ತನ್ನದೇ ಆದ ವೈಶಿಷ್ಟ್ಯದಿಂದ ಬೆಳೆಯುವಂತೆ ಮಾಡಲು ನಿಃಸ್ವಾರ್ಥ ಪ್ರೇಮದ ಆಸರೆ ಬೇಕು.

ಈ ನಿಃಸ್ವಾರ್ಥ ಪ್ರೀತಿಯನ್ನು ಯಾರು ನೀಡಬಲ್ಲರು?

‘ಜನರೇನೋ ಪ್ರೀತಿಸುವುದು ಅತ್ಯಂತ ಸುಲಭಕಾರ್ಯವೆಂದು ನಂಬಿದ್ದಾರೆ. ಪ್ರತಿಯೊಬ್ಬ ವ್ಯಕ್ತಿಯಲ್ಲೂ ಪ್ರೀತಿಸುವ ಸಾಮರ್ಥ್ಯವಿದ್ದರೂ ನಿಜವಾಗಿ ಪ್ರೀತಿಸುವುದು ಒಂದು ಅಸಾಮಾನ್ಯ ಸಿದ್ಧಿಯೇ ಸರಿ’ ಎಂದು ಎರಿಕ್ ಫ್ರೋಂ ಹೇಳುತ್ತಾರೆ. ತನ್ನ ಹೆಂಡತಿಯಾಗುವವಳು ದಕ್ಷಳೂ, ಚುರುಕುಬುದ್ಧಿಯವಳೂ, ಸುಂದರಿಯೂ ಆಗಿರುವುದರಿಂದ ತಾನವಳನ್ನು ಪ್ರೀತಿಸುತ್ತಿದ್ದೇನೆ ಎಂದು ವ್ಯಕ್ತಿಯೊಬ್ಬ ತಿಳಿದುಕೊಂಡಿರಬಹುದು. ಆದರೆ ಅದನ್ನು ಪ್ರೀತಿ ಎನ್ನಲಾಗದು. ಅದು ಮೆಚ್ಚುಗೆ, ಒಪ್ಪಿಗೆ ಮಾತ್ರ. ಪ್ರೀತಿಸಲ್ಪಡುವ ವಸ್ತುವಿನ ಗುಣವೈಶಿಷ್ಟ್ಯಗಳನ್ನು ಹೊಂದಿಕೊಂಡು ಪ್ರೀತಿ ಸ್ಥಿರವಾಗಿರುವುದೆಂದಲ್ಲ. ಅದು ವ್ಯಕ್ತಿಯ ಪ್ರೀತಿಸುವ ಸಾಮರ್ಥ್ಯವನ್ನು ಹೊಂದಿಕೊಂಡು, ಸ್ಥಾಯಿಯೋ, ಅಸ್ಥಾಯಿಯೋ ಆಗುವುದು. ಅದೇನೋ ಕೆಲವರು ತಿಳಿದುಕೊಳ್ಳುವಂತೆ ಸ್ವಾಭಾವಿಕವಾಗಿ ಬಂದುಬಿಡುವುದಿಲ್ಲ. ‘ಪ್ರೀತಿಸುವ ಕಲೆಯನ್ನು ತಂದೆ, ತಾಯಿಗಳು ಮಕ್ಕಳಿಗೆ ಕಲಿಸುವುದು ಶ್ರೇಯಸ್ಕರ’ ಎನ್ನುತ್ತಾರೆ ವಿಲಿಯಂ ಸಿ. ಮೆನ್ನಿಂಗರ್.


\section*{ಪ್ರೀತಿಯ ರೀತಿ}

\addsectiontoTOC{ಪ್ರೀತಿಯ ರೀತಿ}

ಪ್ರೀತಿ ಎಂದರೆ ಕೇವಲ ಭಾವಪರವಶತೆ, ಗಂಡು ಹೆಣ್ಣುಗಳ ದೈಹಿಕ ಆಕರ್ಷಣೆ ಎಂದೇ ಹೆಚ್ಚಿನವರು ತಿಳಿದುಕೊಂಡಿದ್ದಾರೆ. ಕಾಮ ಮತ್ತು ಸೌಂದರ್ಯ ಭಾವನೆಗಳ ಮಿಶ್ರಣದಿಂದ ಬಿಡಿಸಿ, ಪ್ರೀತಿಯ ಸ್ವರೂಪವನ್ನು ತಿಳಿದುಕೊಳ್ಳಲು ನಾವು ಯತ್ನಿಸಬೇಕು. ಹಾಗೆ ತಿಳಿದುಕೊಂಡರೇನೇ ಪ್ರೀತಿಯ ವ್ಯಾಪಕಶಕ್ತಿಯನ್ನೂ, ಪ್ರಭಾವವನ್ನೂ ಅರಿತುಕೊಳ್ಳಬಹುದು.

‘ಪ್ರೀತಿ ಎಂದರೆ ನಾವು ಅರ್ಥೈಸುವ ಹಾಗೆ ಅನುರಾಗದ ಭಾವಪರವಶತೆಯಲ್ಲ. ಪ್ರೀತಿಪಾತ್ರ ವ್ಯಕ್ತಿಯನ್ನು ತನ್ನ ಅಧೀನದಲ್ಲಿರಿಸಿಕೊಳ್ಳುವ ಭಾವೋದ್ವೇಗವೂ ಅಲ್ಲ. ಪ್ರತಿಯೊಬ್ಬನಲ್ಲೂ ಹುದು\-ಗಿರುವ ವೈಶಿಷ್ಟ್ಯವನ್ನು ಸರಿಯಾಗಿ ಗುರುತಿಸಿ, ಅವನ ಅಭ್ಯುದಯಕ್ಕಾಗಿ, ನಿರಂತರ ಶುಭಸಂಕಲ್ಪದಿಂದ ದುಡಿಯುವ ಪ್ರವೃತ್ತಿ ಅದು’ ಎಂದು ಎಡ್ಲೆ ಸ್ಟೀವನ್​ಸನ್ ಹೇಳುತ್ತಾರೆ.\footnote{\engfoot{By love... I do not mean sentimentality or possessive emotion; but the steady recognition of others’ uniqueness, and a sustained intention to seek their good.}\hfill\engfoot{ –Adlai Stevenson}}

ಎರಿಕ್ ಫ್ರೋಂ ಕೂಡ ಇದೇ ಅಭಿಪ್ರಾಯವನ್ನು ಧ್ವನಿಸುತ್ತಾರೆ–

‘ಒಬ್ಬ ವ್ಯಕ್ತಿಯನ್ನು ಪ್ರೀತಿಸುವುದೆಂದರೆ ಅವನ ಬೆಳವಣಿಗೆಯ ಬಗ್ಗೆ, ಸರ್ವತೋಮುಖ ಪ್ರಗತಿಯ ಬಗ್ಗೆ, ತಾನು ಜವಾಬ್ದಾರನೆಂಬ ಭಾವನೆಯನ್ನು ತಳೆದು, ಆತನನ್ನು ಗಮನವಿತ್ತು ನೋಡಿಕೊಳ್ಳುವುದು ಎಂಬ ಅರ್ಥ ಸೂಚಿತವಾಗುವುದು.’

\newpage

ಕೊಳಚೆ ಪ್ರದೇಶದ ಮಕ್ಕಳಿಗೆ ವಿದ್ಯೆಯನ್ನು ನೀಡಿದ ಶೀಲಾ ರೂರ್​ಕೆ ತೋರಿದ ಪ್ರೀತಿ ಮೇಲಿನ ರೀತಿಯದು.

ಆಕೆ ಪ್ರತಿಯೊಬ್ಬ ವಿದ್ಯಾರ್ಥಿಯ ಕಡೆಗೂ ಸಾಕಷ್ಟು ಗಮನವಿತ್ತು ಅವರ ಅಭಿರುಚಿ, ಸಂಸ್ಕಾರ, ಸಾಮರ್ಥ್ಯ, ದೌರ್ಬಲ್ಯಗಳನ್ನು ಪರಿಶೀಲಿಸಿ ಅವರ ಅಭ್ಯುದಯಕ್ಕೇನು ಮಾಡಬೇಕೆಂದು\break ಯೋಚಿಸಿದ್ದಳು–ಎಂದರೆ ಮಕ್ಕಳನ್ನು ತಿಳಿದುಕೊಳ್ಳಲು ಸಮರ್ಥಳಾಗಿದ್ದಳು. ಅವನ ನೋವು, ನಲಿವುಗಳನ್ನು ತಿಳಿದುಕೊಂಡು ಅವನ ಅಭಿವೃದ್ಧಿಯ ಆವಶ್ಯಕತೆಗಳನ್ನು ಪೂರೈಸಲೂ ಪ್ರಯತ್ನಿ\-ಸಿದ್ದಳು. ಪ್ರತಿಯೊಬ್ಬ ವಿದ್ಯಾರ್ಥಿಯೂ ಬಾಲಸಹಜ ಚಾಪಲ್ಯದಿಂದ ನೀಡಿದ ಕಿರುಕುಳವನ್ನು ನೊಂದುಕೊಳ್ಳದೆ ಕ್ಷಮಿಸಿದ್ದಳು. ಪ್ರೀತಿಯನ್ನು ತೋರಿಸಿದ್ದಳೆಂದು ಅವರನ್ನು ಮನಬಂದಂತೆ ವರ್ತಿಸಲು ಬಿಟ್ಟವಳಲ್ಲ. ಏಟು ಕೊಡಬೇಕಾದ ಸನ್ನಿವೇಶದಲ್ಲಿ ಅಪರಾಧಕ್ಕನುಗುಣವಾಗಿ ಬೆತ್ತದ ಉಪಯೋಗ ಮಾಡಿದ್ದಳು. ತಪ್ಪು ಮಾಡಿದಾಗ ಶಿಕ್ಷೆ ನೀಡಿದಂತೆ ಮಕ್ಕಳ ಗುಣವೈಶಿಷ್ಟ್ಯಗಳನ್ನು ಗುರುತಿಸಿ ಪ್ರೋತ್ಸಾಹ ಪ್ರಶಂಸೆಗಳನ್ನೂ ನೀಡಿದ್ದಳು. ಆ ಮಕ್ಕಳ ಶುಭಚಿಂತನೆಯಿಂದ ಶ್ರಮವಹಿಸಿ ದುಡಿಯುತ್ತ ಅವರನ್ನು ತಿದ್ದಿಬೆಳೆಸಲು ದೀರ್ಘಕಾಲ ಯತ್ನಿಸಿದಳು. ಸಂಬಳ ತೆಗೆದುಕೊಂಡದ್ದಕ್ಕೆ ಕನಿಷ್ಠಮಟ್ಟದ ಕೆಲಸ ಮಾಡಿ ಮುಗಿಸಿದ್ದರೆ ಯಾರೂ ಅವಳನ್ನು ದೂರುತ್ತಿರಲಿಲ್ಲ. ಅವಳ ಸದ್ಭಾವನೆ, ಸದಾಶಯಗಳನ್ನು ಅರಿಯದವರು ‘ಇವಳೇಕೆ? ಇಷ್ಟೊಂದು ಒದ್ದಾಟ ಮಾಡುತ್ತಿದ್ದಾಳೆ? ಸಂಸಾರದ ತಾಪತ್ರಯವಿಲ್ಲವೆಂದು ಕಾಣುತ್ತದೆ! ಬೇರೆ ಕೆಲಸವಿಲ್ಲ! ಇಲ್ಲವಾದರೆ ಈ ಮಕ್ಕಳನ್ನು ಸರಿಪಡಿಸಲು ಯಾರಿಗೆ ಸಾಧ್ಯ? ಬೇರೇನಾದರೂ ಲಾಭವಿದ್ದರೂ ಇರಬಹುದೆ?’... ಎಂದಿರಬಹುದು! ಆದರೆ ಪ್ರೀತಿಯ ಶಕ್ತಿ ಅಂಥದು! ಪ್ರೀತಿಸುವ ವ್ಯಕ್ತಿ, ಪ್ರೀತಿಪಾತ್ರನಿಗಾಗಿ ಮಾಡುವ ತ್ಯಾಗ ಮತ್ತು ಸೇವೆ, ಇತರರಿಗೆ ‘ಅತಿ’ಯಾಗಿ ಕಾಣಬಹುದು. ಮಗು ರೋಗಪೀಡಿತನಾಗಿ ಅತ್ತು ಕರೆಯುತ್ತಿರುವಾಗ ತಾಯಿ ಹಾಯಾಗಿ ನಿದ್ರಿಸಬಲ್ಲಳೆ? ನಿದ್ರೆಯನ್ನು ತೊರೆದು, ಹತ್ತು ಹಲವು ತೊಂದರೆಗಳನ್ನು ಎದುರಿಸಿ, ಆಕೆ ಮಗುವಿನ ಆರೈಕೆ ಮಾಡಲು ಸರ್ವಪ್ರಯತ್ನ ಮಾಡುವಳು. ಪ್ರೀತಿಯ ಇಂಥ ಅಭಿವ್ಯಕ್ತಿಯಿಂದಲೇ ಮನುಷ್ಯ ತನ್ನ ಶೈಶವದ ಅಸಹಾಯ ಸ್ಥಿತಿಯಿಂದ ಪಾರಾಗುತ್ತಾನೆ. ಪ್ರೀತಿಯ ಇಂಥ ಅಭಿವ್ಯಕ್ತಿಯುಂಟಾದಾಗಲೇ ಮಕ್ಕಳಲ್ಲಿ ಆತ್ಮಗೌರವ, ಆತ್ಮಶ್ರದ್ಧೆಯಿಂದ ತನ್ನದೇ ಆದ ವಿಶಿಷ್ಟ ವ್ಯಕ್ತಿತ್ವ ರೂಪುಗೊಳ್ಳಲು ಸಾಧ್ಯ. ಅಧ್ಯಾಪಕಿಯಾಗಿದ್ದ ಶೀಲಾ ರೂರ್​ಕೆ ಅಂಥ ಪರಿಶುದ್ಧ ಪ್ರೀತಿಯನ್ನು ಆ ನತದೃಷ್ಟ ಮಕ್ಕಳೆಡೆಗೆ ಹರಿಯಿಸಲು ಸಮರ್ಥಳಾದುದರಿಂದಲೇ ಅವರು ತಮ್ಮ ಕಾಲ ಮೇಲೆ ತಾವು ನಿಂತು ಸಭ್ಯಜೀವನ ನಡೆಸಲು ಸಮರ್ಥರಾದರು! ‘ಪ್ರತಿಯೊಬ್ಬ ವಿದ್ಯಾರ್ಥಿಯನ್ನೂ ಪ್ರೀತ್ಯಾದರಗಳಿಂದ ನೋಡಿಕೊಂಡಿದ್ದೆ’ ಎಂಬ ಸಾಮಾನ್ಯವಾಗಿ ತೋರುವ ವಾಕ್ಯದ ಹಿಂದೆ ಎಷ್ಟೊಂದು ಸ್ವಾರ್ಥತ್ಯಾಗ ಮತ್ತು ಸೇವಾ ಮನೋಭಾವದಿಂದ ಮಾಡಿದ ಕಾರ್ಯ ಅಡಗಿದೆ ಎಂಬುದು ನಿಮಗೀಗ ಸ್ಪಷ್ಟವಾಗಿರಬೇಕು. ನಿಮ್ಮಲ್ಲೂ ಉಕ್ಕಲಿ–ಪ್ರೀತಿಗಾಗಿ ತ್ಯಾಗ, ಸೇವೆಗಾಗಿ ಪ್ರೀತಿ.


\section*{ನಿಮ್ಮ ಆದರ್ಶ}

\addsectiontoTOC{ನಿಮ್ಮ ಆದರ್ಶ}

ನೀವು ಜನಪ್ರಿಯರಾಗಬೇಕೆಂದಿರುವಿರಾ? ದಾರಿ ಕಾಣದಾಗಿದೆಯೆ? ಉಪಾಯ ಇಲ್ಲಿದೆ: ಜನರ ವೈಶಿಷ್ಟ್ಯ ಸಾಮರ್ಥ್ಯ ದಕ್ಷತೆಗಳನ್ನು ಗಮನಿಸಿ, ಗುರುತಿಸಿ, ಹೃತ್ಪೂರ್ವಕವಾಗಿ ಆನಂದಿಸಬಲ್ಲಿರಾ? ಅವರನ್ನು ಪ್ರೋತ್ಸಾಹಿಸಬಲ್ಲಿರಾ? ತಾಳ್ಮೆಯಿಂದ ಅವರ ಹಿತಚಿಂತನೆ ಮಾಡಬಲ್ಲಿರಾ? ಅವರಿಗಾಗಿ ದೇವರಲ್ಲಿ ಪ್ರಾರ್ಥಿಸಬಲ್ಲಿರಾ? ಎಂದರೆ ಜನರಿಗೆ ನಿಃಸ್ವಾರ್ಥ ಪ್ರೀತಿಯನ್ನು ನೀಡಬಲ್ಲಿರಾ? ಹೌದು, ಎಂದಾದಲ್ಲಿ ನೀವು ಜನರಿಂದ ಅಪಾರ ಪ್ರೀತಿಯನ್ನು ಗಳಿಸಬಲ್ಲಿರಿ.

ನೀವು ಸುಲಭವಾಗಿ ಒಳ್ಳೆಯ ಚಿಂತನಶೀಲ ವ್ಯಕ್ತಿಗಳೂ, ಸಂಶೋಧಕರೂ ಆಗಬೇಕೆಂದಿರು\-ವಿರಾ? ನೀವು ಆರಿಸಿಕೊಂಡ ಅಧ್ಯಯನದ ವಿಷಯವನ್ನು ಮನಸಾರೆ ಪ್ರೀತಿಸಿರಿ. ಆಗ ಎಂಥೆಂಥ ಅದ್ಭುತ ಯೋಚನೆಗಳು ಓಡೋಡಿ ಬರುತ್ತವೆಂದು ನೋಡಿ.

ನೀವು ಆರೋಗ್ಯವಂತರೂ, ಬಲಶಾಲಿಗಳೂ ಆಗಬೇಕೆಂದಿರುವಿರಾ? ಶುಭೇಚ್ಛೆ ಹಾಗೂ ಪ್ರೀತಿಯ ತರಂಗಗಳನ್ನು ನಿಮ್ಮ ಸುತ್ತಮುತ್ತಲಿನ ಜನ ಮಾತ್ರವಲ್ಲ, ಪಶುಪಕ್ಷಿ, ಕ್ರಿಮಿಕೀಟ, ತರುವೃಕ್ಷ, ಫಲ, ಪುಷ್ಪಗಳೆಡೆಗೆ, ಎಂದರೆ ನಿಸರ್ಗದೆಡೆ ಹರಿಸಿ. ಆಗ ಇದರ ಪರಿಣಾಮ ನಿಮ್ಮ ಅಂತರ್ಮನಸ್ಸಿನ ಮೇಲಾಗಿ ರಕ್ತದ ಕಣಗಳ ಮೇಲೂ ಸತ್​ಪರಿಣಾಮ ಉಂಟಾಗುವುದು!

ದ್ವೇಷವನ್ನು ಬೆಳೆಸಿ, ಬಿತ್ತರಿಸಿದಿರಾದರೆ ಶರೀರ ಹಾಗೂ ಮನಸ್ಸು ವಿಷಯುಕ್ತವಾಗುತ್ತವೆ.

ಹೌದು, ಪ್ರೀತಿ ಮಾತ್ರ ಅತ್ಯಂತ ಹಿತಕರ ಹಾಗೂ ಆರೋಗ್ಯವರ್ಧಕ ರಾಸಾಯನಿಕ ಪರಿಣಾಮಗಳನ್ನು ರಕ್ತದಲ್ಲಿ ಉಂಟುಮಾಡುತ್ತದೆ. ಅಂದಮೇಲೆ ನೀವು ಅನುಸರಿಸಬೇಕಾದ ಆದರ್ಶಪಥ ಯಾವುದೆಂದು ಬೇರೆ ಹೇಳಬೇಕೆ?


\section*{ಒಲವು ಗೆಲವು}

\addsectiontoTOC{ಒಲವು ಗೆಲವು}

.... ‘ಸ್ವಾರ್ಥರಹಿತ ಪರಿಶುದ್ಧ ಪ್ರೀತಿಗೆ ಅಪಾರಶಕ್ತಿ ಇದೆ. ಬಹುಮಂದಿ ಭಾವಿಸಿರುವು\-ದಕ್ಕಿಂತ ಮಿಗಿಲಾದ ರಚನಾತ್ಮಕ ಹಾಗೂ ರೋಗನಿರ್ಮೂಲನಶಕ್ತಿಯನ್ನು ಇದು ಹೊಂದಿದೆ. ಶಾರೀರಿಕ, ಮಾನಸಿಕ ಹಾಗೂ ನೈತಿಕ ಆರೋಗ್ಯಕ್ಕೆ ಅತ್ಯಾವಶ್ಯಕವಾದ ಜೀವನದಾಯಿ ಶಕ್ತಿಯೇ ಈ ಪ್ರೀತಿ. ಅಹಂಕಾರಿಗಳಿಗಿಂತ ಪರಹಿತಾಸಕ್ತರಾದ, ನಿಃಸ್ವಾರ್ಥ ಪ್ರೀತಿಯನ್ನು ನೀಡುವ ಜನ ದೀರ್ಘಾಯುಗಳಾಗಬಲ್ಲರು.ಈ ಪರಿಶುದ್ಧ ಪ್ರೀತಿಯಿಂದ ವಂಚಿತರಾದ ಎಳೆಯ ಮಕ್ಕಳು ಈ ಸಮಾಜದಲ್ಲಿ ನೈತಿಕವಾಗಿ ಸಾಮಾಜಿಕವಾಗಿ ಹಿಂದುಳಿಯುತ್ತಾರೆ. ದೈಹಿಕ ಅನಾರೋಗ್ಯ, ಕುಡಿತದ ಚಟ, ಅಪರಾಧ ಹಾಗೂ ಆತ್ಮಹತ್ಯೆಯ ಪ್ರವೃತ್ತಿಗಳಿಗೆ, ಪ್ರೀತಿಯೆಂಬುದು ಅತ್ಯಂತ ಪರಿಣಾಮಕಾರಿಯಾದ ದಿವ್ಯೌಷಧ. ಅದು ದ್ವೇಷ, ಭಯ ಹಾಗೂ ನರಮಂಡಲದ ದೌರ್ಬಲ್ಯವೇ ಮೊದಲಾದ ತೊಂದರೆಗಳನ್ನು ದೂರಮಾಡಬಲ್ಲ ಸಾಮರ್ಥ್ಯವನ್ನು ಹೊಂದಿದೆ. ಇಂತಹ ಪರಿಶುದ್ಧ ಪ್ರೀತಿಯನ್ನು ಹೊಂದದೇ ದಿವ್ಯವಾದ ಹಾಗೂ ಚಿರಂತನವಾದ ಸಂತೋಷವನ್ನು ನಾವು ಪಡೆಯಲಾರೆವು. ನಿಜವಾದ ಪ್ರೀತಿ ಇರುವಲ್ಲಿಯೇ ಸ್ವಾತಂತ್ರ್ಯ ಹಾಗೂ ಒಳ್ಳೆಯತನ ಗೌರೀಶಂಕರದಷ್ಟು ಎತ್ತರವನ್ನು ತಲುಪುತ್ತವೆ. ಜನಾಂಗವನ್ನು ಉನ್ನತಸ್ತರಗಳಿಗೇರಿಸಿ, ಅವರ ಧ್ಯೇಯೋದ್ದೇಶಗಳನ್ನು ಉದಾತ್ತಗೊಳಿಸಬಲ್ಲ ಶಕ್ತಿಯೆ ಈ ಪರಿಶುದ್ಧ ನಿಃಸ್ವಾರ್ಥ ಪ್ರೀತಿ. ಇದು ಅತ್ಯಂತ ಸೂಕ್ಷ್ಮವಾದ, ಆದರೆ, ಪ್ರಬಲ ಪರಿಣಾಮಕಾರಿಯಾದ ಶಕ್ತಿ. ಸೀಮಾರಹಿತವಾಗಿ ಈ ಪರಿಶುದ್ಧ ಪ್ರೀತಿಯ ಶಕ್ತಿಯನ್ನು ಸಮಸ್ತ ಜೀವಿಗಳೆಡೆಗೆ ಹರಿಯಿಸುವುದರ ಮೂಲಕವೇ ಮನುಷ್ಯನ ಅಸ್ತಿತ್ವಕ್ಕೆ ಕುಠಾರಾಘಾತ\-ವಾಗುವಂತಹ, ಮನುಷ್ಯ ಮನುಷ್ಯರೊಳಗಣ ದ್ವೇಷಪೂರಿತ ಒಳಜಗಳಗಳನ್ನು ಪರಿಹರಿಸಬಹುದು. ಈ ಪ್ರೀತಿಯನ್ನು ಬಿಟ್ಟು, ಮಾರಕಾಸ್ತ್ರಗಳ ಸಂಗ್ರಹದಿಂದಾಗಲೀ, ನಿರ್ಬಂಧಕ ಪೋಲೀಸು ದಳ\-ಗಳಿಂ\-ದಾಗಲೀ, ಆರ್ಥಿಕ ಹಾಗೂ ರಾಜಕೀಯ ವಿಧಾನಗಳಿಂದಾಗಲೀ, ಹೈಡ್ರೋಜನ್ ಬಾಂಬಿ\-ನಿಂದಾ\-ಗಲೀ, ಜಗತ್ತು ಎದುರಿಸಬೇಕಾದ ಬರಲಿರುವ ದುರಂತವನ್ನು ತಪ್ಪಿಸಲು ಸಾಧ್ಯವಿಲ್ಲ. ಪ್ರೀತಿ ಮಾತ್ರ ಈ ಒಂದು ಪವಾಡವನ್ನು ಮಾಡಬಲ್ಲದು. ಆದರೆ ಪ್ರೀತಿಯ ಸ್ವರೂಪ, ಅದನ್ನು ಸರಿಯಾದ ರೀತಿಯಲ್ಲಿ ಉಂಟುಮಾಡುವ ವಿಧಾನ, ಅದರ ಸಂಗ್ರಹ ಹಾಗೂ ಉಪಯೋಗ ಇವುಗಳನ್ನು ನಾವು ಚೆನ್ನಾಗಿ ತಿಳಿದಿರಬೇಕು’ ಎಂಬುದು ಖ್ಯಾತ ಸಮಾಜಶಾಸ್ತ್ರಜ್ಞರಾದ ಪಿಟ್ರಿಮ್ ಎ. ಸೊರೊಕಿನ್​ರ ಸದುಕ್ತಿ.

ಮಕ್ಕಳ ಸರಿಯಾದ ಬೆಳವಣಿಗೆಗೆ ತಾಯಿಯ ಪ್ರೀತಿವಾತ್ಸಲ್ಯ ಅತ್ಯಂತ ಆವಶ್ಯಕವಾದುದು ಎಂಬುದು ಅನುಭವ ಹಾಗೂ ಸಂಶೋಧನೆಯ ಮೂಲಕ ಸಾಬೀತಾಗಿದೆ. ಸಾಂಕ್ರಾಮಿಕ ರೋಗದಿಂದ, ಹಸಿವೆಯಿಂದ, ಅಯೋಗ್ಯ ಆಹಾರದಿಂದ ಮಕ್ಕಳು ದುರ್ಬಲರಾಗಿ ಸಾಯುವಂತೆ, ತಾಯಿಯ ಒಲವನ್ನೇ ಕಂಡರಿಯದ ಮಕ್ಕಳೂ ಬಹುಬೇಗನೇ ರೋಗಿಷ್ಠರೂ, ದುರ್ಬಲರೂ ಆಗಿ ಸಾಯುತ್ತಾರೆಂದು ಈ ಕ್ಷೇತ್ರದಲ್ಲಿ ಸಂಶೋಧನೆ ಮಾಡಿದ ಡಾ.\ ರೀನೆ ಎ. ಸ್ಪೀಟ್ಸ್​ ಹೇಳುತ್ತಾರೆ. ಅನಾಥಾಲಯ ಒಂದರ ಮೂವತ್ತನಾಲ್ಕು ಅನಾಥಶಿಶುಗಳ ಸಾವನ್ನು ಯಥಾವತ್ತಾಗಿ ಚಿತ್ರೀಕರಿಸಿ ಅದರ ವರದಿಯನ್ನು ಅವರು ನೀಡಿದ್ದಾರೆ. ತಾಯಿಯ ಪ್ರೀತಿಯೊಂದನ್ನು ಬಿಟ್ಟು, ಬೇರೆಲ್ಲ ಆವಶ್ಯಕತೆಗಳನ್ನೂ, ರಕ್ಷಣೆಯನ್ನೂ, ವೈದ್ಯಕೀಯ ಸೌಲಭ್ಯಗಳನ್ನೂ ಆ ಅನಾಥಶಿಶುಗಳ ಆಲಯದಲ್ಲಿ ಒದಗಿಸಲಾಗಿತ್ತು. ಮಮತಾಮಯಿ ಮಾತೆಯ ಪ್ರೀತಿಯ ಅಭಾವವೇ ಆ ಶಿಶುಗಳ ಮರಣಕ್ಕೆ ಕಾರಣವಾಯಿತು ಎನ್ನುತ್ತಾರವರು. ತಾಯಿಯಿಲ್ಲದ ಮಕ್ಕಳ ದುಃಸ್ಥಿತಿಯ ಸಂಪೂರ್ಣ ವಿವರಣೆಯನ್ನು ಡಾಕ್ಟರ್ ಸ್ಪಿಟ್ಸ್ ತಯಾರಿಸಿದ ಚಲನಚಿತ್ರದಲ್ಲಿ ನಾವು ನೋಡಬಹುದು. ತಂದೆ ತಾಯಿಗಳ ಅಗಲುವಿಕೆಯ ಮೂರು ತಿಂಗಳುಗಳಲ್ಲೇ ಶಿಶುಗಳು ಹಸಿವೆಯನ್ನೇ ಮರೆತವು. ನಿದ್ರಿಸಲು ಅಸಮರ್ಥವಾದುವು. ಅವುಗಳ ಮುಖಗಳು ಕಳೆಗುಂದಿದ್ದವು. ಅವು ಅತ್ತು ಅತ್ತು ಸುಸ್ತಾದವು. ಇನ್ನೆರಡು ತಿಂಗಳು ಅದೇ ಸ್ಥಿತಿಯಲ್ಲಿ ನರಳಿದಾಗ ಬುದ್ಧಿಹೀನ ಮಕ್ಕಳಂತೆ ಕಾಣತೊಡಗಿದವು. ಒಂದು ವರುಷ ಮುಗಿಯುವುದರೊಳಗಾಗಿಯೇ ಇಪ್ಪತ್ತೇಳು ಮಕ್ಕಳು ತೀರಿಕೊಂಡವು. ಏಳು ಮಕ್ಕಳು ಎರಡನೇ ವರ್ಷದಲ್ಲಿ ತೀರಿಕೊಂಡವು. ಉಳಿದ ಇಪ್ಪತ್ತೊಂದು ಮಂದಿ ಬದುಕಿದ್ದರೂ ಪ್ರೀತಿಯ ಕೊರತೆಯಿಂದಾಗಿ ಬುದ್ಧಿಹೀನರಾಗಿದ್ದವು.

‘ಮಾನಸಿಕ ಹಾಗೂ ನೈತಿಕ ಕುಸಿತಗಳನ್ನು ತಡೆಗಟ್ಟಲು, ಸರಿಪಡಿಸಲು, ಪ್ರೀತಿಯ ಶಕ್ತಿಯು ಅತ್ಯಂತ ಆವಶ್ಯಕ. ಪ್ರೀತಿಯ ಎರಡು ಮುಖಗಳಾದ ಪ್ರೀತಿಸುವುದು, ಹಾಗೂ ಪ್ರೀತಿಸಲ್ಪಡು\-ವುದು–ನವಜಾತಶಿಶುಗಳು ನೈತಿಕ ಹಾಗೂ ಬೌದ್ಧಿಕವಾಗಿ ಬೆಳೆದು ಉತ್ತಮ ಪ್ರಜೆಗಳಾಗಲು ಅತ್ಯಂತ ಆವಶ್ಯಕ. ಈ ಪ್ರೀತಿ, ವ್ಯಕ್ತಿಗಳ ಮತ್ತು ಎಲ್ಲ ಜೀವಿಗಳ ದುರ್ಬಲತೆಗಳನ್ನು ನಿವಾರಣೆ ಮಾಡುವು\-ದಲ್ಲದೆ, ಮನಸ್ಸಿಗೆ ನವಚೇತನವನ್ನೂ ನೀಡುತ್ತದೆ. ಮಾನಸಿಕ ನೈತಿಕ ಹಾಗೂ ಸಾಮಾಜಿಕ ಅಭ್ಯುದಯಕ್ಕೆ ಅಭಿವೃದ್ಧಿ ಅಥವಾ ಪ್ರಗತಿಗೆ, ಒಟ್ಟಿನಲ್ಲಿ, ಬದುಕಿನ ಸರ್ವತೋಮುಖ ಏಳ್ಗೆಗೆ ಈ ಪ್ರೀತಿ ಅತ್ಯಂತ ಮುಖ್ಯವಾದುದೆಂಬುದು ಒಂದು ಸಂಶಯರಹಿತ ಸತ್ಯಸಂಗತಿ. ಪ್ರೀತಿಸುವುದು, ಪ್ರೀತಿಸಲ್ಪಡುವುದು–ಒಂದು ಅತ್ಯಂತ ಮುಖ್ಯ ಜೀವಸತ್ವ ಅಥವಾ ಆಹಾರಸತ್ವವಾಗಿ ಪರಿಣಮಿಸುತ್ತದೆ. ಸಂತಸದ ಬಾಳ್ವೆಗೆ, ಸರ್ವತೋಮುಖವಾದ ಬೆಳವಣಿಗೆಗೆ ಪ್ರೀತಿಯ ಈ ವಿಟಮಿನ್ ಅನಿ\-ವಾರ್ಯ\-ವಾದ ಅಂಶ.’ (ಡಾ.\ ರೀನೆ ಎ.\ ಸ್ಪೀಟ್ಸ್​)


\section*{ಸೇವಾಭಾವ}

\addsectiontoTOC{ಸೇವಾ\-ಭಾವ}

ಸಮರ್ಪಣೆಯ ಭಾವದಿಂದ ಪ್ರೀತಿಪಾತ್ರ ವ್ಯಕ್ತಿಗಳನ್ನು ಸೇವಿಸುವ ಹಲವರಿರಬಹುದು. ಆದರೆ ಸೇವೆ ಮಾಡುವವನು ‘ಸೇವ್ಯನನ್ನು’, ಎಂದರೆ, ಸೇವೆಯನ್ನು ಪಡೆಯುವವನನ್ನು, ತನ್ನ ಅಧೀನ\-ದಲ್ಲಿರಿಸಿಕೊಳ್ಳಲು ಯತ್ನಿಸುವುದುಂಟು. ಅವರ ಮೇಲೆ ಪ್ರೀತಿಯ ಹೆಸರಿನಲ್ಲೇ ಅಧಿಕಾರ\break ಚಲಾಯಿಸಲೂ, ಶೋಷಿಸಲೂ ಹವಣಿಸುವುದುಂಟು. ಸೇವ್ಯರ ಯಾವತ್ತೂ ಅಭ್ಯುದಯಕ್ಕೆ ತಾವೇ ಕಾರಣರೆಂಬುದನ್ನು ಸೂಚ್ಯವಾಗಿಯೋ, ವ್ಯಕ್ತವಾಗಿಯೋ ಹೇಳುತ್ತ ತಿರುಗುವುದುಂಟು. ತಾನಿಲ್ಲದಿದ್ದರೆ ಇವನ ಬೇಳೆ ಬೇಯುವುದುಂಟೆ? ಎನ್ನುವುದುಂಟು. ತಾನಿಷ್ಟು ಸಹಾಯಮಾಡಿ, ಸೇವೆ ಮಾಡಿ, ಕಾಸಿನ ಉಪಯೋಗವೂ ತನಗೆ ಆಗಲಿಲ್ಲ ಎನ್ನುವುದುಂಟು. ತನ್ನ ಸೇವೆಯನ್ನು ಪಡೆದವನು ಸದಾ ‘ಕೃತಜ್ಞನಾಗಿದ್ದೇನೆ’ ಎಂದು ಹೇಳುತ್ತಿರಬೇಕೆಂದು ನಿರೀಕ್ಷಿಸುವುದುಂಟು. ಯಾವ ಸೇವೆಯನ್ನೂ ಮಾಡದ ಸ್ವಾರ್ಥಿಗಿಂತ ಸೇವೆಯನ್ನು ಮಾಡಿ ಹೆಗ್ಗಳಿಕೆಯನ್ನು ಮೆರೆಯುವವರು ಮೇಲು ಎನ್ನುವುದನ್ನು ಅಲ್ಲಗಳೆಯಲು ಸಾಧ್ಯವಿಲ್ಲ. ಆದರೆ ಅದು ಶುದ್ಧ ನಿಃಸ್ವಾರ್ಥ ಪ್ರೇಮವಲ್ಲ. ನಿಃಸ್ವಾರ್ಥ ಪ್ರೇಮದ ಮಜಲನ್ನು ತಲುಪಲು ಅದು ಮೆಟ್ಟಲು ಎಂದೇನೋ ಹೇಳಬಹುದು. ಆದರೆ ಎಲ್ಲರೂ ಆ ಮೆಟ್ಟಲನ್ನು ಏರಿ ಹೋಗುವರೆಂದೇನೂ ಇಲ್ಲ.

ನಿಜವಾದ ಪ್ರೀತಿ ವ್ಯಕ್ತಿಗೌರವ ಹಾಗೂ ವ್ಯಕ್ತಿಸ್ವಾತಂತ್ರ್ಯಕ್ಕೆ ಹಾನಿ ಬರದಂತೆ, ಸ್ವಲಾಭ, ಸಂಕುಚಿತ ಸ್ವಾರ್ಥಾಭಿಲಾಷೆಯಿಲ್ಲದೆ ವ್ಯಕ್ತವಾಗುತ್ತದೆ. ಆಧ್ಯಾತ್ಮಿಕ ಹಿನ್ನೆಲೆಯುಳ್ಳ ಮಾತೃತ್ವದ ಮಹಾ ಆದರ್ಶವನ್ನು ಮೈಗೂಡಿಸಿಕೊಂಡ ತಾಯಿಯಲ್ಲಿ ನಿಃಸ್ವಾರ್ಥ ದೈವೀಪ್ರೇಮವನ್ನು ಕಾಣಬಹುದು.


\section*{ಸಲುಗೆಯ ಇತಿಮಿತಿ}

\addsectiontoTOC{ಸಲುಗೆಯ ಇತಿಮಿತಿ}

{\parfillskip=0pt ಒಮ್ಮೆ ಸತ್ಯಭಾಮೆ ದ್ರೌಪದಿಯನ್ನು ಸಂಧಿಸಿದಾಗ ‘ನಿನ್ನ ಗಂಡಂದಿರನ್ನು ನೀನು ವಶದಲ್ಲಿಟ್ಟು\par}\newpage\noindent ಕೊಂಡಿರುವುದು ಹೇಗೆ? ನಿನ್ನಿಂದ ಆ ಉಪಾಯವನ್ನು ತಿಳಿದು ನಾನೂ ಕೃಷ್ಣನನ್ನು ಒಲಿಸಿಕೊಳ್ಳುತ್ತೇನೆ’ ಎಂದಳು. ಕೃಷ್ಣ ಸತ್ಯಭಾಮೆಯ ವಶದಲ್ಲಿ ಆಗಲೇ ಇದ್ದಾನೆಂಬುದು ದ್ರೌಪದಿಗೆ\break ಗೊತ್ತಿದ್ದರೂ, ತಾನು ತನ್ನ ಗಂಡಂದಿರ ಒಲುಮೆ ಗಳಿಸಿದ ಮಾರ್ಗವನ್ನು ಹೀಗೆಂದು ಹೇಳಿದಳು\footnote{ ಮಾಸ್ತಿ ವೆಂಕಟೇಶ ಅಯ್ಯಂಗಾರ್, ‘ಭಾರತ ತೀರ್ಥ’}– ‘ನನ್ನ ಗಂಡಂದಿರು ಮೃದುಸ್ವಭಾವದವರು, ಸತ್ಯಶೀಲರು, ಸತ್ಯಧರ್ಮಗಳನ್ನು ಪಾಲಿಸುವವರು. ಆದರೂ ಕ್ರೂರಿಗಳೂ, ಕೋಪಿಷ್ಠರೂ ಆಗಿದ್ದರೆ ಹೇಗೋ ಹಾಗೆ ಎಚ್ಚರದಿಂದ ಪರಿಚರಿಸಿದ್ದೇನೆ. ಪತಿಯ ಆಶ್ರಯ ಸ್ತ್ರೀಯರಿಗೆ ಧರ್ಮ. ನನ್ನ ತಿಳಿವಳಿಕೆಯಂತೆ ಹೆಂಡತಿಗೆ ಗಂಡನೇ ದೇವರು, ಅವನೇ ಗತಿ. ಅವನಿಗೆ ವಿಧೇಯತೆಯೇ ಆಕೆಯ ಸಾಧನೆ. ಇನ್ನು ಗುರು ಹಿರಿಯರನ್ನು ನಾನು ಭಕ್ತ್ಯಾದರಗಳಿಂದ ನೋಡುತ್ತೇನೆ. ನನ್ನ ಅತ್ತೆಯ ಮಾತಿಗೆ ಒಮ್ಮೆಯೂ ಪ್ರತಿ ನುಡಿದವಳಲ್ಲ. ನಾನೇ ನಿತ್ಯವೂ ಆಕೆಯನ್ನು ಪರಿಚರಿಸಿ, ಸ್ನಾನ ಭೋಜನ ಆಚ್ಛಾದನಗಳನ್ನು ನೋಡಿಕೊಳ್ಳುತ್ತೇನೆ. ಅವಳಿಗೆ ಇಲ್ಲದ ಊಟ, ನಿದ್ದೆಯನ್ನು ನಾನು ಅನುಭವಿಸುವುದಿಲ್ಲ. ಆಕೆಯನ್ನು ಪೃಥಿವಿಗೆ ಸಮಾನಳೆಂದು ಕಂಡಿದ್ದೇನೆ. ಸಾವಿರಾರು ಜನ ಸ್ನಾತಕರು, ಬ್ರಾಹ್ಮಣರು, ಅರಮನೆಗೆ ಬಂದು ಹೋಗುತ್ತಿದ್ದ ಕಾಲದಲ್ಲಿ, ಒಬ್ಬರನ್ನೂ ಬಿಡದೆ ಪ್ರತಿಯೊಬ್ಬರ ಸತ್ಕಾರವನ್ನೂ ನಾನು ನೋಡಿಕೊಳ್ಳುತ್ತಿದ್ದೆನು. ಸಾವಿರಾರು ದಾಸಿಯರು ಅರಮನೆಯಲ್ಲಿ ಇರುತ್ತಿರಲಾಗಿ ನನಗೆ ಅವರೆಲ್ಲರ ಗುರುತು ಇತ್ತು, ಹೆಸರು ತಿಳಿದಿತ್ತು. ನಾನು ಅವರೆಲ್ಲರ ಒಳಿತು ಕೆಡಕುಗಳನ್ನೂ, ಕಷ್ಟ ಸುಖಗಳನ್ನೂ ಅರಿತಿದ್ದೆನು. ಸಾವಿರಾರು ಆನೆ, ಕುದುರೆಗಳು ಯುಧಿಷ್ಠಿರನ ಲಾಯಗಳಲ್ಲಿದ್ದಾಗ ಅವುಗಳ ಲಾಲನೆಪಾಲನೆಯ ಮೇಲ್ವಿಚಾರಣೆಯನ್ನು ಮಾಡುತ್ತಿದ್ದೆ. ನನ್ನ ಗಂಡಂದಿರು ಸಂಸಾರದ ಎಲ್ಲ ಹೊರೆಯನ್ನು ನನ್ನ ಮೇಲೆ ಬಿಟ್ಟು ತಮ್ಮ ಧರ್ಮದ ಅನುಷ್ಠಾನದಲ್ಲಿ ನಿಶ್ಚಿಂತರಾಗಿ ಮುಳುಗಿದ್ದರು. ಇಷ್ಟೆಲ್ಲ ಕೆಲಸಗಳನ್ನು ನೋಡಿಕೊಳ್ಳುವುದು ಸುಲಭವಾಗಿರಲಿಲ್ಲ, ಆದರೆ ಕಷ್ಟವಾಯಿತೆಂದು ನಾನು ಸೋಲಲಿಲ್ಲ, ಹಗಲು ರಾತ್ರಿ ದುಡಿದೆನು; ಎಲ್ಲವನ್ನೂ ನೋಡಿಕೊಂಡೆನು. ನಾನು ದಿನವೂ ಎಲ್ಲರಿಗೂ ಮೊದಲು ಎದ್ದಿದ್ದೇನೆ, ಎಲ್ಲರೂ ಮಲಗಿದ ಮೇಲೆ ಮಲಗಿದ್ದೇನೆ. ನಾನು ಗಂಡಂದಿರನ್ನು ವಶಪಡಿಸಿಕೊಳ್ಳಲು ಮಾಡಿದ ತಪಸ್ಸು ಇದು.’

ಸತ್ಯಭಾಮೆ ಕೃಷ್ಣನನ್ನು ಅದೆಷ್ಟು ಪ್ರೀತಿಸುತ್ತಿದ್ದರೂ, ಸಲುಗೆಯಿಂದ ಅವನಲ್ಲಿ ಅವಜ್ಞೆ\break ತೋರಿಸಿದ ಘಳಿಗೆಗಳೂ ಇದ್ದವು. ದ್ರೌಪದಿಯ ಈ ಮಾತು ಆಕೆಯ ತಪ್ಪನ್ನು ತೋರಿಸಿ ತಿದ್ದುಪಡಿಗೆ ದಾರಿ ಮಾಡಿಕೊಟ್ಟಿತು.

ಆದ್ದರಿಂದ ಪ್ರೀತಿ ಎಂದೂ ಸಲುಗೆಯಾಗಿ ಅನಾದರಕ್ಕೆ ಕಾರಣವಾಗಬಾರದು. ಜೋಕೆ!


\section*{ಪ್ರೀತಿಗೂ ಬರ}

\addsectiontoTOC{ಪ್ರೀತಿಗೂ ಬರ}

{\parfillskip=0pt ಸಂಕುಚಿತ ಸ್ವಾರ್ಥದ ದೆಸೆಯಿಂದ ಈ ಪರಿಶುದ್ಧ ಪ್ರೀತಿಯ ಅಭಾವದ ರೋಗ ಅಥವಾ ಬರ\par}\newpage\noindent ಎಲ್ಲೆಲ್ಲೂ ಹಬ್ಬಿಕೊಂಡಿದೆ. ಶ‍್ರೀಮಂತ, ಬಡವ, ತಜ್ಞ, ಅಜ್ಞ ಎಂಬ ಭೇದವಿಲ್ಲದೇ ಎಲ್ಲರನ್ನೂ ಅದು ಹಿಡಿದುಕೊಂಡಿದೆ. ಕೊಳಚೆಪ್ರದೇಶ, ಶುಚಿಯಾದ ಪ್ರದೇಶ ಎಂಬ ತಾರತಮ್ಯ ಅದಕ್ಕಿಲ್ಲ. ಜಾತಿ, ಮತ, ಪಥ, ಪಂಗಡ ಎಂಬ ಭೇದವೂ ಇಲ್ಲ. ಈ ನಿಃಸ್ವಾರ್ಥ ಪ್ರೀತಿಯ ಪುನಃಪ್ರತಿಷ್ಠೆಯಾಗದೆ ಮನುಷ್ಯಕುಲಕ್ಕೆ ಶಾಂತಿ, ತೃಪ್ತಿಗಳು ಲಭ್ಯವಾಗವು. ಆ ನಿಟ್ಟಿನಲ್ಲಿ ದೊಡ್ಡ ವಿದ್ಯಾಸಂಸ್ಥೆಯ ಕಾರ್ಯದರ್ಶಿನಿಯೊಬ್ಬರ ಮನಕರಗಿಸುವ ಒಂದು ಅನುಭವ ಇಲ್ಲಿದೆ:

‘ವಯಸ್ಸು ಒಂಬತ್ತು ವರ್ಷ. ಹುಡುಗ ಎರಡನೇ ತರಗತಿಯಲ್ಲಿ ಓದುತ್ತಿದ್ದ. ಶಾಲೆಗೆ ಬರುವ ಅವನ ಕ್ಲಾಸಿನ ಹುಡುಗರ ಹತ್ತಿರ, ಹೆಚ್ಚಾಗಿ, ಹೆಣ್ಣುಮಕ್ಕಳ ಹತ್ತಿರ, ಹಣ ಕೇಳುತ್ತಿದ್ದ. ಕೆಲವೊಮ್ಮೆ ಒಳ್ಳೆಯ ಮಾತುಗಳಿಂದ, ಕೆಲವೊಮ್ಮೆ ಗದರಿಸಿ ಬೆದರಿಸಿ, ಪ್ರತಿದಿನವೂ ಯಾರಿಂದಲಾದರೂ ಐದು ಪೈಸೆಯಿಂದ ಇಪ್ಪತ್ತೈದು ಪೈಸೆಗಳವರೆಗೆ ಸಾಲ ಎತ್ತುತ್ತಿದ್ದ. “ನನ್ನ ತಾಯಿ ಊರಿನಿಂದ ಬಂದೊಡನೆ ನಿಮ್ಮ ಹಣವನ್ನು ಹಿಂದಿರುಗಿಸುತ್ತೇನೆ” ಎಂದು ಆಶ್ವಾಸನೆ ನೀಡಿದ್ದ. ಮಕ್ಕಳ ಹಿರಿಯರು ಆ ಹುಡುಗನ ಕಾಟವನ್ನು ಶಾಲೆಯ ಮುಖ್ಯಾಧ್ಯಾಪಕಿಯ ಹತ್ತಿರ ಹೇಳಿಕೊಂಡರು. ಮುಖ್ಯಾಧ್ಯಾಪಕಿ ತರಗತಿಯ ಅಧ್ಯಾಪಕಿಯ ಜೊತೆಗೂಡಿ ಹುಡುಗನನ್ನು ನನ್ನ ಹತ್ತಿರ ಕರೆತಂದು ಅವನ ವರ್ತನೆಯ ವಿವರಗಳನ್ನು ತಿಳಿಸಿದರು. ಹುಡುಗನನ್ನು ಗದರಿಸದೆ, ಯಾವ ಕಠಿಣೋಕ್ತಿಗಳನ್ನೂ ಆಡದೆ ಮಾತನಾಡಿಸಿದೆ. ನನ್ನ ಪ್ರಶ್ನೆಗಳಿಗೆ ಬಾಲಕ ಸ್ವಲ್ಪವೂ ಸಂಕೋಚ ಪಟ್ಟುಕೊಳ್ಳದೆ ಧೈರ್ಯವಾಗಿಯೇ ಉತ್ತರಿಸಿದ. “ಪ್ರತಿದಿನ ಬೆಳಿಗ್ಗೆ ಮನೆಯಲ್ಲಿ ಏನೂ ತಿಂಡಿ ಸಿಗುತ್ತಿಲ್ಲ. ಆದುದರಿಂದ ತಿಂಡಿಗಾಗಿ ಅವರಿವರ ಹತ್ತಿರ ದುಡ್ಡು ಕೇಳಿದೆ” ಎಂದ. ಅವನನ್ನು ನೋಡಿಕೊಳ್ಳುತ್ತಿದ್ದ ಬಂಧುಗಳು ಶಾಲೆಯಲ್ಲಿ ನೀಡುವ ಬಿಟ್ಟಿ ಊಟವನ್ನು ತೆಗೆದುಕೊಳ್ಳಲು ಅನುಮತಿ ನೀಡಿರಲಿಲ್ಲ. ಹುಡುಗನ ತಂದೆ ಸಿಂಗಪುರದಲ್ಲಿದ್ದರು. ಅವನ ತಾಯಿ ತೀರಿಕೊಂಡು ಒಂದು ವರ್ಷವೇ ಆಗಿದ್ದರೂ ತಾಯಿಯ ಸಾವಿನ ವಿಚಾರ ಬಾಲಕನಿಗೆ ತಿಳಿಸಿರಲಿಲ್ಲ. ತನ್ನ ತಾಯಿ ಸಿಂಗಪುರದಿಂದ ಹಿಂದಿರುಗಿ ಬಂದು, ತನ್ನನ್ನು ಪ್ರೀತಿಯಿಂದ ಕಾಣದ ಬಂಧುಗಳ ಹಿಡಿತದಿಂದ ಬಿಡಿಸಿ ಕರೆದೊಯ್ಯುವಳು ಎಂಬ ಮಧುರ ನಿರೀಕ್ಷೆಯಲ್ಲಿ ಅವನು ಕಾಲಯಾಪನೆ ಮಾಡುತ್ತಿದ್ದ. ತುಂಬ ಹೊತ್ತು ಮಾತನಾಡಿದ ಬಳಿಕ ಹೇಳಿದೆ: “ಯಾರ ಹತ್ತಿರವೂ ಹಣ ಕೇಳಬೇಡ, ಸಾಲ ಮಾಡಬೇಡ. ಶಾಲೆಯಲ್ಲೇ ನಿನಗೆ ಬೆಳಗಿನ ತಿಂಡಿ ಮಾತ್ರವಲ್ಲ, ಪುಸ್ತಕ ಪೆನ್ಸಿಲುಗಳ ವ್ಯವಸ್ಥೆ ಮಾಡುತ್ತೇನೆ. ಚಿಂತೆ ಬೇಡ. ಈಗ ಹೇಳು. ನಿನಗೇನು ಬೇಕು?” ನಿರ್ಭೀತಿಯ ದೃಢ ಮನಸ್ಸಿನವನೆಂದು ಕಾಣಿಸುತ್ತಿದ್ದ ಆ ಬಾಲಕ ಥಟ್ಟನೆ ಕಂಬನಿದುಂಬಿ ಅಳುತ್ತ ಹೇಳಿದ: “ನನಗೆ ಅಮ್ಮ ಬೇಕು. ನನ್ನ ತಾಯಿಯನ್ನು ಕರೆಸಿ. ನನ್ನ ತಾಯಿಯನ್ನು ನನಗೆ ಕೊಡಿ.” ಹುಡುಗನ ದುಃಖತಪ್ತ ಮನಸ್ಸಿನಿಂದ ಹೊರಹೊಮ್ಮಿದ ಕ್ರಂದನವನ್ನು ಕೇಳಿ ನಮಗೆಲ್ಲ ಕಣ್ಣೀರು ತಡೆಯುವುದು ಕಷ್ಟವಾಯಿತು. ಅವನನ್ನು ಹತ್ತಿರ ಕರೆದು ಪ್ರೀತಿಯಿಂದ ಸಮಾಧಾನ ಮಾಡಿ ಶ‍್ರೀಮಾತೆ ಶಾರದಾದೇವಿಯವರ ಒಂದು ಭಾವಚಿತ್ರವನ್ನು ಕೊಟ್ಟು “ಇವರನ್ನು ನಿನ್ನ ತಾಯಿಯೆಂದು ತಿಳಿದು ಪ್ರಾರ್ಥಿಸು. ಅವರ ಕೃಪೆ ಮಾಡಿ ನಿನಗೆ ಮಾರ್ಗದರ್ಶನ ಮಾಡುತ್ತಾರೆ” ಎಂದೆ. ಹುಡುಗ ಸಂತುಷ್ಟನಾದಂತೆ ಕಂಡಿತು. ಆದರೆ ಮಾರನೇ ದಿನದಿಂದ ಅವನು ಶಾಲೆಗೆ\break ಬರಲಿಲ್ಲ. ಶಾಲೆಯಲ್ಲಿ ನಡೆದ ವಿಚಾರಣೆಯ ಸಂಗತಿಯನ್ನು ತಿಳಿದ ಅವನ ರಕ್ಷಕರು ವರ್ಗಾ\-ವಣೆಯ ಪತ್ರವನ್ನು ತೆಗೆದುಕೊಂಡು ಹುಡುಗನನ್ನು ಬೇರೆ ಶಾಲೆಗೆ ಸೇರಿಸಿಬಿಟ್ಟರು! ಪರಿಶುದ್ಧ ನಿಃಸ್ವಾರ್ಥ ಪ್ರೀತಿಯ ಮಾತು ದೂರವಿರಲಿ ಸಾಮಾನ್ಯ ಪ್ರೀತಿಯನ್ನೂ ಕಾಣದ ಹುಡುಗನಾತ!

‘ತಂದೆಯ ಪ್ರೀತಿಯಿಂದ ವಂಚಿತಳಾದ ಎಂಟು ವರ್ಷದ ಹುಡುಗಿಯೊಬ್ಬಳು ತಾಯಿಯ ಬಗ್ಗೆ ಅನುಕಂಪದಿಂದ ಆಡಿದ ಮಾತುಗಳಿವು: “ನಾನು ನನ್ನ ತಾಯಿಯ ಸ್ಥಾನದಲ್ಲಿದ್ದರೆ ಗಂಡನನ್ನು ಅವನ ಪಾಡಿಗೇ ಬಿಟ್ಟು ಕುಟುಂಬದಿಂದ ಹೊರಗೆ ನಡೆದುಬಿಡುತ್ತಿದ್ದೆ.”

‘ಅದರ ಹಿನ್ನೆಲೆ ಹೀಗಿದೆ. ಅವರದು ದೊಡ್ಡ, ಒಳ್ಳೆಯ ಕುಟುಂಬವೇ. ಮನೆಯಲ್ಲಿ ಬಾಂಧವರು ಆಳುಕಾಳುಗಳಿಗೆ ಕೊರತೆ ಇಲ್ಲ. ಅವರ ಆರ್ಥಿಕಸ್ಥಿತಿ ಕಳಪೆಯಲ್ಲ. ಆದರೆ ಮೇಲಿನ ಮಾತನ್ನು ಹೇಳಿದ ಹುಡುಗಿಯ ತಂದೆ ಕುಡಿತ ಮತ್ತು ಜೂಜಿನ ಚಟಕ್ಕೆ ಬಲಿಯಾದವನು. ರಾತ್ರಿ ಹೊತ್ತು ಆತ ವೇಳೆ ಮೀರಿ ಮನೆಗೆ ಬಂದು ಹೆಂಡತಿಯೊಡನೆ ಸಣ್ಣಪುಟ್ಟ ವಿಚಾರಗಳಿಗೆ ರೇಗಿ, ಜಗಳಾಡಿ, ಹೊಲಸು ಮಾತುಗಳಿಂದ ಬೈದು, ಹೊಡೆದು ರಾದ್ಧಾಂತ ಮಾಡುತ್ತಿದ್ದ. ಈ ಚಿಕ್ಕ ಹುಡುಗಿ ದಿನ ನಿತ್ಯವೂ ನಿದ್ರೆಯಿಂದೆದ್ದು ತಂದೆಯ ಕ್ರೌರ್ಯ ಮತ್ತು ದುರ್ವ್ಯವಹಾರವನ್ನೂ, ತಾಯಿಯ ಸಂಕಟವನ್ನೂ ಕಂಡು, ತಾಯಿಯ ಜೊತೆ ತಾನೂ ಸ್ವಲ್ಪ ಹೊತ್ತು ಅಳುತ್ತ, ಸ್ವಲ್ಪ ಸಮಯದ ನಂತರ ನಿದ್ರಿಸುತ್ತಿದ್ದಳು. ತನ್ನ ತಂದೆಯನ್ನು ಮನಸ್ವೀ ಬೈಯಲು ಅವಳ ಪಾಲಿಗೆ ದೊರೆತ ಜಾಗ ಶಾಲೆ ಮಾತ್ರ. ಒಟ್ಟು ಮನೆಯ ವಾತಾವರಣ ಅವಳ ಮನಸ್ಸಿನಲ್ಲಿ ಒಂದೆಡೆ ಭಯದ ಭಾವವನ್ನೂ, ಇನ್ನೊಂದೆಡೆ ಪುರುಷ ಜಾತಿಯ ಮೇಲೆ ದ್ವೇಷಭಾವವನ್ನೂ ಬಿತ್ತಿ ಅವಳ ವ್ಯಕ್ತಿತ್ವದ ಮೇಲೆ ದುಷ್ಪರಿಣಾಮ ಮಾಡಿದುದರಲ್ಲಿ ಆಶ್ಚರ್ಯವೇನಿದೆ?’

‘ಸೇವಾಕ್ಷೇತ್ರದಲ್ಲಿ ಜನರ ಜೊತೆಗೆ ಇಪ್ಪತ್ತು ವರ್ಷಗಳ ಕಾಲ ದುಡಿದ ಬಳಿಕ ನನ್ನ ಮನಸ್ಸಿಗೆ ಒಂದು ಸಂಗತಿ ಸ್ಪಷ್ಟವಾಗುತ್ತಲಿದೆ. ತಾನು “ಯಾರಿಗೂ ಬೇಡವಾದವನು” ಎಂಬ ಭಾವನೆಯ ರೋಗ–ಮನುಷ್ಯರು ಅನುಭವಿಸಬಹುದಾದ ಅತಿ ಕೆಟ್ಟರೀತಿಯ ರೋಗ. ಕುಷ್ಠರೋಗಕ್ಕೆ ನಾವು ಔಷಧವನ್ನು ಕಂಡುಹಿಡಿದಿದ್ದೇವೆ. ಸಾವಿರಾರು ಜನರು ಆ ರೋಗದಿಂದ ಬಿಡುಗಡೆ ಹೊಂದಿದ್ದಾರೆ. ಕ್ಷಯ ರೋಗಕ್ಕೂ ಔಷಧ ಕಂಡುಹಿಡಿಯಲಾಗಿದೆ. ಸಾವಿರಾರು ಮಂದಿ ಕ್ಷಯರೋಗದಿಂದ ಮುಕ್ತರಾಗಿದ್ದಾರೆ. ಇತರ ನೂರಾರು ರೋಗಗಳಿಗೂ ಔಷಧ ಕಂಡುಹಿಡಿಯಲಾಗಿದೆ. ಆದರೆ ಪ್ರೀತಿಯನ್ನು ಕಂಡರಿಯದ ಈ “ಕೆಟ್ಟ” ರೋಗಕ್ಕೆ ಔಷಧವೆಲ್ಲಿದೆ? ಹೃದಯದಲ್ಲಿ ಪ್ರೀತಿಯನ್ನು ತುಂಬಿಕೊಂಡು, ಕೈಗಳಿಂದ ಸೇವೆ ಮಾಡಲು ಸಿದ್ಧರಾದ ವ್ಯಕ್ತಿಗಳಿಂದ ಮಾತ್ರವೇ ಈ ಕೆಟ್ಟರೋಗ ದೂರವಾಗಲು ಸಾಧ್ಯ’ ಎಂದರು ಮದರ್ ಥೆರೇಸಾ.

ಹಿರಿಯರ ಬಂಧುಬಾಂಧವರ ಪ್ರೀತಿಯನ್ನು ಕಾಣದೆ ಹತಾಶರಾದ ಮಕ್ಕಳ ಸಂತಾಪವನ್ನು ದೂರ ಮಾಡುವವರಾರು? ಪ್ರೀತಿಯ ವರ್ತನೆಗೆ ಖರ್ಚೇನೂ ಇಲ್ಲ. ಪ್ರೀತಿಯ ನಿಧಿ ಪ್ರತಿಯೊಬ್ಬರ ಆಂತರ್ಯದಲ್ಲೂ ಇದೆ. ಪ್ರೀತಿಯನ್ನು ನೀಡುವವರೂ, ಪಡೆಯುವವರೂ ಧನ್ಯರು. ಇಬ್ಬರಿಗೂ ಆನಂದ, ತೃಪ್ತಿ ಉಂಟಾಗುತ್ತದೆ. ಇಬ್ಬರ ವ್ಯಕ್ತಿತ್ವವೂ ಅರಳುತ್ತದೆ. ಆದರೆ ಈ ಜಗತ್ತಿನಲ್ಲಿ ಈ ಸಾಮಾನ್ಯ ಪ್ರೀತಿಯನ್ನಾದರೂ ನೀಡುವವರು ಅಲ್ಪಸಂಖ್ಯಾತರಾಗಲು ಕಾರಣವೇನು? ‘ಅರೇ ಪ್ಯಾರ್ ಕಾ ಪ್ಯಾಲಾ ರಹತೇ ಭಾಯಾ ಹೈಂ ಕ್ಯೋಂ ಜಹರ್ ತುಮ್ಹೇಂ?’ ಎಂದಂತೆ ‘ಅಯ್ಯೋ, ಪ್ರೀತಿಯ ಅಮೃತಕುಂಭವಿರುತ್ತ ದ್ವೇಷದ ವಿಷವೇಕೆ ನಿನಗೆ ಹಿಡಿಸಿದೆ?’ ಕಾರಣ ಇಷ್ಟೆ: ಅಜ್ಞಾನದ ಪ್ರಭಾವ ಮತ್ತು ತರಬೇತಿಯ ಅಭಾವ. ಸಾಮಾನ್ಯ ಪ್ರೀತಿಯನ್ನೋ, ನಿಜವಾದ ನಿಃಸ್ವಾರ್ಥ ದೈವೀಪ್ರೀತಿಯನ್ನೋ, ಅನುಭವಿಸದ ವ್ಯಕ್ತಿ, ಅದರ ಮಹತ್ವವನ್ನು ತಿಳಿಯದ ವ್ಯಕ್ತಿ, ಅದನ್ನು ಇತರರಿಗೆ ನೀಡುವುದಾದರೂ ಹೇಗೆ? ಆದರೆ ಒಂದಂತೂ ನಿಜ–ಪ್ರೀತಿಯ ಮಹಾಪೂರಕ್ಕೆ ಹೃದಯಗಳ ದೋಷ, ದುರಿತಗಳೆಲ್ಲ ಕೊಚ್ಚಿ ಹೋಗಿ, ಬಾಂಧವ್ಯದ ಬೆಸುಗೆ ಬಲವಾಗುತ್ತದೆ.


\section*{ಪ್ರೀತಿಯ ಮಹಿಮೆ}

\addsectiontoTOC{ಪ್ರೀತಿಯ\break ಮಹಿಮೆ}

ಸ್ವಾಮಿ ವಿವೇಕಾನಂದರು ಹೀಗೆನ್ನುತ್ತಾರೆ: ‘ಜಗತ್ತೇ ಪರಿಶುದ್ಧ ಪ್ರೀತಿಗಾಗಿ ಆತುರದಿಂದ ಹಾತೊರೆ\-ಯುತ್ತಿದೆ. ಪ್ರತಿಫಲವನ್ನು ಅಪೇಕ್ಷಿಸದೆ ನಾವು ಅದನ್ನು ಹಂಚಬೇಕು. ಕೊಡು, ಕೊಡು, ಪ್ರತಿಫಲಾಪೇಕ್ಷೆ ಇಲ್ಲದೆ.’

‘ಪ್ರೀತಿಗೆ ಎಂದೂ ಅಪಜಯವಿಲ್ಲ. ಇಂದೋ, ನಾಳೆಯೋ, ಶತಮಾನಗಳ ಬಳಿಕವೋ, ಸತ್ಯವೇ ಜಯಿಸುವುದು. ಪ್ರೀತಿಯು ಜಯವನ್ನು ತಂದೇ ತರುವುದು. ನೀವು ನಿಮ್ಮ ಸಹೋದರ\-ರನ್ನು ಪ್ರೀತಿಸುವಿರೇನು?.... ಪತ್ರಿಕೆಗಳು ಏನು ಹೇಳುತ್ತವೆ ಎಂಬುದನ್ನು ನಾನು ಪರಿಶೀಲಿಸುವುದಿಲ್ಲ. ಪ್ರೀತಿಯ ಸರ್ವಶಕ್ತತೆಯಲ್ಲಿ ನಂಬಿಕೆ ಇಡಿ. ನಿಮ್ಮಲ್ಲಿ ಪರಿಶುದ್ಧ ಪ್ರೀತಿ ಇದೆಯೇ? ನೀವು ಸರ್ವಶಕ್ತರು.’

ಯಾವುದು ನಮ್ಮಲ್ಲಿ ಏಕತಾ ಭಾವನೆಯನ್ನುಂಟುಮಾಡುವುದೋ ಅದು ಸತ್ಯ. ಪ್ರೀತಿಯೇ ಸತ್ಯ, ದ್ವೇಷವೇ ಅನೃತ. ಏಕೆಂದರೆ ದ್ವೇಷವು ನಮ್ಮನ್ನು ಪರಸ್ಪರ ವಿಂಗಡಿಸುತ್ತದೆ; ಅನೈಕ್ಯಕ್ಕೆ ಕಾರಣವಾಗುತ್ತದೆ. ಮನುಷ್ಯ–ಮನುಷ್ಯರನ್ನು ಬೇರ್ಪಡಿಸುತ್ತದೆ. ಈ ದ್ವೇಷ. ಆದುದರಿಂದಲೇ ಅದು ದೋಷಯುಕ್ತ ಹಾಗೂ ಅನೃತ. ದ್ವೇಷ ನಮ್ಮನ್ನು ನಾಶಗೊಳಿಸುವ ಶಕ್ತಿ. ಅದು ನಮ್ಮನ್ನು ಬೇರ್ಪಡಿಸಿ ನಾಶಗೊಳಿಸುತ್ತದೆ. ಪ್ರೀತಿ ಒಂದುಗೂಡಿಸುತ್ತದೆ; ನೀವು ಒಂದಾಗುತ್ತೀರಿ. ತಾಯಿ, ಮಗು, ಕುಟುಂಬ, ನಗರಗಳು, ಇಡಿಯ ಜಗತ್ತು, ಪ್ರಾಣಿಗಳು ಪ್ರೀತಿಯಿಂದ ಒಂದಾಗುತ್ತವೆ. ಪ್ರೀತಿಯೇ ಜೀವಜೀವನದ ಸಾಮರಸ್ಯದ ಸೂತ್ರಧಾರ.


\section*{ಪ್ರೀತಿಯ ಪ್ರತಿಧ್ವನಿ}

\addsectiontoTOC{ಪ್ರೀತಿಯ ಪ್ರತಿಧ್ವನಿ}

ಇಲ್ಲಿದೆ ನಡೆದ ಒಂದು ಘಟನೆ. ತಾಯಿತಂದೆ ಭಾರತವಾಸಿಗಳು. ಒಬ್ಬನೇ ಮಗ ಅಮೇರಿಕದಲ್ಲಿ ಹೆಸರಾಂತ ಡಾಕ್ಟರಾಗಿ ಕೆಲಸ ಮಾಡುತ್ತಿದ್ದ. ಜನಾದರಣೆಯನ್ನೂ, ವಿಶ್ವಾಸವನ್ನೂ ಗಳಿಸಿದರೂ, ಬಾಳಸಂಗಾತಿಯೊಂದಿಗೆ ವೈಮನಸ್ಯ ಬೆಳೆದು, ಜೀವನದಲ್ಲಿ ಜುಗುಪ್ಸೆ ಬಂದು, ತನ್ನ ಎದೆಗೆ ತಾನೇ ಗುಂಡುಹಾರಿಸಿಕೊಂಡು ಆತ ಇಹಲೋಕವನ್ನು ತ್ಯಜಿಸಿದ. ಅದೇ ದಿನ ಸಹಸ್ರಾರು ಮೈಲು ದೂರದಲ್ಲಿದ್ದ ತಾಯಿಗೆ ಸ್ವಪ್ನ–ತನ್ನ ಮಗ ರಕ್ತದ ಮಡುವಿನಲ್ಲಿ ಸತ್ತುಬಿದ್ದಂತೆ. ಎಚ್ಚೆತ್ತ ಮೇಲೆ ವಿಪರೀತ ಹೊಟ್ಟೆಸಂಕಟ. ಮರುದಿನ ದೂರವಾಣಿಯ ಮೂಲಕ ಈ ದುರಂತದ ವಾರ್ತೆ ತಿಳಿದಾಗ ಹೆತ್ತಕರುಳನ್ನು ಮತ್ತಷ್ಟು ಹಿಂಡಿದಂತಾಯಿತು.

ತನ್ನ ಪ್ರೀತಿಗೆ ಪಾತ್ರನಾದ ವ್ಯಕ್ತಿಗೆ ಆಘಾತವಾದಾಗ ಹೆತ್ತಕರುಳಿಗೆ ಹೇಗೆ ನೋವು ಮುಟ್ಟುತ್ತದೆ ಎಂಬುದಕ್ಕೆ ಇದೊಂದೇ ಉದಾಹರಣೆಯಲ್ಲ. ಬೇಕಷ್ಟು ಉದಾಹರಣೆಗಳು ದೊರಕುತ್ತವೆ. ರಷ್ಯಾ ಈ ಬಗ್ಗೆ ಬೇಕಷ್ಟು ಸಂಶೋಧನೆ ನಡೆಸಿದೆ. ಪ್ರೀತಿಯ ಸ್ಪಂದನ ರೇಡಿಯೋ ಅಲೆಗಳಂತೆ ಚಲಿಸಿ, ಅದಕ್ಕೆ ಕಾತರಿಸುವ ಹೃದಯವನ್ನು ಮುಟ್ಟಬಲ್ಲದು, ತಟ್ಟಬಲ್ಲದು.


\section*{ವಿಜ್ಞಾನಿಗಳ ಅಜ್ಞಾನ?}

\addsectiontoTOC{ವಿಜ್ಞಾನಿಗಳ ಅಜ್ಞಾನ?}

ವಿಜ್ಞಾನಿಗಳು, ವಿದ್ಯುಚ್ಛಕ್ತಿ, ಅಣುಶಕ್ತಿ, ಉಗಿಶಕ್ತಿ, ಸೌರಶಕ್ತಿ–ಇವುಗಳನ್ನು ಕುರಿತು ಸಾಕಷ್ಟು\break ಸಂಶೋಧನೆಗಳನ್ನು ಕೈಗೊಂಡು ಜಗತ್ತಿನ ಮುಖವನ್ನು ಬದಲಿಸಿದ್ದಾರೆ. ಆದರೆ ನಿಃಸ್ವಾರ್ಥ ಪ್ರೇಮದ ಶಕ್ತಿಯನ್ನು ಕುರಿತು ಸಂಶೋಧನೆಗಳನ್ನು ಕೈಗೊಂಡು, ಅದರ ಉತ್ಪತ್ತಿ, ಶೇಖರಣೆ ಮತ್ತು ಸರಬರಾಜನ್ನು ಮಾಡುವಂತಾದರೆ ಜಗತ್ತು ನಂದನವನವಾಗುವುದು. ಈ ನಿಃಸ್ವಾರ್ಥ ಪ್ರೇಮದ ಶಕ್ತಿ–ಧಾರ್ಮಿಕ ಪ್ರವಚನಕಾರರ ಉಪದೇಶಕ್ಕೆ ವಿನಿಯೋಗವಾಗುವ ಸರಕೆಂದು ವೈಜ್ಞಾನಿಕ ಮನೋ\-ವೃತ್ತಿಯ ವಿಚಾರವಂತರು ಹಿಂದೆ ತಿಳಿದದ್ದುಂಟು. ವಿಜ್ಞಾನಿಗಳೂ, ರಾಜಕೀಯಸ್ಥರೂ ಈ ಬಗ್ಗೆ ಗಮನವನ್ನೇ ಕೊಟ್ಟಿರಲಿಲ್ಲ.\footnote{\engfoot{.... Moreover pre-war Science was much more interested in the study of criminals than of saints, of the insane than of genius, of the struggle for existence than of mutual aid and of hate and selfishness than of compassion and love.}\hfill\engfoot{ –Pitrim A. Sorokin}} ಎರಡು ಜಾಗತಿಕ ಮಹಾಯುದ್ಧಗಳಿಂದ ಜನರು ತತ್ತರಿಸಿದರು. ಎಲ್ಲೆಲ್ಲೂ ಶಾಂತಿಯ ಪರದೆಯನ್ನು ಹಾಕಿಕೊಂಡಿದ್ದರೂ ಒಳಗೊಳಗೆ ರಣ ಸನ್ನಾಹಕ್ಕೆ ಪ್ರಾಧಾನ್ಯ. ವಿಜ್ಞಾನ ನೀಡಿದ ನಾನಾ ತೆರನಾದ ಸುಖಸೌಕರ್ಯಗಳನ್ನೂ ಜನ ಪಡೆದಿದ್ದಾರೆ ನಿಜ. ಆದರೆ ಸ್ವಾರ್ಥತೆ, ಸ್ವೈರವೃತ್ತಿ, ಇಂದ್ರಿಯಪರಾಯಣತೆ, ದುರಾಸೆ, ಕೊರಳಿರಿ ಯುವ ಸ್ಪರ್ಧಾಮನೋ\-ಭಾವ–ಇವುಗಳಿಂದ ಸಮಾಜ ರೋಗಗ್ರಸ್ತವಾಗಿ ದುರಂತದತ್ತ ಧಾವಿಸುತ್ತಿದೆ. ಇದೀಗ ಮುಖ್ಯವಾಗಿ ಹಿಂಸಾರತಿ, ಯುದ್ಧ, ರಕ್ತಕ್ರಾಂತಿ, ಕೊಲೆಗಡುಕತನ ಇವುಗಳನ್ನು ದೂರಮಾಡಲೂ, ಜನರಲ್ಲಿ ಪರಸ್ಪರ ಸಹಕಾರ, ಸದ್ಭಾವನೆಗಳನ್ನು ಬೆಳೆಸಿ ಶಾಂತಿಯುತ ಸಹಬಾಳ್ವೆಯನ್ನುಂಟುಮಾಡಲೂ, ನಿಃಸ್ವಾರ್ಥಪ್ರೀತಿಯ ಶಕ್ತಿಯನ್ನು ಕುರಿತು ಸಂಶೋಧನೆಗಳನ್ನು ನಡೆಯಿಸಬೇಕೆಂದು ಸೊರೊಕಿನ್ನರ ಅಭಿಪ್ರಾಯ. ಮದ್ದುಗುಂಡುಗಳಿಗಾಗಿ ಖರ್ಚುಮಾಡುವ ಕೋಟಿಗಟ್ಟಲೆ ಡಾಲರ್​ಗಳಲ್ಲಿ ಒಂದಂಶ ಈ ನಿಃಸ್ವಾರ್ಥ ಪ್ರೀತಿಯ ಶಕ್ತಿ ಹಾಗೂ ಅದರ ಬಳಕೆಯ ವಿಧಾನದ ಸಂಶೋಧನೆಗಾಗಿ ವ್ಯಯಿಸಿದರೆ ಹೆಚ್ಚಿನ ಉಪಕಾರವಾಗುತ್ತದೆನ್ನುತ್ತಾರವರು. ನಿಜವಾಗಿಯೂ ಈ ನಿಃಸ್ವಾರ್ಥ ಪ್ರೀತಿಯ\break ಮಹಿಮೆಯನ್ನು ಸರಿಯಾಗಿ ತಿಳಿದುಕೊಂಡರೆ, ಪ್ರತಿಯೊಬ್ಬ ವ್ಯಕ್ತಿಯೂ ತನ್ನ ದೈನಂದಿನ ವ್ಯವಹಾರದಲ್ಲಿ ದ್ವೇಷ, ಅಸೂಯೆ, ಕ್ಷುದ್ರ ಸ್ವಾರ್ಥವನ್ನು ಕಡಿಮೆಮಾಡಿಕೊಳ್ಳಬಲ್ಲ. ತನಗೂ, ಇತರರಿಗೂ ದಿವ್ಯ ಆನಂದವನ್ನು ನೀಡಬಲ್ಲ ಈ ಪರಿಶುದ್ಧ ಪ್ರೀತಿಯನ್ನು ಪ್ರಾಕ್ಟೀಸ್ ಮಾಡಬಲ್ಲ. ಇದರಿಂದ ಕ್ರಮೇಣ ಜಗತ್ತು ಎದುರಿಸುವ ಹಾನಿಯನ್ನು ತಪ್ಪಿಸಬಹುದು.


\section*{ಪ್ರೀತಿಯ ಪವಾಡ}

\addsectiontoTOC{ಪ್ರೀತಿಯ  ಪವಾಡ}

ವಿಜ್ಞಾನಿಗಳು ಈ ಕ್ಷೇತ್ರದಲ್ಲಿ ಸಂಶೋಧನೆ ಮಾಡಲಿ, ಬಿಡಲಿ! ಅವರು ಸಂಶೋಧನೆಗಳನ್ನು ಮಾಡಿ ಪ್ರಕಟಿಸಬಹುದಾದ ತಥ್ಯಗಳನ್ನು ಕಾಯುತ್ತ ಕುಳಿತಿರಲು ನಮಗೆ ಸಾಧ್ಯವಿಲ್ಲ. ಮನೋ ದೈಹಿಕ ಬೇನೆಗಳನ್ನು ಕುರಿತ ಸಂಶೋಧನೆಗಳಿಂದ ಗೊತ್ತಾಗಿರುವ ತಥ್ಯಗಳು ಇವು: ದ್ವೇಷ ಕೆಟ್ಟದು, ಪ್ರೀತಿ ಒಳ್ಳೆಯದು. ದ್ವೇಷ, ಮತ್ಸರ, ದುರಭಿಮಾನ, ಚಿಂತೆ, ಸೇಡಿನ ಮನೋಭಾವ, ಅವಸಾದ ಶರೀರಾರೋಗ್ಯಕ್ಕೆ ತೀವ್ರ ಕೆಡುಕನ್ನುಂಟುಮಾಡುವ ವಿಷಗಳು. ಪ್ರೀತಿ ಶರೀರಾರೋಗ್ಯಕ್ಕೆ ಮಾತ್ರವಲ್ಲ, ಮಾನಸಿಕ, ನೈತಿಕ ಸುಸ್ಥಿತಿಗೂ ಬೇಕು. ಶಸ್ತ್ರಚಿಕಿತ್ಸೆಯಲ್ಲಿ ಅದ್ವಿತೀಯ ತಜ್ಞನೆನಿಸಿದ ಜಾನ್ ಹಂಟರ್ ಹೃದಯಬೇನೆಯಿಂದ ನರಳುತ್ತಿದ್ದ. ಅವನೊಮ್ಮೆ ಹೀಗೆಂದ: ‘ನನ್ನನ್ನು ಸಿಟ್ಟಿಗೇಳುವಂತೆ ಮಾಡುವವನ ಕೈಯಲ್ಲಿ ನನ್ನ ಪ್ರಾಣ ಇದೆ.’ ಎಂದರೆ, ತನ್ನನ್ನು ಸಿಟ್ಟಿಗೇಳುವಂತೆ ಮಾಡಿದರೆ ತಾನು ಹೃದಯಾಘಾತವಾಗಿ ಸಾಯುವುದು ಖಂಡಿತ ಎಂಬುದು ಆ ಮಾತಿನ ತಾತ್ಪರ್ಯ. ಹಾಗೆ ಸಿಟ್ಟಿಗೆದ್ದಾಗಲೇ ಅವನು ಸಾವನ್ನಪ್ಪಿದನಂತೆ! ತಾನು ಬೆಳೆಸಿಕೊಂಡ ದುರ್ಭಾವನೆಗಳು ತನ್ನ ನಾಶಕ್ಕೇ ಕಾರಣವಾಗುವುವು ಎಂದು ತಿಳಿದಾಗ ಮನುಷ್ಯ ಅವುಗಳಿಂದ ಬಿಡಿಸಿಕೊಳ್ಳಲು ಯತ್ನಿಸುತ್ತಾನಷ್ಟೆ! ಯಾವುದು ವೈಯಕ್ತಿಕವಾಗಿ, ನೈತಿಕವಾಗಿ, ಸಾಮಾಜಿಕವಾಗಿ, ಒಳ್ಳೆಯದನ್ನುಂಟುಮಾಡುವುದೋ, ಅದು ರಾಜಕೀಯವಾಗಿ ಕೆಟ್ಟದ್ದಾಗಲು ಸಾಧ್ಯವಿದೆಯೇ? ಇಂದು ನಮ್ಮ ರಾಜಕೀಯಸ್ಥರೂ, ವಿಚಾರವಾದಿಗಳೂ, ಕ್ರಾಂತಿಕಾರಿಗಳೆನಿಸಿಕೊಂಡು ಸಾಮಾಜಿಕ ಪ್ರಗತಿಯ ಬಗ್ಗೆ ಕಾಳಜಿ\break ತೋರ್ಪಡಿಸುವವರೂ ಪಕ್ಷಸ್ವಾರ್ಥದಿಂದ ಜನರಲ್ಲಿ ಸಾಧ್ಯವಿದ್ದಷ್ಟೂ ದ್ವೇಷವನ್ನು ಪ್ರಚಾರ ಮಾಡು\-ತ್ತಾರೆ. ಕ್ರಾಂತಿಯ ಹೆಸರಲ್ಲಿ ಕೆಡುಕಿನ ಬೀಜಗಳನ್ನು ಬಿತ್ತುತ್ತಾರೆ. ಪ್ರಜಾಪ್ರಭುತ್ವದಲ್ಲಿ ವಾಕ್ ಸ್ವಾತಂತ್ರ್ಯವಿದೆ ಎಂದು, ರಚನಾತ್ಮಕ ಟೀಕೆಯ ಮುಸುಕಿನಲ್ಲಿ, ದ್ವೇಷ ಭಾವನೆಯಿಂದ ತಮ್ಮ ಅಭಿಪ್ರಾಯಗಳನ್ನೊಪ್ಪದವರ ಚಾರಿತ್ರ್ಯಹನನ ಮಾಡುತ್ತಾರೆ. ಅನೃತ, ಕಲಹ, ಪರನಿಂದೆಯ ಪರಂಪರೆಯನ್ನೇ ಸೃಷ್ಟಿಸುತ್ತಾರೆ. ಮಾನವಕಲ್ಯಾಣದ ಮುಖವಾಡವನ್ನು ಧರಿಸಿ ಗೋಮುಖ\-ವ್ಯಾಘ್ರರಂತೆ ವರ್ತಿಸುತ್ತಾರೆ.

ಈ ವಿಷವರ್ತುಲದಿಂದ ಬಿಡುಗಡೆಯಾಗಲು ಬಾಲ್ಯದಿಂದಲೇ ಎಳೆಯರಲ್ಲಿ ಪ್ರೀತಿಯ\break ಮಹಿಮೆ, ಮಾಹಾತ್ಮ್ಯಗಳ ಬಗ್ಗೆ ದೃಢವಾದ ನಂಬಿಕೆಯನ್ನುಂಟು ಮಾಡಬೇಕು. ಬದುಕಿನ ವಿಭಿನ್ನ ಕ್ಷೇತ್ರಗಳಲ್ಲಿ ಪ್ರೀತಿಯ ಅಭಿವ್ಯಕ್ತಿಯ ಬಗೆಗೆ ತರಬೇತಿಯನ್ನು ನೀಡಬೇಕು.

\newpage

ವಿಭಿನ್ನ ಪಕ್ಷ, ಪಂಗಡಗಳೊಳಗೆ ಸಾಮರಸ್ಯವನ್ನುಂಟು ಮಾಡಲು ದೇಶ ಪ್ರೇಮಿಗಳು, ಪ್ರೀತಿ, ಸಾಮರಸ್ಯ ನಿರ್ಮಾಣಕ್ಕೆ, ಅಧಿಕಾರಕ್ಕೆ ಆಸೆಪಡೆದ ಪಕ್ಷವೊಂದನ್ನು ಸ್ಥಾಪಿಸಿದರೆ ಒಳಿತು. ನಿಜವಾಗಿಯೂ ಇದು ಧಾರ್ಮಿಕರು ಮಾಡಬೇಕಾದ ಕೆಲಸ. ಆದರೆ ಅಯ್ಯೋ! ಅವರೂ ಪರಸ್ಪರ ಜಗಳಾಡುತ್ತಿದ್ದಾರೆ!

ಶಿಕ್ಷಣತಜ್ಞ ಜಾಫ್ರೀ ಫರ್ಸ್ಟ್ ಅವರ ಮಾತನ್ನು ಗಮನಿಸಿ:

‘ಜೀವಶಾಸ್ತ್ರ, ಆರೋಗ್ಯರಕ್ಷಣೆ, ಲೈಂಗಿಕ ವಿಜ್ಞಾನ–ಇವುಗಳನ್ನು ಕುರಿತು ಪಾಠ ಹೇಳಿದ ಅನುಭವ ನನಗಿದೆ. ನಾನು ಏನು ಹೇಳುತ್ತಿದ್ದೇನೆ ಎಂಬುದು ನನಗೆ ತಿಳಿದಿದೆ. ಪಾಠಪಟ್ಟಿಯಲ್ಲಿ ಪರಿಶುದ್ಧ ಪ್ರೀತಿಯ ಮಹಿಮಾತಿಶಯಗಳನ್ನು ಸರಿಯಾಗಿ ತಿಳಿಸುವ ಯತ್ನಮಾಡದಿರುವುದು ನಮ್ಮ ವಿದ್ಯಾಭ್ಯಾಸ ಪದ್ಧತಿಯಲ್ಲಿ ಒಂದು ಮಹಾದೋಷವೆಂದು ನನಗನ್ನಿಸಿದೆ. ಪ್ರೀತಿ ಎಂದೊಡನೆ ಲೈಂಗಿಕತೆಯೊಂದಿಗೆ ಅದನ್ನು ಜೊತೆಗೂಡಿಸುವುದು ಸರ್ವಥಾ ಸರಿಯಲ್ಲ. ಹೆಣ್ಣುಗಂಡುಗಳ ದೈಹಿಕ ಆಕರ್ಷಣೆಗೇ ಪ್ರೀತಿಯ ಅಭಿವ್ಯಕ್ತಿಯನ್ನು ಸೀಮಿತಗೊಳಿಸುವುದನ್ನು ಬಿಟ್ಟು, ಪ್ರೀತಿಯ ಎಲ್ಲ ಮುಖಗಳ ಪರಿಚಯವನ್ನು ಶೈಶವದಿಂದಲೇ ಕಾಲೇಜು ಶಿಕ್ಷಣದವರೆಗೆ ಸರಿಯಾದ ಪಾಠ ಪಟ್ಟಿಗಳ ಮೂಲಕ ತಿಳಿಸಬೇಕಿತ್ತು. ಈ ಇನ್ನೂರು ವರ್ಷಗಳಲ್ಲಿ ಹಾಗೆ ತಿಳಿಸಿಕೊಟ್ಟಿದ್ದರೆ– ಇಂದು ಸಾಮಾಜಿಕ, ರಾಜಕೀಯ ಮತ್ತು ಜನಾಂಗಕ್ಕೆ ಸಂಬಂಧಿಸಿದ ಹಲವು ಗೊಂದಲದ ಸಮಸ್ಯೆಗಳು ಇಲ್ಲವಾಗುತ್ತಿದ್ದವು.’\footnote{\engfoot{Having taught classes in Biology, Health, Hygiene, and Sex Education, I feel that I know of what I speak. In my opinion, if all aspects of Love, aside from sexuality, had been served to our nation’s youth as applied courses from kindergarten through college over the past two hundred years, we would not be experiencing many of the problems we have in our social, political and racial relationships. \textit{–Edgar Cayee’s Story of Attitudes and Emotions,} Edited and arranged by Jeffrey Furst}}


\section*{ನಿಃಸ್ವಾರ್ಥತೆಯೇ ಒರೆಗಲ್ಲು}

\addsectiontoTOC{ನಿಃಸ್ವಾರ್ಥ\-ತೆಯೇ ಒರೆಗಲ್ಲು}

ಕರ್ಮಯೋಗವನ್ನು ಕುರಿತ ತಮ್ಮ ಒಂದು ಪ್ರವಚನದಲ್ಲಿ ಸ್ವಾಮಿ ವಿವೇಕಾನಂದರು ‘ನಿಃಸ್ವಾರ್ಥ\-ತೆಯೇ ದೇವರು’ ಎಂಬ ಮಾತನ್ನು ಹೇಳುತ್ತಾರೆ. ವರವಿತ್ತು ಸಲಹುವವನೂ, ನಮ್ಮ ದುಃಖ ದೂರ ಮಾಡುವವನೂ ದೇವರು ಎಂದುಕೊಳ್ಳುವ ನಮಗೆ ಮೇಲಿನ ಮಾತಿನ ಅರ್ಥ ಸ್ಪಷ್ಟವಾಗುವುದು ತುಸು ಕಷ್ಟವೇ. ನಿಃಸ್ವಾರ್ಥತೆಯ ಮಹಾಗುಣ ಮನುಷ್ಯನನ್ನು ದೇವತ್ವಕ್ಕೆ ಏರಿಸುವುದೆಂಬುದು ಅವರ ಮಾತಿನ ಅರ್ಥ. ದೈವತ್ವ ಎಂದರೆ ಸ್ವಯಂ ಪರಿಪೂರ್ಣ ಸ್ಥಿತಿ. ಈ ಸ್ಥಿತಿಯನ್ನೇರುವುದು ಜೀವಿತದ ಗುರಿ. ‘ತಾಯಿ ನಿನ್ನ ಪಾಲಿಗೆ ದೇವರಾಗಲಿ’ ಎಂಬ ವೈದಿಕ ಅನುಶಾಸನವನ್ನು ನಾವು ಕೇಳಿದ್ದೇವೆ. ತಾಯಿಯ ನಿಃಸ್ವಾರ್ಥತೆ ಯಾರಿಗೆ ತಾನೇ ತಿಳಿದಿಲ್ಲ? ತಾಯಿ ತನ್ನ ಸುಖವನ್ನು ಮರೆತು ಮಗುವಿನ ಭರಣೆ, ಪೋಷಣೆಗಳಲ್ಲಿ ಮೈಮರೆಯುತ್ತಾಳೆ. ಕೆಲವೊಮ್ಮೆ ತನ್ನ ಕಂದನನ್ನು ರಕ್ಷಿಸಲು ತನ್ನ ಪ್ರಾಣವನ್ನು ಕೊಡಲೂ ಹಿಂಜರಿಯಳು. ಪ್ರೀತಿಯ ಒಂದು ಲಕ್ಷಣವೆ ಪ್ರೀತಿಸಲ್ಪಡುವ ವ್ಯಕ್ತಿಯ ಸರ್ವತೋಮುಖ ಅಭ್ಯುದಯಕ್ಕಾಗಿ ನಿರಂತರ ಯತ್ನ ಶೀಲತೆ. ಇಂಥ ನಿಃಸ್ವಾರ್ಥ ಪ್ರೇಮದಿಂದುದಿಸಿದ ತ್ಯಾಗ, ಸೇವೆಗಳಿಂದಲ್ಲವೆ ಅಸಂಖ್ಯ ಜೀವಿಗಳು ತಮ್ಮ ಶೈಶವ, ಬಾಲ್ಯದ ಅಸಹಾಯಸ್ಥಿತಿಯಿಂದ ಪಾರಾಗಿ, ಮನುಷ್ಯರಾಗಲೂ, ತಮ್ಮ ಕಾಲ ಮೇಲೆ ತಾವು ನಿಲ್ಲಲೂ ಸಮರ್ಥರಾದುದು? ಮಾತ್ರವಲ್ಲ, ಇಂಥ ನಿಃಸ್ವಾರ್ಥ, ಪರಿಶುದ್ಧ ಪ್ರೀತಿಯ ಅಭಿವ್ಯಕ್ತಿಯಾದರೆ, ತಾಯಿಯ ತಾಯ್ತನ ಸಾರ್ಥಕ ಅಥವಾ ಕೃತಕೃತ್ಯ. ಮಗುವು ಆನಂದದಿಂದ ನಕ್ಕುನಲಿಯುವುದನ್ನು ಕಂಡು ಯಾವ ತಾಯಿ ಆನಂದಿಸುವುದಿಲ್ಲ? ಮಗು ಸದಾ ಆನಂದದಿಂದ ನಗುನಗುತ್ತ ಬೆಳೆಯಲಿ ಎಂಬ ಹಾರೈಕೆ ಯಾವ ತಾಯಿಯ ಹೃದಯದಲ್ಲಿರುವುದಿಲ್ಲ? ತಾಯಿಯ ಪ್ರೀತಿ, ತಾಯಿ–ಮಗು ಇಬ್ಬರಿಗೂ ಆನಂದವನ್ನುಂಟುಮಾಡುವುದು. ತಾಯಿಯ ಪ್ರೀತಿ ಮಗುವಿನಲ್ಲಿ ರಕ್ಷಣೆಯ ಭಾವವನ್ನು ಉಂಟುಮಾಡಿ ಸುಖ ಕೊಡುತ್ತದೆ. ಮಗುವು ಸುರಕ್ಷಿತವಾಗಿರುವುದನ್ನು ಕಂಡು ತಾಯಿಗೂ ಹಿತವಾಗಿ ತಾನೂ ಸುರಕ್ಷಿತ ಭಾವನೆಯನ್ನು ಅನುಭವಿಸುತ್ತಾಳೆ. ತಾಯಿ ಮಗುವನ್ನು ಪ್ರೀತಿಸಲಾರಳಾದರೆ, ಅವಳಲ್ಲಿ ಸಂತೃಪ್ತಿಯ ಧನ್ಯತೆಯ ಭಾವ ಮೂಡಲಾರದು. ಮಗುವಾದರೋ ತಾಯಿಯ ಪ್ರೀತಿಯನ್ನು ಕಂಡರಿಯದಿದ್ದರೆ, ಆತ್ಮವಿಶ್ವಾಸವನ್ನಾಗಲೀ, ಇತರರಲ್ಲಿ ವಿಶ್ವಾಸವನ್ನು ಇಡುವುದನ್ನಾಗಲೀ ಕಲಿಯಲಾರದು. ತಾಯಿಯ ನಿಜವಾದ ಪ್ರೀತಿಯನ್ನು ಪಡೆದ ಮಗು ಕೃತಜ್ಞತೆಯ ಮಹಾಗುಣದೊಂದಿಗೆ, ತಾಯಿಯ ಪ್ರೇಮದಿಂದ ದೂರವಾಗದ ಆಪ್ತತೆ ಅಥವಾ ಆತ್ಮೀಯ ಭಾವನೆಯನ್ನು ಪಡೆಯುವುದು. ಮಾತೃತ್ವದ ಆದರ್ಶವನ್ನು ಮೈಗೂಡಿಸಿಕೊಂಡ ತಾಯಿಯಲ್ಲಿ ಈ ಪ್ರೀತಿ ವಿಶೇಷ ರೀತಿಯಿಂದ ವ್ಯಕ್ತವಾಗುವುದನ್ನು ಕಾಣುತ್ತೇವೆ. ಸಾಮಾನ್ಯವಾಗಿ ಈ ಪರಿಶುದ್ಧ, ನಿಃಸ್ವಾರ್ಥಪ್ರೇಮ ಪರಿಪೂರ್ಣ ರೀತಿಯಲ್ಲಿ ಅಭಿವ್ಯಕ್ತವಾಗದಿದ್ದರೂ, ತಾಯಿಯನ್ನು ಗೌರವಿಸು. ಅವಳು ತೋರಿದ ಪ್ರೀತಿ, ವಾತ್ಸಲ್ಯ, ಕರುಣೆಗಳನ್ನೂ, ಮಾಡಿದ ಸೇವೆಯನ್ನೂ ಮರೆಯಬೇಡ. ಕೃತಘ್ನನಾಗಬೇಡ. ಹಾಗೆ ಗೌರವಿಸಿದರೆ ಆ ಸದ್ಗುಣಗಳು ನಿನ್ನಲ್ಲಿಯೂ ಮೂಡುತ್ತವೆ. ನಿನ್ನ ಬದುಕಿಗೂ ಸಂತೃಪ್ತಿಯನ್ನು ನೀಡುವುವು. ಆದುದರಿಂದಲೇ ‘ಮಾತೃದೇವೋಭವ’ ಎಂದರು ಋಷಿಗಳು. ತಾಯಿ ಸ್ವಾರ್ಥಿಯಾದರೆ ಎಂಥ ಭೀಕರ ಪರಿಣಾಮವಾದೀತು ಎಂಬುದನ್ನು ಕಲ್ಪಿಸಿಕೊಳ್ಳಬಹುದು. ಪಶ್ಚಿಮ ದೇಶಗಳಲ್ಲಿ (ಇಂಥವರು ಎಲ್ಲ ಕಡೆಗಳಲ್ಲೂ ಇದ್ದಾರೆ!) ತಮಗೆ ಮಕ್ಕಳು ಬೇಡ, ತಮ್ಮ ಸೌಂದರ್ಯ ನಾಶವಾಗಿಬಿಡುತ್ತದೆ, ಹೆರಿಗೆಯ ಕಷ್ಟ, ಗೋಳು ಬೇಡ–ಎಂದು ಹಲವರು ಮಕ್ಕಳನ್ನು ಹೆತ್ತು, ಲಾಲಿಸಿ, ಪಾಲಿಸುವ ಗೋಜಿಗೆ ಹೋಗುವುದಿಲ್ಲ. ವಿವಾಹಿತರಾದ ಅವರು ಗರ್ಭಿಣಿಯರಾದಾಗ ದುಃಖಿತರೇ ಆಗುತ್ತಾರೆ. ಮಕ್ಕಳನ್ನು ಪಡೆದ ಬಳಿಕ ಅವುಗಳ ಕ್ರಂದನ, ಕಾಟ, ತುಂಟತನಗಳನ್ನು ಸಹಿಸಲಾರದೇ ಕುಪಿತರಾಗಿ ತೀವ್ರತರದ ದೈಹಿಕ ತೊಂದರೆಯನ್ನುಂಟು ಮಾಡುವ ಶಿಕ್ಷೆಯನ್ನು ವಿಧಿಸುತ್ತಾರೆಂದು ಅಧಿಕೃತ ವರದಿ ತಿಳಿಸುತ್ತದೆ.\footnote{\engfoot{Many women neither love nor take care of their sons and daughters–in fact, as we will see, women by thousands each year desert, mutilate or even kill their children. Fathers are often worse. –Holt, Rinchant and Winston, Understanding Human Behaviour}} ಮಕ್ಕಳು ಬದುಕಿನ ಉದ್ದಕ್ಕೂ ದೈಹಿಕ, ಮಾನಸಿಕ ಯಾತನೆಯನ್ನು ಇದರಿಂದ ಅನುಭವಿಸಬೇಕಾಗುವುದು. ಇಲ್ಲಿ ತಾಯಿ ಸ್ವಾರ್ಥಿಯಾದರೆ, ತನ್ನ ತಾಯ್ತನದ ಆದರ್ಶದಿಂದ ದೂರಸರಿದರೆ, ಅವಳ ಪಾಲಿಗೂ, ಮಕ್ಕಳ ಪಾಲಿಗೂ, ತನ್ಮೂಲಕ ಸಮಾಜದ ಪಾಲಿಗೂ, ಎಂಥ ದುರಂತ ಕಾದಿರುತ್ತದೆ ಎಂಬುದನ್ನು ಯೋಚಿಸಬಹುದು.

ಪ್ರೀತಿ ಎಷ್ಟರಮಟ್ಟಿಗೆ ಪರಿಶುದ್ಧವಾಗಿದೆ ಎಂಬುದನ್ನು ತಿಳಿಯಲು ನಿಃಸ್ವಾರ್ಥತೆಯೇ ಒರೆಗಲ್ಲು. ನಿಸರ್ಗ ಇದನ್ನು ತಾಯಿಯ ಮೂಲಕ ಕಲಿಸಿಕೊಡುತ್ತದೆ. ತಾಯ್ತನದ ಶ್ರೇಷ್ಠ ಜವಾಬ್ದಾರಿಯನ್ನು ನಿಸರ್ಗ ಮಹಿಳೆಗೇ ವಹಿಸಿಕೊಟ್ಟಿದೆ. ಆದರೆ ಆಧುನಿಕ ವಿದ್ಯಾಭ್ಯಾಸ ಮತ್ತು ಯಂತ್ರ ಯುಗದ ಸೌಲಭ್ಯಗಳನ್ನು ಪಡೆದು ಸ್ವಾತಂತ್ರ್ಯ ಎಂದರೆ ಸ್ವೇಚ್ಛಾಚಾರ ಎನ್ನುವಂತೆ ವರ್ತಿಸುವ ಮನೋವೃತ್ತಿ ಇಂದು ಮುಂದುವರಿದ ದೇಶಗಳಲ್ಲಿ ಬೆಳೆಯುತ್ತಿದೆ. ಸ್ತ್ರೀಸ್ವಾತಂತ್ರ್ಯ–ಇದು ಕ್ರಮೇಣ ತಾಯ್ತನದ ಆದರ್ಶಕ್ಕೆ ಮಾರಕವಾಗುತ್ತಲಿದೆ. ಸ್ತ್ರೀ ಪುರುಷರ ಸಮಾನತೆಯನ್ನು ಸಾರುವ ಪೊಳ್ಳು ಸುಧಾರಣಾವಾದಿಗಳು ಧಾರ್ಮಿಕ ಹಿನ್ನೆಲೆಯಿಂದ ರಚಿಸಿದ ನಿಯಮಗಳಲ್ಲಿರುವ ಯಾವ ಒಳಿತನ್ನೂ ಕಾಣಲಾರದಾಗಿದ್ದಾರೆ! ಮನುಷ್ಯರನ್ನು ಮೆಟ್ಟಲು ಮೆಟ್ಟಲಾಗಿ ನಿಃಸ್ವಾರ್ಥತೆಯ ಮೂಲಕ ದಿವ್ಯಾ\-ನಂದದ ಸ್ಥಿತಿಗೆ ಕೊಂಡೊಯ್ಯುವುದೇ ಎಲ್ಲ ಧಾರ್ಮಿಕ ನಿಯಮಗಳ ಉದ್ದೇಶ. ಈ ಆದರ್ಶವನ್ನು ಸರಿಯಾಗಿ ತಿಳಿದುಕೊಳ್ಳದೇ ಅಲ್ಲಗಳೆಯುವುದರಿಂದ ಸಮಾಜವನ್ನು ಬೇಗನೇ ದುಃಸ್ಥಿತಿಗೆ ಕೊಂಡೊಯ್ಯಲು ಸಾಧ್ಯವಾಗುವುದಷ್ಟೆ.


\section*{ತಾಯಿಯ ಮಡಿಲು}

\addsectiontoTOC{ತಾಯಿಯ ಮಡಿಲು}

ತಮ್ಮ ತಾಯಿಯನ್ನು ಸ್ಮರಿಸಿಕೊಂಡು ಇಳಿವಯಸ್ಸಿನ ಹಿರಿಯ ಆಫೀಸರ್​ ಒಬ್ಬರು ಹೀಗೆಂದರು:‘ನನ್ನ ತಾಯಿ, ನನ್ನ ಹಾಗೂ ನನ್ನ ಸೋದರ ಸೋದರಿಯರ ಮತ್ತು ಬಂಧುಬಳಗದವರೆಲ್ಲರ ಪಾಲಿಗೆ ಭಗವಂತನ ಪ್ರೇಮದ ಪ್ರತೀಕವೇ ಆಗಿದ್ದರೆಂದು ಅವರನ್ನು ಕುರಿತು ಚಿಂತಿಸಿದಾಗಲೆಲ್ಲ ನನಗನ್ನಿಸು\-ತ್ತಿದೆ. ಅವರ ಹೆಸರಾದರೋ ಕೃಪಾ. ಅವರು ಭಗವಂತನ ಕೃಪೆಯ ಮೂರ್ತಿವತ್ತಾದ ರೂಪ ಎಂದೇ ನಾನು ನಂಬುವಂತಾಗಿತ್ತು. ಎಂದೆಂದೂ ಕುಂದದ, ಬತ್ತಿ ಬರಿದಾಗದ ಪ್ರೀತಿಯ ಸ್ವಭಾವ, ಕಷ್ಟ ಕಂಟಕಗಳನ್ನು ಎದುರಿಸುವ ದೃಢಮನಸ್ಸು, ಒಳ್ಳೆಯದಾಗಿಯೇ ತೀರುವುದೆಂಬ ಭರವಸೆ, ತಪ್ಪುಗಳನ್ನು ತಿದ್ದುವಲ್ಲಿ ಅಪಾರ ತಾಳ್ಮೆ, ಹೊಗಳಿಕೆಗೆ ಬಾಯ್ಬಿಡದ ಮನೋವೃತ್ತಿ, ಭಗವಂತನಲ್ಲಿ ಮತ್ತು ಆತನ ಮಾರ್ಗದರ್ಶನದಲ್ಲಿ ಎಂದೆಂದೂ ಅಲುಗಾಡದ ಶ್ರದ್ಧೆ, ಯಾರಿಂದಲೂ ಏನನ್ನೂ ಅಪೇಕ್ಷಿಸದ ನಿತ್ಯ ಆಂತರಿಕ ತೃಪ್ತಿಯ ನಡೆ–ಇವು ನನ್ನ ಮನಸ್ಸಿನಲ್ಲಿ ಮೂಡಿಸಿದ ಪ್ರಭಾವ ಅಳಿಸಲಾಗದ್ದು. ಆ ಸದ್ಗುಣಗಳ ಸುಗಂಧ ಅಲೆಅಲೆಯಾಗಿ ಬಂದು ನನ್ನನ್ನು ಮೈಮರೆಯುವಂತೆ ಮಾಡುತ್ತದೆ. ತೀರಿಕೊಳ್ಳುವುದಕ್ಕೆ ಒಂದೆರಡು ವರ್ಷಗಳ ಮೊದಲಂತೂ ಅವರ ನಿರ್ಲಿಪ್ತತೆ, ಎಲ್ಲರೆಡೆಗಿನ ಪ್ರೀತಿ, ತಪ್ಪು ತಡೆಗಳಿಲ್ಲದ ಅಂತರ್ದೃಷ್ಟಿ–ಇವು ಅವರನ್ನು ಒಬ್ಬ ತಪಸ್ವಿನಿಯ ಮಟ್ಟಕ್ಕೆ ಏರಿಸಿದ್ದವು. ಅವರು ನಮ್ಮೊಂದಿಗಿದ್ದಾರೆ ಎಂಬ ಯೋಚನೆಯೇ ಮನಸ್ಸಿಗೆ ನೆಮ್ಮದಿಯನ್ನುಂಟು\-ಮಾಡುತ್ತಿತ್ತು. ಉದ್ಯೋಗಕ್ಕಾಗಿ ಬೇರೆ ಬೇರೆ ಊರುಗಳಲ್ಲಿದ್ದ ನಾವು, ವರ್ಷಕ್ಕೊಂದು ಬಾರಿ ತಾಯಿಯ ಸಾನ್ನಿಧ್ಯದಲ್ಲಿ ಕಾಲಕಳೆಯಲು ಹಾತೊರೆಯುತ್ತಿದ್ದೆವು. ಯೌವನದಲ್ಲಿ ಅವರನ್ನು ಪ್ರೀತಿ, ಸಲಿಗೆಗಳಿಂದ ನೋಡುತ್ತಿದ್ದರೂ, ಇಂದು ಗೌರವಮಿಶ್ರಿತ ಭಕ್ತಿ ನನ್ನ ಹೃದಯದಲ್ಲಿ ತುಂಬಿಬರುತ್ತಲಿದೆ–ಅಯ್ಯೋ, ಅವರಿದ್ದಿದ್ದರೆ! ಅವರನ್ನು ಕಳೆದುಕೊಂಡೆವಲ್ಲ! ಎನ್ನಿಸುತ್ತಿದೆ.’

ಇಡಿಯ ಕುಟುಂಬ ಇಂಥ ತಾಯಿಯ ಚಾರಿತ್ರ್ಯದಿಂದ ಪ್ರಭಾವಿತವಾಗದಿರಲು ಸಾಧ್ಯ\-ವಿದೆಯೇ?


\section*{ಪ್ರೀತಿಯಲ್ಲಿದೆ ಸಂಸಾರದ ಸಾರ}

\addsectiontoTOC{ಪ್ರೀತಿ\-ಯಲ್ಲಿದೆ ಸಂಸಾರದ ಸಾರ}

ಅದೊಂದು ಸಂಸಾರ. ಶ‍್ರೀಮಂತಿಕೆ. ದೊಡ್ಡಮನೆ. ಮನೆ ತುಂಬ ಜನ. ಅತ್ಯಾಧುನಿಕ ಸುಖ ಸೌಕರ್ಯಗಳಲ್ಲಿ ಯಾವುದಕ್ಕೂ ಅಲ್ಲಿ ಕೊರತೆಯಿಲ್ಲ. ಆದರೆ ಅವರ ಮನಗಳು ಒಂದಾಗಿಲ್ಲ. ಅಶಾಂತಿ, ಅಸಮಾಧಾನಗಳ ಬಿರುಕು ಬಾಯ್ಬಿಟ್ಟಿದೆ. ಅದನ್ನು ಬೆಸೆಯುವ ಪ್ರೀತಿಗೇ ಅಲ್ಲಿ ಬರಗಾಲ!

ಯಜಮಾನನಿಗೆ ಮಾತ್ರ ಅಲ್ಲಿ ಬೇಗ ಬೆಳಗಾಗುವುದು. ನಿತ್ಯಕರ್ಮಗಳನ್ನು ಪೂರೈಸಿ, ಕಾಫಿ ಮಾಡಿಕೊಡೆಂದು ಕಾಯಿಲೆ ಬಿದ್ದ ಹೆಂಡತಿಯನ್ನು ಆತ ಎಬ್ಬಿಸುವುದು, ಎಷ್ಟೋಬಾರಿ ವಿಫಲನಾಗಿ ತನ್ನ ಪ್ರಾರಬ್ಧವನ್ನು ಹಳಿಯುತ್ತಾ ತಾನೇ ಕಾಫಿ ಮಾಡಿ ಕುಡಿಯುವುದು, ಗೊಣಗುತ್ತಲೇ ಸ್ನಾನ, ಪೂಜೆ ಮಾಡಿ ಮುಗಿಸಿದರೂ ಇನ್ನೂ ಎಚ್ಚರಗೊಳ್ಳದ ಮಕ್ಕಳನ್ನು ಬಯ್ದು ಭಂಗಿಸಿ ಬಡಿದೆಬ್ಬಿಸುವುದು, ನಿದ್ರಾಸುಖದಿಂದ ವಂಚಿತರಾದ ಆ ಮಕ್ಕಳು ಸಿಡುಕಿನಿಂದ ಕೂಗಿ ರೇಗಾಡುವುದು– ಇವೆಲ್ಲ ಅಲ್ಲಿ ನಿತ್ಯವಿಧಿ, ದಿನದ ಗೊಂದಲ ಗಲಿಬಿಲಿಗೆ ಅದೇ ಪೀಠಿಕೆ.

‘ಕೆಲಸಕ್ಕೆ ಹೋಗಬೇಕಲ್ಲಾ! ಕೆಲಸ ಮಾಡಬೇಕಲ್ಲಾ!’ ಎಂಬ ಜಡರಾಗ ಎಳೆಯುತ್ತಾ ತಮ್ಮ ಕೆಲಸ ಕಾರ್ಯಗಳಿಗೆ ತೆರಳುವ ಅವರದು ಕಣ್ಣಿಗೆ ಮಣ್ಣೆರಚುವ ಕೆಲಸ. ಶಾಲೆಗೆ ಹೋಗುವ ಮಕ್ಕಳೂ ಇದಕ್ಕೆ ಹೊರತಲ್ಲ. ಆ ಮಕ್ಕಳ ಹೆಚ್ಚುಗಾರಿಕೆ ವೇಷಭೂಷಣಗಳಲ್ಲೇ ಹೊರತು ನಡೆ ನುಡಿಗಳಲ್ಲಲ್ಲ. ಬೇಕಷ್ಟು ಐಶ್ವರ್ಯ ಇದ್ದರೂ ಜಿಪುಣತನದ ಕಟ್ಟಿನಿಂದಾಗಿ, ಸುಖಿಸುವ ಭಾಗ್ಯ ಅವರಿಗೆ ಇಲ್ಲ. ಹೊಟ್ಟೆಬಟ್ಟೆ ಕಟ್ಟಿಯಾದರೂ ದುಡ್ಡುಮಾಡುವ ಧೋರಣೆ ಯಜಮಾನನದ್ದಾದರೆ ಕದ್ದುಮುಚ್ಚಿಯಾದರೂ ಲಪಟಾಯಿಸಿ ದಿಲ್​ಖುಶ್ ಆಗಿರುವ ಜಾಯಮಾನ ಮಕ್ಕಳದ್ದು. ವ್ಯವಹಾರಸ್ಥ ತಂದೆ, ಗಿರಾಕಿಗಳೊಡನೆ ನಗುತ್ತಾ ಮಾತನಾಡಿದಂತೆ, ಪ್ರೀತಿಯಿಂದ ಒಮ್ಮೆಯೂ ಮಕ್ಕಳೊಡನೆ ಮಾತನಾಡಿದ್ದಿಲ್ಲ. ಮಕ್ಕಳನ್ನು ಬಯ್ಯುವುದೇ ರೂಢಿ ಆಗಿರುವ ಅವನೊಡನೆ ಮಕ್ಕಳೂ ಮನಬಿಚ್ಚಿ ಮಾತಾಡಿದ್ದಿಲ್ಲ; ಒಮ್ಮನಸ್ಸಿನಿಂದ ನಡೆದದ್ದಿಲ್ಲ. ಪರಸ್ಪರ ಹೊಂದಾಣಿಕೆಯ ಕೊಂಡಿಯೇ ಅಲ್ಲಿ ಕಳಚಿಹೋಗಿದೆ. ತಾನು, ತನ್ನದು, ತನ್ನಿಂದಲೇ ಎಂಬುದೇ ಎಲ್ಲರ ಮಂತ್ರವಾಗಿ ಬಿಟ್ಟಿದೆ. ಅದಕ್ಕಾಗಿಯೇ ಎಲ್ಲರ ಹೋರಾಟ, ಹಾಗಾಗಿ ಯಾವಾಗಲೂ ಅಲ್ಲಿ ಗೌಜು, ಗಲಾಟೆ, ಅವಿಶ್ವಾಸ, ದ್ವೇಷ, ಅಸೂಯೆಗಳಿಂದಾಗಿ ಮನೆಮಂದಿಯಲ್ಲೇ ಒಳಜಗಳ ಬೇರೆ. ಈ ಒಳತೋಟಿಯ ಬಿರುಸಿಗೆ ಆ ಸಂಸಾರದ ಸುಖ ಶಾಂತಿ ಕೊಚ್ಚಿಹೋಗಿದೆ.

ಸಂಸಾರದ ಆ ನೆಮ್ಮದಿಯನ್ನು ಅಲ್ಲಿರುವ ಸುಖಜೀವನದ ಸಾಧನ ಸೌಕರ್ಯಗಳಾವುವೂ ಉತ್ಪಾದಿಸಲಾರವು. ಆ ನೆಮ್ಮದಿಯಾದರೂ ಪರಸ್ಪರ ಪ್ರೀತಿಯ ಪಕ್ವಫಲ. ಅದೇ ಇಲ್ಲವಾದರೆ ಸಂಸಾರವೇ ನಿಸ್ಸಾರ. ಶರೀರವೇನೋ ಸುಂದರ ಸಾಲಂಕೃತ; ಆದರೆ ಜೀವವೇ ಇಲ್ಲ ಎಂಬಂತೆಯೇ ಅದು!

ಇದು ಇನ್ನೊಂದು ಸಂಸಾರ. ಬಡತನ ಎದ್ದುಕಾಣುತ್ತಿದೆ. ಹರುಕು ಮುರುಕು ಮನೆ. ಹೊಲಿದು ತೇಪೆ ಹಾಕಿದ ಬಟ್ಟೆಬರೆಗಳು. ಆದರೆ ಎಲ್ಲೆಡೆಯೂ ಅಚ್ಚುಕಟ್ಟುತನ, ಚೊಕ್ಕತನ. ಒಂದು ದಿನ ಮೈ ಮುರಿದು ದುಡಿಯದಿದ್ದರೆ ಎಲ್ಲರೂ ಉಪವಾಸ ಬೀಳುವ ಪರಿಸ್ಥಿತಿ. ಅಷ್ಟಾದರೂ ಅಲ್ಲಿ ಅಶಾಂತಿ, ಅಸಮಾಧಾನಗಳ ಸುಳಿವಿಲ್ಲ. ಎಲ್ಲರ ಮುಖದಲ್ಲೂ ನಗುವಿದೆ, ಹೊಂದಿ ಬಾಳುವ ಛಲವಿದೆ. ಪ್ರೀತಿ ಅವರನ್ನೆಲ್ಲ ಒಂದಾಗಿಸಿದೆ.

ಇಲ್ಲಿ ತಂದೆಗೆ ರೋಜಿನ ಅಡಿಗೆಯ ಕೆಲಸ. ರುಬ್ಬುವುದೇ ಮೊದಲಾದ ಕೆಲಸಗಳಿಗೆ ತಾಯಿಯ ಸಹಕಾರ. ಬೆಳ್ಳಂಬೆಳಗಾಗುವಾಗಲೆ ಅವರು ಮನೆಗೆಲಸ ಪೂರೈಸಿ ತಯಾರಾಗಿ ಹೊರಡಬೇಕು. ನಸುಕಿಗೆ ಏಳುವ ಮಕ್ಕಳ ಒಮ್ಮನಸ್ಸಿನ ದುಡಿಮೆಯಿಂದಾಗಿ ಅವರೆಂದೂ ತಡವಾದದ್ದಿಲ್ಲ. ಮನೆ ಎದುರು ಗುಡಿಸಿ ಸಾರಿಸಿ ರಂಗೋಲಿ ಹಾಕುವುದು, ಮನೆಗೆಲಸಕ್ಕೂ ಹೂ ಗಿಡಗಳಿಗೂ ದೂರದ ಬಾವಿಯಿಂದ ನೀರು ಸೇದಿ ತರುವುದು, ದನಕರುಗಳ ಚಾಕರಿ ಮಾಡುವುದು ಮೊದಲಾದ ಎಲ್ಲ ಕೆಲಸಗಳನ್ನೂ ಮಕ್ಕಳೇ ಹಂಚಿಕೊಂಡು, ಸಂತೋಷದಿಂದ ಮಾಡಿ ಮುಗಿಸುತ್ತಾರೆ. ದೇವರ ಪೂಜೆಯ ವೇಳೆ ಎಲ್ಲರೂ ಒಂದಾಗಿ ಭಕ್ತಿಯಿಂದ ಬಾಗಿ ಬೇಡುತ್ತಾರೆ. ತಿಂಡಿ ತಿನ್ನುವುದು ಆಮೇಲೆ ಒಟ್ಟಾಗಿ. ಹೆಚ್ಚೇನಲ್ಲ–ಒಂದೆರಡು ಒಣರೊಟ್ಟಿ.

ತಂದೆ ತಾಯಿಗಳು ಹೊರಗೆ ಕಷ್ಟಪಟ್ಟು ದುಡಿಯುವಾಗ ಮನೆಯ ನಿರ್ವಹಣೆ ಮಕ್ಕಳದೇ. ಮನೆಯಲ್ಲಿ ದೊಡ್ಡ ಮಕ್ಕಳು ಚಿಕ್ಕ ಮಕ್ಕಳಿಗೆ ಊಟೋಪಚಾರ ಮಾಡಿಸುತ್ತಾರೆ. ತೇಪೆ ಹಾಕಿದ್ದಾದರೂ ಚೊಕ್ಕ ಬಟ್ಟೆಗಳನ್ನೇ ಮಕ್ಕಳಿಗೆ ತೊಡಿಸಿ ಶಾಲೆಗೆ ಕಳುಹಿಸುತ್ತಾರೆ. ಆ ಮಕ್ಕಳ ವೇಷಭೂಷಣಗಳಲ್ಲಿ ಬಡತನವಿದ್ದರೂ ಗುಣನಡತೆಗಳಲ್ಲಿಲ್ಲ. ಶ‍್ರೀಮಂತರ ಮನೆಯ ಮಕ್ಕಳ ಷೋಕಿ ಜೀವನಕ್ಕೆ ಬಾಯಿ ನೀರು ಬಿಡುವ ಸ್ವಭಾವ ಅವರದ್ದಲ್ಲ. ಆ ಮಕ್ಕಳಿಗೂ ಇದ್ದುದರಲ್ಲೇ ತೃಪ್ತಿ.

ದಿನವಿಡೀ ದುಡಿದು ದಣಿದು ಸಂಜೆಗೆ ತಂದೆತಾಯಂದಿರು ಮನೆಗೆ ಬರುವಾಗ, ಆ ಮಕ್ಕಳ ನಗುಮುಖದ ಸ್ವಾಗತ ಕಾದಿರುತ್ತದೆ. ಆಮೇಲೆ ಅಂದಿನ ಸುದ್ದಿ ಸಮಾಚಾರ, ಖರ್ಚುವೆಚ್ಚ, ಕಷ್ಟ ನಷ್ಟಗಳ ಬಗ್ಗೆ ಲೋಕಾಭಿರಾಮವಾಗಿ ಮಾತನಾಡಿ, ರಾತ್ರಿ ಎಲ್ಲರೂ ಒಟ್ಟಾಗಿ ಕುಳಿತು ಇದ್ದುದನ್ನು ಹಂಚಿಕೊಂಡು ಸಂತೋಷದಿಂದ ಊಟ ಮಾಡುತ್ತಾರೆ. ಆ ಸಾದಾ ಊಟಕ್ಕೂ ಮೃಷ್ಟಾನ್ನದ ಸವಿ ಬರುವುದು ಒಗ್ಗಟ್ಟಿನಿಂದಲೇ.

ಬಡತನವಿದ್ದರೂ ಅವರಲ್ಲಿ ಜಿಪುಣತನವಿಲ್ಲ. ಹಬ್ಬಹರಿದಿನಗಳಲ್ಲೂ, ಧರ್ಮಕಾರ್ಯಗಳಲ್ಲೂ ಉದಾರಮನಸ್ಸಿನಿಂದ ಉತ್ತಮವಾದ ವಸ್ತುಗಳನ್ನೇ ಅವರು ದಾನವಾಗಿ ಕೊಡುತ್ತಾರೆ. ಇನ್ನೊಬ್ಬರು ಸಂತೋಷಪಡುವುದನ್ನು ನೋಡುವುದರಿಂದಲೇ ಅವರಿಗೆ ಸಂತೋಷ. ಅವರಲ್ಲಿ ಚಿನ್ನದ ಒಡವೆ\-ಗಳಿಲ್ಲ, ಆಸ್ತಿಪಾಸ್ತಿಗಳಿಲ್ಲ. ಅವರಲ್ಲಿರುವುದೆಲ್ಲ ಸದ್ಭಾವನೆ ಬೀರುವ ಮನಸ್ಸು, ಪರರ ಕಷ್ಟಕ್ಕೆ ಮಿಡಿಯುವ ಹೃದಯ. ಅವುಗಳನ್ನೆಣಿಸದೆ ಚಿನ್ನದ ಬಗ್ಗೆ, ಅಸ್ತಿಪಾಸ್ತಿಗಳ ಬಗ್ಗೆ ಅವರೊಡನೆ ವಿಚಾರಿಸುವವರಿಗೆ ಸಿಗುವ ಉತ್ತರ ಒಂದೇ–“ನೋಡುವವರ ಕಣ್ಣಿಗೆ ನಾವು ಬಡವರು. ಆದರೆ ನಾವು ನಿಜವಾಗಿಯೂ ಶ‍್ರೀಮಂತರೇ. ಕಳ್ಳರ ಕಣ್ಣಿಗೆ ಬೀಳದಿರಲೆಂದು ನಾವು ಚಿನ್ನದ ಒಡವೆಗಳನ್ನು ಹಾಕಿಕೊಳ್ಳುವುದಿಲ್ಲ ಅಷ್ಟೇ! ಸರ್ಕಾರದವರು ಟ್ಯಾಕ್ಸ್ ಹೇರಿ ಪೀಡಿಸುತ್ತಾರೆಂದು ಹೊಲ, ಮನೆ ಕೊಂಡಿಲ್ಲ ಅಷ್ಟೇ!” ಎಂದು.

ಅವರ ಶ‍್ರೀಮಂತಿಕೆ ಜನರ ಕಣ್ಣಿಗೆ ಬೀಳುವಂಥಾದ್ದಲ್ಲ ನಿಜ. ಅದು ಹೃದಯದ ಸಂಪತ್ತು– ಜೀವಜೀವನದ ಬಗ್ಗೆ ಉಕ್ಕಿಹರಿಯುವ ಪರಿಶುದ್ಧ ಪ್ರೀತಿಯ ಸಂಪತ್ತು.

ಹೌದು, ಸಂಸಾರದ ಸುಖ ಒಡವೆವಸ್ತ್ರಗಳಲ್ಲಿಲ್ಲ. ಸಾಧನಸೌಕರ್ಯಗಳಲ್ಲಿಯೂ ಇಲ್ಲ. ಅದಿರುವುದು ಪ್ರೀತಿಯ ಚೌಕಟ್ಟಿನಲ್ಲಿ, ಪ್ರೀತಿಯೆ ಸಂಸಾರದ ಸುಖದ ಕೀಲಿಕೈ. ಸಂಸಾರದ ಹೃದಯವೇ ಆ ಪ್ರೀತಿಯಲ್ಲಿದೆ. ಆ ಹೃದಯ ಸಂಪತ್ತಿನಿಂದಲೇ ಸಂಸಾರದ ಸುಖ ಅದೇ ಇಲ್ಲವಾದರೆ ಆ ಸುಖಕ್ಕೇ ಸಂಚಕಾರ.


\section*{ದೈವೀಪ್ರೀತಿಯ ದಿವ್ಯನಿಧಿ}

\addsectiontoTOC{ದೈವೀ ಪ್ರೀತಿಯ ದಿವ್ಯನಿಧಿ}

ಭಗವಾನ್ ಶ‍್ರೀರಾಮಕೃಷ್ಣರ ಶಿಷ್ಯರಲ್ಲೊಬ್ಬರಾದ ಸ್ವಾಮಿ ವಿಜ್ಞಾನಾನಂದಜಿ ಅವರನ್ನು ಒಮ್ಮೆ ಭಕ್ತರೊಬ್ಬರು ಪ್ರಶ್ನಿಸಿದರು–“ನಿಮ್ಮ ಮನಸ್ಸನ್ನು ಮೊದಲಿಗೆ ಸೆಳೆದ ಶ‍್ರೀರಾಮಕೃಷ್ಣರ ವೈಶಿಷ್ಟ್ಯ ಯಾವುದು?” ಎಂದು. ವಿಜ್ಞಾನಾನಂದಜಿ ಉತ್ತರಿಸಿದರು–“ಅವರ ತಪಸ್ಸು, ತ್ಯಾಗ ಅವರ ದಿವ್ಯದರ್ಶನಗಳು, ಬೋಧನೆ–ಇವಾವುವೂ ನನಗೆ ಆಗ ಅರ್ಥವಾಗಿರಲಿಲ್ಲ. ಆದರೆ ಅವರ ಪ್ರೀತಿ ಮಾತ್ರ ಅಸದೃಶವಾಗಿತ್ತು. ಮೊದಲ ಬಾರಿ ನಾನು ಅವರನ್ನು ಕಂಡಾಗಲೇ ಅತ್ಯಂತ ಆಪ್ತ ಚಿರಪರಿಚಿತನಂತೆ ನನ್ನೊಡನೆ ವ್ಯವಹರಿಸಿದರು. ನನ್ನ ತಂದೆ ತಾಯಿ, ಅಣ್ಣ ತಮ್ಮಂದಿರೂ ಅಷ್ಟು ಪ್ರೀತಿಯಿಂದ ನನ್ನನ್ನು ಕಂಡುಕೊಂಡಿರಲಿಲ್ಲ. ಅವರು ಪ್ರೀತಿಯ ನಿಧಿಯಾಗಿದ್ದರು. ಅವರ ಪ್ರೀತಿಗೆ ಯಾವ ಹೋಲಿಕೆಯೂ ಸಿಗದು. ನಮ್ಮಲ್ಲಿ ಪ್ರತಿಯೊಬ್ಬರ ಹಿತಚಿಂತನೆಯನ್ನೂ ಅವರು ಎಷ್ಟೊಂದು ಮಾಡುತ್ತಿದ್ದರು!”

\newpage

ಶ‍್ರೀರಾಮಕೃಷ್ಣರಲ್ಲಿ ವ್ಯಕ್ತವಾದ ಪರಿಶುದ್ಧ, ನಿಃಸ್ವಾರ್ಥ ಪ್ರೀತಿ ಅವರನ್ನು ಸಮೀಪಿಸಿದ ಪ್ರತಿ\-ಯೊಬ್ಬ ವ್ಯಕ್ತಿಯಲ್ಲೂ ಅದ್ಭುತ ಪ್ರಭಾವವನ್ನು ಬೀರಿತ್ತು. ಆ ಪ್ರೀತಿಯ ಸಂಸ್ಪರ್ಶವನ್ನು ಪಡೆದ ವ್ಯಕ್ತಿಯಲ್ಲಿ ಅದು ಆತ್ಮಶ್ರದ್ಧೆಯನ್ನುಂಟುಮಾಡಿ, ಉನ್ನತಿಯ ಪಥದಲ್ಲಿ ನಡೆಯುವ ಪ್ರೇರಣೆ ನೀಡಿತು.

ಶ‍್ರೀರಾಮಕೃಷ್ಣರು ದಿವ್ಯವಾದ ಬ್ರಹ್ಮಾನಂದದಲ್ಲಿ ಮುಳುಗಿ ತೇಲುತ್ತಿದ್ದರು. ಎಲ್ಲರೂ ಆ ದಿವ್ಯ ಆನಂದವನ್ನು ಪಡೆಯಲು ಯತ್ನಿಸುತ್ತಿಲ್ಲವಲ್ಲ ಎಂಬುದೇ ಅವರ ಚಿಂತೆಯಾಗಿತ್ತು! ತಾವು ಯಾವ ಆನಂದ ಸಾಗರದಲ್ಲಿ ಮುಳುಗಿ ತೇಲುತ್ತಿದ್ದರೋ ಅದನ್ನು ಇತರರಿಗೂ ಹಂಚಿ, ಅವರ ಬದುಕನ್ನು ಕೃತಾರ್ಥ ಮಾಡಬೇಕೆಂಬ ಹಂಬಲ ಸದಾ ಅವರಲ್ಲಿತ್ತು. ಸಂಸಾರದ ತಾಪತ್ರಯಗಳಲ್ಲಿ ಸಿಲುಕಿ ದುಃಖತಪ್ತರಾದ ಜನರನ್ನು ಕಂಡು ಅವರ ಹೃದಯ ಮರುಕದಿಂದ ತುಂಬುತ್ತಿತ್ತು. ಎಷ್ಟೇ ಕಷ್ಟವಾದರೂ ಒಂದೇ ಒಂದು ಜೀವಿಗೆ ಬೆಳಕನ್ನು ನೀಡಲು ಸಾಧ್ಯವಾದರೆ ಅದು ಅವರಿಗೆ ನೆಮ್ಮದಿಯನ್ನೂ, ತೃಪ್ತಿಯನ್ನೂ ನೀಡುತ್ತಿತ್ತು. ಅವರನ್ನು ಸಮೀಪಿಸಿದವರೊಡನೆ ಹೇಳುತ್ತಿದ್ದರು –“ನನಗೆ ಯಾವ ಲೌಕಿಕ ಆಸೆಯೂ ಇಲ್ಲ. ನನಗೆ ದೇಹ ಸುಖ ಬೇಡ, ಕೀರ್ತಿ ಬೇಡ, ಧನಕನಕಾದಿಗಳು ಬೇಡ. ದೇವರ ನಾಮವನ್ನು ಹೇಳುತ್ತ ನಾನು ಸಮಾಧಿಸ್ಥನಾಗುತ್ತೇನೆ; ಸಂಪೂರ್ಣ ಮೈಮರೆಯುತ್ತೇನೆ. ನಾನೇಕೆ ನಿಮ್ಮನ್ನು ಇಷ್ಟೊಂದು ಪ್ರೀತಿಸುತ್ತೇನೆ ಗೊತ್ತೆ? ನೀವಿನ್ನೂ ಚಿಕ್ಕವರು. ನಿಮ್ಮ ಮನಸ್ಸಿನ್ನೂ ಲೋಕದ ಜಂಜಾಟದಲ್ಲಿ ಸಿಕ್ಕಿ ಮಲಿನವಾಗಿಲ್ಲ. ಶ್ರದ್ಧೆ ಮತ್ತು ನಿಷ್ಠೆಯಿಂದ ಸಾಧನೆ ಮಾಡಿದರೆ ನೀವು ಬಹುಬೇಗನೆ ದಿವ್ಯ ಆನಂದವನ್ನು ಪಡೆದು ಧನ್ಯರಾಗುವಿರಿ. ಅದಕ್ಕಾಗಿಯೇ ನಿಮ್ಮನ್ನು ನೋಡಲು ಅಷ್ಟೊಂದು ಕಾತರನಾಗಿರುವೆ.”

ದೈವೀಪ್ರೀತಿಯ ದಿವ್ಯನಿಧಿಯಾದ ಶ‍್ರೀರಾಮಕೃಷ್ಣರ ವಾಕ್ಯಗಳನ್ನು ಮನನ ಮಾಡಿದರೆ, ನಿಃಸ್ವಾರ್ಥ ಪ್ರೇಮದ ಮೂಲ ಯಾವುದೆಂಬುದನ್ನು ತಿಳಿಯಲು ಕಷ್ಟವಿಲ್ಲ.


\section*{ಮಾತೃತ್ವದ ಮಹಾ ಆದರ್ಶ}

\addsectiontoTOC{ಮಾತೃತ್ವದ ಮಹಾ ಆದರ್ಶ}

ಯಾವುದೇ ಆದರ್ಶ ಅಥವಾ ಧ್ಯೇಯ, ವ್ಯಕ್ತಿಯ ಬದುಕಿನಲ್ಲಿ ಪೂರ್ಣ ರೀತಿಯಿಂದ ವ್ಯಕ್ತವಾಗ\-ದಿದ್ದಲ್ಲಿ, ಅದೊಂದು ಒಣ ಸಿದ್ಧಾಂತವಾಗಿಯೇ ಉಳಿಯುವುದು. ಜನಸಾಮಾನ್ಯರ ಪಾಲಿಗೆ ಅದು ಅನುಷ್ಠಾನಕ್ಕೆ ಬೇಕಾದ ಸ್ಫೂರ್ತಿ ಮತ್ತು ಮಾರ್ಗದರ್ಶನ ನೀಡಲಾರದು. ದೇವರ ಮಾತೃ ಭಾವವನ್ನು ಅರ್ಥಮಾಡಿಕೊಳ್ಳಲು ಪರಮಹಂಸರೇ ಮಾತೃತ್ವದ ಮೂರ್ತಿಯೊಂದನ್ನು ಕಡೆದು ನಿಲ್ಲಿಸಿದರು. ಎಲ್ಲರೂ ಯಾರನ್ನು ‘ಅಮ್ಮಾ’ ಎಂದು ಕರೆಯಬಹುದೋ, ಯಾರಿಂದ ಅಭಯ, ಆಶೀರ್ವಾದಗಳನ್ನು ಪಡೆದು ಸಾಂಸಾರಿಕ ಬಂಧನಗಳಿಂದ ಬಿಡುಗಡೆ ಹೊಂದಿ ದಿವ್ಯತೆಯನ್ನೇರ ಬಹುದೋ, ಯಾರ ಜೀವನ ಎಂಬ ಸಂದೇಶವು ಸರ್ವಸಾಧಾರಣರಲ್ಲಿ ಮಾತೃಭಾವದ ಆದರ್ಶವನ್ನು ಪ್ರಸಾರ ಮಾಡಬಲ್ಲದೋ, ಅಂಥ ಮಾತೃಮೂರ್ತಿಯನ್ನು ರಚಿಸಿದರು. ದೇವರು ನಮಗೆ ಮಾತೆಯಂತೆ ಅತ್ಯಂತ ಸಮೀಪದವನು. ನಾವು ಅವನನ್ನು ಪ್ರೀತಿಯಿಂದ ಸಮೀಪಿಸಬಹುದು ಎಂಬುದನ್ನು ಆ ಮೂಲಕ ತಿಳಿಯಪಡಿಸಿದರು. ಕೆಲವರಾದರೂ, ನಿಃಸ್ವಾರ್ಥ ಪ್ರೇಮದ ಉತ್ತುಂಗ ಶಿಖರವನ್ನೇರಲು ಸಾಧ್ಯ ಎಂಬುದನ್ನು ತೋರಿದರು. ಉಳಿದವರು ಆ ದಾರಿಯಲ್ಲಿ ಬಹುದೂರ ಮುನ್ನಡೆಯಲು ಸಾಧ್ಯ ಎನ್ನುವ ಸ್ಫೂರ್ತಿಯನ್ನು ನೀಡಿದರು. ಸ್ತ್ರೀಯರು ಮಾತೃತ್ವದ ಆದರ್ಶದ ಮೂಲಕ ಕೃತಕೃತ್ಯತೆಯನ್ನು ಪಡೆಯುವ ವಿಧಾನವನ್ನು ತಿಳಿಸಿಕೊಟ್ಟರು.


\section*{ಕರುಣೆಯ ಕಡಲು}

\addsectiontoTOC{ಕರುಣೆಯ ಕಡಲು}

ಜಾತಿ, ವರ್ಣ, ಕುಲ, ಗೋತ್ರ ಭೇದವಿಲ್ಲದೆ ಅಸಂಖ್ಯ ಪುತ್ರರೂ, ಪುತ್ರಿಯರೂ ಅವರನ್ನು ‘ತಾಯಿ’ ಎಂದು ಕರೆದರು. ಮಗುವು ತಾಯಿಯನ್ನು ಸಮೀಪಿಸುವಂತೆ, ಭಕ್ತರು ಅವರನ್ನು ಸಮೀಪಿಸಿ, ಅವರ ಸಾನ್ನಿಧ್ಯದಲ್ಲಿ ನಿಶ್ಚಿಂತರಾಗುತ್ತಿದ್ದರು. ಅಪೂರ್ವ ಶಾಂತಿಯನ್ನು ಅನುಭವಿಸುತ್ತಿದ್ದರು. ಮಾರ್ಗದರ್ಶನವನ್ನು ಪಡೆಯುತ್ತಿದ್ದರು. ಧನ್ಯತೆಯ ಭಾವವನ್ನು ಹೊಂದುತ್ತಿದ್ದರು. ಈ ತಾಯಿಯಾದರೋ ಹಗಲೂರಾತ್ರಿ ಮಕ್ಕಳ ಶುಭಕ್ಕಾಗಿ ಪ್ರಾರ್ಥಿಸಿದರು. ಮಕ್ಕಳ ನೂರು ತಪ್ಪುಗಳನ್ನೂ ತಾಳ್ಮೆಯಿಂದ ಕ್ಷಮಿಸುತ್ತ, ಅವರನ್ನು ಸರಿದಾರಿಗೆ ತರಲು ಯತ್ನಿಸುತ್ತಿದ್ದರು. ದಿನ ದಿನ, ತಿಂಗಳು ತಿಂಗಳು, ವರ್ಷ ವರ್ಷಗಳ ಕಾಲ, ಕೆಲವೊಮ್ಮೆ ಸ್ವಲ್ಪವೂ ವಿಶ್ರಾಂತಿಯನ್ನು ತೆಗೆದುಕೊಳ್ಳದೆ, ದೂರದೂರದಿಂದ ಬಂದ ಮಕ್ಕಳ ಕ್ಷೇಮಚಿಂತನೆಯಿಂದ ದುಡಿಯುತ್ತಿದ್ದರು. ಅತಿಥಿಗಳಾಗಿ ಬಂದ ಅವರ ಊಟ ತಿಂಡಿಯ ವ್ಯವಸ್ಥೆ ಆರೋಗ್ಯಲಾಭಕ್ಕಾಗಿ ಮಾರ್ಗದರ್ಶನ, ಅವರ ಅಭ್ಯುದಯ ಚಿಂತನೆ ಮಾಡುತ್ತಲೇ ಇದ್ದರು. ಮಕ್ಕಳ ಒತ್ತಾಯದ ಹಠಗಳನ್ನು ಸೈರಿಸಿಕೊಂಡು ಅವರ ಇಚ್ಛೆಯನ್ನು ನೆರವೇರಿಸುತ್ತಿದ್ದರು. ಯಾರಾದರೂ ಕಠಿಣರೋಗದಿಂದ ನರಳುತ್ತಿದ್ದರೆ, ಯೋಗಶಕ್ತಿಯಿಂದ ಅದನ್ನು ತಮ್ಮ ಶರೀರದ ಮೇಲೆ ಎಳೆದುಕೊಂಡು ತಾವೇ ದಾರುಣ ನೋವನ್ನು ಅನುಭವಿಸುತ್ತಿದ್ದರು. ಇದರ ಮೂಲಕ ಭಕ್ತ ಜನರನ್ನು ತಮ್ಮ ಆಪ್ತರನ್ನಾಗಿ ಮಾಡಿಕೊಂಡು, ತ್ಯಾಗಮಯ, ಸಹನಶೀಲತೆಯ ಜೀವನಾದರ್ಶವನ್ನು ಬೋಧಿಸುತ್ತಿದ್ದರು. ಸಂಸಾರದ ಕಷ್ಟಕಂಟಕಗಳಲ್ಲಿ ಸಿಲುಕಿ, ದಿಕ್ಕೆಟ್ಟು ಬಳಲಿ\-ದವರಿಗೆ, ಗುರುಶಕ್ತಿಯಿಂದ ಮಂತ್ರ ಪ್ರದಾನಮಾಡಿ, ಜಪ, ಪ್ರಾರ್ಥನೆ, ಉಪಾಸನಾ ವಿಧಾನಗಳನ್ನು ತಿಳಿಸಿ, ಅಭಯ ನೀಡುತ್ತಿದ್ದರು. ಅಂತರಂಗದಲ್ಲಿರುವ ಅಪೂರ್ವ ದೈವೀಸ್ವಭಾವವನ್ನು ಮರೆಮಾಡಿ, ಹೆತ್ತುಹೊತ್ತು, ಸಾಕಿಸಲಹಿದ ತಾಯಿಯಂತೆಯೇ ಎಲ್ಲರೆದುರೂ ಅವರು ಕಂಗೊಳಿಸಿದರು. ಅವರದು ನಿರಂತರ ಪರಿಶ್ರಮದ ಬದುಕಾದರೂ ಸದಾ ಆನಂದದಿಂದಲೇ ಇರುತ್ತಿದ್ದರು.

ಈ ಉನ್ನತ ಸ್ಥಿತಿಯನ್ನು ಅವರು ಮೆಟ್ಟಲು ಮೆಟ್ಟಲಾಗಿ ಏರಿದರು. ತ್ಯಾಗ ಮತ್ತು ಸೇವೆಯ ಆದರ್ಶಗಳ ಮೂಲಕ ಸೇವಾಪರಾಯಣೆಯಾದ ಮಗಳಾಗಿ, ಸ್ನೇಹಪರಾಯಣೆಯಾದ ಅಕ್ಕನಾಗಿ, ಪತಿಪರಾಯಣೆಯಾದ ಸಹಧರ್ಮಿಣಿಯಾಗಿ, ಸಂತಾನವತ್ಸಲೆಯಾದ ಮಾತೆಯಾಗಿ, ಆಧ್ಯಾತ್ಮಿಕ ಪಥಪ್ರದರ್ಶಕಳಾದ ಗುರುವಾಗಿ, ದೈವೀ ಪ್ರೇಮ ಅವರಲ್ಲಿ ವ್ಯಕ್ತವಾಯಿತು. ಬಾಲ್ಯದಲ್ಲಿ ವ್ಯರ್ಥ ಕಾಲಹರಣಕ್ಕೆ ಅವಕಾಶಕೊಡದೆ, ಸದಾ ತಂದೆತಾಯಿಗಳಿಗೆ ಕೆಲಸದಲ್ಲಿ ನೆರವಾಗಿ, ಅವರ ಶ್ರಮಭಾರವನ್ನು ಕಡಿಮೆ ಮಾಡಿದರು. ಪಿತೃವಿಯೋಗದ ನಂತರ ಮನೆಯಲ್ಲಿ ಅನ್ನಾಭಾವ ದೂರ ಮಾಡಲು, ತಾವೇ ಭತ್ತವನ್ನು ಕುಟ್ಟಿ ಅಕ್ಕಿಯನ್ನು ತಯಾರಿಸುತ್ತಿದ್ದರು. ಮುದುಕಿಯಾದ ತಾಯಿಯ ಸೇವೆಯನ್ನು ಕೈಗೊಂಡರು. ವೃದ್ಧ ಚಿಕ್ಕತಂದೆಯ ಸೇವೆಯನ್ನು ಮಾಡಿದರು. ಸಹೋದರರನ್ನು ಎತ್ತಿ ಆಡಿಸಿ, ಲಾಲಿಸಿ ಪಾಲಿಸಿ, ದೊಡ್ಡವರನ್ನಾಗಿ ಮಾಡಿದರು. ಸಹೋದರ ಮೃತ್ಯುವಶನಾದಾಗ ಅವನ ಪತ್ನಿ ಮತ್ತು ಮಕ್ಕಳನ್ನು ತಾವೇ ನೋಡಿಕೊಂಡರು. ಇತರ ಸಹೋದರರೊಳಗೆ ಜಗಳವನ್ನು ದೂರಮಾಡಿ, ಸಹಕಾರದಿಂದ ನಡೆಯಲು ಆಜೀವನ ಯತ್ನಿಸಿದರು. ಒತ್ತಾಯ ಒತ್ತಡಗಳಿಂದ ಸಾಂಸಾರಿಕ ಪರಿಸರದಲ್ಲಿ ಕೆಲಸ ಮಾಡುವವರು ಬಹುಮಂದಿ ಇರಬಹುದು. ಇದ್ದಾರೆ. ಆದರೆ, ಹೆಚ್ಚಿನವರು ತಮ್ಮ ಸೇವೆ ಅಥವಾ ಕೆಲಸದ ಹೆಗ್ಗಳಿಕೆಯನ್ನು ಹೇಳಿಕೊಳ್ಳದೆ ಇರಲಾರರು; ಗೊಣಗುತ್ತ ಮಾಡದಿರಲಾರರು. ಶ‍್ರೀಮಾತೆ ಯಾರ ಒತ್ತಾಯಕ್ಕೂ ಮಣಿದು ಸೇವಾಕಾರ್ಯವನ್ನು ಕೈಗೊಂಡವರಲ್ಲ. ಆತ್ಮಶ್ಲಾಘನೆಯನ್ನೆಂದೂ ಮಾಡಿದವರಲ್ಲ. ಅವರ ಪತಿಸೇವಾ ವಿಧಾನ, ಪಾತಿವ್ರತ್ಯದ ಆದರ್ಶ ಅನುಪಮವಾದುದು. ಯಾವತ್ತೂ ದೇಹ ಬುದ್ಧಿ ಇಲ್ಲದ ಆ ದೇವ ದಂಪತಿಗಳ ಈ ಪ್ರೀತಿಯ ಆದರ್ಶ, ಜಗತ್ತಿನ ಇತಿಹಾಸದಲ್ಲೇ ಅಪೂರ್ವ! ಒಂದು ದಿನವಾದರೂ ಆ ಆದರ್ಶ ಮಲಿನವಾಗಲಿಲ್ಲ. ಈ ಮಾತೃಮೂರ್ತಿಯೇ ಶ‍್ರೀಶಾರದಾದೇವಿಯವರು. ಸಿಸ್ಟರ್ ನಿವೇದಿತಾ ಶ‍್ರೀಮಾತೆಯನ್ನು ಹತ್ತಿರದಿಂದ ಕಂಡು ಅವರ ದಿವ್ಯ, ಮಧುರ ಚಾರಿತ್ರ್ಯವನ್ನು ಅರಿತು ಹೀಗೆಂದು ಉದ್ಗರಿಸಿದರು:

‘ಶ‍್ರೀಮಾತೆ ಎಂದರೆ:

ಎಂದೆಂದೂ ನಮ್ಮನ್ನು ತಿರಸ್ಕರಿಸದೇ ಹಂಬಲಿಸುವ ಪ್ರೀತಿಯ ಪ್ರತಿಮೂರ್ತಿ,

ಎಂದೆಂದೂ ನಮ್ಮ ಜೊತೆಯಲ್ಲೇ ಇರುವ ಮಂಗಲಮಯ ಅನುಗ್ರಹ,

ದೂರವಾಗಿ, ಬೇರೆಯಾಗಿ ಬೆಳೆಯಲಾಗದ ದಿವ್ಯಸಾನ್ನಿಧ್ಯ,

ಯಾರಲ್ಲಿ ಪೂರ್ಣ ವಿಶ್ವಾಸ, ಭರವಸೆಗಳನ್ನಿಡಬಹುದೋ ಅಂಥ ಕೋಮಲ ಹೃದಯ,

ಆಳ ಅಳೆಯಲಾಗದ ಮಾಧುರ್ಯ,

ಬಿಡಿಸಿಕೊಳ್ಳಲಾಗದ ಪರಿಶುದ್ಧ ಪ್ರೀತಿಯ ಬಂಧನ,

ಅಕಳಂಕ ಪಾವಿತ್ರ್ಯ,

ಮೇರೆ ಇಲ್ಲದ ವಾತ್ಸಲ್ಯದ ವಾರಿಧಿ

–ಇವೆಲ್ಲವೂ ಆಗಿ ಇವನ್ನೂ ಮೀರಿ ನಿಂತ ಮಹಿಮೆ.’\footnote{\engfoot{Mother is yearning love that can never refuse us, a bendiction that forever abides with us, a presence from which we cannot grow away, a heart in which we are safe, sweetness unfathomed, bond never breakable, holiness without a shadow, all these indeed and more.}\hfill\engfoot{–Sister Nivedita.}}

ತಂದೆಯಷ್ಟು ತಾಯಿ ಪ್ರಸಿದ್ಧಿಯಾಗುವುದಿಲ್ಲ ದಿಟ. ಆದರೆ ತಂದೆಯ ಪ್ರಸಿದ್ಧಿ, ಮಕ್ಕಳ ಅಭ್ಯುದಯಗಳ ಹಿನ್ನೆಲೆಯಲ್ಲಿ ಎಲೆಮರೆಯ ಕಾಯಿಯಂತೆ ದುಡಿಯಬಲ್ಲ ತಾಯಿಯ ಪ್ರೀತಿ ವಾತ್ಸಲ್ಯಗಳ, ತ್ಯಾಗ ಸೇವೆಗಳ ಪರ್ವತವೇ ಇದೆ ಅಲ್ಲವೇ? ತಾಯಿಯ ತ್ಯಾಗ, ಸೇವೆ, ಸಹನೆ– ಇವುಗಳಿಗೆ ಬೆಲೆ ಕಟ್ಟಲಾದೀತೇ? ಮನೆಯಲ್ಲಿ ತಾಯಿಯ ಮೂಲಕ ಈ ದಿವ್ಯಪ್ರೀತಿ ಮಕ್ಕಳೆಡೆಗೆ ಹರಿಯುವಂತಾದರೆ, ಈ ಪ್ರೀತಿಯ ಅಲ್ಪಸ್ವಲ್ಪ ಅಭಿವ್ಯಕ್ತಿಯೂ ಮಕ್ಕಳಲ್ಲಿ ಎಂಥ ಅದ್ಭುತ ಸತ್​\-ಪರಿಣಾಮ ಮಾಡಬಲ್ಲುದು ಎಂಬುದನ್ನು ಯೋಚಿಸಿ ನೋಡಿ.


\section*{ನಲ್ಮೆಯಿಂದ ನಲಿವು}

\addsectiontoTOC{ನಲ್ಮೆಯಿಂದ ನಲಿವು}

ದೈಹಿಕ, ಜೈವಿಕ ಬಯಕೆಗಳನ್ನು ತಣಿಸುವಲ್ಲಿ ಒಂದು ವ್ಯವಸ್ಥೆ ಅಥವಾ ನಿಯಮವನ್ನು ಅನುಸರಿಸಿದಲ್ಲಿ, ಮನುಷ್ಯ ಬಯಕೆಗಳ ಬವಣೆಯಿಂದ ತಪ್ಪಿಸಿಕೊಂಡು, ವ್ಯಕ್ತಿತ್ವದ ಮೇಲಿನ ಮಜಲುಗಳನ್ನು ಏರಬಲ್ಲ. ವ್ಯಕ್ತಿಯ ಶೀಲೋತ್ಕರ್ಷದ ಜೊತೆಜೊತೆಗೆ ಸಮಾಜದ ಹಿತಚಿಂತನೆ ಆಗ ಮಾತ್ರ ಸಾಧ್ಯ. ವೃದ್ಧ ದಂಪತಿಗಳಲ್ಲಿ ಇಂದ್ರಿಯಗಳ ಸೆಳೆತ ನಿಸ್ತೇಜವಾದಾಗ, ವಿಷಯಾಸಕ್ತಿ ಕಡಿಮೆಯಾಗಿ, ಕೇವಲ ಪ್ರೇಮವೊಂದೇ ಘನೀಭೂತವಾಗುವುದು. ಅವರ ಅನುರಾಗದಲ್ಲಿ ರಾಗದ್ವೇಷಗಳ ಅಪಸ್ವರವಿಲ್ಲ, ಕೋಪ ತಾಪಗಳ ಆಲಾಪವಿಲ್ಲ. ಅಲ್ಲಿರುವುದು ಶಾಂತಿ ಸಮಾಧಾನಗಳ ಸಂಚಾರ ಮಾತ್ರ. ಆಗ ಪತಿಯ ಸಮೀಪದಲ್ಲಿ ಸತಿ, ಕರುಣಾಮಯಿ ದಾದಿಯಂತೇ ಇರುವಳು. ಅಂಥ ಸತಿಗೂ ತಾಯಿಗೂ ಭೇದವೆಲ್ಲಿದೆ? ಯಾವುದು ಕಾಲಧರ್ಮದಿಂದ ಸ್ವಾಭಾವಿಕವಾಗುವುದೋ, ಅದನ್ನು ಆಧ್ಯಾತ್ಮಿಕ ಆದರ್ಶಗಳ ಅನುಷ್ಠಾನದಿಂದ, ಬೇಗನೆ ಸಹಜಸ್ವಭಾವವನ್ನಾಗಿ ಪರಿವರ್ತಿಸಬಹುದು. ಪತಿಪತ್ನಿಯರಿಬ್ಬರಲ್ಲೂ ಆಧ್ಯಾತ್ಮಿಕ ಆದರ್ಶವಿದ್ದರೇನೆ ಈ ಪರಿವರ್ತನೆ ಸಾಧ್ಯ ಎಂಬುದು ನಿಶ್ಚಿತ. ಹಾಗಾದಾಗ ಮಾತ್ರವೇ, ಪರಸ್ಪರ ಸಾಮರಸ್ಯದಿಂದ ಇರಬಹುದು. ಇಲ್ಲವಾದಲ್ಲಿ, ಪರಸ್ಪರ ವಿಮುಖರಾಗಿ ದಿನನಿತ್ಯವೂ ವಿರಸವನ್ನನುಭವಿಸಬೇಕಾದೀತು!

\enginline{Love is blind} ಎನ್ನುವ ಮಾತೊಂದಿದೆ. ಪ್ರೀತಿ ಕುರುಡು ಎಂದರೆ ಪ್ರೀತಿಗೆ ಎಲ್ಲರೂ ಒಂದೇ, ತಾರತಮ್ಯದ ದೃಷ್ಟಿ ಅದರಲ್ಲಿಲ್ಲ ಎಂದೇ ಅರ್ಥ. ಆದರೆ ಅದೆಷ್ಟೋ ಬಾರಿ ಪ್ರೀತಿ ಪಾತ್ರರೇ, ಕಾಮವನ್ನೇ ಪ್ರೇಮವೆಂದೆಣಿಸಿ, ವಿವೇಕ ವಿಮರ್ಶೆ ಕಳೆದುಕೊಂಡು, ನೀತಿನಿಯಮಕ್ಕೆ ಕುರುಡಾಗುತ್ತಾರೆ!

ಮದುವೆಯಾದ ಹೊಸದರಲ್ಲಿ ದಂಪತಿಗಳಲ್ಲಿ ಪ್ರೇಮದಲ್ಲಿ ಕಾಮದ ಕಾವೇ ಜಾಸ್ತಿಯಾದರೂ, ಅವರ ನಲ್ಮೆ ಕೇವಲ ದೈಹಿಕ ಆಕರ್ಷಣೆ ಆಸಕ್ತಿಗಳಲ್ಲಿ ವ್ಯಕ್ತವಾದರೂ, ಆಧ್ಯಾತ್ಮಿಕ ಹಿನ್ನೆಲೆಯಿಂದ ಕ್ರಮೇಣ ಅದನ್ನು ಉದಾತ್ತಗೊಳಿಸಬಹುದು. ದೃಢ ಆಧ್ಯಾತ್ಮಿಕ ನಿಷ್ಠೆಯಿಂದ ಪತಿಯು ಸಾಧನೆಯಲ್ಲಿ ತೊಡಗಿ ಭಕ್ತಿ, ಶ್ರದ್ಧೆ, ವ್ಯಾಕುಲತೆಗಳಿಂದ ದೇವರಲ್ಲಿ ಮಣಿದು ಮೊರೆಯುತ್ತಾನೆ. ಸತಿಗೆ ತನ್ನ ಗಂಡನೇ ದೇವರಾಗುತ್ತಾನೆ, ಆತನ ಸೇವೆಯೇ ಸಾಧನೆಯಾಗುತ್ತದೆ. ರೈಲ್ವೇ ಇಂಜಿನ್ನನ್ನು ಅನುಸರಿಸುವ ಬೋಗಿಗಳಂತೆ ಆತನಿಗೆ ಪೂರ್ಣವಿಧೇಯಳಾಗಿರುವುದೇ ಆಕೆಯ ಪಾಲಿನ ತಪಸ್ಸಾಗುತ್ತದೆ. ಆತನ ನೋವುನಲಿವು, ಸುಖದುಃಖಗಳಲ್ಲಿ ಒಂದಾಗಿ, ತಾಳ್ಮೆಯ ಗಣಿಯಾಗಿ, ಪ್ರೀತಿ ವಾತ್ಸಲ್ಯಗಳ ತವರಾಗಿ, ಮನೆಯ ಕೆಲಸಕಾರ್ಯಗಳನ್ನೆಲ್ಲ ದೇವರ ಪೂಜೆ ಎಂದೇ ಮಾಡುತ್ತಾ, ತ್ಯಾಗ ಸೇವೆಗಳಲ್ಲೇ ಆಕೆ ಸುಖ ಕಾಣುತ್ತಾಳೆ. ಅವರಲ್ಲಿ ಸಮರ್ಪಣ ಭಾವ ಹೆಚ್ಚಿದಂತೆ ದೈಹಿಕ ಆಕರ್ಷಣೆ ತನ್ನಿಂದ ತಾನಾಗಿ ದೂರಾಗುತ್ತದೆ. ಆಗ ಅವರ ಪ್ರೀತಿ ಪಕ್ವವಾಗಿ ನಲಿವಿನ ಸಮಾರಾಧನೆಯನ್ನೇ ಮಾಡಿಸುತ್ತದೆ.


\section*{ಆಧುನಿಕತೆಯ ಅಟ್ಟಹಾಸ}

\addsectiontoTOC{ಆಧುನಿಕತೆಯ ಅಟ್ಟಹಾಸ}

ಇಂದಿನ ದಿನಗಳಲ್ಲಿ ಆಧ್ಯಾತ್ಮಿಕ ಆದರ್ಶದ ಹಿನ್ನೆಲೆಯಿಂದ, ಉನ್ನತವಾದ ಬಾಳನ್ನು ಬಾಳಿದಂಥವರ ಚಾರಿತ್ರ್ಯದ ಮಾಧುರ್ಯ, ಮಾಹಾತ್ಮ್ಯಗಳನ್ನು ತಿಳಿದುಕೊಳ್ಳಲು ಶಿಕ್ಷಣದಲ್ಲಿ ಯಾವ ವಿಶಿಷ್ಟ ವ್ಯವಸ್ಥೆಯೂ ಇಲ್ಲ. ಮೌಲ್ಯಗಳ ಬಗ್ಗೆ ಸರ್ವತ್ರ ಅಜ್ಞಾನ ಕಂಡುಬರುತ್ತದೆ. ಬದುಕಿಗೊಂದು ಉನ್ನತ ಗುರಿಯನ್ನೂ, ಸುಖ ಶಾಂತಿ, ತೃಪ್ತಿಗಳನ್ನೂ ನೀಡುವ, ಎಲ್ಲರಿಗೂ ನೀರು, ಗಾಳಿಗಳಂತೆ ಅತ್ಯಂತ ಆವಶ್ಯಕವಾದ, ಮೂಲಭೂತ ಮೌಲ್ಯಗಳ ವಿಚಾರವನ್ನು ಯುವಕ ಯುವತಿಯರಿಗೆ ಮನಮುಟ್ಟುವಂತೆ ತಿಳಿಸುವವರಾರು? ಒಂದೆಡೆ ಶೈಶವದಿಂದಲೇ ಪರೀಕ್ಷೆ ಪಾಸು ಮಾಡುವುದಕ್ಕಾಗಿ, ಅಸಂಖ್ಯ ವಿಷಯಗಳನ್ನು ಕುರಿತು ವಿಚಾರಸಂಗ್ರಹ ನಡೆಯುತ್ತಿದೆ; ವಿಜ್ಞಾನದ ವೈಭವವನ್ನು ಹಾಡಿ ಹೊಗಳಲಾಗುತ್ತಿದೆ. ಇನ್ನೊಂದೆಡೆ ಅಗ್ಗದ ಭಾವುಕತೆ, ಕಾಮಪ್ರಚೋದಕ ವಿಚಾರಗಳು, ದರೋಡೆ, ಕಳ್ಳತನಗಳು, ಹಿಂಸಾರತಿ–ಇವುಗಳ ಅತಿರಂಜಿತ ಸಿನಿಮೀಯ ವರ್ಣನೆಗಳಿಂದ, ಮುಗ್ಧ ಮನಸ್ಸಿಗೆ ತಪ್ಪು ಮಾರ್ಗದಲ್ಲಿ ನಡೆಯುವ ಪ್ರೇರಣೆ ನೀಡಲು, ಸದ್ಯದ ಪರಿಸರ ಬದ್ಧ\-ಕಂಕಣ\-ವಾದಂ\-ತಿದೆ. ವಿಚಾರವಂತರೂ, ಶೈಕ್ಷಣಿಕ ಕ್ಷೇತ್ರದಲ್ಲಿ ದುಡಿಯುವವರೂ ಈ ಬಗ್ಗೆ ಚಿಂತನೆ ಮಾಡುವ ಲಕ್ಷಣ ಕಾಣುತ್ತಿಲ್ಲ. ಬದಲಾಗಿ, ಮನುಷ್ಯರಲ್ಲಿ ಕಂಡುಬರುವ ದೌರ್ಬಲ್ಯಗಳು ಸಹಜ ಸ್ವಭಾವ; ಒಂದು ಆದರ್ಶಕ್ಕಾಗಿ ಹೋರಾಡುವುದು ಒಂದು ತೆರನಾದ ‘ಹಿಪಾಕ್ರಸಿ’ ಎನ್ನುವ ಅನಿಸಿಕೆಯನ್ನು ಬುದ್ಧಿಜೀವಿಗಳೆನ್ನಿಸಿಕೊಂಡವರು ಪ್ರತ್ಯಕ್ಷವಾಗಿಯೋ, ಪರೋಕ್ಷವಾಗಿಯೋ, ಮಕ್ಕಳ ಮನಸ್ಸಿನಲ್ಲಿ ಬಿಂಬಿಸುವುದುಂಟು. ವೈಜ್ಞಾನಿಕತೆ, ವೈಚಾರಿಕತೆಯ ಹೆಸರಿನಲ್ಲಿ, ಈಗ ಲಂಪಟತೆ, ಸ್ವೇಚ್ಛಾಚಾರ, ಸ್ವೈರವೃತ್ತಿಗಳ ಪ್ರಕಟ ಹಾಗೂ ಪ್ರಚ್ಛನ್ನ ಪ್ರಸಾರ ಸರ್ವತ್ರ ಕಂಡುಬರುತ್ತಲಿದೆ. ಇತ್ತೀಚೆಗೆ, ರ್ಯಾಗಿಂಗ್ ಒಂದು ಮಟ್ಟದಲ್ಲಿ ಕೆಡುಕಲ್ಲ ಎಂದು ಮಂತ್ರಿ ಮಹೋದಯರು ಸಾರ್ವಜನಿಕ ಸಭೆಯಲ್ಲಿ ಹೇಳಿದಾಗ, ಯುವಕರು ಐದು ನಿಮಿಷಗಳ ಕಾಲ ಚಪ್ಪಾಳೆ ತಟ್ಟಿ, ತಮ್ಮ ಆನಂದವನ್ನು ವ್ಯಕ್ತಪಡಿಸಿದರೆಂದು ವರದಿಯಾಗಿದೆ. ಪಶ್ಚಿಮ ದೇಶಗಳಲ್ಲಿ ಕಂಡು ಬರುವ ಸ್ವಾತಂತ್ರ್ಯಪ್ರಿಯತೆ, ಭಾರತೀಯ ಯುವಕರನ್ನೂ ಆಕರ್ಷಿಸಿದೆ. ಅವರ ದೇಶಪ್ರೇಮ, ಉದ್ಯಮಶೀಲತೆ, ಪೌರಪ್ರಜ್ಞೆ, ಸಂಘಟಿತರಾಗಿ ಮಹತ್ಕಾರ್ಯವನ್ನು ಸಾಧಿಸುವ ಛಲ–ಇಂಥ ಗುಣಗಳಿಂದ ಪ್ರಯೋಜನ\break ಪಡೆ\-ಯುವ ಬದಲು, ಅವರ ದೌರ್ಬಲ್ಯಗಳನ್ನು ತಮ್ಮದಾಗಿಸಿಕೊಳ್ಳಲು ಇಲ್ಲಿನ ವಿದ್ಯಾವಂತರು ಯತ್ನಿಸುತ್ತಿರುವಂತಿದೆ! ಖ್ಯಾತ ಇತಿಹಾಸ ತಜ್ಞ, ಲಂಡನ್ ವಿಶ್ವವಿದ್ಯಾಲಯದ ಪ್ರಾಧ್ಯಾಪಕ ಎ.\ ಎಲ್.\ ಬಾಶಮ್ ೧೯೬೪ನೇ ಇಸವಿಯಲ್ಲಿ ಭಾರತವನ್ನು ಸಂದರ್ಶಿಸಿದಾಗ, ಪತ್ರಿಕಾ ಪ್ರತಿನಿಧಿ\-ಗಳೊಡನೆ ಹೀಗೆಂದರು:

\newpage

‘ಪಶ್ಚಿಮ ದೇಶಗಳ ಚಲಚ್ಚಿತ್ರ, ಸಾಹಿತ್ಯ, ನೃತ್ಯ ಮತ್ತು ಸಂಗೀತ–ಇವು ಭಾರತೀಯ ಯುವಕ\-ರನ್ನು, ಯೂರೋಪಿನ ಯುವಕರು ಅನುಭವಿಸಿದ ಸ್ವಾತಂತ್ರ್ಯವನ್ನು ಪಡೆಯುವಂತೆ ಪ್ರೇರಿಸಿವೆ. ಆ ಸ್ವಾತಂತ್ರ್ಯವು ಯೂರೋಪು ದೇಶಗಳ ಯುವಕರಿಗೆ ಸುಖವನ್ನು ತಂದಿಲ್ಲ. ಆದರೆ ಭಾರತೀಯ ಯುವಕರು ಆ ಸ್ವಾತಂತ್ರ್ಯವು ಸುಖವನ್ನು ತಂದೀತೆಂದು ನಂಬಿದ್ದಾರೆ. ಇನ್ನು ಒಂದೆರಡು ತಲೆಮಾರುಗಳಲ್ಲಿ ಅವರು ಆ ಸ್ವಾತಂತ್ರ್ಯವನ್ನು ಪಡೆದಾರು. ಆದರೆ ಅವರು ಸುಖಿಯಾಗುವರೇ? ಸ್ತ್ರೀಪುರುಷರ ನಿಕಟ ಸಂಪರ್ಕಕ್ಕೆ ನಿರ್ಬಂಧ, ಹಳೆಯ ವಿವಾಹ ಪದ್ಧತಿ–ಈ ದೇಶದಲ್ಲಿ ಕುಟುಂಬದ ಸ್ಥೈರ್ಯವನ್ನು ಕಾಪಾಡಿದ್ದವು. ಪಶ್ಚಿಮದಲ್ಲಿ ಸ್ವಾತಂತ್ರ್ಯವು ಸ್ವೇಚ್ಛಾಚಾರಕ್ಕೆಳೆಸಿದೆ. ಕುಟುಂಬ ಒಡೆದು ಚೂರಾಗುತ್ತಿದೆ. ಮಕ್ಕಳು ತಂದೆತಾಯಿಗಳಿಂದ ವಂಚಿತರಾಗಿ ಅನಾಥಾಲಯಗಳಲ್ಲಿ ಬೆಳೆಯುತ್ತಿದ್ದಾರೆ.’

ಕೌಟುಂಬಿಕ ಜೀವನದ ಕುಸಿತಕ್ಕೆ ಕಾರಣವಾದ ಪಶ್ಚಿಮ ದೇಶಗಳಲ್ಲಿನ ಸ್ವಚ್ಛಂದ ಲೈಂಗಿಕ ಮನೋವೃತ್ತಿಯನ್ನು ಭಾರತೀಯರೂ ಅನುಸರಿಸಿದರೆ, ಅದು ನಮ್ಮನ್ನು ಎಂಥ ದುರಂತಕ್ಕೆ ಎಳೆಸೀತು ಎಂಬುದರ ಮುನ್ನೋಟ ಇದು. ಪಾಶ್ಚಾತ್ಯದೇಶಗಳಲ್ಲಿನ ಜನ ತಮ್ಮ ತಪ್ಪಿನ ಅರಿವಾಗಿ ಈಗ ಜಾಗ್ರತರಾಗಿ ನಮ್ಮ ಆರ್ಷೇಯ ಸಂಸ್ಕೃತಿಯ ಕಡೆ ಆಕರ್ಷಿತರಾಗುತ್ತಿದ್ದರೆ, ನಮ್ಮ ಜನ ತಮ್ಮ ಸಂಸ್ಕೃತಿಯನ್ನು ಕಡೆಗಣಿಸಿ ಆ ಕಡೆಗೆ ಓಡುತ್ತಿದ್ದಾರಲ್ಲ! ಇದೆಂಥ ವಿಪರ್ಯಾಸ!


\section*{ದುರುಪಯೋಗದಿಂದ ದುರಂತ}

\addsectiontoTOC{ದುರುಪಯೋಗದಿಂದ ದುರಂತ}

ಆಸ್ಕರ್ ವೈಲ್ಡ್ ಹೇಳಿದ ಕತೆಯಿದು–ಒಮ್ಮೆ ಏಸುಕ್ರಿಸ್ತನು ಗ್ರಾಮೀಣ ಪ್ರದೇಶದಿಂದ ನಗರದ ಸಮೀಪ ಬರುತ್ತಿದ್ದ. ನಗರದ ಹೊರವಲಯದ ರಸ್ತೆಯಲ್ಲಿ ನಡೆದು ಬರುತ್ತಿದ್ದಾಗ, ಯುವಕನೊಬ್ಬ ಚರಂಡಿಯಲ್ಲಿ ಬಿದ್ದುಕೊಂಡಿರುವುದನ್ನು ಕಂಡ. ಆ ಯುವಕನನ್ನು ಕುರಿತು ‘ಅಯ್ಯಾ, ನೀನೇಕೆ ಕುಡಿತದ ಅಮಲಿನಲ್ಲಿ ಈ ರೀತಿ ಕಾಲ ಕಳೆಯುತ್ತಿದ್ದೀಯೆ?’ ಎಂದು ಕೇಳಿದ. ಯುವಕ ಉತ್ತರಿಸಿದ: ‘ಪ್ರಭು, ನಾನು ಕುಷ್ಠರೋಗದಿಂದ ನರಳುತ್ತಿದ್ದೆ. ದಯಾಮಯನಾದ ನೀನು ನನ್ನನ್ನು ಗುಣ\-ಪಡಿಸಿದೆ. ನಾನಿನ್ನೇನು ಮಾಡಲಿ?’ ಎಂದು. ಏಸುವು ನಿಟ್ಟುಸಿರುಬಿಟ್ಟು ಮುನ್ನಡೆಯುತ್ತಿದ್ದಂತೆ, ತರುಣನೊಬ್ಬ ವೇಶ್ಯೆಯನ್ನು ಹಿಂಬಾಲಿಸುತ್ತಿದ್ದುದನ್ನು ಕಂಡು ಮರುಕದಿಂದ, ‘ಅಯ್ಯಾ, ನೀನೇಕೆ ನಿನ್ನ ಚೇತನವನ್ನು ವ್ಯಭಿಚಾರದಲ್ಲಿ ಮುಳುಗಿಸುತ್ತಿದ್ದೀಯೆ?’ ಎಂದು ಪ್ರಶ್ನಿಸಿದ. ಆ ತರುಣ ‘ಪ್ರಭು, ನಾನು ಕುರುಡನಾಗಿದ್ದೆ. ಕೃಪಾಪರವಶನಾಗಿ ನೀನು ನನಗೆ ದೃಷ್ಟಿ ದಾನವನ್ನು ಮಾಡಿದೆ. ನಾನಿನ್ನೇನು ಮಾಡಲಿ?’ ಎಂದನಂತೆ. ಕೊನೆಗೆ ನಗರದ ಮಧ್ಯಭಾಗಕ್ಕೆ ಬರುವ ವೇಳೆ, ವೃದ್ಧ\-ನೊಬ್ಬನು ಗೋಳಿಡುತ್ತ ಹೊರಳಾಡುವುದನ್ನು ಕಂಡು ಏಸು ಅವನ ಅಳುವಿಗೆ ಕಾರಣ ಕೇಳಿದಾಗ, ಆತನೆಂದ: ‘ಪ್ರಭು, ನಾನು ಸತ್ತುಹೋಗಿದ್ದೆ. ನೀನು ನನ್ನನ್ನು ಬದುಕಿಸಿದೆ– ಪ್ರಾಣದಾನ ಮಾಡಿದೆ. ನಾನಿನ್ನೇನು ಮಾಡಲು ಶಕ್ತ?’ ಎಂಬುದಾಗಿ, ಏಸುವಿನ ಕೃಪೆಯನ್ನು ಅವರೆಲ್ಲ ದುರುಪಯೋಗ ಮಾಡಿಕೊಂಡಿದ್ದರು.

ಏಸುವು ಅತಿಮಾನುಷ ದೈವೀಶಕ್ತಿಯಿಂದ ಮಾಡಿದ ಇಂಥ ಪವಾಡವನ್ನು ಇಂದು ವಿಜ್ಞಾನ ಸಹಜವಾಗಿ ಮಾಡುತ್ತಿದೆ. ಶಸ್ತ್ರಚಿಕಿತ್ಸೆಯ ಪವಾಡಸದೃಶ ವಿಧಾನಗಳಿಂದ, ವಿಸ್ಮಯಕಾರಕ ಔಷಧಗಳ ಅನ್ವೇಷಣೆಯಿಂದ ವಿಜ್ಞಾನವು ರೋಗಗಳನ್ನು ಓಡಿಸುತ್ತಿದೆ; ವೃದ್ಧಾಪ್ಯವನ್ನು ಮುಂದಕ್ಕೆ ತಳ್ಳುತ್ತಿದೆ. ವಂಶವಾಹಿಗಳ ಪೃಥಕ್ಕರಣ–ಪರಿಷ್ಕರಣಗಳಿಂದ ಪ್ರಾಯಃ ಮೃತ್ಯುವನ್ನು ಮುಂದೂ\-ಡಲೂ, ಜಯಿಸಲೂ(?) ವಿಜ್ಞಾನಿಯು ಸಮರ್ಥನಾದಾನು. ಆದರೆ ಮಾನವನು ತನಗೆ ಲಭ್ಯವಾದ ಈ ಅಪೂರ್ವ ಅವಕಾಶಗಳನ್ನು ವೈಯಕ್ತಿಕ ಅಭಿವೃದ್ಧಿಗೂ, ಸಮಾಜದ ಕಲ್ಯಾಣಕ್ಕೂ ಕಾರಣವಾಗುವಂತೆ ಉಪಯೋಗಿಸುತ್ತಿರುವನೇ? ಅಥವಾ ಆಸ್ಕರ್ ವೈಲ್ಡ್ ಹೇಳಿದ ಕತೆಯಂತೆ ಬದುಕಿನಲ್ಲಿ ಆತ್ಮಶಕ್ತಿಯ ಅಭಿವೃದ್ಧಿಗೆ ಬೇರಾವ ಮೌಲ್ಯಗಳನ್ನೂ ನೆಚ್ಚಿಕೊಳ್ಳದೆ, ಮಾನವ ಇಂದು ಇಂದ್ರಿಯಪರಾಯಣನಾಗಿ, ದೊರೆತ ಅವಕಾಶಗಳ ದುರುಪಯೋಗ ಮಾಡುತ್ತ, ದುರಂತದತ್ತ ಧಾವಿಸುತ್ತಿರುವನೆ?

‘ಈ ಭೂಮಂಡಲದ ನಿವಾಸಿಗಳ ಪೈಕಿ ಇಬ್ಬರಲ್ಲಿ ಒಬ್ಬನು ಪುಷ್ಟಿಕರ ಆಹಾರವಿಲ್ಲದೆ ಹಸಿವಿನಿಂದ ಬಳಲುತ್ತಿದ್ದಾನೆ. ಪ್ರಪಂಚದ ಮಿಲಿಟರಿ ವೆಚ್ಚದಲ್ಲಿ ನೂರರಲ್ಲಿ ಕೇವಲ ಒಂದು ಅಂಶವನ್ನು ತೆಗೆದಿರಿಸಿದರೆ, ಹಸಿವಿನಿಂದ ಬಳಲುತ್ತಿರುವ ಇಪ್ಪತ್ತುಕೋಟಿ ಮಕ್ಕಳಿಗೆ ಆಹಾರ ಸಾಕಾಗುತ್ತದೆ’–ಇದು ವಿಶ್ವಸಂಸ್ಥೆಯ ತಜ್ಞರ ಅಭಿಮತ. ಆದರೆ, ಹಾಗೆ ಹಣವನ್ನು ಯಾರೂ ತೆಗೆದಿರಿಸ ಹೊರಟಿಲ್ಲ. ಶಸ್ತ್ರಾಸ್ತ್ರಗಳ ಸಂಗ್ರಹಕ್ಕೆ ಹಣ ಸುರಿಯುತ್ತಿದ್ದಾರೆ. ಇದನ್ನು ಕಂಡಾಗ ಬರ್ಟ್ರಾಂಡ್ ರಸ್ಸೆಲ್ಲರ ಒಂದು ಮಾತು ನೆನಪಿಗೆ ಬಾರದೇ ಇರದು: ‘ನಮಗೆ ಶತ್ರುಗಳ ವಿನಾಶವಾಗಬೇಕು ಎನ್ನುವ ಬಗ್ಗೆ ಇರುವ ಆಸೆ ಆಕಾಂಕ್ಷೆ, ನಮ್ಮ ಸ್ನೇಹಿತರ ಅಭ್ಯುದಯವಾಗಬೇಕು ಎನ್ನುವ ಬಗೆಗೆ ಇಲ್ಲ. ಇಂಥ ಮನೋವೃತ್ತಿ ಜಗತ್ತಿನ ಹಿತಕ್ಕೆ ಸಾಧಕವಲ್ಲ.’ ಕೆಲವೊಮ್ಮೆ, ನಿಜವಾದ ಶತ್ರುಗಳ ವಿನಾಶಕ್ಕಿಂತಲೂ, ಕಲ್ಪಿತ ಶತ್ರುಗಳ ವಿನಾಶಕ್ಕಾಗಿಯೇ ನಮ್ಮ ಸಂಪತ್ತು, ಶಕ್ತಿ ವಿನಿಯೋಗವಾಗುತ್ತದೆ!

ಮನುಷ್ಯನ ಹೃದಯದಲ್ಲಿ ದಯೆ, ಕರುಣೆ, ಆತ್ಮಸಂಯಮ, ನಿಃಸ್ವಾರ್ಥತೆ, ಪರೋಪಕಾರ ಬುದ್ಧಿ ಬೆಳೆಯದೆ, ಕೇವಲ ಯಾಂತ್ರಿಕ, ತಾಂತ್ರಿಕ ಪ್ರಗತಿಯಾದ ಮಾತ್ರದಿಂದ ಬಹಳ ಪ್ರಯೋಜನ\-ವಾಗಲಿಲ್ಲ. ಪ್ರಯೋಜನವಾಗದಿರುವುದು ಮಾತ್ರವಲ್ಲ, ವ್ಯಾಪಕ ಪ್ರಮಾಣದಲ್ಲಿ ಕೆಡುಕೂ ಆಗಬಹುದು. ಈ ತಾಂತ್ರಿಕ ಪ್ರಗತಿ ಮನುಷ್ಯರ ಅಭ್ಯುದಯಕ್ಕೆ ಬದಲು ಅವರ ಹಿನ್ನಡೆಗೆ ಕಾರಣವಾಗುತ್ತದೆ. ಆದದ್ದೇನು ಎಂಬುದನ್ನು ಕವಿವಾಕ್ಯಗಳಲ್ಲಿ ಹೇಳುವುದಾದರೆ–

\begin{verse}
ಸಂಪತ್ತಿದೆ ವಿದ್ವತ್ತಿದೆ\\
 ಕೈಸೇರಿದೆ ವಿಜ್ಞಾನ\\
 ವಿಷಯಾಗ್ನಿಯ ಸುಖದಾಸೆಗೆ\\
 ಕಲ್ಲೆಣ್ಣೆಯ ಪಾನ\\ 
 ಭುಗಿಭುಗಿಲೆನೆ ಧಗಧಗಿಸಿದೆ\\
 ರಣರೋಷದ ಕೆಂಗಿಚ್ಚು\\ 
 ಹಲ್ಮಸೆದಿದೆ ಬಿಲ್ಲೆತ್ತಿದೆ\\ 
 ಸಂಗ್ರಹಗೈಯುವ ಹುಚ್ಚು.\footnote{ಕುವೆಂಪು, “ಶತಮಾನ ಸಂಧ್ಯೆ.”}
\end{verse}


\section*{ದೌರ್ಜನ್ಯದ ತಾಂಡವ}

\addsectiontoTOC{ದೌರ್ಜನ್ಯದ ತಾಂಡವ}

ವಿಜ್ಞಾನಿಗಳೂ, ವಿಚಾರಪ್ರಿಯರೂ, ತಮ್ಮ ಜೀವನದ ಸರ್ವಸ್ವವನ್ನು ಪಣವಾಗಿಟ್ಟು, ಸತ್ಯಾನ್ವೇಷಣೆ ಮಾಡುತ್ತ, ವಿವಿಧ ಕ್ಷೇತ್ರಗಳಲ್ಲಿ ಜ್ಞಾನವನ್ನು ಹೆಚ್ಚಿಸುತ್ತಿದ್ದಾರೆ. ಪ್ರಕೃತಿಶಕ್ತಿಗಳ ವಿವಿಧ ತೆರನಾದ ಉಪಯೋಗಗಳನ್ನು ಕಂಡುಹಿಡಿಯುತ್ತಿದ್ದಾರೆ. ಆದರೆ ಅವರ ಶ್ರಮದ ಫಲವು ಮನುಷ್ಯನ ಹೃದಯದಲ್ಲಿರುವ ದುರಿತವನ್ನು ಪೋಷಿಸುತ್ತಿದೆಯಲ್ಲ! ಸತತ ಪರಿಶ್ರಮದಿಂದ ಪ್ರಕೃತಿ ರಹಸ್ಯವನ್ನು ಭೇದಿಸಿ ಅವರು ಕಂಡುಹಿಡಿದ ಶಕ್ತಿಗಳು, ಅಧರ್ಮಿಗಳ ಕೈಯಲ್ಲಿ ಸಿಕ್ಕಿ, ಮನುಕುಲದ ಅಮಂಗಲಕ್ಕೆ, ಅನಾಹುತಕ್ಕೆ ಕಾರಣವಾಗುತ್ತಿವೆಯಲ್ಲ! ವಿಜ್ಞಾನದ ವಿಧ್ವಂಸಕ ಮುಖವನ್ನೂ ನಾವು ನೋಡಬೇಡವೇ?

೧೯೪೫ನೇ ಇಸವಿ, ಆಗಸ್ಟ್ ತಿಂಗಳ ೬ನೇ ದಿನಾಂಕ, ಸೋಮವಾರ. ಬೆಳಗಿನ ಹೊತ್ತು ಎಂಟು ಗಂಟೆಯಾಗಿ ಹದಿನೈದು ನಿಮಿಷ ಕಳೆದಿತ್ತು. ಸುಮಾರು ಅರುವತ್ತು ಸಾವಿರ ಗಂಡಸರು, ಹೆಂಗಸರು, ಮಕ್ಕಳು, ಏಕಕಾಲದಲ್ಲಿ ನಾಶವಾದರು. ಒಂದು ಲಕ್ಷಕ್ಕಿಂತ ಹೆಚ್ಚಿನ ಜನರು ಗಾಯಗೊಂಡರು. ಅತಿ ದೊಡ್ಡ ಬಂದರು ನಾಶವಾಯಿತು. ಮಹಾನಗರ ಒಂದು ಧ್ವಂಸವಾಯಿತು.

ಕೆಲವೇ ನಿಮಿಷಗಳಲ್ಲಿ ಈ ವಿಧ್ವಂಸಕ ಕೃತ್ಯ ನಡೆಯಿತು. ಜಪಾನಿನ ಹಿರೋಶಿಮಾದ ಮೇಲೆ ಅಮೇರಿಕನ್ನರ ಪ್ರಥಮ ಅಣುಬಾಂಬು ಸ್ಫೋಟದ ಪರಿಣಾಮ ಅದು.

ಇತ್ತೀಚಿನ ವರ್ಷಗಳಲ್ಲಿ, ವಿಜ್ಞಾನ ಕೊಲ್ಲುವ ಕಲೆಯಲ್ಲಿ ಅತ್ಯಂತ ಭಯಾನಕ ಪ್ರಗತಿಯನ್ನು ಸಾಧಿಸಿದೆ!

ಹೆಚ್ಚಿನ ವಿವರಣೆ ಅಗತ್ಯವಿಲ್ಲ. ಇತ್ತೀಚಿನ ಪತ್ರಿಕೆಯ ಅಧಿಕೃತ ವರದಿ ಇದು:

‘ವಿವಿಧ ರೀತಿಯ, ಸುಮಾರು ಐವತ್ತು ಸಾವಿರ ಪ್ರಬಲ ಮಾರಕ ಅಣ್ವಸ್ತ್ರಗಳನ್ನು, ಜಗತ್ತಿನ ಐದು ರಾಷ್ಟ್ರಗಳು ಸಂಗ್ರಹಿಸಿಟ್ಟುಕೊಂಡಿವೆ. ಹಿರೋಶಿಮಾದ ಮೇಲೆ ಎಸೆದ ಬಾಂಬಿಗಿಂತ, ಹತ್ತು ಲಕ್ಷ ಪಾಲು ಹೆಚ್ಚು ನಾಶಕಾರಕ ಶಕ್ತಿಯುಳ್ಳವು ಇವು. ಇವುಗಳಲ್ಲಿ ಸುಮಾರು ಶೇಕಡಾ ೯೫ರಿಂದ ೯೭ ಅಸ್ತ್ರಗಳು, ಎರಡು ಪ್ರಮುಖ ಪ್ರಬಲ ರಾಷ್ಟ್ರಗಳ ಕೈಯಲ್ಲಿವೆ. ಉಳಿದವುಗಳು ಚೀನ, ಇಂಗ್ಲೆಂಡ್ ಮತ್ತು ಫ್ರಾನ್ಸ್​–ಈ ದೇಶಗಳ ಹತ್ತಿರ ಇವೆ. ಈ ಎರಡು ಪ್ರಬಲ ರಾಷ್ಟ್ರಗಳು ದಿನ ದಿನವೂ ಹತ್ತು ಮಿಲಿಯ ಡಾಲರುಗಳನ್ನು ಅಣ್ವಸ್ತ್ರಗಳನ್ನು ಪಡೆಯುವುದಕ್ಕಾಗಿಯೂ, ನೂರು ಮಿಲಿಯ ಡಾಲರುಗಳನ್ನು ಅವುಗಳ ನಿರ್ಯಾತ, ಪ್ರಯೋಗ ಮತ್ತು ಕೂಡಿಡುವ ವ್ಯವಸ್ಥೆಗಾಗಿಯೂ ಖರ್ಚು ಮಾಡಿವೆ. ಇಷ್ಟಾದರೂ ಅಣ್ವಸ್ತ್ರಗಳ ತಯಾರಿಕೆಯನ್ನು ನಿಲ್ಲಿಸುವ ಯೋಚನೆ ಏನೂ ಇಲ್ಲ. ತೋರಿಕೆಯ ಯೋಚನೆ, ಒಪ್ಪಂದ ಮಾಡಿಕೊಂಡರೂ, ಯಾರೂ ನಿಲ್ಲಿಸಲು ಸಿದ್ಧರಿಲ್ಲ.’

ಮಾರಕಾಸ್ತ್ರಗಳಿಂದ ಮಾನವಕುಲದ ವಿನಾಶಕಾರ್ಯದಲ್ಲಿ ಸಾಧಿಸಿದ ಪ್ರಗತಿ–ಕ್ಷಿಪ್ರತೆಯನ್ನು ಕುರಿತು ಕೆಲ ವಿವರಗಳು ಇಂತಿವೆ:

‘ಯೂರೋಪಿನಲ್ಲಿ, ಹತ್ತೊಂಬತ್ತನೆ ಶತಮಾನಕ್ಕೆ ಮೊದಲು ಒಂದು ಸಾವಿರ ವರ್ಷಗಳಲ್ಲಿ, ಯುದ್ಧಗಳಲ್ಲಿ ಮಡಿದವರ ಒಟ್ಟುಸಂಖ್ಯೆ ಇಪ್ಪತ್ತು ಸಾವಿರ ಮಿಲಿಯ. ಆದರೆ ಈ ಶತಮಾನ ಒಂದರಲ್ಲೇ ಯುದ್ಧಗಳಲ್ಲಿ ಮಡಿದವರು ಎಪ್ಪತ್ತುಸಾವಿರ ಮಿಲಿಯ. ಇದು ಎರಡನೇ ಮಹಾ\-ಯುದ್ಧ\-ದವರೆಗಿನ ಅಂಕೆ ಸಂಖ್ಯೆ. ಎರಡನೇ ಮಹಾಯುದ್ಧದ ನಂತರವೂ ನಡೆದ ಹಲವು ಯುದ್ಧಗಳಲ್ಲಿ, ಒಂದು ಕೋಟಿಗೂ ಹೆಚ್ಚು ಜನ ಮಡಿದಿದ್ದಾರೆ!’


\section*{ವಿಜ್ಞಾನದ ಕರಾಳ ವದನ}

\addsectiontoTOC{ವಿಜ್ಞಾನದ ಕರಾಳ ವದನ}

ನಮ್ಮ ಸುಖ ಸೌಕರ್ಯಗಳ ಯತ್ನಕ್ಕೆ ವಿಜ್ಞಾನದ ದೈತ್ಯಶಕ್ತಿಯ ಬೆಂಬಲ ದೊರೆತಿದೆ. ಆದರೆ ಅದರಿಂದ ಈ ಸುಖಸೌಕರ್ಯಗಳ ಆಕರ್ಷಣೆ, ಅವುಗಳ ಬಯಕೆಯ ತೀವ್ರತೆ, ಅವು ತರುವ ಬಂಧನ–ಇವುಗಳ ತೀವ್ರತೆಯೂ ಹೆಚ್ಚುತ್ತಿದೆ.

‘ಬ್ರಿಟನ್ನಿನಲ್ಲಿ ವಾಹನಗಳ ಆಘಾತದಿಂದ ಒಂದು ವರ್ಷದಲ್ಲಿ ಮಡಿಯುವವರ ಸಂಖ್ಯೆ ಏಳು ಸಾವಿರ. ಗಾಯಗೊಳ್ಳುವವರು ಒಂದು ಲಕ್ಷ ಮಂದಿ. ಅಮೇರಿಕದಲ್ಲಿ ಒಂದು ವರ್ಷದಲ್ಲಿ ಅವಘಡಗಳಲ್ಲಿ ಮಡಿಯುವವರು ನಲ್ವತ್ತೈದು ಸಾವಿರ ಮಂದಿ. ಅಪಘಾತದಿಂದ ಗಾಯಗೊಳ್ಳುವವರು ಹಲವು ಲಕ್ಷಮಂದಿ.’

ನಿಜ, ವಿಜ್ಞಾನವು ನಮ್ಮ ಸುಖವನ್ನು ಅಪಾರ ಮಟ್ಟದಲ್ಲಿ ಹೆಚ್ಚಿಸಿದೆ. ‘ನಮಗಿಂದು ಸೌಕರ್ಯ ಹೆಚ್ಚಾಗಿದೆ, ಆಯುಸ್ಸೂ ಹೆಚ್ಚಿದೆ. ಆದರೆ ಕುಡಿತವೂ ಹೆಚ್ಚಿದೆ’–ಇದು ೧೯೮೨ನೇ ಇಸವಿಯಲ್ಲಿ, ‘ಲಂಡನ್ ಟೈಂಸ್​’ ಪತ್ರಿಕೆಯಲ್ಲಿ ಪ್ರಕಟಗೊಂಡ ಲೇಖನಮಾಲೆಯ ಸಾರ.

ನಾಗರಿಕತೆಯ ಪ್ರಗತಿಯ ಬಗ್ಗೆ ಪ್ರತಿಯೊಬ್ಬನಿಗೂ ಹೆಮ್ಮೆ ಇದೆ. ಆದರೆ ಮದ್ಯಕುಡಿದು ಪ್ರಾಣಿಗಿಂತಲೂ ಹೊಲಸುಮಟ್ಟದಲ್ಲಿ ವರ್ತಿಸುವವರ ಸಂಖ್ಯೆ ಹೆಚ್ಚುತ್ತಿದೆಯಲ್ಲ!

ಪಶ್ಚಿಮ ಜರ್ಮನಿಯಲ್ಲಿ ೧೦೯೧ರಲ್ಲಿ ೧೫೦೦ ಕೋಟಿ ರೂಪಾಯಿ ಮದ್ಯ ಮತ್ತು ಸಿಗರೇಟು\-ಗಳಿಗಾಗಿ ಖರ್ಚಾಗಿತ್ತು. ಈಗ ಈ ಸಂಖ್ಯೆ ಎರಡು ಪಾಲೋ ನಾಲ್ಕು ಪಾಲೋ ಹೆಚ್ಚಿರಬೇಕು. ಮದ್ಯಕುಡಿದು ಕಾರು ನಡೆಯಿಸುವುದರಿಂದ ಉಂಟಾಗುವ ಅವಘಡಗಳು ವರ್ಷದಿಂದ ವರ್ಷಕ್ಕೆ ಹೆಚ್ಚುತ್ತಲಿವೆ. ಬ್ರಿಟನ್ನಿನಲ್ಲಿ ವರ್ಷವೊಂದಕ್ಕೆ, ನಾಲ್ಕುಸಾವಿರ ಮಿಲಿಯ ಲೀಟರಿಗೂ ಹೆಚ್ಚಾಗಿ ಮದ್ಯ ಸೇವನೆ ನಡೆದಿದೆ. ಮದ್ಯ ಮತ್ತು ಸಿಗರೇಟುಗಳ ಮೇಲಿನ ತೆರಿಗೆಯಿಂದ ಅಲ್ಲಿಯ ಸರಕಾರಕ್ಕೆ ವರ್ಷವೊಂದರಲ್ಲಿ ಬರುವ ಆದಾಯ ೧೧೭೦ ಕೋಟಿ ರೂಪಾಯಿಗಳು!

‘ಅತಿ ಹೆಚ್ಚಿನ ವೈಜ್ಞಾನಿಕ ಸಾಧನ ಸಂಪತ್ತಿರುವ ಅಮೇರಿಕದಲ್ಲಿ ಅರ್ಧಗಂಟೆಗೊಂದು ಕೊಲೆ, ಅರ್ಧಗಂಟೆಗೊಂದು ಅತ್ಯಾಚಾರ, ಗಂಟೆಗೆ ಹತ್ತು ದರೋಡೆ, ಒಂದು ಗಂಟೆಯಲ್ಲಿ ನಲ್ವತ್ತು ಕಾರುಗಳ ಕಳವು–ಕಳ್ಳಸಾಗಣೆ’ ಎಂದು ತಜ್ಞರು ವರದಿ ಮಾಡುತ್ತಾರೆ. ಇವುಗಳ ಒಟ್ಟು ಸಂಖ್ಯೆ ವರ್ಷಕ್ಕೆ ಎರಡು ಮಿಲಿಯಕ್ಕಿಂತ ಹೆಚ್ಚು. ಕೊಲೆಗಡುಕರು ಅತ್ಯುತ್ತಮ ವೈಜ್ಞಾನಿಕ ಸಲಕರಣೆಗಳಿಂದ ಸುಸಜ್ಜಿತರಾಗಿ ಬಹಳ ನಾಜೂಕಾಗಿ ತಮ್ಮ ಕಾರ್ಯವನ್ನು ಕೈಗೊಂಡು ಶೋಧನೆಗೆ ಪೋಲೀಸರು ಮಾಡುವ ಭಗೀರಥ ಪ್ರಯತ್ನವನ್ನು ವಿಫಲಗೊಳಿಸುತ್ತಾರೆ! ಕೊಲೆ ಪತ್ತೆಯಾಗುವುದು ನಾಲ್ಕರಲ್ಲಿ ಒಂದು ಮಾತ್ರ!’

೧೯೦೦ರಿಂದ ಇಂದಿನವರೆಗೆ ಕೇವಲ ಕೊಲೆಗಡುಕರ ಗುಂಡುಗಳಿಂದ ಹತರಾದವರು ಎಂಟು ಲಕ್ಷಮಂದಿ! ಎಂದರೆ, ಜನಹೃದಯದ ದುರಿತ, ದೌರ್ಜನ್ಯಗಳ ಪರಿಚಯ ಆಗುತ್ತದೆ. ಕಳೆದ ಒಂದು ವರ್ಷದಲ್ಲೇ ಗುಂಡು ಹೊಡೆದುಕೊಂಡು ಆತ್ಮಹತ್ಯೆ ಮಾಡಿಕೊಂಡ ಮಂದಿ ಹತ್ತು ಸಾವಿರ ಎಂದಾಗ ಅವರ ಮಾನಸಿಕ ತುಮುಲ ಮತ್ತು ವ್ಯಗ್ರತೆ ಎಂಥ ತೀವ್ರ ಮಟ್ಟದ್ದು ಎನ್ನುವ ಸಂಗತಿ ಅರ್ಥವಾದೀತು.

ಮಾನಸಿಕ ತುಮುಲ ಮತ್ತು ವ್ಯಗ್ರತೆಯನ್ನು ಸೂಚಿಸುವ ಇತರ ವಿವರಗಳನ್ನೂ ತಜ್ಞರು ನೀಡುತ್ತಾರೆ. ‘ಅಮೇರಿಕದಲ್ಲಿ ಒಂದು ವರ್ಷದಲ್ಲಿ ಜನ ನುಂಗುವ ಎಸ್ಪೆರಿನ್ ಅಥವಾ ತತ್ಸಮಾನ ಮಾತ್ರೆಗಳ ಪ್ರಮಾಣ ಇಪ್ಪತ್ತೆಂಟು ಸಾವಿರ ಟನ್. ನರಗಳ ಉದ್ವೇಗ ತಣಿಸಲು ನೂರಾರುಕೋಟಿ ರೂಪಾಯಿಗಳ ಔಷಧ ಸೇವಿಸುತ್ತಾರೆ. ನಿದ್ರಾಸುಖದಿಂದ ವಂಚಿತರಾದ ಅಸಂಖ್ಯ ಮಂದಿ ನಿದ್ರೆ ಗುಳಿಗೆಗಳನ್ನು ನುಂಗಲೇಬೇಕಾಗಿದೆ. ಇನ್ನೊಂದು ಹೊಸಚಟ, ಮತ್ತುಬರಿಸುವ ಮಾದಕ ದ್ರವ್ಯ ಸೇವನೆಯ ಚಟ, ಶಾಲಾ ಕಾಲೇಜುಗಳಲ್ಲಿ ಓದುತ್ತಿರುವ ಯುವಕ ಯುವತಿಯರಲ್ಲಿ ಸಾಕಷ್ಟು ಪ್ರಸಾರವಾಗುತ್ತಿದೆ!’

೧೯೭೦ನೇ ಇಸವಿ ನವೆಂಬರ್ ತಿಂಗಳ ‘ರೀಡರ್ಸ್ ಡೈಜೆಸ್ಟ್​’ ಮಾಸಪತ್ರಿಕೆಯಲ್ಲಿ, ಆರ್ಟ್ ಲಿಂಕ್ ಲೆಟರ್ ಒಂದು ಹೃದಯವಿದ್ರಾವಕ ಲೇಖನ ಬರೆದಿದ್ದ ತನ್ನ ಇಪ್ಪತ್ತು ವರ್ಷ ವಯಸ್ಸಿನ ಮಗಳು ಎಲ್. ಎಸ್. ಡಿ. ಸೇವಿಸಿ ಮಹಡಿಯಿಂದ ಹಾರಿ ಪ್ರಾಣ ಕಳೆದುಕೊಂಡ ಘಟನೆಯ ವಿವರವನ್ನು ನೀಡಿ, ಅಮೇರಿಕದ ತಂದೆತಾಯಿಗಳನ್ನು ಹೀಗೆಂದು ಎಚ್ಚರಿಸಿದ್ದ:

‘ನಮ್ಮ ಮೇಲೆ ಬಿದ್ದಿರುವ ದುರಂತದ ನೆರಳು ಪ್ರತಿಯೊಂದು ಕುಟುಂಬದ ಮೇಲೂ ಬಿದ್ದಿದೆ. ಶಿಕ್ಷಣದ ಮಟ್ಟ, ಐಶ್ವರ್ಯ, ಸಮಾಜದಲ್ಲಿ ಸ್ಥಾನಮಾನ ಮುಂತಾದ ಯಾವುದೇ ಸಂಗತಿಗಳೂ ಅದಕ್ಕೆ ಅಡ್ಡಿಯಾಗವು. ನಿಮ್ಮ ಹುಡುಗನು ಇಂದು ಇಡೀ ಅಮೇರಿಕದಲ್ಲಿ ಯಾವುದೇ ವಿಶ್ವ\-ವಿದ್ಯಾಲಯ\-ದಲ್ಲಿರಲಿ, ಪ್ರೌಢಶಾಲೆಯಲ್ಲಿರಲಿ ಅಥವಾ ಪ್ರಾಥಮಿಕ ಶಾಲೆಯಲ್ಲಿರಲಿ, ಅವನು ಈಗಲೇ, ಇದೇ ಕ್ಷಣದಲ್ಲೆ ಅಕ್ರಮ ಮಾದಕವಸ್ತುಗಳ ಸಂಪರ್ಕಕ್ಕೆ ಬರುತ್ತಿದ್ದಾನೆ. ಹಾಗಿಲ್ಲವೆಂದು ಭಾವಿಸಿದಲ್ಲಿ ಅಂಥ ಕುರುಡು ನಂಬಿಕೆ ಇನ್ನೊಂದಿಲ್ಲ.’

ಸುಖಲೋಲುಪತೆ, ಮತ್ತು ಸ್ವೇಚ್ಛಾಚಾರಕ್ಕೆಳಸಿದ, ಸ್ವಾತಂತ್ರ್ಯಕ್ಕೆ, ವಿಜ್ಞಾನ ತಾಂತ್ರಿಕತೆಯ ಸಹಾಯ, ಸಹಕಾರ ಎಂಥ ಕ್ಷೋಭೆಯನ್ನುಂಟುಮಾಡಿದೆ! ವಿಯೆಟ್​ನಾಮ್ ಯುದ್ಧದಲ್ಲಿ\break ನಲವತ್ತೈದು ಸಾವಿರ ಯೋಧರು ಮಡಿದರೆ, ಮಾದಕದ್ರವ್ಯ ಸೇವಿಸಿ ಸತ್ತವರ ಸಂಖ್ಯೆ ಒಂದು ಲಕ್ಷದ ನಲವತ್ತು ಸಾವಿರ! ಈ ಮಾದಕ ದ್ರವ್ಯಗಳನ್ನು ಮಾರುವ ಹದಿಮೂರು ಬೃಹತ್ ಸಂಸ್ಥೆಗಳಿವೆ ಎಂದು ಪತ್ತೆಹಚ್ಚಲಾಗಿದೆ. ಇವುಗಳಲ್ಲಿ ಒಂದು ಕಳ್ಳವ್ಯಾಪಾರಿ ಸಂಸ್ಥೆಯು ಒಂದು ವರ್ಷದಲ್ಲಿ ನಡೆಸುತ್ತಿದ್ದ ವ್ಯಾಪಾರ ಎರಡು ಸಾವಿರ ಕೋಟಿ ರೂಪಾಯಿಗಳಷ್ಟು!


\section*{ನೈತಿಕತೆ ನಿರ್ನಾಮ}

\addsectiontoTOC{ನೈತಿಕತೆ ನಿರ್ನಾಮ}

ವಿದೇಶಗಳಲ್ಲಿ ತಜ್ಞರೂ, ಸಂಶೋಧಕರೂ, ಸಾಮಾಜಿಕ ಆರೋಗ್ಯದ ಕ್ಷೇಮಪಾಲಕರೂ, ಆಧುನಿಕ ಯುಗದ ನೈತಿಕ ಕುಸಿತವನ್ನು ಮತ್ತು ವ್ಯಕ್ತಿ ಸ್ವಾತಂತ್ರ್ಯದ ಹೆಸರಿನಲ್ಲಿ ನಡೆಯುವ ಸ್ವೇಚ್ಛಾಚಾರದ ದುಷ್ಪರಿಣಾಮಗಳ ಚಿತ್ರಗಳನ್ನು ನೀಡಿ, ಜನರನ್ನು ಎಚ್ಚರಿಸುವುದುಂಟು. ಸ್ತ್ರೀ ಪುರುಷರು ಹೇಗೆ ಹೆಚ್ಚು ಹೆಚ್ಚು ಸ್ವಾರ್ಥಪರಾಯಣರಾಗುತ್ತಿದ್ದಾರೆ, ಕಾಮದ ಕೆಸರಿನಲ್ಲಿ ಶೀಲದ ಚೆಲ್ಲಾಟಕ್ಕೆ ಹೇಗೆ ಸ್ವಾತಂತ್ರ್ಯ, ಸ್ವಾಭಾವಿಕತೆ, ನೈಜತೆಯ ಮೆರುಗು ನೀಡಲಾಗುತ್ತಿದೆ, ಗರ್ಭ ನಿರೋಧಕ ಸಾಮಗ್ರಿಗಳ ಬಳಕೆ ಹೇಗೆ ಹೆಚ್ಚುತ್ತಿದೆ, ಭ್ರೂಣ ಹತ್ಯೆಯ ಸಂಖ್ಯೆ ಹೇಗೆ ಏರುತ್ತಲಿದೆ, ವಿವಾಹ ವಿಚ್ಛೇದದ ಆಧಿಕ್ಯ ಹೇಗೆ ಯಾವ ಮಟ್ಟಕ್ಕಿಳಿದು ಕುಟುಂಬನಾಶ ಹೇಗೆ ಆಗುತ್ತಲಿದೆ, ಗುಹ್ಯರೋಗಗಳು ಹೇಗೆ ಪ್ರಸಾರವಾಗುತ್ತಿವೆ–ಎಂಬುದನ್ನೆಲ್ಲ ಅಂಕೆ ಸಂಖ್ಯೆಗಳ ಆಧಾರದಿಂದ ಅವರು ವಿವರಿಸುತ್ತಾರೆ. ನಮ್ಮ ದೇಶದಲ್ಲಿ ವಿಜ್ಞಾನವನ್ನು ಹಾಡಿ ಹೊಗಳುವವರು, ಮೌಲ್ಯಗಳ ಮಹಿಮೆಯನ್ನು ಗುರುತಿಸದವರು, ಈ ವಿವರಣೆಗಳತ್ತ ಗಮನ ಹರಿಸಬೇಕು. ಮಕ್ಕಳ ಭವಿಷ್ಯ ನಿರ್ಮಾತೃಗಳಾದ ಅಧ್ಯಾಪಕರೂ, ಶಿಕ್ಷಣವೇತ್ತರೂ, ಪ್ರತಿಯೊಂದು ವಿಷಯದಲ್ಲೂ ಪಶ್ಚಿಮವನ್ನು ಅನುಕರಿಸ ಹೊರಟ ವಿದ್ಯಾ\-ವಂತರೂ, ಈ ವಿಷಯಗಳನ್ನು ಮನನ ಮಾಡಿ ಅದರ ಕಾರಣ, ಪರಿಣಾಮಗಳನ್ನು ಗುರುತಿಸಬೇಕು.

ವಿಜ್ಞಾನದ ದ್ರುತಗತಿಯ ಪ್ರಗತಿಯ ಜೊತೆಜೊತೆಗೇ, ಪಶ್ಚಿಮದಲ್ಲಿ ಧಾರ್ಮಿಕ ಶ್ರದ್ಧೆಯು ದುರ್ಬಲವಾಗುತ್ತ ಹೋಯಿತು. ಸ್ವಾತಂತ್ರ್ಯಪ್ರಿಯತೆಯು ಸ್ವೇಚ್ಛಾಚಾರಕ್ಕೆಳಸಿತು–ಮುಖ್ಯವಾಗಿ ಗಂಡುಹೆಣ್ಣುಗಳ ಸಂಬಂಧಗಳಲ್ಲಿ. ಮನೋವಿಶ್ಲೇಷಣ ಜನಕನಾದ ಫ್ರಾಯ್ಡ್, ಧಾರ್ಮಿಕ ಕಟ್ಟು\-ಪಾಡು\-ಗಳಿಂದ ಮನಸ್ಸನ್ನು ನಿಯಂತ್ರಿಸುವ ವಿಧಾನವನ್ನು ಅಲ್ಲಗಳೆದ. ಇದು ಮುಂದೆ, ಮಕ್ಕಳ ಅಸ್ತವ್ಯಸ್ತ ಬದುಕಿಗೆ ನಾಂದಿಯಾಯಿತು. ‘ಅಸಹಾಯ ಪುಟ್ಟಶಿಶು ತನ್ನ ತಂದೆತಾಯಿಗಳ ಆಸರೆ, ಆಲಂಬನೆಗಳಲ್ಲಿ ಬೆಳೆಯುತ್ತದೆ. ವಯಸ್ಸಾದ ಬಳಿಕ ಕಲ್ಪಿತ ಆಸರೆ, ಆಲಂಬನೆಗಳು ಅವನಿಗೆ ಬೇಕು. ದೇವರು, ಧರ್ಮ–ಇವುಗಳೆಲ್ಲ ಅಂಥ ಕಲ್ಪಿತ ಆಲಂಬನೆಗಳು. ಮನುಷ್ಯ ಬೌದ್ಧಿಕ ಪ್ರಬುದ್ಧತೆ ಮತ್ತು ವೈಜ್ಞಾನಿಕ ಮನೋವೃತ್ತಿಯನ್ನು ಬೆಳೆಸಿಕೊಂಡಮೇಲೆ, ದೇವರ ಮೇಲೆ ನಂಬಿಕೆ ಮತ್ತು ಧಾರ್ಮಿಕ ಕಟ್ಟುಪಾಡುಗಳು ಮಾಯವಾಗುತ್ತವೆ’ ಎಂದ ಫ್ರಾಯ್ಡ್. ‘ಮನುಷ್ಯನ ಬದುಕಿನಲ್ಲಿ ಕಾಮವೇ ಪ್ರಧಾನ ಪ್ರೇರಕಶಕ್ತಿ. ನಿಯಮ, ಸಂಯಮಗಳ ಶೃಂಖಲೆಯಿಂದ ಮನುಷ್ಯ ತನ್ನನ್ನು ಬಂಧಿಸಿಕೊಂಡು ವ್ಯಾಧಿಗ್ರಸ್ತನಾಗಬೇಕಿಲ್ಲ’ ಎಂದನಾತ. ಸ್ವಚ್ಛಂದ ಲೈಂಗಿಕತೆಗೆ ಪೋಷಕವಾದ ಫ್ರಾಯ್ಡನ ದೋಷಪೂರಿತ ಸಿದ್ಧಾಂತ ಪಶ್ಚಿಮದಲ್ಲಿ ವ್ಯಕ್ತಿಗೂ, ಸಮಾಜಕ್ಕೂ, ಎಂಥ\break ಹಾನಿಯನ್ನೂ, ನೈತಿಕ ಕುಸಿತವನ್ನೂ, ಉಂಟು ಮಾಡಿದೆ ಎಂಬುದನ್ನು ಸೊರೊಕಿನ್ ಸಾಕ್ಷ್ಯಾಧಾರ ಸಹಿತ ಸಾಬೀತುಪಡಿಸುತ್ತಾರೆ. ಫ್ರಾಯ್ಡನ ಲೈಂಗಿಕ ಸ್ವಾತಂತ್ರ್ಯ ಸಿದ್ಧಾಂತಕ್ಕೆ ಅತಿ ಹೆಚ್ಚಿನ ಪ್ರಾಧಾನ್ಯ ನೀಡಿ, ಇಂದು ಸಮಾಜ ತೀವ್ರ ಸಂಕಟಗ್ರಸ್ತವಾಗಿದೆ ಎನ್ನುತ್ತಿದ್ದಾರೆ. ಅಮೇರಿಕದ ವಿಶ್ವವಿದ್ಯಾಲಯಗಳ ತಜ್ಞ ಪ್ರಾಧ್ಯಾಪಕರು ಹಿಂದೆ ಬೆತ್ತದ ಛಡಿ ಏಟಿನ ಶಿಕ್ಷೆಯನ್ನು ಬಿಟ್ಟುಬಿಟ್ಟು ಮಕ್ಕಳನ್ನು ರಕ್ಷಿಸಿ ಎಂದಂತೆ, ಇಂದು ಫ್ರಾಯ್ಡನ ಸಿದ್ಧಾಂತವನ್ನು ಬಿಟ್ಟುಬಿಟ್ಟು ಮಕ್ಕಳನ್ನು ರಕ್ಷಿಸಿ ಎನ್ನುವ ಚಳುವಳಿ ಪ್ರಾರಂಭಿಸಬೇಕಾಗಿದೆ ಎಂದವರೂ ಅವರೇ. ಆದರೆ ಈ ಬೇಡಿಯಿಂದ ಬಿಡಿಸಿಕೊಳ್ಳುವುದು ಅಷ್ಟು ಸುಲಭವೆ? ಕಳೆದ ಇನ್ನೂರು ವರ್ಷಗಳಿಂದ ಕಾಮ ಪ್ರಚೋದನೆಗೆ ಪ್ರಾಧಾನ್ಯವನ್ನು ನೀಡುತ್ತ ಬಂದು ಅದನ್ನು ಥಟ್ಟನೆ ನಿಲ್ಲಿಸಲು ಸಾಧ್ಯವಾಗುವುದೇ? ಸಮಾಜಶಾಸ್ತ್ರಜ್ಞ ಸೊರೊಕಿನ್ ಹೇಳುತ್ತಾರೆ: ‘ಕಳೆದ ಇನ್ನೂರು ವರ್ಷಗಳಿಂದ, ಮುಖ್ಯವಾಗಿ–ಕಳೆದ ಹಲವು ದಶಕಗಳಿಂದ, ನಮ್ಮ ಸಂಸ್ಕೃತಿಯ ಎಲ್ಲ ವಿಭಾಗಗಳೂ, ಅತಿ ಲೈಂಗಿಕತೆಯ ಧಾಳಿಗೊಳಗಾಗಿವೆ. ನಮ್ಮ ನಾಗರಿಕತೆ ಲೈಂಗಿಕತೆಯಲ್ಲಿ ಎಷ್ಟು ಮುಳುಗಿದೆ ಎಂದರೆ ಬದುಕಿನ ಎಲ್ಲೆಡೆಗಳಿಂದಲೂ ಅದು ಇಂದು ಹೊರ ಹೊಮ್ಮುತ್ತಿದೆ.’

ಸಾಹಿತ್ಯ, ಕತೆಕಾದಂಬರಿಗಳು, ಕಲೆ, ಚಿತ್ರ, ಸಂಗೀತ, ಚಲಚ್ಚಿತ್ರ, ಟೆಲಿವಿಷನ್, ಪತ್ರಿಕೆ, ಮ್ಯಾಗಜಿನ್, ಜಾಹಿರಾತು–ಈ ಎಲ್ಲ ಕ್ಷೇತ್ರಗಳಲ್ಲೂ ಲೈಂಗಿಕತೆಯ ಹಾವಳಿ ಹೇಗೆ ನಡೆದಿದೆ, ಎಂಥ ದುಷ್ಪರಿಣಾಮವನ್ನು ಬೀರಿದೆ ಎಂಬುದನ್ನು ವಿಪುಲ ಸಾಕ್ಷ್ಯಾಧಾರಗಳಿಂದ ಆಳವಾಗಿ ಚಿಂತನ ಮಂಥನ ನಡೆಯಿಸಿದ್ದಾರೆ ಸೊರೊಕಿನ್. ೧೯೩೦ರಲ್ಲಿ ನಡೆಯಿಸಿದ ಒಂದು ಅಧ್ಯಯನದ ಪ್ರಕಾರ ಪ್ರದರ್ಶಿತ ನೂರು ಚಲಚ್ಚಿತ್ರಗಳಲ್ಲಿ ೪೫ ಚಿತ್ರಗಳು ಲೈಂಗಿಕತೆಗೆ ಸಂಬಂಧಿಸಿದವುಗಳು. ೨೮ ಕೊಲೆ ಹಾಗೂ ಲೈಂಗಿಕತೆಗೆ ಸಂಬಂಧಿಸಿದ ಚಿತ್ರಗಳು. ಅಂದಿನಿಂದ ಇಂಥ ಚಿತ್ರಗಳ ಪ್ರಮಾಣ ಹೆಚ್ಚಿದೆಯೇ ಹೊರತು ಕುಸಿದಿಲ್ಲ. ಇತ್ತೀಚೆಗೆ ಇದು ಎಂಥ ಅತಿರೇಕಕ್ಕೆಳಸಿದೆ ಎಂಬುದನ್ನು ‘ಸ್ವೇಟ್ಸ್ ಮನ್​’ ಪತ್ರಿಕೆ, ತನ್ನ ಸಮೀಕ್ಷೆಯ ಫಲವಾಗಿ ಹೀಗೆ ವರದಿ ಮಾಡಿದೆ:

‘ಪಶ್ಚಿಮ ಜರ್ಮನಿಯಲ್ಲಿ ಹಲವಾರು ವರ್ಷಗಳಿಂದ ಲೈಂಗಿಕ ಕ್ರಿಯೆಗಳನ್ನು ತೋರಿಸುವ ವಿಡಿಯೋ ಚಿತ್ರಗಳ ಪ್ರದರ್ಶನಕ್ಕೆ ಅತಿ ಹೆಚ್ಚಿನ ನೂಕುನುಗ್ಗಲು ಇದೆ. ಆದರೆ ಇತ್ತೀಚೆಗೆ ಅಂಥ ಚಿತ್ರಗಳಿಂದ ತೃಪ್ತರಾಗದೆ ರಕ್ತಹೆಪ್ಪುಗಟ್ಟುವಂಥ ಅಸಹ್ಯ, ಅಶ್ಲೀಲ, ಹಿಂಸಾರತಿ ಹಾಗೂ ಅತಿರೇಕದ ಅತ್ಯಾಚಾರಗಳನ್ನು ವಿವರಿಸುವ ಚಿತ್ರಗಳಿಗೆ ತೀವ್ರ ಬೇಡಿಕೆ ಉಂಟಾಗಿದೆ.’

ಸುಮಾರು ಮೂವತ್ತು ವರ್ಷಗಳ ಹಿಂದೆಯೇ ಟೆಲಿವಿಷನ್ನಿನ ದುರುಪಯೋಗವನ್ನು ಕುರಿತು ಡಾ.\ ಸೊರೊಕಿನ್ನರು ಸಂಶೋಧನಾತ್ಮಕ ವಿವರಗಳನ್ನು ನೀಡಿದ್ದರು. ಈ ಉಪಕರಣವನ್ನು ಜನಮನ\-ಮಾಲಿನ್ಯವೃದ್ಧಿಗಾಗಿ ಹೇಗೆ ಬಳಸುತ್ತಾರೆಂಬುದನ್ನು ಕುರಿತು ವಿಶೇಷ ಅಧ್ಯಯನವನ್ನು ಮಾಡಿ ಫಲಿತಾಂಶವನ್ನೂ ಪ್ರಕಟಿಸಿದ್ದರು. ‘ಟೆಲಿವಿಷನ್ ಮೂಲಕ ಮಿಲಿಯಗಟ್ಟಲೆ ಮನೆಗಳಿಗೆ, ಕಾಮ\-ಪ್ರಚೋದಕ, ಮದ್ಯಪಾನಮತ್ತ ರಾತ್ರಿಕ್ಲಬ್ಬುಗಳ ವಾತಾವರಣವನ್ನೂ, ಹೊಲಸು ವ್ಯವಹಾರ, ಕೊಲೆ, ಸುಲಿಗೆ, ಅತಿ ಲೈಂಗಿಕತೆಯ ನಾಟಕಪ್ರದರ್ಶನಗಳನ್ನೂ ಹರಿಯಗೊಡುತ್ತಾರೆ. ಸಾಮಾನ್ಯ ಸಿನೆಮಾಗಳಲ್ಲಿ ಪ್ರದರ್ಶಿತವಾಗುವುದಕ್ಕಿಂತಲೂ ಕೆಳಮಟ್ಟದ ಹೊಲಸನ್ನು ಟೆಲಿವಿಷನ್​ಗಳಲ್ಲಿ ಬಿತ್ತರಿ\-ಸುತ್ತಾರೆ. ಈ ಕೊಳಕಿನಲ್ಲಿ ಮುಳುಗೆದ್ದ ಮೇಲೆ ಹೆಚ್ಚಿನವರು ದೈಹಿಕ, ಮಾನಸಿಕ, ನೈತಿಕವಾಗಿ ಈ ಕೊಳೆಯಿಂದ ಬಿಡಿಸಿಕೊಳ್ಳಲಾರರು. ಒಬ್ಬಿಬ್ಬರು ಚೇತರಿಸಿಕೊಳ್ಳುವವರಿರಬಹುದು. ಆದರೆ ಲಕ್ಷ ಲಕ್ಷ ಜನ ನೈತಿಕವಾಗಿ ಮೇಲೇಳಬೇಕೆಂಬ ಪ್ರಜ್ಞೆಯನ್ನೂ ಕಳೆದುಕೊಳ್ಳುವ ದೌರ್ಭಾಗ್ಯಕ್ಕೆ ಒಳಗಾಗುವರು’ ಎಂದವರು ಹೇಳಿದ್ದರು.

ಇಂದ್ರಿಯಚಪಲತೆಯನ್ನು ವಿಪರೀತ ಕೆರಳಿಸುತ್ತ, ದೈಹಿಕ ಭೋಗಾಕಾಂಕ್ಷೆಯನ್ನು ಅತಿರೇಕಕ್ಕೆ ಕೊಂಡೊಯ್ಯುತ್ತ, ನೀಚಮಟ್ಟದ ಸ್ವಾರ್ಥಕ್ಕೆ ಮನುಷ್ಯನನ್ನು ಎಳೆಯುತ್ತ, ಅಶಿಸ್ತು, ಅನೈತಿಕತೆಗೆ ಅವನನ್ನು ತಳ್ಳುತ್ತ, ಬದುಕಿನ ಮೌಲ್ಯಗಳನ್ನು ನಾಶಗೈಯುತ್ತ ಸಾಗುವ ಆಧುನಿಕ ಮನೋವೃತ್ತಿಗೆ ವಿಜ್ಞಾನದ ಸಹಾಯ ಸಹಕಾರ ಹೇರಳವಾಗಿಯೇ ಇದೆ! ಇದರ ಪರಿಣಾಮ?

ಹತ್ತೊಂಭತ್ತನೇ ಶತಮಾನದ ಅಂತ್ಯದವರೆಗೂ ಅಧಿಕ ಪ್ರಮಾಣದಲ್ಲಿ ಮದುವೆಗೊಂದು ಧಾರ್ಮಿಕ ತಳಹದಿ, ಹೆಂಗಸರಲ್ಲಿ ಪಾತಿವ್ರತ್ಯಕ್ಕೆ ಗೌರವ, ಕುಟುಂಬದಲ್ಲಿ ಮಕ್ಕಳ ಬಗ್ಗೆ ಮಮತೆ, ಅವರ ಸರ್ವತೋಮುಖ ಬೆಳವಣಿಗೆಯ ಬಗ್ಗೆ ಮನೆಯವರೆಲ್ಲರ ಆಸಕ್ತಿ, ವೃದ್ಧ ಮಾತಾಪಿತೃಗಳಿಗೆ ಮನೆಯಲ್ಲಿ ಸ್ಥಾನಮಾನ–ಇವು ಕಂಡು ಬರುತ್ತಿದ್ದರೆ, ಇಂದು ಆ ಲಕ್ಷಣಗಳೆಲ್ಲ ಮಾಯವಾಗಿವೆ!

ತಂದೆತಾಯಿಗಳ ಬದಲಾವಣೆಯಿಂದ ಮಕ್ಕಳ ಪರಿಸ್ಥಿತಿ ಚಿಂತಾಜನಕವಾಗಿದೆ. ಅಮೇರಿಕ ಸರಕಾರದ ಅನಾಥಾಲಯಗಳಲ್ಲಿ ತಂದೆತಾಯಿಗಳಿಗೆ ಬೇಡವಾದ ಮಕ್ಕಳು ಬಹುಸಂಖ್ಯೆಯಲ್ಲಿದ್ದಾರೆ. ಹೆತ್ತವರ ಪ್ರೀತಿ, ವಿಶ್ವಾಸಗಳನ್ನು ಅರಿಯದ ಅವರೇ ಮುಂದಿನ ಪ್ರಜೆಗಳು! ಈಗಾಗಲೇ ವಿದ್ಯಾರ್ಥಿಗಳ ಅಶಿಸ್ತು, ಅವಿಧೇಯತೆ, ಸ್ವಾರ್ಥತೆ, ಹಿಂಸಾರತಿ ತೀವ್ರವಾಗುತ್ತಲಿದೆ! ತಮ್ಮ ಪ್ರಾಯಸ್ಥ ಮಕ್ಕಳಿಂದ ರಕ್ಷಣೆ ಅಥವಾ ಆರೈಕೆಯನ್ನು ಪಡೆಯದ ವೃದ್ಧರು, ನಿರ್ಗತಿಕರಾಗಿ, ಸಂಕಟಮಯ ಏಕಾಂತವನ್ನು ಸಹಿಸಲಾರದೇ ಆತ್ಮಹತ್ಯೆಯ ಹಾದಿ ಹಿಡಿದಿದ್ದಾರೆ.

ಡಾ.\ ಕೀನ್ಸೆ ಹೇಳಿದ: ‘ವಿವಾಹವಿಚ್ಛೇದ, ಮದುವೆಯ ಮೊದಲು ಮತ್ತು ನಂತರದ ದುರ್ನಡತೆ ಮತ್ತು ಅನೈತಿಕ ವ್ಯವಹಾರಗಳಿಂದ ಉಂಟಾದ ಹಲವು ತೆರನಾದ ವಿಚಿತ್ರ ಮನೋರೋಗಗಳು ಅಮೇರಿಕದ ಸ್ತ್ರೀಯರ ಪಾಲಿಗೆ ಇಂದು ಸಾಕಷ್ಟುದೊರತಿವೆ. ಅಮೇರಿಕದ ಆಸ್ಪತ್ರೆಗಳಲ್ಲಿ ಇತರ ರೋಗಗಳಿಂದ ನರಳುವವರಷ್ಟೇ ಸಂಖ್ಯೆಯ ಹುಚ್ಚರೂ ಈಗಾಗಲೇ ಕಾಣಿಸಿಕೊಂಡಿದ್ದಾರೆ.’

ವಿವಾಹದ ಪಾವಿತ್ರ್ಯ, ಕುಟುಂಬದ ಸ್ಥೈರ್ಯ, ಮಕ್ಕಳ ರಕ್ಷಣೆ–ಇವುಗಳಿಗೆ ತೀವ್ರ ಆಘಾತ\-ವನ್ನುಂಟು\-ಮಾಡಿದ ಈ ವಿಪ್ಲವದ ಹಿನ್ನೆಲೆ ಯಾವುದು? ಮೂಲ ಕಾರಣ ಯಾವುದು? ಇಂದ್ರಿಯ ಸುಖವೇ ಸರ್ವಸ್ವ ಎನ್ನುವ ವಾದ! ಕಾಮಕ್ಕೇ ಅಗ್ರಪಟ್ಟ ಎನ್ನುವ ವಾದ! ಮಾತೃತ್ವದ ಮಹಿಮೆಯನ್ನು ಅಲ್ಲಗಳೆಯುವ ವಾದ! ಗಂಡಸು ಹೆಂಗಸು ಪರಸ್ಪರ ಪ್ರತಿಸ್ಪರ್ಧಿಗಳೆನ್ನುವ ಭಾವ! ಇಹಲೋಕವೇ ಸರ್ವಸ್ವ, ಇಂದ್ರಿಯಭೋಗವೇ ಸರ್ವಸ್ವ ಎನ್ನುವ ಜಡವಾದ–ದೇಹಾತ್ಮವಾದ! ಈ ಜಡವಾದದ ಬೆಂಕಿ ಇಂದು ಲಕ್ಷ ಲಕ್ಷ ನರನಾರಿಯರ ಹೃದಯಗಳನ್ನು ದಹಿಸುತ್ತಿದೆ!!

ಭೋಗ, ಸ್ವಾರ್ಥಗಳೇ ವೈವಾಹಿಕ ಜೀವನದ ಉದ್ದೇಶವಾದರೆ ವಿವಾಹಿತರಲ್ಲಿ ಘರ್ಷಣೆ ತಪ್ಪುವುದೆಂತು? ಗಂಡಹೆಂಡಿರ ಪೈಕಿ ಒಬ್ಬರು ಇನ್ನೊಬ್ಬರಿಂದ ಅಪೇಕ್ಷಿಸುವ ಸುಖಕ್ಕೆ ಅಲ್ಪಸ್ವಲ್ಪ ಧಕ್ಕೆ ಬಂದರೂ, ಅವರು ಪರಸ್ಪರ ಸಿಡಿಮಿಡಿಗೊಂಡು ಸಿಡಿದು ದೂರ ನಿಲ್ಲುತ್ತಾರೆ. ಬಳಿಕ ತಮಗೆ ಬೇಕೆನಿಸಿದ ಸುಖಕ್ಕಾಗಿ ಬೇರೆಡೆ ಹುಡುಕಾಟ ಪ್ರಾರಂಭಿಸುತ್ತಾರೆ. ಅವರ ಸ್ವಾರ್ಥಸುಖವೇ ಅವರಿಗೆ ಮುಖ್ಯವಾದುದರಿಂದ ಮಕ್ಕಳು ಬೀದಿಪಾಲಾಗುತ್ತಾರೆ. ಅವರ ಪಾಲಿಗೆ ಮಕ್ಕಳು ಒಂದು ಅನಾವಶ್ಯಕ ಭಾರ–ಸುಖಕ್ಕೆ ಅಡ್ಡಿ, ಆತಂಕ. ಆಗ ತಾನೇ ಹುಟ್ಟಿದ ಎಳೆಕೂಸನ್ನು ತನ್ನ ಮಗ್ಗುಲಲ್ಲೇ ಮಲಗಿಸಿಕೊಂಡರೆ ತನ್ನ ಸುಖಕ್ಕೆ ಅಡ್ಡಿ, ಶಿಶುವಿಗೆ ಮೊಲೆಹಾಲು ಕೊಟ್ಟಲ್ಲಿ ತನ್ನ ಶರೀರ ಸೌಷ್ಠವಕ್ಕೆ ಕುಂದು! ಒಟ್ಟಿನಲ್ಲಿ ಮಗುವಿಗೆ ಜನ್ಮ ಕೊಡುವುದೆಂದರೇ ತನಗೆ ನೋವು–ಸಂಕಟ, ಮುಂದೆ ಅದರ ಲಾಲನೆ ಪಾಲನೆ ಎಂದರೆ ತನ್ನ ಸ್ವಚ್ಛಂದ ವಿಹಾರಕ್ಕೆ ತೊಡಕು. ಈ ಎಲ್ಲ ಬಂಧನಗಳಿಂದ ಬಿಡುಗಡೆಯಾಗಬೇಡವೇ?


\section*{ಸ್ವೈರತೆಯ ಉಪಾಸನೆ}

\addsectiontoTOC{ಸ್ವೈರ\-ತೆಯ ಉಪಾಸನೆ}

ನಿಯಮ, ಸಂಯಮಗಳನ್ನು ನಿಂದಿಸುವುದು ಆಧುನಿಕ ಮನೋ\-ವೃತ್ತಿ. ಇದಕ್ಕೆ ಸ್ವತಂತ್ರ ಮನೋ\-ವೃತ್ತಿ ಎಂಬ ಬಿರುದೂ ಇದೆ! ‘ನನ್ನ ಮನಸ್ಸಿಗೆ ಬಂದಂತೆ ನಾನು ನಡೆಯುತ್ತೇನೆ, ನೀವಾರು ಕೇಳಲಿಕ್ಕೆ? ನನ್ನ ಸ್ವಾತಂತ್ರ್ಯಕ್ಕೆ ಕಲ್ಲು ಹಾಕಲು ನೀವಾರು? ನಾನು ಯಾವ ನಿಯಮಕ್ಕೂ ಗುಲಾಮನಾಗಲಿಚ್ಛಿಸುವುದಿಲ್ಲ. ನನಗೆ ನಿಯಮಗಳ ಬಂಧನ ಬೇಡ. “ಸ್ಪಾಂಟೇನಿಯಸ್​” ಆಗಿರಲು ನನ್ನ ಇಚ್ಛೆ’–ಎಂಬುದು ಇವರ ಕೂಗು. ಸ್ವಾತಂತ್ರ್ಯವೆಂದರೆ ತನ್ನ ಪರಿಪೂರ್ಣ ಬೆಳವಣಿಗೆಗೆ, ತನ್ನದೇ ವಿಧಾನದಲ್ಲಿ ಮುನ್ನಡೆಯಲು ಇರುವ ಅನುಕೂಲತೆಯ ಸದುಪಯೋಗ. ಪರಿಪೂರ್ಣ ಬೆಳವಣಿಗೆ ಅಥವಾ ವ್ಯಕ್ತಿತ್ವದ ವಿಕಸನ ಸಂಯಮವಿಲ್ಲದೇ ಸಾಧ್ಯವಿರದು. ಸ್ನಾಯು ಬಲ\-ಸಂವರ್ಧನೆ\-ಯಾಗ\-ಬೇಕೆನ್ನು\-ವವನು ನಿಯಮಿತ ರೀತಿಯಲ್ಲಿ ವ್ಯಾಯಾಮ ಮಾಡಬೇಕು. ತನ್ನ ಒಳಿತಿಗಾಗಿ ತಾನೇ ನಿಯಮಗಳನ್ನು ಹಾಕಿಕೊಂಡು ಅದನ್ನು ಅನುಸರಿಸುವುದು ಬಂಧನವೆನಿಸಿದರೂ, ಅದು ಬೆಳವಣಿಗೆಗೆ ಬೇಕಾದ ಬಂಧನವೇ. ದುರ್ಬಲತೆಯಿಂದ ಪಾರಾಗುವುದಕ್ಕಾಗಿ, ಸ್ವಸ್ಥನಾಗು\-ವುದ\-ಕ್ಕಾಗಿ ಹಾಕಿಕೊಂಡ ನಿಯಮ ಅದು. ಬಂಧನಗಳಿಂದ ಅತೀತನಾಗುವುದಕ್ಕಾಗಿ ಹಾಕಿಕೊಂಡ ಬಂಧನ ಅದು. ಪ್ರಕೃತಿ ಎಲ್ಲೆಲ್ಲೂ ನಿಯಮಗಳನ್ನನುಸರಿಸಿ ನಡೆಯುತ್ತದೆ. ಮನುಷ್ಯನೂ ತನ್ನ ಸರ್ವತೋಮುಖ ಪ್ರಗತಿಗೆ ನಿಯಮಾನುಸಾರವಾಗಿ ನಡೆದು, ಬಳಿಕ ನಿಜವಾದ ಸ್ವಾತಂತ್ರ್ಯದ ಆನಂದವನ್ನು ಪಡೆಯಬೇಕಾಗುತ್ತದೆ. ‘ಮೇಲು ಛಾವಣಿ ಇಲ್ಲದ ಮನೆಯೊಳಗೆ ಹೇಗೆ ಮಳೆಯು ಬಲವಾಗಿ ಒಳನುಗ್ಗುವುದೋ, ಅಂತೆಯೇ ನಿಯಮ, ಸಂಯಮಗಳಿಲ್ಲದವನ ಮನಸ್ಸಿನೊಳಗೆ ಎಲ್ಲ ದುರ್ಭಾವನೆಗಳೂ ನುಗ್ಗುವುವು’ ಎಂದ ಗೌತಮ ಬುದ್ಧ ತ್ರಿಕಾಲಾಬಾಧಿತವಾದ ಸತ್ಯವೊಂದನ್ನು ಉಸುರಿದ.

ಸಂತಾನವೃದ್ಧಿಗಾಗಿ ಮನುಷ್ಯರಲ್ಲಿ ಪ್ರಕೃತಿ ಇಟ್ಟಿರುವ ಒಂದು ಸಹಜ ಪ್ರವೃತ್ತಿ ಕಾಮ. ಕಾಮೇಚ್ಛೆಯ ಸಂತೃಪ್ತಿಯಾಗಬೇಕು, ಆದರೆ ವಂಶಾಭಿವೃದ್ಧಿಯ ಜವಾಬ್ದಾರಿ ಬೇಡವೆಂದರೆ\break ಪ್ರಕೃತಿಯ ನಿಯಮಕ್ಕೆ ವಿರೋಧವಾಗಿ ನಡೆದಂತಾಯಿತು.\footnote{\engfoot{Civilized societies which have most strictly limited sexual freedom have developed the highest culutres. In the whole of human history not a single case is found in which a society advanced to the Rationalistic Culture without its women being born and reared in a rigidly enforced pattern of faithfulness to one man. Further, there is no example of a community which has retained its high position on the culture scale after less rigorous sexual customs have replaced more restricting ones. –Pitirim A. Sorokin, \textit{Sane Sex Order,} Bharatiya Vidya Bhavan, Bombay.}} ಮಕ್ಕಳ ಪರಿಪೋಷಣೆ ಪಾಲನೆಗಳಿಂದ ವಂಚಿತಳಾದ, ಸಂಕುಚಿತ, ಸ್ವಾರ್ಥ ಪರಾಯಣೆಯಾದ ಹೆಂಗಸು, ನಿಸರ್ಗ ನಿಯಮಗಳ ಉಲ್ಲಂಘನೆಯ ಶಿಕ್ಷೆಯನ್ನು ಅನುಭವಿಸಬೇಕಾಗುವುದು–ಎಂದು ತಜ್ಞರು ಹೇಳುತ್ತಿದ್ದಾರೆ. ಶಿಶುವಿಗೆ ಮೊಲೆಯೂಡಿಸದ ತಾಯಿ ಸ್ತನಗಳ ಕ್ಯಾನ್ಸರ್ ರೋಗಕ್ಕೆ ತುತ್ತಾಗಬಹುದೆಂದು ಅವರ ಅಂಬೋಣವಿದೆ. ಹೊತ್ತು ಹೆತ್ತು ಸಾಕಿ ಸಲಹಿ, ಕೈಂಕರ್ಯ ಮಾಡದ ಹೆಂಗಸರು ಮನೋ ರೋಗಕ್ಕೆ ತುತ್ತಾಗಿ, ಶೀಘ್ರಕೋಪಿಗಳಾಗಿ, ಸಣ್ಣಪುಟ್ಟ ವಿಚಾರಗಳಿಗೂ ಜಗಳ ಪ್ರಾರಂಭಿಸಿ, ವಿವಾಹ ವಿಚ್ಛೇದಕ್ಕೆ ಕಾರಣರಾಗುತ್ತಾರೆ ಎಂಬುದೂ ತಜ್ಞರ ಅಭಿಮತ. ಸ್ತ್ರೀ ಪುರುಷರ ಸಂಬಂಧದ ಪಾವಿತ್ರ್ಯವನ್ನನುಸರಿಸಿ ಸಮಾಜದ ಸ್ವಾಸ್ಥ್ಯ ಮಾತ್ರವಲ್ಲ, ಅಸ್ತಿತ್ವ, ಪ್ರಗತಿಗಳೂ ಸ್ಥಾಯಿ\-ಯಾಗುವು\-ವೆಂಬುದನ್ನು ಎಲ್ಲರೂ ಒಪ್ಪುವವರೇ. ಈ ವಿಚಾರವನ್ನು ವಿಜ್ಞಾನಿಗಳೂ, ವಿಚಾರವಂತರೂ ಅಲ್ಲಗಳೆಯಲಾರರು.

ಧರ್ಮ ಮತ್ತು ಅಧ್ಯಾತ್ಮಗಳ ಹಿನ್ನೆಲೆಯಿಲ್ಲದೆ ಸ್ತ್ರೀಪುರುಷರ ಪವಿತ್ರ ಸಂಬಂಧದ ಮಾತು ಗಗನಕುಸುಮವೆ. ಅವರು ಒಬ್ಬರನ್ನೊಬ್ಬರು ಶೋಷಿಸದೆ, ಒಬ್ಬರನ್ನೊಬ್ಬರು ಅರಿತುಕೊಂಡು, ಸಹನೆ, ಸಹಕಾರ, ಸದ್ಭಾವನೆಗಳಿಂದ ಬದುಕಬೇಕಾದರೆ, ಉನ್ನತ ಆದರ್ಶದಲ್ಲಿ ದೃಢವಾದ ಶ್ರದ್ಧೆ ಇಲ್ಲದೆ ಸಾಧ್ಯವಿಲ್ಲ. ಮಹಾರಾಷ್ಟ್ರದ ದೇಶಭಕ್ತ ಸಾನೇ ಗುರೂಜಿ ‘ಭಾರತೀಯ ಸಂಸ್ಕೃತಿ’ ಎಂಬ ಗ್ರಂಥದಲ್ಲಿ ವೈವಾಹಿಕ ಜೀವನಾದರ್ಶವನ್ನು ಸುಂದರವಾದ ಮಾತುಗಳಲ್ಲಿ ಹೀಗೆಂದಿದ್ದಾರೆ:

‘ಸ್ತ್ರೀ ಪುರುಷರ ಸಂಬಂಧಗಳು ಪ್ರೀತಿಯಿಂದ ತುಂಬಿರಬೇಕು. ಹೆಣ್ಣು ಯಾರ ಸೊತ್ತೂ ಅಲ್ಲ. ಅವಳಿಗೆ ಹೃದಯ, ಬುದ್ಧಿ, ಭಾವನೆಗಳಿವೆ. ಅವಳಿಗೆ ಸ್ವಾಭಿಮಾನವೂ ಇದೆ, ಆತ್ಮವಿದೆ, ಸುಖದುಃಖಗಳಿವೆ. ಇವನ್ನೆಲ್ಲ ಗಂಡಸರು ನೆನಪಿನಲ್ಲಿಟ್ಟಿರಬೇಕು. ಹೆಣ್ಣು ಜಗತ್ತಿನಲ್ಲಿ ಒಂದು ದೊಡ್ಡ ಶಕ್ತಿ. ಈ ಶಕ್ತಿಯೊಡನೆ ನಡೆದುಕೊಳ್ಳುವ ಗಂಡಸರು ಶಿವಸ್ವರೂಪರಾಗಬೇಕು. ಶಿವಶಕ್ತಿಗಳ ಪ್ರೇಮದ ಮೇಲೆ ಜಗತ್ತಿನ ಪ್ರಾಣವೇ ಅವಲಂಬಿಸಿದೆ. ಶಿವಶಕ್ತಿಯರ ಪ್ರೇಮಮಯ, ಸಂಯಮಯುಕ್ತ ಸಂಬಂಧದಿಂದಲೇ ಮಹಾ ಕರ್ತೃತ್ವಶಾಲಿ ಕುಮಾರನು ಜನ್ಮವೆತ್ತಿದನು. ಈ ದಿವ್ಯ, ಪವಿತ್ರ ಸಂಬಂಧದಿಂದಲೇ ಶೌರ್ಯ, ಧೈರ್ಯಗಳ ಆಗರರೂ, ವಿದ್ಯೆಯ ಸಾಗರರೂ ಜನಿಸುತ್ತಾರೆ.’


\section*{ನವ್ಯರ ನಿರ್ದೇಶದಲ್ಲಿ}

\addsectiontoTOC{ನವ್ಯರ ನಿರ್ದೇಶದಲ್ಲಿ}

ಈ ‘ದಿವ್ಯ, ಭವ್ಯ’ಗಳೆಲ್ಲ ಅರ್ಥಹೀನವೆನ್ನುವ ನವ್ಯರ ನಿರ್ದೇಶನದಲ್ಲಿ, ನಾಗರಿಕತೆ ಎತ್ತ ಸಾಗುತ್ತಲಿದೆ ಎಂಬುದರ ಯಥಾರ್ಥ ಚಿತ್ರವನ್ನು ವಿಚಾರವಂತರೆಲ್ಲರೂ ಗಮನಿಸಬೇಕು. ತಂದೆ ತಾಯಿಗಳ ಸ್ವೇಚ್ಛಾಚಾರದಿಂದ ಯುವಪೀಳಿಗೆ ಹೇಗೆ ರೂಪುಗೊಂಡಿದೆ? ಅಮೇರಿಕದ ‘ಟೈಮ್​’ ಪತ್ರಿಕೆ (ಜುಲೈ ೧೯೭೭) ತನ್ನ ವರದಿಯಲ್ಲಿ ಹೀಗೆಂದಿತು:

‘ಅಮೇರಿಕದಲ್ಲಿ ನಡೆಯುವ ಒಟ್ಟು ಅಪರಾಧಗಳಲ್ಲಿ ಅರ್ಧಾಂಶಕ್ಕಿಂತ ಹೆಚ್ಚಿನವುಗಳನ್ನು ಹತ್ತರಿಂದ ಹದಿನೇಳು ವಯಸ್ಸಿನ ಪೋರರೇ ನಡೆಯಿಸುತ್ತಿದ್ದಾರೆ. ಕೊಲೆಗಳ ಜೊತೆಗೆ ಅತ್ಯಾಚಾರ, ಮಾರಕ ಆಯುಧಗಳಿಂದ ಹೊಡೆತ, ದರೋಡೆ, ಕಳವು, ವಾಹನಗಳ ಕಳವು – ಇವೂ ಸೇರಿ\-ಕೊಂಡಿವೆ.’\footnote{\engfoot{More than half of all serious crimes (murder, rape, aggravated assault, burglary, larceny, motor vehicle theft) in the United States are committed by youths aged 10 to 17–Time, July 1977 }}

ಏಳು ವರ್ಷಗಳ ಬಳಿಕ ಇತ್ತೀಚಿನ ವರದಿ ಈ ದುರಂತದಲ್ಲಿ ಆದ ಪ್ರಗತಿಯನ್ನು ಸಾರುತ್ತದೆ.

ಅಮೇರಿಕದ ಶಾಲೆಗಳಲ್ಲಿ ಕೊಲೆ ಮತ್ತು ಹಿಂಸೆಗಳನ್ನು ಎದುರಿಸಲು ಅಧ್ಯಕ್ಷ ರೇಗನ್ನರು ರಾಷ್ಟ್ರವ್ಯಾಪಿ ಚಳವಳಿಗೆ ಕರೆನೀಡಿದರು. ೧೯೭೮ರಲ್ಲಿ ನಡೆಸಿದ ಒಂದು ಅಧ್ಯಯನದ ವರದಿಯಂತೆ ಪ್ರತಿ ತಿಂಗಳೂ ಸುಮಾರು ಮೂರು ಮಿಲಿಯ ಹದಿಹರೆಯದ ವಿದ್ಯಾರ್ಥಿಗಳು, ಮುಖ್ಯವಾಗಿ ಪಟ್ಟಣದ ವಿದ್ಯಾರ್ಥಿಗಳು, ಹಿಂಸೆ, ಹಲ್ಲೆಗೊಳಗಾಗುತ್ತಾರೆ. ಅಧ್ಯಾಪಕರುಗಳ ಪರಿ ಸ್ಥಿತಿಯೂ ದುಃಖದಾಯಕ. ಇನ್ನು ಹೆಣ್ಣುಮಕ್ಕಳು ಹಿಂಸೆಗೊಳಗಾಗುವ ವಿಧಾನಗಳನ್ನು ಹೇಳಿ ಪೂರೈಸುವಂತಿಲ್ಲ! ವಿಜ್ಞಾನ ಹಾಗೂ ತಾಂತ್ರಿಕ ಪ್ರಗತಿಯ ಜೊತೆಜೊತೆಗೆ ಇದೆಂಥ ನೈತಿಕ ಕುಸಿತ!


\section*{ಇತಿಹಾಸದ ಪಾಠಗಳು}

\addsectiontoTOC{ಇತಿಹಾಸದ ಪಾಠಗಳು}

ಈ ಶ್ರದ್ಧೆ–ಪರಮಾರ್ಥದಲ್ಲಿನ ಶ್ರದ್ಧೆ, ಭಾರತೀಯರ ನೈತಿಕ ಸಮುನ್ನತಿಯ ಕಂಪನ್ನು ವಿಶ್ವಾದ್ಯಂತ ಪಸರಿಸಲು ಕಾರಣವಾಯಿತು ಎಂಬುದು ಐತಿಹಾಸಿಕ ಸತ್ಯ.

ಭಾರತೀಯರ ಶೀಲದ ಹಿರಿಮೆಯನ್ನು ಕುರಿತು ವಿದೇಶೀ ಪರಿಶೀಲಕರೂ, ಯಾತ್ರಿಕರೂ ಅಚ್ಚರಿಯಿಂದ ಪ್ರಶಂಸೆಯ ಮಾತುಗಳನ್ನು ಹೇಳಿದ್ದಾರೆ. ಐತಿಹಾಸಿಕ ಪ್ರಜ್ಞೆ ಇಲ್ಲದ ಆಧುನಿಕ ವಿದ್ಯಾವಂತ ತರುಣರು ಈ ಮಾತುಗಳನ್ನು ಇಂದು ಮನನ ಮಾಡಬೇಕಾಗಿದೆ. ಇದರ ಮೂಲ ಕಾರಣವನ್ನು ಮಥಿಸಿ ತಿಳಿಯಬೇಕಾಗಿದೆ. ಸುಮಾರು ಎರಡು ಸಾವಿರ ವರುಷಗಳ ಹಿಂದೆ ಯಾತ್ರಿಕನಾಗಿ ಬಂದಿದ್ದ ಏರಿಯನ್ ಹೇಳಿದ: ‘ನಿಜಕ್ಕೂ ಯಾವ ಭಾರತೀಯನೂ ಸುಳ್ಳಾಡಿದನೆಂದು ಯಾರೂ ದೂರವುದಿಲ್ಲ!’ ಸಾವಿರದ ಐನೂರು ವರ್ಷಗಳ ಹಿಂದೆ ಚೀನಾದೇಶದಿಂದ ಇಲ್ಲಿಗೆ ಬಂದಿದ್ದ ಬೌದ್ಧ ವಿದ್ವಾಂಸ ಹ್ಯುವೆನ್​ತ್ಸಾಂಗನು ಮಗಧಪ್ರಾಂತದ ಅಪಾರ ಸಂಪತ್ತನ್ನೂ, ಜನರ ಉಚ್ಚಮಟ್ಟದ ರೀತಿನೀತಿಗಳನ್ನೂ ಕಂಡು ಬೆರಗಾಗಿದ್ದ. ಗ್ರೀಕ್ ರಾಯಭಾರಿ ಮೆಗಾಸ್ತ ನೀಸ್ ಭಾರತಕ್ಕೆ ಬಂದಿದ್ದಾಗ ಜನರು ಮನೆಯ ಬಾಗಿಲುಗಳಿಗೆ ಬೀಗ ಹಾಕುತ್ತಿರಲಿಲ್ಲ! ಬೀಗಕ್ಕೆ ಆಗ ಸಂಸ್ಕೃತ ಶಬ್ದಕೂಡ ಇರಲಿಲ್ಲ. ಆತ ಹೇಳಿದ್ದಾನೆ: ‘ಈ ಜನ ತಮ್ಮ ದೇಶದಲ್ಲಿ ಎಂದೂ ಬರಗಾಲ ಬಂದಿಲ್ಲ, ಊಟಕ್ಕೆ ಇಲ್ಲ–ಎಂಬುದನ್ನು ಕಾಣೆವು ಎನ್ನುತ್ತಾರೆ. ಬೇರೆ ದೇಶಗಳಲ್ಲಿ ಯುದ್ಧ ನಡೆಯುವಾಗ ಶತ್ರುಪಕ್ಷ ಆ ದೇಶದ ನೆಲವನ್ನು ಹಾಳುಮಾಡಿ ಜನಕ್ಕೆ ಆಹಾರವಿಲ್ಲದಂತೆ ಮಾಡುವುದು ವಾಡಿಕೆ. ಈ ದೇಶದಲ್ಲಿ ವ್ಯವಸಾಯದಿಂದ ಅನ್ನ ಒದಗಿಸುವವರನ್ನು ಶ್ರೇಷ್ಠ ಮನುಷ್ಯರೆಂದು ಭಾವಿಸುತ್ತಾರೆ. ಅವರಿಗೆ ಕಷ್ಟಕೊಡುವುದಿಲ್ಲ. ಸನಿಹದಲ್ಲೇ ಯುದ್ಧ ನಡೆಯು ತ್ತಿದ್ದರೂ ಬೇಸಾಯಗಾರರು ತಮ್ಮ ಕೆಲಸವನ್ನು ತಾವು ನಡೆಯಿಸುತ್ತಿರುತ್ತಾರೆ. ಸೈನಿಕ ಸೈನಿಕನನ್ನು ಕೊಲ್ಲುತ್ತಾನೆ. ಬೇಸಾಯಗಾರರನ್ನು ಮುಟ್ಟುವುದಿಲ್ಲ. ಶತ್ರುಪಕ್ಷ ಎದುರಾಳಿಯ ರಾಜ್ಯದಲ್ಲಿ ಬೆಂಕಿ ಹಾಕಿ ಏನನ್ನೂ ಸುಡುವುದಿಲ್ಲ; ಗಿಡಗಳನ್ನು ಕಡಿದು ಹಾಳುಗಡೆಹುವುದಿಲ್ಲ.’

ಭೂ ಸಂಚಾರಿ ಮಾರ್ಕೋಪೊಲೋ ಹೇಳಿದ: ‘ಭೂಲೋಕದಲ್ಲೇ ಭಾರತೀಯರು ಅತ್ಯಂತ ಒಳ್ಳೆಯವರೂ, ಸತ್ಯವಂತರೂ ಆದ ವ್ಯಾಪಾರಿಗಳು. ಯಾವ ಕಾರಣದಿಂದಲೂ ಇವರು ಸುಳ್ಳು ಹೇಳುವವರಲ್ಲ.’

ಮಹಮ್ಮದೀಯ ಭೂವಿವರಣೆಗಾರ ಇದ್ರಿಸಿ ‘ಇಂಡಿಯಾ ದೇಶದ ಜನರು ನಂಬಿಕೆ, ಸತ್ಯ ನಿಷ್ಠೆಗೆ ಪ್ರಸಿದ್ಧರಾಗಿದ್ದಾರೆ’ ಎಂದಿದ್ದಾನೆ. ಇದು ಕ್ರಿಸ್ತಶಕ ಹನ್ನೊಂದನೇ ಶತಮಾನದಲ್ಲಿ ಆತನು ಹೇಳಿದ್ದ ಮಾತು.

ಮುನ್ನೂರು ವರ್ಷಗಳ ಹಿಂದೆ ಇಲ್ಲಿಗೆ ಬಂದ ಪೋರ್ಚುಗೀಸರು ಬರೆದಿಟ್ಟರು: ‘ಹಿಂದುಗಳು ಯುದ್ಧಮಾಡುವಾಗ ಮೊದಲು ಸೂಚನೆ ನೀಡದೆ ಎಷ್ಟಕ್ಕೂ ಯುದ್ಧ ಮಾಡರು. ವೀರರಾದ ಅವರು ಶತ್ರುವಿನ ಬಗ್ಗೆ ಸ್ವಲ್ಪವೂ ದ್ವೇಷ ಇಟ್ಟುಕೊಂಡವರಲ್ಲ. ಆದುದರಿಂದ ಯುದ್ಧದ ವಿರಾಮ ಕಾಲದಲ್ಲಿ ಒಂದೇ ನದಿಯಲ್ಲಿ ಸ್ನಾನ ಮಾಡಿ, ಎಲೆಅಡಿಕೆಯನ್ನು ಪರಸ್ಪರ ವಿನಿಮಯ ಮಾಡಿಕೊಳ್ಳುತ್ತಿದ್ದರು. ಅಪಮಾನದ ಬದುಕು ಸಾವಿಗಿಂತ ಕಳಪೆ ಎಂದು ಅವರು ಭಾವಿಸುತ್ತಿದ್ದರು.

‘ತಾವು ಯುದ್ಧದಲ್ಲಿ ಸೆರೆ ಹಿಡಿದವರನ್ನು ಬಿಡುಗಡೆ ಹಣ ತರುವುದಕ್ಕಾಗಿ ದೂರದ ಊರುಗಳಿಗೆ ಹೋಗಲು ಪೋರ್ಚುಗೀಸ್ ಅಧಿಕಾರಿಗಳು ಬಿಡುತ್ತಿದ್ದರು. ಕೆಲವರು ಹಣ ತರಲು ಸಮರ್ಥರಾಗಿದ್ದರೆ, ಹಲವರು ಸಮರ್ಥರಾಗುತ್ತಿರಲಿಲ್ಲ. ಊರಿಗೆ ಹೋದ ಅವರಿಗೆ ತಪ್ಪಿಸಿಕೊಳ್ಳಲು ಅವಕಾಶವಿದ್ದರೂ, ಆಡಿದ ಮಾತಿಗೆ ತಪ್ಪಬಾರದು, ಸುಳ್ಳು ಹೇಳಬಾರದು–ಎಂಬ ದೃಢನಿಷ್ಠೆಯಿಂದ ಮತಾಂತರ ಅಥವಾ ಮರಣದಂಡನೆಯ ಶಿಕ್ಷೆಯನ್ನು ಎದುರಿಸಲು ಹಿಂದಿರುಗಿ ಬರುತ್ತಿದ್ದರು! ಇವರ ಸತ್ಯನಿಷ್ಠೆ, ಶೀಲ ಬಲವನ್ನು ಕಂಡು ಪೋರ್ಚುಗೀಸರೂ ಅಚ್ಚರಿಪಡುತ್ತಿದ್ದರು.’


\section*{ಶೀಲ ಸೌರಭ}

\addsectiontoTOC{ಶೀಲ ಸೌರಭ}

{\parfillskip=0ptಭಾರತೀಯ ರಾಷ್ಟ್ರೀಯ ಸಭೆಯ ಅಧ್ಯಕ್ಷ ಆಲ್​ಫ್ರೆಡ್ ವೆಬ್ ಮಹಾಶಯ ಹೀಗೆಂದಿದ್ದಾನೆ;\par}\newpage\noindent ‘ಹೊಲಿಗೆಯ ಯಂತ್ರದ ವ್ಯಾಪಾರಿಗಳ ಪ್ರತಿನಿಧಿಯ ಅನುಭವದಲ್ಲಿ ಹೊರದೇಶದಲ್ಲಿ ನೂರರಲ್ಲಿ ಹತ್ತರಷ್ಟು ಸಾಲ ಮರುಪಾವತಿಯಾಗದೇ ಉಳಿಯುತ್ತಿತ್ತು. ಭರತವರ್ಷದಲ್ಲಿ ಅಂಥ ಘಟನೆ ನೂರರಲ್ಲಿ ಒಂದರಷ್ಟು ಮಾತ್ರ! ಆ ಒಂದಂಶ ಇಲ್ಲಿರುವ ಯುರೋಪಿನವರ ಮೂಲಕವೇ! ದೇಶೀಯ ಹೊಲಿಗೆಗಾರರು ಸಾಲವನ್ನು ಉಳಿಸಿಕೊಳ್ಳುತ್ತಿರಲಿಲ್ಲ; ತೀರಿಸುವುದ ಕ್ಕಾಗದಿದ್ದರೆ ಯಂತ್ರವನ್ನು ಹಿಂದಿರುಗಿಸುತ್ತಿದ್ದರು. ಇದಕ್ಕೂ ಆಶ್ಚರ್ಯದ ಸಂಗತಿ–ಸಂತೆಯಲ್ಲಿ, ಅಂಗಡಿ ಕಟ್ಟೆಯಲ್ಲಿ, ರೈಲು ನಿಲ್ದಾಣದಲ್ಲಿ, ವ್ಯಾಪಾರಿಗಳಾದವರು ತೆರೆದ ಗಲ್ಲದಲ್ಲಿ ಹಣಹಾಕಿ ಕುಳಿತಿದ್ದರೆ, ಯಾರೂ ಅದಕ್ಕೆ ಕೈಹಾಕುತ್ತಿರಲಿಲ್ಲ. ಯೂರೋಪಿನಲ್ಲಿ ಹಣವನ್ನು ಹೀಗೆ ಬಿಡುವುದು ಸಾಧ್ಯವೇ?’ ಕರ್ನಲ್ ಸ್ಲೀಮ್​ನ ಹೇಳಿಕೆಯನ್ನೂ ಈ ಸಂದರ್ಭದಲ್ಲಿ ಗಮನಿಸಬೇಕು: ‘ಒಂದೇ ಒಂದು ಸುಳ್ಳು ಹೇಳಿದ್ದರೆ ತಮ್ಮ ಜೀವ, ಸ್ವಾತಂತ್ರ್ಯ, ವಿತ್ತ–ಇವುಗಳನ್ನೆಲ್ಲ ಉಳಿಸಿ ಕೊಳ್ಳಲು ಸಾಧ್ಯವಿದ್ದರೂ ಹಿಂದುಗಳು ಸುಳ್ಳು ಹೇಳದ ನೂರಾರು ಸಂದರ್ಭಗಳನ್ನು ಕಂಡಿದ್ದೇನೆ.’\footnote{\engfoot{I have had before me hundreds of cases in which a man’s property, liberty or life has depended upon his telling a lie and he has refused to tell it.}\hfill\engfoot{ –Col. Sleeman}}

ಚಾರ್ಲ್ಸ್ ವೊರ್ಸೆಲಸ್ ಹೇಳಿದ: ‘ಇಂಡಿಯಾದಲ್ಲಿ ಇಪ್ಪತ್ತೆರಡು ವರ್ಷ, ಈ ಸದ್ಯ ಇಂಗ್ಲೆಂಡಿನಲ್ಲಿ ಹದಿನೇಳು ವರ್ಷಗಳ ಕಾಲ ಕಳೆದಿದ್ದೇನೆ. ನನ್ನವರನ್ನು ಕಂಡಷ್ಟೂ ಇಂಡಿಯದ ಜನರನ್ನು ನಾನು ಹೆಚ್ಚು ಮೆಚ್ಚುತ್ತೇನೆ’. ಡಾ. ಗ್ರೆಹ್ಯಾಮ್ ‘ಎಲ್ಲೆಲ್ಲಿ ಹಿಂದುಗಳು ಹೋಗಿದ್ದಾರೋ, ಅಲ್ಲಿನ ಮೂಲನಿವಾಸಿಗಳ ಧಾರ್ಮಿಕ ಮಟ್ಟವನ್ನು ಸುಧಾರಿಸಿದ್ದಾರೆ’ ಎಂದರೆ, ಸಾರ್ ಜಾರ್ಜ್ ಬರ್ಡ್ ವುಡ್ಡೆ ‘ಹಿಂದೂ ಮಹಿಳೆಯರು ದೋಷರಹಿತ ಭಾರ್ಯೆಯರು. ದೇಹಪರಿಶುದ್ಧತೆಯಲ್ಲಿ ಹಿಂದೂಸ್ಥಾನದ ಜನರು ಭೂಲೋಕದ ಎಲ್ಲರಿಗಿಂತ ಮೇಲು’ ಎಂದ. ಇವೆಲ್ಲ ಅವರ ಶೀಲಕ್ಕಿತ್ತ ಪ್ರಮಾಣಪತ್ರಗಳು!

ಕೌಟಿಲ್ಯ, ಚಕ್ರವರ್ತಿಯಾದ ಚಂದ್ರಗುಪ್ತನ ಗುರುವಾಗಿದ್ದರೂ, ಅರಮನೆಯ ವೈಭವ\-ಯುತ ಜೀವನವನ್ನಾಗಲೀ, ರಾಜಭೋಗಗಳನ್ನಾಗಲೀ ಬಯಸದೆ ಒಂದು ಪುಟ್ಟ ಕುಟೀರದಲ್ಲಿ ವಾಸಿಸಿದ. ಸಾರ್ವಭೌಮ ಅಶೋಕ ಮಹಾಶೂರನೆನಿಸಿದ್ದರೂ, ಕಳಿಂಗ ಯುದ್ಧದಲ್ಲಿ ನಡೆದ ಹತ್ಯೆಯನ್ನೂ, ಹರಿದ ರಕ್ತದ ಕಾಲುವೆಯನ್ನೂ, ವಿಧವೆಯರ ಆರ್ತನಾದವನ್ನೂ, ಅನಾಥರ ಗೋಳನ್ನೂ ಕಂಡು, ಕರುಣೆಯಿಂದ ಕರಗಿದ. ಯುದ್ಧವನ್ನು ಪೂರ್ಣವಾಗಿ ತ್ಯಜಿಸಿ, ತ್ಯಾಗ, ಸೇವೆಗಳ ಮಹಾ ಆದರ್ಶವನ್ನು ಮೈಗೂಡಿಸಿಕೊಂಡ. ಧರ್ಮದ ಆಚರಣೆಯಿಂದ ಬರತಕ್ಕ ವಿಜಯ, ನಿಜವಾದ ವಿಜಯ ಎಂಬುದನ್ನು ಮನಗಂಡ. ಧರ್ಮಪ್ರಚಾರಕಾರ್ಯ ತನ್ನ ಕರ್ತವ್ಯ ಎಂದುಕೊಂಡ. ಜಗತ್ತಿನ ಗುಣಗ್ರಾಹಿಗಳೆಲ್ಲರೂ ಗೌರವದಿಂದ ತಲೆಬಾಗುವಂತೆ ಪರಮ ಉದಾತ್ತ ಉದಾರ ಭಾವನೆಗಳನ್ನು ಬದುಕಿನಲ್ಲಿ ಕಾರ್ಯರೂಪಕ್ಕೆ ತಂದ: ‘ರಾಷ್ಟ್ರದ ಪ್ರಜೆಗಳು ಕರುಣೆ, ಔದಾರ್ಯ, ಸತ್ಯ, ಶುಚಿ, ಕನಿಕರ, ಸೌಜನ್ಯವೇ ಮೊದಲಾದ ಗುಣಗಳನ್ನು ಬೆಳೆಸಿಕೊಳ್ಳಬೇಕು. ಪ್ರಭುಗಳ ಅಪ್ಪಣೆ ಎಂದಲ್ಲ, ತಾವೇ ಮಾಡುವ ಆಲೋಚನೆಯಿಂದ ಅವರು ಈ ದಾರಿಯನ್ನು ಹಿಡಿಯಬೇಕು. ಆಜ್ಞೆ ಎಂದು ಸಾಧಿಸುವ ಸೌಜನ್ಯಕ್ಕಿಂತ ಸ್ವಂತ ನಿಶ್ಚಯದಿಂದ ಸಾಧಿಸುವ ಸೌಜನ್ಯ ಶ್ರೇಷ್ಠವಾದುದು.

ಜಗತ್ತಿನ ಅಸಂಖ್ಯ ರಾಜರುಗಳ ಸಾಲಿನಲ್ಲಿ ಅಶೋಕ ಚಕ್ರವರ್ತಿ ಮಾತ್ರ ದೇದೀಪ್ಯಮಾನ ಧ್ರುವ ನಕ್ಷತ್ರದಂತೆ ಕಂಗೊಳಿಸುತ್ತಿದ್ದಾನೆಂದು ಜಗತ್ತಿನ ಇತಿಹಾಸವನ್ನು ಬರೆದ ಎಚ್.\ ಜಿ.\ ವೆಲ್ಸ್ ಮಹಾಶಯ ಹೇಳಿದ ಮಾತು ಅತಿಶಯೋಕ್ತಿ ಅಲ್ಲ. ಆತನ ಪ್ರೀತಿ, ವಾತ್ಸಲ್ಯ, ಕರುಣೆ ಪ್ರವಾಹಾಕಾರವಾಗಿ ಸರ್ವತ್ರ ಹರಿಯಿತು. ಜಗತ್ತಿನ ಇತಿಹಾಸದಲ್ಲೆ ಅಂಥ ಚಕ್ರವರ್ತಿಯನ್ನು ಇನ್ನೆಲ್ಲಿ ಕಾಣ\-ಬಲ್ಲೆವು? ಆತನ ಹೃದಯದ ಪರಿಚಯವನ್ನು ಮಾಡಿಕೊಡುವ ಈ ವಾಕ್ಯಗಳನ್ನು ಗಮನಿಸಿ:

‘ಮಾರ್ಗಗಳಲ್ಲಿ ಜನರಿಗೆ, ಪ್ರಾಣಿಗಳಿಗೆ, ನೆರಳಿರಲಿ ಎಂದು ನಾನು ಆಲದ ಮರಗಳನ್ನು ನೆಡಿಸಿದ್ದೇನೆ. ದಾರಿಯ ಪಕ್ಕದಲ್ಲಿ ಮಾವಿನತೋಪುಗಳನ್ನು ಬೆಳೆಸಿದ್ದೇನೆ. ಒಂದೊಂದು ಅರ್ಧ ಕ್ರೋಶದಲ್ಲೂ ಬಾವಿಗಳನ್ನು ತೋಡಿಸಿದ್ದೇನೆ. ಅಲ್ಲಲ್ಲಿ ಛತ್ರಗಳನ್ನು ಕಟ್ಟಿಸಿದ್ದೇನೆ.

‘ಮಾತೃ ಶುಶ್ರೂಷೆಯೂ ಪಿತೃ ಶುಶ್ರೂಷೆಯೂ ಸಾಧುವಾದದ್ದು; ಪ್ರಾಣಿಗಳನ್ನು ಕೊಲ್ಲದಿರುವುದು ಸಾಧುವಾದದ್ದು; ಅಲ್ಪವಾಗಿ ವ್ಯಯಮಾಡುವುದೂ, ಅಲ್ಪವಾಗಿ ಕೂಡಿಡುವುದೂ ಸಾಧು\-ವಾದದ್ದು.’

ದೇವಾನಾಂಪ್ರಿಯ ಪ್ರಿಯದರ್ಶಿ ರಾಜನು ಹೀಗೆ ಹೇಳುವನು: ‘ನಾನು ಭೋಜನ ಮಾಡು\-ತ್ತಿರಲಿ, ಅಂತಃಪುರದೊಳಗಿರಲಿ, ಒಳಗಣ ಕೋಣೆಯಲ್ಲಿರಲಿ, ಗೋಶಾಲೆಯಲ್ಲಿರಲಿ, ಪಲ್ಲಕ್ಕಿಯ\-ಲ್ಲಿರಲಿ, ಉದ್ಯಾನಗಳಲ್ಲಿರಲಿ, ಎಲ್ಲೇ ಇದ್ದರೂ ವರದಿಗಾರರು ಜನತೆಯ ಜೀವನ ವಿಧಾನಗಳ (ಸಮಸ್ಯೆಗಳ) ಬಗ್ಗೆ ನನಗೆ ತಿಳಿಸಬೇಕು. ನಾನು ಸರ್ವತ್ರ ಅವರ ಸಮಸ್ಯೆಗಳನ್ನು ಗಮನಿಸುತ್ತೇನೆ.’

‘ದೇವಾನಾಂಪ್ರಿಯ ಪ್ರಿಯದರ್ಶಿ ರಾಜನು ಎಲ್ಲಾ ಮತದವರನ್ನೂ, ಸಂನ್ಯಾಸಿಗಳನ್ನೂ, ಗೃಹಸ್ಥರನ್ನೂ, ದಾನದಿಂದಲೂ, ವಿವಿಧ ಪೂಜೆಗಳಿಂದಲೂ ಪೂಜಿಸುತ್ತಿರುವನು. ಸರ್ವಪ್ರಕಾರ\-ಗಳಲ್ಲೂ ಪರಮತದವರನ್ನು ಪೂಜಿಸಲೇಬೇಕು. ಹೀಗೆ ಮಾಡಿದರೆ ತನ್ನ ಮತವೂ ವರ್ಧಿಸುತ್ತದೆ; ಪರಮತಕ್ಕೂ ಉಪಕಾರವಾಗುತ್ತದೆ. ಅನ್ಯಥಾ ಮಾಡಿದರೆ, ತನ್ನ ಮತಕ್ಕೆ ಹಾನಿ, ಪರಮತಕ್ಕೆ ಅಪಕಾರ. ಏಕೆಂದರೆ ಆತ್ಮಮತವನ್ನು ಹೇಗೆ ಬೆಳಗಿಸಲಿ ಎಂಬ ಆತುರದಿಂದ, ಯಾರು ಅದನ್ನೇ ಪೂಜಿಸಿ, ಪರಮತವನ್ನು ಹಳಿಯುತ್ತಾನೋ, ಅವನು ತನ್ನ ವರ್ತನೆಯಿಂದ ತನ್ನ ಮತಕ್ಕೇ ಹೆಚ್ಚು ಹಾನಿಯನ್ನು ಉಂಟುಮಾಡುತ್ತಾನೆ. ಎಂದರೆ ಸಮವಾಯವೇ ಸಾಧುವಾದದ್ದು. ಹೀಗೆ ಒಟ್ಟು ಗೂಡುವುದು ಏತಕ್ಕಾಗಿ ಎಂದರೆ ಇತರರ ಮತವನ್ನು ಕೇಳುವುದಕ್ಕಾಗಿ, ಶುಶ್ರೂಷಿಸುವುದಕ್ಕಾಗಿ. ಇದೇ ದೇವಾನಾಂಪ್ರಿಯನ ಇಚ್ಛೆ–ಎಲ್ಲ ಮತದವರೂ ಬಹುಶ್ರುತರಾಗಲಿ ಎಂದು, ಕಲ್ಯಾಣಾಗಮಗಳನ್ನು ಹೊಂದಲಿ ಎಂದು. ಆಯಾಯ ಮತಗಳಲ್ಲಿ ಶ್ರದ್ಧೆ ಇರುವವರಿಗೆ ಈಗ ಹೀಗೆ ಹೇಳತಕ್ಕದ್ದು–ಸರ್ವಮತದವರಲ್ಲೂ ಸಾರವೃದ್ಧಿಯಾಗಬೇಕು. ದೇವನಾಂಪ್ರಿಯನು ದಾನವನ್ನೂ ಪೂಜೆಯನ್ನೂ ಇದಕ್ಕಿಂತ ಮೇಲಾಗಿ ಎಣಿಸುವುದಿಲ್ಲ.

\newpage

‘ಎಲ್ಲ ಮನುಷ್ಯರೂ ನನ್ನ ಮಕ್ಕಳು. ನನ್ನ ಮಕ್ಕಳು ಇಹಲೋಕದಲ್ಲೂ, ಪರಲೋಕದಲ್ಲೂ ಎಲ್ಲಾ ಹಿತಸುಖಗಳನ್ನು ಹೊಂದಬೇಕೆಂದು ನಾನು ಇಚ್ಛಿಸುತ್ತಿರುವಂತೆಯೇ, ಎಲ್ಲ ಮನುಷ್ಯರ ವಿಷಯದಲ್ಲೂ ನಾನು ಇಚ್ಛಿಸುತ್ತೇನೆ. ಈ ನನ್ನ ಬಯಕೆ ಎಲ್ಲಿಯತನಕ ಹೋಗುತ್ತದೆ ಎಂಬುದನ್ನು ನೀವು ತಿಳಿದಿಲ್ಲ....

‘ಗಬ್ಬವಾಗಿರುವ ಅಥವಾ ಹಾಲು ಕರೆಯುವ ಆಡು, ಕುರಿ, ಹಂದಿ ಇವುಗಳನ್ನು ಕೊಲ್ಲ ಕೂಡದು. ಆರು ತಿಂಗಳುಗಳವರೆಗೂ ಅವುಗಳ ಶಿಶುಗಳನ್ನು ಕೊಲ್ಲಕೂಡದು. ಹುಂಜಗಳಿಗೆ ಹಿಡ ಮಾಡಬಾರದು. ಜೀವಿಗಳು ಸೇರಿಕೊಂಡಿರುವ ಹೊಟ್ಟನ್ನು ಉರಿಸಬಾರದು. ವಿನಾಕಾರಣವಾಗಿಯಾಗಲಿ, ಹಿಂಸ್ರಪ್ರಾಣಿಗಳನ್ನು ಓಡಿಸುವ ಉದ್ದೇಶದಿಂದಾಗಲಿ, ಕಾಡನ್ನು ಉರಿಸಬಾರದು. ಜೀವಿಗಳಿಂದ ಜೀವಗಳನ್ನು ಪೋಷಿಸಲಾಗದು.’

ನಿಜವಾದ ಆಧ್ಯಾತ್ಮಿಕ ಹಿನ್ನೆಲೆ ಅಥವಾ ನಿಜವಾದ ಧರ್ಮಶ್ರದ್ಧೆ ಹೇಗೆ ಸಮಸ್ತ ಮಾನವ ಜನಾಂಗದ ಕಲ್ಯಾಣಚಿಂತನೆಗೆ ಬೇಕಾದ ನಿಃಸ್ವಾರ್ಥ ಪ್ರೇಮದ ಪ್ರವಾಹವನ್ನು ಹರಿಯಿಸಿತು ಎಂಬುದಕ್ಕೆ ಮೇಲಿನ ಮಾತುಗಳು ಜ್ವಲಂತ ಉದಾಹರಣೆಗಳಲ್ಲವೆ?

ಚಕ್ರವರ್ತಿ ಶ‍್ರೀಹರ್ಷ, ಪ್ರತಿ ಐದು ವರ್ಷಗಳಿಗೊಮ್ಮೆ ತಾನು ಸಂಗ್ರಹಿಸಿದ ಐಶ್ವರ್ಯವನ್ನೆಲ್ಲ ವಿದ್ವಾಂಸರಿಗೂ, ಬಡವರಿಗೂ ಸಂಪೂರ್ಣವಾಗಿ ಹಂಚುತ್ತಿದ್ದ. ಇಮ್ಮಡಿ ಪುಲಿಕೇಶಿಯ ರಾಜ್ಯದಲ್ಲಿ ಸುಂದರಿಯಾದ ಸ್ತ್ರೀ ಆಭರಣಭೂಷಿತೆಯಾಗಿ, ಏಕಾಂಗಿಯಾಗಿ ಸಂಚರಿಸಿದರೂ, ಅವಳನ್ನು ಜನರು ದುಷ್ಟದೃಷ್ಟಿಯಿಂದ ನೋಡುತ್ತಿರಲಿಲ್ಲ. ಅಲ್ಲಸಾನಿ ಪೆದ್ದನ ಕವಿಗೆ ಕೃಷ್ಣದೇವರಾಯನು ತನ್ನ ಹಸ್ತದಿಂದಲೇ ರತ್ನಖಚಿತ ಪಾದುಕೆಗಳನ್ನು ತೊಡಿಸಿ, ಕವಿಯ ಮೆರವಣಿಗೆಯಲ್ಲಿ ಪಲ್ಲಕ್ಕಿಗೆ ತಾನೇ ಹೆಗಲುಕೊಟ್ಟು ಗೌರವಿಸಿದ. ಛತ್ರಪತಿ ಶಿವಾಜಿ ತನ್ನ ಗುರು ಸಮರ್ಥರಾಮದಾಸರಿಗೆ, ತಾನು ಅತ್ಯಂತ ಕಷ್ಟದಿಂದ ಗೆದ್ದುಕೊಂಡ ರಾಜ್ಯವನ್ನು ದಾನಪತ್ರ ಮಾಡಿ ಅವರ ಭಿಕ್ಷಾ ಪಾತ್ರೆಗೆ ಹಾಕಲು ಹಿಂಜರಿಯಲಿಲ್ಲ. ಆತ ತನಗೆ ಸೈನ್ಯಾಧಿಕಾರಿಗಳು ಕಾಣಿಕೆಯಾಗಿ ತಂದೊಪ್ಪಿಸಿದ ಸುಂದರಿಯನ್ನು ತಾನು ಸ್ವೀಕರಿಸದೇ, ಅತ್ಯಂತ ಗೌರವದಿಂದ ಅವಳ ಪತಿಯ ಮನೆಗೆ ಕಳುಹಿಸಿದ. ಅವನು ಮುಸ್ಲಿಂ ದೊರೆಗಳ ಅಪಾಯ ಕ್ರೌರ್ಯ, ಹಿಂಸೆಗಳನ್ನು ಕಂಡವನಾದರೂ, ಒಂದೇ ಒಂದು ಮಸೀದಿಯನ್ನೂ, ಕುರಾನನ್ನೂ ನಾಶಮಾಡದೆ, ಅನ್ಯಧರ್ಮಗಳಿಗೆ ಗೌರವವನ್ನೂ ತೋರಿದ. ಔರಂಗಜೇಬನ ಮೊಮ್ಮಗಳು ಸೆರೆಸಿಕ್ಕಿದಾಗ ಅವಳಿಗೆ ಅವಳ ಧರ್ಮದ ಶಿಕ್ಷಣ ನೀಡಲು ದೂರದ ಅಜಮೀರದಿಂದ ಮುಸಲ್ಮಾನ ಅಧ್ಯಾಪಕಿಯನ್ನು ಕರೆಸಿದ್ದ ದುರ್ಗಾದಾಸ ರಾಠೋಡ. ರಜಪೂತ ವೀರರ ಕ್ಷಾತ್ರತೇಜವು ಭೂಲೋಕದಲ್ಲೇ ಎಣೆ ಇಲ್ಲದ್ದಾಗಿತ್ತು. ಈ ಎಲ್ಲ ಉತ್ಕೃಷ್ಟ ಚಾರಿತ್ರ್ಯಗಳ ಹಿನ್ನೆಲೆಯಲ್ಲಿ ಪ್ರಭಾವ ಬೀರಿದ್ದ ಶಕ್ತಿ ಯಾವುದು? ಧರ್ಮ ಸಂಸ್ಕೃತಿಯ ಉನ್ನತ ಆದರ್ಶಗಳಲ್ಲವೆ?


\section*{ಅಜಗಜಾಂತರ}

\addsectiontoTOC{ಅಜಗಜಾಂ\-ತರ}

ಚಾರಿತ್ರ್ಯಶುದ್ಧಿಯ ಆ ಹಿರಿಮೆಯನ್ನು ಪಶ್ಚಿಮದ ವ್ಯಾಪಾರೀ ಜನಾಂಗ ಹೇಗೆ ಕುಲಗೆಡಿಸಿತೆಂಬುದರ ವಿವರಣೆ ಇಲ್ಲಿದೆ.\footnote{ಕೋಟ ವಾಸುದೇವ ಕಾರಂತ, ‘ದಾನ ಮಾಡಬೇಕು.’}

ಪಶ್ಚಿಮ ದೇಶಗಳ ಕ್ರಿಶ್ಚಿಯನ್​ ಜನಾಂಗ ತಮ್ಮ ನೌಕಾಬಲ, ಫಿರಂಗಿ ಮತ್ತು ಕೋವಿಗಳ ಶಕ್ತಿಯನ್ನು ಬೆಳೆಸಿ ಭೂಲೋಕವನ್ನೇ ಕೊಳ್ಳೆ ಹೊಡೆಯಲು ಹೊರಟಿತು. ತಾವು ಹೋಗಿ ಹೊಸದಾಗಿ ನೆಲೆ ಊರಿದ ಯಾವ ದೇಶವೆ ಆಗಲಿ, ಅದು ತಮ್ಮದೇ ರಾಜ್ಯವಾಗಬೇಕು. ತಮ್ಮ ರಾಜನ ಹೆಸರಿನಲ್ಲಿ ಅದನ್ನು ಗೆದ್ದು ವಶಮಾಡಿಕೊಳ್ಳುವ ಹಕ್ಕು ತಮ್ಮದೇ ಹೊರತು, ಅಲ್ಲಿ ಮೊದಲಿನಿಂದ ಬಾಳಿಕೊಂಡು ಬಂದಿದ್ದ ಮೂಲನಿವಾಸಿಗಳದ್ದಲ್ಲ; ತಾವು ಅವರನ್ನು ಕೊಂದರೂ, ಸುಲಿಗೆ ಮಾಡಿದರೂ ಅಥವಾ ಅವರನ್ನು ಅಡಿಯಾಳಾಗಿ ಮಾಡಿಕೊಂಡು ತಮ್ಮ ಊಳಿಗ ಮಾಡಿಸಿಕೊಂಡರೂ, ಅಲ್ಲಿನ ಸಂಪತ್ತು ಸಮೃದ್ಧಿಯನ್ನು ದೋಚಿ ತಮ್ಮ ದೇಶಕ್ಕೆ ಸಾಗಿಸುತ್ತ ಬಂದರೂ ಅವೆಲ್ಲ ತೀರ ನ್ಯಾಯ, ಸಹಜ ಎಂದೇ ತಿಳಿದು ವರ್ತಿಸಿದವರು ಅವರು.

ಇಂಥ ಅಧರ್ಮದ ಕಲ್ಪನೆಯಿಂದ ಇಡೀ ಉತ್ತರ ಅಮೇರಿಕದ ಆಕ್ರಮಣವಾಯಿತು. ಅಲ್ಲಿನ ಮೂಲನಿವಾಸಿಗಳಾದ ರೆಡ್ ಇಂಡಿಯನರ ಅಪಾರಮಟ್ಟದ ಸುಲಿಗೆ ಕೊಲೆ ನಡೆದು, ಅವರನ್ನು ಸದೆಬಡಿದು ತೀರ ಅಲ್ಪಸಂಖ್ಯೆಯ ಜನಾಂಗವನ್ನಾಗಿ ಮಾಡಲಾಯಿತು. ಅತ್ಯಂತ ವಿಶಾಲವಾದ ದಕ್ಷಿಣ ಅಮೇರಿಕಾ ಖಂಡದ ಬಹುಭಾಗ ಸ್ಪೆಯಿನಿನ ವಶವಾಯಿತು. ಆಸ್ಟ್ರೇಲಿಯಾ, ಆಫ್ರಿಕಾ ಖಂಡಗಳ ಅತ್ಯುತ್ತಮ ಭಾಗಗಳೆಲ್ಲ ಅಲ್ಲಿ ನೆಲಸಲಾರಂಭಿಸಿದ ಬಿಳಿ ಜನರ ಕೈ ಸೇರಿತು. ನೀಗ್ರೋಗಳು ಬಿಳಿ ಜನರ ಕೂಲಿ ಕೆಲಸಕ್ಕೆ ಮಾತ್ರ ಉಪಯುಕ್ತರಾದರು!

ಪಾಶ್ಚಾತ್ಯ ಕ್ರಿಶ್ಚಿಯನ್​ ಜನಾಂಗ ಹೀಗೆಯೇ ದಬ್ಬಾಳಿಕೆಯಿಂದ ಭೂಲೋಕದ ಬೇರೆಬೇರೆ ಭಾಗಗಳನ್ನು ಕೊಳ್ಳೆ ಹೊಡೆಯುತ್ತ ಹೋದಾಗ, ಆ ದೇಶದ ಪಾದ್ರಿಗಳು ಅವರನ್ನು ಹಿಂಬಾಲಿಸುತ್ತ ಅಲ್ಲಿನ ಮೂಲ ನಿವಾಸಿಗಳ ‘ಉದ್ಧಾರ’ಕ್ಕೆ ಬಲಾತ್ಕಾರದ ಮತಾಂತರ ಕಾರ್ಯವನ್ನು ಕೈಗೊಂಡರು. ತಮ್ಮ ಇಂಥ ‘ಪವಿತ್ರ ಕಾರ್ಯ’ಕ್ಕೆಂದು ಕೊಳ್ಳೆಯ ಅಪಾರ ಸಂಪತ್ತಿನ ಒಂದಂಶವನ್ನು ಅಲ್ಲಿನ ಶಾಲೆ ಆಸ್ಪತ್ರೆಗಳಿಗಾಗಿ ದಾನ ಮಾಡುವುದೂ ಒಂದು ಕ್ರಮವಾಯಿತು. ಇದು ಪರೋಪಕಾರ ಬುದ್ಧಿಯಿಂದಲ್ಲ–ಅಲ್ಲೆಲ್ಲ ಕ್ರೈಸ್ತ ಧರ್ಮದ ಹಿರಿಮೆ ಸ್ಥಾಪಿಸಲು, ತಮ್ಮ ಪಂಗಡವನ್ನು ಬಲಗೊಳಿಸಲು.

ಅಮೇರಿಕದ ಹಾರ್​ವರ್ಡ್ ವಿಶ್ವವಿದ್ಯಾನಿಲಯದ ಸಮಾಜಶಾಸ್ತ್ರ ವಿಭಾಗದ ಮುಖ್ಯಸ್ಥರೂ, ವಿಖ್ಯಾತರೂ ಆದ ಪಿಟ್ರಿಮ್ ಎ.\ ಸೊರೊಕಿನ್ನರು ಈ ವಿಚಾರವಾಗಿ ‘ಎ ರಿಕನ್​ಸ್ಟ್ರಕ್ಶನ್ ಆಫ್ ಹ್ಯುಮ್ಯಾನಿಟಿ’ಯಲ್ಲಿ ಹೇಳಿದ ಕೆಲವು ಮಾತುಗಳನ್ನೇ ಇಲ್ಲಿ ಉಲ್ಲೇಖಿಸಬಹುದಾಗಿದೆ–‘ಕಳೆದ ಕೆಲವು ಶತಮಾನಗಳಲ್ಲಿ ಜಗತ್ತಿನಲ್ಲೆಲ್ಲ ಅತಿ ಹೆಚ್ಚಿನ ಯುದ್ಧವನ್ನು ಕೆಣಕುವ, ಕಾರಣವಿಲ್ಲದೆ ಆಕ್ರಮಣ ನಡೆಯಿಸುವ, ಅತಿ ಹೆಚ್ಚಿನ ದುರಾಶೆಯ, ದರ್ಪಮತಾಂಧತೆಯ, ಮಾನವ \hbox\bgroup ತಂಡವೆಂದರೆ\egroup\break ಪಾಶ್ಚಾತ್ಯ ಕ್ರಿಶ್ಚಿಯನ್​ ವರ್ಗ. ಈ ಶತಮಾನಗಳಲ್ಲಿ ಅದು ಇತರ ದೇಶಗಳಿಗೆ ನುಗ್ಗಿತು; ಅದರ ಸೈನ್ಯವು ಆ ದೇಶದ ಪಾದ್ರಿಗಳಿಂದ ಮತ್ತು ವ್ಯಾಪಾರಿಗಳಿಂದ ಹಿಂಬಾಲಿಸಲ್ಪಟ್ಟು, ಕ್ರಿಶ್ಚಿಯನ್ ಅಲ್ಲದ ಜನಾಂಗಗಳನ್ನೆಲ್ಲ ಅಡಿಯಾಳಾಗಿ ಮಾಡಿತು, ಅವರನ್ನು ಸುಲಿಯಿತು, ದರೋಡೆ ಮಾಡಿತು. ಅಮೇರಿಕದ ಮೂಲ ನಿವಾಸಿಗಳು, ಆಫ್ರಿಕದವರು, ಆಸ್ಟ್ರೇಲಿಯಾದವರು, ಏಷ್ಯಾದವರು–ಎಲ್ಲರೂ ಈ ತೆರದ “ಕ್ರಿಶ್ಚಿಯನ್ ಪ್ರೀತಿ”ಯ ಅನುಭವ ಹೊಂದಬೇಕಾಯಿತು. ಆ ಪ್ರೀತಿಯ ರೂಪವೆಂದರೆ ಮೂಲನಿವಾಸಿಗಳ ನಾಶ, ದಾಸ್ಯ, ಬಲಾತ್ಕಾರ, ಅವರ ನಾಗರಿಕತೆಯ ನಾಶ, ಅವರ ಸಂಸ್ಕೃತಿಯ ನಾಶ, ಜೀವನ ಕ್ರಮಗಳ ನಾಶ, ಮದ್ಯಪಾನದ ಅಪಾರ ಪ್ರಚಾರ, ಗುಹ್ಯರೋಗದ ಪ್ರಸಾರ, ನೀತಿ ನಿಯಮಗಳಿಲ್ಲದ ವ್ಯಾಪಾರ ವ್ಯವಹಾರ–ಇವೇ ಮೊದಲಾದವುಗಳು. ಕೆಲವು ರೀತಿಯಿಂದ ಅವರು ಆ ದೇಶಗಳಿಗೆ ಉಪಕಾರ ಮಾಡಿದ್ದಾರೇನೋ ದಿಟ; ಆದರೆ ಅದು ಗುಡಾಣ ತುಂಬ ಇರುವ ನೀರಿನಲ್ಲಿ ಒಂದು ಬಿಂದುವಿನಷ್ಟು ಮಾತ್ರ! ಅದು ಅವರ ಕಾರ್ಯದ ಅನಿವಾರ್ಯ ಉಪಲಕ್ಷಣವಾಗಿ ಮಾತ್ರ.’


\section*{ಹಳೆ ಬೇರು, ಹೊಸ ಚಿಗುರು}

\addsectiontoTOC{ಹಳೆ ಬೇರು, ಹೊಸ ಚಿಗುರು}

ನಮ್ಮ ಸಂಸ್ಕೃತಿ ವೃಕ್ಷದ ಸತ್ವ ಬಲಿಷ್ಠವಾಗಿದೆ ಎನ್ನುವುದಕ್ಕೆ ಇಂಥ ಆಧುನಿಕ ಪ್ರತಿಕೂಲ ಪರಿಸ್ಥಿತಿ\-ಯಲ್ಲೂ ಅದು ಪರಿಸರದ ಪ್ರಭಾವವನ್ನು ಮೆಟ್ಟಿನಿಂತು ಸಾಕಷ್ಟು ಸತ್ಫಲಗಳನ್ನು ನೀಡಿದುದೇ ನಿದರ್ಶನ. ಸಂಕಟಮಯ ಸಂಕ್ರಾಂತಿ ಕಾಲದಲ್ಲಿ ಆಧ್ಯಾತ್ಮಿಕ ವೀರರು ಉದಿಸಿ ಜನರಿಗೆ ಮಾರ್ಗ\-ದರ್ಶನ ಮಾಡಿದ ಸಂಗತಿಗಳು ಅತಿಶಯೋಕ್ತಿಯ ದಂತಕಥೆಗಳಲ್ಲ. ಅಥವಾ ಈ ಘಟನೆಗಳು ಪುರಾಣಯುಗಕ್ಕೆ ಮೀಸಲಲ್ಲ. ಬ್ರಿಟಿಷ್ ಚಕ್ರಾಧಿಪತ್ಯ ನೆಲೆಗೊಂಡು, ಭಾರತದ ಕ್ಷಾತ್ರಶಕ್ತಿ ದುರ್ಬಲವಾಗಿ, ಆರ್ಥಿಕವಾಗಿ ಅವನತವಾಗಿ, ಧಾರ್ಮಿಕ ಸಾಂಸ್ಕೃತಿಕವಾಗಿಯೂ ಸತ್ವ ಹೀನವೆಂದು ಸರ್ವತ್ರ ಪ್ರಚಾರವಾಗುತ್ತಿರುವ ವೇಳೆ, ಸ್ವದೇಶೀಯರೇ ವಿದೇಶೀಯರನ್ನು ಗುಲಾಮರಂತೆ ಅನುಕರಿಸಲು ತೊಡಗಿದಾಗ, ದಕ್ಷಿಣೇಶ್ವರದ ದೇವಮಾನವರು ಕಾಣಸಿಕೊಂಡರು. ಅವರೇ ಭಗವಾನ್ ಶ‍್ರೀರಾಮ\-ಕೃಷ್ಣ ಪರಮಹಂಸರು. ಬ್ರಿಟಿಷ್ ಚಕ್ರಾಧಿಪತ್ಯದ ರಾಜಧಾನಿಯಲ್ಲೇ ಅವರು ಕಾಣಿಸಿಕೊಂಡರು. ವಿಶ್ವವಿದ್ಯಾಲಯಗಳಲ್ಲಿ ಆಧುನಿಕ ವಿದ್ಯಾಭ್ಯಾಸ ಮಾಡುತ್ತಿದ್ದ ಯುವಕ ಸಂದೇಹವಾದಿಗಳಿಗೆ\break ಪಂಥಾಹ್ವಾನ ನೀಡಿ ಅವರನ್ನು ತಮ್ಮ ಗರಡಿಯಲ್ಲಿ ಪಳಗಿಸಿದರು. ಸತ್ಯದ ಬೆಳಕಿನಲ್ಲಿ ವಿವಿಧ\break ಧರ್ಮಗಳ ಅರ್ಥವೇನು ಎಂಬುದನ್ನು ಅವರು ಬೋಧಿಸಿದರು. ಧರ್ಮದಲ್ಲಿ ತಿರುಳು\break ಯಾವುದು, ಮರಳು ಯಾವುದು ಎಂಬುದನ್ನು ತೋರಿಸಿದರು. ಕೇವಲ ಬೋಧನೆಯಿಂದಲೇ ಅಲ್ಲ, ಸಾಧನೆಯಿಂದ, ಅನುಭೂತಿಯಿಂದ, ಬದುಕಿನಿಂದ, ವಿಜ್ಞಾನದ ನೂತನ ಸಂಶೋಧನೆಗಳಿಂದ ಉದ್ಭವಿಸಿದ ಸಂಶಯಗಳನ್ನು ಎದುರಿಸಿ ನಿಲ್ಲುವ ಸಾಮರ್ಥ್ಯ ಋಷಿಗಳು ಕಂಡ ಸನಾತನ ನಿಯಮಕ್ಕಿದೆ ಎಂಬುದನ್ನು ಸಾರಿದರು. ಅವರ ಪ್ರಮುಖ ಶಿಷ್ಯರಲ್ಲೊಬ್ಬರಾದ ಸ್ವಾಮಿ\break ವಿವೇಕಾನಂದರು, ಸನಾತನ ಧರ್ಮದ ಸಾಕಾರ ಪ್ರತಿನಿಧಿಯಾಗಿ ಚಿಕಾಗೊ ಸರ್ವಧರ್ಮ\break ಸಮ್ಮೇಳನದಲ್ಲಿ ಭಾರತದ ಧರ್ಮಸಂದೇಶದ ವಿಜಯಪತಾಕೆಯನ್ನು ಹಾರಿಸಿ, ಭಾರತೀಯರ ಆತ್ಮಗ್ಲಾನಿಯನ್ನು ದೂರಮಾಡಿದರು. ತಮ್ಮ ‘ಸ್ವ’ತ್ವವನ್ನೂ, ಸತ್ವಶಕ್ತಿಗಳನ್ನೂ ಉಳಿಸಿಕೊಂಡು, ಉಜ್ವಲ ರಾಷ್ಟ್ರಾಭಿಮಾನಿಗಳಗಾಗಿ, ಪರಕೀಯರ ಆಳ್ವಿಕೆಯಿಂದ ಮುಕ್ತರಾಗುವ ಹಂಬಲಕ್ಕೆ ಸ್ಫೂರ್ತಿ ನೀಡಿದ ಮಹಾಘಟನೆ ಅದು. ಅಂದಿನ ವಿದ್ಯಾವಂತರಲ್ಲಿ ಆತ್ಮಾಭಿಮಾನ ರಾಷ್ಟ್ರಾಭಿಮಾನ\-ವನ್ನುಂಟು\-ಮಾಡಿ, ಬಹುಜನಹಿತ ಬಹುಜನಸುಖಕ್ಕೆ, ತ್ಯಾಗ ಮತ್ತು ಸೇವೆಯ ಪಥವನ್ನು ದಿಗ್ದರ್ಶಿ\-ಸಿದ ಘಟನೆಯೂ ಹೌದು. ಮುಂದೆ ಜಗತ್ತಿನ ಯಾವ ದೇಶವೂ ಸ್ವಾತಂತ್ರ್ಯವನ್ನು ಪಡೆಯದ ರೀತಿಯಲ್ಲಿ, ಮಹಾತ್ಮಾಗಾಂಧೀಜಿ ಅವರ ನೇತೃತ್ವದಲ್ಲಿ ದೇಶಕ್ಕೆ ಸ್ವರಾಜ್ಯ ಸಿದ್ಧಿಯಾಯಿತು. ಜಗತ್ತಿನ ಗುಣಗ್ರಾಹಿಗಳೆಲ್ಲರೂ ಮಹಾತ್ಮರ ನೈತಿಕ ಸಮುನ್ನತಿ, ಶ್ರದ್ಧೆ, ತ್ಯಾಗ, ಸೇವೆಯೇ ಮೊದಲಾದ ಗುಣಗಳನ್ನು ಕಂಡು ಗೌರವದಿಂದ ತಲೆಬಾಗಿದರು. ‘ಭಾರತ ಮಾತ್ರ ಅಂಥ ಮಹಾತ್ಮನನ್ನು ಸೃಷ್ಟಿಸೀತು, ಹಿಂದೂಧರ್ಮ ಮಾತ್ರ ಅಂಥ ಉನ್ನತ ವ್ಯಕ್ತಿಯನ್ನು ಜಗಕ್ಕೆ ನೀಡೀತು’ ಎಂದು ಉದ್ಗರಿಸಿದ ಪಶ್ಚಿಮದ ಮನೀಷಿಗಳಿದ್ದರು! ತೀವ್ರ ತಪಸ್ಸಿನ ಋಷಿಜೀವನವನ್ನು ನಡೆಯಿಸಿ, ಜಗತ್ತಿನ ಸತ್ಯಾನ್ವೇಷಿಗಳನ್ನು ತಮ್ಮೆಡೆಗೆ ಸೆಳೆದ ಶ‍್ರೀ ರಮಣ ಮಹರ್ಷಿಗಳಾಗಲಿ, ಶ‍್ರೀ ಅರವಿಂದರಾಗಲಿ ೨೦ನೇ ಶತಮಾನದ ಐದನೇ ದಶಕದವರೆಗೂ ಬಾಳಿ ಬದುಕಿದವರು. ಅಧಿಕಾರ ಸ್ಥಾನದಲ್ಲಿ ದೀರ್ಘಕಾಲವಿದ್ದ ಇತರ ದೇಶದ ರಾಜಕೀಯ ಮುಖಂಡರು ಹೇಗೆ ನಡೆದು\-ಕೊಂಡ\-\break ರೆಂಬುದು ವಿದ್ಯಾವಂತರಿಗೆ ತಿಳಿದ ವಿಚಾರವೆ. ಸ್ಟಾಲಿನ್, ಹಿಟ್ಲರ್, ಮುಸೊಲಿನಿ ಮೊದಲಾದವರು ನಿರಂಕುಶಪ್ರಭುಗಳಾಗಿ ಅಪಾರ ಸಂಖ್ಯೆಯ ಜನನಾಶ, ಹಿಂಸಾಚರಣೆ ಮಾಡಿದ್ದನ್ನು ಮರೆಯು\-ವಂತಿಲ್ಲ. ಆದರೆ ಭಾರತದಲ್ಲಿ ದೀರ್ಘಾವಧಿ ಪ್ರಧಾನಮಂತ್ರಿಯಾಗಿದ್ದ ನೆಹರೂ ಪ್ರಜಾಪ್ರಭುತ್ವ\-ವನ್ನುಳಿಸಿ ಲೋಕದ ಶಾಂತಿಗಾಗಿ ಎಡೆಬಿಡದೆ ಯತ್ನಿಸಿದರು. ಈ ಘಟನೆ ಸಾಧ್ಯವಾದುದು ನಮ್ಮ ಸಂಸ್ಕೃತಿಯ ಪ್ರಭಾವದಿಂದಲ್ಲವೆ? ಹೀಗೆ ವೈಜ್ಞಾನಿಕತೆ, ವಿಚಾರವಾದ, ಆಧುನೀಕರಣಗಳ ಗಾಳಿ ಅದೆಷ್ಟು ಬಲವಾಗಿ ಬೀಸಿದರೂ, ಬಲವಾಗಿ ಬೇರು ಬಿಟ್ಟು ಸಂಸ್ಕೃತಿವೃಕ್ಷ ಕಂಗೆಡದೆ ಹೊಸ ಚಿಗುರು ಬಿಡುತ್ತಲೇ ಹೋದುದನ್ನು ಇತಿಹಾಸದುದ್ದಕ್ಕೂ ನೋಡಬಹುದು.

ಜನರನ್ನು ಒಂದುಗೂಡಿಸುವ, ರಾಷ್ಟ್ರಹಿತಕ್ಕಾಗಿ ದುಡಿಯುವ ಪ್ರೇರಣೆಯನ್ನು ನೀಡುವ, ನಮ್ಮ ಸಂಸ್ಕೃತಿಯ ಮೂಲಸ್ರೋತವಾದ ಧರ್ಮ ಮತ್ತು ಆಧ್ಯಾತ್ಮಿಕತೆಯ ಮಹತ್ವಿಕೆಯ ಅರಿವನ್ನು ಎಲ್ಲ ದೇಶವಾಸಿಗಳಲ್ಲೂ ಉಂಟುಮಾಡದೆ, ನಮ್ಮ ಜನಾಂಗಕ್ಕೆ ಕ್ಷೇಮವಿದೆಯೆ? ಭವಿಷ್ಯವಿದೆಯೆ?


\section*{ಅಧ್ಯಾತ್ಮವೇ ಪ್ರೀತಿಯ ತವರು}

\addsectiontoTOC{ಅಧ್ಯಾತ್ಮವೇ ಪ್ರೀತಿಯ ತವರು}

ಪ್ರೀತಿಯ ಮಹಿಮೆಯನ್ನು ಹೇಳುತ್ತ, ನಿಃಸ್ವಾರ್ಥ, ದಿವ್ಯಪ್ರೀತಿಯ ಮೂಲಸ್ರೋತ ಅಧ್ಯಾತ್ಮವೇ ಎಂಬುದನ್ನು ವಿವರಿಸಲು ಸುತ್ತುಬಳಸಿ ಬಂದಂತಾಯಿತು. ಅಧ್ಯಾತ್ಮವನ್ನು ತೊರೆದರೆ ಜನಾಂಗಕ್ಕೆ ಭವಿಷ್ಯವಿಲ್ಲ ಎಂಬುದನ್ನು ಪ್ರತಿಯೊಬ್ಬರೂ ಮನಗಾಣಬೇಕಾಗಿದೆ. ಸಮಸ್ತ ಮಾನವ ಜನಾಂಗ\-ದೆಡೆಗೆ ಈ ದೈವೀಪ್ರೀತಿ ಅಥವಾ ನಿಃಸ್ವಾರ್ಥಪ್ರೀತಿ, ಆತ್ಮಸಾಕ್ಷಾತ್ಕಾರ ಮಾಡಿಕೊಂಡ ಮಹಾತ್ಮರ ಮೂಲಕ ವಿಶೇಷ ರೀತಿಯಲ್ಲಿ ಹರಿಯುತ್ತದೆ ಎಂಬುದು ಘಟಿಸಿದ ಸತ್ಯ ಸಂಗತಿ. ಈ ದೇಶದಲ್ಲಿ ಅಂಥವರ ಸಂಖ್ಯೆ ಹೆಚ್ಚಾಗಿ ಕಂಡುಬಂದರೂ, ಎಲ್ಲ ದೇಶಗಳಲ್ಲೂ ಅಂಥ ಮಹಾತ್ಮರು ಉದಿಸಿ ಜನರಿಗೆ ಮಾರ್ಗದರ್ಶನ ಮಾಡಿದ್ದಾರೆ. ಪ್ರತಿಯೊಬ್ಬರೂ ಆ ದೈವೀಪ್ರೀತಿಯನ್ನು ಬೆಳೆಸಿಕೊಳ್ಳಬಹುದು, ಬೆಳೆಸಿಕೊಳ್ಳಬೇಕು. ಈ ಧಾರ್ಮಿಕ ಶಿಕ್ಷಣದ ಅಭಾವದಿಂದ ಜಗತ್ತು ವಿನಾಶದೆಡೆಗೆ ಓಡುತ್ತಲಿದೆ.


\section*{ಕೋಟಿ ತಾಯಂದಿರ ಹೃದಯ}

\addsectiontoTOC{ಕೋಟಿ ತಾಯಂದಿರ ಹೃದಯ}

ಮಹಾಪುರುಷರಲ್ಲಿ ಸಹಸ್ರಾರು ಮಮತಾಮಯಿ ಮಾತೆಯರ ಅನುಕಂಪೆಯ ಹೃದಯವಿರುತ್ತದೆ. ಅಂಥದಕ್ಕೊಂದು ಉಜ್ವಲ ನಿದರ್ಶನ ಸ್ವಾಮಿ ವಿವೇಕಾನಂದರದು. ಅವರ ಪ್ರೀತಿ ಎಷ್ಟೊಂದು ಆಳ, ಎಷ್ಟೊಂದು ವಿಶಾಲ! ಚಿಕಾಗೋ ಸರ್ವಧರ್ಮ ಸಮ್ಮೇಳನದಲ್ಲಿ ಪ್ರಸಿದ್ಧರಾದ ದಿನ ಶ‍್ರೀಮಂತರ ಮನೆ ಬಾಗಿಲುಗಳೆಲ್ಲ ಅವರಿಗೆ ತೆರೆಯಲ್ಪಟ್ಟಾಗ ಅವರು ರಾತ್ರಿ ನಿದ್ರೆ ಮಾಡಲೇ ಇಲ್ಲ. ತಮ್ಮ ವೈಭವದ ಬಗ್ಗೆ, ತಮ್ಮ ವಾಗ್ವೈಖರಿಯ ಬಗ್ಗೆ ಅವರು ಯೋಚಿಸುತ್ತಿದ್ದುದಲ್ಲ. ‘ಅಯ್ಯೋ, ನನ್ನ ದೇಶದಲ್ಲಿ ಜನರಿಗೆ ಹೊತ್ತಿನ ತುತ್ತಿಗೂ ಗತಿ ಇಲ್ಲವಲ್ಲ! ಅವರನ್ನೆಂತು ಉದ್ಧರಿಸಲಿ!’ ಎಂದು ಕಾತರಿಸುತ್ತಾ ಕಂಬನಿಗರೆಯುತ್ತಿದ್ದರವರು. ಅವರ ಹೃದಯ ಕೋಟಿ ತಾಯಂದಿರ ಮಡಿಲಾಗಿತ್ತು. ಕರುಣೆಯ ಕಡಲಾಗಿತ್ತು.

ಅವರೇ ಬಹಳ ಕಷ್ಟಪಟ್ಟು ಬೇಲೂರು ಮಠವನ್ನೇನೋ ಸ್ಥಾಪಿಸಿದರು. ಆದರೆ ಊರಿಗೆ ಊರೇ ಪ್ಲೇಗ್​ನ ಹಾವಳಿಯಿಂದ ಕಂಗೆಡುವ ಕಾಲ ಬಂತು. ನೊಂದ ಆ ಜನರ ನೆರವಿಗೆ ಧಾವಿಸಲು ಅವರ ಹೃದಯ ತುಡಿಯಿತು. ‘ಅವರ ರಕ್ಷಣೆಗಾಗಿ ಮಠವನ್ನು ಮಾರುತ್ತೇನೆ, ಮರದಡಿಯಲ್ಲಿ\break ನಾನಿರುತ್ತೇನೆ’ ಎಂದವರು ವ್ಯಥೆಯಿಂದ ಉದ್ಗರಿಸಿದ್ದುಂಟು! ತಾಯಿಯಾದವಳಿಗೆ ಮಗುವು ನರಳುವಾಗ ‘ಅಯ್ಯೋ!’ ಎಂದು ಕರುಳು ಹಿಡಿದಂತಾಗುವಂತೆ, ಕೋಟಿ ಮಾತೆಯರ ಹೃದಯ ಸಂಪನ್ನರಾದ ಅವರಿಗೆ ಮನುಷ್ಯನ ನೋವು ಅದೆಷ್ಟು ಸಂಕಟ ತರುತ್ತಿರಬಹುದೆಂದು ನೀವೇ ಯೋಚಿಸಿ.

ಒಮ್ಮೆ ಅವರು ಬೇಲೂರು ಮಠದಲ್ಲಿದ್ದಾಗ ಒಂದು ರಾತ್ರಿ ಸುಮಾರು ಒಂದು ಗಂಟೆಯ ಹೊತ್ತಿಗೆ ನಿದ್ರೆಯಿಂದೆದ್ದು ತಮ್ಮ ಕೋಣೆಯ ಹೊರಗಡೆ ವರಾಂಡದಲ್ಲಿ ಶತಪಥ ಮಾಡುತ್ತಿದ್ದರು. ಸ್ವಲ್ಪ ಹೊತ್ತಿನ ಬಳಿಕ ಶ‍್ರೀರಾಮಕೃಷ್ಣರ ಇನ್ನೊಬ್ಬ ಶಿಷ್ಯರಾದ ಸ್ವಾಮಿ ವಿಜ್ಞಾನಾನಂದಜಿ\break ವಿವೇಕಾನಂದರನ್ನು ಪ್ರಶ್ನಿಸಿದರು: ‘ಸ್ವಾಮೀಜಿ, ನಿದ್ರೆ ಬರಲಿಲ್ಲವೇ?’ ವಿವೇಕಾನಂದರು ಉತ್ತರಿಸಿದರು: ‘ನಿದ್ರೆ ಬಂದಿತ್ತು. ಆದರೆ ಒಂದು ಗಂಟೆಯ ಹೊತ್ತಿಗೆ ಬಹುಜನರು ಅತ್ಯಂತ ಸಂಕಟದಿಂದ ಆರ್ತನಾದ ಮಾಡುವುದನ್ನು ಕೇಳಿದಂತಾಗಿ ಎಚ್ಚರವಾಯಿತು ನೋಡು.’ ವಿಜ್ಞಾನಾನಂದಜಿ ಈ ಆರ್ತನಾದ ಕೇಳಿದ ವಿಚಾರ ಅರ್ಥವಾಗದೆ ಸುಮ್ಮನಾಗಿ, ತಮ್ಮ ಕೋಣೆಗೆ ಹಿಂದಿರುಗಿದರು. ಮಾರನೇ ದಿನ ಪತ್ರಿಕೆಯಲ್ಲಿ ವರದಿ ಪ್ರಕಟವಾಗಿತ್ತು: ಸಾವಿರಾರು ಮೈಲು ದೂರದ ಭೂಮಧ್ಯ ಸಮುದ್ರದ ಸಮೀಪ ಭೂಕಂಪವಾಗಿ ಸಹಸ್ರಾರು ಮಂದಿ ಮಡಿದಿದ್ದರು. ಪ್ರಾಣೋತ್ಕ್ರಮಣ ಕಾಲದಲ್ಲಿ ಅವರ ಆರ್ತನಾದಕ್ಕೆ ವಿವೇಕಾನಂದರ ಮಾತೃಹೃದಯ ಸ್ಪಂದಿಸಿತ್ತು.

ಸಾಮಾನ್ಯ ತಾಯಿಯೊಬ್ಬಳು ತನ್ನ ಪ್ರೀತಿಪಾತ್ರರ, ಮಕ್ಕಳ ನೋವನ್ನು ಕಣ್ಣೆದುರು ಕಾಣಿಸ\-ದಿದ್ದರೂ ಕೆಲವೊಮ್ಮೆ ಅನುಭವಿಸಬಲ್ಲಳು. ಆದರೆ ಇತರರ ದುಃಖವನ್ನು ಆ ರೀತಿ ಗುರುತಿಸಲಾರಳು. ಆದರೆ ಆತ್ಮಾನುಭೂತಿಯನ್ನು ಪಡೆದ, ನಿಃಸ್ವಾರ್ಥತೆಯ ಪರಾಕಾಷ್ಠೆಯನ್ನೇರಿದ ಮಹಾತ್ಮರು ಲೋಕ ದುಃಖವನ್ನು ಎಲ್ಲ ಸ್ಥಿತಿಯಲ್ಲೂ ತಿಳಿಯಬಲ್ಲರು. ಕಷ್ಟಸಹಿಷ್ಣುಗಳಾಗಿ ಅಪಾರ ತ್ಯಾಗದಿಂದ, ಯಾರಿಂದ ಏನನ್ನೂ ಅಪೇಕ್ಷಿಸದೆ, ಜೀವಿಗಳ ದುಃಖ ದೂರಮಾಡಲು ಅವರು ಅಹರ್ನಿಶಿ ದುಡಿಯುವರು. ಭೀಕರವಾದ ಸಂಸಾರ ಸಾಗರವನ್ನು ದಾಟಿದ ಅವರು ಯಾವ ಸ್ವಾರ್ಥವೂ ಇಲ್ಲದೇ ಇತರರನ್ನೂ ಸಂಕಟಗಳಿಂದ ಪಾರು ಮಾಡಬಲ್ಲರು ಎನ್ನುವುದು ಅತಿಶಯೋಕ್ತಿಯ\break ವರ್ಣನೆಯಲ್ಲ. ಸತ್ಯಸ್ಯ ಸತ್ಯ.


\section*{ದುಃಖಕ್ಕೆ ಮಿಡಿಯುವ ಹೃದಯ}

\addsectiontoTOC{ದುಃಖಕ್ಕೆ ಮಿಡಿಯುವ ಹೃದಯ}

‘ಮುಖಂಡರಾಗಲು ಇಚ್ಛಿಸುವವರಲ್ಲಿ ಮೊದಲಿಗೆ ಇರಬೇಕಾದುದು ಅದೇ ಹೃದಯಸಂಪನ್ನತೆ’ ಎಂದಿದ್ದರು ವಿವೇಕಾನಂದರು. ಅವರೇ ಹೇಳಿರುವಂತೆ–

‘ಮಹತ್​ಕಾರ್ಯ ಸಾಧನೆಗೆ ಮೂರು ಮುಖ್ಯ ವಿಷಯಗಳು ಆವಶ್ಯಕ. ಮೊದಲು ಹೃದಯ ಪೂರ್ವಕ ಅನುಭವಿಸಿ. ನೀವು ಸಂವೇದನಾಶೀಲರಾಗಿದ್ದೀರಾ? ಕೋಟ್ಯಂತರ ದೇವ ಮತ್ತು ಋಷಿಗಳ ವಂಶಜರು ಪಶುಸದೃಶರಾಗಿರುವರು ಎಂಬ ವ್ಯಥೆ ನಿಮ್ಮ ಹೃದಯದಲ್ಲಿ ಉಂಟಾಗಿದೆಯೆ? ಕೋಟ್ಯಂತರ ಜನ ಉಪವಾಸದಿಂದ ನರಳುತ್ತಿದ್ದಾರೆ, ಕೋಟ್ಯಂತರ ಜನ ಹಿಂದಿನಿಂದಲೂ ಉಪವಾಸದಿಂದ ನರಳುತ್ತಿದ್ದರು ಎಂಬುದು ನಿಮಗೆ ಗೊತ್ತೆ? ಇದು ನಿಮ್ಮನ್ನು ಚಂಚಲರನ್ನಾಗಿ ಮಾಡಿದೆಯೆ? ಇದು ನಿಮ್ಮ ನಿದ್ರೆಗೆ ಭಂಗ ತಂದಿದೆಯೆ? ಇದು ನಿಮ್ಮ ರಕ್ತದಲ್ಲಿ ವ್ಯಾಪಿಸಿದೆಯೆ? ನಾಡಿನಲ್ಲಿ ಸಂಚರಿಸಿ ಹೃದಯಚಲನೆಯಲ್ಲಿ ಸ್ಪಂದಿಸುತ್ತಿದೆಯೇ? ಇದು ಹೆಚ್ಚು ಕಡಿಮೆ ನಿಮ್ಮನ್ನು ಉನ್ಮತ್ತರನ್ನಾಗಿಸಿದೆಯೆ? ಈ ಸರ್ವನಾಶದ ದುಃಖವೊಂದೇ ನಿಮ್ಮನ್ನು ವ್ಯಾಪಿಸಿ, ನಿಮ್ಮ ಕೀರ್ತಿ, ಯಶಸ್ಸು, ಹೆಂಡಿರು, ಮಕ್ಕಳು, ಆಸ್ತಿ ಮತ್ತು ನಿಮ್ಮ ದೇಹವನ್ನೂ ನೀವು ಮರೆತಿರುವಿರಾ? ಇದನ್ನು ನೀವು ಸಾಧಿಸಿರುವಿರಾ? ಇದೇ ದೇಶಭಕ್ತರಾಗುವುದಕ್ಕೆ ಮೊದಲನೇ ಹೆಜ್ಜೆ. ಈ ಸಮಸ್ಯೆಯಿಂದ ಪಾರಾಗುವುದಕ್ಕೆ ಯಾವುದಾದರೊಂದು ಮಾರ್ಗವನ್ನು ಕಂಡುಹಿಡಿದಿರುವಿರಾ? ಅವರನ್ನು ಟೀಕಿಸಿ ನಿಂದಿಸದೆ ಸಹಾಯವನ್ನು ನೀಡುವ, ವರ್ತಮಾನದ ಮೃತ್ಯುಸ್ಥಿತಿಯಿಂದ ಅವರನ್ನು ಮೇಲೆತ್ತಿ ಅವರ ದುಃಖವನ್ನು ಶಮನಗೊಳಿಸುವ ಸಾಂತ್ವನದ ನುಡಿಯನ್ನು ಉಚ್ಚರಿಸಬಲ್ಲಿರಾ? ಇಷ್ಟು ಮಾತ್ರವೇ ಅಲ್ಲ. ಇಡಿಯ ಜಗತ್ತೇ ಕತ್ತಿ ಹಿಡಿದು ನಿಮಗೆದುರಾಗಿ ನಿಂತರೂ ಸರಿಯೆಂದು ಯೋಚಿಸಿ ನಿರ್ಧರಿಸಿದುದನ್ನು ಮಾಡುವ ಧೈರ್ಯ ನಿಮಗಿದೆಯೆ?.... ಅದಕ್ಕೆ ಅಂಟಿಕೊಂಡು ಅದನ್ನೇ ಹಿಂಬಾಲಿಸುತ್ತ ಎಡೆಬಿಡದೆ ಗುರಿಯೆಡೆಗೆ ಧಾವಿಸಬಲ್ಲಿರಾ? ನಿಮ್ಮಲ್ಲಿ ಆ ದೃಢನಿಷ್ಠೆ ಇದೆಯೇ? ನಿಮ್ಮಲ್ಲಿ ಈ ಮೂರು ಗುಣಗಳು ಇದ್ದರೆ ನೀವು ಪ್ರತಿಯೊಬ್ಬರೂ ಕೂಡ ಮಹಾದ್ಭುತ ಪವಾಡದಂಥ ಕಾರ್ಯಗಳನ್ನು ಸಾಧಿಸಬಲ್ಲಿರಿ.’

\vskip 2pt

ಇಂದಿನ ಸಾಮಾಜಿಕ, ರಾಜಕೀಯ, ಧಾರ್ಮಿಕ ಮುಖಂಡರಾಗುವವರಲ್ಲಿ, ಮಹಾಪುರುಷರಲ್ಲಿ ಕಂಡುಬರುವ ಈ ಪ್ರೀತಿ ಹಾಗೂ ಮರುಕದ ಒಂದಂಶವಿದ್ದರೂ ಜನಾಂಗ ಮೇಲೇಳುತ್ತದೆ.


\section*{ಹೆಚ್ಚಲಿ ಜೀವದಯೆ}

\addsectiontoTOC{ಹೆಚ್ಚಲಿ ಜೀವದಯೆ}

ತನ್ನ ಸಮೀಪದ, ಸುತ್ತಮುತ್ತಲಿನ ಜೀವಿಗಳಲ್ಲಿ ಮರುಕ ತೋರುವುದು, ಕಾರುಣ್ಯ ಮತ್ತು ಪ್ರೀತಿಯಿಂದ ನಡೆದುಕೊಳ್ಳುವುದು, ಶ್ರೇಷ್ಠ ಧರ್ಮಾಚರಣೆ ಎಂಬುದು ಮಹಾಭಾರತದ ಒಂದು ಉಪದೇಶ.

\vskip 2pt

ಧರ್ಮರಾಜನು ಜೊತೆಗೆ ಬಂದ ನಾಯಿಯನ್ನು ಅದರ ಪಾಡಿಗೇ ಬಿಟ್ಟು ಸ್ವರ್ಗಕ್ಕೆ ಹೋಗಲು ಒಪ್ಪಲಿಲ್ಲ. ಅಂತೂ ಧರ್ಮನಿಷ್ಠೆಯ ಈ ಕೊನೆಯ ಪರೀಕ್ಷೆಯಲ್ಲಿ ಆತ ಉತ್ತೀರ್ಣನಾಗಿ ಧರ್ಮ\-ಮೂರ್ತಿ ಎನಿಸಿಕೊಂಡ. ಸಾಕುನಾಯಿಯಲ್ಲದಿದ್ದರೂ ಅದನ್ನು ನೋಡಿಕೊಳ್ಳಲು ಜೊತೆಯಲ್ಲಿ ಯಾರೂ ಇಲ್ಲದಿದ್ದ ಕಾರಣ ಅದನ್ನು ಏಕಾಕಿಯಾಗಿ ಬಿಟ್ಟು ತಾನೊಬ್ಬನೇ ಸ್ವರ್ಗವೇರಲಾರೆ ಎಂದ ಈ ಘಟನೆ ಆತನ ಹೃದಯದ ಮರುಕ, ಅನುಕಂಪ ಹಾಗೂ ದಯಾಭಾವವನ್ನು ತೋರಿಸುತ್ತದಷ್ಟೆ. ಈ ಮರುಕವೇ ಆತನನ್ನು ಮಹಾತ್ಮನನ್ನಾಗಿಸಿತು.

\vskip 2pt

ಪರಮ ದಯಾಮಯನಾದ ಭಗವಂತನನ್ನು ಆಶೀರ್ವಾದಮಾಡು, ಕರುಣೆದೋರು,\break ಕಾಪಾಡು ಎಂದೇನೋ ನಾವು ಪ್ರಾರ್ಥಿಸುತ್ತೇವೆ. ಆದರೆ ನಮ್ಮ ಸುತ್ತಮುತ್ತಲಿರುವ ಆತನು ಸೃಷ್ಟಿಸಿದ ಜೀವಜಂತುಗಳಿಗೆ ಕರುಣೆ\-ದೋರಬೇಕು ಎಂಬ ಅರಿವು ನಮಗಿದೆಯೆ? ಹಾಗೆ ಕರುಣೆ\-ದೋರದಿದ್ದರೆ, ನಮ್ಮ ಪ್ರಾರ್ಥನೆ ಎಷ್ಟೊಂದು ಸಂಕುಚಿತ ಹಾಗೂ ಸ್ವಾರ್ಥಪರ ಎಂದಾಗದೆ? ವರ್ಷಕ್ಕೊಂದು ಬಾರಿ ಗೋಪೂಜೆಯನ್ನು ಮಾಡಿ, ಪುಣ್ಯಕಟ್ಟಿಕೊಂಡು, ಗೋವನ್ನು ಸರಿಯಾಗಿ ನೋಡಿಕೊಳ್ಳಲಾರದ ಪರಂಪರೆಯ ಉದಾಹರಣೆಯಂತೆ ಆಗದೇ ಇದು?

\vskip 2pt

ಪ್ರಾಣಿಗಳೊಡನೆ ಕ್ರೂರವಾಗಿ ವರ್ತಿಸುವವರು ಒಳ್ಳೆಯ ವ್ಯಕ್ತಿಗಳಾಗಲಾರರು ಎಂಬುದು ತತ್ತ್ವಜ್ಞಾನಿ ಷೋಫೆನೀರನ ಮತ.

\vskip 2pt

ಮನುಷ್ಯರು ತಮಗೆ ತೋರಿದ ಪ್ರೀತಿಗೆ ಪ್ರಾಣಿಗಳು ಎಂದೂ ಕೃತಘ್ನತೆಯನ್ನು ತೋರವು ಎನ್ನುವುದಕ್ಕೆ ನಿದರ್ಶನಗಳಿವೆ. ಗೋವಾದಲ್ಲಿ ಒಮ್ಮೆ ಹುಚ್ಚು ಮದವೇರಿದ ಆನೆಯೊಂದು ಪೇಟೆಯ ಮಾರುಕಟ್ಟೆಯ ರಸ್ತೆಯಲ್ಲಿ ಎದುರಿಗೆ ಸಿಕ್ಕ ವಸ್ತುಗಳನ್ನೂ, ಜನರನ್ನೂ ನಾಶಮಾಡುತ್ತ ಓಡುತ್ತಿತ್ತು. ದಾರಿಯಲ್ಲಿ ಒಂದು ಶಿಶುವನ್ನು ಕಂಡು, ಅಲ್ಲೆ ಕೊಂಚಕಾಲ ನಿಂತು, ತನ್ನ ಸೊಂಡಿಲಿನಿಂದ ಆ ಮಗುವನ್ನು ಎತ್ತಿ ಯಾವ ತೊಂದರೆಯನ್ನೂ ಮಾಡದೇ ಅಂಗಡಿಯ ಜಗಲಿಯ ಮೇಲಿರಿಸಿ, ಮತ್ತೆ ತಿರುಗಿ ರುದ್ರ ತಾಂಡವಕ್ಕೆ ಪ್ರಾರಂಭಿಸಿತು. ಮಗುವಿನ ತಾಯಿ, ವರ್ಷಗಳ ಹಿಂದೆ, ಆನೆ ಆ ಬೀದಿಯಲ್ಲಿ ಬರುವಾಗ ನಿತ್ಯವೂ ಅದಕ್ಕೆ ಏನಾದರೂ ತಿನಿಸು ಕೊಡುತ್ತಿದ್ದಳು. ಹುಚ್ಚು ಹಿಡಿದ ಸ್ಥಿತಿಯಲ್ಲೂ ಆನೆ ಅದನ್ನು ಮರೆಯಲಿಲ್ಲ!

ಚಿತ್ರಕೂಟ ಪಯೋಷ್ಣೀ ನದಿಯ ಘಾಟಿನಲ್ಲಿ ನಡೆದ ಘಟನೆ ಇದು. ಸಂಜೆಯ ಹೊತ್ತು ಚಿಕ್ಕ ಹುಡುಗನೊಬ್ಬ ಕಾಲುಜಾರಿ ಹರಿಯುವ ನದಿಯಲ್ಲಿ ಬಿದ್ದು ತೇಲಿಕೊಂಡು ಹೋಗುತ್ತಿದ್ದ. ಹತ್ತಿರದಲ್ಲೇ ಇದ್ದ ಆತನ ತಾಯಿ ಅಸಹಾಯಕಳಾಗಿ ದೈನ್ಯದಿಂದ ಅಳುತ್ತ ಏನು ಮಾಡುವುದೆಂದು ತೋಚದೆ ಮರಗಟ್ಟಿ ಹೋದಂತಾಯಿತು. ಥಟ್ಟನೆ ನೀರಿಗೆ ಹಾರಿದ ಸದ್ದು ಕೇಳಿದಂತಾಗಿ ಆ ಕಡೆ ನೋಡಿದಾಗ ದೊಡ್ಡ ಕೋತಿಯೊಂದು ನೀರಿನಲ್ಲಿ ಬಿದ್ದ ಹುಡುಗನನ್ನು ಎತ್ತಿತಂದು ಅಳುತ್ತಿರುವ ತಾಯಿಯ ಎದುರಲ್ಲಿಟ್ಟು ಅಲ್ಲಿಂದ ಮಾಯವಾಯಿತು!

ಕಳೆದ ಮಹಾಯುದ್ಧದಲ್ಲಿ ನಾಯಿಗಳು ಬೇರೆ ಬೇರೆ ರೀತಿಯಲ್ಲಿ ಸೈನಿಕರಿಗೆ ನೆರವಾದ ಘಟನೆಗಳಿವೆ. ಒಂದು ಬೆಲ್ಜಿಯಂ ಪೋಲೀಸುನಾಯಿ ಒಂದು ವರ್ಷದಲ್ಲಿ ಎರಡು ಸಾವಿರ ಜೀವ ಉಳಿಸಿ ಅದ್ಭುತವನ್ನು ಸಾಧಿಸಿತ್ತು!

ಮೂಕಜಂತುಗಳಿಗೆ ಪ್ರೀತಿ ಅರ್ಥವಾಗುವುದಿಲ್ಲವೆ? ಅವುಗಳನ್ನು ಕ್ರೂರ ರೀತಿಯಿಂದ ನೋಡಿಕೊಳ್ಳಬೇಡಿ. ದುರಾಶೆಯಿಂದ ಪೀಡಿತರಾದ ಜನರು ತಮಗೆ ಉಪಕಾರವನ್ನು ಮಾಡಿದ ಪ್ರಾಣಿಗಳಿಗೂ ಎಂಥ ಕೃತಘ್ನತೆಯನ್ನು ತೋರಿಸುತ್ತಾರೆ! ಪ್ರಾಣಿಗಳನ್ನು ಕ್ರೂರ ರೀತಿಯಿಂದ ನೋಡಿಕೊಳ್ಳುವವರಿಗೆ ಇಂಗ್ಲೆಂಡಿನಲ್ಲಿ ಜುಲ್ಮಾನೆ ವಿಧಿಸುತ್ತಾರೆ! ಜನಸಂಖ್ಯೆಯಲ್ಲಿ ಇಂಗ್ಲೆಂಡು ನಮಗಿಂತ ಹತ್ತುಪಾಲು ಕಡಿಮೆಯಾದರೂ ಅಲ್ಲಿ ಅರವತ್ತಮೂರು ಪ್ರಾಣಿಗೃಹಗಳೂ, ಎಪ್ಪತ್ತೊಂಬತ್ತು ಪ್ರಾಣಿ ಔಷಧಾಲಯಗಳೂ, ನೂರೆಂಬತ್ತು ಪ್ರಾಣಿಹಿತ ಕೇಂದ್ರಗಳೂ ಇವೆ. ಸರಕಾರದಿಂದ ಸಹಾಯ ಪಡೆಯದ ಹಲವಾರು ಪ್ರಾಣಿಗೃಹಗಳಿವೆ. ಪ್ರಾಣಿಗಳ ಉಪಚಾರಕ್ಕೆ ಸಂಚಾರೀ ಆಸ್ಪತ್ರೆಗಳಿವೆ. ಅಲ್ಲಿ ವರುಷಕ್ಕೆ ಸುಮಾರು ೨,೯೦,೦೦೦ ಪ್ರಾಣಿಗಳು ಔಷಧೋಪಚಾರವನ್ನು ಪಡೆಯುವುವು ಎಂದು ವರದಿಯಾಗಿದೆ. ವಿದೇಶೀಯರ ಪ್ರಕೃತಿಪ್ರೇಮ ಅನುಕರಣೀಯವೇ ಸರಿ.

ಗೌತಮಬುದ್ಧ ‘ಜಗತ್ತಿಗೆ ಬಂದ ಪ್ರತಿಯೊಂದು ಜೀವಿಯೂ ಸುಖಕ್ಕಾಗಿ ಹಾತೊರೆಯುತ್ತದೆ. ಎಲ್ಲರಿಗೂ ದಯೆ ತೋರಿ’ ಎಂದರೆ, ಬಸವಣ್ಣನವರು ‘ದಯೆ ಇಲ್ಲದ ಧರ್ಮ ಯಾವುದಯ್ಯಾ?’ ಎಂದು ದಯಾಪರರಾಗಲು ಕರೆಯಿತ್ತರು.


\section*{ಅಹಿಂಸೆಯ ಅಭಿವ್ಯಕ್ತಿ}

\addsectiontoTOC{ಅಹಿಂಸೆಯ ಅಭಿವ್ಯಕ್ತಿ}

ಅಹಿಂಸೆಯಲ್ಲಿ ದೃಢಪ್ರತಿಷ್ಠನಾದರೆ ಆತನ ಸಾನ್ನಿಧ್ಯದಲ್ಲಿ ಪ್ರಾಣಿಗಳು ತಮಗೆ ಸಹಜವಾಗಿ ಬಂದ ವೈರವನ್ನು ಬಿಟ್ಟುಬಿಡುತ್ತವೆ ಎಂದು ಯೋಗಸೂತ್ರಕಾರರಾದ ಪತಂಜಲಿ ಮಹಾಮುನಿ ಹೇಳುತ್ತಾರೆ. ಭಾರತೀಯ ತಪೋವನ ವರ್ಣನೆಗಳಲ್ಲಿ ಹುಲ್ಲೆಗಳು ಹುಲಿಯ ಜೊತೆ ಹಿಂಸೆಯ ಭೀತಿ ಇಲ್ಲದೆ ಸಂಚರಿಸುವ ವಿಚಾರವನ್ನು ನಾವು ಓದುತ್ತೇವೆ. ಇದು ಕೇವಲ ಕವಿಕಲ್ಪನೆ ಎನ್ನಲಾಗದು. ಈ ಘಟನೆಗಳಿಗೆ ವಾಸ್ತವಿಕ ಹಿನ್ನೆಲೆ ಇದೆ.

ಅಹಿಂಸೆ ಎಂದರೆ ಮನಸ್ಸು ಮಾತು ಕೃತಿಗಳಲ್ಲಿ, ಇತರರಿಗೆ ಕೆಡುಕನ್ನೂ ನೋವನ್ನೂ ಉಂಟು ಮಾಡದೇ ಇರುವುದು ಎಂಬರ್ಥದಲ್ಲೇ ಪರ್ಯವಸಾನವಾಗುವುದಿಲ್ಲ. ಅಹಿಂಸೆ ಎಂದರೆ ಎಲ್ಲ\-ರೆಡೆಗೂ ಪರಿಶುದ್ಧ ಪ್ರೀತಿ ಎಂದೂ ಆಗುತ್ತದೆ. ಪ್ರೀತಿ ಸ್ನೇಹ ಕರುಣೆಗಳು ಅಹಿಂಸೆಯ ಇನ್ನೊಂದು ಮುಖ. ಸಂತರು ಅಹಿಂಸೆಯ ಪೂರ್ಣಸಿದ್ಧಿಯನ್ನು ಪಡೆದಾಗ ಅವರ ಹೃದಯದಿಂದ ಪರಿಶುದ್ಧ ಪ್ರೀತಿಯ ತರಂಗ, ಎಲ್ಲರೆಡೆಗೂ ಹರಿಯುತ್ತದೆ. ಅಂಥ ಮಹಾತ್ಮರ ಸಾನ್ನಿಧ್ಯಕ್ಕೆ ಹೋದಾಗ, ಧ್ಯಾನ, ಧಾರಣೆಗಳ ಅಭ್ಯಾಸ ಮಾಡಿರದಿದ್ದ ಸಾಮಾನ್ಯರೂ, ಮನಶ್ಶಾಂತಿಯ ಅನುಭವ ಹೊಂದಿದ ಉದಾಹರಣೆಗಳಿವೆ. ಅವರ ಈ ಪ್ರೇಮ ಮತ್ತು ನಿರ್ಭೀತಿಯ ಸ್ಪಂದನದ ಅನುಭವವನ್ನು ಪಶುಪಕ್ಷಿಗಳೂ ಪಡೆಯಬಲ್ಲವು ಎಂಬುದನ್ನು ಅವುಗಳ ವರ್ತನೆಯಿಂದ ತಿಳಿದ ಉದಾ ಹರಣೆಗಳಿವೆ.

ರಮಣ ಮಹರ್ಷಿಗಳು ತಮ್ಮ ಆಶ್ರಮದಲ್ಲಿದ್ದಾಗ ಕಾಡುಪ್ರಾಣಿಗಳೂ, ಪಕ್ಷಿಗಳೂ, ನಿರ್ಭೀತಿಯಿಂದ ಅವರ ಸಮೀಪಕ್ಕೆ ಬಂದು ತಿಂಡಿತಿನಿಸನ್ನು ಪಡೆದುಹೋಗುತ್ತಿದ್ದವು. ಪಶುಪಕ್ಷಿಗಳ ಮೇಲಣ ಅವರ ಪ್ರೀತಿ ಅಸದೃಶ. ಅವರು ಪ್ರಾಣಿಗಳನ್ನು ಕುರಿತು ಮಾತನಾಡುವಾಗ, ನಾವು ಸಾಮಾನ್ಯವಾಗಿ ಎನ್ನುವಂತೆ ‘ಅದು’ ಎನ್ನದೆ ‘ಅವನು’, ‘ಅವಳು’ ಎಂದೇ ಹೇಳುತ್ತಿದ್ದರು. ‘ಮಕ್ಕಳಿಗೆ ಅನ್ನಕೊಟ್ಟರೇ?’ ಎಂದರೆ ‘ನಾಯಿಗಳಿಗೆ ಅನ್ನಕೊಟ್ಟರೇ?’ ಎಂದರ್ಥ. ‘ಲಕ್ಷ್ಮೀ ಊಟ ಮಾಡಿದಳೇ?’ ಎಂದರೆ ‘ಲಕ್ಷ್ಮೀ ಎಂಬ ಹೆಸರಿನ ಹಸು ತಿನಿಸು ತಿಂದಿತೇ?’ ಎಂದರ್ಥ. ನವಿಲುಗಳನ್ನು ಅವುಗಳ ಕೂಗಿನ ಅನುಕರಣೆಯಿಂದವರು ಕರೆಯುತ್ತಿದ್ದರು. ಅವರು ಕೊಟ್ಟ ಬೇಳೆಕಾಳು, ಅಕ್ಕಿ, ಮಾವಿನಹಣ್ಣನ್ನು ತಿಂದು ಅವು ತಮ್ಮ ತಮ್ಮ ಸ್ಥಾನಗಳಿಗೆ ಹಿಂದಿರುಗುತ್ತಿದ್ದವು. ತಮ್ಮ ದೇಹತ್ಯಾಗದ ಮುಂಚಿನ ದಿನ, ಡಾಕ್ಟರರ ಹೇಳಿಕೆಯಂತೆ, ತೀವ್ರ ನೋವಿನ ರೋಗದಿಂದ ಅವರು ನರಳುತ್ತಿದ್ದರೂ, ಸಮೀಪದ ಮರದ ಮೇಲಿದ್ದ ನವಿಲುಗಳ ಕೇಕೆಯನ್ನು ಕೇಳಿ ‘ಅವು ತಮ್ಮ ಪಾಲಿನ ಉಣಿಸನ್ನು ಪಡೆದವೆ?’ ಎಂದು ವಿಚಾರಿಸಿದ್ದರು! ಗುಬ್ಬಚ್ಚಿಗಳು ಅವರು ಕುಳಿತಲ್ಲಿಗೆ ಹಾರಿಬಂದು ಅವರ ಕೈಯಿಂದಲೇ ಕಾಳುಗಳನ್ನು ಆರಿಸಿಕೊಳ್ಳುತ್ತಿದ್ದವು!

ಆಶ್ರಮದ ವಠಾರದಲ್ಲಿ ಅವರು ಹಾವುಗಳನ್ನು ಕೊಲ್ಲಲು ಬಿಡುತ್ತಿರಲಿಲ್ಲ. ‘ನಾವು ಅವುಗಳ ರಾಜ್ಯಕ್ಕೆ ಬಂದಿದ್ದೇವೆ. ಅವುಗಳಿಗೆ ತೊಂದರೆ ಕೊಡುವುದು ಉಚಿತವಲ್ಲ. ಅವು ನಮಗೆ ಕೆಡುಕನ್ನು ಮಾಡುವ ಉದ್ದೇಶದಿಂದ ಇಲ್ಲಿಗೆ ಬರುವುದಿಲ್ಲ’ ಎನ್ನುತ್ತಿದ್ದರವರು. ಒಮ್ಮೆ ಅವರು ಗುಡ್ಡದ ಮೇಲೆ ಕುಳಿತಿದ್ದಾಗ ನಾಗರಹಾವೊಂದು ಅವರ ಕಾಲಿನ ಮೇಲೇರಿ ಅಲ್ಲೇ ಸ್ವಲ್ಪ ಹೊತ್ತಿದ್ದು, ಆಮೇಲೆ ಹೊರಟು ಹೋಯಿತು. ಅವರು ಕೊಂಚವೂ ಅಲುಗಾಡದೆ, ಭೀತರೂ ಆಗದೆ ಕುಳಿತಿದ್ದರು. ಹಾವು ಹರಿದಾಗ ಹೇಗೆನಿಸಿತು ಎಂದು ಕೇಳಿದಾಗ, ‘ತಣ್ಣಗೆ, ಮೆತ್ತಗೆ’ ಎಂದಿದ್ದರು!


\section*{ಹಾವಿನೊಡನೆ ಸರಸಬೇಡ!}

\addsectiontoTOC{ಹಾವಿನೊಡನೆ ಸರಸಬೇಡ !}

ರಮಣಮಹರ್ಷಿಗಳು ಅಲುಗಾಡದೇ ನಿರ್ಭೀತಿಯಿಂದ ಕುಳಿತಿದ್ದುದರಿಂದ ಆ ಹಾವು ಯಾವ ತೊಂದರೆಯನ್ನೂ ಕೊಡದೆ ಹೊರಟುಹೋಯಿತು, ನಿಜ. ಹಾಗೆಂದು ಕ್ರೂರ ಜಂತುಗಳಿಗೆ ಮನುಷ್ಯರ ಪ್ರೀತಿ–ಸದ್ಭಾವನೆ ಎಲ್ಲ ವೇಳೆಯಲ್ಲೂ ಅರ್ಥವಾಗುತ್ತದೆಂದೆಣಿಸುವುದು ಸರಿಯಲ್ಲ. ಹಿರಿಯರೊಬ್ಬರು ತಮ್ಮ ಒಂದು ಅನುಭವವನ್ನು ಹೀಗೆಂದು ಹೇಳಿದ್ದರು: ಒಂದು ಕಡೆ ಭೀಕರ ಪ್ರವಾಹ ಬಂದಾಗ ಒಬ್ಬ ವ್ಯಕ್ತಿ, ಒಂದು ನಾಗರಹಾವು ಇಬ್ಬರೂ ಒಂದೇ ಮರದಲ್ಲಿ ಮೂರುದಿನಗಳ ಕಾಲ ನೆಲಸಿದ್ದರು. ಆಗ ಆ ಹಾವು ಅವರಿಗೆ ಯಾವ ತೊಂದರೆಯನ್ನೂ ಕೊಡದೆ ಅದರ ಪಾಡಿಗೆ ಅದು ಇದ್ದುಕೊಂಡಿತ್ತು. ಪ್ರವಾಹ ಇಳಿದ ಬಳಿಕ ಆ ವ್ಯಕ್ತಿಯನ್ನು ಹುಡುಕಿಕೊಂಡು ಬಂದ ಅವನ ಬಂಧುಗಳು ಅವನನ್ನು ಮರದಿಂದ ಕೆಳಕ್ಕಿಳಿಸಿದರು. ಹಾವಿನಿಂದ ಬೀಳ್ಕೊಳ್ಳುವ ಸಮಯದಲ್ಲಿ ಅವರು, ಮೂರು ದಿನಗಳೂ ತೊಂದರೆ ಕೊಡದೆ ಇದ್ದುದಕ್ಕಾಗಿ ಪ್ರೀತಿವಿಶ್ವಾಸದಿಂದಲೇ ಅದರ ಬೆನ್ನನ್ನು ಸವರಿದರು. ಹಾವು ಥಟ್ಟನೇ ಸಿಟ್ಟಿಗೆದ್ದು ಅವರನ್ನು ಬಲವಾಗಿ ಕಚ್ಚಿ ಬಿಟ್ಟಿತು. ಅಲ್ಲೇ ಕುಸಿದುಬಿದ್ದು ಕೆಲಕ್ಷಣಗಳಲ್ಲೇ ಅವರು ಅಸುನೀಗಿದರು!

ಆಶ್ರಮದ ನಾಯಿಗಳಲ್ಲಿ ಕಮಲಾ ಬುದ್ಧಿವಂತಳು. ಆಶ್ರಮಕ್ಕೆ ಬಂದ ಭಕ್ತರನ್ನೂ ಅತಿಥಿಗಳನ್ನೂ ಕರೆದುಕೊಂಡು ಹೋಗಿ, ಗಿರಿಪ್ರದಕ್ಷಿಣೆ ಮಾಡಿಸಲು ಕೆಲವೊಮ್ಮೆ ಮಹರ್ಷಿಗಳು ಅವಳಿಗೆ ಹೇಳುತ್ತಿದ್ದರು. ಗಿರಿಯ ಸುತ್ತಲೂ ಇದ್ದ ಮೂರ್ತಿ, ಕೊಳ, ಗುಡಿಗಳೇ ಮೊದಲಾದ ಎಲ್ಲಾ ಜಾಗಗಳಿಗೂ ಅತಿಥಿಗಳನ್ನು ಅವಳು ಕರೆದೊಯ್ಯುತ್ತಿದ್ದಳು.

ಪರಮಜ್ಞಾನಿಗಳಿಗೆ ಸಹಜವಾದ ನಿರ್ಭೀತಿ, ಪ್ರೇಮ, ದಯೆ, ಅನುಕಂಪವೇ ಮೊದಲಾದ ಗುಣಗಳಿಂದ ಕೂಡಿದ್ದ ಅವರು, ಪ್ರಾಣಿಗಳ ಕೂಗುಗಳನ್ನೂ, ವರ್ತನೆಯ ವಿಧಾನಗಳನ್ನೂ, ಅವು ಅನುಸರಿಸುವ ಸಾಮಾಜಿಕ ನಿಯಮಗಳನ್ನೂ ಪರಿಶೀಲಿಸಿದ್ದರು. ಮಂಗಗಳ ಗುಂಪುಗಳೊಳಗೆ ಜಗಳವಾದಾಗ ಅವು ಅವರನ್ನು ಸಮೀಪಿಸಿ, ಅವರ ಮಧ್ಯಸ್ಥಿಕೆಯಲ್ಲೆ, ತಮ್ಮ ಜಗಳ ತೀರ್ಮಾನ ಮಾಡಿಕೊಳ್ಳುತ್ತಿದ್ದವೆಂದು ಅವರು ಹೇಳಿದ್ದಾರೆ! ಸಾಮಾನ್ಯವಾಗಿ ಅರಣ್ಯದಿಂದ ಹಿಡಿದು ತಂದು ಸಾಕಿದ ಕಪಿ, ಮನುಷ್ಯರ ಸಹವಾಸವನ್ನು ತೊರೆದು ಕಾಡಿಗೆ ಹಿಂದಿರುಗಿದರೆ, ಅಲ್ಲಿನ ಕಪಿಗಳು ಅದನ್ನು ತಮ್ಮ ಗುಂಪಿಗೆ ಸೇರಿಸಿಕೊಳ್ಳದೆ ಬಹಿಷ್ಕರಿಸುತ್ತವೆ. ಆದರೆ ಮಹರ್ಷಿಗಳನ್ನು ಕಂಡು ಅವರ ಜೊತೆಗಿದ್ದು ಬಂದ ಕಪಿಗೆ ಈ ನಿಯಮವಿರಲಿಲ್ಲ. ಸಂತೋಷದಿಂದಲೇ ಅದನ್ನು ಅವು ತಮ್ಮ ಗುಂಪಿಗೆ ಬರಮಾಡಿಕೊಳ್ಳುತ್ತಿದ್ದವು! ಒಮ್ಮೆ ಮಹರ್ಷಿಗಳು ಗಿರಿಪ್ರದಕ್ಷಿಣೆ ಮಾಡುತ್ತಿದ್ದರು. ಅವರಿಗೆ ಹಸಿವು, ನೀರಡಿಕೆಗಳಾಗುತ್ತಲೇ ಎತ್ತಣಿಂದಲೋ ಮಂಗಗಳ ಗುಂಪೊಂದು ಅವರ ಸಮೀಪದಲ್ಲಿದ್ದ ಕಾಡು ಅತ್ತಿಹಣ್ಣಿನ ಮರವನ್ನೇರಿ ಹಣ್ಣುಗಳನ್ನು ಉದುರಿಸಿ ತಾವು ತಿನ್ನದೆ ಅಲ್ಲಿಂದ ಹೊರಟು ಹೋದವು!

ಪ್ರೀತಿಗೆಂತಹ ಅದ್ಭುತ ಶಕ್ತಿ! ಪ್ರಾಣಿ, ಪಕ್ಷಿಗಳನ್ನೂ ಸದ್ಭಾವನೆಯಿಂದ ನೋಡಿಕೊಳ್ಳಿ ಎಂಬು\-ದನ್ನು ಸರ್ವಭೂತಹಿತೈಷಿಗಳಾದ ಭಗವಾನ್ ರಮಣ ಮಹರ್ಷಿಗಳು ತೋರಿಸಿ ಒಂದು ಮಹಾ ಆದರ್ಶವನ್ನು ಮನುಕುಲಕ್ಕೆ ನೀಡಿದ್ದಾರೆ.


\section*{ದ್ವೇಷ ಮಹಾದೋಷ}

\addsectiontoTOC{ದ್ವೇಷ ಮಹಾದೋಷ}

‘ದ್ವಿಷ್​’ನಿಂದ ದ್ವೇಷ ಶಬ್ದದ ವ್ಯುತ್ಪತ್ತಿ. ದ್ವಿ ವಿಷ ಎಂದರೆ ವಿಷದ ಎರಡು ಪಾಲು–ವಿಷದ ಆಧಿಕ್ಯ ಎಂಬುದಾಗಿ ದ್ವೇಷದ ಅರ್ಥವನ್ನು ತಿಳಿಯಬಹುದು. ಯಾರು ದ್ವೇಷ ಮಾಡುತ್ತಾರೋ ಅವರು ಉರಿಯುತ್ತಾರೆ. ಯಾರನ್ನು ದ್ವೇಷ ಮಾಡುತ್ತಾರೋ ಅವರನ್ನೂ ಉರಿಸುತ್ತಾರೆ. ದ್ವೇಷದಿಂದ ಮನುಷ್ಯ ನಡೆದಾಡುವ ವಿಷದ ಕಾರ್ಖಾನೆಯಾಗುತ್ತಾನೆ. ತಾನು ಕೆಟ್ಟು ಇತರರನ್ನು ನಾಶ ಗೊಳಿಸುತ್ತಾನೆ. ದ್ವೇಷ ಜ್ವಲನಾತ್ಮಕ ಚಿತ್ತವೃತ್ತಿ.

ಪ್ರೀತಿಯಿಂದ ಪರಸ್ಪರ ಹಂಚಿಕೊಳ್ಳುವಿಕೆ, ಪರಸ್ಪರ ಲಕ್ಷ್ಯ, ಪರಸ್ಪರ ಕಾಳಜಿ, ಪರಸ್ಪರ ಒಪ್ಪಿಗೆ, ಸ್ವೀಕಾರ, ಪರಸ್ಪರ ತಿಳಿವಳಿಕೆ, ಸಾಮರಸ್ಯ ಸಾಧ್ಯವಾಗುತ್ತದೆ. ಆದರೆ ದ್ವೇಷದಿಂದ ಪರಸ್ಪರ ಅಸ್ವೀಕಾರ, ಅಸಂಬದ್ಧ ನಡವಳಿಕೆ, ಪರಸ್ಪರ ವಿರಸ, ತಪ್ಪು ತಿಳಿವು, ಒಬ್ಬರ ಬಗ್ಗೆ ಇನ್ನೊಬ್ಬರಿಗೆ ಸ್ವಲ್ಪವೂ ಕಾಳಜಿ ಇಲ್ಲದಿರುವುದು, ಮತ್ಸರ, ಅನ್ಯಾಯ–ಇವು ಕಾಣಿಸಿಕೊಂಡು ಮನುಷ್ಯನನ್ನು ದುಃಖ ದುರಂತಕ್ಕೆಳೆಸುತ್ತದೆ; ಅವನ ನಾಶಕ್ಕೆ ಕಾರಣವಾಗುತ್ತವೆ. ನಿಜವಾಗಿಯೂ ದ್ವೇಷ, ಮಾತ್ಸರ್ಯ\-ಗಳಿಲ್ಲದಿರುತ್ತಿದ್ದರೆ ಈ ಜಗತ್ತು ಸ್ವರ್ಗದ ನಂದನವನವಾಗುತ್ತಿತ್ತು!

ಇಂದಿನ ಜಗತ್ತಿನಲ್ಲಿ ಬುದ್ಧಿವಂತರೂ, ದಡ್ಡರೂ, ವಿದ್ಯಾವಂತರೂ, ಅವಿದ್ಯಾವಂತರೂ, ರಾಜ\-ನೀತಿಜ್ಞರೂ, ಮುಖಂಡರೂ, ಧಾರ್ಮಿಕರೂ, ಅಧಾರ್ಮಿಕರೂ ಈ ದ್ವೇಷದ ಬೆಂಕಿಯನ್ನು ಪ್ರಜ್ವಲಗೊಳಿಸುತ್ತಲೇ ಇರುತ್ತಾರೆ! ಇದೇನಾಶ್ಚರ್ಯ!

ದ್ವೇಷವು ಭಕ್ತಿಯ ವಿರೋಧ ಭಾವವೆಂದು ಭಕ್ತಿಶಾಸ್ತ್ರದ ಆಚಾರ್ಯರು ಹೇಳುತ್ತಾರೆ. ಎಲ್ಲಿಯವರೆಗೆ ಹೃದಯದಲ್ಲಿ ದ್ವೇಷವಿರುವುದೋ ಅಲ್ಲಿಯವರೆಗೆ ಭಕ್ತಿ ಇರಲು ಸಾಧ್ಯವಿಲ್ಲ.

ನೀವು ಯಥಾರ್ಥ ಧಾರ್ಮಿಕರಾದರೆ, ಭಗವಂತನಲ್ಲಿ ನಿಮಗೆ ದೃಢವಾದ ಭಕ್ತಿ ಇದ್ದುದಾದರೆ, ನೀವು ಯಾರನ್ನೂ ದ್ವೇಷಿಸಲಾರಿರಿ.

ನಿಃಸ್ವಾರ್ಥ ಪ್ರೀತಿಯ ಅಮೃತವನ್ನು ಹೃದಯದಲ್ಲಿ ಮೊದಲು ತುಂಬಿಕೊಂಡು, ಬಳಿಕ ಅದನ್ನು ಎಲ್ಲರಿಗೂ ಹಂಚುವವರು, ಸಮಾಜಕ್ಕೆ ಅತೀವ ಕಲ್ಯಾಣವನ್ನುಂಟುಮಾಡುವ ಧರ್ಮಾ\-ತ್ಮರು–ಮಹಾತ್ಮರು.

ದ್ವೇಷದ ವಿಷವನ್ನು ಪ್ರಸಾರ ಮಾಡುವವರು ಸಮಾಜದ ಸ್ವಾಸ್ಥ್ಯವನ್ನು ಕೆಡಿಸುವ ದಾನವರು– ದುರ್ಯೋಧನರು.


\section*{ಹೊಟ್ಟೆಯಲ್ಲಿ ಬೆಂಕಿ!}

\addsectiontoTOC{ಹೊಟ್ಟೆಯಲ್ಲಿ ಬೆಂಕಿ!}

ದ್ವೇಷಪ್ಪನವರ ಪತ್ನಿಯೇ ಮತ್ಸರಮ್ಮ. ಅವರು ಬಹಳ ಅನ್ಯೋನ್ಯವಾಗಿರುತ್ತಾರೆ. ಜೊತೆ ಜೊತೆಯಾಗಿಯೇ ಅವರ ಸಂಚಾರ. ಮತ್ಸರಕ್ಕೆ ಕನ್ನಡದಲ್ಲಿ ಸರಿಯಾದ ಅರ್ಥಪೂರ್ಣ ಶಬ್ದವಿದೆ. ಹೊಟ್ಟೆಕಿಚ್ಚು ಎನ್ನುವ ಆ ಶಬ್ದ, ಹೊಟ್ಟೆಯಲ್ಲಿ ಉಂಟಾಗುವ ತಳಮಳ, ಸಂಕಟ ಹಾಗೂ\break ವೇದನೆಗಳನ್ನು ಸೂಚಿಸುತ್ತದೆ. ಇದು ಆರೋಗ್ಯಕರವಾದ ಕಿಚ್ಚಲ್ಲ. ಪರರ ಅಭ್ಯುದಯವನ್ನು ಕಂಡಾಗ, ಹಲವರ ಹೊಟ್ಟೆಯಲ್ಲಿ ಈ ರೀತಿ ಹೊಟ್ಟೆ ಸಂಕಟವಾಗುತ್ತದೆ. ಇದೇ ಹೊಟ್ಟೆಯ ಕಿಚ್ಚು.

ಅರಿಷಡ್ವರ್ಗಗಳಲ್ಲಿ ಆರನೆಯದೇ ಮತ್ಸರ. ಇದು ಅನೇಕ ಅನಾಹುತಗಳಿಗೆ ಕಾರಣವಾದದ್ದು. ಕಾಮ, ಕ್ರೋಧ, ಲೋಭ, ಮೋಹ, ಮದಗಳನ್ನು ಬಿಟ್ಟವರುಂಟು. ಅಂಥವರೂ ಮತ್ಸರಗ್ರಸ್ತರಾಗಿ, ಇತರರಿಗೆ ಅನ್ಯಾಯ ವಂಚನೆಗಳನ್ನು ಮಾಡಲು ಹಿಂಜರಿಯುವುದಿಲ್ಲ. ಮತ್ಸರವಿಲ್ಲದವರೇ ಮಹಾತ್ಮರು.

ಸ್ವಾಮಿ ವಿವೇಕಾನಂದರು ಅಹಿಂಸೆ ಎಂದರೆ ಮತ್ಸರರಾಹಿತ್ಯ ಎಂಬ ಒಂದು ನಿರೂಪಣೆ\break ನೀಡಿದ್ದಾರೆ.

ಇಬ್ಬರು ತಪಸ್ವಿಗಳು, ನದಿಯ ಇಕ್ಕೆಲಗಳಲ್ಲಿ ಕುಟೀರ ಕಟ್ಟಿಕೊಂಡು ತಪಸ್ಸನ್ನಾಚರಿಸುತ್ತಿದ್ದರು. ದೀರ್ಘಕಾಲದ ನಂತರ ಭಗವಂತ ಅವರಲ್ಲಿ ಒಬ್ಬನಿಗೆ ಪ್ರತ್ಯಕ್ಷನಾಗಿ ‘ನಿನಗೇನು ಬೇಕು?’ ಎಂದು ಕೇಳಿದ. ತಪಸ್ವಿ ಹೇಳಿದ: ‘ನದಿಯ ಆ ಕಡೆಯಲ್ಲಿ ಕುಟೀರ ಕಟ್ಟಿಕೊಂಡು ತಪಸ್ಸು ಮಾಡು\-ತ್ತಿದ್ದಾರಲ್ಲ, ಅವರಿಗೆ ಲಭ್ಯವಾದುದರ ಎರಡರಷ್ಟು ನನಗೆ ಸಿಗುವಂತಾಗಬೇಕು.’ ಭಗವಂತ ನದಿಯ ಆ ಕಡೆಯಲ್ಲಿ ಕಾಣಿಸಿಕೊಂಡು, ಮತ್ತೊಬ್ಬ ತಪಸ್ವಿಯನ್ನು ‘ನಿನಗೇನು ಬೇಕು?’ ಎಂದು ಕೇಳಿದಾಗ, ಆತ ಮೊದಲನೆಯ ತಪಸ್ವಿಯ ಆಕಾಂಕ್ಷೆ ಏನೆಂದು ಮೊದಲು ಭಗವಂತನಿಂದ ತಿಳಿದುಕೊಂಡು, ಆ ಬಳಿಕ ‘ನನ್ನ ಒಂದು ಕಣ್ಣು ಕುರುಡಾಗಲಿ’ ಎಂದು ಬೇಡಿಕೊಂಡ. ಎಂದರೆ ಮೊದಲನೆಯ ತಪಸ್ವಿಗೆ ಎರಡನೆಯ ತಪಸ್ವಿಗೆ ಲಭ್ಯವಾದುದರ ಎರಡರಷ್ಟು ಫಲ ದೊರಕಿತು. ಎಂದರೆ ಎರಡು ಕಣ್ಣುಗಳೂ ಕುರುಡಾದವು!

ಎಂಥೆಂಥ ಜನರನ್ನೂ ಮತ್ಸರ ಹಿಡಿದುಕೊಳ್ಳುತ್ತದೆ. ಎಂಥ ಅನಾಹುತ ಉಂಟುಮಾಡುತ್ತದೆ ಎಂಬುದನ್ನು ಮೇಲಿನ ಕತೆ ಹೇಳುತ್ತದೆ. ತಪಸ್ಸನ್ನಾಚರಿಸಿ ಭಗವಂತನ ಸಾಕ್ಷಾತ್ಕಾರ ಮಾಡಿಕೊಂಡ ಮೇಲೆ, ಅರಿಷಡ್ವರ್ಗಗಳಿಗೆ ಆಶ್ರಯವಾದ ಅಹಂಕಾರವೇ ಉಳಿಯದು. ಇಲ್ಲಿ ಹೇಳಹೊರಟದ್ದು ತಪಸ್ಸು–ತಪಸ್ವಿಗಳ ಮಹಿಮೆಯನ್ನಲ್ಲ. ಮತ್ಸರದ ವ್ಯಾಪಕಜಾಲ ಎಷ್ಟು ಪ್ರಭಾವಶಾಲಿ ಎಂಬುದನ್ನು!

‘ಗುಣವಂತರನ್ನು ದ್ವೇಷಿಸುವುದು ಮಹಾ ಪಾಪಕೃತ್ಯ’ ಎನ್ನುತ್ತದೆ ಮಹಾಭಾರತ. ‘ಸಜ್ಜನರನ್ನು ಗೌರವಿಸದಿದ್ದರೆ, ದೇವರು ಮುನಿಯುತ್ತಾನೆ’ ಎಂದರು ಭಗವಾನ್ ಶ‍್ರೀರಾಮಕೃಷ್ಣರು. ‘ಮಾತ್ಸರ್ಯ, ಸುಖದ ಮಹಾಶತ್ರು’ ಎಂದರು ಬರ್ಟ್ರಾಂಡ್ ರಸ್ಸೆಲ್.

ಚಿಕ್ಕಮಕ್ಕಳಲ್ಲಿಯೂ ಈ ಮಾತ್ಸರ್ಯ ಕಂಡುಬರುತ್ತದೆ. ತಾಯಿ ಒಬ್ಬನಿಗೆ ಸ್ವಲ್ಪ ಪಕ್ಷಪಾತ ಮಾಡಿದರೆ ಇನ್ನೊಬ್ಬ ಅದನ್ನು ತಾಳಲಾರ. ತಂದೆತಾಯಿಗಳು ಸಾಧ್ಯವಿದ್ದಷ್ಟು ಊಟ, ತಿಂಡಿ, ಆರೈಕೆ, ಲಕ್ಷ್ಯ–ಇವುಗಳಲ್ಲಿ ನಿಷ್ಪಕ್ಷಪಾತ ದೃಷ್ಟಿಯಿಂದ ನಡೆದುಕೊಂಡರೆ ಮಕ್ಕಳು ಸಂತೋಷ\-ದಿಂದಿರಬಲ್ಲರು. ದೊಡ್ಡವರು ಮಕ್ಕಳಂತೆ ಮತ್ಸರವನ್ನು ಕೂಡಲೇ ಹೊರಗಡೆ ಪ್ರಕಟಿಸುವುದಿಲ್ಲ. ಆದರೆ ಅವರ ನಡೆನುಡಿಗಳನ್ನು ಸ್ವಲ್ಪ ಪರಿಶೀಲಿಸಿದರೆ, ಅವರ ಉದ್ದೇಶ, ಸಂಕಲ್ಪ ತಿಳಿಯಲು ಕಷ್ಟವಾಗದು. ಇತರರ ಸತ್ಕಾರ್ಯಗಳಿಗೆ ಇವರು ದುರುದ್ದೇಶದ ಚೀಟಿಯನ್ನು ಅಂಟಿಸುತ್ತಾರೆ. ಮನುಷ್ಯರ ಒಳ್ಳೆಯ ಗುಣಗಳನ್ನು ಮತ್ಸರಗ್ರಸ್ತ ವ್ಯಕ್ತಿಗಳು ಖಂಡಿತವಾಗಿಯೂ ಮೆಚ್ಚಲಾರರು. ಮಾತನಾಡುವಾಗ ಬಹಳ ಕಾಳಜಿ ಇರುವಂತೆ, ನ್ಯಾಯಪ್ರಿಯರಂತೆ ತೋರಿಸಿಕೊಂಡು,\break ವಿಮರ್ಶಾತ್ಮಕ ದೃಷ್ಟಿಯಿಂದ ಮಾತನಾಡುವ ನಾಟಕ ಮಾಡುತ್ತಾರೆ–ಷೇಕ್ಸ್​ಪಿಯರನ ‘ಒಥೆಲೋ’ ನಾಟಕದಲ್ಲಿ ಬರುವ ಇಯಾಗೋವಿನಂತೆ.

ಅಧ್ಯಾಪಕರೊಬ್ಬರು ತಮ್ಮ ಒಳ್ಳೆಯ ನಡತೆಯಿಂದ ಮತ್ತು ಅತ್ಯುತ್ತಮ ಬೋಧನೆಯಿಂದ, ಮಕ್ಕಳ ಪ್ರೀತ್ಯಾದರಗಳನ್ನು ವಿಶೇಷವಾಗಿ ಸಂಪಾದಿಸಿದರು. ಮುಖ್ಯ ಅಧ್ಯಾಪಕರನ್ನು ತಂದೆಯಂತೆ ವಿಶೇಷ ಗೌರವಾದರಗಳಿಂದ ನೋಡಿಕೊಂಡರು. ಪ್ರತಿಸ್ಪರ್ಧಿ ಮನೋಭಾವವಿಲ್ಲದ, ವಿನಯಸಂಪನ್ನರಾದ, ಈ ಅಧ್ಯಾಪಕರನ್ನು ಕಂಡರೆ ಮುಖ್ಯ ಅಧ್ಯಾಪಕರಿಗೆ ಇರಿಟೇಶನ್\break (ಅಸಹನೆ). ನಾಲ್ಕಾರು ವರ್ಷಗಳಲ್ಲಿ ಒಮ್ಮೆಯೂ ಮುಖ್ಯ ಅಧ್ಯಾಪಕರು ಈ ಸಹಾಯಕ ಅಧ್ಯಾಪಕರೊಂದಿಗೆ ಮನಬಿಚ್ಚಿ ಮಾತನಾಡಿದವರಲ್ಲ; ಗೌರವ ವಿಶ್ವಾಸಗಳಿಂದ ಕಂಡವರಲ್ಲ. ಏನೋ ಗಂಟುತಿಂದವರಂತೆ ಮುಖ ಸಿಂಡರಿಸಿಕೊಂಡು ತಿರುಗುತ್ತಿದ್ದರು. ಸಹಾಯಕ ಅಧ್ಯಾಪಕರ ಜನ\-ಪ್ರಿಯತೆ ಹೆಚ್ಚಿದಂತೆಲ್ಲ ಇವರ ಮತ್ಸರದ ಪ್ರಮಾಣವೂ ಅಧಿಕವಾಯಿತು. ಕೊನೆಗೆ ಏನೋ ಕಿತಾಪತಿ\-ಮಾಡಿ, ಅವರನ್ನು ವರ್ಗಾವಣೆ ಮಾಡಿಸಿ ಮನಸ್ಸಿಗೆ ಕೊಂಚ ಸಮಾಧಾನ ತಂದುಕೊಂಡರು! ಸಂಸ್ಥೆಯ ಹಿತದೃಷ್ಟಿಯಿಂದ ದುಡಿದು ಎಲ್ಲರಿಗೂ ಒಳಿತನ್ನು ಮಾಡುತ್ತಿದ್ದಾರೆ, ತಮಗೂ ಒಳ್ಳೆಯ ಹೆಸರನ್ನು ತರುತ್ತಿದ್ದಾರೆ ಎಂಬುದನ್ನೂ ಮರೆತರು!

ತನ್ನ ಗಂಡನ ಅಣ್ಣನ ಮಕ್ಕಳಿಗೆ, ಮೆಡಿಕಲ್ ಕಾಲೇಜಿನಲ್ಲಿ ಸೀಟು ಸಿಕ್ಕಿದ ವಿಚಾರ ಕೇಳಿ, ತಂಗಮ್ಮನಿಗೆ ಹೊಟ್ಟೆಯಲ್ಲಿ ಸಂಕಟವಾಯಿತು. ಅವರಿವರಿಗೆ ಕೇಳಿಸುವಂತೆ ಹೇಳುತ್ತಾಳೆ: ‘ಅಯ್ಯೋ, ಈಗಿನ ಕಾಲದಲ್ಲಿ ಬೀದಿಗೊಬ್ಬ ಡಾಕ್ಟರ್ ಅಂತೆ. ಆ ವೃತ್ತಿಯಲ್ಲಿ ಏನೂ ಗಿಟ್ಟುವುದಿಲ್ಲವಂತೆ. ಆ ಉದ್ಯೋಗ ಯಾರಿಗೆ ಬೇಕಪ್ಪಾ. ನನ್ನ ಮಗನಿಗಂತೂ ಬೇಡ.’ ಯಥಾರ್ಥವಾಗಿ ತಂಗಮ್ಮನ ಮಗನಿಗೂ ಮೆಡಿಕಲ್ ಕಾಲೇಜಿನಲ್ಲಿ ಸೀಟಿಗಾಗಿ ಯತ್ನಿಸಿದ್ದರೂ ಅವನಿಗೆ ಕಡಿಮೆ ಅಂಕಗಳು ಬಂದು ಸೀಟು ಸಿಕ್ಕಿರಲಿಲ್ಲ, ಅಷ್ಟೆ.\footnote{\engfoot{We see things not as they are but as we are.}}

ಹೆಂಗಸರಲ್ಲಿ ಮತ್ಸರ ಹೆಚ್ಚು ಎಂದು ಹೇಳುವುದುಂಟು. ಹೆಂಗಸರು ಎಲ್ಲ ಹೆಂಗಸರನ್ನೂ ತಮ್ಮ ಪ್ರತಿಸ್ಪರ್ಧಿಗಳೆಂದು ತಿಳಿಯುತ್ತಾರಂತೆ. ಆದರೆ ಗಂಡಸರು ತಮ್ಮ ಕ್ಷೇತ್ರದಲ್ಲಿ ದುಡಿಯುವವರನ್ನು ಮಾತ್ರ ಸ್ಪರ್ಧಿಗಳೆಂದು ತಿಳಿದು ಮತ್ಸರ ಪಡುತ್ತಾರೆ. ‘ಪಂಡಿತಃ ಪಂಡಿತಂ ದೃಷ್ಟ್ವಾ ಶ್ವಾನವತ್\break ಗುರ್​ಗುರಾಯತೇ’ ಎಂಬ ಮಾತಿದೆ. ಒಬ್ಬ ಸಂಗೀತಗಾರರನ್ನು ಇನ್ನೊಬ್ಬರೆದುರು ಹೊಗಳಿದರೂ ಮಾತ್ಸರ್ಯದ ಧ್ವನಿಯೇ ಹೊರಡುತ್ತದೆ. ಮಾತ್ಸರ್ಯದ ಹೆಡೆಯಡಿಯಲ್ಲಿ ಎಲ್ಲ ಬಗೆಯ ಜನರೂ ಇದ್ದಾರೆ!

\newpage

ಮತ್ಸರಿಗಳು ಸುಖಿಗಳಲ್ಲ. ಅವರು ಉರಿಯುತ್ತಿರುತ್ತಾರೆ. ತಮ್ಮ ಸುಖ, ಭೋಗ, ಭಾಗ್ಯಗಳಿಂದ ಸಂತುಷ್ಟರಾಗುವುದರ ಬದಲು, ಪರರ ಅಭ್ಯುದಯವನ್ನು ಕಂಡು ಕರುಬುತ್ತಾರೆ. ಇತರರ ಅಶುಭವನ್ನು ತಮ್ಮ ಮನದಲ್ಲಿ ಚಿತ್ರಿಸುತ್ತ, ಅವರ ನಾಶವನ್ನು ಹಾರೈಸುತ್ತ, ತಮ್ಮ ಭವಿಷ್ಯದ ಕೆಡುಕಿಗೆ ಬೀಜಗಳನ್ನು ಬಿತ್ತುತ್ತಾರೆ. ಸಾಧ್ಯವಾದರೆ ಪ್ರತ್ಯಕ್ಷವಾಗಿ ಇತರರಿಗೆ ತೊಂದರೆ ಕೊಡುತ್ತಾರೆ.\break ದುಃಖಿಗಳೇ ಆಗುತ್ತಾರೆ.

ಮತ್ಸರದಿಂದ ಪಾರಾಗಲು, ಅದರ ದುಷ್ಪರಿಣಾಮಗಳ ಬಗ್ಗೆ ಮೊದಲು ಸರಿಯಾಗಿ ತಿಳಿದುಕೊಂಡು, ಅದರಿಂದ ತಪ್ಪಿಸಿಕೊಳ್ಳುವ ದೃಢಸಂಕಲ್ಪ ಮಾಡಬೇಕು. ಇತರರಲ್ಲಿರುವ ಸದ್ಗುಣಗಳನ್ನು ಕಂಡು ಆನಂದಿಸಲು ಕಲಿಯಬೇಕು. ಅವರನ್ನು ಹೃತ್ಪೂರ್ವಕ ಪ್ರಶಂಸಿಸಬೇಕು. ಒಬ್ಬ ಪ್ರತಿಭಾಶಾಲಿ ಮೇಧಾವಿ, ತನ್ನ ಪ್ರಯತ್ನ ಮತ್ತು ಸಂಸ್ಕಾರಗಳ ಬಲದಿಂದ ಮೇಲೇರಿದ್ದಾನೆ. ‘ಪ್ರಯತ್ನವಿಲ್ಲದೆ ನನಗೆ ಯಶಸ್ಸು, ವಿಜಯಗಳು ಸಿಗಬೇಕೇ?’ ಎಂದು ತನ್ನನ್ನು ತಾನೇ ಕೇಳಿಕೊಳ್ಳಬೇಕು. ಶ್ರದ್ಧಾಭಕ್ತಿಯಿಂದ ನಿತ್ಯವೂ ದೇವರನ್ನು ಪ್ರಾರ್ಥಿಸಿ, ಮನಸ್ಸಿನಿಂದ ಈ ಕೊಳೆಯನ್ನು ಕಿತ್ತೆಸೆಯಬೇಕು. ದುರ್ಬಲ, ಬಲಿಷ್ಠ ಇಬ್ಬರೂ ಮತ್ಸರ ಬಿಡಬೇಕು. ಇಲ್ಲದಿದ್ದರೆ, ದುರ್ಬಲನಿಗೆ ಇಹದಲ್ಲಿ ದುರ್ಗತಿ; ಬಲಿಷ್ಠನಿಗೆ ಪರದಲ್ಲಿ ದುರ್ಗತಿ.

ಗೀನಾ ಸೆರ್ಮಿನಾರಾ ಎಂದಂತೆ ‘ತನಗಾಗದವರನ್ನೂ, ತನ್ನ ವೈರಿಯನ್ನೂ ಪ್ರೀತಿಸಲು\break ಕಲಿತಾಗಲೇ ವ್ಯಕ್ತಿ ತನ್ನೊಳಗಿನ ದ್ವೇಷದ ದಳ್ಳುರಿಯಿಂದ ಪಾರಾಗಿ ಸುಖ ಪಡೆಯುತ್ತಾನೆ.’


\section*{ದ್ವೇಷದ ವಿಷವರ್ತುಲ}

\addsectiontoTOC{ದ್ವೇಷದ ವಿಷವರ್ತುಲ}

ದ್ವೇಷ ಮಾಡುವ ವ್ಯಕ್ತಿಯ ಮನಸ್ಸು ಸದಾ ಉದ್ವಿಗ್ನವೂ, ಕಹಿಯೂ ಆಗಿರುತ್ತದೆ. ದ್ವೇಷ ಮನಸ್ಸನ್ನು ಸಂಕುಚಿತಗೊಳಿಸುವುದು, ಅರಿವಿನ ಶಕ್ತಿಯನ್ನೂ, ವಿಚಾರಶಕ್ತಿಯನ್ನೂ ತೀವ್ರವಾಗಿ ಮೊಟಕುಗೊಳಿಸುವುದು, ದ್ವೇಷ ಕಪ್ಪುಕನ್ನಡಕ ಧರಿಸಿದಂತೆ. ನಮ್ಮ ಎಲ್ಲ ಅನುಭವಗಳನ್ನೂ ಕಾರ್ಮೋಡ ಆವರಿಸಿದಂತಾಗುತ್ತದೆ ಆಗ. ದ್ವೇಷ ವೃದ್ಧಿಯಾಗುತ್ತ ಹೋಗುವುದು. ಇತರರಲ್ಲಿ ನೀವು ಯಾವುದನ್ನು ತೀವ್ರವಾಗಿ ದ್ವೇಷಿಸುತ್ತೀರೋ, ಅದು ನಿಮ್ಮ ಮನಸ್ಸಿನಲ್ಲೇ ಬೀಡು ಬಿಡುವುದು. ದ್ವೇಷವನ್ನು ಪಡೆಯುವ ವ್ಯಕ್ತಿ ವ್ಯಥಿತನಾಗಿ, ತಾನೂ ದ್ವೇಷ ಮಾಡತೊಡಗುತ್ತಾನೆ. ದ್ವೇಷ ಮಾಡುವ ವ್ಯಕ್ತಿ ಪ್ರತೀಕಾರ ಮನೋಭಾವನೆಯಿಂದ ಉರಿಯುತ್ತಿರುತ್ತಾನೆ. ದ್ವೇಷ ಮಾಡುವ ವ್ಯಕ್ತಿಯ ಆರೋಗ್ಯಕ್ಕೆ ಹಾಗೂ ಮಾನಸಿಕ ಶಾಂತಿಗೆ ವಿಪರೀತ ಹಾನಿ ಕಟ್ಟಿಟ್ಟದ್ದು.\break ದ್ವೇಷವನ್ನು ಅನುಭವಿಸುವ ವ್ಯಕ್ತಿ, ದ್ವೇಷಮಾಡುವವನನ್ನು ತಿರಸ್ಕಾರದಿಂದ ನೋಡುತ್ತಾನೆ. ದ್ವೇಷ ಮಾಡುವವನ ಮನಸ್ಸಿನಲ್ಲಿ ಒತ್ತಡ, ಆತಂಕ; ಅನುಭವಿಸುವವನಲ್ಲಿ ಅಪನಂಬಿಕೆ, ಅನಾಥ ಭಾವನೆ. ದ್ವೇಷ ಮಾಡುವವನು ಕೊನೆಗೆ ಹತಾಶನಾಗಿ ಎಲ್ಲರಿಂದಲೂ ದೂರೀಕರಿಸಲ್ಪಟ್ಟು, ಏಕಾಂತದಲ್ಲಿ ದುಃಖಭಾಗಿಯಾಗಿ ನರಳುತ್ತಾನೆ.

\newpage

ಒಮ್ಮೆ ಉತ್ತರಭಾರತದಲ್ಲಿ ಕೈಯಲ್ಲಿ ಕಾಸನ್ನಿಟ್ಟುಕೊಳ್ಳದ ಮಹಾಸಾಧುವೊಬ್ಬ ಊರಿಂದ\break ಊರಿಗೆ ಸಂಚರಿಸುತ್ತಿದ್ದ. ಅವನು ಹೋದಲ್ಲೆಲ್ಲಾ ಜನರು ಗೌರವಾದರಗಳನ್ನು ತೋರಿದರು. ಅವನ ವಚನಾಮೃತವನ್ನು ಕಿವಿಗೊಟ್ಟು ಆಲಿಸಿದರು. ಜನರು ಸಭ್ಯರೂ, ನಿರುಪದ್ರವಿಗಳೂ ಆಗಬೇಕೆಂದು ಆತ ಬೋಧಿಸಿದ. ಸಾವಿರಾರು ಜನ ಸೇರಿದ ಒಂದು ಸಭೆಯಲ್ಲಿ ಜನರನ್ನು ಉದ್ದೇಶಿಸಿ ಆತ ಈ ಕೆಳಗಿನ ಮಾತನ್ನು ಉಚ್ಚರಿಸಿದ:

‘ದ್ವೇಷಕ್ಕಿಂತ ಭಯಾನಕವಾದ ಹತ್ಯೆ ಬೇರೊಂದಿಲ್ಲ.’

‘ದ್ವೇಷ ಅತ್ಯಂತ ಭೀಕರವಾದ ಜ್ವರ.’

‘ದ್ವೇಷದಿಂದ ದ್ವೇಷ ಎಂದಿಗೂ ಅಳಿಯದು. ಪ್ರೀತಿಯಿಂದ ದ್ವೇಷವನ್ನು ಗೆಲ್ಲಿರಿ.’

ಆ ಸಾಧು ಬೇರಾರೂ ಅಲ್ಲ–ಗೌತಮ ಬುದ್ಧ. ಎರಡು ಸಾವಿರದ ಐನೂರು ವರ್ಷಗಳ ಹಿಂದೆ ಗೌತಮ ಬುದ್ಧ ಜನರಿಗೆ ನೀಡಿದ ಈ ಸಂದೇಶ ಎಲ್ಲ ಯುಗದ ದುಃಖ ಕ್ಲೇಶಗಳನ್ನು ದೂರ ಮಾಡುವ ದಿವ್ಯೌಷಧಿಯಾಗಬಲ್ಲುದು.

ನೀವು ಯಾರನ್ನಾದರೂ ದ್ವೇಷಿಸುತ್ತೀರಾ? ಅದು ನಿಮ್ಮ ಅತಿ ದೊಡ್ಡ ದೌರ್ಬಲ್ಯ–ಅತಿ ದೊಡ್ಡ ದೋಷ.

ನೀವು ನಿಮ್ಮ ಶತ್ರುಗಳನ್ನೋ, ಇತರರನ್ನೋ ದ್ವೇಷಿಸತೊಡಗಿದಾಗ ನಿಮ್ಮ ಮೇಲೆ ಅಧಿಕಾರ ಚಲಾಯಿಸುವ ಪೂರ್ಣ ಹಕ್ಕನ್ನು ಅವರಿಗೆ ಬಿಟ್ಟುಕೊಡುತ್ತೀರಿ. ನಿಮ್ಮ ನಿದ್ರೆ, ಹಸಿವೆ, ರಕ್ತದ ಒತ್ತಡ, ಆರೋಗ್ಯ, ಆನಂದಗಳ ಮೇಲೆ ಅವರು ಅಧಿಕಾರ ಚಲಾಯಿಸತೊಡಗುತ್ತಾರೆ. ತಮ್ಮ ಯೋಚನೆಯಿಂದ ನೀವೆಷ್ಟು ಕಂಗಾಲಾಗಿದ್ದೀರೆಂಬುದು ನಿಮ್ಮ ಶತ್ರುಗಳಿಗೆ ತಿಳಿದರೆ ಸಂತೋಷದಿಂದ ಕುಣಿದಾಡಿಯಾರು! ನಿಮ್ಮ ದ್ವೇಷಭಾವನೆಗಳು ನಿಮ್ಮ ಶತ್ರುಗಳನ್ನು ನೋಯಿಸುತ್ತಿಲ್ಲ. ಬದಲಾಗಿ ನಿಮ್ಮನ್ನು ದುಃಖ, ದುಮ್ಮಾನಗಳ ನರಕಕ್ಕೆ ಕೊಂಡೊಯ್ಯುತ್ತವೆ.

‘ಪ್ರತಿಯೊಂದು ದ್ವೇಷಭಾವನೆಯೂ ಸುಪ್ತಾವಸ್ಥೆಯಲ್ಲಿರುವುದು. ಒಂದಲ್ಲ ಒಂದು ದಿನ ದುಃಖದ ರೂಪಿನಲ್ಲಿ ರಭಸದಿಂದ ನಮ್ಮ ಮೇಲೆ ಬೀಳುವುದು. ಅದನ್ನು ತಪ್ಪಿಸುವುದಕ್ಕೆ ಆಗುವುದಿಲ್ಲ. ದ್ವೇಷ, ಅಸೂಯಾ ಭಾವನೆಗಳನ್ನು ನೀವು ಪಡೆದಿದ್ದರೆ, ಅವು ಚಕ್ರಬಡ್ಡಿ ಸಹಿತ ನಿಮ್ಮ ಮೇಲೆ ಎರಗುವುವು. ಒಂದು ಸಲ ನೀವು ಅದಕ್ಕೆ ಬಲಿಬಿದ್ದರೆ ಅದರ ಪರಿಣಾಮವನ್ನೂ ಸಹಿಸ ಬೇಕಾಗುತ್ತದೆ’ ಎನ್ನುತ್ತಾರೆ ಸ್ವಾಮಿ ವಿವೇಕಾನಂದರು.


\section*{ಸೇಡಿನ ಸಂಚು}

\addsectiontoTOC{ಸೇಡಿನ ಸಂಚು}

ಒಬ್ಬಾಕೆಯನ್ನು ಹುಚ್ಚುನಾಯಿಯೊಂದು ಕಡಿಯಿತು. ಸಕಾಲದಲ್ಲಿ ಅವಳಿಗೆ ಔಷಧೋಪಚಾರ ದೊರೆಯದಿದ್ದ ಕಾರಣ ಅವಳಲ್ಲೂ ಹುಚ್ಚುನಾಯಿ ಕಚ್ಚಿದಾಗ ಬರುವ ಹೈಡ್ರೋಫೋಬಿಯಾ ಲಕ್ಷಣ ಕಾಣಿಸಿಕೊಂಡಿತು. ಅವಳ ಸಂಬಂಧಿಗಳು ಎಚ್ಚರಗೊಂಡು ಆಕೆಯನ್ನು ಆದಷ್ಟು ಬೇಗನೆ ಆಸ್ಪತ್ರೆಗೆ ಸೇರಿಸಿದರು. ತಜ್ಞ ಡಾಕ್ಟರೊಬ್ಬರು ಆಕೆಯ ಔಷಧೋಪಚಾರದ ವ್ಯವಸ್ಥೆಯನ್ನೂ, ಆರೈಕೆಯನ್ನೂ ನೋಡಿಕೊಂಡರು. ಅವಳ ಮನಸ್ಸು ಸಮಾಧಾನದಲ್ಲಿರುವಾಗ ಡಾಕ್ಟರ್ ಅವಳಿಗೆ ಹೀಗೆಂದರು: ‘ಇದೋ ನೋಡು ಕಾಗದ, ಪೆನ್ ಇಲ್ಲಿದೆ. ನಿನ್ನ ಇಚ್ಛಾಪತ್ರ ಅಥವಾ ಉಯಿಲನ್ನು ಬರೆ.’ ಆಕೆ ಬರೆಯಲು ಪ್ರಾರಂಭಿಸಿದವಳು ನಿಲ್ಲಿಸಲೇ ಇಲ್ಲ. ಡಾಕ್ಟರ್ ಅವಳ ಚಿತ್ತ ಸ್ವಾಧೀನವಿಲ್ಲವೆಂಬುದನ್ನು ಅರಿತುಕೊಂಡು ಪ್ರಶ್ನಿಸಿದರು–‘ಏನು ಮಾಡುತ್ತಿದ್ದೀಯಮ್ಮಾ?’ ‘ನಾನು ಕಚ್ಚ ಬೇಕೆಂದಿರುವವರ ಹೆಸರನ್ನು ಪಟ್ಟಿ ಮಾಡುತ್ತಿದ್ದೇನೆ’ ಎಂದಳವಳು! ಸಾವು ಸಮೀಪಿಸಿದರೂ ಕೆಲವರು ದ್ವೇಷಾಸೂಯೆಗಳ ಅಮಲಿನಲ್ಲಿ ಇತರರ ಕೆಡುಕನ್ನು ಹಾರೈಸುತ್ತಿರುತ್ತಾರೆ. ‘ಸತ್ತರೂ ನೀನು ಮಾಡಿದ ಕೃತ್ಯ ಮರೆಯಲಾರೆ. ಪ್ರೇತವಾಗಿ ಬಂದು ಕಾಡುತ್ತೇನೆ’ ಎನ್ನುವವರಿದ್ದಾರೆ! ದೇಹ ನಾಶವಾದರೂ ದ್ವೇಷ ನಾಶವಾಗದು!


\section*{ಬೂದಿ ಮುಚ್ಚಿದ ಕೆಂಡ}

\addsectiontoTOC{ಬೂದಿ ಮುಚ್ಚಿದ ಕೆಂಡ}

ಕೆಲವರು ಗುಪ್ತವಾಗಿ ದ್ವೇಷಿಸಬಲ್ಲರು. ಬಾಹ್ಯ ಆಚಾರವಿಚಾರಗಳಲ್ಲಿ ಅದನ್ನು ನೀವು ಕಂಡು\-ಹಿಡಿಯ\-ಲಾರಿರಿ. ಆದರೆ ಈ ಗುಪ್ತದ್ವೇಷವು ಅವರ ವ್ಯಕ್ತಿತ್ವವನ್ನೇ ದುರ್ಬಲಗೊಳಿಸುವುದು.

ಮೂವತ್ತನಾಲ್ಕು ವರ್ಷ ವಯಸ್ಸಿನ ಮಹಿಳೆಯೊಬ್ಬಳು ವೈದ್ಯಮಿತ್ರರಲ್ಲಿಗೆ ಬಂದಿದ್ದಳು.\break ಐವತ್ತು ವರ್ಷ ವಯಸ್ಸಿನವಳಂತೆ ಕಾಣುತ್ತಿದ್ದ ಆಕೆ ಅನೇಕ ತಿಂಗಳುಗಳಿಂದ ನಿದ್ರಾಹೀನತೆ, ನರ\-ದೌರ್ಬಲ್ಯ ಹಾಗೂ ಅತೀವ ದೈಹಿಕ ಆಯಾಸದಿಂದ ನರಳುತ್ತಿದ್ದಳು. ವೈದ್ಯರ ಸಲಹೆ ಪಡೆದಿದ್ದರೂ ಅದರಿಂದ ಪ್ರಯೋಜನವಾಗಿರಲಿಲ್ಲ. ದೇವರೊಡನೆ ಬೇಡಿ ಕಾಡಲು ಪ್ರಯತ್ನಿಸಿಯೂ ವಿಫಲ\-ಳಾಗಿದ್ದಳು. ಬೇಸತ್ತು ಕಂಗಾಲಾಗಿ ಆಕೆ ಆತ್ಮಹತ್ಯೆ ಯೋಚನೆ ಮಾಡಿದಳು. ವೈದ್ಯರು ಬಗೆಬಗೆಯಾಗಿ ಪ್ರಶ್ನಿಸಿ ಅವಳ ರೋಗದ ನಿಜವಾದ ಕಾರಣವನ್ನು ಕಂಡುಹಿಡಿದರು. ತಾನು ಮದುವೆ\-ಯಾಗಲಿಚ್ಛಿಸಿದ್ದ ವ್ಯಕ್ತಿಯನ್ನು ಮದುವೆಯಾದ ತನ್ನ ಸಹೋದರಿಯ ಬಗ್ಗೆ, ಅನೇಕ ವರ್ಷಗಳಿಂದ ಅವಳು ಅಸಮಾಧಾನದ ಮಡುವಿನಲ್ಲಿ ತೊಳಲಾಡುತ್ತಿದ್ದಳು. ತೋರಿಕೆಗೆ, ವ್ಯಾವಹಾರಿಕವಾಗಿ, ತನ್ನ ಸಹೋದರಿಯ ಬಗ್ಗೆ ಪ್ರೀತಿಯಿಂದ ವರ್ತಿಸಿದ್ದರೂ, ಸುಪ್ತಮನಸ್ಸಿನಲ್ಲಿ ಅವಳಡೆಗೆ ಭಯಂಕರವಾದ ದ್ವೇಷಭಾವನೆಯನ್ನು ಬೆಳೆಸಿಕೊಂಡಿದ್ದಳು. ಅದೇ ಅವಳ ಮಾನಸಿಕ ಹಾಗೂ ದೈಹಿಕ ಆರೋಗ್ಯ ನಾಶಕ್ಕೂ ಕಾರಣವಾಗಿತ್ತು.

ಇದೇ ವೇಳೆಗೆ ಸಾಧುವೊಬ್ಬರು ಆಕೆಗೆ ಸಾಂತ್ವನ ನೀಡುತ್ತಾ ‘ನೋಡು, ದ್ವೇಷಭಾವನೆ ತುಂಬಾ ಕೆಟ್ಟದ್ದು. ಹೃದಯದಲ್ಲಿ ಬೇರೂರಿರುವ ನಿನ್ನ ಸಹೋದರಿಯ ಬಗೆಗಿನ ಅಸಮಾಧಾನವನ್ನು ಹೋಗಲಾಡಿಸಲು ಸಹಾಯಮಾಡಬೇಕೆಂದು ದೀನಳಾಗಿ ನೀನು ದೇವರಲ್ಲಿ ಪ್ರಾರ್ಥಿಸಬೇಕು. ಆಗ ದೇವರು ನಿನ್ನ ಮನಸ್ಸಿಗೆ ಶಾಂತಿಯನ್ನು ಕೊಡುತ್ತಾನೆ’ ಎಂದರು. ಅವಳು ಈ ಸಲಹೆಯಂತೆಯೆ ನಡೆದುಕೊಂಡಳು. ಪ್ರಾರ್ಥನೆಯ ಮೂಲಕ ಹಾಗೂ ತನಗಿಂತಲೂ ಮಿಗಿಲಾದ ಅಪೂರ್ವ ಶಕ್ತಿಯಲ್ಲಿಟ್ಟ ವಿಶ್ವಾಸದಿಂದ ತನ್ನ ಸಹೋದರಿಯ ಬಗ್ಗೆ ಇದ್ದ ಅಸಮಾಧಾನವನ್ನು ನೀಗಲು ಆಕೆಗೆ ಸಾಧ್ಯವಾಯಿತು. ಅವಳ ನಿದ್ರಾನಾಶ, ಖಿನ್ನತೆ ಹಾಗೂ ಚಿಂತೆ ದೂರವಾದವು. ಹಿಂದೆಂದಿಗಿಂತಲೂ ಉತ್ಸಾಹ ಹುರುಪುಳ್ಳ, ಸಂತೋಷ ಚಿತ್ತವುಳ್ಳ ಹೊಸ ವ್ಯಕ್ತಿಯೇ ಆದಳು ಅವಳು.


\section*{ಪ್ರೀತಿಯಿಂದ ಗೆಲ್ಲಿ}

\addsectiontoTOC{ಪ್ರೀತಿಯಿಂದ ಗೆಲ್ಲಿ}

ಜನರ ದುರ್ನಡತೆ, ದುರಹಂಕಾರ, ದುಷ್ಟಹಂಚಿಕೆಗಳು ನಮ್ಮ ಮನನೋಯಿಸದೇ ಇರಲು ಸಾಧ್ಯವಿಲ್ಲ. ಸುಳ್ಳು ಆಪಾದನೆ, ನಿಂದೆ, ಅಪಪ್ರಚಾರ ನಮ್ಮನ್ನು ರೊಚ್ಚಿಗೇಳುವಂತೆ ಮಾಡುತ್ತವೆ. ಸ್ಪರ್ಧೆಯೂ ದ್ವೇಷವನ್ನು ಹುಟ್ಟಿಸುತ್ತದೆ. ನಿಮ್ಮ ಸದುದ್ದೇಶಗಳನ್ನು ತಿಳಿದುಕೊಳ್ಳದ ನಿಮ್ಮ ಆಪ್ತರೆಂದು ನಟಿಸುವವರು, ಹಿಂದಿನಿಂದ ನಿಮ್ಮ ಬಗ್ಗೆ ಸುಳ್ಳು ಪ್ರಚಾರ ಮಾಡಿದಾಗ, ಮನಸ್ಸು ಸಂಕಟಗ್ರಸ್ತವಾಗಿ ಸೇಡಿನ ಮನೋಭಾವ ಸಿಡಿದೇಳುತ್ತದೆ. ಇಂಥ ಸಂದರ್ಭದಲ್ಲಿ, ತೀವ್ರ ತೆರನಾದ ಪ್ರತೀಕಾರ ಮಾಡುವ ಹಂಬಲ, ಸಂಕಟ ಉಂಟಾಗುವುದು ಸಹಜ. ಎಂದರೆ, ನಮ್ಮಲ್ಲಿ ದ್ವೇಷಭಾವ ಮೂಡುವುದು ಸ್ವಾಭಾವಿಕ. ಆದರೆ ಈ ದ್ವೇಷವನ್ನು ಪ್ರೇಮದಿಂದಲೇ ಗೆಲ್ಲ ಬೇಕೆಂಬುದು ಮಹಾತ್ಮರ ಉಪದೇಶ. ನಿತ್ಯವೂ ಧ್ಯಾನಕಾಲದಲ್ಲಿ ಅಥವಾ ರಾತ್ರಿ ಮಲಗುವ ಹೊತ್ತಿನಲ್ಲಿ ನೀವು ಯಾರನ್ನು ದ್ವೇಷಿಸುತ್ತೀರೋ, ಅವರೆಡೆಗೆ ಪ್ರೀತಿಯ ಭಾವನೆಯನ್ನು ಹರಿಯಗೊಡಬೇಕು. ಧರ್ಮ, ದೈವಭಕ್ತಿ–ನಮ್ಮ ಹೃದಯ ಮತ್ತು ಸಾಂಸಾರಿಕ ಸಂಬಂಧಗಳನ್ನು ಪರಿಶುದ್ಧಗೊಳಿಸುವು\-ದಕ್ಕಾಗಿಯೇ ಹೊರತು ಇನ್ನಷ್ಟು ಕಲುಷಿತಗೊಳಿಸಲಿಕ್ಕಲ್ಲ. ದ್ವೇಷಾಸೂಯೆಯ ಜಾಲಗಳಿಂದ ಮೇಲೆತ್ತುವಂತೆ ಪ್ರಾರ್ಥಿಸಿದರೆ–ತೀವ್ರ ನಿಷ್ಠೆಯಿಂದ ಎಡೆಬಿಡದೆ ಪ್ರಾರ್ಥಿಸಿದರೆ, ಉತ್ತಮ\break ಪರಿಣಾಮ ದೊರೆತೇ ದೊರೆಯುವುದು.

ಮನುಷ್ಯ ಸಂಬಂಧಗಳ ಸಂಕೀರ್ಣತೆಯನ್ನು ಸೂಕ್ಷ್ಮವೂ, ವ್ಯಾಪಕವೂ ಆದ ನಿಯಮದ ದೃಷ್ಟಿಯಿಂದ ಅಧ್ಯಯನ ಮಾಡಿದಾಗ, ದ್ವೇಷಾಸೂಯೆಗಳನ್ನು ಎದುರಿಸುವಲ್ಲಿ ಪ್ರತೀಕಾರ ಮನೋ\-ಭಾವನೆ\-ಗಿಂತಲೂ, ಸಹನೆಯ ವರ್ತನೆ ಹೆಚ್ಚು ಲಾಭದಾಯಕವೆಂಬ ಅರಿವು ಮೂಡುವುದು.

ಮನುಷ್ಯನು ಆತ್ಮ; ಅವನಿಗೊಂದು ದೇಹವಿದೆ–ಎನ್ನುವ ಭಾವನೆ ಋಷಿಗಳು ಕಂಡ ಅನುಭವವನ್ನು ಆಧರಿಸಿದೆ. ಜಾತಿ, ವಂಶ, ಮೇಲು–ಕೀಳು ಭಾವನೆಗಳ ಕುರಿತಾದ ರಾಗದ್ವೇಷಗಳನ್ನು ದಾಟಲು ಜೀವಾತ್ಮನ ನಿಜವಾದ ಸ್ವಭಾವ ಏನೆಂದು ತಿಳಿದುಕೊಳ್ಳಬೇಕು. ಜೀವಾತ್ಮನೊಬ್ಬ ಬೇರೆ ಬೇರೆ ಕಾಲಗಳಲ್ಲಿ, ಬೇರೆ ಬೇರೆ ವಂಶ, ಜಾತಿ, ಮತಗಳಿಗೆ ಸೇರಿದ ದೇಹಗಳಲ್ಲಿದ್ದನೆಂದಾದರೆ, ಈಗ ಸದ್ಯ ಯಾರನ್ನು ದ್ವೇಷಮಾಡುವುದು? ಎಲ್ಲ ಜಾತಿಗಳಲ್ಲೂ ಒಮ್ಮೆ ಅವನು ಅವತರಿಸಿದ್ದ ನೆಂದಾದರೆ ಯಾವುದೇ ಜಾತಿಯನ್ನು ಆತ ಮನಸಾ ದ್ವೇಷಿಸಬಲ್ಲನೆ? ದೇಹವೆನ್ನುವುದು ಆಗಾಗ ಬದಲಾಯಿಸಬಹುದಾದ ಆತ್ಮನ ಒಂದು ವಾಹನ ಎಂದಾದರೆ, ಬೇರೆ ಬೇರೆ ಜಾತಿಗಳಿಗೆ ಸೇರಿದ ವ್ಯಕ್ತಿಗಳನ್ನು ದ್ವೇಷಿಸುವುದು, ನಟನೊಬ್ಬನನ್ನು ಬೇರೆ ಬೇರೆ ವೇಷಭೂಷಣಗಳನ್ನು ಧರಿಸಿದ್ದಕ್ಕಾಗಿ ದ್ವೇಷಿಸಿದಂತೆಯೇ ಸರಿ!

ನಮ್ಮ ಅರಿವಿನ ಪರಿಧಿ ವಿಶಾಲವಾದಂತೆಲ್ಲ, ಅಳವಾದಂತೆಲ್ಲ ನಮ್ಮ ದೃಷ್ಟಿಕೋನ ವ್ಯತ್ಯಾಸವಾಗುವುದು. ಅದರಿಂದಾಗಿ ಬದುಕೇ ಹಸನಾಗುವುದು.


\section*{ಅಸಾಧ್ಯವೇನಲ್ಲ!}

\vskip -7pt\addsectiontoTOC{ಅಸಾಧ್ಯವೇ\-ನಲ್ಲ!}

ನಮ್ಮ ನಿಮ್ಮ ಮನೆಯನ್ನು ನಾಶಮಾಡಿದ, ಆರ್ಥಿಕ ಅವನತಿಗೆ ಕಾರಣರಾದ, ನಮ್ಮ ಆಪ್ತೇಷ್ಟರಿಗೆ ಕೆಡಕನ್ನುಂಟುಮಾಡಿದ, ಸ್ತ್ರೀಯರಿಗೆ ಕೆಡುಕನ್ನು ಮಾಡಿದಂಥ ವ್ಯಕ್ತಿಗಳನ್ನು ದ್ವೇಷಿಸದೆ ಇರಲು ಮನುಷ್ಯನಿಗೆ ಸಾಧ್ಯವೇ? ಎಂಬ ಪ್ರಶ್ನೆ ಸಹಜ.

ಒಬ್ಬಾತ ಕ್ರೂರಿ, ಸ್ವೈರಾಚಾರಿ, ಹಿಂಸಾತುರ, ಪಶುಸ್ವಭಾವಿ ಎಂದು ಆತನನ್ನು ದ್ವೇಷಿಸುವುದು, ಒಂದು ದೃಷ್ಟಿಯಲ್ಲಿ, ಬೀಜಗಣಿತ ತಿಳಿಯದ, ಮೂರನೇ ವರ್ಗದಲ್ಲಿ ಪಾಸಾಗದ ಹುಡುಗನೊಬ್ಬನನ್ನು, ಹೈಸ್ಕೂಲ್ ವಿದ್ಯಾರ್ಥಿ ದ್ವೇಷಿಸಿದಂತೆಯೇ ಸರಿ. ಕಲ್ಲೊಂದನ್ನು ಅದು ಕಲ್ಲಾಗಿರುವುದರಿಂದ ನಾವು ದ್ವೇಷಿಸುತ್ತೇವೇನು? ಕಲ್ಲು ಅಕಸ್ಮಾತ್ ಕಾಲಿಗೆ ತಾಗಿದಾಗ, ತಾವು ಜಾಗರೂಕರಾಗುತ್ತೇವೆಯೆ ಹೊರತು ಕಲ್ಲನ್ನು ದ್ವೇಷಿಸುತ್ತೇವೆಯೆ? ಪ್ರಕೃತಿಯ ಅಸಂಖ್ಯ ವಸ್ತುಗಳಲ್ಲಿ ಅದನ್ನೂ ಒಂದು ವಸ್ತುವಾಗಿ ಸ್ವೀಕರಿಸುತ್ತೇವಷ್ಟೆ! ಅಂತೆಯೇ, ಅಲ್ಪರೂ, ಅಜ್ಞರೂ, ದುಷ್ಟರೂ, ಪಶು ಸ್ವಭಾವಿಗಳೂ ಆದ ಜನರಿಂದ ಅನ್ಯಾಯವನ್ನು ಸಹಿಸುವಾಗ, ಇನ್ನೂ ವಿಕಾಸದ ಸಾಮಾನ್ಯ ಹಂತದಲ್ಲಿದ್ದುಕೊಂಡಿರುವ ಅವರನ್ನು ದ್ವೇಷಿಸುವುದು, ಕಲ್ಲನ್ನು ದ್ವೇಷಿಸಿದಂತೆಯೆ! ಗೌತಮ ಬುದ್ಧನು ‘ದ್ವೇಷವನ್ನು ಪ್ರೀತಿಯಿಂದ ಗೆಲ್ಲಿರಿ’ ಎಂದುದೂ, ಯೇಸುವು ಶಿಲುಬೆಗೇರಿಸಿದ ವ್ಯಕ್ತಿಗಳನ್ನು ‘ತಂದೆಯೇ ಅವರನ್ನು ಕ್ಷಮಿಸಿ, ಅವರಿಗೆ ತಾವೇನು ಮಾಡುತ್ತೇವೆಂಬುದು ತಿಳಿಯದು’ ಎಂದುದೂ, ಪತಂಜಲಿಮಹರ್ಷಿ ಯೋಗಸೂತ್ರದಲ್ಲಿ ‘ದುರಾಚಾರೀ, ಪಾಪಪರಾಯಣ, ವ್ಯಕ್ತಿಗಳನ್ನು ಉಪೇಕ್ಷಾದೃಷ್ಟಿಯಿಂದ\break ನೋಡಿ’ ಎಂದುದೂ ಈ ಹಿನ್ನೆಲೆಯಿಂದಲೆ. ದ್ವೇಷವು ನಮ್ಮ ಮಾನಸಿಕ ಶಕ್ತಿಯನ್ನು ನಿಷೇಧಾತ್ಮಕವಾಗಿ ಪರಿಮಿತಗೊಳಿಸುವುದು. ದ್ವೇಷ ಹೇಗೆ \enginline{restrictive –}ಪರಿಮಿತಗೊಳಿಸುತ್ತದೆ ಎಂಬುದನ್ನು ತಿಳಿದುಕೊಳ್ಳಲು ನಡೆದ ಘಟನೆಯೊಂದು ಸಹಕಾರಿಯಾದೀತು:

ಪುಣೆಯ ಸನಾತನಿ ಪತ್ರಿಕೆಯಲ್ಲಿ ಬಂದ ಘಟನೆಯಿದು–ಎಂಟು ವರ್ಷ ಪ್ರಾಯದ ಬಾಲಕ\-ನೊಬ್ಬ ಪಿಶಾಚಿಯ ಬಾಧೆಯಿಂದ ಬಹಳ ಕಷ್ಟಪಡುತ್ತಿದ್ದ. ತಂದೆ ಐಶ್ವರ್ಯವಂತನಿದ್ದು ಮಗನನ್ನು ಗುಣಪಡಿಸಲು ಬಹಳ ಖರ್ಚು ಮಾಡಿದ. ಆದರೆ ಪ್ರಯೋಜನವಾಗಲಿಲ್ಲ. ಕೊನೆಗೆ ನರಸೋಬವಾಡಿಯ ಶ‍್ರೀ ದತ್ತಾತ್ರೇಯ ದೇವಸ್ಥಾನದಲ್ಲಿ ಪಿಶಾಚಿಗಳ ಬಾಧೆ ಪರಿಹಾರವಾಗುತ್ತದೆ ಎಂಬ ಆ ದೇವಸ್ಥಾನದ ಕೀರ್ತಿ ಕೇಳಿ ತಾಯಿ ತಂದೆ ಮಗನೊಡನೆ ಅಲ್ಲಿ ಹೋಗಿ ಎರಡು ತಿಂಗಳುಗಳಿದ್ದರು. ದೇವರಿಗೆ ಬೇರೆ ಬೇರೆ ತರದ ಸೇವೆಗಳನ್ನು ನಡೆಸಿದರು, ಪ್ರಾರ್ಥಿಸಿದರು. ಕೊನೆಗೆ ಒಂದು ದಿನ ಪಿಶಾಚಿ ಬಾಯಿ ಬಿಟ್ಟಿತು. ಹುಡುಗನ ಮೇಲೆ ಆವೇಶ ಬಂದು ಹೇಳಿತು. ‘ನಾನು ಸುಮ್ಮನೆ ದಾರಿಯಲ್ಲಿ ಹೋಗುತ್ತಿದ್ದಾಗ ಇವನು ಕುತ್ತಿಗೆಗೆ ಉರುಳು ಹಾಕಿ ಕೊಲೆ ಮಾಡಿದ, ನನ್ನ ಹತ್ತಿರವಿದ್ದ ಐನೂರು ರೂಪಾಯಿಗಳನ್ನೂ ಅಪಹರಿಸಿದ. ನಾನು ಪಿಶಾಚಿಯಾದೆ. ಸೇಡು ತೀರಿಸಲೇಬೇಕೆಂದು ಕಳೆದ ಏಳು ಜನ್ಮಗಳ ಕಾಲದಲ್ಲಿ ಇವನನ್ನು ಹುಡುಕುತ್ತಿದ್ದೆ. ನಂತರ ಇವನು ನನ್ನ ಕೈಗೆ ಸಿಕ್ಕಿಬಿದ್ದ. ಇವನನ್ನು ಬಿಡುವುದಿಲ್ಲ’ ಎಂದು ಉದ್ಗರಿಸಿತು. ಮಗನ ಕುರಿತಾಗಿ ಬಹಳ ದುಃಖದಲ್ಲಿದ್ದ ತಂದೆ ಅಂಗಲಾಚಿ ಬೇಡುತ್ತ ‘ನಿನ್ನ ಐನೂರು ರೂಪಾಯಿ ಹಾಗೂ ಬಡ್ಡಿ\break ಹಣವನ್ನು ಕೂಡಾ ನೀನು ಹೇಳಿದಂತೆ ವ್ಯವಸ್ಥೆ ಮಾಡುತ್ತೇನೆ. ಮಗನ ಜೀವ ಬಿಟ್ಟುಕೊಡು’ ಎಂದು ಗೋಗರೆದ. ಆ ಪಿಶಾಚಿ ‘ಇವನು ನನ್ನ ಕೊಲೆ ಮಾಡಿದ. ನಾನು ಇವನ ಜೀವ ತೆಗೆಯದೆ ಬಿಡೆನು’ ಎಂದಿತು. ಕೊನೆಗೂ ಪಿಶಾಚಿ ಹುಡುಗನನ್ನು ಬಿಟ್ಟು ತೊಲಗಲಿಲ್ಲ. ನಂತರ ಶೀಘ್ರದಲ್ಲೇ ಆ ಹುಡುಗ ಮರಣ ಹೊಂದಿದ!

ಶಿಕ್ಷೆಯ ಹೊಣೆ ತಾನೇ ವಹಿಸಿಕೊಂಡ ಆ ಪಿಶಾಚಿಗೂ ಅಷ್ಟರವರೆಗೆ ಆ ಅಲ್ಪ ಪ್ರೇತದ ಜನ್ಮವೇ ಗತಿಯಾಯಿತು! ಕೊಲೆ ಮಾಡಿದವನು ಕರ್ಮ ನಿಯಮದಂತೆ ಖಂಡಿತ ಶಿಕ್ಷೆ ಅನುಭವಿಸಿಯೇ ತೀರುತ್ತಿದ್ದ. ಆದರೆ ಸೇಡಿನ ಮನೋಭಾವನೆ ವ್ಯಕ್ತಿಯ ವಿಕಾಸಕ್ಕೆ ಮುಳುವಾಗಿ ಅವನನ್ನು ಪರಿಮಿತ ಸ್ಥಿತಿಯಲ್ಲಿ ಏಳು ಜನ್ಮಗಳ ಕಾಲ ಇರಿಸಿತು ಎಂಬುದು ಇಲ್ಲಿ ಗಮನಾರ್ಹ.

ನೀವು ಯಾವುದನ್ನು ದ್ವೇಷಿಸುತ್ತೀರೋ ಅದರೆಡೆಗೇ ನಿಮ್ಮ ಸೆಳೆತ. ಭಯದ ಕಲ್ಪನೆಯು ನಮ್ಮನ್ನು ಭಯದ ಸನ್ನಿವೇಶಕ್ಕೆ ಎಳೆಯುವಂತೆ, ದ್ವೇಷವೂ ದ್ವೇಷಕ್ಕೆ ಪಾತ್ರನಾದವನ ಮಾನಸಿಕ ಸ್ಥಿತಿಗೆ ನಮ್ಮನ್ನು ಎಳೆಯುತ್ತದೆ.\footnote{\engfoot{As soon as a person can truly learn to love the disagreeable or restrictive person with whom he involved, he will be released from the bondage.}\hfill\engfoot{ –Gina Cerminara}}

ಪರಿಶುದ್ಧ ಪ್ರೀತಿ ಮನುಷ್ಯನಲ್ಲಿ ವ್ಯಕ್ತವಾಗಬಹುದಾದ ಅತಿಶ್ರೇಷ್ಠ ಹಾಗೂ ಉದಾತ್ತ ಅಭಿವ್ಯಕ್ತಿ. ಅದು ಕೇವಲ ದೈವೀಗುಣ ಮಾತ್ರವಲ್ಲ; ಅದು ದೇವರೇ. ನಾವು ಪರಿಶುದ್ದಪ್ರೀತಿಯನ್ನು ವ್ಯಕ್ತ\-ಗೊಳಿಸಿದಾಗ ನಮ್ಮಲ್ಲಿರುವ ದೈವೀಶಕ್ತಿಯನ್ನು ವ್ಯಕ್ತಗೊಳಿಸುತ್ತೇವೆ.

ಪರಿಶುದ್ಧ ಪ್ರೀತಿಯನ್ನು ವ್ಯಕ್ತಗೊಳಿಸುವುದು ಅಷ್ಟೇನೂ ಸುಲಭದ ಕೆಲಸವಲ್ಲ. ದೀರ್ಘಕಾಲದ ಪ್ರಯತ್ನದಿಂದ ಅದು ಸಾಧ್ಯ. ಒಂದಲ್ಲ ಒಂದು ದಿನ ಆ ಸ್ಥಿತಿಯನ್ನು ಪಡೆಯದೆ ಗತ್ಯಂತರವಿಲ್ಲ. ಆದುದರಿಂದ ಅದು ‘ಅಸಾಧ್ಯ’, ‘ಅವ್ಯಾವಹಾರಿಕ’ ಎಂದು ಕೈಬಿಡಬೇಕಿಲ್ಲ. ಅಲ್ಪ ಸ್ವಲ್ಪವಾದರೂ ಪ್ರಯತ್ನ ಮುಂದುವರಿಯಲೇಬೇಕು. ನಂತರ ಮಹತ್ ಪರಿಣಾಮ ತಿಳಿಯುವುದು.

ಕೋಲ್​ರಿಡ್ಜ್ ಹೇಳುತ್ತಾನೆ: ‘ಚೆನ್ನಾಗಿ ಪ್ರೀತಿಸಬಲ್ಲವನೇ ಚೆನ್ನಾಗಿ ಪ್ರಾರ್ಥಿಸಬಲ್ಲನು.\break ಮನುಷ್ಯ, ಪ್ರಾಣಿ, ಪಕ್ಷಿಗಳೆಲ್ಲವನ್ನೂ ಪ್ರೀತಿಸುವನವನು. ಅತ್ಯುತ್ತಮ ಪ್ರೀತಿಯನ್ನು ನೀಡುವವನೇ ಅತ್ಯುತ್ತಮವಾಗಿ ಪ್ರಾರ್ಥಿಸುವವನು. ಸಣ್ಣಪುಟ್ಟದ್ದೆಂಬ ಭೇದವಿಲ್ಲದೆ ಯಾವ ಪರಮಪ್ರಿಯ ದೇವನು ನಮ್ಮನ್ನೆಲ್ಲ ಸೃಜಿಸಿಹನೋ, ನಮ್ಮನ್ನು ಪ್ರೀತಿಸುತ್ತಿಹನೋ, ಅವನು ಎಲ್ಲರನ್ನೂ ಪ್ರೀತಿ\-ಸುವನು.’


\section*{ಸ್ಪರ್ಧೆಯಿಂದ ದ್ವೇಷ}

\addsectiontoTOC{ಸ್ಪರ್ಧೆಯಿಂದ ದ್ವೇಷ}

ಸಮಾಜಶಾಸ್ತ್ರಜ್ಞ ಪಿಟ್ರಿಮ್ ಎ. ಸೊರೊಕಿನ್ ಸ್ಪರ್ಧಾಕ್ರಮದ ಹಾನಿಯನ್ನು ಹೀಗೆಂದು ಹೇಳಿದ್ದಾರೆ:

‘ಅತಿ ಚಿಕ್ಕಮಟ್ಟದಿಂದ ಕೊರಳು ಕತ್ತರಿಸುವ ಮಟ್ಟದವರೆಗಿನ ವಿವಿಧ ರೂಪಾಂತರಗಳು ಈ ಸ್ಪರ್ಧಾ ವಿಧಾನವನ್ನು ನೆಚ್ಚಿದ್ದರಿಂದ ಉಂಟಾಗಿದೆ. ಅದು ವ್ಯಕ್ತಿ, ವ್ಯಕ್ತಿಗಳೊಳಗೆ ಮತ್ತು ಜನರ ವಿವಿಧ ಗುಂಪುಗಳೊಳಗೆ ಜಗಳ ಹಾಗೂ ಹೋರಾಟದ ಮನೋಭಾವನೆಯನ್ನು ಬೆಳೆಸುತ್ತದೆಯೇ ಹೊರತು ಪರಸ್ಪರ ಸಹಕಾರದ ಭಾವನೆಯನ್ನಲ್ಲ. ಪರಸ್ಪರ ಪ್ರೀತಿಯನ್ನಲ್ಲ.’\footnote{\engfoot{Being incarnations of the principle of competition from its mildest to its most cut-throat forms, they engender in individuals and groups, the spirit of mutual struggle rather than of mutual aids, the pathos of aggressiveness instead of love.}\hfill\engfoot{ –Pitirim A. Sorokin}}

ಸ್ಪರ್ಧೆಯೊಂದರಲ್ಲಿ ಭಾಗವಹಿಸಿದ ಅಭ್ಯರ್ಥಿ ತಾನೇ ಗೆಲ್ಲಬೇಕೆಂದು ಹಾರೈಸುತ್ತಾನೆ. ಅದಕ್ಕಾಗಿ ಎಲ್ಲ ತರದ ಉಪಾಯ, ತಂತ್ರಗಳನ್ನೂ ಮಾಡುತ್ತಾನೆ. ಆರೋಗ್ಯಕರ ಸ್ಪರ್ಧೆ ಒಳಿತೇ. ಆದರೆ ಇದು ಸೀಮಿತ ಕ್ಷೇತ್ರದಲ್ಲಿ ಸಾಧ್ಯ. ಪ್ರತಿಷ್ಠೆಯ ಒಂದು ಹುದ್ದೆಗಾಗಿಯೂ, ಚುನಾವಣೆಯ ಮೂಲಕ ಒಂದು ಅಧಿಕಾರದ ಸ್ಥಾನಕ್ಕಾಗಿಯೂ, ವ್ಯಾಪಾರ ವ್ಯವಹಾರಗಳಲ್ಲಿ ತಮ್ಮ ವಸ್ತುಗಳ ಮಾರಾಟಕ್ಕಾಗಿಯೂ ಸ್ಪರ್ಧೆ ಪುಟಿದೇಳುತ್ತದೆ. ಈ ಸ್ಪರ್ಧಾ ಮನೋಭಾವ ಕ್ರಮೇಣ ಜನರಲ್ಲಿ ಹುದುಗಿರುವ ದುಷ್ಟಸ್ವಭಾವವನ್ನು ಉದ್ರೇಕಿಸುತ್ತದೆ. ಛಲ ಬಲ ಕಪಟಗಳಿಂದ, ಲಂಚ ವಂಚನೆ ವಿಧಾನಗಳಿಂದ ತನ್ನ ಕಾರ್ಯ ಸಾಧಿಸಲು ಮನುಷ್ಯ ಯತ್ನಿಸತೊಡಗುತ್ತಾನೆ.

ಪಶ್ಚಿಮದೇಶಗಳ ಸ್ಪರ್ಧೆಯ ಈ ಕ್ರಮಕ್ಕಿಂತ ವೃತ್ತಿಯಿಂದ ಜಾತಿ ಎಂಬ ಭಾರತದ ವಿಧಾನ ಮೇಲ್ಮಟ್ಟದ್ದು ಎಂಬುದನ್ನು ತಿಳಿದುಕೊಳ್ಳಲು ನಮ್ಮ ದೇಶದ ಹಿಂದಿನ ಕಾಲದ ಹಳ್ಳಿಗಳ ಜನ ಜೀವನವನ್ನು ಅಧ್ಯಯನ ಮಾಡಬಹುದು. ಬಡಗಿ, ಚಿನಿವಾರ, ಕ್ಷೌರಿಕ, ಮಡಿವಾಳ ಮೊದಲಾದ ವೃತ್ತಿಗಳಲ್ಲಿ ಸ್ಪರ್ಧೆಯ ಮನೋಭಾವಕ್ಕೆ ಅವಕಾಶವಿರಲಿಲ್ಲ. ಅವರು ಊರಿನ ಎಲ್ಲರಿಗೂ ಬೇಕಾದವರು. ತಮ್ಮ ಮಕ್ಕಳನ್ನು ಜೀವಿಕೆಗಾಗಿ ತರಬೇತಿ ನೀಡಲು ಅವರು ಚಿಂತಿಸಬೇಕಿರಲಿಲ್ಲ, ಹೊಸ ವೃತ್ತಿಯ ಆರಂಭಕ್ಕಾಗಿ, ಅವರಿವರ ಹತ್ತಿರ ಅಲೆದು ಭಿಕ್ಷಾಪಾತ್ರೆಯನ್ನು ಹಿಡಿದು ಸಂಚರಿಸಬೇಕಾಗಿರಲಿಲ್ಲ. ತನ್ನ ವೃತ್ತಿಯ ಇನ್ನೊಬ್ಬನು ಬಂದು ಸ್ಪರ್ಧಿಸಿ, ತನ್ನನ್ನು ಸೋಲಿಸಲು ಅನ್ಯಾಯದ ಮಾರ್ಗವನ್ನು ಹಿಡಿದರೇನು ಗತಿ, ಎಂಬ ಭಯ, ಉದ್ವೇಗಗಳಿರಲಿಲ್ಲ. ಹೊಟ್ಟೆಯ ಪಾಡಿಗೆ ಅನುಕೂಲವಾದ, ಸರಳ, ಶಾಂತ ರೀತಿಯ ಜೀವನವನ್ನು ನಡೆಯಿಸಲು ಅನುಕೂಲವಾದ ಈ ಕ್ರಮ, ದ್ವೇಷಾಸೂಯೆ\-ಗಳನ್ನು ವೃದ್ಧಿಸುವ ಸ್ಪರ್ಧಾಕ್ರಮಕ್ಕಿಂತ ಮೇಲಲ್ಲವೆ? ಈ ವಿಧಾನವನ್ನು ಪುನಃ ಪ್ರತಿಷ್ಠಾಪನೆ ಮಾಡಬೇಕೆಂಬುದನ್ನು ಪ್ರತಿಪಾದಿಸಲು ಇಲ್ಲಿ ಹೊರಟಿಲ್ಲ.\footnote{ಆದರೆ ವೃತ್ತಿಯು ಜಾತಿಯಾಗಿ ಹೆಪ್ಪುಗಟ್ಟುವ ಸಂಭವ ಇಂದಿನ ನಿರುದ್ಯೋಗದ ದಿನಗಳಲ್ಲಿ ಅಸಾಧ್ಯ ವೆನ್ನಲಾರೆವು. ಅಲ್ಲಲ್ಲಿ ಉದ್ಯೋಗಕ್ಕಾಗಿ ಪರದಾಡುವುದಕ್ಕಿಂತ, ರೈಲ್ವೆ ಇಲಾಖೆಯಲ್ಲಿ ದುಡಿಯುವವರು ತಮ್ಮ ಮಕ್ಕಳಿಗೆ ರೈಲ್ವೆಯಲ್ಲಿ ಒಂದೆರಡು ಉದ್ಯೋಗ ರಿಸರ್ವ್ ಇರಿಸಲಿ, ಪಿ ಎಂಡ್ ಟಿ ಅವರು ಉದ್ಯೋಗಕ್ಕೆ ಅಲ್ಲಲ್ಲಿ ಅಲೆದಾಡುವುದಕ್ಕಿಂತ ತಾವು ರಿಟೈರ್ ಆದ ಬಳಿಕ ತಮ್ಮ ಮಕ್ಕಳಿಗೆ ಅದೇ ಇಲಾಖೆಯಲ್ಲಿ ಉದ್ಯೋಗ ದೊರೆಯಲಿ ಎಂದರೆ ಆಶ್ಚರ್ಯವಿದೆಯೆ? ಅದೇ ಮುಂದೆ ಮುಂದುವರಿದರೆ ಒಂದು ಜಾತಿಯಾಗಿ ಪರಿ ಣಮಿಸುವುದು, ದರ್ಜೆ, ಕಮ್ಮಾರ ಇದ್ದಂತೆ, ರೈಲ್ವೆ, ಪೋಸ್ಟಲ್ ಇತ್ಯಾದಿ ಜಾತಿ. ಆಗ ರೈಲ್ವೆ, ಪೋಸ್ಟಲ್ ಜಗಳಾಡಬೇಕೆ? ದ್ವೇಷ ಸಾಧಿಸಬೇಕೆ?} ಸ್ಪರ್ಧೆಯಿಂದುಂಟಾಗುವ ದ್ವೇಷಾ\-ಸೂಯೆ\-ಗಳನ್ನು ದೂರಮಾಡಲು ಅದು ಹೇಗೆ ಸಹಕಾರಿಯಾಯಿತೆಂಬುದನ್ನು ಮಾತ್ರ ಹೇಳ ಹೊರಟೆ. ಸ್ವಾತಂತ್ರ್ಯ ಲಭ್ಯವಾದ ಮೇಲೆ, ಚುನಾವಣೆ ಎನ್ನುವ ಸ್ಪರ್ಧೆ ಬರುವವರೆಗೆ, ಹಳ್ಳಿಗಳಲ್ಲಿ ‘ನಮ್ಮ ಆಚಾರ ನಮಗೆ, ನಿಮ್ಮ ಆಚಾರ ನಿಮಗೆ’ ಎಂದು ಐಕಮತ್ಯದಿಂದ ಬಾಳುತ್ತಿದ್ದುದನ್ನು ಹಳೆಯ ತಲೆಮಾರಿನವರು ಈಗಲೂ ಹೇಳುವುದುಂಟು. ಚುನಾವಣೆಯಲ್ಲಿ ಜಾತೀಯ ಭಾವನೆಗಳನ್ನು ಕೆರಳಿಸಿ, ಜಾತೀಯತೆಯ ಪ್ರೇರಣೆಯಿಂದ ಓಟುಗಳಿಸಿ, ‘ಜಾತೀಯತೆಯನ್ನು ನಾಶಮಾಡಿ’ ಎಂದು ವೇದಿಕೆಯಿಂದ ಭಾಷಣಮಾಡಿ, ದ್ವೇಷದ ಉರಿಯನ್ನು ಹೊತ್ತಿಸಿ, ಹಬ್ಬಿಸುವ ಮುಖಂಡರು ಎಲ್ಲೆಲ್ಲೂ ಕಂಡುಬರುತ್ತಿದ್ದಾರೆ. ನೀವು ಯಾವುದಾದರೊಂದು ಜಾತಿಯನ್ನು ದ್ವೇಷಿಸಲು ಆರಂಭಿಸಿ\-ದರೆ, ಅವರಲ್ಲಿ ಯಾವ ಒಳ್ಳೆಯದನ್ನೂ ಕಾಣಲಾರಿರಿ. ಈ ಹಿಂದೆ ಎಂದೂ ಕಾಣದ ದೋಷಗಳು ಅವರಲ್ಲಿ ಈಗ ಕಾಣಲು ಆರಂಭವಾಗುವುವು! ಒಟ್ಟಿನಲ್ಲಿ ವ್ಯಕ್ತಿ, ಸಮಾಜ, ಹಾಗೂ ದೇಶದ ಅಭಿವೃದ್ಧಿಗೆ ಜಾತಿಜಾತಿಗಳೊಳಗಿನ ದ್ವೇಷ ಹಾನಿಕರ. ಇದರಿಂದ ಪಾರಾಗದಿದ್ದರೆ ಏನೂ ಸಾಧಿಸಿದಂತಾಗಲಿಲ್ಲ.


\section*{ಒಡೆದರು ಸಿಡಿದರು}

\addsectiontoTOC{ಒಡೆದರು ಸಿಡಿದರು}

ಬ್ರಿಟಿಷರು ಭಾರತೀಯರ ಅಳಿದುಳಿದ ಐಕಮತ್ಯವನ್ನು ಮುರಿಯಲು ಅತ್ಯಂತ ಸೂಕ್ಷ್ಮ ಜಾಲವನ್ನು ಹಬ್ಬಿದರೆಂಬುದು ಇಂದಿನ ಭಾರತೀಯ ವಿದ್ಯಾವಂತರಿಗೂ, ಮುಖಂಡರಿಗೂ ಹೆಚ್ಚು ತಿಳಿದಂತಿಲ್ಲ. ತಾರಾಚಂದರು ‘ಭಾರತೀಯ ಸ್ವಾತಂತ್ರ್ಯದ ಚಳವಳಿಯ ಇತಿಹಾಸ’ ಎಂಬ ಗ್ರಂಥದಲ್ಲಿ ವುಡ್ ಮಹಾಶಯನು ಅಂದಿನ ವೈಸ್​ರಾಯ್​ಗೆ ಬರೆದ ಪತ್ರದ ಸಾರಾಂಶವನ್ನು ಕೊಟ್ಟಿದ್ದಾರೆ. ವುಡ್ ಹೀಗೆಂದಿದ್ದಾನೆ: ‘ಭಾರತೀಯರಲ್ಲಿರುವ ಒಮ್ಮತವನ್ನು ತಡೆಯಲು ಸಾಧ್ಯವಾದುದನ್ನೆಲ್ಲ ಮಾಡು. ನಾವು ಅಲ್ಲಿ ನಮ್ಮ ಆಡಳಿತವನ್ನು ಉಳಿಸಿಕೊಂಡದ್ದೆ ಒಂದು ಗುಂಪಿನ ಮೇಲೆ ಇನ್ನೊಂದನ್ನು ಎತ್ತಿಕಟ್ಟಿ. ಏನೇ ಆಗಲಿ, ಇದನ್ನು ಸರ್ವಥಾ ಮುಂದುವರಿಸಿಕೊಂಡು ಹೋಗಬೇಕು.’\footnote{\engfoot{We have maintained our power by playing off one part against the other and we must continue to do so. Do what you can therefore, to prevent all having a common feeling.–Mr. Wood to Elgin, Quoted from: Tarachand, \textit{History of Freedom Movement in India.}}}

ಬ್ರಿಟಿಷರು ಭಾರತ ಬಿಟ್ಟು ತೊಲಗಿದರೂ ಗುಲಾಮಗಿರಿಯ ಪ್ರತೀಕವಾದ ಪರಸ್ಪರ ದ್ವೇಷಾ\-ಸೂಯೆಯನ್ನು ನಮ್ಮ ದೇಶದಲ್ಲಿ ಉಳಿಸಿ, ಬೆಳೆಸಿಕೊಳ್ಳಲು ನಿಷ್ಠೆಯಿಂದ ಯತ್ನಿಸುತ್ತಲಿದ್ದಾರೆ. ಈ ದ್ವೇಷಾಸೂಯೆಯನ್ನು ಕಿತ್ತೆಸೆಯದೆ ನಾವು ಸತ್ಪ್ರಜೆಗಳಾಗಿ ಬದುಕಲಾರೆವು ಎಂಬ ಭಾವನೆಯನ್ನು ವಿದ್ಯಾರ್ಥಿಗಳಲ್ಲಾದರೂ ಉಂಟುಮಾಡಲು ನಮ್ಮ ಶಿಕ್ಷಣ ಪದ್ಧತಿ ಸಮರ್ಥವಾಗಿಲ್ಲ ಎಂಬುದು ಎಂಥ ವಿಪರ್ಯಾಸ! ಎಂಥ ದುರಂತ! ಸ್ವತಂತ್ರ ಭಾರತದಲ್ಲಿದ್ದೂ, ಗುಲಾಮರಾಗಿರಲು ನಾವು ಯತ್ನಿಸುತ್ತಲೇ ಇದ್ದೇವೆ!!

\newpage

ಬ್ರಿಟಿಷ್ ಆಡಳಿತ ಭಾರತದಲ್ಲಿ ಬಲವಾಗಿ ಬೇರೂರಿದ ಮೇಲೆ ಸರಕಾರಿ ಹುದ್ದೆಗಳಿಗಾಗಿ ಉಂಟಾದ ಸ್ಪರ್ಧೆಯೇ ಹಿಂದೂಗಳಲ್ಲಿ ಬ್ರಾಹ್ಮಣಜಾತಿಯ ಜನರ ಮೇಲೆ ಇತರ ಜಾತಿಯ ಜನರ ದ್ವೇಷ ಬೆಳೆಯಲು ಕಾರಣವಾಯಿತು. ಈ ವಿಚಾರದಲ್ಲಿ ಬ್ರಾಹ್ಮಣಜಾತಿಯ ಜನ ಇತರರಿಗೆ ಅನ್ಯಾಯ ಮಾಡಿ ದ್ರೋಹ ಮಾಡಿ ಈ ದ್ವೇಷ ಉಂಟಾದುದಲ್ಲ. ದೇಶದ ಬೇರೆ ಬೇರೆ ಪ್ರಾಂತಗಳಲ್ಲಿ ಈ ದ್ವೇಷವಿದೆ ಎನ್ನಲೂ ಸಾಧ್ಯವಿಲ್ಲ. ಬ್ರಾಹ್ಮಣೇತರರೇ ಹೆಚ್ಚು ಸಂಖ್ಯೆಯಲ್ಲಿ ಸರಕಾರಿ ಉದ್ಯೋಗಗಳಲ್ಲಿರುವ ಕಡೆ ಈ ದ್ವೇಷ ಅಷ್ಟೊಂದು ಕಂಡು ಬಂದಿಲ್ಲ. ಉದಾಹರಣೆಗೆ ಬಂಗಾಲ ಪ್ರಾಂತ.

ಆದರೆ ಈ ಮೇಲುವರ್ಗದವರೆನಿಸಿಕೊಂಡವರು (ನಿಜವಾಗಿಯೂ ಇವರು ಮೇಲು\-ವರ್ಗ\-ದವರೇ? – ಮೌಢ್ಯ, ಬಡತನ, ಸಂಕುಚಿತತೆ, ಸ್ವಾರ್ಥತೆ, ದುಃಖ–ಇವು ಎಲ್ಲ ಜಾತಿಯವರಿಗೂ ಹಾಸಿ ಹೊದೆಯುವಷ್ಟು ಇವೆ!) ಈಗ ಸದ್ಯ ಏನೇ ಕಾರಣ ಕೊಡಲಿ, ಹಿಂದುಳಿದವರನ್ನು ಪ್ರೀತಿ ವಿಶ್ವಾಸದಿಂದ ನೋಡಿಕೊಳ್ಳಲಿಲ್ಲ. ಅವರ ಏಳ್ಗೆಗಾಗಿ, ಅಲ್ಲೊಬ್ಬ ಇಲ್ಲೊಬ್ಬ ಶ್ರಮಿಸಿದರೂ ಸಂಘಟಿತರಾಗಿ ದುಡಿಯಲಿಲ್ಲ. ಧರ್ಮದ ತಿರುಳನ್ನು ಅವರಿಗೆ ಪ್ರಾಮಾಣಿಕರಾಗಿ ತಿಳಿಸಲು ಯತ್ನಿಸಲಿಲ್ಲ. ಹೃತ್ಪೂರ್ವಕವಾಗಿ ಅವರ ಸರ್ವತೋಮುಖ ಪ್ರಗತಿಗೆ ಮೇಲ್ಜಾತಿಯವರಾಗಿ ಏನು ಮಾಡಬಹುದೆಂದು ಯೋಚಿಸಿ, ಯೋಜನೆ ಕೈಗೊಳ್ಳಲಿಲ್ಲ. ಶತಮಾನಗಳಿಂದ ವಿದ್ಯೆ ಮತ್ತು ಸಂಸ್ಕೃತಿಗಳಿಂದ ವಂಚಿತರಾಗಿ, ತಿರಸ್ಕಾರಕ್ಕೊಳಗಾದ ಹಿಂದುಳಿದವರು, ಆಧುನಿಕ ವಿದ್ಯಾಭ್ಯಾಸವನ್ನು ಪಡೆದು ಕೀಳರಿಮೆಯನ್ನು ತ್ಯಜಿಸಿ, ಆತ್ಮವಿಶ್ವಾಸದಿಂದ ಮೆಲ್ಲನೇ ಮೇಲೇರುತ್ತ, ತಮಗಾದ ನೋವು, ಕಷ್ಟಗಳನ್ನು ತೋಡಿಕೊಳ್ಳಲು, ರೊಚ್ಚನ್ನು ತೋರಿಸಿದರೆ ಆಶ್ಚರ್ಯವಾದರೂ ಏನು? ಆದರೆ ಈ ರೊಚ್ಚಿಗೆ ಕಿಚ್ಚು ಹಚ್ಚುವ ಕೆಲಸ ಮಾಡಿ ತಮ್ಮ ಬೇಳೆಬೇಯಿಸಿಕೊಳ್ಳಲು ಬ್ರಿಟಿಷರು ಹೂಡಿದ ತಂತ್ರವನ್ನು ಯಾರೂ ತಿಳಿಯದಾದರು!

ಹಿಂದೆ ಬ್ರಾಹ್ಮಣಜಾತಿಗೆ ಸೇರಿದವರು ಹಳ್ಳಿಯ ಐಗಳಾಗಿ ಓದು ಬರಹ ಕಲಿಸುತ್ತಿದ್ದರು. ಹಳ್ಳಿಯ ಕಲಿತವರು ಕೃತಜ್ಞತೆಯಿಂದ ಅವರಿಗೆ ಜೀವನೋಪಾಯಕ್ಕಾಗಿ ಏನನ್ನಾದರೂ ಕೊಡು\-ತ್ತಿದ್ದರು. ಬಹಳ ದೊಡ್ಡ ವಿದ್ವಾಂಸರುಗಳಿಗೆ ರಾಜಾಶ್ರಯವಿರುತ್ತಿತ್ತು. ಬ್ರಿಟಿಷರು ತಮ್ಮ ಆಡಳಿತ ನಡೆಯಿಸಲು ಅನುಕೂಲವಾಗುವಂತೆ ಆಂಗ್ಲ ಕ್ರಮದ ಶಾಲೆಗಳನ್ನು ಪ್ರಾರಂಭಿಸಿದರು. ಐಗಳ ಶಾಲೆಗೆ ಸ್ಪರ್ಧಿಯಾಗಿ ಬಂದ ಸರಕಾರದ ಶಾಲೆಗಳೆ ಪ್ರಬಲವಾದವು. ವಿದ್ಯೆಯೇ ತಮ್ಮ ಕಸುಬಾಗಿದ್ದ ಈ ಜನ ಸರಕಾರಿ ಶಾಲೆಗಳಲ್ಲಿ ವಿದ್ಯೆ ಕಲಿಯಲು ಮುಂದಾಗಬೇಕಾಯಿತು. ವಿದ್ಯೆ ಕಲಿತು ಸಣ್ಣಪುಟ್ಟ ಹುದ್ದೆ ಪಡೆಯುತ್ತ ಹೋದರು. ಉಳಿದವರು ಈ ಹೊಸ ವಿದ್ಯೆಯನ್ನು ಕಲಿಯಬಾರದೆಂಬ ನಿರ್ಬಂಧವಿರಲಿಲ್ಲ. ಜೀವನೋಪಾಯಕ್ಕೆ ಪರಂಪರೆಯಿಂದ ಬಂದ ವೃತ್ತಿಗಳಿಗೆ ತೊಂದರೆಗಳಿರ\-ದಿದ್ದುದ\-ರಿಂದ, ಆಗ ಅವರು ವಿದ್ಯಾಭ್ಯಾಸಕ್ಕೆ ಗಮನ ಕೊಡಲಿಲ್ಲ. ಸರಕಾರಿ ಚಾಕರಿಗಾಗಿ ಮನೆ ಮಾರುಬಿಟ್ಟು ಎಲ್ಲೆಲ್ಲೋ ಅಲೆಯುವುದಕ್ಕೆ ಅವರು ಸಿದ್ಧರಿರಲಿಲ್ಲ. ಕ್ರಮೇಣ ಬ್ರಾಹ್ಮಣರಿಗೆ ಬ್ರಿಟಿಷರು ದೊಡ್ಡ ದೊಡ್ಡ ಹುದ್ದೆಗಳನ್ನು ಕೊಟ್ಟರು. ಸಾಕಷ್ಟು ಸಂಬಳ, ಗೌರವದ ಸ್ಥಾನಗಳೂ ದೊರೆತವು. ಸರಕಾರಿ ಚಾಕರಿಯ ಸ್ಥಾನಗಳೆಲ್ಲ ಇವರಿಂದ ತುಂಬತೊಡಗಿದವು. ಇವರ ಮಕ್ಕಳು ಆಧುನಿಕ ವಿದ್ಯೆಯನ್ನು ಪಡೆದು ಸ್ವಾತಂತ್ರ್ಯದ ಮಾತೆತ್ತಿ ಬ್ರಿಟಿಷರನ್ನು ಭಾರತದಿಂದ ಓಡಿಸಬೇಕೆನ್ನತೊಡಗಿದರು. ಬ್ರಿಟಿಷರು ಇವರಿಗೆ ಹೇಗೆ ಮದ್ದರೆಯಬೇಕೆಂಬುದನ್ನು ತಿಳಿಯರೇ? ‘ನೋಡಿರಿ! ಸರಕಾರಿ ಚಾಕರಿಯ ಎಲ್ಲ ಸ್ಥಾನಗಳಿಗೂ ಇವರದೇ ಪಾರುಪತ್ಯ. ನೀವು ಶತಮಾನಗಳಿಂದ ಶೋಷಿತರು. ನೀವು ಈ ಸ್ಥಾನಗಳನ್ನು ಪಡೆಯದೆ ಕೈಕಟ್ಟಿ ಕೂತುಕೊಳ್ಳುವಿರಾ?’–ಎಂದರು. ಬ್ರಿಟಿಷರ ಈ ಕುಹಕವನ್ನು ಉಚ್ಚವರ್ಗದ ಜನ, ಇತರ ವರ್ಗದ ಜನರ ಪ್ರೀತಿ ವಿಶ್ವಾಸವನ್ನು ಗಳಿಸಿ ಯಥಾರ್ಥವೇನೆಂಬುದನ್ನು ಅವರಿಗೆ ತಿಳಿಸಲು ಅಸಮರ್ಥರಾದರು. ಈ ಮಧ್ಯೆ ಬ್ರಾಹ್ಮಣೇತರರೂ ವಿದ್ಯಾಭ್ಯಾಸವನ್ನು ಪಡೆಯತೊಡಗಿದ್ದರು. ಆದರೆ ಎಲ್ಲರಿಗೂ ಸಿಗುವಷ್ಟು ಉದ್ಯೋಗಾವಕಾಶವಿರಲಿಲ್ಲ. ಸ್ಪರ್ಧೆ ಪ್ರಾರಂಭವಾಯಿತು. ವಿದ್ಯಾಭ್ಯಾಸದಲ್ಲಿ ಪರಂಪರೆಯಿಂದ ಸ್ವಲ್ಪ ಹೆಚ್ಚು ಕುಶಲತೆಯನ್ನು ಪಡೆದ ಬ್ರಾಹ್ಮಣರು ಯಶಸ್ವಿಯಾಗುವುದು ಕಂಡು ಬಂದಿತು. ಇತರರಿಗೆ ಈ ಬಗ್ಗೆ ದ್ವೇಷ ಉಂಟಾದುದು ಆಶ್ಚರ್ಯವಲ್ಲ. ಆಮೇಲೆ ದೂಷಣೆ, ದೋಷಾ\-ರೋಪಣೆ ಹೆಚ್ಚಿ ಅವರಿಗೆ ಈಗ ಉದ್ಯೋಗ ದೊರೆಯದಂತೆ ಪ್ರಯತ್ನ ನಡೆಯುತ್ತಿದೆ. ಈ ದ್ವೇಷಾಸೂಯೆ, ದೂಷಣೆಯ ಮನೋವೃತ್ತಿಯ ದುರುಪಯೋಗ ಪಡೆದು ನಮ್ಮನ್ನು ಒಂದು ಗೂಡದಂತೆ ಮಾಡಲು ಹಲವು ಹೊರಗಿನ ಗುಂಪುಗಳು ಕೆಲಸ ಮಾಡುತ್ತಿವೆ ಎಂಬ ಪ್ರಜ್ಞೆಯು ಬ್ರಾಹ್ಮಣರಲ್ಲಾಗಲೀ, ಬ್ರಾಹ್ಮಣೇತರರಲ್ಲಾಗಲೀ ಇನ್ನೂ ಇಲ್ಲದಿರುವುದು ಸ್ಪಷ್ಟ. ಗುಲಾಮಗಿರಿಯ ಒಂದು ಲಕ್ಷಣ ಪರಸ್ಪರ ದ್ವೇಷ, ಅಸೂಯೆ ಎನ್ನುತ್ತಾರೆ. ಈ ಪ್ರವೃತ್ತಿಯನ್ನು ನಮ್ಮ ಜನ ಉಳಿಸಿಕೊಂಡಿದ್ದಾರೆ, ಬೆಳೆಸಿಕೊಂಡಿದ್ದಾರೆ. ಇದು ದುರ್ದೈವದ ಸಂಗತಿ. ಹೀಗೆ ವಿದೇಶೀಯರ ಕೈ ಕೆಳಗೆ ಗುಲಾಮ ಜನಾಂಗದ ವಿವಿಧ ಗುಂಪುಗಳಲ್ಲಿ ದ್ವೇಷ, ಅಸೂಯೆ, ವಿಜೃಂಭಿಸ ತೊಡಗಿತ್ತು. ಇದು ನಿಲ್ಲುವ ಲಕ್ಷಣವೂ ಇಲ್ಲವೆನ್ನಿ!

ಯಾವ ಸಮಾಜದಲ್ಲೇ ಆಗಲಿ, ಮುಂದುವರಿದವರು ಹಿಂದುಳಿದವರನ್ನು ಎತ್ತಲು ತ್ಯಾಗ ಮತ್ತು ಸಮರ್ಪಣೆಯ ಭಾವದಿಂದ ದೀರ್ಘಕಾಲ ದುಡಿಯದೆ, ತಮ್ಮ ಪಾಡಿಗೆ ತಾವು ಎಂದಿದ್ದು\-ಕೊಂಡರೆ, ಅವರ ಏಳ್ಗೆಗೆ ಅವರೇ ಕಲ್ಲು ಹಾಕಿಕೊಂಡಂತೆ. ಉದ್ಯೋಗದ ಸ್ಪರ್ಧೆಯಲ್ಲಿ ಜಯಶಾಲಿ\-ಗಳಾದ ಬ್ರಾಹ್ಮಣ ಜಾತಿಯ ಜನರು ಸುಖಿಗಳಾದರೆ? ಹೊಟ್ಟೆ ಪಾಡಿಗಾಗಿ, ತಮ್ಮ ಜನರಿಂದ ದೂರಸರಿದು, ಎಲ್ಲೆಲ್ಲೊ ಅಲೆಯುತ್ತ, ತಮ್ಮ ಧಾರ್ಮಿಕ ಶ್ರದ್ಧೆಯನ್ನು ಕಳೆದುಕೊಂಡು, ಇತರ ವರ್ಗದ ದ್ವೇಷವನ್ನೂ ಸಂಪಾದಿಸಿಕೊಂಡು, ತಮ್ಮ ಮಕ್ಕಳ ಭವಿಷ್ಯದ ಬಗ್ಗೆ ಚಿಂತಿತರಾದರು.

ಪಶ್ಚಿಮದ ಕ್ರಮವನ್ನನುಸರಿಸುವ ಭಾರತದಲ್ಲಿ ಇಂದು ಸ್ಪರ್ಧೆ ಅನಿವಾರ್ಯ. ಆದರೆ ದ್ವೇಷವೆಂಬ ದೋಷವನ್ನು ಕಡಿಮೆ ಮಾಡಿಕೊಳ್ಳಲು ಎಲ್ಲರೂ ಪ್ರಾಮಾಣಿಕರಾಗಿ ದುಡಿದರೆ, ಅದು ಅಸಾಧ್ಯವಾದ ಕೆಲಸವಲ್ಲ. ಈ ದಿಸೆಯಲ್ಲಿ ಆಧ್ಯಾತ್ಮಿಕ ದೃಷ್ಟಿಕೋನವೊಂದೇ ನಿಜವಾದ ಸಹಾಯ ನೀಡಬಲ್ಲದು. ಗುಂಪುಗೂಡುವುದು, ಜಾತಿಗಳಾಗಿ ವಿಂಗಡವಾಗುವುದು, ನೈಸರ್ಗಿಕ ನಿಯಮವೊ ಎನ್ನುವಷ್ಟು ಸಹಜ. ಆದರೆ, ವಿಭಿನ್ನ ಜಾತಿಮತಗಳೊಡನೆ ದ್ವೇಷವನ್ನು ಬಿತ್ತಿದರೆ, ಅದು ಕ್ರಮೇಣ ಸರ್ವನಾಶಕ್ಕೆ ನಾಂದಿ. ಧಾರ್ಮಿಕ ಕ್ಷೇತ್ರವನ್ನು ನೋಡಿದರೆ, ಎಲ್ಲ ಧರ್ಮಗಳಲ್ಲೂ ಅನೇಕ ಗುಂಪುಗಳಿವೆ. ಕ್ರೈಸ್ತಸಂಪ್ರದಾಯದಲ್ಲಿ ಇನ್ನೂರ ಐವತ್ತು ಪಂಗಡಗಳಿವೆ. ಮುಸಲ್ಮಾನರಲ್ಲೂ ಇವೆ. ಲೌಕಿಕ ಮಟ್ಟದಲ್ಲೂ, ರಾಜಕೀಯದಲ್ಲೂ, ಹಲವಾರು ಪಕ್ಷ, ಪಂಗಡಗಳಾಗುತ್ತಿವೆ. ಇದು ಆವಶ್ಯಕವಿರಬಹುದು. ಆದರೆ, ದ್ವೇಷಕ್ಕೆ ಇಂಬುಕೊಡದಂತೆ ಯತ್ನಿಸಬೇಕು. ದ್ವೇಷ ಮಹಾದೋಷ ಎಂಬುದನ್ನು ಮರೆಯಬಾರದು. ವಿಶ್ವಾಸ ಸೌಹಾರ್ದಗಳನ್ನು ಬೆಳೆಯಿಸಲು, ಪ್ರತಿಯೊಂದು ಜಾತಿ ಮತದವರು, ಅನ್ಯ ಜಾತಿಮತದವರ ಗುಣಗಳನ್ನು, ವೈಶಿಷ್ಟ್ಯಗಳನ್ನು ಎತ್ತಿ ಆಡಬೇಕು. ದೋಷವನ್ನು ಭೂತಕನ್ನಡಿಯಿಂದ ನೋಡಿ ನಿಂದಿಸಬಾರದು. ಇದು ಅತ್ಯಂತ ಕೆಡುಕಿಗೆ ಕಾರಣವಾದ್ದರಿಂದ ಪಾಪಕಾರ್ಯವಾಗುತ್ತದೆ.


\section*{ಎದೆಗಾರಿಕೆ ಪ್ರೀತಿಯ ಫಲ}

\addsectiontoTOC{ಎದೆಗಾರಿಕೆ ಪ್ರೀತಿಯ ಫಲ}

೧೯೧೪ನೇ ಇಸವಿಯವರೆಗೂ ಜೀವಿಸಿದ್ದ ಮಹಾ ಅನುಭಾವಿಗಳಾದ ಗೊಂದಾವಳೇಕರ್ ಬ್ರಹ್ಮ\-ಚೈತನ್ಯ ಮಹಾರಾಜರು ಶಿಷ್ಯನನ್ನು ಉದ್ಧರಿಸಿದ ಒಂದು ಘಟನೆ ಮನನೀಯವಾಗಿದೆ. ಬ್ರಹ್ಮಚೈತನ್ಯರಿಂದ ಮಂತ್ರದೀಕ್ಷೆಯನ್ನು ಪಡೆದಿದ್ದ ಒಬ್ಬಾತ, ದುಸ್ಸಂಗದಿಂದ ದುರ್ವ್ಯಸನಕ್ಕೆ ಬಲಿಯಾದ. ವೇಶ್ಯೆಯೊಬ್ಬಳ ಮೋಹ ಜಾಲದಲ್ಲಿ ಸಿಲುಕಿ, ತನ್ನ ಸಂಪತ್ತನ್ನೆಲ್ಲ ಕಳೆದುಕೊಂಡ. ನೆಂಟರಿಷ್ಟರೂ ಅವನನ್ನು ಬಿಟ್ಟು ದೂರವಾದರು. ಬ್ರಹ್ಮಚೈತನ್ಯರ ಶಿಷ್ಯರೂ, ಪರಿವಾರದವರೂ ಆತನನ್ನು ಹತ್ತಿರ ಸೇರಿಸುತ್ತಿರಲಿಲ್ಲ. ಒಮ್ಮೆ ಗದಗಿನಲ್ಲಿ ಭೀಮರಾಯರ ಮನೆಯಲ್ಲಿ ಬ್ರಹ್ಮ ಚೈತನ್ಯರು ತಂಗಿದ್ದರು. ಆತ ಭೀಮರಾಯರ ಮನೆಯ ಬಾಗಿಲಿನಲ್ಲಿ ಭಿಕಾರಿಯಂತೆ ನಿಂತು ‘ನಾನು ಒಳಗೆ ಬರಬಹುದೆ?’ ಎಂದು ಹೇಳಿ ಕಳುಹಿಸಿದ. ಬ್ರಹ್ಮಚೈತನ್ಯರು ಕೂಡಲೇ ಅವನನ್ನು ಬರಮಾಡಿಕೊಂಡರು. ಆತ ಬಂದವನೇ ಅವರ ಪಾದಕ್ಕೆ ಸಾಷ್ಟಾಂಗವೆರಗಿ ಕೈ ಮುಗಿದು ವಿನಮ್ರನಾಗಿ ನಿಂತುಕೊಂಡ. ಆಗ ಬ್ರಹ್ಮಚೈತನ್ಯರು ‘ಜನರು ನಿನ್ನನ್ನು ಬಿಟ್ಟು ಬಿಟ್ಟಿದ್ದರೂ, ನಾನು ನಿನ್ನನ್ನು ಎಂದಿಗೂ ಬಿಡುವುದಿಲ್ಲ’ ಎಂದು ಹೇಳಿದರು. ಈ ಅಭಯ ವಾಕ್ಯಗಳನ್ನು ಆ ಮಹಾತ್ಮರಿಂದ ಕೇಳಿದ ಆತ ಬಿಕ್ಕಿಬಿಕ್ಕಿ ಅಳತೊಡಗಿದ. ದುಃಖ ಸ್ವಲ್ಪ ಶಮನವಾದ ಮೇಲೆ ಶ‍್ರೀಮಹಾ ರಾಜರು ‘ನಿನ್ನ ಪಶ್ಚಾತ್ತಾಪ ಸತ್ಯವಾಗಿದ್ದಲ್ಲಿ, ಆಗಿ ಹೋಗಿರುವ ಪಾಪಕ್ಕೆ ನಾನು ಹೊಣೆಯಾಗುತ್ತೇನೆ. ಆದರೆ ಪುನಃ ಈ ತರಹದ ಕೃತ್ಯ ಎಂದಿಗೂ ಮಾಡುವುದಿಲ್ಲವೆಂದು ರಾಮನ ಪಾದಗಳಲ್ಲಿ ಕೈಯಿಟ್ಟು ಶಪಥ ಮಾಡು’ ಎಂದರು. ಆತ ಹಾಗೆಯೇ ಶಪಥ ಮಾಡಿದ. ಅವರು ತೋರಿಸಿದ ಪ್ರೀತಿ ಕರುಣೆಗಳಿಂದ ಆತನಲ್ಲಿ ಕೀಳರಿಮೆ ಕರಟಿ ಧೈರ್ಯ ಉಕ್ಕಿತು. ಮುಂದೆ ದೇವರಲ್ಲಿ ಶರಣಾಗಿ ಚೆನ್ನಾಗಿ ಸಾಧನೆಯನ್ನು ಮಾಡಿ ಆಧ್ಯಾತ್ಮಿಕವಾಗಿ ಮುನ್ನಡೆದ.

ಜೀವನದ ಬಗೆಗಿನ ತಮ್ಮ ದೃಷ್ಟಿಕೋನ ಹಾಗೂ ಸ್ಥೈರ್ಯ–ಇವುಗಳ ಸಹಾಯದಿಂದ ಅಂಧ ಮಹಿಳೆಯೊಬ್ಬರು ಅನಿವಾರ್ಯವನ್ನು ಎದುರಿಸಿದ ಕ್ರಮ, ವಿಧಾನ ಇಲ್ಲಿದೆ.\footnote{ಡಾ. ಪ್ರಭುಶಂಕರ, “ಜನಮನ”.}ದೇವರು, ಅಧ್ಯಾತ್ಮದ ವಿಚಾರವನ್ನು ಆಕೆ ಹೇಳಿರದಿದ್ದರೂ, ನಡೆದುಕೊಂಡ ವಿಧಾನ, ವರ್ತನೆಯ ಔನ್ನತ್ಯ, ಆಧ್ಯಾತ್ಮಿಕ ವೀರರ ಚಾರಿತ್ರ್ಯವನ್ನೇ ಹೋಲುತ್ತದೆ.

ಶ‍್ರೀಮತಿ ಹ್ಯಾರಿಯೆಟ್ ಅಮೇರಿಕದಿಂದ ಭಾರತೀಯ ಸಂಗೀತ ಅಭ್ಯಸಿಸಲು ಬಂದ ಒಬ್ಬ ಅಂಧಮಹಿಳೆ. ಬಾಲ್ಯದಲ್ಲೇ ಅಪಘಾತವೊಂದರಲ್ಲಿ ದೃಷ್ಟಿಯನ್ನು ಕಳೆದುಕೊಂಡರೂ ಬ್ರೈಲ್ ಭಾಷೆ ಕಲಿತು ಪದವೀಧರೆಯಾದರು. ಲೇಖಕರೊಡನೆ ಮಾತನಾಡುತ್ತ ತಮ್ಮ ತಂದೆತಾಯಿಗಳು ತಮ್ಮ ಅಭಿವೃದ್ಧಿಗೆ ಸಹಕರಿಸಿದ್ದರ ಕುರಿತು ಹೇಳಿದರು:

‘ನಮ್ಮ ತಂದೆ ತಾಯಿಗಳು ಧೈರ್ಯಸ್ಥರು. ಕುರುಡಿ ಮಗಳ ತಂದೆ ತಾಯಿಗಳಿಗೆ ದೊರಕ\-ಬಹು\-ದಾದ ಅನುಭವಗಳನ್ನು ಅವರು ಹೃತ್ಪೂರ್ವಕವಾಗಿ ಸ್ವಾಗತಿಸಿ, ಈ ಹೊಸ ರೀತಿಯ ಹೊಣೆಗಾರಿಕೆಗೆ ಹೊಂದಿಕೊಳ್ಳುತ್ತಾ ಹೋದರು. ಅವರು ನನ್ನನ್ನು ತುಂಬ ಪ್ರೀತಿಯಿಂದ ಮಾತ್ರವಲ್ಲ ತಮ್ಮ ಇತರ ಮಕ್ಕಳಂತೆಯೇ ಕಂಡರು.’

‘ಅಂದರೆ?’

‘ಅಂದರೆ, ನಾನು ಕುರುಡಿ, ಅದಕ್ಕಾಗಿ ನನ್ನನ್ನು ವಿಶೇಷವಾಗಿ ಪ್ರೀತಿಸಬೇಕು ಎಂದೇನೂ ಭಾವಿಸಲಿಲ್ಲ, ಅಲ್ಲಿ ಹೋಗಬೇಡ, ಇಲ್ಲಿ ಹೋಗಬೇಡ, ಅದು ಮಾಡಬೇಡ, ಇದು ಮಾಡಬೇಡ ಎಂದು ಎಂದೂ ಹೇಳಲಿಲ್ಲ. ನಾನು ಓಡುತ್ತಿದ್ದೆ, ಈಜುತ್ತಿದ್ದೆ, ಕುಣಿಯುತ್ತಿದ್ದೆ, ಅವರು ಪದೇ ಪದೇ ನನ್ನ ಓಡಾಟ ಆಟಪಾಠಗಳಲ್ಲಿ ತಲೆಹಾಕುತ್ತಿರಲಿಲ್ಲ. ನನ್ನನ್ನು ಸುಮ್ಮನೆ ಗಮನಿಸುತ್ತಿದ್ದರು. ತೀರ ಅಗತ್ಯವಾದಾಗ, ನನಗೆ ಏನಾದರೂ ಅಪಾಯವಾಗುತ್ತದೆ ಎಂದು ಕಂಡಾಗ ಮಾತ್ರ ಸಹಾಯ ಮಾಡುವುದಕ್ಕೆ ಬರುತ್ತಿದ್ದರು. ಇದರಿಂದ ನನಗೆ ತುಂಬ ಉಪಕಾರವಾಯಿತು.’

‘ಹೇಗೆ?’

‘ನನಗೆ ತುಂಬ ಧೈರ್ಯ ಬಂತು, ನಾನು ಸ್ವಾವಲಂಬಿ ಆದೆ. ನಾನು ಕಣ್ಣು ಕಾಣ\-ದವಳು, ನನಗೆ ಏನೋ ಕೊರತೆಯಿದೆ, ನಾನು ಇತರರ ಸಮಕ್ಕೆ ಬರಲಾರೆ ಎಂಬ ಭಾವನೆಯೇ ಬರಲಿಲ್ಲ.’

‘ನಿಮ್ಮ ಊಟ ಉಪಚಾರ ನಡೆಯುವುದು ಹೇಗೆ?’

‘ನಾನೇ ಅಡಿಗೆ ಮಾಡಿಕೊಳ್ಳುತ್ತೇನೆ. ನನಗೆ ಅಡುಗೆ ಮಾಡುವುದು ಎಂದರೆ ಪ್ರೀತಿ. ನನ್ನ ತಾಯಿಗೆ ಆಶ್ಚರ್ಯ ಆಗುವಷ್ಟು ಚೆನ್ನಾಗಿ ಅಡುಗೆ ಮಾಡುತ್ತೇನೆ. ಪದಾರ್ಥಗಳ ಡಬ್ಬಗಳನ್ನು, ಪೊಟ್ಟಣಗಳನ್ನು ಮುಟ್ಟಿಯೇ, ಇಂಥದು ಎಂದು ತಿಳಿದುಕೊಳ್ಳುತ್ತೇನೆ. ಕೈಯ ಅಳತೆಯೇ ಅಳತೆ. ಹಾಲು ಕಾದಿದೆ ಎಂಬುದು ಅದರ “ಸೊಂಯ್​” ಶಬ್ದದಿಂದಲೇ ಗೊತ್ತಾಗುತ್ತದೆ. ಇಂಥದು ಬೇಯೋದಕ್ಕೆ ಇಷ್ಟು ಹೊತ್ತುಬೇಕು ಎಂದೂ ಗೊತ್ತು. ಅಷ್ಟು ಹೊತ್ತಾಯಿತು ಎಂದು ನನಗೆ ಕರಾರುವಾಕ್ಕಾಗಿ ಗೊತ್ತಾಗುತ್ತದೆ. ಅಡುಗೆ ನನಗೆ ಸಮಸ್ಯೆಯೇ ಅಲ್ಲ.’

‘ಹ್ಯಾರಿಯೆಟ್, ನಿಮ್ಮ ಬಗ್ಗೆ ನನಗೆ ಇನ್ನೊಂದು ಆಶ್ಚರ್ಯ–ಸದಾ ನೀವು ಸಂತೋಷ\-ವಾಗಿರು\-ತ್ತೀರಿ. ಮಹಡಿಯ ಮೇಲೆ ನಿಮ್ಮ ಸ್ನೇಹಿತೆಯು ಮನೆಗೆ ಬಂದಾಗಲಂತೂ ನಿಮ್ಮ ನಗುವಿನಿಂದ ಛಾವಣಿ ಹಾರಿಯೇಹೋಗುತ್ತದೆ–ಎಂಬಂತೆ ಕೇಕೆ ಹಾಕುತ್ತೀರಿ. ಈ ಸಂತೋಷಕ್ಕೆ ಕಾರಣ ಏನು?’

‘ಬೇಕಾದಷ್ಟು! ಹಾಗೇ ಯೋಚನೆ ಮಾಡಿ. ಪ್ರಪಂಚದಲ್ಲಿ ಎಷ್ಟೋ ಜನಕ್ಕೆ ಊಟ ಇಲ್ಲ. ಬಟ್ಟೆಯಿಲ್ಲ, ಮನೆ ಇಲ್ಲ. ನಮಗೆ ಇವೆಲ್ಲ ಇವೆ. ಜೊತೆಗೆ ವಿದ್ಯೆ ಇದೆ, ಬುದ್ಧಿ ಇದೆ. ಇವೆಲ್ಲ ಕಡಿಮೆ ಸಂಪತ್ತು ಎಂದು ಭಾವಿಸುವಿರಾ? ಇದರಿಂದ ನಾವು ಸಂತೋಷವಾಗಿರದೆ, ಬೇರೆ ಯಾವ ರೀತಿ ಇರುವುದು ಸಾಧ್ಯ?’

‘ನಿಮಗೆ ಕಣ್ಣು ಕಾಣುವುದಿಲ್ಲವಲ್ಲ. ಬೇಸರ ಇಲ್ಲವೇ?’

‘ಇಲ್ಲ ಎಂದಿಗೂ ಇಲ್ಲ. ಜಗತ್ತು ನಮಗೆ ಬೇಕಾದಷ್ಟನ್ನು ಕೊಟ್ಟಿದೆ. ಇಷ್ಟಕ್ಕೂ ನಾವು ಜಗತ್ತಿಗೆ ಹಿಂದಿರುಗಿಸುವುದಾದರೂ ಎಷ್ಟು? ತೀರ ಕಡಿಮೆ. ಆದರೂ ಜಗತ್ತು ನಮಗೆ ಎಷ್ಟೊಂದನ್ನು ನೀಡುತ್ತಿದೆ. ಇದಕ್ಕಿಂತ ಹೆಚ್ಚಾಗಿ ನಿರೀಕ್ಷಿಸುವುದು ತಪ್ಪು. ಅದು ಇನ್ನೂ ಕೊಡಬೇಕು ಎಂದು ಕೇಳಲು ನಮಗೆ ಹಕ್ಕಿದೆಯೇ? ಇದನ್ನು ಸದಾ ಮನಸ್ಸಿನ ಮೂಲೆಯಲ್ಲಿ ಉಳಿಸಿಕೊಂಡರೇ ಸಾಕು–ಸಂತೋಷವಾಗಿರಬಹುದು.’

ಪಾಲಿಗೆ ಬಂದದ್ದು ಪಂಚಾಮೃತವೆಂದು ಸಂತೋಷದಿಂದ ಸ್ವೀಕರಿಸಿ, ಜೀವನದ ಕುಂದು ಕೊರತೆಗಳಿಗೆ ಕುರುಡಾಗಿ ಮುನ್ನಡೆವ ಹ್ಯಾರಿಯೆಟ್​ರ ಎದೆಗಾರಿಕೆ ನಮಗೆಲ್ಲ ಆದರ್ಶವಾಗಬೇಕು.

ಒಮ್ಮೆ ಹ್ಯಾರಿಯೆಟ್ ಬ್ಯಾಂಕಿಗೆ ಹೋಗಿದ್ದರು. ಅಲ್ಲಿ ಹತ್ತು ನಿಮಿಷದ ಕೆಲಸಕ್ಕೆ–ಒಂದು ಚೆಕ್ ಕಟ್ಟುವುದು, ಮತ್ತೊಂದರಿಂದ ಸ್ವಲ್ಪ ಹಣ ತೆಗೆದುಕೊಳ್ಳುವುದು, ಒಂದು ಗಂಟೆ ಕಾಯಬೇಕಾಗಿ ಬಂತು. ಯಾರೊಬ್ಬರೂ ಈ ಕುರುಡಿಯನ್ನು ಕಂಡು, ಸೌಜನ್ಯಕ್ಕಾದರೂ, ‘ನಿಮಗೇನಾದರೂ ಸಹಾಯ ಬೇಕೆ?’ ಎಂದು ಕೇಳಲಿಲ್ಲ. ಆಕೆ, ಅಂಥ ವಿಶೇಷ ಸವಲತ್ತುಗಳನ್ನು ನಿರೀಕ್ಷಿಸುವವರೂ ಅಲ್ಲ. ಹ್ಯಾರಿಯೆಟ್ ಸ್ಪರ್ಶದಿಂದಲೇ, ಇದು ಅಮೇರಿಕದಿಂದ ಬಂದ ಚೆಕ್ಕು–ಇಲ್ಲಿ ಕಟ್ಟಬೇಕಾದದ್ದು, ಇದು ನನ್ನ ಚೆಕ್ಕು ಹಣ ಪಡೆಯಬೇಕಾದದ್ದು ಎಂದು ಹೇಳಿಬಿಟ್ಟರು. ಒಂದು ಸಣ್ಣ ರಟ್ಟಿನ ಚೂರಿನ ಸಹಾಯದಿಂದ ರುಜು ಹಾಕಿದರು. ತಮಗೆ ಯಾವುದೇ ರೀತಿಯ ಅನಾನುಕೂಲವಾಯಿತು ಎಂದು ಒಂದು ಕ್ಷಣವಾದರೂ ಮುಖ ಮುದುಡಲಿಲ್ಲ.

ಮತ್ತೊಮ್ಮೆ ಅವರ ಸ್ನೇಹಿತೆಗೆ ಆಪರೇಷನ್ ಆಗಿ ಆಸ್ಪತ್ರೆಯಲ್ಲಿದ್ದರು. ಉಳಿದವರು ಎಷ್ಟು ಹೇಳಿದರೂ ಕೇಳದೆ ತಾವೇ ಆಸ್ಪತ್ರೆಯ ವಾರ್ಡಿನಲ್ಲಿ ಹಗಲು ರಾತ್ರಿ ಇದ್ದು, ಕಣ್ಣು ಕಾಣದೆ ಇದ್ದರೂ, ಗೋಡೆ ಹಿಡಿದುಕೊಂಡೇ ಓಡಾಡಿ ಅವರ ಸೇವೆ ಮಾಡಿದರು.

ಹ್ಯಾರಿಯೆಟ್​ರವರಂತೆ, ನಮ್ಮ ಜೀವನದ ಎಡರು, ತೊಡರುಗಳು ನಮ್ಮ ವ್ಯಕ್ತಿತ್ವವನ್ನು ಮೊಟಕುಗೊಳಿಸದಂತೆ, ನಾವು ಅವನ್ನು ಎದುರಿಸುತ್ತಿದ್ದೇವೆಯೇ? ಉಳಿದ ಅನೇಕರಿಗಿಂತ ನಮಗೆ ಎಷ್ಟೆಲ್ಲ ಇದೆ. ಆದರೂ ನಮ್ಮ ‘ಬೇಕು’ಗಳಿಗೆ ಮಿತಿಯಿಲ್ಲ; ಗೊಣಗಾಟ ನಿಂತಿಲ್ಲ. ಇಷ್ಟಾದರೂ ನಮ್ಮದು ಸದಾ ಅಳುಮೂತಿ. ನಮ್ಮ ವ್ಯಕ್ತಿತ್ವದ ಉತ್ತಮ ಅಂಶಗಳನ್ನು ಪೋಷಿಸಿ, ಪರಿಪಾಲಿಸಿ, ನಮ್ಮ ಕುಂದುಕೊರತೆಗಳು ನಮ್ಮ ಸಾರ್ಥಕ ಜೀವನಕ್ಕೆ ಅಡ್ಡಿಯಾಗುವಂತೆ ನೋಡಿಕೊಂಡಲ್ಲಿ, ನಮ್ಮ ಬದುಕು ಸುಂದರವಾದೀತು, ಅರ್ಥಪೂರ್ಣವಾದೀತು!


\section*{ಟೀಕೆಗೆ ಕಿವುಡಾಗಿ}

\vskip -7pt\addsectiontoTOC{ಟೀಕೆಗೆ ಕಿವುಡಾಗಿ}

ಇತರರು ನಿಮ್ಮನ್ನು ಟೀಕಿಸಿದರೆ ಅದನ್ನು ನೀವು ಹೇಗೆ ಸ್ವೀಕರಿಸುತ್ತೀರಿ? ಅದು ಯಥಾರ್ಥವೂ, ರಚನಾತ್ಮಕವೂ ಆಗಿದ್ದರೆ ಅದನ್ನು ಒಪ್ಪಲು ನಿಮಗೆ ಕಷ್ಟವಾಗುವುದೇ? ನಿಮ್ಮ ದೋಷದೌರ್ಬಲ್ಯಗಳನ್ನು ತಿದ್ದಿಕೊಳ್ಳುವುದಕ್ಕೆ ನೀವು ಹಿಂಜರಿಯುತ್ತೀರಾ? ನಿಮ್ಮ ಅಭಿವೃದ್ಧಿಯ ದಾರಿಯಲ್ಲಿ ಮುಂದುವರಿಯಲು ನೀವೇಕೆ ಸಂಕೋಚಪಡಬೇಕು?

ಅವರು ಏನು ಹೇಳುತ್ತಾರೆ, ಇವರು ಏನು ಹೇಳುತ್ತಾರೆ–ಎನ್ನುವುದರ ಕುರಿತು ನೀವು ತಲೆ ಕೆಡಿಸಿಕೊಳ್ಳಬೇಕಾಗಿಲ್ಲ. ಆ ಅವರಿಗೂ ಈ ಇವರಿಗೂ ನಿಮ್ಮ ವಿಚಾರವನ್ನೇ ಯೋಚಿಸಲು ಸಮಯವಿದೆ ಎಂದು ನಿಮಗೆ ಹೇಳಿದವರಾರು? ಡೇಲ್ ಕಾರ್ನೆಗೀ ಹೇಳುವಂತೆ ನಮ್ಮ ನಿಮ್ಮ ಸಾವಿಗಿಂತಲೂ, ನಮ್ಮ ನಿಮ್ಮ ಸುದ್ದಿಗಿಂತಲೂ, ಅವರ ತಲೆನೋವೇ ಅವರಿಗೆ ಬೃಹತ್ ಸಮಸ್ಯೆಯಾಗಿರಬಹುದು. ಅವರೇನಾದರೂ ಹೇಳಿದರೆನ್ನಿ. ನಿಮ್ಮ ಸಮಸ್ಯೆಯನ್ನು ಸರಿಯಾಗಿ ಪರಿಶೀಲಿಸಿ, ಸಹಾನುಭೂತಿಯಿಂದ ಹೇಳುತ್ತಾರೇನು? ‘ಹೊಟ್ಟೆಯಿಂದ ಕರುಳು ಕಿತ್ತುಕೊಟ್ಟರೆ ಹುರಿಹಗ್ಗ\-ವೆಂದ’ ಎಂಬ ಗಾದೆಯ ಮಾತನ್ನು ನೀವು ಕೇಳಿರಬಹುದು. ನಿಮ್ಮಲ್ಲಿ ಅವರಿಗೆ ನಿಜವಾದ ಸಹಾನುಭೂತಿ ಹಾಗೂ ಪ್ರೀತಿ ಇಲ್ಲದೆ ಹೋದರೆ, ನಿಮ್ಮ ಪರಿಸ್ಥಿತಿಯನ್ನು ಅವರು ಅರ್ಥಮಾಡಿಕೊಳ್ಳುವುದಾದರೂ ಹೇಗೆ?

ಇತರರು ಮಾಡುವ ಟೀಕೆಗಳಲ್ಲಿ ಕೆಲವು ಯಥಾರ್ಥವಾದವು ಇರಬಹುದು. ಆದರೆ, ಬಹುಮಟ್ಟಿಗೆ ಟೀಕೆಗಳಲ್ಲಿ, ಯಥಾರ್ಥತೆಗಿಂತಲೂ ಕಲ್ಪನಾವಿಲಾಸ, ಶಬ್ದಚಮತ್ಕಾರ, ಇನ್ನೊಬ್ಬರನ್ನು ಹೀನಾಯಮಾಡಬೇಕೆಂಬ ಉದ್ದೇಶವಿರುತ್ತದೆ. ಅಂಥ ಟೀಕೆಗಳ ಹಿನ್ನೆಲೆಯಲ್ಲಿ ಟೀಕಾಕಾರರ ಅಸೂಯಾಪರ ಮನೋಭಾವ ಹುದುಗಿದ್ದು ಅವರು ಆ ಮೂಲಕ ತೃಪ್ತಿ ಪಡೆಯುತ್ತಿರಬಹುದು. ಅಂಥವರ ಮಾತುಗಳಿಂದ ನಿಮ್ಮಲ್ಲಿ ತೀವ್ರವಾದ ಅಶಾಂತಿಯುಂಟಾದರೆ ನಿಮ್ಮಲ್ಲಿ ಆತ್ಮ ವಿಶ್ವಾಸ ಕಡಿಮೆಯಾಗಿದೆಯೆಂದು ತಿಳಿಯಬಹುದು. ಕರ್ತವ್ಯನಿಷ್ಠೆ, ಆತ್ಮವಿಶ್ವಾಸ ನಿಮ್ಮಲ್ಲಿರುವುದಾದರೆ, ಇತರರ ವ್ಯರ್ಥ ಟೀಕೆ ಮಾತುಗಳು ನಿಮ್ಮನ್ನು ಕಂಗೆಡಿಸಲಾರವು.

ಇಂಥ ಟೀಕೆಮಾಡುವವರನ್ನು ಕುರಿತು, ಒಂದು ಸಂಗತಿ ನಿಮಗೆ ತಿಳಿದಿರಲಿ: ಇತರರನ್ನು ಟೀಕಿಸುವಾಗ ಅವರ ‘ಅಹಂ’ನ ತೂಕ ಕೊಂಚ ಏರಿರುತ್ತದೆ. ಇತರರ ದೋಷಗಳನ್ನು ಟೀಕಿಸುವಾತ ಪರೋಕ್ಷವಾಗಿ ತಾನು ಸ್ವಲ್ಪ ಮೇಲ್ಮಟ್ಟದಲ್ಲಿದ್ದೇನೆ, ಎಂದು ಭ್ರಮಿಸುತ್ತಾನೆ; ತಾನೇನೂ ಕಡಿಮೆ ಬುದ್ಧಿವಂತನಲ್ಲ ಎಂಬುದನ್ನು ಪ್ರದರ್ಶಿಸಲು ಹೊರಡುತ್ತಾನೆ.

ತಮಗಿಂತ ಹೆಚ್ಚು ವಿದ್ಯಾವಂತರಾದವರನ್ನು, ಹೆಚ್ಚು ಯಶಸ್ವಿಗಳಾದವರನ್ನು ಟೀಕಿಸುವಾಗ, ಅವರಲ್ಲಿರುವ ಏನಾದರೊಂದು ಹುಳುಕನ್ನು ಕಂಡುಹಿಡಿಯುವಾಗ, ಕೆಲವರಿಗೆ ಒಂದು ರೀತಿಯ ಅನಾಗರಿಕ ಆನಂದವುಂಟಾಗುತ್ತದೆ. ಇನ್ನು ಕೆಲವರಿಗೆ ಮಾತಿನ ಚಮತ್ಕಾರಕ್ಕೇ ಪ್ರಾಧಾನ್ಯ ಕೊಡುವ ಚಪಲ. ಅಂಥವರು ಅರ್ಥಕ್ಕಿಂತಲೂ ಶಬ್ದಕ್ಕೆ ಹೆಚ್ಚು ಪ್ರಾಮುಖ್ಯ ಕೊಟ್ಟು ಮಾತನಾಡುವಾಗ, ತಾವೆಂಥ ಟೀಕೆಗಳನ್ನು ಮಾಡುತ್ತಿದ್ದೇವೆಂಬುದನ್ನು ತಿಳಿಯುವುದಿಲ್ಲ.

ಇನ್ನು ಕೆಲವು ಬುದ್ಧಿವಂತರೆನಿಸಿಕೊಂಡವರಲ್ಲಿ ‘ಇತರರ ಒಳ್ಳೆಯ ಗುಣಗಳನ್ನು ಮೆಚ್ಚುವುದೆಂದರೆ ನಾವು ಅವರಿಗಿಂತ ಕೆಳಮಟ್ಟದವರಾಗಬಹುದು’ ಎಂಬ ಪ್ರಬಲವಾದ ಮನೋಭಾವವಿರುತ್ತದೆ. ಇಂಥ ಮನುಷ್ಯರು ಅಸಂಬದ್ಧವಾಗಿಯಾದರೂ ಯಾರನ್ನಾದರೂ ಟೀಕಿಸಿಯಾರೇ ಹೊರತು, ಒಂದೂ ಮೆಚ್ಚಿಗೆಯ ಮಾತನ್ನಾಡಲಾರರು.

ತಾಯಿಗೆ, ತನ್ನ ಮಗ ಉದ್ದವಾಗಿದ್ದು ತೆಳ್ಳಗಿದ್ದರೆ ಬಳ್ಳಿಬಳ್ಳಿಯಾಗಿ ಕಾಣುತ್ತಾನೆ. ಕರ್ರಗಿದ್ದರೂ ಲಕ್ಷಣವಾಗಿ ತೋರುತ್ತಾನೆ. ಆದರೆ ಅಂತಹ ನೆರೆಮನೆಯ ಹುಡುಗ ಆಕೆಯ ಕಣ್ಣಿಗೆ ‘ಅಡಕೆಯ ಮರದಂತೆ ಕೈಕಾಲು, ಮಡಕೆಯ ಮಸಿಯಂತೆ ಮುಖದ ಬಣ್ಣ’ದ್ದಾಗಿ ಕಾಣಿಸುತ್ತಾನೆ. ಒಮ್ಮೆ ಪರಿಚಿತರೊಬ್ಬರೊಡನೆ ಹೇಳಿದೆ: ‘..... ಊರಲ್ಲಿ ಪ್ರವಾಹದಿಂದ ನೂರು ಮಂದಿ ಮಡಿದರಂತೆ.’ ಅವರೆಂದರು ‘ಅಯ್ಯೋ, ವಿಷವೇರಿದಂತೆ ಏರುತ್ತಿರುವ ಜನಸಂಖ್ಯೆ ಕಡಿಮೆ ಮಾಡಲು, ದೇವರು ಏನಾದರೊಂದು ಉಪಾಯ ಮಾಡಬೇಕಲ್ಲ. ಹುಟ್ಟಿದವರು ಒಂದಲ್ಲ ಒಂದು ದಿನ ಸಾಯಬೇಕು.’ ಆದರೆ ಅದೇ ಊರಿನಲ್ಲಿ ಅವರ ಸಂಬಂಧಿಕನೊಬ್ಬನಿದ್ದು, ಅವನೂ ಅಪಘಾತದಲ್ಲಿ ಸಿಕ್ಕಿಬಿದ್ದಾಗ, ‘ಅವನು ಜನಸಂಖ್ಯೆ ಕಡಿಮೆ ಮಾಡಲು ಹೋಗಿದ್ದಾನೆ’–ಎಂದು ಅವರನ್ನು ಸಮಾಧಾನಪಡಿಸಲು ಹೋದರೆ ಏನಾದೀತು?

ಮನುಷ್ಯರು ತಮ್ಮ ಸುಖದುಃಖಗಳ್ನು ಭಾವಪೂರ್ಣವಾಗಿ ಕಾಣುತ್ತಾರೆ. ಇತರರ ಕಷ್ಟಸಂಕಟಗಳನ್ನು ಬುದ್ಧಿಯಿಂದ ಅಳೆಯುತ್ತಾರೆ, ವಿಶ್ಲೇಷಿಸುತ್ತಾರೆ. ಹಲವಾರು ಟೀಕೆಗಳು ಇಂಥ ಜಾತಿಯವು. ಇತರರ ದುಃಖ ಕ್ಲೇಶಗಳನ್ನು ಸಹಾನುಭೂತಿಯಿಂದ ಕಾಣುವ ಗುಣ, ಉತ್ತಮ ಸಂಸ್ಕಾರದ ಫಲ. ಅದು ಎಲ್ಲರಲ್ಲೂ ಕಾಣಲು ಸಿಗುತ್ತದೆಯೇ?

ನೀವು ಯಾವುದೇ ಕೆಲಸವನ್ನು ಎಷ್ಟೇ ಉತ್ತಮವಾಗಿ ಮಾಡಿದರೂ, ಒಂದಲ್ಲ ಒಂದು ತೆರನಾದ ಟೀಕೆ ಬಂದೇ ಬರುತ್ತದೆ. ಟೀಕೆ ನೀವು ಜಾಗ್ರತರಾಗಿರಬೇಕೆನ್ನುವುದಕ್ಕೆ ಎಚ್ಚರಿಕೆಯ ಕರೆಗಂಟೆ ಅಷ್ಟೆ. ಟೀಕೆಯಿಂದ ಸಂಪೂರ್ಣವಾಗಿ ತಪ್ಪಿಸಿಕೊಳ್ಳಬೇಕಾದರೆ ‘ನೀವು ಏನೂ ಹೇಳಬಾರದು, ಏನೂ ಮಾಡಬಾರದು, ಏನೂ ಆಗಬಾರದು’ ಎಂದು ಒಬ್ಬ ಆಂಗ್ಲ ಮೇಧಾವಿ ಹೇಳಿದ.

ಬೈರಾಗಿಯೊಬ್ಬ ಬಯಲಿನಲ್ಲಿ ಮಣ್ಣಿನ ದಿಣ್ಣೆಯ ಮೇಲೆ ತಲೆಯಿರಿಸಿ ಮಲಗಿಕೊಂಡಿದ್ದ. ದಾರಿಗನೊಬ್ಬ ಅದನ್ನು ಕಂಡು ‘ಎಲ್ಲವನ್ನೂ ಬಿಟ್ಟ ಬೈರಾಗಿಗೆ ತಲೆದಿಂಬು ಹೇಗೆ ಬೇಕಾಗುತ್ತದೆಂದು ನೋಡಿ’ ಎಂದು ಹಂಗಿಸಿದ. ಮರುದಿನ ಬೈರಾಗಿ ಬರಿನೆಲದಲ್ಲಿ ಮಲಗಿದ್ದ. ಅವನೇ ಹಿಂತಿರುಗಿ ಬರುತ್ತ ಉದ್ಗರಿಸಿದ: ‘ಅಯ್ಯೋ, ಇವನೆಂಥ ಭೈರಾಗಿ! ಯಾರೇನು ಹೇಳುತ್ತಾರೆಂಬುದರ ಕಡೆಗೇ ಇವನ ಗಮನ!’ ಆದ್ದರಿಂದಲೇ ದಾಸರ ಆಲಾಪ, ‘ಹರನೇ ನಿನ್ನನ್ನು ಒಲಿಸಲುಬಹುದು, ನರರನು ಒಲಿಸಲು ಬಲು ಕಷ್ಟ!’

‘ಮೌನದಿಂದಿರುವವರನ್ನೂ ಅವರು ನಿಂದಿಸುತ್ತಾರೆ. ಮಾತುಗಾರರನ್ನೂ ನಿಂದಿಸುತ್ತಾರೆ, ಮಿತಭಾಷಿಗಳನ್ನೂ ನಿಂದಿಸುತ್ತಾರೆ. ಲೋಕದಲ್ಲಿ ನಿಂದೆಯನ್ನು ಕೇಳದವರು ಯಾರೂ ಇಲ್ಲ’– ಹೀಗೆಂದು ಗೌತಮ ಬುದ್ಧ ಸಾರಿದ.

ಟೀಕೆಗಳಿಂದ ಸಂಪೂರ್ಣವಾಗಿ ತಪ್ಪಿಸಿಕೊಂಡವರು ಯಾರೂ ಇಲ್ಲ. ಮಹಾವಿಜ್ಞಾನಿ ಐನ್‍ಸ್ಟೀನರ ಸಾಪೇಕ್ಷತಾವಾದ ಪ್ರಕಟವಾದಾಗ ಅದೊಂದು ಹುಚ್ಚುವಾದವೆಂದು ವಿಜ್ಞಾನಿಗಳೇ ಗೇಲಿ ಮಾಡಿದ್ದರು! ಟೀಕಾಸ್ತ್ರಕ್ಕೆ ಹೃದಯವೇ ಇಲ್ಲ!

ಹೇಳುವವರು ಬೇಕಷ್ಟು ಹೇಳಲಿ. ಅದು ಯಥಾರ್ಥವಲ್ಲವೆಂದು ನಿಮಗೆ ಖಚಿತವಾದರೆ ಅದರ ಕಡೆಗೇನೂ ಗಮನವೀಯಬೇಕಾಗಿಲ್ಲ. ಅವರ ಮಾತಿಗೆ ‘ಕಿವುಡನ ಮಾಡಯ್ಯ ತಂದೆ’ ಎಂದು ಸರ್ವಶಕ್ತನ ಮೊರೆ ಹೋಗಿ.


\section*{ಸ್ವವಿಮರ್ಶೆಯ ನಿಕಷ}

\addsectiontoTOC{ಸ್ವವಿಮರ್ಶೆಯ ನಿಕಷ}

‘ಪ್ರತಿಯೊಬ್ಬ ಮನುಷ್ಯನೂ ದಿನದಲ್ಲಿ ಕನಿಷ್ಠ ಪಕ್ಷ ಐದು ನಿಮಿಷಗಳ ಕಾಲವಾದರೂ ಮೂರ್ಖ\-ನಾಗಿರು\-ತ್ತಾನೆ’ ಎಂದು ಹಬ್ಬರ್ಡ್ ಹೇಳಿದ. ನೀವೆಷ್ಟು ಬುದ್ಧಿವಂತರಾದರೂ, ಭಾಗ್ಯವಂತರಾದರೂ, ದೋಷರಹಿತರಲ್ಲ; ಪರಿಪೂರ್ಣರಲ್ಲವೆಂಬುದೇನೋ ಖಂಡಿತ. ನೀವೆಷ್ಟೇ ಪ್ರತಿಭಾಶಾಲಿ\-ಗಳಾಗಿ\-ದ್ದರೂ, ಅನುಭವಿಗಳಾಗಿದ್ದರೂ, ತಪ್ಪಿಬೀಳುವ ಸಂದರ್ಭಗಳು ನಿಮಗೂ ಇವೆ–ಅಂಥ ಅನೇಕ ಸನ್ನಿವೇಶಗಳಲ್ಲಿ ನೀವು ಮಾಡಿದ ತಪ್ಪು ನಿಮಗೆ ತಿಳಿಯದೇ ಹೋಗಬಹುದು. ತಿಳಿದರೂ ನೀವದಕ್ಕೆ ಮಹತ್ವ ಕೊಡದೆ ತಪ್ಪಿಸಿಕೊಳ್ಳಬಹುದು ಅಥವಾ ಆ ತಪ್ಪಿನಿಂದ ಉಂಟಾದ ಕಹಿಫಲವನ್ನು ಅನುಭವಿಸಿದರೂ ಅದನ್ನು ಅಲಕ್ಷಿಸಿರಬಹುದು. ಆ ಕುರಿತು ಇತರರಿಂದ ಟೀಕೆ ಬರಬಹುದು–ಇದು ವಾಸ್ತವಿಕ ಹಾಗೂ ರಚನಾತ್ಮಕ ಟೀಕೆಯಾದರೆ ಅದನ್ನು ನೀವೇಕೆ ಅಲಕ್ಷಿಸಬೇಕು?

‘ನಮ್ಮನ್ನು ಕುರಿತು ನಮಗಿರುವ ಅಭಿಪ್ರಾಯಗಳಿಗಿಂತಲೂ ನಮ್ಮ ಶತ್ರುಗಳ ಅಭಿಪ್ರಾಯವೇ ಸತ್ಯಕ್ಕೆ ಹತ್ತಿರವಾಗಿರುತ್ತದೆ’ ಎಂದು ಒಬ್ಬ ಜ್ಞಾನಿಯ ಮಾತು. ನಮ್ಮ ಬೆನ್ನು ನಮಗೆ ಕಾಣದು. ಹಾಗೆಯೇ ನಮ್ಮ ದೋಷ ನಮಗೆ ಸುಲಭವಾಗಿ ತಿಳಿಯುವುದಿಲ್ಲ. ನಮ್ಮ ದೌರ್ಬಲ್ಯಗಳ ಅರಿವು ಎಷ್ಟೋ ಬಾರಿ ನಮಗಾಗುವುದಿಲ್ಲ. ಅದುದರಿಂದ ಇತರರ ಯಥಾರ್ಥ ವಿಮರ್ಶೆಯು ನಮಗೆ ಕಲ್ಪನಾತ್ಮಕವಾದ ಟೀಕೆಯಾಗಿ ತೋರಿ ನಮ್ಮನ್ನು ಮೋಸಗೊಳಿಸಬಲ್ಲದು. ಆದುದರಿಂದ ಟೀಕೆಯನ್ನು ಸಂಪೂರ್ಣವಾಗಿ ಅಲಕ್ಷಿಸುವುದು ಒಳಿತಲ್ಲ. ಅದರಲ್ಲೇನೂ ಹುರುಳಿಲ್ಲವೆಂದಾದರೆ ಅಲಕ್ಷಿಸುವುದು ಉಚಿತ.

ಟೀಕೆಗಳಲ್ಲಿ ಹೆಚ್ಚಿನಂಶವನ್ನು ದೂರಮಾಡುವ ಉಪಾಯ ಒಂದಿದೆ–ಅದೇ ನೀವೇ ನಿಮ್ಮ ಅತ್ಯಂತ ಕಟು ವಿಮರ್ಶಕರಾಗುವುದು.

ಇಷ್ಟೇ ಪ್ರಮುಖವಾದ ಇನ್ನೊಂದು ಸಂಗತಿಯನ್ನು ಮರೆಯಬೇಡಿ–ಟೀಕಾಸುರರನ್ನು\break ಪುನಃ ಟೀಕಿಸಿ ಬಾಯಿ ಮುಚ್ಚಿಸಲು ಸಾಧ್ಯವಿಲ್ಲ. ನಮ್ಮ ಗೆಳೆಯರನ್ನು ನೋಡುವಂತೆಯೇ ಅವರನ್ನು ಉದಾರ ಹೃದಯದಿಂದ, ಒಳ್ಳೆಯ ಭಾವನೆಯಿಂದ ನೋಡುತ್ತೀರಿ. ನೀವು \hbox\bgroup ಅವರಲ್ಲಿ\egroup\break ವಿಶ್ವಾಸ ವಿಟ್ಟಿರುವಿರೆಂದು ಅವರು ತಿಳಿದರೆ, ಅವರ ನಾಲಗೆಯ ತೀಕ್ಷ್ಣತೆ ತನ್ನಿಂದ ತಾನೇ\break ಕಡಿಮೆ\-ಯಾಗುತ್ತದೆ!

ನಿಮಗೆ ಜನರ ಹೊಗಳಿಕೆಯೇ ಬೇಕೇನು? ಗ್ರಂಥಕರ್ತರೊಬ್ಬರು ಹೇಳಿದ ಮಾತು: ‘ಹತ್ತಾರು ವರ್ಷಗಳ ಕಾಲ ಶ್ರಮಪಟ್ಟು ಬರೆದ ಪುಸ್ತಕವನ್ನು ಐದು ನಿಮಿಷದಲ್ಲಿ ಮಗುಚಿ ಹಾಕಿ “ನೀವು ಬರೆದ ಪುಸ್ತಕ ಚೆನ್ನಾಗಿದೆ” ಎಂದು ಹೇಳುತ್ತಾರೆ. ಅಂಥ ಹೊಗಳಿಕೆ ಕೇಳಿ ಸಾಕಾಗಿದೆ.’ ಕೆಲವರು ನಿಮ್ಮ ಮೇಲಿನ ಗೌರವದಿಂದ ನಿಮ್ಮಲಿಲ್ಲದ ಗುಣಗಳನ್ನು ಕಲ್ಪಿಸಿಕೊಂಡು ನಿಮ್ಮನ್ನು ಹೊಗಳಬಹುದು. ಇತರರು ನಿಮ್ಮ ಇದಿರು ಹೊಗಳಿ ಹಿಂದಿನಿಂದ ‘ನೋಡಿ, ನೋಡಿ, ಎರಡು ಹೊಗಳಿಕೆಯ ಮಾತನ್ನು ಹೇಳಿದೆ–ಅಟ್ಟಕ್ಕೇರಿದ. ಹ್ಹ!’ ಎನ್ನುತ್ತಾರೆ. ಇನ್ನು ಕೆಲವರು ತಮ್ಮ ಸ್ವಾರ್ಥಸಾಧನೆಯಾಗುವ ತನಕ ನಿಮ್ಮನ್ನು ಒಂದೇಸಮನೆ ಹೊಗಳುತ್ತಾರೆ. ನಿಮ್ಮನ್ನು ನಿಜವಾಗಿ ಅರಿತು, ಅಭಿಪ್ರಾಯಗಳನ್ನು ಕೊಡಲು ಯತ್ನಿಸುವವರೆಷ್ಟು ಮಂದಿ? ಎಂದು ಯೋಚಿಸಿದಾಗ, ಹೊಗಳಿಕೆಯ ಕೈಗೊಂಬೆ ನೀವಾಗಲಾರಿರಿ.

ಸ್ವಾಮಿ ರಾಮತೀರ್ಥರನ್ನು ಕಾಣಲು ಬಂದ ಒಬ್ಬನು ಹೀಗೆಂದ: ‘ಜನರು ನಿಮ್ಮನ್ನು ಮೆಚ್ಚುತ್ತಿಲ್ಲ....’

ಸ್ವಾಮಿ ರಾಮತೀರ್ಥರು ಉತ್ತರಿಸಿದರು: ‘ಅವರು ಸೇಬು ಹಣ್ಣನ್ನು ಮೆಚ್ಚಿದಾಗ ಅದನ್ನು ತಿನ್ನುತ್ತಾರೆ. ಪ್ಲಂ ಹಣ್ಣನ್ನು ಮೆಚ್ಚಿದಾಗಲೂ ಅದನ್ನು ತಿನ್ನುತ್ತಾರೆ. ಮಾಂಸ, ಕರುಳು, ಮಿದುಳು ಇವನ್ನು ಮೆಚ್ಚಿದಾಗ ಅವನ್ನೂ ನುಂಗುತ್ತಾರೆ. ಅವರು ನನ್ನನ್ನು ಮೆಚ್ಚದಿದ್ದುದು ನನ್ನ ಭಾಗ್ಯವೇ ಸರಿ. ಮೆಚ್ಚಿದರೆ ನನ್ನನ್ನೂ ಕಬಳಿಸುತ್ತಿದ್ದರು!’

ಜನರ ಪೊಳ್ಳು ಹೊಗಳಿಕೆಗೆ ದಾಸರಾದರೆ ನಾವು, ನಮ್ಮ ಗುರಿಯಿಂದ ಜಾರುತ್ತೇವೆ. ಆ ಹೊಗಳಿಕೆಯನ್ನು ಉಳಿಸಿಕೊಳ್ಳುವ ಹಂಬಲ ಹೆಚ್ಚುತ್ತದೆ. ಆಗ ನಮ್ಮ ಬದುಕು ನಮ್ಮ ಅಧೀನವಿರದು–ನಮ್ಮ ಸಮಯವೂ ನಮ್ಮದಾಗದು.

ಎಮರ್ಸನ್ ಹೇಳಿದಂತೆ ‘ಜಗತ್ತು ಏನು ಹೇಳೀತು? ಜನರು ಏನು ಹೇಳಿಯಾರು? ಎಂದು ಯೋಚಿಸದೆ ತನ್ನ ಕಾರ್ಯದಲ್ಲಿಯೇ ಮಗ್ನನಾಗಿ, ಆ ಕಾರ್ಯದ ಯಶಸ್ಸಿನತ್ತ ಲಕ್ಷ್ಯ ಪೂರೈಸುವವನೇ ಎಲ್ಲರಿಗಿಂತ ಸುಖಿಯಾಗಿರುತ್ತಾನೆ.’


\section*{ದಕ್ಷತೆಯಿಂದ ಧೈರ್ಯ}

\addsectiontoTOC{ದಕ್ಷತೆಯಿಂದ\break ಧೈರ್ಯ}

ಜೀವವಿಮಾ ನಿಗಮದ ಹಿರಿಯ ಅಧಿಕಾರಿಯೊಬ್ಬರು ತಮ್ಮ ದೀರ್ಘಸೇವಾ ಜೀವನದಲ್ಲಿ\break ನೂರಾರು ನೌಕರರನ್ನು ಪ್ರೀತಿ, ವಿಶ್ವಾಸಗಳ ಮೋಡಿಯಿಂದ ಹೇಗೆ ನಿಯಂತ್ರಣದಲ್ಲಿಟ್ಟುಕೊಂಡೆ ಎಂಬುದನ್ನು ಹೀಗೆ ವಿವರಿಸಿದರು–

‘ಯಾವುದೇ ಆಫೀಸಿನಲ್ಲಿರುವಂತೆ ನಮ್ಮ ಆಫೀಸಿನಲ್ಲಿಯೂ ಮೂರು ಬಗೆಯ ನೌಕರರಿದ್ದರು. ಒಂದು: ಕರ್ತವ್ಯನಿಷ್ಠರೂ, ಪ್ರಾಮಾಣಿಕರೂ, ದಕ್ಷರೂ, ವಿಧೇಯರೂ ಆಗಿದ್ದ ಎರಡು ಮೂರು ಮಂದಿ. ಎರಡು: ಇದಕ್ಕೆ ತದ್ವಿರುದ್ಧವಾದ ಪ್ರವೃತ್ತಿಯುಳ್ಳ, ಅಂದರೆ ಶಿಸ್ತುಗೇಡಿಗಳೂ, ವಿಧ್ವಂಸಕ ಪ್ರವೃತ್ತಿಯುಳ್ಳವರೂ, ಕಿಡಿಗೇಡಿಗಳೂ, ಕರ್ತವ್ಯದಲ್ಲಿ ಅನಾಸಕ್ತರೂ ಆದ ಎರಡು ಮೂರು ಮಂದಿ. ಉಳಿದವರು, ಈ ಎರಡರ ಮಧ್ಯೆ ಬರತಕ್ಕ ಮಧ್ಯಮವರ್ಗೀಯರು ಎನ್ನಬಹುದು. ಇವರು ಒಮ್ಮೆ ಹಾಗೂ, ಒಮ್ಮೆ ಹೀಗೂ, ವಾಲುವವರು. ಮನಸ್ಸು ಮಾಡಿದರೆ, ಒಳ್ಳೆಯ ಕೆಲಸ ಮಾಡಬಲ್ಲ ಚಂಚಲ ಪ್ರವೃತ್ತಿಯವರು. ಇವರಲ್ಲಿ ಎಲ್ಲರಿಗೂ ಒಂದೇ ಚಿಕಿತ್ಸೆ ಪ್ರಯೋಜನಕಾರಿಯಾಗಲಾರದು. ಏಕೆಂದರೆ, ಪ್ರತಿಯೊಬ್ಬರ ರೋಗವಿಧಾನವೂ ಬೇರೆ. ಹೀಗಾಗಿ ಇವರಲ್ಲಿ ಪ್ರತಿಯೊಬ್ಬರ ಹಿನ್ನೆಲೆಯನ್ನೂ, ಸ್ವಭಾವವನ್ನೂ ಅಭ್ಯಸಿಸಿ, ಅವರವರ ವರ್ತನೆಗೆ ಕಾರಣಗಳನ್ನು ಶೋಧಿಸಬೇಕಾಯಿತು. ಇಂತಹ ಕಾರಣಗಳೂ ಅನೇಕ. ಕೌಟುಂಬಿಕ ಸಮಸ್ಯೆಗಳು, ಆಫೀಸಿನಲ್ಲಿ ಶೀಘ್ರ ಉನ್ನತಿಗೇರಲು ಅವಕಾಶವಿಲ್ಲವೆಂಬ ನಿರಾಸೆ, ತಮಗೆ ಬೇಕಾದ ಊರಿಗೆ ವರ್ಗ ಸಿಗಲಿಲ್ಲವೆಂದು ಬೇಸರ, ಹಿಂದಿನ ಆಫೀಸರರೊಂದಿಗೆ ನಡೆದ ಜಗಳದಿಂದುಂಟಾಗಿದ್ದ ಕಹಿ, ಸೋಮಾರಿತನ ಚಟವಾಗಿರುವುದು, ಲೀಡರ್ ಎನಿಸಿಕೊಳ್ಳುವ ಆಕಾಂಕ್ಷೆ, ಬರೀ ಹುಡುಗಾಟಿಕೆಯ ಕಿಡಿಗೇಡಿತನ–ಇವೆಲ್ಲವೂ ಒಂದಲ್ಲ ಒಂದು ಪ್ರಮಾಣದಲ್ಲಿ ಹೆಚ್ಚಿನವರ ಮನಸ್ಸನ್ನು ಕಾಡುತ್ತಿದ್ದ ಸಮಸ್ಯೆಗಳು. ಈ ಸಮಸ್ಯೆಗಳಲ್ಲಿ\break ಯಾವುದನ್ನು ಪರಿಹರಿಸುವುದೂ ನನ್ನ ಶಕ್ತಿಗೆ ಮೀರಿದ ವಿಷಯವಾದುದರಿಂದ, ಅವುಗಳ ಪ್ರಭಾವ ಕಡಿಮೆಯಾಗುವಂತೆ ಮಾಡಿ ಕೆಲಸ ಸಾಗುವಂತೆ ಮಾಡುವುದೊಂದೇ ನಾನು ಸಾಧಿಸಬೇಕಾದ ಕಾರ್ಯವಾಗಿತ್ತು.

‘ಕರ್ತವ್ಯನಿಷ್ಠರೆಂದು ಹೇಳಿದ ಇಬ್ಬರು ಮೂವರ ಬಗ್ಗೆ ನಾನು ಹೆಚ್ಚೇನೂ ಮಾಡಬೇಕಾಗಿರಲಿಲ್ಲ. ಅವರ ಕಾರ್ಯದಕ್ಷತೆಗೆ ಆಗಾಗ ಮೆಚ್ಚುಗೆ ವ್ಯಕ್ತಪಡಿಸುತ್ತಾ, ಅವರ ಬಗ್ಗೆ ನನಗಿರುವ ಪ್ರೀತಿ, ವಿಶ್ವಾಸ, ಅಭಿಮಾನವನ್ನು ತೋರ್ಪಡಿಸುತ್ತಾ ಇರುತ್ತಿದ್ದೆ. ಹಾಗೆ ಮಾಡುವಾಗ ಅದು ನಾಟಕೀಯವಾಗದಂತೆ ಜಾಗ್ರತೆ ವಹಿಸುತ್ತಿದ್ದೆ. ಆಫೀಸಿನ ವಾತಾವರಣ ಅತಿ ಸೂಕ್ಷ್ಮ. ಒಬ್ಬ ಒಳ್ಳೆಯ ಕೆಲಸಗಾರರನ್ನು ಹೊಗಳಿದರೆ ಉಳಿದವರು ಅಸೂಯೆಗೊಂಡು ತಮ್ಮ ಕೆಲಸವನ್ನು ಇನ್ನೂ ಕಡಿಮೆ ಮಾಡುವ ಅಪಾಯ ಸದಾ ಇತ್ತು. ಆದ್ದರಿಂದ ಮೆಚ್ಚಿಗೆಯನ್ನು ವ್ಯಕ್ತಪಡಿಸುವಾಗಲೂ, ಅತಿ ಜಾಗರೂಕತೆಯಿಂದ ಇರಬೇಕಾಗುತ್ತಿತ್ತು. ಹಾಗಾಗಿ ಅನೇಕ ಬಾರಿ ಮೆಚ್ಚುಗೆಯನ್ನು ವೈಯಕ್ತಿಕವಾಗಿ, ಏಕಾಂತದಲ್ಲಿ ವ್ಯಕ್ತಪಡಿಸುತ್ತಿದ್ದೆ. ಈ ಮೆಚ್ಚಿಕೆ ನಾಟಕವಲ್ಲ ಎಂದು ತೋರಿಸಲು ತಕ್ಕ ದಾಖಲೆಗಳನ್ನು ಸಾಧ್ಯವಾದಾಗ ನಿರ್ಮಿಸುತ್ತಿದ್ದೆ.

‘ಇನ್ನೊಂದು ವರ್ಗಕ್ಕೆ ಸೇರಿದ ಎರಡು, ಮೂರು ಮಂದಿ ಸಾಮಾನ್ಯವಾದ ಯಾವ ಅಸ್ತ್ರಕ್ಕೂ ಬಾಗುವವರಲ್ಲ. ಇವರು ಆಫೀಸರನ್ನು, ವರ್ಗಶತ್ರು(ಕ್ಲಾಸ್​ಎನಿಮಿ)ವಾಗಿ ನೋಡುವ ದೃಷ್ಟಿಯನ್ನು ಬೆಳೆಸಿಕೊಂಡ ಯೂನಿಯನ್ ಲೀಡರ್​ಗಳು. ಇವರು ಬುದ್ಧಿವಂತರೂ, ಹೃದಯವಂತರೂ ಆದರೂ, ತಾವು ಬೆಳೆಸಿಕೊಂಡಿರುವ “ಇಮೇಜಿ”ಗೆ ಬದ್ಧರಾಗಿ, ಅದರಿಂದ ತಪ್ಪಿಸಿಕೊಳ್ಳಲಾರದೆ, ಕೆಲವೊಮ್ಮೆ ಒಳ್ಳೆಯ ಕೆಲಸ ಮಾಡಬೇಕೆಂದು ಅಂತರಾತ್ಮ ಹೇಳಿದರೂ, ಅದನ್ನು ಅದುಮಿ ಹಿಡಿ\-ಯಲು ಪ್ರಯತ್ನಿಸುತ್ತಿರುವವರೆಂಬುದನ್ನು ಕಂಡುಹಿಡಿದೆ. ಯೂನಿಯನ್ ಲೀಡರ್‌ಗಳ ಸಮಸ್ಯೆಗಳನ್ನು ನಾನು ಬಲ್ಲವನಾದ್ದರಿಂದ ಅವರನ್ನೂ ಮಾನವೀಯ ದೃಷ್ಟಿಯಿಂದ, ಸಹಾನುಭೂತಿ ಸೌಹಾರ್ದ ಮತ್ತು ಔದಾರ್ಯದಿಂದ ನೋಡುತ್ತಾ, ಅದೇ ರೀತಿಯಲ್ಲಿ ವ್ಯವಹರಿಸುತ್ತಾ ಬಂದೆ. ಮುಖ್ಯವಾಗಿ ಇದು ನನ್ನ ಸ್ವಭಾವಕ್ಕೆ ಅನುಗುಣವಾದದ್ದು. ಪ್ರತಿಯೊಬ್ಬನ ಮನಸ್ಸಿಗೂ ತನ್ನದೇ ಆದ ಸ್ವಭಾವ, ಸಾಮರ್ಥ್ಯ, ಪರಿಮಿತಿಗಳು ಇರುತ್ತವೆ. ಕೊನೆಗೂ ಗೆಲ್ಲುವುದು ಅದೇ. ಆದ ಕಾರಣ ಅದಕ್ಕೆ ಅನುಗುಣವಾದ ಧೋರಣೆಯನ್ನು ಪ್ರತಿಯೊಬ್ಬನೂ ರೂಪಿಸಿಕೊಳ್ಳಬೇಕೆ ಹೊರತು ಅನುಕರಣೆಯ ಧೋರಣೆಯನ್ನಲ್ಲ. ನನ್ನದು ಪ್ರಾಮಾಣಿಕತೆಯ ಮಾನವೀಯತೆಯ ದಾರಿ. ಇದರ ಪರಿಣಾಮ ಸ್ವಲ್ಪ ತಡವಾದರೂ ಸ್ಥಾಯಿಯಾಗಿ ಉಳಿಯುವಂಥದು. ಬೇರೆ ರೀತಿಯ ತಂತ್ರಗಾರಿಕೆ ನನ್ನಿಂದ ಅಸಾಧ್ಯ. ನನಗೆ ತಾತ್ಕಾಲಿಕ ಪವಾಡ ಪ್ರಯೋಜನವಿಲ್ಲ. ಒಂದೆರೆಡು ಗಂಟೆಗಳ ಕಾಲ ಒಬ್ಬರನ್ನು ಪ್ರಭಾವಗೊಳಿಸಿ, ಕೆಲಸ ಮಾಡಿಸಿಕೊಳ್ಳುವ ತಂತ್ರ ನನ್ನ ಕೆಲಸಕ್ಕೆ ಸಾಲದು. ಏಕೆಂದರೆ ನನ್ನ ನೌಕರರು ನನ್ನ ಜೊತೆಗೆ ನಿತ್ಯವೂ ಮೂರು ನಾಲ್ಕು ವರ್ಷಗಳ ಕಾಲವಾದರೂ ಕೆಲಸ ಮಾಡಬೇಕು. ಆದ್ದರಿಂದ ಇದು ದೀರ್ಘಾವಧಿಯ ಸಂಬಂಧ (ಲಾಂಗ್ ಟರ್ಮ್ ರಿಲೇಷನ್​\-ಷಿಪ್​). ಇದಕ್ಕೆ ತಕ್ಕ ಮಾರ್ಗವನ್ನು ನಾನು ಶೋಧಿಸಿಕೊಳ್ಳಬೇಕಾಯಿತು. ಅಲ್ಲದೆ ನಮ್ಮ ಸಂಸ್ಥೆಯಲ್ಲಿ ಅಧಿಕಾರಿಗಳಿಗೆ ಕಾನೂನುಬದ್ಧ ಅಧಿಕಾರಗಳು ಸಾಕಷ್ಟಿದ್ದರೂ, ಅವೆಲ್ಲಾ ಪುಸ್ತಕದ ಬದನೇಕಾಯಿಗಳು. ಅವುಗಳನ್ನು ನೌಕರರ ಮೇಲೆ ಜಾರಿ ಮಾಡುವುದು ಬಹುಮಟ್ಟಿಗೆ ಅಸಾಧ್ಯ. ಏಕೆಂದರೆ, ಹಾಗೆ ಮಾಡಿದರೆ ಯೂನಿಯನ್ನುಗಳ ಪ್ರಬಲ ವಿರೋಧವನ್ನು ಎದುರಿಸಬೇಕಾಗುತ್ತಿತ್ತು. ಅಲ್ಲದೆ, ಕೇವಲ ಕಾನೂನಿನ ಒತ್ತಡದಿಂದ ಯಾವ ಕೆಲಸವನ್ನೂ, ಸಮರ್ಪಕವಾಗಿ ಮಾಡಿಸಲು ಸಾಧ್ಯವಿಲ್ಲ ಎನ್ನುವ ಸತ್ಯವೂ ನನಗೆ ಅನೇಕ ಬಾರಿ ಮನದಟ್ಟಾಗಿತ್ತು. ಈ ಎಲ್ಲಾ ದೃಷ್ಟಿಗಳಿಂದ, ಮಾನವೀಯತೆಯ–ಪ್ರಾಮಾಣಿಕತೆಯ ಮಾರ್ಗವೊಂದೇ ನನ್ನ ಪಾಲಿನ ರಾಜ\-ಮಾರ್ಗ\-ವಾಯಿತು.

‘ಆರಂಭದ ಒಂದೆರಡು ತಿಂಗಳುಗಳಲ್ಲಿ ಈ ಕೆಲವು ಮಂದಿ ನನ್ನ ಮೇಲೆ ಸಂಶಯವನ್ನು ಉಗ್ರವಾಗಿ ವ್ಯಕ್ತಪಡಿಸಿದ್ದರು. ನಾನು ಅದನ್ನು ಆದಷ್ಟೂ ತಾಳ್ಮೆಯಿಂದ ಸಹಿಸಿದೆ. ಅನಿವಾರ್ಯ\-ವಾದಾಗ ಸರಿಯಾದ ತರ್ಕದಿಂದ ಉತ್ತರಿಸಿದೆ. ನನ್ನ ವರ್ತನೆಯಲ್ಲಿ ನನಗೇ ತಿಳಿಯದ ತಪ್ಪುಗಳಾಗಿದ್ದರೆ ಅವನ್ನು ಪ್ರಾಂಜಲವಾಗಿ ಒಪ್ಪಿಕೊಂಡೆ. ಸರಿಪಡಿಸಲು ಪ್ರಾಮಾಣಿಕವಾಗಿ ಪ್ರಯತ್ನಿಸ ತೊಡಗಿದೆ. ಕೆಲಸದಲ್ಲಿ ಅಪಾರ ಶ್ರದ್ಧೆ ಬೆಳೆಸಿಕೊಂಡೆ. ಆ ದಿನಗಳಲ್ಲಿ ಬೆಳಿಗ್ಗೆ ೯–೩೦ರಿಂದ ರಾತ್ರಿ ೯ರವರೆಗೆ ದಿನವೂ ದುಡಿಯುತ್ತಿದ್ದೆ. ಅಷ್ಟು ಕೆಲಸವಿತ್ತು! ಮುಖ್ಯವಾಗಿ ನಾನು ನನ್ನ ವ್ಯಕ್ತಿತ್ವದ ವಿಶ್ವಾಸಾರ್ಹತೆ(ಕ್ರಿಡಿಬಿಲಿಡಿ)ಯನ್ನು ಸ್ಥಾಪಿಸಿಕೊಳ್ಳುವುದಕ್ಕೆ ಇದು ಅಗತ್ಯವಾಗಿತ್ತು. ಹಾಗೆಯೇ ಕೆಲಸ ಮಾಡುತ್ತಾ ಹೋದಂತೆ, ಕೆಲಸದಲ್ಲಿ ತಕ್ಕಷ್ಟು ಜ್ಞಾನ ಉಂಟಾಯಿತಲ್ಲದೆ ಆತ್ಮವಿಶ್ವಾಸ, ನೈತಿಕ ಧೈರ್ಯ, ಆಶ್ಚರ್ಯಕರವಾಗಿ ಬೆಳೆಯಿತು. ಇಡೀ ಆಫೀಸಿನಲ್ಲಿ ಎಲ್ಲರಿಗಿಂತ ಹೆಚ್ಚುಕಾಲ, ಎಲ್ಲರಿಗಿಂತ ಹೆಚ್ಚು ಪ್ರಮಾಣದ ಕೆಲಸ ಮಾಡುವವನೇ ನಾನಾದ್ದರಿಂದ ಸಹಜವಾಗಿಯೇ ನನ್ನ ಒಂದು ಶ್ರೇಷ್ಠತೆ (ಸುಪೀರಿಯಾರಿಟಿ) ನಿಧಾನವಾಗಿ ಪ್ರಾಪ್ತವಾಗತೊಡಗಿತು. ಇನ್ನೊಬ್ಬರಿಗೆ ಕೆಲಸ ಹೇಳಲು ಒಂದು ಬಗೆಯ ನೈತಿಕಶಕ್ತಿ ನನ್ನಲ್ಲುಂಟಾಯಿತು. ಈ ಶಕ್ತಿಯೊಂದಿಗೆ ನಾನು ಬೆಳೆಸಿಕೊಂಡ–ತುಂಬ ಕಷ್ಟಪಟ್ಟು ಸಾಧಿಸಿದ ಇನ್ನೊಂದು ಶಕ್ತಿ: ಸಹನೆ. ಉದ್ವೇಗದ ಮೇಲಿನ ಸಂಯಮ ನನಗೆ ಮಹಾನ್ ಸವಾಲಾಗಿತ್ತು. ಆದರೆ ಉದ್ವೇಗವನ್ನು, ಅಭಿವ್ಯಕ್ತಿಯ ಆಕಾಂಕ್ಷೆಯನ್ನು ತಡೆಹಿಡಿಯದೆ, ಒಂದು ಸಣ್ಣ ಸಮಸ್ಯೆ ಎಷ್ಟು ದೊಡ್ಡದಾಗಿ ಬೆಳೆಯಬಹುದು ಎಂಬುದನ್ನು ನಾನು ಒಂದೆರಡು ಬಾರಿ ಅನುಭವಿಸಿದ ನಂತರ ಸಹನೆ–ಸಂಯಮಗಳ ಸಾಧನೆಯನ್ನು ತಪಸ್ಸಿನಂತೆ ಮಾಡಲು ನಿರ್ಧರಿಸಿದೆ. ಆರಂಭದಲ್ಲಿ ಇದು ತೀವ್ರ ಮಾನಸಿಕ ಹಿಂಸೆಯನ್ನು ನೀಡಿತ್ತು. ಆದರೆ ಮಾತಿನ ಚಪಲ ನನ್ನ ಕಾರ್ಯಕ್ಷೇತ್ರದಲ್ಲಿ ಹೇಗೆ ಕೆಲಸ ಕೆಡಿಸಬಲ್ಲದೆನ್ನುವುದು ಚೆನ್ನಾಗಿ ಅನುಭವಕ್ಕೆ ಬಂದದ್ದರಿಂದ ಕ್ರಮೇಣ ಸಂಯಮ ಸಾಧ್ಯವಾಯಿತು.

‘(ಇನ್ಶೂರೆನ್ಸ್​) ಏಜೆಂಟರನ್ನು ಉದ್ಯಮದ ಬೆನ್ನೆಲುಬು ಎಂದು ಹೊಗಳುವುದು ರೂಢಿ. ಕೆಲವು ಏಜೆಂಟರು ತುಂಬಾ ಜಗಳಗಂಟರು. ಅವರಿಗೆ ತಾವು ಹೇಳಿದ ಕೆಲವು ಕ್ಷಣಮಾತ್ರದಲ್ಲಾಗಬೇಕು. ಆಗದಿದ್ದರೆ ಗಲಾಟೆ, ಕಳ್ಳ ಅರ್ಜಿ, ಮೇಲಧಿಕಾರಿಗೆ ದೂರು–ಇತ್ಯಾದಿಗಳನ್ನು ಪ್ರಯೋಗಿಸದೆ ಬಿಡುತ್ತಿರಲಿಲ್ಲ. ಅಂಥ ಒಬ್ಬ ಏಜೆಂಟ್, ಒಮ್ಮೆ ಆಫೀಸಿನಲ್ಲಿ ಕಾನೂನು ಪ್ರಕಾರ ಸಾಧ್ಯವಾಗದ ಒಂದು ಕೆಲಸವನ್ನು ಅಪೇಕ್ಷಿಸಿದ. ಅದಕ್ಕೆ ಸಂಬಂಧಪಟ್ಟ ನೌಕರ ಆಗುವುದಿಲ್ಲವೆಂದ. ಏಜೆಂಟ್ ಉದ್ವಿಗ್ನನಾದ. ಸಂಸ್ಥೆಯನ್ನೂ, ನಮ್ಮನ್ನೂ ಟೀಕಿಸುತ್ತಾ “ಹಾಗಾದರೆ ಏಜೆಂಟರನ್ನು ಉದ್ಯಮದ ಬೆನ್ನೆಲುಬು ಅನ್ನುವುದೇಕೆ?” ಎಂದು ಸವಾಲು ಹಾಕಿದ. ಆಗ ನಾನು–ನನ್ನ ನೌಕರರನ್ನು ರಕ್ಷಿಸುವುದಕ್ಕಾಗಿ, \enginline{“Backbone back}ನಲ್ಲಿಯೇ ಇರಬೇಕು. \enginline{Front}ಗೆ ಬರಬಾರದು” ಎಂದುತ್ತರಿ\-ಸಿದೆ. ಇಡೀ ಆಫೀಸು ಗೊಳ್ಳೆಂದು ನಕ್ಕಿತು! ಏಜೆಂಟ್ ಮುಖಭಂಗಗೊಂಡಂತಾಗಿ ಮರಳಿದ. ನೌಕರರಿಗೇನೋ ಖುಷಿಯಾಗಿತ್ತು. ಆದರೆ ಅನಂತರ ಆತ ಮೇಲಧಿಕಾರಿಗಳಿಗೆ ನನ್ನ ಬಗ್ಗೆ ದೂರಿತ್ತನೆಂದು ಗೊತ್ತಾಯಿತು. ಮೇಲಧಿಕಾರಿಗಳಿಗೆ ನನ್ನ ಕಾರ್ಯದಕ್ಷತೆಯ ಬಗ್ಗೆ ವಿಶ್ವಾಸ ವಿದ್ದುದರಿಂದ, ಅವರು ತಪ್ಪು ತಿಳಿವಳಿಕೆಗೊಳಗಾಗಲಿಲ್ಲ. ಅವರೂ ಅದನ್ನು ತೇಲಿಸಿಬಿಟ್ಟರು. ಇದೊಂದಂತಲ್ಲ, ಇಂತಹ ಇನ್ನೂ ಕೆಲವು ಪ್ರಕರಣಗಳಲ್ಲೂ ಹಾಗೆಯೇ ಆಯಿತು!’

ಪ್ರೀತಿಗೆ ವರ್ಗಭೇದವಿಲ್ಲ. ಪರಿಶುದ್ಧ ಪ್ರೀತಿ ಕೈ ಕೆಳಗಿನವರು ಮಣಿಯುವಂತೆಯೂ ಮಾಡೀತು. ಮೇಲಧಿಕಾರಿಗಳು ತಲೆದೂಗುವಂತೆಯೂ ಮಾಡೀತು. ಹೌದು, ಇದೇ ಪ್ರೀತಿಯ ಪವಿತ್ರ ಗಾರುಡಿ!


\section*{‘ಅಹಂ’ಗೆ ಆಘಾತ ಬೇಡ}

\addsectiontoTOC{‘ಅಹಂ’ಗೆ ಆಘಾತ ಬೇಡ}

ಭಾರವಿಯ ಕತೆ ಕೇಳಿದ್ದೀರಾ?

ಆತ ಸಂಪ್ರದಾಯಸ್ಥ ಮನೆತನದ ಯುವಕ. ವಿದ್ಯಾಭ್ಯಾಸದಲ್ಲಿ ಚುರುಕು. ಕೆಲಸದಲ್ಲಿ ದಕ್ಷನಾಗಿದ್ದ. ಜೊತೆಗೆ ಸಭ್ಯತೆ ಶಿಷ್ಟಾಚಾರಗಳು ಅವನಲ್ಲಿ ಮಿಲಿತವಾಗಿದ್ದವು.

ಅಂಥ ವ್ಯಕ್ತಿ ತನ್ನ ತಂದೆಯನ್ನು ಕೊಲೆ ಮಾಡಬೇಕೆಂದು ಒಂದು ದಿನ ರಾತ್ರಿ ತಂದೆಯು ಮಲಗಿರುವ ಕೋಣೆಯ ಅಟ್ಟದ ಮೇಲೆ ದೊಡ್ಡ ಕಲ್ಲು ಗುಂಡೊಂದನ್ನು ಹಿಡಿದುಕೊಂಡು ಕಾಯು\-ತ್ತಿದ್ದ. ತಂದೆಯು ಮಲಗಿದೊಡನೆಯೇ ಕಲ್ಲನ್ನು ಹೊತ್ತುಹಾಕಿ ಅವನನ್ನು ಎಂದೆಂದೂ ಏಳದಂತೆ ಮಾಡುವ ಕರಾಳ ಕೃತ್ಯಕ್ಕಾಗಿ ಕಾದಿದ್ದ.

ಏನಾಗಿತ್ತು ಅವನಿಗೆ? ಹುಚ್ಚು ಹಿಡಿದಿತ್ತೇ?

ಸಾಮಾನ್ಯ ಅರ್ಥದಲ್ಲಿ ಹಾಗಾಗಿರಲಿಲ್ಲ, ನಿಜ. ಆದರೆ, ಅವನ ಮನಸ್ಸು ಅನುಭವಿಸುತ್ತಿದ್ದ ರೋಷದ ಉದ್ವೇಗ ಹುಚ್ಚಿನ ಮಿತಿಗೆ ಬಂದಿತ್ತು. ಅದಕ್ಕೂ ಮುಂದುವರಿದಿತ್ತು.

ಕಾರಣ?

ಭಾರವಿ ಏನೇ ಒಳ್ಳೆಯ ಕೆಲಸ ಮಾಡಲಿ, ಎಷ್ಟೇ ದಕ್ಷತೆಯಿಂದ ದುಡಿಯಲಿ, ಎಂಥದೇ ಪ್ರತಿಭಾನ್ವಿತ ಕೃತಿ ರಚಿಸಲಿ, ಅವನ ತಂದೆಯದು ಒಂದೇ ಪ್ರತಿಕ್ರಿಯೆ–ಟೀಕೆ, ಕಟು ಟೀಕೆ. ಒಂದೇ ಒಂದು ಮೆಚ್ಚುಗೆಯ ಶಬ್ದವೂ ಅವನ ಬಾಯಿಯಿಂದ ಹೊರಡದು. ಭಾರವಿಯ ಅಸಹನೆ ಮೀರಿತು. ತಂದೆಯ ಕಟು ವಿಮರ್ಶೆ ಅವನ ಪ್ರತಿಭೆಯ ತೊರೆಯನ್ನು ಬತ್ತಿಸಿತು. ರೋಷದ ಜ್ವಾಲೆ ಧಗಧಗಿಸಿತು. ಆ ಉನ್ಮಾದದಲ್ಲಿ ಆತ ತಂದೆಯನ್ನು ಕೊಲೆಮಾಡುವಂತಹ ಕರಾಳ ಕೃತ್ಯವನ್ನು ಕೈಗೊಂಡ.

ಆದರೆ, ತಂದೆ ಮಗ ಇಬ್ಬರೂ ಬದುಕಬೇಕಾಗಿತ್ತು!

ತಂದೆಯೊಂದಿಗೆ ಮಾತನಾಡುತ್ತಿದ್ದ ತಾಯಿಯ ಸ್ವರ ಅಟ್ಟದ ಮೇಲೆ ಕಲ್ಲುಗುಂಡನ್ನು ಇಟ್ಟುಕೊಂಡು ಕುಳಿತಿದ್ದ ಭಾರವಿಗೆ ಕೇಳಿಸಿತು. ಆಕೆ ಹೇಳುತ್ತಿದ್ದಳು, “ಮಗು ಈಗೀಗ ಬಹಳ ಬೇಸರ\-ದಲ್ಲಿರುತ್ತಾನೆ. ಅವನು ಮಾಡಿದ್ದಕ್ಕೆಲ್ಲಾ ನೀವು ಬೈಯುತ್ತೀರೆಂದು ದುಃಖಿತನಾಗಿದ್ದಾನೆ. ನಿಮಗಾ\-ದರೂ ಅವನ ಮೇಲೇಕೆ ಅಷ್ಟು ನಿರ್ದಯತೆ?”

ತಂದೆಯ ಉತ್ತರ ಕೇಳಿಸಿತು: “ನಾನೆಷ್ಟು ಹೃತ್ಪೂರ್ವಕವಾಗಿ ಅವನನ್ನು ಮೆಚ್ಚುತ್ತಿದ್ದೇನೆಂದು ನಿನಗೆ ತಿಳಿದಿಲ್ಲವೇ? ಮಕ್ಕಳನ್ನು ಎದುರಲ್ಲೇ ಹೊಗಳಬಾರದೆಂದು ಪೂರ್ವಿಕರ ಹೇಳಿಕೆ. ಅದರಿಂದ ಅವರ ಅಹಂಕಾರ ಹೆಚ್ಚುತ್ತದೆ.”

ಅಷ್ಟೇ ಸಾಕಾಯಿತು ಭಾರವಿಗೆ. ಕಲ್ಲುಗುಂಡನ್ನು ದೂರವಿಟ್ಟು ಹೊಸ ಜೀವನವನ್ನು\break ಆರಂಭಿಸಿದ. ಕವಿಯಾಗಿ, ರಸಋಷಿಯಾಗಿ ಕಿರಾತಾರ್ಜುನೀಯದಂಥ ಮೇರು ಕೃತಿಯನ್ನು ರಚಿಸಿ ತುಂಬು ಬಾಳನ್ನು ನಡೆಸಿ ಕೀರ್ತಿಶಾಲಿಯಾದ.

ಮೆಚ್ಚುಗೆಯ ಒಂದು ಮಾತು–ಅದೂ ಕದ್ದು ಕೇಳಿದ್ದು, ಅವನ ಜೀವನದ ಸಾರ್ಥಕತೆಗೆ ಕಾರಣವಾಯಿತು. ಘೋರಕೃತ್ಯದಿಂದ ತಪ್ಪಿಸಿತು. ಮಾನವನನ್ನಾಗಿ ಮಾಡಿತು. ಮೆಚ್ಚುಗೆಯ ಮಧುರ ಶಬ್ದ ಕೇಳದ ಅವನು ಎಷ್ಟೇ ಮೇಧಾವಿ ಚತುರನಾಗಿದ್ದರೂ ತಂದೆಯನ್ನೇ ಕೊಲೆ ಗೈಯುವ ಕೆಲಸಕ್ಕೆ ಸಿದ್ಧನಾದ.

ಭಾರವಿಯ ಕತೆಯ ನೀತಿ ಏನು ಎಂಬುದು ಈಗ ನಿಮಗೂ ಅರ್ಥವಾಗುತ್ತಾ ಬಂದಿರಬಹುದು. ಅಲ್ಲಿನ ನೀತಿ ಎಲ್ಲರೂ ಅನುಕರಿಸುವಂಥದು ಎನ್ನಬಹುದು. ಮನುಷ್ಯ–ಮನುಷ್ಯ\-ರೊಳಗೆ ಎಲ್ಲೆಲ್ಲಿ ಸಂಬಂಧ ಸಂಭವವೋ ಅಲ್ಲೆಲ್ಲ ಭಾರವಿಯ ಕತೆಯ ನೀತಿ ಪ್ರತ್ಯಕ್ಷವಾಗುತ್ತದೆ. ಆ ಸಂಬಂಧ ಇಬ್ಬರು ಸ್ನೇಹಿತರದ್ದಿರಲಿ, ಗಂಡ–ಹೆಂಡಿರಲ್ಲಿರಲಿ, ಪ್ರೇಮಿಗಳಲ್ಲಿರಲಿ, ಮಾಲೀಕ–ನೌಕರರೊಳಗಿರಲಿ, ಅಧ್ಯಾಪಕ–ವಿದ್ಯಾರ್ಥಿಗಳಲ್ಲಿರಲಿ, ಭಾಷಣಕಾರ–ಸಭಿಕರಲ್ಲಿರಲಿ, ವ್ಯಾಪಾರಿ–ಗಿರಾಕಿಗಳಲ್ಲಿರಲಿ–ಎಲ್ಲೇ ಆದರೂ ಈ ನೀತಿಯ ಪ್ರಾಮುಖ್ಯ ಅನಿವಾರ್ಯವಾಗುತ್ತದೆ. ಇನ್ನೊಬ್ಬರ ‘ಅಹಂ’ಗೆ ಆಘಾತ ಮಾಡಬೇಡಿ. ಇನ್ನೊಬ್ಬರನ್ನು ಮೆಚ್ಚಲು ಕಲಿಯಿರಿ. ಅರ್ಥಮಾಡಿಕೊಳ್ಳಲು ಕಲಿಯಿರಿ. ಇದೇ ಮಧುರ ಮಾನವೀಯ ಸಂಬಂಧಕ್ಕೆ ಸೋಪಾನ. ಇದರ ಅಭಾವವೇ ಎಲ್ಲ ವಿಧದ ಅಶಾಂತಿ. ಅತೃಪ್ತಿ, ದ್ವೇಷ ದ್ರೋಹಗಳ ಕಾರಣ” ಎಂದು ಸಾರುತ್ತದೆ ಈ ನೀತಿ.

“ಓಹ್​! ಇದೆಂಥ ತತ್ವಜ್ಞಾನ! ಎಲ್ಲ ಕೇಳಿ ನೋಡಿ ಆಗಿದೆ” ಎನ್ನುತ್ತೀರಿ ನೀವು. ಈ ಹಿಂದೆ ನೀವು ಕೇಳದ ಒಂದು ಅದ್ಭುತ ತತ್ವವೊಂದನ್ನು ಮೂರೇ ದಿನಗಳಲ್ಲಿ ನಿಮ್ಮ ಸಮಸ್ಯೆಗಳನ್ನು ಪರಿಹರಿಸಿ ನಿಮ್ಮನ್ನು “ಶ‍್ರೀಮಂತ, ಧೀಮಂತ, ಕಾಂತಿವಂತ, ಕೀರ್ತಿವಂತ”ರಾಗಿ ಮಾಡುವ ಮಾಂತ್ರಿಕ ಕ್ರಮವೊಂದನ್ನು ಇಲ್ಲಿ ಹೇಳಹೊರಟಿಲ್ಲ. ಅಂಥ ಒಂದು ತತ್ವವನ್ನು ಯಾರೂ ಕಂಡುಹಿಡಿದಿಲ್ಲ. ಅತ್ಯಂತ ಪ್ರಯೋಜನಕಾರಿಯಾದ ಗಾಳಿ–ನೀರು ಇವು ಪ್ರಕೃತಿಯಲ್ಲಿ ಬೇಕಷ್ಟು (ಬೆಲೆ ಕೊಡದೆ) ಸಿಗುವಂತೆ ನಮ್ಮ ಬದುಕು ಹಸನಾಗಲು ಅತ್ಯಂತ ಅವಶ್ಯವಾದ ಹಾಗೂ ಪ್ರಯೋಜನಕಾರಿಯಾದ ತತ್ವಗಳು ಸರಳವಾಗಿವೆ. ಯಥೇಷ್ಟವಾಗಿ ಸಿಗುತ್ತವೆ. ಪ್ರಪಂಚದ ಎಲ್ಲ ಧಾರ್ಮಿಕ ಮುಖಂಡರೂ ಮನಃಶಾಸ್ತ್ರಜ್ಞರೂ ವಿಜ್ಞಾನಿಗಳೂ ಅವನ್ನು ಪದೇ ಪದೇ ಜನತೆಗೆ ನೀಡುತ್ತಿದ್ದಾರೆ. ಆದರೆ, ನಾವು ಅದನ್ನು ಉಪಯೋಗಿಸುತ್ತಿದ್ದೇವೆಯೇ? ಎಂಬುದು ಮುಖ್ಯ ಪ್ರಶ್ನೆ.

ಒಂದು ಸವಿನುಡಿಯಲ್ಲಿ ವಿಶ್ವವನ್ನು ಬೊಮ್ಮಗಿಹ ಶಕ್ತಿ ಇದೆ. ಬಿರುನುಡಿಯ ಬಸಿರಿನಲಿ ಘೋರಾಂಧಕಾರವಿದೆ.\footnote{ಜಿ. ಎಸ್. ಶಿವರುದ್ರಪ್}

\begin{verse}
ಮೃದು ವಚನವೇ ಸಕಲ ತಪಂಗಳಯ್ಯಾ\\
 ಮೃದು ವಚನವೇ ಸಕಲ ಜಪಂಗಳಯ್ಯ\footnote{ಬಸವಣ್ಣನವರು}
\end{verse}

ಅಂತರರಾಷ್ಟ್ರೀಯ ಖ್ಯಾತಿಯ ಜನಪ್ರಿಯ ಮುಖಂಡನೆನಿಸಿಕೊಂಡ ಬೆಂಜಮಿನ್ ಫ್ರಾಂಕ್ಲಿನ್ ಅವನ ಅದ್ಭುತ ಯಶಸ್ಸಿನ ರಹಸ್ಯವೇನೆಂದು ಪ್ರಶ್ನಿಸಿದಾಗ ಆತ ಹೇಳಿದನಂತೆ: “ಯಾರ ಕುರಿ\-ತಾಗಿಯೂ ನಾನು ಕೆಟ್ಟ ಶಬ್ದಗಳನ್ನಾಡುವುದಿಲ್ಲ.”


\section*{ಏಕೆ ಮೆಚ್ಚುತ್ತಿಲ್ಲ?}

\addsectiontoTOC{ಏಕೆ ಮೆಚ್ಚುತ್ತಿಲ್ಲ?}

ಜನರು ಅವನನ್ನು ಮೆಚ್ಚುತ್ತಿಲ್ಲ, ಅವನಿಂದ ದೂರ ಸರಿಯುತ್ತಾರೆ. ಕಾರಣವೇನು?

೧. ಅವನು ನಂಬಿಕೆಗೆ ಅರ್ಹನಾದ ವ್ಯಕ್ತಿಯಲ್ಲ. ಕಾಗದ ಬರೆದರೆ ಸರಿಯಾಗಿ ಉತ್ತರಿಸುವ ಅಭ್ಯಾಸ ಅವನಿಗಿಲ್ಲ. ಓದಲು ಕೊಂಡುಕೊಂಡ ಪುಸ್ತಕವನ್ನು ಸಮಯಕ್ಕೆ ಸರಿಯಾಗಿ ಅವನು ಹಿಂದಿರುಗಿಸುವುದಿಲ್ಲ. ಬಸ್ ಸ್ಟೇಷನ್​ನಲ್ಲಿ ಆರು ಗಂಟೆಗೆ ಸಿಗುತ್ತೇನೆಂದು ಮಾತು ಕೊಟ್ಟರೂ ಎಂಟು ಗಂಟೆಯಾದರೂ ಅಲ್ಲಿಗೆ ಬರುವುದಿಲ್ಲ. ಸಣ್ಣಪುಟ್ಟ ಕೆಲಸಗಳಲ್ಲಿ ಸ್ವಲ್ಪ ಸಹಾಯ\break ಮಾಡುತ್ತೀಯಾ ಎಂದರೆ ‘ಸಮಯವಿಲ್ಲ’ ಎನ್ನುತ್ತಾನೆ. ಕೆಲಸ ಮಾಡಲು ಬಂದರೆ ಗಂಟು ಅಪಹರಿಸಿದವರಂತೆ ಮುಖ ಬೀಗಿಸಿಕೊಳ್ಳುತ್ತಾನೆ.

೨. ಅವನು ವಾದಪ್ರಿಯ. ಇತರರ ಪ್ರತಿಯೊಂದು ಮಾತಿಗೂ ವಿರೋಧ ಹೇಳುತ್ತಾನೆ. ಮುಖಕ್ಕೆ ಹೊಡೆದಂತೆ ಬಿರುಸಾಗಿ ಮಾತನಾಡುತ್ತಾನೆ. ಇತರರ ಮಾತಿನ ಅಭಿಪ್ರಾಯದ ಕಡೆಗೆ ಗಮನವೀಯದೆ ಸಣ್ಣಪುಟ್ಟ ದೋಷಗಳನ್ನು ಎತ್ತಿ ಹೇಳುತ್ತಾನೆ. ಅವನಿಗೆ ಸತ್ಯಸ್ಥಿತಿಯನ್ನು ತಿಳಿಯುವ ಕುತೂಹಲವಿಲ್ಲ. ಬುದ್ಧಿವಂತಿಕೆ ಪ್ರದರ್ಶನ ಮಾಡುವ ಹವ್ಯಾಸ ಅವನಿಗಿದೆ.

೩. ಅವನು ಯಾವಾಗಲೂ ತನ್ನ ವಿಚಾರ ಮಾತ್ರ ಮಾತನಾಡುತ್ತಿರುತ್ತಾನೆ. ತನ್ನ ಸುಖದುಃಖಗಳನ್ನೇ ಹೇಳುತ್ತಿರುತ್ತಾನೆ. ‘ಆಫೀಸಿಗೆ ಹೊತ್ತಾಗಿದೆ, ಹೋಗುತ್ತೇನೆ’ ಎಂದು ಹೇಳಿದರೂ ಅವರನ್ನು ನಿಲ್ಲಿಸಿಕೊಂಡು ಮಾತನಾಡುತ್ತಾನೆ. ನಮ್ಮ ಕುರಿತು ನಮಗೆ ಅತ್ಯಂತ ಹೆಚ್ಚಿನ ಅಭಿರುಚಿ ಇರುವು\-ದಾದರೂ ಇತರರಿಗೆ ನಾವೂ ಅಭಿರುಚಿಯ ವಿಷಯವಾಗಿರುವುದಿಲ್ಲ ಎಂಬುದನ್ನು ಆತ ಮರೆತಿ\-ದ್ದಾನೆ. ಇತರರ ಸುಖದುಃಖಗಳಲ್ಲಿ ಸಹಾನುಭೂತಿ ತೋರಿಸುವುದಿರಲಿ; ಅದನ್ನು ಕೇಳುವ ತಾಳ್ಮೆಯೇ ಅವನಿಗಿಲ್ಲ. ನೆರೆಯ ಕೋಣೆಯಲ್ಲಿ ಅವರು ಜ್ವರದಿಂದ ಮಲಗಿದ್ದರೂ ಅವನು ‘ಧಡಾರ್​’ ಎಂದು ಬಾಗಿಲು ಎಳೆದುಕೊಳ್ಳುತ್ತಾನೆ. ಸಮೀಪದಲ್ಲಿ ಪರೀಕ್ಷೆಗಾಗಿ ವಿದ್ಯಾರ್ಥಿ ಓದುತ್ತಿದ್ದರೆ ಅವನು ಅಪಸ್ವರದಲ್ಲಿ ಸಿನೆಮಾ ಹಾಡುಗಳನ್ನು ಕಿರುಚುತ್ತಿರುತ್ತಾನೆ.

೪. ಇತರರ ದೋಷಗಳನ್ನು ವರ್ಣನೆ ಮಾಡುವಾಗ ಅವನು ದಶಮುಖನಾಗುತ್ತಾನೆ. ಇತರರನ್ನು ಹಿಂದಿನಿಂದ ಹೀನಾಯವಾಗಿ ಟೀಕಿಸುವುದೆಂದರೆ ಆತನಿಗೆ ಜೇನು ಸವಿದಂತೆ. ಕೊಂಕು\-ನುಡಿ, ವ್ಯಂಗ್ಯೋಕ್ತಿಗಳ ಒಂದು ಸಂಗ್ರಹ, ಆತ ನಾಲ್ಕು ಜನ ಮಧ್ಯದಲ್ಲೇ ವ್ಯಕ್ತಿಯ ಇದಿರೇನೇ ಅವನ ದೌರ್ಬಲ್ಯಗಳನ್ನು ಹಾಸ್ಯದ ನೆಪದಲ್ಲಿ ಗಾಳಿಗೆ ಹಿಡಿಯುವುದೆಂದರೆ ಅವನಿಗೆ ಮೋಜೆನಿಸುತ್ತದೆ.

೫. ನಾಲ್ಕು ಮಂದಿಯೊಡನೆ ಬೆರೆಯುವಾಗ ಆತ ಇತರರ ಅಸ್ತಿತ್ವಕ್ಕೆ ಬೆಲೆ ಕೊಡುವುದಿಲ್ಲ. ತಾನೊಬ್ಬನೇ ಕೆಲಸಗಾರ, ಇತರರೆಲ್ಲ ನಿದ್ರೆ ಹೊಡೆಯುವ ಸೋಮಾರಿಗಳು ಎಂದು ತೋರಿಸಿಕೊಳ್ಳುತ್ತಾನೆ. ಗಟ್ಟಿಯಾಗಿ ಕೂಗಾಡುತ್ತಾ ಗೊಂದಲವೆಬ್ಬಿಸಿ ಸಣ್ಣ ಕೆಲಸವೊಂದನ್ನು ಮಾಡುತ್ತಿದ್ದರೂ ಪರ್ವತ ಉರುಳಿಸುವವರಂತೆ ನಟಿಸುತ್ತಾನೆ.

೬. ಇತರರ ಎಲ್ಲ ವಿಚಾರ ಅವನಿಗೆ ಬೇಕು. ಅವರು ಏನು ಮಾಡುತ್ತಾರೆ? ಎಲ್ಲಿಗೆ ಹೋಗುತ್ತಾರೆ? ಅವರ ದುರಭ್ಯಾಸಗಳೇನು? ಇತ್ಯಾದಿ.

೭. ತನಗೆ ಗೊತ್ತಿಲ್ಲದ ವಿಚಾರದಲ್ಲಿ ಬಾಯಿ ಹಾಕುವ ಅಭ್ಯಾಸ ಅವನಿಗಿದೆ. ‘ನನಗೆ ಆ ವಿಚಾರ ತಿಳಿಯದು’ ಎಂದು ಅವನು ಹೇಳಲಾರ.

೮. ಹತ್ತು ಜನರ ಹಕ್ಕು ಬಾಧ್ಯತೆಗಳನ್ನು ಗಮನಿಸದೆ ‘ಎಲ್ಲದರಲ್ಲೂ ನಾನು ಮುಂದೆ’ ಎನ್ನುವ ಪ್ರವೃತ್ತಿ ಅವನಿಗೆ ಇದೆ.

೯. ಅವನಿಗೆ ಯಾವ ಆದರ್ಶವೂ ಇಲ್ಲ, ವೈಶಿಷ್ಟ್ಯವೂ ಇಲ್ಲ. ಅವನು ಪ್ರಾಮಾಣಿಕನಲ್ಲ.

ಈ ಮನುಷ್ಯ ಎಲ್ಲೆಲ್ಲೂ ಇದ್ದಾನೆ. ಎಲ್ಲರಲ್ಲೂ ಯಾವುದಾದರೊಂದು ಪ್ರಮಾಣದಲ್ಲಿ ಆವಾಹಿತನಾಗಿದ್ದಾನೆ. ಈ ಕುರಿತು ನಮ್ಮಲ್ಲೇ ಹುಡುಕಾಟ ನಡೆಸಿ ಅವನ ಪ್ರಭಾವ ಕಡಿಮೆ ಮಾಡಿಕೊಳ್ಳುವ ಅಭ್ಯಾಸ ನಮ್ಮ ದುರ್ಲಭವಾದ ಬದುಕಿಗೆ ಸ್ವಲ್ಪವಾದರೂ ಸುಖ–ಶಾಂತಿಯ ಸೌಖ್ಯವನ್ನು ತಂದುಕೊಡಬಲ್ಲುದು; ಬದುಕಿಗೆ ಸಾರ್ಥಕತೆಯನ್ನೀಯಬಲ್ಲುದು, ಹಾಗೂ ರಾಷ್ಟ್ರಕ್ಕೆ ಶ್ರೇಯಸ್ಸನ್ನುಂಟು ಮಾಡಬಲ್ಲುದು.


\section*{ಪ್ರೀತಿಯೇ ಪರಮೌಷಧ}

\addsectiontoTOC{ಪ್ರೀತಿಯೇ ಪರಮೌಷಧ}

‘ಯುದ್ಧ ನಡೆಯುವುದು ಮೊದಲು ಅಂತರಂಗದಲ್ಲಿ, ನಂತರ ರಣರಂಗದಲ್ಲಿ’ ಎಂಬ ಮಾತಿದೆ. ಹೌದು, ಯುದ್ಧಕ್ಕೆ ಕಾರಣ ಪರಸ್ಪರ ದ್ವೇಷ, ಅಸೂಯೆ ಹಾಗೂ ವೈರ ಮನೋಭಾವವೇ, ಅರ್ಥಾತ್ ಪ್ರೀತಿಯ ಕೊರತೆಯೇ. ಅದರಿಂದಾಗಿ ಮನದಲ್ಲೆದ್ದ ಕಿಡಿ ಹೃದಯವನ್ನು ಸುಡತೊಡಗುತ್ತದೆ. ಪರಸ್ಪರ ಅಸೂಯೆ, ಅಶಾಂತಿ, ಅಸಮಾಧಾನಗಳು ಭುಗಿಲೇಳುತ್ತವೆ. ಕ್ರೋಧ ತಾಂಡವವಾಡುತ್ತದೆ. ಪೂರ್ವಾಪರ ವಿವೇಕ ಕಣ್ಮರೆಯಾಗುತ್ತದೆ. ಪ್ರೀತಿ ಕಣ್ಮರೆಯಾಯಿತು ಎಂದರೆ ಆಗದವನ ರೀತಿ ನೀತಿ, ಅಂದ–ಆಯ, ಆಚಾರ ವಿಚಾರ, ಅಲಂಕಾರ ವ್ಯವಹಾರಗಳೆಲ್ಲ ಅಸಹನೀಯವಾಗುತ್ತವೆ. ಆ ಬಗ್ಗೆ ಮನಸ್ಸು ರೋಸಿಹೋಗುತ್ತದೆ. ಅವನ ಅವನತಿಯ ಹಾರೈಕೆಯೇ ಹಿತವೆನಿಸುತ್ತದೆ. ಆತನ ನೋವು ಈತನ ನಲಿವಿಗೆ ಕಾರಣವಾಗುತ್ತದೆ; ಆತನ ಯಾತನೆ ಈತನ ನೆಮ್ಮದಿಗೆ ಮೂಲವಾಗುತ್ತದೆ. ಎಂಥ ವಿಕಟ ವಿಪರ್ಯಾಸ!

ಪ್ರಕೃತಿಯನ್ನು ಕಣ್ತೆರೆದು ನೋಡಿ, ಅದು ನೀಡುತ್ತಿರುವ ಪ್ರೀತಿಯ ಸಂದೇಶ ಗ್ರಹಿಸಿದವನ ಮನವೆಂದೂ ಅಂಥ ಕ್ರೂರತನಕ್ಕೆಳಸದು. ಅರಳಿದ ಕುಸುಮಗಳತ್ತ ಹಾತೊರೆದು ಬರುವ ದುಂಬಿಗಳನ್ನು ದ್ವೇಷಿಸಿ ಆ ಪುಷ್ಪಗಳೆಂದಾದರೂ ಮುದುಡಿಕೊಂಡದ್ದಿದೆಯೇ? ತುಂಬಿದ ಕೆಚ್ಚಲಿನಿಂದ ಹಾಲು ಹಿಂಡುವ ಗೊಲ್ಲನ ಕೈಗಳನ್ನು ಕಾಮಧೇನು ಸಿಟ್ಟಿನಿಂದ ತುಳಿದು ತುಂಡರಿಸಿದ್ದಿದೆಯೇ? ಜಲರಾಶಿಯಲ್ಲಿ ಒಂದಾಗಿ ವಾಸಿಸುವ ಮೀನುಗಳ ಸಂತತಿಯೊಡನೆ ವೈರ ಸಾಧಿಸಲು ಸಮುದ್ರ ಬತ್ತಿ ಬರಿದಾದದ್ದಿದೆಯೆ? ಬೀಸುವ ಗಾಳಿಯಲ್ಲಿ ಹಾರುವ ಪಕ್ಷಿಗಳನ್ನು ಮಾತ್ಸರ್ಯದ ಜಾಲದಲ್ಲಿ ಕೆಡಹಿ ಗಾಳಿ ಸ್ತಬ್ಧವಾದದ್ದಿದೆಯೆ? ಕೊಲೆಗೈಯುತ್ತಿರುವ ಕಟುಕನ ಕ್ರೌರ್ಯಕ್ಕೆ ಕೆರಳಿ ಬೆಳಕು ಮಾಯವಾದದ್ದಿದೆಯೆ? ಇಲ್ಲ, ಖಂಡಿತ ಇಲ್ಲ. ಪ್ರಕೃತಿಯಲ್ಲಿ ನಾವು ನೋಡುವುದು ಪರಿಶುದ್ಧ ಪ್ರೀತಿಯ ಪ್ರವಾಹವನ್ನೇ. ಅದೇ ಪ್ರಕೃತಿ ಹೇಳುವ ಪ್ರೀತಿಯ ಪಾಠ; ಮನುಕುಲದ ಶ್ರೇಯಸ್ಸಿಗೆ ನೀಡುವ ಆತ್ಮೀಯ ಬೋಧೆ.

ಹೌದು, ಮಾನವನೂ ಪರಿಶುದ್ಧ ಹೃದಯವಂತನಾದರೆ ಪ್ರೀತಿಯ ತವರಾಗುತ್ತಾನೆ; ಪ್ರೇಮದ ನಿಧಿಯಾಗುತ್ತಾನೆ; ವಾತ್ಸಲ್ಯದ ಗಣಿಯಾಗುತ್ತಾನೆ. ಆ ಪ್ರೀತಿಯೇ ಸರ್ವತ್ರವೂ, ಸರ್ವಕಾಲ\-ಗಳಲ್ಲಿಯೂ ಸರ್ವವಿಧದಲ್ಲಿಯೂ, ಅಶಾಂತಿ ಅಸಮಾಧಾನಗಳನ್ನು ಬಡಿದೋಡಿಸುವ ಪರಮಾಸ್ತ್ರ; ದೋಷ ದುರಿತಗಳನ್ನು ನೀಗುವ ಪರಮಮಂತ್ರ; ಅಸೂಯೆ ಅತೃಪ್ತಿಗಳನ್ನು ನಿರ್ನಾಮ ಮಾಡುವ ಪರಮೌಷಧ. ನಾವು ಇನ್ನಾದರೂ ಪ್ರಕೃತಿಯ ಪ್ರೀತಿಯ ಕರೆಗೆ ಓಗೊಟ್ಟು ಪರಿಶುದ್ಧ ಪ್ರೀತಿಯ ಪ್ರತೀಕರಾಗೋಣ; ಸರ್ವತ್ರ ಸುಖ ಶಾಂತಿ ಅಕ್ಷಯವಾಗುವಂತೆ ಯತ್ನಿಸೋಣ.

\begin{center}
ಸರ್ವೇ ಜನಾಃ ಸುಖಿನೋ ಭವಂತು!
\end{center}

\chapterend

\addtocontents{toc}{\protect\par\egroup}

