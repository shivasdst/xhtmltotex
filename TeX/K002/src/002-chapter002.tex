
\chapter{ನಿಮ್ಮಲ್ಲಿದೆ ಅಪಾರ ಶಕ್ತಿ!}

\indentsecionsintoc

\begin{itemize}
\item ಮನುಷ್ಯನಾಗಿ ವಿಜೃಂಭಿಸುವ ಆತ್ಮದ ಪರಮ ಮಹಿಮೆಯ ಯಥಾವತ್ ಗುಣಗಾನ ಮಾಡಲು ವಿಜ್ಞಾನದ, ಧರ್ಮದ ಅಥವಾ ಇನ್ನಿತರ ಯಾವುದೇ ಗ್ರಂಥಗಳಿಗೆ ಈ ತನಕ ಸಾಧ್ಯವಾಗಿಲ್ಲ. ಭೂಮಿಯ ಮೇಲೆ ಇದ್ದ, ಇರುವ ಹಾಗೂ ಇರಬಲ್ಲ ಸರ್ವಶಕ್ತ, ಸರ್ವ ವೈಭವೋ ಪೇತ ದೇವರೆಂದರೆ ಮನುಷ್ಯ ಮಾತ್ರನೇ.\hfill \enginline{—}ಸ್ವಾಮಿ ವಿವೇಕಾನಂದ

 \item ನಮ್ಮ ಶಕ್ತಿಸಾಮರ್ಥ್ಯದ ನೆಲೆಯಾಗಿ, ಪರಿಪೂರ್ಣತೆಯ ಗಣಿಯಾಗಿರುವ ಸಂಪತ್ತೆಷ್ಟೋ ಮನದಂತರಾಳದಲ್ಲಿ ಇಂದಿಗೂ ಶೋಧಿಸಲ್ಪಡದೆ ಜಡವಾಗಿಯೇ ಬಿದ್ದಿದೆ.\par~\hfill \enginline{—}ಡಾ. ಅಲೆಕ್ಸಿಸ್ ಕೆರೆಲ್​

 \item ಗೌರವಾದರಗಳಿಂದ ಮೆರೆದಾಡಿ, ಅಹಂಕಾರ ಮಮಕಾರಗಳಿಂದ ಹಾರಾಡಿ, ತನ್ನ ಬಗ್ಗೆ ಯಥಾವತ್ತಾಗಿ ಏನನ್ನೂ ತಿಳಿಯದೆ ಸಾವನ್ನಪ್ಪಿ ಈ ಲೋಕವನ್ನೇ ತ್ಯಜಿಸುವ ಮನುಷ್ಯನ ಸ್ಥಿತಿ ಅದೆಷ್ಟು ಶೋಚನೀಯ ಅಲ್ಲವೆ?\hfill \enginline{—}ಅಜ್ಞಾತ

 \item ಬಂದೊದಗುವ ಎಂಥ ಜಟಿಲ ಸನ್ನಿವೇಶವನ್ನೂ ಅಪಾರ ಎದೆಗಾರಿಕೆಯಿಂದ ಮೆಟ್ಟಿ ನಿಂತೆ ಎಂದಾದರೆ, ಭವಿಷ್ಯದಲ್ಲಿ ಯಾವುದೇ ದುರ್ಗಮ ಪರಿಸ್ಥಿತಿಯನ್ನೂ ನೀನು ಯಶಸ್ವಿಯಾಗಿ ಖಂಡಿತವಾಗಿಯೂ ಎದುರಿಸಬಲ್ಲೆ.\hfill \enginline{—}ವಿಲಿಯಂ ಜೇಮ್ಸ್ 

 \item \enginline{No books, no scriputres, no Science can even imagine the glory of the self that appears as man, the most glorious God that ever was, the only God that ever existed.\general{\hfill} —Swami Vivekananda}

 \item \enginline{Our deepest source of power and perfection has been left miserably undeveloped.\general{\hfill} —Dr. Alexis Carrel}

 \item \enginline{How wretched is the man, with honours crowned, who having not the one thing needful found, dies known to all but himself unknown.\general{\hfill} —Anon}

 \item \enginline{Infinite power of the Spirit brought to bear upon matter evolves material development, made to act upon thought evolves intellectuality, made to act upon itself makes of man a God.}\par~\hfill\enginline{ —Swami Vivekananda}

 \item \enginline{Acceptance of what has happened is the first step to overcome the consequences of any misfortune.\general{\hfill} —William James}

 \item \enginline{Doing is very good, but that comes from thinking. Fill the brain therefore with high thoughts, highest ideals, place them day and night before you and out of that will come great work.}\par~\hfill\enginline{ —Swami Vivekananda}

\end{itemize}


\section*{ಕಣಕಣದಲ್ಲೂ ಶಕ್ತಿಸಾಗರ}

\addsectiontoTOC{ಕಣಕಣದಲ್ಲೂ ಶಕ್ತಿಸಾಗರ}

ಯಾವುದೇ ಒಂದು ವಸ್ತುವಿನ ಒಂದು ಗ್ರಾಮಿನಷ್ಟು ತುಣುಕನ್ನು ತೆಗೆದುಕೊಂಡು ಶಕ್ತಿಯಾಗಿ ಪರಿವರ್ತಿಸಿದರೆ ಆ ಶಕ್ತಿಯ ಪರಿಮಾಣ ಎಷ್ಟು ಗೊತ್ತೇ?

ಮೂರು ಕೋಟಿ ಮುವ್ವತ್ತು ಲಕ್ಷ ಕುದುರೆಗಳು ಒಟ್ಟಿಗೆ ಒಂದು ಗಂಟೆ ಕಾಲ ಮಾಡುವ ಕೆಲಸಕ್ಕೆ ಸಮವಾಗುವಷ್ಟು ಅಗಾಧ!

ನಮ್ಮ ಇಂದ್ರಿಯಗಳಿಗೆ ಗೋಚರವಾಗುವ ಒಂದು ತುಣುಕು ವಸ್ತುವಿನಲ್ಲಿ ಅಡಗಿರುವ ಅಗೋಚರ ಶಕ್ತಿ ಹೋಲಿಕೆಗೇ ಸಿಗದಷ್ಟು ಮಹತ್ತರವಾಗಿದೆ ಎಂಬುದನ್ನು ಮಹಾವಿಜ್ಞಾನಿ ಆಲ್ಬರ್ಟ್ ಐನ್​ಸ್ಟೀನ್ ಸೂತ್ರವೊಂದರ ಮೂಲಕ ಶ್ರುತಪಡಿಸಿದರು. ಆ ಸೂತ್ರ ಇಂತಿದೆ:

\enginline{\textit{E=mc\supskpt{2}}}

ಯಾವುದೇ ದ್ರವ್ಯರಾಶಿಯಲ್ಲಿ ಅಡಗಿರುವ ಶಕ್ತಿಯ ಮೊತ್ತವನ್ನು ಸೂಚಿಸುವ ಸೂತ್ರ ಇದು. ಇದರಲ್ಲಿ \enginline{\textit{c}}ಯು ಬೆಳಕಿನ ವೇಗವನ್ನು ಸೂಚಿಸಿದರೆ, \enginline{E} ಯು \enginline{\textit{m}} ದ್ರವ್ಯರಾಶಿಯಲ್ಲಿರುವ ಶಕ್ತಿಯನ್ನು ಸೂಚಿಸುತ್ತದೆ.

ಈ ಸೂತ್ರ ವಸ್ತುವಿನ ನೈಜ ಸ್ವರೂಪದ ಬಗೆಗೆ ಒಂದು ನಿಶ್ಚಿತ ಜ್ಞಾನವನ್ನೂ, ವಸ್ತುಪ್ರಪಂಚದ ಏಕತೆಯನ್ನೂ ಸೂಚಿಸುತ್ತದೆ. ವಸ್ತುವನ್ನು ಶಕ್ತಿಯಾಗಿಯೂ, ಶಕ್ತಿಯನ್ನು ವಸ್ತುವಾಗಿಯೂ\break ಪರಿವರ್ತಿಸಬಹುದೆಂಬುದನ್ನು ಇಂದು ವಿಜ್ಞಾನಿ ಕಂಡುಕೊಂಡಿದ್ದಾನೆ. ವಸ್ತು ಮತ್ತು ಶಕ್ತಿ ಅವಿನಾಶಿ ಯಾದಂಥವುಗಳು ಎಂಬುದು ಈ ಸೂತ್ರದ ಮೂಲಕ ತಿಳಿಯುವ ಇನ್ನೊಂದು ತಥ್ಯ.\break ಒಟ್ಟಿನಲ್ಲಿ ಆಧುನಿಕ ಯುಗದ ಒಂದು ಮೂಲಭೂತ ಮಹಾಸಂಶೋಧನೆ ಇದು ಎಂಬ ಬಗೆಗೆ ಎರಡು ಮಾತುಗಳಿಲ್ಲ.

ನಿಜ, ನಮ್ಮ ಕಣ್ಣಿಗೆ ಕಾಣಿಸುವ ಅಥವಾ ಇಂದ್ರಿಯಗೋಚರವಾದ ಈ ಜಗತ್ತನ್ನು ಕುರಿತ ಅಸಂಖ್ಯ ರಹಸ್ಯಗಳನ್ನು ವಿಜ್ಞಾನಿ ಭೇದಿಸಿದ್ದಾನೆ. ಆದರೆ ಈ ರಹಸ್ಯಭೇದನೆಯ ಅತ್ಯದ್ಭುತ ಯಂತ್ರವಾದ ಮನುಷ್ಯನ ಮನಸ್ಸು, ಬುದ್ಧಿಗಳ ಸ್ವಭಾವ ಮತ್ತು ನೈಜಸ್ವರೂಪಗಳನ್ನು ಕುರಿತ ರಹಸ್ಯವನ್ನು ಭೇದಿಸಿದ್ದಾನೆಯೇ? ಮನುಷ್ಯನ ಮನಸ್ಸಿನ ಶಕ್ತಿಯ ಆಳ, ಅಗಲಗಳನ್ನು ಅರಿತಿದ್ದಾನೆಯೇ? ನಾವು ಇಂದು ಕಾಣುವ ಅಸಂಖ್ಯ ವಿಷಯಗಳನ್ನು ಕುರಿತ ಅಪಾರ ಜ್ಞಾನರಾಶಿ ಬಂದುದಾದರೂ ಎಲ್ಲಿಂದ? ಮಾನವನು ನಡೆಯಿಸಿದ ಚಿಂತನ–ಮಂಥನಗಳಿಂದ, ಪ್ರಯೋಗ– ಪರೀಕ್ಷಣಗಳಿಂದ, ವಿಚಾರ – ವಿಮರ್ಶೆಗಳಿಂದ, ಎಂದರೆ ಮನುಷ್ಯನ ಮನಸ್ಸಿನಿಂದಲೇ ತಾನೇ? ಅಂದರೆ ಈ ಮನಸ್ಸಿನ ಮಹಿಮೆ ಅಪಾರ ಅಲ್ಲವೆ?


\section*{ಬಿಂದುವಿನಲ್ಲಿ ಸಿಂಧು!}

\addsectiontoTOC{ಬಿಂದುವಿನಲ್ಲಿ ಸಿಂಧು !}

ವಿಶ್ವದ ವಿರಾಟ್ ಸ್ವರೂಪದೊಂದಿಗೆ ಮನುಷ್ಯನನ್ನು ಹೋಲಿಸಿದಾಗ ಮನುಷ್ಯ ಒಂದು ಧೂಳಿನ ಕಣವೂ ಅಲ್ಲ, ಸಿಂಧುವಿನಲ್ಲಿ ಅವನೊಂದು ಬಿಂದು ಅಷ್ಟೇ ಎನ್ನುವುದು ಆಕಾಶಕಾಯಗಳ ಅಧ್ಯಯನ ಮಾಡಿದಾಗ ತಿಳಿಯುವ ಅಂಶ. ಆದರೆ ಈ ‘ಬಿಂದು’ವಿನಲ್ಲಿರುವ ‘ಸಿಂಧು’ವನ್ನೂ ವಿಜ್ಞಾನವಿಂದು ಬಯಲು ಮಾಡಿದೆ. ಮನುಷ್ಯಶರೀರದ ರಚನೆ, ಅಂಗಾಂಗಗಳ ಕಾರ್ಯ ವಿಧಾನ, ಪ್ರತಿಯೊಂದು ಜೀವಕೋಶದಲ್ಲಿನ ಸಂಕೀರ್ಣ ಚಟುವಟಿಕೆಯ ವೈಖರಿ–ಇವುಗಳನ್ನೆಲ್ಲ ಕಂಡಾಗ ಪರಮಾಶ್ಚರ್ಯವೆನಿಸುತ್ತದೆ. ನಮ್ಮ ರಕ್ತದ ಒಂದು ಬಿಂದುವಿನಲ್ಲಿ \enginline{5}ಮಿಲಿಯ ಕೆಂಪು ರಕ್ತಕಣಗಳೂ, \enginline{10}ಸಾವಿರ ಬಿಳಿ ರಕ್ತಕಣಗಳೂ, \enginline{5}ಲಕ್ಷ ಪ್ಲೇಟ್​ಲೆಟ್​ಗಳೂ ಇವೆ ಎಂದರೆ ಅಚ್ಚರಿ ಎನಿಸದೇ? ಹೌದು, ವಿಶ್ವದ ಅಗಾಧತೆ ಒಂದು ಅದ್ಭುತವಾದರೆ, ಸೂಕ್ಷ್ಮಾತಿಸೂಕ್ಷ್ಮಕಣಗಳ ಗಹನತೆ ಇನ್ನೊಂದು ಅದ್ಭುತ! ಆದರೆ ಈ ಎರಡೂ ಅದ್ಭುತಗಳನ್ನು ವೀಕ್ಷಿಸುವ ಮಾನವನ ಮನಸ್ಸು ಪರಮಾದ್ಭುತ ಅಲ್ಲವೆ?

ಡಾ. ಅಲೆಕ್ಸಿಸ್ ಕೆರೆಲ್​ ಹೇಳುವಂತೆ ‘ಗಣಿತದ ಅಮೂರ್ತ ಚಿಂತನೆಯ ಮೂಲಕ ಮಾನವನ ಮನಸ್ಸು ಸೂಕ್ಷ್ಮಾತಿಸೂಕ್ಷ್ಮವಾದ ಇಲೆಕ್ಟ್ರಾನನ್ನೂ, ದೂರದ ಗಗನದಲ್ಲಿ ಮಿನುಗುವ ನಕ್ಷತ್ರವನ್ನೂ ತಿಳಿಯಬಲ್ಲದು.’ ‘ವಿಶ್ವವು ತನ್ನ ವಿಸ್ತಾರ ವೈಶಾಲ್ಯಗಳಿಂದ ನನ್ನ ಸ್ಥಾನವನ್ನು ಒಂದು ಸೂಜಿ ಮೊನೆಯಷ್ಟು ಜಾಗಕ್ಕೆ ಪರಿಮಿತಗೊಳಿಸಿದೆಯಾದರೂ, ನನ್ನ ಯೋಚನೆಯ ಮೂಲಕ ಈ ವಿಶ್ವವನ್ನೇ ನಾನು ಅರಿತುಕೊಳ್ಳುತ್ತಿದ್ದೇನೆ’\footnote{\engfoot{In space the Universe engulfs me and reduces me to a pin point. But throgh thought, I understand that Universe.}\hfill\engfoot{ –Pascal}} ಎಂಬುದು ವಿಖ್ಯಾತ ಗಣಿತಜ್ಞ, ಅನುಭಾವಿ ಪಾಸ್ಕಲನ ಮಾತು. ಅವೆಲ್ಲವೂ ಧ್ವನಿಸುವುದು ಮನುಷ್ಯನ ಮನಸ್ಸಿನ ಮಾಹಾತ್ಮ್ಯೆಯನ್ನು.

ಒಂದು ವಿಚಾರವಂತೂ ಸತ್ಯ. ಆಕಾಶಕಾಯಗಳ ವೈವಿಧ್ಯವನ್ನೋ ನಕ್ಷತ್ರ ಲೋಕಗಳ ವಿಸ್ತಾರ ವನ್ನೋ ನೋಡಿಕೊಂಡಲ್ಲ. ಬದಲು, ಮನಸ್ಸಿನ ಆಳದಲ್ಲಡಗಿರುವ ಶಕ್ತಿಯ ಪರಿಚಯದಿಂದ, ಅದು ವರ್ತಿಸುವ ವಿಧಾನದ ಅರಿವಿನಿಂದ, ಅದು ಅನುಸರಿಸುವ ನಿಯಮಗಳ ತಿಳಿವಿನಿಂದ, ಮನಸ್ಸಿನ ಮೂಲಭೂತ ಸ್ವರೂಪ ಸ್ವಭಾವಗಳ ಅನುಭೂತಿಯಿಂದ ಮನುಷ್ಯನ ಅಸ್ತಿತ್ವ ಸ್ವರೂಪಗಳನ್ನು ನಿರ್ಣಯಿಸುವುದು, ಜೀವನದ ಅರ್ಥ ಉದ್ದೇಶಗಳನ್ನು ಅರಿಯುವುದು.


\section*{ರಹಸ್ಯದ ಕೀಲಿಕೈ}

\addsectiontoTOC{ರಹಸ್ಯದ ಕೀಲಿಕೈ}

ಬಯಸಿದ್ದನ್ನು ನೀಡುವ ಅದ್ಭುತ ರಹಸ್ಯ ಶಕ್ತಿಯನ್ನು ಎಲ್ಲಿ ಅಡಗಿಸಿಡುವುದು ಎಂಬ ವಿಚಾರವಾಗಿ ದೇವತೆಗಳಲ್ಲಿ ಒಮ್ಮೆ ಚರ್ಚೆ ನಡೆಯಿತು. ಒಬ್ಬನು ಅದನ್ನು ಸಮುದ್ರದ ಆಳದಲ್ಲಿ ಹುದುಗಿಸೋಣ ಎಂದನಂತೆ. ಇನ್ನೊಬ್ಬ ಎತ್ತರ ಪರ್ವತದ ಶಿಖರದಲ್ಲಿ ಹೂತಿಡೋಣ ಎಂದ. ಮಗದೊಬ್ಬ ಅರಣ್ಯದ ಗುಹೆಯೇ ಅದಕ್ಕೆ ಸರಿಯಾದ ತಾಣ ಎಂದ. ದೇವತೆಗಳಲ್ಲಿ ಪ್ರಮುಖನೂ, ಮಹಾಮೇಧಾವಿಯೂ ಆದವನೊಬ್ಬ ಹೀಗೆಂದ: ‘ಆ ಶಕ್ತಿಯನ್ನು ಮನುಷ್ಯ ಮನಸ್ಸಿನ ಆಳದಲ್ಲೇ ಅಡಗಿಸಿ ಇಡೋಣ, ಅದು ಅಲ್ಲಿರಬಹುದೆಂಬ ಸಂದೇಹವೇ ಅವನಲ್ಲಿ ಉದಿಸದು. ಏಕೆಂದರೆ ಬಾಲ್ಯದಿಂದಲೇ ಅವನ ಮನಸ್ಸು ಹೊರಗಡೆ ಹರಿಯುತ್ತಿರುತ್ತದೆ. ಆ ಅದ್ಭುತ ಶಕ್ತಿ ತನ್ನಲ್ಲಿ ಇರಬಹುದೆಂದು ಅವನು ಯೋಚಿಸದೆ, ಹೊರಜಗತ್ತಿನಲ್ಲೇ ಅದನ್ನು ಅವನು ಹುಡುಕುತ್ತಾನೆ. ಆದುದರಿಂದ ಆತನ ಮನಸ್ಸಿನ ಆಳದ ಪದರಗಳಲ್ಲೇ ಆ ‘ಚಿಂತಾಮಣಿ’ಯನ್ನು ಅಡಗಿಸಿಡೋಣ’ ಎಂದ. ಎಲ್ಲರೂ ಅವನ ಯೋಜನೆಗೆ ತಮ್ಮ ಸಮ್ಮತಿಯನ್ನು ಸೂಚಿಸಿದರು. ಅಂತೆಯೇ ಅಲ್ಲಿ ಆ ಅದ್ಭುತ ರಹಸ್ಯ ಶಕ್ತಿಯನ್ನು ದೇವತೆಗಳು ಬಚ್ಚಿಟ್ಟರಂತೆ!

‘ಮನುಷ್ಯನ ಆಂತರ್ಯದ ಆಳದಲ್ಲಿ ಅಪಾರ ಶಕ್ತಿ ಇದೆ’–ಕತೆ ಹೇಳುವ ಸತ್ಯ ಇದು. ಅವನಿಗೆ ಯಾವುದು ಬೇಕೆನಿಸಿದರೂ ಅದನ್ನು ಪಡೆಯಬಲ್ಲ, ಅವನ ಪಾಲಿಗೆ ಯಾವುದೂ ಅಸಾಧ್ಯವಲ್ಲ. ಆದರೆ ಅಯ್ಯೋ! ತನ್ನಲ್ಲಿ ಅಪಾರ ಶಕ್ತಿ ಇದೆ ಎನ್ನುವ ಸತ್ಯದಲ್ಲಿ ಅವನಿಗೇ ನಂಬಿಕೆ ಇಲ್ಲ. ತನ್ನ ಬಗೆಗೆ ತಾನೇ ಒಂದು ಪರಿಮಿತ ಹಾಗೂ ಸಂಕುಚಿತ ಭಾವನೆಯನ್ನು ಬಲವಾಗಿ ರೂಢಿಸಿ ಕೊಂಡು ಅದರಿಂದ ಬಂಧಿತನಾಗಿದ್ದಾನೆ ಆತ. ಕಣ್ಣಿಗೆ ಕೈಮುಚ್ಚಿಕೊಂಡು ಕತ್ತಲೆ ಎಂದು ಅಳುವಂತಿದೆ ಅವನ ಸ್ಥಿತಿ!

ಅವನ ಈ ಸ್ಥಿತಿಗೆ ಕಾರಣವಾದರೂ ಏನು? ಅಜ್ಞಾನವೇ ಕಾರಣ. ತುಣುಕು ವಸ್ತುವಿನಲ್ಲಿ ಅಪಾರ ಶಕ್ತಿ ಅಡಗಿದೆ ಎಂಬುದನ್ನು ಐನ್​ಸ್ಟೀನ್ ಹೇಳುವುದಕ್ಕೆ ಮೊದಲು ತಿಳಿಯಲಾಗಲಿಲ್ಲವೇಕೆ? ಇದಕ್ಕೂ ಕಾರಣ ಅಜ್ಞಾನವೇ! ಸಹಸ್ರಾರು ವರ್ಷಗಳಿಂದ ಜನ ಭೂಮಿ ಚಪ್ಪಟೆ ಎಂದೇ ತಿಳಿದಿದ್ದರು. ಕಾರಣವೇನು? ತೋರಿಕೆಯನ್ನೇ ಸತ್ಯವೆಂದು ನಂಬಿದುದೇ ಅಲ್ಲವೇ?

ಜಗತ್ತಿನ ಬೇರೆ ಬೇರೆ ಭಾಗಗಳಲ್ಲಿ ಕಂಗೊಳಿಸಿದ ಮಹಾ ಅನುಭಾವಿಗಳು, ಪುರಾತನ ಭಾರತದ ಪುಷಿಗಳು, ಮನುಷ್ಯನ ನೈಜ ಸ್ವರೂಪದ ಬಗೆಗೆ ತಮ್ಮ ಅಲೌಕಿಕ ಅನುಭವದ ಬಲದಿಂದ ಒಂದು ಅದ್ಭುತ ಸೂತ್ರವನ್ನು ಕಂಡುಕೊಂಡಿದ್ದರು. ಆಧುನಿಕ ವಿಜ್ಞಾನ ಕ್ರಮದಲ್ಲಿ ನಡೆದ ಅಸಂಖ್ಯ ಸಂಶೋಧನೆಗಳು ಆ ಸೂತ್ರವನ್ನು ಪುಷ್ಟೀಕರಿಸುತ್ತವೆ ಎಂಬುದು ನಿಮಗೆ ಈ ಅಧ್ಯಾಯವನ್ನು ಓದಿ ಮುಗಿಸುವುದರೊಳಗೆ ನಿಚ್ಚಳವಾಗುವುದು. ಅದನ್ನು ಸರಿಯಾಗಿ ತಿಳಿದುಕೊಳ್ಳುವಂತಾದರೆ ವ್ಯಕ್ತಿ ಜೀವನದಲ್ಲೂ, ಸಾಮಾಜಿಕ ಜೀವನದಲ್ಲೂ, ಆಶ್ಚರ್ಯಕರ ಹಾಗೂ ಕ್ರಾಂತಿಕಾರಿ ಬದಲಾವಣೆ\-ಗಳನ್ನುಂಟು ಮಾಡಲು ಸಾಧ್ಯ. ನಮ್ಮ ವ್ಯಕ್ತಿತ್ವದ ನಿಗೂಢ ರಹಸ್ಯಗಳನ್ನು ಬಿಡಿಸಿ, ಬದುಕಿನ ಅರ್ಥ ಉದ್ದೇಶಗಳನ್ನು ತಿಳಿಸಿಕೊಡುವ ಸೂತ್ರ ಅದು.


\section*{ಹೊರತನದ ಬಯಲು}

\addsectiontoTOC{ಹೊರ\-ತನದ ಬಯಲು}

ವಿಜ್ಞಾನದ ಮುನ್ನೂರು ವರ್ಷಗಳ ಅದ್ಭುತ ಪ್ರಗತಿಯ ಕತೆ ಬಾಹ್ಯ ಪ್ರಕೃತಿ ಅಥವಾ ಹೊರ ಜಗತ್ತಿನ ಕುರಿತಾದುದು. ಮನುಷ್ಯನ ಮನಸ್ಸಿನ ರಚನೆ, ಸ್ವರೂಪ ಮತ್ತು ಶಕ್ತಿಗಳನ್ನು ಕುರಿತ ಅನ್ವೇಷಣೆ ಇತ್ತೀಚಿನದು, ಎಂದರೆ ಐವತ್ತು ವರ್ಷಗಳ ಈಚಿನಿಂದ ತೀವ್ರವಾಗಿ ನಡೆದಿರುವಂಥದು. ಸೂಕ್ಷ್ಮದರ್ಶಕ, ದೂರದರ್ಶಕಗಳ ಮೂಲಕ ನಮ್ಮ ಇಂದ್ರಿಯಗಳಿಗೆ ಗೋಚರವಾಗುವ ಹೊರ\-ಜಗತ್ತಿನ ಅಸಂಖ್ಯ ಸೂಕ್ಷ್ಮವಿಷಯಗಳನ್ನು ವಿಜ್ಞಾನಿ ಚೆನ್ನಾಗಿ ತಿಳಿದಿರುವುದು ನಿಜ. ಆದರೆ ನಿರಂತರ ಹೊರಮುಖವಾಗಿ ಹರಿಯುವ ಮನಸ್ಸಿನ ಸ್ವಭಾವವೇ ಅಂತರ್ಜ್ಞಾನದಿಂದ ದೊರೆಯಬಹುದಾದ ಇತರ ಆಯಾಮಗಳನ್ನು ನೋಡದಂತೆ ತಡೆ ಮಾಡುತ್ತಿದೆ ಎಂಬುದು ತಡವಾಗಿಯಾದರೂ ವಿಜ್ಞಾನಿಗಳಲ್ಲಿ ಅನೇಕರಿಗೆ ತಿಳಿಯುತ್ತಿದೆ.

ಪ್ರಕೃತಿ ರಹಸ್ಯಗಳನ್ನು ಭೇದಿಸಿ ಅಸಂಖ್ಯ ನೈಸರ್ಗಿಕ ನಿಯಮಗಳನ್ನು ಕಂಡು ಹಿಡಿದು ನಾನಾ ತೆರನಾದ ಸುಖ ಸೌಕರ್ಯಗಳನ್ನು ವಿಜ್ಞಾನಿ ಇಂದು ನಮಗೆ ನೀಡಿದ್ದಾನೆ. ಮಾನವನ ಅನ್ವೇಷಣಾ ಬುದ್ಧಿ ಏರಬಹುದಾದ ಎತ್ತರವನ್ನೂ, ಧೀಮಂತಿಕೆಯ ಪರಾಕಾಷ್ಠೆಯನ್ನೂ ತೋರಿದ್ದಾನೆ ಎನ್ನೋಣ. ಆದರೆ ಪ್ರಾಮಾಣಿಕತೆ, ನ್ಯಾಯಪರತೆ, ಸಹೋದರ ಭಾವ ಮತ್ತು ಪರಸ್ಪರ ಅರಿವು ಸಹಕಾರಗಳಿಂದ ಕೂಡಿದ, ಶಾಂತಿ ಸಹನೆಗಳಿಂದ ತುಂಬಿದ ಸುಂದರ ಜಗತ್ತೊಂದನ್ನು ನಿರ್ಮಿಸಲು ಅವನಿಂದ ಸಾಧ್ಯವಿಲ್ಲವೇ? ಏಕೆ ಸಾಧ್ಯವಾಗಿಲ್ಲ?

ಮನುಷ್ಯರ ಸದ್ಯದ ಸಂಕಟಮಯ ಸನ್ನಿವೇಶದ ಪ್ರತಿಬಿಂಬ ಈ ಕವಿವಾಣಿ–

\begin{verse}
ಜನಕಂಡರೆ ಜನ ಹೆದರಿದೆ\\ಎಲ್ಲೆಲ್ಲಿಯು ಸಂದೇಹ\\ಶಾಂತಿಯ ಪರದೆಯ ಹಿಂಗಡೆ\\ಕ್ರಾಂತಿಯ ರಣಸನ್ನಾಹ\\ನವ ನೀಚತೆ ಮಾರೀಚತೆ ದೇಶಪ್ರೇಮದ ನೆವದಿ\\ದಂಷ್ಟ್ರಾಶ್ರಿತ ರಾಷ್ಟ್ರಂಗಳು ಅಸುಹಿಂಡುತ್ತಿವೆ ಜವದಿ\\ಹೇಳೆನದೋ ಪಶ್ಚಿಮದೆಡೆ\\ರಂಜಿಪ ರಕ್ತಜ್ವಾಲೆ\\ಉನ್ಮಾದದ ರಣಮೋದದ\\ಮದ್ದಿನ ಗುಂಡಿನ ಲೀಲೆ!\\ಧಗಧಗಿಸಿದೆ ರಣರೋಷದ ಕೆಂಗಿಚ್ಚು
\end{verse}

~\hfill ಕುವೆಂಪು, ‘ಶತಮಾನ ಸಂಧ್ಯೆ’

ಪಂಚಭೂತಗಳನ್ನು ಕೈಬೆರಳೆಣಿಕೆಯಲ್ಲಿ ಕುಣಿಸುವ, ನೆಲಜಲಾಗಸಗಳಲ್ಲಿ ಅದ್ಭುತ ವೇಗದಿಂದ ಪಯಣಿಸುವ, ಮನುಷ್ಯಕುಲದ ರೋಗರುಜಿನಗಳು ಪಲಾಯನಸೂತ್ರ ಪಠಿಸುವಂತೆ ಮಾಡುವ ವಿಜ್ಞಾನಿ ‘ಬಾಳಿ ಬದುಕಿ ಬಳಲಿ ಬೀಳಲಿರುವ’ ಜನರ ಹೃದಯದಲ್ಲಿ ಒಂದು ಬಂಧುತ್ವದ ಭಾವವನ್ನುಂಟು ಮಾಡಲಾರನೇ? ಶಾಂತಿ ಸೌಹಾರ್ದ ಸಜ್ಜನಿಕೆಗಳನ್ನು ಸ್ಥಾಪಿಸಲಾರನೇ? ಪ್ರೀತಿಯ ಅಮೃತರಸವನ್ನು ಸುರಿಸಿ ದ್ವೇಷದ ದಾವಾನಲವನ್ನು ತಣಿಸಲಾರನೇ?

ಮೇಲಿನ ಪ್ರಶ್ನೆಗೆ ಉತ್ತರ ಹೀಗಿದೆ:

ಪ್ರಗತಿ ಎನ್ನುವಂಥದು ಎರಡು ಕ್ಷೇತ್ರಗಳಲ್ಲಿ ಆಗಬೇಕಾದುದು. ಅದು ಪ್ರಕೃತಿ ಅಥವಾ ನಿಸರ್ಗದ ಎರಡು ಮುಖಗಳನ್ನು ಕುರಿತಾದುದು–

ಒಂದು ಬಾಹ್ಯ ಪ್ರಕೃತಿ, ಇನ್ನೊಂದು ಮಾನವನ ಆಂತರಿಕ ಪ್ರಕೃತಿ. ಒಂದು ನಾವು ಕಾಣುವ ಜಗತ್ತನ್ನು ಕುರಿತಾದುದು. ಇನ್ನೊಂದು ಜಗತ್ತನ್ನು ಕಾಣುವ ಮನುಷ್ಯನ ಅಂತರಂಗವನ್ನು ಕುರಿತು ಆದುದು.

ವಾತಾವರಣ ಅಥವಾ ಪರಿಸರದ ಹಿಡಿತಗಳಿಂದ ಬಿಡಿಸಿಕೊಳ್ಳುವ ಶಕ್ತಿ ಮತ್ತು ಸ್ವಾತಂತ್ರ್ಯ ವೃದ್ಧಿಯಾದರೆ, ವ್ಯಕ್ತಿಯ ಪಾಲಿಗೆ ಪರಿಸರದ ಮೇಲೆ ಹೆಚ್ಚು ನಿಯಂತ್ರಣ ಸಾಧ್ಯವಾದರೆ ವಿಕಾಸ ಎಂದು ನಾವು ಕರೆಯುವ ಬದಲಾವಣೆ ಪ್ರಗತಿಯ ದಿಕ್ಕಿನತ್ತ ಸಾಗಿದೆ ಎನ್ನಬಹುದು. ಈ ದೃಷ್ಟಿಯಿಂದ ಬಾಹ್ಯ ಪ್ರಕೃತಿಯನ್ನು ನಿಗ್ರಹಿಸಲು ವಿಜ್ಞಾನದ ಅಸಂಖ್ಯ ಶೋಧನೆಗಳು ನಮಗೆ ಅಪಾರ ಸಹಾಯವನ್ನಿತ್ತಿವೆ. ನಾವು ಬಾಹ್ಯ ವಿಕಾಸ ಅಥವಾ ಪ್ರಗತಿಯತ್ತ ಮುನ್ನಡೆಯುತ್ತಿದ್ದೇವೆ ಎಂದು ಹೇಳಲು ಸಂಶಯ ಪಡಬೇಕಿಲ್ಲ. ಆದರೆ ಆಂತರ್ಯದ ವಿಕಾಸ ಆಗುತ್ತಿದೆಯೇ?


\section*{ಆಂತರ್ಯದ ಆಳದಲ್ಲಿ}

\addsectiontoTOC{ಆಂತರ್ಯದ ಆಳದಲ್ಲಿ}

ಮಾನವನ ಎಲ್ಲ ಚಟುವಟಿಕೆಗಳಿಗೆ ಮೂಲವಾದ ಮನಸ್ಸು, ಬುದ್ಧಿ, ಅಹಂ ಮತ್ತು ಆತ್ಮ— ಇವುಗಳಿಗೆ ಸಂಬಂಧಿಸಿದ ಆಂತರಿಕ ವಾತಾವರಣದ ಸ್ಥಿತಿ ಹೇಗಿದೆ? ಆಂತರಿಕ ಪ್ರಕೃತಿಯ ಮಾಲಿನ್ಯ ದೂರವಾಗುತ್ತಿದೆಯೇ? ಎಂದರೆ ಆಂತರಿಕ ನಿಗ್ರಹ ಸಾಧ್ಯವಾಗುತ್ತಿದೆಯೇ? ಆ ನಿಗ್ರಹವನ್ನು ಸಾಧಿಸಲು ಅನುಸರಿಸಬೇಕಾದ ನಿಯಮಗಳು ಇವೆಯೇ? ಅವು ಯಾವುವು? ಅಹಂಕಾರ, ಸ್ವಾರ್ಥತೆ, ದುರಾಸೆ, ಸ್ವೈರವೃತ್ತಿಯೇ ಮೊದಲಾದ ಕುಪ್ರಸಿದ್ಧವಾದ ಆರು ತೆರೆಗಳು (ಕಾಮ, ಕ್ರೋಧ, ಲೋಭ, ಮೋಹ, ಮದ, ಮತ್ಸರ) ಅಳತೆ ಮೀರಿ ಏರಿ ಬಂದಾಗ ಹೊರಜಗತ್ತಿನ ಪ್ರಗತಿ ಎಷ್ಟೇ ಆಗಿದ್ದರೂ, ಮಾನವ ಹೃದಯವನ್ನು ಕಬಳಿಸುವ ದುಃಖದೌರಾತ್ಮ್ಯಗಳನ್ನು ಶಮನ ಮಾಡಲು ಸಾಧ್ಯವೇ? ಒಬ್ಬಾತ ಧನಕುಬೇರನ ಪುತ್ರನಾಗಿದ್ದು ಗೃಹ ವಾಹನ ಸೇವಕರು ಪಾನಪೀಠ ಪರಿಕರಗಳೇ ಮೊದಲಾದ ಸಕಲವಿಧ ಸೌಕರ್ಯಗಳನ್ನು ಹೊಂದಿದ್ದರೂ ಮಾನಸಿಕ ಆರೋಗ್ಯ ಚೆನ್ನಾಗಿಲ್ಲವೆಂದಾದರೆ ಅದು ಎಂಥ ಆಭಾಸವಾಗಬಹುದಲ್ಲವೇ? ಹೊರಗಿನ ಯಾವತ್ತೂ ಸುಖ ಸಾಧನಗಳು ಅವನ ಪಾಲಿಗೆ ಅರ್ಥಹೀನವಷ್ಟೆ! ಪ್ರಸಿದ್ಧ ಬೆಲ್ಜಿಯಂ ಲೇಖಕ ನೊಬೆಲ್ ಪ್ರಶಸ್ತಿ ವಿಜೇತ ಮೌರಿಸ್ ಮೆಟರ್​ಲಿಂಕ್ ಒಂದು ಮಾತನ್ನು ತಮ್ಮ \enginline{The Great Beyond}(ದ ಗ್ರೇಟ್ ಬಿಯಾಂಡ್​) ಎನ್ನುವ ಗ್ರಂಥದ ಪೀಠಿಕೆಯಲ್ಲಿ ಹೇಳುತ್ತಾರೆ: ‘ಅತಿ ದೊಡ್ಡ ಇಂಜಿನಿಯರ್, ಗಣಿತಜ್ಞ, ವೈದ್ಯ ಅಥವಾ ಆಕಾಶ ವಿಜ್ಞಾನಿಯೂ ಕೂಡ ಶೋಷಕನೂ, ಮಹಾ ಮೂರ್ಖನೂ ಆಗಬಲ್ಲ ಎಂಬುದನ್ನು ಜನ ಗಮನಿಸುತ್ತಿಲ್ಲ ಅಥವಾ ಮರೆಯುತ್ತಾರೆ.’ ಕೇವಲ ವಿಚಾರ ಸಂಗ್ರಹ ಅಥವಾ ದಕ್ಷತೆ ಮನುಷ್ಯನನ್ನು ಸಜ್ಜನನನ್ನಾಗಿ ಮಾಡುವುದೇ? ವ್ಯಕ್ತಿಯ ಈ ಆಂತರಿಕ ಪರಿಸರ ಮಾಲಿನ್ಯದ ಭೂಮರೂಪವೇ ಜನಾಂಗದ ಪತನಕ್ಕೆ ಕಾರಣವಾಗುತ್ತದೆ ಎಂಬುದನ್ನು ಇತಿಹಾಸಕಾರರು ತಿಳಿಸುತ್ತಾರೆ. ಜಗತ್ತಿನ ಸುಮಾರು ಇಪ್ಪತ್ತೊಂದು ನಾಗರಿಕತೆಗಳಲ್ಲಿ ಹತ್ತೊಂಬತ್ತು ನಾಗರಿಕತೆಗಳು ಅವಸಾನ ಹೊಂದಿದುದು ಬಾಹ್ಯ ಆಕ್ರಮಣಗಳಿಂದಲ್ಲ, ಆಂತರಿಕ ನೈತಿಕ ಕುಸಿತದಿಂದ ಎಂದು ಪ್ರಸಿದ್ಧ ಇತಿಹಾಸಕಾರ ಅರ್ನಾಲ್ಡ್ ಟೋಯ್ನಬೀ ಹೇಳಿದ್ದಾರೆ.

ಬಾಹ್ಯ ಜಗತ್ತಿನ ಸಂಶೋಧನೆಗಳಿಂದಾದ ಪ್ರಗತಿ ಮಾತ್ರದಿಂದ ಮಾನವನ ಅಂತರಂಗದ ಪ್ರಗತಿಯಾಗಿ ಬಿಡುತ್ತದೆ ಎನ್ನುವ ಭರವಸೆ ಇಲ್ಲ. ಅಂತರಂಗದ ಪ್ರಗತಿಯ ನಿಯಮಗಳನ್ನು ತಿಳಿಯಲು ನಮ್ಮ ಆಂತರ್ಯದ ಅಥವಾ ಒಳಗಿನ ನಿಜವಾದ ಸ್ವರೂಪ ಸ್ವಭಾವಗಳನ್ನು ತಿಳಿಯಬೇಕು. ಪ್ರತಿಯೊಬ್ಬ ವ್ಯಕ್ತಿಯ ಮನಸ್ಸಿನ ಆಳದಲ್ಲಡಗಿರುವ ಅಮೂಲ್ಯವಾದ ರತ್ನದ ಸ್ಪಷ್ಟ ಪರಿಚಯ ಮಾಡಿಕೊಳ್ಳಬೇಕು. ಎಲ್ಲರ ಬದುಕಿಗೂ ಅನ್ವಯಿಸುವ ಅಭ್ಯುದಯ ಮತ್ತು ಸುಖದ ಸರ್ವಮಾನ್ಯ ಮೂಲ ಸೂತ್ರವನ್ನು ಕಂಡು ಹಿಡಿಯಬೇಕು. ಸತ್ಯದ ಅಸ್ತಿಭಾರದ ಮೇಲೆ ನಿಂತ ಸ್ವನಿರ್ದೇಶನಾ ಸಾಮರ್ಥ್ಯವನ್ನೀಯುವ ಮಾರ್ಗದರ್ಶಕ ಸೂತ್ರಗಳನ್ನು ಕಂಡುಹಿಡಿಯಬೇಕು. ತಿನ್ನುವುದು, ಕುಡಿಯುವುದು, ಕುಣಿಯುವುದು, ಮಕ್ಕಳನ್ನು ಪಡೆಯುವುದು ಮೊದಲಾದ\break ಇಂದ್ರಿಯ ವ್ಯಾಪಾರಗಳಿಗಿಂತ ಹಿರಿದಾದ ಗುರಿ ಮಾನವನ ಪಾಲಿಗಿದೆಯೇ? ಇದ್ದರೆ ಅದು ಏನು? ಎಂಬುದನ್ನು ಅರಿತುಕೊಳ್ಳಬೇಕು. ಸುಖದುಃಖಗಳ ಮೂಲವೇನು? ಬದುಕಿನಲ್ಲಿ ಅವುಗಳ ನಿಜವಾದ ಪಾತ್ರವೇನು? ಜೀವನದ ಅರ್ಥ ಉದ್ದೇಶಗಳೇನು? ಎಂಬುದನ್ನು ಸ್ಪಷ್ಟಪಡಿಸಿಕೊಳ್ಳಬೇಕು. ಇದು ಸಾಧ್ಯವಾಗುವವರೆಗೂ ಮನುಷ್ಯನ ಬದುಕಿಗೊಂದು ನೆಲೆ ಬೆಲೆ, ಗೊತ್ತು ಗುರಿ ಇಲ್ಲದಂತೆ ಆಗುತ್ತದೆ.

‘ಕೇವಲ ಯಂತ್ರಶಾಸ್ತ್ರದ ಪ್ರಗತಿಯಿಂದ, ಅಂದರೆ ಹೊಸತಾದ ಯಂತ್ರಗಳ ಆವಿಷ್ಕಾರದಿಂದ, ಮಾನವನ ಜೀವನವು ಉತ್ತಮವಾಗಲಿಲ್ಲ. ಭೌತ, ರಸಾಯನ, ಖಗೋಳಾದಿಶಾಸ್ತ್ರಗಳು ತಮಗೆ ಸಲ್ಲತಕ್ಕುದಕ್ಕಿಂತಲೂ ಹೆಚ್ಚು ಗೌರವವನ್ನು ಮಾನವನಿಂದ ಪಡೆದಿವೆ. ಇದರಿಂದ ಬಹಳ ಹಾನಿಯಾಗಲಿಲ್ಲವಾದರೂ ಈ ಶಾಸ್ತ್ರಗಳ ಪ್ರಮೇಯಗಳನ್ನು ಜೀವನದ ವಿಶ್ಲೇಷಣದ ಬಗ್ಗೆ ಉಪ ಯೋಗಿಸಿಕೊಳ್ಳುವಾಗ ಅವುಗಳು ನಿರರ್ಥಕವೆಂದು ಕಂಡುಬರುವುದರಲ್ಲಿ ಸಂದೇಹವಿಲ್ಲ. ಏತ ಕ್ಕೆಂದರೆ, ಮೇಲಿನ ಶಾಸ್ತ್ರಗಳ ಸಹಾಯದಿಂದ ಮಾನವನು ತನ್ನ ಸುಖ, ಸೌಕರ್ಯ ಮೊದಲಾದುವುಗಳನ್ನು ಸುಲಭವಾಗಿ ಹೆಚ್ಚಿಸಿದ್ದಾನೆ. ಆದರೆ ಇವುಗಳೊಡನೆಯೇ ಅವನನ್ನು ಮುಸುಕಿರುವ ನೈತಿಕ ಮತ್ತು ಬೌದ್ಧಿಕ ಪತನ\enginline{(Mental Degeneration)}ಗಳನ್ನು ನಿವಾರಿಸಲು ಶಕ್ತನಾಗಿ ಇರುವನೆ? ಇಲ್ಲ. ಇದರ ಬದಲಾಗಿ ಪ್ರತಿಯೊಂದು ಜನಾಂಗದಲ್ಲೂ ಈ ವಿದ್ಯೆಗಳು ಪತನಕ್ಕೆ, ಶೀಲದ ಹ್ರಾಸಕ್ಕೆ ಕಾರಣವಾಗಿವೆ. ತನ್ನನ್ನರಿತುಕೊಳ್ಳದಿರುವ ಮನುಷ್ಯನು, ತನ್ನ ಭಾವನೆ, ಪಿಪಾಸೆಗಳ ಹೊಲಬನ್ನೇ ಅರಿಯದ ವ್ಯಕ್ತಿಯು, ಕೇವಲ ಹಡಗುಗಳ ರಚನೆಯಲ್ಲಾಗಲೀ, ದೂರದಲ್ಲಿರುವ ನೀಹಾರಿಕೆಗಳ ಬೆಳಕಿನ ವಿಶ್ಲೇಷಣೆಯಲ್ಲಾಗಲೀ, ದೂರದರ್ಶಿ ಮತ್ತು ಸೂಕ್ಷ್ಮದರ್ಶಿ ಯಂತ್ರಗಳ ನಿರ್ಮಾಣದಲ್ಲಾಗಲೀ ಹೆಚ್ಚಿನ ಆಸ್ಥೆಯನ್ನು ತೋರುತ್ತಿರುವ ಮಾತ್ರಕ್ಕೆ ಜೀವನದ ಮೇಲ್ಮೆಯನ್ನು ಸಾಧಿಸಿದಂತಾಯಿತೇ? ಇದಲ್ಲದೆ ಕೆಲವು ಕ್ಷಣಗಳಲ್ಲೇ ಚೀನಕ್ಕಾಗಲೀ, ಜರ್ಮನಿಗಾಗಲೀ ಹಾರಾಡಿದನೆಂದರೆ ಆತನ ಮನುಷ್ಯತ್ವವು ಪ್ರಕಟವಾದಂತಾಯಿತೆ? ಅಥವಾ ಒಂದಷ್ಟು ಕಚ್ಚಾವಸ್ತುವಿನಿಂದ ಅತ್ಯಧಿಕ ಸಿದ್ಧವಸ್ತುವನ್ನು ತಯಾರಿಸಿದ ಮಾತ್ರಕ್ಕೆ ಸಮಸ್ತ ಜೀವನದ ಗುರಿಯನ್ನು ಮುಟ್ಟಿ ದಂತಾಯಿತೆ? ಇಲ್ಲ. ಹಾಗಾದರೆ ಯಂತ್ರ, ಭೌತ, ರಸಾಯನ, ಜೀವ, ಖಗೋಳಾದಿಶಾಸ್ತ್ರಗಳು ನಮ್ಮ ನೈಸರ್ಗಿಕ ಅಂತಃಶಕ್ತಿಗಳ ಉದಾತ್ತೀಕರಣಕ್ಕೆ ಕಾರಣವಾಗಿಲ್ಲ. ಅವುಗಳು ಅತ್ಯಧಿಕ ಮತ್ತು ಸಮರ್ಪಕ ಬುದ್ಧಿಶಕ್ತಿಯನ್ನಾಗಲೀ, ನೀತಿಪರತೆಯನ್ನಾಗಲೀ, ಆರೋಗ್ಯವನ್ನಾಗಲೀ, ನರ ವ್ಯೂಹದ ನಿರೋಧಶಕ್ತಿ ಮತ್ತು ಸಮತೋಲವನ್ನಾಗಲೀ, ಜೀವಿತದಲ್ಲಿ ಭದ್ರತೆ ಮತ್ತು ಶಾಂತಿಯನ್ನಾಗಲೀ ಒದಗಿಸಿರುವುದಿಲ್ಲವೆಂಬುದು ನಿರ್ವಿವಾದ. ಇಂದಿನ ಉದ್ಯಮ ಮತ್ತು ವಿಜ್ಞಾನಗಳ ಕ್ರಾಂತಿಕಾರಿ ಬದಲಾವಣೆಗಳನ್ನು ಸ್ವಾಗತಿಸಿದ ರಾಷ್ಟ್ರಗಳು ತೀವ್ರವೇಗದಿಂದ ಪತನದತ್ತ ಸಾಗುತ್ತಿವೆ. ನಾಗರಿಕತೆಯಿಂದ ಆಗಬೇಕಾದುದು ಮನುಷ್ಯತ್ವದ ಪ್ರಗತಿ, ಕೇವಲ ವೈಜ್ಞಾನಿಕ ತಾಂತ್ರಿಕ ಪ್ರಗತಿ ಅಲ್ಲ’\footnote{\engfoot{The aim of civilisation is the progress, not of science and machinery, but of mankind}\hfill\engfoot{ –Alexis Carrel}} ಎಂದು ಅಲೆಕ್ಸಿಸ್ ಕೆರಲ್ ಹೇಳಿದ್ದರು. ‘ಗೆಲಿಲಿಯೋ, ನ್ಯೂಟನ್, ಲೆವೋ ಸಿಯರ್ ಪ್ರಭೃತಿಗಳು ತಮ್ಮ ಬುದ್ಧಿಬಲವನ್ನು ಮಾನವ ಶರೀರ ಮತ್ತು ಪ್ರಜ್ಞೆಗಳ ಅಧ್ಯಯನಕ್ಕಾಗಿ ಮೀಸಲಿಟ್ಟಿದ್ದರೆ ನಮ್ಮ ಜಗತ್ತು ಇಂದು ಬೇರೆಯೇ ರೂಪತಾಳುತ್ತಿತ್ತು’ ಎಂದೂ ಡಾ.\ ಕೆರೆಲ್​ ಹೇಳಿದಾಗ ಮನುಷ್ಯನ ನೈಜಸ್ವರೂಪದ ಅಧ್ಯಯನದ ಆವಶ್ಯಕತೆಯನ್ನೇ ಒತ್ತಿ ಹೇಳಿದರು ಎಂಬುದು ಗಮನಾರ್ಹ. ಇದೇ ಅಭಿಪ್ರಾಯವನ್ನು ಐನ್​ಸ್ಟೈನ್ ಇನ್ನೊಂದು ರೀತಿಯಿಂದ ಹೇಳಿದ್ದರು: ‘ವಸ್ತು ವಿಜ್ಞಾನ ನಮ್ಮ ಬದುಕಿನ ಕೆಲವೊಂದು ಉದ್ದೇಶಗಳನ್ನು ಪೂರೈಸಿಕೊಳ್ಳಲು ಶಕ್ತಿ ಶಾಲಿಯಾದ ಉಪಕರಣಗಳನ್ನು ನೀಡುತ್ತದೆ. ಆದರೆ ಜೀವನದ ಕೊನೆಯ ಗುರಿಯನ್ನೂ ಅದನ್ನು ಸಾಧಿಸುವ ಹಂಬಲವನ್ನೂ ಬೇರೆಯೇ ಮಾರ್ಗದಿಂದ ಅಥವಾ ಮೂಲದಿಂದ ಕಂಡುಕೊಳ್ಳ ಬೇಕು.’ ಗ್ರೀಕ್ ದಾರ್ಶನಿಕರೂ ಉಪನಿಷತ್ತಿನ ಪುಷಿಗಳೂ ಸಾರಿದ ‘ನಿನ್ನನ್ನು ನೀನು ತಿಳಿ’ –‘ಆತ್ಮಾನಂ ವಿದ್ಧಿ’ ಎನ್ನುವ ವಾಣಿ ಈಗ ವಿಜ್ಞಾನಿಗಳ ಬಾಯಿಯಿಂದ ಪುನಃ ಮಾರ್ದನಿಸುತ್ತಿದೆಯಷ್ಟೇ.\footnote{\engfoot{Psychoneuros must be understood as the suffering of a human being who has not discovered what life means to him.}\hfill\engfoot{ –C.G. Jung.}

\engfoot{The 20th Century neurosis is the neurosis of purposelessness, meaning- lessness, valuelessness, hollowness and emptiness.}

\engfoot{\general{\hfill} –T.M. Thomas,\textit{ Images of Man}}}

ತನ್ನ ವಾಸ್ತವ ಸ್ವರೂಪ ಅಥವಾ ನಿಜದ ನೆಲೆ ಮತ್ತು ಬದುಕಿನ ಗುರಿಯನ್ನು ಕಾಣದ ಮನುಷ್ಯನ ಗೊಂದಲದ ಸ್ಥಿತಿಯನ್ನು ಶ‍್ರೀ ದ.\ ರಾ.\ ಬೇಂದ್ರೆಯವರು ತಮ್ಮ ಒಂದು ಕವಿತೆಯಲ್ಲಿ ಹೀಗೆಂದು ವರ್ಣಿಸಿದ್ದಾರೆ–

\begin{verse}
ಹೆರತನದ ಹೆದರಿಕೆಯು ಹೃದಯವನು ಹೊಕ್ಕಿಹುದು\\ನರತನದ ಹುದುರೊಳಗೆ ಜೀವ ಸಿಕ್ಕಿಹುದು\\ಹೊರತನದ ಬಯಲೊಳಗೆ ಬುದ್ಧಿ ಬಳಲಾಡುವುದು\\ಪರೆ ನಿನ್ನ ಕಿಂಕರಗೆ ಅಭಯಕರ ನೀಡೌ
\end{verse}

ಅಂತರ್ಮುಖಿಯಾಗದೆ, ಆಂತರ್ಯದ ಆಳದಲ್ಲಿ ಸತ್ಯದರ್ಶನ ಪಡೆಯದೆ, ಕಟ್ಟಕಡೆಗೆ ಸತ್ಯ ದರ್ಶನ ಪಡೆಯುವ ಹಂಬಲವನ್ನಾದರೂ ಇಟ್ಟುಕೊಳ್ಳದೆ, ಕೊನೆಯಪಕ್ಷ ಸತ್ಯದರ್ಶನ ಪಡೆದವರ ಮಾತುಗಳಲ್ಲಾದರೂ ವಿಶ್ವಾಸವಿಟ್ಟು ಆ ದಿಕ್ಕಿನಲ್ಲಿ ನಡೆಯಲು ಪ್ರಾಮಾಣಿಕವಾಗಿ ಯತ್ನಿಸದಿದ್ದರೆ ಕೇವಲ ಭೌತಪ್ರಪಂಚದ ಸುಖಸೌಕರ್ಯಗಳನ್ನೇ ಚರಮ ಲಕ್ಷ್ಯವೆಂದರಿತು ಅದನ್ನು ಹಿಂಬಾಲಿಸು\-ವವನ ಬದುಕಿನಲ್ಲಿ ಶೂನ್ಯತೆ, ಅರ್ಥಹೀನತೆಗಳು ಹೇಗೆ ಹಾಸುಹೊಕ್ಕಾಗಿರುವುವು ಎಂಬುದನ್ನು ಶ‍್ರೀ ಕುವೆಂಪು ತಮ್ಮ ‘ಶತಮಾನ ಸಂಧ್ಯೆ’ ಎನ್ನುವ ಕವನದಲ್ಲಿ ಧ್ವನಿಸಿದ್ದಾರೆ. ಅದು ವಿಜ್ಞಾನದ ಸಂಶೋಧನೆಗಳ ಮಹತ್ವವನ್ನು ಅಲ್ಲಗಳೆಯುವ ಉದ್ಗಾರವಲ್ಲ. ‘ನಾನು ಏನು? ಬದುಕಿನ ಗುರಿ ಏನು?’ ಎಂಬುದನ್ನು ತಿಳಿಯುವ ಆವಶ್ಯಕತೆಯನ್ನು ಒತ್ತಿ ಹೇಳುತ್ತ ಕೊಡುತ್ತಿರುವ ಕರೆ–

\begin{verse}
ನೊಣ ಮೀಸೆಯ ಹುಳುಹೆಜ್ಜೆಯ\\ಎಣಿಸುವ ಬಿಜ್ಜೆಯ ಬಲ್ಲ\\ತನ್ನಾತ್ಮವ ತಾನರಿಯುವ\\ಸಾಧನೆಯೊಂದನು ಒಲ್ಲ\\ಅನ್ವೇಷಣೆ ಅನ್ವೇಷಣೆ\\ಸುಖಿಸಲು ಪುರುಸೊತ್ತಿಲ್ಲ\\ತಿಳಿದೂ ತಿಳಿದೂ ತಿಳಿದೂ\\ಕೊನೆಗೇನೂ ಗೊತ್ತಿಲ್ಲ!
\end{verse}

ಇಂದು, ಈ ವಿಷಯವನ್ನು ತಿಳಿದುಕೊಳ್ಳಲು ವಿಜ್ಞಾನಶಾಸ್ತ್ರವೇ ನಮ್ಮ ನೆರವಿಗೆ ಬರುತ್ತಿದೆ. ವಿಜ್ಞಾನಿಗಳಲ್ಲಿ ಅನೇಕರು ತಮ್ಮ ಸ್ವಂತ ಅನುಭವ, ಪ್ರಯೋಗ ಮತ್ತು ಪರಿಶೀಲನೆ–ಇವುಗಳಿಂದ ಮನುಷ್ಯನಲ್ಲಡಗಿರುವ ಅಪಾರ ಶಕ್ತಿಯನ್ನು ಕುರಿತು ಖಚಿತವಾಗಿ ಹೇಳುತ್ತಿದ್ದಾರೆ.

\vskip 2pt


\section*{ಮನೋಮಂಡಲದ ವಿಶ್ವರೂಪ}

\vskip 0.5pt\addsectiontoTOC{ಮನೋಮಂಡಲದ ವಿಶ್ವರೂಪ}

ಜೀವಶಾಸ್ತ್ರದಲ್ಲಿನ ಸಂಶೋಧನೆಗಾಗಿ ನೊಬೆಲ್ ಪಾರಿತೋಷಕವನ್ನು ಪಡೆದು \enginline{\textit{‘Man, the Unknown’}}(ಅಜ್ಞಾತ ಮಾನವ) ಗ್ರಂಥದ ಮೂಲಕವೂ ತಮ್ಮ ಬುದ್ಧಿಬಲವನ್ನು ಮೆರೆಸಿದ ಜಗದ್ವಿಖ್ಯಾತ ವಿಜ್ಞಾನಿ ಡಾ.\ ಅಲೆಕ್ಸಿಸ್ ಕೆರೆಲರೆಂದಂತೆ ‘ಜೀವಿಯಲ್ಲಿರುವ ಅದ್ಭುತ ಶಕ್ತಿಶಾಲಿ ಯಾದ ಮನಸ್ಸನ್ನು ಶರೀರಶಾಸ್ತ್ರಜ್ಞರೂ, ಅರ್ಥಶಾಸ್ತ್ರಜ್ಞರೂ ನಿರ್ಲಕ್ಷಿಸಿದ್ದಾರೆ. ವೈದ್ಯರ ಗಮನವೂ ಆ ಕಡೆಗೆ ಹರಿದಿಲ್ಲ.’\footnote{\engfoot{The mind is hidden within the living matter completely neglected by physiologists and economists, almost unnoticed by physicians. Yet it is the most colossal power of this world.}\hfill\engfoot{ –Dr. Alexis Carrel}} ‘ನಮ್ಮಮನಸ್ಸಿನ ಆಳದ ಸ್ತರಗಳಲ್ಲಿ ಇದುವರೆಗೂ ನಾವು ಕಲ್ಪಿಸಿಕೊಂಡಿ ರದ ಶಕ್ತಿ, ಬಲ, ಧೈರ್ಯ–ಇವುಗಳ ಮೂಲಸ್ರೋತವಿದೆ. ಪ್ರತಿಯೊಬ್ಬನೂ ಆಳದ ಈ ಸ್ತರಗಳಲ್ಲಿ ಅಡಗಿರುವ ಶಕ್ತಿಯನ್ನು ತಮ್ಮ ಪಾಲಿಗೆ ಬಂದಿರುವ ಅವಕಾಶವನ್ನು ಉಪಯೋಗಿಸಿ ಕೊಂಡೇ ಬಡಿದೆಬ್ಬಿಸುವ ಸಾಧ್ಯತೆ ಇದೆ. ಅದೃಷ್ಟವಶಾತ್ ಅದನ್ನು ಅರಿತು ಉಪಯೋಗಿಸಿ ಕೊಂಡರೆ ಈವರೆಗೂ ಕೇಳಿರದ ಅಪಾರ ಶಕ್ತಿಯ ಮೂಲವಿದೆ ಎಂಬುದು ತಿಳಿಯುವುದು’ ಎಂಬುದು ಅಮೇರಿಕದ ಮನೋವಿಜ್ಞಾನಿ ಸ್ಮೈಲಿ\break ಬ್ಲಾಂಟನ್​ರ ಅಭಿಮತ. ಈ ದಿಸೆಯಲ್ಲಿ ರಷ್ಯಾ ದೇಶದ ಹಿರಿಯ ಸಂಶೋಧಕರಾದ ಬರ್ನಾಡ್ ಬಿ.\ ಕೆಜಿನಕ್ಸಿ ಮತ್ತು ಡಾ.\ ಪಾವೆಲ್ ನ್ಯೂಮಾವ್ ಅವರ ನಿಲುವೂ ಅದೇ–‘ಮನಸ್ಸೊಂದು ಸಾಮಾನ್ಯ ಜಡವಸ್ತುವಲ್ಲ. ಸಾಗರದಾಳದಲ್ಲಿ ಅಮೂಲ್ಯವಾದ ಮುತ್ತುರತ್ನಗಳಿರುವಂತೆ ನಮ್ಮ ಮನಸ್ಸಿನ ಆಳದಲ್ಲಿ ಅದ್ಭುತ ಶಕ್ತಿ ಇದೆ. ಇದು ನಮ್ಮ ಸಾಮಾನ್ಯ ಇಂದ್ರಿಯಗಳ ಮೂಲಕ ಬರುವ ಅರಿವಿನ ಪರಿಧಿಗೆ ಮೀರಿದುದು ಅಷ್ಟೆ.’ ಈ ಶತಮಾನದ ಪ್ರಸಿದ್ಧ ಮನೋವಿಜ್ಞಾನಿ ಎಂದು ಪರಿಗಣಿತರಾಗಿರುವ ಸಿ.\ ಜಿ.\ ಯೂಂಗರ ಪ್ರಕಾರ ‘ನಮ್ಮ ಜೀವತತ್ತ್ವದ ಒಂದು ಭಾಗ ದೇಶ ಕಾಲಕ್ಕೆ ಅತೀತವಾಗಿದೆ,\break ಯಾವುದು ದೇಶಕಾಲಕ್ಕೆ ಅತೀತವೋ ಅದು ಕಾರ್ಯಕಾರಣ ನಿಯಮಕ್ಕೂ ಅತೀತ.’\footnote{\engfoot{Part of psyche is beyond time and space and is unknown to us. That which is beyond time and space is also beyond causation.}\hfill\engfoot{ –C.G.Jung}}

\vskip 2pt

ಮನಸ್ಸಿನ ಅಸಾಮಾನ್ಯ ಶಕ್ತಿಗಳ ಬಗೆಗೆ ದೀರ್ಘಕಾಲದಿಂದ ಅಧ್ಯಯನ, ಪ್ರಯೋಗ ಮತ್ತು ಸಂಶೋಧನೆಗಳನ್ನು ಮಾಡುತ್ತ ಬಂದ ಅಮೇರಿಕದ ಡ್ಯೂಕ್ ವಿಶ್ವವಿದ್ಯಾನಿಲಯದ ಡಾ.\ ಜೆ.\ ಬಿ.\ ರ್ಹೈನ್ ಸುಮಾರು ನಲ್ವತ್ತು ವರ್ಷಗಳ ಹಿಂದೆಯೇ ಹೇಳಿದ್ದ ಮಾತುಗಳು ಇಂತಿವೆ–‘ದೇಶ ಕಾಲಗಳನ್ನು ಮೀರಿಹೋಗುವ ಶಕ್ತಿ ಮನಸ್ಸಿಗಿದೆ. ಎಂದರೆ ಅದು ಸಾಮಾನ್ಯ ಭೌತಿಕ ನಿಯಮಗಳ\break ಅಧೀನವಲ್ಲ ಎಂದಾಯಿತು. ಮನಸ್ಸನ್ನು ಆಗ ಭೌತಿಕವಸ್ತು ಎನ್ನುವುದಕ್ಕಿಂತಲೂ ಅಭೌತಿಕ ಅಥವಾ ಆತ್ಮ ಸಂಬಂಧಿಯಾದ ತತ್ತ್ವ ಎನ್ನಬೇಕಾಗುವುದು.’

‘ಭೌತವಿಜ್ಞಾನದ ಬೆಳವಣಿಗೆಯೊಂದಿಗೆ ಮನುಷ್ಯನ ಬಗೆಗೆ ತಳೆದ ಭೌತಿಕ ಕಲ್ಪನೆ ಇಂದು ಸಂಪೂರ್ಣವಾಗಿ ಸರಿಯಲ್ಲವೆಂದು ಸಿದ್ಧವಾಗಿದೆ–ಎಂದು ನಾವು ದೃಢವಾಗಿ ಹೇಳುವಂತಾಗಿದೆ. ಎಷ್ಟು, ಏನು, ಎಂದು ನಿರ್ದಿಷ್ಟವಾಗಿ ಹೇಳಲು ಆಗದಿದ್ದರೂ ಮನುಷ್ಯರಲ್ಲಿ ಭೌತಿಕವಲ್ಲದ ದೇಹಾತೀತವಾದುದು ಏನೋ ಇದ್ದೇ ಇದೆ ಎಂಬುದರಲ್ಲಿ ಸಂದೇಹವಿಲ್ಲ. ದೇಶಕಾಲದ ನಿಯಮಗಳಿಗೆ ಅಧೀನವಾಗದ ಸತ್ಯವೊಂದು ಮನುಷ್ಯನಲ್ಲಿದೆ.’ ಅಂದರೆ ಪ್ರತಿಯೊಬ್ಬರಲ್ಲೂ ಸಾವಿಲ್ಲದ ನೋವಿಲ್ಲದ ಸತ್ತ್ವ ಅಡಗಿದೆ ಎಂಬುದನ್ನು ಈ ಎಲ್ಲ ಮಾತುಗಳು ಧ್ವನಿಸುತ್ತವಲ್ಲವೆ?

‘ನಮ್ಮ ಸಾಮಾನ್ಯ ಎಚ್ಚರದ ಪ್ರಜ್ಞೆಯಿಂದ ದೂರವಿರುವ, ಸಾಮಾನ್ಯ ಪ್ರಜ್ಞೆಗೆ ನಿಲುಕದ, ಆಳದ ಸ್ತರಗಳಲ್ಲಿರುವ ಅತೀಂದ್ರಿಯ ಶಕ್ತಿಗಳನ್ನು ಸುಪ್ತ ಮನಸ್ಸು ಹೊಂದಿರುತ್ತದೆ. ನೀವು ಗಮನವಿರಿಸದೆ, ಮರೆತಿರಬಹುದಾದ ಅನೇಕ ಶತಮಾನಗಳ ಜ್ಞಾನ ಮತ್ತು ಅನುಭವಗಳನ್ನು ಅದು ವಿಶಿಷ್ಟ ರೀತಿಯಲ್ಲಿ ಒಳಗೊಂಡಿದೆ. ತಪ್ಪುದಾರಿಯಲ್ಲಿ ನಡೆಯುವ ಮೋಹವನ್ನು ತಡೆಯಬಲ್ಲೆವಾದರೆ ಈ ಸುಪ್ತ ಮನಸ್ಸು ಮನುಷ್ಯನಿಗೆ ವಿಶಿಷ್ಟ ಮಾರ್ಗದರ್ಶಕನಾಗಿ ಸೇವೆ ಸಲ್ಲಿಸಬಲ್ಲದು’ ಎನ್ನುತ್ತಾರೆ ಮನೋವಿಜ್ಞಾನಿ ಸಿ. ಜಿ. ಯೂಂಗ್.


\section*{ಸುಪ್ತಮನದ ಗುಪ್ತನಿಧಿ}

\addsectiontoTOC{ಸುಪ್ತ\-ಮನದ ಗುಪ್ತನಿಧಿ}

ಆ ಸುಪ್ತಮನಸ್ಸಿನ ಗುಪ್ತನಿಧಿಯನ್ನು ಸೂರೆಗೊಳ್ಳಲು ಸುಮಾರು ನೂರೈವತ್ತು ವರ್ಷಗಳಿಂದ\break ಹಿಪ್ನಾಸಿಸ್ ಅಥವಾ ಸುಪ್ತಿ ಆವಾಹನೆಯನ್ನು ಕುರಿತು ಕ್ರಮಬದ್ಧ ಅಧ್ಯಯನ ನಡೆಯುತ್ತ ಬಂದಿದೆ. ನಾನಾ ತೆರನಾದ ವಿಘ್ನವಿರೋಧಗಳ ಬಳಿಕ ಅದು ವಿಜ್ಞಾನಿಗಳ ಸ್ವೀಕಾರ ಗೌರವಗಳಿಗೆ ಪಾತ್ರವಾಗು\-ತ್ತಿರುವುದು ಇತ್ತೀಚೆಗೆ. ಈ ಸಂಬಂಧವಾಗಿ ಜೆ.\ ಬಿ.\ ಎಸ್.\ ಹಾಲ್ಡೇನ್​ ಅವರಂಥ ಹಿರಿಯ ವಿಜ್ಞಾನಿಯ ಅಭಿಪ್ರಾಯ ಗಮನಾರ್ಹ. ‘ಸುಪ್ತಿ ಆವಾಹನೆ ಮತ್ತು ಸೂಚನೆಗಳನ್ನು ಕುರಿತು ಒಂದು ಉದಾಹರಣೆಯನ್ನಾದರೂ ಪರಿಶೀಲಿಸಿದವನು ಜಗತ್ತಿನಲ್ಲಿ ಮುಂದೆ ಜರುಗ ಬಹುದಾದ ಅಪೂರ್ವ ಬದಲಾವಣೆಗಳನ್ನೂ, ಮನುಷ್ಯನ ಅಸ್ತಿತ್ವದ ಬಗೆಗಿನ ಅದ್ಭುತ ಸಾಧ್ಯತೆಗಳನ್ನೂ ತಿಳಿದುಕೊಳ್ಳಬಲ್ಲ. ವೈದ್ಯಕೀಯ ವಿಜ್ಞಾನದ ಸಂಶೋಧನೆಗಳ ಪ್ರಗತಿಯಿಂದ ಪಡೆದ ಅಸಂಖ್ಯ ಔಷಧಗಳು\break ಮೊದಲಿಗೆ ಪವಾಡವೆನಿಸಿದರೂ ಜನರ ಆರೋಗ್ಯ ಕ್ಷೇತ್ರದಲ್ಲಿ ಕ್ರಾಂತಿ ಯನ್ನೇ ಮಾಡಿದುವು. ಅಂತೆಯೇ ಮಾನಸಿಕ ಶಕ್ತಿಗಳ ಪ್ರಭಾವವನ್ನು ನಿಯಂತ್ರಿಸುವ ಸಾಮಾನ್ಯ ನಿಯಮಗಳನ್ನು ಸರ್ವ\-ಜನಗ್ರಾಹಿಯಾಗುವ ಮಟ್ಟಕ್ಕೆ ತಂದಾಗ ಅವು ಜಗತ್ತಿನ ಮುಖವನ್ನು ಸಂಪೂರ್ಣ ಬದಲಿಸಬಲ್ಲವು.’

ಸುಪ್ತ್ಯಾವಾಹನೆಯಿಂದ ಪಡೆಯಬಹುದಾದ ಎಷ್ಟೋ ರಹಸ್ಯ ವಿಷಯಗಳು ಸಾವಿರಾರು ವರ್ಷಗಳ ಹಿಂದೆಯೇ ಎಲ್ಲ ಜನಾಂಗಗಳಲ್ಲೂ ಕೆಲವೇ ವ್ಯಕ್ತಿಗಳಿಗೆ ತಿಳಿದಿತ್ತು. ಈ ಬಗ್ಗೆ ಈಗ ವೈಜ್ಞಾನಿಕವಾಗಿ ಸಾಕಷ್ಟು ಪ್ರಯೋಗ ಪರೀಕ್ಷೆಗಳು ನಡೆದಿವೆ. ಹತ್ತು ವರ್ಷಗಳ ಹಿಂದಿನ ವರದಿಯ ಪ್ರಕಾರ ಅಮೇರಿಕದಲ್ಲಿ ನಾಲ್ಕು ಸಾವಿರ ದಂತವೈದ್ಯರು ಸುಪ್ತ್ಯಾವಾಹನೆಯಲ್ಲಿ ತರಬೇತಿ ಪಡೆದವರು. ಇಂದು ಈ ಸಂಖ್ಯೆಗೂ ಅಧಿಕವಾಗಿ ಜನ ತರಬೇತಿ ಪಡೆಯುತ್ತಿದ್ದಾರೆ. ೧೯೫೦ರ ಹೊತ್ತಿಗೆ ಸುಮಾರು ಆರು ಸಾವಿರಕ್ಕಿಂತ ಹೆಚ್ಚಿನ ವೈದ್ಯರು ಮತ್ತು ತಜ್ಞರು ಈ ವಿದ್ಯೆಯಲ್ಲೂ ವಿಶೇಷ ತರಬೇತಿ ಪಡೆದು ರೋಗಿಗಳ ಉಪಚಾರ ಶುಶ್ರೂಷೆಯಲ್ಲಿ ಅದನ್ನು ಬಳಸಿದರು. ಮನೋವಿಜ್ಞಾನಿಗಳನ್ನೂ ಮನೋರೋಗತಜ್ಞರನ್ನೂ ಈ ಸಂಖ್ಯೆಯಲ್ಲಿ ಸೇರಿಸಿಲ್ಲ. \enginline{1958}ರಲ್ಲಿ ಅಮೇರಿಕದ ವೈದ್ಯಕೀಯ ಸಂಸ್ಥೆ ಈ ವಿಷಯವಾಗಿ ಎರಡು ವರ್ಷಗಳ ಕಾಲ ಅಧ್ಯಯನ ನಡೆಸಲು ಒಂದು ಮಂಡಲಿಯನ್ನು ರಚಿಸಿತ್ತು. ಕೆಲವೊಂದು ರೋಗಗಳನ್ನು ಗುಣಪಡಿಸಲು ಸುಪ್ತ್ಯಾವಾಹನೆಯನ್ನು ಆಸ್ಪತ್ರೆಗಳಲ್ಲಿ ಉಪಯೋಗಿಸಬಹುದೆಂದು ಆ ಮಂಡಲಿ ಶಿಫಾರಸು ಮಾಡಿತ್ತು. ಆದರೂ ಅಜ್ಞಾನಿಗಳ, ಅರ್ಧಜ್ಞಾನಿಗಳ, ಸ್ವಾರ್ಥಿಗಳ ಮತ್ತು ವಂಚಕರ ಕೈಯಲ್ಲಿ ಈ ವಿದ್ಯೆ ಶೋಷಣೆಯ ಸಾಧನೆಯಾಗಿ ದುರುಪಯೋಗವಾಗಬಹುದೆಂಬ ಅಷ್ಟೊಂದು ಸಂಗತವಲ್ಲದ ಭೀತಿ ಹಲವರಲ್ಲಿ ಉಳಿದು ಕೊಂಡಿದೆ.

ಈ ಕ್ಷೇತ್ರದಲ್ಲಿ ರಷ್ಯಾ ಸಾಧಿಸಿದ ಪ್ರಗತಿ ಕಣ್ಣುಕೋರೈಸುವಂಥದು. ರಷ್ಯಾ ದೇಶದ ತಜ್ಞರು ಈ ವಿಚಾರದಲ್ಲಿ ಸುಮಾರು ಮೂವತ್ತೈದು ವರ್ಷಗಳ ಹಿಂದೆಯೇ ಪರೀಕ್ಷೆ ಪರಿಶೀಲನೆಗಳ ಸಾಮಾನ್ಯ ಪರಿಧಿಯನ್ನು ದಾಟಿ ಸುಪ್ತಿ ಆವಾಹನೆಯ ನಿರ್ದಿಷ್ಟ ಉಪಯೋಗವನ್ನು ಕಂಡುಕೊಂಡಿದ್ದರು. ರೋಗಿಗಳ ಉಪಚಾರದಲ್ಲಿ ಅರಿವಳಿಕೆಯನ್ನುಂಟುಮಾಡುವುದರಲ್ಲಿ ಈ ವಿದ್ಯೆಯನ್ನು ಉಪಯೋಗಿ\-ಸಿದ್ದರು. ಆಧುನಿಕ ಔಷಧಗಳಿಗೆ ಬಗ್ಗದ ಜಗ್ಗದ ಅಸಂಖ್ಯ ರೋಗಗಳನ್ನು ನಿಯಂತ್ರಿಸುವ ಕಲೆಯಲ್ಲಿ ಪರಿಣತರಾಗಿದ್ದರು. ಮದ್ಯ ಸೇವನೆಯ ಚಟಕ್ಕೆ ಬಲಿಬಿದ್ದ ಹಲವರನ್ನು ಶಾಶ್ವತವಾಗಿ ಆ ದುರಭ್ಯಾಸದಿಂದ ಬಿಡಿಸಿದ್ದರು. ಮಾರ್ಫಿನ್ ಔಷಧ ಸೇವಿಸಿ ಮರಣಾಂತಿಕ ಸ್ಥಿತಿಗೆ ಹೋಗಿದ್ದ ಹಲವರನ್ನು ಈ ಕ್ರಮದಿಂದ ಉಳಿಸಿದ್ದರು. ಈ ಪದ್ಧತಿಯ ಪ್ರಯೋಗದಿಂದ ನಾನಾ ತೆರನಾದ ಮನೋ ದೈಹಿಕ ಬೇನೆಗಳನ್ನು ದೂರಮಾಡುವಲ್ಲಿ ಯಶಸ್ವಿಗಳಾದುದಲ್ಲದೇ ಮನಸ್ಸಿನ ಆಳದ ಸ್ತರಗಳಲ್ಲಡಗಿರುವ ಸೂಕ್ಷ್ಮ ಸಂಕೀರ್ಣ ಗಂಟು, ಗಂತಿಗಳನ್ನೂ ತೀವ್ರ ನೋವು ಅಥವಾ ಆಘಾತಗಳ ಸಂಸ್ಕಾರಗಳನ್ನೂ ಗುರುತಿಸಿ, ವ್ಯಕ್ತಿತ್ವವನ್ನೇ ಮೊಟಕುಗೊಳಿಸುವ ಅವುಗಳನ್ನು ದೂರ ಮಾಡಲು ಸಮರ್ಥರಾಗಿದ್ದರು.

ಈ ವಿಧಾನಗಳಿಂದ ಮನುಷ್ಯನ ಮನಸ್ಸಿನ ಶಕ್ತಿ ಸ್ವರೂಪಗಳನ್ನು ತಿಳಿಯುವ ಆವಶ್ಯಕತೆಯನ್ನೂ, ಇಂದ್ರಿಯಾತೀತ ಅನುಭವಗಳ ಅನ್ವೇಷಣೆಯ ಮಹತ್ವವನ್ನೂ ತಿಳಿಸುವ ಇನ್ನಿಬ್ಬರು ವಿಜ್ಞಾನಿಗಳ ಅಂಬೋಣವನ್ನು ಈ ಸಂದರ್ಭದಲ್ಲಿ ನೆನಪಿಗೆ ತಂದುಕೊಳ್ಳಬೇಕು. ಹಿರಿಯ ಮನೋ\-ವಿಜ್ಞಾನಿ ಸಿಗ್ಮಂಡ್ ಫ್ರಾಯಿಡ್ ‘ನಾನು ಇನ್ನೊಮ್ಮೆ ನನ್ನ ಜೀವನವನ್ನು ಪ್ರಾರಂಭಿಸುವು\-ದಾದಲ್ಲಿ ಮನೋವಿಶ್ಲೇಷಣೆಗಿಂತಲೂ ಅತೀಂದ್ರಿಯ ವಿಚಾರಗಳ ಸಂಶೋಧನೆಗೆ ನನ್ನ ಜೀವನವನ್ನು ಮೀಸ\-ಲಾಗಿಡುತ್ತಿದ್ದೆ’ ಎಂದಿದ್ದರೆ ‘ವಿಜ್ಞಾನವು ಅಭೌತಿಕ ಅಥವಾ ಅತೀಂದ್ರಿಯ ಘಟನೆಗಳ ಮೂಲವನ್ನು ಎಂದು ಸಂಶೋಧಿಸತೊಡಗುವುದೊ ಆಗ ಒಂದೇ ದಶಕದಲ್ಲಿ ಅದು ಹಿಂದಿನ ಶತಮಾನಗಳ\-ಲ್ಲೆಲ್ಲ ಸಾಧಿಸಿದುದಕ್ಕಿಂತ ಹೆಚ್ಚಿನ ಪ್ರಗತಿಯನ್ನು ಸಾಧಿಸುತ್ತದೆ’ ಎಂದು ನಿಕೊಲಾಯ್ ಟೆಸ್ಲಾ ಹೇಳಿದ್ದಾರೆ.

ರಷ್ಯಾದ ಡಾ.\ ವೆಸಿಲಿವ್ ಅನ್ನುವಂತೆ ಅತೀಂದ್ರಿಯ ಜ್ಞಾನದ ಹಿನ್ನೆಲೆಯಲ್ಲಿರುವ ಶಕ್ತಿಯ ಅನ್ವೇಷಣೆಯು ಅಣುಶಕ್ತಿಯ ಅನ್ವೇಷಣೆಗೆ ಸಮನಾದುದು. ವಾಸ್ತವತೆಯ ಆರಾಧಕರಾದ ಮತ್ತು ಅತೀಂದ್ರಿಯ ಸಂಶೋಧನೆಗಳ ಪ್ರಾಥಮಿಕ ಪರಿಚಯವಿಲ್ಲದ ನಮ್ಮ ದೇಶದ ವೈಜ್ಞಾನಿಕ ಮನೋ ವೃತ್ತಿಯವರೆನಿಸಿಕೊಂಡ, ಬುದ್ಧಿಜೀವಿಗಳೆನಿಸಿಕೊಂಡ ಹಲವರಿಗೆ, ಈ ದಿಸೆಯಲ್ಲಿ ರಷ್ಯಾ ತನ್ನ ಅನ್ವೇಷಣೆಯನ್ನು ತೀವ್ರನಿಷ್ಠೆಯಿಂದ ಮುಂದುವರಿಸುತ್ತಿದೆ ಎಂಬುದು ನುಂಗಲಾಗದ ತುತ್ತಾಗ ಬಹುದು. ಸತ್ಯಾನ್ವೇಷಣೆಗಿಂತಲೂ ತಮಗೆ ತಿಳಿಯದ್ದನ್ನು ಮೂಢನಂಬಿಕೆಗಳು ಎಂದು ತೀವ್ರವಾಗಿ ಖಂಡಿಸುವುದೇ ವಿಜ್ಞಾನದ ಗುರಿ ಎಂದು ನಮ್ಮಲ್ಲಿ ಹಲವರು ತಿಳಿದುಕೊಂಡಂತಿದೆ. ಆದರೆ ಸತ್ಯವು ಯಾರ ನಂಬಿಕೆಯನ್ನೂ ಆಧರಿಸಿಕೊಂಡಿಲ್ಲವಷ್ಟೆ.


\section*{ದಿವ್ಯ ಪ್ರಭೆಯ ಭವ್ಯತೆ}

\addsectiontoTOC{ದಿವ್ಯ ಪ್ರಭೆಯ ಭವ್ಯತೆ}

ಮನುಷ್ಯಶರೀರದಿಂದ ಹೊರಹೊಮ್ಮುವ, ಸಾಮಾನ್ಯ ದೃಷ್ಟಿಗೆ ನಿಲುಕದ ಒಂದು ತೆರನಾದ\break ಪ್ರಭಾವಲಯದ ಅಸ್ತಿತ್ವದ ಬಗೆಗೆ ಅನಾದಿಕಾಲದಿಂದಲೂ ಬೇರೆಬೇರೆ ದೇಶಗಳ ಜನರಲ್ಲಿ ನಂಬಿಕೆ ಇತ್ತು. ಸಂತರ, ಭಗವದ್ಭಕ್ತರ ಚಿತ್ರಗಳಲ್ಲಿ ಮುಖ್ಯವಾಗಿ ತಲೆಯ ಸುತ್ತಲೂ ಪ್ರಭಾವಲಯ ಚಿತ್ರಿಸಿರುವುದನ್ನು ನಾವು ಕಾಣುತ್ತೇವೆ. ಕೆಲವೊಮ್ಮೆ ಮಕ್ಕಳಿಗೆ, ಅಪರೂಪವಾಗಿ ಅತೀಂ ದ್ರಿಯ ಪ್ರಜ್ಞೆ ಜಾಗೃತವಾದ ಕೆಲವು ವ್ಯಕ್ತಿಗಳಿಗೆ, ಪ್ರತಿಯೊಬ್ಬ ಮನುಷ್ಯನಲ್ಲೂ ಇರುವ ಪ್ರಭಾವಲಯವು ವಿವಿಧ ವರ್ಣಗಳಲ್ಲಿ ಕಾಣಿಸುವುದುಂಟು. ಅವುಗಳ ವೈವಿಧ್ಯ ಮತ್ತು ಸಾಂದ್ರತೆಯನ್ನು ಪರಿಶೀಲಿಸಿ ವ್ಯಕ್ತಿಯ ದೈಹಿಕ ಮಾನಸಿಕ ಆರೋಗ್ಯದ ಸಂಪೂರ್ಣ ವಿವರಗಳನ್ನು ಹೇಳಲು ಸಾಧ್ಯವಿದೆ. ಇದು ಹಿಂದೆ ಕೆಲವೇ ವ್ಯಕ್ತಿಗಳ ಪಾಲಿನ ಸಿದ್ಧಿಯಾಗಿದ್ದರೆ, ರಷ್ಯನ್ ತಜ್ಞರು ಇಂದು ಕೀರ್ಲಿಯನ್ ಉಪಕರಣಗಳ ಮೂಲಕ ಈ ವಿವಿಧ ವರ್ಣಗಳನ್ನೂ ಅವು ಸೂಚಿಸುವ ಎಕ್ಸ್ ರೇಗೂ ನಿಲುಕದ ರೋಗ ಮೂಲವನ್ನೂ, ಸ್ಪಷ್ಟವಾಗಿ ಗುರುತಿಸಲು ಸಮರ್ಥರಾಗಿದ್ದಾರೆ. ನಾವು ಹೇಳಬಹುದಾದ ಒಂದು ಸಣ್ಣ ಸುಳ್ಳನ್ನೂ ಕೀರ್ಲಿಯನ್ ಉಪಕರಣ ಗುರುತಿಸಬಲ್ಲುದು. ಮಾನಸಿಕ ಏರುಪೇರುಗಳನ್ನೂ, ಮಾನಸಿಕ ವ್ಯಾಧಿಗಳ ಕಾರಣವನ್ನೂ ಅದು ಗುರುತಿಸಬಲ್ಲದು. ವಿಜ್ಞಾನದ ವಿವಿಧ ಕ್ಷೇತ್ರಗಳಲ್ಲಿ ಸಂಶೋಧನಾನಿರತರು ಈ ಉಪಕರಣವನ್ನು ತಮ್ಮ ಅನ್ವೇಷಣೆಯಲ್ಲೂ ಉಪಯೋಗಿಸುತ್ತಿದ್ದಾರೆ. ರಷ್ಯಾದ ವಿಜ್ಞಾನಿಗಳು ರೋಗದ ಕಾರಣವನ್ನು ಕಂಡು ಹಿಡಿಯುವಲ್ಲಿ ಕೀರ್ಲಿಯನ್ ವಿಧಾನವನ್ನು ಅನುಸರಿಸಿ ಯಶಸ್ವಿಯಾದ ವರದಿ ಬಂದಿದೆ.\footnote{\engfoot{The work on the Kirlian Effect's diagnostic properties alone is being conducted not only in Russia but in Rumania, Bulgaria, Hungary, Czecho- slovakia and East Germany by over 1000 high level scientists, physicists and biologists, medical doctors with a total staff of as many as 50,000 laboratory assistants.–}\hfill\engfoot{\textit{The New Soviet Psychic Discoveries}}}

ಪ್ರತಿಯೊಬ್ಬ ವ್ಯಕ್ತಿಯಲ್ಲೂ ಕಂಡುಬರುವ ಮೇಲೆ ಸೂಚಿಸಿದ ದಿವ್ಯ ಪ್ರಭೆಯ ಜೊತೆಗೆ ಮನಸ್ಸಿನ ಇನ್ನಿತರ ಅಸಾಮಾನ್ಯ ಶಕ್ತಿಗಳೂ ಅಡಗಿಕೊಂಡಿವೆ. ಇವುಗಳ ಸ್ಥೂಲ ಪರಿಚಯ ಮಾಡಿ ಕೊಳ್ಳುವುದಕ್ಕೆ ಮೊದಲು ಕೀರ್ಲಿಯನ್ ಉಪಕರಣಗಳ ಮೂಲಕ ವಿಜ್ಞಾನಿಗಳ ಅನುಭವಕ್ಕೆ ಬಂದೊದಗಿದ ಒಂದು ಮುಖ್ಯ ವಿಷಯವನ್ನು ಮರೆಯುವಂತಿಲ್ಲ. ಸೈಕೀ ಎಂದು ಕರೆಯಲಾದ ನಮ್ಮ ಮನಸ್ತತ್ವ ಅಥವಾ ಜೀವತತ್ವವು ಎಲುಬು ರಕ್ತ ಮಾಂಸ ನರಮಂಡಲಗಳಿಂದ ಕೂಡಿದ ನಮ್ಮ ಸ್ಥೂಲದೇಹದಿಂದ ಭಿನ್ನವಾದುದು. ಸತ್ತುಹೋದ ಪ್ರಾಣಿಗಳಲ್ಲಾಗಲೀ ಮನುಷ್ಯ ಶರೀರ ದಲ್ಲಾಗಲೀ ಪ್ರಭಾವಲಯದಿಂದ ಕೂಡಿದ ಈ ಶಕ್ತಿಶರೀರವು ಕಾಣಿಸುವುದಿಲ್ಲ. ಎಂದರೆ ಶಕ್ತಿ ಶರೀರದ ಅಸ್ತಿತ್ವವೇ ನಮ್ಮ ಸ್ಥೂಲ ದೇಹದ ಸಮಸ್ತ ಚಟುವಟಿಕೆಗಳಿಗೂ ಕಾರಣವೆಂದಾಯಿತು. ಈ ಶಕ್ತಿ ಶರೀರವನ್ನೇ ನಮ್ಮ ದೇಶದ ಮನೀಷಿಗಳು ಸೂಕ್ಷ್ಮಶರೀರವೆಂದು ನಿರ್ದೇಶಿಸಿದ್ದರು.

\vskip 3pt

ಮಿದುಳಿಗಿಂತ ಬೇರೆಯಾಗಿ ಮನಸ್ಸಿಗೆ ಸ್ವತಂತ್ರ ಅಸ್ತಿತ್ವವನ್ನು ಇಂದಿನವರೆಗೂ ಮನೋ ವಿಜ್ಞಾನಿಯಾಗಲಿ, ಇತರ ವಿಜ್ಞಾನವೆಂದು ಪರಿಗಣಿಸಲ್ಪಟ್ಟ ಶಾಸ್ತ್ರಗಳಾಗಲಿ ಒಪ್ಪಿಲ್ಲವೆಂದೇ ಹೇಳ ಬೇಕು. ಶತಮಾನಗಳಿಂದ ಈ ನಂಬಿಕೆಯ ದಾರಿಯಲ್ಲಿ ನಡೆದು ಬಂದ ವೈಜ್ಞಾನಿಕ ಮನೋ ವೃತ್ತಿಯ ಜನರು ಇದಕ್ಕೆ ವಿರೋಧವಾದುದೆಂದು ತೋರುವ ಹೊಸ ತಥ್ಯಗಳನ್ನು ಥಟ್ಟನೇ ಒಪ್ಪುವುದಿಲ್ಲ. ಇದು ವೈಜ್ಞಾನಿಕ ಕ್ಷೇತ್ರದಲ್ಲಿ ನಡೆದುಬಂದ ರೂಢಿ. ಆದರೆ ಇತ್ತೀಚೆಗೆ ಜಗತ್ಪ್ರಸಿದ್ಧ ನರವ್ಯೂಹ ತಜ್ಞನೂ, ಮಿದುಳಿನ ಶಸ್ತ್ರಚಿಕಿತ್ಸೆಯಲ್ಲಿ ಸಿದ್ಧಹಸ್ತನೂ ಆದ ಡಾ.\ ವೈಲ್ಡರ್ ಪೆನ್​ಫೀಲ್ಡ್ \enginline{\textit{Secret of Mind}} ಎನ್ನುವ ತನ್ನ ದೀರ್ಘಕಾಲದ ಅಧ್ಯಯನ ಅನುಭವಗಳ ಫಲವಾದ ಒಂದು ಗ್ರಂಥದಲ್ಲಿ ಮಿದುಳಿಗಿಂತ ಮನಸ್ಸು ಸ್ವತಂತ್ರ ಅಸ್ತಿತ್ವವನ್ನು ಹೊಂದಿದೆ ಎಂದು ಒಪ್ಪಿಕೊಳ್ಳುವ ಕಾಲ ಸನ್ನಿಹಿತವಾಗಿದೆ ಎನ್ನುತ್ತಾನೆ. ಕರಣಿಕ ಅಥವಾ ಕಾರ್ಯ ಯೋಜಕನೊಬ್ಬ ಯಾವಾಗಲೂ ಕಂಪ್ಯೂಟರನ್ನು ಉಪಯೋಗಿಸುತ್ತಿರಬಹುದು. ತನ್ನೆಲ್ಲ ಚಟುವಟಿಕೆಗಳಿಗೆ ಅದನ್ನೇ ಹೊಂದಿಸಿಕೊಂಡಿರಬಹುದು. ಆದರೆ ಆ ಕಂಪ್ಯೂಟರ್ ಉಪಕರಣವನ್ನು ಬಿಟ್ಟು ಆತ ಸ್ವತಂತ್ರವಾಗಿ ವರ್ತಿಸುತ್ತಾನಷ್ಟೆ. ಅದೇ ರೀತಿ ಮನಸ್ಸು ತನ್ನ ಎಲ್ಲ ಚಟುವಟಿಕೆಗಳಿಗೆ ಮಿದುಳನ್ನೇ ಹೊಂದಿಕೊಂಡಿದ್ದರೂ ತಾನು ಸ್ವತಂತ್ರವಾಗಿಯೂ ವರ್ತಿಸಬಲ್ಲದು ಎಂಬುದು ಪೆನ್ಫೀಲ್ಡನ ಅಭಿಪ್ರಾಯ.


\section*{ಮಲಗಿ ನಿದ್ರಿಪುದಲ್ಲಿ ಚೈತನ್ಯದಗ್ನಿ!}

\addsectiontoTOC{ಮಲಗಿ ನಿದ್ರಿಪುದಲ್ಲಿ ಚೈತನ್ಯ\-ದಗ್ನಿ!}

ನಮ್ಮ ಮನಸ್ಸಿನಲ್ಲಿ ಅತೀಂದ್ರಿಯ ಶಕ್ತಿಗಳು ಇವೆ, ಅವು ಕಲ್ಪನೆ ಅಥವಾ ಮೂಢನಂಬಿಕೆಯಲ್ಲ ಎಂಬುದನ್ನು ಸಾಕಷ್ಟು ಪರಿಶೀಲನೆ ಪ್ರಯೋಗಗಳ ಮೂಲಕ, ಸಾಕ್ಷಿ ಆಧಾರಗಳ ಮೂಲಕ ದೃಢ\-ಪಡಿಸಿಕೊಂಡಿದ್ದಾರೆ. ಅವು ಕೆಲವರಲ್ಲಿ ಬೇಗನೆ ವ್ಯಕ್ತವಾಗಬಹುದು. ಪ್ರಯತ್ನಿಸದಿದ್ದರೆ ಸ್ವಲ್ಪವೂ ವ್ಯಕ್ತವಾಗದಿರಬಹುದು. ಆದರೆ ಆ ಶಕ್ತಿಗಳ ಇರುವಿಕೆಯ ಬಗ್ಗೆ ಸಂಶಯವಿಲ್ಲ. ಆ ಶಕ್ತಿಗಳ ವೈವಿಧ್ಯವನ್ನು ಕುರಿತು ಸೂಕ್ಷ್ಮಪರಿಚಯ ಇಂತಿದೆ.\footnote{ಈ ಬಗ್ಕೆೃ ಹೆಚ್ಚಿನ ವಿವರಗಳಿಗಾಗಿ ಓದಿ: \engfoot{\textit{Future Science: Life Energies And Physics of paranormal Phenomenon}; Edited by John White and Stanley Kripner. Anchor Books: Doubleday and Co.\ In., Garden City, New York.}}

\textbf{\general{\enginline{(i)}} ದೂರಮನಸ್ಪರ್ಶ:} ಒಬ್ಬಾತನ ಮನಸ್ಸು ದೂರದಲ್ಲಿರುವ ಇನ್ನೊಬ್ಬ ವ್ಯಕ್ತಿಯ ಮನಸ್ಸಿ ನೊಂದಿಗೆ ಭಾವ ಪ್ರೇಷಣೆಯ ಮೂಲಕ ಸಂಪರ್ಕ ಸಾಧಿಸುವ ಶಕ್ತಿ.

\textbf{\general{\enginline{(ii)}} ದೇಹಾತೀತ ಅನುಭವ:} ದೇಹದಿಂದ ಹೊರಕ್ಕೆ ಸೂಕ್ಷ್ಮಶರೀರದಲ್ಲಿ ತೇಲಿ ಹೋಗುವ ಅನುಭವ ಮತ್ತು ದೂರದರ್ಶನ ಎಂದರೆ ದೇಹೇಂದ್ರಿಯಗಳು ಒಂದೆಡೆ ಇದ್ದರೂ ನೂರಾರು ಮೈಲಿ ದೂರದಲ್ಲಿ ನಡೆದ ಘಟನೆಗಳನ್ನು ಪ್ರತ್ಯಕ್ಷ ಕಂಡು ವಿವರಿಸುವ ಶಕ್ತಿ.

\textbf{\general{\enginline{(iii)}} ವಿಶಿಷ್ಟ ಅತೀಂದ್ರಿಯ ಶಕ್ತಿ: }ಮನುಷ್ಯ ಶರೀರದಲ್ಲಿರುವ, ಎಕ್ಸ್​ರೇ ಕಂಡು ಹಿಡಿಯಲಾಗದ ರೋಗವನ್ನು ಅತೀಂದ್ರಿಯ ದೃಷ್ಟಿಯಿಂದ ಗುರುತಿಸುವ ಶಕ್ತಿ. ಶರೀರದಿಂದ ಹೊರ ಹೊಮ್ಮುವ ಪ್ರಭಾವಲಯದ ವಿವಿಧ ವರ್ಣಗಳನ್ನು ಕಂಡು ಅರ್ಥೈಸುವ ಶಕ್ತಿ.

\textbf{\general{\enginline{(iv)}} ವ್ಯಕ್ತಿಯ ಅನುಪಸ್ಥಿತಿಯಲ್ಲಿ ಅವನು ಉಪಯೋಗಿಸಿದ್ದ ಯಾವುದೋ ಒಂದು ವಸ್ತುವನ್ನು ಸ್ಪರ್ಶಿಸಿಯೇ ಅವನೆಂಥವನೆಂಬುದನ್ನು, ಅವನ ರೂಪ ಗುಣಗಳನ್ನು ವರ್ಣಿಸುವ ಶಕ್ತಿ.}

\textbf{\general{\enginline{(v)}} ತನ್ನ ಹಿಂದಿನ ಜೀವನಗಳನ್ನೂ, ಇತರರ ಪೂರ್ವ ಜೀವನದ ಸ್ಮೃತಿಗಳನ್ನೂ ಹೇಳುವ ಶಕ್ತಿ.} ಸಾಮಾನ್ಯರ ಅರಿವಿಗೆ ಬಾರದ, ಮರೆತು ಹೋದ, ಬಾಲ್ಯದ, ಶೈಶವದ ಕೆಲವೊಮ್ಮೆ ಗರ್ಭವಾಸದ ವರೆಗಿನ, ಕೆಲವೊಮ್ಮೆ ಅದಕ್ಕೂ ಹಿಂದು ಹಿಂದಿನ ಜೀವನಗಳ ಅನುಭವಗಳ ಸ್ಮೃತಿಯನ್ನೂ ಎಚ್ಚರವಾಗುವಂತೆ ಮಾಡಿ ಅದನ್ನು ತಿಳಿದುಕೊಳ್ಳಲು ಸಾಧ್ಯ ಎಂದು ಈ ಕ್ಷೇತ್ರದಲ್ಲಿ ದುಡಿದ ತಜ್ಞರು ಹೇಳುತ್ತಾರೆ. \enginline{Age regression (}ಪ್ರತಿಗಮನ ವಿಧಾನ) ಎಂದು ಕರೆಯಲಾಗುವ ಈ ವಿಧಾನದ ಮೂಲಕ ಹಿಂದೆ ಪಡೆದ, ಆದರೆ ಪ್ರಕೃತ ಸರ್ವಥಾ ನೆನಪಿಡಲು ಸಾಧ್ಯವಿರದ ಅನುಭವಗಳನ್ನು ಸುಪ್ತ ಸ್ಥಿತಿಯಲ್ಲಿರುವ ವ್ಯಕ್ತಿಯು ಕಣ್ಣಾರೆ ಕಂಡು ಪ್ರತ್ಯಕ್ಷ ಅನುಭವಿಸಿದಂತೆ ಸ್ಪಷ್ಟವಾಗಿ ವಿವರಿಸಬಲ್ಲ. ಅವನ ಮಾನಸಿಕ ಸ್ಪಂದನಕ್ಕೆ ಶ್ರುತಿಗೂಡಿಸುವ ಸಾಮರ್ಥ್ಯವಿರುವ ಇನ್ನೊಬ್ಬನೂ ಆ ಅನುಭವಗಳನ್ನೇ ತಿಳಿಯಬಲ್ಲ. ಈ ಕ್ಷೇತ್ರಗಳಲ್ಲಿ ಸಂಶೋಧನಾನಿರತರಾದ ಅಂತರರಾಷ್ಟ್ರೀಯ ಖ್ಯಾತಿಯ ಪ್ರಸಿದ್ಧ ನಾಲ್ವರು ಮನೋರೋಗತಜ್ಞರು–ಅಲೆಗ್ಸಾಂಡರ್ ಕ್ಯೇನನ್ ಮತ್ತು ಡೆನ್ನಿಸ್ ಕೆಲ್​ಸಿ, ಇಂಗ್ಲೆಂಡಿನವರು; ವರ್​ವಾರಾ ಐವನೋವಾ, ರಷ್ಯನ್ ಮಹಿಳೆ; ಹೆಲನ್ ವಾಂಬಾಕ್, ಅಮೇರಿಕದ ಮನಶ್ಶಾಸ್ತ್ರಜ್ಞೆ. ಇವರು ಮಾಡಿದ ಪ್ರಯೋಗಾತ್ಮಕ ಸಂಶೋಧನೆಗಳು ನಿಜವಾಗಿಯೂ ಕುತೂಹಲಕಾರಿ ಮತ್ತು ಮಾನವನ ಅಂತರ್ಜಗತ್ತಿಗೊಂದು ಕಿಟಕಿ.

\textbf{\general{\enginline{(vi)}} ಭಾವನಾ ಪ್ರಸಾರದಿಂದ ರೋಗ ಗುಣಪಡಿಸುವ ಶಕ್ತಿ.}

\textbf{\general{\enginline{(vii)}} ಮಾನಸಿಕ ಬಲವನ್ನುಪಯೋಗಿಸಿ } ಹೊರಗಣ ವಸ್ತುವನ್ನು ಚಲಿಸುವಂತೆ ಮಾಡುವ ಶಕ್ತಿ.

\textbf{\general{\enginline{(viii)}} ದೇಹದಲ್ಲಿ ನಡೆಯುವ ಅನೈಚ್ಛಿಕ} ಕ್ರಿಯೆಗಳ ನಿಯಂತ್ರಣ ಶಕ್ತಿ. ಹೃತ್ಕ್ರಿಯೆಯನ್ನು ನಿಲ್ಲಿಸುವುದು, ನಾಡಿ ಬಡಿತವನ್ನು ಹೆಚ್ಚು ಕಡಿಮೆ ಮಾಡುವುದು, ಇತ್ಯಾದಿ.

\textbf{\general{\enginline{(ix)}} ಮಾನಸಿಕ ಕ್ರಿಯೆಯಿಂದಲೇ ಛಾಯಾಗ್ರಹಣದ ಫಲಕದ ಮೇಲೆ ಕಲ್ಪಿತ ಬಿಂಬವನ್ನು ಮೂಡಿಸುವ ಶಕ್ತಿ.}

ಇವುಗಳಲ್ಲಿ ಯೋಗಾಭ್ಯಾಸದಿಂದ ಸಾಧ್ಯವೆಂದು ಹೇಳಲಾದ ಸಿದ್ಧಿಗಳನ್ನು ಪ್ರಸ್ತಾಪಿಸಿಲ್ಲ.

ಈ ಶಕ್ತಿ ಸಿದ್ಧಿಗಳ, ದಿವ್ಯ ಪ್ರಭೆಯ ಹಿನ್ನೆಲೆಯಲ್ಲಿ ಏನಿದೆ? ಪರಂಜ್ಯೋತಿಯೇ ಇದೆ ಎಂಬುದು ಮುಂದಿನ ವಾಕ್ಯಗಳನ್ನು ಓದಿದಾಗ ನಿಮಗೆ ಸ್ಪಷ್ಟವಾಗಿ ಮನವರಿಕೆಯಾಗುವುದು.


\section*{ಸೂರ್ಯನನ್ನು ಬೆಳಗುವ ಬೆಳಕು}

\vskip -7pt\addsectiontoTOC{ಸೂರ್ಯನನ್ನು ಬೆಳಗುವ ಬೆಳಕು}

ಜಗತ್ತನ್ನು ಬೆಳಗುವ ಬೆಳಕು ಯಾವುದು? ಎನ್ನುವ ಪ್ರಶ್ನೆಯನ್ನು ಚಿಕ್ಕ ಬಾಲಕನೂ ಉತ್ತರಿಸ ಬಲ್ಲ. ಸೂರ್ಯನಲ್ಲವೇ ಜಗತ್ತನ್ನು ಬೆಳಗುವವನು? ಬೆಳಗುವವನು ಮಾತ್ರವಲ್ಲ, ಭರಣ– ಪೋಷಣಗಳನ್ನೂ ಮಾಡುವವನು. ಈಗ ಅಂಥ ಸೂರ್ಯನನ್ನೇ ಬೆಳಗುವ ಬೆಳಕು ಯಾವುದು ಎಂದು ಪ್ರಶ್ನಿಸಿದರೆ ನೀವು ಸ್ಪಲ್ಪ ಚಕಿತರಾಗಬಹುದು. ಥಟ್ಟನೇ ಉತ್ತರಿಸಲು ಸ್ವಲ್ಪ ಕಷ್ಟವೂ ಆಗಬಹುದು. ಆದರೆ ಸರಿಯಾಗಿ ಯೋಚಿಸಿದರೆ ಉತ್ತರ ನಾವೆಣಿಸಿದಷ್ಟು ಕಷ್ಟವಾಗದು. ಉಪನಿಷತ್ತುಗಳಲ್ಲಿ ಬೇರೆಬೇರೆ ರೀತಿಗಳಿಂದ ಈ ಪ್ರಶ್ನೆಯ ಉತ್ತರವನ್ನು ಬಹಳ ಮಾರ್ಮಿಕವಾಗಿಯೂ, ಸಮರ್ಪಕವಾಗಿಯೂ ನೀಡಿದ್ದಾರೆ.

ವಿದೇಹರಾಜನಾದ ಜನಕನು ಮಹಿರ್ಷಿ ಯಾಜ್ಞವಲ್ಕ್ಯರನ್ನು ಒಮ್ಮೆ ತನ್ನ ಆಸ್ಥಾನಕ್ಕೆ ಬರಮಾಡಿ ಕೊಂಡು ಯೋಗ್ಯರೀತಿಯಿಂದ ಅವರನ್ನು ಸತ್ಕರಿಸಿದ ಬಳಿಕ ಒಂದು ಪ್ರಶ್ನೆಯನ್ನು ಕೇಳಿದ:\break ‘ಮಹಾಶಯ, ಮನುಷ್ಯರಿಗೆ ಬೆಳಕು ಯಾವುದು?’ ಎಂಬುದೇ ಆ ಪ್ರಶ್ನೆ. ನಮ್ಮ ಎಲ್ಲ ವ್ಯವಹಾರಗಳೂ ಸಾಧ್ಯವಾಗಲು ಸಹಾಯಕವಾಗುವ ಬೆಳಕು ಯಾವುದು ಎಂಬುದು ಪ್ರಶ್ನೆಯ ಇಂಗಿತ. ಸಾಮಾನ್ಯ ಮನುಷ್ಯರನ್ನು ಗಮನದಲ್ಲಿರಿಸಿಕೊಂಡು ಈ ಪ್ರಶ್ನೆಯನ್ನು ಕೇಳಲಾಗಿದೆ. ಸಾಮಾನ್ಯ ಮನುಷ್ಯರೆಂದರೆ ಪ್ರಾಕೃತರು–ಎಂದರೆ ಅನಾಗರಿಕರೆಂದಲ್ಲ. ದೇಹವೇ ತಾನು ಎಂದು ತಿಳಿದು ಕೊಂಡವರು ಎಂಬುದನ್ನು ಇಲ್ಲಿ ಗಮನಿಸಬೇಕು. ಈ ಪ್ರಶ್ನೆಯ ಉತ್ತರವನ್ನು ಯಾಜ್ಞವಲ್ಕ್ಯರು ಹಂತ ಹಂತವಾಗಿ ಮೊದಲು ಹೊರಜಗತ್ತಿನಿಂದ ಪ್ರಾರಂಭಿಸಿ ಕೊನೆಗೆ ಅಂತರ್ಜಗತ್ತಿಗೆ ಹೇಗೆ ಕೊಂಡೊಯ್ಯುತ್ತಾರೆಂಬುದನ್ನು ನೋಡೋಣ.

ಅವರು ‘ಸೂರ್ಯನೇ ಬೆಳಕು’ ಎಂದರು. ಸೂರ್ಯನು ಉದಿಸಿದನೆಂದರೆ ಮನುಷ್ಯನು ಜಾಗೃತನಾಗಿ ನಾನಾ ರೀತಿಯ ವ್ಯವಹಾರಗಳಲ್ಲಿ ತೊಡಗುತ್ತಾನೆ ಸರಿ. ಸೂರ್ಯನು ಅಸ್ತಮಿಸಿದಾಗ ಮನುಷ್ಯನಿಗೆ ಯಾವುದು ಬೆಳಕು? ಚಂದ್ರನೇ ಆ ಬೆಳಕು ಎಂಬುದು ಈ ಪ್ರಶ್ನೆಯ ಉತ್ತರ. ನಿಜ, ಸೂರ್ಯನೂ, ಚಂದ್ರನೂ ಅಸ್ತಮಿತರಾದಾಗ ಯಾವುದು ಬೆಳಕು? ‘ಅಗ್ನಿಯೇ ಬೆಳಕು’ ಎಂದರು ಯಾಜ್ಞವಲ್ಕ್ಯರು. ಕತ್ತಲಾದಾಗ ದೀಪದ ಬೆಳಕಿನಲ್ಲಿ ನಾವು ದಾರಿ ನಡೆಯುತ್ತೇವೆ, ನಮ್ಮ ಎಲ್ಲ ಕೆಲಸ ಮಾಡುತ್ತೇವೆ. ಸರಿ, ಈಗ ಮುಂದಿನ ಪ್ರಶ್ನೆ–ಸೂರ್ಯಚಂದ್ರರೂ ಅಸ್ತಮಿಸಿದಾಗ, ಅಗ್ನಿಯೂ ಆರಿಹೋದಾಗ ಮನುಷ್ಯರಿಗೆ ಯಾವುದು ಬೆಳಕು? ‘ಶಬ್ದ’ವೇ ಬೆಳಕು ಎಂಬುದು ಉತ್ತರ.

ಶಬ್ದವು ಹೇಗೆ ಬೆಳಕು ಆಗಲು ಸಾಧ್ಯ? ಎನ್ನಬಹುದು ನೀವು. ಕಲ್ಪಿಸಿಕೊಳ್ಳಿ. ಅಮಾವಾಸ್ಯೆಯ ದಿನ. ಕತ್ತಲಾಗಿದೆ. ಆಕಾಶವನ್ನು ಕಾರ್ಮೋಡಗಳು ಮುಸುಕಿವೆ. ಸಾಲದ್ದಕ್ಕೆ ನೀವು ಪೇಟೆಯಿಂದ ಹಿಂದಿರುಗುವಾಗ ಧಾರಾಕಾರವಾಗಿ ಮಳೆಯೂ ಸುರಿಯುತ್ತಿದೆ. ಇದ್ದಕ್ಕಿದ್ದಂತೆ ದೀಪಗಳೆಲ್ಲ ನಂದಿ ಹೋಗಿವೆ. ಎಲ್ಲೆಲ್ಲೂ ಗಾಢಾಂಧಕಾರ. ನಮ್ಮ ಕೈಗಳೇ ನಮಗೆ ಕಾಣಿಸುವುದಿಲ್ಲ. ಅಂದರೆ ಯಾವ ಹೊರಗಿನ ಬೆಳಕೂ ನಮ್ಮ ಸಹಾಯಕ್ಕಿಲ್ಲ. ಏನು ಮಾಡುವುದೆಂದು ತೋಚದೆ ನಿಂತಲ್ಲೇ ನಿಂತುಕೊಳ್ಳುತ್ತೇವೆ. ಆಗ ಮನೆಯ ನಾಯಿ ಬೊಗಳುತ್ತದೆ. ಸರಿ, ನಮ್ಮ ಟೈಗರ್ ಕರೆಯುತ್ತಿದ್ದಾನೆ ಎಂದು ಶಬ್ದ ಬಂದ ದಿಕ್ಕನ್ನು ಅನುಸರಿಸಿ ಹೋಗಿ ಮನೆಯನ್ನು ಸೇರುತ್ತೇವೆ. ಇಲ್ಲಿ ಶಬ್ದವೇ ನಮ್ಮ ಪಾಲಿಗೆ ಬೆಳಕಾಯಿತಲ್ಲವೇ? ಆ ಶಬ್ದವು ನಮ್ಮ ಶ್ರೋತ್ರ ಇಂದ್ರಿಯವನ್ನು ಉದ್ದೀಪಿಸಿ ಮನಸ್ಸಿನಲ್ಲಿ ವಿವೇಕವನ್ನುಂಟು ಮಾಡಿ ನಮ್ಮನ್ನು ಕೆಲಸದಲ್ಲಿ ಪ್ರೇರಿಸಿತು. ಇದೇ ರೀತಿ ಕತ್ತಲಲ್ಲಿದ್ದಾಗ ನಮ್ಮ ಹೂದೋಟದಲ್ಲಿ ಅರಳಿದ ಹೂವಿನಸುವಾಸನೆಯೂ ನಮಗೆ ಬೆಳಕಾಗಬಹುದು. ಈಗ ನಾವು ಜಾಗ್ರತಾವಸ್ಥೆಯಲ್ಲಿ ಇಲ್ಲದಿದ್ದಾಗ ಯಾವುದು ಬೆಳಕು?

\vskip 2pt

ಎಷ್ಟೋ ಮಂದಿ ವಿಜ್ಞಾನಿಗಳು ಗಣಿತಜ್ಞರು ಸ್ವಪ್ನಗಳಲ್ಲಿ ತಮ್ಮ ಸಮಸ್ಯೆಗಳಿಗೆ ಪರಿಹಾರವನ್ನು, ಪ್ರಶ್ನೆಗಳಿಗೆ ಉತ್ತರವನ್ನು ಕಂಡ ಘಟನೆಗಳಿವೆ. ನಮ್ಮ ದೇಶದ ಮಹಾಗಣಿತಜ್ಞರಾದ ಶ‍್ರೀನಿವಾಸ ರಾಮಾನುಜನ್ ಅಂಥವರಲ್ಲಿ ಒಬ್ಬರು. ನಿದ್ರೆಯ ಹೊತ್ತಿನಲ್ಲಿ ನಮ್ಮ ಇಂದ್ರಿಯಗಳೆಲ್ಲ ತಮ್ಮ ಬಾಗಿಲುಗಳನ್ನು ಮುಚ್ಚಿವೆ. ಆಗ ಹೊರಗಿನ ಯಾವ ಬೆಳಕೂ ಅಲ್ಲಿ ಕೆಲಸ ಮಾಡುವುದಿಲ್ಲ. ನಿದ್ರಿಸುವ ವ್ಯಕ್ತಿ ಹೊರಜಗತ್ತಿನ ಎಲ್ಲ ಬೆಳಕುಗಳಿಗೂ ಅತೀತನಾಗಿದ್ದಾನೆ. ಆದರೂ ಆತನು ಗಣಿತದ ಸಮಸ್ಯೆಗಳನ್ನೋ, ವಿಜ್ಞಾನದ ಒಂದು ಒಗಟನ್ನೋ ಬಿಡಿಸುತ್ತಾನೆ. ಆ ಸೂಕ್ಷ್ಮ ಮಾನಸಿಕ ಕ್ರಿಯೆಗೆ ಸಹಾಯಕವಾದ ಬೆಳಕು ಯಾವುದು? ಅದು ಹೊರಗಿನ ಯಾವ ಬೆಳಕೂ ಅಲ್ಲವಷ್ಟೆ! ಗಾಢನಿದ್ರೆಯಲ್ಲಿ ಯಾವ ಅರಿವೂ ಇರುವುದಿಲ್ಲ ಎಂದು ನಾವು ಹೇಳಬಲ್ಲೆವೇ? ವಿಚಾರ ಮಾಡಿ ನೋಡಿದರೆ ಅನುಭವರಹಿತ ಜ್ಞಾನಾಭಾವದ ಅವಸ್ಥೆ ಅದಲ್ಲ. ‘ನಾನು ಸುಖವಾಗಿ ನಿದ್ದೆ ಹೋದೆ, ನನಗೆ ಏನೂ ಗೊತ್ತಾಗಲಿಲ್ಲ’ ಎನ್ನುವ ಮಾತಿನಲ್ಲಿ ಸುಖದ ಅನುಭವವೂ ಏನೂ ಗೊತ್ತಾಗಲಿಲ್ಲ ಎನ್ನುವುದರ ಅರಿವೂ ಅಡಗಿದೆಯಲ್ಲವೆ?

\vskip 2pt

೧೭೭೪ನೇ ಇಸವಿ ಸೆಪ್ಟೆಂಬರ್ ತಿಂಗಳ ೨೧ನೇ ತಾರೀಕಿನ ಪ್ರಾತಃಕಾಲ. ಪಾದ್ರಿ ಅಲ್ಫೆನ್ಸೊ ಡಿ ಲಗೌರಿ ಅವರು ಎರಿಜ್ಜೋ ಎಂಬ ಊರಿನ ಜೈಲಿನಲ್ಲಿ ಕೈದಿಗಳಿಗಾಗಿ ಸಾಮೂಹಿಕ ಪ್ರಾರ್ಥನೆಗೆ ಸಿದ್ಧತೆ ಮಾಡುತ್ತಿದ್ದವರು ಅಕಸ್ಮಾತ್ತಾಗಿ ಕುಸಿದು ಬಿದ್ದು ಗಾಢನಿದ್ರೆಯಲ್ಲಿ ಮುಳುಗಿದರು. ಸುಮಾರು ಎರಡು ಗಂಟೆಗಳ ಬಳಿಕ ಎಚ್ಚೆತ್ತು ತಾನು ರೋಮ್ ನಗರದಿಂದ ಆಗತಾನೇ ಹಿಂದಿರುಗಿದೆ\-ನೆಂದೂ, ಹದಿನಾಲ್ಕನೇ ಪೋಪ್ ಕ್ಲೆಮೆಂಟ್ ಮರಣ ಹೊಂದಿದುದನ್ನು ಕಂಡೆನೆಂದೂ, ಪ್ರಾರ್ಥನೆಯಲ್ಲಿ ಭಾಗವಹಿಸಿ ಬಂದೆನೆಂದೂ ಹೇಳಿದರು. ಮೊದಲಿಗೆ ಈ ಘಟನೆಯನ್ನು ವಿಚಾರವಂತರು\break ಸ್ವಪ್ನವೆಂದು ಬಗೆದರು. ಎರಡು ದಿನಗಳ ಬಳಿಕ ಪೋಪ್ ತೀರಿಕೊಂಡದ್ದು ನಿಜವೆಂಬ ವರ್ತಮಾನ ಬಂದಮೇಲೆ ಅದನ್ನು ‘ಆಕಸ್ಮಿಕ’ ಎಂದು ಕರೆದರು. ಪೋಪರ ಮೃತ್ಯುಶಯ್ಯೆಯ ಬಳಿ ಇದ್ದವರು ಅಲ್ಫೆನ್ಸೊ ಅವರನ್ನು ತಾವು ನೋಡಿ ಮಾತನಾಡಿದ್ದೆವು ಮಾತ್ರವಲ್ಲ, ಆ ದಿನದ ಪ್ರಾರ್ಥನೆಯನ್ನು ಮುನ್ನಡೆಸಿದವರೂ ಅವರೇ ಎಂಬುದನ್ನು ತಿಳಿಸಿದಾಗ ಎಲ್ಲರೂ ಮೂಕವಿಸ್ಮಿತರಾದರು!

\vskip 2pt

ರಕ್ತಮಾಂಸಗಳಿಂದ ಕೂಡಿ ದೇಹೇಂದ್ರಿಯ ನರಮಂಡಲಗಳ ಯಂತ್ರದಿಂದ ಜಿಗಿದು ದೇಶ ಕಾಲದ ಪರಿಮಿತಿಯನ್ನು ದಾಟಿ, ನೂರಾರು ಮೈಲು ದೂರದಲ್ಲಿದ್ದ ಪೋಪರ ಮೃತ್ಯುಶಯ್ಯೆಯ ಬಳಿ ಸಾರಿ, ಅಲ್ಲಿ ಪ್ರಾರ್ಥನೆ ಮಾಡಿ ಅಲ್ಲಿಂದ ಹಿಂದಿರುಗಿ ದೇಹವನ್ನು ಪ್ರವೇಶಿಸುವವರೆಗಿನ ಅನುಭವಗಳನ್ನೂ, ತನ್ನ ಸೂಕ್ಷ್ಮ ಶರೀರವನ್ನೂ, ಅದರ ಸಂಚಾರವನ್ನು ವೀಕ್ಷಿಸಿದ ಆ ಬೆಳಕು\break ಯಾವುದು?

ಸುಪ್ತ ನಿದ್ರೆಗೊಳಗಾದವರಲ್ಲಿ ಕೆಲವರು ತಮ್ಮ ಹಿಂದಿನ ಜನ್ಮದ ಘಟನೆಗಳ ವಿವರಗಳನ್ನು ತಿಳಿಸಿದುದಲ್ಲದೆ ಈ ಜೀವನದಲ್ಲಿ ಎಂದೂ ಕಲಿಯದ, ಹೆಸರನ್ನೂ ಕೇಳದ ಭಾಷೆಯಲ್ಲಿ ಮಾತ ನಾಡಿ ಮನಶ್ಶಾಸ್ತ್ರಜ್ಞರನ್ನು ಚಕಿತಗೊಳಿಸಿದ ಉದಾಹರಣೆಗಳಿವೆ. ಮನಸ್ಸಿನ ಸೂಕ್ಷ್ಮಾತಿಸೂಕ್ಷ್ಮ ಸ್ತರಗಳಲ್ಲಿ ಸಂಗ್ರಹವಾಗಿರುವ ಬೇರೆ ಬೇರೆ ದೇಹೇಂದ್ರಿಯಗಳಿಂದ ಪಡೆದ ಅನುಭವಗಳ ಸಂಸ್ಕಾರವನ್ನು ಬೆಳಗುವ ಆ ಬೆಳಕು ಯಾವುದು?

ಗೌತಮ ಬುದ್ಧನು ತನ್ನ ಹಿಂದಣ ಐನೂರು ಜನ್ಮಗಳ ಘಟನಾವಳಿಗಳನ್ನು ನೆನಪಿಸಿಕೊಳ್ಳ ಬಲ್ಲವನಾಗಿದ್ದನೆಂದು ಹೇಳಲಾಗಿದೆ. ಬೇರೆ ಬೇರೆ ಕಾಲಗಳಲ್ಲಿ ಬೇರೆ ಬೇರೆ ದೇಶಗಳಲ್ಲಿ ವಿಭಿನ್ನ ದೇಹೇಂದ್ರಿಯ ಮನಸ್ಸುಗಳಿಂದ ಪಡೆದ ಅನುಭವಗಳ ಸೂಕ್ಷ್ಮ ಸಂಸ್ಕಾರಗಳನ್ನು ಬೆಳಗುವ ಆ ನಿತ್ಯ ನಿರಂತರವಾದ ಬೆಳಕು ಯಾವುದು?

ಅದೇ ಸತ್ಯದ ಬೆಳಕು, ಸಚ್ಚಿದಾನಂದದ ಬೆಳಕು, ಸಾವಿಲ್ಲದ ನೋವಿಲ್ಲದ ಶಾಶ್ವತವಾದ ಅರಿವಿನ ಬೆಳಕು. ಶರೀರೇಂದ್ರಿಯಗಳಿಂದ ವ್ಯತಿರಿಕ್ತವಾಗಿದ್ದರೂ ಅವುಗಳನ್ನು ಬೆಳಗಿಸುತ್ತ ತಾನು ಮಾತ್ರ ಯಾರಿಂದಲೂ ಬೆಳಗದ ಆತ್ಮನ ಬೆಳಕು ಅದೇ. ಆ ಬೆಳಕು ಇರುವುದರಿಂದಲೇ ನಮಗೆ ಸೂರ್ಯನನ್ನು ನೋಡಲು ಸಾಧ್ಯವಾಯಿತು. ಆದುದರಿಂದ ಜಗತ್ತನ್ನು ಬೆಳಗುವ ಬೆಳಕಾದ ಸೂರ್ಯನನ್ನು ಬೆಳಗುವ ಅರಿವಿನ ಬೆಳಕು ಆತ್ಮದ ಬೆಳಕೇ. ಜಾತಿ ಮತ ಕುಲ ಗೋತ್ರ ಭೇದವಿಲ್ಲದೆ, ಬಲಿಷ್ಠ ದುರ್ಬಲ ಶ‍್ರೀಮಂತ ಬಡವ ಎಂಬ ತಾರತಮ್ಯವಿಲ್ಲದೆ ಪ್ರತಿಯೊಬ್ಬ ವ್ಯಕ್ತಿಯಲ್ಲೂ ಈ ಜ್ಯೋತಿ ಬೆಳಗುತ್ತಿದೆ. ಎಂದೆಂದೂ ನಂದದ ಅದ್ಭುತ ಅಮರಜ್ಯೋತಿ ಅದು. ಪ್ರತಿಯೊಬ್ಬ ವ್ಯಕ್ತಿಯ ಹಿನ್ನೆಲೆಯಲ್ಲಿರುವ ಪರಮಸತ್ಯ ಅದೊಂದೇ. ಅದು ಪರಿಪೂರ್ಣ. ಅದು ಪರಿಶುದ್ಧ. ಸಾವಿಲ್ಲದ ನೋವಿಲ್ಲದ ನಿತ್ಯ ನಿರಂತರವಾದ ದೇಶಕಾಲ, ಕಾರ್ಯಕಾರಣಗಳನ್ನು ಮೀರಿನಿಂತ ಸತ್ಯ ಅದು. ಸಂತನಲ್ಲೂ ಸಾಮಾನ್ಯನಲ್ಲೂ, ಸಂತುಷ್ಟನಲ್ಲೂ ದುಃಖಿತನಲ್ಲೂ, ಸುಂದರ ವ್ಯಕ್ತಿ\-ಯಲ್ಲೂ ಕುರೂಪಿಯಲ್ಲೂ, ನೀತಿನಿಷ್ಠನಲ್ಲೂ ನೀತಿಬಾಹಿರ ದುಷ್ಟನಲ್ಲೂ, ಮನುಷ್ಯರಲ್ಲೂ ಪ್ರಾಣಿಗಳಲ್ಲೂ, ಎಲ್ಲರಲ್ಲೂ ಎಲ್ಲೆಲ್ಲೂ ಬೆಳಗುತ್ತಿರುವ ಜ್ಯೋತಿ ಅದೊಂದೇ. ಅದನ್ನೇ ಆತ್ಮ ಎನ್ನುವುದು.

ನಮ್ಮೆಲ್ಲರಲ್ಲೂ ಅಡಗಿರುವ ಈ ಪರಂಜ್ಯೋತಿಯ ಸ್ವರೂಪ ಸ್ವಭಾವಗಳನ್ನು ತಿಳಿದುಕೊಂಡೆ ವೆಂದರೆ ಅತ್ಯಂತ ನೀಚರಿಗೂ ಮಹಾಪಾಪಿಗಳೆನ್ನಿಸಿಕೊಂಡವರಿಗೂ ಕೂಡ ಉನ್ನತಿಯ ಪಥದಲ್ಲಿ ಮುನ್ನಡೆಯಲು ಬೇಕಾದ ಅಪಾರ ಆತ್ಮವಿಶ್ವಾಸ, ಆಸೆ ಆಸಕ್ತಿಗಳು ಚಿಗುರಲು ಸಾಧ್ಯವಿದೆ ಎಂಬುದು ಒಂದು ಮಹಾಸತ್ಯ.


\section*{ಶಕ್ತಿಯ ಸೆಲೆ, ಮುಕ್ತಿಯ ನೆಲೆ}

\addsectiontoTOC{ಶಕ್ತಿಯ ಸೆಲೆ, ಮುಕ್ತಿಯ ನೆಲೆ}

{\parfillskip=0pt ‘ವ್ಯಕ್ತಿ, ಸಮಾಜ, ಅಷ್ಟೇಕೆ, ಇಡಿಯ ಜಗತ್ತನ್ನೇ ಉನ್ನತ ಸ್ಥಿತಿಗೊಯ್ಯುವ ಶಕ್ತಿ ಈ ಪರಮ\par}\newpage\noindent ಶ್ರೇಷ್ಠ ಸತ್ಯವಾದ ಆತ್ಮನಲ್ಲಿಡುವ ಶ್ರದ್ಧೆಯಿಂದ ಬರುವುದು’ ಎಂದು ಸ್ವಾಮಿ ವಿವೇಕಾನಂದರು ವೀರವೇದಾಂತದ ಕಹಳೆ ಊದಿದರು.

‘ಈ ಪರಮಶ್ರೇಷ್ಠ ಸತ್ಯವಾದ ಆತ್ಮನನ್ನು ನಾವು ನಂಬಬೇಕು. ಆ ನಂಬಿಕೆಯಿಂದ ನಮ್ಮಲ್ಲಿ ಶಕ್ತಿ ಉದಿಸುವುದು. ನೀವು ಏನನ್ನು ಯೋಚಿಸುವಿರೋ ಅದೇ ಆಗುವಿರಿ. ನೀವು ದುರ್ಬಲರೆಂದೇ ಯೋಚಿಸುತ್ತಿದ್ದರೆ ದುರ್ಬಲರೇ ಆಗುತ್ತೀರಿ. ನೀವು ಬಲಶಾಲಿಗಳೆಂದು ಯೋಚಿಸುತ್ತ ಬಲಶಾಲಿ\-ಗಳಾ\-ಗುವಿರಿ. ನೀವು ಅಶುದ್ಧರೆಂದು ಯೋಚಿಸುತ್ತಿದ್ದರೆ ಅಶುದ್ಧರಾಗುವಿರಿ. ಪರಿಶುದ್ಧರೆಂದು ಯೋಚಿಸುತ್ತಿದ್ದರೆ ಪರಿಶುದ್ಧರೇ ಆಗುವಿರಿ. ಈ ಮಾತು ನಾವು ನಮ್ಮನ್ನು ದುರ್ಬಲರೆಂದೂ ಪಾಪಿಗಳೆಂದೂ ತಿಳಿಯಬಾರದು ಎಂಬುದನ್ನು ಬೋಧಿಸುವುದು. ನನ್ನಲ್ಲಿ ಸರ್ವಶಕ್ತಿ ಇದೆ. ಅಗಾಧ ಜ್ಞಾನವಿದೆ. ನಾನು ಅದನ್ನು ಇನ್ನೂ ವ್ಯಕ್ತಗೊಳಿಸದೆ ಇದ್ದಿರಬಹುದು. ಆದರೆ ಅದರ ಇರುವಿಕೆಗೆ ಬಾಧಕವಿಲ್ಲ. ನನ್ನಲ್ಲಿ ಅದು ಇದ್ದೇ ಇದೆ. ಸರ್ವಜ್ಞಾನವೂ ಶಕ್ತಿಯೂ ಪವಿತ್ರತೆಯೂ ಸ್ವಾತಂತ್ರ್ಯವೂ ನನ್ನಲ್ಲಿದೆ. ನಾನೇಕೆ ಇದನ್ನು ವ್ಯಕ್ತಗೊಳಿಸುತ್ತಿಲ್ಲ? ಏಕೆಂದರೆ ನನಗೆ ಆ ವಿಷಯ ದಲ್ಲಿ ಅಥವಾ ಸತ್ಯ ಸಂಗತಿಯಲ್ಲಿ ನಂಬಿಕೆ ಅಥವಾ ಶ್ರದ್ಧೆ ಇಲ್ಲ. ನಾನು ದೃಢನಂಬಿಕೆಯನ್ನಿರಿಸಿ ದುದಾದರೆ ಅವು ಬಂದೇ ಬರಬೇಕು. ನಿರಾಕಾರ ಆತ್ಮಭಾವನೆ ಇದನ್ನು ಬೋಧಿಸುವುದು. ನಿಮ್ಮ ಮಕ್ಕಳನ್ನು ಬಾಲ್ಯಾರಭ್ಯ ಬಲಿಷ್ಠರನ್ನಾಗಿ ಮಾಡಿ. ಅವರಿಗೆ ದುರ್ಬಲತೆಯನ್ನು ಬೋಧಿಸಬೇಡಿ. ಶಕ್ತಿಯನ್ನು ನೀಡುವ ಬೋಧನೆ ಕೊಡಿ. ಅವರು ವಿಶ್ವವಿಜಯಿಗಳಾಗುವಂತೆ, ತಮ್ಮ ಕಾಲಮೇಲೆ ತಾವು ನಿಲ್ಲುವಂತೆ, ಸಕಲ ಕಷ್ಟಸಂಕಟ ವಿಘ್ನ ವಿಪತ್ತುಗಳನ್ನು ಎದುರಿಸಿ ಮುನ್ನಡೆಯುವ ಕೆಚ್ಚನ್ನು ಉಂಟುಮಾಡುವ ತರಬೇತು ನೀಡಿ. ಎಲ್ಲಕ್ಕೂ ಮೊದಲು ಈ ಆತ್ಮನ ಪರಮವೈಭವವನ್ನು ಅವರು ಮನಗಾಣಲಿ. ಸರಿಯಾಗಿ ತಿಳಿದುಕೊಳ್ಳಲಿ. ಈ ಅದ್ಭುತ ಭಾವನೆ ವೇದಾಂತದ ಕೊಡುಗೆ. ಪೂರ್ಣ ಜಗತ್ತನ್ನೇ ಕ್ರಾಂತಿಮಾಡಿ ಬದಲಿಸಬಲ್ಲ ಮತ್ತು ವಿಜ್ಞಾನ ಧರ್ಮಗಳೊಳಗೆ ಸಾಮರಸ್ಯ ತರಬಲ್ಲ ಭಾವನೆ ಇದು. ಸರ್ವತ್ರ ಸರ್ವರಿಗೂ ಉಪಯೋಗವಾಗುವ ಭಾವನೆ ಇದು’ ಎನ್ನುತ್ತಾ ಅವರು ನಮ್ಮಲ್ಲಿರುವ ಅಪಾರ ಶಕ್ತಿಯನ್ನು ಬೆಟ್ಟುಮಾಡಿ ತೋರಿಸುತ್ತಿದ್ದಾರೆ.


\section*{‘ನಾನು’ ಯಾರು?}

\addsectiontoTOC{‘ನಾನು’ ಯಾರು?}

ನಮ್ಮಲ್ಲಿ ಅಪಾರ ಶಕ್ತಿ ಇದೆ. ಆಂತರ್ಯದ ಆಳದಲ್ಲೇ ಚಿನ್ನದ ಗಣಿ ಅಥವಾ ಸ್ವರ್ಗದ ರಾಜ್ಯ ಇದೆ ಎಂಬುದನ್ನು ನಾವು ತಿಳಿದಿಲ್ಲ. ತನ್ನ ದೌರ್ಬಲ್ಯ ದೀನಹೀನತೆಗಳೇ ತಾನು ಎಂದು ತಿಳಿದು ಕೊಂಡು ಮನುಷ್ಯ ಎಷ್ಟೋ ವೇಳೆ ಮೂಢನಂತೆ ವರ್ತಿಸುತ್ತಾನೆ. ಕೆಲವೊಮ್ಮೆ ಬುದ್ಧಿಭ್ರಮಣೆ\break ಯಾದಂತೆ ನಡೆದುಕೊಳ್ಳುತ್ತಾನೆ. ಕೊಲೆಗಡುಕನಂತೆ ವರ್ತಿಸುತ್ತಾನೆ. ಹಾಗಾದರೆ ಆ ‘ನಾನು’ ನಿಜವಾಗಿ ಯಾರು?

ನಮ್ಮ ಸಾಮಾನ್ಯ ಅರಿವಿಗೆ ಬಾರದ, ಆದರೆ ನಮ್ಮಲ್ಲಿ ಆಗಲೇ ಇರುವ ಶಕ್ತಿಯ ನೆಲೆಯನ್ನು ತಿಳಿದಾಗ ನಾವು ನಮ್ಮ ಪರಿಮಿತಿಗಳಿಂದ ರಾಗ ದ್ವೇಷಗಳಿಂದ ಪಾರಾಗುತ್ತೇವೆ, ಸ್ವತಂತ್ರರಾಗು\-ತ್ತೇವೆ, ಪ್ರಾಜ್ಞರಾಗುತ್ತೇವೆ. ನಾವು ಯಾವುದೇ ಸ್ಥಾನದಲ್ಲಿರಲಿ ನಮ್ಮ ನಡೆನುಡಿ, ಆಚಾರ ವ್ಯವಹಾರಗಳಲ್ಲಿ ಆಗ ಆಶ್ಚರ್ಯಕರ ಬದಲಾವಣೆ ಪ್ರಾರಂಭವಾಗುತ್ತದೆ. ‘ನಿನ್ನೊಳಗೆ ನೀ ಹೊಕ್ಕು ನಿನ್ನನ್ನೇ ನೀ ಕಂಡು ನೀನೆ ನೀನಾಗು ಗೆಳೆಯ’ ಎಂದರು ಕವಿ ಬೇಂದ್ರೆ. ನೀನೇ ನೀನಾಗುವುದು ಅಥವಾ ನಾವೇ ನಾವಾಗು\-ವುದು ಎಂತು? ವಿಚಾರಯುಕ್ತಿ ಅನುಭವಗಳಿಂದ ನಮ್ಮ ನೈಜ ಸ್ವರೂಪವನ್ನು ತಿಳಿಯಲು ಯತ್ನಿಸಿದಾಗ ನಾವು ನಮ್ಮೊಳಗೆ ಮುಳುಗುತ್ತೇವೆ. ಆಳದಲ್ಲಡಗಿರುವ ಗಭೀರ ಜ್ಯೋತಿಯನ್ನು ಸಂದರ್ಶಿಸುತ್ತೇವೆ. ಇದು ಕಲ್ಪನೆಯಲ್ಲ. ಪರಮ ಶ್ರೇಷ್ಠ ಸತ್ಯ ಎಂಬುದನ್ನು ಮನಗಾಣುತ್ತೇವೆ.

ಈ ‘ನಾನು’ ಅನ್ನು ಛೇದಿಸಿ ಭೇದಿಸಿ ನೋಡಿದಾಗ ಅದರ ಹಿನ್ನೆಲೆಯಲ್ಲಿರುವ ಗೂಢತತ್ವ ಮತ್ತು ಶಕ್ತಿಯ ಪರಿಚಯ ನಮಗಾಗುವುದು. ಮಾತಿಗೆ ನಿಲುಕದ ಪರಮಾದ್ಭುತ ಶಕ್ತಿಯುಳ್ಳ ಅದು ಹೇಗೆ ತನ್ನನ್ನು ತಾನು ತನ್ನ ಬಗೆಗಿನ ಕಲ್ಪನೆಗಳ ಮೂಲಕ ಪರಿಮಿತಿಗೊಳಿಸಿಕೊಳ್ಳುತ್ತದೆ, ಪರಿಮಿತಿಯ ಭಾವನೆಗಳಲ್ಲಿ ಒಂದಾಗಿ ತನ್ನನ್ನು ತಾನು ಹೇಗೆ ಬದ್ಧವಾಗಿಸಿಕೊಂಡು ನರಳುತ್ತದೆ, ಆ ನರಳಾಟದಿಂದ ಹೇಗೆ ಪಾರಾಗುತ್ತದೆ ಎಂಬುದು ಅತ್ಯಂತ ಕುತೂಹಲಕಾರಿಯೂ ಆಶ್ಚರ್ಯದಾಯಕವೂ ಆದ ಅನ್ವೇಷಣೆ. ಈ ಅನ್ವೇಷಣೆ ಮಾಡಲು ಯಾವುದೇ ಪ್ರಯೋಗ ಶಾಲೆಗಳಲ್ಲಿ ನಾವು ದುಡಿಯಬೇಕಿಲ್ಲ. ನಮ್ಮ ಮನಸ್ಸೇ ಪ್ರಯೋಗಶಾಲೆ. ವಿಚಾರವೆಂಬ ಖಡ್ಗವೇ ಸಮಸ್ಯೆಯ ಕಗ್ಗಂಟನ್ನು ಕಡಿಯುವ ಉಪಕರಣ. ಆ ಖಡ್ಗ ಪ್ರಯೋಗದಿಂದ ನಮ್ಮಲ್ಲಿರುವ ಅಪಾರ ಶಕ್ತಿಯ ಅರಿವಾಗುವುದೆಂತು?


\section*{ಕುರಿಯಾದ ಹುಲಿ}

\addsectiontoTOC{ಕುರಿಯಾದ ಹುಲಿ}

ದೊಡ್ಡದಾದ ಹುಲ್ಲುಗಾವಲಿನಲ್ಲಿ ಒಂದು ಕುರಿಮಂದೆ ಹಾಯಾಗಿ ಸಂಚರಿಸುತ್ತಿತ್ತು. ಹುಲ್ಲು ಗಾವಲಿನ ಉತ್ತರ ದಿಕ್ಕಿನಲ್ಲಿ ಒಂದು ದಟ್ಟಡವಿ. ಒಂದಾನೊಂದು ದಿನ ಹಸಿದ ಹುಲಿಯೊಂದು ಆ ಅಡವಿಯಲ್ಲಿನ ಒಂದು ಬಂಡೆಯ ಮೇಲೆ ನಿಂತು ಹುಲ್ಲುಗಾವಲಿನಲ್ಲಿದ್ದ ಕುರಿಮಂದೆಯನ್ನು ಕಂಡಿತು. ‘ಇಂದೆನಗೆ ಆಹಾರ ಸಿಕ್ಕಿತು’ ಎಂದು ಅದು ಬಾಯಿ ಚಪ್ಪರಿಸಿತು. ಬಂಡೆಯ\break ಮೇಲಿನಿಂದ ಒಂದೇ ನೆಗೆತಕ್ಕೆ ಜಿಗಿದು ಆ ಹುಲಿ ಕುರಿಮಂದೆಯ ಹತ್ತಿರ ಬಂದು ಬಿದ್ದುಬಿಟ್ಟಿತು, ಬೀಳುವಾಗ ಅಕಸ್ಮಾತ್ ಅದರ ತಲೆಗೆ ಮೊನಚಾದ ಒಂದು ಕಲ್ಲು ಬಡಿದು ನೆತ್ತರು ಸುರಿಯಿತು. ತುಂಬು ಗರ್ಭವಾಗಿದ್ದ ಆ ಹುಲಿ ಒಂದು ಪುಟ್ಟ ಮರಿಯನ್ನು ಹೆತ್ತು ಸತ್ತು ಹೋಯಿತು. ಭಯದಿಂದ ದಿಕ್ಕು ಪಾಲಾಗಿ ಓಡಿ ಹೋಗಿದ್ದ ಕುರಿಗಳು ಹೆದರುತ್ತಲೆ ಮೆಲ್ಲಮೆಲ್ಲನೆ ಹಿಂದಿರುಗಿ ನೋಡಿದಾಗ ಹುಲಿ ಮಲಗಿಕೊಂಡೇ ಇರುವುದನ್ನು ಕಂಡವು. ಅದು ಸತ್ತುಹೋಗಿತ್ತು. ಬಹಳ ಹೊತ್ತಿನ ಮೇಲೆ ಹುಲಿಯನ್ನು ಸಮೀಪಿಸಿದವು. ‘ಅಯ್ಯೋ ಪಾಪ! ಹುಲಿ ಮರಿ ತಬ್ಬಲಿಯಾಯಿತು’ ಎಂದು ಕೊಂಡವು. ಮರಿಯನ್ನು ಕಂಡು ಕರುಣೆಯಿಂದ ಅದನ್ನು ನೆಕ್ಕಿ ಎತ್ತಿ ತಮ್ಮೊಡನೆ ಕೊಂಡೊಯ್ದವು. ಅದಕ್ಕೆ ಹಾಲು ಕುಡಿಸಿ ಹುಲ್ಲು ತಿನ್ನಿಸಿ ಸಾಕಿ ಸಲಹಿದವು. ಆ ಹುಲಿಮರಿ ತನ್ನನ್ನು ತಾನು ಕುರಿ ಎಂದೇ ತಿಳಿದಿತ್ತು. ಅದು ಕುರಿಗಳಂತೆಯೇ ‘ಬ್ಯಾ’, ‘ಬ್ಯಾ’ ಎಂದೇ ಕೂಗುತ್ತಿತ್ತು. ಕುರಿ ಕಾಯುವವನೂ ಈ ವಿಚಿತ್ರ ಘಟನೆಯನ್ನು ಕುತೂಹಲದಿಂದ ವೀಕ್ಷಿಸುತ್ತಿದ್ದ. ಒಂದು ದಿನ ಇತರ ಕುರಿಗಳ ಜೊತೆ ಈ ‘ಕುರಿ ಹುಲಿ’ಯೂ ಹುಲ್ಲು ಮೇಯುತ್ತಿದ್ದಾಗ ಅದೇ ಅರಣ್ಯಕ್ಕೆ ಬಂದ ಹೊಸ ಹುಲಿಯೊಂದು ಕುರಿಮಂದೆಯಲ್ಲಿ ತನ್ನವನೊಬ್ಬನನ್ನು ಕಂಡು ಅಚ್ಚರಿಪಟ್ಟಿತು. ಓಡಿ ಹೋಗಿ ತನ್ನನ್ನು ತಾನು ಕುರಿ ಎಂದೇ ತಿಳಿದುಕೊಂಡಿದ್ದ ಆ ‘ಕುರಿಹುಲಿ’ಯನ್ನು ಹಿಡಿದುತಂದು ‘ನೀನು ಹುಲಿ! ಅದೇಕೆ ಬ್ಯಾ ಬ್ಯಾ ಎಂದು ಅರುಚುತ್ತೀಯೆ! ಹುಲ್ಲು ತಿನ್ನುತ್ತಿದ್ದೀಯೇ?’ ಎಂದು ಗದರಿಸಿತು. ಆಗ ಆ ‘ಕುರಿಹುಲಿ’, ‘ಅಯ್ಯೋ! ನನ್ನನ್ನು ಬಿಟ್ಟುಬಿಡು. ನಾನು ಕುರಿಮರಿ. ನಾನು ಹುಲ್ಲು ತಿಂದುಕೊಂಡು ಹಾಯಾಗಿರುತ್ತೇನೆ’ ಎಂದಿತು. ಆಗ ಹಿರಿಯ ಹುಲಿ ‘ಛೀ, ಬುದ್ಧಿಹೀನ’ ಎನ್ನುತ್ತ ಆ ‘ಹುಲಿ ಕುರಿ’ಯನ್ನು ಒಂದು ಕೊಳದ ಹತ್ತಿರ ಕರೆದೊಯ್ದು ನೀರಿನಲ್ಲಿ ಬಿದ್ದ ಪ್ರತಿ ಬಿಂಬಗಳನ್ನು ತೋರಿಸಿ ಅವುಗಳಲ್ಲಿರುವ ಸಮಾನ ಗುಣಧರ್ಮವನ್ನು ಮನಗಾಣಿಸಿತು. ತನ್ನೊಂದಿಗೆ ಅರಣ್ಯಕ್ಕೆ ಕರೆದೊಯ್ದು ಬೇಟೆಯಾಡಿ ತೋರಿಸಿತು. ಹಲವು ಬಾರಿ ಗರ್ಜಿಸಿ ಓಡಾಡಿತು. ಕೆಲವು ದಿನಗಳವರೆಗೂ ಈ ‘ಕುರಿಹುಲಿ’ಗೆ ತನ್ನ ಬಗೆಗಿನ ಸಂಶಯ ಭಾವನೆ ದೂರವಾಗಿರಲಿಲ್ಲ. ಕ್ರಮೇಣ ತಾನು ಕುರಿಯಲ್ಲವೆಂಬ ಭಾವನೆ ದೃಢವಾಯಿತು. ಆಗ ಅದು ಹುಲಿಯಂತೆಯೆ ಓಡಾಡ ತೊಡಗಿತು.

ಹುಲಿಯಾಗಿದ್ದರೂ ತನ್ನನ್ನು ತಾನು ಕುರಿಯೆಂದೇ ನಂಬಿಕೊಂಡು ಕುರಿಯಂತೆಯೇ ವರ್ತಿ ಸಿತು ಆ ಹುಲಿಮರಿ! ತನ್ನಲ್ಲಿ ಅದ್ಭುತ ಶಕ್ತಿ ಅಡಗಿದ್ದರೂ ತಾನು ಅಲ್ಪಶಕ್ತಿಯ ಕುರಿ ಎಂಬ ಕಲ್ಪನೆಗೆ ದೃಢವಾಗಿ ಕಟ್ಟುಬಿದ್ದು ಕಿಂಚಿತ್ತೂ ಸಂಶಯವಿಲ್ಲದೇ ಆ ಕಲ್ಪನೆಗೆ ಅನುಗುಣವಾಗಿಯೆ ನಡೆದುಕೊಂಡಿತು. ಕುರಿಯೆಂದು ತಿಳಿದು ಅಂತೆಯೇ ವರ್ತಿಸಿದಾಗಲೂ ಅದು ನಿಜವಾಗಿಯೂ ಹುಲಿಯಾಗಿಯೇ ಇತ್ತು. ಹಾಗಾದರೆ ಯಾವುದು ಅದನ್ನು ಕುರಿಯಾಗಿಸಿದ್ದು? ಹೆಪ್ಪುಗಟ್ಟಿದ ತಪ್ಪು ಕಲ್ಪನೆ ಅಥವಾ ಅಜ್ಞಾನ. ತನ್ನ ನಿಜವಾದ ಸ್ವರೂಪದ ಬಗೆಗಿನ ಅಜ್ಞಾನ. ಅಜ್ಞಾನ ಕಳೆದು ತನ್ನ ಸ್ವರೂಪ ದರ್ಶನವಾಗುತ್ತಲೇ ಎಲ್ಲ ಸಂಶಯ, ವಿಪರೀತ ಭಾವನೆಗಳೂ ದೂರವಾದವು. ಆಗ ಅದು ಹುಲಿಯಂತೆ ವರ್ತಿಸಿತು. ಆತ್ಮವಿಶ್ವಾಸವೆಂದರೆ ಅಹಂಕಾರವಲ್ಲ. ಅಹಂಕಾರ ಅಜ್ಞಾನದಿಂದ ಉಂಟಾಗುತ್ತದೆ; ಆತ್ಮವಿಶ್ವಾಸ ಪ್ರತಿಯೊಬ್ಬನಲ್ಲಿ ಅಡಗಿರುವ ಶಕ್ತಿಯ ಅರಿವಿನಿಂದ ವೃದ್ಧಿಸುತ್ತದೆ.

ಯಾವುದನ್ನು ಸತ್ಯವೆಂದು ನಾವು ದೃಢವಾಗಿ ನಂಬುತ್ತೇವೋ ಆ ನಂಬಿಕೆ ನಮ್ಮ ಯೋಚನೆ, ಭಾವನೆ ಮತ್ತು ಚಟುವಟಿಕೆಗಳ ಮೇಲೆಯೇ ಏಕೆ–ಸಂಪೂರ್ಣ ವ್ಯಕ್ತಿತ್ವದ ಮೇಲೆ ಪ್ರಬಲ ಪ್ರಭಾವ ಬೀರುತ್ತದೆಂಬುದು ನಿಸ್ಸಂದಿಗ್ಧವಾದ ಒಂದು ಸತ್ಯ. ಸತ್ಯದ ಬಗೆಗಿನ ನಮ್ಮ ಕಲ್ಪನೆ ನಮ್ಮ ತಿಳಿ\-ವಳಿಕೆ\-ಯನ್ನೇ ಹೊಂದಿಕೊಂಡಿರುತ್ತದೆ. ಪರಮಸತ್ಯದ ಬಗೆಗಿನ ನಮ್ಮ ಸಿದ್ಧಾಂತ ನಮ್ಮ ಅರಿವಿನ ಪರಿಧಿಯ ವಿಸ್ತಾರ ವಿಕಾಸಗಳನ್ನವಲಂಬಿಸಿ ಬದಲಾವಣೆಯಾಗುತ್ತಿರುತ್ತದೆ ಎಂಬುದು ಇನ್ನೊಂದು ತಥ್ಯ.


\section*{ನೆರಳು—ಬೆಳಕು}

\addsectiontoTOC{ನೆರಳು—ಬೆಳಕು}

{\parfillskip=0pt ನಾನು, ನೀನು, ಅವನು, ಅವಳು, ಅದು ಎಂಬ ಭಾವನೆಗಳು ಸುಮಾರು ಎರಡು ವರ್ಷಗಳಾಗುವಾಗ ಮಗುವಿನಲ್ಲಿ ಉದಿಸುತ್ತವೆ ಎನ್ನುತ್ತಾರೆ. ಈ ‘ನಾನು’ವಿಗೆ ನಾನಾ ತೆರನಾದ ಪ್ರತ್ಯಯಗಳು \par}\newpage\noindent ಉಪಾಧಿಗಳು ಸೇರಿಕೊಂಡು ಸಾವಕಾಶವಾಗಿ ಹೇಗೆ ವ್ಯಕ್ತಿತ್ವ ನಿರ್ಮಾಣವಾಗುತ್ತದೆಂಬುದನ್ನು ನಾವು ಪರಿಶೀಲಿಸಬಹುದು. ಮನುಷ್ಯನ ವ್ಯಕ್ತಿತ್ವ ನಿರ್ಮಾಣದಲ್ಲಿ ಸಾಮಾನ್ಯ ದೃಷ್ಟಿಗೆ ಅಗೋಚರವಾದ ಕೆಲವು ದ್ರವ್ಯಗಳು ಸೇರಿಕೊಂಡಿವೆ. ನಮ್ಮೆದುರಿಗೆ ಇರುವ ವ್ಯಕ್ತಿಯ ದೇಹವನ್ನೇ ನಾವು ಕಾಣುತ್ತೇವೆ. ಕಣ್ಣುಗಳು ಚಲಿಸುವುದು, ಕಣ್ಣುಮುಚ್ಚಿ ತೆರೆಯುವುದು, ಬಾಯಿ ಮತ್ತು ತುಟಿಗಳ ಚಲನೆ, ಮುಖ ತಿರುಗಿಸುವುದು, ಯಾವುದೋ ಒಂದು ಕೆಲಸದಲ್ಲಿ ಇಡೀ ದೇಹದ ಚಲನೆ ಮತ್ತು ಚೇಷ್ಟೆ, ದೇಹದ ನಿಲುವು ಮತ್ತು ಮೈಬಣ್ಣ–ಇವುಗಳೆಲ್ಲ ನಮ್ಮ ಗಮನಕ್ಕೆ ಬರುವಂಥ ಸಂಗತಿಗಳು. ಆದರೆ ಮನುಷ್ಯನ ನಿಜವಾದ ವ್ಯಕ್ತಿತ್ವ ಅಗೋಚರವಾದುದಲ್ಲವೆ? ಮನುಷ್ಯನ ಹೊರ ರೂಪವನ್ನು ಸುಲಭವಾಗಿ ತಿಳಿದುಕೊಂಡಂತೆ ಅಗೋಚರವಾದ ಅಂಶಗಳನ್ನೂ ತಿಳಿಯುವಂತಾದರೆ ನಾವು ಒಂದು ಹೊಸ ಜಗತ್ತಿನಲ್ಲೇ ಬದುಕಿದಂತಾಗುತ್ತದೆ. ನಾವು ಬದುಕಿಕೊಂಡಿರುವುದು ನಮಗೆ ಗೋಚರಿಸುವ ಮಾನವ ಜಗತ್ತಿನಲ್ಲಿ, ಅಂದರೆ ತೋರಿಕೆಯ ಜಗತ್ತಿನಲ್ಲಿ.

ನಮ್ಮ ಯೋಚನೆಗಳು, ಭಾವನೆಗಳು, ಕಲ್ಪನೆಗಳು, ಸ್ಮೃತಿಗಳು, ಹಗಲುಗನಸುಗಳು, ಆಸೆ ಆಸಕ್ತಿ ಅನಿಸಿಕೆಗಳು ಅಗೋಚರವಾಗಿವೆ. ನಮ್ಮ ರಹಸ್ಯಗಳು, ಯೋಜನೆಗಳು, ಅಭೀಪ್ಸೆ ಆಕಾಂಕ್ಷೆಗಳು, ತರ್ಕವಿತರ್ಕಗಳು, ಶ್ರದ್ಧೆ ಸಂಶಯಗಳು, ಹಸಿವೆ ಹಂಬಲಗಳು, ರಾಗದ್ವೇಷಗಳು, ಬೇಕು ಬೇಡಗಳು–ಇವುಗಳೆಲ್ಲ ಅಗೋಚರವೇ. ಆದರೆ ಅವುಗಳೆಲ್ಲ ನಮ್ಮ ವ್ಯಕ್ತಿತ್ವದ ಮೂಲ\-ದ್ರವ್ಯ\-ಗಳಲ್ಲವೇ? ವ್ಯಕ್ತಿತ್ವ ನಿರ್ಮಾಣದ ಈ ಮೂಲದ್ರವ್ಯಗಳು ‘ಅಹಂ’ ಅಥವಾ ‘ನಾನು’ ಭಾವನೆಯನ್ನು ಸುತ್ತುವರಿದಿರುತ್ತವೆ. ನನ್ನ ಬಗೆಗಿನ ನಾನು ಬೆಳೆಸಿಕೊಂಡು ಬಂದಿರುವ ರೂಢಮೂಲವಾದ ಕಲ್ಪನೆಯ ತಾಯಿಬೇರಿನಿಂದ ಅವು ಪೋಷಿತವಾಗಿ ನನ್ನ ವ್ಯಕ್ತಿತ್ವ ಮತ್ತು ಬದುಕನ್ನು ರೂಪಿಸುತ್ತವೆ. ಬೆಳಕಿನಲ್ಲಿ ಕಂಗೊಳಿಸುವ ಭವ್ಯ ಕಟ್ಟಡದ ನೀಲಿನಕಾಶೆ ನೆರಳಿನಲ್ಲಿ ಎಲ್ಲೋ ಅಡಗಿರುವಂತೆ, ಗೋಚರಿಸುವ ವ್ಯಕ್ತಿಯ ಹಿನ್ನೆಲೆಯಲ್ಲಿ ಎಂದರೆ ಸುಪ್ತಮನದ ಆಳ ದಲ್ಲಿ ಅಡಗಿವೆ, ಈ ‘ನಾನು’ ಮತ್ತು ಅದನ್ನು ಸುತ್ತುವರಿದ ಯೋಚನೆ ಭಾವನೆ ಮತ್ತು ಸಂಸ್ಕಾರಗಳು.


\section*{ಸ್ವವ್ಯಕ್ತಿತ್ವ ಚಿತ್ರ}

\addsectiontoTOC{ಸ್ವವ್ಯಕ್ತಿತ್ವ ಚಿತ್ರ}

ಮ್ಯಾಕ್ಸ್​ವೆಲ್ ಮಾಲ್ಟ್ಸ್ ಜಗತ್ಪ್ರಸಿದ್ಧ ಪ್ಲಾಸ್ಟಿಕ್ ಸರ್ಜನ್. ಎಷ್ಟೋ ಮಂದಿ ಜಾಂಬವಂತರನ್ನು ಮನ್ಮಥರನ್ನಾಗಿ ಪರಿವರ್ತಿಸಿದ ಮಹಾನುಭಾವರು ಅವರು. ಅನೇಕ ‘ಅಡ್ಡಮೋರೆಯ ಗಂಟು ಮೂಗಿನ, ಗುಜ್ಜು ಗೊರಲಿನ ಬೊಜ್ಜು ದೇಹದ, ದೊಡ್ಡ ಕೈಕಾಲುಗಳ, ಉದುರಿದ ರೋಮ ಮೀಸೆಗಳ’ ಜನರ ಅಂಕುಡೊಂಕುಗಳನ್ನು ತಿದ್ದಿ ತೀಡಿದವರವರು. ದೈಹಿಕ ಅಂಕು ಡೊಂಕುಗಳು ದೂರವಾಗುತ್ತಿರುವಾಗಲೇ ಅವರ ಮನಸ್ಸಿನಲ್ಲಿ ಆತ್ಮವಿಶ್ವಾಸ ಧೈರ್ಯೋತ್ಸಾಹಗಳು ಗರಿಗೆದರುವುದನ್ನು ಡಾ.\ ಮಾಲ್ಟ್ಸ್ ಕಂಡಿದ್ದರು. ಅವರು ತಮ್ಮ ಬಗೆಗೆ ಕಲ್ಪಿಸಿಕೊಂಡಿದ್ದ ಕೀಳರಿಮೆಯ ಭಾವನೆಗಳು ದೂರವಾಗುವುದನ್ನು ನೋಡಿದ್ದರು. ಉದಾಹರಣೆಗೆ, ಸೀಳುತುಟಿಯ ಹುಡುಗನೊಬ್ಬ ತನ್ನ ಬಗೆಗೆ ನಾಚಿಕೊಂಡು ಉಳಿದವರಿಂದ ದೂರವಾಗಿ ಎಲ್ಲರೂ ತನ್ನನ್ನು ನೋಡಿ ಅಪಹಾಸ್ಯ ಮಾಡುತ್ತಾರೆಂದುಕೊಂಡು ಕೀಳರಿಮೆ ಬೆಳೆಸಿಕೊಂಡು ಮೂಲೆ ಸೇರುತ್ತಾನೆನ್ನಿ. ಶಸ್ತ್ರಚಿಕಿತ್ಸೆಯ ಕೆಲದಿನಗಳ ನಂತರ ಕನ್ನಡಿಯಲ್ಲಿ ತನ್ನ ಮುಖ ನೋಡಿಕೊಂಡು ಸಂತೋಷಪಟ್ಟು ತನ್ನ ವ್ಯಕ್ತಿತ್ವದ ಬಗೆಗೆ ಆತ್ಮವಿಶ್ವಾಸ ತಾಳಿ ಎಲ್ಲರೊಡನೆ ಬೆರೆತು ಅಧ್ಯಯನದಲ್ಲೂ, ಆಟೋಟಗಳಲ್ಲೂ ಮುನ್ನಡೆದು ಹೊಸ ವ್ಯಕ್ತಿಯೇ ಆದ. ಎಷ್ಟೋ ಮಂದಿಯಲ್ಲಿ ಈ ರೀತಿಯ ಶಸ್ತ್ರಚಿಕಿತ್ಸೆಗಳಿಂದ ಹಠಾತ್ತಾದ ಹಾಗೂ ಆಶ್ಚರ್ಯಕರ ಬದಲಾವಣೆಗಳು ಕಂಡುಬಂದಿವೆ. ಆದರೆ ಹಲವು ಸಂದರ್ಭಗಳಲ್ಲಿ ಶಸ್ತ್ರಚಿಕಿತ್ಸೆ ಯಶಸ್ವಿಯಾದರೂ, ಮುಖದ ಮೇಲಿನ ಅಂಕುಡೊಂಕುಗಳು ದೂರವಾಗಿ ಸೌಂದರ್ಯ ವೃದ್ಧಿಯಾದರೂ ವ್ಯಕ್ತಿಯಲ್ಲಿ ನಿರೀಕ್ಷಿಸಿದ ಯಾವ ಬದಲಾವಣೆಗಳೂ ಕಂಡುಬರಲಿಲ್ಲ. ಅವರ ಆತ್ಮವಿಶ್ವಾಸಹೀನತೆ ದೂರವಾಗಲಿಲ್ಲ. ಸೋಲಿನ ಮನೋಭಾವ ಅವರನ್ನು ಬಿಡಲಿಲ್ಲ. ನಿರಾಸೆ ಮತ್ತು ನಿರುತ್ಸಾಹಗಳಿಂದ ಅವರು ಪರಿತಪಿಸುತ್ತಿದ್ದರು. ಅವರ ಹೊರ ಮುಖವಲ್ಲ, ಮನಸ್ಸಿನ ಮುಖ ಬದಲಾಗಬೇಕಿತ್ತು. ಮನಸ್ಸಿನ ಆಳದಲ್ಲಿ ಬೇರೂರಿದ ತಮ್ಮ ವ್ಯಕ್ತಿತ್ವದ ಬಗೆಗಿನ ಅವರ ಕಲ್ಪನೆಗಳು ಬದಲಾಗಬೇಕಿತ್ತು ಎಂಬುದನ್ನು ಡಾ.\ ಮಾಲ್ಟ್ಸ್ ಕಂಡು ಕೊಳ್ಳಬೇಕಾಯಿತು. ಆ ಬದಲಾವಣೆಯಿಂದ ಅವರ ಬದುಕಿನಲ್ಲಿ ಅಪೂರ್ವ ಪರಿವರ್ತನೆಯಾದುದನ್ನು ಕಂಡು ಒಂದು ಸಿದ್ಧಾಂತವನ್ನು ಕಂಡುಕೊಂಡರು. ಅದೇ ‘ಸ್ವವ್ಯಕ್ತಿತ್ವಚಿತ್ರ\enginline{’ (Self-image)}. ಅದು ಬದಲಾವಣೆಯಾದರೆ ವ್ಯಕ್ತಿಯಲ್ಲಿ ಅಪೂರ್ವ ತಿರುವು ಸಾಧ್ಯವಾಗುತ್ತದೆಂಬುದನ್ನು ಸಹಸ್ರಾರು ಪ್ರಯೋಗ ಅನುಭವಗಳಿಂದ ಕಂಡುಕೊಂಡು ಒಂದು ಗ್ರಂಥವನ್ನೇ ಬರೆದರು. ಈ ಪದ್ಧತಿಯ ಪ್ರಯೋಜನ ಪಡೆದುಕೊಂಡ ಮೇಲೆ, ದಡ್ಡರೆಂದು ಪರಿಗಣಿತರಾಗಿ ಫೇಲಾ ಗಿದ್ದ ವಿದ್ಯಾರ್ಥಿಗಳು ಪ್ರಥಮಶ್ರೇಣಿಯಲ್ಲಿ ಪಾಸಾದರು. ಹುಟ್ಟುಗಟ್ಟಿದ ಸೋಲಿನ ಮನೋ ಭಾವದವರು ಉತ್ಸಾಹಿಗಳಾಗಿ ಉದ್ಯಮಶೀಲರಾದರು. ದಕ್ಷತೆ ಸಾಮರ್ಥ್ಯಗಳಲ್ಲಿ ಹಿಂದೆ ಬಿದ್ದಿದ್ದವರು ಉನ್ನತ ಮಟ್ಟಕ್ಕೇರಿದರು. ಅಂತರ್ಮುಖತೆ, ನಾಚಿಕೆಯ ಸ್ವಭಾವದವರು ಕೂಡ ನಿರ್ಭೀತಿಯಿಂದ ಎಲ್ಲರೊಂದಿಗೆ ವ್ಯವಹರಿಸತೊಡಗಿದರು.

ಎಲ್ಲರ ಬದುಕಿಗೂ ಉಪಯುಕ್ತವಾಗುವಂಥ, ಮುಖ್ಯವಾಗಿ ಹಿಂದುಳಿದವರಿಗೆ ತಾರಕವಾಗು ವಂಥ ಒಂದು ಮನೋವೈಜ್ಞಾನಿಕ ಸಂಶೋಧನೆ ಅದು. ನಮಗೆ ಗೊತ್ತಿರಲಿ ಬಿಡಲಿ, ನಮ್ಮ ಬಗೆಗೆ ಪ್ರತಿಯೊಬ್ಬರೂ ಮಾನಸಿಕವಾಗಿ ಒಂದು ನೀಲಿನಕಾಶೆ ಅಥವಾ ಚಿತ್ರ ಮಾಡಿಕೊಂಡಿರುತ್ತೇವೆ. ಇದನ್ನು ಆಂಗ್ಲ ಭಾಷೆಯಲ್ಲಿ \enginline{Self-image (}ಸೆಲ್ಫ್ ಇಮೇಜ್​) ಎನ್ನುತ್ತಾರೆ. ಇದು ನಮ್ಮ ಸಾಮಾನ್ಯ ಅರಿವಿಗೆ ಗೋಚರಿಸುವುದಿಲ್ಲ. ನಾನೊಬ್ಬ ಇಂತಿಂಥ ಜಾತಿಮತ ಕುಲಕ್ಕೆ ಸೇರಿದ, ಇಂತಿಂಥ ರೀತಿಯ ಮನುಷ್ಯ ಎಂಬ ಭಾವನೆ ನಮ್ಮಲ್ಲಿ ಪ್ರತಿಯೊಬ್ಬರಲ್ಲೂ ಇದೆ. ಸಾಮಾನ್ಯವಾಗಿ ಇದು ನಮಗೆ ಗೋಚರಿಸುವುದಿಲ್ಲ. ಆದರೆ ನಮ್ಮ ಮನಸ್ಸಿನ ಆಳದಲ್ಲಂತೂ ನಿಸ್ಸಂಶಯವಾಗಿ ಅದರ ಅಸ್ತಿತ್ವವಿದೆ. ಅದು ಬಾಲ್ಯದಿಂದಲೇ ನಮ್ಮದೇ ಆದ ನಂಬಿಕೆಗಳಿಂದ ಬಲಿತು ಬೆಳೆದಿರುತ್ತದೆ. ಕೆಲವೊಂದು ನಂಬಿಕೆಗಳು ಸರಿಯಾಗಿ ಬುದ್ಧಿತಿಳಿಯುವುದಕ್ಕೂ ಮೊದಲಿನಿಂದಲೇ ಬಲಿತುಬಿಡುತ್ತವೆ. ನಾವು ಪಡೆದ ಅನುಭವಗಳು, ಸಾಧನೆ ಸಿದ್ಧಿಗಳು, ಗೆಲವು ಯಶಸ್ಸುಗಳು, ನೋವು ಆಶಾಭಂಗ ಅಪಮಾನಗಳು, ಸ್ಥಿತಿಗತಿಯಲ್ಲಿ ಉಂಟಾದ ಏರುಪೇರುಗಳು, ಇತರರು ನಮ್ಮ ಮೇಲೆ ಬಾಲ್ಯದಿಂದಲೇ ತೋರುತ್ತಿದ್ದ ಪ್ರತಿಕ್ರಿಯೆಗಳು–ಇವೆಲ್ಲವುಗಳಿಂದ ಈ ನಂಬಿಕೆಗಳು ರೂಪತಾಳುತ್ತವೆ. ಒಟ್ಟು ಅವುಗಳ ಮೊತ್ತದಿಂದ ನಮ್ಮದೇ ಆದ ವ್ಯಕ್ತಿತ್ವದ ಚಿತ್ರವೊಂದು ನಮ್ಮ ಸಾಮಾನ್ಯ ಅರಿವಿಗೆ ಗೋಚರಿಸದೇ ಸಿದ್ಧವಾಗಿರುತ್ತದೆ. ಅದಕ್ಕನುಗುಣವಾಗಿಯೇ ನಾವು ವರ್ತಿಸುತ್ತೇವೆ ಎನ್ನುತ್ತಾರೆ ಮನೋವಿಜ್ಞಾನಿಗಳು.

ನಮ್ಮ ನಡತೆ, ಕಾರ್ಯವಿಧಾನ, ಭಾವನೆ ವರ್ತನೆಯ ಕ್ರಮಗಳು ಸಾಮರ್ಥ್ಯ ವೈಶಿಷ್ಟ್ಯಗಳೂ ಕೂಡ ದೃಢವಾಗಿ ಬೇರುಬಿಟ್ಟ ಸ್ವವ್ಯಕ್ತಿತ್ವ ಚಿತ್ರ ಅಥವಾ ನೀಲಿ ನಕಾಶೆಗಳಿಗನುಗುಣವಾಗಿಯೇ ಇರುತ್ತವೆ. ನಮ್ಮನ್ನು ನಾವು ಎಂಥ ವ್ಯಕ್ತಿ ಎಂದು ಮನಸ್ಸಿನ ಆಳದಲ್ಲಿ ಚಿತ್ರಿಸಿಕೊಂಡಿರುತ್ತೇವೋ ಅದಕ್ಕನುಗುಣವಾಗಿಯೇ ವರ್ತಿಸುತ್ತೇವೆ. \footnote{\engfoot{All Your actions, feelings, behaviour, even your abilities are always consistent with this self-image. In short, your will “act like” the sort of person you conceive yourself to be, not only this, but you literally cannot act otherwise, in spite of all your conscious efforts or will-power.}\hfill\engfoot{ –Dr. Maxwell Maltz, Psycho-Cybernatics}}

ಮೇಲಿನ ಈ ಸಿದ್ಧಾಂತ ಎಷ್ಟರ ಮಟ್ಟಿಗೆ ಸತ್ಯಸಂಗತಿ? ವ್ಯಕ್ತಿತ್ವ ನಿರ್ಮಾಣದಲ್ಲಿ ‘ನಾನು’ ಕಲ್ಪನೆ ಎಂಥ ಪಾತ್ರವಹಿಸುತ್ತದೆ? ಎಂಬುದನ್ನು ಅರ್ಥಮಾಡಿಕೊಳ್ಳಲು ಮುಂದಿನ ಘಟನೆ ಸಹಕಾರಿಯಾಗುತ್ತದೆ–


\section*{ಅಗಸನಾದ ಅರಸುಪುತ್ರ}

\addsectiontoTOC{ಅಗಸನಾದ ಅರಸುಪುತ್ರ}

ವಿಷಗಳಿಗೆಯಲ್ಲಿ ಜನಿಸಿದ ರಾಜನ ಶಿಶು ಮುಂದೆ ಕುಲಕ್ಕೆ ಕುಠಾರಪ್ರಾಯನಾಗುತ್ತಾನೆಂದು ಭವಿಷ್ಯಜ್ಞರಿಂದ ತಿಳಿದ ರಾಜನು, ಸೇವಕರಿಗೆ ಶಿಶುವನ್ನು ಕಾಡಿನಲ್ಲಿ ಬಿಟ್ಟು ಬರಲು ಆಜ್ಞಾಪಿಸು ತ್ತಾನೆ. ಸೇವಕರು ರಾಜಾಜ್ಞೆಯನ್ನು ಪಾಲಿಸಿ ಹಿಂದಿರುಗಿದರು. ಮರದಡಿಯಲ್ಲಿ ಬಂಡೆಯ ಮೇಲೆ ಅತ್ತು ಬಳಲಿ ಮಲಗಿ ನಿದ್ರಿಸುತ್ತಿದ್ದ ಸುಂದರ ಶಿಶುವನ್ನು ಅಕಸ್ಮಾತ್ತಾಗಿ ಆ ದಾರಿಯಲ್ಲಿ ನಡೆದು ಬರುತ್ತಿದ್ದ ಅಗಸ ದಂಪತಿಗಳು ಕಂಡು ಮೂಕವಿಸ್ಮಿತರಾದರು. ಬಹುಕಾಲದಿಂದ ಮಕ್ಕಳಿಲ್ಲದ ತಮಗೆ ದೇವರೇ ಆ ಶಿಶುವನ್ನು ಕರುಣಿಸಿದನೆಂದು ತಿಳಿದು ಅದನ್ನು ತಮ್ಮ ಮನೆಗೆ ಕರೆದೊಯ್ದು ಸಾಕಿ ಸಲಹಿದರು. ತಮ್ಮ ಹೊಟ್ಟೆಯಲ್ಲಿ ಹುಟ್ಟಿದ ಮಗನಲ್ಲ, ಸಾಕುಮಗ ಎಂಬುದನ್ನು ಅವನಿಗೆ ಎಂದೂ ತಿಳಿಯಗೊಡಲಿಲ್ಲ. ಮಗು ಬೆಳೆದು ದೊಡ್ಡವನಾದ. ತಂದೆತಾಯಿಗಳ ಕೆಲಸದಲ್ಲಿ ನೆರವಾಗತೊಡಗಿದ. ಕತ್ತೆ ಕಾಯುತ್ತಿದ್ದ. ಬಟ್ಟೆ ಒಗೆಯುತ್ತಿದ್ದ. ಕತ್ತೆ ಕಾಯುತ್ತ ಒಂದು ದಿನ ಮರದಡಿಯ ಕಲ್ಲಿನ ಮೇಲೆ ಮಲಗಿಕೊಂಡಿದ್ದ. ಸಂತರೊಬ್ಬರು ಆ ಮಾರ್ಗವಾಗಿ ಬರುತ್ತ, ಮಲಗಿದ್ದ ಯುವಕನ ಅಂಗಲಕ್ಷಣಗಳನ್ನು ಕಂಡು ಚಕಿತರಾದರು. ಯುವಕನನ್ನು ಎಬ್ಬಿಸಿ ಅವನು ಯಾರ ಮಗ ಎಂದೆಲ್ಲ ವಿಚಾರಿಸಿದರು. ಯುವಕನಾದರೊ ತಾನು ಅಗಸನ ಮಗನೆಂದೂ, ಅಲ್ಲಿ ಕತ್ತೆಗಳನ್ನು ನೋಡಿಕೊಂಡಿರುವಾಗ ಆಯಾಸವಾಗಿ ಮಲಗಿಕೊಂಡಿರುವೆನೆಂದೂ ಹೇಳಿದ. ಅವರು ‘ನಿನ್ನ ಭವಿಷ್ಯದ ಯೋಜನೆಗಳೇನು?’ ಎಂದು ಕೇಳಿದರು. ‘ಇನ್ನೂ ಹೆಚ್ಚು ಕತ್ತೆಗಳನ್ನು ಕೊಂಡುಕೊಂಡು ನಮ್ಮ ಅಂಗಡಿಯನ್ನು ವಿಸ್ತರಿಸಬೇಕೆಂದಿದ್ದೇನೆ’ ಎಂದ. ಸಂತರು ಯುವಕನ ಸಾಕುತಂದೆಯಾದ ಅಗಸನನ್ನು ಗುಟ್ಟಾಗಿ ಕಂಡು ಯುವಕನ ಪೂರ್ವೋತ್ತರಗಳನ್ನೆಲ್ಲ ಕೇಳಿದರು. ‘ಯಾರ ಮಗು ಎಂದು ತಿಳಿಯದು, ಕಾಡಿನಲ್ಲಿ ಸಿಕ್ಕಿದ ವಿಚಾರ ಸತ್ಯ’ ಎಂದನಾತ. ಸಂತರು ರಾಜನನ್ನು ಕಂಡು ವಿಚಾರಿಸಿ ‘ರಾಜ್ಯಕ್ಕೆ ಆತನ ಪುತ್ರನಿಂದ ಕೆಡುಕಿಲ್ಲ, ಗಂಡಯೋಗ ಕಳೆದುಹೋಗಿದೆ –ಶುಭವಾಗುತ್ತದೆ. ಹಿಂದಕ್ಕೆ ಕರೆಸಿಕೊಳ್ಳಬೇಕು’ ಎಂದರು. ರಾಜನು ಅವರ ಮಾತಿಗೆ ಸಮ್ಮತಿಯನ್ನು ಸೂಚಿಸಿದ.

ಸದ್ಯ ತನ್ನನ್ನು ಅಗಸನ ಮಗನಾಗಿಯೇ ತಿಳಿದುಕೊಂಡ ಆ ಯುವಕನನ್ನು ‘ನೀನು ರಾಜಪುತ್ರ ಕಣಯ್ಯ’ ಎಂದರೆ ಆತ ಏನು ಹೇಳಿಯಾನು? ‘ರಾಜಪುತ್ರನೆಂದು ನನ್ನನ್ನೇಕೆ ಅಪಮಾನ ಮಾಡು ತ್ತೀರಿ?’ ಎಂದಾನಲ್ಲವೆ?

‘ನಿನ್ನನ್ನು ಅರಮನೆಗೆ ಕರೆದೊಯ್ಯುತ್ತೇನೆ’ ಎಂದರೆ ‘ಅಲ್ಲಿ ಒಗೆಯಲು ಬೇಕಷ್ಟು ಬಟ್ಟೆಗಳು ದೊರಕಬಹುದಲ್ಲವೇ’ ಎನ್ನುತ್ತಾನಲ್ಲವೆ?

ದೀರ್ಘಕಾಲದಿಂದ ಅನುಭವಕ್ಕೆ ಬಂದ, ಅವನ ಪಾಲಿಗೆ ಸತ್ಯವೆಂದು ಸಾಬೀತಾದ ಸಂಗತಿಯನ್ನು ಒಂದು ಮಾತಿನಿಂದ ಅಥವಾ ಬಹಳ ದೃಢವಾದ ಶ್ರದ್ಧೆಯಿಂದ ‘ನೀನು ರಾಜಪುತ್ರ’ ಎಂದು ಗಟ್ಟಿಯಾಗಿ ಉಚ್ಚರಿಸಿದ ಮಾತ್ರಕ್ಕೆ ಅವನು ಅದನ್ನು ಅರ್ಥವಿಸಿಕೊಂಡು ಸ್ವವ್ಯಕ್ತಿತ್ವ ಚಿತ್ರವನ್ನು ಬದಲಾಯಿಸಿಕೊಳ್ಳಲು ಸಾಧ್ಯವೆ? ಮಹಾಭಾರತದ ಕರ್ಣನ ಅಂತರ್ದ್ವಂದ್ವವನ್ನು ನಾವಿಲ್ಲಿ ನೆನಪಿಸಿಕೊಳ್ಳಬಹುದು. ದೀರ್ಘಕಾಲದಿಂದ ತನ್ನನ್ನು ತಾನು ಕುರಿಯೆಂದೇ ತಿಳಿದು ಕೊಂಡಿದ್ದ ಕತೆಯಲ್ಲಿ ಬರುವ ‘ಕುರಿಹುಲಿ’ಗೆ ‘ನೀನು ನಿಜವಾಗಿಯೂ ಹುಲಿ’ ಎಂದು ಕಾಡಿನ ಹುಲಿ ಹೇಳಿದ ಮಾತ್ರದಿಂದ ಅದಕ್ಕೆ ತನ್ನ ವ್ಯಕ್ತಿತ್ವ ಚಿತ್ರವನ್ನು ಬದಲಿಸಲು ಸಾಧ್ಯವಾಯಿತೆ?


\section*{ಉದ್ಧಾರದ ಹೆದ್ದಾರಿ}

ಸ್ವವ್ಯಕ್ತಿತ್ವ ಚಿತ್ರವನ್ನು ಖಂಡಿತವಾಗಿಯೂ ಬದಲಿಸಲು ಸಾಧ್ಯ. ದೀರ್ಘಕಾಲದಿಂದ ಯುವಕನಲ್ಲಿ ಹುಟ್ಟುಗಟ್ಟಿದ ಭಾವನೆಯನ್ನು ಹಂತಹಂತವಾಗಿ ತಪ್ಪು ಎಂಬುದನ್ನು ತಿಳಿಸಿಕೊಡಬೇಕು. ಯಾವುದನ್ನು ಅನುಭವ ಸಾಕ್ಷ್ಯಾಧಾರಗಳಿಂದ ಸತ್ಯವೆಂದು ತಿಳಿದುಕೊಂಡಿದ್ದನೊ ಅದನ್ನು ಇನ್ನೊಂದು ಅನುಭವ ಮತ್ತು ಸಾಕ್ಷ್ಯಾಧಾರಗಳಿಂದ ಸರಿಯಲ್ಲವೆಂದು ಸ್ಪಷ್ಟಪಡಿಸಬೇಕು.

ವಿದ್ಯಾರ್ಥಿಗಳಲ್ಲಿ ಬಹಳ ಮಂದಿ ತಪ್ಪಾಗಿ ಸ್ವವ್ಯಕ್ತಿತ್ವ ಚಿತ್ರವನ್ನು ನಿರ್ಮಿಸಿಕೊಳ್ಳುತ್ತಾರೆ. ತಮಗಿಂತ ಹೆಚ್ಚು ಬೇಗನೇ ವಿಷಯಗಳನ್ನು ಗ್ರಹಿಸುವ ಶಕ್ತಿ ಇರುವ ವಿದ್ಯಾರ್ಥಿಗಳೊಂದಿಗೆ ತಮ್ಮನ್ನು ಹೋಲಿಸಿಕೊಂಡು ತಾವು ಅಪ್ರಯೋಜಕ ವ್ಯಕ್ತಿಗಳು, ಕಲಿಕೆಯಲ್ಲಿ ತಮಗೆ ಅಂಥ ಬುದ್ಧಿ ಇಲ್ಲ ಎಂದು ತಿಳಿದುಕೊಳ್ಳಲು ಶುರುಮಾಡುತ್ತಾರೆ. ಪ್ರಾರಂಭದ ಸಣ್ಣಪುಟ್ಟ ತಪ್ಪುಗಳಾದಾಗ ಅಧ್ಯಾಪಕರ ನಿರುತ್ಸಾಹದಾಯಕ ಬೈಗುಳು ಭರ್ತ್ಸನೆಗಳಿಂದ ಹೆದರಿ ಆತ್ಮವಿಶ್ವಾಸ ಕಳೆದು ಕೊಳ್ಳುತ್ತಾರೆ. ಚೆನ್ನಾಗಿ ಕಲಿಯುವುದಿಲ್ಲ ಈತ–‘ತಲೆ ಇಲ್ಲ’ ಎಂಬ ಹಿರಿಯರ ಕಟು ನುಡಿಯೂ ಆಗಾಗ ಅವರ ಮನಸ್ಸಿ\-ನೊಳಗೆ ಪ್ರವೇಶಿಸಿರುತ್ತದೆ. ಬುದ್ಧಿವಂತನಾದ ಹಿರಿಯಣ್ಣನನ್ನೊ, ಚುರುಕುಬುದ್ಧಿಯ ಕಿರಿಯ ತಂಗಿಯನ್ನೊ ಹೋಲಿಸಿ ತನ್ನನ್ನು ಮೂಢನೆಂದು ಜರೆದ ತಮ್ಮ ತಂದೆಯ ಮಾತುಗಳು ಅವರ ಕಿವಿಯಲ್ಲಿ ಹಲವು ಬಾರಿ ಮೊಳಗಿ ನೋಯಿಸಿವೆ. ಹೀಗೆ ತಮ್ಮ ಬುದ್ಧಿಶಕ್ತಿ ಮತ್ತು ಅಧ್ಯಯನ ಸಾಮರ್ಥ್ಯದ ಬಗೆಗೆ ನಿಷೇಧಾತ್ಮಕ ಭಾವನೆಗಳನ್ನು ಮನಸ್ಸಿನೊಳಗೆ ಹರಿಯಗೊಡುತ್ತಾರೆ. ಒಮ್ಮೆ ಸ್ವವ್ಯಕ್ತಿತ್ವ ಚಿತ್ರದಲ್ಲಿ ನಿಷೇಧಾತ್ಮಕ ಭಾವನೆಗಳು ಸೇರಿಕೊಂಡ ವೆಂದರೆ ಅವರು ಎಷ್ಟೇ ಪ್ರಯತ್ನ\-ಪಟ್ಟರೂ, ಹೋರಾಟ ನಡೆಯಿಸಿದರೂ ಯಶಸ್ಸು ದೊರಕುವು ದಿಲ್ಲ. ಹುಟ್ಟು ಆಶಾವಾದಿಗಳಾದ ಯುವಕರ ಮನಸ್ಸು ಸೋಲಿನಿಂದ ಎಷ್ಟೊಂದು ಸಂಕಟಗ್ರಸ್ತವಾಗುತ್ತದೆ! ಬಹುಮಾನದ ಆಸೆ ತೋರಿಸಿದರೂ, ಧೈರ್ಯ ನೀಡಿದರೂ, ಆತ್ಮವಿಶ್ವಾಸ ಬೇಕು ಎಂದು ಸ್ಫೂರ್ತಿ ನೀಡಿದರೂ ಅವರಲ್ಲಿ ಬದಲಾವಣೆ ಸಾಧ್ಯವಿಲ್ಲ. ‘ನನಗೆ ಓದು ಬಾರದು. ನನ್ನ ಮನಸ್ಸಿಗೆ ಹಿಡಿಸುವುದಿಲ್ಲ’ ಎಂಬ ಮಾತಿನಲ್ಲಿ ‘ಇನ್ನೂ ಓದಲು ಯತ್ನಿಸಿ ವಿಫಲನಾಗಿ ಎಲ್ಲ ರಿಂದಲೂ ಅಪಹಾಸ್ಯಕ್ಕೆ ಗುರಿಯಾಗಲು ಸಾಧ್ಯವಿಲ್ಲ’ ಎಂಬ ನೋವೇ ಅಡಗಿರುತ್ತದೆ. ಆದರೆ ಇಂಥ ವ್ಯಕ್ತಿಗಳ ಸ್ವವ್ಯಕ್ತಿತ್ವ ಚಿತ್ರ ಬದಲಿಸಿದರೆ ಅವರು ಪ್ರಥಮ ಶ್ರೇಣಿಯಲ್ಲಿ ಪಾಸಾಗಬಲ್ಲರು. ಅವರ ‘ಅಹಂ’ಗೆ ಸ್ವಲ್ಪವೂ ಆಘಾತವಾಗದಂತೆ ತಾಳ್ಮೆ ಮತ್ತು ಪ್ರೀತಿಯಿಂದ ಮಾರ್ಗದರ್ಶನ ಮಾಡಬೇಕು. ಎಲ್ಲವನ್ನೂ ಪ್ರಾರಂಭದಿಂದಲೇ ಪ್ರಾರಂಭಿಸಬೇಕು. ಮೊದಲು ಸಣ್ಣಪುಟ್ಟ ಸಮಸ್ಯೆಗಳನ್ನು ಬಿಡಿಸಲು ಪ್ರೇರಣೆ ನೀಡಿ ಅದರಲ್ಲಿ ಅವರಿಗೆ ಜಯ ದೊರಕುವಂತೆ ಮಾಡಬೇಕು. ಜಯ ದೊರಕಿದಾಗಲೆಲ್ಲ ಪ್ರೋತ್ಸಾಹದ ಮಾತುಗಳನ್ನು ಹೇಳಬೇಕು. ಅಂಥ ನೂರಾರು ಸಣ್ಣ ಪುಟ್ಟ ವಿಜಯಗಳನ್ನು ಪಡೆಯುತ್ತಲೇ ದೀರ್ಘಕಾಲದಿಂದ ಮನೆಮಾಡಿಕೊಂಡಿದ್ದ ಅವರ ಅಪ ಜಯದ ನಿಷೇಧಾತ್ಮಕ ಭಾವನೆ ಮೆಲ್ಲನೆ ಮಾಯವಾಗತೊಡಗಿ ಸ್ವವ್ಯಕ್ತಿತ್ವ ಚಿತ್ರ ಬದಲಾಗುತ್ತದೆ. ಆತ್ಮವಿಶ್ವಾಸ ಮರಳುತ್ತದೆ. ನಿಜವಾಗಿಯೂ ಅವರು ಹೊಸ ವ್ಯಕ್ತಿಗಳಾಗುತ್ತಾರೆ.

ಸ್ವವ್ಯಕ್ತಿತ್ವ ಚಿತ್ರ ಅನುಭವಗಳಿಂದ ಸ್ಥಿರೀಕೃತವಾದುದು. ಅದನ್ನು ಇನ್ನಿತರ ರಚನಾತ್ಮಕ ಅನುಭವಗಳಿಂದ ಸರಿಪಡಿಸಲು ಸಾಧ್ಯ. ಒಟ್ಟಿನಲ್ಲಿ ‘ನಾನು’ ಚಿತ್ರವನ್ನು ಅನುಸರಿಸಿ ಬದುಕಿನ ವಿವಿಧ ಗುರಿಯ ನಿರ್ಧಾರ ನಡೆಯುತ್ತದೆ. ನಮ್ಮ ಆಸೆ ಆಕಾಂಕ್ಷೆಗಳು, ರಾಗದ್ವೇಷಗಳು, ನಡೆ ನುಡಿಗಳು ಆ ಚಿತ್ರವನ್ನು ಹೊಂದಿಕೊಂಡಿರುತ್ತವೆ.\footnote{\engfoot{You usually do not realise that your physical body is created by you at each moment as a direct result of your inner conception of what you are, or that it changes in important chemical and electro-magnetic ways with ever moving pace of your own thought.–}\hfill\engfoot{\textit{Seth Speaks,} The Eternal Validity of the Soul}}


\section*{ನಾನಾ ಬಣ್ಣಗಳಲ್ಲಿ ‘ನಾನು’}

\addsectiontoTOC{ನಾನಾ ಬಣ್ಣಗಳಲ್ಲಿ ‘ನಾನು’}

{\parfillskip=0pt ‘ನಾನು’ವಿನ ಬಗೆಗಿನ ಹುಟ್ಟುಗಟ್ಟಿದ ಕಲ್ಪನೆ ಅಥವಾ ರೂಢಮೂಲ ಭಾವನೆಗಳನ್ನು ಅನುಸರಿಸಿ\par}\newpage\noindent ಮನುಷ್ಯರ ನಡವಳಿಕೆಯಲ್ಲಿ, ಪ್ರವೃತ್ತಿಯಲ್ಲಿ ಹೇಗೆ ಬದಲಾವಣೆಯಾಗಬಹುದು ಎಂಬುದನ್ನು ಪರಿಶೀಲಿಸಿ. ಇದೇ ತಾದಾತ್ಮ್ಯ ಸಂಬಂಧ \enginline{(Identification Orientation)}. ಅವರವರ ಸ್ವವ್ಯಕ್ತಿತ್ವಚಿತ್ರಕ್ಕನುಗುಣವಾಗಿ ಪ್ರತಿಯೊಂದು ತಾದಾತ್ಮ್ಯವೂ ಬೇರೆ ಬೇರೆ ಪ್ರತಿಕ್ರಿಯೆಗಳನ್ನು ವ್ಯಕ್ತಿಯಲ್ಲಿ ಉಂಟುಮಾಡುತ್ತದೆ ಎಂಬುದನ್ನು ಕೆಳಗೆ ಸೂಚಿಸಿದ ಉದಾಹರಣೆಗಳಲ್ಲಿ ನೋಡ ಬಹುದು. ‘ನಾನು ಆರೋಗ್ಯವಾಗಿದ್ದೇನೆ’ ಎಂದುಕೊಳ್ಳುವ ವ್ಯಕ್ತಿ ತನ್ನ ಆರೋಗ್ಯವನ್ನು ಕುರಿತು ಮಾತ್ರ ಚಿಂತಿಸಿ ಸುಮ್ಮನಾಗುತ್ತಾನೆಯೇ? ತನ್ನ ಆರೋಗ್ಯದ ಬಗೆಗೆ ಅಭಿಮಾನವನ್ನು ಬೆಳೆಸಿ ಕೊಂಡ ಆತ, ವಿದ್ಯಾವಂತ, ಧನವಂತ ಜನಗಳಂತೆ ಗುಪ್ತವಾಗಿಯೋ ವ್ಯಕ್ತವಾಗಿಯೋ ಜಂಭವನ್ನು ಪ್ರದರ್ಶಿಸಬಹುದು. ತನ್ನ ಆರೋಗ್ಯದ ಬಗೆಗೆ ಯಾವುದೋ ಕಾರಣಗಳನ್ನು ಕೊಡಹೊರಡ ಬಹುದು. ನಾನಾ ತೆರನಾದ ರೋಗ ರುಜಿನಗಳಿಂದ ನರಳುವ ಜನರನ್ನು ಕೊಂಚ ತಿರಸ್ಕಾರಪೂರಿತ ದೃಷ್ಟಿಯಿಂದ ನೋಡಬಹುದು. ‘ನಾನು ಒಂದು ದಿನ ಮಾತ್ರೆ ಬಾಯಿಯಲ್ಲಿ ಹಾಕಿಕೊಂಡವನಲ್ಲ. ಡಾಕ್ಟರಿಗೆ ಸಲಾಮು ಹೊಡೆದವನಲ್ಲ’ ಎಂದು ಅಸೌಖ್ಯದಿಂದ ನರಳುವವನೆದುರೇ ಹೇಳಿಕೊಂಡು ಆತ್ಮತೃಪ್ತಿಯ ಕಿರುನಗೆಯನ್ನು ಬೀರಬಹುದು. ಈ ವ್ಯಕ್ತಿ ಆರೋಗ್ಯದೊಂದಿಗೆ ಸ್ವಲ್ಪ ಶ್ರಮವಹಿಸಿ ದುಡಿಯುವವನಾಗಿದ್ದು ಯಶಸ್ಸೂ ದೊರಕಿದರೆ ಸುತ್ತಮುತ್ತಲಿನ ಜನರನ್ನು ‘ಕೆಲಸಗಳ್ಳರು,\break ಸೋಮಾರಿಗಳು’ ಎಂದೂ ಕರೆಯತೊಡಗಬಹುದು. ‘ಚೆನ್ನಾಗಿ ಕೆಲಸ ಮಾಡಿದರೆ ರೋಗವೆಲ್ಲ ಓಟಕೀಳುತ್ತದೆ’ಎಂದೂ ಉಪದೇಶಿಸಬಹುದು. ಈ ನಡವಳಿಕೆಗಳೆಲ್ಲ ವ್ಯಕ್ತಿತ್ವ ನಿರ್ಮಾಣದ ಸಮಯದಲ್ಲಿ ಅಜ್ಞಾತವಾಗಿ ಸಂಗ್ರಹವಾದ ಯಾರದೋ ಅನುಕರಣೆಯೂ ಆಗಿರ ಬಹುದು.\break ಒಟ್ಟಿನಲ್ಲಿ ಸ್ವವ್ಯಕ್ತಿತ್ವದ ಚಿತ್ರಕ್ಕನುಗುಣವಾಗಿ ಅವನ ಯೋಚನಾಲಹರಿ, ವರ್ತನೆ, ವರ್ತನೆಯನ್ನು ಸಮರ್ಥಿಸಿಕೊಳ್ಳುವ ವಿಧಾನಗಳು ಪ್ರಕಟವಾಗುತ್ತವೆ.

ಮೂರ್ತಿಪೂಜಾ ವಿಧಾನವನ್ನು ಬಾಲ್ಯದಿಂದಲೇ ಅರ್ಥಹೀನ, ಮೂಢನಂಬಿಕೆ ಎಂದು\break ದ್ವೇಷಿಸುವಂತೆ ತರಬೇತಿಯನ್ನು ಪಡೆದು ಬಂದ ಮಗುವನ್ನೋ ಯುವಕನನ್ನೋ ಪರಿಶೀಲಿಸಿ. ತನ್ನ ನಂಬಿಕೆಯ, ತಾನು ಅನುಸರಿಸುವ ಧರ್ಮ ಮಾತ್ರ ಅತ್ಯಂತ ಶ್ರೇಷ್ಠ, ತನ್ನ ಧರ್ಮದ ಪೂಜಾ ವಿಧಾನಗಳು ಮಾತ್ರ ಸತ್ಯ, ಇತರರದು ಅರ್ಥಹೀನ ಎಂಬ ಭಾವನೆ ರೂಢಮೂಲವಾದರೆ ಮೂರ್ತಿ ಪೂಜಕರ ದೇವತಾ ಮೂರ್ತಿಗಳನ್ನು ಕಂಡಾಗ ಅವನಲ್ಲಿ ಗೌರವ ಮತ್ತು ಭಕ್ತಿಭಾವ ಉಂಟಾಗಲು ಸಾಧ್ಯವಾಗುವುದೇ? ಮೂರ್ತಿಪೂಜೆಯ ಅರ್ಥ, ಉಪಾಸನೆಯಲ್ಲಿ ಸಂಕೇತಗಳ ಪಾತ್ರ, ಸಾಧಕನ ಬದುಕಿನಲ್ಲಿ ಅವುಗಳ ಆವಶ್ಯಕತೆ, ಸರ್ವತ್ರ ಸರ್ವವ್ಯಾಪಿಯಾಗಿರುವ ಸರ್ವಶಕ್ತ ನಾದ ದೇವರು ಭಕ್ತನಿಗಾಗಿ ಮೂರ್ತಿಯಲ್ಲಿ ಆವಿರ್ಭವಿಸುವ ಸಾಧ್ಯತೆ–ಇವುಗಳನ್ನು ಕುರಿತು ನೀವು ವಿಶದವಾಗಿ ತಿಳಿಸಲು ಯತ್ನಿಸಿದರೂ ಆತನ ‘ನಾನು’ ಅದನ್ನು ಹೃತ್ಪೂರ್ವಕವಾಗಿ ಸ್ವೀಕರಿಸಲು ಸಾಧ್ಯವೇ? ಅವಕಾಶ ದೊರೆತರೆ ಮೂರ್ತಿಯನ್ನು ಕೆಡಹಿಬಿಡೋಣವೆಂದೂ ಆತನಿಗೆ ತೋರಬಹುದು. ಜನರ ನಿಂದೆ, ಟೀಕೆ, ಕೋಪ, ಪೋಲೀಸರ ಭರ್ತ್ಸನೆ ಅವನನ್ನು ಹಾಗೆ ಮಾಡದಂತೆ ತಡೆಯಬಹುದು. ಆದರೆ ಅವರ ಕಣ್ಣು ತಪ್ಪಿಸಿ ಮೂರ್ತಿಭಂಜನೆ ಮಾಡಿದಾಗ ಯಾವುದೋ ಪುಣ್ಯಕಾರ್ಯದಿಂದೊದಗುವ ಧನ್ಯತೆಯ ಭಾವವನ್ನು ಅವನು ಅನುಭವಿಸಬಹುದಲ್ಲವೇ?

ಅವನ ‘ಅಹಂ’ಗೆ ಬಲವಾಗಿ ಅಂಟಿಕೊಂಡ ಭಾವನೆಗಳು ಅವನನ್ನು ಆ ವಿಧ್ವಂಸಕ ಕಾರ್ಯ ಕ್ಕೆಳಸುವುವು. ಅವನು ಬೆಳೆದು ಬಂದ ವಾತಾವರಣದ ಪ್ರಭಾವ, ಅವನಿಗೆ ದೀರ್ಘಕಾಲದಿಂದ ಸಿಕ್ಕಿದ ತರಬೇತಿ ಪ್ರೇರಣೆ ಪ್ರೋತ್ಸಾಹಗಳ ಪರಿಧಿಯನ್ನು ದಾಟುವುದು ಅವನ ಪಾಲಿಗೆ ಸಾಧ್ಯವೇ? ಅವನು ದ್ವೇಷಕ್ಕೆ ಪಾತ್ರನೆ? ಕನಿಕರಕ್ಕೆ ಪಾತ್ರನೆ? ಸಾವಕಾಶವಾಗಿ ಅವನ ಸ್ವವ್ಯಕ್ತಿತ್ವ ಚಿತ್ರವನ್ನು ಬದಲಿಸಲು ಸಾಧ್ಯವಿಲ್ಲವೆ? ಯೋಚಿಸಿ ನೋಡಿ.


\section*{ಸುಧಾರಣೆಯ ಸೋಪಾನ}

\addsectiontoTOC{ಸುಧಾರಣೆಯ\break ಸೋಪಾನ}

ಸಮಸ್ಯೆಯ ಸಂಕೀರ್ಣತೆಯನ್ನಾಗಲೀ ರಚನಾತ್ಮಕವಾಗಿ ಏನು ಮಾಡಬೇಕೆಂಬುದನ್ನು ಕುರಿತು ಯಾವ ಯೋಜನೆ ಅಥವಾ ಸಿದ್ಧತೆಯನ್ನಾಗಲೀ ಮಾಡದೆ, ಪರಿಣತರ ಸಲಹೆಯನ್ನು ಪಡೆಯದೆ, ನಿಜವಾಗಿ ಆ ಕ್ಷೇತ್ರದಲ್ಲಿ ಪ್ರಾಮಾಣಿಕರಾಗಿ ದುಡಿದವರಿದ್ದಾರೆಂದೂ ಅರಿತುಕೊಳ್ಳದೆ, ನಮ್ಮಲ್ಲಿ ಕ್ರಾಂತಿ ಮನೋಭಾವದವರೆನ್ನಿಸಿಕೊಂಡ ಜನ ವೇದಿಕೆಯ ಮೇಲೆ ನಿಂತು ಜಾತೀಯತೆಯನ್ನು ನಾಶಮಾಡಿ ಎಂದು ಭಾಷಣ ಬಿಗಿಯುತ್ತಾರೆ. ಹಾಗೆ ಭಾಷಣ ಮಾಡುವವರು ತಾವು ಯಾವ ಜಾತಿಯವರೆಂಬುದನ್ನು ನೇರವಾಗಿ ಹೇಳದಿದ್ದರೂ ತಮ್ಮ ಮಾತಿನಿಂದ ಅಪ್ರತ್ಯಕ್ಷವಾಗಿ ತಮಗರಿ\-ವಿಲ್ಲದೆ ಕ್ಷಣ ಕ್ಷಣವೂ ವ್ಯಕ್ತಗೊಳಿಸುತ್ತಾರಷ್ಟೆ. ತಾವು ನಿಜವಾಗಿ ಜಾತ್ಯತೀತರಾಗಿ ಅವರು ‘ಜಾತಿಯನ್ನು ನಿರ್ಮೂಲಿಸಿ’, ಎಂದು ಹೇಳುತ್ತಿದ್ದಾರೆಯೆ? ಅವರು ತಮ್ಮ ವೈಯಕ್ತಿಕ ಅಹಂಕಾರದಿಂದ ಮೇಲೆದ್ದು ಎಲ್ಲ ಜಾತಿಯ ಜನರ ಸಮಸ್ಯೆಗಳನ್ನು ತಲಸ್ಪರ್ಶಿಯಾಗಿ, ಪೂರ್ವಾಗ್ರಹವಿಲ್ಲದೆ ಅರಿತು ಉದಾರ ಭಾವನೆಯಿಂದ ಮಾರ್ಗದರ್ಶನ ಮಾಡುವ ದಕ್ಷತೆಯನ್ನು ಸಂಪಾದಿಸಿದ್ದಾರೆಯೆ? ಹಾಗೆ ಮಾಡದಿದ್ದರೆ ಕ್ರಾಂತಿ ಮತ್ತು ವಿಮರ್ಶೆಯ ಹೆಸರಲ್ಲಿ ದ್ವೇಷ ಮತ್ತು ದೋಷಾ ರೋಪಣೆಯ ಬೀಜವನ್ನೇ ಬಿತ್ತಿದಂತಾಗುತ್ತದಲ್ಲವೆ? ಕತ್ತಲನ್ನು ಕುರಿತು ಉಗ್ರವಾಗಿ ಖಂಡಿಸಿದ ಮಾತ್ರಕ್ಕೆ ಕೆಲಸ ನಡೆಯಲಿಲ್ಲ. ಬೆಳಕನ್ನು ತರುವ ಕೆಲಸ ನಡೆಯಬೇಕು. ಬೆಳಕನ್ನು ತರುವವರು ಎಂತಿರಬೇಕು? ತಾವು ಕತ್ತಲೆಯಲ್ಲಿ ತಡಕಾಡುವವರು, ಇತರರಿಗೆ ಬೆಳಕನ್ನು ನೀಡಲು ಸಾಧ್ಯವೆ? ತಾವೇ ಪ್ರವಾಹದಲ್ಲಿ ಕೊಚ್ಚಿಕೊಂಡು ಹೋಗುತ್ತಿರುವವರು ಇತರರನ್ನು ಎತ್ತಿಹಿಡಿದು ದಡಕ್ಕೆ ಎಸೆಯಬಲ್ಲರೇ? ಬಂಡೆಯ ಮೇಲೆ ದೃಢವಾಗಿ ನಿಲ್ಲಲು ಸಾಧ್ಯವಿದ್ದವರು ಪ್ರವಾಹದ ವೇಗವನ್ನು ತಡೆದುಕೊಂಡು ತೇಲುತ್ತಿರುವವರಿಗೆ ಆಸರೆಯಾಗಬಲ್ಲರು.

ನಿಜವಾಗಿ ಮೂಲ ‘ನಾನು’ವಿಗೆ ಯಾವ ಜಾತಿಯೂ ಇಲ್ಲ. ನಿಜ ‘ನಾನು’ವಿನ ಬಗೆಗೆ ತಿಳಿವಳಿಕೆಯಾಗುವವರೆಗೂ ಅದಕ್ಕಂಟಿಕೊಂಡ ಬೇರೆ ಬೇರೆ ಭಾವನೆಗಳನ್ನು ದೂರಕ್ಕೋಡಿಸು ವುದು ಸುಲಭವಲ್ಲ. ಇದನ್ನು ಸರಿಯಾಗಿ ತಿಳಿದುಕೊಂಡವನು ಮಾತ್ರ–ಹೇಗೆ ‘ನಾನು’ವಿನಲ್ಲಿ ಬಾಲ್ಯ\-ದಿಂದಲೇ ನಾನಾ ತೆರನಾದ ಜಾತಿಮತ ಕುಲಗೋತ್ರ ಭಾವನೆಗಳ ಬಾಲಗ್ರಹ ಅಂಟಿ ಕೊಂಡು ಬೆಳೆದಿರುತ್ತದೆ, ಅದು ಸ್ವವ್ಯಕ್ತಿತ್ವ ಚಿತ್ರದಲ್ಲಿ ಹೇಗೆ ಸೇರಿಕೊಂಡಿರುತ್ತದೆ, ವ್ಯಕ್ತಿಯನ್ನು ಸ್ವತಂತ್ರವಾಗಿ ವಿಚಾರ ಮಾಡಲು ಹೇಗೆ ತಡೆಯುತ್ತದೆ–ಎಂಬುದನ್ನು ಕುರಿತ ಸಮಸ್ಯೆಯನ್ನು ಪರಿಹರಿಸಬಲ್ಲ. ಯಾರಿಗೂ ಕೆಡುಕಾಗದ, ಎಲ್ಲರನ್ನೂ ನಿಂತ ಸ್ಥಾನದಿಂದ ಮೇಲಕ್ಕೇರಿಸುವ ಸೂತ್ರವನ್ನು ಕಂಡು\-ಹಿಡಿಯಲು ತಾಳ್ಮೆಯಿಂದ ಆತ ಮಾತ್ರ ದುಡಿಯಬಲ್ಲ.

ಮನುಷ್ಯನ ಅರಿವಿನ ಪರಿಧಿಯನ್ನು ವಿಸ್ತರಿಸಿದಾಗ ತನ್ನ ಸಂಸ್ಕಾರಗಳನ್ನು ಸಂಸ್ಕರಿಸುವ ರಹಸ್ಯವನ್ನು ತಿಳಿಸಿಕೊಟ್ಟು ತನ್ನನ್ನು ಮೇಲೆತ್ತಿಕೊಳ್ಳುವ ಆಸರೆ ಆಧಾರವನ್ನು ತೋರಿಸಿದಾಗ ಸಮಸ್ಯೆಯು ಪೂರ್ಣ ಪ್ರಮಾಣದಲ್ಲಿ ಪರಿಹಾರವಾಗದಿದ್ದರೂ ಸಮಸ್ಯೆಯ ಉಲ್ಬಣತೆ ಬಹು\-ಮಟ್ಟಿಗೆ ಮಾಯ\-ವಾಗುತ್ತದೆ–ಎಂಬುದು ದಿಟ. ಮಾತ್ರವಲ್ಲ ಅದು ನಿಜವಾದ ಪ್ರಗತಿಗೆ ಸಾಧಕ. ಸಂಕುಚಿತತೆಗೆ ಬಾಧಕ.


\section*{ಸತ್ಯಾನ್ವೇಷಣೆ}

\addsectiontoTOC{ಸತ್ಯಾನ್ವೇಷಣೆ}

ತನ್ನನ್ನು ತಾನು ವೈಜ್ಞಾನಿಕ ಮನೋಭಾವದವನು ಎಂದು ತಿಳಿದುಕೊಂಡವನಿಗೂ ಸತ್ಯಾನ್ವೇಷಿಗೂ ವ್ಯತ್ಯಾಸವಿದೆ ಎಂಬುದನ್ನು ನಾನು ತಿಳಿದುಕೊಳ್ಳಲೇಬೇಕಾದಂಥ ಒಂದು ಘಟನೆ ನಡೆಯಿತು. ವೈಜ್ಞಾನಿಕ ಗ್ರಂಥಗಳನ್ನು ಅಧ್ಯಯನ ಮಾಡಿ, ತುಂಬ ಓದಿಕೊಂಡ ನನ್ನ ಮಿತ್ರ ಆಗಾಗ\break ‘ದೇವರು, ಮತ–ಇವು ಭೀತಿಯಿಂದ ಹುಟ್ಟಿದ ಮನೋಧರ್ಮ ಅಷ್ಟೆ’ ಎಂದೇ ಹೇಳುತ್ತಿದ್ದ. ಪಶ್ಚಿಮದ ಯಾರೋ ಪ್ರತಿಭಾಶಾಲಿ ಎಂಬ ಬಿರುದಾಂಕಿತನ ಅಭಿಪ್ರಾಯವನ್ನು ಅವನು ತನ್ನದಾಗಿ ಮಾಡಿಕೊಂಡಿದ್ದನೆಂದು ನನಗನ್ನಿಸಿತ್ತು. ಪುಣ್ಯವಿಶೇಷದಿಂದ ಶೋಷಣೆಯೇ ಧರ್ಮ\break ಮೂಲವೆಂದು ಆತ ಹೇಳಲಿಲ್ಲ. ‘ಕೆಲವೊಂದು ಆದಿಮ ಜನಾಂಗಗಳಲ್ಲಿ ದೇವರು ಧರ್ಮಗಳಿಗೆ ಭೀತಿಮೂಲವಿರಬಹುದು ಎಂದಷ್ಟೆ ಹೇಳಬಹುದು. ಆದರೆ ಸುಸಂಸ್ಕೃತ ನಾಗರಿಕ ಜನಾಂಗಗಳ ಧಾರ್ಮಿಕ ಭಾವನೆಗಳಲ್ಲಿ ಧರ್ಮ–ದೇವರು ಭೀತಿಮೂಲವಲ್ಲ. ಕವಲು ದಾರಿಯ ಸಮೀಪದಲ್ಲಿ ನಿಂತವನು ಯಾವ ದಾರಿ ಹಿಡಿಯಬೇಕು ಎಂದು ಯೋಚಿಸಿದಂತೆ ಕೆಲವೊಂದು ಮೂಲಭೂತ ಸಮಸ್ಯೆಗಳಿಗೆ ಉತ್ತರಗಳನ್ನು ಹುಡುಕುವಲ್ಲಿ ಇವುಗಳ ಮೂಲ ಅಡಗಿದೆ ಎಂಬುದು ಸುಸಂಗತ ಅಭಿಪ್ರಾಯ. ಗೌತಮ ಬುದ್ಧನಿಗೇನು ಭೀತಿ ಇತ್ತು? ಶಂಕರಾಚಾರ್ಯರಿಗೇನು ಹೆದರಿಕೆ ಇತ್ತು? ಉಪನಿಷತ್ತುಗಳ ದ್ರಷ್ಟಾರರನ್ನು ಸತ್ಯಾನ್ವೇಷಿಗಳೆನ್ನದೆ ಪಲಾಯನವಾದಿಗಳೆನ್ನಬೇಕೆ? ಭಾರತ ದಲ್ಲಂತೂ ಸಂಕಟಮಯ ಸನ್ನಿವೇಶದಲ್ಲೇ ತತ್ತ್ವಜ್ಞಾನ ಧರ್ಮಗಳ ಉದಯ ಎಂದರೆ ತಪ್ಪಾ ಗದು. ಮಹಾವೀರನಾದ ಅರ್ಜುನನಿಗೆ ಸಂಕಟಮಯ ಅಂತರ್ದ್ವಂದ್ವ ಉಂಟಾದಾಗ ಗೀತೋಪದೇಶ\-ವಾಯಿತು. ಭಾರತೀಯ ತತ್ತ್ವಶಾಸ್ತ್ರಕ್ಕೆ ಮೌಲಿಕವಾದ ಕಾಣಿಕೆಯನ್ನಿತ್ತ ಸಾಂಖ್ಯರು ದುಃಖ ಕ್ಲೇಶ ಪರಿಮಿತಿಗಳಿಂದ ಪಾರಾಗುವ ವಿಧಾನವನ್ನು ಹುಡುಕುತ್ತಾ ವಿಚಾರವಾದಕ್ಕೆ ನಾಂದಿ ಹಾಕಿದರು. ಒಟ್ಟಿನಲ್ಲಿ ಧರ್ಮಮೂಲದಲ್ಲಿ, ಪರಿಮಿತಿಗಳಿಂದ ಪಾರಾಗುವ ಅನುಭೂತಿಯಲ್ಲಿ ಕೊನೆಗೊಳ್ಳುವ ವಿಚಾರ ಯುಕ್ತಿಗಳೇ ಕಂಡುಬರುತ್ತವೆ; ಸತ್ಯಾನ್ವೇಷಣೆಯ ತೀವ್ರ ಹಂಬಲವೇ ಕಂಡು ಬರುತ್ತದೆ. ಸತ್ಯಾನ್ವೇಷಣೆ ಅಂತರ್ಮುಖಿಯಾದದ್ದು ಅಷ್ಟೆ’ ಎಂದೆಲ್ಲ ಹೇಳಿದೆ. ಆದರೆ ನನ್ನ ಗೆಳೆಯ ಮಹಾತ್ಮರ ನಡವಳಿಕೆಯನ್ನು ‘ಸೈಕಲಾಜಿಕಲ್ ಒಬ್ಸೆಶ್ಶನ್​’ ಎಂದು ಹೇಳಹೊರಟ: ‘ಅದು ಯಾವ ಶತಮಾನದ ‘ಸೈಕಾಲಜಿ’ ಹೇಳುವ ಸಿದ್ಧಾಂತ?’ ಎಂದು ನಾನು ಕೇಳಬೇಕಾಯಿತು. ಏಕೆಂದರೆ, ಇತ್ತೀಚಿನ ಸಂಶೋಧನೆಗಳು ಕೆಲವು ವರ್ಷಗಳ ಹಿಂದಿನ ಸಂಶೋಧನೆಗಳನ್ನು ಅಲ್ಲಗಳೆಯುತ್ತವಷ್ಟೆ!

\vskip 1.4pt

ಜೀವನದ ಅರ್ಥ ಉದ್ದೇಶಗಳನ್ನು ಕಂಡುಕೊಳ್ಳುವ ಪ್ರಯತ್ನ ಹೆದರಿಕೆಯಿಂದ ಉದ್ಭವಿಸು ತ್ತದೆಯೇ? ಗಭೀರ ಕುತೂಹಲ, ಸತ್ಯವನ್ನು ತಿಳಿಯುವ ತೀವ್ರ ಹಂಬಲ ಭಯದಿಂದ ಉದ್ಭವಿಸು ತ್ತವೆಯೇ? ಸತ್ಯಾನ್ವೇಷಣೆಯ ತೀವ್ರ ಹಂಬಲ ಒಬ್ಸೆಶ್ಶನ್​’ ಎಂದಾದರೆ ವೈಜ್ಞಾನಿಕ ಸಂಶೋಧನೆ\-ಯಲ್ಲಿರುವ ಹಂಬಲವೂ ಒಬ್ಸೆಶ್ಶನ್ ಅಲ್ಲವೇ?

\vskip 1.4pt

ಈ ಗೆಳೆಯನ ದೃಷ್ಟಿಕೋನದ ಹಿನ್ನೆಲೆಯಲ್ಲಿ ಕೆಲಸಮಾಡುತ್ತಿರುವ, ಆಗಲೇ ಮನಸ್ಸಿನಲ್ಲಿ ಸ್ವೀಕೃತವಾಗಿ ಹೆಪ್ಪುಗಟ್ಟಿದ ನಂಬಿಕೆಗಳು ಹೀಗಿರಬಹುದು:

\vskip 1.4pt

ಬುದ್ಧಿವಂತರೂ, ತೀಕ್ಷ್ಣಮೇಧಾಸಂಪನ್ನ ವ್ಯಕ್ತಿಗಳೂ ವಿಜ್ಞಾನದ ಪ್ರಾದುರ್ಭಾವಕ್ಕೆ ಮೊದಲು ಹುಟ್ಟಿರಲು ಸಾಧ್ಯವಿಲ್ಲ.

\vskip 1.4pt

ಧರ್ಮವಿರೋಧ ಭಾವನೆಯೇ ವೈಜ್ಞಾನಿಕ ಮನೋವೃತ್ತಿಯ ಲಕ್ಷಣ ಎನ್ನುವ ಅಪಕ್ವ ಸಿದ್ಧಾಂತ ಪ್ರಾಯಃ ಬಾಲ್ಯದಲ್ಲೇ ಅವನ ಸುಪ್ತಮನಸ್ಸನ್ನು ಅಜ್ಞಾತವಾಗಿ ಪ್ರವೇಶಿಸಿರಬಹುದು.

\vskip 1.4pt

ಧಾರ್ಮಿಕ ಒಳಜಗಳ, ಕ್ಷುದ್ರತನ, ಧರ್ಮದ ಹೆಸರಲ್ಲಿ ನಡೆಯುವ ಶೋಷಣೆ ವೈರ ವಿದ್ವೇಷಗಳು ಜುಗುಪ್ಸೆಯನ್ನುಂಟುಮಾಡಿ ಧರ್ಮವನ್ನು ಮತ್ತು ಧಾರ್ಮಿಕ ಪವಿತ್ರ ಗ್ರಂಥಗಳನ್ನು ಸರಿ ಯಾಗಿ ಅಧ್ಯಯನ ಮಾಡದಂತೆ ಅವನನ್ನು ಪ್ರೇರಿಸಿರಬಹುದು.

\vskip 1.4pt

ಮನೋರೋಗಿಗಳನ್ನು ಅಧ್ಯಯನ ಮಾಡಿ ಕಂಡುಕೊಂಡ ಅರೆಬೆಂದ ಮನೋವೈಜ್ಞಾನಿಕ ಸಿದ್ಧಾಂತಗಳನ್ನು ಸರಿಯಾಗಿ ಅರಗಿಸಿಕೊಳ್ಳದೆ ಆ ನೆಲೆಯಲ್ಲಿ ಮಹಾತ್ಮರ ಬಗೆಗೂ ತನ್ನ ತೀರ್ಮಾನಗಳನ್ನು ರೂಪಿಸಿಕೊಂಡಿರಬಹುದು.

\vskip 1.4pt

ತಮ್ಮ ಅಧ್ಯಯನದ ಕ್ಷೇತ್ರದಲ್ಲಿ ಉನ್ನತಮಟ್ಟದ ಸಿದ್ಧಿ ಅಥವಾ ಯಶಸ್ಸನ್ನು ಪಡೆದ ವ್ಯಕ್ತಿಗಳಿಗೆ ತಮಗೆ ಪ್ರವೇಶವಿರದ ಇತರ ವಿಷಯಗಳ ಮೇಲೂ ಅಧಿಕಾರಯುತ ತೀರ್ಮಾನ ನೀಡುವ ಸಾಮರ್ಥ್ಯ ಬಂದುಬಿಡುತ್ತದೆಂಬ ಅಜ್ಞಾತ ಅವಿಚಾರಿತ ಆತ್ಮವಿಶ್ವಾಸ ಉಂಟಾಗುವುದು ಅಸ್ವಾಭಾವಿಕವಲ್ಲ. ಸುಪ್ರಸಿದ್ಧ ಮನೋವಿಜ್ಞಾನಿ ಪ್ರೊ.\ ಐಸೆಂಕ್ ಒಂದೆಡೆ ಹೀಗೆನ್ನುತ್ತಾರೆ:\footnote{\engfoot{Scientists especially when they leave the particular field in which they have specialised are just as ordinary pig-headed and unreasonable as anybody else and the unsually high intelligence only makes their prejudices all the more dangerous.}\hfill\engfoot{ –\textit{Prof.\ Eysenck}}} ‘ತಾವು ವಿಶೇಷ ರೀತಿಯಿಂದ ಅಧ್ಯಯನ ಮಾಡಿದ ಕ್ಷೇತ್ರವನ್ನು ಬಿಟ್ಟು ಇತರ ವಿಷಯಗಳ ಬಗೆಗೆ ವಿಜ್ಞಾನಿಗಳೂ ಸಾಮಾನ್ಯರಷ್ಟೇ ಮೂಢರೂ ವಿಚಾರಹೀನರೂ ಆಗಿರುತ್ತಾರೆ. ಮಾತ್ರವಲ್ಲ ಅವರ ಅಸಾಮಾನ್ಯ\-ವೆನಿಸುವ ಬುದ್ಧಿಶಕ್ತಿಯು ಅವರ ಪೂರ್ವಾಗ್ರಹಗಳನ್ನೂ ದ್ವೇಷಗಳನ್ನೂ ಇನ್ನಷ್ಟು ಅಪಾಯಕಾರಿಯಾಗಿ ಮಾಡುತ್ತದೆ.’


\section*{ಶೋಷಣೆಯ ಇತಿಮಿತಿ}

\addsectiontoTOC{ಶೋಷಣೆಯ ಇತಿಮಿತಿ}

ದಬ್ಬಾಳಿಕೆ ಮತ್ತು ದಾಸ್ಯಕ್ಕೊಳಗಾದ ಜನಾಂಗಗಳಲ್ಲಿ ಆಳುವವರು ನೀಡುವ ದುರ್ಭಾವನೆಗಳು ಶೋಷಿತರ ಮನಸ್ಸಿನ ಆಳದಲ್ಲಿ ಉಳಿದುಬಿಡುತ್ತವೆ ಎಂದು ಮನೋವಿಜ್ಞಾನಿಗಳೂ ಸಮಾಜ\break ಶಾಸ್ತ್ರಜ್ಞರೂ ಹೇಳುತ್ತಾರೆ.

“ಅನಾಥಾಶ್ರಮ ಎಂದು ಹೆಸರಿಸುವುದು ಬೇಡ. ಅದು ಮಕ್ಕಳಲ್ಲಿ ತಾವು ‘ಅನಾಥ’ ಮಕ್ಕಳು ಎನ್ನುವ ಭಾವನೆಯನ್ನು ಬೇರೂರುವಂತೆ ಮಾಡುತ್ತದೆ. ಬಾಲಕಾಶ್ರಮ ಎಂದು ಹೆಸರಿಡಿ” ಎಂದು ಶ‍್ರೀರಾಮಕೃಷ್ಣರ ಪ್ರಮುಖ ಶಿಷ್ಯರಲ್ಲೊಬ್ಬರಾದ ಸ್ವಾಮಿ ಶಾರದಾನಂದರು ಒಮ್ಮೆ ಮಕ್ಕಳ ಸಂಸ್ಥೆಯ ಮೇಲ್ವಿಚಾರಣೆ ನೋಡಿಕೊಳ್ಳುವ ಸ್ವಾಮಿಗಳೊಬ್ಬರಿಗೆ ಹೇಳಿದ್ದರು.

ಸ್ವಾಮಿ ಶಾರದಾನಂದರನ್ನು ‘ದಲಿತರ ಉನ್ನತಿಯ ಮಾರ್ಗವೇನು?’ ಎಂದು ಸ್ವಾಮಿ\-ಗಳೊ\-ಬ್ಬರು ಹಿಂದುಳಿದ ಜನಗಳ ಬಗೆಗೆ ಕಳಕಳಿಯಿಂದ ಪ್ರಶ್ನಿಸಿದ್ದರು. ಶಾರದಾನಂದರ ಉತ್ತರ: ‘ದಲಿತ, ತುಳಿತಕ್ಕೊಳಗಾದವನು ಎಂಬ ಭಾವನೆಯನ್ನು ಮನಸ್ಸಿನೊಳಗೆ ಪ್ರವೇಶಿಸಬಿಡುವುದು ಉಚಿತವಲ್ಲ. ಅದರಿಂದ ಉನ್ನತಿಯನ್ನು ಹೊಂದದ ವ್ಯಕ್ತಿಗೆ ಇನ್ನಷ್ಟು ಕೆಡುಕೇ ಆಗುತ್ತದೆ. ತಾವು ಶಕ್ತಿಹೀನರು, ಮುಂದುವರಿದವರೆಲ್ಲರೂ ತಮ್ಮನ್ನು ಕೆಳಗೇ ಇರುವಂತೆ ಮಾಡಿದ್ದರೆಂದೇ ತಿಳಿಯ ತೊಡಗುತ್ತಾರೆ. ಇದರಿಂದ ಅವರ ಸ್ವಾಭಾವಿಕ ಶಕ್ತಿಯ ಜಾಗರಣವಾಗಲು ವಿಳಂಬವಾಗುತ್ತದೆ. ಅಷ್ಟೇ ಅಲ್ಲ, ಈ ಭಾವನೆ ಮನಸ್ಸಿನ ಆಳವನ್ನು ಪ್ರವೇಶಿಸಿದರೆ ನಿಮ್ನ ಶ್ರೇಣಿಯಲ್ಲಿರುವ ಜನರಿಗೆ ಸದಾ ಸರ್ವದಾ ಮುಂದುವರಿದವರ ಮೇಲೆ ತೀವ್ರತರದ ಅಪನಂಬಿಕೆ ಮತ್ತು ವಿದ್ವೇಷಭಾವನೆ ಜಾಗರೂಕವಾಗಿರುತ್ತದೆ. ಇದರಿಂದ ಉನ್ನತಸ್ಥಾನದಲ್ಲಿರುವವರಿಗೂ ಕೇಡು. ಕೆಳಗಿರುವವರ ಉನ್ನತಿಗೂ ಬಹಳ ಬಾಧಕ. ಪರಸ್ಪರ ಅಪನಂಬಿಕೆ, ಸಂಶಯಗಳಿಂದ ಯಾವ ಒಳ್ಳೆಯ ಕಾರ್ಯವನ್ನೇ ಆಗಲಿ ಯಶಸ್ವಿಯಾಗಿ ಪೂರೈಸಲು ಸಾಧ್ಯವೇ?’ ಒಬ್ಬ ತಾಯಿಯ ಮಕ್ಕಳಲ್ಲಿ ಕೆಲವರು ಕೊಂಚ ಪ್ರಗತಿ ಹೊಂದಿದ್ದಾರೆ, ಕೆಲವರು ಅಷ್ಟೊಂದು ಪ್ರಗತಿ ಹೊಂದಿಲ್ಲ ಎನ್ನುವುದು ಉಚಿತವಾದ ರೀತಿ. ಪ್ರಗತಿಪಥದಲ್ಲಿ ಅಷ್ಟೊಂದು ಮುನ್ನಡೆಯದಿದ್ದವರು ಪಾಲಿಗೆ ಬಂದ ಅವಕಾಶವನ್ನು ಉಪಯೋಗಿಸಿಕೊಂಡು ಪ್ರಯತ್ನವನ್ನು ಹೆಚ್ಚಿಸಿದಂತೆ ಸಕಾಲದಲ್ಲಿ ಪ್ರಗತಿ ಹೊಂದುತ್ತಾರೆ. ಅವರ ಹಿಂದುಳಿಯುವಿಕೆ ಆ ದೃಷ್ಟಿಯಿಂದ ದುಃಖಪಡುವ ಅಥವಾ ನಾಚಿಕೆಗೇಡಿನ ವಿಷಯವಲ್ಲ. ಎರಡನೇ ತರಗತಿಯಲ್ಲಿ ಓದುತ್ತಿರುವ ವಿದ್ಯಾರ್ಥಿ ಮೂರನೇ ತರಗತಿಯಲ್ಲಿ ಓದುವ ವಿದ್ಯಾರ್ಥಿಯನ್ನು ಕಂಡು ಕರುಬಬೇಕಿಲ್ಲ. ಇವನೂ ಸಕಾಲದಲ್ಲಿ ಮುಂದುವರಿಯುತ್ತಾನೆ. ಹೀಗೆಂದ ಮಾತ್ರಕ್ಕೆ ಮುಂದು\-ವರಿದವರಲ್ಲಿ ಶೋಷಕರಿಲ್ಲವೆಂದಾಗಲಿ, ಹಿಂದುಳಿದವರೆಲ್ಲರೂ ಶೋಷಿತರೆಂದಾಗಲಿ ಅರ್ಥವಲ್ಲ. ತಮಗಾಗಿರುವ ಅನ್ಯಾಯವನ್ನು ಅವರು ಸಂಘಟಿತರಾಗಿ ವಿರೋಧಿಸಬಾರದೆಂದೂ ಅಲ್ಲ. ಒಟ್ಟಿನಲ್ಲಿ ಕೀಳರಿಮೆಯ ಸ್ವವ್ಯಕ್ತಿತ್ವ ಚಿತ್ರದಿಂದ ಉದ್ಭವಿಸುವ ದ್ವೇಷಸಿದ್ಧಾಂತ, ಸ್ವನಾಶದ ಬೀಜವನ್ನು ಒಳಗೊಂಡಿದೆ. ಅದು ಬಸವಣ್ಣನವರು ಹೇಳುವಂತೆ ಮೊದಲು ತನ್ನನ್ನು ಸುಟ್ಟು ಮತ್ತೆ ನೆರೆಹೊರೆ\-ಯವರನ್ನು ಸುಡಬಲ್ಲುದು ಎನ್ನುವುದು ಮರೆಯಲಾಗದ ಸತ್ಯ.

ಮನೋರೋಗತಜ್ಞರೂ, ಮನುಷ್ಯಚಾರಿತ್ರ್ಯ ಸುಧಾರಕರೂ ಈ ಒಂದು ದುರಂತ ಸತ್ಯವನ್ನು ಕಂಡುಕೊಂಡಿದ್ದಾರೆ. ತಿರಸ್ಕಾರ, ನಿಂದೆ ಮತ್ತು ಶೋಷಣೆಗೊಳಗಾದ ಯಾವುದೇ ಸಾಮಾಜಿಕ ವ್ಯವಸ್ಥೆಯಲ್ಲಾಗಲಿ ಮೇಲಿನವರು ಅಥವಾ ಆಳುವವರು ತಮ್ಮ ಮೇಲೆ ಹೇರುವ ‘ದುಷ್ಟ, ನೀಚ, ಅಯೋಗ್ಯ’, ‘ಕೆಲಸಕ್ಕೆ ಬಾರದವ’, ‘ಅಪ್ರಯೋಜಕ’ ಎಂಬಂಥ ಭಾವನೆಗಳು ತಿರಸ್ಕೃತರ ಅಥವಾ ದಬ್ಬಾಳಿಕೆಗೊಳಗಾದವರ ಸ್ವವ್ಯಕ್ತಿತ್ವ ಚಿತ್ರದಲ್ಲಿ ಅಂಟಿಕೊಂಡು ಬಿಡುತ್ತವೆ. ತನ್ನಲ್ಲಿಯೇ ಅಜ್ಞಾತವಾಗಿ ಅಪನಂಬಿಕೆ ಬೆಳೆದು ತಾನು ‘ಕೆಲಸಕ್ಕೆ ಬಾರದವನು’ ಎಂಬ ಭಾವನೆ ದೃಢವಾಗಿ ತನ್ನನ್ನೇ ತಾನು ದ್ವೇಷಿಸಿಕೊಳ್ಳುವ ಪ್ರವೃತ್ತಿಯನ್ನು ಉಂಟುಮಾಡುತ್ತದೆ. ಒಟ್ಟಿನಲ್ಲಿ ಈ ಭಾವನೆಗಳು ವ್ಯಕ್ತಿಯ ನಿಜವಾದ ಪರಿಪೂರ್ಣ ಬೆಳವಣಿಗೆಗೆ ದೊಡ್ಡ ಆಘಾತವಾಗುತ್ತವೆ.\footnote{\engfoot{Therapeutic as well as reformist efforts verify the sad truth that in any system based on exclusion, suppression and exploitation, the suppressed, excluded and exploited unconsciously accept the evil image they are made to represent by those who are dominant.}

\engfoot{\general{\hfill} –Erik Erikson, Identity, \textit{Youth and Crisis.}}}

ಆರ್ಥಿಕವಾಗಿ, ಸಾಮಾಜಿಕವಾಗಿ, ಸಾಂಸ್ಕೃತಿಕವಾಗಿ, ಬೌದ್ಧಿಕವಾಗಿ, ಧಾರ್ಮಿಕವಾಗಿ,\break ಆಧ್ಯಾತ್ಮಿಕವಾಗಿ ಮುನ್ನಡೆದವರು ಯಾವ ಜಾತಿಮತ ಕುಲದವರಾದರೂ ತಮಗಿಂತ ಸ್ವಲ್ಪ ಕೆಳಗಡೆ ಇರುವವರ ಎದುರು ತಮ್ಮ ಬಲಿತ ‘ಅಹಂ’, ‘ಮಮ’ಗಳನ್ನು ತಿಳಿದೊ ತಿಳಿಯದೆಯೊ ಪ್ರದರ್ಶಿಸು\-ತ್ತಾರೆ. ಇವರು ಅಜ್ಞಾತವಾಗಿ ಅಭಿಮಾನದಿಂದ ಸೆಟೆದು ತಮಗಿಂತ ಕೆಳಗಿರುವವರನ್ನು ‘ಮೂರ್ಖ’, ‘ಮೂಢ’, ‘ಕೆಲಸಕ್ಕೆ ಬಾರದ ಗುಂಪಿಗೆ ಸೇರಿದವರು’ ಎಂಬ ನಿಂದೆ ತಿರಸ್ಕಾರ ಮತ್ತು ಅಪನಂಬಿಕೆಯ ದೃಷ್ಟಿಯಿಂದ ನೋಡುತ್ತಾರೆ. ಈ ಚಟವನ್ನು ಅವರು ಬೇಗ ತೊರೆದಷ್ಟೂ ಅವರಿಗೂ ಸಮಾಜಕ್ಕೂ ಒಳಿತಾಗುವುದು ಖಂಡಿತ. ಗುಲಾಮಗಿರಿಗೆ ಸೊಪ್ಪುಹಾಕುವವರಲ್ಲೂ, ಮುಂದುವರಿಯದವರು ಎನಿಸಿಕೊಂಡ ಜನರಲ್ಲೂ ಕಂಡು ಬರುವ ಶ‍್ರೀಮದ್ಗಾಂಭೀರ್ಯದ ಈ ಮನೋವೃತ್ತಿ, ಶಿಕ್ಷಣ ಹಾಗೂ ಸಂಸ್ಕೃತಿಯ ಸೌಲಭ್ಯಗಳಿಂದ ವಂಚಿತರಾದ ಜನರಿಗೆ ಮೇಲೇಳಲು ಬೇಕಾಗುವ ಆತ್ಮ\-ವಿಶ್ವಾಸಕ್ಕೆ ಬಹಳಷ್ಟು ಆಘಾತವನ್ನುಂಟುಮಾಡಿದೆ. ತಮ್ಮ ಆ ದೃಷ್ಟಿ ಕೋನ ಮತ್ತು ವರ್ತನೆಗಳಿಂದ ತಮ್ಮಷ್ಟು ಮೇಲೇರದ ಜನರಿಗೆ ಎಂಥ ಆಘಾತವಾಗುವುದೆಂಬುದನ್ನು ಅವರು ಯೋಚಿಸಬಲ್ಲರೇ? ಒಂದು ತೆರನಾದ ಅಸಹಾಯ ಸ್ಥಿತಿಯಲ್ಲಿ ಕೆಳಗಡೆ ಇರುವವರ ಸ್ವವ್ಯಕ್ತಿತ್ವ ಚಿತ್ರದಲ್ಲಿ ನಿಷೇಧಾತ್ಮಕ ಭಾವನೆಗಳನ್ನು ತುಂಬಿ ಅವರ ಬದುಕಿಗೆ ಅಪ್ರತ್ಯಕ್ಷವಾಗಿ ಕೆಡುಕನ್ನುಂಟುಮಾಡಿ, ಕ್ರಮೇಣ ತಮ್ಮ ಬದುಕಿಗೂ ಅವರಿಂದ ಕೆಡುಕಾಗುವದಾರಿಯನ್ನು ಸಿದ್ಧಗೊಳಿಸಿದಂತಾಗುತ್ತದೆಂಬುದನ್ನು ತಿಳಿಯು\-ವಷ್ಟು ದೂರದೃಷ್ಟಿ ಇಲ್ಲದಿದ್ದರೆ ಅವರ ‘ಮೇಲ್ಮೆ’ಗೆ ಏನು ಅರ್ಥ? ಸದ್ಯದ ಪರಿಮಿತಿ ಮತ್ತು ದೋಷಗಳಿಂದಲೇ ಮನುಷ್ಯನ ಯೋಗ್ಯತೆಯನ್ನು ಅಳೆಯದೇ ಅವನ ಸಂಭಾವ್ಯ ಸಾಧ್ಯತೆಗಳನ್ನು ಗಮನಿಸಿ ಅವನನ್ನು ಗೌರವ ದೃಷ್ಟಿಯಿಂದ ನೋಡುವ ಪರಿಪಾಠ ಬೆಳೆಯದಿದ್ದರೆ ನಮ್ಮ ಜನಾಂಗ ಮುನ್ನಡೆಯ ದಾರಿಯಲ್ಲಿ ಸರ್ವತ್ರ ವಿಘ್ನ ಪರಂಪರೆಗಳನ್ನೇ ಸೃಷ್ಟಿಸಿಕೊಳ್ಳುವುದು. ಈಗಾಗಲೇ ಸಾಕಷ್ಟು ಸೃಷ್ಟಿಸಿಕೊಂಡಿದೆ. ಈ ಪ್ರಜ್ಞೆ ವಿದ್ಯಾವಂತರಲ್ಲಿ ಬೆಳೆಯುವ ಲಕ್ಷಣಗಳಿವೆಯೇ?\break ಹಿರಿಯರೂ, ವಿದ್ಯಾವಂತರೂ ಬಾಲ್ಯದಿಂದಲೇ ಮಕ್ಕಳಲ್ಲಿ ‘ವ್ಯಕ್ತಿ ಗೌರವ’ದ ಈ ಭಾವನೆಯನ್ನು ಬೋಧನೆಯ ಜೊತೆ ತಮ್ಮ ವರ್ತನೆಗಳ ಮೂಲಕ ತಿಳಿಸಿಕೊಡಬೇಕು. ಈ ವಿಚಾರದಲ್ಲಿ ರ್ಯಾಂಕ್ ವಿಜೇತರು, ಉನ್ನತಮಟ್ಟದ ಬುದ್ಧಿಶಾಲಿಗಳು, ಉನ್ನತ ಸ್ಥಾನದಲ್ಲಿರುವವರು, ಅಧಿಕಾರಿಗಳು, ಮೇಲುವರ್ಗದವರೆನ್ನಿಸಿ ಕೊಂಡವರು–ಇವರುಗಳ ಜವಾಬ್ದಾರಿ ಅತಿ ಹೆಚ್ಚಿನದು. ತಾವು ಕೆಳಗಿದ್ದೇವೆಂದು ತಿಳಿದುಕೊಂಡವರಲ್ಲಿ ಆತ್ಮವಿಶ್ವಾಸದ ಭಾವನೆಗಳನ್ನು ಬಿತ್ತಲು ಇವರು ತಮ್ಮ ಹಿತದೃಷ್ಟಿ\-ಯಿಂದಲಾದರೂ ವಿಶೇಷವಾಗಿ ಶ್ರಮಿಸಬೇಕು. ಏನೂ ಮಾಡಲಾಗದಿದ್ದರೆ ಹೃದಯ ತುಂಬಿ ಅವರ ಶುಭಕ್ಕಾಗಿ ಹೃತ್ಪೂರ್ವಕವಾಗಿ ಪ್ರಾರ್ಥಿಸುವ ಸಜ್ಜನಿಕೆಯನ್ನಾದರೂ ಬೆಳಸಿಕೊಳ್ಳಬೇಕು. ಇಲ್ಲಿ ನಮ್ಮ ಬುದ್ಧಿ ಜೀವಿಗಳೆನಿಸಿಕೊಂಡವರಲ್ಲಿ ಹೆಚ್ಚಿನವರು ಸೋತಿದ್ದಾರೆಂದರೆ ಉತ್ಪ್ರೇಕ್ಷೆ ಇಲ್ಲ. ಅವರು ಈ ಒಂದು ಮಾನವೀಯ ಸಮಸ್ಯೆಯನ್ನೇ ಅರ್ಥಮಾಡಿಕೊಂಡಿಲ್ಲವೆನ್ನಬಹುದು.


\section*{ಕೀಳರಿಮೆ ಕೀಳಲು}

\addsectiontoTOC{ಕೀಳರಿಮೆ\break ಕೀಳಲು}

ಸಾಂಸ್ಕೃತಿಕವಾಗಿ, ಸಾಮಾಜಿಕವಾಗಿ ಮತ್ತು ಬೌದ್ಧಿಕವಾಗಿ ತಮ್ಮನ್ನು ತಾವು ಕೆಳಗಿದ್ದೇವೆಂದು ತಿಳಿದು\-ಕೊಂಡವರು ಮೊದಲು ತಮ್ಮ ಕೀಳರಿಮೆಯನ್ನು ಕಿತ್ತೆಸೆಯಬೇಕು. ಬಹಳ ಆಳವಾಗಿ ಅದು ಬೇರು ಬಿಟ್ಟಿದ್ದರೂ ಅದನ್ನು ದೂರಕ್ಕೋಡಿಸಲು ಖಂಡಿತ ಸಾಧ್ಯ. ಇದನ್ನು ನಾವು ನಮಗೆ ಬೇಡವಾದ ವಸ್ತುಗಳನ್ನು ಹೊರಕ್ಕೆ ಎಸೆದಂತೆ ಎಸೆಯಲು ಸಾಧ್ಯವಿಲ್ಲ. ನಮ್ಮಲ್ಲಿ ಸುಪ್ತವಾಗಿರುವ ಅಪಾರ ಶಕ್ತಿಯಲ್ಲಿ ಅಚಲವಿಶ್ವಾಸವಿಡುವುದು ಅತ್ಯಂತ ಆವಶ್ಯಕ ಮಾತ್ರವಲ್ಲ, ಮೊದಲ ಕರ್ತವ್ಯವೂ ಹೌದು. ಈ ವಿಶ್ವಾಸ ಮೂಢನಂಬಿಕೆಯಲ್ಲ. ಮನುಷ್ಯನಾಗಿ ಗೋಚರಿಸುವ ವ್ಯಕ್ತಿಯ ಹಿನ್ನೆಲೆಯಲ್ಲಿ ಅದ್ಭುತ ಶಕ್ತಿ ಅಡಗಿದೆ ಎಂಬುದು ಹಿಂದೆ ಮಹಾಪುರುಷರ ಮತ್ತು ಅನುಭಾವಿಗಳ ಸ್ಪಷ್ಟವಾದ ಅನುಭವವಾಗಿದ್ದರೆ ಇಂದು ಮನೋವಿಜ್ಞಾನಿಗಳು ಈ ಮಾತನ್ನು ಪ್ರಯೋಗಾತ್ಮಕ ಅನುಭವಗಳಿಂದ ಸತ್ಯವೆಂದು ಸಾರುತ್ತಿದ್ದಾರೆ. ದೃಢವಿಶ್ವಾಸದ ಬಲದಿಂದಲೇ ಆ ಶಕ್ತಿಯನ್ನು ಎಚ್ಚರಿಸಬೇಕು. ನಾವು ನಿಂತ ಸ್ಥಾನದಿಂದ ಮೊದಲು ಏರಲು ಯತ್ನಿಸಬೇಕು. ನಮ್ಮ ಅಭಿರುಚಿಯ ಕೆಲಸದಲ್ಲಿ ಮೆಲ್ಲಮೆಲ್ಲನೆ ಮೆಟ್ಟಲು ಮೆಟ್ಟಲಾಗಿ ಮೇಲೇರುತ್ತ ದುರ್ಬಲಗೊಳಿಸುವ ನಿಷೇಧಾತ್ಮಕ ಭಾವನೆಗಳನ್ನು ಹೊರತಳ್ಳಬೇಕು. ಈ ಮೂಲಕ ಆತ್ಮವಿಶ್ವಾಸವನ್ನು ವೃದ್ಧಿಸುವ ಉಪಾಯವನ್ನು ಕಂಡುಕೊಳ್ಳಬೇಕು. ಇದು ಒಂದೇ ದಿನದಲ್ಲಿ ಸಾಧ್ಯವಾಗುವ ಕೆಲಸವಲ್ಲ. ತಾಳ್ಮೆ ಮತ್ತು ಪ್ರಾಮಾಣಿಕ ಪ್ರಯತ್ನದಿಂದ ಮೊದಲು ಅಸಾಧ್ಯವೆಂದು ತೋರಿದರೂ ಕೊನೆಯಲ್ಲಿ ಸಾಧ್ಯವಾಗಿಯೇ ಆಗುತ್ತದೆ. ತಾಳ್ಮೆ ಮತ್ತು ಪ್ರಾಮಾಣಿಕ ಪ್ರಯತ್ನಕ್ಕೆ ಸ್ಫೂರ್ತಿಯಾಗಿ ಹಿನ್ನೆಲೆಯಲ್ಲಿ ಕೆಲಸ ಮಾಡುವ ಶಕ್ತಿಯೇ ಆತ್ಮವಿಶ್ವಾಸ–ತನ್ನಲ್ಲಿ ತನಗೆ ನಂಬಿಕೆ. ತನ್ನ ಆಂತರ್ಯದ ಆಳದಲ್ಲಿ ಅಪಾರಶಕ್ತಿ ಇದೆ ಎಂಬ ನಂಬಿಕೆ. ಈ ನಂಬಿಕೆ ಅಥವಾ ವಿಶ್ವಾಸದ ಅಭಾವವೇ ಭಯ ಸಂಶಯಗಳಿಗೂ, ಕಾರ್ಯಶಕ್ತಿಯ ಹೀನತೆಗೂ, ಎಲ್ಲತೆರನಾದ ದೌರ್ಬಲ್ಯಗಳಿಗೂ ಕಾರಣವಾಗುತ್ತದೆ. ಮೇಲಿನವರ ಕೃಪೆಗಾಗಿ ಕಾಯದೆ ಅಥವಾ ಅವರನ್ನು ದ್ವೇಷಿಸದೆ, ಹತಾಶ ಮನೋಭಾವನೆಯಿಂದ ನರಳದೆ ಮೇಲೆದ್ದು ತನ್ನ ಕಾಲ ಮೇಲೆ ತಾನೇ ನಿಲ್ಲುವ ಪ್ರೇರಣೆಯನ್ನು ನೀಡುವುದು ಈ ವಿಶ್ವಾಸ. ತನಗೆ ಕೊಡಲ್ಪಟ್ಟ ಅವಕಾಶವನ್ನು ವ್ಯರ್ಥವಾಗದಂತೆ ಸರಿಯಾಗಿ ಉಪಯೋಗಿಸಿ ಕೊಂಡೇ ಮೇಲೇರಬೇಕೆಂಬ ದೃಷ್ಟಿಕೋನಕ್ಕೂ ಈ ವಿಶ್ವಾಸವೇ ತಳಹದಿ.

‘ತನ್ನನ್ನು ಅಶಕ್ತನೆಂದೂ ಪಾಪಿಯೆಂದೂ ಭಾವಿಸುವುದೇ ಅತಿ ದೊಡ್ಡ ಪಾಪ; ನೀವು ಏನನ್ನಾ ದರೂ ನಂಬಬೇಕಿದ್ದರೆ, ನಾವು ಭಗವಂತನ ಮಕ್ಕಳೆಂದೂ ಅವನ ಅಂಶಗಳೆಂದೂ, ಅವನ ಅನಂತ ಶಕ್ತಿ ಮತ್ತು ಆನಂದದ ಪಾಲುಗಾರರೆಂದೂ ನಂಬಿರಿ’ ಎಂದರು ಸ್ವಾಮಿ ವಿವೇಕಾನಂದರು.

ಈ ಮಾತುಗಳು ಆತ್ಮವಿಶ್ವಾಸವನ್ನು ಹುಟ್ಟಿಸಲು ಹೇಳಿದ ರ್ಹೆಟರಿಕ್ ಜಾತಿಯ ಮಾತುಗಳಲ್ಲ. ಮನುಷ್ಯನ ಅಂತರಂಗದಲ್ಲಿ ದೇವತ್ವವನ್ನೂ ಅಪಾರ ಶಕ್ತಿಯನ್ನೂ ಪ್ರತ್ಯಕ್ಷವಾಗಿ ಕಂಡ ದ್ರಷ್ಟಾ ರರು ಹೇಳಿದ ಮಾತು. ಈ ಮಾತನ್ನು ಸ್ವಲ್ಪವಾದರೂ ಕಾರ್ಯರೂಪಕ್ಕೆ ತರಲು ಯತ್ನಿಸಿದವರ ಪಾಲಿಗೆ ಅದು ನೀಡುವ ಶಕ್ತಿ ಸಾಮರ್ಥ್ಯಗಳ ಅನುಭವವಾಗುತ್ತದೆಂಬುದರಲ್ಲಿ ಸಂದೇಹವಿಲ್ಲ.


\section*{ದ್ವೇಷದ ದಳ್ಳುರಿ}

\addsectiontoTOC{ದ್ವೇಷದ ದಳ್ಳುರಿ}

ದ್ವೇಷವೂ ಸ್ವವ್ಯಕ್ತಿತ್ವ ಚಿತ್ರಕ್ಕೆ ಪುಷ್ಟಿ ಕೊಡುತ್ತದೆ ಎನ್ನುವ ಸ್ವದೋಷ ಸಮರ್ಥನೆಯವಾದವೂ ಇತ್ತೀಚೆಗೆ ಸಾಮಾಜಿಕವಾಗಿ ಸಾಂಸ್ಕೃತಿಕವಾಗಿ ಹಿಂದುಳಿದವರಲ್ಲಿ ಅವರ ಕ್ರಾಂತಿಕಾರಿ ಮುಖಂಡರಿಂದ ಪ್ರಚಾರವಾಗುತ್ತಿರುವುದು ಒಂದು ದುರಂತ. ಮನುಷ್ಯನು ಬೆಳೆಯಬೇಕಿದ್ದರೆ, ಎತ್ತರವನ್ನು ತಲುಪಬೇಕಿದ್ದರೆ ತನ್ನ ಕ್ಷೇತ್ರದಲ್ಲಿ ಪ್ರಾಮಾಣಿಕನಾಗಿ ದುಡಿಯಬೇಕು. ಇದು ಸಾವಕಾಶವಾಗಿ ಆಗಬೇಕಾದ ಪ್ರಗತಿ. ಶೋಷಕರ ದಬ್ಬಾಳಿಕೆಯನ್ನು ತೀವ್ರ ಹಿಂಸಾತ್ಮಕವಾಗಿ ವಿರೋಧಿಸದೆ ಗಾಂಧೀಜಿ ಬೋಧಿಸಿದ ಸತ್ಯಾಗ್ರಹ ವಿಧಾನವನ್ನು ಅನುಸರಿಸಿದರೆ ಶೋಷಿತರು ಹ್ಯಾಪರಾಗಿ ಉಳಿಯಬೇಕಾಗುತ್ತದೆಂದು ಈ ಮುಖಂಡರು ಬೋಧಿಸುತ್ತಾರೆ. ಎತ್ತರದಲ್ಲಿರುವವರೆಂದು ತಿಳಿಯ\-ಲಾದ ಜನರ ಸಾಂಸ್ಕೃತಿಕ ಮೌಲ್ಯಗಳನ್ನು ತುಚ್ಛವೂ, ಹೇಯವೂ ಎಂದು ಕಾಣುವ ಮೂಲಕ ತಮ್ಮ ‘ಸೆಲ್ಫ್ ಇಮೇಜಿ’ಗೆ ಬಲ ಬರುತ್ತದೆಂದು ದ್ವೇಷ ಪ್ರಚಾರದಿಂದ ಆಗಲೇ ತುಳಿತ ಕ್ಕೊಳಗಾದವರನ್ನು ಇವರು ಅಡ್ಡದಾರಿಗಳೆಯುತ್ತಾರೆ. ನೀಗ್ರೋ ಲೇಖಕನೊಬ್ಬ ಬಿಳಿಯರು ಸೇರಿದ್ದ ದೊಡ್ಡ ಸಭೆಯಲ್ಲಿ ‘ನಿಮ್ಮ ದೊಡ್ಡ ಕವಿಗಳಾದ ಪೌಂಡ್ ಮತ್ತು ಎಲಿಯಟ್ಟರನ್ನು ಬಹಳ ಸೊಗಸಾದ ಶೈಲಿಯಲ್ಲಿ ಬರೆಯುತ್ತಾರೆಂದು ಇಷ್ಟಪಟ್ಟರೆ ಅದರ ಪರಿಣಾಮ ನನ್ನಂಥವರ ಮೆಲೆ ಏನಾಗುತ್ತದೆ ಗೊತ್ತೇ? ನನ್ನನ್ನೇ ನಾನು ಅವಮಾನಪಡಿಸಿಕೊಂಡಂತೆ ಆಗುತ್ತದೆ. ಏಕೆಂದರೆ ಅವರು ಬಳಸುವ ಸಾಂಸ್ಕೃತಿಕ ಸರಕೆಲ್ಲವೂ ನನ್ನಂಥ ಜನರು ಗುಲಾಮರಾಗಿರಲು ಮಾತ್ರ ಯೋಗ್ಯರು ಎಂಬ ಬಹಳ ಹಳೇಕಾಲದಿಂದ ಹೇಳಿಕೊಂಡು ಬಂದಂಥವು. ಆ ಸರಕಿನ ಹಿಂದೆ ಅಂಥದೇ ಜನಾಂಗಭೇದದ ಸೂಕ್ಷ್ಮ ದೃಷ್ಟಿಕೋನವಿದೆ. ಅಂಥದನ್ನು ಶೈಲಿಗಾಗಿ ಮೆಚ್ಚಿ ನನ್ನನ್ನೇ ನಾನು ದ್ವೇಷಿಸಿಕೊಳ್ಳುವಂತಾಗಬೇಕೇ? ಆದ್ದರಿಂದ ಈ ಇಡೀ ಸಾಹಿತ್ಯ ಪರಂಪರೆಯನ್ನು ನಾನು ತಿರಸ್ಕರಿಸುತ್ತೇನೆ’\footnote{ ಡಾ. ಯು. ಆರ್. ಅನಂತಮೂರ್ತಿ, ‘ಸಮಕ್ಷಮ’} ಎಂದನಂತೆ.

ಮನುಷ್ಯನನ್ನು ಪ್ರಾಣಿಯ ಸ್ಥಿತಿಯಿಂದ ಮೇಲೆತ್ತುವ, ಆತ್ಮವಿಶ್ವಾಸ ಮತ್ತು ಭರವಸೆಗಳನ್ನು ನೀಡುವ, ಶಕ್ತಿಶಾಲಿಯಾದ ಒಂದು ರಹಸ್ಯ ಸಿದ್ಧಾಂತ ಇದುವರೆಗೆ ನಿಮಗೆ ದೊರೆತಿರಲಿಲ್ಲ. ಅದನ್ನು ಪಡೆದುಕೊಂಡಿದ್ದೇವೆಂದು ತಿಳಿದುಕೊಂಡವರು, ಸಂಕುಚಿತ ಸ್ವಾರ್ಥತೆಯಿಂದ ಅದನ್ನು ನಿಮಗೆ ನೀಡದೆ ಮಹಾ ಅಪಕಾರ ಮಾಡಿದ್ದಾರೆಂಬುದು ನಿಜ. ಆದರೆ ಅವರಲ್ಲಿ ಹಲವರು ಅದನ್ನು ತಾವೂ ಸರಿಯಾಗಿ ತಿಳಿದು ಅದರ ಉಪಯೋಗ ಪಡೆದುಕೊಂಡವರಲ್ಲ. ಏನಿದ್ದರೂ ಈಗ ಅದನ್ನು ಅವರು ನಿಮಗೆ ನೀಡುವ ಸಂಭವವಿದೆ. ಅದನ್ನು ನೀವು ಪಡೆದುಕೊಂಡು ಉಪಯೋಗಿಸುವ ಅವಕಾಶವಿದೆ. ಆದರೆ ‘ಅದು ನಮ್ಮನ್ನು ದಬ್ಬಾಳಿಕೆಯಲ್ಲಿಟ್ಟು ಆಳಿದ ಜನಾಂಗದ ಭಾವನೆ, ಅದನ್ನು ಸ್ವೀಕರಿಸಿದರೆ ನಮ್ಮ ವ್ಯಕ್ತಿಗೌರವಕ್ಕೆ ಕುಂದು’ ಎಂದು ತಿರಸ್ಕರಿಸಿದರೆ, ‘ನಮ್ಮತನ’ ಪ್ರದರ್ಶಿಸಿ\-ದಂತಾದರೂ ಆ ದೃಷ್ಟಿಕೋನದಿಂದ ನಷ್ಟ ಉಂಟಾಗುವುದು ಯಾರಿಗೆ?

ಈ ಮುಖಂಡನ ಮನಸ್ಸಿನ ಆಳದಲ್ಲಿ ಹತಾಶ ಭಾವನೆ ನೋವು ಅಡಗಿರುವುದು ಸತ್ಯ. ಅದು ಅವನ ಬಾಯಿಯಿಂದ ಈ ದ್ವೇಷ ನುಡಿಗಳನ್ನು ಹೊರಸೂಸುವಂತೆ ಮಾಡಿರುವುದೂ ಸಹಜ. ಆದರೆ ಈತನ ದ್ವೇಷಕ್ಕೆ ಮೇಲಿನ ವರ್ಗದವರು ಮಾತ್ರ ಬಲಿ ಎನ್ನಲಾಗದು; ಅವನ ಅನುಯಾಯಿಗಳಾದ ಮುಗ್ಧ ದಲಿತರೂ ಬಲಿಯಲ್ಲವೆ? ಏಕೆಂದರೆ, ಅವರೂ ಒಳ್ಳೆಯ ವಿಚಾರಗಳಿಂದ ವಂಚಿತರಾಗುತ್ತಾರಷ್ಟೆ.

ಅಪಮಾನ, ಹತಾಶೆ ಮತ್ತು ದ್ವೇಷ–ಇವುಗಳಿಂದ ಉದ್ಭವಿಸಿದ ಆಪಾದನೆಯೇ ಇಲ್ಲಿ\break ವಿಮರ್ಶೆಯ ಹೆಸರಲ್ಲಿ ಹೊರಹೊಮ್ಮಿದೆ ಎಂಬುದನ್ನು ತಿಳಿಯಲು ಮನೋವಿಜ್ಞಾನಿ ಬೇಕಿಲ್ಲ. ಸಾಮಾನ್ಯ ವ್ಯವಹಾರಕುಶಲನು ತಿಳಿಯಬಲ್ಲ. ಈ ದೃಷ್ಟಿಯಿಂದ ತಮ್ಮನ್ನು ಗುಲಾಮಗಿರಿಯಲ್ಲಿಟ್ಟು ಬಿಳಿಯರು ಕಂಡುಹಿಡಿದ ವಿವಿಧ ಔಷಧೋಪಕರಣಗಳನ್ನಾಗಲೀ, ಯಂತ್ರಗಳ ನ್ನಾಗಲೀ, ಅವರ ಧಾರ್ಮಿಕ ಕಲಾಪಗಳನ್ನಾಗಲೀ ಒಟ್ಟಿನಲ್ಲಿ ಅವರ ಯಾವತ್ತೂ ಸಾಧನೆ ಸಿದ್ಧಿಗಳನ್ನಾಗಲೀ ದ್ವೇಷಿಸದೇ ಇರುವುದು ಸರಿಯಾಗುವುದೆ? ದ್ವೇಷಿಸದೆ ಇರುವುದು ಸಾಧ್ಯವೆ? ಈ ದ್ವೇಷ ಮುಖಂಡನಿಗೆ ಪ್ರಸಿದ್ಧಿಯನ್ನು ತರಬಹುದಾದರೂ ಅವನ ಅನುಯಾಯಿಗಳಿಗೆ ಶುಭವನ್ನುಂಟುಮಾಡುವುದೆ?

ದ್ವೇಷದ ದಾವಾನಲದ ದುರಂತ ಚಿತ್ರಿಸುವ ಒಂದು ಘಟನೆ ಇಲ್ಲಿದೆ:

ಅವರಿಬ್ಬರೇ, ಅವರಿಗಿಬ್ಬರೇ ಮಕ್ಕಳು; ಹಿರಿಯವನು ಹುಡುಗ, ಕಿರಿಯವಳು ಆತನ ತಂಗಿ, ಹುಡುಗಿ. ವಯಸ್ಸು ಹತ್ತಾಗಿದ್ದರೂ ಹುಡುಗ ಹಾಸಿಗೆಯಲ್ಲಿ ಮೂತ್ರ ಮಾಡಿಕೊಳ್ಳುತ್ತಿದ್ದ.\break ಔಷಧೋಪಚಾರಗಳಿಂದ ಪ್ರಯೋಜನವಾಗಲಿಲ್ಲ. ತಂದೆತಾಯಿಗಳಿಗೆ ಸ್ವಲ್ಪ ತಲೆನೋವೇ ಆದ\-ನಾತ. ಕಿರಿಯವಳು ಚುರುಕು ಬುದ್ಧಿಯವಳು. ಕಲಿಕೆಯಲ್ಲಿ ಮುಂದು. ಕೆಲಸದಲ್ಲಿ ಅಚ್ಚುಕಟ್ಟು. ಎಲ್ಲರೂ ಹೊಗಳುವವರೆ. ತಿಳಿದೊ ತಿಳಿಯದೆಯೋ ಹುಡುಗನನ್ನು ಅಜ್ಞಾತವಾಗಿ ಕೊಂಚ ತಿರಸ್ಕಾರ ಮತ್ತು ನಿಕೃಷ್ಟ ದೃಷ್ಟಿಯಿಂದಲೇ ತಾಯಿ ತಂದೆ ನೋಡುತ್ತ ಬಂದರು. ಅವನ ಮನಸ್ಸಿನಲ್ಲಿ ಕೀಳರಿಮೆ ಅಥವಾ ಹೀನ ಮನೋಭಾವ ಆಗಲೇ ಪ್ರವೇಶಿಸಿತ್ತು. ಶಾಲೆಯಲ್ಲೂ ಹಿನ್ನಡೆ. ಕ್ರಮೇಣ ಬೀದಿ ಹುಡುಗರ ಜೊತೆ ಸೇರಿಕೊಂಡು ತುಂಟಾಟ ಪ್ರಾರಂಭಿಸಿದ. ತಂದೆ ಎಚ್ಚೆತ್ತು ಬುದ್ಧಿ ಹೇಳಿ ನೋಡಿದರು. ಹಿರಿಯರಿಂದ ಬುದ್ಧಿ ಹೇಳಿಸಿದರು. ಹೊಡೆದು ಬಡಿದೂ ನೋಡಿದರು. ಬೈದು ಭಂಗಿಸಿ ಕೊನೆಗೆ ತಿರಸ್ಕಾರದಿಂದ ‘ಈ ಪ್ರಾಣಿಯನ್ನು ಸರಿಪಡಿಸಲು ಬ್ರಹ್ಮ ನಿಂದಲೂ ಸಾಧ್ಯವಿಲ್ಲ’ ಎಂದು ಆಗಾಗ ಹೇಳುತ್ತಿದ್ದರು. ಒಂದು ದಿನ ತಾಯಿ, ತಂಗಿಯ ಸದ್ಗುಣಗಳ ಸರಮಾಲೆಯನ್ನು ನೇಯ್ದರು. ಆತನ ಮೂರ್ಖತೆಯ ನಡೆನುಡಿಗಳನ್ನು ಕಣ್ಣಿಗೆ ಕಟ್ಟುವಂತೆ ವಿವರವಾಗಿ ವರ್ಣಿಸಿದರು. ಇದ್ದಕ್ಕಿದ್ದಂತೆ ಅವನು ವ್ಯಗ್ರನಾದ. ಅವನ ಮನಸ್ಸಿನ ಆಳದಿಂದ ಪಾಶವಿಕ ಪ್ರವೃತ್ತಿಯೊಂದು ಹೆಡೆಯಾಡತೊಡಗಿತು. ತಂಗಿಯನ್ನು ಕೊಂದೇ ಹಾಕಿ ಬಿಡುತ್ತೇನೆ ಎಂದು ಅರಿವಿಲ್ಲದೆ ಗಟ್ಟಿಯಾಗಿ ಕಿರಿಚಿಕೊಂಡ. ಕೂಡಲೇ ಆತನ ಮಾನಸಿಕ ಪರಿ ಸ್ಥಿತಿಯನ್ನು ಅರಿತುಕೊಂಡು ತಂದೆತಾಯಿಗಳಿಬ್ಬರೂ ಅವನನ್ನು ಆಸ್ಪತ್ರೆಗೆ ಸೇರಿಸಿ ಔಷಧೋಪ ಚಾರ ಮಾಡಲು ಯತ್ನಿಸಿದರು. ಹುಡುಗನ ಮಾನಸಿಕ ಸ್ಥಿತಿ ಮತ್ತಷ್ಟು ಉಗ್ರವಾಯಿತು. ಆರು ತಿಂಗಳ ಬಳಿಕ ಅವನನ್ನು ಹಿಂದಿರುಗಿ ಮನೆಗೆ ಕರೆತಂದರು. ಆಗ ಅನುಭವಿಗಳಾದ ಹಿರಿಯ ಸಜ್ಜನ ರೊಬ್ಬರು ಹೀಗೆ ಸೂಚನೆ ನೀಡಿದರು: ‘ಅವಸರವಾಗಿ ಸುಧಾರಣೆ ಮಾಡಹೋಗಬೇಡಿ. ಕೋಪ ವ್ಯಗ್ರತೆ ತೋರಿಸಬೇಡಿ. ಅವನ ದೋಷವನ್ನು ಗಮನಿಸಿದರೂ ಗಮನಿಸದಂತಿರಿ. ಅವನಲ್ಲಿ ಯಾವುದಾದರೂ ಒಳ್ಳೆಯದನ್ನು ಕಂಡಾಗ ಸಂತೋಷ ವ್ಯಕ್ತಪಡಿಸಿ. ಈ ಎಲ್ಲ ದೋಷಗಳಿದ್ದರೂ ನೀನು ನಮ್ಮವನೇ ಎಂಬಂಥ ಆತ್ಮೀಯ ಭಾವನೆ ಬೆಳೆದು ಬರಲಿ, ಅವನ ಅಭಿರುಚಿಯ ಯಾವು ದಾದರೂ ಕೆಲಸದಲ್ಲಿ ಸಣ್ಣಪುಟ್ಟ ಯಶಸ್ಸು ದೊರಕುವಂತೆ ಮಾಡಿ. ಅದು ದೊರಕಿದಾಗ ಒಂದು ಪ್ರಶಂಸೆಯ ಮಾತನ್ನು ಹೇಳಿ. ಅವನು ಖಂಡಿತವಾಗಿ ಒಳ್ಳೆಯ ದಾರಿ ಹಿಡಿಯುತ್ತಾನೆಂಬ ದೃಢ ನಂಬಿಕೆಯಿಂದ ವರ್ತಿಸಿ, ತಂಗಿಯ ಸದ್ಗುಣಗಳೊಂದಿಗೆ ಅವನ ದುರ್ಗಣಗಳ ಹೋಲಿಕೆ ಮಾಡಿ ವಿಮರ್ಶಿಸ ಹೋಗಬೇಡಿ.’

ಅವರ ಮಾತಿನ ಪಾಲನೆಯಿಂದ ಸ್ವಲ್ಪ ಸುಧಾರಣೆ ಕಂಡರೂ ತಾಪತ್ರಯವಿನ್ನೂ ತಪ್ಪಿಲ್ಲ. ಮಗನ ಒಳಿತನ್ನೇ ಸದಾ ಹಾರೈಸಿದ ಈ ತಂದೆ ತಾಯಿಗಳಿಗೇಕೆ ಈ ಶಿಕ್ಷೆ?

ಸಾಮಾನ್ಯ ದೃಷ್ಟಿಗೆ ಗೋಚರವಲ್ಲ. ಉದ್ದೇಶಪೂರಿತವೂ ಅಲ್ಲ. ಆದರೆ ಒಂದು ತೆರನಾದ ದಮನಕಾರರಿಗೂ, ದಮನಕ್ಕೊಳಗಾದವರಿಗೂ ಇರುವ ಸಂಬಂಧವನ್ನು ಇಲ್ಲಿ ಕಾಣಬಹುದಲ್ಲವೇ?

ಪರಸ್ಪರ ಅಪನಂಬಿಕೆ, ದ್ವೇಷ ತಿರಸ್ಕಾರಗಳು ತಂದೆ ತಾಯಿ ಮತ್ತು ಮಗನ ನಡುವೆ ಅಜ್ಞಾತವಾಗಿ ಬೆಳೆದು ಬಂದುದೇ ಇಲ್ಲಿ ಘರ್ಷಣೆಗೆ ಕಾರಣವಲ್ಲವೆ?

ಹುಡುಗನಲ್ಲಿ ಕಾಣಿಸಿಕೊಂಡ ತಂಗಿಯ ಮೇಲಣ ದ್ವೇಷ ಅವನ ‘ಅಹಂ’ನ ಅಸ್ತಿತ್ವಕ್ಕೆ ಬೇಕಾ ದುದೇ ಎಂದುವಾದಿಸಬಹುದಾದರೂ ಅದು ಆರೋಗ್ಯಕರ ಭಾವನೆ ಎನ್ನಲಾದೀತೆ? ಅದು ಮಾನಸಿಕ ವೈಪರೀತ್ಯವಲ್ಲವೇ?


\section*{ಒಲವಿನ ಮಾರ್ದನಿ}

\addsectiontoTOC{ಒಲವಿನ ಮಾರ್ದನಿ}

ತಪ್ಪು ಮಾಡಿದಾಗ ಅಥವಾ ಅಕಸ್ಮಾತ್ ತಪ್ಪುದಾರಿಯಲ್ಲಿ ನಡೆದಾಗ ಅಂಥವರನ್ನು ತಿರಸ್ಕಾರದಿಂದ ನೋಡಿ, ನಿಂದಿಸಿ ಅಪಹಾಸ್ಯಕ್ಕೊಂದು ವಸ್ತುವನ್ನಾಗಿಸುವ ಮಹಾಮಡಿವಂತರು ನಮ್ಮಲ್ಲಿ ನೂರಾರು ಮಂದಿ ಇದ್ದಾರೆ. ತಪ್ಪನ್ನು ತೋರಿಸಿ ಪ್ರೀತಿಯಿಂದ ತಿದ್ದಿ ವ್ಯಕ್ತಿಯ ಆತ್ಮವಿಶ್ವಾಸ ಆತ್ಮಾಭಿಮಾನಕ್ಕೆ ಕುಂದಾಗದಂತೆ ನೋಡಿಕೊಂಡು ಅವನನ್ನು ಸರಿದಾರಿಗೆ ತರುವ ತಾಳ್ಮೆಯಿಂದ ಲಭ್ಯವಾಗುವ ಕಲೆ ನಮ್ಮಲ್ಲಿ ಎಷ್ಟು ಮಂದಿಗೆ ತಿಳಿದಿದೆ? ಸಿದ್ಧಿಸಿದೆ? ನಮಗೆ ವಿದ್ಯೆ ಹೇಳುವ ಅಧ್ಯಾಪಕರು, ಹೊತ್ತು ಹೆತ್ತು ಸಾಕಿ ಸಲಹಿದ ಮಾತಾಪಿತೃಗಳು ಈ ವಿಚಾರದಲ್ಲಿ ಎಷ್ಟರಮಟ್ಟಿಗೆ ನೆರವಾಗಬಲ್ಲರು? ತಪ್ಪಿ ನಡೆದಾಗ ಸಹನೆ ಕಳೆದುಕೊಂಡಾಗ ಉಗ್ರವಾಗಿ ಬೈದು ಭಂಗಿಸಿ ‘ಹಾಗೆ ಮಾಡಬಾರದು’, ‘ಹೀಗೆ ಮಾಡಬಾರದು’ ಎಂದು ಬೋಧನೆ ನೀಡಿ ಮಕ್ಕಳ ಪಾಡಿಗೆ ಅವರನ್ನು ಬಿಟ್ಟುಬಿಟ್ಟರೆ ಹಿರಿಯರ ಕರ್ತವ್ಯ ಮುಗಿಯಿತು! ಒಬ್ಬ ಸ್ವಲ್ಪ ಕಾಲುಜಾರಿದರೆ, ಆದಷ್ಟು ಬೇಗನೆ ಅವನನ್ನು ಬೀಳುವಂತೆ ಮಾಡಿ ಮೇಲೇಳದಂತೆ ನೋಡಿಕೊಳ್ಳುವಷ್ಟು ವ್ಯಂಗ್ಯಟೀಕೆಗಳನ್ನೂ ನಿಷೇಧಾತ್ಮಕ ಭಾವನೆಗಳನ್ನೂ ಬಿತ್ತಿ ಆಮೇಲೆ ‘ಅಯ್ಯೋ ಪಾಪ!’ ಎನ್ನುವಂಥ ತೋರಿಕೆಯ, ಅನುಕಂಪೆಯ ಮಾತಿನ ಮಳೆಗರೆಯುವವರೇ ಹೆಚ್ಚು ಮಂದಿ! ಹಾಗೆಂದು ಇಂಥ ಜನಗಳ ಮೇಲೆ ದೋಷಾರೋಪಣೆ ಮಾಡಿದ ಮಾತ್ರದಿಂದ ಬಹಳ ದೊಡ್ಡ ಕೆಲಸವಾಗಲಿಲ್ಲ. ಅವರು ಇದನ್ನು ಯಾವುದೋ ಹಳೆಯ ಜಾಡಿನಲ್ಲಿ ನಡೆಯುವ ಅಭ್ಯಾಸ ಬಲದಿಂದ ಮಾಡುತ್ತಾರಲ್ಲದೆ ಉದ್ದೇಶ ಪೂರ್ವಕವಾಗಿ ಮಾಡುತ್ತಾರೆನ್ನುವಂತಿಲ್ಲ. ಅತ್ತೆಯಾದವಳು ತಾನು ಸೊಸೆಯಾಗಿರುವಾಗ ತನ್ನ ಅತ್ತೆಯು ನೀಡಿದ ಅಸಂಖ್ಯ ಬೈಗಳು ಮತ್ತು ಕುಹಕದ ಮಾತುಗಳನ್ನು ನುಂಗಿದವಳು. ತನ್ನ ಸೊಸೆ ಬಂದಾಗ ತಾನು ತಿಂದು ಅರಗಿಸಿಕೊಂಡವಾಗ್ಬಾಣಗಳಲ್ಲಿ ಕೆಲವನ್ನಾದರೂ ಸೊಸೆಯ ಮೇಲೆ ಎಸೆಯಲು ಅಜ್ಞಾತವಾಗಿ ಸಜ್ಜಾಗಿರುತ್ತಾಳೆ. ಹಿರಿಯ ವಿದ್ಯಾರ್ಥಿಗಳು ಹಾಸ್ಟೆಲ್ ಸೇರಿದಾಗ ರ್ಯಾಗಿಂಗ್ ಮಾಡಿಸಿಕೊಂಡು ನೋವು ತಿಂದವರು. ಹೊಸವರುಷ ಕಿರಿಯ ವಿದ್ಯಾರ್ಥಿಗಳು ಹಾಸ್ಟೆಲ್ ಸೇರಲು\break ಬಂದಾಗ ತಾವು ಬಳುವಳಿಯಾಗಿ ಪಡೆದುದನ್ನು ಅವರಿಗೆ ನೀಡಿ ಸಂತೃಪ್ತಿ ಪಡೆಯುತ್ತಾರೆ. ತಾವು ಮಕ್ಕಳಾಗಿದ್ದಾಗ ತಮ್ಮ ಅಧ್ಯಾಪಕರು ಹೊಡೆದು ಬಡಿದು ಕ್ರುದ್ಧರಾಗಿ ನೋಡಿಕೊಂಡ ವಿಧಾನವನ್ನು ಮೆಚ್ಚಿರದಿದ್ದರೂ, ಈಗ ಅಧ್ಯಾಪಕನಾದವನು ಆ ಪದ್ಧತಿಯನ್ನೇ ಮುಂದುವರಿಸಿಕೊಂಡು ಬರುತ್ತಾನೆ. ತಾಳ್ಮೆ ಮತ್ತು ಪ್ರೀತಿ–ಇವು ಅನುಸರಿಸುವುದಕ್ಕಲ್ಲ, ಭಾಷಣ ಮತ್ತು ಬರಹಕ್ಕೆ ಮಾತ್ರ ಉಪಯೋಗವಾಗುವ ಸರಕೆಂದು ಭಾವಿಸುತ್ತಾನೆ. ಗುಲಾಮಗಿರಿಯಲ್ಲಿ ಬಾಳಿದ ಜನಾಂಗದಲ್ಲಿ ಕಂಡುಬರುವ ಒಂದು ಚಟ ಇದು: ತಮಗಿಂತ ಮೇಲಿರುವವರ ಎದುರಲ್ಲಿ ದೈನ್ಯಪ್ರದರ್ಶನ, ತಮಗಿಂತ ಸ್ವಲ್ಪ ಕೆಳಗಿನ ಸ್ಥಾನದಲ್ಲಿರುವವರ ಮೇಲೆ ದರ್ಪ ಪ್ರದರ್ಶನ.

ಪಶ್ಚಿಮದ ಪ್ರವಾಸೀ ಕಲಾವಿದನೊಬ್ಬ ಭಾರತಕ್ಕೆ ಬಂದು ಸಂಚಾರ ಮುಗಿಸಿ ಅಲ್ಲಿಗೆ ಹಿಂದಿರುಗಿದಾಗ ಹೀಗೆಂದು ತನ್ನ ಅನುಭವವನ್ನು ಹೇಳಿದ: “ಭಾರತದ ಎಲ್ಲ ಕಡೆ ನೋಡುತ್ತಲಿದ್ದೇನೆ. ಪ್ರತಿಯೊಬ್ಬರೂ ಈ ದೇಶದಲ್ಲಿ ಇನ್ನೊಬ್ಬರನ್ನು ಸಂಧಿಸಿದಾಗ ಈತ ತನಗಿಂತ ಅಧಿಕಾರ ಸ್ಥಾನಮಾನಗಳಲ್ಲಿ ಮೇಲಿನವನೆ? ಕೆಳಗಿನವನೆ? ಎಂದು ತಿಳಿದು ಆ ರೀತಿ ವರ್ತಿಸುತ್ತಾರೆಯೆ ಹೊರತು ಈತ ತನ್ನ ‘ಸಹೋದ್ಯೋಗಿ’ ‘ಸಮಾನಶೀಲ’ ಎಂಬ ಭಾವನೆಯಿಂದಲ್ಲ.” ನಮ್ಮ ಸದ್ಯದ ಸಾಮಾಜಿಕ, ಸಾಂಸ್ಕೃತಿಕ ಶೈಕ್ಷಣಿಕ ಪರಂಪರೆ ಪರಿಸ್ಥಿತಿಗಳಲ್ಲಿ ಇದು ಎಲ್ಲೆಡೆ ಸಾಧ್ಯವಿಲ್ಲವೆ ನ್ನೋಣ. ನಮ್ಮಲ್ಲಿ ಕ್ರಾಂತಿಕಾರಿಗಳು, ಸಮತೆಯ ಪ್ರವಾದಿಗಳು, ವಿಚಾರವಂತರೆನಿಸಿಕೊಂಡವರು, ಮುಖಂಡರು ಕಾರ್ಯಕರ್ತರೊಂದಿಗೆ ‘ಮೇಸ್ತ್ರಿ ಕೂಲಿ’ ಸಂಬಂಧವನ್ನೇ ಹೆಚ್ಚಾಗಿ ಮೆಚ್ಚು ವುದನ್ನು ಕಾಣಬಹುದು. ಪರಸ್ಪರ ವಿಚಾರ ವಿನಿಮಯ, ಸಹಕಾರ ಮತ್ತು ಜವಾಬ್ದಾರಿಯಿಂದ ಒಂದು ಉಪಯುಕ್ತ ಕೆಲಸವನ್ನು ಎಲ್ಲರೂ ಸೇರಿ ನಿರ್ವಹಿಸುವಲ್ಲಿ ಈ ಭಾವನೆ ಎಂಥ ತಡೆಯನ್ನು ತಂದೀತೆಂಬುದನ್ನು ನಮ್ಮಲ್ಲಿ ರಚನಾತ್ಮಕ ಕಾರ್ಯಗಳಲ್ಲಿ ಅನುಭವವಿಲ್ಲದ ಹೆಚ್ಚಿನ ಬುದ್ಧಿ ಜೀವಿಗಳು ತಿಳಿದಿಲ್ಲವೆಂದೇ ಹೇಳಬೇಕು. ತಿಳಿದರೂ ಕಾರ್ಯರೂಪಕ್ಕೆ ಹೇಗೆ ತರಬೇಕೆಂದು ಯೋಚಿಸುತ್ತಿಲ್ಲ. ಯೋಚಿಸಿದರೂ ಹಾಕಿಕೊಂಡ ಯೋಜನೆಗಳನ್ನು ಅನುಸರಿಸುತ್ತಿಲ್ಲ.

ಸಾಮಾಜಿಕ ಕ್ರಾಂತಿ, ಜಾತಿ ನಿರ್ಮೂಲನ ಮೊದಲಾದ ವಿಷಯಗಳನ್ನು ಕುರಿತು ಕಾಳಜಿ ವಹಿಸಿ ಭಾವೋದ್ವೇಗದಿಂದ ಮಾತನಾಡುವ ಉನ್ನತ ಸ್ಥಾನದಲ್ಲಿದ್ದ ದೊಡ್ಡ ಆಫೀಸಿನ ಪರಿಚಿತ ಮುಖ್ಯಸ್ಥರನ್ನೊಮ್ಮೆ ಪ್ರಶ್ನಿಸಿದೆ: ‘ನೀವು ಎಂದಾದರೂ ನಿಮ್ಮ ಇಲಾಖೆಯ ಸಿಬ್ಬಂದಿವರ್ಗದವರನ್ನು ಸೇರಿಸಿಕೊಂಡು ಎಲ್ಲರೂ ಪಾಲುಗೊಳ್ಳುವ ಸಭೆ ಸಮಾರಂಭ ಕೂಟ ನಡೆಸಿದ್ದಿದೆಯೇ?’ ‘ಇಲ್ಲ’ ಎಂದರವರು. ವರ್ಷಗಳ ಕಾಲ ಒಂದೇ ಇಲಾಖೆ, ಒಂದೇ ಮಾಡಿನಡಿಯಲ್ಲಿ ದುಡಿಯುವ ಹೆಚ್ಚು ಕಡಿಮೆ ಸಂಬಳ ಬರುವ ಜನರಲ್ಲಿ ಒಂದು ತೆರನಾದ ಯಾಂತ್ರಿಕ ಮುಖ ಪರಿಚಯವಿತ್ತೇ ಹೊರತು ಪರಸ್ಪರ ಪ್ರೀತಿ ವಿಶ್ವಾಸ ಸಹಕಾರಗಳಿಂದೊಡಗೂಡಿದ ಬಂಧುತ್ವದ ಭಾವನೆ ಮೂಡಿರ ಲಿಲ್ಲ. ಮುಖ್ಯಸ್ಥರು ಇನ್ನೊಂದು ಇಲಾಖೆಯ ಮುಖ್ಯಸ್ಥರೊಂದಿಗೆ ಮಾತ್ರ ಬೆರೆಯುತ್ತಿದ್ದರು.

ಶಿಕ್ಷಣ ಮತ್ತು ತರಬೇತಿ ಶೋಷಿತರಿಗೆ ಮಾತ್ರವಲ್ಲ, ಶೋಷಕರಿಗೂ ಅಗತ್ಯವಲ್ಲವೆ?\break ಶ‍್ರೀಸಾಮಾನ್ಯರಿಗೆ ಮಾತ್ರವಲ್ಲ, ಮುಖಂಡರಿಗೂ ಬೇಕಲ್ಲವೆ? ಅವಿದ್ಯಾವಂತರಿಗೆ ಮಾತ್ರವಲ್ಲ, ವಿದ್ಯಾವಂತರಿಗೂ ಆವಶ್ಯಕವಲ್ಲವೆ?

ಎಲ್ಲೆಲ್ಲೂ ಪ್ರಾರಂಭದಿಂದಲೇ ಹೊಡೆತವನ್ನು ತಿಂದು, ದಿಕ್ಕೆಟ್ಟು ಓಡಿ ಭಯದಿಂದ ಕಂಗೆಟ್ಟ ಬೆಕ್ಕೊಂದು ತಡೆಯಲಾರದ ಹಸಿವಿನಿಂದ ಶ‍್ರೀಮಂತರೊಬ್ಬರ ಮನೆಯ ಸಮೀಪದ ತೋಟದ ಮೂಲೆಯಲ್ಲಿ ಹೆದರಿಕೊಂಡು ಕುಳಿತಿತ್ತು. ಪ್ರಾಣಿ ಪಕ್ಷಿಗಳನ್ನು ಪ್ರೀತಿಯಿಂದ ಕಂಡು ಅವುಗಳನ್ನು ಸಾಕಿ ಸಲಹುವುದು ಆ ಮನೆಯ ಮಕ್ಕಳು ಬೆಳೆಸಿಕೊಂಡು ಬಂದ ಹವ್ಯಾಸ. ಬೆಕ್ಕಿನ ಸದ್ಯದ ಸ್ಥಿತಿಯನ್ನು ಕಂಡೊಡನೆ ಮಕ್ಕಳು ಅದನ್ನು ಸಮೀಪಿಸಿ ಪ್ರೀತಿಯಿಂದ ಕರೆದು ಹಸಿದು ಕಂಗಾಲಾ ಗಿದ್ದ ಅದಕ್ಕೆ ಆಹಾರ ನೀಡಿದರು. ಬೆಕ್ಕು ಕುಳಿತ ಜಾಗದಿಂದ ಕದಲಿಲ್ಲ. ಹಸಿದಿದ್ದರೂ ಆಹಾರವನ್ನು ಮುಟ್ಟಲಿಲ್ಲ. ಮಕ್ಕಳು ಹತ್ತಿರ ಹೋದರೆ ಭಯದಿಂದ ಕೋಪ ತೋರಿಸಿ ಬುಸುಗುಟ್ಟು ತ್ತಿತ್ತು. ಮಕ್ಕಳೆಲ್ಲ ಮನೆಯೊಳಗೆ ಬಂದು ಅದಕ್ಕೆ ಕಾಣಿಸದಂತೆ ಏನು ಮಾಡುತ್ತಿದೆ ಎಂದು ಇಣುಕಿ ನೋಡತೊಡಗಿದರು. ಬೆಕ್ಕು ಅತ್ತಿತ್ತ ಬೆದರುಗಣ್ಣಿನಿಂದ ನೋಡಿ ಯಾರೂ ಇಲ್ಲವೆಂದು ತಿಳಿದು ಕೊಟ್ಟದ್ದನ್ನು ತಿಂದು ಹೊರಟು ಹೋಯಿತು. ದಿನವೂ ಅದು ತೋಟದ ಮೂಲೆಯಲ್ಲಿ ಬಂದು ಕುಳಿತು ಮಕ್ಕಳು ಕೊಟ್ಟ ಹಾಲನ್ನು ಕುಡಿದು ಹೋದರೂ ಅವರನ್ನು ಸಂಶಯ ಮತ್ತು ಬೆದರುಗಣ್ಣುಗಳಿಂದಲೇ ನೋಡುತ್ತಿತ್ತು. ಆರು ತಿಂಗಳ ಕಾಲ ಹಾಲು ಕುಡಿದ ಮೇಲೆ ಮೆಲ್ಲನೆ ಮಕ್ಕಳನ್ನು ಕಂಡು ‘ಮಿಯಾವ್​’ ಎಂದಿತು. ಹೆಚ್ಚು ಕಡಿಮೆ ಒಂದು ವರ್ಷ ಕಾಲ ಮಕ್ಕಳ ನಿರಂತರ ಪ್ರೀತಿಯನ್ನು ಕಂಡು ವಿಶ್ವಾಸ ತಾಳಿ ಅವರ ಬಳಿ ಬಂದು ಅವರ ಮಡಿಲಲ್ಲಿ ಕುಳಿತು ಆಡತೊಡಗಿತು. ಹುಟ್ಟಿದಂದಿನಿಂದ ಹೊಡೆತ ನೋವುಗಳನ್ನು ಅನುಭವಿಸಿದ ಅದು ಮೊದ ಮೊದಲು ಮಕ್ಕಳ ಪ್ರೀತಿಯನ್ನು ಅರ್ಥಮಾಡಿಕೊಳ್ಳುವ ಸ್ಥಿತಿಯಲ್ಲೇ ಇರಲಿಲ್ಲ!

ಮುನ್ನಡೆದವರೇ! ಬುದ್ಧಿವಂತರೇ! ದಕ್ಷರಾದವರೇ! ಸುಸಂಸ್ಕೃತರೇ! ತುಳಿತಕ್ಕೊಳಗಾದವರಿಗೆ ಬೇಕು ನಿರ್ಮಲ ಪ್ರೀತಿ. ಕ್ಷಣಿಕ ಉದ್ವೇಗದಿಂದ ಹೊರಹೊಮ್ಮುವ ‘ನಾನು ದಾನಿ’ತನದಿಂದ ಮಾಡುವ ಜಾಹೀರಾತಿನ ಉಪಕಾರವಲ್ಲ.

ಈ ಸಹನೆ ಮತ್ತು ನಿರ್ಮಲ ಪ್ರೀತಿಯನ್ನು ಯಾರು ನೀಡಬಲ್ಲರು? ತುಳಿತಕ್ಕೊಳಗಾದವರ ಮುಖಂಡನು ಈ ನಿರ್ಮಲ ಪ್ರೀತಿಯನ್ನು ತಾನು ಕಂಡಿದ್ದರೆ, ಉಂಡಿದ್ದರೆ, ತನ್ನ ಬಂಧುಗಳಿಗೂ ಅದನ್ನು ನೀಡುತ್ತಿದ್ದ. ಇಲ್ಲವಾದರೆ ಅದನ್ನು ಅವನು ಅವರಿಗೆ ಹೇಗೆ ನೀಡಲಾಪ? ನಮಗೆ ಸಾಧ್ಯವಿರಲಿ ಇಲ್ಲದಿರಲಿ, ಈ ನಿಷ್ಕಲ್ಮಷ ಪ್ರೀತಿಯೊಂದೇ ಅಭ್ಯುದಯದ ರಾಜಮಾರ್ಗ ಎಂಬುದನ್ನು ಮರೆಯಲಾಗದು. ಇದನ್ನು ಮೊದಲು ಎತ್ತರದಲ್ಲಿದ್ದವರು ತಿಳಿದುಕೊಂಡು ಕೆಳಗಿನ ಹಂತಗಳಲ್ಲಿರು\-ವವರಿಗೆ ಶ್ರದ್ಧೆ ಮತ್ತು ವಿನಮ್ರ ಭಾವದಿಂದ ಹಂಚಬೇಕು.


\section*{‘ಅಹಂ’ನ ಅಸಂಖ್ಯ ವೇಷಗಳು}

\addsectiontoTOC{‘ಅಹಂ’ನ ಅಸಂಖ್ಯ ವೇಷಗಳು}

ಮನುಷ್ಯನ ಅಸ್ತಿತ್ವದ ಸಮಸ್ಯೆಗಳೆಲ್ಲ ಮುಖ್ಯವಾಗಿ ಹೊಂದಾಣಿಕೆ ಅಥವಾ ಸಮನ್ವಯದ ಸಮಸ್ಯೆ\-ಗಳೇ – ಬಗೆಹರಿಸಲಾರದ್ದೆಂದು ತಿಳಿದುಕೊಂಡ ಒಡಕು ಅಥವಾ ಬಿರುಕಿನಿಂದ ಉದ್ಭವಿಸಿದ ಸಮಸ್ಯೆಗಳೆ. ಒಡಕನ್ನು ಮುಚ್ಚುವ ಅಥವಾ ಸರಿಪಡಿಸುವ ಸೂತ್ರವಿದೆ. ನಾವೆಲ್ಲರೂ ಒಪ್ಪಲೇ\-ಬೇಕಾದ ಮೂಲತತ್ತ್ವದ ಅರಿವಿನ ಅಭಾವದಿಂದ ಸಮಸ್ಯೆಗಳು ಉದ್ಭವಿಸಿ ಉಳಿದು ಕೊಳ್ಳುತ್ತವೆ. ಆ ಮೂಲತತ್ತ್ವದ ಅರಿವು ದೊರೆತೊಡನೆ ಸಮಸ್ಯೆಯ ಪರಿಹಾರ ತನ್ನಿಂದ ತಾನೇ ಆಗುತ್ತದೆ. ಈ ವಿಚಾರವನ್ನು ‘ಜೀವಶಿವ ಸೂತ್ರ’ದಲ್ಲಿ ಹೆಚ್ಚು ವಿಶದವಾಗಿ ಸ್ಪಷ್ಟಪಡಿಸಲಾಗಿದೆ.

ಪ್ರಾಣಿಗಳಲ್ಲಿ ಹೊಟ್ಟೆಪಾಡಿನ ಹೋರಾಟವೇ ಮುಖ್ಯ. ಮನುಷ್ಯನಲ್ಲಿ ಹೊಟ್ಟೆಯ ಚಿಂತೆ ಮುಗಿದ ಮೇಲೆ ‘ನಾನು’ ಅಥವಾ ‘ಅಹಂ’ನ ಅಸ್ತಿತ್ವವನ್ನು ಕಾಯ್ದುಕೊಳ್ಳುವ ಹೋರಾಟ ನಡೆದಿರುತ್ತದೆ. ಮನುಷ್ಯ–ಮನುಷ್ಯರೊಳಗಿನ ಸಂಬಂಧದ ಸಮಸ್ಯೆಗಳಲ್ಲಿ ಈ ‘ಅಹಂ’ನ ಆಟ ತಾಕಲಾಟಗಳಿಂದ ಉಂಟಾಗುವವುಗಳು ಬಹಳ ಪ್ರಧಾನಪಾತ್ರ ವಹಿಸುತ್ತವೆ. ‘ಅಹಂ’ನ ಅಸಂಖ್ಯ ವೇಷಗಳು ತಾದಾತ್ಮ್ಯ ಸಂಬಂಧದಿಂದ ಹೇಗೆ ಉಂಟಾಗಬಹುದೆಂಬುದನ್ನು ಮುಂದೆ ಸೂಚಿಸಿದೆ. ಈ ವರೆಗೆ ನೀಡಿದ ವಿವರಣೆಗಳ ಹಿನ್ನೆಲೆಯಿಂದ ಅದನ್ನು ಹೆಚ್ಚು ವಿವರವಾಗಿ ಅರ್ಥೈಸಿಕೊಳ್ಳಬಹುದು. ತಮ್ಮ ಸುತ್ತಮುತ್ತಲ ಜನರಲ್ಲಿ ಅದು ವ್ಯಕ್ತವಾಗುವ ವಿಧಾನವನ್ನೂ ಓದುಗರು ಪರಿಶೀಲಿಸಬಹುದು–

(ಅ) ಸ್ಥೂಲದೇಹ ಮತ್ತು ಅದರ ಗುಣಧರ್ಮಗಳೊಂದಿಗೆ ‘ನಾನು’ ನಾನಾ ತೆರನಾಗಿ ಸೇರಿ ಕೊಂಡಾಗ ಅಥವಾ ತಾದಾತ್ಮ್ಯಗೊಂಡಾಗ–

\begin{longtable}{@{}l@{\hspace{1cm}}l@{}}
ನಾನು & ಆರೋಗ್ಯವಾಗಿದ್ದೇನೆ \\
ನಾನು & ತೆಳ್ಳಗಿದ್ದೇನೆ \\
ನಾನು & ಬೆಳ್ಳಗಿದ್ದೇನೆ \\
ನಾನು & ಹಸಿದಿದ್ದೇನೆ \\
ನಾನು & ಬಾಯಾರಿ ಬಳಲಿದ್ದೇನೆ \\
ನಾನು & ಕುಂಟ \\
ನಾನು & ಕುರುಡ \\
ನಾನು & ಯುವಕ \\
ನಾನು & ವೃದ್ಧ \\
ನಾನು & ಹಿಂದೂ \\
ನಾನು & ಕ್ರಿಶ್ಚಿಯನ್ \\
ನಾನು & ಮುಸಲ್ಮಾನ \\
ನಾನು & ಬಿಳಿಯ \\
\end{longtable}

(ಆ) ‘ನಾನು’ ಮಾನಸಿಕ ಗುಣಧರ್ಮಗಳ ಏರಿಳಿತಗಳೊಂದಿಗೆ ಸೇರಿಕೊಂಡಾಗ–

\begin{longtable}{@{}l@{\hspace{1cm}}l@{}}
ನಾನು & ಕೋಪಿಷ್ಠ \\
ನಾನು & ಸರಳ ವ್ಯಕ್ತಿ \\
ನಾನು & ಸುಖಿ \\
ನಾನು & ಧನ್ಯ \\
ನಾನು & ದುಃಖಿ \\
ನಾನು & ನಿರಹಂಕಾರಿ \\
\end{longtable}

(ಇ) ‘ನಾನು’ ಮಾನಸಿಕ ಶಕ್ತಿ, ಬುದ್ಧಿಶಕ್ತಿ, ದಕ್ಷತೆ ಮತ್ತು ಸಾಮರ್ಥ್ಯಗಳೊಂದಿಗೆ ಸೇರಿ\-ಕೊಂಡಾಗ–

\begin{longtable}{@{}l@{\hspace{1cm}}l@{}}
ನಾನು & ವೈಜ್ಞಾನಿಕ ಮನೋಭಾವದವನು \\
ನಾನು & ಬರಹಗಾರ \\
ನಾನು & ಕಲಾವಿದ \\
ನಾನು & ಬುದ್ಧಿಜೀವಿ \\
ನಾನು & ವಿಜ್ಞಾನಿ \\
ನಾನು & ದಡ್ಡ \\
ನಾನು & ಪ್ರಚಂಡ ಮೇಧಾವಿ \\
ನಾನು & ಸತ್ಯಾನ್ವೇಷಿ \\
\end{longtable}

(ಈ) ‘ನಾನು’ ನೈಸರ್ಗಿಕ ಕಾರ್ಯ, ಕರ್ತೃತ್ವ ಕರ್ತವ್ಯಗಳೊಂದಿಗೆ ಸೇರಿ\-ಕೊಂಡಾಗ–

\begin{longtable}{@{}l@{\hspace{1cm}}l@{}}
ನಾನು & ತಂದೆ \\
ನಾನು & ತಾಯಿ \\
ನಾನು & ಹೆಂಡತಿ \\
ನಾನು & ಅಧ್ಯಾಪಕ \\
ನಾನು & ಆಫೀಸರ್​ \\
ನಾನು & ಸಾಧು \\
\end{longtable}

(ಉ) ‘ನಾನು’ ಸಂಪತ್ತು, ಸ್ಥಿತಿಗತಿ ಮತ್ತು ಸಲಕರಣೆಗಳೊಂದಿಗೆ ತಾದಾತ್ಮ್ಯಗೊಂಡಾಗ–

\begin{longtable}{@{}l@{\hspace{1cm}}l@{}}
ನಾನು & ಬಡವ \\
ನಾನು & ದಲಿತ \\
ನಾನು & ಶೋಷಿತ \\
ನಾನು & ಅನಾಥ \\
ನಾನು & ಈ ಮನೆಯ ಒಡೆಯ \\
ನಾನು & ಆರ್ಥಿಕವಾಗಿ ಗಟ್ಟಿಕುಳ \\
\end{longtable}

(ಊ) ‘ನಾನು’ವಿವಿಧ ಜಾತಿಮತ ಪಂಥ ಪಂಗಡಗಳೊಂದಿಗೆ ತಾದಾತ್ಮ್ಯಗೊಂಡಾಗ–

\begin{longtable}{@{}l@{\hspace{1cm}}l@{}}
ನಾನು & ಒಕ್ಕಲಿಗ \\
ನಾನು & ಲಿಂಗಾಯತ \\
ನಾನು & ಬ್ರಾಹ್ಮಣ \\
ನಾನು & ಸ್ಮಾರ್ತ \\
ನಾನು & ಮಾಧ್ವ \\
ನಾನು & ಶ್ರೀವೈಷ್ಣವ \\
ನಾನು & ಹರಿಜನ \\
ನಾನು & ಗಿರಿಜನ \\
ನಾನು & ಸಸ್ಯಾಹಾರಿ \\
ನಾನು & ವಿಚಾರವಾದಿ \\
ನಾನು & ಸಾಮ್ಯವಾದಿ \\
ನಾನು & ಸಾರ್ತರ್​ವಾದಿ \\
\end{longtable}

(ಪು) ‘ನಾನು’ ಅವಸ್ಥಾತ್ರಯ ಮತ್ತು ಆತ್ಮದ ಸ್ಥಿತಿಗಳಿಗೆ ಪ್ರತೀಕವಾದಾಗ–

\begin{longtable}{@{}l@{\hspace{1cm}}>{\raggedright}p{5cm}@{}}
ನಾನು & ಎಚ್ಚರದಲ್ಲಿರುವಾಗ \tabularnewline
ನಾನು & ಸ್ವಪ್ನಾವಸ್ಥೆಯಲ್ಲಿ ಅದ್ಭುತ ದೃಶ್ಯ ಕಂಡಾಗ \tabularnewline
ನಾನು & ಗಾಢ ನಿದ್ರೆಯಲ್ಲಿ ಮುಳುಗಿದಾಗ \tabularnewline
ನಾನು & ದೇಹಾತೀತ ಅನುಭವಗಳನ್ನು ಪಡೆದಾಗ \tabularnewline
ನಾನು & ಒಮ್ಮೆ ಇಂಗ್ಲೆಂಡಿನಲ್ಲಿ ಹುಟ್ಟಿದ್ದಾಗ \tabularnewline
ನಾನು & ಕ್ರಿಸ್ತಪೂರ್ವ ಐದು ಸಾವಿರ ವರ್ಷಗಳ ಹಿಂದೆ ಮಧ್ಯ ಪ್ರಾಚ್ಯದಲ್ಲಿ ಹುಟ್ಟಿದಾಗ \tabularnewline
ನಾನು & ಗಂಡುಶರೀರದಲ್ಲಿದ್ದಾಗ \tabularnewline
ನಾನು & ಹೆಣ್ಣಾಗಿ ಹುಟ್ಟಿದಾಗ \tabularnewline
\end{longtable}

(ಪೂ) ‘ನಾನು’ ಆಧ್ಯಾತ್ಮಿಕ ಭಾವದೊಂದಿಗೆ ತಾದಾತ್ಮ್ಯಗೊಂಡಾಗ–

\begin{longtable}{@{}l@{\hspace{1cm}}l@{}}
ನಾನು & ದೇವರ ಭಕ್ತ \\
ನಾನು & ದೇವರ ಸೇವಕ \\
ನಾನು & ದೇವರ ಅಂಶ \\
ನಾನು & ಆತ್ಮ \\
\end{longtable}


\section*{‘ನಾನು’ವಿನ ಆಳದಲ್ಲಿ ನಾವೆಲ್ಲ ಒಂದು}

\addsectiontoTOC{‘ನಾನು’ವಿನ ಆಳದಲ್ಲಿ ನಾವೆಲ್ಲ ಒಂದು}

ಮೇಲಿನ ವಾಕ್ಯಗಳಲ್ಲಿ ‘ನಾನು’ ಎನ್ನುವುದು ಸ್ಥಾಯಿಯಾಗಿದೆ. ಎಲ್ಲ ಅನುಭವಗಳಿಗೂ ಸಾಕ್ಷಿ ಯಾಗಿ ನಿರಂತರವೂ ಉಳಿದುಕೊಂಡಿದೆ. ಉಳಿದ ಎಲ್ಲ ಅನುಭವಗಳೂ ಉದಯಿಸುತ್ತವೆ, ಬದಲಾಗುತ್ತವೆ, ಅಸ್ತಮಿಸುತ್ತಿವೆ. ದೃಶ್ಯವೂ, ನಿರಂತರ ಪರಿವರ್ತನಶೀಲವೂ, ವಿನಾಶಶೀಲವೂ ಆದ ಆ ಅನುಭವಗಳನ್ನೆಲ್ಲ ಬೆಳಗುತ್ತ ಸ್ಥಾಯಿಯಾಗಿ ನಿಂತಿರುವ ಈ ‘ನಾನು’ವಿನ ಮೂಲಕ ವ್ಯಕ್ತವಾಗುವ ತತ್ತ್ವ ಯಾವುದು? ‘ನಾನು’ ಯಾವಾಗಲೂ ಇದ್ದುಕೊಂಡಿದೆ ಎನ್ನುವುದಕ್ಕಿಂತಲೂ ನೈಜಸ್ಥಿತಿಯಲ್ಲಿ ಅದು ‘ಇರುವಿಕೆಯೇ ಆಗಿದೆ’ ಎನ್ನುವುದು ಹೆಚ್ಚು ಸಮಂಜಸ ಎಂಬುದು ಸ್ವಲ್ಪ ಗಾಢವಾಗಿ ಯೋಚಿಸಿದರೆ ಸ್ಪಷ್ಟವಾಗುವುದು. ‘ನಾನು ಇದ್ದೇನೆ’ ಎಂಬ ಮಾತಿನಲ್ಲಿ ‘ನಾನು ಇದ್ದೇನೆ ಎಂಬುದು ನನಗೆ ತಿಳಿದಿದೆ’ ಎಂಬುದನ್ನೇ ಸೂಚ್ಯವಾಗಿ ಹೇಳುತ್ತಿದ್ದೇನಷ್ಟೆ. ದೈನಂದಿನ ವ್ಯವಹಾರದಲ್ಲಿ ಇರುವಿಕೆ ಅರಿಯುವಿಕೆ ಬೇರೆ ಬೇರೆ ಎಂದು ತೋರಿದರೂ ನನ್ನ ಅಸ್ತಿತ್ವ ಅಥವಾ ಇರವಿನ ಅರಿವು ಸಮಸಮಯದಲ್ಲೇ ಉಂಟಾಗುವುದರಿಂದ ‘ನಾನು’ ಎನ್ನುವುದು ಅರಿವಿರವುಗಳ ಗಟ್ಟಿಯೇ ಆಗಿದೆ. ಯಥಾರ್ಥವಾಗಿ ವಿವಿಧ ಪರಿಮಿತಿಗಳಾದ ಕಾಮ, ಮೋಹ, ಸಂಶಯ, ವಿಪರೀತ ಭಾವನೆಗಳಿಂದ ಅದನ್ನು ಬಿಡಿಸಿ, ವಿಂಗಡಿಸಿ ತಿಳಿದುಕೊಂಡಾಗ ಅದು ಅರಿವಿರವು ಒಲವು ಅಥವಾ ಸತ್ ಚಿತ್ ಆನಂದದ ಸಾಗರವೇ ಆಗಿದೆ ಎಂಬುದು ಅನುಭವಿಯ ಪಾಲಿಗೆ ಮಹಾ ಸತ್ಯವಾಗುವುದು. ನಮ್ಮೆಲ್ಲರ ನಿಜದ ನೆಲೆ ಅದು. ಅಲ್ಲಿ ನಾವೆಲ್ಲರೂ ಒಂದು. ಸತ್ಯಾನ್ವೇಷಣೆಯ ಹಂಬಲವಿರುವವರಿಗೆ ವಿಚಾರ ಯುಕ್ತಿಗಳಿಂದಲೂ ಈ ಪರಮಸತ್ಯದ ಒಂದು ಬೌದ್ಧಿಕವಾದ ಕ್ಷಣಿಕ ದರ್ಶನ ಅಥವಾ ಇಣುಕುನೋಟ ಸಾಧ್ಯ.

ಮನುಷ್ಯನ ಅಂತರಂಗದಲ್ಲಿರುವ ಮೂರು ಮುಖ್ಯ ಸೆಳೆತಗಳು ಸ್ವಾಭಾವಿಕ, ಸರ್ವಸಾಮಾನ್ಯ, ಸರ್ವವೇದ್ಯ. ಇವು ನಮ್ಮ ನೈಜಸ್ವರೂಪವನ್ನು ತಿಳಿಸುವ ಮೂಲಭೂತ ಅನಿಸಿಕೆಗಳು. ಈ ಮೂರು ಸೆಳೆತಗಳು ಮೂರು ವಿಧವಾಗಿ ಪ್ರಕಟವಾಗುವುದನ್ನು ಕಾಣುತ್ತೇವೆ. ಬದುಕಬೇಕು—ನಾಶ\-ವಾಗ\-ಬಾರದು, ತಿಳಿದುಕೊಳ್ಳಬೇಕು—ಅಜ್ಞಾನಿ\-ಯಾಗಿರ\-ಬಾರದು, ಆನಂದಿಸ\-ಬೇಕು—ದುಃಖಿ\-ಯಾಗ\-ಬಾರದು ಎಂಬಿವುಗಳೇ ಆ ಮೂರು ಹಂಬಲಗಳು. ಈ ನಮ್ಮ ಜಗತ್ತಿನಲ್ಲಿ ಸ್ವಸ್ಥನಾದ ವ್ಯಕ್ತಿಯೊಬ್ಬ\-ನಲ್ಲಿ ಈ ಮೂರು ಗುಣಗಳನ್ನು ಕಾಣದಿರಲು ಸಾಧ್ಯವಿಲ್ಲ. ಯಾರಲ್ಲಾದರೂ ಈ ಮೂರು ಗುಣಗಳನ್ನು ಕಾಣದಿದ್ದರೆ ಆತನು ಸಾಮಾನ್ಯ ಮಾನವರ ಗುಂಪಿಗೆ ಸೇರುವುದಿಲ್ಲ. ಒಂದೋ ಆತ ಅತಿಮಾನವ, ಇಲ್ಲವೇ ಅವನಲ್ಲಿ ಏನೋ ಐಬು ಅಥವಾ ಕೊರತೆ ಇದೆ ಎಂದು ಅರ್ಥ.

ಈ ಸೆಳೆತ, ಅನಿಸಿಕೆ ಅಥವಾ ಆಸೆಗೆ ಗೊತ್ತಾದ ಯಾವುದೇ ಒಂದು ಸೀಮೆ ಅಥವಾ ಪರಿಧಿ ಇದೆ ಎನ್ನುವಂತಿಲ್ಲ. ಎಂದೆಂದೂ ಬದುಕಿರಬೇಕೆಂದು ಹಂಬಲಿಸುತ್ತೇವೆ. ಆದರೆ ಸಾವು ಅನಿವಾರ್ಯ, ಯಮರಾಯ ಎಂದಾದರೊಂದು ದಿನ ನಮ್ಮಲ್ಲಿಗೆ ಬಂದೇ ಬರುತ್ತಾನೆಂಬುದು ನಮಗೆ ಗೊತ್ತಿದೆ. ಆದರೆ ನಾವು ಇಲ್ಲವಾಗಿ ಬಿಡುತ್ತೇವೆ ಎನ್ನುವ ಕಲ್ಪನೆ ನಮಗೆ ಬರುವುದಿಲ್ಲ. ಹಾಗೆಂದು ಯೋಚಿಸುವುದು ನಮ್ಮ ಅಂತರಂಗದ ಸ್ವಭಾವಕ್ಕೆ ಸಾಧ್ಯವೇ ಇಲ್ಲವೆಂದು ತೋರುವುದು.

ನಮ್ಮಲ್ಲಿ ಅರಿವಿನ ಹಂಬಲ ಎಷ್ಟೊಂದು ತೀವ್ರವಾಗಿ ಇದೆ! ವಸ್ತುಗಳ ಹೊರರೂಪ, ಅದರ ಹಿನ್ನೆಲೆಯಲ್ಲಿ ಅಡಗಿರುವ ತತ್ತ್ವ, ಅದರಿಂದ ಪಡೆಯಬಹುದಾದ ಉಪಯೋಗ, ಅದರ ಸಾರ ಸರ್ವಸ್ವವನ್ನೇ ತಿಳಿಯಬೇಕೆನ್ನುವ ಆಸೆ ನಮ್ಮದು. ದೂರದ ನಕ್ಷತ್ರ ನೀಹಾರಿಕೆಗಳಲ್ಲಿ ಏನಿದೆ, ಯಾವ ಮೂಲದ್ರವ್ಯಗಳಿಂದ ಅದು ರಚಿತವಾಗಿದೆ ಎಂಬುದನ್ನೂ ಸಾಗರದ ಆಳದಲ್ಲಿ ಏನಡಗಿದೆ, ಜಲಚರಗಳ ಜೀವನ ವಿಧಾನ ವೈಶಿಷ್ಟ್ಯಗಳೇನು ಎಂಬುದನ್ನೂ ಅರಿತುಕೊಳ್ಳುವ ತವಕ ನಮಗೆ. ಮನುಷ್ಯನ ಮನಸ್ಸೇನು? ಅದರ ಆಳದ ಪದರಗಳಲ್ಲಿ ಏನಿದೆ? ಅದು ಯಾವ ನಿಯಮಗಳನ್ನು ಅನುಸರಿಸಿ ಹೇಗೆ ವರ್ತಿಸುತ್ತದೆ ಎಂಬುದನ್ನು ತಿಳಿಯುವ ಕುತೂಹಲ ನಮ್ಮನ್ನು ಆವರಿಸಿ ಕೊಂಡಿದೆ. ಸಾವು ಏನು? ಸಾವಿನ ಆಚೆಗೆ ಜೀವನವಿದೆಯೇ? ಭಗವಂತನು ಇದ್ದಾನೆಯೆ? ಇದ್ದರೆ ಸುಖದುಃಖಗಳಿಗೇನು ಅರ್ಥ?–ಎಂಬುದನ್ನು ತಿಳಿಯುವ ಹಂಬಲ ಯಾರಿಗಿಲ್ಲ? ಒಟ್ಟಿನಲ್ಲಿ ಪ್ರಪಂಚದ ಯಾವ ಜ್ಞಾನವನ್ನೂ ಪಡೆಯದೆ, ಅರಿವನ್ನು ಹೆಚ್ಚಿಸಿಕೊಳ್ಳದೇ ಅಜ್ಞಾನಿಯಾಗಿಯೇ ಬಿದ್ದಿರಲು ಯಾರೂ ಇಚ್ಛಿಸುವುದಿಲ್ಲ.

ಇನ್ನೊಂದು ಮೂಲಭೂತ ಸೆಳೆತ–ಸುಖಕ್ಕಾಗಿ ನಮ್ಮ ಆತುರ ಮತ್ತು ಕಾತರ. ನಮ್ಮಲ್ಲಿ ಬುದ್ಧಿ ಸರಿಯಾಗಿರುವ ಯಾರಾದರೂ ನಾಳೆಯಿಂದ ನನ್ನ ಬಾಳು ದುಃಖಮಯವಾಗಲಿ ಎನ್ನ ಬಲ್ಲನೇ? ನಾವು ಸದಾ ಸಂತುಷ್ಟರಾಗಿರಲು ಆಶಿಸುತ್ತೇವೆ. ಇತರರಿಂದ ಸಂತೋಷ ಪಡೆಯಲು ಇಚ್ಛಿಸುತ್ತೇವೆ. ಸಂತೋಷವನ್ನು ನೀಡಬಹುದಾದ ವಸ್ತುಗಳು, ದೃಶ್ಯಗಳು, ಯೋಚನೆಗಳು, ಭೋಗ್ಯವಸ್ತುಗಳು ನಮಗೆ ಬೇಕು. ಯಾವೆಲ್ಲ ಸಾಧನೆಗಳಿಂದ ಆನಂದ ದೊರಕಬಹುದೋ ಅವೆಲ್ಲವನ್ನೂ ಸಾಧ್ಯವಿದ್ದಷ್ಟು ಬೇಗನೆ ಪಡೆದು ಅತ್ಯಂತ ಹೆಚ್ಚಿನ ಆನಂದವನ್ನು ದೀರ್ಘಕಾಲ ಹೊಂದುವ ಹಂಬಲ ನಮ್ಮದು.

ಸರ್ವತ್ರ ಸರ್ವವ್ಯಾಪಿಯಾಗಿರುವ ಈ ಅದಮ್ಯವಾದ ತ್ರಿವಿಧ ಹಂಬಲದ ಮೂಲವೆಲ್ಲಿ? ಮನುಷ್ಯನ ಅನಂತವಾದ ನೈಜಸ್ವರೂಪದ ಹಿನ್ನೆಲೆಯಿಂದ ಇವು ಉದ್ಭವಿಸುತ್ತವೆ. ನಿಜವಾದ ಮನುಷ್ಯ ಎಂದರೆ ಆತ್ಮ. ಆತ್ಮನ ನೈಜಸ್ಥಿತಿಯನ್ನು ಸಚ್ಚಿದಾನಂದ ಸ್ವರೂಪ ಎನ್ನುತ್ತಾರೆ. ಮನುಷ್ಯನಿಗೆ ತಿಳಿದಿರಲಿ, ತಿಳಿಯದಿರಲಿ ಈ ಸಚ್ಚಿದಾನಂದದ ಸೆಳೆತ ಅವನ ರಕ್ತದ ಕಣಕಣದಲ್ಲಿಯೂ ವ್ಯಾಪಿಸಿಕೊಂಡಿದೆ. ಯಾವುದೇ ಕ್ಷೇತ್ರದಲ್ಲಿ ಯಾವುದೇ ಕೆಲಸ ಮಾಡುತ್ತಿರಲಿ ಈ ಮೂರು ಹಂಬಲಗಳು ಅವನಲ್ಲಿ ಸದಾ ಜಾಗೃತವಾಗಿರುತ್ತವೆ. ತನ್ನ ಅಸ್ತಿತ್ವದ ನಾಶವನ್ನಾಗಲಿ, ಅಜ್ಞಾನವನ್ನಾಗಲಿ, ನಿರಂತರ ದುಃಖವನ್ನಾಗಲಿ ಮನುಷ್ಯನು ಎಂದೂ ಹಂಬಲಿಸಲಾರ. ಅದನ್ನು ಅವನು ಸರ್ವ ಪ್ರಕಾರಗಳಲ್ಲಿಯೂ ವಿರೋಧಿಸುತ್ತಾನೆ. ದೀರ್ಘ ಕಾಲದ ಬದುಕಿನ ಹೋರಾಟಗಳೆಲ್ಲವೂ ಪರಿ\-ಪೂರ್ಣ ಆನಂದ ಮತ್ತು ಪರಿಪೂರ್ಣ ಜ್ಞಾನವನ್ನು ಪಡೆಯುವುದಕ್ಕಾಗಿಯೆ. ಒಟ್ಟಿನಲ್ಲಿ ‘ನಾನು’ವಿನ ಸ್ವರೂಪ ಸ್ವಭಾವಗಳನ್ನು ಹೀಗೆ ಸಂಕ್ಷಿಪ್ತವಾಗಿ ವಿವರಿಸಬಹುದು–

\begin{center}
ನಾನು
\end{center}

\begin{itemize}
\item ಪರಮಸತ್ಯ \footnote{\engfoot{Ultimate reality or Supreme Consciousness.}}ಅಥವಾ ವಿಶ್ವಪ್ರಜ್ಞೆ ನಾನು’ವಿನ ನೈಜಸ್ವರೂಪವಾದ ಸಚ್ಚಿದಾನಂದ

 \item ಅಜ್ಞಾತ ಜೈವಸ್ರೋತ\footnote{\engfoot{Unconscious life current.}}

 \item ಅಹಂಪ್ರಜ್ಞೆ ವ್ಯಾವಹಾರಿಕ ‘ನಾನು’\footnote{\engfoot{Individualised Consciousness.}}

 \item ಮನಸ್ಸುದೇಹೇಂದ್ರಿಯಗಳು \footnote{\engfoot{Mind and body}}

\end{itemize}

೧. ಆಕಾಶದಂತೆ ಸರ್ವವ್ಯಾಪಿಯಾಗಿ ಎಲ್ಲವನ್ನೂ ಬೆಳಗುವ ವಿಶ್ವಪ್ರಜ್ಞೆ ಅಥವಾ ಸತ್ ಚಿತ್ ಆನಂದ; ನಿತ್ಯವಾದ, ಸಾವಿಲ್ಲದ ನೋವಿಲ್ಲದ ಪರಮಾರ್ಥ ತತ್ವ.

೨. ನಾವು ಎಚ್ಚರದ ಸ್ಥಿತಿಯಲ್ಲೇ ಇರಲಿ, ಕನಸಿನಲ್ಲೇ ಇರಲಿ, ಗಾಢನಿದ್ರೆಯಲ್ಲೇ ಇರಲಿ, ದೇಹದ ಸಂಕೀರ್ಣ ಚಟುವಟಿಕೆಗಳು ಹುಟ್ಟಿನಿಂದ ಸಾಯುವ ತನಕ ಕ್ರಮ ಬದ್ಧವಾಗಿ, ಆಘಾತ– ಆತಂಕ–ರೋಗರುಜಿನಗಳ ಸಮಯದಲ್ಲೂ ನಡೆಯುತ್ತಿರುತ್ತವೆ. ಹೃದಯದ ಬಡಿತ, ಶ್ವಾಸೋ\-ಚ್ಛ್ವಾಸ, ರಕ್ತಪರಿಚಲನೆ, ಶುದ್ಧೀಕರಣ ಕ್ರಿಯೆ, ಜೀರ್ಣಾಂಗಗಳ ಕಾರ್ಯ–ಒಂದೇ ಎರಡೇ? ದೇಹದ ಸಮಸ್ತ ಚಟುವಟಿಕೆಗಳೂ ವ್ಯವಸ್ಥಿತ ರೀತಿಯಲ್ಲಿ ನಮ್ಮ ಪ್ರಯತ್ನವಿಲ್ಲದೆ ನಡೆಯು ತ್ತವೆ. ನಮ್ಮ ದೇಹ ಮನಸ್ಸುಗಳು ಸದಾ ಬದಲಾಯಿಸುತ್ತಿರುತ್ತವೆ. ಆದರೆ ನಮ್ಮ ಯಾವ ಪ್ರಯತ್ನವೂ ಇಲ್ಲದೆ ಅವು ತಮ್ಮತನವನ್ನು ಉಳಿಸಿಕೊಂಡಿರುತ್ತವೆ. ಯೋಚಿಸಿ ನೋಡಿದಾಗ ನಮಗೆ ತಿಳಿಯುವುದಿಷ್ಟು; ನಮ್ಮ ಜೀವನವೆಂಬುದು ಪ್ರವಾಹಾಕಾರವಾಗಿ ಹರಿಯುತ್ತಿರುವ ಒಂದು ಮಹಾ ಅಜ್ಞಾತಶಕ್ತಿ. ‘ಅಹಂ’ ಅಥವಾ ‘ನಾನು’ ಎಂದುಕೊಳ್ಳುವಂಥದು ಈ ಶಕ್ತಿಯ ಪ್ರವಾಹದಲ್ಲಿ ಮೇಲಕ್ಕೆ ಪುಟಿದೇಳುವ ಗುಳ್ಳೆಯಂತೆ. ನಮ್ಮ ಎಚ್ಚರದ ಅವಸ್ಥೆಯಲ್ಲಿ ಸ್ಪಷ್ಟವಾಗಿಯೂ ಸ್ವಪ್ನಾವಸ್ಥೆಯಲ್ಲಿ ಅಸ್ಪಷ್ಟವಾಗಿಯೂ ಕಾಣಿಸಿಕೊಳ್ಳುತ್ತದೆ. ಗಾಢನಿದ್ರೆಯಲ್ಲಿ ಸಂಪೂರ್ಣ ಅಜ್ಞಾತ ಶಕ್ತಿಯಲ್ಲಿ ಅಡಗಿದಂತೆ ಭಾಸವಾಗುತ್ತದೆ. (ನಮ್ಮ ಬದುಕಿನ ಮೂರನೇ ಒಂದು ಭಾಗ ನಿದ್ರೆ ಎನ್ನುವ ಅಜ್ಞಾತ ಅವಸ್ಥೆಯಲ್ಲಿ ಕಳೆಯುತ್ತದೆ).

ಆದರೆ ಈ ‘ಅಹಂ’ನ ಮೂಲವಿರುವುದು ಪ್ರವಾಹಾಕಾರವಾದ ಅಜ್ಞಾತ ಶಕ್ತಿಯಲ್ಲಲ್ಲ. ‘ಅಹಂ’ ಪ್ರಜ್ಞೆಯ ಮೂಲವಿರುವುದು ವಿಶ್ವಪ್ರಜ್ಞೆ ಅಥವಾ ನಾಶರಹಿತವಾದ ತ್ರಿಕಾಲದಲ್ಲೂ ಇರುವ ಆತ್ಮನಲ್ಲಿ. ‘ಅಹಂ’ಅನ್ನು ಈ ಅಜ್ಞಾತ ಜೈವ ಶಕ್ತಿಯು ಆಘಾತಗೊಳಿಸುವಂತೆ ಕಂಡರೂ ಅದಕ್ಕೆ ಯಾವ ಬಾಧಕವೂ ಇಲ್ಲ ಎಂಬುದನ್ನು ತಿಳಿದುಕೊಂಡಾಗ ನಾವು ನಿರ್ಭೀತರಾಗುತ್ತೇವೆ.

೩. ಕಿಟಕಿ ಬಾಗಿಲು ಮುಚ್ಚಿದ ಕೋಣೆಯಲ್ಲಿ ಸಣ್ಣ ಬಿರುಕಿನ ಮೂಲಕ ಬೆಳಕು ಒಳಗೆ ಪ್ರವೇಶಿಸಿ ಕೊಂಚ ಜಾಗವನ್ನು ಬೆಳಗುತ್ತದೆ. ಕೋಣೆಯ ಹೊರಗಿನಿಂದ ನೋಡುವವರಿಗೆ ಬೆಳಕು ಪ್ರವೇಶ ಮಾಡುವ ಬಿರುಕು ಕಪ್ಪಾದ ಒಂದು ರಂಧ್ರ ಮಾತ್ರ. ಆದರೆ ಕೋಣೆಯಲ್ಲಿರುವವರಿಗೆ ಸ್ವಲ್ಪ ಭಾಗವನ್ನದು ಬೆಳಗುವುದು ಕಾಣುತ್ತದೆ. ಆ ಬೆಳಕನ್ನು ನೀಡುವ ಬಿರುಕು ತನ್ನಿಂದಲೇ ಬೆಳಕು ಬರುತ್ತಿದೆಯೆಂದು ತಿಳಿದುಕೊಂಡರೆ ಹೇಗೋ ಹಾಗಿದೆ ನಮ್ಮ ‘ನಾನು’, ‘ನಾನು’ ಎಂದುಕೊಳ್ಳುವ ಅಹಂಕಾರ. ಈ ಅಹಂಕಾರದ ಹಿನ್ನೆಲೆಯಲ್ಲಿ ಅಗಾಧ ಅಸೀಮ ಸಚ್ಚಿದಾನಂದ ಅಥವಾ ಪರಿಶುದ್ಧ ಪ್ರಜ್ಞೆ ಇದೆ. ಈ ಪುಟ್ಟ ‘ಅಹಂ’ ತನ್ನ ಹಿನ್ನೆಲೆಯಲ್ಲಿರುವ ಸಚ್ಚಿದಾನಂದ ಸಾಗರದ ಆಧಾರವನ್ನು ಮರೆತರೆ, ಇತರ ‘ಅಹಂ’ಗಳು ಎಂಥ ಸ್ಥಿತಿಯಲ್ಲಿದ್ದರೂ ವಿಕಾಸದ ಪ್ರಾಥಮಿಕ ಹಂತದಲ್ಲಿದ್ದರೂ ಮೂಲತಃ ತನ್ನಂತೆಯೇ ಎಂಬುದನ್ನು ತಿಳಿಯದೆ ಪೂರ್ವಾಗ್ರಹ ಅಥವಾ ದ್ವೇಷದೃಷ್ಟಿಯನ್ನು ಬೆಳೆಸಿಕೊಂಡರೆ ಇದಕ್ಕೆ (‘ಅಹಂ’ಗೆ) ರೋಗ ತಗಲಿದಂತಾಗುತ್ತದೆ. ಓದುಗರಲ್ಲಿ ಕೆಲವರಿಗಾದರೂ ನಿಜವಾದ ಧಾರ್ಮಿಕ ದೃಷ್ಟಿ ಇಲ್ಲಿದೆ ಎನ್ನಿಸದಿರದು.\footnote{ ಡಾ. ಯೂಂಗ್ ಅವರ ಈ ಮಾತುಗಳನ್ನು ಓದಿ: \engfoot{The ego is ill for the very reason that it is cut off from the whole and lost its connection with mankind as well as the spirit}\hfill\engfoot{ –Dr. C. G. Jung}}

೪. ಮನಸ್ಸು ಮತ್ತು ದೇಹ.


\section*{ಅಸ್ತಿತ್ವ ಮೀಮಾಂಸೆ}

\addsectiontoTOC{ಅಸ್ತಿತ್ವ ಮೀಮಾಂಸೆ}

ಮನುಷ್ಯರೆಲ್ಲರೂ ಸುಖಕ್ಕಾಗಿ ನಿರಂತರ ಹೋರಾಡುತ್ತಾರೆ. ಒಂದಲ್ಲ ಒಂದು ತೆರನಾದ ಸಂಕಟದಲ್ಲಿ ಸಿಲುಕಿಕೊಳ್ಳುತ್ತಾರೆ. ಸ್ವಾತಂತ್ರ್ಯಕ್ಕಾಗಿ ಹೋರಾಡುತ್ತಾರೆ. ಒಂದಲ್ಲ ಒಂದು ತೆರನಾದ ಬಂಧನದಲ್ಲಿ ಸಿಕ್ಕಿಕೊಳ್ಳುತ್ತಾರೆ. ಕೀರ್ತಿಗಾಗಿ ಅವರ ಹೋರಾಟ ನಡದೇ ಇದೆ. ಅಪಕೀರ್ತಿ ಅವರನ್ನು ಹೊಂಚು ಹಾಕಿ ನಿಂತಂತಿದೆ. ಒಳಿತನ್ನೇ ಆಚರಿಸಬೇಕೆಂಬ ಹಂಬಲವೇನೋ ಅವರಿಗಿದೆ. ಆದರೆ ಕೆಡುಕಿನ ಹಾದಿಯಲ್ಲೇ ನಡೆಯುವ ಚಟದಿಂದ ತಮ್ಮನ್ನು ತಾವು ಬಿಡಿಸಿಕೊಳ್ಳ ಲಾರರು. ಸಾವಿನಿಂದ ತಪ್ಪಿಸಿಕೊಳ್ಳುವ ಪ್ರಯತ್ನ ಸದಾ ನಡೆದಿದೆ, ಆದರೆ ಸಾವಿನ ಭಯ ಅವರನ್ನು ಸುತ್ತವರಿದಿದೆ. ಹುಟ್ಟಿದ ವ್ಯಕ್ತಿ ಸಾಯುವವರೆಗೂ ನಗು ಅಳು, ಆಸೆ ನಿರಾಸೆ, ರಾಗ ದ್ವೇಷ, ಗೆಲವು ಸೋಲು–ಈ ದ್ವಂದ್ವಗಳ ಹೆದ್ದೆರೆಗಳಲ್ಲಿ ತೇಲುತ್ತಿರುತ್ತಾನೆ. ಜೇಡರ ಬಲೆಯಲ್ಲಿ ಸಿಕ್ಕ ನೊಣ ದಂತೆ ಕೆಲವೊಮ್ಮೆ ಅಸಹಾಯಕನಾಗಿಬಿಡುತ್ತಾನೆ. ಹಳ್ಳಿಯ ಹೈದನೂ, ದಿಲ್ಲಿಯ ಮಾನ್ಯನೂ, ಅರಮನೆ ಗುರು\-ಮನೆಗಳಲ್ಲಿರುವವನೂ, ಗುಹೆ ಗುಡಿಸಲಲ್ಲಿರುವವನೂ, ಪಶ್ಚಿಮ ದೇಶದ ಬಿಳಿಯನೂ, ಆಫ್ರಿಕಾದ ಕರಿಯನೂ ಈ ದ್ವಂದ್ವಗಳನ್ನು ದಾಟಲಾರರು. ಹಾಗಾದರೆ ಮನುಷ್ಯನ ಸ್ಥಿತಿ ಇಷ್ಟೆಯೇ? ಇದೇ ಅವನ ಗತಿಯೇ? ಹಾಗೆಂದಾದರೆ ಅವನೇಕೆ ಪರಿಸ್ಥಿತಿಯೊಂದಿಗೆ ಹೊಂದಾಣಿಕೆ ಮಾಡಿಕೊಂಡಿಲ್ಲ? ಪರಿಸ್ಥಿತಿಯ ಪರಿಮಿತಿಯಿಂದ ಬಿಡಿಸಿಕೊಳ್ಳಲು ತೀವ್ರ ಹಂಬಲದ ಹೋರಾಟವೇಕೆ? ಪಾರತಂತ್ರ್ಯದ ಸಂಕೋಲೆಯಿಂದ ತಪ್ಪಿಸಿಕೊಳ್ಳುವ ಪ್ರಯತ್ನವೇಕೆ? ಅಪೂರ್ಣ ಮಾನವ ಪೂರ್ಣತೆಯ ಚಿತ್ರವನ್ನೇಕೆ ಚಿತ್ರಿಸಿಕೊಳ್ಳಬೇಕು? ಸತ್ಯ ಸಂಗತಿ ಇದು: ಮನುಷ್ಯ ನಿಜವಾಗಿಯೂ ಮರ್ತ್ಯನಲ್ಲ, ಬದ್ಧನೂ ಅಲ್ಲ, ಅಪೂರ್ಣನೂ ಅಲ್ಲ. ಆದರೆ ಮರ್ತ್ಯನೆಂದೂ, ಬದ್ಧನೆಂದೂ, ಅಪೂರ್ಣನೆಂದೂ ದೀರ್ಘಕಾಲದ ಅಜ್ಞಾನದಿಂದ ತಿಳಿದುಕೊಂಡಿದ್ದರೂ ಅದನ್ನು ಅವನ ಆಂತ ರಿಕ ಸತ್ತ್ವ ನಿರಂತರ ವಿರೋಧಿಸುತ್ತಿರುತ್ತದೆ. ಅಜ್ಞಾನ ಕಳೆಯದೆ ಈ ಆಂತರಿಕ ಸತ್ತ್ವದ ಪರಿಚಯವಾಗದು, ಅವನ ಹೋರಾಟಕ್ಕೂ ಅಂತ್ಯವಿರದು.

ಸೂರ್ಯಕಿರಣಗಳ ಕಾವಿನಿಂದ ಸಮುದ್ರದ ನೀರು ಜಲಕಣಗಳಾಗಿ ಮೋಡವಾಗಿ ಮೇಲೇರಿ ಹಿಮಾಲಯ ಪರ್ವತ ಪ್ರದೇಶವನ್ನು ಸಮೀಪಿಸುತ್ತದೆ. ಈ ಜಲಕಣಗಳ ನಿಜವಾದ ಸ್ವರೂಪ ಸಮುದ್ರವೇ. ತಮ್ಮ ನಿಜದ ನೆಲೆಯಾದ ಸಮುದ್ರಕ್ಕೆ ಹಿಂದಿರುಗಲು ಈ ಜಲಕಣಗಳ ಸತತ ಪ್ರಯತ್ನ ನಡೆದೇ ಇದೆ. ಆದುದರಿಂದಲೇ ಅವು ಅಷ್ಟು ಚಂಚಲ. ತಮ್ಮ ನೈಜ ಸ್ವರೂಪವಾದ ಅಥವಾ ನಿಜದ ನೆಲೆಯಾದ ಸಮುದ್ರವನ್ನು ಸೇರುವವರೆಗೂ ಅವುಗಳ ಚಾಂಚಲ್ಯವಾಗಲಿ, ಹೋರಾಟವಾಗಲಿ ನಿಲ್ಲದು. ತಮ್ಮ ನೈಜ ಸಚ್ಚಿದಾನಂದ ಸ್ವರೂಪದಿಂದ ಚ್ಯುತರಾದ ಜೀವಿಗಳಿಗೆ, ನೈಜ ಆನಂದ ಪಡೆಯುವ, ನಿಜದ ನೆಲೆಯಲ್ಲಿ ನಿಲ್ಲುವ ಸತತವಾದ ಇಚ್ಛೆ ಇದೆ ಮತ್ತು ಯತ್ನವೂ ನಡೆದೇ ಇದೆ. ಆದುದರಿಂದಲೇ ಜೀವನು ಅಷ್ಟೊಂದು ಚಂಚಲನಾಗಿ ಪತ್ನಿಯಲ್ಲಿ ಪುತ್ರರಲ್ಲಿ ವಿತ್ತದಲ್ಲಿ ಹಾಗೂ ಇನ್ನಿತರ ವಿಷಯವಸ್ತುಗಳಲ್ಲಿ ಆನಂದವನ್ನು ಹುಡುಕಾಡುತ್ತಿರುತ್ತಾನೆ. ಆನಂದ ಲಾಭದ ಆಸೆಯಿಂದಲೇ ವ್ಯಕ್ತಿಯು ನಾನಾ ರೀತಿಯ ಒಳಿತು ಕೆಡಕುಗಳನ್ನುಂಟು ಮಾಡುವ ಕಾರ್ಯಗಳಲ್ಲಿ ಮುಳುಗಿರುತ್ತಾನೆ.ಆದರೆ ಆತನ ಪ್ರಯತ್ನಗಳೆಲ್ಲ ಪೂರ್ಣದೃಷ್ಟಿಯಿಂದ ಪ್ರೇರಿತವಲ್ಲ; ಆಂಶಿಕದೃಷ್ಟಿಯಿಂದ ಪ್ರೇರಿತನಾಗಿ ಅಥವಾ ಅಜ್ಞಾನವಶನಾಗಿದ್ದುಕೊಂಡು ಮಾಡುವುದರಿಂದ ಅವನನ್ನು ಗೊಂದಲ ದುಃಖಗಳಿಗೆ ಈಡುಮಾಡುತ್ತವೆ. ಸರಿಯಾದ ಅರಿವು ದೊರೆತಾಗ ವ್ಯಕ್ತಿ ನಿಜದ ಆನಂದದೊಂದಿಗೆ ಬಂಧನವನ್ನೂ ದಾಟಿ ಪೂರ್ಣತೆಯನ್ನು ಪಡೆಯುತ್ತಾನೆ. 

ನಮ್ಮ ಸಾಮಾಜಿಕ, ರಾಜಕೀಯ, ಆರ್ಥಿಕ ಸುಧಾರಣೆಯ ಕ್ರಮಗಳೆಲ್ಲ ಹೊಟ್ಟೆ ಬಟ್ಟೆ ವಸತಿಯೇ ಮೊದಲಾದ ದೈಹಿಕ ಜೈವಿಕ ಅಸ್ತಿತ್ವಕ್ಕಿರುವ ತಡೆತೊಂದರೆಗಳನ್ನು ದೂರಮಾಡುವುದಕ್ಕಾಗಿ ನಡೆದಿವೆ. ಆದರೆ ಮನುಷ್ಯರು ‘ಅಹಂ’ನ ಅಸ್ತಿತ್ವಕ್ಕಾಗಿ ನಡೆಯಿಸುವ ಹೋರಾಟದ ಭಾರವನ್ನು ಕಡಿಮೆ ಮಾಡುವ ಕಲೆಯನ್ನು ಇನ್ನೂ ಕಲಿತಿಲ್ಲ. ತಾನು ಹುಟ್ಟಿ ಬೆಳೆದ ತನ್ನ ಕುಟುಂಬದ ಪರಿಸರದಲ್ಲಿ, ತನ್ನ ಕಾರ್ಯಕ್ಷೇತ್ರದಲ್ಲಿ, ಸ್ನೇಹಿತ ಸಹೋದ್ಯೋಗಿಗಳ ಮಧ್ಯದಲ್ಲಿ ಇತರ ಸಾಮಾಜಿಕ ವ್ಯವಹಾರಗಳಲ್ಲಿ ‘ಅಹಂ’ ತನ್ನ ವೈಶಿಷ್ಟ್ಯ, ಸ್ಥಾನಮಾನಗಳನ್ನು ಉಳಿಸಿಕೊಳ್ಳಲು ಹೆಣಗುತ್ತದೆ. ಇದರಿಂದ ಸಮಸ್ಯೆಗಳ ಜಟಿಲ ಜಾಲ ಹೇಗೆ ಉದಿಸುತ್ತದೆ ಎಂಬುದನ್ನು ಕಲ್ಪಿಸಿಕೊಳ್ಳಬಹುದು. ಅಸ್ತಿತ್ವವಾದಿ ಬರಹಗಾರರು ತಮ್ಮ ಕತೆಕಾದಂಬರಿಗಳಲ್ಲಿ ಇದನ್ನು ಅರ್ಥವತ್ತಾಗಿ ಚಿತ್ರಿಸುವಲ್ಲಿ ನೈಪುಣ್ಯ ತೋರಿಸುತ್ತಾರೆ. ಬರ್ಟ್ರಾಂಡ್ ರಸೆಲ್ಲರು ಹೇಳುವಂತೆ ‘ಜೀವನಕ್ಕಾಗಿ ಹೋರಾಟ’ ಎನ್ನುವ ಜನರ ಮಾತಿನಲ್ಲಿ ತಮ್ಮ ತಮ್ಮ ವೈಯಕ್ತಿಕ ಆಶೋತ್ತರಗಳ ನೆರವೇರಿಕೆ ಅಥವಾ ಯಶಸ್ಸಿಗಾಗಿ ಹೋರಾಟ ಎಂಬುದೇ ಸೂಚಿತವಾಗುತ್ತದೆ. ನಾಳಿನ ತಿಂಡಿ ಅಥವಾ ಊಟ ದೊರೆಯುವುದೇ ಇಲ್ಲವೇ ಎಂಬುದು ಅವರ ಸಮಸ್ಯೆಯಲ್ಲ. ತಮ್ಮ ನೆರೆಹೊರೆಯ ಜನರನ್ನು ಮೀರಿಸುವಲ್ಲಿ ತಾವು ಸೋಲಬಹುದೆಂಬ ಭಯ ಅವರನ್ನು ಬಾಧಿಸುತ್ತದೆ.

ಈ ‘ಅಹಂ’ ಅಸ್ತಿತ್ವ ಮತ್ತು ಮಟ್ಟವನ್ನು ಕಾಯ್ದುಕೊಳ್ಳುವ ಹಂಬಲ ಹೋರಾಟಗಳು\break ಪಶ್ಚಿಮದ ಚಿಂತನಶೀಲವ್ಯಕ್ತಿಗಳ ಅಂಬೋಣಕ್ಕನುಗುಣವಾಗಿ ನಮ್ಮ ಮೂಲಭೂತ ಅಸ್ತಿತ್ವದ\break ಅನಿವಾರ್ಯ ಸಮಸ್ಯೆಗಳಲ್ಲೊಂದು. ಅಸ್ತಿತ್ವವಾದದ ಜನಕನೆನಿಸಿದ ಕಿರ್ಕೆಗಾರ್ಡ್​ನ ಅಭಿಪ್ರಾಯದಲ್ಲಿ ಮನುಷ್ಯನು ಪಡೆಯಬಹುದಾದ ಅನುಭವಗಳೆಲ್ಲ ಈ ‘ಅಹಂ’ನ ಸಂಕುಚಿತ ಬಾಗಿಲಿನ ಮೂಲಕವೇ ಬರಬೇಕು. ಮನುಷ್ಯ ಜೀವನದ ಪ್ರಮುಖ ಲಕ್ಷಣಗಳನ್ನು ಆತನು ಸಂಕ್ಷಿಪ್ತವಾಗಿ, ಸಾರರೂಪದಲ್ಲಿ, ನಾಲ್ಕೇ ಶಬ್ದಗಳಲ್ಲಿ ಸೂಚಿಸುತ್ತಾನೆ. ವ್ಯಕ್ತಿತ್ವ, ವಿರೋಧ, ಆಯ್ಕೆ ಮತ್ತು ಭಯ–ಇವುಗಳೇ ಆ ಪ್ರಮುಖ ಲಕ್ಷಣಗಳು. ಈ ಮಾತುಗಳಿಂದ ಅಹಮಿಕೆಯ ಅಸಂಖ್ಯ ಅಲೆತಗಳನ್ನೂ, ಘರ್ಷಣೆಗಳನ್ನೂ, ಪರಿಮಿತಿ ಪರಿಣಾಮಗಳನ್ನೂ ಆತ ತಿಳಿಸುತ್ತಾನೆ. ಮನುಷ್ಯನು ಹೇಗೆ ತನ್ನ ಹೆಗಲ ಮೇಲೇರಿ ತಾನು ಕುಳಿತುಕೊಳ್ಳಲಾರನೋ ಅದೇ ರೀತಿ ಈ ನಾಲ್ಕು ಪರಿಮಿತಿಗಳನ್ನು ದಾಟಲಾರನು. ‘ನಾನು’ ‘ನಾನು’ ಎಂದುಕೊಳ್ಳುವ, ದೇಶ ಕಾಲ ಪರಂಪರೆ ಮತ್ತು ಪರಿಸರದ ಒತ್ತಡಗಳಿಗೆ ಬದ್ಧನಾದ, ನಾನಾ ಆಸೆ ಆಸಕ್ತಿ ಕ್ರಿಯೆ ಪ್ರತಿಕ್ರಿಯೆ ಪ್ರಭಾವ ಪ್ರತಿಭಟನೆಗಳ ಸೆಳೆತಕ್ಕೆ ಸಿಕ್ಕಿಕೊಂಡ ವ್ಯಕ್ತಿಯ ಸಾಂತತೆ ಅಥವಾ ಪರಿಮಿತಿಯನ್ನು ‘ವ್ಯಕ್ತಿತ್ವ’ ಎನ್ನುವ ಮಾತು ತಿಳಿಸುತ್ತದೆ. ವ್ಯಕ್ತಿಯ ಬದುಕಿನ ಗುಣಮಟ್ಟ ಅವನು ಆರಿಸಿಕೊಳ್ಳುವ ಉದ್ಯಮ ಉದ್ಯೋಗ ಮತ್ತು ದೃಷ್ಟಿಕೋನವನ್ನು ಹೊಂದಿಕೊಂಡಿದೆ. ಇಲ್ಲೂ ಅವನಿಗೆ ಸಂಪೂರ್ಣ ಸ್ವಾತಂತ್ರ್ಯವಿದೆಯೇ? ಅವನಿಗೆ ಅಸಾಧ್ಯವಾವುದೂ ಇಲ್ಲವೆಂದು ತೋರುವುದು. ಅದೇ ವೇಳೆ ಅವನು ನಾಲ್ಕು ಹೆಜ್ಜೆ ಮುಂದುವರಿಯುವಾಗ ಸುತ್ತಲೂ ಅಭೇದ್ಯವಾದ ಕೋಟೆ ಕಾಣಿಸುವುದು.\footnote{\engfoot{Man is capable of a little and a lot, of everything and a nothing.}

\engfoot{\general{\hfill}–Pascal.}} ಅವನ ಅಸಹಾಯಕತೆ ಅವನಿಗೆ ಸ್ಪಷ್ಟವಾಗುವುದು. ಅವನ ಜ್ಞಾನ ಆನಂದ ಅಸ್ತಿತ್ವಗಳ ಪರಿಮಿತಿ ಹೆಜ್ಜೆಹೆಜ್ಜೆಗೂ \hbox{ಅವನಿಗೆ} ತಿಳಿಯುತ್ತಿರುತ್ತದೆ. ಬುಡ ಬೇರುಗಳಿಲ್ಲದ ವೃಕ್ಷ, ಅಸ್ತಿಭಾರವಿಲ್ಲದ ಸೌಧ, ಚುಕ್ಕಾಣಿ ಇಲ್ಲದ ದೋಣಿ–ಇವುಗಳಂತೆ ಅವನ ಬದುಕು ಎಂದೂ ತೋರುವುದು. ಸಾವು ಕ್ಷಣಾರ್ಧದಲ್ಲಿ ಎಲ್ಲವನ್ನೂ ಕಬಳಿಸುವುದು. ಈ ಪರಸ್ಪರ ವಿರೋಧಗಳ ಕಂತೆಯ ಹಿಂದೆ ಏಕತೆ ಅಥವಾ ಒಂದು ಸಮನ್ವಯದ ಸೂತ್ರವು ಅವನ ಬುದ್ಧಿಗೆ ಕೆಲವೊಮ್ಮೆಯಾದರೂ ಅಗೋಚರವೇನೂ ಅಲ್ಲ.

ಪ್ರಸಿದ್ಧ ವಿಜ್ಞಾನಿ ಆರ್ಕಿಮಿಡೀಸ್​ನು ನಿಲ್ಲಲು ಭದ್ರನೆಲೆ, ಒಂದು ಆನಿಕೆ ಮತ್ತು ಸನ್ನೆಗೋಲು ಇಷ್ಟಿದ್ದರೆ ಭೂಮಿಯನ್ನು ಎತ್ತಿ ಹಾಕಬಲ್ಲೆ ಎಂದ. ಚಲಿಸುತ್ತಿರುವ ಭೂಮಿಯಲ್ಲೇ ನಿಂತು ಆ ಕಾರ್ಯ ಮಾಡಲು ಸಾಧ್ಯವಿಲ್ಲ. ಅಂತೆಯೇ ‘ಅಹಂ’ನ ಪ್ರವಾಹದಲ್ಲೇ ತೇಲುತ್ತ ತತ್ಸಂಬಂಧವಾದ ಸಮಸ್ಯೆಗಳನ್ನು ಬಿಡಿಸಲು ಸಾಧ್ಯವಿಲ್ಲ. ‘ಅಹಂ’ನ್ನು ಅತಿಕ್ರಮಿಸಿ ಅದರ ಪರಿಮಿತಿಗಳನ್ನು ದಾಟಿ, ಅದರ ಆದಿ ಅಂತ್ಯಗಳನ್ನು ಪರಿಶೀಲಿಸಲು ಪ್ರಜ್ಞೆಯ ಏಣಿ ಮೆಟ್ಟಲಿನ ತುದಿಯನ್ನು ಏರಬೇಕು. ಅಲ್ಲಿ ನಿಂತಾಗ ಪೂರ್ಣದೃಷ್ಟಿಯಿಂದ ಸಮಸ್ಯೆಗಳನ್ನು ಸರಿಯಾಗಿ ಅರ್ಥಮಾಡಿ ಕೊಳ್ಳುವ ಶಕ್ತಿಯು ದೊರೆಯುವುದು. ಸಮಸ್ಯೆಯು ಪೂರ್ಣವಾಗಿ ಪರಿಹಾರವಾಗದಿದ್ದರೂ, ಅದರ ತೀವ್ರತೆ, ವಿಷಮತೆ, ಹಾನಿಕಾರಕ ಪ್ರಭಾವ ಬಹುಮಟ್ಟಿಗೆ ಕಡಿಮೆಯಾಗುವುದು.

ಭಾರತದಲ್ಲಿ ಸಾವಿರಾರು ವರ್ಷಗಳ ಹಿಂದೆಯೇ ಅಸ್ತಿತ್ವದ ಸಮಸ್ಯೆಗಳಿಗೆ ಪ್ರಜ್ಞೆಯ ಉನ್ನತ ಭೂಮಿಕೆಯಲ್ಲಿ ಪುಷಿಗಳು ಪರಿಹಾರವನ್ನು ಕಂಡುಕೊಂಡಿದ್ದರು. ‘ಅಹಂ’ನ ಮೂಲಸ್ವರೂಪ ಸ್ವಭಾವಗಳನ್ನು ಸ್ಪಷ್ಟವಾಗಿ ಅಧ್ಯಯನ ಮಾಡಿ ‘ಅಹಂ’ನ್ನು ದಾಟಿ ಪರಿಶುದ್ಧ ಪ್ರಜ್ಞೆಯ ಸ್ತರದಲ್ಲಿ ನಿಂತ ಮಹನೀಯರು ಈ ದೇಶದಲ್ಲಿ ಆಗಿಹೋಗಿರುವರು. ಐನ್​ಸ್ಟೀನ್ ಹೇಳುವ ‘ವೈಯಕ್ತಿಕ ಅಹಂಕಾರದಿಂದ ಮುಕ್ತರಾದ’ ಅವರು ಸಮಾಜದ ಮಾರ್ಗದರ್ಶಕರಾಗಿದ್ದರು. ಧರ್ಮಾನು\-ಯಾಯಿಗಳಲ್ಲಿ ಕೆಲವರು ತಮ್ಮ ದುರ್ಬಲತೆಯಿಂದ ಧರ್ಮಕ್ಕೆ ಕೆಟ್ಟಹೆಸರನ್ನು ತಂದಿರುವುದು ನಿಜ. ಆದರೆ ಮೂಲ ಪ್ರವರ್ತಕರು ಸತ್ಯವನ್ನು ಸಾಕ್ಷಾತ್ಕರಿಸಿಕೊಂಡವರಾದುದರಿಂದ, ಅವರ ಅನುಯಾಯಿಗಳಲ್ಲಿ ಹಲವರು ಸತ್ಯನಿಷ್ಠರಾಗಿ ದುಡಿದುದರಿಂದ ಧರ್ಮವೃಕ್ಷ ಉನ್ನತ ಚಾರಿತ್ರ್ಯದ ಅತ್ಯುತ್ತಮ ಫಲಗಳನ್ನು ನೀಡಿರುವುದೂ ದಿಟ.

ಪ್ರಜ್ಞೆಯ ಉನ್ನತಸ್ತರಗಳನ್ನು ತಲುಪುವ ಯತ್ನವೇ ನಿಜವಾದ ಆಧ್ಯಾತ್ಮಿಕ ಸಾಧನೆ. ಮಂತ್ರ ಜಪ, ಉಪಾಸನೆ, ಧ್ಯಾನ, ನಿಷ್ಕಾಮಕರ್ಮ ಇವುಗಳೆಲ್ಲ ಸೋಪಾನಗಳು. ‘ಅನುಭೂತಿಯೇ ಧರ್ಮ’, ‘ಇಂದ್ರಿಯಗಳ ಮೇರೆಯನ್ನು ಮೀರಿಹೋಗುವ ಯತ್ನವೇ ಧರ್ಮ’ ಎನ್ನುವ ಸ್ವಾಮಿ ವಿವೇಕಾನಂದರ ಧರ್ಮ ಸಂಬಂಧವಾದ ನಿರೂಪಣೆಗಳು ಸಾಧನೆ ಮತ್ತು ಗುರಿಯನ್ನು ಕುರಿತ ಸ್ಪಷ್ಟ ಅರಿವನ್ನು ಮೂಡಿಸುತ್ತವೆ.

ಇಂದ್ರಿಯಗಳ ಮೇರೆಯನ್ನು ಮೀರಿಹೋಗಿ ಅನುಭೂತಿಯಿಂದ ಅಹುದು ಎನ್ನಿಸಿಕೊಳ್ಳುವ ಸತ್ಯದ ಯಥಾರ್ಥ ಸ್ವರೂಪವನ್ನು ವಿಚಾರದ ಒರೆಗಲ್ಲಿನಲ್ಲಿ ವಿಶ್ಲೇಷಿಸಿ ನೋಡೋಣ.


\section*{ಜೀವ ಶಿವ ಸೂತ್ರ}

\addsectiontoTOC{ಜೀವ ಶಿವ ಸೂತ್ರ}

ತನ್ನನ್ನು ತಾನು ಕುರಿಯೆಂದೇ ದೃಢವಾಗಿ ನಂಬಿ ಕುರಿಯಂತೆ ವರ್ತಿಸಿದ ಕತೆಯಲ್ಲಿ ಬರುವ ಹುಲಿಯ ಉದಾಹರಣೆಯಂತೆ ಅಥವಾ ತನ್ನನ್ನು ತಾನು ಅಗಸನ ಮಗನೆಂದು ತಿಳಿದುಕೊಂಡ ರಾಜಪುತ್ರನಂತೆ ಮನುಷ್ಯರೂ ತಮ್ಮ ನೈಜಸ್ವರೂಪವಾದ ಸತ್​ಚಿತ್ ಆನಂದವನ್ನಾಗಲೀ, ತಮ್ಮಲ್ಲಿ ಅಡಗಿರುವ ಅಪಾರ ಶಕ್ತಿಯನ್ನಾಗಲೀ ಅರಿಯದೆ, ತಮ್ಮ ಬಗೆಗಿನ ಸೀಮಿತ ಕಲ್ಪನೆಗೇ ಕಟ್ಟುಬಿದ್ದು ಸತ್ಯವೆಂದೇ ನಂಬಿ ಆ ಕಲ್ಪನೆಗೆ ಅನುಗುಣವಾಗಿಯೇ ನಡೆದುಕೊಳ್ಳುತ್ತಾರೆ. ಅಜ್ಞಾನದ ಸ್ಥಿತಿಯಲ್ಲಿ ತನ್ನನ್ನು ಕುರಿ ಎಂದು ತಿಳಿದುಕೊಂಡಿದ್ದಾಗಲೂ ಯಥಾರ್ಥವಾಗಿ ಅದು ಹುಲಿಯೇ ಆಗಿತ್ತು. ಅಜ್ಞಾನದ ಸ್ಥಿತಿಯಲ್ಲಿ ತನ್ನನ್ನು ತಾನು ಅಗಸನ ಮಗನೆಂದು ತಿಳಿದುಕೊಂಡಿದ್ದಾಗಲೂ ಅವನು ನಿಜವಾಗಿಯೂ ರಾಜಪುತ್ರನೇ ಆಗಿದ್ದನಲ್ಲವೆ? ಅಂತೆಯೇ ಮಾನವನಾಗಿ ತೋರುವ, ಮಾನವನೆಂದೇ ತಿಳಿದು ತನ್ನ ದುರ್ಬಲ ಪರಿಮಿತ ಭಾವನೆಗಳಿಗೇ ಅಂಟಿಕೊಂಡು ತದನುಗುಣವಾಗಿ ವರ್ತಿಸುವ ವ್ಯಕ್ತಿಯೂ ನಿಜವಾಗಿಯೂ ಅನಂತಾತ್ಮನೇ. ಕತೆಗಳಲ್ಲಿ ಬರುವ ಘಟನೆಯ ಆಧಾರದಿಂದ ನಾವು ಈ ಒಂದು ಸಮೀಕರಣವನ್ನು ರಚಿಸಬಹುದು.

ಕುರಿ = ಹುಲಿ

ಅಗಸನ ಮಗ = ರಾಜಪುತ್ರ

ರೂಢಮೂಲವಾದ ಅಜ್ಞಾನವೊಂದೇ ಇಲ್ಲಿ ಹುಲಿಯನ್ನು ಕುರಿಯಾಗಿಸಿದ್ದು, ರಾಜಪುತ್ರನನ್ನು ಅಗಸನ ಮಗನನ್ನಾಗಿಸಿದ್ದು. ವ್ಯಾವಹಾರಿಕ ದೃಷ್ಟಿಯಿಂದ, ಕುರಿ ಮಂದೆಯಲ್ಲಿದ್ದುಕೊಂಡು ಕುರಿಗಳಂತೆ ಹುಲ್ಲು ತಿಂದ, ಕುರಿಗಳಂತೆಯೇ ಅರಚುತ್ತ ತನ್ನನ್ನು ತಾನು ಕುರಿ ಎಂದೇ ತಿಳಿದುಕೊಂಡ ಆ ಹುಲಿಯನ್ನು ಕುರಿ ಎಂದೇ ಕರೆಯೋಣ. ಆದರೆ ಸ್ವರೂಪತಃ ಎಂದರೆ ನಿಜವಾಗಿಯೂ ಅದು ಹುಲಿಯೇ ಅಲ್ಲವೆ? ಅದು ತನ್ನನ್ನು ತಾನು ಹುಲಿಯೆಂದು ತಿಳಿದುಕೊಂಡಾಗ ಮಾಯವಾದ ಅಥವಾ ಕಾಣೆಯಾದ ಅಂಶ ಯಾವುದು? ಅಜ್ಞಾನ ಮತ್ತು ಆ ಅಜ್ಞಾನದಿಂದ ಉಂಟಾದ ನಡವಳಿಕೆ ತಾನೆ? ವಸ್ತುವಿನಲ್ಲಿ ಅಡಗಿರುವ ಅಪಾರ ಶಕ್ತಿ ಸೂತ್ರವನ್ನು ಐನ್​ಸ್ಟೀನ್ ತಿಳಿಸುವವರೆಗೆ, ನಾವು ತುಣುಕು ವಸ್ತುವನ್ನು ಅದರ ತೋರಿಕೆಯ ಗುಣ ಗಾತ್ರಗಳಿಗೇ ಸೀಮಿತಗೊಳಿಸಿ ಅದನ್ನು ಅಲ್ಪ, ನಗಣ್ಯ ಎಂದೇ ಭಾವಿಸಿದ್ದೆವು. ತೋರಿಕೆಯೇ ಪೂರ್ಣಸತ್ಯವಲ್ಲ ಎಂಬುದು ಸ್ಪಷ್ಟವಾಗಿ ಸಾಬೀತಾದ ಬಳಿಕ ನಮ್ಮ ದೃಷ್ಟಿಕೋನ ಬದಲಿಸಿತು. ಆದರೆ ವಸ್ತುವನ್ನು ಶಕ್ತಿಯಾಗಿ ಪರಿವರ್ತಿಸಲು ಕೆಲವೊಂದು ವಿಧಾನಗಳನ್ನು ಅನುಸರಿಸಬೇಕಾಗುತ್ತದೆಂಬುದು ಮರೆಯತಕ್ಕ ಸಂಗತಿಯಲ್ಲ.

ಐನ್​ಸ್ಟೀನ್ ಕಂಡುಹಿಡಿದ ಸೂತ್ರಕ್ಕಿಂತ ಮಾನವನಿಗೆ ಹೆಚ್ಚು ಉಪಕಾರಕವಾದ ಇನ್ನೊಂದು ಸೂತ್ರವಿದೆ. ಇದು ನಿತ್ಯಸತ್ಯವೊಂದನ್ನು ತಿಳಿಸಿಕೊಡುವ ಸೂತ್ರ. ಮಾನವನ ನೈಜ ಸ್ವಭಾವವನ್ನು ತಿಳಿಸಿಕೊಡುವ ಸೂತ್ರ. ಪ್ರತಿಯೊಬ್ಬರೂ ತಮ್ಮ ಬದುಕಿನಲ್ಲಿ ವೈಜ್ಞಾನಿಕ ವಿಚಾರಗಳ ಹಿನ್ನೆಲೆಯಿಂದ ದೃಢಪಡಿಸಿಕೊಂಡು ಪ್ರಯೋಗ ಮತ್ತು ಅನುಭವಗಳ ಮೂಲಕ ತಿಳಿದುಕೊಳ್ಳಬಹು ದಾದ ಸೂತ್ರ. ಬದುಕಿನ ಅರ್ಥ ಉದ್ದೇಶಗಳ ರಹಸ್ಯೋದ್ಘಾಟನೆ ಮಾಡುವ ಸೂತ್ರ. ವ್ಯಕ್ತಿ ಯೊಬ್ಬನಿಗೆ ನಿಂತ ನೆಲದಿಂದ ಮೇಲಕ್ಕೇರಲು ಬೇಕಾದ ಆತ್ಮವಿಶ್ವಾಸ ಧೈರ್ಯ ಸಾಹಸ ಉತ್ಸಾಹಗಳನ್ನು ನೀಡುವ ಅದ್ಭುತ ಸೂತ್ರ. ಶತಶತಮಾನಗಳ ಹಿಂದೆಯೆ ಪ್ರಪಂಚದ ಮಹಾಧರ್ಮ ಪ್ರವರ್ತಕರು ಅನುಭಾವಿಗಳು ಮುಖ್ಯವಾಗಿ ಭಾರತೀಯ ಪುಷಿಗಳು ಇದನ್ನು ಕಂಡುಹಿಡಿದಿದ್ದರೂ ಆಧುನಿಕ ಯುಗದಲ್ಲಿ ಕೇವಲ ಒಂದು ಸಿದ್ಧಾಂತ ಮಾತ್ರವಾಗಿ ಅಲ್ಲದೆ, ಪ್ರಾಯೋಗಿಕ ದೃಷ್ಟಿಯ ಹಿನ್ನೆಲೆಯಿಂದ ಸಕಲ ಮಾನವ ಜನಾಂಗದ ಅಭ್ಯುದಯಕ್ಕೆ ಮಾರ್ಗದರ್ಶನವಾಗಬಲ್ಲ ಈ ಸೂತ್ರವನ್ನು ನಿರ್ಭೀತಿಯಿಂದ ಬೋಧಿಸಿದ ಶ್ರೇಯಸ್ಸು ಸ್ವಾಮಿ ವಿವೇಕಾನಂದರಿಗೆ ಸಲ್ಲುತ್ತದೆ. ಹಿಂದೆ ಕೆಲವೇ ಸಂನ್ಯಾಸಿ ಮತ್ತು ಗೃಹಸ್ಥ ಸಂಪ್ರದಾಯವಾದಿಗಳ ಸ್ವತ್ತಾಗಿದ್ದ ಈ ತತ್ತ್ವವನ್ನು ಸರ್ವಜನಗ್ರಾಹಿಯಾಗುವಂತೆ ತಿಳಿಸಿದರೆ ವ್ಯಕ್ತಿಯ ಸರ್ವತೋಮುಖ ಅಭ್ಯುದಯ, ಸಮಾಜದ ಕಲ್ಯಾಣ ನೇರ ದಾರಿಯಿಂದ ತ್ವರಿತವಾಗಿ ಕೈಗೂಡುತ್ತದೆಂಬುದು ಈ ಮಹಾತ್ಮರ ಅನುಭವಸಿದ್ಧ ದೃಢವಿಶ್ವಾಸವಾಗಿತ್ತು. ಅವರೆಲ್ಲ ಹೇಳಿದ ಸೂತ್ರ ಇಂತಿದೆ:

ಜೀವ = ಶಿವ, ಅಥವಾ

ನರ = ಹರ, ಅಥವಾ

ಮಾನವ = ಮಾಧವ

\enginline{E=\textit{mc\supskpt{2}}}ಎನ್ನುವ ಸೂತ್ರಕ್ಕಿಂತ ಹಿರಿದಾದ ಹೆಚ್ಚಿನ ಸತ್ಯವನ್ನು ಜೀವ ಶಿವ ಸೂತ್ರ ವ್ಯಕ್ತಪಡಿಸುತ್ತದೆ. ಈ ಜೀವ ಶಿವ ಸೂತ್ರದಲ್ಲಿ \enginline{\textit{c\supskpt{2}}}ಎನ್ನುವ ನಿಯತಾಂಕದ ಅಭಾವ ನಿಮಗೆ ಗೋಚರವಾಗುತ್ತದೆಯಷ್ಟೆ. ನಿಯತಾಂಕವಾದ \enginline{\textit{c\supskpt{2}}}ಅತಿ ಹೆಚ್ಚಿನ ಪರಿಣಾಮದ್ದಾದರೂ ಒಂದು ನಿಶ್ಚಿತ ಸೀಮಿತ ಸಂಖ್ಯೆ. ದ್ರವ್ಯರಾಶಿ ಅಥವಾ ವಸ್ತುವಿನ ಒಂದು ತುಣುಕಿನಲ್ಲಿ ಗೋಚರವಾಗುವ ಗುಣಗಾತ್ರಕ್ಕಿಂತ ಅತಿ ಹೆಚ್ಚಿನ ಶಕ್ತಿ ಎಂಥ ಪ್ರಮಾಣದಲ್ಲಿ ಅಡಗಿದೆ ಎಂಬುದನ್ನು ಆ ನಿಯತಾಂಕ ತೋರಿಸಿ ಕೊಡುತ್ತದೆ. ಆದರೆ ಜೀವ ಶಿವ ಸೂತ್ರದಲ್ಲಿ ಮೊದಲ ನೋಟಕ್ಕೆ ಒಂದೆಡೆಯಲ್ಲಿ ಅನಂತತೆಯನ್ನೂ ಇನ್ನೊಂದೆಡೆ ಸಾಮಾನ್ಯವಾಗಿ ಸೀಮಿತವೆಂದು ತೋರುವ ವಸ್ತುವನ್ನೂ ನೋಡುವಾಗ ಅಚ್ಚರಿಯಾಗದಿರದು. ದೃಷ್ಟಿಗೆ ಗೋಚರಿಸುವ ಅಥವಾ ನಮ್ಮ ಅನುಭವಕ್ಕೆ ಬರುವಷ್ಟಕ್ಕೆ ವಸ್ತುವಿನ ಸ್ವರೂಪವನ್ನು ಸೀಮಿತಗೊಳಿಸುವ ನಮ್ಮ ಸಾಮಾನ್ಯ ದೃಷ್ಟಿಯಿಂದ ಈ ಸಮಸ್ಯೆ ಉದ್ಭವಿಸುತ್ತದೆ. ಮಾನವನೆಂದರೆ ಆರಡಿಯಷ್ಟು ಎತ್ತರದ ರಕ್ತಮಾಂಸಮಜ್ಜೆ ನರವ್ಯೂಹಗಳಿಂದ ಕೂಡಿದ ಸ್ವಲ್ಪ ಮಾನಸಿಕ ಬೌದ್ಧಿಕ ಶಕ್ತಿಯನ್ನು ಪಡೆದುಕೊಂಡ ಒಂದು ಪ್ರಾಣಿ ಎಂದು ನಾವು ಸಾಮಾನ್ಯವಾಗಿ ತಿಳಿದಿದ್ದೇವೆ. ಈ ನಂಬಿಕೆಗೆ ಮೇಲಿನ ಸೂತ್ರ ಕುಠಾರಾಘಾತವಾಗಿ ಪರಿಣಮಿಸುತ್ತದೆ. ಮಾತ್ರವಲ್ಲ, ವ್ಯಕ್ತಿಯಲ್ಲಿರುವ ಸುಪ್ತ ಶಕ್ತಿಯನ್ನು ವಿಕಾಸಗೊಳಿಸಲು ಅನಂತ ಅವಕಾಶವನ್ನು ತೋರಿಸಿಕೊಡುತ್ತದೆ.

‘ಮನುಷ್ಯನಾಗಿ ಗೋಚರಿಸುವ ಆತ್ಮದ ಪರಮ ಮಹಿಮೆ ವೈಭವಗಳನ್ನು ಗ್ರಂಥಗಳಾಗಲಿ ವಿಜ್ಞಾನಶಾಸ್ತ್ರವಾಗಲಿ ಧರ್ಮಗ್ರಂಥಗಳಾಗಲಿ ಊಹಿಸಲೂ ಅಸಮರ್ಥವಾಗಿವೆ. ಭೂಮಿಯ ಮೇಲೆ ಇದ್ದ, ಇರುವ ಮತ್ತು ಇರಬಲ್ಲ ಅತ್ಯಂತ ಮಹಿಮಾಮಯ ದೈವವೇ ಮನುಷ್ಯ.\footnote{\engfoot{No books, no scriputres, no science can ever imagine the glory of the self that appears as man, the most glorious God that ever was, the only God that ever existed. }\hfill\engfoot{ –Swami Vivekananda}} ನಾನೇ ಅವನು. ಅವನೇ ನಾನು. ನಾನು ಶುದ್ಧ ಸಚ್ಚಿದಾನಂದವಾಗಿದ್ದೆನು ಮತ್ತು ಈ ಯಃಕಶ್ಚಿತ್ ನಾನು ಆಗಿರಲಿಲ್ಲ. ಸೂತ್ರದಲ್ಲಿ ಕಾಣಸಿಗದ ಆ ನಿಯತಾಂಕ ಅನಂತವೇ ಆಗಿದ್ದು ಅದನ್ನು ಸಾಕ್ಷಾತ್ಕರಿಸಿದಾಗ ಅಥವಾ ಕಂಡುಹಿಡಿದಾಗ ಸಮೀಕರಣವನ್ನು ಬಿಡಿಸಿದಂತಾಗುತ್ತದೆ. ನಮ್ಮ ಬದುಕು ಕೇವಲ ಕುತೂಹಲ ಪರಿಹಾರಕ್ಕಾಗಿ, ತೃಷ್ಣೆಗಳ ತೃಪ್ತಿಗಾಗಿಯೇ ಇರದೆ ಕರ್ತವ್ಯ ಕರ್ಮ ಮತ್ತು ನೈಜ ಸ್ವರೂಪದ ಸಾಕ್ಷಾತ್ಕಾರಕ್ಕಾಗಿದೆ, ಸ್ವಸ್ಥರಾಗುವುದಕ್ಕಾಗಿದೆ. ಪರಿಪೂರ್ಣತೆಯೆಡೆಗೆ ಪಯಣಿಸುವುದಕ್ಕಾಗಿದೆ.

ಈ ಸೂತ್ರದಂತೆ ನಿಜವಾದ ಮನುಷ್ಯ ಎಂದರೆ ಅನಂತನೂ ವಿಶ್ವವ್ಯಾಪಿಯೂ ಆದ ಆತ್ಮ. ದೇಶಕಾಲಗಳಿಂದ ಬಂಧಿತವಾಗದ ಕಾರ್ಯಕಾರಣದ ಬಲೆಗೆ ಸಿಲುಕದ ಸರ್ವಬಂಧಮುಕ್ತನಾದ ಸಚ್ಚಿದಾನಂದವೇ ಅದು. ಜೀವಾತ್ಮನು ಸರ್ವಶಕ್ತನ ಕಿಡಿಯಾದುದರಿಂದ, ಆತ್ಮರೂಪಿ ಮನುಷ್ಯನೂ ಸರ್ವಶಕ್ತನೇ. ಆತನೂ ಅಪಾರ ಶಕ್ತಿಯ ಅಕ್ಷಯ ನಿಧಿಯೇ. ಈ ಸತ್ಯದ ಅನುಭೂತಿಯನ್ನೇ ಆತ್ಮಸಾಕ್ಷಾತ್ಕಾರ ಎನ್ನುವುದು.

‘ಮಾನವನು ಎರಡು ಅನಂತಗಳ ನಡುವೆ ಇದ್ದಾನೆ. ಒಂದು ಸೂಕ್ಷ್ಮಾತಿಸೂಕ್ಷ್ಮ ಲೋಕಕ್ಕೆ ಸಂಬಂಧಿಸಿದ್ದಾದರೆ, ಇನ್ನೊಂದು ಮಹತ್ತಿಗಿಂತಲೂ ಮಹತ್ತಾದ ಬೃಹತ್ ಲೋಕಕ್ಕೆ ಸಂಬಂಧಿಸಿದ್ದು\footnote{\engfoot{Man stands between two infinitie; the infinitely small and the infinitely big.}\hfill\engfoot{ –Bertrand Rusell}} ಎಂದು ಬರ್ಟ್ರಾಂಡ್ ರಸೆಲ್ ಹೇಳಿದಾಗ ಅವರು ಪಾಸ್ಕಲನ್ಣ\footnote{\engfoot{What is man in the Universe? A nothing compared to the infinite, a something compared to the nothing – middle between nothing and everything }

\hfill\engfoot{ –Pascal}} ಮಾತನ್ನೇ ಪ್ರತಿಧ್ವನಿಸಿದರು. ಅವುಗಳಲ್ಲಿ ಒಂದು ಅಣುಕಣಗಳ ಸೂಕ್ಷ್ಮಾತಿಸೂಕ್ಷ್ಮಸ್ಥಿತಿಯನ್ನು ಸೂಚಿಸಿದರೆ ಇನ್ನೊಂದು ವಿಶ್ವದ ಪಾರಾವಾರವಿಲ್ಲದ ವೈಶಾಲ್ಯವನ್ನು ಸೂಚಿಸುವಂಥದು. ಆದರೆ ರಸೆಲ್ ಒಂದು ಸಂಗತಿಯನ್ನು ಹೇಳಲು ಮರೆತರು. ಪ್ರಾಯಃ ಅವರು ಅದನ್ನು ಹೇಳಲಾರರೆಂದು ತೋರುತ್ತದೆ. ಈ ಎರಡು ಅನಂತಗಳನ್ನೂ ಅಚ್ಚರಿಯಿಂದ ವೀಕ್ಷಿಸುವ ಮನುಷ್ಯನೂ ತನ್ನನ್ನು ತಾನು ಸಾಂತನೆಂದು ತಿಳಿದುಕೊಂಡ ಒಂದು ಅನಂತ. ಪಾಸ್ಕಲನ ಮಾತು ಇದಕ್ಕೆ ಸಮೀಪವಾಗಿದೆ. ಇನ್ನೊಂದು ವಿಚಾರ: ಎರಡು ಅಥವಾ ಮೂರು ಅನಂತಗಳು ಎಂದರೆ ಅಸಂಬದ್ಧವಾಗುತ್ತದೆ. ಒಂದು ಇನ್ನೊಂದನ್ನು ಸಾಂತಗೊಳಿಸುತ್ತದೆಯಲ್ಲವೇ? ಆಕಾಶದಂತೆ ಸರ್ವತ್ರ ಸರ್ವವ್ಯಾಪಿಯಾಗಿರುವ ಅನಂತ ಒಂದೇ. ಇದು ಭಾರತೀಯ ದಾರ್ಶನಿಕ ಶ್ರೇಷ್ಠರು ಕಂಡುಕೊಂಡ ಮಹೋನ್ನತ ಸತ್ಯ. ಈ ಭಾವನೆ ವ್ಯಕ್ತಿಗೆ, ಘನತೆ ಗೌರವಗಳನ್ನು ನೀಡಿ ಆತ್ಮವಿಶ್ವಾಸ ಮತ್ತು ಅಪಾರ ಶಕ್ತಿಯ ಹೊನಲನ್ನೇ ಹರಿಸುವಲ್ಲಿ ಅದ್ಭುತಪಾತ್ರ ನಿರ್ವಹಿಸುತ್ತದೆ.


\section*{ಕಾಪ್ರಾರ ‘ಭೌತಶಾಸ್ತ್ರದ ಪಥ’}

\addsectiontoTOC{ಕಾಪ್ರಾರ ‘ಭೌತಶಾಸ್ತ್ರದ ಪಥ’}

ಡಾ.\ ಫ್ರಿಜೋ ಕಾಪ್ರಾ ಪ್ರಸಿದ್ಧ ಅಣುಭೌತವಿಜ್ಞಾನಿ. ಪಶ್ಚಿಮದ ಬೇರೆ ಬೇರೆ ವಿಶ್ವವಿದ್ಯಾಲಯಗಳಲ್ಲಿ ಸಂಶೋಧನೆಗಳನ್ನು ಕೈಗೊಂಡು ಹೆಸರು ಪಡೆದವರು. ಆಧುನಿಕ ಭೌತವಿಜ್ಞಾನ ಮತ್ತು ಪೂರ್ವದೇಶಗಳ (ಉಪನಿಷತ್ತು, ಬೌದ್ಧಧರ್ಮ ಮತ್ತು ಟಾವೊ ಮತಗಳಲ್ಲಿ ಕಂಡುಬರುವ) ಅನುಭಾವದ ಸಂಪ್ರದಾಯಗಳಲ್ಲಿ ವಿಶ್ವದ ಮೂಲಭೂತ ತತ್ತ್ವವನ್ನು ಕುರಿತು ಕಂಡುಬರುವ ಸಾಮ್ಯಹೋಲಿಕೆಗಳನ್ನು ವಿಸ್ತೃತವಾಗಿ ಅಧ್ಯಯನ ಮಾಡಿ \enginline{‘Tao of Physics’–‘}ಭೌತಶಾಸ್ತ್ರದ ಟಾವೋ’ ಅಥವಾ ‘ಭೌತಶಾಸ್ತ್ರದ ಪಥ’ ಎನ್ನುವ ಒಂದು ಗ್ರಂಥವನ್ನು ಬರೆದರು. ಈ ಗ್ರಂಥ ವಿಜ್ಞಾನದ ವಿವಿಧ ಕ್ಷೇತ್ರಗಳಲ್ಲಿ ದುಡಿಯುವವರ ಮತ್ತು ತಾತ್ತ್ವಿಕರ ಮನಸ್ಸನ್ನು ಸೆಳೆಯಿತು. ಭೌತವಿಜ್ಞಾನದ ಕ್ಷೇತ್ರದಲ್ಲಿ ದೀರ್ಘಕಾಲದಿಂದ ತರಬೇತಿಯನ್ನು ಪಡೆಯುತ್ತ ಬಂದಿದ್ದರೂ ಹಲವಾರು ವರ್ಷಗಳಿಂದ ತತ್ಸಂಬಂಧವಾದ ಸಂಶೋಧನೆಗಳಲ್ಲಿ ಮುಳುಗಿದ್ದರೂ ಅವರಿಗೆ ಪೌರಸ್ತ್ಯ ದೇಶಗಳ ಅನುಭಾವೀ ಸಾಹಿತ್ಯ ಪರಂಪರೆಯಲ್ಲಿ ವಿಶೇಷ ಆಸಕ್ತಿ ಇತ್ತು. ಯಾವ ವಿಷಯ ಅವರ ಪಾಲಿಗೆ ಗ್ರಾಫಿನ ಮೂಲಕ, ಗಣಿತದ ಸಿದ್ಧಾಂತಗಳ ಮೂಲಕ ವಿಭಿನ್ನ ಆಕೃತಿಗಳ ಮೂಲಕ, ಭೌತ ವಿಜ್ಞಾನದಲ್ಲಿ ನಡೆದ ವಿಭಿನ್ನ ಸಂಶೋಧನೆಗಳನ್ನು ಕುರಿತು ಆಳವಾದ ಚಿಂತನೆಯ ಮೂಲಕ ಪರಿಚಿತವಾಗಿತ್ತೊ, ಅದನ್ನು ಕುರಿತಾದ ಅನುಭಾವದ ಮಟ್ಟದ ಅಥವಾ ಅಂತರ್ ದೃಷ್ಟಿಯ ಅನುಭವವೊಂದು ಅವರ ಬದುಕಿನಲ್ಲಿ ಆಮೂಲಾಗ್ರ ಪರಿಣಾಮವನ್ನುಂಟುಮಾಡಿತು. ಕೇವಲ ವಿಶ್ಲೇಷಣಾತ್ಮಕ ವೈಜ್ಞಾನಿಕ ಚಿಂತನೆಯಲ್ಲೇ ಮಗ್ನರಾಗಿದ್ದ ಅವರಿಗೆ, ಕಾಣಿಸಿದ ‘ಶಕ್ತಿ ಖೇಲನ’ದ ಬೃಹತ್​ದರ್ಶನ ಭಾವವಿಹ್ವಲತೆಯನ್ನೇ ಉಂಟುಮಾಡಿ ಕಂಬನಿಯನ್ನು ಹರಿಯಿ ಸಿತು. ಅಂದಿನಿಂದ ಅವರು ವೈಜ್ಞಾನಿಕ ಕ್ಷೇತ್ರದಲ್ಲಿ ದುಡಿಯುತ್ತಿದ್ದರೂ ಪೌರಸ್ತ್ಯ ಧರ್ಮಗಳ ಮೂಲಭೂತ ಭಾವನೆಗಳನ್ನು ಅಧ್ಯಯನ ಮಾಡುತ್ತ ಬಂದರು. ಅವರ ಆಳವಾದ ಅನುಭವ ಅಧ್ಯಯನಗಳ ಫಲವೇ ಈ ವಿಖ್ಯಾತ ಗ್ರಂಥ. ನಮ್ಮ ದೇಶದಲ್ಲಿ ಸತ್ಯಾನ್ವೇಷಿಗಳು ಪರಮಾರ್ಥ ಅಥವಾ ಕೊನೆಯ ಸತ್ಯವನ್ನು ತಿಳಿಯಲು ಮೂರು ವಿಧಾನಗಳನ್ನು ಅನುಸರಿಸುತ್ತಾರೆ. ಮೊದಲನೆಯದೇ ಪವಿತ್ರವೂ ಅಪೌರುಷೇಯವೂ ಎನಿಸಿದ ಗ್ರಂಥಗಳ ಅಧ್ಯಯನ. ಗ್ರಂಥಗಳಲ್ಲಿ ಹೇಳಿದ್ದನ್ನು ಬಾಯಿಪಾಠ ಮಾಡುವುದು ಅಧ್ಯಯನದ ಒಂದು ವಿಧಾನವಾದರೂ ಸರಿಯಾಗಿ ನಿಸ್ಸಂ ಶಯವಾಗಿ ಅರ್ಥ ನಿರ್ಣಯ ಮಾಡುವ ಸಾಮರ್ಥ್ಯವನ್ನು ಪಡೆಯುವುದೇ ಅಧ್ಯಯನದ ಮುಖ್ಯ ಉದ್ದೇಶ. ಬಳಿಕ ತರ್ಕಯುಕ್ತಿಗಳ ಮೂಲಕ ಗ್ರಂಥದಲ್ಲಿ ಹೇಳಿದ ವಿಷಯಗಳನ್ನು ಎಲ್ಲ ತೆರನಾದ ಸಂಶಯ ಮತ್ತು ವಿಪರೀತ ಭಾವನೆಗಳನ್ನು ದೂರಮಾಡಿ ಸ್ಪಷ್ಟವಾಗಿ ತಿಳಿದುಕೊಳ್ಳುವುದು. ಕೊನೆಯಲ್ಲಿ, ನಿಸ್ಸಂಶಯವಾಗಿ ಬೌದ್ಧಿಕಮಟ್ಟದಲ್ಲಿ ತಿಳಿದುಕೊಂಡ ವಿಷಯವನ್ನು ಧ್ಯಾನದ ಮೂಲಕ ತಾವೇ ಅನುಭವಿಸುವುದು. ಹೀಗೆ ಶ್ರುತಿ, ಯುಕ್ತಿ ಮತ್ತು ಅನುಭವಗಳೆಲ್ಲಾ ಒಂದೇ ಗುರಿಯನ್ನು ಸೂಚಿಸಿ ಅದು ಸಂಗತವೆಂದಾದರೆ ಅದನ್ನು ಸತ್ಯ ಎಂದು ಸ್ವೀಕರಿಸುವುದು ರೂಢಿ. ಹೀಗೆ ಡಾ.\ ಕಾಪ್ರಾ ಅವರ ಮಾತುಗಳು ವಿಶೇಷ ರೀತಿಯಿಂದ ಗಮನಾರ್ಹ. ಏಕೆಂದರೆ ಅವರು ಮೇಲಿನ ಮೂರು ವಿಧಾನಗಳೊಂದಿಗೆ ನಾಲ್ಕನೆಯದಾದ ವಿಜ್ಞಾನದ ಒರೆಗಲ್ಲಿಗೆ ಹಚ್ಚಿ ನೋಡುವ ಒಬ್ಬ ಸಂಶೋಧಕ ವಿಜ್ಞಾನಿ. ಅವರದು ಸತ್ಯಾನ್ವೇಷಿಯ ಸಮನ್ವಯದೃಷ್ಟಿ. ಅವರ ಕೆಲವೊಂದು ಪರಿಶೀಲನೆಗಳನ್ನೂ, ಇತರ ವಿಜ್ಞಾನಿಗಳ ದೀರ್ಘಕಾಲದ ಚಿಂತನೆಯ ಫಲವಾದ ಕೆಲವು ಅನಿಸಿಕೆಗಳನ್ನೂ ಮುಂದೆ ಸಂಗ್ರಹಿಸಲಾಗಿದೆ. ಅವೆಲ್ಲವೂ ಮನುಷ್ಯನು ಚೈತನ್ಯದ ಚಿಲುಮೆ ಎಂಬುದನ್ನೇ ಸಾರುತ್ತಿವೆ.


\section*{ವಿಜ್ಞಾನಿ ಹೇಳಿದ ವೇದಾಂತತತ್ತ್ವ}

\addsectiontoTOC{ವಿಜ್ಞಾನಿ ಹೇಳಿದ ವೇದಾಂತ\-ತತ್ತ್ವ}

ವಿಶ್ವದ ಮೂಲಭೂತ ಏಕತೆಯೇ ಪೌರಸ್ತ್ಯ ಅನುಭಾವಿಗಳ ಅನುಭವ ವಾಣಿಯ ಒಂದು ಪ್ರಮುಖ ಲಕ್ಷಣ. ಅದು ಆಧುನಿಕ ಭೌತವಿಜ್ಞಾನದಲ್ಲಿ ನಮಗೆ ಗೋಚರಿಸುವ ಒಂದು ಪ್ರಮುಖ ಸತ್ಯವೂ ಹೌದು. ಈ ಏಕತೆ ಅಣುವಿನ ಹಂತದಲ್ಲೇ ವ್ಯಕ್ತವಾಗುತ್ತದೆ. ಇಲ್ಲಿನ ಮಾತುಗಳೇ ಅದಕ್ಕೆ ಸಾಕ್ಷಿ–

‘ಕ್ವಾಂಟಂ ಸಿದ್ಧಾಂತವು ವಿಶ್ವವನ್ನು ಭೌತವಸ್ತುಗಳ ಒಂದು ಸಂಗ್ರಹದಂತಲ್ಲ, ಒಂದೇ ಅಖಂಡ ವಸ್ತುವಿನ ವಿವಿಧ ಭಾಗಗಳ ನಡುವೆ ಇರುವ ಸಂಬಂಧಗಳ ಜಟಿಲ ಜಾಲದಂತೆ ನೋಡಲು ನಮ್ಮನ್ನು ನಿರ್ಬಂಧಿಸುತ್ತದೆ.’\footnote{\engfoot{Quantum theory forces us to see the universe not as a collection of physical objects, but rather as a complicated web of relation between the various parts of unified whole. }\hfill\engfoot{ –Dr. Capra}}–ಡಾ. ಕಾಪ್ರಾ

‘ಇಡಿಯ ವಿಶ್ವವೇ ಒಂದು ಮೂಲಭೂತ ಕ್ಷೇತ್ರದಂತೆ ಕಾಣಿಸುವುದು. ಅಲ್ಲದೇ ಪ್ರತಿ ಯೊಂದು ನಕ್ಷತ್ರ, ಪರಮಾಣು, ಅಲೆಯುವ ಧೂಮಕೇತು, ಮೆಲ್ಲನೇ ಸುತ್ತುವ ನೀಹಾರಿಕೆ, ತಿರು ಗುವ ಎಲೆಕ್ಟ್ರಾನು–ಇವೆಲ್ಲವೂ ಆ ಕ್ಷೇತ್ರದಲ್ಲಿನ ಒಂದೊಂದು ಅಲೆ ಅಥವಾ ಗುಳ್ಳೆಗಳಂತೆ ಕಾಣುವುವು. ಇವೆಲ್ಲವುಗಳ ಹಿಂದೆ ದೇಶಕಾಲಗಳ ಏಕತೆ ಇದೆ.\footnote{\engfoot{Far from its august perspective the universe is revealed as one elemental field in which each star, each atom, each wandering comet and slow-wheeling galaxy and flying electron is seen to be but a ripple or tumescence in the underlying space-time unity.}}

‘ಹೀಗೆ ಮಾನವನ ಎಲ್ಲ ಜಾಗತಿಕ ಪ್ರತ್ಯಕ್ಷಾನುಭವಗಳೂ ಮತ್ತು ಸತ್ಯದ ಸೂಕ್ಷ್ಮ ಅಂತಃ ಸಂವೇದನೆಗಳೂ ಅಂತ್ಯದಲ್ಲಿ ಒಟ್ಟಿಗೆ ಒಂದುಗೂಡುವುವು. ಆಗ ವಿಶ್ವದ ಈ ಮೂಲಭೂತ ಏಕತೆಯು ಸುಸ್ಪಷ್ಟವಾಗಿ ವ್ಯಕ್ತವಾಗುವುದು.’– ಲಿಂಕನ್ ಬರ್ನೆಟ್ \footnote{\engfoot{Thus, all man's perceptions of the world and all his abstract intuitions of reality merge finally into one, and the deep underlying unity of the universe is laid bare.}\hfill\engfoot{ –Lincoln Barnett}}

‘ಅಗಾಧ ನಕ್ಷತ್ರಗಳು ಅತ್ಯಂತ ಸೂಕ್ಷ್ಮವಾದ ಪರಮಾಣುಗಳು ಒಮ್ಮುಖವಾಗಿ, ಮೂಲಭೂತ ಅಖಂಡ ವಸ್ತುವಿನ ಅವಯವಗಳಂತೆ ಸಮಗ್ರ ಸಮನ್ವಯ ಏಕತೆಯನ್ನು ಪಡೆದು ನಿಂತಿವೆ.’\break–ವಿ. ರೆಡ್ನಿಕ್ \footnote{\engfoot{Infinite stars and minute atoms not only converge but also exist as integral units}\hfill\hbox{\engfoot{ –V. Rednick}}}

‘ಚಲನೆ ಮತ್ತು ಬದಲಾವಣೆಗಳು ವಸ್ತುವಿನ ಪ್ರಾಮುಖ್ಯ ಲಕ್ಷಣಗಳಷ್ಟೆ. ಚಲನೆಗೆ ಕಾರಣವಾದ ಯಾವುದೊ ಅವ್ಯಕ್ತ ಶಕ್ತಿ, ವಸ್ತುವಿನ ಹೊರಗೆ ಎಲ್ಲಿಯೋ ಇದೆ ಎಂದಲ್ಲ. ವಸ್ತುವಿನಲ್ಲೇ ಅಡಗಿರುವ ಸ್ವಭಾವಸಿದ್ಧವಾದ ಒಂದು ಲಕ್ಷಣ ಅದು. ಇದರಂತೆಯೇ ಪೌರಸ್ತ್ಯ \hbox{ಅನುಭಾವಿಗಳ} ದೇವತ್ವದ ಕಲ್ಪನೆಯೂ ಕೂಡ. ಜಗತ್ತನ್ನು ಆಳುವವನೊಬ್ಬ ಎಲ್ಲಿಯೋ ಕುಳಿತುಕೊಂಡು ಸರ್ವಾಧಿ\-ಕಾರಿಯಂತೆ ಆದೇಶ ನೀಡುವವನು ಎಂದಲ್ಲ. ಪ್ರತಿಯೊಂದನ್ನೂ ಒಳಗಿನಿಂದಲೇ ನಿಯಂತ್ರಿಸುವ ತತ್ತ್ವ ಅದು.’–ಡಾ. ಕಾಪ್ರಾ\footnote{\engfoot{Since motion and change are essential properties of things, the forces causing the motion are not outside the objects, but are an intrinsic property of matter. Correspondingly, the Eastern image of the Divine is not that of a ruler who directs the world from above, but of a principle that controls everything from within.}\hfill\engfoot{ -Dr. Capra}}

ಹೀಗೆ ವಸ್ತುವಿನ ಆಳಕ್ಕೆ ಮುಳುಗಿದಂತೆಲ್ಲ ಅವುಗಳಲ್ಲಿನ ಮೂಲಭೂತ ಏಕತೆಯು ಹೆಚ್ಚು ಹೆಚ್ಚು ಗೋಚರವಾಗುತ್ತದೆ. ಎಲ್ಲ ವಸ್ತು ಮತ್ತು ಶಕ್ತಿಗಳ ಏಕತೆಯೇ ಆಧುನಿಕ ಭೌತವಿಜ್ಞಾನ ಮತ್ತು ಪೌರಸ್ತ್ಯ ದರ್ಶನಗಳನ್ನು ಹೋಲಿಸಿ ನೋಡಿದಾಗಲೆಲ್ಲ ಮರುಕಳಿಸುವ ತಥ್ಯ. ಅತ್ಯಂತ ಸೂಕ್ಷ್ಮ ಪರಮಾಣು ವಿಜ್ಞಾನದ ವಿಭಿನ್ನ ಮಾದರಿಗಳನ್ನು ನಾವು ಅಧ್ಯಯನ ಮಾಡಿದಾಗ ಈ ಒಂದು ಅಂತರ್ ದೃಷ್ಟಿಯು ಬೇರೆ ಬೇರೆ ರೀತಿಗಳಲ್ಲಿ ತಿರುತಿರುಗಿ ವ್ಯಕ್ತವಾಗುವುದನ್ನು ನೋಡಬಹುದು. ವಸ್ತುವಿನ ಮೂಲಘಟಕಗಳೂ, ಅವುಗಳಿಂದ ಉದ್ಭವಿಸುವ ಘಟನೆಗಳೂ ಅಂತಸ್ಸಂಬಂಧಿತ, ಅಂತ\-ಸ್ಸಂಪರ್ಕಿತ, ಅಂತರಾಧಾರಿತ; ಅವುಗಳನ್ನು ಪ್ರತ್ಯೇಕ ವಸ್ತುವನ್ನಾಗಿ ಬೇರ್ಪಡಿಸಿ ಅರಿಯಲು ಸಾಧ್ಯವಿಲ್ಲ. ಒಂದು ಪರಿಪೂರ್ಣ ತತ್ತ್ವದ ಅಂಶಗಳನ್ನಾಗಿಯೇ ತಿಳಿಯಲು ಸಾಧ್ಯ.


\section*{ವಿಕಾಸದ ವಿಧಾನ}

\addsectiontoTOC{ವಿಕಾಸದ ವಿಧಾನ}

ಎಲ್ಲೆಡೆ ವ್ಯಕ್ತವಾಗಿರುವ ಶಕ್ತಿ ಒಂದೇ ಎಂಬುದನ್ನು ಮನಗಂಡ ಪ್ರಸಿದ್ಧ ಜೀವಶಾಸ್ತ್ರಜ್ಞ ಶ‍್ರೀ ಜೆ.\ ಎಸ್.\ ಹಾಲ್ಡೇನ್​ ಎನ್ನುವಂತೆ \footnote{\engfoot{It is only a narrow view of what is natural that prevents our recognising the presence of God every where within and around us. Nothing is real except God, and relations of time and space are only the order of His manifestation. Nature is just the manifestation of God and evolution is no more biological or physical phenomenon, but the order in time relations of His Manifestations.}\hfill\engfoot{ J. S. Haldane, The Science and Philosophy.}}“ಸ್ವಾಭಾವಿಕವಾದುದನ್ನು ಅಥವಾ ನಿಸರ್ಗವನ್ನು ಸಂಕುಚಿತ ದೃಷ್ಟಿಯಿಂದ ನೋಡುವುದರಿಂದ ಒಳಗೂ ಹೊರಗೂ ಎಲ್ಲೆಲ್ಲೂ ಇರುವ ದೇವರ ಅಸ್ತಿತ್ವವನ್ನು ನಾವು ತಿಳಿಯಲಾರೆವು. ದೇವರಲ್ಲದೇ ಬೇರಾವುದೂ ಸತ್ಯವಲ್ಲ. ದೇಶಕಾಲಗಳ ಸಂಬಂಧವು ಅವನು ವ್ಯಕ್ತವಾಗುವ ವಿಧಾನ. ಪ್ರಕೃತಿ ಅಥವಾ ನಿಸರ್ಗ ಎನ್ನುವುದು ದೇವರ ಆವಿರ್ಭಾವವೇ. ವಿಕಾಸವೆನ್ನುವುದು ಕೇವಲ ಜೈವಿಕ ಅಥವಾ ಭೌತಿಕ ಘಟನಾವಳಿ ಮಾತ್ರವಲ್ಲ, ಕಾಲದಲ್ಲಿ ಅವನ ಆವಿರ್ಭಾವವಾಗುವ ಕ್ರಮ ಅಥವಾ ವಿಧಾನ ವಿಶೇಷ.”

‘ವಿಕಾಸದ ಪ್ರೇರಕಶಕ್ತಿಯು ಒಂದಾನೊಂದು ಅತಿ ದೊಡ್ಡ ಜೀವಂತ ಶಕ್ತಿಯಿಂದ ನಿರ್ದೇಶಿತವೊ ಎಂಬಂತೆ ಇದೆ,’\footnote{\engfoot{....this evolutionary drive is `directed' as if by some single gigantic organism.}\hfill\hbox{\engfoot{ Chardin Phenomenon of Man}}} ಎಂದು ಚಾರ್ಡಿನ್ ಎಂದರೆ ಡಾ. ಜೆ. ಎಸ್. ಹಾಲ್ಡೇನ್​\enginline{ ‘Unity and Diversity of Life’} ಎಂಬ ತಮ್ಮ ಗ್ರಂಥದಲ್ಲಿ ಹೀಗೆ ಹೇಳಿದ್ದಾರೆ; ‘ಭೌತಶಾಸ್ತ್ರದಲ್ಲಿ ನಡೆದ ಈ ಶತಮಾನದ ಎರಡು ಪ್ರಮುಖ ಸಂಶೋಧನೆಗಳು ಆಳವಾದ ತಾತ್ತ್ವಿಕ ಅರ್ಥವನ್ನು ಹೊಂದಿವೆ. ಆ ಎರಡು ಶೋಧನೆಗಳು ಇವು: ಐನ್​ಸ್ಟೀನ್ ಕಂಡು ಹಿಡಿದ “ಕಾಲದೇಶಗಳು ಒಂದೇ ತೆರನಾದ ಸಂಬಂಧದ ವಿಭಿನ್ನ ಮುಖಗಳು” ಎಂಬುದು ಒಂದು. “ಸೂಕ್ಷ್ಮಾತಿಸೂಕ್ಷ್ಮ ಕಣಗಳಲ್ಲಿರುವ ವ್ಯತ್ಯಾಸ ಕೊನೆಯದಲ್ಲ” ಎಂಬುದು ಇನ್ನೊಂದು. ನನಗೂ ನಿಮಗೂ ಅಥವಾ ನನಗೂ ಹತ್ತಿರದ ಸೊಳ್ಳೆಗೂ ಇರುವ ವ್ಯತ್ಯಾಸ ಅಂತಿಮವಲ್ಲ ಎಂಬುದನ್ನು ನಂಬಲು ಈ ಸಿದ್ಧಾಂತ ಸಹಾಯ ಮಾಡುತ್ತದೆ... ಮುನ್​ಸ್ಟರ್​ನಲ್ಲಿನ ಪ್ರಾಣಿಶಾಸ್ತ್ರದ ಪ್ರಯೋಗಾಲಯದ ದೊಡ್ಡ ಕೋಣೆಯ ಗೋಡೆಯ ಮೇಲೆ ಪ್ರೊ.\ ರೆಂಚ್ “ತತ್ತ್ವಮಸಿ” ಎನ್ನುವ ಉಪನಿಷತ್ತಿನ ಮಹಾವಾಕ್ಯವನ್ನು ಬರೆದಿರಿಸಿಕೊಂಡದ್ದು ಇದೇ ಭಾವದಿಂದಲೇ.’

‘ನಮ್ಮ ದುರ್ಬಲವೂ ದೋಷಯುಕ್ತವೂ ಆದ ಮನಸ್ಸಿಗೆ ಪರಿಶೀಲಿಸಲು ಸಾಧ್ಯವಾಗುವ ಅತ್ಯಂತ ಸಾಮಾನ್ಯ ಹಾಗೂ ಸಣ್ಣ ಪುಟ್ಟ ಸಂಗತಿಗಳಲ್ಲೂ ತನ್ನಿಂದ ತಾನೇ ಪ್ರಕಾಶಿತವಾಗುವ ಆ ಅಸೀಮವಾದ ಅತ್ಯಂತ ಶ್ರೇಷ್ಠ ಚೇತನದಲ್ಲಿ ವಿನಯ ಪೂರ್ವಕವಾದ ಗೌರವ ಭಾವನೆಯೇ ನನ್ನ ಧಾರ್ಮಿಕ ದೃಷ್ಟಿ. ಅರಿಯಲಶಕ್ಯವಾದ ಈ ವಿಶ್ವದಲ್ಲಿ ವ್ಯಕ್ತವಾಗುವ ಶ್ರೇಷ್ಠ ವಿವೇಚನಾ ಶಕ್ತಿಯಲ್ಲಿ (ಎಂದರೆ ಸಮಸ್ತ ವಿಶ್ವದ ಹಿನ್ನೆಲೆಯಲ್ಲಿ ಇದ್ದುಕೊಂಡು ಸರ್ವವ್ಯಾಪಿಯಾಗಿರುವ ವಿಶ್ವನಿಯಾಮಕ ಶಕ್ತಿಯಲ್ಲಿ ಅಥವಾ ದೇವರಲ್ಲಿ) ಆಳವೂ ಭಾವನಾತ್ಮಕವೂ ಆದ ದೃಢ ವಿಶ್ವಾಸವೇ ದೇವರನ್ನು ಕುರಿತ ನನ್ನ ಭಾವನೆಯಾಗಿದೆ'\footnote{\engfoot{My religion consists of a humble admiration of the illimitable Superior Spirit who reveals Himself in the slight details we are able to preceive with our frail and feeble minds. That deeply emotional conviction of the presence of a Superior Reasoning Power, which is revealed in he incomprehensible universe forms my idea of God.}\hfill\engfoot{ Einstein}} ಎಂಬುದು ಈ ನಿಟ್ಟಿನಲ್ಲಿ ಐನ್ ಸ್ಟೀನ್ ಮತ.

೧೯೬೯ನೇ ಇಸವಿಯ 'ಟ್ರಾನ್ಸ್​ಪರ್ಸನಲ್ ಸೈಕಾಲಜಿ’ಯ ಚಳಿಗಾಲದ ಸಂಚಿಕೆಯಲ್ಲಿ ಒಂದು ವಿಶಿಷ್ಟಲೇಖನ ಪ್ರಕಟವಾಯಿತು. ಭೌತ, ಖಭೌತ, ಆಕಾಶ ವಿಜ್ಞಾನಿಗಳ ಹಾಗೂ ಇತರ ವಿಜ್ಞಾನದ ವಿವಿಧ ಕ್ಷೇತ್ರಗಳಲ್ಲಿ ದುಡಿದು ಪ್ರಸಿದ್ಧರಾದ ವಿಜ್ಞಾನಿಗಳ ಮತ್ತು ಅನುಭಾವಿಗಳ ವಾಕ್ಯಗಳನ್ನು ಈ ಲೇಖನದ ಬೇರೆ ಬೇರೆ ಭಾಗಗಳಲ್ಲಿ ಅವರ ಹೆಸರನ್ನು ಸೂಚಿಸದೆ ಉದ್ಧರಿಸ ಲಾಗಿತ್ತು. ಲೇಖನದ ಕೊನೆಯಲ್ಲಿ ನೀಡಿದ ಅಡಿಟಿಪ್ಪಣಿಯಲ್ಲಿ ಮಾತ್ರ ಅವತರಣಿಕೆಗಳ ಮೂಲ ವನ್ನೂ ಗ್ರಂಥಕರ್ತರ ಹೆಸರುಗಳನ್ನೂ ಸೂಚಿಸಿದ್ದರು. ಅಡಿಟಿಪ್ಪಣಿಯನ್ನು ನೋಡದೇ ಅನುಭಾವಿಗಳ ವಾಕ್ಯಗಳಾವುವು? ವಿಜ್ಞಾನಿಗಳ ವಾಕ್ಯಗಳಾವುವು? ಎಂದು ಹೇಳಲು ಸಾಧ್ಯವಿರಲಿಲ್ಲ. ಎಂದರೆ ಆಧುನಿಕ ವಿಜ್ಞಾನಿಗಳ ಮತ್ತು ಪುರಾತನ ಅನುಭಾವಿಗಳ ವಾಕ್ಯಗಳಲ್ಲಿ ಅಷ್ಟೊಂದು ಸಾಮ್ಯ ಕಂಡು ಬರುತ್ತಿತ್ತು. ಈ ದಿಕ್ಕಿನಲ್ಲಿ ಅಭಿರುಚಿಯುಳ್ಳವರು ಡಾ. ಕಾಪ್ರಾ ಅವರ ಗ್ರಂಥವನ್ನು ಅವಶ್ಯವಾಗಿ ಓದಬೇಕು.

ಐನ್​ಸ್ಟೀನ್ ತಮ್ಮ ಸಾಪೇಕ್ಷತಾಸಿದ್ಧಾಂತವನ್ನು ಪ್ರಕಟಿಸುವ ಎಂಟು ವರ್ಷಗಳ ಮೊದಲು ಸ್ವಾಮಿ ವಿವೇಕಾನಂದರು ಚಿಕಾಗೋ ಸರ್ವಧರ್ಮ ಸಮ್ಮೇಳನದಲ್ಲಿ ಭಾಗವಹಿಸಿ ಭಾರತಕ್ಕೆ\break ಹಿಂದಿರುಗಿದಾಗ ಕುಂಭಕೋಣದಲ್ಲಿ ನೀಡಿದ ಸ್ವಾಗತಕ್ಕೆ ಉತ್ತರವಾಗಿ ಮಾಡಿದ ಒಂದು ಉಪನ್ಯಾಸದಲ್ಲಿ ಹೀಗೆಂದರು:

‘ಯೋಚನಾಶೀಲರಾದ ಯೂರೋಪಿನ ಜನರು, ಏಕೆ, ವಿಶ್ವದ ವಿಚಾರಶೀಲರೆಲ್ಲ ನಮ್ಮಿಂದ ಒಂದು ಮಹಾಸತ್ಯವನ್ನು ನಿರೀಕ್ಷಿಸುತ್ತಿರುವರು. ಅದೇ ಸಮಸ್ತ ವಿಶ್ವದ ಆಧ್ಯಾತ್ಮಿಕ ಏಕತೆಯನ್ನು ಸಾರುವ ಸನಾತನ ಮಹಾತತ್ತ್ವ. ಈ ತತ್ತ್ವ ಇಂದು ಉಚ್ಚವರ್ಗದವರಿಗಿಂತ ಕೆಳವರ್ಗದವರಿಗೆ ಅಥವಾ ದಲಿತರಿಗೆ ಹೆಚ್ಚು ಬೇಕಾಗಿದೆ. ವಿದ್ಯಾವಂತರಿಗಿಂತ ಅಜ್ಞಾನಿಗಳಿಗೆ ಹೆಚ್ಚು ಆವಶ್ಯಕವಾಗಿದೆ, ಬಲಾಢ್ಯರಿಗಿಂತ ಬಲಹೀನರಿಗೆ ಹೆಚ್ಚು ಬೇಕಾಗಿದೆ, ಸುಸಂಸ್ಕೃತರಿಗಿಂತ ಶ‍್ರೀಸಾಮಾನ್ಯ ರಿಗೆ ಹೆಚ್ಚು ಆವಶ್ಯಕವಾಗಿದೆ. ಆಧುನಿಕ ವೈಜ್ಞಾನಿಕ ಸಂಶೋಧನೆಗಳು (ವಸ್ತುವಿನ ನೈಜಸ್ವಭಾವಗಳ ಅರಿವಿನಿಂದ) ವಾಸ್ತವಿಕವಾಗಿ ಸಮಸ್ತವಿಶ್ವದ ಏಕತೆಯನ್ನು ಪ್ರಯೋಗಾತ್ಮಕವಾಗಿ ಸಾರಿ ಹೇಳುತ್ತಿವೆಯೆಂಬು\-ದನ್ನು ಮದರಾಸು ವಿಶ್ವವಿದ್ಯಾಲಯದ ಪದವೀಧರರಿಗೆ ನಾನು ಹೇಳಬೇಕಾದುದಿಲ್ಲ. ವಸ್ತುವಿನ ಮೂಲ ಸ್ವರೂಪದ ದೃಷ್ಟಿಯಿಂದ ಅಥವಾ ವಸ್ತುವಿಜ್ಞಾನದ ಭಾಷೆಯಲ್ಲಿ ಹೇಳುವು ದಾದರೆ, ನಾನು, ನೀವು, ಸೂರ್ಯಚಂದ್ರ, ನಕ್ಷತ್ರ ಎಲ್ಲವೂ ಅನಂತ ವಸ್ತುವಿನ ಸಾಗರದಲ್ಲಿ ಸಣ್ಣಪುಟ್ಟ ಅಲೆಗಳು ಮಾತ್ರ. ಶತಮಾನಗಳ ಹಿಂದೆ ಭಾರತೀಯ ಮನಶ್ಶಾಸ್ತ್ರವು ಮನಸ್ಸು ಮತ್ತು ದೇಹ–ಇವು ಸಮಷ್ಟಿ ಅಥವಾ ಸಮಗ್ರ ವಸ್ತುಸಾಗರದಲ್ಲಿ ಅಲೆಗಳು ಮಾತ್ರ ಎಂಬುದನ್ನು ಹೇಗೆ ಸಾರಿದ್ದವು!ಉಪನಿಷತ್ತು ಇನ್ನೂ ಒಂದು ಹೆಜ್ಜೆ ಮುನ್ನಡೆದು ಪ್ರಕೃತಿ ಅಥವಾ ಸಮಸ್ತ ಜಗತ್ತಿನ ಏಕತೆಯ ಹಿನ್ನೆಲೆಯಲ್ಲಿ ಇರುವುದೊಂದೇ ಆತ್ಮವೆಂದು ಸಾರಿತು. ಈ ವಿಶ್ವಬ್ರಹ್ಮಾಂಡದಲ್ಲಿ ರುವ ಪರಮ ಆತ್ಮ ಒಂದೇ ಅಥವಾ ಒಬ್ಬನೇ. ಕೊನೆಯಲ್ಲಿ ಎಲ್ಲವೂ ಒಂದೇ.’

ವಸ್ತುವಿನ ತಳವನ್ನು ಪ್ರವೇಶಿಸಿದರೆ ಏಕತೆ ಕಾಣುವುದು. ಮನುಷ್ಯ ಮನುಷ್ಯರಲ್ಲಿ ಬೇರೆ ಬೇರೆ ಜನಾಂಗಗಳಲ್ಲಿ, ಉನ್ನತ ನೀಚರಲ್ಲಿ ಶ‍್ರೀಮಂತ ಬಡವರಲ್ಲಿ, ದೇವಮಾನವರಲ್ಲಿ ಮಾನವ ಮತ್ತು ಪ್ರಾಣಿಗಳಲ್ಲಿ ಎಲ್ಲೆಲ್ಲೂ ತೋರಿಕೆಯನ್ನು ಭೇದಿಸಿ ಆಳಕ್ಕೆ ಮುಳುಗಿದಾಗ ಈ ಏಕತೆ ಗೋಚರಿಸುವುದು. ಅವೆಲ್ಲ ಒಂದೇ ತತ್ತ್ವದ ವಿಭಿನ್ನ ಆವಿರ್ಭಾವಗಳು.


\section*{ಪರಮಾಣು ಲೀಲಾವಿನೋದ}

\addsectiontoTOC{ಪರಮಾಣು ಲೀಲಾವಿನೋದ}

ಪರಮಾಣು ಮಟ್ಟಕ್ಕಿಳಿದಾಗ ವಿಜ್ಞಾನಿಗೂ ಈ ಏಕತೆಯ ಅರಿವಾಗುತ್ತದೆ. ಪರಮಾಣು ಸೂಕ್ಷ್ಮ ಕಣಗಳ ದೃಷ್ಟಿಯಿಂದ ವಸ್ತುಗಳು ಗುಣ ಗಾತ್ರ ರೂಪ ಆಕಾರಗಳಲ್ಲಿ ಎಷ್ಟೇ ಭಿನ್ನವಾಗಿದ್ದರೂ ಅವುಗಳೆಲ್ಲ ಮೂಲತಃ ಪರಮಾಣುಗಳೆಂಬ ಇಟ್ಟಿಗೆಗಳಿಂದಲೇ ರಚಿತವಾದವುಗಳು.

ಸಾಮಾನ್ಯ ಪೆನ್ಸಿಲ್ ಮೊನೆಯಿಂದ ಗುರುತಿಸಬಹುದಾದ ಒಂದು ಪುಟ್ಟ ಬಿಂದುವಿನಲ್ಲಿ ಒಂದು ಮಿಲಿಯ ಪರಮಾಣುಗಳನ್ನು ತುಂಬಿಸಿದ ಮೇಲೆಯೂ ಜಾಗ ಮಿಗುತ್ತದೆಯೆಂದರೆ ಪರಮಾಣು ಅದೆಷ್ಟು ಸೂಕ್ಷ್ಮಾತಿಸೂಕ್ಷ್ಮವಾಗಿದೆ ಎಂಬ ಕಲ್ಪನೆ ನಮ್ಮ ಮನಸ್ಸಿನಲ್ಲಿ ಮೂಡೀತು. ಅಂತಹ ಪರಮಾಣುವಿನ ಮಧ್ಯೆ ದ್ರಾಕ್ಷಿ ಹಣ್ಣಿನ ನಡುವೆ ಬೀಜ ಇರುವಂತೆ ಬೀಜಕೇಂದ್ರವಿದೆ. ಅದರಲ್ಲಿ ಪ್ರೋಟಾನು ನ್ಯೂಟ್ರಾನುಗಳೆಂಬ ಸೂಕ್ಷ್ಮಕಣಗಳಿವೆ. ಅವು ಸೆಕೆಂಡಿಗೆ ನಲ್ವತ್ತು ಸಾವಿರ ಮೈಲು ವೇಗದಲ್ಲಿ ಅಲ್ಲಿ ತಿರುಗುತ್ತಿವೆ. ಇಲೆಕ್ಟ್ರಾನುಗಳೆಂಬ ಮತ್ತೂ ಸೂಕ್ಷ್ಮವಾದ ಕಣಗಳು ಬೀಜ ಕೇಂದ್ರದ ಸುತ್ತ ಸುತ್ತುತ್ತಿವೆ. ಫ್ಯಾನು ಜೋರಾಗಿ ತಿರುಗಿದಾಗ ಅದರ ಅಲಗುಗಳು ಕಾಣಿಸದೇ ವೃತ್ತವೇ ತಿರುಗುತ್ತಿರುವಂತೆ ನಮಗೆ ಭಾಸವಾಗುವ ಹಾಗೆ ಸೆಕೆಂಡಿಗೆ ಆರುನೂರು ಮೈಲುವೇಗದಿಂದ ಸುತ್ತುವ ಇಲೆಕ್ಟ್ರಾನುಗಳಿಂದ ಬೀಜಕೇಂದ್ರಕ್ಕೂ ಇಲೆಕ್ಟ್ರಾನು ಪಥಕ್ಕೂ ನಡುವಣ ಎಡೆಯೆಲ್ಲ ತುಂಬಿದಂತೆ ತೋರಿ ವಸ್ತುವು ಅಖಂಡವಾಗಿ ಕಾಣಿಸುತ್ತದೆ. ವಸ್ತುವಿನ ಮೂಲಘಟಕಗಳಾದ ಈ ಶಕ್ತಿಯ ಕಣಗಳು ಗಿರಗಿರನೆ ತಿರುಗುತ್ತಿದ್ದರೂ, ಗ್ರಹಿಸಲಸಾಧ್ಯವಾದ ಈ ಶಕ್ತಿಯ ಕಣಗಳೇ ನಮ್ಮ ಪಾಲಿಗೆ ತಿನ್ನಲು ಆಹಾರ ಪದಾರ್ಥಗಳಾಗಿ, ಉರಿಸಲು ಇದ್ದಲಾಗಿ ಎಣ್ಣೆಯಾಗಿ, ಉಡಲು ಬಟ್ಟೆಬರೆಗಳಾಗಿ ಇನ್ನಿತರ ಅಸಂಖ್ಯ ವಸ್ತುಗಳಾಗಿ ಒಂದೆಡೆ ಪರಿವರ್ತಿತವಾದರೆ, ಇನ್ನೊಂದೆಡೆ ವಿಭಿನ್ನ ಶರೀರಗಳಾಗಿ ಅವಯವ ಅಂಗಾಂಗಗಳಾಗಿ ಅವುಗಳ ಬೆಳವಣಿಗೆ ಅಭಿವರ್ಧನೆಯ ಸಾಧನವಾಗಿ, ಮನುಷ್ಯನಾಗಿ, ಅತಿಸೂಕ್ಷ್ಮ ಬೀಜಕಣಗಳ ಮೂಲಕ ವಂಶವಾಹಿಗಳಾಗಿ ರೂಪಾಂತರಿತವಾಗಿವೆ.

ಅಚ್ಚಿನ ಮೊಳೆಗಳು ಕಾಗದದ ಮೇಲೆ ಬೀಳುವುದರಿಂದ ಪುಸ್ತಕವು ನಿರ್ಮಿತವಾಗುತ್ತದೆ. ಆದರೆ ಆ ಮೊಳೆಗಳು ಯಾವುದೋ ಒಂದು ಕ್ರಮಬದ್ಧ ರೀತಿಯಲ್ಲಿ ಕಾಗದದ ಮೇಲೆ ಬೀಳುತ್ತವಷ್ಟೆ. ಅಲ್ಲಿ ಒಂದು ವ್ಯವಸ್ಥೆ, ಶಿಸ್ತು, ಕ್ರಮ–ಇವುಗಳನ್ನು ಕಾಣಬಹುದು ಮಾತ್ರವಲ್ಲ, ಅವು ಉಂಟು ಮಾಡುವ ಗುರುತುಗಳನ್ನು, ಅಂದರೆ ಅಕ್ಷರಗಳನ್ನು ಪರಿಶೀಲಿಸಿದಾಗ ನಮ್ಮ ಮನಸ್ಸಿಗೆ ಅರ್ಥವೂ ತಿಳಿದುಬರುವುದು. ಆದುದರಿಂದ ಯಾವುದೋ ಅರ್ಥವನ್ನುಂಟುಮಾಡುವಂತೆ ಅಚ್ಚಿನ ಮೊಳೆಗಳನ್ನು ಜೋಡಿಸುವ ವ್ಯವಸ್ಥೆಯು ಗ್ರಂಥಕರ್ತನ ಇಚ್ಛೆಯಿಂದ ನಡೆದಿದೆ ಎನ್ನಲು ನಾವು ಸಂದೇಹಪಡುವುದಿಲ್ಲ. ನಾವು ಕಾಣುವ, ಎಣಿಸಲಾಗದಷ್ಟು ಸಂಕೀರ್ಣವೂ ವೈವಿಧ್ಯಪೂರ್ಣವೂ ವಿಚಿತ್ರವೂ ರಹಸ್ಯಾತ್ಮಕವೂ ಆದ ವಸ್ತುಗಳಿಂದ ಕೂಡಿಕೊಂಡ ಈ ಜಗತ್ತು ಇಷ್ಟ ಬಂದಂತೆ, ಅಸಂಬದ್ಧವಾಗಿ, ಚೆಲ್ಲಾಪಿಲ್ಲಿಯಾಗಿ ನಿಯಮರಹಿತವಾಗಿ ನಡೆಯುತ್ತಿಲ್ಲ, ಗಣಿತಸೂತ್ರ ಹೇಳುವ ಕಟ್ಟುನಿಟ್ಟಿಗೆ ಸಮನಾಗಿ ಅದು ನಡೆಯುತ್ತಿದೆ ಎಂಬುದನ್ನು ಐನ್​ಸ್ಟೀನ್ ಸಾಪೇಕ್ಷವಾದದ ಮೂಲಕ ನಾವು ತಿಳಿಯಬಹುದು. ಜೊತೆಗೆ ಕಟ್ಟುನಿಟ್ಟಿನಂತೆ ನಡೆಯುವ ನಿಯಮಗಳ ಹಿನ್ನೆಲೆಯಲ್ಲಿ ಸರ್ವತ್ರ ಸರ್ವವ್ಯಾಪಿಯಾಗಿರುವ ಅಲೌಕಿಕವಾದ, ಅಣುಕಣಗಳ ಅರ್ಥಪೂರ್ಣ ವ್ಯವಸ್ಥೆಗೆ ಕಾರಣವಾದ ಪರಮ ಪ್ರಜ್ಞೆಯ ಅಥವಾ ಪರಮಾತ್ಮನ ನೆರವಿನ ಹೊಳಹೂ ನಮಗಿಲ್ಲಿ ದೊರೆಯುವುದು. ‘ದೇವರು ಪರಮಸೂಕ್ಷ್ಮ ಆದರೆ ಪಕ್ಷಪಾತಿಯಲ್ಲ.’ ‘ದೇವರು ಈ ಜಗತ್ತಿನೊಡನೆ ಪಗಡೆಯ ದಾಳ ಎಸೆಯುವಂತೆ ಆಟವಾಡುವುದಿಲ್ಲ,’ ಎಂಬ ಐನ್​ಸ್ಟೀನ್ ಮಾತಿನಲ್ಲಿ ಯಾವ ಅವ್ಯವಸ್ಥೆ ಅಥವಾ ಕ್ರಮಹೀನತೆಯೂ ಇರುವುದಿಲ್ಲ. ಅವ್ಯವಸ್ಥೆ ಅಥವಾ ಕ್ರಮಹೀನತೆ ಎಂದು ನಮಗೆ ತೋರಿಬರುವಲ್ಲೂ, ಆ ಅವ್ಯವಸ್ಥೆಯ ಹಿಂದೆ ವ್ಯವಸ್ಥೆ, ಆ ಅಕ್ರಮದ ಹಿಂದೆ ಕ್ರಮ ಗೋಚರಿಸುತ್ತದೆ ಎನ್ನುವುದು ಸೂಚಿತವಾಗುತ್ತದೆ. ಸರ್ವಶಕ್ತನ ಅರ್ಥಪೂರ್ಣ ವ್ಯವಸ್ಥೆಯ ಕ್ಷಣಿಕ ದರ್ಶನ ಇಲ್ಲಿ ಸಿಗುವುದು. ಎಷ್ಟೊಂದು ಸೂಕ್ಷ್ಮವಾದ ಆದರೆ ನಿಖರವಾದ, ವ್ಯವಸ್ಥಿತವಾದ ಮತ್ತು ಅರ್ಥಪೂರ್ಣವಾದ ಸೃಷ್ಟಿ ಆ ದೇವರದ್ದು! ಅದಕ್ಕೆ ಐನ್​ಸ್ಟೀನರೆಂದದ್ದು–‘ಎಲ್ಲ ವಸ್ತುಗಳ ಆಳವನ್ನೂ ಶೋಧಿಸುವ ಶಕ್ತಿ ಇದ್ದ ವ್ಯಕ್ತಿಯಲ್ಲಿ ಧರ್ಮವಿಲ್ಲದ ವೈಜ್ಞಾನಿಕ ಮನಸ್ಸನ್ನು ಕಾಣುವುದು ಕಷ್ಟ.’ ಎಂದರೆ ಪೂರ್ಣ ಸತ್ಯಾನ್ವೇಷಣೆಯ ಸಾಮರ್ಥ್ಯ ಮತ್ತು ಹಂಬಲವಿರುವ ವಿಜ್ಞಾನಿ ಧಾರ್ಮಿಕನಾಗದೆ ಇರಲಾರ. ಅವರೆಂದಂತೆ ‘ಧರ್ಮವಿಲ್ಲದ ವಿಜ್ಞಾನ ಕುಂಟು, ವಿಜ್ಞಾನವಿಲ್ಲದ ಧರ್ಮ ಕುರುಡು!’


\section*{ವಿಶ್ವದ ಹಿನ್ನೆಲೆಯಲ್ಲಿ ಪರಮಾತ್ಮ}

\addsectiontoTOC{ವಿಶ್ವದ ಹಿನ್ನೆಲೆಯಲ್ಲಿ ಪರಮಾತ್ಮ}

ಎಲ್ಲ ಕಡೆಯಲ್ಲಿದ್ದು, ಎಲ್ಲ ಕಾಲಗಳಲ್ಲೂ, ಎಲ್ಲವನ್ನೂ ನಿಯಂತ್ರಿಸುವ ತತ್ತ್ವವೇ ಪರಮಾರ್ಥ ತತ್ತ್ವ ಅಥವಾ ಪರಮಾತ್ಮ ತತ್ತ್ವ. ಸರ್ವಶಕ್ತನಾದ ಭಗವಂತನು ತಾನು ಸರ್ವಭೂತಗಳ ಹಿನ್ನೆಲೆಯಲ್ಲಿ ಅಂತರ್ಯಾಮಿಯಾಗಿ ನಿರ್ವಿಕಾರನಾಗಿದ್ದುಕೊಂಡು ಸರ್ವಭೂತಗಳನ್ನೂ ತನ್ನ ಮಾಯಾ\-ಶಕ್ತಿ\-ಯಿಂದ ಯಂತ್ರವತ್ ಪರಿಭ್ರಮಿಸುವಂತೆ ಮಾಡುತ್ತಾನೆ ಎನ್ನುವ ಮಾತು ಭಗವದ್ಗೀತೆಯಲ್ಲಿ ಬರುತ್ತದೆ. ಅಣುಕಣಗಳ ಅರ್ಥಪೂರ್ಣ ಚಲನವಲನಗಳನ್ನು ಕುರಿತು ಓದಿದವರಿಗೆ ಮೇಲಿನ ಈ ಮಾತು ಅತ್ಯಂತ ಸುಸಂಗತವಾಗಿ ಕಾಣದಿರದು. ಉಪನಿಷತ್ತುಗಳಲ್ಲಿ ಕಂಡುಬರುವ ದೇವರನ್ನು ಕುರಿತ ನಿರೂಪಣೆ ಎಲ್ಲ ಧರ್ಮಗಳ ಭಗವದ್ಭಾವನೆಗಳನ್ನೂ ಒಳಗೊಂಡಂತಿದೆ. ಸಗುಣಸಾಕಾರದ ಭಾವನೆಗಳನ್ನೂ, ನಿರ್ಗುಣನಿರಾಕಾರದ ಭಾವನೆಗಳನ್ನೂ ಅಲ್ಲಿ ಏಕ ಸೂತ್ರದಲ್ಲಿ ಹಿಡಿದಿಡಲಾಗಿದೆ. ಮಾತ್ರವಲ್ಲದೆ ವಿಜ್ಞಾನಿಗಳು ಹೇಳುವ ವಿಶ್ವನಿಯಾಮಕ ಶಕ್ತಿಯ ಬಗೆಗಿನ ಭಾವನೆಗೆ ನಿಕಟವಾಗಿದ್ದು ಅದಕ್ಕೂ ಮೀರಿದ ಅರ್ಥವ್ಯಾಪ್ತಿಯನ್ನೂ, ಪೂರ್ಣತೆಯನ್ನೂ ಹೊಂದಿದೆ. ಶ್ವೇತಾಶ್ವತರ\break ಉಪನಿಷತ್ತಿನ \enginline{(6.11)} ಒಂದು ಮಂತ್ರದಲ್ಲಿ ಬರುವ ಗಭೀರ ಅರ್ಥ ವ್ಯಾಪ್ತಿಯುಳ್ಳ ಹನ್ನೊಂದು ಶಬ್ದಗಳನ್ನು ಪ್ರತಿಯೊಬ್ಬರೂ ಪರಿಶೀಲಿಸಬಹುದು:

\begin{verse}
ಏಕೋ ದೇವಃ ಸರ್ವಭೂತೇಷು ಗೂಢಃ\\
 ಸರ್ವವ್ಯಾಪೀ ಸರ್ವಭೂತಾಂತರಾತ್ಮಾ~।\\
 ಕರ್ಮಾಧ್ಯಕ್ಷಃ ಸರ್ವಭೂತಾಧಿವಾಸಃ \\
 ಸಾಕ್ಷೀ ಚೇತಾ ಕೇವಲೋ ನಿರ್ಗುಣಶ್ಚ~॥ (ಶ್ವೇ. \enginline{6.11)}\footnote{ ಹೆಚ್ಚಿನ ವಿವರಣೆಗೆ ಓದಿ–ಡಿ.ವಿ.ಜಿ. ಅವರ ‘ದೇವರು: ಒಂದು ವಿಚಾರಲಹರಿ’}
\end{verse}

ಯಾರು ಎಲ್ಲ ಭೂತಗಳಲ್ಲೂ (ಸೂಕ್ಷ್ಮಾತಿಸೂಕ್ಷ್ಮನಾಗಿ) ಅಡಗಿರುವನೊ, ಸರ್ವವ್ಯಾಪಿಯೋ ಸರ್ವಭೂತಗಳ ಅಂತರಾತ್ಮನೋ, ಕರ್ಮಗಳಿಗೆ ತಕ್ಕ ಫಲವನ್ನು ಕೊಡುವ ಅಧಿಷ್ಠಾತೃವೋ ಸರ್ವ ಪ್ರಾಣಿಗಳಿಗೆ ಆಶ್ರಯನೋ, ಸಾಕ್ಷಿಯೋ ಎಂದರೆ ಎಲ್ಲವನ್ನೂ ನೋಡುವವನೋ ಎಲ್ಲರಿಗೂ ಚೇತನವನ್ನು ಕೊಡುವವನೋ, ಸತ್ವಾದಿಗುಣಗಳಿಗೆ ಅತೀತನೋ ಆ ದೇವನು ಒಬ್ಬನೇ.

ಆತ್ಮನ ನೈಜಸ್ವರೂಪವನ್ನು ಹೇಳಹೊರಟ ಉಪನಿಷತ್ತಿನಲ್ಲಿ ಮತ್ತೆ ಮತ್ತೆ ಪ್ರತಿಪಾದಿಸಲ್ಪಡುವ ಒಂದು ಪ್ರಾಮುಖ್ಯ ಸಂಗತಿಯೇ ಪರಮಾರ್ಥತತ್ತ್ವದ ಸರ್ವವ್ಯಾಪಿತ್ವ ಮತ್ತು ಏಕತೆ.

ದೇವರು ಎಲ್ಲರಲ್ಲೂ ಇದ್ದಾನೆ, ಆದರೆ ಎಲ್ಲರೂ ದೇವರಲ್ಲಿಲ್ಲ ಎಂಬುದು ಪರಮಹಂಸರ ಒಂದು ವಾಣಿ. ಪರತತ್ತ್ವ ಎಲ್ಲರಲ್ಲಿದೆ; ಅದು ಅತ್ಯಂತ ಸೂಕ್ಷ್ಮವಾದುದರಿಂದ ಸಾಮಾನ್ಯವಾಗಿ ಅದರ ಅಸ್ತಿತ್ವದ ಅರಿವಾಗುತ್ತಿಲ್ಲ. ತನ್ನ ಸಾಮಾನ್ಯ ಅರಿವಿಗೆ ನಿಲುಕುವಷ್ಟೇ ಮನುಷ್ಯನ ಸತ್ಯದ ಕಲ್ಪನೆ ಸೀಮಿತವಾಗುತ್ತದೆ. ತಾನು ಹುಟ್ಟಿ ಬೆಳೆದು ಬಂದ ಕುಟುಂಬ ಮತ್ತು ಪರಿಸರ–ಇವು ನೀಡಿದ ಭಾವನೆಗಳಿಂದ ಬಂಧಿತನವನು. ಕುರಿಗಳ ಜೊತೆಗೆ ಹುಟ್ಟಿದಾರಭ್ಯ ಬೆಳೆದು ಬಂದ ಹುಲಿ ತನ್ನನ್ನು ತಾನು ಕುರಿ ಎಂದೇ ತಿಳಿದುಕೊಂಡಿತು. ರಾಜಪುತ್ರ ಶೈಶವದಿಂದಲೇ ಅಗಸನ ಮನೆಯಲ್ಲಿ ಬೆಳೆದು ತನ್ನನ್ನು ಅಗಸನೆಂದೇ ತಿಳಿದುಕೊಂಡ. ರಕ್ತಮಾಂಸ ನರವ್ಯೂಹಗಳಿಂದ ಕೂಡಿಕೊಂಡು ಯಾವುದೋ ಜಾತಿಮತಕುಲಗೋತ್ರಕ್ಕೆ ಸೇರಿಕೊಂಡಂಥವನು ‘ನಾನು’ ಎಂಬ ಭಾವನೆಯಿಂ ದಲೇ ಮನುಷ್ಯನ ವ್ಯವಹಾರ ನಡೆಯುತ್ತದೆ. ಪ್ರಜ್ಞೆಯ ಆಳಕ್ಕೆ ಮುಳುಗಿ ತನ್ನ ನೈಜ ಸ್ವರೂಪವನ್ನು ಅವನು ತಿಳಿದುಕೊಳ್ಳಲಾರ. ತಿಳಿಯುವುದಕ್ಕೆ ಮೊದಲು ಯುಕ್ತಿಯುಕ್ತವಾಗಿ ತಿಳಿಸಿಕೊಟ್ಟರೂ ನಂಬಲಾರ. ಆದರೆ ಇಂದು ವೈಜ್ಞಾನಿಕ ಸಂಶೋಧನೆಗಳ ಬೆಳಕಿನಲ್ಲಿ ಇದನ್ನು ತಿಳಿಯುವ ಹಂಬಲ ವಿರುವವರಿಗೆ ತಿಳಿಯುವುದು ಸುಲಭಸಾಧ್ಯ.

‘ಇದೆಲ್ಲವೂ ನಿಜವಾಗಿಯೂ ಪರತತ್ತ್ವವೇ’, ‘ಇದೆಲ್ಲವೂ ಪುರುಷನೇ’, ‘ಆ ಪರತತ್ತ್ವವು ಏಕ ಮೇವಾದ್ವಿತೀಯ’ ಎಂದರೆ ತನಗೆರಡನೆಯದಿಲ್ಲದ ಒಂದೇ ತತ್ತ್ವ–ಇದನ್ನು ಅರ್ಥೈಸಿಕೊಳ್ಳಲು ಆಕಾಶದ ಕಲ್ಪನೆ ನಮಗೆ ಸಹಾಯಕ. ‘ಆತ್ಮತತ್ತ್ವವೇ ಪರತತ್ತ್ವ’. ‘ನಾನು ಪರತತ್ತ್ವವೇ’, ‘ನೀನು ಆ ಪರತತ್ತ್ವವೇ ಆಗಿದ್ದೀಯೇ’, ‘ಆ ಪರತತ್ತ್ವವು ಸತ್ಯಜ್ಞಾನಗಳ ಅನಂತ’. ‘ಇದೆಲ್ಲವು ಆತ್ಮವೇ’, ‘ಆ ಪರತತ್ತ್ವವೇ ಈ ವಿಶ್ವ.’

ಮೇಲಿನ ಮಾತುಗಳು ತಾತ್ತ್ವಿಕನ ದೃಷ್ಟಿಕೋನವೆಂದಿಟ್ಟುಕೊಂಡರೆ ಧಾರ್ಮಿಕರ ಅಥವಾ ಆಸ್ತಿಕರ ದೃಷ್ಟಿಯಲ್ಲಿ ಅವು ಈ ರೂಪವನ್ನು ತಾಳುತ್ತವೆ:

ಆತನ ಸಾನ್ನಿಧ್ಯ ಸದಾ ನಮ್ಮಲ್ಲಿ ಇದ್ದೇ ಇದೆ, ನಾವು ಅವನಿಂದಲೇ ಅಸ್ತಿತ್ವ ಸತ್ತ್ವಶಕ್ತಿಗಳನ್ನು ಪಡೆದು ಅವನಲ್ಲೇ ಬದುಕಿ ಅವನೆಡೆಗೇ ಹಿಂದಿರುಗುತ್ತೇವೆ... ಸಮಸ್ತ ಸೃಷ್ಟಿಯ ಹಿನ್ನೆಲೆಯಲ್ಲಿ ಆತನಿದ್ದಾನೆ. ಏಕಮೇವಾದ್ವಿತೀಯನವನು. ಆ ಸರ್ವವ್ಯಾಪಿಯೂ ಏಕವೂ ಆದ ತತ್ತ್ವವೆಂದೆಂದೂ ಇದೆ. ಆದರೆ ಅದನ್ನು ನಾವು ಸರಿಯಾಗಿ ಅರಿತುಕೊಂಡರೇನೇ ನಮ್ಮಲ್ಲಿ ಬದಲಾವಣೆ ಸಾಧ್ಯ, ಸರಿಯಾಗಿ ಅರಿತುಕೊಳ್ಳುವುದಕ್ಕೆ ಮೊದಲು ಅದು ಸತ್ಯ ಎಂಬುದು ನಮ್ಮ ಮನಸ್ಸಿಗೆ ಬೌದ್ಧಿಕ ಮಟ್ಟದಲ್ಲಿ ಖಚಿತವಾಗಬೇಕು. ಅದನ್ನು ನಮ್ಮ ಮನಸ್ಸು ಪೂರ್ತಿಯಾಗಿ ಒಪ್ಪಿಕೊಳ್ಳ ಬೇಕು. ಆಗಲೇ ನಮ್ಮ ಬದುಕಿನಲ್ಲಿ ಅಭ್ಯುದಯದ ಅರುಣೋದಯ ಆರಂಭವಾಗುತ್ತದೆ.

ಆ ಸರ್ವವ್ಯಾಪಿಯಾದ ಸರ್ವಶಕ್ತನು ನಮ್ಮ ದೈನಂದಿನ ಜೀವನದಲ್ಲಿ ಅನುಭವಕ್ಕೆ ಬರುವ, ನಾವು ಸಾಮಾನ್ಯವಾಗಿ ಕಲ್ಪಿಸಿಕೊಳ್ಳುವಂಥ ಒಬ್ಬ ವ್ಯಕ್ತಿ ಅಲ್ಲ. ಆದರೆ ಆತನ ಸಾನ್ನಿಧ್ಯಬೋಧೆ ಗಾಗಿ ಯಥಾರ್ಥ ವ್ಯಾಕುಲತೆ ಅಥವಾ ಹಂಬಲ ಉಂಟಾದವರ ಪಾಲಿಗೆ ವ್ಯಕ್ತಿಯಾಗಿಯೂ ಪರಮಾಪ್ತನಾಗಿಯೂ ವ್ಯಕ್ತನಾಗಬಲ್ಲ. ತನಗೆ ಯಾವ ನಿರ್ದಿಷ್ಟ ರೂಪವಿಲ್ಲದಿದ್ದರೂ (ಅನಿರ್ದೇಶ್ಯ\-ವಪುಃ) ಭಕ್ತನಿಗೆ ಇಷ್ಟವಾದ ರೂಪದಲ್ಲಿ ಕಾಣಿಸಿಕೊಳ್ಳಬಲ್ಲ.

ಯಾವ ಸ್ಥಿತಿಯಲ್ಲೇ ಇರಲಿ ಅವನ ಸ್ಮರಣೆಯು ಮನುಷ್ಯರ ಒಳ ಹೊರಗನ್ನು ಪವಿತ್ರ\-ಗೊಳಿ\-ಸುವುದು.


\section*{‘ಮಹಾಘನ’ದ ಮಾಹಾತ್ಮ್ಯೆ}

\addsectiontoTOC{‘ಮಹಾಘನ’ದ ಮಾಹಾತ್ಮ್ಯೆ}

\begin{verse}
ಹಗಲು ನಾಲ್ಕು ಜಾವ ಅಶನಕ್ಕೆ ಕುದಿವರು\\
 ಇರುಳು ನಾಲ್ಕು ಜಾವ ವ್ಯಸನಕ್ಕೆ ಕುದಿವರು\\
 ಅಗಸ ನೀರೊಳಗಿರ್ದೂ ಬಾಯಾರಿ ಸತ್ತಂತೆ\\
 ತಮ್ಮೊಳಗಿರ್ದ ಮಹಾಘನವರಿಯರು\\
 ಚೆನ್ನಮಲ್ಲಿಕಾರ್ಜುನ
\end{verse}

\hfill–ಅಕ್ಕಮಹಾದೇವಿ

ಹೊರಜಗತ್ತಿನ ಹಿನ್ನೆಲೆಯಲ್ಲಿರುವ ಆ ಮೂಲತತ್ತ್ವವನ್ನು ಶೋಧಿಸಿದಾಗ ಯಾವ ಸತ್ಯ ಅಥವಾ ತತ್ತ್ವ ಗೋಚರಿಸಿತೋ ಅದೇ ಸತ್ಯ ಅಂತರ್ಜಗತ್ತಿನ ಆಳವನ್ನು ಶೋಧಿಸಿದಾಗಲೂ ಕಂಡುಬಂದಿತು. ವ್ಯಕ್ತಿಯ ಹಿನ್ನೆಲೆಯ ಆಳದಲ್ಲಿರುವ ತಳಪಾಯದ ತತ್ತ್ವವೂ ವಿಶ್ವ ಬ್ರಹ್ಮಾಂಡದ ಹಿನ್ನೆಲೆಯ ಆಳದಲ್ಲಿರುವ ತಳಪಾಯದ ತತ್ತ್ವವೂ ಒಂದೇ ಎಂಬುದು ಅನುಭಾವಿಗಳ ಅಂತಿಮ ನಿರ್ಣಯ. ಅದೇ ಇಲ್ಲಿ ಧ್ವನಿತ.

ಎಲ್ಲಕ್ಕೂ ಮೂಲವಾದ ಪರಮಾಣುಲೀಲೆಯನ್ನು ವಿವರಿಸುವ ಕ್ಷೇತ್ರ ಸಿದ್ಧಾಂತವನ್ನರಿತರೆ ವಿಜ್ಞಾನದ ಆಗುಹೋಗುಗಳ ಜಾಡೇ ತಿಳಿಯುತ್ತದೆ ಎನ್ನುತ್ತಾರೆ ವಿಜ್ಞಾನಿಗಳು.

ಇದೆಲ್ಲವೂ ಮೂಲತಃ ಆತ್ಮವೇ ಆಗಿರುವುದರಿಂದ ಹೊರಗಿನ ಎಲ್ಲವನ್ನೂ ಬಿಡಿಬಿಡಿಯಾಗಿ ಅರಿಯುವ ಕಷ್ಟಸಾಧ್ಯವೂ ವ್ಯರ್ಥಪ್ರಯಾಸಕರವೂ ಆದ ಸಾಹಸ ಮಾಡುವುದಕ್ಕಿಂತ ತನಗೆ ಅತ್ಯಂತ ನಿಕಟವಾಗಿರುವ ಆತ್ಮನನ್ನು ತನ್ನಲ್ಲೇ ತಿಳಿದುಕೊಂಡರೆ ಎಲ್ಲವನ್ನೂ ತಿಳಿದುಕೊಂಡಂತೆ ಆಗುವುದು ಎಂದು ಅನುಭಾವಿಗಳು ಸಾರುತ್ತಾರೆ.

ನಾವು ಅನುಭಾವಿಗಳಾಗಬೇಕಿಲ್ಲ. ಅನುಭಾವಿಗಳ ಅನುಭವವಾಣಿಯನ್ನು ನಮ್ಮ ವ್ಯಾವ\-ಹಾರಿಕ ಜೀವನದಲ್ಲಿ ಪೂರ್ಣರೀತಿಯಿಂದ ಉಪಯೋಗಿಸಿಕೊಂಡು ಯಶಸ್ವಿಗಳಾಗುವ ವಿಧಾನವನ್ನು ಅರಿಯೋಣ.

‘ಎಲ್ಲ ಇದೆ, ಎಲ್ಲ ಇದೆ ನಿತ್ಯತೆಯ ಗಬ್ಬದಲಿ’, ‘ಸರ್ವವ್ಯಾಪಿಯಾದ ಆ ನಿತ್ಯತೆ ಅಥವಾ ನಿತ್ಯ ಸತ್ಯ ಎಲ್ಲರಲ್ಲೂ ಇದೆ. ಅದು ಎಲ್ಲ ಶಕ್ತಿಗಳ ಮೂಲ.’ ನಮ್ಮ ಅಭ್ಯುದಯ ಅಭಿವೃದ್ಧಿಗೆ ಬೇಕಾದ ಎಲ್ಲ ಶಕ್ತಿಗಳೂ ನಮ್ಮ ಆಂತರ್ಯದ ಆಳದಲ್ಲಿವೆ. ನಮಗೆ ಅಸಾಧ್ಯವೆನಿಸುವುದು ಯಾವುದೂ ಇಲ್ಲ. ಶ್ರದ್ಧೆ ಆ ಶಕ್ತಿಯನ್ನು ಎಚ್ಚರಿಸುತ್ತದೆ.

ಸತ್ಯವು ಪವಿತ್ರ, ಸತ್ಯವು ಜ್ಞಾನಮಯ, ಸತ್ಯವು ಸರ್ವವನ್ನೂ ಬೆಳಗುವಂಥದು. ಶಕ್ತಿ ಸ್ಥೈರ್ಯಗಳನ್ನು ನೀಡುವಂಥದು. ಅದು ನಮ್ಮ ಒಳಗೊಳಗೇ ಇದೆ. ಶ್ರದ್ಧೆಯೊಂದೇ ಅದನ್ನು ಪ್ರಕಟ ಗೊಳಿಸಲು ಸಹಾಯಮಾಡುತ್ತದೆ.

‘ಅತ್ಯಂತ ಶಕ್ತಿಶಾಲಿಯಾದ, ಜೀವನ ಸಂಜೀವಿನಿಯಾದ ಈ ಮಹಾಸಂದೇಶವನ್ನು ಜಗತ್ತು ನಮ್ಮಿಂದ ನಿರೀಕ್ಷಿಸುತ್ತಿದೆ. ಎಲ್ಲಕ್ಕಿಂತ ಮಿಗಿಲಾಗಿ ಭರತಖಂಡದ ಮೂಕ ಜನಕೋಟಿಯ ಉದ್ಧಾರಕ್ಕೆ ಇದು ಬೇಕಾಗಿದೆ. ಈ ಮಹಾಸತ್ಯದ ವ್ಯಾವಹಾರಿಕ ಪ್ರಯೋಜನವನ್ನು ಪಡೆಯಲು ಪ್ರಾಮಾಣಿಕ ಪ್ರಯತ್ನವನ್ನು ಮಾಡುವವರೆಗೂ ನಾವು ನಮ್ಮ ಮಾತೃಭೂಮಿಯನ್ನು ಮೇಲೆತ್ತ ಲಾರೆವು. ಈ ಸಿದ್ಧಾಂತವನ್ನು ತಿಳಿದುಕೊಳ್ಳುವುದರಿಂದ ಮಾತ್ರ ಪ್ರಚಂಡ ಇಚ್ಛಾಶಕ್ತಿಯನ್ನು ಸೃಷ್ಟಿಸಬಹುದು.’\footnote{ ಸ್ವಾಮಿ ವಿವೇಕಾನಂದರು.}

ಸ್ವಾತಂತ್ರ್ಯ ಪೂರ್ವದ ಹೋರಾಟದಲ್ಲಿ ದೇಶಪ್ರೇಮಿಗಳ ಹೃದಯದಲ್ಲಿ ವಿದ್ಯುತ್ ಸಂಚಾರ ಮಾಡಿಸಿದ ಸ್ಫೂರ್ತಿಯ ಕಿಡಿಯನ್ನು ಹೊತ್ತಿಸಿದ ತ್ಯಾಗ, ಸೇವೆಗಳಿಗೆ ಜನರನ್ನು ದೀಕ್ಷಾಬದ್ಧರಾಗು ವಂತೆ ಪ್ರೇರಿಸಿದ ವಿವೇಕಾನಂದರ ವಾಣಿಯಲ್ಲಿ ತಿರುತಿರುಗಿ ಅನುರಣಿತವಾಗುವ ಒಂದು\break ಬೋಧನೆಯೇ ನಮ್ಮ ಜನಮನದಲ್ಲಿ ಆತ್ಮವಿಶ್ವಾಸದಿಂದ ಎಚ್ಚರಗೊಳಿಸಬಹುದಾದ ಪ್ರಚಂಡ\break ಇಚ್ಛಾಶಕ್ತಿ ಮತ್ತು ಕಾರ್ಯಶಕ್ತಿಯ ವಿಚಾರ. ತನ್ನ ದಿವ್ಯತೆಯಲ್ಲಿ, ತನ್ನಲ್ಲಿ ಅಡಗಿರುವ ಅಪಾರ ಶಕ್ತಿಯಲ್ಲಿ ಮೊದಲು ಮಾನವ ವಿಶ್ವಾಸವಿಟ್ಟಿದ್ದಾದರೆ ಆತ ಅದ್ಭುತವನ್ನು ಸಾಧಿಸಬಹುದು. ತನ್ನ ಸ್ವವ್ಯಕ್ತಿತ್ವ ಚಿತ್ರದಲ್ಲಿ ಶೈಶವದಿಂದಲೇ ಅಂಟಿಕೊಂಡು ಬಂದ ನಿಷೇಧಾತ್ಮಕ ಭಾವನೆಗಳನ್ನೂ ದ್ವೇಷದೌರ್ಬಲ್ಯಗಳ ಸ್ಮೃತಿಯನ್ನೂ ಕಿತ್ತೆಸೆಯಬಹುದು. ಈ ಆತ್ಮವಿಶ್ವಾಸದಿಂದ ಶಕ್ತಿ, ವೈಭವ, ಪವಿತ್ರತೆ, ಏಕಾಗ್ರತೆ, ಸ್ಥೈರ್ಯವೇ ಮೊದಲಾದ ಬದುಕನ್ನು ಬೆಳಗುವ ಗುಣಗಳೆಲ್ಲವೂ ಬರುವುವು. ಮದರಾಸಿನ ಒಂದು ಪ್ರಮುಖ ಉಪನ್ಯಾಸದಲ್ಲಿ ಅವರು ನುಡಿದ ಸಿಡಿಲನುಡಿಯಲ್ಲಿ ಅಡಗಿರುವ ಈ ಮಹಾ\-ಸಂದೇಶದ ಅರ್ಥವ್ಯಾಪ್ತಿಯನ್ನು ಎಲ್ಲರೂ, ಮುಖ್ಯವಾಗಿ ಯುವಕರು ಮನನ ಮಾಡಬೇಕು–

“ಶಾಶ್ವತ ನಿಯಮಗಳನ್ನು ಆಧರಿಸಿ ನಿಂತ ನಮ್ಮ ಧರ್ಮದಲ್ಲಡಗಿರುವ ಮಹಾ ಸತ್ಯಗಳನ್ನು ಶ‍್ರೀಸಾಮಾನ್ಯರಿಗೂ ಬೋಧಿಸಿ. ಶತಶತಮಾನಗಳಿಂದ ಜನರಿಗೆ ಅಧಃಪತನಕ್ಕೆ ಕಾರಣವಾದ ಸಿದ್ಧಾಂತಗಳನ್ನೇ ಬೋಧಿಸಲಾಗಿದೆ. ಅವರು ಏನೂ ಅಲ್ಲ, ಅಪ್ರಯೋಜಕ ವ್ಯಕ್ತಿಗಳು ಎಂದು ಹೇಳುತ್ತ ಬಂದಿದ್ದಾರೆ. ಜಗತ್ತಿನಾದ್ಯಂತ ಎಲ್ಲೆಡೆಯಲ್ಲೂ ಇದೇ ಕತೆ. ದೀನದಲಿತರನ್ನು ಮನುಷ್ಯರೇ ಅಲ್ಲವೆಂಬಂತೆ ನೋಡಲಾಗಿದೆ. ಶತಮಾನಗಳಿಂದ ಅವರನ್ನು ಹೆದರಿಸಿ ಗದರಿಸಿ ಪ್ರಾಣಿ ಮಟ್ಟದ ಕನಿಷ್ಠ ಸ್ಥಿತಿಗೆ ಕೊಂಡೊಯ್ಯಲಾಗಿದೆ. ಅವರು ಈ ಆತ್ಮನ ವೈಭವವನ್ನು ಎಂದೂ ಕೇಳಿರಲಿಲ್ಲ. ಅವರೆಲ್ಲರೂ ಈ ಆತ್ಮ ವಿಚಾರವನ್ನು ಕೇಳುವಂತಾಗಲಿ, ಅತ್ಯಂತ ನಿಕೃಷ್ಟ ಸ್ಥಿತಿಗಿಳಿದವರೂ ಇದನ್ನು ಕೇಳಲಿ. ತಮ್ಮನ್ನು ನೀಚಾತಿನೀಚರೆಂದು ತಿಳಿದುಕೊಂಡವರಲ್ಲೂ ಈ ದೈವೀ ತೇಜಸ್ಸಿದೆ, ಆತ್ಮವಿದೆ. ಬೆಂಕಿ ಸುಡದ, ನೀರು ತೋಯಿಸದ, ಖಡ್ಗ ತುಂಡರಿಸದ, ನಾಶರಹಿತ ಚೈತನ್ಯವಿದೆ. ಸಾವಿಲ್ಲದ ನೋವಿಲ್ಲದ ಸರ್ವಶಕ್ತವಾದ ಆದ್ಯಂತರಹಿತವಾದ ಪರಮಪವಿತ್ರವಾದ ಸರ್ವಜ್ಞವೂ ಸರ್ವವ್ಯಾಪಿಯೂ ಆದ ಆತ್ಮವಿದೆ. ಅವರೆಲ್ಲರೂ ತಮ್ಮಲ್ಲಿ ತಾವು ಶ್ರದ್ಧೆಯನ್ನು ಬೆಳೆಸಿಕೊಳ್ಳಲಿ. ಆಂಗ್ಲರಿಗೂ ನಿಮಗೂ ಇರುವ ವ್ಯತ್ಯಾಸಕ್ಕೆ ಕಾರಣವೇನು? ಅವರು ತಮ್ಮ ಹಕ್ಕು, ಕರ್ತವ್ಯ, ರಾಜನೀತಿ ಮುಂತಾದವುಗಳನ್ನು ಕುರಿತು ಮಾತನಾಡಲಿ, ನಾನು ನಿಜವಾದ ವ್ಯತ್ಯಾಸ ಏನೆಂಬು\-ದನ್ನು ಕಂಡುಕೊಂಡಿದ್ದೇನೆ. ‘ನಾನೊಬ್ಬ ಆಂಗ್ಲ ಏನನ್ನೂ ಸಾಧಿಸಬಲ್ಲೆ’ ಎಂದವನು ದೃಢವಾಗಿ ನಂಬಿದ್ದಾನೆ, ಆ ತೆರನಾದ ಆತ್ಮಶ್ರದ್ಧೆ ಆಗಲೇ ಒಳಗಡೆ ಇರುವ ದೈವೀಶಕ್ತಿಯನ್ನು ಹೊರಹೊಮ್ಮುವಂತೆ ಮಾಡುವುದು. ಆದರೆ ಈ ದೇಶದಲ್ಲಿ ನೀವುಗಳೆಲ್ಲ ಶುದ್ಧ ಅಪ್ರಯೋಜಕ ವ್ಯಕ್ತಿಗಳೆಂದು ತರಬೇತನ್ನು ಪಡೆದುಕೊಂಡು ಹಾಗೆಯೇ ಆಗುತ್ತಿದ್ದೀರಿ. ನಮಗೆ ಶಕ್ತಿಬೇಕು. ಆದುದರಿಂದ ಪ್ರತಿಯೊಬ್ಬರೂ ಮೊದಲು ಈ ಆತ್ಮವಿಶ್ವಾಸದ ಮಹಾ ಪಾಠವನ್ನು ಕಲಿಯಲಿ.”


\section*{ನಂಬಿಕೆಯ ನಾಕ ನರಕ}

\addsectiontoTOC{ನಂಬಿಕೆಯ ನಾಕ ನರಕ}

{\parfillskip=0pt ಎರಡು ಮೂರು ವರ್ಷಗಳ ಹಿಂದೆ ಪತ್ರಿಕೆಯಲ್ಲಿ ಒಂದು ಕುತೂಹಲಕಾರಿ ಘಟನೆಯ ವರದಿ\-\par}\newpage\noindent ಯಾಗಿತ್ತು. ದೆಹಲಿಯ ಸಮೀಪದ ಹಳ್ಳಿಯ ಹೊಲದಲ್ಲಿ ದುಡಿಯುವ ಕೆಲಸಗಾರನೊಬ್ಬ ಆಸ್ತಮಾ ರೋಗದಿಂದ ನರಳುತ್ತಿದ್ದ. ರೋಗ ಉಲ್ಬಣಿಸಿದಾಗ ನಗರಕ್ಕೆ ಹೋಗಿ ಒಂದು ಪ್ರಸಿದ್ಧ ಆಸ್ಪತ್ರೆಯ ಡಾಕ್ಟರುಗಳನ್ನು ಕಂಡು ತನ್ನನ್ನು ಪರೀಕ್ಷಿಸಿ ಔಷಧ ನೀಡಬೇಕೆಂದು ಬೇಡಿಕೊಂಡ. ಡಾಕ್ಟರ್ ಒಬ್ಬರು ತಮ್ಮ ಹೆಸರು ಪದವಿಗಳನ್ನು ಅಚ್ಚಿಸಿದ ದೊಡ್ಡ ಕಾಗದದ ಮೇಲೆ ಔಷಧವನ್ನು ಹೆಸರಿಸಿ ‘ಇದನ್ನು ಹದಿನಾಲ್ಕು ಪಾಲು ಮಾಡಿ ಬೆಳಿಗ್ಗೆ ಮತ್ತು ಸಂಜೆ ಸೇವಿಸಬೇಕು. ಒಂದು ವಾರದ ಬಳಿಕ ತಿರುಗಿಬಂದು ನಿನ್ನ ಆರೋಗ್ಯ ಸುಧಾರಿಸಿದ ವಿಚಾರ ತಿಳಿಸಬೇಕು’ ಎಂದರು. ರೋಗಿಯು ಡಾಕ್ಟರ್ ಹೇಳಿದುದನ್ನು ಕಟ್ಟುನಿಟ್ಟಾಗಿ ಅನುಸರಿಸಿದ. ಒಂದು ವಾರದ ಬಳಿಕ ಡಾಕ್ಟರನ್ನು ಕಾಣುತ್ತಲೇ ಅವರನ್ನು ನಮಸ್ಕರಿಸಿ ‘ನೀವು ಕೊಟ್ಟ ಔಷಧ ರಾಮಬಾಣದ ಹಾಗೆ ಕೆಲಸ ಮಾಡಿತು. ಆಸ್ತಮಾದ ನರಳಾಟದಿಂದ ನನ್ನನ್ನು ರಕ್ಷಿಸಿದಿರಿ, ನಿಮಗೆ ಎಷ್ಟು ಧನ್ಯವಾದ ಹೇಳಿದರೂ ಕಡಿಮೆ’ ಎಂದ. ಡಾಕ್ಟರ್ ತಾವು ಕೊಟ್ಟ ಔಷಧಸೂಚಿಯನ್ನು ತೋರಿಸೆಂದು ಹೇಳಿದಾಗ ಅವರಿಗೊಂದು ಆಶ್ಚರ್ಯ ಕಾದಿತ್ತು. ಅವನು ಡಾಕ್ಟರ್ ಕೊಟ್ಟ ಕಾಗದವನ್ನೇ ಹದಿನಾಲ್ಕು ಪಾಲು ಮಾಡಿ ಪ್ರತಿ ದಿನವೂ ಬೆಳಿಗ್ಗೆ ಮತ್ತು ಸಂಜೆ ಸೇವಿಸಿದ್ದನಂತೆ. ಡಾಕ್ಟರು ಚಕಿತರಾಗಿ ಕ್ಷಣಕಾಲ ಮುಗುಳ್ನಗುತ್ತ ‘ಸರಿ ಒಳಿತಾಯಿತು’ ಎಂದು ಅವನ ನಂಬಿಕೆಗೆ ಆಘಾತವನ್ನು ಮಾಡದೆ ಅವನ ರೋಗ ಗುಣವಾದುದನ್ನು ದೃಢಪಡಿಸಿಕೊಂಡು ಅವನನ್ನು ಊರಿಗೆ ಕಳಿಸಿದರು.

ಇದು ನಡೆದ ಘಟನೆ. ಕಲ್ಪನೆಯ ಕಗ್ಗವಲ್ಲ. ವಿರಳವಾದರೂ ಇಂಥ ಘಟನೆಗಳ ಹಿನ್ನೆಲೆಯಲ್ಲಿ ಕೆಲಸ ಮಾಡುವ ಶಕ್ತಿ ಅಥವಾ ನಿಯಮ ಯಾವುದು? ಸಂಶಯರಹಿತವಾದ ದೃಢನಂಬಿಕೆಯಲ್ಲವೇ? ‘ಈ ರೋಗಕ್ಕೆ ಡಾಕ್ಟರ್ ಔಷಧ ನೀಡಿದ್ದಾರೆ. ಖಂಡಿತವಾಗಿ ನನ್ನ ರೋಗ ವಾಸಿಯಾಗುವುದು’ ಎಂಬ ನಂಬಿಕೆ. ಆ ದೃಢನಂಬಿಕೆ ಎಂಥ ಕೆಲಸ ಮಾಡಿತು! ರೋಗ ನಿರೋಧಕ ಶಕ್ತಿಯನ್ನೇ ದೇಹದಲ್ಲಿ ಉತ್ಪಾದಿಸಿತು.

ನಂಬಿಕೆಯ ಇನ್ನೊಂದು ಮುಖ ನೋಡಿ: ಹಲವು ವರ್ಷಗಳ ಹಿಂದೆ ಕೊಲೆ ಮಾಡಿದ ಕಾರಣಕ್ಕಾಗಿ ನೇಣು ಶಿಕ್ಷೆ ವಿಧಿಸಲ್ಪಟ್ಟ ವ್ಯಕ್ತಿಯೊಬ್ಬನನ್ನು ಮನೋವಿಜ್ಞಾನಿಗಳು ತಮ್ಮ ಒಂದು ಸಂಶೋಧನೆಗಾಗಿ ನ್ಯಾಯಾಧಿಕಾರಿಗಳ ಅನುಮತಿಯನ್ನು ಪಡೆದು ತಮ್ಮ ಪ್ರಯೋಗಾಲಯಕ್ಕೆ ಕರೆ ತಂದರು. ಅವನ ಜೀವವನ್ನು ಉಳಿಸಲು ಅಲ್ಲಿಗೆ ಅವನನ್ನು ಅವರು ಕರೆತಂದುದಲ್ಲ. ನಂಬಿಕೆಯು ಯಾವುದೇ ಮನುಷ್ಯನ ಮನಸ್ಸಿನಮೇಲೆ ಹೇಗೆ ಪರಿಣಾಮ ಬೀರುವುದು ಎಂಬುದನ್ನು ತಿಳಿಯಲು ಒಂದು ಪ್ರಯೋಗವನ್ನು ಮಾಡಲು ಯೋಚಿಸಿಕೊಂಡಿದ್ದರು. ಆತನ ಕಣ್ಣಿಗೆ ಬಟ್ಟೆಯನ್ನು ಸುತ್ತಿ ಒಂದು ಮೇಜಿನ ಮೇಲೆ ಮಲಗಿಸಿದ್ದರು. ಅವನು ಎದ್ದು ಓಡಿಹೋಗದಂತೆ ಕಾಲುಗಳನ್ನು ಬಟ್ಟೆಗಳಿಂದ ಕಟ್ಟಿಹಾಕಿದ್ದರು. ಅವನ ಕತ್ತಿನ ಸಮೀಪದಲ್ಲಿ ಒಂದು ಟ್ಯೂಬನ್ನು ಇಟ್ಟಿದ್ದು, ಟ್ಯೂಬಿನಿಂದ ಕೆಳಗೆ ಸುರಿಯಲ್ಪಡುವ ರಕ್ತ(?)ವನ್ನು ಶೇಖರಿಸಲು ಒಂದು ಪಾತ್ರೆಯನ್ನು ಇಟ್ಟಿದ್ದರು. ಹಿರಿಯ ತಜ್ಞನೊಬ್ಬ ಮಲಗಿದ್ದ ವ್ಯಕ್ತಿಯನ್ನು ಕುರಿತು ಹೀಗೆಂದ: ‘ಮನುಷ್ಯನ ಶರೀರದಿಂದ ಎಷ್ಟು ಪ್ರಮಾಣದ ರಕ್ತವನ್ನು ತೆಗೆದರೆ ಅವನು ಸಾಯುತ್ತಾನೆಂಬುದನ್ನು ಪರೀಕ್ಷಿಸಲು ನಾವು ಈ ಪ್ರಯೋಗ ಮಾಡುತ್ತಿದ್ದೇವೆ. ಸುಮಾರು ಮೂರೂವರೆ ಲೀಟರ್ ರಕ್ತವನ್ನು ದೇಹದಿಂದ ತೆಗೆದಾಗ ಮನುಷ್ಯ ಸಾಯುತ್ತಾನೆ ಎನ್ನುತ್ತಾರೆ. ನಿನಗೆ ನೋವಾಗದಂತೆ ಈ ಕೆಲಸ ಮಾಡುತ್ತೇವೆ. ಗಲ್ಲಿಗೇರಿ ಮರಣ ಹೊಂದುವುದಕ್ಕಿಂತ ಇದು ಎಷ್ಟೋ ಸುಲಭದ ಸಾವಾಗಿ ಪರಿಣಮಿಸುತ್ತದೆ. ನೋವು, ನರಳಾಟ ತಪ್ಪುತ್ತದೆ.’ ಆತ ಸುಮ್ಮನೆ ಯಾವ ಪ್ರತಿಕ್ರಿಯೆಯನ್ನೂ ತೋರದೆ ಮಲಗಿಕೊಂಡಿದ್ದ. ಆತನ ಎಡಕತ್ತಿನಲ್ಲಿ ರಕ್ತನಾಳದ ಮೇಲೆ ಬ್ಲೇಡಿನಿಂದ ಗಾಯವನ್ನು ಮಾಡಿದರು. ಮೊದಲು ಹನಿಹನಿಯಾಗಿ, ಆ ಬಳಿಕ ಧಾರೆಯಾಗಿ ರಕ್ತ(?) ಸುರಿಯಲು ಪ್ರಾರಂಭವಾಯಿತು. ಆತನ ಮುಖ ವಿವರ್ಣವಾದುದನ್ನು ಗುರುತಿಸಬಹುದಿತ್ತು. ಸ್ವಲ್ಪ ಹೊತ್ತಿನಲ್ಲಿ ದ್ರವ ಪದಾರ್ಥ ಟ್ಯೂಬಿನಲ್ಲಿ ಹರಿದು ಪಾತ್ರೆಗೆ ಬೀಳುವ ಸದ್ದು ಹೆಚ್ಚಿತು. ಅಲ್ಲಿ ನೆರೆದ ವಿಜ್ಞಾನಿಗಳು ಈಗ ರಕ್ತದ ಪ್ರಮಾಣ ‘ಮೂರು ಲೀಟರ್ ಆಯಿತು’ ಎಂದರು. ಮಲಗಿದ್ದವನು ಮೇಲುಸಿರು ಬಿಡುತ್ತಿದ್ದ. ‘ಈಗ ಮೂರೂವರೆಲೀಟರ್​’ ಎಂದಾಗ ಅವನ ಉಸಿರಾಟ ನಿಂತಿತ್ತು. ನಿಜವಾಗಿಯೂ ಟ್ಯೂಬಿನ ಮೂಲಕ ಮೂರೂವರೆ ಲೀಟರ್ ನೀರು ಮಾತ್ರ ಶೇಖರವಾಗಿತ್ತು. ಗಾಯದಿಂದ ರಕ್ತವು ಹರಿದಿರಲಿಲ್ಲ. ಆದರೆ ಶರೀರದಿಂದ ರಕ್ತ ಧಾರೆಧಾರೆಯಾಗಿ ಹೋಗುತ್ತಿದೆ ಎಂದು ಸಂಶಯರಹಿತ ದೃಢನಂಬಿಕೆಯಿಂದ ಇದ್ದ ಆತ ದೇಹದಿಂದ ಅಷ್ಟೊಂದು ರಕ್ತ ಹೋದ ಪರಿಣಾಮವನ್ನೇ ಹೊಂದಿದ.


\section*{ಕೇವಲ ಆಕಸ್ಮಿಕವೇ?}

\addsectiontoTOC{ಕೇವಲ ಆಕಸ್ಮಿಕವೇ?}

ಮೇಲಿನ ಉದಾಹರಣೆಯಲ್ಲಿ ಸೂಚಿಸಿರುವಂಥ ಘಟನೆಗಳು ಕೆಲವೊಮ್ಮೆ ಆಕಸ್ಮಿಕವಾಗಿ ನಡೆಯುವುದುಂಟು. ಅದೇನು ಮಹಾ? ಅದರಿಂದ ನಾವು ಒಂದು ಸಾಮಾನ್ಯ ನಿಯಮವನ್ನು ಕಂಡು\-ಹಿಡಿಯು\-ವಂತಿಲ್ಲ ಎನ್ನಬಹುದು ನೀವು. ನಂಬಿಕೆಯಿಂದ ರೋಗ ಗುಣವಾದ ಘಟನೆಗಳು, ಕಂಡುಹಿಡಿಯುವಂತಿಲ್ಲ ಎನ್ನಬಹುದು ನೀವು. ನಂಬಿಕೆಯಿಂದ ರೋಗ ಗುಣವಾದ ಘಟನೆಗಳು, ಭಯ ಸಂಶಯಗಳಿಂದ ಅಪಾಯವಾದ ಘಟನೆಗಳು ಅನಾದಿಕಾಲದಿಂದಲೂ ನಡೆದಿರುವಂಥವುಗಳೇ. ಆದರೆ ಔಷಧಗಳ ಗುಣ ಪರಿಣಾಮ ಪ್ರಭಾವಗಳನ್ನು ಅಲ್ಲಗಳೆಯಲು ಸಾಧ್ಯವೇ? –ಎನ್ನುತ್ತೀರಿ ನೀವು. ಆಧುನಿಕ ಯುಗದ ಸಂಶೋಧನೆಗಳ ಮೂಲಕ ಬೇರೆ ಬೇರೆ ರಾಸಾಯನಿಕ ದ್ರವ್ಯಗಳು ಶರೀರದ ಮೇಲೆ ಉಂಟುಮಾಡುವ ಪರಿಣಾಮಗಳನ್ನು ಪ್ರಯೋಗಗಳಿಂದ ಕರಾರು ವಾಕ್ಕಾಗಿ ತಿಳಿಯಲಾಗಿದೆ. ಈ ‘ದ್ರವ್ಯಗುಣ’ವನ್ನು ಇಲ್ಲವೆನಿಸುವುದು ಅಸಾಧ್ಯ ಎಂಬ ಸಂಶಯ ಸರಿಯಾದುದೆ. ಆದರೆ ನಂಬಿಕೆಯ ಶಕ್ತಿಯನ್ನು ಕುರಿತ ಸಂಶೋಧನೆಗಳೇ ಇಲ್ಲಿ ನಮ್ಮ ಸಹಾಯಕ್ಕೆ ಬರುತ್ತಿವೆ ಎಂಬುದನ್ನು ಮುಂದಿನ ವಾಕ್ಯದಲ್ಲಿ ನೀವು ಕಾಣಬಹುದು.

ಅಮೇರಿಕದ ಬಾಸ್ಟನ್ ನಗರದ ಡಾ.\ ಬೆನ್​ಸನ್ ನಲವತ್ತು ವರ್ಷಗಳ ಕಾಲ ನಡೆಯಿಸಿದ ಸಂಶೋಧನೆಯ ಸಂಕ್ಷಿಪ್ತ ವರದಿಯನ್ನು ೧೯೭೯ನೆಯ ಜೂನ್ ೨೬ನೇ ದಿನಾಂಕದ ಮದರಾಸಿನ ‘ದಿ ಹಿಂದೂ’ ಪತ್ರಿಕೆಯಲ್ಲಿ ಪ್ರಕಟಿಸಿದ್ದರು. ಡಾಕ್ಟರುಗಳ ವ್ಯಕ್ತಿತ್ವ, ಅವರು ನೀಡುವ ಸಲಹೆ ಸೂಚನೆಗಳ ಪ್ರಭಾವ, ರೋಗಿಗಳು ಡಾಕ್ಟರುಗಳಲ್ಲಿ ಇರಿಸಿದ ವಿಶ್ವಾಸ–ಇವುಗಳ ಮಹತ್ವವನ್ನು ಕುರಿತ ಪ್ರಯೋಗಾತ್ಮಕ ಅಧ್ಯಯನದ ವರದಿ ಅದು. ಯಾವ ದ್ರವ್ಯಗುಣವಿಲ್ಲದ ಔಷಧಿಗಳನ್ನು ಸೇವಿಸಿಯೂ ‘ಆ ಮಾತ್ರೆ ಬಹಳ ಒಳ್ಳೆಯ ಕೆಲಸ ಮಾಡುತ್ತದೆ’ ಎಂಬ ಡಾಕ್ಟರುಗಳ ಮಾತನ್ನೇ ನಂಬಿ ರೋಗಮುಕ್ತರಾದವರು ನೂರರಲ್ಲಿ ೮೦ ಮಂದಿ ಎಂದು ವರದಿ ಹೇಳುತ್ತದೆ. ಈ ನಿರುಪಯುಕ್ತ ಔಷಧಿಗಳನ್ನು ಸೇವಿಸಿಯೇ ಎದೆನೋವು, ಹೃದಯ ರೋಗಗಳಿಂದಲೂ ಪಾರಾಗಿದ್ದಾರೆ ಎಂಬುದು ವರದಿ ತಿಳಿಸುವ ಇನ್ನೊಂದು ತಥ್ಯ.

ವೈದ್ಯಕೀಯ ಉಪಚಾರಗಳಲ್ಲಿ ದೃಢನಂಬಿಕೆಯ ಸತ್​ಪರಿಣಾಮವನ್ನು ಕುರಿತ ಉಪಯೋಗ ಉಪಕಾರ ವೈದ್ಯರುಗಳಿಗೆ ಹಿಂದಿನಿಂದಲೂ ತಿಳಿದ ವಿಚಾರವೇ. ಆದರೆ ಸಂಶೋಧಕರು ‘ಈ ನಂಬಿಕೆಯ ಶಕ್ತಿಯಿಂದ ಉಂಟಾಗುವ ಉಪಕಾರ ತಾವು ನಿರೀಕ್ಷಿಸಿದ್ದಕ್ಕಿಂತ ಹೆಚ್ಚಾಗಿದೆ. ತಮ್ಮ ಊಹೆಗೆ ಮೀರಿ ಫಲ ನೀಡಿದೆ’ ಎನ್ನುತ್ತಾರೆ.

ಬಾಸ್ಟನ್ನಿನ ಬೆತ್ ಇಸ್ರೇಲ್ ಆಸ್ಪತ್ರೆಯ ಡಾ. ಹರ್ಬಟ್R ಬೆನ್​ಸನ್ ಅವರ ಸಂಶೋಧನೆಯನ್ನು ‘ನ್ಯೂ ಇಂಗ್ಲೆಂಡ್ ಜರ್ನಲ್ ಆಫ್ ಮೆಡಿಸನ್​’ನಲ್ಲಿ ಪ್ರಕಟಿಸಲಾಗಿದೆ. ಹೃದಯಕ್ಕೆ ಅತ್ಯಂತ ಕಡಿಮೆ ರಕ್ತ ತಲುಪುವ ಕಾರಣದಿಂದ ಬರುವ ಎದೆ ನೋವು ‘ಅಂಜೈನಾ’ ರೋಗಕ್ಕೆ ಮೊದಲು ಡಾಕ್ಟರುಗಳು ನೀಡುತ್ತಿದ್ದ ಔಷಧಿಗಳನ್ನು ಸಂಶೋಧಕರು ಪುನಃ ಪರಿಶೀಲಿಸಿದರು.

ಈಗಿನ ವೈದ್ಯಕೀಯ ಉಪಚಾರದ ಕ್ರಮದಂತೆ ಮೂರು ವಿಧದ ಔಷಧಗಳೂ ಎರಡು ಶಸ್ತ್ರ ಚಿಕಿತ್ಸೆಗಳೂ ಈ ರೋಗಕ್ಕೆ ಉಪಕಾರಿಯಾಗಲಿಲ್ಲ. ವ್ಯರ್ಥವೇ ಆಯಿತು. ಆದರೆ ಹಿಂದೆ ವೈದ್ಯರುಗಳು ನೀಡುತ್ತಿದ್ದ ಔಷಧಗಳ ಪರಿಣಾಮ ಉತ್ತಮವಾಗಿತ್ತು.

ಒಂದು ಸಾವಿರದ ನೂರ ಹದಿನೇಳು ರೋಗಿಗಳನ್ನು ಕುರಿತ ಅಧ್ಯಯನದಲ್ಲಿ ಸುಮಾರು ನೂರ ಹದಿಮೂರು ಕೇಸುಗಳನ್ನು ವೈದ್ಯರು ಪರಿಶೀಲಿಸಿದ್ದರು. ನಿಷ್ಪ್ರಯೋಜಕ ಔಷಧಗಳನ್ನು ಸೇವಿಸಿ ನೂರರಲ್ಲಿ ಸುಮಾರು ಎಂಬತ್ತೆರಡು ಮಂದಿ ಗುಣಮುಖರಾದರು.

ಔಷಧಕ್ಕಿಂತಲೂ ವೈದ್ಯರ ಉತ್ಸಾಹ, ಪ್ರೀತಿ ಮತ್ತು ಕಳಕಳಿ, ವಿಶ್ವಾಸಾರ್ಹ ನಡವಳಿಕೆ ರೋಗಿಗಳ ನೋವನ್ನು ಕಡಿಮೆ ಮಾಡಿದ್ದೆಂದು ಸಂಶೋಧಕರು ನಿರ್ಧರಿಸಿದರು. ಡಾ.\ ಬೆನ್​ಸನ್ ಶೇಕಡಾ ೮೨ರಷ್ಟು ನಂಬಿಕೆಯಿಂದಲೇ ಗುಣ ಹೊಂದಿದವರು ಎಂದರೆ ಬಹಳ ಹೆಚ್ಚಿನ ಸಂಖ್ಯೆ ಎಂದು ಆಶ್ಚರ್ಯದಿಂದ ಉದ್ಗರಿಸುತ್ತಾರೆ. ಆದರೆ ನಾಲ್ಕು ದಶಕಗಳಿಂದಲೂ ಬೇರೆ ಬೇರೆ ಕಡೆಗಳಲ್ಲಿ ನಡೆದ ಹಲವಾರು ಸಂಶೋಧನೆಗಳಿಂದ ‘ನಂಬಿಕೆಯ ಬಗೆಗಿನ ಈ ಸತ್ಯ’ ಒಂದೇ ರೀತಿಯಾಗಿರುವುದು ಕಂಡುಬಂದಿದೆ ಎನ್ನುತ್ತಾರೆ!

ಹದಿನಾರನೇ ಶತಮಾನದ ಜರ್ಮನ್ ದೇಶದ ಪ್ರಸಿದ್ಧ ವೈದ್ಯ ಪೆರಾಸಲ್ಸಸ್ ಎಂಬಾತ ನಂಬಿಕೆ\-ಯಿಂದಲೇ ಎಲ್ಲ ರೋಗಗಳನ್ನೂ ಗುಣಪಡಿಸಬಹುದು ಎಂದಿದ್ದನಂತೆ. ಆದರೆ ಎಲ್ಲರಿಗೂ ಈ ದೃಢನಂಬಿಕೆ ಸಾಧ್ಯವೆಂದಲ್ಲ. ಅಭ್ಯಾಸ ತರಬೇತಿಗಳಿಂದ ಅಸಾಧ್ಯವೆಂದೂ ಅಲ್ಲ. ಔಷಧಗಳ ಪ್ರಭಾವ ಇಲ್ಲವೆಂದೂ ಅಲ್ಲ. ಸಂಶಯರಹಿತ ದೃಢನಂಬಿಕೆಯ ಅದ್ಭುತ ಶಕ್ತಿಯನ್ನು ಕುರಿತು ಮಾತ್ರ ಇಲ್ಲಿ ಹೇಳಹೊರಟದ್ದು.


\section*{ವಶ್ಯಸುಪ್ತಿಯಲ್ಲಿ ಮನೋಲೀಲೆ}

\vskip -7pt\addsectiontoTOC{ವಶ್ಯಸುಪ್ತಿಯಲ್ಲಿ ಮನೋಲೀಲೆ}

ಕೆಲವೊಂದು ಯೋಚನೆಗಳನ್ನು ಮನುಷ್ಯರ ಒಳಮನಸ್ಸು ಸಂಪೂರ್ಣವಾಗಿ ಸ್ವೀಕರಿಸಿದರೆ, ಆ ಯೋಚನೆಗಳಿಗೆ ಸತ್ಯದ ಆಧಾರವಿರದೆ ಅವು ಕೇವಲ ಕಲ್ಪಿತವಾಗಿದ್ದರೂ ಮನಸ್ಸು ಯಥಾರ್ಥವನ್ನು ಕಂಡು ಪ್ರತಿಕ್ರಿಯೆ ತೋರಿಸುವಂತೆಯೇ ವರ್ತಿಸುತ್ತದೆ! ಈ ಸಂಗತಿಯನ್ನು ವಶ್ಯಸುಪ್ತಿ ಗೊಳಗಾದವರಲ್ಲೆಲ್ಲ ನಾವು ಸ್ಪಷ್ಟವಾಗಿ ಮನಗಾಣಬಹುದು. ಡಾ.\ ಸಾಲಿಗ್ಮ್ಯಾನ್ ತಮ್ಮ ಒಂದು ಬರಹದಲ್ಲಿ ಪ್ರಯೋಗಾತ್ಮಕ ಅನುಭವವನ್ನು ನೀಡಿದ್ದಾರೆ. ವಶ್ಯಸುಪ್ತಿಗೊಳಗಾದ ವ್ಯಕ್ತಿಯನ್ನು ಉದ್ದೇಶಿಸಿ ಕಾದ ಕಬ್ಬಿಣದಿಂದ ಅವನ ತೋಳನ್ನು ಮುಟ್ಟಲಾಗುತ್ತಿದೆ ಎಂದೂ, ಬೊಕ್ಕೆ ಏಳುವುದೆಂದೂ ಹೇಳಿದರು. ನಿಜವಾಗಿಯೂ ಡಾ.\ ಹೇಡ್ ಫೀಲ್ಡ್ ಆತನನ್ನು ಬೆರಳಿನಿಂದ ಮುಟ್ಟಿದ್ದು, ಕಾದ ಕಬ್ಬಿಣದಿಂದಲ್ಲ. ಆದರೆ ಅವನು ಸುಟ್ಟ ನೋವಿನಿಂದ ನರಳುವಂತೆಯೇ ನರಳಿದ. ಸುಪ್ತಸ್ಥಿತಿಯಲ್ಲಿದ್ದ ಆತನ ತೋಳಿಗೆ ಬ್ಯಾಂಡೇಜ್ ಕಟ್ಟಿದ್ದರು. ಆರು ಘಂಟೆಗಳ ನಂತರ ಬ್ಯಾಂಡೇಜನ್ನು\break ಬಿಚ್ಚಿದಾಗ ಡಾ.\ ಹೇಡ್ ಫೀಲ್ಡ್ ಮುಟ್ಟಿದ ಜಾಗದಲ್ಲಿ ಒಂದು ಬೊಕ್ಕೆ ಎದ್ದಿತ್ತು. ಮಧ್ಯೆ ಬಿಳಿಯ ಸತ್ತ ಚರ್ಮ. ಕೆಳಭಾಗದಲ್ಲಿ ಸ್ವಲ್ಪ ದ್ರವ ಕಾಣಿಸಿತು. ಬೊಕ್ಕೆಯು ಬೆಳೆಯುತ್ತ ಹೋಗಿ ಮರು ದಿನದ ಹೊತ್ತಿಗೆ ಬಹಳಷ್ಟು ನೀರಿನಂತಹ ದ್ರವ ಸೇರಿಕೊಂಡು ಶಾಖದಿಂದಲೇ ಉಂಟಾದ ಬೊಕ್ಕೆಯಂತೆ ಕಂಡುಬಂದಿತು. ಈ ಪ್ರಯೋಗವನ್ನೇ ತಜ್ಞರ ಎದುರಿನಲ್ಲಿ ಮತ್ತೊಮ್ಮೆ ಮಾಡಿ ನೋಡಿದಾಗಲೂ ಪರಿಣಾಮ ಒಂದೇ ಆಗಿತ್ತು.

ಸುಪ್ತಿಗೊಳಗಾದವನ ತೋಳನ್ನು ಕೆಂಪಾಗಿ ಕಾದ ಸ್ಟೀಲಿನ ತುಂಡಿನಿಂದ ಸ್ಪರ್ಶಿಸಿ, ಬೆರಳಿನಿಂದ ಮುಟ್ಟಿದ್ದೆಂದು ಹೇಳಿ ಏನೂ ನೋವಿರುವುದಿಲ್ಲವೆಂದು ಹೇಳಿದರು. ಚರ್ಮವನ್ನು ಮುಟ್ಟಿದಾಗಲೂ ಅನಂತರವೂ ನೋವಿನ ಯಾವ ಪ್ರತಿಕ್ರಿಯೆಯನ್ನೂ ಆತ ತೋರಲಿಲ್ಲ. ಹಿಂದಿನ ಬೊಕ್ಕೆಯಲ್ಲಿ ಕಂಡು ಬಂದ ರಕ್ತರಹಿತ ದ್ರವಾಂಶ ಅಲ್ಲಿ ತೋರಿಬರಲಿಲ್ಲ. ಕಾದ ಸ್ಟೀಲಿನ ತುಂಡಿ ನಿಂದ ಮುಟ್ಟಿದ ಜಾಗದಲ್ಲಿ ಒಂದು ಸಣ್ಣ ಕೆಂಪು ವೃತ್ತ ಉಂಟಾಗಿತ್ತು. ಅದು ನೋವನ್ನುಂಟು ಮಾಡಲಿಲ್ಲ. ಮಾತ್ರವಲ್ಲ ಬಹುಬೇಗನೇ ವಾಸಿಯೂ ಆಯಿತು!

ಭಾರ ಎತ್ತುವ ಪ್ರಸಿದ್ಧ ಚಾಂಪಿಯನ್ ಒಬ್ಬನನ್ನು ಸುಪ್ತ ನಿದ್ರೆಗೊಳಪಡಿಸಿ ‘ಎಷ್ಟು ಪ್ರಯತ್ನಿ\-ಸಿದರೂ ಮೇಜಿನ ಮೇಲಿರುವ ಪೆನ್ಸಿಲನ್ನು ಎತ್ತಲು ನಿನಗೆ ಸಾಧ್ಯವಿಲ್ಲ’ ಎಂಬ ಸೂಚನೆ ನೀಡಿದರು. ನಾನೂರು ಪೌಂಡ್ ಭಾರದ ವಸ್ತುವನ್ನು ಲೀಲಾಜಾಲವಾಗಿ ಎತ್ತುವ ಆತ ಪ್ರಯತ್ನಿಸಿ ಹೋರಾಡಿದರೂ ಈಗ ಒಂದು ಪೆನ್ಸಿಲನ್ನೂ ಎತ್ತಲಾರದಾದ. ಅವನ ಒಳಮನಸ್ಸು ತಜ್ಞನ ಸೂಚನೆಯನ್ನು ಸಂಪೂರ್ಣ ಸ್ವೀಕರಿಸಿ ಅಂತೆಯೇ ವರ್ತಿಸುವಂತೆ ಪ್ರಭಾವ ಬೀರಿತು!

ಹಾಗಾದರೆ ಸುಪ್ತಾವಸ್ಥೆಯಲ್ಲಿರುವ ಆತನು ದುರ್ಬಲನಾದನೆಂದರ್ಥವೇ? ಅಲ್ಲ. ಅವನಲ್ಲಿ ಮೊದಲಿನಷ್ಟೆ ಬಲ ಇತ್ತು. ಆದರೆ ಅದನ್ನು ಸರಿಯಾಗಿ ಅರಿತುಕೊಳ್ಳದೇ ತನಗೆ ತಾನೇ ವಿರೋಧವಾಗಿ ವರ್ತಿಸುವಂತಾಯಿತು! ಇತ್ತ ಸ್ವಶಕ್ತಿಯಿಂದ ನರಗಳನ್ನು ಬಿಗಿಗೊಳಿಸಿ ಪೆನ್ಸಿಲನ್ನು ಎತ್ತಲು ಪ್ರಯತ್ನಿಸುತ್ತಿರುತ್ತಾನೆ, ಆದರೆ ‘ಎತ್ತಲು ಸಾಧ್ಯವಿಲ್ಲ’ ಎಂಬ ಸೂಚನೆ ಪ್ರಬಲ ಪರಿಣಾಮಕಾರಿ ಯಾಗಿ ಕೆಲಸ ಮಾಡಿಬಿಡುತ್ತದೆ. ತನ್ನ ನಿಜವಾದ ಶಕ್ತಿಯನ್ನು ವ್ಯಕ್ತಗೊಳಿಸದಂತೆ ಇಲ್ಲಿ ಆ ನಿಷೇಧಾತ್ಮಕ ಸೂಚನೆ ಕೆಲಸಮಾಡಿ ಸೋಲನ್ನು ತಂದಿತೆಂಬುದನ್ನು ಗಮನಿಸಬೇಕು.

ಸಭೆಯಲ್ಲಿ ಮಾತನಾಡಲು ಅಂಜುವ, ಸಭಾಕಂಪದಿಂದ ನರಳುವ ವ್ಯಕ್ತಿಯನ್ನು ಸುಪ್ತಾವಸ್ಥೆಗೆ ಕೊಂಡೊಯ್ದು ‘ನೀನು ಚೆನ್ನಾಗಿ ಸಲೀಸಾಗಿ ಮಾತನಾಡುತ್ತಿ’ ಎಂಬ ಸೂಚನೆಯನ್ನು ತಜ್ಞ ನೀಡಿದ. ಸುಪ್ತಾವಸ್ಥೆಯಲ್ಲೇ ಎದ್ದುನಿಂತು ಆತ ನಿರ್ಭೀತಿಯಿಂದ ಮಾತನಾಡಿ ತಾನೂ ಚಕಿತನಾಗಿ ಎಲ್ಲರನ್ನೂ ಬೆರಗುಗೊಳಿಸಿದ. ಸಾಮಾನ್ಯ ಸ್ಥಿತಿಯಲ್ಲಿ ಜಟಿಲವಾದ ಗಣಿತದ ಸಮಸ್ಯೆಯೊಂದನ್ನು ಬಿಡಿಸಲು ಒಂದು ಗಂಟೆ ಬೇಕಿತ್ತು. ಸೂಚನೆಯ ನಂತರ ಒಬ್ಬಾತ ಅದನ್ನು ಇಪ್ಪತ್ತು ನಿಮಿಷಗಳಲ್ಲೇ ಮುಗಿಸಲು ಸಮರ್ಥನಾದ! ಈ ಮಾತುಗಾರಿಕೆಯಾಗಲಿ, ಗಣಿತಜ್ಞತೆಯಾಗಲಿ ವಶ್ಯಸುಪ್ತಿಯ ತಜ್ಞನನ್ನು ನೋಡುವ ಮೊದಲೇ ಆ ವ್ಯಕ್ತಿಗಳಲ್ಲಿತ್ತು. ಅದು ತಜ್ಞನ ಮಾಂತ್ರಿಕ ಶಕ್ತಿ ಅಥವಾ ಪವಾಡಶಕ್ತಿಯಾಗಿರಲಿಲ್ಲ. ತನ್ನಲ್ಲಡಗಿದೆ ಎಂಬ ನಂಬಿಕೆ ಆ ವ್ಯಕ್ತಿಗಳಲ್ಲಿರಲಿಲ್ಲ ಅಷ್ಟೆ. ನಿಷೇಧಾ ತ್ಮಕ ಭಾವನೆಗಳಿಂದ ಅದನ್ನು ಭದ್ರವಾಗಿ ಮುಚ್ಚಲಾಗಿತ್ತು. ತಜ್ಞನು ಆ ಮುಚ್ಚಳವನ್ನು ದೂರ ಸರಿಯುವಂತೆ ಮಾಡಿದ.

ಬಾಲ್ಯದಿಂದಲೇ ನಿಷೇಧಮಯ ತರಬೇತಿ, ವಾತಾವರಣಗಳಿದ್ದರೂ ಕೆಲವೇ ಕೆಲವು ಕೈಬೆರಳಣಿ ಕೆಯ ಮೇಧಾವಿಗಳು ಅದನ್ನು ಮೀರಿ ಮೇಲೇಳುವಂಥ ಆತ್ಮವಿಶ್ವಾಸವನ್ನು ಮೈಗೂಡಿಸಿ\-ಕೊಂಡಿರು\-ತ್ತಾರೆ. ಆ ಆತ್ಮವಿಶ್ವಾಸವೇ ಅಡಗಿರುವ ಅಪಾರ ಶಕ್ತಿಯನ್ನು ವ್ಯಕ್ತಗೊಳಿಸುವ ಕೀಲಿಕೈ.

ಸುಳ್ಳಾದರೂ ಯಾವುದನ್ನು ಮನಸ್ಸು ಸತ್ಯವೆಂದು ನಿಸ್ಸಂಶಯವಾಗಿ ಸ್ವೀಕರಿಸುವುದೋ, ಅಂತೆಯೇ ವರ್ತಿಸುವುದು ಎಂಬುದು ಮೇಲಿನ ಘಟನೆಯಿಂದ ಸ್ಪಷ್ಟ. ನಮ್ಮ ಹೊರಮನಸ್ಸು ಕೆಲಸ ಮಾಡದೇ ವಶ್ಯಸುಪ್ತಿಯಂಥ ವಿಶ್ರಾಂತ ಸ್ಥಿತಿಯಲ್ಲಿದ್ದಾಗ ನಿಷೇಧಾತ್ಮಕ ಯೋಚನೆಗಳು ಮನಸ್ಸಿನ ಆಳವನ್ನು ಪ್ರವೇಶಿಸಿದರೆ ಅವು ಎಂಥ ಪ್ರಭಾವ ಬೀರಬಲ್ಲವು! ಎಂಥ ಪ್ರಮಾದ\-ವನ್ನುಂಟು\-ಮಾಡ\-ಬಲ್ಲವು! ಎಂಬುದನ್ನು ಮೇಲಿನ ಘಟನೆಯ ಆಧಾರದಿಂದ ಕಲ್ಪಿಸಿಕೊಳ್ಳ ಬಹುದು. ಸಾಮಾನ್ಯವಾಗಿ ಪ್ರಕೃತಿಯ ರಕ್ಷಣೆ ನಮಗಿದೆ. ನಮ್ಮ ಹೊರ ಮನಸ್ಸು ಚಟುವಟಿಕೆಯಿಂದಿದ್ದಾಗ ಯಾವುದೇ ನೂತನ ಸೂಚನೆಯನ್ನಾಗಲಿ, ಅಭಿಪ್ರಾಯವನ್ನಾಗಲಿ ಥಟ್ಟನೆ ಸ್ವೀಕರಿಸುವುದಿಲ್ಲ. ಯೋಚಿಸಿ, ವಿಮರ್ಶಿಸಿ ತನ್ನ ಹಿಂದಿನ ಅನುಭವಗಳೊಡನೆ ಹೋಲಿಸಿ, ಸಾವಕಾಶವಾಗಿ ಸ್ವೀಕರಿಸುತ್ತದೆ. ಇಲ್ಲವಾದರೆ ನಮ್ಮ ಪ್ರತಿಯೊಂದು ಭಯ ಸಂಶಯಗಳೂ ಕಾರ್ಯ ರೂಪಕ್ಕೆ ಬರುತ್ತಿದ್ದವು. ‘ನನಗೆ ಜ್ವರ ಬರುತ್ತಿದೆ’, ‘ನಾನು ದುರ್ಬಲನಾಗುತ್ತಿದ್ದೇನೆ’ ಎಂಬಂಥ ಯೋಚನೆಗಳನ್ನು ಬಹುಬೇಗನೇ ಸುಪ್ತಮನಸ್ಸು ಕಾರ್ಯತಃ ಮಾಡಿ ತೋರಿಸುತ್ತಿತ್ತು. ಆದರೆ ನಮ್ಮ ಎಚ್ಚರದ ಮನಸ್ಸು ಒಂದೇ ಯೋಚನೆಯನ್ನು ತೀವ್ರ ಭಾವೋದ್ವೇಗದಿಂದ, ಭಯದಿಂದ ಮೆಲುಕಾಡುತ್ತಿದ್ದರೆ ಸುಪ್ತಮನಸ್ಸು ಅದನ್ನು ಬಹುಬೇಗನೇ ಕಾರ್ಯಗತಗೊಳಿಸುತ್ತದೆ!

ಮಕ್ಕಳಲ್ಲಿ ಹೊರಮನಸ್ಸು ಬಲಿಯುವುದಕ್ಕೆ ಮೊದಲೇ ನಿಷೇಧಾತ್ಮಕ ಸೂಚನೆಗಳನ್ನಿತ್ತರೆ ಅವು ವಶ್ಯಸುಪ್ತಿಯಲ್ಲಿ ನೀಡಿದ ಸೂಚನೆಯಂತೆಯೇ ಹೆಚ್ಚು ಕಡಿಮೆ ಕೆಲಸ ಮಾಡಿಯಾವು! ಆದುದ ರಿಂದಲೇ ಮಕ್ಕಳನ್ನು ತಿದ್ದುವವರು ಎಷ್ಟೊಂದು ಜಾಗರೂಕರಾಗಿರಬೇಕು! ಅವರ ಭವಿಷ್ಯದ ಬುನಾದಿಯ ಇಟ್ಟಿಗೆಗಳು ಬಿರುಕು ಬಿಡದಂತೆ ನೋಡಿಕೊಳ್ಳಬೇಕು. ವ್ಯಕ್ತಿಯನ್ನು ಮಾತ್ರವಲ್ಲ, ಒಂದು ಜನಾಂಗವನ್ನೇ ದುರ್ಬಲಗೊಳಿಸುವ ಅಥವಾ ಸಶಕ್ತವನ್ನಾಗಿಸುವ ಮೂಲಸೂತ್ರವಿಲ್ಲಿದೆಯಲ್ಲವೇ?

‘ಹಗಲೂರಾತ್ರಿ ಒಬ್ಬಾತ, ಅಯ್ಯೋ! ನಾನು ದೀನ, ದುಃಖಿ, ಯಾವುದಕ್ಕೂ ಪ್ರಯೋಜನ\-ವಿಲ್ಲದವನು’ ಎಂದು ಯೋಚಿಸುತ್ತಿದ್ದರೆ ಅವನು ನಿಜವಾಗಿಯೂ ನಿಷ್ಪ್ರಯೋಜಕನೇ ಆಗುತ್ತಾನೆ. ಅಯೋಗ್ಯ ಎಂದು ಭಾವಿಸುತ್ತಿದ್ದರೆ ಅಯೋಗ್ಯನೇ ಆಗುತ್ತಾನೆ. ‘ನಾವೆಲ್ಲ ಭಗವಂತನ ಮಕ್ಕಳು ಅನಂತಾತ್ಮನ ಅಥವಾ ಪರಂಜ್ಯೋತಿಯ ಕಿಡಿಗಳು. ನಾವು ಹೇಗೆ ಅಯೋಗ್ಯರಾಗಬಲ್ಲೆವು? ನಮ್ಮಿಂದ ಯಾವ ಕಾರ್ಯವೂ ಅಸಾಧ್ಯವಲ್ಲ. ಮಾನವನು ಏನನ್ನು ಬೇಕಾದರೂ ಸಾಧಿಸಬಲ್ಲ. ಈ ವಿಷಯವನ್ನು ಎಂದೂ ಮರೆಯಬೇಡಿ’ ಎಂದರು ಸ್ವಾಮಿ ವಿವೇಕಾನಂದರು.

‘ನಿಮ್ಮಪುರಾಣಗಳಲ್ಲಿ ಬರುವ ಮೂವತ್ತಮೂರು ಕೋಟಿ ದೇವರನ್ನು ನೀವು ನಂಬಿದರೂ ನಿಮ್ಮಲ್ಲಿ ನಿಮಗೆ ವಿಶ್ವಾಸವಿರದಿದ್ದರೆ ನಿಮಗೆ ಮುಕ್ತಿ ಇಲ್ಲ. ನಿಮ್ಮಲ್ಲಿ ನಿಮಗೆ ಶ್ರದ್ಧೆ ಅಥವಾ ವಿಶ್ವಾಸವಿರಲಿ, ಆ ಶ್ರದ್ಧೆಯ ಅಡಿಗಲ್ಲಿನ ಮೇಲೆ ನಿಂತು ಬಲಾಢ್ಯರಾಗಿ. ನಮಗಿಂದು ಬೇಕಾಗಿರುವುದು ಶ್ರದ್ಧೆ.’

‘ವ್ಯಕ್ತಿಯೇ ಆಗಲಿ ಜನಾಂಗವೇ ಆಗಲಿ ಶ್ರದ್ಧೆಯನ್ನು ಕಳೆದುಕೊಂಡರೆ ನಾಶ ಸಮೀಪಿಸಿದಂತೆ’ ಎಂದೂ ಹೇಳಿದ್ದರವರು.


\section*{ರಕ್ತದಲ್ಲೇ ರಚನಾತ್ಮಕತೆ!}

\addsectiontoTOC{ರಕ್ತದಲ್ಲೇ ರಚನಾತ್ಮ\-ಕತೆ!}

ಜಪಾನ್​ದೇಶದಲ್ಲಿ ಮಕ್ಕಳಿಗೆ ಪ್ರಾಥಮಿಕ ಶಾಲೆಯನ್ನು ಸೇರಿದ ಮೊದಲ ದಿನದಿಂದಲೇ ತಮ್ಮ ದೇಶದ ಮಹಿಮೆಯನ್ನು ಬೋಧಿಸುತ್ತಾರೆಂದು ಜಪಾನ್ ದೇಶವನ್ನು ಸಂದರ್ಶಿಸಿ ಬಂದ ಹಿರಿಯರು ಹೇಳಿದರು. ತಮ್ಮ ದೇಶದ ಮಹಿಮರ ಮಹಾನುಭಾವರ ಬಗೆಗೆ, ಸಂಸ್ಕೃತಿಯ ಬಗೆಗೆ, ತಮ್ಮ ಪೂರ್ವಜರು ಸಾಧಿಸಿ ಸಂಪಾದಿಸಿದ ದಕ್ಷತೆ, ಶಕ್ತಿ ಸಾಮರ್ಥ್ಯಗಳ ಬಗೆಗೆ ಗೌರವ ಆದರಗಳನ್ನು ಉಂಟುಮಾಡುವಂಥ, ಸ್ವಾಭಿಮಾನ, ಸ್ವಜನಾಭಿಮಾನ, ಸ್ವದೇಶಾಭಿಮಾನಗಳನ್ನುಂಟು ಮಾಡು\-ವಂಥ ಸಂಗತಿಗಳನ್ನು ಮಕ್ಕಳು ಚೆನ್ನಾಗಿ ತಿಳಿದುಕೊಳ್ಳುತ್ತಾರೆ. ತಾವು ಎಷ್ಟು ಶ್ರೇಷ್ಠ ಜನಾಂಗಕ್ಕೆ ಸೇರಿದವರೆಂದರೆ ದೇವರೇ ತಮ್ಮನ್ನು ವಿಶೇಷವಾಗಿ ಪ್ರೀತಿಸುತ್ತಾನೆ! ಸೂರ್ಯದೇವನ ಬೆಳಕಿನ ಕಿರಣಗಳು ದಿನವೂ ಮೊತ್ತಮೊದಲು ತಮ್ಮ ದೇಶದ ಮೇಲೇ ಬೀಳುತ್ತಿರುವುದು ಆ ಪ್ರೀತಿಯ ಸಂಕೇತ! ಜಗತ್ತಿನಲ್ಲೇ ತಮ್ಮದು ಅಗ್ರಮಾನ್ಯ ದೇಶ! ತಾವು ಉನ್ನತಮಟ್ಟದ ಜನಾಂಗ! ಉದಯರವಿಯ ನಾಡಿನವರು! ಈ ಭಾವನೆಗಳನ್ನು ಬಾಲ್ಯದಲ್ಲೇ ಮಕ್ಕಳ ಮನಸ್ಸಿನಲ್ಲಿ ಸರಿಯಾಗಿ ಬಿಂಬಿಸುತ್ತಾರೆ. ಅದು ಅವರ ಸ್ವವ್ಯಕ್ತಿತ್ವ ಚಿತ್ರದಲ್ಲಿ ನೆಲೆನಿಂತು ಎಂಥ ರಚನಾತ್ಮಕ ಕಾರ್ಯಕ್ಕೆ ಒಂದು ಜನಾಂಗವನ್ನೇ ಪ್ರೇರಿಸಿದೆ ಎಂಬುದು ಇಂದು ಎಲ್ಲರ ಕಣ್ಣಿಗೂ ಕಾಣಿಸುವ ಸಂಗತಿ. ಕಾಯಕ ಪ್ರೇಮ, ಉದ್ಯಮಶೀಲತೆ, ಸಹಕಾರ ಭಾವನೆ, ಪ್ರತಿಯೊಂದು ಕ್ಷೇತ್ರದಲ್ಲೂ ಪರಿಪೂರ್ಣತೆಯನ್ನು ಮುಟ್ಟುವ ಹಂಬಲ, ಅಪಾರವಾದ ದೇಶಪ್ರೇಮ–ಇವುಗಳನ್ನು ಬಾಲ್ಯದಿಂದಲೇ ಮೈಗೂಡಿಸಿಕೊಂಡ ಆ ಪುಟ್ಟದೇಶ ನಾನಾ ತೆರನಾದ ನೈಸರ್ಗಿಕ ಪರಿಮಿತಿಗಳನ್ನು ದಾಟಿ ಜಗತ್ತಿನ ಮುಂದುವರಿದ ಮಹಾರಾಷ್ಟ್ರಗಳಿಗೆ ಎಲ್ಲ ಕ್ಷೇತ್ರಗಳಲ್ಲೂ ಪಂಥಾಹ್ವಾನ ನೀಡಬಲ್ಲ ಅಗ್ರಸ್ಥಾನವನ್ನು ಸಂಪಾದಿಸಿದೆಯಲ್ಲವೆ?


\section*{ದೌರ್ಬಲ್ಯದ ದಾರ್ಶನಿಕರು}

\addsectiontoTOC{ದೌರ್ಬಲ್ಯದ ದಾರ್ಶನಿಕರು}

ಕಳೆದ ಶತಮಾನಗಳಲ್ಲಿ ನಮ್ಮ ದೇಶದ ಮಕ್ಕಳು ಪಡೆಯುವ, ಸಾಮ್ರಾಜ್ಯಶಾಹಿಗಳು ರೂಪಿಸಿದ ಶಿಕ್ಷಣವನ್ನು ಕುರಿತು ಸ್ವಾಮಿ ವಿವೇಕಾನಂದರು ಹೀಗೆಂದಿದ್ದರು: ‘ನೀವು ಪಡೆಯುವ ಶಿಕ್ಷಣದಲ್ಲಿ ಕೆಲವೊಂದು ಗುಣಗಳಿವೆ. ಆದರೆ ಅದರಿಂದಾಗುವ ಅಪಕಾರದ ಪ್ರಮಾಣ ಎಷ್ಟು ಹೆಚ್ಚಿದೆ ಎಂದರೆ ತಕ್ಕಡಿಯ ಒಂದು ತಟ್ಟೆಯಲ್ಲಿ ಒಳ್ಳೆಯ ಅಂಶದ ಭಾರವೇ ಇಲ್ಲವೆನ್ನುವಂತಾಗಿದೆ. ಮಗುವನ್ನು ಶಾಲೆಗೆ ಕರೆದೊಯ್ದಾಗ ಮೊದಲು ಅದು ಕಲಿಯುವುದು ತನ್ನ ತಂದೆ ಒಬ್ಬ ಮೂರ್ಖ ಎಂದು, ಎರಡನೆಯದೇ ತನ್ನ ಅಜ್ಜ ಹುಚ್ಚ ಎಂದು, ಮೂರನೆಯದೆ ತನ್ನ ಗುರುಗಳೆಲ್ಲ ಮೋಸಗಾರರೆಂದು, ನಾಲ್ಕನೆಯದೇ ತನ್ನ ದೇಶದ ಪವಿತ್ರ ಗ್ರಂಥಗಳೆಲ್ಲ ಸುಳ್ಳು, ಮೂಢನಂಬಿಕೆಗಳ ಕಂತೆ ಎಂದು. ಐದನೆಯದೇ ತಾವೆಲ್ಲ ಕೆಲಸಕ್ಕೆ ಬಾರದವರು, ತಮ್ಮ ದೇಶದಲ್ಲಿ ಮಹಾವ್ಯಕ್ತಿಗಳೇ ಹುಟ್ಟಿಲ್ಲ ಎಂದು. ಹೀಗೆ ಬೆಳೆದ ಹುಡುಗನೊಬ್ಬ ೧೬ನೆ ವಯಸ್ಸಿನ ಹೊತ್ತಿಗೆ ಶಕ್ತಿ ಉತ್ಸಾಹಗಳಿಲ್ಲದ ನಿಷೇಧಾತ್ಮಕ ಭಾವನೆಗಳ ಮುದ್ದೆಯಾಗುತ್ತಾನೆ. ನಿಷೇಧಾತ್ಮಕ ವಿದ್ಯಾಭ್ಯಾಸ ಅಥವಾ ನಿಷೇಧಮಯ ಭಾವನೆಗಳ ಮೇಲೆ ನಿಂತ ಯಾವುದೇ ತರಬೇತಿ ಸಾವಿಗಿಂತಲೂ ಭಯಾನಕ.’

ಭಾರತದ ಪುರಾತನ ಸಂಸ್ಕೃತಿಯ ಪ್ರಬಲ ಬೌದ್ಧಿಕ ಸಾಮರ್ಥ್ಯಕ್ಕೆ ಸಾಕಷ್ಟು ಸಾಕ್ಷ್ಯಗಳಿದ್ದರೂ, ಜಗತ್ತೇ ಅದನ್ನು ಕಂಡು ಗೌರವದಿಂದ ತಲೆಬಾಗಿದರೂ, ಸ್ವಾತಂತ್ರ್ಯಪೂರ್ವದಲ್ಲಿ ಆ ಸಂಸ್ಕೃತಿಯ ಸಂದೇಶದಿಂದಲೇ ಸ್ಫೂರ್ತಿಪಡೆದ ಈ ದೇಶದ ದೂರದೃಷ್ಟಿಯ ಧೀಮಂತ ದೇಶಪ್ರೇಮಿಗಳು ಗುಲಾಮಗಿರಿಯಿಂದ, ವಿದೇಶೀಯರ ಸುಲಿಗೆಯಿಂದ ದೇಶವನ್ನು ಪಾರುಮಾಡಲು ಪ್ರಾಣವನ್ನು ಪಣವಾಗಿಟ್ಟು ಹೋರಾಡಿದ್ದರೂ, ಇಂದಿನ ದಿನಗಳಲ್ಲಿ ಶಾಲಾಕಾಲೇಜುಗಳಿಂದ ಹೊರ ಬರುವ ಲಕ್ಷಲಕ್ಷ ನರನಾರಿಯರಲ್ಲಿ ದೇಶದ ಹಿರಿಮೆ ಗರಿಮೆಗಳ ಬಗೆಗೆ ಕಿಂಚಿತ್ತಾದರೂ ಅಭಿಮಾನ ಕಂಡು\-ಬರುತ್ತಿದೆಯೇ? ಪ್ರಚಾರ ವ್ಯವಸ್ಥೆಯ ಸಂಘಟಿತ ಬಲವಿಲ್ಲದಿದ್ದರೂ ತನ್ನ ಸತ್ವಶಕ್ತಿಗಳಿಂದಲೇ ವಿದೇಶೀ ವಿಚಾರವಂತರ ಮನಸೆಳೆದ ಯೋಗವೇದಾಂತಗಳಾಗಲಿ, ನೃತ್ಯ ಸಂಗೀತಗಳಾಗಲಿ, ವಾಸ್ತು\-ಶಿಲ್ಪಗಳಾಗಲಿ, ತತ್ತ್ವಶಾಸ್ತ್ರ, ಸಂತರ ಜೀವನ, ಮತೀಯ ಸಾಧನೆ ಬೋಧನೆಗಳಾಗಲಿ, ಕುಶಲಕಲೆ\-ಗಳಾಗಲಿ ಇಲ್ಲಿನ ಆಧುನಿಕ ವಿದ್ಯಾವಂತರಿಗೆ ಆಪ್ಯಾಯಮಾನವೆನಿಸಿಯಾವೇ? ಈ ಸಂಗತಿಗಳನ್ನು ಕುರಿತ ಪ್ರಾಥಮಿಕ ಪರಿಚಯ, ಸ್ವಲ್ಪಮಟ್ಟಿನ ಅಭಿರುಚಿ ಅಭಿಮಾನಗಳನ್ನಾದರೂ ಈ ದೇಶದ ಯುವಕರಲ್ಲಿ ಕಾಣಬಹುದೆ? ಕಾಣುವ ಸಾಧ್ಯತೆಯಾದರೂ ಇದೆಯೆ? ದೇಶದ ಪರಂಪರೆಗೆ ಸಂಬಂಧಿಸಿದ ಎಲ್ಲ ಸಾಧನೆ, ಸಿದ್ಧಿ, ಮೌಲ್ಯಗಳನ್ನೂ ‘ಶೋಷಣೆ ಮತ್ತು ಮೂಢ ನಂಬಿಕೆ’ ಎಂಬ ಎರಡು ಶಬ್ದಗಳಿಂದ ವಿವರಿಸುವುದೇ ನಮ್ಮ ಕ್ರಾಂತಿಕಾರಿಗಳ, ಮಹಾಬುದ್ಧಿ ಜೀವಿಗಳ ಲಕ್ಷಣ ಎಂಬಂತೆ ಆಗಿದೆಯಲ್ಲ! ರಾಷ್ಟ್ರೀಯತಾಭಾವನೆಯಿಂದ ಪ್ರೇರಿತರಾಗಿ ಹೋರಾಡಿ, ಅಪಾರ ಬಲಿದಾನ, ತ್ಯಾಗಗಳಿಂದ ಸ್ವಾತಂತ್ರ್ಯವನ್ನು ಪಡೆದು ಸ್ವಾತಂತ್ರ್ಯಾನಂತರ ಆ ಭಾವನೆಗೆ ಕುಠಾರಾಘಾತವಾಗುವಂಥ ಕೃತ್ಯ ಇತ್ತೀಚಿನ ವರ್ಷಗಳಲ್ಲಿ ಸರ್ವತ್ರ ಕಂಡುಬರುತ್ತಿದೆಯಲ್ಲ! ಭಾಷಾವಿವಾದ, ಜಾತಿಗಳಲ್ಲಿ ಪರಸ್ಪರ ದ್ವೇಷ, ವ್ಯಕ್ತಿಸ್ವಾರ್ಥ, ಬುದ್ಧಿ ಜೀವಿಗಳ ಪರಾನುಕರಣೆ ಮತ್ತು ಅಸಹಾಯಕತೆ, ಅಪಾರಮಟ್ಟದ ಭ್ರಷ್ಟಾಚಾರ, ನೈತಿಕ ಅವನತಿ –ಇವು ದೇಶವನ್ನು ತರಿದು ತಿನ್ನುತ್ತಿವೆಯಲ್ಲ! ‘ರಾಜಕಾರಣಿ ಅಯೋಗ್ಯ, ಉತ್ಪಾದಕ ಲಾಭ ಬಡುಕ, ವರ್ತಕ ಕಾಳ ಸಂತೆಯ ಪ್ರವರ್ತಕ, ಸರಕಾರಿ ನೌಕರ ಸ್ವಾರ್ಥಿ ಮತ್ತು ಲಂಚಕೋರ, ಕೆಲಸಗಾರ ಮೈಗಳ್ಳ ಮುಷ್ಕರನಿರತ, ವಿದ್ಯಾರ್ಥಿ ಶಿಸ್ತು ಮತ್ತು ವಿದ್ಯಾಸಕ್ತಿ ಇಲ್ಲದವನು.’ ಇದೆಂಥ ಪರಿಸ್ಥಿತಿ! ರಾಜಕೀಯದ ರಾಗದ್ವೇಷಗಳ ದುರ್ಬೀಜವನ್ನು ಬಿತ್ತಿಯೇ ದೇಶದ ಉನ್ನತಿ ಎಂಬ ಸತ್ಫಲವನ್ನು ಬೆಳೆಯಲು ಸಾಧ್ಯವೇ? ರಾಜಕೀಯ ಭ್ರಷ್ಟಾಚಾರಗಳು ಹೆಚ್ಚುತ್ತಲೇ ಹೋಗುತ್ತಿವೆ. ಆದರೆ ತಪ್ಪಿತಸ್ಥರನ್ನು ಶಿಕ್ಷಿಸುವವರು ಯಾರೂ ಇಲ್ಲ. ಬಡ ಪ್ರಜಾಜನರ ಕಷ್ಟಗಳು ಹೆಚ್ಚುತ್ತಲೇ ಇವೆ. ಇದಕ್ಕೆ ಪರಿಹಾರವೇ ಇಲ್ಲ. ಎಲ್ಲರೂ ಬಡವರ ಉದ್ಧಾರಕ್ಕೆ ಕಂಕಣಬದ್ಧರಾದವರೇ. ಸುತ್ತಲೂ ಕತ್ತಲೆ ಕವಿಯುತ್ತಿದ್ದರೂ ಎತ್ತರ ಸ್ಥಾನದಲ್ಲಿರುವವರು ರಾಷ್ಟ್ರದ ಹಿತರಕ್ಷಣೆಯ ಸೂತ್ರವನ್ನು ಹಿಡಿದವರು ‘ಎಲ್ಲ ಕಡೆ ಬೆಳಕೇ ’ ಎಂದು ಪ್ರಚಾರ ಮಾಡಿದರೆ ಸಾಕೇ? ಸರ್ವತ್ರ ದುರಂತದೆಡೆ ಧಾವಿಸುತ್ತಿರುವ ಒಂದು ಜನಾಂಗವನ್ನು ರಕ್ಷಿಸುವ ಯಾವ ಸೂತ್ರವೂ ಇಲ್ಲವೆ? ಎಲ್ಲ ಪಕ್ಷಗಳವರೂ ಎಲ್ಲ ಜಾತಿಮತಗಳ ಜನರೂ ಒಪ್ಪುವ, ಎಲ್ಲರಿಗೂ ಶಕ್ತಿಯನ್ನೀವ ಒಂದು ಶ್ರದ್ಧೆಯ ಕೇಂದ್ರಬಿಂದುವಿಲ್ಲವೆ? ಜನಾಂಗದ ಸ್ವವ್ಯಕ್ತಿತ್ವ ಚಿತ್ರದಲ್ಲಿ ಎಂಥ ನಿಷೇಧಮಯ ಭಾವನೆಗಳು ತುಂಬಿಕೊಂಡಿವೆ! ತತ್ಫಲವಾಗಿ ಎಂಥ ದುರಂತ ಸನ್ನಿವೇಶ ಸನ್ನಿಹಿತವಾಗಿದೆ!


\section*{ಬೇಕಾಗಿದೆ ಕಾಯಕಲ್ಪ}

\addsectiontoTOC{ಬೇಕಾಗಿದೆ ಕಾಯಕಲ್ಪ}

ಇತ್ತೀಚೆಗೆ ನಮ್ಮ ದೇಶದ ಧೀಮಂತ ವ್ಯಕ್ತಿಗಳಲ್ಲೊಬ್ಬರಾದ ನ್ಯಾಯವಾದಿ ನಾಪಾಲ್ಕೀವಾಲರು ಜಪಾನ್​ದೇಶದ ಮಂತ್ರಿಗಳೊಬ್ಬರನ್ನು ಭೇಟಿಯಾದಾಗ ಒಂದು ಪ್ರಶ್ನೆಯನ್ನು ಕೇಳಿದರು: ‘ನಿಜ ಹೇಳುವುದಾದರೆ, ನಾವು ಜಪಾನ್ ದೇಶದ ಜನರಿಗಿಂತ ಕಡಿಮೆ ಬುದ್ಧಿಶಾಲಿಗಳೇನೂ ಅಲ್ಲ. ನೀವಾದರೋ ಜಗತ್ತಿನ ಉನ್ನತ ರಾಷ್ಟ್ರಗಳಲ್ಲಿ ಅತ್ಯುನ್ನತ ಸ್ಥಾನಗಳಿಸಿಕೊಂಡಿದ್ದೀರಿ. ಆದರೆ ನಾವಿನ್ನೂ ಕೊಳೆ ಕೊಂಪೆಯಲ್ಲಿ ನರಳುತ್ತಿದ್ದೇವೆ. ಹೇಳಿ, ಇದಕ್ಕೇನು ಕಾರಣ?’

ಜಪಾನ್ ದೇಶದ ಮಂತ್ರಿಗಳು ನೀಡಿದ ಅರ್ಥಪೂರ್ಣ ಉತ್ತರ ಹೀಗಿತ್ತು: ‘ನನ್ನ ಸ್ನೇಹಿತರೆ! ಜಪಾನಿನಲ್ಲಿ ನಾವು ಒಂದು ಮಿಲಿಯ ನಾಗರಿಕರಿದ್ದೇವೆ. ನಿಮ್ಮಲ್ಲಿ ಆರುನೂರು ಮಿಲಿಯ ವ್ಯಕ್ತಿ\-ಗಳಿದ್ದಾರೆ!’

ನಮ್ಮಪಾಲಿಗೆ ಬಂದ ಶಿಕ್ಷಣ ವಿಧಾನ ನಮ್ಮನ್ನೂ, ನಮ್ಮ ಯುವಕರನ್ನೂ ನಾಗರಿಕರನ್ನಾಗಿ ಮಾಡುವ ಸಂಭವವಿದೆಯೆ? ಅಥವಾ ಸಂಕುಚಿತ ಸ್ವಾರ್ಥಕ್ಕೇ ತನ್ನ ಸರ್ವಸ್ವವನ್ನೂ ಧಾರೆಯೆರೆ ಯುವ ಸ್ವಾರ್ಥಪರಾಯಣ ‘ವ್ಯಕ್ತಿ’ಗಳನ್ನೇ ಆಗಿಸಲು ತರಬೇತಿ ನೀಡೀತೆ? ಪ್ರಾಯಃ ಜಪಾನಿನ ಮಂತ್ರಿಗಳು ‘ವ್ಯಕ್ತಿ’ಗಳೆಂದೇ ಹೇಳಿ ಸುಮ್ಮನಾದರೂ, ಕ್ಷುದ್ರ ಎಂಬ ವಿಶೇಷಣ ಸೇರಿಸಲು ಮರೆತರೇ? ಅಥವಾ ಅದು ಅವರ ಇಂಗಿತವಿದ್ದರೂ ಇರಬಹುದೆ?

‘ಭಾರತ ಎಂದರೆ ಅತ್ಯಂತ ಹೆಚ್ಚಿನ ಸಂಪನ್ಮೂಲಗಳುಳ್ಳ ಆದರೆ ಬಡವರು ವಾಸವಾಗಿರುವ ಶ‍್ರೀಮಂತ ದೇಶ’ ಎಂದು ವಿದೇಶೀ ಯಾತ್ರಿಕನೊಬ್ಬ ಹೇಳಿದ. ಎಂಥ ವಿರೋಧಾಭಾಸ! ಇದಕ್ಕೆ ಕಾರಣವೇನಿರಬಹುದು?

ಇತ್ತೀಚಿನ ದಶಕಗಳಲ್ಲಿ ಒಂದು ರಾಷ್ಟ್ರವಾಗಿ ಎದ್ದು ನಿಂತ ಇಸ್ರೇಲ್​ನವರ ದೇಶಪ್ರೇಮ, ಪ್ರತಿಭೆ ಪರಾಕ್ರಮಗಳೂ ಅನನ್ಯವಾದವು. ಎರಡು ಸಾವಿರವರುಷಗಳ ಹಿಂದೆ ತಮ್ಮ ಮಾತೃ ಭೂಮಿಯನ್ನು ಕಳೆದುಕೊಂಡು, ಹರಿದು ಹಂಚಿಹೋಗಿದ್ದರು ಅವರು. ಇಂದು ತಮ್ಮ ಅಂಗೈ ಅಗಲದ ಭೂಮಿಯಲ್ಲಿ ದೃಢವಾಗಿ ನಿಂತು ರಕ್ತಪಿಪಾಸು ವಿರೋಧಿಗಳ ಆಕ್ರಮಣವನ್ನು ಎದುರಿಸಿ ಅಪ್ರತಿಮ ಪ್ರಗತಿಯನ್ನು ಸಾಧಿಸಿ ತೋರಿಸಿದ್ದಾರೆ. ಪ್ರತಿಯೊಂದು ಕ್ಷೇತ್ರದಲ್ಲೂ ಅದ್ಭುತ ಪ್ರಗತಿಯ ಜಯಭೇರಿ ಬಾರಿಸಿದ್ದಾರೆ! ಮರುಭೂಮಿಯನ್ನು ನಂದನವನವನ್ನಾಗಿಸಿದ್ದಾರೆ!

ಇಸ್ರೇಲಿಗಳೂ, ಜಪಾನೀಯರೂ, ಮಾನವ ಕುಟುಂಬಕ್ಕೆ ಸೇರಿದವರೆ! ಆದರೆ ವ್ಯತ್ಯಾಸದ ಮೂಲವೆಲ್ಲಿ?

ಮೊದಲಿನ ಎರಡು ಜನಾಂಗಗಳಿಗೂ ವೈಯಕ್ತಿಕ ಪ್ರಯತ್ನಗಳನ್ನು ಸಮನ್ವಯಗೊಳಿಸಿ ಸಾರ್ಥಕ\-ಗೊಳಿಸುವ ಸ್ವಾಭಿಮಾನ ಸ್ವಜನಾಭಿಮಾನ ಸ್ವದೇಶಾಭಿಮಾನಗಳನ್ನುಂಟುಮಾಡುವ ಆದರ್ಶ\break ಅಥವಾ ಗುರಿ ಇದೆ. ಭಾರತದ ಜನ ತಮ್ಮ ‘ಸ್ವ’ ಏನೆಂಬುದನ್ನು ತಿಳಿದುಕೊಳ್ಳದೇ ಗೊಂದಲಕ್ಕೊಳ\-ಗಾಗಿದ್ದಾರೆ! ರಾಷ್ಟ್ರೀಯತೆ ಉಳಿಯುವುದು ಸಂಸ್ಕೃತಿಯ ಮೂಲತತ್ವಗಳ ಮೇಲೆ. ಆ ಮೂಲತತ್ವಗಳು ವ್ಯಕ್ತಿಯ ಅಭ್ಯುದಯಕ್ಕೂ ರಾಷ್ಟ್ರದ ಕಲ್ಯಾಣಕ್ಕೂ ಕಾರಣವಾಗು ವಂಥವುಗಳು. ಜೀವನದಾಯಿಯಾದ ಆ ತತ್ವಗಳನ್ನು ವಿಜ್ಞಾನದ ಬೆಳಕಿನಲ್ಲಿ ಪರೀಕ್ಷಿಸಿ ನೋಡಿ ಅದರ ಸತ್ಯತೆಯನ್ನು ಜನಮಾನಸಕ್ಕೆ ಮುಟ್ಟಿಸಿದರೆ ವ್ಯಕ್ತಿಯೊಡನೆ ರಾಷ್ಟ್ರವೂ ಎದ್ದು ನಿಲ್ಲುವ ಸಂಭವವಿದೆ.

ಸಾಮ್ರಾಜ್ಯಶಾಹಿ ಸರಕಾರ ಅಂದು ತನ್ನ ಅಗತ್ಯಕ್ಕಾಗಿ, ಕಾರಕೂನರ ಪಡೆಯ ನಿರ್ಮಾಣಕ್ಕಾಗಿ ರೂಪಿಸಿಕೊಂಡ ಶಿಕ್ಷಣ ಇಂದೂ ಮುಂದುವರಿಯುತ್ತಿದೆ. ನಮ್ಮ ಪರಂಪರೆ ಸಂಸ್ಕೃತಿಗಳ ಮೇಲೆ, ಕೊನೆಗೆ ನಮ್ಮ ಮೇಲೆಯೇ ನಮಗೆ ತಿರಸ್ಕಾರ ಹುಟ್ಟುವಂತೆ, ವಿದೇಶೀಯರ ಜೀವನ ವಿಧಾನಗಳ ಮೇಲೆ ಆದರ ಹುಟ್ಟುವಂತೆ ಮೆಕಾಲೆ ಯೋಜಿಸಿದ ಶಿಕ್ಷಣಪದ್ಧತಿ ತನ್ನ ಪ್ರಭಾವ ಬೀರಿತು. ಈಗ ವೈಜ್ಞಾನಿಕ ತಾಂತ್ರಿಕ ಶಿಕ್ಷಣಗಳೂ ಸೇರಿಕೊಂಡಿವೆಯೆಂಬುದೇನೊ ನಿಜ. ಆದರೆ ಶಿಕ್ಷಣದಲ್ಲಿ ವ್ಯಕ್ತಿತ್ವದ ಪೂರ್ಣ ವಿಕಾಸಕ್ಕೆ ಎಡೆ ಇದೆಯೆ? ಶಿಕ್ಷಿತ ವ್ಯಕ್ತಿ ಉತ್ಪಾದಕ ಘಟಕನಾಗುತ್ತಿದ್ದಾನೆಯೆ? ಸ್ವಾವಲಂಬಿಯಾಗುತ್ತಿದ್ದಾನೆಯೆ? ಪರಾವಲಂಬಿ ಪರೋಪಜೀವಿಯಾಗುತ್ತಿದ್ದಾನಲ್ಲ! ನಮ್ಮ ಶಿಕ್ಷಣ ಓದು ಬರಹ ಮತ್ತು ವಿಚಾರ ಸಂಗ್ರಹಗಳಲ್ಲಿ ಕೆಲವೊಂದು ಯಾಂತ್ರಿಕ ಪರಿಣತಿಯಲ್ಲಿ ತರಬೇತು ನೀಡಿದರೂ ಒಂದು ಜನಾಂಗವಾಗಿ ಬಾಳಲು ಬೇಕಾದ ಸಂಸ್ಕಾರಗಳನ್ನು ಯುವಕರಿಗೆ ಕೊಡಲು ಏಕೆ ಸಮರ್ಥವಾಗಿಲ್ಲ? ಸ್ವತಂತ್ರ ಮನೋವೃತ್ತಿ, ರಚನಾತ್ಮಕ ಪ್ರವೃತ್ತಿ, ಸಮಾಜದ ಒಟ್ಟು ಹಿತದ ಪ್ರಜ್ಞೆ, ಬುದ್ಧಿಪೂರ್ವಕ ಪರಿಶ್ರಮ ಮಾಡುವ ಹಂಬಲ, ಕನಿಷ್ಠ ನೈತಿಕ ಮಟ್ಟ ಇವುಗಳನ್ನು ನೀಡದಿದ್ದರೆ ಶಿಕ್ಷಣ ಯುವಕರನ್ನು ದುರ್ಬಲರನ್ನಾಗಿಸಲು ಯತ್ನಿಸಿದಂತಾಗದೆ?

ಬ್ರಿಟಿಷರು ಬಿತ್ತಿದ ‘ನೀವು ಗುಲಾಮರಾಗಿರಲು ಯೋಗ್ಯರು’ ಎನ್ನುವ ಭಾವನೆ ಭಾರತೀಯರ ಸ್ವವ್ಯಕ್ತಿತ್ವ ಚಿತ್ರದಲ್ಲಿ ಗಟ್ಟಿಯಾಗಿ ಅಂಟಿಕೊಂಡಿದೆಯೇ? ಅಂಟಿಕೊಂಡಿದ್ದರೂ ಅದನ್ನು ಸರಿ ಪಡಿಸಬಹುದು, ಸರಿಪಡಿಸುವ ವಿಧಾನಗಳಿವೆ. ಜಾತಿಮತ ಭೇದವಿಲ್ಲದೆ ಎಲ್ಲರಿಗೂ ಉಪಕಾರಿಯಾದ ಅಪ್ಪಟ ಭಾರತೀಯ ಸುಲಭ ವಿಧಾನವಿದೆ. ಅದೇ ಶ್ರದ್ಧೆಯ ವೃದ್ಧಿ.


\section*{ಶ್ರದ್ಧೆಯಿಂದ ಶಕ್ತಿವೃದ್ಧಿ!}

\addsectiontoTOC{ಶ್ರದ್ಧೆಯಿಂದ ಶಕ್ತಿವೃದ್ಧಿ!}

ಸಂಶಯರಹಿತ ದೃಢ ನಂಬಿಕೆಯ ಪ್ರಭಾವ ಪರಿಣಾಮಗಳು ರೋಗರುಜಿನಗಳ ಪರಿಹಾರ\-ದಲ್ಲಿ ಮಾತ್ರ ಉಪಯೋಗಕ್ಕೆ ಬರುವಂಥವುಗಳೇ? ಜೀವನದ ವಿಭಿನ್ನ ಕ್ಷೇತ್ರಗಳಲ್ಲಿ ವ್ಯಕ್ತಿಯ ಅಭ್ಯುದಯಕ್ಕೆ ನಂಬಿಕೆ ಹೇಗೆ ಸಹಕಾರಿಯಾಗಬಲ್ಲುದು ಎಂಬುದನ್ನು ನಾವು ತಿಳಿಯಲೇಬೇಕು. ನಂಬಿಕೆ ಅಥವಾ ಶ್ರದ್ಧೆ ಕೆಲಸ ಮಾಡುವ ಕ್ರಮ, ನಿಯಮ ಯಾವುದು? ಅದು ಬದುಕನ್ನು ಬೆಳಗುವ ವಿಧಾನ ವೈವಿಧ್ಯಗಳೇನು? ಅದರ ಸತ್ವಶಕ್ತಿಗಳೇನು? ತತ್ವ ರಹಸ್ಯಗಳೇನು?

ಶ್ರದ್ಧೆ ಎನ್ನುವುದು ಕೇವಲ ಧರ್ಮಕ್ಕೆ ಸೀಮಿತವೇ? ಶ್ರದ್ಧೆಯನ್ನು ಅದು ಧರ್ಮಸಂಬಂಧಿ ಯಾದುದರಿಂದ ಅಂಧಶ್ರದ್ಧೆ ಎಂದು ಹೆಸರಿಸಿ ಅಪಹಾಸ್ಯ ಮಾಡುವುದು ಬುದ್ಧಿವಂತರೆನಿಸಿ\break ಕೊಂಡವರ ಒಂದು ಚಟ. ಈ ಚಟದ ಮೂಲ ಪಶ್ಚಿಮದಲ್ಲಿ ಧಾರ್ಮಿಕರಿಗೂ ವೈಜ್ಞಾನಿಕರಿಗೂ ನಡೆದ ಚಕಮಕಿಯ ಅಂಧಾನುಕರಣೆಯಲ್ಲಿದೆ ಎಂಬುದನ್ನು ಈ ವಿಚಾರವಂತರಲ್ಲಿ ಹಲವರು ತಿಳಿದಿಲ್ಲ. ಶ್ರದ್ಧೆಗೆ ‘ಅಂಧ’ ಎನ್ನುವ ವಿಶೇಷಣ ಸೇರಿಸುವುದು, ಸಂಶಯಗ್ರಸ್ತತೆಯೇ ವೈಚಾರಿಕತೆ ಎನ್ನುವುದು ಆಧುನಿಕರಲ್ಲಿ ಹಲವರ ಒಂದು ಅವಿಚಾರಿತ ಫ್ಯಾಷನ್ ಆಗಿಬಿಟ್ಟಿದೆ. ಧರ್ಮದ ಹೆಸರಿನಲ್ಲಿ ಅಸಂಖ್ಯ ಅರ್ಥಹೀನ ಆಚಾರಗಳು ಇಲ್ಲವೆಂದಲ್ಲ. ಮೂಢನಂಬಿಕೆ ಮಾನವನ ಪ್ರಗತಿಗೆ ಮಹಾವೈರಿ ಎಂಬುದೂ ಸತ್ಯವೇ. ಆದರೆ ತನ್ನ ಕುದುರೆಗೆ ಮೂರೇ ಕಾಲು ಎಂದುವಾದಿಸುವ ಮನೋವೃತ್ತಿಯ ಮತಾಂಧತೆ ಮೂಢನಂಬಿಕೆಗಿಂತಲೂ ಬಲವಾದ ಶತ್ರು. ಶ್ರದ್ಧೆಯನ್ನು ಟೀಕಿಸುವ ಜನರು ಶ್ರದ್ಧೆಯ ಅರ್ಥ ಮಹತ್ವಗಳನ್ನು ತಿಳಿಯಲು ಯತ್ನಿಸುತ್ತಾರೆಯೆ? ಅದನ್ನು ಸರಿಯಾಗಿ ತಿಳಿಯುವ ಪ್ರಯತ್ನ ಮಾಡದೆ ಅದನ್ನು ನಿಂದಿಸಹೊರಟರೆ ಅದೊಂದು ಅಂಧ ವಿಮರ್ಶೆಯಾಗದೆ? ನಿಜವಾದ ಶ್ರದ್ಧೆಯು ಮಾಡುವ ಉಪಕಾರಕ ಪ್ರಭಾವವನ್ನು ತಿಳಿಯುವಷ್ಟು ಅವರು ತಮ್ಮ ಬುದ್ಧಿಶಕ್ತಿಯನ್ನು ಉಪಯೋಗಿಸಿದ್ದಾರೆಯೇ? ಶ್ರದ್ಧೆ ನೀಡುವ ಶಕ್ತಿಯ ಬದಲು ಅವರು ಜನರಿಗೆ ಬೇರಾವ ಶಕ್ತಿಯನ್ನು ನೀಡಲಾಪರು? ಶ್ರದ್ಧೆ ಜೀವನದ ಒಂದು ಪ್ರಮುಖ ಆವಶ್ಯಕತೆ. ಪ್ರಕೃತಿಯು ಒಂದು ನಿಯಮವನ್ನನುಸರಿಸಿ ನಡೆಯುತ್ತದೆ ಎಂಬ ಶ್ರದ್ಧೆಯಿಂದಲೇ ವಿಜ್ಞಾನಿ ಕೂಡ ಸಂಶೋಧನೆಗೆ ತೊಡಗುತ್ತಾನಷ್ಟೆ. ನಮ್ಮ ಆಸೆ ಆಸಕ್ತಿ ಆಕಾಂಕ್ಷೆಗಳು ಕಾರ್ಯಗತವಾಗಲೂ, ರೂಪುಗೊಳ್ಳಲೂ ಈ ಶ್ರದ್ಧೆಯ ಆಧಾರ ಅಡಿಪಾಯಗಳಿಲ್ಲದೆ ಸಾಧ್ಯವೇ? ಮೂಢನಂಬಿಕೆಯನ್ನೂ, ಅಂಧಶ್ರದ್ಧೆಯನ್ನೂ ನಿಂದಿಸುವುದು ಒಳ್ಳೆಯ ಕೆಲಸವೆ. ಆದರೆ ಮೂಢವಲ್ಲದ ನಿಜವಾದ ನಂಬಿಕೆ ಏನೆಂಬುದನ್ನು ತಿಳಿಸುವುದು ಇನ್ನೂ ಒಳ್ಳೆಯ ಕೆಲಸ. ಕತ್ತಲೆ ಕತ್ತಲೆ ಎಂದು ಕೂಗಾಡುತ್ತ ಸತ್ಯಾಗ್ರಹ ಕೈಗೊಳ್ಳುವುದಕ್ಕಿಂತ ಬೆಳಕನ್ನು ನೀಡಿದರೆ ಕತ್ತಲೆ ಪಲಾಯನ ಸೂತ್ರ ಪಠಿಸುವುದಲ್ಲವೆ? ಕತ್ತಲೆಯನ್ನು ದ್ವೇಷಿಸಿದರೆ ಸಾಲದು. ಬೆಳಕನ್ನೂ ಪ್ರೀತಿಸಬೇಕು. ಅಸತ್ಯವನ್ನು ಅಲ್ಲಗಳೆಯುವುದು ಶ್ಲಾಘ್ಯವೇ. ಆದರೆ ಸತ್ಯವನ್ನು ತೋರಿಸಿಕೊಡುವುದು ಬಹಳ ಉಪಯುಕ್ತವಾದ ಕೆಲಸ. ಕತ್ತಲನ್ನು ದ್ವೇಷಿಸುವವರು ಕತ್ತಲನ್ನೆ ತೀವ್ರವಾಗಿ ಚಿಂತಿಸುತ್ತ ಧ್ಯಾನಿ ಸುತ್ತ ಕತ್ತಲೆಡೆಗೇ ಹೆಚ್ಚು ಆಕರ್ಷಿತರಾಗುವ ಸಂಭವವಿದೆ. ಅಜ್ಞಾತವಾಗಿ ಕತ್ತಲಿನ ಪ್ರವಾದಿಗಳೂ ಆಗಿ ಪರಿವರ್ತಿತರಾಗಿ ಬಿಡಬಹುದು. ಆದ್ದರಿಂದ ಬೆಳಕಿನೆಡೆಗೇ ನಮ್ಮ ಗಮನ ಹರಿಸುವುದು ಲೇಸು.

ಶ್ರದ್ಧೆ ಒಂದು ಆತ್ಮಗುಣ. ಮನಸ್ಸಿನ ಒಂದು ಮಹಾಶಕ್ತಿ. ಅದರ ಉಪಯೋಗವನ್ನು ಎಲ್ಲರೂ ಎಲ್ಲ ಕಾಲದಲ್ಲೂ ಪಡೆಯಬಹುದು. ನಮ್ಮ ಆಯುಸ್ಸು, ಆರೋಗ್ಯ, ಉದ್ಯಮ, ಆನಂದ –ಈ ಎಲ್ಲವುಗಳ ಮೇಲೆ ಶ್ರದ್ಧೆಯ ಸತ್​ಪರಿಣಾಮ ಶತಃಸಿದ್ಧ. ವಿಶ್ವದಲ್ಲಿರುವ ಎಲ್ಲ ರಹಸ್ಯದ ಬಾಗಿಲುಗಳನ್ನು ತೆರೆಯುವ ಕೀಲಿಕೈ ಶ್ರದ್ಧೆ. ಪ್ರಪಂಚದ ಎಲ್ಲ ಶಕ್ತಿಗಳೂ ಹೊರ ಹೊಮ್ಮುವ ಮೂಲಸ್ರೋತವಾದ ಪರಮಾರ್ಥಕ್ಕೂ ಅದು ಪರಿಪೂರ್ಣವಾದ ಮಾರ್ಗ. ಅಪಾರ ಶಕ್ತಿಯನ್ನು ಸಂಪಾದಿಸಲು ಇರುವ ರಾಜಮಾರ್ಗ ಈ ಶ್ರದ್ಧೆಯೇ. ಸರ್ವನಾಶದ ಮಧ್ಯೆ ಆಶಾವಾದಿಯಾಗಿ ಅಲುಗಾಡದೇ ನಿಲ್ಲುವ ಸ್ಥೈರ್ಯವನ್ನು ನೀಡುವುದೂ ಈ ಶ್ರದ್ಧೆಯೇ! ನಿಜವಾದ ಶ್ರದ್ಧೆ ಇತ್ತೆಂದರೆ ದಿವ್ಯಜ್ಞಾನವನ್ನೂ ಪಡೆಯಬಹುದೆಂಬುದು ಭಗವದ್ಗೀತೆಯ ಒಂದು ಉಕ್ತಿ.

ಶ್ರದ್ಧೆಯ ಶಕ್ತಿಸ್ವರೂಪಗಳನ್ನು ತಿಳಿಸಿಕೊಡುವ ನೈಜ ಘಟನೆಗಳನ್ನು ಈಗ ಪರಿಶೀಲಿಸೋಣ. ಶ್ರದ್ಧೆಯ ಶಕ್ತಿಯಿಂದ ಶಕ್ತರಾಗಿ ಮೇಲಕ್ಕೇರಿದ ಶ‍್ರೀಸಾಮಾನ್ಯರನ್ನು ಕುರಿತ ಸ್ವಾಮಿ ವಿವೇಕಾನಂದರ ಸ್ವಂತ ಅನುಭವದ ಕಥನ ಇಲ್ಲಿದೆ: “ನಾನು ನ್ಯೂಯಾರ್ಕ್​ನಲ್ಲಿದ್ದಾಗ ಐರ್ಲೆಂಡಿನಿಂದ ಬರುವ ಪ್ಯಾಟ್​ನನ್ನು ನೋಡಿದ್ದೆ. ದಬ್ಬಾಳಿಕೆಗೆ ತುತ್ತಾಗಿ ಜೋಲುಮುಖದಿಂದ, ಮನೆಯಲ್ಲಿ ತನ್ನದೆಂಬ ಒಂದು ಸಾಮಾನೂ ಇಲ್ಲದೆ ತಲೆಗೊಂದು ಟೋಪಿಯೂ ಇಲ್ಲದ ನಿರ್ಗತಿಕನಾಗಿದ್ದ ಆತ. ಅವನ ದೇಹ ದುರ್ಬಲವಾಗಿ ಕಣ್ಣಿನಲ್ಲಿ ಭೀತಿ ಕಾಣಿಸುತ್ತಿತ್ತು. ಆದರೆ ಆರು ತಿಂಗಳಲ್ಲಿ ದೃಶ್ಯ ಸಂಪೂರ್ಣ ಬದಲಾಯಿಸಿತು. ಅಮೇರಿಕಾದಲ್ಲಿ ಅವನನ್ನು ಭೇಟಿಯಾದಾಗ ಆತನ ಕಣ್ಣಿನಲ್ಲಿ ಅಂಜಿಕೆಯ ಸುಳಿವೇ ಇಲ್ಲ. ಆತ ಧೈರ್ಯದಿಂದ ನಡೆದಾಡುತ್ತಿದ್ದ. ಇದಕ್ಕೇನು ಕಾರಣ? ಐರ್ಲೆಂಡಿನಲ್ಲಿದ್ದಾಗ ಸುತ್ತಲೂ ಅನಾದರಣೀಯ ವಾತಾವರಣ ಆವರಿಸಿತ್ತು. ಪ್ರಕೃತಿಯೆಲ್ಲ ಒಂದೇ ಧ್ವನಿಯಿಂದ, ‘ಪ್ಯಾಟ್, ನಿನಗೇನೂ ಗತಿಯಿಲ್ಲ. ನೀನು ಹುಟ್ಟು ಗುಲಾಮ. ನಿನಗೆ ಯಾವ ಅಭಿವೃದ್ಧಿಯೂ ಸಾಧ್ಯವಿಲ್ಲ’ ಎಂದು ಹೇಳುತ್ತಿತ್ತು. ಹುಟ್ಟಿದಾಗಿನಿಂದಲೇ ಅಂಥ ಮಾತುಗಳನ್ನು ಕೇಳಿದ ಪ್ಯಾಟನ ಮನಸ್ಸು ಅದನ್ನು ಸಂಪೂರ್ಣವಾಗಿ ಸ್ವೀಕರಿಸಿತು; ತಾನೊಬ್ಬ ಅತ್ಯಂತ ಹೀನ ನೆಂದು ತಿಳಿಯ ತೊಡಗಿದ\-ನಾತ. ಆದರೆ ಅಮೇರಿಕಾ ದೇಶಕ್ಕೆ ಬಂದೊಡನೆಯೇ ‘ಪ್ಯಾಟ್ ನಮ್ಮಂತೆಯೇ ಒಬ್ಬ ಮನುಷ್ಯ. ಇದನ್ನೆಲ್ಲ ಮಾಡಿದವನು ಮನುಷ್ಯ. ನಿನ್ನಂತಹ ಮನುಷ್ಯರು ಏನನ್ನು ಬೇಕಾದರೂ ಸಾಧಿಸಬಹುದು’ ಎಂಬ ಧ್ವನಿ ಸುತ್ತಲೂ ಅನುರಣಿತವಾಗಿತ್ತು. ಪ್ಯಾಟನು ತಲೆ ಎತ್ತಿ ನೋಡಿದ. ಅದು ನಿಜವಾಗಿತ್ತು. ಪ್ರಕೃತಿ ‘ಏಳು ಜಾಗ್ರತನಾಗು, ಗುರಿಯು ದೊರಕುವವರೆಗೂ ನಿಲ್ಲಬೇಡ’ ಎನ್ನುವಂತೆ ತೋರಿತು.

“ಹಿಂದೆ ಪ್ಯಾಟ್ ದೀನ ಹೀನ ದಯನೀಯ ಸ್ಥಿತಿಯಲ್ಲಿದ್ದ. ಅವನ ಹೃದಯವನ್ನು ಆತ್ಮಶ್ರದ್ಧೆ ಪ್ರವೇಶಿಸಿತು. ಅವನೊಬ್ಬ ಬೇರೆಯೇ ವ್ಯಕ್ತಿಯಾದ!

“ಅಮೇರಿಕದಲ್ಲಿ ನಡೆದ ಸರ್ವಧರ್ಮಸಮ್ಮೇಳನದಲ್ಲಿ ನೀಗ್ರೋ ಯುವಕನೊಬ್ಬ ಸುಂದರವಾದ ಭಾಷಣಮಾಡಿದ. ಅವನಲ್ಲಿ ಆಸಕ್ತನಾಗಿ ಹಲವು ವೇಳೆ ಅವನ ವಿಚಾರವನ್ನು ಕುರಿತು ಕೇಳಿದೆ. ಆದರೆ ಆತ ಏನನ್ನೂ ಹೇಳಲಿಲ್ಲ. ಮುಂದೆ ಇಂಗ್ಲೆಂಡಿಗೆ ಹೋದಾಗ ಕೆಲವು ಮಂದಿ ಆಫ್ರಿಕದ ಪ್ರಮುಖರು ಅವನ ವಿಚಾರ ತಿಳಿಸಿದರು. ಆ ಯುವಕ ನೀಗ್ರೋ ಮುಖಂಡನೊಬ್ಬನ ಮಗನಾಗಿದ್ದ. ಇನ್ನೊಂದು ಪಂಗಡದ ಮುಖಂಡನಿಗೂ ಈತನ ತಂದೆಗೂ ಜಗಳವಾಗಿ ಯುವಕನ ತಾಯಿಯನ್ನು ವಿರೋಧಿಗಳು ಕೊಂದು ಮಾಂಸ ಬೇಯಿಸಿ ತಿಂದೇಬಿಟ್ಟರು. ಈ ಯುವಕನನ್ನೂ ಕೊಲ್ಲುವಂತೆ ಆಜ್ಞಾಪಿಸಿದ್ದರು. ಆದರೆ ಆತ ಹೇಗೋ ತಪ್ಪಿಸಿಕೊಂಡು ನೂರಾರು ಮೈಲಿ ಓಡಿ ಕಡಲಕರೆಯನ್ನು ಸೇರಿದ. ಅಲ್ಲಿ ಒಂದು ಹಡಗಿನಲ್ಲಿ ಆಶ್ರಯ ಪಡೆದು ಅಮೇರಿಕಾ ದೇಶ ತಲುಪಿದ. ಅಲ್ಲಿ ವಿದ್ಯಾಭ್ಯಾಸ ಮಾಡಿ ಆತ ಅಂಥ ವಿದ್ವತ್ಪೂರ್ಣ ಉಪನ್ಯಾಸ ನೀಡುವ ಸಾಮರ್ಥ್ಯವನ್ನು ಸಂಪಾದಿಸಿದ್ದ!”

ತಲೆತಲಾಂತರದಿಂದ ಓದುಬರಹದ ಗಂಧಗಾಳಿ ಇಲ್ಲದ ಕುಟುಂಬದಿಂದ ಬಂದವನು ಆತ. ಅಧ್ಯಯನದ ಹೊಸ ಸಾಹಸಕ್ಕೆ ಬೇಕಾಗುವ ಆತ್ಮಶ್ರದ್ಧೆ ಅವನಿಗೆ ಆ ನೂತನ ವಾತಾವರಣದಲ್ಲಿ ದೊರೆತಿತ್ತು. ತನ್ನಿಂದ ಓದು ಖಂಡಿತ ಸಾಧ್ಯ ಎನ್ನುವ ಶ್ರದ್ಧೆ ಅವನನ್ನು ಓದಿನಲ್ಲಿ ಮುನ್ನಡೆಯಿಸಿತು. ಅವನ ಪ್ರತಿಭೆಯನ್ನು ಅರಳಿಸಿತು. ಅಂತರಂಗದ ಆಳದಲ್ಲಿ ಪ್ರತಿಯೊಬ್ಬರಲ್ಲೂ ಮಹಾಘನ ಇದೆ, ಅಪಾರ ಶಕ್ತಿ ಇದೆ. ಶ್ರದ್ಧೆ ಅದನ್ನು ಎಚ್ಚರಗೊಳಿಸುತ್ತದೆ.


\section*{ಪರಂಪರೆಯ ಪರಿಮಿತಿಯಿಂದ ಪಾರಾಗಲು}

\addsectiontoTOC{ಪರಂಪರೆಯ ಪರಿಮಿತಿಯಿಂದ ಪಾರಾಗಲು}

ಪರಂಪರೆ ಅಥವಾ ವಂಶವಾಹಿಗಳ ಮೂಲಕ ಎಲ್ಲ ಗುಣಗಳನ್ನೂ ಪಡೆದುಕೊಂಡು ಬಂದಿರ ಬೇಕು; ಇಲ್ಲವಾದರೆ ಯಾವುದೇ ಕ್ಷೇತ್ರದಲ್ಲಿ ಅತ್ಯುನ್ನತ ಮಟ್ಟದ ಸಿದ್ಧಿಯನ್ನು ಪಡೆಯಲು ಅಸಾಧ್ಯ ಎನ್ನುವ ನಂಬಿಕೆ ನಮ್ಮ ದೇಶದ ಹೆಚ್ಚಿನ ಜನರಲ್ಲಿ ಅಜ್ಞಾತವಾಗಿ ಹುಟ್ಟುಗಟ್ಟಿದಂತಿದೆ. ಮನೆಯಲ್ಲಿ ತಂದೆತಾಯಿಗಳೂ, ಶಾಲೆಯಲ್ಲಿ ಅಧ್ಯಾಪಕರೂ ತಿಳಿದೋ, ತಿಳಿಯದೆಯೋ ಈ ಒಂದು ವಿಚಾರವನ್ನು ಮಕ್ಕಳ ಮನಸ್ಸಿನಲ್ಲಿ ಪ್ರತ್ಯಕ್ಷವಾಗಿ ಪರೋಕ್ಷವಾಗಿ ಬಾಲ್ಯದಿಂದಲೇ ಬಿತ್ತುತ್ತ ಬರುತ್ತಾರೆ. ಕೆಲವರೇನೋ ಇದು ಮೂಢನಂಬಿಕೆ ಎಂದು ಭಾಷಣ ಹೊಡೆದರೂ ಈ ಮೂಢನಂಬಿಕೆಯ ಬಿಗಿಹಿಡಿತಗಳಿಂದ ಬಿಡಿಸಿಕೊಳ್ಳುವ ರಚನಾತ್ಮಕ ವಿಧಾನವನ್ನು ತೋರ ಲಾರರು. ಆದರೆ ಸತ್ಯ ಇದು: ಒಳ್ಳೆಯ ಆರೋಗ್ಯವಿರುವ ಮತ್ತು ಸಾಮಾನ್ಯ ಬುದ್ಧಿಶಕ್ತಿಯುಳ್ಳ ಮಕ್ಕಳೂ ಅತ್ಯುತ್ತಮ ಮಾರ್ಗ\-ದರ್ಶಕರ ಕೈಯಲ್ಲಿ ಶೈಶವದಿಂದಲೇ ಪಳಗಿದರೆ ಯಾವ ಕ್ಷೇತ್ರ ದಲ್ಲೇ ಆಗಲಿ ಜಗತ್ಪ್ರಸಿದ್ಧ ಮಹಾ ಮೇಧಾವಿಗಳಾಗದಿದ್ದರೂ ಉನ್ನತಮಟ್ಟವನ್ನು ಏರಬಲ್ಲರು. ಇಲ್ಲಿ ಅತ್ಯುತ್ತಮ ಮಾರ್ಗದರ್ಶಕ\-ರೆಂದರೆ ಮಹಾಮೇಧಾವಿಗಳೆಂದಾಗಲೀ ಅತಿಹೆಚ್ಚಿನ ಡಾಕ್ಟರೇಟುಗಳನ್ನು ಪಡೆದವರೆಂದಾಗಲಿ ಅಲ್ಲ. ಮಕ್ಕಳನ್ನು ತಿದ್ದುವಲ್ಲಿ ದಕ್ಷಳಾದ ತಾಯಿಯೊಬ್ಬಳ ಪ್ರೀತಿ ಮತ್ತು ತಾಳ್ಮೆಯನ್ನು ತೋರಬಲ್ಲ, ತಮ್ಮ ಅಧ್ಯಯನದ ವಿಷಯದಲ್ಲಿ ಅಭ್ಯಾಸಶೀಲರಾದ ಪರಿಣತರು ಅವರಾಗಿದ್ದರೆ ಸಾಕು. ಮಾತುಮಾತಿಗೆ ಸಿಟ್ಟಿಗೆದ್ದು ಬೈದು ಭಂಗಿಸುವ ತಾಳ್ಮೆಗೆಟ್ಟ ಅಧ್ಯಾಪಕರ ಕೈಯಲ್ಲಿ ಮಕ್ಕಳು ಸಿಕ್ಕಿಕೊಂಡರೆ ಅವರ ಪರಿಸ್ಥಿತಿ ಎಂತಾಗಬಹುದು? ಮಗುವಿನ ಭವಿಷ್ಯದ ಬುನಾದಿ ಎಂಥ ನಿಷೇಧಾತ್ಮಕ ಇಟ್ಟಿಗೆಗಳಿಂದ ಕಟ್ಟಿದಂತಾಗಬಹುದು?–ಎಂಬುದನ್ನು ವಿವರಿಸುವುದಕ್ಕಿಂತ ಕಲ್ಪಿಸಿಕೊಳ್ಳುವುದೇ ಲೇಸು. ಈ ದಿಸೆಯಲ್ಲಿ ಪ್ರಯೋಗಾತ್ಮಕ ಸಂಶೋಧನೆಗಳನ್ನು ಕೈಗೊಂಡು ಮಕ್ಕಳ ಪ್ರತಿಭೆಯನ್ನು ನಿರ್ದಿಷ್ಟ ಕ್ಷೇತ್ರದಲ್ಲಿ ಪರಿಪೂರ್ಣವಾಗಿ ಅರಳುವಂತೆ ಮಾಡಿದ ಮದರಾಸು ಆಕಾಶವಾಣಿಯ ಕಲಾವಿದ, ಗೋಟುವಾದ್ಯ ಪ್ರವೀಣ ಶ‍್ರೀ ನರಸಿಂಹನ್ ಅವರ ಪ್ರಯೋಗಾತ್ಮಕ ಅನುಭವ ಗಮನಾರ್ಹ. ಸಂಗೀತದ ಗಂಧಗಾಳಿ ಇಲ್ಲದ ಸಂಗೀತದ ದ್ವೇಷಿಗಳ ಮನೆಯಿಂದ ಬರುವ ಮಗುವನ್ನೂ\break ‘ಪ್ರತಿಭಾವಂತ ಗಾನಕಲಾಪರಿಣತ’ ನನ್ನಾಗಿ ಮಾಡಬಹುದು ಎನ್ನುತ್ತಾರವರು. ಆ ಮಗುವನ್ನು ಇನ್ನೂ ಮೂರು ತಿಂಗಳ ಶಿಶುವಾಗಿರುವಾಗಲೇ ಅವರ ಹತ್ತಿರ ತರಬೇತಿಗಾಗಿ ಬಿಡಬೇಕು. ‘ಹನ್ನೆರಡು ವರ್ಷಗಳ ಕಾಲ ಆಳವಾದ ಸಂಶೋಧನೆ ನಡೆಯಿಸಿ ಒಂದು ವಿಧಾನವನ್ನು ಕಂಡುಹಿಡಿದಿದ್ದೇನೆ. ಅದನ್ನು ಪ್ರಾಮಾಣಿಕವಾಗಿ ಕಾರ್ಯರೂಪಕ್ಕೆ ತರಲು ಯತ್ನಿಸಿದರೆ ಯಾವುದೇ ಕುಟುಂಬದಲ್ಲಿ ಮನೆಯಲ್ಲಿದ್ದುಕೊಂಡೇ ಸಂಗೀತದ ಮಹಾಪ್ರತಿಭೆ ಅರಳುವಂತೆ ಮಾಡಬಹುದು’ ಎನ್ನುತ್ತಾರೆ ನರಸಿಂಹನ್. ಅವರು ಹೇಳಿರುವ ಸೂಚನೆಗಳನ್ನು ಗಮನಿಸಿ:

‘ಪರಿಸರವು ಶಾಂತವೂ ಸಂತೋಷದಾಯಕವೂ ಆಗಿರಬೇಕು. ಮಗುವಿನ ಕೋಮಲವಾದ ಶ್ರವಣೇಂದ್ರಿಯಕ್ಕೆ ಕರ್ಕಶ ಶಬ್ದ ಬೀಳದಂತೆ ಜಾಗರೂಕತೆ ವಹಿಸಬೇಕು. ತನ್ನನ್ನು ಅತ್ಯಂತ ಪ್ರೀತಿಯಿಂದ ನೋಡಿಕೊಳ್ಳುತ್ತಿದ್ದಾರೆ ಎಂದು ಮಗುವಿಗೆ ಅನುಭವವಾಗಬೇಕು. ಕಟುನುಡಿಗಳಿಂದ ಬೈದುಭಂಗಿಸುವುದಾಗಲೀ, ಸಣ್ಣ ಪುಟ್ಟ ತಪ್ಪುಗಳನ್ನು ಎತ್ತಿತೋರಿ ಅಪಹಾಸ್ಯ ನಿಂದೆಯನ್ನಾಗಲಿ ಮಾಡಬಾರದು. ಮಗುವು ನಿದ್ರೆಯಿಂದ ಎದ್ದಾಗ, ಆಹಾರ ಸೇವಿಸುವಾಗ, ನಿದ್ರಿಸ ಹೊರಟಾಗ ಮೆಲ್ಲಮೆಲ್ಲನೆ ಸರಳವಾದ ರಾಗಗಳನ್ನು ಉಸುರಬೇಕು. ಆದರೆ ಒತ್ತಾಯ, ಬಲವಂತ ಕೂಡದು. ವೇಗವನ್ನೆಂದಿಗೂ ತೀವ್ರಗೊಳಿಸಬಾರದು. ಮೂರು ತಿಂಗಳಲ್ಲಿ ಮಗುವು ಸಾಮಾನ್ಯ ರಾಗಗಳಿಗೆ ಪ್ರತಿಕ್ರಿಯೆ ವ್ಯಕ್ತಪಡಿಸತೊಡಗುತ್ತದೆ. ಆಗಾಗ ಒಳ್ಳೆಯ ಸಂಗೀತವನ್ನು ಕೇಳಿಸಬೇಕು. ಒಮ್ಮೊಮ್ಮೆ ಮಗು ಆಟವಾಡುತ್ತಿದ್ದಾಗ ಹಾಸ್ಯಗಾರರಂತೆ ಹೊಸ ರಾಗವನ್ನು ಸ್ವಲ್ಪ ವಿಚಿತ್ರವಾಗಿ ಹಾಡಬಹುದು. ಮಗು ಆಸಕ್ತಿಯನ್ನು ತೋರಿ ನಗುತ್ತದೆ. ಮತ್ತೆ ಮತ್ತೆ ಅದೇ ರಾಗವನ್ನು ಹಾಗೆಯೇ ಹಾಡಿ ತೋರಿಸುವಂತೆ ಹೇಳುತ್ತದೆ. ಅಭ್ಯಾಸವೇನೋ ಆವಶ್ಯಕ. ಆದರೆ ಗದರಿಸಿ ಒತ್ತಾಯದಿಂದ ಮಾಡಿಸುವ ಯಾಂತ್ರಿಕ ಕೆಲಸವಾಗಬಾರದು. ಈ ವಿಧಾನದಿಂದ ನಿಜವಾಗಿಯೂ ಬಾಲಕನು ಸಂಗೀತ ಕಲೆಯಲ್ಲಿ ಬಹಳ ಉನ್ನತಮಟ್ಟವನ್ನು ತಲುಪಲು ಸಾಧ್ಯ.’

ಚೈತನ್ಯದ ಚಿಲುಮೆಯಾದ ಯುವಜನರ ಉತ್ಸಾಹ ಕಾರ್ಯಶಕ್ತಿಗಳನ್ನು ಯೋಗ್ಯ ಮತ್ತು ಉಪಕಾರವಾದ ಮಾರ್ಗದಲ್ಲಿ ಹರಿಯುವಂತೆ ಮಾಡಿ ಅವರಲ್ಲಿ ಅಪಾರ ಆತ್ಮವಿಶ್ವಾಸವನ್ನು\break ಮೂಡಿಸುವ ರಹಸ್ಯ ಕಲೆ ಇಲ್ಲಿದೆಯಲ್ಲವೆ?

ಪ್ರತಿಯೊಬ್ಬ ವ್ಯಕ್ತಿಯಲ್ಲೂ ಅಡಗಿರುವ ಶಕ್ತಿಯನ್ನು ಎಚ್ಚರಿಸುವ ವಿಧಾನದ ರಹಸ್ಯ ಇಲ್ಲಿ ಬಯಲಾಗಿದೆಯಲ್ಲವೆ?

ಯಾವುದೇ ಒಂದು ವಿಶಿಷ್ಟ ಕ್ಷೇತ್ರದಲ್ಲಿ ಸಿದ್ಧಿ ಅಥವಾ ಪರಿಣತಿಯನ್ನು ಪಡೆಯಲು ಈ ವಿಧಾನ ಉಪಯುಕ್ತವಲ್ಲವೆ?

ಶ್ರದ್ಧೆಯ ಚಿಕಿತ್ಸೆ ಎಲ್ಲರಿಗೂ ಬೇಕು. ಬೆಳೆಯುತ್ತಿರುವ ಮಕ್ಕಳಿಗೆ, ಹಿಂದುಳಿದವರಿಗೆ ವಿಶೇಷವಾಗಿ ಬೇಕು. ಎಲ್ಲರೂ ಅದರ ಬಲದಿಂದ ಎತ್ತರವನ್ನು ಏರಬಲ್ಲರು, ಸರ್ವತೋಮುಖವಾದ ಪ್ರಗತಿಯನ್ನು ಹೊಂದಬಲ್ಲರು. ಆದರೆ ಅದನ್ನು ಪಡೆಯಲು ಡಾ.\ ತಾಳ್ಮೆ, ಡಾ.\ ಒಲ್ಮೆ–ಇವರಿಬ್ಬರ ಹೃತ್ಪೂರ್ವಕ ಸಹಕಾರ ಬೇಕು. ಅಷ್ಟೇ ಅಲ್ಲ, ಡಾ.\ ಶಾಂತಮ್ಮ ವಾತಾವರಣ್ ಅವರ ನಿರಂತರ ಸಾನ್ನಿಧ್ಯವಿರಬೇಕು. ಇವರ ಮೂಲಕ ಬಾಲ್ಯದಿಂದಲೇ ಈ ಚಿಕಿತ್ಸೆ ಪಡೆಯುವಂತಾದರೆ ವ್ಯಕ್ತಿತ್ವ ಪೂರ್ಣರೀತಿಯಿಂದ ವಿಕಸಿತವಾಗುವ ಸಾಧ್ಯತೆಯಿದೆ ಎಂಬುದರಲ್ಲಿ ಸ್ವಲ್ಪವೂ ಸಂದೇಹವಿಲ್ಲ.

\medskip


\section*{ಮಕ್ಕಳ ಲಾಲನೆ ಪಾಲನೆ}

\addsectiontoTOC{ಮಕ್ಕಳ ಲಾಲನೆ\break ಪಾಲನೆ}

ಮಕ್ಕಳ ಬೆಳವಣಿಗೆಯ ಬಗೆಗೆ (ಅವರ ಜೊತೆಯಲ್ಲೇ ಇದ್ದುಕೊಂಡು) ಪ್ರಯೋಗಾತ್ಮಕ ಅಧ್ಯಯನ ಮಾಡಿ ಸಹಸ್ರಾರು ಮಕ್ಕಳ ಬದುಕನ್ನು ಬೆಳಗಿದ ರಷ್ಯಾದೇಶದ ಬಾಲಶಿಕ್ಷಣತಜ್ಞ ಸುಖೋಮ್ಲಿಂಸ್ಕಿ ಅವರ ಆಳ ಅನುಭವದ ಮಾತುಗಳಲ್ಲಿ ಕೆಲವನ್ನು ಇಲ್ಲಿ ಸಂಗ್ರಹಿಸಿದ್ದೇನೆ. ತಮ್ಮ ಮಕ್ಕಳ ಅಭ್ಯುದಯಾಕಾಂಕ್ಷಿಗಳು ಗಮನಿಸಬೇಕಾದ ಉಪಯುಕ್ತ ವಿಚಾರಗಳು ಇವು–

\vskip 4.5pt

‘ತಾನು ಕೈಗೊಂಡ ಕೆಲಸದಲ್ಲಿ ಗೆಲುವಿನ ಆಸೆ ಕಂಡುಬರದಿದ್ದರೆ ಮಗುವಿನ ಜ್ಞಾನದ ಹಂಬಲವನ್ನು ಹತ್ತಿಕ್ಕಿದಂತಾಗುತ್ತದೆ. ತೀವ್ರತರದ ಕಟುಮನೋಭಾವ, ನೋವು ಅದರ ಹೃದಯವನ್ನು ಆವರಿಸುತ್ತದೆ. ತಿರುಗಿ ಅಭಿರುಚಿ ಉತ್ಸಾಹಗಳುಂಟಾದರೆ ಮಾತ್ರ ಆ ಕಟು ಮನೋಭಾವ ದೂರವಾಗಬಹುದು. ಆದರೆ ಪುನಃ ಅಭಿರುಚಿ, ಉತ್ಸಾಹ ಅಧ್ಯಯನದ ಹಂಬಲ–ಇವುಗಳನ್ನುಂಟು ಮಾಡುವುದು ಅತಿ ಕಷ್ಟದ ಕೆಲಸ.\footnote{\engfoot{If a child sees no hope of success in his work, his eagerness for knowledge will be stifled, and cold bitterness will grip a child's heart which no effort whatsoever will be able to thaw out until the spark of eagerness lights up again (and kindling it a second time is an infinitely difficult task); a child loses faith in his capacity, shuts up like a clam, becomes wary and prickly, responds with brazen resentment to advice and emonstrations from his teachers. Or worse still–his sense of dignity is undermined and he reconciles himself to the thought that he has no real ability. My heart is always filled with anger and indignation, when I see one of these apathetic, resigned children who is ready to listen patiently to a teacher's exhortations for hours at a stretch and completely indifferent to the words of his classmates; when they reproach him with lagging behind or repeating a class... There is nothing more immoral than kill another person's sense of dignity!}

\engfoot{Interest in learning is only to be found where there is inspiration born of success.}\hfill\engfoot{ –V. Sukhomlinsky, \textit{On Education}}}

\vskip 4.5pt

‘ತನ್ನ ಶಕ್ತಿ ಸಾಮರ್ಥ್ಯಗಳಲ್ಲಿ ಮಗುವು ಒಮ್ಮೆ ಶ್ರದ್ಧೆಯನ್ನು ಕಳೆದುಕೊಂಡರೆ ಅದು ತನ್ನ ಮನಸ್ಸಿನ ಬಾಗಿಲುಗಳನ್ನು ಮುಚ್ಚಿಕೊಳ್ಳುತ್ತದೆ. ಅತಿ ನಾಜೂಕು ಮನೋಭಾವನೆಯನ್ನು\break ತಳೆಯುತ್ತದೆ. ಅಧ್ಯಾಪಕರ ಸೂಚನೆ ಬುದ್ಧಿವಾದ ಭರ್ತ್ಸನೆಗಳಿಗೆ ದುರಭಿಮಾನ ಮತ್ತು ಕೋಪಗಳಿಂದ ಪ್ರತಿಕ್ರಿಯೆ ತೋರುತ್ತದೆ.’

‘ನೀವು ಮಗುವಿನಲ್ಲಿ ಹೆಚ್ಚು ಹೆಚ್ಚು ಅಪನಂಬಿಕೆಯನ್ನು ತೋರಿಸಿದಂತೆ ಅದು ಮೋಸ ಮತ್ತು ವಂಚನೆಯ ಕಲೆಗಳಲ್ಲಿ ಇನ್ನೂ ದಕ್ಷನಾಗುತ್ತದೆ. ಅದರ ಮನಸ್ಸು ಆಲಸ್ಯ ಮತ್ತು ಪ್ರತಿಯೊಂದನ್ನು ಅಲಕ್ಷ್ಯದಿಂದ ನೋಡುವ ಪ್ರವೃತ್ತಿಯ ತಾಣವಾಗುತ್ತದೆ. ಮಕ್ಕಳಲ್ಲಿ ಅಪನಂಬಿಕೆಯಿಂದ ನೇರವಾಗಿ ಹೊರಹೊಮ್ಮುವ ಫಲವೇ ಆಲಸ್ಯ.’

‘ತಿರಸ್ಕಾರವು ಮನುಷ್ಯರ ಸುಪ್ತಮನಸ್ಸಿನ ಆಳದಿಂದ ಒರಟಾದ, ಕೆಲವೊಮ್ಮೆ ಅತ್ಯಂತ ಪಾಶವಿಕ ಪ್ರವೃತ್ತಿಗಳನ್ನು ಹೊರತರುತ್ತದೆ.’

‘ಅಧ್ಯಾಪಕರು ಒಮ್ಮೆ ಮಗುವಿನ ನಿಜವಾದ ಸ್ನೇಹಿತರಾದರೆ ಮತ್ತು ಆ ಸ್ನೇಹವು ಉದಾತ್ತ ಹಾಗೂ ಮನಸೆಳೆಯುವ ಅಭಿರುಚಿಯ ವಿಷಯದಲ್ಲಿ ರಚನಾತ್ಮಕ ಕಾರ್ಯಕ್ಕೆ ಪ್ರೇರಕವಾದರೆ ಮಗುವಿನ ಮನಸ್ಸು ಕೆಟ್ಟ ಯೋಚನೆಗಳಿಂದ ದೂರವಾಗುತ್ತದೆ.’

ಕಲಿಯಬೇಕು ಎಂಬ ಹಂಬಲ ಮಗುವಿನ ಮನಸ್ಸಿನಲ್ಲಿ ಉದಿಸದಿದ್ದರೆ, ಉತ್ಸಾಹ ಉಂಟಾಗದಿದ್ದರೆ ವಿದ್ಯಾಭ್ಯಾಸದ, ಏಕೆ, ನಮ್ಮ ಸಮಸ್ತ ಯೋಜನೆಗಳೂ ನಿಷ್ಫಲವಲ್ಲವೆ?

ಚೆನ್ನಾಗಿ ಓದಿ ಬರೆದು ಎಸ್.ಎಸ್.ಎಲ್.ಸಿ.ಯಲ್ಲಿ ಪ್ರಥಮ ದರ್ಜೆಯಲ್ಲಿ ಪಾಸಾಗಿ ಕಾಲೇ ಜನ್ನು ಸೇರಿ ಪದವಿ ಪರೀಕ್ಷೆಗೆ ವಾಣಿಜ್ಯೋದ್ಯಮ ವಿಷಯವನ್ನು ಆರಿಸಿಕೊಂಡ ವಿದ್ಯಾರ್ಥಿಯ ಪರಿಚಯ ನನಗಿತ್ತು. ಒಮ್ಮೆ ‘ನಿನ್ನ ಓದು ಹೇಗಿದೆ?’ ಎಂದು ಕೇಳಿದಾಗ ಆತ ‘ಬಹಳ ಬೇಜಾರು, ಬೋರು, ಸ್ವಾಮೀಜಿ. ಉಳಿದ ನನ್ನ ಕ್ಲಾಸಿನ ವಿದ್ಯಾರ್ಥಿಗಳು ಬೇಗಬೇಗನೇ ಬ್ಯಾಲೆನ್ಸ್​ಶೀಟ್ ತಯಾರುಮಾಡುತ್ತಾರೆ. ನನಗೋ ತಲೆಚಿಟ್ಟು ಹಿಡಿದು ಹೋಗುತ್ತದೆ. ನನ್ನ ಮೇಲೇ ನನಗೆ ಬೇಸರ ಬಂದು, ಏಕಪ್ಪಾ ಈ ವಿಷಯ ತೆಗೆದುಕೊಂಡೆ. ಸತ್ತೇ ಹೋದರೂ ವಾಸಿ ಎನ್ನಿಸುತ್ತದೆ’ ಎಂದು ತನ್ನ ಆಂತರ್ಯದ ಅಳಲನ್ನು ನೇರವಾಗಿ ತೋಡಿಕೊಂಡಿದ್ದ. ‘ನೀನು ಪ್ರಥಮ ದರ್ಜೆಯ ವಿದ್ಯಾರ್ಥಿಯಾಗಿದ್ದಿ ಎಂಬುದನ್ನು ಮರೆಯಬೇಡ. ಇತರ ವಿದ್ಯಾರ್ಥಿಗಳ ಹಿನ್ನೆಲೆ, ಪರಂಪರೆ, ತರಬೇತಿ ಏನೆಂದು ನಿನಗೆ ತಿಳಿಯದು. ಸುಮ್ಮನೆ ಅವರ ಈಗಿನ ದಕ್ಷತೆಯನ್ನು ಮಾತ್ರ ನೋಡಿ ನಿನ್ನ ಸದ್ಯದ ಸ್ಥಿತಿಯನ್ನು ಹೋಲಿಸಿ ಹೆದರಿ ಆತ್ಮವಿಶ್ವಾಸ ಕಳೆದುಕೊಳ್ಳಬೇಡ. ಸುಲಭದಿಂದ ಕಠಿಣದೆಡೆಗೆ ಹೆಜ್ಜೆಹೆಜ್ಜೆಯಾಗಿ ಮೇಲೇರಬೇಕು, ಪ್ರಯತ್ನಿಸಿ ನೋಡು,’ ಎಂದು ಧೈರ್ಯ ಹೇಳಿದೆ. ಅವನು ಮುಂದೆ ಪದವಿ ಪರೀಕ್ಷೆಯಲ್ಲಿ ಎರಡನೇ ದರ್ಜೆಯಲ್ಲಿ ಹೆಚ್ಚು ಅಂಕಗಳನ್ನೇ ಪಡೆದು ಪಾಸಾದ. ಆದರೆ ಇತರ ವಿದ್ಯಾರ್ಥಿಗಳ ಯಶಸ್ಸು ದಕ್ಷತೆಗಳನ್ನು ಕಂಡು ಮೊದಲು ಹೆದರಿ ಕೊಂಡಿದ್ದ. ಪರೀಕ್ಷೆಯ ಭೂತವನ್ನು ಹೇಗೆ ಎದುರಿಸಬೇಕೆಂಬುದನ್ನು ನಮ್ಮ ವಿದ್ಯಾರ್ಥಿಗಳಿಗೆ ತಿಳಿಸಿಕೊಡುವವರಾರು? ಇದನ್ನು ಪ್ರಾರಂಭದಿಂದಲೇ ತಿಳಿಸಿಕೊಡಬೇಕಲ್ಲವೇ?


\section*{ಶ್ರದ್ಧಾವಾನ್ ಲಭತೇ ಜ್ಞಾನಮ್​}

\addsectiontoTOC{ಶ್ರದ್ಧಾವಾನ್ ಲಭತೇ ಜ್ಞಾನಮ್​}

ಇಂದು ಭಾರತದ ಎಲ್ಲ ರಂಗಗಳಲ್ಲೂ ನೇರದಾರಿಗಿಂತ ಒಳದಾರಿಗಳೇ ಹೆಚ್ಚಾಗಲು ಕಾರಣವೇನು? ವಿದ್ಯಾವಂತರೆನಿಸಿಕೊಂಡ ಜನರಲ್ಲಿ ಅವಿದ್ಯಾವಂತರೆನಿಸಿಕೊಂಡ ಜನರಿಗಿಂತ ಹೆಚ್ಚಿನ ಪ್ರಾಮಾಣಿಕತೆ ಕಾಣಿಸುತ್ತದೆ ಎನ್ನಬಲ್ಲೆವೆ? ಇದಕ್ಕೆ ಮೂಲಕಾರಣವೇನಿರಬಹುದು? ಬಾಲ್ಯದಿಂದಲೇ, ಪ್ರಾಥಮಿಕ ತರಗತಿಗಳಿಂದಲೇ ನಮ್ಮ ಯುವಕರು ಬೆಳೆಸಿಕೊಂಡು ಬಂದ ಸಣ್ಣ ಪುಟ್ಟ ವಿಜಯವನ್ನು ಪಡೆಯದ ಸೋಲಿನ ಮನೋಭಾವ, ಆತ್ಮವಿಶ್ವಾಸಹೀನತೆ, ಭವಿಷ್ಯದ ಭಯ, ಸುರಕ್ಷಿತತೆಯ ಅಭಾವ, ದುಡಿಮೆಯಿಲ್ಲದ ಸಿನೆಮಾ ಕಲ್ಪನೆಯ ವಿಲಾಸ ಜೀವನಾಕಾಂಕ್ಷೆ–ಇವು ಒಂದು ಕಾರಣವಂತೂ ಹೌದು. ಈ ಆತ್ಮಶ್ರದ್ಧೆಯ ಅಭಾವ ನಮ್ಮನ್ನು ಅಧಃಪತನದೆಡೆಗೆ ಹೇಗೆ ಕೊಂಡೊಯ್ಯುತ್ತಿದೆ ಎಂಬುದನ್ನು ಯೋಚಿಸಲೂ ಅಸಾಧ್ಯವಾಗಿದೆ ನಮಗೆ!

ಸ್ವಾಮಿ ವಿವೇಕಾನಂದರು ಆತ್ಮಶ್ರದ್ಧೆಯನ್ನು ರೂಢಿಸಿಕೊಳ್ಳುವ ಆವಶ್ಯಕತೆಯನ್ನು ಮತ್ತೆ ಮತ್ತೆ ಬೋಧಿಸಿದ್ದರು: ‘ಆತ್ಮಶ್ರದ್ಧೆಯ ಆದರ್ಶ ನಮ್ಮ ಬದುಕಿಗೆ ಅತ್ಯಂತ ಉಪಕಾರಿಯಾದದ್ದು. ನಮ್ಮಲ್ಲಿ ನಮಗೆ ಶ್ರದ್ಧೆಯುಂಟಾಗುವಂತೆ ಬೋಧಿಸಿ ಅನುಷ್ಠಾನ ಮಾಡಿಸಿದ್ದರೆ ಪ್ರಪಂಚದಲ್ಲಿ ಕಾಣಬರುವ ದುಷ್ಟತನ, ದುರಂತಗಳು ದೂರವಾಗುತ್ತಿದ್ದವು ಎಂಬುದರಲ್ಲಿ ನನಗೆ ಸ್ವಲ್ಪವೂ ಸಂದೇಹವಿಲ್ಲ. ಮಾನವನ ಇತಿಹಾಸದಲ್ಲೆಲ್ಲಾ ಕಂಗೊಳಿಸುವ ಮಹಾಪುರುಷರ ಜೀವನದಲ್ಲಿ ಎಲ್ಲಕ್ಕಿಂತ ಮಿಗಿಲಾಗಿ ಅವರ ಅದ್ಭುತ ಕಾರ್ಯಶಕ್ತಿಗೆ ಪ್ರೇರಣೆಯಾಗಿ ಕೆಲಸ ಮಾಡಿದುದು ಈ ಆತ್ಮಶ್ರದ್ಧೆಯೇ.\break ಮಹತ್ತಾದುದೇನನ್ನಾದರೂ ಸಾಧಿಸಬೇಕೆಂಬ ಪ್ರಜ್ಞೆಯಿಂದಲೇ ಉದಿಸಿದ ಅವರು ಮಹತ್ವಿಕೆ\-ಯನ್ನೇ\-ರಿದರು. ಒಬ್ಬ ಮನುಷ್ಯನು ನೀಚತೆಯ ರಸಾತಳಕ್ಕೆ ಬೇಕಾದರೆ ಹೋಗಲಿ, ಅವನ ಬದುಕಿನಲ್ಲಿ ಒಂದಲ್ಲ ಒಂದು ದಿನ ಬರುವುದು–ಆಗ ಇನ್ನೂ ಆಳಕ್ಕೆ ಇಳಿಯಲಾರದ ಹತಾಶೆಯಿಂದಲೇ ಅವನು ಹಿಂದಿರುಗಿ ಮೇಲಕ್ಕೇರಿ ತನ್ನಲ್ಲಿ ತಾನು ಶ್ರದ್ಧೆಯನ್ನಿಡಲು ಕಲಿಯುವನು. ಆದರೆ ಇದನ್ನು ಮೊದಲೇ ತಿಳಿದುಕೊಂಡಿದ್ದರೆ ಬಹಳಷ್ಟು ಉಪಕಾರವಾಗುವುದು. ಶ್ರದ್ಧೆಯ ಈ ಮಹಾಶಕ್ತಿಯನ್ನು ತಿಳಿದು ಅದನ್ನು ನಮ್ಮದಾಗಿ ಮಾಡಿ ಕೊಳ್ಳಲು ದುಃಖಕ್ಲೇಶಗಳಿಂದ ಕೂಡಿದ ಅನುಭವಗಳನ್ನು ಪಡೆಯುತ್ತ ಹೋಗಬೇಕೇ? ಮನುಷ್ಯ ಮನುಷ್ಯರಲ್ಲಿರುವ ವ್ಯತ್ಯಾಸಕ್ಕೆ ಶ್ರದ್ಧೆಯ ತಾರತಮ್ಯವೇ ಕಾರಣ. ಜನರ ಹೃದಯದಲ್ಲಿ ಪುನಃ ಈ ಶ್ರದ್ಧೆಯನ್ನು ಪ್ರತಿಷ್ಠಾಪಿಸಬೇಕು.’

ನಮ್ಮಲ್ಲಿ ಪ್ರತಿಯೊಬ್ಬರಲ್ಲೂ ಅಪಾರಶಕ್ತಿ ಅಡಗಿದೆ. ಆದರೆ ನಾವೇ ನಮ್ಮ ಸೀಮಿತ ಕಲ್ಪನೆಗಳಿಂದ ನಮ್ಮ ಮೇಲೆ ವಶೀಕರಣ ಸುಪ್ತಿಯ ಪ್ರಭಾವವನ್ನು ಬೀರಿಕೊಂಡಂತಾಗಿದೆ. ಆತ್ಮಶ್ರದ್ಧೆ ಅಥವಾ ನಮ್ಮಲ್ಲಿ ನಮಗಿರುವ ಶ್ರದ್ಧೆಯೇ ಆ ಶಕ್ತಿಯನ್ನು ಚಿಮ್ಮಿಸಿ ಮಹತ್ಕಾರ್ಯದ ಸಾಧಕರು\break ನಾವಾಗುವಂತೆ ಮಾಡುತ್ತದೆ.


\section*{ಬದುಕಿಗೊಂದು ಬೆಳಕು}

\vskip -5pt\addsectiontoTOC{ಬದುಕಿಗೊಂದು ಬೆಳಕು}

ನಮ್ಮ ನೈಜಸ್ವರೂಪ ಸ್ವಭಾವಗಳ ತಿಳಿವು ಹೆಚ್ಚು ಸ್ಪಷ್ಟವಾಗಿ ಮನವರಿಕೆಯಾಗುವವರೆಗೆ ನಮ್ಮ ಬೆಳವಣಿಗೆಗೆ ನಾವೇ ತೊಂದರೆಗಳನ್ನು ಹೇಗೆ ತಂದುಕೊಳ್ಳುತ್ತೇವೆಂಬುದು ಈ ಅಧ್ಯಾಯವನ್ನು ಓದಿದವರಿಗೆ ತಿಳಿಯುತ್ತದೆ.

ದಿವ್ಯತೆ ನಮ್ಮ ಆಂತರ್ಯದ ಆಳದಲ್ಲೇ ಇದೆ. ಅದರ ಅನುಭವ ಪಡೆಯುವುದಕ್ಕಾಗಿ ಮಾಡುವ ಪ್ರಯತ್ನ ಮತ್ತು ಪಾಲಿಸುವ ನಿಯಮಗಳಲ್ಲಿಯೇ ಮನುಷ್ಯನ ಮನಸ್ಸಿನ ಪುನರ್ ವ್ಯವಸ್ಥೆ ಮತ್ತು ಸಮಾಜದ ವಿಭಿನ್ನ ವ್ಯಕ್ತಿಗಳ ಮಧುರ ಸಂಬಂಧದ ರಹಸ್ಯ ಅಡಗಿದೆ. ಎಲ್ಲ ಧರ್ಮಗಳ ಮೂಲವಿರುವುದೂ ಅಲ್ಲಿಯೆ. ಸೌಧವೇರುವವನು ಕೆಳಗಿನ ಮೆಟ್ಟಲುಗಳನ್ನು ದಾಟಿಯೇ ಮೇಲೇರಿ ತುತ್ತತುದಿಯನ್ನು ಸೇರುವಂತೆ ಉದಾತ್ತ ಗುರಿಯ ಸಾಧನೆಗಾಗಿ ಸಾಮಾನ್ಯ ಆಸೆ ಆಸಕ್ತಿಗಳನ್ನು ಕ್ರಮವಾಗಿ ತೊರೆಯುವುದೇ ತ್ಯಾಗ. ಸತ್ಯವನ್ನು ತಿಳಿಯಬೇಕಾದರೆ ಅಸತ್ಯವನ್ನು ತ್ಯಜಿಸಬೇಕು. ಒಳ್ಳೆಯದು ಬೇಕಾದರೆ ಕೆಟ್ಟದ್ದನ್ನು ತ್ಯಜಿಸಬೇಕು. ನಿತ್ಯವು ಬೇಕಾದರೆ ಅನಿತ್ಯವನ್ನು ತ್ಯಜಿಸಬೇಕು. ನಮ್ಮೆಲ್ಲರ ನಿಜದ ನೆಲೆಯೂ ಸ್ವರೂಪವೂ ಆದ ದಿವ್ಯತೆಯನ್ನು ಪಡೆಯಲಿಚ್ಛಿಸುವವನು ಅಂತರ್ಮುಖಿಯಾಗಲೇಬೇಕಾಗುವುದು. ಬಹಿರ್ವ್ಯಾಪಾರಗಳನ್ನು ಸಾವಕಾಶವಾಗಿ ಯಾದರೂ ಕಡಿಮೆಮಾಡಿಕೊಳ್ಳಬೇಕಾಗುವುದು. ದಿವ್ಯತೆಯ ಗುರಿಯನ್ನು ಮರೆಯದೆ ಮನಸ್ಸಿನ ಎದುರಲ್ಲಿ ಇರಿಸಿಕೊಂಡು ಎಲ್ಲ ಲೌಕಿಕ ಕಾರ್ಯಗಳನ್ನು ಅನಾಸಕ್ತಿಯಿಂದ ನೆರವೇರಿಸಲು ಸಾಧ್ಯ. ಬಹುಪಾಲು ಮಾನವರು ಸೋಪಾನಕ್ರಮದಿಂದ ಮೇಲೇರಲು ಈ ದಾರಿಯಲ್ಲೇ ಮುನ್ನಡೆಯ ಬೇಕು. ಸ್ವಾಮಿ ವಿವೇಕಾನಂದರನ್ನು ಒಮ್ಮೆ ಸನಾತನ ಧರ್ಮವನ್ನು ಅತ್ಯಂತ ಸಂಕ್ಷೇಪವಾಗಿ ವಿವರಿಸಿ ಎಂದಾಗ ಅವರು ಎರಡು ಶಬ್ದಗಳನ್ನು ಉಚ್ಚರಿಸಿದರು. ಅವು ಇಂತಿವೆ: ‘ಪ್ರವೃತ್ತಿ ಮತ್ತು ನಿವೃತ್ತಿ.’ ಭಗವಂತನ ರಾಜ್ಯದಲ್ಲಿ ಆತನ ನಿಯಮಗಳನ್ನುಲ್ಲಂಘಿಸದೆ ಜಗತ್ತು ನೀಡುವ ವೈಭವಗಳನ್ನೂ ಸುಖಭೋಗಗಳನ್ನೂ ಅನುಭವಿಸಬಹುದು ಅಥವಾ ನೇರವಾಗಿ ಅನಂತ ಶಾಂತಿ – ಆನಂದಗಳ ಆಗರವಾದ ಆತನನ್ನೇ ತ್ಯಾಗಪಥವನ್ನು ಅನುಸರಿಸಿ ಸೇರಬಹುದು ಎಂಬುದು ಆ ಮಾತಿನ ಅರ್ಥ.

ಅವರೆನ್ನುವಂತೆ, ‘ಮನುಷ್ಯರಲ್ಲಿ ಅಡಗಿರುವ ಅಪಾರ ಶಕ್ತಿಯನ್ನು ಹೊರಜಗತ್ತಿನೆಡೆಗೆ ಕ್ರಮ ವರಿತು ಹರಿಯಿಸಿದರೆ ವೈಜ್ಞಾನಿಕ, ಭೌತಿಕ ಬೆಳವಣಿಗೆ ಸಾಧ್ಯವಾಗುತ್ತದೆ. ಅದೇ ಶಕ್ತಿಯನ್ನು ಭಾವನೆ ಯೋಚನೆಗಳೆಡೆ ಹರಿಯಿಸಿದರೆ ಸಾಹಿತ್ಯ ಕಲೆ, ಮತ್ತು ತತ್ತ್ವಶಾಸ್ತ್ರ ಮೊದಲಾದ ಬೌದ್ಧಿಕ ಬೆಳವಣಿಗೆಗೆ ಕಾರಣವಾಗುತ್ತದೆ. ಅದೇ ಶಕ್ತಿಯನ್ನು ಪೂರ್ಣರೀತಿಯಿಂದ ಮೂಲಸ್ರೋತದೆಡೆಗೇ ಹಿಂದಿರುಗಿಸಿದರೆ ಮಾನವನು ದೇವಾತ್ಮನಾಗುತ್ತಾನೆ.’\footnote{\engfoot{Infinite power of the Spirit brought to bear upon matter evolves material development, made to act upon thought evolves intellectuality, made to act upon Itself makes of man a God.}\hfill\engfoot{ –Swami Vivekananda.}} ಸರ್ವಬಂಧನಗಳಿಂದ ಬಿಡುಗಡೆ ಹೊಂದಿ ಮುಕ್ತನಾಗುತ್ತಾನೆ.

ದಿವ್ಯತ್ವದಲ್ಲಿ ವಿಶ್ವಾಸವಿಟ್ಟು ಮುನ್ನಡೆಯಲು ಯತ್ನಿಸುವವನ ದೃಷ್ಟಿಕೋನ ಹೇಗಿರುತ್ತದೆ? ಎಲ್ಲರ ಶುಭವನ್ನೂ ಅವನು ಹಾರೈಸುತ್ತಾನೆ, ಕಾರಣ ಪ್ರತಿಯೊಬ್ಬ ವ್ಯಕ್ತಿಯೂ ತತ್ತ್ವಶಃ ಸಚ್ಚಿದಾನಂದ\-ದಿಂದಲೇ ಸಿಡಿದು ಬಂದವನು. ಅದು ಆತನಲ್ಲಿ ಇನ್ನೂ ಪ್ರಕಟವಾಗದಿದ್ದಿರಬಹುದು. ಆದರೆ ಎಲ್ಲರೂ ದಿವ್ಯ ಜ್ಯೋತಿ ಸಾಗರದಿಂದಲೇ ಬಂದವರೆಂಬುದು ಸತ್ಯ. ಮನುಷ್ಯನಾಗಲೀ ಪ್ರಾಣಿ\-ಯಾಗಲೀ ಯಾವುದೇ ವಸ್ತುವಾಗಲೀ ಅದು ನಮ್ಮಷ್ಟೇ ಮೂಲತಃ ದೈವೀಚೇತನ. ಪ್ರತಿಯೊಬ್ಬ ವ್ಯಕ್ತಿಯೂ ಆತನ ಸದ್ಯದ ಪರಿಸ್ಥಿತಿ ಹೇಗೇ ಇರಲಿ, ಗೌರವಕ್ಕೆ ಪಾತ್ರ, ವಿಶ್ವಾಸಕ್ಕೆ ಅರ್ಹ, ಪ್ರೀತಿಗೆ ಯೋಗ್ಯ. ಜನರು ತಮ್ಮ ದೈವೀ ಮೂಲ ಹಾಗೂ ಸ್ವಭಾವವನ್ನು ಮರೆತಿರಬಹುದು. ಅವರ ವರ್ತನೆ ಹೇಯವೂ, ಪಾಶವಿಕವೂ ಆಗಿರಬಹುದು. ಆದರೆ ಅವರು ಅಜ್ಞಾನದಿಂದ ಹಾಗೆ ಮಾಡುತ್ತಾರೆಂಬುದು ಮರೆಯತಕ್ಕ ಸಂಗತಿಯಲ್ಲ.

ಇತರರಿಗೆ ನಮ್ಮಿಂದಾದ ಸಹಾಯವನ್ನು ಮಾಡುವ ಪ್ರವೃತ್ತಿಯನ್ನು ಬೆಳೆಸಿಕೊಳ್ಳುವ ಉತ್ತಮ ಉಪಾಯ ಯಾವುದು? ಅವರ ಹೊರಗಿನ ರೂಪ, ಬಣ್ಣ, ಜಾತಿ, ಮತ–ಇವುಗಳು ಏನೇ ಇರಲಿ, ಅವರಲ್ಲಿ ದೈವತ್ವವಿದೆ ಎಂಬ ದೃಢವಿಶ್ವಾಸ ನಮ್ಮಲ್ಲಿ ಉದಿಸಿರಬೇಕು. ಪ್ರತಿಯೊಬ್ಬನಲ್ಲೂ ಅಡಗಿರುವ ಈ ದೈವತ್ವದ ವಿಚಾರವನ್ನು ಸ್ವಲ್ಪವೂ ಅರಿತಿರದ ಜನರನ್ನೂ ಕೂಡ ಶುಭ ಅಥವಾ ಲೇಸನ್ನು ಬಯಸುವ ದೃಷ್ಟಿಯಿಂದ ನೋಡಬೇಕು. ಅವನು ಕೊಲೆಗಡುಕನೂ, ಕಳ್ಳನೂ ಆಗಿರಬಹುದು. ಸಂತನೋ, ಮಹಾತ್ಮನೋ ಇರಬಹುದು. ನಿಜವಾದ ಧಾರ್ಮಿಕ ಭಾವನೆಯ ವ್ಯಕ್ತ ರೂಪವೆ ಶುಭಾಕಾಂಕ್ಷೆ ಅಥವಾ ಇತರರಿಗೆ ಲೇಸಬಯಸುವುದು. ಇದನ್ನು ಒಳ್ಳೆಯವರಿಗೆ ಮಾತ್ರ ತೋರಿಸಿ ದುಷ್ಟರಿಂದ ಬಚ್ಚಿಡಬಾರದು. ಹಾಗೆಂದು ದುಷ್ಟರ ಸಂಚಿಗೆ ಬಲಿಬಿದ್ದು ನರಳಬೇಕೆಂದಲ್ಲ. ದುಷ್ಕೃತ್ಯ ಎಸಗಿದವನನ್ನು ಶಿಕ್ಷಿಸಬಾರದೆಂದೂ ಅಲ್ಲ. ಅವರ ಮೇಲೆ ದ್ವೇಷ ಸಾಧಿಸುವುದಕ್ಕಿಂತ ಅವನ ದುಷ್ಟತನದ ಹಿನ್ನೆಲೆಯಲ್ಲಿ ಅಜ್ಞಾನ ಮತ್ತು ಅವನಿಗೆ ಸಿಕ್ಕಿದ ತರಬೇತಿ ಹೇಗೆ ಕೆಲಸ ಮಾಡುತ್ತಿದೆ, ಅದನ್ನು ಹೇಗೆ ಬದಲಿಸಬಹುದು ಎಂಬುದನ್ನು ಯೋಚಿಸಲು ಈ ದೃಷ್ಟಿಕೋನ ಪ್ರೇರಣೆ ನೀಡುವುದು.

ದೇವರನ್ನು ಅಥವಾ ದಿವ್ಯತೆಯನ್ನು ಪ್ರೀತಿಸುತ್ತೇನೆ ಎಂದುಕೊಳ್ಳುವ ವ್ಯಕ್ತಿ ತನ್ನ ಸಹೋದರ\-ರನ್ನೂ, ನೆರೆಕೆರೆಯವರನ್ನೂ ದ್ವೇಷಿಸಲಾರ. ಏಕೆಂದರೆ ಯಾವ ದಿವ್ಯತೆ ತನ್ನಲ್ಲಡಗಿದೆಯೋ ಅದೇ ತತ್ತ್ವ ಅಥವಾ ಭಗವಂತನ ಅಂಶ ನೆರೆಕೆರೆಯವರಲ್ಲೂ ಇದೆ. ಸಮಸ್ತ ಜೀವಿಗಳೆಡೆಗೆ ಪ್ರೀತಿ ಮತ್ತು ಸಹೋದರಭಾವ ನಮ್ಮಲ್ಲಿ ವ್ಯಕ್ತವಾಗುತ್ತಿದ್ದರೆ ದೇವರಲ್ಲಿ ನಾವಿಟ್ಟ ಶ್ರದ್ಧಾಭಕ್ತಿಗಳು ಸತ್ಯ ಎಂದಾಗುತ್ತದೆ.


\section*{ಬಹುಜನ ಹಿತಾಯ}

\addsectiontoTOC{ಬಹುಜನ ಹಿತಾಯ}

ಇತ್ತೀಚೆಗೆ ಮಿತ್ರರೊಬ್ಬರು ನಡೆದ ಘಟನೆಯೊಂದನ್ನು ಹೇಳಿದರು. ಸುಮಾರು ಐದು ಸಾವಿರ ಜನಸಂಖ್ಯೆ ಸಣ್ಣ ಪೇಟೆಯಂತಿರುವ ಬಡಾವಣೆಯಲ್ಲಿ ಒಂದು ಎಮ್ಮೆಯ ಕರು ಬಾವಿಗೆ ಬಿತ್ತು. ಬಾವಿಗೆ ಕಲ್ಲು ಮಣ್ಣು ಅಥವಾ ಸಿಮೆಂಟಿನಿಂದ ಮಾಡಿದ ಆವರಣವಿದ್ದಿರಲಿಲ್ಲ. ಮರದ ಗೂಟ ಹಾಕಿದ್ದರು. ಮನೆಯಲ್ಲಿ ಹೆಂಗಸರು ಮಕ್ಕಳು ಮಾತ್ರ ಇದ್ದರು. ಕರು ಬಾವಿಗೆ ಬಿದ್ದುದನ್ನು ಕಂಡು ಅವರು ಗಟ್ಟಿಯಾಗಿ ಕೂಗಿಕೊಂಡರು. ಆಚೀಚಿನ ವಿದ್ಯಾವಂತರೆನಿಸಿಕೊಂಡ ಉದ್ಯೋಗಸ್ಥ ಜನ ಅಲ್ಲಿ ಸೇರಿದರು. ‘ನೀವು ಎಮ್ಮೆ ತಂದದ್ದು ಯಾವಾಗ? ಎಲ್ಲಿಂದ ತಂದಿರಿ? ಬೆಲೆ ಎಷ್ಟು? ಬಾವಿಯ ಸುತ್ತಲೂ ಗೂಟಗಳನ್ನು ಹಾಕಿರುವಾಗ ಕರು ಬಾವಿಗೆ ಬಿದ್ದುದಾದರೂ ಹೇಗೆ? ಯಾರಾದರೂ ಕರುವನ್ನು ಬಾವಿಗೆ ದೂಡಿದರೇ? ಹೇಗೆ?’ ಎಂದೆಲ್ಲ ಪ್ರಶ್ನಿಸತೊಡಗಿದರು. ಕೆಲವರು ಆ ಮನೆಯವರು ಯಾವ ರಾಜಕೀಯ ಪಕ್ಷಕ್ಕೆ ಸೇರಿದವರು, ಅವರ ಜಾತಿಮತಕುಲಗೋತ್ರ ಯಾವುದು ಎನ್ನುವ ಬಗೆಗೆ ಯೋಚಿಸತೊಡಗಿದರು. ಸಣ್ಣ ಪುಟ್ಟ ಚಿಲ್ಲರೆ ಉಪಕಾರಗಳಿಂದೇನು? ಒಟ್ಟು ಸಾಮಾಜಿಕ ಕ್ರಾಂತಿಯಾಗದೆ ನಮ್ಮ ಜನಾಂಗದ ಅಭಿವೃದ್ಧಿಯಾಗದು ಎನ್ನುವ ತೀವ್ರವಾದಿಗಳಾದ ಯುವಕರೂ ಅಲ್ಲಿದ್ದರು. ಆದರೆ ಯಾರೂ ಕರುವನ್ನು ಬಾವಿಯಿಂದ ಮೇಲೆತ್ತುವ ಗೋಜಿಗೇ ಹೋಗಲಿಲ್ಲ. ಜನಸ್ತೋಮವನ್ನು ನೋಡಿ ಹಳ್ಳಿಯ ರೈತನೊಬ್ಬ ಅತ್ತ ಬಂದವನು ನೇರವಾಗಿ ಬಾವಿಯಲ್ಲಿ ಇಳಿದು ಕರುವನ್ನು ಮೇಲಕ್ಕೆತ್ತಿ ಜೀವದಾನ ಮಾಡಿ ಅಲ್ಲಿಂದ ಕಣ್ಮರೆ ಯಾದ. ಚರ್ಚೆಯಲ್ಲಿ ನಿರತರಾದವರು ಸ್ವರ ಏರಿಸಿ ಚರ್ಚೆಯನ್ನು ಮುಂದುವರಿಸುತ್ತಲೇ ಇದ್ದರು.

ಈ ಘಟನೆ ಏನು ಹೇಳುತ್ತದೆಂಬುದನ್ನು ವಿವರಿಸಬೇಕಿಲ್ಲವಷ್ಟೆ?

ಕಷ್ಟಕಾಲದಲ್ಲಿ ನೆರವಾಗುವ ಗುಣವನ್ನು ನಾವು ಬೆಳೆಸಿಕೊಳ್ಳಬೇಕು. ಒಂದೆಡೆ ಸ್ವಾಮಿ ವಿವೇಕಾನಂದರು ‘ವ್ಯಕ್ತಿಯು ಯಾವುದೇ ಸಾಮಾಜಿಕ ವ್ಯವಸ್ಥೆಯಲ್ಲಿರಲಿ ಅವನ ಅಪೂರ್ಣತೆಯನ್ನು ದೂರಗೊಳಿಸಲು ನೆರವಾಗುವವನೇ ಮಾನವ ಜನಾಂಗದ ನಿಜವಾದ ಉಪಕಾರಿ’\break ಎಂದಿದ್ದರು. ‘ಕೆಟ್ಟ ರೂಢಿಗಳನ್ನು ಕಾನೂನುಗಳನ್ನು ಸತ್ಪುರುಷರು ಲೆಕ್ಕಿಸುವುದಿಲ್ಲ. ಅವುಗಳ ಜಾಗದಲ್ಲಿ ಆತ್ಮೀಯತೆ, ಸಹಿಷ್ಣುತೆ ಮತ್ತು ಒಗ್ಗಟ್ಟಿನ ಲಿಖಿತ ಸಂವಿಧಾನವಿರದಿದ್ದರೂ ಅತ್ಯಂತ ಪ್ರಭಾವಶಾಲಿ ಯಾದ ಪರಂಪರೆಯನ್ನು ಅವರು ಬೆಳೆಸುತ್ತಾರೆ. ಸತ್ಪುರುಷರು ಎಲ್ಲ ಕಾನೂನು ಕಾಯಿದೆಗಳನ್ನೂ ದಾಟಿ ತಮ್ಮ ಬಾಂಧವರು ಯಾವುದೇ ಪರಿಸ್ಥಿತಿಯಲ್ಲಿರಲಿ ಅವರ ಉದ್ಧಾರಕ್ಕಾಗಿ ಶ್ರಮಿಸುತ್ತಾರೆ.’ ಬಹು ಜನರ ಹಿತಕ್ಕಾಗಿ, ಸುಖಕ್ಕಾಗಿ ಶ್ರಮಿಸುವುದೇ ಅವರ ಧ್ಯೇಯ.


\section*{ಜೀವ ಶಿವ ಸೇವೆ}

\addsectiontoTOC{ಜೀವ ಶಿವ ಸೇವೆ}

ದಿವ್ಯತೆಯಲ್ಲಿನ ದೃಢವಿಶ್ವಾಸದಿಂದ ಉದ್ಭವಿಸುವ ಇನ್ನೊಂದು ಮನೋವೃತ್ತಿಯೇ ಸೇವಾ\break ಪರಾಯಣತೆ. ಸೇವೆ ಎನ್ನುವ ಶಬ್ದ ಇಂದು ಸವಕಲು ನಾಣ್ಯವಾಗಿರಬಹುದು. ಸ್ವಾರ್ಥಿ ಮತ್ತು ದುರ್ಬಲ ವ್ಯಕ್ತಿಗಳು ಯಾವ ಕ್ಷೇತ್ರವನ್ನು ಪ್ರವೇಶಿಸಿದರೂ ಅಲ್ಲಿ ಮಸಿ ಬಳಿಯುತ್ತಾರೆ. ಕೊಳೆತ ಹಣ್ಣನ್ನು ನೋಡಿ ಮರದ ನಿಜವಾದ ಸ್ಥಿತಿಗತಿಯ ಬಗೆಗೆ ತೀರ್ಮಾನ ನೀಡಬಾರದೆಂದು ಕ್ರಿಸ್ತ ವಾಣಿ. ಅಂತೆಯೇ ಸೇವಾಧರ್ಮದ ಅರ್ಥ ಮಹತ್ವಗಳನ್ನು ಅದರ ದುರ್ಬಲ ಅನುಯಾಯಿಗಳ ಮೂಲಕ ನಿರ್ಣಯಿಸಬಾರದು. ಎಲ್ಲರೆಡೆಗಿನ ಶುಭಾಶಯ ಮತ್ತು ಪರಿಶುದ್ಧ ಪ್ರೀತಿ\break ಇವುಗಳಿಂದ ಪ್ರೇರಿತವಾಗಿರಬೇಕು ಸೇವೆ. ವಿಕಾಸದ ಮೇಲಿನ ಹಂತದಲ್ಲಿರುವವರು ತಮಗಿಂತ ಕೆಳಗಿನ ಮೆಟ್ಟಲುಗಳಲ್ಲಿರುವವರಿಗೆ ಅಯಾಚಿತವಾಗಿ ಸಹಾಯ ಹಸ್ತ ನೀಡುವಲ್ಲಿ ಈ ಸೇವೆಯ ಭಾವನೆ ವ್ಯಕ್ತವಾಗುವುದು. ಒಳಿತನ್ನು ಇತರರಿಗೆ ಹಂಚಲೂ, ಹತ್ತು ಮಂದಿ ಸೇರಿ ಮಾಡುವ ಕೆಲಸದಲ್ಲಿ ತಾನೊಂದು ನಿಮಿತ್ತವಾಗಲು ನೀಡುವ ಸಹಕಾರದಲ್ಲೂ ಇದು ವ್ಯಕ್ತವಾಗುವುದು. ಪರಿಶುದ್ಧ ಪ್ರೀತಿ ಪ್ರತಿಯೊಬ್ಬರಲ್ಲೂ ಹುದುಗಿರುವ ದಿವ್ಯತೆಯ ಒಂದು ಪ್ರಕಾಶ. ಇತರರಿಂದ ಏನನ್ನೂ ಪ್ರತಿಫಲವಾಗಿ ಬಯಸದೆ ಅವರಿಗೆ ಒಳಿತನ್ನು ಮಾಡುವಲ್ಲಿ ಅದು ವ್ಯಕ್ತವಾಗುವುದು. ಲೋಕ ಲೋಚನಕ್ಕೆ ಗೋಚರವಾಗುವುದಕ್ಕಾಗಿ ಜಾಹೀರಾತಿನ ಪ್ರಸಿದ್ಧಿ ತರುವಂಥ ‘ಜಗದ ಪೊಗಳಿಕೆಗೆ ಬಾಯಬಿಡುವಂಥ’ ಕೆಲಸ ಸೇವೆಯಲ್ಲ. ತಾನು ಹುಟ್ಟಿ ಬೆಳೆದು ಬಂದ ತಾಣದಲ್ಲಿ ತನ್ನ ಪಾಲಿಗೆ ಸಾಧ್ಯವಾಗುವ ಬಹುಜನ ಹಿತದಲ್ಲಿ ನೆರವಾಗುವ ಪುಟ್ಟ ಕೆಲಸವೂ ಸೇವೆಯಾಗುವುದು. ದೃಷ್ಟಿ ಕೋನ ಮನೋಭಾವನೆಗಳು ಮುಖ್ಯ, ಮಾಡುವ ಕೆಲಸವಲ್ಲ. ಕೆಲಸವನ್ನು ಮಾಡುವ ವಿಧಾನ ಮುಖ್ಯ. ಕೆಲಸದ ಹಿನ್ನೆಲೆಯಲ್ಲಿ ಅಡಗಿರುವ ಉದ್ದೇಶ ಮುಖ್ಯ.

‘ಅಷ್ಟೊಂದು ತಪಸ್ಸಿನ ನಂತರ ನಾನು ನಿಜವಾದ ಈ ಪರಮ ಸತ್ಯವನ್ನು ಕಂಡುಕೊಂಡಿದ್ದೇನೆ. ಪ್ರತಿಯೊಂದು ಜೀವಿಯಲ್ಲೂ ಭಗವಂತನಿದ್ದಾನೆ. ನರರ ಸೇವೆಯನ್ನು ಮಾಡುವವನು ನಾರಾಯಣನ ಸೇವೆಯನ್ನು ಮಾಡುತ್ತಾನೆ. ನಿಮ್ಮ ಸಹೋದರರ ಸೇವೆ ಮಾಡದ ನೀವು, ಕಣ್ಣಿಗೆ ಕಾಣಿಸುವ ವ್ಯಕ್ತದೇವತೆಯ ಸೇವೆ ಮಾಡದ ನೀವು, ಇಂದ್ರಿಯಗಳಿಗೆ ಅಗೋಚರನಾದ, ವ್ಯಕ್ತನಲ್ಲದ ಭಗವಂತನ ಸೇವೆಯನ್ನು ಹೇಗೆ ಮಾಡಬಲ್ಲಿರಿ? ನೀವು ಪೂಜೆ ಸಲ್ಲಿಸಲು ಒಂದು ದೇವಾಲಯವನ್ನು ನಿರ್ಮಿಸಬಹುದು. ಅದು ಚೆನ್ನಾಗಿಯೂ ಇರಬಹುದು. ಆದರೆ ಇನ್ನೂ ಶ್ರೇಷ್ಠವಾದುದು ಉತ್ಕೃಷ್ಟವಾದುದು ಆಗಲೇ ಇದೆ. ಅದೇ ಮಾನವ ದೇಹ’ ಎಂದರು ಸ್ವಾಮಿ ವಿವೇಕಾನಂದರು.

ದೇವಸ್ಥಾನದಲ್ಲಿ ಮಾತ್ರ ದೇವರನ್ನು ಪೂಜಿಸಿ, ತನ್ನಂತೆಯೇ ಇತರರೂ ಕೂಡ ದೇವರ ಮಕ್ಕಳು ಎಂಬುದನ್ನು ಅರಿಯದೆ, ಅವರ ಹಿತಾಕಾಂಕ್ಷೆಯಿಂದ ಸಹಾಯ ಸಹಕಾರ ನೀಡದವನು ನಿಜವಾದ ಧಾರ್ಮಿಕನೆನಿಸಲಾರ. ದೇವರ ಸೇವೆಯನ್ನು ಮಾಡಲಿಚ್ಛಿಸುವವನು ದೇವರ ಮಕ್ಕಳ ಸೇವೆಯನ್ನು ಮೊದಲು ಮಾಡಬೇಕು ಎನ್ನುತ್ತಾರೆ. ‘ಯಾರು ದೀನರಲ್ಲಿ, ದುರ್ಬಲರಲ್ಲಿ, ರೋಗಿಗಳಲ್ಲಿ ಶಿವನನ್ನು ನೋಡುವರೋ ಅವರು ನಿಜವಾಗಿ ಶಿವನನ್ನು ಪೂಜಿಸುವರು. ಜಾತಿ ಮತ ಕುಲ ಗೋತ್ರಗಳನ್ನು ಲೆಕ್ಕಿಸದೆ ಯಾರು ಒಬ್ಬ ಬಡವನಲ್ಲಿ ಶಿವನನ್ನು ನೋಡಿ ಅವನ ಸೇವೆ ಮಾಡುವರೋ ಅವನಿಗೆ ಸಹಾಯ ನೀಡಿರುವರೋ ಅಂಥವರ ಮೇಲೆ ಶಿವನಿಗೆ ತನ್ನನ್ನು ಕೇವಲ ವಿಗ್ರಹದಲ್ಲೇ ನೋಡುವವರಿಗಿಂತ ಹೆಚ್ಚು ಪ್ರೀತಿ’ ಎನ್ನುವ ವಿವೇಕಾನಂದರ ವಾಣಿಯಲ್ಲಿ ಸೇವಾ ಧರ್ಮದ ಸಾರ, ಮಹತ್ವ ಅಡಗಿದೆ.

ದಿವ್ಯತೆಯಲ್ಲಿಡುವ ದೃಢ ವಿಶ್ವಾಸದಿಂದ ಮತ್ತು ಒಳಗಿರುವ ಅಪಾರ ಶಕ್ತಿಯ ಬಲದಿಂದ ಪ್ರತಿಯೊಬ್ಬ ವ್ಯಕ್ತಿಯಲ್ಲೂ ಉದಿಸುವ ಸದ್ಗುಣಗಳಾದ ಸರ್ವರೆಡೆಗೆ ಶುಭಾಕಾಂಕ್ಷೆ, ತಾಳ್ಮೆ, ಆಶಾವಾದ, ಪ್ರಯತ್ನಶೀಲತೆ ಮತ್ತು ಸೇವಾಪರಾಯಣತೆ ಇವು ವ್ಯಕ್ತಿಯ ಸರ್ವತೋಮುಖ\break ಅಭ್ಯುದಯ ಮತ್ತು ಸಮಾಜದ ಸಮಗ್ರ ಪ್ರಗತಿಗೆ ಕಾರಣವಾಗಬಲ್ಲಂಥವುಗಳು ಎಂಬುದು ಸಂಶಯ ರಹಿತ ಸತ್ಯ.

ನಮ್ಮ ಓದು, ಪರಿಸರ, ಶಿಕ್ಷಣ–ಇವು ವ್ಯಕ್ತಿಯಲ್ಲಿ ಈ ಗುಣಗಳನ್ನೇ ಬೆಳೆಸುವಂತಾದರೆ ನಮ್ಮಸಮಾಜ ಎದ್ದು ನಿಲ್ಲುವುದು. ನಮ್ಮಲ್ಲಿರುವ ಅಪಾರ ಶಕ್ತಿಯನ್ನು ನಾವು ಮನಗಂಡಾಗಲೇ ಭಾಗ್ಯೋದಯದ ಹೊಸಬೆಳಕು ಹರಿಯುವುದು.

\chapterend

\addtocontents{toc}{\protect\par\egroup}

