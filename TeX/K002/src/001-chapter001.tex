
\chapter{ಪ್ರಯತ್ನದಿಂದ ಪರಮಾರ್ಥ}

\indentsecionsintoc

\begin{itemize}
\itemsep=3pt
\item ಸ್ವಪ್ರಯತ್ನದಿಂದ ಸರ್ವತೋಮುಖ ಏಳ್ಗೆ ಸಾಧಿಸಬಲ್ಲ ಒಬ್ಬನೇ ಒಬ್ಬ ವ್ಯಕ್ತಿ ಈ ವಿಶ್ವದಲ್ಲಿ ಖಂಡಿತವಾಗಿಯೂ ಇದ್ದಾನೆ–ಆತ ನೀನೇ.\hfill \enginline{—}ಆಲ್ಡಸ್ ಹಕ್ಸ್​ಲೀ

 \item ಏಳು! ಎದ್ದೇಳು! ನಿನ್ನ ಭವಿಷ್ಯವನ್ನು ರೂಪಿಸುವ ಸಂಪೂರ್ಣ ಜವಾಬ್ದಾರಿಯನ್ನು ಹೊರು. ನಿನ್ನ ಉತ್ತರೋತ್ತರ ಶ್ರೇಯಸ್ಸಿಗೆ ಬೇಕಾದ ಎಲ್ಲ ಶಕ್ತಿ ಸಾಮರ್ಥ್ಯಗಳೂ ನಿನ್ನೊಳಗೇ ಇವೆ. ನೋಡು, ಮನಮಾಡು. ಏಳು ಎದ್ದೇಳು, ಭವ್ಯ ಭವಿಷ್ಯತ್ತಿನ ನಿರ್ಮಾಣಕ್ಕೆ ಇಂದೇ ಅಡಿ ಇಡು.\hfill \enginline{—}ಸ್ವಾಮಿ ವಿವೇಕಾನಂದ

 \item ತನ್ನ ಆಸುಪಾಸಿನ ಜನಸಾಮಾನ್ಯರು ಬಿರುಗಾಳಿಗೊಡ್ಡಿದ ತರಗೆಲೆಗಳಂತೆ ಚೆದುರಿ\break ನಿರ್ನಾಮವಾಗುವಾಗ, ಏಕನಿಷ್ಠೆ ಹಾಗೂ ಏಕಾಗ್ರತೆಯಿಂದ ಬಿಡದ ಛಲದೊಂದಿಗೆ ಮುನ್ನುಗ್ಗಿ, ಮನೋಚಾಂಚಲ್ಯವನ್ನು ಬಡಿದಟ್ಟುವ ಪ್ರಯತ್ನಶೀಲ ವ್ಯಕ್ತಿ ಮಾತ್ರನೇ, ಮುಗಿಲೆತ್ತರಕ್ಕೇರಿ ಗೋಪುರದಂತೆ ದೃಢವಾಗಿ ಕಂಗೊಳಿಸುತ್ತಾನೆ.\hfill \enginline{—}ವಿಲಿಯಂ ಜೇಮ್ಸ್.

 \item ಬಹಳಷ್ಟು ವಿಶ್ಲೇಷಣೆಯ ಕೊನೆಯಲ್ಲಿ ಹೀಗೆನ್ನಬಹುದು: ಆಯಾ ದಿನದ ಪ್ರಾಮಾಣಿಕ ಕೆಲಸ, ಆಯಾ ದಿನದ ಪ್ರಾಮಾಣಿಕ ಚಟುವಟಿಕೆ, ಆಯಾ ದಿನದ ಔದಾರ್ಯದ ಮಾತುಗಳು, ಆಯಾ ದಿನದ ಸತ್ಕರ್ಮಗಳು–ಇವುಗಳನ್ನು ಬಿಟ್ಟು ಮನುಷ್ಯನ ಪ್ರಗತಿಗೆ ಬೇರಾವ ಉಪಾಯವೂ ಇಲ್ಲ.\hfill \enginline{—}ಕ್ಲೇರ್ ಬೂತ್ ಲೂಸ್​

 \item \enginline{There is one corner of the Universe you can be certain of improving and that is your own self.\general{\hfill} –\textit{Aldous Huxley}}

 \item \enginline{Stand up, be bold, be strong. Take the whole responsiblity on your shoulders and know that you are the creator of your own destiny. All the strength and succour you want is within yourself. Therefore make your own future.\general{\hfill} –\textit{Swami Vivekananda}}

 \item \enginline{The man who has daily inured himself to habits of concentrated attention, energetic volition and self denial will stand like a tower when everything rocks around him and when his softer fellow mortals are winnowed like chaff in the blast.\general{\hfill} —\textit{William James}}

 \item \enginline{In the final analysis there is no other solution to man’s progress but the day’s honest work, the day’s honest activities, the day’s generous utterances and the day’s good deed.\general{\hfill\hbox\bgroup} —\textit{Clare Booth Luce}\general{\egroup}}

\end{itemize}


\section*{ಏನಿದು ವಿಚಿತ್ರ?}

\addsectiontoTOC{ಏನಿದು ವಿಚಿತ್ರ ?}

ಮೆಡಿಟರೇನಿಯನ್ ಪ್ರದೇಶದ ಸಿಸಿಲಿ ದ್ವೀಪ. ಮಧ್ಯಾಹ್ನದ ಸುಡು ಬಿಸಿಲು ಇಳಿಮುಖವಾಗುತ್ತ\-ಲಿತ್ತು. ತಮ್ಮ ತಮ್ಮ ವ್ಯಾವಹಾರಿಕ ಚಿಂತೆಗಳ ಸಾಮ್ರಾಜ್ಯದಲ್ಲಿ ಮಗ್ನರಾದ ಜನರೇ ಬೀದಿಯಲ್ಲಿ ತುಂಬಿ\-ಕೊಂಡಿದ್ದರು. ಥಟ್ಟನೇ ಕರ್ಕಶ ಕೂಗೊಂದು—‘ಯುರೇಕಾ, ಯುರೇಕಾ’– ಕೇಳಿಬಂತು. ನಡುಬೀದಿಯಲ್ಲಿ ಒಬ್ಬಾತ ಪೂರ್ಣ ನಗ್ನನಾಗಿ ಓಡುತ್ತಿದ್ದಾನೆ. ಅವನನ್ನು ಕಂಡು ಈ ಜನರೆಲ್ಲ ದಂಗಾಗಿದ್ದಾರೆ. ಆತ ‘ಯುರೇಕಾ, ಯುರೇಕಾ’ ಎಂದು ಅರಚುತ್ತ ಅಪಾರ ಆನಂದ ಸಂಭ್ರಮದಿಂದ, ದೇಹದ ಮೇಲಿನ ಪರಿವೆಯೇ ಇಲ್ಲದಂತೆ ಓಡುತ್ತಿದ್ದಾನೆ. ಜನಸಮೂಹ ತದೇಕ ದೃಷ್ಟಿಯಿಂದ ಆತನನ್ನೇ ಹಿಂಬಾಲಿಸುತ್ತಲಿದೆ, ಏನಿದು? ಏನಾಶ್ಚರ್ಯ!

\vskip 2pt

ಯಾರು ಆ ಮನುಷ್ಯ? ಆತ ಸಾಮಾನ್ಯ ವ್ಯಕ್ತಿ ಅಲ್ಲ, ಮಹಾ ವಿಜ್ಞಾನಿ, ‘ನನಗೆ ನಿಲ್ಲಲು ಭದ್ರ ನೆಲೆ, ಹಿಡಿಯಲು ದೊಡ್ಡ ಸನ್ನೆಕೋಲು, ಜೊತೆಗೆ ಸೂಕ್ತ ಆನಿಕೆಗಳನ್ನೊದಗಿಸಿದರೆ ಈ ಭೂಲೋಕವನ್ನೇ ಎತ್ತಿಬಿಡಬಲ್ಲೆ’ ಎಂದ ಗಣಿತಜ್ಞ ಆತ. ಆತನ ಹೆಸರು ಆರ್ಕಿಮಿಡಿಸ್. ಆತ ಬತ್ತಲೆಯಾಗಿ ಆ ದಿನ ಅಪರಾಹ್ನ ಜನನಿಬಿಡ ಬೀದಿಯಲ್ಲಿ ‘ಯುರೇಕಾ’ (ಕಂಡೆ ನಾ) ಎನ್ನುತ್ತ ಓಡಿದ್ದಾದರೂ ಯಾಕೆ? ಆತ ಕಂಡುದಾದರೂ ಏನು?

\vskip 2pt

ಆ ಘಟನೆಯ ಹಿನ್ನೆಲೆ ಹೀಗಿದೆ–ಸಿರಾಕ್ಯೂಸ್​ನ ದೊರೆ ಹೀರೋ ಚಿನ್ನದ ಕಿರೀಟವೊಂದನ್ನು ತಯಾರಿಸಲು ಅಕ್ಕಸಾಲಿಗನಿಗೆ ಆದೇಶ ನೀಡಿದ್ದ. ಕಿರೀಟದ ಕೆಲಸವನ್ನು ಪೂರೈಸಿ ಆತ ಅದನ್ನು ಅರಸನಿಗೆ ಅರ್ಪಿಸಿದ. ಕಿರೀಟದಲ್ಲಿ ಚಿನ್ನವಲ್ಲದೇ ಇತರ ಹೀನ ಲೋಹಗಳ ಮಿಶ್ರಣ\-ವಿರ\-ಬಹುದೆಂಬ ಸಂದೇಹ ಬಂತು ದೊರೆಗೆ. ಪ್ರತ್ಯಕ್ಷಸಾಕ್ಷಿಗೆ ಅವಕಾಶವಿರಲಿಲ್ಲ. ಆಸ್ಥಾನದ ವಿಖ್ಯಾತ ಪಂಡಿತರಿಗಂದೇ ಪರೀಕ್ಷಿಸಿ ತಿಳಿಸುವ ಪಂಥಾಹ್ವಾನವಾಯಿತು. ಅದನ್ನು ಸ್ವೀಕರಿಸಿದವನೇ\break ಆರ್ಕಿಮಿಡಿಸ್.

\vskip 2pt

ಅರಸನಿಂದ ಅಪ್ಪಣೆ ಪಡೆದ ಆರ್ಕಿಮಿಡಿಸ್ ಕಿರೀಟದಲ್ಲಿ ಇತರ ಲೋಹಗಳ ಮಿಶ್ರಣವಿದೆಯೇ ಎಂಬುದನ್ನು ತಿಳಿಯಲು ಯೋಚಿಸತೊಡಗಿದ. ಲೆಕ್ಕ ಗುಣಿಸಿದ. ಸಮಸ್ಯೆಯನ್ನು ಹಲವು ದೃಷ್ಟಿಕೋನಗಳಿಂದ ವಿಶ್ಲೇಷಿಸಿದ. ಏಕಾಗ್ರಚಿತ್ತನಾಗಿ ಚಿಂತನೆ ನಡೆಯಿಸಿದ. ಅನವರತ ತದೇಕ ಧ್ಯಾನ ಪರನಾದ. ಸ್ನಾನದ ಹೊತ್ತಿನಲ್ಲೂ ಅದನ್ನೇ ಯೋಚಿಸುತ್ತಿದ್ದ. ನೀರು ತುಂಬಿದ ತೊಟ್ಟಿಯೊಳಕ್ಕೆ ಇಳಿಯುತ್ತಿದ್ದಾಗ ಅದರೊಳಗೆ ಜಾರಿಬಿದ್ದ. ಒಂದಿಷ್ಟು ನೀರು ತುಳುಕಿ ಹೊರಚೆಲ್ಲಿತು. ಅದೋ! ತೇಲುವಿಕೆಯ ನಿಯಮ–ಸಾಪೇಕ್ಷ ಸಾಂದ್ರತೆಯ ಸಿದ್ಧಾಂತದ ಸಾಕ್ಷಾತ್ಕಾರ ಅವನಿಗಾಯಿತು! ರಾಜನ ಕಿರೀಟದ ಖೋಟಾತನ ಬಯಲಿಗೆಳೆಯಲು ಬೇಕಾದುದು ಅದೇ. ತಡೆಯಲಾರದ ಆನಂದದಿಂದ ಮೈಮರೆತು ಸ್ನಾನದ ತೊಟ್ಟಿಯಿಂದಲೇ ಆತ ಮೇಲೆ ಜಿಗಿದು ‘ಯುರೇಕಾ’ ಎಂದು ಕೂಗುತ್ತ ಓಡಿದ.

ಆರ್ಕಿಮಿಡಿಸ್ ಸಮಸ್ಯೆಯ ಪರಿಹಾರಕ್ಕಾಗಿ ಏಕಾಗ್ರಚಿತ್ತದಿಂದ ಬಿಡದೆ ಚಿಂತಿಸಿದ. ಮನಸ್ಸಿನ ಪೂರ್ಣ ಶಕ್ತಿಯನ್ನು ಉಪಯೋಗಿಸಿ ನಿರಂತರ ಶ್ರಮಿಸಿದ. ಹಗಲಿರುಳೂ ಕಾತರಿಸಿದ. ಆ ಸತ್ಯ ಸಾಕ್ಷಾತ್ಕಾರವಾಗುತ್ತಲೇ ಆನಂದಾತಿರೇಕದಿಂದ ಮೈಮರೆತ. ಸಾರ್ಥಕ ದುಡಿಮೆಯ ಮಹಿಮೆ ಅಂಥದು!


\section*{ವ್ಯಕ್ತಿಯಲ್ಲ, ಶಕ್ತಿ}

\addsectiontoTOC{ವ್ಯಕ್ತಿಯಲ್ಲ, ಶಕ್ತಿ}

ಮಕ್ಕಳ ಆಸಕ್ತಿ ಅರಳಿಸುವ, ವೈಜ್ಞಾನಿಕ ಮತ್ತಿತರ ನೂರಾರು ವಿಚಾರಗಳನ್ನೊಳಗೊಂಡ ಸುಮಾರು ಮೂರು ಸಾವಿರ ಪುಟಗಳ ‘ಬಾಲ ಪ್ರಪಂಚ’ವನ್ನು ಕೇವಲ ಮೂರು ತಿಂಗಳೊಳಗೆ ಬರೆಯ\-ಲಾಯಿತು. ಬರೆದವರು ಯಾರು ಗೊತ್ತೇ?

\vskip 2pt

ವಿಶ್ವವಿದ್ಯಾಲಯಗಳ ಪದವೀಧರ ಪಂಡಿತ ಮಂಡಲಿಯಲ್ಲ. ಬಿಳಿಯ ಪಾಶ್ಚಾತ್ಯ ವಿದ್ವಾಂಸರಲ್ಲ. ಒಬ್ಬ ಭಾರತೀಯರು, ಒಬ್ಬ ಕನ್ನಡಿಗರೇ. ಅಧ್ಯಯನಕ್ಕೆ ಸಂಶೋಧನೆಗೆ ವಿಶೇಷ ಅನುಕೂಲತೆಗಳಿರುವ ವಿಶ್ವವಿದ್ಯಾಲಯಗಳಲ್ಲಿ ವ್ಯಾಸಂಗ ಮಾಡಿದ ಪದವೀಧರರು ಅವರಲ್ಲ. ಮಾಧ್ಯಮಿಕ ವಿದ್ಯಾಭ್ಯಾಸವನ್ನು ಮುಗಿಸಿ ಕಾಲೇಜಿನಲ್ಲಿ ಕೆಲ ದಿನಗಳನ್ನು ಮಾತ್ರ ಕಳೆದವರು ಅವರು. ಆದರೆ ಯಾವ ಪದವೀಧರನೂ ಯೋಚಿಸದ, ಯಾಚಿಸದ ಜ್ಞಾನದಾಹ ಅವರದು.

\vskip 2pt

ಮೇಲೆ ಹೇಳಿದ ಪುಸ್ತಕವೊಂದನ್ನೇ ಅವರು ಬರೆದುದಲ್ಲ. ನೂರಕ್ಕೂ ಹೆಚ್ಚಿನ ಕಾದಂಬರಿಗಳು, ಅನುವಾದಿತ ಪ್ರಬಂಧಗಳು, ಪ್ರವಾಸ ಸಾಹಿತ್ಯ, ವಿಡಂಬನೆಯ ಬರಹಗಳು, ನಾಟಕಗಳು, ಗೀತ ನಾಟಕಗಳು, ಕಥೆಗಳು, ಯಕ್ಷಗಾನ, ಚಿತ್ರಕಲೆ, ವಾಸ್ತುಶಿಲ್ಪಕ್ಕೆ ಸಂಬಂಧಿಸಿದ ಗ್ರಂಥಗಳು, ಪಠ್ಯ ಪುಸ್ತಕಗಳು, ಶಬ್ದಕೋಶ, ಜನಪ್ರಿಯ ವಿಜ್ಞಾನ ಪ್ರಪಂಚ, ಒಂದೇ ಎರಡೇ? ಅವರ ಕ್ರಿಯಾ ಶೀಲತೆ ಕೇವಲ ಗ್ರಂಥ ಸಾಹಿತ್ಯ ಪ್ರಪಂಚಕ್ಕೆ ಮೀಸಲಾಗದೇ ಸಾಮಾಜಿಕ, ರಾಜಕೀಯ, ನೃತ್ಯ ನಾಟಕ ಸಿನೆಮಾಗಳವರೆಗೂ ವ್ಯಾಪಿಸಿದೆ. ಜ್ಞಾನಪೀಠ ಪ್ರಶಸ್ತಿ ವಿಜೇತರಾದ ಅವರ ಹೆಸರು ಕೋಟ ಶಿವರಾಮ ಕಾರಂತ ಎಂದು.

\vskip 2pt

ಬೃಹತ್ ಸಂಸ್ಥೆಯೊಂದು ಮಾಡಬಹುದಾದ ಕೆಲಸವನ್ನು ವ್ಯಕ್ತಿಯೊಬ್ಬನೇ ಮಾಡಿ ತೋರಿಸಿ\-ದುದು ಪರಮಾದ್ಭುತ ಅಲ್ಲವೆ? ಹೌದು, ಅವರ ಅದ್ಭುತ ಕಾರ್ಯಶಕ್ತಿಯ ರಹಸ್ಯ ಏನು? ಹಿನ್ನಲೆ ಏನು? ಇದನ್ನು ತಿಳಿಯಲೋಸುಗವೇ ನಾನು ಒಮ್ಮೆ ಕಾರಂತರನ್ನು ಭೇಟಿಯಾಗಿ ವಿಚಾರಿಸಿದ್ದೆ– ‘ಅದ್ಭುತವೆನಿಸುವ ನಿಮ್ಮ ಈ ಕಾರ್ಯಶಕ್ತಿಯ ರಹಸ್ಯವೇನು?’

‘ರಹಸ್ಯವೆಂಥದು? ಕೆಲಸ ಮಾಡುತ್ತ ಹೋಗುತ್ತೇನೆ. ಕೆಲಸ ಆನಂದ ತರಬಲ್ಲದು, ಉತ್ಸಾಹ ಸ್ಫೂರ್ತಿಯನ್ನು ನೀಡಬಲ್ಲದು.’ ಹೌದು, ಕಾರಂತರು ಕೆಲಸ ಮಾಡುತ್ತ ಹೋಗುತ್ತಾರೆ. ಕೆಲಸದಲ್ಲೇ ಲೀನರಾಗಿ ಮೈಮರೆಯುತ್ತಾರೆ. ಅದು ನೀಡುವ ಆನಂದವೇ ಇಳಿವಯಸ್ಸಿನಲ್ಲೂ ಕಾರಂತರ ಕಾರ್ಯೋತ್ಸಾಹ ಹೆಚ್ಚಲು ಸ್ಫೂರ್ತಿ.


\section*{ಪ್ರತಿಭೆಯ ಹಿಂದಿದೆ ಪರಿಶ್ರಮ}

\addsectiontoTOC{ಪ್ರತಿಭೆಯ ಹಿಂದಿದೆ ಪರಿಶ್ರಮ}

ಜಗತ್ತಿನ ಇತಿಹಾಸವನ್ನು ನೋಡಿ. ಏನಾದರೂ ಮಹತ್ತನ್ನು ಸಾಧಿಸಿದವರೆಲ್ಲರೂ ಶ್ರಮಪಟ್ಟು ದುಡಿಯುತ್ತಿದ್ದರು, ತಮ್ಮ ಕೆಲಸದಲ್ಲೇ ಮಗ್ನರಾಗುತ್ತಿದ್ದರು. ಈ ದುಡಿಮೆ ಅವರಿಗೆ ಅದ್ಭುತ ವೆನಿಸುವ ಆನಂದ, ತೃಪ್ತಿ ತಂದಿತು. ಅಪಾರ ಶಕ್ತಿ, ಆತ್ಮವಿಶ್ವಾಸ ನೀಡಿತು.

ಎರಡು ಸಾವಿರಕ್ಕೂ ಹೆಚ್ಚು ಹೊಸ ಯಂತ್ರಗಳನ್ನು ಸೃಷ್ಟಿಸಿ ಅಸಾಧಾರಣ ಪ್ರತಿಭಾಶಾಲಿ ಎನ್ನಿಸಿಕೊಂಡ ಥಾಮಸ್ ಆಲ್ವ ಎಡಿಸನ್ ಹೇಳಿದಂತೆ ‘ಆವಿಷ್ಕಾರಗಳು ಆಕಸ್ಮಿಕಗಳಲ್ಲ–ಅವು ನಿರಂತರ ಶ್ರಮಕ್ಕೆ ಒಲಿದ ವರಗಳು. ಪ್ರತಿಭೆಯಲ್ಲಿ ನಿಜವಾಗಿಯೂ ಸ್ಫೂರ್ತಿಯ ಪಾಲು ಒಂದಾ ದರೆ, ಉಳಿದ ತೊಂಬತ್ತೊಂಬತ್ತು ಪಾಲು ಸತತ ಪರಿಶ್ರಮದ ಫಲ.’ ನಿಜವಾದ ಸಫಲತೆ ಮತ್ತು ಸತತ ಪರಿಶ್ರಮ–ಇವು ಯಾವಾಗಲೂ ಒಡನಾಡಿಗಳು. ಅವಿರತ ಪರಿಶ್ರಮದಿಂದ ಮಾತ್ರ ಅತ್ಯಂತ ಅಮೂಲ್ಯ ವಸ್ತುವನ್ನು ಪಡೆಯಬಹುದೆಂಬುದನ್ನು ನಾವು ಮತ್ತೆ ಮತ್ತೆ ನೆನಪಿಗೆ ತಂದು ಕೊಳ್ಳಬೇಕು. ಹೌದು, ಶಬ್ದಕೋಶದಲ್ಲಿ ಮಾತ್ರ ‘ಗೆಲವು’ ‘ಶ್ರಮ’ಕ್ಕೆ ಮೊದಲೇ ಬರುತ್ತದೆ.\footnote{\engfoot{Success comes before work only in the dictionary.}}

‘ಒಂದು ಆದರ್ಶವನ್ನೊ, ಗುರಿಯನ್ನೊ ಆರಿಸಿಕೊಳ್ಳಿ. ಅದನ್ನು ನಿಮ್ಮ ಬದುಕಿನ ಉಸಿರಾಗಿ ಮಾಡಿಕೊಳ್ಳಿ. ಬೇರೆ ಯೋಚನೆಗಳನ್ನೆಲ್ಲ ದೂರಕ್ಕೆ ತಳ್ಳಿ. ನಿಮ್ಮ ಪೂರ್ಣ ಮನಸ್ಸು, ದೇಹ, ಮಾಂಸಖಂಡಗಳು, ನರವ್ಯೂಹ–ಅದೊಂದೇ ವಿಷಯವನ್ನು ಚಿಂತಿಸಲಿ, ಕನಸು ಕಾಣಲಿ. ಮಹಾತ್ಮರಾಗಲು ದಾರಿ ಅದೊಂದೇ’–ಎಂಬುದು ಮಹಾತ್ಮರೊಬ್ಬರ ವಚನ.

\begin{verse}
ನಾವು ಮಹಾತ್ಮರೆನ್ನಿಸಿಕೊಳ್ಳಲು ಹೆಣಗಬೇಕಿಲ್ಲ.\\ಸಂಶೋಧನೆಗಳನ್ನು ಮಾಡಿ ವಿಜ್ಞಾನಿಗಳಾಗಬೇಕಿಲ್ಲ.\\ಒಬ್ಬರಂತೆ ಒಬ್ಬರಲ್ಲ. ಒಬ್ಬರಂತೆ ಒಬ್ಬರಿಲ್ಲ.
\end{verse}

ನಡೆ–ನುಡಿ, ಅಭಿರುಚಿ ಆಸಕ್ತಿ ಸಂಸ್ಕಾರ ಸಾಮರ್ಥ್ಯಗಳಲ್ಲಿ ವ್ಯತ್ಯಾಸವಿದೆ, ನಿಜ. ನಾವು ಯಾರನ್ನೂ ಅನುಕರಣೆ ಮಾಡಬೇಕಿಲ್ಲ. ಆದರೆ ಹಿಡಿದ ಕೆಲಸದಲ್ಲಿ ಸಫಲತೆಯನ್ನು ಪಡೆಯ ಬಲ್ಲೆವೇನು? ಉತ್ಸಾಹ ಆನಂದದಿಂದ ಇರಬಲ್ಲೆವೇನು? ನಿಶ್ಚಿಂತೆ ನಿರ್ಭೀತಿಯಿಂದ ಬದುಕಿನ ದಾರಿಯನ್ನು ಕ್ರಮಿಸಬಲ್ಲೆವೇನು? ಅನಾಥ ಪ್ರಜ್ಞೆಯನ್ನು ಮೆಟ್ಟಿ ನಿಲ್ಲಬಲ್ಲೆವೇನು? ದೈನ್ಯ ದಾಸ್ಯ ಕೀಳರಿಮೆಯ ಭಾವನೆಗಳಿಂದ ಮೇಲೇರಬಲ್ಲೆವೇನು?

ಧರ್ಮ, ಕಲೆ, ಸಾಹಿತ್ಯ, ಕೃಷಿ, ವಿಜ್ಞಾನ, ಯಾವುದೇ ಕ್ಷೇತ್ರಗಳಲ್ಲಿ ಅತ್ಯುನ್ನತ ಸಿದ್ಧಿಯು ಪ್ರಾಮಾಣಿಕ ಅವಿಶ್ರಾಂತ ಶ್ರಮದಿಂದಲೇ ಸಾಧ್ಯ. ಜನಾಂಗದ ಜ್ಞಾನ ಭಂಡಾರವನ್ನು ಉಳಿಸಿ ಬೆಳೆಸಿದವರೆಲ್ಲ ಕಾಯಕದ ಮಹಿಮೆ, ಮಹಾತ್ಮ್ಯೆಗಳಿಗೆ ಉಜ್ವಲ ನಿದರ್ಶನರಾಗಿ ರಾರಾಜಿಸು\-ತ್ತಾರೆ. ಅವರ ಅಭೀಪ್ಸೆಯ ತೀವ್ರತೆ, ಅವರು ಪಟ್ಟ ಶ್ರಮ, ಅವರ ಮಾನಸಿಕ ಏಕಾಗ್ರತೆ, ಅವರು ಏರಿದ ಎತ್ತರ–ಬದುಕಿನ ಯಾವ ಕ್ಷೇತ್ರದಲ್ಲಿ ದುಡಿಯುವವರಿಗೂ ನಿತ್ಯಸ್ಫೂರ್ತಿಯ ಸೆಲೆ ಯಲ್ಲವೇ? ನಾವೂ ಹಿಡಿದ ಕೆಲಸದಲ್ಲಿ ಸಫಲತೆಯನ್ನು ಪಡೆಯಬೇಕಾದರೆ, ಉತ್ಸಾಹ ಆನಂದ ದಿಂದ ಅದನ್ನು ಮಾಡುತ್ತ ಹೋಗಬೇಕು. ಧೈರ್ಯದಿಂದ ಮುನ್ನಡೆಯುತ್ತಾ ಸಾಗಬೇಕು.


\section*{ಉದ್ಯಮದಿಂದ ಉನ್ನತಿ}

\addsectiontoTOC{ಉದ್ಯಮದಿಂದ ಉನ್ನತಿ}

ಎಂಟು ಗಂಟೆಗಳಿಗೂ ಹೆಚ್ಚು ಕಾಲ ಕೆಲಸದಲ್ಲಿ ಮಗ್ನನಾಗಿದ್ದು ನಗುಮುಖದಿಂದ ಮನೆಗೆ ಮರಳುವ ಯುವಕನೊಬ್ಬನ ಪರಿಚಯ ನನಗಿದೆ. ಕೆಲಸದ ಬಳಿಕ ಆತನ ಉತ್ಸಾಹ, ಸಂತೋಷ ಕಡಿಮೆ\-ಯಾದು\-ದನ್ನು ನಾನು ಕಂಡಿಲ್ಲ. ಆಫೀಸಿನಿಂದ ಮೂರು ಕಿಲೋಮೀಟರ್ ದೂರದ ಹಳ್ಳಿಯ ಒಂದು ಪುಟ್ಟ ಕೋಣೆಯಲ್ಲಿ ಆತನ ವಾಸ. ಅಲ್ಲಿ ಯಾವ ಆಧುನಿಕ ಸೌಕರ್ಯಗಳೂ ಇಲ್ಲ. ಕೆಲಸದ ನಂತರ ನಡೆದುಕೊಂಡೇ ಅಲ್ಲಿಗೆ ಮರಳುತ್ತಾನೆ. ಸ್ವತಃ ಅಡುಗೆ ಮಾಡಿಕೊಳ್ಳುತ್ತಾನೆ. ದಿನದಲ್ಲಿ ಒಂದೆರಡು ಗಂಟೆಗಳ ಕಾಲ ಅಧ್ಯಯನ ಮಾಡುತ್ತಾನೆ. ಕಾಲೇಜಿಗೆ ಹೋಗಲು ಅವಕಾಶವಾಗದ ಆತ ಖಾಸಗಿ ಪರೀಕ್ಷೆ ಕಟ್ಟುತ್ತ ಪದವೀಧರನೂ ಆದ. ಆಫೀಸಿನಲ್ಲಿ ಆತನ ಹಿರಿಯ ಮೇಲ್ವಿಚಾರಕರಿಂದ ಆರಂಭಿಸಿ ಸಹೋದ್ಯೋಗಿಗಳಾಗಲೀ ಇತರರಾಗಲೀ ಅವನನ್ನು ಕುರಿತು ಒಂದು ಕೊಂಕು ಬಿರುನುಡಿಯನ್ನೂ ಆಡಿಲ್ಲ. ಎಲ್ಲರೂ ಆತನನ್ನು ಸಹಜವಾದ ವಿಶ್ವಾಸ ಗೌರವಗಳಿಂದಲೇ ಕಾಣುತ್ತಾರೆ. ‘ವಿದ್ಯೆಯು ವಿನಯವನ್ನು ನೀಡುತ್ತದೆ’ ಎನ್ನುವ ಮಾತು ಆತನ ಬಗೆಗೆ ಸಾರ್ಥಕವಾಗಿತ್ತು. ತನ್ನ ಕಾರ್ಯಕ್ಕಾಗಿ ಯಾರಿಗೂ ಸಲಾಮು ಹೊಡೆಯದೆ, ವಾಮಮಾರ್ಗ ಹಿಡಿಯದೇ ಬೆಳಗಿದ ಚೇತನ ಆತ. ಕಾರ್ಯದಕ್ಷತೆ, ನಿಯಮನಿಷ್ಠೆ, ತಾಳ್ಮೆ ಮತ್ತು ಎಂದಿಗೂ ಬತ್ತದ ಕಾರ್ಯೋತ್ಸಾಹ–ಇವುಗಳ ಬಲದಿಂದಲೇ ಎಲ್ಲರೂ ಹಾತೊರೆಯುವ ಸಂಸ್ಥೆಯ ಬಹು ಉನ್ನತ ಸ್ಥಾನಕ್ಕೆ ಏರಿದ ಆತ. ತನ್ನ ಆಫೀಸಿನ ನೌಕರರಿಗೆ ಉದಾರಿ ಉಪಕಾರಿಯಾದರೂ ಸೌಜನ್ಯದ ದುರುಪಯೋಗಮಾಡಲು ಅವಕಾಶ ನೀಡದ ಗಂಭೀರ ಸ್ವಭಾವದವನೂ ಹೌದು. ಅನ್ಯಾಯ ದೌರಾತ್ಮ್ಯಗಳಿಗೆ ಹೆದರದೇ ಎದೆಯೊಡ್ಡಿ ಎದುರಿಸಬಲ್ಲ ಛಾತಿ ಇರುವ ನ್ಯಾಯನಿಷ್ಠ ಹೋರಾಟಗಾರನೂ ಹೌದು. ಇಂದೀಗ ಸದ್ಗೃಹಸ್ಥನಾಗಿ ಮೂರು ಮಕ್ಕಳ ತಂದೆಯಾದರೂ, ಯಾವ ದುರಭ್ಯಾಸಕ್ಕೂ ಒಳಗಾಗದ ಸಂಯಮದ ಜೀವನ ಅವನದು. ದಣಿವಿಲ್ಲದ ಕಾರ್ಯೋತ್ಸಾಹ, ಬಹು ಜನರಿಗೆ ಉಪಕಾರ ಮಾಡಿದಾಗ ವ್ಯಕ್ತಿಯೊಬ್ಬನಿಗೆ ಬರಬಹುದಾದ ತೃಪ್ತಿ ಮತ್ತು ಧನ್ಯತೆಯ ಭಾವ, ಆಂತರಿಕ ಸಂತೃಪ್ತಿಯನ್ನು ಸೂಚಿಸುವ ಮುಖಮುದ್ರೆ–ಇವು ಆತನಲ್ಲಿ ನಾನು ಕಂಡ ವೈಶಿಷ್ಟ್ಯ. ಆತನನ್ನು ಸಂಧಿಸಿದಾಗ ನಾನು ಕೇಳಿದ ಪ್ರಶ್ನೆಗಳು ಇವು: ‘ನೀನು ನಿನ್ನ ಬದುಕಿನಲ್ಲಿ ಕಂಡುಹಿಡಿದ ಸಫಲತೆಯ ಸೂತ್ರ ಯಾವುದು? ನಿನ್ನ ಕಾರ್ಯವೈಶಿಷ್ಟ್ಯ ಏನು? ಬದುಕಿನ ಸಮಸ್ಯೆಗಳನ್ನು ನೀನು ಎದುರಿಸಿದ ಬಗೆ ಹೇಗೆ?’

ಆತನಂದ: ‘ಹೇಳಿಕೊಳ್ಳುವ ಅಂಥ ವೈಶಿಷ್ಟ್ಯವಿದೆ ಎನ್ನಲಾರೆ. ಒಟ್ಟಿನಲ್ಲಿ ದೇವರ ಕೃಪೆ ಎನ್ನುತ್ತೇನೆ. ಆದರೆ ಇಷ್ಟು ಹೇಳಬಹುದೋ ಏನೋ! ಮಾಡಬೇಕಾದ ಕೆಲಸಗಳನ್ನೆಲ್ಲಾ ಶ್ರದ್ಧೆಯಿಂದ ಮಾಡಬೇಕು ಎಂಬ ಮಾತನ್ನು ಹಿರಿಯರಿಂದ ಆಗಾಗ ಕೇಳಿದ್ದೆ. ಬಾಲ್ಯದಿಂದಲೇ ಕೆಲವೊಂದು ಜವಾಬ್ದಾರಿಯ ಕೆಲಸಗಳನ್ನು ನಿರ್ವಹಿಸಲೇಬೇಕಿತ್ತು. ಅದನ್ನು ಸರಿಯಾಗಿ ನಿರ್ವಹಿಸದಿದ್ದಲ್ಲಿ ನಿಂದೆ, ಬೈಗಳು, ಅಪವಾದಗಳಿಗೆ ಗುರಿಯಾಗುವ ಸಂಭವವಿತ್ತು. ಮಾನಸಿಕ ಚಂಚಲತೆ, ಚಪಲತೆಗಳಿಗೆ ಕೊರತೆ ಇರದಿದ್ದರೂ ಅಪವಾದ ನಿಂದೆಗಳಿಂದ ತಪ್ಪಿಸಿಕೊಳ್ಳಲು ಕೆಲಸ ಮಾಡಿದೆನೆಂದು ತೋರುತ್ತದೆ. ಯಾರೋ ‘ಹುಡುಗ ಚೆನ್ನಾಗಿ ಕೆಲಸ ಮಾಡುತ್ತಾನೆ’ ಎಂದು ಪರೋಕ್ಷದಲ್ಲಿ ಹೇಳಿದ ಮಾತು ಕಿವಿಗೆ ಬಿತ್ತು. ಅದು ಉತ್ಸಾಹದ ಗರಿಗೆದರಲು ಸಹಾಯ ಮಾಡಿರಬಹುದು. ಹಿರಿಯರೊಬ್ಬರು ಆಗಾಗ ‘ನೀನು ಯಾವುದೇ ಕೆಲಸವನ್ನು ಚೆನ್ನಾಗಿ ಮಾಡಬಲ್ಲೆಯಾದರೆ\break ಯಾರಿಗೂ ಹೆದರಬೇಕಾದ ಸಂದರ್ಭ ಬರುವುದಿಲ್ಲ’ ಎಂದು ಹೇಳಿದ್ದು ನೆನಪಿದೆ. ಇದೇ ಸಂದರ್ಭದಲ್ಲಿ ಒಂದು ಸಂಗತಿಯನ್ನು ಬಹುಬೇಗನೆ ತಿಳಿದುಕೊಂಡೆ. ಕೆಲಸವನ್ನು ಚೆನ್ನಾಗಿ ನಿರ್ವಹಿಸಲು ಕಲಿತವನ ಪಾಲಿಗೆ ಆತ್ಮವಿಶ್ವಾಸ ಅಪಾರಮಟ್ಟದಲ್ಲಿ ವೃದ್ಧಿಯಾಗುತ್ತದೆ, ಸ್ವಂತಿಕೆ ಬೆಳೆಯುತ್ತದೆ. ಇತರರಿಗೆ ಮಾರ್ಗದರ್ಶನ, ಉಪಕಾರ ಮಾಡುವ ಅವಕಾಶಗಳು ದೊರಕುತ್ತವೆ. ಸೋಲನ್ನು ಎದುರಿಸುವ ಸಾಮರ್ಥ್ಯ ಉಂಟಾಗುತ್ತದೆ. ಮೇಲಿನ ಸ್ಥಾನಗಳಿಗೆ ಏರಲು ಸ್ವಾಭಾವಿಕವಾಗಿ ಬಾಗಿಲು ತೆರೆದಂತಾಗುತ್ತದೆ. ದಕ್ಷತೆಯ ದುಡಿಮೆಯಿಂದ ದೊರಕುವ ಒಂದು ಸಣ್ಣ ಯಶಸ್ಸೂ, ವಿಜಯವೂ ಮುಂದಿನ ಪ್ರಗತಿಗೆ ಪ್ರೇರಕವಾಗುತ್ತದೆ. ಅಹಂಕಾರ, ಅಭಿಮಾನ, ಒಣ ಜಂಭಗಳಿಲ್ಲದಿದ್ದರೆ ತಡವಾಗಿಯಾದರೂ ಸಹೋದ್ಯೋಗಿಗಳ, ಕ್ರಮೇಣ ಮೇಲಧಿಕಾರಿಗಳ ಮತ್ತು ಜನರ ವಿಶ್ವಾಸ ಗೌರವಗಳೂ, ಅವರು ಬಾಯಿಬಿಟ್ಟು ಹೇಳದಿದ್ದರೂ ಲಭಿಸಿಯೇ ಲಭಿಸುತ್ತವೆ. ಇದರ ಜೊತೆಗೆ ದಿನದಿನವೂ ನಮ್ಮ ಕಾರ್ಯಸಾಮರ್ಥ್ಯ ಹೆಚ್ಚುತ್ತ ಹೋಗುತ್ತದೆ.’

‘ಚೆನ್ನಾಗಿ ದುಡಿಯುವವರಿಗೇ ಕೆಲವೊಮ್ಮೆ ಕೆಲಸದ ಹೊರೆ ಹೊರಿಸುತ್ತಾರೆ ಎಂದು ಕೇಳಿದ್ದೇನೆ –ಬಗ್ಗಿದವನಿಗೆ ಒಂದು ಗುದ್ದು ಹೆಚ್ಚು ಎಂದಂತೆ.’

‘ಹೌದು, ಹೊರೆಯಾಗುವುದೂ ಹೌದು. ಮೊದಲು ಇಂಥ ಸಣ್ಣ ಪುಟ್ಟ ತಲೆನೋವು ತರುವ ಹೊರೆ ಹೊರಲು ಸಿದ್ಧರಿರಬೇಕು. ಇಂಥ ಸಂದರ್ಭಗಳಲ್ಲಿ ಕೆಲವೊಮ್ಮೆ ಸೋತಂತೆ ಎನಿಸಿದರೂ ತಾಳ್ಮೆ ಇದ್ದರೆ, ಅದನ್ನು ನಮ್ಮ ಕಾರ್ಯ ಸಾಮರ್ಥ್ಯಕ್ಕೆ ಸವಾಲಾಗಿ ಸ್ವೀಕರಿಸಿದರೆ ಅದು ಸೋಲಲ್ಲ ಎಂದೇ ನನ್ನ ಅನುಭವ. ಕೆಲಸದ ಬಗ್ಗೆ ಒಂದು ಸಿದ್ಧಾಂತವನ್ನು ಆವಿಷ್ಕರಿಸಿದ್ದೇನೆ. ಉದ್ಯೋಗ\-ವೆಂದರೆ ಕೇವಲ ಹೊಟ್ಟೆಹೊರೆದುಕೊಳ್ಳಲು ಮಾಡುವ ಒಂದು ದುಡಿಮೆ ಮಾತ್ರವಲ್ಲ, ಪ್ರತಿಯೊಬ್ಬನ ಆಂತರ್ಯದಲ್ಲೂ ಅಡಗಿರುವ ಆನಂದದ ಬುಗ್ಗೆಯನ್ನು ತಡೆಹಿಡಿಯುವ ಅಡ್ಡಗೋಡೆಯನ್ನು ದೂರಕ್ಕೆಸೆಯುವ ಉಪಾಯವೇ ಕೆಲಸ. ಆ ಉದ್ಯೋಗ ಈ ಉದ್ಯೋಗವೆಂದಿಲ್ಲ, ಪಾಲಿಗೆ ಬಂದ ಯಾವ ಕೆಲಸವನ್ನೇ ಆಗಲಿ ಪೂರ್ಣ ಏಕಾಗ್ರತೆಯಿಂದ ಮಾಡಿದರೆ ಮನುಷ್ಯರ ಹೃದಯದಲ್ಲಿ ಆನಂದ ಸಂತೃಪ್ತಿ ಉಕ್ಕುತ್ತವೆ. ಕೇವಲ ಸಂಬಳ ಬಡ್ತಿಗಳಿಂದಲೇ ನಮ್ಮ ಕೆಲಸದ ಗುಣ ಮಟ್ಟ ವೃದ್ಧಿಯಾಗುತ್ತದೆ ಎನ್ನುವುದು ಸುಳ್ಳು. ಕೆಲಸದ ಬಗೆಗಿನ ದೃಷ್ಟಿಕೋನ ಮತ್ತು ದಕ್ಷತೆಯ ನಿರ್ವಹಣೆಗಳಿಂದ ಅದು ವೃದ್ಧಿಯಾಗುತ್ತದೆ. ಕೈಗೊಂಡ ಕಾರ್ಯದಲ್ಲಿ ಪೂರ್ಣ ಮನಸ್ಸು ಕೊಟ್ಟು ಅತ್ಯುತ್ತಮ ಸಿದ್ಧಿ ಪಡೆದರೇ ನೈಜ ಆನಂದ ಲಭ್ಯವಾಗುವುದೆಂಬುದನ್ನು ತಿಳಿದಾತ ಅದು ಎಷ್ಟು ಸಣ್ಣ ಕೆಲಸವಾದರೂ ತನ್ನ ಪಾಲಿನ ಕರ್ತವ್ಯವನ್ನು ಅಸಡ್ಡೆಯಿಂದ ಮಾಡಲಾರ.’

ನಾನು ಕೇಳಿದೆ–‘ಈ ಮಾತು ಕೇವಲ ಯಾಂತ್ರಿಕ ಮತ್ತು ಏಕತಾನತೆಯ ಕೆಲಸದಲ್ಲೂ ಸಫಲ\-ವಾಗುತ್ತದೆಯೇ? ಸೃಜನಾತ್ಮಕ ಕೆಲಸಗಳಲ್ಲಿರುವ ಸಹಜ ಆನಂದ ಸ್ಫೂರ್ತಿಗಳು ‘ಬೋರ್​’ ಎನ್ನಿಸುವ ಕೇವಲ ಯಾಂತ್ರಿಕವಾದ ಕೆಲಸಗಳಲ್ಲೂ ಲಭಿಸುತ್ತವೆಯೇ?’

‘ಸ್ವಲ್ಪವೂ ಯಾಂತ್ರಿಕತೆ ಇಲ್ಲದ ಸೃಜನಾತ್ಮಕ ಕೆಲಸವೆಂಬುದೊಂದು ಇದೆಯೇ ಎಂಬುದು ನನಗೆ ತಿಳಿಯದು. ಪ್ರತಿಯೊಂದು ಪ್ರತಿಭೆಯ ಹಿಂದೆ, ಕಲಾಕೃತಿಯ ಹಿಂದೆ ಶ್ರಮವಿದ್ದೇ ಇದೆ\-ಯಲ್ಲವೇ? ಆ ವಿಚಾರ ಏನೇ ಇರಲಿ, ಮಾಡಲೇ ಬೇಕಾದ ಕೆಲಸವನ್ನು ಸರಿಯಾಗಿ ಮಾಡಲಾಗದಿದ್ದರೆ ಅದೊಂದು ಭಾರವಾದ ಹೊರೆಯಾದಂತಾಗಿ ನಮ್ಮ ಮನಸ್ಸಿನ ಮೇಲೆ ತನ್ನ ದುಷ್ಪರಿಣಾಮ ಬೀರುವುದು ಎಲ್ಲರಿಗೂ ತಿಳಿದ ವಿಷಯ. ನೀರಸವೆನಿಸಿದರೂ ಕರ್ತವ್ಯವನ್ನು ಉತ್ತಮ ರೀತಿಯಲ್ಲಿ ಮಾಡಿ ಮುಗಿಸಿದಾಗ ಶಾಂತಿ ಸಮಾಧಾನ ಸಂತೃಪ್ತಿಯ ಭಾವನೆಯುಂಟಾಗುವುದು ಎಲ್ಲರ ಅನುಭವಕ್ಕೆ ಬರುವ ವಿಚಾರವೇ.’

ಪಾಲಿಗೆ ಬಂದ ಕೆಲಸವನ್ನು ಅತ್ಯುತ್ತಮ ರೀತಿಯಲ್ಲಿ ನಿರ್ವಹಿಸುವ ಅಭ್ಯಾಸವನ್ನು ನೀವು ಗಳಿಸಿಕೊಂಡಿದ್ದೀರಾ? ಆ ಅಭ್ಯಾಸ ನಿಮ್ಮ ಸತ್ವಶಕ್ತಿಗಳನ್ನು ಅಪಾರ ಮಟ್ಟದಲ್ಲಿ ಹೆಚ್ಚಿಸಬಲ್ಲದು. ಕೆಲಸದ ಗುಣಮಟ್ಟ ಹೆಚ್ಚಿದಂತೆ ನಿಮ್ಮ ವ್ಯಕ್ತಿತ್ವದ ಗುಣಮಟ್ಟವೂ ವೃದ್ಧಿಯಾಗಲೇಬೇಕು.


\section*{ಸಾಮಾನ್ಯನು ಸಮರ್ಥನಾದ}

\addsectiontoTOC{ಸಾಮಾನ್ಯನು ಸಮರ್ಥನಾದ}

ಕೆನಡಾದ ಅತ್ಯಂತ ಪ್ರಸಿದ್ಧರಾದ ವೈದ್ಯಶಾಸ್ತ್ರಜ್ಞರು ಡಾ.\ ವಿಲಿಯಂ ಒಸ್ಲರ್. ಅವರು ತಮ್ಮ ವಿದ್ಯಾರ್ಥಿಜೀವನದಲ್ಲಿ ಅತಿಯಾದ ಚಿಂತೆಗೊಳಗಾದ ವ್ಯಕ್ತಿಯಾಗಿದ್ದರು. ಮೋಂಟ್ರೀಲಿನ ಜನರಲ್ ಆಸ್ಪತ್ರೆಯಲ್ಲಿ ವೈದ್ಯಕೀಯ ವ್ಯಾಸಂಗ ಮಾಡುತ್ತಿದ್ದ ಅವರಿಗೆ ತಮ್ಮ ಭವಿಷ್ಯವನ್ನು ಕುರಿತ ಚಿಂತೆ ಭಯ ಉದ್ವೇಗ ಸಂದೇಹ ಗೊಂದಲಗಳ ಪೀಡೆ ಪ್ರಾರಂಭವಾಗಿತ್ತು. ಕೊನೆಯ ವರ್ಷದ ಪರೀಕ್ಷೆಗಾಗಿ ಅಷ್ಟೊಂದು ಬೃಹತ್ ಗ್ರಂಥಗಳ ಅಧ್ಯಯನ ತಮ್ಮಿಂದ ಸಾಧ್ಯವಾಗುವುದೇ? ಅಷ್ಟೊಂದು ವಿಚಾರಗಳನ್ನು ಒಂದು ವರ್ಷದಲ್ಲೇ ತಲೆಯೊಳಗೆ ತುಂಬಿಕೊಳ್ಳಲಾಗುವುದೇ? ಅವುಗಳನ್ನೆಲ್ಲ ನೆನಪಿನಲ್ಲಿರಿಸಿಕೊಂಡು ಪರೀಕ್ಷೆ ಬರೆಯಲು ತನ್ನಿಂದಾದೀತೆ? ಎಂಬ ಭಯ ಸಂದೇಹಗಳು ಅವರನ್ನು ಕಾಡುತ್ತಿದ್ದವು. ಒಂದು ವೇಳೆ ಪರೀಕ್ಷೆಯಲ್ಲಿ ಉತ್ತೀರ್ಣರಾದರೂ ಮುಂದೆ ನೌಕರಿಯ ಚಿಂತೆ ಬೇರೆ. ಸ್ವಂತ ಪ್ರಾಕ್ಟೀಸ್ ಮಾಡಲೇ ಹೇಗೆ? ಅದಕ್ಕಾದರೊ ಹಣ ಬೇಕು, ಅದೃಷ್ಟವೂ ಬೇಕು. ಈ ಸ್ಪರ್ಧಾಮಯ ಜಗತ್ತಿನಲ್ಲಿ ಜೀವನಯಾತ್ರೆ ಸುಗಮವಾದೀತೆ? ಒಟ್ಟಿನಲ್ಲಿ ತಮ್ಮ ಭವಿಷ್ಯ ಜೀವನದಲ್ಲಿ ಯಶಸ್ಸು ವಿಜಯ ಸಿಕ್ಕೀತೇ, ದಕ್ಕೀತೇ ಎಂಬಂಥ ಉದ್ವೇಗ ಸಂಶಯ ಚಿಂತೆಗಳ ಕಾರ್ಮೋಡಗಳು ಅವರ ಮನಸ್ಸನ್ನು ಸಾಕಷ್ಟು ತಲ್ಲಣಗೊಳಿಸಿದ್ದವು. ಆದರೆ ಅಕಸ್ಮಾತ್ತಾಗಿ ಅವರ ಕಣ್ಣಿಗೆ ಬಿದ್ದ ಕಾರ್ಲೈಲನ ಒಂದು ನುಡಿ, ಒಂದೇ ಒಂದು ನುಡಿ ‘ನಮ್ಮ ಅತಿ ಮುಖ್ಯವಾದ ಕರ್ತವ್ಯ ಸಮೀಪದಲ್ಲಿರುವ ಕಾರ್ಯವನ್ನು ಯಶಸ್ವಿಯಾಗಿ ನಿರ್ವಹಿಸುವುದರಲ್ಲಿದೆ. ದೂರದ ದಿಗಂತದಲ್ಲಿ ಅಸ್ಪಷ್ಟವಾಗಿ ಗೋಚರಿಸುವ ಭವಿಷ್ಯದೆಡೆ ನೋಡುವುದರಲ್ಲಲ್ಲ,’\footnote{\engfoot{Our main business is not to see what lies dimly at a distance, but to do what lies clearly at hand}\hfill\engfoot{ —Thomas Carlyle}} –ಅವರ ಬದುಕಿನ ದಿಕ್ಕನ್ನೇ ಬದಲಿಸಿತು, ವಿಷಣ್ಣ ಮನಃಸ್ಥಿತಿಯಿಂದ ಅವರನ್ನು ಮೇಲಕ್ಕೆತ್ತಿತು. ಅಸಾಮಾನ್ಯ ವ್ಯಕ್ತಿಯಾಗಲು ಬೇಕಾದ ಸ್ಫೂರ್ತಿಯ ಕಿಡಿಯನ್ನು ಹೃದಯದಲ್ಲಿ ಹೊತ್ತಿಸಿತು. ಅವರನ್ನು ಮಹಾ ಕರ್ಮವೀರರನ್ನಾಗಿಸಿತು. ಆ ನುಡಿ ಅವರ ಬಾಳಿನ ಯಶಸ್ಸು ವಿಜಯಗಳ ಮೂಲಮಂತ್ರವಾಯಿತು.

\vskip 2pt

ಮುಂದೆ ಅವರು ಹೆಸರಾಂತ ಜಾನ್ ಹಾಪ್​ಕಿನ್ಸ್ ಸಂಸ್ಥೆಯ ಸರ್ವತೋಮುಖ ಪ್ರಗತಿಗೆ ಕಾರಣರಾದರು. ನಾಲ್ಕು ವಿಶ್ವವಿದ್ಯಾಲಯಗಳಲ್ಲಿ ಪ್ರಾಧ್ಯಾಪಕರಾಗಿ ದುಡಿದರು. ಆಕ್ಸ್​ಫರ್ಡಿನ ವೈದ್ಯವಿಭಾಗದ ಮಹಾಪ್ರಾಧ್ಯಾಪಕರಾದರು. ಇಂಗ್ಲೆಂಡ್ ಕೊಡಮಾಡಿದ ಅನೇಕ ಗೌರವಗಳಿಗೆ ಪಾತ್ರರಾದರು.

\vskip 2pt

ವಿಲಿಯಂ ಒಸ್ಲರ್ ಒಮ್ಮೆ ಯೇಲ್ ವಿಶ್ವವಿದ್ಯಾನಿಲಯದ ವಿದ್ಯಾರ್ಥಿಗಳನ್ನು ಉದ್ದೇಶಿಸಿ ಭಾಷಣ ಮಾಡುತ್ತಾ ‘ಜನಪ್ರಿಯ ಗ್ರಂಥಕರ್ತನಾಗಿ, ಆಕ್ಸ್​ಫರ್ಡ್​ನಂಥ ಉನ್ನತ ಮಟ್ಟದ ವಿಶ್ವ\-ವಿದ್ಯಾ\-ನಿಲಯದ ಮಹಾಪ್ರಾಧ್ಯಾಪಕನಾಗಿ ಪ್ರಸಿದ್ಧನಾದ ನನ್ನನ್ನು ಹುಟ್ಟಿನಿಂದಲೇ ಮಹಾ ಮೇಧಾವಿ ಎಂದು ಹಲವರು ಊಹಿಸಿರಬಹುದು. ಆದರೆ ವಸ್ತುಸ್ಥಿತಿ ಹಾಗಿಲ್ಲ. ನಾನು ಅತ್ಯಂತ ಸಾಮಾನ್ಯ ಮಿದುಳು ಶಕ್ತಿಯುಳ್ಳ ವ್ಯಕ್ತಿಯಾಗಿದ್ದೆನೆಂಬುದು ನನ್ನ ಆಪ್ತ ಮಿತ್ರರಿಗೆ ತಿಳಿದಿದೆ’ ಎಂದರಂತೆ.


\section*{ರಹಸ್ಯವೇನು?}

\addsectiontoTOC{ರಹಸ್ಯವೇನು ?}

ಹಾಗಾದರೆ ಅವರ ಅದ್ಭುತವೆನಿಸುವ ಈ ವಿಜಯದ ರಹಸ್ಯವೇನು? ಒಸ್ಲರ್ ‘ವರ್ತಮಾನ ಕಾಲದಲ್ಲಿ’ ಬದುಕಿದೆ ಎನ್ನುತ್ತಾರೆ. ಎಂದರೆ? ಭವಿಷ್ಯದ ಸುಖಸ್ವಪ್ನಗಳನ್ನು ಕಲ್ಪಿಸಿಕೊಳ್ಳುತ್ತ ಸೋಮಾರಿಯಾಗಿ ಕುಳಿತುಕೊಳ್ಳಲಿಲ್ಲ. ಗತಕಾಲದ ಘಟನೆಗಳನ್ನು ಮೆಲುಕು ಹಾಕುತ್ತ, ಕೊರ ಗುತ್ತ ಅನಾವಶ್ಯಕ ಚಿಂತೆಯಲ್ಲಿ ಕಾಲ ಕಳೆಯಲಿಲ್ಲ. ಅವರ ಮಾನಸಿಕ ಶಕ್ತಿಯು ವರ್ತಮಾನದಲ್ಲಿ ಮಾಡಲೇಬೇಕೆಂದು ಯೋಚಿಸಿಕೊಂಡಿದ್ದ ಕೆಲಸಗಳನ್ನು ದಕ್ಷತೆಯಿಂದ ಮಾಡುವುದರಲ್ಲೇ ಲೀನ ವಾಯಿತು. ಈ ಸಾಧನೆಗಳಿಂದಲೇ ಅವರು ಅತ್ಯುನ್ನತ ಸಿದ್ಧಿಯನ್ನು ಪಡೆದರು.

\vskip 2pt

ಒಸ್ಲರ್ ಹೇಳಿದ ಮಾತುಗಳಿವು: ‘ಭವಿಷ್ಯವು ಇಂದೇ ರೂಪಿತವಾಗುತ್ತದೆ. ನಿನ್ನೆಯ ಚಿಂತೆ ಬಿಡಿ. ನಾಳಿನ ಘಟನೆಗಳ ಬಗೆಗೆ ಕಾತರತೆಯಿಂದ ಕನಸು ಕಾಣುತ್ತಾ ಉದ್ವಿಗ್ನರಾಗಬೇಡಿ. ಭೂತ ಮತ್ತು ಭವಿಷ್ಯತ್ಕಾಲಗಳ ಬಾಗಿಲನ್ನು ಭದ್ರವಾಗಿ ಮುಚ್ಚಿ. ವರ್ತಮಾನದಲ್ಲೇ ನಿರ್ದಿಷ್ಟ ಕಾರ್ಯ ಸೀಮೆಯನ್ನು ಯೋಚಿಸಿ ನಿಶ್ಚಿಂತೆಯಿಂದ ದುಡಿಯುವ ಅಭ್ಯಾಸಮಾಡಿ. ಈ ಕ್ಷಣ ಕಾರ್ಯೋನ್ಮುಖರಾಗಿ.’

ಯೋಜಿಸಿಕೊಂಡ ದಿನದ ಕೆಲಸವನ್ನು ಅತ್ಯುತ್ತಮ ರೀತಿಯಲ್ಲಿ ನಿರ್ವಹಿಸುವಲ್ಲಿಗೇ ನಿಮ್ಮ ಮಹತ್ವಾಕಾಂಕ್ಷೆ ಸೀಮಿತವಾಗಲಿ. ವಿಜಯದ ದಾರಿಯಲ್ಲಿ ನಡೆಯುವ ಯಾತ್ರಿಕರು ವರ್ತಮಾನದಲ್ಲೇ ಹೆಚ್ಚು ಚಟುವಟಿಕೆಯಿಂದಿರುತ್ತಾರೆ. ನಾಳಿನ ಯೋಚನೆ ಮನಸ್ಸನ್ನು ಸ್ವಲ್ಪವೂ ಪ್ರವೇಶಿಸದಂತೆ ನೋಡಿಕೊಳ್ಳುತ್ತಾರೆ. ನಿನ್ನೆ ನಾಳೆಗಳ ಚಿಂತೆ ಚಿಂತನೆಗಳು ಬೇಡ. ನಿಮ್ಮ ಪೂರ್ಣಶಕ್ತಿಯು ದಿನದ ಕಾರ್ಯವನ್ನು ದಕ್ಷತೆಯಿಂದ ಮಾಡುವುದರಲ್ಲೇ ಮಗ್ನವಾಗುವಂಥ ಮಹತ್ವಾಕಾಂಕ್ಷೆಯನ್ನು ಹೊಂದಿರಲಿ.’

ಒಸ್ಲರ್ ಹೇಳಿದ ಮಾತುಗಳನ್ನೆ ಭಾರತದ ಪುಷಿಗಳು ಸಾವಿರಾರು ವರ್ಷಗಳ ಹಿಂದೆಯೇ ಹೇಳಿದ್ದರು. ನಾವು ಅದನ್ನೀಗ ಮರೆತಿದ್ದೇವೆ. ನಮ್ಮ ಬದುಕಿನಲ್ಲಿ ನಾವು ಅನುಭವಿಸುವ ಸೋಲು ಸಂಕಟಗಳಿಗೆ ಈ ಮಾತನ್ನು ಮರೆತುದೇ ಕಾರಣ. ಅದನ್ನು ಕಾರ್ಯಕ್ಕಿಳಿಸದುದೇ ಕಾರಣ. ಆ ಮಾತುಗಳು ಇಂತಿವೆ–‘ಬುದ್ಧಿವಂತರು ಗತಕಾಲದ ಘಟನೆಗಳನ್ನು ಕುರಿತು ಶೋಕಿಸುವುದಿಲ್ಲ. ಭವಿಷ್ಯದ ಬಗೆಗೆ ಚಿಂತಿಸುವುದಿಲ್ಲ. ಅವರು ವರ್ತಮಾನ ಕಾಲದಲ್ಲೇ ಕಾರ್ಯನಿರ್ವಹಿಸುತ್ತಾರೆ.’\footnote{ಗತಶೋಕಂ ನ ಕುರ್ವೀತ ಭವಿಷ್ಯಂ ನೈವ ಚಿಂತಯೇತ್~।\\\phantom{ವರ್ತ} ವರ್ತಮಾನೇಷು ಕಾರ್ಯೇಷು ವರ್ತಯಂತಿ ವಿಚಕ್ಷಣಾಃ~॥}

‘ಇಂದಿನ ದಿನ ಸುದಿನ, ನಾಳೆ ಎಂದರೆ ಅದು ಕಠಿಣ’ ಎಂದರು ಹರಿದಾಸರು. ‘ನೀನು ಈ ದಿನವನ್ನು ನೋಡಿಕೋ, ಅದು ನಾಳೆಯನ್ನು ನೋಡಿಕೊಳ್ಳುತ್ತದೆ’ ಎಂದು ಒಬ್ಬ ಆಂಗ್ಲ ವಿದ್ವಾಂಸ ಹೇಳಿದ, ‘ಈ ದಿನ’ ಎಂಬ ಒಂದೇ ಶಬ್ದವನ್ನು ತನ್ನ ಮೇಜಿನ ಮೇಲೆ ದೊಡ್ಡ ಅಕ್ಷರಗಳಲ್ಲಿ ಬರೆದಿರಿಸಿಕೊಂಡಿದ್ದ ಅಮರ ಸಾಹಿತಿ ರಸ್ಕಿನ್. ‘ನಿನ್ನೆ ಸತ್ತಿಹುದೀಗ ನಾಳೆ ಹುಟ್ಟದೇ ಇಹುದು, ಇಂದು ಸೊಗವಿರಲದನು ಮರೆತಳುವುದೇಕೆ?’ ಎಂದ ಉಮರ್ ಖಯ್ಯಾಮ್. ಇವರೆಲ್ಲರೂ ಉಪಯೋಗಿಸಿದ ಶಬ್ದಗಳ ಅರ್ಥ ಒಂದೇ–ವರ್ತಮಾನ ಕಾಲದಲ್ಲಿ ಕಾರ್ಯನಿರತರಾಗಿರುವುದು ಅವಶ್ಯ ಎಂದು.

ಈ ಹೊತ್ತು ಉತ್ತಮ ಬದುಕನ್ನು ಬದುಕಿದ್ದಾದರೆ ನಾಳೆಗದು ಒಂದು ಸಂತಸದ ಸವಿನೆನಪಾಗಿ ಮನಸ್ಸಿಗೆ ನೆಮ್ಮದಿ ನೀಡುವುದು. ಭವಿಷ್ಯವಿನ್ನೂ ನಮ್ಮ ಕೈಗೆ ಸಿಕ್ಕಿಲ್ಲ. ವರ್ತಮಾನ ಮಾತ್ರ ನಮ್ಮ ಕೈಯಲ್ಲಿದೆ. ಅದನ್ನು ಸರ್ವಪ್ರಯತ್ನದಿಂದ ಸದುಪಯೋಗಗೊಳಿಸಬೇಕು.

ಹಾಗಾದರೆ ನಾವು ಭವಿಷ್ಯವನ್ನು ಕುರಿತು ಯೋಚಿಸಬೇಡವೇ? ನಡೆದು ಬಂದ ದಾರಿಯನ್ನು ನೋಡಬೇಡವೇ? ನಮ್ಮ ಭವಿಷ್ಯ ನಿರ್ಮಾಣಕ್ಕೆ ತಕ್ಕ ಯೋಜನೆಗಳನ್ನು ಹಾಕಿಕೊಳ್ಳಬೇಡವೇ?


\section*{ಯೋಜನೆಯಿಂದ ಸಾಧನೆ}

\vskip -8pt\addsectiontoTOC{ಯೋಜನೆಯಿಂದ ಸಾಧನೆ}

ಯೋಜನೆ–‘ಪ್ಲಾನಿಂಗ್​’–ಆಧುನಿಕ ಯುಗದಲ್ಲಿ ವಿಶೇಷ ಬಳಕೆಯಲ್ಲಿರುವ ಶಬ್ದ. ಕನ್ನಡ ನಾಡಿನಲ್ಲಿ ಮೊದಲಿಗೆ ಬಹುವಿಧಯೋಜನೆಗಳನ್ನು ಕಾರ್ಯರೂಪಕ್ಕೆ ತಂದು ಸಮೃದ್ಧಿಯತ್ತ ಪಥ ಪ್ರದರ್ಶನ ಮಾಡಿದವರು ಸರ್ ಎಂ. ವಿಶ್ವೇಶ್ವರಯ್ಯನವರು. ಜಪಾನ್ ಸಾಧಿಸಿದ ಪ್ರಗತಿ ಅಭಿವೃದ್ಧಿಗಳು ಆ ದೇಶದವರು ಕೈಗೊಂಡ ಯೋಜನೆಗಳ ವಿಜಯ ಎನ್ನುತ್ತಾರೆ.

ಮೊದಲ ಬಾರಿಗೆ ಚಂದ್ರಗ್ರಹದ ಮೇಲೆ ಒಬ್ಬ ವ್ಯಕ್ತಿಯನ್ನು ಏರಿಸಲು ಹಾರಿಸಲು ಸುಮಾರು ಹತ್ತುಸಾವಿರ ತಾಂತ್ರಿಕ ತಜ್ಞರು ಹತ್ತು ವರ್ಷಗಳ ಕಾಲ ದುಡಿದಿದ್ದರು. ಈ ಯುಗದಲ್ಲಿ ಒಂದು ಬೃಹತ್ ಯೋಜನೆ ನೀಡಿದ ಮಹತ್ಫಲ ಅದು. ಭಾರತದ ಭವ್ಯ ಭವಿತವ್ಯದ ಕನಸು ಕಂಡವರು ಹಲವಾರು ಪಂಚವಾರ್ಷಿಕ ಯೋಜನೆಗಳನ್ನು ಹಾಕಿಕೊಂಡಿದ್ದರು. ತೀವ್ರಗತಿಯಿಂದ ಏರುತ್ತಿರುವ ಜನಸಂಖ್ಯೆಯಿಂದ ಮುಂದೆ ಕೆಲವೇ ವರ್ಷಗಳಲ್ಲಿ ಹೊತ್ತಿನ ತುತ್ತಿಗೂ ತತ್ತರಿಸುವ ಸ್ಥಿತಿ ಬರಬಹುದೆಂದು ತಿಳಿದು ಅದನ್ನು ತಡೆಯಲು ಕುಟುಂಬ ಯೋಜನೆಯ ಪ್ರಚಾರ ನಡೆದಿದೆ. ವಿಮಾ ಯೋಜನೆ ಸರ್ವವ್ಯಾಪಿಯಾಗಿಬಿಟ್ಟಿದೆ. ಸಮಾಜ, ಸರಕಾರ, ಸಂಘಸಂಸ್ಥೆಗಳು ತಮ್ಮ ತಮ್ಮ ವಿವಿಧಾವಧಿಯ ಯೋಜನೆಗಳನ್ನು ಮುಂದಿರಿಸಿಕೊಂಡಿವೆ. ಆಧುನಿಕ ಪ್ರಜ್ಞೆ ಇರುವ ಯಾವ ವ್ಯಕ್ತಿಯೂ ಭವಿಷ್ಯದ ಬಗೆಗೆ ಯೋಚನೆ ಮಾಡದಿರಲಾರ, ಯೋಜನೆ ಕೈಗೊಳ್ಳದಿರಲಾರ. ನಮ್ಮ ದೇಶದ ಪೂರ್ವಪುರುಷರು ಬದುಕಿನ ವ್ಯವಸ್ಥೆಯ ಬಗೆಗೂ, ಬದುಕಿನ ಆಚೆಗೆ ನೆಗೆಯುವ ಬಗೆಗೂ, ಅಲ್ಲಿಯ ಜೀವನದ ಬಗೆಗೂ, ಯೋಜನೆಗಳನ್ನು ಹಾಕಿಕೊಂಡಿದ್ದರು. ರಘುವಂಶದ ದೊರೆಗಳು ಶೈಶವದಲ್ಲಿ ಸಮಸ್ತ ವಿದ್ಯೆಯನ್ನೂ, ಯೌವನದಲ್ಲಿ ಸುಖಭೋಗವನ್ನು ಅನುಭವಿಸು ವಾಗಲೇ ಗೃಹಸ್ಥ ಜೀವನದ ಕರ್ತವ್ಯ ನಿರ್ವಹಣೆಯನ್ನೂ, ವಾರ್ಧಕ್ಯದಲ್ಲಿ ಧ್ಯಾನ ಮೌನ ಶಾಸ್ತ್ರಚಿಂತನೆ ಏಕಾಂತಪ್ರಿಯತೆ ಇವುಗಳಿಂದ ಮುನಿವೃತ್ತಿಯನ್ನೂ, ಯೋಗಶಕ್ತಿಯಿಂದ ತಮ್ಮ ದೇಹತ್ಯಾಗವನ್ನೂ ಮಾಡಬಲ್ಲಂಥವರು ಎಂದು ಕಾಳಿದಾಸ ವರ್ಣಿಸಿದ್ದಾನೆ. ಒಂದು ದೇಶದ ಭವಿಷ್ಯವನ್ನು ನಿರ್ಮಿಸಲು ಹೊರಟ ಮುಖಂಡರು ಆ ದೇಶವು ಸಹಸ್ರಾರು ವರ್ಷಗಳ ಹಿಂದಿನಿಂದ ನಡೆದು ಬಂದ ದಾರಿಯನ್ನು ನೋಡಬೇಕು ಎಂದು ವಿನ್​ಸ್ಟನ್ ಚರ್ಚಿಲ್​ರು ಹೇಳಿದ್ದರು. ವ್ಯಕ್ತಿಜೀವನದ ಭವಿಷ್ಯ ನಿರ್ಮಾಣದಲ್ಲೂ ಕೌಟುಂಬಿಕ ಪರಂಪರೆ ಮತ್ತು ಪರಿಸರದ ಪ್ರಜ್ಞೆ ಅತ್ಯಾವಶ್ಯಕ ಎಂಬುದನ್ನೂ ಅಲ್ಲಗಳೆಯಲಾಗದು.

ಭೂತಕಾಲದ ಘಟನಾವಳಿಗಳನ್ನು ಅವಲೋಕಿಸಿ ನಮ್ಮ ದೋಷ ದೌರ್ಬಲ್ಯಗಳನ್ನು, ಶಕ್ತಿ ಸೌಲಭ್ಯಗಳನ್ನು ತಿಳಿಯಬಾರದೆಂದಲ್ಲ. ‘ವರ್ತಮಾನದಲ್ಲಿ ಬದುಕು’ ಎಂದರೆ ಭವಿಷ್ಯದ ಬಗೆಗೆ ನಿರ್ದಿಷ್ಟ ಯೋಜನೆಗಳನ್ನು ಹಾಕಿಕೊಳ್ಳಬಾರದೆಂದಲ್ಲ. ಯೋಜನೆಗಳಲ್ಲಿ ಸಣ್ಣಪುಟ್ಟ ಬದಲಾವಣೆ ಕೂಡದೆಂದೂ ಅಲ್ಲ. ಆದರೆ ಹಾಕಿಕೊಂಡ ಯೋಜನೆಗಳಲ್ಲಿ ಯಶಸ್ಸು ಪಡೆಯಬೇಕಾದರೆ ವರ್ತಮಾನದ ಪೂರ್ಣ ಸದುಪಯೋಗವಾಗಬೇಕು, ಪ್ರಯತ್ನದ ಪರಾಕಾಷ್ಠೆಯನ್ನೇರಬೇಕು\break ಎಂಬುದು ಒಸ್ಲರ್ ಅವರು ಹೇಳಿದ ಮಾತಿನ ಮಥಿತಾರ್ಥ.

ಒಮ್ಮೆ ಒಬ್ಬ ಮನೋವಿಜ್ಞಾನಿ ಸುಮಾರು ಮೂರುಸಾವಿರ ಜನರನ್ನು ಕಂಡು ‘ನಿಮ್ಮ ಬದುಕಿನ ಆಧಾರವೇನು? ಏನು ಮಾಡಿಕೊಂಡಿದ್ದೀರಿ?’ ಎಂದು ಪ್ರಶ್ನಿಸಿದ. ಅವರು ನೀಡಿದ ವಿವಿಧ ತೆರನಾದ ಉತ್ತರಗಳನ್ನು ಕೇಳಿ ಚಕಿತನೂ ಆದ. ನೂರರಲ್ಲಿ ತೊಂಬತ್ನಾಲ್ಕು ಮಂದಿ ತಮ್ಮ ಬದುಕಿನ ಭವ್ಯ ಭವಿಷ್ಯವನ್ನು ಕಾಯುತ್ತ ವರ್ತಮಾನವನ್ನು ನಿರ್ದಿಷ್ಟ ಉದ್ದೇಶ ಹಾಗೂ ನಿಯಮಿತ ಕಾರ್ಯಗಳಿಲ್ಲದೇ ಕಳೆಯುತ್ತಿದ್ದರು. ಮುಂದೆ ಒಳ್ಳೆಯ ದಿನಗಳು ಬರಬಹುದೆಂದು ಕೆಲವರು, ಮಕ್ಕಳು ದೊಡ್ಡವರಾಗಿ ಎಲ್ಲಾದರೂ ಹೋಗಿ ನಾಲ್ಕು ಕಾಸು ದುಡಿದು ತಂದಾರು ನೋಡೋಣ ಎಂದು ಕೆಲವರು, ಬರುವ ವರ್ಷ ಗ್ರಹಗತಿ ಒಳ್ಳೆಯದಿದೆ ಅಲ್ಲಿಯವರೆಗೂ ಏನೂ ಮಾಡು\-ವಂತಿಲ್ಲ ಎಂದು ಕೆಲವರು, ಆ ಸ್ಥಾನದಲ್ಲಿದ್ದಾನಲ್ಲ ಅವನ ತಲೆ ಉರುಳದೇ ನಮ್ಮ ಉನ್ನತಿಯಾಗದೆಂದು ಅವನ ಸಾವನ್ನು ಹಾರೈಸುತ್ತ ಕೆಲವರು, ನಾಳೆ, ಬರುವ ತಿಂಗಳು, ಮುಂದಿನ ವರ್ಷ ನೋಡೋಣ ಎಂದು ಕೆಲವರು ಕಾಲಕ್ಷೇಪ ಮಾಡುತ್ತಿದ್ದರು. ಸಮುದ್ರದಲ್ಲಿ ಅಲೆಗಳು ನಿಂತ ಮೇಲೆ ಸ್ನಾನ ಮಾಡೋಣವೆಂದು ಕಾಯುವಂಥ ಮಹನೀಯರೂ ಇದ್ದರು.

ನೀವು ಬದುಕಿನಲ್ಲಿ ಯಶಸ್ವಿಗಳಾಗಲೇಬೇಕು. ಆದರೆ ಕೈಯಲ್ಲಿರುವ ಹಣ್ಣನ್ನು ದೂರಕ್ಕೆಸೆದು ಮರದ ಮೇಲಿನ ಹಣ್ಣಿಗಾಗಿ ಹಾತೊರೆಯುವ ಭ್ರಾಂತರಾಗುವಿರೇನು? ನಿಂತ ಸ್ಥಾನದಿಂದ ಮೇಲಕ್ಕೇರಲು ಈಗಿಂದೀಗಲೇ ಯತ್ನಿಸದೇ ಎಂದೋ ಒಂದು ದಿನ ಎಲ್ಲವೂ ಥಟ್ಟನೇ ಒಳ್ಳೆಯದಾಗಿ ಬಿಡುತ್ತದೆ ಎಂಬ ತಿರುಕನ ಕನಸಿನ ನಂಬಿಕೆಯನ್ನೇ ನೀವು ಆಶ್ರಯಿಸುವಿರೇನು?

ಕೆಲವರ ‘ವರ್ತಮಾನ’ದ ತುಂಬ ಭೂತಭವಿಷ್ಯಗಳ ಸುದ್ದಿಯೇ! ನೀವು ಎಲ್ಲಿದ್ದೀರಿ? ನಿಮ್ಮ ವರ್ತಮಾನದ ವಾರ್ತೆ ಏನು ಎಂಥದು? ಎಂಬುದನ್ನು ಅವಲೋಕಿಸಿ, ವರ್ತಮಾನದಲ್ಲಿ ಕಾರ್ಯನಿರತನಾದರೆ ಮಾತ್ರ ಸಾಮಾನ್ಯನು ಸಮರ್ಥನಾಗಬಲ್ಲ.


\section*{ಕ್ರೂರಿ ಅಕ್ರೂರನಾದ}

\addsectiontoTOC{ಕ್ರೂರಿ ಅಕ್ರೂರನಾದ}

ದುಷ್ಟರ ಸಹವಾಸದಲ್ಲಿ ಸಿಲುಕಿ ಕೊಲೆ, ದರೋಡೆ, ಕ್ರೌರ್ಯ, ಹಿಂಸೆಗಳಲ್ಲಿ ಪರಿಣತಿಯನ್ನು ಪಡೆದು ಕುಖ್ಯಾತರಾದ ಬಾಲ ಅಪರಾಧಿಗಳನ್ನು ಸುಧಾರಣೆ ಮಾಡಲು ಪಣತೊಟ್ಟ ಅಮೇರಿಕದ ಫಾದರ್ ಫ್ಲೆನಾಗನ್ ಕೆಥೊಲಿಕ್ ಸಂಪ್ರದಾಯದಲ್ಲಿ ತರಬೇತಿ ಪಡೆದು ಬಂದ ಒಬ್ಬ ಪಾದ್ರಿ. ಅವರು ನಿರ್ಮಿಸಿದ ‘ಮಕ್ಕಳ ಪಟ್ಟಣ’ದಲ್ಲಿ ನೀಗ್ರೊಗಳನ್ನೂ ಒಳಗೊಂಡು ಎಲ್ಲ ಜಾತಿಮತಗಳ ಅನಾಥ ಬಾಲಕರು ಕೂಡ ಇದ್ದರು. ಮಕ್ಕಳನ್ನು ಉನ್ನತ ಮಟ್ಟದ ಚಾರಿತ್ರ್ಯವಂತರನ್ನಾಗಿ ಪರಿವರ್ತಿಸಲು ಅವರು ಪಟ್ಟ ಶ್ರಮ, ತೋರಿಸಿದ ತಾಳ್ಮೆ ಅಪೂರ್ವವಾದವು. ನಾನಾ ರೀತಿಯ ಕುಕೃತ್ಯ ಮತ್ತು ಅಪರಾಧಗಳಿಗಾಗಿ ಪೋಲೀಸರಿಂದ ಬಂಧನಕ್ಕೊಳಗಾದ ಮಕ್ಕಳನ್ನು ಮಕ್ಕಳ ಪಟ್ಟಣಕ್ಕೆ ಕರೆ\-ತರು\-ತ್ತಿದ್ದರು. ಅವರಲ್ಲಿ ಹೆಪ್ಪುಗಟ್ಟಿದ ದುಷ್ಟತನವನ್ನು ಲೆಕ್ಕಿಸದೆ ಪ್ರತಿಯೊಬ್ಬ ಬಾಲಕನೂ ಸ್ವಭಾವತಃ ಒಳ್ಳೆಯವನು ಎಂಬ ವಿಶ್ವಾಸವನ್ನಿಟ್ಟು ಪ್ರೀತಿಯ ಮಳೆಗರೆದು ಅವರು ಕೊಟ್ಟ ಅಪಾರ ಕಿರುಕುಳಗಳನ್ನೂ, ತೊಂದರೆಗಳನ್ನೂ ಸಹಿಸಿ, ಅವರ ಒಳಿತಿಗಾಗಿ ಹೃತ್ಪೂರ್ವಕ ಪ್ರಾರ್ಥನೆ ಸಲ್ಲಿಸಿ ದುಡಿದು ಮಕ್ಕಳ ಬದುಕನ್ನು ಬೆಳಗುವುದರಲ್ಲಿ ಯಶಸ್ವಿಗಳಾದ, ಸರ್ವಜನಾದರಣೀಯರಾದ ಫ್ಲೆನಾಗನ್​; ಪ್ರೀತಿ, ತಾಳ್ಮೆ ಮತ್ತು ಸೇವೆಗಳ ಜ್ವಲಂತ ಮೂರ್ತಿ.

ಮಕ್ಕಳು ತಮ್ಮ ಬದುಕನ್ನು ತಿದ್ದಿಕೊಳ್ಳಬೇಕಾದರೆ ಟೀಕೆ, ವಿಮರ್ಶೆಗಳಿಗಿಂತಲೂ ಅವರೆದುರಿಗೆ ಮಾದರಿಯಾಗಬಲ್ಲ ಆದರ್ಶ ನಡತೆಯ ವ್ಯಕ್ತಿಗಳಿರುವುದು ಹೆಚ್ಚು ಆವಶ್ಯಕ ಎಂಬುದು ಫ್ಲೆನಾಗನ್​ರ ನುಡಿ.

ಕೊಲೆಗಡುಕನೂ ಕ್ರೂರಿಯೂ ಆದ ಹುಡುಗನನ್ನು ಫ್ಲೆನಾಗನ್ ಪರಿವರ್ತಿಸಿದ ಕತೆಯ ಹಿನ್ನೆಲೆಯಲ್ಲಿ ಅಡಗಿದ ರಹಸ್ಯ ಏನೆಂಬುದನ್ನು ಸಾರುವ ಈ ಘಟನೆಯನ್ನು ‘ಮಕ್ಕಳ ಪಟ್ಟಣದ ಫಾದರ್ ಫ್ಲೆನಾಗನ್​’ ಎಂಬ ಆಂಗ್ಲಭಾಷೆಯ ಪುಟ್ಟ ಪುಸ್ತಕದಿಂದ ಸಂಗ್ರಹಿಸಲಾಗಿದೆ–

ಎಡ್ಡಿ ಎಂಬ ಹುಡುಗ ನಾಲ್ಕನೇ ವಯಸ್ಸಿನಲ್ಲಿ ತನ್ನ ತಂದೆತಾಯಿಗಳನ್ನು ಕಳೆದುಕೊಂಡು ಅನಾಥನಾದ. ಎಂಟನೇ ವಯಸ್ಸಿಗೇ ಆತ ತುಂಟರ ತಂಡದ ಮುಖ್ಯಸ್ಥನಾಗುವ ಸಾಮರ್ಥ್ಯ ದಕ್ಷತೆಗಳನ್ನು ಪಡೆದಿದ್ದ. ಆಶ್ಚರ್ಯದ ಸಂಗತಿ ಎಂದರೆ ಅವನ ತಂಡದಲ್ಲಿದ್ದ ತುಂಟರಲ್ಲಿ ಹೆಚ್ಚಿ\-ನವರು ಆತನಿಗಿಂತ ವಯಸ್ಸಿನಲ್ಲಿ ಹಿರಿಯರು. ಹದಿಹರೆಯದ ಮಕ್ಕಳೂ ಅವನನ್ನು ಗುರುವಾಗಿ ಸ್ವೀಕರಿಸಿದ್ದರು. ಎಡ್ಡಿ ಕೊಲೆಗಳನ್ನು ನಡೆಸಿದ್ದ. ತನ್ನ ಪಂಥ ಪರಾಕ್ರಮ ಮೆರೆಸಲು ಏಕಾಂಗಿ ಯಾಗಿ ಬ್ಯಾಂಕ್ ಒಂದನ್ನು ನುಗ್ಗಿ ಸಾವಿರಾರು ಡಾಲರ್ ಕದ್ದಿದ್ದ. ಒಂದು ಪಿಸ್ತೂಲನ್ನು ಸಂಪಾದಿಸಿ ಎಷ್ಟೋ ತಿಂಡಿಯಂಗಡಿಗಳನ್ನು ಲೂಟಿ ಮಾಡಿದ್ದ. ಆದರೆ ಒಮ್ಮೆ ಆತ ಒಬ್ಬ ಮುದುಕಿಯನ್ನು ಕೊಲ್ಲಲು ಪಿಸ್ತೂಲು ಸಜ್ಜುಗೊಳಿಸುತ್ತಿರುವಷ್ಟರಲ್ಲಿ ಪೋಲೀಸರಿಂದ ಬಂಧಿತನಾದ.

‘ಮಕ್ಕಳ ಪಟ್ಟಣ’ಕ್ಕೆ ಬಂದಾಗ ಆತನಿಗೆ ಪೋಲೀಸರ ಭೀತಿ ಇರಲಿಲ್ಲ. ಮನಬಂದಂತೆ ವರ್ತಿಸತೊಡಗಿದ. ತನ್ನ ದರ್ಪವನ್ನು ಮೆರೆಸಿದ. ಆತ ಸಿಕ್ಕಿದ ವಸ್ತುಗಳನ್ನು ಚೆಲ್ಲಾಪಿಲ್ಲಿಯಾಗಿ ಎಸೆಯುತ್ತಿದ್ದ. ಕೊಳಕು ಮಾತುಗಳಿಂದ ಕ್ಲಾಸಿನಲ್ಲಿದ್ದ ಇತರ ಹುಡುಗರನ್ನು ಬೈಯುತ್ತಿದ್ದ. ಎಲ್ಲರ ಗೌರವವನ್ನೂ ಪಡೆದಿದ್ದ ಫ್ಲೆನಾಗನ್ ಅವರನ್ನು ಅವಾಚ್ಯ ಶಬ್ದಗಳಿಂದ ನಿಂದಿಸುತ್ತಿದ್ದ. ಆಟಗಳಾಗಲಿ, ಬ್ಯಾಂಡ್ ಬಾರಿಸುವುದಾಗಲಿ, ಹೊಲದಲ್ಲಿ ದುಡಿಯುವುದಾಗಲಿ ತನಗೆ ‘ಬೋರ್​’ ಬೇಸರ ಎನ್ನುತ್ತಾ ಎಲ್ಲವನ್ನು ತಿರಸ್ಕರಿಸುತ್ತಿದ್ದ. ಆತ ಪ್ರಾರ್ಥನೆಯ ಸಮಯದಲ್ಲಿ ಬೆಕ್ಕು ಕೂಗಿದಂತೆ ಸದ್ದು ಮಾಡುತ್ತಿದ್ದ. ಒಬ್ಬ ವಿದ್ಯಾರ್ಥಿ ಗಂಟೆಗಟ್ಟಲೆ ಕುಳಿತು ಮಾಡಿದ ಯಾವುದೋ ಒಂದು ಕೈಗೆಲಸವನ್ನು ಕ್ಷಣಾರ್ಧದಲ್ಲಿ ಹಾಳು ಮಾಡುತ್ತಿದ್ದ. ಬಂದ ಆರು ತಿಂಗಳಲ್ಲಿ ಯಾರೂ ಆತನ ಮುಖದಲ್ಲಿ ಒಂದು ನಗುವನ್ನಾಗಲೀ, ಕಣ್ಣೀರ ಹನಿಯನ್ನಾಗಲೀ ಕಂಡವರಿಲ್ಲ. ‘ಕಾಲಿನಿಂದ ತಲೆಯ ತನಕ ದ್ವೇಷ ತುಂಬಿಕೊಂಡ ಜೀವ’ ಎಂದೇ ಎಲ್ಲ ಜನ ಹೇಳುತ್ತಿದ್ದರು.

ಮಕ್ಕಳನ್ನು ನೋಡಿಕೊಳ್ಳುವ ಅಧ್ಯಾಪಕರಿಗೆ ಸಾಕುಬೇಕಾಯಿತು. ಅವರು ಫ್ಲೆನಾಗನ್ ಅವರಿಗೆ ಪತ್ರ ಬರೆದರು–

‘ಪ್ರಿಯ ಫಾದರ್ ಫ್ಲೆನಾಗನ್,

\newpage

ನೀವು ಕೆಟ್ಟ ಹುಡುಗನೆಂಬುವನು ಈ ಜಗತ್ತಿನಲ್ಲಿ ಇಲ್ಲ ಎಂದು ಹೇಳುವುದನ್ನು ಕೇಳಿದ್ದೇನೆ. ಈ ಹುಡುಗನನ್ನು ಏನೆಂದು ಕರೆಯುವಿರಿ! ದಯವಿಟ್ಟು ತಿಳಿಸಬಲ್ಲಿರಾ?’–ಎಂದು.

ಒಂದು ರಾತ್ರಿ ಎಡ್ಡಿ ನಿದ್ರೆಯಲ್ಲಿ ನರಳಾಡುತ್ತಿದ್ದ. ಫ್ಲೆನಾಗನ್ ಅವನ ಮುಖವನ್ನು ನೋಡಿಯೇ ಆತನಿಗೆ ವಿಪರೀತ ಜ್ವರ ಬಂದಿದೆ ಎಂಬುದನ್ನು ತಿಳಿದುಕೊಂಡರು. ಆತನ ತಡೆಯಲಾರದ ತುಂಟತನದಿಂದ ಅವರು ನೊಂದುಕೊಂಡಿದ್ದರೂ, ಈಗ ಎಲ್ಲವನ್ನೂ ಮರೆತು ಇತರ ಎಲ್ಲ ವಿದ್ಯಾರ್ಥಿ\-ಗಳಿಗಿಂತಲೂ ಹೆಚ್ಚು ಪ್ರೀತಿ ಕರುಣೆಗಳಿಂದಲೇ ಅವನನ್ನು ನೋಡಿಕೊಂಡರು.

ಅವನು ಸೌಖ್ಯ ಹೊಂದಿದಾಗ ಅವನನ್ನು ಫ್ಲೆನಾಗನ್ ಮತ್ತು ಇತರ ಅಧ್ಯಾಪಕರೂ, ಮಕ್ಕಳೂ ವಿಶೇಷ ಗಮನವಿತ್ತು ವಿಶ್ವಾಸದಿಂದ ನೋಡಿಕೊಂಡರು. ಹಿರಿಯ ವಿದ್ಯಾರ್ಥಿಗಳು ಅವನನ್ನು ಸಿನಿಮಾ ನೋಡಲು ಕರೆದುಕೊಂಡು ಹೋದರು. ಹೆಚ್ಚಿನ ವೇಳೆ ಒಂದೇ ಚಿತ್ರವನ್ನು ಆತ ಎರಡೆರಡು ಬಾರಿ ನೋಡುತ್ತಿದ್ದ. ಊಟ ತಿಂಡಿಗಳಲ್ಲಿ ಎಲ್ಲರಿಗಿಂತಲೂ ಮುಂದಿನ ಸ್ಥಾನ ಪಡೆದಿದ್ದ. ಯಾವ ಕೊರತೆಯೂ ಆಗದಂತೆ ಎಲ್ಲರೂ ಅವನನ್ನು ನೋಡಿಕೊಂಡರು. ಆದರೂ ಅವನ ಮುಖದಲ್ಲಿ ಒಂದು ನಗುವೂ ಮಿನುಗಲಿಲ್ಲ.

ಒಂದು ದಿನ ಎಡ್ಡಿ ನೇರವಾಗಿ ಫ್ಲೆನಾಗನ್ ಅವರ ಆಫೀಸನ್ನು ಪ್ರವೇಶಿಸಿ ಹೀಗೆಂದ: ‘ನೀವು ನನ್ನನ್ನು ಒಳ್ಳೆಯವನನ್ನಾಗಿ ಸರಿಪಡಿಸಲು ಯತ್ನಿಸುತ್ತಿದ್ದೀರಿ. ಆದರೆ ಅದೆಷ್ಟು ಸಫಲ? ಈಗ ತಾನೆ ನಾನು ಸಿಸ್ಟರಿಗೊಂದು ಒದೆ ಕೊಟ್ಟು ಬಂದೆ. ನೀವು ಏನು ಹೇಳುತ್ತೀರಿ?’

‘ಈಗಲೂ ಹೇಳುತ್ತೇನೆ, ನೀನು ಒಳ್ಳೆಯ ಹುಡುಗನೇ’ ಫ್ಲೆನಾಗನ್ ದೃಢವಾಗಿ ಉತ್ತರಿಸಿದರು.

‘ಸರಿ, ನಾನು ನಿಮಗೀಗ ಹೇಳಿದ್ದೇನು? ನೀವು ಅದೇ ಸುಳ್ಳನ್ನೇ ಮತ್ತೆ ಹೇಳುತ್ತಿದ್ದೀರಿ. ನಿಮಗೆ ಗೊತ್ತಿದೆ ನಾನೊಬ್ಬ ಒಳ್ಳೆಯ ಹುಡುಗನಲ್ಲ ಎಂಬುದು, ಆದರೂ ಒಳ್ಳೆಯವನೆಂದು ಸುಳ್ಳು ಹೇಳುತ್ತಿದ್ದೀರಿ. ಆ ಸುಳ್ಳನ್ನೇ ಮತ್ತೆಮತ್ತೆ ಹೇಳುವುದರಿಂದ ನೀವು ಮಹಾಸುಳ್ಳುಗಾರರೆಂಬುದು ಸ್ಪಷ್ಟವಾಗುತ್ತದೆ. ಅಲ್ಲವೇ?’

ಫ್ಲೆನಾಗನ್ ಒಂದು ಕ್ಷಣ ಯೋಚಿಸಿದರು. ಹುಡುಗನ ಬದುಕಿನಲ್ಲಿ ಒಂದು ತಿರುವು ಉಂಟಾ\-ಗುವ ಸಂದರ್ಭ ಸನ್ನಿಹಿತವಾಗಿದೆ ಎಂದುಕೊಂಡು ಎಡ್ಡಿಯನ್ನು ಪ್ರಶ್ನಿಸಿದರು, ‘ಒಳ್ಳೆಯ ಹುಡುಗ ಹೇಗಿರುತ್ತಾನಪ್ಪಾ? ಹಿರಿಯರಿಗೆ ವಿಧೇಯನಾಗಿರುತ್ತಾನಲ್ಲವೇ ಎಡ್ಡಿ?’

ಎಡ್ಡಿ ‘ಹೌದು’ ಎಂಬಂತೆ ತಲೆ ಆಡಿಸಿದ.

‘ಅಧ್ಯಾಪಕರು ಹೇಳಿದಂತೆ ನಡೆದುಕೊಳ್ಳುತ್ತಾನೆ ತಾನೆ?’

‘ಹೌದು’ ಎಂದ ಎಡ್ಡಿ.

‘ಹಾಗಾದರೆ ಅದನ್ನೇ ನೀನು ಮಾಡುತ್ತಿದ್ದುದು. ಆದರೆ, ಎಡ್ಡಿ! ಇಲ್ಲಿಯವರೆಗೆ ನಿನಗೆ ಸಿಕ್ಕಿದವರು ಒಳ್ಳೆಯ ಅಧ್ಯಾಪಕರಲ್ಲ ಅಷ್ಟೆ. ಬೀದಿ ಅಲೆಯುವ ತುಂಟರು ನಿನಗೆ ಇಷ್ಟರವರೆಗೆ ಮಾರ್ಗದರ್ಶನ ಮಾಡಿದವರು. ನೀನು ಅವರಿಗೆ ವಿಧೇಯನಾಗಿದ್ದೆ. ಅವರು ನಿನಗೆ ಕೆಟ್ಟದ್ದನ್ನು ಹೇಳಿಕೊಟ್ಟರು. ಅದನ್ನೇ ನೀನು ಅನುಸರಿಸುತ್ತ ಬಂದು ನಿನ್ನನ್ನು ನೀನು ಕೆಟ್ಟವನೇ ಎಂದು ತಿಳಿದುಕೊಂಡೆ. ಈಗ ನೀನು ಇಲ್ಲಿ ಒಳ್ಳೆಯ ಅಧ್ಯಾಪಕರಿಗೆ ವಿಧೇಯನಾದರೆ ಅತ್ಯುತ್ತಮ ವ್ಯಕ್ತಿ ಯಾಗುವೆ’ ಎಂದರು ಫ್ಲೆನಾಗನ್.

ಈ ಮಾತು ಅವನ ಮನಸ್ಸಿನ ಆಳವನ್ನು ಕಲಕಿತು. ಕ್ಷಣಕಾಲ ಅವನು ಯೋಚಿಸುತ್ತ ಮೌನ\-ವಾಗಿಯೇ ಇದ್ದ. ಫ್ಲೆನಾಗನ್ ಅವರ ಮಾತು ಹೌದೆನಿಸಿತು, ಆ ಮಾತು ಸ್ವಭಾವತಃ ತಾನೊಬ್ಬ ದುಷ್ಟ ಎನ್ನುವ ರೂಢಮೂಲವಾದ ದುರ್ಭಾವನೆಯನ್ನು ಹೊಡೆದೋಡಿಸಿತು. ಫ್ಲೆನಾಗನ್ ಅವರ ಮೇಜನ್ನು ಬಳಸಿ ಎಡ್ಡಿ ನೇರವಾಗಿ ಅವರನ್ನು ಸಮೀಪಿಸಿದ. ಮಗುವನ್ನು ಎತ್ತಿಕೊಳ್ಳಲು ಕೈ ಚಾಚುವಂತೆ ಅವರು ಅವನನ್ನು ಸ್ವಾಗತಿಸಿ ತಬ್ಬಿಕೊಂಡರು. ಹುಡುಗನ ಕಣ್ಣುಗಳಿಂದ ನೀರು ಹರಿದು ಕಪೋಲಗಳು ಒದ್ದೆಯಾದವು.

ಹತ್ತು ವರ್ಷಗಳ ನಂತರ ಎಡ್ಡಿ ಪದವಿ ಪರೀಕ್ಷೆಯಲ್ಲಿ ಉನ್ನತ ಸ್ಥಾನವನ್ನು ಗಳಿಸಿದ. ಸೈನ್ಯಕ್ಕೆ ಸೇರಿ ಯುದ್ಧದಲ್ಲಿ ಭಾಗವಹಿಸಿದ. ಹಲವಾರು ಫಲಕ ಪ್ರಶಸ್ತಿಗಳನ್ನೂ ಪಡೆದ. ತನ್ನ ಪರಿಚಿತ ವ್ಯಕ್ತಿಗಳ, ಗೆಳೆಯರ, ಸಹೋದ್ಯೋಗಿಗಳ ವಿಶ್ವಾಸ ಪ್ರೀತಿಗಳನ್ನು ಗಳಿಸಿದ. ನಂಬಿಕೆಗೆ ಅರ್ಹನಾದ ಶೀಲವಂತ ವ್ಯಕ್ತಿ ಎನಿಸಿಕೊಂಡ.

ಸ್ವಭಾವತಃ ಖಂಡಿತವಾಗಿಯೂ ನೀನು ಒಳ್ಳೆಯವನೇ ಎಂಬ ದೃಢವಿಶ್ವಾಸದ ನುಡಿ ಆತನ ಮನದಂತರಾಳದಲ್ಲಿ ಬಲವಾಗಿ ಬೇರುಬಿಟ್ಟಿದ್ದ ‘ಕೆಟ್ಟವ’ ಎಂಬ ಭಾವನೆಯನ್ನು ಹೊರದೂಡಿ\-ತಲ್ಲವೇ?

ಫ್ಲೆನಾಗನ್ ಅವರು ಮಗುವಿನ ದಿವ್ಯತ್ವದಲ್ಲಿ ದೃಢವಾದ ಶ್ರದ್ಧೆಯನ್ನಿಟ್ಟಿದ್ದರು. ಅಡ್ಡದಾರಿಯನ್ನು ಹಿಡಿದ, ಸಂಶಯಗ್ರಸ್ತನಾದ ಹುಡುಗನೂ, ಕ್ರಮೇಣ ಹೇಗೆ ತನ್ನ ಒಳ್ಳೆಯತನದಲ್ಲಿ ವಿಶ್ವಾಸವಿಟ್ಟು ಮೇಲಕ್ಕೇರಿದ ಎಂಬುದನ್ನು ನೋಡಿ. ಜೊತೆ ಜೊತೆಗೆ ಫ್ಲೆನಾಗನ್ ಅವರ ದೃಢವಿಶ್ವಾಸ ಮತ್ತು ತಾಳ್ಮೆಯಿಂದ ಕೂಡಿದ ಪ್ರಯತ್ನಗಳೂ ಮರೆಯತಕ್ಕ ಸಂಗತಿಗಳಲ್ಲ.


\section*{ಬೆಳಕನ್ನು ತನ್ನಿ}

\addsectiontoTOC{ಬೆಳಕನ್ನು ತನ್ನಿ}

“ಮನುಷ್ಯರಲ್ಲಿರುವ ದೌರ್ಬಲ್ಯಕ್ಕೆ ದೌರ್ಬಲ್ಯವನ್ನು ಕುರಿತು ಚಿಂತಿಸುವುದೇ ಔಷಧವಲ್ಲ. ಶಕ್ತಿಯನ್ನು ಕುರಿತು ಚಿಂತಿಸುವುದೇ ಪರಿಹಾರೋಪಾಯ. ಮನುಷ್ಯರಿಗೆ ಅವರಲ್ಲಿ ಆಗಲೇ ಅಡಗಿರುವ ಶಕ್ತಿಯನ್ನು ಕುರಿತು ಬೋಧಿಸಿ. ಮನುಷ್ಯರನ್ನು ಪಾಪಿಗಳು, ನೀಚರು ಎಂದು ಹೇಳುವುದಕ್ಕೆ ಬದಲಾಗಿ ವೇದಾಂತವು ‘ನೀನು ಆಗಲೇ ಪರಿಶುದ್ಧ ಪರಿಪೂರ್ಣ, ಯಾವುದನ್ನು ಪಾಪ ದೋಷ ದೌರ್ಬಲ್ಯವೆಂದು ಕರೆಯುವಿಯೋ ಅದು ನಿನಗೆ ನಿಜವಾಗಿ ಸೇರಿದ್ದಲ್ಲ’ ಎಂದು ಸಾರುತ್ತದೆ. ಇತರರಿಗಾಗಲೀ, ನಿಮಗೆ ನೀವೇ ಆಗಲಿ ‘ದರಿದ್ರ, ದುರ್ಬಲ’ ಎಂದು ಹೇಳಬೇಡಿ” ಎಂದರು ಸ್ವಾಮಿ ವಿವೇಕಾನಂದರು.

“ಶತಮಾನಗಳಿಂದ ಕೋಣೆಯೊಂದರಲ್ಲಿ ಕತ್ತಲೆ ಕವಿದುಕೊಂಡಿದೆ. ನೀವು ಅಲ್ಲಿಗೆ ಹೋಗಿ ‘ಅಯ್ಯೋ! ಕತ್ತಲೆ ಕತ್ತಲೆ’ ಎಂದು ಅಳಲು ಪ್ರಾರಂಭಿಸಿದರೇನು ಪ್ರಯೋಜನ? ಅಥವಾ ಕತ್ತಲೆಯನ್ನು ತೀವ್ರವಾಗಿ ಬೈದು ಭಂಗಿಸಿದರೆ ಏನು ಮಾಡಿದ ಹಾಗಾಯಿತು? ಬೆಳಕನ್ನು ತನ್ನಿ. ಕತ್ತಲೆ ಕೂಡಲೇ ಮಾಯವಾಗುವುದು.” ಮನುಷ್ಯರನ್ನು ಉನ್ನತ ಮಟ್ಟಕ್ಕೆಳೆಯುವ ಅಥವಾ ಪರಿವರ್ತಿ ಸುವ ಉಪಾಯ ಇದು. ಜನರಿಗೆ ಉನ್ನತವಾದುದು ಯಾವುದೆಂಬುದನ್ನು ಸೂಚಿಸಿ. ಮೊದಲು ಮನುಷ್ಯರಲ್ಲಿ ವಿಶ್ವಾಸವಿಡಿ. ಮನುಷ್ಯ ನೀಚತೆಯ ಆಳಕ್ಕೆ ಮುಳುಗಿದವನು ಎಂಬ ನಂಬಿಕೆಯಿಂದೇಕೆ ಸುಧಾರಣಾಕಾರ್ಯವನ್ನು ಪ್ರಾರಂಭಿಸಬೇಕು? “ಅತಿ ದುಷ್ಟತನ ವ್ಯಕ್ತಗೊಳಿಸು\-ವವನಲ್ಲೂ ನಾನು ನನ್ನ ನಂಬಿಕೆಯನ್ನು ಕಳೆದುಕೊಂಡವನಲ್ಲ. ಮೊದಲು ನನ್ನ ನಂಬಿಕೆಯು ಆಶಾ ಜನಕವಾಗಿರದಿದ್ದರೂ ಕೊನೆಯಲ್ಲಿ ನಂಬಿಕೆಗೆ ಅನುಗುಣವಾಗಿಯೇ ಯಶಸ್ಸು ದೊರಕಿರುವುದನ್ನು ಕಂಡಿದ್ದೇನೆ” ಎಂದು ಸ್ವಾಮೀಜಿ ಹೇಳಿದರು. “ಮಾನವನ ಸಾಮರ್ಥ್ಯದಲ್ಲಿ ವಿಶ್ವಾಸವಿಡಿ. ಪ್ರಾಜ್ಞನಲ್ಲೂ, ಅಜ್ಞನಲ್ಲೂ–ಒಬ್ಬನು ದೇವತೆಯಾಗಿ ಕಂಡರೂ, ಇನ್ನೊಬ್ಬನು ದೆವ್ವವಾಗಿ ಕಂಡರೂ, ಅವನ ಆಂತರ್ಯದಲ್ಲಿ ಅದಾಗಲೇ ಅಡಗಿರುವ ಪರಿಪೂರ್ಣತೆಯಲ್ಲಿ ನಂಬಿಕೆಯನ್ನಿಡಿ. ಮೊದಲು ಆತನಲ್ಲಿ ವಿಶ್ವಾಸವಿಡಿ. ಆ ಬಳಿಕ ಅವನಲ್ಲಿ ದೋಷ ಕಂಡುಬಂದರೆ, ಅವನು ತಪ್ಪು ದಾರಿಯನ್ನು ಹಿಡಿದರೆ, ಅವನು ಅಸಾಧುವೂ, ನೀಚವೂ ಆದ ಸಿದ್ಧಾಂತಗಳನ್ನು ಅನುಸರಿಸಿದರೆ ಅವೆಲ್ಲಕ್ಕೂ ಕಾರಣ ಅವನ ನೈಜಸ್ವಭಾವ ಅಲ್ಲವೆಂದೂ, ಉತ್ತಮ ಆದರ್ಶ ಅವನ ಮುಂದಿಲ್ಲ ದಿರುವುದೇ ಆಗಿದೆ ಎಂದೂ ತಿಳಿಯಿರಿ. ಹೌದು, ಸನ್ಮಾರ್ಗವನ್ನು ಕಾಣಲಾರದ್ದರಿಂದಲ್ಲವೇ ಅವನು ದುರ್ಮಾರ್ಗವನ್ನು ಅವಲಂಬಿಸಿದ್ದು? ತಪ್ಪನ್ನು ತಿದ್ದುವ ರೀತಿ ಇದೊಂದೇ: ದುರ್ಮಾರ್ಗದಲ್ಲಿ ನಡೆಯುತ್ತಿರುವವನಿಗೆ ಸನ್ಮಾರ್ಗವನ್ನು ತೋರಿಸಿ. ಅಲ್ಲಿಗೆ ನಿಮ್ಮ ಕೆಲಸ ಮುಗಿಯಿತು. ಅವನು ಎರಡು ಮಾರ್ಗಗಳನ್ನು ಹೋಲಿಸಿ ನೋಡಲಿ. ತನ್ನಲ್ಲಿ ಮೊದಲು ಇದ್ದ ಅಸತ್ಯವನ್ನೂ, ನೀವು ನೀಡಿದ ಸತ್ಯವನ್ನೂ ಸರಿದೂಗಲಿ. ಆಗ ನೀವು ಅವನಿಗೆ ತಿಳಿಸಿರುವುದು ಸತ್ಯವಾಗಿದ್ದರೆ ಅಸತ್ಯವು ಮಾಯವಾಗಲೇ ಬೇಕು. ಬೆಳಕು ಕತ್ತಲನ್ನು ಓಡಿಸಲೇಬೇಕು. ಸತ್ಯವು ಶ್ರೇಯಸ್ಸನ್ನೂ, ಬಲವನ್ನೂ, ವಿಜಯವನ್ನೂ ಎಂದಾದರೂ ವ್ಯಕ್ತಪಡಿಸಲೇಬೇಕು” ಎಂದು ಅವರು ಹೇಳಿದರು.


\section*{ದಡ್ಡನು ಧೀರನಾದ}

\addsectiontoTOC{ದಡ್ಡನು ಧೀರನಾದ}

ನಮ್ಮ ಆಶ್ರಮ ನಡೆಸುವ ಒಂದು ಹಾಸ್ಟೆಲಿನಲ್ಲಿ ಒಬ್ಬ ವಿದ್ಯಾರ್ಥಿಯಲ್ಲಾದ ಆಮೂಲಾಗ್ರ\break ಬದಲಾವಣೆಯನ್ನು ಕುರಿತು ಸೋದರ ಸ್ವಾಮೀಜಿ ಒಮ್ಮೆ ಹೇಳಿದ್ದರು. ಅವರ ಮಾತಿನಲ್ಲೇ ಆ ಘಟನೆಯನ್ನು ಇಲ್ಲಿ ನೀಡುತ್ತಿದ್ದೇನೆ–

“ಒಂದು ದಿನ ಹಾಸ್ಟೆಲಿನ ಎಲ್ಲ ಹುಡುಗರನ್ನೂ ಹತ್ತಿರ ಕುಳ್ಳಿರಿಸಿಕೊಂಡು ಮಾತನಾಡುತ್ತಿದ್ದೆ. ನಮ್ಮ ದೇಶದಲ್ಲಿ ಎಷ್ಟು ಮಂದಿ ಹುಡುಗರಿಗೆ ಒಳ್ಳೆಯ ರೀತಿಯ ಅಧ್ಯಯನ ಮಾಡಲು ಅವಕಾಶವಿದೆ? ನೋಡಿ, ಮನೆಯವರಿಗೆ ಓದು ಬರಹ ಬಾರದು. ದೂರದಲ್ಲಿರುವ ಹಳ್ಳಿಯ ಶಾಲೆ. ಅಲ್ಲಿ ಸೌಕರ್ಯ ಸಾಲದು. ತಪ್ಪದೇ ಶಾಲೆಗೆ ಹೋಗಲು ಅನುಕೂಲವಿಲ್ಲ. ಶಾಲೆಗೆ ಹೋದರೆ ಒಬ್ಬರೋ, ಇಬ್ಬರೋ ಅಧ್ಯಾಪಕರು ಇನ್ನೂರು ಮಕ್ಕಳಿಗೆ ಪಾಠ ಹೇಳಬೇಕು. ನಾಲ್ಕೈದು\break ವರ್ಷಗಳ ಕಾಲ ಶಾಲೆಗೆ ಹೋಗಿದ್ದರೂ ಪೇಟೆ ಪಟ್ಟಣಗಳಲ್ಲಿ ಓದಿದ ಹುಡುಗರನ್ನು ಕಂಡಾಗ ಹಳ್ಳಿಯ ಹುಡುಗರಿಗೆ ತಮ್ಮ ಅಭಿವೃದ್ಧಿಯ ಬಗೆಗೆ ಸಂಕೋಚವೆನಿಸಬಹುದು, ಭಯವೂ ಆಗಬಹುದು. ನಿಮಗಾದರೆ ಚೆನ್ನಾಗಿ ಓದಿ ಮುನ್ನಡೆಯಲು ಎಷ್ಟೊಂದು ಅವಕಾಶಗಳಿವೆ. ಒಳ್ಳೆಯ ಪುಸ್ತಕಾಲಯವಿದೆ. ಹಿರಿಯ ವಿದ್ಯಾರ್ಥಿಗಳಿದ್ದಾರೆ. ಓದುವಾಗ ನಿಮಗೆ ಯಾವ ತೊಂದರೆಯಾಗದಂತೆ ನೋಡಿ ಕೊಳ್ಳಲಾಗುತ್ತದೆ. ಸಣ್ಣಪುಟ್ಟ ನಿಯಮಿತವಾದ ಕೆಲಸಗಳನ್ನು ಬಿಟ್ಟರೆ ಬೇರಾವ ತಾಪತ್ರಯವಿಲ್ಲ, ಬೇರಾವ ಜವಾಬ್ದಾರಿ, ಚಿಂತೆಗಳಿಲ್ಲ. ನಿಮಗೆ ತೊಂದರೆಗಳೇನಾದರೂ ಇದ್ದರೆ ನಾವು ಅದನ್ನು ದೂರ ಮಾಡಲು ಯತ್ನಿಸುತ್ತೇವೆ. ಆದರೂ ದೇವರು ಕೊಟ್ಟ ಈ ಅವಕಾಶವನ್ನು ನೀವು ಸರಿಯಾಗಿ ಉಪಯೋಗಿಸಿಕೊಳ್ಳುತ್ತಿದ್ದೀರಾ? ಯೋಚಿಸಿ! ನಮ್ಮ ಪಾಲಿಗೆ ಬಂದ ಅವಕಾಶವನ್ನು ಸರಿಯಾಗಿ ಉಪಯೋಗಿಸಿಕೊಳ್ಳದಿದ್ದರೆ, ನಾವು ನಿಂತ ಜಾಗದಿಂದ ಮೇಲೇರಲು ಬೇರಾವ ದಾರಿಯೂ ಇಲ್ಲ. ಈ ಬಗ್ಗೆ ಮುಂದೆ ಪರಿತಪಿಸಬೇಕಾಗುತ್ತದೆ” ಎಂದೆಲ್ಲ ಹೇಳಿದೆ. ಹುಡುಗರೆಲ್ಲರೂ ಮೌನವಾಗಿ ಆಲಿಸುತ್ತಿದ್ದರು.

ರಾತ್ರಿ ಒಂಬತ್ತರ ಹೊತ್ತಿಗೆ ಏಳನೇ ತರಗತಿಯಲ್ಲಿ ಓದುತ್ತಿದ್ದ ಹುಡುಗ ನಿರಂಜನ ನನ್ನ ಕೋಣೆಯೊಳಗೆ ಬಂದು, ‘ಸ್ವಾಮೀಜಿ, ನಿಮ್ಮ ಹತ್ತಿರ ಒಂದು ವಿಚಾರ ಕೇಳಬೇಕೆಂದಿದ್ದೇನೆ. ಕೇಳಲೇ?’ ಎಂದ.

‘ಖಂಡಿತ ಕೇಳಬಹುದು’ ಎಂದೆ.

‘ದಯವಿಟ್ಟು ನನಗೊಬ್ಬನಿಗೇ ಸ್ವಲ್ಪ ಲೆಕ್ಕ ಹೇಳಿಕೊಡುತ್ತೀರಾ?’ಎಂದು ಬಹಳ ದೈನ್ಯದಿಂದ ಕೇಳಿದ.

‘ಆಗಲಿ ಹೇಳಿಕೊಡುತ್ತೇನೆ. ನಿತ್ಯವೂ ಅರ್ಧಗಂಟೆ ಅದಕ್ಕಾಗಿ ಮೀಸಲಿಡುತ್ತೇನೆ. ಏನೂ ಸಂಕೋಚಬೇಡ. ಊಟವಾದ ಮೇಲೆ ಬಾ’ ಎಂದೆ.

ಹಾಸ್ಟೆಲಿನಲ್ಲಿ ಚೆನ್ನಾಗಿ ಅಧ್ಯಯನ ಮಾಡದ, ‘ಸೋಮಾರಿ’, ‘ದಡ್ಡ’, ‘ಪರೀಕ್ಷೆಯಲ್ಲಿ ಕಾಪಿ, ಮೇಷ್ಟರಿಗೆ ಟೋಪಿ’ ಎಂಬೆಲ್ಲ ಬಿರುದುಗಳನ್ನು ಪಡೆದ ಹುಡುಗನವನು ಎಂಬುದು ನನಗೆ ತಿಳಿದ ವಿಷಯವೇ ಆಗಿತ್ತು.

ಸಮಯಕ್ಕೆ ಸರಿಯಾಗಿ ಅವನು ಪುಸ್ತಕ ಪೆನ್ಸಿಲುಗಳನ್ನು ತೆಗೆದುಕೊಂಡು ನನ್ನ ಕೋಣೆಯಲ್ಲಿ ಹಾಜರಾದ. ಅವನ ಗಣಿತಜ್ಞಾನದ ಮಟ್ಟ ೩ನೆಯ ತರಗತಿಯಷ್ಟರದು. ಅದಕ್ಕಿಂತ ಹೆಚ್ಚಿನದಲ್ಲ. ಇವನಿಗೆ ಗಣಿತ ಬೋಧನೆ ಮಾಡಿ ಪರೀಕ್ಷೆಯಲ್ಲಿ ಪಾಸುಮಾಡಿಸಲು ನನ್ನಿಂದ ಸಾಧ್ಯವಾಗುವುದೇ ಹೇಗೆ? ಎಂಬ ಯೋಚನೆ ಮನಸ್ಸಿನಲ್ಲಿ ಸುಳಿದು ಮಾಯವಾಯಿತು. ಗಣಿತ ಕಲಿಸಲು ಸಾಧ್ಯ ವಾಗಲಿ ಬಿಡಲಿ, ಆಗಲೇ ‘ಲೆಕ್ಕದಲ್ಲಿ ತಲೆ ಇಲ್ಲ’, ‘ಮನಸ್ಸು ಕೊಟ್ಟು ಕಲಿಯುವುದಿಲ್ಲ’, ‘ತಲೆ ಯಲ್ಲಿ ಹುಲ್ಲು, ಮಿದುಳಿಲ್ಲ’ ಮುಂತಾದ ವಾಕ್ಯಗಳಿಂದ ಆಘಾತವಾದ ಆತನ ಮನಸ್ಸಿಗೆ ಇನ್ನಿಷ್ಟು ತಿರಸ್ಕಾರದ ಮಾತುಗಳನ್ನಾಡಿ ನೋಯಿಸಲು ನಾನು ಸರ್ವಥಾ ಸಿದ್ಧನಿರಲಿಲ್ಲ. ಏಕೆಂದರೆ ನಾನು ಬುದ್ಧಿವಂತರೆನಿಸಿಕೊಂಡ ಹೆಚ್ಚಿನವರ ಅನಾವಶ್ಯಕ ಅಹಂಕಾರದಿಂದ ನೊಂದವನು. ಬುದ್ಧಿವಂತರು, ಸರಳರೂ, ನಿರಹಂಕಾರಿಗಳೂ, ನಿರಾಡಂಬರಿಗಳೂ ಆಗಿದ್ದರೆ ಚಿನ್ನ ಪರಿಮಳಿಸಿದಂತೆ ಎಂಬುದನ್ನು ಕಂಡವನು. ನಿಂದೆಯ ನುಡಿಯನ್ನು ಕೇಳಿ ನೊಂದುಕೊಂಡ ಆತನಿಗೆ ಹೇಗಾದರೂ ಮಾಡಿ, ಎಷ್ಟೇ ಕಷ್ಟಪಟ್ಟಾದರೂ ಕಲಿಸಲು ಸಾಧ್ಯವೇ ಎಂದು ನೋಡಬೇಕೆಂಬ ಹಂಬಲ ಉದಿಸಿತು. ಏನೇ ಆಗಲಿ, ಪ್ರಯತ್ನಿಸಿ ನೋಡೋಣ ಎಂದು ತೋರಿತು.

ಸುಮಾರು ಮೂರೂವರೆ ತಿಂಗಳುಗಳ ಕಾಲ ಒಂದು ದಿನವೂ ತಪ್ಪದೆ ನಿಷ್ಠೆಯಿಂದ ಅವನು ಲೆಕ್ಕ ಹೇಳಿಸಿಕೊಳ್ಳಲು ಬಂದ. ಆತನಲ್ಲಿ ಮೊದಲು ಆತ್ಮವಿಶ್ವಾಸ ಮತ್ತು ಅಭಿರುಚಿ ಉಂಟಾಗಲೆಂದು ಅವನು ಒಂದೂ ತಪ್ಪು ಮಾಡಲು ಸಾಧ್ಯವಿರದಂಥ ಸುಮಾರು ನೂರು ಲೆಕ್ಕಗಳನ್ನು ಕೊಟ್ಟಿದ್ದೆ. ಈ ಲೆಕ್ಕಗಳು ಮೊದಮೊದಲು ಬಹಳ ಸುಲಭವಾಗಿದ್ದು ಐವತ್ತರ ನಂತರ ಕ್ರಮವಾಗಿ ಸ್ವಲ್ಪ ಕಠಿಣವಾಗಿದ್ದವು. ಮೊದಲ ದಿನ ಸುಮಾರು ಹತ್ತು ಲೆಕ್ಕಗಳನ್ನು ಸರಿಯಾಗಿ ಮಾಡಿದಾಗ ‘ವೆರಿಗುಡ್, ಸರಿಯಾಗಿ ಮಾಡಿದಿ. ನಿನಗೆ ಲೆಕ್ಕದಲ್ಲಿ ಅಭಿರುಚಿ ಇದೆ’ ಎಂದೆ. ನೂರು ಲೆಕ್ಕಗಳನ್ನು ತಪ್ಪಿಲ್ಲದೆ ಮಾಡುವಂತಾಗಬೇಕೆಂಬುದು ನಾನು ಹಾಕಿಕೊಂಡ ಯೋಜನೆಯಾಗಿತ್ತು. ದೇವರ ದಯೆಯಿಂದ ಅದು ಯಶಸ್ವಿಯಾಯಿತು. ನೂರು ಲೆಕ್ಕಗಳನ್ನೂ ತಪ್ಪಿಲ್ಲದೇ ಮಾಡಿ ಮುಗಿಸಿದ ಮೇಲೆ ‘ಯಾರಯ್ಯ ನಿನಗೆ ಲೆಕ್ಕ ಮಾಡಲು ಬರುವುದಿಲ್ಲ ಎಂದು ಹೇಳಿದವರು? ನೀನು ಒಂದೇ ಒಂದು ತಪ್ಪು ಮಾಡಲಿಲ್ಲ! ನೋಡು ಈ ಬಾರಿ ನಿನಗೆ ಎಪ್ಪತ್ತು ಪರ್ಸೆಂಟ್ ಬರುತ್ತದೆ ಪರೀಕ್ಷೆ ಯಲ್ಲಿ!’ ಎಂದೆ. ಅವನ ಆತ್ಮವಿಶ್ವಾಸ ಚಿಗುರಿ ಉತ್ಸಾಹ ಗರಿಗೆದರಿದುದನ್ನು ನೋಡಿ ನನಗೆ ಸಂತೋಷವಾಯಿತು. ರಾತ್ರಿ ಎಂಟೂವರೆ ಗಂಟೆಯಿಂದ ಕೆಲವೊಮ್ಮೆ ಹತ್ತೂವರೆಯವರೆಗೂ ಅವನು ಬಿಡದೆ ಕುಳಿತು ಲೆಕ್ಕ ಮಾಡುತ್ತಿದ್ದ. ಮುಂದೆ ಅವನು ನಾಲ್ಕು ತಿಂಗಳ ನಂತರ ಪರೀಕ್ಷೆ ಯಲ್ಲಿ ಕಾಪಿ ಹೊಡೆಯದೇ ಸ್ವಂತ ಬುದ್ಧಿಯನ್ನುಪಯೋಗಿಸಿ ನೂರರಲ್ಲಿ ಎಪ್ಪತ್ತು ಅಂಕಗಳನ್ನು ಪಡೆದ. ಆಗ ಅವನಲ್ಲಿ ಕಂಡು ಬಂದ ಆತ್ಮತೃಪ್ತಿ, ಆನಂದಗಳನ್ನು ವರ್ಣಿಸಲು ಶಬ್ದಗಳಿಗೆ ಶಕ್ತಿ ಸಾಲದು! ಒಂದು ಜೀವ ಹೂವು ಅರಳಿದಂತೆ ಹಿರಿಹಿರಿ ಹಿಗ್ಗಿದುದನ್ನು ನಾನು ಕಂಡೆ. ಒಂದು ದೊಡ್ಡ ಪಾಠವನ್ನೂ ಕಲಿತೆ. ಅವನ ಅಭಿವೃದ್ಧಿಯ ಕತೆಯನ್ನು ಕೇಳಿ ತಿಳಿದ ಅವನ ತಂದೆತಾಯಿ ಆಶ್ರಮಕ್ಕೆ ಬಂದು ನನ್ನನ್ನು ಕಂಡು ಸಂತೋಷದ ಕಂಬನಿಯಿಂದ ಕೃತಜ್ಞತೆಯ ಮಾತುಗಳನ್ನು ಹೇಳಿ ವಂದಿಸಿದರು.”


\section*{ಆತ್ಮವಿಶ್ವಾಸ ಕುದುರಲಿ}

\addsectiontoTOC{ಆತ್ಮವಿಶ್ವಾಸ ಕುದುರಲಿ}

ನಿಜವಾಗಿಯೂ ವಿದ್ಯಾರ್ಥಿಯು ದಡ್ಡನಾಗಿರಲಿಲ್ಲ. ಅವನ ತಂದೆ ಓದು ಬರಹ ಬಲ್ಲ ವಿದ್ಯಾ ವಂತರು ಮಾತ್ರವಲ್ಲ, ವಿದ್ವಾಂಸರೂ ಆಗಿದ್ದರು. ಅವರ ವೈಶಿಷ್ಟ್ಯವೋ, ದೋಷವೋ, ಮಕ್ಕಳು ಬೇಗನೆ ಓದು ಕಲಿತು ಮುಂದೆ ಬರಬೇಕು ಎಂಬ ಆತುರದವರು. ಅವರು ತಾವು ಕಲಿಯುತ್ತಿದ್ದಾಗ ಎಲ್ಲ ಕ್ಲಾಸುಗಳಲ್ಲೂ ಪ್ರಥಮದರ್ಜೆಯವರಂತೆ. ಯಾವುದೇ ವಿಚಾರವನ್ನು ಬಹುಬೇಗನೇ ಗ್ರಹಿಸಬಲ್ಲವರು. ಮಗನನ್ನು ಮಹಾ ಮೇಧಾವಿಯನ್ನಾಗಿ ಮಾಡಲು ನಾಲ್ಕನೇ ವಯಸ್ಸಿನಿಂದಲೇ ಅವನ ತಲೆಯಲ್ಲಿ ಗಣಿತವನ್ನು ತುರುಕಲು ತೊಡಗಿದ್ದರು. ಅವನು ಥಟ್ಟನೆ ಗ್ರಹಿಸದಿದ್ದಾಗ ತಾಳ್ಮೆ ಗೆಟ್ಟ ಅವರು ಬೈದು ಹೊಡೆದು ಹೇಳಿಕೊಡಲು ಯತ್ನಿಸಿ ವಿಫಲರಾದರು. ಅವರು ಹೆಚ್ಚು ಹೆಚ್ಚು ವ್ಯಗ್ರರಾಗಿ ಬೋಧಿಸಹೊರಟಂತೆ ಹುಡುಗನ ಮನಸ್ಸು ಅಷ್ಟಷ್ಟೂ ಭಯದಿಂದ ಒರಟಾಗುತ್ತ ಬಂದಿತು. ಆ ಬಳಿಕ ತಮ್ಮ ಗಣಿತ ಬೋಧನೆಯಿಂದ ಹುಡುಗನನ್ನು ಬಾಧಿಸದಿದ್ದರೂ, ಆಗಾಗ ‘ಕಲಿಯುವುದರಲ್ಲಿ ಹಿಂದೆ, ಊಟದಲ್ಲಿ ಮುಂದೆ’ ಎಂದು ಚಟಾಕಿ ಹಾರಿಸುತ್ತಿದ್ದರು. ಹುಡುಗ ಆಗಲೇ ತನಗೆ ಲೆಕ್ಕಬಾರದು ಎಂದುಕೊಂಡು ತನಗರಿವಿಲ್ಲದೆ ಲೆಕ್ಕವನ್ನು ದ್ವೇಷಿಸತೊಡಗಿದ. ಆತ ಶಾಲೆ ಸೇರಿಕೊಂಡಾಗ ಅಲ್ಲಿಯೂ ಇತರ ವಿದ್ಯಾರ್ಥಿಗಳೊಂದಿಗೆ ತನ್ನನ್ನು ಹೋಲಿಸಿಕೊಂಡಾಗ ತಾನು ಲೆಕ್ಕದಲ್ಲಿ ಹಿಂದೆ ಎಂಬುದನ್ನು ಇಷ್ಟವಿಲ್ಲದಿದ್ದರೂ ಒಪ್ಪಬೇಕಾಯಿತು. ದಿನದಿನವೂ ಉಳಿದ ಹುಡುಗರು, ಅಧ್ಯಾಪಕರು ‘ಅರ್ಥವಾಯಿತೆ?’ ಎಂದು ಕೇಳಿದಾಗ ‘ಹೌದು’ ಎಂದು ಒಕ್ಕೊರಲಿನಿಂದ ಹೇಳುವಾಗ ತಾನು ಮಾತ್ರ ಅರ್ಥವಾಗದೆ ಭಯದಿಂದ ‘ಹೌದು’ ಎನ್ನುವ ಹೌದಪ್ಪರ ಜೊತೆ ಸೇರಿಕೊಳ್ಳುತ್ತಿದ್ದ. ಒಂದು ದಿನ ಅರೆನಿದ್ರೆಯಲ್ಲಿರುವಾಗ ತಂದೆ ತನ್ನ ತಾಯಿಯ ಹತ್ತಿರ ‘ಹುಡುಗನಿಗೆ ಲೆಕ್ಕದಲ್ಲಿ ತಲೆ ಇಲ್ಲವೆಂದು ಮೇಷ್ಟರು ಹೇಳುತ್ತಾರೆ, ಏನು ಮಾಡುವುದೊ?’ ಎಂದದ್ದು ಕೇಳಿಸಿತು. ಮುಂದೆ ತ್ರೈಮಾಸಿಕ ಪರೀಕ್ಷೆಯ ನಂತರ ಮಾರ್ಕ್ಸ್ ಕಾರ್ಡ್ ಬಂದಾಗ ಬಹಳ ಕಡಿಮೆ ಅಂಕಗಳು ಬಂದು ಫೇಲಾದುದು ಗೊತ್ತಾಯಿತು. ಹೀಗೆ ಹಲವಾರು ಸಾಕ್ಷ್ಯಾಧಾರ ಮತ್ತು ಸ್ವಂತ ಅನುಭವಗಳಿಂದ ತನಗೆ ಲೆಕ್ಕ ಬರುವುದಿಲ್ಲ ಮಾತ್ರವಲ್ಲ, ತನಗೆ ಲೆಕ್ಕದಲ್ಲಿ ತಲೆಯೇ ಇಲ್ಲ ಎನ್ನುವ ಸಂಗತಿ ಸತ್ಯ ಎಂಬುದನ್ನು ಅವನ ಒಳ ಮನಸ್ಸು ಸ್ವೀಕರಿಸಿತ್ತು. ಮುಂದೆ ಲೆಕ್ಕದ ಅಧ್ಯಾಪಕರು ಕ್ಲಾಸಿಗೆ ಬಂದಾಗ ಅವನಿಗೆ ತನ್ನ ಎದೆ ಬಡಿತ ಕೇಳಲು ಶುರುವಾಯಿತು. ಪರೀಕ್ಷೆಯಲ್ಲಿ ಲೆಕ್ಕದ ಪೇಪರು ಕಂಡಾಗ ಸರ್ವಾಂಗವೂ ಬೆವೆತು, ತನ್ನ ಜಾಗವನ್ನು ಬಿಟ್ಟು ಬೇಗನೆ ಹೊರಗೆ ಹೋಗೋಣವೆನಿಸುತ್ತಿತ್ತು. ಸ್ನೇಹಿತರಲ್ಲಿ ಇವನಿಗಿಂತ ಹೆಚ್ಚು ಬುದ್ಧಿವಂತರನ್ನೇನೊ ‘ಲೆಕ್ಕ ಹೇಳಿಕೊಡುತ್ತೀಯಾ?’ ಎಂದು ಕೇಳಿದ್ದ. ಅವರು ಆಸೆ ತೋರಿಸಿ ದರೂ ತಾಳ್ಮೆಯಿಂದ ಹೇಳಿಕೊಡಲಿಲ್ಲ. ಆಗ ಬೇರೆ ದಾರಿ ಕಾಣದೆ ಮೇಷ್ಟರಿಗೆ ಟೋಪಿ ಹಾಕಿ ಪರೀಕ್ಷೆಯಲ್ಲಿ ಕಾಪಿ ಹೊಡೆಯುವ ವಿದ್ಯೆಯನ್ನು ಕರಗತಮಾಡಿಕೊಂಡ.

ತನಗೆ ಲೆಕ್ಕ ಬರುವುದಿಲ್ಲ ಎನ್ನುವ ಬೇರುಬಿಟ್ಟ ಭಾವನೆಯನ್ನು ಪ್ರಯತ್ನದಿಂದ ಕಿತ್ತೆಸೆಯ ಬಹುದು. ಯಾವೆಲ್ಲ ಅನುಭವ ಸಾಕ್ಷಿ ಆಧಾರಗಳಿಂದ ಆ ಭಾವನೆ ಬೇರು ಬಿಟ್ಟಿದೆಯೊ ಅದಕ್ಕೆ ವಿರೋಧವಾದ ಅನುಭವ ಸಾಕ್ಷಿ ಆಧಾರಗಳಿಂದ ಅದನ್ನು ದೂರೀಕರಿಸಬಹುದು. ಶಾಲೆಗೆ ಸೇರಿದ ಪ್ರಥಮದಿಂದಲೇ ಅಥವಾ ಮೊದಲೇ ಗಣಿತದಲ್ಲಿ ಮುಗ್ಗರಿಸಿ ಆತ್ಮವಿಶ್ವಾಸವನ್ನು ಕ್ರಮವಾಗಿ ಕಳೆದುಕೊಂಡ ಆತನಿಗೆ ಪ್ರಾರಂಭದ ಹಂತದಿಂದಲೇ ಹಲವಾರು ಸಣ್ಣಪುಟ್ಟ ವಿಜಯಗಳನ್ನು ತಂದುಕೊಟ್ಟು ಆತ್ಮವಿಶ್ವಾಸವನ್ನು ಪುನಃ ಚಿಗುರುವಂತೆ ಮಾಡಬೇಕು. ಅಪಜಯದ ನಿಷೇಧಾ ತ್ಮಕ ಭಾವನೆಯನ್ನು ಹೊರತಳ್ಳುವ ದಾರಿ ಅದೊಂದೆ. ಇದು ಖಂಡಿತ ಸಾಧ್ಯವಾಗುವ ಕೆಲಸ ವಾದರೂ ಸ್ವಲ್ಪಮಟ್ಟಿಗೆ ಕಷ್ಟಸಾಧ್ಯವೇ. ಏಕೆಂದರೆ ತರಬೇತಿ ನೀಡುವ ಅಧ್ಯಾಪಕರಲ್ಲಿ ಅಪಾರ ತಾಳ್ಮೆ, ಪ್ರೀತಿ, ದೃಢ ವಿಶ್ವಾಸಬೇಕು. ಇವರು ವಿಶ್ವಾಸಕ್ಕೆ ಅರ್ಹರು, ತನ್ನ ಮೂರ್ಖತೆಯನ್ನು ಕಂಡು ಅಪಹಾಸ್ಯ ಮಾಡರು, ತಾಳ್ಮೆ ಕಳೆದುಕೊಳ್ಳಲಾರರು, ಅನುಕಂಪೆಯಿಂದ ಹೇಳಿಕೊಡುವರು ಎಂದಾದರೆ ಹುಡುಗನಿಗೆ ಧೈರ್ಯಬರುವುದು. ಅವನು ತನ್ನ ಕಾಲಮೇಲೆ ತಾನೇ ನಿಲ್ಲಲು ಆಶಿಸಲಾರನೇ? ಖಂಡಿತ ಆಶಿಸುತ್ತಾನೆ. ಆದರೆ ತನ್ನ ಮನಸ್ಸು ಮತ್ತು ಪರಿಸರದ ಜನ ತನಗೇ ವಿರೋಧವಾಗಿ ಕೆಲಸ ಮಾಡುವುದನ್ನು ಅವನು ಅಸಹಾಯನಾಗಿ ನೋಡುತ್ತ ಬಂದಿದ್ದಾನೆ ಅಷ್ಟೆ. ನಿರಂಜನ ನನಗೊಬ್ಬನಿಗೇ ಸ್ವಲ್ಪ ಲೆಕ್ಕ ಹೇಳಿಕೊಡುತ್ತೀರಾ ಎಂದೇಕೆ ಕೇಳಿದ? ಇನ್ನೊಬ್ಬ ವಿದ್ಯಾರ್ಥಿಯನ್ನು ಹೋಲಿಸಿ ತನ್ನನ್ನು ಎಲ್ಲಿಯಾದರೂ ತೆಗಳಿದರೆ ಎಂಬ ಭೀತಿಯಿಂದಲ್ಲವೇ?

ದಿನದಿನವೂ ಕಲಿಕೆಯಲ್ಲಿ ಮುಂದಿರುವ ಗೆಳೆಯರೊಂದಿಗೆ ತನ್ನನ್ನು ಹೋಲಿಸಿಕೊಳ್ಳುತ್ತ, ನಿಂದೆ ತಿರಸ್ಕಾರದ ನುಡಿಗಳನ್ನು ಕೇಳುತ್ತ ಮಗುವಿನಲ್ಲಿ ನಾನೊಬ್ಬ ಅಪ್ರಯೋಜಕ ವ್ಯಕ್ತಿ ಎಂಬ ಭಾವನೆ ಬೇರುಬಿಟ್ಟರೆ ಅವನ ಬದುಕಿನುದ್ದಕ್ಕೂ ಅದು ಹೇಗೆ ಮಾರಕಪ್ರಾಯವಾಗಿ ಅವನಿಗೆ ಸಂಕಟವನ್ನು ತಂದೀತು ಎಂಬುದನ್ನು ನಮ್ಮಲ್ಲಿ ಎಷ್ಟು ಜನ ಯೋಚಿಸಬಲ್ಲರು? ಮಗುವಿನ ಶುಭವನ್ನು ಹಾರೈಸುವ ತಂದೆ ತಾಯಂದಿರೆ ಮಗುವಿನ ಹಿತಕ್ಕೆ ಸದುದ್ದೇಶದಿಂದಲೇ ಅಜ್ಞಾತವಾಗಿ ಹೇಗೆ ಹಾನಿ ಮಾಡುತ್ತಾರಲ್ಲವೇ? ಇದೊಂದು ಎಂಥ ಆಭಾಸ! ಅಜ್ಞಾನದ ವಿಜೃಂಭಣೆ, ಅಷ್ಟೆ. ಇನ್ನು ಕ್ಲಾಸಿನಲ್ಲಿ ಮಗುವಿನ ಹೆಸರೇ ಗೊತ್ತಿಲ್ಲದ, ಸರಿಯಾಗಿ ಮಕ್ಕಳ ಪರಿಚಯ ಮಾಡಿಕೊಳ್ಳುವುದೇ ಕಷ್ಟ ಎನ್ನುವ ಅಧ್ಯಾಪಕರು ಈ ಸಮಸ್ಯೆಯನ್ನು ಹೇಗೆ ಪರಿಹರಿಸಬಲ್ಲರು? ತಾಳ್ಮೆಯಿಂದ ಕೂಡಿದ ಪ್ರಯತ್ನದಿಂದ ಎಲ್ಲವೂ ಸಾಧ್ಯ ಎಂಬುದು ನೆನಪಿರಲಿ.


\section*{ಡಾಕ್ಟರ್ ಸ್ವಾಮೀಜೀಯೊಬ್ಬರ ಅನುಭವ}

\addsectiontoTOC{ಡಾಕ್ಟರ್ ಸ್ವಾಮೀಜೀಯೊಬ್ಬರ ಅನುಭವ}

ವೈದ್ಯಕೀಯ ವಿಜ್ಞಾನವನ್ನು ಓದಿ ಎಂ.ಬಿ.ಬಿಎಸ್.\ ಪಾಸು ಮಾಡಿದ್ದ ನಮ್ಮ ಮಿಷನ್ನಿನ ಸ್ವಾಮಿ ಗಳೊಬ್ಬರು ತಮ್ಮ ಅನುಭವವನ್ನು ಕುರಿತು ಹೀಗೆಂದರು: ‘ಎಂ.ಬಿ.ಬಿ.ಎಸ್.\ ಕೊನೆಯ ಪರೀಕ್ಷೆ ಯಲ್ಲಿ ಉತ್ತೀರ್ಣನಾದ ಕೂಡಲೆ ಔಷಧೋಪಚಾರ ವಿಭಾಗದಲ್ಲಿ ನನಗೆ ಡೆಮೊನ್​ಸ್ಟ್ರೇಟರ್ ಹುದ್ದೆ ದೊರಕಿತ್ತು. ಮೂರನೇ ವರ್ಷದ ಎಂ.ಬಿ.ಬಿ.ಎಸ್. ವಿದ್ಯಾರ್ಥಿಗಳಿಗೆ ರೋಗಿಗಳನ್ನು ಪರೀಕ್ಷಿಸಿ ರೋಗದ ಕಾರಣವನ್ನು ತಿಳಿಯುವ ವಿಧಾನವನ್ನು ಬೋಧಿಸುವ ಕೆಲಸ ನನ್ನದಾಗಿತ್ತು. ಹದಿಮೂರು ಮಂದಿ ವಿದ್ಯಾರ್ಥಿಗಳನ್ನು ಆರು ವಾರಗಳ ಕಾಲ–ಮುಂದೆ ಅದು ಹನ್ನೆರಡು ವಾರಗಳ ಅವಧಿಗೆ ವಿಸ್ತರಿಸಿತು–ನಾನು ನೋಡಿಕೊಳ್ಳಬೇಕಾಗಿತ್ತು. ಆಸ್ಪತ್ರೆಯ ವಾರ್ಡಿನಲ್ಲೇ ದಿನವೂ ಎರಡು ಮೂರು ಗಂಟೆಗಳ ಕಾಲ, ಒಂದು ವಾರದಲ್ಲಿ ನಾಲ್ಕು ದಿನಗಳು ಕ್ಲಾಸನ್ನು ತೆಗೆದು ಕೊಳ್ಳುತ್ತಿದ್ದೆ. ರೋಗದ ಲಕ್ಷಣಗಳನ್ನು ಪರಿಶೀಲಿಸಿ ಅದನ್ನು ಸರಿಯಾಗಿ ಹೇಗೆ ಅರ್ಥೈಸಿಕೊಳ್ಳ ಬೇಕೆಂಬುದನ್ನು ವಿವರಿಸುತ್ತಿದ್ದೆ. ಆಗ ನಾನು ಸ್ವಾಮಿ ವಿವೇಕಾನಂದರ ಬರಹಗಳನ್ನು ಓದಿ ಸ್ಫೂರ್ತಿ ಪಡೆದುಕೊಂಡಿದ್ದೆ. ಪ್ರತಿಯೊಬ್ಬ\-ನಲ್ಲೂ ಅಪಾರಶಕ್ತಿ ಅಡಗಿದೆ, ಆ ಶಕ್ತಿಯನ್ನು ಶ್ರದ್ಧೆ, ಆತ್ಮವಿಶ್ವಾಸ ಮತ್ತು ಸರಿಯಾದ ಪ್ರಯತ್ನಗಳಿಂದ ಎಚ್ಚರಿಸಿ ಏನನ್ನೂ ಸಾಧಿಸಬಹುದು. ಸಾಮಾನ್ಯನೂ ಸಮರ್ಥನಾಗಬಹುದು ಎಂಬ ಅವರ ಮಾತುಗಳನ್ನು ನಾನು ಮನನ ಮಾಡುತ್ತಿದ್ದೆ. ಅವರ ವಿಚಾರಧಾರೆಯಿಂದ ಆಗ ತುಂಬ ಪ್ರಭಾವಿತನಾಗಿದ್ದೆ. ಆಗ ನನ್ನಲ್ಲಿ ಆತ್ಮವಿಶ್ವಾಸ, ಉತ್ಸಾಹ, ಏನನ್ನೂ ಸಾಧಿಸಬಲ್ಲೆ ಎನ್ನುವ ಛಾತಿ ತುಂಬಿತುಳುಕುತ್ತಿದ್ದವು ಎಂದರೆ ಅತಿಶಯೋಕ್ತಿಯಲ್ಲ.

ಪ್ರತಿಯೊಬ್ಬ ವಿದ್ಯಾರ್ಥಿಯ ಆಂತರ್ಯದಲ್ಲಿ ಅಪಾರ ಶಕ್ತಿ ಅಡಗಿದೆ. ಅವರೆಲ್ಲರೂ ತಮ್ಮ ಅಧ್ಯಯನ ಕ್ಷೇತ್ರದಲ್ಲಿ ಹೆಚ್ಚಿನ ಪರಿಣತಿ ಸಾಧಿಸಬಲ್ಲರೆಂದು ನಾನು ಬೋಧಿಸಹೊರಡದಿದ್ದರೂ ಅಂಥ ಸಾಧ್ಯತೆಯ ಬಗ್ಗೆ ನನ್ನಲ್ಲಿ ಅಪಾರ ವಿಶ್ವಾಸವಿತ್ತು. ಹಿಂದಿನ ಪರೀಕ್ಷೆಯಲ್ಲಿ ಫೇಲಾದ ಹದಿಮೂರು ಮಂದಿ ವಿದ್ಯಾರ್ಥಿಗಳ ಈ ಗುಂಪು ಒಂದು ಸೆಮಿಸ್ಟರ್ ಕಳೆದುಕೊಂಡವರಾಗಿದ್ದರು.

ಪ್ರತಿಭಾಶಾಲಿಗಳಲ್ಲದ ವಿದ್ಯಾರ್ಥಿಗಳ ಗುಂಪಿಗೆ ಸೇರಿದ ಇವರು ತಮಗೆ ನಿರ್ವಹಿಸಲು ಹೇಳಿದ ಕೆಲಸ ಕಾರ್ಯಗಳನ್ನು ಅಚ್ಚುಕಟ್ಟಾಗಿ ಮಾಡುತ್ತಿರಲಿಲ್ಲ. ರೋಗಿಗಳ ಬಗ್ಗೆ ಸ್ಪಷ್ಟ ಚಿತ್ರಣ, ವಿವರಣೆ ಸರಿಯಾಗಿ ಬರೆಯುತ್ತಿರಲಿಲ್ಲ. ಕ್ಲಾಸಿಗೆ ಸಮಯಕ್ಕೆ ಸರಿಯಾಗಿ ಬರಲೂ ಅವರಿಗೆ ಸಾಧ್ಯ ವಾಗುತ್ತಿರಲಿಲ್ಲ. ಆದರೆ ನಾನು ಅತ್ಯಂತ ಪ್ರಾಮಾಣಿಕನಾಗಿ ಅವರೊಂದಿಗೆ ಅವರ ಪ್ರಗತಿ ಸಾಧ್ಯ–ಖಂಡಿತ ಸಾಧ್ಯ ಎನ್ನುವ ದೃಢಶ್ರದ್ಧೆಯಿಂದ ದುಡಿಯತೊಡಗಿದೆ. ಅವರಲ್ಲೇ ಒಬ್ಬನಾಗಿ ಸ್ನೇಹಿತನಂತೆ ಕೆಲವೊಮ್ಮೆ ಪ್ರೋತ್ಸಾಹದ ಮಾತುಗಳನ್ನಾಡುತ್ತ ಪಾಠಪ್ರವಚನಗಳನ್ನು ನಡೆಯಿಸಿದೆ. ವಿದ್ಯಾರ್ಥಿಯೊಬ್ಬ ಕ್ಲಾಸಿಗೆ ಗೈರುಹಾಜರಾಗಿದ್ದರೆ ಆದಷ್ಟು ಬೇಗನೇ ಅವನ ಕಡೆ ವಿಶೇಷ ಗಮನವಿತ್ತು ಹಿಂದಿನ ಕ್ಲಾಸುಗಳಲ್ಲಿ ತಿಳಿದಿರದಿದ್ದ ಅನೇಕ ಸಂಗತಿಗಳನ್ನು ವಿವರಿಸುತ್ತಿದ್ದೆ. ಮುಂದಿನ ಕ್ಲಾಸುಗಳಲ್ಲಿ ಅಧ್ಯಯನ ಮುಂದುವರಿಸಲು ತೊಂದರೆಯಾಗದಂತೆ ನೋಡಿಕೊಳ್ಳು ತ್ತಿದ್ದೆ. ಪ್ರತಿಯೊಬ್ಬ ವಿದ್ಯಾರ್ಥಿಯ ಬಗೆಗೂ ವಿಶೇಷ ಗಮನವೀಯುತ್ತ ಆತನ ತೊಂದರೆಗಳೇ ನೆಂದು ತಿಳಿದು ಪರಿಹಾರ ಸೂಚಿಸುತ್ತಿದ್ದೆ. ಇದೆಲ್ಲದರ ಪರಿಣಾಮವಾಗಿ ಮುಂದಿನ ಪರೀಕ್ಷೆಯಲ್ಲಿ ಎಲ್ಲರೂ ಪಾಸಾದರು ಮಾತ್ರವಲ್ಲ ನನ್ನ ಸಹಾಯವಿಲ್ಲದೇ ಇತರ ಅನೇಕ ಪರೀಕ್ಷೆಗಳಲ್ಲಿ ಪಾಸಾ ಗುತ್ತ ಮುನ್ನಡೆದರು. ಅವರಲ್ಲೊಬ್ಬ ಪ್ರಥಮ ದರ್ಜೆಯಲ್ಲಿ ಪಾಸಾದ. ಸ್ವಾಮಿ ವಿವೇಕಾನಂದರ ದಿವ್ಯವಾಣಿ ನೀಡಿದ ಆತ್ಮವಿಶ್ವಾಸದ ಅಪರಿಮಿತಶಕ್ತಿಯನ್ನು ಕುರಿತ ಸ್ಫೂರ್ತಿಯೇ ನನ್ನ ಈ ವಿಜಯಕ್ಕೆ ಕಾರಣವಾಯಿತು.’

ಓದುಗರು ಇಲ್ಲಿ ಒಂದು ವಿಚಾರವನ್ನು ಗಮನಿಸಬೇಕು. ಅಧ್ಯಾಪಕರಿಗೆ ತಮ್ಮಲ್ಲಿ ಮಾತ್ರ ಅಪಾರ ವಿಶ್ವಾಸವಿದ್ದುದಷ್ಟೇ ಅಲ್ಲ. ಹಿಂದುಳಿದ ವಿದ್ಯಾರ್ಥಿಗಳೂ ಖಂಡಿತವಾಗಿ ಮೇಲೇರ ಬಲ್ಲರು, ಅವರಲ್ಲಿ ಆ ಶಕ್ತಿ ಅಡಗಿದೆ, ಅದನ್ನು ಎಚ್ಚರಿಸಲು ಸಾಧ್ಯ ಎನ್ನುವ ಸಂಗತಿಯಲ್ಲೂ ಅಪಾರ ವಿಶ್ವಾಸವಿತ್ತು. ಎಷ್ಟೋ ಮಂದಿ ಮೇಧಾವಿಗಳೂ ಅತ್ಯುತ್ತಮ ಮಟ್ಟದ ಅಧ್ಯಾಪಕರೂ ಇದ್ದಾರೆ ನಿಜ. ಆದರೆ ವಿದ್ಯಾರ್ಥಿಗಳು ವಿದ್ಯುದ್ವೇಗದಿಂದ ವಿಷಯಗಳನ್ನು ಗ್ರಹಿಸದಿದ್ದರೆ ಅವರನ್ನು ಸ್ವಲ್ಪ ನಿಕೃಷ್ಟವಾಗಿ ‘ಅಪ್ರಯೋಜಕ’, ‘ಅಧ್ಯಯನಕ್ಕೆ ಯೋಗ್ಯತೆ ಇಲ್ಲ’, ‘ಮೂರ್ಖ’ ಎಂದು ಬಾಯ್ಬಿಟ್ಟು ಹೇಳದಿದ್ದರೂ ಆ ದೃಷ್ಟಿಯಿಂದ ನೋಡಿದರೂ ಅದರ ದುಷ್ಪರಿಣಾಮ ವಿದ್ಯಾರ್ಥಿಗಳ ಮೇಲಾಗುವುದು!


\section*{ಬದುಕಿನ ಸಾರ್ಥಕತೆ}

\vskip -8pt\addsectiontoTOC{ಬದುಕಿನ ಸಾರ್ಥ\-ಕತೆ}

‘ಏನು ಆಗಬೇಕೆಂದು ಯೋಚಿಸಿ ಚಿತ್ರಿಸಿ ಕಲ್ಪಿಸಿಕೊಂಡಿರುತ್ತೇವೆಯೋ ಅದನ್ನು ಕುರಿತು ನಾವು ಅರ್ಧಜಾಗೃತರಾಗಿರುತ್ತೇವೆ. ನಮ್ಮ ದೈಹಿಕ ಹಾಗೂ ಮಾನಸಿಕ ಶಕ್ತಿಗಳ ಬಹು ಅಲ್ಪಭಾಗವನ್ನು ಮಾತ್ರ ನಾವು ಉಪಯೋಗಿಸುತ್ತೇವೆ. ಒಟ್ಟಿನಲ್ಲಿ ಮಾನವನು ತನ್ನ ಶಕ್ತಿಯ ವಿಸ್ತಾರ ಪರಿಧಿ ಯೊಳಗಣ ತೀರ ಸೀಮಿತ ಕ್ಷೇತ್ರದಲ್ಲೇ ಬದುಕುತ್ತಾನೆ. ನಾನಾವಿಧದ ಶಕ್ತಿಗಳನ್ನು ಪಡೆದೂ, ಅವನ್ನು ಉಪಯೋಗಿಸಲು ಹಂಬಲಿಸದೇ, ಆ ನಿಟ್ಟಿನಲ್ಲಿ ಯತ್ನಿಸದೆ ಅಸಮರ್ಥನಾಗಿಯೇ ಬದುಕುತ್ತಾನೆ.’ –ಹೀಗೆಂದು ಹೇಳಿದವರು ಪ್ರಸಿದ್ಧ ಮನೋವಿಜ್ಞಾನಿ ವಿಲಿಯಮ್ ಜೇಮ್ಸ್.

ವಿಶಾಲವಾದ ಮೈದಾನದಲ್ಲಿ ಹಾಯಾಗಿ ಹುಲ್ಲು ಮೇದು ತಿರುಗಾಡುವಷ್ಟು ಉದ್ದವಾದ ಹಗ್ಗ ವಿದೆ. ಆದರೂ ಗೂಟಕ್ಕೆ ಕಟ್ಟಿದ ಹಸು ಗೂಟದ ಬಳಿಯೇ ಹಗ್ಗ ಸುತ್ತಿಕೊಂಡು ತನ್ನ ಸಂಚಾರದ ಸ್ವಾತಂತ್ರ್ಯ, ಸೌಲಭ್ಯಗಳನ್ನು ಮೊಟಕುಗೊಳಿಸಿಕೊಂಡಂತೆ ಇದು. ಮನುಷ್ಯರೂ ತಮ್ಮಲ್ಲಿರುವ ಶಕ್ತಿಸಾಧ್ಯತೆಗಳನ್ನರಿಯದೆ ಬಹಳಷ್ಟು ಪರಿಮಿತ ಪರಿಧಿಯಲ್ಲೇ ಬದುಕಿಕೊಂಡಿರುತ್ತಾರೆಂಬುದು ಮನೋವಿಜ್ಞಾನಿಯ ಮಾತಿನ ಅರ್ಥ.

ಎಂಥ ಅದ್ಭುತ ಯಂತ್ರ ನಮ್ಮ ದೇಹ! ಎಷ್ಟು ಅದ್ಭುತ ಶಕ್ತಿ ಹುದುಗಿದೆ ನಮ್ಮಲ್ಲಿ! ಆದರೂ ಕಣ್ಣನ್ನು ಮುಚ್ಚಿಕೊಂಡು ಕತ್ತಲೆ ಎಂದು ಅಳುವವರು ನಾವು!

ನಮಗೆ ಜೀವನದಲ್ಲಿ ಮಹತ್ವಾಕಾಂಕ್ಷೆಗಳಿಲ್ಲ–ನಿರ್ದಿಷ್ಟ ಉದ್ದೇಶಗಳಿಲ್ಲ. ನಮ್ಮ ಶಕ್ತಿಯಲ್ಲಿ ನಮಗೆ ವಿಶ್ವಾಸವೂ ಇಲ್ಲ. ನಿಯಮಗಳನ್ನನುಸರಿಸಿ ನಾವು ಕೆಲಸ ಮಾಡುತ್ತಿಲ್ಲ. ಹರಟೆ ಹೊಡೆಯಲು ಮಾತ್ರ ಇಚ್ಛಿಸುತ್ತೇವೆ. ನಾವು ಯಶಸ್ವಿಗಳಾಗಬಲ್ಲೆವೆ?

ಸಂತನೊಬ್ಬ ತನ್ನ ಬದುಕಿನ ಕೊನೆಯ ದಿನಗಳಲ್ಲಿ ಜೀವನಾವಲೋಕನ ಮಾಡುತ್ತ ‘ನಾನು ಒಳ್ಳೆಯ ರೀತಿಯಲ್ಲೇ ಹೋರಾಡಿದ್ದೇನೆ’ ಎಂದನಂತೆ. ಸತತ ಪ್ರಯತ್ನ ಹಾಗೂ ಬಿಡದ ಛಲ ಗಳಿಂದ ಅನವರತ ಕಾರ್ಯನಿರತನಾಗಿ ಕಷ್ಟ ಕಾರ್ಪಣ್ಯಗಳನ್ನು ಕೆಚ್ಚೆದೆಯಿಂದ ಎದುರಿಸಿ ಬದುಕಿದ ಬಾಳಿನ ಸಮೀಕ್ಷೆ ಅದು. ನಮ್ಮ ಜೀವನದ ಸಾಫಲ್ಯವು ದಿನದ ಪ್ರತಿಯೊಂದು ಘಂಟೆಯಲ್ಲಿ ಪ್ರತಿಯೊಂದು ಕ್ಷಣದಲ್ಲಿ ನಾವು ಬದುಕುವ ರೀತಿಯನ್ನು ಹೊಂದಿಕೊಂಡಿದೆ. ನಾಳೆ ನೋಡೋಣ, ಮತ್ತೆ ಮಾಡೋಣ ಎನ್ನುವವರು ತಮ್ಮ ಕೊನೆಯ ದಿನಗಳಲ್ಲಿ ಏನು ಹೇಳಿ ಯಾರು?–

‘ಅಯ್ಯೋ, ವ್ಯರ್ಥ ವಿಚಾರಗಳಲ್ಲೇ ನನ್ನ ಸಮಯ ಕಳೆದು ಹೋಯಿತು.’

‘ಉಪಯುಕ್ತ ವಸ್ತುಗಳನ್ನು ಬಿಟ್ಟು ಮರೀಚಿಕೆಯನ್ನು ಹಿಂಬಾಲಿಸಿದೆನಲ್ಲ?’

‘ಕೆಟ್ಟ ಮೇಲೆ ಬುದ್ಧಿ ಬಂದರೆ ಏನು ಪ್ರಯೋಜನ!’

‘ನಾನು ನನ್ನ ಸ್ನೇಹಿತರ ನಿರೀಕ್ಷೆಯ ಮಟ್ಟವನ್ನೇರದೆ ನನ್ನ ವಿರೋಧಿಗಳು ಹಲ್ಲು ಕಿರಿಯು\-ವಂತೆ ಮಾಡಿದೆ.’

‘ಎಲ್ಲ ವಿಷಯಗಳಲ್ಲೂ ಸುಲಭವಾದುದನ್ನೇ ಆರಿಸಿ ಮೋಸ ಹೋದೆ.’

‘ನಾನು ಸೋತೆನೆಂದು ಹೇಳಲಾಗದಿದ್ದರೂ ನನ್ನ ಜಯವು ಹೇಳುವಂಥದಲ್ಲ!’

ನೀವು ಮುಂದೆ ಹೀಗೆ ಹೇಳಲು ಇಚ್ಛಿಸುವಿರಾ?

ನಿಮ್ಮ ಜೀವನದಲ್ಲಿ ನಿರ್ದಿಷ್ಟ ಉದ್ದೇಶಗಳ ಅಭಾವಕ್ಕೆ ಹಲವು ಕಾರಣಗಳಿರಬಹುದು. ನಿಮ್ಮ ಅಭಿರುಚಿಯ ವಿಷಯದಲ್ಲಿ ನಿಮಗೆ ಬಾಲ್ಯದಿಂದಲೇ ತರಬೇತಿಯ ಅಭಾವವಿರಬಹುದು. ನಿಮ್ಮ ಅನಾರೋಗ್ಯ ದೌರ್ಬಲ್ಯಗಳು ನಿಮ್ಮ ದಾರಿಯಲ್ಲಿ ತೊಡಕುಗಳಾಗಿರಬಹುದು. ನೀವು ಸ್ವಾಭಾವಿಕವಾಗಿ ಚಂಚಲ ಮನಸ್ಕರಾಗಿದ್ದು ದೃಢತೆಯಿಂದ, ಏಕಾಗ್ರತೆಯಿಂದ ಹಿಡಿದ ಕೆಲಸವನ್ನು ಮಾಡದೇ ಇರಬಹುದು. ನೀವು ಮಾಡಬೇಕಾಗಿರುವ ಉದ್ಯೋಗದಲ್ಲಿ ನಿಮಗೆ ಅಭಿರುಚಿಯಾಗಲಿ,\break ಯೋಗ್ಯತೆಯಾಗಲಿ ಇಲ್ಲದೇ ನಿಮ್ಮ ಸತ್ವಶಕ್ತಿಗಳನ್ನು ಪ್ರಕಟಿಸಲು ಅವಕಾಶವಿಲ್ಲದಿರಬಹುದು. ಜೀವನದ ಹಲವು ಕ್ಷೇತ್ರಗಳಲ್ಲಿ ಸೋಲನ್ನು ಕಂಡ ನೀವು ನಿರಾಶಾವಾದಿಗಳಾಗಿರಬಹುದು. ಎಲ್ಲವೂ ವಿಧಿವಶ ಹಣೆಯ ಬರಹ, ಪ್ರಾರಬ್ಧ ಕರ್ಮ ಎನ್ನುತ್ತಿರಬಹುದು.

ನಿಮ್ಮ ಬದುಕಿನ ರೀತಿಯನ್ನು ಸ್ವಲ್ಪ ವಿಮರ್ಶಿಸಿಕೊಳ್ಳಿ. ಸ್ವಲ್ಪ ಜಾಗರೂಕತೆಯಿಂದ ನಿಮ್ಮ ಉದ್ದೇಶವನ್ನು ಕಂಡುಕೊಳ್ಳಿ. ಉದ್ದೇಶವಿಲ್ಲದ ಜೀವನ ಚುಕ್ಕಾಣಿ ಇಲ್ಲದ ದೋಣಿಯಂತೆ ಎಲ್ಲೆಲ್ಲೋ ಸಾಗುವುದು.


\section*{ಬಾಳಿಗೊಂದು ಗುರಿ}

\addsectiontoTOC{ಬಾಳಿಗೊಂದು ಗುರಿ}

ನಿಮ್ಮಲ್ಲಿರುವ ಸುಪ್ತಶಕ್ತಿಗಳನ್ನು ಬಡಿದೆಬ್ಬಿಸಬೇಕಾದರೆ ನಿಮ್ಮ ಜೀವನಕ್ಕೊಂದು ಗುರಿ ಇರಬೇಕು. ಗುರಿ ಎಂದರೆ ಮೋಕ್ಷಾಕಾಂಕ್ಷೆ ಒಂದೇ ಆಗಬೇಕೆಂದಿಲ್ಲ. ಜೀವನದಲ್ಲಿ ನಿರ್ದಿಷ್ಟವಾದ ಹಾಗೂ ಸ್ಪಷ್ಟವಾದ ಉದ್ದೇಶಬೇಕು. ಒಂದು ಸಣ್ಣ ತಾತ್ಕಾಲಿಕ ಅಭಿಲಾಷೆ ಒಂದು ಉದ್ದೇಶವಲ್ಲ. ನೀವು ಕೈಗೊಂಡ ಕೆಲಸದಲ್ಲೆಲ್ಲ ಯಶಸ್ವಿಯಾಗಬೇಕೆಂಬ ಕೇವಲ ಆಸೆ ಮಾತ್ರ ಇದ್ದರೆ ಸಾಲದು. ಗೊತ್ತು\-ಪಡಿಸಿ\-ಕೊಂಡ ಕೆಲಸವನ್ನು ಗೊತ್ತುಪಡಿಸಿದ ಸಮಯದಲ್ಲಿ ಸರಿಯಾದ ರೀತಿಯಲ್ಲಿ ಮಾಡಿ ಮುಗಿ\-ಸುವ ಉದ್ದೇಶ ಬಹಳ ಆವಶ್ಯಕ. ಕಾರ್ಯ ಯೋಜನೆಯೊಂದು ಇಲ್ಲದಿದ್ದರೆ ನಿಮ್ಮ ಅಮೂಲ್ಯವಾದ ಮಾನಸಿಕ ಶಕ್ತಿ ಮತ್ತು ಸಮಯ ವ್ಯರ್ಥವಾಗಿ ವ್ಯಯವಾಗುತ್ತದೆ.

ನೀವು ಸಭೆಯೊಂದರಲ್ಲಿ ಭಾಷಣ ಮಾಡಲು ಯೋಚಿಸಿರಬಹುದು. ವ್ಯಾಪಾರ ನಡೆಯಿಸಲು ಯೋಜನೆ ಹಾಕಿರಬಹುದು. ಕಲಾವಿದನಾಗಲು, ಕಾದಂಬರಿಗಾರನಾಗಲು ಇಚ್ಛಿಸಿರಬಹುದು. ನಿಮ್ಮ ಉದ್ಯೋಗದಲ್ಲೇ ಚೆನ್ನಾಗಿ ದುಡಿದು ಹೆಚ್ಚು ಹಣಗಳಿಸುವ ಯೋಚನೆ ಇರಬಹುದು. ಏನೇ ಇರಲಿ, ನಿಮ್ಮ ಉದ್ದೇಶವನ್ನು ಕುರಿತು ಸ್ಪಷ್ಟವಾದ ಅರಿವು ನಿಮಗಿರಬೇಕಾಗುತ್ತದೆ. ಆ ಒಂದು ವಿಶಿಷ್ಟ ಆಸೆ ಅಥವಾ ಭಾವನೆ ನಿಮ್ಮನ್ನು ಕಾರ್ಯಪ್ರವೃತ್ತರನ್ನಾಗಿ ಮಾಡುತ್ತದೆ.

ಅನಿಶ್ಚಿತತೆಯಿಂದ ಪಾರಾಗಲು ನಾನು ಜೀವನದಲ್ಲಿ ಅತ್ಯಂತ ಹೆಚ್ಚಾಗಿ ಆಶಿಸುವುದು\break ಯಾವುದು ಎಂಬ ಪ್ರಶ್ನೆಯನ್ನು ನಿಮಗೆ ನೀವೇ ಕೇಳಿಕೊಳ್ಳಬೇಕು. ಆಗ ನಿಮ್ಮ ಜೀವನದಲ್ಲಿ ನೀವಿರಿಸಿಕೊಳ್ಳಬಹುದಾದ ಮುಖ್ಯ ಉದ್ದೇಶದ ಚಿತ್ರ ಹೆಚ್ಚು ಸ್ಪಷ್ಟವಾಗುವುದು. ಆಗ ನಿಮ್ಮ ಮಾನಸಿಕ ಶಕ್ತಿ ಆ ದಿಸೆಯಲ್ಲಿ ಹರಿಯಲು ತೊಡಗುವುದು.


\section*{ಹಂಬಲದ ಬೆಂಬಲ}

\vskip -8pt\addsectiontoTOC{ಹಂಬಲದ ಬೆಂಬಲ}

ಜರ್ಮನ್ ದೇಶದ ಫ್ರಾಂಕ್​ಫರ್ಟ್ ನಗರದ ಒಂದು ಗಲ್ಲಿಯಲ್ಲಿನ ಪುಸ್ತಕಾಲಯ. ಪುಸ್ತಕಗಳಿಂದ ಆವೃತರಾಗಿ ಏಕಾಗ್ರಚಿತ್ತ ಸಮಾಧಿಸ್ಥ ಯೋಗಿಯಂತೆ ಒಬ್ಬರು ಅಲ್ಲಿ ಕುಳಿತಿದ್ದಾರೆ. ಅದು ಅವರ ಸ್ವಂತ ಗ್ರಂಥಾಲಯ. ಅಲ್ಲಿ ಸುಮಾರು ಮೂವತ್ತು ಸಾವಿರ ಪುಸ್ತಕಗಳಿವೆ. ಬಹಳಷ್ಟು ಭಾಷಾವಿಜ್ಞಾನಕ್ಕೆ ಸಂಬಂಧಿಸಿದ ಪುಸ್ತಕಗಳೇ. ಅವರು ಒಂದಲ್ಲ ಎರಡಲ್ಲ, ಹತ್ತಲ್ಲ, ನೂರಲ್ಲ– ಸುಮಾರು ಮುನ್ನೂರು ಭಾಷೆಗಳನ್ನು ಬಲ್ಲರು. ಓದು ಬರಹ ಮಾತ್ರವಲ್ಲ, ಅನುವಾದ ಕಾರ್ಯವನ್ನೂ ಮಾಡಬಲ್ಲರು. ಸರಳವಾಗಿ ಮಾತನಾಡಬಲ್ಲರು. ಇದು ಕಲ್ಪಿತ ಕಥೆಯಲ್ಲ. ವಾಸ್ತವ ವಿಚಾರ. ಅವರ ಹೆಸರು ಡಾ.\ ಹೆರಾಲ್ಡ್ ಸ್ರುಜ್. ಡಾ.\ ಸ್ರುಜ್ ಬಹುಭಾಷಾತಜ್ಞರಷ್ಟೇ ಅಲ್ಲ, ಶ್ರೇಷ್ಠ ಕವಿಗಳೂ ಹೌದು.

ಇಷ್ಟೊಂದು ಭಾಷೆಗಳನ್ನು ಹೇಗೆ ಕಲಿತಿರಿ? ಎಂದು ಅವರನ್ನು ಪ್ರಶ್ನಿಸಿದಾಗ ಅವರು ನಗುತ್ತ ಉತ್ತರಿಸಿದರು: ‘ಬಹುಭಾಷಾಜ್ಞಾನ ಪಡೆಯಲು ಮೂರು ಸಂಗತಿಗಳು ಬೇಕೇ ಬೇಕು. ಮೊದಲನೆ ಯದು ಕಲಿಯಬೇಕು, ತಿಳಿಯಬೇಕು ಎಂಬ ತೀವ್ರ ಹಂಬಲ, ಎರಡನೆಯದು ಅನವರತ ಶ್ರದ್ಧೆಯಿಂದ ಕೂಡಿದ ಪರಿಶ್ರಮ ಮತ್ತು ಮೂರನೆಯದೇ ಅವಕಾಶ. ಬಹುಭಾಷೆಗಳನ್ನು ಕಲಿಯುವ ತೀವ್ರ ಹಂಬಲ ಬಾಲ್ಯದಿಂದಲೇ ನನ್ನಲ್ಲಿ ಬಲವಾಗಿ ಬೇರೂರಿತ್ತು. ಮುಂದೆ ಅವಕಾಶ ದೊರೆತುದರಿಂದ ಶ್ರದ್ಧೆಯಿಂದ ಪ್ರಯತ್ನಿಸಿದೆ–ಸಾಧ್ಯವಾಯಿತು’ ಎಂದು.

ಡಾ. ಸ್ರುಜ್ ಮುನ್ನೂರು ಭಾಷೆಗಳನ್ನು ಚೆನ್ನಾಗಿ ಕಲಿಯಬಲ್ಲರು! ನಾವು ಒಂದು ಭಾಷೆಯನ್ನು ಚೆನ್ನಾಗಿ ಕಲಿಯಲಾರೆವು. ಏಕೆ? ನಮಗೆ ಹಂಬಲದ ಬೆಂಬಲವಿಲ್ಲ. ಕಲಿಯಬೇಕೆಂಬ ಆಸೆ ಇಲ್ಲ. ಅಬ್ರಹಾಂ ಲಿಂಕನ್ ಲಾಯರ್ ಆಗಲು ತೀವ್ರವಾಗಿ ಹಂಬಲಿಸಿದ್ದ. ಬ್ಲ್ಯಾಕ್ ಸ್ಟೋನ್ ಅವರ ಗ್ರಂಥಗಳನ್ನು ಪಡೆಯಲು ಒಮ್ಮೆ ನಲ್ವತ್ತು ಮೈಲಿ ದೂರ ನಡೆದು ಹೋಗಿದ್ದ ನಂತೆ!

ತೀವ್ರ ಹಂಬಲ, ತೀವ್ರ ಅಭೀಪ್ಸೆಗಳೇ ಗುರಿ ಸಾಧಿಸಲು ಯೋಗ್ಯವಾದ ವಾತಾವರಣದೆಡೆಗೆ ನಿಮ್ಮನ್ನು ಸೆಳೆಯುವುವು! ಆಧ್ಯಾತ್ಮಿಕ ಜಗತ್ತಿನಲ್ಲೂ ತೀವ್ರ ಹಂಬಲ, ತೀವ್ರ ವ್ಯಾಕುಲತೆ ಇತ್ತು ಎಂದಾದರೆ ದೇವರನ್ನು ಪಡೆಯಬಹುದು. ‘ತೀವ್ರ ವ್ಯಾಕುಲತೆ ಇತ್ತು ಎಂದರೆ ಅರುಣೋದಯ ವಾದ ಹಾಗೆ. ಇನ್ನು ಸೂರ್ಯೋದಯಕ್ಕೆ ತಡವಿಲ್ಲ’ ಎಂದು ಭಗವಾನ್ ಶ‍್ರೀರಾಮಕೃಷ್ಣರು ಹೇಳಿ ದರು. ನೋಡಿದಿರಾ ಈ ಹಂಬಲದ ಮಹಿಮೆ!

‘ಯಾವುದಾದರೊಂದು ವಸ್ತುವನ್ನು ಕುರಿತು ಅರಿತುಕೊಳ್ಳಬೇಕೆಂಬ ತೀವ್ರ ಹಂಬಲ ಮಾನವನ ಹೃದಯದಲ್ಲಿ ಪ್ರಬಲವಾಗಿ ಉದಯಿಸದಿದ್ದರೆ, ವ್ಯಾಕುಲತೆ ಉಂಟಾಗದಿದ್ದರೆ ಅವನ ಕಣ್ಣುಗಳ ಎದುರಿಗೆ ಅಂಧಕಾರವೇ ಇರುವುದು. ಆ ವಸ್ತು ಅವನಿಗೆ ಗೋಚರಿಸುವುದೇ ಇಲ್ಲ!’

‘ಸತ್ಯದ ಮಹಾ ಸಾಗರದಲ್ಲಿ ಮನುಷ್ಯನು ಏನನ್ನು ಬಯಸುತ್ತಾನೋ ಅದನ್ನು ಪಡೆಯುತ್ತಾನೆ. ದೇವರಿಗಾಗಿ ಕಾತರಿಸಿದ ಅಸ್ಸಿಸಿಯ ಸಂತ ಫ್ರಾನ್ಸಿಸ್ ದೈವ ಸಾಕ್ಷಾತ್ಕಾರ ಪಡೆದ. ಸೃಷ್ಟಿಯ ಹಿನ್ನೆಲೆಯ ನಿಯಮಗಳನ್ನು ಹುಡುಕಾಡಿದ ಐನ್​ಸ್ಟೀನ್ ವಿಶ್ವವ್ಯಾಪಕ ನಿಯಮಗಳನ್ನು ಕಂಡು ಹಿಡಿದ. ರುಯ್ಸ್​ಬ್ರೋಕ್ ಮಹಾನುಭಾವನೆನ್ನುವಂತೆ ದೇಶಕಾಲಾತೀತನಾಗಿ ಪ್ರಜ್ಞೆಯ ಅನಿರ್ ವಚನೀಯ ರಾಜ್ಯದಲ್ಲಿ ವಿಜೃಂಭಿಸುವ ಆ ದೇವರೂ, ಪ್ರೀತಿಯ ವ್ಯಾಕುಲ ಮೊರೆಗೆ ಮಾತ್ರ ಲಭ್ಯನು’ ಎಂಬುದು ನೊಬೆಲ್ ಪಾರಿತೋಷಕ ವಿಜೇತ ವಿಜ್ಞಾನಿ ಡಾ. ಅಲೆಕ್ಸಿಸ್ ಕೆರೆಲ್​ ಹೇಳಿದ ಮಾತು!\footnote{\engfoot{In the shoreless ocean of reality, man finds only what he seeks. Saint Francis of Assisi found God, Einstein, the Laws of the Cosmos. God can only be encountered outside the dimensions of space and time, behind the intellect, in that indefinable realm, which according to Ruysbrock the admirable can only be penetrated by love and longing.}\hfill\engfoot{ –Dr. Alexis Carrel, `\textit{Reflections of Life,}’ Jaico}}

ಅಂದರೆ ಪ್ರಯತ್ನದಿಂದ ಪರಮಾರ್ಥ ಸಾಧ್ಯ ಎಂದಾಯಿತಲ್ಲವೇ?

ನಮ್ಮ ದೇಶದ ದೇವಾಲಯಗಳಲ್ಲಿ ದೇವರ ವಿಭಿನ್ನ ಮೂರ್ತಿಗಳನ್ನು ನೀವು ನೋಡಿರಬಹುದು. ಆ ಮೂರ್ತಿಗಳು ತೋರಿಸುವ ಮುದ್ರೆಗಳನ್ನು ನೀವು ಎಂದಾದರೂ ಪರಿಶೀಲಿಸಿದ್ದೀರಾ? ಅವು ಬಹಳ ಅರ್ಥಪೂರ್ಣವಾಗಿವೆ. ಆ ದೇವತಾ ಮೂರ್ತಿಗಳು ಅಭಯ, ವರದಮುದ್ರೆಗಳ ಮೂಲಕ ಜೀವಿಗೆ ‘ಹೆದರದಿರು, ನಿನ್ನ ಆಸೆ ನೆರವೇರುತ್ತದೆ’ ಎಂಬುದನ್ನು ಸದಾ ಸೂಚಿಸುತ್ತವೆ. ಆಸೆ ಇತ್ತು ಎಂದಾದರೆ ಅದು ಎಂದಾದರೊಂದು ದಿನ ನೆರವೇರಲೇಬೇಕು. ಸರ್ವಶಕ್ತನೂ, ಸರ್ವಜ್ಞನೂ ಆದ ದೇವರು ಜಗತ್ತನ್ನು ಏಕೆ ಸೃಷ್ಟಿಸಿದ ಎನ್ನುವ ಪ್ರಶ್ನೆಗೆ ಜೀವಿಗಳ ಆಶೋತ್ತರ\-ಗಳನ್ನು ಪೂರೈಸಲು, ಅವರವರ ಕರ್ಮಫಲಗಳನ್ನು ಅನುಭವಿಸಲು ತಕ್ಕುದಾದ ವ್ಯವಸ್ಥೆಗಾಗಿ ಎನ್ನುತ್ತದೆ ಶಾಸ್ತ್ರ. ನಿಮ್ಮ ಆಸೆಯು ಇಂದೋ ನಾಳೆಯೋ, ಮುಂದೆ ಎಂದಾದರೊಂದು ದಿನ ನೆರವೇರಿಯೇ ತೀರುತ್ತದೆ. ಎಲ್ಲ ಆಸೆಗಳಿಂದ ಬಿಡುಗಡೆ ಹೊಂದಿ ದಿವ್ಯಸ್ಥಿತಿಗೇರಲೂ, ಮುಳ್ಳಿನಿಂದ ಮುಳ್ಳನ್ನು ತೆಗೆದಂತೆ ಈ ತೀವ್ರ ತೆರನಾದ ಆಸೆಯೇ ನಮಗೆ ಆಶ್ರಯ. ಹಾಗಾದರೆ ದುರಾಸೆಗಳೂ ನೆರವೇರುತ್ತವೆಯೇ ಎಂದು ಕೇಳಬಹುದು ನೀವು. ಹೌದು, ನೆರವೇರುತ್ತವೆ. ಆದರೆ ಪರಿಣಾಮ ಮಾತ್ರ ಭಯಾನಕ. ರಾವಣ ದುರ್ಯೋಧನರುಗಳ ದುರಾಸೆ ನೆರವೇರಿದಂತೆ ಕಂಡರೂ ಅದು ದುರಂತ ಮತ್ತು ಸರ್ವನಾಶದೆಡೆಗೇ ಅವರನ್ನು ಸೆಳೆಯಿತು.

ನಿಮಗೇನು ಬೇಕೋ ಅದು ದೊರತೇ ದೊರೆಯಬೇಕು. ಆದರೆ ಏನು ಬೇಕೆಂಬುದೇ ತಿಳಿಯದಿದ್ದರೆ ಪರಿಣಾಮ ಶೂನ್ಯಸಂಪಾದನೆಯಾದೀತು. ಇದು ಪ್ರಸಿದ್ಧ ಸಂತ ಅಲ್ಲಮಪ್ರಭುವಿನ ಶೂನ್ಯಸಂಪಾದನೆಯಲ್ಲ–ಅಲ್ಲಮನದು ಸಿದ್ಧಿ ಸಂಪಾದನೆ. ಗೊತ್ತು ಗುರಿ ಇಲ್ಲದ ಬದುಕಿನಿಂದ ನಾವು ಸಂಪಾದಿಸುವುದು ಇಷ್ಟೇ: ನಿರಾಶಾಮನೋಭಾವ, ದೈನ್ಯ, ಕೀಳರಿಮೆ ಮತ್ತು ದುಃಖ, ಒಟ್ಟಿನಲ್ಲಿ ದುರ್ಬಲ ವ್ಯಕ್ತಿತ್ವ, ದುರಂತ ಜೀವನ.


\section*{ಮುನ್ನಡೆಯ ಮೂಲಮಂತ್ರ}

\addsectiontoTOC{ಮುನ್ನಡೆಯ ಮೂಲ\-ಮಂತ್ರ}

ಹಾಗಾದರೆ ಬದುಕಿನಲ್ಲಿ ಸದ್ಯ ನಿಂತ ಸ್ಥಾನದಿಂದ ಉನ್ನತಿಯ ಪಥದಲ್ಲಿ ಮುನ್ನಡೆಯಬೇಕೆನ್ನು\-ವವರಿಗೆ ಯಾವ ಸಾಮಗ್ರಿಗಳು ಬೇಕು?

ಮುನ್ನಡೆಯಬೇಕು, ಮೇಲೇರಬೇಕು, ಸರ್ವತೋಮುಖವಾದ ಪ್ರಗತಿಯಾಗಬೇಕು ಎನ್ನುವ ಪ್ರಬಲ ಆಸೆ ಬೇಕು.

ಯಾವ ದಿಕ್ಕಿನಲ್ಲಿ ಮುನ್ನಡೆಯಬೇಕೆಂಬುದನ್ನು ಸ್ಪಷ್ಟವಾಗಿ ನಿರ್ದಿಷ್ಟ ಪಡಿಸಿಕೊಳ್ಳಬೇಕು. ಎಂದರೆ ನಿಮ್ಮ ಗುರಿಯನ್ನು ಕುರಿತು ಸ್ಪಷ್ಟ ಅರಿವು ಬೇಕು.

ನಿಮಗೇನು ಬೇಕು ಎಂಬುದನ್ನು ಸ್ಥಿರಚಿತ್ತರಾಗಿ ಯೋಚಿಸಿ, ಅವು ವಾಸ್ತವವೂ, ಸಾಧ್ಯವೂ ಆಗುವ ಬೇಕುಗಳಾಗಲಿ. ಜಗುಲಿ ಹಾರದೇ ಗಗನ ಹಾರುವ ಕಲ್ಪನೆಯ ಗಾಳಿಗೋಪುರದ ಬೇಕುಗಳಾಗದಿರಲಿ.

ನಿಮ್ಮ ಮುಖ್ಯ ‘ಬೇಕು’ವಿಗೆ ಪೂರಕವಾದ ಪುಟ್ಟ ‘ಬೇಕು’ಗಳ ಒಂದು ಪಟ್ಟಿಯನ್ನು ಮಾಡಿ. ಅವುಗಳು ನಿಮಗೇಕೆ ಬೇಕು ಎಂಬುದಕ್ಕೆ ಕಾರಣಕೊಡಿ.

ನೀವು ಒಂಭತ್ತನೇ ತರಗತಿಯಲ್ಲಿ ಓದುತ್ತಿರುವಾಗ ವಿಮಾನ ಚಾಲಕನಾಗಬೇಕೆಂದು ನಿಮ್ಮ ನೆಚ್ಚಿನ ಮುಖ್ಯ ‘ಬೇಕು’ವಿನ ಬಗೆಗೆ ನಿರ್ಧರಿಸಿಕೊಂಡಿರಬಹುದು. ಆ ಬೇಕು ಅಸಾಧ್ಯವೇನಲ್ಲ. ಆದರೆ ಆ ‘ಮುಖ್ಯ ಬೇಕು’ ಪೂರೈಸಬೇಕಾದರೆ ಇತರ ‘ಪುಟ್ಟ ಬೇಕುಗಳ’ ಪೂರಣವಾಗುವುದು ಉಚಿತ. ದೃಷ್ಟಿಪಾಟವ, ಆರೋಗ್ಯವಂತ ಶರೀರ, ಶಾಂತ ನಿರ್ಭೀತ ಮನಸ್ಸು, ಇವುಗಳ ಜೊತೆಗೆ ವರ್ಷಾಂತ್ಯದ ಪರೀಕ್ಷೆಯಲ್ಲಿ ಅತ್ಯುತ್ತಮ ರೀತಿಯಲ್ಲಿ ಪಾಸಾಗುವುದು–ಇವುಗಳ ಪೂರೈಕೆಯಾಗದೆ ಮುಖ್ಯ ‘ಬೇಕು’ ದೊರೆಯುವುದೇ?

ನಿಮ್ಮ ‘ಬೇಕು’ಗಳನ್ನು ನಿಶ್ಚಿತವಾಗಿರಿಸಿಕೊಳ್ಳಿ. ಅದನ್ನು ಪಡೆಯಲು ಕೂಡಲೇ ಕಾರ್ಯಮಗ್ನ ರಾಗಿ. ನೀವು ಆ ದಿಕ್ಕಿನಲ್ಲಿ ನಡೆದು ಜಯ ಗಳಿಸಿದ ಜನರ ಜೀವನವನ್ನು ಅಧ್ಯಯನ ಮಾಡಬೇಕು. ಅವರು ಪಟ್ಟ ಶ್ರಮ, ಎದುರಿಸಿದ ಕಷ್ಟಗಳೆಂಥವು ಎಂಬುದನ್ನು ಮನಸ್ಸಿಗೆ ತಂದುಕೊಳ್ಳಬೇಕು. ಅಂಥ ವ್ಯಕ್ತಿಗಳ ಸಂಪರ್ಕ ಸಾಧ್ಯವಿದ್ದರೆ, ಅವರನ್ನು ಕಂಡು ಮಾತನಾಡಿಸಿ ನಿಮ್ಮ ಸಮಸ್ಯೆ\-ಗಳಿಗೆ ಅವರು ಸೂಚಿಸುವ ಪರಿಹಾರವೇನೆಂಬುದನ್ನು ತಿಳಿದುಕೊಳ್ಳಬೇಕು.

ಮೋಸ, ವಂಚನೆ, ಅಡ್ಡದಾರಿಗಳಿಂದ ಯಾವ ಮಹತ್ವದ ಒಳ್ಳೆಯ ಕೆಲಸವೂ ಸಾಧಿತ\-ವಾಗುವು\-ದಿಲ್ಲ\-ವೆಂಬುದನ್ನು ಎಂದಿಗೂ ಮರೆಯದಿರಬೇಕು.

ಸೋಲಿನಿಂದ ಧೃತಿಗೆಡದೇ, ಸೋಲಿನ ಕಾರಣವನ್ನು ಕಂಡುಹಿಡಿದು ತಾಳ್ಮೆಯಿಂದ\break ಪ್ರಾಮಾಣಿಕ ಪ್ರಯತ್ನವನ್ನು ಮುಂದುವರಿಸಬೇಕು.

\newpage


\section*{ಗುರಿ ಮತ್ತು ದಾರಿ}

\vskip -8pt\addsectiontoTOC{ಗುರಿ ಮತ್ತು ದಾರಿ}

ಸ್ವಾಮಿ ವಿವೇಕಾನಂದರೆಂದರು: ‘ಗುರಿಯಷ್ಟೇ, ಅದನ್ನು ಹೊಂದುವ ಮಾರ್ಗದ ಕಡೆಗೂ ಗಮನ ಕೊಡಬೇಕೆಂಬುದು ನಾನು ಕಲಿತ ಅತ್ಯಂತ ಮಹತ್ವದ ಪಾಠಗಳಲ್ಲೊಂದು. ನಾನು ಯಾರಿಂದ ಈ ಪಾಠ ಕಲಿತೆನೋ ಆ ಮಹಾತ್ಮನ ಸ್ವಂತ ಜೀವನವೆ ಈ ಒಂದು ಪ್ರಮುಖ ತತ್ವಕ್ಕೆ ಒಂದು ಉದಾಹರಣೆಯಂತಿತ್ತು. ಆ ಒಂದು ತತ್ವದಿಂದಲೇ ನಾನು ಅನೇಕ ಪಾಠಗಳನ್ನು ಕಲಿಯು ತ್ತಿದ್ದೇನೆ. ವಿಜಯದ ಎಲ್ಲ ರಹಸ್ಯವೂ ಆ ತತ್ವದಲ್ಲಡಗಿದೆ ಎಂದು ನನಗೆ ಭಾಸವಾಗುತ್ತಿದೆ.

‘ಗುರಿಯ ಕಡೆಗೇ ನಮಗೆ ಹೆಚ್ಚು ಸೆಳೆತ. ಇದೇ ನಮ್ಮ ಜೀವನದ ಅತಿ ದೊಡ್ಡ ದೋಷ. ಗುರಿಯು ನಮಗೆ ಎಷ್ಟು ಹೆಚ್ಚು ಆಕರ್ಷಕವೂ, ಮೋಹಕವೂ, ಮನಸ್ಸಿನ ಪರಿಧಿಯಲ್ಲಿ\break ಬೃಹದಾಕಾರವಾಗಿಯೂ ಆಗುವುದೆಂದರೆ, ಸಣ್ಣ ಪುಟ್ಟ ವಿಷಯಗಳನ್ನು ಕುರಿತು ನಾವು ಸಂಪೂರ್ಣ ಅಲಕ್ಷ್ಯರಾಗುತ್ತೇವೆ’.

ಸೋಲು ಬಂದಾಗ ನಾವು ಅದನ್ನು ವಿಮರ್ಶಾತ್ಮಕವಾಗಿ ವಿಶ್ಲೇಷಿಸಿದರೆ ನೂರರಲ್ಲಿ ತೊಂಬತ್ತು ಬಾರಿಯೂ ನಾವು ಗುರಿಯನ್ನು ಹೊಂದುವ ಮಾರ್ಗದ ಕಡೆಗೆ ಗಮನವಿತ್ತಿಲ್ಲ ಎಂಬುದನ್ನು ಕಾಣುವೆವು. ಈ ಮಾರ್ಗವನ್ನು ಬಲಗೊಳಿಸುವ, ಪೂರ್ಣಗೊಳಿಸುವುದರ ಕಡೆಗೆ ನಮ್ಮ ಗಮನವಿರಬೇಕಾದುದು ಆವಶ್ಯಕ. ದಾರಿಯು ಸಮರ್ಪಕವಾಯಿತೆಂದರೆ ಗುರಿಯು ದೊರೆಯಲೇ ಬೇಕು. ಯಾವುದೇ ಪರಿಣಾಮ ತನ್ನಿಂದ ತಾನೇ ಉಂಟಾಗದು. ಕಾರಣವು ಸುಸ್ಪಷ್ಟವೂ, ನಿಶ್ಚಿತವೂ, ಬಲಯುತವೂ ಆಗಿದ್ದರೆ ಮಾತ್ರ ಅದರಿಂದ ಪರಿಣಾಮ ಉಂಟಾಗುವುದು. ಒಮ್ಮೆ ಆದರ್ಶ ಅಥವಾ ಗುರಿಯನ್ನು ಹುಡುಕಿ ಅದು ಎಟಕುವ ಮಾರ್ಗವನ್ನು ದೃಢಪಡಿಸಿಕೊಂಡೆವೆಂದರೆ, ನಂತರ ಆ ಮಾರ್ಗದಲ್ಲಿ ತದೇಕ ನಿಷ್ಠೆಯಿಂದ ನಿಶ್ಚಿಂತರಾಗಿ ಮುನ್ನಡೆಯುವುದಷ್ಟೇ ನಮ್ಮ ಕರ್ತವ್ಯವಾಗಬೇಕು. ನಿಧಾನವಾದರೂ ಸರಿ, ಸಮರ್ಪಕ ದಾರಿಯಲ್ಲಿಯೇ ಹೆಜ್ಜೆಯಿಟ್ಟು ಮುನ್ನಡೆ ಯುವುದರಿಂದ ಗುರಿ ಸೇರುವುದಂತೂ ಖಂಡಿತ.

ಪಶ್ಚಿಮದ ಪ್ರಸಿದ್ಧ ಸಂಗೀತಗಾರ ಬಿಥೋವನ್ ಒಮ್ಮೆ ಗಾನಗೋಷ್ಠಿಯನ್ನು ಮುಗಿಸಿ ಮೇಲೆ ದ್ದಾಗ ಅವನ ಅದ್ಭುತ ಕಲಾನೈಪುಣ್ಯವನ್ನು ಕಂಡು ಕೇಳಿ ಚಕಿತರಾದ ಜನರೂ, ಸ್ನೇಹಿತರೂ, ಅಭಿಮಾನಿಗಳೂ ಅವನನ್ನು ಸುತ್ತುವರಿದರು. ಅವರೆಲ್ಲರೂ ಮೂಕವಿಸ್ಮಿತರಾಗಿ ಅವನನ್ನೇ ಬೆರಗು\-ಗಣ್ಣುಗಳಿಂದ ನೋಡುತ್ತಿದ್ದರು. ಪ್ರಶಂಸೆಯ ಮಾತುಗಳನ್ನು ಹೇಳಲಾರದಷ್ಟು ಭಾವ ತನ್ಮಯ\-ರಾಗಿದ್ದರು ಅವರೆಲ್ಲ. ತನ್ನ ನಾದಮಾಧುರ್ಯದಿಂದ ಅಷ್ಟೊಂದು ಜನರ ಹೃನ್ಮನಗಳನ್ನು ಸೆಳೆದಿದ್ದ\-ನಾತ. ಆ ಗಂಭೀರ ಮೌನವನ್ನು ಉತ್ಸಾಹಿ ಮಹಿಳೆಯೊಬ್ಬಳು ತನ್ನ ಮಾತಿನಿಂದ ಭೇದಿಸಿದಳು: ‘ಮಹಾಶಯ, ದೇವರು ನನಗೆ ಈ ಅಸಾಧಾರಣ ಪ್ರತಿಭೆಯ ಕೊಡುಗೆಯನ್ನು ಕೊಟ್ಟಿದ್ದರೆ...’ ಎಂದವಳು ಉದ್ಗರಿಸಿದಳು.

ಬಿಥೋವನ್ ಹೀಗೆಂದ–‘ನನ್ನಲ್ಲಿರುವುದೆಂದು ನೀನಂದುಕೊಂಡಿರುವ ಈ ಅಸಾಧಾರಣ\break ಪ್ರತಿಭೆ ದೇವರ ಕೊಡುಗೆ ಅನ್ನುತ್ತೀಯಾ? ನೀನೂ ಈ ಕೊಡುಗೆಯನ್ನು ಪಡೆಯಬಹುದು. ಏನು ಮಾಡಬೇಕು ಗೊತ್ತೆ? ದಿನವೂ ಎಂಟು ಗಂಟೆಗಳ ಕಾಲ ನಲ್ವತ್ತು ವರ್ಷಗಳವರೆಗೆ ಬಿಡದೇ ಪಿಯಾನೊ ಅಭ್ಯಾಸ ಮಾಡು. ಅಷ್ಟೇ ಸಾಕು, ನೀನೂ ನನ್ನಂತೆಯೇ ಅದ್ಭುತ ಕೊಡುಗೆಯನ್ನು ಪಡೆಯುತ್ತಿ.’

ಈ ಅನುಭವದ ಮಾತಿನ ಹಿನ್ನೆಲೆಯಲ್ಲಿ ದಾರಿಯು ಸರಿಯಾಯಿತೆಂದರೆ ಗುರಿಯು ದೊರೆತೇ ದೊರೆಯುತ್ತದೆನ್ನುವ ಅಮರತತ್ವ ಅಡಗಿದೆಯಲ್ಲವೇ?

ವಾಗ್ಗೇಯಕಾರ ಶ‍್ರೀ ವಾಸುದೇವಾಚಾರ್ಯರು ತಮ್ಮ ‘ನೆನಪುಗಳು’ ಎನ್ನುವ ಪುಸ್ತಕದಲ್ಲಿ ಪಿಟೀಲು ವಾದನದಲ್ಲಿ ಅದ್ವಿತೀಯರೆನಿಸಿಕೊಂಡ ಒಬ್ಬ ವಿದ್ವಾಂಸರನ್ನು ಸ್ಮರಿಸುತ್ತಾರೆ. ಅವರ ಹೆಸರು ಕೃಷ್ಣಯ್ಯರ್. ವಾದ್ಯ ಬಾರಿಸುವುದರಲ್ಲಿ ಅವರಿಗೆ ಎಂಥ ಸ್ವಾಮಿತ್ವವಿತ್ತೆನ್ನುವುದಕ್ಕೆ ಆಚಾರ್ಯರು ತಾವು ಕಂಡ ಒಂದು ನಿದರ್ಶನವನ್ನಿತ್ತಿದ್ದಾರೆ. ಒಂದು ಕಛೇರಿಯಲ್ಲಿ ಕೃಷ್ಣಯ್ಯರ್ ಪಕ್ಕವಾದ್ಯ ನುಡಿಸುತ್ತಿದ್ದಾಗ ಅಕಸ್ಮಾತ್ತಾಗಿ ಪಿಟೀಲಿನ ತಂತಿ ಕಿತ್ತುಹೋಯಿತು. ತಾನು ಹಾಡಿದ ಸಂಗೀತವನ್ನು ನುಡಿಸಲು ಸಾಧ್ಯವಾಗದೇ ಬೇಕೆಂತಲೇ ತಂತಿಯನ್ನು ಕಿತ್ತು ಆಟ ಹೂಡಿದ್ದಾರೆ ಎಂಬರ್ಥದ ವ್ಯಂಗ್ಯ ನಗುವನ್ನು ಬೀರಿದ ಗಾಯಕ. ಕೃಷ್ಣಯ್ಯರ್ ಕೋಪದಿಂದ ‘ನಿಮ್ಮ ಸಂಗೀತಕ್ಕೆ ಪಕ್ಕವಾದ್ಯ ನುಡಿಸಲು ನಾಲ್ಕು ತಂತಿಗಳು ಬೇಕೆ? ಇದೊ ಇನ್ನೂ ಎರಡು ತಂತಿಯನ್ನು ಕಿತ್ತೆಸೆಯುತ್ತೇನೆ’ ಎಂದು ಒಂದೇ ತಂತಿಯಲ್ಲಿ ಕಛೇರಿಯನ್ನು ನಿರ್ವಹಿಸಿದರಂತೆ!

ಆ ಅದ್ಭುತ ಕಾರ್ಯದ ಹಿನ್ನೆಲೆಯಲ್ಲಿ ಎಷ್ಟೊಂದು ದೀರ್ಘಕಾಲದ, ನಿಷ್ಠೆಯಿಂದ ಕೂಡಿದ ನಿರಂತರ ಪರಿಶ್ರಮದ ಅಭ್ಯಾಸವಿದೆ ಎಂಬುದನ್ನು ಯಾರೂ ಊಹಿಸಿಕೊಳ್ಳಬಹುದು.

ದೊಡ್ಡವರು ಬದುಕನ್ನು ಪ್ರಾರಂಭಿಸುವಾಗಲೇ ದೊಡ್ಡವರಾಗಿರಲಿಲ್ಲ ಅಥವಾ ಮಹಾ\break ಮೇಧಾವಿಗಳಾಗಿರಲಿಲ್ಲ. ಆದರೆ ಕಲಿಯುತ್ತ, ಕಲಿಯುತ್ತ, ಕಲಿಯುತ್ತಲೇ ತಮ್ಮ ಮೇಲ್ಮೆಯನ್ನು ಸಾಧಿಸಿದರು. ದಾರಿಯು ಸರಿಯಾಯಿತೆಂದರೆ ಗುರಿಯು ದೊರೆಯುವುದು ಎನ್ನುವುದಕ್ಕೆ ಅವರ ಸಿದ್ಧಿ ಸಾಕ್ಷಿ ನೀಡುತ್ತದೆ. ಆದರೆ ನಾವು ಮಾಡುವುದೇನು? ಗುರಿಯನ್ನೇ ಮೆಲಕು ಹಾಕುತ್ತಾ, ಎಲ್ಲೆಂದರಲ್ಲಿ ಹೇಗೆಂದರೆ ಹಾಗೆ ಮುನ್ನಡೆದು ಇರುವ ಶಕ್ತಿಯನ್ನೆಲ್ಲ ವ್ಯಯಿಸಿಕೊಂಡು, ಕೊನೆಗೆ ದಾರಿ ಕಾಣದಾಗಿ, ಅಳುಮೋರೆ ಹಾಕಿಕೊಂಡು ಕೈಗೆಟುಕದ ದ್ರಾಕ್ಷಿ ಹಣ್ಣು ಹುಳಿ ಎಂದ ನರಿಯ ತಂತ್ರದಿಂದ, ಆ ಗುರಿಯನ್ನೇ ಹಳಿದು ಹಂಗಿಸುವುದು. ಇನ್ನಾದರೂ ನಾವು ಸರಿದಾರಿ ತುಳಿಯಲು ಯತ್ನಿಸೋಣ.


\section*{ಗುಂಡಣ್ಣ ಕಾಲೇಜಿಗೆ ಹೋದ}

\addsectiontoTOC{ಗುಂಡಣ್ಣ ಕಾಲೇಜಿಗೆ ಹೋದ}

ಗುಂಡಣ್ಣ ಹಳ್ಳಿಯಿಂದ ಕಾಲೇಜು ವಿದ್ಯಾಭ್ಯಾಸಕ್ಕಾಗಿ ಪಟ್ಟಣ ಸೇರಿದ. ಸಾಮಾನ್ಯ ಆರ್ಥಿಕ ಶಕ್ತಿಯುಳ್ಳ ಕುಟುಂಬದಿಂದ ಬಂದವನಾತ. ಅವನ ಮನೆಯವರಿಗೆ ಕಾಲೇಜು ಶಿಕ್ಷಣದ ವೆಚ್ಚವನ್ನು ನಿರ್ವಹಿಸುವ ಸಾಮರ್ಥ್ಯವಿರಲಿಲ್ಲ. ಆದರೆ ಅವನ ಒತ್ತಾಯಕ್ಕೂ ಅಧ್ಯಾಪಕರ ಪ್ರೋತ್ಸಾಹಕ್ಕೂ ಮಣಿದು, ಹುಡುಗನಿಗೆ ಭವ್ಯ ಭವಿಷ್ಯವಿದೆ ಎಂದೆಣಿಸಿ ತಂದೆ ಸಾಲ ಮಾಡಿ ಮಗನನ್ನು\break ಪದವೀಧರನನ್ನಾಗಿ ಮಾಡಲು ನಿಶ್ಚಯಿಸಿದರು. ತಾನು ಹುಟ್ಟಿ ಬೆಳದು ಓಡಾಡಿದ ಊರನ್ನು ಬಿಟ್ಟು ಪಟ್ಟಣಕ್ಕೆ ಹೊರಟು ನಿಂತಾಗ ಅವನ ತಾಯಿ ಕಂಬನಿದುಂಬಿ ಹೇಳಿದ್ದರು, ‘ಮಗೂ, ಚೆನ್ನಾಗಿ ಕಲಿಯಬೇಕು. ನಿನ್ನ ತಂದೆ ಎಷ್ಟು ಕಷ್ಟಪಡುತ್ತಿದ್ದಾರೆಂದು ನೀನು ನೋಡಿದ್ದಿ. ನಿನ್ನ ತಂಗಿ ತಮ್ಮಂದಿರೂ ಕಲಿಯಬೇಕು, ನೆನಪಿರಲಿ. ವೇಳೆ ವ್ಯರ್ಥವಾಗದಂತೆ ನೋಡಿಕೋ.’ ಎಷ್ಟು ಕಳಕಳಿಯಿಂದ ಅವರು ಈ ಮಾತು ಹೇಳಿದರೆಂದರೆ ಆತ ಗದ್ಗದಿತನಾಗಿ ‘ಆಗಲಮ್ಮ’ ಎಂದಷ್ಟೇ ಉತ್ತರಿಸಿದ್ದ. ಹೊರಡುವಾಗ ಊರ ಹೊರಗಿನ ಆಂಜನೇಯಸ್ವಾಮಿಗೆ ನಮಿಸಿ ಪ್ರಾರ್ಥಿಸಿಯೂ ಇದ್ದ.

ಕಾಲೇಜು ಮತ್ತು ಹಾಸ್ಟೆಲುಗಳ ವಾತಾವರಣಕ್ಕೆ ಹೊಂದಿಕೊಳ್ಳಲು ಅವನಿಗೆ ಕೆಲವು ದಿನಗಳೇ ಬೇಕಾದವು. ನೂರಾರು ವಿದ್ಯಾರ್ಥಿಗಳು ತುಂಬಿದ ಕ್ಲಾಸುಗಳಲ್ಲಿ ಅವನೊಬ್ಬ ನಗಣ್ಯ ವ್ಯಕ್ತಿ. ಏನು? ಯಾರು? ಎತ್ತ? ಎಂದು ವಿಚಾರಿಸುವವರಿಲ್ಲ. ಯಾರ ಮಾತನ್ನೋ ಕೇಳಿ ತನಗೆ ಅಷ್ಟೊಂದು ಹಿಡಿಸದ ಅಭಿರುಚಿ ಇರದ ವಿಷಯವನ್ನೇ ಅವನು ಅಧ್ಯಯನಕ್ಕಾಗಿ ಆರಿಸಿಕೊಂಡ. ನಿತ್ಯ ನಿಯಮಿತ ರೀತಿಯಲ್ಲಿ ಚೆನ್ನಾಗಿ ಅಧ್ಯಯನ ಮಾಡಬೇಕೆಂದು ಖಂಡಿತವಾಗಿಯೂ ಅವನು ಸಂಕಲ್ಪಿಸಿದ್ದ. ಆದರೆ ಓದು ಅಷ್ಟೊಂದು ಚೆನ್ನಾಗಿ ಸಾಗಲಿಲ್ಲ. ಮುಂದೆ ಕ್ಲಾಸಿನ ಪಾಠಗಳು ಅರ್ಥವಾಗದೇ ಬೋರ್ ಹೊಡೆಯಲು ಪ್ರಾರಂಭವಾಯಿತು. ಮೆಲ್ಲಮೆಲ್ಲನೇ ಅವನ ಆತ್ಮ ವಿಶ್ವಾಸ ದುರ್ಬಲವಾಗುತ್ತಿತ್ತು. ಒಂದು ತೆರನಾದ ಅವ್ಯಕ್ತ ಭಯ ಅವನನ್ನು ಆವರಿಸುತ್ತಿತ್ತು. ತಾನೇನು ಮಾಡುತ್ತಿದ್ದೇನೆ ಎಂದು ತನಗೆ ತಾನೇ ಕೇಳಿಕೊಳ್ಳುತ್ತಿದ್ದ. ಅಧ್ಯಾಪಕರ ಮುಖ ಪರಿಚಯವಿದ್ದರೂ ಆತ್ಮೀಯತೆಯಿಂದ ತನ್ನ ಮನದಳಲನ್ನು ಹೇಳಿಕೊಳ್ಳುವಂತಿರಲಿಲ್ಲ. ಸ್ನೇಹಿತರೇ ಅವನ ಸಹಾಯಕ್ಕೆ ಬಂದರು. ‘ಏನೂ ಹೆದರಬೇಡ ಮಗು. ಪರೀಕ್ಷೆಗೆ ಮೊದಲು ಒಂದರೆಡು\break ತಿಂಗಳು ಪಟ್ಟಾಗಿ ಕುಳಿತರಾಯಿತು ಫಸ್ಟ್​ಕ್ಲಾಸ್​’ ಎಂದು ಧೈರ್ಯ ಕೊಟ್ಟರು. ಆತ ಕಾಲೇಜಿನ ಪಠ್ಯೇತರ ಚಟುವಟಿಕೆಗಳಲ್ಲಿ ಭಾಗವಹಿಸಿದ. ನಾಟಕ ಸಿನೆಮಾ ಎಂದು ಅಲೆದ. ಆಧುನಿಕ ಸಮಾಜದ ಕಣ್ಣು ಕೋರೈಸುವ ವಿಲಾಸ ಮತ್ತು ಆಡಂಬರದ ಜೀವನ, ನಗರದ ನಾನಾ ಆಕರ್ಷಣೆಗಳು ಅವನ ಮನಸ್ಸನ್ನು ಮುಗ್ಧಗೊಳಿಸುತ್ತಿದ್ದವು. ಅವನು ಕಲ್ಪನೆಯ ಲೋಕದಲ್ಲಿ ವಿಹರಿಸುತ್ತಿದ್ದ. ಸಮಯ ಸಿಕ್ಕಿದಾಗ ಇಸ್ಪೀಟ್ ಆಟಗಾರರ ಗುಂಪಿನಲ್ಲಿ ತನ್ನ ಪ್ರತಿಭೆಯನ್ನು ಮೆರೆಸಲು ಮರೆಯಲಿಲ್ಲ. ದಿನಗಳು ಕಳೆದವು. ಪರೀಕ್ಷೆ ಸಮೀಪಿಸಿತು. ಆತ ಈಗ ಎಚ್ಚರಗೊಂಡ. ಗುಳಿಗೆ ಸೇವಿಸಿ ನಿದ್ರೆ, ಆಹಾರಗಳ ಕಡೆಗೆ ಗಮನವೀಯದೆ ಓದತೊಡಗಿದ. ಸ್ಫೂರ್ತಿಗಾಗಿ ಸಿಗರೇಟಿನ ಹೊಗೆಯನ್ನು ಹಾರಿಸಿದ. ಪರೀಕ್ಷೆ ಸಮೀಪಿಸಿದಂತೆಲ್ಲ ಅವನ ಉದ್ವೇಗಕ್ಕೆ ತುದಿ ಮೊದಲಿಲ್ಲ. ಪುಸ್ತಕಗಳು ಯಮಭಾರವಾಗಿ ಕಾಣತೊಡಗಿದವು. ಈ ಬಾರಿ ಪರೀಕ್ಷೆಗೆ ಕುಳಿತುಕೊಳ್ಳದೇ, ಇನ್ನೊಮ್ಮೆ ಚೆನ್ನಾಗಿ ಅಧ್ಯಯನ ಮಾಡಿ ಸೆಪ್ಟೆಂಬರದಲ್ಲಿ ಕುಳಿತುಕೊಳ್ಳೋಣವೆಂದು ಹಲವು ಬಾರಿ ಅವನಿಗೆ ಈ ಮಧ್ಯೆ ಅನ್ನಿಸಿದ್ದುಂಟು. ಕೊನೆಗೂ ಆರೋಗ್ಯ ಕೆಡಿಸಿಕೊಂಡು ಚಿಂತೆ ಹಚ್ಚಿ ಕೊಂಡು ಪರೀಕ್ಷೆಯಲ್ಲಿ ಏನೋ ಬರೆದು ಬಂದ. ಮನೆಯವರು ಅವನ ಕಳೆಗುಂದಿದ ಮುಖವನ್ನು ಕಂಡು ಬಹಳ ಕಷ್ಟಪಟ್ಟು ಓದಿ ಸುಸ್ತಾಗಿರಬೇಕೆಂದು ಆರೈಕೆ ಮಾಡಿದರು. ಪರೀಕ್ಷೆಯ ಫಲಿತಾಂಶ ಹೊರಬಿತ್ತು. ಅವನು ಪಾಸಾಗಿರಲಿಲ್ಲ. ಆಗ ವಿಧಿಯನ್ನು ಶಪಿಸತೊಡಗಿದ. ಗ್ರಹ ಚಾರ ಎಂದ. ಕಾಲೇಜಿನ ಅಧ್ಯಾಪಕರ ಹೊಣೆಗೇಡಿತನವೇ ತನ್ನ ಈ ಸ್ಥಿತಿಗೆ ಕಾರಣವೆಂದ. ಕೆಲವೊಮ್ಮೆ ಅವನ ಮನಸ್ಸಿಗೆ ಎಷ್ಟು ದುಃಖವಾಗಿತ್ತೆಂದರೆ ರೈಲಿನ ಕಂಬಿಗೆ ತಲೆಕೊಟ್ಟು ಇಹ ಲೋಕದ ಯಾತ್ರೆಗೆ ಮುಕ್ತಾಯ ಹೇಳೋಣ ಎಂದು ಅನ್ನಿಸಿದ್ದುಂಟು.

ಗುಂಡಣ್ಣ ಗುರಿಯನ್ನೇನೋ ಇಟ್ಟುಕೊಂಡಿದ್ದ. ಶುಭ ಸಂಕಲ್ಪದಿಂದಲೇ ಅವನು ಕಾಲೇಜು ಸೇರಿದ್ದ. ಆದರೆ ಗುರಿಯನ್ನು ಹೊಂದುವ ಮಾರ್ಗದ ಕಡೆಗೆ ಸಾಕಷ್ಟು ಗಮನವೀಯಲಿಲ್ಲ. ತನ್ನ ಅಭಿರುಚಿಯ ವಿಷಯಗಳನ್ನು ಅಧ್ಯಯನ ಮಾಡಲು ಆರಿಸಿಕೊಳ್ಳಲಿಲ್ಲ. ಯೋಗ್ಯ ಮಿತ್ರರ ಸಹಕಾರ ಪಡೆಯಲು ಸಮರ್ಥನಾಗಲಿಲ್ಲ. ದಿನದಿನವೂ ಹಠಹಿಡಿದು ಏಕಾಗ್ರತೆಯಿಂದ\break ಅಧ್ಯಯನ ಮಾಡಲಿಲ್ಲ. ನಿಶ್ಚಿತ ನಿಯಮಗಳನ್ನನುಸರಿಸಿ ಶ್ರಮಪಡಲಿಲ್ಲ. ಆತ ಮಾರ್ಗದ ಕಡೆಗೆ ಗಮನವೇ ಜೀವನದ ಒಂದು ಪರಮ ರಹಸ್ಯ ಎಂಬುದನ್ನು ಮರೆತಿದ್ದ.


\section*{ಛಲದಿಂದ ಬಲ}

\addsectiontoTOC{ಛಲದಿಂದ ಬಲ}

ಹೈಸ್ಕೂಲು ವಿದ್ಯಾಭ್ಯಾಸವನ್ನು ಮುಗಿಸಿದ ಮಿತ್ರರೊಬ್ಬರು ಒಂದು ಸಂಸ್ಥೆಯಲ್ಲಿ ದುಡಿಯು ತ್ತಿದ್ದರು. ದಿನದ ಎಂಟು ಗಂಟೆಗಳ ಕೆಲಸದೊಂದಿಗೆ ಸಂಸಾರ ನಿರ್ವಹಣೆಯ ಭಾರವು ಅವರ ಮೇಲಿತ್ತು. ಖಾಸಗಿಯಾಗಿಯೇ ಎಂ.\ ಎ.\ ಪರೀಕ್ಷೆಯಲ್ಲಿ ಪಾಸಾದರು. ‘ಖಾಸಗಿಯವರೆಂದು ಪ್ರಥಮ ದರ್ಜೆ ಕೊಡಲಿಲ್ಲ. ಐವತ್ತೊಂಬತ್ತು ಪರ್ಸೆಂಟ್ ಬಂದಿದೆ’ ಎಂದರು. ಅವರೇನೂ ಹುಟ್ಟಿನಿಂದಲೇ ಮಹಾ ಮೇಧಾವಿಯಾಗಿರಲಿಲ್ಲ. ‘ಈ ನಾನಾ ಕಾರ್ಯಗಳ ಮಧ್ಯೆ ಅಷ್ಟೊಂದು ಅಧ್ಯಯನ ಹೇಗೆ ನಡೆಯಿಸಿದಿರಿ?’ ಎಂದು ಕೇಳಿದೆ. ಅವರ ಉತ್ತರ ಕುತೂಹಲಕಾರಿಯಾಗಿತ್ತು: ‘ಒಂದೊಂದೇ ಮೆಟ್ಟಲು ಮೇಲೇರುತ್ತ ಹೋದೆ. ಕಷ್ಟವಾಗಲಿಲ್ಲ’ ಎಂದರು. ದಿನವೂ ಎರಡು ಗಂಟೆಗಳ ಕಾಲ ನಿತ್ಯ ನಿಯಮಿತ ರೀತಿಯಿಂದ ಅಧ್ಯಯನ ಮಾಡಿದೆ ಎಂಬುದನ್ನು ಅವರು ಮೇಲಿನ ಉದಾಹರಣೆಯಿಂದ ತಿಳಿಸಿದರು.

ಯಾವುದೇ ಕ್ಷೇತ್ರದಲ್ಲಿ ಅತ್ಯುನ್ನತ ಸಿದ್ಧಿಯನ್ನು ಪಡೆದವರನ್ನು ಕಂಡಾಗ ಅವರು ಒಳ್ಳೆಯ ಆಟಗಾರರಿರಲಿ, ಸಂಗೀತತಜ್ಞರಿರಲಿ, ವಾಗ್ಮಿಗಳಿರಲಿ ಅವರಂತೆ ನಾವಾಗಲು ಸಾಧ್ಯವೇ ಇಲ್ಲ– ಅದೊಂದು ದೈವದತ್ತವಾದ ವರ ಎಂದು ನಾವು ತಿಳಿಯುವುದುಂಟು. ಅವರು ಆ ಮಟ್ಟವನ್ನು ಮುಟ್ಟುವುದಕ್ಕೆ ಮೊದಲು ವರ್ಷಗಳ ಕಾಲ ಅಭ್ಯಾಸ ಮಾಡಿದ್ದಾರೆ ಎಂಬುದನ್ನು ನಾವು ಮರೆಯು ತ್ತೇವೆ. ನಾವು ಸದ್ಯ ಮಾಡಲಸಾಧ್ಯವಾದುದನ್ನು ಅಷ್ಟು ವರ್ಷಗಳ ಪ್ರಯತ್ನದ ಬಲದಿಂದ ಅವರು ಸುಲಭವಾಗಿ ಮಾಡುವುದು ಸ್ವಾಭಾವಿಕವಲ್ಲವೇ? ಬಿಡದ ಛಲವೇ ಸಾಧನೆಯ ಚಾಲನ ಶಕ್ತಿ.

ಥಟ್ಟನೆ ಎಲ್ಲರನ್ನೂ ಚಕಿತಗೊಳಿಸುವಂಥ ಅದ್ಭುತವನ್ನು ಸಾಧಿಸುವ ಆತುರ ನಮ್ಮನ್ನು ಎಷ್ಟೋ ವೇಳೆ ಕಂಗೆಡಿಸುತ್ತದೆ, ಹತಾಶರನ್ನಾಗಿ ಮಾಡುತ್ತದೆ. ಸಾವಿರ ಮೈಲಿನ ಯಾತ್ರೆ ಒಂದು ಹೆಜ್ಜೆಯಿಂದ ಪ್ರಾರಂಭವಾಗುತ್ತದೆ. ಸಾವಿರಾರು ಪುಟಗಳ ಗ್ರಂಥ ಒಂದು ಅಕ್ಷರದಿಂದ ಪ್ರಾರಂಭವಾಗುತ್ತದೆ. ಎಲ್ಲ ದೊಡ್ಡ ಕೆಲಸಗಳೂ ಸಣ್ಣ ಕೆಲಸಗಳ ಮೊತ್ತ. ಸಣ್ಣ ಕೆಲಸವನ್ನು ಪೂರ್ಣ ಮನಸ್ಸು ಕೊಟ್ಟು ಚೆನ್ನಾಗಿ ಮಾಡಲು ಸಮರ್ಥನಾದರೆ ದೊಡ್ಡ ಕೆಲಸವೂ ಚೆನ್ನಾಗಿಯೇ ಯಶಸ್ವಿಯಾಗುವುದು.

ಸ್ವಾಮಿ ವಿವೇಕಾನಂದರು ಒಮ್ಮೆ ‘ನನಗೆ ವಯಸ್ಸಾಗುತ್ತ ಬಂದಂತೆಲ್ಲ ನಾನು ಸಂಧಿಸುವ ಹಿರಿಯರು, ಮಹಾತ್ಮರು ಮೊದಲಾದವರು ಸಣ್ಣ ಕೆಲಸವನ್ನು ಹೇಗೆ ಮಾಡುತ್ತಾರೆ ಎಂಬುದನ್ನು ಗಮನಿಸುತ್ತೇನೆ’ ಎಂದರು.

‘ನೀವು ವ್ಯಕ್ತಿಯೊಬ್ಬನ ಚಾರಿತ್ರ್ಯದ ಬಗೆಗೆ ತಿಳಿದುಕೊಂಡು “ಇವನು ಇಂಥವನು” ಎನ್ನುವ ತೀರ್ಮಾನಕ್ಕೆ ಬರಬೇಕಾದರೆ ಅವನು ಮಾಡಿದ ಯಾವುದೇ ದೊಡ್ಡ ಕಾರ್ಯವನ್ನು ಮಾತ್ರ ಪರಿಶೀಲಿಸಿದರೆ ಸಾಲದು. ಎಂಥ ಸಾಮಾನ್ಯನೂ ಒಂದಲ್ಲ ಒಂದು ಅವಕಾಶ ದೊರೆತಾಗ ಮಹಾ ವೀರನಂತೆ ವರ್ತಿಸಬಹುದು. ಮನುಷ್ಯರು ದಿನದಿನವೂ ಮಾಡುವ ಸಾಮಾನ್ಯವೆಂದು ತೋರುವ ಕೆಲಸಗಳನ್ನು ಪರಿಶೀಲಿಸಿ. ಅವು ನಿಜವಾದ ಉನ್ನತ ಚಾರಿತ್ರ್ಯವನ್ನು ಕುರಿತು ಸ್ಪಷ್ಟ ಚಿತ್ರವನ್ನು ನೀಡಬಲ್ಲವು, ದೊಡ್ಡದೊಂದು ಅವಕಾಶ ಅಲ್ಪನನ್ನು ದೊಡ್ಡವನನ್ನಾಗಿಸಬಹುದು, ಆದರೆ ನಿಜವಾದ ಮಹಾತ್ಮನು ಎಲ್ಲಾ ಸಂದರ್ಭಗಳಲ್ಲೂ–ಜನರು ಅವನನ್ನು ಪರಿಶೀಲಿಸಲಿ ಬಿಡಲಿ, ಒಂದೇ ತೆರನಾಗಿರುವನು’ ಎಂದು ಅವರು ಹೇಳಿದ್ದರು.

ಮಹಾತ್ಮಾ ಗಾಂಧೀಜಿಯವರ ಗುಣವೈಭವವನ್ನು ಮಕ್ಕಳಿಗಾಗಿ ಸರಳ ಸುಂದರ ರೀತಿಯಲ್ಲಿ ಅನು ಬಂದೋಪಾಧ್ಯಾಯರು ಬರೆದು ವಿವರಿಸಿದ್ದಾರೆ. ಪುಸ್ತಕದ ಹೆಸರು ‘ಬಹುರೂಪಿ ಗಾಂಧಿ.’ ಆಂಗ್ಲ ಭಾಷೆಯಲ್ಲಿರುವ ಆ ಪುಸ್ತಕವನ್ನು ಓದಿ ನಾನು ಮುಗ್ಧನಾದೆ. ಅದನ್ನೇಕೆ ಮಕ್ಕಳಿಗೆ ಉಪಪಠ್ಯವನ್ನಾಗಿಯಾದರೂ ಇಡಬಾರದು ಎಂದು ನನಗನ್ನಿಸಿತು. ಬೆಳೆಯುತ್ತಿರುವ ತರುಣ ಪೀಳಿಗೆಗೆ, ಕೆಲವರಿಗಾದರೂ ಅದು ದುಡಿಮೆಯ ವಿಧಾನ ಮತ್ತು ಮಹಿಮೆಯನ್ನು ತಿಳಿಸುವ ಸ್ಫೂರ್ತಿಯ ನಿಧಿಯಾಗಬಲ್ಲುದು. ಗಾಂಧೀಜಿ ಅವರ ದಿನದಿನದ, ಕ್ಷಣಕ್ಷಣದ ಪ್ರತಿಯೊಂದು ಕೆಲಸದಲ್ಲೂ ದೊಡ್ಡತನ ಹೇಗೆ ಪ್ರಕಟವಾಗುತ್ತಿತ್ತು ಎಂಬುದು ಆ ಗ್ರಂಥವನ್ನು ಓದುವವರ ಪಾಲಿಗೆ ಸ್ಪಷ್ಟವಾಗಿ ತಿಳಿಯುತ್ತದೆ.

ಕೆಲವರಿಗೆ ಸಣ್ಣ ಪುಟ್ಟ ಕೆಲಸಗಳೆಂದರೆ ಬೇಸರ, ಕಿರಿಕಿರಿ. ಅಂಥ ಕೆಲಸಗಳನ್ನವರು ಚಿಲ್ಲರೆ ವಿಷಯಗಳೆಂದು ತಿಳಿಯುತ್ತಾರೆ. ಅವನ್ನು ಮಾಡುವುದು ತಮಗೆ ಅಗೌರವವೆಂದೇ ತಿಳಿಯುತ್ತಾರೆ. ದೊಡ್ಡ ಮನುಷ್ಯನು ಚಿಲ್ಲರೆ ವಿಷಯಗಳಲ್ಲೂ ದೊಡ್ಡತನದಿಂದ ನಡೆದುಕೊಳ್ಳುತ್ತಾನೆಂಬ ಮಾತನ್ನು ಕೇಳಿದ್ದೀರಾ? ಸಣ್ಣಪುಟ್ಟ ವಿಷಯಗಳನ್ನು ಆತ ಅಲಕ್ಷಿಸುವುದಿಲ್ಲ. ಹೆಜ್ಜೆ ಹೆಜ್ಜೆ ಸರಿಯಾದಲ್ಲಿ ಮಾತ್ರ ದಾರಿ ಸಲೀಸಲ್ಲವೇ?


\section*{ಯೋಗ್ಯತೆಯ ಅಳತೆಗೋಲು}

\addsectiontoTOC{ಯೋಗ್ಯತೆಯ ಅಳತೆ\-ಗೋಲು}

ಪ್ರಸಿದ್ಧ ಶಿಲ್ಪಕಲಾವಿದ ಮೈಕೇಲ್ ಎಂಜೆಲೋ ಅವನ ಕಲಾಕೇಂದ್ರಕ್ಕೆ ಒಮ್ಮೆ ಕಲಾಪ್ರೇಮಿ ಸಂದರ್ಶಕ\-ನೊಬ್ಬ ಬಂದನಂತೆ. ಮೈಕೇಲ್ ತಾನು ಸದ್ಯ ತಯಾರಿಸುತ್ತಿದ್ದ ಮೂರ್ತಿಯನ್ನು ಕುರಿತು\break ವಿವರಿಸುತ್ತಿದ್ದ. ಆ ಮೂರ್ತಿಯ ನಿರ್ಮಾಣ ಕಾರ್ಯವನ್ನು ನೋಡಲು ಆ ಸಂದರ್ಶಕನೇ ಹಿಂದೊಮ್ಮೆ ಬಂದಿದ್ದ. ಅವನು ಬಂದು ಹೋದಂದಿನಿಂದ ಏನೆಲ್ಲ ಕೆಲಸಗಳು ನಡೆದವು ಎಂಬುದನ್ನು ಮೈಕೇಲ್ ಹೇಳಿದ: ‘ಆ ಭಾಗವನ್ನು ಸ್ವಲ್ಪ ಬದಲಿಸಿದೆ, ಕೈಗಳ ಮಾಂಸಖಂಡ ಬಲಿಷ್ಠವಾಗಿ ತೋರುವಂತೆ ಮಾಡಿದೆ, ತುಟಿಗಳಲ್ಲಿ ಕೊಂಚ ಭಾವ ಪ್ರಕಾಶವಾಗುವಂತೆ ಮಾಡಿದೆ, ಕಾಲುಗಳು ಶಕ್ತವಾಗಿ ಕಾಣುವಂಥ ಕೆತ್ತನೆಯ ಕೆಲಸ ಅಲ್ಪಸ್ವಲ್ಪ ನಡೆಯಿತು’ ಎಂದ. ಸಂದರ್ಶಕ ಇನ್ನೂ ಕೆಲಸ ಮುಗಿಯಲಿಲ್ಲವೆನ್ನುವ ಭಾವನೆಯಿಂದಿರಬೇಕು, ‘ಓಹ್, ಇಷ್ಟು ದಿನವೂ ಅತಿ ಸಾಮಾನ್ಯವಾದ ಸಣ್ಣಪುಟ್ಟ ಕೆಲಸಗಳನ್ನೇ ಮಾಡಿದುದೇ?’ ಎಂದು ಉದ್ಗರಿಸಿದ. ಮೈಕೇಲ್ ಅರ್ಥಪೂರ್ಣವಾಗಿ ನಕ್ಕು ಹೇಳಿದ: ‘ನೆನಪಿರಲಿ; ಸಾಮಾನ್ಯವೆನಿಸುವ ಈ ಸಣ್ಣ ಪುಟ್ಟ ಕೆಲಸಗಳಿಂದ ‘ಪರಿ ಪೂರ್ಣತೆ’ ಉಂಟಾಗುತ್ತದೆ. ಆದರೆ ಪರಿಪೂರ್ಣತೆ ಎನ್ನುವುದು ಸಾಮಾನ್ಯವಲ್ಲ.’\footnote{\engfoot{Trifles make perfection and perfection is not trifle.}\hfill\engfoot{ –Michelangelo}}

ಇಂತಿದೆ ಸಣ್ಣ ಕೆಲಸದ ಮಹತ್ವ–ಇದು ಎಲ್ಲ ಕ್ಷೇತ್ರಗಳಿಗೂ ಅನ್ವಯಿಸುವ ಸಂದೇಶ.

ಬರಹಗಾರನು ತನ್ನ ಮಸಿಕುಡಿಕೆ ಮತ್ತು ಲೇಖನಿಯನ್ನು ಇರಿಸಿಕೊಂಡ ಬಗೆಯಿಂದಲೇ ಅವನ ಯೋಗ್ಯತೆಯನ್ನು ಗುರುತಿಸಬಹುದು ಎನ್ನುವ ಮಾತನ್ನು ನೀವು ಕೇಳಿರಬಹುದು. ಅತಿ ಪುಟ್ಟ ಕೆಲಸವೊಂದನ್ನು ಮಾಡುವ ರೀತಿಯನ್ನು ಕಂಡೇ ವ್ಯಕ್ತಿಯೊಬ್ಬನ ಕಾರ್ಯಸಾಮರ್ಥ್ಯವನ್ನೂ, ಅವನ ನಡತೆಯನ್ನೂ ತಿಳಿದುಕೊಳ್ಳಬಹುದು.

ಬಾಲಕನೊಬ್ಬನನ್ನು ಕರೆದು ಕಿಟಕಿಯಲ್ಲಿ ಸಂಗ್ರಹವಾದ ಧೂಳನ್ನು ಝಾಡಿಸಿ ಶುಚಿಗೊಳಿಸಲು ಹೇಳಿ, ಅವನ ಕೆಲಸವನ್ನು ಪರಿಶೀಲಿಸಿ. ಕಿಟಕಿಯ ಮಧ್ಯಭಾಗವನ್ನು ಮಾತ್ರ ಗುಡಿಸಿ ಸಂದು ಗೊಂದುಗಳಲ್ಲಿರುವ ಧೂಳನ್ನು ಗುಡಿಸದೇ ಹಾಗೆಯೇ ಬಿಟ್ಟರೆ ಅವನ ಕಾರ್ಯಸಾಮರ್ಥ್ಯವನ್ನು ಹೊಗಳುವಹಾಗಿಲ್ಲ.

ಹುಡುಗಿಯೊಬ್ಬಳು ಕೋಣೆಯನ್ನು ಶುಚಿಗೊಳಿಸುವ ವಿಧಾನವನ್ನು ಪರಿಶೀಲಿಸಿ. ಸುಮ್ಮ ಸುಮ್ಮನೇ ಕೆಲಸ ಮಾಡುವಂತೆ ನಟಿಸುತ್ತಾ ಕಸಕೊಳೆಯನ್ನು ಮೂಲೆಯಲ್ಲೇ ಸೇರಿಸುವವಳು ಒಳ್ಳೆಯ ಮನೆಯೊಡತಿಯಾಗುವುದು ಕಷ್ಟ.

ಮೋಟರ್ ಷೆಡ್ಡಿನಲ್ಲಿ ಕಾರನ್ನು ತೊಳೆಯುತ್ತಿರುವ ಆಳನ್ನು ಸ್ವಲ್ಪ ಹೊತ್ತು ದಿಟ್ಟಿಸಿ. ಆತನ ಕೆಲಸವನ್ನು ನೋಡುತ್ತಲೇ ನೀವು ಹೇಳಬಹುದು ಅವನಿಗೆ ಉದ್ಯೋಗದಲ್ಲಿ ಪ್ರಗತಿ ಸಾಧ್ಯವೇ, ಇಲ್ಲವೇ, ಎಂದು.

ಪ್ರತಿಯೊಬ್ಬ ಮನುಷ್ಯನ ಕೆಲಸ ಅವನ ವ್ಯಕ್ತಿತ್ವದ ಸಾಕ್ಷಿಚಿತ್ರ. ಸೋಮಾರಿಯೂ, ಅವ್ಯವಸ್ಥಿತ ಚಿತ್ತನೂ ಆದ ವ್ಯಕ್ತಿ ಅತ್ಯಂತ ಸುಲಭ ಹಾಗೂ ಸರಳವಾದ ಕೆಲಸವನ್ನು ಕೆಟ್ಟರೀತಿಯಲ್ಲಿ ಮಾಡುತ್ತಾನೆ. ಮಹತ್ವದ ಫಲಗಳು ಒಮ್ಮೆಗೇ ದೊರೆಯುತ್ತವೆಯೇ? ಅದಕ್ಕಾಗಿ ನಾವು ನಡೆಯುವಾಗ ಹೆಜ್ಜೆಯಿಡುತ್ತ ಮುಂದೆ ಸಾಗುವಂತೆ ಕ್ರಮವಾಗಿ ಸಾಧಿಸುತ್ತಾ ಮುಂದುವರಿಯಬೇಕು. ನೆನಪಿಡಿ:

ಸಣ್ಣ ಕೆಲಸವನ್ನು ಅಲಕ್ಷಿಸಬೇಡಿ–ನಿಷ್ಠೆಯಿಂದ ನಿರ್ವಹಿಸಿ.

ತಾಳ್ಮೆಗೆಡದೆ ಮೆಟ್ಟಲು ಮೆಟ್ಟಲಾಗಿ ಮೇಲೇರಿ.

ದಾರಿಯು ಸರಿಯಾಗಿದ್ದರೆ ಗುರಿಯು ದೊರೆತೇ ದೊರೆಯುತ್ತದೆ.


\section*{ಏಕಿಷ್ಟು ಅವಸರ?}

\addsectiontoTOC{ಏಕಿಷ್ಟು ಅವಸರ ?}

ಕೆಲವರ ಅವಸರ ಹೇಳತೀರದು. ಅವರು ಆಫೀಸಿನಿಂದ ಮನೆಗೆ ಓಡೋಡಿ ಬರುತ್ತಾರೆ, ಬೇಗ ಬೇಗನೇ ಊಟ ಮಾಡುತ್ತಾರೆ, ಎಲ್ಲ ಕೆಲಸಗಳಲ್ಲೂ ತಲೆ ಹಾಕುತ್ತಾರೆ. ಯಾವುದನ್ನೂ ಪೂರ್ಣ ಮಾಡುವುದಿಲ್ಲ. ‘ಬೇಗ ಬೇಗನೇ’ ಎನ್ನುವ ಪಲ್ಲವಿಯನ್ನು ಹಾಡುತ್ತಿರುತ್ತಾರೆ. ಐದು ವರ್ಷ ದಾಟಿರದಿದ್ದರೂ ಮಗುವನ್ನು ನೇರವಾಗಿ ಶಾಲೆಯಲ್ಲಿ ಮೂರನೇ ಕ್ಲಾಸಿಗೆ ಸೇರಿಸುವ ಹುಮ್ಮಸ್ಸು ಅವರದ್ದು. ಸ್ಟೇಷನ್​ನಿಂದ ಟ್ರೈನ್ ಹೊರಟ ಮೇಲೇ ತರಲು ಮರೆತಿದ್ದ ಅತಿ ಮುಖ್ಯವಾದ ಕಾಗದಪತ್ರಗಳನ್ನು ನೆನಪಿಸಿಕೊಂಡು ಅವರು ಚಡಪಡಿಸುವರು. ಆಫೀಸಿನಿಂದ ಕಂಗಾಲಾಗಿ ಮನೆಗೆ ಬರುವುದನ್ನು ಕಂಡು ಅವರ ಹೆಂಡತಿ ಯಾವ ಹೊತ್ತಿಗೆ ಏನು ಗಂಡಾಂತರ ಕಾದಿದೆಯೋ ಎಂದು ನಡುಗುತ್ತಾಳೆ. ಅಲ್ಲೇ ಇರುವ ರೇಡಿಯೊ ಕೊಂಚ ತಿರುಗಿಸಿ ಸಂಗೀತ ಕೇಳಲು ಹೊರಡುತ್ತಾರೆ. ಒಂದೆರಡು ನಿಮಿಷ ಕೇಳಿದಂತೆ ಮಾಡಿ ಅಲ್ಲಿಗೇ ನಿಲ್ಲಿಸಿ, ಹತ್ತು ದಿನಗಳ ಹಿಂದೆ ಪುಸ್ತಕಾಲಯದಿಂದ ತಂದ ಗ್ರಂಥ ಬಿಡಿಸುತ್ತಾರೆ. ಅದನ್ನು ಹಿಂದಿರುಗಿಸಬೇಕು, ಪೂರ್ಣ ಓದಿ ಆಗಲಿಲ್ಲ ಎಂಬ ತವಕ. ಸ್ವಲ್ಪ ಹೊತ್ತು ಓದುವ ಶಾಸ್ತ್ರ ಮಾಡಿ ಅದನ್ನು ಅಲ್ಲೇ ಎಸೆಯುತ್ತಾರೆ. ‘ಛೆ ಬೋರ್​’, ‘ಬೇಸರ’, ‘ಆಯಾಸ’ ಮೊದಲಾದ ಶಬ್ದಗಳನ್ನು ಅವರು ಜಪಿಸುತ್ತಾರೆ. ತಿಂಡಿ ಮುಗಿಸಿ ಆಟಕ್ಕೆ ಹೊರಟ ಮಗನನ್ನು ಕರೆದು ‘ನೋಡೋ, ಅಲ್ಲಿ ಹೋಗಿ ಸಾಮಾನನ್ನು ಇಟ್ಟು ಬಾ’ ಎಂದು ಆಜ್ಞಾಪಿಸುತ್ತಾರೆ. ‘ಎಲ್ಲಿಗೆ? ಯಾವ ಸಾಮಾನು?’ ಎಂದವನು ಚಕಿತನಾದರೆ, ‘ಅಯ್ಯೋ, ಯಾರಪ್ಪಾ ಇವನ ಹತ್ತಿರ ಬಡಿದುಕೊಳ್ಳುವುದು!’ ಎಂದು ಕಿರುಚುತ್ತಾರೆ. ಸಾಮಾನಿನ ಹೆಸರು, ಅದನ್ನು ಇಡಬೇಕಾದ ಜಾಗವನ್ನು ಆಮೇಲೆ ನಿರ್ದೇಶಿಸುತ್ತಾರೆ. ಏಕಕಾಲದಲ್ಲಿ ನಾಲ್ಕು ದಿಕ್ಕುಗಳ ಕಡೆ ಓಡುವಂತಿದೆ ಅವರ ಅವಸ್ಥೆ.

ಹೌದು, ಅವರದು ವ್ಯವಸ್ಥಿತ ಜೀವನವಲ್ಲ. ಅವರ ಹೊಲಗಳಲ್ಲಿ ಉತ್ಪತ್ತಿ ಚೆನ್ನಾಗಿದೆ. ಹಣಕ್ಕೆ ಕೊರತೆ ಇಲ್ಲ. ಅವರೇನು ದುಂದುಗಾರರಲ್ಲ. ಆದರೂ ಒಂದಲ್ಲ ಒಂದು ಗೊಂದಲದಲ್ಲಿ ಸಿಕ್ಕಿ ಕೊಳ್ಳುತ್ತಾರೆ. ಅವರಿಗೆ ತಮ್ಮ ಕೆಲಸವನ್ನು ವ್ಯವಸ್ಥಿತ ರೀತಿಯಲ್ಲಿ ಹೇಗೆ ನಿರ್ವಹಿಸಬೇಕೆಂಬುದು ತಿಳಿಯದು.

ಕೆಲವು ವರ್ಷಗಳ ಹಿಂದೆ ಕನ್ನಡದ ಒಂದು ಮಾಸಪತ್ರಿಕೆಯಲ್ಲಿ ತಾವು ಕಲ್ಕತ್ತದಲ್ಲಿ ಕಂಡ ಒಂದು ಘಟನೆಯನ್ನು ಒಬ್ಬರು ಬರೆದಿದ್ದರು. ಅವಸರ ಗೊಂದಲಗಳ ಮಾನಸಿಕ ಸ್ಥಿತಿಯ ಒಂದು ಚಿತ್ರ ಅದು–

\newpage

ಬಂಗಾಳದ ವಿಧಾನಸಭೆಯ ಸದಸ್ಯರೊಬ್ಬರು ಅಂಚೆ ಕಛೇರಿಗೆ ಬಂದು ಅಲ್ಲಿದ್ದ ಯುವಕನ ಹತ್ತಿರ ‘ಯಾವುದೊ ಕಂಪನಿಯ ಜನರಲ್ ಮ್ಯಾನೇಜರನೊಡನೆ ಮಾತನಾಡಬೇಕಾಗಿದೆ, ಫೋನ್ ಮಾಡುತ್ತೇನೆ’ ಎಂದರಂತೆ. ‘ಇನ್ನೂ ಗಂಟೆ ಒಂಬತ್ತು, ಹತ್ತೂವರೆ ಹೊತ್ತಿಗೆ ಅವರು ತಮ್ಮ ಆಫೀಸಿಗೆ ಹೋಗಬಹುದು. ಆಗ ನೀವು ಫೋನು ಮಾಡಿರಿ’ ಎಂದ ಯುವಕ. ತಮ್ಮ ಹಿರಿತನದ ಗಾಂಭೀರ್ಯದಿಂದ ಅವನ ಮಾತನ್ನು ಕೇಳಿಸಿಕೊಂಡರೂ ಕೇಳಿಸದವರಂತೆ ಸ್ವಲ್ಪಹೊತ್ತು ಸುಮ್ಮನಿದ್ದು, ಬಳಿಕ ರಿಸೀವರ್ ತೆಗೆದುಕೊಂಡು ಡಯಲ್​ಮಾಡಿ ‘ಯಾರು ಮಾತನಾಡುತ್ತಿರುವುದು?’ ಎಂದು ವಿಚಾರಿಸಿದರು. ‘ಚಪರಾಸಿ’ ಎಂದು ಉತ್ತರ ದೊರೆಯಿತು. ಸದಸ್ಯ ಮಹಾಶಯರು ಬಿರುಸಾಗಿ ಫೋನನ್ನು ಕೆಳಗೆ ಕುಕ್ಕಿದರು. ಮತ್ತರ್ಧ ಗಂಟೆಯ ನಂತರ ತಿರುಗಿ ಫೋನ್ ಮಾಡಿದಾಗ ಅದೇ ಉತ್ತರ ಬಂದಿರಬೇಕು. ‘ದುರ್​’ ಎಂದು ಮುಖ ಕೆಂಪಗೆ ಮಾಡಿಕೊಂಡರು. ಕನ್ನಡದಲ್ಲಿ ‘ಥೂ, ಛೀ’ ಎನ್ನುವುದಕ್ಕೆ ಬಂಗಾಲಿಯಲ್ಲಿ ‘ದುರ್​’ ಎನ್ನುತ್ತಾರೆ. ಸುಮಾರು ಹತ್ತೂವರೆಗೆ ತಿರುಗಿ ಫೋನ್ ಮಾಡಿ ‘ಯಾರು ಚಪರಾಸಿನಾ ಮಾತಾಡೋದು?’ ಎಂದು ಕೇಳಿದರು. ಆದರೆ ಆಗ ರಿಸೀವರ್ ತೆಗೆದುಕೊಂಡವರು ಜನರಲ್ ಮ್ಯಾನೇಜರ್ ಆಗಿದ್ದರು. ಇವರ ಮಾತನ್ನು ಕೇಳಿದಾ ಕ್ಷಣ ಅವರು ಏನೂ ಉತ್ತರ ಕೊಡದೇ ರಿಸೀವರ್ ಕೆಳಗೆ ಇಟ್ಟಿರಬೇಕು. ಸದಸ್ಯರು ಆಫೀಸಿನಲ್ಲಿದ್ದ ಯುವಕನನ್ನು ಮಾತನಾಡಿಸದೇ ಜೋಲುಮೋರೆ ಹಾಕಿಕೊಂಡು ಅಲ್ಲಿಂದ ಹೊರಬಿದ್ದರಂತೆ.

ಅಲ್ಲಿ ಅವರು ಸುಮಾರು ಒಂದೂವರೆ ಘಂಟೆಗಳಷ್ಟು ಕಾಲ ವ್ಯರ್ಥವಾಗಿ ಕಳೆದಂತಾಯಿತಲ್ಲ! ಅದಕ್ಕೇನು ಕಾರಣ ಬಲ್ಲಿರಾ?

ಕಾರಣ ಇಲ್ಲಿದೆ: ಅವಸರ, ಉದ್ವೇಗ, ಗೊಂದಲ, ಅನ್ಯಮನಸ್ಕತೆ. ಇವು ಅವರ ವಿವೇಕವನ್ನು ದೂರಕ್ಕೋಡಿಸಿದ್ದವು.

ಮಾತುಕತೆ, ಸಂಚಾರ, ಆಟ ಊಟ, ಇತರ ಎಲ್ಲ ವ್ಯವಹಾರಗಳಲ್ಲೂ ಈ ಆತುರತೆಯಿಂದ ಆಗುವ ಅನಾಹುತಗಳು ಹಲವು.

ಈ ಆತುರ, ಅವಸರ, ಗೊಂದಲಗಳಿಗೇನು ಕಾರಣ?

‘ಯಾವುದೇ ಕೆಲಸ ಕೈಗೊಳ್ಳುವುದಕ್ಕೆ ಮೊದಲು ಶಾಂತಚಿತ್ತರಾಗಿ ಸರಿಯಾಗಿ ಯೋಚಿಸದೇ ಇರುವುದು.’

‘ಹಲವು ಕೆಲಸಗಳನ್ನು ಬೇಗನೇ ಮಾಡಿ ಮುಗಿಸುವ ಚಪಲ.’

‘ವರ್ಷಗಟ್ಟಲೆ ಬೇಕಾಗುವ ಕೆಲಸವನ್ನು ಕೆಲವೇ ದಿನಗಳಲ್ಲಿ ಮಾಡುವ ಹುಚ್ಚು ಉತ್ಸಾಹ.’

‘ಮಾಡಲೇಬೇಕಾದ ಕೆಲಸವನ್ನು ಆದಷ್ಟು ಮುಂದೆ ಹಾಕುತ್ತ ಕಾಲ ಕಳೆದು, ಕೊನೆಯಗಳಿಗೆ ಯಲ್ಲಿ ಅನಿವಾರ್ಯ ಪರಿಸ್ಥಿತಿ ಎಂದಾದಮೇಲೆ ಅತ್ಯಾತುರದಿಂದ ಮಾಡಹೊರಡುವುದು.’

‘ತಪ್ಪಿನಿಂದ ಪಾಠ ಕಲಿಯದೆ ಅದು ತನ್ನ ವೈಶಿಷ್ಟ್ಯವೆಂದೇ ತಿಳಿದುಕೊಂಡು ಆ ಅವ್ಯವಸ್ಥೆಯ ಅಭ್ಯಾಸವನ್ನು ಬಲಪಡಿಸುತ್ತ ಹೋಗುವುದು.’

ಅವಸರ, ಆತುರ, ಉದ್ವೇಗ, ಗೊಂದಲ ಗಲಿಬಿಲಿಗಳು ಮನಸ್ಸನ್ನು ಬಳಲಿಸಿ ದೇಹವನ್ನು ದಣಿಯುವಂತೆ ಮಾಡುತ್ತವೆ. ನೀವು ರಕ್ತದ ಏರೊತ್ತಡ ಮತ್ತು ನರಮಂಡಲದ ದೌರ್ಬಲ್ಯಗಳಿಂದ ನರಳುವ ಪ್ರಸಂಗವನ್ನವು ತಂದೊಡ್ಡುತ್ತವೆ. ಅವುಗಳಿಂದ ಪಾರಾಗುವ ಬಗೆ ಹೇಗೆ?


\section*{ಬೇಗ ಬೇಗನೆ, ಮೆಲ್ಲ ಮೆಲ್ಲನೆ}

\addsectiontoTOC{ಬೇಗ ಬೇಗನೆ, ಮೆಲ್ಲ ಮೆಲ್ಲನೆ}

ಬೇಗ ಬೇಗನೇ ಮುಂದುವರಿಯಲು ನೀವು ಮೆಲ್ಲ ಮೆಲ್ಲನೇ ನಡೆಯಬೇಕು. ಇದೆಂಥ ವಿರೋಧಾ\-ಭಾಸ ಎನ್ನಬಹುದು ನೀವು. ಯೋಚಿಸಿ ನೋಡಿ. ಆಂಗ್ಲ ಭಾಷೆಯಲ್ಲಿನ \enginline{hasten slowly} ಎನ್ನುವ ಉಕ್ತಿಯನ್ನು ಕೇಳಿದ್ದೀರಾ?

ಒಂದು ಪತ್ರಿಕಾ ಕಛೇರಿಗೆ ಒಬ್ಬ ಟೈಪಿಸ್ಟ್ ಹೊಸದಾಗಿ ಸೇರಿಕೊಂಡ. ಅವನು ವೇಗವಾಗಿ ತಪ್ಪಿಲ್ಲದೇ ಟೈಪ್ ಮಾಡುತ್ತಿದ್ದ. ಮ್ಯಾನೇಜರ್ ಅವನ ದಕ್ಷತೆಯನ್ನು ಕಂಡು ಸಂತುಷ್ಟರಾಗಿ ‘ಉಳಿದವರಿಗೂ ವೇಗದ ರಹಸ್ಯವನ್ನು ತಿಳಿಸಿಕೊಡು’ ಎಂದು ಆತನನ್ನು ಕೇಳಿಕೊಂಡರು. ‘ವೇಗವಾಗಿ ಟೈಪು ಮಾಡಬೇಕಾದರೆ ಮೆಲ್ಲನೇ ಟೈಪ್ ಮಾಡಬೇಕು’ ಎಂದು ಅವನು ರಹಸ್ಯ ಭೇದಿಸಿ ದಾಗ ಎಲ್ಲರೂ ಚಕಿತರಾದರು. ಅವನ ಅಭಿಪ್ರಾಯ ಇದು: ‘ವೇಗವಾಗಿ ಟೈಪ್ ಮಾಡಬೇಕಾದರೆ ಮೊದಲು ಮೆಲ್ಲ ಮೆಲ್ಲನೇ ಟೈಪ್ ಮಾಡಬೇಕು. ಯಾವ ಆತುರ ಉದ್ವೇಗಗಳಿಲ್ಲದೇ ಒದ್ದಾಟ ಮಾಡದೇ, ಕೂಡಲೇ ವೇಗವನ್ನು ವರ್ಧಿಸುವ ಹಂಬಲವಿಲ್ಲದೇ ಟೈಪ್ ಮಾಡಬೇಕು. ಹೀಗೆ ಮಾಡಲು ಸದ್ಯದ ನಮ್ಮ ವೇಗವನ್ನು ಅರ್ಧಕ್ಕಿಳಿಸಬಹುದು. ಹಾಗೆ ವೇಗವನ್ನು ಅರ್ಧಕ್ಕಿಳಿಸಿದಾಗ ನಮ್ಮ ಗಲಿಬಿಲಿ ಗೊಂದಲ, ಎಲ್ಲಿ ತಪ್ಪಾಗುತ್ತದೆಯೋ ಎನ್ನುವ ಭೀತಿ ದೂರವಾಗುತ್ತದೆ. ಶಾಂತ ಚಿತ್ತರಾಗಿ ಆನಂದದಿಂದಲೇ ನಾವು ಟೈಪು ಮಾಡುತ್ತೇವೆ. ಹಾಗೆ ಒಂದು ವಾರ ಟೈಪು ಮಾಡುವಾಗ ನಮಗೆ ಗೊತ್ತಿರದಂತೆ ವೇಗವು ಹೆಚ್ಚಾಗಿರುವುದನ್ನು ಕಾಣುತ್ತೇವೆ.’

‘ಅವಸರ ಆತುರಗಳಿಗೂ, ನಮ್ಮ ಕೆಲಸದ ವೇಗಕ್ಕೂ ಸಂಬಂಧವಿಲ್ಲ.’ ಈ ವಿಚಾರವನ್ನು ಸರಿ ಯಾಗಿ ತಿಳಿದುಕೊಂಡರೆ ಎಲ್ಲ ಕೆಲಸಗಳನ್ನೂ ನಾವು ಆಸಕ್ತಿ, ಏಕಾಗ್ರತೆ ಮತ್ತು ಸಂತೋಷದಿಂದ ಮಾಡಬಹುದು.

ಯಕ್ಷಿಣಿ ಆಟಗಾರನೊಬ್ಬ ಒಮ್ಮೆ ಶಾಲೆಗೆ ಬಂದಿದ್ದ. ಆತ ಏಕಕಾಲದಲ್ಲಿ ಆರು ಚೆಂಡುಗಳನ್ನು ಎರಡು ಕೈಗಳಿಂದ ವೇಗವಾಗಿ ಮೇಲಕ್ಕೆಸೆಯುತ್ತಲೂ, ಅವು ಕೆಳಕ್ಕೆ ಬರುತ್ತಿರುವಷ್ಟರಲ್ಲೇ ಸ್ಪರ್ಶ ಮಾತ್ರದಿಂದಲೋ ಎಂಬಂತೆ ತಿರುಗಿ ಮೇಲೇರುವಂತೆ ಮಾಡುತ್ತಲೂ ಇದ್ದ. ಚೆಂಡುಗಳು ಅವನ ಹಸ್ತವನ್ನು ಮುಟ್ಟುತ್ತಲೇ ಪುಟ ನೆಗೆಯುವಂತೆ ತೋರುತ್ತಿದ್ದವು. ಎಷ್ಟು ಎತ್ತರಕ್ಕೆ ಹಾರಿಸಿದರೂ ಅವು ಅವನ ಕೈಯೆಡೆಗೆ ಕಾಣದ ಸೂತ್ರದಿಂದೆಳೆದಂತೆ ಓಡೋಡಿ ಬರುತ್ತಿದ್ದವು. ವಿದ್ಯಾರ್ಥಿಗಳು ಆ ಚಮತ್ಕಾರವನ್ನು ಕಂಡು ಚಕಿತರೂ, ಆನಂದಭರಿತರೂ ಆದರು. ಆಟಗಾರ ಹೊರಟು ಹೋದ ಕೂಡಲೇ ಕೆಲವು ಉತ್ಸಾಹಿಗಳು ಕೈಗೆ ಸಿಕ್ಕಿದ ಲಿಂಬೆಹಣ್ಣು, ಮೂಸಂಬಿ, ಚೆಂಡುಗಳನ್ನು ಮೇಲೆಕ್ಕೆಸೆದು ಮುಖ ಮೂತಿಗಳ ಮೇಲೆ ಬೀಳಿಸಿಕೊಂಡರು. ಕೆಲವರು ಒಂದೆ ರಡು ದಿನ ಯತ್ನಿಸಿ ‘ಅಯ್ಯೊ, ಇದು ನಮ್ಮಿಂದಾಗದ ಕೆಲಸ’ ಎಂದು ಕೈಬಿಟ್ಟರು. ಹದಿನೈದು ದಿನಗಳ ಬಳಿಕ ಮಕ್ಕಳ ಮನೋರಂಜನಾ ಕಾರ್ಯಕ್ರಮದಲ್ಲಿ ಒಬ್ಬ ವಿದ್ಯಾರ್ಥಿ ನಾಲ್ಕು ಚೆಂಡನ್ನು ಎರಡೂ ಕೈಗಳಿಂದ ಸರಸರನೆ ಹಾರಿಸಿ ಹಿಡಿದು ಎಲ್ಲರನ್ನೂ ಅಚ್ಚರಿಗೊಳಿಸಿದ. ವಿದ್ಯಾರ್ಥಿಗಳು ಜೋರಾಗಿ ಚಪ್ಪಾಳೆ ಹೊಡೆದು ತಮ್ಮ ಆನಂದವನ್ನು ವ್ಯಕ್ತಪಡಿಸಿದರು. ಕಾರ್ಯಕ್ರಮ ಮುಗಿದ ಬಳಿಕ ವಿದ್ಯಾರ್ಥಿಗಳೆಲ್ಲ ‘ಹೇಗೆ ಕಲಿತೆ?’, ‘ನನಗೂ ಹೇಳಿಕೊಡು’ ಎಂದು ಅವನನ್ನು ಮುತ್ತಿ ಕೊಂಡರು. ಅವನು ಥಟ್ಟನೇ ಗುಟ್ಟು ಬಿಟ್ಟುಕೊಡಲಿಲ್ಲ. ‘ಮ್ಯಾಜಿಕ್ ಮಾಡಿದೆ’ ಎಂದ. ನಾನು ಅವನನ್ನು ಪ್ರತ್ಯೇಕವಾಗಿ ಹತ್ತಿರ ಕರೆದು ಕೇಳಿದಾಗ ಅವನೆಂದ: ‘ಪ್ರತಿನಿತ್ಯ ಯಾರೂ ಇಲ್ಲದ ಜಾಗದಲ್ಲಿ ಮೊದಲು ಒಂದೇ ಚೆಂಡನ್ನು ಮೆಲ್ಲಮೆಲ್ಲನೇ ಒಂದೇ ಕೈಯಿಂದ ಮೇಲಕ್ಕೆ ಹಾರಿಸಿ ಅದೇ ಕೈಯಿಂದ ಹಿಡಿದೆ. ಸುಮಾರು ನೂರು ಬಾರಿ ಹಾಗೆ ಮಾಡಿದ ಮೇಲೆ ಎರಡು ಚೆಂಡುಗಳನ್ನು ಒಂದೇ ಕೈಯಿಂದ ಒಂದಾದ ಮೇಲೊಂದರಂತೆ ಹಾರಿಸಿ ಹಿಡಿದೆ. ಧೈರ್ಯಬಂತು. ನನಗೂ ಸಾಧ್ಯ ಎಂದು ತೋರಿತು. ಹತ್ತು ದಿನಗಳ ನಂತರ ಒಂದು ತಪ್ಪೂ ಇಲ್ಲದೆ ವೇಗವಾಗಿ ಹಾರಿಸಿ ಹಿಡಿಯಲು ಸಾಧ್ಯವಾಯಿತು.’

\vskip 2pt

ಬೇಗಬೇಗನೇ ಎನ್ನುವುದು ಮೆಲ್ಲಮೆಲ್ಲನೆಯಿಂದಲೇ ಸಾಧ್ಯ ಎಂಬುದನ್ನು ಆತ ಕಂಡು ಹಿಡಿದಿದ್ದ.

\vskip 2pt

ನಮ್ಮ ಸುಪ್ತಮನಸ್ಸು ನಾವು ಪಡೆಯುವ ಎಲ್ಲ ಅನುಭವಗಳನ್ನೂ ಸಂಗ್ರಹಿಸಿರುತ್ತದೆ. ಅಭ್ಯಾಸಗಳ ನಿರ್ಮಾಣ ಮತ್ತು ನಿಯಂತ್ರಣ ಸುಪ್ತ ಮನಸ್ಸಿನ ಒಂದು ಮುಖ್ಯ ಕಾರ್ಯ. ಅದನ್ನು ನಾವು ಹೇಗೆ ನಡೆಯಿಸಿಕೊಳ್ಳುತ್ತೇವೋ ಅದು ಹಾಗೇ ನಡೆಯುತ್ತದೆ. ನಮ್ಮ ಉದ್ದೇಶದ ಸಾಫಲ್ಯಕ್ಕೆ ಸಹಾಯಕವಾಗುವಂತೆಯೂ, ವಿರೋಧವಾಗುವಂತೆಯೂ, ಅದನ್ನು ದುಡಿಸಿಕೊಳ್ಳ ಬಹುದು. ಆತುರ, ಅಸಹನೆ, ಗಲಿಬಿಲಿ, ಗೊಂದಲಗಳನ್ನು ಅದಕ್ಕಿತ್ತರೆ ಅದನ್ನೇ ಅದು ವೃದ್ಧಿಸುತ್ತದೆ. ಕ್ರಮ, ನಿಯಮ, ಶಾಂತತೆ, ಏಕಾಗ್ರತೆಯ ಕೆಲಸವನ್ನೇ ಕೊಟ್ಟರೆ ಅದನ್ನು ವೃದ್ಧಿಸಲೂ ತನ್ನ ಸಹಾಯ ಮಾಡುತ್ತದೆ.

\vskip 2pt

ಹಾರ್ಮೋನಿಯಮ್, ಪಿಟೀಲು ವಾದನಪಟುಗಳ ಬೆರಳುಗಳು ವಾದ್ಯದ ಮೂಲಕ ಹೊರಡಿ ಸುವ ಮಧುರ ಧ್ವನಿಯನ್ನು ನೀವು ಕೇಳಿದ್ದೀರಿ. ಗಾಯಕನು ಹಾಡುವ ರಾಗದ ಏರಿಳಿತಗಳನ್ನು ಕ್ಷಣಾರ್ಧದಲ್ಲಿ ಅವರು ಹಿಂಬಾಲಿಸುತ್ತಾರಷ್ಟೆ. ಹಾಗೆ ಬಾರಿಸುವಾಗ ಗಾಯಕನಿಗೆ ಮುಖ ಕೊಟ್ಟು ನಗುತ್ತ, ತಮ್ಮ ಕೈಚಳಕದಿಂದ ಆತನ ಸವಾಲನ್ನು ಎದುರಿಸುತ್ತ, ತಾಳಲಯದ ಕಡೆಗೂ ಗಮನ ಹರಿಯಿಸುತ್ತ, ತಮ್ಮ ಚಾತುರ್ಯಕ್ಕೆ ಸಭಿಕರು ತೋರಿಸುವ ಆನಂದ ಅಭಿನಂದನೆಗಳನ್ನು ಆಂತರಿಕ ಆತ್ಮತೃಪ್ತಿಯ ನಗುವಿನಿಂದ ಸ್ವೀಕರಿಸುತ್ತಿರುತ್ತಾರೆ. ಗಂಟೆಗಟ್ಟಲೆ ವಾದ್ಯ ಬಾರಿಸಿದರೂ ಅವನ ಆನಂದ ಸ್ಫೂರ್ತಿಗಳಿಗೆ ಕೊರತೆಯಿಲ್ಲ.

\vskip 2pt

ಕಛೇರಿ ನಡೆಯುವ ವೇಳೆ ವಾದ್ಯ ಬಾರಿಸುವವರು ಎಲ್ಲೆಲ್ಲಿ ಹೇಗೆ ಬೆರಳಿಡಬೇಕು ಎಂಬುದಕ್ಕೆ ಸ್ವಲ್ಪವಾದರೂ ಗಮನ ಕೊಡುವರೇ? ಇಲ್ಲ. ಹಾಗೆ ಗಮನವಿತ್ತರೆ ಅವರಿಂದ ಸರಿಯಾಗಿ ಬಾರಿ ಸಲು ಸಾಧ್ಯವಾಗುವುದೇ? ಖಂಡಿತ ಸಾಧ್ಯವಿಲ್ಲ. ಅವರ ಹೊರಮನಸ್ಸು ಹಲವು ಕಡೆ ಗಮನವಿತ್ತರೂ ಒಳಮನಸ್ಸು ಅವರ ಬೆರಳುಗಳನ್ನು ಹೇಗೆ ಬೇಕೋ ಹಾಗೆ ನಡೆಸುತ್ತದಲ್ಲವೇ? ಇದರ ರಹಸ್ಯವೇನು?

ಯಾವುದೇ ಕಲೆ ಅಥವಾ ಕೆಲಸದಲ್ಲಿ ನಮಗೆ ಪೂರ್ಣ ದಕ್ಷತೆ ಬರಬೇಕಾದರೆ ಅದರ ಕಾರ್ಯ ವಿಧಾನ ನಮ್ಮ ಸುಪ್ತಮನಸ್ಸಿನಲ್ಲಿ ಸರಿಯಾಗಿ ನೆಲೆನಿಲ್ಲಬೇಕು. ಅದನ್ನು ಮಾಡುವಾಗ ಲಕ್ಷ್ಯ ಕೊಡದೆ ಸರಾಗವಾಗಿ ಮಾಡುವಂತಾಗಬೇಕು. ತನ್ಮೂಲಕ ಸಂತೋಷವೂ ಉಂಟಾಗಬೇಕು. ಕೆಲಸವನ್ನೋ, ಕಲೆಯನ್ನೋ ಅವಸರ ಆತುರದ ಭಾರ ಹೊರಿಸದೇ ಮೆಲ್ಲಮೆಲ್ಲನೇ ಕಲಿಯಲು ಯತ್ನಿಸಿದರೆ ಅದನ್ನು ಸರಿಯಾಗಿ, ಪೂರ್ಣವಾಗಿ ಕಲಿಯಬಲ್ಲೆವು. ಮಾತ್ರವಲ್ಲ, ಯಾವ ಕಷ್ಟ ವನ್ನೂ ಪಡದೆ ಅಥವಾ ಭಯವಿಲ್ಲದೇ ಮುಂದೆ ಅವಕಾಶ ಬಂದಾಗ ಪುನಃ ಅದನ್ನು ಮಾಡ ಬಲ್ಲೆವು.

ನೀವು ಇತ್ತೀಚೆಗೆ ಸೈಕಲ್ ಸವಾರಿ ಮಾಡದೇ ಸುಮಾರು ಹತ್ತು ವರ್ಷಗಳೇ ಆಗಿರಬಹುದು. ಆದರೆ ಒಮ್ಮೆ ತಿರುಗಿ ಸವಾರಿ ಮಾಡಲು ಯತ್ನಿಸಿ. ನಿಮ್ಮ ಸುಪ್ತ ಮನಸ್ಸು ಖಂಡಿತವಾಗಿ ನಿಮಗೆ ಸಹಾಯಗೈಯುವುದು. ನೀವು ಭಯ ಸಂಶಯವಿಲ್ಲದೇ ಸಂತೋಷದಿಂದ ಸವಾರಿ ಮಾಡುವಿರಿ.

ಸಣ್ಣ ಕೆಲಸವಾದರೂ ಮನಸ್ಸು ಕೊಟ್ಟು ಚೆನ್ನಾಗಿ ಮಾಡಬೇಕು. ಏಕೆ ಎಂಬುದು ನಿಮಗೀಗ ಗೊತ್ತಾಗಿರಬಹುದಲ್ಲವೇ?

ಕಲಿಕೆಯಲ್ಲಿ ಈ ತತ್ವದ ರಹಸ್ಯ ಹಲವು ವಿದ್ಯಾರ್ಥಿಗಳಿಗೆ ಅಪಾರ ಸ್ಫೂರ್ತಿಯನ್ನು ನೀಡ ಬಲ್ಲದು. ಅವರ ಅಧ್ಯಯನದ, ಅಲ್ಲ ಬದುಕಿನ ದಿಕ್ಕನ್ನೇ ಬದಲಿಸಬಲ್ಲುದು.


\section*{ನನ್ನ ಸಮಯ ಅಮೂಲ್ಯ}

\addsectiontoTOC{ನನ್ನ ಸಮಯ ಅಮೂಲ್ಯ}

ಪ್ರಸಿದ್ಧ ಜನನಾಯಕರೂ, ಉದ್ಯಮಿಗಳೂ ಆದ ಹಿರಿಯರೊಬ್ಬರನ್ನು ನಾನು ಅವರ ಆಫೀಸಿನಲ್ಲಿ ಭೇಟಿಯಾದೆ. ಅವರ ಕೋಣೆಯಲ್ಲಿದ್ದ ಮೇಜಿನ ತುದಿಗೆ ‘ನನ್ನ ಸಮಯ ಅಮೂಲ್ಯವಾಗಿದೆ’ ಎಂಬ ಪುಟ್ಟ ಬೋರ್ಡನ್ನು ತೂಗಹಾಕಿದ್ದರು. ಏನಿದು? ಇವರ ಸಮಯ ಅಮೂಲ್ಯ, ನನ್ನ ಸಮಯ ಅಮೂಲ್ಯವಲ್ಲವೇ? ಎಂದು ಯೋಚಿಸುತ್ತ ತಿರುಗಿ ಅದೇ ವಾಕ್ಯವನ್ನು ಓದಿದೆ. ಯಾರು ಅಲ್ಲ ಎಂದವರು? ತನ್ನ ಸಮಯ ಅಮೂಲ್ಯವಾಗಿದೆ ಎಂಬುದನ್ನು ತಿಳಿಯದ ಹಲವರು ಇತರರ ಸಮಯವನ್ನು ಹಾಳು ಮಾಡಲೂ ಹಿಂಜರಿಯುವುದಿಲ್ಲ. ಅಂಥವರಿಗೆ ಎಚ್ಚರಿಕೆಯಾಗಿ ಅವರು ಅಲ್ಲಿ ಆ ಮಾತನ್ನು ಬರೆದಿದ್ದರು.

‘ನನ್ನ ಸಮಯ ಅಮೂಲ್ಯವಾಗಿದೆ’ ಎಂದು ಬರೆದಿರಿಸಿಕೊಂಡ ಅವರು ದಿನದಲ್ಲಿ ಹದಿನಾರು ಗಂಟೆಗಳಿಗಿಂತಲೂ ಹೆಚ್ಚು ಹೊತ್ತು ದುಡಿಯಬಲ್ಲರು. ವಿಶಾಲವಾಗಿ ಬೆಳೆದ ಸಂಸ್ಥೆಯ ನೂರಾರು ಸಮಸ್ಯೆಗಳಿಗೆ ಸಮಾಧಾನವನ್ನೂ, ಸೂಕ್ತ ಸಲಹೆಗಳನ್ನೂ ನೀಡಬಲ್ಲರು. ಅಲ್ಲಿ ಬರೆದ ಮಾತಿನ ಅರ್ಥವನ್ನು ಅವರು ಬದುಕಿನಲ್ಲಿ ಆಚರಿಸಿ, ಉಪದೇಶಿಸುವಂತಿತ್ತು.

ನೀವು ನಿಮ್ಮ ಮನಸ್ಸಿನ ಎದುರು ಆ ಮಾತುಗಳನ್ನು ಬರೆದ ಒಂದು ಬೋರ್ಡನ್ನು ತೂಗು ಹಾಕಿ. ನಿಜವಾಗಿಯೂ ನಿಮ್ಮ ಸಮಯ ಅಮೂಲ್ಯವಾಗಿದೆ.

ಒಂದು ಸಣ್ಣ ಲೆಕ್ಕ ಮಾಡುತ್ತೀರಾ? ಒಂದು ವರ್ಷದಲ್ಲಿ ೮೭೬೦ ಗಂಟೆಗಳಿವೆಯಷ್ಟೆ! ಈ ಸಮಯವನ್ನು ನೀವು ಹೇಗೆ ಕಳೆಯುತ್ತೀರೆಂಬುದನ್ನು ಸ್ವಲ್ಪ ಯೋಚಿಸಿ. ಇಷ್ಟು ಸಮಯ ನಿದ್ರೆಗೆ, ಇಷ್ಟು ಸಮಯ ಕೆಲಸಕ್ಕೆ, ಆಹಾರ ಸೇವನೆಗೆ, ವಿನೋದ ವಿಹಾರಕ್ಕೆ ಇಷ್ಟು–ಹೀಗೆಂದು ವಿಂಗಡಿ ಸಿರಿ. ಒಂದು ಪೆನ್ಸಿಲ್ ತೆಗೆದುಕೊಂಡು ಕಾಗದದ ಮೇಲೆ ಶಾಂತಚಿತ್ತರಾಗಿ ಬರೆಯಿರಿ. ನೀವು ಆಶ್ಚರ್ಯಚಕಿತರಾಗುತ್ತೀರಿ–ನಿಮಗೆ ಎಷ್ಟು ಹೊತ್ತು ವಿರಾಮ ಕಾಲವಿದೆಯೆಂದು ಆಗ ಅರ್ಥ ವಾಗುವುದು. ಇನ್ನೊಂದು ಆಶ್ಚರ್ಯ ನಿಮ್ಮನ್ನು ಕಾದಿರುತ್ತದೆ. ಅಷ್ಟು ವಿರಾಮ ಕಾಲವನ್ನು ನೀವು ಹೇಗೆ ಸಾರ್ಥಕವಾಗಿಯೋ, ನಿರರ್ಥಕವಾಗಿಯೋ ಕಳೆದಿರೆಂಬುದು. ಸಮಯದ ಕಡೆಗೆ ಗಮನ ಕೊಡದೆ ಹಣವನ್ನು ಕೂಡಿ ಹಾಕುವುದು ಮಾತ್ರ ಬುದ್ಧಿವಂತಿಕೆ ಎಂದು ತಿಳಿದರೆ ನಿಮ್ಮ ಅರ್ಥ ಶಾಸ್ತ್ರಿಕೆಯನ್ನು ಮೆಚ್ಚುವಂತಿಲ್ಲ. ನೆನಪಿಡಿ: ನಿಮ್ಮ ಸಮಯದ ಸದ್ವಿನಿಯೋಗವೇ ಯಶಸ್ಸಿನ ಗುಟ್ಟು.

ದಿನದಲ್ಲಿ ನೀವು ಒಂದೆರಡು ಗಂಟೆಗಳ ಕಾಲ ವ್ಯರ್ಥಹರಟೆಯಲ್ಲಿ ಅಥವಾ ಯಾವ ಕೆಲಸವೂ ಇಲ್ಲದೆ ಕಳೆದಿರೆಂದರೆ ಒಂದು ವರ್ಷದ ನಿಮ್ಮ ಅಮೂಲ್ಯ ಸಮಯದಲ್ಲಿ ನೀವು ಸಾಧಿಸಬಹುದಾಗಿದ್ದ ಅದ್ಭುತ ಹೇಗೆ ಮಾಯವಾಗುವುದೆಂಬುದನ್ನು ಯೋಚಿಸಿ. ಮತ್ತೆ ನೋಡೋಣವೆಂದು ನಿಮ್ಮ ಕೆಲಸವನ್ನು ಮುಂದಕ್ಕಿರಿಸಿಕೊಳ್ಳುತ್ತೀರಾ? ಕಳೆದ ಸಮಯ ಹಿಂತಿರುಗಿ ಬರುವುದೇ? ನಮ್ಮ ಜೀವನ ದೀರ್ಘವಾಗಿದೆಯೇ ಅದನ್ನು ಸಿಕ್ಕಾಪಟ್ಟೆ ಉಪಯೋಗಿಸಿ ಮತ್ತೆ ಸರಿಪಡಿಸಿಕೊಳ್ಳುವುದಕ್ಕೆ? ನಿಜವಾದ ಮಿತವ್ಯಯಿಯ ಮನಸ್ಸಿನಲ್ಲಿ ನಿರರ್ಥಕ ವಿಚಾರಗಳಿಗೆ ಸ್ಥಾನವಿರದು, ಉದ್ದೇಶ ಹೀನ ಚರ್ಚೆಯಲ್ಲಿ ನಾವೇಕೆ ಮಗ್ನರಾಗಬೇಕು? ನಿಶ್ಚಿತ ಸಮಯವನ್ನು ಗೊತ್ತುಪಡಿಸಿಕೊಂಡು ನೀವು ಕಾರ್ಯಮಗ್ನರಾಗಿ. ಒಸ್ಲರ್ ಹೇಳಿದ ಮಾತನ್ನು ನೀವು ಮರೆಯುತ್ತೀರಾ? ನೀವು ಈಗಿಂದೀಗ ಜಾಗರೂಕರಾಗದಿದ್ದರೆ ಮತ್ತೆ ಪಶ್ಚಾತ್ತಾಪ ಪಡಬೇಕಾಗುವುದು. ಉಮರ್ ಖಯ್ಯಾಮ್ ಹೇಳಿದಂತೆ:

‘ನಮ್ಮಾಯುಸ್ಸಿನ ಪಕ್ಕಿ ಪಾರ್ವ ದೂರವೆ ಕಿರಿದುಅದು ರೆಕ್ಕೆ ಎತ್ತಿಹುದು ನೋಡು ಮನ ಮಾಡು.’

‘ಈ ದಿನ ಎಂದಿಗೂ ಹಿಂದಿರುಗಿ ಬರಲಾರದೆಂಬುದನ್ನು ಯೋಚಿಸು’ ಎಂದು ದಾಂತೆ ಸಾರಿದ. ನಂಬಲಸಾಧ್ಯವಾದ ವೇಗದಿಂದ ನಮ್ಮ ಜೀವನ ಜಾರಿ ಹೋಗುತ್ತಿದೆ. ಸೆಕೆಂಡಿಗೆ 19 ಮೈಲುಗಳ ವೇಗದಲ್ಲಿ ನಮ್ಮ ಯಾನ ಸಾಗಿದೆ, ಅಂತರಿಕ್ಷದಲ್ಲಿ, ಈ ದಿನ, ಈ ಕ್ಷಣ ಮಾತ್ರ ನಮ್ಮ ಅಮೂಲ್ಯ ಸೊತ್ತು. ಹೌದು, ಅದು ನಮ್ಮ ಏಕಮಾತ್ರ ನಿಧಿ. ಭೂತಭವಿಷ್ಯಗಳ ಚಿಂತೆಯನ್ನು ಮರೆತು ಸರ್ವ ಪ್ರಯತ್ನಗಳಿಂದಲೂ ಈ ಕ್ಷಣದಲ್ಲಿ ಯಶಸ್ವಿಯಾಗಿ ಬದುಕಲು ದಿನದಿನವೂ ಯತ್ನಿಸೋಣ. ಆಗ ಮಾತ್ರ ನಮ್ಮ ಬಾಳಿಗೆ ಹುರುಪು, ಹೊಸತನ, ಬೆಳಕು–ಇವು ಬಂದೇ ಬರುತ್ತವೆ. ಒಂದೊಂದೇ ದಿನಗಳು ಸೇರಿ ವರ್ಷಗಳಾಗುತ್ತವೆ. ಜೊತೆಗೆ ದಿನ ದಿನವೂ ಮಾಡಿದ ಒಳ್ಳೆಯ ಕೆಲಸಗಳ ಫಲವೂ ಸಂಸ್ಕಾರಗಳೂ ಕಲೆತು ವೃದ್ಧಿಯಾಗುತ್ತವೆ.

‘ನೀವು ಗಡ್ಡವನ್ನು ಏಕೆ ಕ್ಷೌರ ಮಾಡಿಕೊಳ್ಳುತ್ತಿಲ್ಲ’ ಎಂದು ಯಾರೋ ಜಾರ್ಜ್ ಬರ್ನಾರ್ಡ್‍ಷಾ ಅವರನ್ನು ಕೇಳಿದರು. ಆ ಪ್ರಶ್ನೆಗೆ ಅವರು ನೀಡಿದ ಉತ್ತರ ಹೀಗಿತ್ತು: ‘ಗಡ್ಡ ಮಾಡಿಕೊಂಡರೆ ಸಮಯ ವ್ಯರ್ಥವಾಗುತ್ತದೆ. ನಾನು ದಿನಕ್ಕೆ ನಾಲ್ಕು ಮಿನಿಟುಗಳಂತೆ ಜೀವನದಲ್ಲಿ ಹತ್ತು ತಿಂಗಳುಗಳನ್ನು ಗಳಿಸಿದ್ದೇನೆ.’

ಗಡ್ಡಧಾರಿಗಳೆಲ್ಲ ಸಮಯದ ಸದುಪಯೋಗ ಮಾಡುತ್ತಾರೆಂದಲ್ಲ. ಸಮಯದ ಬೆಲೆಯನ್ನು ಅರಿತ, ಅತ್ಯಂತ ಎಚ್ಚರಿಕೆಯಿಂದ ತನ್ನ ಪಾಲಿಗೆ ದೊರೆತ ಸಮಯವನ್ನು ಸ್ವಲ್ಪವೂ ವ್ಯರ್ಥವಾಗದಂತೆ ನೋಡಿಕೊಂಡ ಹಿರಿಯ ಅನುಭವಿಯೊಬ್ಬರ ಮಾತು ಅದು.

ಹೌದು, ಜೀವನ ಬಹಳ ದೀರ್ಘವಾಗಿಲ್ಲ. ಸಮಯ ಅಮೂಲ್ಯ. ಈ ಕ್ಷಣವೇ ಮುಂದಾಗಿ.


\section*{ನಿಮಗೆ ನೀವೇ}

\addsectiontoTOC{ನಿಮಗೆ ನೀವೇ}

ಸತ್ತಮೇಲೆ ಜೀವಾತ್ಮನ ಅವಸ್ಥೆಯನ್ನು ಕುರಿತು ಪುರಾಣಗಳು ಹೇಳುವ ಕತೆಯನ್ನು ನೀವು ಕೇಳಿ\-ದ್ದೀರಾ? ಈ ಲೋಕದಲ್ಲಿ ಆತ ಮಾಡಿದ ತಪ್ಪಿಗಾಗಿ ನರಕದಲ್ಲಿ ಭೀಕರವಾದ ದಂಡನೆಯನ್ನು ಅನುಭವಿಸಬೇಕಾಗುತ್ತದಲ್ಲವೇ? ಅದನ್ನು ನೀವು ನರಕಯಾತನೆ ಎನ್ನುತ್ತೀರಷ್ಟೆ? ಅದಕ್ಕಿಂತಲೂ ಭೀಕರವಾದ ಯಮಯಾತನೆಯನ್ನು ನಾವಿಲ್ಲಿ ಅನುಭವಿಸಬೇಕಾಗುತ್ತದೆಂದು ಜೇಮ್ಸ್​ ಹೇಳಿದ. ಕಾರಣ ಗೊತ್ತೇನು?–ನಮ್ಮ ದುರಭ್ಯಾಸ ದುರ್ನಡತೆಗಳು.

‘ತಪ್ಪು ದಾರಿಯಲ್ಲಿ ನಡೆಯುವ ಅಭ್ಯಾಸದಿಂದ ಈ ಜಗತ್ತಿನಲ್ಲಿ ನಾವು ನಿರ್ಮಿಸಿಕೊಳ್ಳುವ ನರಕ, ಪುರಾಣಗಳು ಹೇಳುವ ನರಕಕ್ಕಿಂತ ಕನಿಷ್ಠವಾದುದಲ್ಲ’\footnote{\engfoot{The hell to be endured hereafter of whch theology tells, is no worse than the hell we make ourselves in this world by habitually fashioning our character in the wrong way.}\hfill\hbox{\engfoot{ –Willam James}}} ಎಂದು ಆತ ಸಾರಿದ.

ಹೌದು! ಪುರಾಣಗಳು ಹೇಳುವ ನರಕಕ್ಕಿಂತಲೂ ನಮ್ಮ ದುರ್ನಡತೆಗಳಿಂದ ನಾವೇ ನಿರ್ಮಿಸಿ ಕೊಳ್ಳುವ ನರಕ ಭಯಾನಕವಾದುದು.

ಆಶ್ಚರ್ಯ! ನಮ್ಮ ನರಕವನ್ನು ನಿರ್ಮಿಸಿಕೊಳ್ಳುವವರು ನಾವೇ! ನರಕ ಮಾತ್ರವಲ್ಲ, ಸ್ವರ್ಗವನ್ನು ನಿರ್ಮಿಸಿಕೊಳ್ಳುವವರೂ ಹೌದು.

ತಂದೆ ಮಾಡಿದ ಸಾಲವನ್ನು ಮಗ ತೀರಿಸಬಹುದು. ಅಣ್ಣ ಕಟ್ಟಿಸಲು ಪ್ರಾರಂಭಿಸಿದ ಸೌಧವನ್ನು ತಮ್ಮನು ಪೂರ್ಣಗೊಳಿಸಬಹುದು. ಆದರೆ ನಿಮ್ಮ ಹಸಿವು ನೀರಡಿಕೆಗಳನ್ನು ನೀವೇ ಪರಿಹರಿಸಿಕೊಳ್ಳಬೇಕಲ್ಲವೇ? ನಿಮ್ಮ ಸಮಸ್ಯೆಗಳಿಗೆ ನೀವೇ ಉತ್ತರ ಹುಡುಕಬೇಕು. ಇತರರೇನೋ ನಿಮಗೆ ಮಾರ್ಗದರ್ಶನ ಮಾಡಬಹುದು. ಆದರೆ ಆ ಮಾರ್ಗದಲ್ಲಿ ನಡೆಯಬೇಕಾದವರು ನೀವೇ. ಆಗ ಮಾತ್ರ ನಿಮ್ಮ ಬಾಳಿಗೊಂದು ಸಾರ್ಥಕತೆ ಬಂದೀತು. ವ್ಯವಸ್ಥಿತ ಜೀವನವನ್ನು ನಡೆಯಿಸಿ ನೀವು ಭಾಗ್ಯಶಾಲಿಗಳೂ ಆಗಬಲ್ಲಿರಿ. ಎಲ್ಲವೂ ಕಾಲಾಧೀನವೆನ್ನುತ್ತ ಯಾವ ಕೆಲಸವನ್ನೂ ಆಸ್ಥೆಯಿಂದ ನಿರ್ವಹಿಸದೇ ನಿಮ್ಮ ಬಾಳನ್ನು ಹಾಳುಗೆಡಹಿ ಗೋಳನ್ನು ಆಹ್ವಾನಿಸಲೂ ಬಲ್ಲಿರಿ.


\section*{ಶಕ್ತಿಯೇ ಜೀವನ}

\vskip -8pt\addsectiontoTOC{ಶಕ್ತಿಯೇ ಜೀವನ}

ಜಗತ್ತು ಪೂಜಿಸುವುದು ಶಕ್ತಿಯನ್ನು; ದುರ್ಬಲತೆಯನ್ನಲ್ಲ ಎಂಬುದನ್ನು ಯುವಕರು ಸರಿಯಾಗಿ ತಿಳಿದಿರಬೇಕು. ಕ್ರಿಕೆಟ್ ಮೈದಾನದಲ್ಲಿ ದಾಂಡಿಗ ನಾಲ್ಕು ಸಿಕ್ಸರ್ ಬಾರಿಸಿದರೆ ಸಹಸ್ರಾರು ಯುವಕರು ಚಪ್ಪಾಳೆ ತಟ್ಟಿ, ಹುಚ್ಚೆದ್ದು ಕುಣಿದು ಅವನನ್ನು ಹೆಗಲಮೇಲೇರಿಸಿಕೊಂಡು ರಥೋತ್ಸವ ಮಾಡುವು\-ದಿಲ್ಲವೇ? ಅದು ಅವನ ಶಕ್ತಿ ಸಾಮರ್ಥ್ಯ ದಕ್ಷತೆಗಳಿಗೆ ಸಲ್ಲಿಸಿದ ಗೌರವವಲ್ಲವೇ? ಶಕ್ತಿಯನ್ನು ಎಲ್ಲರೂ ಭಕ್ತಿ ಗೌರವಗಳಿಂದ ನೋಡುತ್ತಾರೆ. ‘ಶಕ್ತನಾದರೆ ನೆಂಟರೆಲ್ಲ ಹಿತರು, ಅಶಕ್ತನಾದರೆ ಆಪ್ತ ಜನರೇ ವೈರಿಗಳು’ ಎಂದು ಪುರಂದರದಾಸರು ಹೇಳಿದ ಮಾತು ಎಷ್ಟು ಸತ್ಯ! ತಂದೆತಾಯಂದಿರು ಪ್ರೀತಿಯಿಂದ ಸಾಕಿ ಸಲಹಿದ ಮಗುವು ಬೆಳೆದು ದುರ್ಬಲನೂ, ರೋಗಿಷ್ಠನೂ, ದಡ್ಡನೂ, ಮೂರ್ಖನೂ ಆದರೆ ಏನೆನ್ನುವರು ಬಲ್ಲಿರಾ? ‘ಯಾಕಾಗಿ ಹುಟ್ಟಿದನಿವನು, ಕುಲಕ್ಕೆ ಕಳಂಕ’ ಎನ್ನುವರಲ್ಲವೇ? ಹೌದು, ಶಕ್ತನಿಗೆ ಎಲ್ಲರೂ ಆಪ್ತರು. ಅಶಕ್ತನನ್ನು ಎಲ್ಲರೂ ತಿರಸ್ಕರಿಸುವರು, ಮೂಲೆಗೆ ತಳ್ಳುವರು. ದೇಹ ಮನಸ್ಸು ಸರಿಯಾಗಿರುವ ನಾವು ನಮಗೆ ಕೊಡ ಮಾಡಿದ ಅವಕಾಶಗಳನ್ನು ಸರಿಯಾಗಿ ಉಪಯೋಗಿಸಿಕೊಂಡು ನಮ್ಮ ಶಕ್ತಿಯನ್ನು ಸರ್ವತೋಮುಖವಾಗಿ ಹೆಚ್ಚಿಸಿಕೊಳ್ಳಬೇಕಲ್ಲವೇ? ದೈಹಿಕ, ಮಾನಸಿಕ, ಬೌದ್ಧಿಕ, ನೈತಿಕ ಬಲಗಳ ಜೊತೆಗೆ ನಾವು ಆತ್ಮಶಕ್ತಿಯನ್ನೂ ಬೆಳೆಸಿಕೊಳ್ಳಬೇಕು. ‘ಶಕ್ತಿಯೇ ಜೀವನ, ದುರ್ಬಲತೆಯೇ ಮರಣ.’ ಇದೊಂದು ದೊಡ್ಡ ಸತ್ಯ. ‘ಶಕ್ತಿ ಸೌಭಾಗ್ಯ, ಚಿರಜೀವನದ ಒಳಗುಟ್ಟು. ದುರ್ಬಲತೆ–ಚಿಂತೆ, ದುಃಖಗಳ ಮೂಲ’ ಎಂದ ಸ್ವಾಮಿ ವಿವೇಕಾನಂದರು ‘ಬಲಿಷ್ಠರಾಗಿ, ಕಬ್ಬಿಣದ ಮಾಂಸಖಂಡ, ಉಕ್ಕಿನ ನರಮಂಡಲ, ವಿದ್ಯುತ್ತಿನ ಇಚ್ಛಾಶಕ್ತಿಯ ಬಲ–ಇವುಗಳನ್ನು ಬೆಳೆಸಿಕೊಂಡು ನಿಮ್ಮ ಕಾಲ ಮೇಲೆ ನಿಲ್ಲಿ’ ಎಂದು ತಿರುತಿರುಗಿ ಸಾರಿದರು. ದುರ್ಬಲನು ತಾನು ನಾಶವಾಗುವುದಲ್ಲದೇ ಇತರರಲ್ಲಿ ಶೋಷಣೆ ಮತ್ತು ಹಿಂಸೆಯ ಪ್ರವೃತ್ತಿಯನ್ನು ಹುಟ್ಟಿಸುತ್ತಾನೆ. ನಮ್ಮ ಪರಂಪರೆ ನೀಡುವ ಮುಖ್ಯ ಬೋಧನೆಯೇ ಶಕ್ತಿಯನ್ನು ಬೆಳೆಸಿಕೊಳ್ಳಿ ಎಂಬುದು. ನಿರ್ವೀರ್ಯನಾಗಬೇಡ, ದುರ್ಬಲನಾಗಬೇಡ ಎಂದು ಶ‍್ರೀಕೃಷ್ಣ ಅರ್ಜುನನಿಗೆ ಬೋಧಿಸಿದ. ದೇವತೆಗಳೂ ದುರ್ಬಲರನ್ನೇ ನಾಶ ಮಾಡುತ್ತಾರೆ, ಬಲಿಷ್ಠರನ್ನಲ್ಲ ಎಂಬ ಅರ್ಥಪೂರ್ಣ ಉಕ್ತಿ ಇದೆ. ‘ಕುದುರೆಯನ್ನು ಬಲಿ ಕೊಡುವುದಿಲ್ಲ, ಆನೆಯನ್ನು ಬಲಿಕೊಡುವುದಿಲ್ಲ. ಹುಲಿಯನ್ನಂತೂ ಇಲ್ಲವೇ ಇಲ್ಲ. ಆದರೆ ಮೇಕೆಯನ್ನು ಹಿಡಿದು ಬಲಿಕೊಡುತ್ತಾರೆ. ಅಯ್ಯೋ! ವಿಧಿಯು ದುರ್ಬಲರನ್ನು ನಾಶಗೊಳಿಸುತ್ತದೆ!’

ನಮ್ಮನ್ನು ದುರ್ಬಲಗೊಳಿಸುವ ನೂರಾರು ಕತೆಗಳು, ಘಟನೆಗಳು, ಚಿತ್ರಗಳು, ಧಾರ್ಮಿಕ ಮತ್ತು ರಾಜಕೀಯದ ರಾಗದ್ವೇಷಗಳು–ನಿತ್ಯವೂ ಪ್ರಸಾರವಾಗುತ್ತಿರಬಹುದು. ಅವು ಯುವಕರ ಮನಸ್ಸನ್ನು ತಲ್ಲಣಗೊಳಿಸಲೂಬಹುದು. ಆದರೆ ಬದುಕು ಎಷ್ಟೇ ನಿರಾಶಾದಾಯಕವಾದರೂ ಅದನ್ನು ದಾಟಿ ಮೇಲೇರಲು ಬೇಕಾದ ಶಕ್ತಿ ನಮ್ಮಲ್ಲೇ ಇದೆ, ನಮ್ಮ ಸಮೀಪದಲ್ಲೇ ಇದೆ, ನಮ್ಮ ಆಂತರ್ಯದ ಆಳದಲ್ಲೇ ಇದೆ. ಜಾತಿ ಮತ ಕುಲ ಗೋತ್ರ ಭೇದವಿಲ್ಲದೇ ಪ್ರತಿಯೊಬ್ಬ ಮನುಷ್ಯನ ಮನಸ್ಸಿನ ಆಳದಲ್ಲಿ ಸಾವಿಲ್ಲದ, ನೋವಿಲ್ಲದ ದೈವೀಶಕ್ತಿ ಇದೆ. ಮನೋವಿಜ್ಞಾನಿ ಇಂದು ಈ ಮಾತನ್ನು ಹೇಳುವ ಸ್ಥಿತಿಯಲ್ಲಿದ್ದಾನೆ. ಒಟ್ಟಿನಲ್ಲಿ ನಮಗೆ ಬೇಕಾದ ಮಾರ್ಗದರ್ಶನ, ಶಕ್ತಿ, ಸಹಾಯ, ಸ್ಫೂರ್ತಿ, ಶಾಂತಿ ಇವೇ ಮೊದಲಾದ ಬದುಕನ್ನು ಬೆಳಗುವ ಅಂಶಗಳು ನಮ್ಮಲ್ಲಿವೆ. ಇಂಥ ಅಪಾರ ಶಕ್ತಿಯನ್ನು ಒಳಗಡೆ ಇರಿಸಿಕೊಂಡು ಗುಲಾಮರಂತೆ ವರ್ತಿಸಲು ಕಾರಣವೇನು? ಒಂದು: ಅಜ್ಞಾನ. ಇನ್ನೊಂದು: ಬಾಲ್ಯದಿಂದ ಬೆಳೆಸಿಕೊಂಡು ಬಂದ ನಿಷೇಧಾತ್ಮಕ ಭಾವನೆ. ಪ್ರತಿಯೊಂದು ನಿಷೇಧಾತ್ಮಕ ಭಾವನೆ ಅಥವಾ ಯೋಚನೆ, ಒಳಗಿನ ಆ ಅಪಾರ ವಾದ ಶಕ್ತಿಯನ್ನು ಪರಿಮಿತಗೊಳಿಸಿ ನಮ್ಮನ್ನು ದುರ್ಬಲರನ್ನಾಗಿ ಮಾಡುತ್ತದೆ. ಪ್ರತಿಯೊಂದು ಆತ್ಮವಿಶ್ವಾಸದ ಮತ್ತು ರಚನಾತ್ಮಕ ಯೋಚನೆಯು ನಮ್ಮ ಅಭ್ಯುದಯ ಅಭಿವೃದ್ಧಿಗೆ ನೆರವಾಗುತ್ತದೆ.

ಈ ವಿವೇಕವಾಣಿಯನ್ನು ಮತ್ತೆ ಮತ್ತೆ ಮನನ ಮಾಡಿ:

‘ನಿಮ್ಮ ಭವಿಷ್ಯವನ್ನು ನಿರ್ಮಿಸಿಕೊಳ್ಳುವವರು ನೀವೇ ಎಂಬುದನ್ನು ಮರೆಯಬೇಡಿ. ಎಲ್ಲ ಹೊಣೆಯನ್ನು ಹೊರಲು ಸಿದ್ಧರಾಗಿ, ಮಿಂಚಿಹೋದ ಕಾರ್ಯವನ್ನು ಕುರಿತು ಚಿಂತಿಸುತ್ತ ವ್ಯರ್ಥ ಕಾಲಹರಣ ಬೇಡ. ಅನಂತ ಭವಿಷ್ಯ ನಿಮ್ಮ ಮುಂದೆ ಇದೆ. ನೀವು ಈ ಸಂಗತಿಯನ್ನು ಸದಾ ನೆನಪಿನಲ್ಲಿಡಬೇಕು. ನಿಮ್ಮ ಪ್ರತಿಯೊಂದು ಯೋಚನೆ, ಮಾತು ಮತ್ತು ಕೆಲಸ–ಇವೇ ನಿಮ್ಮ ಭವಿಷ್ಯತ್ತಿನ ಬೆಳಸಿನ ಬುತ್ತಿಯಾಗಿ ಪರಿಣಮಿಸುವುವು. ಕೆಟ್ಟ ಯೋಚನೆ, ಕೆಟ್ಟ ಕೆಲಸಗಳು ಕ್ರೂರ ಹುಲಿಯಂತೆ ನಿಮ್ಮ ಮೇಲೆ ನೆಗೆಯಲು ಸಿದ್ಧವಾಗಿದ್ದರೆ, ಒಳ್ಳೆಯ ಯೋಚನೆ ಮತ್ತು ಒಳ್ಳೆಯ ಕಾರ್ಯಗಳು ನೂರಾರು, ಸಾವಿರಾರು ದೇವತೆಗಳ ಸಾತ್ತ್ವಿಕ ಶಕ್ತಿಯಿಂದ ನಿಮ್ಮನ್ನು ಸದಾ ರಕ್ಷಿಸಲು ಸಿದ್ಧವಾಗಿರುತ್ತವೆ ಎಂಬ ದೃಢವಿಶ್ವಾಸವು ನಿಮ್ಮ ಪಾಲಿಗೆ ಸತ್ಕಾರ್ಯದ ಮಹಾ ಪ್ರೇರಕ ಶಕ್ತಿಯಾಗುತ್ತವೆ.’

ಈ ಮಾತುಗಳು ಶಕ್ತಿ ಸಂಜೀವಿನಿಯ ಕಿಡಿಗಳಲ್ಲವೆ? ಸತ್ಕರ್ಮಕ್ಕೆ ಪ್ರೇರಣೆ ನೀಡುವ ಸ್ಫೂರ್ತಿಯ ನಿಧಿಯಲ್ಲವೇ? ಇವುಗಳನ್ನು ಗಾಳಿಗೆ ತೂರಿ ನಾವು ಅಭ್ಯುದಯ, ಅಭಿವೃದ್ಧಿಯ ಪಥದಲ್ಲಿ ಮುನ್ನಡೆಯಬಲ್ಲೆವೇ?

ನಿಮಗೆ ಈ ವಿಚಾರ ತಿಳಿದಿದೆಯೇ?–

‘ಪ್ರತಿಯೊಂದು ದೈಹಿಕ ಸಂವೇದನೆ ಹಾಗೂ ಬಾಹ್ಯ ಪ್ರೇರಣೆಗೆ ನಿಮ್ಮ ಪ್ರತಿಕ್ರಿಯೆ–ಇವು ನಿಮ್ಮ ಮಿದುಳಿನ ಹತ್ತು ಸಾವಿರ ಮಿಲಿಯ ಕಣಗಳ ಮೇಲೆ ತಮ್ಮ ಮುದ್ರೆಯನ್ನು ಒತ್ತುತ್ತವೆ. ಈ ಪ್ರಭಾವ ಶಾಶ್ವತವಾಗಿದ್ದು ದಿನೇ ದಿನೇ ಬೆಳೆಯುತ್ತ ಹೋಗುತ್ತದೆ. ಅವುಗಳ ಒಟ್ಟು ಮೊತ್ತವೇ ನಿಮ್ಮ ವ್ಯಕ್ತಿತ್ವ ಹಾಗೂ ನಡತೆ.’

‘ನಮ್ಮ ತಪ್ಪುಗಳನ್ನು ದೇವರು ಕ್ಷಮಿಸಿಯಾನು! ಆದರೆ ನಮ್ಮ ನರಮಂಡಲ ಕ್ಷಮಿಸಲಾರದು’\footnote{\engfoot{The Lord may forgive our sins, but the nervous system never does.}

\engfoot{\general{~\hfill}–Willjiam James}} ಎಂದು ಮನೋವಿಜ್ಞಾನಿ ಹೇಳುತ್ತಾನೆ.

ಒಳಿತೋ ಕೆಡುಕೋ, ನಿಮ್ಮ ಅದೃಷ್ಟವನ್ನು ನೀವೇ ರೂಪಿಸಿಕೊಳ್ಳುತ್ತೀರಿ. ಆದುದರಿಂದ ಸ್ಯಾಮ್ಯುಯೆಲ್ ಸ್ಮಾೖಲ್ಸ್ ಹೇಳುವಂತೆ ‘ಒಂದು ಒಳ್ಳೆಯ ಯೋಚನೆ ಮಾಡಿ–ಅದು ನಿಮ್ಮನ್ನು ಒಂದು ಒಳ್ಳೆಯ ಕೆಲಸಕ್ಕೆ ಪ್ರೇರೇಪಿಸೀತು. ಒಂದು ಒಳ್ಳೆಯ ಕಾರ್ಯವನ್ನು ಕೈಗೊಳ್ಳಿರಿ–ಅದು ನಿಮ್ಮನ್ನು ಒಂದು ಒಳ್ಳೆಯ ಅಭ್ಯಾಸಕ್ಕೆಳೆಸೀತು. ಒಂದು ಒಳ್ಳೆಯ ಅಭ್ಯಾಸವನ್ನು ರೂಢಿಸಿಕೊಳ್ಳಿ –ಅದು ನಿಮ್ಮ ನಡತೆಗೆ ಪುಟಗೊಟ್ಟೀತು. ಒಳ್ಳೆಯ ನಡತೆಯಿಂದ ನಿಮ್ಮ ಅದೃಷ್ಟ ಕುದುರಲೇ ಬೇಕು.’ ಅಭ್ಯಾಸದ ಹಿರಿಮೆಗರಿಮೆಯೇ ಇಲ್ಲಿ ಧ್ವನಿತವಾಗಿದೆ!

ಇನ್ನಾದರೂ ನೀವು ಸ್ವಲ್ಪ ಜಾಗರೂಕರಾಗುತ್ತೀರಾ? ಅದ್ಭುತಯಂತ್ರವಾದ ನಿಮ್ಮ ದೇಹ ಮನಸ್ಸುಗಳಲ್ಲಿ ಅಡಗಿರುವ ಸುಪ್ತಶಕ್ತಿಯನ್ನು ಎಚ್ಚರಿಸಿ ಅದರಿಂದ ಹೆಚ್ಚಿನ ಪ್ರಯೋಜನ\break ಪಡೆಯುವಿರಾ? ಹೌದು, ಆಗ ನೀವು ತೀವ್ರ ಬದಲಾವಣೆಯನ್ನು ನಿಮ್ಮ ಬದುಕಿನಲ್ಲಿ ತಂದು\-ಕೊಳ್ಳುವಿರಿ. ಹೇಗೆ ಎನ್ನುವಿರಾ? ಒಂದು ದಾರಿ ಇದೆ.

ಕ್ರಮಬದ್ಧತೆಯ ನಿಯಮವನ್ನು ಅನ್ವಯಿಸುವುದರಿಂದ ನಮ್ಮಲ್ಲಿನ ಸುಪ್ತಶಕ್ತಿಯನ್ನು\break ಎಚ್ಚರಿಸಬಹುದು. ನಿತ್ಯ ಜೀವನದ ಎಲ್ಲ ಭಾಗಕ್ಕೂ ಕ್ರಮಬದ್ಧತೆಯ ನಿಯಮವನ್ನು ಅನ್ವಯಿಸಬೇಕು. ಗೊತ್ತುಪಡಿಸಿಕೊಂಡ ಸಮಯದಲ್ಲಿ ನಿರ್ದಿಷ್ಟ ಕೆಲಸ ಮಾಡುವ ಕಲೆ ಅದು. ಕೆಲವರಿಗೆ ಈ ನಿಯಮ ಬಾಲ್ಯದಿಂದಲೇ ಸ್ವಭಾವ ಸಿದ್ಧವಾಗಿರಬಹುದು. ಇನ್ನು ಕೆಲವರು ಪ್ರಯತ್ನದಿಂದ ಈ ಕಲೆಯನ್ನು ಕಲಿಯಬೇಕಾಗಬಹುದು.

ಹಾಕಿಕೊಂಡ ನಿಯಮಗಳನ್ನು ನಿತ್ಯನಿಯಮಿತ ರೀತಿಯಲ್ಲಿ ಪಾಲಿಸುವುದೇ ನಿಷ್ಠೆ. ನಿಷ್ಠೆ\-ಯಿಲ್ಲದೇ ಯಾವ ಪ್ರಗತಿಯೂ ಇಲ್ಲ. ನಮ್ಮ ಜ್ಞಾನಸಂಗ್ರಹವು ನಿಜವಾದ ಶಕ್ತಿಯಾಗಿ ಪರಿಣಮಿಸ\-ಬೇಕಾದರೆ ನಿಯಮಪಾಲನೆಯಲ್ಲಿ ದೃಢನಿಷ್ಠೆ ಇರಬೇಕು. ಅನಿಯಮಿತ ಜೀವನದಿಂದ ಯಾವ ಕಾರ್ಯದಲ್ಲೂ ಯಶಸ್ವಿಯಾಗಲಾರೆವು. ನಿಮ್ಮ ಶಕ್ತಿ ವೃದ್ಧಿಸಬೇಕಾದರೆ ನಿಯಮಪಾಲನೆ ಮಾಡ\-ಬೇಕಾದವರು ನೀವೇ.

ಹೌದು, ನಿಮಗೆ ನೀವೇ. ನೀವೇ ನಿಮ್ಮ ಶತ್ರುಗಳಾಗಬಲ್ಲಿರಿ, ನೀವೇ ನಿಮ್ಮ ಮಿತ್ರರೂ ಆಗಬಲ್ಲಿರಿ. ಹೇಗೆನ್ನುವಿರಾ? ಮುಂದೆ ಓದಿ.


\section*{ಯೋಚನೆಯ ಪತ್ತೆದಾರ}

\addsectiontoTOC{ಯೋಚನೆಯ ಪತ್ತೆದಾರ}

ರಷ್ಯಾ ದೇಶದ ವೂಲ್ಫ್ ಮೆಸ್ಸಿಂಗ್ ತನ್ನ ಅತೀಂದ್ರಿಯ ಶಕ್ತಿಗಾಗಿ ಪ್ರಸಿದ್ಧನಾಗಿದ್ದ. ಇತರರ ಮನಸ್ಸಿನಲ್ಲಿ ಏನಿದೆ? ಎಂಬುದನ್ನು ಅವನು ಸರಿಯಾಗಿ ಸ್ಪಷ್ಟವಾಗಿ ಓದಬಲ್ಲ ಸಿದ್ಧಿಯನ್ನು ಪಡೆದಿದ್ದ. ಸರ್ವಾಧಿಕಾರಿ ಸ್ಟ್ಯಾಲಿನ್​ನಿಂದ ಪರೀಕ್ಷಿತನಾಗಿ ಸೈ ಎನ್ನಿಸಿಕೊಂಡವನಾತ. ಜಗತ್ತಿನ ಅಂದಿನ ಪ್ರಮುಖರಲ್ಲಿ ಐನ್​ಸ್ಟೀನ್, ಫ್ರಾೖಡ್ ಮತ್ತು ಗಾಂಧೀಜಿ–ಇವರನ್ನು ಭೇಟಿಯಾಗಿ ತನ್ನ ಸಿದ್ಧಿಯನ್ನು ಪ್ರದರ್ಶಿಸಿ ಅವರನ್ನು ಬೆರಗುಗೊಳಿಸಿದ ಆತ, ೧೯೨೭ರಲ್ಲಿ ಭಾರತಕ್ಕೆ ಬಂದಾಗ ಸಬರಮತಿ ಆಶ್ರಮದಲ್ಲಿ ಗಾಂಧೀಜಿಯನ್ನು ಕಂಡು ಮೌನವಾಗಿ ಅವರು ನೀಡಿದ ಆದೇಶವನ್ನು ಮಾಡಿ ತೋರಿಸಿದವನು. ಎರಡನೇ ಮಹಾ ಯುದ್ಧದ ಹೊತ್ತಿಗೆ ಹಿಟ್ಲರನ ಮೃತ್ಯುಪಾಶವನ್ನು ಹೇಗೋ ತಪ್ಪಿಸಿಕೊಂಡು ಪೋಲೆಂಡನ್ನೂ ಬಿಟ್ಟು ರಷ್ಯಾಕ್ಕೆ ಬಂದು, ಸಾರ್ವಜನಿಕ ಪ್ರದರ್ಶನಗಳನ್ನು ನೀಡುತ್ತಿದ್ದರೂ, ವೈಯಕ್ತಿಕ ಹಾಗೂ ರಾಜಕೀಯ ಭವಿಷ್ಯಗಳನ್ನು ಕುರಿತು ಆತ ಹೇಳುತ್ತಿರಲಿಲ್ಲ. ನೆರೆದ ಜನರಲ್ಲಿ ಯಾರಾದರೂ ಮೌನವಾಗಿದ್ದುಕೊಂಡು ಆತನಿಗೆ ಮಾನಸಿಕವಾಗಿ ಆಜ್ಞೆ ನೀಡಿದರೆ, ಅದನ್ನವನು ಗ್ರಹಿಸಿಕೊಂಡು ಅಂತೆಯೇ ವರ್ತಿಸುತ್ತಿದ್ದ. ‘ನಾನು ಮಾಡಬೇಕೆಂದು ನೀವು ಹೇಳಲು ಬಯಸಿರುವ ಅಥವಾ ಸಂಕಲ್ಪಿಸಿರುವ ಆಜ್ಞೆಯನ್ನು ಮಾತ್ರ ಮನಸ್ಸಿನಲ್ಲಿ ಸ್ಪಷ್ಟವಾಗಿ ಚಿಂತಿಸಿ’ ಎಂದು ಪರೀಕ್ಷಕರಲ್ಲಿ ನಿವೇದಿಸಿಕೊಳ್ಳುವುದು ಅವನು ಅನುಸರಿಸುತ್ತಿದ್ದ ಕ್ರಮ. ಕೆಲವು ವರ್ಷಗಳ ಹಿಂದೆ ವೈದ್ಯಕೀಯ ವಿದ್ವಾಂಸರ ಮತ್ತು ವಿಜ್ಞಾನಿಗಳ ಎದುರಲ್ಲಿ ಆತ ಒಂದು ಪ್ರದರ್ಶನವನ್ನಿತ್ತ. ಪ್ರದರ್ಶನಕ್ಕೆ ಮೊದಲು ಯಾರ ಮುಖಭಾವವನ್ನೂ ಪರಿಶೀಲಿಸಬಾರ ದೆಂದು ಅವನ ಕಣ್ಣಿಗೆ ಬಟ್ಟೆ ಕಟ್ಟಿದ್ದರು. ಸಭೆಯಲ್ಲಿದ್ದ ಒಬ್ಬ ಡಾಕ್ಟರ್ ಮೊದಲೇ ನಿಶ್ಚಯ ಮಾಡಿಕೊಂಡಂತೆ ಮೌನವಾಗಿದ್ದುಕೊಂಡು ಒಂದು ಸೂಚನೆಯನ್ನು ಮನಸ್ಸಿನಲ್ಲೇ ಏಕಾಗ್ರತೆ ಯಿಂದ ಮೆಲುಕು ಹಾಕುತ್ತಿದ್ದರು. ಮೆಸ್ಸಿಂಗ್ ಮೊದಲು ಮೌನವಾಗಿದ್ದು ಅದನ್ನು ಗ್ರಹಿಸಿ, ಬಳಿಕ ಕಣ್ಣಿನ ಬಟ್ಟೆಯನ್ನು ತೆಗೆದುಹಾಕಿ ನಾಲ್ಕನೇ ನಂಬರಿನ ಕುರ್ಚಿಯ ಬಳಿ ನಿಂತುಕೊಂಡ. ಅಲ್ಲಿ ಕುಳಿತ ವ್ಯಕ್ತಿಯ ಕೋಟಿನ ಬಲಕಿಸೆಯಿಂದ ಒಂದು ಸ್ಪಾಂಜನ್ನೂ, ಕತ್ತರಿಯನ್ನೂ ತೆಗೆದುಕೊಂಡು ನೆರೆದ ಜನರಿಗೆ ತೋರಿಸಿ ‘ಸ್ಪಾಂಜನ್ನು ಕತ್ತರಿಸಲು ಇಷ್ಟವಿಲ್ಲ’ ಎಂದು ಹೇಳಿ ಸೀಮೆಸುಣ್ಣದಿಂದ ಒಂದು ನಾಯಿಯ ಚಿತ್ರವನ್ನು ಆ ಸ್ಪಾಂಜಿನ ಮೇಲೆ ಬರೆದ.

ಈ ಪ್ರಯೋಗದಲ್ಲಿ ಮುಖ್ಯ ತೀರ್ಪುಗಾರರಾಗಿ ಬಂದಿದ್ದ ಹಿರಿಯರು ಮಾನಸಿಕ ಆಜ್ಞೆಯನ್ನು ನೀಡಿದ ಡಾಕ್ಟರನ್ನು ವಿಚಾರಿಸಿದರು. ಆಜ್ಞೆಯನ್ನು ನೀಡಿದ ಡಾಕ್ಟರ್ ತಾನು ನಾಲ್ಕನೇ ನಂಬರಿನ ಕುರ್ಚಿಯಲ್ಲಿ ಕುಳಿತ ತನ್ನ ಸ್ನೇಹಿತನ ಕೋಟಿನ ಜೇಬಿನಲ್ಲಿರುವ ಸ್ಪಾಂಜನ್ನು ನಾಯಿಯ ಆಕಾರದಲ್ಲಿ ಕತ್ತರಿಸುವಂತೆ ಕಲ್ಪಿಸಿದ್ದು ನಿಜವೆಂದು ಒಪ್ಪಿಕೊಂಡ. ಮೆಸ್ಸಿಂಗ್ ಅವನ ಕಲ್ಪನೆಯನ್ನು ಓದಿ ಅಂತೆಯೇ ನಡೆದಿದ್ದ. ಆದರೆ ಒಂದು ಸಣ್ಣ ಬದಲಾವಣೆ ಅವರನ್ನು ಕೇಳಿ ಮಾಡಿದ್ದ. ಎಲ್ಲರೂ ಮೆಸ್ಸಿಂಗನ ಅದ್ಭುತ ಸಿದ್ಧಿಯನ್ನು ಕಂಡು ಚಕಿತರಾದರು. ಮೆಸ್ಸಿಂಗ್ ಈ ಒಂದು ಪ್ರದರ್ಶನವನ್ನು ಮಾತ್ರ ನೀಡಿದುದಲ್ಲ ಎಂಬುದು ನೆನಪಿನಲ್ಲಿಡಬೇಕಾದ ಸಂಗತಿ.

ಹಲವರು ಮೆಸ್ಸಿಂಗನನ್ನು ಮುತ್ತಿಕೊಂಡು ‘ಇತರರ ಮನಸ್ಸಿನಲ್ಲಿರುವ ಯೋಚನೆಗಳನ್ನು ನೀನು ಹೇಗೆ ಓದುತ್ತಿ?’ ಎಂದು ಕೇಳಿದರು. ಮೆಸ್ಸಿಂಗನ ಉತ್ತರ ಹೀಗಿದೆ:

‘ಜನರು ಸ್ಪಷ್ಟವಾಗಿ ಮಾಡಿಕೊಂಡ ಯೋಚನೆಗಳು, ನೀಡುವ ಆದೇಶಗಳು ನನ್ನ ಮನಸ್ಸಿಗೆ ಚಿತ್ರರೂಪದಲ್ಲಿ ಬರುತ್ತವೆ.’\footnote{\engfoot{People's thoughts come to me as pictures. I usually see visual images of a specific action. Thoughts of deaf and dumb are easier to get, probably because they think much more visually than the rest of us.}\hfill\engfoot{–Wolf Messing}}

ಏನಾಶ್ಚರ್ಯ! ನಮ್ಮ ಯೋಚನೆಗಳೆಲ್ಲ ಚಿತ್ರಗಳೇ!

ಏನು ಅವುಗಳಿಗೆ ಸಂಚಾರಶಕ್ತಿಯೂ ಇದೆಯೆ?

‘ಹೌದು’ ಎಂದಿದ್ದರು ನಮ್ಮ ದೇಶದ ಮನೀಷಿಗಳು. ‘ಹೌದು’ ಎನ್ನುತ್ತಿದ್ದಾರೆ ಈ ವಿಚಾರದಲ್ಲಿ ಸಂಶೋಧನೆ ಮಾಡಿದ ಆಧುನಿಕ ಪ್ರಯೋಗ ಪರಿಣತರು.

‘ಹೌದು’ ಎಂದಿದ್ದರು ನಮ್ಮ ದೇಶದ ಮನೀಷಿಗಳು. ‘ಹೌದು’ ಎನ್ನುತ್ತಿದ್ದಾರೆ ಈ ವಿಚಾರದಲ್ಲಿ ಸಂಶೋಧನೆ ಮಾಡಿದ ಆಧುನಿಕ ಪ್ರಯೋಗ ಪರಿಣತರು.


\section*{‘ಚಿತ್ರಗುಪ್ತ’ನಿದ್ದಾನೆ! ಎಚ್ಚರಿಕೆ!}

\addsectiontoTOC{‘ಚಿತ್ರಗುಪ್ತ’ನಿದ್ದಾನೆ ! ಎಚ್ಚರಿಕೆ !}

ನಾವು ಯೋಚಿಸುವುದು ಚಿತ್ರದ ಮೂಲಕ. ಶಬ್ದಗಳು ಚಿತ್ರಗಳನ್ನು ನೆನಪಿಗೆ ತರುವ ಸಂಕೇತಗಳು ಅಷ್ಟೇ. ‘ಮರ’ ಎಂದು ಉಚ್ಚರಿಸಿದಾಗಲೇ ಮರದ ಚಿತ್ರ ಮನಸ್ಸಿನಲ್ಲಿ ಮೂಡುತ್ತದೆ. ‘ಆಲದ ಮರ, ಮಾವಿನ ಮರ’ ಎಂದಾಗ ಚಿತ್ರಗಳಲ್ಲಿ ವೇಗವಾಗಿ ಉಂಟಾಗುವ ಬದಲಾವಣೆಯನ್ನು ನಾವು ಗ್ರಹಿಸಬಹುದು.

‘ನಾನು ಡಾಕ್ಟರ್ ಆಗುತ್ತೇನೆ. ದಿಲ್ಲಿಗೆ ಹೋಗುತ್ತೇನೆ, ಮನೆ ಕಟ್ಟುತ್ತೇನೆ, ವಿಜ್ಞಾನ ವಿಚಾರ ತಿಳಿಯುತ್ತೇನೆ, ಆಫೀಸರ್​ ಆಗುತ್ತೇನೆ, ಮದುವೆಯಾಗುತ್ತೇನೆ, ನಟನಾಗುತ್ತೇನೆ, ನಾನಾ ಊರುಗಳನ್ನು ಸಂಚರಿಸುತ್ತೇನೆ, ಶತ್ರುವಿಗೆ ಬುದ್ಧಿ ಕಲಿಸುತ್ತೇನೆ’ ಎಲ್ಲವೂ ಮಾನಸಿಕ ಚಿತ್ರಗಳೇ! ಮನಸ್ಸು ಇವುಗಳ ಬಗೆಗೆ ಇನ್ನೂ ವಿವರಗಳನ್ನು ಕಲ್ಪಿಸುತ್ತ ಹೋಗುತ್ತದೆ.

ನಾವು ಬಾಲ್ಯದಿಂದಲೇ ಸಂಗ್ರಹಿಸುತ್ತಿರುವ ಅಸಂಖ್ಯ ಚಿತ್ರಗಳು ಅಥವಾ ಯೋಚನೆಗಳು ನಮ್ಮ ಮನಸ್ಸಿನಲ್ಲಿರುತ್ತವೆ. ಅವು ಮನಸ್ಸಿನ ಆಳದಲ್ಲಿ ಹುದುಗಿರುತ್ತವೆ. ಬೇಕಾದಾಗ ಮೇಲಕ್ಕೆ ಬರುತ್ತವೆ.

‘ಹಲಸಿನ ಹಣ್ಣು’ ಎಂದಾಗ ಹಣ್ಣಿನ ಚಿತ್ರ ಮತ್ತು ಬಣ್ಣದೊಂದಿಗೆ ಹಿಂದೆ ನೀವು ತಿಂದ ಹಣ್ಣಿನ ರುಚಿಯೂ, ಅದರ ಸುವಾಸನೆಯೂ ಮನಸ್ಸಿಗೆ ಬಂದು ಬಾಯಲ್ಲಿ ನೀರೂರುವುದಿಲ್ಲವೇ? ಎಂದರೆ ಕೇವಲ ಚಿತ್ರ ಮಾತ್ರವಲ್ಲ, ನೀವು ಪಡೆದ ಪ್ರತಿಯೊಂದು ಅನುಭವವೂ, ಪ್ರತಿ ಕ್ರಿಯೆಯೂ ಆ ಚಿತ್ರದೊಂದಿಗೆ ಅಡಗಿಕೊಂಡಿದೆ ಎಂದಾಯಿತು. ಇದನ್ನು ನಮ್ಮ ದೇಶದ ವಿಚಾರ ವಂತರು ‘ಸಂಸ್ಕಾರ’ ಎಂದು ಕರೆದರು.

ಎಂಥ ಅದ್ಭುತ ಟೇಪ್ ರೆಕಾರ್ಡರ್ ನಮ್ಮ ಮನಸ್ಸು!

ಎಷ್ಟೊಂದು ಸೂಕ್ಷ್ಮ ವಿಷಯಗಳನ್ನೂ ಅದು ಸಂಗ್ರಹಿಸುತ್ತದೆ!

ಅದು ಒಳ್ಳೆಯ ಭಾವಚಿತ್ರಗ್ರಾಹಿಯಾದ ಕ್ಯಾಮರಾ ಕೂಡ ಹೌದು!

ಕೇವಲ ಧ್ವನಿ ಮತ್ತು ಚಿತ್ರಗಳನ್ನು ಮಾತ್ರ ಅದು ಸಂಗ್ರಹಿಸುವುದಲ್ಲ! ವಿವಿಧ ರುಚಿಯ, ಸ್ಪರ್ಶ ಮತ್ತು ವಾಸನೆಗಳ ಅನುಭವವನ್ನೂ ವ್ಯವಸ್ಥಿತ ರೀತಿಯಲ್ಲಿ ಸಂಗ್ರಹಿಸಿಬಿಡುತ್ತದೆ!

ಬೇರೆ ಬೇರೆ ಸನ್ನಿವೇಶಗಳಲ್ಲಿ ನಾವು ಅನುಭವಿಸಿದ ಸುಖದುಃಖಗಳನ್ನೂ, ಅವುಗಳಿಗೆ ನಾವು ತೋರಿಸಿದ ಪ್ರತಿಕ್ರಿಯೆಗಳನ್ನೂ, ನಮ್ಮ ರಾಗ ದ್ವೇಷಗಳನ್ನೂ ಅದು ದಾಖಲೆ ಮಾಡುತ್ತದೆ.\break ಮನಸ್ಸೆಂಬ ಈ ಚಿತ್ರಗುಪ್ತನ ಕಣ್ಣು ತಪ್ಪಿಸಲು ಯಾರಿಗೂ ಸಾಧ್ಯವಿಲ್ಲ. ಇದು ಹಿಂದೆ ಸಂಗ್ರಹವಾದ ಅನುಭವಗಳ ದಾಖಲೆಗಳೊಂದಿಗೆ ನೂತನ ವಿಚಾರಗಳನ್ನೂ, ಅನುಭವಗಳನ್ನೂ ಸಂಗ್ರಹಿಸಿ, ಹಿಂದಿನ ಅನುಭವಗಳ ಸ್ಮೃತಿಯೊಂದಿಗೆ ನಿಯಮಿತ ರೀತಿಯಲ್ಲಿ ಅವುಗಳನ್ನು ಜೋಡಿಸಿಡುತ್ತದೆ!

ನಮ್ಮ ಮುಂದಿನ ಯೋಚನೆಗಳೂ, ನಾವು ಮುಂದೆ ಪಡೆಯಬಹುದಾದ ಅನುಭವಗಳೂ, ನಾವು ಆಗಲೇ ಪಡೆದಿರಿಸಿಕೊಂಡ ಯೋಚನೆ, ಅನುಭವಗಳು ಎಂಥವು ಎಂಬುದನ್ನು ಹೊಂದಿ ಕೊಂಡಿರುತ್ತವೆ.

ವಿದ್ಯಾರ್ಥಿಯೊಬ್ಬ ಏಳೆಂಟು ದಿನಗಳಿಂದ ಒಂದು ಸಿನೆಮಾ ನೋಡುತ್ತ ಬಂದ. ಸಿನೆಮಾ ಕತೆ ಯಲ್ಲಿ ಬರುವ ದರೋಡೆಕಾರನು ಬೀದಿಹೋಕರನ್ನು ಹೊಡೆದು ಹಿಂಸಿಸಿ ಸುಲಿಗೆ ಮಾಡುವ ಚಿತ್ರವನ್ನು ಏಕಾಗ್ರತೆಯಿಂದ ಮನಸ್ಸು ಕೊಟ್ಟು ನೋಡಿದ. ಕತೆಗೆ ಸಂಬಂಧಿಸಿದ ಮೂಲ ಲೇಖಕನ ಪುಸ್ತಕವನ್ನೂ ಮತ್ತೆ ಮತ್ತೆ ಓದಿದ. ಮನನ ಮಾಡಿದ. ಸಂದರ್ಭ ಸಿಕ್ಕಿದರೆ ತಾನೂ ಅಂಥ ಒಂದು ಸಾಹಸವನ್ನು ಏಕೆ ಕೈಗೊಳ್ಳಬಾರದು ಎಂಬ ಯೋಚನೆ ಬರತೊಡಗಿತು! ಮನಸ್ಸಿನಲ್ಲೇ ‘ಹೇಗೆ, ಏನು, ಎತ್ತ?’ ಎಂದು ಚಿತ್ರಿಸಿದ. ಕಾರ್ಯರೂಪಕ್ಕೂ ತಂದ. ಆದರೆ ಚಿತ್ರದಲ್ಲಿ ದರೋಡೆಕಾರನು ತಪ್ಪಿಸಿಕೊಂಡಂತೆ ಇವನು ತಪ್ಪಿಸಿಕೊಳ್ಳಲಾರದೇ ಪೋಲೀಸರ ಕೈಗೆ ಸಿಕ್ಕಿ, ಅವರ ಬೂಟ್ಸಿನ ಒದೆತ ತಿಂದು ತೆಪ್ಪಗಾದ.

ನಿಮ್ಮ ಭವಿಷ್ಯ ಜೀವನ ರೂಪಿತವಾಗುವುದು ನೀವು ಸೃಷ್ಟಿಸುವ ಚಿತ್ರಗಳಿಂದ. ಕೆಟ್ಟ ಚಿತ್ರಗಳನ್ನು ಚಿತ್ರಿಸುತ್ತ ಒಳ್ಳೆಯ ಬದುಕನ್ನು ರೂಪಿಸಲು ಸಾಧ್ಯವೇ? ನೀವೆಂಥ ಚಿತ್ರಗಳನ್ನು ರೂಪಿಸುತ್ತಿದ್ದೀರಿ ಎಂಬುದನ್ನು ಯೋಚಿಸಿದ್ದೀರಾ?

ಪುಷಿಗಳು ‘ಹೇ ದೇವತೆಗಳಿರಾ, ನಮ್ಮ ಕಣ್ಣುಗಳು ಮಂಗಳಕರವಾದ ಚಿತ್ರಗಳನ್ನು ನೋಡು ವಂತಾಗಲಿ, ನಮ್ಮ ಕಿವಿಗಳು ಒಳ್ಳೆಯ, ಶುಭಕರವಾದ ಶಬ್ದಗಳನ್ನು ಕೇಳುವಂತಾಗಲಿ’ ಎಂದು ಪ್ರಾರ್ಥಿಸಿದರು. ಈ ಪ್ರಾರ್ಥನೆ ಎಷ್ಟೊಂದು ಅರ್ಥಪೂರ್ಣವಾಗಿದೆ ಎಂದು ನಿಮಗನ್ನಿಸದಿರದು.

ಪುರಾಣಗಳಲ್ಲಿ ಮೃತ್ಯುದೇವತೆಯಾದ ಯಮನ ವರ್ಣನೆ ಬರುತ್ತದಷ್ಟೆ. ಯಮನ\break ಇನ್ನೊಂದು ಹೆಸರು ‘ಕಾಲ’ ಎಂದು. ಕಾಲನು ಸೂರ್ಯನ ಮಗ. ಕಾಲದ ಪ್ರಜ್ಞೆ ಸೂರ್ಯೋದಯ ಸೂರ್ಯಾಸ್ತಮಾನಗಳಿಂದ. ಕಾಲನ ಆಸರೆಯಲ್ಲೇ ಬದುಕಿ ಬಾಳಿ ಕಣ್ಮರೆಯಾಗುವವರು ನಾವು. ನಾವೆಲ್ಲರೂ ಕಾಲನ ಅಧೀನರು ಎಂದರೆ ತಪ್ಪಿಲ್ಲ. ನಮ್ಮ ಬದುಕಿನ ಎಲ್ಲ ಘಟನೆಗಳ ಮತ್ತು ಅನುಭವಗಳ ಚಿತ್ರಗಳನ್ನು ದಾಖಲೆ ಮಾಡುವವನು ಕಾಲನ ಕರಣಿಕ. ಅವನೇ ಚಿತ್ರಗುಪ್ತ. ನಾವು ಕಂಡ ದೃಶ್ಯಗಳನ್ನೆಲ್ಲ, ಪಡೆದ ಅನುಭವಗಳನ್ನೆಲ್ಲ, ಮಾಡಿದ ಸತ್ಕರ್ಮ, ದುಷ್ಕರ್ಮಗಳ ವಿಧಾನ ಗಳನ್ನೆಲ್ಲ, ಸುಪ್ತಮನಸ್ಸಿನಲ್ಲಿ ಗುಪ್ತರೂಪವಾಗಿಡುವವನು ಅವನು. ಯಮನ ಕರಣಿಕ ನಮ್ಮ ಕಣ್ಣಿಗೆ ಕಾಣಿಸುವಂಥವನಲ್ಲ. ಆದರೆ ನಮ್ಮ ‘ಮನಸ್ಸು’ ಎಂಬ ಚಿತ್ರಗುಪ್ತನ ಪರಿಚಯವನ್ನು ಎಲ್ಲರೂ ಕೊಂಚ ಯೋಚಿಸಿದರೆ ಪಡೆಯಬಹುದು. ಇಂದು ಮನೋವಿಜ್ಞಾನಿಗಳು ಇವನ ಬಗೆಗೆ ಹೆಚ್ಚು ತಿಳಿವು ನೀಡುತ್ತಿದ್ದಾರೆ.

ವ್ಯಕ್ತಿಯೊಬ್ಬನನ್ನು ಸುಪ್ತನಿದ್ರೆಗೊಳಪಡಿಸಿದರೆ ಅವನ ಬದುಕಿನಲ್ಲಿ ನಡೆದ ಪ್ರತಿಯೊಂದು ಅನುಭವದ ಸ್ಮೃತಿಯನ್ನೂ ಕೆದಕಿ ಜೀವಂತವಾಗಿಸಬಹುದೆಂಬುದನ್ನು ಮನೋವಿಜ್ಞಾನಿಗಳು\break ಪ್ರಯೋಗದ ಮೂಲಕ ಕಂಡುಕೊಂಡಿದ್ದಾರೆ. ಎಚ್ಚರದ ಈಗಿನ ಸ್ಥಿತಿಯಲ್ಲಿ ಹಳೆಯ ಸ್ಮೃತಿಗಳನ್ನು ನೆನಪಿಸಲು ಎಷ್ಟು ಯತ್ನಿಸಿದರೂ ಅವನ ಮನಸ್ಸಿಗೆ ಅವು ಬಾರದಿರಬಹುದು. ಆದರೆ ಸುಪ್ತಾವಸ್ಥೆಯಲ್ಲಿ ಅವನನ್ನು ಶೈಶವದ ದಿನಗಳ ಅನುಭವಗಳನ್ನು ವರ್ಣಿಸಲು ಕೇಳಿಕೊಂಡರೆ, ಸಾಮಾನ್ಯ ಎಚ್ಚರದ ಸ್ಥಿತಿಯಲ್ಲಿ ಅಸಾಧ್ಯವಾದ ಎಲ್ಲ ವಿವರಗಳನ್ನೂ ಆತ ಅತ್ಯಂತ ಸ್ಪಷ್ಟವಾಗಿ ನೀಡಬಲ್ಲ, ಜೀವಂತ ಅನುಭವಿಸುತ್ತಿರುವವನಂತೆ ವರ್ಣಿಸಬಲ್ಲ.

ಲಿಸ್ಲಿ ಲೇಕ್ರೊನ್ ಬರೆದ ‘ಪ್ರಾಯೋಗಿಕ ಸುಪ್ತಿ ಶಾಸ್ತ್ರ’ ನೀಡುವ ಅಸಂಖ್ಯ ಉದಾಹರಣೆಗಳಲ್ಲಿ ಒಂದನ್ನು ಗಮನಿಸಿ: ನಲ್ವತ್ತೈದು ವರ್ಷ ವಯಸ್ಸಿನ ವ್ಯಕ್ತಿಯನ್ನು ಸುಪ್ತಾವಸ್ಥೆಗೆ, ಮೆಲ್ಲಮೆಲ್ಲನೇ ಹಿಂದು ಹಿಂದಿನ ಅನುಭವಗಳ ವಿವಿಧ ಸ್ಮೃತಿಯ ಸ್ತರಗಳಿಗೆ ಕೊಂಡೊಯ್ಯ ಲಾಯಿತು. ಕೊನೆ ಯಲ್ಲಿ ಮೂರು ವರ್ಷ ವಯಸ್ಸಿನ ಅನುಭವಗಳ ಸ್ಮೃತಿ ಚಿತ್ರದ ಸ್ಥಿತಿಯೊಂದಿಗೆ ತಾದಾತ್ಮ್ಯವನ್ನು ಉಂಟು ಮಾಡಿದರು. ಆಗ ಆತನ ಉಸಿರಾಟದಲ್ಲಿ ವಿಚಿತ್ರ ಬದಲಾವಣೆ ಕಾಣಿಸ ತೊಡಗಿತು. ಗೂರಲು ಕಾಣಿಸಿತು. ಕೆಮ್ಮು ತೀವ್ರವಾಯಿತು. ಉಸಿರಾಡುವಾಗ ‘ಸೊಂಯ್, ಸೊಂಯ್​’ ಸದ್ದು ಪ್ರಾರಂಭವಾಯಿತು. ನಾಡಿಯ ಬಡಿತದ ವೇಗವು ತೀವ್ರವಾಗಿತ್ತು. ಆಸ್ತಮಾ ರೋಗದ ಎಲ್ಲ ಲಕ್ಷಣಗಳೂ ಅವನಲ್ಲಿ ಕಂಡವು. ಎಚ್ಚರದ ಸ್ಥಿತಿಯಲ್ಲಿ ಅವನಿಗೆ ಆ ರೋಗ ದಿಂದ ನರಳಿದ ನೆನಪೇ ಇರಲಿಲ್ಲ. ತಜ್ಞರು ಅವನ ತಾಯಿಯನ್ನು ಕಂಡು ಮಾತನಾಡಿದಾಗ ಅವನ ಮೂರನೆಯ ವಯಸ್ಸಿನಲ್ಲಿ ಅವನಿಗೆ ಆ ರೋಗ ಬಂದಿತ್ತೆಂದು ಅವಳಾಗಿಯೇ ಹೇಳಿದಳು.

ಏನಾಶ್ಚರ್ಯ! ಅನುಭವಿಸಿದ ಸುಖದುಃಖಗಳೆಲ್ಲವುಗಳ ಸ್ಮೃತಿಚಿತ್ರ ಸುಪ್ತ ಮನಸ್ಸಿನಲ್ಲಿ ಗುಪ್ತರೂಪದಿಂದ ಅಡಗಿದೆ ಎಂದಾಯಿತು.

ವಿಜ್ಞಾನಿಗಳಿಗೆ ಸವಾಲಾದ ಇಂತಹ ಇನ್ನೊಂದು ಘಟನೆ ‘ಸೂಪರ್ ಸೈಕ್​’ ಎನ್ನುವ ಗ್ರಂಥದಲ್ಲಿ ಪ್ರಕಟವಾಗಿದೆ–

ಅಮೇರಿಕಾದ ವೈದ್ಯರೊಬ್ಬರನ್ನು ಸುಪ್ತಾವಸ್ಥೆಯಲ್ಲಿ ಅವರ ಜನನದ ಸಮಯದ ದೃಶ್ಯವನ್ನು ವಿವರಿಸುವಂತೆ ಕೇಳಿದಾಗ, ಅವರ ತಾಯಿ ಆಸ್ಪತ್ರೆಯಲ್ಲಿ ಮಲಗಿದ್ದುದನ್ನೂ, ವೈದ್ಯರು ಆಕೆಯ ಬಲಪಾರ್ಶ್ವದಲ್ಲಿ ನಿಂತುದನ್ನೂ ಕಣ್ಣಿಗೆ ಕಟ್ಟುವಂತೆ ವಿವರಿಸಿದ್ದರು. ಮಾತ್ರವಲ್ಲ, ‘ಹೆಚ್ಚು ಕಾಲ ವ್ಯಯ ಮಾಡುವುದು ಸರಿಯಲ್ಲ, ಮಗು ಉಳಿಯುತ್ತದೆಂಬ ಭರವಸೆ ನನಗಿಲ್ಲ’ ಎಂಬುದಾಗಿ ವೈದ್ಯರು ಹೇಳಿದ ನುಡಿಗಳನ್ನೂ ಪುನರುಚ್ಚರಿಸಿದರು.

ನಿಜವಾಗಿಯೂ ಅವರು ಗರ್ಭದಲ್ಲಿ ಪೂರ್ಣ ಬೆಳವಣಿಗೆಯಾಗುವುದಕ್ಕೆ ಆರು ವಾರಗಳ ಮೊದಲೇ 1.6 ಕೆ. ಜಿ. ತೂಕವನ್ನಷ್ಟೇ ಪಡೆದು ಜನಿಸಿದ್ದರು.

ಸುಪ್ತಮನಸ್ಸಿನಲ್ಲಿ ಹುದುಗಿದ ಗುಪ್ತವಿಚಾರಗಳನ್ನು ಕುರಿತು ಇಂಥ ನೂರಾರು ಉದಾಹರಣೆ ಗಳಿವೆ. ಅವುಗಳನ್ನೆಲ್ಲಾ ಉಲ್ಲೇಖಿಸಲು ಅಸಾಧ್ಯವಾದರೂ ಮನಸ್ಸು ಚಿತ್ರಗುಪ್ತ ಎನ್ನುವ ಬಗೆಗೆ ನಿಮಗೆ ಖಾತ್ರಿಯಾಗಿರಬೇಕು. ಒಟ್ಟಿನಲ್ಲಿ–

ಮನುಷ್ಯನು ಯೋಚಿಸುವುದು ಮಾನಸಿಕ ಚಿತ್ರದ ಮೂಲಕ. ನಮ್ಮ ಜೀವನದಲ್ಲಿ ಪಡೆದ ಅನುಭವಗಳ ಪ್ರತಿಯೊಂದು ಮಾನಸಿಕ ಚಿತ್ರದ ಜೊತೆಗೆ ತತ್ಕಾಲದ ಒಂದು ಭಾವನೆಯೂ ನಮ್ಮ ಪ್ರಜ್ಞೆಯಲ್ಲಿ, ಮನಸ್ಸಿನ ಆಳದಲ್ಲಿ ದಾಖಲಾಗಿರುತ್ತದೆ. ಪ್ರತಿಯೊಬ್ಬ ವ್ಯಕ್ತಿಯೂ ಇಂದು ಏನಾಗಿರು\-ವನೋ ಅದು ಅವನಿಗೊದಗಿ ಬಂದ ಅನುಭವಗಳಿಗೆ ಅವನು ತೋರಿಸಿದ ವಿಭಿನ್ನ ಬೌದ್ಧಿಕ ಮತ್ತು ಭಾವನಾತ್ಮಕ ಪ್ರತಿಕ್ರಿಯೆಗಳ ಮೊತ್ತವೇ.

ಒಬ್ಬನು ಎಂಥ ವ್ಯಕ್ತಿ, ಏನನ್ನು ಪ್ರತಿನಿಧಿಸುತ್ತಾನೆ, ಎಂಬುದು ಅವನು ಸದ್ಯ ತನ್ನ ಬಗ್ಗೆ ಏನೆಂದು ಕಲ್ಪಿಸಿಕೊಂಡಿರುತ್ತಾನೆ, ಜೀವನವನ್ನೂ, ಇತರರನ್ನೂ ಯಾವ ದೃಷ್ಟಿಯಿಂದ ನೋಡುತ್ತಾನೆ, ಎಂಬುದನ್ನು ಹೊಂದಿಕೊಂಡಿದೆ.


\section*{ಯೋಚನೆಯೇ ರೂವಾರಿ}

\addsectiontoTOC{ಯೋಚನೆಯೇ ರೂವಾರಿ}

ಮನುಷ್ಯ ಜೀವನದ ಮೇಲೆ ಪ್ರಭಾವ ಬೀರುವ ಅತ್ಯಂತ ಪ್ರಮುಖವಾದ ಸಂಗತಿಗಳು – ಯೋಚನೆಗಳು ಮತ್ತು ಭಾವನೆಗಳು ಎಂಬುದು ಬೆಳಕಿನಷ್ಟು ಸ್ಪಷ್ಟ.

ಪ್ರಗತಿ ಪಥದಲ್ಲಿ ನಡೆಯಬೇಕೆನ್ನುವ ಯಾವ ವ್ಯಕ್ತಿಯೇ ಆಗಲಿ ಯೋಚನೆಯ ಈ ಮಹಾಶಕ್ತಿಯನ್ನು ಅರ್ಥಮಾಡಿಕೊಳ್ಳಬೇಕು.

ಪ್ರತಿಯೊಂದು ಯೋಚನೆ ಭಯ, ಸಂಶಯಗಳಿಂದ ಕೂಡಿಕೊಂಡು ಹೊರಮುಖವಾಗಿ ಹರಿದು ಕಾರ್ಯರೂಪಕ್ಕೆ ಬಂದಾಗ ಅದು ನಿಷೇಧಾತ್ಮಕವಾಗಿ ಕೊನೆಗೊಳ್ಳುವುದು.

ಪ್ರತಿಯೊಂದು ಯೋಚನೆ ಆತ್ಮವಿಶ್ವಾಸ ಮತ್ತು ಆಶಾಭಾವನೆಯಿಂದ ಕೂಡಿ ಹೊರಮುಖವಾಗಿ ಹರಿದು ಕಾರ್ಯರೂಪಕ್ಕೆ ಬಂದಾಗ ಅದು ರಚನಾತ್ಮಕವಾಗಿ ಪರ್ಯವಸಾನವಾಗುವುದು.

ಮನಸ್ಸಿನ ಶಕ್ತಿಯನ್ನು ಕುರಿತು ನಿಸ್ಸಂದಿಗ್ಧವಾದ ಒಂದು ನಿಯಮ ಹೀಗಿದೆ: ‘ಸದೃಶವಾದ ಯೋಚನೆಗಳು ತತ್​ಸದೃಶ ಯೋಚನೆಗಳೆಡೆಗೆ ಎಳೆಯುತ್ತವೆ’\footnote{\engfoot{Like attracts like, like begets like, like becomes like.}}.

ಇದನ್ನು ಸರಳವಾದ ಮಾತಿನಲ್ಲಿ ಹೇಳುವುದಾದರೆ ಒಳ್ಳೆಯ ಯೋಚನೆಗಳನ್ನು ಮಾಡುತ್ತಾ ಹೋದರೆ ಅವು ಒಳ್ಳೆಯ ವ್ಯಕ್ತಿಗಳೆಡೆಗೆ, ವಸ್ತುಗಳೆಡೆಗೆ ನಮ್ಮನ್ನು ಕೊಂಡೊಯ್ಯುತ್ತವೆ.

ಕೆಟ್ಟ ಯೋಚನೆಗಳನ್ನು ಮಾಡುತ್ತ ಹೋದರೆ ಅವು ಕೆಟ್ಟ ವ್ಯಕ್ತಿಗಳೆಡೆಗೆ, ಕೆಟ್ಟ ವಸ್ತುಗಳೆಡೆಗೆ ನಮ್ಮನ್ನು ಸೆಳೆಯುತ್ತವೆ.

ಆಧುನಿಕ ಪ್ರಯೋಗ ಪರೀಕ್ಷಣಗಳಿಂದ ಸತ್ಯವೆಂದು ಸಾರಲ್ಪಡುವ ಈ ವಿಚಾರವನ್ನು ಎರಡೂ ವರೆ ಸಹಸ್ರವರ್ಷಗಳ ಹಿಂದೆ ಗೌತಮ ಬುದ್ಧನು ನೀಡಿದ ಉಪದೇಶಗಳೊಂದಿಗೆ ಹೋಲಿಸಿ ನೋಡಿ–

‘ನಾವು ಸದ್ಯ ಏನಾಗಿರುವೆವೋ ಅದು ನಮ್ಮ ಯೋಚನೆಗಳ ಫಲವೇ. ಯೋಚನೆಗಳೇ ಸದ್ಯದ ಸ್ಥಿತಿಗೆ ತಳಹದಿ. ನಮ್ಮ ಯೋಚನೆಗಳಿಂದಲೇ ಅದು ರಚಿತವಾಗಿದೆ. ಒಬ್ಬಾತ ಶುಭಯೋಚನೆ ಯಿಂದೊಡಗೂಡಿ ಮಾತನಾಡಿದರೆ, ಅದಕ್ಕನುಗುಣವಾಗಿ ನಡೆದುಕೊಂಡರೆ, ಮನುಷ್ಯರನ್ನು\break ನೆರಳು ಹಿಂಬಾಲಿಸುವಂತೆ ಸುಖವು ಎಂದಿಗೂ ಬೆಂಬಿಡದೆ ಅವನನ್ನು ಹಿಂಬಾಲಿಸುವುದು. ಒಬ್ಬಾತ ಕೆಟ್ಟದ್ದನ್ನು ಯೋಚಿಸುತ್ತ ಅದಕ್ಕನುಗುಣವಾಗಿ ಮಾತನಾಡಿದರೆ, ನಡೆದುಕೊಂಡರೆ ಬಂಡಿಯನ್ನೆಳೆ\-ಯುವ ಎತ್ತಿನ ಕಾಲುಗಳನ್ನು ಚಕ್ರಗಳು ಹಿಂಬಾಲಿಸುವಂತೆ ದುಃಖ ಅವನನ್ನು ಹಿಂಬಾಲಿಸುವುದು.’

ಶಾಶ್ವತಸತ್ಯವನ್ನು ತಿಳಿಸಿಕೊಡುವ ಮಹಾತ್ಮನ ಈ ಮಹಾವಾಕ್ಯಗಳು ಯೋಚನೆಯ ಮಹಿಮೆಗೆ ಹಿಡಿದ ಕೈಗನ್ನಡಿ ಅಲ್ಲವೇ?

ಇಲ್ಲಿಯವರೆಗೂ ಅಸಂಖ್ಯ ದೀನ, ಹೀನ ಯೋಚನೆಗಳಲ್ಲಿ ನಾವು ಮುಳುಗಿದ್ದಿರಬಹುದು. ವಿಕಾಸದ ಪಥದಲ್ಲಿ ಮುನ್ನಡೆಯಬೇಕಾದರೆ ಇಂದಿನಿಂದಲೇ ಒಂದೊಂದೇ ಒಳ್ಳೆಯ ಯೋಚನೆಗಳನ್ನು ನಮ್ಮದಾಗಿಸಿಕೊಂಡು ನಮ್ಮ ಬದುಕನ್ನು ಖಂಡಿತವಾಗಿಯೂ ತಿದ್ದಿಕೊಳ್ಳಬಹುದು. ಮೈಗೆ ಕೊಳೆ ಅಂಟಿಕೊಂಡಿದ್ದರೆ ‘ಕೊಳೆ, ಕೊಳೆ’ಎಂದು ಕೂಗಿಕೊಂಡರೆ ಅದು ಹೋಗದು. ಕೊಳೆಯನ್ನು ಕೊಳೆಯಿಂದ ತೆಗೆದೆಸೆಯಲು ಸಾಧ್ಯವಿಲ್ಲ. ಶುದ್ಧ ನೀರಿನಿಂದ ತೊಳೆಯಬೇಕು. ಅಂತೆಯೇ ಒಳ್ಳೆಯ ಯೋಚನೆಗಳನ್ನು ಬೆಳೆಸಿಕೊಳ್ಳಬೇಕು, ಉಳಿಸಿಕೊಳ್ಳಬೇಕು. ಆಗ ಮೆಲ್ಲ ಮೆಲ್ಲನೇ ದುಷ್ಟ ಯೋಚನೆಗಳು ದೂರವಾಗುವುವು. ಸತ್ಯಾನ್ವೇಷಿಗಳ ಮತ್ತು ವೈಯಕ್ತಿಕ ಅಹಂ ಭಾವನೆಯಿಂದ ಮುಕ್ತರಾದ ಮಹಾತ್ಮರ ಜೀವನ ಸಂದೇಶಗಳ ಮನನ ಈ ದಾರಿಯಲ್ಲಿ ನಮಗೆ ಬಹಳಷ್ಟು ಸಹಾಯಕ.

ಯೋಚನೆಗಳ ನೈಜಸ್ವರೂಪ ಸ್ವಭಾವವನ್ನು ಕುರಿತು ಕೆಲ ವಿಚಾರಗಳು ಇಂತಿವೆ: ‘ಒಮ್ಮೆ ಒಂದು ಯೋಚನೆಯನ್ನು ಮಾಡಿದಿರಾದರೆ ಅದೇ ಯೋಚನೆ ಪುನಃ ಮನಸ್ಸಿನಲ್ಲಿ ಮೂಡುವ ಸಂಭವವಿದೆ. ಒಂದು ಯೋಚನೆಯನ್ನು ಐದು, ಹತ್ತು ಅಥವಾ ಇಪ್ಪತ್ತು ಬಾರಿ ಆವರ್ತನೆ ಮಾಡಿದರೆ ಅದು ನಮ್ಮ ಮನಸ್ಸಿನ ಒಂದು ಭಾಗವಾಗಿ ಬಿಡುವುದು. ನಾವು ಪ್ರಯತ್ನಪೂರ್ವಕವಾಗಿ ಅದನ್ನು ಯೋಚಿಸದಿದ್ದರೂ ಅದು ಮನಸ್ಸಿನ ಆಳದಲ್ಲಿ ಮನೆ ಮಾಡಿಕೊಂಡಿರುತ್ತದೆ. ಅವಕಾಶ ಬಂದಾಗ ಮೇಲೇಳುತ್ತದೆ. ನಿಮ್ಮ ಮನಸ್ಸಿನಲ್ಲಿರುವ ಒಂದು ಪ್ರಬಲ ಯೋಚನೆ ನಿಮ್ಮ ಮೇಲೆ ಮಾತ್ರವಲ್ಲ, ಸಮೀಪದಲ್ಲಿರುವ ಇತರರ ಮೇಲೂ ತನ್ನ ಪ್ರಭಾವ ಬೀರುತ್ತದೆ. ಯೋಚನೆಯ ಶಕ್ತಿ ಎಷ್ಟೆಂದರೆ, ಒಬ್ಬ ವ್ಯಕ್ತಿಯ ಕೆಡುಕನ್ನು ನೀವು ಚಿಂತಿಸುತ್ತಿದ್ದರೆ ಆ ವ್ಯಕ್ತಿಯೂ ನಿಮ್ಮ ಕೆಡುಕನ್ನು ಅಜ್ಞಾತವಾಗಿ ಚಿಂತಿಸುವ ಸಂಭವವಿದೆ. ಇತರರ ಹಿತಚಿಂತನೆ ನೀವು ನಡೆಸಿದ್ದರೆ, ಅವರೂ ನಿಮ್ಮೆಡೆಗೆ ಶುಭಚಿಂತನೆಯನ್ನೇ ನೀಡುವರು.’

‘ಯೋಚನೆಗಳು ಅಲೆಯಂತೆ ಚಲಿಸುತ್ತವೆ. ತಮಗೆ ವ್ಯಕ್ತವಾಗಲು ಅನುಕೂಲವಾದ ಮನಸ್ಸನ್ನು ಹುಡುಕುತ್ತವೆ’ ಎಂದು ಶ‍್ರೀ ಅರವಿಂದರು ಹೇಳಿದ್ದರು. ‘ಸ್ಪಷ್ಟವಾಗಿ ಕಾಣುವ ಹಾಗೆ ಯೋಚನೆಗಳು ಹೊರಗಡೆಯಿಂದ ಮನಸ್ಸನ್ನು ಪ್ರವೇಶಿಸುತ್ತವೆ ಎಂಬುದನ್ನು ಶ‍್ರೀ ಲೇಲೆ ಅವರು ಹೇಳುವವರೆಗೆ ನಾನು ತಿಳಿದಿರಲಿಲ್ಲ. ಅವರ ಆದೇಶದಂತೆ ಮನಸ್ಸನ್ನು ಸ್ತಬ್ಧಗೊಳಿಸಿದೆ. ಪರ್ವತಾಗ್ರದಲ್ಲಿನ ನಿರ್ವಾತ ಪ್ರದೇಶದಂತೆ ಮನಸ್ಸು ಶಾಂತವಾಗಿತ್ತು. ಆಗ ಒಂದಾದ ಮೇಲೊಂದು ಯೋಚನೆ ಮೂರ್ತರೂಪವಾಗಿ ನನ್ನ ಸಮೀಪ ಬರುವುದನ್ನು ಕಂಡೆ. ಅವುಗಳನ್ನು ಹೊರಕ್ಕೆ ತಳ್ಳಿ ನನ್ನ ಮನಸ್ಸನ್ನು ಸಂಪೂರ್ಣವಾಗಿ ನಿಸ್ತರಂಗಗೊಳಿಸಿದೆ’ ಎಂದೂ ಅವರು ಹೇಳಿದ್ದರು.

ನೀವು ಎಂಥ ಯೋಚನೆಗಳನ್ನು ಸ್ವಾಗತಿಸಿ ಸ್ವೀಕರಿಸುತ್ತಿದ್ದೀರಿ? ಎಂಥ ಯೋಚನೆಗಳನ್ನು ಹೊರಗೆ ಕಳುಹಿಸುತ್ತಿದ್ದೀರಿ? ಒಮ್ಮೆ ಸಿಂಹಾವಲೋಕನ ಮಾಡುವುದೊಳಿತಲ್ಲವೇ? ರಾತ್ರಿ ನಿದ್ರಿಸುವುದಕ್ಕೆ ಮೊದಲು ಎಂಥ ಯೋಚನೆಗಳು ನಮ್ಮನ್ನು ಆಳುತ್ತಿವೆ ಎನ್ನುವುದರ ಬಗೆಗೆ ನಾವು ಜಾಗರೂಕರಾಗಿರಬೇಕು. ಆಗ ನಮ್ಮ ಜಾಗ್ರತ ಮನಸ್ಸು ವಿಶ್ರಾಂತಿ ಪಡೆಯಲು ಸಿದ್ಧತೆ ನಡೆಸಿರುತ್ತದೆ. ಸುಪ್ತಮನಸ್ಸು ಒಳಿತು ಕೆಡಕುಗಳ ವಿಚಾರ ಮಾಡಲು ಹೋಗುವುದಿಲ್ಲ. ಕೊಟ್ಟದ್ದನ್ನು ಸ್ವೀಕರಿಸಿ ತನ್ನದನ್ನೂ ಸೇರಿಸಿ ಸಕಾಲದಲ್ಲಿ ಜಾಗ್ರತ ಮನಸ್ಸಿಗೆ ಹಿಂದಿರುಗಿಸುತ್ತದೆ. ಕಂಪ್ಯೂಟರಿಗೆ ನೀವು ತಪ್ಪು ಮಾಹಿತಿಗಳನ್ನು ಒದಗಿಸಿದಂತೆ ಕೋಪ ಮತ್ತು ದ್ವೇಷಮೂಲವಾದ ಯೋಚನೆಗಳನ್ನೂ, ಇತರ ವಿಷಮಯ ಭಾವನೆಗಳನ್ನೂ ಅದು ಮೆಲುಕಾಡುತ್ತಿದ್ದರೆ ನಿಮ್ಮ ನಿದ್ರೆಗೆ ಖಂಡಿತವಾಗಿಯೂ ಭಂಗ ತರುತ್ತದೆ, ಮಾತ್ರವಲ್ಲ, ಎಚ್ಚೆತ್ತ ಮೇಲೆ ನಿಮ್ಮ ವ್ಯಕ್ತಿತ್ವವನ್ನು ಬಿರುಕುಗೊಳಿ ಸುವ ನಿಷೇಧಾತ್ಮಕ ಕಾರ್ಯಕ್ಕೆ ಪ್ರೇರಿಸುತ್ತದೆ.

ಅದಕ್ಕೆಂದೇ ಪ್ರಸಿದ್ಧ ಬರಹಗಾರ ವಿಕ್ಟರ್ ಹ್ಯೂಗೊ ಹೇಳಿದ:

‘ನಿದ್ರಿಸುವುದಕ್ಕೆ ಮೊದಲು ಶುಭನಿರೀಕ್ಷೆ, ಪ್ರೀತಿ, ಕ್ಷಮೆ–ಇವುಗಳನ್ನು ನಿಮ್ಮ ತಲೆದಿಂಬಾಗಿ ಇರಿಸಿಕೊಳ್ಳಿ. ಆಗ ಆನಂದದಿಂದ ಗುಣಗುಣಿಸುತ್ತ ಬೆಳಗ್ಗೆ ಏಳುವಿರಿ ನೀವು!’

‘ಚಿತ್ರಗುಪ್ತ’ನಿದ್ದಾನೆ! ನೆನಪಿರಲಿ.


\section*{ಅಭ್ಯಾಸದ ಅದ್ಭುತಗಳು}

\addsectiontoTOC{ಅಭ್ಯಾಸದ ಅದ್ಭುತಗಳು}

ಸ್ಯಾಮ್ಯುಯೆಲ್ ಸ್ಮಾೖಲ್ಸ್ ತಮ್ಮ ‘ಸ್ವಸಹಾಯ’ ಎನ್ನುವ ಪುಸ್ತಕದಲ್ಲಿ ರಾಬರ್ಟ್ ಪೀಲ್ ಅವರನ್ನು ಕುರಿತ ಒಂದು ಸುಂದರ ಉದಾಹರಣೆ ನೀಡಿದ್ದಾರೆ. ಬಾಲ್ಯದಿಂದಲೇ ಪೀಲ್ ಅವರಿಗೆ ಭಾಷಣ ಕಲೆಯನ್ನು ಕಲಿಸಲು ಅವರ ತಂದೆ ಅವರನ್ನು ಒಂದು ವೇದಿಕೆಯ ಮೇಲೆ ನಿಲ್ಲಿಸಿ ಮಾತನಾಡುವಂತೆ ಪ್ರೋತ್ಸಾಹಿಸುತ್ತಿದ್ದರು. ಚರ್ಚಿನಲ್ಲಿ ಕೇಳಿದ್ದ ಭಾನುವಾರದ ಉಪನ್ಯಾಸವನ್ನು ಪೀಲ್ ಪುನರುಚ್ಚರಿಸಬೇಕಾಗಿತ್ತು. ಮೊದಮೊದಲು ಅಂಥ ಯಾವ ಪ್ರಗತಿಯೂ ಕಾಣಿಸಲಿಲ್ಲ. ಕ್ರಮೇಣ ನಿಯಮಿತ ನಿರಂತರ ಪ್ರಯತ್ನದಿಂದ ಅಸಾಧಾರಣ ಅವಧಾನ ಏಕಾಗ್ರತೆಗಳನ್ನು ಅವರು ಗಳಿಸಿದರು. ದೀರ್ಘಕಾಲ ಮಾಡಿದ ಉಪನ್ಯಾಸಗಳನ್ನೂ ಯಾವ ಟಿಪ್ಪಣಿಯ ಸಹಾಯವಿಲ್ಲದೇ ಚಾಚೂ ತಪ್ಪದೇ ಒಂದೇ ಒಂದು ಶಬ್ದವನ್ನೂ ಬಿಡದೆ ಪುನರುಚ್ಚರಿಸುವ ಸಾಮರ್ಥ್ಯವನ್ನು ಪಡೆದರು. ಮುಂದೆ ಪಾರ್ಲಿಮೆಂಟಿನಲ್ಲಿ ವಿರೋಧಿಗಳ ಹಲವಾರು ಪ್ರಶ್ನೆಗಳನ್ನು ಕ್ರಮವಾಗಿ ಉತ್ತರಿಸುವ ಅವರ ಅದ್ಭುತ ಸ್ಮೃತಿಶಕ್ತಿಯು ಎಲ್ಲರನ್ನೂ ಅಚ್ಚರಿಗೊಳಿಸಿತು.

ಆ ಶಕ್ತಿಯ ಬೆಳವಣಿಗೆಗೆ ಬಾಲ್ಯದಲ್ಲೇ ತಂದೆಯ ಪ್ರೇರಣೆಯಂತೆ ಅವರು ರೂಢಿಸಿಕೊಂಡಿದ್ದ ಅಭ್ಯಾಸದ ವಿಚಾರ ಎಲ್ಲರಿಗೂ ತಿಳಿದಿರಲಿಲ್ಲ.

ಓದು ಬರಹದ ಆವಿಷ್ಕಾರಕ್ಕೆ ಮೊದಲೇ, ನಿಯಮಿತ ಅಭ್ಯಾಸದಿಂದ ವೇದೋಪನಿಷತ್ತುಗಳನ್ನೂ ಸಹಸ್ರಾರು ಶ್ಲೋಕಗಳನ್ನೂ ಕಂಠಸ್ಥವಾಗಿಸಿಕೊಂಡು, ತಲೆ ತಲಾಂತರದಿಂದ ಬಂದ ವಾಙ್ಮಯ ಸಂಪತ್ತನ್ನು ಒಂದಕ್ಷರ ತಪ್ಪಿಲ್ಲದೇ ಮುಂದಿನ ಪೀಳಿಗೆಗೆ ವರ್ಗಾವಣೆ ಮಾಡುವ ಒಂದು ಸಿದ್ಧಿಯನ್ನು ನಮ್ಮ ಜನ ಸಾಧಿಸಿದ್ದರು. ಸ್ಮೃತಿಶಕ್ತಿಯ ಈ ಸಾಹಸದಲ್ಲಿ ಯಾಂತ್ರಿಕತೆ ಕಂಡುಬಂದರೂ, ಅಭ್ಯಾಸದಿಂದ ಪಡೆಯಬಹುದಾದ ಅದ್ಭುತ ಸಾಮರ್ಥ್ಯಕ್ಕೊಂದು ಜೀವಂತ ನಿದರ್ಶನ ಇದು!

ಆಂಗ್ಲ ಭಾಷೆಯಲ್ಲಿ ‘ಹ್ಯಾಬಿಟ್​’ ಎನ್ನುವ ಶಬ್ದದ ಸಮಾನಾರ್ಥಕ ನಾವು ಉಪಯೋಗಿಸುವ ಕನ್ನಡ ಶಬ್ದ ಅಭ್ಯಾಸ. ದುರಭ್ಯಾಸವನ್ನು ಚಟ ಎನ್ನುವುದುಂಟು. ‘ಪ್ರ್ಯಾಕ್ಟೀಸ್​’ ಎನ್ನುವ ಶಬ್ದ ವನ್ನೂ ‘ಅಭ್ಯಾಸ’ ಎಂದೇ ಭಾಷಾಂತರಿಸುವ ಪರಿಪಾಠ ಬಂದು ಬಿಟ್ಟಿದೆ. ಆದರೆ ಸಾಧನೆ ಅಥವಾ ಅನುಷ್ಠಾನ ಎಂಬವು ಸೂಕ್ತ ಪದಗಳು. ಪುನಃ ಪುನಃ ಮಾಡುವ ಪ್ರಯತ್ನ ಅಥವಾ ಸಾಧನೆಯಿಂದ ಅಭ್ಯಾಸ ಸಿದ್ಧಿಸುವುದು.

ಸರ್ಕಸ್ಸಿನಲ್ಲಿ ತರಬೇತಿ ಪಡೆದ ಆಟಗಾರರು ಮಾಡಿ ತೋರಿಸುವ ಸಾಹಸಗಳು ಅಭ್ಯಾಸದಿಂದ ಪಡೆಯಬಹುದಾದ ಅದ್ಭುತ ಶಕ್ತಿ ಮತ್ತು ಸಿದ್ಧಿಗಳಿಗೆ ಮನಸೆಳೆಯುವ ನಿದರ್ಶನಗಳು. ‘ಅಮೇರಿಕದಲ್ಲಿ ಗೊರೂರು’ ಎಂಬ ಪುಸ್ತಕದಲ್ಲಿ ಅಲ್ಲಿ ಕಂಡ ಸರ್ಕಸ್ಸಿನಲ್ಲಿ ರಷ್ಯಾ ದೇಶದ ಸರ್ಕಸ್ ಪಟುಗಳು ತೋರಿಸಿದ ಬೆರಗುಗೊಳಿಸುವ ಅಸಾಮಾನ್ಯ ಚಟುವಟಿಕೆಗಳನ್ನು ಗ್ರಂಥಕರ್ತರು ಸುಂದರವಾಗಿ ಕಣ್ಣಿಗೆ ಕಟ್ಟುವಂತೆ ವಿವರಿಸಿದ್ದಾರೆ. ಆರೋಗ್ಯವಂತ ವ್ಯಕ್ತಿಯೊಬ್ಬ ಆಸ್ಪತ್ರೆಯನ್ನು ಪ್ರವೇಶಿಸಿದಾಗ ನರಳುತ್ತಿರುವ ರೋಗಿಗಳ ಶೋಕ ದುಃಖ ಆಕ್ರಂದನಗಳ ಕ್ಷಣಿಕ ದರ್ಶನದಿಂದ ಸಂತಾಪವನ್ನು ಹೊಂದಿ ಚಿಂತನಶೀಲನಾಗುತ್ತಾನೆ. ಅಂತೆಯೆ ಸರ್ಕಸ್ಸನ್ನು ನೋಡಿಬಂದಾಗ ಸಾಹಸ ಮತ್ತು ಅಭ್ಯಾಸಗಳ ಮಹಿಮೆಯ ಕ್ಷಣಿಕ ಅರಿವು ನಮ್ಮಲ್ಲೂ ಉತ್ಸಾಹದ ತರಂಗವೊಂದನ್ನು ಎಬ್ಬಿಸಬಹುದು. ಆದರೆ ಕೇವಲ ಸಂಕಲ್ಪದಿಂದ ನೂತನ ಅಭ್ಯಾಸ ಕೈಗೂಡದು. ಅದಕ್ಕೆ ಸಾಧನೆಯ ನೀರೆರೆದು ಪೋಷಿಸಿ ಪಾಲಿಸಬೇಕು.


\section*{ನಿಮ್ಮ ಅಭ್ಯಾಸಗಳೇ ನೀವು}

\addsectiontoTOC{ನಿಮ್ಮ ಅಭ್ಯಾಸಗಳೇ ನೀವು}

ಮನುಷ್ಯ ಜೀವನದ ಎಲ್ಲ ಕ್ಷೇತ್ರಗಳಲ್ಲೂ ಒಳಿತು ಕೆಡಕುಗಳ ಮತ್ತು ಚಾರಿತ್ರ್ಯ ನಿರ್ಮಾಣದ ಹಿನ್ನೆಲೆಯಲ್ಲಿ, ಯೋಚನೆ, ಭಾವನೆ ಮತ್ತು ಚಟುವಟಿಕೆ ಇವುಗಳಿಗೆ ಸಂಬಂಧಿಸಿದ ಅಭ್ಯಾಸಗಳು ಪ್ರಮುಖ ಪಾತ್ರವಹಿಸುತ್ತವೆ. ಒಳಿತೋ, ಕೆಡುಕೋ ಒಂದೇ ಕೆಲಸವನ್ನು ಪದೇ ಪದೇ ಮಾಡುತ್ತ ಅದು ನಮ್ಮ ಮನಸ್ಸು ಮತ್ತು ನರಮಂಡಲವನ್ನು ಎಷ್ಟು ಬಲವತ್ತರವಾಗಿ ವ್ಯಾಪಿಸಿಬಿಡುತ್ತದೆ ಎಂದರೆ ರೂಢ ಕ್ರಿಯೆಗಳಂತೆ ಸಂದರ್ಭವೊದಗಿದಾಗ ಪ್ರಯತ್ನವಿಲ್ಲದೆ ಸಹಜ ಪ್ರತಿಕ್ರಿಯೆ ನಡೆಯುತ್ತದೆ!

ನಿವೃತ್ತ ಸೈನಿಕನೊಬ್ಬ ಬೀದಿಯಲ್ಲಿ ಆಹಾರ ಸಾಮಗ್ರಿಯ ಹೊರೆಯನ್ನು ಹೊತ್ತುಕೊಂಡು ಹೋಗುತ್ತಿದ್ದ. ತುಂಟ ಹುಡುಗನೊಬ್ಬ ಆ ಸೈನಿಕನು ಆ ಮಾರ್ಗವಾಗಿ ಹೋಗುತ್ತಿರುವುದನ್ನು ಹಲವು ಬಾರಿ ನೋಡಿದ್ದ. ಒಂದು ದಿನ ಚರಂಡಿಯ ಸಮೀಪದಲ್ಲಿ ಸೈನಿಕ ಹೊರೆಯನ್ನು ಹೊತ್ತು ಅನ್ಯಮನಸ್ಕನಾಗಿ ನಡೆದು ಹೋಗುತ್ತಿದ್ದಾಗ ಹುಡುಗ ಗಟ್ಟಿಯಾಗಿ ‘ಅಟೆನ್​ಶನ್​’ ಎಂದು ಕೂಗಿ ಕೊಂಡ. ‘ಅಟೆನ್​ಶನ್​’ ಶಬ್ದ ಕಿವಿಯ ಮೇಲೆ ಬೀಳುತ್ತಲೇ ಸೈನಿಕ ತಲೆಯ ಮೇಲಿನ ಹೊರೆಯನ್ನು ಆಧರಿಸಿ ಹಿಡಿದಿದ್ದ ಕೈಗಳನ್ನು ಥಟ್ಟನೆ ಕೆಳಕ್ಕೆ ಬಿಟ್ಟು ನೆಟ್ಟಗೆ ನಿಂತು ಸೆಲ್ಯೂಟ್ ಹೊಡೆಯಲು ಸಿದ್ಧನಾದ. ಹೊತ್ತುಕೊಂಡಿದ್ದ ಹೊರೆ ಕೆಳಗೆ ಚರಂಡಿಯಲ್ಲಿ ಬಿದ್ದು ಸಂಗ್ರಹಿಸಿದ ಸಾಮಾನು ಚೆಲ್ಲಾಪಿಲ್ಲಿಯಾಗಿ ಚೆದುರಿತು. ಹುಡುಗನ ತುಂಟತನ ತಿಳಿಯುವ ಮೊದಲೆ ಕಾರ್ಯ ಮಿಂಚಿ ಹೋಗಿತ್ತು. ದೀರ್ಘಕಾಲ ಕವಾಯತಿನ ಅಭ್ಯಾಸಕ್ಕೆ ಒಗ್ಗಿಕೊಂಡಿದ್ದ ಅವನ ಚಿತ್ತ ತತ್​ಕ್ಷಣವೇ ಕಾರ್ಯವೆಸಗಿದ್ದೆ ಅದಕ್ಕೆ ಕಾರಣ!

ಅಭ್ಯಾಸಗಳ ಅದ್ಭುತ ನಿಯಂತ್ರಣ ಶಕ್ತಿಯನ್ನು ಹಲವರು ಯೋಚಿಸುತ್ತಿಲ್ಲ. ನಾವು ಸಿಲುಕಿ ಕೊಂಡಿರುವ ಅತ್ಯಂತ ಕ್ಲೇಶಕರವಾದ ಹೇವರಿಕೆಯನ್ನುಂಟುಮಾಡುವಂಥ ಸನ್ನಿವೇಶ ಅಥವಾ ಪರಿಸರಗಳಿಂದ ಬಿಡಿಸಿಕೊಳ್ಳಲೂ ತಡೆಯಾಗಿ ನಿಲ್ಲುವವು ಈ ಅಭ್ಯಾಸಗಳೆ. ದಿನದಿನವೂ ನಿಂತ ಜಾಗ ಮತ್ತು ಪರಿಸರಕ್ಕೆ ಅಂಟಿಕೊಳ್ಳುವಂತೆ, ಹೊಂದಿಕೊಳ್ಳುವಂತೆ ಅಜ್ಞಾತವಾಗಿ ನಾವು ಅಭ್ಯಾಸ ಮಾಡುತ್ತಿರುತ್ತೇವೆ. ಬೆಳಗಿನ ಉಪಾಹಾರ ಮತ್ತು ಕಾಫಿ ಸೇವನೆಯ ನಂತರ ಸಿಗರೇಟು ಸೇದುವ ಅಭ್ಯಾಸವಿರುವವರನ್ನು ವೀಕ್ಷಿಸಿ. ಉಪಾಹಾರದ ನಂತರ ಜೇಬಿನಲ್ಲಿ ಸಿಗರೇಟು ಮುಗಿದಿದ್ದರೆ ಅಂಗಡಿ ಎರಡು ಮೈಲಿ ದೂರವಿದ್ದರೂ ಅಲ್ಲಿಗೆ ನಡೆದುಕೊಂಡಾದರೂ ಹೋಗಿ ಹೊಗೆಬತ್ತಿಯನ್ನು ಕೊಂಡು ತಂದು ಧೂಮಪಾನ ಮಾಡದಿರಲು ಅವರಿಂದ ಸಾಧ್ಯವೇ? ಅಭ್ಯಾಸವೆಂದರೆ ದುರಭ್ಯಾಸಗಳೆಂದು ಅರ್ಥವಲ್ಲ. ಎಲ್ಲ ಅಭ್ಯಾಸಗಳೂ ತಮ್ಮ ಪರಿಣಾಮಗಳ ಮೂಲಕ ವ್ಯಕ್ತಿಯನ್ನು ನಿಯಂತ್ರಿಸುತ್ತವೆ, ರೂಪಿಸುತ್ತವೆ. ನಮ್ಮ ಸಾಧನೆ ಸಿದ್ಧಿಗಳು, ವಿಚಾರ ವೈದುಷ್ಯಗಳು, ಅಭಿರುಚಿ ಕಲಿಕೆಗಳು, ರಾಗ ದ್ವೇಷಗಳು, ಕೋಪ ತಾಪಗಳು, ಮರೆ ಮೋಸಗಳು, ಅಹಂಕಾರ ಅಭಿಮಾನಗಳು, ಒಲವು ಅನಿಸಿಕೆಗಳು, ಕುಹಕ ಕೊಂಕುಗಳು–ಇವು ಪರಿಸರದಿಂದ ಪ್ರತ್ಯಕ್ಷವಾಗಿಯೋ, ಪರೋಕ್ಷವಾಗಿಯೋ ನಾವು ಸಂಗ್ರಹಿಸಿ ರೂಢಿಸಿಕೊಳ್ಳುವ ಅಭ್ಯಾಸಗಳೇ. ರೂಢ ಮೂಲವಾದ ಈ ಅಭ್ಯಾಸಗಳಿಂದ ಪಾರಾಗುವುದು ಕಷ್ಟ ಎನ್ನುವುದನ್ನು ಸೂಚಿಸುವ ಚಮತ್ಕಾರದ ನುಡಿಯೊಂದು ಆಂಗ್ಲಭಾಷೆಯಲ್ಲಿದೆ. \enginline{Habit} ಎನ್ನುವ ಪದದಿಂದ ನೀವು \enginline{H}ಅನ್ನು ತೆಗೆದರೆ \enginline{ a bit} ಉಳಿಯುತ್ತದೆ. \enginline{a} ಅನ್ನು ತೆಗೆದರೆ \enginline{bit} ಉಳಿದುಕೊಳ್ಳುತ್ತದೆ. \enginline{b} ಅನ್ನೂ ತೆಗೆದರೆ \enginline{it} ಉಳಿದು ಕೊಳ್ಳುತ್ತದೆ. ನೋಡಿದಿರಾ ಅಭ್ಯಾಸಗಳ ಆಳವಾದ ಪ್ರಭಾವವನ್ನು!


\section*{ಚಾರಿತ್ರ್ಯ ನಿರ್ಮಾಣದ ಸೂತ್ರಧಾರ}

\addsectiontoTOC{ಚಾರಿತ್ರ್ಯ ನಿರ್ಮಾಣದ ಸೂತ್ರಧಾರ}

ನಾವು ಮಾಡುವ ಪ್ರತಿಯೊಂದು ಕ್ರಿಯೆಯೂ ಸರೋವರದ ಮೇಲ್ಭಾಗದಲ್ಲಿ ಸ್ಪಂದಿಸುವ ಅಲೆಯಂತೆ. ಸರೋವರದಲ್ಲಿ ಸ್ಪಂದಿಸುವ ಅಲೆಯಾದರೋ ಕೆಲವೇ ನಿಮಿಷಗಳಲ್ಲಿ ಮಾಯವಾಗುತ್ತದೆ. ಆದರೆ ಮನಸ್ಸಿನಲ್ಲಿ ಮೂಡುವ ಯೋಚನೆ ಭಾವನೆಗಳಾಗಲಿ, ನಾವು ಮಾಡುವ ಚಟುವಟಿಕೆಯಾಗಲಿ ಮಾಯವಾದಂತೆ ಕಂಡರೂ ಮನಸ್ಸಿನ ಆಳದ ಪದರುಗಳಲ್ಲಿ ಸಂಸ್ಕಾರ ರೂಪದಲ್ಲಿ ಉಳಿದಿರುತ್ತವೆ. ಇಂಥ ಅನೇಕಾನೇಕ ಸಂಸ್ಕಾರಗಳೇ ಕಲೆತು ನಮ್ಮ ನಡತೆ ಅಥವಾ ಚಾರಿತ್ರ್ಯ ನಿರ್ಮಾಣವಾಗುವುದು. ಅಭ್ಯಾಸವು ಮನುಷ್ಯನ ಎರಡನೇ ಸ್ವಭಾವ ಎನ್ನುವ ಆಂಗ್ಲ ಗಾದೆ ಇದೆ. ಯೋಚಿಸಿ ನೋಡಿದರೆ ಅಭ್ಯಾಸ ಮನುಷ್ಯನ ಸರ್ವ ಸ್ವಭಾವವೆಂದು ತಿಳಿಯುವುದು.

‘ನಡತೆ ಅಥವಾ ಚಾರಿತ್ರ್ಯ ಎಂದರೆ ಸಂಪೂರ್ಣವಾಗಿ ರೂಪುಗೊಂಡ ಇಚ್ಛಾಶಕ್ತಿ’ ಎಂಬುದು ಜಾನ್ ಸ್ಟೂಯರ್ಟ್ ಮಿಲ್ ಮಹಾಶಯ ನೀಡಿದ ನಿರೂಪಣೆ. ಇಚ್ಛಾಶಕ್ತಿ ಎಂದರೆ ಕಾರ್ಯ ಪ್ರೇರಕವಾದ ಅನಿಸಿಕೆಗಳ ಒಟ್ಟು ಮೊತ್ತ. ಅವು ನಿಶ್ಚಿತ, ನಿಯಮಿತ, ರಚನಾತ್ಮಕ ರೀತಿಯಲ್ಲಿ ವ್ಯಕ್ತವಾದರೆ ಅಂಥ ವ್ಯಕ್ತಿಯ ಚಾರಿತ್ರ್ಯ ದೃಢವಾಗಿದೆ ಎಂದರ್ಥ. ‘ಯಾರನ್ನೂ ಅಪ್ರಯೋಜಕ ವ್ಯಕ್ತಿ ಎನ್ನಬೇಡಿ. ಏಕೆಂದರೆ ಆತನೊಂದು ಚಾರಿತ್ರ್ಯವನ್ನು ಪ್ರತಿನಿಧಿಸುತ್ತಾನೆ: ಎಂದರೆ ತಾನು ಒಂದು ತೆರನಾದ ಅಭ್ಯಾಸಗಳ ಮೂಟೆ ಎಂಬುದನ್ನು ಸೂಚಿಸುತ್ತಾನೆ. ಹಳೆಯ ಅಭ್ಯಾಸಗಳನ್ನು ನೂತನ ಉಪಯುಕ್ತ ಅಭ್ಯಾಸಗಳಿಂದ ಗೆಲ್ಲಬಹುದು. ನಡತೆ ಎಂದರೆ ಪುನರಾವರ್ತಿತ ಅಭ್ಯಾಸಗಳ ಮೊತ್ತ. ಪುನರಾವರ್ತಿತ ಕ್ರಿಯೆಗಳ ಮೂಲಕವೆ ನಡತೆಯಲ್ಲಿ ಸುಧಾರಣೆಗಳಾಗಬೇಕು’ ಎಂದರು ಸ್ವಾಮಿ ವಿವೇಕಾನಂದರು.

ಆದರೆ ಇದು ಸಾಧ್ಯವಾಗಬೇಕಾದರೆ ನಿಷ್ಠೆಯಿಂದ ಕೂಡಿದ ನಿಯಮಿತವಾದ ನಿರಂತರ ಪ್ರಯತ್ನ ಬೇಕು.


\section*{ಪರಿಸರದ ಪ್ರಭಾವ}

\addsectiontoTOC{ಪರಿಸರದ ಪ್ರಭಾವ}

ಪರಿಸರದ ಪ್ರಭಾವದಿಂದ ತಪ್ಪಿಸಿಕೊಳ್ಳಲು ಯೋಗ್ಯ ಮಾರ್ಗದರ್ಶನ ಮತ್ತು ತೀವ್ರ ಹಂಬಲದ ಪ್ರಯತ್ನಗಳಿಲ್ಲದಿದ್ದರೆ ಮನುಷ್ಯನ ಬಾಳು ಹೇಗೆ ರೂಪುಗೊಳ್ಳಬಹುದೆಂಬುದನ್ನು ತಿಳಿಸುವ ಒಂದು ನಿದರ್ಶನವಿದೆ. ಸಾಮಾನ್ಯ ವ್ಯಕ್ತಿಯೊಬ್ಬ ಕೊಲೆಗಡುಕನಾಗಿ ಪರಿವರ್ತಿತನಾಗ ಬಹುದಾದ ವಿಧಾನದ ಬಗೆಗೆ ತಜ್ಞ ಸಂಶೋಧಕನ ಅಂಬೋಣ ಅದು.

‘ಪ್ರಥಮ ಬಾರಿ ಕೊಲೆಯನ್ನು ಪ್ರತ್ಯಕ್ಷ ನೋಡಿದವನು ತೀವ್ರವಾಗಿ ಸಂಕಟಪಟ್ಟು ವಿಪರೀತ ಅಸಹ್ಯಭಾವನೆಯನ್ನು ಬೆಳೆಸಿಕೊಳ್ಳುವನು. ಕೆಲವು ಕಾಲ ಅಂಥ ಕೃತ್ಯಗಳ ಸಮೀಪದಲ್ಲೇ ಇರು ತ್ತಿದ್ದರೆ ಮೆಲ್ಲನೆ ಹೊಂದಿಕೊಳ್ಳುತ್ತ ಅದನ್ನು ಸಹಿಸಿಕೊಳ್ಳುವನು. ದೀರ್ಘಕಾಲ ಕೊಲೆಗಡುಕರ ಸನಿಹದಲ್ಲಿದ್ದು ಕೊಲೆ ಹಿಂಸಾಕೃತ್ಯಗಳನ್ನು ಕಾಣುತ್ತ ಅವುಗಳ ಸಂಸ್ಪರ್ಶದಲ್ಲೇ ಇದ್ದವನು ಕ್ರಮೇಣ ಅವುಗಳಿಂದ ಪ್ರಭಾವಿತನಾಗಿ ಯಾವ ಹಿಂಸಾಕೃತ್ಯಕ್ಕಿಳಿಯಲೂ ಹಿಂಜರಿಯನು.’

ಜೈಲಿನಲ್ಲೇ ಬದುಕಿನ ಹೆಚ್ಚಿನ ದಿನಗಳನ್ನು ಕಳೆದ ಕೈದಿಗಳು ಶಿಕ್ಷಾವಧಿಯ ನಂತರ ಬಿಡುಗಡೆ ಹೊಂದಿ ಹೊರಗೆ ಬಂದರೂ ದಿಕ್ಕುತೋಚದಂತಾಗಿ ಅಧಿಕಾರಿಯನ್ನು ‘ತಿರುಗಿ ಜೈಲಿಗೆ ಸೇರಿಸಿ ಕೊಳ್ಳಿ’ ಎಂದು ಕೇಳಿಕೊಳ್ಳುವುದೂ ಇದಕ್ಕೆ!

ಹೊಲಸಿನ ಮಧ್ಯೆ ವಾಸಿಸುತ್ತ ಕ್ರಮೇಣ ಮನುಷ್ಯರು ಎಲ್ಲ ಶುಚಿತ್ವದ ನಿಯಮಗಳನ್ನೂ ಮರೆತು ಅಂಥ ಪರಿಸರದಿಂದ ಮೇಲಕ್ಕೇಳಲಾರದೆ ಹೇಗೋ ಹೊಂದಿಕೊಂಡು ಬಿಡುತ್ತಾರೆ. ಪರಿಸರದ ಪ್ರಭಾವ ಅದು.

ಇಂಥ ಜನ ಪತನದಿಂದ ಮೇಲೇರಲು ಸಾಧ್ಯವೇ?

ಪರಂಪರೆ ಮತ್ತು ಪರಿಸರದಿಂದ ಪಡೆದುಕೊಂಡು ಬಂದ ದುರಭ್ಯಾಸಗಳ ಆಳವಾದ ಕಂದಕದಲ್ಲಿ ಸಿಕ್ಕಿ ನಿರ್ನಾಮವಾಗುತ್ತಿದ್ದ ಒಂದು ನರಭಕ್ಷಕ ಜನಾಂಗವನ್ನೇ ಮೇಲಕ್ಕೆತ್ತಲು ಅಪೂರ್ವ ತ್ಯಾಗ, ಶ್ರಮ ಮತ್ತು ಸಮರ್ಪಣೆಯ ಭಾವದಿಂದ ಹೋರಾಡಿದ ತರುಣ ಆಂಗ್ಲ ವ್ಯಾಪಾರಿ ಡಂಕನ್​ನ ಸಾಹಸಕತೆ ಯಾವುದೇ ಜನಾಂಗಕ್ಕೆ ಅಭಿವೃದ್ಧಿಯ ಪಥದಲ್ಲಿ ಮುನ್ನಡೆಯಲು ಮಾರ್ಗ ದರ್ಶನವಾಗುವಂಥ ಸ್ಫೂರ್ತಿಯನ್ನು ನೀಡಬಲ್ಲುದು.

೧೮೫೭ರ ಸುಮಾರಿಗೆ ಕೆನಡಾದ ವಾಯವ್ಯದಲ್ಲಿರುವ ಅಲಾಸ್ಕಾದ ಶಾಂತಸಾಗರದಲ್ಲಿನ ಒಂದು ದ್ವೀಪದಲ್ಲಿ ಬ್ರಿಟಿಷ್ ವ್ಯಾಪಾರಿ ಹಡಗೊಂದು ಲಂಗರು ಹಾಕಿತು. ಹಡಗಿನಲ್ಲಿದ್ದ ತರುಣ ವ್ಯಾಪಾರಿ ಸಮುದ್ರ ತೀರದಲ್ಲಿ ಭಯಾನಕ ದೃಶ್ಯವನ್ನು ಕಂಡು ತತ್ತರಿಸಿಹೋದ. ಸುಮಾರು ಇಪ್ಪತ್ತಕ್ಕೂ ಹೆಚ್ಚಿನ ಜನರ ಛಿನ್ನಭಿನ್ನವಾದ ಶವಗಳು ಅಲ್ಲಿ ಅಸ್ತವ್ಯಸ್ತವಾಗಿ ಬಿದ್ದಿದ್ದವು. ತರುಣ ಡಂಕನ್​ನ ಆಶ್ಚರ್ಯ ಉದ್ವೇಗಗಳನ್ನು ಕಂಡ ಹಡಗಿನ ಅಧಿಕಾರಿ ಹೀಗೆಂದ:

‘ಅವರು ಸಿಮಶೀನ್ ಇಂಡಿಯನ್ಸ್. ಈ ಜನ ತಮ್ಮೊಳಗೆ ಯಾವಾಗಲೂ ಪರಸ್ಪರ ಕಾದಾಡು\-ತ್ತಾರೆ. ಜಗಳಾಡುತ್ತಲೇ ಸಾಯುತ್ತಾರೆ. ಅಲ್ಲದೇ ಅವರು ನರಭಕ್ಷಕರು ಕೂಡ ಹೌದು. ಕೊಲೆ ಮಾಡುವುದು ಅವರಿಗೊಂದು ಆಟ. ಕುಡಿತದ ಚಟಕ್ಕೆ ಸಂಪೂರ್ಣ ಬಲಿಯಾದವರು ಅವರು. ದಿನವಿಡೀ ದುಡಿದು ತಂದ ಫರ್ ಚರ್ಮವನ್ನು ಅಲ್ಪಬೆಲೆಗೆ ಮಾರಿ ಶರಾಬು ಕುಡಿಯುತ್ತಾರೆ. ಶರಾಬಿಗಾಗಿ ತಮ್ಮ ಹೊಟ್ಟೆಯಲ್ಲಿ ಹುಟ್ಟಿದ ಹೆಣ್ಣು ಮಕ್ಕಳನ್ನೂ ಮಾರಲು ಹೇಸದ ಜನರು ಇವರು...’

ತಂದೆ ತನ್ನ ಅಳುತ್ತಿರುವ ಹುಡುಗಿಯರನ್ನು ಒತ್ತಾಯ ಮತ್ತು ಬಲಾತ್ಕಾರದಿಂದ ಹಣಕ್ಕಾಗಿ ಸಿಪಾಯಿಗಳಿಗೊಪ್ಪಿಸುವ ದಾರುಣ ದೃಶ್ಯವನ್ನು ಪ್ರತ್ಯಕ್ಷವಾಗಿ ಕಂಡ ಡಂಕನ್ ಹೌಹಾರಿದ. ಸಂತಪ್ತನಾದ. ‘ಇವರಿಗೊಬ್ಬ ಮಾರ್ಗದರ್ಶಕ ಬೇಕು, ಇವರಿಗೊಬ್ಬ ಸುಧಾರಕ ಬೇಕು’ ಎಂಬ ಮಾತು ಅವನ ಮನಸ್ಸಿನಲ್ಲಿ ಮಾರ್ದನಿಗೊಂಡಿತು. ಆಗ ಮಾನವ ಪ್ರೇಮಿಯಾದ ಆತ ಒಂದು ಭೀಷ್ಮ ಪ್ರತಿಜ್ಞೆಯನ್ನೇ ಮಾಡಿದ. ಅದುವರೆಗೂ ಬಿಳಿಜಾತಿಯ ಯಾವ ವ್ಯಕ್ತಿಯೂ ಮಾಡಿರದಂಥ ಮಹಾಕಾರ್ಯವನ್ನು ತಾನೆಸಗಬೇಕೆಂದು ದೃಢ ನಿರ್ಧಾರ ಮಾಡಿ ಸೇವಾಕ್ಷೇತ್ರವನ್ನು ಪ್ರವೇಶಿಸಿಯೇ ಬಿಟ್ಟ.

ತನ್ನ ಬದುಕನ್ನೇ ಗಂಡಾಂತರಕ್ಕೊಡ್ಡಿ ಆ ಅನಾಗರಿಕ ಜನರ ಮಧ್ಯೆ ನೆಲೆಸಿದ. ಅವರ ಭಾಷೆಯನ್ನು ಕಲಿತುಕೊಂಡ. ಹೃತ್ಪೂರ್ವಕ ಪ್ರೀತಿಯ ವ್ಯವಹಾರದಿಂದ ಅವರ ವಿಶ್ವಾಸವನ್ನು ಗಳಿಸಿ ಕೊಳ್ಳಲು ಯತ್ನಿಸಿದ. ಮನೆ ಮನೆಗೂ ಹೋಗಿ ಜನಸಂಪರ್ಕ ಬೆಳೆಸಿ ಶುಚಿತ್ವದ ನಿಯಮಗಳನ್ನು ಬೋಧಿಸಿದ. ಯುವಕರನ್ನು ಒಂದೆಡೆ ಕೂಡಿಸಿ ತರಬೇತಿ ನೀಡಿ, ಶ್ರದ್ಧಾವಂತ ಕೆಲಸಗಾರರ ತಂಡವನ್ನು ನಿರ್ಮಿಸಿದ. ಜನರ ಸಾಮಾಜಿಕ ಹಾಗೂ ಆರ್ಥಿಕ ಅಭಿವೃದ್ಧಿಗೆ ಯೋಜನೆಗಳನ್ನು ಕೈ ಗೊಂಡ. ಅವರಿಗೆ ವಿವಿಧ ಉದ್ಯೋಗಗಳು ದೊರೆಯುವಂತೆ ಮಾಡಿ, ದುಡಿಮೆಯ ಮಹಿಮೆ ಮತ್ತು ಉತ್ಪತ್ತಿಯ ವಿಧಾನವನ್ನು ತಿಳಿಸಿಕೊಟ್ಟ. ತಾನೇ ಚರ್ಚುಗಳನ್ನು ಸ್ಥಾಪಿಸಿ ಧಾರ್ಮಿಕ ಭಾವನೆಗಳನ್ನು ಬಿತ್ತರಿಸಿದ. ಅವರಲ್ಲಿ ಬಲವಾಗಿ ಬೇರುಬಿಟ್ಟಿದ್ದ ದುಷ್ಟ ಚಟಗಳನ್ನು ದೂರ ಮಾಡಲು ಅಪೂರ್ವ ಸಹನೆ ಮತ್ತು ದೃಢತೆಯಿಂದ ಹಲವು ವರ್ಷಗಳ ಕಾಲ ದಣಿವಿಲ್ಲದ ಹೋರಾಟ ನಡೆಯಿಸಿದ. ಈ ಎಲ್ಲ ಕೆಲಸಗಳನ್ನು ನಿರ್ವಹಿಸುವುದು ಸುಲಭವಾಗಿರಲಿಲ್ಲ. ‘ಒಳ್ಳೆಯ ಕೆಲಸ ಮಾಡಹೊರಟವರಿಗೆ ಬಹುವಿಘ್ನಗಳು’ ಎನ್ನುವ ಮಾತು ಅವನ ಪಾಲಿಗೆ ದಿನ ದಿನದ ಅನುಭವವಾಯಿತು. ಯಾರನ್ನು ಸೇವಿಸಹೊರಟನೋ ಆ ಜನರಿಂದಲೇ ವಿರೋಧ, ಆ ಜನಾಂಗದ ಮುಖಂಡರುಗಳಿಂದ ವಿರೋಧ, ಆ ದುರ್ದೈವೀ ಜನರನ್ನು ಶೋಷಿಸಿ ಲಾಭ ಪಡೆ ಯುವ ಅಭ್ಯಾಸವನ್ನು ರೂಢಿಸಿಕೊಂಡ ವ್ಯಾಪಾರೀ ವರ್ಗದಿಂದ ವಿರೋಧ, ಕ್ರೈಸ್ತಧರ್ಮಾಧಿಕಾರಿ ಗಳಿಂದ ವಿರೋಧ, ಈ ಎಲ್ಲ ತಡೆಗಳನ್ನು ಏಕಾಂಗಿಯಾಗಿ ಅಮಿತ ಸಾಹಸ, ಉತ್ಸಾಹ, ಆತ್ಮ ವಿಶ್ವಾಸ, ನಿರಂತರ ಹೋರಾಟ–ಇವುಗಳ ಬಲದಿಂದ ಎದುರಿಸಿ ಯಶಸ್ವಿಯಾಗಿ ಬೆಳಗಿದ.

ಅರವತ್ತು ವರ್ಷಗಳ ಸಂಘರ್ಷ ಹಾಗೂ ವಿರೋಧದ ಬಳಿಕ ಅವನ ಉದ್ದೇಶ ಸಫಲ\-ವಾಯಿತು. ಭಯಾನಕ, ಅಸಭ್ಯ ಅನಾಗರೀಕತೆಯಿಂದ ಕೆಟ್ಟ ಹೆಸರು ಪಡೆದಿದ್ದ ಸಿಮಶೀನ್ ಜಾತಿಯೇ ಇಂದು ಶಾಂತಿಪ್ರಿಯವೂ, ಉನ್ನತ ಶೀಲವೂ ಸುಸಂಸ್ಕೃತವೂ ಆಗಿದೆ. ಇಂದು ಅಲ್ಲಿ ಬರ್ಬರ ಅಪರಾಧಗಳು ನಡೆಯುವುದಿಲ್ಲ. ಇಂದು ಯಾರೊಬ್ಬರೂ ಅಲ್ಲಿ ಮದ್ಯಸೇವನೆ ಮಾಡುವುದಿಲ್ಲ.

ಹಳೆಯ ಅಭ್ಯಾಸಗಳನ್ನು ನೂತನ ಅಭ್ಯಾಸಗಳಿಂದ ಗೆಲ್ಲಬಹುದು ಎಂಬ ವಿವೇಕವಾಣಿಯನ್ನು ಪ್ರಯೋಗ ಪರೀಕ್ಷಣಗಳಿಂದ ಸತ್ಯವೆಂದು ಸಾರಿದ ಡಂಕನ್​! ಆದರೆ ಎಷ್ಟೊಂದು ದೀರ್ಘಕಾಲದ ಶ್ರಮ! ಎಂಥ ಮಹಾತ್ಯಾಗ! ಎಷ್ಟೊಂದು ಹೋರಾಟ!

೧೯೧೮ನೇ ಇಸವಿಯಲ್ಲಿ, ತನ್ನ ಎಂಬತ್ತಾರನೇ ವಯಸ್ಸಿನಲ್ಲಿ ಡಂಕನ್ ಇಹಲೋಕದಿಂದ ಕಣ್ಮರೆಯಾದ. ಆದರೆ ಸಿಮಶೀನ್ ಜನಾಂಗದ ಕಣ್ಮಣಿಯಾಗಿ ಪ್ರಾತಃಸ್ಮರಣೀಯನಾದ.


\section*{ವಿಭೂತಿ ಪೂಜೆ}

\addsectiontoTOC{ವಿಭೂತಿ ಪೂಜೆ}

ನೀವು ಯಾರನ್ನು ಹೃತ್ಪೂರ್ವಕವಾಗಿ ಗೌರವಿಸಿ ಪ್ರೀತಿಸುವಿರೊ, ಅವರ ಗುಣಗಳನ್ನೂ ಆದರ್ಶವನ್ನೂ ನಿಮ್ಮದಾಗಿ ಮಾಡಿಕೊಳ್ಳುವಿರಿ. ಸ್ವಾತಂತ್ರ್ಯಪೂರ್ವದಲ್ಲಿ ಮಹಾತ್ಮಾ ಗಾಂಧೀಜಿಯವರನ್ನು ಜನಾಂಗವೇ ಪ್ರೀತಿಸಿ ಗೌರವಿಸಿತು. ಎಂಥ ಸ್ವಾರ್ಥತ್ಯಾಗಿ ಮೃತ್ಯುಂಜಯ ಕರ್ಮವೀರರು ಆಗ ಉದಯಿಸಿದರು!

ಅಲೆಕ್ಸಿಸ್ ಕೆರೆಲ್​ ಅವರ ಮಾತನ್ನು ಕೇಳಿ: ‘ಉದಾತ್ತ ಗುಣಗಳಿಂದ ಶೋಭಿಸುವ ಉನ್ನತ ಮಟ್ಟದ ವ್ಯಕ್ತಿಗಳನ್ನು ಶ್ರದ್ಧಾಭಕ್ತಿಗಳಿಂದ ನೋಡುವ “ವಿಭೂತಿ ಪೂಜೆ” ಮಾನವನ ಆವಶ್ಯಕತೆಗಳಲ್ಲಿ ಒಂದು. ಅದು ಮಾನಸಿಕ ಪ್ರಗತಿಗೆ ಅವಶ್ಯವೂ ಹೌದು. ಪ್ರಜಾಸತ್ತಾತ್ಮಕ ದೇಶಗಳಲ್ಲಿ ಎಳೆಯರಿಗೆ ಮಾದರಿಯಾಗಬಲ್ಲ ವ್ಯಕ್ತಿಗಳೆಲ್ಲ ಸುದೈವದಿಂದ ಸಮಾಜದಲ್ಲಿ ಜೀವಂತರಾಗಿರುವವರಲ್ಲದೆ ಗತಿಸಿದ ವ್ಯಕ್ತಿಗಳೂ ಇದ್ದಾರೆ. ಅವರು ನಮ್ಮ ನಡುವೆ ಜೀವಂತವಾಗಿಯೇ ಇರುತ್ತಾರೆ. ಅವರನ್ನೂ ಅವರ ಜೀವನ ಸಂದೇಶಗಳನ್ನೂ ಕುರಿತು ನಾವು ಯೋಚಿಸಬಹುದು. ಅವರ ದನಿಯನ್ನು ಆಲಿಸಬಹುದು. ಸಿನೆಮಾ ನಟನಟಿಯರನ್ನು ಕುರಿತ ವಿಚಾರಕ್ಕಿಂತ ಕೋಲಾ, ದಾಂತೆ, ಪ್ಯಾಶ್ಚರ್ ಇವರ ಸತ್ಸಂಗ ಒಳಿತಲ್ಲವೇ? ಮಹಾವಿದ್ವಾಂಸರ, ವೀರರ, ಸಂತರ ಜೀವನ ಕಥೆಗಳಲ್ಲಿ ಅವ್ಯಯವಾದ ಆತ್ಮಶಕ್ತಿಯ ಸಂಚಯವಿದೆ. ಅವರು ಬಯಲು ನೆಲದಿಂದ ಎದ್ದುನಿಂತ ಪರ್ವತ ಶಿಖರಗಳಂತಿದ್ದಾರೆ. ಎಷ್ಟು ಎತ್ತರಕ್ಕೆ ನಾವು ಏರಬಹುದು, ಮಾನವ ಪ್ರಜ್ಞೆಯು ಹಂಬಲಿಸುವ ಗುರಿ ಎಷ್ಟು ಉದಾತ್ತವಾಗಿರಬಹುದು ಎಂಬುದನ್ನೂ ಅವರು ನಮಗೆ ತೋರಿಸಿ ಕೊಡುತ್ತಾರೆ. ದೇಶಗಳ ಅವನತಿಯ ಕಾಲದಲ್ಲಿ ಮುಂದಾಳುಗಳು ವೈಶಿಷ್ಟ್ಯರಹಿತರಾಗಿರುತ್ತಾರೆ. ಪೂಜಾರ್ಹರಾದ ವ್ಯಕ್ತಿ\-ಗಳಿಲ್ಲದಿದ್ದರೆ ಶ‍್ರೀಸಾಮಾನ್ಯರಿಗೆ ಕಷ್ಟವೆನಿಸುತ್ತದೆ.’ ಆ ದೃಷ್ಟಿಯಿಂದಲೇ ನೀಗ್ರೊ ಜನಾಂಗವನ್ನು ಮೇಲೆತ್ತಲು ರಚನಾತ್ಮಕ ಯೋಜನೆಗಳಿಂದ ದುಡಿದು ಹೋರಾಡಿದ ನೀಗ್ರೊ ಮಹಾನಾಯಕ ಬೂಕರ್ ಟಿ.\ ವಾಷಿಂಗ್ಟನ್ ಹೇಳಿದ: ‘ಸುಸಂಸ್ಕೃತ ಸಜ್ಜನರ ಒಡನಾಟ ದಿಂದ ಲಭಿಸುವ ಕಲಿಕೆಗೆ ಸಮನಾದ ವಿದ್ಯೆಯನ್ನು ಯಾವ ಪುಸ್ತಕವೂ, ಬೆಲೆಬಾಳುವ ಉಪಕರಣ ಗಳೂ ನೀಡಲಾರವು ಎಂಬುದು ವಯಸ್ಸಾದಂತೆಲ್ಲ ನನ್ನ ಮನಸ್ಸಿಗೆ ಸ್ಪಷ್ಟವೂ ದೃಢವೂ ಆಗುತ್ತ ಲಿದೆ’ ಎಂದು.

\vskip 2pt

ನಮ್ಮ ದೇಶದ ಸಂತರು, ಮಹಾತ್ಮರು, ಅನುಭವದಿಂದ ಮುಪ್ಪುರಿಗೊಂಡ ಹಿರಿಯರು, ಶಿಕ್ಷಣತಜ್ಞರು ಈ ಒಂದು ವಿಷಯದಲ್ಲಿ ಭಿನ್ನಾಭಿಪ್ರಾಯ ತಳೆದಿಲ್ಲ. ಅದು: ದುರ್ಜನರ ಸಹವಾಸವನ್ನು ಸರ್ವಪ್ರಯತ್ನದಿಂದ ಬಿಡಬೇಕು. ಎಷ್ಟೇ ಕಷ್ಟವಾದರೂ ಸಜ್ಜನರ ಸಹವಾಸವನ್ನು ತೊರೆಯಬಾರದು. ಸಜ್ಜನರನ್ನು ಗೌರವಿಸದ, ಅವರಿಂದ ಮಾರ್ಗದರ್ಶನವನ್ನು ಪಡೆಯದ ಜನಾಂಗ ಮೇಲೇರಲು ಸಾಧ್ಯವಿಲ್ಲ.

\vskip 2pt

ಸಜ್ಜನರ ಸಂಗವದು ಹೆಜ್ಜೇನು ಸವಿದಂತೆ, ದುರ್ಜನರ ಕೂಡೆ ಒಡನಾಟ ಬಚ್ಚಲಿನ ರೊಚ್ಚಿ ನಂತಿಹುದು–ಎಂದು ಸರ್ವಜ್ಞ ಹೇಳಿದ ಮಾತು ಒಂದು ಶಾಶ್ವತ ಸತ್ಯ.

\vskip 2pt

ಮನುಷ್ಯರು ಪ್ರಜ್ಞಾಪೂರ್ವಕವಾಗಿ ಕಲಿಯುವಂತೆ, ಸ್ವಾಭಾವಿಕವಾಗಿ ತನಗೆ ಅರಿವಿಲ್ಲದೆ ಅನುಕರಣೆಯಿಂದ ಕಲಿಯುವ ರೀತಿ ಅತ್ಯಂತ ಪರಿಣಾಮಕಾರಿಯಾಗಿರುತ್ತದೆ. ಈ ವಿಚಾರವನ್ನು ತಿಳಿದವರಿಗೆ ಜ್ಞಾನಿಗಳ ಈ ಎಚ್ಚರಿಕೆಯ ಮಾತುಗಳು ಅರ್ಥಪೂರ್ಣವಾಗುತ್ತವೆ.


\section*{ಮೂರು ವರ್ಷದ ಬುದ್ಧಿ}

\addsectiontoTOC{ಮೂರು ವರ್ಷದ ಬುದ್ಧಿ}

ತಾಯಿಯೊಬ್ಬಳು ಶಿಕ್ಷಣತಜ್ಞರನ್ನು ಕಂಡು ತನ್ನ ಮಗುವಿಗೆ ಎಂದಿನಿಂದ ಶಿಕ್ಷಣ ಪ್ರಾರಂಭಿಸ ಬೇಕೆಂದು ಕೇಳಿದಳಂತೆ. ತಜ್ಞರು ‘ಮಗುವಿನ ವಯಸ್ಸು ಎಷ್ಟು?’ ಎಂದು ಕೇಳಿದಾಗ ‘ಇನ್ನೂ ಮೂರು ವರ್ಷವಷ್ಟೇ’ ಎಂದಳವಳು, ‘ಅಯ್ಯೋ ಮೂರು ವರ್ಷವೇ? ಇನ್ನೂ ಶಿಕ್ಷಣ\break ಆರಂಭಿಸಿಲ್ಲವೇ? ಬೇಗನೇ ಮನೆಗೆ ಹೋಗಿ ಆರಂಭಿಸಿ. ಆಗಲೇ ಮೂರು ವರ್ಷಗಳು ಸಂದುಹೋದವು’ ಎಂದರಂತೆ.

ಮಗುವು ಈ ಜಗತ್ತಿನ ಬೆಳಕನ್ನು ಕಂಡ ದಿನದಿಂದಲೇ ಅದರ ಶಿಕ್ಷಣ ಪ್ರಾರಂಭವಾಗುತ್ತದೆ ಎನ್ನುತ್ತಾರೆ ಮನೋವಿಜ್ಞಾನಿಗಳು. ನಮ್ಮ ದೇಶದ ಹಿರಿಯರು ಶಿಶುವು ಮಾತೆಯ ಗರ್ಭದಲ್ಲಿರು ವಾಗಲೇ ಕಲಿಯಲು ಆರಂಭಿಸುತ್ತದೆ ಎನ್ನುತ್ತಾರೆ. ಎಂದರೆ ತಾಯಿಯ ನೋವು ನಲಿವುಗಳಿಗೆ, ಅಭಿರುಚಿ ಅನಿಸಿಕೆಗಳಿಗೆ ಅನುಗುಣವಾಗಿ ಅದು ಸ್ಪಂದಿಸುತ್ತದೆ, ಅದಕ್ಕನುಗುಣವಾಗಿ ಅದರ ವ್ಯಕ್ತಿತ್ವ ಚಾರಿತ್ರ್ಯ ರೂಪಿತವಾಗುತ್ತಿರುತ್ತದೆ–ಎನ್ನುವುದು ಅವರ ಅಂಬೋಣ. ತಾಯಿಯ ಪ್ರೀತಿ ಮತ್ತು ಸುರಕ್ಷೆಯನ್ನು ಕಂಡರಿಯದ ಶಿಶು ತನ್ನ ವಯಸ್ಕ ಜೀವನದಲ್ಲಿ ವ್ಯಕ್ತಿತ್ವದ ಪೂರ್ಣ ಬೆಳವಣಿಗೆಗೆ ಬೇಕಾದ ಮುಖ್ಯಾಂಶವನ್ನು ಕಳೆದುಕೊಂಡಿರುತ್ತದೆ. ಆದುದರಿಂದಲೇ ‘ಮಕ್ಕಳಿ ಸ್ಕೂಲ್ ಮನೇಲಲ್ವೆ’, ‘ಹಿರಿಯರ ವರ್ತನೆ ಕಿರಿಯರ ಕಾಪಿಪುಸ್ತಕ’, ‘ಹಿರಿಯಕ್ಕನ ಚಾಳಿ ಮನೆ ಮಂದಿಗೆ’, ‘ಮೂರು ವರ್ಷದ ಬುದ್ಧಿ ನೂರು ವರ್ಷದ ತನಕ’ ಎನ್ನುವ ಅನುಭವದ ಮಾತುಗಳು ಮಗುವಿನ ಮನಸ್ಸನ್ನು ರೂಪಿಸುವಲ್ಲಿ ಮನೆಯ ವಾತಾವರಣ ಹಾಗೂ ತಂದೆತಾಯಂದಿರ ಹೊಣೆಗಾರಿಕೆಯ ಮಹತ್ವವನ್ನು ಸೂಚಿಸುತ್ತವೆ.

ಶಿಕ್ಷಣವೆಂದರೆ ಓದು ಬರಹ ಮಾತ್ರವೆಂದು ನಾವು ತಿಳಿದಂತಿದೆ. ಮಗು ಶಾಲೆಗೆ ಹೋಗುವ ಮೊದಲು ಭಾಷೆಯನ್ನು ಕಲಿಯುವುದೆಂತು ಎಂದು ಪರಿಶೀಲಿಸಿದರೆ ಸಾಕು ಅದರ ಕಲಿಕೆಯ ರಹಸ್ಯ ನಮಗೆ ಮನವರಿಕೆಯಾಗುವುದು. ಮಗು ಅನುಕರಣೆಯಿಂದಲೇ ಎಲ್ಲವನ್ನೂ ಕಲಿಯುತ್ತದೆ. ತಾಯಿತಂದೆಗಳ, ಸೋದರ ಸೋದರಿಯರ ಮಾತುಗಳನ್ನು ಕೇಳುತ್ತ ಮೆಲ್ಲಮೆಲ್ಲನೆ ಶಬ್ದಗಳನ್ನು ಉಚ್ಚರಿಸುತ್ತ ಸ್ವಾಭಾವಿಕವಾಗಿ ಭಾಷೆಯನ್ನು ಕಲಿತುಬಿಡುತ್ತದೆ. ಭಾಷೆ ಮಾತ್ರವಲ್ಲ, ಹಿರಿಯರ ಪ್ರತಿಯೊಂದು ವರ್ತನೆಯ ವಿಧಾನವನ್ನೂ ಅನುಸರಿಸುತ್ತದೆ. ಹಿರಿಯರ ಆಸೆ ಆಸಕ್ತಿಗಳನ್ನೂ, ರಾಗದ್ವೇಷಗಳನ್ನೂ, ರಚನಾತ್ಮಕವೊ ನಿಷೇಧಾತ್ಮಕವೊ ಆದ ಭಾವನೆಗಳನ್ನೂ, ಮತಾಂಧತೆ ಮತಿಹೀನತೆಯ ವರ್ತನೆಗಳನ್ನೂ ತನ್ನದಾಗಿ ಮಾಡಿಕೊಳ್ಳುತ್ತದೆ. ಮಗುವಿನ ಭವಿಷ್ಯವನ್ನು ರೂಪಿ ಸುವುದರಲ್ಲಿ ಹಿರಿಯರ ಹೊಣೆಗಾರಿಕೆ ಎಷ್ಟೊಂದು ಮಹತ್ವದ್ದು ಎಂಬುದನ್ನು ಹೆಚ್ಚಿನವರು ಗಮನಿಸುತ್ತಿಲ್ಲ.


\section*{ಮಕ್ಕಳ ಮೇಲ್ಮೆಗೆ}

\addsectiontoTOC{ಮಕ್ಕಳ ಮೇಲ್ಮೆಗೆ}

ಈಗಿನ ದಿನಗಳಲ್ಲಿ ಎಷ್ಟೋ ಮಂದಿ ತಂದೆತಾಯಂದಿರು ತಮ್ಮ ಮಕ್ಕಳ ಶಿಕ್ಷಣಕ್ಕಾಗಿ ಹಣ ವೆಚ್ಚ ಮಾಡಲು ಹಿಂಜರಿಯುವುದಿಲ್ಲ. ಮಕ್ಕಳ ಭವಿಷ್ಯ ಭವ್ಯವಾಗಬೇಕೆಂಬ ಅವರ ಕನಸು ಸಹಜ\-ವಾದುದೇ. ಆದರೆ ತಮ್ಮ ಮಕ್ಕಳು ಕಲಿಕೆಯ ಸಮಯದಲ್ಲಿ ಎಂಥ ಯೋಚನೆ ಭಾವನೆಗಳನ್ನು ಸಂಗ್ರಹಿಸುತ್ತಿದ್ದಾರೆ, ಎಂಥ ಅಭ್ಯಾಸಗಳನ್ನು ರೂಢಿಸಿಕೊಳ್ಳುತ್ತಿದ್ದಾರೆ ಎಂಬುದನ್ನು ಕುರಿತು ಅವರು ಜಾಗರೂಕರಾಗಿದ್ದಾರೆಯೇ? ಓದು ಬರಹ ಕಲಿತ ಮಕ್ಕಳು ಜೀವನ ಸಂಗ್ರಾಮವನ್ನು ಎದುರಿಸಲು ಸಹಾಯಕವಾಗುವ ವಿಚಾರಗಳಿಗಿಂತಲೂ ಮನುಷ್ಯರ ದೌರ್ಬಲ್ಯಗಳನ್ನು ರಂಜಕವಾಗಿ ತೋರಿ\-ಸುವ, ವರ್ಣಿಸುವ ಅಸಂಖ್ಯ ಚಲನಚಿತ್ರ ಮತ್ತು ಬರಹಗಳಿಂದ ಸಾಕಷ್ಟು ಪ್ರಭಾವಿತರಾಗುತ್ತಾರೆ.\break ಸಂಶೋಧಕ ವಿಜ್ಞಾನಿಗಳ, ಸಾಹಸೀ ಉದ್ಯಮಶೀಲ ವ್ಯಕ್ತಿಗಳ, ಸ್ವಾರ್ಥ ತ್ಯಾಗಿಗಳ, ಕರ್ಮವೀರರ, ದೇಶಪ್ರೇಮಿ ಸಮಾಜಸೇವಕರ, ದೈವವನ್ನು ನೆಚ್ಚಿ ಬದುಕಿದ ಸಾಧು ಸಂತರ ಜೀವನ ಸಂದೇಶಗಳನ್ನು ಅವರು ಓದುವುದೇ ಇಲ್ಲವೆಂದರೂ ಸಲ್ಲುವುದು. ಮನುಷ್ಯರ ಶೀಲ ಸಂವರ್ಧನೆಗೆ ಬೇಕಾಗುವ ಸಾಮಗ್ರಿಗಳನ್ನು ಬೆಳವಣಿಗೆಯ ಅತ್ಯುತ್ತಮ ಸಮಯದಲ್ಲಿ ಮಕ್ಕಳಿಗೆ ನೀಡಲಾಗದಿರುವುದು ಒಂದು ದೊಡ್ಡ ದೌರ್ಭಾಗ್ಯ ಮಾತ್ರವಲ್ಲ, ನಮ್ಮ ಶಿಕ್ಷಣತಜ್ಞ ರೆನ್ನಿಸಿಕೊಂಡವರ, ಹೆತ್ತವರ ಅಸಹಾಯಕತೆ ಅಥವಾ ದೂರದರ್ಶಿತ್ವದ ಅಭಾವಕ್ಕೆ ಹಿಡಿದ ಕನ್ನಡಿಯೂ ಹೌದು. ವ್ಯಕ್ತಿಯ ಚಾರಿತ್ರ್ಯ ನಿರ್ಮಾಣದಲ್ಲಿ ಯೋಚನೆ ಮತ್ತು ಒಳ್ಳೆಯ ಸಂಸ್ಕಾರಗಳನ್ನು ನಿರ್ಮಿಸುವ ಅಭ್ಯಾಸಗಳ ಪಾತ್ರದ ಮೌಲ್ಯ ಮಾಹಾತ್ಮ್ಯೆಗಳನ್ನು ತಿಳಿದವರಿಗೆ ಮಾತ್ರ ಈ ಮಾತು ಅರ್ಥವಾದೀತು.

ಅಭ್ಯಾಸಗಳು ಮೊದಲಿಗೆ ಜೇಡನ ಬಲೆಯ ನೂಲಿನಂತೆ. ಆದರೆ ಬರಬರುತ್ತ ಕಬ್ಬಿಣದ ತಂತಿ ಗಳಂತೆ ನಮ್ಮನ್ನು ಬಲವಾಗಿ ಹಿಡಿದುಕೊಳ್ಳುತ್ತವೆ ಎನ್ನುವ ಗಾದೆ ಬಾಲ್ಯದಿಂದಲೇ ಒಳ್ಳೆಯ ಅಭ್ಯಾಸಗಳನ್ನು ರೂಢಿಸಿಕೊಳ್ಳುವ ಆವಶ್ಯಕತೆಯನ್ನು ಸಾರುತ್ತದೆ. ಎಲ್ಲ ಅಭ್ಯಾಸಗಳಿಗಿಂತಲೂ ಅತ್ಯಂತ ಪ್ರಮುಖವಾದ ಅಭ್ಯಾಸ ಯಾವುದು? ‘ಒಂದು ಒಳ್ಳೆಯ ಅಭ್ಯಾಸವನ್ನು ರೂಢಿಸಿ ಕೊಳ್ಳಲು ನಾವು ಬೆಳೆಸಿಕೊಳ್ಳಬೇಕಾದ ಎಚ್ಚರ ಅಥವಾ ಜಾಗರೂಕತೆಯ ಅಭ್ಯಾಸ’. ಒಳ್ಳೆಯತನದಲ್ಲಿ ವಿಶ್ವಾಸ, ಒಳ್ಳೆಯವನಾಗಬೇಕೆಂಬ ಸಂಕಲ್ಪಗಳೇ ನಮ್ಮನ್ನು ದೋಷಮುಕ್ತರನ್ನಾಗಿ ಮಾಡುವ\break ಶಕ್ತಿಯನ್ನು ನೀಡುವುದಿಲ್ಲ. ನಾವು ರೂಢಿಸಿಕೊಂಡ ಬೇಡದ ಅಭ್ಯಾಸಗಳನ್ನು ಮುರಿಯ ಬೇಕು. ಹೊಸ ಅಭ್ಯಾಸಗಳನ್ನು ನಮ್ಮದಾಗಿ ಮಾಡಿಕೊಳ್ಳಬೇಕು. ವ್ಯಕ್ತಿಯೊಬ್ಬ ಇಪ್ಪತ್ತರಿಂದ ಮೂವತ್ತು ವರ್ಷ ವಯಸ್ಸಿನೊಳಗೆ ಬುದ್ಧಿಶಕ್ತಿಯ ಬೆಳವಣಿಗೆ ಮತ್ತು ಉದ್ಯೋಗಕ್ಕೆ ಸಂಬಂಧಿ ಸಿದ ಅಭ್ಯಾಸಗಳನ್ನು ಸಂಗ್ರಹ ಮಾಡಿರುತ್ತಾನೆ. ವೈಯಕ್ತಿಕ ಶುಚಿತ್ವದ ನಿಯಮ, ಮಾತು ಮತ್ತು ಉಚ್ಚಾರಣೆಯ ವಿಧಾನ, ಹಾವಭಾವ, ಅಂಗಚೇಷ್ಟೆ ನಡೆ–ಇವುಗಳನ್ನು ಇಪ್ಪತ್ತರೊಳಗೇ ಕಲಿತು ಬಿಡುತ್ತಾನೆ\break ಎನ್ನುತ್ತಾರೆ.

‘ಕೆಲವೆ ದಿನಗಳಲ್ಲಿ ನಡೆದಾಡುವ ಅಭ್ಯಾಸದ ಮೂಟೆಗಳು ನಾವಾಗುತ್ತೇವೆ’ ಎಂಬುದನ್ನು ಬದುಕಿಗೆ ರೂಪ ನೀಡಲು ಸಾಧ್ಯವಾಗುವ ಬಾಲ್ಯಕಾಲದಲ್ಲೇ ಚಿಕ್ಕವರು ತಿಳಿದುಕೊಂಡರೆ ಅವರು ತಮ್ಮ ನಡತೆಯ ಕಡೆಗೆ ಹೆಚ್ಚು ಗಮನವೀಯಬಲ್ಲರು. ‘ಎಂದೆಂದೂ ಬದಲಿಸಲಾಗದ ಅದೃಷ್ಟವನ್ನು ನಾವು ನಿತ್ಯವೂ ಹೆಣೆಯುತ್ತಿರುತ್ತೇವೆ’ ಎಂದು ಮನೋವಿಜ್ಞಾನಿ ಹೇಳುತ್ತಾನೆ. ಆದುದರಿಂದ ಅಕ್ಷರ ವಿದ್ಯೆಯ ಜೊತೆಗೆ ಬದುಕನ್ನು ಹಸನಾಗಿಸುವ ವಿವಿಧ ಸಂಸ್ಕಾರಗಳ ನಿರ್ಮಾಣದ ಕಡೆಗೂ ಹೆತ್ತವರೂ, ಶಿಕ್ಷಕರೂ ಎಷ್ಟೊಂದು ಗಮನವೀಯಬೇಕು!

ಆವಾಸಿಕ (ರೆಸಿಡೆನ್ಸಿಯಲ್​) ಶಿಕ್ಷಣ ಸಂಸ್ಥೆಯಲ್ಲಿ ದೀರ್ಘಕಾಲ ಸೇವೆ ಸಲ್ಲಿಸಿದ ಹಿರಿಯ ರೊಬ್ಬರು ತಮ್ಮ ಅನುಭವವನ್ನು ಹೀಗೆಂದು ಹೇಳಿದ್ದರು–

ನಮ್ಮ ಆವಾಸಿಕ ಶಿಕ್ಷಣ ಸಂಸ್ಥೆಯಲ್ಲಿ ಐದು ವರ್ಷಗಳ ಕಾಲವಿದ್ದು ಚೆನ್ನಾಗಿ ಓದಿ ಪಾಸು ಮಾಡಿ ಇದೀಗ ಒಳ್ಳೆಯ ಉದ್ಯೋಗ, ಸ್ಥಾನಮಾನ ಗಳಿಸಿದ ಹಳೆಯ ವಿದ್ಯಾರ್ಥಿಯನ್ನು ಪ್ರಶ್ನಿ ಸಿದೆ: ‘ನೀನು ಬೇರೆ ಕಡೆ ಕಲಿಯಲು ಅನುಕೂಲವಿರದೆ ಇಲ್ಲಿ ಕಲಿಯಲು ಸಾಧ್ಯವಾದ ಒಂದು ವಿಶಿಷ್ಟ ಅಭ್ಯಾಸ ಅಥವಾ ನಡವಳಿಕೆ ಯಾವುದು?’

ಆತ ಉತ್ತರಿಸಿದ: ‘ಸರ್, ನಿಮ್ಮ ಹತ್ತಿರ ಸತ್ಯ ಹೇಳುತ್ತೇನೆ. ನಾನು ಮೊದಲು ಮೂರು ನಾಲ್ಕು ದಿನಗಳಿಗೊಮ್ಮೆ ಸ್ನಾನಮಾಡುತ್ತಿದ್ದೆ. ಐದು ವರ್ಷಗಳು ಇಲ್ಲಿದ್ದು ದಿನನಿತ್ಯ ಸ್ನಾನದ ಅಭ್ಯಾಸವಾಗಿಬಿಟ್ಟಿದೆ. ಈಗ ಒಂದು ದಿನವೂ ಸ್ನಾನ ಬಿಡಲಾರೆ.’

ಸ್ನಾನ ಮಾಡುವುದು ಒಳಿತು ಎಂದು ತಿಳಿದ ಮಾತ್ರಕ್ಕೆ ದಿನನಿತ್ಯ ಸ್ನಾನ ಮಾಡುವ ಅಭ್ಯಾಸ\-ವಾಗಿ ಬಿಡುವುದಿಲ್ಲ. ದೀರ್ಘಕಾಲ ತಪ್ಪದೇ ನಿತ್ಯವೂ ಸ್ನಾನ ಮಾಡಿದಾಗ ಅದು ಅಭ್ಯಾಸವಶವಾಗುತ್ತದೆ. ಬಾಲ್ಯದಿಂದಲೇ ಹಿರಿಯರು ದಿನವೂ ಮಾಡಿಸಿದರೆ ಅದು ಸಾಧ್ಯವಾಗುತ್ತದೆ.

ಆವಾಸಿಕ ಶಿಕ್ಷಣ ಸಂಸ್ಥೆಯಲ್ಲಿದ್ದ ನನಗೆ ಬೇರೆ ಬೇರೆ ಪರಿಸರಗಳಿಂದ ಬಂದ ವಿದ್ಯಾರ್ಥಿಗಳನ್ನೂ, ಅವರ ಸಂಸ್ಕಾರ ಸಾಮರ್ಥ್ಯ ಅಭಿರುಚಿಗಳನ್ನೂ ಪರಿಶೀಲಿಸುವ ಅವಕಾಶವಿತ್ತು. ಒಂದು ಪ್ರಾತಃಕಾಲ ನೂತನ ವಿದ್ಯಾರ್ಥಿಗಳು ಮುಖಮಾರ್ಜನ ಮಾಡುವ ವಿಧಾನವನ್ನು ಕಂಡು ಆಶ್ಚರ್ಯ\-ವಾಯಿತು. ಕಲ್ಲಿಗೆ ಅಂಟಿಕೊಂಡ ಪಾಚಿಯನ್ನು ಮರಳು ಹಾಕಿ ಬಲವಾಗಿ ಉಜ್ಜುವಂತೆ ಪೇಸ್ಟನ್ನು ಸವರಿ ಬ್ರಶ್ಶಿನಿಂದ ಜೋರಾಗಿ ಸದ್ದುಬರುವಂತೆ ಹಲ್ಲುಜ್ಜುತ್ತಿದ್ದರು. ವಸಡಿನಿಂದ ರಕ್ತ ಬರುವಷ್ಟು ಜೋರಾಗಿ ಹಲ್ಲುಜ್ಜುವ ಅಭ್ಯಾಸ ಕೆಲವರದು! ಎಲ್ಲ ಹಲ್ಲುಗಳೂ ಸ್ಪಷ್ಟವಾಗಿ ಕಾಣಿಸುವ ತಲೆಬುರುಡೆಯನ್ನು ತೋರಿಸಿ, ಆಹಾರದ ತುಣುಕುಗಳು ಹಲ್ಲಿನ ಸಂದುಗಳಲ್ಲಿ ಹೇಗೆ ಸಿಕ್ಕಿಕೊಳ್ಳುತ್ತವೆ, ಅವು ಹಲ್ಲನ್ನು ಹೇಗೆ ಹಾಳುಮಾಡುತ್ತವೆ, ವಸಡಿಗೆ ಆಘಾತವಾಗದಂತೆ ಬ್ರಶ್ಶನ್ನು ವಿವಿಧ ರೀತಿಯಲ್ಲಿ ಹಲ್ಲುಗಳ ಒಳಭಾಗ ಮತ್ತು ಹೊರಗೆ ಹೇಗೆ ತಿರುಗಿಸಬೇಕು ಎಂಬು ದನ್ನು ವಿವರಿಸಿ ತೋರಿಸಿದೆ. ಹೇಳಿಕೊಟ್ಟ ನೂತನ ವಿಧಾನವನ್ನು ಅನುಸರಿಸುತ್ತಿದ್ದಾರೆಯೇ ಎಂಬು ದನ್ನೂ ಎರಡು ಮೂರು ದಿನಗಳ ಕಾಲ ಪರಿಶೀಲಿಸಿದೆ. ಹಲವರು ಹಳೆಯ ಅಭ್ಯಾಸಕ್ಕೆ ಅಂಟಿ ಕೊಂಡಿದ್ದರು.

‘ಹಲ್ಲನ್ನು ಉಜ್ಜಿ ಶುಚಿಯಾಗಿಟ್ಟುಕೊಳ್ಳಿ’... ಎಂಬ ಬೋಧನೆ ಮಾತ್ರದಿಂದ ವಿದ್ಯಾರ್ಥಿಗಳು ಆ ಅಭ್ಯಾಸವನ್ನು ರೂಢಿಸಿಕೊಳ್ಳುವರೇ?

ನಾವು ಸ್ವತಃ ಮಾಡಿ ತೋರಿಸಿ, ಕೆಲವೊಮ್ಮೆ ಪ್ರೀತಿಯಿಂದ, ಕೆಲವೊಮ್ಮೆ ಕೆಲವರನ್ನು ಗದರಿಸಿ, ಹೇಳಿ ತಿದ್ದಿ ತೀಡಿ ಆ ಅಭ್ಯಾಸ ದೃಢವಾಗುವಂತೆ ಮಾಡಬೇಕಲ್ಲವೇ?

‘ತಟ್ಟೆ, ಲೋಟ, ಚಮಚೆಗಳನ್ನು ನಾವೇ ತೊಳೆದುಕೊಳ್ಳೋಣ. ಅದಕ್ಕಾಗಿ ಬೇರೆಯೇ ಕೆಲಸ ಗಾರರು ಬೇಡ. ಸ್ವಲ್ಪ ಉಳಿತಾಯ ಸಾಧ್ಯವಾಗುತ್ತದೆ ಎಂದು ಸೂಚಿಸಿದೆ. ಆದರೆ ಎಲ್ಲರೂ ನಾನೊಬ್ಬ ವ್ಯವಹಾರಜ್ಞಾನವಿಲ್ಲದ ಪ್ರಾಣಿ ಎನ್ನುವಂತೆ ಗೇಲಿ ಮಾಡಿದರು’–ತಾಂತ್ರಿಕ ಶಿಕ್ಷಣಾಲಯ\-ದಲ್ಲಿ ಅಧ್ಯಯನ ಮಾಡುತ್ತಿದ್ದ ವಿದ್ಯಾರ್ಥಿಯೊಬ್ಬ ತನ್ನ ಒಂದು ಅನುಭವವನ್ನು ಹಾಗೆಂದು ಹೇಳಿಕೊಂಡ. ಹಾಸ್ಟೆಲಿನ ಮೀಟಿಂಗಿನಲ್ಲಿ ವಿದ್ಯಾರ್ಥಿ ಕಾರ್ಯನಿರ್ವಾಹಕರು ಕೆಲಸಗಾರರ\break ಸಮಸ್ಯೆಯನ್ನು ಚರ್ಚಿಸುತ್ತಿದ್ದಾಗ ಆತ ಕೊಟ್ಟ ಸೂಚನೆಗೆ ಅವನ ಮಿತ್ರರು ತೋರಿಸಿದ ಪ್ರತಿಕ್ರಿಯೆ ಅದು.

ಓದುಬರಹ ಕಲಿತು ವಿದ್ಯಾವಂತರೆನಿಸಿಕೊಂಡವರಲ್ಲಿ ಹೆಚ್ಚಿನವರು ಯಾವುದೇ ದೈಹಿಕ ಶ್ರಮವನ್ನು ಕೀಳಾಗಿ ಭಾವಿಸುವ ಅಭ್ಯಾಸವನ್ನು ರೂಢಿಸಿಕೊಂಡು ಬಿಟ್ಟಿದ್ದಾರೆ ನಮ್ಮ ದೇಶದಲ್ಲಿ!

ಬೇಸಗೆಯ ರಜೆ ಮುಗಿಸಿ ಶಾಲೆಗೆ ಹಿಂದಿರುಗಿದ ವಿದ್ಯಾರ್ಥಿಯೊಬ್ಬ ಹಿರಿಯರೊಡನೆ ನಿಷ್ಕಪಟ ಭಾವದಿಂದ ಹೀಗೆಂದ: ‘ಸಾರ್, ಈ ಬಾರಿ ಮನೆಗೆ ಹೋದಾಗ ಶೌಚಕ್ಕೆ ಮೊದಲು ಮತ್ತು ಆಮೇಲೆ ಕಮೋಡಿಗೆ ನೀರು ಹಾಕಲು ಮರೆಯಲಿಲ್ಲ.’

ಶುಚಿತ್ವದ ಪ್ರಾಥಮಿಕ ನಿಯಮಗಳನ್ನು ಕಾರ್ಯರೂಪಕ್ಕೆ ತರುವಂಥ ಸಂಸ್ಕಾರ ನಿರ್ಮಾಣದಲ್ಲಿ ಅವನೊಂದು ಹೆಜ್ಜೆ ಇಟ್ಟಂತಾಯಿತಲ್ಲವೇ?

ವೈಯಕ್ತಿಕ ಮತ್ತು ಸಾಮಾಜಿಕ ಶುಚಿತ್ವದ ಅಭ್ಯಾಸಗಳು, ವಿಧೇಯತೆ, ಕ್ರಮಬದ್ಧತೆ, ಸಮಯ ಪ್ರಜ್ಞೆ, ಪರಸ್ಪರ ಸಹಕಾರ ಭಾವದಿಂದ ಒಂದು ಕೆಲಸದಲ್ಲಿ ಭಾಗಿಯಾಗುವುದು, ಗುರುಹಿರಿಯರಲ್ಲಿ ಗೌರವ, ಅನುಕಂಪೆಯ ಮನೋಭಾವ, ನಡವಳಿಕೆಯಲ್ಲಿ ವಿನಯ, ಅತಿಥಿ ಅಭ್ಯಾಗತರನ್ನು ಸತ್ಕರಿಸುವ ವಿಧಾನ, ಒಳ್ಳೆಯದನ್ನು ಕಂಡು ಮುಕ್ತಕಂಠದಿಂದ ಪ್ರಶಂಸಿಸುವುದು–ಈ ಎಲ್ಲ ಗುಣಗಳನ್ನು ಆರ್ಜಿಸಲು ಯೋಗ್ಯ ಮಾರ್ಗದರ್ಶಕರ ನೇತೃತ್ವದಲ್ಲಿ ತರಬೇತಿ ಅಗತ್ಯ. ವ್ಯಕ್ತಿಯಾಗಿ ತಾನು ಸುಖಿಯಾಗುವುದಕ್ಕೂ, ಸಾಮಾಜಿಕವಾಗಿ ಸಮಾಜಕ್ಕೆ ಸುಖವಾಗುವಂತೆ ನಡೆದುಕೊಳ್ಳುವುದಕ್ಕೂ ಅವಶ್ಯವೆನಿಸುವ ಸಂಸ್ಕಾರಗಳ ನಿರ್ಮಾಣವಾಗದಿದ್ದರೆ ಶಿಕ್ಷಣ ಅಪೂರ್ಣ ಮಾತ್ರವಲ್ಲ, ಅಪಾಯ\-ಕಾರಿಯೂ ಹೌದು.


\section*{ಅಸಾಧ್ಯವೂ ಸಾಧ್ಯ}

\addsectiontoTOC{ಅಸಾಧ್ಯವೂ ಸಾಧ್ಯ}

ಸುಮಾರು ಐವತ್ತು ವರ್ಷಗಳ ಹಿಂದೆ ಆಂಗ್ಲ ಮಾಸಪತ್ರಿಕೆಯೊಂದರಲ್ಲಿ ಜಪಾನ್ ಪ್ರವಾಸದಿಂದ ಹಿಂದಿರುಗಿದ ಭಾರತೀಯರೊಬ್ಬರು ‘ಜಪಾನ್ ಮಹಾರಾಷ್ಟ್ರವಾಗಲು ಕಾರಣವೇನು?’ ಎಂಬ ಲೇಖನವನ್ನು ಪ್ರಕಟಿಸಿದರು. ಅವರ ಹಲವು ಅನುಭವಗಳಲ್ಲಿ ಒಂದೆರಡನ್ನು ಮಾತ್ರ ಇಲ್ಲಿ ಓದುಗರ ಅವಗಾಹನೆಗೆ ತರುತ್ತಲಿದ್ದೇನೆ. ಅರ್ಧಶತಮಾನದ ಹಿಂದೆಯೇ ಒಂದು ಜನಾಂಗ ಸಾಮಾಜಿಕವಾಗಿ ಶಿಸ್ತು, ಶಾಂತಿ ಸಹಕಾರ ಭಾವನೆಯನ್ನು ಕಾರ್ಯರೂಪಕ್ಕೆ ತಂದ ಒಂದು ಘಟನೆ ಅದು:

‘ಜಪಾನಿನ ಒಂದು ಹಳ್ಳಿಯಲ್ಲಿ ಎರಡು ದಿನ ತಂಗಿದ್ದೆ. ಅಲ್ಲೊಂದು ಬಿಸಿ ನೀರಿನ ಊಟೆ (ಬುಗ್ಗೆ) ಇತ್ತು. ಅಲ್ಲಿಗೆ ಒಂದು ಜಪಾನೀ ಪ್ರವಾಸಿಗಳ ಗುಂಪು ಬಂದಿತ್ತು. ಸುಮಾರು ಐನೂರು ಮಂದಿ ವಯಸ್ಕರೂ, ನೂರೈವತ್ತು ಮಂದಿ ಮಕ್ಕಳೂ ಆ ತಂಡದಲ್ಲಿದ್ದರು. ಅಷ್ಟು ಮಂದಿ ಅಲ್ಲಿ ಸೇರಿದ್ದಾರೆಂಬುದನ್ನು ಪ್ರತ್ಯಕ್ಷ ನೋಡದಿದ್ದರೆ ಊಹಿಸುವಂತೆಯೂ ಇರಲಿಲ್ಲ. ಎಲ್ಲರೂ ಅಷ್ಟೊಂದು ಮೌನವಾಗಿದ್ದುಕೊಂಡು ತಮ್ಮ ತಮ್ಮ ಕೆಲಸಗಳಲ್ಲಿ ಮಗ್ನರಾಗಿರುವ ಅಭ್ಯಾಸವನ್ನು ರೂಢಿಸಿಕೊಂಡಿದ್ದರು! ಅಲ್ಲಿ ಒಂದು ಸಭಾಂಗಣದಲ್ಲಿ ಸಹಸ್ರಾರು ಮಂದಿ ಕಿಕ್ಕಿರಿದು ತುಂಬಿದ್ದರೂ ಪೂರ್ಣ ಮೌನ ನೆಲಸಿರುತ್ತದೆ!’

ನಡತೆಯಲ್ಲಿ ನಯ ಗಾಂಭೀರ್ಯ, ದುಡಿಮೆಯಲ್ಲಿ ಆಸಕ್ತಿ ಮತ್ತು ಏಕಾಗ್ರತೆ, ಪ್ರಾಮಾಣಿಕತೆ, ಸರಳ ಜೀವನವೇ ಮೊದಲಾದ ಗುಣಗಳನ್ನು ಗುಂಪಿನಲ್ಲೂ ಕಾಣಬಹುದೆಂಬುದನ್ನು ಲೇಖಕರು ತಮ್ಮ ಅನುಭವಗಳ ಆಧಾರದಿಂದ ಹೇಳಿದ್ದಾರೆ. ಜಪಾನೀಯರು ಬೆಳೆಸಿಕೊಂಡ ಕೃತಜ್ಞತಾ ಬುದ್ಧಿಯನ್ನೇ ಕುರಿತ ಹಲವು ನಿದರ್ಶನಗಳನ್ನು ನೀಡಿದ್ದಾರೆ.

‘ಗೆಣಸನ್ನು ಮೊದಲು ಒಂದು ಹಳ್ಳಿಗೆ ತಂದು ಜನರಿಗೆ ಪರಿಚಯ ಮಾಡಿಕೊಟ್ಟ ಓರ್ವ ಜಪಾನ್ ಪ್ರಜೆಯ ಸವಿನೆನಪಿಗಾಗಿ ಒಂದು ಸ್ಮೃತಿ ಸ್ತಂಭವನ್ನು ನಿಲ್ಲಿಸಿದ್ದಾರೆ. ಒಂದು ಒಳ್ಳೆಯ ಮರವನ್ನು ನೆಟ್ಟ ವ್ಯಕ್ತಿಗೆ, ಸಹಕಾರ ಸಂಘವನ್ನು ಸ್ಥಾಪಿಸಿ ಜನರಿಗೆ ಉಪಕಾರ ಮಾಡಿದವನೊಬ್ಬನಿಗೆ, ರಷ್ಯಾ ಚೈನಾ ಯುದ್ಧಗಳಲ್ಲಿ ಮಡಿದ ಸೈನಿಕರಿಗೆ, ಕುಸ್ತಿಯ ಪಂದ್ಯದಲ್ಲಿ ಗೆದ್ದ ಯುವಕನಿಗೆ ಅವರವರ ಗ್ರಾಮಗಳಲ್ಲಿ ಒಂದೊಂದು ಸ್ಮೃತಿ ಸಂಕೇತಗಳಿವೆ. ಆ ಸ್ಮೃತಿಸಂಕೇತಗಳು ಸರಳವಾಗಿದ್ದರೂ, ಸಮಾಜದ ಅಭ್ಯುದಯ ಅಭಿವೃದ್ಧಿಗೆ ಅತಿ ಪುಟ್ಟ ಉಪಕಾರ ಮಾಡಿದವರನ್ನು ನೆನೆಯುವ ಜನರ ಸೌಜನ್ಯವನ್ನು ಸೂಚಿಸುವ ಸಾಕ್ಷಿಗಳಾಗಿವೆ. ಎಳೆಯರು ಒಳ್ಳೆಯದನ್ನು ಕಂಡು ಗುರುತಿಸಿ ಕೃತಜ್ಞತೆಯಿಂದ ಅದನ್ನು ಸ್ಮರಿಸಿ ಗೌರವಿಸುವ ಸಂಸ್ಕಾರವನ್ನು ರೂಢಿಸಿಕೊಳ್ಳಲು ಪ್ರೇರಕವಾಗುವ ಆದರ್ಶ ಅದು.’

ಕೆಲವು ವರುಷಗಳ ಹಿಂದೆ ಕನ್ನಡದಲ್ಲಿ ಒಂದು ಪುಸ್ತಕ ಪ್ರಕಟವಾಯಿತು. ಅದರ ಹೆಸರು ‘ಎಕ್ಸ್​ಪೊ ೭೦’. ಪ್ರಕಟಿಸಿದವರು ಬೆಂಗಳೂರಿನ ಕರ್ನಾಟಕ ಸಹಕಾರಿ ಪ್ರಕಾಶನ ಮಂದಿರದವರು. ಅದರಲ್ಲಿ ಯುವಜನ ನಿಯೋಗದ ಸದಸ್ಯರ ಪ್ರವಾಸಾನುಭವವನ್ನು ವಿವರಿಸುವ ಇಪ್ಪತ್ತು ಪ್ರಬಂಧಗಳಿವೆ. ಪ್ರತಿಯೊಬ್ಬರೂ ತಾವು ಜಪಾನಿನಲ್ಲಿ ಕಂಡ ಜನಜೀವನ, ಆ ಜನಾಂಗದ ಸಾಧನೆ, ಸಿದ್ಧಿಗಳನ್ನು ಹೃದಯಂಗಮವಾಗಿ ತಿಳಿಸಿದ್ದಾರೆ. ನಮ್ಮ ನಾಡಿನ ತರುಣರೆಲ್ಲರೂ ಓದ ಬೇಕಾದ ಪುಸ್ತಕ ಅದು. ಅಲ್ಲಿ ಕಾಣಸಿಗುವ ಒಂದು ಘಟನೆಯ ವಿವರಣೆಯನ್ನು ಓದಿ:

‘ಟೋಕಿಯೋ ವಿಶ್ವವಿದ್ಯಾಲಯದಲ್ಲಿ ಮೂವತ್ತೇಳು ಲಕ್ಷ ಪುಸ್ತಕಗಳಿವೆ. ಅನೇಕ ವರ್ಷಗಳಿಂದ ಅಲ್ಲಿ ಪುಸ್ತಕ ಕಳುವಾಗಿಲ್ಲ. ಪುಸ್ತಕವನ್ನು ದುರುಪಯೋಗ ಮಾಡುವ ವಿದ್ಯಾರ್ಥಿಗೆ ಶಿಕ್ಷೆ ಏನು ಎಂದು ಗ್ರಂಥಪಾಲರನ್ನು ಕೇಳಿದಾಗ ಆತ ತನಗೆ ತಿಳಿಯದು ಎಂದು ಹೇಳಿ ನಿಬಂಧನೆಗಳನ್ನು ನೋಡ ಹೋದ. ಅಂತಹ ಪ್ರಸಂಗವೇ ಆತನ ಅನುಭವದಲ್ಲಿ ಬಂದಿಲ್ಲ. ಆತನಿಗೆ ಹೇಗೆ ತಿಳಿದಿರಬೇಕು!’

ಒಂದು ಜನಾಂಗ ತನ್ನ ತರುಣ ಪೀಳಿಗೆಯಲ್ಲಿ ಎಂಥ ಉನ್ನತಮಟ್ಟದ ಸಂಸ್ಕಾರಗಳನ್ನು ನಿರ್ಮಿಸಬಲ್ಲದು! ಬಾಲ್ಯದಿಂದಲೇ ತರಬೇತನ್ನು ನೀಡಿ ಎಂಥ ಸುಯೋಗ್ಯ ಪ್ರಜೆಗಳನ್ನಾಗಿಸ ಬಲ್ಲದು! ಯೋಚಿಸಿ ನೋಡಿ.

‘ಅಮೇರಿಕದ ಮಾಯಾಲೋಕವೆಂದು ಪ್ರಸಿದ್ಧವಾದ ಡಿಸ್ನೀ ಲ್ಯಾಂಡಿಗೆ ದಿನವೂ ಸುಮಾರು ನಲ್ವತ್ತು ಐವತ್ತು ಸಹಸ್ರ ಜನ ಸಂದರ್ಶಕರು ಬರುತ್ತಾರೆ. ರಜಾ ದಿನಗಳಲ್ಲಿ ಆ ಸಂಖ್ಯೆ ಒಂದು\break ಲಕ್ಷದವರೆಗೂ ಏರುತ್ತದೆ. ಆದರೆ ಅಲ್ಲೆಲ್ಲೂ ನೂಕು ನುಗ್ಗಲು ಇಲ್ಲ. ವ್ಯವಸ್ಥೆ ಎಷ್ಟು ಅಚ್ಚು ಕಟ್ಟಾಗಿರುವುದೆಂದರೆ ಕೆಲವು ಕಡೆ ಒಂದು ಫರ್ಲಾಂಗು ಕ್ಯೂನಲ್ಲಿ ನಿಂತಿದ್ದರೂ ತುಂಬ ಕಾಯು ತ್ತಿದ್ದೇವೆ ಎಂಬ ಭಾವನೆ ಬರುತ್ತಿರಲಿಲ್ಲ. ಬಹುಬೇಗನೆ ತಮ್ಮ ಸರದಿ ಬರುತ್ತಿದೆ ಎನ್ನಿಸುತ್ತಿತ್ತು.’ ಇತ್ತೀಚೆಗೆ ಅಮೇರಿಕದಿಂದ ಹಿಂದಿರುಗಿದ ಮಿತ್ರರು ಹೇಳಿದ ಮಾತಿದು.


\section*{ಅಭ್ಯಾಸದ ಹಿನ್ನೆಲೆ}

\addsectiontoTOC{ಅಭ್ಯಾಸದ ಹಿನ್ನೆಲೆ}

ಒಂದು ಹೊಸ ಅಭ್ಯಾಸವನ್ನು ಕಲಿತು ರೂಢಿಸಿ ದೃಢಪಡಿಸಿಕೊಂಡು ಅದನ್ನು ಮನಸ್ಸಿನಲ್ಲಿ ನೆಲೆ ನಿಲ್ಲುವಂತೆ ಮಾಡಲು ಅತ್ಯಂತ ದೃಢಸಂಕಲ್ಪದಿಂದ ಪ್ರಾರಂಭಿಸಬೇಕು. ಚಂಚಲತೆ, ಅನಿಶ್ಚಿತತೆ–ಇವು ಯಾವ ಶಿಸ್ತಿಗೂ ಒಳಪಡದ ದುರ್ಬಲ ಮನಸ್ಸಿನ ಸ್ಥಿತಿ. ಈ ತೆರನಾದ ಮನಸ್ಸಿನ ಶಕ್ತಿ ಏಕಕಾಲದಲ್ಲಿ ಹಲವು ಕಡೆ ಹರಿದು ವ್ಯರ್ಥವಾಗಿ ವ್ಯಯವಾಗುತ್ತದೆ. ಇಂಥ ವ್ಯಕ್ತಿ ಗಳಿಂದ ಯಾವ ಉತ್ತಮ ಕಾರ್ಯವೂ ಸಾಧ್ಯವಾಗದು. ಸಣ್ಣ ಕೆಲಸವನ್ನಾದರೂ ನಿಯಮ ಪೂರ್ವಕವಾಗಿ ಮನಗೊಟ್ಟು ಮಾಡುವುದರಿಂದ ಕೆಲಸ ಮಾಡುವ ಶಕ್ತಿ ಸಂಚಯವಾಗುವುದು.

\vskip 2pt

ಪರಿಣತ ಸೈಕಲ್ ಸವಾರನನ್ನು ಪರಿಶೀಲಿಸಿ. ಸೈಕಲ್ ಸವಾರಿ ಮಾಡುತ್ತಿರುವಾಗಲೇ ಸ್ನೇಹಿತನ ಹತ್ತಿರ ಮಾತನಾಡುತ್ತ ಸುತ್ತಲಿನ ಪ್ರಾಕೃತಿಕ ಸೌಂದರ್ಯವನ್ನು ಸವಿಯುತ್ತ, ರಸ್ತೆಯಲ್ಲಿ ಎದುರು ಗಡೆ ಬರುತ್ತಿರುವ ವಾಹನ ಮತ್ತು ಜನರ ಮಧ್ಯೆ, ಭಯ ಉದ್ವೇಗ ಗೊಂದಲಗಳಿಲ್ಲದೆ ದಾರಿ ಮಾಡಿಕೊಂಡು ಮುನ್ನುಗ್ಗುತ್ತಾನೆ. ಎಲ್ಲಿ ಬ್ರೇಕ್ ಹಾಕಬೇಕೋ ಅಲ್ಲಿ ತಾನೇ ತಾನಾಗಿ, ಎಂದರೆ ಗಮನವಿತ್ತು ಯೋಚಿಸದೇ ಸರಿಯಾಗಿ ಬ್ರೇಕ್ ಹಾಕಿಬಿಡುತ್ತಾನೆ. ಒಂದು ಅಭ್ಯಾಸದಿಂದ ಬಹಳಷ್ಟು ಶಕ್ತಿ ಉಳಿತಾಯವಾಗುತ್ತದೆ. ರೂಢಮೂಲವಾದ ಆ ಒಂದು ಅಭ್ಯಾಸದೊಂದಿಗೆ ಇನ್ನೊಂದು ಪುಟ್ಟ ರಚನಾತ್ಮಕ ಅಭ್ಯಾಸವನ್ನು ಪ್ರಾರಂಭಿಸಬಹುದು. ನಿತ್ಯವೂ ಸ್ನಾನ ಮಾಡು ತ್ತಿರುವಾಗ ಗಣಿತದ ಒಂದು ಸೂತ್ರವನ್ನೊ, ಕವಿತೆಯ ತುಣಕನ್ನೊ, ಶ್ಲೋಕವನ್ನೊ ಕಂಠಪಾಠ ಮಾಡಬಹುದು. ಸ್ನಾನವು ಯಾವ ತಡೆ ಇಲ್ಲದೇ ನಡೆದಿರುತ್ತದೆ. ಜೊತೆಗೆ ಉತ್ತಮ ವಿಷಯ ಸಂಗ್ರಹವೂ ಆಗುತ್ತಿರುತ್ತದೆ.

\vskip 2pt

ಅಭ್ಯಾಸಸಿದ್ಧಿಗಾಗಿ ನಿಷ್ಠೆಯಿಂದ ನಿರಂತರ ಸಾಧನೆ ಮಾಡಲು ಪ್ರೇರಕವಾಗುವ ಪ್ರತಿಯೊಂದು ಅವಕಾಶವನ್ನೂ ಎಚ್ಚರಿಕೆಯಿಂದ ಗಮನಿಸಿ ಉಪಯೋಗಿಸಿಕೊಳ್ಳಬೇಕು. ಹಳೆಯ ಜಾಡಿನಲ್ಲಿ ನಿಮ್ಮನ್ನು ತಿರುಗಿ ಎಳೆಯುವಂಥ ಅವಕಾಶಗಳಿಗೆ ವಿರೋಧವಾದ ದೇಶಕಾಲ ಸಂದರ್ಭಗಳನ್ನು ಸೃಷ್ಟಿಸಿಕೊಳ್ಳಬೇಕು. ಒಟ್ಟಿನಲ್ಲಿ ನಿಮ್ಮ ಪ್ರತಿಜ್ಞೆ ಅಥವಾ ಸಂಕಲ್ಪವನ್ನು ಕಾಯ್ದುಕೊಳ್ಳಲು ಪ್ರತಿ ಯೊಂದು ಸಂದರ್ಭವನ್ನೂ ಉಪಯೋಗಿಸಿಕೊಳ್ಳಬೇಕು. ಹೀಗೆ ಮಾಡಿದಾಗ ಮಾತ್ರ ಪ್ರತಿಜ್ಞಾ ಭಂಗ ಮಾಡುವ ಚಂಚಲತೆ, ಚಪಲತೆಗಳಿಂದ ಪಾರಾಗಬಹುದು.

\vskip 2pt

ಹೊಸ ಅಭ್ಯಾಸವು ಮನಸ್ಸು ನರವ್ಯೂಹಗಳಲ್ಲಿ ಆಳವಾಗಿ ಬೇರು ಬಿಡುವವರೆಗೂ ಯೋಜಿಸಿ ಕೊಂಡ ಸಾಧನೆಯನ್ನು ಒಂದೇ ಒಂದು ದಿನದ ಮಟ್ಟಿಗೂ ಬಿಡಬಾರದು. ಒಂದು ದಿನ ಕಾರಣಾಂತರಗಳಿಂದ ಅಭ್ಯಾಸ ತಪ್ಪಿದರೆ, ಮರುದಿನ ಹೇಗಾದರೂ ನಿಯಮಗಳಿಂದ ಜಾರಿ ಕೊಳ್ಳಲು ಮನಸ್ಸು ಕಾರಣ ಹುಡುಕುವುದು.

ವಿಖ್ಯಾತ ಬ್ಯಾಡ್​ಮಿಂಟನ್ ಆಟಗಾರ ಶ‍್ರೀ ಪ್ರಕಾಶ್ ಪಡುಕೋಣೆ ಅವರ ನಿಷ್ಠಾಯುಕ್ತ ಅಭ್ಯಾಸದ ಅದ್ಭುತ ಸಾಹಸವನ್ನು ಅವರು ಗಳಿಸಿದ ಪ್ರಶಸ್ತಿ ಯಶಸ್ಸುಗಳ ಕತೆಯನ್ನು ಕೇಳಿ, ನಗರದ ಹಲವು ಭಾಗಗಳಲ್ಲಿ ಯುವಕರು ಬ್ಯಾಡ್​ಮಿಂಟನ್ ಕ್ಲಬ್ಬನ್ನು ಸ್ಥಾಪಿಸಿದರು. ನಿತ್ಯವೂ ನಿಷ್ಠೆಯಿಂದ ಅಭ್ಯಾಸ ಮಾಡಲು ಅವರು ಸಂಕಲ್ಪಿಸಿದರು. ಉತ್ಸಾಹದಿಂದಲೇ ದಿನವೂ ಕ್ಲಬ್ಬಿಗೆ ಹೋಗುತ್ತಿದ್ದರು. ಏಳೆಂಟು ದಿನಗಳ ಅಭ್ಯಾಸದ ಬಳಿಕ ನಾನಾ ಕಾರಣಗಳಿಂದ ಒಬ್ಬೊಬ್ಬರೇ ಗೈರುಹಾಜರಾಗಲು ತೊಡಗಿದರು. ಒಂದು ತಿಂಗಳೊಳಗೆ ಅವರ ಉತ್ಸಾಹದೊಂದಿಗೆ ಕ್ಲಬ್ಬೂ ಮಾಯವಾಯಿತು! ಶ‍್ರೀ ಪಡುಕೋಣೆ ಎಡೆಬಿಡದೆ ಹದಿನೇಳು ವರ್ಷಗಳ ಕಾಲ ನಿತ್ಯ ನಿರಂತರ ನಿಷ್ಠೆಯಿಂದ ಅಭ್ಯಾಸ ನಡೆಸಿದ್ದರೆ, ಆರಂಭಶೂರರಾದ ಈ ಯುವಕರು ಹದಿನೇಳು ದಿನಗಳ ಕಾಲವಾದರೂ ಅಭ್ಯಾಸವನ್ನು ಮುಂದುವರಿಸಲಾರದಾದರು.

ಬರಿಯ ಸತ್ ಸಂಕಲ್ಪ ಕ್ಷಣಿಕ ಭಾವೋದ್ವೇಗ ಉತ್ಸಾಹಗಳೇ ಕಾರ್ಯ ಸಿದ್ಧಿಗೆ ಸಾಕಾಗವು. ನಿಯಮಪಾಲನೆಯಿಂದ ತಪ್ಪಿಸಿಕೊಳ್ಳಲು ಮನಸ್ಸು ಸದಾ ಒಂದಲ್ಲ ಒಂದು ಉಪಾಯವನ್ನು ಹುಡುಕುವುದು. ನಿಯಮಕ್ಕೆ ಬಗ್ಗುವಂತೆ, ಒಗ್ಗುವಂತೆ ದಿನವೂ ತಪ್ಪದೇ ತಾಳ್ಮೆಯಿಂದ ಯತ್ನಿಸ ಬೇಕಲ್ಲವೆ?


\section*{ಆತ್ಮವಿಮರ್ಶೆಯ ಅಭ್ಯಾಸ}

\addsectiontoTOC{ಆತ್ಮವಿಮರ್ಶೆಯ ಅಭ್ಯಾಸ}

ನಮ್ಮ ವ್ಯಕ್ತಿತ್ವವನ್ನು ರೂಪಿಸುವ ಹಂಬಲ ನಮಗಿದ್ದರೆ, ನಮ್ಮ ಏಳ್ಗೆಗೆ ನಾವೇ ಶಿಲ್ಪಿಗಳಾಗ ಬೇಕೆಂದಿದ್ದರೆ, ಸಿಂಹಾವಲೋಕನದ ಅಭ್ಯಾಸವನ್ನು ರೂಢಿಸಿಕೊಳ್ಳಬೇಕೆಂಬುದು ಅನುಭವಿಗಳ ಆದೇಶ. ಅರಣ್ಯದಲ್ಲಿ ಸಂಚರಿಸುವ ಸಿಂಹ ಒಮ್ಮೊಮ್ಮೆ ತನ್ನ ಮುಖ ತಿರುಗಿಸಿ ನಡೆದು ಬಂದ ದಾರಿಯನ್ನು ಅವಲೋಕಿಸುತ್ತದೆ ಎನ್ನುತ್ತಾರೆ. ಇದನ್ನು ಸಿಂಹಾವಲೋಕನ ಎನ್ನುತ್ತಾರಷ್ಟೆ. ದಿನದ ಚಟುವಟಿಕೆಗಳನ್ನೂ ಸಿಂಹಾವಲೋಕನ ಕ್ರಮದಿಂದ ಪರಿಶೀಲಿಸಬೇಕು. ‘ಮನವ ಶೋಧಿಸಬೇಕು ನಿಚ್ಚ, ಅನುದಿನ ಮಾಡುವ ಪಾಪಪುಣ್ಯದ ವೆಚ್ಚ’ ಎನ್ನುವ ದಾಸವಾಣಿಯಲ್ಲೂ ಆ ಅಭ್ಯಾಸದ ಆವಶ್ಯಕತೆ ಸೂಚಿತವಾಗಿದೆ. ದಿನದ ಕೊನೆಯ ಭಾಗದಲ್ಲಿ ನಿದ್ರಿಸುವ ಮೊದಲು ‘ನನ್ನ ಪಾಲಿಗೆ ಬಂದ ಆಯುಸ್ಸಿನಲ್ಲಿ ಈ ಒಂದು ದಿನವನ್ನು ಹೇಗೆ ಕಳೆದೆ? ನನಗೆ ಸಿಕ್ಕಿದ ಅವಕಾಶಗಳನ್ನು ಹೇಗೆ ಉಪಯೋಗಿಸಿಕೊಂಡೆ? ಎಲ್ಲಿ ತಪ್ಪುಗಳಾದವು? ಕಾರಣಗಳೇನು? ನಾಳಿನ ಅಂಥ ಸಂದರ್ಭಗಳಲ್ಲಿ ಹೇಗೆ ನಡೆದುಕೊಳ್ಳಬಹುದು? ಈ ಹೊತ್ತು ಮಾಡಬೇಕಾಗಿದ್ದ ಕೆಲಸಗಳನ್ನು ಹೊರಗಿನ ಒತ್ತಾಯ ಒತ್ತಡಗಳಿಂದ ಗೊಣಗುತ್ತ ಮಾಡಿದೆನೆ? ಸ್ವಂತ ಯೋಚನೆ ಯೋಜನೆಗಳನ್ನನು\-ಸರಿಸುತ್ತ ಉತ್ಸಾಹದಿಂದ ಮಾಡಿದೆನೆ? ಇನ್ನೂ ಚೆನ್ನಾಗಿ ಮಾಡಬಹುದಿತ್ತೇ? ಸ್ನೇಹಿತರೊಡನೆ, ಬಂಧುಬಾಂಧವರೊಡನೆ ಮತ್ತು ನೆರೆಹೊರೆಯವರೊಂದಿಗೆ ಹೇಗೆ ವರ್ತಿಸಿದೆ?’–ಇವೇ ಮೊದಲಾದ ಪ್ರಶ್ನೆಗಳನ್ನು ಕೇಳುತ್ತ ಪರದೆಯ ಮೇಲಣ ಚಿತ್ರವನ್ನು ವೀಕ್ಷಿಸುವಂತೆ ನಿರ್ಲಿಪ್ತ\-ರಾಗಿ, ನಡೆದ ಘಟನೆಗಳನ್ನು ನೋಡಿ ತಿದ್ದುಪಡಿಗಳನ್ನು ಮಾಡಿಕೊಳ್ಳಬಹುದು. ಕನ್ನಡಿಯಲ್ಲಿ ನಮ್ಮ ಮುಖವನ್ನು ಕಂಡು ಸರಿಪಡಿಸಿಕೊಂಡಂತೆ ಈ ಆತ್ಮವಿಮರ್ಶೆಯ ವಿಧಾನದಿಂದ ನಮಗೂ, ಇತರರಿಗೂ ಸಹ್ಯವಲ್ಲದ ಗುಣಗಳನ್ನು ಬಿಟ್ಟುಬಿಡಲು ಸಾಧ್ಯ, ಉತ್ತಮ ಗುಣಗಳನ್ನು ಆರ್ಜಿಸಲು ಸಾಧ್ಯ ಎಂಬುದನ್ನು ಬೆಂಜಮಿನ್ ಫ್ರಾಂಕ್ಲಿನ್ ಸ್ವಾನುಭವದಿಂದ ಸಾರಿದ್ದಾನೆ.

ಅಭ್ಯಾಸ ನಿರ್ಮಾಣದಲ್ಲಿ ಮೊದಲು ಜಗಲಿ ಹಾರಿ ಆಮೇಲೆ ಗಗನ ಹಾರಲು ಯತ್ನಿಸಬೇಕು. ಪ್ರಥಮದಲ್ಲೇ ಹೊರಲಾರದ ಹೊರೆಯನ್ನು ಹೊತ್ತವನ ಪಾಡಾದರೆ ಮುಂದಿನ ಪ್ರಯತ್ನದಲ್ಲಿ ಉತ್ಸಾಹಹೀನನಾಗುವ ಸಂಭವವಿದೆ. ಪ್ರಥಮ ಯತ್ನದಲ್ಲಿ ಉಂಟಾದ ಯಶಸ್ಸು ಮುಂದಿನ ಪ್ರಯತ್ನದ ಪ್ರೇರಕಶಕ್ತಿಯಾಗುತ್ತದೆ. ಆದುದರಿಂದ ಮೊದಮೊದಲು ಸುಲಭವಾದ, ಅಲ್ಪ ಸಮಯ ಬೇಕಾಗುವ ಸ್ವಲ್ಪ ಶ್ರಮದಿಂದ ಸಾಧ್ಯವಾಗುವ ಅಭ್ಯಾಸಗಳನ್ನು ನಿಷ್ಠೆಯಿಂದ ಸಾಧಿಸಬೇಕು.

‘ಯಾರ ಬದುಕಿನಲ್ಲಿ ಯಾವ ನಿರ್ದಿಷ್ಟ ಅಭ್ಯಾಸಗಳೂ ಇಲ್ಲದೇ ಎಲ್ಲವೂ ಅನಿಶ್ಚಿತ ಹಾಗೂ ಅಸ್ತವ್ಯಸ್ತವೊ, ಬೆಳಿಗ್ಗೆ ಹಾಸಿಗೆ ಬಿಟ್ಟೇಳುವುದು, ರಾತ್ರಿ ವಿಶ್ರಾಂತಿ ಪಡೆಯುವುದು, ಆಹಾರ ಸೇವನೆ, ಒಂದು ಸಿಗರೇಟು ಹೊತ್ತಿಸುವುದು, ಒಂದು ಲೋಟ ನೀರು ಕುಡಿಯುವುದೇ ಮೊದಲಾದ ಅತಿ ಸಣ್ಣ ಪುಟ್ಟ ಕೆಲಸಗಳನ್ನು ಪ್ರಜ್ಞಾಪೂರ್ವಕವಾಗಿ ಚಿಂತನ ಮಂಥನ ನಡೆಯಿಸಿ ಮಾಡ\-ಬೇಕಾ\-ಗುವುದೊ, ಅವನಂಥ ದುಃಖಿ ಮತ್ತೊಬ್ಬನಿಲ್ಲ. ಅವನ ಅರ್ಧಸಮಯ ಏನು ಮಾಡ ಬಹುದೆಂದು ನಿರ್ಣಯಿಸುವುದರಲ್ಲಿ, ಇಲ್ಲವಾದರೆ ಸುಮ್ಮನೆ ಕಾಲಹರಣವಾಯಿತೆಂದು ಪರಿತಪಿ ಸುವುದರಲ್ಲಿ ವ್ಯಯವಾಗುತ್ತದೆ. ಅವನ ಜಾಗ್ರತ ಪ್ರಜ್ಞೆಯಲ್ಲಿ ಮಾಡಬೇಕಾದ ಅಗತ್ಯ ಕೆಲಸಗಳ ಬಗೆಗೆ ಮಾಡಬೇಕೆ, ಬೇಡವೇ? ಎಂಬಂಥ ಸಂದೇಹಗಳೆ ಬರಬಾರದು. ಅವನಿಗರಿವಿಲ್ಲದೆಯೇ ಹುಟ್ಟು ಗುಣಗಳ ಪ್ರಕ್ರಿಯೆಗಳಂತೆ ದಿನದ ಕರ್ತವ್ಯಗಳನ್ನು ಸಹಜವಾಗಿ, ಒದ್ದಾಟವಿಲ್ಲದೆ ಮಾಡುವಂತಾಗ ಬೇಕು. ಹೀಗೆ ದಿನದ ಕರ್ತವ್ಯ ನಿರ್ವಹಣೆ ಸಹಜಗುಣವಾಗದವರು ಇಂದಿಂದೇ ಅದನ್ನು ಸರಿಪಡಿಸಿಕೊಳ್ಳಲು ಯತ್ನಿಸಲಿ’ ಎಂದು ಮನೋವಿಜ್ಞಾನಿ ಕರೆಕೊಡುತ್ತಾನೆ.


\section*{ಪ್ರಯತ್ನವೇ ಪರಮ ಪೂಜೆ}

\addsectiontoTOC{ಪ್ರಯತ್ನವೇ ಪರಮ ಪೂಜೆ}

ನಮ್ಮ ದೇಶದ ಸುಪ್ರಸಿದ್ಧ ತಾತ್ತ್ವಿಕ ಗ್ರಂಥವಾದ ‘ಯೋಗವಾಸಿಷ್ಠ’ದಲ್ಲಿ ಸ್ವಪ್ರಯತ್ನವನ್ನು ಕುರಿತು ಹೇಳಿದ ಮಾತುಗಳನ್ನು ಕೇಳಿದ್ದೀರಾ?

‘ಸರಿಯಾದ ಮತ್ತು ಉತ್ಸಾಹಪೂರಿತ ಪ್ರಯತ್ನದಿಂದ ಪಡೆಯಲು ಸಾಧ್ಯವಾಗದ ವಸ್ತು ಯಾವುದೂ ಈ ಪ್ರಪಂಚದಲ್ಲಿಲ್ಲ. ಯಾರಾದರೂ ಯಾವುದೇ ವಸ್ತುವನ್ನು ಪಡೆಯಲು ಆಶಿಸಿ, ಅದನ್ನು ಹೊಂದಲು ಯತ್ನಿಸಿದುದೇ ಆದರೆ ಅವರು ಖಂಡಿತವಾಗಿಯೂ ಸಫಲರಾಗಬಲ್ಲರು. ಆದರೆ ಹಿಡಿದ ದಾರಿಯಲ್ಲಿ ಹಿಂದೇಟು ಹಾಕದೆ ಮುಂದುವರಿಯುತ್ತಲೇ ಸಾಗಬೇಕು.

\newpage

‘ಈ ಪ್ರಪಂಚದಲ್ಲಿ ಹಲವು ಮಂದಿ ಅತ್ಯಂತ ನಿಮ್ನ ಪರಿಸ್ಥಿತಿಗಳಿಂದ, ದುಃಖ ದಾರಿದ್ರ್ಯ\-ಗಳಿಂದ ಪಾರಾಗಿ ಪರಮಭಾಗ್ಯಶಾಲಿಗಳಾದರು. ಬುದ್ಧಿಶಾಲಿಗಳಾದವರು ಪ್ರಯತ್ನ ಮಾತ್ರದಿಂದಲೆ ಅತ್ಯಂತ ಅಪಾಯಕರ ಸನ್ನಿವೇಶಗಳನ್ನು ದಾಟಿ ಜಯಶಾಲಿಗಳಾದರಲ್ಲದೆ ಅದೃಷ್ಟದಲ್ಲಿ ಅರ್ಥಹೀನ ನಂಬಿಕೆಯನ್ನಿಟ್ಟಲ್ಲ. ಯಾರು ಯಾವುದಕ್ಕಾಗಿ ಹಾತೊರೆದಿದ್ದನೋ, ಯಾವುದನ್ನು ಪಡೆಯುವುದಕ್ಕಾಗಿ ಹೋರಾಡಿದ್ದನೋ ಅದನ್ನು ಪಡೆದೇ ಪಡೆಯುತ್ತಾನೆ–ಸೋಮಾರಿಯಾಗಿ ಕುಳಿತು ಯಾರು ಏನನ್ನೂ ಸಾಧಿಸಿಲ್ಲ. ಪ್ರತಿಯೊಬ್ಬನೂ ತನಗೆ ತಾನೇ ಶತ್ರು ಅಥವಾ ತನಗೆ ತಾನೇ ಮಿತ್ರ ಎಂಬುದನ್ನು ತಿಳಿಯಬೇಕು. ಯಾರು ತನ್ನನ್ನು ತಾನು ರಕ್ಷಿಸಿಕೊಳ್ಳಲಾರನೋ ಅವನನ್ನು ರಕ್ಷಿಸಲು ಯಾರೂ ಇರಲಾರರು. ಖಂಡಿತವಾಗಿಯೂ ಒಬ್ಬನು ತನ್ನ ಸ್ವಪ್ರಯತ್ನದಿಂದಲೇ ದುಃಖದಾಯಕ ಸನ್ನಿವೇಶಗಳಿಂದ ಪಾರಾಗಬಲ್ಲನು. ಆದುದರಿಂದ ಪ್ರತಿಯೊಬ್ಬನೂ ಸರಿಯಾದ ಮಾರ್ಗದಲ್ಲಿ ನಿಷ್ಠೆಯಿಂದ ಮುಂದುವರಿಯಬೇಕು. ಧೈರ್ಯಶಾಲಿಗಳು, ಸಾಹಸಿಗಳು, ಜ್ಞಾನಿಗಳು ಆದ ಮಹಾತ್ಮರಾರೂ ವಿಧಿಯನ್ನು ಕಾದು ಕುಳಿತಿರಲಿಲ್ಲ.

‘ಅದೃಷ್ಟದ ಕಲ್ಪನೆ ಅನಗತ್ಯವಾದುದು–ಕಾರಣ ನಾವು ಎತ್ತ ನೋಡಿದರೂ ಪ್ರಯತ್ನಶೀಲತೆ ಹಾಗೂ ಪರಿಶ್ರಮ ಫಲದಾಯಕವಾದುದನ್ನು ಕಾಣುತ್ತೇವೆಯೇ ಹೊರತು ಕರ್ಮಹೀನತೆಯಲ್ಲ. ಕರ್ಮಶೀಲತೆ ಸಂಪೂರ್ಣವಾಗಿ ಮಾಯವಾದ ಶವದಲ್ಲಿ ಅದೃಷ್ಟದಿಂದ ಏನಾದರೂ ಸಂಭವಿಸಿ ದುದನ್ನು ನಾವು ಎಂದೂ ಕೇಳಿಲ್ಲ. ಅದೃಷ್ಟ ಏನನ್ನೂ ಮಾಡುವುದಿಲ್ಲ. ನಮ್ಮ ಕಲ್ಪನೆಯಲ್ಲಿ ಮಾತ್ರ ಅದೃಷ್ಟಕ್ಕೆ ಅಸ್ತಿತ್ವ–ಅದೃಷ್ಟಕ್ಕೆ ತನ್ನದೇ ಆದ ಅಸ್ತಿತ್ವವಿಲ್ಲ.

‘ಅದೃಷ್ಟವೆಂದರೆ ನಾವು ಆಗಲೇ ಮಾಡಿದ ಕಾರ್ಯಗಳ ಒಳ್ಳೆಯ ಅಥವಾ ಕೆಟ್ಟ ಫಲಗಳಲ್ಲದೇ ಬೇರೇನೂ ಅಲ್ಲ. ಒಬ್ಬನ ಸ್ವಪ್ರಯತ್ನದ ಫಲವಾಗಿ ಯಾವುದು ಸಂಭವನೀಯವೋ ಅದನ್ನು ನಾವು ಅದೃಷ್ಟವೆನ್ನುತ್ತೇವೆ. ಒಬ್ಬ ವ್ಯಕ್ತಿಯ ಪ್ರಯತ್ನವನ್ನೇ ಹೊಂದಿಕೊಂಡಿದೆ ಆತನ ಯಶಸ್ಸು.’


\section*{ವಿಧಿಯೋ, ಪೌರುಷವೋ}

\addsectiontoTOC{ವಿಧಿಯೋ,\break ಪೌರುಷವೋ}

ನಮ್ಮ ದೇಶದಲ್ಲಿ ‘ತೇನ ವಿನಾ ತೃಣಮಪಿ ನ ಚಲತಿ’ ಎನ್ನುತ್ತ ಒಂದು ಹುಲ್ಲು ಕಡ್ಡಿಯೂ ಭಗವಂತನ ಕೃಪೆಯಿಲ್ಲದೆ ಅಲುಗಾಡದು ಎನ್ನುವ ವಿಧಿವಾದಿಗಳಿದ್ದಾರೆ. ಸ್ವಪ್ರಯತ್ನದಿಂದ ಸರ್ವವೂ ಸಾಧ್ಯ ಎನ್ನುವ ಪೌರುಷವಾದಿಗಳಿದ್ದಾರೆ. ಮನುಷ್ಯನು ಪರಿಸ್ಥಿತಿಯ ದಾಸ ಎನ್ನುವುದು ಕೆಲವರ ಮತ. ತನ್ನ ವ್ಯಕ್ತಿತ್ವದ ಪ್ರಭಾವದಿಂದ ಪರಿಸ್ಥಿತಿಯನ್ನೇ ಬದಲಾಯಿಸಬಲ್ಲ ಸಾಮರ್ಥ್ಯ ಮಾನವನಿಗೆ ಇದೆ ಎಂಬುದು ಇತರರ ವಾದ. ಸ್ವಪ್ರಯತ್ನ ಮತ್ತು ವಿಧಿ ರಥದ ಎರಡು ಚಕ್ರ ಗಳಿದ್ದಂತೆ, ಕತ್ತರಿಯ ಎರಡು ಅಲಗುಗಳಂತೆ ಎಂದು ಹೇಳುವುದುಂಟು. ‘ಹಣೆಯ ಬರಹ’, ‘ಗ್ರಹಗತಿ’, ‘ಕರ್ಮಫಲ’, ಇವು ನಾವು ಯಾವುದೇ ಕಾರ್ಯದಲ್ಲಿ ಸೋತಾಗ ಹೇಳುವ ಮಾತುಗಳು. ತಾತ್ತ್ವಿಕ ದೃಷ್ಟಿಯಿಂದ ಅವುಗಳಿಗೆ ಏನೇ ಅರ್ಥವಿರಲಿ, ನಮ್ಮ ಬದುಕನ್ನು ಉನ್ನತವಾಗಿ ಸಲು ನಾವು ಹೋರಾಡಲೇಬೇಕು. ಸಾಹಸದಿಂದ ಮುನ್ನುಗ್ಗಲೇಬೇಕು. ಸ್ವಪ್ರಯತ್ನದಲ್ಲಿ ವಿಶ್ವಾಸ ವಿಟ್ಟು ಶ್ರದ್ಧೆಯಿಂದ ದುಡಿಯಲೇಬೇಕು. ‘ಆರರವರೆಗೆ ಮನುಷ್ಯನ ಪ್ರಯತ್ನ, ಏಳನೆಯದು ದೈವಕೃಪೆ’ ಎನ್ನುವ ಮಾತಿದೆ. ಸಾಧನೆಯಿಲ್ಲದೆ, ಉತ್ಸಾಹವಿಲ್ಲದೆ, ನಿರಂತರ ಪರಿಶ್ರಮವಿಲ್ಲದೆ ಯಾರಾದರೂ ಪ್ರಪಂಚದಲ್ಲಿ ಉನ್ನತಿ ಹೊಂದಿದ್ದಾರೆಯೇ?

ವಿಧಿ ಅಥವಾ ಅದೃಷ್ಟವನ್ನೇ ಕಾಯುತ್ತ ಸೋಮಾರಿಯಾಗಿ ಕಾಲ ಕಳೆದರೆ ಯಾವುದನ್ನಾದರೂ ಯಶಸ್ವಿಯಾಗಿ ನೆರವೇರಿಸಲು ಆಗುವುದೇ? ಯಾವ ಕೆಲಸವೂ ಸಾಧ್ಯವಿಲ್ಲ. ಅಂಥ ಮನೋ ಭಾವನೆ ಮನುಷ್ಯನನ್ನು ಹೇಡಿಯನ್ನಾಗಿ ಮಾಡುವುದು, ತಾಮಸ ಪ್ರವೃತ್ತಿಯ ದಾಸನನ್ನಾಗಿ ಮಾಡುವುದು. ಜನರು ತಮ್ಮ ದುರ್ಬಲತೆಯಿಂದ ಅನೇಕ ತಪ್ಪುಗಳನ್ನೆಸಗುವರು. ಆದರೆ ದೋಷವನ್ನು ವಿಧಿಯ ಮೇಲೆ, ಅದೃಷ್ಟದ ಮೇಲೆ, ಗ್ರಹ ನಕ್ಷತ್ರಗಳ ಮೇಲೆ ಆರೋಪಿಸುತ್ತಾರೆ. ಅಂಥವನೊಬ್ಬ ಅಜಾಗರೂಕತೆಯಿಂದ ಕಾಲುಜಾರಿ ಬಿದ್ದು ನೆಲವನ್ನು ದೂರಿದನಂತೆ, ವಿಧಿಯನ್ನು ಹಳಿದನಂತೆ!

ಗುರಿಯನ್ನು ಸೇರುವುದಕ್ಕೆ ತಡೆಗಳನ್ನೊಡ್ಡುವ ವಿಧಿ ಎಂಬುದು ಇದ್ದರೆ ಅದನ್ನು ನಮ್ಮ ಸ್ವಂತ ಪರಿಶ್ರಮದ ಮೂಲಕ, ಪ್ರತಿಯೊಬ್ಬನಲ್ಲಿಯೂ ಹುದುಗಿರುವ ಅಂತಶ್ಶಕ್ತಿಯನ್ನು ಜಾಗೃತಗೊಳಿ ಸುವುದರಿಂದ ಜಯಿಸಬಹುದು. ವಿಧಿಯೇ ಸರ್ವಶಕ್ತಿಯುಳ್ಳದ್ದಾಗಿದ್ದರೆ ನ್ಯಾಯಾನ್ಯಾಯಗಳು, ಸುಗುಣ ದುರ್ಗುಣಗಳು, ಆತ್ಮಶಕ್ತಿ–ಇವುಗಳ ಮಾತೇ ಇರುತ್ತಿರಲಿಲ್ಲ. ಮನುಷ್ಯರು ಕಲ್ಲು ಕೊರಡುಗಳಲ್ಲ. ‘ನಾನು ಆ ರೀತಿ ಮಾಡುತ್ತಿರುವುದಕ್ಕೆ ವಿಧಿಯೇ ಕಾರಣ, ನನ್ನ ಕೆಲಸಗಳಿಗೆ ನಾನು ಜವಾಬ್ದಾರನಲ್ಲ, ನಾನು ಅಸಹಾಯಕನಾಗಿ ಅದೃಷ್ಟದಿಂದ ಮುಂದೂಡಲ್ಪಡುತ್ತಿದ್ದೇನೆ’ ಎನ್ನುವ ಭಾವನೆಯಿಂದ ಯಾವ ಮನುಷ್ಯನಿಗೂ ಏಳಿಗೆ ಸಾಧ್ಯವಿಲ್ಲ. ತಾನು ದುರ್ಬಲನೆಂದೂ, ಅಗೋ ಚರವಾದ ಅದೃಷ್ಟದ ಕರುಣೆಯಲ್ಲಿ ತಾನಿರುವೆನೆಂದೂ ತಿಳಿಯುವುದು ಮನುಷ್ಯನ ಅಜ್ಞಾನ ಮತ್ತು ಅಧೋಗತಿಯಲ್ಲದೆ ಬೇರಲ್ಲ.

ಯಾರು ದೇವರಲ್ಲಿ ಶರಣಾಗಿ ಪೂರ್ಣ ಮನಸ್ಸಿನಿಂದ ಬೇಡಿ ಬಾಳಲು ಶಕ್ತನೋ, ಅಂಥವನು ಮಾತ್ರ ‘ನಾನು ಯಂತ್ರ, ದೇವರು ಈ ಯಂತ್ರದ ಚಾಲಕ’ ಎಂದು ಹೇಳಬಲ್ಲ. ಅವನು ಕನಸು ಮನಸಿನಲ್ಲಿಯೂ ತಪ್ಪು ಹೆಜ್ಜೆ ಇಡಲಾರ. ಯಾರಿಗೂ ಕೆಡಕು ಮಾಡಲಾರ. ಸದಾ ಸರ್ವರ ಶುಭಾಕಾಂಕ್ಷಿ ಆತ. ನಿಸ್ವಾರ್ಥತೆಯ ತವರಾಗಿ, ಪರರ ಒಳಿತಿಗಾಗಿ ಪ್ರಾರ್ಥಿಸುವ ಮಹಿಮ ಆತ.


\section*{ಅರ್ಹತೆಗೆ ಒಲಿದ ಅದೃಷ್ಟ}

\addsectiontoTOC{ಅರ್ಹತೆಗೆ ಒಲಿದ ಅದೃಷ್ಟ}

ಇತ್ತೀಚೆಗೆ ಯುವಕನೊಬ್ಬ ಪ್ರಸಿದ್ಧ ಕ್ರೀಡಾಪತ್ರಿಕೆ ಏರ್ಪಡಿಸಿದ ಒಂದು ಸ್ಪರ್ಧೆಯಲ್ಲಿ ಭಾಗವಹಿಸಿ ಹತ್ತು ಸಾವಿರ ರೂಪಾಯಿಗಳ ಬಹುಮಾನ ಪಡೆದ. ಎಲ್ಲರೂ ಅವನ ಅದೃಷ್ಟವನ್ನು ಕಂಡು ಚಕಿತ\-ರಾದರು. ಸಂತೋಷವನ್ನು ವ್ಯಕ್ತಪಡಿಸಿ ಧನ್ಯವಾದಗಳನ್ನು ಅರ್ಪಿಸಿದರು. ಆತ ನಮ್ಮ ಶಾಲೆಯ ಹಳೆ ವಿದ್ಯಾರ್ಥಿ. ‘ಅಂಥ ಒಂದು ಬಹುಮಾನ ಸಿಕ್ಕೀತೆಂದು ಕನಸಿನಲ್ಲೂ ಯೋಚಿಸಿರಲಿಲ್ಲ’ ಎಂದು ಅವನು ನನಗೆ ಹೇಳಿದ. ‘ಅದೃಷ್ಟವಲ್ಲದೆ ಮತ್ತೇನು?’ ಎಂದವನ ಮಿತ್ರರು ದನಿ ಗೂಡಿಸಿದರು. ಅವನೂ ನಗುತ್ತ ಹೌದೆಂಬಂತೆ ತಲೆತೂಗಿದ. ಆದರೆ ಅದು ಲಾಟರಿಯ ಸ್ಪರ್ಧೆಯಾಗಿರಲಿಲ್ಲ. ಮಿತ್ರರೆಲ್ಲ ದೂರ ಸರಿದ ಮೇಲೆ ಆತನೊಬ್ಬನೇ ಇದ್ದಾಗ ನಾನೊಂದು ಪ್ರಶ್ನೆ ಕೇಳಿದೆ. ‘ನೀನು ಎಷ್ಟು ವರ್ಷಗಳಿಂದ ಆ ಕ್ರೀಡಾ ಪತ್ರಿಕೆಯನ್ನು ಓದುತ್ತಿದ್ದಿ?’ ‘ಆರನೇ ತರಗತಿಯಿಂದ’ ಎಂದನಾತ. ಈಗ ಅವನು ಪಿ.ಯು.ಸಿ. ಮುಗಿಸಿದ್ದಾನೆ. ಈ ಆರು ವರ್ಷ ಕಾಲ ಪ್ರಕಟವಾದ ಪತ್ರಿಕೆಯ ಪ್ರತಿಯೊಂದು ಸಂಚಿಕೆಯನ್ನೂ ತಪ್ಪದೇ ಓದಿದ್ದ. ಆಟೋಟದ ಸ್ಪರ್ಧೆಗಳಲ್ಲಿ ತತ್ಸಂಬಂಧವಾದ ವಾರ್ತೆಗಳಲ್ಲಿ ಆತನಿಗೆ ಮೊದಲಿನಿಂದಲೂ ಆಸಕ್ತಿ. ಮಕ್ಕಳು ಸಿಹಿ ತಿಂಡಿಗಾಗಿ ಹಾತೊರೆಯುವಂತೆ ಆ ಪತ್ರಿಕೆಗಾಗಿ ಅವನ ಹಂಬಲವಿತ್ತು. ಅದು ಕೈಗೆ ಬಂದೊಡನೆ ಮೊದಲಿನಿಂದ ಕೊನೆಯವರೆಗೆ ಹಲವು ಬಾರಿ ಓದಿದವನು ಆತ. ಈ ಆರು ವರ್ಷಗಳಲ್ಲಿ ಪಂದ್ಯಾಟಕ್ಕೆ ಸಂಬಂಧಿಸಿದ ಎಷ್ಟೊಂದು ವಿಚಾರಗಳನ್ನು ಆತ ಸಂಗ್ರಹಿಸಿರಬಹುದು ಎಂಬುದನ್ನು ಯೋಚಿಸಿ ನೋಡಿ! ಆ ಜ್ಞಾನದ ಹಿನ್ನೆಲೆಯಿಂದ ಲಕ್ಷಾಂತರ ಜನರು ಭಾಗವಹಿಸಿದ ಸ್ಪರ್ಧೆಯಲ್ಲಿ ಅತ್ಯುನ್ನತ ಸ್ಥಾನವನ್ನು ಪಡೆಯುವ ಸಾಮರ್ಥ್ಯವನ್ನು ಆತ ಗಳಿಸಿದ. ಬಹುಮಾನದ ಬಗೆಗೆ ಅವನು ಎಂದೂ ಯೋಚಿಸಿರಲಿಲ್ಲ. ಅರ್ಹತೆಯನ್ನು ಹುಡುಕಿಕೊಂಡು ಅದೇ ಅವನ ಬಳಿ ಬಂದಿತಲ್ಲವೆ?


\section*{ಸಹನೆಯಿಂದ ಸಿದ್ಧಿ}

\addsectiontoTOC{ಸಹನೆಯಿಂದ ಸಿದ್ಧಿ}

ಪ್ರಭುಶಕ್ತಿಗಿಂತಲೂ, ವಾಕ್​ಶಕ್ತಿ ಪ್ರಬಲ ಪರಿಣಾಮಕಾರಿ ಎಂಬುದನ್ನು ತೋರಿಸಿಕೊಟ್ಟ\break ಡಿಮೋಸ್ತನೀಸ್ ಕ್ರಿ.\ ಪೂ.\ ೩೮೪ರಲ್ಲಿ ಗ್ರೀಸ್ ದೇಶದಲ್ಲಿ ಜನಿಸಿದಾಗ ಜ್ಯೋತಿಷ್ಕರು ಅವನು ಸಾಮಾನ್ಯ ಮನುಷ್ಯನಾಗಿ ಬಾಳುವನೆಂದು ಭವಿಷ್ಯ ನುಡಿದಿದ್ದರು. ಉಗ್ಗುದನಿಯ ಕುಗ್ಗುನುಡಿಯ ಹುಡುಗನಾದ ಆತ ಅಶಕ್ತ ರೋಗಿಯಾಗಿ ಬೆಳೆದು, ತನ್ನ ಸಹಪಾಠಿ ಸ್ನೇಹಿತರ ಪರಿಹಾಸ್ಯ ಹಾಗೂ ಕನಿಕರಕ್ಕೆ ಪಾತ್ರನಾಗಿ, ಬಾಲ್ಯದಲ್ಲೇ ತಂದೆಯನ್ನು ಕಳೆದುಕೊಂಡು ತಬ್ಬಲಿಯಾದ. ಆತನ ಚಿಕ್ಕಪ್ಪ ಈ ಬೆಪ್ಪುತಕ್ಕಡಿಯನ್ನು ಹೇಗೊ ಸೋಲಿಸಬಹುದೆಂದು ಅವನ ಆಸ್ತಿಯನ್ನು ಅಪಹರಿಸಿಬಿಟ್ಟ.\break ಡಿಮೋಸ್ತನೀಸ್ ತನಗಾದ ಅನ್ಯಾಯವನ್ನು ನ್ಯಾಯಾಲಯದಲ್ಲಿ ದೂರಿಕೊಂಡ. ಆದರೆ ಅವನ ಮಾತಿಗೆ ಬೆಲೆ ಸಿಗಲಿಲ್ಲ. ಅದೇ ವೇಳೆಗೆ, ಆ ಕಾಲದ ಪ್ರಸಿದ್ಧ ಗ್ರೀಕ್ ವಾಗ್ಮಿಯೊಬ್ಬನ ಮಾತುಗಾರಿಕೆಯನ್ನು ಕೇಳಿದ ಜನ ಮೋಹಿತರಾಗಿ ಆತನಿಗೆ ತೋರಿಸಿದ ಗೌರವಾದರಗಳು ಡಿಮೋಸ್ತನೀಸ್​ನ ಮನಸ್ಸಿನಲ್ಲಿ ಅಚ್ಚಳಿಯದೆ ಉಳಿಯಿತು. ತನ್ನೆಲ್ಲ ಕಷ್ಟನಷ್ಟಗಳ ಮಧ್ಯದಲ್ಲೇ ಮಹತ್ತಾದುದನ್ನು ಸಾಧಿಸುವ ಆಕಾಂಕ್ಷೆಯನ್ನದು ಮೊಳೆಯಿಸಿತು.

ವಾಗ್ಮಿಯಾಗಲು ಡಿಮೋಸ್ತನೀಸ್ ದೃಢನಿಶ್ಚಯ ಮಾಡಿದ್ದ. ಆದರೆ ಅವನು ಹಲವು ತೊಡಕುಗಳನ್ನು ಎದುರಿಸಬೇಕಿತ್ತು. ಮಾತನಾಡಲು ಪ್ರಾರಂಭಿಸಿದೊಡನೆ ಉಗ್ಗು ಅವನನ್ನು ಬಾಧಿಸುತ್ತಿತ್ತು. ದೀರ್ಘವಾದ ವಾಕ್ಯವನ್ನು ಒಮ್ಮಲೇ ಹೇಳುವ ಸಾಮರ್ಥ್ಯ ಅವನಿಗಿರಲಿಲ್ಲ. ನಿರಂತರ ಪ್ರಯತ್ನದಿಂದ ಅವನು ಈ ದೌರ್ಬಲ್ಯಗಳನ್ನು ಗೆದ್ದುಬಿಟ್ಟ. ಒಬ್ಬ ವೈದ್ಯನ ಹೇಳಿಕೆಯಂತೆ ನಾಲಗೆಯ ಮೇಲೆ ಬೆಣಚು ಕಲ್ಲುಗಳನ್ನಿಟ್ಟು ಸ್ಪಷ್ಟವಾಗಿ ಗಟ್ಟಿಯಾಗಿ ಶಬ್ದಗಳನ್ನು ಉಚ್ಚರಿಸ ತೊಡಗಿದ.\break ಎತ್ತರವಾದ ಬೆಟ್ಟಗಳನ್ನು ಏರಿ ಇಳಿಯುತ್ತ ದೀರ್ಘವಾಗಿ ಉಸಿರೆಳೆಯುವ ಅಭ್ಯಾಸವನ್ನು ಮಾಡಿ, ಅತ್ಯಂತ ಉದ್ದವಾದ ವಾಕ್ಯಗಳನ್ನೂ ತಡೆಯಿಲ್ಲದೇ ಉಚ್ಚರಿಸತೊಡಗಿದ. ದಿನವೂ ಸಮುದ್ರದ ಬಳಿ ನಿಂತು ಸಮುದ್ರ ಘೋಷವನ್ನು ಮೀರಿಸುವ ದನಿಯಿಂದ ಮಾತುಗಾರಿಕೆಯನ್ನು ಅಭ್ಯಸಿಸಿದ. ಕಾಯಿದೆ ಕಾನೂನು ಗ್ರಂಥಗಳನ್ನೂ, ಗ್ರೀಕ್ ಮಹಾಕಾವ್ಯಗಳನ್ನೂ ದಿನದಲ್ಲಿ ಹದಿನಾರು ಗಂಟೆ\-ಗಳಿಗೂ ಮಿಕ್ಕಿ ಅಧ್ಯಯನ ನಡೆಯಿಸಿದ. ತನ್ನ ಅಧ್ಯಯನ ಕಾಲದಲ್ಲಿ ಆತ ಜನರೊಂದಿಗೆ ಬೆರೆಯುತ್ತಿರಲಿಲ್ಲ. ಜನರೊಡನೆ ಯಾವತ್ತೂ ವ್ಯವಹಾರದಿಂದ ತನ್ನ ಅಭ್ಯಾಸಕ್ಕೆ ವಿಘ್ನ ಬರಕೂಡ\-ದೆಂದು ತಲೆಯನ್ನು ಅರ್ಧ ಬೋಳಿಸಿ ವಿಕಾರಮಾಡಿಕೊಂಡು ನೆಲಮಾಳಿಗೆಯಲ್ಲಿ ಕುಳಿತಿರುತ್ತಿದ್ದ. ಪ್ರಭಾವೀ ಹಾವಭಾವ ಮತ್ತು ಅಂಗಾಭಿನಯಕ್ಕಾಗಿ ದೊಡ್ಡ ನಿಲುಗನ್ನಡಿಯ ಎದುರು ದಿನವೂ ನಾಟಕ ನಡೆಸುತ್ತಿದ್ದ. ಮೂರು ವರ್ಷಗಳ ಅಜ್ಞಾತವಾಸದಿಂದ ಆತ ಹೊರ ಬಿದ್ದಾಗ ಜ್ಞಾನನಿಧಿ\-ಯಾಗಿದ್ದ. ಗ್ರೀಕ್ ದೊರೆ ಫಿಲಿಪ್ ಹೇಳಿದ: ‘ಇಡೀ ಜಗತ್ತನ್ನೇ ಜಯಿಸ ಬಹುದು; ಆದರೆ ಡಿಮೋಸ್ತನೀಸ್​ನ ನಾಲಗೆಯನ್ನು ಜಯಿಸುವುದು ಅಸಾಧ್ಯ.’ ವಾಗ್ದೇವಿ ಅವನ ನಾಲಿಗೆಯಲ್ಲಿ ‘ನಲಿದೊಲಿದು ನರ್ತಿಸುತ್ತಿದ್ದಳು.’

\vskip 1pt

ಡಿಮೋಸ್ತನೀಸ್​ನ ತಾಳ್ಮೆ ಮತ್ತು ಸಾಹಸ ಅದ್ವಿತೀಯವಾಗಿತ್ತು. ತತ್ಪರಿಣಾಮವಾಗಿ ಆತನು ಪಡೆದ ಫಲವೂ ಅದ್ವಿತೀಯವಾಗಿತ್ತು.

\vskip 1pt

ಡಾ.\ ಅಂಬೇಡ್ಕರ್ ವಿದೇಶದಲ್ಲಿ ವಿದ್ಯಾರ್ಥಿಯಾಗಿದ್ದಾಗ ತಮ್ಮ ಅಧ್ಯಯನ ಅಭ್ಯಾಸಗಳಲ್ಲಿ ಮುಳುಗಿಹೋಗಿದ್ದರು. ಜ್ಞಾನ ಪಿಪಾಸುವಿಗೆ ನಿದ್ರೆ ಮತ್ತು ಸುಖ ಇಲ್ಲ ಎಂಬ ಸುಭಾಷಿತ ಅವರ ಪಾಲಿಗೆ ಅಕ್ಷರಶಃ ನಿಜವಾಗಿತ್ತು. ವಿದ್ಯಾಭ್ಯಾಸಕ್ಕೆಂದು ತಾಯ್ನಾಡಿನಿಂದ ಬಹುದೂರ ಹೋಗಿ ಕಾಲ ಹರಣ ಮಾಡಿದರೆ, ಭೋಗ ವಿಲಾಸ ವೈಭವಗಳ ಬೆನ್ನಟ್ಟಿದರೆ ದೇಶದ್ರೋಹವಾಗುತ್ತದೆಂದು ಅವರು ಯೋಚಿಸಿದರು. ಹಾಗಾಗಿ ಸಿನೆಮಾ ನೋಡುವ ಚಟವಾಗಲಿ, ಊರುಕೇರಿ ಅಲೆಯುವ ಹವ್ಯಾಸವಾಗಲಿ ಅವರ ಮನಸ್ಸನ್ನೆಂದೂ ಸೆಳೆಯಲಿಲ್ಲ. ಅಂಬೇಡ್ಕರ್ ಮಹಾ ಪುಸ್ತಕಪ್ರೇಮಿಗಳು. ಪುಸ್ತಕಗಳನ್ನು ಕಂಡರೆ ಅವರಿಗೆ ದೇಹಾಯಾಸ ಮಾನಸಿಕ ವ್ಯಥೆಗಳೂ ದೂರವಾಗುತ್ತಿದ್ದವು. ಲಂಡನ್ನಿಗೆ ಅವರು ವಿಶೇಷ ಅಧ್ಯಯನಕ್ಕಾಗಿ ಹೋದಾಗಲಂತೂ, ಪುಸ್ತಕಾಲಯದ ಕಾವಲುಗಾರ ಕಿಟಕಿಬಾಗಿಲುಗಳನ್ನು ಮುಚ್ಚಿ ಬಂದು ಅವರನ್ನು ಎಚ್ಚರಿಸುವ ತನಕವೂ ತಮ್ಮ ಆಸನದಲ್ಲೇ ಕುಳಿತು ಓದಿನಲ್ಲಿ ಮಗ್ನರಾಗಿರುತ್ತಿದ್ದರು. ಅಲ್ಪಾಹಾರಿಯಾಗಿದ್ದುಕೊಂಡು ಅವರು ತಮ್ಮ ಸಾಹಸದ ಜ್ಞಾನಯಾತ್ರೆಯನ್ನು ತಡೆಯಿಲ್ಲದೇ ನಡೆಸಿದರು. ಅವರ ಜ್ಞಾನದಾಹ ಅಂತಹದಾಗಿತ್ತು.

\vskip 1pt

‘ಮಹತ್ಕಾರ್ಯಗಳು ಹಠಾತ್ ಬಲದಿಂದಲ್ಲ – ನಿರಂತರ ಪರಿಶ್ರಮದಿಂದ ಸಾಧ್ಯವಾಗಿವೆ’\break ಎಂದು ಡಾ.\ ಜಾನ್ಸನ್ ಹೇಳಿದ.

\vskip 1pt

‘ಹಿಡಿದ ಕೆಲಸವನ್ನು ಪರಿಪೂರ್ಣಗೊಳಿಸುವ ದೃಢನಿರ್ಧಾರವೇ ಬಲಿಷ್ಠನಿಗೂ, ದುರ್ಬಲ ನಿಗೂ ಇರುವ ಅಂತರವನ್ನು ಸೂಚಿಸುತ್ತದೆ’ಎಂದು ಕಾರ್ಲೈಲ್ ಹೇಳಿದ.

‘ನಿರಂತರ ಸಾಧನೆ ಎಲ್ಲ ಕಷ್ಟಗಳನ್ನೂ ದೂರ ಮಾಡುವುದು’ ಎಂದು ಒಂದು ಲ್ಯಾಟಿನ್ ಗಾದೆ ಹೇಳುತ್ತದೆ.

‘ಸರಿಯಾದ ಮತ್ತು ಉತ್ಸಾಹಪೂರಿತ ಪ್ರಯತ್ನದಿಂದ ಪಡೆಯಲು ಸಾಧ್ಯವಾಗದ ವಸ್ತು ಯಾವುದೂ ಈ ಪ್ರಪಂಚದಲ್ಲಿಲ್ಲ’ ಎಂಬುದು ‘ಯೋಗವಾಸಿಷ್ಠ’ದ ಮತ.

ಯಾರೂ ಕಂಡರಿಯದ ಕಡೆಗೆ ತನ್ನ ಹಡಗನ್ನು ತಿರುಗಿಸಿ ಹೊಸ ಜಗತ್ತನ್ನು ಕಂಡುಹಿಡಿದ ಕೊಲಂಬಸನ ಧೈರ್ಯವೆಂಥದು! ಪೂರ್ವನಿಶ್ಚಿತ ಯೋಜನೆಯಂತೆ ಅವನಿಗೆ ತನ್ನ ಗುರಿಯನ್ನು ಸೇರಲು ಅಸಾಧ್ಯವಾದಾಗ, ಅವನ ಸಂಗಡಿಗರು ಬೇಸತ್ತು ಆತನನ್ನು ವಿರೋಧಿಸುತ್ತ ಸಮುದ್ರ ಕ್ಕೆಸೆದು ಬಿಡುತ್ತೇವೆಂದು ಬೆದರಿಕೆ ಹಾಕಿದರು. ಕಷ್ಟ ಪರಂಪರೆಗಳ ಮಧ್ಯದಲ್ಲಿ ದಿಕ್ಕುಕಾಣದ ಆ ಜಲರಾಶಿಯ ನಡುವೆ, ತನ್ನವರ ಪ್ರಬಲ ವಿರೋಧವನ್ನೂ ಲೆಕ್ಕಿಸದೆ ಏಕಾಂಗಿಯಾಗಿ ಅಮಿತ ಸಾಹಸ ಹಾಗೂ ಆತ್ಮವಿಶ್ವಾಸದಿಂದ ಮುಂದುವರಿಯುತ್ತ ಆತ ವಿಜಯಿಯಾದ.

ಜಾರ್ಜ್ ಸ್ಟೀವನ್​ಸನ್ ತನ್ನ ಯಂತ್ರವನ್ನು ಪರಿಷ್ಕರಿಸಲು ಹದಿನೈದು ವರ್ಷಗಳ ಕಾಲ ದುಡಿದ. ಜೇಮ್ಸ್​ ವಾಟ್ ಮೂವತ್ತು ವರ್ಷಗಳ ಕಾಲ ನಿರಂತರ ಶ್ರಮದಿಂದ ತಾನು ರಚಿಸಲು ತೊಡಗಿದ ಯಂತ್ರವನ್ನು ಪೂರ್ಣಗೊಳಿಸಿದ. ಪಿಚ್ ಬ್ಲೆಂಡ್ ಅದಿರಿನಿಂದ ರೇಡಿಯಮ್ ಅನ್ನು ಬೇರ್ಪಡಿಸಲು ಕ್ಯೂರಿ ದಂಪತಿಗಳು ಎಡೆಬಿಡದೆ ನಾಲ್ಕು ವರ್ಷಗಳ ಕಾಲ ಆ ಅದಿರನ್ನು ಕುಲುಮೆ ಯಲ್ಲಿ ಕಾಯಿಸಿದರು. ಸಾಧನೆಯ ಶಿಖರದಲ್ಲಿ ಕಂಗೊಳಿಸುವವರೆಲ್ಲರ ಯಶಸ್ಸಿನ ಗುಟ್ಟು– ತಾಳ್ಮೆಯಿಂದ ನಿರಂತರ ಪ್ರಯತ್ನಶೀಲರಾದುದೇ–ಆಗಿದೆ.

ವಿದ್ಯುದ್ದೀಪದ ಆವಿಷ್ಕಾರವನ್ನು ಮಾಡಿ ಎಡಿಸನ್ ತನ್ನ ಪ್ರಯೋಗಗಳಲ್ಲಿ ಒಂಬೈನೂರು ಬಾರಿ ವಿಫಲನಾದ. ಆದರೂ ತಾಳ್ಮೆಯಿಂದ ಹೋರಾಟ ನಡೆಯಿಸಿ ಯಶಸ್ವಿಯಾದ. ‘ಇಷ್ಟೊಂದು ಸೋಲುಗಳಿಂದ ನಿಮಗೆ ಆಶಾಭಂಗವಾಗುವುದಿಲ್ಲವೇ? ಈ ಸಂಶೋಧನಾ ಕಾರ್ಯದಲ್ಲೇ ಬೇಸರ ಬರುವುದಿಲ್ಲವೇ?’ ಎಂದು ಎಡಿಸನ್​ನನ್ನು ಪ್ರಶ್ನಿಸಿದಾಗ ‘ಇಲ್ಲ, ನನಗೆ ಬೇಸರವಿಲ್ಲ; ತಪ್ಪು\break ಯಾವುದು, ಸರಿ ಯಾವುದು ಎಂದು ತಿಳಿಯುವ ಯತ್ನದಲ್ಲಿ ಒಂಬೈನೂರು ತಪ್ಪುಗಳನ್ನು ಬಿಟ್ಟು ಸರಿಯಾದುದನ್ನು ಹಿಡಿದಿದ್ದೇನೆ ಎನ್ನುವ ತೃಪ್ತಿ ಇದೆ’ ಎಂದನಂತೆ!

ಐಸಾಕ್ ನ್ಯೂಟನ್ ವರ್ಷಗಟ್ಟಲೆ ದುಡಿದು ಸಂಶೋಧಿಸಿದ ತನ್ನ ಬರಹವನ್ನು ತನ್ನ ಅಭ್ಯಾಸದ ಕೋಣೆಯ ಮೇಜಿನ ಮೇಲಿರಿಸಿದ್ದ. ಅವನ ಸಾಕುನಾಯಿ ಉರಿಯುತ್ತಿದ್ದ ಮೇಣದ ಬತ್ತಿಯನ್ನು ಆ ಹಾಳೆಗಳ ಮೇಲೆ ಕೆಡವಿತು. ಬರಹ ಪೂರ್ಣವಾಗಿ ಸುಟ್ಟುಹೋದ ಬಳಿಕ ನ್ಯೂಟನ್ ಅದನ್ನು ಕಂಡು ಬಹಳಷ್ಟು ನೊಂದುಕೊಂಡ. ಆದರೆ ಧೃತಿಗೆಡಲಿಲ್ಲ. ತಿರುಗಿ ಆ ಕೆಲಸವನ್ನು ಮಾಡುವೆ ನೆಂದು ದೃಢಸಂಕಲ್ಪ ಮಾಡಿದ. ತಾಳ್ಮೆಯಿಂದ ದುಡಿದು ಮುಗಿಸಿಯೂ ಬಿಟ್ಟ.

ವಯಸ್ಸಿಗೂ, ಯಶಸ್ಸಿಗೂ ಸಂಬಂಧ ಕಲ್ಪಿಸಲು ಸಾಧ್ಯವಿಲ್ಲ. ಕೆಲವರು ತಮ್ಮ ಬದುಕಿನ ಪ್ರಾರಂಭದ ಹಂತಗಳಲ್ಲಿ ಯಶಸ್ಸನ್ನು ಗಳಿಸಿದರೆ ಇನ್ನು ಕೆಲವರು ಮಧ್ಯ ವಯಸ್ಸಿನಲ್ಲಿ, ಇತರರು ಅಪರ ವಯಸ್ಸಿನಲ್ಲಿ. ಇಂಗ್ಲೆಂಡಿನಲ್ಲಿ ಪಿಟ್ ಎಂಬಾತ ಇಪ್ಪತ್ತನಾಲ್ಕನೇ ವಯಸ್ಸಿಗೆ ಪ್ರಧಾನಿ ಯಾಗಿದ್ದರೆ ಗ್ಲಾಡ್​ಸ್ಟೋನ್ ಪ್ರಧಾನಿಯಾದದ್ದು ತನ್ನ ಎಂಭತ್ತಮೂರನೇ ವಯಸ್ಸಿನಲ್ಲಿ. ಗಯಟೆ ಹತ್ತನೇ ವಯಸ್ಸಿನಲ್ಲಿ ತನ್ನ ಬರವಣಿಗೆಯ ಕೆಲಸ ಪ್ರಾರಂಭಿಸಿದ್ದ. ಆದರೆ ಆತನ ಅಮೂಲ್ಯ ಕೃತಿ \enginline{Faust (}ಫೌಸ್ಟ್​) ಬೆಳಕು ಕಂಡದ್ದು ಎಂಭತ್ತನೇ ವಯಸ್ಸಿನಲ್ಲಿ. ಕೋಲ್​ರಿಜ್ ತನ್ನ ಪ್ರಸಿದ್ಧ \enginline{Ancient Mariner (}ಏನ್ಶಿಯಂಟ್ ಮ್ಯಾರಿನರ್​) ಬರೆದದ್ದು ಇಪ್ಪತ್ತೈದನೆ ವಯಸ್ಸಿನಲ್ಲಿ. ಲಿಯೋನಾರ್ಡೊ ಡಾ ವಿಂಚಿಯ \enginline{Last Supper (}ಲಾಸ್ಟ್ ಸಪ್ಪರ್​) ಎನ್ನುವ ಪ್ರಸಿದ್ಧ ಕಲಾಕೃತಿ ಹೊರಹೊಮ್ಮಿದುದು ಎಪ್ಪತ್ತೇಳರ ಅಪರ ವಯಸ್ಸಿನಲ್ಲಿ. ಕೆಲ್​ವಿನ್ ತನ್ನ ಪ್ರಥಮ ವೈಜ್ಞಾನಿಕ ಸಂಶೋಧನೆ ಮಾಡಿದ್ದು ಹದಿನೆಂಟನೇ ವಯಸ್ಸಿನಲ್ಲಾದರೆ, ನಾವಿಕರ ದಿಕ್ಸೂಚಿಗೆ ಸುಧಾರಿತ ರೂಪ ನೀಡಿದ್ದು ಎಂಭತ್ತ ಮೂರರಲ್ಲಿ. ದೇಹಾರೋಗ್ಯ ಮತ್ತು ಸ್ವಸ್ಥ ಶರೀರವೇ ಯಶಸ್ಸಿಗೆ ಕಾರಣವೆನ್ನಲು ಸಾಧ್ಯವೇ? ಮಿಲ್ಟನ್ ಕುರುಡ, ನೆಪೋಲಿಯನ್ ಚರ್ಮ ವ್ಯಾಧಿ ಪೀಡಿತ. ಜೂಲಿಯಸ್ ಸೀಸರ್ ಮೂರ್ಛೆ ರೋಗಗ್ರಸ್ತ, ಬಿಥೋವನ್ ಕಿವುಡ, ಬೈರನ್ ಕೂಡ ಕಿವುಡ, ಮಹಾವಾಗ್ಮಿ ಡಿಮೋಸ್ತನೀಸ್ ಮೊದಲು ಉಗ್ಗುತ್ತಿದ್ದ!


\section*{ನಿರಂತರ ಪರಿಶ್ರಮಕ್ಕೆ ಜೀವಂತ ನಿದರ್ಶನ}

\addsectiontoTOC{ನಿರಂತರ ಪರಿಶ್ರಮಕ್ಕೆ ಜೀವಂತ ನಿದರ್ಶನ}

ಗಾಂಧೀಜೀಯವರು ತಮ್ಮ ಪರಿಮಿತಿ ಮತ್ತು ದೌರ್ಬಲ್ಯಗಳನ್ನು ಮೆಟ್ಟಿ ಮೇಲೇರುವ ಪ್ರಾಮಾಣಿಕ ಪ್ರಯತ್ನದ ಜೊತೆಜೊತೆಗೇ ರಾಷ್ಟ್ರದ ಜನರು ತಮ್ಮ ದೌರ್ಬಲ್ಯ ಪರಿಮಿತಿಗಳನ್ನು ದಾಟಲು ಪ್ರೇರಣೆ ನೀಡಿದವರು. ತಮ್ಮ ತಪೋಮಯ ಜೀವನದ ನೈತಿಕ, ಆಧ್ಯಾತ್ಮಿಕ ಬಲದಿಂದ ಜನಮಾನಸದಲ್ಲಿ ರಾಷ್ಟ್ರದ ಸ್ವಾತಂತ್ರ್ಯ ಹೋರಾಟಕ್ಕೆ ಸಾತ್ವಿಕ ಸ್ಫೂರ್ತಿಯನ್ನು ತುಂಬಿದವರು ಅವರು. ವೃದ್ಧಾಪ್ಯದಲ್ಲೂ ದಿನದ ಇಪ್ಪತ್ತು ಗಂಟೆಗಳ ಕಾಲ ಜನರ ಹಿತಾಭ್ಯುದಯಗಳನ್ನೇ ಚಿಂತನೆ ಮಾಡುತ್ತ, ಸದಾ ಜಾಗರೂಕರಾಗಿದ್ದುಕೊಂಡು ದುಡಿಯುತ್ತ, ದುಡಿಮೆಯ ಮಹಿಮೆಯನ್ನು ಸಾರಿದ ಧೀರರು. ಬಾಯಿಮಾತಿನ ಅನುಕಂಪವನ್ನು ವ್ಯಕ್ತಪಡಿಸುವುದರಲ್ಲಿ ಕಾಲಕಳೆಯದೆ ದೀನದಲಿತರ ದುಃಖ ದುಮ್ಮಾನಗಳನ್ನು ದೂರ ಮಾಡಲು ನಿರಂತರ ಶ್ರಮಿಸಿ ಸೇವಾಪರಾಯಣರಾಗಿ, ನಿಸ್ವಾರ್ಥತೆ ಮತ್ತು ನೈತಿಕ ಸಮುನ್ನತಿಯ ಶಿಖರವನ್ನೇರಿದವರು. ಬ್ರಿಟಿಷ್ ಚಕ್ರಾಧಿಪತ್ಯವನ್ನು ಎದುರಿಸಿ ನಿಂತು ಸ್ವಾತಂತ್ರ್ಯ ಹೋರಾಟದ ಅಗ್ರದೂತರಾಗಿ ಮಹಾತ್ಮರೆನ್ನಿಸಿಕೊಂಡವರೂ ಅವರೇ. ಮಹಾತ್ಮಾ\-ಗಾಂಧೀಜಿ ಸಾಮಾನ್ಯರಲ್ಲಿ ಜನಿಸಿದವರು, ಸಾಮಾನ್ಯರಂತೆ ತಪ್ಪುಗಳ ನ್ನೆಸಗಿದವರು. ಆದರೆ ಅಸಾಮಾನ್ಯರಾಗಿ ಅಗ್ರಗಣ್ಯರಾದವರು. ಅವರು ಆರ್ಜಿಸಿದ ಸದ್ಗುಣಗಳು ಅನುಕರಣೀಯ\-ವಲ್ಲವೆ? ಅವರನ್ನು ಜನ ಮರೆಯಲು ಸಾಧ್ಯವೇ? ಅವರ ಬದುಕು ದೃಢನಿರ್ಧಾರ, ಧ್ಯೇಯನಿಷ್ಠೆ ಮತ್ತು ತಾಳ್ಮೆಯಿಂದ ಕೂಡಿದ ನಿರಂತರ ಪರಿಶ್ರಮಕ್ಕೊಂದು ಜೀವಂತ ನಿದರ್ಶನವಲ್ಲವೇ?\break ಅಂತಹುದೇ ಒಂದು ಬದುಕು ನೀಗ್ರೋ ಜನಾಂಗದ ಮಹಾನಾಯಕ ಬೂಕರ್ ಟಿ.\break ವಾಶಿಂಗ್ಟನ್​ನದು.

\newpage


\section*{ದಲಿತರ ಧೀರನಾಯಕ}

\addsectiontoTOC{ದಲಿತರ ಧೀರನಾಯಕ}

ಗುಲಾಮಗಿರಿಯ ಉಸಿರುಕಟ್ಟುವಂಥ ಕೊಳಚೆಯ ವಾತಾವರಣದಲ್ಲಿ ಜನಿಸಿದವನು ಆತ.\break ಬಿಳಿಯರ ಕುಹಕ, ನಿಂದೆ, ತಿರಸ್ಕಾರ, ಪೀಡೆಗಳನ್ನೇ ಉಂಡು ನಿರಂತರ ನಿರುತ್ಸಾಹದ ಪರಿಸರವನ್ನೇ ಕಂಡವನು. ಯಾವ ಶಿಕ್ಷಣಕ್ಕೂ, ಅಭಿವೃದ್ಧಿಗೂ ಅವಕಾಶವಿಲ್ಲದಿದ್ದರೂ, ಬಿಳಿಯರ ಮಕ್ಕಳು ಓದುವುದನ್ನು ಕಂಡು ಓದು ಕಲಿಯುವ ಹಂಬಲ ಅವನಲ್ಲಿ ಉದಿಸಿತಂತೆ. ಯಾವ ಅಧ್ಯಾಪಕರ ಸಹಾಯವನ್ನೂ ಮೊದಲಿಗೆ ಪಡೆಯದೆ ತನ್ನ ತಾಯಿ ಕೊಟ್ಟ ಅಕ್ಷರಮಾಲೆಯ ಪುಟ್ಟ ಪುಸ್ತಕವನ್ನು ನೋಡುತ್ತ ನೋಡುತ್ತ ಕಲಿಯಲು ಆರಂಭಿಸಿ ಕೆಲವೇ ವಾರಗಳಲ್ಲಿ ಅದನ್ನು ಕರಗತ ಮಾಡಿಕೊಂಡನಂತೆ. ಬಿಳಿಯರ ಮಕ್ಕಳು ಶಾಲೆಯಲ್ಲಿ ವಿದ್ಯಾಭ್ಯಾಸ ಮಾಡುವುದನ್ನು ಕಂಡು ಬಾಲಕ ಬೂಕರನಿಗೆ ‘ಕಲಿಯಬೇಕು ಕಲಿಯಲೇಬೇಕು’ ಎನ್ನುವ ತೀವ್ರ ಆಶೆ ಅಂಕುರಿಸಿತು. ಈ ತೀವ್ರ ಅಭೀಪ್ಸೆ ಅವನನ್ನು ಎಂಥ ಉನ್ನತ ಮಟ್ಟಕ್ಕೆ ಏರಿಸಿತು! ಅವನ ಕಲಿಕೆ ಎಷ್ಟು ಸರ್ವತೋಮುಖ\-ವಾದದ್ದು ಎಂಬುದು ರೋಮಾಂಚಕಾರಿಯಾದ ಕತೆ. ಬಡತನದ ಬೇಗೆಯಲ್ಲಿ ಬೆಂದು ಕಷ್ಟನಷ್ಟ, ದುಃಖ\-ದುಮ್ಮಾನಗಳನ್ನು ಅಪಾರ ತಾಳ್ಮೆಯಿಂದ ಸಹಿಸಿ, ಮೂಳೆ ಮುರಿಯುವಂಥ ದುಡಿಮೆಯ ಬಲದಿಂದ ಮೇಲಕ್ಕೇರಿದ ಮಹಾನುಭಾವನ ಕತೆ ಅದು. ಅಭಿವೃದ್ಧಿಶೀಲ ವ್ಯಕ್ತಿ ಗಳೂ, ರಾಷ್ಟ್ರಗಳೂ, ಮೇಲೇರಲು ಬೇಕಾದ ಸ್ಫೂರ್ತಿಯನ್ನು ಆತನ ಜೀವನದಿಂದ ಪಡೆಯು ವಷ್ಟು ಉನ್ನತ ಮಟ್ಟಕ್ಕೆ ಅವನು ಏರಿದ್ದ. ಸ್ಥಾನಮಾನ ಗೌರವಗಳು ಅವನನ್ನು ಹುಡುಕಿಕೊಂಡು ಬಂದಾಗ ಅಹಂಕಾರದಿಂದ ಸ್ವಲ್ಪವೂ ಮತ್ತನಾಗದೆ, ನಿಸ್ವಾರ್ಥತೆಯ ಪರಾಕಾಷ್ಠೆಯನ್ನೇರಿ ದಲಿತ ಜನಾಂಗವನ್ನು ಮೇಲಕ್ಕೆತ್ತಲು ಹಗಲು ರಾತ್ರಿಯೆನ್ನದೆ ದುಡಿದ ಈ ಧೀರ ಕರ್ಮಯೋಗಿಯ ರೀತಿನೀತಿಗಳು ಜಗತ್ತಿಗೇ ಒಂದು ಮಹಾ ಆದರ್ಶ ನಿದರ್ಶನ. ಸಮಾಜದ ಹಿತಚಿಂತನೆ ಮಾಡ ಬಲ್ಲ ಯಾವ ವ್ಯಕ್ತಿಗಾದರೂ ಈತನು ಆರ್ಜಿಸಿದ ಸದ್ಗುಣಗಳ ಸೌಂದರ್ಯವನ್ನು ಕಂಡು ರೋಮಾಂಚನವಾಗ ದಿರದು! ಶ್ರದ್ಧೆ, ತಾಳ್ಮೆ, ಸ್ವಪ್ರಯತ್ನ, ಆತ್ಮಶಕ್ತಿಯ ಬಲ–ಇವುಗಳಿಂದ ಮನುಷ್ಯನು ಪರಿಸರದ ಪ್ರಬಲ ಪ್ರತಿಗಾಮಿ ಶಕ್ತಿಗಳನ್ನು ಮೀರಿ ನಿಲ್ಲಬಲ್ಲ ಎಂಬುದಕ್ಕೆ ಆತನ ಜೀವನ ಕಾರ್ಯಗಳು ಜ್ವಲಂತ ಜೀವಂತ ನಿದರ್ಶನಗಳು. ಅವನೆಂದಂತೆ–‘ಯಶಸ್ಸಿನ ಹಿರಿಮೆಯನ್ನು ಒಬ್ಬಾತ ಗಳಿಸಿದ ಸ್ಥಾನಮಾನಗಳಿಂದ ಅಳೆಯುವುದಕ್ಕಿಂತಲೂ, ಆ ಹೋರಾಟದಲ್ಲಿ ಆತ ಎದುರಿಸಿ ಪಾರಾದ ವಿಘ್ನ ಪರಂಪರೆಗಳಿಂದ ಅಳೆಯುವುದು ಸೂಕ್ತ.’\footnote{\engfoot{I have learned that success is to be measured not so much by the position that one has reached in life as by the obstacles which he has overcome while trying to succeed.}\hfill\engfoot{ –Booker T. Washington}}

‘ಯಾರು ಇತರರಿಗಾಗಿ ಹೆಚ್ಚು ದುಡಿಯುವರೋ ಅವರೇ ಅತ್ಯಂತ ಸುಖಿಗಳು. ಇತರರಿಗಾಗಿ ಯಾವ ಸಹಾಯವನ್ನೂ ಮಾಡದವರು ಅತ್ಯಂತ ದುಃಖಿಗಳು.’ ‘ದುರ್ಬಲನಿಗೆ ಮಾಡಿದ ಸಹಾಯವು ಸಹಾಯ ಮಾಡಿದವನಿಗೆ ಬಲವನ್ನು ನೀಡುವುದು; ದೌರ್ಭಾಗ್ಯಶಾಲಿಯನ್ನು\break ಪೀಡಿಸುವುದರಿಂದ ದುರ್ಬಲತೆ ಪ್ರಾಪ್ತವಾಗುವುದು.’

ದೈಹಿಕ ಶ್ರಮವನ್ನು ಲೆಕ್ಕಿಸದೆ, ಸಂಬಳವನ್ನು ಗಮನಿಸದೆ, ತನ್ನ ಜನಾಂಗದವರನ್ನು ಸುಶಿಕ್ಷಿತ\-ರನ್ನಾಗಿ ಮಾಡಲು ಅವನು ಹಗಲಿರುಳು ಶ್ರಮಿಸಿದ. ಶಾಲೆಗಳನ್ನು ತೆರೆದು ಕಲಿಸಿದ. ಖಾಸಗಿ ಯಾಗಿಯೂ ಅನೇಕರಿಗೆ ಪಾಠ ಹೇಳಿದ. ಮುಂದೆ ಟಸ್ಕಗೀ ಎಂಬಲ್ಲಿ ಶಿಕ್ಷಣ ಸಂಸ್ಥೆಯನ್ನು ಕಟ್ಟಿದ. ಆತನ ನಿರಂತರ ದುಡಿಮೆ ಮತ್ತು ಸಾಹಸದಿಂದ ಕೆಲವೇ ವರ್ಷಗಳಲ್ಲಿ ಆ ಸಂಸ್ಥೆ ಪ್ರಸಿದ್ಧಿ ಪಡೆಯಿತು. ಜಗತ್ತಿನ ಬೇರೆಬೇರೆ ಭಾಗಗಳಿಂದ ಅಲ್ಲಿಗೆ ವಿದ್ಯಾರ್ಥಿಗಳು ಬಂದರು.

ಇಪ್ಪತ್ತು ವರ್ಷಗಳಲ್ಲಿ ಎರಡು ಸಾವಿರದ ಮುನ್ನೂರು ಎಕರೆಗಳನ್ನು ಆ ಸಂಸ್ಥೆ ಪಡೆಯಿತು. ಅದರಲ್ಲಿ ಏಳುನೂರು ಎಕರೆಗಳಷ್ಟು ಭೂಮಿಯನ್ನು ವ್ಯವಸಾಯಕ್ಕೆ ಉಪಯೋಗಿಸಲಾಗುತ್ತಿತ್ತು. ವಿದ್ಯಾರ್ಥಿಗಳೇ ಹೊಲದಲ್ಲಿ ದುಡಿದು ಬೆಳೆಯನ್ನು ಬೆಳೆಯುತ್ತಿದ್ದರು. ತಮಗೆ ಬೇಕಾದ ಕಟ್ಟಡ ನಿರ್ಮಾಣವನ್ನೂ ಕೈಗೊಳ್ಳುತ್ತಿದ್ದರು. ಸಾಮಾನ್ಯ ಶಿಕ್ಷಣದೊಂದಿಗೆ ವ್ಯವಸಾಯ ಮತ್ತು\break ಕೈಗಾರಿಕೆಯ ಶಿಕ್ಷಣವನ್ನೂ ನೀಡುತ್ತಿದ್ದರು. ಜೊತೆಗೆ ಧಾರ್ಮಿಕ ಶಿಕ್ಷಣಕ್ಕೂ ಸ್ಥಾನವಿತ್ತು. ವಿದ್ಯಾರ್ಥಿಗಳನ್ನು ಆಫೀಸಿನಿಂದ ಆಫೀಸಿಗೆ ಅಲೆದು ಉದ್ಯೋಗಕ್ಕಾಗಿ ಕೈಯೊಡ್ಡಿ ಬೇಡುವ ಭಿಕ್ಷುಕರನ್ನಾಗಿ ಮಾಡುವುದು ವಿದ್ಯಾಭ್ಯಾಸದ ಗುರಿ ಅಲ್ಲ ಎಂಬುದನ್ನು ಈ ದಲಿತರ ನಾಯಕ ತನ್ನ ಸಂಸ್ಥೆಯ ಮೂಲಕ ತೋರಿಸಿಕೊಟ್ಟ. ಸ್ವಪ್ರಯತ್ನ, ಸ್ವಾವಲಂಬನೆ, ಸಾಹಸ, ಉದ್ಯಮಶೀಲತೆಗಳ ಅಡಿಗಲ್ಲಿನ ಮೇಲೆ ಅವರ ಜೀವನವನ್ನು ರೂಪಿಸಿಕೊಳ್ಳುವಂತೆ ಪ್ರೇರಣೆ ನೀಡಿದ.

‘ಯಾವುದಾದರೊಂದು ರೀತಿಯಲ್ಲಿ ನಮ್ಮ ದೈನಂದಿನ ಜೀವನಕ್ಕೆ ಸಂಬಂಧಿಸದ ವಿದ್ಯಾಭ್ಯಾಸವನ್ನು ವಿದ್ಯಾಭ್ಯಾಸವೆಂದೇ ಕರೆಯಲಾಗುವುದಿಲ್ಲ’ ಎಂದವನು ಹೇಳಿದ್ದಾನೆ. ‘ವಿದ್ಯಾಭ್ಯಾಸವು ದೈಹಿಕ ಶ್ರಮದಿಂದ ಪಾರಾಗುವ ಸಾಧನವಾಗಿರದೆ ದೈಹಿಕ ಶ್ರಮವನ್ನೇ ಉನ್ನತಿಗೇರಿಸಿ ಅದಕ್ಕೆ ಗೌರವಸ್ಥಾನವನ್ನು ನೀಡುವುದು. ಆದುದರಿಂದ ವಿದ್ಯಾಭ್ಯಾಸವು ಪರೋಕ್ಷವಾಗಿ ಜನ ಸಾಮಾನ್ಯ ರನ್ನು ಮೇಲೆತ್ತಲು ಮತ್ತು ಅವರಿಗೆ ಒಂದು ಗೌರವಸ್ಥಾನ ನೀಡಲು ಸಹಾಯಕವಾಗುವುದು’ ಎಂಬುದು ಅವನು ಕಂಡುಕೊಂಡ ವಿದ್ಯಾಭ್ಯಾಸದ ಗುರಿಯಾಗಿತ್ತು.

‘ಕೆಲಸವು ಎಂದೂ ನನ್ನನ್ನು ನಿಯಂತ್ರಿಸದಂತೆ ಅದನ್ನು ಸಂಪೂರ್ಣ ನನ್ನ ಹತೋಟಿಯಲ್ಲಿ ಇಟ್ಟಿದ್ದೇನೆ. ನಾನು ಎಂದೆಂದೂ ಕೆಲಸದ ಆಳಾಗದೆ ಅದರ ಒಡೆಯನಾಗಿರುವಷ್ಟು ಅದಕ್ಕಿಂತ ಮುಂದಿರುತ್ತೇನೆ. ವ್ಯಕ್ತಿಯೊಬ್ಬ ತನ್ನ ಕೆಲಸದಲ್ಲಿ ಪೂರ್ಣ ಒಡೆತನ ಸಂಪಾದಿಸಿದ್ದರೆ ಆ ಒಡೆತನ ಅಥವಾ ದಕ್ಷತೆಯ ಪ್ರಜ್ಞೆ, ದೈಹಿಕ ಮಾನಸಿಕ ಮತ್ತು ಆಧ್ಯಾತ್ಮಿಕ ಸುಖವನ್ನು ನೀಡುವುದು. ಮನಸ್ಸಿಗೆ ಅದಮ್ಯ ಶಕ್ತಿ ಸ್ಥೈರ್ಯ ಮತ್ತು ಸ್ಫೂರ್ತಿಗಳನ್ನು ನೀಡುವುದು. ನಿಜವಾಗಿಯೂ ಕೆಲಸವು ಶ್ರಮದಾಯಕವಾಗುವುದರ ಬದಲಾಗಿ ದೇಹಕ್ಕೆ ಬಲವನ್ನು ನೀಡಿ ಮನಸ್ಸನ್ನು ಚುರುಕುಗೊಳಿಸು ವುದು. ಒಬ್ಬಾತ ತನ್ನ ಕೆಲಸವನ್ನು ಪ್ರೀತಿಸುವ ಸ್ಥಿತಿಗೆ ಬಂದಾಗ ಅದು ಒಂದು ತೆರನಾದ ಅತ್ಯಂತ ಹೆಚ್ಚಿನ ಮೌಲ್ಯದ ಶಕ್ತಿಯನ್ನುಂಟುಮಾಡುವುದು.

‘ಪ್ರತಿಯೊಬ್ಬ ವ್ಯಕ್ತಿಯೂ ತನ್ನ ಜೀವನದ ಪ್ರತಿಯೊಂದು ದಿನವನ್ನೂ ಅತ್ಯುತ್ತಮ ರೀತಿಯಲ್ಲಿ ಕಳೆಯಲು ನಿರ್ಧರಿಸಿದನಾದರೆ–ಎಂದರೆ ಪ್ರತಿಯೊಂದು ದಿನವೂ ಪವಿತ್ರತೆ, ನಿಸ್ವಾರ್ಥತೆ ಮತ್ತು ಉಪಯುಕ್ತತೆಯ ಪರಮಾವಧಿಯನ್ನು ಮುಟ್ಟುವ ರೀತಿಯಲ್ಲಿ ದುಡಿಯಲು ಯತ್ನಿಸಿದರೆ, ಆತನ ಜೀವನವು ಸದಾ ಸ್ಫೂರ್ತಿ ಉತ್ಸಾಹಗಳಿಂದ ತುಂಬಿರುತ್ತದೆ. ಇತರರ ಜೀವನವು ಉಪಯುಕ್ತವೂ ಸುಖಮಯವೂ ಆಗಲು ಪ್ರಯತ್ನಿಸುವುದರಿಂದ ಲಭಿಸುವ ಆನಂದ ಮತ್ತು ತೃಪ್ತಿಯನ್ನು ಯಾವನು ತನ್ನ ಜೀವನದಲ್ಲಿ ಅನುಭವಿಸಿಲ್ಲವೋ, ಅವನು ಬಿಳಿಯನೇ ಆಗಲಿ ಕರಿಯನೇ ಆಗಲಿ, ನಾನು ಅವನ ಬಗೆಗೆ ತೀವ್ರ ಮರುಕವನ್ನು ತೋರಿಸುತ್ತೇನೆ’ ಎಂಬುದು ಆತನ ಸ್ವಾನು ಭವದ ನುಡಿ, ನಿಜವಾಗಿಯೂ ಸ್ಫೂರ್ತಿಯ ಕಿಡಿ.


\section*{ಪಾವ್​ಲೋವ್​ನ ಚರಮಸಂದೇಶ}

\vskip -6pt\addsectiontoTOC{ಪಾವ್​ಲೋವ್​ನ ಚರಮ\-ಸಂದೇಶ}

ರಷ್ಯಾದೇಶದ ಮನೋವಿಜ್ಞಾನಿ ಪಾವ್​ಲೋವ್ ಮರಣಶಯ್ಯೆಯಲ್ಲಿದ್ದಾಗ ಆತನ ಪ್ರೀತಿಯ\break ವಿದ್ಯಾರ್ಥಿಗಳು ಬದುಕಿಗೆ ಉಪಯುಕ್ತವಾಗುವ ಗೆಲುವಿನ ರಹಸ್ಯವನ್ನು ತಿಳಿಸಿಕೊಡುವ ಒಂದು ಉಪದೇಶವನ್ನು ನೀಡಬೇಕೆಂದು ಕೇಳಿಕೊಂಡರು. \enginline{‘Passion and Gradualness’} ಎಂದಾತ ಉಸುರಿದನಂತೆ. ಪ್ಯಾಶನ್ ಎಂದರೆ ತೀವ್ರ ಹಂಬಲ. ಗ್ರ್ಯಾಜುಯಲ್​ನೆಸ್ ಎಂದರೆ ಮೆಟ್ಟಲು ಮೆಟ್ಟಲಾಗಿ ಮೇಲೇರುವ ತಾಳ್ಮೆ ಎಂದರ್ಥ. ಯಾವುದೇ ಕ್ಷೇತ್ರದಲ್ಲಿ ಉನ್ನತ ಮಟ್ಟದ ವಿಜಯ ಸಾಧ್ಯವಾಗಬೇಕಾದರೆ ತೀವ್ರ ಹಂಬಲದೊಂದಿಗೆ ಕ್ರಮವಾಗಿ, ಮೆಟ್ಟಲು ಮೆಟ್ಟಲಾಗಿ, ಆದರೆ ದೃಢನಿಷ್ಠೆಯಿಂದ ಮುನ್ನಡೆಯಬೇಕು ಎಂಬುದು ಆ ಮಾತಿನ ತಾತ್ಪರ್ಯ. ‘ಆತನಿಗೆ ಸಂಗೀತದ ಹುಚ್ಚು. ಇಡೀ ದಿನ ಕಿರಿಚುತ್ತಿರುತ್ತಾನಪ್ಪ’ ಎನ್ನುವ ಮಾತಿನಲ್ಲಿ ಸಂಗೀತವನ್ನು ಕಲಿಯುವ ತೀವ್ರ ಹಂಬಲ ವ್ಯಕ್ತವಾಗುತ್ತದೆ. ತೀವ್ರ ಹಂಬಲವೆಂದಾಗ ಗುರಿಯನ್ನು ಬೇಗನೇ ಹೇಗಾದರೂ ಮಾಡಿ ತಲಪಬೇಕೆಂದಾಗದು. ಗುರಿ ಸೇರುವವರೆಗೆ ಸೋಲಿನ ಭಾವನೆ ಮನಸ್ಸನ್ನು ಪ್ರವೇಶಿಸದಂತೆ ಮುಗ್ಗರಿಸದೆ ಮುನ್ನಡೆಯಬೇಕು ಎಂದರ್ಥ. ಸೋಲಿನ ಮನೋಭಾವನೆ ಬಾರದಂತಾಗಬೇಕಾ ದರೆ ಪ್ರಾರಂಭದಿಂದಲೇ ಗೆಲುವನ್ನು ತಂದುಕೊಡುವ ಕೆಲಸಗಳನ್ನು ಪೂರ್ಣಮನಸ್ಸಿನಿಂದ ಮಾಡಬೇಕು. ಇದು ಅಭಿರುಚಿಯನ್ನು ಅರಳಿಸಿ ಏಕಾಗ್ರತೆಯ ಜೊತೆಗೆ ಶಕ್ತಿ, ಉತ್ಸಾಹ, ಆನಂದಗಳನ್ನು ವೃದ್ಧಿಸುವುದು. ಭಾರ ಎತ್ತುವವರು, ಜಟ್ಟಿಗಳು, ಸಂಗೀತಾಭ್ಯಾಸಿಗಳು, ಸರ್ಕಸ್ಸಿನ ಸಾಹಸಿಗಳು–ಇವರೆಲ್ಲರ ಯಶಸ್ಸಿನ ಗುಟ್ಟೂ ಈ ರೀತಿಯ ಅಭ್ಯಾಸ.


\section*{ತಾಳಿದವ ಬಾಳಿಯಾನು}

\vskip -6pt\addsectiontoTOC{ತಾಳಿದವ ಬಾಳಿಯಾನು}

ಪಾವ್​ಲೋವ್ ಹೇಳಿದ ಮಾತನ್ನು ಇನ್ನೊಂದು ರೀತಿಯಿಂದಲೂ ಅರ್ಥೈಸಬಹುದು. ಯಾವುದೇ ಕ್ಷೇತ್ರದಲ್ಲಿ ಯಾವುದೇ ವಿಷಯದಲ್ಲಿ ದಕ್ಷತೆ ಅಥವಾ ಉನ್ನತ ಮಟ್ಟದ ಸಿದ್ಧಿ ಬೇಕಾದರೆ ಆ ವಿಷಯದಲ್ಲಿ ಪ್ರೀತಿ, ಸಾಧನೆಯಲ್ಲಿ ಸಾವಕಾಶವಾಗಿ ಮುನ್ನಡೆಯಲು ತಾಳ್ಮೆ ಬೇಕು. ಹೌದು. ಪ್ರೀತಿಯ ಹೆಬ್ಬಾಗಿಲಿಗೆ ತಾಳ್ಮೆಯೇ ಹೊಸ್ತಿಲು.

ನಮ್ಮಲ್ಲಿ ಅಪಾರ ಶಕ್ತಿ ಅಡಗಿದ್ದರೂ ಅಂಬೆಗಾಲಿಡುವುದಕ್ಕೆ ಮೊದಲೇ ಇಪ್ಪತ್ತು ಮೈಲು\break ಓಟದ ಸ್ಪರ್ಧೆಯಲ್ಲಿ ಭಾಗವಹಿಸಲು ಸಾಧ್ಯವಿಲ್ಲವಷ್ಟೆ. ನಾವು ನಿಂತ ಸ್ಥಾನದಿಂದ ಮುನ್ನಡೆಯ ಬೇಕು, ಮೇಲೇರಬೇಕು. ಸದ್ಯದ ಸತ್ವ ಸಾಮರ್ಥ್ಯಗಳೇನು? ಇದನ್ನು ಎಷ್ಟರಮಟ್ಟಿಗೆ, ಹೇಗೆ, ಹೆಚ್ಚಿಸಿಕೊಳ್ಳಬಹುದು? ಎಂಬುದನ್ನು ತಿಳಿಯದೇ ಇತರರಿಗೆ ಹೋಲಿಸಿಕೊಂಡು ಅನುಕರಿಸ\-ಹೊರಟರೆ, ದುಡುಕಿದರೆ, ಧೃತಿಗೆಟ್ಟು ಮುಗ್ಗರಿಸಿ ಬೀಳಬೇಕಾಗಬಹುದು. ಹೀಗೆ ಬಿದ್ದವನು ಬೀಳುವು ದಕ್ಕೆ ಕಾರಣ ತನ್ನಲ್ಲಿಲ್ಲ, ನೆಲದ ಅಂಕುಡೊಂಕು ಓರೆಕೋರೆಯಲ್ಲಿದೆ ಎನ್ನಲು ತೊಡಗುವು ದುಂಟು. ತಪ್ಪುದಾರಿಯಲ್ಲಿ ನಡೆದೂ ತನ್ನನ್ನು ತಾನು ಸಮರ್ಥಿಸಿಕೊಂಡು ತನ್ನ ಅಭ್ಯುದಯಕ್ಕೆ ತಾನೇ ಆಘಾತ ಮಾಡಿಕೊಳ್ಳುವುದುಂಟು. ಮುನ್ನಡೆದವರ ಬಗೆಗೆ ಕಹಿಭಾವನೆ ತಳೆಯುವು ದುಂಟು. ತಾಳ್ಮೆಯ ಅಭಾವದ ಪರಿಣಾಮ ಇದು. ಆದುದರಿಂದ ಪ್ರಗತಿಪಥದಲ್ಲಿ ಮುನ್ನಡೆಯು ವವನು ತಾಳ್ಮೆ ಎಂಬ ಮಹಾ ಗುಣವನ್ನು ದಿನದಿನವೂ ರೂಢಿಸಿಕೊಳ್ಳಬೇಕು.

ವ್ಯಕ್ತಿತ್ವದ ಪೂರ್ಣವಿಕಸನದಲ್ಲಿ ತಾಳ್ಮೆಯ ವ್ಯಾಪಕ ಮಹತ್ವವನ್ನು ಮನಗಂಡ ಶ‍್ರೀ ವಾದಿರಾಜ ಯತಿವರ್ಯರು ಅದನ್ನು ತಪಸ್ಸೆಂದು ಪರಿಗಣಿಸಿ ತಾಳಿಬಾಳಲು ನೀಡಿದ ಸಂದೇಶ ಈ\break ಸೊಲ್ಲುಗಳಲ್ಲಿದೆ–

\begin{verse}
ನೆಟ್ಟ ಸಸಿ ಫಲ ಬರುವ ತನಕ ಶಾಂತಿಯ ತಾಳು,\\ಕಟ್ಟು ಬುತ್ತಿಯ ಮುಂದೆ ಉಣಲುಂಟು ತಾಳು,\\ಕಷ್ಟ ಬಂದರೆ ತಾಳು ಕಂಗೆಡದೆ ತಾಳು,\\ದುಷ್ಟ ಮನುಜರು ಪೇಳ್ವ ನಿಷ್ಠುರದ ನುಡಿ ತಾಳು,\\ಉಕ್ಕು ಹಾಲಿಗೆ ನೀರನಿಕ್ಕುವಂದದಿ ತಾಳು.
\end{verse}

ನೀವು ನೆಟ್ಟ ಸಸಿಗೆ ಗೊಬ್ಬರವನ್ನು ಚೆನ್ನಾಗಿ ಹಾಕಿರಬಹುದು. ನೀರನ್ನೂ ಕಾಲ ಕಾಲದಲ್ಲಿ ಎರೆದಿರಬಹುದು. ಸಸಿಯ ಸುತ್ತಲೂ ಬೇಲಿಯನ್ನು ಕಟ್ಟಿ ಅದನ್ನು ಚೆನ್ನಾಗಿ ರಕ್ಷಿಸಿರಬಹುದು. ಆದರೆ ಬೇಗನೆ ಫಲಕೊಡುವಂತೆ ಸಸಿಯನ್ನು ಅವಸರ ಪಡಿಸಲು ಸಾಧ್ಯವಿಲ್ಲ. ಬೇಗನೆ ಫಲ ಕೊಡುವ ಸಸಿಯನ್ನು ನೆಟ್ಟಿದ್ದರೂ ಅದು ಗೊತ್ತಾದ ಸಮಯದಲ್ಲೇ ಫಲ ನೀಡೀತು. ಫಲ ಬರುವ ವರೆಗೆ ಶಾಂತಚಿತ್ತರಾಗಿರುವ ತಾಳ್ಮೆಯನ್ನು ನಾವು ಕಲಿತಿರಬೇಕಲ್ಲವೆ?

ಯಾವುದೇ ಕ್ಷೇತ್ರದಲ್ಲಿ ಅತ್ಯುನ್ನತ ಮಟ್ಟದ ಸಿದ್ಧಿಯು ಪ್ರಾಮಾಣಿಕ ಪ್ರಯತ್ನದಿಂದ ಸಾಧ್ಯ. ಅದು ನಿಯಮ. ಆದುದರಿಂದ ಮೋಸದಿಂದ ಮಹತ್ಕಾರ್ಯ ಸಾಧ್ಯವಿಲ್ಲ. ಸತ್ಕರ್ಮದಿಂದಲೇ ಶ್ರೇಯಸ್ಸು ಎನ್ನುವ ದೃಢಪ್ರಜ್ಞೆ ನಮ್ಮಲ್ಲಿ ಉದಿಸಬೇಕು. ಫಲದ ಬಗೆಗೆ ಅತ್ಯಾಸಕ್ತನಾಗದೆ ಸತ್ಕರ್ಮದಲ್ಲೇ ಮುಳುಗಲು ನಾವು ತಾಳ್ಮೆಯನ್ನು ಅಭ್ಯಸಿಸಬೇಕು. ‘ಒಳ್ಳೆಯ ಸಂಕಲ್ಪದಿಂದ ಸತ್ಕಾರ್ಯವನ್ನು ಮಾಡಿದವನು ಎಂದೂ ದುರ್ಗತಿಯನ್ನು ಹೊಂದುವುದಿಲ್ಲ’ ಎನ್ನುವುದು ಭಗವಂತನ ವಾಣಿ.\footnote{ ಶ‍್ರೀಮದ್ ಭಗವದ್ಗೀತಾ, ೬, ೪೦.} ಈ ವಾಣಿಯಲ್ಲಿಡುವ ದೃಢವಿಶ್ವಾಸ ನಮ್ಮನ್ನು ವಿಘ್ನಗಳಿಗೆ ಅಂಜದೆ, ಸತ್ಯ ಪಥದಿಂದ ವಿಚಲಿತರಾಗದಂತೆ ಪ್ರೇರಣೆ ನೀಡಬೇಕು. \enginline{Short cut will cut you short} ಎನ್ನುವ ಚತುರೋಕ್ತಿ ಆಂಗ್ಲ ಭಾಷೆಯಲ್ಲಿದೆ. ಒಳದಾರಿ ಹಿಡಿಯಲು ದುಡುಕಿದರೆ ನಿಮ್ಮ ಬೆಳವಣಿಗೆಗದು ಕೆಡಕು ಎಂಬುದು ಈ ಮಾತಿನ ಅರ್ಥ.

ಬದುಕಿನಲ್ಲಿ ಎಲ್ಲರೂ ಒಂದಲ್ಲ ಒಂದು ತೆರನಾದ ಸಂಕಟವನ್ನು ಅನುಭವಿಸಲೇಬೇಕು. ಕೆಲವರು ಸಂಕಟ ಬಂದಾಗ ದಿಕ್ಕು ತೋಚದೆ ಹತಾಶರಾಗಿ ನಿರುತ್ಸಾಹಿಗಳೂ, ನಿರಾಶಾವಾದಿಗಳೂ ಆಗುತ್ತಾರೆ. ತಮ್ಮ ಸಂಕಟಕ್ಕೆ ಇತರರೇ ಕಾರಣರೆಂದು ಅವರನ್ನು ನಿಂದಿಸಿ ಆತ್ಮತೃಪ್ತಿಪಟ್ಟು ಕೊಳ್ಳುತ್ತಾರೆ. ಇನ್ನು ಕೆಲವರು ದುಃಖವನ್ನು ತಡೆಯಲಾರದೆ ಆತ್ಮಹತ್ಯೆಯಂಥ ಹೀನ ಕಾರ್ಯ ವೆಸಗುತ್ತಾರೆ. ದುಃಖವಾಗಲೀ, ಸುಖವಾಗಲೀ ಮನುಷ್ಯನ ಬದುಕಿನಲ್ಲಿ ಸ್ಥಾಯಿಯಲ್ಲ. ದುಃಖವನ್ನು ಸ್ವಾಗತಿಸದಿದ್ದರೂ ಅದು ಬರುತ್ತದೆ. ದುಃಖ ಬಂದಾಗ ಹತಾಶರಾಗದೆ ಅದರ ಕಾರಣವನ್ನು ಕಂಡು ಹಿಡಿದು, ಅದನ್ನು ದಾಟಲು ಧೈರ್ಯಗೆಡದಿರುವ ತಾಳ್ಮೆಯನ್ನು ಅಭ್ಯಸಿಸಬೇಕು.

ಎಂಥ ಸತ್ಯನಿಷ್ಠರೂ, ಸಜ್ಜನರೂ, ಪರೋಪಕಾರಿಗಳೂ, ಕೆಟ್ಟ ಮಾತುಗಳನ್ನೂ ನಿಂದೆಯ ನುಡಿಗಳನ್ನೂ ಕೇಳಬೇಕಾಗುವುದು ಎಂದ ಮೇಲೆ ಸಾಮಾನ್ಯರ ಪಾಡೇನು? ಆ ಅಪಪ್ರಚಾರ ಮತ್ತು ನಿಂದೆಯ ನುಡಿಗಳನ್ನು ನುಂಗಿ ಕುಳಿತಿರಬೇಕೆಂದಲ್ಲ. ಎಲ್ಲಿ ಅದನ್ನು ವಿರೋಧಿಸುವುದು ಯುಕ್ತವೋ, ಪ್ರತಿಕ್ರಿಯೆಯನ್ನು ತೋರಿಸುವುದು ಸಹಜವೋ ಅಲ್ಲಿ ಅದನ್ನು ಕರ್ತವ್ಯ ದೃಷ್ಟಿ ಯಿಂದ ಮಾಡಲೇ ಬೇಕು. ಆದರೆ ಜಗತ್ತಿನಲ್ಲಿ ಪ್ರತಿಯೊಬ್ಬರೂ ದಿನ ಬೆಳಗಾದರೆ ಒಂದಲ್ಲ ಒಂದು ಅಸಂಬದ್ಧ ಮಾತನ್ನು ಕೇಳಬೇಕಾಗುತ್ತದೆ. ಇದರಿಂದ ವಿಚಲಿತನಾಗುವುದಾಗಲೀ, ಎಲ್ಲರ ಬಾಯಿ ಮುಚ್ಚಿಸಹೊರಡುವುದಾಗಲೀ ಹುಚ್ಚುತನವೇ ಆಗುತ್ತದೆ. ಸತ್ಯವು ನಿಂದಕರ ಬಾಯಿಯನ್ನು ತನ್ನಿಂದ ತಾನೇ ಮುಚ್ಚುವಂತೆ ಮಾಡುತ್ತದೆ. ಇನ್ನೊಂದು ದೃಷ್ಟಿಯಿಂದ ನಿಂದಕರು ಇರಬೇಕು. ಏಕೆಂದರೆ ನಮ್ಮ ದೋಷಗಳನ್ನು ಎತ್ತಿ ತೋರಿಸಿ ನಮ್ಮನ್ನು ತಿದ್ದುವ ಶಿಕ್ಷಕರು ಅವರಾಗುತ್ತಾರೆ. ತಾಳ್ಮೆಯಿಂದಲೇ ಈ ಅರಿವು ನಮ್ಮ ಪಾಲಿಗೆ ಲಭ್ಯ.\footnote{\engfoot{Through patience we learn to know self, to measure and test our ideals, to use faith and to seek understanding through virtue. Thus all spiritual attributes are embraced in patience.\general{\hfill}\general{\hbox{}–Edgar~Caycee\general{}}}}

ತಾಳ್ಮೆಯನ್ನು ಬೆಳೆಸಿಕೊಳ್ಳಲು ನಾವು ಸದಾ ಜಾಗರೂಕರೂ, ಪ್ರಾರ್ಥನಾಶೀಲರೂ ಆಗಿರ ಬೇಕು. ಉದ್ವೇಗದ ಹೊತ್ತಿನಲ್ಲಿ ಇತರರನ್ನು ನೋಯಿಸುವ ಒಂದು ಬಿರುನುಡಿ ನಮ್ಮಿಂದ ಹೊರ ಹೊಮ್ಮಬಹುದು. ಅದರಿಂದ ಪರಿತಾಪ, ಪಶ್ಚಾತ್ತಾಪಪಡದೆ ಇರಬೇಕಾದರೆ ಈ ತಾಳ್ಮೆಯನ್ನು ನಾವು ಅಭ್ಯಸಿಸಬೇಕು.

ಎಲ್ಲ ಸದ್ಗುಣಗಳೂ ತಾಳ್ಮೆಯಿಂದ ನಮ್ಮ ವಶವರ್ತಿಗಳಾಗುವುವು. ಆದುದರಿಂದಲೇ ತಾಳಿದ ವನು ಬಾಳಿಯಾನು, ತಾಳದವನು ಹಾಳಾದನು!


\section*{ಏಳಿ, ಎದ್ದೇಳಿ!}

\addsectiontoTOC{ಏಳಿ, ಎದ್ದೇಳಿ !}

ಮನುಷ್ಯನ ಅಂತರಂಗದ ಆಳದಲ್ಲಿ ದಿವ್ಯತೆ ಇದೆ. ಸಚ್ಚಿದಾನಂದದ ಕಿಡಿ ಇದೆ. ದೈವೀ ಶಕ್ತಿಯೇ ಇದೆ. ಈ ದಿವ್ಯತೆಯಲ್ಲಿಡುವ ದೃಢವಿಶ್ವಾಸದಿಂದ ನಮ್ಮ ಪಾಲಿಗೆ ಬಂದೊದಗುವ ಒಂದು ಪ್ರಮುಖ ಗುಣವೇ ಭರವಸೆ–ಎಂದೆಂದೂ ನಿರಾಶಾವಾದಕ್ಕೆಳಸದ ಭಾವನೆ ಮತ್ತು ಭವ್ಯ ಭವಿಷ್ಯದಲ್ಲಿ ನಂಬಿಕೆ. ಯಾವುದೇ ಸಂಕಟಮಯ ಸನ್ನಿವೇಶ ಬಂದೊದಗಿದರೂ ಅದರ ಹಿನ್ನೆಲೆ ಯಲ್ಲಿ ಸಾಮಾನ್ಯ ದೃಷ್ಟಿಗೆ ನಿಲುಕದ ಒಂದು ನಿಯಮ ಕೆಲಸ ಮಾಡುತ್ತಿದೆ ಎಂಬ ನಂಬಿಕೆಯೇ ಈ ಭರವಸೆಗೆ ಪೋಷಕ ಶಕ್ತಿ. ಈ ನಿಯಮವು ನನ್ನ ಅಭ್ಯುದಯದ ರಹಸ್ಯ ಸೂತ್ರ ನನ್ನ ಕೈಯಲ್ಲೇ ಇದೆ ಎಂಬುದನ್ನು ತಡವಾಗಿಯಾದರೂ ತಿಳಿಸಿಕೊಡುತ್ತದೆ. ರಾಬರ್ಟ್ ಬ್ರೌನಿಂಗ್ ಕವಿಯ ‘ಮುಪ್ಪೇರು ನನ್ನೊಡನೆ ಮುಂದಿರ್ಪುದುಚ್ಚ ದಶೆ’ ಎಂಬ ಉಕ್ತಿಯಂತೆ, ಗೆಲವಿನ ಹಾದಿಯಲ್ಲಿ ಆಗ ಸೋಲು ಒಂದು ಸೋಪಾನವೇ ಆಗುತ್ತದೆ. ಚಕ್ರದ ಉರುಳಾಟದಲ್ಲಿ ಕೆಳಮುಖವಾಗಿದ್ದ ಭಾಗ ಮೇಲೇರಲೇಬೇಕು. ರಾತ್ರಿ ಕಳೆದು ತಿರುಗಿ ನೇಸರು ಉದಿಸಲೇಬೇಕು. ನೋವು, ಸಂಕಟ, ದುಃಖ–ಇವು ನಮ್ಮ ಬದುಕಿಗೆ ಬೇಕಾದ ತರಬೇತಿ ಮತ್ತು ಶಿಕ್ಷಣಗಳನ್ನು ನೀಡಿ ಹೊರಟು ಹೋಗುತ್ತವೆ. ಅವು ಸ್ಥಾಯಿಯಲ್ಲ. ಬದುಕನ್ನು ಆಂಶಿಕ ದೃಷ್ಟಿಯಿಂದ ನೋಡಿದಾಗ ಈ ಮಾತು ಸರಿ ಎನಿಸದಿದ್ದರೂ ಪೂರ್ಣದೃಷ್ಟಿಯಿಂದ ನೋಡಿದಾಗ ನಿಜ ಎಂಬುದು ಸ್ಪಷ್ಟವಾಗುವುದು. ಇಷ್ಟೇ ಅಲ್ಲ, ನಮಗೆ ನಿಕಟವಾಗಿರುವ ಮತ್ತು ಸರ್ವತ್ರ ಸರ್ವವ್ಯಾಪಿಯಾಗಿರುವ ಆ ದಿವ್ಯ ವಿಶ್ವ ನಿಯಾಮಕ ಶಕ್ತಿ ನಮ್ಮನ್ನು ಪೀಡಿಸುವುದಕ್ಕಾಗಿ ಇಲ್ಲ; ನಮ್ಮನ್ನು ಉದ್ಧಾರದ ಹಾದಿಯಲ್ಲಿ ನಡೆಸುವುದಕ್ಕಾಗಿ ಇದೆ. ನಮ್ಮ ಮನಸ್ಸಿನ ನೇರಕ್ಕೆ ಸರಿಯಾಗಿ ಎಲ್ಲವೂ ನಡೆಯಬೇಕೆಂದು ಬಯ ಸದೆ ಕಷ್ಟಕಾಲದಲ್ಲೂ ಆ ಶಕ್ತಿಯ ಮೊರೆ ಹೋಗಿ, ಮುನ್ನಡೆಯ ನಮ್ಮ ಪ್ರಯತ್ನವನ್ನು ಬಿಡದಿ ದ್ದರೆ ಉದ್ಧಾರ ಖಂಡಿತ.

ಧೀರರ ದಾರಿಯಲ್ಲಿ ಪ್ರಕೃತಿಯು ಅತ್ಯಂತ ಭೀಷಣ ತಡೆಗಳನ್ನಿಟ್ಟರೂ ಅವುಗಳನ್ನು ಅವರು ಕೆಚ್ಚೆದೆಯ ಸಾಮರ್ಥ್ಯದಿಂದ, ನಿರಂತರ ಪರಿಶ್ರಮದಿಂದ, ಅಪಾರ ಆತ್ಮವಿಶ್ವಾಸದಿಂದ, ಅನಂತ ತಾಳ್ಮೆಯಿಂದ ಎದುರಿಸಿ, ಶ್ರೇಯಸ್ಸಿನ ಶಿಖರವನ್ನೇರಿದುದನ್ನು ನಾವು ಕಾಣುತ್ತೇವೆ.

‘ಓ ದೇವರೆ! ಪ್ರಯತ್ನದ ಬೆಲೆಯನ್ನು ತೆತ್ತಾಗ ನೀನು ಪ್ರತಿಯೊಂದನ್ನೂ ಅನುಗ್ರಹಿಸುತ್ತಿ’\footnote{\engfoot{‘O Lord–thou givest us everything at the price of an effort.’}

~\hfill\engfoot{ –Leonardo da Vinci}} ಎಂದ ಲೀಯೋನಾರ್ಡೋ ಡಾ ವಿಂಚಿಯ ಮಾತು ಅಕ್ಷರಶಃ ಸತ್ಯ.

\chapterend

\addtocontents{toc}{\protect\par\egroup}

